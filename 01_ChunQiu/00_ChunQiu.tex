%% -*- coding: utf-8 -*-
%% Time-stamp: <Chen Wang: 2021-11-01 18:12:07>

\chapter{春秋{\tiny(BC770-BC403)}}

\section{简介}

春秋时期(公元前770年-公元前476年/公元前403年),简称春秋, 是东周的前半段时期。

春秋时代周天子的势力减弱,群雄纷争,齐桓公、宋襄公、晋文公、秦穆公、楚庄王相继称霸,史称“春秋五霸”。當時齊桓公提出「尊周室,攘夷狄,禁篡弑,抑兼併」(尊王攘夷)的思想,因此周天子於表面上仍獲尊重。

春秋时期因孔子修订的《春秋》而得名。这部书记载了从鲁隐公元年(前722年)到鲁哀公十四年(前481年)的历史,共二百四十二年。后史学家为了方便起见,一般从周平王元年(前770年)东周立国,平王東遷到洛邑起,到周敬王四十三年(前477年)或四十四年(前476年)为止(也有学者认为应到《左傳》記載之終(前468年)、三家灭智(前453年)或三家分晋(前403年)),称为“春秋时期”。春秋时期之后是战国时期。

据史书记载,春秋二百四十二年间,有三十六名君主被臣下或敌国杀之,五十二个諸侯國被灭。。有大小战事四百八十多次,诸侯的朝聘和盟会四百五十餘次。鲁国朝王三次,聘周四次。

前771年,因周幽王宠信褒姒,废太子宜臼。宜臼逃至申国,他外公申侯联合曾侯、許文公及犬戎(外族)推翻周幽王,都城宗周被毁坏,後周平王上任,前770年周平王被迫将国都从镐京迁至成周(雒邑)。因雒邑在镐京之东,此后的周朝史称东周。

周室衰微:申侯引犬戎攻入京師,害死女婿周幽王,以恢复外孙周平王的太子地位,擁立平王,使平王有弒父之嫌,因而使周天子在諸侯間的威望下降,其次各诸侯国势力逐渐强大,互相攻伐,故平王東遷後,周室漸漸衰落。周王室放弃了原本的关中地区(宗周),只有洛邑周边(成周)一小塊王畿,而失去对其他诸侯国的控制。

卿士鄭莊公:另外,由於卿士鄭莊公連打勝仗,勢力越來越大,逐漸不把周王室放在眼裡。周平王看到鄭國太驕橫了,不願把處理朝政的大權都交給鄭莊公,想將一半的權力交給另一個卿士虢公忌父,鄭莊公知道後很不滿。

周平王不敢得罪鄭莊公,公元前720年,就將王子狐作為人質讓他住到鄭國去;而鄭國公子忽也作為人質住到都城雒邑,史稱“周鄭交質”。这兩件事使周天子的地位大為降低。

周桓王十三年(前707年),由於鄭莊公不尊王室的問題與鄭國爭執,周桓王率陈、蔡、卫等国军队讨伐郑国,郑莊公领兵抗拒,两军战于繻葛(今河南长葛北),史称“繻葛之战”,郑庄公打败了王师,一箭射中周王的肩膀。说明周王的地位已严重下降,只是还保存着天下共主的虚名罢了,诸侯争霸的时代正式到来。

齐桓公称霸:前685年,齐国君主齐桓公继位,以管仲为相,实施变法,废除井田制度,按土地的肥瘠,确定赋税,设盐、铁官和铸钱,增加财政收入,寓兵于农,将基层行政组织和军事组织合为一体,增加了兵源和作战能力,迅速成为华夏各国中最富强的国家。然后就打起了“尊王攘夷”的口号,多次大会诸侯,帮助或干涉其他国家,抗击夷狄。

周惠王二十一年(前656年),齐桓公带领八个诸侯国的联军,陈兵“蛮夷”楚国边境,质询楚国为何不向周王室朝贡,迫使楚国签订召陵之盟,又於公元前651年舉行葵丘会盟,成为春秋五霸之首。自此,齊桓公建立了會盟霸主的制度。

宋楚之争:齐桓公死后,五公子夺位,齐国内乱不止。傳说,齐桓公的五个儿子互相战争,箭矢射到了齐桓公的尸体上,都没有人顾及。南方的楚国兴起,自称为王,消灭了其北方的几个小国之后将矛头指向中原。宋襄公试图效法齐桓公,以抵抗楚国进攻为名,再次大会诸侯以成为霸主,但宋国实力与威望都不足。宋襄公十五年(前638年),宋楚两军交战于泓水。楚军渡河时宋大司马子鱼建议宋襄公“半渡击之”,宋襄公称趁敌渡河时攻击是为不仁不义拒绝建议;楚军渡河后子鱼建议趁楚军列阵混乱之时攻击,宋襄公再次以不仁不义为由拒绝。楚军列阵完毕后发起攻击,宋军大败,宋襄公大腿中箭,次年因伤重而死。楚成王称雄一时。

晋文公践土之盟:在北方的晉國,與周室同宗。晉獻公时期晋国向四面扩张,领土和国力大增。但献公寵信愛姬,废嫡立幼,致使國政大亂。前636年,晉獻公之子重耳在秦穆公派出的军队护送下继承晋国君位,是为晋文公。他改革政治,发展经济,整军经武,取信于民,安定王室,友好秦国(秦晋之好),在诸侯中威信很高。周襄王二十年(前633年),楚军包围宋国都城商丘。次年初,晋文公率兵救宋,在城濮之战大败楚军,然后会盟于践土,成为中原霸主。

秦穆公独霸西戎:晋文公死后,秦晋联盟被瓦解,秦穆公谋求向东方发展,被晋所阻。秦晋殽之战(前627年),秦全军覆没,大将孟明视被俘虏,隔年在彭衙之战再败,虽然以后有王官之戰的胜利,但终没法挑战晋在中原的地位,惟有转而向西发展,秦穆公任用由余,吞并了一些戎狄部族,寬地千里,独霸西戎。

楚庄王问鼎中原:楚国在城濮战后,向东发展,灭了许多小国,势力南到今云南,北达黄河。楚庄王改革内政,平息暴乱,啟用賢臣孫叔敖兴修水利,改革軍制,国力更为强大,在攻克陆浑戎后,竟陈兵周郊,向周定王的使者询问象徵國家政權的傳國寶器 - 九鼎的大小轻重,意在灭周自立,此即“问鼎”一词的来源。周定王十年(前597年),楚与晋会战于邲(邲之战),大胜晉国。前594年,楚围宋,宋告急于晋,晋不能救,宋遂与楚言和,尊楚。这时中原各国除晋、齐、鲁之外,尽尊楚庄王为霸主。

晋楚大战与弭兵会盟:晋楚两大国之间连续不断的战争给人民带来巨大的灾难,也引起中小国家的厌倦,加以晋楚两大国势均力敌,谁都无法击垮对方。于是由宋国发起,于周简王七年(前579年)举行第一次“弭兵”会盟,是为华元弭兵。但是不久之后,会盟破裂。晉楚兩國再度爆發兩次大規模戰役(前576年的鄢陵之戰、前557年的湛阪之戰),雖皆以晉國獲勝收場,但楚國在中原地區仍與晉國保持勢均力敵的態勢,很多中原小国都备受到影响,疲惫不堪。周灵王二十六年(前546年),出于地缘政治的影响,宋国再次出面斡旋,邀请晋楚和各诸侯国举行第二次“弭兵”会盟,此后战争大大减少。史稱『向戌弭兵』。

吴越雄霸东南:当中原诸侯争霸接近尾声时,地处江浙的吴、越开始发展。吴王阖闾重用孙武、伍子胥等人。周敬王十四年(前506年),吴王以伍子胥为大将,统兵伐楚。吴军攻进楚都郢,伍子胥为父兄报仇,掘楚平王墓,鞭尸三百。周敬王二十四年(前496年)吴军挥师南进伐越。越王勾践率兵迎战,越大夫灵姑浮一戈击中阖闾,阖闾因伤逝世。周敬王二十六年(前494年),吴王夫差为父报仇,兴兵败越。勾践求和,贿赂吴臣伯嚭并送给吴王珍宝和美女西施,自己亲自为夫差牵马。吴王拒绝了伍子胥联齐灭越的建议,接受越国求和,转兵向北进击,大败齐军,成为小霸。勾践卧薪尝胆,十年生聚,十年教训,终于在周元王三年(前473年)消灭吴国,夫差羞愤自杀。勾践北上与齐晋会盟于徐,成为最后一个霸主。

三家分晋:在晉文公回晉即位的時候,有不少隨從隨他回國,結果這些人漸漸在晉國成為世襲貴族,而晉國的國政亦落入這些貴族(士大夫)的手上。前455年,晉國貴族只餘下智、趙、韓、魏四家。智氏出兵攻赵氏,并胁迫魏韩两氏出兵。战事持续两年后,赵氏游说魏韩两家倒戈,灭智氏,瓜分智地并把持晋国国政,史稱三家分晋。到晋幽公仅余绛、曲沃两地。前403年周威烈王册立韩赵魏三家为侯国,即为资治通鉴中春秋战国的分界点。

春秋五霸的崛起:春秋时期,周王室衰微,实际上和一个中等诸侯国地位相近。各国之间互相攻伐,战争持续不断,小国被吞并。各国内部,卿大夫势力强大,动乱时有发生,弑君现象屡见不鲜。《春秋》和《左传》中记载的弑君事件达43次之多,主要集中在春秋前期,这也反映了西周东周交替时权力的急剧变化。「春秋戰國之時,已漸由封建而變為郡縣。」「周初千八百國,至春秋之初,僅存百二十四國。春秋諸國,吞併小弱,大抵以其國地為縣。因滅國而特置縣,因置縣而特命官,封建之制遂漸變為郡縣之制。」

經濟:春秋时期,铁制农具开始普及,春秋时期除使用块炼铁外,还掌握了冶炼生铁的先进技术。铁器的使用使大规模开垦荒地成为可能,促进了私田的发展,同时也为手工业提供了锐利的工具,牛耕渐趋普遍起来,牛耕技术的发展,只有与铁器的使用相配合,方可发挥出它的功能。在青铜冶铸方面发明了错金、错银、嵌红铜等新工艺。侯马大批铸造陶范的出土,显示出这一时期青铜冶铸业和采矿业的规模很大、水平很高。春秋中期以后,各诸侯国已经大量使用货币。金属货币的流通,促进了手工业、商业的发展。

文化:春秋战国是中国文化发展的时期。周天子及其诸侯政治权威的动摇与衰落,造成学在官府的局面被打破,如儒家的孔子创办私学,首开私学风气。孔子提倡有教无类的办学思想,促进教育事业的发展,以及为人们提高自己的社会地位提供了途径。而随之而出现的学术下移、典籍文化走向民间等社会方方面面的变化,又引起了人们思想观念的某种改变,这些变化正是春秋时期思想文化转型得以实现的历史条件,后这为战国的百家争鸣奠定了基础。

这一时期,由于政治的不稳定,禮樂崩壞,学术受政治影响小,学术思想得以获得发展,开始产生了不同的学派。如道家的老子等。老子著有《道德经》,道德经阐述了中国古代朴素的宇宙观,世界观,人生观,对后世中国文化影响深远。《论语》是孔子弟子将孔子的主要言行记录下来整理的。其后,儒家开始发韧,在学术上逐步占据主导地位。尤其至汉武帝“罢黜百家,独尊儒术”后,更是占据中国文化的统治地位,长达两千余年。

藝術:春秋时代的艺术,主要是青铜器上面的雕刻。著名的三足羊首鼎就是春秋时代的青铜艺术品。1923年,在新郑市出土了大量春秋时代的青铜鼎、爵,和西周时期的青铜器相比之下工艺已经大大发展。青铜器上的纹饰也很讲究。

春秋時代的木雕藝術以南方的楚國最為聞名。春秋時代的青銅祭器數量極多,且大小各異,西元前六至五世紀發展出來的精緻裝飾為其特色。相較之下,春秋時代這類的青銅器較通常不加裝飾的戰國時代青銅器為重要。考古挖掘出的春秋時代印章為目前所知最早的,然而有文獻證據顯示印章出現的時間更早。此外,中國的金器製作亦在春秋時代開始普及。

科技:铁器和牛耕在春秋时期得到推广,推动了历史的发展。在天文学、物理学、医学方面。

中国传统农业在春秋时期才开始形成。春秋时期的人们发明了以前没有的铁犁铧、铁锄、连枷、石磨等新农具。

春秋时期青铜器铸造也是这一时代的特征,以曾国和楚国、徐国的青铜器为代表。

建築:春秋时期诸侯国渐强盛,兴建大量城市和宫殿。其时多为以阶梯形夯土台为基的台榭式建筑。以夯土台为中心,附建木质结构房屋,形成多层次宫殿。

而楚國在楚靈王時期建的章華台更是春秋時的建築代表。

山东省临淄县郎家庄春秋时代墓葬出土的漆器残片,中画圆形,四面画四座建筑,柱顶上有栱,承托脊檩。窗仍为井字格,但另加小格。这种四室相背的建筑可能和台榭建筑有关。

歷史同一時期:前753年:古罗马进入王政时代(前753年-前509年) 前550年:波斯帝國的阿契美尼德王朝成立(前550年-前330年) 前509年:羅馬共和國時期開始

%% -*- coding: utf-8 -*-
%% Time-stamp: <Chen Wang: 2021-11-02 15:39:05>

\section{东周}

\subsection{簡介}

东周(前770年至前256年),是自周平王東遷以后對周朝的称呼,相对于之前国都在镐京的時期即西周。东周也是「春秋時代」的开始。

東周京都於前770年自镐京(今陝西省西安市),东迁至雒邑(今河南省洛阳市)。传25王,前后515年。 这一时期是中国的社会制度剧烈转变的时期,以铁器的广泛使用为标志。

周幽王死後,諸侯擁立原先被廢的太子宜臼為王,是为周平王。他即位第二年,見鎬京被戰火破壞,又受到犬戎侵扰,便遷都雒邑,史稱「東周」,以別於在這以前的西周。東周的前半期,諸侯爭相稱霸,持續了二百多年,稱為「春秋時代」;東周的後半期,周天子地位漸失,亦持續了二百多年,稱為「戰國時代」。

周平王東遷以後,管轄範圍大減,形同一個小國,加上被指有弑父之嫌,在諸侯中的威望已经大不如前。面對諸侯之間互相攻伐和兼併,邊境的外族又乘機入侵,周天子不能擔負共主的責任,經常要向一些強大的諸侯求助。在這情況下,強大的諸侯便自居霸主。中原諸侯對四夷侵擾則以「尊王攘夷」口號團結自衛,战国时代徐州相王、五国相王后各大诸侯纷纷僭越称王(吴、越、楚三国春秋時代已称王),周王权威進一步受損。

周襄王十七年(前635年),发生“子带之乱”,襄王不能平,求救于晋文公,文公诛叔带,遂为伯而得河内地。周襄王二十年(前632年),襄王为晋文公所迫,于河阳踐土會盟。

周定王元年(前606年),楚庄王伐陆浑之戎,欲观九鼎。定王使王孙满应设以辞,楚人遂去。

周赧王时,東周国势益弱,同时内部争斗不休,以至分为东周国和西周国,赧王迁都西周國。周赧王八年(前307年),秦借道两周之间攻韩,周人两边都不敢得罪,左右为难。东西两周位于诸强国之间,不能同心协力,反而彼此攻杀。至赧王六十年(前256年),西周国为秦所灭,赧王死,七年后,东周国亦为秦所灭。

春秋时代的得名,是因孔子修订《春秋》而得名。一般而言,春秋时代从周平王五十年(前722年)起,直到周敬王四十三年(前477年)或四十四年(前476年)为止,也有学者认为春秋時代应到三家灭智(前453年)或三家分晋(前403年),原因是即使到三家分晉,除秦、楚、齊等國外,還有其他大小王國。

战国时代由三家分晉,春秋時代結束,直到秦統一中國(公元前221年)這段時間,一般稱為戰國時代。

值得注意的是,東周王朝在戰國後期(前256年)即已被秦所滅,所以戰國時代在時間上並不全然包含在東周裡面。

这一时期是中国歷史的社会制度转变的时期,这一转变是以铁器的广泛使用为标志的,東周基本上進入了鐵器時代。

在东周时期,铁器被广泛使用。农业有了长足的发展,从整体上来看人口不断成长,原来各诸侯国之间的无人地带,已不存在。各国因争夺土地或者水利资源,冲突时起。铜钱开始流行,甚至在楚国还出现了金币—郢钱,出现一定的商品经济和商人阶层。教育向平民普及。贵族与平民间的界限也被冲破。社会产生了一种革命性的变化,周王朝建立的宗法封建制度,已经不能适应这一变化。

東周與西周的地理位置差異反應在藝術表現上,尤其是東周晚期的藝術作品,展現了多元的風貌與高水準的技術。或許受到孔子反對以人殉葬的影響,以低溫燒製的陪葬塑像(明器,又稱「冥器」、「盟器」)數量增加。

東周時期亦出現低溫綠色鉛釉器皿、質地鬆的打磨黑色器皿、高溫釉器皿等。有些陶器仿效最新流行的漆器,色彩鮮明,有些則仿效青銅器。另有模製與裝飾的陶瓦、陶磚。西周時期較少見的玉雕再次成為重要的陪葬品與個人飾物。青銅的應用不限於宗教禮儀用途,變得較為世俗,常用作結婚贈禮之居家裝飾。青銅鐘及青銅鏡逐漸流行,動物和怪獸圖騰則由色彩繽紛而樣式化的裝飾圖案所取代。東周墓葬出土有最早繪於絲絹上的畫作。此外,亦發現了漢代及唐代陪葬陶器的前身。

%% -*- coding: utf-8 -*-
%% Time-stamp: <Chen Wang: 2021-11-01 18:13:59>

\subsection{平王宜臼{\tiny(BC772-BC720)}}

\subsubsection{生平}

周平王(前781年-前720年),姓姬,名宜臼,東周第一位天子。周幽王之子,母申后為申侯之女,後母褒姒。

姬宜臼本是周幽王太子,後因周幽王寵愛美女褒姒,褒姒生下一子伯服,得寵更甚。褒姒卻與權臣勾結,陰謀奪嫡,而幽王竟為了討美人歡心,便廢掉王后申后與太子宜臼母子,改封褒姒與其子伯服為王后和太子。宜臼逃奔申國。

前771年,周幽王被杀。申侯、缯侯及许文公,在申国擁立宜臼为周天子,即「周平王」。而虢公翰则在携,拥立周幽王之弟姬余臣为周天子,即「携王」(携惠王),形成「二王并立」的局面。

前770年,由於鎬京在戰後已殘破不堪,宜臼為避犬戎,在晉國和鄭國的支持下迁都雒邑,史稱「平王东迁」。周朝至此以後,為史家稱作「東周」。

前750年,携王被晋文侯所弑。前720年,周王宜臼去世,諡號平王。

當時傳聞平王和母親申后因被其父周幽王廢掉後一直懷恨在心,故意聯合諸侯國繒和外族犬戎入宮殺害幽王和褒姒,然後即位為天子,故此周平王有弑父之嫌,亦令周朝的國勢一落千丈。除晉鄭二國之外,其餘諸侯認為宜臼有弒父之嫌疑,故都不再聽從天子,平王唯“晉、鄭是依”,勉強支運國勢,而造成春秋時代的開始。

\subsubsection{年表}

% \centering
\begin{longtable}{|>{\centering\scriptsize}m{2em}|>{\centering\scriptsize}m{1.3em}|>{\centering}m{8.8em}|}
  % \caption{秦王政}\\
  \toprule
  \SimHei \normalsize 年数 & \SimHei \scriptsize 公元 & \SimHei 大事件 \tabularnewline
  % \midrule
  \endfirsthead
  \toprule
  \SimHei \normalsize 年数 & \SimHei \scriptsize 公元 & \SimHei 大事件 \tabularnewline
  \midrule
  \endhead
  \midrule
  % 元年 & -770 & \tabularnewline\hline
  % 二年 & -769 & \tabularnewline\hline
  % 三年 & -768 & \tabularnewline\hline
  % 四年 & -767 & \tabularnewline\hline
  % 五年 & -766 & \tabularnewline\hline
  % 六年 & -765 & \tabularnewline\hline
  % 七年 & -764 & \tabularnewline\hline
  % 八年 & -763 & \tabularnewline\hline
  % 九年 & -762 & \tabularnewline\hline
  % 十年 & -761 & \tabularnewline\hline
  % 十一年 & -760 & \tabularnewline\hline
  % 十二年 & -759 & \tabularnewline\hline
  % 十三年 & -758 & \tabularnewline\hline
  % 十四年 & -757 & \tabularnewline\hline
  % 十五年 & -756 & \tabularnewline\hline
  % 十六年 & -755 & \tabularnewline\hline
  % 十七年 & -754 & \tabularnewline\hline
  % 十八年 & -753 & \tabularnewline\hline
  % 十九年 & -752 & \tabularnewline\hline
  % 二十年 & -751 & \tabularnewline\hline
  % 二一年 & -750 & \tabularnewline\hline
  % 二二年 & -749 & \tabularnewline\hline
  % 二三年 & -748 & \tabularnewline\hline
  % 二四年 & -747 & \tabularnewline\hline
  % 二五年 & -746 & \tabularnewline\hline
  % 二六年 & -745 & \tabularnewline\hline
  % 二七年 & -744 & \tabularnewline\hline
  % 二八年 & -743 & \tabularnewline\hline
  % 二九年 & -742 & \tabularnewline\hline
  % 三十年 & -741 & \tabularnewline\hline
  % 三一年 & -740 & \tabularnewline\hline
  % 三二年 & -739 & \tabularnewline\hline
  % 三三年 & -738 & \tabularnewline\hline
  % 三四年 & -737 & \tabularnewline\hline
  % 三五年 & -736 & \tabularnewline\hline
  % 三六年 & -735 & \tabularnewline\hline
  % 三七年 & -734 & \tabularnewline\hline
  % 三八年 & -733 & \tabularnewline\hline
  % 三九年 & -732 & \tabularnewline\hline
  % 四十年 & -731 & \tabularnewline\hline
  % 四一年 & -730 & \tabularnewline\hline
  % 四二年 & -729 & \tabularnewline\hline
  % 四三年 & -728 & \tabularnewline\hline
  % 四四年 & -727 & \tabularnewline\hline
  % 四五年 & -726 & \tabularnewline\hline
  % 四六年 & -725 & \tabularnewline\hline
  % 四七年 & -724 & \tabularnewline\hline
  % 四八年 & -723 & \tabularnewline\hline
  四九年 & -722 & \tabularnewline\hline
  五十年 & -721 & \tabularnewline\hline
  五一年 & -720 & \tabularnewline
  \bottomrule
\end{longtable}

%%% Local Variables:
%%% mode: latex
%%% TeX-engine: xetex
%%% TeX-master: "../../Main"
%%% End:

%% -*- coding: utf-8 -*-
%% Time-stamp: <Chen Wang: 2021-11-01 18:25:56>

\subsection{桓王林\tiny{(BC719-BC697)}}

\subsubsection{生平}

周桓王(?-前697年),姓姬,名林,東周第二代君主,諡桓。他是周平王之孫,因平王駕崩時,太子洩父早死,其作为洩父之子得以繼承天子之位。

桓王(前720年)即位後,計劃削弱卿權以加強王權。加上周朝領地與鄭國領地鄰接,鄭國多次越界取禾,故桓王罷免了鄭莊公卿士之職。莊公不滿,便不再朝見天子。周王室始與鄭國交惡。

桓王二年(前718年)春天,晉國曲沃封君曲沃莊伯賄賂周桓王,聯合鄭國、邢國攻打晉國都城翼城,晉鄂侯戰敗,逃奔隨邑。同年夏天,晉鄂侯去世,曲沃莊伯於是再度攻打晉國。由於當時曲沃莊伯背叛周桓王,同年秋天,周桓王便反過來支持晉國,並派遣虢公 率領軍隊討伐曲沃莊伯,曲沃莊伯兵敗,只得逃回曲沃防守。周桓王立晉鄂侯之子晉哀侯為君。

桓王十二年(前708年)周桓王聯合秦國出兵包圍芮國,俘虜芮國國君芮伯萬。

桓王十三年(前707年),桓王率蔡、衞、陳聯軍攻鄭,大敗於繻葛。桓王本人更於此役中為鄭將祝聃射箭中傷。此後桓王雖然仍能影響虢國,但已無力阻止周王室轉衰之势,也無力阻止諸侯間的互相攻伐。

桓王十八年(前703年),虢仲向周桓王進讒言誣陷大夫詹父。周桓王認為詹父有理,詹父於是帶領周天子的軍隊進攻虢國。同年夏天,虢公逃亡到虞國

桓王二十三年(前697年),桓王三月因病駕崩,儿子王子佗继位,是為周莊王。

\subsubsection{年表}

% \centering
\begin{longtable}{|>{\centering\scriptsize}m{2em}|>{\centering\scriptsize}m{1.3em}|>{\centering}m{8.8em}|}
  % \caption{秦王政}\\
  \toprule
  \SimHei \normalsize 年数 & \SimHei \scriptsize 公元 & \SimHei 大事件 \tabularnewline
  % \midrule
  \endfirsthead
  \toprule
  \SimHei \normalsize 年数 & \SimHei \scriptsize 公元 & \SimHei 大事件 \tabularnewline
  \midrule
  \endhead
  \midrule
  元年 & -719 & \tabularnewline\hline
  二年 & -718 & \tabularnewline\hline
  三年 & -717 & \tabularnewline\hline
  四年 & -716 & \tabularnewline\hline
  五年 & -715 & \tabularnewline\hline
  六年 & -714 & \tabularnewline\hline
  七年 & -713 & \tabularnewline\hline
  八年 & -712 & \tabularnewline\hline
  九年 & -711 & \tabularnewline\hline
  十年 & -710 & \tabularnewline\hline
  十一年 & -709 & \tabularnewline\hline
  十二年 & -708 & \tabularnewline\hline
  十三年 & -707 & \tabularnewline\hline
  十四年 & -706 & \tabularnewline\hline
  十五年 & -705 & \tabularnewline\hline
  十六年 & -704 & \tabularnewline\hline
  十七年 & -703 & \tabularnewline\hline
  十八年 & -702 & \tabularnewline\hline
  十九年 & -701 & \tabularnewline\hline
  二十年 & -700 & \tabularnewline\hline
  二一年 & -699 & \tabularnewline\hline
  二二年 & -698 & \tabularnewline\hline
  二三年 & -697 & \tabularnewline
  \bottomrule
\end{longtable}

%%% Local Variables:
%%% mode: latex
%%% TeX-engine: xetex
%%% TeX-master: "../../Main"
%%% End:

%% -*- coding: utf-8 -*-
%% Time-stamp: <Chen Wang: 2021-11-02 14:37:57>

\subsection{莊王佗\tiny{(BC696-BC682)}}

\subsubsection{生平}

周莊王(?-前682年),姓姬,名佗,東周第三代國王,諡莊。他是周桓王之兒。莊王三年(前694年),周公黑肩欲殺莊王,而立莊王弟王子克,事泄,黑肩被莊王與辛伯所殺,克奔南燕(河南延津)。

在位期间执政为虢公林父、周公黑肩。

\subsubsection{年表}

% \centering
\begin{longtable}{|>{\centering\scriptsize}m{2em}|>{\centering\scriptsize}m{1.3em}|>{\centering}m{8.8em}|}
  % \caption{秦王政}\\
  \toprule
  \SimHei \normalsize 年数 & \SimHei \scriptsize 公元 & \SimHei 大事件 \tabularnewline
  % \midrule
  \endfirsthead
  \toprule
  \SimHei \normalsize 年数 & \SimHei \scriptsize 公元 & \SimHei 大事件 \tabularnewline
  \midrule
  \endhead
  \midrule
  元年 & -696 & \tabularnewline\hline
  二年 & -695 & \tabularnewline\hline
  三年 & -694 & \tabularnewline\hline
  四年 & -693 & \tabularnewline\hline
  五年 & -692 & \tabularnewline\hline
  六年 & -691 & \tabularnewline\hline
  七年 & -690 & \tabularnewline\hline
  八年 & -689 & \tabularnewline\hline
  九年 & -688 & \tabularnewline\hline
  十年 & -687 & \tabularnewline\hline
  十一年 & -686 & \tabularnewline\hline
  十二年 & -685 & \tabularnewline\hline
  十三年 & -684 & \tabularnewline\hline
  十四年 & -683 & \tabularnewline\hline
  十五年 & -682 & \tabularnewline  
  \bottomrule
\end{longtable}

%%% Local Variables:
%%% mode: latex
%%% TeX-engine: xetex
%%% TeX-master: "../../Main"
%%% End:

%% -*- coding: utf-8 -*-
%% Time-stamp: <Chen Wang: 2021-11-02 14:42:43>

\subsection{僖王胡齐\tiny{(BC681-BC677)}}

\subsubsection{生平}

周僖王(?-前677年),又作周釐王,姓姬,名胡齐,東周第四代君主,在位5年,號僖。

僖王爲周莊王長子,莊王雖偏爱姚姬生的少子王子頹,但未能廢長立幼。前681年,僖王即位。

前679年,曲沃克晉後,曲沃武公把晉國的寶器獻給僖王,僖王承認曲沃武公為晉君,列為諸侯。前678年,遭晉軍攻打並殺害夷邑大夫詭諸,執政大臣周公忌父逃奔虢國。

前677年,周僖王崩。

在位期间执政为虢公醜、周公忌父。

\subsubsection{年表}

% \centering
\begin{longtable}{|>{\centering\scriptsize}m{2em}|>{\centering\scriptsize}m{1.3em}|>{\centering}m{8.8em}|}
  % \caption{秦王政}\\
  \toprule
  \SimHei \normalsize 年数 & \SimHei \scriptsize 公元 & \SimHei 大事件 \tabularnewline
  % \midrule
  \endfirsthead
  \toprule
  \SimHei \normalsize 年数 & \SimHei \scriptsize 公元 & \SimHei 大事件 \tabularnewline
  \midrule
  \endhead
  \midrule
  元年 & -681 & \tabularnewline\hline
  二年 & -680 & \tabularnewline\hline
  三年 & -679 & \tabularnewline\hline
  四年 & -678 & \tabularnewline\hline
  五年 & -677 & \tabularnewline
  \bottomrule
\end{longtable}

%%% Local Variables:
%%% mode: latex
%%% TeX-engine: xetex
%%% TeX-master: "../../Main"
%%% End:

%% -*- coding: utf-8 -*-
%% Time-stamp: <Chen Wang: 2021-11-02 14:59:42>

\subsection{惠王閬\tiny{(BC676-BC652)}}

\subsubsection{生平}

周惠王(?-前652年),姓姬,名閬,又名聞,東周第五代君主,諡惠。他是周僖王之子。

周惠王在前677年繼位後,佔用蒍國的園圃飼養野獸,蒍國的人民不滿,惠王二年,有五大夫作亂,立王子頹為周天子,惠王奔溫(今河南溫縣南),鄭厲公在櫟地(今禹州市)收容惠王,並在惠王四年與虢國攻入周朝,協助平定“子頹之亂”,惠王復辟,鄭國因功獲賜予虎牢(今河南滎陽汜水鎮)以東的地方,虢國也獲賜土地。

周惠王晚年宠爱幼子王子带,欲立為嗣,约郑联楚、晋以成此事,但此时齐桓公稱霸天下,与诸侯会盟力挺太子,周惠王未能如愿。周惠王駕崩后,太子周襄王即位。

《史記·周本紀》稱惠王在位25年,《左傳》稱周惠王在魯僖公七年(前653年)冬天駕崩。

在位期间执政为虢公醜、周公忌父、宰孔。

\subsubsection{年表}

% \centering
\begin{longtable}{|>{\centering\scriptsize}m{2em}|>{\centering\scriptsize}m{1.3em}|>{\centering}m{8.8em}|}
  % \caption{秦王政}\\
  \toprule
  \SimHei \normalsize 年数 & \SimHei \scriptsize 公元 & \SimHei 大事件 \tabularnewline
  % \midrule
  \endfirsthead
  \toprule
  \SimHei \normalsize 年数 & \SimHei \scriptsize 公元 & \SimHei 大事件 \tabularnewline
  \midrule
  \endhead
  \midrule
  元年 & -676 & \tabularnewline\hline
  二年 & -675 & \tabularnewline\hline
  三年 & -674 & \tabularnewline\hline
  四年 & -673 & \tabularnewline\hline
  五年 & -672 & \tabularnewline\hline
  六年 & -671 & \tabularnewline\hline
  七年 & -670 & \tabularnewline\hline
  八年 & -669 & \tabularnewline\hline
  九年 & -668 & \tabularnewline\hline
  十年 & -667 & \tabularnewline\hline
  十一年 & -666 & \tabularnewline\hline
  十二年 & -665 & \tabularnewline\hline
  十三年 & -664 & \tabularnewline\hline
  十四年 & -663 & \tabularnewline\hline
  十五年 & -662 & \tabularnewline\hline
  十六年 & -661 & \tabularnewline\hline
  十七年 & -660 & \tabularnewline\hline
  十八年 & -659 & \tabularnewline\hline
  十九年 & -658 & \tabularnewline\hline
  二十年 & -657 & \tabularnewline\hline
  二一年 & -656 & \tabularnewline\hline
  二二年 & -655 & \tabularnewline\hline
  二三年 & -654 & \tabularnewline\hline
  二四年 & -653 & \tabularnewline\hline
  二五年 & -652 & \tabularnewline  
  \bottomrule
\end{longtable}

%%% Local Variables:
%%% mode: latex
%%% TeX-engine: xetex
%%% TeX-master: "../../Main"
%%% End:

%% -*- coding: utf-8 -*-
%% Time-stamp: <Chen Wang: 2021-11-02 15:06:10>

\subsection{襄王鄭\tiny{(BC651-BC619)}}

\subsubsection{生平}

周襄王(?-前619年),姬姓,名鄭,東周第六代君主,諡襄。襄王是周惠王之子, 《史記·周本紀》稱襄王在位32年,《左傳》稱襄王崩於魯文公八年(前619年)秋。

惠王死後,襄王懼怕異母弟王子帶爭奪王位繼承權,秘不發喪,並派人向齊國求援,襄王直到大局已定才公佈父王死訊。

周襄王時,因鄭伯不聽王命,曾經要求狄人攻打鄭國。在狄人擊敗鄭國後,周襄王娶隗姓狄人為妻,又將隗后退回。此舉引發狄人不滿,前636年,王子帶聯合狄人,攻打周襄王,周襄王逃到鄭國。

當時晉文公勢力強大,在前635年出兵助襄王,殺王子帶,迎接周襄王返回洛陽復位。前632年,晉文公召襄王,襄王親自到踐土(今河南原陽西南)會見他。

周襄王在位期间宋襄公、晋文公、秦穆公相继称霸。在位期间执政为宰孔、周公忌父、王子虎、周公閱、王叔桓公。

\subsubsection{年表}

% \centering
\begin{longtable}{|>{\centering\scriptsize}m{2em}|>{\centering\scriptsize}m{1.3em}|>{\centering}m{8.8em}|}
  % \caption{秦王政}\\
  \toprule
  \SimHei \normalsize 年数 & \SimHei \scriptsize 公元 & \SimHei 大事件 \tabularnewline
  % \midrule
  \endfirsthead
  \toprule
  \SimHei \normalsize 年数 & \SimHei \scriptsize 公元 & \SimHei 大事件 \tabularnewline
  \midrule
  \endhead
  \midrule
  元年 & -651 & \tabularnewline\hline
  二年 & -650 & \tabularnewline\hline
  三年 & -649 & \tabularnewline\hline
  四年 & -648 & \tabularnewline\hline
  五年 & -647 & \tabularnewline\hline
  六年 & -646 & \tabularnewline\hline
  七年 & -645 & \tabularnewline\hline
  八年 & -644 & \tabularnewline\hline
  九年 & -643 & \tabularnewline\hline
  十年 & -642 & \tabularnewline\hline
  十一年 & -641 & \tabularnewline\hline
  十二年 & -640 & \tabularnewline\hline
  十三年 & -639 & \tabularnewline\hline
  十四年 & -638 & \tabularnewline\hline
  十五年 & -637 & \tabularnewline\hline
  十六年 & -636 & \tabularnewline\hline
  十七年 & -635 & \tabularnewline\hline
  十八年 & -634 & \tabularnewline\hline
  十九年 & -633 & \tabularnewline\hline
  二十年 & -632 & \tabularnewline\hline
  二一年 & -631 & \tabularnewline\hline
  二二年 & -630 & \tabularnewline\hline
  二三年 & -629 & \tabularnewline\hline
  二四年 & -628 & \tabularnewline\hline
  二五年 & -627 & \tabularnewline\hline
  二六年 & -626 & \tabularnewline\hline
  二七年 & -625 & \tabularnewline\hline
  二八年 & -624 & \tabularnewline\hline
  二九年 & -623 & \tabularnewline\hline
  三十年 & -622 & \tabularnewline\hline
  三一年 & -621 & \tabularnewline\hline
  三二年 & -620 & \tabularnewline\hline
  三三年 & -619 & \tabularnewline  
  \bottomrule
\end{longtable}

%%% Local Variables:
%%% mode: latex
%%% TeX-engine: xetex
%%% TeX-master: "../../Main"
%%% End:

%% -*- coding: utf-8 -*-
%% Time-stamp: <Chen Wang: 2021-11-02 15:08:18>

\subsection{頃王壬臣\tiny{(BC618-BC613)}}

\subsubsection{生平}

周頃王(?-前613年),姓姬,名壬臣,為周襄王之子。周頃王在前618年繼位為東周第七代君主,當時王畿已縮小,王室財政一貧如洗,無法安葬襄王,頃王只得派毛伯衛向魯國討錢。後來魯文公派使者送錢到都城,才安葬了襄王。

頃王在前613年春天去世,在位6年,由兒子周匡王繼位。

在位期间执政为周公閱、王叔桓公、王孫蘇。

\subsubsection{年表}

% \centering
\begin{longtable}{|>{\centering\scriptsize}m{2em}|>{\centering\scriptsize}m{1.3em}|>{\centering}m{8.8em}|}
  % \caption{秦王政}\\
  \toprule
  \SimHei \normalsize 年数 & \SimHei \scriptsize 公元 & \SimHei 大事件 \tabularnewline
  % \midrule
  \endfirsthead
  \toprule
  \SimHei \normalsize 年数 & \SimHei \scriptsize 公元 & \SimHei 大事件 \tabularnewline
  \midrule
  \endhead
  \midrule
  元年 & -618 & \tabularnewline\hline
  二年 & -617 & \tabularnewline\hline
  三年 & -616 & \tabularnewline\hline
  四年 & -615 & \tabularnewline\hline
  五年 & -614 & \tabularnewline\hline
  六年 & -613 & \tabularnewline  
  \bottomrule
\end{longtable}

%%% Local Variables:
%%% mode: latex
%%% TeX-engine: xetex
%%% TeX-master: "../../Main"
%%% End:

%% -*- coding: utf-8 -*-
%% Time-stamp: <Chen Wang: 2021-11-02 15:10:26>

\subsection{匡王班\tiny{(BC612-BC607)}}

\subsubsection{生平}

周匡王(?-前607年),姓姬,名班,中国东周第8代君主,前612年至前607年在位,共6年。匡王是周頃王之子。前607年十月,周王班崩,諡“匡”,其弟王子瑜繼位,即周定王。

在位期间执政为周公閱、王孫蘇、召桓公、毛伯衛。

\subsubsection{年表}

% \centering
\begin{longtable}{|>{\centering\scriptsize}m{2em}|>{\centering\scriptsize}m{1.3em}|>{\centering}m{8.8em}|}
  % \caption{秦王政}\\
  \toprule
  \SimHei \normalsize 年数 & \SimHei \scriptsize 公元 & \SimHei 大事件 \tabularnewline
  % \midrule
  \endfirsthead
  \toprule
  \SimHei \normalsize 年数 & \SimHei \scriptsize 公元 & \SimHei 大事件 \tabularnewline
  \midrule
  \endhead
  \midrule
  元年 & -612 & \tabularnewline\hline
  二年 & -611 & \tabularnewline\hline
  三年 & -610 & \tabularnewline\hline
  四年 & -609 & \tabularnewline\hline
  五年 & -608 & \tabularnewline\hline
  六年 & -607 & \tabularnewline
  \bottomrule
\end{longtable}

%%% Local Variables:
%%% mode: latex
%%% TeX-engine: xetex
%%% TeX-master: "../../Main"
%%% End:

%% -*- coding: utf-8 -*-
%% Time-stamp: <Chen Wang: 2021-11-02 15:14:57>

\subsection{定王瑜\tiny{(BC606-BC586)}}

\subsubsection{生平}

周定王(?-前586年),姓姬,名瑜,中国東周第9代天子,前606年—前586年在位,周定王是匡王之弟。定王在位21年而卒,子夷立,為簡王。

在位期间执政为王孫蘇、召桓公、劉康公、毛伯衛、單襄公。

楚莊王为稱霸天下,不斷北侵並打敗了晋国、齐国、宋国、郑国、陈国、蔡国等國,在定王元年征伐陆浑之戎,進軍到周京雒邑的南郊,向周王耀武示威。定王不敢責問楚莊王,便派王孫滿去慰勞楚軍,楚莊王詢問周朝鎮國之寶的九鼎大小輕重,欲逼周取天下。後王孫滿以婉辭說服了楚莊王,使楚不敢輕舉妄動去取代周朝,便撤兵回國。

\subsubsection{年表}

% \centering
\begin{longtable}{|>{\centering\scriptsize}m{2em}|>{\centering\scriptsize}m{1.3em}|>{\centering}m{8.8em}|}
  % \caption{秦王政}\\
  \toprule
  \SimHei \normalsize 年数 & \SimHei \scriptsize 公元 & \SimHei 大事件 \tabularnewline
  % \midrule
  \endfirsthead
  \toprule
  \SimHei \normalsize 年数 & \SimHei \scriptsize 公元 & \SimHei 大事件 \tabularnewline
  \midrule
  \endhead
  \midrule
  元年 & -606 & \tabularnewline\hline
  二年 & -605 & \tabularnewline\hline
  三年 & -604 & \tabularnewline\hline
  四年 & -603 & \tabularnewline\hline
  五年 & -602 & \tabularnewline\hline
  六年 & -601 & \tabularnewline\hline
  七年 & -600 & \tabularnewline\hline
  八年 & -599 & \tabularnewline\hline
  九年 & -598 & \tabularnewline\hline
  十年 & -597 & \tabularnewline\hline
  十一年 & -596 & \tabularnewline\hline
  十二年 & -595 & \tabularnewline\hline
  十三年 & -594 & \tabularnewline\hline
  十四年 & -593 & \tabularnewline\hline
  十五年 & -592 & \tabularnewline\hline
  十六年 & -591 & \tabularnewline\hline
  十七年 & -590 & \tabularnewline\hline
  十八年 & -589 & \tabularnewline\hline
  十九年 & -588 & \tabularnewline\hline
  二十年 & -587 & \tabularnewline\hline
  二一年 & -586 & \tabularnewline  
  \bottomrule
\end{longtable}

%%% Local Variables:
%%% mode: latex
%%% TeX-engine: xetex
%%% TeX-master: "../../Main"
%%% End:

%% -*- coding: utf-8 -*-
%% Time-stamp: <Chen Wang: 2021-11-02 15:13:42>

\subsection{簡王夷\tiny{(BC585-BC572)}}

\subsubsection{生平}

周簡王(?-前572年),姓姬,名夷,為周定王之子。在位14年,此期間晋、楚、秦,宋、郑等国相互攻伐不止,吴国兴起,攻入楚国,幾乎亡楚。前572年九月,周王夷病死,諡号为简王。

在位期间执政为單襄公、劉康公、周公楚、尹武公。

子周灵王、儋季。

\subsubsection{年表}

% \centering
\begin{longtable}{|>{\centering\scriptsize}m{2em}|>{\centering\scriptsize}m{1.3em}|>{\centering}m{8.8em}|}
  % \caption{秦王政}\\
  \toprule
  \SimHei \normalsize 年数 & \SimHei \scriptsize 公元 & \SimHei 大事件 \tabularnewline
  % \midrule
  \endfirsthead
  \toprule
  \SimHei \normalsize 年数 & \SimHei \scriptsize 公元 & \SimHei 大事件 \tabularnewline
  \midrule
  \endhead
  \midrule
  元年 & -585 & \tabularnewline\hline
  二年 & -584 & \tabularnewline\hline
  三年 & -583 & \tabularnewline\hline
  四年 & -582 & \tabularnewline\hline
  五年 & -581 & \tabularnewline\hline
  六年 & -580 & \tabularnewline\hline
  七年 & -579 & \tabularnewline\hline
  八年 & -578 & \tabularnewline\hline
  九年 & -577 & \tabularnewline\hline
  十年 & -576 & \tabularnewline\hline
  十一年 & -575 & \tabularnewline\hline
  十二年 & -574 & \tabularnewline\hline
  十三年 & -573 & \tabularnewline\hline
  十四年 & -572 & \tabularnewline  
  \bottomrule
\end{longtable}

%%% Local Variables:
%%% mode: latex
%%% TeX-engine: xetex
%%% TeX-master: "../../Main"
%%% End:

%% -*- coding: utf-8 -*-
%% Time-stamp: <Chen Wang: 2021-11-02 15:17:28>

\subsection{靈王泄心\tiny{(BC571-BC545)}}

\subsubsection{生平}

周靈王(?-前545年),姓姬,名泄心,是周简王之子,東周第11代國王,在位27年。《幼学琼林》中说他出生时便有胡须。

《列仙傳》中記載:周灵王的长子太子晋天性聪明,善吹笙,立他为太子,不幸早逝。公元前545年十一月的某天夜裡,周灵王梦见太子骑着白鹤来迎接他。传位于次子王子贵,癸巳日,病死。孔子在周靈王二十一年出生于魯。

在位期間執政為王叔陳生、伯輿、單靖公。

\subsubsection{年表}

% \centering
\begin{longtable}{|>{\centering\scriptsize}m{2em}|>{\centering\scriptsize}m{1.3em}|>{\centering}m{8.8em}|}
  % \caption{秦王政}\\
  \toprule
  \SimHei \normalsize 年数 & \SimHei \scriptsize 公元 & \SimHei 大事件 \tabularnewline
  % \midrule
  \endfirsthead
  \toprule
  \SimHei \normalsize 年数 & \SimHei \scriptsize 公元 & \SimHei 大事件 \tabularnewline
  \midrule
  \endhead
  \midrule
  元年 & -571 & \tabularnewline\hline
  二年 & -570 & \tabularnewline\hline
  三年 & -569 & \tabularnewline\hline
  四年 & -568 & \tabularnewline\hline
  五年 & -567 & \tabularnewline\hline
  六年 & -566 & \tabularnewline\hline
  七年 & -565 & \tabularnewline\hline
  八年 & -564 & \tabularnewline\hline
  九年 & -563 & \tabularnewline\hline
  十年 & -562 & \tabularnewline\hline
  十一年 & -561 & \tabularnewline\hline
  十二年 & -560 & \tabularnewline\hline
  十三年 & -559 & \tabularnewline\hline
  十四年 & -558 & \tabularnewline\hline
  十五年 & -557 & \tabularnewline\hline
  十六年 & -556 & \tabularnewline\hline
  十七年 & -555 & \tabularnewline\hline
  十八年 & -554 & \tabularnewline\hline
  十九年 & -553 & \tabularnewline\hline
  二十年 & -552 & \tabularnewline\hline
  二一年 & -551 & \tabularnewline\hline
  二二年 & -550 & \tabularnewline\hline
  二三年 & -549 & \tabularnewline\hline
  二四年 & -548 & \tabularnewline\hline
  二五年 & -547 & \tabularnewline\hline
  二六年 & -546 & \tabularnewline\hline
  二七年 & -545 & \tabularnewline  
  \bottomrule
\end{longtable}

%%% Local Variables:
%%% mode: latex
%%% TeX-engine: xetex
%%% TeX-master: "../../Main"
%%% End:

%% -*- coding: utf-8 -*-
%% Time-stamp: <Chen Wang: 2021-11-02 15:21:09>

\subsection{景王貴\tiny{(BC544-BC520)}}

\subsubsection{生平}

周景王(?-前520年),姓姬,名貴,中國東周君主,諡號景,為周靈王之子。周景王在位时,财政困難,連器皿都要向各国索讨。

有一次,景王宴请晋国大臣知文子荀跞,指着鲁国送来的酒壶说:“各国都有器物送给天子,为何晋国没有?”荀跞答不出来,請副使籍谈答覆,籍谈说当初晋国受封时,未赐以礼器,現在晋国忙于对付戎狄,當然無法送出礼物。周景王列数王室赐给晋的土地器物,讽刺其“数典而忘其祖”,这是“数典忘祖”的典故。此時周天子的地位已经一落千丈。

周景王太子寿早死,立王子猛为太子,卻宠爱庶长子王子朝,還一度打算刺殺支持王子猛的單穆公、劉文公。公元前520年四月,周景王病重,嘱咐宾孟要擁立王子朝。王子朝未及立为嗣君,景王却突然病死,由王子猛即位[2]。

周景王在位期間執政為單靖公、劉定公、成簡公、單獻公、單成公、劉獻公、單穆公。

\subsubsection{年表}

% \centering
\begin{longtable}{|>{\centering\scriptsize}m{2em}|>{\centering\scriptsize}m{1.3em}|>{\centering}m{8.8em}|}
  % \caption{秦王政}\\
  \toprule
  \SimHei \normalsize 年数 & \SimHei \scriptsize 公元 & \SimHei 大事件 \tabularnewline
  % \midrule
  \endfirsthead
  \toprule
  \SimHei \normalsize 年数 & \SimHei \scriptsize 公元 & \SimHei 大事件 \tabularnewline
  \midrule
  \endhead
  \midrule
  元年 & -544 & \tabularnewline\hline
  二年 & -543 & \tabularnewline\hline
  三年 & -542 & \tabularnewline\hline
  四年 & -541 & \tabularnewline\hline
  五年 & -540 & \tabularnewline\hline
  六年 & -539 & \tabularnewline\hline
  七年 & -538 & \tabularnewline\hline
  八年 & -537 & \tabularnewline\hline
  九年 & -536 & \tabularnewline\hline
  十年 & -535 & \tabularnewline\hline
  十一年 & -534 & \tabularnewline\hline
  十二年 & -533 & \tabularnewline\hline
  十三年 & -532 & \tabularnewline\hline
  十四年 & -531 & \tabularnewline\hline
  十五年 & -530 & \tabularnewline\hline
  十六年 & -529 & \tabularnewline\hline
  十七年 & -528 & \tabularnewline\hline
  十八年 & -527 & \tabularnewline\hline
  十九年 & -526 & \tabularnewline\hline
  二十年 & -525 & \tabularnewline\hline
  二一年 & -524 & \tabularnewline\hline
  二二年 & -523 & \tabularnewline\hline
  二三年 & -522 & \tabularnewline\hline
  二四年 & -521 & \tabularnewline\hline
  二五年 & -520 & \tabularnewline  
  \bottomrule
\end{longtable}

%%% Local Variables:
%%% mode: latex
%%% TeX-engine: xetex
%%% TeX-master: "../../Main"
%%% End:

%% -*- coding: utf-8 -*-
%% Time-stamp: <Chen Wang: 2021-11-02 15:22:09>

\subsection{悼王猛\tiny{(BC520-BC520)}}

\subsubsection{生平}

周悼王(?-前520年),姓姬,名猛,中國東周君主,諡號悼王,未即位时称王子猛,即位后称王猛。他是周景王的儿子,景王病重時,囑咐大夫賓孟立王子朝。景王死,國人立王子猛為王,是為悼王。悼王後來被王子朝殺死。

周悼王为周景王之子。《左传》记载周景王十八年(前527年),其太子寿和王后去世。其后围绕着周景王的太子人选,朝中形成了对立的两派。一派为支持王子猛的大夫单子单穆公、刘子刘文公。另一派支持王子朝,以王子朝之傅宾起(又称为“宾孟”)为代表。周景王宠爱王子朝,并杀王子猛之傅下门子(见《国语·周语下第三》,取徐元诰《国语集解》之解释)。

周景王二十五年夏四月,景王命令公卿随同自己前往北山田猎。景王计划趁此杀死单子、刘子,然后册立王子朝为太子。但计划尚未实行,周景王突然驾崩,按《左传》的说法是死于心脏病。王子猛即位为王,单子杀宾起,王猛之位初定。即位之时王猛的身份,据《史记·周本纪》是“长子”,按《国语集解》之说是景王朝的太子,对此《左传》无明确记载。《左传·昭公二十六年》记载的王子朝告诸侯文,强调了王后无子的情况下,应该立长子为太子的礼制,而批评单子、刘子为私利而立少子的行为。由此可见,王子朝和王子猛都不是景王的王后之子。而王子朝比王子猛年长。

\subsubsection{年表}

% \centering
\begin{longtable}{|>{\centering\scriptsize}m{2em}|>{\centering\scriptsize}m{1.3em}|>{\centering}m{8.8em}|}
  % \caption{秦王政}\\
  \toprule
  \SimHei \normalsize 年数 & \SimHei \scriptsize 公元 & \SimHei 大事件 \tabularnewline
  % \midrule
  \endfirsthead
  \toprule
  \SimHei \normalsize 年数 & \SimHei \scriptsize 公元 & \SimHei 大事件 \tabularnewline
  \midrule
  \endhead
  \midrule
  元年 & -520 & \tabularnewline  
  \bottomrule
\end{longtable}

%%% Local Variables:
%%% mode: latex
%%% TeX-engine: xetex
%%% TeX-master: "../../Main"
%%% End:

%% -*- coding: utf-8 -*-
%% Time-stamp: <Chen Wang: 2021-11-02 15:24:59>

\subsection{敬王匄\tiny{(BC519-BC476)}}

\subsubsection{生平}

周敬王(前6世纪?-前477年),姬姓,名匄(音同「丐」),中國東周君主,諡號敬王。他是周景王的兒子,周悼王同母弟。

周景王的庶長子王子朝在悼王病死後,自立為王。晉國派兵攻打王子朝,立王子匄為王。此後敬王與王子朝不時仍有衝突。前516年王子朝逃到楚國。前505年春,楚国被吴国击败,险些亡国,周敬王趁机派人在楚地殺死王子朝。儋翩帶領王子朝支持者在次年起兵舉事,敬王出逃,在前503年得晉國幫助下回都。

東周自周平王開始以雒邑(洛邑,又稱成周)為都城,平王東遷後,又稱雒邑為王城。敬王時,因王子朝在雒邑勢大,乃遷都至雒邑之東,稱新都為成周,稱舊都為王城。

周敬王二十六年(前494年),吳王夫差為父報仇,起兵擊敗越國,越王勾踐求和,並獻上美女西施給夫差。《左傳·哀公十九年》記載,冬,周敬王去世,葬于三王陵(今河南省洛阳市西南10里处)。

在位期間執政為單穆公、劉文公、單武公、劉桓公、萇弘、單平公。

\subsubsection{年表}

% \centering
\begin{longtable}{|>{\centering\scriptsize}m{2em}|>{\centering\scriptsize}m{1.3em}|>{\centering}m{8.8em}|}
  % \caption{秦王政}\\
  \toprule
  \SimHei \normalsize 年数 & \SimHei \scriptsize 公元 & \SimHei 大事件 \tabularnewline
  % \midrule
  \endfirsthead
  \toprule
  \SimHei \normalsize 年数 & \SimHei \scriptsize 公元 & \SimHei 大事件 \tabularnewline
  \midrule
  \endhead
  \midrule
  元年 & -519 & \tabularnewline\hline
  二年 & -518 & \tabularnewline\hline
  三年 & -517 & \tabularnewline\hline
  四年 & -516 & \tabularnewline\hline
  五年 & -515 & \tabularnewline\hline
  六年 & -514 & \tabularnewline\hline
  七年 & -513 & \tabularnewline\hline
  八年 & -512 & \tabularnewline\hline
  九年 & -511 & \tabularnewline\hline
  十年 & -510 & \tabularnewline\hline
  十一年 & -509 & \tabularnewline\hline
  十二年 & -508 & \tabularnewline\hline
  十三年 & -507 & \tabularnewline\hline
  十四年 & -506 & \tabularnewline\hline
  十五年 & -505 & \tabularnewline\hline
  十六年 & -504 & \tabularnewline\hline
  十七年 & -503 & \tabularnewline\hline
  十八年 & -502 & \tabularnewline\hline
  十九年 & -501 & \tabularnewline\hline
  二十年 & -500 & \tabularnewline\hline
  二一年 & -499 & \tabularnewline\hline
  二二年 & -498 & \tabularnewline\hline
  二三年 & -497 & \tabularnewline\hline
  二四年 & -496 & \tabularnewline\hline
  二五年 & -495 & \tabularnewline\hline
  二六年 & -494 & \tabularnewline\hline
  二七年 & -493 & \tabularnewline\hline
  二八年 & -492 & \tabularnewline\hline
  二九年 & -491 & \tabularnewline\hline
  三十年 & -490 & \tabularnewline\hline
  三一年 & -489 & \tabularnewline\hline
  三二年 & -488 & \tabularnewline\hline
  三三年 & -487 & \tabularnewline\hline
  三四年 & -486 & \tabularnewline\hline
  三五年 & -485 & \tabularnewline\hline
  三六年 & -484 & \tabularnewline\hline
  三七年 & -483 & \tabularnewline\hline
  三八年 & -482 & \tabularnewline\hline
  三九年 & -481 & \tabularnewline\hline
  四十年 & -480 & \tabularnewline\hline
  四一年 & -479 & \tabularnewline\hline
  四二年 & -478 & \tabularnewline\hline
  四三年 & -477 & \tabularnewline\hline
  四四年 & -476 & \tabularnewline  
  \bottomrule
\end{longtable}

%%% Local Variables:
%%% mode: latex
%%% TeX-engine: xetex
%%% TeX-master: "../../Main"
%%% End:

%% -*- coding: utf-8 -*-
%% Time-stamp: <Chen Wang: 2021-11-02 15:26:32>

\subsection{元王仁\tiny{(BC475-BC469)}}

\subsubsection{生平}

周元王(?-前469年),姓姬,名仁,中國東周君主,在位8年,為周敬王之子。

周元王四年(前473年),越王句踐滅吳。句踐隨後北上遷都琅琊,與齊國、晉國等諸侯會盟於徐州(今山東滕縣南),「越兵橫行於江、淮東,諸侯畢賀,號稱霸王」,周元王正式承認句踐為霸主。

\subsubsection{年表}

% \centering
\begin{longtable}{|>{\centering\scriptsize}m{2em}|>{\centering\scriptsize}m{1.3em}|>{\centering}m{8.8em}|}
  % \caption{秦王政}\\
  \toprule
  \SimHei \normalsize 年数 & \SimHei \scriptsize 公元 & \SimHei 大事件 \tabularnewline
  % \midrule
  \endfirsthead
  \toprule
  \SimHei \normalsize 年数 & \SimHei \scriptsize 公元 & \SimHei 大事件 \tabularnewline
  \midrule
  \endhead
  \midrule
  元年 & -475 & \tabularnewline\hline
  二年 & -474 & \tabularnewline\hline
  三年 & -473 & \tabularnewline\hline
  四年 & -472 & \tabularnewline\hline
  五年 & -471 & \tabularnewline\hline
  六年 & -470 & \tabularnewline\hline
  七年 & -469 & \tabularnewline  
  \bottomrule
\end{longtable}

%%% Local Variables:
%%% mode: latex
%%% TeX-engine: xetex
%%% TeX-master: "../../Main"
%%% End:

%% -*- coding: utf-8 -*-
%% Time-stamp: <Chen Wang: 2021-11-02 15:29:48>

\subsection{貞定王介\tiny{(BC468-BC441)}}

\subsubsection{生平}

周貞定王(?-前441年),姓姬,名介,東周君主,周元王子,在位28年,諡號貞定王。

周貞定王十六年(前453年),晋国大夫韩康子、赵襄子、魏桓子共同攻灭了晉國最大勢力智伯瑤,是為三家滅智。

清朝学者黄式三在其《周季编略》中认为周王介的谥号貞定王的说法是一个错误。他指出,《史记·周本纪》中周王介被称为定王,与周定王瑜同谥,黄式三认为此处史记是沿袭了《国语》的错误记载。黄式三认为皇甫谧在《帝王世纪》中,按照《世本》和《史记》等称周王介为贞王或定王的记载,臆造了周貞定王的称谓,司马贞《史记索隐》就对皇甫谧的做法提出批评。而后世学者多从皇甫谧,黄式三认为应该根据《国语》韦昭注和司马贞《史记索隐》的说法,而称周王介为周贞王。

\subsubsection{年表}

% \centering
\begin{longtable}{|>{\centering\scriptsize}m{2em}|>{\centering\scriptsize}m{1.3em}|>{\centering}m{8.8em}|}
  % \caption{秦王政}\\
  \toprule
  \SimHei \normalsize 年数 & \SimHei \scriptsize 公元 & \SimHei 大事件 \tabularnewline
  % \midrule
  \endfirsthead
  \toprule
  \SimHei \normalsize 年数 & \SimHei \scriptsize 公元 & \SimHei 大事件 \tabularnewline
  \midrule
  \endhead
  \midrule
  元年 & -468 & \tabularnewline\hline
  二年 & -467 & \tabularnewline\hline
  三年 & -466 & \tabularnewline\hline
  四年 & -465 & \tabularnewline\hline
  五年 & -464 & \tabularnewline\hline
  六年 & -463 & \tabularnewline\hline
  七年 & -462 & \tabularnewline\hline
  八年 & -461 & \tabularnewline\hline
  九年 & -460 & \tabularnewline\hline
  十年 & -459 & \tabularnewline\hline
  十一年 & -458 & \tabularnewline\hline
  十二年 & -457 & \tabularnewline\hline
  十三年 & -456 & \tabularnewline\hline
  十四年 & -455 & \tabularnewline\hline
  十五年 & -454 & \tabularnewline\hline
  十六年 & -453 & \tabularnewline\hline
  十七年 & -452 & \tabularnewline\hline
  十八年 & -451 & \tabularnewline\hline
  十九年 & -450 & \tabularnewline\hline
  二十年 & -449 & \tabularnewline\hline
  二一年 & -448 & \tabularnewline\hline
  二二年 & -447 & \tabularnewline\hline
  二三年 & -446 & \tabularnewline\hline
  二四年 & -445 & \tabularnewline\hline
  二五年 & -444 & \tabularnewline\hline
  二六年 & -443 & \tabularnewline\hline
  二七年 & -442 & \tabularnewline\hline
  二八年 & -441 & \tabularnewline  
  \bottomrule
\end{longtable}

%%% Local Variables:
%%% mode: latex
%%% TeX-engine: xetex
%%% TeX-master: "../../Main"
%%% End:

%% -*- coding: utf-8 -*-
%% Time-stamp: <Chen Wang: 2021-11-02 15:31:46>

\subsection{哀王去疾\tiny{(BC441-BC441)}}

\subsubsection{生平}

周哀王(?-前441年),姓姬,名去疾,東周君主,為周貞定王長子。前441年即位,《史記·周本紀》稱哀王在位僅三個月,為弟叔襲殺害,諡號為哀王。

《搜神记·卷六》記载周哀王八年郑国一妇女共生育四十個子女,其中只有二十個長大成人。哀王九年,晉國有頭豬生了個人。

\subsubsection{年表}

% \centering
\begin{longtable}{|>{\centering\scriptsize}m{2em}|>{\centering\scriptsize}m{1.3em}|>{\centering}m{8.8em}|}
  % \caption{秦王政}\\
  \toprule
  \SimHei \normalsize 年数 & \SimHei \scriptsize 公元 & \SimHei 大事件 \tabularnewline
  % \midrule
  \endfirsthead
  \toprule
  \SimHei \normalsize 年数 & \SimHei \scriptsize 公元 & \SimHei 大事件 \tabularnewline
  \midrule
  \endhead
  \midrule
  元年 & -441 & \tabularnewline  
  \bottomrule
\end{longtable}

%%% Local Variables:
%%% mode: latex
%%% TeX-engine: xetex
%%% TeX-master: "../../Main"
%%% End:

%% -*- coding: utf-8 -*-
%% Time-stamp: <Chen Wang: 2021-11-02 15:32:57>

\subsection{思王叔襲\tiny{(BC441-BC441)}}

\subsubsection{生平}

周思王(?-前441年),姓姬,名叔襲,東周君主,為周貞定王之子,周哀王之弟。

前441年,叔襲殺害周哀王即位,為周思王;在位僅五個月,八月又被弟王子嵬所殺。

\subsubsection{年表}

% \centering
\begin{longtable}{|>{\centering\scriptsize}m{2em}|>{\centering\scriptsize}m{1.3em}|>{\centering}m{8.8em}|}
  % \caption{秦王政}\\
  \toprule
  \SimHei \normalsize 年数 & \SimHei \scriptsize 公元 & \SimHei 大事件 \tabularnewline
  % \midrule
  \endfirsthead
  \toprule
  \SimHei \normalsize 年数 & \SimHei \scriptsize 公元 & \SimHei 大事件 \tabularnewline
  \midrule
  \endhead
  \midrule
  元年 & -441 & \tabularnewline
  \bottomrule
\end{longtable}

%%% Local Variables:
%%% mode: latex
%%% TeX-engine: xetex
%%% TeX-master: "../../Main"
%%% End:

%% -*- coding: utf-8 -*-
%% Time-stamp: <Chen Wang: 2021-11-02 15:34:25>

\subsection{考王嵬\tiny{(BC440-BC426)}}

\subsubsection{生平}

周考王(?-前426年),又稱周考哲王,姓姬,名嵬,為中國東周第十九代國王,在位15年,為周貞定王之子、周哀王與周思王之弟。

前441年,姬嵬殺害周思王自立,是為周考王,以前440年為考王元年。前440年,周考王封其弟姬揭於王畿(位於今河南),是為西周桓公。西周桓公死後,子威公立。惠公繼承威公之位,在周顯王二年(前367年)又封少子姬班於鞏(今河南省鞏义市西南),史稱“東周”。此時周朝王畿內再分出「西周」、「東周」两小國,王畿便更為縮小。

周考王時期處於春秋時期與戰國時期之間,或戰國初期。

前426年周考王死去,其子姬午繼位,是為周威烈王。

\subsubsection{年表}

% \centering
\begin{longtable}{|>{\centering\scriptsize}m{2em}|>{\centering\scriptsize}m{1.3em}|>{\centering}m{8.8em}|}
  % \caption{秦王政}\\
  \toprule
  \SimHei \normalsize 年数 & \SimHei \scriptsize 公元 & \SimHei 大事件 \tabularnewline
  % \midrule
  \endfirsthead
  \toprule
  \SimHei \normalsize 年数 & \SimHei \scriptsize 公元 & \SimHei 大事件 \tabularnewline
  \midrule
  \endhead
  \midrule
  元年 & -440 & \tabularnewline\hline
  二年 & -439 & \tabularnewline\hline
  三年 & -438 & \tabularnewline\hline
  四年 & -437 & \tabularnewline\hline
  五年 & -436 & \tabularnewline\hline
  六年 & -435 & \tabularnewline\hline
  七年 & -434 & \tabularnewline\hline
  八年 & -433 & \tabularnewline\hline
  九年 & -432 & \tabularnewline\hline
  十年 & -431 & \tabularnewline\hline
  十一年 & -430 & \tabularnewline\hline
  十二年 & -429 & \tabularnewline\hline
  十三年 & -428 & \tabularnewline\hline
  十四年 & -427 & \tabularnewline\hline
  十五年 & -426 & \tabularnewline
  \bottomrule
\end{longtable}

%%% Local Variables:
%%% mode: latex
%%% TeX-engine: xetex
%%% TeX-master: "../../Main"
%%% End:


%%% Local Variables:
%%% mode: latex
%%% TeX-engine: xetex
%%% TeX-master: "../../Main"
%%% End:
 %東周
% %% -*- coding: utf-8 -*-
%% Time-stamp: <Chen Wang: 2019-12-26 22:28:51>

\section{鲁国}

\subsection{简介}

鲁国,是周朝的一個姬姓諸侯國,為周成王的四叔周公旦及其子伯禽的封国。鲁国先後傳二十五世,三十六位君主,歷時八百餘年。首都在曲阜,疆域在泰山以南,略有今山东省西南部,國力鼎盛時期勢力遍及河南、江蘇及安徽三省。另外,魯国亦是孔子的出生地。

立國:西周初年周公輔佐天子周成王,周公东征打败了伙同武庚叛乱的殷商旧属国,之後周公长子伯禽代替周公前往受封的奄国故土建立鲁国。

伯禽到达封国之后,把曲阜作为自己封国的都城,然后依照周国的制度、习俗来进行治理。因为要去除当地的旧习俗,伯禽前前后后用了三年时间才完成了初步的稳定,然后返回成周报告政绩。而鲁的邻国齐却只用了五个月就返回成周报告结果了,这是因为齐国采取了和鲁国完全相反的政策。齐国的封君简化了周的制度,并依照当地风俗来治理封国,于是很快地稳定下来了。在管叔、蔡叔联合武庚作乱时,东方的淮夷、徐戎等也兴兵作乱,前来攻打鲁国。伯禽率领鲁国的军队前往抵抗,奋战两年最终在周、齐的帮助下平定了鲁国。伯禽在位四十余年,坚持使用周礼治理鲁国,又加上成王赋予了鲁国“郊祭文王”、“奏天子礼乐”的资格,鲁国因此在立国之初就奠定了丰厚的周文化基础。而在后来礼坏乐崩的时代,鲁国则成为了典型周礼的保存者和实施者,世人称“周礼尽在鲁矣”。

周室强藩:周王朝历来有厚同姓、薄异姓的国策,而周成王赋予鲁国“郊祭文王”、“奏天子礼乐”的资格,不仅仅是对周公旦功劳的一种追念,更是希望作为宗邦的鲁国能够“大启尔宇,为周室辅”。这是鲁国在政治上的优势。伐灭管蔡之乱,平定徐戎之叛,鲁国得到“殷民六族”。而本来是王族的殷商之民,拥有较高的文化水平,同时也善于发展经济;而鲁国地处东方海滨,盐铁等重要资源丰富。鲁国历经鲁公伯禽、考公酋(系本作“就”,邹诞本作“遒”)、炀公熙(一作怡,考公弟)、幽公宰(系本名圉)、魏公晞(幽公弟)、厉公擢(系本作“翟”)、献公具(厉公弟)、真公濞(本亦多作“慎公”),一直都是周室强藩,震慑并管理东方,充分发挥了宗邦的作用。此时的鲁国“奄有龟蒙,遂荒大东。至于海邦,淮夷来同”,其国力之强,使得国人和夷狄之民“莫我敢承”、“莫不率从”。这种情形一直延续到春秋,彼时曹、滕、薛、纪、杞、彀、邓、邾、牟、葛诸侯仍旧时常朝觐鲁国。

废长立幼:鲁真公薨,其弟敖立,是为鲁武公。武公有长子括、少子戏。武公九年,武公带着两个儿子,西去朝拜周宣王。宣王很喜欢戏,有意立戏为鲁国的太子。長子括為魯武公的嫡子(與正室夫人所生)、少子戲為魯武公的庶子(與側室夫人所生),依照當時的宗法,只有在正室夫人無子或所生之子死亡時才能立庶子為太子,宣王的做法嚴重犯了宗法的大忌。宣王的卿大夫樊仲山父说:「这个废长立幼,不合规矩。若您執意違背规矩的话,日后鲁国一定会违背您的旨意。」周宣王不顾重臣意见,下命令立戏为鲁国太子。鲁武公郁郁不乐,回到鲁国后就去世了。太子戏繼位,是为鲁懿公。懿公之後被其兄括的儿子伯御殺掉。伯御安安稳稳地做了十一年鲁国国君,最后被周宣王发兵伐魯,把伯御给诛杀了,再立魯懿公的弟弟称,是为鲁孝公。伐魯令周朝天子的威信受損,以後周诸侯国弑其君的事情时有发生。

隐公居摄:鲁孝公薨,子弗湟立,是为鲁惠公。鲁惠公的元配没有生子就死了,妾室声子倒是生了个儿子,名叫做息(一作息姑)。后来,惠公听说宋国有个女子生来手掌就有“鲁夫人”的纹状,于是就把她娶回鲁国,是为仲子。仲子为惠公生了个儿子,名叫做允(一作轨)。惠公没有立太子就死掉了。年长的公子息颇得鲁人的拥戴,于是他摄行君位。但是又担心其他人不服,于是立公子允为惠公太子,说是等他长大后就把政权返给他。历史上把公子息称作“隐公”,谥法:“不尸其位曰隐。”。

隐公时期,卿大夫羽父位高权重,逐渐掌握实权。羽父野心大,渴望与国君平起平坐,何况隐公甚至还不是名义上的国君。羽父要隐公立他为太宰,太宰就是周天子的王室正卿,就地位而言,跟诸侯平起平坐。隐公不答应。羽父于是到太子允前,谗言隐公想要霸占权位。太子允于是授命羽父派人弑杀了隐公。太子允即位,是为鲁桓公。魯隐公及桓公時(前722年-前662年),魯國多次戰勝齊、宋等国,且不断侵襲杞國、莒國等小国。鲁桓公初期,羽父还挺有权势,但是到了后来就不见经传,或许是桓公疏远了他也未可知。

三桓時期:春秋中期之後,魯國政權轉入貴族大臣手中。魯莊公的三个弟弟季友、叔牙及慶父的子孫長期掌握魯國實權,称為季孫氏、叔孫氏、孟孫氏三家,由於三家都是魯桓公之後,被称為「三桓」,魯國從此「政在大夫」。鲁桓公有庶长子庆父、太子同、公子牙、公子友。庆父、叔牙、季友的后代分别是孟氏、叔孙氏及季氏,合称三桓。三桓为孟氏、叔孙氏及季氏,而非孟孙氏、叔孙氏及季孙氏。以往有众多学者认为孟孙、叔孙、季孙皆为氏称,实误。“孙”为尊称,对于孟氏和季氏,“孟孙某”、“季孙某”仅限于宗主的称谓,宗族一般成员只能称“孟某”、“季某”。所以,“孟孙”、“季孙”并不是氏称。考之《左传》,只有“孟氏”、“季氏”的字样,而无“孟孙氏”、“季孙氏”的字样。叔孙氏的情况比较特殊,起先为叔氏,后来公子牙(字子叔)之后立叔氏,原来的叔氏改称叔孙氏。

桓公薨,太子同立,是为鲁庄公。庄公夫人哀姜,哀姜娣叔姜为庄公生子开。庄公晚年,筑高台,看到大夫党氏的女儿孟任,很是欢喜,就跟着她走。最后,庄公许诺说立孟任为夫人,如果她给自己生了儿子,就立为太子。孟任生子般(一作“斑”)。庄公想立般为太子,又担心其他臣子有意见。到了庄公三十二年,庄公病笃,又想到立太子的事情,就询问自己的兄弟叔牙、季友。叔牙说庆父有才能,季友则说就算死也要立公子般。庄公让季友派人赐鸩酒给叔牙。叔牙饮鸩而死,立其后为叔氏,后改称叔孙氏。

鲁庄公立般为太子,而季友辅佐。叔牙死后不久,庄公薨。于是季友立太子般为国君,为庄公治丧,因此尚未正式即位。而庆父发难,派人弑杀了在党氏居住的子般。季友惊慌之间,逃往陈国。庆父与庄公夫人哀姜一向都有私通,因此发难之后,他立哀姜陪嫁的叔姜之子,公子开为国君,是为鲁闵公(一作湣公)。庆父立闵公之后,跟哀姜私通,後來想把闵公也杀了,自己当国君。齐国仲孙湫就预言“不去庆父,鲁难未已”([子说庆父不死,鲁难未已。比喻不清除制造内乱的罪魁祸首,国家就得不到安宁。亦指了结或停止危害的关键事物。)。鲁闵公二年,庆父派大夫卜齮袭杀闵公于武闱。季友听闻,由陈國走到邾國,接庄公妾成风之子申,请鲁人以其为国君。庆父忧惧,出逃到莒國。于是,季友送公子申入鲁,并重金贿赂莒人,抓庆父回国。庆父请求让他出逃,季友不肯。于是庆父自杀。立其后为孟氏。关于孟氏,《春秋》又作仲氏。因为当初庆父虽为长兄,但为了表示君臣之别,于是自称仲,史称共仲。实际上,当时的人都以其年长而叫他的后代为孟氏。

季友立公子申,是为鲁僖公(史记作“釐公”)。僖公元年,季友帅师败“莒师于郦,获莒拏”,“公赐季友汶阳之田及费”,季友为鲁国相。季友相僖公,执政多年,把鲁国治理得井井有条。鲁人作《诗·鲁颂》称赞。僖公十六年,季友卒,谥成,史称“成季”,其后立为季氏。

公卿争权:鲁僖公、文公、宣公、成公、襄公、昭公、定公、哀公及悼公九位鲁侯在位期间,作为卿家的三桓与公室争权夺利,尤其是以季氏的执政与公室的反击最为激烈。鲁穆公时期实行改革,任命博士公仪休为鲁相,才遂渐从三桓手中收回政权。成季死后,庄公的公子遂(即襄仲)及其儿子公孙归父相继掌权,是为东门氏执政时期,而孟氏一度被东门氏赶出鲁国。然而,成季的孙子季孙行父(即季文子)利用三桓的势力,魯宣公十五年(前594年)實行「初税畝」,开初税亩,使得私田兴起,而“隐民”剧增,获得鲁国平民阶层的人心。公子遂杀嫡立庶,以公子俀为国君,是为鲁宣公。

宣公发现三桓日益强盛,同时有民不知君、只知三桓的说法甚嚣尘上,于是他欲去三桓,以张大公室。他与执政的公孙归父商量,是不是起兵灭了三桓,但是国人明显倾心于三桓,使用国内兵马或许不妥。于是,公孙归父前往晋国借兵。可惜公孙归父还没成功搬来晋国军队,宣公就死了,而季文子趁机发难,备述襄仲当政时的弊端,斥责他“南通于楚,既不能固,又不能坚事齐、晋”,使鲁国没有强援。鲁国司寇表示愿意随季文子除乱。公孙归父听到这样的消息,连忙逃到齐国躲起来。季文子开始执政。从此开启了季氏祖孙几代人的执政专权之路。

季文子、季武子、季平子辅佐鲁文公、宣公、成公、襄公、昭公及定公六位鲁侯,位列三卿之首,独专国政。魯成公元年(前590年)行「作丘甲」。季武子时期,通过一系列的政策从不同角度削弱公室的勢力。襄公十一年, 增设三军。季武子、叔孙穆叔、孟献子分三军,一卿主一军之征赋,由是三桓强于公室。当年,周武王封周公旦于鲁,按周礼“天子六军,诸侯大国三军”,鲁有三军。自文公以来,鲁国弱而从霸主之令,若军多则贡多,遂自减中军,只剩上下二军,属于公室,“有事,三卿更帅以征伐”不得专其民。季武子欲专其民,遂增设中军,三桓分三军之民。襄公十二年,三桓“十二分其国民,三家得七,公得五,国民不尽属公,公室已是卑矣”。

昭公外逃:昭公五年,季武子罢中军。四分公室,季孙称左师,孟氏称右师,叔孙氏则自以叔孙为军名,“三家自取其税,减已税以贡于公,国民不复属于公,公室弥益卑矣”。公室奋起反击,昭公二十五年,在郈昭伯、公若等人的劝说下,鲁昭公发兵伐季氏。而孟氏、叔孙氏认为唇亡齿寒,三桓是一荣俱荣、一损俱损,于是发兵救援。结果昭公外逃,而季平子专权,摄行君位将近十年。魯昭公被三家驅逐,客死異鄉。其后不久,三桓属下的家臣陽虎等人控制国政,一度形成“陪臣執国命”的局面。魯定公時(前509年~前495年),陽虎失敗出奔 ,三桓重新掌權 ,後魯哀公(前494年~前468年在位)圖謀恢復君權,同三家大臣衝突加劇,終致流亡越国。

隳三都:季平子的僭越行为,导致其家臣奋起模仿,其中影响最大的莫过于阳虎。定公五年,季平子、叔孙成子相继去世,阳虎发难,囚禁季桓子,逐仲梁怀,随后执掌鲁国权位长达三年。虽然阳虎被三桓赶出了鲁国,但是三桓的影响日渐削弱、公卿之别君臣之礼日渐败坏也成了趋势。这个时候,在位的鲁定公决心削弱三桓,而这个时候三桓内部并不稳定,因为季氏的专权,导致其他两家的不满。定公十年,齐鲁会盟,作为司仪的孔子不仅言谈之间退发难的莱夷之人,更以口舌之利,使得齐国归还汶阳之田。于是,定公以此为契机,重用孔子, 而孔子为了恢复公卿之别、君臣之分,决定以隳三都的方式,逐步消解三桓的强盛势力。季桓子出于防止家臣犯上的考虑,同意隳三都,并派仲由等臣子率兵毁掉自己的费城。然而三桓之中,孟氏反对,他坚持不毁掉自己的成城,结果定公发兵讨伐,却无法攻下。而定公在季氏的唆使下观齐女乐,败坏礼数,更寒了孔子的心。结果,三桓把公室的坚定拥护者孔子赶出了鲁国。

费国独立:魯哀公即位,哀公十二年(前483年)行「用田賦」。哀公十七年(西元前478年),孔子的弟子於曲阜孔子故里建孔庙。根据《史记》的记载,当时孔子的弟子将其“故所居堂”立庙祭祀,庙屋三间,内藏衣、冠、琴、车、书等孔子遗物。哀公想要伐灭三桓,结果反被三桓逐赶,死于有山氏。哀公死后,三桓立公子宁,是为鲁悼公。悼公时期,三桓胜,鲁如小侯,卑于三桓之家。魯元公時(前436年~前416年),三桓逐漸失勢,直到鲁穆公时期(前415年-前383年),鲁国实行改革,任命博士公仪休为鲁相,遂渐从三桓手中收回政权,国政开始奉法循理,摆脱了三桓专政的问题,重新确立了公室的权威。而三桓之一的季氏则据其封邑费、卞,独立成为了费国。

戰国時期

楚灭鲁国:戰国時期魯国國力已衰弱,仍多次與齊国作戰。前323年,鲁景公卒,鲁平公即位,此时正是韩、魏、赵、燕、中山五国相王之年。鲁顷公二年(前278年),秦国破楚国首都郢,楚顷王东迁至陈国。顷公十九年(前261年),楚伐鲁取徐州。顷公二十四年(前256年),鲁国为楚考烈王所灭,迁顷公于下邑,封鲁君于莒。后七年(前249年)鲁顷公死于柯(今山东东阿),鲁国绝祀。

秦朝末年,楚後懷王曾封項羽為魯公。項羽死後,楚地人民都投降漢高祖劉邦,只有魯國不歸順,劉邦本來要以重兵屠殺魯國,後認為魯國長老嚴守禮義,為主死節,所以把項羽的首級拿出來給魯國的長老看,並答應禮葬項羽。之後,劉邦以「魯公」的公爵禮儀,在穀城埋葬項羽,並親自為其哭喪,魯國長老才投降。

汉平帝时期,封鲁顷公八世孙公子宽为褒鲁侯,奉周公祀,公子宽死后谥为“节”,其子公孙相如袭爵。王莽新朝时期,又封公孙相如后裔姬就为褒鲁子。


%% -*- coding: utf-8 -*-
%% Time-stamp: <Chen Wang: 2019-12-26 22:49:48>

\subsection{隐公{\tiny(BC722-BC712)}}

\subsubsection{生平}

魯隱公(?-前712年),姬姓,名息姑,魯国第十四代国君,前722年-前712年在位。魯惠公之子,生母是声子。傳世的魯國史書《春秋》及其三傳的記事都是從魯隱公開始的。

公元前722年,魯惠公死,嫡妻所生的公子軌當時還年幼,所以國人共立息姑攝政,行君事。在位十一年,隱公始終牢記自己是攝政行君事,一心等待公子軌長大,把國君的位置禪讓給他。因此,當公子翬提出請求殺死公子軌時,隱公断然拒絶;結果公子翬怕消息走漏,反而先跟公子軌合作,把隱公刺殺而死,公子軌即位,是為魯桓公。

隱公行君事期間,重視政治、外交,魯國國力較強。隱公除了到棠地觀人捕魚,被認為不合于禮法之外,處理政事、軍事都還比較謹慎公正,與鄰國修好,所以周圍小國如滕國、薛國等國都到魯國朝拜。與鄭國、齊國等強國也結好。

在位期間的卿為公子翬、无骇、公子益师、公子彄、挟、公子豫。

\subsubsection{年表}

% \centering
\begin{longtable}{|>{\centering\scriptsize}m{2em}|>{\centering\scriptsize}m{1.3em}|>{\centering}m{8.8em}|}
  % \caption{秦王政}\\
  \toprule
  \SimHei \normalsize 年数 & \SimHei \scriptsize 公元 & \SimHei 大事件 \tabularnewline
  % \midrule
  \endfirsthead
  \toprule
  \SimHei \normalsize 年数 & \SimHei \scriptsize 公元 & \SimHei 大事件 \tabularnewline
  \midrule
  \endhead
  \midrule
  元年 & -722 & \begin{enumerate}
    \tiny
  \item 三月,公及\xpinyin*{邾}儀父盟于蔑\footnote{公及邾儀父盟于蔑,邾子克也,未王命,故不書爵,曰儀父,貴之也,公攝位,而欲求好於邾,故為蔑之盟。}。
  \item 夏,五月,鄭伯克段于鄢\footnote{初,鄭武公娶于申,曰武姜。生莊公及共叔段。莊公寤生,驚姜氏,故名曰寤生,遂惡之。愛共叔段,欲立之。亟請於武公,公弗許。及莊公即位,為之請制。公曰:「制,巖邑也,虢叔死焉,佗邑唯命。」請京,使居之,謂之「京城大叔」。祭仲曰:「都城過百雉,國之害也,先王之制:大都不過參國之一;中,五之一;小,九之一。今京不度,非制也,君將不堪。」公曰:「姜氏欲之,焉辟害?」對曰:「姜氏何厭之有!不如早為之所,無使滋蔓。蔓,難圖也。蔓草猶不可除,況君之寵弟乎!」公曰:「多行不義,必自斃,子姑待之。」既而大叔命西鄙、北鄙貳於己。公子呂曰:「國不堪貳,君將若之何?欲與大叔,臣請事之;若弗與,則請除之,無生民心。」公曰:「無庸,將自及。」大叔又收貳以為己邑,至于廩延。子封曰:「可矣,厚將得眾。」公曰:「不義不暱,厚將崩。」大叔完聚,繕甲兵,具卒乘,將襲鄭。夫人將啟之。公聞其期,曰:「可矣。」命子封帥車二百乘以伐京。京叛大叔段,段入于鄢,公伐諸鄢。五月辛丑,大叔出奔共。}。
  \item 八月,紀人伐夷。
  \item 秋,七月,天王使宰咺來歸惠公仲子之賵\footnote{天子七月而葬,同軌畢至,諸侯五月,同盟至,大夫三月,同位至,士踰月,外姻至,贈死不及尸,弔生不及哀,豫凶事,非禮也。}。
  \item 九月,及宋人盟于宿。
  \end{enumerate} \tabularnewline\hline
  二年 & -721 & \tabularnewline\hline
  三年 & -720 & \tabularnewline\hline
  四年 & -719 & \tabularnewline\hline
  五年 & -718 & \tabularnewline\hline
  六年 & -717 & \tabularnewline\hline
  七年 & -716 & \tabularnewline\hline
  八年 & -715 & \tabularnewline\hline
  九年 & -714 & \tabularnewline\hline
  十年 & -713 & \tabularnewline\hline
  十一年 & -712 & \tabularnewline
  \bottomrule
\end{longtable}



%%% Local Variables:
%%% mode: latex
%%% TeX-engine: xetex
%%% TeX-master: "../../Main"
%%% End:

% %% -*- coding: utf-8 -*-
%% Time-stamp: <Chen Wang: 2018-07-12 22:55:13>

\subsection{桓公{\tiny(BC711-BC694)}}

% \centering
\begin{longtable}{|>{\centering\scriptsize}m{2em}|>{\centering\scriptsize}m{1.3em}|>{\centering}m{8.8em}|}
  % \caption{秦王政}\\
  \toprule
  \SimHei \normalsize 年数 & \SimHei \scriptsize 公元 & \SimHei 大事件 \tabularnewline
  % \midrule
  \endfirsthead
  \toprule
  \SimHei \normalsize 年数 & \SimHei \scriptsize 公元 & \SimHei 大事件 \tabularnewline
  \midrule
  \endhead
  \midrule
  元年 & -711 & \tabularnewline\hline
  二年 & -710 & \tabularnewline\hline
  三年 & -709 & \tabularnewline\hline
  四年 & -708 & \tabularnewline\hline
  五年 & -707 & \tabularnewline\hline
  六年 & -706 & \tabularnewline\hline
  七年 & -705 & \tabularnewline\hline
  八年 & -704 & \tabularnewline\hline
  九年 & -703 & \tabularnewline\hline
  十年 & -702 & \tabularnewline\hline
  十一年 & -701 & \tabularnewline\hline
  十二年 & -700 & \tabularnewline\hline
  十三年 & -699 & \tabularnewline\hline
  十四年 & -698 & \tabularnewline\hline
  十五年 & -697 & \tabularnewline\hline
  十六年 & -696 & \tabularnewline\hline
  十七年 & -695 & \tabularnewline\hline
  十八年 & -694 & \tabularnewline
  \bottomrule
\end{longtable}

%%% Local Variables:
%%% mode: latex
%%% TeX-engine: xetex
%%% TeX-master: "../../Main"
%%% End:

% %% -*- coding: utf-8 -*-
%% Time-stamp: <Chen Wang: 2018-07-12 23:02:54>

\subsection{庄公{\tiny(BC693-BC662)}}

% \centering
\begin{longtable}{|>{\centering\scriptsize}m{2em}|>{\centering\scriptsize}m{1.3em}|>{\centering}m{8.8em}|}
  % \caption{秦王政}\\
  \toprule
  \SimHei \normalsize 年数 & \SimHei \scriptsize 公元 & \SimHei 大事件 \tabularnewline
  % \midrule
  \endfirsthead
  \toprule
  \SimHei \normalsize 年数 & \SimHei \scriptsize 公元 & \SimHei 大事件 \tabularnewline
  \midrule
  \endhead
  \midrule
  元年 & -693 & \tabularnewline\hline
  二年 & -692 & \tabularnewline\hline
  三年 & -691 & \tabularnewline\hline
  四年 & -690 & \tabularnewline\hline
  五年 & -689 & \tabularnewline\hline
  六年 & -688 & \tabularnewline\hline
  七年 & -687 & \tabularnewline\hline
  八年 & -686 & \tabularnewline\hline
  九年 & -685 & \tabularnewline\hline
  十年 & -684 & \tabularnewline\hline
  十一年 & -683 & \tabularnewline\hline
  十二年 & -682 & \tabularnewline\hline
  十三年 & -681 & \tabularnewline\hline
  十四年 & -680 & \tabularnewline\hline
  十五年 & -679 & \tabularnewline\hline
  十六年 & -678 & \tabularnewline\hline
  十七年 & -677 & \tabularnewline\hline
  十八年 & -676 & \tabularnewline\hline
  十九年 & -675 & \tabularnewline\hline
  二十年 & -674 & \tabularnewline\hline
  二一年 & -673 & \tabularnewline\hline
  二二年 & -672 & \tabularnewline\hline
  二三年 & -671 & \tabularnewline\hline
  二四年 & -670 & \tabularnewline\hline
  二五年 & -669 & \tabularnewline\hline
  二六年 & -668 & \tabularnewline\hline
  二七年 & -667 & \tabularnewline\hline
  二八年 & -666 & \tabularnewline\hline
  二九年 & -665 & \tabularnewline\hline
  三十年 & -664 & \tabularnewline\hline
  三一年 & -663 & \tabularnewline\hline
  三二年 & -662 & \tabularnewline
  \bottomrule
\end{longtable}

%%% Local Variables:
%%% mode: latex
%%% TeX-engine: xetex
%%% TeX-master: "../../Main"
%%% End:

% %% -*- coding: utf-8 -*-
%% Time-stamp: <Chen Wang: 2021-11-01 17:59:15>

\subsection{闵公啟{\tiny(BC661-BC660)}}

\subsubsection{子斑生平}

子般(?-前662年),姬姓,名般,一作斑,,称子表示此时是其父鲁庄公死的当年,魯莊公之子。魯莊公的夫人哀姜是齊國人,無子。莊公臨死前欲立庶子般為嗣君,莊公弟叔牙建議立庄公庶长兄公子慶父,另一位弟弟季友則支持立子般,季友於是借莊公之命赐死叔牙。

三十二年八月,莊公病逝,季友立子般為君,十月慶父殺子般,立莊公另一庶子啟為魯君,是為魯閔公。季友逃亡陳國。

\subsubsection{閔公啟生平}

魯閔公(前669年?-前660年),即姬啟,一名啟方,為春秋諸侯國魯國君主之一,是魯國第十七任君主。他為魯莊公、叔姜的兒子。近人考証謂於周惠王八年(前669年)出生,至周惠王十七年(前660年)去世,年約十歲。(楊伯峻《春秋左傳注》,頁254)

莊公死前,弟弟叔牙建議立莊公庶長兄慶父,另一位弟弟季友則支持立子般,季友於是借莊公之命賜死叔牙,莊公病逝,季友立子般為君,十月慶父殺子般,立莊公另一庶子啟為魯君,即魯閔公,魯閔公亦是齊桓公的外甥,對齊桓公很尊敬,因此齊魯無大事,直到兩年後公子慶父以毒餅殺死魯閔公,齊桓公才派兵迎立魯閔公之弟魯釐公。

在位期間的卿為公子慶父、季友。

\subsubsection{年表}

% \centering
\begin{longtable}{|>{\centering\scriptsize}m{2em}|>{\centering\scriptsize}m{1.3em}|>{\centering}m{8.8em}|}
  % \caption{秦王政}\\
  \toprule
  \SimHei \normalsize 年数 & \SimHei \scriptsize 公元 & \SimHei 大事件 \tabularnewline
  % \midrule
  \endfirsthead
  \toprule
  \SimHei \normalsize 年数 & \SimHei \scriptsize 公元 & \SimHei 大事件 \tabularnewline
  \midrule
  \endhead
  \midrule
  元年 & -661 & \tabularnewline\hline
  二年 & -660 & \tabularnewline
  \bottomrule
\end{longtable}

%%% Local Variables:
%%% mode: latex
%%% TeX-engine: xetex
%%% TeX-master: "../../Main"
%%% End:

% %% -*- coding: utf-8 -*-
%% Time-stamp: <Chen Wang: 2018-07-12 23:08:29>

\subsection{僖公{\tiny(BC659-BC627)}}

% \centering
\begin{longtable}{|>{\centering\scriptsize}m{2em}|>{\centering\scriptsize}m{1.3em}|>{\centering}m{8.8em}|}
  % \caption{秦王政}\\
  \toprule
  \SimHei \normalsize 年数 & \SimHei \scriptsize 公元 & \SimHei 大事件 \tabularnewline
  % \midrule
  \endfirsthead
  \toprule
  \SimHei \normalsize 年数 & \SimHei \scriptsize 公元 & \SimHei 大事件 \tabularnewline
  \midrule
  \endhead
  \midrule
  元年 & -659 & \tabularnewline\hline
  二年 & -658 & \tabularnewline\hline
  三年 & -657 & \tabularnewline\hline
  四年 & -656 & \tabularnewline\hline
  五年 & -655 & \tabularnewline\hline
  六年 & -654 & \tabularnewline\hline
  七年 & -653 & \tabularnewline\hline
  八年 & -652 & \tabularnewline\hline
  九年 & -651 & \tabularnewline\hline
  十年 & -650 & \tabularnewline\hline
  十一年 & -649 & \tabularnewline\hline
  十二年 & -648 & \tabularnewline\hline
  十三年 & -647 & \tabularnewline\hline
  十四年 & -646 & \tabularnewline\hline
  十五年 & -645 & \tabularnewline\hline
  十六年 & -644 & \tabularnewline\hline
  十七年 & -643 & \tabularnewline\hline
  十八年 & -642 & \tabularnewline\hline
  十九年 & -641 & \tabularnewline\hline
  二十年 & -640 & \tabularnewline\hline
  二一年 & -639 & \tabularnewline\hline
  二二年 & -638 & \tabularnewline\hline
  二三年 & -637 & \tabularnewline\hline
  二四年 & -636 & \tabularnewline\hline
  二五年 & -635 & \tabularnewline\hline
  二六年 & -634 & \tabularnewline\hline
  二七年 & -633 & \tabularnewline\hline
  二八年 & -632 & \tabularnewline\hline
  二九年 & -631 & \tabularnewline\hline
  三十年 & -630 & \tabularnewline\hline
  三一年 & -629 & \tabularnewline\hline
  三二年 & -628 & \tabularnewline\hline
  三三年 & -627 & \tabularnewline
  \bottomrule
\end{longtable}

%%% Local Variables:
%%% mode: latex
%%% TeX-engine: xetex
%%% TeX-master: "../../Main"
%%% End:

% %% -*- coding: utf-8 -*-
%% Time-stamp: <Chen Wang: 2018-07-12 23:10:00>

\subsection{文公{\tiny(BC626-BC609)}}

% \centering
\begin{longtable}{|>{\centering\scriptsize}m{2em}|>{\centering\scriptsize}m{1.3em}|>{\centering}m{8.8em}|}
  % \caption{秦王政}\\
  \toprule
  \SimHei \normalsize 年数 & \SimHei \scriptsize 公元 & \SimHei 大事件 \tabularnewline
  % \midrule
  \endfirsthead
  \toprule
  \SimHei \normalsize 年数 & \SimHei \scriptsize 公元 & \SimHei 大事件 \tabularnewline
  \midrule
  \endhead
  \midrule
  元年 & -626 & \tabularnewline\hline
  二年 & -625 & \tabularnewline\hline
  三年 & -624 & \tabularnewline\hline
  四年 & -623 & \tabularnewline\hline
  五年 & -622 & \tabularnewline\hline
  六年 & -621 & \tabularnewline\hline
  七年 & -620 & \tabularnewline\hline
  八年 & -619 & \tabularnewline\hline
  九年 & -618 & \tabularnewline\hline
  十年 & -617 & \tabularnewline\hline
  十一年 & -616 & \tabularnewline\hline
  十二年 & -615 & \tabularnewline\hline
  十三年 & -614 & \tabularnewline\hline
  十四年 & -613 & \tabularnewline\hline
  十五年 & -612 & \tabularnewline\hline
  十六年 & -611 & \tabularnewline\hline
  十七年 & -610 & \tabularnewline\hline
  十八年 & -609 & \tabularnewline
  \bottomrule
\end{longtable}

%%% Local Variables:
%%% mode: latex
%%% TeX-engine: xetex
%%% TeX-master: "../../Main"
%%% End:

% %% -*- coding: utf-8 -*-
%% Time-stamp: <Chen Wang: 2018-07-12 23:11:20>

\subsection{宣公{\tiny(BC608-BC591)}}

% \centering
\begin{longtable}{|>{\centering\scriptsize}m{2em}|>{\centering\scriptsize}m{1.3em}|>{\centering}m{8.8em}|}
  % \caption{秦王政}\\
  \toprule
  \SimHei \normalsize 年数 & \SimHei \scriptsize 公元 & \SimHei 大事件 \tabularnewline
  % \midrule
  \endfirsthead
  \toprule
  \SimHei \normalsize 年数 & \SimHei \scriptsize 公元 & \SimHei 大事件 \tabularnewline
  \midrule
  \endhead
  \midrule
  元年 & -608 & \tabularnewline\hline
  二年 & -607 & \tabularnewline\hline
  三年 & -606 & \tabularnewline\hline
  四年 & -605 & \tabularnewline\hline
  五年 & -604 & \tabularnewline\hline
  六年 & -603 & \tabularnewline\hline
  七年 & -602 & \tabularnewline\hline
  八年 & -601 & \tabularnewline\hline
  九年 & -600 & \tabularnewline\hline
  十年 & -599 & \tabularnewline\hline
  十一年 & -598 & \tabularnewline\hline
  十二年 & -597 & \tabularnewline\hline
  十三年 & -596 & \tabularnewline\hline
  十四年 & -595 & \tabularnewline\hline
  十五年 & -594 & \tabularnewline\hline
  十六年 & -593 & \tabularnewline\hline
  十七年 & -592 & \tabularnewline\hline
  十八年 & -591 & \tabularnewline
  \bottomrule
\end{longtable}

%%% Local Variables:
%%% mode: latex
%%% TeX-engine: xetex
%%% TeX-master: "../../Main"
%%% End:

% %% -*- coding: utf-8 -*-
%% Time-stamp: <Chen Wang: 2021-11-01 17:57:14>

\subsection{成公黑肱{\tiny(BC590-BC573)}}

\subsubsection{生平}

魯成公(?-前573年),姬姓,名黑肱,為東周春秋時期諸侯國魯國的一位君主,是魯國第二十一任君主,承襲父親魯宣公擔任該國君主,在位18年。

在位期間執政為季孫行父、仲孫蔑、叔孫僑如。

魯成公元年(前590年),季孫行父在魯國實行一種「作丘(地區單位)甲」的地區編制。

魯成公二年(前589年)春天,齊國要攻打魯、衛兩國。魯、衛兩國大夫請求晉國出兵,晉國以郤克為主將率兵救討伐齊國,以救援魯、衛二國。同一年,齊頃公親率齊軍南下攻打魯國龍邑(今山東泰安東南),寵臣盧蒲就癸被殺,頃公怒而攻至巢丘(今山東泰安境內)。季孫行父率魯軍幫晉、衛、曹等國,去攻打齊國的鞍(今山東濟南市)。齊頃公在鞍之戰大敗,齊頃公被晉軍追逼,「差點被俘,幸得大夫逢丑父相救,二人互換衣服,佯命齊頃公到山腳華泉取水,得以逃走。同年十一月,魯國的魯成公同蔡景侯、許靈公、秦國右大夫說、宋國華元、陳國公孫寧、衛國孫良夫、鄭國子良、齊國大夫、曹、邾、薛、鄫等多國代表參與由楚國公子嬰齊在蜀(今山東省泰安市東南)所主辦的會盟。

魯成公七年(前584年),吳國攻打鄰近魯國的郯國,郯國被納入吳國的領土的事,因此季孫行父向魯國國君發出「中國不振旅,蠻夷(指吳國)來伐」的警告。

魯成公十六年(前575年),魯成公的夫人定姒產下魯成公的兒子姬午。

魯成公十八年(前573年),魯成公薨,姬午即位(即後來的魯襄公)。

\subsubsection{年表}

% \centering
\begin{longtable}{|>{\centering\scriptsize}m{2em}|>{\centering\scriptsize}m{1.3em}|>{\centering}m{8.8em}|}
  % \caption{秦王政}\\
  \toprule
  \SimHei \normalsize 年数 & \SimHei \scriptsize 公元 & \SimHei 大事件 \tabularnewline
  % \midrule
  \endfirsthead
  \toprule
  \SimHei \normalsize 年数 & \SimHei \scriptsize 公元 & \SimHei 大事件 \tabularnewline
  \midrule
  \endhead
  \midrule
  元年 & -590 & \tabularnewline\hline
  二年 & -589 & \tabularnewline\hline
  三年 & -588 & \tabularnewline\hline
  四年 & -587 & \tabularnewline\hline
  五年 & -586 & \tabularnewline\hline
  六年 & -585 & \tabularnewline\hline
  七年 & -584 & \tabularnewline\hline
  八年 & -583 & \tabularnewline\hline
  九年 & -582 & \tabularnewline\hline
  十年 & -581 & \tabularnewline\hline
  十一年 & -580 & \tabularnewline\hline
  十二年 & -579 & \tabularnewline\hline
  十三年 & -578 & \tabularnewline\hline
  十四年 & -577 & \tabularnewline\hline
  十五年 & -576 & \tabularnewline\hline
  十六年 & -575 & \tabularnewline\hline
  十七年 & -574 & \tabularnewline\hline
  十八年 & -573 & \tabularnewline
  \bottomrule
\end{longtable}

%%% Local Variables:
%%% mode: latex
%%% TeX-engine: xetex
%%% TeX-master: "../../Main"
%%% End:

% %% -*- coding: utf-8 -*-
%% Time-stamp: <Chen Wang: 2021-11-01 17:58:32>

\subsection{襄公午{\tiny(BC572-BC542)}}

\subsubsection{生平}

魯襄公(前575年-前542年),姬姓,名午,春秋時代魯國的第二十二代君主,魯成公之子,於魯成公十六年(公元前575年)誕生。

魯成公十八年(前573年),魯成公去世,由四歲的太子午即君主之位,是為魯襄公。執政為正卿司徒季孫行父,保持魯國的相對穩定。

魯襄公五年(前568年),魯襄公九歲,執政正卿為季孫行父、仲孫蔑。季孫行父去世,行父以薄葬來進行下葬儀式,這時魯襄公很感動地稱讚的說行父是個廉吏,於是襄公給行父的諡號為「文」。

\subsubsection{年表}

% \centering
\begin{longtable}{|>{\centering\scriptsize}m{2em}|>{\centering\scriptsize}m{1.3em}|>{\centering}m{8.8em}|}
  % \caption{秦王政}\\
  \toprule
  \SimHei \normalsize 年数 & \SimHei \scriptsize 公元 & \SimHei 大事件 \tabularnewline
  % \midrule
  \endfirsthead
  \toprule
  \SimHei \normalsize 年数 & \SimHei \scriptsize 公元 & \SimHei 大事件 \tabularnewline
  \midrule
  \endhead
  \midrule
  元年 & -572 & \tabularnewline\hline
  二年 & -571 & \tabularnewline\hline
  三年 & -570 & \tabularnewline\hline
  四年 & -569 & \tabularnewline\hline
  五年 & -568 & \tabularnewline\hline
  六年 & -567 & \tabularnewline\hline
  七年 & -566 & \tabularnewline\hline
  八年 & -565 & \tabularnewline\hline
  九年 & -564 & \tabularnewline\hline
  十年 & -563 & \tabularnewline\hline
  十一年 & -562 & \tabularnewline\hline
  十二年 & -561 & \tabularnewline\hline
  十三年 & -560 & \tabularnewline\hline
  十四年 & -559 & \tabularnewline\hline
  十五年 & -558 & \tabularnewline\hline
  十六年 & -557 & \tabularnewline\hline
  十七年 & -556 & \tabularnewline\hline
  十八年 & -555 & \tabularnewline\hline
  十九年 & -554 & \tabularnewline\hline
  二十年 & -553 & \tabularnewline\hline
  二一年 & -552 & \tabularnewline\hline
  二二年 & -551 & \tabularnewline\hline
  二三年 & -550 & \tabularnewline\hline
  二四年 & -549 & \tabularnewline\hline
  二五年 & -548 & \tabularnewline\hline
  二六年 & -547 & \tabularnewline\hline
  二七年 & -546 & \tabularnewline\hline
  二八年 & -545 & \tabularnewline\hline
  二九年 & -544 & \tabularnewline\hline
  三十年 & -543 & \tabularnewline\hline
  三一年 & -542 & \tabularnewline
  \bottomrule
\end{longtable}

%%% Local Variables:
%%% mode: latex
%%% TeX-engine: xetex
%%% TeX-master: "../../Main"
%%% End:

% %% -*- coding: utf-8 -*-
%% Time-stamp: <Chen Wang: 2021-11-01 18:02:17>

\subsection{昭公稠{\tiny(BC541-BC510)}}

\subsubsection{子野生平}

子野(?-前542年),姬姓,名野,子表示此时是其父鲁襄公死的当年。子野是魯襄公的庶子,魯昭公和鲁定公之兄,母为敬归。

前542年六月二十八日,魯襄公去世,鲁国人拥立太子子野即位,住在季氏那里。九月十一日,子野由于哀痛过度而死。

魯國人便擁立子野生母敬归的妹妹齊歸生的儿子公子裯為国君,是為魯昭公。

\subsubsection{昭公稠生平}

魯昭公(?-前510年),姬姓,名稠,魯国之二十四代君主。前542年即位,前517年,魯昭公伐季孙氏,但大败,魯昭公逃到齐国,前510年,昭公死。在其任內,他嘗試與季平子政治角力,演變成「鬥雞之變」,使昭公逃到齊國。

在位期間執政為季孫宿、叔孫婼、仲孫貜。

魯昭公二十三年(前519年),叔孫昭子將魯政讓位給季孫意如。

鬥雞之變後,在位期間執政為仲孫何忌、叔孫不敢。

\subsubsection{年表}

% \centering
\begin{longtable}{|>{\centering\scriptsize}m{2em}|>{\centering\scriptsize}m{1.3em}|>{\centering}m{8.8em}|}
  % \caption{秦王政}\\
  \toprule
  \SimHei \normalsize 年数 & \SimHei \scriptsize 公元 & \SimHei 大事件 \tabularnewline
  % \midrule
  \endfirsthead
  \toprule
  \SimHei \normalsize 年数 & \SimHei \scriptsize 公元 & \SimHei 大事件 \tabularnewline
  \midrule
  \endhead
  \midrule
  元年 & -541 & \tabularnewline\hline
  二年 & -540 & \tabularnewline\hline
  三年 & -539 & \tabularnewline\hline
  四年 & -538 & \tabularnewline\hline
  五年 & -537 & \tabularnewline\hline
  六年 & -536 & \tabularnewline\hline
  七年 & -535 & \tabularnewline\hline
  八年 & -534 & \tabularnewline\hline
  九年 & -533 & \tabularnewline\hline
  十年 & -532 & \tabularnewline\hline
  十一年 & -531 & \tabularnewline\hline
  十二年 & -530 & \tabularnewline\hline
  十三年 & -529 & \tabularnewline\hline
  十四年 & -528 & \tabularnewline\hline
  十五年 & -527 & \tabularnewline\hline
  十六年 & -526 & \tabularnewline\hline
  十七年 & -525 & \tabularnewline\hline
  十八年 & -524 & \tabularnewline\hline
  十九年 & -523 & \tabularnewline\hline
  二十年 & -522 & \tabularnewline\hline
  二一年 & -521 & \tabularnewline\hline
  二二年 & -520 & \tabularnewline\hline
  二三年 & -519 & \tabularnewline\hline
  二四年 & -518 & \tabularnewline\hline
  二五年 & -517 & \tabularnewline\hline
  二六年 & -516 & \tabularnewline\hline
  二七年 & -515 & \tabularnewline\hline
  二八年 & -514 & \tabularnewline\hline
  二九年 & -513 & \tabularnewline\hline
  三十年 & -512 & \tabularnewline\hline
  三一年 & -511 & \tabularnewline\hline
  三二年 & -510 & \tabularnewline
  \bottomrule
\end{longtable}

%%% Local Variables:
%%% mode: latex
%%% TeX-engine: xetex
%%% TeX-master: "../../Main"
%%% End:

% %% -*- coding: utf-8 -*-
%% Time-stamp: <Chen Wang: 2021-11-01 18:02:49>

\subsection{定公宋{\tiny(BC509-BC495)}}

\subsubsection{生平}

魯定公(前556年-前495年),姬姓,名宋,為中國春秋時期諸侯國魯國君主之一,是魯國第二十五任君主。他為魯昭公的庶弟,承襲魯昭公擔任該國君主,在位15年。

在位期間執政為季孫意如、叔孫不敢、仲孫何忌、季孫斯、叔孫州仇,其中公元前505年~前503年,執政主官為季孫氏家宰陽虎。前501年~前497年,任命孔子為大司寇。期間行攝相事。

\subsubsection{年表}

% \centering
\begin{longtable}{|>{\centering\scriptsize}m{2em}|>{\centering\scriptsize}m{1.3em}|>{\centering}m{8.8em}|}
  % \caption{秦王政}\\
  \toprule
  \SimHei \normalsize 年数 & \SimHei \scriptsize 公元 & \SimHei 大事件 \tabularnewline
  % \midrule
  \endfirsthead
  \toprule
  \SimHei \normalsize 年数 & \SimHei \scriptsize 公元 & \SimHei 大事件 \tabularnewline
  \midrule
  \endhead
  \midrule
  元年 & -509 & \tabularnewline\hline
  二年 & -508 & \tabularnewline\hline
  三年 & -507 & \tabularnewline\hline
  四年 & -506 & \tabularnewline\hline
  五年 & -505 & \tabularnewline\hline
  六年 & -504 & \tabularnewline\hline
  七年 & -503 & \tabularnewline\hline
  八年 & -502 & \tabularnewline\hline
  九年 & -501 & \tabularnewline\hline
  十年 & -500 & \tabularnewline\hline
  十一年 & -499 & \tabularnewline\hline
  十二年 & -498 & \tabularnewline\hline
  十三年 & -497 & \tabularnewline\hline
  十四年 & -496 & \tabularnewline\hline
  十五年 & -495 & \tabularnewline
  \bottomrule
\end{longtable}

%%% Local Variables:
%%% mode: latex
%%% TeX-engine: xetex
%%% TeX-master: "../../Main"
%%% End:

% %% -*- coding: utf-8 -*-
%% Time-stamp: <Chen Wang: 2021-11-01 18:03:17>

\subsection{哀公將{\tiny(BC494-BC467)}}

\subsubsection{生平}

鲁哀公(约前508年-前468年),姬姓,名將,為春秋諸侯國魯國君主之一,是魯國第二十六任君主。魯定公之子,承襲魯定公擔任該國君主,在位27年。

鲁哀公在位时,鲁国大权被卿大夫家族把持,史称三桓,即所谓“政在大夫”。鲁哀公曾经试图恢复君主权力,同三家大夫冲突加剧,终致流亡越国。魯哀公27年,鲁哀公通过邾国逃到越国。

在位期間執政的士大夫為季孫斯、叔孫州仇、仲孫何忌、季孫肥、叔孫舒、仲孫彘。

\subsubsection{年表}

% \centering
\begin{longtable}{|>{\centering\scriptsize}m{2em}|>{\centering\scriptsize}m{1.3em}|>{\centering}m{8.8em}|}
  % \caption{秦王政}\\
  \toprule
  \SimHei \normalsize 年数 & \SimHei \scriptsize 公元 & \SimHei 大事件 \tabularnewline
  % \midrule
  \endfirsthead
  \toprule
  \SimHei \normalsize 年数 & \SimHei \scriptsize 公元 & \SimHei 大事件 \tabularnewline
  \midrule
  \endhead
  \midrule
  元年 & -494 & \tabularnewline\hline
  二年 & -493 & \tabularnewline\hline
  三年 & -492 & \tabularnewline\hline
  四年 & -491 & \tabularnewline\hline
  五年 & -490 & \tabularnewline\hline
  六年 & -489 & \tabularnewline\hline
  七年 & -488 & \tabularnewline\hline
  八年 & -487 & \tabularnewline\hline
  九年 & -486 & \tabularnewline\hline
  十年 & -485 & \tabularnewline\hline
  十一年 & -484 & \tabularnewline\hline
  十二年 & -483 & \tabularnewline\hline
  十三年 & -482 & \tabularnewline\hline
  十四年 & -481 & \tabularnewline\hline
  十五年 & -480 & \tabularnewline\hline
  十六年 & -479 & \tabularnewline\hline
  十七年 & -478 & \tabularnewline\hline
  十八年 & -477 & \tabularnewline\hline
  十九年 & -476 & \tabularnewline\hline
  二十年 & -475 & \tabularnewline\hline
  二一年 & -474 & \tabularnewline\hline
  二二年 & -473 & \tabularnewline\hline
  二三年 & -472 & \tabularnewline\hline
  二四年 & -471 & \tabularnewline\hline
  二五年 & -470 & \tabularnewline\hline
  二六年 & -469 & \tabularnewline\hline
  二七年 & -468 & \tabularnewline\hline
  二八年 & -467 & \tabularnewline
  \bottomrule
\end{longtable}

%%% Local Variables:
%%% mode: latex
%%% TeX-engine: xetex
%%% TeX-master: "../../Main"
%%% End:

% %% -*- coding: utf-8 -*-
%% Time-stamp: <Chen Wang: 2018-07-12 23:19:27>

\subsection{悼公{\tiny(BC466-BC429)}}

% \centering
\begin{longtable}{|>{\centering\scriptsize}m{2em}|>{\centering\scriptsize}m{1.3em}|>{\centering}m{8.8em}|}
  % \caption{秦王政}\\
  \toprule
  \SimHei \normalsize 年数 & \SimHei \scriptsize 公元 & \SimHei 大事件 \tabularnewline
  % \midrule
  \endfirsthead
  \toprule
  \SimHei \normalsize 年数 & \SimHei \scriptsize 公元 & \SimHei 大事件 \tabularnewline
  \midrule
  \endhead
  \midrule
  元年 & -466 & \tabularnewline\hline
  二年 & -465 & \tabularnewline\hline
  三年 & -464 & \tabularnewline\hline
  四年 & -463 & \tabularnewline\hline
  五年 & -462 & \tabularnewline\hline
  六年 & -461 & \tabularnewline\hline
  七年 & -460 & \tabularnewline\hline
  八年 & -459 & \tabularnewline\hline
  九年 & -458 & \tabularnewline\hline
  十年 & -457 & \tabularnewline\hline
  十一年 & -456 & \tabularnewline\hline
  十二年 & -455 & \tabularnewline\hline
  十三年 & -454 & \tabularnewline\hline
  十四年 & -453 & \tabularnewline\hline
  十五年 & -452 & \tabularnewline\hline
  十六年 & -451 & \tabularnewline\hline
  十七年 & -450 & \tabularnewline\hline
  十八年 & -449 & \tabularnewline\hline
  十九年 & -448 & \tabularnewline\hline
  二十年 & -447 & \tabularnewline\hline
  二一年 & -446 & \tabularnewline\hline
  二二年 & -445 & \tabularnewline\hline
  二三年 & -444 & \tabularnewline\hline
  二四年 & -443 & \tabularnewline\hline
  二五年 & -442 & \tabularnewline\hline
  二六年 & -441 & \tabularnewline\hline
  二七年 & -440 & \tabularnewline\hline
  二八年 & -439 & \tabularnewline\hline
  二九年 & -438 & \tabularnewline\hline
  三十年 & -437 & \tabularnewline\hline
  三一年 & -436 & \tabularnewline\hline
  三二年 & -435 & \tabularnewline\hline
  三三年 & -434 & \tabularnewline\hline
  三四年 & -433 & \tabularnewline\hline
  三五年 & -432 & \tabularnewline\hline
  三六年 & -431 & \tabularnewline\hline
  三七年 & -430 & \tabularnewline\hline
  三八年 & -429 & \tabularnewline
  \bottomrule
\end{longtable}

%%% Local Variables:
%%% mode: latex
%%% TeX-engine: xetex
%%% TeX-master: "../../Main"
%%% End:

% %% -*- coding: utf-8 -*-
%% Time-stamp: <Chen Wang: 2018-07-12 23:20:46>

\subsection{元公{\tiny(BC428-BC408)}}

% \centering
\begin{longtable}{|>{\centering\scriptsize}m{2em}|>{\centering\scriptsize}m{1.3em}|>{\centering}m{8.8em}|}
  % \caption{秦王政}\\
  \toprule
  \SimHei \normalsize 年数 & \SimHei \scriptsize 公元 & \SimHei 大事件 \tabularnewline
  % \midrule
  \endfirsthead
  \toprule
  \SimHei \normalsize 年数 & \SimHei \scriptsize 公元 & \SimHei 大事件 \tabularnewline
  \midrule
  \endhead
  \midrule
  元年 & -428 & \tabularnewline\hline
  二年 & -427 & \tabularnewline\hline
  三年 & -426 & \tabularnewline\hline
  四年 & -425 & \tabularnewline\hline
  五年 & -424 & \tabularnewline\hline
  六年 & -423 & \tabularnewline\hline
  七年 & -422 & \tabularnewline\hline
  八年 & -421 & \tabularnewline\hline
  九年 & -420 & \tabularnewline\hline
  十年 & -419 & \tabularnewline\hline
  十一年 & -418 & \tabularnewline\hline
  十二年 & -417 & \tabularnewline\hline
  十三年 & -416 & \tabularnewline\hline
  十四年 & -415 & \tabularnewline\hline
  十五年 & -414 & \tabularnewline\hline
  十六年 & -413 & \tabularnewline\hline
  十七年 & -412 & \tabularnewline\hline
  十八年 & -411 & \tabularnewline\hline
  十九年 & -410 & \tabularnewline\hline
  二十年 & -409 & \tabularnewline\hline
  二一年 & -408 & \tabularnewline
  \bottomrule
\end{longtable}

%%% Local Variables:
%%% mode: latex
%%% TeX-engine: xetex
%%% TeX-master: "../../Main"
%%% End:

% %% -*- coding: utf-8 -*-
%% Time-stamp: <Chen Wang: 2018-07-12 23:23:22>

\subsection{穆公{\tiny(BC407-BC376)}}

% \centering
\begin{longtable}{|>{\centering\scriptsize}m{2em}|>{\centering\scriptsize}m{1.3em}|>{\centering}m{8.8em}|}
  % \caption{秦王政}\\
  \toprule
  \SimHei \normalsize 年数 & \SimHei \scriptsize 公元 & \SimHei 大事件 \tabularnewline
  % \midrule
  \endfirsthead
  \toprule
  \SimHei \normalsize 年数 & \SimHei \scriptsize 公元 & \SimHei 大事件 \tabularnewline
  \midrule
  \endhead
  \midrule
  元年 & -407 & \tabularnewline\hline
  二年 & -406 & \tabularnewline\hline
  三年 & -405 & \tabularnewline\hline
  四年 & -404 & \tabularnewline\hline
  五年 & -403 & \tabularnewline
  \bottomrule
\end{longtable}

%%% Local Variables:
%%% mode: latex
%%% TeX-engine: xetex
%%% TeX-master: "../../Main"
%%% End:



%%% Local Variables:
%%% mode: latex
%%% TeX-engine: xetex
%%% TeX-master: "../../Main"
%%% End:

% %% -*- coding: utf-8 -*-
%% Time-stamp: <Chen Wang: 2019-12-26 22:14:20>

\section{郑}

%% -*- coding: utf-8 -*-
%% Time-stamp: <Chen Wang: 2018-07-16 21:59:24>

\subsection{庄公{\tiny(BC743-BC701)}}

% \centering
\begin{longtable}{|>{\centering\scriptsize}m{2em}|>{\centering\scriptsize}m{1.3em}|>{\centering}m{8.8em}|}
  % \caption{秦王政}\\
  \toprule
  \SimHei \normalsize 年数 & \SimHei \scriptsize 公元 & \SimHei 大事件 \tabularnewline
  % \midrule
  \endfirsthead
  \toprule
  \SimHei \normalsize 年数 & \SimHei \scriptsize 公元 & \SimHei 大事件 \tabularnewline
  \midrule
  \endhead
  \midrule
  % 元年 & -743 & \tabularnewline\hline
  % 二年 & -742 & \tabularnewline\hline
  % 三年 & -741 & \tabularnewline\hline
  % 四年 & -740 & \tabularnewline\hline
  % 五年 & -739 & \tabularnewline\hline
  % 六年 & -738 & \tabularnewline\hline
  % 七年 & -737 & \tabularnewline\hline
  % 八年 & -736 & \tabularnewline\hline
  % 九年 & -735 & \tabularnewline\hline
  % 十年 & -734 & \tabularnewline\hline
  % 十一年 & -733 & \tabularnewline\hline
  % 十二年 & -732 & \tabularnewline\hline
  % 十三年 & -731 & \tabularnewline\hline
  % 十四年 & -730 & \tabularnewline\hline
  % 十五年 & -729 & \tabularnewline\hline
  % 十六年 & -728 & \tabularnewline\hline
  % 十七年 & -727 & \tabularnewline\hline
  % 十八年 & -726 & \tabularnewline\hline
  % 十九年 & -725 & \tabularnewline\hline
  % 二十年 & -724 & \tabularnewline\hline
  % 二一年 & -723 & \tabularnewline\hline
  二二年 & -722 & \tabularnewline\hline
  二三年 & -721 & \tabularnewline\hline
  二四年 & -720 & \tabularnewline\hline
  二五年 & -719 & \tabularnewline\hline
  二六年 & -718 & \tabularnewline\hline
  二七年 & -717 & \tabularnewline\hline
  二八年 & -716 & \tabularnewline\hline
  二九年 & -715 & \tabularnewline\hline
  三十年 & -714 & \tabularnewline\hline
  三一年 & -713 & \tabularnewline\hline
  三二年 & -712 & \tabularnewline\hline
  三三年 & -711 & \tabularnewline\hline
  三四年 & -710 & \tabularnewline\hline
  三五年 & -709 & \tabularnewline\hline
  三六年 & -708 & \tabularnewline\hline
  三七年 & -707 & \tabularnewline\hline
  三八年 & -706 & \tabularnewline\hline
  三九年 & -705 & \tabularnewline\hline
  四十年 & -704 & \tabularnewline\hline
  四一年 & -703 & \tabularnewline\hline
  四二年 & -702 & \tabularnewline\hline
  四三年 & -701 & \tabularnewline
  \bottomrule
\end{longtable}

%%% Local Variables:
%%% mode: latex
%%% TeX-engine: xetex
%%% TeX-master: "../../Main"
%%% End:

%% -*- coding: utf-8 -*-
%% Time-stamp: <Chen Wang: 2018-07-16 22:01:47>

\subsection{昭公{\tiny(BC701)}}

% \centering
\begin{longtable}{|>{\centering\scriptsize}m{2em}|>{\centering\scriptsize}m{1.3em}|>{\centering}m{8.8em}|}
  % \caption{秦王政}\\
  \toprule
  \SimHei \normalsize 年数 & \SimHei \scriptsize 公元 & \SimHei 大事件 \tabularnewline
  % \midrule
  \endfirsthead
  \toprule
  \SimHei \normalsize 年数 & \SimHei \scriptsize 公元 & \SimHei 大事件 \tabularnewline
  \midrule
  \endhead
  \midrule
  元年 & -701 & \tabularnewline
  \bottomrule
\end{longtable}

%%% Local Variables:
%%% mode: latex
%%% TeX-engine: xetex
%%% TeX-master: "../../Main"
%%% End:

%% -*- coding: utf-8 -*-
%% Time-stamp: <Chen Wang: 2018-07-16 22:03:12>

\subsection{厉公{\tiny(BC700-BC697)}}

% \centering
\begin{longtable}{|>{\centering\scriptsize}m{2em}|>{\centering\scriptsize}m{1.3em}|>{\centering}m{8.8em}|}
  % \caption{秦王政}\\
  \toprule
  \SimHei \normalsize 年数 & \SimHei \scriptsize 公元 & \SimHei 大事件 \tabularnewline
  % \midrule
  \endfirsthead
  \toprule
  \SimHei \normalsize 年数 & \SimHei \scriptsize 公元 & \SimHei 大事件 \tabularnewline
  \midrule
  \endhead
  \midrule
  元年 & -700 & \tabularnewline\hline
  二年 & -699 & \tabularnewline\hline
  三年 & -698 & \tabularnewline\hline
  四年 & -697 & \tabularnewline
  \bottomrule
\end{longtable}

%%% Local Variables:
%%% mode: latex
%%% TeX-engine: xetex
%%% TeX-master: "../../Main"
%%% End:

%% -*- coding: utf-8 -*-
%% Time-stamp: <Chen Wang: 2018-07-16 22:09:37>

\subsection{昭公复辟{\tiny(BC696-BC695)}}

% \centering
\begin{longtable}{|>{\centering\scriptsize}m{2em}|>{\centering\scriptsize}m{1.3em}|>{\centering}m{8.8em}|}
  % \caption{秦王政}\\
  \toprule
  \SimHei \normalsize 年数 & \SimHei \scriptsize 公元 & \SimHei 大事件 \tabularnewline
  % \midrule
  \endfirsthead
  \toprule
  \SimHei \normalsize 年数 & \SimHei \scriptsize 公元 & \SimHei 大事件 \tabularnewline
  \midrule
  \endhead
  \midrule
  元年 & -696 & \tabularnewline\hline
  二年 & -695 & \tabularnewline
  \bottomrule
\end{longtable}

%%% Local Variables:
%%% mode: latex
%%% TeX-engine: xetex
%%% TeX-master: "../../Main"
%%% End:

%% -*- coding: utf-8 -*-
%% Time-stamp: <Chen Wang: 2018-07-16 22:11:25>

\subsection{子亹{\tiny(BC694)}}

% \centering
\begin{longtable}{|>{\centering\scriptsize}m{2em}|>{\centering\scriptsize}m{1.3em}|>{\centering}m{8.8em}|}
  % \caption{秦王政}\\
  \toprule
  \SimHei \normalsize 年数 & \SimHei \scriptsize 公元 & \SimHei 大事件 \tabularnewline
  % \midrule
  \endfirsthead
  \toprule
  \SimHei \normalsize 年数 & \SimHei \scriptsize 公元 & \SimHei 大事件 \tabularnewline
  \midrule
  \endhead
  \midrule
  元年 & -694 & \tabularnewline
  \bottomrule
\end{longtable}

%%% Local Variables:
%%% mode: latex
%%% TeX-engine: xetex
%%% TeX-master: "../../Main"
%%% End:

%% -*- coding: utf-8 -*-
%% Time-stamp: <Chen Wang: 2018-07-16 22:13:17>

\subsection{子婴{\tiny(BC693-BC680)}}

% \centering
\begin{longtable}{|>{\centering\scriptsize}m{2em}|>{\centering\scriptsize}m{1.3em}|>{\centering}m{8.8em}|}
  % \caption{秦王政}\\
  \toprule
  \SimHei \normalsize 年数 & \SimHei \scriptsize 公元 & \SimHei 大事件 \tabularnewline
  % \midrule
  \endfirsthead
  \toprule
  \SimHei \normalsize 年数 & \SimHei \scriptsize 公元 & \SimHei 大事件 \tabularnewline
  \midrule
  \endhead
  \midrule
  元年 & -693 & \tabularnewline\hline
  二年 & -692 & \tabularnewline\hline
  三年 & -691 & \tabularnewline\hline
  四年 & -690 & \tabularnewline\hline
  五年 & -689 & \tabularnewline\hline
  六年 & -688 & \tabularnewline\hline
  七年 & -687 & \tabularnewline\hline
  八年 & -686 & \tabularnewline\hline
  九年 & -685 & \tabularnewline\hline
  十年 & -684 & \tabularnewline\hline
  十一年 & -683 & \tabularnewline\hline
  十二年 & -682 & \tabularnewline\hline
  十三年 & -681 & \tabularnewline\hline
  十四年 & -680 & \tabularnewline
  \bottomrule
\end{longtable}

%%% Local Variables:
%%% mode: latex
%%% TeX-engine: xetex
%%% TeX-master: "../../Main"
%%% End:

%% -*- coding: utf-8 -*-
%% Time-stamp: <Chen Wang: 2018-07-16 22:33:25>

\subsection{厉公复辟{\tiny(BC679-BC673)}}

% \centering
\begin{longtable}{|>{\centering\scriptsize}m{2em}|>{\centering\scriptsize}m{1.3em}|>{\centering}m{8.8em}|}
  % \caption{秦王政}\\
  \toprule
  \SimHei \normalsize 年数 & \SimHei \scriptsize 公元 & \SimHei 大事件 \tabularnewline
  % \midrule
  \endfirsthead
  \toprule
  \SimHei \normalsize 年数 & \SimHei \scriptsize 公元 & \SimHei 大事件 \tabularnewline
  \midrule
  \endhead
  \midrule
  元年 & -679 & \tabularnewline\hline
  二年 & -678 & \tabularnewline\hline
  三年 & -677 & \tabularnewline\hline
  四年 & -676 & \tabularnewline\hline
  五年 & -675 & \tabularnewline\hline
  六年 & -674 & \tabularnewline\hline
  七年 & -673 & \tabularnewline
  \bottomrule
\end{longtable}

%%% Local Variables:
%%% mode: latex
%%% TeX-engine: xetex
%%% TeX-master: "../../Main"
%%% End:

%% -*- coding: utf-8 -*-
%% Time-stamp: <Chen Wang: 2018-07-16 22:15:22>

\subsection{文公{\tiny(BC672-BC628)}}

% \centering
\begin{longtable}{|>{\centering\scriptsize}m{2em}|>{\centering\scriptsize}m{1.3em}|>{\centering}m{8.8em}|}
  % \caption{秦王政}\\
  \toprule
  \SimHei \normalsize 年数 & \SimHei \scriptsize 公元 & \SimHei 大事件 \tabularnewline
  % \midrule
  \endfirsthead
  \toprule
  \SimHei \normalsize 年数 & \SimHei \scriptsize 公元 & \SimHei 大事件 \tabularnewline
  \midrule
  \endhead
  \midrule
  元年 & -672 & \tabularnewline\hline
  二年 & -671 & \tabularnewline\hline
  三年 & -670 & \tabularnewline\hline
  四年 & -669 & \tabularnewline\hline
  五年 & -668 & \tabularnewline\hline
  六年 & -667 & \tabularnewline\hline
  七年 & -666 & \tabularnewline\hline
  八年 & -665 & \tabularnewline\hline
  九年 & -664 & \tabularnewline\hline
  十年 & -663 & \tabularnewline\hline
  十一年 & -662 & \tabularnewline\hline
  十二年 & -661 & \tabularnewline\hline
  十三年 & -660 & \tabularnewline\hline
  十四年 & -659 & \tabularnewline\hline
  十五年 & -658 & \tabularnewline\hline
  十六年 & -657 & \tabularnewline\hline
  十七年 & -656 & \tabularnewline\hline
  十八年 & -655 & \tabularnewline\hline
  十九年 & -654 & \tabularnewline\hline
  二十年 & -653 & \tabularnewline\hline
  二一年 & -652 & \tabularnewline\hline
  二二年 & -651 & \tabularnewline\hline
  二三年 & -650 & \tabularnewline\hline
  二四年 & -649 & \tabularnewline\hline
  二五年 & -648 & \tabularnewline\hline
  二六年 & -647 & \tabularnewline\hline
  二七年 & -646 & \tabularnewline\hline
  二八年 & -645 & \tabularnewline\hline
  二九年 & -644 & \tabularnewline\hline
  三十年 & -643 & \tabularnewline\hline
  三一年 & -642 & \tabularnewline\hline
  三二年 & -641 & \tabularnewline\hline
  三三年 & -640 & \tabularnewline\hline
  三四年 & -639 & \tabularnewline\hline
  三五年 & -638 & \tabularnewline\hline
  三六年 & -637 & \tabularnewline\hline
  三七年 & -636 & \tabularnewline\hline
  三八年 & -635 & \tabularnewline\hline
  三九年 & -634 & \tabularnewline\hline
  四十年 & -633 & \tabularnewline\hline
  四一年 & -632 & \tabularnewline\hline
  四二年 & -631 & \tabularnewline\hline
  四三年 & -630 & \tabularnewline\hline
  四四年 & -629 & \tabularnewline\hline
  四五年 & -628 & \tabularnewline
  \bottomrule
\end{longtable}

%%% Local Variables:
%%% mode: latex
%%% TeX-engine: xetex
%%% TeX-master: "../../Main"
%%% End:

%% -*- coding: utf-8 -*-
%% Time-stamp: <Chen Wang: 2018-07-16 22:15:48>

\subsection{穆公{\tiny(BC627-BC606)}}

% \centering
\begin{longtable}{|>{\centering\scriptsize}m{2em}|>{\centering\scriptsize}m{1.3em}|>{\centering}m{8.8em}|}
  % \caption{秦王政}\\
  \toprule
  \SimHei \normalsize 年数 & \SimHei \scriptsize 公元 & \SimHei 大事件 \tabularnewline
  % \midrule
  \endfirsthead
  \toprule
  \SimHei \normalsize 年数 & \SimHei \scriptsize 公元 & \SimHei 大事件 \tabularnewline
  \midrule
  \endhead
  \midrule
  元年 & -627 & \tabularnewline\hline
  二年 & -626 & \tabularnewline\hline
  三年 & -625 & \tabularnewline\hline
  四年 & -624 & \tabularnewline\hline
  五年 & -623 & \tabularnewline\hline
  六年 & -622 & \tabularnewline\hline
  七年 & -621 & \tabularnewline\hline
  八年 & -620 & \tabularnewline\hline
  九年 & -619 & \tabularnewline\hline
  十年 & -618 & \tabularnewline\hline
  十一年 & -617 & \tabularnewline\hline
  十二年 & -616 & \tabularnewline\hline
  十三年 & -615 & \tabularnewline\hline
  十四年 & -614 & \tabularnewline\hline
  十五年 & -613 & \tabularnewline\hline
  十六年 & -612 & \tabularnewline\hline
  十七年 & -611 & \tabularnewline\hline
  十八年 & -610 & \tabularnewline\hline
  十九年 & -609 & \tabularnewline\hline
  二十年 & -608 & \tabularnewline\hline
  二一年 & -607 & \tabularnewline\hline
  二二年 & -606 & \tabularnewline
  \bottomrule
\end{longtable}

%%% Local Variables:
%%% mode: latex
%%% TeX-engine: xetex
%%% TeX-master: "../../Main"
%%% End:

%% -*- coding: utf-8 -*-
%% Time-stamp: <Chen Wang: 2018-07-16 22:16:22>

\subsection{灵公{\tiny(BC605)}}

% \centering
\begin{longtable}{|>{\centering\scriptsize}m{2em}|>{\centering\scriptsize}m{1.3em}|>{\centering}m{8.8em}|}
  % \caption{秦王政}\\
  \toprule
  \SimHei \normalsize 年数 & \SimHei \scriptsize 公元 & \SimHei 大事件 \tabularnewline
  % \midrule
  \endfirsthead
  \toprule
  \SimHei \normalsize 年数 & \SimHei \scriptsize 公元 & \SimHei 大事件 \tabularnewline
  \midrule
  \endhead
  \midrule
  元年 & -605 & \tabularnewline
  \bottomrule
\end{longtable}

%%% Local Variables:
%%% mode: latex
%%% TeX-engine: xetex
%%% TeX-master: "../../Main"
%%% End:

%% -*- coding: utf-8 -*-
%% Time-stamp: <Chen Wang: 2018-07-16 22:17:12>

\subsection{襄公{\tiny(BC604-BC587)}}

% \centering
\begin{longtable}{|>{\centering\scriptsize}m{2em}|>{\centering\scriptsize}m{1.3em}|>{\centering}m{8.8em}|}
  % \caption{秦王政}\\
  \toprule
  \SimHei \normalsize 年数 & \SimHei \scriptsize 公元 & \SimHei 大事件 \tabularnewline
  % \midrule
  \endfirsthead
  \toprule
  \SimHei \normalsize 年数 & \SimHei \scriptsize 公元 & \SimHei 大事件 \tabularnewline
  \midrule
  \endhead
  \midrule
  元年 & -604 & \tabularnewline\hline
  二年 & -603 & \tabularnewline\hline
  三年 & -602 & \tabularnewline\hline
  四年 & -601 & \tabularnewline\hline
  五年 & -600 & \tabularnewline\hline
  六年 & -599 & \tabularnewline\hline
  七年 & -598 & \tabularnewline\hline
  八年 & -597 & \tabularnewline\hline
  九年 & -596 & \tabularnewline\hline
  十年 & -595 & \tabularnewline\hline
  十一年 & -594 & \tabularnewline\hline
  十二年 & -593 & \tabularnewline\hline
  十三年 & -592 & \tabularnewline\hline
  十四年 & -591 & \tabularnewline\hline
  十五年 & -590 & \tabularnewline\hline
  十六年 & -589 & \tabularnewline\hline
  十七年 & -588 & \tabularnewline\hline
  十八年 & -587 & \tabularnewline
  \bottomrule
\end{longtable}

%%% Local Variables:
%%% mode: latex
%%% TeX-engine: xetex
%%% TeX-master: "../../Main"
%%% End:

%% -*- coding: utf-8 -*-
%% Time-stamp: <Chen Wang: 2018-07-16 22:18:44>

\subsection{悼公{\tiny(BC586-BC585)}}

% \centering
\begin{longtable}{|>{\centering\scriptsize}m{2em}|>{\centering\scriptsize}m{1.3em}|>{\centering}m{8.8em}|}
  % \caption{秦王政}\\
  \toprule
  \SimHei \normalsize 年数 & \SimHei \scriptsize 公元 & \SimHei 大事件 \tabularnewline
  % \midrule
  \endfirsthead
  \toprule
  \SimHei \normalsize 年数 & \SimHei \scriptsize 公元 & \SimHei 大事件 \tabularnewline
  \midrule
  \endhead
  \midrule
  元年 & -586 & \tabularnewline\hline
  二年 & -585 & \tabularnewline
  \bottomrule
\end{longtable}

%%% Local Variables:
%%% mode: latex
%%% TeX-engine: xetex
%%% TeX-master: "../../Main"
%%% End:

%% -*- coding: utf-8 -*-
%% Time-stamp: <Chen Wang: 2018-07-16 22:18:36>

\subsection{成公{\tiny(BC584-BC571)}}

% \centering
\begin{longtable}{|>{\centering\scriptsize}m{2em}|>{\centering\scriptsize}m{1.3em}|>{\centering}m{8.8em}|}
  % \caption{秦王政}\\
  \toprule
  \SimHei \normalsize 年数 & \SimHei \scriptsize 公元 & \SimHei 大事件 \tabularnewline
  % \midrule
  \endfirsthead
  \toprule
  \SimHei \normalsize 年数 & \SimHei \scriptsize 公元 & \SimHei 大事件 \tabularnewline
  \midrule
  \endhead
  \midrule
  元年 & -584 & \tabularnewline\hline
  二年 & -583 & \tabularnewline\hline
  三年 & -582 & \tabularnewline\hline
  四年 & -581 & \tabularnewline\hline
  五年 & -580 & \tabularnewline\hline
  六年 & -579 & \tabularnewline\hline
  七年 & -578 & \tabularnewline\hline
  八年 & -577 & \tabularnewline\hline
  九年 & -576 & \tabularnewline\hline
  十年 & -575 & \tabularnewline\hline
  十一年 & -574 & \tabularnewline\hline
  十二年 & -573 & \tabularnewline\hline
  十三年 & -572 & \tabularnewline\hline
  十四年 & -571 & \tabularnewline
  \bottomrule
\end{longtable}

%%% Local Variables:
%%% mode: latex
%%% TeX-engine: xetex
%%% TeX-master: "../../Main"
%%% End:

%% -*- coding: utf-8 -*-
%% Time-stamp: <Chen Wang: 2018-07-16 22:19:42>

\subsection{僖公{\tiny(BC570-BC566)}}

% \centering
\begin{longtable}{|>{\centering\scriptsize}m{2em}|>{\centering\scriptsize}m{1.3em}|>{\centering}m{8.8em}|}
  % \caption{秦王政}\\
  \toprule
  \SimHei \normalsize 年数 & \SimHei \scriptsize 公元 & \SimHei 大事件 \tabularnewline
  % \midrule
  \endfirsthead
  \toprule
  \SimHei \normalsize 年数 & \SimHei \scriptsize 公元 & \SimHei 大事件 \tabularnewline
  \midrule
  \endhead
  \midrule
  元年 & -570 & \tabularnewline\hline
  二年 & -569 & \tabularnewline\hline
  三年 & -568 & \tabularnewline\hline
  四年 & -567 & \tabularnewline\hline
  五年 & -566 & \tabularnewline
  \bottomrule
\end{longtable}

%%% Local Variables:
%%% mode: latex
%%% TeX-engine: xetex
%%% TeX-master: "../../Main"
%%% End:

%% -*- coding: utf-8 -*-
%% Time-stamp: <Chen Wang: 2018-07-16 22:20:30>

\subsection{简公{\tiny(BC565-BC530)}}

% \centering
\begin{longtable}{|>{\centering\scriptsize}m{2em}|>{\centering\scriptsize}m{1.3em}|>{\centering}m{8.8em}|}
  % \caption{秦王政}\\
  \toprule
  \SimHei \normalsize 年数 & \SimHei \scriptsize 公元 & \SimHei 大事件 \tabularnewline
  % \midrule
  \endfirsthead
  \toprule
  \SimHei \normalsize 年数 & \SimHei \scriptsize 公元 & \SimHei 大事件 \tabularnewline
  \midrule
  \endhead
  \midrule
  元年 & -565 & \tabularnewline\hline
  二年 & -564 & \tabularnewline\hline
  三年 & -563 & \tabularnewline\hline
  四年 & -562 & \tabularnewline\hline
  五年 & -561 & \tabularnewline\hline
  六年 & -560 & \tabularnewline\hline
  七年 & -559 & \tabularnewline\hline
  八年 & -558 & \tabularnewline\hline
  九年 & -557 & \tabularnewline\hline
  十年 & -556 & \tabularnewline\hline
  十一年 & -555 & \tabularnewline\hline
  十二年 & -554 & \tabularnewline\hline
  十三年 & -553 & \tabularnewline\hline
  十四年 & -552 & \tabularnewline\hline
  十五年 & -551 & \tabularnewline\hline
  十六年 & -550 & \tabularnewline\hline
  十七年 & -549 & \tabularnewline\hline
  十八年 & -548 & \tabularnewline\hline
  十九年 & -547 & \tabularnewline\hline
  二十年 & -546 & \tabularnewline\hline
  二一年 & -545 & \tabularnewline\hline
  二二年 & -544 & \tabularnewline\hline
  二三年 & -543 & \tabularnewline\hline
  二四年 & -542 & \tabularnewline\hline
  二五年 & -541 & \tabularnewline\hline
  二六年 & -540 & \tabularnewline\hline
  二七年 & -539 & \tabularnewline\hline
  二八年 & -538 & \tabularnewline\hline
  二九年 & -537 & \tabularnewline\hline
  三十年 & -536 & \tabularnewline\hline
  三一年 & -535 & \tabularnewline\hline
  三二年 & -534 & \tabularnewline\hline
  三三年 & -533 & \tabularnewline\hline
  三四年 & -532 & \tabularnewline\hline
  三五年 & -531 & \tabularnewline\hline
  三六年 & -530 & \tabularnewline
  \bottomrule
\end{longtable}

%%% Local Variables:
%%% mode: latex
%%% TeX-engine: xetex
%%% TeX-master: "../../Main"
%%% End:

%% -*- coding: utf-8 -*-
%% Time-stamp: <Chen Wang: 2018-07-16 22:21:18>

\subsection{定公{\tiny(BC529-BC514)}}

% \centering
\begin{longtable}{|>{\centering\scriptsize}m{2em}|>{\centering\scriptsize}m{1.3em}|>{\centering}m{8.8em}|}
  % \caption{秦王政}\\
  \toprule
  \SimHei \normalsize 年数 & \SimHei \scriptsize 公元 & \SimHei 大事件 \tabularnewline
  % \midrule
  \endfirsthead
  \toprule
  \SimHei \normalsize 年数 & \SimHei \scriptsize 公元 & \SimHei 大事件 \tabularnewline
  \midrule
  \endhead
  \midrule
  元年 & -529 & \tabularnewline\hline
  二年 & -528 & \tabularnewline\hline
  三年 & -527 & \tabularnewline\hline
  四年 & -526 & \tabularnewline\hline
  五年 & -525 & \tabularnewline\hline
  六年 & -524 & \tabularnewline\hline
  七年 & -523 & \tabularnewline\hline
  八年 & -522 & \tabularnewline\hline
  九年 & -521 & \tabularnewline\hline
  十年 & -520 & \tabularnewline\hline
  十一年 & -519 & \tabularnewline\hline
  十二年 & -518 & \tabularnewline\hline
  十三年 & -517 & \tabularnewline\hline
  十四年 & -516 & \tabularnewline\hline
  十五年 & -515 & \tabularnewline\hline
  十六年 & -514 & \tabularnewline
  \bottomrule
\end{longtable}

%%% Local Variables:
%%% mode: latex
%%% TeX-engine: xetex
%%% TeX-master: "../../Main"
%%% End:

%% -*- coding: utf-8 -*-
%% Time-stamp: <Chen Wang: 2018-07-16 22:24:08>

\subsection{献公{\tiny(BC513-BC501)}}

% \centering
\begin{longtable}{|>{\centering\scriptsize}m{2em}|>{\centering\scriptsize}m{1.3em}|>{\centering}m{8.8em}|}
  % \caption{秦王政}\\
  \toprule
  \SimHei \normalsize 年数 & \SimHei \scriptsize 公元 & \SimHei 大事件 \tabularnewline
  % \midrule
  \endfirsthead
  \toprule
  \SimHei \normalsize 年数 & \SimHei \scriptsize 公元 & \SimHei 大事件 \tabularnewline
  \midrule
  \endhead
  \midrule
  元年 & -513 & \tabularnewline\hline
  二年 & -512 & \tabularnewline\hline
  三年 & -511 & \tabularnewline\hline
  四年 & -510 & \tabularnewline\hline
  五年 & -509 & \tabularnewline\hline
  六年 & -508 & \tabularnewline\hline
  七年 & -507 & \tabularnewline\hline
  八年 & -506 & \tabularnewline\hline
  九年 & -505 & \tabularnewline\hline
  十年 & -504 & \tabularnewline\hline
  十一年 & -503 & \tabularnewline\hline
  十二年 & -502 & \tabularnewline\hline
  十三年 & -501 & \tabularnewline
  \bottomrule
\end{longtable}

%%% Local Variables:
%%% mode: latex
%%% TeX-engine: xetex
%%% TeX-master: "../../Main"
%%% End:

%% -*- coding: utf-8 -*-
%% Time-stamp: <Chen Wang: 2018-07-16 22:25:12>

\subsection{声公{\tiny(BC500-BC463)}}

% \centering
\begin{longtable}{|>{\centering\scriptsize}m{2em}|>{\centering\scriptsize}m{1.3em}|>{\centering}m{8.8em}|}
  % \caption{秦王政}\\
  \toprule
  \SimHei \normalsize 年数 & \SimHei \scriptsize 公元 & \SimHei 大事件 \tabularnewline
  % \midrule
  \endfirsthead
  \toprule
  \SimHei \normalsize 年数 & \SimHei \scriptsize 公元 & \SimHei 大事件 \tabularnewline
  \midrule
  \endhead
  \midrule
  元年 & -500 & \tabularnewline\hline
  二年 & -499 & \tabularnewline\hline
  三年 & -498 & \tabularnewline\hline
  四年 & -497 & \tabularnewline\hline
  五年 & -496 & \tabularnewline\hline
  六年 & -495 & \tabularnewline\hline
  七年 & -494 & \tabularnewline\hline
  八年 & -493 & \tabularnewline\hline
  九年 & -492 & \tabularnewline\hline
  十年 & -491 & \tabularnewline\hline
  十一年 & -490 & \tabularnewline\hline
  十二年 & -489 & \tabularnewline\hline
  十三年 & -488 & \tabularnewline\hline
  十四年 & -487 & \tabularnewline\hline
  十五年 & -486 & \tabularnewline\hline
  十六年 & -485 & \tabularnewline\hline
  十七年 & -484 & \tabularnewline\hline
  十八年 & -483 & \tabularnewline\hline
  十九年 & -482 & \tabularnewline\hline
  二十年 & -481 & \tabularnewline\hline
  二一年 & -480 & \tabularnewline\hline
  二二年 & -479 & \tabularnewline\hline
  二三年 & -478 & \tabularnewline\hline
  二四年 & -477 & \tabularnewline\hline
  二五年 & -476 & \tabularnewline\hline
  二六年 & -475 & \tabularnewline\hline
  二七年 & -474 & \tabularnewline\hline
  二八年 & -473 & \tabularnewline\hline
  二九年 & -472 & \tabularnewline\hline
  三十年 & -471 & \tabularnewline\hline
  三一年 & -470 & \tabularnewline\hline
  三二年 & -469 & \tabularnewline\hline
  三三年 & -468 & \tabularnewline\hline
  三四年 & -467 & \tabularnewline\hline
  三五年 & -466 & \tabularnewline\hline
  三六年 & -465 & \tabularnewline\hline
  三七年 & -464 & \tabularnewline\hline
  三八年 & -463 & \tabularnewline
  \bottomrule
\end{longtable}

%%% Local Variables:
%%% mode: latex
%%% TeX-engine: xetex
%%% TeX-master: "../../Main"
%%% End:

%% -*- coding: utf-8 -*-
%% Time-stamp: <Chen Wang: 2018-07-16 22:25:59>

\subsection{哀公{\tiny(BC462-BC455)}}

% \centering
\begin{longtable}{|>{\centering\scriptsize}m{2em}|>{\centering\scriptsize}m{1.3em}|>{\centering}m{8.8em}|}
  % \caption{秦王政}\\
  \toprule
  \SimHei \normalsize 年数 & \SimHei \scriptsize 公元 & \SimHei 大事件 \tabularnewline
  % \midrule
  \endfirsthead
  \toprule
  \SimHei \normalsize 年数 & \SimHei \scriptsize 公元 & \SimHei 大事件 \tabularnewline
  \midrule
  \endhead
  \midrule
  元年 & -462 & \tabularnewline\hline
  二年 & -461 & \tabularnewline\hline
  三年 & -460 & \tabularnewline\hline
  四年 & -459 & \tabularnewline\hline
  五年 & -458 & \tabularnewline\hline
  六年 & -457 & \tabularnewline\hline
  七年 & -456 & \tabularnewline\hline
  八年 & -455 & \tabularnewline
  \bottomrule
\end{longtable}

%%% Local Variables:
%%% mode: latex
%%% TeX-engine: xetex
%%% TeX-master: "../../Main"
%%% End:

%% -*- coding: utf-8 -*-
%% Time-stamp: <Chen Wang: 2018-07-16 22:26:47>

\subsection{共公{\tiny(BC454-BC424)}}

% \centering
\begin{longtable}{|>{\centering\scriptsize}m{2em}|>{\centering\scriptsize}m{1.3em}|>{\centering}m{8.8em}|}
  % \caption{秦王政}\\
  \toprule
  \SimHei \normalsize 年数 & \SimHei \scriptsize 公元 & \SimHei 大事件 \tabularnewline
  % \midrule
  \endfirsthead
  \toprule
  \SimHei \normalsize 年数 & \SimHei \scriptsize 公元 & \SimHei 大事件 \tabularnewline
  \midrule
  \endhead
  \midrule
  元年 & -454 & \tabularnewline\hline
  二年 & -453 & \tabularnewline\hline
  三年 & -452 & \tabularnewline\hline
  四年 & -451 & \tabularnewline\hline
  五年 & -450 & \tabularnewline\hline
  六年 & -449 & \tabularnewline\hline
  七年 & -448 & \tabularnewline\hline
  八年 & -447 & \tabularnewline\hline
  九年 & -446 & \tabularnewline\hline
  十年 & -445 & \tabularnewline\hline
  十一年 & -444 & \tabularnewline\hline
  十二年 & -443 & \tabularnewline\hline
  十三年 & -442 & \tabularnewline\hline
  十四年 & -441 & \tabularnewline\hline
  十五年 & -440 & \tabularnewline\hline
  十六年 & -439 & \tabularnewline\hline
  十七年 & -438 & \tabularnewline\hline
  十八年 & -437 & \tabularnewline\hline
  十九年 & -436 & \tabularnewline\hline
  二十年 & -435 & \tabularnewline\hline
  二一年 & -434 & \tabularnewline\hline
  二二年 & -433 & \tabularnewline\hline
  二三年 & -432 & \tabularnewline\hline
  二四年 & -431 & \tabularnewline\hline
  二五年 & -430 & \tabularnewline\hline
  二六年 & -429 & \tabularnewline\hline
  二七年 & -428 & \tabularnewline\hline
  二八年 & -427 & \tabularnewline\hline
  二九年 & -426 & \tabularnewline\hline
  三十年 & -425 & \tabularnewline\hline
  三一年 & -424 & \tabularnewline
  \bottomrule
\end{longtable}

%%% Local Variables:
%%% mode: latex
%%% TeX-engine: xetex
%%% TeX-master: "../../Main"
%%% End:

%% -*- coding: utf-8 -*-
%% Time-stamp: <Chen Wang: 2018-07-16 22:27:22>

\subsection{幽公{\tiny(BC423)}}

% \centering
\begin{longtable}{|>{\centering\scriptsize}m{2em}|>{\centering\scriptsize}m{1.3em}|>{\centering}m{8.8em}|}
  % \caption{秦王政}\\
  \toprule
  \SimHei \normalsize 年数 & \SimHei \scriptsize 公元 & \SimHei 大事件 \tabularnewline
  % \midrule
  \endfirsthead
  \toprule
  \SimHei \normalsize 年数 & \SimHei \scriptsize 公元 & \SimHei 大事件 \tabularnewline
  \midrule
  \endhead
  \midrule
  元年 & -423 & \tabularnewline
  \bottomrule
\end{longtable}

%%% Local Variables:
%%% mode: latex
%%% TeX-engine: xetex
%%% TeX-master: "../../Main"
%%% End:

%% -*- coding: utf-8 -*-
%% Time-stamp: <Chen Wang: 2018-07-16 22:28:46>

\subsection{繻公{\tiny(BC422-BC396)}}

% \centering
\begin{longtable}{|>{\centering\scriptsize}m{2em}|>{\centering\scriptsize}m{1.3em}|>{\centering}m{8.8em}|}
  % \caption{秦王政}\\
  \toprule
  \SimHei \normalsize 年数 & \SimHei \scriptsize 公元 & \SimHei 大事件 \tabularnewline
  % \midrule
  \endfirsthead
  \toprule
  \SimHei \normalsize 年数 & \SimHei \scriptsize 公元 & \SimHei 大事件 \tabularnewline
  \midrule
  \endhead
  \midrule
  元年 & -422 & \tabularnewline\hline
  二年 & -421 & \tabularnewline\hline
  三年 & -420 & \tabularnewline\hline
  四年 & -419 & \tabularnewline\hline
  五年 & -418 & \tabularnewline\hline
  六年 & -417 & \tabularnewline\hline
  七年 & -416 & \tabularnewline\hline
  八年 & -415 & \tabularnewline\hline
  九年 & -414 & \tabularnewline\hline
  十年 & -413 & \tabularnewline\hline
  十一年 & -412 & \tabularnewline\hline
  十二年 & -411 & \tabularnewline\hline
  十三年 & -410 & \tabularnewline\hline
  十四年 & -409 & \tabularnewline\hline
  十五年 & -408 & \tabularnewline\hline
  十六年 & -407 & \tabularnewline\hline
  十七年 & -406 & \tabularnewline\hline
  十八年 & -405 & \tabularnewline\hline
  十九年 & -404 & \tabularnewline\hline
  二十年 & -403 & \tabularnewline% \hline
  % 二一年 & -402 & \tabularnewline\hline
  % 二二年 & -401 & \tabularnewline\hline
  % 二三年 & -400 & \tabularnewline\hline
  % 二四年 & -399 & \tabularnewline\hline
  % 二五年 & -398 & \tabularnewline\hline
  % 二六年 & -397 & \tabularnewline\hline
  % 二七年 & -396 & \tabularnewline
  \bottomrule
\end{longtable}

%%% Local Variables:
%%% mode: latex
%%% TeX-engine: xetex
%%% TeX-master: "../../Main"
%%% End:



%%% Local Variables:
%%% mode: latex
%%% TeX-engine: xetex
%%% TeX-master: "../../Main"
%%% End:



%%% Local Variables:
%%% mode: latex
%%% TeX-engine: xetex
%%% TeX-master: "../Main"
%%% End:
