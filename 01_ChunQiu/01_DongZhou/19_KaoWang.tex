%% -*- coding: utf-8 -*-
%% Time-stamp: <Chen Wang: 2021-11-02 15:34:25>

\subsection{考王嵬\tiny{(BC440-BC426)}}

\subsubsection{生平}

周考王(?-前426年),又稱周考哲王,姓姬,名嵬,為中國東周第十九代國王,在位15年,為周貞定王之子、周哀王與周思王之弟。

前441年,姬嵬殺害周思王自立,是為周考王,以前440年為考王元年。前440年,周考王封其弟姬揭於王畿(位於今河南),是為西周桓公。西周桓公死後,子威公立。惠公繼承威公之位,在周顯王二年(前367年)又封少子姬班於鞏(今河南省鞏义市西南),史稱“東周”。此時周朝王畿內再分出「西周」、「東周」两小國,王畿便更為縮小。

周考王時期處於春秋時期與戰國時期之間,或戰國初期。

前426年周考王死去,其子姬午繼位,是為周威烈王。

\subsubsection{年表}

% \centering
\begin{longtable}{|>{\centering\scriptsize}m{2em}|>{\centering\scriptsize}m{1.3em}|>{\centering}m{8.8em}|}
  % \caption{秦王政}\\
  \toprule
  \SimHei \normalsize 年数 & \SimHei \scriptsize 公元 & \SimHei 大事件 \tabularnewline
  % \midrule
  \endfirsthead
  \toprule
  \SimHei \normalsize 年数 & \SimHei \scriptsize 公元 & \SimHei 大事件 \tabularnewline
  \midrule
  \endhead
  \midrule
  元年 & -440 & \tabularnewline\hline
  二年 & -439 & \tabularnewline\hline
  三年 & -438 & \tabularnewline\hline
  四年 & -437 & \tabularnewline\hline
  五年 & -436 & \tabularnewline\hline
  六年 & -435 & \tabularnewline\hline
  七年 & -434 & \tabularnewline\hline
  八年 & -433 & \tabularnewline\hline
  九年 & -432 & \tabularnewline\hline
  十年 & -431 & \tabularnewline\hline
  十一年 & -430 & \tabularnewline\hline
  十二年 & -429 & \tabularnewline\hline
  十三年 & -428 & \tabularnewline\hline
  十四年 & -427 & \tabularnewline\hline
  十五年 & -426 & \tabularnewline
  \bottomrule
\end{longtable}

%%% Local Variables:
%%% mode: latex
%%% TeX-engine: xetex
%%% TeX-master: "../../Main"
%%% End:
