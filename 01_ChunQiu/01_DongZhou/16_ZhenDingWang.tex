%% -*- coding: utf-8 -*-
%% Time-stamp: <Chen Wang: 2021-11-02 15:29:48>

\subsection{貞定王介\tiny{(BC468-BC441)}}

\subsubsection{生平}

周貞定王(?-前441年),姓姬,名介,東周君主,周元王子,在位28年,諡號貞定王。

周貞定王十六年(前453年),晋国大夫韩康子、赵襄子、魏桓子共同攻灭了晉國最大勢力智伯瑤,是為三家滅智。

清朝学者黄式三在其《周季编略》中认为周王介的谥号貞定王的说法是一个错误。他指出,《史记·周本纪》中周王介被称为定王,与周定王瑜同谥,黄式三认为此处史记是沿袭了《国语》的错误记载。黄式三认为皇甫谧在《帝王世纪》中,按照《世本》和《史记》等称周王介为贞王或定王的记载,臆造了周貞定王的称谓,司马贞《史记索隐》就对皇甫谧的做法提出批评。而后世学者多从皇甫谧,黄式三认为应该根据《国语》韦昭注和司马贞《史记索隐》的说法,而称周王介为周贞王。

\subsubsection{年表}

% \centering
\begin{longtable}{|>{\centering\scriptsize}m{2em}|>{\centering\scriptsize}m{1.3em}|>{\centering}m{8.8em}|}
  % \caption{秦王政}\\
  \toprule
  \SimHei \normalsize 年数 & \SimHei \scriptsize 公元 & \SimHei 大事件 \tabularnewline
  % \midrule
  \endfirsthead
  \toprule
  \SimHei \normalsize 年数 & \SimHei \scriptsize 公元 & \SimHei 大事件 \tabularnewline
  \midrule
  \endhead
  \midrule
  元年 & -468 & \tabularnewline\hline
  二年 & -467 & \tabularnewline\hline
  三年 & -466 & \tabularnewline\hline
  四年 & -465 & \tabularnewline\hline
  五年 & -464 & \tabularnewline\hline
  六年 & -463 & \tabularnewline\hline
  七年 & -462 & \tabularnewline\hline
  八年 & -461 & \tabularnewline\hline
  九年 & -460 & \tabularnewline\hline
  十年 & -459 & \tabularnewline\hline
  十一年 & -458 & \tabularnewline\hline
  十二年 & -457 & \tabularnewline\hline
  十三年 & -456 & \tabularnewline\hline
  十四年 & -455 & \tabularnewline\hline
  十五年 & -454 & \tabularnewline\hline
  十六年 & -453 & \tabularnewline\hline
  十七年 & -452 & \tabularnewline\hline
  十八年 & -451 & \tabularnewline\hline
  十九年 & -450 & \tabularnewline\hline
  二十年 & -449 & \tabularnewline\hline
  二一年 & -448 & \tabularnewline\hline
  二二年 & -447 & \tabularnewline\hline
  二三年 & -446 & \tabularnewline\hline
  二四年 & -445 & \tabularnewline\hline
  二五年 & -444 & \tabularnewline\hline
  二六年 & -443 & \tabularnewline\hline
  二七年 & -442 & \tabularnewline\hline
  二八年 & -441 & \tabularnewline  
  \bottomrule
\end{longtable}

%%% Local Variables:
%%% mode: latex
%%% TeX-engine: xetex
%%% TeX-master: "../../Main"
%%% End:
