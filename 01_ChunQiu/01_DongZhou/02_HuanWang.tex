%% -*- coding: utf-8 -*-
%% Time-stamp: <Chen Wang: 2021-11-01 18:25:56>

\subsection{桓王林\tiny{(BC719-BC697)}}

\subsubsection{生平}

周桓王(?-前697年),姓姬,名林,東周第二代君主,諡桓。他是周平王之孫,因平王駕崩時,太子洩父早死,其作为洩父之子得以繼承天子之位。

桓王(前720年)即位後,計劃削弱卿權以加強王權。加上周朝領地與鄭國領地鄰接,鄭國多次越界取禾,故桓王罷免了鄭莊公卿士之職。莊公不滿,便不再朝見天子。周王室始與鄭國交惡。

桓王二年(前718年)春天,晉國曲沃封君曲沃莊伯賄賂周桓王,聯合鄭國、邢國攻打晉國都城翼城,晉鄂侯戰敗,逃奔隨邑。同年夏天,晉鄂侯去世,曲沃莊伯於是再度攻打晉國。由於當時曲沃莊伯背叛周桓王,同年秋天,周桓王便反過來支持晉國,並派遣虢公 率領軍隊討伐曲沃莊伯,曲沃莊伯兵敗,只得逃回曲沃防守。周桓王立晉鄂侯之子晉哀侯為君。

桓王十二年(前708年)周桓王聯合秦國出兵包圍芮國,俘虜芮國國君芮伯萬。

桓王十三年(前707年),桓王率蔡、衞、陳聯軍攻鄭,大敗於繻葛。桓王本人更於此役中為鄭將祝聃射箭中傷。此後桓王雖然仍能影響虢國,但已無力阻止周王室轉衰之势,也無力阻止諸侯間的互相攻伐。

桓王十八年(前703年),虢仲向周桓王進讒言誣陷大夫詹父。周桓王認為詹父有理,詹父於是帶領周天子的軍隊進攻虢國。同年夏天,虢公逃亡到虞國

桓王二十三年(前697年),桓王三月因病駕崩,儿子王子佗继位,是為周莊王。

\subsubsection{年表}

% \centering
\begin{longtable}{|>{\centering\scriptsize}m{2em}|>{\centering\scriptsize}m{1.3em}|>{\centering}m{8.8em}|}
  % \caption{秦王政}\\
  \toprule
  \SimHei \normalsize 年数 & \SimHei \scriptsize 公元 & \SimHei 大事件 \tabularnewline
  % \midrule
  \endfirsthead
  \toprule
  \SimHei \normalsize 年数 & \SimHei \scriptsize 公元 & \SimHei 大事件 \tabularnewline
  \midrule
  \endhead
  \midrule
  元年 & -719 & \tabularnewline\hline
  二年 & -718 & \tabularnewline\hline
  三年 & -717 & \tabularnewline\hline
  四年 & -716 & \tabularnewline\hline
  五年 & -715 & \tabularnewline\hline
  六年 & -714 & \tabularnewline\hline
  七年 & -713 & \tabularnewline\hline
  八年 & -712 & \tabularnewline\hline
  九年 & -711 & \tabularnewline\hline
  十年 & -710 & \tabularnewline\hline
  十一年 & -709 & \tabularnewline\hline
  十二年 & -708 & \tabularnewline\hline
  十三年 & -707 & \tabularnewline\hline
  十四年 & -706 & \tabularnewline\hline
  十五年 & -705 & \tabularnewline\hline
  十六年 & -704 & \tabularnewline\hline
  十七年 & -703 & \tabularnewline\hline
  十八年 & -702 & \tabularnewline\hline
  十九年 & -701 & \tabularnewline\hline
  二十年 & -700 & \tabularnewline\hline
  二一年 & -699 & \tabularnewline\hline
  二二年 & -698 & \tabularnewline\hline
  二三年 & -697 & \tabularnewline
  \bottomrule
\end{longtable}

%%% Local Variables:
%%% mode: latex
%%% TeX-engine: xetex
%%% TeX-master: "../../Main"
%%% End:
