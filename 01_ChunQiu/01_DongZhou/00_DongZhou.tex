%% -*- coding: utf-8 -*-
%% Time-stamp: <Chen Wang: 2021-11-02 15:39:05>

\section{东周}

\subsection{簡介}

东周(前770年至前256年),是自周平王東遷以后對周朝的称呼,相对于之前国都在镐京的時期即西周。东周也是「春秋時代」的开始。

東周京都於前770年自镐京(今陝西省西安市),东迁至雒邑(今河南省洛阳市)。传25王,前后515年。 这一时期是中国的社会制度剧烈转变的时期,以铁器的广泛使用为标志。

周幽王死後,諸侯擁立原先被廢的太子宜臼為王,是为周平王。他即位第二年,見鎬京被戰火破壞,又受到犬戎侵扰,便遷都雒邑,史稱「東周」,以別於在這以前的西周。東周的前半期,諸侯爭相稱霸,持續了二百多年,稱為「春秋時代」;東周的後半期,周天子地位漸失,亦持續了二百多年,稱為「戰國時代」。

周平王東遷以後,管轄範圍大減,形同一個小國,加上被指有弑父之嫌,在諸侯中的威望已经大不如前。面對諸侯之間互相攻伐和兼併,邊境的外族又乘機入侵,周天子不能擔負共主的責任,經常要向一些強大的諸侯求助。在這情況下,強大的諸侯便自居霸主。中原諸侯對四夷侵擾則以「尊王攘夷」口號團結自衛,战国时代徐州相王、五国相王后各大诸侯纷纷僭越称王(吴、越、楚三国春秋時代已称王),周王权威進一步受損。

周襄王十七年(前635年),发生“子带之乱”,襄王不能平,求救于晋文公,文公诛叔带,遂为伯而得河内地。周襄王二十年(前632年),襄王为晋文公所迫,于河阳踐土會盟。

周定王元年(前606年),楚庄王伐陆浑之戎,欲观九鼎。定王使王孙满应设以辞,楚人遂去。

周赧王时,東周国势益弱,同时内部争斗不休,以至分为东周国和西周国,赧王迁都西周國。周赧王八年(前307年),秦借道两周之间攻韩,周人两边都不敢得罪,左右为难。东西两周位于诸强国之间,不能同心协力,反而彼此攻杀。至赧王六十年(前256年),西周国为秦所灭,赧王死,七年后,东周国亦为秦所灭。

春秋时代的得名,是因孔子修订《春秋》而得名。一般而言,春秋时代从周平王五十年(前722年)起,直到周敬王四十三年(前477年)或四十四年(前476年)为止,也有学者认为春秋時代应到三家灭智(前453年)或三家分晋(前403年),原因是即使到三家分晉,除秦、楚、齊等國外,還有其他大小王國。

战国时代由三家分晉,春秋時代結束,直到秦統一中國(公元前221年)這段時間,一般稱為戰國時代。

值得注意的是,東周王朝在戰國後期(前256年)即已被秦所滅,所以戰國時代在時間上並不全然包含在東周裡面。

这一时期是中国歷史的社会制度转变的时期,这一转变是以铁器的广泛使用为标志的,東周基本上進入了鐵器時代。

在东周时期,铁器被广泛使用。农业有了长足的发展,从整体上来看人口不断成长,原来各诸侯国之间的无人地带,已不存在。各国因争夺土地或者水利资源,冲突时起。铜钱开始流行,甚至在楚国还出现了金币—郢钱,出现一定的商品经济和商人阶层。教育向平民普及。贵族与平民间的界限也被冲破。社会产生了一种革命性的变化,周王朝建立的宗法封建制度,已经不能适应这一变化。

東周與西周的地理位置差異反應在藝術表現上,尤其是東周晚期的藝術作品,展現了多元的風貌與高水準的技術。或許受到孔子反對以人殉葬的影響,以低溫燒製的陪葬塑像(明器,又稱「冥器」、「盟器」)數量增加。

東周時期亦出現低溫綠色鉛釉器皿、質地鬆的打磨黑色器皿、高溫釉器皿等。有些陶器仿效最新流行的漆器,色彩鮮明,有些則仿效青銅器。另有模製與裝飾的陶瓦、陶磚。西周時期較少見的玉雕再次成為重要的陪葬品與個人飾物。青銅的應用不限於宗教禮儀用途,變得較為世俗,常用作結婚贈禮之居家裝飾。青銅鐘及青銅鏡逐漸流行,動物和怪獸圖騰則由色彩繽紛而樣式化的裝飾圖案所取代。東周墓葬出土有最早繪於絲絹上的畫作。此外,亦發現了漢代及唐代陪葬陶器的前身。

%% -*- coding: utf-8 -*-
%% Time-stamp: <Chen Wang: 2021-11-01 18:13:59>

\subsection{平王宜臼{\tiny(BC772-BC720)}}

\subsubsection{生平}

周平王(前781年-前720年),姓姬,名宜臼,東周第一位天子。周幽王之子,母申后為申侯之女,後母褒姒。

姬宜臼本是周幽王太子,後因周幽王寵愛美女褒姒,褒姒生下一子伯服,得寵更甚。褒姒卻與權臣勾結,陰謀奪嫡,而幽王竟為了討美人歡心,便廢掉王后申后與太子宜臼母子,改封褒姒與其子伯服為王后和太子。宜臼逃奔申國。

前771年,周幽王被杀。申侯、缯侯及许文公,在申国擁立宜臼为周天子,即「周平王」。而虢公翰则在携,拥立周幽王之弟姬余臣为周天子,即「携王」(携惠王),形成「二王并立」的局面。

前770年,由於鎬京在戰後已殘破不堪,宜臼為避犬戎,在晉國和鄭國的支持下迁都雒邑,史稱「平王东迁」。周朝至此以後,為史家稱作「東周」。

前750年,携王被晋文侯所弑。前720年,周王宜臼去世,諡號平王。

當時傳聞平王和母親申后因被其父周幽王廢掉後一直懷恨在心,故意聯合諸侯國繒和外族犬戎入宮殺害幽王和褒姒,然後即位為天子,故此周平王有弑父之嫌,亦令周朝的國勢一落千丈。除晉鄭二國之外,其餘諸侯認為宜臼有弒父之嫌疑,故都不再聽從天子,平王唯“晉、鄭是依”,勉強支運國勢,而造成春秋時代的開始。

\subsubsection{年表}

% \centering
\begin{longtable}{|>{\centering\scriptsize}m{2em}|>{\centering\scriptsize}m{1.3em}|>{\centering}m{8.8em}|}
  % \caption{秦王政}\\
  \toprule
  \SimHei \normalsize 年数 & \SimHei \scriptsize 公元 & \SimHei 大事件 \tabularnewline
  % \midrule
  \endfirsthead
  \toprule
  \SimHei \normalsize 年数 & \SimHei \scriptsize 公元 & \SimHei 大事件 \tabularnewline
  \midrule
  \endhead
  \midrule
  % 元年 & -770 & \tabularnewline\hline
  % 二年 & -769 & \tabularnewline\hline
  % 三年 & -768 & \tabularnewline\hline
  % 四年 & -767 & \tabularnewline\hline
  % 五年 & -766 & \tabularnewline\hline
  % 六年 & -765 & \tabularnewline\hline
  % 七年 & -764 & \tabularnewline\hline
  % 八年 & -763 & \tabularnewline\hline
  % 九年 & -762 & \tabularnewline\hline
  % 十年 & -761 & \tabularnewline\hline
  % 十一年 & -760 & \tabularnewline\hline
  % 十二年 & -759 & \tabularnewline\hline
  % 十三年 & -758 & \tabularnewline\hline
  % 十四年 & -757 & \tabularnewline\hline
  % 十五年 & -756 & \tabularnewline\hline
  % 十六年 & -755 & \tabularnewline\hline
  % 十七年 & -754 & \tabularnewline\hline
  % 十八年 & -753 & \tabularnewline\hline
  % 十九年 & -752 & \tabularnewline\hline
  % 二十年 & -751 & \tabularnewline\hline
  % 二一年 & -750 & \tabularnewline\hline
  % 二二年 & -749 & \tabularnewline\hline
  % 二三年 & -748 & \tabularnewline\hline
  % 二四年 & -747 & \tabularnewline\hline
  % 二五年 & -746 & \tabularnewline\hline
  % 二六年 & -745 & \tabularnewline\hline
  % 二七年 & -744 & \tabularnewline\hline
  % 二八年 & -743 & \tabularnewline\hline
  % 二九年 & -742 & \tabularnewline\hline
  % 三十年 & -741 & \tabularnewline\hline
  % 三一年 & -740 & \tabularnewline\hline
  % 三二年 & -739 & \tabularnewline\hline
  % 三三年 & -738 & \tabularnewline\hline
  % 三四年 & -737 & \tabularnewline\hline
  % 三五年 & -736 & \tabularnewline\hline
  % 三六年 & -735 & \tabularnewline\hline
  % 三七年 & -734 & \tabularnewline\hline
  % 三八年 & -733 & \tabularnewline\hline
  % 三九年 & -732 & \tabularnewline\hline
  % 四十年 & -731 & \tabularnewline\hline
  % 四一年 & -730 & \tabularnewline\hline
  % 四二年 & -729 & \tabularnewline\hline
  % 四三年 & -728 & \tabularnewline\hline
  % 四四年 & -727 & \tabularnewline\hline
  % 四五年 & -726 & \tabularnewline\hline
  % 四六年 & -725 & \tabularnewline\hline
  % 四七年 & -724 & \tabularnewline\hline
  % 四八年 & -723 & \tabularnewline\hline
  四九年 & -722 & \tabularnewline\hline
  五十年 & -721 & \tabularnewline\hline
  五一年 & -720 & \tabularnewline
  \bottomrule
\end{longtable}

%%% Local Variables:
%%% mode: latex
%%% TeX-engine: xetex
%%% TeX-master: "../../Main"
%%% End:

%% -*- coding: utf-8 -*-
%% Time-stamp: <Chen Wang: 2021-11-01 18:25:56>

\subsection{桓王林\tiny{(BC719-BC697)}}

\subsubsection{生平}

周桓王(?-前697年),姓姬,名林,東周第二代君主,諡桓。他是周平王之孫,因平王駕崩時,太子洩父早死,其作为洩父之子得以繼承天子之位。

桓王(前720年)即位後,計劃削弱卿權以加強王權。加上周朝領地與鄭國領地鄰接,鄭國多次越界取禾,故桓王罷免了鄭莊公卿士之職。莊公不滿,便不再朝見天子。周王室始與鄭國交惡。

桓王二年(前718年)春天,晉國曲沃封君曲沃莊伯賄賂周桓王,聯合鄭國、邢國攻打晉國都城翼城,晉鄂侯戰敗,逃奔隨邑。同年夏天,晉鄂侯去世,曲沃莊伯於是再度攻打晉國。由於當時曲沃莊伯背叛周桓王,同年秋天,周桓王便反過來支持晉國,並派遣虢公 率領軍隊討伐曲沃莊伯,曲沃莊伯兵敗,只得逃回曲沃防守。周桓王立晉鄂侯之子晉哀侯為君。

桓王十二年(前708年)周桓王聯合秦國出兵包圍芮國,俘虜芮國國君芮伯萬。

桓王十三年(前707年),桓王率蔡、衞、陳聯軍攻鄭,大敗於繻葛。桓王本人更於此役中為鄭將祝聃射箭中傷。此後桓王雖然仍能影響虢國,但已無力阻止周王室轉衰之势,也無力阻止諸侯間的互相攻伐。

桓王十八年(前703年),虢仲向周桓王進讒言誣陷大夫詹父。周桓王認為詹父有理,詹父於是帶領周天子的軍隊進攻虢國。同年夏天,虢公逃亡到虞國

桓王二十三年(前697年),桓王三月因病駕崩,儿子王子佗继位,是為周莊王。

\subsubsection{年表}

% \centering
\begin{longtable}{|>{\centering\scriptsize}m{2em}|>{\centering\scriptsize}m{1.3em}|>{\centering}m{8.8em}|}
  % \caption{秦王政}\\
  \toprule
  \SimHei \normalsize 年数 & \SimHei \scriptsize 公元 & \SimHei 大事件 \tabularnewline
  % \midrule
  \endfirsthead
  \toprule
  \SimHei \normalsize 年数 & \SimHei \scriptsize 公元 & \SimHei 大事件 \tabularnewline
  \midrule
  \endhead
  \midrule
  元年 & -719 & \tabularnewline\hline
  二年 & -718 & \tabularnewline\hline
  三年 & -717 & \tabularnewline\hline
  四年 & -716 & \tabularnewline\hline
  五年 & -715 & \tabularnewline\hline
  六年 & -714 & \tabularnewline\hline
  七年 & -713 & \tabularnewline\hline
  八年 & -712 & \tabularnewline\hline
  九年 & -711 & \tabularnewline\hline
  十年 & -710 & \tabularnewline\hline
  十一年 & -709 & \tabularnewline\hline
  十二年 & -708 & \tabularnewline\hline
  十三年 & -707 & \tabularnewline\hline
  十四年 & -706 & \tabularnewline\hline
  十五年 & -705 & \tabularnewline\hline
  十六年 & -704 & \tabularnewline\hline
  十七年 & -703 & \tabularnewline\hline
  十八年 & -702 & \tabularnewline\hline
  十九年 & -701 & \tabularnewline\hline
  二十年 & -700 & \tabularnewline\hline
  二一年 & -699 & \tabularnewline\hline
  二二年 & -698 & \tabularnewline\hline
  二三年 & -697 & \tabularnewline
  \bottomrule
\end{longtable}

%%% Local Variables:
%%% mode: latex
%%% TeX-engine: xetex
%%% TeX-master: "../../Main"
%%% End:

%% -*- coding: utf-8 -*-
%% Time-stamp: <Chen Wang: 2021-11-02 14:37:57>

\subsection{莊王佗\tiny{(BC696-BC682)}}

\subsubsection{生平}

周莊王(?-前682年),姓姬,名佗,東周第三代國王,諡莊。他是周桓王之兒。莊王三年(前694年),周公黑肩欲殺莊王,而立莊王弟王子克,事泄,黑肩被莊王與辛伯所殺,克奔南燕(河南延津)。

在位期间执政为虢公林父、周公黑肩。

\subsubsection{年表}

% \centering
\begin{longtable}{|>{\centering\scriptsize}m{2em}|>{\centering\scriptsize}m{1.3em}|>{\centering}m{8.8em}|}
  % \caption{秦王政}\\
  \toprule
  \SimHei \normalsize 年数 & \SimHei \scriptsize 公元 & \SimHei 大事件 \tabularnewline
  % \midrule
  \endfirsthead
  \toprule
  \SimHei \normalsize 年数 & \SimHei \scriptsize 公元 & \SimHei 大事件 \tabularnewline
  \midrule
  \endhead
  \midrule
  元年 & -696 & \tabularnewline\hline
  二年 & -695 & \tabularnewline\hline
  三年 & -694 & \tabularnewline\hline
  四年 & -693 & \tabularnewline\hline
  五年 & -692 & \tabularnewline\hline
  六年 & -691 & \tabularnewline\hline
  七年 & -690 & \tabularnewline\hline
  八年 & -689 & \tabularnewline\hline
  九年 & -688 & \tabularnewline\hline
  十年 & -687 & \tabularnewline\hline
  十一年 & -686 & \tabularnewline\hline
  十二年 & -685 & \tabularnewline\hline
  十三年 & -684 & \tabularnewline\hline
  十四年 & -683 & \tabularnewline\hline
  十五年 & -682 & \tabularnewline  
  \bottomrule
\end{longtable}

%%% Local Variables:
%%% mode: latex
%%% TeX-engine: xetex
%%% TeX-master: "../../Main"
%%% End:

%% -*- coding: utf-8 -*-
%% Time-stamp: <Chen Wang: 2021-11-02 14:42:43>

\subsection{僖王胡齐\tiny{(BC681-BC677)}}

\subsubsection{生平}

周僖王(?-前677年),又作周釐王,姓姬,名胡齐,東周第四代君主,在位5年,號僖。

僖王爲周莊王長子,莊王雖偏爱姚姬生的少子王子頹,但未能廢長立幼。前681年,僖王即位。

前679年,曲沃克晉後,曲沃武公把晉國的寶器獻給僖王,僖王承認曲沃武公為晉君,列為諸侯。前678年,遭晉軍攻打並殺害夷邑大夫詭諸,執政大臣周公忌父逃奔虢國。

前677年,周僖王崩。

在位期间执政为虢公醜、周公忌父。

\subsubsection{年表}

% \centering
\begin{longtable}{|>{\centering\scriptsize}m{2em}|>{\centering\scriptsize}m{1.3em}|>{\centering}m{8.8em}|}
  % \caption{秦王政}\\
  \toprule
  \SimHei \normalsize 年数 & \SimHei \scriptsize 公元 & \SimHei 大事件 \tabularnewline
  % \midrule
  \endfirsthead
  \toprule
  \SimHei \normalsize 年数 & \SimHei \scriptsize 公元 & \SimHei 大事件 \tabularnewline
  \midrule
  \endhead
  \midrule
  元年 & -681 & \tabularnewline\hline
  二年 & -680 & \tabularnewline\hline
  三年 & -679 & \tabularnewline\hline
  四年 & -678 & \tabularnewline\hline
  五年 & -677 & \tabularnewline
  \bottomrule
\end{longtable}

%%% Local Variables:
%%% mode: latex
%%% TeX-engine: xetex
%%% TeX-master: "../../Main"
%%% End:

%% -*- coding: utf-8 -*-
%% Time-stamp: <Chen Wang: 2021-11-02 14:59:42>

\subsection{惠王閬\tiny{(BC676-BC652)}}

\subsubsection{生平}

周惠王(?-前652年),姓姬,名閬,又名聞,東周第五代君主,諡惠。他是周僖王之子。

周惠王在前677年繼位後,佔用蒍國的園圃飼養野獸,蒍國的人民不滿,惠王二年,有五大夫作亂,立王子頹為周天子,惠王奔溫(今河南溫縣南),鄭厲公在櫟地(今禹州市)收容惠王,並在惠王四年與虢國攻入周朝,協助平定“子頹之亂”,惠王復辟,鄭國因功獲賜予虎牢(今河南滎陽汜水鎮)以東的地方,虢國也獲賜土地。

周惠王晚年宠爱幼子王子带,欲立為嗣,约郑联楚、晋以成此事,但此时齐桓公稱霸天下,与诸侯会盟力挺太子,周惠王未能如愿。周惠王駕崩后,太子周襄王即位。

《史記·周本紀》稱惠王在位25年,《左傳》稱周惠王在魯僖公七年(前653年)冬天駕崩。

在位期间执政为虢公醜、周公忌父、宰孔。

\subsubsection{年表}

% \centering
\begin{longtable}{|>{\centering\scriptsize}m{2em}|>{\centering\scriptsize}m{1.3em}|>{\centering}m{8.8em}|}
  % \caption{秦王政}\\
  \toprule
  \SimHei \normalsize 年数 & \SimHei \scriptsize 公元 & \SimHei 大事件 \tabularnewline
  % \midrule
  \endfirsthead
  \toprule
  \SimHei \normalsize 年数 & \SimHei \scriptsize 公元 & \SimHei 大事件 \tabularnewline
  \midrule
  \endhead
  \midrule
  元年 & -676 & \tabularnewline\hline
  二年 & -675 & \tabularnewline\hline
  三年 & -674 & \tabularnewline\hline
  四年 & -673 & \tabularnewline\hline
  五年 & -672 & \tabularnewline\hline
  六年 & -671 & \tabularnewline\hline
  七年 & -670 & \tabularnewline\hline
  八年 & -669 & \tabularnewline\hline
  九年 & -668 & \tabularnewline\hline
  十年 & -667 & \tabularnewline\hline
  十一年 & -666 & \tabularnewline\hline
  十二年 & -665 & \tabularnewline\hline
  十三年 & -664 & \tabularnewline\hline
  十四年 & -663 & \tabularnewline\hline
  十五年 & -662 & \tabularnewline\hline
  十六年 & -661 & \tabularnewline\hline
  十七年 & -660 & \tabularnewline\hline
  十八年 & -659 & \tabularnewline\hline
  十九年 & -658 & \tabularnewline\hline
  二十年 & -657 & \tabularnewline\hline
  二一年 & -656 & \tabularnewline\hline
  二二年 & -655 & \tabularnewline\hline
  二三年 & -654 & \tabularnewline\hline
  二四年 & -653 & \tabularnewline\hline
  二五年 & -652 & \tabularnewline  
  \bottomrule
\end{longtable}

%%% Local Variables:
%%% mode: latex
%%% TeX-engine: xetex
%%% TeX-master: "../../Main"
%%% End:

%% -*- coding: utf-8 -*-
%% Time-stamp: <Chen Wang: 2021-11-02 15:06:10>

\subsection{襄王鄭\tiny{(BC651-BC619)}}

\subsubsection{生平}

周襄王(?-前619年),姬姓,名鄭,東周第六代君主,諡襄。襄王是周惠王之子, 《史記·周本紀》稱襄王在位32年,《左傳》稱襄王崩於魯文公八年(前619年)秋。

惠王死後,襄王懼怕異母弟王子帶爭奪王位繼承權,秘不發喪,並派人向齊國求援,襄王直到大局已定才公佈父王死訊。

周襄王時,因鄭伯不聽王命,曾經要求狄人攻打鄭國。在狄人擊敗鄭國後,周襄王娶隗姓狄人為妻,又將隗后退回。此舉引發狄人不滿,前636年,王子帶聯合狄人,攻打周襄王,周襄王逃到鄭國。

當時晉文公勢力強大,在前635年出兵助襄王,殺王子帶,迎接周襄王返回洛陽復位。前632年,晉文公召襄王,襄王親自到踐土(今河南原陽西南)會見他。

周襄王在位期间宋襄公、晋文公、秦穆公相继称霸。在位期间执政为宰孔、周公忌父、王子虎、周公閱、王叔桓公。

\subsubsection{年表}

% \centering
\begin{longtable}{|>{\centering\scriptsize}m{2em}|>{\centering\scriptsize}m{1.3em}|>{\centering}m{8.8em}|}
  % \caption{秦王政}\\
  \toprule
  \SimHei \normalsize 年数 & \SimHei \scriptsize 公元 & \SimHei 大事件 \tabularnewline
  % \midrule
  \endfirsthead
  \toprule
  \SimHei \normalsize 年数 & \SimHei \scriptsize 公元 & \SimHei 大事件 \tabularnewline
  \midrule
  \endhead
  \midrule
  元年 & -651 & \tabularnewline\hline
  二年 & -650 & \tabularnewline\hline
  三年 & -649 & \tabularnewline\hline
  四年 & -648 & \tabularnewline\hline
  五年 & -647 & \tabularnewline\hline
  六年 & -646 & \tabularnewline\hline
  七年 & -645 & \tabularnewline\hline
  八年 & -644 & \tabularnewline\hline
  九年 & -643 & \tabularnewline\hline
  十年 & -642 & \tabularnewline\hline
  十一年 & -641 & \tabularnewline\hline
  十二年 & -640 & \tabularnewline\hline
  十三年 & -639 & \tabularnewline\hline
  十四年 & -638 & \tabularnewline\hline
  十五年 & -637 & \tabularnewline\hline
  十六年 & -636 & \tabularnewline\hline
  十七年 & -635 & \tabularnewline\hline
  十八年 & -634 & \tabularnewline\hline
  十九年 & -633 & \tabularnewline\hline
  二十年 & -632 & \tabularnewline\hline
  二一年 & -631 & \tabularnewline\hline
  二二年 & -630 & \tabularnewline\hline
  二三年 & -629 & \tabularnewline\hline
  二四年 & -628 & \tabularnewline\hline
  二五年 & -627 & \tabularnewline\hline
  二六年 & -626 & \tabularnewline\hline
  二七年 & -625 & \tabularnewline\hline
  二八年 & -624 & \tabularnewline\hline
  二九年 & -623 & \tabularnewline\hline
  三十年 & -622 & \tabularnewline\hline
  三一年 & -621 & \tabularnewline\hline
  三二年 & -620 & \tabularnewline\hline
  三三年 & -619 & \tabularnewline  
  \bottomrule
\end{longtable}

%%% Local Variables:
%%% mode: latex
%%% TeX-engine: xetex
%%% TeX-master: "../../Main"
%%% End:

%% -*- coding: utf-8 -*-
%% Time-stamp: <Chen Wang: 2021-11-02 15:08:18>

\subsection{頃王壬臣\tiny{(BC618-BC613)}}

\subsubsection{生平}

周頃王(?-前613年),姓姬,名壬臣,為周襄王之子。周頃王在前618年繼位為東周第七代君主,當時王畿已縮小,王室財政一貧如洗,無法安葬襄王,頃王只得派毛伯衛向魯國討錢。後來魯文公派使者送錢到都城,才安葬了襄王。

頃王在前613年春天去世,在位6年,由兒子周匡王繼位。

在位期间执政为周公閱、王叔桓公、王孫蘇。

\subsubsection{年表}

% \centering
\begin{longtable}{|>{\centering\scriptsize}m{2em}|>{\centering\scriptsize}m{1.3em}|>{\centering}m{8.8em}|}
  % \caption{秦王政}\\
  \toprule
  \SimHei \normalsize 年数 & \SimHei \scriptsize 公元 & \SimHei 大事件 \tabularnewline
  % \midrule
  \endfirsthead
  \toprule
  \SimHei \normalsize 年数 & \SimHei \scriptsize 公元 & \SimHei 大事件 \tabularnewline
  \midrule
  \endhead
  \midrule
  元年 & -618 & \tabularnewline\hline
  二年 & -617 & \tabularnewline\hline
  三年 & -616 & \tabularnewline\hline
  四年 & -615 & \tabularnewline\hline
  五年 & -614 & \tabularnewline\hline
  六年 & -613 & \tabularnewline  
  \bottomrule
\end{longtable}

%%% Local Variables:
%%% mode: latex
%%% TeX-engine: xetex
%%% TeX-master: "../../Main"
%%% End:

%% -*- coding: utf-8 -*-
%% Time-stamp: <Chen Wang: 2021-11-02 15:10:26>

\subsection{匡王班\tiny{(BC612-BC607)}}

\subsubsection{生平}

周匡王(?-前607年),姓姬,名班,中国东周第8代君主,前612年至前607年在位,共6年。匡王是周頃王之子。前607年十月,周王班崩,諡“匡”,其弟王子瑜繼位,即周定王。

在位期间执政为周公閱、王孫蘇、召桓公、毛伯衛。

\subsubsection{年表}

% \centering
\begin{longtable}{|>{\centering\scriptsize}m{2em}|>{\centering\scriptsize}m{1.3em}|>{\centering}m{8.8em}|}
  % \caption{秦王政}\\
  \toprule
  \SimHei \normalsize 年数 & \SimHei \scriptsize 公元 & \SimHei 大事件 \tabularnewline
  % \midrule
  \endfirsthead
  \toprule
  \SimHei \normalsize 年数 & \SimHei \scriptsize 公元 & \SimHei 大事件 \tabularnewline
  \midrule
  \endhead
  \midrule
  元年 & -612 & \tabularnewline\hline
  二年 & -611 & \tabularnewline\hline
  三年 & -610 & \tabularnewline\hline
  四年 & -609 & \tabularnewline\hline
  五年 & -608 & \tabularnewline\hline
  六年 & -607 & \tabularnewline
  \bottomrule
\end{longtable}

%%% Local Variables:
%%% mode: latex
%%% TeX-engine: xetex
%%% TeX-master: "../../Main"
%%% End:

%% -*- coding: utf-8 -*-
%% Time-stamp: <Chen Wang: 2021-11-02 15:14:57>

\subsection{定王瑜\tiny{(BC606-BC586)}}

\subsubsection{生平}

周定王(?-前586年),姓姬,名瑜,中国東周第9代天子,前606年—前586年在位,周定王是匡王之弟。定王在位21年而卒,子夷立,為簡王。

在位期间执政为王孫蘇、召桓公、劉康公、毛伯衛、單襄公。

楚莊王为稱霸天下,不斷北侵並打敗了晋国、齐国、宋国、郑国、陈国、蔡国等國,在定王元年征伐陆浑之戎,進軍到周京雒邑的南郊,向周王耀武示威。定王不敢責問楚莊王,便派王孫滿去慰勞楚軍,楚莊王詢問周朝鎮國之寶的九鼎大小輕重,欲逼周取天下。後王孫滿以婉辭說服了楚莊王,使楚不敢輕舉妄動去取代周朝,便撤兵回國。

\subsubsection{年表}

% \centering
\begin{longtable}{|>{\centering\scriptsize}m{2em}|>{\centering\scriptsize}m{1.3em}|>{\centering}m{8.8em}|}
  % \caption{秦王政}\\
  \toprule
  \SimHei \normalsize 年数 & \SimHei \scriptsize 公元 & \SimHei 大事件 \tabularnewline
  % \midrule
  \endfirsthead
  \toprule
  \SimHei \normalsize 年数 & \SimHei \scriptsize 公元 & \SimHei 大事件 \tabularnewline
  \midrule
  \endhead
  \midrule
  元年 & -606 & \tabularnewline\hline
  二年 & -605 & \tabularnewline\hline
  三年 & -604 & \tabularnewline\hline
  四年 & -603 & \tabularnewline\hline
  五年 & -602 & \tabularnewline\hline
  六年 & -601 & \tabularnewline\hline
  七年 & -600 & \tabularnewline\hline
  八年 & -599 & \tabularnewline\hline
  九年 & -598 & \tabularnewline\hline
  十年 & -597 & \tabularnewline\hline
  十一年 & -596 & \tabularnewline\hline
  十二年 & -595 & \tabularnewline\hline
  十三年 & -594 & \tabularnewline\hline
  十四年 & -593 & \tabularnewline\hline
  十五年 & -592 & \tabularnewline\hline
  十六年 & -591 & \tabularnewline\hline
  十七年 & -590 & \tabularnewline\hline
  十八年 & -589 & \tabularnewline\hline
  十九年 & -588 & \tabularnewline\hline
  二十年 & -587 & \tabularnewline\hline
  二一年 & -586 & \tabularnewline  
  \bottomrule
\end{longtable}

%%% Local Variables:
%%% mode: latex
%%% TeX-engine: xetex
%%% TeX-master: "../../Main"
%%% End:

%% -*- coding: utf-8 -*-
%% Time-stamp: <Chen Wang: 2021-11-02 15:13:42>

\subsection{簡王夷\tiny{(BC585-BC572)}}

\subsubsection{生平}

周簡王(?-前572年),姓姬,名夷,為周定王之子。在位14年,此期間晋、楚、秦,宋、郑等国相互攻伐不止,吴国兴起,攻入楚国,幾乎亡楚。前572年九月,周王夷病死,諡号为简王。

在位期间执政为單襄公、劉康公、周公楚、尹武公。

子周灵王、儋季。

\subsubsection{年表}

% \centering
\begin{longtable}{|>{\centering\scriptsize}m{2em}|>{\centering\scriptsize}m{1.3em}|>{\centering}m{8.8em}|}
  % \caption{秦王政}\\
  \toprule
  \SimHei \normalsize 年数 & \SimHei \scriptsize 公元 & \SimHei 大事件 \tabularnewline
  % \midrule
  \endfirsthead
  \toprule
  \SimHei \normalsize 年数 & \SimHei \scriptsize 公元 & \SimHei 大事件 \tabularnewline
  \midrule
  \endhead
  \midrule
  元年 & -585 & \tabularnewline\hline
  二年 & -584 & \tabularnewline\hline
  三年 & -583 & \tabularnewline\hline
  四年 & -582 & \tabularnewline\hline
  五年 & -581 & \tabularnewline\hline
  六年 & -580 & \tabularnewline\hline
  七年 & -579 & \tabularnewline\hline
  八年 & -578 & \tabularnewline\hline
  九年 & -577 & \tabularnewline\hline
  十年 & -576 & \tabularnewline\hline
  十一年 & -575 & \tabularnewline\hline
  十二年 & -574 & \tabularnewline\hline
  十三年 & -573 & \tabularnewline\hline
  十四年 & -572 & \tabularnewline  
  \bottomrule
\end{longtable}

%%% Local Variables:
%%% mode: latex
%%% TeX-engine: xetex
%%% TeX-master: "../../Main"
%%% End:

%% -*- coding: utf-8 -*-
%% Time-stamp: <Chen Wang: 2021-11-02 15:17:28>

\subsection{靈王泄心\tiny{(BC571-BC545)}}

\subsubsection{生平}

周靈王(?-前545年),姓姬,名泄心,是周简王之子,東周第11代國王,在位27年。《幼学琼林》中说他出生时便有胡须。

《列仙傳》中記載:周灵王的长子太子晋天性聪明,善吹笙,立他为太子,不幸早逝。公元前545年十一月的某天夜裡,周灵王梦见太子骑着白鹤来迎接他。传位于次子王子贵,癸巳日,病死。孔子在周靈王二十一年出生于魯。

在位期間執政為王叔陳生、伯輿、單靖公。

\subsubsection{年表}

% \centering
\begin{longtable}{|>{\centering\scriptsize}m{2em}|>{\centering\scriptsize}m{1.3em}|>{\centering}m{8.8em}|}
  % \caption{秦王政}\\
  \toprule
  \SimHei \normalsize 年数 & \SimHei \scriptsize 公元 & \SimHei 大事件 \tabularnewline
  % \midrule
  \endfirsthead
  \toprule
  \SimHei \normalsize 年数 & \SimHei \scriptsize 公元 & \SimHei 大事件 \tabularnewline
  \midrule
  \endhead
  \midrule
  元年 & -571 & \tabularnewline\hline
  二年 & -570 & \tabularnewline\hline
  三年 & -569 & \tabularnewline\hline
  四年 & -568 & \tabularnewline\hline
  五年 & -567 & \tabularnewline\hline
  六年 & -566 & \tabularnewline\hline
  七年 & -565 & \tabularnewline\hline
  八年 & -564 & \tabularnewline\hline
  九年 & -563 & \tabularnewline\hline
  十年 & -562 & \tabularnewline\hline
  十一年 & -561 & \tabularnewline\hline
  十二年 & -560 & \tabularnewline\hline
  十三年 & -559 & \tabularnewline\hline
  十四年 & -558 & \tabularnewline\hline
  十五年 & -557 & \tabularnewline\hline
  十六年 & -556 & \tabularnewline\hline
  十七年 & -555 & \tabularnewline\hline
  十八年 & -554 & \tabularnewline\hline
  十九年 & -553 & \tabularnewline\hline
  二十年 & -552 & \tabularnewline\hline
  二一年 & -551 & \tabularnewline\hline
  二二年 & -550 & \tabularnewline\hline
  二三年 & -549 & \tabularnewline\hline
  二四年 & -548 & \tabularnewline\hline
  二五年 & -547 & \tabularnewline\hline
  二六年 & -546 & \tabularnewline\hline
  二七年 & -545 & \tabularnewline  
  \bottomrule
\end{longtable}

%%% Local Variables:
%%% mode: latex
%%% TeX-engine: xetex
%%% TeX-master: "../../Main"
%%% End:

%% -*- coding: utf-8 -*-
%% Time-stamp: <Chen Wang: 2021-11-02 15:21:09>

\subsection{景王貴\tiny{(BC544-BC520)}}

\subsubsection{生平}

周景王(?-前520年),姓姬,名貴,中國東周君主,諡號景,為周靈王之子。周景王在位时,财政困難,連器皿都要向各国索讨。

有一次,景王宴请晋国大臣知文子荀跞,指着鲁国送来的酒壶说:“各国都有器物送给天子,为何晋国没有?”荀跞答不出来,請副使籍谈答覆,籍谈说当初晋国受封时,未赐以礼器,現在晋国忙于对付戎狄,當然無法送出礼物。周景王列数王室赐给晋的土地器物,讽刺其“数典而忘其祖”,这是“数典忘祖”的典故。此時周天子的地位已经一落千丈。

周景王太子寿早死,立王子猛为太子,卻宠爱庶长子王子朝,還一度打算刺殺支持王子猛的單穆公、劉文公。公元前520年四月,周景王病重,嘱咐宾孟要擁立王子朝。王子朝未及立为嗣君,景王却突然病死,由王子猛即位[2]。

周景王在位期間執政為單靖公、劉定公、成簡公、單獻公、單成公、劉獻公、單穆公。

\subsubsection{年表}

% \centering
\begin{longtable}{|>{\centering\scriptsize}m{2em}|>{\centering\scriptsize}m{1.3em}|>{\centering}m{8.8em}|}
  % \caption{秦王政}\\
  \toprule
  \SimHei \normalsize 年数 & \SimHei \scriptsize 公元 & \SimHei 大事件 \tabularnewline
  % \midrule
  \endfirsthead
  \toprule
  \SimHei \normalsize 年数 & \SimHei \scriptsize 公元 & \SimHei 大事件 \tabularnewline
  \midrule
  \endhead
  \midrule
  元年 & -544 & \tabularnewline\hline
  二年 & -543 & \tabularnewline\hline
  三年 & -542 & \tabularnewline\hline
  四年 & -541 & \tabularnewline\hline
  五年 & -540 & \tabularnewline\hline
  六年 & -539 & \tabularnewline\hline
  七年 & -538 & \tabularnewline\hline
  八年 & -537 & \tabularnewline\hline
  九年 & -536 & \tabularnewline\hline
  十年 & -535 & \tabularnewline\hline
  十一年 & -534 & \tabularnewline\hline
  十二年 & -533 & \tabularnewline\hline
  十三年 & -532 & \tabularnewline\hline
  十四年 & -531 & \tabularnewline\hline
  十五年 & -530 & \tabularnewline\hline
  十六年 & -529 & \tabularnewline\hline
  十七年 & -528 & \tabularnewline\hline
  十八年 & -527 & \tabularnewline\hline
  十九年 & -526 & \tabularnewline\hline
  二十年 & -525 & \tabularnewline\hline
  二一年 & -524 & \tabularnewline\hline
  二二年 & -523 & \tabularnewline\hline
  二三年 & -522 & \tabularnewline\hline
  二四年 & -521 & \tabularnewline\hline
  二五年 & -520 & \tabularnewline  
  \bottomrule
\end{longtable}

%%% Local Variables:
%%% mode: latex
%%% TeX-engine: xetex
%%% TeX-master: "../../Main"
%%% End:

%% -*- coding: utf-8 -*-
%% Time-stamp: <Chen Wang: 2021-11-02 15:22:09>

\subsection{悼王猛\tiny{(BC520-BC520)}}

\subsubsection{生平}

周悼王(?-前520年),姓姬,名猛,中國東周君主,諡號悼王,未即位时称王子猛,即位后称王猛。他是周景王的儿子,景王病重時,囑咐大夫賓孟立王子朝。景王死,國人立王子猛為王,是為悼王。悼王後來被王子朝殺死。

周悼王为周景王之子。《左传》记载周景王十八年(前527年),其太子寿和王后去世。其后围绕着周景王的太子人选,朝中形成了对立的两派。一派为支持王子猛的大夫单子单穆公、刘子刘文公。另一派支持王子朝,以王子朝之傅宾起(又称为“宾孟”)为代表。周景王宠爱王子朝,并杀王子猛之傅下门子(见《国语·周语下第三》,取徐元诰《国语集解》之解释)。

周景王二十五年夏四月,景王命令公卿随同自己前往北山田猎。景王计划趁此杀死单子、刘子,然后册立王子朝为太子。但计划尚未实行,周景王突然驾崩,按《左传》的说法是死于心脏病。王子猛即位为王,单子杀宾起,王猛之位初定。即位之时王猛的身份,据《史记·周本纪》是“长子”,按《国语集解》之说是景王朝的太子,对此《左传》无明确记载。《左传·昭公二十六年》记载的王子朝告诸侯文,强调了王后无子的情况下,应该立长子为太子的礼制,而批评单子、刘子为私利而立少子的行为。由此可见,王子朝和王子猛都不是景王的王后之子。而王子朝比王子猛年长。

\subsubsection{年表}

% \centering
\begin{longtable}{|>{\centering\scriptsize}m{2em}|>{\centering\scriptsize}m{1.3em}|>{\centering}m{8.8em}|}
  % \caption{秦王政}\\
  \toprule
  \SimHei \normalsize 年数 & \SimHei \scriptsize 公元 & \SimHei 大事件 \tabularnewline
  % \midrule
  \endfirsthead
  \toprule
  \SimHei \normalsize 年数 & \SimHei \scriptsize 公元 & \SimHei 大事件 \tabularnewline
  \midrule
  \endhead
  \midrule
  元年 & -520 & \tabularnewline  
  \bottomrule
\end{longtable}

%%% Local Variables:
%%% mode: latex
%%% TeX-engine: xetex
%%% TeX-master: "../../Main"
%%% End:

%% -*- coding: utf-8 -*-
%% Time-stamp: <Chen Wang: 2021-11-02 15:24:59>

\subsection{敬王匄\tiny{(BC519-BC476)}}

\subsubsection{生平}

周敬王(前6世纪?-前477年),姬姓,名匄(音同「丐」),中國東周君主,諡號敬王。他是周景王的兒子,周悼王同母弟。

周景王的庶長子王子朝在悼王病死後,自立為王。晉國派兵攻打王子朝,立王子匄為王。此後敬王與王子朝不時仍有衝突。前516年王子朝逃到楚國。前505年春,楚国被吴国击败,险些亡国,周敬王趁机派人在楚地殺死王子朝。儋翩帶領王子朝支持者在次年起兵舉事,敬王出逃,在前503年得晉國幫助下回都。

東周自周平王開始以雒邑(洛邑,又稱成周)為都城,平王東遷後,又稱雒邑為王城。敬王時,因王子朝在雒邑勢大,乃遷都至雒邑之東,稱新都為成周,稱舊都為王城。

周敬王二十六年(前494年),吳王夫差為父報仇,起兵擊敗越國,越王勾踐求和,並獻上美女西施給夫差。《左傳·哀公十九年》記載,冬,周敬王去世,葬于三王陵(今河南省洛阳市西南10里处)。

在位期間執政為單穆公、劉文公、單武公、劉桓公、萇弘、單平公。

\subsubsection{年表}

% \centering
\begin{longtable}{|>{\centering\scriptsize}m{2em}|>{\centering\scriptsize}m{1.3em}|>{\centering}m{8.8em}|}
  % \caption{秦王政}\\
  \toprule
  \SimHei \normalsize 年数 & \SimHei \scriptsize 公元 & \SimHei 大事件 \tabularnewline
  % \midrule
  \endfirsthead
  \toprule
  \SimHei \normalsize 年数 & \SimHei \scriptsize 公元 & \SimHei 大事件 \tabularnewline
  \midrule
  \endhead
  \midrule
  元年 & -519 & \tabularnewline\hline
  二年 & -518 & \tabularnewline\hline
  三年 & -517 & \tabularnewline\hline
  四年 & -516 & \tabularnewline\hline
  五年 & -515 & \tabularnewline\hline
  六年 & -514 & \tabularnewline\hline
  七年 & -513 & \tabularnewline\hline
  八年 & -512 & \tabularnewline\hline
  九年 & -511 & \tabularnewline\hline
  十年 & -510 & \tabularnewline\hline
  十一年 & -509 & \tabularnewline\hline
  十二年 & -508 & \tabularnewline\hline
  十三年 & -507 & \tabularnewline\hline
  十四年 & -506 & \tabularnewline\hline
  十五年 & -505 & \tabularnewline\hline
  十六年 & -504 & \tabularnewline\hline
  十七年 & -503 & \tabularnewline\hline
  十八年 & -502 & \tabularnewline\hline
  十九年 & -501 & \tabularnewline\hline
  二十年 & -500 & \tabularnewline\hline
  二一年 & -499 & \tabularnewline\hline
  二二年 & -498 & \tabularnewline\hline
  二三年 & -497 & \tabularnewline\hline
  二四年 & -496 & \tabularnewline\hline
  二五年 & -495 & \tabularnewline\hline
  二六年 & -494 & \tabularnewline\hline
  二七年 & -493 & \tabularnewline\hline
  二八年 & -492 & \tabularnewline\hline
  二九年 & -491 & \tabularnewline\hline
  三十年 & -490 & \tabularnewline\hline
  三一年 & -489 & \tabularnewline\hline
  三二年 & -488 & \tabularnewline\hline
  三三年 & -487 & \tabularnewline\hline
  三四年 & -486 & \tabularnewline\hline
  三五年 & -485 & \tabularnewline\hline
  三六年 & -484 & \tabularnewline\hline
  三七年 & -483 & \tabularnewline\hline
  三八年 & -482 & \tabularnewline\hline
  三九年 & -481 & \tabularnewline\hline
  四十年 & -480 & \tabularnewline\hline
  四一年 & -479 & \tabularnewline\hline
  四二年 & -478 & \tabularnewline\hline
  四三年 & -477 & \tabularnewline\hline
  四四年 & -476 & \tabularnewline  
  \bottomrule
\end{longtable}

%%% Local Variables:
%%% mode: latex
%%% TeX-engine: xetex
%%% TeX-master: "../../Main"
%%% End:

%% -*- coding: utf-8 -*-
%% Time-stamp: <Chen Wang: 2021-11-02 15:26:32>

\subsection{元王仁\tiny{(BC475-BC469)}}

\subsubsection{生平}

周元王(?-前469年),姓姬,名仁,中國東周君主,在位8年,為周敬王之子。

周元王四年(前473年),越王句踐滅吳。句踐隨後北上遷都琅琊,與齊國、晉國等諸侯會盟於徐州(今山東滕縣南),「越兵橫行於江、淮東,諸侯畢賀,號稱霸王」,周元王正式承認句踐為霸主。

\subsubsection{年表}

% \centering
\begin{longtable}{|>{\centering\scriptsize}m{2em}|>{\centering\scriptsize}m{1.3em}|>{\centering}m{8.8em}|}
  % \caption{秦王政}\\
  \toprule
  \SimHei \normalsize 年数 & \SimHei \scriptsize 公元 & \SimHei 大事件 \tabularnewline
  % \midrule
  \endfirsthead
  \toprule
  \SimHei \normalsize 年数 & \SimHei \scriptsize 公元 & \SimHei 大事件 \tabularnewline
  \midrule
  \endhead
  \midrule
  元年 & -475 & \tabularnewline\hline
  二年 & -474 & \tabularnewline\hline
  三年 & -473 & \tabularnewline\hline
  四年 & -472 & \tabularnewline\hline
  五年 & -471 & \tabularnewline\hline
  六年 & -470 & \tabularnewline\hline
  七年 & -469 & \tabularnewline  
  \bottomrule
\end{longtable}

%%% Local Variables:
%%% mode: latex
%%% TeX-engine: xetex
%%% TeX-master: "../../Main"
%%% End:

%% -*- coding: utf-8 -*-
%% Time-stamp: <Chen Wang: 2021-11-02 15:29:48>

\subsection{貞定王介\tiny{(BC468-BC441)}}

\subsubsection{生平}

周貞定王(?-前441年),姓姬,名介,東周君主,周元王子,在位28年,諡號貞定王。

周貞定王十六年(前453年),晋国大夫韩康子、赵襄子、魏桓子共同攻灭了晉國最大勢力智伯瑤,是為三家滅智。

清朝学者黄式三在其《周季编略》中认为周王介的谥号貞定王的说法是一个错误。他指出,《史记·周本纪》中周王介被称为定王,与周定王瑜同谥,黄式三认为此处史记是沿袭了《国语》的错误记载。黄式三认为皇甫谧在《帝王世纪》中,按照《世本》和《史记》等称周王介为贞王或定王的记载,臆造了周貞定王的称谓,司马贞《史记索隐》就对皇甫谧的做法提出批评。而后世学者多从皇甫谧,黄式三认为应该根据《国语》韦昭注和司马贞《史记索隐》的说法,而称周王介为周贞王。

\subsubsection{年表}

% \centering
\begin{longtable}{|>{\centering\scriptsize}m{2em}|>{\centering\scriptsize}m{1.3em}|>{\centering}m{8.8em}|}
  % \caption{秦王政}\\
  \toprule
  \SimHei \normalsize 年数 & \SimHei \scriptsize 公元 & \SimHei 大事件 \tabularnewline
  % \midrule
  \endfirsthead
  \toprule
  \SimHei \normalsize 年数 & \SimHei \scriptsize 公元 & \SimHei 大事件 \tabularnewline
  \midrule
  \endhead
  \midrule
  元年 & -468 & \tabularnewline\hline
  二年 & -467 & \tabularnewline\hline
  三年 & -466 & \tabularnewline\hline
  四年 & -465 & \tabularnewline\hline
  五年 & -464 & \tabularnewline\hline
  六年 & -463 & \tabularnewline\hline
  七年 & -462 & \tabularnewline\hline
  八年 & -461 & \tabularnewline\hline
  九年 & -460 & \tabularnewline\hline
  十年 & -459 & \tabularnewline\hline
  十一年 & -458 & \tabularnewline\hline
  十二年 & -457 & \tabularnewline\hline
  十三年 & -456 & \tabularnewline\hline
  十四年 & -455 & \tabularnewline\hline
  十五年 & -454 & \tabularnewline\hline
  十六年 & -453 & \tabularnewline\hline
  十七年 & -452 & \tabularnewline\hline
  十八年 & -451 & \tabularnewline\hline
  十九年 & -450 & \tabularnewline\hline
  二十年 & -449 & \tabularnewline\hline
  二一年 & -448 & \tabularnewline\hline
  二二年 & -447 & \tabularnewline\hline
  二三年 & -446 & \tabularnewline\hline
  二四年 & -445 & \tabularnewline\hline
  二五年 & -444 & \tabularnewline\hline
  二六年 & -443 & \tabularnewline\hline
  二七年 & -442 & \tabularnewline\hline
  二八年 & -441 & \tabularnewline  
  \bottomrule
\end{longtable}

%%% Local Variables:
%%% mode: latex
%%% TeX-engine: xetex
%%% TeX-master: "../../Main"
%%% End:

%% -*- coding: utf-8 -*-
%% Time-stamp: <Chen Wang: 2021-11-02 15:31:46>

\subsection{哀王去疾\tiny{(BC441-BC441)}}

\subsubsection{生平}

周哀王(?-前441年),姓姬,名去疾,東周君主,為周貞定王長子。前441年即位,《史記·周本紀》稱哀王在位僅三個月,為弟叔襲殺害,諡號為哀王。

《搜神记·卷六》記载周哀王八年郑国一妇女共生育四十個子女,其中只有二十個長大成人。哀王九年,晉國有頭豬生了個人。

\subsubsection{年表}

% \centering
\begin{longtable}{|>{\centering\scriptsize}m{2em}|>{\centering\scriptsize}m{1.3em}|>{\centering}m{8.8em}|}
  % \caption{秦王政}\\
  \toprule
  \SimHei \normalsize 年数 & \SimHei \scriptsize 公元 & \SimHei 大事件 \tabularnewline
  % \midrule
  \endfirsthead
  \toprule
  \SimHei \normalsize 年数 & \SimHei \scriptsize 公元 & \SimHei 大事件 \tabularnewline
  \midrule
  \endhead
  \midrule
  元年 & -441 & \tabularnewline  
  \bottomrule
\end{longtable}

%%% Local Variables:
%%% mode: latex
%%% TeX-engine: xetex
%%% TeX-master: "../../Main"
%%% End:

%% -*- coding: utf-8 -*-
%% Time-stamp: <Chen Wang: 2021-11-02 15:32:57>

\subsection{思王叔襲\tiny{(BC441-BC441)}}

\subsubsection{生平}

周思王(?-前441年),姓姬,名叔襲,東周君主,為周貞定王之子,周哀王之弟。

前441年,叔襲殺害周哀王即位,為周思王;在位僅五個月,八月又被弟王子嵬所殺。

\subsubsection{年表}

% \centering
\begin{longtable}{|>{\centering\scriptsize}m{2em}|>{\centering\scriptsize}m{1.3em}|>{\centering}m{8.8em}|}
  % \caption{秦王政}\\
  \toprule
  \SimHei \normalsize 年数 & \SimHei \scriptsize 公元 & \SimHei 大事件 \tabularnewline
  % \midrule
  \endfirsthead
  \toprule
  \SimHei \normalsize 年数 & \SimHei \scriptsize 公元 & \SimHei 大事件 \tabularnewline
  \midrule
  \endhead
  \midrule
  元年 & -441 & \tabularnewline
  \bottomrule
\end{longtable}

%%% Local Variables:
%%% mode: latex
%%% TeX-engine: xetex
%%% TeX-master: "../../Main"
%%% End:

%% -*- coding: utf-8 -*-
%% Time-stamp: <Chen Wang: 2021-11-02 15:34:25>

\subsection{考王嵬\tiny{(BC440-BC426)}}

\subsubsection{生平}

周考王(?-前426年),又稱周考哲王,姓姬,名嵬,為中國東周第十九代國王,在位15年,為周貞定王之子、周哀王與周思王之弟。

前441年,姬嵬殺害周思王自立,是為周考王,以前440年為考王元年。前440年,周考王封其弟姬揭於王畿(位於今河南),是為西周桓公。西周桓公死後,子威公立。惠公繼承威公之位,在周顯王二年(前367年)又封少子姬班於鞏(今河南省鞏义市西南),史稱“東周”。此時周朝王畿內再分出「西周」、「東周」两小國,王畿便更為縮小。

周考王時期處於春秋時期與戰國時期之間,或戰國初期。

前426年周考王死去,其子姬午繼位,是為周威烈王。

\subsubsection{年表}

% \centering
\begin{longtable}{|>{\centering\scriptsize}m{2em}|>{\centering\scriptsize}m{1.3em}|>{\centering}m{8.8em}|}
  % \caption{秦王政}\\
  \toprule
  \SimHei \normalsize 年数 & \SimHei \scriptsize 公元 & \SimHei 大事件 \tabularnewline
  % \midrule
  \endfirsthead
  \toprule
  \SimHei \normalsize 年数 & \SimHei \scriptsize 公元 & \SimHei 大事件 \tabularnewline
  \midrule
  \endhead
  \midrule
  元年 & -440 & \tabularnewline\hline
  二年 & -439 & \tabularnewline\hline
  三年 & -438 & \tabularnewline\hline
  四年 & -437 & \tabularnewline\hline
  五年 & -436 & \tabularnewline\hline
  六年 & -435 & \tabularnewline\hline
  七年 & -434 & \tabularnewline\hline
  八年 & -433 & \tabularnewline\hline
  九年 & -432 & \tabularnewline\hline
  十年 & -431 & \tabularnewline\hline
  十一年 & -430 & \tabularnewline\hline
  十二年 & -429 & \tabularnewline\hline
  十三年 & -428 & \tabularnewline\hline
  十四年 & -427 & \tabularnewline\hline
  十五年 & -426 & \tabularnewline
  \bottomrule
\end{longtable}

%%% Local Variables:
%%% mode: latex
%%% TeX-engine: xetex
%%% TeX-master: "../../Main"
%%% End:


%%% Local Variables:
%%% mode: latex
%%% TeX-engine: xetex
%%% TeX-master: "../../Main"
%%% End:
