%% -*- coding: utf-8 -*-
%% Time-stamp: <Chen Wang: 2021-11-01 18:13:59>

\subsection{平王宜臼{\tiny(BC772-BC720)}}

\subsubsection{生平}

周平王(前781年-前720年),姓姬,名宜臼,東周第一位天子。周幽王之子,母申后為申侯之女,後母褒姒。

姬宜臼本是周幽王太子,後因周幽王寵愛美女褒姒,褒姒生下一子伯服,得寵更甚。褒姒卻與權臣勾結,陰謀奪嫡,而幽王竟為了討美人歡心,便廢掉王后申后與太子宜臼母子,改封褒姒與其子伯服為王后和太子。宜臼逃奔申國。

前771年,周幽王被杀。申侯、缯侯及许文公,在申国擁立宜臼为周天子,即「周平王」。而虢公翰则在携,拥立周幽王之弟姬余臣为周天子,即「携王」(携惠王),形成「二王并立」的局面。

前770年,由於鎬京在戰後已殘破不堪,宜臼為避犬戎,在晉國和鄭國的支持下迁都雒邑,史稱「平王东迁」。周朝至此以後,為史家稱作「東周」。

前750年,携王被晋文侯所弑。前720年,周王宜臼去世,諡號平王。

當時傳聞平王和母親申后因被其父周幽王廢掉後一直懷恨在心,故意聯合諸侯國繒和外族犬戎入宮殺害幽王和褒姒,然後即位為天子,故此周平王有弑父之嫌,亦令周朝的國勢一落千丈。除晉鄭二國之外,其餘諸侯認為宜臼有弒父之嫌疑,故都不再聽從天子,平王唯“晉、鄭是依”,勉強支運國勢,而造成春秋時代的開始。

\subsubsection{年表}

% \centering
\begin{longtable}{|>{\centering\scriptsize}m{2em}|>{\centering\scriptsize}m{1.3em}|>{\centering}m{8.8em}|}
  % \caption{秦王政}\\
  \toprule
  \SimHei \normalsize 年数 & \SimHei \scriptsize 公元 & \SimHei 大事件 \tabularnewline
  % \midrule
  \endfirsthead
  \toprule
  \SimHei \normalsize 年数 & \SimHei \scriptsize 公元 & \SimHei 大事件 \tabularnewline
  \midrule
  \endhead
  \midrule
  % 元年 & -770 & \tabularnewline\hline
  % 二年 & -769 & \tabularnewline\hline
  % 三年 & -768 & \tabularnewline\hline
  % 四年 & -767 & \tabularnewline\hline
  % 五年 & -766 & \tabularnewline\hline
  % 六年 & -765 & \tabularnewline\hline
  % 七年 & -764 & \tabularnewline\hline
  % 八年 & -763 & \tabularnewline\hline
  % 九年 & -762 & \tabularnewline\hline
  % 十年 & -761 & \tabularnewline\hline
  % 十一年 & -760 & \tabularnewline\hline
  % 十二年 & -759 & \tabularnewline\hline
  % 十三年 & -758 & \tabularnewline\hline
  % 十四年 & -757 & \tabularnewline\hline
  % 十五年 & -756 & \tabularnewline\hline
  % 十六年 & -755 & \tabularnewline\hline
  % 十七年 & -754 & \tabularnewline\hline
  % 十八年 & -753 & \tabularnewline\hline
  % 十九年 & -752 & \tabularnewline\hline
  % 二十年 & -751 & \tabularnewline\hline
  % 二一年 & -750 & \tabularnewline\hline
  % 二二年 & -749 & \tabularnewline\hline
  % 二三年 & -748 & \tabularnewline\hline
  % 二四年 & -747 & \tabularnewline\hline
  % 二五年 & -746 & \tabularnewline\hline
  % 二六年 & -745 & \tabularnewline\hline
  % 二七年 & -744 & \tabularnewline\hline
  % 二八年 & -743 & \tabularnewline\hline
  % 二九年 & -742 & \tabularnewline\hline
  % 三十年 & -741 & \tabularnewline\hline
  % 三一年 & -740 & \tabularnewline\hline
  % 三二年 & -739 & \tabularnewline\hline
  % 三三年 & -738 & \tabularnewline\hline
  % 三四年 & -737 & \tabularnewline\hline
  % 三五年 & -736 & \tabularnewline\hline
  % 三六年 & -735 & \tabularnewline\hline
  % 三七年 & -734 & \tabularnewline\hline
  % 三八年 & -733 & \tabularnewline\hline
  % 三九年 & -732 & \tabularnewline\hline
  % 四十年 & -731 & \tabularnewline\hline
  % 四一年 & -730 & \tabularnewline\hline
  % 四二年 & -729 & \tabularnewline\hline
  % 四三年 & -728 & \tabularnewline\hline
  % 四四年 & -727 & \tabularnewline\hline
  % 四五年 & -726 & \tabularnewline\hline
  % 四六年 & -725 & \tabularnewline\hline
  % 四七年 & -724 & \tabularnewline\hline
  % 四八年 & -723 & \tabularnewline\hline
  四九年 & -722 & \tabularnewline\hline
  五十年 & -721 & \tabularnewline\hline
  五一年 & -720 & \tabularnewline
  \bottomrule
\end{longtable}

%%% Local Variables:
%%% mode: latex
%%% TeX-engine: xetex
%%% TeX-master: "../../Main"
%%% End:
