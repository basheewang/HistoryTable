%% -*- coding: utf-8 -*-
%% Time-stamp: <Chen Wang: 2021-11-02 15:21:09>

\subsection{景王貴\tiny{(BC544-BC520)}}

\subsubsection{生平}

周景王(?-前520年),姓姬,名貴,中國東周君主,諡號景,為周靈王之子。周景王在位时,财政困難,連器皿都要向各国索讨。

有一次,景王宴请晋国大臣知文子荀跞,指着鲁国送来的酒壶说:“各国都有器物送给天子,为何晋国没有?”荀跞答不出来,請副使籍谈答覆,籍谈说当初晋国受封时,未赐以礼器,現在晋国忙于对付戎狄,當然無法送出礼物。周景王列数王室赐给晋的土地器物,讽刺其“数典而忘其祖”,这是“数典忘祖”的典故。此時周天子的地位已经一落千丈。

周景王太子寿早死,立王子猛为太子,卻宠爱庶长子王子朝,還一度打算刺殺支持王子猛的單穆公、劉文公。公元前520年四月,周景王病重,嘱咐宾孟要擁立王子朝。王子朝未及立为嗣君,景王却突然病死,由王子猛即位[2]。

周景王在位期間執政為單靖公、劉定公、成簡公、單獻公、單成公、劉獻公、單穆公。

\subsubsection{年表}

% \centering
\begin{longtable}{|>{\centering\scriptsize}m{2em}|>{\centering\scriptsize}m{1.3em}|>{\centering}m{8.8em}|}
  % \caption{秦王政}\\
  \toprule
  \SimHei \normalsize 年数 & \SimHei \scriptsize 公元 & \SimHei 大事件 \tabularnewline
  % \midrule
  \endfirsthead
  \toprule
  \SimHei \normalsize 年数 & \SimHei \scriptsize 公元 & \SimHei 大事件 \tabularnewline
  \midrule
  \endhead
  \midrule
  元年 & -544 & \tabularnewline\hline
  二年 & -543 & \tabularnewline\hline
  三年 & -542 & \tabularnewline\hline
  四年 & -541 & \tabularnewline\hline
  五年 & -540 & \tabularnewline\hline
  六年 & -539 & \tabularnewline\hline
  七年 & -538 & \tabularnewline\hline
  八年 & -537 & \tabularnewline\hline
  九年 & -536 & \tabularnewline\hline
  十年 & -535 & \tabularnewline\hline
  十一年 & -534 & \tabularnewline\hline
  十二年 & -533 & \tabularnewline\hline
  十三年 & -532 & \tabularnewline\hline
  十四年 & -531 & \tabularnewline\hline
  十五年 & -530 & \tabularnewline\hline
  十六年 & -529 & \tabularnewline\hline
  十七年 & -528 & \tabularnewline\hline
  十八年 & -527 & \tabularnewline\hline
  十九年 & -526 & \tabularnewline\hline
  二十年 & -525 & \tabularnewline\hline
  二一年 & -524 & \tabularnewline\hline
  二二年 & -523 & \tabularnewline\hline
  二三年 & -522 & \tabularnewline\hline
  二四年 & -521 & \tabularnewline\hline
  二五年 & -520 & \tabularnewline  
  \bottomrule
\end{longtable}

%%% Local Variables:
%%% mode: latex
%%% TeX-engine: xetex
%%% TeX-master: "../../Main"
%%% End:
