%% -*- coding: utf-8 -*-
%% Time-stamp: <Chen Wang: 2021-11-02 14:59:42>

\subsection{惠王閬\tiny{(BC676-BC652)}}

\subsubsection{生平}

周惠王(?-前652年),姓姬,名閬,又名聞,東周第五代君主,諡惠。他是周僖王之子。

周惠王在前677年繼位後,佔用蒍國的園圃飼養野獸,蒍國的人民不滿,惠王二年,有五大夫作亂,立王子頹為周天子,惠王奔溫(今河南溫縣南),鄭厲公在櫟地(今禹州市)收容惠王,並在惠王四年與虢國攻入周朝,協助平定“子頹之亂”,惠王復辟,鄭國因功獲賜予虎牢(今河南滎陽汜水鎮)以東的地方,虢國也獲賜土地。

周惠王晚年宠爱幼子王子带,欲立為嗣,约郑联楚、晋以成此事,但此时齐桓公稱霸天下,与诸侯会盟力挺太子,周惠王未能如愿。周惠王駕崩后,太子周襄王即位。

《史記·周本紀》稱惠王在位25年,《左傳》稱周惠王在魯僖公七年(前653年)冬天駕崩。

在位期间执政为虢公醜、周公忌父、宰孔。

\subsubsection{年表}

% \centering
\begin{longtable}{|>{\centering\scriptsize}m{2em}|>{\centering\scriptsize}m{1.3em}|>{\centering}m{8.8em}|}
  % \caption{秦王政}\\
  \toprule
  \SimHei \normalsize 年数 & \SimHei \scriptsize 公元 & \SimHei 大事件 \tabularnewline
  % \midrule
  \endfirsthead
  \toprule
  \SimHei \normalsize 年数 & \SimHei \scriptsize 公元 & \SimHei 大事件 \tabularnewline
  \midrule
  \endhead
  \midrule
  元年 & -676 & \tabularnewline\hline
  二年 & -675 & \tabularnewline\hline
  三年 & -674 & \tabularnewline\hline
  四年 & -673 & \tabularnewline\hline
  五年 & -672 & \tabularnewline\hline
  六年 & -671 & \tabularnewline\hline
  七年 & -670 & \tabularnewline\hline
  八年 & -669 & \tabularnewline\hline
  九年 & -668 & \tabularnewline\hline
  十年 & -667 & \tabularnewline\hline
  十一年 & -666 & \tabularnewline\hline
  十二年 & -665 & \tabularnewline\hline
  十三年 & -664 & \tabularnewline\hline
  十四年 & -663 & \tabularnewline\hline
  十五年 & -662 & \tabularnewline\hline
  十六年 & -661 & \tabularnewline\hline
  十七年 & -660 & \tabularnewline\hline
  十八年 & -659 & \tabularnewline\hline
  十九年 & -658 & \tabularnewline\hline
  二十年 & -657 & \tabularnewline\hline
  二一年 & -656 & \tabularnewline\hline
  二二年 & -655 & \tabularnewline\hline
  二三年 & -654 & \tabularnewline\hline
  二四年 & -653 & \tabularnewline\hline
  二五年 & -652 & \tabularnewline  
  \bottomrule
\end{longtable}

%%% Local Variables:
%%% mode: latex
%%% TeX-engine: xetex
%%% TeX-master: "../../Main"
%%% End:
