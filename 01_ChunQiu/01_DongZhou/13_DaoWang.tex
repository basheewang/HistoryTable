%% -*- coding: utf-8 -*-
%% Time-stamp: <Chen Wang: 2021-11-02 15:22:09>

\subsection{悼王猛\tiny{(BC520-BC520)}}

\subsubsection{生平}

周悼王(?-前520年),姓姬,名猛,中國東周君主,諡號悼王,未即位时称王子猛,即位后称王猛。他是周景王的儿子,景王病重時,囑咐大夫賓孟立王子朝。景王死,國人立王子猛為王,是為悼王。悼王後來被王子朝殺死。

周悼王为周景王之子。《左传》记载周景王十八年(前527年),其太子寿和王后去世。其后围绕着周景王的太子人选,朝中形成了对立的两派。一派为支持王子猛的大夫单子单穆公、刘子刘文公。另一派支持王子朝,以王子朝之傅宾起(又称为“宾孟”)为代表。周景王宠爱王子朝,并杀王子猛之傅下门子(见《国语·周语下第三》,取徐元诰《国语集解》之解释)。

周景王二十五年夏四月,景王命令公卿随同自己前往北山田猎。景王计划趁此杀死单子、刘子,然后册立王子朝为太子。但计划尚未实行,周景王突然驾崩,按《左传》的说法是死于心脏病。王子猛即位为王,单子杀宾起,王猛之位初定。即位之时王猛的身份,据《史记·周本纪》是“长子”,按《国语集解》之说是景王朝的太子,对此《左传》无明确记载。《左传·昭公二十六年》记载的王子朝告诸侯文,强调了王后无子的情况下,应该立长子为太子的礼制,而批评单子、刘子为私利而立少子的行为。由此可见,王子朝和王子猛都不是景王的王后之子。而王子朝比王子猛年长。

\subsubsection{年表}

% \centering
\begin{longtable}{|>{\centering\scriptsize}m{2em}|>{\centering\scriptsize}m{1.3em}|>{\centering}m{8.8em}|}
  % \caption{秦王政}\\
  \toprule
  \SimHei \normalsize 年数 & \SimHei \scriptsize 公元 & \SimHei 大事件 \tabularnewline
  % \midrule
  \endfirsthead
  \toprule
  \SimHei \normalsize 年数 & \SimHei \scriptsize 公元 & \SimHei 大事件 \tabularnewline
  \midrule
  \endhead
  \midrule
  元年 & -520 & \tabularnewline  
  \bottomrule
\end{longtable}

%%% Local Variables:
%%% mode: latex
%%% TeX-engine: xetex
%%% TeX-master: "../../Main"
%%% End:
