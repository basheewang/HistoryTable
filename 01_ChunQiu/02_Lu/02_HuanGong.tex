%% -*- coding: utf-8 -*-
%% Time-stamp: <Chen Wang: 2021-11-01 17:48:29>

\subsection{桓公允{\tiny(BC711-BC694)}}

\subsubsection{生平}

魯桓公(约前731年10月7日-前694年4月14日),姬姓,名允,一名軌,魯惠公之子,魯隱公之弟。魯國第十五代國君,在位十八年。

公子允,是惠公正室夫人仲子所生,所以被立為世子,又因惠公去世時尚且年幼,由庶兄息姑即位,是為魯隱公。

大臣羽父勸隱公殺死公子允,隱公善良,不願意,羽父怕以後公子允得知之後會報復自己,所以反過來聯合公子允殺了隱公,隱公被殺後,公子允前711年即位,是為魯桓公。魯桓公前694年死于齊国,在位18年。

據《左傳》記載,魯桓公带著夫人文姜訪問齊国,齊襄公与文姜通奸(文姜是襄公之妹)。之後魯桓公加以指責。同年夏四月,齊襄公派公子彭生駕駛魯桓公的馬車,魯桓公被搚幹而死。時人猜測可能是齊襄公命彭生在車上殺了魯桓公,以便与文姜通姦。在魯国的壓力下,齊襄公殺了彭生。

在位期間的卿為羽父、柔。

魯桓公嫡長子為魯莊公,繼承國君之位。另有三子,庶長子孟慶父、次子叔牙、嫡次子季友,都被封為卿大夫,後代皆在魯國掌權,三人因皆出自魯桓公,被稱為三桓。

\subsubsection{年表}

% \centering
\begin{longtable}{|>{\centering\scriptsize}m{2em}|>{\centering\scriptsize}m{1.3em}|>{\centering}m{8.8em}|}
  % \caption{秦王政}\\
  \toprule
  \SimHei \normalsize 年数 & \SimHei \scriptsize 公元 & \SimHei 大事件 \tabularnewline
  % \midrule
  \endfirsthead
  \toprule
  \SimHei \normalsize 年数 & \SimHei \scriptsize 公元 & \SimHei 大事件 \tabularnewline
  \midrule
  \endhead
  \midrule
  元年 & -711 & \tabularnewline\hline
  二年 & -710 & \tabularnewline\hline
  三年 & -709 & \tabularnewline\hline
  四年 & -708 & \tabularnewline\hline
  五年 & -707 & \tabularnewline\hline
  六年 & -706 & \tabularnewline\hline
  七年 & -705 & \tabularnewline\hline
  八年 & -704 & \tabularnewline\hline
  九年 & -703 & \tabularnewline\hline
  十年 & -702 & \tabularnewline\hline
  十一年 & -701 & \tabularnewline\hline
  十二年 & -700 & \tabularnewline\hline
  十三年 & -699 & \tabularnewline\hline
  十四年 & -698 & \tabularnewline\hline
  十五年 & -697 & \tabularnewline\hline
  十六年 & -696 & \tabularnewline\hline
  十七年 & -695 & \tabularnewline\hline
  十八年 & -694 & \tabularnewline
  \bottomrule
\end{longtable}

%%% Local Variables:
%%% mode: latex
%%% TeX-engine: xetex
%%% TeX-master: "../../Main"
%%% End:
