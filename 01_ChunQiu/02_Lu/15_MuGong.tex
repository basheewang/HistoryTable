%% -*- coding: utf-8 -*-
%% Time-stamp: <Chen Wang: 2021-11-01 18:06:26>

\subsection{穆公顯{\tiny(BC407-BC376)}}

\subsubsection{生平}

魯穆公(?-前377年),即姬顯,為戰國諸侯國魯國君主之一,是魯國第二十九任君主。他為魯元公兒子,承襲魯元公擔任該國君主,在位33年。在位期間實行改革,擺脫了哀、悼、元三代三桓大夫專政的問題,確立了魯公室的權威,並與鄰國齊國展開多次戰爭。

元年(前415年),魯穆公實行改革,任命博士公儀休為魯相,從三桓收回政權,國政開始奉法循理。季孫氏據其封邑費、卞、東野成為獨立小國,而孟孫氏和叔孫氏先後亡於齊國。

四年(前412年),齊國攻魯的莒、安陽(今山東陽穀縣東北),命吳起為將,打敗了齊軍。

五年(前411年),吳起奔魏。齊伐魯取都(一作「取一城」)。

八年(前408年),齊取魯郕。

廿二年(前394年),齊伐魯,取郕。韓救魯。

廿六年(前390年),魯國打敗齊國于平陸。

鲁穆公向子思询问道:“我听说庞{米间}氏的孩子不孝顺,他的行为怎么样?”于思回答说:“君子尊重贤人来祟尚道德,提倡好事来给民众作出表率。至于错误行为,那是小人才会记住的,我不知道。”子思出去了。子服厉伯进见,穆公问他庞{米间}氏孩子的劣行,子服厉伯回答说:“这孩子的过错有三条。”都是穆公不曾听说过的。从此以后,穆公看重子思而看轻子服厉伯。 有人说:鲁国的君权,三代都被季孙氏控制着,不是应该的吗?明君发现好事就给予赏赐,发觉坏事就给予惩罚,两者目的是一致的。所以把好事报告给君主的人,也就是和君主同样喜欢好事的;把坏事报告给君主的人,也就是和君主同样厌恶坏事的:都是应该奖赏和赞誉的。不把坏事报告给君主,是和君主离心离德而和坏人紧密勾结的行为,这是应该贬斥相处罚的。现在于思不把庞子的过错告知穆公,穆公却尊重他;厉伯把庞子的过错告知穆公,穆公却鄙视他。人的心情都是喜欢受尊重而厌恶被鄙视的,所以季氏已酿成祸乱了,却没人向上报告,这就是鲁君被挟持的原因。况且这种亡国的风气,是陬、鲁地方的人自我欣赏的东西,而穆公偏偏予以推崇,不是弄反了吗?

魯穆公問子思道:「什么樣的才能叫做忠臣呢?」子思說:「總是指出君主做的壞事的人,就可以稱為忠臣了。」魯穆公(聞言)不高興,子思作揖後就退下了。成孫戈覲見,魯穆公說:「剛才我問子思忠臣的事,子思說:『總是指出君主做的壞事的人,就可以稱為忠臣了。』寡人對此很困惑,不能有所得。」成孫戈說:「咦,這話說得好呀!為了君王的緣故而失去生命的人,這種人是有的。總是指出君主做的壞事的人卻從未有過。為了君王的緣故而失去生命的人,不過是盡忠於爵祿。總是指出君主做的壞事的人,是遠離爵祿的。為了義理而遠離爵祿,如果不是子思,我是不會聽說這種事的。」

\subsubsection{年表}

% \centering
\begin{longtable}{|>{\centering\scriptsize}m{2em}|>{\centering\scriptsize}m{1.3em}|>{\centering}m{8.8em}|}
  % \caption{秦王政}\\
  \toprule
  \SimHei \normalsize 年数 & \SimHei \scriptsize 公元 & \SimHei 大事件 \tabularnewline
  % \midrule
  \endfirsthead
  \toprule
  \SimHei \normalsize 年数 & \SimHei \scriptsize 公元 & \SimHei 大事件 \tabularnewline
  \midrule
  \endhead
  \midrule
  元年 & -407 & \tabularnewline\hline
  二年 & -406 & \tabularnewline\hline
  三年 & -405 & \tabularnewline\hline
  四年 & -404 & \tabularnewline\hline
  五年 & -403 & \tabularnewline
  \bottomrule
\end{longtable}

%%% Local Variables:
%%% mode: latex
%%% TeX-engine: xetex
%%% TeX-master: "../../Main"
%%% End:
