%% -*- coding: utf-8 -*-
%% Time-stamp: <Chen Wang: 2021-11-01 17:57:14>

\subsection{成公黑肱{\tiny(BC590-BC573)}}

\subsubsection{生平}

魯成公(?-前573年),姬姓,名黑肱,為東周春秋時期諸侯國魯國的一位君主,是魯國第二十一任君主,承襲父親魯宣公擔任該國君主,在位18年。

在位期間執政為季孫行父、仲孫蔑、叔孫僑如。

魯成公元年(前590年),季孫行父在魯國實行一種「作丘(地區單位)甲」的地區編制。

魯成公二年(前589年)春天,齊國要攻打魯、衛兩國。魯、衛兩國大夫請求晉國出兵,晉國以郤克為主將率兵救討伐齊國,以救援魯、衛二國。同一年,齊頃公親率齊軍南下攻打魯國龍邑(今山東泰安東南),寵臣盧蒲就癸被殺,頃公怒而攻至巢丘(今山東泰安境內)。季孫行父率魯軍幫晉、衛、曹等國,去攻打齊國的鞍(今山東濟南市)。齊頃公在鞍之戰大敗,齊頃公被晉軍追逼,「差點被俘,幸得大夫逢丑父相救,二人互換衣服,佯命齊頃公到山腳華泉取水,得以逃走。同年十一月,魯國的魯成公同蔡景侯、許靈公、秦國右大夫說、宋國華元、陳國公孫寧、衛國孫良夫、鄭國子良、齊國大夫、曹、邾、薛、鄫等多國代表參與由楚國公子嬰齊在蜀(今山東省泰安市東南)所主辦的會盟。

魯成公七年(前584年),吳國攻打鄰近魯國的郯國,郯國被納入吳國的領土的事,因此季孫行父向魯國國君發出「中國不振旅,蠻夷(指吳國)來伐」的警告。

魯成公十六年(前575年),魯成公的夫人定姒產下魯成公的兒子姬午。

魯成公十八年(前573年),魯成公薨,姬午即位(即後來的魯襄公)。

\subsubsection{年表}

% \centering
\begin{longtable}{|>{\centering\scriptsize}m{2em}|>{\centering\scriptsize}m{1.3em}|>{\centering}m{8.8em}|}
  % \caption{秦王政}\\
  \toprule
  \SimHei \normalsize 年数 & \SimHei \scriptsize 公元 & \SimHei 大事件 \tabularnewline
  % \midrule
  \endfirsthead
  \toprule
  \SimHei \normalsize 年数 & \SimHei \scriptsize 公元 & \SimHei 大事件 \tabularnewline
  \midrule
  \endhead
  \midrule
  元年 & -590 & \tabularnewline\hline
  二年 & -589 & \tabularnewline\hline
  三年 & -588 & \tabularnewline\hline
  四年 & -587 & \tabularnewline\hline
  五年 & -586 & \tabularnewline\hline
  六年 & -585 & \tabularnewline\hline
  七年 & -584 & \tabularnewline\hline
  八年 & -583 & \tabularnewline\hline
  九年 & -582 & \tabularnewline\hline
  十年 & -581 & \tabularnewline\hline
  十一年 & -580 & \tabularnewline\hline
  十二年 & -579 & \tabularnewline\hline
  十三年 & -578 & \tabularnewline\hline
  十四年 & -577 & \tabularnewline\hline
  十五年 & -576 & \tabularnewline\hline
  十六年 & -575 & \tabularnewline\hline
  十七年 & -574 & \tabularnewline\hline
  十八年 & -573 & \tabularnewline
  \bottomrule
\end{longtable}

%%% Local Variables:
%%% mode: latex
%%% TeX-engine: xetex
%%% TeX-master: "../../Main"
%%% End:
