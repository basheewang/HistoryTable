%% -*- coding: utf-8 -*-
%% Time-stamp: <Chen Wang: 2021-11-01 17:54:51>

\subsection{僖公申{\tiny(BC659-BC627)}}

\subsubsection{生平}

魯僖公(?-前627年),《史記》作魯釐公,姬姓,名申,為春秋諸侯國魯國君主之一,是魯國第十八任君主。他為魯莊公庶子,承襲魯閔公擔任該國君主,在位33年。在位期間執政為季友、臧文仲、公孙兹、孟穆伯、公子买、东门襄仲。

魯莊公死後,季友立公子般繼位,為魯君子般。後孟慶父勾结私通的鲁莊公夫人哀姜,殺死了鲁君子般,魯莊公兒子啟方繼位,是為魯閔公,季友逃走。前660年,慶父與哀姜謀殺閔公,想自立為君,魯人不服,要殺慶父,慶父逃到莒國。季友回國,立公子申為魯僖公,並迫使慶父自縊。

《春秋》不书魯僖公即位。《公羊传》、《谷梁传》解释为继承被弑君主不书即位;《左传》解释为魯僖公曾经犯下出奔的大恶,因此不书即位。

魯僖公是孔子在《春秋》中出現最多次的君主,很多《春秋》中的大小記事、諸侯國之間的國際情勢概要都是以魯僖公年間發生的。

\subsubsection{年表}

% \centering
\begin{longtable}{|>{\centering\scriptsize}m{2em}|>{\centering\scriptsize}m{1.3em}|>{\centering}m{8.8em}|}
  % \caption{秦王政}\\
  \toprule
  \SimHei \normalsize 年数 & \SimHei \scriptsize 公元 & \SimHei 大事件 \tabularnewline
  % \midrule
  \endfirsthead
  \toprule
  \SimHei \normalsize 年数 & \SimHei \scriptsize 公元 & \SimHei 大事件 \tabularnewline
  \midrule
  \endhead
  \midrule
  元年 & -659 & \tabularnewline\hline
  二年 & -658 & \tabularnewline\hline
  三年 & -657 & \tabularnewline\hline
  四年 & -656 & \tabularnewline\hline
  五年 & -655 & \tabularnewline\hline
  六年 & -654 & \tabularnewline\hline
  七年 & -653 & \tabularnewline\hline
  八年 & -652 & \tabularnewline\hline
  九年 & -651 & \tabularnewline\hline
  十年 & -650 & \tabularnewline\hline
  十一年 & -649 & \tabularnewline\hline
  十二年 & -648 & \tabularnewline\hline
  十三年 & -647 & \tabularnewline\hline
  十四年 & -646 & \tabularnewline\hline
  十五年 & -645 & \tabularnewline\hline
  十六年 & -644 & \tabularnewline\hline
  十七年 & -643 & \tabularnewline\hline
  十八年 & -642 & \tabularnewline\hline
  十九年 & -641 & \tabularnewline\hline
  二十年 & -640 & \tabularnewline\hline
  二一年 & -639 & \tabularnewline\hline
  二二年 & -638 & \tabularnewline\hline
  二三年 & -637 & \tabularnewline\hline
  二四年 & -636 & \tabularnewline\hline
  二五年 & -635 & \tabularnewline\hline
  二六年 & -634 & \tabularnewline\hline
  二七年 & -633 & \tabularnewline\hline
  二八年 & -632 & \tabularnewline\hline
  二九年 & -631 & \tabularnewline\hline
  三十年 & -630 & \tabularnewline\hline
  三一年 & -629 & \tabularnewline\hline
  三二年 & -628 & \tabularnewline\hline
  三三年 & -627 & \tabularnewline
  \bottomrule
\end{longtable}

%%% Local Variables:
%%% mode: latex
%%% TeX-engine: xetex
%%% TeX-master: "../../Main"
%%% End:
