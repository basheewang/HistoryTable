%% -*- coding: utf-8 -*-
%% Time-stamp: <Chen Wang: 2021-11-01 17:59:15>

\subsection{闵公啟{\tiny(BC661-BC660)}}

\subsubsection{子斑生平}

子般(?-前662年),姬姓,名般,一作斑,,称子表示此时是其父鲁庄公死的当年,魯莊公之子。魯莊公的夫人哀姜是齊國人,無子。莊公臨死前欲立庶子般為嗣君,莊公弟叔牙建議立庄公庶长兄公子慶父,另一位弟弟季友則支持立子般,季友於是借莊公之命赐死叔牙。

三十二年八月,莊公病逝,季友立子般為君,十月慶父殺子般,立莊公另一庶子啟為魯君,是為魯閔公。季友逃亡陳國。

\subsubsection{閔公啟生平}

魯閔公(前669年?-前660年),即姬啟,一名啟方,為春秋諸侯國魯國君主之一,是魯國第十七任君主。他為魯莊公、叔姜的兒子。近人考証謂於周惠王八年(前669年)出生,至周惠王十七年(前660年)去世,年約十歲。(楊伯峻《春秋左傳注》,頁254)

莊公死前,弟弟叔牙建議立莊公庶長兄慶父,另一位弟弟季友則支持立子般,季友於是借莊公之命賜死叔牙,莊公病逝,季友立子般為君,十月慶父殺子般,立莊公另一庶子啟為魯君,即魯閔公,魯閔公亦是齊桓公的外甥,對齊桓公很尊敬,因此齊魯無大事,直到兩年後公子慶父以毒餅殺死魯閔公,齊桓公才派兵迎立魯閔公之弟魯釐公。

在位期間的卿為公子慶父、季友。

\subsubsection{年表}

% \centering
\begin{longtable}{|>{\centering\scriptsize}m{2em}|>{\centering\scriptsize}m{1.3em}|>{\centering}m{8.8em}|}
  % \caption{秦王政}\\
  \toprule
  \SimHei \normalsize 年数 & \SimHei \scriptsize 公元 & \SimHei 大事件 \tabularnewline
  % \midrule
  \endfirsthead
  \toprule
  \SimHei \normalsize 年数 & \SimHei \scriptsize 公元 & \SimHei 大事件 \tabularnewline
  \midrule
  \endhead
  \midrule
  元年 & -661 & \tabularnewline\hline
  二年 & -660 & \tabularnewline
  \bottomrule
\end{longtable}

%%% Local Variables:
%%% mode: latex
%%% TeX-engine: xetex
%%% TeX-master: "../../Main"
%%% End:
