%% -*- coding: utf-8 -*-
%% Time-stamp: <Chen Wang: 2021-11-01 17:51:13>

\subsection{庄公同{\tiny(BC693-BC662)}}

\subsubsection{生平}

魯莊公(前706年10月7日-前662年8月11日),姬姓,名同,為中國春秋時期魯國第十六任國君。他是魯桓公之子,在桓公死後繼位,在位32年(前693年—前662年)。

魯桓公三年,娶齊襄公之妹姜氏。桓公六年,公子同出生,後立爲太子。

莊公八年,齊公子糾與管仲逃到魯國。次年齊桓公發兵擊敗魯國,魯國殺子糾(《左傳》稱「齊人取子糾殺之」)。齊向魯索回管仲,魯人施伯認為齊欲重用管仲,將會對魯不利,勸莊公殺管仲,莊公不聽,把管仲歸還齊。

齊桓公回國繼位幾年後,對於魯國積恨難消,管仲、鮑叔牙等勸不止,再次派兵攻擊魯國,爆發了「長勺之戰」。這次魯國有準備,把來犯的齊軍打退,兩國和解,直到自己的外甥魯閔公被自己的姊妹所謀殺後,才再次對魯大動干戈。

十三年,魯莊公會齊桓公於柯,曹沫劫持齊桓公,逼他退還齊侵佔魯的土地,桓公答應後才釋放他。桓公欲背約,管仲諫之,終於歸還齊侵佔魯的土地。

莊公的夫人哀姜是齊國人,無子。莊公臨死前欲立庶子斑(或作般)為嗣君,莊公弟叔牙建議立長弟慶父,另一弟季友則支持立斑,季友以莊公之名逼叔牙飲毒酒自殺死。

三十二年八月,莊公病逝,季友立子斑為君,十月慶父殺子斑,立莊公另一庶子啟為魯君。

在位期間的卿為公子慶父、季友、臧文仲、叔牙、公子溺、公子结。

據《春秋公羊傳·莊公元年》,魯桓公曾在酒醉時懷疑姬同非自己的兒子,乃是齊襄公與其妹文姜通姦所生。魯桓公薨後,文姜與齊襄公又有多次私會。

\subsubsection{年表}

% \centering
\begin{longtable}{|>{\centering\scriptsize}m{2em}|>{\centering\scriptsize}m{1.3em}|>{\centering}m{8.8em}|}
  % \caption{秦王政}\\
  \toprule
  \SimHei \normalsize 年数 & \SimHei \scriptsize 公元 & \SimHei 大事件 \tabularnewline
  % \midrule
  \endfirsthead
  \toprule
  \SimHei \normalsize 年数 & \SimHei \scriptsize 公元 & \SimHei 大事件 \tabularnewline
  \midrule
  \endhead
  \midrule
  元年 & -693 & \tabularnewline\hline
  二年 & -692 & \tabularnewline\hline
  三年 & -691 & \tabularnewline\hline
  四年 & -690 & \tabularnewline\hline
  五年 & -689 & \tabularnewline\hline
  六年 & -688 & \tabularnewline\hline
  七年 & -687 & \tabularnewline\hline
  八年 & -686 & \tabularnewline\hline
  九年 & -685 & \tabularnewline\hline
  十年 & -684 & \tabularnewline\hline
  十一年 & -683 & \tabularnewline\hline
  十二年 & -682 & \tabularnewline\hline
  十三年 & -681 & \tabularnewline\hline
  十四年 & -680 & \tabularnewline\hline
  十五年 & -679 & \tabularnewline\hline
  十六年 & -678 & \tabularnewline\hline
  十七年 & -677 & \tabularnewline\hline
  十八年 & -676 & \tabularnewline\hline
  十九年 & -675 & \tabularnewline\hline
  二十年 & -674 & \tabularnewline\hline
  二一年 & -673 & \tabularnewline\hline
  二二年 & -672 & \tabularnewline\hline
  二三年 & -671 & \tabularnewline\hline
  二四年 & -670 & \tabularnewline\hline
  二五年 & -669 & \tabularnewline\hline
  二六年 & -668 & \tabularnewline\hline
  二七年 & -667 & \tabularnewline\hline
  二八年 & -666 & \tabularnewline\hline
  二九年 & -665 & \tabularnewline\hline
  三十年 & -664 & \tabularnewline\hline
  三一年 & -663 & \tabularnewline\hline
  三二年 & -662 & \tabularnewline
  \bottomrule
\end{longtable}

%%% Local Variables:
%%% mode: latex
%%% TeX-engine: xetex
%%% TeX-master: "../../Main"
%%% End:
