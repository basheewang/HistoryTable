%% -*- coding: utf-8 -*-
%% Time-stamp: <Chen Wang: 2019-12-27 10:00:40>

\section{鲁国}

\subsection{简介}

鲁国,是周朝的一個姬姓諸侯國,為周成王的四叔周公旦及其子伯禽的封国。鲁国先後傳二十五世,三十六位君主,歷時八百餘年。首都在曲阜,疆域在泰山以南,略有今山东省西南部,國力鼎盛時期勢力遍及河南、江蘇及安徽三省。另外,魯国亦是孔子的出生地。

立國:西周初年周公輔佐天子周成王,周公东征打败了伙同武庚叛乱的殷商旧属国,之後周公长子伯禽代替周公前往受封的奄国故土建立鲁国。

伯禽到达封国之后,把曲阜作为自己封国的都城,然后依照周国的制度、习俗来进行治理。因为要去除当地的旧习俗,伯禽前前后后用了三年时间才完成了初步的稳定,然后返回成周报告政绩。而鲁的邻国齐却只用了五个月就返回成周报告结果了,这是因为齐国采取了和鲁国完全相反的政策。齐国的封君简化了周的制度,并依照当地风俗来治理封国,于是很快地稳定下来了。在管叔、蔡叔联合武庚作乱时,东方的淮夷、徐戎等也兴兵作乱,前来攻打鲁国。伯禽率领鲁国的军队前往抵抗,奋战两年最终在周、齐的帮助下平定了鲁国。伯禽在位四十余年,坚持使用周礼治理鲁国,又加上成王赋予了鲁国“郊祭文王”、“奏天子礼乐”的资格,鲁国因此在立国之初就奠定了丰厚的周文化基础。而在后来礼坏乐崩的时代,鲁国则成为了典型周礼的保存者和实施者,世人称“周礼尽在鲁矣”。

周室强藩:周王朝历来有厚同姓、薄异姓的国策,而周成王赋予鲁国“郊祭文王”、“奏天子礼乐”的资格,不仅仅是对周公旦功劳的一种追念,更是希望作为宗邦的鲁国能够“大启尔宇,为周室辅”。这是鲁国在政治上的优势。伐灭管蔡之乱,平定徐戎之叛,鲁国得到“殷民六族”。而本来是王族的殷商之民,拥有较高的文化水平,同时也善于发展经济;而鲁国地处东方海滨,盐铁等重要资源丰富。鲁国历经鲁公伯禽、考公酋(系本作“就”,邹诞本作“遒”)、炀公熙(一作怡,考公弟)、幽公宰(系本名圉)、魏公晞(幽公弟)、厉公擢(系本作“翟”)、献公具(厉公弟)、真公濞(本亦多作“慎公”),一直都是周室强藩,震慑并管理东方,充分发挥了宗邦的作用。此时的鲁国“奄有龟蒙,遂荒大东。至于海邦,淮夷来同”,其国力之强,使得国人和夷狄之民“莫我敢承”、“莫不率从”。这种情形一直延续到春秋,彼时曹、滕、薛、纪、杞、彀、邓、邾、牟、葛诸侯仍旧时常朝觐鲁国。

废长立幼:鲁真公薨,其弟敖立,是为鲁武公。武公有长子括、少子戏。武公九年,武公带着两个儿子,西去朝拜周宣王。宣王很喜欢戏,有意立戏为鲁国的太子。長子括為魯武公的嫡子(與正室夫人所生)、少子戲為魯武公的庶子(與側室夫人所生),依照當時的宗法,只有在正室夫人無子或所生之子死亡時才能立庶子為太子,宣王的做法嚴重犯了宗法的大忌。宣王的卿大夫樊仲山父说:「这个废长立幼,不合规矩。若您執意違背规矩的话,日后鲁国一定会违背您的旨意。」周宣王不顾重臣意见,下命令立戏为鲁国太子。鲁武公郁郁不乐,回到鲁国后就去世了。太子戏繼位,是为鲁懿公。懿公之後被其兄括的儿子伯御殺掉。伯御安安稳稳地做了十一年鲁国国君,最后被周宣王发兵伐魯,把伯御给诛杀了,再立魯懿公的弟弟称,是为鲁孝公。伐魯令周朝天子的威信受損,以後周诸侯国弑其君的事情时有发生。

隐公居摄:鲁孝公薨,子弗湟立,是为鲁惠公。鲁惠公的元配没有生子就死了,妾室声子倒是生了个儿子,名叫做息(一作息姑)。后来,惠公听说宋国有个女子生来手掌就有“鲁夫人”的纹状,于是就把她娶回鲁国,是为仲子。仲子为惠公生了个儿子,名叫做允(一作轨)。惠公没有立太子就死掉了。年长的公子息颇得鲁人的拥戴,于是他摄行君位。但是又担心其他人不服,于是立公子允为惠公太子,说是等他长大后就把政权返给他。历史上把公子息称作“隐公”,谥法:“不尸其位曰隐。”。

隐公时期,卿大夫羽父位高权重,逐渐掌握实权。羽父野心大,渴望与国君平起平坐,何况隐公甚至还不是名义上的国君。羽父要隐公立他为太宰,太宰就是周天子的王室正卿,就地位而言,跟诸侯平起平坐。隐公不答应。羽父于是到太子允前,谗言隐公想要霸占权位。太子允于是授命羽父派人弑杀了隐公。太子允即位,是为鲁桓公。魯隐公及桓公時(前722年-前662年),魯國多次戰勝齊、宋等国,且不断侵襲杞國、莒國等小国。鲁桓公初期,羽父还挺有权势,但是到了后来就不见经传,或许是桓公疏远了他也未可知。

三桓時期:春秋中期之後,魯國政權轉入貴族大臣手中。魯莊公的三个弟弟季友、叔牙及慶父的子孫長期掌握魯國實權,称為季孫氏、叔孫氏、孟孫氏三家,由於三家都是魯桓公之後,被称為「三桓」,魯國從此「政在大夫」。鲁桓公有庶长子庆父、太子同、公子牙、公子友。庆父、叔牙、季友的后代分别是孟氏、叔孙氏及季氏,合称三桓。三桓为孟氏、叔孙氏及季氏,而非孟孙氏、叔孙氏及季孙氏。以往有众多学者认为孟孙、叔孙、季孙皆为氏称,实误。“孙”为尊称,对于孟氏和季氏,“孟孙某”、“季孙某”仅限于宗主的称谓,宗族一般成员只能称“孟某”、“季某”。所以,“孟孙”、“季孙”并不是氏称。考之《左传》,只有“孟氏”、“季氏”的字样,而无“孟孙氏”、“季孙氏”的字样。叔孙氏的情况比较特殊,起先为叔氏,后来公子牙(字子叔)之后立叔氏,原来的叔氏改称叔孙氏。

桓公薨,太子同立,是为鲁庄公。庄公夫人哀姜,哀姜娣叔姜为庄公生子开。庄公晚年,筑高台,看到大夫党氏的女儿孟任,很是欢喜,就跟着她走。最后,庄公许诺说立孟任为夫人,如果她给自己生了儿子,就立为太子。孟任生子般(一作“斑”)。庄公想立般为太子,又担心其他臣子有意见。到了庄公三十二年,庄公病笃,又想到立太子的事情,就询问自己的兄弟叔牙、季友。叔牙说庆父有才能,季友则说就算死也要立公子般。庄公让季友派人赐鸩酒给叔牙。叔牙饮鸩而死,立其后为叔氏,后改称叔孙氏。

鲁庄公立般为太子,而季友辅佐。叔牙死后不久,庄公薨。于是季友立太子般为国君,为庄公治丧,因此尚未正式即位。而庆父发难,派人弑杀了在党氏居住的子般。季友惊慌之间,逃往陈国。庆父与庄公夫人哀姜一向都有私通,因此发难之后,他立哀姜陪嫁的叔姜之子,公子开为国君,是为鲁闵公(一作湣公)。庆父立闵公之后,跟哀姜私通,後來想把闵公也杀了,自己当国君。齐国仲孙湫就预言“不去庆父,鲁难未已”([子说庆父不死,鲁难未已。比喻不清除制造内乱的罪魁祸首,国家就得不到安宁。亦指了结或停止危害的关键事物。)。鲁闵公二年,庆父派大夫卜齮袭杀闵公于武闱。季友听闻,由陈國走到邾國,接庄公妾成风之子申,请鲁人以其为国君。庆父忧惧,出逃到莒國。于是,季友送公子申入鲁,并重金贿赂莒人,抓庆父回国。庆父请求让他出逃,季友不肯。于是庆父自杀。立其后为孟氏。关于孟氏,《春秋》又作仲氏。因为当初庆父虽为长兄,但为了表示君臣之别,于是自称仲,史称共仲。实际上,当时的人都以其年长而叫他的后代为孟氏。

季友立公子申,是为鲁僖公(史记作“釐公”)。僖公元年,季友帅师败“莒师于郦,获莒拏”,“公赐季友汶阳之田及费”,季友为鲁国相。季友相僖公,执政多年,把鲁国治理得井井有条。鲁人作《诗·鲁颂》称赞。僖公十六年,季友卒,谥成,史称“成季”,其后立为季氏。

公卿争权:鲁僖公、文公、宣公、成公、襄公、昭公、定公、哀公及悼公九位鲁侯在位期间,作为卿家的三桓与公室争权夺利,尤其是以季氏的执政与公室的反击最为激烈。鲁穆公时期实行改革,任命博士公仪休为鲁相,才遂渐从三桓手中收回政权。成季死后,庄公的公子遂(即襄仲)及其儿子公孙归父相继掌权,是为东门氏执政时期,而孟氏一度被东门氏赶出鲁国。然而,成季的孙子季孙行父(即季文子)利用三桓的势力,魯宣公十五年(前594年)實行「初税畝」,开初税亩,使得私田兴起,而“隐民”剧增,获得鲁国平民阶层的人心。公子遂杀嫡立庶,以公子俀为国君,是为鲁宣公。

宣公发现三桓日益强盛,同时有民不知君、只知三桓的说法甚嚣尘上,于是他欲去三桓,以张大公室。他与执政的公孙归父商量,是不是起兵灭了三桓,但是国人明显倾心于三桓,使用国内兵马或许不妥。于是,公孙归父前往晋国借兵。可惜公孙归父还没成功搬来晋国军队,宣公就死了,而季文子趁机发难,备述襄仲当政时的弊端,斥责他“南通于楚,既不能固,又不能坚事齐、晋”,使鲁国没有强援。鲁国司寇表示愿意随季文子除乱。公孙归父听到这样的消息,连忙逃到齐国躲起来。季文子开始执政。从此开启了季氏祖孙几代人的执政专权之路。

季文子、季武子、季平子辅佐鲁文公、宣公、成公、襄公、昭公及定公六位鲁侯,位列三卿之首,独专国政。魯成公元年(前590年)行「作丘甲」。季武子时期,通过一系列的政策从不同角度削弱公室的勢力。襄公十一年, 增设三军。季武子、叔孙穆叔、孟献子分三军,一卿主一军之征赋,由是三桓强于公室。当年,周武王封周公旦于鲁,按周礼“天子六军,诸侯大国三军”,鲁有三军。自文公以来,鲁国弱而从霸主之令,若军多则贡多,遂自减中军,只剩上下二军,属于公室,“有事,三卿更帅以征伐”不得专其民。季武子欲专其民,遂增设中军,三桓分三军之民。襄公十二年,三桓“十二分其国民,三家得七,公得五,国民不尽属公,公室已是卑矣”。

昭公外逃:昭公五年,季武子罢中军。四分公室,季孙称左师,孟氏称右师,叔孙氏则自以叔孙为军名,“三家自取其税,减已税以贡于公,国民不复属于公,公室弥益卑矣”。公室奋起反击,昭公二十五年,在郈昭伯、公若等人的劝说下,鲁昭公发兵伐季氏。而孟氏、叔孙氏认为唇亡齿寒,三桓是一荣俱荣、一损俱损,于是发兵救援。结果昭公外逃,而季平子专权,摄行君位将近十年。魯昭公被三家驅逐,客死異鄉。其后不久,三桓属下的家臣陽虎等人控制国政,一度形成“陪臣執国命”的局面。魯定公時(前509年~前495年),陽虎失敗出奔 ,三桓重新掌權 ,後魯哀公(前494年~前468年在位)圖謀恢復君權,同三家大臣衝突加劇,終致流亡越国。

隳三都:季平子的僭越行为,导致其家臣奋起模仿,其中影响最大的莫过于阳虎。定公五年,季平子、叔孙成子相继去世,阳虎发难,囚禁季桓子,逐仲梁怀,随后执掌鲁国权位长达三年。虽然阳虎被三桓赶出了鲁国,但是三桓的影响日渐削弱、公卿之别君臣之礼日渐败坏也成了趋势。这个时候,在位的鲁定公决心削弱三桓,而这个时候三桓内部并不稳定,因为季氏的专权,导致其他两家的不满。定公十年,齐鲁会盟,作为司仪的孔子不仅言谈之间退发难的莱夷之人,更以口舌之利,使得齐国归还汶阳之田。于是,定公以此为契机,重用孔子, 而孔子为了恢复公卿之别、君臣之分,决定以隳三都的方式,逐步消解三桓的强盛势力。季桓子出于防止家臣犯上的考虑,同意隳三都,并派仲由等臣子率兵毁掉自己的费城。然而三桓之中,孟氏反对,他坚持不毁掉自己的成城,结果定公发兵讨伐,却无法攻下。而定公在季氏的唆使下观齐女乐,败坏礼数,更寒了孔子的心。结果,三桓把公室的坚定拥护者孔子赶出了鲁国。

费国独立:魯哀公即位,哀公十二年(前483年)行「用田賦」。哀公十七年(西元前478年),孔子的弟子於曲阜孔子故里建孔庙。根据《史记》的记载,当时孔子的弟子将其“故所居堂”立庙祭祀,庙屋三间,内藏衣、冠、琴、车、书等孔子遗物。哀公想要伐灭三桓,结果反被三桓逐赶,死于有山氏。哀公死后,三桓立公子宁,是为鲁悼公。悼公时期,三桓胜,鲁如小侯,卑于三桓之家。魯元公時(前436年~前416年),三桓逐漸失勢,直到鲁穆公时期(前415年-前383年),鲁国实行改革,任命博士公仪休为鲁相,遂渐从三桓手中收回政权,国政开始奉法循理,摆脱了三桓专政的问题,重新确立了公室的权威。而三桓之一的季氏则据其封邑费、卞,独立成为了费国。

戰国時期

楚灭鲁国:戰国時期魯国國力已衰弱,仍多次與齊国作戰。前323年,鲁景公卒,鲁平公即位,此时正是韩、魏、赵、燕、中山五国相王之年。鲁顷公二年(前278年),秦国破楚国首都郢,楚顷王东迁至陈国。顷公十九年(前261年),楚伐鲁取徐州。顷公二十四年(前256年),鲁国为楚考烈王所灭,迁顷公于下邑,封鲁君于莒。后七年(前249年)鲁顷公死于柯(今山东东阿),鲁国绝祀。

秦朝末年,楚後懷王曾封項羽為魯公。項羽死後,楚地人民都投降漢高祖劉邦,只有魯國不歸順,劉邦本來要以重兵屠殺魯國,後認為魯國長老嚴守禮義,為主死節,所以把項羽的首級拿出來給魯國的長老看,並答應禮葬項羽。之後,劉邦以「魯公」的公爵禮儀,在穀城埋葬項羽,並親自為其哭喪,魯國長老才投降。

汉平帝时期,封鲁顷公八世孙公子宽为褒鲁侯,奉周公祀,公子宽死后谥为“节”,其子公孙相如袭爵。王莽新朝时期,又封公孙相如后裔姬就为褒鲁子。


\subsection{伯禽生平}

伯禽(約前1068年-前998年),生年月不詳,姬姓,亦称禽父。周朝諸侯国魯国第一任君主,周公旦长子。《史記》記載就任年在周公東征,即周成王元年(約前1042年)。

周公東征之後,周成王将商朝遺民六族和泰山之南的原奄国土地、人民封給周公,為魯国。由于周公需要留在朝中,因此派其長子伯禽赴魯国就任。

伯禽到任之後,在齐太公的齊国軍隊支援下平定了淮夷和徐戎的叛乱,奠定了周朝在淮河以北地区的统治。在進軍過程中,伯禽在費地作《費誓》激励士氣,這篇文辞被記记載在尚書之中。

伯禽与齐国第二代君主齐丁公、卫国第二代君主卫康伯以及晋国第二代君主晋侯燮共事周康王。周康王分三位诸侯以珍宝之器。而同事周康王的楚君熊绎却无分。春秋时期的前530年,楚灵王仍忿然提起此事。

伯禽在位共46年,魯国在他的統治下成為著名的“礼儀之邦”,疆域北至泰山、南達徐州、東至黄海、西抵陽穀一带,成為在今山東境内与齊国抗衡的大国。

中国历史博物馆藏禽簋,《殷周金文集成》编号“七·四〇四一”。其铭文记载了周成王讨伐东方的奄侯,周公谋划这次征伐,而“禽”也就是当时为周王室大祝的伯禽,在助祭时宣读祝辞。

\subsection{世系图}

\noindent 伯禽 → 魯考公 → 魯煬公 → 魯幽公 → 魯魏公 → 魯厲公 → 魯獻公 → 魯真公 → 魯武公 → 魯懿公 → 伯御 → 魯孝公 → 魯惠公 → 鲁隐公 → 鲁桓公 → 魯莊公 → 子般 → 魯閔公 → 魯僖公 → 魯文公 → 魯宣公 → 魯成公 → 鲁襄公 → 子野 → 魯昭公 → 魯定公 → 魯哀公 → 魯悼公 → 魯元公 → 魯穆公 → 魯共公 → 魯康公 → 魯景公 → 魯平公 → 鲁文公 → 魯頃公

%% -*- coding: utf-8 -*-
%% Time-stamp: <Chen Wang: 2019-12-26 22:49:48>

\subsection{隐公{\tiny(BC722-BC712)}}

\subsubsection{生平}

魯隱公(?-前712年),姬姓,名息姑,魯国第十四代国君,前722年-前712年在位。魯惠公之子,生母是声子。傳世的魯國史書《春秋》及其三傳的記事都是從魯隱公開始的。

公元前722年,魯惠公死,嫡妻所生的公子軌當時還年幼,所以國人共立息姑攝政,行君事。在位十一年,隱公始終牢記自己是攝政行君事,一心等待公子軌長大,把國君的位置禪讓給他。因此,當公子翬提出請求殺死公子軌時,隱公断然拒絶;結果公子翬怕消息走漏,反而先跟公子軌合作,把隱公刺殺而死,公子軌即位,是為魯桓公。

隱公行君事期間,重視政治、外交,魯國國力較強。隱公除了到棠地觀人捕魚,被認為不合于禮法之外,處理政事、軍事都還比較謹慎公正,與鄰國修好,所以周圍小國如滕國、薛國等國都到魯國朝拜。與鄭國、齊國等強國也結好。

在位期間的卿為公子翬、无骇、公子益师、公子彄、挟、公子豫。

\subsubsection{年表}

% \centering
\begin{longtable}{|>{\centering\scriptsize}m{2em}|>{\centering\scriptsize}m{1.3em}|>{\centering}m{8.8em}|}
  % \caption{秦王政}\\
  \toprule
  \SimHei \normalsize 年数 & \SimHei \scriptsize 公元 & \SimHei 大事件 \tabularnewline
  % \midrule
  \endfirsthead
  \toprule
  \SimHei \normalsize 年数 & \SimHei \scriptsize 公元 & \SimHei 大事件 \tabularnewline
  \midrule
  \endhead
  \midrule
  元年 & -722 & \begin{enumerate}
    \tiny
  \item 三月,公及\xpinyin*{邾}儀父盟于蔑\footnote{公及邾儀父盟于蔑,邾子克也,未王命,故不書爵,曰儀父,貴之也,公攝位,而欲求好於邾,故為蔑之盟。}。
  \item 夏,五月,鄭伯克段于鄢\footnote{初,鄭武公娶于申,曰武姜。生莊公及共叔段。莊公寤生,驚姜氏,故名曰寤生,遂惡之。愛共叔段,欲立之。亟請於武公,公弗許。及莊公即位,為之請制。公曰:「制,巖邑也,虢叔死焉,佗邑唯命。」請京,使居之,謂之「京城大叔」。祭仲曰:「都城過百雉,國之害也,先王之制:大都不過參國之一;中,五之一;小,九之一。今京不度,非制也,君將不堪。」公曰:「姜氏欲之,焉辟害?」對曰:「姜氏何厭之有!不如早為之所,無使滋蔓。蔓,難圖也。蔓草猶不可除,況君之寵弟乎!」公曰:「多行不義,必自斃,子姑待之。」既而大叔命西鄙、北鄙貳於己。公子呂曰:「國不堪貳,君將若之何?欲與大叔,臣請事之;若弗與,則請除之,無生民心。」公曰:「無庸,將自及。」大叔又收貳以為己邑,至于廩延。子封曰:「可矣,厚將得眾。」公曰:「不義不暱,厚將崩。」大叔完聚,繕甲兵,具卒乘,將襲鄭。夫人將啟之。公聞其期,曰:「可矣。」命子封帥車二百乘以伐京。京叛大叔段,段入于鄢,公伐諸鄢。五月辛丑,大叔出奔共。}。
  \item 八月,紀人伐夷。
  \item 秋,七月,天王使宰咺來歸惠公仲子之賵\footnote{天子七月而葬,同軌畢至,諸侯五月,同盟至,大夫三月,同位至,士踰月,外姻至,贈死不及尸,弔生不及哀,豫凶事,非禮也。}。
  \item 九月,及宋人盟于宿。
  \end{enumerate} \tabularnewline\hline
  二年 & -721 & \tabularnewline\hline
  三年 & -720 & \tabularnewline\hline
  四年 & -719 & \tabularnewline\hline
  五年 & -718 & \tabularnewline\hline
  六年 & -717 & \tabularnewline\hline
  七年 & -716 & \tabularnewline\hline
  八年 & -715 & \tabularnewline\hline
  九年 & -714 & \tabularnewline\hline
  十年 & -713 & \tabularnewline\hline
  十一年 & -712 & \tabularnewline
  \bottomrule
\end{longtable}



%%% Local Variables:
%%% mode: latex
%%% TeX-engine: xetex
%%% TeX-master: "../../Main"
%%% End:

% %% -*- coding: utf-8 -*-
%% Time-stamp: <Chen Wang: 2018-07-12 22:55:13>

\subsection{桓公{\tiny(BC711-BC694)}}

% \centering
\begin{longtable}{|>{\centering\scriptsize}m{2em}|>{\centering\scriptsize}m{1.3em}|>{\centering}m{8.8em}|}
  % \caption{秦王政}\\
  \toprule
  \SimHei \normalsize 年数 & \SimHei \scriptsize 公元 & \SimHei 大事件 \tabularnewline
  % \midrule
  \endfirsthead
  \toprule
  \SimHei \normalsize 年数 & \SimHei \scriptsize 公元 & \SimHei 大事件 \tabularnewline
  \midrule
  \endhead
  \midrule
  元年 & -711 & \tabularnewline\hline
  二年 & -710 & \tabularnewline\hline
  三年 & -709 & \tabularnewline\hline
  四年 & -708 & \tabularnewline\hline
  五年 & -707 & \tabularnewline\hline
  六年 & -706 & \tabularnewline\hline
  七年 & -705 & \tabularnewline\hline
  八年 & -704 & \tabularnewline\hline
  九年 & -703 & \tabularnewline\hline
  十年 & -702 & \tabularnewline\hline
  十一年 & -701 & \tabularnewline\hline
  十二年 & -700 & \tabularnewline\hline
  十三年 & -699 & \tabularnewline\hline
  十四年 & -698 & \tabularnewline\hline
  十五年 & -697 & \tabularnewline\hline
  十六年 & -696 & \tabularnewline\hline
  十七年 & -695 & \tabularnewline\hline
  十八年 & -694 & \tabularnewline
  \bottomrule
\end{longtable}

%%% Local Variables:
%%% mode: latex
%%% TeX-engine: xetex
%%% TeX-master: "../../Main"
%%% End:

% %% -*- coding: utf-8 -*-
%% Time-stamp: <Chen Wang: 2018-07-12 23:02:54>

\subsection{庄公{\tiny(BC693-BC662)}}

% \centering
\begin{longtable}{|>{\centering\scriptsize}m{2em}|>{\centering\scriptsize}m{1.3em}|>{\centering}m{8.8em}|}
  % \caption{秦王政}\\
  \toprule
  \SimHei \normalsize 年数 & \SimHei \scriptsize 公元 & \SimHei 大事件 \tabularnewline
  % \midrule
  \endfirsthead
  \toprule
  \SimHei \normalsize 年数 & \SimHei \scriptsize 公元 & \SimHei 大事件 \tabularnewline
  \midrule
  \endhead
  \midrule
  元年 & -693 & \tabularnewline\hline
  二年 & -692 & \tabularnewline\hline
  三年 & -691 & \tabularnewline\hline
  四年 & -690 & \tabularnewline\hline
  五年 & -689 & \tabularnewline\hline
  六年 & -688 & \tabularnewline\hline
  七年 & -687 & \tabularnewline\hline
  八年 & -686 & \tabularnewline\hline
  九年 & -685 & \tabularnewline\hline
  十年 & -684 & \tabularnewline\hline
  十一年 & -683 & \tabularnewline\hline
  十二年 & -682 & \tabularnewline\hline
  十三年 & -681 & \tabularnewline\hline
  十四年 & -680 & \tabularnewline\hline
  十五年 & -679 & \tabularnewline\hline
  十六年 & -678 & \tabularnewline\hline
  十七年 & -677 & \tabularnewline\hline
  十八年 & -676 & \tabularnewline\hline
  十九年 & -675 & \tabularnewline\hline
  二十年 & -674 & \tabularnewline\hline
  二一年 & -673 & \tabularnewline\hline
  二二年 & -672 & \tabularnewline\hline
  二三年 & -671 & \tabularnewline\hline
  二四年 & -670 & \tabularnewline\hline
  二五年 & -669 & \tabularnewline\hline
  二六年 & -668 & \tabularnewline\hline
  二七年 & -667 & \tabularnewline\hline
  二八年 & -666 & \tabularnewline\hline
  二九年 & -665 & \tabularnewline\hline
  三十年 & -664 & \tabularnewline\hline
  三一年 & -663 & \tabularnewline\hline
  三二年 & -662 & \tabularnewline
  \bottomrule
\end{longtable}

%%% Local Variables:
%%% mode: latex
%%% TeX-engine: xetex
%%% TeX-master: "../../Main"
%%% End:

% %% -*- coding: utf-8 -*-
%% Time-stamp: <Chen Wang: 2021-11-01 17:59:15>

\subsection{闵公啟{\tiny(BC661-BC660)}}

\subsubsection{子斑生平}

子般(?-前662年),姬姓,名般,一作斑,,称子表示此时是其父鲁庄公死的当年,魯莊公之子。魯莊公的夫人哀姜是齊國人,無子。莊公臨死前欲立庶子般為嗣君,莊公弟叔牙建議立庄公庶长兄公子慶父,另一位弟弟季友則支持立子般,季友於是借莊公之命赐死叔牙。

三十二年八月,莊公病逝,季友立子般為君,十月慶父殺子般,立莊公另一庶子啟為魯君,是為魯閔公。季友逃亡陳國。

\subsubsection{閔公啟生平}

魯閔公(前669年?-前660年),即姬啟,一名啟方,為春秋諸侯國魯國君主之一,是魯國第十七任君主。他為魯莊公、叔姜的兒子。近人考証謂於周惠王八年(前669年)出生,至周惠王十七年(前660年)去世,年約十歲。(楊伯峻《春秋左傳注》,頁254)

莊公死前,弟弟叔牙建議立莊公庶長兄慶父,另一位弟弟季友則支持立子般,季友於是借莊公之命賜死叔牙,莊公病逝,季友立子般為君,十月慶父殺子般,立莊公另一庶子啟為魯君,即魯閔公,魯閔公亦是齊桓公的外甥,對齊桓公很尊敬,因此齊魯無大事,直到兩年後公子慶父以毒餅殺死魯閔公,齊桓公才派兵迎立魯閔公之弟魯釐公。

在位期間的卿為公子慶父、季友。

\subsubsection{年表}

% \centering
\begin{longtable}{|>{\centering\scriptsize}m{2em}|>{\centering\scriptsize}m{1.3em}|>{\centering}m{8.8em}|}
  % \caption{秦王政}\\
  \toprule
  \SimHei \normalsize 年数 & \SimHei \scriptsize 公元 & \SimHei 大事件 \tabularnewline
  % \midrule
  \endfirsthead
  \toprule
  \SimHei \normalsize 年数 & \SimHei \scriptsize 公元 & \SimHei 大事件 \tabularnewline
  \midrule
  \endhead
  \midrule
  元年 & -661 & \tabularnewline\hline
  二年 & -660 & \tabularnewline
  \bottomrule
\end{longtable}

%%% Local Variables:
%%% mode: latex
%%% TeX-engine: xetex
%%% TeX-master: "../../Main"
%%% End:

% %% -*- coding: utf-8 -*-
%% Time-stamp: <Chen Wang: 2018-07-12 23:08:29>

\subsection{僖公{\tiny(BC659-BC627)}}

% \centering
\begin{longtable}{|>{\centering\scriptsize}m{2em}|>{\centering\scriptsize}m{1.3em}|>{\centering}m{8.8em}|}
  % \caption{秦王政}\\
  \toprule
  \SimHei \normalsize 年数 & \SimHei \scriptsize 公元 & \SimHei 大事件 \tabularnewline
  % \midrule
  \endfirsthead
  \toprule
  \SimHei \normalsize 年数 & \SimHei \scriptsize 公元 & \SimHei 大事件 \tabularnewline
  \midrule
  \endhead
  \midrule
  元年 & -659 & \tabularnewline\hline
  二年 & -658 & \tabularnewline\hline
  三年 & -657 & \tabularnewline\hline
  四年 & -656 & \tabularnewline\hline
  五年 & -655 & \tabularnewline\hline
  六年 & -654 & \tabularnewline\hline
  七年 & -653 & \tabularnewline\hline
  八年 & -652 & \tabularnewline\hline
  九年 & -651 & \tabularnewline\hline
  十年 & -650 & \tabularnewline\hline
  十一年 & -649 & \tabularnewline\hline
  十二年 & -648 & \tabularnewline\hline
  十三年 & -647 & \tabularnewline\hline
  十四年 & -646 & \tabularnewline\hline
  十五年 & -645 & \tabularnewline\hline
  十六年 & -644 & \tabularnewline\hline
  十七年 & -643 & \tabularnewline\hline
  十八年 & -642 & \tabularnewline\hline
  十九年 & -641 & \tabularnewline\hline
  二十年 & -640 & \tabularnewline\hline
  二一年 & -639 & \tabularnewline\hline
  二二年 & -638 & \tabularnewline\hline
  二三年 & -637 & \tabularnewline\hline
  二四年 & -636 & \tabularnewline\hline
  二五年 & -635 & \tabularnewline\hline
  二六年 & -634 & \tabularnewline\hline
  二七年 & -633 & \tabularnewline\hline
  二八年 & -632 & \tabularnewline\hline
  二九年 & -631 & \tabularnewline\hline
  三十年 & -630 & \tabularnewline\hline
  三一年 & -629 & \tabularnewline\hline
  三二年 & -628 & \tabularnewline\hline
  三三年 & -627 & \tabularnewline
  \bottomrule
\end{longtable}

%%% Local Variables:
%%% mode: latex
%%% TeX-engine: xetex
%%% TeX-master: "../../Main"
%%% End:

% %% -*- coding: utf-8 -*-
%% Time-stamp: <Chen Wang: 2018-07-12 23:10:00>

\subsection{文公{\tiny(BC626-BC609)}}

% \centering
\begin{longtable}{|>{\centering\scriptsize}m{2em}|>{\centering\scriptsize}m{1.3em}|>{\centering}m{8.8em}|}
  % \caption{秦王政}\\
  \toprule
  \SimHei \normalsize 年数 & \SimHei \scriptsize 公元 & \SimHei 大事件 \tabularnewline
  % \midrule
  \endfirsthead
  \toprule
  \SimHei \normalsize 年数 & \SimHei \scriptsize 公元 & \SimHei 大事件 \tabularnewline
  \midrule
  \endhead
  \midrule
  元年 & -626 & \tabularnewline\hline
  二年 & -625 & \tabularnewline\hline
  三年 & -624 & \tabularnewline\hline
  四年 & -623 & \tabularnewline\hline
  五年 & -622 & \tabularnewline\hline
  六年 & -621 & \tabularnewline\hline
  七年 & -620 & \tabularnewline\hline
  八年 & -619 & \tabularnewline\hline
  九年 & -618 & \tabularnewline\hline
  十年 & -617 & \tabularnewline\hline
  十一年 & -616 & \tabularnewline\hline
  十二年 & -615 & \tabularnewline\hline
  十三年 & -614 & \tabularnewline\hline
  十四年 & -613 & \tabularnewline\hline
  十五年 & -612 & \tabularnewline\hline
  十六年 & -611 & \tabularnewline\hline
  十七年 & -610 & \tabularnewline\hline
  十八年 & -609 & \tabularnewline
  \bottomrule
\end{longtable}

%%% Local Variables:
%%% mode: latex
%%% TeX-engine: xetex
%%% TeX-master: "../../Main"
%%% End:

% %% -*- coding: utf-8 -*-
%% Time-stamp: <Chen Wang: 2018-07-12 23:11:20>

\subsection{宣公{\tiny(BC608-BC591)}}

% \centering
\begin{longtable}{|>{\centering\scriptsize}m{2em}|>{\centering\scriptsize}m{1.3em}|>{\centering}m{8.8em}|}
  % \caption{秦王政}\\
  \toprule
  \SimHei \normalsize 年数 & \SimHei \scriptsize 公元 & \SimHei 大事件 \tabularnewline
  % \midrule
  \endfirsthead
  \toprule
  \SimHei \normalsize 年数 & \SimHei \scriptsize 公元 & \SimHei 大事件 \tabularnewline
  \midrule
  \endhead
  \midrule
  元年 & -608 & \tabularnewline\hline
  二年 & -607 & \tabularnewline\hline
  三年 & -606 & \tabularnewline\hline
  四年 & -605 & \tabularnewline\hline
  五年 & -604 & \tabularnewline\hline
  六年 & -603 & \tabularnewline\hline
  七年 & -602 & \tabularnewline\hline
  八年 & -601 & \tabularnewline\hline
  九年 & -600 & \tabularnewline\hline
  十年 & -599 & \tabularnewline\hline
  十一年 & -598 & \tabularnewline\hline
  十二年 & -597 & \tabularnewline\hline
  十三年 & -596 & \tabularnewline\hline
  十四年 & -595 & \tabularnewline\hline
  十五年 & -594 & \tabularnewline\hline
  十六年 & -593 & \tabularnewline\hline
  十七年 & -592 & \tabularnewline\hline
  十八年 & -591 & \tabularnewline
  \bottomrule
\end{longtable}

%%% Local Variables:
%%% mode: latex
%%% TeX-engine: xetex
%%% TeX-master: "../../Main"
%%% End:

% %% -*- coding: utf-8 -*-
%% Time-stamp: <Chen Wang: 2021-11-01 17:57:14>

\subsection{成公黑肱{\tiny(BC590-BC573)}}

\subsubsection{生平}

魯成公(?-前573年),姬姓,名黑肱,為東周春秋時期諸侯國魯國的一位君主,是魯國第二十一任君主,承襲父親魯宣公擔任該國君主,在位18年。

在位期間執政為季孫行父、仲孫蔑、叔孫僑如。

魯成公元年(前590年),季孫行父在魯國實行一種「作丘(地區單位)甲」的地區編制。

魯成公二年(前589年)春天,齊國要攻打魯、衛兩國。魯、衛兩國大夫請求晉國出兵,晉國以郤克為主將率兵救討伐齊國,以救援魯、衛二國。同一年,齊頃公親率齊軍南下攻打魯國龍邑(今山東泰安東南),寵臣盧蒲就癸被殺,頃公怒而攻至巢丘(今山東泰安境內)。季孫行父率魯軍幫晉、衛、曹等國,去攻打齊國的鞍(今山東濟南市)。齊頃公在鞍之戰大敗,齊頃公被晉軍追逼,「差點被俘,幸得大夫逢丑父相救,二人互換衣服,佯命齊頃公到山腳華泉取水,得以逃走。同年十一月,魯國的魯成公同蔡景侯、許靈公、秦國右大夫說、宋國華元、陳國公孫寧、衛國孫良夫、鄭國子良、齊國大夫、曹、邾、薛、鄫等多國代表參與由楚國公子嬰齊在蜀(今山東省泰安市東南)所主辦的會盟。

魯成公七年(前584年),吳國攻打鄰近魯國的郯國,郯國被納入吳國的領土的事,因此季孫行父向魯國國君發出「中國不振旅,蠻夷(指吳國)來伐」的警告。

魯成公十六年(前575年),魯成公的夫人定姒產下魯成公的兒子姬午。

魯成公十八年(前573年),魯成公薨,姬午即位(即後來的魯襄公)。

\subsubsection{年表}

% \centering
\begin{longtable}{|>{\centering\scriptsize}m{2em}|>{\centering\scriptsize}m{1.3em}|>{\centering}m{8.8em}|}
  % \caption{秦王政}\\
  \toprule
  \SimHei \normalsize 年数 & \SimHei \scriptsize 公元 & \SimHei 大事件 \tabularnewline
  % \midrule
  \endfirsthead
  \toprule
  \SimHei \normalsize 年数 & \SimHei \scriptsize 公元 & \SimHei 大事件 \tabularnewline
  \midrule
  \endhead
  \midrule
  元年 & -590 & \tabularnewline\hline
  二年 & -589 & \tabularnewline\hline
  三年 & -588 & \tabularnewline\hline
  四年 & -587 & \tabularnewline\hline
  五年 & -586 & \tabularnewline\hline
  六年 & -585 & \tabularnewline\hline
  七年 & -584 & \tabularnewline\hline
  八年 & -583 & \tabularnewline\hline
  九年 & -582 & \tabularnewline\hline
  十年 & -581 & \tabularnewline\hline
  十一年 & -580 & \tabularnewline\hline
  十二年 & -579 & \tabularnewline\hline
  十三年 & -578 & \tabularnewline\hline
  十四年 & -577 & \tabularnewline\hline
  十五年 & -576 & \tabularnewline\hline
  十六年 & -575 & \tabularnewline\hline
  十七年 & -574 & \tabularnewline\hline
  十八年 & -573 & \tabularnewline
  \bottomrule
\end{longtable}

%%% Local Variables:
%%% mode: latex
%%% TeX-engine: xetex
%%% TeX-master: "../../Main"
%%% End:

% %% -*- coding: utf-8 -*-
%% Time-stamp: <Chen Wang: 2021-11-01 17:58:32>

\subsection{襄公午{\tiny(BC572-BC542)}}

\subsubsection{生平}

魯襄公(前575年-前542年),姬姓,名午,春秋時代魯國的第二十二代君主,魯成公之子,於魯成公十六年(公元前575年)誕生。

魯成公十八年(前573年),魯成公去世,由四歲的太子午即君主之位,是為魯襄公。執政為正卿司徒季孫行父,保持魯國的相對穩定。

魯襄公五年(前568年),魯襄公九歲,執政正卿為季孫行父、仲孫蔑。季孫行父去世,行父以薄葬來進行下葬儀式,這時魯襄公很感動地稱讚的說行父是個廉吏,於是襄公給行父的諡號為「文」。

\subsubsection{年表}

% \centering
\begin{longtable}{|>{\centering\scriptsize}m{2em}|>{\centering\scriptsize}m{1.3em}|>{\centering}m{8.8em}|}
  % \caption{秦王政}\\
  \toprule
  \SimHei \normalsize 年数 & \SimHei \scriptsize 公元 & \SimHei 大事件 \tabularnewline
  % \midrule
  \endfirsthead
  \toprule
  \SimHei \normalsize 年数 & \SimHei \scriptsize 公元 & \SimHei 大事件 \tabularnewline
  \midrule
  \endhead
  \midrule
  元年 & -572 & \tabularnewline\hline
  二年 & -571 & \tabularnewline\hline
  三年 & -570 & \tabularnewline\hline
  四年 & -569 & \tabularnewline\hline
  五年 & -568 & \tabularnewline\hline
  六年 & -567 & \tabularnewline\hline
  七年 & -566 & \tabularnewline\hline
  八年 & -565 & \tabularnewline\hline
  九年 & -564 & \tabularnewline\hline
  十年 & -563 & \tabularnewline\hline
  十一年 & -562 & \tabularnewline\hline
  十二年 & -561 & \tabularnewline\hline
  十三年 & -560 & \tabularnewline\hline
  十四年 & -559 & \tabularnewline\hline
  十五年 & -558 & \tabularnewline\hline
  十六年 & -557 & \tabularnewline\hline
  十七年 & -556 & \tabularnewline\hline
  十八年 & -555 & \tabularnewline\hline
  十九年 & -554 & \tabularnewline\hline
  二十年 & -553 & \tabularnewline\hline
  二一年 & -552 & \tabularnewline\hline
  二二年 & -551 & \tabularnewline\hline
  二三年 & -550 & \tabularnewline\hline
  二四年 & -549 & \tabularnewline\hline
  二五年 & -548 & \tabularnewline\hline
  二六年 & -547 & \tabularnewline\hline
  二七年 & -546 & \tabularnewline\hline
  二八年 & -545 & \tabularnewline\hline
  二九年 & -544 & \tabularnewline\hline
  三十年 & -543 & \tabularnewline\hline
  三一年 & -542 & \tabularnewline
  \bottomrule
\end{longtable}

%%% Local Variables:
%%% mode: latex
%%% TeX-engine: xetex
%%% TeX-master: "../../Main"
%%% End:

% %% -*- coding: utf-8 -*-
%% Time-stamp: <Chen Wang: 2021-11-01 18:02:17>

\subsection{昭公稠{\tiny(BC541-BC510)}}

\subsubsection{子野生平}

子野(?-前542年),姬姓,名野,子表示此时是其父鲁襄公死的当年。子野是魯襄公的庶子,魯昭公和鲁定公之兄,母为敬归。

前542年六月二十八日,魯襄公去世,鲁国人拥立太子子野即位,住在季氏那里。九月十一日,子野由于哀痛过度而死。

魯國人便擁立子野生母敬归的妹妹齊歸生的儿子公子裯為国君,是為魯昭公。

\subsubsection{昭公稠生平}

魯昭公(?-前510年),姬姓,名稠,魯国之二十四代君主。前542年即位,前517年,魯昭公伐季孙氏,但大败,魯昭公逃到齐国,前510年,昭公死。在其任內,他嘗試與季平子政治角力,演變成「鬥雞之變」,使昭公逃到齊國。

在位期間執政為季孫宿、叔孫婼、仲孫貜。

魯昭公二十三年(前519年),叔孫昭子將魯政讓位給季孫意如。

鬥雞之變後,在位期間執政為仲孫何忌、叔孫不敢。

\subsubsection{年表}

% \centering
\begin{longtable}{|>{\centering\scriptsize}m{2em}|>{\centering\scriptsize}m{1.3em}|>{\centering}m{8.8em}|}
  % \caption{秦王政}\\
  \toprule
  \SimHei \normalsize 年数 & \SimHei \scriptsize 公元 & \SimHei 大事件 \tabularnewline
  % \midrule
  \endfirsthead
  \toprule
  \SimHei \normalsize 年数 & \SimHei \scriptsize 公元 & \SimHei 大事件 \tabularnewline
  \midrule
  \endhead
  \midrule
  元年 & -541 & \tabularnewline\hline
  二年 & -540 & \tabularnewline\hline
  三年 & -539 & \tabularnewline\hline
  四年 & -538 & \tabularnewline\hline
  五年 & -537 & \tabularnewline\hline
  六年 & -536 & \tabularnewline\hline
  七年 & -535 & \tabularnewline\hline
  八年 & -534 & \tabularnewline\hline
  九年 & -533 & \tabularnewline\hline
  十年 & -532 & \tabularnewline\hline
  十一年 & -531 & \tabularnewline\hline
  十二年 & -530 & \tabularnewline\hline
  十三年 & -529 & \tabularnewline\hline
  十四年 & -528 & \tabularnewline\hline
  十五年 & -527 & \tabularnewline\hline
  十六年 & -526 & \tabularnewline\hline
  十七年 & -525 & \tabularnewline\hline
  十八年 & -524 & \tabularnewline\hline
  十九年 & -523 & \tabularnewline\hline
  二十年 & -522 & \tabularnewline\hline
  二一年 & -521 & \tabularnewline\hline
  二二年 & -520 & \tabularnewline\hline
  二三年 & -519 & \tabularnewline\hline
  二四年 & -518 & \tabularnewline\hline
  二五年 & -517 & \tabularnewline\hline
  二六年 & -516 & \tabularnewline\hline
  二七年 & -515 & \tabularnewline\hline
  二八年 & -514 & \tabularnewline\hline
  二九年 & -513 & \tabularnewline\hline
  三十年 & -512 & \tabularnewline\hline
  三一年 & -511 & \tabularnewline\hline
  三二年 & -510 & \tabularnewline
  \bottomrule
\end{longtable}

%%% Local Variables:
%%% mode: latex
%%% TeX-engine: xetex
%%% TeX-master: "../../Main"
%%% End:

% %% -*- coding: utf-8 -*-
%% Time-stamp: <Chen Wang: 2021-11-01 18:02:49>

\subsection{定公宋{\tiny(BC509-BC495)}}

\subsubsection{生平}

魯定公(前556年-前495年),姬姓,名宋,為中國春秋時期諸侯國魯國君主之一,是魯國第二十五任君主。他為魯昭公的庶弟,承襲魯昭公擔任該國君主,在位15年。

在位期間執政為季孫意如、叔孫不敢、仲孫何忌、季孫斯、叔孫州仇,其中公元前505年~前503年,執政主官為季孫氏家宰陽虎。前501年~前497年,任命孔子為大司寇。期間行攝相事。

\subsubsection{年表}

% \centering
\begin{longtable}{|>{\centering\scriptsize}m{2em}|>{\centering\scriptsize}m{1.3em}|>{\centering}m{8.8em}|}
  % \caption{秦王政}\\
  \toprule
  \SimHei \normalsize 年数 & \SimHei \scriptsize 公元 & \SimHei 大事件 \tabularnewline
  % \midrule
  \endfirsthead
  \toprule
  \SimHei \normalsize 年数 & \SimHei \scriptsize 公元 & \SimHei 大事件 \tabularnewline
  \midrule
  \endhead
  \midrule
  元年 & -509 & \tabularnewline\hline
  二年 & -508 & \tabularnewline\hline
  三年 & -507 & \tabularnewline\hline
  四年 & -506 & \tabularnewline\hline
  五年 & -505 & \tabularnewline\hline
  六年 & -504 & \tabularnewline\hline
  七年 & -503 & \tabularnewline\hline
  八年 & -502 & \tabularnewline\hline
  九年 & -501 & \tabularnewline\hline
  十年 & -500 & \tabularnewline\hline
  十一年 & -499 & \tabularnewline\hline
  十二年 & -498 & \tabularnewline\hline
  十三年 & -497 & \tabularnewline\hline
  十四年 & -496 & \tabularnewline\hline
  十五年 & -495 & \tabularnewline
  \bottomrule
\end{longtable}

%%% Local Variables:
%%% mode: latex
%%% TeX-engine: xetex
%%% TeX-master: "../../Main"
%%% End:

% %% -*- coding: utf-8 -*-
%% Time-stamp: <Chen Wang: 2021-11-01 18:03:17>

\subsection{哀公將{\tiny(BC494-BC467)}}

\subsubsection{生平}

鲁哀公(约前508年-前468年),姬姓,名將,為春秋諸侯國魯國君主之一,是魯國第二十六任君主。魯定公之子,承襲魯定公擔任該國君主,在位27年。

鲁哀公在位时,鲁国大权被卿大夫家族把持,史称三桓,即所谓“政在大夫”。鲁哀公曾经试图恢复君主权力,同三家大夫冲突加剧,终致流亡越国。魯哀公27年,鲁哀公通过邾国逃到越国。

在位期間執政的士大夫為季孫斯、叔孫州仇、仲孫何忌、季孫肥、叔孫舒、仲孫彘。

\subsubsection{年表}

% \centering
\begin{longtable}{|>{\centering\scriptsize}m{2em}|>{\centering\scriptsize}m{1.3em}|>{\centering}m{8.8em}|}
  % \caption{秦王政}\\
  \toprule
  \SimHei \normalsize 年数 & \SimHei \scriptsize 公元 & \SimHei 大事件 \tabularnewline
  % \midrule
  \endfirsthead
  \toprule
  \SimHei \normalsize 年数 & \SimHei \scriptsize 公元 & \SimHei 大事件 \tabularnewline
  \midrule
  \endhead
  \midrule
  元年 & -494 & \tabularnewline\hline
  二年 & -493 & \tabularnewline\hline
  三年 & -492 & \tabularnewline\hline
  四年 & -491 & \tabularnewline\hline
  五年 & -490 & \tabularnewline\hline
  六年 & -489 & \tabularnewline\hline
  七年 & -488 & \tabularnewline\hline
  八年 & -487 & \tabularnewline\hline
  九年 & -486 & \tabularnewline\hline
  十年 & -485 & \tabularnewline\hline
  十一年 & -484 & \tabularnewline\hline
  十二年 & -483 & \tabularnewline\hline
  十三年 & -482 & \tabularnewline\hline
  十四年 & -481 & \tabularnewline\hline
  十五年 & -480 & \tabularnewline\hline
  十六年 & -479 & \tabularnewline\hline
  十七年 & -478 & \tabularnewline\hline
  十八年 & -477 & \tabularnewline\hline
  十九年 & -476 & \tabularnewline\hline
  二十年 & -475 & \tabularnewline\hline
  二一年 & -474 & \tabularnewline\hline
  二二年 & -473 & \tabularnewline\hline
  二三年 & -472 & \tabularnewline\hline
  二四年 & -471 & \tabularnewline\hline
  二五年 & -470 & \tabularnewline\hline
  二六年 & -469 & \tabularnewline\hline
  二七年 & -468 & \tabularnewline\hline
  二八年 & -467 & \tabularnewline
  \bottomrule
\end{longtable}

%%% Local Variables:
%%% mode: latex
%%% TeX-engine: xetex
%%% TeX-master: "../../Main"
%%% End:

% %% -*- coding: utf-8 -*-
%% Time-stamp: <Chen Wang: 2018-07-12 23:19:27>

\subsection{悼公{\tiny(BC466-BC429)}}

% \centering
\begin{longtable}{|>{\centering\scriptsize}m{2em}|>{\centering\scriptsize}m{1.3em}|>{\centering}m{8.8em}|}
  % \caption{秦王政}\\
  \toprule
  \SimHei \normalsize 年数 & \SimHei \scriptsize 公元 & \SimHei 大事件 \tabularnewline
  % \midrule
  \endfirsthead
  \toprule
  \SimHei \normalsize 年数 & \SimHei \scriptsize 公元 & \SimHei 大事件 \tabularnewline
  \midrule
  \endhead
  \midrule
  元年 & -466 & \tabularnewline\hline
  二年 & -465 & \tabularnewline\hline
  三年 & -464 & \tabularnewline\hline
  四年 & -463 & \tabularnewline\hline
  五年 & -462 & \tabularnewline\hline
  六年 & -461 & \tabularnewline\hline
  七年 & -460 & \tabularnewline\hline
  八年 & -459 & \tabularnewline\hline
  九年 & -458 & \tabularnewline\hline
  十年 & -457 & \tabularnewline\hline
  十一年 & -456 & \tabularnewline\hline
  十二年 & -455 & \tabularnewline\hline
  十三年 & -454 & \tabularnewline\hline
  十四年 & -453 & \tabularnewline\hline
  十五年 & -452 & \tabularnewline\hline
  十六年 & -451 & \tabularnewline\hline
  十七年 & -450 & \tabularnewline\hline
  十八年 & -449 & \tabularnewline\hline
  十九年 & -448 & \tabularnewline\hline
  二十年 & -447 & \tabularnewline\hline
  二一年 & -446 & \tabularnewline\hline
  二二年 & -445 & \tabularnewline\hline
  二三年 & -444 & \tabularnewline\hline
  二四年 & -443 & \tabularnewline\hline
  二五年 & -442 & \tabularnewline\hline
  二六年 & -441 & \tabularnewline\hline
  二七年 & -440 & \tabularnewline\hline
  二八年 & -439 & \tabularnewline\hline
  二九年 & -438 & \tabularnewline\hline
  三十年 & -437 & \tabularnewline\hline
  三一年 & -436 & \tabularnewline\hline
  三二年 & -435 & \tabularnewline\hline
  三三年 & -434 & \tabularnewline\hline
  三四年 & -433 & \tabularnewline\hline
  三五年 & -432 & \tabularnewline\hline
  三六年 & -431 & \tabularnewline\hline
  三七年 & -430 & \tabularnewline\hline
  三八年 & -429 & \tabularnewline
  \bottomrule
\end{longtable}

%%% Local Variables:
%%% mode: latex
%%% TeX-engine: xetex
%%% TeX-master: "../../Main"
%%% End:

% %% -*- coding: utf-8 -*-
%% Time-stamp: <Chen Wang: 2018-07-12 23:20:46>

\subsection{元公{\tiny(BC428-BC408)}}

% \centering
\begin{longtable}{|>{\centering\scriptsize}m{2em}|>{\centering\scriptsize}m{1.3em}|>{\centering}m{8.8em}|}
  % \caption{秦王政}\\
  \toprule
  \SimHei \normalsize 年数 & \SimHei \scriptsize 公元 & \SimHei 大事件 \tabularnewline
  % \midrule
  \endfirsthead
  \toprule
  \SimHei \normalsize 年数 & \SimHei \scriptsize 公元 & \SimHei 大事件 \tabularnewline
  \midrule
  \endhead
  \midrule
  元年 & -428 & \tabularnewline\hline
  二年 & -427 & \tabularnewline\hline
  三年 & -426 & \tabularnewline\hline
  四年 & -425 & \tabularnewline\hline
  五年 & -424 & \tabularnewline\hline
  六年 & -423 & \tabularnewline\hline
  七年 & -422 & \tabularnewline\hline
  八年 & -421 & \tabularnewline\hline
  九年 & -420 & \tabularnewline\hline
  十年 & -419 & \tabularnewline\hline
  十一年 & -418 & \tabularnewline\hline
  十二年 & -417 & \tabularnewline\hline
  十三年 & -416 & \tabularnewline\hline
  十四年 & -415 & \tabularnewline\hline
  十五年 & -414 & \tabularnewline\hline
  十六年 & -413 & \tabularnewline\hline
  十七年 & -412 & \tabularnewline\hline
  十八年 & -411 & \tabularnewline\hline
  十九年 & -410 & \tabularnewline\hline
  二十年 & -409 & \tabularnewline\hline
  二一年 & -408 & \tabularnewline
  \bottomrule
\end{longtable}

%%% Local Variables:
%%% mode: latex
%%% TeX-engine: xetex
%%% TeX-master: "../../Main"
%%% End:

% %% -*- coding: utf-8 -*-
%% Time-stamp: <Chen Wang: 2018-07-12 23:23:22>

\subsection{穆公{\tiny(BC407-BC376)}}

% \centering
\begin{longtable}{|>{\centering\scriptsize}m{2em}|>{\centering\scriptsize}m{1.3em}|>{\centering}m{8.8em}|}
  % \caption{秦王政}\\
  \toprule
  \SimHei \normalsize 年数 & \SimHei \scriptsize 公元 & \SimHei 大事件 \tabularnewline
  % \midrule
  \endfirsthead
  \toprule
  \SimHei \normalsize 年数 & \SimHei \scriptsize 公元 & \SimHei 大事件 \tabularnewline
  \midrule
  \endhead
  \midrule
  元年 & -407 & \tabularnewline\hline
  二年 & -406 & \tabularnewline\hline
  三年 & -405 & \tabularnewline\hline
  四年 & -404 & \tabularnewline\hline
  五年 & -403 & \tabularnewline
  \bottomrule
\end{longtable}

%%% Local Variables:
%%% mode: latex
%%% TeX-engine: xetex
%%% TeX-master: "../../Main"
%%% End:



%%% Local Variables:
%%% mode: latex
%%% TeX-engine: xetex
%%% TeX-master: "../../Main"
%%% End:
