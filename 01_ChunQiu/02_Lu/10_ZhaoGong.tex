%% -*- coding: utf-8 -*-
%% Time-stamp: <Chen Wang: 2021-11-01 18:02:17>

\subsection{昭公稠{\tiny(BC541-BC510)}}

\subsubsection{子野生平}

子野(?-前542年),姬姓,名野,子表示此时是其父鲁襄公死的当年。子野是魯襄公的庶子,魯昭公和鲁定公之兄,母为敬归。

前542年六月二十八日,魯襄公去世,鲁国人拥立太子子野即位,住在季氏那里。九月十一日,子野由于哀痛过度而死。

魯國人便擁立子野生母敬归的妹妹齊歸生的儿子公子裯為国君,是為魯昭公。

\subsubsection{昭公稠生平}

魯昭公(?-前510年),姬姓,名稠,魯国之二十四代君主。前542年即位,前517年,魯昭公伐季孙氏,但大败,魯昭公逃到齐国,前510年,昭公死。在其任內,他嘗試與季平子政治角力,演變成「鬥雞之變」,使昭公逃到齊國。

在位期間執政為季孫宿、叔孫婼、仲孫貜。

魯昭公二十三年(前519年),叔孫昭子將魯政讓位給季孫意如。

鬥雞之變後,在位期間執政為仲孫何忌、叔孫不敢。

\subsubsection{年表}

% \centering
\begin{longtable}{|>{\centering\scriptsize}m{2em}|>{\centering\scriptsize}m{1.3em}|>{\centering}m{8.8em}|}
  % \caption{秦王政}\\
  \toprule
  \SimHei \normalsize 年数 & \SimHei \scriptsize 公元 & \SimHei 大事件 \tabularnewline
  % \midrule
  \endfirsthead
  \toprule
  \SimHei \normalsize 年数 & \SimHei \scriptsize 公元 & \SimHei 大事件 \tabularnewline
  \midrule
  \endhead
  \midrule
  元年 & -541 & \tabularnewline\hline
  二年 & -540 & \tabularnewline\hline
  三年 & -539 & \tabularnewline\hline
  四年 & -538 & \tabularnewline\hline
  五年 & -537 & \tabularnewline\hline
  六年 & -536 & \tabularnewline\hline
  七年 & -535 & \tabularnewline\hline
  八年 & -534 & \tabularnewline\hline
  九年 & -533 & \tabularnewline\hline
  十年 & -532 & \tabularnewline\hline
  十一年 & -531 & \tabularnewline\hline
  十二年 & -530 & \tabularnewline\hline
  十三年 & -529 & \tabularnewline\hline
  十四年 & -528 & \tabularnewline\hline
  十五年 & -527 & \tabularnewline\hline
  十六年 & -526 & \tabularnewline\hline
  十七年 & -525 & \tabularnewline\hline
  十八年 & -524 & \tabularnewline\hline
  十九年 & -523 & \tabularnewline\hline
  二十年 & -522 & \tabularnewline\hline
  二一年 & -521 & \tabularnewline\hline
  二二年 & -520 & \tabularnewline\hline
  二三年 & -519 & \tabularnewline\hline
  二四年 & -518 & \tabularnewline\hline
  二五年 & -517 & \tabularnewline\hline
  二六年 & -516 & \tabularnewline\hline
  二七年 & -515 & \tabularnewline\hline
  二八年 & -514 & \tabularnewline\hline
  二九年 & -513 & \tabularnewline\hline
  三十年 & -512 & \tabularnewline\hline
  三一年 & -511 & \tabularnewline\hline
  三二年 & -510 & \tabularnewline
  \bottomrule
\end{longtable}

%%% Local Variables:
%%% mode: latex
%%% TeX-engine: xetex
%%% TeX-master: "../../Main"
%%% End:
