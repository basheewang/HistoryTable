%% -*- coding: utf-8 -*-
%% Time-stamp: <Chen Wang: 2019-12-27 10:46:49>

\subsection{隐公{\tiny(BC722-BC712)}}

\subsubsection{生平}

魯隱公(?-前712年),姬姓,名息姑,魯国第十四代国君,前722年-前712年在位。魯惠公之子,生母是声子。傳世的魯國史書《春秋》及其三傳的記事都是從魯隱公開始的。

公元前722年,魯惠公死,嫡妻所生的公子軌當時還年幼,所以國人共立息姑攝政,行君事。在位十一年,隱公始終牢記自己是攝政行君事,一心等待公子軌長大,把國君的位置禪讓給他。因此,當公子翬提出請求殺死公子軌時,隱公断然拒絶;結果公子翬怕消息走漏,反而先跟公子軌合作,把隱公刺殺而死,公子軌即位,是為魯桓公。

隱公行君事期間,重視政治、外交,魯國國力較強。隱公除了到棠地觀人捕魚,被認為不合于禮法之外,處理政事、軍事都還比較謹慎公正,與鄰國修好,所以周圍小國如滕國、薛國等國都到魯國朝拜。與鄭國、齊國等強國也結好。

在位期間的卿為公子翬、无骇、公子益师、公子彄、挟、公子豫。

\subsubsection{年表}

% \centering
\begin{longtable}{|>{\centering\scriptsize}m{2em}|>{\centering\scriptsize}m{1.3em}|>{\centering}m{8.8em}|}
  % \caption{秦王政}\\
  \toprule
  \SimHei \normalsize 年数 & \SimHei \scriptsize 公元 & \SimHei 大事件 \tabularnewline
  % \midrule
  \endfirsthead
  \toprule
  \SimHei \normalsize 年数 & \SimHei \scriptsize 公元 & \SimHei 大事件 \tabularnewline
  \midrule
  \endhead
  \midrule
  元年\footnote{惠公元妃孟子,孟子卒,繼室以聲子,生隱公,宋武公生仲子,仲子生而有文在其手,曰為魯夫人,故仲子歸于我,生桓公而惠公薨,是以隱公立而奉之。} & -722 & \begin{enumerate}
    \tiny
  \item 三月,公及\xpinyin*{邾}儀父盟于蔑\footnote{公及邾儀父盟于蔑,邾子克也,未王命,故不書爵,曰儀父,貴之也,公攝位,而欲求好於邾,故為蔑之盟。}。
  \item 夏,五月,鄭伯克段于鄢\footnote{初,鄭武公娶于申,曰武姜。生莊公及共叔段。莊公寤生,驚姜氏,故名曰寤生,遂惡之。愛共叔段,欲立之。亟請於武公,公弗許。及莊公即位,為之請制。公曰:「制,巖邑也,虢叔死焉,佗邑唯命。」請京,使居之,謂之「京城大叔」。祭仲曰:「都城過百雉,國之害也,先王之制:大都不過參國之一;中,五之一;小,九之一。今京不度,非制也,君將不堪。」公曰:「姜氏欲之,焉辟害?」對曰:「姜氏何厭之有!不如早為之所,無使滋蔓。蔓,難圖也。蔓草猶不可除,況君之寵弟乎!」公曰:「多行不義,必自斃,子姑待之。」既而大叔命西鄙、北鄙貳於己。公子呂曰:「國不堪貳,君將若之何?欲與大叔,臣請事之;若弗與,則請除之,無生民心。」公曰:「無庸,將自及。」大叔又收貳以為己邑,至于廩延。子封曰:「可矣,厚將得眾。」公曰:「不義不暱,厚將崩。」大叔完聚,繕甲兵,具卒乘,將襲鄭。夫人將啟之。公聞其期,曰:「可矣。」命子封帥車二百乘以伐京。京叛大叔段,段入于鄢,公伐諸鄢。五月辛丑,大叔出奔共。}。
  \item 八月,紀人伐夷,有蜚,不為災。
  \item 秋,七月,天王使宰咺來歸惠公仲子之賵\footnote{天子七月而葬,同軌畢至,諸侯五月,同盟至,大夫三月,同位至,士踰月,外姻至,贈死不及尸,弔生不及哀,豫凶事,非禮也。}。
  \item 九月,及宋人盟于宿。
  \item 冬十月,庚申,改葬惠公\footnote{惠公之薨也,有宋師,太子少,葬故有闕,是以改葬。},公弗臨。
  \end{enumerate} \tabularnewline\hline
  二年 & -721 & \begin{enumerate}
    \tiny
  \item 春,公會戎于潛。八月,庚辰,公及戎盟于唐。
  \item 冬,十月,伯姬歸于紀。
  \item 十有二月,乙卯,夫人子氏薨。鄭人伐衛\footnote{公孫滑之亂也}。
  \end{enumerate} \tabularnewline\hline
  三年 & -720 & \begin{enumerate}
    \tiny
  \item 三月,庚戌,天王\footnote{周平王也。}崩。
  \item 
  \end{enumerate} \tabularnewline\hline
  四年 & -719 & \tabularnewline\hline
  五年 & -718 & \tabularnewline\hline
  六年 & -717 & \tabularnewline\hline
  七年 & -716 & \tabularnewline\hline
  八年 & -715 & \tabularnewline\hline
  九年 & -714 & \tabularnewline\hline
  十年 & -713 & \tabularnewline\hline
  十一年 & -712 & \tabularnewline
  \bottomrule
\end{longtable}



%%% Local Variables:
%%% mode: latex
%%% TeX-engine: xetex
%%% TeX-master: "../../Main"
%%% End:
