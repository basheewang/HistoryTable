%% -*- coding: utf-8 -*-
%% Time-stamp: <Chen Wang: 2019-12-26 22:36:59>

\subsection{隐公{\tiny(BC722-BC712)}}

\subsubsection{生平}

魯隱公(?-前712年),姬姓,名息姑,魯国第十四代国君,前722年-前712年在位。魯惠公之子,生母是声子。傳世的魯國史書《春秋》及其三傳的記事都是從魯隱公開始的。

公元前722年,魯惠公死,嫡妻所生的公子軌當時還年幼,所以國人共立息姑攝政,行君事。在位十一年,隱公始終牢記自己是攝政行君事,一心等待公子軌長大,把國君的位置禪讓給他。因此,當公子翬提出請求殺死公子軌時,隱公断然拒絶;結果公子翬怕消息走漏,反而先跟公子軌合作,把隱公刺殺而死,公子軌即位,是為魯桓公。

隱公行君事期間,重視政治、外交,魯國國力較強。隱公除了到棠地觀人捕魚,被認為不合于禮法之外,處理政事、軍事都還比較謹慎公正,與鄰國修好,所以周圍小國如滕國、薛國等國都到魯國朝拜。與鄭國、齊國等強國也結好。

在位期間的卿為公子翬、无骇、公子益师、公子彄、挟、公子豫。

\subsubsection{年表}

% \centering
\begin{longtable}{|>{\centering\scriptsize}m{2em}|>{\centering\scriptsize}m{1.3em}|>{\centering}m{8.8em}|}
  % \caption{秦王政}\\
  \toprule
  \SimHei \normalsize 年数 & \SimHei \scriptsize 公元 & \SimHei 大事件 \tabularnewline
  % \midrule
  \endfirsthead
  \toprule
  \SimHei \normalsize 年数 & \SimHei \scriptsize 公元 & \SimHei 大事件 \tabularnewline
  \midrule
  \endhead
  \midrule
  元年 & -722 & \begin{enumerate}
    \tiny
  \item 三月,公及邾儀父盟于蔑\footnote{公及邾儀父盟于蔑,邾子克也,未王命,故不書爵,曰儀父,貴之也,公攝位,而欲求好於邾,故為蔑之盟。}。
  \item 夏,五月,鄭伯克段于鄢。
  \item 秋,七月,天王使宰咺來歸惠公仲子之賵。
  \item 九月,及宋人盟于宿。
  \end{enumerate} \tabularnewline\hline
  二年 & -721 & \tabularnewline\hline
  三年 & -720 & \tabularnewline\hline
  四年 & -719 & \tabularnewline\hline
  五年 & -718 & \tabularnewline\hline
  六年 & -717 & \tabularnewline\hline
  七年 & -716 & \tabularnewline\hline
  八年 & -715 & \tabularnewline\hline
  九年 & -714 & \tabularnewline\hline
  十年 & -713 & \tabularnewline\hline
  十一年 & -712 & \tabularnewline
  \bottomrule
\end{longtable}

%%% Local Variables:
%%% mode: latex
%%% TeX-engine: xetex
%%% TeX-master: "../../Main"
%%% End:
