%% -*- coding: utf-8 -*-
%% Time-stamp: <Chen Wang: 2018-07-08 21:41:54>

\chapter{秦朝}

\section{秦王政}

% \centering
\begin{longtable}{|>{\centering}m{2em}|>{\centering}m{2em}|>{\centering}m{8.3em}|}
  % \caption{秦王政}\\
  \toprule
  \SimHei 年数 & \SimHei 公元 & \SimHei 大事件 \tabularnewline
  % \midrule
  \endfirsthead
  \toprule
  \SimHei 年数 & \SimHei 公元 & \SimHei 大事件 \tabularnewline
  \midrule
  \endhead
  \midrule
  一年 & -246 & \begin{enumerate}
    \tiny
  \item 秦王\CJKunderline{政}即位。
  \item 韩国人\CJKunderline{郑国}\footnote{郑国,战国时期韩国卓越的水利专家,出生于韩国都城新郑(现在河南省新郑市)。成年后,郑国曾任韩国管理水利事务的水工(官名),参与过治理荥泽水患以及整修鸿沟之渠等水利工程。后来被韩王派去秦国修建水利工事,从而“疲秦”,而郑国渠修建之后,关中成为天下粮仓,赢得了“天府之国”的美名。虽然郑国作为间谍不成功,但是作为一名卓越的水利专家,治理水患,改变了关中农业区的面貌,使八百里秦川成为富饶之乡。郑国渠和都江堰、灵渠并称为秦代三大水利工程。}始建郑国渠。
  \item 晋阳被秦所占领。
  \item \CJKunderline{蒙骜}\footnote{蒙骜(?—公元前240年),《战国策》作蒙傲,战国末期秦国著名将领。蒙骜本是齐国人,后来投靠秦国,官至上卿。蒙骜历仕秦昭襄王、秦孝文王、秦庄襄王、秦始皇四朝,数次率军出征,屡立战功。先后夺取韩国十余座城池、赵国三十余座城池、魏国五十余座城池,使秦国得以设立三川郡和东郡,并让秦国疆域与齐国相接,对韩国、魏国形成三面包围之势,为日后秦始皇统一六国打下坚定的基础。公元前240年,蒙骜去世,时年七十多岁。其子蒙武、其孙蒙恬、蒙毅都是秦国名将。}击定之。
  \end{enumerate} \tabularnewline\hline
  二年 & -245 & \begin{enumerate}
    \tiny
  \item 赵孝成王逝世,悼襄王即位。
  \item \CJKunderline{廉颇}攻繁阳。
  \item 秦将\CJKunderline{麃公}\footnote{麃(biāo)公,姓名、生卒年不详,战国时期秦国的将军。秦王政元年(公元前246年),秦王政即位,麃公与蒙骜、王齮同为将军。秦王政二年(公元前245年),麃公率军攻打卷城,斩首三万人。}攻伐魏国卷地(今河南原阳西北),斩首三万。
  \item 吕不韦得河间十城。
  \item 鲁仲连\footnote{鲁仲连(约前305~前245)战国时名士。亦称鲁连。今茌平人。善于出谋划策,常周游各国,为其排难解纷。赵孝王九年,秦军围困赵国国都邯郸。迫于压力,魏王派使臣劝赵王尊秦为帝,赵王犹豫不决。鲁仲连以利害说赵、魏两国联合抗秦。两国接受其主张,秦军以此撤军。20余年后,燕将攻占齐国的聊城。齐派田单收复聊城却久攻不下,双方损兵折将,死伤严重。鲁仲连闻之赶来,书写了一封义正辞言的书信,射入城中,燕将读后,忧虑、惧怕,遂拔剑自刎,于是齐军轻而易举攻下聊城。赵、齐诸国大臣皆欲奏上为其封官嘉赏。他一一推辞,退而隐居。《汉书·艺文志》载有《鲁仲连子》14篇,今佚。}逝世。
  \end{enumerate} \tabularnewline\hline
  三年 & -244 & \begin{enumerate}
    \tiny
  \item 赵边将\CJKunderline{李牧}\footnote{李牧(?-公元前229年),嬴姓,李氏,名牧,战国时期赵国柏仁(今河北省邢台市隆尧县)人,战国时期的赵国名将、军事家,与白起、王翦、廉颇并称“战国四大名将”。战国末期,李牧是赵国赖以支撑危局的唯一良将,素有“李牧死,赵国亡”之称。}率军大规模反击匈奴,斩杀匈奴10余万骑兵。
  \item 燕王\CJKunderline{喜}使太子\CJKunderline{丹}入质于秦。
  \item \CJKunderline{蒙骜}攻韩十三城,攻魏氏畼、有诡。
  \end{enumerate} \tabularnewline\hline

  \bottomrule
\end{longtable}

%%% Local Variables:
%%% mode: latex
%%% TeX-engine: xetex
%%% TeX-master: "Main"
%%% End:
