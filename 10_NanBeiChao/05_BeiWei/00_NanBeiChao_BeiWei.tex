%% -*- coding: utf-8 -*-
%% Time-stamp: <Chen Wang: 2019-12-23 15:22:02>


\section{北魏\tiny(386-534)}

\subsection{简介}

北魏(386年-534年)是北朝時期的第一個朝代,由鲜卑人拓跋珪所建立,定都平城(遗址在今山西省大同市)。439年,魏太武帝统一华北,与南方的汉人政权对峙。

494年,魏孝文帝迁都洛阳。495年,孝文帝下诏首先在宫廷中禁止包括鲜卑语在内的诸北语,改说汉语,但对三十岁以上的人有所宽限。496年,孝文帝诏令鲜卑八大贵族全部改为汉姓,并将皇族姓氏拓跋改为元姓。534年,北魏分裂為被高歡掌控的東魏(都邺城)與被宇文泰掌控的西魏(都長安)。东魏武定八年(550年),高洋废魏孝静帝,代东魏自立,建立北齐。西魏于恭帝三年(557年)魏恭帝被权臣宇文护逼迫禅位于堂弟宇文觉,建立北周,172年的元魏历史才正式宣告结束。

拓跋氏自称是黄帝后裔,黄帝发源地为战国时魏国所在,又“魏”有美好之意,故以此名国号。以其领土位于中国北方,又是北朝的第一个政权,故史称“北魏”。为别于此前三国時期的曹魏政权,某些史书因此别称为“后魏”,但由于史学界不称曹魏为“前魏”,故“后魏”之称很少使用。又以其王室姓拓跋,后改姓元,故又别称拓跋魏(东魏和西魏虽然姓拓跋,但是多數史学家并不如此称呼这两个政权)、元魏。

在公元四世纪初,拓跋鲜卑在今山西北部和今內蒙古等地建立代國。376年被前秦所吞并。淝水之战后,前秦统治瓦解。386年,拓跋珪即代王位,重建代国。同年四月,改国号为魏。398年(天興元年)建都平城,次年称帝。即為魏道武帝。

396年道武帝改元皇兴,率40万大军征讨后燕,一举攻下中山、信度、邺城,平定中原。经过明元帝时代的承平岁月,到北魏太武帝时,于427年攻破夏国首都统万城,428年占领安定,逐走赫连夏后主赫连定。436年攻破和龙,灭亡北燕冯氏。太延五年(公元439年)吞并北凉沮渠氏。442年西凉残余势力李宝投降北魏。443年仇池杨保炽投降北魏。至此北魏完成了兼併华北地区和北方,這時华南地区和南方早已是劉宋,南北各自为政,形成互不隶属的對峙之局。

在統一華北以前,北魏就有多次與南朝政權在黃淮下游交手的經驗。顯祖獻文帝皇興三年(469),北魏上黨公、征南大將軍慕容白曜攻下南朝宋所屬的青州治所東陽城,至此以後,現今山東半島,又屬黃淮下游古稱青齊的地區(《尚書‧禹貢》稱「海岱之地」)就歸北魏所管,並被割劃為青州、齊州、濟州、光州等區。

早在馮太后掌政時代,馮太后已推行了一系列措施建立國家規模,如在太和九年(485年)推行均田制,把之前因为戰亂而遺下的無主荒地按制度分給存活的農民,一部分可永久擁有,一部分則身死後交還公家。又施行租調制,農民按制度上數字,定期向朝廷納稅。

孝文帝親政後更在文化上開始修整,在风雨飘摇之中的背景下為了维持和巩固政权,進行了許多大刀阔斧的改革,即後世之所謂孝文漢化,其舉措大略如下:

一、遷洛陽:孝文帝以舊都平城(今山西省大同市)為用武之地,非可文治,而洛邑為歷史名都,物富民豐,交通便利,便於經略海內,控制中原,魏太和十七年(493年),以南伐為名,進駐河洛,定為京師。遷都洛阳後,戎裝以外,官民皆著漢服。

二、改漢姓:《魏書‧官氏志》記載了一百一十八個胡人改姓的例子,如皇族拓跋氏改元姓、步六孤改陸姓、賀賴氏改賀姓、獨孤改姓劉。

三、斷胡語:凡三十歲以下官員一律使用漢語,如果仍用鮮卑語,即降爵黜官。

四、通婚姻:鼓勵與漢族世家通婚,並從己身開始迎娶漢族士族女子。

五、重教育:祀孔子,尊儒教,尋古書,設立太學、小學。

自此胡漢界線开始逐漸消弭,對当时和后世發展意義非凡。

第八任皇帝魏宣武帝元恪立他的儿子元诩当太子时,没有按旧制处死太子的母亲胡贵嫔,導致外戚及士族掌權。元诩即位后,胡贵嫔为皇太后,後六鎮之亂爆發。胡党毒死元诩,立元钊,大将尔朱荣趁势讨伐,立元子攸,在河阴之变之后掌控朝政。元颢、元悦等宗室都因河阴之变而南下投靠梁朝。

529年,梁武帝派陈庆之攻陷洛阳,立元颢为帝。陈庆之目睹洛阳的衣冠、礼仪、人才不输南朝,心生感慨。元颢政权不久为尔朱荣所败。

孝莊帝元子攸不能容忍尔朱荣跋扈,在尔朱荣回朝后设计铲除之,梁武帝亦于530年趁机派兵拥立元悦为帝。但不久后孝庄帝就被尔朱家族所杀害,元悦见状亦放弃称帝而南归。尔朱氏立元晔为帝,又废元晔改立元恭。

尔朱家族大将高欢倒戈,立元朗为帝,讨伐尔朱家族,并取得胜利。高欢在532年以元恭为尔朱氏所立、元朗世系疏远为由,皆废黜,立元修为帝。曾为皇帝的元晔、元恭、元朗及北返的元悦皆被元修所杀。元修不能容忍高欢掌握朝政,在534年投奔长安的宇文泰,而宇文泰杀元修,另立元宝炬为帝,建都长安,史称“西魏”。

高欢另立元善见为帝,建都邺城(今河北临漳),史称“东魏”。北魏就此分裂。

北魏的宮廷為了避免外戚干政,實施殘酷的子貴母死制度,即後宮女性只要生下皇子就得被賜死,以避免母以子貴的情況發生。但幼子還是需要人照顧,因此就有所謂的保太后,即以太子的保母在太子繼位後成為皇太后。北魏有三種皇太后,一種是皇帝的生母,另一種是皇帝的保姆,還有一種是未曾替前任皇帝生皇子因而存活的皇后。如北魏献文帝乃由漢人女子李贵人所生,然李贵人在生下献文帝以後即被賜死,由身為太后的馮氏所養大。献文帝曾求當時當權的馮太后廢除舊法,但被拒絕。後來直到篤信佛教的北魏宣武帝,才終於取消子貴母死,但他卻導致北魏的外戚爭權,最終導致北魏滅亡及分裂。

北魏文成帝拓跋濬是北魏太武帝的孙子,其父拓跋晃没有做过皇帝,文成帝也并非以储君身份登基,故其生母郁久闾氏未曾被赐死,在文成帝登基之初尚在人世,但不久后,身为在位皇帝之母的她也因“子贵母死”制度所累而死。

魏孝文帝改革之前,北魏的税收由部落贡纳、牧民的畜牧税为以及一般农民的租调为主,其中农民的租调为最大收入。北魏规定租调税收为“户调帛二匹、絮二斤、丝一斤、粟二十石;又入帛一匹二丈,委之州库,以供调外之费。”。不过这是个一般办法,政府需要的时候可以增加征收物品的种类和数量。租调是按户收取的,户的大小没有限制,孝文帝改革之前,三五十家组成一户的情形很普遍。除了这种一般性税收外,政府经常因为战争而加开新税,官吏因为没有官俸,也常常以各种借口征税,给百姓带来很大的负担。

徭役方面,兵役方面由鲜卑人担任,因而兵役较轻。而力役的情况因为缺少史料,无法得知。只知道,为政府工作的工、杂役非常多。他们被编为隶户、军户、营户、府户、绫罗户、乐户等等。

孝文帝改革后,为了给官僚机构提供俸禄,以减少官吏欺压百姓。提高了税率,魏孝文帝定每户增调帛三匹、谷二斛九斗,充百官俸禄。又在太和九年(485年)实行均田制,办法大致有四项:十五岁以上的男丁和妇人均可授田,男丁授露田四十亩,妇人二十亩,授田视轮休需要加倍或再加倍。如果有牛一头则授田三十亩,最多四头牛,多出的不授田。老少病残或者缺乏男丁的家庭十一岁以上和有病者均授予半夫之田。奴婢一样按照男丁和妇人的标准授田。授田不准买卖,年老或身死还田,但七十以上授田者不必归还。男丁授桑田20亩。桑田不必还给国家,可传给子孙,也可以可卖出多于20亩的部分,也可买桑田补足20亩。产麻地男子授麻田10亩,妇人50亩,年老及身死后还田。多余土地可以借给农民耕种,政府严格控制农民迁徙,只允许迁往空荒地区。规定驻地长官在所在地给予公田,刺史十五顷,太守十顷,治中别驾八顷,县令郡丞六顷,不许买卖。

政府在均田制的基础上重新规定了税收制度,一夫一妻应缴纳的租调为:“其民调,一夫一妇帛一匹,粟二石。民年十五以上未娶者,四人出一夫一妇之调;奴任耕、婢任织者八口当未娶者四;耕牛二十头当奴婢八。其麻布之乡,一夫一妇布一匹,下至牛,以此为降。”

北魏兵民分开,兵用于打仗,民从事耕桑。而兵主要由鲜卑及其他少数民族组成,农业主要由汉人从事。兵民之分也就是胡汉之分,也是胡汉分治的体现。

而士兵里面也分两种,一种是鲜卑兵,另外一种是非鲜卑兵。

鲜卑兵由代北部落的鲜卑人组成,主要担任北魏的禁旅和边防六镇的士兵。这种兵带贵族性质,地位颇高,但在魏文帝汉化之后有所改变。

非鲜卑兵中,以高车兵最为重要,禁军和六镇边兵都有高车人。此外还有部分少数民族和汉人军队。

北魏经历了游牧部落联盟而迅速转移到国家的历史,拓跋鲜卑人有自己的语言而没有文字。北魏时期的主要宗教是佛教、道教和琐罗亚斯德教,其中最重要的是佛教,僧尼的人数曾发展到二百多万。北魏道教,主要是经过寇谦之改良的天师道。当时佛道两家的斗争十分激烈,太武帝拓跋燾曾经大舉灭佛。琐罗亚斯德教在中国称为祆教或拜火教,主神被称为“胡天”,主要在入华的粟特人当中传播。孝文帝在平城(大同)开凿了云岗石窟。

北魏大部分時期,由於國家及私人贊助,佛教藝術十分興盛。雲崗佛教石窟約興建於西元四六〇年,由上千位工匠歷時約三十五年後完工,洞窟內有雕塑及與繪畫。之後,北魏孝文帝亦於龍門興建石窟。雲崗石窟的佛像屬較靜態的罽賓風格,龍門的造像形式則較流線飄逸,開始展現中國風格的影響。北魏的陪葬陶器亦受到佛教影響,強調「正面性」(frontality) 及對稱。

%% -*- coding: utf-8 -*-
%% Time-stamp: <Chen Wang: 2021-11-01 15:09:15>

\subsection{道武帝拓跋珪\tiny(386-409)}

\subsubsection{生平}

魏道武帝拓跋珪(371年-409年11月6日),又名涉珪、什翼圭、翼圭、開,北魏开国皇帝,代王拓跋什翼犍之孙,獻明帝拓跋寔和贺夫人之子。

拓跋珪出生于371年8月4日。376年,前秦滅代國,拓跋珪將要被強遷至秦都長安,但代王左長史燕鳳以拓跋珪年幼,力勸前秦天王苻堅讓拓跋珪留在部中,稱待拓跋珪長大後為首領,會念及苻堅施恩給代國。苻堅同意,拓跋珪得以留下。其時,代國舊部由劉庫仁及劉衞辰分掌,拓跋珪母賀氏帶拓跋珪、拓跋儀及拓跋觚從賀蘭部遷至獨孤部,與南部大人長孫嵩等人同屬劉庫仁統領。劉庫仁本亦為南部大人,拓跋珪等人到後仍盡忠侍奉他們,並沒有因代國滅亡、自己改受前秦官位而變節,又招撫接納離散的部人,甚有恩信。

383年,苻堅於淝水之戰中戰敗,其後國中大亂,劉庫仁助秦軍對抗後燕,但於次年遭慕輿文夜襲殺害,其弟劉頭眷代領其眾。385年,劉庫仁之子劉顯殺頭眷自立,又想要殺拓跋珪。劉顯弟劉亢埿的妻子是拓跋珪的姑姑,並將劉顯的意圖告訴賀氏。劉顯謀主梁六眷是拓跋什翼犍的甥子,也派部人穆崇、奚牧將此事密報拓跋珪。賀氏於是約劉顯飲酒,將其灌醉,讓拓跋珪與舊臣長孫犍、元他等人乘夜逃至賀蘭部。不久,劉顯部中內亂,賀氏得以到賀蘭部與拓跋珪等會合。但其時賀氏弟賀染干忌憚拓跋珪得人心,曾試圖殺害他,但都因尉古真告密及賀氏出面而失敗。而拓跋珪的堂曾祖父拓跋紇羅及拓跋建就勸賀蘭部首領賀訥推拓跋珪為主。

登國元年正月六日(386年2月20日),拓跋珪得到以賀蘭部為首的諸部支持在牛川大會諸部,召開部落大會,即位為代王,年號登國。拓跋珪任用賢能,勵精圖治,重興代國。即位不久,便移都代國原都盛樂,並推動農業,讓人民休養生息。同年四月,改稱魏王,稱國號為魏,史稱北魏。

北魏建立時四週有強敵環伺,北有賀蘭部、南有獨孤部、東有庫莫奚部、西邊在河套一帶有匈奴鐵弗部、陰山以北為柔然部和高車部、太行山以東為慕容垂建立的後燕及以西的慕容永統治的西燕。因為叔父拓跋窟咄為了爭位與劉顯勾結,企圖取拓跋珪而代之形成內部不穩,于桓等人意圖殺害拓跋珪以響應窟咄,莫題等人亦與窟咄通訊。拓跋珪殺死于桓等五人,赦免莫題等七姓,但都因恐懼內亂而往依賀蘭部,借陰山作屏障防守,又派人向後燕求援。

同年十一月,拓跋窟咄逼近,部眾惶恐不安。慕容垂之子慕容麟帶領的後燕援軍此時仍未到,於是先讓北魏使者安同先回去,讓魏人知燕軍已在附近,穩定人心。拓跋珪於是領兵會合後燕援軍,在高柳大敗拓跋窟咄。窟咄帶領殘兵敗將西逃,依附鐵弗部,被鐵弗部首領劉衞辰殺死,拓跋珪接收其部眾。十二月,後燕任命拓跋珪為西單于,封上谷王,但拓跋珪不受。

次年,拓跋珪與後燕聯手擊敗劉顯,逼劉顯出奔西燕。六月,拓跋珪又於弱落水大敗庫莫奚部;七月再擊敗來攻的庫莫奚。登國四年(388年),拓跋珪大破高車諸部。登國五年(389年),拓跋珪又西征高車袁紇部,並在鹿渾海大敗對方,俘獲人口及牲畜共計二十多萬。不久更聯同慕容麟所率的後燕軍進攻賀蘭部、紇突隣部及紇奚部,後兩者向北魏請降。七月,賀蘭部遭鐵弗部攻擊,賀訥於是向北魏投降求援,拓跋珪於是領兵去救援,擊退鐵弗,並將賀訥等人遷至東界。

拓跋珪進擊高車諸部,唯獨柔然不肯降魏,遂於登國七年(391年)進攻柔然。柔然當時率眾退避,拓跋珪追擊,軍糧用盡後以備乘戰馬作軍糧,終在南牀山追及,並俘獲其一半部眾。接著拓跋珪繼續派兵追擊餘部,逼令首領縕紇提投降。同年,拓跋珪進攻鐵弗,直攻代來城,擒獲直力鞮,衞辰被部下殺害。拓跋珪更盡誅劉衞辰宗族共五千多人,將屍體丟在黃河中。此戰後,黃河以南諸部都向北魏投降。北魏至此亦已擊敗大部份強鄰,國力亦大增。

北魏與後燕皆是386年建立,後燕強而北魏弱,拓跋珪與後燕結好,而北魏開國之初的內亂,後燕亦曾出兵支援拓跋珪,每年兩回亦派使者往來。登国六年(391年),賀蘭部內亂,賀染干和賀訥互相攻擊,拓跋珪亦自請為響導,請後燕出兵討伐。但同年,后燕將來使拓跋觚扣留,以向北魏求名马。拓跋珪拒絕,拓跋觚亦一直遭扣留,此后两国关系惡化。北魏轉而聯結西燕对付後燕。但後燕帝慕容垂於登国九年(394年)六月出兵進攻西燕,圍攻長子,西燕帝慕容永曾向北魏求援,拓跋珪遂派陳留公拓跋虔及庾岳救援西燕,可是援軍尚未趕到,長子就失陷。慕容永及其公卿大將三十多人都被誅殺,西燕滅亡。華北一帶就剩下北魏與後燕两国互相對峙。

登国十年(395年)北魏侵逼後燕附塞諸部,慕容垂就於同年五月派其太子慕容寶伐魏。拓跋珪知大軍前來,率眾到河西避戰。燕軍於七月到五原後收降魏別部三萬多家人,又收穄田穀物及造船打算渡河進攻。拓跋珪亦進軍河邊,與燕軍對峙。北魏一方面派許謙向後秦請求援兵,一面卻派兵堵截燕軍與後燕都城中山的道路,並抓住取道去前線的燕國使者。因著慕容垂在出兵時已經患病,而堵截道路令慕容寶久久都不知道國內消息,拓跋珪於是逼令抓到的使者向燕軍謊稱慕容垂的死訊,成功動搖燕軍將士的軍心。燕魏兩軍自九月起隔河對峙至十月,燕軍終因內亂而被逼燒船撤退。其時黃河河水未結,魏軍未能及時渡河追擊。但次月大風令河面結冰後,拓跋珪即下令渡河並派二萬多精騎追擊燕軍。魏军在参合陂打败燕军,俘獲大量燕軍將士及官員,拓跋珪除了選用有才的如賈閏、賈彜等人留下外,將其他官員都送回後燕,但同時將燕兵都坑殺。史称參合陂之戰。

登国十一年(396年)三月,慕容垂率軍再度伐魏,攻陷平城(今山西大同市),留守平城的拓跋虔戰死,守城的三萬餘家部落皆被俘。接著慕容垂更派慕容寶等進逼拓跋珪。拓跋珪此時十分驚懼,打算離開盛樂避兵,而諸部因驍勇善戰的拓跋虔戰死,亦有異心,令拓跋珪不知所措。可是慕容垂因見參合陂堆積如山的燕兵屍體而發病,被逼退兵,並病逝于上谷。同年七月,拓跋珪建天子旌旗,並改元皇始,並正式圖取後燕所佔的中原土地。

皇始元年(396年)八月,拓跋珪就大舉伐燕,親率四十多萬大軍南出馬邑,越過句注南攻後燕并州,同時又命封真率偏師進攻後燕幽州。九月,魏軍進至晉陽,守城的慕容農出戰但大敗,晉陽城守將此時叛燕逼使慕容農率眾東走。長孫肥率眾追擊,在潞川追上,慕容農妻兒被擄,只能與三騎逃回中山。北魏遂奪取後燕并州 ( 今山西地區 )之地,並置官員治理當地。

隨後,拓跋珪命于栗磾及公孫蘭等暗中開通昔日韓信在井陘用過的路,並在同年十月,越過太行山率軍取道該路進攻後燕京師中山城 ( 河北省定縣 )。其時燕軍決意嬰城自守,打持久戰,於是拓跋珪在攻下常山後,其東各郡縣的官員不是棄城就是投降,北魏於是輕易地得到中原大部分郡縣歸附,僅餘中山城、鄴城及信都城三城仍然拒守。拓跋珪於是兵分三路分攻三城:自攻中山,拓跋儀攻鄴及王建、李栗攻信都。然而,拓跋珪在攻中山城時遭燕軍力拒,於是暫時放棄中山城,改而南取其餘二城。

皇始二年(397年)正月,拓跋珪加入進攻信都城,終於逼得守將慕容鳳棄城出走,但其時慕容德卻成功離間進攻鄴城的拓跋儀及賀賴盧,令他們退兵,並乘機從後追擊,大破魏軍。

上一年,為拓跋珪憎惡的魏將沒根自疑而叛魏投燕,其侄兒醜提恐怕會被株連,於是決定自并州率部回北魏後方作亂。拓跋珪見內亂起,於是自後燕求和,但慕容寶卻意圖乘此機反擊,拒絕之餘更派步兵十二萬及騎兵三萬七千出屯柏肆,在滹沱水以北阻擊魏軍。魏軍在滹沱水南岸設營,燕軍於是乘夜渡水進攻,以萬餘兵突襲魏營,並乘風勢放火。魏軍此時大亂,拓跋珪慌忙起來棄營逃跑,僅而避過攻到其帳下的燕將乞特真。可是,燕軍此時卻無故自亂,互相攻擊,拓跋珪在營外見到,就擊鼓收拾餘眾,集結好後進攻營內燕軍,並乘勢進攻營北作支援的慕容寶軍,逼使慕容寶退回北岸。此戰後,燕軍士氣大降,而魏軍卻已重整。拓跋珪乘慕容寶撤退的機會追擊,屢敗燕軍。慕容寶恐懼下更拋下大軍率二萬騎兵速返中山;又怕被追上,命令士兵拋棄戰衣及兵器輕裝撤還。其時大量燕兵因大風雪而凍死,很多後燕朝臣及兵將都被俘或投降。

三月,慕容寶向拓跋珪求和,並說要送還拓跋觚,並割讓常山以西土地。拓跋珪已答允,但慕容寶卻反悔,拓跋珪於是進圍中山。最終慕容寶等人棄中山城出走,拓跋珪原本打算在該晚入城,王建則以士兵會乘夜盜取城中財寶為由勸阻,拓跋珪於是等到日出才入城。可是慕容詳卻趁機自立為主,閉門拒守,拓跋珪試圖強攻但攻了幾日都不果,於是試圖勸降,可是城中軍民卻表示擔心會有昔日在參合陂被殺的燕降卒一樣的下場,所以堅守到最後。拓跋珪想起當日勸他殺俘的正是王建,導致現在難取中山,於是向其吐口水。至五月,拓跋珪撤圍,到河間補充軍糧。在圍攻中山的同時,拓跋珪派庾岳率兵討平國內叛變的賀蘭部、紇鄰部及紇奚部,成功解決內亂。

九月,時據中山的慕容麟因飢荒而出據新市,拓跋珪於是主動進攻,並在次月於義臺大破慕容麟。慕容麟出走後,拓跋珪入據中山。皇始三年(398年),鄴城也因慕容德棄守而落入魏軍手中,拓跋珪於鄴置行臺後回到中山,並打算回盛樂,於是修治由望都至代的直道,設中山行臺以防變亂,又下令強遷新佔之山東六州官民和外族人士到代郡充實人口。

皇始三年(398年)七月,拓跋珪迁都平城,營建宮殿、宗廟、社稷。同年十二月二日(399年1月24日),改年號天興,即皇帝位。

天興二年(399年)正月,拓跋珪即位後不久便北巡,並分三道進攻高車各部,至二月會師時大破高車三十餘部,另拓跋儀又以三萬騎兵攻破高車殘餘的七部,皆大有所獲。同年三月二十日,拓跋珪派遣建義將軍庾真及越騎校尉奚斤進攻北方的庫狄部及宥連部,將他們擊敗並逼令庫狄部的沓亦干歸附。庾真等軍接著又擊破侯莫陳部,俘獲十多萬頭牲畜並一直追擊到大峨谷。

拓跋珪曾派賀狄干向後秦獻馬一千匹並請結婚姻,不過其時拓跋珪已立慕容氏為皇后,故此後秦君主姚興拒絕了婚姻要求並強留賀狄干,兩國遂有嫌隙。天興五年(402年)後秦高平公沒弈干和屬部黜弗及素古延分別遭北魏常山王拓跋遵及材官將軍和突領兵進侵,其中拓跋遵軍更曾追擊至瓦亭,另魏平陽太守貮塵又進攻秦河東之地。這些行動威脅到秦都長安,關中各城白天都閉著城門,亦令得後秦準備進攻北魏。拓跋珪亦在該年舉行閱兵,又命并州各郡送穀物到平陽郡的乾壁儲存以防備秦軍進攻。

天興五年(402年)六月,後秦派軍進攻北魏,攻陷了乾壁。拓跋珪則派毗陵王拓跋順及豫州刺史長孫肥為前鋒迎擊,自率大軍在後。八月,拓跋珪至永安(今山西霍縣東北),秦將姚平派二百精騎視察魏軍但盡數被擒,於是撤走,但在柴壁遭拓跋珪追上,於是據守柴壁。拓跋珪圍困柴壁,而姚興則率軍來救援姚平,並要據天渡運糧給姚平。

拓跋珪接著增厚包圍圈,防止姚平突圍或姚興強攻,另又聽從安同所言,築浮橋渡汾河,並在西岸築圍拒秦軍,引秦軍走汾東的蒙阬。姚興到後果走蒙阬,遭拓跋珪擊敗。拓跋珪又派兵各據險要,阻止秦軍接近柴壁。至十月,姚平糧盡突圍但失敗,於是率部投水自殺,拓跋珪更派擅長游泳的人下水打撈自殺者,又生擒狄伯支等四十多名後秦官員,二萬多名士兵亦束手就擒。姚興雖然能夠與姚平遙相呼應,但無力救援,柴壁敗後多次派人請和,但拓跋珪不准,反而要進攻蒲阪,只是當時姚緒堅守不戰,且早於394年背魏再興的柔然汗國要攻魏,逼使拓跋珪撤兵。

拓跋珪晚年因服食寒食散,剛愎自用、猜忌多疑,更常因想起昔日一點不滿就要誅殺大臣。大臣們大都惶恐度日,影響辦事能力,以至偷竊等行為十分猖獗。

天賜四年(407年)至天賜六年(409年)間,拓跋珪先後誅殺了司空庾岳、北部大人賀狄干兄弟及高邑公莫題父子。往日曾與穆崇共謀刺殺拓跋珪的拓跋儀雖然因拓跋珪念其功勳而沒被追究,但眼見拓跋珪殺害大臣,於是自疑逃亡,但還是被追兵抓住,並被賜死。

天赐六年冬十月戊辰(409年11月6日),次子拓跋紹母賀夫人有过失,拓跋珪幽禁她於宮中,准备处死。到黃昏時仍未決。賀氏秘密向拓跋紹求救。拓跋紹與宮中守兵及宦官串通,當晚带人翻墙入宮,刺殺拓跋珪。拓跋珪在拓跋紹來到時驚醒,試圖找武器反擊但不果,終為其所殺,享年三十九歲。

其子拓跋嗣登位後,於永興二年(410年)諡拓跋珪為宣武皇帝,廟號烈祖,泰常五年(420年)才改諡為道武皇帝,太和十五年(491年)改廟號為太祖。

北齊史官魏收於《魏書》的「史臣曰」評論說:「晉氏崩離,戎羯乘釁,僭偽紛糾,犲狼競馳。太祖顯晦安危之中,屈伸潛躍之際,驅率遺黎,奮其靈武,克剪方難,遂啟中原,朝拱人神,顯登皇極。雖冠履不暇,栖遑外土,而制作經謨,咸存長世。所謂大人利見,百姓與能,抑不世之神武也。而屯厄有期,禍生非慮,將人事不足,豈天實為之。嗚呼!」

唐代某貴族「公子」與世族虞世南的對話:「公子曰:『魏之道武,始立大號,觀其器用,足為一時之杰乎?』先生曰:『道武經略之志,將立霸階,而才不逮也。末年沈痼,加以精虐,不能任下,禍及方悟,不亦晚乎!』;公子曰:『魏之太祖、太武,孰與為輩?』先生曰:『太祖、太武,俱有異人之姿,故能辟土擒敵,窺覦江外。然善戰好殺,暴桀雄武,稟崆峒之氣焉。至於安忍誅殘,石季龍之儔也。』」

北宋司馬光評論說:「後魏之先,世居朔野,有國久矣。道武帝乘燕氏之衰,悉舉引弓之眾,以馮陵中夏;馬首所向,無不望風奔潰。南取并州,東舉幽、冀;兵不留行,而數千里之地定矣!」

\subsubsection{登国}

\begin{longtable}{|>{\centering\scriptsize}m{2em}|>{\centering\scriptsize}m{1.3em}|>{\centering}m{8.8em}|}
  % \caption{秦王政}\
  \toprule
  \SimHei \normalsize 年数 & \SimHei \scriptsize 公元 & \SimHei 大事件 \tabularnewline
  % \midrule
  \endfirsthead
  \toprule
  \SimHei \normalsize 年数 & \SimHei \scriptsize 公元 & \SimHei 大事件 \tabularnewline
  \midrule
  \endhead
  \midrule
  元年 & 386 & \tabularnewline\hline
  二年 & 387 & \tabularnewline\hline
  三年 & 388 & \tabularnewline\hline
  四年 & 389 & \tabularnewline\hline
  五年 & 390 & \tabularnewline\hline
  六年 & 391 & \tabularnewline\hline
  七年 & 392 & \tabularnewline\hline
  八年 & 393 & \tabularnewline\hline
  九年 & 394 & \tabularnewline\hline
  十年 & 395 & \tabularnewline\hline
  十一年 & 396 & \tabularnewline
  \bottomrule
\end{longtable}

\subsubsection{皇始}

\begin{longtable}{|>{\centering\scriptsize}m{2em}|>{\centering\scriptsize}m{1.3em}|>{\centering}m{8.8em}|}
  % \caption{秦王政}\
  \toprule
  \SimHei \normalsize 年数 & \SimHei \scriptsize 公元 & \SimHei 大事件 \tabularnewline
  % \midrule
  \endfirsthead
  \toprule
  \SimHei \normalsize 年数 & \SimHei \scriptsize 公元 & \SimHei 大事件 \tabularnewline
  \midrule
  \endhead
  \midrule
  元年 & 396 & \tabularnewline\hline
  二年 & 397 & \tabularnewline\hline
  三年 & 398 & \tabularnewline
  \bottomrule
\end{longtable}

\subsubsection{天兴}

\begin{longtable}{|>{\centering\scriptsize}m{2em}|>{\centering\scriptsize}m{1.3em}|>{\centering}m{8.8em}|}
  % \caption{秦王政}\
  \toprule
  \SimHei \normalsize 年数 & \SimHei \scriptsize 公元 & \SimHei 大事件 \tabularnewline
  % \midrule
  \endfirsthead
  \toprule
  \SimHei \normalsize 年数 & \SimHei \scriptsize 公元 & \SimHei 大事件 \tabularnewline
  \midrule
  \endhead
  \midrule
  元年 & 398 & \tabularnewline\hline
  二年 & 399 & \tabularnewline\hline
  三年 & 400 & \tabularnewline\hline
  四年 & 401 & \tabularnewline\hline
  五年 & 402 & \tabularnewline\hline
  六年 & 403 & \tabularnewline\hline
  七年 & 404 & \tabularnewline
  \bottomrule
\end{longtable}

\subsubsection{天赐}

\begin{longtable}{|>{\centering\scriptsize}m{2em}|>{\centering\scriptsize}m{1.3em}|>{\centering}m{8.8em}|}
  % \caption{秦王政}\
  \toprule
  \SimHei \normalsize 年数 & \SimHei \scriptsize 公元 & \SimHei 大事件 \tabularnewline
  % \midrule
  \endfirsthead
  \toprule
  \SimHei \normalsize 年数 & \SimHei \scriptsize 公元 & \SimHei 大事件 \tabularnewline
  \midrule
  \endhead
  \midrule
  元年 & 404 & \tabularnewline\hline
  二年 & 405 & \tabularnewline\hline
  三年 & 406 & \tabularnewline\hline
  四年 & 407 & \tabularnewline\hline
  五年 & 408 & \tabularnewline\hline
  六年 & 409 & \tabularnewline
  \bottomrule
\end{longtable}


%%% Local Variables:
%%% mode: latex
%%% TeX-engine: xetex
%%% TeX-master: "../../Main"
%%% End:

%% -*- coding: utf-8 -*-
%% Time-stamp: <Chen Wang: 2019-12-23 14:52:34>

\subsection{明元帝\tiny(409-423)}

\subsubsection{生平}

魏明元帝拓跋嗣(392年-423年12月24日),鮮卑名木末,北魏第二位皇帝,409年—423年在位。

北魏道武帝拓跋珪長子,登國七年(392年)生于雲中宫,天興六年(403年),封齊王,拜相國,加車騎大將軍。天赐六年十月,拓跋嗣被其父北魏道武帝拓跋珪立為太子,生母劉貴人按子贵母死的制度被道武帝赐死,拓跋嗣知道后悲傷不已,因而被其父北魏道武帝拓跋珪怒斥出宫。

十月十三日(409年11月6日),道武帝拓跋珪被其次子清河王拓跋绍所杀,太子拓跋嗣在宫中衛士的擁戴下殺了拓跋紹,拓跋嗣在同年十月十七日(11月10日)登基,改年号“永興”,為北魏明元帝。

泰常八年(423年),明元帝拓跋嗣進攻刘宋得勝回来,此役稱為南北朝第一次南北戰爭,北魏獲得勝利,攻佔虎牢關,奪取劉宋領土三百里。十一月己巳(12月24日),明元帝因攻戰勞顿成疾而终,享年32岁。

拓跋嗣雖英年早逝,但上承其父北魏開國君主太祖道武帝拓拔珪的武功,後有兒子北魏太武帝拓拔燾滅北方諸國一統北方,媲美五胡十六國初期前秦苻堅統一北方功勳。因此拓拔嗣在北魏開國歷史中具有承先啟後的重要地位。

拓跋嗣谥号为明元皇帝,庙号太宗。

\subsubsection{永兴}

\begin{longtable}{|>{\centering\scriptsize}m{2em}|>{\centering\scriptsize}m{1.3em}|>{\centering}m{8.8em}|}
  % \caption{秦王政}\
  \toprule
  \SimHei \normalsize 年数 & \SimHei \scriptsize 公元 & \SimHei 大事件 \tabularnewline
  % \midrule
  \endfirsthead
  \toprule
  \SimHei \normalsize 年数 & \SimHei \scriptsize 公元 & \SimHei 大事件 \tabularnewline
  \midrule
  \endhead
  \midrule
  元年 & 409 & \tabularnewline\hline
  二年 & 410 & \tabularnewline\hline
  三年 & 411 & \tabularnewline\hline
  四年 & 412 & \tabularnewline\hline
  五年 & 413 & \tabularnewline
  \bottomrule
\end{longtable}

\subsubsection{神瑞}

\begin{longtable}{|>{\centering\scriptsize}m{2em}|>{\centering\scriptsize}m{1.3em}|>{\centering}m{8.8em}|}
  % \caption{秦王政}\
  \toprule
  \SimHei \normalsize 年数 & \SimHei \scriptsize 公元 & \SimHei 大事件 \tabularnewline
  % \midrule
  \endfirsthead
  \toprule
  \SimHei \normalsize 年数 & \SimHei \scriptsize 公元 & \SimHei 大事件 \tabularnewline
  \midrule
  \endhead
  \midrule
  元年 & 414 & \tabularnewline\hline
  二年 & 415 & \tabularnewline\hline
  三年 & 416 & \tabularnewline
  \bottomrule
\end{longtable}

\subsubsection{泰常}

\begin{longtable}{|>{\centering\scriptsize}m{2em}|>{\centering\scriptsize}m{1.3em}|>{\centering}m{8.8em}|}
  % \caption{秦王政}\
  \toprule
  \SimHei \normalsize 年数 & \SimHei \scriptsize 公元 & \SimHei 大事件 \tabularnewline
  % \midrule
  \endfirsthead
  \toprule
  \SimHei \normalsize 年数 & \SimHei \scriptsize 公元 & \SimHei 大事件 \tabularnewline
  \midrule
  \endhead
  \midrule
  元年 & 416 & \tabularnewline\hline
  二年 & 417 & \tabularnewline\hline
  三年 & 418 & \tabularnewline\hline
  四年 & 419 & \tabularnewline\hline
  五年 & 420 & \tabularnewline\hline
  六年 & 421 & \tabularnewline\hline
  七年 & 422 & \tabularnewline\hline
  八年 & 423 & \tabularnewline
  \bottomrule
\end{longtable}


%%% Local Variables:
%%% mode: latex
%%% TeX-engine: xetex
%%% TeX-master: "../../Main"
%%% End:

%% -*- coding: utf-8 -*-
%% Time-stamp: <Chen Wang: 2021-11-01 15:09:27>

\subsection{太武帝拓跋焘\tiny(423-452)}

\subsubsection{生平}

魏太武帝拓跋焘(408年-452年3月11日),鮮卑本名佛狸伐。佛狸是官號,突厥語狼büri或böri的音譯。伐或bäg是官稱,且是魏晉時期鮮卑諸部使用最為廣泛的政治名號。北魏第三位皇帝(423年12月27日—452年3月11日在位),北魏明元帝拓跋嗣的长子,北魏道武帝拓拔珪的長孫,在位28年,谥号太武皇帝。

拓拔焘即位时,只有十六岁,大臣们都拿他当小孩子看。于是拓拔焘决定先整顿吏治,励精图治,令人刮目相看,北魏国力进入鼎盛。427年,拓跋燾在連續兩年突擊統萬城之後,占領胡夏的北部地區(包含首都統萬),並一度攻下關中,胡夏雖遷都至平涼,卻於次年(428年)打敗魏軍並收復關中。北魏在429年北伐柔然大獲全勝之後,趁著柔然近十年都難以恢復的良機,把軍隊主力向南進攻,於430年大敗劉宋與胡夏的聯合攻勢,不但占領胡夏大部分的關隴領土(包含平涼、關中、隴西郡),更在431年從宋軍手中奪回河南四鎮(洛陽、虎牢等),拓拔焘返回首都平城,祭告太廟並舉行盛大的慶功典禮。

撤退到上邽的夏主赫連定,雖於431年滅西秦而稍微挽救了國勢,並意圖再滅北涼以維持胡夏,但卻在432年,被吐谷渾君主慕容慕璝襲擊而俘虜。同年赫連定被送給北魏,拓拔焘將其處死,胡夏亡。436年拓跋燾派軍東征北燕,燕主馮弘在高句麗大軍的保護之下,將首都人民全部東遷高句麗,而魏軍主帥忌憚高軍,坐看燕人東撤;北燕雖然滅亡,但只得空地空城,因此拓跋燾大怒之下處罰了征燕主帥娥清、古弼。439年拓拔焘率大軍圍攻北涼首都姑臧,涼主沮渠牧犍出降,北涼亡。至此,北魏統一華北,与江东的刘宋王朝对峙,形成南北朝的局面。

自前涼张氏以来,河西地方文化学术比较发达,号称多士。北魏自道武帝以后,政治上使用汉族高门,汲取不少魏晋典制。431年,藉由同年打敗劉宋的威勢,拓拔焘下詔,徵聘關東地區的數百名士(多為領導地方的世家大族)入朝為官,也就是把山東郡姓如范陽盧氏、博陵崔氏、趙郡李氏等勢力一網打盡,強迫他們到平城擔任無薪水的官職,讓漢人世族的勢力與北魏政權相結合。當時被徵召的名士高允,後來寫了一篇文章〈徵士頌〉來追憶、讚揚此盛事。439年北魏吞并河西后,又有大批河西文士进入北魏统治区域,不少人被徵召到平城去做官,受到重用,北魏的儒学才开始兴盛。

之後,拓拔焘又击溃吐谷渾、柔然,扩地千餘里。他一共七次率军进攻柔然,太平真君十年(449)大败柔然,收民畜凡百余万,柔然可汗远遁,北方边塞再度得到安静。

他在450-451年对宋的战争中,雖然大勝,但人馬死傷近半,又使軍民疲憊,怨聲不已。末期又刑罰殘酷,使国内政治混乱。譬如崔浩修国史详实记载魏先世事迹,可能涉及某些鲜卑习俗和隐私,有伤体面,拓跋焘不惜发动國史之獄,将三朝功臣司徒崔浩处死,连清河崔氏与浩同宗者以及崔浩姻亲范阳卢氏、太原郭氏、河东柳氏都遭族灭。事后拓跋焘说 “崔司徒可惜”,有后悔之意;再如監國執政的太子,也在父子權力衝突下,被宦官宗愛的讒言害死。正平二年二月甲寅(452年3月11日)拓跋焘被宗爱杀死,享年四十五歲,谥号太武帝,庙号世祖。

拓跋焘统治时期,氐、羌、屠各,以及所谓“杂虏”、“杂人”的各族暴亂非常频繁。太平真君六年(445年)卢水胡盖吴在关中杏城(今陕西黄陵西南)发动的起义,声势最为浩大。盖吴建号秦地王,有众十余万,得到安定卢水胡刘超、河东蜀薛永宗的响应,拓跋焘调动强大的兵力才镇压下去。

拓跋焘受崔浩、寇谦之影响,奉道排佛。镇压盖吴过程中,在长安佛寺中发现大量兵器,认为佛寺与盖吴通谋,太平真君七年(446年),詔:「諸有佛圖、形像及胡經,盡皆擊破焚燒,沙門無少長悉坑之。」,是為北魏太武帝滅佛,三武滅佛之一(另外兩位是北周武帝和唐武宗)。

拓跋焘天生將才,为人勇健,善于指挥。战阵亲犯矢石,神色自若,命将出师,违其节度者多败,因此将士畏服,为之盡力。有知人之明,常从士伍中选拔人才。赏不遗贱,罚不避贵,虽所爱之人亦不宽假。他放棄父親拓跋嗣築邊城防禦柔然的政策,主動攻擊柔然並獲得成功。他自奉俭朴,而赏赐功臣绝无吝嗇,幾乎把資源都用在主動出擊的軍功賞賜之上。认为元老功臣勤劳日久,应让他们以爵归第,随时朝见饷宴,百官职务则可另简贤能。这样就保证了行政效率,使政治多少能健全发展。他倚重汉人,李顺、崔浩、李孝伯等先后掌握朝权,但個性果於殺戮,處死大臣後常懊悔自己太快動刀。

北齊史官魏收於《魏書》的「史臣曰」評論說:「世祖聰明雄斷,威靈傑立,藉二世之資,奮征伐之氣,遂戎軒四出,周旋險夷。掃統萬,平秦隴,翦遼海,盪河源,南夷荷擔,北蠕削跡,廓定四表,混一戎華,其為功也大矣。遂使有魏之業,光邁百王,豈非神叡經綸,事當命世。至於初則東儲不終,末乃釁成所忽。固本貽防,殆弗思乎?」

唐代某貴族「公子」與世族虞世南的對話:「公子曰:『魏之太祖、太武,孰與為輩?』先生曰:『太祖、太武,俱有異人之姿,故能辟土擒敵,窺覦江外。然善戰好殺,暴桀雄武,稟崆峒之氣焉。至於安忍誅殘,石季龍之儔也。』」

北宋司馬光評論說:「(北魏)繼以明元、太武,兼有青、兗,包司、豫,摧赫連,開關中,梟馮弘,吞遼碣,擄沮渠,并河右,高車入臣,蠕蠕遠遁;自河以北,逾於大漠,悉為其有;子孫稱帝者,百有餘年。左袵之盛,未之有也。」

资治通鉴记载: 魏主(指太武帝)為人,壯健鷙勇,臨城對陣,親犯矢石,左右死傷相繼,神色自若;由是將士畏服,咸盡死力。性儉率,服御飲膳,取給而已。群臣請增峻京城及修宮室曰: 「《易》云:『王公設險,以守其國。』又蕭何云:『天子以四海為家,不壯不麗,無以重威。』」帝曰:「古人有言:『在德不在險。』屈丐蒸土築城而朕滅之。 豈在城也?今天下未平,方須民力,土功之事,朕所未為。蕭何之對,非雅言也。」每以為財者軍國之本,不可輕費。至於賞賜,皆死事勳績之家,親戚貴寵未嘗橫有所及。命將出師,指授節度,違之者多致負敗。明於知人,或拔干於卒伍之中,唯其才用所長,不論本末。聽察精敏,下無遁情,賞不遺賤,罰不避貴,雖所甚愛之人,終無寬假。常曰:「法者,朕與天下共之,何敢輕也。」然性殘忍,果於殺戮,往往已殺而復悔之。

太平真君四年(443年)拓拔焘遣大臣李敞所刻的石刻祝文,存於嘎仙洞内的石壁上。1980年7月30日,中国考古学家米文平等人在此洞发现石刻祝文,结合当时在洞内发现的陶器碎片等,认定此处即为史书中记载的北魏祖庭。但该洞是否确实就是拓跋鲜卑的发源地,史学界尚有争论。

江蘇省南京市六合區东南有瓜步山,山上有佛狸祠。

《魏书·世祖纪下》记载:北魏太武帝拓跋焘于宋元嘉二十七年击败王玄谟的军队以后,在山上建立行宫,即后来的「佛狸祠」。

南宋诗人辛弃疾有《永遇乐·京口北固亭怀古》:「可堪回首,佛狸祠下,一片神鸦社鼓」。后又有《水调歌头·舟次扬州和杨济翁周显先韵》:「谁道投鞭飞渡,忆昔鸣髇血污,风雨佛狸愁。」

太延元年(435年)十月,太武帝东巡冀州、定州,二十日甲辰到定州,驻驾于新城宫。十一月十六日己巳,在广川(河北景县)校猎。二十三日丙子到达邺城(河北临漳),祭祀密太后(太武帝母杜氏)庙,并慰问老年族人,褒礼贤俊。十二月二十日癸卯派遣使者到北岳恒山祭祀。次年正月初二甲寅从五回道返回平城。

在东巡至河北易县南管头之南画猫村古徐水河谷时,见山岩险峭,景观奇丽,太武帝即兴演示射术,又命左右将士善射者进行射箭比试。镇东将军、定州刺史、乐浪公乞伏某请求立碑纪念。到太延三年丁丑(437年)碑刻完工,乐浪公已去职,新任刺史征东将军、张掖公秃发保周)接手此事。

东巡碑碑额题【皇帝东巡之碑】,史籍最早提到北魏太武帝东巡碑,是郦道元《水经注》。郦书之后,宋代乐史《太平寰宇记》卷六七易州满城县条,也曾提及此碑,称引的内容有溢出郦书者。此后东巡碑湮没无闻将近千年,直到1935年,由徐森玉(鸿宝)先生在河北易县觅得原碑,把20份拓本带回北平,次年傅增湘、周肇祥也前往摹拓,东巡碑才重新现身,为艺林所重。今碑已破碎,仅剩残片若干块。

\subsubsection{始光}

\begin{longtable}{|>{\centering\scriptsize}m{2em}|>{\centering\scriptsize}m{1.3em}|>{\centering}m{8.8em}|}
  % \caption{秦王政}\
  \toprule
  \SimHei \normalsize 年数 & \SimHei \scriptsize 公元 & \SimHei 大事件 \tabularnewline
  % \midrule
  \endfirsthead
  \toprule
  \SimHei \normalsize 年数 & \SimHei \scriptsize 公元 & \SimHei 大事件 \tabularnewline
  \midrule
  \endhead
  \midrule
  元年 & 424 & \tabularnewline\hline
  二年 & 425 & \tabularnewline\hline
  三年 & 426 & \tabularnewline\hline
  四年 & 427 & \tabularnewline\hline
  五年 & 428 & \tabularnewline
  \bottomrule
\end{longtable}

\subsubsection{神䴥}

\begin{longtable}{|>{\centering\scriptsize}m{2em}|>{\centering\scriptsize}m{1.3em}|>{\centering}m{8.8em}|}
  % \caption{秦王政}\
  \toprule
  \SimHei \normalsize 年数 & \SimHei \scriptsize 公元 & \SimHei 大事件 \tabularnewline
  % \midrule
  \endfirsthead
  \toprule
  \SimHei \normalsize 年数 & \SimHei \scriptsize 公元 & \SimHei 大事件 \tabularnewline
  \midrule
  \endhead
  \midrule
  元年 & 428 & \tabularnewline\hline
  二年 & 429 & \tabularnewline\hline
  三年 & 430 & \tabularnewline\hline
  四年 & 431 & \tabularnewline
  \bottomrule
\end{longtable}

\subsubsection{延和}

\begin{longtable}{|>{\centering\scriptsize}m{2em}|>{\centering\scriptsize}m{1.3em}|>{\centering}m{8.8em}|}
  % \caption{秦王政}\
  \toprule
  \SimHei \normalsize 年数 & \SimHei \scriptsize 公元 & \SimHei 大事件 \tabularnewline
  % \midrule
  \endfirsthead
  \toprule
  \SimHei \normalsize 年数 & \SimHei \scriptsize 公元 & \SimHei 大事件 \tabularnewline
  \midrule
  \endhead
  \midrule
  元年 & 432 & \tabularnewline\hline
  二年 & 433 & \tabularnewline\hline
  三年 & 434 & \tabularnewline\hline
  四年 & 435 & \tabularnewline
  \bottomrule
\end{longtable}

\subsubsection{太延}

\begin{longtable}{|>{\centering\scriptsize}m{2em}|>{\centering\scriptsize}m{1.3em}|>{\centering}m{8.8em}|}
  % \caption{秦王政}\
  \toprule
  \SimHei \normalsize 年数 & \SimHei \scriptsize 公元 & \SimHei 大事件 \tabularnewline
  % \midrule
  \endfirsthead
  \toprule
  \SimHei \normalsize 年数 & \SimHei \scriptsize 公元 & \SimHei 大事件 \tabularnewline
  \midrule
  \endhead
  \midrule
  元年 & 435 & \tabularnewline\hline
  二年 & 436 & \tabularnewline\hline
  三年 & 437 & \tabularnewline\hline
  四年 & 438 & \tabularnewline\hline
  五年 & 439 & \tabularnewline\hline
  六年 & 440 & \tabularnewline
  \bottomrule
\end{longtable}

\subsubsection{太平真君}

\begin{longtable}{|>{\centering\scriptsize}m{2em}|>{\centering\scriptsize}m{1.3em}|>{\centering}m{8.8em}|}
  % \caption{秦王政}\
  \toprule
  \SimHei \normalsize 年数 & \SimHei \scriptsize 公元 & \SimHei 大事件 \tabularnewline
  % \midrule
  \endfirsthead
  \toprule
  \SimHei \normalsize 年数 & \SimHei \scriptsize 公元 & \SimHei 大事件 \tabularnewline
  \midrule
  \endhead
  \midrule
  元年 & 440 & \tabularnewline\hline
  二年 & 441 & \tabularnewline\hline
  三年 & 442 & \tabularnewline\hline
  四年 & 443 & \tabularnewline\hline
  五年 & 444 & \tabularnewline\hline
  六年 & 445 & \tabularnewline\hline
  七年 & 446 & \tabularnewline\hline
  八年 & 447 & \tabularnewline\hline
  九年 & 448 & \tabularnewline\hline
  十年 & 449 & \tabularnewline\hline
  十一年 & 450 & \tabularnewline\hline
  十二年 & 451 & \tabularnewline
  \bottomrule
\end{longtable}

\subsubsection{正平}

\begin{longtable}{|>{\centering\scriptsize}m{2em}|>{\centering\scriptsize}m{1.3em}|>{\centering}m{8.8em}|}
  % \caption{秦王政}\
  \toprule
  \SimHei \normalsize 年数 & \SimHei \scriptsize 公元 & \SimHei 大事件 \tabularnewline
  % \midrule
  \endfirsthead
  \toprule
  \SimHei \normalsize 年数 & \SimHei \scriptsize 公元 & \SimHei 大事件 \tabularnewline
  \midrule
  \endhead
  \midrule
  元年 & 451 & \tabularnewline\hline
  二年 & 452 & \tabularnewline
  \bottomrule
\end{longtable}


%%% Local Variables:
%%% mode: latex
%%% TeX-engine: xetex
%%% TeX-master: "../../Main"
%%% End:

%% -*- coding: utf-8 -*-
%% Time-stamp: <Chen Wang: 2019-12-23 15:02:48>

\subsection{拓跋余\tiny(452)}

\subsubsection{生平}

拓跋余(5世纪-452年10月29日),鮮卑名可博真,北魏太武帝拓跋燾之子,生母闾左昭仪。北魏皇帝,為中常侍宗愛所立,但同年又被其殺害,所在位232天。

拓跋余生年不详。太平真君三年(442年),拓跋余獲封為吳王。正平元年(451年)九月,太武帝南征,以拓跋余留守平城。十二月改封南安王。

正平二年二月五日(452年3月11日),中常侍宗愛弒太武帝,尚书左仆射兰延、侍中吴兴公和疋、侍中太原公薛提等秘不发丧。兰延、和疋认为太武帝嫡孙拓跋濬冲幼,欲立太武帝在世长子,于是召东平王拓跋翰,置于秘室。薛提则坚持立拓跋濬,兰延等犹豫未决。宗爱得知。宗爱曾得罪拓跋晃,素来厌恶拓跋翰,却和拓跋余关系好,于是秘密迎拓跋余从中宫便门入宫,矫皇后令征召兰延等,斩于殿堂,再杀死拓跋翰,立拓跋余为帝,为太武帝发丧,改元承平(或作永平),大赦。

拓跋余即位後,因自己不是作为先帝长子继位,即厚待群下以取悅眾人,以宗爱为大司马、大将军、太师、都督中外诸军事,领中秘书,封冯翊王,以兄长原太子拓跋晃辅臣尚书令古弼为司徒,兼太尉张黎为太尉,羽林中郎、幢将乌程子建威将军吕罗汉典宿卫。但他也徹夜暢飲,夜夜笙歌,很快即令國庫空虛,又多次出獵,即使邊境有事,亦不加体恤,百姓皆憤怒,而他不作改變。南朝宋文帝刘义隆遣将檀和之侵犯济州,拓跋余令侍中、尚书左仆射、征南将军韩茂讨之,檀和之遁走。

另外,宗愛自拓跋余登位後掌權日久,朝野內外皆忌憚他,而拓跋余則懷疑宗愛另有所圖,密謀削奪宗愛權力,宗愛於是于十月一日(10月29日)使小黄门贾周等乘夜趁拓跋余祭祀东庙而杀之(《宋书》作与宗爱同被兄拓跋谭所杀,疑误),隐秘其事,只有羽林中郎刘尼知道。

拓跋余被杀后,百官不知道立谁为新君,刘尼、吕罗汉等迎立拓跋濬为北魏文成帝,诛杀宗爱。文成帝以王禮安葬拓跋余,諡為隱王。

先前太平真君七年(439年)八月,月犯荧惑;八月至十一月,又犯轩辕。是岁正月,太白经天。九月火犯太微。这些星象被认为是从拓跋晃去世到拓跋余被杀一系列事的预兆。

\subsubsection{承平}

\begin{longtable}{|>{\centering\scriptsize}m{2em}|>{\centering\scriptsize}m{1.3em}|>{\centering}m{8.8em}|}
  % \caption{秦王政}\
  \toprule
  \SimHei \normalsize 年数 & \SimHei \scriptsize 公元 & \SimHei 大事件 \tabularnewline
  % \midrule
  \endfirsthead
  \toprule
  \SimHei \normalsize 年数 & \SimHei \scriptsize 公元 & \SimHei 大事件 \tabularnewline
  \midrule
  \endhead
  \midrule
  元年 & 452 & \tabularnewline\hline
  \bottomrule
\end{longtable}


%%% Local Variables:
%%% mode: latex
%%% TeX-engine: xetex
%%% TeX-master: "../../Main"
%%% End:

%% -*- coding: utf-8 -*-
%% Time-stamp: <Chen Wang: 2019-12-23 15:13:21>

\subsection{文成帝\tiny(452-465)}

\subsubsection{生平}

魏文成帝拓跋濬(440年-465年),鮮卑名烏雷,拥有直懃头衔,南北朝時期北魏的第四代皇帝。是魏太武帝的嫡孫,景穆太子拓跋晃長子。452年-465年在位,在位十三年。

正平元年(451年),太武帝北征,任太子拓跋晃為監國,但宦官中常侍宗愛對太子多加干預,又與太子部屬給事仇尼道盛和侍郎任城互有閒隙,宗愛怕日後太子登基對己不利,於是與東宮勢力展開權力鬥爭,仗着拓跋燾的信任誣陷太子及其手下人意圖造反,不曾想皇帝拓跋燾竟然相信了。继而下令整肅太子府,且誅殺了許多太子近臣。太子拓跋晃因此積憂成疾,一病而死,時年才24歲。

後來魏太武帝拓跋燾知道太子是清白的,非常懊悔。但宗愛一見到此,怕被皇帝誅殺,先下手為強於正平二年(452年)三月弒太武帝。太武帝死後,朝廷欲立太武帝最年长的在世儿子第三子東平王拓跋翰為帝,但宗愛與拓跋翰關係不好,因此假立太武皇后之命,將拓跋翰殺掉,又假借皇后之命,將擁立東平王的大臣尚書僕射蘭延、侍中吳興公和疋及侍中太原公薛提殺死,然後立太武帝幼子南安王拓跋余為帝,宗愛自為大司馬、大將軍、太師,總督中外軍事、領中祕書,封馮翊王,大權在握。拓跋余想奪回皇權,又于十月一日遭宗愛所弒。短短數月,宗愛連殺兩位皇帝,引起朝野震動。

羽林郎中刘尼、太子少傅游雅、殿中尚书源贺、尚书陆丽、尚书长孙渴侯五人密谋,十月三日(10月31日),由太子少傅游雅、源贺、长孙渴侯率禁军守卫宫廷,陆丽与刘尼一起迎皇孙拓跋濬入宫即位。拓跋濬改元兴安。

拓跋濬即位后,便诛杀了宗爱、贾周等人,都动用五刑,灭三族。兴安元年(452年)十一月初九,文成帝追谥父亲拓跋晃为景穆皇帝,母亲闾氏为恭皇后,尊乳母常氏为保太后。

太武帝崇信道教,一度太武滅佛。兴安元年,拓跋濬下令复兴佛教。兴安二年,令建造云冈石窟。

拓跋濬不再继续太武帝四处用兵的政策,停止南侵南朝宋,休养生息。但也有征伐。太安四年(458年),拓跋濬亲率10万骑兵、15万辆战车,进攻柔然。柔然处罗可汗吐贺真远远逃走。柔然别部统帅乌朱驾颓等人率领几千个帳幕所聚的部落投降。

和平六年五月十一日(465年6月21日),拓跋濬去世,时年僅25岁。六月初二,定谥号为文成皇帝,庙号高宗。八月,安葬云中的金陵。

\subsubsection{兴安}

\begin{longtable}{|>{\centering\scriptsize}m{2em}|>{\centering\scriptsize}m{1.3em}|>{\centering}m{8.8em}|}
  % \caption{秦王政}\
  \toprule
  \SimHei \normalsize 年数 & \SimHei \scriptsize 公元 & \SimHei 大事件 \tabularnewline
  % \midrule
  \endfirsthead
  \toprule
  \SimHei \normalsize 年数 & \SimHei \scriptsize 公元 & \SimHei 大事件 \tabularnewline
  \midrule
  \endhead
  \midrule
  元年 & 452 & \tabularnewline\hline
  二年 & 453 & \tabularnewline\hline
  三年 & 454 & \tabularnewline
  \bottomrule
\end{longtable}

\subsubsection{兴光}

\begin{longtable}{|>{\centering\scriptsize}m{2em}|>{\centering\scriptsize}m{1.3em}|>{\centering}m{8.8em}|}
  % \caption{秦王政}\
  \toprule
  \SimHei \normalsize 年数 & \SimHei \scriptsize 公元 & \SimHei 大事件 \tabularnewline
  % \midrule
  \endfirsthead
  \toprule
  \SimHei \normalsize 年数 & \SimHei \scriptsize 公元 & \SimHei 大事件 \tabularnewline
  \midrule
  \endhead
  \midrule
  元年 & 454 & \tabularnewline\hline
  二年 & 455 & \tabularnewline
  \bottomrule
\end{longtable}

\subsubsection{太安}

\begin{longtable}{|>{\centering\scriptsize}m{2em}|>{\centering\scriptsize}m{1.3em}|>{\centering}m{8.8em}|}
  % \caption{秦王政}\
  \toprule
  \SimHei \normalsize 年数 & \SimHei \scriptsize 公元 & \SimHei 大事件 \tabularnewline
  % \midrule
  \endfirsthead
  \toprule
  \SimHei \normalsize 年数 & \SimHei \scriptsize 公元 & \SimHei 大事件 \tabularnewline
  \midrule
  \endhead
  \midrule
  元年 & 455 & \tabularnewline\hline
  二年 & 456 & \tabularnewline\hline
  三年 & 457 & \tabularnewline\hline
  四年 & 458 & \tabularnewline\hline
  五年 & 459 & \tabularnewline
  \bottomrule
\end{longtable}

\subsubsection{和平}

\begin{longtable}{|>{\centering\scriptsize}m{2em}|>{\centering\scriptsize}m{1.3em}|>{\centering}m{8.8em}|}
  % \caption{秦王政}\
  \toprule
  \SimHei \normalsize 年数 & \SimHei \scriptsize 公元 & \SimHei 大事件 \tabularnewline
  % \midrule
  \endfirsthead
  \toprule
  \SimHei \normalsize 年数 & \SimHei \scriptsize 公元 & \SimHei 大事件 \tabularnewline
  \midrule
  \endhead
  \midrule
  元年 & 460 & \tabularnewline\hline
  二年 & 461 & \tabularnewline\hline
  三年 & 462 & \tabularnewline\hline
  四年 & 463 & \tabularnewline\hline
  五年 & 464 & \tabularnewline\hline
  六年 & 465 & \tabularnewline
  \bottomrule
\end{longtable}


%%% Local Variables:
%%% mode: latex
%%% TeX-engine: xetex
%%% TeX-master: "../../Main"
%%% End:

%% -*- coding: utf-8 -*-
%% Time-stamp: <Chen Wang: 2019-12-23 15:13:53>

\subsection{献文帝\tiny(465-471)}

\subsubsection{生平}

魏獻文帝拓跋弘(454年-476年),鮮卑名第豆胤,魏文成帝拓跋濬長子,南北朝時期北魏第六位皇帝。

太安二年(456年)正月,立为太子。

和平六年(465年)五月,父亲拓跋濬逝世,随后,拓跋弘登基为帝。

皇興元年(467年)八月,獻文帝之妃李夫人誕下拓跋宏。   

皇興三年(469年)六月,李夫人被按照子貴母死的制度殺死。葬在金陵,承明元年(476年)追諡思皇后,配饗太廟。

太和二年(478年)十二月,李夫人全家都被馮太后處死。

皇興五年(471年),年僅17歲的魏献文帝因不滿馮太后長期攝政及專權,原本要禪讓給三叔京兆王拓跋子推,被眾臣勸阻後作罷,禪讓予太子拓跋宏,且因群臣奏称“三皇澹泊无为,所以称皇;西汉高祖之父被尊为太上皇,是不统治天下的,而皇帝年幼,陛下仍然执政”,不称太上皇,而称尊号太上皇帝。雖然作為太上皇,但魏献文帝仍掌握有部分的皇帝權力,並欲與馮太后爭權。

延興二年(472年),柔然來犯,魏献文帝以太上皇之姿,御駕親征,大敗柔然,並追至大漠。

承明元年(476年),拓跋弘被馮太后毒死,崩於永安殿,年僅二十三歲,諡曰獻文皇帝,廟號顯祖,葬於金陵。

\subsubsection{天安}

\begin{longtable}{|>{\centering\scriptsize}m{2em}|>{\centering\scriptsize}m{1.3em}|>{\centering}m{8.8em}|}
  % \caption{秦王政}\
  \toprule
  \SimHei \normalsize 年数 & \SimHei \scriptsize 公元 & \SimHei 大事件 \tabularnewline
  % \midrule
  \endfirsthead
  \toprule
  \SimHei \normalsize 年数 & \SimHei \scriptsize 公元 & \SimHei 大事件 \tabularnewline
  \midrule
  \endhead
  \midrule
  元年 & 466 & \tabularnewline\hline
  二年 & 467 & \tabularnewline
  \bottomrule
\end{longtable}

\subsubsection{皇兴}

\begin{longtable}{|>{\centering\scriptsize}m{2em}|>{\centering\scriptsize}m{1.3em}|>{\centering}m{8.8em}|}
  % \caption{秦王政}\
  \toprule
  \SimHei \normalsize 年数 & \SimHei \scriptsize 公元 & \SimHei 大事件 \tabularnewline
  % \midrule
  \endfirsthead
  \toprule
  \SimHei \normalsize 年数 & \SimHei \scriptsize 公元 & \SimHei 大事件 \tabularnewline
  \midrule
  \endhead
  \midrule
  元年 & 467 & \tabularnewline\hline
  二年 & 468 & \tabularnewline\hline
  三年 & 469 & \tabularnewline\hline
  四年 & 470 & \tabularnewline\hline
  五年 & 471 & \tabularnewline
  \bottomrule
\end{longtable}


%%% Local Variables:
%%% mode: latex
%%% TeX-engine: xetex
%%% TeX-master: "../../Main"
%%% End:

%% -*- coding: utf-8 -*-
%% Time-stamp: <Chen Wang: 2021-11-01 15:10:55>

\subsection{孝文帝元宏\tiny(471-499)}

\subsubsection{生平}

魏孝文帝元宏(467年10月13日-499年4月26日),本姓拓跋,是北魏献文帝拓跋弘的长子,北魏第七位皇帝(471年9月20日—499年4月26日在位),後改姓「元」,在位28年,卒年33岁,其所推行的孝文帝改革,以漢化運動為主體,俗稱孝文漢化,其改革措施有利于缓解民族隔阂和阶级矛盾,为社会经济的恢复和发展发挥积极作用。雖然因推動漢化急進而最終導致六鎮起義及北魏解體,卻為北朝的胡漢融合作出貢獻。

孝文帝去世后,庙号高祖,谥号孝文皇帝。

北魏献文帝皇兴元年八月二十九日(467年10月13日),元宏生于北魏首都平城(今山西省大同市)紫宫,生母李夫人。

皇兴五年八月二十日(471年9月20日),父親獻文帝禪讓于太华前殿,大赦,改元延兴元年。

元宏即位时只有5岁,獻文帝死後,由祖母冯太后攝政。冯太后是汉人,对鲜卑人建立的北魏朝廷进行了一系列中央集权化的改革,孝文帝便受此影响。

太和十四年(490年)冯太后去世后亲政,秉承馮太后的政策,繼續进行了汉化改革,而且做得比馮太后更大刀闊斧。他先整顿吏治,立三长法,实行均田制;太和十八年(494年),他以“南伐”为名迁都洛阳,全面改革鲜卑旧俗:规定以汉服代替鲜卑服,以汉语代替鲜卑语,迁洛鲜卑人以洛阳为籍贯,改鲜卑姓为汉姓,自己也改姓“元”。并鼓励鲜卑贵族与汉士族联姻,又参照南朝典章,修改北魏政治制度,并严厉镇压反对改革的守旧贵族,处死太子恂,这一举动使鲜卑经济、文化、社会、政治、军事等方面大大的发展,缓解了民族隔阂,史称“孝文帝改革”。

太和二十三年三月丙戌(499年4月6日),魏孝文帝在南征途中生病。三月庚子(499年4月20日),魏孝文帝病重。四月初一日(499年4月26日),孝文帝崩于谷塘原之行宫。孝文帝去世后,庙号高祖,谥号孝文皇帝。

陵墓位於河南省洛陽市孟津縣官庄村村東南800米處的兩個大型土丘,兩塚相距约100米,大塚是魏孝文帝陵墓「長陵」,小塚是第三位皇后文昭皇后(即宣武帝生母)的「終寧陵」。

然而孝文帝去世以后仅25年,北魏边镇鲜卑军事集团就发动反汉化运动六镇起义。534年,北魏分裂成东魏、西魏,之后更分别被北齐和北周取代。

北齐官修正史《魏书》魏收的评价是:“史臣曰:有魏始基代朔,廓平南夏,辟壤经世,咸以威武为业,文教之事,所未遑也。高祖幼承洪绪,早著睿圣之风。时以文明摄事,优游恭己,玄览独得,著自不言,神契所标,固以符于冥化。及躬总大政,一日万机,十许年间,曾不暇给;殊途同归,百虑一致。至夫生民所难行,人伦之高迹,虽尊居黄屋,尽蹈之矣。若乃钦明稽古,协御天人,帝王制作,朝野轨度,斟酌用舍,焕乎其有文章,海内生民咸受耳目之赐。加以雄才大略,爱奇好士,视下如伤,役己利物,亦无得而称之。其经纬天地,岂虚谥也!”

《資治通鑑》曰:高祖友愛諸弟,始終無間。嘗從容謂咸陽王禧等曰:「我後子孫解逅不肖,汝等觀望,可輔則輔之,不可輔則取之,勿為它人有也。」親任賢能,從善如流,精勤庶務,朝夕不倦。常曰:「人主患不能處心公平,推誠於物。能是二者,則胡、越之人皆可使如兄弟矣。」用法雖嚴,於大臣無所容貸,然人有小過,常多闊略。嘗於食中得蟲,又左右進羹誤傷帝手,皆笑而赦之。天地五郊、宗廟二分之祭,未嘗不身親其禮。每出巡遊及用兵,有司奏修道路,帝輒曰:「粗修橋樑,通車馬而已,勿去草鏟令平也。」在淮南行兵,如在境內,禁士卒無得踐傷粟稻;或伐民樹以供軍用,皆留絹償之。宮室非不得已不修,衣弊,浣濯而服之,鞍勒用鐵木而已。幼多力善射,能以指彈碎羊骨,射禽獸無不命中;及年十五,遂不復畋獵。常謂史官曰:「時事不可以不直書。人君威福在己,無能制之者;若史策復不書其惡,將何所畏忌邪!」

\subsubsection{延兴}

\begin{longtable}{|>{\centering\scriptsize}m{2em}|>{\centering\scriptsize}m{1.3em}|>{\centering}m{8.8em}|}
  % \caption{秦王政}\
  \toprule
  \SimHei \normalsize 年数 & \SimHei \scriptsize 公元 & \SimHei 大事件 \tabularnewline
  % \midrule
  \endfirsthead
  \toprule
  \SimHei \normalsize 年数 & \SimHei \scriptsize 公元 & \SimHei 大事件 \tabularnewline
  \midrule
  \endhead
  \midrule
  元年 & 471 & \tabularnewline\hline
  二年 & 472 & \tabularnewline\hline
  三年 & 473 & \tabularnewline\hline
  四年 & 474 & \tabularnewline\hline
  五年 & 475 & \tabularnewline\hline
  六年 & 476 & \tabularnewline
  \bottomrule
\end{longtable}

\subsubsection{承明}

\begin{longtable}{|>{\centering\scriptsize}m{2em}|>{\centering\scriptsize}m{1.3em}|>{\centering}m{8.8em}|}
  % \caption{秦王政}\
  \toprule
  \SimHei \normalsize 年数 & \SimHei \scriptsize 公元 & \SimHei 大事件 \tabularnewline
  % \midrule
  \endfirsthead
  \toprule
  \SimHei \normalsize 年数 & \SimHei \scriptsize 公元 & \SimHei 大事件 \tabularnewline
  \midrule
  \endhead
  \midrule
  元年 & 476 & \tabularnewline
  \bottomrule
\end{longtable}

\subsubsection{太和}

\begin{longtable}{|>{\centering\scriptsize}m{2em}|>{\centering\scriptsize}m{1.3em}|>{\centering}m{8.8em}|}
  % \caption{秦王政}\
  \toprule
  \SimHei \normalsize 年数 & \SimHei \scriptsize 公元 & \SimHei 大事件 \tabularnewline
  % \midrule
  \endfirsthead
  \toprule
  \SimHei \normalsize 年数 & \SimHei \scriptsize 公元 & \SimHei 大事件 \tabularnewline
  \midrule
  \endhead
  \midrule
  元年 & 477 & \tabularnewline\hline
  二年 & 478 & \tabularnewline\hline
  三年 & 479 & \tabularnewline\hline
  四年 & 480 & \tabularnewline\hline
  五年 & 481 & \tabularnewline\hline
  六年 & 482 & \tabularnewline\hline
  七年 & 483 & \tabularnewline\hline
  八年 & 484 & \tabularnewline\hline
  九年 & 485 & \tabularnewline\hline
  十年 & 486 & \tabularnewline\hline
  十一年 & 487 & \tabularnewline\hline
  十二年 & 488 & \tabularnewline\hline
  十三年 & 489 & \tabularnewline\hline
  十四年 & 490 & \tabularnewline\hline
  十五年 & 491 & \tabularnewline\hline
  十六年 & 492 & \tabularnewline\hline
  十七年 & 493 & \tabularnewline\hline
  十八年 & 494 & \tabularnewline\hline
  十九年 & 495 & \tabularnewline\hline
  二十年 & 496 & \tabularnewline\hline
  二一年 & 497 & \tabularnewline\hline
  二二年 & 498 & \tabularnewline\hline
  二三年 & 499 & \tabularnewline
  \bottomrule
\end{longtable}


%%% Local Variables:
%%% mode: latex
%%% TeX-engine: xetex
%%% TeX-master: "../../Main"
%%% End:

%% -*- coding: utf-8 -*-
%% Time-stamp: <Chen Wang: 2021-11-01 15:11:15>

\subsection{宣武帝元恪\tiny(499-515)}

\subsubsection{生平}

魏宣武帝元恪(483年-515年2月12日),河南郡洛阳县(今河南省洛阳市东)人,魏孝文帝元宏次子,生母貴人高照容,是南北朝时期北魏的第八代皇帝。499年-515年在位,在位十六年。

太和七年(483年),異母兄元恂之生母林氏按北魏子貴母死之慣例而被賜死,元恂由嫡曾祖母馮太后養育。太和十七年(493年)七月,長兄元恂被立為皇太子。孝文帝遠征南齊,十歲的元恂留守新都洛陽。元恂嫌河南酷暑,穿胡服。

太和二十年(496年),十三歲的元恂逃至平城,得到反對漢化和南遷的貴族的支持。其父孝文帝返回後平息了變亂,廢黜元恂為庶人,囚禁在河陽,衣食僅夠維生。不久,又派人將元恂賜死。元恪的長兄元恂死時年僅15歲。

太和二十二年(498年),孝文帝改立十六歲的元恪為皇太子。

翌年,孝文帝崩。十七歲的元恪即位。

宣武帝的統治初期(499年至508年),他的叔父北魏宗室咸陽王元禧(獻文帝拓跋弘次子,孝文帝元宏之異母弟)輔政、尚書令王肅輔佐。

北魏對南朝發動了一系列戰爭,攻取南朝梁的四川之地、北撃柔然,北魏疆域大大向南拓展,國勢盛極一時。因篤信佛教,宣武帝取消子貴母死制度,讓宣武靈皇后活著。

他在位的後半期,外戚高肇專權,朝政一片黑暗,北魏逐漸衰弱。延昌四年正月,宣武帝崩於式乾殿。

\subsubsection{景明}

\begin{longtable}{|>{\centering\scriptsize}m{2em}|>{\centering\scriptsize}m{1.3em}|>{\centering}m{8.8em}|}
  % \caption{秦王政}\
  \toprule
  \SimHei \normalsize 年数 & \SimHei \scriptsize 公元 & \SimHei 大事件 \tabularnewline
  % \midrule
  \endfirsthead
  \toprule
  \SimHei \normalsize 年数 & \SimHei \scriptsize 公元 & \SimHei 大事件 \tabularnewline
  \midrule
  \endhead
  \midrule
  元年 & 500 & \tabularnewline\hline
  二年 & 501 & \tabularnewline\hline
  三年 & 502 & \tabularnewline\hline
  四年 & 503 & \tabularnewline\hline
  五年 & 504 & \tabularnewline
  \bottomrule
\end{longtable}

\subsubsection{正始}

\begin{longtable}{|>{\centering\scriptsize}m{2em}|>{\centering\scriptsize}m{1.3em}|>{\centering}m{8.8em}|}
  % \caption{秦王政}\
  \toprule
  \SimHei \normalsize 年数 & \SimHei \scriptsize 公元 & \SimHei 大事件 \tabularnewline
  % \midrule
  \endfirsthead
  \toprule
  \SimHei \normalsize 年数 & \SimHei \scriptsize 公元 & \SimHei 大事件 \tabularnewline
  \midrule
  \endhead
  \midrule
  元年 & 504 & \tabularnewline\hline
  二年 & 505 & \tabularnewline\hline
  三年 & 506 & \tabularnewline\hline
  四年 & 507 & \tabularnewline\hline
  五年 & 508 & \tabularnewline
  \bottomrule
\end{longtable}

\subsubsection{永平}

\begin{longtable}{|>{\centering\scriptsize}m{2em}|>{\centering\scriptsize}m{1.3em}|>{\centering}m{8.8em}|}
  % \caption{秦王政}\
  \toprule
  \SimHei \normalsize 年数 & \SimHei \scriptsize 公元 & \SimHei 大事件 \tabularnewline
  % \midrule
  \endfirsthead
  \toprule
  \SimHei \normalsize 年数 & \SimHei \scriptsize 公元 & \SimHei 大事件 \tabularnewline
  \midrule
  \endhead
  \midrule
  元年 & 508 & \tabularnewline\hline
  二年 & 509 & \tabularnewline\hline
  三年 & 510 & \tabularnewline\hline
  四年 & 511 & \tabularnewline\hline
  五年 & 512 & \tabularnewline
  \bottomrule
\end{longtable}

\subsubsection{延昌}

\begin{longtable}{|>{\centering\scriptsize}m{2em}|>{\centering\scriptsize}m{1.3em}|>{\centering}m{8.8em}|}
  % \caption{秦王政}\
  \toprule
  \SimHei \normalsize 年数 & \SimHei \scriptsize 公元 & \SimHei 大事件 \tabularnewline
  % \midrule
  \endfirsthead
  \toprule
  \SimHei \normalsize 年数 & \SimHei \scriptsize 公元 & \SimHei 大事件 \tabularnewline
  \midrule
  \endhead
  \midrule
  元年 & 512 & \tabularnewline\hline
  二年 & 513 & \tabularnewline\hline
  三年 & 514 & \tabularnewline\hline
  四年 & 515 & \tabularnewline
  \bottomrule
\end{longtable}


%%% Local Variables:
%%% mode: latex
%%% TeX-engine: xetex
%%% TeX-master: "../../Main"
%%% End:

%% -*- coding: utf-8 -*-
%% Time-stamp: <Chen Wang: 2019-12-23 15:16:31>

\subsection{孝明帝\tiny(515-528)}

\subsubsection{生平}

魏孝明帝元诩(510年-528年3月31日),河南郡洛阳县(今河南省洛阳市东)人,魏宣武帝元恪第二子,生母胡充华,是南北朝时期北魏的皇帝,在位13年。

延昌四年(515年),宣武帝病逝,太子元詡繼位,由太傅、侍中元懌輔政。胡太后和元懌相愛,常招元懌夜宿宮中。領軍元乂和長秋卿劉騰等人密謀,將元懌殺害,又把胡太后幽禁在北宮的宣光殿。胡太后又結識鄭儼、李神軌、徐紇諸情人。鄭儼和徐紇把持內外,時稱“徐鄭”。

正光年间以后,国家入不敷出,政府决定预先征收六年的租調,导致人民生活愈加艰苦。除此之外,政府停止对官员供应酒,但每季所举行的祭祀祖宗、神明的仪式及外交所需费用不算在内。由于匪徒越来越多,大量器械被劫掠,关西地区匪患尤其严重。政府的积蓄于是枯竭。在此境地下,政府又下令减少对官员及外国使节百分之五十的粮食及肉供应。孝昌年间,京城的田每亩征税五升,借公田耕种者每亩征税一斗。并且还对市场,商业店铺征税。

武泰元年(528年),孝明帝妃潘外憐生一女,胡太后宣稱生男孩,大赦天下。武泰元年二月癸丑(528年3月31日),鄭儼率御林軍來到顯陽殿,將孝明帝毒死。爾朱榮聞訊,追查孝明帝的死因,另立長樂王元子攸。

\subsubsection{熙平}

\begin{longtable}{|>{\centering\scriptsize}m{2em}|>{\centering\scriptsize}m{1.3em}|>{\centering}m{8.8em}|}
  % \caption{秦王政}\
  \toprule
  \SimHei \normalsize 年数 & \SimHei \scriptsize 公元 & \SimHei 大事件 \tabularnewline
  % \midrule
  \endfirsthead
  \toprule
  \SimHei \normalsize 年数 & \SimHei \scriptsize 公元 & \SimHei 大事件 \tabularnewline
  \midrule
  \endhead
  \midrule
  元年 & 516 & \tabularnewline\hline
  二年 & 517 & \tabularnewline\hline
  三年 & 518 & \tabularnewline
  \bottomrule
\end{longtable}

\subsubsection{神龟}

\begin{longtable}{|>{\centering\scriptsize}m{2em}|>{\centering\scriptsize}m{1.3em}|>{\centering}m{8.8em}|}
  % \caption{秦王政}\
  \toprule
  \SimHei \normalsize 年数 & \SimHei \scriptsize 公元 & \SimHei 大事件 \tabularnewline
  % \midrule
  \endfirsthead
  \toprule
  \SimHei \normalsize 年数 & \SimHei \scriptsize 公元 & \SimHei 大事件 \tabularnewline
  \midrule
  \endhead
  \midrule
  元年 & 518 & \tabularnewline\hline
  二年 & 519 & \tabularnewline\hline
  三年 & 520 & \tabularnewline
  \bottomrule
\end{longtable}

\subsubsection{正光}

\begin{longtable}{|>{\centering\scriptsize}m{2em}|>{\centering\scriptsize}m{1.3em}|>{\centering}m{8.8em}|}
  % \caption{秦王政}\
  \toprule
  \SimHei \normalsize 年数 & \SimHei \scriptsize 公元 & \SimHei 大事件 \tabularnewline
  % \midrule
  \endfirsthead
  \toprule
  \SimHei \normalsize 年数 & \SimHei \scriptsize 公元 & \SimHei 大事件 \tabularnewline
  \midrule
  \endhead
  \midrule
  元年 & 520 & \tabularnewline\hline
  二年 & 521 & \tabularnewline\hline
  三年 & 522 & \tabularnewline\hline
  四年 & 523 & \tabularnewline\hline
  五年 & 524 & \tabularnewline\hline
  六年 & 525 & \tabularnewline
  \bottomrule
\end{longtable}

\subsubsection{孝昌}

\begin{longtable}{|>{\centering\scriptsize}m{2em}|>{\centering\scriptsize}m{1.3em}|>{\centering}m{8.8em}|}
  % \caption{秦王政}\
  \toprule
  \SimHei \normalsize 年数 & \SimHei \scriptsize 公元 & \SimHei 大事件 \tabularnewline
  % \midrule
  \endfirsthead
  \toprule
  \SimHei \normalsize 年数 & \SimHei \scriptsize 公元 & \SimHei 大事件 \tabularnewline
  \midrule
  \endhead
  \midrule
  元年 & 525 & \tabularnewline\hline
  二年 & 526 & \tabularnewline\hline
  三年 & 527 & \tabularnewline\hline
  四年 & 528 & \tabularnewline
  \bottomrule
\end{longtable}

\subsubsection{武泰}

\begin{longtable}{|>{\centering\scriptsize}m{2em}|>{\centering\scriptsize}m{1.3em}|>{\centering}m{8.8em}|}
  % \caption{秦王政}\
  \toprule
  \SimHei \normalsize 年数 & \SimHei \scriptsize 公元 & \SimHei 大事件 \tabularnewline
  % \midrule
  \endfirsthead
  \toprule
  \SimHei \normalsize 年数 & \SimHei \scriptsize 公元 & \SimHei 大事件 \tabularnewline
  \midrule
  \endhead
  \midrule
  元年 & 528 & \tabularnewline
  \bottomrule
\end{longtable}


%%% Local Variables:
%%% mode: latex
%%% TeX-engine: xetex
%%% TeX-master: "../../Main"
%%% End:

%% -*- coding: utf-8 -*-
%% Time-stamp: <Chen Wang: 2019-12-23 15:19:26>

\subsection{元氏生平}

元氏(528年2月12日-?),女,河南洛阳(今河南省洛阳市东)人,真名不详,南北朝时期北魏第10位皇帝(未获后世普遍承认),是北魏孝明帝元詡与宫嫔潘外憐的女儿,也是孝明帝唯一的骨肉。出生后因时局危险,所以她的祖母、掌握帝国实际大权的皇太后胡氏对外宣称本为女性(即皇女)的她是男性(即为皇子),以安人心。不久,孝明帝暴崩,尚在襁褓中的“皇子”元氏以先帝唯一子嗣的身份继位(528年4月1日),在名义上成为了北魏皇帝。元氏即位当天便被废黜,次日由幼主元钊继位,之后史书上便没有了对她的记载。

北魏孝明帝孝昌四年春正月初七乙丑日(528年2月12日),皇女元氏出生。她是孝明帝第一个、也是唯一的孩子,她的母亲是孝明帝九嫔之一的充华潘外憐,而她的祖母即为临朝称制的灵太后胡氏。胡太后对外宣称潘外憐生下的是皇子,并于第二天(正月初八丙寅日,即2月13日)颁布诏书,大赦天下,改元武泰。胡太后之所以诈称皇女为皇子,和她与儿子元诩的矛盾有关。胡太后本为北魏宣武帝众妾之一的充华,但却为宣武帝生下了唯一存活下来的子嗣,即后来继位的元诩。宣武帝死后,元诩继皇帝位,是为孝明帝,尊胡充华为皇太妃,后又加尊为皇太后。因孝明帝年幼,胡太后临朝称制。

但孝明帝日渐长大,胡太后却不肯放权归政,而且在政治上排斥异己,生活上不检点,不但引起朝臣反感,连孝明帝对她也大为不满,甚至招致天下人的厌恶,尤其是孝明帝把与胡太后私通的清河王元怿处死后,胡太后对儿子也恨之入骨,母子之间的裂痕越来越深。经过了几次反对胡太后的失败的政变,孝明帝密令大将尔朱荣进兵首都洛阳(今河南省洛阳市),试图胁迫胡太后归政。胡太后知道后,与近臣商议对策,适逢孝明帝的充華潘外憐生了皇女,所以胡太后假称是皇子,大赦天下,转移朝臣的视线,暗中谋划除去孝明帝。

武泰元年二月廿五癸丑日(528年3月31日),孝明帝元詡突然驾崩,是被胡太后暗中串通近臣用毒酒毒死的。第二天(即武泰元年二月廿六甲寅日,528年4月1日),胡太后伪称皇女元氏为皇太子,拥立元氏登基为皇帝,胡太后继续临朝称制,再次大赦天下。当时的元氏出生刚满50天。由于没有跨年,未改元,仍然沿用“武泰”年号。这样,元氏成为名义上的北魏皇帝。当然,由于这位女婴皇帝才出生一个多月,不可能真正地行使皇帝权力,实权仍然掌握在她的祖母、临朝称制的胡太后手中,元氏只不过是胡太后的傀儡。

胡太后立皇女元氏为帝才一天不到,见人心已经安定,当天便发下诏书宣布皇帝本是女儿身,废黜女婴皇帝,改立已故宗室临洮王元宝晖的世子元钊为皇帝。

胡太后的诏书全文如下:“皇家握历受图,年将二百;祖宗累圣,社稷载安。高祖以文思先天,世宗以下武经世,股肱惟良,元首穆穆。及大行在御,重以宽仁,奉养率由,温明恭顺。朕以寡昧,亲临万国,识谢涂山,德惭文母。属妖逆递兴,四郊多故。实望穹灵降祐,麟趾众繁。自潘充华有孕椒宫,冀诞储两,而熊罴无兆,维虺遂彰。于时直以国步未康,假称统胤,欲以底定物情,系仰宸极。何图一旦,弓剑莫追,国道中微,大行绝祀。皇曾孙故临洮王宝晖世子钊,体自高祖,天表卓异,大行平日养爱特深,义齐若子,事符当璧。及翊日弗愈,大渐弥留,乃延入青蒲,受命玉几。暨陈衣在庭,登策靡及,允膺大宝,即日践阼。朕是用惶惧忸怩,心焉靡洎。今丧君有君,宗祏惟固,宜崇赏卿士,爰及百辟,凡厥在位,并加陟叙。内外百官文武、督将征人,遭艰解府,普加军功二阶;其禁卫武官,直阁以下直从以上及主帅,可军功三阶;其亡官失爵,听复封位。谋反大逆削除者,不在斯限。清议禁锢,亦悉蠲除。若二品以上不能自受者,任授兒弟。可班宣远迩,咸使知之。”

胡太后诏书的大意就是孝明帝死得仓促,来不及指定继承人,只好让他唯一的女儿暂且以皇子身份继承皇位,后来发现元钊是皇位的合适人选,便废女婴皇帝而让元钊当皇帝。

元钊于胡太后发下诏书后的第二天(即武泰元年二月廿七乙卯日,528年4月2日)正式即位,是为北魏幼主。

实际上,这一废一立都是胡太后试图长期掌握帝国最高权力而使出的手段,因为元钊虽然比女婴皇帝大几岁,但也只有三岁,胡太后立他为皇帝是因为他年幼不能管理国家,她可以继续临朝称制,统治天下了。实际上,胡太后自被孝明帝尊为太后开始就是北魏的实际统治者了,她不但临朝称制,还自称为“朕”(秦始皇以后的皇帝自称),让朝臣们称她为“陛下”(臣下对皇帝的尊称)。她不惜先毒杀亲生儿子,后立尚在襁褓中的孙女,再立刚满三岁的宗室幼子,就是因为她是妇人之身,在当时的情况下不能直接行使皇帝权力,立年幼无知的傀儡皇帝可以保证自己通过临朝称制的方法继续掌权专制,而其最后终于身败名裂,遂被后世认为是不成功的野心家和导致北魏分裂亡国的罪人。

女婴皇帝被废後無紀錄可循,再也不知所終,但她因為即位為帝,導致的政治影響很大。

而短时间的废立让天下震惊,認定胡太后害死孝明帝,大将尔朱荣遂以胡太后肆意废立为藉口带兵讨伐,而其中一条重要的理由就是胡太后欺瞒上天和朝臣,立女婴为帝。尔朱荣又另立元子攸为皇帝,是为孝莊帝,这样北魏出现了两帝并立的局面。不过这种局面很快就被打破,15日后,尔朱荣的军队占领京师洛阳,胡太后和幼主元钊被俘。尔朱荣将幼主和胡太后押送至黄河南岸边。胡太后在尔朱荣面前说了许多好话求饶,尔朱荣不听,下令将幼主和胡太后沉入黄河,后又屠杀大臣两千多人,史称河阴之变。从此,尔朱荣完全掌握了北魏的实际大权,而北魏则开始了由军阀权臣掌控的时代,直接导致了国家的分裂。

元氏的皇帝身份常有爭議,就事實上是當了皇帝沒錯,但短暫的女皇帝身份普遍不被后世所承认,而且史书中,尤其是在正史中从来不把她列入正统的帝系,一来是因为她是胡太后的傀儡,且在位时间只有一天不到,二来是因为她是以冒名男婴而即帝位的。因此,至今为止,武周王朝的武则天仍然被普遍认为是中国历史上的第一位、也是唯一的一位女皇帝。不过,学者成扬则认为,元氏的登基虽然是北魏统治集团内部权力斗争的产物尤其是胡太后一手造成的,但其作为“第一个登上皇帝宝座的女性,这一事实却不容抹杀”,并建议将武则天的身份从“(中国)历史上唯一的女皇帝”修改为“中国历史上有作为的女皇帝”。对于成扬的说法,另一位研究武则天的专家罗元贞予以驳斥,他认为元氏的女皇帝地位连封建时代的史家都不承认,“在社会主义的新中国居然承认”,斥之为“标新立异、沽名钓誉之一例”,他本人则坚持认为武则天才是“真正的”中国第一位、也是唯一的一位女皇帝。


\subsection{幼主生平}

元钊(526年-528年5月17日),河南郡洛阳县(今河南省洛阳市东)人,魏孝文帝元宏曾孙,临洮王元寶暉之子,北魏皇帝。

魏孝明帝元诩厌恶灵太后的宠臣郑俨、徐纥等人,因为受到灵太后的威逼,不能将他们赶走,就秘密的命令尔朱荣带领军队前往洛阳,希望以此胁迫灵太后。尔朱荣以高欢为先锋,抵达上党,魏孝明帝又下诏让尔朱荣停止进军。郑俨和徐纥担心祸事惹到身上,就阴谋与灵太后用毒酒谋害魏孝明帝。武泰元年二月癸丑(528年3月31日),魏孝明帝突然死去。二月甲寅(528年4月1日),灵太后立魏孝明帝之女元氏为皇帝,大赦。既而下诏称:“潘充华本来生的是女儿,已故临洮王元宝晖的世子元钊,是高祖的后裔,适合接受皇位。”二月乙卯(528年4月2日),元钊即位。元钊当年才虚龄三岁,灵太后希望长久的独揽大权,所以贪图元钊年幼而册立他为皇帝。

建义元年四月庚子(528年5月17日),尔朱荣派遣骑兵逮捕灵太后和元钊,送到河阴,灵太后对尔朱荣多方辩解自己的行为,尔朱荣拂袖起身,灵太后和元钊都被沉入黄河,灵太后的妹妹胡玄辉将灵太后和元钊的尸体收敛在双灵寺中。

\subsection{孝庄帝\tiny(528-530)}

\subsubsection{孝庄帝生平}

魏孝莊帝元子攸(507年-531年1月26日),字彦达,河南郡洛阳县(今河南省洛阳市东)人,魏献文帝拓跋弘之孙,彭城王元勰第三子,母親為王妃李媛華。孝莊帝是尔朱荣拥立的傀儡皇帝,最终被尔朱兆绞杀。

元子攸姿貌很俊美,有勇力。自幼在宫为孝明帝元诩担任伴读,与魏孝明帝颇为友爱,官至中书侍郎,封武城县开国公,527年,被特封长乐王。

528年5月15日(農曆四月十一日、建義元年),元子攸被尔朱荣拥立为皇帝。

孝莊帝永安二年(529年)鑄永安五銖錢。九月二十五日(530年11月1日、永安三年),孝莊帝伏兵明光殿,聲稱皇后大尔朱氏生下了太子,派元徽向爾朱榮報喜。爾朱榮跟元天穆一起入朝。元子攸听说尔朱荣进宫臉色緊張,連忙喝酒以遮掩。尔朱荣见到光祿少卿魯安、典御李侃晞從東廂門执刀闖入,便撲向元子攸。元子攸用藏在膝下的刀砍到尔朱荣,魯安等揮刀亂砍,殺爾朱榮與元天穆等人。

十月三十日(530年12月5日、永安三年),爾朱兆另立元曄為帝。十二月三日(531年1月6日),爾朱兆攻入洛陽,殺死孝莊帝在襁褓中的儿子,孝莊帝被俘囚於永寧寺、後解送囚於晉陽三級寺。

永安三年十二月甲子(531年1月26日),魏孝莊帝於晋阳城(今太原市晋源区境)三級寺被尔朱兆絞殺,虚岁二十四。臨終前孝莊帝向佛祖禮拜,發願生生世世不做皇帝,並賦詩明志:「權去生道促,憂來死路長。懷恨出國門,含悲入鬼鄉。隧門一時閉,幽庭豈復光。思鳥吟青松,哀風吹白楊。昔來聞死苦,何言身自當。」中兴二年(532年),元朗给元子攸上谥号为武怀皇帝,魏孝武帝元修即位后,因为武怀谥号犯魏孝武帝父亲元怀名讳,于太昌元年(532年)改元子攸谥号为孝庄皇帝,庙号敬宗。十一月,葬于静陵。

\subsubsection{建义}

\begin{longtable}{|>{\centering\scriptsize}m{2em}|>{\centering\scriptsize}m{1.3em}|>{\centering}m{8.8em}|}
  % \caption{秦王政}\
  \toprule
  \SimHei \normalsize 年数 & \SimHei \scriptsize 公元 & \SimHei 大事件 \tabularnewline
  % \midrule
  \endfirsthead
  \toprule
  \SimHei \normalsize 年数 & \SimHei \scriptsize 公元 & \SimHei 大事件 \tabularnewline
  \midrule
  \endhead
  \midrule
  元年 & 528 & \tabularnewline
  \bottomrule
\end{longtable}

\subsubsection{永安}

\begin{longtable}{|>{\centering\scriptsize}m{2em}|>{\centering\scriptsize}m{1.3em}|>{\centering}m{8.8em}|}
  % \caption{秦王政}\
  \toprule
  \SimHei \normalsize 年数 & \SimHei \scriptsize 公元 & \SimHei 大事件 \tabularnewline
  % \midrule
  \endfirsthead
  \toprule
  \SimHei \normalsize 年数 & \SimHei \scriptsize 公元 & \SimHei 大事件 \tabularnewline
  \midrule
  \endhead
  \midrule
  元年 & 528 & \tabularnewline\hline
  二年 & 529 & \tabularnewline\hline
  三年 & 530 & \tabularnewline
  \bottomrule
\end{longtable}


%%% Local Variables:
%%% mode: latex
%%% TeX-engine: xetex
%%% TeX-master: "../../Main"
%%% End:

%% -*- coding: utf-8 -*-
%% Time-stamp: <Chen Wang: 2021-11-01 15:12:37>

\subsection{东海王元晔\tiny(530-531)}

\subsubsection{生平}

元晔(?-532年12月26日),字华兴,小字盆子,河南郡洛阳县(今河南省洛阳市东)人,追尊魏景穆帝拓跋晃曾孙,鄯善镇将、扶风王元怡之子,北魏宗室、官员,一度被尔朱兆与尔朱世隆拥立为皇帝。

元晔性格轻浮急躁,有体力,以秘书郎为起家官,略微升迁至通直散骑常侍。建义元年四月甲辰(528年5月21日),魏孝庄帝元子攸封秘书郎中元晔为长广王,食邑一千户,外任太原郡太守,代理并州刺史。尔朱荣死后,尔朱世隆逃回并州的建兴郡高都县,尔朱兆从晋阳县前来汇合,于是在永安三年十月壬申(530年12月5日)推举元晔为皇帝,大赦所管辖的地区,年号建明,所有官员加四级。元晔小名盆子,听说此事的人都认为类似赤眉军之事。

建明元年十二月戊申(531年1月10日),元晔在攻克洛阳后大赦天下。尔朱世隆与兄弟密谋,担心元晔的母亲卫氏干预朝政,观察卫氏出行,派遣数十骑兵装扮成劫匪,在京城小巷子里杀死了卫氏。公家和私人都感到惊愕,不知道什么原因。很快又张贴公告悬赏,以一千万钱悬赏劫匪。百姓知道后,没有不垂头丧气的。尔朱世隆很快认为元晔是北魏宗室远支,又不是众望所推的人,想要推举宗室近支广陵王元恭为皇帝。建明元年春二月己巳(531年4月1日),元晔前往邙山南,尔朱世隆等人在洛阳东城外奉迎广陵王元恭,尔朱世隆等人写好禅让的册文,以泰山郡太守窦瑗执马鞭进入行宫,启奏元晔说:“上天和百姓的愿望,都在广陵王身上,请实行尧舜禅让的礼仪。”元晔因此退位。

魏节闵帝元恭登基后,于普泰元年三月癸酉(531年4月5日)封元晔为东海王,食邑一万户。太昌元年十一月甲辰(532年12月26日),安定王元朗和元晔在家中被赐令自杀。元晔的爵位被削除。

\subsubsection{建明}

\begin{longtable}{|>{\centering\scriptsize}m{2em}|>{\centering\scriptsize}m{1.3em}|>{\centering}m{8.8em}|}
  % \caption{秦王政}\
  \toprule
  \SimHei \normalsize 年数 & \SimHei \scriptsize 公元 & \SimHei 大事件 \tabularnewline
  % \midrule
  \endfirsthead
  \toprule
  \SimHei \normalsize 年数 & \SimHei \scriptsize 公元 & \SimHei 大事件 \tabularnewline
  \midrule
  \endhead
  \midrule
  元年 & 530 & \tabularnewline\hline
  二年 & 531 & \tabularnewline
  \bottomrule
\end{longtable}


%%% Local Variables:
%%% mode: latex
%%% TeX-engine: xetex
%%% TeX-master: "../../Main"
%%% End:

%% -*- coding: utf-8 -*-
%% Time-stamp: <Chen Wang: 2021-11-01 15:12:42>

\subsection{节闵帝元恭\tiny(531)}

\subsubsection{生平}

魏节闵帝元恭(498年-532年6月21日;在位531年-532年),字修业,河南郡洛阳县(今河南省洛阳市东)人,魏献文帝拓跋弘之孙,广陵惠王元羽之子,母王氏,是南北朝时期北魏的皇帝。

先前元恭的伯父北魏孝文帝元宏为鼓励鲜卑与汉族通婚而将元羽嫡妻降为妾室,另聘娶郑始容为元羽嫡妻。史书没有记载元恭的母亲王氏是否就是元羽被降为妾室的元配,仅记载元羽死后袭爵广陵王的是元恭,而非元恭的哥哥元欣。

建明二年二月廿九日(531年4月1日),爾朱榮堂弟爾朱世隆废元晔,立元恭為帝。军阀高欢则立渤海太守安定王元朗为帝。高欢打败尔朱氏后,考虑到元朗世系疏远,一度想尊奉元恭,派仆射魏兰根慰谕洛阳观察节闵帝为人。魏兰根认为节闵帝神采高明,日后难制,与侍中、司空高乾兄弟及黄门侍郎崔㥄强调节闵帝系尔朱氏所立,共劝高欢为了讨伐尔朱氏师出有名而废帝。532年6月(农历四月),节闵帝被高欢所废,囚禁于崇训佛寺。元朗亦被高欢所迫禅位给平阳王元修即北魏孝武帝。节闵帝赋诗:“朱門久可患,紫極非情玩。顛覆立可待,一年三易換。時運正如此,唯有修真觀。”

五月丙申(532年6月21日),魏节闵帝在门下外省被孝武帝毒死,虚岁三十五,葬以亲王殊礼;加九旒、銮辂、黄屋、左纛,班剑百二十人,二卫、羽林备仪卫。东魏称广陵王或前废帝,西魏谥节闵帝。

2013年,洛阳市文物考古研究院完成北魏节愍帝的陵墓考古挖掘工作,陵墓中出土东罗马帝国金币一枚。

\subsubsection{普泰}

\begin{longtable}{|>{\centering\scriptsize}m{2em}|>{\centering\scriptsize}m{1.3em}|>{\centering}m{8.8em}|}
  % \caption{秦王政}\
  \toprule
  \SimHei \normalsize 年数 & \SimHei \scriptsize 公元 & \SimHei 大事件 \tabularnewline
  % \midrule
  \endfirsthead
  \toprule
  \SimHei \normalsize 年数 & \SimHei \scriptsize 公元 & \SimHei 大事件 \tabularnewline
  \midrule
  \endhead
  \midrule
  元年 & 531 & \tabularnewline
  \bottomrule
\end{longtable}


%%% Local Variables:
%%% mode: latex
%%% TeX-engine: xetex
%%% TeX-master: "../../Main"
%%% End:

%% -*- coding: utf-8 -*-
%% Time-stamp: <Chen Wang: 2019-12-23 15:21:36>

\subsection{后废帝\tiny(531-532)}

\subsubsection{生平}

元朗(513年-532年12月26日;在位531年-532年),字仲哲,河南郡洛阳县(今河南省洛阳市东)人,追尊魏景穆帝拓跋晃玄孙,章武王元融的儿子,是中國南北朝时期北魏的第14代君主,無廟號,史稱後廢帝。

元朗在531年10月30日(农历十月六日)被高欢立为皇帝,532年6月(农历四月)高欢因元朗世系疏远,迫使他让位于孝武帝元修。太昌元年十一月甲辰(532年12月26日),元朗和元晔都在门下外省被杀死,虚岁二十。永熙二年葬于鄴西南野馬岡。

\subsubsection{中兴}

\begin{longtable}{|>{\centering\scriptsize}m{2em}|>{\centering\scriptsize}m{1.3em}|>{\centering}m{8.8em}|}
  % \caption{秦王政}\
  \toprule
  \SimHei \normalsize 年数 & \SimHei \scriptsize 公元 & \SimHei 大事件 \tabularnewline
  % \midrule
  \endfirsthead
  \toprule
  \SimHei \normalsize 年数 & \SimHei \scriptsize 公元 & \SimHei 大事件 \tabularnewline
  \midrule
  \endhead
  \midrule
  元年 & 531 & \tabularnewline\hline
  二年 & 532 & \tabularnewline
  \bottomrule
\end{longtable}


%%% Local Variables:
%%% mode: latex
%%% TeX-engine: xetex
%%% TeX-master: "../../Main"
%%% End:

%% -*- coding: utf-8 -*-
%% Time-stamp: <Chen Wang: 2019-12-23 15:22:30>

\subsection{孝武帝\tiny(532-534)}

\subsubsection{生平}

魏孝武帝元修,一说元脩(510年-535年2月3日),字孝則,河南郡洛阳县(今河南省洛阳市东)人,魏孝文帝元宏之孙,廣平武穆王元懷第三子,母親是李氏,是北魏最後一位皇帝。

他曾被封為汝陽縣公、通直散騎侍郎、中書侍郎。建義年間,辭散騎常侍,為平東將軍、太常卿,後來又為鎮東將軍、宗正卿。530年封為平陽王。普泰初年,轉任侍中、鎮東將軍、儀同三司、兼為尚書右僕射,後又改加侍中、尚書左僕射。河阴之变后政局一片混乱,诸王大多各自逃生,元修逃亡民间,隐为乡农。

中興二年(532年)高欢击败尔朱氏,欲立元悦为帝,因无法服众,只得退而选择元修。斛斯椿从元修的心腹好友王思政辗转寻到元修,元修道:“这该不是把我出卖了吧?”高欢遂亲自前来陳誠,泣下沾襟。元修方才入京,四月廿五日(6月13日)即位。

元修即位后与高欢的长女结婚,夫妻彼此都没有感情。元修与三个堂姊妹姘居,将她们都封为公主。534年與高歡決裂,高歡帶兵從晉陽南下时,元修于七月廿八日(8月22日)率一部分兵众,偕同情妇元明月及元明月的哥哥元宝炬等入關中投奔其妹妹的未婚夫宇文泰。

先前梁武帝因天象称“荧惑入南斗,天子下殿走”,就赤脚下殿以应天象,得知元修西奔,羞惭地说:“索虏也应天象吗?”因为这一天象应在元修身上就意味着上天认为元修才是正统天子。

同年十月十七日(11月8日),高歡以元修弃国逃跑,另立元善見為帝,十日后遷都鄴。宇文泰让元氏诸王杀死元明月,元修非常不高兴,闰十二月癸巳(535年2月3日),宇文泰用毒酒毒死了元修,改立元宝炬为帝。

元修死後被宇文泰下令埋进草堂佛寺,十余年后才得正式落葬。西魏上諡號為孝武皇帝。

\subsubsection{太昌}

\begin{longtable}{|>{\centering\scriptsize}m{2em}|>{\centering\scriptsize}m{1.3em}|>{\centering}m{8.8em}|}
  % \caption{秦王政}\
  \toprule
  \SimHei \normalsize 年数 & \SimHei \scriptsize 公元 & \SimHei 大事件 \tabularnewline
  % \midrule
  \endfirsthead
  \toprule
  \SimHei \normalsize 年数 & \SimHei \scriptsize 公元 & \SimHei 大事件 \tabularnewline
  \midrule
  \endhead
  \midrule
  元年 & 532 & \tabularnewline
  \bottomrule
\end{longtable}

\subsubsection{永兴}

\begin{longtable}{|>{\centering\scriptsize}m{2em}|>{\centering\scriptsize}m{1.3em}|>{\centering}m{8.8em}|}
  % \caption{秦王政}\
  \toprule
  \SimHei \normalsize 年数 & \SimHei \scriptsize 公元 & \SimHei 大事件 \tabularnewline
  % \midrule
  \endfirsthead
  \toprule
  \SimHei \normalsize 年数 & \SimHei \scriptsize 公元 & \SimHei 大事件 \tabularnewline
  \midrule
  \endhead
  \midrule
  元年 & 532 & \tabularnewline
  \bottomrule
\end{longtable}

\subsubsection{永熙}

\begin{longtable}{|>{\centering\scriptsize}m{2em}|>{\centering\scriptsize}m{1.3em}|>{\centering}m{8.8em}|}
  % \caption{秦王政}\
  \toprule
  \SimHei \normalsize 年数 & \SimHei \scriptsize 公元 & \SimHei 大事件 \tabularnewline
  % \midrule
  \endfirsthead
  \toprule
  \SimHei \normalsize 年数 & \SimHei \scriptsize 公元 & \SimHei 大事件 \tabularnewline
  \midrule
  \endhead
  \midrule
  元年 & 532 & \tabularnewline\hline
  二年 & 533 & \tabularnewline\hline
  三年 & 534 & \tabularnewline
  \bottomrule
\end{longtable}


%%% Local Variables:
%%% mode: latex
%%% TeX-engine: xetex
%%% TeX-master: "../../Main"
%%% End:



%%% Local Variables:
%%% mode: latex
%%% TeX-engine: xetex
%%% TeX-master: "../../Main"
%%% End:
