%% -*- coding: utf-8 -*-
%% Time-stamp: <Chen Wang: 2019-12-23 15:16:05>

\subsection{宣武帝\tiny(499-515)}

\subsubsection{生平}

魏宣武帝元恪(483年-515年2月12日),河南郡洛阳县(今河南省洛阳市东)人,魏孝文帝元宏次子,生母貴人高照容,是南北朝时期北魏的第八代皇帝。499年-515年在位,在位十六年。

太和七年(483年),異母兄元恂之生母林氏按北魏子貴母死之慣例而被賜死,元恂由嫡曾祖母馮太后養育。太和十七年(493年)七月,長兄元恂被立為皇太子。孝文帝遠征南齊,十歲的元恂留守新都洛陽。元恂嫌河南酷暑,穿胡服。

太和二十年(496年),十三歲的元恂逃至平城,得到反對漢化和南遷的貴族的支持。其父孝文帝返回後平息了變亂,廢黜元恂為庶人,囚禁在河陽,衣食僅夠維生。不久,又派人將元恂賜死。元恪的長兄元恂死時年僅15歲。

太和二十二年(498年),孝文帝改立十六歲的元恪為皇太子。

翌年,孝文帝崩。十七歲的元恪即位。

宣武帝的統治初期(499年至508年),他的叔父北魏宗室咸陽王元禧(獻文帝拓跋弘次子,孝文帝元宏之異母弟)輔政、尚書令王肅輔佐。

北魏對南朝發動了一系列戰爭,攻取南朝梁的四川之地、北撃柔然,北魏疆域大大向南拓展,國勢盛極一時。因篤信佛教,宣武帝取消子貴母死制度,讓宣武靈皇后活著。

他在位的後半期,外戚高肇專權,朝政一片黑暗,北魏逐漸衰弱。延昌四年正月,宣武帝崩於式乾殿。

\subsubsection{景明}

\begin{longtable}{|>{\centering\scriptsize}m{2em}|>{\centering\scriptsize}m{1.3em}|>{\centering}m{8.8em}|}
  % \caption{秦王政}\
  \toprule
  \SimHei \normalsize 年数 & \SimHei \scriptsize 公元 & \SimHei 大事件 \tabularnewline
  % \midrule
  \endfirsthead
  \toprule
  \SimHei \normalsize 年数 & \SimHei \scriptsize 公元 & \SimHei 大事件 \tabularnewline
  \midrule
  \endhead
  \midrule
  元年 & 500 & \tabularnewline\hline
  二年 & 501 & \tabularnewline\hline
  三年 & 502 & \tabularnewline\hline
  四年 & 503 & \tabularnewline\hline
  五年 & 504 & \tabularnewline
  \bottomrule
\end{longtable}

\subsubsection{正始}

\begin{longtable}{|>{\centering\scriptsize}m{2em}|>{\centering\scriptsize}m{1.3em}|>{\centering}m{8.8em}|}
  % \caption{秦王政}\
  \toprule
  \SimHei \normalsize 年数 & \SimHei \scriptsize 公元 & \SimHei 大事件 \tabularnewline
  % \midrule
  \endfirsthead
  \toprule
  \SimHei \normalsize 年数 & \SimHei \scriptsize 公元 & \SimHei 大事件 \tabularnewline
  \midrule
  \endhead
  \midrule
  元年 & 504 & \tabularnewline\hline
  二年 & 505 & \tabularnewline\hline
  三年 & 506 & \tabularnewline\hline
  四年 & 507 & \tabularnewline\hline
  五年 & 508 & \tabularnewline
  \bottomrule
\end{longtable}

\subsubsection{永平}

\begin{longtable}{|>{\centering\scriptsize}m{2em}|>{\centering\scriptsize}m{1.3em}|>{\centering}m{8.8em}|}
  % \caption{秦王政}\
  \toprule
  \SimHei \normalsize 年数 & \SimHei \scriptsize 公元 & \SimHei 大事件 \tabularnewline
  % \midrule
  \endfirsthead
  \toprule
  \SimHei \normalsize 年数 & \SimHei \scriptsize 公元 & \SimHei 大事件 \tabularnewline
  \midrule
  \endhead
  \midrule
  元年 & 508 & \tabularnewline\hline
  二年 & 509 & \tabularnewline\hline
  三年 & 510 & \tabularnewline\hline
  四年 & 511 & \tabularnewline\hline
  五年 & 512 & \tabularnewline
  \bottomrule
\end{longtable}

\subsubsection{延昌}

\begin{longtable}{|>{\centering\scriptsize}m{2em}|>{\centering\scriptsize}m{1.3em}|>{\centering}m{8.8em}|}
  % \caption{秦王政}\
  \toprule
  \SimHei \normalsize 年数 & \SimHei \scriptsize 公元 & \SimHei 大事件 \tabularnewline
  % \midrule
  \endfirsthead
  \toprule
  \SimHei \normalsize 年数 & \SimHei \scriptsize 公元 & \SimHei 大事件 \tabularnewline
  \midrule
  \endhead
  \midrule
  元年 & 512 & \tabularnewline\hline
  二年 & 513 & \tabularnewline\hline
  三年 & 514 & \tabularnewline\hline
  四年 & 515 & \tabularnewline
  \bottomrule
\end{longtable}


%%% Local Variables:
%%% mode: latex
%%% TeX-engine: xetex
%%% TeX-master: "../../Main"
%%% End:
