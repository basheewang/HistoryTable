%% -*- coding: utf-8 -*-
%% Time-stamp: <Chen Wang: 2021-11-01 15:10:27>

\subsection{献文帝拓跋弘\tiny(465-471)}

\subsubsection{生平}

魏獻文帝拓跋弘(454年-476年),鮮卑名第豆胤,魏文成帝拓跋濬長子,南北朝時期北魏第六位皇帝。

太安二年(456年)正月,立为太子。

和平六年(465年)五月,父亲拓跋濬逝世,随后,拓跋弘登基为帝。

皇興元年(467年)八月,獻文帝之妃李夫人誕下拓跋宏。   

皇興三年(469年)六月,李夫人被按照子貴母死的制度殺死。葬在金陵,承明元年(476年)追諡思皇后,配饗太廟。

太和二年(478年)十二月,李夫人全家都被馮太后處死。

皇興五年(471年),年僅17歲的魏献文帝因不滿馮太后長期攝政及專權,原本要禪讓給三叔京兆王拓跋子推,被眾臣勸阻後作罷,禪讓予太子拓跋宏,且因群臣奏称“三皇澹泊无为,所以称皇;西汉高祖之父被尊为太上皇,是不统治天下的,而皇帝年幼,陛下仍然执政”,不称太上皇,而称尊号太上皇帝。雖然作為太上皇,但魏献文帝仍掌握有部分的皇帝權力,並欲與馮太后爭權。

延興二年(472年),柔然來犯,魏献文帝以太上皇之姿,御駕親征,大敗柔然,並追至大漠。

承明元年(476年),拓跋弘被馮太后毒死,崩於永安殿,年僅二十三歲,諡曰獻文皇帝,廟號顯祖,葬於金陵。

\subsubsection{天安}

\begin{longtable}{|>{\centering\scriptsize}m{2em}|>{\centering\scriptsize}m{1.3em}|>{\centering}m{8.8em}|}
  % \caption{秦王政}\
  \toprule
  \SimHei \normalsize 年数 & \SimHei \scriptsize 公元 & \SimHei 大事件 \tabularnewline
  % \midrule
  \endfirsthead
  \toprule
  \SimHei \normalsize 年数 & \SimHei \scriptsize 公元 & \SimHei 大事件 \tabularnewline
  \midrule
  \endhead
  \midrule
  元年 & 466 & \tabularnewline\hline
  二年 & 467 & \tabularnewline
  \bottomrule
\end{longtable}

\subsubsection{皇兴}

\begin{longtable}{|>{\centering\scriptsize}m{2em}|>{\centering\scriptsize}m{1.3em}|>{\centering}m{8.8em}|}
  % \caption{秦王政}\
  \toprule
  \SimHei \normalsize 年数 & \SimHei \scriptsize 公元 & \SimHei 大事件 \tabularnewline
  % \midrule
  \endfirsthead
  \toprule
  \SimHei \normalsize 年数 & \SimHei \scriptsize 公元 & \SimHei 大事件 \tabularnewline
  \midrule
  \endhead
  \midrule
  元年 & 467 & \tabularnewline\hline
  二年 & 468 & \tabularnewline\hline
  三年 & 469 & \tabularnewline\hline
  四年 & 470 & \tabularnewline\hline
  五年 & 471 & \tabularnewline
  \bottomrule
\end{longtable}


%%% Local Variables:
%%% mode: latex
%%% TeX-engine: xetex
%%% TeX-master: "../../Main"
%%% End:
