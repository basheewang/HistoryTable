%% -*- coding: utf-8 -*-
%% Time-stamp: <Chen Wang: 2019-12-23 15:19:26>

\subsection{元氏生平}

元氏(528年2月12日-?),女,河南洛阳(今河南省洛阳市东)人,真名不详,南北朝时期北魏第10位皇帝(未获后世普遍承认),是北魏孝明帝元詡与宫嫔潘外憐的女儿,也是孝明帝唯一的骨肉。出生后因时局危险,所以她的祖母、掌握帝国实际大权的皇太后胡氏对外宣称本为女性(即皇女)的她是男性(即为皇子),以安人心。不久,孝明帝暴崩,尚在襁褓中的“皇子”元氏以先帝唯一子嗣的身份继位(528年4月1日),在名义上成为了北魏皇帝。元氏即位当天便被废黜,次日由幼主元钊继位,之后史书上便没有了对她的记载。

北魏孝明帝孝昌四年春正月初七乙丑日(528年2月12日),皇女元氏出生。她是孝明帝第一个、也是唯一的孩子,她的母亲是孝明帝九嫔之一的充华潘外憐,而她的祖母即为临朝称制的灵太后胡氏。胡太后对外宣称潘外憐生下的是皇子,并于第二天(正月初八丙寅日,即2月13日)颁布诏书,大赦天下,改元武泰。胡太后之所以诈称皇女为皇子,和她与儿子元诩的矛盾有关。胡太后本为北魏宣武帝众妾之一的充华,但却为宣武帝生下了唯一存活下来的子嗣,即后来继位的元诩。宣武帝死后,元诩继皇帝位,是为孝明帝,尊胡充华为皇太妃,后又加尊为皇太后。因孝明帝年幼,胡太后临朝称制。

但孝明帝日渐长大,胡太后却不肯放权归政,而且在政治上排斥异己,生活上不检点,不但引起朝臣反感,连孝明帝对她也大为不满,甚至招致天下人的厌恶,尤其是孝明帝把与胡太后私通的清河王元怿处死后,胡太后对儿子也恨之入骨,母子之间的裂痕越来越深。经过了几次反对胡太后的失败的政变,孝明帝密令大将尔朱荣进兵首都洛阳(今河南省洛阳市),试图胁迫胡太后归政。胡太后知道后,与近臣商议对策,适逢孝明帝的充華潘外憐生了皇女,所以胡太后假称是皇子,大赦天下,转移朝臣的视线,暗中谋划除去孝明帝。

武泰元年二月廿五癸丑日(528年3月31日),孝明帝元詡突然驾崩,是被胡太后暗中串通近臣用毒酒毒死的。第二天(即武泰元年二月廿六甲寅日,528年4月1日),胡太后伪称皇女元氏为皇太子,拥立元氏登基为皇帝,胡太后继续临朝称制,再次大赦天下。当时的元氏出生刚满50天。由于没有跨年,未改元,仍然沿用“武泰”年号。这样,元氏成为名义上的北魏皇帝。当然,由于这位女婴皇帝才出生一个多月,不可能真正地行使皇帝权力,实权仍然掌握在她的祖母、临朝称制的胡太后手中,元氏只不过是胡太后的傀儡。

胡太后立皇女元氏为帝才一天不到,见人心已经安定,当天便发下诏书宣布皇帝本是女儿身,废黜女婴皇帝,改立已故宗室临洮王元宝晖的世子元钊为皇帝。

胡太后的诏书全文如下:“皇家握历受图,年将二百;祖宗累圣,社稷载安。高祖以文思先天,世宗以下武经世,股肱惟良,元首穆穆。及大行在御,重以宽仁,奉养率由,温明恭顺。朕以寡昧,亲临万国,识谢涂山,德惭文母。属妖逆递兴,四郊多故。实望穹灵降祐,麟趾众繁。自潘充华有孕椒宫,冀诞储两,而熊罴无兆,维虺遂彰。于时直以国步未康,假称统胤,欲以底定物情,系仰宸极。何图一旦,弓剑莫追,国道中微,大行绝祀。皇曾孙故临洮王宝晖世子钊,体自高祖,天表卓异,大行平日养爱特深,义齐若子,事符当璧。及翊日弗愈,大渐弥留,乃延入青蒲,受命玉几。暨陈衣在庭,登策靡及,允膺大宝,即日践阼。朕是用惶惧忸怩,心焉靡洎。今丧君有君,宗祏惟固,宜崇赏卿士,爰及百辟,凡厥在位,并加陟叙。内外百官文武、督将征人,遭艰解府,普加军功二阶;其禁卫武官,直阁以下直从以上及主帅,可军功三阶;其亡官失爵,听复封位。谋反大逆削除者,不在斯限。清议禁锢,亦悉蠲除。若二品以上不能自受者,任授兒弟。可班宣远迩,咸使知之。”

胡太后诏书的大意就是孝明帝死得仓促,来不及指定继承人,只好让他唯一的女儿暂且以皇子身份继承皇位,后来发现元钊是皇位的合适人选,便废女婴皇帝而让元钊当皇帝。

元钊于胡太后发下诏书后的第二天(即武泰元年二月廿七乙卯日,528年4月2日)正式即位,是为北魏幼主。

实际上,这一废一立都是胡太后试图长期掌握帝国最高权力而使出的手段,因为元钊虽然比女婴皇帝大几岁,但也只有三岁,胡太后立他为皇帝是因为他年幼不能管理国家,她可以继续临朝称制,统治天下了。实际上,胡太后自被孝明帝尊为太后开始就是北魏的实际统治者了,她不但临朝称制,还自称为“朕”(秦始皇以后的皇帝自称),让朝臣们称她为“陛下”(臣下对皇帝的尊称)。她不惜先毒杀亲生儿子,后立尚在襁褓中的孙女,再立刚满三岁的宗室幼子,就是因为她是妇人之身,在当时的情况下不能直接行使皇帝权力,立年幼无知的傀儡皇帝可以保证自己通过临朝称制的方法继续掌权专制,而其最后终于身败名裂,遂被后世认为是不成功的野心家和导致北魏分裂亡国的罪人。

女婴皇帝被废後無紀錄可循,再也不知所終,但她因為即位為帝,導致的政治影響很大。

而短时间的废立让天下震惊,認定胡太后害死孝明帝,大将尔朱荣遂以胡太后肆意废立为藉口带兵讨伐,而其中一条重要的理由就是胡太后欺瞒上天和朝臣,立女婴为帝。尔朱荣又另立元子攸为皇帝,是为孝莊帝,这样北魏出现了两帝并立的局面。不过这种局面很快就被打破,15日后,尔朱荣的军队占领京师洛阳,胡太后和幼主元钊被俘。尔朱荣将幼主和胡太后押送至黄河南岸边。胡太后在尔朱荣面前说了许多好话求饶,尔朱荣不听,下令将幼主和胡太后沉入黄河,后又屠杀大臣两千多人,史称河阴之变。从此,尔朱荣完全掌握了北魏的实际大权,而北魏则开始了由军阀权臣掌控的时代,直接导致了国家的分裂。

元氏的皇帝身份常有爭議,就事實上是當了皇帝沒錯,但短暫的女皇帝身份普遍不被后世所承认,而且史书中,尤其是在正史中从来不把她列入正统的帝系,一来是因为她是胡太后的傀儡,且在位时间只有一天不到,二来是因为她是以冒名男婴而即帝位的。因此,至今为止,武周王朝的武则天仍然被普遍认为是中国历史上的第一位、也是唯一的一位女皇帝。不过,学者成扬则认为,元氏的登基虽然是北魏统治集团内部权力斗争的产物尤其是胡太后一手造成的,但其作为“第一个登上皇帝宝座的女性,这一事实却不容抹杀”,并建议将武则天的身份从“(中国)历史上唯一的女皇帝”修改为“中国历史上有作为的女皇帝”。对于成扬的说法,另一位研究武则天的专家罗元贞予以驳斥,他认为元氏的女皇帝地位连封建时代的史家都不承认,“在社会主义的新中国居然承认”,斥之为“标新立异、沽名钓誉之一例”,他本人则坚持认为武则天才是“真正的”中国第一位、也是唯一的一位女皇帝。


\subsection{幼主生平}

元钊(526年-528年5月17日),河南郡洛阳县(今河南省洛阳市东)人,魏孝文帝元宏曾孙,临洮王元寶暉之子,北魏皇帝。

魏孝明帝元诩厌恶灵太后的宠臣郑俨、徐纥等人,因为受到灵太后的威逼,不能将他们赶走,就秘密的命令尔朱荣带领军队前往洛阳,希望以此胁迫灵太后。尔朱荣以高欢为先锋,抵达上党,魏孝明帝又下诏让尔朱荣停止进军。郑俨和徐纥担心祸事惹到身上,就阴谋与灵太后用毒酒谋害魏孝明帝。武泰元年二月癸丑(528年3月31日),魏孝明帝突然死去。二月甲寅(528年4月1日),灵太后立魏孝明帝之女元氏为皇帝,大赦。既而下诏称:“潘充华本来生的是女儿,已故临洮王元宝晖的世子元钊,是高祖的后裔,适合接受皇位。”二月乙卯(528年4月2日),元钊即位。元钊当年才虚龄三岁,灵太后希望长久的独揽大权,所以贪图元钊年幼而册立他为皇帝。

建义元年四月庚子(528年5月17日),尔朱荣派遣骑兵逮捕灵太后和元钊,送到河阴,灵太后对尔朱荣多方辩解自己的行为,尔朱荣拂袖起身,灵太后和元钊都被沉入黄河,灵太后的妹妹胡玄辉将灵太后和元钊的尸体收敛在双灵寺中。

\subsection{孝庄帝\tiny(528-530)}

\subsubsection{孝庄帝生平}

魏孝莊帝元子攸(507年-531年1月26日),字彦达,河南郡洛阳县(今河南省洛阳市东)人,魏献文帝拓跋弘之孙,彭城王元勰第三子,母親為王妃李媛華。孝莊帝是尔朱荣拥立的傀儡皇帝,最终被尔朱兆绞杀。

元子攸姿貌很俊美,有勇力。自幼在宫为孝明帝元诩担任伴读,与魏孝明帝颇为友爱,官至中书侍郎,封武城县开国公,527年,被特封长乐王。

528年5月15日(農曆四月十一日、建義元年),元子攸被尔朱荣拥立为皇帝。

孝莊帝永安二年(529年)鑄永安五銖錢。九月二十五日(530年11月1日、永安三年),孝莊帝伏兵明光殿,聲稱皇后大尔朱氏生下了太子,派元徽向爾朱榮報喜。爾朱榮跟元天穆一起入朝。元子攸听说尔朱荣进宫臉色緊張,連忙喝酒以遮掩。尔朱荣见到光祿少卿魯安、典御李侃晞從東廂門执刀闖入,便撲向元子攸。元子攸用藏在膝下的刀砍到尔朱荣,魯安等揮刀亂砍,殺爾朱榮與元天穆等人。

十月三十日(530年12月5日、永安三年),爾朱兆另立元曄為帝。十二月三日(531年1月6日),爾朱兆攻入洛陽,殺死孝莊帝在襁褓中的儿子,孝莊帝被俘囚於永寧寺、後解送囚於晉陽三級寺。

永安三年十二月甲子(531年1月26日),魏孝莊帝於晋阳城(今太原市晋源区境)三級寺被尔朱兆絞殺,虚岁二十四。臨終前孝莊帝向佛祖禮拜,發願生生世世不做皇帝,並賦詩明志:「權去生道促,憂來死路長。懷恨出國門,含悲入鬼鄉。隧門一時閉,幽庭豈復光。思鳥吟青松,哀風吹白楊。昔來聞死苦,何言身自當。」中兴二年(532年),元朗给元子攸上谥号为武怀皇帝,魏孝武帝元修即位后,因为武怀谥号犯魏孝武帝父亲元怀名讳,于太昌元年(532年)改元子攸谥号为孝庄皇帝,庙号敬宗。十一月,葬于静陵。

\subsubsection{建义}

\begin{longtable}{|>{\centering\scriptsize}m{2em}|>{\centering\scriptsize}m{1.3em}|>{\centering}m{8.8em}|}
  % \caption{秦王政}\
  \toprule
  \SimHei \normalsize 年数 & \SimHei \scriptsize 公元 & \SimHei 大事件 \tabularnewline
  % \midrule
  \endfirsthead
  \toprule
  \SimHei \normalsize 年数 & \SimHei \scriptsize 公元 & \SimHei 大事件 \tabularnewline
  \midrule
  \endhead
  \midrule
  元年 & 528 & \tabularnewline
  \bottomrule
\end{longtable}

\subsubsection{永安}

\begin{longtable}{|>{\centering\scriptsize}m{2em}|>{\centering\scriptsize}m{1.3em}|>{\centering}m{8.8em}|}
  % \caption{秦王政}\
  \toprule
  \SimHei \normalsize 年数 & \SimHei \scriptsize 公元 & \SimHei 大事件 \tabularnewline
  % \midrule
  \endfirsthead
  \toprule
  \SimHei \normalsize 年数 & \SimHei \scriptsize 公元 & \SimHei 大事件 \tabularnewline
  \midrule
  \endhead
  \midrule
  元年 & 528 & \tabularnewline\hline
  二年 & 529 & \tabularnewline\hline
  三年 & 530 & \tabularnewline
  \bottomrule
\end{longtable}


%%% Local Variables:
%%% mode: latex
%%% TeX-engine: xetex
%%% TeX-master: "../../Main"
%%% End:
