%% -*- coding: utf-8 -*-
%% Time-stamp: <Chen Wang: 2021-11-01 15:10:55>

\subsection{孝文帝元宏\tiny(471-499)}

\subsubsection{生平}

魏孝文帝元宏(467年10月13日-499年4月26日),本姓拓跋,是北魏献文帝拓跋弘的长子,北魏第七位皇帝(471年9月20日—499年4月26日在位),後改姓「元」,在位28年,卒年33岁,其所推行的孝文帝改革,以漢化運動為主體,俗稱孝文漢化,其改革措施有利于缓解民族隔阂和阶级矛盾,为社会经济的恢复和发展发挥积极作用。雖然因推動漢化急進而最終導致六鎮起義及北魏解體,卻為北朝的胡漢融合作出貢獻。

孝文帝去世后,庙号高祖,谥号孝文皇帝。

北魏献文帝皇兴元年八月二十九日(467年10月13日),元宏生于北魏首都平城(今山西省大同市)紫宫,生母李夫人。

皇兴五年八月二十日(471年9月20日),父親獻文帝禪讓于太华前殿,大赦,改元延兴元年。

元宏即位时只有5岁,獻文帝死後,由祖母冯太后攝政。冯太后是汉人,对鲜卑人建立的北魏朝廷进行了一系列中央集权化的改革,孝文帝便受此影响。

太和十四年(490年)冯太后去世后亲政,秉承馮太后的政策,繼續进行了汉化改革,而且做得比馮太后更大刀闊斧。他先整顿吏治,立三长法,实行均田制;太和十八年(494年),他以“南伐”为名迁都洛阳,全面改革鲜卑旧俗:规定以汉服代替鲜卑服,以汉语代替鲜卑语,迁洛鲜卑人以洛阳为籍贯,改鲜卑姓为汉姓,自己也改姓“元”。并鼓励鲜卑贵族与汉士族联姻,又参照南朝典章,修改北魏政治制度,并严厉镇压反对改革的守旧贵族,处死太子恂,这一举动使鲜卑经济、文化、社会、政治、军事等方面大大的发展,缓解了民族隔阂,史称“孝文帝改革”。

太和二十三年三月丙戌(499年4月6日),魏孝文帝在南征途中生病。三月庚子(499年4月20日),魏孝文帝病重。四月初一日(499年4月26日),孝文帝崩于谷塘原之行宫。孝文帝去世后,庙号高祖,谥号孝文皇帝。

陵墓位於河南省洛陽市孟津縣官庄村村東南800米處的兩個大型土丘,兩塚相距约100米,大塚是魏孝文帝陵墓「長陵」,小塚是第三位皇后文昭皇后(即宣武帝生母)的「終寧陵」。

然而孝文帝去世以后仅25年,北魏边镇鲜卑军事集团就发动反汉化运动六镇起义。534年,北魏分裂成东魏、西魏,之后更分别被北齐和北周取代。

北齐官修正史《魏书》魏收的评价是:“史臣曰:有魏始基代朔,廓平南夏,辟壤经世,咸以威武为业,文教之事,所未遑也。高祖幼承洪绪,早著睿圣之风。时以文明摄事,优游恭己,玄览独得,著自不言,神契所标,固以符于冥化。及躬总大政,一日万机,十许年间,曾不暇给;殊途同归,百虑一致。至夫生民所难行,人伦之高迹,虽尊居黄屋,尽蹈之矣。若乃钦明稽古,协御天人,帝王制作,朝野轨度,斟酌用舍,焕乎其有文章,海内生民咸受耳目之赐。加以雄才大略,爱奇好士,视下如伤,役己利物,亦无得而称之。其经纬天地,岂虚谥也!”

《資治通鑑》曰:高祖友愛諸弟,始終無間。嘗從容謂咸陽王禧等曰:「我後子孫解逅不肖,汝等觀望,可輔則輔之,不可輔則取之,勿為它人有也。」親任賢能,從善如流,精勤庶務,朝夕不倦。常曰:「人主患不能處心公平,推誠於物。能是二者,則胡、越之人皆可使如兄弟矣。」用法雖嚴,於大臣無所容貸,然人有小過,常多闊略。嘗於食中得蟲,又左右進羹誤傷帝手,皆笑而赦之。天地五郊、宗廟二分之祭,未嘗不身親其禮。每出巡遊及用兵,有司奏修道路,帝輒曰:「粗修橋樑,通車馬而已,勿去草鏟令平也。」在淮南行兵,如在境內,禁士卒無得踐傷粟稻;或伐民樹以供軍用,皆留絹償之。宮室非不得已不修,衣弊,浣濯而服之,鞍勒用鐵木而已。幼多力善射,能以指彈碎羊骨,射禽獸無不命中;及年十五,遂不復畋獵。常謂史官曰:「時事不可以不直書。人君威福在己,無能制之者;若史策復不書其惡,將何所畏忌邪!」

\subsubsection{延兴}

\begin{longtable}{|>{\centering\scriptsize}m{2em}|>{\centering\scriptsize}m{1.3em}|>{\centering}m{8.8em}|}
  % \caption{秦王政}\
  \toprule
  \SimHei \normalsize 年数 & \SimHei \scriptsize 公元 & \SimHei 大事件 \tabularnewline
  % \midrule
  \endfirsthead
  \toprule
  \SimHei \normalsize 年数 & \SimHei \scriptsize 公元 & \SimHei 大事件 \tabularnewline
  \midrule
  \endhead
  \midrule
  元年 & 471 & \tabularnewline\hline
  二年 & 472 & \tabularnewline\hline
  三年 & 473 & \tabularnewline\hline
  四年 & 474 & \tabularnewline\hline
  五年 & 475 & \tabularnewline\hline
  六年 & 476 & \tabularnewline
  \bottomrule
\end{longtable}

\subsubsection{承明}

\begin{longtable}{|>{\centering\scriptsize}m{2em}|>{\centering\scriptsize}m{1.3em}|>{\centering}m{8.8em}|}
  % \caption{秦王政}\
  \toprule
  \SimHei \normalsize 年数 & \SimHei \scriptsize 公元 & \SimHei 大事件 \tabularnewline
  % \midrule
  \endfirsthead
  \toprule
  \SimHei \normalsize 年数 & \SimHei \scriptsize 公元 & \SimHei 大事件 \tabularnewline
  \midrule
  \endhead
  \midrule
  元年 & 476 & \tabularnewline
  \bottomrule
\end{longtable}

\subsubsection{太和}

\begin{longtable}{|>{\centering\scriptsize}m{2em}|>{\centering\scriptsize}m{1.3em}|>{\centering}m{8.8em}|}
  % \caption{秦王政}\
  \toprule
  \SimHei \normalsize 年数 & \SimHei \scriptsize 公元 & \SimHei 大事件 \tabularnewline
  % \midrule
  \endfirsthead
  \toprule
  \SimHei \normalsize 年数 & \SimHei \scriptsize 公元 & \SimHei 大事件 \tabularnewline
  \midrule
  \endhead
  \midrule
  元年 & 477 & \tabularnewline\hline
  二年 & 478 & \tabularnewline\hline
  三年 & 479 & \tabularnewline\hline
  四年 & 480 & \tabularnewline\hline
  五年 & 481 & \tabularnewline\hline
  六年 & 482 & \tabularnewline\hline
  七年 & 483 & \tabularnewline\hline
  八年 & 484 & \tabularnewline\hline
  九年 & 485 & \tabularnewline\hline
  十年 & 486 & \tabularnewline\hline
  十一年 & 487 & \tabularnewline\hline
  十二年 & 488 & \tabularnewline\hline
  十三年 & 489 & \tabularnewline\hline
  十四年 & 490 & \tabularnewline\hline
  十五年 & 491 & \tabularnewline\hline
  十六年 & 492 & \tabularnewline\hline
  十七年 & 493 & \tabularnewline\hline
  十八年 & 494 & \tabularnewline\hline
  十九年 & 495 & \tabularnewline\hline
  二十年 & 496 & \tabularnewline\hline
  二一年 & 497 & \tabularnewline\hline
  二二年 & 498 & \tabularnewline\hline
  二三年 & 499 & \tabularnewline
  \bottomrule
\end{longtable}


%%% Local Variables:
%%% mode: latex
%%% TeX-engine: xetex
%%% TeX-master: "../../Main"
%%% End:
