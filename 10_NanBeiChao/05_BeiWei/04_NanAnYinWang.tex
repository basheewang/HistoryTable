%% -*- coding: utf-8 -*-
%% Time-stamp: <Chen Wang: 2019-12-23 15:02:48>

\subsection{拓跋余\tiny(452)}

\subsubsection{生平}

拓跋余(5世纪-452年10月29日),鮮卑名可博真,北魏太武帝拓跋燾之子,生母闾左昭仪。北魏皇帝,為中常侍宗愛所立,但同年又被其殺害,所在位232天。

拓跋余生年不详。太平真君三年(442年),拓跋余獲封為吳王。正平元年(451年)九月,太武帝南征,以拓跋余留守平城。十二月改封南安王。

正平二年二月五日(452年3月11日),中常侍宗愛弒太武帝,尚书左仆射兰延、侍中吴兴公和疋、侍中太原公薛提等秘不发丧。兰延、和疋认为太武帝嫡孙拓跋濬冲幼,欲立太武帝在世长子,于是召东平王拓跋翰,置于秘室。薛提则坚持立拓跋濬,兰延等犹豫未决。宗爱得知。宗爱曾得罪拓跋晃,素来厌恶拓跋翰,却和拓跋余关系好,于是秘密迎拓跋余从中宫便门入宫,矫皇后令征召兰延等,斩于殿堂,再杀死拓跋翰,立拓跋余为帝,为太武帝发丧,改元承平(或作永平),大赦。

拓跋余即位後,因自己不是作为先帝长子继位,即厚待群下以取悅眾人,以宗爱为大司马、大将军、太师、都督中外诸军事,领中秘书,封冯翊王,以兄长原太子拓跋晃辅臣尚书令古弼为司徒,兼太尉张黎为太尉,羽林中郎、幢将乌程子建威将军吕罗汉典宿卫。但他也徹夜暢飲,夜夜笙歌,很快即令國庫空虛,又多次出獵,即使邊境有事,亦不加体恤,百姓皆憤怒,而他不作改變。南朝宋文帝刘义隆遣将檀和之侵犯济州,拓跋余令侍中、尚书左仆射、征南将军韩茂讨之,檀和之遁走。

另外,宗愛自拓跋余登位後掌權日久,朝野內外皆忌憚他,而拓跋余則懷疑宗愛另有所圖,密謀削奪宗愛權力,宗愛於是于十月一日(10月29日)使小黄门贾周等乘夜趁拓跋余祭祀东庙而杀之(《宋书》作与宗爱同被兄拓跋谭所杀,疑误),隐秘其事,只有羽林中郎刘尼知道。

拓跋余被杀后,百官不知道立谁为新君,刘尼、吕罗汉等迎立拓跋濬为北魏文成帝,诛杀宗爱。文成帝以王禮安葬拓跋余,諡為隱王。

先前太平真君七年(439年)八月,月犯荧惑;八月至十一月,又犯轩辕。是岁正月,太白经天。九月火犯太微。这些星象被认为是从拓跋晃去世到拓跋余被杀一系列事的预兆。

\subsubsection{承平}

\begin{longtable}{|>{\centering\scriptsize}m{2em}|>{\centering\scriptsize}m{1.3em}|>{\centering}m{8.8em}|}
  % \caption{秦王政}\
  \toprule
  \SimHei \normalsize 年数 & \SimHei \scriptsize 公元 & \SimHei 大事件 \tabularnewline
  % \midrule
  \endfirsthead
  \toprule
  \SimHei \normalsize 年数 & \SimHei \scriptsize 公元 & \SimHei 大事件 \tabularnewline
  \midrule
  \endhead
  \midrule
  元年 & 452 & \tabularnewline\hline
  \bottomrule
\end{longtable}


%%% Local Variables:
%%% mode: latex
%%% TeX-engine: xetex
%%% TeX-master: "../../Main"
%%% End:
