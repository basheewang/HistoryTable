%% -*- coding: utf-8 -*-
%% Time-stamp: <Chen Wang: 2019-12-23 15:20:31>

\subsection{东海王\tiny(530-531)}

\subsubsection{生平}

元晔(?-532年12月26日),字华兴,小字盆子,河南郡洛阳县(今河南省洛阳市东)人,追尊魏景穆帝拓跋晃曾孙,鄯善镇将、扶风王元怡之子,北魏宗室、官员,一度被尔朱兆与尔朱世隆拥立为皇帝。

元晔性格轻浮急躁,有体力,以秘书郎为起家官,略微升迁至通直散骑常侍。建义元年四月甲辰(528年5月21日),魏孝庄帝元子攸封秘书郎中元晔为长广王,食邑一千户,外任太原郡太守,代理并州刺史。尔朱荣死后,尔朱世隆逃回并州的建兴郡高都县,尔朱兆从晋阳县前来汇合,于是在永安三年十月壬申(530年12月5日)推举元晔为皇帝,大赦所管辖的地区,年号建明,所有官员加四级。元晔小名盆子,听说此事的人都认为类似赤眉军之事。

建明元年十二月戊申(531年1月10日),元晔在攻克洛阳后大赦天下。尔朱世隆与兄弟密谋,担心元晔的母亲卫氏干预朝政,观察卫氏出行,派遣数十骑兵装扮成劫匪,在京城小巷子里杀死了卫氏。公家和私人都感到惊愕,不知道什么原因。很快又张贴公告悬赏,以一千万钱悬赏劫匪。百姓知道后,没有不垂头丧气的。尔朱世隆很快认为元晔是北魏宗室远支,又不是众望所推的人,想要推举宗室近支广陵王元恭为皇帝。建明元年春二月己巳(531年4月1日),元晔前往邙山南,尔朱世隆等人在洛阳东城外奉迎广陵王元恭,尔朱世隆等人写好禅让的册文,以泰山郡太守窦瑗执马鞭进入行宫,启奏元晔说:“上天和百姓的愿望,都在广陵王身上,请实行尧舜禅让的礼仪。”元晔因此退位。

魏节闵帝元恭登基后,于普泰元年三月癸酉(531年4月5日)封元晔为东海王,食邑一万户。太昌元年十一月甲辰(532年12月26日),安定王元朗和元晔在家中被赐令自杀。元晔的爵位被削除。

\subsubsection{建明}

\begin{longtable}{|>{\centering\scriptsize}m{2em}|>{\centering\scriptsize}m{1.3em}|>{\centering}m{8.8em}|}
  % \caption{秦王政}\
  \toprule
  \SimHei \normalsize 年数 & \SimHei \scriptsize 公元 & \SimHei 大事件 \tabularnewline
  % \midrule
  \endfirsthead
  \toprule
  \SimHei \normalsize 年数 & \SimHei \scriptsize 公元 & \SimHei 大事件 \tabularnewline
  \midrule
  \endhead
  \midrule
  元年 & 530 & \tabularnewline\hline
  二年 & 531 & \tabularnewline
  \bottomrule
\end{longtable}


%%% Local Variables:
%%% mode: latex
%%% TeX-engine: xetex
%%% TeX-master: "../../Main"
%%% End:
