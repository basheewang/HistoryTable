%% -*- coding: utf-8 -*-
%% Time-stamp: <Chen Wang: 2019-12-23 15:01:57>

\subsection{太武帝\tiny(423-452)}

\subsubsection{生平}

魏太武帝拓跋焘(408年-452年3月11日),鮮卑本名佛狸伐。佛狸是官號,突厥語狼büri或böri的音譯。伐或bäg是官稱,且是魏晉時期鮮卑諸部使用最為廣泛的政治名號。北魏第三位皇帝(423年12月27日—452年3月11日在位),北魏明元帝拓跋嗣的长子,北魏道武帝拓拔珪的長孫,在位28年,谥号太武皇帝。

拓拔焘即位时,只有十六岁,大臣们都拿他当小孩子看。于是拓拔焘决定先整顿吏治,励精图治,令人刮目相看,北魏国力进入鼎盛。427年,拓跋燾在連續兩年突擊統萬城之後,占領胡夏的北部地區(包含首都統萬),並一度攻下關中,胡夏雖遷都至平涼,卻於次年(428年)打敗魏軍並收復關中。北魏在429年北伐柔然大獲全勝之後,趁著柔然近十年都難以恢復的良機,把軍隊主力向南進攻,於430年大敗劉宋與胡夏的聯合攻勢,不但占領胡夏大部分的關隴領土(包含平涼、關中、隴西郡),更在431年從宋軍手中奪回河南四鎮(洛陽、虎牢等),拓拔焘返回首都平城,祭告太廟並舉行盛大的慶功典禮。

撤退到上邽的夏主赫連定,雖於431年滅西秦而稍微挽救了國勢,並意圖再滅北涼以維持胡夏,但卻在432年,被吐谷渾君主慕容慕璝襲擊而俘虜。同年赫連定被送給北魏,拓拔焘將其處死,胡夏亡。436年拓跋燾派軍東征北燕,燕主馮弘在高句麗大軍的保護之下,將首都人民全部東遷高句麗,而魏軍主帥忌憚高軍,坐看燕人東撤;北燕雖然滅亡,但只得空地空城,因此拓跋燾大怒之下處罰了征燕主帥娥清、古弼。439年拓拔焘率大軍圍攻北涼首都姑臧,涼主沮渠牧犍出降,北涼亡。至此,北魏統一華北,与江东的刘宋王朝对峙,形成南北朝的局面。

自前涼张氏以来,河西地方文化学术比较发达,号称多士。北魏自道武帝以后,政治上使用汉族高门,汲取不少魏晋典制。431年,藉由同年打敗劉宋的威勢,拓拔焘下詔,徵聘關東地區的數百名士(多為領導地方的世家大族)入朝為官,也就是把山東郡姓如范陽盧氏、博陵崔氏、趙郡李氏等勢力一網打盡,強迫他們到平城擔任無薪水的官職,讓漢人世族的勢力與北魏政權相結合。當時被徵召的名士高允,後來寫了一篇文章〈徵士頌〉來追憶、讚揚此盛事。439年北魏吞并河西后,又有大批河西文士进入北魏统治区域,不少人被徵召到平城去做官,受到重用,北魏的儒学才开始兴盛。

之後,拓拔焘又击溃吐谷渾、柔然,扩地千餘里。他一共七次率军进攻柔然,太平真君十年(449)大败柔然,收民畜凡百余万,柔然可汗远遁,北方边塞再度得到安静。

他在450-451年对宋的战争中,雖然大勝,但人馬死傷近半,又使軍民疲憊,怨聲不已。末期又刑罰殘酷,使国内政治混乱。譬如崔浩修国史详实记载魏先世事迹,可能涉及某些鲜卑习俗和隐私,有伤体面,拓跋焘不惜发动國史之獄,将三朝功臣司徒崔浩处死,连清河崔氏与浩同宗者以及崔浩姻亲范阳卢氏、太原郭氏、河东柳氏都遭族灭。事后拓跋焘说 “崔司徒可惜”,有后悔之意;再如監國執政的太子,也在父子權力衝突下,被宦官宗愛的讒言害死。正平二年二月甲寅(452年3月11日)拓跋焘被宗爱杀死,享年四十五歲,谥号太武帝,庙号世祖。

拓跋焘统治时期,氐、羌、屠各,以及所谓“杂虏”、“杂人”的各族暴亂非常频繁。太平真君六年(445年)卢水胡盖吴在关中杏城(今陕西黄陵西南)发动的起义,声势最为浩大。盖吴建号秦地王,有众十余万,得到安定卢水胡刘超、河东蜀薛永宗的响应,拓跋焘调动强大的兵力才镇压下去。

拓跋焘受崔浩、寇谦之影响,奉道排佛。镇压盖吴过程中,在长安佛寺中发现大量兵器,认为佛寺与盖吴通谋,太平真君七年(446年),詔:「諸有佛圖、形像及胡經,盡皆擊破焚燒,沙門無少長悉坑之。」,是為北魏太武帝滅佛,三武滅佛之一(另外兩位是北周武帝和唐武宗)。

拓跋焘天生將才,为人勇健,善于指挥。战阵亲犯矢石,神色自若,命将出师,违其节度者多败,因此将士畏服,为之盡力。有知人之明,常从士伍中选拔人才。赏不遗贱,罚不避贵,虽所爱之人亦不宽假。他放棄父親拓跋嗣築邊城防禦柔然的政策,主動攻擊柔然並獲得成功。他自奉俭朴,而赏赐功臣绝无吝嗇,幾乎把資源都用在主動出擊的軍功賞賜之上。认为元老功臣勤劳日久,应让他们以爵归第,随时朝见饷宴,百官职务则可另简贤能。这样就保证了行政效率,使政治多少能健全发展。他倚重汉人,李顺、崔浩、李孝伯等先后掌握朝权,但個性果於殺戮,處死大臣後常懊悔自己太快動刀。

北齊史官魏收於《魏書》的「史臣曰」評論說:「世祖聰明雄斷,威靈傑立,藉二世之資,奮征伐之氣,遂戎軒四出,周旋險夷。掃統萬,平秦隴,翦遼海,盪河源,南夷荷擔,北蠕削跡,廓定四表,混一戎華,其為功也大矣。遂使有魏之業,光邁百王,豈非神叡經綸,事當命世。至於初則東儲不終,末乃釁成所忽。固本貽防,殆弗思乎?」

唐代某貴族「公子」與世族虞世南的對話:「公子曰:『魏之太祖、太武,孰與為輩?』先生曰:『太祖、太武,俱有異人之姿,故能辟土擒敵,窺覦江外。然善戰好殺,暴桀雄武,稟崆峒之氣焉。至於安忍誅殘,石季龍之儔也。』」

北宋司馬光評論說:「(北魏)繼以明元、太武,兼有青、兗,包司、豫,摧赫連,開關中,梟馮弘,吞遼碣,擄沮渠,并河右,高車入臣,蠕蠕遠遁;自河以北,逾於大漠,悉為其有;子孫稱帝者,百有餘年。左袵之盛,未之有也。」

资治通鉴记载: 魏主(指太武帝)為人,壯健鷙勇,臨城對陣,親犯矢石,左右死傷相繼,神色自若;由是將士畏服,咸盡死力。性儉率,服御飲膳,取給而已。群臣請增峻京城及修宮室曰: 「《易》云:『王公設險,以守其國。』又蕭何云:『天子以四海為家,不壯不麗,無以重威。』」帝曰:「古人有言:『在德不在險。』屈丐蒸土築城而朕滅之。 豈在城也?今天下未平,方須民力,土功之事,朕所未為。蕭何之對,非雅言也。」每以為財者軍國之本,不可輕費。至於賞賜,皆死事勳績之家,親戚貴寵未嘗橫有所及。命將出師,指授節度,違之者多致負敗。明於知人,或拔干於卒伍之中,唯其才用所長,不論本末。聽察精敏,下無遁情,賞不遺賤,罰不避貴,雖所甚愛之人,終無寬假。常曰:「法者,朕與天下共之,何敢輕也。」然性殘忍,果於殺戮,往往已殺而復悔之。

太平真君四年(443年)拓拔焘遣大臣李敞所刻的石刻祝文,存於嘎仙洞内的石壁上。1980年7月30日,中国考古学家米文平等人在此洞发现石刻祝文,结合当时在洞内发现的陶器碎片等,认定此处即为史书中记载的北魏祖庭。但该洞是否确实就是拓跋鲜卑的发源地,史学界尚有争论。

江蘇省南京市六合區东南有瓜步山,山上有佛狸祠。

《魏书·世祖纪下》记载:北魏太武帝拓跋焘于宋元嘉二十七年击败王玄谟的军队以后,在山上建立行宫,即后来的「佛狸祠」。

南宋诗人辛弃疾有《永遇乐·京口北固亭怀古》:「可堪回首,佛狸祠下,一片神鸦社鼓」。后又有《水调歌头·舟次扬州和杨济翁周显先韵》:「谁道投鞭飞渡,忆昔鸣髇血污,风雨佛狸愁。」

太延元年(435年)十月,太武帝东巡冀州、定州,二十日甲辰到定州,驻驾于新城宫。十一月十六日己巳,在广川(河北景县)校猎。二十三日丙子到达邺城(河北临漳),祭祀密太后(太武帝母杜氏)庙,并慰问老年族人,褒礼贤俊。十二月二十日癸卯派遣使者到北岳恒山祭祀。次年正月初二甲寅从五回道返回平城。

在东巡至河北易县南管头之南画猫村古徐水河谷时,见山岩险峭,景观奇丽,太武帝即兴演示射术,又命左右将士善射者进行射箭比试。镇东将军、定州刺史、乐浪公乞伏某请求立碑纪念。到太延三年丁丑(437年)碑刻完工,乐浪公已去职,新任刺史征东将军、张掖公秃发保周)接手此事。

东巡碑碑额题【皇帝东巡之碑】,史籍最早提到北魏太武帝东巡碑,是郦道元《水经注》。郦书之后,宋代乐史《太平寰宇记》卷六七易州满城县条,也曾提及此碑,称引的内容有溢出郦书者。此后东巡碑湮没无闻将近千年,直到1935年,由徐森玉(鸿宝)先生在河北易县觅得原碑,把20份拓本带回北平,次年傅增湘、周肇祥也前往摹拓,东巡碑才重新现身,为艺林所重。今碑已破碎,仅剩残片若干块。

\subsubsection{始光}

\begin{longtable}{|>{\centering\scriptsize}m{2em}|>{\centering\scriptsize}m{1.3em}|>{\centering}m{8.8em}|}
  % \caption{秦王政}\
  \toprule
  \SimHei \normalsize 年数 & \SimHei \scriptsize 公元 & \SimHei 大事件 \tabularnewline
  % \midrule
  \endfirsthead
  \toprule
  \SimHei \normalsize 年数 & \SimHei \scriptsize 公元 & \SimHei 大事件 \tabularnewline
  \midrule
  \endhead
  \midrule
  元年 & 424 & \tabularnewline\hline
  二年 & 425 & \tabularnewline\hline
  三年 & 426 & \tabularnewline\hline
  四年 & 427 & \tabularnewline\hline
  五年 & 428 & \tabularnewline
  \bottomrule
\end{longtable}

\subsubsection{神䴥}

\begin{longtable}{|>{\centering\scriptsize}m{2em}|>{\centering\scriptsize}m{1.3em}|>{\centering}m{8.8em}|}
  % \caption{秦王政}\
  \toprule
  \SimHei \normalsize 年数 & \SimHei \scriptsize 公元 & \SimHei 大事件 \tabularnewline
  % \midrule
  \endfirsthead
  \toprule
  \SimHei \normalsize 年数 & \SimHei \scriptsize 公元 & \SimHei 大事件 \tabularnewline
  \midrule
  \endhead
  \midrule
  元年 & 428 & \tabularnewline\hline
  二年 & 429 & \tabularnewline\hline
  三年 & 430 & \tabularnewline\hline
  四年 & 431 & \tabularnewline
  \bottomrule
\end{longtable}

\subsubsection{延和}

\begin{longtable}{|>{\centering\scriptsize}m{2em}|>{\centering\scriptsize}m{1.3em}|>{\centering}m{8.8em}|}
  % \caption{秦王政}\
  \toprule
  \SimHei \normalsize 年数 & \SimHei \scriptsize 公元 & \SimHei 大事件 \tabularnewline
  % \midrule
  \endfirsthead
  \toprule
  \SimHei \normalsize 年数 & \SimHei \scriptsize 公元 & \SimHei 大事件 \tabularnewline
  \midrule
  \endhead
  \midrule
  元年 & 432 & \tabularnewline\hline
  二年 & 433 & \tabularnewline\hline
  三年 & 434 & \tabularnewline\hline
  四年 & 435 & \tabularnewline
  \bottomrule
\end{longtable}

\subsubsection{太延}

\begin{longtable}{|>{\centering\scriptsize}m{2em}|>{\centering\scriptsize}m{1.3em}|>{\centering}m{8.8em}|}
  % \caption{秦王政}\
  \toprule
  \SimHei \normalsize 年数 & \SimHei \scriptsize 公元 & \SimHei 大事件 \tabularnewline
  % \midrule
  \endfirsthead
  \toprule
  \SimHei \normalsize 年数 & \SimHei \scriptsize 公元 & \SimHei 大事件 \tabularnewline
  \midrule
  \endhead
  \midrule
  元年 & 435 & \tabularnewline\hline
  二年 & 436 & \tabularnewline\hline
  三年 & 437 & \tabularnewline\hline
  四年 & 438 & \tabularnewline\hline
  五年 & 439 & \tabularnewline\hline
  六年 & 440 & \tabularnewline
  \bottomrule
\end{longtable}

\subsubsection{太平真君}

\begin{longtable}{|>{\centering\scriptsize}m{2em}|>{\centering\scriptsize}m{1.3em}|>{\centering}m{8.8em}|}
  % \caption{秦王政}\
  \toprule
  \SimHei \normalsize 年数 & \SimHei \scriptsize 公元 & \SimHei 大事件 \tabularnewline
  % \midrule
  \endfirsthead
  \toprule
  \SimHei \normalsize 年数 & \SimHei \scriptsize 公元 & \SimHei 大事件 \tabularnewline
  \midrule
  \endhead
  \midrule
  元年 & 440 & \tabularnewline\hline
  二年 & 441 & \tabularnewline\hline
  三年 & 442 & \tabularnewline\hline
  四年 & 443 & \tabularnewline\hline
  五年 & 444 & \tabularnewline\hline
  六年 & 445 & \tabularnewline\hline
  七年 & 446 & \tabularnewline\hline
  八年 & 447 & \tabularnewline\hline
  九年 & 448 & \tabularnewline\hline
  十年 & 449 & \tabularnewline\hline
  十一年 & 450 & \tabularnewline\hline
  十二年 & 451 & \tabularnewline
  \bottomrule
\end{longtable}

\subsubsection{正平}

\begin{longtable}{|>{\centering\scriptsize}m{2em}|>{\centering\scriptsize}m{1.3em}|>{\centering}m{8.8em}|}
  % \caption{秦王政}\
  \toprule
  \SimHei \normalsize 年数 & \SimHei \scriptsize 公元 & \SimHei 大事件 \tabularnewline
  % \midrule
  \endfirsthead
  \toprule
  \SimHei \normalsize 年数 & \SimHei \scriptsize 公元 & \SimHei 大事件 \tabularnewline
  \midrule
  \endhead
  \midrule
  元年 & 451 & \tabularnewline\hline
  二年 & 452 & \tabularnewline
  \bottomrule
\end{longtable}


%%% Local Variables:
%%% mode: latex
%%% TeX-engine: xetex
%%% TeX-master: "../../Main"
%%% End:
