%% -*- coding: utf-8 -*-
%% Time-stamp: <Chen Wang: 2019-12-23 15:22:30>

\subsection{孝武帝\tiny(532-534)}

\subsubsection{生平}

魏孝武帝元修,一说元脩(510年-535年2月3日),字孝則,河南郡洛阳县(今河南省洛阳市东)人,魏孝文帝元宏之孙,廣平武穆王元懷第三子,母親是李氏,是北魏最後一位皇帝。

他曾被封為汝陽縣公、通直散騎侍郎、中書侍郎。建義年間,辭散騎常侍,為平東將軍、太常卿,後來又為鎮東將軍、宗正卿。530年封為平陽王。普泰初年,轉任侍中、鎮東將軍、儀同三司、兼為尚書右僕射,後又改加侍中、尚書左僕射。河阴之变后政局一片混乱,诸王大多各自逃生,元修逃亡民间,隐为乡农。

中興二年(532年)高欢击败尔朱氏,欲立元悦为帝,因无法服众,只得退而选择元修。斛斯椿从元修的心腹好友王思政辗转寻到元修,元修道:“这该不是把我出卖了吧?”高欢遂亲自前来陳誠,泣下沾襟。元修方才入京,四月廿五日(6月13日)即位。

元修即位后与高欢的长女结婚,夫妻彼此都没有感情。元修与三个堂姊妹姘居,将她们都封为公主。534年與高歡決裂,高歡帶兵從晉陽南下时,元修于七月廿八日(8月22日)率一部分兵众,偕同情妇元明月及元明月的哥哥元宝炬等入關中投奔其妹妹的未婚夫宇文泰。

先前梁武帝因天象称“荧惑入南斗,天子下殿走”,就赤脚下殿以应天象,得知元修西奔,羞惭地说:“索虏也应天象吗?”因为这一天象应在元修身上就意味着上天认为元修才是正统天子。

同年十月十七日(11月8日),高歡以元修弃国逃跑,另立元善見為帝,十日后遷都鄴。宇文泰让元氏诸王杀死元明月,元修非常不高兴,闰十二月癸巳(535年2月3日),宇文泰用毒酒毒死了元修,改立元宝炬为帝。

元修死後被宇文泰下令埋进草堂佛寺,十余年后才得正式落葬。西魏上諡號為孝武皇帝。

\subsubsection{太昌}

\begin{longtable}{|>{\centering\scriptsize}m{2em}|>{\centering\scriptsize}m{1.3em}|>{\centering}m{8.8em}|}
  % \caption{秦王政}\
  \toprule
  \SimHei \normalsize 年数 & \SimHei \scriptsize 公元 & \SimHei 大事件 \tabularnewline
  % \midrule
  \endfirsthead
  \toprule
  \SimHei \normalsize 年数 & \SimHei \scriptsize 公元 & \SimHei 大事件 \tabularnewline
  \midrule
  \endhead
  \midrule
  元年 & 532 & \tabularnewline
  \bottomrule
\end{longtable}

\subsubsection{永兴}

\begin{longtable}{|>{\centering\scriptsize}m{2em}|>{\centering\scriptsize}m{1.3em}|>{\centering}m{8.8em}|}
  % \caption{秦王政}\
  \toprule
  \SimHei \normalsize 年数 & \SimHei \scriptsize 公元 & \SimHei 大事件 \tabularnewline
  % \midrule
  \endfirsthead
  \toprule
  \SimHei \normalsize 年数 & \SimHei \scriptsize 公元 & \SimHei 大事件 \tabularnewline
  \midrule
  \endhead
  \midrule
  元年 & 532 & \tabularnewline
  \bottomrule
\end{longtable}

\subsubsection{永熙}

\begin{longtable}{|>{\centering\scriptsize}m{2em}|>{\centering\scriptsize}m{1.3em}|>{\centering}m{8.8em}|}
  % \caption{秦王政}\
  \toprule
  \SimHei \normalsize 年数 & \SimHei \scriptsize 公元 & \SimHei 大事件 \tabularnewline
  % \midrule
  \endfirsthead
  \toprule
  \SimHei \normalsize 年数 & \SimHei \scriptsize 公元 & \SimHei 大事件 \tabularnewline
  \midrule
  \endhead
  \midrule
  元年 & 532 & \tabularnewline\hline
  二年 & 533 & \tabularnewline\hline
  三年 & 534 & \tabularnewline
  \bottomrule
\end{longtable}


%%% Local Variables:
%%% mode: latex
%%% TeX-engine: xetex
%%% TeX-master: "../../Main"
%%% End:
