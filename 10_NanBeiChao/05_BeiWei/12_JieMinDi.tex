%% -*- coding: utf-8 -*-
%% Time-stamp: <Chen Wang: 2019-12-23 15:20:56>

\subsection{节闵帝\tiny(531)}

\subsubsection{生平}

魏节闵帝元恭(498年-532年6月21日;在位531年-532年),字修业,河南郡洛阳县(今河南省洛阳市东)人,魏献文帝拓跋弘之孙,广陵惠王元羽之子,母王氏,是南北朝时期北魏的皇帝。

先前元恭的伯父北魏孝文帝元宏为鼓励鲜卑与汉族通婚而将元羽嫡妻降为妾室,另聘娶郑始容为元羽嫡妻。史书没有记载元恭的母亲王氏是否就是元羽被降为妾室的元配,仅记载元羽死后袭爵广陵王的是元恭,而非元恭的哥哥元欣。

建明二年二月廿九日(531年4月1日),爾朱榮堂弟爾朱世隆废元晔,立元恭為帝。军阀高欢则立渤海太守安定王元朗为帝。高欢打败尔朱氏后,考虑到元朗世系疏远,一度想尊奉元恭,派仆射魏兰根慰谕洛阳观察节闵帝为人。魏兰根认为节闵帝神采高明,日后难制,与侍中、司空高乾兄弟及黄门侍郎崔㥄强调节闵帝系尔朱氏所立,共劝高欢为了讨伐尔朱氏师出有名而废帝。532年6月(农历四月),节闵帝被高欢所废,囚禁于崇训佛寺。元朗亦被高欢所迫禅位给平阳王元修即北魏孝武帝。节闵帝赋诗:“朱門久可患,紫極非情玩。顛覆立可待,一年三易換。時運正如此,唯有修真觀。”

五月丙申(532年6月21日),魏节闵帝在门下外省被孝武帝毒死,虚岁三十五,葬以亲王殊礼;加九旒、銮辂、黄屋、左纛,班剑百二十人,二卫、羽林备仪卫。东魏称广陵王或前废帝,西魏谥节闵帝。

2013年,洛阳市文物考古研究院完成北魏节愍帝的陵墓考古挖掘工作,陵墓中出土东罗马帝国金币一枚。

\subsubsection{普泰}

\begin{longtable}{|>{\centering\scriptsize}m{2em}|>{\centering\scriptsize}m{1.3em}|>{\centering}m{8.8em}|}
  % \caption{秦王政}\
  \toprule
  \SimHei \normalsize 年数 & \SimHei \scriptsize 公元 & \SimHei 大事件 \tabularnewline
  % \midrule
  \endfirsthead
  \toprule
  \SimHei \normalsize 年数 & \SimHei \scriptsize 公元 & \SimHei 大事件 \tabularnewline
  \midrule
  \endhead
  \midrule
  元年 & 531 & \tabularnewline
  \bottomrule
\end{longtable}


%%% Local Variables:
%%% mode: latex
%%% TeX-engine: xetex
%%% TeX-master: "../../Main"
%%% End:
