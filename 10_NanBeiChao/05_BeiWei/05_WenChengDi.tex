%% -*- coding: utf-8 -*-
%% Time-stamp: <Chen Wang: 2019-12-23 15:13:21>

\subsection{文成帝\tiny(452-465)}

\subsubsection{生平}

魏文成帝拓跋濬(440年-465年),鮮卑名烏雷,拥有直懃头衔,南北朝時期北魏的第四代皇帝。是魏太武帝的嫡孫,景穆太子拓跋晃長子。452年-465年在位,在位十三年。

正平元年(451年),太武帝北征,任太子拓跋晃為監國,但宦官中常侍宗愛對太子多加干預,又與太子部屬給事仇尼道盛和侍郎任城互有閒隙,宗愛怕日後太子登基對己不利,於是與東宮勢力展開權力鬥爭,仗着拓跋燾的信任誣陷太子及其手下人意圖造反,不曾想皇帝拓跋燾竟然相信了。继而下令整肅太子府,且誅殺了許多太子近臣。太子拓跋晃因此積憂成疾,一病而死,時年才24歲。

後來魏太武帝拓跋燾知道太子是清白的,非常懊悔。但宗愛一見到此,怕被皇帝誅殺,先下手為強於正平二年(452年)三月弒太武帝。太武帝死後,朝廷欲立太武帝最年长的在世儿子第三子東平王拓跋翰為帝,但宗愛與拓跋翰關係不好,因此假立太武皇后之命,將拓跋翰殺掉,又假借皇后之命,將擁立東平王的大臣尚書僕射蘭延、侍中吳興公和疋及侍中太原公薛提殺死,然後立太武帝幼子南安王拓跋余為帝,宗愛自為大司馬、大將軍、太師,總督中外軍事、領中祕書,封馮翊王,大權在握。拓跋余想奪回皇權,又于十月一日遭宗愛所弒。短短數月,宗愛連殺兩位皇帝,引起朝野震動。

羽林郎中刘尼、太子少傅游雅、殿中尚书源贺、尚书陆丽、尚书长孙渴侯五人密谋,十月三日(10月31日),由太子少傅游雅、源贺、长孙渴侯率禁军守卫宫廷,陆丽与刘尼一起迎皇孙拓跋濬入宫即位。拓跋濬改元兴安。

拓跋濬即位后,便诛杀了宗爱、贾周等人,都动用五刑,灭三族。兴安元年(452年)十一月初九,文成帝追谥父亲拓跋晃为景穆皇帝,母亲闾氏为恭皇后,尊乳母常氏为保太后。

太武帝崇信道教,一度太武滅佛。兴安元年,拓跋濬下令复兴佛教。兴安二年,令建造云冈石窟。

拓跋濬不再继续太武帝四处用兵的政策,停止南侵南朝宋,休养生息。但也有征伐。太安四年(458年),拓跋濬亲率10万骑兵、15万辆战车,进攻柔然。柔然处罗可汗吐贺真远远逃走。柔然别部统帅乌朱驾颓等人率领几千个帳幕所聚的部落投降。

和平六年五月十一日(465年6月21日),拓跋濬去世,时年僅25岁。六月初二,定谥号为文成皇帝,庙号高宗。八月,安葬云中的金陵。

\subsubsection{兴安}

\begin{longtable}{|>{\centering\scriptsize}m{2em}|>{\centering\scriptsize}m{1.3em}|>{\centering}m{8.8em}|}
  % \caption{秦王政}\
  \toprule
  \SimHei \normalsize 年数 & \SimHei \scriptsize 公元 & \SimHei 大事件 \tabularnewline
  % \midrule
  \endfirsthead
  \toprule
  \SimHei \normalsize 年数 & \SimHei \scriptsize 公元 & \SimHei 大事件 \tabularnewline
  \midrule
  \endhead
  \midrule
  元年 & 452 & \tabularnewline\hline
  二年 & 453 & \tabularnewline\hline
  三年 & 454 & \tabularnewline
  \bottomrule
\end{longtable}

\subsubsection{兴光}

\begin{longtable}{|>{\centering\scriptsize}m{2em}|>{\centering\scriptsize}m{1.3em}|>{\centering}m{8.8em}|}
  % \caption{秦王政}\
  \toprule
  \SimHei \normalsize 年数 & \SimHei \scriptsize 公元 & \SimHei 大事件 \tabularnewline
  % \midrule
  \endfirsthead
  \toprule
  \SimHei \normalsize 年数 & \SimHei \scriptsize 公元 & \SimHei 大事件 \tabularnewline
  \midrule
  \endhead
  \midrule
  元年 & 454 & \tabularnewline\hline
  二年 & 455 & \tabularnewline
  \bottomrule
\end{longtable}

\subsubsection{太安}

\begin{longtable}{|>{\centering\scriptsize}m{2em}|>{\centering\scriptsize}m{1.3em}|>{\centering}m{8.8em}|}
  % \caption{秦王政}\
  \toprule
  \SimHei \normalsize 年数 & \SimHei \scriptsize 公元 & \SimHei 大事件 \tabularnewline
  % \midrule
  \endfirsthead
  \toprule
  \SimHei \normalsize 年数 & \SimHei \scriptsize 公元 & \SimHei 大事件 \tabularnewline
  \midrule
  \endhead
  \midrule
  元年 & 455 & \tabularnewline\hline
  二年 & 456 & \tabularnewline\hline
  三年 & 457 & \tabularnewline\hline
  四年 & 458 & \tabularnewline\hline
  五年 & 459 & \tabularnewline
  \bottomrule
\end{longtable}

\subsubsection{和平}

\begin{longtable}{|>{\centering\scriptsize}m{2em}|>{\centering\scriptsize}m{1.3em}|>{\centering}m{8.8em}|}
  % \caption{秦王政}\
  \toprule
  \SimHei \normalsize 年数 & \SimHei \scriptsize 公元 & \SimHei 大事件 \tabularnewline
  % \midrule
  \endfirsthead
  \toprule
  \SimHei \normalsize 年数 & \SimHei \scriptsize 公元 & \SimHei 大事件 \tabularnewline
  \midrule
  \endhead
  \midrule
  元年 & 460 & \tabularnewline\hline
  二年 & 461 & \tabularnewline\hline
  三年 & 462 & \tabularnewline\hline
  四年 & 463 & \tabularnewline\hline
  五年 & 464 & \tabularnewline\hline
  六年 & 465 & \tabularnewline
  \bottomrule
\end{longtable}


%%% Local Variables:
%%% mode: latex
%%% TeX-engine: xetex
%%% TeX-master: "../../Main"
%%% End:
