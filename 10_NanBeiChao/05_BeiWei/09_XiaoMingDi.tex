%% -*- coding: utf-8 -*-
%% Time-stamp: <Chen Wang: 2019-12-23 15:16:31>

\subsection{孝明帝\tiny(515-528)}

\subsubsection{生平}

魏孝明帝元诩(510年-528年3月31日),河南郡洛阳县(今河南省洛阳市东)人,魏宣武帝元恪第二子,生母胡充华,是南北朝时期北魏的皇帝,在位13年。

延昌四年(515年),宣武帝病逝,太子元詡繼位,由太傅、侍中元懌輔政。胡太后和元懌相愛,常招元懌夜宿宮中。領軍元乂和長秋卿劉騰等人密謀,將元懌殺害,又把胡太后幽禁在北宮的宣光殿。胡太后又結識鄭儼、李神軌、徐紇諸情人。鄭儼和徐紇把持內外,時稱“徐鄭”。

正光年间以后,国家入不敷出,政府决定预先征收六年的租調,导致人民生活愈加艰苦。除此之外,政府停止对官员供应酒,但每季所举行的祭祀祖宗、神明的仪式及外交所需费用不算在内。由于匪徒越来越多,大量器械被劫掠,关西地区匪患尤其严重。政府的积蓄于是枯竭。在此境地下,政府又下令减少对官员及外国使节百分之五十的粮食及肉供应。孝昌年间,京城的田每亩征税五升,借公田耕种者每亩征税一斗。并且还对市场,商业店铺征税。

武泰元年(528年),孝明帝妃潘外憐生一女,胡太后宣稱生男孩,大赦天下。武泰元年二月癸丑(528年3月31日),鄭儼率御林軍來到顯陽殿,將孝明帝毒死。爾朱榮聞訊,追查孝明帝的死因,另立長樂王元子攸。

\subsubsection{熙平}

\begin{longtable}{|>{\centering\scriptsize}m{2em}|>{\centering\scriptsize}m{1.3em}|>{\centering}m{8.8em}|}
  % \caption{秦王政}\
  \toprule
  \SimHei \normalsize 年数 & \SimHei \scriptsize 公元 & \SimHei 大事件 \tabularnewline
  % \midrule
  \endfirsthead
  \toprule
  \SimHei \normalsize 年数 & \SimHei \scriptsize 公元 & \SimHei 大事件 \tabularnewline
  \midrule
  \endhead
  \midrule
  元年 & 516 & \tabularnewline\hline
  二年 & 517 & \tabularnewline\hline
  三年 & 518 & \tabularnewline
  \bottomrule
\end{longtable}

\subsubsection{神龟}

\begin{longtable}{|>{\centering\scriptsize}m{2em}|>{\centering\scriptsize}m{1.3em}|>{\centering}m{8.8em}|}
  % \caption{秦王政}\
  \toprule
  \SimHei \normalsize 年数 & \SimHei \scriptsize 公元 & \SimHei 大事件 \tabularnewline
  % \midrule
  \endfirsthead
  \toprule
  \SimHei \normalsize 年数 & \SimHei \scriptsize 公元 & \SimHei 大事件 \tabularnewline
  \midrule
  \endhead
  \midrule
  元年 & 518 & \tabularnewline\hline
  二年 & 519 & \tabularnewline\hline
  三年 & 520 & \tabularnewline
  \bottomrule
\end{longtable}

\subsubsection{正光}

\begin{longtable}{|>{\centering\scriptsize}m{2em}|>{\centering\scriptsize}m{1.3em}|>{\centering}m{8.8em}|}
  % \caption{秦王政}\
  \toprule
  \SimHei \normalsize 年数 & \SimHei \scriptsize 公元 & \SimHei 大事件 \tabularnewline
  % \midrule
  \endfirsthead
  \toprule
  \SimHei \normalsize 年数 & \SimHei \scriptsize 公元 & \SimHei 大事件 \tabularnewline
  \midrule
  \endhead
  \midrule
  元年 & 520 & \tabularnewline\hline
  二年 & 521 & \tabularnewline\hline
  三年 & 522 & \tabularnewline\hline
  四年 & 523 & \tabularnewline\hline
  五年 & 524 & \tabularnewline\hline
  六年 & 525 & \tabularnewline
  \bottomrule
\end{longtable}

\subsubsection{孝昌}

\begin{longtable}{|>{\centering\scriptsize}m{2em}|>{\centering\scriptsize}m{1.3em}|>{\centering}m{8.8em}|}
  % \caption{秦王政}\
  \toprule
  \SimHei \normalsize 年数 & \SimHei \scriptsize 公元 & \SimHei 大事件 \tabularnewline
  % \midrule
  \endfirsthead
  \toprule
  \SimHei \normalsize 年数 & \SimHei \scriptsize 公元 & \SimHei 大事件 \tabularnewline
  \midrule
  \endhead
  \midrule
  元年 & 525 & \tabularnewline\hline
  二年 & 526 & \tabularnewline\hline
  三年 & 527 & \tabularnewline\hline
  四年 & 528 & \tabularnewline
  \bottomrule
\end{longtable}

\subsubsection{武泰}

\begin{longtable}{|>{\centering\scriptsize}m{2em}|>{\centering\scriptsize}m{1.3em}|>{\centering}m{8.8em}|}
  % \caption{秦王政}\
  \toprule
  \SimHei \normalsize 年数 & \SimHei \scriptsize 公元 & \SimHei 大事件 \tabularnewline
  % \midrule
  \endfirsthead
  \toprule
  \SimHei \normalsize 年数 & \SimHei \scriptsize 公元 & \SimHei 大事件 \tabularnewline
  \midrule
  \endhead
  \midrule
  元年 & 528 & \tabularnewline
  \bottomrule
\end{longtable}


%%% Local Variables:
%%% mode: latex
%%% TeX-engine: xetex
%%% TeX-master: "../../Main"
%%% End:
