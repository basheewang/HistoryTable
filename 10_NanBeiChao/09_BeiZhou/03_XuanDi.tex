%% -*- coding: utf-8 -*-
%% Time-stamp: <Chen Wang: 2021-11-01 15:15:27>

\subsection{宣帝宇文贇\tiny(578-579)}

\subsubsection{生平}

周宣帝宇文\xpinyin*{贇}(559年-580年6月22日),字乾伯,自稱天元皇帝,代郡武川县(今内蒙古自治区呼和浩特市武川县)人,北周武帝宇文邕長子,北周第四代皇帝(578年-579年),在位只有一年。

北周宣帝武成元年生於同州,是個暴虐荒淫的皇帝。宇文贇即位前,父親武帝對他管教極為嚴格,曾派人監視他的言行舉止,甚至只要犯錯就會嚴厲懲罰。

建德二年(573年),迎娶隨國公楊堅的長女楊麗華。

宣政元年(578年)武帝去世後,遺詔太子宇文贇襲統大寶。宇文贇即位,史稱「周天元」。宇文贇在父親死後,面無哀戚,抚摸着脚上被打的杖痕,大声对着武帝的棺材喊道:“死得太晚了!”

宣帝即位后,整日沉迷于酒色,极其荒淫无道,史称:“宣帝初立,即逞奢欲。”最後甚至五位皇后並立,此舉打破劉聰的“三后並立”的記錄,又大肆裝飾宮殿,且濫施刑罰,經常派親信監視大臣言行,宣政元年(578年),殺宇文憲。北周國勢日漸衰落。

大成元年(579年),在位僅一年的宣帝禪位於長子宇文闡(北周靜帝),自称天元皇帝,杨丽华为天元皇后,住处称为“天台”,对臣下自称为“天”,仍實際掌控朝政,在後宮享樂。大臣朝见時,必须事先吃斋三天、净身一天。又於全國大选美女,以充实後宮,大将军陈山提的第八女陈月仪,仪同元晟的第二女元乐尚最受寵愛。由于纵欲过度,嬉遊無度,宇文赟的健康恶化,大象二年五月己酉(580年6月22日),宣帝因享樂過度有疾,禪位後次年去世,時年22歲。

次年,杨坚廢靜帝(宇文衍)自立,改國號為隋,北周滅亡。

周宣帝死后,大象二年七月丙申葬定陵,具体位置不详。北周武帝孝陵以西已知其为北周重臣葬地,北周诸陵当在咸阳市渭城区底张镇一带。

唐令狐德棻《周書》評價宇文贇擢髮難數、罄竹難書,罪惡多端,最終沒有受屠戮而亡,實在是很幸運:「高祖識嗣子之非才,顧宗祏之至重,滯愛同於晉武,則哲異於宋宣。但欲威之以檟楚,期之於懲肅,義方之教,豈若是乎。卒使昏虐君臨,姦回肆毒,善無小而必棄,惡無大而弗為。窮南山之簡,未足書其過;盡東觀之筆,不能記其罪。然猶獲全首領,及子而亡,幸哉。」

\subsubsection{大成}

\begin{longtable}{|>{\centering\scriptsize}m{2em}|>{\centering\scriptsize}m{1.3em}|>{\centering}m{8.8em}|}
  % \caption{秦王政}\
  \toprule
  \SimHei \normalsize 年数 & \SimHei \scriptsize 公元 & \SimHei 大事件 \tabularnewline
  % \midrule
  \endfirsthead
  \toprule
  \SimHei \normalsize 年数 & \SimHei \scriptsize 公元 & \SimHei 大事件 \tabularnewline
  \midrule
  \endhead
  \midrule
  元年 & 579 & \tabularnewline
  \bottomrule
\end{longtable}


%%% Local Variables:
%%% mode: latex
%%% TeX-engine: xetex
%%% TeX-master: "../../Main"
%%% End:
