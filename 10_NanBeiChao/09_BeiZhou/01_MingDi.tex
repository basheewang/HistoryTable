%% -*- coding: utf-8 -*-
%% Time-stamp: <Chen Wang: 2019-12-23 16:08:56>

\subsection{明帝\tiny(557-560)}

\subsubsection{生平}

周明帝宇文毓(534年-560年5月30日,在位:557年-560年),小名統萬突。是南北朝时期北周的天王及皇帝。宇文泰庶长子。母亲是夫人姚氏。

557年九月因唯一的嫡弟宇文觉被堂兄晋公宇文护所废而即天王位,559年称帝。明帝明敏有识量,为宇文护所惮。武成二年(560年)夏四月,宇文护暗中命令李安在加糖的蒸饼中下毒,进献给周明帝,四月庚子(560年5月29日)周明帝病危,四月辛丑(560年5月30日)去世。临终前自知为宇文护所害,口述遗诏五百余字,称诸子年幼不堪大任,称赞异母弟鲁公宇文邕宽仁大度海内所闻,能昌大周朝的必是他,意即以宇文邕继位。宇文护也最终拥立宇文邕继位。

後人普遍認同明帝是一位治国有方的明君,在位期间颇有作为,勵精圖治。

周明帝死后葬昭陵,具体位置不详。北周武帝孝陵以西已知其为北周重臣葬地,北周诸陵当在咸阳市渭城区底张镇一带。

\subsubsection{武成}

\begin{longtable}{|>{\centering\scriptsize}m{2em}|>{\centering\scriptsize}m{1.3em}|>{\centering}m{8.8em}|}
  % \caption{秦王政}\
  \toprule
  \SimHei \normalsize 年数 & \SimHei \scriptsize 公元 & \SimHei 大事件 \tabularnewline
  % \midrule
  \endfirsthead
  \toprule
  \SimHei \normalsize 年数 & \SimHei \scriptsize 公元 & \SimHei 大事件 \tabularnewline
  \midrule
  \endhead
  \midrule
  元年 & 559 & \tabularnewline\hline
  二年 & 560 & \tabularnewline
  \bottomrule
\end{longtable}


%%% Local Variables:
%%% mode: latex
%%% TeX-engine: xetex
%%% TeX-master: "../../Main"
%%% End:
