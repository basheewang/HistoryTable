%% -*- coding: utf-8 -*-
%% Time-stamp: <Chen Wang: 2019-12-23 16:14:47>


\section{北周\tiny(557-581)}

\subsection{简介}

北周(557年—581年)是中國歷史上南北朝的北朝之一。又称後周(宋朝以后鲜用),由宇文氏建立,定都長安,北周自建國後,統治實權一直在霸府宇文護身上,皇帝無力與之抗阻,為了擺脫宇文護的束縛,經過一連串的計畫與鬥爭,北周武帝終於殺死了宇文護,掌握大權,並以德施政,人民安樂,在位時更成功滅北齊,統一北朝。但他死後三年,北周便被楊堅的隋所取代,後由隋滅陳,統一中國。

北周由宇文泰奠定根基。北魏在六鎮之亂時,宇文泰投靠權臣爾朱榮,隨其入關中討伐叛逆,後來投于以關中隴西為根據地的大將賀拔岳的麾下,並漸漸受重用。

控制洛陽的另一權臣高歡認為賀拔岳有不臣之心,故使隴西秦州軍人刺殺賀拔岳。賀拔岳所屬將領在賀遇刺後,擁立宇文泰為統帥。宇文泰只是表面上服從高歡,其實控制關隴。

北魏孝武帝在討伐高歡失敗後,逃奔關中。宇文泰雖收容了他。但不久就將孝武帝殺害,改擁立西魏文帝建立西魏(535年)。而東方的高歡在孝武帝逃入關中後擁立東魏孝靜帝,把朝廷遷到河北鄴城,建立東魏(534年)。

西魏建立後,宇文泰成為大丞相。宇文泰在三次戰役中大敗東魏,奠定宇文氏在關中的基礎。宇文泰任用蘇綽等人改革,使西魏進一步強盛。進而攻入南梁的成都,奪取西川地盤。

西魏恭帝三年(556年),宇文泰病死,由嫡长子宇文觉承袭为安定郡公、太师、大冢宰。次年,宇文泰之侄宇文護迫西魏恭帝禪讓,由宇文覺即位天王,建立北周,建都長安(即今陝西西安)。

宇文覺不滿宇文護專權,企圖剷除宇文護,但反被其所殺。宇文護擁立其庶兄宇文毓,是為北周明帝。幾年後,明帝被殺,又擁立其兄弟宇文邕為北周武帝。宇文護執掌政權十五年,成為北周實際上的主宰。他承繼宇文泰、蘇綽的政策,消滅威脅政權的軍閥,使北周政權更鞏固。北周武帝年間,宇文護的兒子亂政害民,宇文護的威望大降。天和七年(572年)三月,北周武帝乘機刺殺了宇文護,重奪政權。

北周武帝執政後,積極推廣漢化並勵精圖治。575年發兵征北齊,577年,北周滅北齊,統一華北,仅北齐营州刺史高宝宁未降,奉逃奔突厥的皇子高绍义为帝。北周統一華北後國力一度興盛,但北周武帝英年早逝,其繼位者北周宣帝宇文贇奢侈浮華,沉緬酒色,政治腐敗。周宣帝生前即传位年幼的儿子北周靜帝宇文闡。580年6月8日宣帝病死,外戚楊堅以大丞相身份輔政,乘機將北周重臣外遣,進而把持朝政。相州總管尉遲迥、鄖州總管司馬消難與益州總管王謙等人不滿楊堅專權,聯合叛變反抗楊堅,爆發尉遲迥之亂,但被楊堅所派的韋孝寬、王誼與高熲等人平定。期间杨坚亦诛杀北周明帝长子太师雍州牧毕王宇文贤及尚在人世的宇文泰五子,并与突厥通好,突厥他钵可汗遂将高绍义交给北周。581年3月4日,北周靜帝禪讓帝位於楊堅,楊堅受禅称帝,改國號隋,北周享國二十四年而亡。杨坚建国不久,就将北周近支宗室诛杀殆尽,将宇文洛封为介国公作为北周奉祀。

此時期佛教藝術創作,大多數位於長安。印度笈多王朝雕像為許多大型佛像的原型。在敦煌千佛洞則有一些北周風格的壁畫,在這些壁畫中,山水畫固然重要,不過仍遜於人物畫。

北周人口盛时约1250万。


\subsection{文帝简介}

宇文泰(507年-556年11月21日),字黑獭(一作黑泰),代郡武川县(今内蒙古自治区呼和浩特市武川县)人,鲜卑宇文部后裔,漢化鮮卑人,北朝西魏權臣,也是北周政權的奠基者,掌權22年。后追尊为文王,庙号太祖,武成元年(559年)追尊为文帝。

宇文泰先世为宇文部酋长。东汉末,宇文部加入鲜卑部落联盟,遂被鲜卑化,游牧于今内蒙古自治区西拉木伦河上游。

北魏末年六镇起义中,宇文泰随父宇文肱加入鲜于修礼的起义队伍。起义被尔朱荣镇压后,宇文泰成为其部将贺拔岳麾下。永安三年(530年),魏孝庄帝杀尔朱荣,但军权仍然操在尔朱氏手中。不久,尔朱氏败灭,高欢位居丞相,并由此掌权。魏孝武帝密诏贺拔岳,欲以之牵制高欢。

永熙三年(534年)贺拔岳為侯莫陈悦所杀,由寇洛接手部隊以稳定军心,在平凉,寇洛自认“智能本阙,不宜统帅”,赵贵建议迎宇文泰,“诸将犹豫未决”,赫连达亦称宇文泰“明略过人,一时之杰。今日之事,非此公不济”,旧部才往夏州迎宇文泰。宇文泰率领帐下轻骑兵迅速向平凉进发,当时高欢也派长史侯景招揽贺拔岳的部众,宇文泰到安定,在休息的住所遇到了侯景,宇文泰吐出正吃的食物上战马,对侯景说:“贺拔公虽死,宇文泰还在,您想要干什么?”侯景脸色苍白,回答说:“我也只是箭而已,随着他人射,怎么能自己决定。”侯景至此返回。宇文泰到了平涼,为贺拔岳哭泣非常悲痛,贺拔岳的将士们既悲伤又高兴的说:“宇文公来了,再没有担心的。”

宇文泰上表北魏孝武帝元修,相约共扶王室,元修遂下诏以宇文泰为大都督、雍州刺史兼尚书令。同年,宇文泰平定秦、陇,孝武帝封官为侍中、骠骑大将军、开府仪同三司,关西大都督,略阳县公,地位仅次于高欢。是年元修攜情婦元明月及宗室數人從前綫逃跑,投奔宇文泰。十月,高欢另立元善见为帝,迁都于邺(今河北省临漳县),是为东魏。北魏遂分裂。

元修性格強硬,不守禮法,與宇文泰關係搞得非常僵。永熙三年(535年)十二月,宇文泰杀元修及元明月,另立元明月之兄元宝炬为帝,是为西魏,而实际政权控制在宇文泰手中。同年,宇文泰與元修的妹妹馮翊公主結婚。

宇文泰足智多谋,有很强的指挥能力。与东魏多次交锋,互有胜负。大统三年(537年)春,东魏进攻潼关,宇文泰大败之。秋,东魏十万人进沙苑(今陕西大荔),宇文泰以不满万人乘东魏军轻敌,亲自鸣鼓奋战,获得大胜,俘虏七万人,史称“沙苑之战”。

大统十三年(547年),西魏守將韋孝寬以七千人馬留守位置險要的玉璧城,頂住高歡十萬鮮卑鐵騎長達五十餘天的輪番衝擊。高歡喪師達七萬,智力用盡,玉璧城卻始終屹立不倒,高歡愁悶無處發泄,被活活氣死。

经济上,劝课农桑,恢复了均田制。并注意屯田以资军用。曾采纳苏绰建议进行改革,制定了「墨入朱出」(臣子上奏用黑筆寫,上級回覆用紅筆寫)公文格式,以朱色、墨色区别财政支出与收入(中文「赤字」的由來),定出户籍册和胪列次年课役大数的计帐制度。大统十三年的计帐,在敦煌石窟里有残卷保存下来。后又针对地方官员制定六条诏书:清心、敦教化、尽地利、擢贤良、恤狱讼、均赋役。

宇文泰改革军队统辖系统,建立府兵制,以扩大兵源。这个制度为隋唐所沿用。形式上采取鲜卑旧八部制,立八柱国,实为六军。每个柱国大将军下设有两个大将军,共12個大将军;每个大将军下有两个开府,共24个开府;每个开府下有两个仪同,共48个仪同;一个仪同领兵千人。这样,六柱国合计有兵四万八千人左右。这就是历史上著名的府兵。

外交上,宇文泰采取了和北攻南的政策,对于北方的突厥、柔然曾通好,但对于南朝蕭梁则采取攻势,先后进占了益州和荆雍等中國西南地區。

政治上,宇文泰实行以德治教化为主,法治为辅的原则。法律上,主张不苛不暴,而“法不阿贵”。思想文化上,推崇儒学,曾在行台设学。俘虏王褒、宗懔等均受到礼遇。后又令卢辩仿周礼更改官制,实行北周六官制,甚至政府文告也要仿先秦体。

宇文泰恢复鲜卑旧姓,如恢复皇族元氏为拓跋氏。而所将士卒也改从主将的胡姓。从形式上胡化一批的汉人,楊忠授普六茹氏,李虎授大野氏。

魏恭帝三年九月,宇文泰巡视北方回到牵屯山时患病,十月乙亥(556年11月21日)在云阳宫去世,时年虚岁五十,遗体运回长安才发丧。

\subsection{闵帝简介}

周孝閔帝宇文覺(542年-557年),字陀羅尼,代郡武川县(今内蒙古自治区呼和浩特市武川县)人,追尊周文帝宇文泰嫡长子,南北朝時代北周的开国君主,號稱天王,但實際上是權臣宇文護的傀儡。

七歲(周書記為九歲)時,被封為略陽郡公。當時有善於面相者史元華為他面相,私下告訴他的親人:「這位公子有至貴之相,但可惜的是他的寿命与地位不相称。」

556年三月,西魏恭帝拓跋廓命宇文覺為安定公世子;四月,拜為大司馬。到了十月,宇文泰過世,由宇文覺繼承他太師、安定公等官爵。十二月,拓跋廓又下詔以岐陽之地封宇文觉為周公,不久即禅位,派济北公拓跋迪将皇帝的玉玺和绶带送给宇文觉。並於557年正式即天王位,是為北周的開始。

宇文覺是宇文泰的第三子,生性剛毅果敢,對於其堂兄宇文護專政感到相當不滿,而司會李植與軍司馬孫恆也對宇文護權高位重頗有微詞,便與乙弗鳳、賀拔提等人一同私下向宇文覺請求誅殺宇文護,宇文覺同意。他們又找了張光洛一同行事,不料張光洛卻將此事告訴宇文護。於是宇文護改封李植為梁州刺史,孫恆為潼州刺史,將他們外放。乙弗鳳因此表示他將把宇文護引進宮後誅殺,但張光洛又將此事告訴宇文護,宇文護反而與尉遲綱合謀廢立之事,先設計誅殺乙弗鳳,並使宇文覺身邊沒有侍衛;接著派賀蘭祥逼迫宇文覺退位,將他貶為略陽公並幽禁,不久將他殺害,死时只有15岁。

孝闵帝为人温和并体恤民情,多次免除百姓的繁重税务,算是北朝时期的一位明君。

後來他的庶弟周武帝宇文邕誅殺宇文護,追認了宇文覺的开国皇帝身份,派遣蜀國公尉遲迥在南郊上諡其為孝閔皇帝,稱其陵墓為靜陵。北周武帝孝陵以西已知其为北周重臣葬地,北周诸陵当在咸阳市渭城区底张镇一带。

%% -*- coding: utf-8 -*-
%% Time-stamp: <Chen Wang: 2019-12-23 16:08:56>

\subsection{明帝\tiny(557-560)}

\subsubsection{生平}

周明帝宇文毓(534年-560年5月30日,在位:557年-560年),小名統萬突。是南北朝时期北周的天王及皇帝。宇文泰庶长子。母亲是夫人姚氏。

557年九月因唯一的嫡弟宇文觉被堂兄晋公宇文护所废而即天王位,559年称帝。明帝明敏有识量,为宇文护所惮。武成二年(560年)夏四月,宇文护暗中命令李安在加糖的蒸饼中下毒,进献给周明帝,四月庚子(560年5月29日)周明帝病危,四月辛丑(560年5月30日)去世。临终前自知为宇文护所害,口述遗诏五百余字,称诸子年幼不堪大任,称赞异母弟鲁公宇文邕宽仁大度海内所闻,能昌大周朝的必是他,意即以宇文邕继位。宇文护也最终拥立宇文邕继位。

後人普遍認同明帝是一位治国有方的明君,在位期间颇有作为,勵精圖治。

周明帝死后葬昭陵,具体位置不详。北周武帝孝陵以西已知其为北周重臣葬地,北周诸陵当在咸阳市渭城区底张镇一带。

\subsubsection{武成}

\begin{longtable}{|>{\centering\scriptsize}m{2em}|>{\centering\scriptsize}m{1.3em}|>{\centering}m{8.8em}|}
  % \caption{秦王政}\
  \toprule
  \SimHei \normalsize 年数 & \SimHei \scriptsize 公元 & \SimHei 大事件 \tabularnewline
  % \midrule
  \endfirsthead
  \toprule
  \SimHei \normalsize 年数 & \SimHei \scriptsize 公元 & \SimHei 大事件 \tabularnewline
  \midrule
  \endhead
  \midrule
  元年 & 559 & \tabularnewline\hline
  二年 & 560 & \tabularnewline
  \bottomrule
\end{longtable}


%%% Local Variables:
%%% mode: latex
%%% TeX-engine: xetex
%%% TeX-master: "../../Main"
%%% End:

%% -*- coding: utf-8 -*-
%% Time-stamp: <Chen Wang: 2019-12-23 16:12:35>

\subsection{武帝\tiny(560-578)}

\subsubsection{生平}

周武帝宇文邕yōng(543年-578年6月21日),字祢罗突,代郡武川县(今内蒙古自治区呼和浩特市武川县)人,追尊周文帝宇文泰第四子,北周第三位皇帝(560年—578年在位),期間推動建德毀佛,以求富國強兵,是三武滅佛之一。滅亡北齐,統一逾三分之二的中國本部,為12年後隋滅陳之戰打下基礎。在位18年。

宇文邕在西魏时封辅城郡公,北周封鲁国公,其堂兄宇文护专横跋扈,连杀二帝,又立宇文邕为帝。宇文邕18歲即位,因其兄弟先後被宇文护所殺,武帝即位後,為免自身也遭殺身之禍,對宇文护表示恭敬,讓他主理國家大事,以靜待時機。武帝不甘做傀儡,最終於572年杀死宇文护,得以亲政。

在位期间,武帝不像其父欲恢復鲜卑旧俗,反而极力摆脱鲜卑旧俗並接受漢文化,且自己也整顿吏治,使北周政治清明,百姓生活安定,国势强盛。宇文邕生活俭朴,能够及时关心民间疾苦。据史书记载,他“身布袍,寝布被……后宫不过十余人。”他的漢文化政策為日後楊堅的統一奠定基礎。

另外他还聽從道士衛元嵩和张宾意見大举灭佛,捣毁全国大量佛塔、佛寺,严令僧尼还俗,这是“求武器于塔庙之间、以士兵于僧侣之下”的富國強兵运动,是为建德毀佛。而在宇文邕禁止佛教之外,而且卫元嵩自己没想到连道教也被禁止,自己和众多的道士也被迫还俗。

正当北周日益强盛的时候,北齐却日衰。建德四年(575年)末,宇文邕於是出兵大举进攻腐朽的北齐,并于一年半后(即建德六年,577年)灭北齐。

宣政元年(578年)宇文邕率軍分五道伐突厥,未出發即病死,年僅36岁,谥号武帝,庙号高祖。他可以說是南北朝兩百多年的亂世中少數稱得上有作為的君主。

在史書中,他是一位嚴父,曾對其繼承人、教而不善的太子宇文贇(后来的北周宣帝)施用体罚,並多次威脅要廢去其太子地位,但最後都沒有實行。这样的举措反而收到了反效果,让宇文赟对他记恨,而更加不听从他的说教。宣帝繼位后荒淫无度,不到三年内其子就被杨坚篡位,北周灭亡。

宇文邕發明類似樗蒲、打馬的擲賽遊戲,史稱北周象戲,并編著有《象经》一书。

唐令狐德棻《周書》評價宇文邕沉著、毅力且有智謀,韜光晦跡、除國害。之後勵精圖治、除卻奢靡、凡事從儉,戰爭時與軍士同喜悲。令狐德棻認為,再一兩年,宇文邕就能天一大一統:「帝沉毅有智謀。初以晉公護專權,常自晦跡,人莫測其深淺。及誅護之後,始親萬機。克己勵精,聽覽不怠。用法嚴整,多所罪殺。號令懇惻,唯屬意於政。羣下畏服,莫不肅然。性旣明察,少於恩惠。凡布懷立行,皆欲踰越古人。身衣布袍,寢布被,無金寶之飾,諸宮殿華綺者,皆撤毀之,改為土階數尺,不施櫨栱。其雕文刻鏤,錦繡纂組,一皆禁斷。後宮嬪禦,不過十餘人。勞謙接下,自強不息。以海內未康,銳情教習。至於校兵閱武,步行山谷,履涉勤苦,皆人所不堪。平齊之役,見軍士有跣行者,帝親脫靴以賜之。每宴會將士,必自執杯勸酒,或手付賜物。至於征伐之處,躬在行陣。性又果決,能斷大事。故能得士卒死力,以弱制強。破齊之後,遂欲窮兵極武,平突厥,定江南,一二年間,必使天下一統,此其志也。」

唐令狐德棻《周書》評價宇文邕認真治國、同匹夫節儉度日,成為一時明君。雖然因為長年征戰,被稱窮兵黷武,但他的鴻圖遠略,是能凌駕古代王者:「自東西否隔,二國爭強,戎馬生郊,干戈日用,兵連禍結,力敵勢均,疆埸之事,一彼一此。高祖纘業,未親萬機,慮遠謀深,以蒙養正。及英威電發,朝政惟新,內難旣除,外略方始。乃苦心焦思,克己勵精,勞役為士卒之先,居處同匹夫之儉。脩富民之政,務強兵之術,乘讐人之有釁,順大道而推亡。五年之間,大勳斯集。攄祖宗之宿憤,拯東夏之阽危,盛矣哉,其有成功者也。若使翌日之瘳無爽,經營之志獲申,黷武窮兵,雖見譏於良史,雄圖遠略,足方駕於前王者歟。」


\subsubsection{保定}

\begin{longtable}{|>{\centering\scriptsize}m{2em}|>{\centering\scriptsize}m{1.3em}|>{\centering}m{8.8em}|}
  % \caption{秦王政}\
  \toprule
  \SimHei \normalsize 年数 & \SimHei \scriptsize 公元 & \SimHei 大事件 \tabularnewline
  % \midrule
  \endfirsthead
  \toprule
  \SimHei \normalsize 年数 & \SimHei \scriptsize 公元 & \SimHei 大事件 \tabularnewline
  \midrule
  \endhead
  \midrule
  元年 & 561 & \tabularnewline\hline
  二年 & 562 & \tabularnewline\hline
  三年 & 563 & \tabularnewline\hline
  四年 & 564 & \tabularnewline\hline
  五年 & 565 & \tabularnewline
  \bottomrule
\end{longtable}

\subsubsection{天和}

\begin{longtable}{|>{\centering\scriptsize}m{2em}|>{\centering\scriptsize}m{1.3em}|>{\centering}m{8.8em}|}
  % \caption{秦王政}\
  \toprule
  \SimHei \normalsize 年数 & \SimHei \scriptsize 公元 & \SimHei 大事件 \tabularnewline
  % \midrule
  \endfirsthead
  \toprule
  \SimHei \normalsize 年数 & \SimHei \scriptsize 公元 & \SimHei 大事件 \tabularnewline
  \midrule
  \endhead
  \midrule
  元年 & 566 & \tabularnewline\hline
  二年 & 567 & \tabularnewline\hline
  三年 & 568 & \tabularnewline\hline
  四年 & 569 & \tabularnewline\hline
  五年 & 570 & \tabularnewline\hline
  六年 & 571 & \tabularnewline\hline
  七年 & 572 & \tabularnewline
  \bottomrule
\end{longtable}

\subsubsection{建德}

\begin{longtable}{|>{\centering\scriptsize}m{2em}|>{\centering\scriptsize}m{1.3em}|>{\centering}m{8.8em}|}
  % \caption{秦王政}\
  \toprule
  \SimHei \normalsize 年数 & \SimHei \scriptsize 公元 & \SimHei 大事件 \tabularnewline
  % \midrule
  \endfirsthead
  \toprule
  \SimHei \normalsize 年数 & \SimHei \scriptsize 公元 & \SimHei 大事件 \tabularnewline
  \midrule
  \endhead
  \midrule
  元年 & 572 & \tabularnewline\hline
  二年 & 573 & \tabularnewline\hline
  三年 & 574 & \tabularnewline\hline
  四年 & 575 & \tabularnewline\hline
  五年 & 576 & \tabularnewline\hline
  六年 & 578 & \tabularnewline
  \bottomrule
\end{longtable}

\subsubsection{宣政}

\begin{longtable}{|>{\centering\scriptsize}m{2em}|>{\centering\scriptsize}m{1.3em}|>{\centering}m{8.8em}|}
  % \caption{秦王政}\
  \toprule
  \SimHei \normalsize 年数 & \SimHei \scriptsize 公元 & \SimHei 大事件 \tabularnewline
  % \midrule
  \endfirsthead
  \toprule
  \SimHei \normalsize 年数 & \SimHei \scriptsize 公元 & \SimHei 大事件 \tabularnewline
  \midrule
  \endhead
  \midrule
  元年 & 578 & \tabularnewline
  \bottomrule
\end{longtable}


%%% Local Variables:
%%% mode: latex
%%% TeX-engine: xetex
%%% TeX-master: "../../Main"
%%% End:

%% -*- coding: utf-8 -*-
%% Time-stamp: <Chen Wang: 2021-11-01 15:15:27>

\subsection{宣帝宇文贇\tiny(578-579)}

\subsubsection{生平}

周宣帝宇文\xpinyin*{贇}(559年-580年6月22日),字乾伯,自稱天元皇帝,代郡武川县(今内蒙古自治区呼和浩特市武川县)人,北周武帝宇文邕長子,北周第四代皇帝(578年-579年),在位只有一年。

北周宣帝武成元年生於同州,是個暴虐荒淫的皇帝。宇文贇即位前,父親武帝對他管教極為嚴格,曾派人監視他的言行舉止,甚至只要犯錯就會嚴厲懲罰。

建德二年(573年),迎娶隨國公楊堅的長女楊麗華。

宣政元年(578年)武帝去世後,遺詔太子宇文贇襲統大寶。宇文贇即位,史稱「周天元」。宇文贇在父親死後,面無哀戚,抚摸着脚上被打的杖痕,大声对着武帝的棺材喊道:“死得太晚了!”

宣帝即位后,整日沉迷于酒色,极其荒淫无道,史称:“宣帝初立,即逞奢欲。”最後甚至五位皇后並立,此舉打破劉聰的“三后並立”的記錄,又大肆裝飾宮殿,且濫施刑罰,經常派親信監視大臣言行,宣政元年(578年),殺宇文憲。北周國勢日漸衰落。

大成元年(579年),在位僅一年的宣帝禪位於長子宇文闡(北周靜帝),自称天元皇帝,杨丽华为天元皇后,住处称为“天台”,对臣下自称为“天”,仍實際掌控朝政,在後宮享樂。大臣朝见時,必须事先吃斋三天、净身一天。又於全國大选美女,以充实後宮,大将军陈山提的第八女陈月仪,仪同元晟的第二女元乐尚最受寵愛。由于纵欲过度,嬉遊無度,宇文赟的健康恶化,大象二年五月己酉(580年6月22日),宣帝因享樂過度有疾,禪位後次年去世,時年22歲。

次年,杨坚廢靜帝(宇文衍)自立,改國號為隋,北周滅亡。

周宣帝死后,大象二年七月丙申葬定陵,具体位置不详。北周武帝孝陵以西已知其为北周重臣葬地,北周诸陵当在咸阳市渭城区底张镇一带。

唐令狐德棻《周書》評價宇文贇擢髮難數、罄竹難書,罪惡多端,最終沒有受屠戮而亡,實在是很幸運:「高祖識嗣子之非才,顧宗祏之至重,滯愛同於晉武,則哲異於宋宣。但欲威之以檟楚,期之於懲肅,義方之教,豈若是乎。卒使昏虐君臨,姦回肆毒,善無小而必棄,惡無大而弗為。窮南山之簡,未足書其過;盡東觀之筆,不能記其罪。然猶獲全首領,及子而亡,幸哉。」

\subsubsection{大成}

\begin{longtable}{|>{\centering\scriptsize}m{2em}|>{\centering\scriptsize}m{1.3em}|>{\centering}m{8.8em}|}
  % \caption{秦王政}\
  \toprule
  \SimHei \normalsize 年数 & \SimHei \scriptsize 公元 & \SimHei 大事件 \tabularnewline
  % \midrule
  \endfirsthead
  \toprule
  \SimHei \normalsize 年数 & \SimHei \scriptsize 公元 & \SimHei 大事件 \tabularnewline
  \midrule
  \endhead
  \midrule
  元年 & 579 & \tabularnewline
  \bottomrule
\end{longtable}


%%% Local Variables:
%%% mode: latex
%%% TeX-engine: xetex
%%% TeX-master: "../../Main"
%%% End:

%% -*- coding: utf-8 -*-
%% Time-stamp: <Chen Wang: 2021-11-01 15:15:40>

\subsection{静帝宇文闡\tiny(579-581)}

\subsubsection{生平}

周靜帝宇文\xpinyin*{闡}(573年8月1日-581年7月9日),原名宇文衍,代郡武川县(今内蒙古自治区呼和浩特市武川县)人,北周末代皇帝(第五代,579年—581年在位),周宣帝宇文贇長子。母親是朱滿月。

建德二年六月壬子日(573年8月1日)出生。

大象元年二月辛巳(579年4月1日),受宣帝內禪即位,時年七歲。次年,宣帝崩。刘昉、鄭译決定以楊堅为輔政大臣(后李德林提议下成为大丞相)。期间杨坚平定尉迟迥之乱、剪除北周宗室,逐渐形成代国之势。

大象三年(581年)北周静帝禅让帝位于杨坚,杨坚登基。至此北周滅亡,隋朝建立。楊堅封宇文阐为介国公,食邑一万户,车服礼乐仍按北周天子的旧制,上书皇帝不称为表,皇帝回复不称诏。虽有这样的规定,实际上未能实行。

開皇元年五月壬申日(《隋书》作五月辛未日,相差一天),杨坚暗中派人殺死介国公宇文阐,时年九岁,后表示大为震惊,发布死讯,在朝堂举哀,隆重祭悼,谥为静皇帝,葬在恭陵;以周静帝的堂叔祖宇文洛继为介国公。

\subsubsection{大象}

\begin{longtable}{|>{\centering\scriptsize}m{2em}|>{\centering\scriptsize}m{1.3em}|>{\centering}m{8.8em}|}
  % \caption{秦王政}\
  \toprule
  \SimHei \normalsize 年数 & \SimHei \scriptsize 公元 & \SimHei 大事件 \tabularnewline
  % \midrule
  \endfirsthead
  \toprule
  \SimHei \normalsize 年数 & \SimHei \scriptsize 公元 & \SimHei 大事件 \tabularnewline
  \midrule
  \endhead
  \midrule
  元年 & 579 & \tabularnewline\hline
  二年 & 580 & \tabularnewline
  \bottomrule
\end{longtable}

\subsubsection{大定}

\begin{longtable}{|>{\centering\scriptsize}m{2em}|>{\centering\scriptsize}m{1.3em}|>{\centering}m{8.8em}|}
  % \caption{秦王政}\
  \toprule
  \SimHei \normalsize 年数 & \SimHei \scriptsize 公元 & \SimHei 大事件 \tabularnewline
  % \midrule
  \endfirsthead
  \toprule
  \SimHei \normalsize 年数 & \SimHei \scriptsize 公元 & \SimHei 大事件 \tabularnewline
  \midrule
  \endhead
  \midrule
  元年 & 581 & \tabularnewline
  \bottomrule
\end{longtable}


%%% Local Variables:
%%% mode: latex
%%% TeX-engine: xetex
%%% TeX-master: "../../Main"
%%% End:



%%% Local Variables:
%%% mode: latex
%%% TeX-engine: xetex
%%% TeX-master: "../../Main"
%%% End:
