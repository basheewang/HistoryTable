%% -*- coding: utf-8 -*-
%% Time-stamp: <Chen Wang: 2021-11-01 15:15:11>

\subsection{武帝宇文邕\tiny(560-578)}

\subsubsection{生平}

周武帝宇文邕yōng(543年-578年6月21日),字祢罗突,代郡武川县(今内蒙古自治区呼和浩特市武川县)人,追尊周文帝宇文泰第四子,北周第三位皇帝(560年—578年在位),期間推動建德毀佛,以求富國強兵,是三武滅佛之一。滅亡北齐,統一逾三分之二的中國本部,為12年後隋滅陳之戰打下基礎。在位18年。

宇文邕在西魏时封辅城郡公,北周封鲁国公,其堂兄宇文护专横跋扈,连杀二帝,又立宇文邕为帝。宇文邕18歲即位,因其兄弟先後被宇文护所殺,武帝即位後,為免自身也遭殺身之禍,對宇文护表示恭敬,讓他主理國家大事,以靜待時機。武帝不甘做傀儡,最終於572年杀死宇文护,得以亲政。

在位期间,武帝不像其父欲恢復鲜卑旧俗,反而极力摆脱鲜卑旧俗並接受漢文化,且自己也整顿吏治,使北周政治清明,百姓生活安定,国势强盛。宇文邕生活俭朴,能够及时关心民间疾苦。据史书记载,他“身布袍,寝布被……后宫不过十余人。”他的漢文化政策為日後楊堅的統一奠定基礎。

另外他还聽從道士衛元嵩和张宾意見大举灭佛,捣毁全国大量佛塔、佛寺,严令僧尼还俗,这是“求武器于塔庙之间、以士兵于僧侣之下”的富國強兵运动,是为建德毀佛。而在宇文邕禁止佛教之外,而且卫元嵩自己没想到连道教也被禁止,自己和众多的道士也被迫还俗。

正当北周日益强盛的时候,北齐却日衰。建德四年(575年)末,宇文邕於是出兵大举进攻腐朽的北齐,并于一年半后(即建德六年,577年)灭北齐。

宣政元年(578年)宇文邕率軍分五道伐突厥,未出發即病死,年僅36岁,谥号武帝,庙号高祖。他可以說是南北朝兩百多年的亂世中少數稱得上有作為的君主。

在史書中,他是一位嚴父,曾對其繼承人、教而不善的太子宇文贇(后来的北周宣帝)施用体罚,並多次威脅要廢去其太子地位,但最後都沒有實行。这样的举措反而收到了反效果,让宇文赟对他记恨,而更加不听从他的说教。宣帝繼位后荒淫无度,不到三年内其子就被杨坚篡位,北周灭亡。

宇文邕發明類似樗蒲、打馬的擲賽遊戲,史稱北周象戲,并編著有《象经》一书。

唐令狐德棻《周書》評價宇文邕沉著、毅力且有智謀,韜光晦跡、除國害。之後勵精圖治、除卻奢靡、凡事從儉,戰爭時與軍士同喜悲。令狐德棻認為,再一兩年,宇文邕就能天一大一統:「帝沉毅有智謀。初以晉公護專權,常自晦跡,人莫測其深淺。及誅護之後,始親萬機。克己勵精,聽覽不怠。用法嚴整,多所罪殺。號令懇惻,唯屬意於政。羣下畏服,莫不肅然。性旣明察,少於恩惠。凡布懷立行,皆欲踰越古人。身衣布袍,寢布被,無金寶之飾,諸宮殿華綺者,皆撤毀之,改為土階數尺,不施櫨栱。其雕文刻鏤,錦繡纂組,一皆禁斷。後宮嬪禦,不過十餘人。勞謙接下,自強不息。以海內未康,銳情教習。至於校兵閱武,步行山谷,履涉勤苦,皆人所不堪。平齊之役,見軍士有跣行者,帝親脫靴以賜之。每宴會將士,必自執杯勸酒,或手付賜物。至於征伐之處,躬在行陣。性又果決,能斷大事。故能得士卒死力,以弱制強。破齊之後,遂欲窮兵極武,平突厥,定江南,一二年間,必使天下一統,此其志也。」

唐令狐德棻《周書》評價宇文邕認真治國、同匹夫節儉度日,成為一時明君。雖然因為長年征戰,被稱窮兵黷武,但他的鴻圖遠略,是能凌駕古代王者:「自東西否隔,二國爭強,戎馬生郊,干戈日用,兵連禍結,力敵勢均,疆埸之事,一彼一此。高祖纘業,未親萬機,慮遠謀深,以蒙養正。及英威電發,朝政惟新,內難旣除,外略方始。乃苦心焦思,克己勵精,勞役為士卒之先,居處同匹夫之儉。脩富民之政,務強兵之術,乘讐人之有釁,順大道而推亡。五年之間,大勳斯集。攄祖宗之宿憤,拯東夏之阽危,盛矣哉,其有成功者也。若使翌日之瘳無爽,經營之志獲申,黷武窮兵,雖見譏於良史,雄圖遠略,足方駕於前王者歟。」


\subsubsection{保定}

\begin{longtable}{|>{\centering\scriptsize}m{2em}|>{\centering\scriptsize}m{1.3em}|>{\centering}m{8.8em}|}
  % \caption{秦王政}\
  \toprule
  \SimHei \normalsize 年数 & \SimHei \scriptsize 公元 & \SimHei 大事件 \tabularnewline
  % \midrule
  \endfirsthead
  \toprule
  \SimHei \normalsize 年数 & \SimHei \scriptsize 公元 & \SimHei 大事件 \tabularnewline
  \midrule
  \endhead
  \midrule
  元年 & 561 & \tabularnewline\hline
  二年 & 562 & \tabularnewline\hline
  三年 & 563 & \tabularnewline\hline
  四年 & 564 & \tabularnewline\hline
  五年 & 565 & \tabularnewline
  \bottomrule
\end{longtable}

\subsubsection{天和}

\begin{longtable}{|>{\centering\scriptsize}m{2em}|>{\centering\scriptsize}m{1.3em}|>{\centering}m{8.8em}|}
  % \caption{秦王政}\
  \toprule
  \SimHei \normalsize 年数 & \SimHei \scriptsize 公元 & \SimHei 大事件 \tabularnewline
  % \midrule
  \endfirsthead
  \toprule
  \SimHei \normalsize 年数 & \SimHei \scriptsize 公元 & \SimHei 大事件 \tabularnewline
  \midrule
  \endhead
  \midrule
  元年 & 566 & \tabularnewline\hline
  二年 & 567 & \tabularnewline\hline
  三年 & 568 & \tabularnewline\hline
  四年 & 569 & \tabularnewline\hline
  五年 & 570 & \tabularnewline\hline
  六年 & 571 & \tabularnewline\hline
  七年 & 572 & \tabularnewline
  \bottomrule
\end{longtable}

\subsubsection{建德}

\begin{longtable}{|>{\centering\scriptsize}m{2em}|>{\centering\scriptsize}m{1.3em}|>{\centering}m{8.8em}|}
  % \caption{秦王政}\
  \toprule
  \SimHei \normalsize 年数 & \SimHei \scriptsize 公元 & \SimHei 大事件 \tabularnewline
  % \midrule
  \endfirsthead
  \toprule
  \SimHei \normalsize 年数 & \SimHei \scriptsize 公元 & \SimHei 大事件 \tabularnewline
  \midrule
  \endhead
  \midrule
  元年 & 572 & \tabularnewline\hline
  二年 & 573 & \tabularnewline\hline
  三年 & 574 & \tabularnewline\hline
  四年 & 575 & \tabularnewline\hline
  五年 & 576 & \tabularnewline\hline
  六年 & 578 & \tabularnewline
  \bottomrule
\end{longtable}

\subsubsection{宣政}

\begin{longtable}{|>{\centering\scriptsize}m{2em}|>{\centering\scriptsize}m{1.3em}|>{\centering}m{8.8em}|}
  % \caption{秦王政}\
  \toprule
  \SimHei \normalsize 年数 & \SimHei \scriptsize 公元 & \SimHei 大事件 \tabularnewline
  % \midrule
  \endfirsthead
  \toprule
  \SimHei \normalsize 年数 & \SimHei \scriptsize 公元 & \SimHei 大事件 \tabularnewline
  \midrule
  \endhead
  \midrule
  元年 & 578 & \tabularnewline
  \bottomrule
\end{longtable}


%%% Local Variables:
%%% mode: latex
%%% TeX-engine: xetex
%%% TeX-master: "../../Main"
%%% End:
