%% -*- coding: utf-8 -*-
%% Time-stamp: <Chen Wang: 2021-11-01 15:15:40>

\subsection{静帝宇文闡\tiny(579-581)}

\subsubsection{生平}

周靜帝宇文\xpinyin*{闡}(573年8月1日-581年7月9日),原名宇文衍,代郡武川县(今内蒙古自治区呼和浩特市武川县)人,北周末代皇帝(第五代,579年—581年在位),周宣帝宇文贇長子。母親是朱滿月。

建德二年六月壬子日(573年8月1日)出生。

大象元年二月辛巳(579年4月1日),受宣帝內禪即位,時年七歲。次年,宣帝崩。刘昉、鄭译決定以楊堅为輔政大臣(后李德林提议下成为大丞相)。期间杨坚平定尉迟迥之乱、剪除北周宗室,逐渐形成代国之势。

大象三年(581年)北周静帝禅让帝位于杨坚,杨坚登基。至此北周滅亡,隋朝建立。楊堅封宇文阐为介国公,食邑一万户,车服礼乐仍按北周天子的旧制,上书皇帝不称为表,皇帝回复不称诏。虽有这样的规定,实际上未能实行。

開皇元年五月壬申日(《隋书》作五月辛未日,相差一天),杨坚暗中派人殺死介国公宇文阐,时年九岁,后表示大为震惊,发布死讯,在朝堂举哀,隆重祭悼,谥为静皇帝,葬在恭陵;以周静帝的堂叔祖宇文洛继为介国公。

\subsubsection{大象}

\begin{longtable}{|>{\centering\scriptsize}m{2em}|>{\centering\scriptsize}m{1.3em}|>{\centering}m{8.8em}|}
  % \caption{秦王政}\
  \toprule
  \SimHei \normalsize 年数 & \SimHei \scriptsize 公元 & \SimHei 大事件 \tabularnewline
  % \midrule
  \endfirsthead
  \toprule
  \SimHei \normalsize 年数 & \SimHei \scriptsize 公元 & \SimHei 大事件 \tabularnewline
  \midrule
  \endhead
  \midrule
  元年 & 579 & \tabularnewline\hline
  二年 & 580 & \tabularnewline
  \bottomrule
\end{longtable}

\subsubsection{大定}

\begin{longtable}{|>{\centering\scriptsize}m{2em}|>{\centering\scriptsize}m{1.3em}|>{\centering}m{8.8em}|}
  % \caption{秦王政}\
  \toprule
  \SimHei \normalsize 年数 & \SimHei \scriptsize 公元 & \SimHei 大事件 \tabularnewline
  % \midrule
  \endfirsthead
  \toprule
  \SimHei \normalsize 年数 & \SimHei \scriptsize 公元 & \SimHei 大事件 \tabularnewline
  \midrule
  \endhead
  \midrule
  元年 & 581 & \tabularnewline
  \bottomrule
\end{longtable}


%%% Local Variables:
%%% mode: latex
%%% TeX-engine: xetex
%%% TeX-master: "../../Main"
%%% End:
