%% -*- coding: utf-8 -*-
%% Time-stamp: <Chen Wang: 2019-12-23 15:31:38>

\subsection{废帝\tiny(551-554)}

\subsubsection{生平}

元钦(525年-554年),河南郡洛阳县(今河南省洛阳市东)人,西魏文帝元宝炬长子、母為乙弗皇后。

大統元年(535年),元寶炬登基建立西魏時,元钦被立為皇太子。不久丞相宇文泰主動許配女兒給元欽,成為元欽的岳父。大統十七年(551年),文帝駕崩,太子元欽即位,沿用文帝年号,次年(552年)去年号,称元年。次年(553年),宇文泰主動去丞相位。同年十一月,尚書元烈謀誅安定公宇文泰,反被宇文泰所處死。此後,元欽對此事常有怨言,欲誅宇文泰,竟聯合宇文泰的女婿李基、李暉和于翼。結果被三人告密。登位第三年(554年)二月被宇文泰废黜,安置雍州。宇文泰立其庶弟齊王元廓。元钦史稱「廢帝」。恭帝元年四月庚戌(554年),宇文泰用毒酒毒死了元钦。

\subsubsection{无年号}

\begin{longtable}{|>{\centering\scriptsize}m{2em}|>{\centering\scriptsize}m{1.3em}|>{\centering}m{8.8em}|}
  % \caption{秦王政}\
  \toprule
  \SimHei \normalsize 年数 & \SimHei \scriptsize 公元 & \SimHei 大事件 \tabularnewline
  % \midrule
  \endfirsthead
  \toprule
  \SimHei \normalsize 年数 & \SimHei \scriptsize 公元 & \SimHei 大事件 \tabularnewline
  \midrule
  \endhead
  \midrule
  元年 & 551 & \tabularnewline\hline
  二年 & 552 & \tabularnewline\hline
  三年 & 553 & \tabularnewline\hline
  四年 & 554 & \tabularnewline
  \bottomrule
\end{longtable}


%%% Local Variables:
%%% mode: latex
%%% TeX-engine: xetex
%%% TeX-master: "../../Main"
%%% End:
