%% -*- coding: utf-8 -*-
%% Time-stamp: <Chen Wang: 2021-11-01 15:13:24>

\subsection{文帝元宝炬\tiny(535-551)}

\subsubsection{生平}

魏文帝元宝炬(507年-551年3月28日),河南郡洛阳县(今河南省洛阳市东)人,魏孝文帝元宏之孙,京兆王元愉第三子,母杨奥妃,南北朝時代西魏建立者。

元宝炬在魏宣武帝時因為父親獲罪受牽連,年幼的一眾兄弟皆被幽禁在宗正寺,在宣武帝死後才重獲自由,被叔叔元懌收養。武泰年间封邵县侯,530年被封為南陽王。534年跟隨堂弟魏孝武帝元修入關中投靠宇文泰。

永熙三年闰十二月癸巳(535年2月3日)宇文泰毒死魏孝武帝后,秘密不发布丧事,宇文泰与群臣商议册立新皇帝,众人大多推举魏孝武帝哥哥的儿子广平王元赞为嗣君。濮阳王元顺在正室以外的房间流下眼泪对宇文泰说:“高欢逼迫驱逐先帝,册立幼年的君主来独揽大权,您应该反其道而行之。广平王虽然与先帝亲近,但是年幼,不适合居于皇帝之位,不如册立年长的君主来奉戴。”宇文泰深深地赞同元顺的意见,因此发布国家的丧事,向太宰、南阳王元宝炬奉上帝号,改元大统,政權實際上由宇文泰控制。大统十七年三月庚戌(551年3月28日),元宝炬去世,虚岁四十五,四月庚辰(551年4月27日)葬于永陵。嫡長子皇太子元钦嗣位。

\subsubsection{大统}

\begin{longtable}{|>{\centering\scriptsize}m{2em}|>{\centering\scriptsize}m{1.3em}|>{\centering}m{8.8em}|}
  % \caption{秦王政}\
  \toprule
  \SimHei \normalsize 年数 & \SimHei \scriptsize 公元 & \SimHei 大事件 \tabularnewline
  % \midrule
  \endfirsthead
  \toprule
  \SimHei \normalsize 年数 & \SimHei \scriptsize 公元 & \SimHei 大事件 \tabularnewline
  \midrule
  \endhead
  \midrule
  元年 & 535 & \tabularnewline\hline
  二年 & 536 & \tabularnewline\hline
  三年 & 537 & \tabularnewline\hline
  四年 & 538 & \tabularnewline\hline
  五年 & 539 & \tabularnewline\hline
  六年 & 540 & \tabularnewline\hline
  七年 & 541 & \tabularnewline\hline
  八年 & 512 & \tabularnewline\hline
  九年 & 513 & \tabularnewline\hline
  十年 & 544 & \tabularnewline\hline
  十一年 & 545 & \tabularnewline\hline
  十二年 & 546 & \tabularnewline\hline
  十三年 & 547 & \tabularnewline\hline
  十四年 & 548 & \tabularnewline\hline
  十五年 & 549 & \tabularnewline\hline
  十六年 & 550 & \tabularnewline\hline
  十七年 & 551 & \tabularnewline
  \bottomrule
\end{longtable}


%%% Local Variables:
%%% mode: latex
%%% TeX-engine: xetex
%%% TeX-master: "../../Main"
%%% End:
