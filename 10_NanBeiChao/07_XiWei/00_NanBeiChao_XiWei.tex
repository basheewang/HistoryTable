%% -*- coding: utf-8 -*-
%% Time-stamp: <Chen Wang: 2019-12-23 15:28:32>


\section{西魏\tiny(535-557)}

\subsection{简介}


西魏(535年-557年)是中國魏晉南北朝時期中的北朝的一個地方政权,是由鮮卑人宇文泰擁立北魏孝文帝元宏的孫子元宝炬為帝,與高歡所掌控的東魏對立,建都長安。至557年被北周取代,總止經歷兩代三帝,享國二十二年。

在整個西魏統治時期,一直都由權臣宇文泰以霸府的形態控制著政權,在他努力下,北方經濟逐漸恢復,人民安居樂業,而且屢勝東魏。

大統元年(535年)春正月戊申,鮮卑人宇文泰擁立北魏孝文帝元宏的孫子元宝炬為帝,大赦並改元大統,追尊皇考元愉為文景皇帝,皇妣楊奧妃為皇后。春正月己酉,進丞相、略陽公宇文泰都督中外諸軍、錄尚書事、大行臺,改封安定郡公。以尚書令斛斯椿為太保。春正月乙卯,立妃乙氏為皇后,立皇子元欽為皇太子。春正月甲子,以廣陵王元欣為太傅,以儀同三司萬俟壽樂幹為司空。大統元年(535年)春二月,前南青州刺史大野拔斬兗州刺史樊子鵠,投降東魏。

宇文泰在三次戰役中大敗東魏,奠定宇文氏在關中的基礎。宇文泰任用漢人蘇綽等官員進行改革,使西魏進一步強盛。

大統十七年(551年),元宝炬駕崩,太子元欽即位,沿用西魏文帝年號。

552年,去年號,稱元年。

553年,宇文泰主動去丞相位。此时,梁元帝为与占据益州的萧纪争夺帝位,请西魏攻取益州。秋八月,大將軍尉遲迥攻克成都,平定劍南;萧纪随即为萧绎所灭。同年冬十一月,尚書元烈謀誅安定公宇文泰,反被宇文泰所處死。元欽對此事常有怨言,欲誅宇文泰,竟聯合宇文泰的女婿李基、李暉和于翼。結果被三人告密。

554年春正月,元欽被宇文泰所廢,立其弟齊王元廓。四月庚戌,宇文泰用毒酒毒死元欽。后西魏君臣又恢复鲜卑姓氏。

555年,西魏攻破南朝梁都城江陵,迫使梁元帝投降(但并没有灭亡梁朝)。

556年,宇文泰病死,由宇文泰的嫡長子宇文覺承襲為安定郡公、太師、大冢宰。

557年,宇文泰之侄宇文護迫西魏恭帝禪讓,由宇文覺即位天王,建立北周,建都長安(即今陝西西安)。

曾因接纳东魏叛将侯景而与东魏交战。但侯景与西魏皆未完全信任对方,侯景很快叛投南朝梁。

552年,柔然人被突厥土門可汗擊敗,其汗國崩潰。柔然王室由鄧叔子率領,西支柔然南逃至西魏,西魏太師宇文泰不敢收留,將此部三千餘人收捕交還突厥使者,全數斬殺於長安青門外。

%% -*- coding: utf-8 -*-
%% Time-stamp: <Chen Wang: 2021-11-01 15:13:24>

\subsection{文帝元宝炬\tiny(535-551)}

\subsubsection{生平}

魏文帝元宝炬(507年-551年3月28日),河南郡洛阳县(今河南省洛阳市东)人,魏孝文帝元宏之孙,京兆王元愉第三子,母杨奥妃,南北朝時代西魏建立者。

元宝炬在魏宣武帝時因為父親獲罪受牽連,年幼的一眾兄弟皆被幽禁在宗正寺,在宣武帝死後才重獲自由,被叔叔元懌收養。武泰年间封邵县侯,530年被封為南陽王。534年跟隨堂弟魏孝武帝元修入關中投靠宇文泰。

永熙三年闰十二月癸巳(535年2月3日)宇文泰毒死魏孝武帝后,秘密不发布丧事,宇文泰与群臣商议册立新皇帝,众人大多推举魏孝武帝哥哥的儿子广平王元赞为嗣君。濮阳王元顺在正室以外的房间流下眼泪对宇文泰说:“高欢逼迫驱逐先帝,册立幼年的君主来独揽大权,您应该反其道而行之。广平王虽然与先帝亲近,但是年幼,不适合居于皇帝之位,不如册立年长的君主来奉戴。”宇文泰深深地赞同元顺的意见,因此发布国家的丧事,向太宰、南阳王元宝炬奉上帝号,改元大统,政權實際上由宇文泰控制。大统十七年三月庚戌(551年3月28日),元宝炬去世,虚岁四十五,四月庚辰(551年4月27日)葬于永陵。嫡長子皇太子元钦嗣位。

\subsubsection{大统}

\begin{longtable}{|>{\centering\scriptsize}m{2em}|>{\centering\scriptsize}m{1.3em}|>{\centering}m{8.8em}|}
  % \caption{秦王政}\
  \toprule
  \SimHei \normalsize 年数 & \SimHei \scriptsize 公元 & \SimHei 大事件 \tabularnewline
  % \midrule
  \endfirsthead
  \toprule
  \SimHei \normalsize 年数 & \SimHei \scriptsize 公元 & \SimHei 大事件 \tabularnewline
  \midrule
  \endhead
  \midrule
  元年 & 535 & \tabularnewline\hline
  二年 & 536 & \tabularnewline\hline
  三年 & 537 & \tabularnewline\hline
  四年 & 538 & \tabularnewline\hline
  五年 & 539 & \tabularnewline\hline
  六年 & 540 & \tabularnewline\hline
  七年 & 541 & \tabularnewline\hline
  八年 & 512 & \tabularnewline\hline
  九年 & 513 & \tabularnewline\hline
  十年 & 544 & \tabularnewline\hline
  十一年 & 545 & \tabularnewline\hline
  十二年 & 546 & \tabularnewline\hline
  十三年 & 547 & \tabularnewline\hline
  十四年 & 548 & \tabularnewline\hline
  十五年 & 549 & \tabularnewline\hline
  十六年 & 550 & \tabularnewline\hline
  十七年 & 551 & \tabularnewline
  \bottomrule
\end{longtable}


%%% Local Variables:
%%% mode: latex
%%% TeX-engine: xetex
%%% TeX-master: "../../Main"
%%% End:

%% -*- coding: utf-8 -*-
%% Time-stamp: <Chen Wang: 2021-11-01 15:13:30>

\subsection{废帝元钦\tiny(551-554)}

\subsubsection{生平}

元钦(525年-554年),河南郡洛阳县(今河南省洛阳市东)人,西魏文帝元宝炬长子、母為乙弗皇后。

大統元年(535年),元寶炬登基建立西魏時,元钦被立為皇太子。不久丞相宇文泰主動許配女兒給元欽,成為元欽的岳父。大統十七年(551年),文帝駕崩,太子元欽即位,沿用文帝年号,次年(552年)去年号,称元年。次年(553年),宇文泰主動去丞相位。同年十一月,尚書元烈謀誅安定公宇文泰,反被宇文泰所處死。此後,元欽對此事常有怨言,欲誅宇文泰,竟聯合宇文泰的女婿李基、李暉和于翼。結果被三人告密。登位第三年(554年)二月被宇文泰废黜,安置雍州。宇文泰立其庶弟齊王元廓。元钦史稱「廢帝」。恭帝元年四月庚戌(554年),宇文泰用毒酒毒死了元钦。

\subsubsection{无年号}

\begin{longtable}{|>{\centering\scriptsize}m{2em}|>{\centering\scriptsize}m{1.3em}|>{\centering}m{8.8em}|}
  % \caption{秦王政}\
  \toprule
  \SimHei \normalsize 年数 & \SimHei \scriptsize 公元 & \SimHei 大事件 \tabularnewline
  % \midrule
  \endfirsthead
  \toprule
  \SimHei \normalsize 年数 & \SimHei \scriptsize 公元 & \SimHei 大事件 \tabularnewline
  \midrule
  \endhead
  \midrule
  元年 & 551 & \tabularnewline\hline
  二年 & 552 & \tabularnewline\hline
  三年 & 553 & \tabularnewline\hline
  四年 & 554 & \tabularnewline
  \bottomrule
\end{longtable}


%%% Local Variables:
%%% mode: latex
%%% TeX-engine: xetex
%%% TeX-master: "../../Main"
%%% End:

%% -*- coding: utf-8 -*-
%% Time-stamp: <Chen Wang: 2019-12-23 15:32:24>

\subsection{恭帝\tiny(554-557)}

\subsubsection{生平}

魏恭帝拓跋廓(537年-557年),原姓元,河南郡洛阳县(今河南省洛阳市东)人,西魏文帝元宝炬四子,南北朝时期西魏的皇帝。

554年即位,去年号称元年,并且復姓拓跋。557年被迫禅位于宇文觉,西魏灭亡。恭帝被封为宋公,住在宇文觉的堂兄大司马宇文护府上,不久被杀。

\subsubsection{无年号}


\begin{longtable}{|>{\centering\scriptsize}m{2em}|>{\centering\scriptsize}m{1.3em}|>{\centering}m{8.8em}|}
  % \caption{秦王政}\
  \toprule
  \SimHei \normalsize 年数 & \SimHei \scriptsize 公元 & \SimHei 大事件 \tabularnewline
  % \midrule
  \endfirsthead
  \toprule
  \SimHei \normalsize 年数 & \SimHei \scriptsize 公元 & \SimHei 大事件 \tabularnewline
  \midrule
  \endhead
  \midrule
  元年 & 554 & \tabularnewline\hline
  二年 & 555 & \tabularnewline\hline
  三年 & 556 & \tabularnewline\hline
  四年 & 557 & \tabularnewline
  \bottomrule
\end{longtable}


%%% Local Variables:
%%% mode: latex
%%% TeX-engine: xetex
%%% TeX-master: "../../Main"
%%% End:



%%% Local Variables:
%%% mode: latex
%%% TeX-engine: xetex
%%% TeX-master: "../../Main"
%%% End:
