%% -*- coding: utf-8 -*-
%% Time-stamp: <Chen Wang: 2019-12-19 17:14:34>

\chapter{南北朝\tiny(420-589)}

\section{简介}

南北朝(420年—589年)是中國歷史上的一段時期,由420年刘裕篡东晋建立宋開始,至589年隋滅陳為止,上承兩晉、五胡十六國、下接隋朝。因為南北长时间对立,所以稱南北朝。南朝(420年—589年)包含宋、齐、梁、陈等四朝;北朝(439年—581年)包含北魏、东魏、西魏、北齐和北周等五朝。

由于军权转移,南朝皇族主要出身于寒門或庶族。初期經濟逐渐恢复,但由於戰略錯誤與北朝军力強盛,使得疆界逐次南移。皇帝與宗室為了皇位時常血腥鬥爭。南梁在梁武帝在位期間國力改善,使國力再度強盛,晚年國家糜爛,侯景之乱使南朝实力大减,并四分五裂,獨霸政局的僑姓世族完全崩潰。雖由南陳的陈文帝統一南朝,但國力大跌,包括現在四川在內的西部大片原屬南梁領土被西魏佔領,南陳只能依長江抵禦北朝。北朝承繼五胡十六国,為胡漢融合的新興朝代。北魏皇室為鮮卑族,漢族官員受五胡文化影響,鮮卑皇室也受到漢文化的薰陶,彼此通婚。北魏被北方的柔然牽制,直到較友好的突厥并吞柔然後才全力對付南朝。後期在六镇之乱和农民暴动之后造成实力大衰。北魏分裂成東魏及西魏後,不久分別被北齊及北周取代。北周主要由六鎮集團組成,初期军力強盛。最後藉由宇文泰开创的北周關隴集團,吞并政治日趋腐败的北齊。此時統一中國的天平已朝向北周傾斜,周武帝去世後,漢人楊堅掌握朝廷,通过授禅北周静帝建立隋朝,经营八年之后,發兵灭南陳統一中國。

北朝戰爭不斷、各阶级對立严重,而南朝經濟持續成長、局势比較穩定,出現元嘉之治與永明之治等治世。中原人口自黃巾之亂和永嘉之乱后就开始南移,为南方帶來大量勞動力與先进的生產技術。江南的繁荣,使得中國的經濟重心南移。在文化方面,乱世为思想自由提供沃土肥壤,提出務實求治和無君論等觀點,在文學、藝術、科技等方面,開創出獨到的見解與理論。玄學、佛教與道教都很興盛。其中佛教帶動石窟的發展,敦煌莫高窟、麦积山石窟、雲岡石窟與龙门石窟名揚後世。对外交流也很兴旺,东到日本和朝鮮半島,西到西域、中亚、西亚(埃兰沙赫尔),南到东南亚與印度。

南北朝初期仍是世族政治,社會階層分為世族、齊民編戶、依附人及奴隸。世族擁有大量不需付稅的依附人從事生產與作戰,影響朝廷的稅收。雖然南朝皇帝仍然需要主流世族的擁護,不過也扶持寒門以平衡政治勢力,並且在南梁時出现了科舉制度的萌芽。南朝世族因為長期安逸而逐漸衰退,在侯景之亂後徹底崩潰。北朝胡人缺乏中原政治的經驗,所以重用漢人世族,引起雙方的文化採借,久之形成文化混合,以北魏孝文帝的漢化運動最盛。混合的過程產生激烈的思想衝突、政治鬥爭或種族衝突,例如六镇之乱、北齊的排漢運動。而北周建立關中本位政策,融合鮮卑及漢文化以消除胡漢隔閡。在隋朝統一天下後,開創出具開放性和包容性的隋唐帝國。

南朝時期從420年東晉權臣刘裕篡東晉開始,至589年隋滅陳為止。南朝經歷宋、齐、梁、陈四個朝代,這些國家皆建都於建康,只有南梁曾遷都。这四朝与之前同样建都于建康的孫吳和东晋合称「六朝」。

宋武帝劉裕原為東晉北府军的將領,在桓玄之亂後掌握朝廷。他為了獲得聲望來篡晉,發動了兩次北伐,收復了山東、河南及關中地區(關中後被夏國佔領)。之後劉裕殺晉安帝,改立晉恭帝,並在兩年後(420年)如同魏晉舊事篡位,建國劉宋,東晉亡。南方進入「南北朝時期」。440年,北魏統一北方後,方與劉宋形成南北對峙。宋武帝出身於軍旅,為人剛毅儉樸,稱帝後仍力行節儉,一時政風甚佳。但是他似乎不重視皇室教育,以至於所托非人,釀成巨變。他察覺當時世族權勢盛重,君主威權移墜,所以在朝政上重用寒族掌握機要,軍權重鎮則託付給宗室皇族。宗室掌握軍權及政區,因而心生篡位之意,所以皇帝與宗室之间發生多次骨肉相殘的惨剧。

宋武帝去世後,宋少帝繼立,因為嬉戲失德,被輔政大臣徐羨之、傅亮及谢晦所殺,改立宜都王義隆,是為宋文帝。他後來與北府名將檀道濟剷除把持國政的徐羨之等人,至此政局穩定。宋文帝提倡節儉並澄清吏治,開創了「元嘉之治」。430年起,宋文帝屢次北伐,由于軍力不足,再加上文帝的错误指挥,以致「兵荒財單」,國力大減。名将檀道济因军功被宋文帝猜忌而被铲除,又使劉宋失去能與北魏抗衡的大將。當北魏發生蓋吳起事時,劉宋沒能即時北伐。到445年時,北魏趁勁敵柔然暫衰時開始發動多次南征,雙方軍隊於淮水一線來回拉鋸,450年時,六十萬魏軍的主力一度逼近長江。劉宋在452年後無力再舉。

453年,宋文帝被太子邵所殺,三子劉駿趁機率軍奪位,即宋孝武帝。《宋书》說他為人驕淫奢侈,姦淫堂姊妹,發生兩起宗室戰事,最後還屠廣陵城,誅殺男丁三千口,女眷小孩全沒為奴婢。當時傳到北魏的民謠說:「遙望建康城,小江逆流縈,前見子殺父,後見弟殺兄」。其子前廢帝繼立後又大殺宗室,後為湘東王劉彧所殺,是為宋明帝。然而他因得位不正,所以更是大殺宗室,使宋孝武帝的子孫盡亡。其子宋後廢帝繼立後政局動盪,履有叛變,將軍蕭道成漸漸掌握軍權。477年後廢帝被蕭道成弒殺後,蕭道成擁立宋顺帝,獨攬朝政。在消滅政敵袁粲與沈攸之後,於479年篡位,建國南齊,史稱齊高帝,劉宋亡。

齊高帝屬於兰陵萧氏的次等世族,原本地位不高,所以稱帝後曾說自己原來是「布衣素族」(素族指非皇族的高門)。他的政風也如同宋初,為政節儉,在位四年即去世,由太子賾繼立,即齐武帝。齊武帝為政清明,與北魏無戰事,安民保境,史稱「永明之治」,但推行的检籍政策卻造成一次不小的叛亂(唐寓之暴亂)。當時皇帝利用典籤官作為耳目,來監察諸州政事及宗室諸王。

齊武帝去世後,由皇太孫萧昭业繼立,由萧子良與蕭鸞輔政。然而齊帝蕭昭業奢侈荒戲,不但國政由輔政大臣蕭鸞掌握,連父、祖留給他的親信護衛蕭諶、蕭坦之等人,都主動投靠蕭鸞。蕭鸞有意篡位,殺蕭昭業後改立其弟萧昭文,不久廢帝自立,是為齊明帝。齊明帝繼位後,利用典籤官大殺宗室諸王,高帝與武帝的子孫都被殺盡。蕭鸞晚年病重,相當尊重道教與厭勝之術,將所有的服裝都改為紅色;而且蕭鸞還特地下詔向官府徵求銀魚以為藥劑。498年蕭鸞病故,葬於興安陵。

齊明帝去世後由太子蕭寶卷繼立。他狠辣剛暴,殺害顧命大臣,激起各地方鎮叛亂。亂事平定後,他又殺平亂有功的尚书令蕭懿。501年蕭懿之弟雍州刺史蕭衍宣佈舉兵,在江陵立其弟寶融為帝,是為齐和帝。蕭衍在攻入建康後,齊帝寶卷被將軍王珍國所殺。在次年蕭衍篡位,建國南梁,史稱梁武帝,南齊亡。

梁武帝为兰陵萧氏的旁支,為人節儉,勤政愛民。使得梁朝前期開創出天監之治,國力勝過逐漸混亂的北魏。鑑於宋齊宗室的屠殺,梁武帝對其宗室十分寬容,即使犯罪也不追究。他學問淵博,提倡學術發展,使得南梁教育發達,南朝的文化發展至極致。然而在梁武帝後期,他喜聽人奉承,又迷信佛教,三次捨身同泰寺。由於僧侶道士不用賦稅,以致近一半的戶口記名其下,國家財政蒙受重大損失。當時的宗室及官員貪財奢侈,沉陷在紙醉金迷中而不可自拔。

梁武帝初期,北魏在漢化運動後矛盾叢生,國力漸漸輸給南梁。至503年始,北魏與南梁會戰於淮南地區,最後昌义之、曹景宗、韋叡在鍾離之戰大敗魏軍。梁武帝至此積極北伐,但範圍不出淮南地區。516年的壽陽之戰雖然擊潰魏軍,但因為損失過大而暫停北伐,十年後才奪下壽陽。此外梁武帝喜用降將,以期不勞而獲。北魏發生六镇之乱時,梁武帝派陈庆之護送北魏北海王元颢北返繼位。陳慶之雖能以七千騎兵攻至洛陽,但因孤軍無援,最後失敗。東西魏時期,東魏將侯景受东魏及西魏逼迫投奔南梁,梁武帝任用他北伐東魏。但在梁軍戰敗後,梁武帝意圖送還侯景以求和。他得知後舉兵叛變,南攻建康,史稱侯景之乱。梁临贺王蕭正德引他渡江,使侯景攻入建康,梁武帝退至台城,侯景立萧正德为帝。之後各地雖有勤王之師,但皆觀望。侯景聞知勤王師後一度和談,但最後叛約並攻陷台城。建康淪陷後他屠殺江南世族,為南朝政治帶來毀滅性打擊。侯景废杀萧正德,梁武帝最後餓死,侯景立皇太子萧纲为梁简文帝,又迫其禅位给故太子孙萧栋,又杀简文帝,最後篡位建國漢。

然而侯景勢力僅在江東、江汉平原,梁州和益州一帶依舊由梁室掌控,只是各軍互相牽制,坐视不理,有的求救于北齐和西魏。最後由廣州刺史部的番禺郡太守陈霸先與湘東王萧繹派遣的王僧辯聯合攻滅侯景。湘東王萧繹也於江陵繼位,為梁元帝,他拒绝还都建康,并派手下杀死萧栋兄弟。據守益州的武陵王紀已稱帝,这时发兵攻擊江陵。梁元帝向西魏求救,西魏攻滅武陵王紀後也佔領益州。次年,蕭詧引西魏軍趁機攻陷江陵,梁元帝被殺,西魏立他為傀儡,史稱西梁或后梁。梁元帝被殺後,陳霸先與王僧辯立晉安王蕭方智為梁王,准备拥立为帝。而後北齊迎蕭淵明南下,梁軍被擊敗,王僧辯屈事而迎立萧渊明為梁帝。陳霸先反對而率軍殺王僧辯,復立萧方智为梁敬帝。之後陸續擊潰北齊南侵及王僧辯餘黨,專政梁廷。最後於557年篡位,建國號陳,史稱南陳,為陳武帝,南梁亡。但西梁继续存在直至587年被隋朝取消,梁将王琳也据长江中上游地区拥立元帝孙萧庄抵抗南陳直至573年。

陳武帝是吳興人,為南方吴人。當時僑姓世族及吳姓世族皆因侯景之乱而嚴重受創,許多地方勢力亦紛紛割據。由於陳武帝無法盡數平定所以採取安撫的方式。武帝死後其姪陳蒨繼位,即陳文帝。盤據兩湖的王琳率先發難,聯合北齐、北周大軍東征建康。陳文帝先是擊潰王琳北齊聯軍,繼而封鎖巴丘阻止北周順江東進。至此國勢方定。陳文帝在位時期,勵精圖治,復甦江南經濟,使南陳國勢恢復。

文帝去世後由太子伯宗繼位,即陳廢帝。不久,其叔安成王頊廢帝自立,即陈宣帝。當時北周意圖滅北齊,於是邀南陳共伐北齊。陳宣帝為了想要收復淮南而同意,於573年派吳明徹北伐,兩年後收復淮南。當時北齊衰落,陳宣帝本能趁機攻滅,但他只想固守兩淮一帶。北周趁機攻滅北齊後,在577年南征奪兩淮,陳軍慘敗,南陳岌岌可危。然而北周武帝突然去世,權臣杨坚準備篡位,北周遂無意繼續南征。陈宣帝在楊堅建立隋朝後去世,太子叔寶繼位,即陈后主。他荒淫奢侈,致國政大亂,朝政極度腐敗。當時官吏剝削嚴重,人民苦不堪言。隋文帝採納高熲的策略,在南方收成季節火燒南方的田地,這使得南陳的國力衰退。588年隋文帝任楊廣為主將,發動隋滅陳之戰。陳叔寶恃長江天險,照常歌舞。隔年,隋軍攻入建康,陳叔寶與愛妃張麗華、孔貴人避入井中後被俘,南陳遂亡。

北朝時期自439年北魏滅北涼統一華北開始,至589年隋滅陳為止。經歷北魏、东魏西魏對峙、北齊北周對峙三個時期,並包括隋立國至滅陳時期。北魏、东魏、西魏及北周由鮮卑族建立,北齊則由胡化漢人所建。

北魏於十六國時期由拓跋鮮卑所建,前身為代國。前秦於淝水之戰崩潰後,代王拓跋什翼犍之孫拓跋珪舉兵復國,都盛樂,改國號為「魏」,史稱北魏。北魏在道武帝(拓跋珪)、明元帝及太武帝的經營下逐步壯大。拓跋珪與後燕交惡而發生多次戰爭,最後在張袞幫助下於參合陂之戰擊潰燕軍。拓跋珪不久率軍攻破後燕首都,遷都至平城,並在次年稱帝,即道武帝。道武帝性情殘忍,後為其子拓跋紹所殺。同年道武帝長子拓跋嗣平亂繼位,即明元帝。他攻下南朝宋的河南地,但不久去世。其子拓跋燾繼位,即太武帝。他勵精圖治,國力大盛,並屢次攻掠南朝宋。在解除北方柔然的威脅後展開統一華北戰爭。在439年攻滅北涼後結束「五胡十六國時期」,與南朝宋對峙。北方至此正式進入「南北朝時期」。然而,還有後仇池,至443年方亡於北魏。

北魏初期國家組織與經濟的建立皆仰賴崔宏與崔浩父子。雖然軍力鼎盛,但北有強敵柔然,以致不能全力南征。太武帝統一華北後又滅西域五大強國之一的鄯善,控制了西域。在450年又南征刘宋,直逼瓜步,並揚言渡江。之後掠奪五萬戶北返,至此北朝軍力壓倒南朝,但軍力大損。信仰佛教的盧水胡人蓋吳率各族百姓起事,太武帝平定此事後打擊佛教,成為三武灭佛之一。太武帝於後期刑罰殘酷,最後被宦官宗愛所殺,宗愛之亂至文成帝時方平定。

獻文帝執政時,被其母馮太后毒殺。馮太后改立獻文帝之子拓跋宏(即孝文帝),並把持朝政。馮太后猜忌多智且濫刑,但使國政平穩。孝文帝傾慕漢文化,認為鮮卑人應該要深入漢化。他為人英明好學,在親政後擴建首都平城為漢城。基於洛陽較平城繁華,地理位置控制全國易發兵於江南,可擺脫保守派勢力,於493年假借南征南朝齊名義,率眾南遷洛陽。孝文帝在遷都後的三年間推動漢化運動,全用漢官官制、禁胡服胡語、推廣教育、改姓氏(包括改拓拔氏为元氏)並同漢人世族通婚、禁止歸葬及度量衡採漢制。漢化運動為南遷的鮮卑人提升文化素質,為北魏的政治與經濟帶來發展,但使得暮氣重重的鮮卑貴族由尚武精神趨向奢侈及文弱。而後孝文帝在多次南征南朝齊後皆無功而返。至於留在北方六镇的鮮卑貴族由於不願南遷,逐漸不受洛陽朝廷重視而失勢,這使得北魏內部分裂成鮮卑化與漢化兩大集團,成為日後六镇之乱的原因之一。494年太子元恂意圖北返平城,孝文帝得知後廢太子並賜死。保守派穆泰、陆叡於平城擁王兵變,被鎮壓後孝文帝還親自北巡安撫。當孝文帝死后,北魏開始走入下坡。杨大眼是北魏孝文帝、宣武帝時名將,軍功顯赫,守衛南疆對抗南朝齊、梁。

499年孝文帝去世後由宣武帝繼任。他沉迷佛教,國政大亂,貴族競相奢侈。孝明帝繼任後,由胡太后執政。胡太后奢侈,私通清河王元懌並寵信元乂、劉騰。元劉二人因與清河王不合而叛變,並掌控朝政。劉騰去世後,到525年孝明帝與胡太后方平定亂黨。但胡太后依舊如故,並與孝明帝不合。而後北方發生六镇之乱,北魏走向滅亡之路。

早在北魏初年,為了避免柔然入侵北都平城,於阴山黄河一帶設置沃野、懷朔、武川、撫冥、柔玄及懷荒等六鎮來拱衛首都。六鎮將領由鮮卑貴族擔任,將士多是鮮卑族或漢族的高門子弟。他們被視為「國之肺腑」,可隨時返京任職。但在遷都洛陽後,六鎮地位下降。由於仍保有鮮卑原始習性,被漢化的貴族歧視為「代北寒人」,將領陞遷備受壓抑,心懷不滿。最後北方貴族與屯兵於523年發生六镇之乱,秦隴、關東等地各族人民也陸續起事。此事歷經三年方定,並形成許多軍閥。其中以鎮守晉陽的尔朱荣的勢力最大,他曾攻滅關東勢力最大的葛荣。

孝明帝意圖聯合爾朱榮對付胡太后,但被胡太后毒死。胡太后先後立孝明帝獨女及堂姪元钊為帝。同年尔朱荣以替孝明帝報仇為由,率軍攻佔洛陽,掌控朝政,史稱河阴之变。他在河陰將北魏幼主和胡太后沉入黃河溺斃,殺死大臣兩千餘人,改立孝莊帝,於晉陽遙控朝政。孝莊帝憤為傀儡,於530年在爾朱榮晉見時親自殺掉。而後爾朱榮之子尔朱兆及從弟爾朱世隆擁長廣王元晔為帝,攻下洛陽後殺孝莊帝,改立節閔帝。同年軍閥高歡於信都擁元朗為帝,並在532年攻下洛陽後,改立孝武帝,元晔、節閔帝、元朗皆被废杀。

孝武帝為高欢所制,有意聯合關中鎮將贺拔岳對付高歡。高歡先發制人,於534年殺賀拔岳。孝武帝則任宇文泰代之,並與高歡決裂,投奔宇文泰。高歡追之不及,改立清河王世子善見為帝,即東魏孝靜帝,遷都邺城。孝武帝西奔後不久被宇文泰所殺,改立南陽王寶炬為帝,即西魏文帝,定都长安。北魏於534年分裂成东魏及西魏後滅亡。

東魏及西魏表面上由拓跋氏後裔所繼承,實際上分別由高欢及宇文泰控制。所以在十餘年後分別篡奪,形成北周與北齐的對峙。基本上東西魏為沿山西陝西的邊河黄河為界。由於东魏繼承北魏的國力較多,所以不論在軍力、經濟或文化上均勝過西魏,但東魏在多次進攻後皆失利,雙方的對峙至此已定。

高歡所控制的東魏,是由鮮卑化的六鎮流民及河北世族所組成,高歡本身也是胡化漢人,使得在政治上較倚重鲜卑族。後來北齐皇帝也都有意保持鲜卑习俗,提倡说鲜卑语及武事。高歡用人惟才是用,朝中不少名臣都是其夥伴,這些皆為后来的北齐打下堅固基礎。然而他戰術不及宇文泰,三次戰役屢敗於焉。536年高歡率竇泰等人西征西魏,於潼关戰敗,竇泰自殺。隔年高歡趁關中大饑時率軍再度西征,於沙苑之役敗給軍力不多的宇文泰。至此分裂局勢大定,戰場也轉向河東地區。546年高歡再率十萬大軍西征,於玉壁和西魏守將韦孝宽發生玉壁之戰。最後高歡戰敗,死傷七萬餘人,隔年病死於晉陽。高歡死後,長子高澄繼之。他凶橫暴烈,姦淫大臣妻子,後被家奴刺死。高洋繼任後於550年廢殺東魏帝,並屠殺東魏皇室,東魏亡。他建國北齐,史稱北齊文宣帝。

宇文泰所控制的西魏,在八柱国等將領協助下,有效地抵抗東魏的多次進攻,鞏固西魏局勢。當時西魏在經濟、文化與軍事皆不如東魏與南朝梁,例如邙山之戰敗於東魏。他讓蘇綽等人推行「六條詔書」,建立關中本位政策使胡漢將領同心協力,設置府兵制以建立職業軍人,維持尚武精神。這些皆使西魏國力強盛,也影響隋唐的政治制度與集團分佈。宇文泰趁南朝梁於侯景之乱後諸王內鬥之際,先後攻下蜀地及江陵,並立西梁為傀儡國(詳見江陵之戰)。西魏帝後由廢帝、恭帝相繼繼立。556年宇文泰去世後,其侄宇文护專政。他於隔年廢西魏恭帝,建國北周,立宇文泰子宇文覺為北周孝閔帝,西魏亡。

北齊繼承東魏疆域,於550年由齐文宣帝建國。齊文宣帝先後擊敗庫莫奚、契丹、柔然、山胡(屬匈奴族)等族。並攻下南朝梁的淮南地區。在經濟方面,農業、鹽鐵業、瓷器業都相當發達。北齊大致上同北魏,持續推行均田制。這些使得北齊的國力在初期均勝過北周及南朝陳。然而齊文宣帝在位後期荒淫殘暴,並為了維護鮮卑貴族利益,屠殺漢人世族。北齊對人民的壓迫更重,使得北齊國勢衰落。齊廢帝繼立後,由其叔高演輔政。但高演不久即篡位殺帝,是為齐孝昭帝。齊孝昭帝在位期間,國力漸漸復元,還親征庫莫奚。但於兩年後去世,由其弟長廣王湛繼立,即齐武成帝。齊武成帝昏庸好色,北齊國力大衰,不久去世,由後主高緯繼立。高緯同其父同樣昏庸好色,還誅殺名將斛律光、高長恭,使國政更陷混亂。之後北齊被南朝陳攻下淮南,並在577年亡於北周。

北周繼承西魏疆域,於556年由周孝閔帝立國,但朝政由堂兄宇文護掌握。孝閔帝意圖聯合赵贵、独孤信剷除宇文護。然而被其發現,趙及獨孤二人被殺,周孝閔帝於隔年被先廢後殺。宇文護改立宇文毓為帝,即周明帝,但於560年又毒死周明帝改立宇文邕,即周武帝。周武帝宇文邕採韜晦之計,在十二年後成功殺死宇文護,親掌朝政。執政之後,周武帝宇文邕推動多方面的改革,使北周國力更盛。577年周武帝宇文邕發兵东征北齊,於隔年攻克鄴城,北齊亡。周武帝宇文邕在統一華北後,獲得李德林等關東世族的歸附,國力更強。周武帝立即南征南朝陳,但於同年逝世。而後北周發生內亂,使得南朝陳得以維持下去。

楊堅為北周開國元勳楊忠之子,他的女兒為太子妃。578年周武帝去世後,由太子宇文贇繼立,即周宣帝。他荒淫昏庸,迷信佛道二教,立五位皇后並奪人妻子。他殺宗室功臣宇文憲並大撤諸王就國。這些皆為楊堅的篡位鋪好路。楊堅開始集結周廷文武諸臣,形成一股龐大的集團。周宣帝去世後,其子宇文闡繼位,即周静帝,由外戚楊堅專政。尉遲迥、司馬消難等人不滿楊堅專權,起兵反楊。楊堅得李德林策劃,以韦孝宽等人平定。581年楊堅篡位為帝,即隋文帝,建國隋,改元开皇,北周亡。587年隋文帝廢西梁後主蕭琮,西梁亡。588年隋文帝發動隋滅陳之战。以楊廣為主將,同賀若弼和韓擒虎等名將發兵攻陳。隔年隋軍攻陷南朝陳都城建康,南朝陳亡,中國地區再度統一。自永嘉之乱以來,中國分裂二百八十年之久,至此「南北朝時期」結束,進入「隋朝時期」。不久即開創隋唐盛世,將中國歷史推向另一個高峰。

南朝政區承襲东晋,實行州郡縣三級制。而僑州郡縣及雙頭州郡也因為土斷而變成一般州郡。自汉灵帝中平五年(188年)实施的的州郡縣三級制,到隋平南陳後改為州縣二級制而結束。南朝的州設刺史,郡設太守,只有丹陽郡因為是首都所在地而設尹。縣設令、長。自劉宋以後,令多於長。與郡同級的有王國和公國,設內史和相。還有特為習稱蠻民及僚族、俚族等少數民族設置左郡、左縣和僚郡、俚郡。例如有南陳左郡、東宕渠僚郡等。當時州郡縣有等級之分,大致上以距離首都遠近為品級高低之分,諸州佐吏則按州的等級設置官員。揚、荊二州還有「二陝」之稱呼。

南朝疆域方面,刘宋繼承東晉疆域,基本上為二十二州上下。宋武帝时代是顶峰,但是不久河南地於永初三年(422年)之後逐漸被北魏併吞,后改以淮水為界。萧齊基本上同劉宋為為二十二州上下,可是相繼失去雍州沔北及淮南豫州之地。萧梁時州郡設置和疆域變化很大,因北伐獲得淮北之地,一度達河南地。又開拓閩、越、平俚洞,破牂柯。到539年共有大小不一的一百零七個州。侯景之乱後,北齊趁机佔領江北淮南之地,西魏趁机佔領漢中巴蜀。西魏又受蕭詧之托,率軍奪下萧梁江陵以北之地,建立附庸國西梁。南陳成立後疆域不多,至569年開始陸續收復淮南及部份淮北之地,並且一度奪下北齐長江以北之地(573年—577年)。到陈宣帝太建十一年(579年)時被北周宣帝占领淮南江北使得國土減少,僅剩長江以南至交廣地區。

北朝政區承襲十六国,如同南朝一樣為州郡縣三級制。然而州轄區不大,州刺史可越郡級直接管理縣,使得郡級逐漸虛級化,到583年隋朝正式定為州縣兩級制。北魏也設有僑州郡縣和雙頭州郡(如南雍州),並將州郡縣按人口數分等級。為了防範新附或異姓叛變,於406年將各級行政長官分立三位,其中州刺史方面須一位為宗室。北魏原設有負責地方軍政的行台及管理數州軍事的都督。到北齊定為行台制,北周則為總管制,都是負責數個州郡軍事與行政的政區單位。北齊因為州轄區越分越細,於是設置行台兼管數州民政及軍事。西魏則改稱都督為總管,性質同北齊行台。北周時,總管一般兼任駐州刺史,並以所駐之州為名。北魏還特為鮮卑本族或其他民族(漢人除外)設置領民酋長來管理該族,地位只次於州刺史。又延續十六國政區,設有管理州境內其他民族的護軍。其地位等同郡守,至457年廢除。還有鎮戍制,於重要的軍事要地設鎮。鎮由鎮將管理,下置戍,由戍主管理。其中又以鞏固首都平城的六鎮最重要,至孝文帝遷都後勢微。六鎮起義後,北朝的鎮戍專管軍事,不再具政區性質。

北朝疆域方面,北魏自代北之地(今山西省北部)崛起,至439年統一華北而結束五胡乱华時期。其屢次入侵劉宋,佔領山東、河南與淮北地。又取南齊淮南地及南梁漢中、劍閣一帶。至此疆域北至漠南草原,西抵西域東部,東達遼西,南達江漢流域。在擴充領土期間,州郡多因時制宜,到487年開始整頓。到北魏孝明帝之後領土減少,州郡濫置。魏分東西後東魏有八十州,西魏有三十三州。北齊建立後,開始整頓政區規劃,廢除三州、一百五十三郡及五百八十九縣。北齊江淮之地後被南陳佔領。西魏屢次攻佔南梁巴蜀之地與江漢之地。北周開國之初招徠南中(中國西南地區),置寧州(即南寧州)。北周武帝滅北齊,取南陳江淮之地,領土大大擴充。到隋文帝开皇八年(588年)11月時出征南陳,开皇九年(589年)3月底統一南北。

南北朝時期的世族雖然權力極盛,但南朝世族已随着时间的推移而漸漸走向僵化的趋势,寒門開始興起;而北朝世族受皇帝影響極大,使得權力並不穩定。

南朝皇帝主要是兴起于寒門或庶族,而世族的特權沒有馬上被動搖。南朝規定,世族的子弟二十歲登朝做官,寒門子弟三十歲才能試做小吏。這使得世族升遷極快,短時期內就可「坐致公卿」。寒門為了加入世族,除了改注籍狀並學習中原官话,假裝是名門遠親以外,就只能投靠名門高官(包括王族)作門生,運氣好就可以被提拔作官,未來當到高官後就會自動提升門第,成為中低層的世族;或者更有效的辦法,是投筆從戎,在戰場殺敵立功,最後甚至有可能當到最高的三公,晉升為頂層的高門。南朝有不少三公即是出於軍功,如王玄谟、張敬兒、王敬则、陳顯達等,他們的子弟不但可與王、謝世族中地位較低的支系聯姻,也可以靠著新興的高門身分,去輕視、貶低門第較低者。

世族為了維護社會地位,並且盛行祖譜。例如賈弼之祖孫三代專精祖譜學,撰《十八州士族譜》,共七百多卷。劉宋劉湛、南齊王儉、南梁王僧孺也都有祖譜學專著。祖譜學是吏部選官的重要依據,是維護世族政治的工具。世族同寒門保持著嚴格的界限,不同寒門通婚共坐,然而在南朝中後期也逐漸崩解。南朝世族既不會帶兵打仗,又不能有效管理政事,完全成為寄生於社會的廢物,在南梁侯景之亂後全面崩盤,一蹶不起。寒門主要指無特權的地主和商人,他們不甘心受到排擠,通過考試等各種途徑登上政治舞台,在梁武帝時期萌芽出科舉制度。南朝的開國皇帝,就是通過領兵打仗、控制軍權而上升起來的寒門。

由於北朝胡族君主需要熟悉中國典章制度的人才,而且也為了笼络一些有勢力的北方世族,於是與北方漢人世族合作治理國家。而北方世族為了延续下去也願意配合,但是雙方也會因為風俗習慣或政治觀點不同而引發殺機。北魏孝文帝推動漢化運動後,將鮮卑貴族融入北方世族之中,明定漢和鲜卑世族的等級和地位分為膏粱、華腴、甲、乙、丙、丁六等,漢人世族的郡姓與漢化鮮卑世族的虏姓。並且進一步實行漢和鲜卑世族聯姻,使得雙方的隔閡逐漸減少。六鎮民變後,洛陽鮮卑世族受損嚴重。其後北魏分裂成東西,並分別由北齊與北周繼承。北齊君主並不是十分重用漢人世族,偏重於提倡鮮卑文化與武功;而北周採取關中本位政策,融合胡漢文化,重用蘇綽、卢辩等世族。戰勝北齊、南陳後,政治要津皆為關中胡漢世族所壟斷,關東世族與江南世族都難以抗衡,影響日後隋唐的政治環境。

宋與北魏都出现了后来隋代的三省制度的雏形。北周時按周禮,設置六官,即天官、地官、春官、夏官、秋官、冬官六府,是隋唐六部制度的源頭。但是在整個南北朝時期,三省六部制仍與三公九卿制共存,直至隋朝建立後才廢止三公九卿制。門下省負責獻計策和勸諫皇帝,參與機密之事,又成為大權掌握的機構。南朝皇帝為了避免被世族控制,所以讓寒門擔任皇帝身邊的機要職務。例如通事舍人不僅替皇帝起草詔令,又掌管政令,成為天子身邊的實權職務。另一個要職是典簽。南朝君主鑒於東晉方鎮勢強,威脅中央,因此多以宗室子弟為州鎮軍政長官,這就是「擬周之分陝」。典簽則負責控制州鎮要事。典簽每年數次回京向皇帝報告刺史好壞。齊武帝時典簽權勢達到極盛,諸王刺史都非常害怕典簽,所以當時有「諸州惟聞有簽帥,不聞有刺史」的說法;之後篡位的齊明帝也利用典簽之權,將宗王刺史大肆誅殺,但等到權力較為穩固之後,就抑制典簽之權,典簽因此失權,重新變回刺史的實質部下(但仍兼朝廷耳目)。

南北兩朝彼此關係始終處於敵對狀態,南朝稱北朝為「索虜」,北朝稱南朝為「島夷」。雙方時常發生戰爭,然而於戰间期仍有開市以互相貿易。

當時常稱南方的土著民族為「南蠻」。長江流域以板楯蛮、盤瓠蠻與廩君蠻实力最大,嶺南以俚族為主。這些民族與漢人雜居,從事農業,受漢文化影響,用漢姓,在南朝後期逐漸融入漢族。板循蠻又稱賨人,原居益州巴郡閬中一帶,之後經渝水北遷漢中、關中。廩君蠻原在益州巴郡、荊州江陵一帶,後來擴展到長江漢水與淮西一帶。史書上提到的巴東蠻、宜都建平蠻都是指廩君蠻。盤瓠蠻又稱「溪人」,發揚地在辰州,分佈現在的湖南與江西一帶。

俚族的範圍在南嶺、今貴州南部到海南島、越南北部一帶。有名的有萧梁的冼珍,她於侯景之乱、廣州歐陽紇之亂時安定交廣,保護當地俚漢人民,被尊為「聖母」。交州以南則是林邑國,林邑王范陽邁屢次近犯南朝的日南、九德等郡,林邑國到了南朝後期成功佔領日南郡。南朝的交州曾數度發生割據抗命的事件。468年起李長仁與李叔獻兄弟據交州抵制刘宋朝廷,齊高帝採用劉善明的建議,安撫李叔獻為交州刺史。齊武帝時,叔獻阻截外國貢獻,武帝乘機派兵攻佔交州。南梁時,505年交州刺史李凱據州叛變,梁廷派李畟討平,並斬當地反抗者阮宗孝。541年李賁起事,攻陷州城龍編,於544年建國萬春。隔年梁廷遣楊瞟、陳霸先等率兵擊敗李賁軍隊。546年李賁退保屈獠洞時被殺,其部下趙光復仍據龍編,兄長李天寶據屈獠洞。李佛子繼承後於571年攻滅趙光復。隋朝時隋廷派劉方南征,李佛子遂向隋朝請降。

中國西南的南中地區,自東漢晚期至東晉時的長期战争,中原朝廷無力兼顧,遂形成豪族爨氏割據一隅。宋齊梁時期,南中地區雖仍為屬地,並置寧州刺史之職,然而大多遙領,未能至任,爨氏方為當地實際控制者。梁朝寧州刺史徐文盛曾在當地有所作為,但因爆發侯景之亂而離去。西魏乘機於553年攻取巴蜀地區,置益州刺史,其後宇文氏取代西魏,建立北周,即令益州刺史尉遲迥兼理寧州軍事,招徠南中地區,任爨瓚為寧州刺史,南中遂屬北周。

柔然是曾隸屬於拓跋鮮卑的別部,與拓拔部(北魏前身)關係較近,是漠南漠北第一強國。自北魏道武帝時期常攻打北魏,並且經略河西走廊,領有突厥、高車等從屬國。東西魏時期,552年突厥首領土門可汗建立突厥汗国,兩年後攻滅柔然。突厥併吞高車餘眾,與薩珊王朝合併在558年滅白匈奴,疆域擴張到東至布列亞河,北至贝加尔湖和外兴安岭,南抵中原與阿姆河,西達鹹海。突厥汗國以今阿爾泰山為界,形成土門系的東突厥和室點密系的西突厥。時東北民族可分為兩大語族,扶餘國(豆莫婁)、庫莫奚、契丹與室韋的扶餘語族與分成七個部族的勿吉(靺鞨)語族。扶餘國以農業和畜牧業為主,盛產名馬、赤玉、大珠,貂皮。社會盛行巫術,也會在戰爭時祭天占卜以預知吉凶。勿吉人則是「相與偶耕,土多栗、麥、穄,菜則有葵」,以打獵為業。扶餘國於東漢初期時強盛,臣服勿吉等國,其國人與後來的高句丽、百济有關聯。到北魏時期,北魏孝文帝延興六年(475年)勿吉逐漸興盛起來,不久勿吉、扶餘國、高句麗等東北諸國遣使向北魏朝貢。北魏為了安定這個地區,開市於和龍、密雲之間,與東北各國熱烈的貿易與使臣交往。478年勿吉向北魏請求和百濟南北夾攻高句麗,北魏就勸阻這場戰爭發生。北魏衰退後,493年勿吉滅亡扶餘國,領土擴展到整個松遼平原,成為當時東北一支強大部落。

在朝鲜半岛有高句麗、百濟與新羅等三國以及日本的倭國。高句丽為強國,威服百濟與新羅,併吞扶餘遺民,受南北各朝冊封國王,550年后因為內部鬥爭而逐漸衰退。百济受到高句麗與新羅的排擠,所以努力維繫蕭梁、倭國等關係,在貿易與朝貢都有蓬勃的發展。新罗受到高句麗的威脅與百濟結盟,於六世紀開始強盛並併吞倭國最後勢力任那。倭国在晉末南朝時期由倭五王多次遣使到建康,要求獲得南朝皇帝的冊封。其中倭王武更向宋顺帝提到要進軍朝鮮半島。倭國藉著南朝皇帝的封冊,鞏固在任那的勢力。而南朝則是逐漸增加對倭王的冊封,承認倭國在朝鮮半島南部的統治。

仇池國亡於北魏後,分別建立武興國與陰平國。472年楊文度建國武興,與南、北兩朝周旋和交戰。505年楊紹先稱帝,次年敗於北魏而被俘。534年楊紹先乘北魏分裂時復興武興,於553年亡於西魏。。477年楊廣香建國陰平,受北朝支配,通好於南齊、梁。580年沙州楊永安響應益州總管王謙,起兵反抗楊堅,最終戰敗而滅國。

西域在南北朝時期於經濟文化上成長進步,當時絲路已經有了三條:自高昌轉西、與中道合龜茲爲北新道;自鄯善向于閬爲南道。有名的國家或城市有高車、高昌、鄯善、龜茲與于闐。高車本來屬於鐵勒副伏羅部,臣屬柔然。487年領主阿伏至羅率部西遷至今吐魯番西北建高車國。高車國曾威盛西域,並且短暫控制高昌。之後陸續被嚈噠與柔然夾攻,到541年被柔然攻滅,餘眾於546年併入突厥汗國。北魏太武帝時期在西域有兩個據點,一個是高昌郡,另一個是派萬度歸攻下,比同郡縣的鄯善國。在北魏文成帝時期柔然扶闞伯周為王,高昌國正式建立。經四周諸國爭奪,到499年由麴嘉建立麴氏高昌。龜茲盛產煤鐵,擅長鑄治。另外尚產銅、鉛、良馬、胡粉與安息香等等產物,是西域著名的經濟中心。于闐國王與百姓都信仰佛法,寺院的僧侶很多。中原很多名僧都到于闐取得經書,百姓皆提供僧房給遠方的僧侶,這些都使于闐成為了佛教文化中心。

在南北朝時期,朝代常因為軍權流入權臣手中而更替。南朝的軍事制度大至延續兩晉兵制,然而世兵制衰落,所以以募兵制為主。北朝在軍事制度方面,在北魏初期採行兵民合一的部族兵制,統一華北後逐漸成為世兵制。北朝後期出現府兵制,成為隋唐兵制的基礎。但西魏、北周的府兵制并不是如同隋唐府兵制:平时为民,战时为兵;兵不识将,将不知兵,而是鲜卑兵制,是部酋分属制,是兵农分离制,是特殊贵族制。

南朝兵種以步兵和水軍為主,騎兵較少。兵源原本來自世兵制。但是因為戰爭的消耗、士兵的逃亡和被私家分割,部分兵戶變為民戶,兵源趨於枯竭,於是改以募兵制為主。南朝軍隊區分為中軍(亦稱台軍)及外軍。中軍直屬中央,平時駐守京城,有事出征。在劉宋時,宋武帝劉裕加強皇宮兵力,以圖扭轉東晉以來外強內弱的局面。然而,由於宗室自相殘殺而失敗,歷朝屢次有篡位之事發生。外軍則歸各地都督管制。都督多兼刺史,而且常與中央抗衡。

北朝方面,北魏軍隊在初期以鮮卑騎兵為主力,其補給是由各部自行掠取。在統一華北過程中,漢族逐漸加入軍隊。當攻城戰增加後變為步、騎兵混合。之後步兵成為主力兵種。北魏統一華北後,軍隊分為中軍、鎮戍兵和州郡兵。中軍在平時守衛京城,有事則成為對外作戰的主力。鎮戍兵是為保衛邊防而設置的。鎮相當於州、戍相當於郡。初時僅設於北部邊境,後來擴展到南部邊境。州郡兵,是維持諸州治安的軍隊,有時也充作鎮戍兵或是隨軍出征。北魏後期也逐漸形成固定的兵戶。

东魏和北齐的軍隊主要由六鎮鎮民和洛陽的鮮卑兵所組成,在北齊時又編成「百保鮮卑」。另外,也選漢族勇士來防備邊界。西魏和北周受到鲜卑傳統和汉文化的影響,於550年創立府兵制。該制度將遷至關中的六鎮軍民編成六軍,並設立八位柱國大將軍。西魏權臣宇文泰為最高統帥,西魏廣陵王元欣無實權,其他六個柱國則分領府兵,各督署2大將軍。北周時又擴增柱國人數,並將兵權集中在皇帝手中。但此时府兵制不同于隋唐府兵制,他们仍然是职业兵,宇文泰将府兵将领(及其士卒)改从鲜卑姓,并使之与土地结合。组成一支隶籍关中、职业为军人、民族为胡人、组织为部落式的强大的军队。且府兵雖為主力,但仍有守衛京師的中軍、地方的鎮戍兵及州郡兵等其他軍隊。而且世家豪族勢力強大,大都擁有實力不弱的私兵。

南北朝社會的人口很複雜,大致上可分為四個階層:名門豪族的世族;自耕農、新民等從事農工商的編戶齊民;屬於部曲、佃客、衣食客、門生舊故等依附世族的依附人,受政府控管的雜戶、百工戶、兵戶與營戶也是依附人;最後是奴婢、生口、隸戶及被俘擄遷移的城民,這些都屬於奴隸。

魏晉南北朝是世族政治時期,雖然北方豪族的地位與權力遜於南朝,但也居於極高的地位。世族控制的人口有部曲、佃客與奴隸,不經「自贖」或「放遣」,是不能獲得自由的。部曲主要用於作戰,由於戰事減少所以也參於生產活動。由於南朝大家族制的衰亡使得部曲逐漸受國家控制。佃客的來源有政府依官品賜給與私自招誘。奴隸的主要來源是破產的農民或是流民,他們是地主的私產,因而可以抵押或買賣。為了防止逃亡,奴隸都被「黥面」。奴隸可以經由「糜喃為客」、「發奴為兵」等方式轉化為地主的佃客和國家的士兵。自耕農是當時農業生產的重要力量。他們對朝廷負擔租調、雜稅、徭役以及兵役,這些都使許多自耕農破產流亡,淪為世族的部曲和佃客。南北朝實行三国以來的世兵制,兵戶世代當兵,平時還需要交納租調。由於手工業者很缺,故官府對雜戶或百工戶的控制極嚴,百工戶從民間徵調到官府作坊後,與配到作坊里的刑徒為伍,終年勞作,世代相襲。如果貴族、官僚私佔百工戶往往受到懲治。在北朝還有新民和城民。新民是北魏道武帝為了充實國力,大規模遷徒各族人民或工匠至首都地區的人民,計口授田。城民是被征服、被遷移的人民,被配置在各州內,身分如同奴隸。城民民族複雜,分佈廣大。

南方約在晉末宋初由大家庭制轉化為小家庭,在同一家族不同職業的十家就有七八家之多,互相漠視。這是因為宗族發展後各家庭親疏貧富不同,若無共同外患就容易分離;朝廷課稅方式對大家族制無益而導致的。而北方面對異族,需要團結合作,仍然保持大家族制。通過參與胡人政權的機會,逐漸將中國傳統文化及典章制度灌輸給異族。但是也有留下一些不良風俗,例如財婚的盛行。

南北朝時,遊牧民族與農業民族由互相衝突演化成文化的整合或汉化,形成胡漢融合文化。北方草原遊牧民族在進入中原的過程中不斷漢化,而中原世族為了逃避戰亂則紛紛舉家南遷,促進漢族與南方民族的接觸與融合。所以隋唐時期的漢族已非秦漢時期的漢族,而是黃河、長江兩大流域以原漢族為主體的各民族融合而成的新漢族。永嘉之乱使得大量北方漢人南下江南,东晋初期先設立僑州郡縣的方式安撫這些北方流民(僑居白籍),並且給予低稅優惠。但是僑州郡縣遷徙不定、僑民與當地人民混雜,影響了政府賦稅收入。於是在东晋中後期實施「土斷」,讓北方僑民就地入籍(在地黃籍),與當地人民共同負擔國家賦役。南朝時期實行5次土斷,以413年劉宋劉裕進行的義熙土斷成效最為顯著。這導致南朝境內的僑居州縣陸續消失。永嘉之乱使得北方漢族南遷,但仍有部分滯留在北方與游牧民族相處。由於胡族缺乏統治中國的經驗,所以重用漢人世族治理國家,這引起雙方的文化採借,久之形成文化融合。例如五胡十六国時期各族君主與漢人世族的合作,最後使部分中原胡人轉化成漢人;北魏孝文帝時期推動漢化運動,融合鮮卑皇族與漢人世族;西魏宇文泰採用蘇綽建議,建立關中本位政策,這些都融合了胡漢民族。然而融合的過程難免會產生思想衝突、政治鬥爭或種族衝突。例如北魏太武帝因修國史事件滅重臣崔浩一族,牽連范陽盧氏、太原郭氏與河東柳氏,皆滅族。孝文帝的漢化運動使得洛陽鮮卑貴族與六鎮鮮卑貴族產生矛盾,並發生鮮卑化運動以反抗漢文化,最後引發六镇之乱。主要以六鎮鮮卑人與胡化漢人為主的東魏北齊,保持尚武精神,提倡鮮卑文化、西胡化,極力排斥漢文化。最後由具備胡漢融合文化的北周與替代之的汉化政权隋朝攻滅政治與經濟混亂的北齊與南陳,建立具開創性、「天下一家」性質的隋唐帝國。

南北朝經濟主要是莊園經濟。世族與寺院的莊園大部分都是多方經營,具有自給自足的性質。農田有良好的水利系統供種植稻、麥、粟、桑、麻、蔬菜等作物,還可以種植竹木果樹、養魚、畜牧等等。還有紡織、釀造、生產工具等手工業。世族的莊園生產主要交給佃客、部曲和奴隸,而寺院是一般僧侣與民戶。由地主集中開墾,這對於地區的開發起一定的作用。由於世族享有特權,佛教較為盛行,致使地主莊園與寺院莊園膨脹,並且大量隱匿農戶。加上戰爭頻繁又使得社會精壯勞動力損失極大,導致國家與地主、寺院間互相爭奪土地和勞動力而爆發流血衝突。例如三武灭佛中的北魏太武帝灭佛與北周武帝灭佛。最後,由於各民族經濟交流加強,並逐漸融合為一體,社會經濟的發展從這種民族融合中汲入了許多新的發展能量。而江南地區已進入全面開發階段,使中國經濟重心南移,最後促成隋唐大運河的建立。

農業是莊園經濟的重心,深受朝廷與世族關切。北方农业规模较南方为小,且大多生产小麦、小米、高粱及黄豆。由于气候的关系,北方深受饥荒之苦。土地兼併的情形直到南朝仍然十分嚴重,朝廷難以禁止世族兼併土地,宋孝武帝大明元年(457年)朝廷乾脆承認佔領山林川澤的法令以限制世族搶佔範圍。然而法令頒布後反而刺激豪門權貴兼并山澤土地的活動。最後整個南朝在佔田奪土、兼山並澤的事例一直是史不絕書的。南朝相對北朝來說比較安定,南渡的移民仍然絡繹不絕,農業生産繼續有所發展。比較突出的地區有荊揚二州,而益州居次。揚州是南朝最發達的地區,其中以建康及其周圍地區發展最大。而三吳地區(吴郡、吳興、會稽)是南朝各種支出的主要來源。洞庭湖周圍的荊、湘地區發展也很快。此外淮南地區原本也是糧食重鎮之一,但是於451年的魏宋战爭中遭到破壞。經過齊、梁二朝的經營才獲得恢復。南朝各代興修不少水利事業,例如有宋、齊、梁各朝於壽陽(今安徽壽縣南)修治的芍陂;南齊在齊郡(治所今山東臨淄)開墾二百頃廢田,用沈湖水灌溉。基本上,江東帶海傍湖和延江之地,乃至江南腹地即南川、湘川地區等都已經開發了。

北朝前期因為五胡十六國時期的戰爭使得農業發展滯後於南朝,太和九年(485年)北魏孝文帝採納李安世的建議推動均田制,將戰爭時遺下的大量荒地按制度分給農民。這個制度日後推行於西魏、东魏、北齐、北周和隋唐,分配農地主要有露田(種穀物之田)和桑田。露田主要為國有地,給於男女奴婢均可,不可買賣,死後須收回國有。桑田為私有地,給予男子,可以買賣。最後地方官吏可按品級授給公田。北魏實行的均田制與三长制、租庸調制互相配合,促进了農民生產。北魏孝文帝推動漢化運動改善了吏治,使得農業能夠發展成長。直到神龜三年(520年)政治虽然逐漸腐敗,但官府糧倉還是相當充實。北齊的均田制與租調制比北魏限制多,並且增加奴婢的租調。然而東魏、北齊一朝貪污風氣極盛,到齊後主高緯時,荒淫腐朽,大量修建宮殿與寺院,窮極奢華。西魏北周方面,宇文泰採用蘇綽的建議,建立租賦預算和戶籍制度,以保證朝廷收入。由於北周世族的力量不大,加以吏治比較清明,比較容易推行均田制。由於北朝的鮮卑人為遊牧民族,其畜牧業原本就很發達,主要產地在漠南地區〔今河套地区〕。在北魏孝文帝時期又在河南設牧揚,養戰馬十萬匹。有名的有尔朱荣的父親尔朱新兴,他擁有「牛羊駝馬,色別為群,谷量而已」。

由於朝廷大力提倡農桑,戶調征絹布,當時絹布的地位等同貨幣,這些都促進紡織業的生產。南朝的紡織業與養蠶業比較發達,產地以荊、揚二州為主。由於絲、綿、絹、布等是南朝調稅的主要項目,因此紡織是民間普遍的副業。織錦業則在益州為主,劉裕滅后秦,把關中的織錦戶遷到江南,到萧齐和萧梁时期繁榮。當時富豪人家穿綉裙,著錦履,以彩帛作雜花,綾作服飾,錦作屏障。南朝朝廷設有專官管理礦冶,用水排鼓風冶鑄。鍊鋼則使用一種雜煉生鐵和熟鐵的灌鋼法。這種方法可以鍊出優質鋼,用來製造寶劍和刀。瓷器的燒制技術早在三國、晉朝時期成熟。南朝時以青瓷為主,產地集中在會稽郡(浙江紹興)。其硬度高,釉料勻,通體青瑩。江南其餘地區的制瓷技術各有自己的特點。南朝的紙張潔白勻稱,完全取代了簡牘,藤紙與麻紙都很流行。造船業也十分興盛,最大可以載重二萬斛。

北朝的紡織業主要是絲織業和毛紡織業,是手工業中最發達的一個產業。絲織中心有涇州(今陝西涇川)、雍州(今陝西西安市)、定州(今河北定縣)等地。絹布產量增加使得絹價下降。由初期每匹絹千錢,到北魏孝文帝後降到二三百錢。官方的絲織業規模巨大,有宮內或京城內的官方工場、作坊,為朝廷生產綾羅錦繡等絲織品。民間有專業的綾羅戶、細繭羅縠戶,分散在今河北、山東一帶,其產量大且質量精美。有山東的大文綾、連珠孔雀羅,阿縣的縞。毛紡織業的產品主要是氈,用途很廣,利潤頗豐。可用來作襦(短衣)、袴(套褲)、靴墊及帳蓬等。北朝朝廷也設有專官管理礦冶,以冶鐵業最為發達。鐵產量很高,450年劉宋軍隊攻克北魏碻磝戍(今山東聊城東)時就繳獲大量鐵器。資量方面,相州的牽口冶造的刀,為全中國地區之最,皆送入京師武庫。北齊綦毋怀文所造的宿鐵刀,既有非常強的硬度,又有韌性,斬甲過三十札。

南朝農業和手工業發達,加上江河交通便利,使得商業發達。由於政治松弛,幣制废弛,質量不精。市場上有普通的生產用品、生活用品與奢侈品。商賈小者坐販於列肆,大者轉運於四方。商稅是朝廷收入的大宗,然而世族有免關稅權,在任期屆滿時帶著大批貨物作為「還資」,然後轉販各地。商業重鎮有建康、江陵、成都、廣州、广陵等地。建康是三吴的經濟中心。南朝梁武帝時建康城內有居民28萬,貢使商旅,方舟萬計。會稽、吴郡、餘杭居次。廣州是海上貿易重鎮,貿易對象有东南亚各國、天竺、獅子國、波斯等國。江陵是關中、豫州、益州、荆州、交州、梁州的轉運站。成都不僅商業繁盛,也是蜀錦的重要產地。

北朝商業在北魏漢化運動後也逐漸成長起来。貨幣方面,原本以絹布穀物的物物交换作為交換媒介,孝文帝改制後鑄五銖錢改善,但仍然難以流通。商業重鎮有洛陽、鄴和長安。洛陽是北方的貿易中心,西陽門外有大市,周圍八里,十分繁榮。當時除了商販和大商人外,許多貴族、官僚也從事經商。貿易對象有中亚、西亚諸國、高句麗、百濟、伽耶、新羅與日本商人。西域商人经营的主要是金银珠玉、珍物器玩,以换回中国的丝织品、工艺品等。儘管南北朝彼此戰爭不斷,停戰期的貿易仍然活躍。雙方常以穀物、布帛代替貨幣在市場上流通。貿易方式主要有官方互市與個人走私,其中走私在民間、官員、軍隊中都有。關於商品的需求,北朝需要甘蔗、荔枝、芒果、香蕉、菠蘿、杨梅、橘柚等水果以及北方官員、貴族享用的奢侈品;南朝則需要北朝的馬匹、骆驼與毛織品等。總體來說,北朝商業仍然不及南朝活躍。

儒學獨尊的地位在晉朝被破除後,到了南北朝時期已經形成多元化的思想。在諸多的思想流派中,出現了以法治國、務實求治的主張(三國曹操與诸葛亮、東晉王導)和《無君論》(東晉鲍敬言)、《神滅論》(南梁范縝)、提倡「人死則神滅」(北魏邢邵與北齊樊遜)等有價值的觀點,也產生了消極頹廢、遁世遊仙的思想。影響最大的是玄學思想。玄學在南朝十分興盛,宋文帝時設至玄學館,玄學與史、文、儒並列四學,清談益盛。到梁武帝時鼓勵提倡經學,但此時經學已經受清談影響,只注重於辯論之說。隋朝統一後清談漸漸衰退,直到唐朝中期才終止。由于佛教過度膨胀與糜爛,出现了不少反對過度崇佛的思想家如南梁范縝、北魏邢邵與北齊樊遜,這些人的思想衍生出无神论。

范缜是齊梁之間人士,489年在竟陵王蕭子良的宴席上發表了反對佛教因果報應論,認為灵魂並不存在。他的觀點主要是唯物論的「變化的形一元論」,認為身體與精神都是物質,整個宇宙就是一個物質的動態變化。神滅論處理了「體」、「用」、「變化」、「關係」諸主題,完整概括了「唯物本體論」所應處理之主要範疇。范縝著有《神滅論》與答覆反對派曹思文的《答曹舍人》。他的論點引起朝野反彈,曹思文作《難神滅論》,蕭琛引「杜伯關弓」、「伯有被介」故事駁之。篤信佛教的梁武帝也展開論戰,作《敕答臣下神滅論》,命令他放棄觀點。梁武帝組織僧俗六十多人發表文章對范縝進行圍攻,引發思想界一場關於「神」(靈魂)滅不滅之理論大戰。然而變化的形一元論的涵義與變化廣大,幾可做無窮盡的推論引申,范縝辯才無礙,眾人難以反駁他,最後判范縝為「異端」而流放他。

北朝的邢邵與樊遜等思想家主張无神论。邢邵是北魏後期東魏初期人士,當時佛教受到鮮卑皇室提倡而興盛。他反對「神不滅論」,主張人死則靈魂就會消失,否定人死為鬼的理論,認為「欲使土化為人,木生眼鼻,造化神明,不應如此。」他還主張類化論,只有同纇事物,才可以轉化;不同類事物不能轉化。類化說顯現出物種產生的多元性與差異性。邢邵思想以及他和杜弼爭論轮回等問題等都紀錄在《北史·卷五十五·杜弼傳》與《北齐书·杜弼傳》中。樊遜北齊人,554年北齐文宣帝欲封禅泰山,樊遜向齊帝勸諫道教、神仙皆為虚妄。

顏之推南梁人,後被迫仕北朝。他主張早教,認為「人在小的時候,精神專一;長大以後,思想分散,不易學習。」他所撰有的《顏氏家訓》對往後中國社會有深遠影響力,後世視之為家訓的典範作品,獲得儒家學者及佛教徒的重視。

南北朝时期的文學发展迅速,其中南朝風格偏向華麗纖巧,而北朝風格偏向豪放粗曠。北朝代表人物是北地三才,即邢邵、魏收、溫子升。南朝方面,文學代表是駢文,講究格律、詞藻、用典。內容多脫離實際生活,抒發一些富貴閒愁。以庾信文章為代表。詩風流行元嘉體與永明體。元嘉體是代表劉宋元嘉年間的詩風,代表人物有「元嘉三大家」謝靈運、顏延之與鮑照。他們的共同功績是把古體詩推進到完全成熟階段,並且注意聲律和對偶的運用,並且逐漸發展出近體詩。而永明體(亦稱“新體詩”)是南齊武帝永明年間形成的一種詩體,受印度梵音學特別是佛經轉讀及梵唄經聲的啟發,周颐首先發現了漢語平、上、去、入四種聲調,著《四聲切韻韻》。詩人沈約又根據自己對四聲的理解,撰為《四聲譜》。王融等人為之扇揚,並在創作中進行試驗,永明聲律論盛極一時。永明詩人在詩中力求做到“一簡之內,音韻盡殊;兩句之中,輕重悉異”,避免所謂八病(即平頭、上尾、蜂腰、鶴膝、大韻、小韻、旁紐、正紐),於是產生了永明體,這是唐代格律詩的源頭。南陳徐摛、徐陵與北周庾信的徐庾體文章綺艷,也是很有名。江淹與鮑照並稱南朝辭賦大家,江淹的《恨赋》、《别赋》與鮑照的《蕪城賦》、《舞鶴賦》並稱南朝辭賦的絕唱。江淹在獄中寫的《詣建平王書》,辭氣激昂高亢,不亢不卑,字行間流露出真實情感。江郎才盡也是指他晚年時減少寫作的情形。

敘事長詩方面以北朝的《木蘭詩》和南朝的《孔雀東南飛》為代表。民歌方面,由于南北文化不同,呈現出不同的色彩和情調。《乐府诗集》即有「豔曲興于南朝,胡音生於北俗」的說法。小說受到名士清談的影響,促成軼事小說的出現,可分為「志怪小說」和「志人小說」。比較有名的有劉義慶的《世说新语》,為後世文學作品提供大量典故和成語。道教影響了中國藝術及科學。例如《遊仙詩》等文學,描述神仙飄逸之妙或藉由神仙之說抒發情懷。道教名士陶弘景、陆修静均擅長神仙文學。《玉臺新詠》爲陳後主妃子張麗華所撰錄,主要收錄男女閨情之作。

文學研究方面,刘勰的《文心雕龙》成為中国第一部系统文艺理论巨著,主張實用的「攡文必在緯軍國」的落實文風,反對不切實用的浮靡文風。南梁武帝的长子萧统组织文人编选的《昭明文选》是中国现存的最早一部诗文总集。於唐朝時與五經並駕齊驅,盛極一時。直至北宋的民間尚傳謠曰:「文選爛,秀才半」。这两部巨著都对后来中国文学的发展产生了深远影响。鍾嶸的《詩品》也是重要的專著,開創中國古代詩論、詩評的體制。其專注於漢朝至南梁的五言詩,確定文章風格的來源,一派學《詩經》、一派學《楚辭》。然而三品評判過於牽強,如下品的曹操、中品的陶潛、嵇康、曹丕等人在今日已獲得較高評價。

南北朝繼承了漢代以來設官修史之制。宋設著作官(宋齊梁陳時官名及分職屢有更改),負責撰修國史(本王朝史)及帝王起居注。南齊始有國史與前朝史之分。北魏亦設著作官及起居令史,使修史官與起居官逐漸分職。北齊始設史館(或稱史閣),為專門修史機構,影響中國日後的官方修史制度。西魏、北周亦設著作官制度。南北諸朝又有大臣監修史書。此外,南梁時始行編修武帝、梁元帝的「實錄」,是為唐代開始一朝接一朝修實錄的濫觴。

纪传体斷代史書在南北朝史學仍佔一席位。官修的有如沈约《宋书》、萧子显《南齐书》、魏收《魏书》,私修的有如范曄《后汉书》。

反映社會各種狀況的史書,亦在南北朝盛行。如范曄《後漢書》、沈約《宋書》當中新增「獨行」、「逸民」(或「隱逸」)、「列女」等類傳記各種人物面貌;宗教史籍有慧皎《高僧傳》;記述寺院建築的有楊衒之《洛陽伽藍記》;地理類著作方面,以郦道元《水经注》為南北朝集大成之作。少數民族歷史亦因五胡各族建政權而深受重視,成就較高的有崔鴻《十六國春秋》、蕭方等《三十国春秋》。

譜學(或叫譜牒學)在南北朝門閥政治影響下而大盛一時。各豪族郡望為求鞏固社會地位和政治權利,乃撰修家牒,以彰顯自身血統、門第及婚宦。繼家譜出現後,又有了家譜學的研究,當時便出現「統譜」、「百家譜」等書籍。

南北朝的注史之學,具代表性的有如裴松之《三国志注》。裴注著重資料搜集、補充史事,不再局限於對音訓及解釋史文,對中國的注史方法產生有相當影響。裴松之對史料相互考異,日後史家有所繼承,如司马光撰《資治通鑑考異》。裴注裡又有對前代史家的評論,這推動了中國史學批評的發展。

本時期的宗教逐漸以佛道為主流,並與玄學互相競爭。佛教在南北朝的蓬勃發展,已脫離先前依附儒、道的困境,於北魏刘宋時開始流行並逐漸中土化,時人对菩萨的信仰十分流行。當時佛教逐漸產生出學派,有名的有三论宗、涅槃宗、天台宗、律宗以及禅宗。三论宗奠基於南北朝僧肇、遼東僧朗、興皇法朗、茅山大明、吉藏大師。因依鳩摩羅什所譯的《中論》、《十二門論》和《百論》等三論立宗,故名為三論宗,屬大乘中觀派。淨土宗的思想主要在《往生論》內,其代表曇鸞提倡他力、易行思想等思想。涅槃宗方面,《十地經論》有勒那摩提、慧光等所形成的地論學派,以及曇無讖翻譯的《大般涅槃經》(北本),傳入南朝後弘揚更廣。律宗始於南北朝法顯、慧光,着重研习及传持佛教戒律、严肃佛教戒规而得名。天台宗是中國佛教最早創立的一個宗派,始祖智顗主要依據《妙法莲华经》,所以又稱法華宗。該宗主張實相和止觀,以實相闡明理論,用止觀指導實修。禅宗達摩主張「教外別傳、不立文字」,提倡以心修禪,出世後還需度化他人。達摩的禪法,簡明深入。。與寶誌禪師、傅大士合稱梁代三大士,有名的還有「一葦渡江」。佛教的高度發展也導致政府抑制、儒道二教激烈的問難。由於大量寺廟與僧侶減少稅收與兵源,各國開始限制信仰佛教的人數與撲滅佛教。其中北魏太武帝、周武帝的灭法運動最有名,與後世的唐武宗合稱三武灭佛。佛教進入劉宋代愈為一般人士所好,從而引起與儒教及道教為種種問題而論爭。如「三世因果之真偽」、「精神之滅不滅」、「佛之在否」等問題盛為諍論。随着佛教的传播,空前的发展出佛教藝術如佛像、壁画、石窟寺院等。其中敦煌千佛洞、雲崗石窟、龙门石窟、麦积山石窟成为中国造像艺术宝库之中的瑰宝。

道教的改革頗多成就,五胡十六國晚期的寇谦之受东晋灵宝派的影響,制作《雲中音誦新科之誡》等經八十餘卷,在精義經理方面無所創新,卻明確聲稱要清理道教。寇謙之對道教進行改革的的總原則是「以禮度為首」,除去五斗米道的三張(張陵、张衡、张鲁)偽法、租米錢稅及男女合氣之術。嚴格齋戒禮拜,使道教組織更為嚴密,道規教儀更為完備,使道教「專以禮度為首,而加以服食閉煉」。由於寇謙之的道教改革,不僅在宗旨、組織、道經、齋儀等各方面創立了新道教的基本規模,而且將北朝君主和漢、鲜卑士族加入道教。將道教發揚至社會各階,一度成為北朝的國教。劉宋的廬山道士陸修靜則收羅以往道教典藉,參考當時的制度級佛教修持儀式,改革南朝的天師道。陸修靜對南朝道教的改革主要體現在他的《陸先生道門科略》中。不過它的發展很快被上清派和靈寶派所遮掩。寇謙之與陸修靜的改革使道教的教規、儀範逐漸定型。而後的陶弘景繼續吸收儒佛兩家思想,充實道教內容,構築道教神仙譜系,敘述道教傳授歷史,主張三教合流,對後世道教的發展影響極大。他融合南方葛洪的金丹道教、楊羲的上清經道教及陸修靜的南天師道後,開創了茅山宗。道教在南北朝時期還造作了大量的經書,道教經書的分類方法三洞四輔十二類就是這時形成的。

在北魏尚有祆教(即琐罗亚斯德教),是古代波斯帝國的國教。祆教的思想屬二元論(即光明神與黑暗神),主神被称为「胡天」,主要經典是《阿維斯陀》。祆教主要由西域進來的粟特人所傳播,當時西域各國都信仰祆教,北朝也有部分皇室信仰,北魏靈太后所祀之胡天神就是祆神。

南北朝時藝術興盛,南朝以绘画為主,北朝以雕塑為主。北朝雕刻的盛行與佛教流行有關。佛寺大量興建,無論是木造、磚築或是石窟建造都有,並且雕刻无数大大小小的佛像。在佛像石窟雕像上,著名的有366年前秦時開鑿的敦煌莫高窟、384年后秦時開鑿麦积山石窟、北魏孝文帝時開鑿的雲岡石窟、龙门石窟、北齐文宣帝時開鑿的天龙山石窟等。石窟藝術最雄偉的是雲岡石窟20窟的座像。佛像容貌豐滿,兩肩寬厚,衣褶線條緊貼身軀而雕,莊嚴中寓有慈祥,表現出佛的胸懷氣度。彩塑最好的是敦煌莫高窟第259窟的造像,含蓄微笑的神態,給人以恬靜的美感。洛陽永寧寺塔基出土的北魏泥塑殘像最為精緻,面目傳神。天龍山石窟還是為北齊佛教文化的代表,以漫山阁及九连洞著称。 綜合印度佛塔及漢朝塔樓為基礎而建立的木塔,為當時重要的建築發展。中國本地的儒道思想亦為藝術提供新主題、新風格。儒家思想多與孝道結合,道家藝術家則偏好自然山水及民間傳說。世俗藝術傳統亦產生變化,尤其展現於繪畫方面。中國山水畫始祖顧愷之為此時期的代表。陶器工藝在中國南方亦有顯著進展,最著名的為「越窯」的綠釉「越瓷」。越窯十分耐用,甚至外銷海外,遠達埃及、菲律賓。六朝也是中國開始出現文人畫家、書法家、重要私人藝術收藏的時期,文學、藝術批評亦有所發展。

南朝陵墓石刻方面,歷代帝王、貴族陵墓前有神道石柱、石碑、石獸等。石獸又稱避邪,由獅子造型演變而來,有驅邪、求福或升天之含意。其中以南朝齊武帝陵前的天祿(雙角石獸)、齊景帝陵前的麒麟最具代表。風格承襲漢代石獸雕刻,善於利用整塊石材,已洗練的手法表現雄偉的氣勢。

繪畫方面,中国古代山水画兴起于南北朝。由於玄學流行,老莊的自然觀和江南秀麗的山水結合,使得繪畫脫離儒學的限制,朝向純藝術的方向發展,有名的畫家有劉宋陸探微與南梁張僧繇。張僧繇畫的龍非常神妙,畫龍點睛就是出源於他的畫工。張僧繇一生苦學,「手不釋筆,俾夜作晝,未曾倦怠,數紀之內,無須臾之閒」,繪有《五星二十八宿神形圖》、《雪山紅樹圖》等等名作。由於山水诗的出现,使得长期以来的以表现人物为主的绘画传统转变為山水景色。南朝有不少以人物画著称的画家。例如劉宋宗炳是中國最早的山水畫理論著述。其《畫山水序》最為著名,精闢地理解「山水以形媚道」之外,在自然山水的觀察,歸納出展現物體遠近的繪畫方法。劉宋山水畫家王微,著有《敘畫》一篇,強調觀察自然和主觀能動作用。南梁萧贲可在在团扇畫出「咫尺之内,而瞻万里之遥;方寸之中,乃辨千寻之峻。」,表现出遥远的空间距离感。绘画理论在此時期已經成熟,南齐谢赫撰著研究绘画理论的重典《古画品录》。這本書分為論繪畫六法論與畫品等兩個部分。他所提出繪畫品鑑的六法,已不仅限于人物画,对后世有很大影响。其中「氣韻生動」的理論更被歷代畫評家奉為最高水準。

魏晉南北朝是楷書發展的青年期,其中碑刻是楷書的寶庫。北朝碑刻即所謂的魏碑書法,魄力雄渾,氣象渾穆,體態多變。《龍門二十品》是指在龍門石窟中發現的北魏時期二十方造像記,這些作品被認為是魏碑書法的代表。《张猛龙碑》备受书法家们的推崇。清朝碑刻學家杨守敬评论:「书法潇洒古淡,奇正相生。」,成就遠超過唐人作品。《鄭文公碑》是北魏书法家郑道昭的作品。清朝碑刻學家叶昌炽认为:「其笔力之健,可以刲犀兕,搏龙蛇,而游刃于虚,全以神运」唐朝書法家歐陽詢和虞世南都深受此碑影響。其他還有《石門銘》等眾多的墓誌銘文。南朝有名的碑如《爨龙颜碑》、《瘗鹤铭》等。

在整個魏晉南北朝期間,由於邊疆民族內遷、北方人群的南下,造成文化大交流及混成。由於儒學一統的局面打破以及玄道佛的興起,使得學術研究朝向多元化。各國為了生存或戰爭,多少推行一些改革措施以確保某些地區農業與手工業的發展。這些都使得科學技術大幅提升。

郦道元從小志於地理學研究。由於當時地理著作不夠完備,所以他在各地做官之際進行實地考察,最後以《水经》為藍本,完成《水经注》。《水經註》為陸地水文地理,他以水道為綱,詳細描繪其本身性質與週邊環境,為中國古代地理學做出傑出貢獻。《水經注》不僅講河流,還詳細記載了河流所經的地貌、地質礦物和動植物。《水经注》中记载了许多古生物残骸化石和遗迹化石,渭水上游成纪县(今甘肃庄浪)僵人峡還有人类化石。後世可以從中了解古代的耕作制度、古代植物種類和植被分佈,動物的地區分佈及其活動的季節怙,以及古人如何利用它們取得經濟效益。

贾思勰為北魏農學家,家境較為富裕,他面對當時的天災人禍及連年飢荒,主張重視農業,並以齊民謀生為己任,所以決定寫出一本農書。他所寫的《齊民要術》在中國或世界的農學史上均佔有重要地位。《齊民要術》以當時黃河中下游地區,特別是山東地區為重點。描述當時農業生產概貌,並介紹中國傳統農業細耕和多種經營方法和包括食品加工技術在內的農業科技的高超水準。書中正文分成10卷,92篇,收錄1500年前中國農藝、園藝、造林、蠶桑、畜牧、獸醫、配種、釀造、烹飪、儲備,以及治荒的方法。援引古籍近200種,其中《氾勝之書》、《四民月令》等漢晉重要農書現已失傳,後人可以從此書了解的漢晉時期的農業運作。這是經營莊園的地主和農民所需的知識,也是北朝農業技術發展的重要指標。

祖冲之為劉宋人,家族歷代成員大多熟悉天文歷算,所以祖沖之從小對天算有興趣。他平生著作豐富,天文方面有《上「大明歷」表》、《駁議》;數學方面有《缀术》、《九章術義註》、《重差術》。他寫有《安邊論》一文,建議朝廷開墾荒地,發展農業,安定民生,鞏固國防。祖沖之的主要成就在數學、天文曆法和機械製造三個領域,在張衡、劉徽的基礎上,將圓周率數值精準到小數點後七位數字。他是世界上第一個將「歲差」數值記入曆法運算中,他還將置閏規則修整,這是唐代之前最好的方法。祖冲之和他的儿子祖暅共同提出推算球体体积的公式,稱為“祖氏定理”,比西方早一千年以上。在機械方面,製作出指南车、木牛流马、千里船(即腳踏輪船)、水力運轉的水碓模,以同一原動輪帶動碓和磨作功。

北朝張子信初步發現了日行盈縮的規律。北魏末年,張子信避葛荣兵亂而隱居海島,用圓儀測天三十年。大約在565年前後,他發現關於太陽運動不均勻性、五星運動不均勻性和月亮視差對日食的影響的現象,並且提出計算方法,在中國古代天文學史上具劃時代的意義。經由張子信的學生張孟賓、劉孝孫等人的努力,這三大發現及其計算方法在孟賓歷和孝孫歷576年中大約已被應用。

南朝醫學發達,有名的有南梁的徐之才與姚僧垣。徐之才為醫學世家,他隨梁豫章王蕭綜投奔北魏後獲北魏皇帝重用。他對本草藥物及方劑研究較深,撰有《藥對》及《小兒方》。對婦科也有一定的見解,其《逐月養胎法》實本自先秦時期《青史子》中胎教法而作,對於孕婦之衛生及優生均有重要意義。徐氏一家由南仕北,對於南北地區醫藥之交流,也有積極的意義。姚僧垣,曾多次為梁武帝、梁元帝治病。之後因為戰亂投奔北周,為北周大臣醫治疾病而出名,其治病藥方留存到唐代,成為治療「氣兼水身面腫」的重要藥方。此外,較重要的醫學學者有陶弘景,撰有反映漢末以來醫藥發展的《本草經集注》(今存第一卷)、《名醫別錄》(附入《本草經集注》,已佚)等醫籍。而可使人中毒甚至死亡的寒食散,南北朝的諸王及大臣們也常有服用。

%% -*- coding: utf-8 -*-
%% Time-stamp: <Chen Wang: 2019-12-19 17:28:49>


\section{刘宋\tiny(420-479)}

\subsection{简介}

宋(420年-479年)史稱劉宋或稱南宋(跟其他南朝政權,南齊、南梁及南陳看齊,然而為了避免跟趙氏南遷的政權混淆,大部分人會多用前者劉宋為主)是中国歷史上南北朝时期南朝的第一个朝代,也是南朝版圖最大的朝代,當時所謂「七分天下,而有其四」。439年,北魏統一中国北方後,與劉宋形成南北對峙。劉宋强盛时,其统治地区北以秦岭、黄河与北魏相邻,西至四川大雪山,西南包括云南,南至越南中部横山、林邑一带。

420年,宋武帝刘裕取代东晋政权而建立。国号宋,定都建康(今江苏省南京市),因国君姓刘,为与后来赵匡胤建立的宋朝相区别,故又称为刘宋。

以刘裕世居彭城为春秋时宋国故地,故以此为国号。又以五德終始說,刘宋为水德,故别称水宋。

开国皇帝刘裕出于行伍,自幼家贫。时值东晋末期,民變此起彼伏,朝廷内部斗争也十分激烈。402年,东晋大将桓玄乘朝廷实力虚弱,起兵篡位,国号“楚”。刘裕与刘毅等起兵勤王,并最终消灭了桓玄的力量。此后,刘裕率军南征北伐,其势力不断得到稳固壮大,并先后攻灭刘毅、司马休之等实力派,最终迫使晋恭帝将帝位禅让给他,420年劉裕建宋,年號永初。刘裕即位後,因為他已在晉末實行各種改革如土斷、壓制豪強、澄清吏治、強化軍隊等等,所以在位的三年中,除了恢復漢代舉孝廉、策秀孝的制度並強化官僚法制之外,主要政策仍是在休養生息、恢復晋末北伐的國力損傷,並計畫在422年出征北魏。結果422年五月刘裕得病驾崩,北伐取消。

422年劉裕死,太子劉義符即位,北魏趁機派十萬大軍南侵,占領洛陽等河南地區,逼退宋將檀道濟。424年,徐羨之、謝晦等託孤輔政大臣,害怕失德無禮的劉義符會敗壞國政,遂以「廢昏立明」為名號,廢殺劉義符,改立劉裕的第三個兒子劉義隆為皇帝,是為宋文帝,年號元嘉。宋文帝在426年除掉權臣徐羨之等人後親政,他在位三十年,励精图治、知人善任、提倡節儉並澄清吏治,国家生产经济因此大力提升,被稱為元嘉之治,為六朝治世之典範(也是江東第一個治世)。

430年起,宋文帝首次北伐,到彥之率領的五萬宋軍,成功占領河南洛陽等地。但由于軍力不足,加上文帝的過度指挥,以致北魏逼退宋軍數萬主力後,於431年重佔河南。436年名将檀道濟因军功被宋文帝猜忌而被铲除,又使南朝宋失去能與北魏制衡的大將。445-446年當北魏發生蓋吳起事時,南朝宋沒能即時北伐。到445年時,北魏趁勁敵柔然暫衰時開始發動多次小規模南征,雙方於淮南來回拉鋸,元嘉二十七年(450年),宋文帝出兵六萬北伐北魏之河南地(二次北伐),但卻被北魏太武帝拓跋燾所之率六十萬大軍正面擊敗,六十萬魏軍遂引兵南下,威逼建康。魏軍所過之處大肆搶掠燒殺,江淮地區損失慘重、「邑里蕭條」,元嘉之治因此衰落。宋文帝在452年趁拓跋燾遇弒之機會,派軍進行第三次北伐,但仍無功而返,此後劉宋無力再舉,註定日後國防線逐漸南撤的命運。

453年,宋文帝长子刘劭發生巫蠱事件,弒父即位,其三弟劉駿起義兵攻劉劭,獲得各方軍鎮的支持,於是斬劉劭於台城,劉駿自立為帝,是為宋孝武帝。

宋孝武帝統治期間,雖有諸王劉義宣、劉誕等相繼叛亂,但大多很快平定,和北魏的戰事也只限於山東半島,雖小勝北魏但影響不大,因此總體來說,孝武帝統治的十一年算是個相對安定的時期,孝武帝的積極政策也促進了江南的開發與貨幣經濟的深化。一直到463年底至464年,浙江等地發生大旱災,造成慘重的大饑荒,浙江十分之六的戶口餓死逃散。

宋孝武帝是一個頗有作為、積極改革制度的皇帝。他加強中央集權,撤除「錄尚書事」職銜,並分割州、郡以削弱藩鎮實力,並開始以中書舍人戴法興、巢尚之等人處理中樞機要事務,形成後代所謂「寒人掌機要」的政治局面,孝武帝的集權化統治也被史書稱為「主威獨運,官置百司,權不外假」。孝武帝同時重用江東寒門沈慶之與傖荒北人柳元景,依照兩人的功績,先後提拔為三公,開啟吳興沈氏與河東柳氏攀升為南朝高門的起始之路,並開創南朝寒門、寒人以軍功升為三公的先例。

孝武帝另外也對門閥制度進行一定程度的整頓,給予世族制度新的生機。除了拔用上述的沈、柳為三公之高門,更提拔孤寒衰微的袁粲為員外散騎侍郎和侍中;拔擢寒門的顏竣、寒素的顏師伯成為高官重臣;任用南北之望的名士琅邪王彧、會稽孔覬為散騎常侍,一度矯正過去散騎常侍受人輕視的不良習慣;甚至從461年開始,把與商人等通婚、私下經商的士族,開除士族資格並黜為將吏,是為檢籍政策的先聲。

464年夏季,孝武帝死,其子劉子業繼位,荒淫残暴,朝廷內外人情汹汹,心懷恐懼,劉子業不久被湘東王劉彧弒殺。劉彧在建康自立為帝之後(宋明帝),因為得位不正,面臨孝武帝第三子劉子勛登基為帝、聯合兄弟方鎮圍攻建康的艱鉅情勢。宋明帝政權雖然領土、人口都不到劉子勛政權的十分之一,但是以伐亂為名,憑藉量少質精的中央軍,採取各種積極手段:採用才幹名士蔡興宗的意見,撫慰叛亂將士在京師的親人,安定人心。重用沈攸之、張永、蕭道成等才幹武將。放權給諸弟劉休仁等人積極平亂。

於是上下一心、兵強將勇,因此打敗劉子勛並平定江南與淮南各地區,最後全面誅殺孝武帝子孫。但是淮北方鎮薛安都等人為了自保而向北魏求援,於是北魏大軍在四年之內陸續攻下淮北、山東半島地區,劉宋戰亂不斷,國力大衰,人民痛苦指數飆升。又因為必須對有功的軍人加官晉爵、大肆封賞,於是造成士族制度的嚴重破壞,清濁不分、官品淆亂。

472年,宋明帝死,太子劉昱繼立,宋明帝遺詔命蔡興宗、袁粲、褚淵、劉勔、沈攸之五人託孤顧命大臣,分別掌控內外重區,另外命令蕭道成為衛尉,參掌機要。其中遺詔雖任命袁粲、褚淵在中央秉政,但實際上接受宋明帝秘密遺命,就近輔佐新帝劉昱,掌控宮中內外大權的人物,是宋明帝最親信的側近權倖——王道隆與阮佃夫二人。

宋明帝在死前,為了穩固兒子的皇位,大肆誅除有能力的皇弟宗室、功臣武將和高門士族,造成劉昱繼位後中央和地方軍鎮互相猜忌、攻伐的政治亂象,使得武將蕭道成因此崛起,逐漸掌握中央軍權。特別是474年,桂陽王劉休範以清君側之名造反,殺死了權倖王道隆與顧命大將劉勔,幾乎就要攻下建康城,但蕭道成即時回軍,平定亂事。事後蕭道成接替劉勔的地位,上升為與宰相袁粲並列的「四貴」之一,更受到權倖阮佃夫的倚重,因此交結地方軍鎮都督,權勢日漸擴大。476年,文帝在世长孙建平王刘景素在京口起兵,亦被萧道成等镇压。477年,年滿15歲的劉昱在殺掉權臣阮佃夫後,與蕭道成發生激烈衝突,但卻意外被蕭道成弒殺,蕭道成趁機改立明帝第三子劉準為皇帝,即宋順帝。蕭道成獨攬軍政大權後,挾持軟弱的褚淵,以武力平定忠宋大臣袁粲、沈攸之的起義。479年,年幼的宋顺帝劉準把帝位禅让给了蕭道成,宋被南齐所取代。當時民間以一首歌謠傳述蕭道成殺袁粲篡宋的事業:「可憐石頭城,寧為袁粲死,不作褚淵生!」

從宋武帝劉裕時代(420-422年)以黃河為界,與北魏對峙的局面,到劉裕死後的423年,北魏趁機攻下河南三鎮(洛陽、虎牢、滑台),從項城到濟南大致形成一條國界線,分開北魏與劉宋,劉宋仍保有山東半島與江蘇省北部的淮北地區。但與北魏為界的大塊面積,因為屢遭兩方進攻掠奪,成為所謂「邊荒」地區,僅剩居民聚集在榛木所圍成的山寨堡壘,被稱為「榛人」。項城到濟南的「邊荒」國界維持了長達四十三年,四十三年中兩國大軍屢次越界征伐,但屢得屢失,兩方都無法把佔領的土地長久地穩固下來,因此維持了四十多年勢均力敵的局面。

一直到466年,劉宋山東、淮北的鎮將薛安都、崔道固等人,因為害怕篡位自立的宋明帝討伐他們,而向北魏求援。北魏趁機派五萬以上的大軍,於打敗宋明帝的北征軍張永之後,陸續在四年內攻下了山東、淮北的所有城鎮,劉宋被迫以淮河與北魏為界。雖然宋明帝心有不甘,屢次派沈攸之等人北征,但從此(469年後)南朝與北魏的淮河國界就大致固定下來,一直到31年後(500年南齊末)才因為壽陽鎮將裴叔業投降北魏,使國界進一步往南退。

刘宋的行政区划袭承东晋,实行州、郡、县三级制。

州是第一级行政区。州的最高行政长官称刺史,劉宋一朝的州數大致在二十州上下,至宋末穩定為二十二州。其中不少州是僑寓州,為寄住在南方州郡上,不一定有實土。

尹、郡、王国、公国(部分)是第二级行政区。尹的最高行政长官称尹,郡的最高行政长官称太守,王国的最高行政长官称内史,公国的最高行政长官称相。

县、公国(部分)、侯国、伯国、子国、男国是第三级行政区。县的最高行政长官称令或长,公国、侯国、伯国、子国、男国的最高行政长官称相。

劉宋前期繼承「東晉門閥政治」的地理格局,以荊州(鎮江陵)和南徐州(鎮京口)為核心軍鎮,所以劉裕規定兩州必由劉氏宗王擔任刺史。其中荊州因為州大民多、「地廣兵強」,又統攝雍、南梁、益等州,支撐劉宋西半的安危,故有「分陝」之稱,劉裕遺詔說荊州刺史需「諸子次第居之」,說明荊州的重要性略高於北府兵根本的南徐州。劉宋宗王擔任荊州刺史的結果,是促使荊州士族與揚州士族的合流,大致結束東晉百年荊、揚對立的局面,代表人物即是江陵士族劉柳、劉湛父子,曾各自取得相當於副宰相的官位與權勢。

中期因為454年荊州刺史劉義宣,趁著新帝劉駿即位的弱勢格局,發動十萬荊州鎮軍挑戰建康,因此劉駿在同年平定劉義宣之亂後,即刻從荊州東部分出新的一州名郢州,並廢除荊州重兵來源的南蠻校尉,其營戶兵力遷至建康,有效地削弱荊州,瓦解其「分陝」地位。之後劉駿又土斷雍州,大幅強化雍州的實力,不但讓原來的大荊州地域,陷入荊、雍、郢三州相互牽制的局面,後來隨著雍州軍力的不斷加強,至宋末沈攸之起義失敗之後(478年),「江陵素畏襄陽人」的局面已大致形成。

劉宋選官制度仍以九品中正制為主,但宋初門閥制度的整體格局,卻是從東晉末年義熙時代(405-419年)劉裕等京口北府的軍事集團崛起開始,延續繼承下來的新格局。也就是說,大量京口將領混入世族門閥的結構中(多成為中下層世族),擠壓了原來名門舊族的地位與空間,因此南朝士族從劉宋開始,常會刻意去排擠寒門、寒人,好顯示自己的清高地位。於是高門就有許多「士庶區別,國之章也」、「士庶之際,實自天隔」的言論出現,這實際上是名門舊族的一種防禦性反應。

雖然宋初就有武人將領合法地進入世族結構中,但劉宋前半的元嘉之治,士族制度卻是極其完備的,一般也認為是文化士族的「全盛期」(同時也是「最後的榮光時期」)。在宋文帝治下,史稱其「綱維備舉,條禁明密,罰有恆科,爵無濫品。故能內清外晏,四海謐如也」,當時的江南社會:「閭閻之間,講誦相聞;士敦操尚,鄉恥輕薄。江左風俗,於斯為美,後之言政治者,皆稱元嘉焉」。這是因為宋文帝除了重用並放權給兼具才幹與名望的風雅士族,如王華、王曇首、殷景仁、謝弘微、劉湛、范曄、江湛、王僧綽等,更重要的是,文帝也能夠尊重王敬弘、王球這一類缺乏理政才幹的高門清望,雖不給實權,但仍任命為副宰相與吏部尚書,在用人上保持住門閥制度的清濁流品,因此能激清揚濁,使得「士敦操尚,鄉恥輕薄」。

劉宋中期的門閥制度,雖然因為450年宋文帝大舉北伐失敗後的困局,使寒人得以竄改籍注或詐列士籍,混亂士族的流品,但在宋孝武帝的整頓革新之下,仍使門閥制度獲得一定的生機。一直到465年宋明帝自立為帝後,才因為廣募部曲、濫賞軍功,造成士族制度嚴重破壞,成為劉宋滅亡的重要原因。

劉宋前期為對付北魏,積極聯絡柔然、胡夏、北燕、高句麗、吐谷渾,希望對北魏包夾圍攻,但都被魏軍以優勢機動力各個擊破,無法發揮包夾的效果。其中劉宋與柔然的聯盟關係最穩固也最持久(延續到南齊),對北魏的危害也最大,因此北魏常要先北向摧毀柔然的主力,然後才敢在隔年大舉南伐劉宋。

劉宋與北魏的對峙,除了幾次大規模的戰爭衝突以外,其實一半以上的時間,雙方保持相對和平的外交關係。雖然劉宋稱北魏為「索虜」、北魏稱南朝為「島夷」,但他們仍不定時的互派使者「交聘」,維持南北的交涉往來。南北通使往來,在南北史書上的記載雖然各有偏頗扭曲,如魏書記載,421年劉裕派沈範、索季孫等到北魏「朝貢」(宋書記為「報使」),但是實際上是一種平等的對等關係。而且使者代表國家,南北的競爭不只是軍事武力的競爭,文化與氣度上也有互別高低的意味,從劉宋中期開始,南北雙方開始精選使者的素養氣質,如清代史家趙翼所稱「必妙選行人,擇其容止可觀,文學優瞻者,以充聘使」。如果出使有失國體,使者回國後則會被嚴加懲處,這多發生在北魏前期文化素養不高的條件下。劉宋中期時,南北曾有五年的互市貿易。453年宋孝武帝登基後,北魏派使者「求通戶市」,宋孝武帝在與公卿大臣廣泛議論後,決定答應互市。兩國官方的貿易關係,大約持續到458年邊境發生小規模戰事而止。

劉宋與東北亞的倭國、百濟則有密切的朝貢關係,倭國曾在晉末、劉宋對江東朝貢十多次,史書記載有五位倭王,是為倭五王;百濟在劉宋後期與江東的關係更形緊密,似乎結成軍事同盟,共抗北魏。因為469年北魏完全奪取劉宋在山東半島的城鎮後,百濟因為早前渡海在遼西或山東半島沿岸設有港岸據點,因此與南朝同樣面臨北魏的軍事壓力,479年宋齊易代之後,北魏還曾在488年派軍進攻過百濟的城鎮,卻被百濟打敗。令外又因為百濟與倭國對於朝鮮半島東南部的伽倻(任那)可能有爭奪關係,因此兩國在對劉宋的朝貢外交中常常是互相牽制的對立關係。也因為兩國都想要討好劉宋,所以頻頻向劉宋朝貢示好。譬如倭國在朝貢時,一直希望劉宋冊封倭王為「都督」百濟在內的大將軍,讓倭王有統治百濟的名分,但要求總是被劉宋拒絕。

高句麗因為是朝鮮半島與東北亞中最強的國家(更是東亞世界中僅次於魏、宋的第三強國),所以對劉宋的朝貢關係不太緊密,兩國關係主要是針對北魏而結成的鬆散軍事聯盟。高句麗常會單方面中斷對劉宋的朝貢,說明劉宋對高句麗只存在形式上且薄弱的君臣關係。不過在465年之前,因為劉宋強盛的水軍能夠在渤海沿岸執行任務,相對於遠離海岸且毫無水軍的北魏勢力,高句麗在465年前一直選擇劉宋作交往、朝貢的對象。劉宋曾在438年派將領王白駒,率水軍七千人渡海到高句麗的遼東,想迎接兩前年滅國的北燕主馮弘來到南方,結果高句麗先把馮弘處死,並派兵把王白駒繳械,強制遣送王白駒等回劉宋,隔年再回送八百匹馬給宋作賠禮;到了465年後,可能因為當年劉宋發生劇烈的內鬥(見劉子勛條),高句麗從此改對北魏進行較為緊密的朝貢關係,並在469年後長期疏遠南朝(北魏在469年攻下山東半島),只和南朝保存微弱的外交聯繫。475年高句麗更大破劉宋的盟友百濟,破其國都、殺百濟王,佔領漢江流域,國力達到極盛。這次劉宋沒有再派出水軍到遼東、朝鮮,說明劉宋的無力干涉與衰落。

劉宋交州(越南北部)的南邊與林邑國(今越南之中南部)接鄰。東晉末期,林邑有數年的內亂,劉宋建國元年420年,交州刺史杜慧度派兵萬人南征林邑,林邑請降,並向宋廷致送大象、金銀、古貝等禮物。421年,林邑王陽邁一世遣使到宋廷入貢,並獲宋武帝冊封。但到陽邁二世時,於427年入侵日南、九德等郡。431年,林邑入貢宋廷。432年,陽邁二世派水兵入侵九真,交州刺史阮彌之派軍抵抗,驅逐至區粟而回。433年,陽邁二世遣使到宋廷,要求「領交州」,宋廷不許,陽邁二世因此大為憤恨,雖常遣使入貢,但亦常派兵入侵交州。

446年,宋文帝派龍驤將軍交州刺史檀和之、太尉府振武將軍宗愨等征討林邑。戰前,文帝提示檀和之,倘若林邑國能夠誠心求和,便可答允。檀和之派人向陽邁二世諭以恩信時,陽邁二世竟加以扣留,於是雙方進行交戰。林邑軍先以大象軍取得首勝,後來宗愨提議用獅子的外型去威嚇大象,可以取勝。主帥檀和之採納計策,果然大敗林邑軍。宗愨部隊更一舉攻克首都林邑(Campapura),擄獲無數珍寶、黃金數十萬斤,陽邁二世出逃。此一征戰令林邑元氣大傷,「家國荒殄,時人靡存」。此後,林邑國沒有再起兵進犯交州,對劉宋甚為恭順,多次遣使到建康訪問進貢。

自晉室南遷之後,苟延殘喘地偏安江南。原本居於華北的漢人氏族為了逃難而向南遷徙,大量來自中原的移民士族改變了江南地區的人文景觀,甚至口頭語言也逐漸與古河洛語言接軌。[來源請求]

南朝宋時期,主要把土著蠻夷分成蠻人、俚人、僚人三種類型,三者有時被通稱為「南蠻」。蠻人在長江流域以板循蠻、盤瓠蠻與廩君蠻实力最大,板循蠻又稱賨人,原居益州巴郡閬中一帶,之後經渝水北遷漢中、關中。廩君蠻原在益州巴郡、荊州江陵一帶,後來擴展到長江漢水與淮西一帶。史書上提到的巴東蠻、宜都建平蠻都是指廩君蠻。盤瓠蠻又稱「溪人」,發揚地在辰州,分佈現在的湖南與江西一帶。

南朝劉宋政府為了對付蠻人,在荊州置南蠻校尉、在雍州置寧蠻校尉,專責教化及討伐南蠻。為了在荊雍的強大蠻族群體,南朝劉宋政府在440至470年代曾發動大規模地討蠻運動,有兩次的主將分別為雍府大將沈慶之、荊州刺史沈攸之,捕獲數十萬的蠻族人力。而在450年代,沈慶之、王玄謨大致討平淮水蠻,強化了劉宋在淮南地區的國防。

俚族的範圍在南嶺、今貴州南部到海南島、越南北部一帶。468年起李長仁與李叔獻兄弟據交州抵制刘宋朝廷,當時的宋明帝政權因為正與北魏全力爭奪山東、淮北地區,無力征討交州,只好承認李長仁的刺史名號,維持劉宋對交州名義上的統治,並於471年在交、廣兩州交界地新設越州,以防禦李氏兄弟。李氏兄弟很可能具有交州俚僚族群的血統,他們在交州的割據一直維持到485年才被齊武帝討平。

僚人主要分散在四川、漢中的山谷空地,與賨人的分布區頗有重疊。當時「僚人與夏人(漢人)參居者,頗輸租賦」,說明其編戶化與華化的趨勢較重。

劉宋常態兵力大約二、三十萬,極限動員時可能有四十萬,,但劉宋在淮水以北征伐時,因為受限於後勤供應,只能發動五、六萬兵力,使得北伐經常失敗。劉宋前期北府兵獨大,成為中央軍與荊州、北徐州方鎮的主力來源,故此時仍以世兵制為主;中期荊雍兵崛起,逐漸取得一定的優勢,學者田餘慶認為:「北府兵力日衰,荊雍兵力日盛,是同一個歷史過程的兩個方面」。此時世兵制衰落,軍隊主力逐漸被募兵制和徵兵制取代,特別是將領自招部曲的募兵制,更成為宋末軍隊中的精銳核心。譬如469年後流亡南方的青齊豪族,就被蕭道成收納招募為將官、部曲,成為蕭道成建齊易宋的主力。學者有的稱此武力集團為「淮陰集團」,有的稱之為「青徐集團」。

宋孝武帝大明八年(464年)官方紀錄,全國有901,769戶,5,174,074人,但因為十多年前發生北魏破壞江北的燒殺屠掠,江北人口大減,以及463-464年浙江等地發生大飢荒,浙江人口死亡逃散十分之六,所以劉宋盛世年代(元嘉之治)的官方戶口數字,應當超過一百萬戶、六百萬口。

劉宋的江北地區主要是村塢型經濟,常受戰亂影響而發展有限;江南社會主要是莊園經濟。世族與寺院的莊園大部分都是多方經營,從自給自足的性質,朝向商品經濟發展。農田有良好的水利系統供種植稻、麥、粟、桑、麻、蔬菜等作物,還可以種植竹木果樹、養魚、畜牧等等。還有紡織、釀造、生產工具等手工業。世族的莊園生產主要交給佃客、部曲和奴隸,而寺院是一般僧侣與民戶。由地主集中開墾,這對於地區的開發起一定的作用。由於世族享有特權,佛教較為盛行,致使地主莊園與寺院莊園膨脹,並且隱匿許多農戶。

農業是莊園經濟的重心,深受朝廷與世族關切。開墾山林與土地兼併的情形在劉宋一直非常旺盛,朝廷雖有禁令,但難以禁止世族兼併土地或霸占山澤,宋孝武帝大明元年(457年)朝廷乾脆承認佔領山林川澤的法令以限制世族搶佔範圍。法令頒布後果然刺激豪門權貴兼併山澤土地的活動,也因此促進了商品經濟的發展。劉宋相對北魏來說比較安定,南渡的移民在初期與末期仍然絡繹不絕,農業生産繼續有所發展。比較突出的地區有荊、揚二州,而益州居次。揚州是劉宋最發達的地區,其中以建康及其周圍地區發展最大。而三吳地區(吳郡、吳興、義興)是中央財庫、各種支出的主要來源。

由於朝廷大力提倡農桑,戶調征絹布,當時絹布的地位等同貨幣,這些都促進紡織業的生產。劉宋的紡織業與養蠶業比較發達,產地以荊、揚二州為主。由於絲、綿、絹、布等是國家調稅的主要項目,因此紡織是民間普遍的副業。織錦業則在益州為主,劉裕滅後秦,把關中的織錦戶遷到江南,開始在江南發展織錦業。當時富豪人家穿綉裙,著錦履,以彩帛作雜花,綾作服飾,錦作屏障。朝廷設有專官管理礦冶,用水排鼓風冶鑄。鍊鋼則使用一種雜煉生鐵和熟鐵的灌鋼法。這種方法可以鍊出優質鋼,用來製造寶劍和刀。瓷器的燒制技術早在三國、晉朝時期成熟。劉宋時以青瓷為主,產地集中在會稽郡(浙江紹興)。其硬度高,釉料勻,通體青瑩。江南其餘地區的制瓷技術各有自己的特點。劉宋的紙張潔白勻稱,完全取代了簡牘,藤紙與麻紙都很流行。造船業也十分興盛,如宋末沈攸之起義反蕭道成時,荊州作部曾「裝戰艦數百千艘」,而且三吳運河網也持續修造,到南齊時已大致完成,暢通了三吳與建康的交通。

劉宋農業和手工業發達,加上江河交通便利,使得商業日漸發達,江南社會穩定地朝貨幣經濟與商品經濟發展,甚至連江北的漢中地區,也在劉宋中期開始使用貨幣。但由於國家控制的銅礦不足,使得幣制屢變,質量不精。市場上有普通的生產用品、生活用品與奢侈品,商賈小者坐販於列肆,大者轉運於四方,而凡是大批運進的商品買賣,多是世族莊園所生產的經濟作物。商稅是朝廷收入的大宗,然而世族有免關稅權,在任期屆滿時帶著大批貨物作為「還資」,然後轉販各地。商業重鎮有建康、江陵、成都、廣州、广陵等地。建康是三吴的經濟中心。會稽、吳郡、餘杭居次。廣州是海上貿易重鎮,貿易對象有东南亚各國、天竺、獅子國、波斯等國。江陵是關中、豫州、益州、荆州、交州、梁州的轉運站。成都不僅商業繁盛,也是蜀錦的重要產地。

劉宋詩風流行的是元嘉體。元嘉體是宋文帝元嘉年間的詩風,代表人物有「元嘉三大家」謝靈運、顏延之與鮑照。他們的共同功績是把古體詩推進到完全成熟階段,並且注意聲律和對偶的運用,並且逐漸發展出近體詩;袁淑、謝莊亦為有名詩人。民間詩人則以劉宋初期的陶淵明最具代表性,其擅長描述田園生活,風格清新樸實,提升古體詩內涵,表現出高遠純潔的情操。其作品《桃花源記》寓意追求一個可供逃避亂世的和諧世界,富有哲理。其詩歌、散文及辭賦廣泛影響後世名家如王維、李白、杜甫、蘇軾、辛棄疾、陸游等人。

小說受到名士清談的影響,促成軼事小說的出現,最有名的是宗王劉義慶招集文人才士所編寫的《世說新語》,為後世文學作品提供大量典故和成語,也是唐代晉書編撰的重要史料來源。

劉宋繼承了漢代以來設官修史之制。宋設著作官,負責撰修國史(本王朝史)及帝王起居注。宋代最著名的兩本史籍,是范曄的《後漢書》與裴松之的《三國志注》。《後漢書》新增「獨行」、「逸民」(或「隱逸」)、「列女」等類傳記各種人物面貌,最被稱道;裴注著重資料搜集、補充史事,不再局限於對音訓及解釋史文,對中國的注史方法產生有相當影響。裴松之對史料相互考異,日後史家有所繼承,如司馬光撰《資治通鑑考異》。裴注裡又有對前代史家的評論,這推動了中國史學批評的發展。

劉宋時譜學(或叫譜牒學)在門閥社會影響下而開始盛行。各家士族郡望為求鞏固社會地位和政治權利,乃撰修家牒,以彰顯自身血統、門第及婚宦。繼家譜出現後,又有了家譜學的研究,當時便出現「統譜」、「百家譜」等書籍。

劉宋在東晉之後,延續晉代的文化發展。由於玄學流行,老莊的自然觀和江南秀麗的山水結合,使得繪畫脫離儒學的限制,朝向純藝術的方向發展,陸探微為宋明帝時期著名的宮廷畫家,然而其作品均已失傳。由於山水诗的出现,使得长期以来的以表现人物为主的绘画传统转变為山水景色,例如宗炳是中國最早的山水畫理論著述。其《畫山水序》最為著名,精闢地理解「山水以形媚道」之外,在自然山水的觀察,歸納出展現物體遠近的繪畫方法;另外山水畫家王微,著有《敘畫》一篇,強調觀察自然和主觀能動作用。

江南社會的人口很複雜,大致上可分為四個階層:名門豪族的世族;自耕農、新民等從事農工商的編戶齊民;屬於部曲、佃客、衣食客、門生舊故等依附世族的依附人,受政府控管的雜戶、百工戶、兵戶與營戶也是依附人;最後是奴婢、生口、隸戶,這些都屬於奴隸。

雖然文化士族的實力大削,但劉宋仍維持世族社會的結構;而江北豪族的地位與權力雖遜於江南的僑吳士族,但在經濟力與軍事實力方面,卻高出甚多。世族控制的人口有部曲、佃客與奴隸,不經「自贖」或「放遣」,是不能獲得自由的。由於南朝大家族制的衰亡使得部曲逐漸受國家控制。佃客的來源有政府依官品賜給與私自招誘。奴隸的主要來源是破產的農民或是流民,他們是地主的私產,因而可以抵押或買賣。為了防止逃亡,奴隸都被「黥面」。奴隸可以經由「糜喃為客」、「發奴為兵」等方式轉化為地主的佃客和國家的士兵。自耕農是當時農業生產的重要力量。他們對朝廷負擔租調、雜稅、徭役以及兵役,這些都使許多自耕農破產流亡,淪為世族的部曲和佃客。劉宋實行三國以來的世兵制,兵戶世代當兵,平時還需要交納租調。由於手工業者很缺,故官府對雜戶或百工戶的控制極嚴,百工戶從民間徵調到官府作坊後,與配到作坊里的刑徒為伍,終年勞作,世代相襲。如果世族、官僚私佔百工戶往往受到懲治。

江南社會約在晉末宋初由大家庭制轉化為小家庭,在同一家族不同職業的十家就有七八家之多,互相漠視。這是因為宗族發展後各家庭親疏貧富不同,若無共同外患就容易分離;朝廷課稅方式對大家族制無益而導致的。


%% -*- coding: utf-8 -*-
%% Time-stamp: <Chen Wang: 2021-11-01 15:03:03>

\subsection{武帝劉裕\tiny(420-422)}

\subsubsection{生平}

宋武帝劉裕(363年4月16日-422年6月26日),字德輿,小字寄奴,彭城綏輿里(今江蘇省徐州市銅山区)人,東晉末年至南北朝初期的軍事家、政治家,南北朝時期劉宋開國皇帝。早年出身十分貧寒,劉裕最初為北府將領孫無終的司馬,在孫恩之亂中展現其軍事才能,及後更發起義軍擊敗篡位的桓玄,恢復了東晉政權,並獲得了極高名望,並在不久之後掌握朝政大權。

劉裕趁南燕内讧之际而出兵滅燕,隨後又平定了盧循之亂,以及消滅了劉毅、諸葛長民及司馬休之等異己,鞏固了在東晉國內的地位。接著又乘後秦内乱而北伐,收復了洛陽及關中地區,受封宋公並得九錫,終篡奪了東晉政權,建立劉宋,正式開始了南北朝時代。

劉裕家族在早年隨晉室南渡,長居京口(今江蘇鎮江市),《宋書》說他是漢高祖劉邦的弟弟楚王劉交第二十一世孫。《魏書》則猜測其祖先可能姓項。劉裕於興寧元年三月壬寅日(363年4月16日)出生,其時家境貧苦,母親更因分娩後疾病去世,父親劉翹無力請乳母給劉裕哺乳,一度打算拋棄他,只因劉懷敬之母伸出援手,養育劉裕,才得以活下來。劉裕早年曾以賣鞋为生,但卻又常賭博樗蒲,傾盡家財,遭鄉里賤視,亦因不修品行而不為當世人們所賞識。不过,刘裕才能出眾,且有大志,當時出身琅琊王氏的王謐就十分敬重他,更曾向他說:「你應當會成為一代英雄。」。

劉裕及後從軍,最初就任北府軍將領、冠軍將軍孫無終的司馬。隆安三年(399年),孫恩起兵反抗晉朝,自海島攻下會稽,三吳各郡皆響應他,孫恩之亂由而爆發。另一北府將領、前將軍劉牢之率軍鎮壓,當時他就請了劉裕參府軍事。

當時劉裕奉命率數十人去偵察敵軍,卻遇上數千人的敵軍並發生戰鬥,雖然所帶的人大多戰死了,但劉裕仍揮動長刀抵抗,殺傷多人。劉牢之子劉敬宣派兵搜尋劉裕,見劉裕獨力抵抗,都讚歎劉裕的能力,並率軍進攻,俘殺一千多人。不久諸軍擊敗孫恩各軍,又攻下會稽郡治所山陰(今浙江紹興市),令孫恩退回海島。

次年(400年)五月,孫恩再襲會稽,殺害駐鎮會稽的謝琰,至十一月時劉牢之率軍前往才擊退孫恩。劉牢之及後命劉裕守句章(今浙江寧波市)。當時句章城小兵弱,而劉裕就常做好作戰準備。翌年(401年)二月孫恩就率眾自浹口(今浙江鎮海)進攻句章,而劉裕就身先士卒,每戰都摧其鋒銳,致令孫恩無法攻下句章,反為劉牢之所敗。三月,孫恩轉戰海鹽(今浙江海鹽縣),劉裕跟隨其進攻方向,於海鹽築城抵抗,又大敗來攻的孫恩。

孫恩後循海路至丹徒(今江蘇鎮江市丹徒區),劉裕率不足千人的部隊趕路,與孫恩同時趕至。當時劉裕軍隊疲累,丹徒守軍亦無鬥志,但面對孫恩來襲,劉裕仍能率眾大敗對方,逼其狼狽登船撤離岸上。孫恩不久轉屯郁洲(今江蘇灌雲縣東北),朝廷以劉裕為建武將軍、下邳太守,討伐孫恩,多次交戰後大破對方,令其勢力轉弱而南撤。劉裕接著追擊,又再敗孫恩,令其再度逃到海島。次年孫恩就被消滅。

元興元年(402年),驃騎大將軍司馬元顯下令討伐荊州刺史桓玄,並以劉牢之為前鋒。桓玄率軍兵臨建康時,劉裕請求進攻,但劉牢之不肯,反而想叛歸桓玄。劉裕當時與何無忌極力諫止但都無效,劉牢之終向桓玄請降,桓玄亦順利消滅司馬元顯的勢力,掌握朝政。

事後桓玄調劉牢之為會稽內史以削其軍權,劉牢之圖據廣陵(今江蘇揚州市)叛桓玄,但劉裕認為人心已去,事必不成而拒絕與劉牢之合作,最終劉牢之因失去僚屬的支持而自殺。桓脩後以劉裕為其中兵參軍,並於同年參與討伐統領孫恩餘黨的盧循、徐道覆。當時桓玄诛殺了多名北府舊將,但劉裕仍領兵討伐盧循部眾,更獲加任彭城內史。及至桓玄篡位後次年(404年),劉裕跟從桓脩入朝建康,桓玄亦十分賞識他,出遊都殷勤接引,賞賜亦甚為豐厚。當時桓玄皇后劉氏就勸桓玄除去劉裕,但桓玄仍圖借助劉裕攻略中原,拒絕加害。

早在劉牢之失敗之時,劉裕就向何無忌說:「桓玄若果守著臣子的忠節,就應與你輔助他;否則,就要與你對付他。」及至劉裕入朝後回到京口,就與何無忌、劉毅、孟昶、諸葛長民、王元德等人合謀舉兵討伐桓玄,並準備在京口、廣陵、歷陽(今安徽和縣)及建康(今江蘇南京市)四地同時起兵。元興三年二月乙卯(404年3月24日),劉裕託詞遊獵而外出募眾,終得百多人。次日(3月25日)早上起兵,何無忌殺桓脩,當時桓脩司馬刁弘率眾前來,劉裕則假稱江州刺史郭昶之已在尋陽(今江西九江市)迎晉安帝復位,桓玄更已被處決,自己只是奉密詔誅除桓氏叛黨。刁弘信以為真,劉裕不久就誅除刁弘,控制了京口。同時孟昶等亦成功控制了廣陵,並率眾至京口與劉裕會合,劉裕獲眾人推舉為盟主,總督徐州事,並於次日(3月26日)起兵進攻建康。

桓玄先派吳甫之及皇甫敷抵抗劉裕,劉裕先於江乘(今江蘇句容北)殺吳甫之,至江乘以南的羅落橋時奮力作戰,又殺皇甫敷,繼續進攻。三月己未日(3月28日),劉裕進攻覆舟山,並命弱兵登山,持著旗幟分道而行,營造四周皆有士兵,數量很多的假象;而又因桓玄守軍大多是北府軍出身,面對劉裕都沒有鬥志,劉裕於是與諸軍進攻,順利以火攻擊潰桓玄守軍,而桓玄亦棄城西逃。

劉裕於三月壬戌日(3月31日)獲王謐等人推舉為使持節、都督揚兗豫青冀幽并八州諸軍事、鎮軍將軍,徐州刺史。不久,劉裕奉武陵王司馬遵承制總百官行事。劉裕在進建康城後派諸將追擊桓玄,終於當年六月誅殺了桓玄,並讓在江陵(今湖北江陵)的晉安帝復位。然而,桓氏勢力仍在荊州盤據,並反攻江陵,直至義熙元年(405年)才再收復江陵,驅逐當地桓氏勢力,並自江陵迎晉安帝回建康,不久劉裕就還鎮丹徒。

義熙二年(406年),劉裕因功受封為豫章郡公。義熙四年正月(407年),因上年年末揚州刺史、錄尚書事王謐去世,劉裕聽從劉穆之的勸言入朝議繼任人選,終獲授侍中、車騎將軍、開府儀同三司、揚州刺史、錄尚書事、徐兗二州刺史,入掌朝政大權。

盧循、徐道覆趁劉裕領兵在外,於義熙六年(410年)起兵,進攻江州。當時朝廷急徵劉裕,而當時劉裕剛滅南燕,收到詔書就撤還建康。劉裕至山陽(今江蘇淮安市)時知江州刺史何無忌已戰死,於是加速回防建康,並於四月趕至。五月,豫州刺史劉毅大敗給盧循,盧循繼續東下,而劉裕當時就招募兵眾,修治石頭城並於當地聚兵。不過,由於劉裕急急南返,士卒多有傷病,而建康兵力亦不過千人,面對有十多萬人的盧循大軍顯得實力懸殊,然而劉裕堅決不肯接受諸葛長民及孟昶奉安帝北歸廣陵避敵的建議,決意死戰。

盧循軍到後停駐蔡洲(今江蘇江寧縣西南江中),劉裕就以木柵阻斷石頭城及淮口,修治越城(今江寧縣南)並建查浦、藥園、廷尉三個堡壘,分兵戍守以禦盧循,盧循曾分疑兵進攻白石及查浦,自率大軍進攻丹陽郡,但都沒有取勝,而且在各縣中都無法搶掠到物資,被逼於七月退兵江州。同年十月,劉裕率劉藩、檀韶、劉敬宣等人進攻盧循,並於十二月以火攻擊敗盧循船隊。盧循敗後試圖於左里(今鄱陽湖口)擋住劉裕,但劉裕率軍奮戰,盧循軍無法阻擋而大敗,盧循因而南逃廣州。劉裕早於盧循撤出蔡洲後就已派了孫處及沈田子經海路攻佔了盧循根據地番禺,盧循敗逃廣州後於義熙七年(411年)又於廣州敗於沈田子等人,終在交州被刺史杜慧度所殺。

劉裕於義熙七年(411年)班師回到建康,並受太尉、中書監職位。次年(412年)四月,劉裕以劉毅為荊州刺史。劉毅自以能力不亞於劉裕,甚不服在劉裕以下,他亦得朝中有名望人士歸心交結,故此遷鎮荊州時就將大部分豫州府屬及江州萬多人的軍隊都帶去荊州,到任後又重新調度荊州郡縣首長,更以患病為由請堂弟劉藩去做他副手。劉裕知其有異心,於是假意答允其請求,但就乘劉藩自兗州治所廣陵入朝時就指稱他與謝混圖謀不軌,將二人賜死,接著就率軍自建康出發討伐劉毅。劉裕派王鎮惡為前鋒,沿路聲言是劉藩前來去騙倒各戍和民眾,直至江陵城外五六里時才被發現,但已趕及在劉毅關閉城門前率兵入城,並在城內作戰。城中民眾知劉裕在率軍前來都十分驚恐,劉毅不敵王鎮惡,唯有出逃,並於牛牧寺自殺。劉裕隨後來到江陵,誅殺了劉毅親信郗僧施,消滅了劉毅勢力。

劉裕征劉毅時以諸葛長民守留府事,但諸葛長民見劉毅敗死,自己亦深感不安,更意圖作亂,劉裕回建康時故意拖慢進度,讓等待迎接他的諸葛長民及其他官員接連幾日都等不到劉裕。劉裕卻乘輕舟快快進城,進入了官邸東府。諸葛長民知道劉裕突然回來了,於是拜訪,劉裕暗中命壯士丁旿埋伏,故意和諸葛長民閒話家常,乘諸葛長民警覺性下降時命丁旿將其殺死,接著又誅殺了長民弟諸葛黎民等人。劉裕接著就加鎮西將軍、豫州刺史,接掌諸葛長民的原職。清除京口武將中的異己勢力之後,劉裕在412年底發動晉滅譙蜀之戰,隔年(413年)西征主將朱齡石成功滅譙蜀,使劉裕加授羽葆、鼓吹及班劍二十人。

412年征討劉毅時,劉裕以晉宗室司馬休之接任荊州刺史。司馬休之頗得當地人心,而劉裕就懷疑他有異心;在義熙十年(414年),其子司馬文思又在建康招集輕俠,令劉裕十分厭惡,司馬文思終因被揭發殺害官吏而被捕,劉裕誅殺其黨眾而免文思死,反送他到司馬休之那裏,要他親自教誨他,實質就是要司馬休之將其處死。然而,司馬休之並沒有殺文思,只是上表廢掉文思的譙王爵位,並寫信向劉裕道歉。這舉動令劉裕對其大感不滿,立刻就命江州刺史孟懷玉戒備。

義熙十一年(415年),劉裕收殺司馬休之在建康的次子司馬文寶及侄兒司馬文祖,並出兵討伐司馬休之,自加黃鉞,領荊州刺史。司馬休之則上表劉裕罪狀,派兵抵抗;當時雍州刺史魯宗之自感不被劉裕所容,故與司馬休之聯結。劉裕前鋒徐逵之初戰敗於魯軌,眾將除蒯恩外皆戰死,劉裕大怒。然而當他到時,魯軌及司馬文思率軍在懸岸峭壁上列陣,令劉裕難以登岸,胡藩當時就冒險攀登,司馬文思等竟不能抵擋,劉裕就乘對方後撤的機會登岸進攻,終擊潰司馬休之的軍隊,攻下江陵,司馬休之及魯宗之北投後秦。

劉裕在消滅司馬休之後獲劍履上殿、入朝不趨、贊拜不名的崇禮,次年(416年)正月更獲加領平北將軍、兗州刺史、都督南秦州諸軍事,至此其一人已經都督徐州、南徐、豫、南豫、兗、南兗、青、冀、幽、并、司、郢、荊、江、湘、雍、梁、益、寧、交、廣、南秦共二十二州。

在魏晋十六国时期,东晋虽偏安江南,却始终没有放弃收复中原等漢地北部地區,所以屡次发动北伐战争。后秦、南燕出于内乱而败亡;公元397年北魏军攻下中山,后燕官吏兵投降两万余人,后燕的疆域被切断为南燕和北燕二部,405年南燕又发生政变;416年姚兴卒,后秦内乱不断,镇守蒲坂和岭北的姚懿、姚恢先后率叛军进攻长安。刘裕趁后秦、南燕内乱之际,乘机出兵,并一举攻灭。这次收复中原的版图之多,是东晋历次北伐中最成功、影响最深远的一次,也是以前的多次北伐都无法与之比拟的。

義熙五年(409年),南燕皇帝慕容超因為缺乏太樂伎人,派兵侵略淮北的宿豫城(今江蘇宿遷縣東南),大掠民眾北歸。及後又派兵進攻淮北,擄去陽平和濟南兩郡太守,俘擄千多家。

劉裕因此上表北伐,並於同年四月出發。當時劉裕認為燕軍短視,不會據守大峴山(今山東臨朐縣東南)天險並堅壁清野,只會進據臨朐(今山東臨朐縣),退守都城廣固(今山東青州市),而當時南燕軍的行動亦果然如此。慕容超知晉軍過了大峴山就親自率軍到臨朐,劉裕前鋒先於巨蔑水擊退燕軍,接著攻臨朐城。晉燕兩軍於臨朐以南作戰,胡藩獻計出奇兵突襲臨朐城內,最終成功攻克,慕容超倉皇自城中逃至城南大軍那裏,而此時劉裕命軍隊進攻,大敗燕軍並斬殺其十多名大將,慕容超於是逃回廣固。劉裕接著乘勝追擊至廣固,並成功攻克其外城。慕容超據守小城抵抗,劉裕就築圍圍困廣固。劉裕圍城戰爭一直維持至次年二月才攻下廣固,並俘殺慕容超,滅了南燕。

劉裕在當日平滅南燕後就已經有攻略後秦的打算,只因盧循作亂才逼令他班師建康,而劉裕在消滅了國內主要異己後,又再重拾昔日計劃。劉裕在獲加督至二十二州後月餘,又獲加中外大都督,解徐兗二州刺史而改領司、豫二州刺史,並奉琅琊王司馬德文北伐,打著晉朝皇室旗號安撫北方漢人。至五月又加北雍州刺史。終在八月,劉裕正式自建康出兵,進軍至彭城(今江蘇徐州市)後又加北徐州刺史。十月,劉裕所派的檀道濟等進攻洛陽(今河南洛陽市),守將姚洸出降,成功收復洛陽。

次年(417年)正月,劉裕自彭城率水軍西進,進入黃河。劉裕一直進軍至潼關,命王鎮惡率軍經渭河進攻後秦都城長安(今陝西西安市),王鎮惡於渭橋大敗姚丕,姚泓所率的軍隊亦因遭姚丕敗兵踐踏而潰亂,最終姚泓於八月出降,後秦滅亡。劉裕於次月到達長安,大賞將士並誅殺歸降的後秦宗室姚璞、姚讚及其百多名宗族。

同年十一月,留守建康的劉穆之去世,當時劉裕還想以長安做基地進攻西北北涼等國,只是諸將都思鄉,大多都不想留下;劉裕向來倚重的劉穆之去世更令他覺得建康根本之地已空虛無靠,於是下了決心班師東歸。劉裕於是留了當時僅得十一歲的次子劉義真鎮守長安,並留下王鎮惡、王脩、沈田子、毛德祖等將領協助他。當地人民知道劉裕要走都向他哭訴,希望他回心轉意,然而劉裕去意已決,還是在當年十二月出發離開。

然而,劉裕走後次年,諸將內訌,沈田子殺王鎮惡,王脩殺沈田子,而劉義真又在諸將唆擺下命人殺害王脩,於是關中大亂,夏國乘機進攻關中,劉裕唯有召還劉義真,派朱齡石等代鎮長安,更指令若關中不能守下去就可放棄。最終晉軍還是撤出長安,關中地區遭夏國佔領。

義熙十四年(418年),劉裕接受相國、總百揆、揚州牧的官職,以十郡建「宋國」,受封為宋公,並接受九錫的特殊禮待。同年,劉裕命令中書侍郎王韶之與晉安帝左右侍從密謀以毒酒毒殺安帝,王韶之於是乘司馬德文因病出宮的機會下手,縊殺安帝。當時劉裕因為相信預言書說:「昌明(晉孝武帝)之後尚有二帝」,於是聲稱依照遺詔,立了司馬德文為皇帝,即晉恭帝。

元熙元年(419年),劉裕進爵為宋王,又加十郡增益宋國,令宋國包括了二十郡。年末劉裕又獲加皇帝規格的的十二旒冕、天子旌旗等一系列特殊禮待。元熙二年(420年),劉裕入輔,傅亮知劉裕想要晉恭帝禪讓帝位予他但難於啓齒,於是代為向恭帝暗示,恭帝於是在六月禪讓帝位給劉裕,東晉滅亡,劉裕即位為帝,改國號為「宋」,改元永初。劉裕在稱帝之後,為了斬草除根,還殺掉了恭帝。此行為可謂劉裕一生中一個汙點,因為其行為開啟了前朝遜位之主不得善終之先(新朝王莽之於西漢孺子嬰、曹魏文帝曹丕之於東漢獻帝劉協、西晉武帝司馬炎之於曹魏元帝曹奐,都沒有加害前朝末主),至此,南朝末主除了陳後主陳叔寶其亡國非遭逢篡位外,全都俱被新立的政權所殺。

永初三年(422年),劉裕患病,五月病重時遺命司空徐羨之、尚書僕射傅亮、領軍將軍謝晦及護軍將軍檀道濟四人為顧命大臣,輔助太子劉義符。劉裕於五月癸亥日(6月26日)去世,享年六十歲。廟號高祖,謚為武皇帝,葬在初寧陵(今江蘇南京紫金山)。

刘裕自他繼王謐以錄尚書事掌權起直至其去世,一直掌握著東晉以及南朝宋的軍政大權,曾对当时积弊已久的政治、经济状况有所整顿。

門閥士族兼併土地的行為令百姓流離失所,無法保護其產業,劉裕則一改東晉以來對這種事寬松的規管,重訂規管並展示公眾,大大抑制了門閥豪強的兼并行為。及至會稽虞氏的虞亮藏匿一千多名脫離戶籍逃亡的人,劉裕將之處死,連時任會稽內史的司馬休之也遭免官。另劉裕又針對當時門閥豪強私佔山澤,人民去砍柴釣魚都受限制的問題,禁止豪強這種行為。刁氏一族向來富有,奴客亦多,在其宗族桓玄敗死後被誅滅時,劉裕亦將刁家的資產都分發給人民,讓人們按己力取用,賑濟當時處於饑荒及戰亂中的人民。劉裕亦於義熙九年(413年)將臨沂、湖熟原屬皇后所有,用來資助其化妝品開銷的田地分配給窮人。如此削奪了世族以及皇室的私產,用來資濟人民。即位為帝後更派大使巡行四方,舉善旌賢,訪問民間疾苦。

劉裕選才用人不重門第而重其才能,故對於寒門出身的劉穆之一直予以重任,在收復建康後讓他主持政局,大改官場之風,及至在劉裕領兵在外時更讓其主掌中樞重任。劉裕在晉時見州郡推薦的很多秀才、孝廉水平都不合要求,於是上請申明舊制,以策試考核他們。至登位後更曾到延賢堂為各秀才、孝廉作策試。而曾與劉裕起兵討伐桓玄的魏順之在盧循之亂時因為不敢救援部將謝寶,反倒退卻;魏順之雖為功臣,亦是魏詠之的弟弟,但劉裕大怒之下仍將其處死,此舉亦震懾其他桓玄之役中的功臣,都聽籨其命令。

劉裕於義熙九年(413年)再度實行土斷,各地人民依界土斷,只有僑居於晉陵的徐、兗、青三州人民不受影響,而當時很多僑郡僑縣都在這次土斷中被裁撤,重新整理了全國戶籍,便利於統計藏匿人口及增加賦稅收入。永初元年(420年),劉裕更下令所有逃避戶籍的人只要在限期內自首就能獲赦,並免去他們兩年的租賦,但凡有黃籍或證明文件的人都可恢復其原籍,再次減少國內藏匿人口。

刘裕消滅劉毅後,下令嚴禁荊、江二州地方官吏滥征租税、徭役,规定租税、徭役,都以现存户口为准。凡是州、郡、县的官吏利用官府之名,占据屯田、园地獲利的,皆一律废除。劉裕即位後,下令凡宫府需要的物资都要到市場採購,照价给钱,不得向人民征调。又下令官員不可徵去人民車牛,亦不能以官威逼迫人民獻出車牛,另亦將繁多的交易稅項作出減省,便利市場商業交易。

刘裕对政治、经济的整顿,进一步打击了門閥士族的势力,改善了政治和社会状况,对劳动人民的痛苦亦有所减轻。

而劉裕在建立南朝宋後亦削弱强藩,集权中央,於是限制了荆州州府置將和官吏數額,前者不可多於二千人,後者亦不可多於一萬人;另其他州府置將及官吏數亦不分別不得多於五百人及五千人。为防止权臣擁兵,他特別下诏命不得再別置軍府,宰相領揚州刺史的話可置一千兵。而凡大臣外任要職要需軍隊防衞,或要出兵討伐,一律配以朝廷军队,事情完結後軍隊都需交回朝廷。另劉裕為防外戚亂政,下令有幼主的話都委事宰相,不需太后臨朝。

劉裕高七尺六寸,氣質奇特。

劉裕行軍法令嚴明,軍隊軍容齊整,絕不擾民。而他在軍事行動的分析亦常常精準無誤,例如伐南燕時料定燕軍不會據守大峴山抵抗,而慕容德果然否決公孫五樓守大峴的計劃。命令朱齡石征伐西蜀時亦預計敵方會猜測晉軍循內水進攻,必以重兵守涪城,於是指令要從外水進攻,改派疑兵引誘涪城重兵,以圖直取成都。最終亦正如劉裕預計那樣,朱齡石成功繞過涪城重兵,直取成都,獲得勝利。

在生活上刘裕崇节俭,不爱珍宝,不喜豪华,宫中嫔妃也少。宁州地方官曾经奉献琥珀枕,是无价之宝,他不稀罕。在出征後秦时,有人说琥珀能够治疗伤口,他就命人将它砸碎,分给将领作为治伤药。平定关中後,他得到了美女姚氏(後秦天王姚興的姪女),十分宠爱。臣下谢晦劝谏他不要因女色而荒废政务,他当晚就将姚氏送出宫去。後來劉裕進封宋公,東西堂將要放置以金塗釘釘製的局腳牀,但劉裕以節為由而改用鐵釘釘製的直腳牀。又一次廣州進貢一匹筒細布,劉裕因其過於精巧瑰麗,製作必定擾民,故此下令彈劾獻布那郡的太守,將布匹送還並下令禁止再製作這種布。劉裕因患有熱病,常常要有冰冷的物件去降溫,於是有人就獻上石床。劉裕躺上冰冷的石床,感到十分舒服,但又感木牀已經很耗人力,大石頭要磨成牀就更甚了,於是下令將石床砸毀。劉裕更加下令將自己昔日的農具收起,留給後人。其子宋文帝一次看見,得知內情後大感慚愧。而其孫宋孝武帝拆毀劉裕生前的臥室而建玉燭殿,發現牀頭上有土帳,牆上掛著葛布製的燈籠及麻製蠅拂,袁顗稱許劉裕有儉素之德,但孝武帝沒有說甚麼,只說:「老農夫有這些東西,已經過於富裕了。」

劉裕不擅文才,故劉毅曾在宴會中特地賦詩:「六國多雄士,正始出風流」特意展示其文學造詣勝過劉裕。劉裕書法亦差,曾被劉穆之規勸,並在其指示下改寫大字。

劉裕不信神祇,登位後更曾下令將民間廟宇拆毀,只有先賢以及以有勳德的人的廟祠才得豁免。劉裕去世前患病,群臣上請劉裕祈求神祇庇佑,但劉裕不接受,只派了謝方明去太廟告知祖先。

昔日劉裕曾欠下刁逵三萬錢,無力償還,被刁逵抓著,王謐則去見刁逵,並替劉裕償還欠款,劉裕才得釋放;而當時劉裕既無名聲亦貧賤,不被其他具名望人士看重,唯有王謐去與他結交。王謐後在桓玄篡位時奉天子玉璽及冊文給桓玄,在桓楚任司徒,更獲封公爵,甚為禮侍。劉裕義軍攻下建康後,王謐仍任司徒,領揚州刺史、錄尚書事,但王謐既因在桓楚任高職,甚得寵待,故很不安心,最終出奔。然而劉裕沒有向王謐問罪,並念及昔日恩情,請武陵王司馬遵追還王謐,並讓其官復原職。而昔日為其債主的刁逵,在桓楚任豫州刺史,並為桓玄收捕起義失敗的諸葛長民。他在桓玄敗後出奔,終被部下抓住,可是刁氏一族接著卻遭誅殺,只有刁聘獲赦,然而不久刁聘亦因謀反而被誅,令刁氏族滅。

傳說劉裕出生時有神光照亮室內,當晚還降甘露。

劉裕曾到京口竹林寺,並獨自躺臥在寺內講堂內。一眾僧人竟看見他上面有五色龍形物體出現,大感吃驚並告知劉裕,劉裕則十分高興起說:「僧人是不會說謊的。」

有言曲阿、丹徒有天子之氣,而劉翹的墓就在丹徒,當時一個叫孔恭的人擅長占卜墓穴吉凶,劉裕一次就在父親墓前問孔恭,孔恭就言那是不平凡的墓地。劉裕聽後更為自負。更劉裕又覺得身邊總有兩條小龍,連旁人也曾看見過,至劉裕名聲漸高時,小龍也變大了。

傳說劉裕一次去伐木砍柴,射傷了一條大蛇。翌日再去時卻聽見有杵臼搗藥的聲音,發現有幾個小童正在製藥。劉裕於是問他們為何要製藥,小童則答:「我們的王被劉寄奴射傷,所以要製藥醫治。」劉裕追問:「你們的王既有神通,為何不殺了他?」小童卻答:「劉寄奴是王者,不可以殺。」劉裕喝跑了小童,拿走他們的藥。後來一次到下邳遊玩,一個僧人向他說:「江南地區會有動亂,令此地安定的人就是你呀。」僧人又給了劉裕一些傷藥,接著就失去了蹤影。劉裕手部有傷患,一直都無法痊癒,但用了僧人的藥一次後卻痊癒了。劉裕於是視剩餘的的傷藥及當日在小童那裏的藥為珍寶,每次受了傷,用那些藥都能醫好。

盧循譏諷劉裕智窮,劉裕則以續命湯反譏盧循命不長。典出藝文類聚·卷八十七:果部下:益智。

劉裕是兩晉南北朝時期最卓越的軍事統帥之一。劉裕在不到二十年時間裡,對內平息戰亂,先後平定孫恩、盧循的叛亂,消滅桓玄、劉毅等軍事集團;對外致力於北伐,取譙蜀、伐南燕、滅後秦,從一名普通軍人成長為名垂青史的軍事統帥,取得世人矚目的成就,更徹底改變晋朝政權對征服漢地北部的塞外各民族一直處於被動的局面。北魏謀臣崔浩在評價劉裕時說:「劉裕奮起寒微,不階尺土,討滅桓玄,興復晉室,北擒慕容超,南梟盧循,所向無前,非其才之過人,安能如是乎!」崔浩亦說:「劉裕之平禍亂,司馬德宗之曹操也。」何去非在《備論》中也說:「宋武帝以英特之姿,攘袂而起,平靈寶于舊楚,定劉毅于荊豫,滅南燕于二齊,克譙縱於庸蜀,殄盧循於交廣,西執姚泓而滅後秦,蓋舉無遺策而天下憚服矣。北方之寇,獨關東之拓跋,隴北之赫連耳。方其入關,魏人雖強,不敢南指西顧以議其後。」《南史》評論說:「宋武地非齊、晉,眾無一旅,曾不浹旬,夷凶翦暴,誅內清外,功格上下。若夫樂推所歸,謳歌所集,校之魏、晉,可謂收其實矣。」

劉裕的軍事生涯,指揮無數次作戰,最大特點是以少勝多,而且作戰中常身先士卒,所以能夠贏得廣大將士尊敬。劉裕北伐是中國戰爭史上最成功的北伐之一,成就不但遠較以前東晉各次北伐高,中國歷史上僅次於朱元璋,所以辛棄疾用「金戈鐵馬,氣吞萬里如虎」的詩句來形容劉裕北伐時的氣勢。司马光叙述刘裕北伐成功后匆忙东归,关中复失时,大发感叹:「惜乎,百年之寇,千里之土,得之艰难,失之造次,使丰、鄗之都复输寇手。荀子曰:『兼并易能也,坚凝之难。』信哉。」而王夫之直指劉裕是為了急急篡位而放棄關中,說:「刘裕灭姚秦,欲留长安经略西北,不果而归,而中原遂终于沦没。史称将佐思归,裕之饰说也。王、沈、毛、傅之獨留,豈繄不有思歸之念乎?西征之士,一歲而已,非久役也。新破人國,子女玉帛足系其心,梟雄者豈必故土之安乎?固知欲留經略者,裕之初志,而造次東歸者,裕之轉念也。夫裕欲归而急于篡,固其情已。」但王夫之仍然肯定了「然使裕據關中,撫雒陽,捍拓拔嗣而營河北,拒屈丐而固秦雍,平沮渠蒙遜而收隴右,勛愈大,威愈張,晉之天下其將安往?曹丕在鄴,而漢獻遙奉以璽綬,奚必反建康以面受之於晉廷乎?蓋裕之北伐,非徒示威以逼主攘夺,而无志于中原者,青泥既败,长安失守,登高北望,慨然流涕,志欲再举,止之者謝晦、鄭鮮之也。蓋當日之貪佐命以弋利祿者,既無遠志,抑無定情,裕欲孤行其志而不得,則急遽以行篡弒,裕之初心亦絀矣。」他还稱刘裕「為功于天下,烈于曹操,而其植人才以贊成其大計,不如操遠矣。操方舉事據兗州,他務未遑,而亟于用人;逮其後而丕與叡猶多得剛直明敏之才,以匡其闕失。」显然也包括了对刘裕北伐成功的肯定。「裕起自寒微,以敢戰立功名,而雄俠自喜,與士大夫之臭味不親,故胡藩言:一談一詠,搢紳之士輻湊歸之、不如劉毅。當時在廷之士,無有為裕心腹者,孤恃一機巧汰縱之劉穆之,而又死矣;傅亮、徐羡之、謝晦,皆輕躁而無定情者也。孤危遠處于外,求以制朝廷而遙授以天下也,既不可得,且有反面相距之憂,此裕所以汔濟濡尾而僅以偏安艸竊終也。當代無才,而裕又無馭才之道也。身殂而弒奪興,況望其能相佐以成底定之功哉?曹操之所以得志于天下,而待其子始篡者,得人故也。豈徒奸雄為然乎?聖人以仁義取天下,亦視其人而已矣。」

呂思勉則認為,劉裕急急篡位的說法只是史家附會王買德的話,說:「宋武代晉,在當日,業已勢如振槁,即無關、洛之績,豈慮無成?苟其急於圖,篡平司馬休之後,逕篡可矣,何必多伐秦一舉?武帝之於異己,雖云肆意翦除,亦特其庸中佼佼者耳,反之子必尚多。劉穆之死,後路無所付託,設有竊發,得不更詒大局之憂?欲攘外者必先安內,則武帝之南歸,亦不得訾其專為私計心也。義真雖云年少,留西之精兵良將,不為不多。王鎮惡之死,在正月十四日(應為十五),而勃勃之圖長安,仍歷三時而後克,可見兵力實非不足。長安之陷,其關鍵,全在王脩之死。義真之信讒,庸非始料所及,此尤不容苟責者也。」

劉裕在對待刁逵及王謐截然不同的態度,招來了不少批評,南朝梁湘東世子蕭方等就曾言:「夫蛟龍潛伏,魚蝦褻之。是以漢高赦雍齒,魏武赦梁鵠,安可以布衣之嫌而成萬乘之隙也!今王謐為公,刁逵亡族,醻恩報怨,何其狹哉。」裴子野亦言:「刁逵,玄之爪牙;王謐,楚之上相,論逆則王重,定罪則逵輕。稚遠以舊德錄萬機,長民以宿憾夷七族,以為晉政偏頗甚矣!且神龍伏於罟網,漁者安知其靈化;霸王匿於人庶,庸夫何以悟其英雄!苟在不悟則驕之者,眾可勝怨乎?是知宋高祖之非弘亮也,同盟多貮宜乎哉!」

劉裕攻下南燕都城廣固後,因為怨恨城池久久不下,故此意圖將城內人民全部坑殺,並將其妻女賞賜給將士,只因韓範勸止才不實行,但仍然盡殺南燕王公共三千人,並抄沒萬餘人。此意圖亦招來司馬光批評:「晉自濟江以來,威靈不競,戎狄橫騖,虎噬中原。劉裕始以王師翦平東夏,不於此際旌禮賢俊,忍撫疲民,宣愷悌之風,滌殘穢之政,使群士嚮風,遺黎企踵,而更恣行屠戮以快忿心;迹其施設,曾苻、姚之不如,宜其不能蕩壹四海,成美大之業,豈非雖有智勇而無仁義使之然哉!」

王夫之在《读通鉴论》评论刘裕:“宋武兴,东灭慕容超,西灭姚泓,拓跋嗣、赫连勃勃敛迹而穴处。自刘渊称乱以来,祖逖、庾翼、桓温、谢安经营百年而无能及此。后乎此者,二萧、陈氏无尺土之展,而浸以削亡。然则永嘉以降,仅延中国生人之气者,唯刘氏耳。舉晉人坐失之中原,責宋以不蕩平,沒其撻伐之功而黜之,亦大不平矣。君天下者,道也,非勢也。如以勢而已矣,則東周之季,荊、吳、徐、越割土稱王,遂將黜周以與之一等;而嬴政統一六寓,賢于五帝、三王也遠矣。拓拔氏安得抗宋而與並肩哉?唐臣隋矣,宋臣周矣,其樂推以為正者,一天下爾。以義則假禪之名,以篡而與劉宋奚擇焉?中原喪于司馬氏之手,且愛其如綫之緒以存之;徒不念中華冠帶之區,而忍割南北為華、夷之界乎?半以委匪類而使為君,顧抑撻伐有功之主以不與唐、宋等倫哉?汉之后,唐之前,唯宋室犹可以为中国主也。”


\subsubsection{永初}

\begin{longtable}{|>{\centering\scriptsize}m{2em}|>{\centering\scriptsize}m{1.3em}|>{\centering}m{8.8em}|}
  % \caption{秦王政}\
  \toprule
  \SimHei \normalsize 年数 & \SimHei \scriptsize 公元 & \SimHei 大事件 \tabularnewline
  % \midrule
  \endfirsthead
  \toprule
  \SimHei \normalsize 年数 & \SimHei \scriptsize 公元 & \SimHei 大事件 \tabularnewline
  \midrule
  \endhead
  \midrule
  元年 & 420 & \tabularnewline\hline
  二年 & 421 & \tabularnewline\hline
  三年 & 422 & \tabularnewline
  \bottomrule
\end{longtable}


%%% Local Variables:
%%% mode: latex
%%% TeX-engine: xetex
%%% TeX-master: "../../Main"
%%% End:

%% -*- coding: utf-8 -*-
%% Time-stamp: <Chen Wang: 2021-11-01 15:03:14>

\subsection{少帝刘义符\tiny(422-424)}

\subsubsection{生平}

刘义符(406年-424年8月4日),小字车兵,彭城綏輿里(今江蘇省銅山縣)人。中国南北朝時期宋朝的第二位皇帝,宋武帝刘裕长子,母親是夫人張闕。永初三年(422年)即位為帝,但兩年後就因居喪行為不當而遭顧命大臣徐羨之、傅亮等廢黜,不久被殺。

刘义符於義熙二年(406年)出生,其時劉裕已經四十四歲,他對得到這個遲來的兒子亦十分高興,後來更立其為豫章公世子。義熙十二年(416年)劉裕北伐後秦時,劉義符就受任中軍將軍,監太尉留府事,留守建康留府。義熙十四年(418年),劉裕受封宋公,建宋國,又以劉義符為宋公世子。劉裕於元熙元年(419年)進封宋王,並加殊禮,劉義符成為宋王太子。至永初元年(420年)劉裕篡晉自立,劉義符亦被立為皇太子。

永初三年(422年),劉裕病重,特意召見劉義符並告誡他:「檀道濟雖然有才幹和謀略,但沒有遠大志向,絕不像他兄長檀韶那麼難駕御。徐羨之、傅亮,應該沒有異心。謝晦數次跟我出征,頗為機智,若有人懷有異心,那肯定是他了。」並以徐羨之、傅亮、謝晦及檀道濟四人為顧命大臣,又明令往後有幼主繼位都不用母后臨朝,朝事都全交給宰相。劉裕在五月癸亥日(6月26日)去世後,劉義符就於同日即位為帝。

劉義符即位後仍需守父喪,但他在期間表現無禮,常與身邊的人十分親密,遊樂無度。當時特進范泰曾寫書勸諫,但劉義符不聽。而謝晦早年見劉義符身邊的都是小人,就曾向劉裕表示劉義符不是繼承宋室的人選,徐羨之、傅亮及謝晦最終在景平二年(424年)謀廢劉義符,並召江州刺史王弘及南兗州刺史檀道濟入朝,命中書舍人邢安泰及潘盛為內應,然後讓皇太后下令廢劉義符為營陽王。那天早上,謝晦、檀道濟及徐羨之領兵自雲龍門入宮,其時潘盛已經撤去宿衞軍隊,故此無人阻攔謝晦等軍。而劉義符那時在華林園設下市場,並親身去賣物;又開了水道,前一天和身邊親近乘船高歌大叫,並遊玩到天淵池後睡在船上,到了那天早上還沒醒。士兵進宮後殺害劉義符身邊兩個侍者,更傷了其手指,接著就將其扶出東閤,沒收皇帝璽綬,終將其送到吳郡幽禁。六月癸丑日(8月4日),徐羨之命邢安泰於金昌亭弒殺劉義符,劉義符極力反抗,要逃出昌門,但遭追捕者用門閂絆倒,最終遇害,享年十九歲。

徐羨之隨後立了宜都王劉義隆為帝。至元嘉九年(432年)以劉義恭長子劉朗為南豐縣王,作為劉義符的後嗣。

\subsubsection{景平}

\begin{longtable}{|>{\centering\scriptsize}m{2em}|>{\centering\scriptsize}m{1.3em}|>{\centering}m{8.8em}|}
  % \caption{秦王政}\
  \toprule
  \SimHei \normalsize 年数 & \SimHei \scriptsize 公元 & \SimHei 大事件 \tabularnewline
  % \midrule
  \endfirsthead
  \toprule
  \SimHei \normalsize 年数 & \SimHei \scriptsize 公元 & \SimHei 大事件 \tabularnewline
  \midrule
  \endhead
  \midrule
  元年 & 423 & \tabularnewline\hline
  二年 & 424 & \tabularnewline
  \bottomrule
\end{longtable}


%%% Local Variables:
%%% mode: latex
%%% TeX-engine: xetex
%%% TeX-master: "../../Main"
%%% End:

%% -*- coding: utf-8 -*-
%% Time-stamp: <Chen Wang: 2021-11-01 15:03:23>

\subsection{文帝劉義隆\tiny(424-453)}

\subsubsection{生平}

宋文帝劉義隆(407年-453年3月16日),小字車兒,宋武帝劉裕的第三子,劉宋第三任皇帝。劉義隆生於東晉末年,南朝宋立國後,受封宜都王。宋少帝被廢後獲擁立為帝,即位後改元元嘉。劉義隆在位近三十年,在位期間,建立制度、賞罰分明、鼓勵農桑,減免賦稅力役,使得國家大治,「內清外晏,四海謐如」,此治世因其年號元嘉而稱為「元嘉之治」。劉義隆亦銳意北伐,曾先後三次發起大規模北伐戰爭圖收復北魏所佔的河南土地,然而三次皆失敗,其中發生在元嘉後期的第二次更讓魏軍南攻至江北瓜步,一度威脅建康。元嘉北伐亦對國內經濟民生造成嚴重打擊,《資治通鑑》對北伐的創傷寫道「元嘉之政,自此衰矣。」

劉義隆於東晉義熙三年(407年)生於京口(今江蘇省鎮江市)。義熙六年(410年),時值盧循之亂,盧循叛軍逼近建康,劉裕因應京口位置重要,遂命劉粹輔佐年僅四歲的劉義隆鎮守京口。義熙十一年(415年),因前年劉裕指令朱齡石成功滅亡譙蜀,收復蜀地,晉廷封劉義隆彭城縣公。義熙十三年(417年),劉裕北伐,率水軍自彭城(今江蘇省徐州市)兵向關中,令劉義隆行冠軍將軍留守,東晉朝廷加封其為使持節、監徐兗青冀四州諸軍事、徐州刺史。義熙十四年(418年),劉裕收復關中、還軍彭城,原本想讓世子劉義符出鎮荊州,遂授劉義隆為監司州豫州之淮西兗州之陳留諸軍事、前將軍、司州刺史,並命其鎮守洛陽(今河南省洛陽市),然而因張邵諫止劉裕讓世子外任,劉裕遂改義隆為都督荊益寧雍梁秦六州豫州之河南廣平揚州之義成松滋四郡諸軍事、西中郎將、荊州刺史,鎮守江陵(今湖北省荊州市)。不過,由於劉義隆年紀尚輕,州府事皆由司馬張邵處理。

永初元年(420年),宋武帝劉裕篡晉登位,封劉義隆為宜都王,食邑三千戶。不久加號鎮西將軍,並先後獲進督北秦州及湘州。

劉裕於永初三年(422年)死後,宋少帝劉義符即位,但因為居喪無禮,有多過失,在景平二年(424年)即因顧命大臣徐羨之、傅亮及謝晦為首發動的政變廢黜,將其幽禁並派人殺害。因義符無子,義符次弟劉義真應當繼位,然因為徐羨之認為他不宜為君,故在廢帝以前就先廢義真為庶人,後更派人殺害。廢帝後,侍中程道惠曾請改立武帝五子劉義恭,然而徐羨之屬意劉義隆,百官於是上表迎作為武帝三子劉義隆為皇帝。

時傅亮率行臺到江陵迎劉義隆入京。當時已時是七月中,江陵已聽聞少帝遇害的消息,劉義隆及一些官員都對來迎隊伍有所懷疑,不敢東下,但在王華、王曇首及到彥之的勸告下決定出發並在八月八日(八月丙申日,424年9月16日)到達建康,次日即位為帝,改元「元嘉」。

宋文帝自江陵東下起一直在提防徐羨之等人,即在東下行程上,隨行的荊州州府官員都嚴兵自衞,行臺百官都無法接近,中兵參軍朱容子更在行程數十日內一直抱刀在船艙外守衞。即位後又將親信王華及王曇首召進京內任官,更拒絕徐羨之讓當時暫鎮襄陽(今湖北襄陽市)的到彥之出任雍州刺史的建議,堅持要召其入京為中領軍,統領軍事。傅亮及謝晦亦試圖和王華等人交結,以圖安心。徐羨之及謝晦亦在元嘉二年(425年)上表歸政,讓劉義隆正式親政。不過,王華及孔甯子其時多次向劉義隆中傷徐羨之等人,劉義隆亦有了誅殺權臣的意圖,慮及謝晦當時以荊州刺史坐鎮荊州重地,於是託辭北伐及拜謁陵墓以修建船艦,其時朝廷行事異常,圖謀差點就泄露了。

元嘉三年(426年),劉義隆宣布徐羨之、傅亮及謝晦擅殺少帝及劉義真的罪行,要將徐羨之及傅亮治罪,並決定親征謝晦,命雍州刺史劉粹、南兗州刺史檀道濟及中領軍到彥之先行出兵。徐羨之聞訊自殺,傅亮被捕處死,謝晦則出兵反抗,但知檀道濟協助劉義隆討伐即惶恐不已,無計可施,最終檀道濟到後朝廷軍隊軍勢強盛,謝晦軍隊潰散,謝晦試圖逃走但被擒處死,遂消滅了三個權臣的勢力。

劉義隆殺徐羨之後,揚州刺史一職由司徒王弘出任,不過王弘卻一直試圖讓彭城王劉義康入朝他共掌朝政,以收斂當時琅邪王氏人物掌握朝廷要職的鋒芒。最終劉義康於元嘉六年(429年)得以司徒、錄尚書事身份和王弘共輔朝政,然而當時王弘常因患病而將政事推給義康處理,遂令義康漸得專掌朝政。元嘉九年(432年)王弘去世後,劉義隆更授義康揚州刺史,義康獨掌政事。

時劉義隆常常患病,政事其實都由劉義康處理,而且劉義康更衣不解帶去照料劉義隆,內廷和外朝事遂由義康所掌握。乃至元嘉十三年(436年),因應劉義隆病重,劉義康擔心一旦劉義隆去世,無人能駕馭功高才大的司空檀道濟,於是假作詔書,並在宋文帝的同意下收殺檀道濟一家及其親將。不過,劉義康自以皇帝是至親,率性而行,行事都不避嫌,沒有君臣之禮。其時劉義康親信劉湛等人更力圖想將義康推上帝位,趁義隆病重時稱應以長君繼位,甚至去儀曹處拿去東晉時晉康帝兄終弟及的資料,更去誣陷一些忠於國家,不和劉湛一夥的大臣。劉義隆病愈後知道這些事,即令兄弟之間生了嫌隙,最終劉義隆在元嘉十七年(440年)誅殺了劉湛等人,並應劉義康上表求退而讓他外調江州。隨後劉義隆將司徒、錄尚書事及揚州刺史分別授予江夏王劉義恭及尚書僕射殷景仁,然劉義恭鑑於義康被貶,雖然擔當實質宰相,行事小心謹慎,只奉行文書,卻得劉義隆安心,主相之爭以權力歸回劉義隆手中結束。

北魏在永初三年十月曾乘劉裕去世大舉南侵,奪取包括虎牢(今河南滎陽汜水鎮)、洛陽及滑臺(今河南滑县)等黃河以南土地,故劉義隆自即位以來便有收復黃河以南土地的志向。元嘉七年(430年)三月,劉義隆以到彥之為主帥,率領王仲德及兗州刺史竺靈秀率水軍至黃河,另遣段宏率八千精騎攻虎牢。到彥之軍一日只行軍約十里,到七月才到須昌(今山東東平縣西北),其時北魏以碻磝(今山東荏平西南)、滑臺、虎牢、及洛陽四鎮兵少,先後讓守將棄城北退,宋軍遂輕易奪回四鎮。然而到十月,北魏反攻,魏將安頡進攻洛陽金鏞城,守將杜驥因城池殘破且無糧食而棄守南撤;另一方面虎牢亦失陷。接著,魏將叔孫建及長孫道生等於十一月渡過黃河,到彥之見諸軍相繼敗陣,不理垣護之支援青州的諫言,南退至歷城(今山東濟南歷城區)後就燒船率軍直奔彭城,守須昌的竺靈秀於是也退,更在湖陸大敗給叔孫建。魏軍亦進攻滑臺,檀道濟雖然在十一月率軍北上救援,但次年正月起因叔孫建等人的干擾而無法支援滑臺,滑臺遂於二月失陷,檀道濟全軍撤返。北伐以失敗告終。不過後來王玄謨常常都進獻北伐策略,劉義隆聽後心動,曾對殷景仁說:「聽王玄謨說的話,令人也想在狼居胥山祭天呀。」。

元嘉二十七年(450年)-二月,北魏以步騎十萬南侵,並強攻不滿千兵的懸瓠(今河南汝南縣),守將陳憲苦戰力保不失,劉義隆遣臧質與劉康祖救援,逼退魏軍。當時義隆也命令徐兗二州刺史劉駿派兵進攻攻佔汝陽郡的魏軍,但所派的劉泰之軍卻慘敗予魏軍,泰之更戰死。魏軍在四月撤兵後,劉義隆即欲伐魏。他得到親信徐湛之、江湛及王玄謨支持,然沈慶之進諫:「步兵對陣騎兵向來處於劣勢,請放棄出征之事,而且當日檀道濟再戰無功而返,到彥之更是失利敗還。現在看王玄謨等人都比不上這兩位將軍,軍隊戰力也不及當時,這恐怕會再度戰敗,難以得志。」然劉義隆卻說:「我軍戰敗自有別的原因,這是因為檀道濟放任著敵人以圖鞏固自己地位,到彥之行軍中途病發。北虜恃著的就只是馬,夏天多雨水,河流暢通,只要派船進攻北方,那碻磝敵軍肯定會退走,滑臺守軍亦很易攻破。攻取了這兩城後送糧食慰問人民,那虎牢、洛陽人心自然不穩。等到冬天做好城間防守,待北虜騎兵過河,那就一網成擒。」

於是劉義隆堅持不聽沈慶之、太子劉劭及蕭思話勸阻,於當年(450年)七月下詔北伐,以青冀二州刺史蕭斌為六萬軍主帥,節下的王玄謨(先鋒)率沈慶之和申坦領主力進入黃河,更別遣其他四軍東西並進,大舉伐魏。不久北魏碻磝守軍就棄城,王玄謨遂攻滑臺,但強攻數月仍不能攻下,等到十月號稱百萬的北魏援軍渡過黃河,他才撤退,卻在追擊中大敗,死了萬多人。劉義隆見玄謨戰敗,魏軍一直深入,於是召還正在攻魏的各路軍隊,最終魏軍南攻至瓜步(今江蘇南京六合區瓜埠鎮),一度威脅渡江攻打建康,劉義隆唯有答應議和息兵。魏軍遂於次年自瓜步退軍,當時在彭城坐鎮的太尉劉義恭,認為碻磝不可守,就命一直守城的王玄謨退回歷城,碻磝遂失。此戰不但無功而還,且更被魏軍攻至長江,大肆燒殺擄掠,《資治通鑑》所謂「丁壯者即加斬截,嬰兒貫於槊上,盤舞以為戲。所過郡縣,赤地無餘,春燕歸,巢於林木」、「自是邑里蕭條,元嘉之政衰矣」。

元嘉二十九年(452年),劉義隆以北魏太武帝去世,命蕭思話督冀州刺史張永攻碻磝,可是自七月開始攻城起一直都無法攻破,至八月更被魏軍燒了攻城器具和軍營,蕭思話即使率兵增援,攻了十多日都沒法攻下,眼見兵糧不足,只有退兵。另一邊在攻虎牢的魯爽等知蕭思話退兵後亦撤走,北伐結束。

元嘉二十二年(445年),左衞將軍、太子詹事范曄與員外散騎侍郎孔熙先等人被揭發圖以劉義康造反,皆被誅殺,劉義隆亦因而廢劉義康為庶人。劉義隆第二次北伐失敗,令魏軍兵至瓜步,此時他憂心有人會借機擁被廢為庶人的劉義康作亂,遂於元嘉二十八年(451年)正月賜死劉義康。同時,太子劉劭將北伐失敗的罪責歸咎於當日一力支持並與持反對意見的沈慶之論戰的徐湛之及江湛,雖然劉義隆將責任歸於自己,但劉劭已經和二人極度不和。

後來,劉劭與始興王劉濬聽信女巫嚴道育,為了不再讓劉義隆知道他們做過的過失而責罵他們,就施以巫蠱,在含章殿前埋下代表劉義隆的玉雕人像。此事黃門慶國亦有參與,後來為了自保就報告給劉義隆知道。劉義隆知道後既驚訝又嘆惜,下令收捕另一同謀王鸚鵡,在其家中找到了劉劭和劉濬寫的數百張寫有咒詛之言的紙,又將那人像找到出來。劉義隆詰責二人,二人恐懼無言,只能一直道歉。劉義隆於是有了廢太子和賜死劉濬的打算,就與江湛、徐湛之及王僧綽商量;他想立建平王劉宏,徐湛之就支持女婿隨王劉誕,江湛就支持妹夫南平王劉鑠,可是久久都沒決定。王僧綽慮及機密可能泄露,勸劉義隆快作決定,但還是作不了決定。

元嘉三十年(453年)二月,劉義隆得知劉劭和劉濬還與嚴道育來往,決定實行廢太子和殺劉濬的計劃。劉義隆將此事告訴了劉濬生母潘淑妃,潘淑妃則告訴劉濬,劉劭再從劉濬口中得知,遂決定發動政變。二月二十日(3月15日)夜晚,劉劭召蕭斌及袁淑入宮,告知其計劃並表示翌日天亮就行事,蕭斌在劉劭威嚇下決定加入,堅拒的袁淑遂被殺。劉劭與蕭斌率軍在明早(3月16日)天亮時聲言受了敕命,帶著軍隊從萬春門入禁宮。那一晚,劉義隆又與徐湛之整夜討論事情,至劉劭軍隊攻入時蠟燭還亮著。劉劭齋帥張超之入殿,劉義隆舉起几桌抵抗,卻被砍斷五指,接著被殺,享年四十七歲。

劉劭隨後登位,並為劉義隆上諡號景皇帝,廟號中宗,並於三月二十日(三月癸巳日,4月14日),葬劉義隆於長寧陵。同年宋孝武帝劉駿起兵殺劉劭即位,改諡號文皇帝,廟號太祖。

劉義隆在消滅徐羨之等權臣後下詔派大使巡行四方,奏報地方官員的表現優劣,整頓吏治;又宣布一些年老、喪偶、年幼喪父及患重疾而生活困難者可向郡縣求助獲得支援,更廣開言路,歡迎人民進納有益意見和謀策。劉義隆亦多次去延賢堂聽審刑訟。元嘉十七年更下令開放禁止平民使用的山澤地區,又禁止徵老弱當兵的這些傷治害民的措施,要求各官依從法令行事。另在歷次天災時都會賑施或減免當年賦稅以撫慰人民。

劉義隆亦鼓勵農桑,元嘉八年即下詔命郡縣獎勵勤於耕作養蠶的農戶和教導正確農作方法,並將一些特別優秀的農戶上報。元嘉十七年又下令酌量減免農民欠下政府的「諸逋債」,後更於元嘉二十一年悉數免除元嘉十九年以前的欠「諸逋債」,又下令租借種子口糧給一些想參與農耕但物資缺乏的人,更賜布帛獎勵營治千畝田地的官民;元嘉二十一年夏季因連續下雨而出現水災,影響農業,劉義隆除了下令賑濟外,還在秋季命官員大力勵農民耕作米麥,又令開垦田地以備來年耕作,並於元嘉二十二年重新開墾湖熟的千頃廢田。

劉義隆重視文化建設,元嘉十五年(438年)召雷次宗在京城雞籠山(今南京市北極閣)開設「儒學館」講學,使儒學與玄學、文學、史學合稱「四學」;又於元嘉十九年(442年)下詔建國子學,待一眾冑子集合後於次年復立國子學,並於二十三年至國子學策試學生。不過,因北伐原因,劉義隆在在元嘉二十七(450年)年又廢止國子學。因陳壽所著《三國志》過於精簡,劉義隆便詔命裴松之為其作注,並於完成後親自御覽,讚道:「此為不朽矣!」。

劉義隆身高七尺五寸,博涉經史,亦擅長隸書。他喜好文儒,對文士亦十分禮待,或加以親任,甚至得劉義隆寛免罪過。

劉義隆生於京口,對京口亦留有特別感情,元嘉二十六年曾下詔以原本僑置於京口的州治北遷原地令當地不復當年繁華,從各州人民中招募數千戶人充實京口,並賜予田宅。又因懷念當時生活,命找尋當年在京口生活的官民並一一上報,去世者則酌情賞賜其子孫。

史載劉義隆儉約,不好奢侈,既曾在元嘉八年下詔「直存簡約,以應事實。內外可通共詳思,務令節儉」,他本人亦曾經因老舊的乘輦蓬蓋未壞和紫色輦席貴為由拒絕車庫令更換的建議。但他卻在元嘉二十三年修築北堤建玄武湖,甚至想在湖中建方丈、蓬萊及瀛洲三座仙山,惟因何尚之反對而作罷;同年他又在華林園修築景陽山,何尚之亦諫,認為應該給工人在盛暑休息一下,但義隆不肯,反稱他們常常曝曬,在盛暑烈日下工作不叫辛勞。

會稽長公主劉興弟是義隆嫡姐,義隆亦十分尊敬她,尤其怕她號哭,如就曾經帶著武敬皇后為劉裕造的納衣去哭罵義隆,終讓義隆不殺徐湛之。劉義康被奪相權,外調江州時,會稽長公主亦曾要求義隆不要加害義康,當時義隆亦答應,並將二人對飲中的那壺酒賜給義康。然而劉義隆於瓜步之戰後仍違背諾言賜死劉義康。

南齊的史家沈約評論宋文帝:「太祖幼年特秀,顧無保傅之嚴,而天授和敏之姿,自稟君人之德。及正位南面,歷年長久,綱維備舉,條禁明密,罰有恆科,爵無濫品。故能內清外晏,四海謐如也。昔漢氏東京常稱建武、永平故事,自茲厥後,亦每以元嘉爲言,斯固盛矣。授將遣帥,乖分閫之命,才謝光武,而遙制兵略,至於攻日戰時,莫不仰聽成旨。雖覆師喪旅,將非韓、白,而延寇慼境,抑此之由。及至言漏衾衽,難結商豎,雖禍生非慮,蓋亦有以而然也。嗚呼哀哉!」

蕭梁的史家裴子野評述文帝:「太祖寬肅宣惠,大臣光表,超越二昆,來應寶命,沈明內斷,不欲政由寧氏,克滅權逼,不使芒刺在躬,親臨朝事,率尊恭德,斟酌先王之典,強宣當時之宜,吏久其職,育孫長子,民樂其生,鮮陷刑辟,仁厚之化,既已播流,率土忻欣,無思不服……上亦蘊籍義文,思弘儒術,庠序建於國都,四學聞乎家巷,天子乃移蹕下輦以從之,束帛宴語以勸之,士莫不敦悅詩書,沐浴禮義,淑慎規矩,斐然向方……然值北虜方強,周、韓歲擾,金墉、虎牢,代失其禦,二十七年,偏師克復河南,橫挑強胡百萬之眾,匈奴遂跨彭、沛,航淮浦,設穹廬於瓜步……于時精兵猛將,嬰城而不敢鬥,謀臣智士,折撓而無可稱……我守既嚴,胡兵亦怠,且知大川所以限南北也,疲老而退,我追奔之師,橐弓裹足,系虜之民,流離道路,江淮以北蕭然矣。重以含章巫盅,始自二逆,弒帝合殿,史籍未聞,仲尼以為非一朝一夕之故,其所由來者漸矣,辨之不早辨也。元嘉之禍,其有以焉。」

唐朝的虞世南:「夫立人之道,曰仁曰义,仁有爱育之功,义有断割之用,宽猛相济,然後为善。文帝沈吟於废立之际,沦溺於嬖宠之间,当断不断,自贻其祸。孽由自作,岂命也哉。」

北宋的司馬光評論:「文帝勤於為治,子惠庶民,足為承平之良主;而不量其力,橫挑強胡,使師徒殲於河南,戎馬飲於江津。及其末路,狐疑不決,卒成子禍,豈非文有餘而武不足耶?」

南宋的辛棄疾於〈永遇樂·京口北固亭懷古〉一詞中諷喻文帝北伐:「元嘉草草,封狼居胥,贏得倉皇北顧。四十三年,望中猶記、烽火揚州路。可堪回首,佛貍祠下,一片神鴉社鼓!憑誰問:廉頗老矣,尚能飯否?」,暗喻當時南宋權臣韓侂冑的北伐失敗。

清代初期的王夫之評430年北伐:「元嘉之北伐也,文帝诛权奸,修内治,息民六年而用之,不可谓无其具;拓跋氏伐赫连,伐蠕蠕,击高车,兵疲于西北,备弛于东南,不可谓无其时;然而得地不守,瓦解蝟缩,兵歼甲弃,并淮右之地而失之,何也?将非其人也。到彦之、萧思话大溃于青、徐,(南宋孝宗之)邵弘渊、李显忠大溃于符离,一也,皆将非其人,以卒与敌者也。文帝、孝宗皆图治之英君,大有为于天下者,其命将也,非信左右佞幸之推引,如燕之任骑劫、赵之任赵葱也;所任之将,亦当时人望所归,小试有效,非若曹之任公孙彊、蜀汉之任陈祗也;意者当代有将才而莫之能用邪?然自是以后,未见有人焉,愈于彦之、思话而当时不用者,将天之吝于生材乎?非也。天生之,人主必有以鼓舞而培养之,当世之士,以人主之意指为趋,而文帝、孝宗之所信任推崇以风示天下者,皆拘葸异谨之人,谓可信以无疑,而不知其适以召败也。道不足以消逆叛之萌,智不足以驭枭雄之士,于是乎摧抑英尤而登进柔輭;则天下相戒以果敢机谋,而生人之气为之坐痿;故举世无可用之才,以保国而不足,况欲与猾虏争生死于中原乎!」

\subsubsection{元嘉}

\begin{longtable}{|>{\centering\scriptsize}m{2em}|>{\centering\scriptsize}m{1.3em}|>{\centering}m{8.8em}|}
  % \caption{秦王政}\
  \toprule
  \SimHei \normalsize 年数 & \SimHei \scriptsize 公元 & \SimHei 大事件 \tabularnewline
  % \midrule
  \endfirsthead
  \toprule
  \SimHei \normalsize 年数 & \SimHei \scriptsize 公元 & \SimHei 大事件 \tabularnewline
  \midrule
  \endhead
  \midrule
  元年 & 424 & \tabularnewline\hline
  二年 & 425 & \tabularnewline\hline
  三年 & 426 & \tabularnewline\hline
  四年 & 427 & \tabularnewline\hline
  五年 & 428 & \tabularnewline\hline
  六年 & 429 & \tabularnewline\hline
  七年 & 430 & \tabularnewline\hline
  八年 & 431 & \tabularnewline\hline
  九年 & 432 & \tabularnewline\hline
  十年 & 433 & \tabularnewline\hline
  十一年 & 434 & \tabularnewline\hline
  十二年 & 435 & \tabularnewline\hline
  十三年 & 436 & \tabularnewline\hline
  十四年 & 437 & \tabularnewline\hline
  十五年 & 438 & \tabularnewline\hline
  十六年 & 439 & \tabularnewline\hline
  十七年 & 440 & \tabularnewline\hline
  十八年 & 441 & \tabularnewline\hline
  十九年 & 442 & \tabularnewline\hline
  二十年 & 443 & \tabularnewline\hline
  二一年 & 444 & \tabularnewline\hline
  二二年 & 445 & \tabularnewline\hline
  二三年 & 446 & \tabularnewline\hline
  二四年 & 447 & \tabularnewline\hline
  二五年 & 448 & \tabularnewline\hline
  二六年 & 449 & \tabularnewline\hline
  二七年 & 450 & \tabularnewline\hline
  二八年 & 451 & \tabularnewline\hline
  二九年 & 452 & \tabularnewline\hline
  三十年 & 453 & \tabularnewline\hline
  \bottomrule
\end{longtable}


%%% Local Variables:
%%% mode: latex
%%% TeX-engine: xetex
%%% TeX-master: "../../Main"
%%% End:

%% -*- coding: utf-8 -*-
%% Time-stamp: <Chen Wang: 2021-11-01 15:03:33>

\subsection{孝武帝刘劭\tiny(453-464)}

\subsubsection{刘劭生平}

刘劭(424年-453年5月27日),字休远,彭城綏輿里(今江蘇省徐州市銅山區)人。他是中国南北朝時期南朝宋宋文帝刘义隆的长子,母為皇后袁齊媯。宋文帝晚年因劉劭與女巫嚴道育交往及行巫蠱而謀廢其太子之位,劉劭於是先發制人發起兵變弒父奪位。但即位不久即遭三弟武陵王劉駿為首的軍隊討伐,兵敗被殺,在位僅一百日。史書因劉劭殺父奪位,不用劉劭為文帝上的廟號及諡號,亦不承認劉劭為刘宋的正統皇帝。

元嘉元年(424年),刘劭出生,時正值劉義隆在服父喪,於是一直到喪期結束,於元嘉三年閏正月丙戌(426年2月28日)才正式宣布長子誕生。元嘉六年三月丁巳(429年5月14日),刘劭被立為皇太子,居於永福省。元嘉十二年(435年),劉劭出居東宮,並娶殷淳女殷氏為太子妃。劉劭愛讀史書,尤其喜歡武事,而不管是親覽宮廷事務還是接待賓客,只要他想作,宋文帝都會順從他。為保護東宮,文帝亦讓東宮兵力與羽林衞兵力相同。

元嘉二十七年(450年),宋文帝北伐,劉劭與沈慶之、蕭思話等人極力反對但不果,但戰事不利,魏軍南攻至長江北岸的瓜步,震動建康,劉劭領兵出守石頭,總統水軍。時北魏遣使求婚,包括劉劭以內群臣都認為應該准許,但一直力主北伐的江湛認為外族無信,反對和親,劉劭於是大怒,厲聲斥責他,並在眾人離去時命隨從推逼江湛,差點將他推倒。劉劭又對文帝說:「北伐敗辱,數州淪破,獨有斬江湛,可以謝天下。」但文帝以北伐乃己意為由拒絕對江湛問罪。不過此後,劉劭每次辦宴都沒邀請江湛,又常和文帝說江湛是佞人,不應親近。

文帝為提倡農耕和種桑養蠶,特意在宮內養蠶。當時有一個叫嚴道育的女巫自稱能夠通靈及使喚鬼怪,劉劭姊東陽公主在婢女王鸚鵡告知之下,向文帝假稱道育擅長養蠶而召入宮中。道育入宮後稱述服用丹藥之事,預告符瑞事後又果真看見奇異事情,劉劭及東陽公主於是都對道育能力深信不疑。文帝次子始興王劉濬一向都依附劉劭,又因二人常有過失,為了不讓文帝知道,就請嚴道育幫忙,最終道育教他們巫蠱之事,以文帝形像造一個玉雕人像,埋在含章殿前。這件事除了他們三人知道外,王鸚鵡、與王鸚鵡私通的養子奴僕陳天興以及黃門慶國都有參與,劉劭更給了陳天興一個隊主職位。東陽公主死後,王鸚鵡亦該嫁人,劉劭及劉濬為守住巫蠱秘密,就自行決定將王鸚鵡嫁給劉濬府佐沈懷遠為妾,也不報告給文帝,只稍稍和臨賀公主提及了。不過,文帝及後知陳天興任隊主,亦知天興與王鸚鵡養母子的關係,特意派人詰問劉劭有關二人之事,劉劭就答稱天興身體壯健故給其職位,而表示鸚鵡還未嫁。鸚鵡當時已嫁給沈懷遠,劉劭怕事情被揭穿,立即通報給劉濬及臨賀公主要他們都稱鸚鵡未定婚嫁;王鸚鵡亦怕陳天興會令二人私通的事曝光,於是請求劉劭殺了天興滅口。不過,天興之死令黃門慶國擔心自身安危,於是將巫蠱之事告知文帝。文帝聞訊既震驚又哀惋,立即就命人收捕王鸚鵡,並在其家中搜到數百張劉劭及劉濬所寫的紙,全都是詛咒巫蠱的文句,又挖出含章殿前的人像。嚴道育就逃亡,成功躲過搜捕之人,並易服為尼姑,匿藏在東宮,有時隨劉濬出京口,有時又住在平民張旿的家。而劉劭及劉濬面對文帝的詰責,驚懼得無法答話,只有一直道歉。元嘉二十八年(451年)至元嘉三十年(453年)幾次的天象變異,令文帝再加東宮兵眾,令東宮擁有一萬兵。

元嘉三十年(453年)二月,劉濬轉任荊州刺史,遂自京口入朝,並載著嚴道育回東宮,打算帶她同赴荊州。不過,那時就有人告發嚴道育化身尼姑,常出入劉濬府內,文帝起初不信,但派人查問下終從兩個婢女口中得悉那真是嚴道育。文帝知二子仍然和嚴道育往來十分傷心,於是命京口送二婢到來,並決定廢太子及賜死劉濬,為此與王僧綽、江湛及徐湛之商討,但久未有決定。文帝亦將決定向劉濬生母潘淑妃透露,潘淑妃就將此事告知劉濬,劉濬報告劉劭後劉劭就決定起事,於是晚晚設宴款待將士,又與心腹張超之、陳叔兒、詹叔兒及任建之籌劃。

二月二十日晚(3月15日),兩婢快將到來,劉劭假傳詔命:「魯秀謀反,汝可平明守闕,率眾入。」於是命張超之召集二千多名士兵作準備,又召各幢隊主聲稱有討伐之事。當晚,劉劭又召見了前太子中庶子蕭斌、太子左衞率袁淑、太子中舍人殷仲素及左積弩將軍王正見入宮,對他們哭著說:「主上信讒,將見罪廢。內省無過,不能受枉。明旦便當行大事,望相與勠力。」蕭斌與袁淑立即就表示反對,勸他再作考慮,劉劭聽後表現憤怒。蕭斌在驚嚇下轉為支持,但袁淑仍舊反對,惟未能讓劉劭回心轉意,劉劭接著向袁淑等人分派袴褶,又分派幾段錦布讓其縛好袴子,作出戰準備。天亮時,劉劭與蕭斌同車準備好出發,停在奉化門等袁淑,但袁淑久久不到,到後又不肯上車,劉劭遂命左右殺害袁淑,接著命部眾如同平常入朝一樣走進宮中。經萬春門時,由於違反東宮兵入宮城的規定,劉劭於是對門衞聲稱受到敕命要帶兵入宮收討,遂成功進宮。接著張超之等數十人就直入雲龍門、東中華門及齋閤,拔刀直奔上合殿。當晚文帝又與徐湛之徹夜密談,當值衞兵至此時仍然在寢,張超之就上前砍殺文帝,並殺掉徐湛之。劉劭知文帝被殺後出坐東堂,由蕭斌持刀侍衞,並派人殺害江湛;左細仗主卜天與率眾進攻劉劭,但失敗被殺。

當日劉劭就即位稱帝,寫詔道:「徐湛之、江湛弒逆無狀,吾勒兵入殿,已無所及,號惋崩衄,肝心破裂。今罪人斯得,元凶克殄,可大赦天下,改元嘉三十年為太初元年。文並賜位二等,諸科一依丁卯。」太初年號是劉劭與嚴道育所商定的,來朝的官員才數十人,劉劭就等不及要即位,即位後又稱疾退入永福省,升文帝靈柩至太極前殿。劉劭及後將先前給諸王及各處的武器回武庫,又誅殺徐湛之、江湛等人的黨羽,並封賞幫助他篡位的官員,後來更將與其有宿怨的長沙王劉瑾等人宗室殺害;又在查閱文帝巾箱時發現王僧綽亦有參與廢太子的圖謀,亦將其殺害。文帝大殮時,劉劭稱疾不敢親往,至入殮後才穿上喪服至文帝靈前,表現得痛心哀慟。然後又向四方派大使,對一眾官員求問治國之道,又減輕賦稅及減少徭役,減省出遊耗費,又分配一些田野山澤給貧民。又先後立妃殷氏為皇后,長子劉偉之為太子。

不過,江州刺史武陵王劉駿、荊州刺史南譙王劉義宣、雍州刺史臧質及會稽太守隨王劉誕等人都拒命,起兵討伐劉劭,並以劉駿為主。劉劭弒父後正值劉駿典籤董元嗣回到建康,於是就命元嗣將自己聲稱的徐湛之弒逆版本報告給劉駿,但元嗣回去後就以實情報告。劉駿一方面派元嗣奉表還都,另一方面卻謀起兵。劉劭知劉駿起兵就責問元嗣,元嗣表示出發時尚未有此事,但劉劭不信,加以拷打後元嗣仍然不招,最終被打死。劉劭亦曾密書當時與劉駿一同討蠻的沈慶之,命其殺死劉駿,但慶之就支持劉駿,反助其統兵東下。面對大軍來攻,劉劭下令中外戒嚴,又自以自己向來習武,對百官說“卿等但助我理文書,勿措意戎陳。若有寇難,吾當自出,唯恐賊不敢動爾”,並由皇后叔父司隸校尉殷沖掌文符,左衞將軍尹弘為軍隊準備衣服,由蕭斌總掌眾事。另又將劉駿及義宣諸子分別軟禁在侍中下省及太倉空屋,並打算殺害尋陽、江陵、會稽三鎮士庶官員留在建康的家眷,但在劉義恭及何尚之勸阻下改變主意,才改為下書表明不問罪。劉劭又命褚湛之守石頭,劉思考鎮東府,蕭斌及劉濬就力勸劉劭率水軍迎擊討伐軍,不過劉義恭則建議以逸待勞,劉劭取信了義恭之策,即使蕭斌力陳對方形勢佔上風,應盡快一決勝負亦未能讓劉劭改變主意。不過,其實劉劭始終都不太相信朝廷舊臣,留義恭住尚書下省外,亦軟禁義恭十二個兒子在侍中下省;另厚待王羅漢及魯秀,將兵權交給二人,大加賞賜財寶和美女以取悅二人,又天天出外慰勞將士,自督修建船艦,又焚毀秦淮河南岸,將百姓都趕到北岸。

大軍臨近,但守石頭的龐秀之卻先一步轉投劉駿陣營,大大動搖人心。四月十九日(5月12日),討伐軍前鋒已到新林,劉劭親上石頭城烽火樓觀敵。二十一日(5月14日),討伐軍進至新亭,柳元景在依山建新亭壘據險自守,而劉劭就召魯秀與王羅漢駐朱雀門,讓蕭斌率步兵、褚湛之率水軍;當時詹叔兒察知討伐軍營壘尚未建立,勸劉劭乘時進擊,但劉劭不肯。翌日,劉劭才命蕭斌率魯秀、王羅漢等精兵共萬人進攻新亭壘,劉劭亦親自登上朱雀門督戰。由於士兵都得劉劭賞賜,故都為他拼命作戰,戰事佔據上風。但就在新亭壘將被攻下時,魯秀突然收兵,柳元景抓緊機會反擊,終扭轉戰局。劉劭在蕭斌等敗後又親率心腹再戰,又遭柳元景擊敗,死傷更大,劉劭斬殺撤退者以圖遏止潰敗之勢,但失敗,劉劭只好走經朱雀門還宮。

此戰敗後,褚湛之、檀和之、魯秀及劉義恭先後叛歸劉駿,劉劭只好向神明之力求助,將蔣侯神像運到宮內,拜他為大司馬,封鍾山郡王;又以蘇侯為驃騎將軍,命南平王劉鑠寫祝文向其宣告劉駿罪狀。五月,劉劭派往抵御劉誕所領東軍的部隊在曲阿戰敗,為了遏阻他進攻,劉劭就焚毀都水西裝及左尚方,以及破壞柏崗及方山土壩,又命守家未服兵役的男丁緣秦淮河竪起舶船,並在上築上大弩作防禦,又命人以柵欄阻斷班瀆、白石等水道。其時男丁不足,甚至要動用婦女完成工事。

五月三日(5月26日),魯秀率五百人進攻大航,將之攻克,守將王羅漢酒醉中驚聞敵軍已渡河,於是棄杖投降,其餘部隊亦都隨之潰散。當晚,劉劭關閉六門拒守,在門內鑿出護城河及柵欄。不過城內混亂,已無秩序,尹弘及孟宗嗣等人出降,蕭斌知大航失守後亦命所屬軍隊解甲投降但被殺。翌日(5月27日),劉義恭登朱雀門,總領諸軍進攻宣陽門,先前劉劭召還的陳叔兒部於建陽門遠遠望見討伐軍就棄杖逃走;原本屯駐閶闔門的劉劭部隊亦逃還殿內,程天祚及譚金等人因而攻入殿內,其餘眾軍繼進,臧質亦從廣莫門進入,會師太極殿前。劉劭穿過西垣入武庫井內,但為高禽所捕。當時劉劭問高禽:「天子何在?」高禽答:「至尊近在新亭。」高禽將劉劭帶到殿前,臧質問其為何行逆,劉劭答:「先朝當見枉廢,不能作獄中囚,問計於蕭斌,斌見勸如此。」將罪責推及蕭斌,又問臧質可否代為請求劉駿流放他到遠地。臧質遂將劉劭縛在馬上,將要衞送到劉駿軍門,但到牙旗下時劉義恭率眾觀望,並詰問劉劭何以殺其十二子,劉劭亦答道這是有負於義恭。江湛妻庾氏及龐秀之亦罵劉劭,但劉劭卻大聲回罵:「汝輩復何煩爾!」劉劭四子皆被殺,劉劭對劉鑠說:「此何有哉。」接著劉劭亦於牙旗下被殺,死前嘆道:「不圖宗室一至於此。」

劉駿在較早前已即位為帝,劉劭死後亂事被平定,劉劭妻殷氏與劉劭、劉濬諸子都被賜死,其他幫助劉劭的大臣如殷沖、尹弘、王羅漢及張超之等都被殺或賜死,嚴道育及王鸚鵡都在街上被鞭殺,焚屍揚灰江上;劉劭及劉濬屍體都被棄到長江中,枭首大航,劉劭東宮住所亦被毀。

據說劉劭初生時,尚為宜都王妃的袁皇后對兒子詳細端視後就命人向劉義隆表示:「此兒形貌異常,必破國亡家,不可舉。」並要下手殺掉他。劉義隆狼狽地趕去阻止才讓劉劭得以長大。

文帝死前一天夜晚,太史曾上奏預測東方有兵突襲,建議在太極前殿列兵萬人作銷災。文帝不許,最終讓劉劭成功篡位,聞言就嘆道:「幾誤我事。」又問太史令他還有多少年壽命,太史當時回答十年,但退下後就對人稱只有十旬日,劉劭知道後大怒,將太史殺掉,最終劉劭果十旬而亡。

刘劭的年号是太初(453年二月—453年五月),共计三個月。

\subsubsection{孝武帝生平}

宋孝武帝劉駿(430年9月19日-464年7月12日),字休龍,小字道民,宋文帝劉義隆的第三子,南朝宋第五任皇帝。453年3月16日深夜,皇太子劉劭於京城建康(今南京市)行凶,殺害父皇宋文帝劉義隆,自稱皇帝;時為武陵王的劉駿在沈慶之的輔佐下,於江州(今九江市)起兵宣討。同年5月20日,於新亭(今南京市西南)即皇帝位。5月27日攻下京城,擒斬長兄劉劭、二兄劉濬。隔年(454年)2月14日改元,年號孝建;457年2月10日二度改元,年號大明。

劉駿在位期間,加強中央集權,撤除「錄尚書事」職銜,並分割州、郡以削弱藩鎮實力;誅中書令王僧達、丹陽令顏竣,討誅隨王劉誕,剷除強臣。崇禮佛教,尊奉高僧僧導,率公卿親臨瓦官寺聽宣《維摩詰經》;詔令整肅佛門,勒令不法僧人還俗;史載劉駿天性好色,臨幸不避戚誼,並有與母后路惠男亂倫之嫌疑,流傳後世。

464年7月12日,劉駿病逝於建康宮玉燭殿,享年三十五歲,在位十一年。8月27日,奉葬景寧陵。

史載劉駿其人機警聰慧,博學多聞並文采華美,讀書能七行俱下,又雄豪尚武,擅長騎射。劉駿病逝後,吏部尚書蔡興宗稱其為「守道之君」(「以道始終」);然而劉駿生性喜奢、欲求無度,晚年「尤貪財利」、不聽善諫,以致原本讚許他德行的士族,也感嘆「天下失望」;更兼大明末年,浙江大旱,通貨膨脹失控、浙江的人民餓死十分之六、七,依《宋書‧州郡志》記載之戶口推算,飢餓致死者最高可能有三十萬人。南朝梁史家裴子野總結劉駿「威可以整法,智足以勝奸,人君之略,幾將備矣。」卻也嘆道:「夫以世祖才明,少以禮度自肅,思武皇之節儉,追太祖之寬恕,則漢之文景,曾何足云!」

劉駿生於南朝宋文帝元嘉年間(430年9月19日),為宋文帝第三子。435年,年僅六歲便受封武陵王,食邑二千戶;439年,時年十歲,受詔都督湘州諸軍事、征虜將軍、湘州刺史,領石頭戍事;440年,遷使持節、都督南豫、豫、司、雍、并五州諸軍事、南豫州刺史,仍任征虜將軍,戍守石頭城;444年,加都督秦州,進號撫軍將軍;隔年(445年),時年十六歲,受詔改任都督雍、梁、南北秦四州,荊州之襄陽、竟陵、南陽、順陽、新野、隨六郡諸軍事、甯蠻校尉、雍州刺史,持節,仍任撫軍將軍。自東晉偏安江東後,劉駿為南朝第一位出鎮襄陽的皇室子弟。449年,受詔改任都督南兗、徐、兗、青、冀、幽六州、豫州之梁郡諸軍事、安北將軍、徐州刺史,持節如故,北鎮彭城。不久宋文帝又下詔加任劉駿為兗州刺史,次子始興王劉濬為南兗州刺史,因此劉駿都督南兗州的職銜當即撤銷。

450年,北魏太武帝拓跋燾率兵南侵,宋文帝詔令劉駿領兵北襲屯駐於汝陽的北魏永昌王拓跋仁。劉駿領一千五百兵馬進襲汝陽,魏兵因無防備而潰敗。但之後探得宋軍並無援軍,因而反戈一擊,宋軍大敗,士兵僅有九百人生還。5月19日,劉駿因汝陽戰敗,降號為鎮軍將軍。451年3月19日,魏軍解圍盱眙北還。4月13日,因防禦北魏入侵無功,宋文帝再下詔降劉駿為北中郎將。

452年,劉駿時年二十三歲,加封都督南兗州軍事,擔任南兗州刺史,鎮守山陽,不久改任都督江州、荊州之江夏、豫州之西陽、晉熙、新蔡四郡諸軍事、南中郎將、江州刺史,持節如故。當時江寇橫行,宋文帝派遣步兵校尉沈慶之討賊,由劉駿全權統領征討大軍。劉駿的親信顏竣,曾於彭城假托沙門僧語,散佈劉駿當為「真人」的符讖謠言,並傳至京師。宋文帝欲行加罪,卻因爆發太子劉劭詛咒皇帝的巫蠱事件,故對劉駿和顏竣暫時不予治罪。

453年3月16日深夜,劉駿長兄、皇太子劉劭趁夜帶兵入宮弒君,宋文帝遇害。劉劭稱帝,進號劉駿為征南將軍、加任散騎常侍,以示攏絡,卻矚使步兵校尉沈慶之殺害劉駿。沈慶之受命後求見劉駿,劉駿稱病不敢接見。沈慶之便闖至劉駿面前,將劉劭的手書呈遞。劉駿涕泣請求沈慶之讓自己與母親路淑媛訣別。沈慶之說:「下官受先帝厚恩,常願報德,今日之事,唯力是視,殿下是何疑之深!」劉駿聽此言,便起座再拜說:「家國安危,在於將軍。」遂由沈慶之處分內外。453年4月11日,劉駿戒嚴示眾,起兵討逆。荊州刺史南譙王劉義宣、雍州刺史臧質響應義舉。5月1日,劉駿移檄建康(今南京市);14日,冠軍將軍柳元景與劉劭大戰於新亭,劉劭敗逃;三天後,劉駿兵進江寧;18日,江夏王劉義恭來降,奉表上尊號;隔日,劉駿進駐新亭,使散騎侍郎徐爰草制即位禮儀。

453年5月20日,武陵王劉駿於新亭即皇帝位,大赦天下,時年二十四歲;27日,攻陷建康城,斬偽皇帝劉劭及二兄劉濬。

454年3月17日,南郡王劉義宣、江州刺史臧質、豫州刺史魯爽、兗州刺史徐遺寶舉兵造反。因新皇即位日淺,朝廷得報大懼。劉駿甚至想奉呈乘輿法物迎劉義宣即位,竟陵王劉誕當即阻止,說:「奈何持此座與人?」劉駿乃止。4月19日,安北司馬夏侯祖歡擊破徐遺寶;6月1日,鎮軍將軍沈慶之於曆陽之小峴大破魯爽,將其斬決;29日,劉義宣及臧質率軍攻梁山營壘,豫州刺史王玄謨派遣遊擊將軍垣護之、竟陵太守薛安都出壘迎戰,擊敗臧質。垣護之因風縱火,劉義宣及臧質大敗而逃;7月13日,臧質遭斬;8月4日,賜死劉義宣於江陵獄中。

455年8月29日,因武昌王劉渾自號楚王、擅訂年號(永光),潛越禮制,下詔將其廢為庶人,賜死。

459年,劉駿暗示有司核奏竟陵王劉誕不法,貶爵為侯,並任命垣閬為兗州刺史,以赴鎮所為名,趁機襲擊劉誕。事泄失敗,垣閬被殺。6月4日,劉誕聚眾造反,佔據廣陵城,劉駿派遣車騎大將軍沈慶之率兵平叛;9月22日,攻下廣陵,將劉誕斬首,殺光城內的三千男丁,女子賞賜給兵士。

劉駿是一個頗有作為、積極改革制度的皇帝。他加強中央集權,撤除「錄尚書事」職銜,並分割州、郡以削弱藩鎮實力。454年7月28日,因揚、荊二州地大兵多,刺史易生異志,劉駿下詔分割揚州、浙東五郡為「東揚州」,並由荊、湘、江、豫四州分割出八郡,劃歸「郢州」,荊、揚二州自此削弱;撤除「南蠻校尉」一職,戍兵移鎮建康,增強京師武備。同年(454年),劉駿因劉義宣叛亂,有意削弱諸王侯權勢,江夏王劉義恭於是奏請裁損諸王侯車服器用、樂舞制度九條,劉駿准奏後,更另有司增訂至二十四條,全面抑制藩王地位,威福獨專。宗王兄弟中只有七弟劉宏被親愛重用,455年成為宰相(458年卒)。孝武帝同時重用江東寒門沈慶之與傖荒北人柳元景,依照兩人的功績,先後提拔為三公,開啟吳興沈氏與河東柳氏攀升為南朝高門的起始之路,並開創南朝寒門、寒人以軍功升為三公的先例。

458年(大明二年),在外放顏竣並處死王僧達後,劉駿欲大權獨攬、專擅朝綱,因此除了高門蔡興宗與袁顗以外,從此不再放權給宗王兄弟與高門強族的大臣,專委任倖臣充作耳目,隱刺朝政,形成後代所謂「寒人掌機要」的政治局面,孝武帝的集權統治也被史書稱為「主威獨運,官置百司,權不外假」。倖臣當中,戴法興、巢尚之、戴明寶、徐爰四人,最有理政才幹,因此大受寵幸,事必與議。巢尚之及徐爰尤知謹慎,惟戴法興及戴明寶卻因此作威作福、納賄受貨,門庭若市,身價並達千金。戴明寶尤其驕縱,放任長子戴敬出錢搶買皇帝的御用物,甚至於劉駿出巡時,騎馬於御輦旁來回奔馳,毫無顧忌。劉駿大怒,下令處死戴敬並將戴明寶下獄,不久仍釋放,委以重任如初。而戴法興於劉子業任皇太子時即奉命侍從,後更受劉駿遺命託孤,輔佐劉子業繼位(宋前廢帝),以致宋前廢帝時有民間謠言:「戴法興為真天子,皇帝為假天子。」之語,權重若此。

劉駿生性嚴峻寡恩,對待左右侍臣,動輒屠戮;甚且自詡風流,晚年專喜戲謔大臣,各取綽號,無禮之至,惟吏部尚書蔡興宗方直嚴肅,劉駿憚怕之,不敢侵狎;平時飲食起居極盡奢華,宮殿牆柱及地板皆鋪錦繡,又嫌宮廷狹小,特命建「玉燭殿」以供享樂,並破壞其祖父、宋武帝生前所居密室,做為地基,並率大臣圍觀動工。見床頭用土作鄣,牆上掛葛燈籠、麻繩拂,侍中袁顗便稱讚宋武帝有節儉樸素之德,劉駿自以為名士派頭,瞧不起沒文化的祖父劉裕,批評說:「田舍公得此,以為過矣!」(「鄉下人能用這些東西,已經太過了!」)

劉駿生性好賭,揮霍不少,加上國家戰亂之後,中央府庫空虛、無錢可使,便效法桓玄手段,以賭博斂財。詔命凡各州刺史及二千石官員,卸職還都時須獻奉財物,限期繳納。其後更召入宮中賭博作樂,賺盡地方官於其任上所積錢財,方准離去。這種收稅辦法被後任的宋、齊皇帝沿用並發揚光大,直接強逼刺史「獻奉」,省略掉賭博這種相對體面的手法。;劉駿晚年喜好飲酒,常飲至深夜,隔日起床洗漱完畢後,便繼續喝至大醉,整日嗜睡。然而有奏疏馳至,便立刻整理好儀容,毫無醉態。宮中內外都佩服他的機神明肅,不敢偷懶懈怠。

大明七年(463年)底至八年(464年),浙江等地因為劇烈旱災,造成嚴重的大饑荒,浙江十分之六的戶口餓死逃散。宋朝史家司馬光因此批評劉駿,說他晚年好酒奢靡,以致原本強盛的劉宋,在他執政末年中衰。

454年,劉駿召幸南郡王劉義宣(六叔)的幾個女兒,劉義宣於是憎恨劉駿,隨後在江州刺史臧質的慫恿下,起兵造反。造反失敗,劉義宣遭誅。劉駿可能便秘密納娶其中一位堂妹(劉駿為避人耳目,冊封其為殷淑儀),並與其生下第八子劉子鸞等五子一女,但也有說法認為殷淑儀並非劉氏女。

史載劉駿與母親路太后有亂倫之嫌疑。南朝人沈約所著《宋書》之記載較為含蓄,內文如下:「上於閨房之內,禮敬甚寡,有所御幸,或留止太后房內,故民間喧然,咸有醜聲。宮掖事秘,莫能辨也。」——《宋書‧列傳第一‧后妃傳》

《宋書》指劉駿常於路太后所居顯陽殿中臨幸宮女,因停留時間過久,以致民間謠傳其間有不可告人之事。《宋書》作者沈約並無否認,只模稜兩可地表示:「宮掖事秘,莫能辨也。」

然而由北朝人魏收所著的《魏書》就沒有顧忌,直接指涉劉駿與其母亂倫:「駿淫亂無度,蒸其母路氏,穢汙之聲,布於甌越。」——《魏書‧列傳第八十五‧島夷劉裕傳》

魏收還記述劉駿天性好色、狎褻無度,以致其兒子、宋前廢帝劉子業即位後,指著劉駿的畫像罵:「此渠大好色,不擇尊卑!」

但也有人認為記載不實。唐朝史家劉知幾在其著作《史通》中辯誣說:「沈氏著書,好誣先代,於晉則故造奇說,在宋則多出謗言,前史所載,已譏其謬矣。而魏收黨附北朝,尤苦南國,承其詭妄,重以加諸。遂云馬睿出於牛金,劉駿上淫路氏。可謂助桀為虐,幸人之災。」

464年7月12日,劉駿病逝於玉燭殿,享年三十五歲,在位十一年。皇太子劉子業繼位,是為宋前廢帝。8月27日,奉葬位於丹陽郡秣陵縣岩山(今南京市江寧區秣陵鎮)的景寧陵,予諡「孝武皇帝」,廟號「世祖」。


\subsubsection{孝建}

\begin{longtable}{|>{\centering\scriptsize}m{2em}|>{\centering\scriptsize}m{1.3em}|>{\centering}m{8.8em}|}
  % \caption{秦王政}\
  \toprule
  \SimHei \normalsize 年数 & \SimHei \scriptsize 公元 & \SimHei 大事件 \tabularnewline
  % \midrule
  \endfirsthead
  \toprule
  \SimHei \normalsize 年数 & \SimHei \scriptsize 公元 & \SimHei 大事件 \tabularnewline
  \midrule
  \endhead
  \midrule
  元年 & 454 & \tabularnewline\hline
  二年 & 455 & \tabularnewline\hline
  三年 & 456 & \tabularnewline
  \bottomrule
\end{longtable}

\subsubsection{大明}

\begin{longtable}{|>{\centering\scriptsize}m{2em}|>{\centering\scriptsize}m{1.3em}|>{\centering}m{8.8em}|}
  % \caption{秦王政}\
  \toprule
  \SimHei \normalsize 年数 & \SimHei \scriptsize 公元 & \SimHei 大事件 \tabularnewline
  % \midrule
  \endfirsthead
  \toprule
  \SimHei \normalsize 年数 & \SimHei \scriptsize 公元 & \SimHei 大事件 \tabularnewline
  \midrule
  \endhead
  \midrule
  元年 & 457 & \tabularnewline\hline
  二年 & 458 & \tabularnewline\hline
  三年 & 459 & \tabularnewline\hline
  四年 & 460 & \tabularnewline\hline
  五年 & 461 & \tabularnewline\hline
  六年 & 462 & \tabularnewline\hline
  七年 & 463 & \tabularnewline\hline
  八年 & 464 & \tabularnewline
  \bottomrule
\end{longtable}


%%% Local Variables:
%%% mode: latex
%%% TeX-engine: xetex
%%% TeX-master: "../../Main"
%%% End:

%% -*- coding: utf-8 -*-
%% Time-stamp: <Chen Wang: 2021-11-01 15:03:41>

\subsection{前废帝刘子业\tiny(464-465)}

\subsubsection{生平}

刘子业(449年2月25日-466年1月1日),小字法师,中國歷史南北朝時期南朝宋皇帝,史稱「前廢帝」。他是宋孝武帝刘骏长子,生母為文穆皇后王憲嫄。年号“永光”、“景和”。宋前廢帝以皇太子身份即位,但即位之初受制於掌權大臣而難以專政,遂於即位一年後就先將主政大臣戴法興誅殺,接著又將圖謀廢立的三名顧命大臣殺害,其中更殘忍肢解了叔祖父江夏王劉義恭。此後前廢帝肆意行事,荒淫无道,做了很多殘暴甚至亂倫的行為,約半年後就在阮佃夫等人策劃下,被主衣壽寂之刺殺。

劉子業於元嘉二十六年正月十七日(449年2月25日)出生,四年後就發生了太子劉劭弒宋文帝奪位的事件,因為孝武帝起兵討伐劉劭,劉子業被劉劭囚於侍中下省。同年,孝武帝即位,於翌年孝建元年(454年)立了子業為皇太子。不過,子業一直居於永福省,在大明二年(458年)才出居東宮。子業在東宮多有犯錯,而孝武帝亦寵愛殷淑儀以及和她生下的皇子劉子鸞,於是一度有了廢子業,立子鸞的想法,但時為侍中的袁顗稱讚子業好學,天天進步,終也保住了其太子之位。

大明八年閏五月廿三日(464年7月12日),孝武帝去世,同日子業以皇太子繼位,是為宋前廢帝。孝武帝死前指定了江夏王劉義恭、柳元景、顏師伯、沈慶之及王玄謨五人為顧命大臣,分掌朝事以及軍旅之事。不過,前廢帝即位後朝事其實都繼續由孝武帝寵臣越騎校尉戴法興及中書通事舍人巢尚之掌握,義恭等雖錄尚書事仍只守空名。前廢帝即位後不久獲尊為皇太后的生母王憲嫄病重,遂派人召廢帝前來,但廢帝卻說:「病人房間裡有很多鬼,太可怕了,這怎麼能去呢?」竟拒絕探望母親,不久太后便過世。而前廢帝的命令和活動此時亦受戴法興所約束,意願常常被法興壓下,法興甚至多次對廢帝說:「你這樣的作為,想成為營陽王嗎?」這令廢帝很不滿,於是與痛恨戴法興的宦官華願兒勾結誣陷法興,終於永光元年八月初一(465年9月6日)賜死了戴法興。

前廢帝又為了削弱時任尚書僕射的顏師伯的權力,故意重設左右僕射,以王彧為右僕射,更加奪其兼丹陽尹之職,令師伯深感不安。而前廢帝日漸顯露的狂悖行徑亦令柳元景、劉義恭等人十分憂心,於是義恭與元景、師伯等人陰謀廢帝而立義恭,但久未有決定,又嘗試尋求沈慶之支持,但慶之卻向廢帝告發圖謀。永光元年八月十三日(465年9月18日),廢帝親領禁軍宿衞去收捕柳元景,就地將其殺害;又領兵到義恭府第殺害義恭,更下令肢解義恭,甚至將義恭眼晴拿出來浸在蜜糖中,稱之為「鬼目粽」。二人皆被夷滅三族,顏師伯、劉德願等亦被誅殺。

沈慶之因與義恭並不親厚,又與師伯有私憾,遂告發了圖謀,廢帝亦以沈慶之為太尉以褒賞他。袁顗當日在孝武帝面前保廢帝太子之位,廢帝本亦感其恩德,加上沈慶之亦念在袁顗提拔之恩,袁顗遂得以在義恭等人被殺後入為吏部尚書,與慶之、徐爰領選事。然而,很快袁顗就因不合廢帝心意而獲罪,白衣領職,袁顗在恐懼之下自求外任,終獲授雍州刺史,遠赴襄陽。而留在朝中的沈慶之盡心對廢帝的荒唐行為作出規勸,也令廢帝很不滿。廢帝後來將姑姑新蔡公主劉英媚納於後宮,向外謊稱她是謝貴嬪,宣稱公主已死並以一個婢女的屍體冒充,送到公主丈夫何邁那裏供治喪用。何邁本亦已受猜忌,早有廢立的準備,打算趁廢帝出遊時下手,但圖謀外泄,景和元年十一月,何邁又遭廢帝親自領兵誅殺。殺何邁前,廢帝知沈慶之一定會來反對,於是命人封鎖清豁諸橋阻止其前來,年已八十的慶之無法進見後回府,廢帝更派了與慶之有過節的沈攸之送藥賜死他。

新安王劉子鸞一度危及廢帝太子之位,廢帝在誅除義恭等人後開始對其進行報復,景和元年九月十一日(465年10月16日),就下令賜死子鸞,同為殷淑儀所生的十二皇女以及劉子師都一同被賜死,並開挖殷淑儀的墓穴,又怪罪寫了《宋孝武宣貴妃誄》的謝莊,甚至想掘開父親陵墓景寧陵,只因太史勸阻而不成事,不過仍然用糞便弄污陵墓。另一方面,前廢帝亦忌憚一眾叔父威脅他的地位,其中九叔義陽王劉昶早在孝武一朝就有謀反的傳言,到廢帝在位年間就更盛,廢帝亦常對旁人稱他即位以來未試過戒嚴,有所不快。劉昶在義恭等人被殺後上表入朝,廢帝就向陪使者入都的劉昶典籤籧法生宣稱劉昶與義恭合謀反叛,入朝正好,但又斥責法生沒有通報劉昶謀反的訊息。法生聞言恐懼,於是立即逃到彭城告知劉昶,而廢帝就以此為由北討,於九月己酉親征彭城,並宣布內外戒嚴。劉昶試圖起兵抵抗但無人支持,只好逃到北魏。

剩下諸叔,廢帝將他們都囚於殿內毆打欺凌,其中他最忌憚較年長的十一叔湘東王劉彧、十二叔建安王劉休仁及十三叔山陽王劉休祐,因他們都很肥壯而命人用竹籠載著他們量度體重,最重的劉彧被稱為「豬王」,休仁及休祐分別獲得「殺王」及「賊王」之號,並時常命他們跟著自己。才能差劣的八叔東海王劉褘也被稱為「驢王」,只有年紀尚輕的桂陽王劉休範及巴陵王劉休若過得好點。廢帝曾經脫光劉彧,將他放到坑中,並將飯菜都倒在木槽中混合,讓坑中的劉彧像豬一樣到木槽進食,以作娛樂;又常想殺害三王,但因劉休仁用其計策取悅廢帝,廢帝將就一直沒有殺他們。不過,廢帝卻屢次逼姦宮中妃主,例如命身邊官員侍從當著休仁面前逼姦休仁生母楊太妃,又曾威逼南平王妃江氏就範,在她堅拒後殺掉了她的三個兒子南平王劉敬猷、廬陽王劉敬先及南安縣侯劉敬淵。因著文帝及孝武帝在兄弟中皆排名第三,得入繼帝位,廢帝對三弟晉安王劉子勛亦很猜忌,而何邁的廢立圖謀都是以子勛取代廢帝。何邁失敗後,廢帝乘此指控子勛與何邁謀反,派了使者賜死子勛,以鄧琬為首的子勛屬官們最終決定起兵抗命,在十一月十九日於尋陽宣布戒嚴。

前廢帝表現無道,蔡興宗更曾經向在軍中有威望的沈慶之及王玄謨明言起事推翻廢帝,又曾向掌禁軍的右衞將軍劉道隆暗示,但都遭對方拒絕。相反,前廢帝以美女財帛等東西賜給宗越、譚金、童太一及沈攸之等將領,讓他們甘心為前廢帝服務,為其爪牙。而湘東世子師阮佃夫見劉彧被囚於殿內,常面臨被殺危機,就與王道隆、李道兒、禁中將領柳光世等人以及淳于文祖、繆方盛等前廢帝身邊近臣密謀廢立。景和元年十一月二十九日,前廢帝打算出巡荊湘二州,並欲在出發前將劉彧等三王殺害,阮佃夫在前廢帝早上出華林園時將圖謀密告主衣壽寂之、細鎧主姜產之等人,獲得響應,其中防守華林閤的隊主樊僧整也加入了。當晚入夜後,廢帝在竹林堂前與巫師射鬼,壽寂之就領著姜產之等人衝進去行刺廢帝,廢帝見寂之衝過來就用箭射他,但沒有命中,於是逃跑,但被寂之追上,被殺,得年十七歲。

史載前廢帝幼而狷急,故任太子期間屢遭孝武帝指責,如孝武帝西巡時命廢帝參侍侯起居,就因為字跡差而被罵,甚至被孝武帝指他「素都懈怠,狷戾日甚,何以固乃爾邪!」廢帝即位後初亦受制於戴法興等人,但自法興等被殺後,就沒有人敢阻遏廢帝行事,很多大臣都被打,人心騷動。

廢帝雖然多有猜忌逼害宗室的舉動,但對於同胞所生的劉子尚及山陰公主劉楚玉卻相當親近,經常一同行動,子尚性情亦有如廢帝那樣。廢帝曾應公主的要求賜其面首三十,並命當時的美男子尚書吏部郎褚淵侍候公主十天。

不過廢帝年輕時都有好學一面,故也對古事有一定認識,所作的《世祖誄》及一些雜篇都不乏有文采的地方,又曾仿效曹操設立發丘中郎將及摸金校尉兩職。

廢帝在母親病重時不肯探望,母親死後卻在其夢中出現,並說:「汝不仁不孝,本無人君之相,子尚愚悖如此,亦非運祚所及。孝武險虐滅道,怨結人神,兒子雖多,並無天命;大命所歸,應還文帝之子。」

廢帝曾在華林園竹林堂命宮女們裸身追逐以供自己取樂,其中一名宮女不肯,廢帝就殺了她。不久,廢帝卻夢見後堂有一女子罵他,廢帝醒來後大怒,遂命人在宮中找出一個和夢中女子相貌相似的宮人,又將她殺死。就在當晚,這個宮人就在廢帝夢中出現,說已經將被枉殺的事告知上帝。後巫師宣稱竹林堂有鬼,才有廢帝前往射鬼,遭壽寂之刺殺的事。

\subsubsection{永光}

\begin{longtable}{|>{\centering\scriptsize}m{2em}|>{\centering\scriptsize}m{1.3em}|>{\centering}m{8.8em}|}
  % \caption{秦王政}\
  \toprule
  \SimHei \normalsize 年数 & \SimHei \scriptsize 公元 & \SimHei 大事件 \tabularnewline
  % \midrule
  \endfirsthead
  \toprule
  \SimHei \normalsize 年数 & \SimHei \scriptsize 公元 & \SimHei 大事件 \tabularnewline
  \midrule
  \endhead
  \midrule
  元年 & 465 & \tabularnewline
  \bottomrule
\end{longtable}

\subsubsection{景和}

\begin{longtable}{|>{\centering\scriptsize}m{2em}|>{\centering\scriptsize}m{1.3em}|>{\centering}m{8.8em}|}
  % \caption{秦王政}\
  \toprule
  \SimHei \normalsize 年数 & \SimHei \scriptsize 公元 & \SimHei 大事件 \tabularnewline
  % \midrule
  \endfirsthead
  \toprule
  \SimHei \normalsize 年数 & \SimHei \scriptsize 公元 & \SimHei 大事件 \tabularnewline
  \midrule
  \endhead
  \midrule
  元年 & 465 & \tabularnewline
  \bottomrule
\end{longtable}


%%% Local Variables:
%%% mode: latex
%%% TeX-engine: xetex
%%% TeX-master: "../../Main"
%%% End:

%% -*- coding: utf-8 -*-
%% Time-stamp: <Chen Wang: 2019-12-20 13:44:50>

\subsection{明帝\tiny(465-472)}

\subsubsection{生平}

宋明帝劉彧(439年12月9日-472年5月10日),字休炳,小字榮期,南朝宋第七任皇帝。劉彧生於元嘉年間,為宋文帝劉義隆第十一子,先後受封淮陽王、湘東王。宋前廢帝劉子業即位,顧慮諸叔威脅皇位,趁劉彧入朝時將其拘留殿中,並因劉彧體胖而封其為「豬王」,大肆羞辱,且屢次欲加殺害,都因始安王劉休仁諂媚化解,才保全性命。劉子業遭壽寂之殺害後,劉休仁便奉迎劉彧即皇帝位,改元泰始,大赦天下。

劉彧在位六年半,執政前期眾親王及方鎮相繼叛變,朝廷頻繁動武平亂,國力逐漸耗損。北魏也趁機侵略,佔領山東、淮北等地區,北朝國力自此超越南朝;劉彧為防範宋孝武帝劉駿諸子奪取皇位,殺盡諸姪子,致使劉駿絕後;晚年尤多忌諱,文書奏折不得出現諱字,犯禁者一律誅殺。

472年5月10日,劉彧逝世,享年三十四歲,庙号太宗,谥号明帝,奉葬高寧陵。

史載劉彧個性寬和仁慈,儀容端雅,喜好文學。即位後雖然四方反抗但用人不疑,能使將士效忠不貳。然而晚年好猜忌,對待皇族及侍臣動輒殘忍刑戮;國家連年征伐,國庫空虛,而劉彧卻奢侈無度,致使「天下騷然,民不堪命」,劉宋國運自此衰敗。

劉彧生於南朝宋文帝元嘉十六年十月戊寅(439年12月9日),九歲時受封淮陽王,食邑二千戶。452年,改封湘東王。劉彧三哥、宋孝武帝劉駿即位後,歷任郡太守、中護軍、侍中、衛尉、游擊將軍、左衞將軍、都官尚書、領軍將軍等職銜,並獲賜鼓吹一部。453年,劉彧生母沈容姬逝世,劉彧時年十四歲,由路太后撫養於宮中,特受寵愛,時常侍奉路太后醫藥,也因此為劉駿所親愛,不招致猜忌。宋前廢帝劉子業繼位後,任命劉彧為州刺史,都督州郡軍事,並得以本號開府儀同三司。

宋前廢帝劉子業即位後荒淫無道,殺戮群臣,並恐諸叔覬覦皇位,欲加殺盡。劉彧於景和年間入朝,遭拘留宮中,百般毆打凌辱。劉彧與始安王劉休仁、山陽王劉休祐皆體型肥胖,被劉子業封為「豬王」、「殺王」、「賊王」,並將三人用竹籠囚禁。劉子業又命人掘地為坑,灌滿泥水,再以木槽盛飯,並用雜食攪和後置於坑前,命劉彧裸體於泥坑中以口對木槽中就食,戲謔為豬。劉彧曾因抗拒羞辱而惹怒劉子業,劉子業命將其裸體後用竹杖綁住四肢抬付太官,說:「即日屠豬。」劉休仁在旁笑說:「豬今日未應死。」劉子業問何故,劉休仁答說:「待皇太子生,殺豬取其肝肺。」劉子業聽後怒火漸息,命交付廷尉,劉彧才逃過死劫。466年,劉子業欲南遊荊州及湘州,決定明日殺害劉彧等諸叔父後,即行出發。劉彧遂與心腹阮佃夫、李道兒等共謀弒君。1月1日夜,阮佃夫與李道兒暗中結交宮中侍臣壽寂之,於華林園將劉子業殺害。劉子業死後,劉休仁隨即奉迎劉彧入宮即位,並令人備皇帝羽儀。由於事起倉促,劉彧半途失落鞋子,跛著走入西堂,仍戴著象徵臣子的烏紗帽,劉休仁讓人給其戴上白紗帽後,便擁劉彧登上御座召見文武百官,接受歡呼禮拜,凡事以「令書」頒布施行。隔天(1月2日),劉彧下令殺掉劉子業的同母次弟劉子尚,以絕後患。

泰始元年十二月丙寅(466年1月9日),劉彧於宮中太極前殿登基為帝,大赦天下。

465年底,宋孝武帝劉駿第三子、晉安王劉子勛為反抗劉子業謀害己命,在鄧琬等人輔佐下,於江州起兵叛亂。劉彧弒君自立後,授姪子劉子勛官爵遭拒。劉子勛甚至在鄧琬的主導下傳檄天下,改討劉彧。466年2月7日,鄧琬、袁顗等奉年僅十歲的劉子勛於尋陽城登極稱帝,年號「義嘉」,另立政府。江州的義嘉政權得到幾乎全國的承認與響應,南朝各州郡皆向劉子勛上表稱臣,改用義嘉年號,並向尋陽奉貢。劉彧所統治區域僅限京師建康(今江蘇省南京市)附近的丹陽、淮南等郡百餘里土地而已,形勢極為嚴峻。

劉彧聞變後隨即任命劉休仁為征討大都督,統帥全軍,王玄謨為副手。任用吳喜、江方興等為東路軍將領,討平會稽、吳、吳興、晉州等東南各州郡,俘虜劉駿第六子、尋陽王劉子房;任用劉勔、張永、蕭道成等為北路軍將領,擊敗殷琰、薛安都等敵對將領,抵擋住北方的攻勢。任命沈攸之、張興世等為西路軍將領討伐劉子勛的尋陽政權,擊敗袁顗、劉胡等人,攻入尋陽,捕斬敵對天子劉子勛,義嘉政權滅亡。隨後宋軍陸續平定江南及淮南各地,「義嘉之難」平息。

由於劉彧登基時,其諸弟(宋文帝劉義隆子嗣)皆在京師,多擁戴兄長劉彧為帝;而劉子勛起兵地方,方鎮都督則多為劉子勛的兄弟(宋孝武帝劉駿子嗣),皆起兵支持劉子勛的義嘉政權。南朝宋即形成文帝系諸王與孝武帝系諸王的內戰局面。劉彧鑑於此,於戰事平定後接受劉休仁的建議,將劉駿在世諸子皆诛殺殆盡,劉駿二十八子自此滅絕。

劉彧於平定叛亂後欲逞兵威,命張永及沈攸之率領重兵,往迎已於義嘉之難後投降的徐州刺史薛安都。薛安都恐劉彧趁機圖己,便向北魏輸誠,乞師自救,汝南太守常珍奇也舉懸瓠城降魏。467年初,北魏派遣尉元、孔伯恭領兵救徐州,另派拓跋石、張窮奇領兵救懸瓠,兗州刺史畢眾敬望風迎降。魏將尉元隨後於呂梁一帶大敗宋將張永及沈攸之,宋軍幾乎全軍覆沒,張永、沈攸之隻身逃回江南,徐、兗二州淪陷;467年2月,青州刺史沈文秀、冀州刺史崔道固投降北魏,旋即又於4月歸降劉宋。北魏遂派遣長孫陵、慕容白曜往攻青州,劉彧命沈攸之領兵救援,卻於睢清口遭魏將孔伯恭擊敗,退守淮陰。青州與冀州待援不至,被圍攻數年,先後降魏,青、冀二州也淪陷。

南朝宋本地處江南,國狹民脊,自此再失四州,國力更形衰弱;再加上戰亂不斷,劉彧為獎賞有功將士,大肆封賞封官,造成國庫空虛、士族制度嚴重破壞,削弱南朝宋的執政根基,北朝國力從此超越南朝。

劉彧晚年害怕諸弟在他死後奪取太子劉昱的皇位,於是接受倖臣王道隆與阮佃夫的建議,大殺立過軍功的諸弟,只有劉休範因為人才凡弱而留下未殺。王道隆與阮佃夫掌權後擅用威權、官以賄成,富逾公室。劉彧同時殺害可能會不利於太子的重要大臣,如功臣武將壽寂之、吳喜與高門名士王景文(皇后王貞風之兄、劉彧的大舅子),結果造成劉昱繼位後中央和地方軍鎮互相猜忌、攻伐的政治亂象,使得武將蕭道成因此崛起,最後篡宋建齊。

472年,宋明帝死,太子劉昱繼立,宋明帝遺詔命蔡興宗、袁粲、褚淵、劉勔、沈攸之五人託孤顧命大臣,分別掌控內外重區,另外命令蕭道成為衛尉,參掌機要。其中遺詔雖任命袁粲、褚淵在中央秉政,但實際上接受宋明帝秘密遺命,就近輔佐新帝劉昱,掌控宮中內外大權的人物,是宋明帝最親信的側近權倖——王道隆與阮佃夫二人。

史書大多記載,劉彧晚年失去了生育能力,所以他的兒子們都是借腹生子取來的,他把諸弟新生的男嬰抱為自己的兒子,然後殺掉男嬰的生母。但是史家呂思勉認為這是《宋書》作者沈約,為了迎合當時南齊皇帝所捏造的誣蔑之詞,不足採信,而《南史》與《資治通鑑》則是沿用沈約的說法。呂思勉認為宋明帝生前因為猜忌諸弟而狠心殺弟、流放諸姪,不可能殺其父而養其子、流其兄而立其弟。曾懷疑《宋書》等史料的記載,認為宋明帝的皇后王貞風既然有兩個女兒,說明宋明帝可以生育,因此《宋書》應該是為了強化南齊的合法性,故意加工偽造史料,並被後人延用。

毛泽东在阅读南北朝的史书关于刘彧的传记中,写下了:(登基)“可谓奇矣”。

《宋書》記載劉彧:「少而和令,風姿端雅……好讀書,愛文義……及即大位,四方反叛,以寬仁待物,諸軍帥有父兄子弟同逆者,並授以禁兵,委任不易,故眾為之用,莫不盡力。平定天下,逆黨多被全,其有才能者,並見授用,有如舊臣。才學之士,多蒙引進,參侍文籍,應對左右」、「末年好鬼神,多忌諱,言語文書,有禍敗凶喪及疑似之言應回避者,數百千品,有犯必加罪戮」、「泰始、泰豫之際,更忍虐好殺,左右失旨忤意,往往有斮刳斷截者。時經略淮、泗,軍旅不息,荒弊積久,府藏空竭。內外百官,並日料祿俸;而上奢費過度,務為彫侈。每所造制,必為正御三十副,御次、副又各三十,須一物輒造九十枚,天下騷然,民不堪命……親近讒慝,剪落皇枝,宋氏之業,自此衰矣」

沈約評論劉彧:「太宗因易隙之情,據已行之典,剪落洪枝,願不待慮。既而本根無庇,幼主孤立,神器以勢弱傾移,靈命隨樂推回改。斯蓋履霜有漸,堅冰自至,所從來遠也」

北宋的司馬光評論:「(明)帝猜忍奢侈,宋道益衰」、「夫以孝武之驕淫、明帝之猜忍,得保首領以沒於牖下,幸矣,其何後之有?」

蕭梁的史家裴子野評論:「景和(劉子業)申之以淫虐,太宗易之以昏縱,師旅薦興,邊鄙蹙迫,人懷苟且,朝無紀綱,內寵方議其安,外物已睹其敗矣。」

\subsubsection{泰始}

\begin{longtable}{|>{\centering\scriptsize}m{2em}|>{\centering\scriptsize}m{1.3em}|>{\centering}m{8.8em}|}
  % \caption{秦王政}\
  \toprule
  \SimHei \normalsize 年数 & \SimHei \scriptsize 公元 & \SimHei 大事件 \tabularnewline
  % \midrule
  \endfirsthead
  \toprule
  \SimHei \normalsize 年数 & \SimHei \scriptsize 公元 & \SimHei 大事件 \tabularnewline
  \midrule
  \endhead
  \midrule
  元年 & 465 & \tabularnewline\hline
  二年 & 466 & \tabularnewline\hline
  三年 & 467 & \tabularnewline\hline
  四年 & 468 & \tabularnewline\hline
  五年 & 469 & \tabularnewline\hline
  六年 & 470 & \tabularnewline\hline
  七年 & 471 & \tabularnewline
  \bottomrule
\end{longtable}

\subsubsection{泰豫}

\begin{longtable}{|>{\centering\scriptsize}m{2em}|>{\centering\scriptsize}m{1.3em}|>{\centering}m{8.8em}|}
  % \caption{秦王政}\
  \toprule
  \SimHei \normalsize 年数 & \SimHei \scriptsize 公元 & \SimHei 大事件 \tabularnewline
  % \midrule
  \endfirsthead
  \toprule
  \SimHei \normalsize 年数 & \SimHei \scriptsize 公元 & \SimHei 大事件 \tabularnewline
  \midrule
  \endhead
  \midrule
  元年 & 472 & \tabularnewline
  \bottomrule
\end{longtable}


%%% Local Variables:
%%% mode: latex
%%% TeX-engine: xetex
%%% TeX-master: "../../Main"
%%% End:

%% -*- coding: utf-8 -*-
%% Time-stamp: <Chen Wang: 2021-11-01 15:03:58>

\subsection{后废帝劉昱\tiny(472-477)}

\subsubsection{生平}

劉\xpinyin*{昱}(463年3月1日-477年8月1日),字德融,小字慧震,是劉宋第八任皇帝,史稱「後廢帝」,宋明帝長子,母是貴妃陳妙登。在位五年暴戾荒誕,令朝野憂心不已,雖經歷過兩次宗室反亂仍未有改善,反倒更為放肆,終為楊玉夫等人所弒。

劉昱在大明七年正月辛丑日(463年3月1日)出生於衞尉府。宋明帝即位後,在消滅反對自己的晉安王劉子勛政權後,於十月戊寅日(466年11月17日)冊立劉昱為皇太子。翌年才正式取名為「昱」。劉昱五、六歲時才讀書,雖然他有過目不忘的本事,無論做金銀器飾還是衣帽都很優秀;即使未曾學過吹篪,拿到手竟也能吹奏。可是他卻不愛學習,只愛玩樂,當時主管他的官員無法制約,只好向明帝報告,明帝也只命令陳貴妃嚴加督促。泰始六年(470年)劉昱正式出居東宮,制訂了太子元會朝賀之禮及袞冕九章衣,並娶了出身濟陽江氏的江簡珪為太子妃。

泰豫元年四月己亥日(472年5月10日),明帝去世,遺詔以尚書令袁粲、護軍將軍褚淵、尚書右僕射劉勔、征西將軍荊州刺史蔡興宗及安西將軍郢州刺史沈攸之五人為顧命大臣。翌日(5月11日)劉昱正式繼位,但朝政實權其實一直都掌握在明帝倖臣阮佃夫、王道隆和楊運長等人手中,在大臣們和太后王貞風阻遏下,即位之初未能放肆而行。

元徽二年(474年)發生了叔父桂陽王劉休範縱兵進犯建康的事件,顧命大臣劉勔及權臣王道隆戰死,但亂事終在右衞將軍蕭道成指揮下獲平定。同年十一月,劉昱加元服,但此後劉昱就又見變態,其在東宮時已有作的隨意動手打人以及赤腳蹲坐的無禮行為故態復萌,眾人都無法制約他了。元徽三年(475年)秋冬之間劉昱曾多次出行,生母陳太妃則多次乘車跟著看顧他,但他卻愈來愈放肆,太妃也無能為力了。劉昱出行每每都丟低部隊,只帶著身邊隨從就四處去,一直到日落才回來。當時朝野對皇帝行為如斯都相當失望,反而都希望年長而又禮遇士人的建平王劉景素能夠入繼大統;不過陳太妃外戚集團以及阮佃夫等權臣卻憂心這樣會破壞他們的利益,故對景素處處防範。終在元徽四年(476年),劉景素於京口起兵,佃夫等人已作預備,藉蕭道成等人將之消滅。可是,景素被消滅後更讓劉昱變本加厲,竟去到每日都外出的程度,每天都和身邊隨從解僧智及張五兒互相追逐,時而夜出晨歸,時而晨出夜歸,跟著他的人都帶著鋋矛傷害經過的行人和牲畜,百姓不堪其滋擾乾脆日夜都閉戶,街上幾乎都沒行人了。廢帝又更加暴力,不再穿齊衣冠,反常常穿方便活動的小袴褶,隨便就動手打人;又為數十根大棍都各改名號,身邊一定帶著針椎、鑿子和鋸等刑具。

元徽五年(477年),阮佃夫眼見皇帝的行為,決定行廢立,他看準劉昱出遊愛丟低部隊的特點而聯結直閤將軍申伯宗、步兵校尉朱幼及于天寶,要趁他出遊時召其部隊回建康,據城反叛,接著抓住他將之廢掉,改立安成王劉準。不過,那次劉昱沒有如原定北出江乘,于天寶就將圖謀向劉昱告發,劉昱遂將佃夫等人誅殺。佃夫心腹張羊當時逃跑,但還是被抓住,劉昱竟親自駕車在承明門將他輾死。不久,劉昱又忌與阮佃夫交好的散騎常侍杜幼文等人,一次出遊時在幼文府外聽到傳出的音樂聲後決意殺掉他,連同司徒左長史沈勃及游擊將軍孫超之都在劉昱率領宿衞軍下被劉昱親手殺害。杜幼文兄長長水校尉杜叔文在玄武湖北被捕,劉昱就自己執矟騎馬,親自殺死叔文。其殺人取樂的沉溺程度更到只要一日不殺人就悶悶不樂的情況。於是百官都人人自危,朝不保夕。

雖然蕭道成在劉昱即位後先後平定劉休範及劉景素的起事,功勳和名聲都愈來愈大,但劉昱卻更猜忌他。劉昱曾經帶數十人突擊蕭道成的居所,當時蕭道成因暑熱而赤膊臥睡,劉昱命道成站著,然後將道成的腹部當箭靶,拉弓就要射,只因王天恩巧言勸說下才改以無箭頭的箭射。但及後劉昱仍想殺他,命人用木頭刻了道成的身形,在其腹畫箭靶,供自己和身邊隨從射擊,更曾襲擊時道成所在的領軍將軍府,想逼他出來接著殺害,但道成不動,劉昱無可奈何,但仍時時想手殺道成;陳太妃看不過眼,出言責罵,劉昱才收斂下來。

不過,時以「四貴」當政蕭道成因此此時已圖廢立,秘密連絡同為四貴之一的司徒袁粲、褚淵,向他們表示廢立之願。當時袁粲認為劉昱所為是年幼導致,反對廢立,使得蕭道成無法成事。蕭道成遂另結直閤將軍王敬則,而王敬則亦歸款道成,並與劉昱身邊的楊玉夫、陳奉伯等二十五人聯結,伺機行事。元徽五年七月七日(477年8月1日),劉昱再度出行,如常丟下部隊先走,期間張五兒的馬墮進湖中,劉昱一怒之下命人將這匹馬抓到明光亭前,親自將牠殺死宰割,及後又和隨從們玩羌人胡伎小樂,在山崗上鬥跳高,乘露車到青園尼寺。再晚點,劉昱到了新安寺偷狗,再到曇度道人處把狗煮了下酒,到了晚上才醉著回宮中。楊玉夫原本也算得劉昱親信,但劉昱卻突然憎惡了他,更向人說:「明日當殺小子,取肝肺。」這令楊玉夫很恐懼。就在這晚,劉昱命令玉夫等織女星經過時通報他,接著就和內人們穿針,穿完後不勝醉意睡了在仁壽殿東阿氊幄之中。由於廢帝出入無定,宮省不管日夜都是開著門,但人們害怕被被突然發怒的劉昱波及,都不敢出,故宮禁內外都沒有聯絡。玉夫等到二更時確定劉昱已熟睡,與楊萬年進帳以廢帝防身刀殺了他,享年十五歲。楊玉夫等人將劉昱的頭顱割下來,交給王敬則運送到蕭道成府前,大聲敲門通知劉昱之死,但蕭道成卻認為外頭是劉昱派來的軍隊,為了騙他開門而假稱皇上已死,堅持不開門。王敬則無可奈何,只好把劉昱的頭顱越牆丟進府內,蕭道成確認頭顱是劉昱本人之後,這才騎馬直衝皇宮,眾人知道劉昱被殺後都大呼萬歲。死後的劉昱被廢為蒼梧王。

雖然劉昱是宋明帝與貴妃陳妙登的長子,但是由於陳妙登曾經為李道兒的侍妾,所以劉昱的身世也一直被質疑。“民中皆呼废帝为李氏子。废帝后每自称李将军,或自谓李统”。

劉昱葬在丹阳郡秣陵县郊坛西。

\subsubsection{元徽}

\begin{longtable}{|>{\centering\scriptsize}m{2em}|>{\centering\scriptsize}m{1.3em}|>{\centering}m{8.8em}|}
  % \caption{秦王政}\
  \toprule
  \SimHei \normalsize 年数 & \SimHei \scriptsize 公元 & \SimHei 大事件 \tabularnewline
  % \midrule
  \endfirsthead
  \toprule
  \SimHei \normalsize 年数 & \SimHei \scriptsize 公元 & \SimHei 大事件 \tabularnewline
  \midrule
  \endhead
  \midrule
  元年 & 473 & \tabularnewline\hline
  二年 & 474 & \tabularnewline\hline
  三年 & 475 & \tabularnewline\hline
  四年 & 476 & \tabularnewline\hline
  五年 & 477 & \tabularnewline
  \bottomrule
\end{longtable}


%%% Local Variables:
%%% mode: latex
%%% TeX-engine: xetex
%%% TeX-master: "../../Main"
%%% End:

%% -*- coding: utf-8 -*-
%% Time-stamp: <Chen Wang: 2021-11-01 15:04:07>

\subsection{顺帝劉準\tiny(477-479)}

\subsubsection{生平}

宋順帝劉準(467年8月8日-479年6月23日),字仲謨,小字智觀,為劉宋的末代皇帝,為宋明帝劉彧的第三子。但《宋書》卻說宋明帝晚年陽痿,不能人道,所以劉準其實是刘彧之弟桂陽王劉休範与姬妾的儿子,由陈法容所抚养。不過這項記載被史家呂思勉駁斥,認為是《宋書》的作者沈約(南齊的史官)為了討好南齊皇帝,故意編造這樣的史料,好掩飾蕭道成篡位、弒君的罪惡。

泰始五年七月癸丑生,泰始七年封為安成王。477年,後廢帝劉昱被弒之後,劉準在蕭道成的擁立下,于元徽五年七月壬辰即位,是為宋順帝,並封蕭道成為相國、齊王;雖然劉準名義上是皇帝,但是權力都被蕭道成掌握。昇明三年(479年),蕭道成要求劉準禪讓,並且派部將王敬則率兵解送出宮。479年四月,劉準禅位與蕭道成,至此,劉宋滅亡。

蕭道成即位之後,封劉準為汝陰王,遷居丹陽並派兵監管。建元元年五月己未(479年6月23日),監視劉準的兵士聽得門外馬蹄聲雜亂,以為發生了變亂,便殺害劉準,得年12歲(虛龄13歲)。

「願後身世世勿復生天王家。」 被清初三大學者之一黃宗羲的知名政治思想著作《明夷待訪錄》首篇《原君》中引用:「昔人(指南朝宋順帝)願世世無生帝王家,⋯」便是表明若將國家當做產業看待,則全天下眾多覬覦這份產業的人是擋不住的,頂多傳個幾代,殺身之禍報應在他(國君)的子孫上。像是劉準被逼讓位所言、明思宗自殺前對公主所說「妳為何要出生在我們帝王之家?」這些話是多麼悲痛。

\subsubsection{昇明}

\begin{longtable}{|>{\centering\scriptsize}m{2em}|>{\centering\scriptsize}m{1.3em}|>{\centering}m{8.8em}|}
  % \caption{秦王政}\
  \toprule
  \SimHei \normalsize 年数 & \SimHei \scriptsize 公元 & \SimHei 大事件 \tabularnewline
  % \midrule
  \endfirsthead
  \toprule
  \SimHei \normalsize 年数 & \SimHei \scriptsize 公元 & \SimHei 大事件 \tabularnewline
  \midrule
  \endhead
  \midrule
  元年 & 477 & \tabularnewline\hline
  二年 & 478 & \tabularnewline\hline
  三年 & 479 & \tabularnewline
  \bottomrule
\end{longtable}


%%% Local Variables:
%%% mode: latex
%%% TeX-engine: xetex
%%% TeX-master: "../../Main"
%%% End:



%%% Local Variables:
%%% mode: latex
%%% TeX-engine: xetex
%%% TeX-master: "../../Main"
%%% End:

%% -*- coding: utf-8 -*-
%% Time-stamp: <Chen Wang: 2019-12-20 14:46:48>


\section{南齐\tiny(479-502)}

\subsection{简介}

齊(479年-502年)是中国历史上南北朝时期南朝第二个朝代。为蕭道成所建。史称南齐(以与北朝的北齐相区别)或萧齐。

以齐为国号,源于谶纬之说。《谶书》云:“金刀利刃齐\xpinyin*{刈}之”,意即“齐”将取代“宋”(因为南朝宋皇族为刘姓)。

南齐的开国之君萧道成是刘宋将领,在宋明帝在位期间担任右军将军。宋明帝去世后,他与尚书令袁粲共同掌管朝政。474年,萧道成平定江州刺史桂阳王刘休范的反叛,进爵为公,迁中领军将军,掌握了禁卫军,督五州军事。此时刘宋政权内鬥激烈,萧道成逐渐掌握大权。477年,后废帝刘昱在睡梦中被自己卫士所杀。萧道成立劉準继位。萧道成被封齐王。在这之后,萧道成铲除了忠于刘宋的袁粲、沈攸之等人。479年,萧道成迫使刘准禅让,刘宋灭亡,南齐建立。

萧道成崇尚节俭,反对奢靡,并以身作则,将宫殿、御用仪仗等凡用金、铜制作的器具全部用铁器替代,衣服上的玉佩、挂饰等统统取消。高帝萧道成在位时经常吊在嘴边的一句话是“使我治天下十年,当使黄金与土同价”,可见他的提倡节俭与身体力行。齐高帝提倡节俭的政策减轻了人民的负担。他也与北魏和好,维护边境安定。这使得新生的南齐政权迅速走向轨道。

482年,齐高帝萧道成去世,由长子萧赜继位,即齐武帝。当时,庶族地主为了免除所承担的赋役,往往向官吏行贿,在政府的黄籍上注入伪造的父祖爵位,改成免役免税的士族。刘宋以来,这种改注籍状,诈入仕流的庶族地主很多。萧道成在继位的第二年(480年),實施檢籍政策,专门设立校籍官和置令史,负责清查户籍。齐武帝登基后,继续其父的政策。那些被认为有假的户籍,都须退还本地,称为“却籍”。核查出本应服役纳赋而户籍上造假的,便恢复原来的户籍,继续承担赋役,称为“正籍”。檢籍政策虽然增加了赋税,却严重伤害了庶族地主的利益。

485年富阳唐寓之为此起兵叛乱,虽然这次叛乱被齐武帝迅速平息,但检籍的政策依然受到庶族的激烈反对。最终,在490年,齐武帝被迫妥协,宣布“却籍”无效,对因为“却籍”而被发配戍边的人民准许返归故乡,恢复刘宋升明时期户籍所注的原状。

尽管如此,齐武帝依然是一个英明的君主。他基本继承了齐高帝的作风,對外崇尚节俭,并且与北魏保持边界和平,使得南齐的国力大幅增强,史稱「永明之治」。

齐武帝在登基时,立长子萧长懋为皇太子。萧长懋在齐武帝在位期间去世,齐武帝选择皇太孙萧昭业作为继承人。493年,齐武帝去世,萧昭业继位。为武帝发丧之日,萧昭业刚送葬车出端门,便稱自己有病不能前去墓地。回宫后,马上召集乐工大奏胡曲表演歌舞,喇叭胡琴,声彻内外。萧昭业登基后,赏赐自己的亲信,一次赏赐就百数十万。每次看见宫中财宝,就自言自语:“我从前想你们一个也难得,看我今天怎么用你们!”他刚继位时,御库中总共有钱八亿万之巨,金银布帛不可胜数。萧昭业继位不到一年,已挥霍大半,赏赐给得意的左右、宫人。甚至打碎宫中宝物作为娱乐。他爱好斗鸡,甚至花数千钱来买斗鸡。

萧昭业贪图享受,任意賞賜,甚至寵幸宦官徐龍駒和侍衛周奉叔等人,弄權不法。辅政大臣萧鸾多次劝谏,萧昭业剛開始拒聽,後來才勉強答應蕭鸞,處死了徐龍駒和周奉叔。雖然蕭昭業一直打算铲除萧鸾,可是找不到願意支持他的宗王與大臣,只好暫時繼續忍受蕭鸞的專政。近卫军首领萧谌、萧坦之看到萧昭业私德日渐敗壞,都依附萧鸾,准备发动政变罢黜他。494年,萧鸾带兵入宫,诛杀萧昭业。萧鸾以太后的名义废萧昭业为郁林王,迎立其弟新安王萧昭文为帝。不到四个月,萧鸾废萧昭文为海陵王。萧鸾自立为帝,史称齐明帝。

齐明帝萧鸾是萧道成的侄子,在軍隊中累積很高的威望。他在武帝去世时受命辅佐萧昭业。在通过政变手段上台之后,为避免历史重演,他遂大肆诛杀齐高帝、齐武帝子孙以防后患。他性情阴险,每次誅殺宗王子孫時,都要對天焚香,痛哭流涕,表現自己是萬不得已的姿態,最后在生前杀光了高、武、文三帝诸子,还险些连高、武二帝的孙子们也杀尽。他極力崇尚节俭,但有時飲宴却頗為奢侈。他崇信道术,每次出行都要占卜吉凶。向南出行则宣称是向西行,向东出行则宣称是向北行。萧鸾晚年病重,对外却一直隐瞒病情,直到萧鸾特地下诏向官府征求银鱼以为药剂,外界才知道萧鸾患病。498年,萧鸾去世,由其子萧宝卷继承。

萧宝卷刚即位,其父留下的六個輔政大臣,分別是揚州刺史始安王蕭遙光、尚書令徐孝嗣、右僕射江祏、右將軍蕭坦之、侍中江祀、衛尉劉暄。是為「六貴」。未幾萧宝卷在一年內把「六貴」分別殺死。

萧宝卷性格讷涩,很少说话,不喜欢跟大臣接触,常常出宫闲逛,出游時常拆毁民居、驱逐居民,闹得民不聊生。后宫失火被焚,他就新造仙华、神仙、玉寿三座豪华宫殿。他又凿金为莲花,贴放于地,令宠妃潘氏行走其上,称为“步步生莲花」。

面对这种局面,始安王萧遥光、太尉江州刺史陳顯達和将军崔慧景连续发起三次兵变,试图起事推翻萧宝卷,但都被平定。萧宝卷因此更加放纵,派人毒杀平叛最力的尚書令,亦為宗室的萧懿,萧懿之弟、鎮守襄陽的萧衍遂与荆州行府事萧颖胄合谋拥戴荆州刺史南康王萧宝融为帝,即齐和帝,由萧衍起兵攻建康,于是南齐出现了两帝并立的局面。萧颖胄死后,萧衍成为义军的唯一首领。公元501年,萧衍攻陷建康,萧宝卷被将军王珍国所杀。萧衍奉萧昭业母王宝明称制,迎和帝回建康,期间以王宝明名义追废萧宝卷为东昏侯,进封自己为建安郡公、梁公、梁王,并诛杀明帝的儿子们和三个侄子,只有残疾的晋安王萧宝义和北逃的建安王萧宝夤得免(庐陵王萧宝源虽也未被杀但不久病死)。

502年,齐和帝还未到建康就被迫禅讓於萧衍。萧衍改国号为梁,是為梁武帝,南齐灭亡。后来萧衍在高帝孙萧子恪、萧子范面前将自己起兵夺位解释为替齐高帝、齐武帝子孙报仇。

527年,已成为北魏将领的萧宝夤在关中地区叛魏称帝,复建齐国,直至次年败亡。但因其在位短暂、立国时南齐亡国已久也没有掌控原属南齐的领地,他并不被视为南齐君主。

齊高帝(479年—482年在位)收集大量藝術作品。南京宮廷的肖像畫家謝赫,是當時重要的畫家。現存最早的中國繪畫理論,即出自其筆下,對後世影響甚鉅。座落於南京東方二十公里棲霞山(攝山)的棲霞寺,當地附近洞窟內有佛像源自南齊時期。

\subsection{宣帝生平}

萧承之(384年-447年),字嗣伯。刘宋南兰陵郡人 ,居晉陵縣武進縣東城里。南朝宋將領,官至右軍將軍。其子蕭道成後來篡宋自立,建立南齊,獲追諡為宣皇帝。

蕭承之少有大志,才力过人,同族的丹阳尹萧摹之及北兖州刺史萧源之都很看重他。萧承之初为建威將軍朱齡石的参军,在义熙九年(413年)隨朱齡石入蜀,滅掉譙蜀政權,並迁扬武将军、安固汶山二郡太守,任內善於撫綏。

刘宋元嘉初年,徙萧承之为武烈将军、济南太守。元嘉七年(430年),右将军到彦之北伐大败,北魏乘胜進攻南朝宋青州各郡縣,其中魏將叔孫建領兵进攻济南,萧承之率数百人拒战,击退魏军。及後魏军聚集济南城下,萧承之使偃兵开城门,用空城计。部下谏道:“贼众我寡,为何如此轻敌!”萧承之道:“今日孤悬在外守著困逼的城池已是很危急的了,若果還繼續示弱必然會被他們攻陷,只應該以強勢應付。”魏军疑有伏兵,遂引兵而去。

元嘉八年(431年),在到彥之敗後領兵北援的征南大将军檀道济在寿张(今山東東平縣西南)擊敗了叔孫健等軍,但因為缺糧而被逼在歷城(今山東濟南歷城區)撤兵,一直堅守的滑台(今山東滑縣)失去援兵之下最終失陷。青州刺史蕭思話知道濟退兵,擔憂遭到魏軍攻擊而決定棄守治所東陽城,蕭承之堅決反對卻不獲聽從;最終東陽城沒有被魏軍攻擊,但留下的大批糧草就被當地人焚毀。戰後宋文帝以萧承之有保全济南城之功,曾手書長沙王劉義欣表示打算以承之為兗州刺史,並命他將建議交給檀道濟參詳,然而因為承之與檀道濟素來沒有交情,事情就被擱下了。後轉輔國、鎮北中兵參軍及員外郎。

元嘉十年(432年),萧思话出任梁南秦二州刺史,萧承之为他的横野府司马、汉中太守。就在同年十一月,仇池王杨难当趁刺史交接間进攻汉中(今陝西漢中市),留守的原梁州刺史甄法护弃治所南城(今陝西褒中縣)逃走,萧思话那時才到襄阳(今湖北襄陽市),便停駐當地,改派萧承之率五百兵先赴梁州,又派西戎長史蕭汪之為其統率。蕭順之在路上招集兵眾,招得一千精兵前進。元嘉十一年(433年)正月,蕭順之到達磝頭 (今陝西石泉縣城),時楊難當在焚燒掠奪漢中過後領兵撤走,留下他任命的梁秦二州刺史趙溫守漢中,另有魏兴太守薛健守黄金山。蕭順之就先派陰平太守蕭坦攻陷黃金山下的鐵城戍,至二月時趙溫與薛健及仇池馮翊太守蒲早子合力進攻鐵城戍,反為蕭坦所敗; 隨後蕭承之獲得荊州刺史臨川王劉義慶派的裴方明部三千兵增援,遂將黃金戍也拿下。黃金戍是東漢末年佔據漢中的張魯所修築,南接漢中而北通驛道,兼城戍極其險要,承之據此作為基地後就向盤據漢中的趙溫等進行攻擊。趙溫逼於承之威脅而退守漢中小城,另外薛健及蒲甲子就守下桃城。承之遂與從襄陽到來的蕭思話主力部隊合力進攻,屢敗趙溫等人,其中承之曾在峨公山被呂平圍攻,但他在蕭汪之等支援下大敗對方。

楊難當派兒子楊和領萬多人增援趙溫等,並圍攻承之,至三月時和承之相持了達四十日,包圍圈厚達十多重。由於楊和援軍都穿上堅韌的犀甲,士兵近身戰時戈矛刀刃都不能刺穿,承之最終想出將長槊折成數尺長,指向敵軍後讓人在槊末用大斧鎚擊,一支就連殺數個犀甲兵。閏三月,敵軍無法抵抗之下唯有燒營逃走,承之就乘勝追擊至南城,再敗敵軍,俘殺甚多,最終收復了梁州。蕭承之亦因功獲授龍驤將軍。後蕭思話進寧朔將軍,承之亦隨轉寧朔司馬、兼漢中太守。

蕭承之後來入朝任太子屯騎校尉,隨後又調為江夏王劉義恭的司徒中兵參軍、龍驤將軍、南泰山太守 ,並封晉興縣五等男,食邑三百四十戶。及後又轉任右軍將軍。

蕭承之於元嘉二十四年(447年)去世,享年六十四。梁州人因為思念承之,在峨公山為他立庙祭祀。升明二年(478年),赠散骑常侍、金紫光禄大夫。其子萧道成建立南齐,承之獲追尊为宣皇帝,葬於永安陵。

%% -*- coding: utf-8 -*-
%% Time-stamp: <Chen Wang: 2021-11-01 15:04:36>

\subsection{高帝萧道成\tiny(479-482)}

\subsubsection{生平}

齐高帝萧道成(427年-482年4月11日),字绍伯,南北朝時代,南朝第二個皇朝南齊开国皇帝。

蕭道成出自兰陵萧氏,父親蕭承之仕於劉宋為右將軍,蕭道成亦在劉宋擔任軍官,宋明帝駕崩,蕭道成以右衛將軍領衛尉的名銜,與其他數位大臣受遺詔掌機要,為輔政大臣。劉昱即位皇帝後,桂陽王兼江州刺史劉休範叛變,為蕭道成領軍所平定,權勢日隆。

但青春期的劉昱喜好弓馬,時常殺人取樂,又濫殺無辜,一次突然跑到蕭道成家中,道成是個大胖子,「方坦而晝臥,腹大如瓠」。劉昱玩心頓起,覺得這麼大的肚子是個絕好的箭靶,拿起箭來就「彎弓欲射其腹」。蕭道成苦苦哀求,左右隨從也紛紛勸解說:蕭大人肚子這麼大,這麼好的目標,一箭就射死太可惜了!以後想射就沒有了!好說歹說之下,劉昱答應去掉箭鏃才射,結果一箭射中肚臍,歡呼高歌而去。剩下嚇得一身冷汗,死裡逃生心有餘悸的權臣蕭道成。

477年,劉昱被蕭道成的黨羽楊玉夫所弒,蕭道成改立宋順帝劉準,獨攬朝政。並在同年和隔年(478年),分別消滅了忠於宋室的尚書令袁粲、荊州刺史沈攸之,宰制全國。

479年,蕭道成篡宋自立為天子,國號齊,他為政務節儉,實施檢籍政策,清查詐入士族籍貫的寒人。

與南朝宋的劉裕一樣,在位僅四年去世,終年56歲。

除了在政治上的功業,蕭道成也廣覽經學、史學書籍,善寫作文、書法和下棋。

北宋的司馬光評論他:「高帝以功名之盛,不容於昏暴之朝,逆取而順守之,亦一時之良主也。」

\subsubsection{建元}

\begin{longtable}{|>{\centering\scriptsize}m{2em}|>{\centering\scriptsize}m{1.3em}|>{\centering}m{8.8em}|}
  % \caption{秦王政}\
  \toprule
  \SimHei \normalsize 年数 & \SimHei \scriptsize 公元 & \SimHei 大事件 \tabularnewline
  % \midrule
  \endfirsthead
  \toprule
  \SimHei \normalsize 年数 & \SimHei \scriptsize 公元 & \SimHei 大事件 \tabularnewline
  \midrule
  \endhead
  \midrule
  元年 & 479 & \tabularnewline\hline
  二年 & 480 & \tabularnewline\hline
  三年 & 481 & \tabularnewline\hline
  四年 & 482 & \tabularnewline
  \bottomrule
\end{longtable}


%%% Local Variables:
%%% mode: latex
%%% TeX-engine: xetex
%%% TeX-master: "../../Main"
%%% End:

%% -*- coding: utf-8 -*-
%% Time-stamp: <Chen Wang: 2021-11-01 15:04:48>

\subsection{武帝蕭賾\tiny(482-493)}

\subsubsection{生平}

齊武帝蕭\xpinyin*{賾}(440年-493年8月27日),字宣遠,齊高帝蕭道成長子,母刘智容。南齊第二任皇帝,病死時54歲,葬景安陵。年号永明。

武帝十分關心百姓疾苦,即位後,就下詔說:“比歲未稔,貧窮不少,京師二岸,多有其弊。遣中書舍人優量賑恤。”不久,再次下詔說,“水雨頻降,潮流薦滿,二岸居民,多所淹漬。遣中書舍人與兩縣官長優量賑恤。”

第二年,他又下詔酌情遣返軍中的囚徒,大赦囚犯,對於百姓中的鰥寡和貧窮之人,要加以賑濟。他提倡並獎勵農桑,災年時,還減免租稅。在位第四年,他下詔說: “揚、南徐二州,今年戶租三分二取見布,一分取錢。來歲以後,遠近諸州輸錢處,並減布直,匹準四百,依舊折半,以為永制。”

武帝還下令多辦學校,挑選有學問之人任教,以培育人們的德行。武帝以富國為先,不喜歡遊宴、奢靡之事,提倡節儉。他曾下令舉辦婚禮時不得奢侈。

齐武帝登基后,延續其父蕭道成的檢籍政策。那些被认为有假的户籍,都须退还本地,称为“却籍”。核查出本应服役纳赋而户籍上造假的,便恢复原来的户籍,继续承担赋役,称为“正籍”。檢籍政策虽然增加了赋税,却严重伤害了庶族地主的利益。

公元485年富阳唐寓之为此起兵叛乱,虽然这次叛乱被齐武帝迅速平息,但检籍的政策依然受到庶族的激烈反对。最终,在490年,齐武帝被迫妥协,宣布“却籍”无效,对因为“却籍”而被发配戍边的人民准许返归故乡,恢复刘宋升明时期户籍所注的原状。

武帝對於其後事,特意下詔說:“我識滅之後,身上著夏衣,畫天衣,純烏犀導,應諸器悉不得用寶物及織成等,唯裝復裌衣各一通。常所服身刀長短二口鐵環者,隨我入梓宮。祭敬之典,本在因心,東鄰殺牛,不如西家禴祭。我靈上慎勿以牲為祭,唯設餅、茶飲、幹飯、酒脯而已。天下貴賤,咸同此制。未山陵前,朔望設菜食。陵墓萬世所宅,意嘗恨休安陵未稱,今可用東三處地最東邊以葬我,名為景安陵。喪禮每存省約,不須煩民。百官停六時入臨,朔望祖日可依舊。諸主六宮,並不須從山陵。內殿鳳華、壽昌、耀靈三處,是吾所治制。”

齊武帝時,還與北魏通好,邊境比較安定。高帝和武帝的清明統治使江南經濟也有了一定的發展,社會也暫時安定。

大體而言,齐武帝是一个英明剛斷的君主。他基本继承了齐高帝的作风,對外崇尚节俭,努力實施富國政策。

\subsubsection{永明}

\begin{longtable}{|>{\centering\scriptsize}m{2em}|>{\centering\scriptsize}m{1.3em}|>{\centering}m{8.8em}|}
  % \caption{秦王政}\
  \toprule
  \SimHei \normalsize 年数 & \SimHei \scriptsize 公元 & \SimHei 大事件 \tabularnewline
  % \midrule
  \endfirsthead
  \toprule
  \SimHei \normalsize 年数 & \SimHei \scriptsize 公元 & \SimHei 大事件 \tabularnewline
  \midrule
  \endhead
  \midrule
  元年 & 483 & \tabularnewline\hline
  二年 & 484 & \tabularnewline\hline
  三年 & 485 & \tabularnewline\hline
  四年 & 486 & \tabularnewline\hline
  五年 & 487 & \tabularnewline\hline
  六年 & 488 & \tabularnewline\hline
  七年 & 489 & \tabularnewline\hline
  八年 & 490 & \tabularnewline\hline
  九年 & 491 & \tabularnewline\hline
  十年 & 492 & \tabularnewline\hline
  十一年 & 493 & \tabularnewline
  \bottomrule
\end{longtable}


%%% Local Variables:
%%% mode: latex
%%% TeX-engine: xetex
%%% TeX-master: "../../Main"
%%% End:

%% -*- coding: utf-8 -*-
%% Time-stamp: <Chen Wang: 2019-12-20 14:49:17>

\subsection{鬱林王\tiny(493-494)}

\subsubsection{生平}

蕭昭業(473年-494年9月7日),字元尚,小名法身,南朝齊的第三任皇帝,文惠太子蕭長懋之長子,齐武帝之孙。

蕭昭業雖然工於隸書,美容止而獲得祖父與父親的喜愛,但是蕭昭業本人是一個陽奉陰違的人。父亲病重时,他在人前表现出悲伤得连自己的健康都受损的样子,见者无不动容,而一旦回家,即表现出喜状,还要一位杨姓巫婆诅咒他的父亲和祖父,以让自己可以尽早登极。萧长懋不久死去,萧昭业认为杨氏诅咒得力,予以赏赐。武帝对此全然不觉,立蕭昭業為皇太孫。武帝不久也病倒了,萧昭业继续在人前作悲伤状,实则兴奋,在给妻何婧英的信中写了一个大“喜”字,还在其周围写了36个小“喜”字。

齐武帝病重时,和武帝子竟陵王蕭子良相好的中书郎王融想改立萧子良为新君,取代萧昭业,被武帝堂弟西昌侯蕭鸞挫败。493年末齊武帝死後,蕭昭業即位,改年號為隆昌。同時由蕭子良與蕭鸞輔政。

萧昭业尊母王宝明为皇太后,封妻何婧英为皇后。他即位之後便原型畢露,不但濫發賞賜,又與庶母霍氏通姦,並且過著十分浪費奢靡的生活,毫無一國之君的姿態,并架空涉嫌夺位的萧子良,赐死王融,朝政都委託西昌侯蕭鸞處理。还在丧期,就恢复奏乐。宠幸偏爱中书舍人綦毋珍之、朱隆之、直将军曹道刚、周奉叔、宦官徐龙驹等人,纵容近臣公然卖官。萧昭业曾不满萧鸾专权,对徐龙驹说:“我和萧锵(萧昭业叔祖父,鄱阳王)商议杀萧鸾,萧锵不同意,只能先让萧鸾专权一阵子了。”萧鸾迫使萧昭业下令诛杀皇后何婧英的男宠杨珉及徐龙驹;将军萧谌、萧坦之秘密投靠萧鸾,将萧昭业的动向告诉他;萧鸾又设计寻罪名杀死萧昭业近臣周奉叔、杜文谦、綦毋珍之;而萧昭业却因萧子良去世而放松警惕。

萧昭业与何婧英的叔父何胤图谋杀萧鸾,但何胤不敢。萧昭业不再委萧鸾以重任。萧鸾担心生变,与萧谌、萧坦之合谋政变,由二人等人于省诛杀曹道刚、朱隆之等,亲自率兵自尚书省入云龙门進宮。萧昭业正在寿昌殿和霍氏裸体相对,闻变下令关闭宫门,派宦官登兴光楼察看。他不知萧谌、萧坦之已叛变,向萧谌求助,见萧谌杀入殿中,才知。宿卫将士准备抵抗萧谌时,萧昭业逃向爱姬徐氏房,拔剑自刺,不果,以帛缠颈,乘舆出延德殿,宿卫见状要护驾,萧昭业却没有发话,出西弄,被萧谌追上弒殺,尸体运到徐龙驹府中。萧鸾以太后名义下诏追廢蕭昭業為鬱林王,葬以亲王礼。

\subsubsection{隆昌}

\begin{longtable}{|>{\centering\scriptsize}m{2em}|>{\centering\scriptsize}m{1.3em}|>{\centering}m{8.8em}|}
  % \caption{秦王政}\
  \toprule
  \SimHei \normalsize 年数 & \SimHei \scriptsize 公元 & \SimHei 大事件 \tabularnewline
  % \midrule
  \endfirsthead
  \toprule
  \SimHei \normalsize 年数 & \SimHei \scriptsize 公元 & \SimHei 大事件 \tabularnewline
  \midrule
  \endhead
  \midrule
  元年 & 494 & \tabularnewline
  \bottomrule
\end{longtable}


%%% Local Variables:
%%% mode: latex
%%% TeX-engine: xetex
%%% TeX-master: "../../Main"
%%% End:

%% -*- coding: utf-8 -*-
%% Time-stamp: <Chen Wang: 2021-11-01 15:05:05>

\subsection{海陵王蕭昭文\tiny(494)}

\subsubsection{生平}

蕭昭文(480年-494年),字季尚,南朝齊的第四任皇帝,在位僅四個月。母亲为宫人许氏。

蕭昭文為文惠太子蕭長懋的第二子,永明四年(486年)閏正月封為臨汝公,邑千五百户。初为辅国将军、济阳太守。十年(492年)正月,转持节、督南豫州诸军事、南豫州刺史,将军如故。十一年(493年),进号冠军将军。493年萧昭文兄长鬱林王蕭昭業即位後,十月封昭文為新安王。

隆昌元年(494年)闰四月,以萧昭文为扬州刺史。七月,尚书令、镇军大将军、西昌侯蕭鸞刺死萧昭业,拥立萧昭文為帝,改年號為延興,大赦。但是政事俱操於蕭鸞之手,萧昭文的生母许氏也没有得到尊封。

萧昭文刚登基,萧鸾就被任为骠骑大将军、录尚书事、扬州刺史、宣城郡公。九月,萧鸾以萧昭文之名诛杀高帝、武帝诸子,先杀司徒鄱阳王萧锵、中军大将军随郡王萧子隆、南兖州刺史安陆王萧子敬。江州刺史晋安王萧子懋起兵,萧鸾假黄钺,萧子懋败亡,萧鸾又杀湘州刺史南平王萧锐、郢州刺史晋熙王萧銶、南豫州刺史宜都王萧铿。十月,萧鸾被进为太傅,领大将军、扬州牧,加殊礼,进爵为王。萧鸾又杀中军将军桂阳王萧铄、抚军将军衡阳王萧钧、侍中秘书监江夏王萧锋、镇军将军建安王萧子真、左将军巴陵王萧子伦、司徒庐陵王萧子卿,且几乎杀死萧昭文的弟弟荆州刺史萧昭秀。这时萧鸾已掌控萧昭文的起居,有一次萧昭文想吃蒸鱼菜,太官令答没有萧鸾的命令,不给他吃。萧鸾诸子年幼,于是任兄子萧遥光、萧遥欣、萧遥昌以要职。当月,萧昭文被蕭鸞以嫡母皇太后王宝明名义以有病为由廢黜為海陵王,以东汉东海王劉彊之礼安置,给虎贲、旄头、画轮车,供奉很厚。次月,萧昭文便被蕭鸞派去看病的医生殺害,谥号恭。萧鸾以劉彊之礼厚葬,但并没有用皇帝礼。北魏趁机大举入侵。

\subsubsection{延兴}

\begin{longtable}{|>{\centering\scriptsize}m{2em}|>{\centering\scriptsize}m{1.3em}|>{\centering}m{8.8em}|}
  % \caption{秦王政}\
  \toprule
  \SimHei \normalsize 年数 & \SimHei \scriptsize 公元 & \SimHei 大事件 \tabularnewline
  % \midrule
  \endfirsthead
  \toprule
  \SimHei \normalsize 年数 & \SimHei \scriptsize 公元 & \SimHei 大事件 \tabularnewline
  \midrule
  \endhead
  \midrule
  元年 & 494 & \tabularnewline
  \bottomrule
\end{longtable}


%%% Local Variables:
%%% mode: latex
%%% TeX-engine: xetex
%%% TeX-master: "../../Main"
%%% End:

%% -*- coding: utf-8 -*-
%% Time-stamp: <Chen Wang: 2021-11-01 15:05:10>

\subsection{明帝蕭鸞\tiny(494-498)}

\subsubsection{生平}

齊明帝蕭鸞(452年-498年),字景栖,小名玄度,廟號高宗,谥明皇帝,南齊的第五任皇帝。為始安貞王蕭道生之子、齊高帝蕭道成之姪。他在494年至498年期間在位,共5年。

蕭鸞自小父母雙亡,由蕭道成撫養,蕭道成對其視若己出。宋順帝時,蕭鸞擔任安吉令,以嚴格而聞名;後遷淮南、宣城太守,輔國將軍。齊高帝時封西昌侯、任郢州刺史;齊武帝蕭賾時升任侍中,領驍騎將軍。蕭賾死前,萧鸾挫败中书郎王融改立蕭賾次子竟陵王萧子良为新君的图谋。蕭賾以蕭鸞為輔政,輔佐蕭昭業。

自從文惠太子蕭長懋於493年死後,蕭鸞便有爭奪帝位之心;蕭鸞迫使萧昭业处决近臣杨珉、徐龙驹,又寻罪名杀萧昭业近臣周奉叔、杜文谦、綦毋珍之。萧昭业曾与叔祖父萧锵合谋杀萧鸾,萧锵反对,故未果。后萧昭业又和皇后何婧英叔父何胤谋杀萧鸾,何胤不敢,但萧昭业也不再委萧鸾以重任。494年,萧鸾担心有变,与投靠自己的将领萧谌、萧坦之等发动政变,由萧谌殺蕭昭業,并以太后名义追废为鬱林王,改立其弟蕭昭文,愈发控制朝政,甚至控制了萧昭文的饮食,萧昭文曾想吃蒸鱼菜,太官令却因为没有萧鸾的许可而不给;不久萧鸾又廢蕭昭文為海陵王自立為帝。蕭鸞於494年即位後,便壓制宗室力量,並以典籤監視諸王;並且从萧昭文任期开始就大肆屠殺蕭道成、蕭賾二帝诸子,先杀年长者,临终时又杀年幼者,全都誅滅。蕭鸞任內長期深居簡出,要求節儉,停止各地向中央的進獻,並且停止不少工程。

蕭鸞晚年病重,相當尊重道教與厭勝之術,將所有的服裝都改為紅色;而且蕭鸞還特地下詔向官府徵求銀魚以為藥劑,外界才知道蕭鸞患病。498年蕭鸞病故,葬於興安陵。

《南齊書》這樣形容他:“帝明審有吏才,持法無所借,制御親幸,臣下肅清。驅使寒人不得用四幅繖,大存儉約。罷世祖所起新林苑,以地還百姓。廢文帝所起太子東田,斥賣之。永明中輿輦舟乘,悉剔取金銀還主衣庫。太官進御食,有裹蒸,帝曰:‘我食此不盡,可四片破之,餘充晚食。’而世祖掖庭中宮殿服御,一無所改。”

\subsubsection{建武}

\begin{longtable}{|>{\centering\scriptsize}m{2em}|>{\centering\scriptsize}m{1.3em}|>{\centering}m{8.8em}|}
  % \caption{秦王政}\
  \toprule
  \SimHei \normalsize 年数 & \SimHei \scriptsize 公元 & \SimHei 大事件 \tabularnewline
  % \midrule
  \endfirsthead
  \toprule
  \SimHei \normalsize 年数 & \SimHei \scriptsize 公元 & \SimHei 大事件 \tabularnewline
  \midrule
  \endhead
  \midrule
  元年 & 494 & \tabularnewline\hline
  二年 & 495 & \tabularnewline\hline
  三年 & 496 & \tabularnewline\hline
  四年 & 497 & \tabularnewline\hline
  五年 & 498 & \tabularnewline
  \bottomrule
\end{longtable}

\subsubsection{永泰}

\begin{longtable}{|>{\centering\scriptsize}m{2em}|>{\centering\scriptsize}m{1.3em}|>{\centering}m{8.8em}|}
  % \caption{秦王政}\
  \toprule
  \SimHei \normalsize 年数 & \SimHei \scriptsize 公元 & \SimHei 大事件 \tabularnewline
  % \midrule
  \endfirsthead
  \toprule
  \SimHei \normalsize 年数 & \SimHei \scriptsize 公元 & \SimHei 大事件 \tabularnewline
  \midrule
  \endhead
  \midrule
  元年 & 498 & \tabularnewline
  \bottomrule
\end{longtable}


%%% Local Variables:
%%% mode: latex
%%% TeX-engine: xetex
%%% TeX-master: "../../Main"
%%% End:

%% -*- coding: utf-8 -*-
%% Time-stamp: <Chen Wang: 2021-11-01 15:05:19>

\subsection{萧宝卷蕭寶卷\tiny(498-501)}

\subsubsection{生平}

蕭寶卷(483年-501年),字智藏,原名蕭明賢,南齊的第六代皇帝,因殘暴荒淫被殺,後追封東昏侯;為齊明帝蕭鸞第二子。蕭寶卷也被認為是中國歷史上昏庸荒淫的皇帝之一。

蕭寶卷的生母劉惠端是蕭鸞的正妻,早亡,由潘妃撫養。他年少時不喜讀書,以捕老鼠為樂。498年,蕭寶卷在蕭鸞死後即位,這時蕭寶卷才十六歲,並且封潘妃之侄女潘玉兒為貴妃,潘貴妃生下一個女兒,封為公主,但公主卻百日而殤,蕭寶卷制斬衰絰杖,積旬不聽音樂,衣悉粗布。

蕭寶卷性格內向,很少說話,不喜歡跟大臣接觸,常常出宮閒逛,每次出遊都一定要拆毀民居、驅逐居民,並且興建仙華、神仙、玉壽諸殿,並且大量賞賜臣下,造成國家的財政困難。而且蕭寶卷也殺害不少的大臣,即位之後便殺害六位顧命大臣表叔右僕射江祏、侍中江祀、堂兄始安王蕭遙光(作乱事败被杀)、司空徐孝嗣、右將軍蕭坦之、舅父領軍將軍劉暄及重臣曹虎、沈文季等人。也由於蕭寶卷的昏暴,導致大将裴叔业以重镇寿阳投魏;太尉陳顯達與將軍崔慧景先後起兵叛亂,但都兵敗被殺。萧宝卷胞弟江夏王萧宝玄响应崔慧景,几乎被其拥立为帝,萧宝卷平乱后杀萧宝玄。

蕭寶卷平定叛亂之後更加昏暴,除了與潘玉奴、宦官梅蟲兒等人日夜玩樂之外,並且派人毒殺平定崔慧景叛亂最力的尚書令蕭懿,結果導致蕭懿之弟蕭衍發兵進攻建康。萧衍並且与荆州行府事萧颖胄合作,改立皇弟荆州刺史南康王蕭寶融於江陵稱帝,遥废萧宝卷为庶人,又封涪陵王;蕭寶卷就在蕭衍發兵進攻建康的動亂中,被將軍王珍國所殺。

之後蕭寶卷被追廢為庶人,有司请求追封其为零阳侯,不许,请求追封涪陵王,获准。后蕭衍又將其追降為東昏侯,谥号炀。

\subsubsection{永元}

\begin{longtable}{|>{\centering\scriptsize}m{2em}|>{\centering\scriptsize}m{1.3em}|>{\centering}m{8.8em}|}
  % \caption{秦王政}\
  \toprule
  \SimHei \normalsize 年数 & \SimHei \scriptsize 公元 & \SimHei 大事件 \tabularnewline
  % \midrule
  \endfirsthead
  \toprule
  \SimHei \normalsize 年数 & \SimHei \scriptsize 公元 & \SimHei 大事件 \tabularnewline
  \midrule
  \endhead
  \midrule
  元年 & 499 & \tabularnewline\hline
  二年 & 500 & \tabularnewline\hline
  三年 & 501 & \tabularnewline
  \bottomrule
\end{longtable}


%%% Local Variables:
%%% mode: latex
%%% TeX-engine: xetex
%%% TeX-master: "../../Main"
%%% End:

%% -*- coding: utf-8 -*-
%% Time-stamp: <Chen Wang: 2019-12-20 14:53:11>

\subsection{和帝\tiny(501-502)}

\subsubsection{生平}

齊和帝蕭寶融(488年-502年),字智昭,南齊的末代皇帝,齊明帝蕭鸞第八子,母萧鸾原配追赠皇后刘惠端。

494年十一月被封為隨郡王,499年正月改封為南康王並任荊州刺史,駐守江陵。

501年三月,蕭衍發兵攻打蕭寶卷,並且与行荆州府事萧颖胄合谋,让萧宝融自称相国,后又立蕭寶融為皇帝,以萧宝融名义废萧宝卷为庶人,萧颖胄辅佐萧宝融守江陵,萧衍东征。萧衍手下劝他把萧宝融带到襄阳以免便宜了挟天子以令诸侯的萧颖胄,萧衍拒绝,认为此次起兵如果失败了此举就没有意义,如果成功了再寻找夺权的机会也不迟。后来萧颖胄病死,萧衍成为反对萧宝卷的唯一首领。蕭衍進入建康後,便將蕭寶融於502年接入建康,在萧宝融到来前奉废帝萧昭业的母后王宝明称制。同年,王宝明封蕭衍為梁王,不久蕭衍以王宝明名義殺害湘東王蕭寶晊兄弟,後來又殺掉齊明帝其他的兒子。不久蕭寶融便在到达建康前在姑孰被迫奉王宝明诏令禪位予蕭衍,南齊到此滅亡。

蕭衍即位後封蕭寶融為巴陵王,在姑孰建立宮室供其居住;第二天蕭寶融就被蕭衍所殺。

\subsubsection{中兴}

\begin{longtable}{|>{\centering\scriptsize}m{2em}|>{\centering\scriptsize}m{1.3em}|>{\centering}m{8.8em}|}
  % \caption{秦王政}\
  \toprule
  \SimHei \normalsize 年数 & \SimHei \scriptsize 公元 & \SimHei 大事件 \tabularnewline
  % \midrule
  \endfirsthead
  \toprule
  \SimHei \normalsize 年数 & \SimHei \scriptsize 公元 & \SimHei 大事件 \tabularnewline
  \midrule
  \endhead
  \midrule
  元年 & 501 & \tabularnewline\hline
  二年 & 502 & \tabularnewline
  \bottomrule
\end{longtable}


%%% Local Variables:
%%% mode: latex
%%% TeX-engine: xetex
%%% TeX-master: "../../Main"
%%% End:



%%% Local Variables:
%%% mode: latex
%%% TeX-engine: xetex
%%% TeX-master: "../../Main"
%%% End:

%% -*- coding: utf-8 -*-
%% Time-stamp: <Chen Wang: 2019-12-23 14:18:01>


\section{南梁\tiny(502-557)}

\subsection{简介}

梁(502年-557年),又稱南梁,是中国历史上南北朝时期南朝的第三个朝代,由南齐宗室蕭衍称帝,改國號為梁,都建康(今江苏南京)。以蕭衍封地在古梁郡,故国号为梁。因为皇帝姓萧,又称萧梁。

南齐末年,皇帝萧宝卷行事荒淫,并大肆诛杀大臣,杀死尚书令萧懿并追杀其兄弟,萧懿的弟弟雍州刺史萧衍因此联合行荆州府事萧颖胄起兵,拥立荆州刺史南康王萧宝融为帝。萧衍攻入京城,萧宝卷被杀。因萧颖胄已病逝,萧衍成为萧宝融势力唯一领导人物,以太后王宝明名义加封自己为建安郡公、梁公、梁王及杀死萧宝卷的儿子、兄弟、堂兄弟,仅萧宝卷庶兄萧宝义因残疾被留作二王三恪、胞弟萧宝夤北逃而幸免(庶弟萧宝源虽未被杀,但也很快病死)。他虽迎萧宝融进京,但在萧宝融进京前即迫其禅位。

萧衍本是南齐宗室,但与皇室关系疏远,为了给自己制造登基的合法性,他不继承齐朝皇统而是以封号自建国号梁,声称自己推翻萧宝卷之举是为齐高帝、齐武帝子孙报仇,另立政权也非夺取齐高帝、齐武帝的天下。

梁武帝蕭衍於代齐即位後厲行儉約,令南梁前期國勢頗盛。然而,武帝迷信佛教,曾三次出家為僧,令朝臣須用大量金錢為他贖身。他又大建佛寺及翻譯佛經,令佛教大盛,可是佛事太過損害经济,令梁朝國勢開始衰弱。

其後东魏叛將侯景投降,武帝本欲借侯景之力北伐,侯景見南梁國勢衰弱,加上武帝出賣自己,遂有反叛之意,終於548年爆發侯景之亂。皇侄临贺王萧正德曾被过继给武帝,却未能被立为皇太子且回归本宗,心怀不满,与侯景勾结,侯景许诺拥立其为帝。侯景围攻建康,包括皇子宗室们所统领的各地兵马多观望不救,萧正德奉命抵抗时率军倒戈。侯景攻克建康外城后,立萧正德为帝。549年侯景攻克建康城,以武帝名义解散勤王军队,废杀萧正德,武帝亦被其囚禁餓死,這場亂事亦是梁朝滅亡的關鍵。

武帝死后,侯景立皇太子萧纲为傀儡简文帝,把持朝政。同时,不服从侯景的南梁地方势力彼此也互相攻伐及求援于北齐、西魏。武帝第七子湘东王萧绎攻杀萧统子河东王萧誉,迫使其弟岳阳王萧詧以襄阳降西魏,受封梁王;武帝第六子邵陵王萧纶降北齐,亦受封梁王,但因萧绎亦与北齐结盟而失去北齐支持,遭萧绎、侯景打击,最终被西魏所杀;武帝幼子武陵王萧纪据益州称帝。其他地方势力亦有被侯景所灭者。北齊和西魏相繼乘机夺取淮南和中國西南大片土地,梁朝国力急剧衰败,只能偏安長江以南。双方互有胜负,但总體来说在军事上北朝转強,南朝逐渐转弱。

551年,侯景迫简文帝禅位给武帝故太子萧统孙豫章王萧栋,又杀简文帝,同年又迫使萧栋禅位,改国号为汉。552年,萧绎灭侯景,在江陵称帝,史称梁元帝;指示收复建康的手下杀死萧栋兄弟,没有还都建康。年底,他歼灭萧纪势力,但期间他联合西魏致使益州被西魏所得。

553年,北齐出兵意图拥立武帝侄湘阴侯萧退为帝,未果。

因梁元帝与西魏交恶,555年,西魏攻克江陵,迫使梁元帝父子投降,然后杀之,在江陵立萧詧为帝;元帝诸子仅晋安王萧方智幸存,大将王僧辩等在建康拥立他为梁王,以太宰承制,准备拥立为帝,却因被北齐所败,被迫同意北齐所请,改立萧懿子萧渊明为帝,萧渊明亦应王僧辩所请,立萧方智为皇太子。另一大将陈霸先随即以王僧辩投降北齐、抛弃先帝之子为由袭杀王僧辩,迫使萧渊明禅位给萧方智,萧方智史称梁敬帝。陈霸先代表敬帝对北齐称臣,后又击败北齐,亦掌握了朝中大权。

梁敬帝时,將皇位禪讓給陈霸先,陈霸先遂改国號為陈。梁朝前后共10帝,歷時55年。

梁敬帝將皇位禪讓給陳霸先之後,梁朝仍有兩支殘餘勢力與陳朝對抗,分別成為北朝東西兩個政權的傀儡,力爭正統地位。

後梁(555-587)、又稱西梁(據江陵〔今荊州市以南〕地區);傳承三帝蕭詧、蕭巋及蕭琮:西魏及北周支持蕭衍之孫蕭詧。始於西魏恭帝拓跋廓於555年在江陵立蕭詧為梁皇帝以對抗陳霸先;之後北周繼續支持後梁。传三帝共33年,587年亡於隋朝。

東梁(557-573)、據長江中上游地區;傳承蕭莊(557-560)及王琳(560-573):北齊文宣帝高洋則扶植梁元帝孫蕭莊繼承梁朝,對抗陳霸先的篡奪,根據地郢州(今荊州市以北),據有長江中上游地區,主事者為王琳。557年起蕭莊稱帝至560年被陳朝擊敗投奔北齊。後王琳又再據壽陽抵抗至573年勢力方被陳朝消滅,前後共16年。

%% -*- coding: utf-8 -*-
%% Time-stamp: <Chen Wang: 2019-12-20 18:00:48>

\subsection{武帝\tiny(502-549)}

\subsubsection{生平}

梁武帝萧衍(464年-549年),字叔达,小字练儿。南兰陵中都里人(今江苏常州市武进区西北)。南北朝時代南梁開国皇帝,廟號高祖。

萧衍是南齐宗室,亦是蘭陵蕭氏的世家子弟,出生在秣陵(今南京),父亲蕭順之是齐高帝的族弟,封临湘县侯,官至丹阳尹。母张尚柔。萧衍少年時受過良好的儒家教育,私德頗佳、亦不太注重個人享受,是文學名士竟陵八友之一。原為權臣,在其兄長蕭懿被害後,逐漸有帝位之野心,南齐中兴二年(502年),齐和帝被迫禅位于萧衍,南梁建立,是為梁武帝。稱帝後的萧衍改善許多前朝留下的弊政,並多次主持整理經史文書。然而晚年的他多次出家,傾力資助佛教發展直接導致國庫空虛,在侯景之乱爆發後絕食而亡。梁武帝萧衍在位时间長达48年,在南北朝皇帝中名列第一。

萧衍年轻時多才多藝,學識廣博。他的政治、軍事才能,在南朝諸帝中堪稱翹楚,不在另三位開國皇帝之下。在南齊武帝永明年間,他經常在當時的文化中心、竟陵王蕭子良的西邸出入,與沈約、謝脁等人合稱「竟陵八友」,在這期間發表了許多詩作,在學術研究和文學創作上皆有所成就。《梁書》紀載他:“六藝備閑,棋登逸品,陰陽緯候,卜筮占決,並悉稱善。……草隸尺牘,騎射弓馬,莫不奇妙。”他很好學,從小就受到正統的儒家教育,“少時習周孔,弱冠窮六經”,即位之後,“雖萬機多務,猶卷不輟手,燃燭側光,常至午夜”。

齐武帝驾崩时,萧衍没有参与王融意图拥立萧子良的政变,反支持皇太孙萧昭业登基。后又助权臣萧鸾篡位,是為齊明帝。齊明帝的皇叔荆州刺史萧子隆性温和、有文才,明帝欲徵之回朝,恐其不从。萧衍说:“随王(萧子隆)虽有美名,其实能力庸劣,手下没有智谋之士,爪牙只有司马垣历生、武陵太守卞白龙,而且二人唯利是从,若以显职相诱,都会来;随王只需要折简就能召来了。”齊明帝从之,徵垣历生为太子左卫率、卞白龙为游击将军,二人果然都到任。明帝再召萧子隆为侍中、抚军将军,后杀之。

齊明帝死後,繼任的東昏侯蕭寶卷暴虐無道,爆發的亂事在各地将帥們的努力下皆被平息,當中最為得力的是蕭衍的兄長、時任雍州(今湖北省襄陽)刺史的蕭懿。永元二年(500年),蕭懿被誣告謀反,遭東昏侯賜死,由蕭衍接任雍州刺史一職。喜好樂府詩的蕭衍上任後派人搜集當地的民歌,恢復自晉朝以來就已停止的民歌搜集工作。同時他積極招兵,暗中尋找機會推翻東昏侯。他秘密的派人在襄陽大伐竹木,沉於湖底,直到一年後舉兵之時,馬上派人去湖中打撈起事先砍伐好的竹木,並讓早已召集好的數千工匠在最短時間內建造戰船,此即後世 "伐竹沉木" 的典故。

中興元年(501年),蕭衍領兵攻郢城,圍攻兩百餘日,城破,「積屍床下而寢其上,比屋皆滿。」同年十二月,蕭衍發兵攻占首都建康,改立南康王蕭寶融於江陵稱帝,是為齊和帝;東昏侯在政變中被將軍王珍國所殺。中兴二年(502年),皇太后王寶明临朝称制,之後蕭衍受齊和帝禪讓登基,改國號為天監元年(502年),是為梁武帝。

梁武帝昔日的好友沈約、范雲等世族出身的名門後人在梁朝當上宰相,與前朝重臣蕭秀等人合力推動各種改革,改正南齊時施政上的種種問題。此外,武帝登基後對樂府詩的興趣不減當年,仍參與樂府詩的創作及編修。在他的影響和提倡下,梁朝文化的發展達到了東晉以來最繁榮的階段。《南史》作者李延壽評價道:“自江左以來,年逾二百,文物之盛,獨美於茲。”

520年,梁武帝改元普通,這一年被中國歷史學家視為南朝梁發展的分水嶺。在這年開始,梁武帝開始篤信佛法,多次舍身出家。

普通八年(527年)三月八日,梁武帝第一次前往同泰寺舍身出家,三日後返回,大赦天下,改年号大通,是為大通元年(527年)。同年,隸領軍曹仲宗伐渦陽(今安徽蒙城),在關中侯陳慶之的奮鬥下梁軍大敗北魏軍、俘斬甚眾,又乘勝進擊至城父。梁武帝詔下令渦陽之地設置西徐州,並以手詔嘉勉陳慶之:「本非將種,又非豪家,觖望風雲,以至於此。可深思奇略,善克令終。開朱門而待賓,揚聲名於竹帛,豈非大丈夫哉!」

大通三年(529年)九月十五日,梁武帝第二次至同泰寺举行“四部无遮大会”,脱下帝袍,换上僧衣,舍身出家,九月十六日讲解《涅槃经》。當月二十五日群臣捐錢一億,向“三宝”祷告,请求赎回“皇帝菩萨”,二十七日萧衍還俗。

梁武帝因天象称“荧惑入南斗,天子下殿走”,就赤脚下殿以应天象。之後傳來北魏孝武帝西奔的消息,得知此事的武帝羞惭地说道:“綁著辮子的胡虜(索虏)也应天象吗?”(由於天象应於北魏,意味天意认为北魏孝武帝才是正统天子)

大同十二年(546年)四月十日,蕭衍第三次出家,此次群臣用兩億錢將其贖回;太清元年(547年),三月三日蕭衍又第四次出家,在同泰寺住了三十七天,四月十日、朝廷出資一億錢贖回武帝。郭祖深形容:“都下佛寺五百餘所,窮極宏麗。僧尼十餘萬,資產豐沃。”。此時國力日衰。

侯景原为东魏的将领,由于他与东魏丞相高澄的矛盾,于太清元年(547年)正月据河南十三州叛归西魏,但西魏宇文泰对其不信任。迫于无奈,侯景致函萧衍,许愿献出河南十三州来投奔南朝梁。萧衍接纳了侯景,并任命他为大将军,封河南王。不久,东魏攻击侯景,萧衍派姪子萧渊明支援,结果战败,萧渊明被俘。侯景败退后占据寿阳。高澄假意提出和解,意在离间侯景和梁朝。司农卿傅岐认为高澄议和是离间之计。而朱异等人则极力主张与东魏和好。萧衍不听臣下劝告,与东魏使者往来,侯景感到恐慌。

此时,侯景假托东魏名义写信给萧衍,提出用萧渊明交换侯景,萧衍居然表示接受。侯景十分气愤,遂起兵叛变。他以萧正德为内应,轻易渡过了长江,并在公元549年三月围攻建康。城中久被围困,粮食断绝,饥病困扰,人多浮肿气急,横尸满路,能登城抗击者不到四千人。南梁诸王手握重兵,却彼此猜忌按兵不动,无人讨叛。十二日,侯景攻入建康,纵兵洗劫,蕭家宗室、世族琅琊王氏、陳郡謝氏皆遭血洗,史称侯景之亂。

城陷之后,侯景的武士隨意进出皇宫、甚至佩带武器。萧衍见了很奇怪,问左右侍从,侍从说是侯丞相的卫兵。萧衍生气地喝道:“甚么丞相!不就是侯景嗎!”侯景听说了,非常生气,于是派人监视萧衍,萧衍的饮食也被侯景裁减。萧衍忧愤成疾,口苦乾渴,索蜂蜜水,未得实现,怒憤更疾。

据说梁武帝曾经在志公禅师臨終時,向其询问自己寿命。志公说;「我的墓塔倒了,陛下的大限就到了。」志公涅槃後,寺方造了木製的靈塔,梁武帝担心志公的木造靈塔不坚固,就拆除,打算重建,拆了以后不久侯景之乱就发生了。

五月,萧衍在糧食尚足的情況下(身旁有數百顆雞蛋),因激憤不已,病卒於台城內,死前猶喊著軍事戰陣時的「荷,荷」(士兵先退後進的口號),表示他反擊侯景的志願。死时86岁,葬于修陵(今江苏丹阳市陵口)。谥号武帝,庙号高祖。

梁武帝除了帝王的身分,也身為學者在經、史、詩詞、佛學等領域留下大量著述而出名。

在经学方面,他撰有《周易讲疏》、《春秋答问》、《孔子正言》等二百余卷。天监十一年(512年),又制成吉、凶、军、宾、嘉五礼,共一千余卷,八千零十九条,颁布施行。

在史学方面,他不满《汉书》等断代史的写法,因而主持编撰了六百卷的《通史》,并“躬制赞序”。命殷芸将无法入史的剩余材料(主要是异闻杂谈),编入小说。这些著作大都没有流传下来。

在文学方面,梁武帝也非常喜欢诗赋創作,现存古詩、樂府詩等诗歌有80多首。蕭衍和王融、謝朓、任昉、沈約、范雲、蕭琛、陸倕七人共稱竟陵八友,在齊永明時代的文學界頗負盛名。

在宗教方面,今日漢傳佛教的素食主義即以梁武帝為首。佛教的梁皇寶懺是他編製成的,他又著有《大般涅槃經》、《大品般若經》、《淨名經》、《三慧經》等諸經義記數百卷。在道教学说中,他把儒家的“礼法”、道家的“无”和佛教“因果报应”揉合,创立了“三教同源说”,在中国古代思想史上占有极其重要的地位。由於梁武帝對佛教流通的貢獻,寺廟时以梁武帝與其長子昭明太子合祀為護法神。

梁武帝的学问路线,是先习儒,再奉道,后入佛。少年时代是习儒阶段,“少时学周孔,弱冠穷六经”。二十岁以后,改奉道教,一直到即位为帝后,仍未捨道。《隋书·经籍志》载,“武帝弱年好事,先受道法,及即位,犹自上章”。称帝后的萧衍和道士陶弘景的关系极善,他每当遇到国家大事,经常要派人到茅山去向陶弘景请教,以致于陶弘景有“山中宰相”之称。不过,在即位后的第二个年头,即天监三年(504),萧衍就颁布了《捨事道法詔》,宣布捨道归佛。而据其《述三教诗》,则称“晚年开释卷,犹月映众星”。到晚年才开始研读佛经。这也许说明,他虽然已经颁布了事佛诏,实际上还未真正彻底放弃道教。但总的来说,颁诏以后,他是以事佛为主的。有關《捨事道法詔》的真实性在学术界存疑,但无论其真伪,萧衍的奉佛则是事实。

梁武帝对佛教的支持,表现为两大方面:一是亲身修佛,二是从各方面扶持佛教的发展。

梁武帝本人归佛后,逐渐过上了佛教徒的生活。在武帝發表《斷酒肉文》前漢傳佛教「律中無有斷肉法」(反而是與釋迦佛作對的天啟,提倡素食),蕭衍把佛教五戒中的不殺生引申為素食,颁布了《断酒肉文》,禁止僧众吃肉,自己也行素食,開啟了漢傳佛教素食的傳統,之後漢傳佛教僧團開始遵守《梵網經》規定的菩薩戒,不再食肉。对那些敢于饮酒食肉的僧侶,他以世俗的刑法治罪。他又颁布《断杀绝宗庙牺牲诏》,禁止宗庙的牺牲,这是有违儒家祭祀禮儀的,但他坚持推行。他还正式受戒,据《续高僧传》卷六记载,他于天监十八年(519)“发宏誓心,受梵網經菩萨戒”。

梁武帝晚年奉佛更甚,经常日食一餐,過午不食,所食也只是豆羹、粗饭而已。篤信佛教,由於不近女色,曾經四十年無幸后宮,最突出的奉佛行为,是多次舍身出家,先后四次舍身同泰寺,每次都是朝廷花了大量的香火钱才把他赎出来還俗。他的第四次舍身是在太清元年(547)三月,历时一个月,所花赎钱为“一亿万”,这为同泰寺带来了巨额资金。

武帝本人是可以划入“义学”一类的,他对佛经很有研究,尤重《般若经》、《涅槃经》、《法华经》等,他常常为大家讲经说法,召开各种法会,开设过千僧会、无遮大会。中大通元年(529)开设的无遮大会,参加者有道俗五万多人。他的佛教撰述,则有《摩诃般若波罗蜜经注解》(现仅存序)、《三慧经义记》(《三慧经》本是《摩诃般若经》中的《三慧品》,萧衍认为此品最重要,因而独列為《三慧经》)、《制旨大涅槃经讲疏》、《净名经义记》、《制旨大集经讲疏》、《发般若经题论义并问答》(均佚),另著有《立神明成佛义记》、《敕答臣下神灭论》、《为亮法师制涅槃经疏序》、《断酒肉文》、《述三教诗》等,均存。

武帝在哲学上对中国佛教的贡献,突出之处是把中国传统的心性论、灵魂不灭论和佛教的涅槃佛性说结合起来了,他本人是属于涅槃学派的,主张“神明成佛”,所谓“神明”,是指永恒不灭的精神实体,它是众生成佛的内在根据,“神明”也就是佛性。他又提出三教同源论,认为儒、道二教同源于佛教,老子、孔子,都是释迦牟尼的弟子,所以从这个角度来看,三教可以会通,同时,三教的社会作用也是相同的,都是教化人为善。

除了自身奉佛,萧衍还大力扶持佛教事业的发展。他非常支持外僧的译经,僧伽婆罗被他召入五处译场从事译经,所译经典,又请宝唱等人写疏,他甚至“躬临法座,笔受其文,然后乃付译人”。真谛在萧衍门下也受到礼遇,只是因为侯景之乱,真谛的译事难申。萧衍和国内法師的关系也很密切,宝亮、智藏、法云、僧旻等人,都是萧衍非常器重的。他组织僧人编撰佛教著作,编成的作品至少有十二种。他还广造伽藍,所建有大爱敬寺、智度寺、光宅寺、同泰寺等十一座,各寺铸有佛像,大爱敬寺有金铜像,智度寺的正殿铸有金像,光宅寺有丈九无量寿佛铜像,同泰寺有十方银像。

禅宗祖师菩提达摩南北朝时期来中国弘法,与梁武帝会谈。但因理念不合,话不投机,离开梁朝而北去。

在梁武帝的支持下,梁代佛教达到了南朝佛教的最盛期,他最后在侯景之乱时,饥病交加,死于寺中。但武帝之后,梁简文帝和梁元帝也都篤信佛法。

蕭衍登位天子,民望所歸,敢革時政,頗得人心,初期國家興旺繁盛,為一明君。後期太過信仰宗教,企图以佛治民,學者有此評價:一,太過慈悲,不力法治,导致官吏貪污搜刮,百官“缘饰奸谄,深害时政”,奸邪小人纷纷以正人君子的面目出现,官场风气败坏,民间疾苦,国力衰败,而民怨终于为眼光锐利的侯景所利用。二,外交失敗,不能知人,“险躁之心,暮年愈甚”,导致侯景之乱,侯景之乱彻底打击了江南的经济基础、人口基础。

錢穆於《國史大綱》云:“獨有一蕭衍老翁,儉過漢文,勤如王莽,可謂南朝一令主。”
王夫之於《讀通鑑論》亦云:“梁氏享國五十年,天下且小康焉。”

\subsubsection{天监}

\begin{longtable}{|>{\centering\scriptsize}m{2em}|>{\centering\scriptsize}m{1.3em}|>{\centering}m{8.8em}|}
  % \caption{秦王政}\
  \toprule
  \SimHei \normalsize 年数 & \SimHei \scriptsize 公元 & \SimHei 大事件 \tabularnewline
  % \midrule
  \endfirsthead
  \toprule
  \SimHei \normalsize 年数 & \SimHei \scriptsize 公元 & \SimHei 大事件 \tabularnewline
  \midrule
  \endhead
  \midrule
  元年 & 502 & \tabularnewline\hline
  二年 & 503 & \tabularnewline\hline
  三年 & 504 & \tabularnewline\hline
  四年 & 505 & \tabularnewline\hline
  五年 & 506 & \tabularnewline\hline
  六年 & 507 & \tabularnewline\hline
  七年 & 508 & \tabularnewline\hline
  八年 & 509 & \tabularnewline\hline
  九年 & 510 & \tabularnewline\hline
  十年 & 511 & \tabularnewline\hline
  十一年 & 512 & \tabularnewline\hline
  十二年 & 513 & \tabularnewline\hline
  十三年 & 514 & \tabularnewline\hline
  十四年 & 515 & \tabularnewline\hline
  十五年 & 516 & \tabularnewline\hline
  十六年 & 517 & \tabularnewline\hline
  十七年 & 518 & \tabularnewline\hline
  十八年 & 519 & \tabularnewline
  \bottomrule
\end{longtable}

\subsubsection{普通}

\begin{longtable}{|>{\centering\scriptsize}m{2em}|>{\centering\scriptsize}m{1.3em}|>{\centering}m{8.8em}|}
  % \caption{秦王政}\
  \toprule
  \SimHei \normalsize 年数 & \SimHei \scriptsize 公元 & \SimHei 大事件 \tabularnewline
  % \midrule
  \endfirsthead
  \toprule
  \SimHei \normalsize 年数 & \SimHei \scriptsize 公元 & \SimHei 大事件 \tabularnewline
  \midrule
  \endhead
  \midrule
  元年 & 520 & \tabularnewline\hline
  二年 & 521 & \tabularnewline\hline
  三年 & 522 & \tabularnewline\hline
  四年 & 523 & \tabularnewline\hline
  五年 & 524 & \tabularnewline\hline
  六年 & 525 & \tabularnewline\hline
  七年 & 526 & \tabularnewline\hline
  八年 & 527 & \tabularnewline
  \bottomrule
\end{longtable}

\subsubsection{大通}

\begin{longtable}{|>{\centering\scriptsize}m{2em}|>{\centering\scriptsize}m{1.3em}|>{\centering}m{8.8em}|}
  % \caption{秦王政}\
  \toprule
  \SimHei \normalsize 年数 & \SimHei \scriptsize 公元 & \SimHei 大事件 \tabularnewline
  % \midrule
  \endfirsthead
  \toprule
  \SimHei \normalsize 年数 & \SimHei \scriptsize 公元 & \SimHei 大事件 \tabularnewline
  \midrule
  \endhead
  \midrule
  元年 & 527 & \tabularnewline\hline
  二年 & 528 & \tabularnewline\hline
  三年 & 529 & \tabularnewline
  \bottomrule
\end{longtable}

\subsubsection{中大通}

\begin{longtable}{|>{\centering\scriptsize}m{2em}|>{\centering\scriptsize}m{1.3em}|>{\centering}m{8.8em}|}
  % \caption{秦王政}\
  \toprule
  \SimHei \normalsize 年数 & \SimHei \scriptsize 公元 & \SimHei 大事件 \tabularnewline
  % \midrule
  \endfirsthead
  \toprule
  \SimHei \normalsize 年数 & \SimHei \scriptsize 公元 & \SimHei 大事件 \tabularnewline
  \midrule
  \endhead
  \midrule
  元年 & 529 & \tabularnewline\hline
  二年 & 530 & \tabularnewline\hline
  三年 & 531 & \tabularnewline\hline
  四年 & 532 & \tabularnewline\hline
  五年 & 533 & \tabularnewline\hline
  六年 & 534 & \tabularnewline
  \bottomrule
\end{longtable}

\subsubsection{大同}

\begin{longtable}{|>{\centering\scriptsize}m{2em}|>{\centering\scriptsize}m{1.3em}|>{\centering}m{8.8em}|}
  % \caption{秦王政}\
  \toprule
  \SimHei \normalsize 年数 & \SimHei \scriptsize 公元 & \SimHei 大事件 \tabularnewline
  % \midrule
  \endfirsthead
  \toprule
  \SimHei \normalsize 年数 & \SimHei \scriptsize 公元 & \SimHei 大事件 \tabularnewline
  \midrule
  \endhead
  \midrule
  元年 & 535 & \tabularnewline\hline
  二年 & 536 & \tabularnewline\hline
  三年 & 537 & \tabularnewline\hline
  四年 & 538 & \tabularnewline\hline
  五年 & 539 & \tabularnewline\hline
  六年 & 540 & \tabularnewline\hline
  七年 & 541 & \tabularnewline\hline
  八年 & 542 & \tabularnewline\hline
  九年 & 543 & \tabularnewline\hline
  十年 & 544 & \tabularnewline\hline
  十一年 & 545 & \tabularnewline\hline
  十二年 & 546 & \tabularnewline
  \bottomrule
\end{longtable}

\subsubsection{中大同}

\begin{longtable}{|>{\centering\scriptsize}m{2em}|>{\centering\scriptsize}m{1.3em}|>{\centering}m{8.8em}|}
  % \caption{秦王政}\
  \toprule
  \SimHei \normalsize 年数 & \SimHei \scriptsize 公元 & \SimHei 大事件 \tabularnewline
  % \midrule
  \endfirsthead
  \toprule
  \SimHei \normalsize 年数 & \SimHei \scriptsize 公元 & \SimHei 大事件 \tabularnewline
  \midrule
  \endhead
  \midrule
  元年 & 546 & \tabularnewline\hline
  二年 & 547 & \tabularnewline
  \bottomrule
\end{longtable}

\subsubsection{太清}

\begin{longtable}{|>{\centering\scriptsize}m{2em}|>{\centering\scriptsize}m{1.3em}|>{\centering}m{8.8em}|}
  % \caption{秦王政}\
  \toprule
  \SimHei \normalsize 年数 & \SimHei \scriptsize 公元 & \SimHei 大事件 \tabularnewline
  % \midrule
  \endfirsthead
  \toprule
  \SimHei \normalsize 年数 & \SimHei \scriptsize 公元 & \SimHei 大事件 \tabularnewline
  \midrule
  \endhead
  \midrule
  元年 & 547 & \tabularnewline\hline
  二年 & 548 & \tabularnewline\hline
  三年 & 549 & \tabularnewline
  \bottomrule
\end{longtable}


%%% Local Variables:
%%% mode: latex
%%% TeX-engine: xetex
%%% TeX-master: "../../Main"
%%% End:

%% -*- coding: utf-8 -*-
%% Time-stamp: <Chen Wang: 2021-11-01 15:05:44>

\subsection{简文帝蕭綱\tiny(549-551)}

\subsubsection{生平}

梁簡文帝蕭綱(503年-551年),字世讚,一作世纘,小字六通,梁武帝蕭衍第三子,昭明太子蕭統的胞弟,母丁令光。

蕭綱最早封為晉安王,曾經擔任過南徐州刺史,並且曾經參與北伐;531年蕭統病故之後被封為皇太子。

548年侯景叛亂,萧纲助守台城,梁武帝因自认为年老,授权萧纲主军国大事。侯景部下仪同三司范桃棒在被俘的云旗将军陈昕劝说下图谋率所部袭杀侯景部下行台左丞王伟、部将宋子仙,再去建康投降。范桃棒写信射入建康城中,再秘密派陈昕趁夜吊绳入城。武帝大喜,但萧纲担心有诈,犹豫不决。范桃棒又派陈昕写信说:“现在仅带所领五百人,如果到城门,都自己脱甲,乞求朝廷开门赐容。事成之后,保证擒侯景。”萧纲见其恳切,愈发生疑。结果事泄,范桃棒被杀,陈昕出城接应后也被擒杀。

侯景攻陷台城後,梁武帝於549年憂憤而死,但是侯景認為目前仍然不能自立為皇帝,便擁立蕭綱為皇帝,次年改元大寶。但是蕭綱不過是侯景的傀儡。551年,侯景派人廢蕭綱為晉安王,改立豫章王蕭棟為皇帝;蕭綱被囚禁於永福省,蕭綱被廢後兩個月,被侯景派人以棉被悶死,享年49歲。

侯景事後為蕭綱上諡號曰明皇帝,廟號高宗,梁元帝在552年追諡蕭綱為簡文皇帝,廟號太宗。

蕭綱本人文學造詣很高,雅好詩賦,有大量詠物、宮體、閨怨之作,其中五言詩最多,並且與蕭子显、蕭繹、徐擒、庾肩吾等人形成宮體詩流派,萧纲是宫体诗的代表。侯景攻入建康期间,曾经“募人出烧东宫,东宫台殿遂尽。所聚百橱图籍,一皆灰烬”。


\subsubsection{大宝}

\begin{longtable}{|>{\centering\scriptsize}m{2em}|>{\centering\scriptsize}m{1.3em}|>{\centering}m{8.8em}|}
  % \caption{秦王政}\
  \toprule
  \SimHei \normalsize 年数 & \SimHei \scriptsize 公元 & \SimHei 大事件 \tabularnewline
  % \midrule
  \endfirsthead
  \toprule
  \SimHei \normalsize 年数 & \SimHei \scriptsize 公元 & \SimHei 大事件 \tabularnewline
  \midrule
  \endhead
  \midrule
  元年 & 550 & \tabularnewline\hline
  二年 & 551 & \tabularnewline
  \bottomrule
\end{longtable}


%%% Local Variables:
%%% mode: latex
%%% TeX-engine: xetex
%%% TeX-master: "../../Main"
%%% End:

%% -*- coding: utf-8 -*-
%% Time-stamp: <Chen Wang: 2021-11-01 15:06:11>

\subsection{淮陰王萧栋\tiny(551)}

\subsubsection{生平}

蕭棟(?-552年),字元吉,南朝梁朝的第三代皇帝。史稱豫章王、淮陰王。

蕭棟為昭明太子蕭統之孫,豫章王蕭歡之子。蕭統去世後,梁武帝曾經有一度想立蕭歡為皇太孫,但最後沒有,而改封蕭統三弟后来的梁簡文帝蕭綱當太子。萧欢死后,萧栋袭为豫章王。

侯景之乱期间,叛将侯景攻破梁都建康,将萧栋等在京宗室软禁。

551年,侯景被皇弟湘东王萧绎部将王僧辩所败,回到建康,图谋篡位。他以簡文帝名义下诏禅位给萧栋。当时京城一带饥荒,萧栋与王妃张氏正在园中种菜,看到来迎接他为帝的士兵,不知所措,哭着登车,被侯景立为皇帝,升武德殿。当时平地起风,将华盖吹翻直出端门,时人知道萧栋不能善终。改元天正。侯景掌权,萧栋毫无实权。他追封祖父萧统、父萧欢为皇帝,尊母王氏为皇太后,立王妃张氏为皇后。

两个半月後,因侯景所迫,萧栋加侯景殊礼、九锡,侯景不久廢蕭棟為淮陰王,並自立為漢皇帝,並將蕭棟與其弟蕭橋、蕭樛囚於密室之中。

萧绎登基为梁元帝并收復建業後,指使王僧辩杀萧栋,王僧辩拒绝。萧绎于是命宣猛将军朱买臣杀萧栋。当时,蕭棟與其弟都逃出密室,相扶而出,但仍戴着镣铐,遇到将军杜崱后才去除。弟弟们以为逃出生天,但萧栋认为未必,仍然害怕。不久,他们遇到朱买臣,朱买臣请他们上船饮酒,席未终,朱买臣所部士兵抓住他们,沉入扬子江。

\subsubsection{天正}

\begin{longtable}{|>{\centering\scriptsize}m{2em}|>{\centering\scriptsize}m{1.3em}|>{\centering}m{8.8em}|}
  % \caption{秦王政}\
  \toprule
  \SimHei \normalsize 年数 & \SimHei \scriptsize 公元 & \SimHei 大事件 \tabularnewline
  % \midrule
  \endfirsthead
  \toprule
  \SimHei \normalsize 年数 & \SimHei \scriptsize 公元 & \SimHei 大事件 \tabularnewline
  \midrule
  \endhead
  \midrule
  元年 & 551 & \tabularnewline
  \bottomrule
\end{longtable}


%%% Local Variables:
%%% mode: latex
%%% TeX-engine: xetex
%%% TeX-master: "../../Main"
%%% End:

%% -*- coding: utf-8 -*-
%% Time-stamp: <Chen Wang: 2019-12-23 14:07:19>

\subsection{元帝\tiny(552-554)}

\subsubsection{生平}

梁元帝萧绎(508年9月16日-555年1月27日),字世誠,梁武帝蕭衍的第七子,正式谥號為「孝元皇帝」,後世比照西漢的漢元帝和東晉的晋元帝,省「孝」字稱「梁元帝」。

萧绎於514年封湘东王,早年因病而一眼失明。547年出荊州,任荊州刺史、使持節、都督荊雍湘司郢寧梁南北秦九州諸軍事、鎮西將軍。侯景之亂時,梁武帝遣人至荊州宣讀密詔,授蕭繹为侍中、假黄钺、大都督中外诸军事、司徒承制,其余职务如故。萧绎手握强兵,却没有积极勤王。

549年梁武帝餓死台城後,蕭繹首先發兵攻滅自己的侄兒河東王蕭譽與哥哥邵陵王蕭綸,並擊退萧誉弟岳阳王襄陽都督萧詧的來犯,迫使萧詧投靠西魏;之後再命王僧辯率軍東下消滅侯景。其长子萧方等即在与萧誉作战时身亡,次子萧方诸在与侯景交战时被擒杀。萧绎弟益州刺史武陵王蕭紀亦有意出兵共讨侯景,萧绎写信阻止,称“蜀人勇悍,易动难安,弟可镇之,吾自当灭贼。”又写信称“地拟孙、刘,各安境界;情深鲁、卫,书信恒通”以为示好。

552年侯景死後,蕭繹即帝位於江陵,是為梁世祖。当时,群臣中有人建议返回旧都建康,但蕭繹没有同意。并派手下朱买臣在建康杀死侯景所废皇帝萧栋兄弟三人。

萧绎即帝位之前,蕭紀已稱帝於益州;萧纪出兵讨伐侯景,得知侯景已灭,就转为讨伐萧绎。蕭繹便派兵迎战,写信讲和,同时也请求西魏出兵袭取益州。萧纪因此遭受重创,向萧绎求和,萧绎回信拒绝称兄弟情断,最终全歼萧纪势力,但也給了西魏可趁之機,益州因此沦落敌手。萧绎将萧纪父子除宗籍改姓饕餮,并将萧纪二子萧圆照、萧圆正饿死。

554年,蕭繹给西魏权臣宇文泰写信,要求按照旧图重新划定疆界,言辞又极为傲慢。宇文泰大为不满,命令常山公于谨、大將軍楊忠、兄子大将军宇文护等将领以5万兵马进攻江陵(今湖北江陵縣)。梁元帝战败,由御史中丞王孝祀作降文。隨后,便率太子等人到西魏軍營投降。不久為萧詧以土袋悶死,江陵“阖城老幼被虏入关”。

梁元帝也是一個愛好讀書與喜好文學的君主,“四十六岁,自聚书来四十年,得书八万卷”,自稱“韜於文士,愧於武夫。”曾主編《金樓子》等書;江陵被圍城時,承聖三年十二月辛未(555年1月27日),元帝入東閤竹殿,命舍人高善寶放火焚燒圖書14萬卷,包括从建康为避兵灾而转移到江陵的8万卷书,自稱“文武之道,今夜盡矣!”“讀書萬卷,猶有今日,故焚之。”江陵焚書被視為中國的文化浩劫之一。

清朝初年的王船山評論其焚書行徑:「未有不恶其不悔不仁而归咎于读书者,曰书何负于帝哉?此非知读书者之言也。

北宋的司馬光評論:「元帝於兄弟之中,殘忍尤甚,是以雖翦兇渠而克復故業,旋踵之間,身為伏馘;豈特人心之不與哉?亦天地之所誅也。」

唐朝的虞世南:「梁元聪明伎艺,才兼文武,仗顺伐逆,克雪家冤,成功遂事,有足称者。但国难之后,伤夷未复,信强寇之甘言,袭褊心于怀楚,蕃屏宗支自为仇敌,孤远悬僻,莫与同憂,身亡祚灭,生人涂炭,举鄢、郢而弃之,良可惜也。」

陳朝的史家何之元:「世祖聰明特達,才藝兼美,詩筆之麗,罕與為匹,伎能之事,無所不該,極星象之功,窮蓍龜之妙,明筆法于馬室,不愧鄭玄,辨雲物于魯台,無慚梓慎,至於帷籌將略,朝野所推,遂乃撥亂反正,夷凶殄逆,紐地維之已絕,扶天柱之將傾,黔首蒙拯溺之恩,蒼生荷仁壽之惠,微管之力,民其戎乎?鯨鯢既誅,天下且定,早應移鑾西楚,旋駕東都,祀宗土方,清蹕宮闕,西周岳陽之敗績,信口宇文之和通,以萬乘之尊,居二境之上,夷虜乘釁,再覆皇基,率土分崩,莫知攸暨,謀之不善,乃至於斯。」

\subsubsection{承圣}

\begin{longtable}{|>{\centering\scriptsize}m{2em}|>{\centering\scriptsize}m{1.3em}|>{\centering}m{8.8em}|}
  % \caption{秦王政}\
  \toprule
  \SimHei \normalsize 年数 & \SimHei \scriptsize 公元 & \SimHei 大事件 \tabularnewline
  % \midrule
  \endfirsthead
  \toprule
  \SimHei \normalsize 年数 & \SimHei \scriptsize 公元 & \SimHei 大事件 \tabularnewline
  \midrule
  \endhead
  \midrule
  元年 & 552 & \tabularnewline\hline
  二年 & 553 & \tabularnewline\hline
  三年 & 554 & \tabularnewline\hline
  四年 & 555 & \tabularnewline
  \bottomrule
\end{longtable}


%%% Local Variables:
%%% mode: latex
%%% TeX-engine: xetex
%%% TeX-master: "../../Main"
%%% End:

%% -*- coding: utf-8 -*-
%% Time-stamp: <Chen Wang: 2021-11-01 15:06:27>

\subsection{闵帝蕭淵明\tiny(555)}

\subsubsection{生平}

梁閔帝蕭淵明(5世紀?-556年6月2日),字靖通,蕭懿之子,梁武帝蕭衍之姪。

蕭淵明最早封貞陽侯,後擔任豫州刺史;在侯景背叛東魏投降南梁之時,蕭衍命蕭淵明與侯景北伐攻打東魏,結果蕭淵明兵敗被俘。侯景曾伪造东魏书信称愿意放还萧渊明交换侯景,萧衍回信答应“贞阳朝至,侯景夕返”,侯景遂骂萧衍薄心肠而作乱。

後在西魏攻陷江陵殺害梁孝元帝時,王僧辩、陈霸先意图拥立元帝子萧方智为帝,立萧方智为梁王;北齊文宣帝高洋與上黨王高渙於555年送回蕭淵明,準備讓蕭淵明成為北齊支持的傀儡皇帝,王僧辩未能抵抗齐军,于是接受萧渊明,但要求立萧方智为皇太子;蕭淵明於是即帝位,改年號為天成,并立萧方智为皇太子。不久後陳霸先诛王僧辯,並且廢蕭淵明,改立蕭方智為皇帝。蕭淵明改封建安公。

之後,北齊要求南梁送回蕭淵明,陳霸先也準備將蕭淵明送還北齊。但還沒有出發蕭淵明便病故。蕭莊在北齊支持下稱帝之後,諡蕭淵明為閔皇帝。

\subsubsection{天成}

\begin{longtable}{|>{\centering\scriptsize}m{2em}|>{\centering\scriptsize}m{1.3em}|>{\centering}m{8.8em}|}
  % \caption{秦王政}\
  \toprule
  \SimHei \normalsize 年数 & \SimHei \scriptsize 公元 & \SimHei 大事件 \tabularnewline
  % \midrule
  \endfirsthead
  \toprule
  \SimHei \normalsize 年数 & \SimHei \scriptsize 公元 & \SimHei 大事件 \tabularnewline
  \midrule
  \endhead
  \midrule
  元年 & 555 & \tabularnewline
  \bottomrule
\end{longtable}


%%% Local Variables:
%%% mode: latex
%%% TeX-engine: xetex
%%% TeX-master: "../../Main"
%%% End:

%% -*- coding: utf-8 -*-
%% Time-stamp: <Chen Wang: 2019-12-23 14:18:46>

\subsection{敬帝\tiny(555-557)}

\subsubsection{生平}

梁敬帝蕭方智(543年-558年5月5日),字慧相,南梁的末代皇帝,梁元帝蕭繹的第九子。549年,蕭方智被封為興梁侯;552年被封為晉安王,553年被封為江州刺史。

當梁元帝在江陵被殺之時,555年陳霸先、王僧辯擁立蕭方智以太宰承制於建康,立为梁王,但是北齊將元帝堂兄貞陽侯蕭淵明送回之後,王僧辯因无力抵抗,又同意擁立蕭淵明為皇帝,但要求立萧方智为皇太子。萧渊明登基后,立萧方智为皇太子。陳霸先以王僧辩投降北齐、抛弃元帝子为由袭殺王僧辯,萧渊明亦退位,萧方智登基。557年,蕭方智禪位與陳霸先,南朝梁被南陳取代。

陳霸先封蕭方智為江陰王,永定二年四月乙丑(558年5月5日),陈武帝陈霸先派人杀死梁敬帝,陳霸先追諡曰敬帝,封梁武帝十弟鄱陽王蕭恢之孫蕭季卿為江陰王。

\subsubsection{绍泰}

\begin{longtable}{|>{\centering\scriptsize}m{2em}|>{\centering\scriptsize}m{1.3em}|>{\centering}m{8.8em}|}
  % \caption{秦王政}\
  \toprule
  \SimHei \normalsize 年数 & \SimHei \scriptsize 公元 & \SimHei 大事件 \tabularnewline
  % \midrule
  \endfirsthead
  \toprule
  \SimHei \normalsize 年数 & \SimHei \scriptsize 公元 & \SimHei 大事件 \tabularnewline
  \midrule
  \endhead
  \midrule
  元年 & 555 & \tabularnewline\hline
  二年 & 556 & \tabularnewline
  \bottomrule
\end{longtable}

\subsubsection{太平}

\begin{longtable}{|>{\centering\scriptsize}m{2em}|>{\centering\scriptsize}m{1.3em}|>{\centering}m{8.8em}|}
  % \caption{秦王政}\
  \toprule
  \SimHei \normalsize 年数 & \SimHei \scriptsize 公元 & \SimHei 大事件 \tabularnewline
  % \midrule
  \endfirsthead
  \toprule
  \SimHei \normalsize 年数 & \SimHei \scriptsize 公元 & \SimHei 大事件 \tabularnewline
  \midrule
  \endhead
  \midrule
  元年 & 556 & \tabularnewline\hline
  二年 & 557 & \tabularnewline
  \bottomrule
\end{longtable}

\subsection{东梁简介}

蕭莊(548年-577年),南兰陵郡兰陵县(今江苏省常州市武进区)人,梁元帝之孫,武烈世子蕭方等之子。

554年,西魏攻陷江陵、殺害梁元帝時,蕭莊只有七歲,逃匿於民家之中;之後被王琳發現,將蕭莊護送回建康。梁敬帝蕭方智即帝位之後,將蕭莊作為人質送往北齊。

557年,陳霸先廢蕭方智即帝位後,王琳等人要求北齊送還蕭莊,並使其接替南梁皇帝;蕭莊回到南朝之後,王琳立蕭莊為梁皇帝於郢州,據有長江中上游地區,是為東梁。之後蕭莊的南梁與陳霸先的陳朝便持續交戰,560年,當王琳與陳朝的侯瑱在蕪湖交戰時,北周便發兵攻打郢州,結果王琳兵敗,與蕭莊逃亡北齊。蕭莊被封为侯,又封梁王。北齐允諾幫他復興梁朝,但没能实现。齐后主高纬灭亡后,蕭莊在邺城闭气自杀。



%%% Local Variables:
%%% mode: latex
%%% TeX-engine: xetex
%%% TeX-master: "../../Main"
%%% End:

%% -*- coding: utf-8 -*-
%% Time-stamp: <Chen Wang: 2021-11-01 15:07:19>

\subsection{后梁(555-587)}

\subsubsection{简介}

後梁(555年-587年)為中國南北朝時期南梁在蕭詧稱帝後殘存的政权。國都於江陵,统治地区位于原梁朝國都建康的西边,故又稱為西梁。

在南梁末年的侯景之乱中,梁武帝孙、昭明太子萧统子岳阳王萧詧为叔父湘东王萧绎所迫投降西魏,封梁王;萧绎后来登基,史称梁元帝。554年西魏攻陷江陵后,梁元帝投降,被萧詧杀死。萧詧於555年称帝,並且對西魏稱臣;但是後梁由於國土狹小,屬地僅有江陵附近數縣八百里地,先後是西魏、北周和隋朝的附庸。但後梁也一直自居為南朝正統而與陳朝對立。而後梁也由於承續南梁的文化,而成為具有高度文化的國家。

後梁共傳宣帝蕭詧、明帝蕭巋、後主蕭琮三世。587年九月,隋文帝廢除後梁,改蕭琮為莒國公,後梁因此滅亡,存在共三十三年。然而由於蕭氏歷代事奉北周、對隋朝甚為恭謹,蕭巋之女還被隋文帝选为晋王妃即后来隋煬帝楊廣之皇后(炀愍皇后),因此在後梁廢除後,蕭氏在隋朝中央朝廷與江陵仍保有一定的政治影響力。隋末群雄之一的蕭銑即為蕭巋弟蕭巖之孫。

\subsubsection{宣帝蕭詧}

梁宣帝蕭\xpinyin*{詧}(519年-562年),字理孫,梁武帝之孫、昭明太子蕭統之第三子。為後梁(西梁)的建立者,諡號宣皇帝,廟號中宗。

萧统死后,梁武帝立萧纲为皇太子,而进封萧统诸子为王。蕭詧被封為岳陽郡王並被任命為東揚州刺史,鎮守會稽。但他们仍然因未被立储而对梁武帝怀恨。

546年十月,萧詧改任雍州刺史,鎮守襄陽。549年,侯景之乱时,蕭詧兄長湘州刺史河東王蕭譽被他們的叔父湘東王(之後的梁元帝)蕭繹攻擊,邵陵王萧纶劝阻无果,蕭詧試圖救援蕭譽兵敗,部将杜岸叛变反以五百骑攻打襄阳,被留守的蔡大宝和萧詧母龚保林守御而未果。萧詧连夜撤回襄阳,龚保林不知他兵败,以为他是敌军,直到天亮了认出他才放他入城。萧詧自知得罪萧绎,为了自保,據襄陽歸降西魏,西魏於550年三月封蕭詧為梁王。萧纶被西魏军所杀,萧詧葬之。554年冬西魏派大将于谨等攻打江陵,梁元帝開門投降,被蕭詧侮辱后以土袋悶死。

之後西魏於555年在江陵立蕭詧為梁皇帝,年號大定。萧詧部将尹德毅劝他趁西魏人贪婪多有杀伤之机,趁西魏精锐尽在此时设宴,埋伏武士杀死于谨等人,再分头命令果决勇敢的人奇袭魏军营垒,全歼魏军,抚慰江陵百姓,任命文武百官,以救命之恩获得百姓支持,还可以写信招揽王僧辩等人,着朝服、渡长江,登基称帝,继承尧、禹之业。萧詧却说:“您的这条计策,并不是不好。可是魏人待我十分宽厚,我不能违背道德。如果仓促之间依计而行,就会像邓祁侯说的那样,我不会有好下场了。”

果然,西魏除江陵附近八百里之地外,將襄陽等地皆併入西魏,並且將江陵一帶的人民財產擄掠一空,萧詧所辖只有江陵周边八百里。萧詧追悔莫及,见屋宇残破,战乱不息,为自己威略不振而感到羞耻,心中常怀忧愤,于是作《愍时赋》自抒其意。每每读到“老马伏枥,志在千里,烈士暮年,壮心不已”就扬眉举目,握腕激奋,久久叹息不止。即位八年後,562年,蕭詧在抑鬱中病故。

\subsubsection{大定}

\begin{longtable}{|>{\centering\scriptsize}m{2em}|>{\centering\scriptsize}m{1.3em}|>{\centering}m{8.8em}|}
  % \caption{秦王政}\
  \toprule
  \SimHei \normalsize 年数 & \SimHei \scriptsize 公元 & \SimHei 大事件 \tabularnewline
  % \midrule
  \endfirsthead
  \toprule
  \SimHei \normalsize 年数 & \SimHei \scriptsize 公元 & \SimHei 大事件 \tabularnewline
  \midrule
  \endhead
  \midrule
  元年 & 555 & \tabularnewline\hline
  二年 & 556 & \tabularnewline\hline
  三年 & 557 & \tabularnewline\hline
  四年 & 558 & \tabularnewline\hline
  五年 & 559 & \tabularnewline\hline
  六年 & 560 & \tabularnewline\hline
  七年 & 561 & \tabularnewline\hline
  八年 & 562 & \tabularnewline
  \bottomrule
\end{longtable}


\subsubsection{明帝蕭巋}

梁明帝蕭巋(542年-585年),字仁遠,是南北朝時代西梁的第二位君主。正式諡號為「孝明皇帝」,後世比照漢朝和西晉皇帝省略「孝」字,稱「梁明帝」。

西梁是南梁的一個分裂王朝,它的地盤主要在今天湖北襄陽、荊州地區,首都江陵(今湖北省荊州市)。蕭巋的父親蕭詧與梁元帝蕭繹不和,蕭繹繼梁帝位後,蕭詧就投靠西魏,被西魏皇帝封為梁王,在他的統治地盤內他自稱皇帝,但實際上西梁的「皇帝」在他們的領土上並沒有真正的主權,很長時間裡北朝在西梁設有江陵總管,一方面用來監督西梁的君主,另一方面這些總管擁有兵權來保護西梁不被南朝攻擊。蕭詧死後他的兒子蕭巋於562年以皇太子身份繼帝位。

蕭巋的年號是天保,他繼續他父親的政策,聯合北朝(北周)來抵抗南朝(南陳)的威脅。北周武帝宇文邕滅北齊後蕭巋親自赴長安祝賀,因此深得宇文邕的信任。隋文帝楊堅登基後再次親自赴長安祝賀,又贏得了楊堅的信任。後來蕭、楊兩家又通婚,蕭巋的一個女兒還嫁給了楊廣,後來成為隋煬帝的皇后。由於蕭、楊兩家的關係如此親密,因此後來隋將它駐扎在西梁的江陵總管撤回,使得西梁獲得了自主權。

蕭巋是一個相當有學問的皇帝,他曾著《孝經》、《周易義記》、《大小乘幽微》等十四部書。

蕭巋於天保二十四年(585年)五月逝世,諡為孝明皇帝,廟號世宗。

\subsubsection{天保}

\begin{longtable}{|>{\centering\scriptsize}m{2em}|>{\centering\scriptsize}m{1.3em}|>{\centering}m{8.8em}|}
  % \caption{秦王政}\
  \toprule
  \SimHei \normalsize 年数 & \SimHei \scriptsize 公元 & \SimHei 大事件 \tabularnewline
  % \midrule
  \endfirsthead
  \toprule
  \SimHei \normalsize 年数 & \SimHei \scriptsize 公元 & \SimHei 大事件 \tabularnewline
  \midrule
  \endhead
  \midrule
  元年 & 562 & \tabularnewline\hline
  二年 & 563 & \tabularnewline\hline
  三年 & 564 & \tabularnewline\hline
  四年 & 565 & \tabularnewline\hline
  五年 & 566 & \tabularnewline\hline
  六年 & 567 & \tabularnewline\hline
  七年 & 568 & \tabularnewline\hline
  八年 & 569 & \tabularnewline\hline
  九年 & 570 & \tabularnewline\hline
  十年 & 571 & \tabularnewline\hline
  十一年 & 572 & \tabularnewline\hline
  十二年 & 573 & \tabularnewline\hline
  十三年 & 574 & \tabularnewline\hline
  十四年 & 575 & \tabularnewline\hline
  十五年 & 576 & \tabularnewline\hline
  十六年 & 577 & \tabularnewline\hline
  十七年 & 578 & \tabularnewline\hline
  十八年 & 579 & \tabularnewline\hline
  十九年 & 580 & \tabularnewline\hline
  二十年 & 581 & \tabularnewline\hline
  二一年 & 582 & \tabularnewline\hline
  二二年 & 583 & \tabularnewline\hline
  二三年 & 584 & \tabularnewline\hline
  二四年 & 585 & \tabularnewline
  \bottomrule
\end{longtable}


\subsubsection{后主蕭琮}

蕭琮(558年-607年),字溫文,為西梁明帝蕭巋之子,西梁第三位,亦是末代皇帝。

蕭琮最早封東陽王,後被立為皇太子。蕭琮博學有才,善於弓馬,個性倜儻不羈。585年即位為西梁皇帝,改年號為廣運。蕭琮即位之後,隋文帝設立江陵總管監視蕭琮的行為;587年,因蕭琮的叔父蕭巖等人帶了一部分居民逃入陳朝,隋文帝徵召蕭琮入朝,当年10月26日廢除西梁國,蕭琮亦被廢為莒國公。西梁也因此滅亡。

蕭琮在隋朝時仍然受到器重,隋煬帝即位後,因为萧琮是自己的妻兄,对萧琮更厚待,又封蕭琮為梁公、內史令,蕭琮的親族也有不少被提拔入朝廷為官。當時有童謠說:「蕭蕭亦復起」,導致隋煬帝對蕭琮的猜忌,最後蕭琮被免職,不久後在家中過世。

蕭琮死後被贈左光祿大夫,侄萧钜续封为梁公,史书没有记载萧琮子萧铉为何没有袭爵。

隋末割據勢力之一的蕭銑,為蕭琮之堂姪,並在稱帝之後追諡蕭琮為孝靖皇帝。

\subsubsection{广运}

\begin{longtable}{|>{\centering\scriptsize}m{2em}|>{\centering\scriptsize}m{1.3em}|>{\centering}m{8.8em}|}
  % \caption{秦王政}\
  \toprule
  \SimHei \normalsize 年数 & \SimHei \scriptsize 公元 & \SimHei 大事件 \tabularnewline
  % \midrule
  \endfirsthead
  \toprule
  \SimHei \normalsize 年数 & \SimHei \scriptsize 公元 & \SimHei 大事件 \tabularnewline
  \midrule
  \endhead
  \midrule
  元年 & 586 & \tabularnewline\hline
  二年 & 587 & \tabularnewline
  \bottomrule
\end{longtable}


%%% Local Variables:
%%% mode: latex
%%% TeX-engine: xetex
%%% TeX-master: "../../Main"
%%% End:



%%% Local Variables:
%%% mode: latex
%%% TeX-engine: xetex
%%% TeX-master: "../../Main"
%%% End:

%% -*- coding: utf-8 -*-
%% Time-stamp: <Chen Wang: 2019-12-23 14:31:08>


\section{南陈\tiny(557-589)}

\subsection{简介}

陈(557年-589年)是中国历史上南北朝时期南朝最后一个朝代,由陈霸先代梁所建立,以建康(今南京)为首都,国号陈。陈朝名称来自陈霸先即位前被封的陈公、陈王,但陈王的封号来源又有二说,一说认为来自陈王封地中排第一郡的陈留郡(见《陈书》),另一说认为陈霸先姓出于陈,陈为圣人舜后裔,故强迫梁帝封其为陈王(见胡三省《资治通鉴·陳纪一》下注)。

南梁太平二年(557年)梁敬帝蕭方智禪位當時已為陳王的大將陳霸先,陳霸先於是稱帝建立陳朝,史稱南陳。

由於侯景之亂的緣故,南朝受到的破壞極其之大,百廢待興。導致陈朝建立时,已经出现南朝转弱,北朝转强的局面。南朝已失去不少領土,到陳朝時,雖然北朝分裂為西魏和東魏(後被北周及北齊取代),長江以北為北齊所佔,西南(四川)被北周所佔,只能靠長江天險維持南北對峙的局面。

陈朝刚建立时(557年)面临北朝的入侵,形势十分危急。陈朝开国皇帝陈霸先带领军队一举击败敌军,形势有所好转。陳霸先於559年病逝,其侄陳文帝陳蒨即位,先後消滅各地的割據勢力,大力革除前朝蕭梁奢侈之風,使陳朝政治稍為安定。天康元年(566年),文帝死,遺詔太子陳伯宗繼位,568年被文帝弟陳宣帝陈顼以陈霸先章太后的名义所廢。宣帝繼續實行文帝時輕徭薄賦之策,使江南經濟逐漸恢復過來。

陈朝在經濟文化上比较发达,除了陳後主以外的幾位君王也都算得上明主,比起南朝其他三個朝代而言政治情勢較為穩定,但在軍事上卻已難以與北方抗衡,即使北方已分裂為北周及北齊,而且北齊皇帝多為殘暴之君,北周由於宇文護專權之故,政治情勢在武帝親政之前也一直不是十分安定,但在武力上南方仍然望塵莫及。

到了陳宣帝時試圖結好北周,夾擊北齊。太建四年(572年),周陳互派使者。翌年的兩年內,陳宣帝以吳明徹為征討大都督,統兵十萬北伐攻打北齊,佔領淮、陰、泗諸城。王琳等忠于南梁的势力和熊昙朗、周迪、留异、陈宝应等在梁末崛起的半独立势力也相继被消灭。

太建九年(577年),北周滅北齊。翌年,周陳在呂梁展開激戰,陳敗周勝,吳明徹被俘,淮南之地得而復失,江北州郡盡為北周所有,回復南北隔江對峙的被動局面。

太建十四年(582年),宣帝病死,太子陳叔寶繼位,是為後主。陳後主不問政事,荒於酒色,陳朝政治江河日下,後主亦自恃長江天險不思進取,被動固守。至於北朝,建立漢人政权隋朝的隋文帝積極準備滅陳。陳朝最後被北方的隋朝在南征之戰中所滅,南北統一。

首都建康為重要的文化、政治、宗教中心,吸引東南亞、印度的商人及僧侶前來。南朝文化在梁朝时达到巅峰,经历了侯景之乱的文化洗劫,到陳朝时已经进入尾声。文学方面以徐陵为文宗,有文学集《玉台新咏》传世,其中最著名篇章《孔雀东南飞》。艺术方面以姚最的评论《续画品录》影响最大。


%% -*- coding: utf-8 -*-
%% Time-stamp: <Chen Wang: 2019-12-23 14:29:49>

\subsection{武帝\tiny(557-559)}

\subsubsection{生平}

陳武帝陈霸先(503年-559年),字兴国,小字法生,吴兴郡长城县(今浙江长兴)人,南北朝時代陳朝開國皇帝。原是南梁的著名軍事將領。557年接受梁敬帝的禪位建立陳朝,557年至559年在位。死後廟號高祖,諡號武皇帝。

梁武帝天监二年(503年)出生,自幼家境贫寒,却好读兵书。初仕乡为里司,后到建康为油库吏,之后又为新渝縣侯萧暎传教(傳令吏)。当时,萧映是广州刺史,于是陈霸先随萧暎来到广州,任中直兵参军。因陈霸先平乱有功,被提任为西江督护,很快又因平交州李賁之乱有功,封为交州司马兼领武平太守(越南永福省永安市附近),后任振远将军、高要太守。梁武帝萧衍曾命使臣將陳霸先畫像帶回,并授予直阁将军一职,封号新安子。

侯景叛乱,陈霸先于梁大宝元年(550年)正月,在始兴(今广东韶关)起兵讨侯景,次年与征东将军王僧辩会合共进。天正二年(552年)三月,领军围石头城(在今南京),大败侯景。因功授征虏将军、开府仪同三司,封司空,领扬州(非今日之扬州市)刺史,镇京口(今江苏镇江)。

梁承圣三年(554年),西魏破江陵,梁元帝被杀。陈霸先与王僧辩请晋安王萧方智以太宰承制,又遣长史谢哲奉笺劝进,晋安王入居朝堂,称梁王。承圣四年(555年),王僧辩屈事北齐,迎立北齐扶植的萧渊明为梁帝,陈霸先苦劝无效,遂诛王僧辩,立萧方智为帝。后又击退北齐的南下侵略,剷平了王僧辩餘党的反抗,晋封陈公,再封陈王,受九锡。

王僧辩的部下王琳得知陈霸先立萧方智为帝,並不服氣,太平二年五月,進攻陈霸先。六月。陈霸先命平西将军周文育、平南将军侯安都等征讨王琳。侯安都至沌口(今武昌)与王琳对峙多日,侯安都军大败。陈霸先再派遣侯瑱、徐度進攻王琳,再派谢哲調解。八月,王琳退军湘州(今湖南长沙),陈霸先以大軍進驻大雷(今安徽望江)。雙方再度對峙,直到陈霸先病逝。

梁太平二年(557年)梁敬帝萧方智禅位,陈霸先代梁称帝建立陈朝,史稱南陳。王琳也立永嘉王萧庄,称帝于荆州。陈永定三年(559年)六月十二日,生病。六月二十一日病逝。因唯一在世亲子陈昌被北周扣留,遗诏追兄子临川王陳蒨入纂。八月甲午,群臣上谥号曰武皇帝,庙号高祖。丙申,葬万安陵(在今南京市江宁区)。隋滅陳後,王僧辯之子王頒是隋军大将,为报父仇,掘陳霸先之墓,挖出骨骸,焚化成灰水喝進肚裡。

在位僅二年,是魏晉南北朝時期中屬於南朝方面十分難得的英明君主,其個性节俭樸素,“常膳不过数品,私宴用瓦器、蚌盘,肴核充事而已;后宫无金翠之饰,不设女乐”。在政治上宽政廉平,爱育为本,恒崇宽政,不行株连,怀柔攻心,诚贯天下。但因建立南陳時,巴蜀地區及淮南已被北周及北齊攻陷,其統治疆域是南朝四代主要政權疆域最小的一個。在经济上,穩定保持了江南的發展。

南陳的吏部尚書姚察在陳亡被俘到隋朝後,為隋文帝撰寫陳朝歷史,仍認為陳霸先「英略大度,應變無方,」與漢高祖劉邦、魏武帝曹操一樣同屬偉人(《陳書》卷一:英略大度,應變無方,蓋漢高、魏武之亞矣)。

唐散騎常侍姚思廉(557年-637年),字簡之,自幼習史,父親是南陳的末任吏部尚書姚察。姚思廉曾任隋朝代王楊侑侍讀。唐朝李淵稱帝後,為李世民秦王府文學館學士。自玄武門之變,進任太子洗馬。貞觀初年,又任著作郎,「十八學士」之一。官至散騎常侍,受命與魏徵同修梁陳二史。貞觀十年(636年),成《梁書》(50卷)《陳書》(30卷),為二十四史之一。他評價陳霸先「智以綏物、武以寧亂、英謀獨運、人皆莫及」。

唐鄭國文貞公魏徵(580年-643年2月11日),字玄成,唐朝貞觀時諫臣,曾是《隋書》、《周書》、《北齊書》、《梁書》、《陳書》五部史書的總監修官。魏徵認為陳霸先效命舊王朝,立下豐功偉績,功勳不下曹操、劉裕;三分天下,能夠「決機百勝」,雄豪無愧劉備、孫權(高祖拔起壟畝,有雄桀之姿。始佐下藩,奮英奇之略。魏王之延漢鼎祚,宋武之反晉乘輿,懋績鴻勳,無以尚也。決機百勝,成此三分,方諸鼎峙之雄,足以無慚權、備矣)。

唐朝大史學家李延壽評價:用「雄武英略」、「性甚仁愛」、「恆崇寬簡」、「彌厲恭儉」 來稱讚陳霸先一生。

北宋《資治通鑒》編撰者司馬光用「臨戎制勝,英謀獨運」、「為政務崇寬簡」、「性儉素」等語言分別概括了陳霸先治軍、從政、為人的鮮明個性。

明朝南京太僕寺丞歸有光評價:恭儉勤勞,志度弘遠,江左諸帝,號為最賢。赫然陳祖,大業光燦。寂寞沛鄉,吾茲感歎。[來源請求]

中國共產黨中央委員會主席毛澤東說他欣賞的是陳霸先南征北戰所使用的戰術。毛澤東在晚年時曾要求人們讀讀《陳書》,瞭解陳霸先的身世經歷。

中華民國作者柏楊在他的一本名為《中國人史綱》的出版品中評道:「陳帝國是南北朝唯一沒有出過暴君的政權。」

\subsubsection{永定}

\begin{longtable}{|>{\centering\scriptsize}m{2em}|>{\centering\scriptsize}m{1.3em}|>{\centering}m{8.8em}|}
  % \caption{秦王政}\
  \toprule
  \SimHei \normalsize 年数 & \SimHei \scriptsize 公元 & \SimHei 大事件 \tabularnewline
  % \midrule
  \endfirsthead
  \toprule
  \SimHei \normalsize 年数 & \SimHei \scriptsize 公元 & \SimHei 大事件 \tabularnewline
  \midrule
  \endhead
  \midrule
  元年 & 557 & \tabularnewline\hline
  二年 & 558 & \tabularnewline\hline
  三年 & 559 & \tabularnewline
  \bottomrule
\end{longtable}


%%% Local Variables:
%%% mode: latex
%%% TeX-engine: xetex
%%% TeX-master: "../../Main"
%%% End:

%% -*- coding: utf-8 -*-
%% Time-stamp: <Chen Wang: 2021-11-01 15:07:48>

\subsection{文帝陈蒨\tiny(559-566)}

\subsubsection{生平}

陳文帝陈\xpinyin*{蒨}(522年-566年),一作茜,又名昙蒨、荃菺,字子華。中国南北朝时期陈朝第二位皇帝(560年—566年在位),在位7年,年号天嘉。

陈蒨是陈朝开国皇帝陈霸先長兄陳道譚的長子,深受陈霸先的賞識與栽培,更令其總理軍政。後來武帝駕崩,因唯一在世儿子陈昌在北周为人质,皇后章要儿听从陈蒨心腹大臣侯安都等安排,稱武帝遺詔命陈蒨入纂皇統,遂即帝位。

北周闻讯,为了制造内乱故意放陈昌回国。因道路一度为东梁所阻隔,天嘉元年陈昌才出发,因而写信要陈蒨让位。陈蒨很不高兴,说:“太子快回来了,我只好找个地方当藩王去养老。”侯安都说:“自古岂有被代天子?”陈昌入陈境后,陈蒨诏令主书舍人沿途迎接,却在陈昌渡江时由侯安都于无人时将其推入长江淹死,对外宣布陈昌在江中因船只故障而溺死。丧柩至京师,陈蒨亲出临哭,追谥号献,风光大葬,又以子陈伯信为其后嗣。

陈蒨在位期間,励精图治,整顿吏治,注重农桑,兴修水利,恢复江南经济。此时陈朝政治清明,百姓富裕,国势强盛,史稱「天嘉小康」。陳蒨亦因而是南朝皇帝中的明君。566年崩,享年44岁,谥号为文帝,庙号世祖。葬于永宁陵(在今南京棲霞區棲霞街道新合村獅子冲)。

陈蒨有一名貌美如婦的寵臣韓子高,《南史》記曰:“子高年十六,為總角,容貌美麗,狀似婦人。”陈蒨在还是临川王时邂逅了這位美少年,從此讓韓子高隨侍左右,寵愛備至。基於這種曖昧事實,所以後世有些小說、戲曲藉題發揮,露骨地將二人描繪成同性愛關係,例如唐朝李翊的《陳子高傳》(明朝馮夢龍的《情史》有節錄)、明朝王驥德的《男王后》等皆是著名創作。

\subsubsection{天嘉}

\begin{longtable}{|>{\centering\scriptsize}m{2em}|>{\centering\scriptsize}m{1.3em}|>{\centering}m{8.8em}|}
  % \caption{秦王政}\
  \toprule
  \SimHei \normalsize 年数 & \SimHei \scriptsize 公元 & \SimHei 大事件 \tabularnewline
  % \midrule
  \endfirsthead
  \toprule
  \SimHei \normalsize 年数 & \SimHei \scriptsize 公元 & \SimHei 大事件 \tabularnewline
  \midrule
  \endhead
  \midrule
  元年 & 560 & \tabularnewline\hline
  二年 & 561 & \tabularnewline\hline
  三年 & 562 & \tabularnewline\hline
  四年 & 563 & \tabularnewline\hline
  五年 & 564 & \tabularnewline\hline
  六年 & 565 & \tabularnewline\hline
  七年 & 566 & \tabularnewline
  \bottomrule
\end{longtable}

\subsubsection{天康}

\begin{longtable}{|>{\centering\scriptsize}m{2em}|>{\centering\scriptsize}m{1.3em}|>{\centering}m{8.8em}|}
  % \caption{秦王政}\
  \toprule
  \SimHei \normalsize 年数 & \SimHei \scriptsize 公元 & \SimHei 大事件 \tabularnewline
  % \midrule
  \endfirsthead
  \toprule
  \SimHei \normalsize 年数 & \SimHei \scriptsize 公元 & \SimHei 大事件 \tabularnewline
  \midrule
  \endhead
  \midrule
  元年 & 566 & \tabularnewline
  \bottomrule
\end{longtable}


%%% Local Variables:
%%% mode: latex
%%% TeX-engine: xetex
%%% TeX-master: "../../Main"
%%% End:

%% -*- coding: utf-8 -*-
%% Time-stamp: <Chen Wang: 2021-11-01 15:07:58>

\subsection{废帝陳伯宗\tiny(566-568)}

\subsubsection{生平}

陳伯宗(554年6月12日或552年6月20日-570年4月22日),字奉業,小字藥王,南朝陳朝的第三代皇帝,史書稱作「廢帝」,陳文帝陳蒨之嫡出長子,母安德皇后沈妙容。

天康元年(566年)四月,陳伯宗在陳文帝死後即帝位,由於陳伯宗年幼,便以叔父安成王陳頊為司徒、錄尚書事、都督中外諸軍事。於是政局都為陳頊所掌握。567年改年號為光大,陳頊晉位為太傅,准許佩帶劍履上殿。光大二年(568年)11月,陳頊叛逆廢陳伯宗為臨海王,自立為帝,是為陳宣帝。

陳伯宗於被廢之後,於太建二年四月乙卯(570年4月22日)逝世,得年十九歲。

陈伯宗的出生时间,《陈书·废帝本纪》有具体记载是梁承圣三年(554年)五月庚寅生(554年6月12日)。可是《陈书·废帝本纪》又记载:“(陈伯宗)太建二年(570年)四月薨,时年十九”。由此推算陈伯宗出生时间为552年(十九岁是虚岁)。这样一来,《废帝本纪》就在陈伯宗年龄方面自相矛盾了。按照陈伯宗于554年出生的说法推算,陈伯宗死时的年龄为十七岁。而陈伯宗同母弟陈伯茂的年龄,《陈书》记载“光大二年,皇太后令黜废帝为临海王,其日又下令降伯茂为温麻侯。时六门之外有别馆,以为诸王冠昏之所,名为昏第。至是命伯茂出居之,宣帝遣盗殒之于车中,年十八”。由此推算可知陈伯茂大约也生于552年。陈伯宗為陈伯茂兄,出生时间应该是公元552年(552年五月庚寅,即552年6月20日),卒时年十九岁。


\subsubsection{光大}

\begin{longtable}{|>{\centering\scriptsize}m{2em}|>{\centering\scriptsize}m{1.3em}|>{\centering}m{8.8em}|}
  % \caption{秦王政}\
  \toprule
  \SimHei \normalsize 年数 & \SimHei \scriptsize 公元 & \SimHei 大事件 \tabularnewline
  % \midrule
  \endfirsthead
  \toprule
  \SimHei \normalsize 年数 & \SimHei \scriptsize 公元 & \SimHei 大事件 \tabularnewline
  \midrule
  \endhead
  \midrule
  元年 & 567 & \tabularnewline\hline
  二年 & 568 & \tabularnewline
  \bottomrule
\end{longtable}



%%% Local Variables:
%%% mode: latex
%%% TeX-engine: xetex
%%% TeX-master: "../../Main"
%%% End:

%% -*- coding: utf-8 -*-
%% Time-stamp: <Chen Wang: 2019-12-23 14:33:06>

\subsection{宣帝\tiny(568-582)}

\subsubsection{生平}

陈宣帝陈\xpinyin*{顼}(530年8月14日-582年2月17日),又名陈昙顼,字绍世,小字师利,南北朝时期陈朝第四位皇帝(569年—582年在位),正式諡號為「孝宣皇帝」,後世比照漢朝和西晉皇帝省略「孝」字,稱「陳宣帝」,在位14年,年号太建。

陈顼出生于南朝梁中大通二年(530年)七月初六日(530年8月14日)。陳頊是高祖武皇帝陈霸先的侄子,始兴昭烈王陈道谭第二子,世祖文皇帝陈蒨的弟弟。他本来是皇帝陈伯宗的辅佐大臣,后叛逆废掉了陈伯宗,篡位为帝。

他在位期间,兴修水利,开垦荒地,鼓励农民生产,社会经济得到了一定的恢复与发展。573年(太建五年),派大将吴明彻乘北齐大乱之机北伐,攻占了吕梁(在今江苏徐州附近)和寿阳,一度占有淮、泗之地,但最后在577年被北周夺走。总的来说,陈顼在位期间,国家比较安定,政治也较为清明。陈顼于陈太建十四年崩(582年)正月初十日(582年2月17日),享年53岁。

陈顼谥号为孝宣皇帝,庙号高宗。葬显宁陵(在今南京郊区)。

\subsubsection{太建}

\begin{longtable}{|>{\centering\scriptsize}m{2em}|>{\centering\scriptsize}m{1.3em}|>{\centering}m{8.8em}|}
  % \caption{秦王政}\
  \toprule
  \SimHei \normalsize 年数 & \SimHei \scriptsize 公元 & \SimHei 大事件 \tabularnewline
  % \midrule
  \endfirsthead
  \toprule
  \SimHei \normalsize 年数 & \SimHei \scriptsize 公元 & \SimHei 大事件 \tabularnewline
  \midrule
  \endhead
  \midrule
  元年 & 569 & \tabularnewline\hline
  二年 & 570 & \tabularnewline\hline
  三年 & 571 & \tabularnewline\hline
  四年 & 572 & \tabularnewline\hline
  五年 & 573 & \tabularnewline\hline
  六年 & 574 & \tabularnewline\hline
  七年 & 575 & \tabularnewline\hline
  八年 & 576 & \tabularnewline\hline
  九年 & 577 & \tabularnewline\hline
  十年 & 578 & \tabularnewline\hline
  十一年 & 579 & \tabularnewline\hline
  十二年 & 580 & \tabularnewline\hline
  十三年 & 581 & \tabularnewline\hline
  十四年 & 582 & \tabularnewline
  \bottomrule
\end{longtable}



%%% Local Variables:
%%% mode: latex
%%% TeX-engine: xetex
%%% TeX-master: "../../Main"
%%% End:

%% -*- coding: utf-8 -*-
%% Time-stamp: <Chen Wang: 2019-12-23 14:35:03>

\subsection{后主\tiny(582-589)}

\subsubsection{生平}

陈叔宝(553年12月或554年1月-604年),字元秀,小字黄奴。南北朝时期陈朝末代皇帝(第五代,582年—589年在位),史称“后主”,在位7年,年号至德、祯明。

陈叔宝出生于梁朝承圣二年十一月(553年12月21日或,554年1月18日),是陈宣帝陈顼嫡长子,皇后柳敬言所生。

虽然身为太子,但是其皇位却来得十分不易。陈宣帝的次子、陈叔宝的弟弟陈叔陵一直有篡位之心,谋划刺杀陈叔宝。宣帝去世后,叔宝在宣帝灵柩前大哭,叔陵趁机用磨好的刀砍击叔宝,击中颈部,但没有造成致命伤害,叔宝在左右的护卫下逃出,派大将萧摩诃讨伐叔陵。最后叔陵被杀,叔宝即皇帝位,就是陈朝末代皇帝—陈后主。

在位时大建宫室,生活奢侈,不理朝政,日夜与妃嫔、文臣游宴,制作艳词,隋军南下时,自恃长江天险,不以为然。

陈叔宝是一个荒淫無度的皇帝,“奏伎縱酒,作詩不輟”(《南史·陳本紀》),又大建宫室,滥施刑罚,寵愛美女張麗華,朝政极度腐败。張貴妃名叫張麗華,十歲時,充當龔貴嬪的侍女,陳後主一見鍾情,封為貴妃,視為至寶,以至臨朝之際,百官奏事,都讓張麗華坐於膝上或將其抱在懷裡,同決天下大事。特別是張麗華為他生下兒子陳渊之後,使陳後主對張麗華這個傾國傾城的美人,寵愛萬分,在他心目中的地位更加提高、鞏固,陳渊也立刻被立為太子。

陳後主即位時,隋朝的隋文帝楊堅正大舉任賢納諫,減輕賦稅,整飭軍備,消除奢靡之風。隨時準備攻略江南富饒之地,而陳後主竟然奢侈荒淫無度,臣民也流於逸樂,給隋朝以可乘之機。陳後主除寵愛張貴妃之外,還有龔貴嬪、孔貴嬪,還有王、李二美人,還有張、薛二淑媛,還有袁昭儀、何婕妤、江修容等美人。當時陳後主在光照殿前,又建「臨春」、「結綺」、「望仙」三閣,高聳入雲,其窗牖欄檻,都以沉香檀木來做,極盡奢華,宛如人間仙境。陳後主自居臨春閣,張麗華住結綺閣,龔孔二貴嬪同住望仙閣。三閣都有凌空銜接的覆道,陳後主往來於三閣之中,左右逢源,得其所哉!妃嬪們或臨窗靚裝,或倚欄小立,風吹袂起,飄飄焉若神仙。此外陳後主更把中書令江總,以及陳暄、孔范、王瑗等一般文學大臣一齊召進宮來,清歌妙舞,飲酒賦詩,自夕達旦。

隋文帝开皇八年三月,下诏:“天之所覆,无非朕臣,每关听览,有怀伤恻。可出师授律,应机诛诊,在期一举,永清吴越。”于是发兵五十一万八千人,由晋王杨广指揮,進攻陈朝都城建康。晋王杨广由六合出发,秦王杨俊由襄阳顺流而下,清合公杨素由永安誓师,荆州刺史刘思仁由江陵东进,蕲州刺史王世绩由蕲春发兵,庐州总管韩擒虎由庐江急进,还有吴州总管贺若弼及青州总管燕荣分别由庐江与东海赶来会师。陈叔宝恃长江天险,不以为意,只顧與張麗華飲酒玩樂,縱慾淫樂,且说:“王气在此,齐兵三度来,周兵再度至,无不摧没。虏今来者必自败。”翌年(陈祯明三年,即隋开皇九年,589年)正月,南征之戰隋军分道攻入建康(今江苏南京)。其中韩擒虎亲率五百名精锐士卒自横江夜渡采石矶,紧接着贺若弼攻拔京口,形成两路夹击,最先进入朱雀门的是韩擒虎。当时陈后主惊荒失措,他身边的侍臣只有尚書左僕射袁憲一人。當時袁憲建議:「臣願陛下正衣冠,御前殿,依梁武見侯景故事。」(《陳書·袁憲傳》)但陈后主不理会,只说:“非唯朕无德,亦是江南衣冠道尽,吾自有计,卿等不必多言!吾自有计。”与爱妃张丽华、孔贵嬪避入井中,后被俘,隋軍一面掃蕩残敵,令後主手書招降陈朝未降将帅,一面收圖籍,封府庫,又将張麗華、孔貴嬪梟首於青溪中橋及施文慶、沈客卿、阳慧朗、徐析、暨慧景等奸佞斬於右闕下。陳朝宣告覆亡,隋文帝終於统統一了全國。长达四百多年的魏晋南北朝时代结束,中國進入大一統的隋朝。

楊堅對陳叔寶極為優待,准許他以三品官員身分上朝。又常邀請他參加宴會,恐他傷心,不奏江南音樂,而後主卻從未把亡國之痛放在心上。一次,監守他的人報告文帝說:「陳叔寶表示,身無秩位,入朝不便,願得到一個官號。」文帝嘆息說:「陳叔寶全無心肝。」監守人又奏:「叔寶常酗酒致醉,很少有清醒的時候。」隋文帝讓後主節酒,過了不久又說:「由着他的性子喝吧,不這樣,他怎樣打發日子呀!」過了一些時候,隋文帝又問後主有何嗜好,回答說:「好食驢肉。」問飲酒多少,回答說:「每日與子弟飲酒一石。」讓隋文帝相當驚訝。

隋文帝東巡邙山,後主奉召前往,他在宴會上賦詩說:「日月光天德,山川壯帝居,太平無以報,願上東封書。」表請封禪,隋文帝不許。楊堅評價說:「陳叔寶的失敗皆與飲酒有關,如將作詩飲酒的功夫用在國事上,豈能落此下場!當賀若弼攻京口時,邊人告急,叔寶正在飲酒,不予理會;高熲攻克陳朝宮殿,見告急文書還在床下,連封皮都沒有拆,真是愚蠢可笑到了極點,陳亡也是天意呀!」

陈叔宝死于隋仁寿四年(604年),得年五十二岁。追贈大將軍、長城縣公。諡號煬。

有诗《玉树后庭花》传世:“麗宇芳林對高閣,新裝豔質本傾城;映戶凝嬌乍不進,出帷含態笑相迎,妖姬臉似花含露,玉樹流光照後庭。”

據史記載,陳後主某日到沈婺華處,暫入即還,卻寫了一首詩《戲贈沈后》:「留人不留人,不留人去也。此處不留人,自有留人處。」婺華《答後主》:「誰言不相憶,見罷倒成羞。情知不肯住,教遣若為留。」

陳後主詩:「考差蒲未齊,沈漾若浮綠,朱鷺戲蘋藻,徘徊流澗曲。澗曲多巖樹,逶迤復繼續,振振難以明,湯湯今又矚。」

\subsubsection{至德}

\begin{longtable}{|>{\centering\scriptsize}m{2em}|>{\centering\scriptsize}m{1.3em}|>{\centering}m{8.8em}|}
  % \caption{秦王政}\
  \toprule
  \SimHei \normalsize 年数 & \SimHei \scriptsize 公元 & \SimHei 大事件 \tabularnewline
  % \midrule
  \endfirsthead
  \toprule
  \SimHei \normalsize 年数 & \SimHei \scriptsize 公元 & \SimHei 大事件 \tabularnewline
  \midrule
  \endhead
  \midrule
  元年 & 583 & \tabularnewline\hline
  二年 & 584 & \tabularnewline\hline
  三年 & 585 & \tabularnewline\hline
  四年 & 586 & \tabularnewline
  \bottomrule
\end{longtable}

\subsubsection{祯明}

\begin{longtable}{|>{\centering\scriptsize}m{2em}|>{\centering\scriptsize}m{1.3em}|>{\centering}m{8.8em}|}
  % \caption{秦王政}\
  \toprule
  \SimHei \normalsize 年数 & \SimHei \scriptsize 公元 & \SimHei 大事件 \tabularnewline
  % \midrule
  \endfirsthead
  \toprule
  \SimHei \normalsize 年数 & \SimHei \scriptsize 公元 & \SimHei 大事件 \tabularnewline
  \midrule
  \endhead
  \midrule
  元年 & 587 & \tabularnewline\hline
  二年 & 588 & \tabularnewline\hline
  三年 & 589 & \tabularnewline
  \bottomrule
\end{longtable}



%%% Local Variables:
%%% mode: latex
%%% TeX-engine: xetex
%%% TeX-master: "../../Main"
%%% End:



%%% Local Variables:
%%% mode: latex
%%% TeX-engine: xetex
%%% TeX-master: "../../Main"
%%% End:

%% -*- coding: utf-8 -*-
%% Time-stamp: <Chen Wang: 2019-12-23 15:22:02>


\section{北魏\tiny(386-534)}

\subsection{简介}

北魏(386年-534年)是北朝時期的第一個朝代,由鲜卑人拓跋珪所建立,定都平城(遗址在今山西省大同市)。439年,魏太武帝统一华北,与南方的汉人政权对峙。

494年,魏孝文帝迁都洛阳。495年,孝文帝下诏首先在宫廷中禁止包括鲜卑语在内的诸北语,改说汉语,但对三十岁以上的人有所宽限。496年,孝文帝诏令鲜卑八大贵族全部改为汉姓,并将皇族姓氏拓跋改为元姓。534年,北魏分裂為被高歡掌控的東魏(都邺城)與被宇文泰掌控的西魏(都長安)。东魏武定八年(550年),高洋废魏孝静帝,代东魏自立,建立北齐。西魏于恭帝三年(557年)魏恭帝被权臣宇文护逼迫禅位于堂弟宇文觉,建立北周,172年的元魏历史才正式宣告结束。

拓跋氏自称是黄帝后裔,黄帝发源地为战国时魏国所在,又“魏”有美好之意,故以此名国号。以其领土位于中国北方,又是北朝的第一个政权,故史称“北魏”。为别于此前三国時期的曹魏政权,某些史书因此别称为“后魏”,但由于史学界不称曹魏为“前魏”,故“后魏”之称很少使用。又以其王室姓拓跋,后改姓元,故又别称拓跋魏(东魏和西魏虽然姓拓跋,但是多數史学家并不如此称呼这两个政权)、元魏。

在公元四世纪初,拓跋鲜卑在今山西北部和今內蒙古等地建立代國。376年被前秦所吞并。淝水之战后,前秦统治瓦解。386年,拓跋珪即代王位,重建代国。同年四月,改国号为魏。398年(天興元年)建都平城,次年称帝。即為魏道武帝。

396年道武帝改元皇兴,率40万大军征讨后燕,一举攻下中山、信度、邺城,平定中原。经过明元帝时代的承平岁月,到北魏太武帝时,于427年攻破夏国首都统万城,428年占领安定,逐走赫连夏后主赫连定。436年攻破和龙,灭亡北燕冯氏。太延五年(公元439年)吞并北凉沮渠氏。442年西凉残余势力李宝投降北魏。443年仇池杨保炽投降北魏。至此北魏完成了兼併华北地区和北方,這時华南地区和南方早已是劉宋,南北各自为政,形成互不隶属的對峙之局。

在統一華北以前,北魏就有多次與南朝政權在黃淮下游交手的經驗。顯祖獻文帝皇興三年(469),北魏上黨公、征南大將軍慕容白曜攻下南朝宋所屬的青州治所東陽城,至此以後,現今山東半島,又屬黃淮下游古稱青齊的地區(《尚書‧禹貢》稱「海岱之地」)就歸北魏所管,並被割劃為青州、齊州、濟州、光州等區。

早在馮太后掌政時代,馮太后已推行了一系列措施建立國家規模,如在太和九年(485年)推行均田制,把之前因为戰亂而遺下的無主荒地按制度分給存活的農民,一部分可永久擁有,一部分則身死後交還公家。又施行租調制,農民按制度上數字,定期向朝廷納稅。

孝文帝親政後更在文化上開始修整,在风雨飘摇之中的背景下為了维持和巩固政权,進行了許多大刀阔斧的改革,即後世之所謂孝文漢化,其舉措大略如下:

一、遷洛陽:孝文帝以舊都平城(今山西省大同市)為用武之地,非可文治,而洛邑為歷史名都,物富民豐,交通便利,便於經略海內,控制中原,魏太和十七年(493年),以南伐為名,進駐河洛,定為京師。遷都洛阳後,戎裝以外,官民皆著漢服。

二、改漢姓:《魏書‧官氏志》記載了一百一十八個胡人改姓的例子,如皇族拓跋氏改元姓、步六孤改陸姓、賀賴氏改賀姓、獨孤改姓劉。

三、斷胡語:凡三十歲以下官員一律使用漢語,如果仍用鮮卑語,即降爵黜官。

四、通婚姻:鼓勵與漢族世家通婚,並從己身開始迎娶漢族士族女子。

五、重教育:祀孔子,尊儒教,尋古書,設立太學、小學。

自此胡漢界線开始逐漸消弭,對当时和后世發展意義非凡。

第八任皇帝魏宣武帝元恪立他的儿子元诩当太子时,没有按旧制处死太子的母亲胡贵嫔,導致外戚及士族掌權。元诩即位后,胡贵嫔为皇太后,後六鎮之亂爆發。胡党毒死元诩,立元钊,大将尔朱荣趁势讨伐,立元子攸,在河阴之变之后掌控朝政。元颢、元悦等宗室都因河阴之变而南下投靠梁朝。

529年,梁武帝派陈庆之攻陷洛阳,立元颢为帝。陈庆之目睹洛阳的衣冠、礼仪、人才不输南朝,心生感慨。元颢政权不久为尔朱荣所败。

孝莊帝元子攸不能容忍尔朱荣跋扈,在尔朱荣回朝后设计铲除之,梁武帝亦于530年趁机派兵拥立元悦为帝。但不久后孝庄帝就被尔朱家族所杀害,元悦见状亦放弃称帝而南归。尔朱氏立元晔为帝,又废元晔改立元恭。

尔朱家族大将高欢倒戈,立元朗为帝,讨伐尔朱家族,并取得胜利。高欢在532年以元恭为尔朱氏所立、元朗世系疏远为由,皆废黜,立元修为帝。曾为皇帝的元晔、元恭、元朗及北返的元悦皆被元修所杀。元修不能容忍高欢掌握朝政,在534年投奔长安的宇文泰,而宇文泰杀元修,另立元宝炬为帝,建都长安,史称“西魏”。

高欢另立元善见为帝,建都邺城(今河北临漳),史称“东魏”。北魏就此分裂。

北魏的宮廷為了避免外戚干政,實施殘酷的子貴母死制度,即後宮女性只要生下皇子就得被賜死,以避免母以子貴的情況發生。但幼子還是需要人照顧,因此就有所謂的保太后,即以太子的保母在太子繼位後成為皇太后。北魏有三種皇太后,一種是皇帝的生母,另一種是皇帝的保姆,還有一種是未曾替前任皇帝生皇子因而存活的皇后。如北魏献文帝乃由漢人女子李贵人所生,然李贵人在生下献文帝以後即被賜死,由身為太后的馮氏所養大。献文帝曾求當時當權的馮太后廢除舊法,但被拒絕。後來直到篤信佛教的北魏宣武帝,才終於取消子貴母死,但他卻導致北魏的外戚爭權,最終導致北魏滅亡及分裂。

北魏文成帝拓跋濬是北魏太武帝的孙子,其父拓跋晃没有做过皇帝,文成帝也并非以储君身份登基,故其生母郁久闾氏未曾被赐死,在文成帝登基之初尚在人世,但不久后,身为在位皇帝之母的她也因“子贵母死”制度所累而死。

魏孝文帝改革之前,北魏的税收由部落贡纳、牧民的畜牧税为以及一般农民的租调为主,其中农民的租调为最大收入。北魏规定租调税收为“户调帛二匹、絮二斤、丝一斤、粟二十石;又入帛一匹二丈,委之州库,以供调外之费。”。不过这是个一般办法,政府需要的时候可以增加征收物品的种类和数量。租调是按户收取的,户的大小没有限制,孝文帝改革之前,三五十家组成一户的情形很普遍。除了这种一般性税收外,政府经常因为战争而加开新税,官吏因为没有官俸,也常常以各种借口征税,给百姓带来很大的负担。

徭役方面,兵役方面由鲜卑人担任,因而兵役较轻。而力役的情况因为缺少史料,无法得知。只知道,为政府工作的工、杂役非常多。他们被编为隶户、军户、营户、府户、绫罗户、乐户等等。

孝文帝改革后,为了给官僚机构提供俸禄,以减少官吏欺压百姓。提高了税率,魏孝文帝定每户增调帛三匹、谷二斛九斗,充百官俸禄。又在太和九年(485年)实行均田制,办法大致有四项:十五岁以上的男丁和妇人均可授田,男丁授露田四十亩,妇人二十亩,授田视轮休需要加倍或再加倍。如果有牛一头则授田三十亩,最多四头牛,多出的不授田。老少病残或者缺乏男丁的家庭十一岁以上和有病者均授予半夫之田。奴婢一样按照男丁和妇人的标准授田。授田不准买卖,年老或身死还田,但七十以上授田者不必归还。男丁授桑田20亩。桑田不必还给国家,可传给子孙,也可以可卖出多于20亩的部分,也可买桑田补足20亩。产麻地男子授麻田10亩,妇人50亩,年老及身死后还田。多余土地可以借给农民耕种,政府严格控制农民迁徙,只允许迁往空荒地区。规定驻地长官在所在地给予公田,刺史十五顷,太守十顷,治中别驾八顷,县令郡丞六顷,不许买卖。

政府在均田制的基础上重新规定了税收制度,一夫一妻应缴纳的租调为:“其民调,一夫一妇帛一匹,粟二石。民年十五以上未娶者,四人出一夫一妇之调;奴任耕、婢任织者八口当未娶者四;耕牛二十头当奴婢八。其麻布之乡,一夫一妇布一匹,下至牛,以此为降。”

北魏兵民分开,兵用于打仗,民从事耕桑。而兵主要由鲜卑及其他少数民族组成,农业主要由汉人从事。兵民之分也就是胡汉之分,也是胡汉分治的体现。

而士兵里面也分两种,一种是鲜卑兵,另外一种是非鲜卑兵。

鲜卑兵由代北部落的鲜卑人组成,主要担任北魏的禁旅和边防六镇的士兵。这种兵带贵族性质,地位颇高,但在魏文帝汉化之后有所改变。

非鲜卑兵中,以高车兵最为重要,禁军和六镇边兵都有高车人。此外还有部分少数民族和汉人军队。

北魏经历了游牧部落联盟而迅速转移到国家的历史,拓跋鲜卑人有自己的语言而没有文字。北魏时期的主要宗教是佛教、道教和琐罗亚斯德教,其中最重要的是佛教,僧尼的人数曾发展到二百多万。北魏道教,主要是经过寇谦之改良的天师道。当时佛道两家的斗争十分激烈,太武帝拓跋燾曾经大舉灭佛。琐罗亚斯德教在中国称为祆教或拜火教,主神被称为“胡天”,主要在入华的粟特人当中传播。孝文帝在平城(大同)开凿了云岗石窟。

北魏大部分時期,由於國家及私人贊助,佛教藝術十分興盛。雲崗佛教石窟約興建於西元四六〇年,由上千位工匠歷時約三十五年後完工,洞窟內有雕塑及與繪畫。之後,北魏孝文帝亦於龍門興建石窟。雲崗石窟的佛像屬較靜態的罽賓風格,龍門的造像形式則較流線飄逸,開始展現中國風格的影響。北魏的陪葬陶器亦受到佛教影響,強調「正面性」(frontality) 及對稱。

%% -*- coding: utf-8 -*-
%% Time-stamp: <Chen Wang: 2019-12-23 14:50:36>

\subsection{道武帝\tiny(386-409)}

\subsubsection{生平}

魏道武帝拓跋珪(371年-409年11月6日),又名涉珪、什翼圭、翼圭、開,北魏开国皇帝,代王拓跋什翼犍之孙,獻明帝拓跋寔和贺夫人之子。

拓跋珪出生于371年8月4日。376年,前秦滅代國,拓跋珪將要被強遷至秦都長安,但代王左長史燕鳳以拓跋珪年幼,力勸前秦天王苻堅讓拓跋珪留在部中,稱待拓跋珪長大後為首領,會念及苻堅施恩給代國。苻堅同意,拓跋珪得以留下。其時,代國舊部由劉庫仁及劉衞辰分掌,拓跋珪母賀氏帶拓跋珪、拓跋儀及拓跋觚從賀蘭部遷至獨孤部,與南部大人長孫嵩等人同屬劉庫仁統領。劉庫仁本亦為南部大人,拓跋珪等人到後仍盡忠侍奉他們,並沒有因代國滅亡、自己改受前秦官位而變節,又招撫接納離散的部人,甚有恩信。

383年,苻堅於淝水之戰中戰敗,其後國中大亂,劉庫仁助秦軍對抗後燕,但於次年遭慕輿文夜襲殺害,其弟劉頭眷代領其眾。385年,劉庫仁之子劉顯殺頭眷自立,又想要殺拓跋珪。劉顯弟劉亢埿的妻子是拓跋珪的姑姑,並將劉顯的意圖告訴賀氏。劉顯謀主梁六眷是拓跋什翼犍的甥子,也派部人穆崇、奚牧將此事密報拓跋珪。賀氏於是約劉顯飲酒,將其灌醉,讓拓跋珪與舊臣長孫犍、元他等人乘夜逃至賀蘭部。不久,劉顯部中內亂,賀氏得以到賀蘭部與拓跋珪等會合。但其時賀氏弟賀染干忌憚拓跋珪得人心,曾試圖殺害他,但都因尉古真告密及賀氏出面而失敗。而拓跋珪的堂曾祖父拓跋紇羅及拓跋建就勸賀蘭部首領賀訥推拓跋珪為主。

登國元年正月六日(386年2月20日),拓跋珪得到以賀蘭部為首的諸部支持在牛川大會諸部,召開部落大會,即位為代王,年號登國。拓跋珪任用賢能,勵精圖治,重興代國。即位不久,便移都代國原都盛樂,並推動農業,讓人民休養生息。同年四月,改稱魏王,稱國號為魏,史稱北魏。

北魏建立時四週有強敵環伺,北有賀蘭部、南有獨孤部、東有庫莫奚部、西邊在河套一帶有匈奴鐵弗部、陰山以北為柔然部和高車部、太行山以東為慕容垂建立的後燕及以西的慕容永統治的西燕。因為叔父拓跋窟咄為了爭位與劉顯勾結,企圖取拓跋珪而代之形成內部不穩,于桓等人意圖殺害拓跋珪以響應窟咄,莫題等人亦與窟咄通訊。拓跋珪殺死于桓等五人,赦免莫題等七姓,但都因恐懼內亂而往依賀蘭部,借陰山作屏障防守,又派人向後燕求援。

同年十一月,拓跋窟咄逼近,部眾惶恐不安。慕容垂之子慕容麟帶領的後燕援軍此時仍未到,於是先讓北魏使者安同先回去,讓魏人知燕軍已在附近,穩定人心。拓跋珪於是領兵會合後燕援軍,在高柳大敗拓跋窟咄。窟咄帶領殘兵敗將西逃,依附鐵弗部,被鐵弗部首領劉衞辰殺死,拓跋珪接收其部眾。十二月,後燕任命拓跋珪為西單于,封上谷王,但拓跋珪不受。

次年,拓跋珪與後燕聯手擊敗劉顯,逼劉顯出奔西燕。六月,拓跋珪又於弱落水大敗庫莫奚部;七月再擊敗來攻的庫莫奚。登國四年(388年),拓跋珪大破高車諸部。登國五年(389年),拓跋珪又西征高車袁紇部,並在鹿渾海大敗對方,俘獲人口及牲畜共計二十多萬。不久更聯同慕容麟所率的後燕軍進攻賀蘭部、紇突隣部及紇奚部,後兩者向北魏請降。七月,賀蘭部遭鐵弗部攻擊,賀訥於是向北魏投降求援,拓跋珪於是領兵去救援,擊退鐵弗,並將賀訥等人遷至東界。

拓跋珪進擊高車諸部,唯獨柔然不肯降魏,遂於登國七年(391年)進攻柔然。柔然當時率眾退避,拓跋珪追擊,軍糧用盡後以備乘戰馬作軍糧,終在南牀山追及,並俘獲其一半部眾。接著拓跋珪繼續派兵追擊餘部,逼令首領縕紇提投降。同年,拓跋珪進攻鐵弗,直攻代來城,擒獲直力鞮,衞辰被部下殺害。拓跋珪更盡誅劉衞辰宗族共五千多人,將屍體丟在黃河中。此戰後,黃河以南諸部都向北魏投降。北魏至此亦已擊敗大部份強鄰,國力亦大增。

北魏與後燕皆是386年建立,後燕強而北魏弱,拓跋珪與後燕結好,而北魏開國之初的內亂,後燕亦曾出兵支援拓跋珪,每年兩回亦派使者往來。登国六年(391年),賀蘭部內亂,賀染干和賀訥互相攻擊,拓跋珪亦自請為響導,請後燕出兵討伐。但同年,后燕將來使拓跋觚扣留,以向北魏求名马。拓跋珪拒絕,拓跋觚亦一直遭扣留,此后两国关系惡化。北魏轉而聯結西燕对付後燕。但後燕帝慕容垂於登国九年(394年)六月出兵進攻西燕,圍攻長子,西燕帝慕容永曾向北魏求援,拓跋珪遂派陳留公拓跋虔及庾岳救援西燕,可是援軍尚未趕到,長子就失陷。慕容永及其公卿大將三十多人都被誅殺,西燕滅亡。華北一帶就剩下北魏與後燕两国互相對峙。

登国十年(395年)北魏侵逼後燕附塞諸部,慕容垂就於同年五月派其太子慕容寶伐魏。拓跋珪知大軍前來,率眾到河西避戰。燕軍於七月到五原後收降魏別部三萬多家人,又收穄田穀物及造船打算渡河進攻。拓跋珪亦進軍河邊,與燕軍對峙。北魏一方面派許謙向後秦請求援兵,一面卻派兵堵截燕軍與後燕都城中山的道路,並抓住取道去前線的燕國使者。因著慕容垂在出兵時已經患病,而堵截道路令慕容寶久久都不知道國內消息,拓跋珪於是逼令抓到的使者向燕軍謊稱慕容垂的死訊,成功動搖燕軍將士的軍心。燕魏兩軍自九月起隔河對峙至十月,燕軍終因內亂而被逼燒船撤退。其時黃河河水未結,魏軍未能及時渡河追擊。但次月大風令河面結冰後,拓跋珪即下令渡河並派二萬多精騎追擊燕軍。魏军在参合陂打败燕军,俘獲大量燕軍將士及官員,拓跋珪除了選用有才的如賈閏、賈彜等人留下外,將其他官員都送回後燕,但同時將燕兵都坑殺。史称參合陂之戰。

登国十一年(396年)三月,慕容垂率軍再度伐魏,攻陷平城(今山西大同市),留守平城的拓跋虔戰死,守城的三萬餘家部落皆被俘。接著慕容垂更派慕容寶等進逼拓跋珪。拓跋珪此時十分驚懼,打算離開盛樂避兵,而諸部因驍勇善戰的拓跋虔戰死,亦有異心,令拓跋珪不知所措。可是慕容垂因見參合陂堆積如山的燕兵屍體而發病,被逼退兵,並病逝于上谷。同年七月,拓跋珪建天子旌旗,並改元皇始,並正式圖取後燕所佔的中原土地。

皇始元年(396年)八月,拓跋珪就大舉伐燕,親率四十多萬大軍南出馬邑,越過句注南攻後燕并州,同時又命封真率偏師進攻後燕幽州。九月,魏軍進至晉陽,守城的慕容農出戰但大敗,晉陽城守將此時叛燕逼使慕容農率眾東走。長孫肥率眾追擊,在潞川追上,慕容農妻兒被擄,只能與三騎逃回中山。北魏遂奪取後燕并州 ( 今山西地區 )之地,並置官員治理當地。

隨後,拓跋珪命于栗磾及公孫蘭等暗中開通昔日韓信在井陘用過的路,並在同年十月,越過太行山率軍取道該路進攻後燕京師中山城 ( 河北省定縣 )。其時燕軍決意嬰城自守,打持久戰,於是拓跋珪在攻下常山後,其東各郡縣的官員不是棄城就是投降,北魏於是輕易地得到中原大部分郡縣歸附,僅餘中山城、鄴城及信都城三城仍然拒守。拓跋珪於是兵分三路分攻三城:自攻中山,拓跋儀攻鄴及王建、李栗攻信都。然而,拓跋珪在攻中山城時遭燕軍力拒,於是暫時放棄中山城,改而南取其餘二城。

皇始二年(397年)正月,拓跋珪加入進攻信都城,終於逼得守將慕容鳳棄城出走,但其時慕容德卻成功離間進攻鄴城的拓跋儀及賀賴盧,令他們退兵,並乘機從後追擊,大破魏軍。

上一年,為拓跋珪憎惡的魏將沒根自疑而叛魏投燕,其侄兒醜提恐怕會被株連,於是決定自并州率部回北魏後方作亂。拓跋珪見內亂起,於是自後燕求和,但慕容寶卻意圖乘此機反擊,拒絕之餘更派步兵十二萬及騎兵三萬七千出屯柏肆,在滹沱水以北阻擊魏軍。魏軍在滹沱水南岸設營,燕軍於是乘夜渡水進攻,以萬餘兵突襲魏營,並乘風勢放火。魏軍此時大亂,拓跋珪慌忙起來棄營逃跑,僅而避過攻到其帳下的燕將乞特真。可是,燕軍此時卻無故自亂,互相攻擊,拓跋珪在營外見到,就擊鼓收拾餘眾,集結好後進攻營內燕軍,並乘勢進攻營北作支援的慕容寶軍,逼使慕容寶退回北岸。此戰後,燕軍士氣大降,而魏軍卻已重整。拓跋珪乘慕容寶撤退的機會追擊,屢敗燕軍。慕容寶恐懼下更拋下大軍率二萬騎兵速返中山;又怕被追上,命令士兵拋棄戰衣及兵器輕裝撤還。其時大量燕兵因大風雪而凍死,很多後燕朝臣及兵將都被俘或投降。

三月,慕容寶向拓跋珪求和,並說要送還拓跋觚,並割讓常山以西土地。拓跋珪已答允,但慕容寶卻反悔,拓跋珪於是進圍中山。最終慕容寶等人棄中山城出走,拓跋珪原本打算在該晚入城,王建則以士兵會乘夜盜取城中財寶為由勸阻,拓跋珪於是等到日出才入城。可是慕容詳卻趁機自立為主,閉門拒守,拓跋珪試圖強攻但攻了幾日都不果,於是試圖勸降,可是城中軍民卻表示擔心會有昔日在參合陂被殺的燕降卒一樣的下場,所以堅守到最後。拓跋珪想起當日勸他殺俘的正是王建,導致現在難取中山,於是向其吐口水。至五月,拓跋珪撤圍,到河間補充軍糧。在圍攻中山的同時,拓跋珪派庾岳率兵討平國內叛變的賀蘭部、紇鄰部及紇奚部,成功解決內亂。

九月,時據中山的慕容麟因飢荒而出據新市,拓跋珪於是主動進攻,並在次月於義臺大破慕容麟。慕容麟出走後,拓跋珪入據中山。皇始三年(398年),鄴城也因慕容德棄守而落入魏軍手中,拓跋珪於鄴置行臺後回到中山,並打算回盛樂,於是修治由望都至代的直道,設中山行臺以防變亂,又下令強遷新佔之山東六州官民和外族人士到代郡充實人口。

皇始三年(398年)七月,拓跋珪迁都平城,營建宮殿、宗廟、社稷。同年十二月二日(399年1月24日),改年號天興,即皇帝位。

天興二年(399年)正月,拓跋珪即位後不久便北巡,並分三道進攻高車各部,至二月會師時大破高車三十餘部,另拓跋儀又以三萬騎兵攻破高車殘餘的七部,皆大有所獲。同年三月二十日,拓跋珪派遣建義將軍庾真及越騎校尉奚斤進攻北方的庫狄部及宥連部,將他們擊敗並逼令庫狄部的沓亦干歸附。庾真等軍接著又擊破侯莫陳部,俘獲十多萬頭牲畜並一直追擊到大峨谷。

拓跋珪曾派賀狄干向後秦獻馬一千匹並請結婚姻,不過其時拓跋珪已立慕容氏為皇后,故此後秦君主姚興拒絕了婚姻要求並強留賀狄干,兩國遂有嫌隙。天興五年(402年)後秦高平公沒弈干和屬部黜弗及素古延分別遭北魏常山王拓跋遵及材官將軍和突領兵進侵,其中拓跋遵軍更曾追擊至瓦亭,另魏平陽太守貮塵又進攻秦河東之地。這些行動威脅到秦都長安,關中各城白天都閉著城門,亦令得後秦準備進攻北魏。拓跋珪亦在該年舉行閱兵,又命并州各郡送穀物到平陽郡的乾壁儲存以防備秦軍進攻。

天興五年(402年)六月,後秦派軍進攻北魏,攻陷了乾壁。拓跋珪則派毗陵王拓跋順及豫州刺史長孫肥為前鋒迎擊,自率大軍在後。八月,拓跋珪至永安(今山西霍縣東北),秦將姚平派二百精騎視察魏軍但盡數被擒,於是撤走,但在柴壁遭拓跋珪追上,於是據守柴壁。拓跋珪圍困柴壁,而姚興則率軍來救援姚平,並要據天渡運糧給姚平。

拓跋珪接著增厚包圍圈,防止姚平突圍或姚興強攻,另又聽從安同所言,築浮橋渡汾河,並在西岸築圍拒秦軍,引秦軍走汾東的蒙阬。姚興到後果走蒙阬,遭拓跋珪擊敗。拓跋珪又派兵各據險要,阻止秦軍接近柴壁。至十月,姚平糧盡突圍但失敗,於是率部投水自殺,拓跋珪更派擅長游泳的人下水打撈自殺者,又生擒狄伯支等四十多名後秦官員,二萬多名士兵亦束手就擒。姚興雖然能夠與姚平遙相呼應,但無力救援,柴壁敗後多次派人請和,但拓跋珪不准,反而要進攻蒲阪,只是當時姚緒堅守不戰,且早於394年背魏再興的柔然汗國要攻魏,逼使拓跋珪撤兵。

拓跋珪晚年因服食寒食散,剛愎自用、猜忌多疑,更常因想起昔日一點不滿就要誅殺大臣。大臣們大都惶恐度日,影響辦事能力,以至偷竊等行為十分猖獗。

天賜四年(407年)至天賜六年(409年)間,拓跋珪先後誅殺了司空庾岳、北部大人賀狄干兄弟及高邑公莫題父子。往日曾與穆崇共謀刺殺拓跋珪的拓跋儀雖然因拓跋珪念其功勳而沒被追究,但眼見拓跋珪殺害大臣,於是自疑逃亡,但還是被追兵抓住,並被賜死。

天赐六年冬十月戊辰(409年11月6日),次子拓跋紹母賀夫人有过失,拓跋珪幽禁她於宮中,准备处死。到黃昏時仍未決。賀氏秘密向拓跋紹求救。拓跋紹與宮中守兵及宦官串通,當晚带人翻墙入宮,刺殺拓跋珪。拓跋珪在拓跋紹來到時驚醒,試圖找武器反擊但不果,終為其所殺,享年三十九歲。

其子拓跋嗣登位後,於永興二年(410年)諡拓跋珪為宣武皇帝,廟號烈祖,泰常五年(420年)才改諡為道武皇帝,太和十五年(491年)改廟號為太祖。

北齊史官魏收於《魏書》的「史臣曰」評論說:「晉氏崩離,戎羯乘釁,僭偽紛糾,犲狼競馳。太祖顯晦安危之中,屈伸潛躍之際,驅率遺黎,奮其靈武,克剪方難,遂啟中原,朝拱人神,顯登皇極。雖冠履不暇,栖遑外土,而制作經謨,咸存長世。所謂大人利見,百姓與能,抑不世之神武也。而屯厄有期,禍生非慮,將人事不足,豈天實為之。嗚呼!」

唐代某貴族「公子」與世族虞世南的對話:「公子曰:『魏之道武,始立大號,觀其器用,足為一時之杰乎?』先生曰:『道武經略之志,將立霸階,而才不逮也。末年沈痼,加以精虐,不能任下,禍及方悟,不亦晚乎!』;公子曰:『魏之太祖、太武,孰與為輩?』先生曰:『太祖、太武,俱有異人之姿,故能辟土擒敵,窺覦江外。然善戰好殺,暴桀雄武,稟崆峒之氣焉。至於安忍誅殘,石季龍之儔也。』」

北宋司馬光評論說:「後魏之先,世居朔野,有國久矣。道武帝乘燕氏之衰,悉舉引弓之眾,以馮陵中夏;馬首所向,無不望風奔潰。南取并州,東舉幽、冀;兵不留行,而數千里之地定矣!」

\subsubsection{登国}

\begin{longtable}{|>{\centering\scriptsize}m{2em}|>{\centering\scriptsize}m{1.3em}|>{\centering}m{8.8em}|}
  % \caption{秦王政}\
  \toprule
  \SimHei \normalsize 年数 & \SimHei \scriptsize 公元 & \SimHei 大事件 \tabularnewline
  % \midrule
  \endfirsthead
  \toprule
  \SimHei \normalsize 年数 & \SimHei \scriptsize 公元 & \SimHei 大事件 \tabularnewline
  \midrule
  \endhead
  \midrule
  元年 & 386 & \tabularnewline\hline
  二年 & 387 & \tabularnewline\hline
  三年 & 388 & \tabularnewline\hline
  四年 & 389 & \tabularnewline\hline
  五年 & 390 & \tabularnewline\hline
  六年 & 391 & \tabularnewline\hline
  七年 & 392 & \tabularnewline\hline
  八年 & 393 & \tabularnewline\hline
  九年 & 394 & \tabularnewline\hline
  十年 & 395 & \tabularnewline\hline
  十一年 & 396 & \tabularnewline
  \bottomrule
\end{longtable}

\subsubsection{皇始}

\begin{longtable}{|>{\centering\scriptsize}m{2em}|>{\centering\scriptsize}m{1.3em}|>{\centering}m{8.8em}|}
  % \caption{秦王政}\
  \toprule
  \SimHei \normalsize 年数 & \SimHei \scriptsize 公元 & \SimHei 大事件 \tabularnewline
  % \midrule
  \endfirsthead
  \toprule
  \SimHei \normalsize 年数 & \SimHei \scriptsize 公元 & \SimHei 大事件 \tabularnewline
  \midrule
  \endhead
  \midrule
  元年 & 396 & \tabularnewline\hline
  二年 & 397 & \tabularnewline\hline
  三年 & 398 & \tabularnewline
  \bottomrule
\end{longtable}

\subsubsection{天兴}

\begin{longtable}{|>{\centering\scriptsize}m{2em}|>{\centering\scriptsize}m{1.3em}|>{\centering}m{8.8em}|}
  % \caption{秦王政}\
  \toprule
  \SimHei \normalsize 年数 & \SimHei \scriptsize 公元 & \SimHei 大事件 \tabularnewline
  % \midrule
  \endfirsthead
  \toprule
  \SimHei \normalsize 年数 & \SimHei \scriptsize 公元 & \SimHei 大事件 \tabularnewline
  \midrule
  \endhead
  \midrule
  元年 & 398 & \tabularnewline\hline
  二年 & 399 & \tabularnewline\hline
  三年 & 400 & \tabularnewline\hline
  四年 & 401 & \tabularnewline\hline
  五年 & 402 & \tabularnewline\hline
  六年 & 403 & \tabularnewline\hline
  七年 & 404 & \tabularnewline
  \bottomrule
\end{longtable}

\subsubsection{天赐}

\begin{longtable}{|>{\centering\scriptsize}m{2em}|>{\centering\scriptsize}m{1.3em}|>{\centering}m{8.8em}|}
  % \caption{秦王政}\
  \toprule
  \SimHei \normalsize 年数 & \SimHei \scriptsize 公元 & \SimHei 大事件 \tabularnewline
  % \midrule
  \endfirsthead
  \toprule
  \SimHei \normalsize 年数 & \SimHei \scriptsize 公元 & \SimHei 大事件 \tabularnewline
  \midrule
  \endhead
  \midrule
  元年 & 404 & \tabularnewline\hline
  二年 & 405 & \tabularnewline\hline
  三年 & 406 & \tabularnewline\hline
  四年 & 407 & \tabularnewline\hline
  五年 & 408 & \tabularnewline\hline
  六年 & 409 & \tabularnewline
  \bottomrule
\end{longtable}


%%% Local Variables:
%%% mode: latex
%%% TeX-engine: xetex
%%% TeX-master: "../../Main"
%%% End:

%% -*- coding: utf-8 -*-
%% Time-stamp: <Chen Wang: 2019-12-23 14:52:34>

\subsection{明元帝\tiny(409-423)}

\subsubsection{生平}

魏明元帝拓跋嗣(392年-423年12月24日),鮮卑名木末,北魏第二位皇帝,409年—423年在位。

北魏道武帝拓跋珪長子,登國七年(392年)生于雲中宫,天興六年(403年),封齊王,拜相國,加車騎大將軍。天赐六年十月,拓跋嗣被其父北魏道武帝拓跋珪立為太子,生母劉貴人按子贵母死的制度被道武帝赐死,拓跋嗣知道后悲傷不已,因而被其父北魏道武帝拓跋珪怒斥出宫。

十月十三日(409年11月6日),道武帝拓跋珪被其次子清河王拓跋绍所杀,太子拓跋嗣在宫中衛士的擁戴下殺了拓跋紹,拓跋嗣在同年十月十七日(11月10日)登基,改年号“永興”,為北魏明元帝。

泰常八年(423年),明元帝拓跋嗣進攻刘宋得勝回来,此役稱為南北朝第一次南北戰爭,北魏獲得勝利,攻佔虎牢關,奪取劉宋領土三百里。十一月己巳(12月24日),明元帝因攻戰勞顿成疾而终,享年32岁。

拓跋嗣雖英年早逝,但上承其父北魏開國君主太祖道武帝拓拔珪的武功,後有兒子北魏太武帝拓拔燾滅北方諸國一統北方,媲美五胡十六國初期前秦苻堅統一北方功勳。因此拓拔嗣在北魏開國歷史中具有承先啟後的重要地位。

拓跋嗣谥号为明元皇帝,庙号太宗。

\subsubsection{永兴}

\begin{longtable}{|>{\centering\scriptsize}m{2em}|>{\centering\scriptsize}m{1.3em}|>{\centering}m{8.8em}|}
  % \caption{秦王政}\
  \toprule
  \SimHei \normalsize 年数 & \SimHei \scriptsize 公元 & \SimHei 大事件 \tabularnewline
  % \midrule
  \endfirsthead
  \toprule
  \SimHei \normalsize 年数 & \SimHei \scriptsize 公元 & \SimHei 大事件 \tabularnewline
  \midrule
  \endhead
  \midrule
  元年 & 409 & \tabularnewline\hline
  二年 & 410 & \tabularnewline\hline
  三年 & 411 & \tabularnewline\hline
  四年 & 412 & \tabularnewline\hline
  五年 & 413 & \tabularnewline
  \bottomrule
\end{longtable}

\subsubsection{神瑞}

\begin{longtable}{|>{\centering\scriptsize}m{2em}|>{\centering\scriptsize}m{1.3em}|>{\centering}m{8.8em}|}
  % \caption{秦王政}\
  \toprule
  \SimHei \normalsize 年数 & \SimHei \scriptsize 公元 & \SimHei 大事件 \tabularnewline
  % \midrule
  \endfirsthead
  \toprule
  \SimHei \normalsize 年数 & \SimHei \scriptsize 公元 & \SimHei 大事件 \tabularnewline
  \midrule
  \endhead
  \midrule
  元年 & 414 & \tabularnewline\hline
  二年 & 415 & \tabularnewline\hline
  三年 & 416 & \tabularnewline
  \bottomrule
\end{longtable}

\subsubsection{泰常}

\begin{longtable}{|>{\centering\scriptsize}m{2em}|>{\centering\scriptsize}m{1.3em}|>{\centering}m{8.8em}|}
  % \caption{秦王政}\
  \toprule
  \SimHei \normalsize 年数 & \SimHei \scriptsize 公元 & \SimHei 大事件 \tabularnewline
  % \midrule
  \endfirsthead
  \toprule
  \SimHei \normalsize 年数 & \SimHei \scriptsize 公元 & \SimHei 大事件 \tabularnewline
  \midrule
  \endhead
  \midrule
  元年 & 416 & \tabularnewline\hline
  二年 & 417 & \tabularnewline\hline
  三年 & 418 & \tabularnewline\hline
  四年 & 419 & \tabularnewline\hline
  五年 & 420 & \tabularnewline\hline
  六年 & 421 & \tabularnewline\hline
  七年 & 422 & \tabularnewline\hline
  八年 & 423 & \tabularnewline
  \bottomrule
\end{longtable}


%%% Local Variables:
%%% mode: latex
%%% TeX-engine: xetex
%%% TeX-master: "../../Main"
%%% End:

%% -*- coding: utf-8 -*-
%% Time-stamp: <Chen Wang: 2019-12-23 15:01:57>

\subsection{太武帝\tiny(423-452)}

\subsubsection{生平}

魏太武帝拓跋焘(408年-452年3月11日),鮮卑本名佛狸伐。佛狸是官號,突厥語狼büri或böri的音譯。伐或bäg是官稱,且是魏晉時期鮮卑諸部使用最為廣泛的政治名號。北魏第三位皇帝(423年12月27日—452年3月11日在位),北魏明元帝拓跋嗣的长子,北魏道武帝拓拔珪的長孫,在位28年,谥号太武皇帝。

拓拔焘即位时,只有十六岁,大臣们都拿他当小孩子看。于是拓拔焘决定先整顿吏治,励精图治,令人刮目相看,北魏国力进入鼎盛。427年,拓跋燾在連續兩年突擊統萬城之後,占領胡夏的北部地區(包含首都統萬),並一度攻下關中,胡夏雖遷都至平涼,卻於次年(428年)打敗魏軍並收復關中。北魏在429年北伐柔然大獲全勝之後,趁著柔然近十年都難以恢復的良機,把軍隊主力向南進攻,於430年大敗劉宋與胡夏的聯合攻勢,不但占領胡夏大部分的關隴領土(包含平涼、關中、隴西郡),更在431年從宋軍手中奪回河南四鎮(洛陽、虎牢等),拓拔焘返回首都平城,祭告太廟並舉行盛大的慶功典禮。

撤退到上邽的夏主赫連定,雖於431年滅西秦而稍微挽救了國勢,並意圖再滅北涼以維持胡夏,但卻在432年,被吐谷渾君主慕容慕璝襲擊而俘虜。同年赫連定被送給北魏,拓拔焘將其處死,胡夏亡。436年拓跋燾派軍東征北燕,燕主馮弘在高句麗大軍的保護之下,將首都人民全部東遷高句麗,而魏軍主帥忌憚高軍,坐看燕人東撤;北燕雖然滅亡,但只得空地空城,因此拓跋燾大怒之下處罰了征燕主帥娥清、古弼。439年拓拔焘率大軍圍攻北涼首都姑臧,涼主沮渠牧犍出降,北涼亡。至此,北魏統一華北,与江东的刘宋王朝对峙,形成南北朝的局面。

自前涼张氏以来,河西地方文化学术比较发达,号称多士。北魏自道武帝以后,政治上使用汉族高门,汲取不少魏晋典制。431年,藉由同年打敗劉宋的威勢,拓拔焘下詔,徵聘關東地區的數百名士(多為領導地方的世家大族)入朝為官,也就是把山東郡姓如范陽盧氏、博陵崔氏、趙郡李氏等勢力一網打盡,強迫他們到平城擔任無薪水的官職,讓漢人世族的勢力與北魏政權相結合。當時被徵召的名士高允,後來寫了一篇文章〈徵士頌〉來追憶、讚揚此盛事。439年北魏吞并河西后,又有大批河西文士进入北魏统治区域,不少人被徵召到平城去做官,受到重用,北魏的儒学才开始兴盛。

之後,拓拔焘又击溃吐谷渾、柔然,扩地千餘里。他一共七次率军进攻柔然,太平真君十年(449)大败柔然,收民畜凡百余万,柔然可汗远遁,北方边塞再度得到安静。

他在450-451年对宋的战争中,雖然大勝,但人馬死傷近半,又使軍民疲憊,怨聲不已。末期又刑罰殘酷,使国内政治混乱。譬如崔浩修国史详实记载魏先世事迹,可能涉及某些鲜卑习俗和隐私,有伤体面,拓跋焘不惜发动國史之獄,将三朝功臣司徒崔浩处死,连清河崔氏与浩同宗者以及崔浩姻亲范阳卢氏、太原郭氏、河东柳氏都遭族灭。事后拓跋焘说 “崔司徒可惜”,有后悔之意;再如監國執政的太子,也在父子權力衝突下,被宦官宗愛的讒言害死。正平二年二月甲寅(452年3月11日)拓跋焘被宗爱杀死,享年四十五歲,谥号太武帝,庙号世祖。

拓跋焘统治时期,氐、羌、屠各,以及所谓“杂虏”、“杂人”的各族暴亂非常频繁。太平真君六年(445年)卢水胡盖吴在关中杏城(今陕西黄陵西南)发动的起义,声势最为浩大。盖吴建号秦地王,有众十余万,得到安定卢水胡刘超、河东蜀薛永宗的响应,拓跋焘调动强大的兵力才镇压下去。

拓跋焘受崔浩、寇谦之影响,奉道排佛。镇压盖吴过程中,在长安佛寺中发现大量兵器,认为佛寺与盖吴通谋,太平真君七年(446年),詔:「諸有佛圖、形像及胡經,盡皆擊破焚燒,沙門無少長悉坑之。」,是為北魏太武帝滅佛,三武滅佛之一(另外兩位是北周武帝和唐武宗)。

拓跋焘天生將才,为人勇健,善于指挥。战阵亲犯矢石,神色自若,命将出师,违其节度者多败,因此将士畏服,为之盡力。有知人之明,常从士伍中选拔人才。赏不遗贱,罚不避贵,虽所爱之人亦不宽假。他放棄父親拓跋嗣築邊城防禦柔然的政策,主動攻擊柔然並獲得成功。他自奉俭朴,而赏赐功臣绝无吝嗇,幾乎把資源都用在主動出擊的軍功賞賜之上。认为元老功臣勤劳日久,应让他们以爵归第,随时朝见饷宴,百官职务则可另简贤能。这样就保证了行政效率,使政治多少能健全发展。他倚重汉人,李顺、崔浩、李孝伯等先后掌握朝权,但個性果於殺戮,處死大臣後常懊悔自己太快動刀。

北齊史官魏收於《魏書》的「史臣曰」評論說:「世祖聰明雄斷,威靈傑立,藉二世之資,奮征伐之氣,遂戎軒四出,周旋險夷。掃統萬,平秦隴,翦遼海,盪河源,南夷荷擔,北蠕削跡,廓定四表,混一戎華,其為功也大矣。遂使有魏之業,光邁百王,豈非神叡經綸,事當命世。至於初則東儲不終,末乃釁成所忽。固本貽防,殆弗思乎?」

唐代某貴族「公子」與世族虞世南的對話:「公子曰:『魏之太祖、太武,孰與為輩?』先生曰:『太祖、太武,俱有異人之姿,故能辟土擒敵,窺覦江外。然善戰好殺,暴桀雄武,稟崆峒之氣焉。至於安忍誅殘,石季龍之儔也。』」

北宋司馬光評論說:「(北魏)繼以明元、太武,兼有青、兗,包司、豫,摧赫連,開關中,梟馮弘,吞遼碣,擄沮渠,并河右,高車入臣,蠕蠕遠遁;自河以北,逾於大漠,悉為其有;子孫稱帝者,百有餘年。左袵之盛,未之有也。」

资治通鉴记载: 魏主(指太武帝)為人,壯健鷙勇,臨城對陣,親犯矢石,左右死傷相繼,神色自若;由是將士畏服,咸盡死力。性儉率,服御飲膳,取給而已。群臣請增峻京城及修宮室曰: 「《易》云:『王公設險,以守其國。』又蕭何云:『天子以四海為家,不壯不麗,無以重威。』」帝曰:「古人有言:『在德不在險。』屈丐蒸土築城而朕滅之。 豈在城也?今天下未平,方須民力,土功之事,朕所未為。蕭何之對,非雅言也。」每以為財者軍國之本,不可輕費。至於賞賜,皆死事勳績之家,親戚貴寵未嘗橫有所及。命將出師,指授節度,違之者多致負敗。明於知人,或拔干於卒伍之中,唯其才用所長,不論本末。聽察精敏,下無遁情,賞不遺賤,罰不避貴,雖所甚愛之人,終無寬假。常曰:「法者,朕與天下共之,何敢輕也。」然性殘忍,果於殺戮,往往已殺而復悔之。

太平真君四年(443年)拓拔焘遣大臣李敞所刻的石刻祝文,存於嘎仙洞内的石壁上。1980年7月30日,中国考古学家米文平等人在此洞发现石刻祝文,结合当时在洞内发现的陶器碎片等,认定此处即为史书中记载的北魏祖庭。但该洞是否确实就是拓跋鲜卑的发源地,史学界尚有争论。

江蘇省南京市六合區东南有瓜步山,山上有佛狸祠。

《魏书·世祖纪下》记载:北魏太武帝拓跋焘于宋元嘉二十七年击败王玄谟的军队以后,在山上建立行宫,即后来的「佛狸祠」。

南宋诗人辛弃疾有《永遇乐·京口北固亭怀古》:「可堪回首,佛狸祠下,一片神鸦社鼓」。后又有《水调歌头·舟次扬州和杨济翁周显先韵》:「谁道投鞭飞渡,忆昔鸣髇血污,风雨佛狸愁。」

太延元年(435年)十月,太武帝东巡冀州、定州,二十日甲辰到定州,驻驾于新城宫。十一月十六日己巳,在广川(河北景县)校猎。二十三日丙子到达邺城(河北临漳),祭祀密太后(太武帝母杜氏)庙,并慰问老年族人,褒礼贤俊。十二月二十日癸卯派遣使者到北岳恒山祭祀。次年正月初二甲寅从五回道返回平城。

在东巡至河北易县南管头之南画猫村古徐水河谷时,见山岩险峭,景观奇丽,太武帝即兴演示射术,又命左右将士善射者进行射箭比试。镇东将军、定州刺史、乐浪公乞伏某请求立碑纪念。到太延三年丁丑(437年)碑刻完工,乐浪公已去职,新任刺史征东将军、张掖公秃发保周)接手此事。

东巡碑碑额题【皇帝东巡之碑】,史籍最早提到北魏太武帝东巡碑,是郦道元《水经注》。郦书之后,宋代乐史《太平寰宇记》卷六七易州满城县条,也曾提及此碑,称引的内容有溢出郦书者。此后东巡碑湮没无闻将近千年,直到1935年,由徐森玉(鸿宝)先生在河北易县觅得原碑,把20份拓本带回北平,次年傅增湘、周肇祥也前往摹拓,东巡碑才重新现身,为艺林所重。今碑已破碎,仅剩残片若干块。

\subsubsection{始光}

\begin{longtable}{|>{\centering\scriptsize}m{2em}|>{\centering\scriptsize}m{1.3em}|>{\centering}m{8.8em}|}
  % \caption{秦王政}\
  \toprule
  \SimHei \normalsize 年数 & \SimHei \scriptsize 公元 & \SimHei 大事件 \tabularnewline
  % \midrule
  \endfirsthead
  \toprule
  \SimHei \normalsize 年数 & \SimHei \scriptsize 公元 & \SimHei 大事件 \tabularnewline
  \midrule
  \endhead
  \midrule
  元年 & 424 & \tabularnewline\hline
  二年 & 425 & \tabularnewline\hline
  三年 & 426 & \tabularnewline\hline
  四年 & 427 & \tabularnewline\hline
  五年 & 428 & \tabularnewline
  \bottomrule
\end{longtable}

\subsubsection{神䴥}

\begin{longtable}{|>{\centering\scriptsize}m{2em}|>{\centering\scriptsize}m{1.3em}|>{\centering}m{8.8em}|}
  % \caption{秦王政}\
  \toprule
  \SimHei \normalsize 年数 & \SimHei \scriptsize 公元 & \SimHei 大事件 \tabularnewline
  % \midrule
  \endfirsthead
  \toprule
  \SimHei \normalsize 年数 & \SimHei \scriptsize 公元 & \SimHei 大事件 \tabularnewline
  \midrule
  \endhead
  \midrule
  元年 & 428 & \tabularnewline\hline
  二年 & 429 & \tabularnewline\hline
  三年 & 430 & \tabularnewline\hline
  四年 & 431 & \tabularnewline
  \bottomrule
\end{longtable}

\subsubsection{延和}

\begin{longtable}{|>{\centering\scriptsize}m{2em}|>{\centering\scriptsize}m{1.3em}|>{\centering}m{8.8em}|}
  % \caption{秦王政}\
  \toprule
  \SimHei \normalsize 年数 & \SimHei \scriptsize 公元 & \SimHei 大事件 \tabularnewline
  % \midrule
  \endfirsthead
  \toprule
  \SimHei \normalsize 年数 & \SimHei \scriptsize 公元 & \SimHei 大事件 \tabularnewline
  \midrule
  \endhead
  \midrule
  元年 & 432 & \tabularnewline\hline
  二年 & 433 & \tabularnewline\hline
  三年 & 434 & \tabularnewline\hline
  四年 & 435 & \tabularnewline
  \bottomrule
\end{longtable}

\subsubsection{太延}

\begin{longtable}{|>{\centering\scriptsize}m{2em}|>{\centering\scriptsize}m{1.3em}|>{\centering}m{8.8em}|}
  % \caption{秦王政}\
  \toprule
  \SimHei \normalsize 年数 & \SimHei \scriptsize 公元 & \SimHei 大事件 \tabularnewline
  % \midrule
  \endfirsthead
  \toprule
  \SimHei \normalsize 年数 & \SimHei \scriptsize 公元 & \SimHei 大事件 \tabularnewline
  \midrule
  \endhead
  \midrule
  元年 & 435 & \tabularnewline\hline
  二年 & 436 & \tabularnewline\hline
  三年 & 437 & \tabularnewline\hline
  四年 & 438 & \tabularnewline\hline
  五年 & 439 & \tabularnewline\hline
  六年 & 440 & \tabularnewline
  \bottomrule
\end{longtable}

\subsubsection{太平真君}

\begin{longtable}{|>{\centering\scriptsize}m{2em}|>{\centering\scriptsize}m{1.3em}|>{\centering}m{8.8em}|}
  % \caption{秦王政}\
  \toprule
  \SimHei \normalsize 年数 & \SimHei \scriptsize 公元 & \SimHei 大事件 \tabularnewline
  % \midrule
  \endfirsthead
  \toprule
  \SimHei \normalsize 年数 & \SimHei \scriptsize 公元 & \SimHei 大事件 \tabularnewline
  \midrule
  \endhead
  \midrule
  元年 & 440 & \tabularnewline\hline
  二年 & 441 & \tabularnewline\hline
  三年 & 442 & \tabularnewline\hline
  四年 & 443 & \tabularnewline\hline
  五年 & 444 & \tabularnewline\hline
  六年 & 445 & \tabularnewline\hline
  七年 & 446 & \tabularnewline\hline
  八年 & 447 & \tabularnewline\hline
  九年 & 448 & \tabularnewline\hline
  十年 & 449 & \tabularnewline\hline
  十一年 & 450 & \tabularnewline\hline
  十二年 & 451 & \tabularnewline
  \bottomrule
\end{longtable}

\subsubsection{正平}

\begin{longtable}{|>{\centering\scriptsize}m{2em}|>{\centering\scriptsize}m{1.3em}|>{\centering}m{8.8em}|}
  % \caption{秦王政}\
  \toprule
  \SimHei \normalsize 年数 & \SimHei \scriptsize 公元 & \SimHei 大事件 \tabularnewline
  % \midrule
  \endfirsthead
  \toprule
  \SimHei \normalsize 年数 & \SimHei \scriptsize 公元 & \SimHei 大事件 \tabularnewline
  \midrule
  \endhead
  \midrule
  元年 & 451 & \tabularnewline\hline
  二年 & 452 & \tabularnewline
  \bottomrule
\end{longtable}


%%% Local Variables:
%%% mode: latex
%%% TeX-engine: xetex
%%% TeX-master: "../../Main"
%%% End:

%% -*- coding: utf-8 -*-
%% Time-stamp: <Chen Wang: 2021-11-01 15:10:07>

\subsection{南安隱王拓跋余\tiny(452)}

\subsubsection{生平}

拓跋余(5世纪-452年10月29日),鮮卑名可博真,北魏太武帝拓跋燾之子,生母闾左昭仪。北魏皇帝,為中常侍宗愛所立,但同年又被其殺害,所在位232天。

拓跋余生年不详。太平真君三年(442年),拓跋余獲封為吳王。正平元年(451年)九月,太武帝南征,以拓跋余留守平城。十二月改封南安王。

正平二年二月五日(452年3月11日),中常侍宗愛弒太武帝,尚书左仆射兰延、侍中吴兴公和疋、侍中太原公薛提等秘不发丧。兰延、和疋认为太武帝嫡孙拓跋濬冲幼,欲立太武帝在世长子,于是召东平王拓跋翰,置于秘室。薛提则坚持立拓跋濬,兰延等犹豫未决。宗爱得知。宗爱曾得罪拓跋晃,素来厌恶拓跋翰,却和拓跋余关系好,于是秘密迎拓跋余从中宫便门入宫,矫皇后令征召兰延等,斩于殿堂,再杀死拓跋翰,立拓跋余为帝,为太武帝发丧,改元承平(或作永平),大赦。

拓跋余即位後,因自己不是作为先帝长子继位,即厚待群下以取悅眾人,以宗爱为大司马、大将军、太师、都督中外诸军事,领中秘书,封冯翊王,以兄长原太子拓跋晃辅臣尚书令古弼为司徒,兼太尉张黎为太尉,羽林中郎、幢将乌程子建威将军吕罗汉典宿卫。但他也徹夜暢飲,夜夜笙歌,很快即令國庫空虛,又多次出獵,即使邊境有事,亦不加体恤,百姓皆憤怒,而他不作改變。南朝宋文帝刘义隆遣将檀和之侵犯济州,拓跋余令侍中、尚书左仆射、征南将军韩茂讨之,檀和之遁走。

另外,宗愛自拓跋余登位後掌權日久,朝野內外皆忌憚他,而拓跋余則懷疑宗愛另有所圖,密謀削奪宗愛權力,宗愛於是于十月一日(10月29日)使小黄门贾周等乘夜趁拓跋余祭祀东庙而杀之(《宋书》作与宗爱同被兄拓跋谭所杀,疑误),隐秘其事,只有羽林中郎刘尼知道。

拓跋余被杀后,百官不知道立谁为新君,刘尼、吕罗汉等迎立拓跋濬为北魏文成帝,诛杀宗爱。文成帝以王禮安葬拓跋余,諡為隱王。

先前太平真君七年(439年)八月,月犯荧惑;八月至十一月,又犯轩辕。是岁正月,太白经天。九月火犯太微。这些星象被认为是从拓跋晃去世到拓跋余被杀一系列事的预兆。

\subsubsection{承平}

\begin{longtable}{|>{\centering\scriptsize}m{2em}|>{\centering\scriptsize}m{1.3em}|>{\centering}m{8.8em}|}
  % \caption{秦王政}\
  \toprule
  \SimHei \normalsize 年数 & \SimHei \scriptsize 公元 & \SimHei 大事件 \tabularnewline
  % \midrule
  \endfirsthead
  \toprule
  \SimHei \normalsize 年数 & \SimHei \scriptsize 公元 & \SimHei 大事件 \tabularnewline
  \midrule
  \endhead
  \midrule
  元年 & 452 & \tabularnewline\hline
  \bottomrule
\end{longtable}


%%% Local Variables:
%%% mode: latex
%%% TeX-engine: xetex
%%% TeX-master: "../../Main"
%%% End:

%% -*- coding: utf-8 -*-
%% Time-stamp: <Chen Wang: 2019-12-23 15:13:21>

\subsection{文成帝\tiny(452-465)}

\subsubsection{生平}

魏文成帝拓跋濬(440年-465年),鮮卑名烏雷,拥有直懃头衔,南北朝時期北魏的第四代皇帝。是魏太武帝的嫡孫,景穆太子拓跋晃長子。452年-465年在位,在位十三年。

正平元年(451年),太武帝北征,任太子拓跋晃為監國,但宦官中常侍宗愛對太子多加干預,又與太子部屬給事仇尼道盛和侍郎任城互有閒隙,宗愛怕日後太子登基對己不利,於是與東宮勢力展開權力鬥爭,仗着拓跋燾的信任誣陷太子及其手下人意圖造反,不曾想皇帝拓跋燾竟然相信了。继而下令整肅太子府,且誅殺了許多太子近臣。太子拓跋晃因此積憂成疾,一病而死,時年才24歲。

後來魏太武帝拓跋燾知道太子是清白的,非常懊悔。但宗愛一見到此,怕被皇帝誅殺,先下手為強於正平二年(452年)三月弒太武帝。太武帝死後,朝廷欲立太武帝最年长的在世儿子第三子東平王拓跋翰為帝,但宗愛與拓跋翰關係不好,因此假立太武皇后之命,將拓跋翰殺掉,又假借皇后之命,將擁立東平王的大臣尚書僕射蘭延、侍中吳興公和疋及侍中太原公薛提殺死,然後立太武帝幼子南安王拓跋余為帝,宗愛自為大司馬、大將軍、太師,總督中外軍事、領中祕書,封馮翊王,大權在握。拓跋余想奪回皇權,又于十月一日遭宗愛所弒。短短數月,宗愛連殺兩位皇帝,引起朝野震動。

羽林郎中刘尼、太子少傅游雅、殿中尚书源贺、尚书陆丽、尚书长孙渴侯五人密谋,十月三日(10月31日),由太子少傅游雅、源贺、长孙渴侯率禁军守卫宫廷,陆丽与刘尼一起迎皇孙拓跋濬入宫即位。拓跋濬改元兴安。

拓跋濬即位后,便诛杀了宗爱、贾周等人,都动用五刑,灭三族。兴安元年(452年)十一月初九,文成帝追谥父亲拓跋晃为景穆皇帝,母亲闾氏为恭皇后,尊乳母常氏为保太后。

太武帝崇信道教,一度太武滅佛。兴安元年,拓跋濬下令复兴佛教。兴安二年,令建造云冈石窟。

拓跋濬不再继续太武帝四处用兵的政策,停止南侵南朝宋,休养生息。但也有征伐。太安四年(458年),拓跋濬亲率10万骑兵、15万辆战车,进攻柔然。柔然处罗可汗吐贺真远远逃走。柔然别部统帅乌朱驾颓等人率领几千个帳幕所聚的部落投降。

和平六年五月十一日(465年6月21日),拓跋濬去世,时年僅25岁。六月初二,定谥号为文成皇帝,庙号高宗。八月,安葬云中的金陵。

\subsubsection{兴安}

\begin{longtable}{|>{\centering\scriptsize}m{2em}|>{\centering\scriptsize}m{1.3em}|>{\centering}m{8.8em}|}
  % \caption{秦王政}\
  \toprule
  \SimHei \normalsize 年数 & \SimHei \scriptsize 公元 & \SimHei 大事件 \tabularnewline
  % \midrule
  \endfirsthead
  \toprule
  \SimHei \normalsize 年数 & \SimHei \scriptsize 公元 & \SimHei 大事件 \tabularnewline
  \midrule
  \endhead
  \midrule
  元年 & 452 & \tabularnewline\hline
  二年 & 453 & \tabularnewline\hline
  三年 & 454 & \tabularnewline
  \bottomrule
\end{longtable}

\subsubsection{兴光}

\begin{longtable}{|>{\centering\scriptsize}m{2em}|>{\centering\scriptsize}m{1.3em}|>{\centering}m{8.8em}|}
  % \caption{秦王政}\
  \toprule
  \SimHei \normalsize 年数 & \SimHei \scriptsize 公元 & \SimHei 大事件 \tabularnewline
  % \midrule
  \endfirsthead
  \toprule
  \SimHei \normalsize 年数 & \SimHei \scriptsize 公元 & \SimHei 大事件 \tabularnewline
  \midrule
  \endhead
  \midrule
  元年 & 454 & \tabularnewline\hline
  二年 & 455 & \tabularnewline
  \bottomrule
\end{longtable}

\subsubsection{太安}

\begin{longtable}{|>{\centering\scriptsize}m{2em}|>{\centering\scriptsize}m{1.3em}|>{\centering}m{8.8em}|}
  % \caption{秦王政}\
  \toprule
  \SimHei \normalsize 年数 & \SimHei \scriptsize 公元 & \SimHei 大事件 \tabularnewline
  % \midrule
  \endfirsthead
  \toprule
  \SimHei \normalsize 年数 & \SimHei \scriptsize 公元 & \SimHei 大事件 \tabularnewline
  \midrule
  \endhead
  \midrule
  元年 & 455 & \tabularnewline\hline
  二年 & 456 & \tabularnewline\hline
  三年 & 457 & \tabularnewline\hline
  四年 & 458 & \tabularnewline\hline
  五年 & 459 & \tabularnewline
  \bottomrule
\end{longtable}

\subsubsection{和平}

\begin{longtable}{|>{\centering\scriptsize}m{2em}|>{\centering\scriptsize}m{1.3em}|>{\centering}m{8.8em}|}
  % \caption{秦王政}\
  \toprule
  \SimHei \normalsize 年数 & \SimHei \scriptsize 公元 & \SimHei 大事件 \tabularnewline
  % \midrule
  \endfirsthead
  \toprule
  \SimHei \normalsize 年数 & \SimHei \scriptsize 公元 & \SimHei 大事件 \tabularnewline
  \midrule
  \endhead
  \midrule
  元年 & 460 & \tabularnewline\hline
  二年 & 461 & \tabularnewline\hline
  三年 & 462 & \tabularnewline\hline
  四年 & 463 & \tabularnewline\hline
  五年 & 464 & \tabularnewline\hline
  六年 & 465 & \tabularnewline
  \bottomrule
\end{longtable}


%%% Local Variables:
%%% mode: latex
%%% TeX-engine: xetex
%%% TeX-master: "../../Main"
%%% End:

%% -*- coding: utf-8 -*-
%% Time-stamp: <Chen Wang: 2019-12-23 15:13:53>

\subsection{献文帝\tiny(465-471)}

\subsubsection{生平}

魏獻文帝拓跋弘(454年-476年),鮮卑名第豆胤,魏文成帝拓跋濬長子,南北朝時期北魏第六位皇帝。

太安二年(456年)正月,立为太子。

和平六年(465年)五月,父亲拓跋濬逝世,随后,拓跋弘登基为帝。

皇興元年(467年)八月,獻文帝之妃李夫人誕下拓跋宏。   

皇興三年(469年)六月,李夫人被按照子貴母死的制度殺死。葬在金陵,承明元年(476年)追諡思皇后,配饗太廟。

太和二年(478年)十二月,李夫人全家都被馮太后處死。

皇興五年(471年),年僅17歲的魏献文帝因不滿馮太后長期攝政及專權,原本要禪讓給三叔京兆王拓跋子推,被眾臣勸阻後作罷,禪讓予太子拓跋宏,且因群臣奏称“三皇澹泊无为,所以称皇;西汉高祖之父被尊为太上皇,是不统治天下的,而皇帝年幼,陛下仍然执政”,不称太上皇,而称尊号太上皇帝。雖然作為太上皇,但魏献文帝仍掌握有部分的皇帝權力,並欲與馮太后爭權。

延興二年(472年),柔然來犯,魏献文帝以太上皇之姿,御駕親征,大敗柔然,並追至大漠。

承明元年(476年),拓跋弘被馮太后毒死,崩於永安殿,年僅二十三歲,諡曰獻文皇帝,廟號顯祖,葬於金陵。

\subsubsection{天安}

\begin{longtable}{|>{\centering\scriptsize}m{2em}|>{\centering\scriptsize}m{1.3em}|>{\centering}m{8.8em}|}
  % \caption{秦王政}\
  \toprule
  \SimHei \normalsize 年数 & \SimHei \scriptsize 公元 & \SimHei 大事件 \tabularnewline
  % \midrule
  \endfirsthead
  \toprule
  \SimHei \normalsize 年数 & \SimHei \scriptsize 公元 & \SimHei 大事件 \tabularnewline
  \midrule
  \endhead
  \midrule
  元年 & 466 & \tabularnewline\hline
  二年 & 467 & \tabularnewline
  \bottomrule
\end{longtable}

\subsubsection{皇兴}

\begin{longtable}{|>{\centering\scriptsize}m{2em}|>{\centering\scriptsize}m{1.3em}|>{\centering}m{8.8em}|}
  % \caption{秦王政}\
  \toprule
  \SimHei \normalsize 年数 & \SimHei \scriptsize 公元 & \SimHei 大事件 \tabularnewline
  % \midrule
  \endfirsthead
  \toprule
  \SimHei \normalsize 年数 & \SimHei \scriptsize 公元 & \SimHei 大事件 \tabularnewline
  \midrule
  \endhead
  \midrule
  元年 & 467 & \tabularnewline\hline
  二年 & 468 & \tabularnewline\hline
  三年 & 469 & \tabularnewline\hline
  四年 & 470 & \tabularnewline\hline
  五年 & 471 & \tabularnewline
  \bottomrule
\end{longtable}


%%% Local Variables:
%%% mode: latex
%%% TeX-engine: xetex
%%% TeX-master: "../../Main"
%%% End:

%% -*- coding: utf-8 -*-
%% Time-stamp: <Chen Wang: 2021-11-01 15:10:55>

\subsection{孝文帝元宏\tiny(471-499)}

\subsubsection{生平}

魏孝文帝元宏(467年10月13日-499年4月26日),本姓拓跋,是北魏献文帝拓跋弘的长子,北魏第七位皇帝(471年9月20日—499年4月26日在位),後改姓「元」,在位28年,卒年33岁,其所推行的孝文帝改革,以漢化運動為主體,俗稱孝文漢化,其改革措施有利于缓解民族隔阂和阶级矛盾,为社会经济的恢复和发展发挥积极作用。雖然因推動漢化急進而最終導致六鎮起義及北魏解體,卻為北朝的胡漢融合作出貢獻。

孝文帝去世后,庙号高祖,谥号孝文皇帝。

北魏献文帝皇兴元年八月二十九日(467年10月13日),元宏生于北魏首都平城(今山西省大同市)紫宫,生母李夫人。

皇兴五年八月二十日(471年9月20日),父親獻文帝禪讓于太华前殿,大赦,改元延兴元年。

元宏即位时只有5岁,獻文帝死後,由祖母冯太后攝政。冯太后是汉人,对鲜卑人建立的北魏朝廷进行了一系列中央集权化的改革,孝文帝便受此影响。

太和十四年(490年)冯太后去世后亲政,秉承馮太后的政策,繼續进行了汉化改革,而且做得比馮太后更大刀闊斧。他先整顿吏治,立三长法,实行均田制;太和十八年(494年),他以“南伐”为名迁都洛阳,全面改革鲜卑旧俗:规定以汉服代替鲜卑服,以汉语代替鲜卑语,迁洛鲜卑人以洛阳为籍贯,改鲜卑姓为汉姓,自己也改姓“元”。并鼓励鲜卑贵族与汉士族联姻,又参照南朝典章,修改北魏政治制度,并严厉镇压反对改革的守旧贵族,处死太子恂,这一举动使鲜卑经济、文化、社会、政治、军事等方面大大的发展,缓解了民族隔阂,史称“孝文帝改革”。

太和二十三年三月丙戌(499年4月6日),魏孝文帝在南征途中生病。三月庚子(499年4月20日),魏孝文帝病重。四月初一日(499年4月26日),孝文帝崩于谷塘原之行宫。孝文帝去世后,庙号高祖,谥号孝文皇帝。

陵墓位於河南省洛陽市孟津縣官庄村村東南800米處的兩個大型土丘,兩塚相距约100米,大塚是魏孝文帝陵墓「長陵」,小塚是第三位皇后文昭皇后(即宣武帝生母)的「終寧陵」。

然而孝文帝去世以后仅25年,北魏边镇鲜卑军事集团就发动反汉化运动六镇起义。534年,北魏分裂成东魏、西魏,之后更分别被北齐和北周取代。

北齐官修正史《魏书》魏收的评价是:“史臣曰:有魏始基代朔,廓平南夏,辟壤经世,咸以威武为业,文教之事,所未遑也。高祖幼承洪绪,早著睿圣之风。时以文明摄事,优游恭己,玄览独得,著自不言,神契所标,固以符于冥化。及躬总大政,一日万机,十许年间,曾不暇给;殊途同归,百虑一致。至夫生民所难行,人伦之高迹,虽尊居黄屋,尽蹈之矣。若乃钦明稽古,协御天人,帝王制作,朝野轨度,斟酌用舍,焕乎其有文章,海内生民咸受耳目之赐。加以雄才大略,爱奇好士,视下如伤,役己利物,亦无得而称之。其经纬天地,岂虚谥也!”

《資治通鑑》曰:高祖友愛諸弟,始終無間。嘗從容謂咸陽王禧等曰:「我後子孫解逅不肖,汝等觀望,可輔則輔之,不可輔則取之,勿為它人有也。」親任賢能,從善如流,精勤庶務,朝夕不倦。常曰:「人主患不能處心公平,推誠於物。能是二者,則胡、越之人皆可使如兄弟矣。」用法雖嚴,於大臣無所容貸,然人有小過,常多闊略。嘗於食中得蟲,又左右進羹誤傷帝手,皆笑而赦之。天地五郊、宗廟二分之祭,未嘗不身親其禮。每出巡遊及用兵,有司奏修道路,帝輒曰:「粗修橋樑,通車馬而已,勿去草鏟令平也。」在淮南行兵,如在境內,禁士卒無得踐傷粟稻;或伐民樹以供軍用,皆留絹償之。宮室非不得已不修,衣弊,浣濯而服之,鞍勒用鐵木而已。幼多力善射,能以指彈碎羊骨,射禽獸無不命中;及年十五,遂不復畋獵。常謂史官曰:「時事不可以不直書。人君威福在己,無能制之者;若史策復不書其惡,將何所畏忌邪!」

\subsubsection{延兴}

\begin{longtable}{|>{\centering\scriptsize}m{2em}|>{\centering\scriptsize}m{1.3em}|>{\centering}m{8.8em}|}
  % \caption{秦王政}\
  \toprule
  \SimHei \normalsize 年数 & \SimHei \scriptsize 公元 & \SimHei 大事件 \tabularnewline
  % \midrule
  \endfirsthead
  \toprule
  \SimHei \normalsize 年数 & \SimHei \scriptsize 公元 & \SimHei 大事件 \tabularnewline
  \midrule
  \endhead
  \midrule
  元年 & 471 & \tabularnewline\hline
  二年 & 472 & \tabularnewline\hline
  三年 & 473 & \tabularnewline\hline
  四年 & 474 & \tabularnewline\hline
  五年 & 475 & \tabularnewline\hline
  六年 & 476 & \tabularnewline
  \bottomrule
\end{longtable}

\subsubsection{承明}

\begin{longtable}{|>{\centering\scriptsize}m{2em}|>{\centering\scriptsize}m{1.3em}|>{\centering}m{8.8em}|}
  % \caption{秦王政}\
  \toprule
  \SimHei \normalsize 年数 & \SimHei \scriptsize 公元 & \SimHei 大事件 \tabularnewline
  % \midrule
  \endfirsthead
  \toprule
  \SimHei \normalsize 年数 & \SimHei \scriptsize 公元 & \SimHei 大事件 \tabularnewline
  \midrule
  \endhead
  \midrule
  元年 & 476 & \tabularnewline
  \bottomrule
\end{longtable}

\subsubsection{太和}

\begin{longtable}{|>{\centering\scriptsize}m{2em}|>{\centering\scriptsize}m{1.3em}|>{\centering}m{8.8em}|}
  % \caption{秦王政}\
  \toprule
  \SimHei \normalsize 年数 & \SimHei \scriptsize 公元 & \SimHei 大事件 \tabularnewline
  % \midrule
  \endfirsthead
  \toprule
  \SimHei \normalsize 年数 & \SimHei \scriptsize 公元 & \SimHei 大事件 \tabularnewline
  \midrule
  \endhead
  \midrule
  元年 & 477 & \tabularnewline\hline
  二年 & 478 & \tabularnewline\hline
  三年 & 479 & \tabularnewline\hline
  四年 & 480 & \tabularnewline\hline
  五年 & 481 & \tabularnewline\hline
  六年 & 482 & \tabularnewline\hline
  七年 & 483 & \tabularnewline\hline
  八年 & 484 & \tabularnewline\hline
  九年 & 485 & \tabularnewline\hline
  十年 & 486 & \tabularnewline\hline
  十一年 & 487 & \tabularnewline\hline
  十二年 & 488 & \tabularnewline\hline
  十三年 & 489 & \tabularnewline\hline
  十四年 & 490 & \tabularnewline\hline
  十五年 & 491 & \tabularnewline\hline
  十六年 & 492 & \tabularnewline\hline
  十七年 & 493 & \tabularnewline\hline
  十八年 & 494 & \tabularnewline\hline
  十九年 & 495 & \tabularnewline\hline
  二十年 & 496 & \tabularnewline\hline
  二一年 & 497 & \tabularnewline\hline
  二二年 & 498 & \tabularnewline\hline
  二三年 & 499 & \tabularnewline
  \bottomrule
\end{longtable}


%%% Local Variables:
%%% mode: latex
%%% TeX-engine: xetex
%%% TeX-master: "../../Main"
%%% End:

%% -*- coding: utf-8 -*-
%% Time-stamp: <Chen Wang: 2021-11-01 15:11:15>

\subsection{宣武帝元恪\tiny(499-515)}

\subsubsection{生平}

魏宣武帝元恪(483年-515年2月12日),河南郡洛阳县(今河南省洛阳市东)人,魏孝文帝元宏次子,生母貴人高照容,是南北朝时期北魏的第八代皇帝。499年-515年在位,在位十六年。

太和七年(483年),異母兄元恂之生母林氏按北魏子貴母死之慣例而被賜死,元恂由嫡曾祖母馮太后養育。太和十七年(493年)七月,長兄元恂被立為皇太子。孝文帝遠征南齊,十歲的元恂留守新都洛陽。元恂嫌河南酷暑,穿胡服。

太和二十年(496年),十三歲的元恂逃至平城,得到反對漢化和南遷的貴族的支持。其父孝文帝返回後平息了變亂,廢黜元恂為庶人,囚禁在河陽,衣食僅夠維生。不久,又派人將元恂賜死。元恪的長兄元恂死時年僅15歲。

太和二十二年(498年),孝文帝改立十六歲的元恪為皇太子。

翌年,孝文帝崩。十七歲的元恪即位。

宣武帝的統治初期(499年至508年),他的叔父北魏宗室咸陽王元禧(獻文帝拓跋弘次子,孝文帝元宏之異母弟)輔政、尚書令王肅輔佐。

北魏對南朝發動了一系列戰爭,攻取南朝梁的四川之地、北撃柔然,北魏疆域大大向南拓展,國勢盛極一時。因篤信佛教,宣武帝取消子貴母死制度,讓宣武靈皇后活著。

他在位的後半期,外戚高肇專權,朝政一片黑暗,北魏逐漸衰弱。延昌四年正月,宣武帝崩於式乾殿。

\subsubsection{景明}

\begin{longtable}{|>{\centering\scriptsize}m{2em}|>{\centering\scriptsize}m{1.3em}|>{\centering}m{8.8em}|}
  % \caption{秦王政}\
  \toprule
  \SimHei \normalsize 年数 & \SimHei \scriptsize 公元 & \SimHei 大事件 \tabularnewline
  % \midrule
  \endfirsthead
  \toprule
  \SimHei \normalsize 年数 & \SimHei \scriptsize 公元 & \SimHei 大事件 \tabularnewline
  \midrule
  \endhead
  \midrule
  元年 & 500 & \tabularnewline\hline
  二年 & 501 & \tabularnewline\hline
  三年 & 502 & \tabularnewline\hline
  四年 & 503 & \tabularnewline\hline
  五年 & 504 & \tabularnewline
  \bottomrule
\end{longtable}

\subsubsection{正始}

\begin{longtable}{|>{\centering\scriptsize}m{2em}|>{\centering\scriptsize}m{1.3em}|>{\centering}m{8.8em}|}
  % \caption{秦王政}\
  \toprule
  \SimHei \normalsize 年数 & \SimHei \scriptsize 公元 & \SimHei 大事件 \tabularnewline
  % \midrule
  \endfirsthead
  \toprule
  \SimHei \normalsize 年数 & \SimHei \scriptsize 公元 & \SimHei 大事件 \tabularnewline
  \midrule
  \endhead
  \midrule
  元年 & 504 & \tabularnewline\hline
  二年 & 505 & \tabularnewline\hline
  三年 & 506 & \tabularnewline\hline
  四年 & 507 & \tabularnewline\hline
  五年 & 508 & \tabularnewline
  \bottomrule
\end{longtable}

\subsubsection{永平}

\begin{longtable}{|>{\centering\scriptsize}m{2em}|>{\centering\scriptsize}m{1.3em}|>{\centering}m{8.8em}|}
  % \caption{秦王政}\
  \toprule
  \SimHei \normalsize 年数 & \SimHei \scriptsize 公元 & \SimHei 大事件 \tabularnewline
  % \midrule
  \endfirsthead
  \toprule
  \SimHei \normalsize 年数 & \SimHei \scriptsize 公元 & \SimHei 大事件 \tabularnewline
  \midrule
  \endhead
  \midrule
  元年 & 508 & \tabularnewline\hline
  二年 & 509 & \tabularnewline\hline
  三年 & 510 & \tabularnewline\hline
  四年 & 511 & \tabularnewline\hline
  五年 & 512 & \tabularnewline
  \bottomrule
\end{longtable}

\subsubsection{延昌}

\begin{longtable}{|>{\centering\scriptsize}m{2em}|>{\centering\scriptsize}m{1.3em}|>{\centering}m{8.8em}|}
  % \caption{秦王政}\
  \toprule
  \SimHei \normalsize 年数 & \SimHei \scriptsize 公元 & \SimHei 大事件 \tabularnewline
  % \midrule
  \endfirsthead
  \toprule
  \SimHei \normalsize 年数 & \SimHei \scriptsize 公元 & \SimHei 大事件 \tabularnewline
  \midrule
  \endhead
  \midrule
  元年 & 512 & \tabularnewline\hline
  二年 & 513 & \tabularnewline\hline
  三年 & 514 & \tabularnewline\hline
  四年 & 515 & \tabularnewline
  \bottomrule
\end{longtable}


%%% Local Variables:
%%% mode: latex
%%% TeX-engine: xetex
%%% TeX-master: "../../Main"
%%% End:

%% -*- coding: utf-8 -*-
%% Time-stamp: <Chen Wang: 2019-12-23 15:16:31>

\subsection{孝明帝\tiny(515-528)}

\subsubsection{生平}

魏孝明帝元诩(510年-528年3月31日),河南郡洛阳县(今河南省洛阳市东)人,魏宣武帝元恪第二子,生母胡充华,是南北朝时期北魏的皇帝,在位13年。

延昌四年(515年),宣武帝病逝,太子元詡繼位,由太傅、侍中元懌輔政。胡太后和元懌相愛,常招元懌夜宿宮中。領軍元乂和長秋卿劉騰等人密謀,將元懌殺害,又把胡太后幽禁在北宮的宣光殿。胡太后又結識鄭儼、李神軌、徐紇諸情人。鄭儼和徐紇把持內外,時稱“徐鄭”。

正光年间以后,国家入不敷出,政府决定预先征收六年的租調,导致人民生活愈加艰苦。除此之外,政府停止对官员供应酒,但每季所举行的祭祀祖宗、神明的仪式及外交所需费用不算在内。由于匪徒越来越多,大量器械被劫掠,关西地区匪患尤其严重。政府的积蓄于是枯竭。在此境地下,政府又下令减少对官员及外国使节百分之五十的粮食及肉供应。孝昌年间,京城的田每亩征税五升,借公田耕种者每亩征税一斗。并且还对市场,商业店铺征税。

武泰元年(528年),孝明帝妃潘外憐生一女,胡太后宣稱生男孩,大赦天下。武泰元年二月癸丑(528年3月31日),鄭儼率御林軍來到顯陽殿,將孝明帝毒死。爾朱榮聞訊,追查孝明帝的死因,另立長樂王元子攸。

\subsubsection{熙平}

\begin{longtable}{|>{\centering\scriptsize}m{2em}|>{\centering\scriptsize}m{1.3em}|>{\centering}m{8.8em}|}
  % \caption{秦王政}\
  \toprule
  \SimHei \normalsize 年数 & \SimHei \scriptsize 公元 & \SimHei 大事件 \tabularnewline
  % \midrule
  \endfirsthead
  \toprule
  \SimHei \normalsize 年数 & \SimHei \scriptsize 公元 & \SimHei 大事件 \tabularnewline
  \midrule
  \endhead
  \midrule
  元年 & 516 & \tabularnewline\hline
  二年 & 517 & \tabularnewline\hline
  三年 & 518 & \tabularnewline
  \bottomrule
\end{longtable}

\subsubsection{神龟}

\begin{longtable}{|>{\centering\scriptsize}m{2em}|>{\centering\scriptsize}m{1.3em}|>{\centering}m{8.8em}|}
  % \caption{秦王政}\
  \toprule
  \SimHei \normalsize 年数 & \SimHei \scriptsize 公元 & \SimHei 大事件 \tabularnewline
  % \midrule
  \endfirsthead
  \toprule
  \SimHei \normalsize 年数 & \SimHei \scriptsize 公元 & \SimHei 大事件 \tabularnewline
  \midrule
  \endhead
  \midrule
  元年 & 518 & \tabularnewline\hline
  二年 & 519 & \tabularnewline\hline
  三年 & 520 & \tabularnewline
  \bottomrule
\end{longtable}

\subsubsection{正光}

\begin{longtable}{|>{\centering\scriptsize}m{2em}|>{\centering\scriptsize}m{1.3em}|>{\centering}m{8.8em}|}
  % \caption{秦王政}\
  \toprule
  \SimHei \normalsize 年数 & \SimHei \scriptsize 公元 & \SimHei 大事件 \tabularnewline
  % \midrule
  \endfirsthead
  \toprule
  \SimHei \normalsize 年数 & \SimHei \scriptsize 公元 & \SimHei 大事件 \tabularnewline
  \midrule
  \endhead
  \midrule
  元年 & 520 & \tabularnewline\hline
  二年 & 521 & \tabularnewline\hline
  三年 & 522 & \tabularnewline\hline
  四年 & 523 & \tabularnewline\hline
  五年 & 524 & \tabularnewline\hline
  六年 & 525 & \tabularnewline
  \bottomrule
\end{longtable}

\subsubsection{孝昌}

\begin{longtable}{|>{\centering\scriptsize}m{2em}|>{\centering\scriptsize}m{1.3em}|>{\centering}m{8.8em}|}
  % \caption{秦王政}\
  \toprule
  \SimHei \normalsize 年数 & \SimHei \scriptsize 公元 & \SimHei 大事件 \tabularnewline
  % \midrule
  \endfirsthead
  \toprule
  \SimHei \normalsize 年数 & \SimHei \scriptsize 公元 & \SimHei 大事件 \tabularnewline
  \midrule
  \endhead
  \midrule
  元年 & 525 & \tabularnewline\hline
  二年 & 526 & \tabularnewline\hline
  三年 & 527 & \tabularnewline\hline
  四年 & 528 & \tabularnewline
  \bottomrule
\end{longtable}

\subsubsection{武泰}

\begin{longtable}{|>{\centering\scriptsize}m{2em}|>{\centering\scriptsize}m{1.3em}|>{\centering}m{8.8em}|}
  % \caption{秦王政}\
  \toprule
  \SimHei \normalsize 年数 & \SimHei \scriptsize 公元 & \SimHei 大事件 \tabularnewline
  % \midrule
  \endfirsthead
  \toprule
  \SimHei \normalsize 年数 & \SimHei \scriptsize 公元 & \SimHei 大事件 \tabularnewline
  \midrule
  \endhead
  \midrule
  元年 & 528 & \tabularnewline
  \bottomrule
\end{longtable}


%%% Local Variables:
%%% mode: latex
%%% TeX-engine: xetex
%%% TeX-master: "../../Main"
%%% End:

%% -*- coding: utf-8 -*-
%% Time-stamp: <Chen Wang: 2021-11-01 15:11:46>

\subsection{元氏生平}

元氏(528年2月12日-?),女,河南洛阳(今河南省洛阳市东)人,真名不详,南北朝时期北魏第10位皇帝(未获后世普遍承认),是北魏孝明帝元詡与宫嫔潘外憐的女儿,也是孝明帝唯一的骨肉。出生后因时局危险,所以她的祖母、掌握帝国实际大权的皇太后胡氏对外宣称本为女性(即皇女)的她是男性(即为皇子),以安人心。不久,孝明帝暴崩,尚在襁褓中的“皇子”元氏以先帝唯一子嗣的身份继位(528年4月1日),在名义上成为了北魏皇帝。元氏即位当天便被废黜,次日由幼主元钊继位,之后史书上便没有了对她的记载。

北魏孝明帝孝昌四年春正月初七乙丑日(528年2月12日),皇女元氏出生。她是孝明帝第一个、也是唯一的孩子,她的母亲是孝明帝九嫔之一的充华潘外憐,而她的祖母即为临朝称制的灵太后胡氏。胡太后对外宣称潘外憐生下的是皇子,并于第二天(正月初八丙寅日,即2月13日)颁布诏书,大赦天下,改元武泰。胡太后之所以诈称皇女为皇子,和她与儿子元诩的矛盾有关。胡太后本为北魏宣武帝众妾之一的充华,但却为宣武帝生下了唯一存活下来的子嗣,即后来继位的元诩。宣武帝死后,元诩继皇帝位,是为孝明帝,尊胡充华为皇太妃,后又加尊为皇太后。因孝明帝年幼,胡太后临朝称制。

但孝明帝日渐长大,胡太后却不肯放权归政,而且在政治上排斥异己,生活上不检点,不但引起朝臣反感,连孝明帝对她也大为不满,甚至招致天下人的厌恶,尤其是孝明帝把与胡太后私通的清河王元怿处死后,胡太后对儿子也恨之入骨,母子之间的裂痕越来越深。经过了几次反对胡太后的失败的政变,孝明帝密令大将尔朱荣进兵首都洛阳(今河南省洛阳市),试图胁迫胡太后归政。胡太后知道后,与近臣商议对策,适逢孝明帝的充華潘外憐生了皇女,所以胡太后假称是皇子,大赦天下,转移朝臣的视线,暗中谋划除去孝明帝。

武泰元年二月廿五癸丑日(528年3月31日),孝明帝元詡突然驾崩,是被胡太后暗中串通近臣用毒酒毒死的。第二天(即武泰元年二月廿六甲寅日,528年4月1日),胡太后伪称皇女元氏为皇太子,拥立元氏登基为皇帝,胡太后继续临朝称制,再次大赦天下。当时的元氏出生刚满50天。由于没有跨年,未改元,仍然沿用“武泰”年号。这样,元氏成为名义上的北魏皇帝。当然,由于这位女婴皇帝才出生一个多月,不可能真正地行使皇帝权力,实权仍然掌握在她的祖母、临朝称制的胡太后手中,元氏只不过是胡太后的傀儡。

胡太后立皇女元氏为帝才一天不到,见人心已经安定,当天便发下诏书宣布皇帝本是女儿身,废黜女婴皇帝,改立已故宗室临洮王元宝晖的世子元钊为皇帝。

胡太后的诏书全文如下:“皇家握历受图,年将二百;祖宗累圣,社稷载安。高祖以文思先天,世宗以下武经世,股肱惟良,元首穆穆。及大行在御,重以宽仁,奉养率由,温明恭顺。朕以寡昧,亲临万国,识谢涂山,德惭文母。属妖逆递兴,四郊多故。实望穹灵降祐,麟趾众繁。自潘充华有孕椒宫,冀诞储两,而熊罴无兆,维虺遂彰。于时直以国步未康,假称统胤,欲以底定物情,系仰宸极。何图一旦,弓剑莫追,国道中微,大行绝祀。皇曾孙故临洮王宝晖世子钊,体自高祖,天表卓异,大行平日养爱特深,义齐若子,事符当璧。及翊日弗愈,大渐弥留,乃延入青蒲,受命玉几。暨陈衣在庭,登策靡及,允膺大宝,即日践阼。朕是用惶惧忸怩,心焉靡洎。今丧君有君,宗祏惟固,宜崇赏卿士,爰及百辟,凡厥在位,并加陟叙。内外百官文武、督将征人,遭艰解府,普加军功二阶;其禁卫武官,直阁以下直从以上及主帅,可军功三阶;其亡官失爵,听复封位。谋反大逆削除者,不在斯限。清议禁锢,亦悉蠲除。若二品以上不能自受者,任授兒弟。可班宣远迩,咸使知之。”

胡太后诏书的大意就是孝明帝死得仓促,来不及指定继承人,只好让他唯一的女儿暂且以皇子身份继承皇位,后来发现元钊是皇位的合适人选,便废女婴皇帝而让元钊当皇帝。

元钊于胡太后发下诏书后的第二天(即武泰元年二月廿七乙卯日,528年4月2日)正式即位,是为北魏幼主。

实际上,这一废一立都是胡太后试图长期掌握帝国最高权力而使出的手段,因为元钊虽然比女婴皇帝大几岁,但也只有三岁,胡太后立他为皇帝是因为他年幼不能管理国家,她可以继续临朝称制,统治天下了。实际上,胡太后自被孝明帝尊为太后开始就是北魏的实际统治者了,她不但临朝称制,还自称为“朕”(秦始皇以后的皇帝自称),让朝臣们称她为“陛下”(臣下对皇帝的尊称)。她不惜先毒杀亲生儿子,后立尚在襁褓中的孙女,再立刚满三岁的宗室幼子,就是因为她是妇人之身,在当时的情况下不能直接行使皇帝权力,立年幼无知的傀儡皇帝可以保证自己通过临朝称制的方法继续掌权专制,而其最后终于身败名裂,遂被后世认为是不成功的野心家和导致北魏分裂亡国的罪人。

女婴皇帝被废後無紀錄可循,再也不知所終,但她因為即位為帝,導致的政治影響很大。

而短时间的废立让天下震惊,認定胡太后害死孝明帝,大将尔朱荣遂以胡太后肆意废立为藉口带兵讨伐,而其中一条重要的理由就是胡太后欺瞒上天和朝臣,立女婴为帝。尔朱荣又另立元子攸为皇帝,是为孝莊帝,这样北魏出现了两帝并立的局面。不过这种局面很快就被打破,15日后,尔朱荣的军队占领京师洛阳,胡太后和幼主元钊被俘。尔朱荣将幼主和胡太后押送至黄河南岸边。胡太后在尔朱荣面前说了许多好话求饶,尔朱荣不听,下令将幼主和胡太后沉入黄河,后又屠杀大臣两千多人,史称河阴之变。从此,尔朱荣完全掌握了北魏的实际大权,而北魏则开始了由军阀权臣掌控的时代,直接导致了国家的分裂。

元氏的皇帝身份常有爭議,就事實上是當了皇帝沒錯,但短暫的女皇帝身份普遍不被后世所承认,而且史书中,尤其是在正史中从来不把她列入正统的帝系,一来是因为她是胡太后的傀儡,且在位时间只有一天不到,二来是因为她是以冒名男婴而即帝位的。因此,至今为止,武周王朝的武则天仍然被普遍认为是中国历史上的第一位、也是唯一的一位女皇帝。不过,学者成扬则认为,元氏的登基虽然是北魏统治集团内部权力斗争的产物尤其是胡太后一手造成的,但其作为“第一个登上皇帝宝座的女性,这一事实却不容抹杀”,并建议将武则天的身份从“(中国)历史上唯一的女皇帝”修改为“中国历史上有作为的女皇帝”。对于成扬的说法,另一位研究武则天的专家罗元贞予以驳斥,他认为元氏的女皇帝地位连封建时代的史家都不承认,“在社会主义的新中国居然承认”,斥之为“标新立异、沽名钓誉之一例”,他本人则坚持认为武则天才是“真正的”中国第一位、也是唯一的一位女皇帝。


\subsection{幼主元钊生平}

元钊(526年-528年5月17日),河南郡洛阳县(今河南省洛阳市东)人,魏孝文帝元宏曾孙,临洮王元寶暉之子,北魏皇帝。

魏孝明帝元诩厌恶灵太后的宠臣郑俨、徐纥等人,因为受到灵太后的威逼,不能将他们赶走,就秘密的命令尔朱荣带领军队前往洛阳,希望以此胁迫灵太后。尔朱荣以高欢为先锋,抵达上党,魏孝明帝又下诏让尔朱荣停止进军。郑俨和徐纥担心祸事惹到身上,就阴谋与灵太后用毒酒谋害魏孝明帝。武泰元年二月癸丑(528年3月31日),魏孝明帝突然死去。二月甲寅(528年4月1日),灵太后立魏孝明帝之女元氏为皇帝,大赦。既而下诏称:“潘充华本来生的是女儿,已故临洮王元宝晖的世子元钊,是高祖的后裔,适合接受皇位。”二月乙卯(528年4月2日),元钊即位。元钊当年才虚龄三岁,灵太后希望长久的独揽大权,所以贪图元钊年幼而册立他为皇帝。

建义元年四月庚子(528年5月17日),尔朱荣派遣骑兵逮捕灵太后和元钊,送到河阴,灵太后对尔朱荣多方辩解自己的行为,尔朱荣拂袖起身,灵太后和元钊都被沉入黄河,灵太后的妹妹胡玄辉将灵太后和元钊的尸体收敛在双灵寺中。

\subsection{孝庄帝元子攸\tiny(528-530)}

\subsubsection{孝庄帝生平}

魏孝莊帝元子攸(507年-531年1月26日),字彦达,河南郡洛阳县(今河南省洛阳市东)人,魏献文帝拓跋弘之孙,彭城王元勰第三子,母親為王妃李媛華。孝莊帝是尔朱荣拥立的傀儡皇帝,最终被尔朱兆绞杀。

元子攸姿貌很俊美,有勇力。自幼在宫为孝明帝元诩担任伴读,与魏孝明帝颇为友爱,官至中书侍郎,封武城县开国公,527年,被特封长乐王。

528年5月15日(農曆四月十一日、建義元年),元子攸被尔朱荣拥立为皇帝。

孝莊帝永安二年(529年)鑄永安五銖錢。九月二十五日(530年11月1日、永安三年),孝莊帝伏兵明光殿,聲稱皇后大尔朱氏生下了太子,派元徽向爾朱榮報喜。爾朱榮跟元天穆一起入朝。元子攸听说尔朱荣进宫臉色緊張,連忙喝酒以遮掩。尔朱荣见到光祿少卿魯安、典御李侃晞從東廂門执刀闖入,便撲向元子攸。元子攸用藏在膝下的刀砍到尔朱荣,魯安等揮刀亂砍,殺爾朱榮與元天穆等人。

十月三十日(530年12月5日、永安三年),爾朱兆另立元曄為帝。十二月三日(531年1月6日),爾朱兆攻入洛陽,殺死孝莊帝在襁褓中的儿子,孝莊帝被俘囚於永寧寺、後解送囚於晉陽三級寺。

永安三年十二月甲子(531年1月26日),魏孝莊帝於晋阳城(今太原市晋源区境)三級寺被尔朱兆絞殺,虚岁二十四。臨終前孝莊帝向佛祖禮拜,發願生生世世不做皇帝,並賦詩明志:「權去生道促,憂來死路長。懷恨出國門,含悲入鬼鄉。隧門一時閉,幽庭豈復光。思鳥吟青松,哀風吹白楊。昔來聞死苦,何言身自當。」中兴二年(532年),元朗给元子攸上谥号为武怀皇帝,魏孝武帝元修即位后,因为武怀谥号犯魏孝武帝父亲元怀名讳,于太昌元年(532年)改元子攸谥号为孝庄皇帝,庙号敬宗。十一月,葬于静陵。

\subsubsection{建义}

\begin{longtable}{|>{\centering\scriptsize}m{2em}|>{\centering\scriptsize}m{1.3em}|>{\centering}m{8.8em}|}
  % \caption{秦王政}\
  \toprule
  \SimHei \normalsize 年数 & \SimHei \scriptsize 公元 & \SimHei 大事件 \tabularnewline
  % \midrule
  \endfirsthead
  \toprule
  \SimHei \normalsize 年数 & \SimHei \scriptsize 公元 & \SimHei 大事件 \tabularnewline
  \midrule
  \endhead
  \midrule
  元年 & 528 & \tabularnewline
  \bottomrule
\end{longtable}

\subsubsection{永安}

\begin{longtable}{|>{\centering\scriptsize}m{2em}|>{\centering\scriptsize}m{1.3em}|>{\centering}m{8.8em}|}
  % \caption{秦王政}\
  \toprule
  \SimHei \normalsize 年数 & \SimHei \scriptsize 公元 & \SimHei 大事件 \tabularnewline
  % \midrule
  \endfirsthead
  \toprule
  \SimHei \normalsize 年数 & \SimHei \scriptsize 公元 & \SimHei 大事件 \tabularnewline
  \midrule
  \endhead
  \midrule
  元年 & 528 & \tabularnewline\hline
  二年 & 529 & \tabularnewline\hline
  三年 & 530 & \tabularnewline
  \bottomrule
\end{longtable}


%%% Local Variables:
%%% mode: latex
%%% TeX-engine: xetex
%%% TeX-master: "../../Main"
%%% End:

%% -*- coding: utf-8 -*-
%% Time-stamp: <Chen Wang: 2021-11-01 15:12:37>

\subsection{东海王元晔\tiny(530-531)}

\subsubsection{生平}

元晔(?-532年12月26日),字华兴,小字盆子,河南郡洛阳县(今河南省洛阳市东)人,追尊魏景穆帝拓跋晃曾孙,鄯善镇将、扶风王元怡之子,北魏宗室、官员,一度被尔朱兆与尔朱世隆拥立为皇帝。

元晔性格轻浮急躁,有体力,以秘书郎为起家官,略微升迁至通直散骑常侍。建义元年四月甲辰(528年5月21日),魏孝庄帝元子攸封秘书郎中元晔为长广王,食邑一千户,外任太原郡太守,代理并州刺史。尔朱荣死后,尔朱世隆逃回并州的建兴郡高都县,尔朱兆从晋阳县前来汇合,于是在永安三年十月壬申(530年12月5日)推举元晔为皇帝,大赦所管辖的地区,年号建明,所有官员加四级。元晔小名盆子,听说此事的人都认为类似赤眉军之事。

建明元年十二月戊申(531年1月10日),元晔在攻克洛阳后大赦天下。尔朱世隆与兄弟密谋,担心元晔的母亲卫氏干预朝政,观察卫氏出行,派遣数十骑兵装扮成劫匪,在京城小巷子里杀死了卫氏。公家和私人都感到惊愕,不知道什么原因。很快又张贴公告悬赏,以一千万钱悬赏劫匪。百姓知道后,没有不垂头丧气的。尔朱世隆很快认为元晔是北魏宗室远支,又不是众望所推的人,想要推举宗室近支广陵王元恭为皇帝。建明元年春二月己巳(531年4月1日),元晔前往邙山南,尔朱世隆等人在洛阳东城外奉迎广陵王元恭,尔朱世隆等人写好禅让的册文,以泰山郡太守窦瑗执马鞭进入行宫,启奏元晔说:“上天和百姓的愿望,都在广陵王身上,请实行尧舜禅让的礼仪。”元晔因此退位。

魏节闵帝元恭登基后,于普泰元年三月癸酉(531年4月5日)封元晔为东海王,食邑一万户。太昌元年十一月甲辰(532年12月26日),安定王元朗和元晔在家中被赐令自杀。元晔的爵位被削除。

\subsubsection{建明}

\begin{longtable}{|>{\centering\scriptsize}m{2em}|>{\centering\scriptsize}m{1.3em}|>{\centering}m{8.8em}|}
  % \caption{秦王政}\
  \toprule
  \SimHei \normalsize 年数 & \SimHei \scriptsize 公元 & \SimHei 大事件 \tabularnewline
  % \midrule
  \endfirsthead
  \toprule
  \SimHei \normalsize 年数 & \SimHei \scriptsize 公元 & \SimHei 大事件 \tabularnewline
  \midrule
  \endhead
  \midrule
  元年 & 530 & \tabularnewline\hline
  二年 & 531 & \tabularnewline
  \bottomrule
\end{longtable}


%%% Local Variables:
%%% mode: latex
%%% TeX-engine: xetex
%%% TeX-master: "../../Main"
%%% End:

%% -*- coding: utf-8 -*-
%% Time-stamp: <Chen Wang: 2019-12-23 15:20:56>

\subsection{节闵帝\tiny(531)}

\subsubsection{生平}

魏节闵帝元恭(498年-532年6月21日;在位531年-532年),字修业,河南郡洛阳县(今河南省洛阳市东)人,魏献文帝拓跋弘之孙,广陵惠王元羽之子,母王氏,是南北朝时期北魏的皇帝。

先前元恭的伯父北魏孝文帝元宏为鼓励鲜卑与汉族通婚而将元羽嫡妻降为妾室,另聘娶郑始容为元羽嫡妻。史书没有记载元恭的母亲王氏是否就是元羽被降为妾室的元配,仅记载元羽死后袭爵广陵王的是元恭,而非元恭的哥哥元欣。

建明二年二月廿九日(531年4月1日),爾朱榮堂弟爾朱世隆废元晔,立元恭為帝。军阀高欢则立渤海太守安定王元朗为帝。高欢打败尔朱氏后,考虑到元朗世系疏远,一度想尊奉元恭,派仆射魏兰根慰谕洛阳观察节闵帝为人。魏兰根认为节闵帝神采高明,日后难制,与侍中、司空高乾兄弟及黄门侍郎崔㥄强调节闵帝系尔朱氏所立,共劝高欢为了讨伐尔朱氏师出有名而废帝。532年6月(农历四月),节闵帝被高欢所废,囚禁于崇训佛寺。元朗亦被高欢所迫禅位给平阳王元修即北魏孝武帝。节闵帝赋诗:“朱門久可患,紫極非情玩。顛覆立可待,一年三易換。時運正如此,唯有修真觀。”

五月丙申(532年6月21日),魏节闵帝在门下外省被孝武帝毒死,虚岁三十五,葬以亲王殊礼;加九旒、銮辂、黄屋、左纛,班剑百二十人,二卫、羽林备仪卫。东魏称广陵王或前废帝,西魏谥节闵帝。

2013年,洛阳市文物考古研究院完成北魏节愍帝的陵墓考古挖掘工作,陵墓中出土东罗马帝国金币一枚。

\subsubsection{普泰}

\begin{longtable}{|>{\centering\scriptsize}m{2em}|>{\centering\scriptsize}m{1.3em}|>{\centering}m{8.8em}|}
  % \caption{秦王政}\
  \toprule
  \SimHei \normalsize 年数 & \SimHei \scriptsize 公元 & \SimHei 大事件 \tabularnewline
  % \midrule
  \endfirsthead
  \toprule
  \SimHei \normalsize 年数 & \SimHei \scriptsize 公元 & \SimHei 大事件 \tabularnewline
  \midrule
  \endhead
  \midrule
  元年 & 531 & \tabularnewline
  \bottomrule
\end{longtable}


%%% Local Variables:
%%% mode: latex
%%% TeX-engine: xetex
%%% TeX-master: "../../Main"
%%% End:

%% -*- coding: utf-8 -*-
%% Time-stamp: <Chen Wang: 2021-11-01 15:12:51>

\subsection{后废帝元朗\tiny(531-532)}

\subsubsection{生平}

元朗(513年-532年12月26日;在位531年-532年),字仲哲,河南郡洛阳县(今河南省洛阳市东)人,追尊魏景穆帝拓跋晃玄孙,章武王元融的儿子,是中國南北朝时期北魏的第14代君主,無廟號,史稱後廢帝。

元朗在531年10月30日(农历十月六日)被高欢立为皇帝,532年6月(农历四月)高欢因元朗世系疏远,迫使他让位于孝武帝元修。太昌元年十一月甲辰(532年12月26日),元朗和元晔都在门下外省被杀死,虚岁二十。永熙二年葬于鄴西南野馬岡。

\subsubsection{中兴}

\begin{longtable}{|>{\centering\scriptsize}m{2em}|>{\centering\scriptsize}m{1.3em}|>{\centering}m{8.8em}|}
  % \caption{秦王政}\
  \toprule
  \SimHei \normalsize 年数 & \SimHei \scriptsize 公元 & \SimHei 大事件 \tabularnewline
  % \midrule
  \endfirsthead
  \toprule
  \SimHei \normalsize 年数 & \SimHei \scriptsize 公元 & \SimHei 大事件 \tabularnewline
  \midrule
  \endhead
  \midrule
  元年 & 531 & \tabularnewline\hline
  二年 & 532 & \tabularnewline
  \bottomrule
\end{longtable}


%%% Local Variables:
%%% mode: latex
%%% TeX-engine: xetex
%%% TeX-master: "../../Main"
%%% End:

%% -*- coding: utf-8 -*-
%% Time-stamp: <Chen Wang: 2019-12-23 15:22:30>

\subsection{孝武帝\tiny(532-534)}

\subsubsection{生平}

魏孝武帝元修,一说元脩(510年-535年2月3日),字孝則,河南郡洛阳县(今河南省洛阳市东)人,魏孝文帝元宏之孙,廣平武穆王元懷第三子,母親是李氏,是北魏最後一位皇帝。

他曾被封為汝陽縣公、通直散騎侍郎、中書侍郎。建義年間,辭散騎常侍,為平東將軍、太常卿,後來又為鎮東將軍、宗正卿。530年封為平陽王。普泰初年,轉任侍中、鎮東將軍、儀同三司、兼為尚書右僕射,後又改加侍中、尚書左僕射。河阴之变后政局一片混乱,诸王大多各自逃生,元修逃亡民间,隐为乡农。

中興二年(532年)高欢击败尔朱氏,欲立元悦为帝,因无法服众,只得退而选择元修。斛斯椿从元修的心腹好友王思政辗转寻到元修,元修道:“这该不是把我出卖了吧?”高欢遂亲自前来陳誠,泣下沾襟。元修方才入京,四月廿五日(6月13日)即位。

元修即位后与高欢的长女结婚,夫妻彼此都没有感情。元修与三个堂姊妹姘居,将她们都封为公主。534年與高歡決裂,高歡帶兵從晉陽南下时,元修于七月廿八日(8月22日)率一部分兵众,偕同情妇元明月及元明月的哥哥元宝炬等入關中投奔其妹妹的未婚夫宇文泰。

先前梁武帝因天象称“荧惑入南斗,天子下殿走”,就赤脚下殿以应天象,得知元修西奔,羞惭地说:“索虏也应天象吗?”因为这一天象应在元修身上就意味着上天认为元修才是正统天子。

同年十月十七日(11月8日),高歡以元修弃国逃跑,另立元善見為帝,十日后遷都鄴。宇文泰让元氏诸王杀死元明月,元修非常不高兴,闰十二月癸巳(535年2月3日),宇文泰用毒酒毒死了元修,改立元宝炬为帝。

元修死後被宇文泰下令埋进草堂佛寺,十余年后才得正式落葬。西魏上諡號為孝武皇帝。

\subsubsection{太昌}

\begin{longtable}{|>{\centering\scriptsize}m{2em}|>{\centering\scriptsize}m{1.3em}|>{\centering}m{8.8em}|}
  % \caption{秦王政}\
  \toprule
  \SimHei \normalsize 年数 & \SimHei \scriptsize 公元 & \SimHei 大事件 \tabularnewline
  % \midrule
  \endfirsthead
  \toprule
  \SimHei \normalsize 年数 & \SimHei \scriptsize 公元 & \SimHei 大事件 \tabularnewline
  \midrule
  \endhead
  \midrule
  元年 & 532 & \tabularnewline
  \bottomrule
\end{longtable}

\subsubsection{永兴}

\begin{longtable}{|>{\centering\scriptsize}m{2em}|>{\centering\scriptsize}m{1.3em}|>{\centering}m{8.8em}|}
  % \caption{秦王政}\
  \toprule
  \SimHei \normalsize 年数 & \SimHei \scriptsize 公元 & \SimHei 大事件 \tabularnewline
  % \midrule
  \endfirsthead
  \toprule
  \SimHei \normalsize 年数 & \SimHei \scriptsize 公元 & \SimHei 大事件 \tabularnewline
  \midrule
  \endhead
  \midrule
  元年 & 532 & \tabularnewline
  \bottomrule
\end{longtable}

\subsubsection{永熙}

\begin{longtable}{|>{\centering\scriptsize}m{2em}|>{\centering\scriptsize}m{1.3em}|>{\centering}m{8.8em}|}
  % \caption{秦王政}\
  \toprule
  \SimHei \normalsize 年数 & \SimHei \scriptsize 公元 & \SimHei 大事件 \tabularnewline
  % \midrule
  \endfirsthead
  \toprule
  \SimHei \normalsize 年数 & \SimHei \scriptsize 公元 & \SimHei 大事件 \tabularnewline
  \midrule
  \endhead
  \midrule
  元年 & 532 & \tabularnewline\hline
  二年 & 533 & \tabularnewline\hline
  三年 & 534 & \tabularnewline
  \bottomrule
\end{longtable}


%%% Local Variables:
%%% mode: latex
%%% TeX-engine: xetex
%%% TeX-master: "../../Main"
%%% End:



%%% Local Variables:
%%% mode: latex
%%% TeX-engine: xetex
%%% TeX-master: "../../Main"
%%% End:

%% -*- coding: utf-8 -*-
%% Time-stamp: <Chen Wang: 2019-12-23 15:25:07>


\section{东魏\tiny(534-550)}

\subsection{简介}

東魏(534年-550年)是中國南北朝時期北方王朝之一,由鮮卑化漢人高歡擁立北魏孝文帝年僅十一歲的曾孫元善見為孝靜帝,並與宇文泰所掌控的西魏對立,建都邺,以高欢霸府「大丞相府」所在地晋阳为别都。

北魏孝武帝为对抗权臣高歡逃奔关中后,高欢另立元善见为帝,东魏開始。

東魏自始,便是霸府政治權臣高欢架空皇帝掌控整个政局。高歡於公元547年病死,其权势由長子高澄所繼承。随即发生大将侯景叛投西魏的事件,但被高澄所平定。高澄权倾人主,曾命手下崔季舒打孝静帝三拳。后来孝静帝君臣图谋高澄,事泄,高澄指责身为皇帝的孝静帝要造反,虽然当时与孝静帝和解,但很快幽禁了孝静帝并诛杀参与图谋的大臣。公元550年,當二十七歲的孝靜帝以為高澄遇刺身亡、自己可以親政時,随即被高澄之弟高洋所廢,東魏亡。東魏只有孝靜帝元善見一个皇帝,用四個年號,共十六年。東魏由北齊取代。

東魏時期的藝術創作仍以佛教為主要啟發。位於泰山的神通寺,是中國現存最古老的石造塔寺,可能建於東魏一代。此時期的佛雕較北魏時期渾厚。

%% -*- coding: utf-8 -*-
%% Time-stamp: <Chen Wang: 2021-11-01 15:13:11>

\subsection{孝静帝元善見\tiny(534-550)}

\subsubsection{生平}

魏孝靜帝元善見(524年-552年1月21日),河南郡洛阳县(今河南省洛阳市东)人,魏孝文帝元宏曾孫,清河文宣王元亶嫡子,母清河王妃胡智,南北朝時期東魏唯一一代皇帝。政权被高欢父子控制。

永熙三年(西元534年)七月,魏孝武帝從洛陽出逃,投靠在長安的宇文泰。元善见父元亶原本自以为可以被权臣高欢拥立为帝,但十月,高歡和百僚詳細商議後,決定立元善見為皇帝,承嗣魏孝明帝元诩。即位於鄴城東北,改元天平,東魏正式建立,年僅十一歲。由於年幼,由高欢輔政。

高歡善於玩弄權術,權傾朝野,令孝靜帝天天都提心吊膽,對高歡頗為畏懼。高歡雖屢敗於勁敵西魏宇文泰,但一直把持权力。而高歡死後,其子高澄承繼父職,權勢更大。孝静帝在邺城东打猎,骑马疾驰,監衛都督從後叫:「天子不要馳馬,高澄發怒。」;高澄侍飲,舉起大酒杯勸酒,孝静帝不悦说“自古无不亡之国,朕何需要這樣活着?”,被高澄令手下崔季舒打了三拳。孝静帝与荀济等密謀要刺殺高澄,但事蹟敗露,高澄带兵入宫,指责孝静帝谋反,孝静帝驳斥,高澄痛哭谢罪,与孝靜帝痛饮到深夜,但三天后就將孝靜帝幽禁于含章堂而将荀济等烹杀于市。公元549年,高洋再繼任父兄之職,他見篡魏之時機已到,於次年迫帝禪位於己,改國號「齊」,東魏亡。

北齊封元善見為中山王,邑一萬戶;上書不稱臣,答不稱詔,載天子旌旗,行魏正朔,乘五時副車;封王諸子為縣公,邑各一千戶;奉絹三萬匹,錢一千萬,粟二萬石,奴婢三百人,水碾一具,田百頃,園一所;於中山國立魏宗廟。

天保二年十二月己酉(552年1月21日),高洋灌醉元善见的妻子高皇后(太原長公主),派人用毒酒毒死了元善见,又杀死了他的三个儿子,元善见时年28岁。

天保三年二月,北齐上谥号孝靜皇帝,将元善见葬于邺县漳河以北。他的陵墓曾经被发掘。

孝靜帝文武雙全,自幼聰明,能洞悉先機,且好文學,美容儀。力能挾石獅子以逾牆,射無不中。嘉辰宴會,多命群臣賦詩,從容沉雅,有孝文風。

天平初年,考虑到各地移民尚未建立起家业,孝静帝诏令政府支付一百三十万石的粟加以救助。天平三年秋,並、肆、汾、建、晉、泰、陝、東雍、南汾等州遇到旱灾,饥民开始逃荒,直到天平四年才下诏救济,死于饥荒者众多。孝静帝时期,政府在沿海地区设置煮盐灶,滄州有灶一千四百八十四,瀛州有四百五十二灶,幽州有一百八十灶,青州有五百四十六灶,邯鄲有四个灶。每年可产盐二十萬九千七百二斛四升,国家从中得益不少。

\subsubsection{天平}

\begin{longtable}{|>{\centering\scriptsize}m{2em}|>{\centering\scriptsize}m{1.3em}|>{\centering}m{8.8em}|}
  % \caption{秦王政}\
  \toprule
  \SimHei \normalsize 年数 & \SimHei \scriptsize 公元 & \SimHei 大事件 \tabularnewline
  % \midrule
  \endfirsthead
  \toprule
  \SimHei \normalsize 年数 & \SimHei \scriptsize 公元 & \SimHei 大事件 \tabularnewline
  \midrule
  \endhead
  \midrule
  元年 & 534 & \tabularnewline\hline
  二年 & 535 & \tabularnewline\hline
  三年 & 536 & \tabularnewline\hline
  四年 & 537 & \tabularnewline
  \bottomrule
\end{longtable}

\subsubsection{元象}

\begin{longtable}{|>{\centering\scriptsize}m{2em}|>{\centering\scriptsize}m{1.3em}|>{\centering}m{8.8em}|}
  % \caption{秦王政}\
  \toprule
  \SimHei \normalsize 年数 & \SimHei \scriptsize 公元 & \SimHei 大事件 \tabularnewline
  % \midrule
  \endfirsthead
  \toprule
  \SimHei \normalsize 年数 & \SimHei \scriptsize 公元 & \SimHei 大事件 \tabularnewline
  \midrule
  \endhead
  \midrule
  元年 & 538 & \tabularnewline\hline
  二年 & 539 & \tabularnewline
  \bottomrule
\end{longtable}

\subsubsection{兴和}

\begin{longtable}{|>{\centering\scriptsize}m{2em}|>{\centering\scriptsize}m{1.3em}|>{\centering}m{8.8em}|}
  % \caption{秦王政}\
  \toprule
  \SimHei \normalsize 年数 & \SimHei \scriptsize 公元 & \SimHei 大事件 \tabularnewline
  % \midrule
  \endfirsthead
  \toprule
  \SimHei \normalsize 年数 & \SimHei \scriptsize 公元 & \SimHei 大事件 \tabularnewline
  \midrule
  \endhead
  \midrule
  元年 & 539 & \tabularnewline\hline
  二年 & 540 & \tabularnewline\hline
  三年 & 541 & \tabularnewline\hline
  四年 & 542 & \tabularnewline
  \bottomrule
\end{longtable}

\subsubsection{武定}

\begin{longtable}{|>{\centering\scriptsize}m{2em}|>{\centering\scriptsize}m{1.3em}|>{\centering}m{8.8em}|}
  % \caption{秦王政}\
  \toprule
  \SimHei \normalsize 年数 & \SimHei \scriptsize 公元 & \SimHei 大事件 \tabularnewline
  % \midrule
  \endfirsthead
  \toprule
  \SimHei \normalsize 年数 & \SimHei \scriptsize 公元 & \SimHei 大事件 \tabularnewline
  \midrule
  \endhead
  \midrule
  元年 & 543 & \tabularnewline\hline
  二年 & 544 & \tabularnewline\hline
  三年 & 545 & \tabularnewline\hline
  四年 & 546 & \tabularnewline\hline
  五年 & 547 & \tabularnewline\hline
  六年 & 548 & \tabularnewline\hline
  七年 & 549 & \tabularnewline\hline
  八年 & 550 & \tabularnewline
  \bottomrule
\end{longtable}


%%% Local Variables:
%%% mode: latex
%%% TeX-engine: xetex
%%% TeX-master: "../../Main"
%%% End:



%%% Local Variables:
%%% mode: latex
%%% TeX-engine: xetex
%%% TeX-master: "../../Main"
%%% End:

%% -*- coding: utf-8 -*-
%% Time-stamp: <Chen Wang: 2019-12-23 15:28:32>


\section{西魏\tiny(535-557)}

\subsection{简介}


西魏(535年-557年)是中國魏晉南北朝時期中的北朝的一個地方政权,是由鮮卑人宇文泰擁立北魏孝文帝元宏的孫子元宝炬為帝,與高歡所掌控的東魏對立,建都長安。至557年被北周取代,總止經歷兩代三帝,享國二十二年。

在整個西魏統治時期,一直都由權臣宇文泰以霸府的形態控制著政權,在他努力下,北方經濟逐漸恢復,人民安居樂業,而且屢勝東魏。

大統元年(535年)春正月戊申,鮮卑人宇文泰擁立北魏孝文帝元宏的孫子元宝炬為帝,大赦並改元大統,追尊皇考元愉為文景皇帝,皇妣楊奧妃為皇后。春正月己酉,進丞相、略陽公宇文泰都督中外諸軍、錄尚書事、大行臺,改封安定郡公。以尚書令斛斯椿為太保。春正月乙卯,立妃乙氏為皇后,立皇子元欽為皇太子。春正月甲子,以廣陵王元欣為太傅,以儀同三司萬俟壽樂幹為司空。大統元年(535年)春二月,前南青州刺史大野拔斬兗州刺史樊子鵠,投降東魏。

宇文泰在三次戰役中大敗東魏,奠定宇文氏在關中的基礎。宇文泰任用漢人蘇綽等官員進行改革,使西魏進一步強盛。

大統十七年(551年),元宝炬駕崩,太子元欽即位,沿用西魏文帝年號。

552年,去年號,稱元年。

553年,宇文泰主動去丞相位。此时,梁元帝为与占据益州的萧纪争夺帝位,请西魏攻取益州。秋八月,大將軍尉遲迥攻克成都,平定劍南;萧纪随即为萧绎所灭。同年冬十一月,尚書元烈謀誅安定公宇文泰,反被宇文泰所處死。元欽對此事常有怨言,欲誅宇文泰,竟聯合宇文泰的女婿李基、李暉和于翼。結果被三人告密。

554年春正月,元欽被宇文泰所廢,立其弟齊王元廓。四月庚戌,宇文泰用毒酒毒死元欽。后西魏君臣又恢复鲜卑姓氏。

555年,西魏攻破南朝梁都城江陵,迫使梁元帝投降(但并没有灭亡梁朝)。

556年,宇文泰病死,由宇文泰的嫡長子宇文覺承襲為安定郡公、太師、大冢宰。

557年,宇文泰之侄宇文護迫西魏恭帝禪讓,由宇文覺即位天王,建立北周,建都長安(即今陝西西安)。

曾因接纳东魏叛将侯景而与东魏交战。但侯景与西魏皆未完全信任对方,侯景很快叛投南朝梁。

552年,柔然人被突厥土門可汗擊敗,其汗國崩潰。柔然王室由鄧叔子率領,西支柔然南逃至西魏,西魏太師宇文泰不敢收留,將此部三千餘人收捕交還突厥使者,全數斬殺於長安青門外。

%% -*- coding: utf-8 -*-
%% Time-stamp: <Chen Wang: 2019-12-23 15:29:32>

\subsection{文帝\tiny(535-551)}

\subsubsection{生平}

魏文帝元宝炬(507年-551年3月28日),河南郡洛阳县(今河南省洛阳市东)人,魏孝文帝元宏之孙,京兆王元愉第三子,母杨奥妃,南北朝時代西魏建立者。

元宝炬在魏宣武帝時因為父親獲罪受牽連,年幼的一眾兄弟皆被幽禁在宗正寺,在宣武帝死後才重獲自由,被叔叔元懌收養。武泰年间封邵县侯,530年被封為南陽王。534年跟隨堂弟魏孝武帝元修入關中投靠宇文泰。

永熙三年闰十二月癸巳(535年2月3日)宇文泰毒死魏孝武帝后,秘密不发布丧事,宇文泰与群臣商议册立新皇帝,众人大多推举魏孝武帝哥哥的儿子广平王元赞为嗣君。濮阳王元顺在正室以外的房间流下眼泪对宇文泰说:“高欢逼迫驱逐先帝,册立幼年的君主来独揽大权,您应该反其道而行之。广平王虽然与先帝亲近,但是年幼,不适合居于皇帝之位,不如册立年长的君主来奉戴。”宇文泰深深地赞同元顺的意见,因此发布国家的丧事,向太宰、南阳王元宝炬奉上帝号,改元大统,政權實際上由宇文泰控制。大统十七年三月庚戌(551年3月28日),元宝炬去世,虚岁四十五,四月庚辰(551年4月27日)葬于永陵。嫡長子皇太子元钦嗣位。

\subsubsection{大统}

\begin{longtable}{|>{\centering\scriptsize}m{2em}|>{\centering\scriptsize}m{1.3em}|>{\centering}m{8.8em}|}
  % \caption{秦王政}\
  \toprule
  \SimHei \normalsize 年数 & \SimHei \scriptsize 公元 & \SimHei 大事件 \tabularnewline
  % \midrule
  \endfirsthead
  \toprule
  \SimHei \normalsize 年数 & \SimHei \scriptsize 公元 & \SimHei 大事件 \tabularnewline
  \midrule
  \endhead
  \midrule
  元年 & 535 & \tabularnewline\hline
  二年 & 536 & \tabularnewline\hline
  三年 & 537 & \tabularnewline\hline
  四年 & 538 & \tabularnewline\hline
  五年 & 539 & \tabularnewline\hline
  六年 & 540 & \tabularnewline\hline
  七年 & 541 & \tabularnewline\hline
  八年 & 512 & \tabularnewline\hline
  九年 & 513 & \tabularnewline\hline
  十年 & 544 & \tabularnewline\hline
  十一年 & 545 & \tabularnewline\hline
  十二年 & 546 & \tabularnewline\hline
  十三年 & 547 & \tabularnewline\hline
  十四年 & 548 & \tabularnewline\hline
  十五年 & 549 & \tabularnewline\hline
  十六年 & 550 & \tabularnewline\hline
  十七年 & 551 & \tabularnewline
  \bottomrule
\end{longtable}


%%% Local Variables:
%%% mode: latex
%%% TeX-engine: xetex
%%% TeX-master: "../../Main"
%%% End:

%% -*- coding: utf-8 -*-
%% Time-stamp: <Chen Wang: 2019-12-23 15:31:38>

\subsection{废帝\tiny(551-554)}

\subsubsection{生平}

元钦(525年-554年),河南郡洛阳县(今河南省洛阳市东)人,西魏文帝元宝炬长子、母為乙弗皇后。

大統元年(535年),元寶炬登基建立西魏時,元钦被立為皇太子。不久丞相宇文泰主動許配女兒給元欽,成為元欽的岳父。大統十七年(551年),文帝駕崩,太子元欽即位,沿用文帝年号,次年(552年)去年号,称元年。次年(553年),宇文泰主動去丞相位。同年十一月,尚書元烈謀誅安定公宇文泰,反被宇文泰所處死。此後,元欽對此事常有怨言,欲誅宇文泰,竟聯合宇文泰的女婿李基、李暉和于翼。結果被三人告密。登位第三年(554年)二月被宇文泰废黜,安置雍州。宇文泰立其庶弟齊王元廓。元钦史稱「廢帝」。恭帝元年四月庚戌(554年),宇文泰用毒酒毒死了元钦。

\subsubsection{无年号}

\begin{longtable}{|>{\centering\scriptsize}m{2em}|>{\centering\scriptsize}m{1.3em}|>{\centering}m{8.8em}|}
  % \caption{秦王政}\
  \toprule
  \SimHei \normalsize 年数 & \SimHei \scriptsize 公元 & \SimHei 大事件 \tabularnewline
  % \midrule
  \endfirsthead
  \toprule
  \SimHei \normalsize 年数 & \SimHei \scriptsize 公元 & \SimHei 大事件 \tabularnewline
  \midrule
  \endhead
  \midrule
  元年 & 551 & \tabularnewline\hline
  二年 & 552 & \tabularnewline\hline
  三年 & 553 & \tabularnewline\hline
  四年 & 554 & \tabularnewline
  \bottomrule
\end{longtable}


%%% Local Variables:
%%% mode: latex
%%% TeX-engine: xetex
%%% TeX-master: "../../Main"
%%% End:

%% -*- coding: utf-8 -*-
%% Time-stamp: <Chen Wang: 2019-12-23 15:32:24>

\subsection{恭帝\tiny(554-557)}

\subsubsection{生平}

魏恭帝拓跋廓(537年-557年),原姓元,河南郡洛阳县(今河南省洛阳市东)人,西魏文帝元宝炬四子,南北朝时期西魏的皇帝。

554年即位,去年号称元年,并且復姓拓跋。557年被迫禅位于宇文觉,西魏灭亡。恭帝被封为宋公,住在宇文觉的堂兄大司马宇文护府上,不久被杀。

\subsubsection{无年号}


\begin{longtable}{|>{\centering\scriptsize}m{2em}|>{\centering\scriptsize}m{1.3em}|>{\centering}m{8.8em}|}
  % \caption{秦王政}\
  \toprule
  \SimHei \normalsize 年数 & \SimHei \scriptsize 公元 & \SimHei 大事件 \tabularnewline
  % \midrule
  \endfirsthead
  \toprule
  \SimHei \normalsize 年数 & \SimHei \scriptsize 公元 & \SimHei 大事件 \tabularnewline
  \midrule
  \endhead
  \midrule
  元年 & 554 & \tabularnewline\hline
  二年 & 555 & \tabularnewline\hline
  三年 & 556 & \tabularnewline\hline
  四年 & 557 & \tabularnewline
  \bottomrule
\end{longtable}


%%% Local Variables:
%%% mode: latex
%%% TeX-engine: xetex
%%% TeX-master: "../../Main"
%%% End:



%%% Local Variables:
%%% mode: latex
%%% TeX-engine: xetex
%%% TeX-master: "../../Main"
%%% End:

%% -*- coding: utf-8 -*-
%% Time-stamp: <Chen Wang: 2019-12-23 15:43:45>


\section{北齐\tiny(550-577)}

\subsection{简介}

北齊(550年—577年)是中国北朝之鲜卑化汉人政权。550年6月9日(庚午年五月戊午日),由文宣帝高洋取代东魏建立,建國號齊,建元天保,遷都鄴城,以晉陽為別都。史稱北齊或後齊,以別於南齊。以皇室姓高,又稱高齊。北齊歷經文宣帝高洋、廢帝高殷、孝昭帝高演、武成帝高湛、后主高緯、幼主高恆六帝,577年被北周消滅,共享國二十八年。

北齊國勢本來頗為強盛,但由於北齊帝王多為殘暴昏庸之主,導致政治情勢混亂,國勢也日漸衰落。

后主時期,北周在北周武帝的统治下日渐兴盛,而北齐则衰落,更枉杀大将斛律光、高长恭。577年北周統一北方,北齊滅亡。北齊滅亡後,境内的士族大多遷到關中,成为北周臣民。范阳王高绍义逃奔突厥投靠他钵可汗。北齐营州刺史高宝宁不降周,奉高绍义为主继续抵抗。后来北周与突厥关系改善,580年,高绍义遭他钵可汗出卖被交给北周。

581年北周外戚楊堅篡位,建國號隋,583年消灭高宝宁势力,589年南下滅陳,結束中原自魏晉南北朝長達四百年的分裂局面。

北齐继承了东魏所控制的地区,占有今黄河下游流域的河北、河南、山东、山西以及苏北、皖北的广阔地区。同时与其并存的王朝有西魏、北周(取代西魏)、梁(含西梁、东梁)、陈(取代梁,但只占有前者部分领土)等。

北齐天保三年(552年)以后,北击库莫奚、东北逐契丹、西北破柔然,西平山胡(属匈奴族),南取淮南,势力一直延伸到长江边,这时北齐的国力达到鼎盛。北齐的农业、盐铁业、瓷器制造业都相当发达,是和与其鼎立的陈、北周三个国家中最富庶的。北齐继续推行均田制,大体上与北魏相同,但也略有变化。例如,北齐取消了受倍田的规定,不过一夫一妇的实际受田数仍相当于倍田,北魏对奴婢受田没有限制。北齐则按官品限制在300人至600人之间。另外还规定了赋税。

此外,魏收於此時編寫了《魏書》。

东魏和北齐初创之际,兵制继承北魏,兵民分离,鲜卑人为兵。在齐文宣帝时改革,军人出现汉人勇夫,但没有改变兵民、汉胡之分。

后在河清三年(564年),出现一种新的兵制,将当兵与种田结合起来,成为隋文帝改革府兵制的模板。

北齐人口盛时约2200万。

佛教及印度、中亞、西亞文化在本時期持續對藝術產生重大影響。部分中國史上最精緻的佛像座落於北齊的佛寺洞窟寺,這些佛像說明當時製作佛雕的工藝,以及北魏以來藝術風格的快速進展。一些大型陶雕源自北齊。北齊的陶器的特色包括雙色以上的釉色,白胎陶器亦於此時期發展。此時期繪畫品質極高,由太原的婁叡墓壁畫可見一斑。

北齐首都邺城繁华昌盛,布局有致,邺城之盛就在北齐时期。

%% -*- coding: utf-8 -*-
%% Time-stamp: <Chen Wang: 2019-12-23 15:42:57>

\subsection{文宣帝\tiny(550-559)}

\subsubsection{生平}

齐文宣帝高洋(526年-559年,在位550年—559年),字子進,鮮卑名侯尼干,勃海郡蓨县(今河北省衡水市景县)人。因其生于晋阳,又名晋阳乐,南北朝时期北齐开国皇帝,在位10年。他是东魏权臣高欢次子,北齐追尊文襄皇帝高澄的同母弟,鮮卑化漢人。

幼時其貌不揚,沉默寡言,其實大智若愚,聰慧過人,雖偶然被兄弟嘲笑或玩弄,但其才能甚得父親高欢欣賞。高澄被蘭京刺殺以後,高洋便牢牢地掌握了大權。東魏孝靜帝元善見只好封他為丞相、齐郡王,加九锡、殊礼。高洋不甘當傀儡皇帝的大臣,就於550年就廢掉了元善見,自立為帝,改元「天保」,建都鄴,北齊建立,年僅25歲。當年十一月,西魏宇文泰率大軍進攻剛剛建立的北齊,高洋親自率軍迎戰。宇文泰看到高洋手下的部隊軍容嚴整,嘆息道:「高歡不死矣。」隨即退軍。

在位初年,留心政務,削減州郡,整頓吏治,訓練軍隊,加強兵防,使北齊在很短的時間內強盛起來。高洋出兵進攻柔然、契丹、高句麗等國,都大獲全勝。高洋亦曾趁南朝梁遭遇侯景之乱意图拥立梁宗室萧退为梁帝,遭梁将王僧辩、陈霸先抵抗而未果,后又拥立萧渊明为梁帝并迫使王僧辩接受,但陈霸先袭杀王僧辩,废黜萧渊明,后即代梁建陈。

同時,北齊的農業、鹽鐵業、瓷器製造業都相當發達,是同陳、西魏鼎立的三個國家中最富庶的。

可是,他在即位六、七年後就腐敗起來,整日不理朝政,沉湎於酒色之中,他在都城鄴(今河南安陽)修築三台宮殿,十分豪華,動用十萬民夫,簡直是奢侈至極。

幾代北齊皇帝幾乎都有精神問題,加上高洋為人殘忍嗜殺,酗酒之後更常失去理智。

其實高洋自己也清楚酒後的荒唐行為,但無法改正。腐化的生活縮短了高洋的壽命。

天保五年(554年)八月,高洋任命尉粲為司徒、侯莫陳相為司空、清河王高岳為太保(上三公)、平陽王高淹為錄尚書事、常山王高演為尚書令、上黨王高渙為左僕射。

天保六年(555年)三月十六日,齐文宣帝高洋返回都城鄴城,封高孝珩為廣寧王、高延宗為安德王,此兩人都是哥哥高澄的兒子。

天保十年(559年),高洋驾崩(最长寿的北齐皇帝),虚龄34歲,葬於武寧陵,謚號為文宣皇帝,廟號為顯祖。后主天统初年(565年),有诏改谥景烈皇帝,庙号威宗。武平初年(570年),改回原来谥号。

高洋死後,北齊統治階級內部愈來愈混亂,最終為北周所滅。

興建高台時,曾單獨爬上最高處,居民看到紛紛膽跳心驚。並時常在街道裸露身體,儘管當時季節正處寒冬。

高洋有次喝醉酒,一氣之下說要將母親婁太后嫁給北方蠻族,母親氣着說自己怎會生出禽獸不如的兒子,高洋略為清醒,想逗母親開心,沒想到將母親摔傷。完全酒醒後,發現自己鑄成大錯,於是痛鞭自己,下決心戒酒,但是最後仍無法戒掉。

高洋曾經非常宠爱一名原為歌妓的薛嬪,容貌倾国,姿色万千,高洋和她如胶似漆、整日厮守在一起,但後來怀疑薛嬪曾与清河王高岳私通,有过暧昧关系,妒火中燒,命高岳自殺。薛嬪当时怀孕,分娩后,抽出匕首把薛嫔杀了,薛嬪遭斬殺肢解,並將頭顱置於自己衣袖裡面,回宮大宴賓客時,突然將人頭丟出,嚇的賓客四散,自己則取出薛嬪的大腿骨(髀骨)當作琵琶,邊流淚邊吟唱:“佳人難再得!”薛嫔出葬时,高洋披头散发,在车后步行跟随,大声哭号。

元魏宗室元韶,因娶高洋长姊某公主(即北魏永熙皇后),是高洋姐夫,有次高洋前去並詢問他:「為何漢朝可以中興?」元韶表示因為新朝沒把漢朝劉姓宗室殺光,於是高洋下令诛杀元魏宗室始平公元世道、东平公元景式等二十五家,其余十九家被囚禁,在东市斩杀七百二十一人,与其余所杀三千人一起投尸到漳水,元韶被囚后餓死。

高洋在位后期残暴不仁,荒淫无道,经常随意出入朝中官员府第,看见长得漂亮的女人就会色心大起,不分贵贱还是人妻,接着就是霸王硬上弓。除此之外,高洋还有观淫癖,征集坊间美女大批,弄入宫中后,然后脱个精光,命令侍从和卫士与这些女人群交,朝夕临视为乐。

\subsubsection{天保}

\begin{longtable}{|>{\centering\scriptsize}m{2em}|>{\centering\scriptsize}m{1.3em}|>{\centering}m{8.8em}|}
  % \caption{秦王政}\
  \toprule
  \SimHei \normalsize 年数 & \SimHei \scriptsize 公元 & \SimHei 大事件 \tabularnewline
  % \midrule
  \endfirsthead
  \toprule
  \SimHei \normalsize 年数 & \SimHei \scriptsize 公元 & \SimHei 大事件 \tabularnewline
  \midrule
  \endhead
  \midrule
  元年 & 550 & \tabularnewline\hline
  二年 & 551 & \tabularnewline\hline
  三年 & 552 & \tabularnewline\hline
  四年 & 553 & \tabularnewline\hline
  五年 & 554 & \tabularnewline\hline
  六年 & 555 & \tabularnewline\hline
  七年 & 556 & \tabularnewline\hline
  八年 & 557 & \tabularnewline\hline
  九年 & 558 & \tabularnewline\hline
  十年 & 559 & \tabularnewline
  \bottomrule
\end{longtable}


%%% Local Variables:
%%% mode: latex
%%% TeX-engine: xetex
%%% TeX-master: "../../Main"
%%% End:

%% -*- coding: utf-8 -*-
%% Time-stamp: <Chen Wang: 2019-12-23 15:44:05>

\subsection{废帝\tiny(559-560)}

\subsubsection{生平}

高殷(545年-561年;在位559年—560年),字正道,北齊第二代皇帝,齊文宣帝嫡長子,母亲是昭信皇后李祖娥。

天保元年(550年),立為皇太子,時年六歲。性敏慧,但偏于柔懦,高洋对此不满,曾打算立次子太原王高紹德。

天保十年(559年),高殷即位,時年十五歲。即位後,以咸陽王斛律金為左丞相,叔父錄尚書事、常山王高演為太傅,叔父司徒、長廣王高湛為太尉,司空段韶為司徒,平陽王高淹為司空,高陽王高湜為尚書左僕射,河間王高孝琬為司州牧,侍中燕子獻為右僕射共同执政,勵精圖治,對民生極為關心,曾分命使者巡省四方,求政得失,省察風俗,問人疾苦;整頓吏治,政治清明;武官年逾六旬皆放免,軍事上淘汰老弱,留下精壯,軍力大增;下詔減徭役,使由天保朝國勢的危急有紓緩。然高演在位高权重兢兢业业之余也开始覬覦皇位,且引起了朝中反对派的不满。終於560年,太后李祖娥等人与高演等人的矛盾白热化,其六叔高演發動政變,高殷被废为濟南王。

高殷被废的时候,其祖母娄昭君命高演发誓决不伤害高殷性命,但最终高演还是虑有后患,于次年将高殷秘密殺害。高殷死時十七歲,谥号「愍悼」,没有关于子女的记载。

\subsubsection{乾明}

\begin{longtable}{|>{\centering\scriptsize}m{2em}|>{\centering\scriptsize}m{1.3em}|>{\centering}m{8.8em}|}
  % \caption{秦王政}\
  \toprule
  \SimHei \normalsize 年数 & \SimHei \scriptsize 公元 & \SimHei 大事件 \tabularnewline
  % \midrule
  \endfirsthead
  \toprule
  \SimHei \normalsize 年数 & \SimHei \scriptsize 公元 & \SimHei 大事件 \tabularnewline
  \midrule
  \endhead
  \midrule
  元年 & 560 & \tabularnewline
  \bottomrule
\end{longtable}


%%% Local Variables:
%%% mode: latex
%%% TeX-engine: xetex
%%% TeX-master: "../../Main"
%%% End:

%% -*- coding: utf-8 -*-
%% Time-stamp: <Chen Wang: 2021-11-01 15:13:58>

\subsection{孝昭帝高演\tiny(560-561)}

\subsubsection{生平}

齊孝昭帝高演(535年-561年;在位560年—561年),字延安,北齐第三任皇帝。他是東魏权臣高欢第六子,文宣帝同母弟,在位一年。

高演長於政術,善於理解事情的細節;天保朝起開始干預朝政,政治經驗逐漸成熟豐富,眼見次兄齊文宣帝沉湎酒色,大臣趨炎附勢,惟高演滿臉憂愁,不時直諫。其兄文宣帝臨終時,表示必要时皇位可以相让,唯不可伤害高殷。廢帝即位,獨攬朝政。560年,高演發動政變,废高殷為濟南王。高演登上皇帝寶座,改元皇建,時年二十六歲。

高演在位期間,文治武功兼盛,『帝留心於政事,積極尋求及任用賢能為朝廷效力,政治清明;帝關心民生,輕徭薄賦,並下詔分遣大使巡省四方,觀察風俗,問人疾苦,考求得失。並親征親戎北討庫莫奚,出長城,虜奔遁,分兵致討,大獲牛馬。』在北齊28年歷史和六帝之中,只有孝昭帝稱得上是明君,可惜他在位時間不長,即位翌年,高演便因墮马事故重伤而死,在位僅兩年,終年僅27歲。

高殷被废的时候,娄昭君命儿子高演发誓决不伤害孙子高殷性命,但最终高演还是虑有后患,于次年将高殷秘密殺害。不久高演即出了意外,传说是齊文宣帝的厉鬼复仇。娄昭君亦对此深感悲愤,不肯原谅高演。為了保住兒子高百年,临终时候高演宣布废掉年幼的太子,傳位於弟弟長廣王高湛。他的谥号为孝昭皇帝,廟號肃宗。

\subsubsection{皇建}

\begin{longtable}{|>{\centering\scriptsize}m{2em}|>{\centering\scriptsize}m{1.3em}|>{\centering}m{8.8em}|}
  % \caption{秦王政}\
  \toprule
  \SimHei \normalsize 年数 & \SimHei \scriptsize 公元 & \SimHei 大事件 \tabularnewline
  % \midrule
  \endfirsthead
  \toprule
  \SimHei \normalsize 年数 & \SimHei \scriptsize 公元 & \SimHei 大事件 \tabularnewline
  \midrule
  \endhead
  \midrule
  元年 & 560 & \tabularnewline\hline
  二年 & 561 & \tabularnewline
  \bottomrule
\end{longtable}


%%% Local Variables:
%%% mode: latex
%%% TeX-engine: xetex
%%% TeX-master: "../../Main"
%%% End:

%% -*- coding: utf-8 -*-
%% Time-stamp: <Chen Wang: 2019-12-23 15:45:27>

\subsection{武成帝\tiny(561-565)}

\subsubsection{生平}

齊武成帝高湛(537年-569年1月13日),小字步落稽,北齐第四位皇帝,561-565年在位,在位四年。東魏權臣高歡第九子,孝昭帝高演之同母弟。

高湛幼時亦其得父親高歡喜愛。北齊建國後,被齊文宣帝封為長廣王。高湛協助其兄高演發動政變廢黜了侄兒高殷。齊孝昭帝高演繼位後,甚為寵信他。後高演傷病兼身,臨終時為了不讓自己的兒子高百年落得與高殷一樣的命運,決定傳位於弟。561年,高湛繼位,改元太寧,是為武成帝。

武成帝昏庸无能,沉湎于美色之中,不思國事,北齊岌岌可危。565年,傳位於太子高緯,自任太上皇,繼續在幕後主政。最后也因为酒色过度而死,年僅三十二歲。年號太寧、河清,谥号武成帝,庙号世祖,葬於永平陵。

高洋皇后李祖娥其子高殷即位,但只有一年便被其叔高演所篡。高演即位為孝昭帝後,將她由皇太后降為昭信皇后,居於昭信宮。

後來高湛繼位為武成帝後,逼李皇后與之相姦。高湛恐嚇她:「如果妳敢不從,我就殺妳兒子。」李皇后因害怕而答應他,從此頗受寵愛。她懷孕的時候,兒子太原王高紹德到她的宮殿,她避不見面,高紹德便怒言:「妳當做兒子的不知道嗎?您是因為肚子大了,所以才不見兒子吧。」李皇后羞愧,等到生下一個女兒,含羞將其掐死。

高湛見女兒被害,怒不可遏,將高紹德捉到宮里,舉刀怒喝:「妳殺我的女兒,我就殺妳的兒子!」高紹德驚慌求饒,高湛又罵高紹德:「想當年我被你父親毒打,你也沒來救過我!」當場將高紹德殺死。李皇后當場大哭起來,高湛更是憤怒,將她衣服脫光,胡亂鞭打,讓她哭天喊地不已。最後將她裝在絹袋裡,也不管她鮮血淋漓,就丟到渠道裡,任水漂流,許久才甦醒。用牛車送到妙勝寺出家為尼。北齊滅亡後入關內,隋朝時才得以送還故鄉。

齐武成帝高湛在位期间昏庸无道,荒淫无度,终日沉迷于女色,再加上好大喜功,是导致北齐逐渐走入灭亡的主因。

\subsubsection{太宁}

\begin{longtable}{|>{\centering\scriptsize}m{2em}|>{\centering\scriptsize}m{1.3em}|>{\centering}m{8.8em}|}
  % \caption{秦王政}\
  \toprule
  \SimHei \normalsize 年数 & \SimHei \scriptsize 公元 & \SimHei 大事件 \tabularnewline
  % \midrule
  \endfirsthead
  \toprule
  \SimHei \normalsize 年数 & \SimHei \scriptsize 公元 & \SimHei 大事件 \tabularnewline
  \midrule
  \endhead
  \midrule
  元年 & 561 & \tabularnewline\hline
  二年 & 562 & \tabularnewline
  \bottomrule
\end{longtable}

\subsubsection{河清}

\begin{longtable}{|>{\centering\scriptsize}m{2em}|>{\centering\scriptsize}m{1.3em}|>{\centering}m{8.8em}|}
  % \caption{秦王政}\
  \toprule
  \SimHei \normalsize 年数 & \SimHei \scriptsize 公元 & \SimHei 大事件 \tabularnewline
  % \midrule
  \endfirsthead
  \toprule
  \SimHei \normalsize 年数 & \SimHei \scriptsize 公元 & \SimHei 大事件 \tabularnewline
  \midrule
  \endhead
  \midrule
  元年 & 562 & \tabularnewline\hline
  二年 & 563 & \tabularnewline\hline
  三年 & 564 & \tabularnewline\hline
  四年 & 565 & \tabularnewline
  \bottomrule
\end{longtable}


%%% Local Variables:
%%% mode: latex
%%% TeX-engine: xetex
%%% TeX-master: "../../Main"
%%% End:

%% -*- coding: utf-8 -*-
%% Time-stamp: <Chen Wang: 2019-12-23 15:45:56>

\subsection{高纬\tiny(565-576)}

\subsubsection{生平}

高纬(556年5月29日-577年11月),字仁纲,南北朝时期北齐第五位皇帝(565年-577年在位),史稱「後主」,北齐武成帝高湛的嫡長子,亦是中國唯一一位無上皇。

高纬与庶兄高绰同日出生,实为高湛次子,但是因为是嫡出,故被视为长子。

高纬即位时,腐朽的北齐政权已经摇摇欲坠,他自己仍然荒淫无道,杀害兄长高绰,导致北齐军队衰弱,政治腐败,尤其最大致命伤是诛杀名将高長恭、斛律光,这使得北齐失去得以抗击北周侵略的有能将领。

577年,北周来攻打北齐,占领晋阳,齐军大败,周军不久破北齐京师邺(今河北临漳),高纬慌忙将皇位传于自己8岁的儿子高恒,然后带着幼主高恒等十余人骑马准备投降江南的陈朝。他们刚逃到青州(今山东青州)就被周军俘虏了,北齐灭亡。高纬投降后,被周武帝封温国公,不久因为被诬陷谋反,而被武帝赐死,终年22岁(《北齊書》将此事记于下年)。

讽刺的是,高纬被俘后,竟对北周武帝宇文邕要求將馮小憐归还给他。周帝說:「朕對於天下,就像脫掉鞋子一樣輕視,一個老太婆有甚麼好跟您爭的呢?」後主寵愛馮小憐,李商隱曾寫詩諷刺道:

「小憐玉體橫陳夜,已報周師入晉陽」—北齊兩首(其一)第二聯

「晋阳已陷休回顾,更请君王猎一围」—北齊兩首(其二)第二聯

这两首诗說明後主在北周入侵時仍然不理政事,荒唐、淫亂。

\subsubsection{天统}

\begin{longtable}{|>{\centering\scriptsize}m{2em}|>{\centering\scriptsize}m{1.3em}|>{\centering}m{8.8em}|}
  % \caption{秦王政}\
  \toprule
  \SimHei \normalsize 年数 & \SimHei \scriptsize 公元 & \SimHei 大事件 \tabularnewline
  % \midrule
  \endfirsthead
  \toprule
  \SimHei \normalsize 年数 & \SimHei \scriptsize 公元 & \SimHei 大事件 \tabularnewline
  \midrule
  \endhead
  \midrule
  元年 & 565 & \tabularnewline\hline
  二年 & 566 & \tabularnewline\hline
  三年 & 567 & \tabularnewline\hline
  四年 & 568 & \tabularnewline\hline
  五年 & 569 & \tabularnewline
  \bottomrule
\end{longtable}

\subsubsection{武平}

\begin{longtable}{|>{\centering\scriptsize}m{2em}|>{\centering\scriptsize}m{1.3em}|>{\centering}m{8.8em}|}
  % \caption{秦王政}\
  \toprule
  \SimHei \normalsize 年数 & \SimHei \scriptsize 公元 & \SimHei 大事件 \tabularnewline
  % \midrule
  \endfirsthead
  \toprule
  \SimHei \normalsize 年数 & \SimHei \scriptsize 公元 & \SimHei 大事件 \tabularnewline
  \midrule
  \endhead
  \midrule
  元年 & 570 & \tabularnewline\hline
  二年 & 571 & \tabularnewline\hline
  三年 & 572 & \tabularnewline\hline
  四年 & 573 & \tabularnewline\hline
  五年 & 574 & \tabularnewline\hline
  六年 & 575 & \tabularnewline\hline
  七年 & 576 & \tabularnewline
  \bottomrule
\end{longtable}

\subsubsection{隆化}

\begin{longtable}{|>{\centering\scriptsize}m{2em}|>{\centering\scriptsize}m{1.3em}|>{\centering}m{8.8em}|}
  % \caption{秦王政}\
  \toprule
  \SimHei \normalsize 年数 & \SimHei \scriptsize 公元 & \SimHei 大事件 \tabularnewline
  % \midrule
  \endfirsthead
  \toprule
  \SimHei \normalsize 年数 & \SimHei \scriptsize 公元 & \SimHei 大事件 \tabularnewline
  \midrule
  \endhead
  \midrule
  元年 & 576 & \tabularnewline
  \bottomrule
\end{longtable}


%%% Local Variables:
%%% mode: latex
%%% TeX-engine: xetex
%%% TeX-master: "../../Main"
%%% End:

%% -*- coding: utf-8 -*-
%% Time-stamp: <Chen Wang: 2021-11-01 15:14:34>

\subsection{安德王高延宗\tiny(576)}

\subsubsection{生平}

高延宗(544年-577年),勃海郡蓨县(今河北省衡水市景县)人,追尊齐文襄帝高澄第五子,母亲为姬妾陈氏,原本是北魏广阳王的家妓。

高延宗幼时,父亲高澄在547年被杀,就被他二叔文宣帝高洋抚养,宠得不成样子,高延宗十二岁,高洋还让他骑在自己的肚腹上,在肚脐里撒尿。抱着他说:“可怜止有此一个。”高洋问高延宗想做什么王,高延宗:“想作冲天王。”高洋于是问杨愔,杨愔答道:“天下无此郡名,愿使安于德。”于是封为安德王。高延宗骄傲恣睢,时常以肮脏花样折腾臣下,及至高洋去世,继位的叔叔高演看不过,命人捉高延宗来打了一百棍,自此高延宗稍有收敛。

武成帝高湛杀河间王高孝琬,高延宗痛哭不已。576年,后主高纬南逃,高延宗留守晋阳,被部下拥立为帝,改元德昌,军心登时大振,反败为胜,几乎将北周武帝宇文邕活捉。但仅两天就因得胜后懈怠防备被卷土重来的周军打败,周军攻克东门、南门,高延宗战不利,出逃到城北民宅被追上俘获。宇文邕亲自下马抓住高延宗的手,高延宗拒绝:“我是死人,我的手怎能接触至尊!”宇文邕说:“我们是两国天子(即承认了高延宗的北齐皇帝身份),彼此之间并非有仇怨,都是老百姓而来的,我终究不会加害于您,不必害怕。”让他重新穿戴好衣帽,以礼相待。后来宇文邕又问高延宗如何夺取邺城,高延宗推辞说这不是亡国之臣所能回答的,宇文邕强迫他回答,他才说:“如果任城王高湝援救邺城,臣下不知北齐能否坚持。如果是后主高纬自己守卫邺城,那么陛下可以兵不血刃就取得胜利。”

577年周武帝带高纬等回长安,与北齐君臣一起饮酒,席间让高纬跳舞,高延宗哭得不能自己,屡次想服毒自杀,被婢女劝止。同年,周武帝称高纬勾结穆提婆谋反,赐高纬及大量北齐皇室成员自尽。高氏皇族多自陈无辜,只有高延宗捋起衣袖哭着不说话。高延宗被以椒塞口而死。次年,妻李氏收葬了他的尸体。

\subsubsection{德昌}

\begin{longtable}{|>{\centering\scriptsize}m{2em}|>{\centering\scriptsize}m{1.3em}|>{\centering}m{8.8em}|}
  % \caption{秦王政}\
  \toprule
  \SimHei \normalsize 年数 & \SimHei \scriptsize 公元 & \SimHei 大事件 \tabularnewline
  % \midrule
  \endfirsthead
  \toprule
  \SimHei \normalsize 年数 & \SimHei \scriptsize 公元 & \SimHei 大事件 \tabularnewline
  \midrule
  \endhead
  \midrule
  元年 & 576 & \tabularnewline
  \bottomrule
\end{longtable}


%%% Local Variables:
%%% mode: latex
%%% TeX-engine: xetex
%%% TeX-master: "../../Main"
%%% End:

%% -*- coding: utf-8 -*-
%% Time-stamp: <Chen Wang: 2019-12-23 15:47:26>

\subsection{高桓\tiny(577)}

\subsubsection{生平}

高恒(570年-570年代577或578年)北齐最后一位皇帝,高纬兒子,母親穆黃花,後母馮小憐,史稱「幼主」。

当时北周不断进攻腐朽的北齐,齐军屡战屡败。577年正月一日,高纬禅位于自己的儿子高恒,改元“承光”,是为北齐幼主。正月廿一日,太上皇高緯再命幼主讓位給任城王高湝,他成為守國天王(或为宗国天王),但讓位的詔書未達高湝處,北齐京师邺(今河北临漳)已經沦陷,太上皇高緯與左皇后馮小憐逃離,幼主等10餘人骑马欲逃往南方的陈朝,但是刚刚走到青州(今山东青州)便被周军俘虏。

建德七年(578年)七月初二,他因被誣陷與宜州刺史穆提婆謀反而於八月被杀,得年8岁(《资治通鉴》将此事记于上年)。

\subsubsection{承光}

\begin{longtable}{|>{\centering\scriptsize}m{2em}|>{\centering\scriptsize}m{1.3em}|>{\centering}m{8.8em}|}
  % \caption{秦王政}\
  \toprule
  \SimHei \normalsize 年数 & \SimHei \scriptsize 公元 & \SimHei 大事件 \tabularnewline
  % \midrule
  \endfirsthead
  \toprule
  \SimHei \normalsize 年数 & \SimHei \scriptsize 公元 & \SimHei 大事件 \tabularnewline
  \midrule
  \endhead
  \midrule
  元年 & 577 & \tabularnewline
  \bottomrule
\end{longtable}


%%% Local Variables:
%%% mode: latex
%%% TeX-engine: xetex
%%% TeX-master: "../../Main"
%%% End:


\subsection{献武帝简介}

高歡(496年-547年2月25日),勃海郡蓨县人(今河北衡水市景县),鲜卑化汉族,鲜卑名賀六渾,为北魏、東魏權臣,也是北齐政权的奠基者。追尊為「神武帝」。

高欢六世祖高隱,西晋玄菟太守。高隐之子高庆,高庆之子高泰,高泰之子高湖,三世仕慕容氏。高湖的儿子高谧被流放懷朔鎮,后世居于此,家族被鮮卑化。高欢雖自称為漢人,但据史载“累世北边,故习其俗,遵同鲜卑”。

《北齐书·神武上》记载他“目有精光,長頭高顴,齒白如玉,少有人傑表。深沉有大度,輕財重士,廣結士人,為豪俠所宗。”高欢的母亲韩期姬是高樹生的正室,在他出生后不久即去世,高樹生将他交给高欢姐姐高娄斤和姐夫尉景抚养长大。在六镇起义爆发后,先后投靠杜洛周、葛荣,后来投奔爾朱荣。他向爾朱榮提出讨伐胡太后亲信郑俨、徐纥而清君侧,受爾朱榮赏识。在河阴之变后,爾朱榮掌握朝政,高欢被封为晋州刺史。

後來北魏孝莊帝殺死爾朱榮,爾朱家族起兵讨伐孝莊帝,孝莊帝战败被杀。爾朱家族立长广王元晔为帝。高歡却没有参与这次行动。后来他设法说服爾朱兆派他统帅镇压六镇之乱得到的降兵,并带领他们前往河北。

爾朱家族残暴不仁,高欢遂产生讨伐爾朱家族的想法。在此期间,爾朱兆听从慕容绍宗的建议,企图一举把高欢解决。但高欢深藏不露,使得爾朱兆与他结为兄弟,不设防备。爾朱度律废元晔,立节闵帝,封高欢为渤海王,并征其入朝。高欢清楚其中有诈,拒不接受。不久之后,高欢在信都起兵,立元朗为帝,正式讨伐爾朱氏。经过一年的战斗,高欢击败了爾朱兆、爾朱世隆、爾朱彥伯、爾朱天光、爾朱度律、爾朱仲远等人,掌握了政权。慕容绍宗归降,被高欢重用。

高欢以元朗世系疏远不是皇帝之选,有心另立皇帝。他最初有意奉戴节闵帝,派仆射魏兰根观察节闵帝为人。但节闵帝神采高明,魏兰根怕日后难制,于是与高乾兄弟及黄门侍郎崔㥄以节闵帝系尔朱氏所立,一旦奉戴则当初起兵无名为由,说服高欢废帝。高欢又因汝南王元悦是北魏孝文帝子,认为他可以继位,于是告知元悦自己有意拥立,但元悦性行轻狂,举止多有过失,高欢於是也放弃了拥立他的打算,废节闵帝和元朗,立孝文帝之孫元修為北魏孝武帝,而将节闵帝囚禁于佛寺。高歡被授大丞相、天柱大将军、太师、世袭定州刺史,增封并前十五万户,辞天柱大将军,减户五万。高欢独揽大权,使孝武帝非常不满。孝武帝联合贺拔岳试图牵制高欢的势力。高欢亲信司空高乾密奏高欢孝武帝有二心,结果被孝武帝杀掉。高欢哭着说:“天子枉害司空!”两人关系迅速恶化(亦有说高乾代表的汉人豪族势力本非高欢嫡系,其死亦有被高欢故意出卖借刀杀人的成分)。高欢命令侯莫陈悦干掉贺拔岳,并派侯景去接收贺拔岳的部队。不料,贺拔岳的部下奉戴宇文泰为主,侯景无功而返。宇文泰用为贺拔岳报仇的名义起兵,并发檄文讨伐高欢。

孝武帝終於在534年逃往關中投靠宇文泰,而高歡另立元善見為孝靜帝,遷都鄴(今河北臨漳西南),史稱東魏,由高歡任相。当年12月,宇文泰杀孝武帝,立元宝炬为帝,定都长安,史称西魏。东西魏对峙的局面形成。

天平四年(537年)春,高欢、高昂、窦泰分三路进攻西魏。窦泰进攻潼关,宇文泰故意示弱,率精锐出潼关左面的小关,攻其不备,东魏军大败,大将窦泰自杀。高欢被迫撤军。

十月,高歡率兵二十萬至蒲津(今山西永濟縣一帶)攻打西魏,志在為竇泰復仇,高歡命令高昂領兵三萬出河南。時關中大饑,宇文泰所將不滿萬人。東魏右長史薛琡提議堅守糧道,不可渡河野戰;侯景也勸高歡分成二軍,相繼而進,但高歡不接受建議。後高歡渡河至馮翊城下,西魏華州刺史王羆有備,不可攖其鋒,乃涉洛水,軍於許原西。宇文泰至渭南,徵諸州兵馬,諸將認為眾寡不敵,請求緩進,不許。宇文泰令造浮橋於渭河,軍隊備有三日糧食,以輕騎渡渭河,至沙苑(今陝西大荔南,洛、渭之間)距東魏軍僅六十里。宇文泰採用李弼的計謀,列陣於渭曲,又命將士將武器藏在蘆葦中,候聞鼓聲而起。不久,高歡遣東魏兵至,見西魏兵少人乏,於是兵馬輕敵冒進,一時行伍亂次。宇文泰遂鳴鼓擊之,于謹等六軍與之合戰,李弼率鐵騎橫擊,東魏兵潰散敗北,喪兵七萬。這時李穆獻計:「高歡膽破矣,逐之可獲。」宇文泰不聽,還軍渭南,這時所徵諸州之兵剛到前線,宇文泰命令士兵每人種樹一株,以旌武功。李弼等十二大將,以功進爵,史稱“沙苑之戰”。

公元538年,高欢部将侯景夺回洛阳金墉城,宇文泰率军救援,一开始东魏气势如虹,宇文泰战马中箭,把宇文泰甩在地上,结果宇文泰差点被俘虏。但不久后西魏军重整旗鼓,侯景被击败,高昂率军追击宇文泰,战败被斩。此战双方打平,但高欢痛失一员大将。

公元543年,高昂的哥哥高仲密以北豫州投降西魏,高欢率十万大军讨伐,宇文泰率军救援。高欢大将彭乐以数千骑兵冲入西魏北军,取得很大胜利,高欢鸣鼓进击,斩首三万余级。高欢派彭乐追击宇文泰。宇文泰狼狈不堪,向彭乐哀求:“彭将军你太傻了!今天你杀掉我,明天你还有用吗?何不还营,把我丢下的金银宝物取走呢?”彭乐闻讯便不再追击,回去跟高欢报告:“宇文泰侥幸逃跑,已经心惊胆战!”高欢听说彭乐放走大敌,气得要命,却无可奈何。

隔日,双方重整旗鼓再战。这一次西魏占了上风,东魏战败,高欢被迫撤退。宇文泰命令贺拔胜率三千兵马追击高欢,贺拔胜的兵器几乎都击到了高欢,贺拔胜边追边喊:“贺六浑,我贺拔破胡(贺拔胜的表字)今天一定宰了你!”所幸高欢部下射死贺拔胜坐骑,这才顺利脱险。高欢回军后,下令把贺拔胜留在东魏的几个儿子统统杀掉,贺拔胜郁郁而终。

武定四年(546年),高歡率十萬大軍在玉璧(山西稷山)與宇文泰交戰,西魏守将韦孝宽积极防守,高欢无懈可击。东魏苦攻玉壁五十多天,战死病死七万多人,高歡因忧愤生病,被迫撤退。西魏造謠高歡中箭病危,高歡回師途中帶病召集群臣,請斛律金高歌〈敕勒歌〉一首:“敕勒川,陰山下,天似穹廬,籠蓋四野。天蒼蒼,野茫茫,風吹草低見牛羊。”曲中高歡親自和唱,哀慟流淚。

武定五年春正月朔(547年2月6日),发生了日食,高欢说:“日食是为了我吗,死了又有什么遗憾。”正月丙午(547年2月25日),高欢向魏孝静帝启禀陈说,当日,高欢在晋阳去世,虚岁五十二,葬于义平陵(不过据《资治通鉴》记载,义平陵是高欢的衣冠冢,实陵潜葬于鼓山石窟)。

高欢之子高洋篡魏登基后,追尊高欢为太祖獻武皇帝,後主時,又改為高祖神武皇帝。

高歡善於玩弄權術,足智多謀,精通權宜之計。从他替爾朱榮出谋划策,到后来击破掌权的爾朱家族都显示了这一点。另外,高欢临终前嘱咐儿子高澄,指出侯景必然造反,但只要用慕容绍宗为将就可讨平。结果不出高欢所料。高歡用人惟才是用,為北齊立國打下了堅固的基礎。

然而,高欢野心太大,未能处理好与孝武帝的关系,致使孝武帝出奔宇文泰,最终造成东西魏对峙之局。而且,高欢控制的东魏实力虽远强于西魏,但他在戰術不及宇文泰,导致他终其一生未能统一北方。高欢亦教子无方,他身后的北齐政权暴君和昏君輩出,朝政混乱,最终被宇文氏的北周消灭。

爾朱榮認識高歡時,對高歡能讓馬乖乖站著讓他清洗,十分驚訝,高歡表示強硬手段才是唯一方法,爾朱榮對他記憶十分深刻,開始拔擢他。後來,高歡幾個兒子有次面對一團繩索難解,其中次子高洋一刀砍斷,高歡十分高興。此為「快刀斬亂麻」一語由來。

\subsection{文襄帝简介}

高澄(521年-549年),字子惠,北魏、东魏权臣高欢之子,父亲死后,任大丞相。北齐追尊为文襄帝。

高澄是高歡与娄昭君所生的長子,生来就非常聪明,对政事有独到见解,自幼深得父亲喜爱和重用。北魏中兴元年(531年),立为渤海王世子。10岁时曾独自出马为高欢招降高敖曹。11岁时以高欢特使的身份两次去洛阳朝觐孝武帝元修。中兴二年(532年),加侍中、开府仪同三司,尚孝静帝妹冯翊公主,史书中赞叹他神情俊爽,恍若成人。天平元年(534年),加使持节、尚书令、大行台、并州刺史。天平三年(536年),入辅朝政,加领左右、京畿大都督。朝臣们虽听说高澄年轻老成,有风度、有见识,但总觉得他是个少年,心里并不服气。当看到他驾驭全局,有胆略、有气魄,在朝堂上做宰相时听断如流,处理问题及时妥切,不由得个个心悦诚服。元象元年(538年),高澄兼任吏部尚书。 兴和二年(540年),高澄加大将军,领中书监,仍代理吏部尚书。北魏从崔亮开始挑选官员就论资排辈,不按才能选取。高澄废除了这一个制度,开始根据才能名望挑选官员,亲自写书征召各地有才学有名望的士子为朝廷效力。当时品德好、有本事的人,都得到了提拔重用,有的一时安排不了相应的位置,高澄就将他们召为宾客,在自己府中供养起来,有时间便与他们一起游园娱乐赋诗,使这些人各得其所,各尽所长。

自从河阴之变后,尔朱荣为了安定朝中人心,上奏滥封官爵。赠荫一事,渐渐变得杂滥无章,平庸无能的官员动辄高官厚禄,被有识之士所非议。武定年间,高澄开始纠正其过失,使得追赠褒扬渐有章法。高澄推荐铁面无私的崔暹为御史中尉,严厉打击那些无法无天的贪官污吏,尤其是窃据高位的权贵,有许多人被绳之以法。官场风气大有改观,人心为之一振。兴和三年(541年),高澄在麟趾阁和群臣编纂议定了律法《麟趾格》,并颁布天下。《麟趾格》是《北齐律》的蓝本,又是隋唐律法的直接渊源,影响一直波及后世。

在高澄的主持下,朝廷将治国的政策书于榜上,公开张贴在街头,供天下百姓自由评论,发表意见。对那些提出建议或批评时事的人,都给予优厚的待遇,即使言过其实或言辞激烈,也予以宽容,不加罪责。由于百姓的称赞,高澄的威望更加上升。在这段时期内,东魏与南方的梁朝关系比较和睦,双方的使节往来频繁。然而,为了显示各自的“国威”,东魏与南朝梁的使节都竭力在言辞、才学方面争锋,常常出现热烈辩论的场面。无论是梁使至邺城(今河北临漳),还是魏使至建康,都是如此,久而成为惯例。高澄则乐于此道,每当设宴招待梁使,高澄或者亲自到场,或者派遣属下与会。凡是东魏方面有所妙论、他都兴奋异常,为之鼓掌助威。他也因此召揽了一大批文人学士.或罗致门下,以为宾客;或推荐给朝廷,出任各级官吏。

东魏兴和三年(541年),有雀衔永安五铢置于高欢座前,高澄令百炉别铸此钱,又称“令公百炉”钱。北魏末年战乱,导致经济紊乱、货币贬值,民间私铸大量假钱。高澄在武定初年开始改革这项弊政,令人前往全国各地,将铸钱用的铜和原有的钱币收集起来,重新铸造。然而民间偷铸假钱的情况仍然屡禁不绝。因此高澄在武定六年(548年)进行新的货币改革,改用悬秤五铢。 东魏武定六年(548年)所铸永安五铢,号称“重如其文”,是一种足重货币。它的铸造是魏晋南北朝货币史上由乱到治的转折点,是后世足重货币“开皇五铢”的先驱。为促进足重货币的流通,高澄还采取了强硬的手段,《魏书·食货志》:“计百钱重一斤四两二十铢,自余皆准此为数。其京邑二市、天下州镇郡县之市,各置二称,悬于市门,私民所用之称,皆准市称以定轻重。” 由于武定六年永安五铢,曾被作为标尺,悬在市场的门上,以称量入市货币的轻重。因此在钱币学上,一般也将武定六年的永安五铢称之为“悬称五铢”。

高歡在547年死後,高澄繼任大丞相,都督中外諸軍,坐鎮晉陽。美姿容,善言笑,氣度高華,聰明過人,愛士好賢,爽直義氣。但又傲慢氣盛,性格暴烈,情慾豪侈,任性恣睢。與高欢之妾原魏廣平王妃鄭大車通姦,高歡死後,其次妻柔然(蠕蠕)的公主,按照柔然習俗,蠕蠕公主改嫁給高澄。親信崔季舒指稱薛置書的夫人元氏甚美,高澄把元氏騙到府中予以姦淫,元氏痛斥高澄是人面獸心。崔季舒將她移送廷尉府治罪,廷尉陸操以無罪釋之。

孝靜帝曾在打猎时骑马疾驰,就被监卫都督乌那罗受工伐劝止,理由是高澄会不悦;孝静帝不满高澄掌权,在被高澄举大酒杯敬酒時说“自古无不亡之国,朕活着有什么意思”,高澄就怒罵道:“朕,朕,狗腳朕!”並令中书黄门郎崔季舒打了他三拳。孝靜帝不堪忧辱,咏谢灵运之诗:“韩亡子房奋,秦帝仲连耻。本自江海人,忠义动君子。”侍讲大臣荀济和尚书祠部郎中元瑾、长秋卿刘思逸、华山王元大器、淮南王元宣洪及济北王元徽等商量,要想辦法除掉高澄,即在皇宫日夜挖掘通往城外的秘密通道。事機不密,被高澄得知,高澄马上带兵直闯进宫,直斥孝静帝图谋造反,虽然当时被孝静帝驳斥得痛哭谢罪并一起痛饮到深夜,但仅三天后,高澄就把孝静帝幽禁在含章堂,将荀济等人在市场上烹杀。549年,高澄计划夺取东魏政权,却在邺城(今河北临漳邺镇一带)被家中廚子蘭京暗杀刺死,享年僅29歲。

550年,其弟高洋正式稱帝,為北齊文宣帝。高洋追尊高澄为文襄皇帝,庙号世宗。

《北史·齐本纪上第六》:“文襄嗣膺霸道,威略昭著。内除奸逆,外拓淮夷,摈斥贪残,存情人物。而志在峻法,急于御下,于前王之德,有所未同。盖天意人心,好生恶杀,虽吉凶报应,未皆影响。总而论之,积善多庆。”


%%% Local Variables:
%%% mode: latex
%%% TeX-engine: xetex
%%% TeX-master: "../../Main"
%%% End:

%% -*- coding: utf-8 -*-
%% Time-stamp: <Chen Wang: 2019-12-23 16:14:47>


\section{北周\tiny(557-581)}

\subsection{简介}

北周(557年—581年)是中國歷史上南北朝的北朝之一。又称後周(宋朝以后鲜用),由宇文氏建立,定都長安,北周自建國後,統治實權一直在霸府宇文護身上,皇帝無力與之抗阻,為了擺脫宇文護的束縛,經過一連串的計畫與鬥爭,北周武帝終於殺死了宇文護,掌握大權,並以德施政,人民安樂,在位時更成功滅北齊,統一北朝。但他死後三年,北周便被楊堅的隋所取代,後由隋滅陳,統一中國。

北周由宇文泰奠定根基。北魏在六鎮之亂時,宇文泰投靠權臣爾朱榮,隨其入關中討伐叛逆,後來投于以關中隴西為根據地的大將賀拔岳的麾下,並漸漸受重用。

控制洛陽的另一權臣高歡認為賀拔岳有不臣之心,故使隴西秦州軍人刺殺賀拔岳。賀拔岳所屬將領在賀遇刺後,擁立宇文泰為統帥。宇文泰只是表面上服從高歡,其實控制關隴。

北魏孝武帝在討伐高歡失敗後,逃奔關中。宇文泰雖收容了他。但不久就將孝武帝殺害,改擁立西魏文帝建立西魏(535年)。而東方的高歡在孝武帝逃入關中後擁立東魏孝靜帝,把朝廷遷到河北鄴城,建立東魏(534年)。

西魏建立後,宇文泰成為大丞相。宇文泰在三次戰役中大敗東魏,奠定宇文氏在關中的基礎。宇文泰任用蘇綽等人改革,使西魏進一步強盛。進而攻入南梁的成都,奪取西川地盤。

西魏恭帝三年(556年),宇文泰病死,由嫡长子宇文觉承袭为安定郡公、太师、大冢宰。次年,宇文泰之侄宇文護迫西魏恭帝禪讓,由宇文覺即位天王,建立北周,建都長安(即今陝西西安)。

宇文覺不滿宇文護專權,企圖剷除宇文護,但反被其所殺。宇文護擁立其庶兄宇文毓,是為北周明帝。幾年後,明帝被殺,又擁立其兄弟宇文邕為北周武帝。宇文護執掌政權十五年,成為北周實際上的主宰。他承繼宇文泰、蘇綽的政策,消滅威脅政權的軍閥,使北周政權更鞏固。北周武帝年間,宇文護的兒子亂政害民,宇文護的威望大降。天和七年(572年)三月,北周武帝乘機刺殺了宇文護,重奪政權。

北周武帝執政後,積極推廣漢化並勵精圖治。575年發兵征北齊,577年,北周滅北齊,統一華北,仅北齐营州刺史高宝宁未降,奉逃奔突厥的皇子高绍义为帝。北周統一華北後國力一度興盛,但北周武帝英年早逝,其繼位者北周宣帝宇文贇奢侈浮華,沉緬酒色,政治腐敗。周宣帝生前即传位年幼的儿子北周靜帝宇文闡。580年6月8日宣帝病死,外戚楊堅以大丞相身份輔政,乘機將北周重臣外遣,進而把持朝政。相州總管尉遲迥、鄖州總管司馬消難與益州總管王謙等人不滿楊堅專權,聯合叛變反抗楊堅,爆發尉遲迥之亂,但被楊堅所派的韋孝寬、王誼與高熲等人平定。期间杨坚亦诛杀北周明帝长子太师雍州牧毕王宇文贤及尚在人世的宇文泰五子,并与突厥通好,突厥他钵可汗遂将高绍义交给北周。581年3月4日,北周靜帝禪讓帝位於楊堅,楊堅受禅称帝,改國號隋,北周享國二十四年而亡。杨坚建国不久,就将北周近支宗室诛杀殆尽,将宇文洛封为介国公作为北周奉祀。

此時期佛教藝術創作,大多數位於長安。印度笈多王朝雕像為許多大型佛像的原型。在敦煌千佛洞則有一些北周風格的壁畫,在這些壁畫中,山水畫固然重要,不過仍遜於人物畫。

北周人口盛时约1250万。


\subsection{文帝简介}

宇文泰(507年-556年11月21日),字黑獭(一作黑泰),代郡武川县(今内蒙古自治区呼和浩特市武川县)人,鲜卑宇文部后裔,漢化鮮卑人,北朝西魏權臣,也是北周政權的奠基者,掌權22年。后追尊为文王,庙号太祖,武成元年(559年)追尊为文帝。

宇文泰先世为宇文部酋长。东汉末,宇文部加入鲜卑部落联盟,遂被鲜卑化,游牧于今内蒙古自治区西拉木伦河上游。

北魏末年六镇起义中,宇文泰随父宇文肱加入鲜于修礼的起义队伍。起义被尔朱荣镇压后,宇文泰成为其部将贺拔岳麾下。永安三年(530年),魏孝庄帝杀尔朱荣,但军权仍然操在尔朱氏手中。不久,尔朱氏败灭,高欢位居丞相,并由此掌权。魏孝武帝密诏贺拔岳,欲以之牵制高欢。

永熙三年(534年)贺拔岳為侯莫陈悦所杀,由寇洛接手部隊以稳定军心,在平凉,寇洛自认“智能本阙,不宜统帅”,赵贵建议迎宇文泰,“诸将犹豫未决”,赫连达亦称宇文泰“明略过人,一时之杰。今日之事,非此公不济”,旧部才往夏州迎宇文泰。宇文泰率领帐下轻骑兵迅速向平凉进发,当时高欢也派长史侯景招揽贺拔岳的部众,宇文泰到安定,在休息的住所遇到了侯景,宇文泰吐出正吃的食物上战马,对侯景说:“贺拔公虽死,宇文泰还在,您想要干什么?”侯景脸色苍白,回答说:“我也只是箭而已,随着他人射,怎么能自己决定。”侯景至此返回。宇文泰到了平涼,为贺拔岳哭泣非常悲痛,贺拔岳的将士们既悲伤又高兴的说:“宇文公来了,再没有担心的。”

宇文泰上表北魏孝武帝元修,相约共扶王室,元修遂下诏以宇文泰为大都督、雍州刺史兼尚书令。同年,宇文泰平定秦、陇,孝武帝封官为侍中、骠骑大将军、开府仪同三司,关西大都督,略阳县公,地位仅次于高欢。是年元修攜情婦元明月及宗室數人從前綫逃跑,投奔宇文泰。十月,高欢另立元善见为帝,迁都于邺(今河北省临漳县),是为东魏。北魏遂分裂。

元修性格強硬,不守禮法,與宇文泰關係搞得非常僵。永熙三年(535年)十二月,宇文泰杀元修及元明月,另立元明月之兄元宝炬为帝,是为西魏,而实际政权控制在宇文泰手中。同年,宇文泰與元修的妹妹馮翊公主結婚。

宇文泰足智多谋,有很强的指挥能力。与东魏多次交锋,互有胜负。大统三年(537年)春,东魏进攻潼关,宇文泰大败之。秋,东魏十万人进沙苑(今陕西大荔),宇文泰以不满万人乘东魏军轻敌,亲自鸣鼓奋战,获得大胜,俘虏七万人,史称“沙苑之战”。

大统十三年(547年),西魏守將韋孝寬以七千人馬留守位置險要的玉璧城,頂住高歡十萬鮮卑鐵騎長達五十餘天的輪番衝擊。高歡喪師達七萬,智力用盡,玉璧城卻始終屹立不倒,高歡愁悶無處發泄,被活活氣死。

经济上,劝课农桑,恢复了均田制。并注意屯田以资军用。曾采纳苏绰建议进行改革,制定了「墨入朱出」(臣子上奏用黑筆寫,上級回覆用紅筆寫)公文格式,以朱色、墨色区别财政支出与收入(中文「赤字」的由來),定出户籍册和胪列次年课役大数的计帐制度。大统十三年的计帐,在敦煌石窟里有残卷保存下来。后又针对地方官员制定六条诏书:清心、敦教化、尽地利、擢贤良、恤狱讼、均赋役。

宇文泰改革军队统辖系统,建立府兵制,以扩大兵源。这个制度为隋唐所沿用。形式上采取鲜卑旧八部制,立八柱国,实为六军。每个柱国大将军下设有两个大将军,共12個大将军;每个大将军下有两个开府,共24个开府;每个开府下有两个仪同,共48个仪同;一个仪同领兵千人。这样,六柱国合计有兵四万八千人左右。这就是历史上著名的府兵。

外交上,宇文泰采取了和北攻南的政策,对于北方的突厥、柔然曾通好,但对于南朝蕭梁则采取攻势,先后进占了益州和荆雍等中國西南地區。

政治上,宇文泰实行以德治教化为主,法治为辅的原则。法律上,主张不苛不暴,而“法不阿贵”。思想文化上,推崇儒学,曾在行台设学。俘虏王褒、宗懔等均受到礼遇。后又令卢辩仿周礼更改官制,实行北周六官制,甚至政府文告也要仿先秦体。

宇文泰恢复鲜卑旧姓,如恢复皇族元氏为拓跋氏。而所将士卒也改从主将的胡姓。从形式上胡化一批的汉人,楊忠授普六茹氏,李虎授大野氏。

魏恭帝三年九月,宇文泰巡视北方回到牵屯山时患病,十月乙亥(556年11月21日)在云阳宫去世,时年虚岁五十,遗体运回长安才发丧。

\subsection{闵帝简介}

周孝閔帝宇文覺(542年-557年),字陀羅尼,代郡武川县(今内蒙古自治区呼和浩特市武川县)人,追尊周文帝宇文泰嫡长子,南北朝時代北周的开国君主,號稱天王,但實際上是權臣宇文護的傀儡。

七歲(周書記為九歲)時,被封為略陽郡公。當時有善於面相者史元華為他面相,私下告訴他的親人:「這位公子有至貴之相,但可惜的是他的寿命与地位不相称。」

556年三月,西魏恭帝拓跋廓命宇文覺為安定公世子;四月,拜為大司馬。到了十月,宇文泰過世,由宇文覺繼承他太師、安定公等官爵。十二月,拓跋廓又下詔以岐陽之地封宇文觉為周公,不久即禅位,派济北公拓跋迪将皇帝的玉玺和绶带送给宇文觉。並於557年正式即天王位,是為北周的開始。

宇文覺是宇文泰的第三子,生性剛毅果敢,對於其堂兄宇文護專政感到相當不滿,而司會李植與軍司馬孫恆也對宇文護權高位重頗有微詞,便與乙弗鳳、賀拔提等人一同私下向宇文覺請求誅殺宇文護,宇文覺同意。他們又找了張光洛一同行事,不料張光洛卻將此事告訴宇文護。於是宇文護改封李植為梁州刺史,孫恆為潼州刺史,將他們外放。乙弗鳳因此表示他將把宇文護引進宮後誅殺,但張光洛又將此事告訴宇文護,宇文護反而與尉遲綱合謀廢立之事,先設計誅殺乙弗鳳,並使宇文覺身邊沒有侍衛;接著派賀蘭祥逼迫宇文覺退位,將他貶為略陽公並幽禁,不久將他殺害,死时只有15岁。

孝闵帝为人温和并体恤民情,多次免除百姓的繁重税务,算是北朝时期的一位明君。

後來他的庶弟周武帝宇文邕誅殺宇文護,追認了宇文覺的开国皇帝身份,派遣蜀國公尉遲迥在南郊上諡其為孝閔皇帝,稱其陵墓為靜陵。北周武帝孝陵以西已知其为北周重臣葬地,北周诸陵当在咸阳市渭城区底张镇一带。

%% -*- coding: utf-8 -*-
%% Time-stamp: <Chen Wang: 2019-12-23 16:08:56>

\subsection{明帝\tiny(557-560)}

\subsubsection{生平}

周明帝宇文毓(534年-560年5月30日,在位:557年-560年),小名統萬突。是南北朝时期北周的天王及皇帝。宇文泰庶长子。母亲是夫人姚氏。

557年九月因唯一的嫡弟宇文觉被堂兄晋公宇文护所废而即天王位,559年称帝。明帝明敏有识量,为宇文护所惮。武成二年(560年)夏四月,宇文护暗中命令李安在加糖的蒸饼中下毒,进献给周明帝,四月庚子(560年5月29日)周明帝病危,四月辛丑(560年5月30日)去世。临终前自知为宇文护所害,口述遗诏五百余字,称诸子年幼不堪大任,称赞异母弟鲁公宇文邕宽仁大度海内所闻,能昌大周朝的必是他,意即以宇文邕继位。宇文护也最终拥立宇文邕继位。

後人普遍認同明帝是一位治国有方的明君,在位期间颇有作为,勵精圖治。

周明帝死后葬昭陵,具体位置不详。北周武帝孝陵以西已知其为北周重臣葬地,北周诸陵当在咸阳市渭城区底张镇一带。

\subsubsection{武成}

\begin{longtable}{|>{\centering\scriptsize}m{2em}|>{\centering\scriptsize}m{1.3em}|>{\centering}m{8.8em}|}
  % \caption{秦王政}\
  \toprule
  \SimHei \normalsize 年数 & \SimHei \scriptsize 公元 & \SimHei 大事件 \tabularnewline
  % \midrule
  \endfirsthead
  \toprule
  \SimHei \normalsize 年数 & \SimHei \scriptsize 公元 & \SimHei 大事件 \tabularnewline
  \midrule
  \endhead
  \midrule
  元年 & 559 & \tabularnewline\hline
  二年 & 560 & \tabularnewline
  \bottomrule
\end{longtable}


%%% Local Variables:
%%% mode: latex
%%% TeX-engine: xetex
%%% TeX-master: "../../Main"
%%% End:

%% -*- coding: utf-8 -*-
%% Time-stamp: <Chen Wang: 2019-12-23 16:12:35>

\subsection{武帝\tiny(560-578)}

\subsubsection{生平}

周武帝宇文邕yōng(543年-578年6月21日),字祢罗突,代郡武川县(今内蒙古自治区呼和浩特市武川县)人,追尊周文帝宇文泰第四子,北周第三位皇帝(560年—578年在位),期間推動建德毀佛,以求富國強兵,是三武滅佛之一。滅亡北齐,統一逾三分之二的中國本部,為12年後隋滅陳之戰打下基礎。在位18年。

宇文邕在西魏时封辅城郡公,北周封鲁国公,其堂兄宇文护专横跋扈,连杀二帝,又立宇文邕为帝。宇文邕18歲即位,因其兄弟先後被宇文护所殺,武帝即位後,為免自身也遭殺身之禍,對宇文护表示恭敬,讓他主理國家大事,以靜待時機。武帝不甘做傀儡,最終於572年杀死宇文护,得以亲政。

在位期间,武帝不像其父欲恢復鲜卑旧俗,反而极力摆脱鲜卑旧俗並接受漢文化,且自己也整顿吏治,使北周政治清明,百姓生活安定,国势强盛。宇文邕生活俭朴,能够及时关心民间疾苦。据史书记载,他“身布袍,寝布被……后宫不过十余人。”他的漢文化政策為日後楊堅的統一奠定基礎。

另外他还聽從道士衛元嵩和张宾意見大举灭佛,捣毁全国大量佛塔、佛寺,严令僧尼还俗,这是“求武器于塔庙之间、以士兵于僧侣之下”的富國強兵运动,是为建德毀佛。而在宇文邕禁止佛教之外,而且卫元嵩自己没想到连道教也被禁止,自己和众多的道士也被迫还俗。

正当北周日益强盛的时候,北齐却日衰。建德四年(575年)末,宇文邕於是出兵大举进攻腐朽的北齐,并于一年半后(即建德六年,577年)灭北齐。

宣政元年(578年)宇文邕率軍分五道伐突厥,未出發即病死,年僅36岁,谥号武帝,庙号高祖。他可以說是南北朝兩百多年的亂世中少數稱得上有作為的君主。

在史書中,他是一位嚴父,曾對其繼承人、教而不善的太子宇文贇(后来的北周宣帝)施用体罚,並多次威脅要廢去其太子地位,但最後都沒有實行。这样的举措反而收到了反效果,让宇文赟对他记恨,而更加不听从他的说教。宣帝繼位后荒淫无度,不到三年内其子就被杨坚篡位,北周灭亡。

宇文邕發明類似樗蒲、打馬的擲賽遊戲,史稱北周象戲,并編著有《象经》一书。

唐令狐德棻《周書》評價宇文邕沉著、毅力且有智謀,韜光晦跡、除國害。之後勵精圖治、除卻奢靡、凡事從儉,戰爭時與軍士同喜悲。令狐德棻認為,再一兩年,宇文邕就能天一大一統:「帝沉毅有智謀。初以晉公護專權,常自晦跡,人莫測其深淺。及誅護之後,始親萬機。克己勵精,聽覽不怠。用法嚴整,多所罪殺。號令懇惻,唯屬意於政。羣下畏服,莫不肅然。性旣明察,少於恩惠。凡布懷立行,皆欲踰越古人。身衣布袍,寢布被,無金寶之飾,諸宮殿華綺者,皆撤毀之,改為土階數尺,不施櫨栱。其雕文刻鏤,錦繡纂組,一皆禁斷。後宮嬪禦,不過十餘人。勞謙接下,自強不息。以海內未康,銳情教習。至於校兵閱武,步行山谷,履涉勤苦,皆人所不堪。平齊之役,見軍士有跣行者,帝親脫靴以賜之。每宴會將士,必自執杯勸酒,或手付賜物。至於征伐之處,躬在行陣。性又果決,能斷大事。故能得士卒死力,以弱制強。破齊之後,遂欲窮兵極武,平突厥,定江南,一二年間,必使天下一統,此其志也。」

唐令狐德棻《周書》評價宇文邕認真治國、同匹夫節儉度日,成為一時明君。雖然因為長年征戰,被稱窮兵黷武,但他的鴻圖遠略,是能凌駕古代王者:「自東西否隔,二國爭強,戎馬生郊,干戈日用,兵連禍結,力敵勢均,疆埸之事,一彼一此。高祖纘業,未親萬機,慮遠謀深,以蒙養正。及英威電發,朝政惟新,內難旣除,外略方始。乃苦心焦思,克己勵精,勞役為士卒之先,居處同匹夫之儉。脩富民之政,務強兵之術,乘讐人之有釁,順大道而推亡。五年之間,大勳斯集。攄祖宗之宿憤,拯東夏之阽危,盛矣哉,其有成功者也。若使翌日之瘳無爽,經營之志獲申,黷武窮兵,雖見譏於良史,雄圖遠略,足方駕於前王者歟。」


\subsubsection{保定}

\begin{longtable}{|>{\centering\scriptsize}m{2em}|>{\centering\scriptsize}m{1.3em}|>{\centering}m{8.8em}|}
  % \caption{秦王政}\
  \toprule
  \SimHei \normalsize 年数 & \SimHei \scriptsize 公元 & \SimHei 大事件 \tabularnewline
  % \midrule
  \endfirsthead
  \toprule
  \SimHei \normalsize 年数 & \SimHei \scriptsize 公元 & \SimHei 大事件 \tabularnewline
  \midrule
  \endhead
  \midrule
  元年 & 561 & \tabularnewline\hline
  二年 & 562 & \tabularnewline\hline
  三年 & 563 & \tabularnewline\hline
  四年 & 564 & \tabularnewline\hline
  五年 & 565 & \tabularnewline
  \bottomrule
\end{longtable}

\subsubsection{天和}

\begin{longtable}{|>{\centering\scriptsize}m{2em}|>{\centering\scriptsize}m{1.3em}|>{\centering}m{8.8em}|}
  % \caption{秦王政}\
  \toprule
  \SimHei \normalsize 年数 & \SimHei \scriptsize 公元 & \SimHei 大事件 \tabularnewline
  % \midrule
  \endfirsthead
  \toprule
  \SimHei \normalsize 年数 & \SimHei \scriptsize 公元 & \SimHei 大事件 \tabularnewline
  \midrule
  \endhead
  \midrule
  元年 & 566 & \tabularnewline\hline
  二年 & 567 & \tabularnewline\hline
  三年 & 568 & \tabularnewline\hline
  四年 & 569 & \tabularnewline\hline
  五年 & 570 & \tabularnewline\hline
  六年 & 571 & \tabularnewline\hline
  七年 & 572 & \tabularnewline
  \bottomrule
\end{longtable}

\subsubsection{建德}

\begin{longtable}{|>{\centering\scriptsize}m{2em}|>{\centering\scriptsize}m{1.3em}|>{\centering}m{8.8em}|}
  % \caption{秦王政}\
  \toprule
  \SimHei \normalsize 年数 & \SimHei \scriptsize 公元 & \SimHei 大事件 \tabularnewline
  % \midrule
  \endfirsthead
  \toprule
  \SimHei \normalsize 年数 & \SimHei \scriptsize 公元 & \SimHei 大事件 \tabularnewline
  \midrule
  \endhead
  \midrule
  元年 & 572 & \tabularnewline\hline
  二年 & 573 & \tabularnewline\hline
  三年 & 574 & \tabularnewline\hline
  四年 & 575 & \tabularnewline\hline
  五年 & 576 & \tabularnewline\hline
  六年 & 578 & \tabularnewline
  \bottomrule
\end{longtable}

\subsubsection{宣政}

\begin{longtable}{|>{\centering\scriptsize}m{2em}|>{\centering\scriptsize}m{1.3em}|>{\centering}m{8.8em}|}
  % \caption{秦王政}\
  \toprule
  \SimHei \normalsize 年数 & \SimHei \scriptsize 公元 & \SimHei 大事件 \tabularnewline
  % \midrule
  \endfirsthead
  \toprule
  \SimHei \normalsize 年数 & \SimHei \scriptsize 公元 & \SimHei 大事件 \tabularnewline
  \midrule
  \endhead
  \midrule
  元年 & 578 & \tabularnewline
  \bottomrule
\end{longtable}


%%% Local Variables:
%%% mode: latex
%%% TeX-engine: xetex
%%% TeX-master: "../../Main"
%%% End:

%% -*- coding: utf-8 -*-
%% Time-stamp: <Chen Wang: 2021-11-01 15:15:27>

\subsection{宣帝宇文贇\tiny(578-579)}

\subsubsection{生平}

周宣帝宇文\xpinyin*{贇}(559年-580年6月22日),字乾伯,自稱天元皇帝,代郡武川县(今内蒙古自治区呼和浩特市武川县)人,北周武帝宇文邕長子,北周第四代皇帝(578年-579年),在位只有一年。

北周宣帝武成元年生於同州,是個暴虐荒淫的皇帝。宇文贇即位前,父親武帝對他管教極為嚴格,曾派人監視他的言行舉止,甚至只要犯錯就會嚴厲懲罰。

建德二年(573年),迎娶隨國公楊堅的長女楊麗華。

宣政元年(578年)武帝去世後,遺詔太子宇文贇襲統大寶。宇文贇即位,史稱「周天元」。宇文贇在父親死後,面無哀戚,抚摸着脚上被打的杖痕,大声对着武帝的棺材喊道:“死得太晚了!”

宣帝即位后,整日沉迷于酒色,极其荒淫无道,史称:“宣帝初立,即逞奢欲。”最後甚至五位皇后並立,此舉打破劉聰的“三后並立”的記錄,又大肆裝飾宮殿,且濫施刑罰,經常派親信監視大臣言行,宣政元年(578年),殺宇文憲。北周國勢日漸衰落。

大成元年(579年),在位僅一年的宣帝禪位於長子宇文闡(北周靜帝),自称天元皇帝,杨丽华为天元皇后,住处称为“天台”,对臣下自称为“天”,仍實際掌控朝政,在後宮享樂。大臣朝见時,必须事先吃斋三天、净身一天。又於全國大选美女,以充实後宮,大将军陈山提的第八女陈月仪,仪同元晟的第二女元乐尚最受寵愛。由于纵欲过度,嬉遊無度,宇文赟的健康恶化,大象二年五月己酉(580年6月22日),宣帝因享樂過度有疾,禪位後次年去世,時年22歲。

次年,杨坚廢靜帝(宇文衍)自立,改國號為隋,北周滅亡。

周宣帝死后,大象二年七月丙申葬定陵,具体位置不详。北周武帝孝陵以西已知其为北周重臣葬地,北周诸陵当在咸阳市渭城区底张镇一带。

唐令狐德棻《周書》評價宇文贇擢髮難數、罄竹難書,罪惡多端,最終沒有受屠戮而亡,實在是很幸運:「高祖識嗣子之非才,顧宗祏之至重,滯愛同於晉武,則哲異於宋宣。但欲威之以檟楚,期之於懲肅,義方之教,豈若是乎。卒使昏虐君臨,姦回肆毒,善無小而必棄,惡無大而弗為。窮南山之簡,未足書其過;盡東觀之筆,不能記其罪。然猶獲全首領,及子而亡,幸哉。」

\subsubsection{大成}

\begin{longtable}{|>{\centering\scriptsize}m{2em}|>{\centering\scriptsize}m{1.3em}|>{\centering}m{8.8em}|}
  % \caption{秦王政}\
  \toprule
  \SimHei \normalsize 年数 & \SimHei \scriptsize 公元 & \SimHei 大事件 \tabularnewline
  % \midrule
  \endfirsthead
  \toprule
  \SimHei \normalsize 年数 & \SimHei \scriptsize 公元 & \SimHei 大事件 \tabularnewline
  \midrule
  \endhead
  \midrule
  元年 & 579 & \tabularnewline
  \bottomrule
\end{longtable}


%%% Local Variables:
%%% mode: latex
%%% TeX-engine: xetex
%%% TeX-master: "../../Main"
%%% End:

%% -*- coding: utf-8 -*-
%% Time-stamp: <Chen Wang: 2021-11-01 15:15:40>

\subsection{静帝宇文闡\tiny(579-581)}

\subsubsection{生平}

周靜帝宇文\xpinyin*{闡}(573年8月1日-581年7月9日),原名宇文衍,代郡武川县(今内蒙古自治区呼和浩特市武川县)人,北周末代皇帝(第五代,579年—581年在位),周宣帝宇文贇長子。母親是朱滿月。

建德二年六月壬子日(573年8月1日)出生。

大象元年二月辛巳(579年4月1日),受宣帝內禪即位,時年七歲。次年,宣帝崩。刘昉、鄭译決定以楊堅为輔政大臣(后李德林提议下成为大丞相)。期间杨坚平定尉迟迥之乱、剪除北周宗室,逐渐形成代国之势。

大象三年(581年)北周静帝禅让帝位于杨坚,杨坚登基。至此北周滅亡,隋朝建立。楊堅封宇文阐为介国公,食邑一万户,车服礼乐仍按北周天子的旧制,上书皇帝不称为表,皇帝回复不称诏。虽有这样的规定,实际上未能实行。

開皇元年五月壬申日(《隋书》作五月辛未日,相差一天),杨坚暗中派人殺死介国公宇文阐,时年九岁,后表示大为震惊,发布死讯,在朝堂举哀,隆重祭悼,谥为静皇帝,葬在恭陵;以周静帝的堂叔祖宇文洛继为介国公。

\subsubsection{大象}

\begin{longtable}{|>{\centering\scriptsize}m{2em}|>{\centering\scriptsize}m{1.3em}|>{\centering}m{8.8em}|}
  % \caption{秦王政}\
  \toprule
  \SimHei \normalsize 年数 & \SimHei \scriptsize 公元 & \SimHei 大事件 \tabularnewline
  % \midrule
  \endfirsthead
  \toprule
  \SimHei \normalsize 年数 & \SimHei \scriptsize 公元 & \SimHei 大事件 \tabularnewline
  \midrule
  \endhead
  \midrule
  元年 & 579 & \tabularnewline\hline
  二年 & 580 & \tabularnewline
  \bottomrule
\end{longtable}

\subsubsection{大定}

\begin{longtable}{|>{\centering\scriptsize}m{2em}|>{\centering\scriptsize}m{1.3em}|>{\centering}m{8.8em}|}
  % \caption{秦王政}\
  \toprule
  \SimHei \normalsize 年数 & \SimHei \scriptsize 公元 & \SimHei 大事件 \tabularnewline
  % \midrule
  \endfirsthead
  \toprule
  \SimHei \normalsize 年数 & \SimHei \scriptsize 公元 & \SimHei 大事件 \tabularnewline
  \midrule
  \endhead
  \midrule
  元年 & 581 & \tabularnewline
  \bottomrule
\end{longtable}


%%% Local Variables:
%%% mode: latex
%%% TeX-engine: xetex
%%% TeX-master: "../../Main"
%%% End:



%%% Local Variables:
%%% mode: latex
%%% TeX-engine: xetex
%%% TeX-master: "../../Main"
%%% End:


%%% Local Variables:
%%% mode: latex
%%% TeX-engine: xetex
%%% TeX-master: "../Main"
%%% End:
