%% -*- coding: utf-8 -*-
%% Time-stamp: <Chen Wang: 2021-11-01 15:07:58>

\subsection{废帝陳伯宗\tiny(566-568)}

\subsubsection{生平}

陳伯宗(554年6月12日或552年6月20日-570年4月22日),字奉業,小字藥王,南朝陳朝的第三代皇帝,史書稱作「廢帝」,陳文帝陳蒨之嫡出長子,母安德皇后沈妙容。

天康元年(566年)四月,陳伯宗在陳文帝死後即帝位,由於陳伯宗年幼,便以叔父安成王陳頊為司徒、錄尚書事、都督中外諸軍事。於是政局都為陳頊所掌握。567年改年號為光大,陳頊晉位為太傅,准許佩帶劍履上殿。光大二年(568年)11月,陳頊叛逆廢陳伯宗為臨海王,自立為帝,是為陳宣帝。

陳伯宗於被廢之後,於太建二年四月乙卯(570年4月22日)逝世,得年十九歲。

陈伯宗的出生时间,《陈书·废帝本纪》有具体记载是梁承圣三年(554年)五月庚寅生(554年6月12日)。可是《陈书·废帝本纪》又记载:“(陈伯宗)太建二年(570年)四月薨,时年十九”。由此推算陈伯宗出生时间为552年(十九岁是虚岁)。这样一来,《废帝本纪》就在陈伯宗年龄方面自相矛盾了。按照陈伯宗于554年出生的说法推算,陈伯宗死时的年龄为十七岁。而陈伯宗同母弟陈伯茂的年龄,《陈书》记载“光大二年,皇太后令黜废帝为临海王,其日又下令降伯茂为温麻侯。时六门之外有别馆,以为诸王冠昏之所,名为昏第。至是命伯茂出居之,宣帝遣盗殒之于车中,年十八”。由此推算可知陈伯茂大约也生于552年。陈伯宗為陈伯茂兄,出生时间应该是公元552年(552年五月庚寅,即552年6月20日),卒时年十九岁。


\subsubsection{光大}

\begin{longtable}{|>{\centering\scriptsize}m{2em}|>{\centering\scriptsize}m{1.3em}|>{\centering}m{8.8em}|}
  % \caption{秦王政}\
  \toprule
  \SimHei \normalsize 年数 & \SimHei \scriptsize 公元 & \SimHei 大事件 \tabularnewline
  % \midrule
  \endfirsthead
  \toprule
  \SimHei \normalsize 年数 & \SimHei \scriptsize 公元 & \SimHei 大事件 \tabularnewline
  \midrule
  \endhead
  \midrule
  元年 & 567 & \tabularnewline\hline
  二年 & 568 & \tabularnewline
  \bottomrule
\end{longtable}



%%% Local Variables:
%%% mode: latex
%%% TeX-engine: xetex
%%% TeX-master: "../../Main"
%%% End:
