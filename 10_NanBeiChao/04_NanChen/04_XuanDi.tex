%% -*- coding: utf-8 -*-
%% Time-stamp: <Chen Wang: 2019-12-23 14:33:06>

\subsection{宣帝\tiny(568-582)}

\subsubsection{生平}

陈宣帝陈\xpinyin*{顼}(530年8月14日-582年2月17日),又名陈昙顼,字绍世,小字师利,南北朝时期陈朝第四位皇帝(569年—582年在位),正式諡號為「孝宣皇帝」,後世比照漢朝和西晉皇帝省略「孝」字,稱「陳宣帝」,在位14年,年号太建。

陈顼出生于南朝梁中大通二年(530年)七月初六日(530年8月14日)。陳頊是高祖武皇帝陈霸先的侄子,始兴昭烈王陈道谭第二子,世祖文皇帝陈蒨的弟弟。他本来是皇帝陈伯宗的辅佐大臣,后叛逆废掉了陈伯宗,篡位为帝。

他在位期间,兴修水利,开垦荒地,鼓励农民生产,社会经济得到了一定的恢复与发展。573年(太建五年),派大将吴明彻乘北齐大乱之机北伐,攻占了吕梁(在今江苏徐州附近)和寿阳,一度占有淮、泗之地,但最后在577年被北周夺走。总的来说,陈顼在位期间,国家比较安定,政治也较为清明。陈顼于陈太建十四年崩(582年)正月初十日(582年2月17日),享年53岁。

陈顼谥号为孝宣皇帝,庙号高宗。葬显宁陵(在今南京郊区)。

\subsubsection{太建}

\begin{longtable}{|>{\centering\scriptsize}m{2em}|>{\centering\scriptsize}m{1.3em}|>{\centering}m{8.8em}|}
  % \caption{秦王政}\
  \toprule
  \SimHei \normalsize 年数 & \SimHei \scriptsize 公元 & \SimHei 大事件 \tabularnewline
  % \midrule
  \endfirsthead
  \toprule
  \SimHei \normalsize 年数 & \SimHei \scriptsize 公元 & \SimHei 大事件 \tabularnewline
  \midrule
  \endhead
  \midrule
  元年 & 569 & \tabularnewline\hline
  二年 & 570 & \tabularnewline\hline
  三年 & 571 & \tabularnewline\hline
  四年 & 572 & \tabularnewline\hline
  五年 & 573 & \tabularnewline\hline
  六年 & 574 & \tabularnewline\hline
  七年 & 575 & \tabularnewline\hline
  八年 & 576 & \tabularnewline\hline
  九年 & 577 & \tabularnewline\hline
  十年 & 578 & \tabularnewline\hline
  十一年 & 579 & \tabularnewline\hline
  十二年 & 580 & \tabularnewline\hline
  十三年 & 581 & \tabularnewline\hline
  十四年 & 582 & \tabularnewline
  \bottomrule
\end{longtable}



%%% Local Variables:
%%% mode: latex
%%% TeX-engine: xetex
%%% TeX-master: "../../Main"
%%% End:
