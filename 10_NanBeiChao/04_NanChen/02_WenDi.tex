%% -*- coding: utf-8 -*-
%% Time-stamp: <Chen Wang: 2021-11-01 15:07:48>

\subsection{文帝陈蒨\tiny(559-566)}

\subsubsection{生平}

陳文帝陈\xpinyin*{蒨}(522年-566年),一作茜,又名昙蒨、荃菺,字子華。中国南北朝时期陈朝第二位皇帝(560年—566年在位),在位7年,年号天嘉。

陈蒨是陈朝开国皇帝陈霸先長兄陳道譚的長子,深受陈霸先的賞識與栽培,更令其總理軍政。後來武帝駕崩,因唯一在世儿子陈昌在北周为人质,皇后章要儿听从陈蒨心腹大臣侯安都等安排,稱武帝遺詔命陈蒨入纂皇統,遂即帝位。

北周闻讯,为了制造内乱故意放陈昌回国。因道路一度为东梁所阻隔,天嘉元年陈昌才出发,因而写信要陈蒨让位。陈蒨很不高兴,说:“太子快回来了,我只好找个地方当藩王去养老。”侯安都说:“自古岂有被代天子?”陈昌入陈境后,陈蒨诏令主书舍人沿途迎接,却在陈昌渡江时由侯安都于无人时将其推入长江淹死,对外宣布陈昌在江中因船只故障而溺死。丧柩至京师,陈蒨亲出临哭,追谥号献,风光大葬,又以子陈伯信为其后嗣。

陈蒨在位期間,励精图治,整顿吏治,注重农桑,兴修水利,恢复江南经济。此时陈朝政治清明,百姓富裕,国势强盛,史稱「天嘉小康」。陳蒨亦因而是南朝皇帝中的明君。566年崩,享年44岁,谥号为文帝,庙号世祖。葬于永宁陵(在今南京棲霞區棲霞街道新合村獅子冲)。

陈蒨有一名貌美如婦的寵臣韓子高,《南史》記曰:“子高年十六,為總角,容貌美麗,狀似婦人。”陈蒨在还是临川王时邂逅了這位美少年,從此讓韓子高隨侍左右,寵愛備至。基於這種曖昧事實,所以後世有些小說、戲曲藉題發揮,露骨地將二人描繪成同性愛關係,例如唐朝李翊的《陳子高傳》(明朝馮夢龍的《情史》有節錄)、明朝王驥德的《男王后》等皆是著名創作。

\subsubsection{天嘉}

\begin{longtable}{|>{\centering\scriptsize}m{2em}|>{\centering\scriptsize}m{1.3em}|>{\centering}m{8.8em}|}
  % \caption{秦王政}\
  \toprule
  \SimHei \normalsize 年数 & \SimHei \scriptsize 公元 & \SimHei 大事件 \tabularnewline
  % \midrule
  \endfirsthead
  \toprule
  \SimHei \normalsize 年数 & \SimHei \scriptsize 公元 & \SimHei 大事件 \tabularnewline
  \midrule
  \endhead
  \midrule
  元年 & 560 & \tabularnewline\hline
  二年 & 561 & \tabularnewline\hline
  三年 & 562 & \tabularnewline\hline
  四年 & 563 & \tabularnewline\hline
  五年 & 564 & \tabularnewline\hline
  六年 & 565 & \tabularnewline\hline
  七年 & 566 & \tabularnewline
  \bottomrule
\end{longtable}

\subsubsection{天康}

\begin{longtable}{|>{\centering\scriptsize}m{2em}|>{\centering\scriptsize}m{1.3em}|>{\centering}m{8.8em}|}
  % \caption{秦王政}\
  \toprule
  \SimHei \normalsize 年数 & \SimHei \scriptsize 公元 & \SimHei 大事件 \tabularnewline
  % \midrule
  \endfirsthead
  \toprule
  \SimHei \normalsize 年数 & \SimHei \scriptsize 公元 & \SimHei 大事件 \tabularnewline
  \midrule
  \endhead
  \midrule
  元年 & 566 & \tabularnewline
  \bottomrule
\end{longtable}


%%% Local Variables:
%%% mode: latex
%%% TeX-engine: xetex
%%% TeX-master: "../../Main"
%%% End:
