%% -*- coding: utf-8 -*-
%% Time-stamp: <Chen Wang: 2019-12-23 14:31:08>


\section{南陈\tiny(557-589)}

\subsection{简介}

陈(557年-589年)是中国历史上南北朝时期南朝最后一个朝代,由陈霸先代梁所建立,以建康(今南京)为首都,国号陈。陈朝名称来自陈霸先即位前被封的陈公、陈王,但陈王的封号来源又有二说,一说认为来自陈王封地中排第一郡的陈留郡(见《陈书》),另一说认为陈霸先姓出于陈,陈为圣人舜后裔,故强迫梁帝封其为陈王(见胡三省《资治通鉴·陳纪一》下注)。

南梁太平二年(557年)梁敬帝蕭方智禪位當時已為陳王的大將陳霸先,陳霸先於是稱帝建立陳朝,史稱南陳。

由於侯景之亂的緣故,南朝受到的破壞極其之大,百廢待興。導致陈朝建立时,已经出现南朝转弱,北朝转强的局面。南朝已失去不少領土,到陳朝時,雖然北朝分裂為西魏和東魏(後被北周及北齊取代),長江以北為北齊所佔,西南(四川)被北周所佔,只能靠長江天險維持南北對峙的局面。

陈朝刚建立时(557年)面临北朝的入侵,形势十分危急。陈朝开国皇帝陈霸先带领军队一举击败敌军,形势有所好转。陳霸先於559年病逝,其侄陳文帝陳蒨即位,先後消滅各地的割據勢力,大力革除前朝蕭梁奢侈之風,使陳朝政治稍為安定。天康元年(566年),文帝死,遺詔太子陳伯宗繼位,568年被文帝弟陳宣帝陈顼以陈霸先章太后的名义所廢。宣帝繼續實行文帝時輕徭薄賦之策,使江南經濟逐漸恢復過來。

陈朝在經濟文化上比较发达,除了陳後主以外的幾位君王也都算得上明主,比起南朝其他三個朝代而言政治情勢較為穩定,但在軍事上卻已難以與北方抗衡,即使北方已分裂為北周及北齊,而且北齊皇帝多為殘暴之君,北周由於宇文護專權之故,政治情勢在武帝親政之前也一直不是十分安定,但在武力上南方仍然望塵莫及。

到了陳宣帝時試圖結好北周,夾擊北齊。太建四年(572年),周陳互派使者。翌年的兩年內,陳宣帝以吳明徹為征討大都督,統兵十萬北伐攻打北齊,佔領淮、陰、泗諸城。王琳等忠于南梁的势力和熊昙朗、周迪、留异、陈宝应等在梁末崛起的半独立势力也相继被消灭。

太建九年(577年),北周滅北齊。翌年,周陳在呂梁展開激戰,陳敗周勝,吳明徹被俘,淮南之地得而復失,江北州郡盡為北周所有,回復南北隔江對峙的被動局面。

太建十四年(582年),宣帝病死,太子陳叔寶繼位,是為後主。陳後主不問政事,荒於酒色,陳朝政治江河日下,後主亦自恃長江天險不思進取,被動固守。至於北朝,建立漢人政权隋朝的隋文帝積極準備滅陳。陳朝最後被北方的隋朝在南征之戰中所滅,南北統一。

首都建康為重要的文化、政治、宗教中心,吸引東南亞、印度的商人及僧侶前來。南朝文化在梁朝时达到巅峰,经历了侯景之乱的文化洗劫,到陳朝时已经进入尾声。文学方面以徐陵为文宗,有文学集《玉台新咏》传世,其中最著名篇章《孔雀东南飞》。艺术方面以姚最的评论《续画品录》影响最大。


%% -*- coding: utf-8 -*-
%% Time-stamp: <Chen Wang: 2021-11-01 15:07:37>

\subsection{武帝陈霸先\tiny(557-559)}

\subsubsection{生平}

陳武帝陈霸先(503年-559年),字兴国,小字法生,吴兴郡长城县(今浙江长兴)人,南北朝時代陳朝開國皇帝。原是南梁的著名軍事將領。557年接受梁敬帝的禪位建立陳朝,557年至559年在位。死後廟號高祖,諡號武皇帝。

梁武帝天监二年(503年)出生,自幼家境贫寒,却好读兵书。初仕乡为里司,后到建康为油库吏,之后又为新渝縣侯萧暎传教(傳令吏)。当时,萧映是广州刺史,于是陈霸先随萧暎来到广州,任中直兵参军。因陈霸先平乱有功,被提任为西江督护,很快又因平交州李賁之乱有功,封为交州司马兼领武平太守(越南永福省永安市附近),后任振远将军、高要太守。梁武帝萧衍曾命使臣將陳霸先畫像帶回,并授予直阁将军一职,封号新安子。

侯景叛乱,陈霸先于梁大宝元年(550年)正月,在始兴(今广东韶关)起兵讨侯景,次年与征东将军王僧辩会合共进。天正二年(552年)三月,领军围石头城(在今南京),大败侯景。因功授征虏将军、开府仪同三司,封司空,领扬州(非今日之扬州市)刺史,镇京口(今江苏镇江)。

梁承圣三年(554年),西魏破江陵,梁元帝被杀。陈霸先与王僧辩请晋安王萧方智以太宰承制,又遣长史谢哲奉笺劝进,晋安王入居朝堂,称梁王。承圣四年(555年),王僧辩屈事北齐,迎立北齐扶植的萧渊明为梁帝,陈霸先苦劝无效,遂诛王僧辩,立萧方智为帝。后又击退北齐的南下侵略,剷平了王僧辩餘党的反抗,晋封陈公,再封陈王,受九锡。

王僧辩的部下王琳得知陈霸先立萧方智为帝,並不服氣,太平二年五月,進攻陈霸先。六月。陈霸先命平西将军周文育、平南将军侯安都等征讨王琳。侯安都至沌口(今武昌)与王琳对峙多日,侯安都军大败。陈霸先再派遣侯瑱、徐度進攻王琳,再派谢哲調解。八月,王琳退军湘州(今湖南长沙),陈霸先以大軍進驻大雷(今安徽望江)。雙方再度對峙,直到陈霸先病逝。

梁太平二年(557年)梁敬帝萧方智禅位,陈霸先代梁称帝建立陈朝,史稱南陳。王琳也立永嘉王萧庄,称帝于荆州。陈永定三年(559年)六月十二日,生病。六月二十一日病逝。因唯一在世亲子陈昌被北周扣留,遗诏追兄子临川王陳蒨入纂。八月甲午,群臣上谥号曰武皇帝,庙号高祖。丙申,葬万安陵(在今南京市江宁区)。隋滅陳後,王僧辯之子王頒是隋军大将,为报父仇,掘陳霸先之墓,挖出骨骸,焚化成灰水喝進肚裡。

在位僅二年,是魏晉南北朝時期中屬於南朝方面十分難得的英明君主,其個性节俭樸素,“常膳不过数品,私宴用瓦器、蚌盘,肴核充事而已;后宫无金翠之饰,不设女乐”。在政治上宽政廉平,爱育为本,恒崇宽政,不行株连,怀柔攻心,诚贯天下。但因建立南陳時,巴蜀地區及淮南已被北周及北齊攻陷,其統治疆域是南朝四代主要政權疆域最小的一個。在经济上,穩定保持了江南的發展。

南陳的吏部尚書姚察在陳亡被俘到隋朝後,為隋文帝撰寫陳朝歷史,仍認為陳霸先「英略大度,應變無方,」與漢高祖劉邦、魏武帝曹操一樣同屬偉人(《陳書》卷一:英略大度,應變無方,蓋漢高、魏武之亞矣)。

唐散騎常侍姚思廉(557年-637年),字簡之,自幼習史,父親是南陳的末任吏部尚書姚察。姚思廉曾任隋朝代王楊侑侍讀。唐朝李淵稱帝後,為李世民秦王府文學館學士。自玄武門之變,進任太子洗馬。貞觀初年,又任著作郎,「十八學士」之一。官至散騎常侍,受命與魏徵同修梁陳二史。貞觀十年(636年),成《梁書》(50卷)《陳書》(30卷),為二十四史之一。他評價陳霸先「智以綏物、武以寧亂、英謀獨運、人皆莫及」。

唐鄭國文貞公魏徵(580年-643年2月11日),字玄成,唐朝貞觀時諫臣,曾是《隋書》、《周書》、《北齊書》、《梁書》、《陳書》五部史書的總監修官。魏徵認為陳霸先效命舊王朝,立下豐功偉績,功勳不下曹操、劉裕;三分天下,能夠「決機百勝」,雄豪無愧劉備、孫權(高祖拔起壟畝,有雄桀之姿。始佐下藩,奮英奇之略。魏王之延漢鼎祚,宋武之反晉乘輿,懋績鴻勳,無以尚也。決機百勝,成此三分,方諸鼎峙之雄,足以無慚權、備矣)。

唐朝大史學家李延壽評價:用「雄武英略」、「性甚仁愛」、「恆崇寬簡」、「彌厲恭儉」 來稱讚陳霸先一生。

北宋《資治通鑒》編撰者司馬光用「臨戎制勝,英謀獨運」、「為政務崇寬簡」、「性儉素」等語言分別概括了陳霸先治軍、從政、為人的鮮明個性。

明朝南京太僕寺丞歸有光評價:恭儉勤勞,志度弘遠,江左諸帝,號為最賢。赫然陳祖,大業光燦。寂寞沛鄉,吾茲感歎。[來源請求]

中國共產黨中央委員會主席毛澤東說他欣賞的是陳霸先南征北戰所使用的戰術。毛澤東在晚年時曾要求人們讀讀《陳書》,瞭解陳霸先的身世經歷。

中華民國作者柏楊在他的一本名為《中國人史綱》的出版品中評道:「陳帝國是南北朝唯一沒有出過暴君的政權。」

\subsubsection{永定}

\begin{longtable}{|>{\centering\scriptsize}m{2em}|>{\centering\scriptsize}m{1.3em}|>{\centering}m{8.8em}|}
  % \caption{秦王政}\
  \toprule
  \SimHei \normalsize 年数 & \SimHei \scriptsize 公元 & \SimHei 大事件 \tabularnewline
  % \midrule
  \endfirsthead
  \toprule
  \SimHei \normalsize 年数 & \SimHei \scriptsize 公元 & \SimHei 大事件 \tabularnewline
  \midrule
  \endhead
  \midrule
  元年 & 557 & \tabularnewline\hline
  二年 & 558 & \tabularnewline\hline
  三年 & 559 & \tabularnewline
  \bottomrule
\end{longtable}


%%% Local Variables:
%%% mode: latex
%%% TeX-engine: xetex
%%% TeX-master: "../../Main"
%%% End:

%% -*- coding: utf-8 -*-
%% Time-stamp: <Chen Wang: 2021-11-01 15:07:48>

\subsection{文帝陈蒨\tiny(559-566)}

\subsubsection{生平}

陳文帝陈\xpinyin*{蒨}(522年-566年),一作茜,又名昙蒨、荃菺,字子華。中国南北朝时期陈朝第二位皇帝(560年—566年在位),在位7年,年号天嘉。

陈蒨是陈朝开国皇帝陈霸先長兄陳道譚的長子,深受陈霸先的賞識與栽培,更令其總理軍政。後來武帝駕崩,因唯一在世儿子陈昌在北周为人质,皇后章要儿听从陈蒨心腹大臣侯安都等安排,稱武帝遺詔命陈蒨入纂皇統,遂即帝位。

北周闻讯,为了制造内乱故意放陈昌回国。因道路一度为东梁所阻隔,天嘉元年陈昌才出发,因而写信要陈蒨让位。陈蒨很不高兴,说:“太子快回来了,我只好找个地方当藩王去养老。”侯安都说:“自古岂有被代天子?”陈昌入陈境后,陈蒨诏令主书舍人沿途迎接,却在陈昌渡江时由侯安都于无人时将其推入长江淹死,对外宣布陈昌在江中因船只故障而溺死。丧柩至京师,陈蒨亲出临哭,追谥号献,风光大葬,又以子陈伯信为其后嗣。

陈蒨在位期間,励精图治,整顿吏治,注重农桑,兴修水利,恢复江南经济。此时陈朝政治清明,百姓富裕,国势强盛,史稱「天嘉小康」。陳蒨亦因而是南朝皇帝中的明君。566年崩,享年44岁,谥号为文帝,庙号世祖。葬于永宁陵(在今南京棲霞區棲霞街道新合村獅子冲)。

陈蒨有一名貌美如婦的寵臣韓子高,《南史》記曰:“子高年十六,為總角,容貌美麗,狀似婦人。”陈蒨在还是临川王时邂逅了這位美少年,從此讓韓子高隨侍左右,寵愛備至。基於這種曖昧事實,所以後世有些小說、戲曲藉題發揮,露骨地將二人描繪成同性愛關係,例如唐朝李翊的《陳子高傳》(明朝馮夢龍的《情史》有節錄)、明朝王驥德的《男王后》等皆是著名創作。

\subsubsection{天嘉}

\begin{longtable}{|>{\centering\scriptsize}m{2em}|>{\centering\scriptsize}m{1.3em}|>{\centering}m{8.8em}|}
  % \caption{秦王政}\
  \toprule
  \SimHei \normalsize 年数 & \SimHei \scriptsize 公元 & \SimHei 大事件 \tabularnewline
  % \midrule
  \endfirsthead
  \toprule
  \SimHei \normalsize 年数 & \SimHei \scriptsize 公元 & \SimHei 大事件 \tabularnewline
  \midrule
  \endhead
  \midrule
  元年 & 560 & \tabularnewline\hline
  二年 & 561 & \tabularnewline\hline
  三年 & 562 & \tabularnewline\hline
  四年 & 563 & \tabularnewline\hline
  五年 & 564 & \tabularnewline\hline
  六年 & 565 & \tabularnewline\hline
  七年 & 566 & \tabularnewline
  \bottomrule
\end{longtable}

\subsubsection{天康}

\begin{longtable}{|>{\centering\scriptsize}m{2em}|>{\centering\scriptsize}m{1.3em}|>{\centering}m{8.8em}|}
  % \caption{秦王政}\
  \toprule
  \SimHei \normalsize 年数 & \SimHei \scriptsize 公元 & \SimHei 大事件 \tabularnewline
  % \midrule
  \endfirsthead
  \toprule
  \SimHei \normalsize 年数 & \SimHei \scriptsize 公元 & \SimHei 大事件 \tabularnewline
  \midrule
  \endhead
  \midrule
  元年 & 566 & \tabularnewline
  \bottomrule
\end{longtable}


%%% Local Variables:
%%% mode: latex
%%% TeX-engine: xetex
%%% TeX-master: "../../Main"
%%% End:

%% -*- coding: utf-8 -*-
%% Time-stamp: <Chen Wang: 2019-12-23 14:31:33>

\subsection{废帝\tiny(566-568)}

\subsubsection{生平}

陳伯宗(554年6月12日或552年6月20日-570年4月22日),字奉業,小字藥王,南朝陳朝的第三代皇帝,史書稱作「廢帝」,陳文帝陳蒨之嫡出長子,母安德皇后沈妙容。

天康元年(566年)四月,陳伯宗在陳文帝死後即帝位,由於陳伯宗年幼,便以叔父安成王陳頊為司徒、錄尚書事、都督中外諸軍事。於是政局都為陳頊所掌握。567年改年號為光大,陳頊晉位為太傅,准許佩帶劍履上殿。光大二年(568年)11月,陳頊叛逆廢陳伯宗為臨海王,自立為帝,是為陳宣帝。

陳伯宗於被廢之後,於太建二年四月乙卯(570年4月22日)逝世,得年十九歲。

陈伯宗的出生时间,《陈书·废帝本纪》有具体记载是梁承圣三年(554年)五月庚寅生(554年6月12日)。可是《陈书·废帝本纪》又记载:“(陈伯宗)太建二年(570年)四月薨,时年十九”。由此推算陈伯宗出生时间为552年(十九岁是虚岁)。这样一来,《废帝本纪》就在陈伯宗年龄方面自相矛盾了。按照陈伯宗于554年出生的说法推算,陈伯宗死时的年龄为十七岁。而陈伯宗同母弟陈伯茂的年龄,《陈书》记载“光大二年,皇太后令黜废帝为临海王,其日又下令降伯茂为温麻侯。时六门之外有别馆,以为诸王冠昏之所,名为昏第。至是命伯茂出居之,宣帝遣盗殒之于车中,年十八”。由此推算可知陈伯茂大约也生于552年。陈伯宗為陈伯茂兄,出生时间应该是公元552年(552年五月庚寅,即552年6月20日),卒时年十九岁。


\subsubsection{光大}

\begin{longtable}{|>{\centering\scriptsize}m{2em}|>{\centering\scriptsize}m{1.3em}|>{\centering}m{8.8em}|}
  % \caption{秦王政}\
  \toprule
  \SimHei \normalsize 年数 & \SimHei \scriptsize 公元 & \SimHei 大事件 \tabularnewline
  % \midrule
  \endfirsthead
  \toprule
  \SimHei \normalsize 年数 & \SimHei \scriptsize 公元 & \SimHei 大事件 \tabularnewline
  \midrule
  \endhead
  \midrule
  元年 & 567 & \tabularnewline\hline
  二年 & 568 & \tabularnewline
  \bottomrule
\end{longtable}



%%% Local Variables:
%%% mode: latex
%%% TeX-engine: xetex
%%% TeX-master: "../../Main"
%%% End:

%% -*- coding: utf-8 -*-
%% Time-stamp: <Chen Wang: 2019-12-23 14:33:06>

\subsection{宣帝\tiny(568-582)}

\subsubsection{生平}

陈宣帝陈\xpinyin*{顼}(530年8月14日-582年2月17日),又名陈昙顼,字绍世,小字师利,南北朝时期陈朝第四位皇帝(569年—582年在位),正式諡號為「孝宣皇帝」,後世比照漢朝和西晉皇帝省略「孝」字,稱「陳宣帝」,在位14年,年号太建。

陈顼出生于南朝梁中大通二年(530年)七月初六日(530年8月14日)。陳頊是高祖武皇帝陈霸先的侄子,始兴昭烈王陈道谭第二子,世祖文皇帝陈蒨的弟弟。他本来是皇帝陈伯宗的辅佐大臣,后叛逆废掉了陈伯宗,篡位为帝。

他在位期间,兴修水利,开垦荒地,鼓励农民生产,社会经济得到了一定的恢复与发展。573年(太建五年),派大将吴明彻乘北齐大乱之机北伐,攻占了吕梁(在今江苏徐州附近)和寿阳,一度占有淮、泗之地,但最后在577年被北周夺走。总的来说,陈顼在位期间,国家比较安定,政治也较为清明。陈顼于陈太建十四年崩(582年)正月初十日(582年2月17日),享年53岁。

陈顼谥号为孝宣皇帝,庙号高宗。葬显宁陵(在今南京郊区)。

\subsubsection{太建}

\begin{longtable}{|>{\centering\scriptsize}m{2em}|>{\centering\scriptsize}m{1.3em}|>{\centering}m{8.8em}|}
  % \caption{秦王政}\
  \toprule
  \SimHei \normalsize 年数 & \SimHei \scriptsize 公元 & \SimHei 大事件 \tabularnewline
  % \midrule
  \endfirsthead
  \toprule
  \SimHei \normalsize 年数 & \SimHei \scriptsize 公元 & \SimHei 大事件 \tabularnewline
  \midrule
  \endhead
  \midrule
  元年 & 569 & \tabularnewline\hline
  二年 & 570 & \tabularnewline\hline
  三年 & 571 & \tabularnewline\hline
  四年 & 572 & \tabularnewline\hline
  五年 & 573 & \tabularnewline\hline
  六年 & 574 & \tabularnewline\hline
  七年 & 575 & \tabularnewline\hline
  八年 & 576 & \tabularnewline\hline
  九年 & 577 & \tabularnewline\hline
  十年 & 578 & \tabularnewline\hline
  十一年 & 579 & \tabularnewline\hline
  十二年 & 580 & \tabularnewline\hline
  十三年 & 581 & \tabularnewline\hline
  十四年 & 582 & \tabularnewline
  \bottomrule
\end{longtable}



%%% Local Variables:
%%% mode: latex
%%% TeX-engine: xetex
%%% TeX-master: "../../Main"
%%% End:

%% -*- coding: utf-8 -*-
%% Time-stamp: <Chen Wang: 2021-11-01 15:08:17>

\subsection{后主陈叔宝\tiny(582-589)}

\subsubsection{生平}

陈叔宝(553年12月或554年1月-604年),字元秀,小字黄奴。南北朝时期陈朝末代皇帝(第五代,582年—589年在位),史称“后主”,在位7年,年号至德、祯明。

陈叔宝出生于梁朝承圣二年十一月(553年12月21日或,554年1月18日),是陈宣帝陈顼嫡长子,皇后柳敬言所生。

虽然身为太子,但是其皇位却来得十分不易。陈宣帝的次子、陈叔宝的弟弟陈叔陵一直有篡位之心,谋划刺杀陈叔宝。宣帝去世后,叔宝在宣帝灵柩前大哭,叔陵趁机用磨好的刀砍击叔宝,击中颈部,但没有造成致命伤害,叔宝在左右的护卫下逃出,派大将萧摩诃讨伐叔陵。最后叔陵被杀,叔宝即皇帝位,就是陈朝末代皇帝—陈后主。

在位时大建宫室,生活奢侈,不理朝政,日夜与妃嫔、文臣游宴,制作艳词,隋军南下时,自恃长江天险,不以为然。

陈叔宝是一个荒淫無度的皇帝,“奏伎縱酒,作詩不輟”(《南史·陳本紀》),又大建宫室,滥施刑罚,寵愛美女張麗華,朝政极度腐败。張貴妃名叫張麗華,十歲時,充當龔貴嬪的侍女,陳後主一見鍾情,封為貴妃,視為至寶,以至臨朝之際,百官奏事,都讓張麗華坐於膝上或將其抱在懷裡,同決天下大事。特別是張麗華為他生下兒子陳渊之後,使陳後主對張麗華這個傾國傾城的美人,寵愛萬分,在他心目中的地位更加提高、鞏固,陳渊也立刻被立為太子。

陳後主即位時,隋朝的隋文帝楊堅正大舉任賢納諫,減輕賦稅,整飭軍備,消除奢靡之風。隨時準備攻略江南富饒之地,而陳後主竟然奢侈荒淫無度,臣民也流於逸樂,給隋朝以可乘之機。陳後主除寵愛張貴妃之外,還有龔貴嬪、孔貴嬪,還有王、李二美人,還有張、薛二淑媛,還有袁昭儀、何婕妤、江修容等美人。當時陳後主在光照殿前,又建「臨春」、「結綺」、「望仙」三閣,高聳入雲,其窗牖欄檻,都以沉香檀木來做,極盡奢華,宛如人間仙境。陳後主自居臨春閣,張麗華住結綺閣,龔孔二貴嬪同住望仙閣。三閣都有凌空銜接的覆道,陳後主往來於三閣之中,左右逢源,得其所哉!妃嬪們或臨窗靚裝,或倚欄小立,風吹袂起,飄飄焉若神仙。此外陳後主更把中書令江總,以及陳暄、孔范、王瑗等一般文學大臣一齊召進宮來,清歌妙舞,飲酒賦詩,自夕達旦。

隋文帝开皇八年三月,下诏:“天之所覆,无非朕臣,每关听览,有怀伤恻。可出师授律,应机诛诊,在期一举,永清吴越。”于是发兵五十一万八千人,由晋王杨广指揮,進攻陈朝都城建康。晋王杨广由六合出发,秦王杨俊由襄阳顺流而下,清合公杨素由永安誓师,荆州刺史刘思仁由江陵东进,蕲州刺史王世绩由蕲春发兵,庐州总管韩擒虎由庐江急进,还有吴州总管贺若弼及青州总管燕荣分别由庐江与东海赶来会师。陈叔宝恃长江天险,不以为意,只顧與張麗華飲酒玩樂,縱慾淫樂,且说:“王气在此,齐兵三度来,周兵再度至,无不摧没。虏今来者必自败。”翌年(陈祯明三年,即隋开皇九年,589年)正月,南征之戰隋军分道攻入建康(今江苏南京)。其中韩擒虎亲率五百名精锐士卒自横江夜渡采石矶,紧接着贺若弼攻拔京口,形成两路夹击,最先进入朱雀门的是韩擒虎。当时陈后主惊荒失措,他身边的侍臣只有尚書左僕射袁憲一人。當時袁憲建議:「臣願陛下正衣冠,御前殿,依梁武見侯景故事。」(《陳書·袁憲傳》)但陈后主不理会,只说:“非唯朕无德,亦是江南衣冠道尽,吾自有计,卿等不必多言!吾自有计。”与爱妃张丽华、孔贵嬪避入井中,后被俘,隋軍一面掃蕩残敵,令後主手書招降陈朝未降将帅,一面收圖籍,封府庫,又将張麗華、孔貴嬪梟首於青溪中橋及施文慶、沈客卿、阳慧朗、徐析、暨慧景等奸佞斬於右闕下。陳朝宣告覆亡,隋文帝終於统統一了全國。长达四百多年的魏晋南北朝时代结束,中國進入大一統的隋朝。

楊堅對陳叔寶極為優待,准許他以三品官員身分上朝。又常邀請他參加宴會,恐他傷心,不奏江南音樂,而後主卻從未把亡國之痛放在心上。一次,監守他的人報告文帝說:「陳叔寶表示,身無秩位,入朝不便,願得到一個官號。」文帝嘆息說:「陳叔寶全無心肝。」監守人又奏:「叔寶常酗酒致醉,很少有清醒的時候。」隋文帝讓後主節酒,過了不久又說:「由着他的性子喝吧,不這樣,他怎樣打發日子呀!」過了一些時候,隋文帝又問後主有何嗜好,回答說:「好食驢肉。」問飲酒多少,回答說:「每日與子弟飲酒一石。」讓隋文帝相當驚訝。

隋文帝東巡邙山,後主奉召前往,他在宴會上賦詩說:「日月光天德,山川壯帝居,太平無以報,願上東封書。」表請封禪,隋文帝不許。楊堅評價說:「陳叔寶的失敗皆與飲酒有關,如將作詩飲酒的功夫用在國事上,豈能落此下場!當賀若弼攻京口時,邊人告急,叔寶正在飲酒,不予理會;高熲攻克陳朝宮殿,見告急文書還在床下,連封皮都沒有拆,真是愚蠢可笑到了極點,陳亡也是天意呀!」

陈叔宝死于隋仁寿四年(604年),得年五十二岁。追贈大將軍、長城縣公。諡號煬。

有诗《玉树后庭花》传世:“麗宇芳林對高閣,新裝豔質本傾城;映戶凝嬌乍不進,出帷含態笑相迎,妖姬臉似花含露,玉樹流光照後庭。”

據史記載,陳後主某日到沈婺華處,暫入即還,卻寫了一首詩《戲贈沈后》:「留人不留人,不留人去也。此處不留人,自有留人處。」婺華《答後主》:「誰言不相憶,見罷倒成羞。情知不肯住,教遣若為留。」

陳後主詩:「考差蒲未齊,沈漾若浮綠,朱鷺戲蘋藻,徘徊流澗曲。澗曲多巖樹,逶迤復繼續,振振難以明,湯湯今又矚。」

\subsubsection{至德}

\begin{longtable}{|>{\centering\scriptsize}m{2em}|>{\centering\scriptsize}m{1.3em}|>{\centering}m{8.8em}|}
  % \caption{秦王政}\
  \toprule
  \SimHei \normalsize 年数 & \SimHei \scriptsize 公元 & \SimHei 大事件 \tabularnewline
  % \midrule
  \endfirsthead
  \toprule
  \SimHei \normalsize 年数 & \SimHei \scriptsize 公元 & \SimHei 大事件 \tabularnewline
  \midrule
  \endhead
  \midrule
  元年 & 583 & \tabularnewline\hline
  二年 & 584 & \tabularnewline\hline
  三年 & 585 & \tabularnewline\hline
  四年 & 586 & \tabularnewline
  \bottomrule
\end{longtable}

\subsubsection{祯明}

\begin{longtable}{|>{\centering\scriptsize}m{2em}|>{\centering\scriptsize}m{1.3em}|>{\centering}m{8.8em}|}
  % \caption{秦王政}\
  \toprule
  \SimHei \normalsize 年数 & \SimHei \scriptsize 公元 & \SimHei 大事件 \tabularnewline
  % \midrule
  \endfirsthead
  \toprule
  \SimHei \normalsize 年数 & \SimHei \scriptsize 公元 & \SimHei 大事件 \tabularnewline
  \midrule
  \endhead
  \midrule
  元年 & 587 & \tabularnewline\hline
  二年 & 588 & \tabularnewline\hline
  三年 & 589 & \tabularnewline
  \bottomrule
\end{longtable}



%%% Local Variables:
%%% mode: latex
%%% TeX-engine: xetex
%%% TeX-master: "../../Main"
%%% End:



%%% Local Variables:
%%% mode: latex
%%% TeX-engine: xetex
%%% TeX-master: "../../Main"
%%% End:
