%% -*- coding: utf-8 -*-
%% Time-stamp: <Chen Wang: 2019-12-23 14:35:03>

\subsection{后主\tiny(582-589)}

\subsubsection{生平}

陈叔宝(553年12月或554年1月-604年),字元秀,小字黄奴。南北朝时期陈朝末代皇帝(第五代,582年—589年在位),史称“后主”,在位7年,年号至德、祯明。

陈叔宝出生于梁朝承圣二年十一月(553年12月21日或,554年1月18日),是陈宣帝陈顼嫡长子,皇后柳敬言所生。

虽然身为太子,但是其皇位却来得十分不易。陈宣帝的次子、陈叔宝的弟弟陈叔陵一直有篡位之心,谋划刺杀陈叔宝。宣帝去世后,叔宝在宣帝灵柩前大哭,叔陵趁机用磨好的刀砍击叔宝,击中颈部,但没有造成致命伤害,叔宝在左右的护卫下逃出,派大将萧摩诃讨伐叔陵。最后叔陵被杀,叔宝即皇帝位,就是陈朝末代皇帝—陈后主。

在位时大建宫室,生活奢侈,不理朝政,日夜与妃嫔、文臣游宴,制作艳词,隋军南下时,自恃长江天险,不以为然。

陈叔宝是一个荒淫無度的皇帝,“奏伎縱酒,作詩不輟”(《南史·陳本紀》),又大建宫室,滥施刑罚,寵愛美女張麗華,朝政极度腐败。張貴妃名叫張麗華,十歲時,充當龔貴嬪的侍女,陳後主一見鍾情,封為貴妃,視為至寶,以至臨朝之際,百官奏事,都讓張麗華坐於膝上或將其抱在懷裡,同決天下大事。特別是張麗華為他生下兒子陳渊之後,使陳後主對張麗華這個傾國傾城的美人,寵愛萬分,在他心目中的地位更加提高、鞏固,陳渊也立刻被立為太子。

陳後主即位時,隋朝的隋文帝楊堅正大舉任賢納諫,減輕賦稅,整飭軍備,消除奢靡之風。隨時準備攻略江南富饒之地,而陳後主竟然奢侈荒淫無度,臣民也流於逸樂,給隋朝以可乘之機。陳後主除寵愛張貴妃之外,還有龔貴嬪、孔貴嬪,還有王、李二美人,還有張、薛二淑媛,還有袁昭儀、何婕妤、江修容等美人。當時陳後主在光照殿前,又建「臨春」、「結綺」、「望仙」三閣,高聳入雲,其窗牖欄檻,都以沉香檀木來做,極盡奢華,宛如人間仙境。陳後主自居臨春閣,張麗華住結綺閣,龔孔二貴嬪同住望仙閣。三閣都有凌空銜接的覆道,陳後主往來於三閣之中,左右逢源,得其所哉!妃嬪們或臨窗靚裝,或倚欄小立,風吹袂起,飄飄焉若神仙。此外陳後主更把中書令江總,以及陳暄、孔范、王瑗等一般文學大臣一齊召進宮來,清歌妙舞,飲酒賦詩,自夕達旦。

隋文帝开皇八年三月,下诏:“天之所覆,无非朕臣,每关听览,有怀伤恻。可出师授律,应机诛诊,在期一举,永清吴越。”于是发兵五十一万八千人,由晋王杨广指揮,進攻陈朝都城建康。晋王杨广由六合出发,秦王杨俊由襄阳顺流而下,清合公杨素由永安誓师,荆州刺史刘思仁由江陵东进,蕲州刺史王世绩由蕲春发兵,庐州总管韩擒虎由庐江急进,还有吴州总管贺若弼及青州总管燕荣分别由庐江与东海赶来会师。陈叔宝恃长江天险,不以为意,只顧與張麗華飲酒玩樂,縱慾淫樂,且说:“王气在此,齐兵三度来,周兵再度至,无不摧没。虏今来者必自败。”翌年(陈祯明三年,即隋开皇九年,589年)正月,南征之戰隋军分道攻入建康(今江苏南京)。其中韩擒虎亲率五百名精锐士卒自横江夜渡采石矶,紧接着贺若弼攻拔京口,形成两路夹击,最先进入朱雀门的是韩擒虎。当时陈后主惊荒失措,他身边的侍臣只有尚書左僕射袁憲一人。當時袁憲建議:「臣願陛下正衣冠,御前殿,依梁武見侯景故事。」(《陳書·袁憲傳》)但陈后主不理会,只说:“非唯朕无德,亦是江南衣冠道尽,吾自有计,卿等不必多言!吾自有计。”与爱妃张丽华、孔贵嬪避入井中,后被俘,隋軍一面掃蕩残敵,令後主手書招降陈朝未降将帅,一面收圖籍,封府庫,又将張麗華、孔貴嬪梟首於青溪中橋及施文慶、沈客卿、阳慧朗、徐析、暨慧景等奸佞斬於右闕下。陳朝宣告覆亡,隋文帝終於统統一了全國。长达四百多年的魏晋南北朝时代结束,中國進入大一統的隋朝。

楊堅對陳叔寶極為優待,准許他以三品官員身分上朝。又常邀請他參加宴會,恐他傷心,不奏江南音樂,而後主卻從未把亡國之痛放在心上。一次,監守他的人報告文帝說:「陳叔寶表示,身無秩位,入朝不便,願得到一個官號。」文帝嘆息說:「陳叔寶全無心肝。」監守人又奏:「叔寶常酗酒致醉,很少有清醒的時候。」隋文帝讓後主節酒,過了不久又說:「由着他的性子喝吧,不這樣,他怎樣打發日子呀!」過了一些時候,隋文帝又問後主有何嗜好,回答說:「好食驢肉。」問飲酒多少,回答說:「每日與子弟飲酒一石。」讓隋文帝相當驚訝。

隋文帝東巡邙山,後主奉召前往,他在宴會上賦詩說:「日月光天德,山川壯帝居,太平無以報,願上東封書。」表請封禪,隋文帝不許。楊堅評價說:「陳叔寶的失敗皆與飲酒有關,如將作詩飲酒的功夫用在國事上,豈能落此下場!當賀若弼攻京口時,邊人告急,叔寶正在飲酒,不予理會;高熲攻克陳朝宮殿,見告急文書還在床下,連封皮都沒有拆,真是愚蠢可笑到了極點,陳亡也是天意呀!」

陈叔宝死于隋仁寿四年(604年),得年五十二岁。追贈大將軍、長城縣公。諡號煬。

有诗《玉树后庭花》传世:“麗宇芳林對高閣,新裝豔質本傾城;映戶凝嬌乍不進,出帷含態笑相迎,妖姬臉似花含露,玉樹流光照後庭。”

據史記載,陳後主某日到沈婺華處,暫入即還,卻寫了一首詩《戲贈沈后》:「留人不留人,不留人去也。此處不留人,自有留人處。」婺華《答後主》:「誰言不相憶,見罷倒成羞。情知不肯住,教遣若為留。」

陳後主詩:「考差蒲未齊,沈漾若浮綠,朱鷺戲蘋藻,徘徊流澗曲。澗曲多巖樹,逶迤復繼續,振振難以明,湯湯今又矚。」

\subsubsection{至德}

\begin{longtable}{|>{\centering\scriptsize}m{2em}|>{\centering\scriptsize}m{1.3em}|>{\centering}m{8.8em}|}
  % \caption{秦王政}\
  \toprule
  \SimHei \normalsize 年数 & \SimHei \scriptsize 公元 & \SimHei 大事件 \tabularnewline
  % \midrule
  \endfirsthead
  \toprule
  \SimHei \normalsize 年数 & \SimHei \scriptsize 公元 & \SimHei 大事件 \tabularnewline
  \midrule
  \endhead
  \midrule
  元年 & 583 & \tabularnewline\hline
  二年 & 584 & \tabularnewline\hline
  三年 & 585 & \tabularnewline\hline
  四年 & 586 & \tabularnewline
  \bottomrule
\end{longtable}

\subsubsection{祯明}

\begin{longtable}{|>{\centering\scriptsize}m{2em}|>{\centering\scriptsize}m{1.3em}|>{\centering}m{8.8em}|}
  % \caption{秦王政}\
  \toprule
  \SimHei \normalsize 年数 & \SimHei \scriptsize 公元 & \SimHei 大事件 \tabularnewline
  % \midrule
  \endfirsthead
  \toprule
  \SimHei \normalsize 年数 & \SimHei \scriptsize 公元 & \SimHei 大事件 \tabularnewline
  \midrule
  \endhead
  \midrule
  元年 & 587 & \tabularnewline\hline
  二年 & 588 & \tabularnewline\hline
  三年 & 589 & \tabularnewline
  \bottomrule
\end{longtable}



%%% Local Variables:
%%% mode: latex
%%% TeX-engine: xetex
%%% TeX-master: "../../Main"
%%% End:
