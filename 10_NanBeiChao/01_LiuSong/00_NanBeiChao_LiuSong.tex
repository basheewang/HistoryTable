%% -*- coding: utf-8 -*-
%% Time-stamp: <Chen Wang: 2019-12-19 17:28:49>


\section{刘宋\tiny(420-479)}

\subsection{简介}

宋(420年-479年)史稱劉宋或稱南宋(跟其他南朝政權,南齊、南梁及南陳看齊,然而為了避免跟趙氏南遷的政權混淆,大部分人會多用前者劉宋為主)是中国歷史上南北朝时期南朝的第一个朝代,也是南朝版圖最大的朝代,當時所謂「七分天下,而有其四」。439年,北魏統一中国北方後,與劉宋形成南北對峙。劉宋强盛时,其统治地区北以秦岭、黄河与北魏相邻,西至四川大雪山,西南包括云南,南至越南中部横山、林邑一带。

420年,宋武帝刘裕取代东晋政权而建立。国号宋,定都建康(今江苏省南京市),因国君姓刘,为与后来赵匡胤建立的宋朝相区别,故又称为刘宋。

以刘裕世居彭城为春秋时宋国故地,故以此为国号。又以五德終始說,刘宋为水德,故别称水宋。

开国皇帝刘裕出于行伍,自幼家贫。时值东晋末期,民變此起彼伏,朝廷内部斗争也十分激烈。402年,东晋大将桓玄乘朝廷实力虚弱,起兵篡位,国号“楚”。刘裕与刘毅等起兵勤王,并最终消灭了桓玄的力量。此后,刘裕率军南征北伐,其势力不断得到稳固壮大,并先后攻灭刘毅、司马休之等实力派,最终迫使晋恭帝将帝位禅让给他,420年劉裕建宋,年號永初。刘裕即位後,因為他已在晉末實行各種改革如土斷、壓制豪強、澄清吏治、強化軍隊等等,所以在位的三年中,除了恢復漢代舉孝廉、策秀孝的制度並強化官僚法制之外,主要政策仍是在休養生息、恢復晋末北伐的國力損傷,並計畫在422年出征北魏。結果422年五月刘裕得病驾崩,北伐取消。

422年劉裕死,太子劉義符即位,北魏趁機派十萬大軍南侵,占領洛陽等河南地區,逼退宋將檀道濟。424年,徐羨之、謝晦等託孤輔政大臣,害怕失德無禮的劉義符會敗壞國政,遂以「廢昏立明」為名號,廢殺劉義符,改立劉裕的第三個兒子劉義隆為皇帝,是為宋文帝,年號元嘉。宋文帝在426年除掉權臣徐羨之等人後親政,他在位三十年,励精图治、知人善任、提倡節儉並澄清吏治,国家生产经济因此大力提升,被稱為元嘉之治,為六朝治世之典範(也是江東第一個治世)。

430年起,宋文帝首次北伐,到彥之率領的五萬宋軍,成功占領河南洛陽等地。但由于軍力不足,加上文帝的過度指挥,以致北魏逼退宋軍數萬主力後,於431年重佔河南。436年名将檀道濟因军功被宋文帝猜忌而被铲除,又使南朝宋失去能與北魏制衡的大將。445-446年當北魏發生蓋吳起事時,南朝宋沒能即時北伐。到445年時,北魏趁勁敵柔然暫衰時開始發動多次小規模南征,雙方於淮南來回拉鋸,元嘉二十七年(450年),宋文帝出兵六萬北伐北魏之河南地(二次北伐),但卻被北魏太武帝拓跋燾所之率六十萬大軍正面擊敗,六十萬魏軍遂引兵南下,威逼建康。魏軍所過之處大肆搶掠燒殺,江淮地區損失慘重、「邑里蕭條」,元嘉之治因此衰落。宋文帝在452年趁拓跋燾遇弒之機會,派軍進行第三次北伐,但仍無功而返,此後劉宋無力再舉,註定日後國防線逐漸南撤的命運。

453年,宋文帝长子刘劭發生巫蠱事件,弒父即位,其三弟劉駿起義兵攻劉劭,獲得各方軍鎮的支持,於是斬劉劭於台城,劉駿自立為帝,是為宋孝武帝。

宋孝武帝統治期間,雖有諸王劉義宣、劉誕等相繼叛亂,但大多很快平定,和北魏的戰事也只限於山東半島,雖小勝北魏但影響不大,因此總體來說,孝武帝統治的十一年算是個相對安定的時期,孝武帝的積極政策也促進了江南的開發與貨幣經濟的深化。一直到463年底至464年,浙江等地發生大旱災,造成慘重的大饑荒,浙江十分之六的戶口餓死逃散。

宋孝武帝是一個頗有作為、積極改革制度的皇帝。他加強中央集權,撤除「錄尚書事」職銜,並分割州、郡以削弱藩鎮實力,並開始以中書舍人戴法興、巢尚之等人處理中樞機要事務,形成後代所謂「寒人掌機要」的政治局面,孝武帝的集權化統治也被史書稱為「主威獨運,官置百司,權不外假」。孝武帝同時重用江東寒門沈慶之與傖荒北人柳元景,依照兩人的功績,先後提拔為三公,開啟吳興沈氏與河東柳氏攀升為南朝高門的起始之路,並開創南朝寒門、寒人以軍功升為三公的先例。

孝武帝另外也對門閥制度進行一定程度的整頓,給予世族制度新的生機。除了拔用上述的沈、柳為三公之高門,更提拔孤寒衰微的袁粲為員外散騎侍郎和侍中;拔擢寒門的顏竣、寒素的顏師伯成為高官重臣;任用南北之望的名士琅邪王彧、會稽孔覬為散騎常侍,一度矯正過去散騎常侍受人輕視的不良習慣;甚至從461年開始,把與商人等通婚、私下經商的士族,開除士族資格並黜為將吏,是為檢籍政策的先聲。

464年夏季,孝武帝死,其子劉子業繼位,荒淫残暴,朝廷內外人情汹汹,心懷恐懼,劉子業不久被湘東王劉彧弒殺。劉彧在建康自立為帝之後(宋明帝),因為得位不正,面臨孝武帝第三子劉子勛登基為帝、聯合兄弟方鎮圍攻建康的艱鉅情勢。宋明帝政權雖然領土、人口都不到劉子勛政權的十分之一,但是以伐亂為名,憑藉量少質精的中央軍,採取各種積極手段:採用才幹名士蔡興宗的意見,撫慰叛亂將士在京師的親人,安定人心。重用沈攸之、張永、蕭道成等才幹武將。放權給諸弟劉休仁等人積極平亂。

於是上下一心、兵強將勇,因此打敗劉子勛並平定江南與淮南各地區,最後全面誅殺孝武帝子孫。但是淮北方鎮薛安都等人為了自保而向北魏求援,於是北魏大軍在四年之內陸續攻下淮北、山東半島地區,劉宋戰亂不斷,國力大衰,人民痛苦指數飆升。又因為必須對有功的軍人加官晉爵、大肆封賞,於是造成士族制度的嚴重破壞,清濁不分、官品淆亂。

472年,宋明帝死,太子劉昱繼立,宋明帝遺詔命蔡興宗、袁粲、褚淵、劉勔、沈攸之五人託孤顧命大臣,分別掌控內外重區,另外命令蕭道成為衛尉,參掌機要。其中遺詔雖任命袁粲、褚淵在中央秉政,但實際上接受宋明帝秘密遺命,就近輔佐新帝劉昱,掌控宮中內外大權的人物,是宋明帝最親信的側近權倖——王道隆與阮佃夫二人。

宋明帝在死前,為了穩固兒子的皇位,大肆誅除有能力的皇弟宗室、功臣武將和高門士族,造成劉昱繼位後中央和地方軍鎮互相猜忌、攻伐的政治亂象,使得武將蕭道成因此崛起,逐漸掌握中央軍權。特別是474年,桂陽王劉休範以清君側之名造反,殺死了權倖王道隆與顧命大將劉勔,幾乎就要攻下建康城,但蕭道成即時回軍,平定亂事。事後蕭道成接替劉勔的地位,上升為與宰相袁粲並列的「四貴」之一,更受到權倖阮佃夫的倚重,因此交結地方軍鎮都督,權勢日漸擴大。476年,文帝在世长孙建平王刘景素在京口起兵,亦被萧道成等镇压。477年,年滿15歲的劉昱在殺掉權臣阮佃夫後,與蕭道成發生激烈衝突,但卻意外被蕭道成弒殺,蕭道成趁機改立明帝第三子劉準為皇帝,即宋順帝。蕭道成獨攬軍政大權後,挾持軟弱的褚淵,以武力平定忠宋大臣袁粲、沈攸之的起義。479年,年幼的宋顺帝劉準把帝位禅让给了蕭道成,宋被南齐所取代。當時民間以一首歌謠傳述蕭道成殺袁粲篡宋的事業:「可憐石頭城,寧為袁粲死,不作褚淵生!」

從宋武帝劉裕時代(420-422年)以黃河為界,與北魏對峙的局面,到劉裕死後的423年,北魏趁機攻下河南三鎮(洛陽、虎牢、滑台),從項城到濟南大致形成一條國界線,分開北魏與劉宋,劉宋仍保有山東半島與江蘇省北部的淮北地區。但與北魏為界的大塊面積,因為屢遭兩方進攻掠奪,成為所謂「邊荒」地區,僅剩居民聚集在榛木所圍成的山寨堡壘,被稱為「榛人」。項城到濟南的「邊荒」國界維持了長達四十三年,四十三年中兩國大軍屢次越界征伐,但屢得屢失,兩方都無法把佔領的土地長久地穩固下來,因此維持了四十多年勢均力敵的局面。

一直到466年,劉宋山東、淮北的鎮將薛安都、崔道固等人,因為害怕篡位自立的宋明帝討伐他們,而向北魏求援。北魏趁機派五萬以上的大軍,於打敗宋明帝的北征軍張永之後,陸續在四年內攻下了山東、淮北的所有城鎮,劉宋被迫以淮河與北魏為界。雖然宋明帝心有不甘,屢次派沈攸之等人北征,但從此(469年後)南朝與北魏的淮河國界就大致固定下來,一直到31年後(500年南齊末)才因為壽陽鎮將裴叔業投降北魏,使國界進一步往南退。

刘宋的行政区划袭承东晋,实行州、郡、县三级制。

州是第一级行政区。州的最高行政长官称刺史,劉宋一朝的州數大致在二十州上下,至宋末穩定為二十二州。其中不少州是僑寓州,為寄住在南方州郡上,不一定有實土。

尹、郡、王国、公国(部分)是第二级行政区。尹的最高行政长官称尹,郡的最高行政长官称太守,王国的最高行政长官称内史,公国的最高行政长官称相。

县、公国(部分)、侯国、伯国、子国、男国是第三级行政区。县的最高行政长官称令或长,公国、侯国、伯国、子国、男国的最高行政长官称相。

劉宋前期繼承「東晉門閥政治」的地理格局,以荊州(鎮江陵)和南徐州(鎮京口)為核心軍鎮,所以劉裕規定兩州必由劉氏宗王擔任刺史。其中荊州因為州大民多、「地廣兵強」,又統攝雍、南梁、益等州,支撐劉宋西半的安危,故有「分陝」之稱,劉裕遺詔說荊州刺史需「諸子次第居之」,說明荊州的重要性略高於北府兵根本的南徐州。劉宋宗王擔任荊州刺史的結果,是促使荊州士族與揚州士族的合流,大致結束東晉百年荊、揚對立的局面,代表人物即是江陵士族劉柳、劉湛父子,曾各自取得相當於副宰相的官位與權勢。

中期因為454年荊州刺史劉義宣,趁著新帝劉駿即位的弱勢格局,發動十萬荊州鎮軍挑戰建康,因此劉駿在同年平定劉義宣之亂後,即刻從荊州東部分出新的一州名郢州,並廢除荊州重兵來源的南蠻校尉,其營戶兵力遷至建康,有效地削弱荊州,瓦解其「分陝」地位。之後劉駿又土斷雍州,大幅強化雍州的實力,不但讓原來的大荊州地域,陷入荊、雍、郢三州相互牽制的局面,後來隨著雍州軍力的不斷加強,至宋末沈攸之起義失敗之後(478年),「江陵素畏襄陽人」的局面已大致形成。

劉宋選官制度仍以九品中正制為主,但宋初門閥制度的整體格局,卻是從東晉末年義熙時代(405-419年)劉裕等京口北府的軍事集團崛起開始,延續繼承下來的新格局。也就是說,大量京口將領混入世族門閥的結構中(多成為中下層世族),擠壓了原來名門舊族的地位與空間,因此南朝士族從劉宋開始,常會刻意去排擠寒門、寒人,好顯示自己的清高地位。於是高門就有許多「士庶區別,國之章也」、「士庶之際,實自天隔」的言論出現,這實際上是名門舊族的一種防禦性反應。

雖然宋初就有武人將領合法地進入世族結構中,但劉宋前半的元嘉之治,士族制度卻是極其完備的,一般也認為是文化士族的「全盛期」(同時也是「最後的榮光時期」)。在宋文帝治下,史稱其「綱維備舉,條禁明密,罰有恆科,爵無濫品。故能內清外晏,四海謐如也」,當時的江南社會:「閭閻之間,講誦相聞;士敦操尚,鄉恥輕薄。江左風俗,於斯為美,後之言政治者,皆稱元嘉焉」。這是因為宋文帝除了重用並放權給兼具才幹與名望的風雅士族,如王華、王曇首、殷景仁、謝弘微、劉湛、范曄、江湛、王僧綽等,更重要的是,文帝也能夠尊重王敬弘、王球這一類缺乏理政才幹的高門清望,雖不給實權,但仍任命為副宰相與吏部尚書,在用人上保持住門閥制度的清濁流品,因此能激清揚濁,使得「士敦操尚,鄉恥輕薄」。

劉宋中期的門閥制度,雖然因為450年宋文帝大舉北伐失敗後的困局,使寒人得以竄改籍注或詐列士籍,混亂士族的流品,但在宋孝武帝的整頓革新之下,仍使門閥制度獲得一定的生機。一直到465年宋明帝自立為帝後,才因為廣募部曲、濫賞軍功,造成士族制度嚴重破壞,成為劉宋滅亡的重要原因。

劉宋前期為對付北魏,積極聯絡柔然、胡夏、北燕、高句麗、吐谷渾,希望對北魏包夾圍攻,但都被魏軍以優勢機動力各個擊破,無法發揮包夾的效果。其中劉宋與柔然的聯盟關係最穩固也最持久(延續到南齊),對北魏的危害也最大,因此北魏常要先北向摧毀柔然的主力,然後才敢在隔年大舉南伐劉宋。

劉宋與北魏的對峙,除了幾次大規模的戰爭衝突以外,其實一半以上的時間,雙方保持相對和平的外交關係。雖然劉宋稱北魏為「索虜」、北魏稱南朝為「島夷」,但他們仍不定時的互派使者「交聘」,維持南北的交涉往來。南北通使往來,在南北史書上的記載雖然各有偏頗扭曲,如魏書記載,421年劉裕派沈範、索季孫等到北魏「朝貢」(宋書記為「報使」),但是實際上是一種平等的對等關係。而且使者代表國家,南北的競爭不只是軍事武力的競爭,文化與氣度上也有互別高低的意味,從劉宋中期開始,南北雙方開始精選使者的素養氣質,如清代史家趙翼所稱「必妙選行人,擇其容止可觀,文學優瞻者,以充聘使」。如果出使有失國體,使者回國後則會被嚴加懲處,這多發生在北魏前期文化素養不高的條件下。劉宋中期時,南北曾有五年的互市貿易。453年宋孝武帝登基後,北魏派使者「求通戶市」,宋孝武帝在與公卿大臣廣泛議論後,決定答應互市。兩國官方的貿易關係,大約持續到458年邊境發生小規模戰事而止。

劉宋與東北亞的倭國、百濟則有密切的朝貢關係,倭國曾在晉末、劉宋對江東朝貢十多次,史書記載有五位倭王,是為倭五王;百濟在劉宋後期與江東的關係更形緊密,似乎結成軍事同盟,共抗北魏。因為469年北魏完全奪取劉宋在山東半島的城鎮後,百濟因為早前渡海在遼西或山東半島沿岸設有港岸據點,因此與南朝同樣面臨北魏的軍事壓力,479年宋齊易代之後,北魏還曾在488年派軍進攻過百濟的城鎮,卻被百濟打敗。令外又因為百濟與倭國對於朝鮮半島東南部的伽倻(任那)可能有爭奪關係,因此兩國在對劉宋的朝貢外交中常常是互相牽制的對立關係。也因為兩國都想要討好劉宋,所以頻頻向劉宋朝貢示好。譬如倭國在朝貢時,一直希望劉宋冊封倭王為「都督」百濟在內的大將軍,讓倭王有統治百濟的名分,但要求總是被劉宋拒絕。

高句麗因為是朝鮮半島與東北亞中最強的國家(更是東亞世界中僅次於魏、宋的第三強國),所以對劉宋的朝貢關係不太緊密,兩國關係主要是針對北魏而結成的鬆散軍事聯盟。高句麗常會單方面中斷對劉宋的朝貢,說明劉宋對高句麗只存在形式上且薄弱的君臣關係。不過在465年之前,因為劉宋強盛的水軍能夠在渤海沿岸執行任務,相對於遠離海岸且毫無水軍的北魏勢力,高句麗在465年前一直選擇劉宋作交往、朝貢的對象。劉宋曾在438年派將領王白駒,率水軍七千人渡海到高句麗的遼東,想迎接兩前年滅國的北燕主馮弘來到南方,結果高句麗先把馮弘處死,並派兵把王白駒繳械,強制遣送王白駒等回劉宋,隔年再回送八百匹馬給宋作賠禮;到了465年後,可能因為當年劉宋發生劇烈的內鬥(見劉子勛條),高句麗從此改對北魏進行較為緊密的朝貢關係,並在469年後長期疏遠南朝(北魏在469年攻下山東半島),只和南朝保存微弱的外交聯繫。475年高句麗更大破劉宋的盟友百濟,破其國都、殺百濟王,佔領漢江流域,國力達到極盛。這次劉宋沒有再派出水軍到遼東、朝鮮,說明劉宋的無力干涉與衰落。

劉宋交州(越南北部)的南邊與林邑國(今越南之中南部)接鄰。東晉末期,林邑有數年的內亂,劉宋建國元年420年,交州刺史杜慧度派兵萬人南征林邑,林邑請降,並向宋廷致送大象、金銀、古貝等禮物。421年,林邑王陽邁一世遣使到宋廷入貢,並獲宋武帝冊封。但到陽邁二世時,於427年入侵日南、九德等郡。431年,林邑入貢宋廷。432年,陽邁二世派水兵入侵九真,交州刺史阮彌之派軍抵抗,驅逐至區粟而回。433年,陽邁二世遣使到宋廷,要求「領交州」,宋廷不許,陽邁二世因此大為憤恨,雖常遣使入貢,但亦常派兵入侵交州。

446年,宋文帝派龍驤將軍交州刺史檀和之、太尉府振武將軍宗愨等征討林邑。戰前,文帝提示檀和之,倘若林邑國能夠誠心求和,便可答允。檀和之派人向陽邁二世諭以恩信時,陽邁二世竟加以扣留,於是雙方進行交戰。林邑軍先以大象軍取得首勝,後來宗愨提議用獅子的外型去威嚇大象,可以取勝。主帥檀和之採納計策,果然大敗林邑軍。宗愨部隊更一舉攻克首都林邑(Campapura),擄獲無數珍寶、黃金數十萬斤,陽邁二世出逃。此一征戰令林邑元氣大傷,「家國荒殄,時人靡存」。此後,林邑國沒有再起兵進犯交州,對劉宋甚為恭順,多次遣使到建康訪問進貢。

自晉室南遷之後,苟延殘喘地偏安江南。原本居於華北的漢人氏族為了逃難而向南遷徙,大量來自中原的移民士族改變了江南地區的人文景觀,甚至口頭語言也逐漸與古河洛語言接軌。[來源請求]

南朝宋時期,主要把土著蠻夷分成蠻人、俚人、僚人三種類型,三者有時被通稱為「南蠻」。蠻人在長江流域以板循蠻、盤瓠蠻與廩君蠻实力最大,板循蠻又稱賨人,原居益州巴郡閬中一帶,之後經渝水北遷漢中、關中。廩君蠻原在益州巴郡、荊州江陵一帶,後來擴展到長江漢水與淮西一帶。史書上提到的巴東蠻、宜都建平蠻都是指廩君蠻。盤瓠蠻又稱「溪人」,發揚地在辰州,分佈現在的湖南與江西一帶。

南朝劉宋政府為了對付蠻人,在荊州置南蠻校尉、在雍州置寧蠻校尉,專責教化及討伐南蠻。為了在荊雍的強大蠻族群體,南朝劉宋政府在440至470年代曾發動大規模地討蠻運動,有兩次的主將分別為雍府大將沈慶之、荊州刺史沈攸之,捕獲數十萬的蠻族人力。而在450年代,沈慶之、王玄謨大致討平淮水蠻,強化了劉宋在淮南地區的國防。

俚族的範圍在南嶺、今貴州南部到海南島、越南北部一帶。468年起李長仁與李叔獻兄弟據交州抵制刘宋朝廷,當時的宋明帝政權因為正與北魏全力爭奪山東、淮北地區,無力征討交州,只好承認李長仁的刺史名號,維持劉宋對交州名義上的統治,並於471年在交、廣兩州交界地新設越州,以防禦李氏兄弟。李氏兄弟很可能具有交州俚僚族群的血統,他們在交州的割據一直維持到485年才被齊武帝討平。

僚人主要分散在四川、漢中的山谷空地,與賨人的分布區頗有重疊。當時「僚人與夏人(漢人)參居者,頗輸租賦」,說明其編戶化與華化的趨勢較重。

劉宋常態兵力大約二、三十萬,極限動員時可能有四十萬,,但劉宋在淮水以北征伐時,因為受限於後勤供應,只能發動五、六萬兵力,使得北伐經常失敗。劉宋前期北府兵獨大,成為中央軍與荊州、北徐州方鎮的主力來源,故此時仍以世兵制為主;中期荊雍兵崛起,逐漸取得一定的優勢,學者田餘慶認為:「北府兵力日衰,荊雍兵力日盛,是同一個歷史過程的兩個方面」。此時世兵制衰落,軍隊主力逐漸被募兵制和徵兵制取代,特別是將領自招部曲的募兵制,更成為宋末軍隊中的精銳核心。譬如469年後流亡南方的青齊豪族,就被蕭道成收納招募為將官、部曲,成為蕭道成建齊易宋的主力。學者有的稱此武力集團為「淮陰集團」,有的稱之為「青徐集團」。

宋孝武帝大明八年(464年)官方紀錄,全國有901,769戶,5,174,074人,但因為十多年前發生北魏破壞江北的燒殺屠掠,江北人口大減,以及463-464年浙江等地發生大飢荒,浙江人口死亡逃散十分之六,所以劉宋盛世年代(元嘉之治)的官方戶口數字,應當超過一百萬戶、六百萬口。

劉宋的江北地區主要是村塢型經濟,常受戰亂影響而發展有限;江南社會主要是莊園經濟。世族與寺院的莊園大部分都是多方經營,從自給自足的性質,朝向商品經濟發展。農田有良好的水利系統供種植稻、麥、粟、桑、麻、蔬菜等作物,還可以種植竹木果樹、養魚、畜牧等等。還有紡織、釀造、生產工具等手工業。世族的莊園生產主要交給佃客、部曲和奴隸,而寺院是一般僧侣與民戶。由地主集中開墾,這對於地區的開發起一定的作用。由於世族享有特權,佛教較為盛行,致使地主莊園與寺院莊園膨脹,並且隱匿許多農戶。

農業是莊園經濟的重心,深受朝廷與世族關切。開墾山林與土地兼併的情形在劉宋一直非常旺盛,朝廷雖有禁令,但難以禁止世族兼併土地或霸占山澤,宋孝武帝大明元年(457年)朝廷乾脆承認佔領山林川澤的法令以限制世族搶佔範圍。法令頒布後果然刺激豪門權貴兼併山澤土地的活動,也因此促進了商品經濟的發展。劉宋相對北魏來說比較安定,南渡的移民在初期與末期仍然絡繹不絕,農業生産繼續有所發展。比較突出的地區有荊、揚二州,而益州居次。揚州是劉宋最發達的地區,其中以建康及其周圍地區發展最大。而三吳地區(吳郡、吳興、義興)是中央財庫、各種支出的主要來源。

由於朝廷大力提倡農桑,戶調征絹布,當時絹布的地位等同貨幣,這些都促進紡織業的生產。劉宋的紡織業與養蠶業比較發達,產地以荊、揚二州為主。由於絲、綿、絹、布等是國家調稅的主要項目,因此紡織是民間普遍的副業。織錦業則在益州為主,劉裕滅後秦,把關中的織錦戶遷到江南,開始在江南發展織錦業。當時富豪人家穿綉裙,著錦履,以彩帛作雜花,綾作服飾,錦作屏障。朝廷設有專官管理礦冶,用水排鼓風冶鑄。鍊鋼則使用一種雜煉生鐵和熟鐵的灌鋼法。這種方法可以鍊出優質鋼,用來製造寶劍和刀。瓷器的燒制技術早在三國、晉朝時期成熟。劉宋時以青瓷為主,產地集中在會稽郡(浙江紹興)。其硬度高,釉料勻,通體青瑩。江南其餘地區的制瓷技術各有自己的特點。劉宋的紙張潔白勻稱,完全取代了簡牘,藤紙與麻紙都很流行。造船業也十分興盛,如宋末沈攸之起義反蕭道成時,荊州作部曾「裝戰艦數百千艘」,而且三吳運河網也持續修造,到南齊時已大致完成,暢通了三吳與建康的交通。

劉宋農業和手工業發達,加上江河交通便利,使得商業日漸發達,江南社會穩定地朝貨幣經濟與商品經濟發展,甚至連江北的漢中地區,也在劉宋中期開始使用貨幣。但由於國家控制的銅礦不足,使得幣制屢變,質量不精。市場上有普通的生產用品、生活用品與奢侈品,商賈小者坐販於列肆,大者轉運於四方,而凡是大批運進的商品買賣,多是世族莊園所生產的經濟作物。商稅是朝廷收入的大宗,然而世族有免關稅權,在任期屆滿時帶著大批貨物作為「還資」,然後轉販各地。商業重鎮有建康、江陵、成都、廣州、广陵等地。建康是三吴的經濟中心。會稽、吳郡、餘杭居次。廣州是海上貿易重鎮,貿易對象有东南亚各國、天竺、獅子國、波斯等國。江陵是關中、豫州、益州、荆州、交州、梁州的轉運站。成都不僅商業繁盛,也是蜀錦的重要產地。

劉宋詩風流行的是元嘉體。元嘉體是宋文帝元嘉年間的詩風,代表人物有「元嘉三大家」謝靈運、顏延之與鮑照。他們的共同功績是把古體詩推進到完全成熟階段,並且注意聲律和對偶的運用,並且逐漸發展出近體詩;袁淑、謝莊亦為有名詩人。民間詩人則以劉宋初期的陶淵明最具代表性,其擅長描述田園生活,風格清新樸實,提升古體詩內涵,表現出高遠純潔的情操。其作品《桃花源記》寓意追求一個可供逃避亂世的和諧世界,富有哲理。其詩歌、散文及辭賦廣泛影響後世名家如王維、李白、杜甫、蘇軾、辛棄疾、陸游等人。

小說受到名士清談的影響,促成軼事小說的出現,最有名的是宗王劉義慶招集文人才士所編寫的《世說新語》,為後世文學作品提供大量典故和成語,也是唐代晉書編撰的重要史料來源。

劉宋繼承了漢代以來設官修史之制。宋設著作官,負責撰修國史(本王朝史)及帝王起居注。宋代最著名的兩本史籍,是范曄的《後漢書》與裴松之的《三國志注》。《後漢書》新增「獨行」、「逸民」(或「隱逸」)、「列女」等類傳記各種人物面貌,最被稱道;裴注著重資料搜集、補充史事,不再局限於對音訓及解釋史文,對中國的注史方法產生有相當影響。裴松之對史料相互考異,日後史家有所繼承,如司馬光撰《資治通鑑考異》。裴注裡又有對前代史家的評論,這推動了中國史學批評的發展。

劉宋時譜學(或叫譜牒學)在門閥社會影響下而開始盛行。各家士族郡望為求鞏固社會地位和政治權利,乃撰修家牒,以彰顯自身血統、門第及婚宦。繼家譜出現後,又有了家譜學的研究,當時便出現「統譜」、「百家譜」等書籍。

劉宋在東晉之後,延續晉代的文化發展。由於玄學流行,老莊的自然觀和江南秀麗的山水結合,使得繪畫脫離儒學的限制,朝向純藝術的方向發展,陸探微為宋明帝時期著名的宮廷畫家,然而其作品均已失傳。由於山水诗的出现,使得长期以来的以表现人物为主的绘画传统转变為山水景色,例如宗炳是中國最早的山水畫理論著述。其《畫山水序》最為著名,精闢地理解「山水以形媚道」之外,在自然山水的觀察,歸納出展現物體遠近的繪畫方法;另外山水畫家王微,著有《敘畫》一篇,強調觀察自然和主觀能動作用。

江南社會的人口很複雜,大致上可分為四個階層:名門豪族的世族;自耕農、新民等從事農工商的編戶齊民;屬於部曲、佃客、衣食客、門生舊故等依附世族的依附人,受政府控管的雜戶、百工戶、兵戶與營戶也是依附人;最後是奴婢、生口、隸戶,這些都屬於奴隸。

雖然文化士族的實力大削,但劉宋仍維持世族社會的結構;而江北豪族的地位與權力雖遜於江南的僑吳士族,但在經濟力與軍事實力方面,卻高出甚多。世族控制的人口有部曲、佃客與奴隸,不經「自贖」或「放遣」,是不能獲得自由的。由於南朝大家族制的衰亡使得部曲逐漸受國家控制。佃客的來源有政府依官品賜給與私自招誘。奴隸的主要來源是破產的農民或是流民,他們是地主的私產,因而可以抵押或買賣。為了防止逃亡,奴隸都被「黥面」。奴隸可以經由「糜喃為客」、「發奴為兵」等方式轉化為地主的佃客和國家的士兵。自耕農是當時農業生產的重要力量。他們對朝廷負擔租調、雜稅、徭役以及兵役,這些都使許多自耕農破產流亡,淪為世族的部曲和佃客。劉宋實行三國以來的世兵制,兵戶世代當兵,平時還需要交納租調。由於手工業者很缺,故官府對雜戶或百工戶的控制極嚴,百工戶從民間徵調到官府作坊後,與配到作坊里的刑徒為伍,終年勞作,世代相襲。如果世族、官僚私佔百工戶往往受到懲治。

江南社會約在晉末宋初由大家庭制轉化為小家庭,在同一家族不同職業的十家就有七八家之多,互相漠視。這是因為宗族發展後各家庭親疏貧富不同,若無共同外患就容易分離;朝廷課稅方式對大家族制無益而導致的。


%% -*- coding: utf-8 -*-
%% Time-stamp: <Chen Wang: 2019-12-19 17:32:15>

\subsection{武帝\tiny(420-422)}

\subsubsection{生平}

宋武帝劉裕(363年4月16日-422年6月26日),字德輿,小字寄奴,彭城綏輿里(今江蘇省徐州市銅山区)人,東晉末年至南北朝初期的軍事家、政治家,南北朝時期劉宋開國皇帝。早年出身十分貧寒,劉裕最初為北府將領孫無終的司馬,在孫恩之亂中展現其軍事才能,及後更發起義軍擊敗篡位的桓玄,恢復了東晉政權,並獲得了極高名望,並在不久之後掌握朝政大權。

劉裕趁南燕内讧之际而出兵滅燕,隨後又平定了盧循之亂,以及消滅了劉毅、諸葛長民及司馬休之等異己,鞏固了在東晉國內的地位。接著又乘後秦内乱而北伐,收復了洛陽及關中地區,受封宋公並得九錫,終篡奪了東晉政權,建立劉宋,正式開始了南北朝時代。

劉裕家族在早年隨晉室南渡,長居京口(今江蘇鎮江市),《宋書》說他是漢高祖劉邦的弟弟楚王劉交第二十一世孫。《魏書》則猜測其祖先可能姓項。劉裕於興寧元年三月壬寅日(363年4月16日)出生,其時家境貧苦,母親更因分娩後疾病去世,父親劉翹無力請乳母給劉裕哺乳,一度打算拋棄他,只因劉懷敬之母伸出援手,養育劉裕,才得以活下來。劉裕早年曾以賣鞋为生,但卻又常賭博樗蒲,傾盡家財,遭鄉里賤視,亦因不修品行而不為當世人們所賞識。不过,刘裕才能出眾,且有大志,當時出身琅琊王氏的王謐就十分敬重他,更曾向他說:「你應當會成為一代英雄。」。

劉裕及後從軍,最初就任北府軍將領、冠軍將軍孫無終的司馬。隆安三年(399年),孫恩起兵反抗晉朝,自海島攻下會稽,三吳各郡皆響應他,孫恩之亂由而爆發。另一北府將領、前將軍劉牢之率軍鎮壓,當時他就請了劉裕參府軍事。

當時劉裕奉命率數十人去偵察敵軍,卻遇上數千人的敵軍並發生戰鬥,雖然所帶的人大多戰死了,但劉裕仍揮動長刀抵抗,殺傷多人。劉牢之子劉敬宣派兵搜尋劉裕,見劉裕獨力抵抗,都讚歎劉裕的能力,並率軍進攻,俘殺一千多人。不久諸軍擊敗孫恩各軍,又攻下會稽郡治所山陰(今浙江紹興市),令孫恩退回海島。

次年(400年)五月,孫恩再襲會稽,殺害駐鎮會稽的謝琰,至十一月時劉牢之率軍前往才擊退孫恩。劉牢之及後命劉裕守句章(今浙江寧波市)。當時句章城小兵弱,而劉裕就常做好作戰準備。翌年(401年)二月孫恩就率眾自浹口(今浙江鎮海)進攻句章,而劉裕就身先士卒,每戰都摧其鋒銳,致令孫恩無法攻下句章,反為劉牢之所敗。三月,孫恩轉戰海鹽(今浙江海鹽縣),劉裕跟隨其進攻方向,於海鹽築城抵抗,又大敗來攻的孫恩。

孫恩後循海路至丹徒(今江蘇鎮江市丹徒區),劉裕率不足千人的部隊趕路,與孫恩同時趕至。當時劉裕軍隊疲累,丹徒守軍亦無鬥志,但面對孫恩來襲,劉裕仍能率眾大敗對方,逼其狼狽登船撤離岸上。孫恩不久轉屯郁洲(今江蘇灌雲縣東北),朝廷以劉裕為建武將軍、下邳太守,討伐孫恩,多次交戰後大破對方,令其勢力轉弱而南撤。劉裕接著追擊,又再敗孫恩,令其再度逃到海島。次年孫恩就被消滅。

元興元年(402年),驃騎大將軍司馬元顯下令討伐荊州刺史桓玄,並以劉牢之為前鋒。桓玄率軍兵臨建康時,劉裕請求進攻,但劉牢之不肯,反而想叛歸桓玄。劉裕當時與何無忌極力諫止但都無效,劉牢之終向桓玄請降,桓玄亦順利消滅司馬元顯的勢力,掌握朝政。

事後桓玄調劉牢之為會稽內史以削其軍權,劉牢之圖據廣陵(今江蘇揚州市)叛桓玄,但劉裕認為人心已去,事必不成而拒絕與劉牢之合作,最終劉牢之因失去僚屬的支持而自殺。桓脩後以劉裕為其中兵參軍,並於同年參與討伐統領孫恩餘黨的盧循、徐道覆。當時桓玄诛殺了多名北府舊將,但劉裕仍領兵討伐盧循部眾,更獲加任彭城內史。及至桓玄篡位後次年(404年),劉裕跟從桓脩入朝建康,桓玄亦十分賞識他,出遊都殷勤接引,賞賜亦甚為豐厚。當時桓玄皇后劉氏就勸桓玄除去劉裕,但桓玄仍圖借助劉裕攻略中原,拒絕加害。

早在劉牢之失敗之時,劉裕就向何無忌說:「桓玄若果守著臣子的忠節,就應與你輔助他;否則,就要與你對付他。」及至劉裕入朝後回到京口,就與何無忌、劉毅、孟昶、諸葛長民、王元德等人合謀舉兵討伐桓玄,並準備在京口、廣陵、歷陽(今安徽和縣)及建康(今江蘇南京市)四地同時起兵。元興三年二月乙卯(404年3月24日),劉裕託詞遊獵而外出募眾,終得百多人。次日(3月25日)早上起兵,何無忌殺桓脩,當時桓脩司馬刁弘率眾前來,劉裕則假稱江州刺史郭昶之已在尋陽(今江西九江市)迎晉安帝復位,桓玄更已被處決,自己只是奉密詔誅除桓氏叛黨。刁弘信以為真,劉裕不久就誅除刁弘,控制了京口。同時孟昶等亦成功控制了廣陵,並率眾至京口與劉裕會合,劉裕獲眾人推舉為盟主,總督徐州事,並於次日(3月26日)起兵進攻建康。

桓玄先派吳甫之及皇甫敷抵抗劉裕,劉裕先於江乘(今江蘇句容北)殺吳甫之,至江乘以南的羅落橋時奮力作戰,又殺皇甫敷,繼續進攻。三月己未日(3月28日),劉裕進攻覆舟山,並命弱兵登山,持著旗幟分道而行,營造四周皆有士兵,數量很多的假象;而又因桓玄守軍大多是北府軍出身,面對劉裕都沒有鬥志,劉裕於是與諸軍進攻,順利以火攻擊潰桓玄守軍,而桓玄亦棄城西逃。

劉裕於三月壬戌日(3月31日)獲王謐等人推舉為使持節、都督揚兗豫青冀幽并八州諸軍事、鎮軍將軍,徐州刺史。不久,劉裕奉武陵王司馬遵承制總百官行事。劉裕在進建康城後派諸將追擊桓玄,終於當年六月誅殺了桓玄,並讓在江陵(今湖北江陵)的晉安帝復位。然而,桓氏勢力仍在荊州盤據,並反攻江陵,直至義熙元年(405年)才再收復江陵,驅逐當地桓氏勢力,並自江陵迎晉安帝回建康,不久劉裕就還鎮丹徒。

義熙二年(406年),劉裕因功受封為豫章郡公。義熙四年正月(407年),因上年年末揚州刺史、錄尚書事王謐去世,劉裕聽從劉穆之的勸言入朝議繼任人選,終獲授侍中、車騎將軍、開府儀同三司、揚州刺史、錄尚書事、徐兗二州刺史,入掌朝政大權。

盧循、徐道覆趁劉裕領兵在外,於義熙六年(410年)起兵,進攻江州。當時朝廷急徵劉裕,而當時劉裕剛滅南燕,收到詔書就撤還建康。劉裕至山陽(今江蘇淮安市)時知江州刺史何無忌已戰死,於是加速回防建康,並於四月趕至。五月,豫州刺史劉毅大敗給盧循,盧循繼續東下,而劉裕當時就招募兵眾,修治石頭城並於當地聚兵。不過,由於劉裕急急南返,士卒多有傷病,而建康兵力亦不過千人,面對有十多萬人的盧循大軍顯得實力懸殊,然而劉裕堅決不肯接受諸葛長民及孟昶奉安帝北歸廣陵避敵的建議,決意死戰。

盧循軍到後停駐蔡洲(今江蘇江寧縣西南江中),劉裕就以木柵阻斷石頭城及淮口,修治越城(今江寧縣南)並建查浦、藥園、廷尉三個堡壘,分兵戍守以禦盧循,盧循曾分疑兵進攻白石及查浦,自率大軍進攻丹陽郡,但都沒有取勝,而且在各縣中都無法搶掠到物資,被逼於七月退兵江州。同年十月,劉裕率劉藩、檀韶、劉敬宣等人進攻盧循,並於十二月以火攻擊敗盧循船隊。盧循敗後試圖於左里(今鄱陽湖口)擋住劉裕,但劉裕率軍奮戰,盧循軍無法阻擋而大敗,盧循因而南逃廣州。劉裕早於盧循撤出蔡洲後就已派了孫處及沈田子經海路攻佔了盧循根據地番禺,盧循敗逃廣州後於義熙七年(411年)又於廣州敗於沈田子等人,終在交州被刺史杜慧度所殺。

劉裕於義熙七年(411年)班師回到建康,並受太尉、中書監職位。次年(412年)四月,劉裕以劉毅為荊州刺史。劉毅自以能力不亞於劉裕,甚不服在劉裕以下,他亦得朝中有名望人士歸心交結,故此遷鎮荊州時就將大部分豫州府屬及江州萬多人的軍隊都帶去荊州,到任後又重新調度荊州郡縣首長,更以患病為由請堂弟劉藩去做他副手。劉裕知其有異心,於是假意答允其請求,但就乘劉藩自兗州治所廣陵入朝時就指稱他與謝混圖謀不軌,將二人賜死,接著就率軍自建康出發討伐劉毅。劉裕派王鎮惡為前鋒,沿路聲言是劉藩前來去騙倒各戍和民眾,直至江陵城外五六里時才被發現,但已趕及在劉毅關閉城門前率兵入城,並在城內作戰。城中民眾知劉裕在率軍前來都十分驚恐,劉毅不敵王鎮惡,唯有出逃,並於牛牧寺自殺。劉裕隨後來到江陵,誅殺了劉毅親信郗僧施,消滅了劉毅勢力。

劉裕征劉毅時以諸葛長民守留府事,但諸葛長民見劉毅敗死,自己亦深感不安,更意圖作亂,劉裕回建康時故意拖慢進度,讓等待迎接他的諸葛長民及其他官員接連幾日都等不到劉裕。劉裕卻乘輕舟快快進城,進入了官邸東府。諸葛長民知道劉裕突然回來了,於是拜訪,劉裕暗中命壯士丁旿埋伏,故意和諸葛長民閒話家常,乘諸葛長民警覺性下降時命丁旿將其殺死,接著又誅殺了長民弟諸葛黎民等人。劉裕接著就加鎮西將軍、豫州刺史,接掌諸葛長民的原職。清除京口武將中的異己勢力之後,劉裕在412年底發動晉滅譙蜀之戰,隔年(413年)西征主將朱齡石成功滅譙蜀,使劉裕加授羽葆、鼓吹及班劍二十人。

412年征討劉毅時,劉裕以晉宗室司馬休之接任荊州刺史。司馬休之頗得當地人心,而劉裕就懷疑他有異心;在義熙十年(414年),其子司馬文思又在建康招集輕俠,令劉裕十分厭惡,司馬文思終因被揭發殺害官吏而被捕,劉裕誅殺其黨眾而免文思死,反送他到司馬休之那裏,要他親自教誨他,實質就是要司馬休之將其處死。然而,司馬休之並沒有殺文思,只是上表廢掉文思的譙王爵位,並寫信向劉裕道歉。這舉動令劉裕對其大感不滿,立刻就命江州刺史孟懷玉戒備。

義熙十一年(415年),劉裕收殺司馬休之在建康的次子司馬文寶及侄兒司馬文祖,並出兵討伐司馬休之,自加黃鉞,領荊州刺史。司馬休之則上表劉裕罪狀,派兵抵抗;當時雍州刺史魯宗之自感不被劉裕所容,故與司馬休之聯結。劉裕前鋒徐逵之初戰敗於魯軌,眾將除蒯恩外皆戰死,劉裕大怒。然而當他到時,魯軌及司馬文思率軍在懸岸峭壁上列陣,令劉裕難以登岸,胡藩當時就冒險攀登,司馬文思等竟不能抵擋,劉裕就乘對方後撤的機會登岸進攻,終擊潰司馬休之的軍隊,攻下江陵,司馬休之及魯宗之北投後秦。

劉裕在消滅司馬休之後獲劍履上殿、入朝不趨、贊拜不名的崇禮,次年(416年)正月更獲加領平北將軍、兗州刺史、都督南秦州諸軍事,至此其一人已經都督徐州、南徐、豫、南豫、兗、南兗、青、冀、幽、并、司、郢、荊、江、湘、雍、梁、益、寧、交、廣、南秦共二十二州。

在魏晋十六国时期,东晋虽偏安江南,却始终没有放弃收复中原等漢地北部地區,所以屡次发动北伐战争。后秦、南燕出于内乱而败亡;公元397年北魏军攻下中山,后燕官吏兵投降两万余人,后燕的疆域被切断为南燕和北燕二部,405年南燕又发生政变;416年姚兴卒,后秦内乱不断,镇守蒲坂和岭北的姚懿、姚恢先后率叛军进攻长安。刘裕趁后秦、南燕内乱之际,乘机出兵,并一举攻灭。这次收复中原的版图之多,是东晋历次北伐中最成功、影响最深远的一次,也是以前的多次北伐都无法与之比拟的。

義熙五年(409年),南燕皇帝慕容超因為缺乏太樂伎人,派兵侵略淮北的宿豫城(今江蘇宿遷縣東南),大掠民眾北歸。及後又派兵進攻淮北,擄去陽平和濟南兩郡太守,俘擄千多家。

劉裕因此上表北伐,並於同年四月出發。當時劉裕認為燕軍短視,不會據守大峴山(今山東臨朐縣東南)天險並堅壁清野,只會進據臨朐(今山東臨朐縣),退守都城廣固(今山東青州市),而當時南燕軍的行動亦果然如此。慕容超知晉軍過了大峴山就親自率軍到臨朐,劉裕前鋒先於巨蔑水擊退燕軍,接著攻臨朐城。晉燕兩軍於臨朐以南作戰,胡藩獻計出奇兵突襲臨朐城內,最終成功攻克,慕容超倉皇自城中逃至城南大軍那裏,而此時劉裕命軍隊進攻,大敗燕軍並斬殺其十多名大將,慕容超於是逃回廣固。劉裕接著乘勝追擊至廣固,並成功攻克其外城。慕容超據守小城抵抗,劉裕就築圍圍困廣固。劉裕圍城戰爭一直維持至次年二月才攻下廣固,並俘殺慕容超,滅了南燕。

劉裕在當日平滅南燕後就已經有攻略後秦的打算,只因盧循作亂才逼令他班師建康,而劉裕在消滅了國內主要異己後,又再重拾昔日計劃。劉裕在獲加督至二十二州後月餘,又獲加中外大都督,解徐兗二州刺史而改領司、豫二州刺史,並奉琅琊王司馬德文北伐,打著晉朝皇室旗號安撫北方漢人。至五月又加北雍州刺史。終在八月,劉裕正式自建康出兵,進軍至彭城(今江蘇徐州市)後又加北徐州刺史。十月,劉裕所派的檀道濟等進攻洛陽(今河南洛陽市),守將姚洸出降,成功收復洛陽。

次年(417年)正月,劉裕自彭城率水軍西進,進入黃河。劉裕一直進軍至潼關,命王鎮惡率軍經渭河進攻後秦都城長安(今陝西西安市),王鎮惡於渭橋大敗姚丕,姚泓所率的軍隊亦因遭姚丕敗兵踐踏而潰亂,最終姚泓於八月出降,後秦滅亡。劉裕於次月到達長安,大賞將士並誅殺歸降的後秦宗室姚璞、姚讚及其百多名宗族。

同年十一月,留守建康的劉穆之去世,當時劉裕還想以長安做基地進攻西北北涼等國,只是諸將都思鄉,大多都不想留下;劉裕向來倚重的劉穆之去世更令他覺得建康根本之地已空虛無靠,於是下了決心班師東歸。劉裕於是留了當時僅得十一歲的次子劉義真鎮守長安,並留下王鎮惡、王脩、沈田子、毛德祖等將領協助他。當地人民知道劉裕要走都向他哭訴,希望他回心轉意,然而劉裕去意已決,還是在當年十二月出發離開。

然而,劉裕走後次年,諸將內訌,沈田子殺王鎮惡,王脩殺沈田子,而劉義真又在諸將唆擺下命人殺害王脩,於是關中大亂,夏國乘機進攻關中,劉裕唯有召還劉義真,派朱齡石等代鎮長安,更指令若關中不能守下去就可放棄。最終晉軍還是撤出長安,關中地區遭夏國佔領。

義熙十四年(418年),劉裕接受相國、總百揆、揚州牧的官職,以十郡建「宋國」,受封為宋公,並接受九錫的特殊禮待。同年,劉裕命令中書侍郎王韶之與晉安帝左右侍從密謀以毒酒毒殺安帝,王韶之於是乘司馬德文因病出宮的機會下手,縊殺安帝。當時劉裕因為相信預言書說:「昌明(晉孝武帝)之後尚有二帝」,於是聲稱依照遺詔,立了司馬德文為皇帝,即晉恭帝。

元熙元年(419年),劉裕進爵為宋王,又加十郡增益宋國,令宋國包括了二十郡。年末劉裕又獲加皇帝規格的的十二旒冕、天子旌旗等一系列特殊禮待。元熙二年(420年),劉裕入輔,傅亮知劉裕想要晉恭帝禪讓帝位予他但難於啓齒,於是代為向恭帝暗示,恭帝於是在六月禪讓帝位給劉裕,東晉滅亡,劉裕即位為帝,改國號為「宋」,改元永初。劉裕在稱帝之後,為了斬草除根,還殺掉了恭帝。此行為可謂劉裕一生中一個汙點,因為其行為開啟了前朝遜位之主不得善終之先(新朝王莽之於西漢孺子嬰、曹魏文帝曹丕之於東漢獻帝劉協、西晉武帝司馬炎之於曹魏元帝曹奐,都沒有加害前朝末主),至此,南朝末主除了陳後主陳叔寶其亡國非遭逢篡位外,全都俱被新立的政權所殺。

永初三年(422年),劉裕患病,五月病重時遺命司空徐羨之、尚書僕射傅亮、領軍將軍謝晦及護軍將軍檀道濟四人為顧命大臣,輔助太子劉義符。劉裕於五月癸亥日(6月26日)去世,享年六十歲。廟號高祖,謚為武皇帝,葬在初寧陵(今江蘇南京紫金山)。

刘裕自他繼王謐以錄尚書事掌權起直至其去世,一直掌握著東晉以及南朝宋的軍政大權,曾对当时积弊已久的政治、经济状况有所整顿。

門閥士族兼併土地的行為令百姓流離失所,無法保護其產業,劉裕則一改東晉以來對這種事寬松的規管,重訂規管並展示公眾,大大抑制了門閥豪強的兼并行為。及至會稽虞氏的虞亮藏匿一千多名脫離戶籍逃亡的人,劉裕將之處死,連時任會稽內史的司馬休之也遭免官。另劉裕又針對當時門閥豪強私佔山澤,人民去砍柴釣魚都受限制的問題,禁止豪強這種行為。刁氏一族向來富有,奴客亦多,在其宗族桓玄敗死後被誅滅時,劉裕亦將刁家的資產都分發給人民,讓人們按己力取用,賑濟當時處於饑荒及戰亂中的人民。劉裕亦於義熙九年(413年)將臨沂、湖熟原屬皇后所有,用來資助其化妝品開銷的田地分配給窮人。如此削奪了世族以及皇室的私產,用來資濟人民。即位為帝後更派大使巡行四方,舉善旌賢,訪問民間疾苦。

劉裕選才用人不重門第而重其才能,故對於寒門出身的劉穆之一直予以重任,在收復建康後讓他主持政局,大改官場之風,及至在劉裕領兵在外時更讓其主掌中樞重任。劉裕在晉時見州郡推薦的很多秀才、孝廉水平都不合要求,於是上請申明舊制,以策試考核他們。至登位後更曾到延賢堂為各秀才、孝廉作策試。而曾與劉裕起兵討伐桓玄的魏順之在盧循之亂時因為不敢救援部將謝寶,反倒退卻;魏順之雖為功臣,亦是魏詠之的弟弟,但劉裕大怒之下仍將其處死,此舉亦震懾其他桓玄之役中的功臣,都聽籨其命令。

劉裕於義熙九年(413年)再度實行土斷,各地人民依界土斷,只有僑居於晉陵的徐、兗、青三州人民不受影響,而當時很多僑郡僑縣都在這次土斷中被裁撤,重新整理了全國戶籍,便利於統計藏匿人口及增加賦稅收入。永初元年(420年),劉裕更下令所有逃避戶籍的人只要在限期內自首就能獲赦,並免去他們兩年的租賦,但凡有黃籍或證明文件的人都可恢復其原籍,再次減少國內藏匿人口。

刘裕消滅劉毅後,下令嚴禁荊、江二州地方官吏滥征租税、徭役,规定租税、徭役,都以现存户口为准。凡是州、郡、县的官吏利用官府之名,占据屯田、园地獲利的,皆一律废除。劉裕即位後,下令凡宫府需要的物资都要到市場採購,照价给钱,不得向人民征调。又下令官員不可徵去人民車牛,亦不能以官威逼迫人民獻出車牛,另亦將繁多的交易稅項作出減省,便利市場商業交易。

刘裕对政治、经济的整顿,进一步打击了門閥士族的势力,改善了政治和社会状况,对劳动人民的痛苦亦有所减轻。

而劉裕在建立南朝宋後亦削弱强藩,集权中央,於是限制了荆州州府置將和官吏數額,前者不可多於二千人,後者亦不可多於一萬人;另其他州府置將及官吏數亦不分別不得多於五百人及五千人。为防止权臣擁兵,他特別下诏命不得再別置軍府,宰相領揚州刺史的話可置一千兵。而凡大臣外任要職要需軍隊防衞,或要出兵討伐,一律配以朝廷军队,事情完結後軍隊都需交回朝廷。另劉裕為防外戚亂政,下令有幼主的話都委事宰相,不需太后臨朝。

劉裕高七尺六寸,氣質奇特。

劉裕行軍法令嚴明,軍隊軍容齊整,絕不擾民。而他在軍事行動的分析亦常常精準無誤,例如伐南燕時料定燕軍不會據守大峴山抵抗,而慕容德果然否決公孫五樓守大峴的計劃。命令朱齡石征伐西蜀時亦預計敵方會猜測晉軍循內水進攻,必以重兵守涪城,於是指令要從外水進攻,改派疑兵引誘涪城重兵,以圖直取成都。最終亦正如劉裕預計那樣,朱齡石成功繞過涪城重兵,直取成都,獲得勝利。

在生活上刘裕崇节俭,不爱珍宝,不喜豪华,宫中嫔妃也少。宁州地方官曾经奉献琥珀枕,是无价之宝,他不稀罕。在出征後秦时,有人说琥珀能够治疗伤口,他就命人将它砸碎,分给将领作为治伤药。平定关中後,他得到了美女姚氏(後秦天王姚興的姪女),十分宠爱。臣下谢晦劝谏他不要因女色而荒废政务,他当晚就将姚氏送出宫去。後來劉裕進封宋公,東西堂將要放置以金塗釘釘製的局腳牀,但劉裕以節為由而改用鐵釘釘製的直腳牀。又一次廣州進貢一匹筒細布,劉裕因其過於精巧瑰麗,製作必定擾民,故此下令彈劾獻布那郡的太守,將布匹送還並下令禁止再製作這種布。劉裕因患有熱病,常常要有冰冷的物件去降溫,於是有人就獻上石床。劉裕躺上冰冷的石床,感到十分舒服,但又感木牀已經很耗人力,大石頭要磨成牀就更甚了,於是下令將石床砸毀。劉裕更加下令將自己昔日的農具收起,留給後人。其子宋文帝一次看見,得知內情後大感慚愧。而其孫宋孝武帝拆毀劉裕生前的臥室而建玉燭殿,發現牀頭上有土帳,牆上掛著葛布製的燈籠及麻製蠅拂,袁顗稱許劉裕有儉素之德,但孝武帝沒有說甚麼,只說:「老農夫有這些東西,已經過於富裕了。」

劉裕不擅文才,故劉毅曾在宴會中特地賦詩:「六國多雄士,正始出風流」特意展示其文學造詣勝過劉裕。劉裕書法亦差,曾被劉穆之規勸,並在其指示下改寫大字。

劉裕不信神祇,登位後更曾下令將民間廟宇拆毀,只有先賢以及以有勳德的人的廟祠才得豁免。劉裕去世前患病,群臣上請劉裕祈求神祇庇佑,但劉裕不接受,只派了謝方明去太廟告知祖先。

昔日劉裕曾欠下刁逵三萬錢,無力償還,被刁逵抓著,王謐則去見刁逵,並替劉裕償還欠款,劉裕才得釋放;而當時劉裕既無名聲亦貧賤,不被其他具名望人士看重,唯有王謐去與他結交。王謐後在桓玄篡位時奉天子玉璽及冊文給桓玄,在桓楚任司徒,更獲封公爵,甚為禮侍。劉裕義軍攻下建康後,王謐仍任司徒,領揚州刺史、錄尚書事,但王謐既因在桓楚任高職,甚得寵待,故很不安心,最終出奔。然而劉裕沒有向王謐問罪,並念及昔日恩情,請武陵王司馬遵追還王謐,並讓其官復原職。而昔日為其債主的刁逵,在桓楚任豫州刺史,並為桓玄收捕起義失敗的諸葛長民。他在桓玄敗後出奔,終被部下抓住,可是刁氏一族接著卻遭誅殺,只有刁聘獲赦,然而不久刁聘亦因謀反而被誅,令刁氏族滅。

傳說劉裕出生時有神光照亮室內,當晚還降甘露。

劉裕曾到京口竹林寺,並獨自躺臥在寺內講堂內。一眾僧人竟看見他上面有五色龍形物體出現,大感吃驚並告知劉裕,劉裕則十分高興起說:「僧人是不會說謊的。」

有言曲阿、丹徒有天子之氣,而劉翹的墓就在丹徒,當時一個叫孔恭的人擅長占卜墓穴吉凶,劉裕一次就在父親墓前問孔恭,孔恭就言那是不平凡的墓地。劉裕聽後更為自負。更劉裕又覺得身邊總有兩條小龍,連旁人也曾看見過,至劉裕名聲漸高時,小龍也變大了。

傳說劉裕一次去伐木砍柴,射傷了一條大蛇。翌日再去時卻聽見有杵臼搗藥的聲音,發現有幾個小童正在製藥。劉裕於是問他們為何要製藥,小童則答:「我們的王被劉寄奴射傷,所以要製藥醫治。」劉裕追問:「你們的王既有神通,為何不殺了他?」小童卻答:「劉寄奴是王者,不可以殺。」劉裕喝跑了小童,拿走他們的藥。後來一次到下邳遊玩,一個僧人向他說:「江南地區會有動亂,令此地安定的人就是你呀。」僧人又給了劉裕一些傷藥,接著就失去了蹤影。劉裕手部有傷患,一直都無法痊癒,但用了僧人的藥一次後卻痊癒了。劉裕於是視剩餘的的傷藥及當日在小童那裏的藥為珍寶,每次受了傷,用那些藥都能醫好。

盧循譏諷劉裕智窮,劉裕則以續命湯反譏盧循命不長。典出藝文類聚·卷八十七:果部下:益智。

劉裕是兩晉南北朝時期最卓越的軍事統帥之一。劉裕在不到二十年時間裡,對內平息戰亂,先後平定孫恩、盧循的叛亂,消滅桓玄、劉毅等軍事集團;對外致力於北伐,取譙蜀、伐南燕、滅後秦,從一名普通軍人成長為名垂青史的軍事統帥,取得世人矚目的成就,更徹底改變晋朝政權對征服漢地北部的塞外各民族一直處於被動的局面。北魏謀臣崔浩在評價劉裕時說:「劉裕奮起寒微,不階尺土,討滅桓玄,興復晉室,北擒慕容超,南梟盧循,所向無前,非其才之過人,安能如是乎!」崔浩亦說:「劉裕之平禍亂,司馬德宗之曹操也。」何去非在《備論》中也說:「宋武帝以英特之姿,攘袂而起,平靈寶于舊楚,定劉毅于荊豫,滅南燕于二齊,克譙縱於庸蜀,殄盧循於交廣,西執姚泓而滅後秦,蓋舉無遺策而天下憚服矣。北方之寇,獨關東之拓跋,隴北之赫連耳。方其入關,魏人雖強,不敢南指西顧以議其後。」《南史》評論說:「宋武地非齊、晉,眾無一旅,曾不浹旬,夷凶翦暴,誅內清外,功格上下。若夫樂推所歸,謳歌所集,校之魏、晉,可謂收其實矣。」

劉裕的軍事生涯,指揮無數次作戰,最大特點是以少勝多,而且作戰中常身先士卒,所以能夠贏得廣大將士尊敬。劉裕北伐是中國戰爭史上最成功的北伐之一,成就不但遠較以前東晉各次北伐高,中國歷史上僅次於朱元璋,所以辛棄疾用「金戈鐵馬,氣吞萬里如虎」的詩句來形容劉裕北伐時的氣勢。司马光叙述刘裕北伐成功后匆忙东归,关中复失时,大发感叹:「惜乎,百年之寇,千里之土,得之艰难,失之造次,使丰、鄗之都复输寇手。荀子曰:『兼并易能也,坚凝之难。』信哉。」而王夫之直指劉裕是為了急急篡位而放棄關中,說:「刘裕灭姚秦,欲留长安经略西北,不果而归,而中原遂终于沦没。史称将佐思归,裕之饰说也。王、沈、毛、傅之獨留,豈繄不有思歸之念乎?西征之士,一歲而已,非久役也。新破人國,子女玉帛足系其心,梟雄者豈必故土之安乎?固知欲留經略者,裕之初志,而造次東歸者,裕之轉念也。夫裕欲归而急于篡,固其情已。」但王夫之仍然肯定了「然使裕據關中,撫雒陽,捍拓拔嗣而營河北,拒屈丐而固秦雍,平沮渠蒙遜而收隴右,勛愈大,威愈張,晉之天下其將安往?曹丕在鄴,而漢獻遙奉以璽綬,奚必反建康以面受之於晉廷乎?蓋裕之北伐,非徒示威以逼主攘夺,而无志于中原者,青泥既败,长安失守,登高北望,慨然流涕,志欲再举,止之者謝晦、鄭鮮之也。蓋當日之貪佐命以弋利祿者,既無遠志,抑無定情,裕欲孤行其志而不得,則急遽以行篡弒,裕之初心亦絀矣。」他还稱刘裕「為功于天下,烈于曹操,而其植人才以贊成其大計,不如操遠矣。操方舉事據兗州,他務未遑,而亟于用人;逮其後而丕與叡猶多得剛直明敏之才,以匡其闕失。」显然也包括了对刘裕北伐成功的肯定。「裕起自寒微,以敢戰立功名,而雄俠自喜,與士大夫之臭味不親,故胡藩言:一談一詠,搢紳之士輻湊歸之、不如劉毅。當時在廷之士,無有為裕心腹者,孤恃一機巧汰縱之劉穆之,而又死矣;傅亮、徐羡之、謝晦,皆輕躁而無定情者也。孤危遠處于外,求以制朝廷而遙授以天下也,既不可得,且有反面相距之憂,此裕所以汔濟濡尾而僅以偏安艸竊終也。當代無才,而裕又無馭才之道也。身殂而弒奪興,況望其能相佐以成底定之功哉?曹操之所以得志于天下,而待其子始篡者,得人故也。豈徒奸雄為然乎?聖人以仁義取天下,亦視其人而已矣。」

呂思勉則認為,劉裕急急篡位的說法只是史家附會王買德的話,說:「宋武代晉,在當日,業已勢如振槁,即無關、洛之績,豈慮無成?苟其急於圖,篡平司馬休之後,逕篡可矣,何必多伐秦一舉?武帝之於異己,雖云肆意翦除,亦特其庸中佼佼者耳,反之子必尚多。劉穆之死,後路無所付託,設有竊發,得不更詒大局之憂?欲攘外者必先安內,則武帝之南歸,亦不得訾其專為私計心也。義真雖云年少,留西之精兵良將,不為不多。王鎮惡之死,在正月十四日(應為十五),而勃勃之圖長安,仍歷三時而後克,可見兵力實非不足。長安之陷,其關鍵,全在王脩之死。義真之信讒,庸非始料所及,此尤不容苟責者也。」

劉裕在對待刁逵及王謐截然不同的態度,招來了不少批評,南朝梁湘東世子蕭方等就曾言:「夫蛟龍潛伏,魚蝦褻之。是以漢高赦雍齒,魏武赦梁鵠,安可以布衣之嫌而成萬乘之隙也!今王謐為公,刁逵亡族,醻恩報怨,何其狹哉。」裴子野亦言:「刁逵,玄之爪牙;王謐,楚之上相,論逆則王重,定罪則逵輕。稚遠以舊德錄萬機,長民以宿憾夷七族,以為晉政偏頗甚矣!且神龍伏於罟網,漁者安知其靈化;霸王匿於人庶,庸夫何以悟其英雄!苟在不悟則驕之者,眾可勝怨乎?是知宋高祖之非弘亮也,同盟多貮宜乎哉!」

劉裕攻下南燕都城廣固後,因為怨恨城池久久不下,故此意圖將城內人民全部坑殺,並將其妻女賞賜給將士,只因韓範勸止才不實行,但仍然盡殺南燕王公共三千人,並抄沒萬餘人。此意圖亦招來司馬光批評:「晉自濟江以來,威靈不競,戎狄橫騖,虎噬中原。劉裕始以王師翦平東夏,不於此際旌禮賢俊,忍撫疲民,宣愷悌之風,滌殘穢之政,使群士嚮風,遺黎企踵,而更恣行屠戮以快忿心;迹其施設,曾苻、姚之不如,宜其不能蕩壹四海,成美大之業,豈非雖有智勇而無仁義使之然哉!」

王夫之在《读通鉴论》评论刘裕:“宋武兴,东灭慕容超,西灭姚泓,拓跋嗣、赫连勃勃敛迹而穴处。自刘渊称乱以来,祖逖、庾翼、桓温、谢安经营百年而无能及此。后乎此者,二萧、陈氏无尺土之展,而浸以削亡。然则永嘉以降,仅延中国生人之气者,唯刘氏耳。舉晉人坐失之中原,責宋以不蕩平,沒其撻伐之功而黜之,亦大不平矣。君天下者,道也,非勢也。如以勢而已矣,則東周之季,荊、吳、徐、越割土稱王,遂將黜周以與之一等;而嬴政統一六寓,賢于五帝、三王也遠矣。拓拔氏安得抗宋而與並肩哉?唐臣隋矣,宋臣周矣,其樂推以為正者,一天下爾。以義則假禪之名,以篡而與劉宋奚擇焉?中原喪于司馬氏之手,且愛其如綫之緒以存之;徒不念中華冠帶之區,而忍割南北為華、夷之界乎?半以委匪類而使為君,顧抑撻伐有功之主以不與唐、宋等倫哉?汉之后,唐之前,唯宋室犹可以为中国主也。”


\subsubsection{永初}

\begin{longtable}{|>{\centering\scriptsize}m{2em}|>{\centering\scriptsize}m{1.3em}|>{\centering}m{8.8em}|}
  % \caption{秦王政}\
  \toprule
  \SimHei \normalsize 年数 & \SimHei \scriptsize 公元 & \SimHei 大事件 \tabularnewline
  % \midrule
  \endfirsthead
  \toprule
  \SimHei \normalsize 年数 & \SimHei \scriptsize 公元 & \SimHei 大事件 \tabularnewline
  \midrule
  \endhead
  \midrule
  元年 & 420 & \tabularnewline\hline
  二年 & 421 & \tabularnewline\hline
  三年 & 422 & \tabularnewline
  \bottomrule
\end{longtable}


%%% Local Variables:
%%% mode: latex
%%% TeX-engine: xetex
%%% TeX-master: "../../Main"
%%% End:

%% -*- coding: utf-8 -*-
%% Time-stamp: <Chen Wang: 2019-12-20 13:52:56>

\subsection{少帝\tiny(422-424)}

\subsubsection{生平}

刘义符(406年-424年8月4日),小字车兵,彭城綏輿里(今江蘇省銅山縣)人。中国南北朝時期宋朝的第二位皇帝,宋武帝刘裕长子,母親是夫人張闕。永初三年(422年)即位為帝,但兩年後就因居喪行為不當而遭顧命大臣徐羨之、傅亮等廢黜,不久被殺。

刘义符於義熙二年(406年)出生,其時劉裕已經四十四歲,他對得到這個遲來的兒子亦十分高興,後來更立其為豫章公世子。義熙十二年(416年)劉裕北伐後秦時,劉義符就受任中軍將軍,監太尉留府事,留守建康留府。義熙十四年(418年),劉裕受封宋公,建宋國,又以劉義符為宋公世子。劉裕於元熙元年(419年)進封宋王,並加殊禮,劉義符成為宋王太子。至永初元年(420年)劉裕篡晉自立,劉義符亦被立為皇太子。

永初三年(422年),劉裕病重,特意召見劉義符並告誡他:「檀道濟雖然有才幹和謀略,但沒有遠大志向,絕不像他兄長檀韶那麼難駕御。徐羨之、傅亮,應該沒有異心。謝晦數次跟我出征,頗為機智,若有人懷有異心,那肯定是他了。」並以徐羨之、傅亮、謝晦及檀道濟四人為顧命大臣,又明令往後有幼主繼位都不用母后臨朝,朝事都全交給宰相。劉裕在五月癸亥日(6月26日)去世後,劉義符就於同日即位為帝。

劉義符即位後仍需守父喪,但他在期間表現無禮,常與身邊的人十分親密,遊樂無度。當時特進范泰曾寫書勸諫,但劉義符不聽。而謝晦早年見劉義符身邊的都是小人,就曾向劉裕表示劉義符不是繼承宋室的人選,徐羨之、傅亮及謝晦最終在景平二年(424年)謀廢劉義符,並召江州刺史王弘及南兗州刺史檀道濟入朝,命中書舍人邢安泰及潘盛為內應,然後讓皇太后下令廢劉義符為營陽王。那天早上,謝晦、檀道濟及徐羨之領兵自雲龍門入宮,其時潘盛已經撤去宿衞軍隊,故此無人阻攔謝晦等軍。而劉義符那時在華林園設下市場,並親身去賣物;又開了水道,前一天和身邊親近乘船高歌大叫,並遊玩到天淵池後睡在船上,到了那天早上還沒醒。士兵進宮後殺害劉義符身邊兩個侍者,更傷了其手指,接著就將其扶出東閤,沒收皇帝璽綬,終將其送到吳郡幽禁。六月癸丑日(8月4日),徐羨之命邢安泰於金昌亭弒殺劉義符,劉義符極力反抗,要逃出昌門,但遭追捕者用門閂絆倒,最終遇害,享年十九歲。

徐羨之隨後立了宜都王劉義隆為帝。至元嘉九年(432年)以劉義恭長子劉朗為南豐縣王,作為劉義符的後嗣。

\subsubsection{景平}

\begin{longtable}{|>{\centering\scriptsize}m{2em}|>{\centering\scriptsize}m{1.3em}|>{\centering}m{8.8em}|}
  % \caption{秦王政}\
  \toprule
  \SimHei \normalsize 年数 & \SimHei \scriptsize 公元 & \SimHei 大事件 \tabularnewline
  % \midrule
  \endfirsthead
  \toprule
  \SimHei \normalsize 年数 & \SimHei \scriptsize 公元 & \SimHei 大事件 \tabularnewline
  \midrule
  \endhead
  \midrule
  元年 & 423 & \tabularnewline\hline
  二年 & 424 & \tabularnewline
  \bottomrule
\end{longtable}


%%% Local Variables:
%%% mode: latex
%%% TeX-engine: xetex
%%% TeX-master: "../../Main"
%%% End:

%% -*- coding: utf-8 -*-
%% Time-stamp: <Chen Wang: 2021-11-01 15:03:23>

\subsection{文帝劉義隆\tiny(424-453)}

\subsubsection{生平}

宋文帝劉義隆(407年-453年3月16日),小字車兒,宋武帝劉裕的第三子,劉宋第三任皇帝。劉義隆生於東晉末年,南朝宋立國後,受封宜都王。宋少帝被廢後獲擁立為帝,即位後改元元嘉。劉義隆在位近三十年,在位期間,建立制度、賞罰分明、鼓勵農桑,減免賦稅力役,使得國家大治,「內清外晏,四海謐如」,此治世因其年號元嘉而稱為「元嘉之治」。劉義隆亦銳意北伐,曾先後三次發起大規模北伐戰爭圖收復北魏所佔的河南土地,然而三次皆失敗,其中發生在元嘉後期的第二次更讓魏軍南攻至江北瓜步,一度威脅建康。元嘉北伐亦對國內經濟民生造成嚴重打擊,《資治通鑑》對北伐的創傷寫道「元嘉之政,自此衰矣。」

劉義隆於東晉義熙三年(407年)生於京口(今江蘇省鎮江市)。義熙六年(410年),時值盧循之亂,盧循叛軍逼近建康,劉裕因應京口位置重要,遂命劉粹輔佐年僅四歲的劉義隆鎮守京口。義熙十一年(415年),因前年劉裕指令朱齡石成功滅亡譙蜀,收復蜀地,晉廷封劉義隆彭城縣公。義熙十三年(417年),劉裕北伐,率水軍自彭城(今江蘇省徐州市)兵向關中,令劉義隆行冠軍將軍留守,東晉朝廷加封其為使持節、監徐兗青冀四州諸軍事、徐州刺史。義熙十四年(418年),劉裕收復關中、還軍彭城,原本想讓世子劉義符出鎮荊州,遂授劉義隆為監司州豫州之淮西兗州之陳留諸軍事、前將軍、司州刺史,並命其鎮守洛陽(今河南省洛陽市),然而因張邵諫止劉裕讓世子外任,劉裕遂改義隆為都督荊益寧雍梁秦六州豫州之河南廣平揚州之義成松滋四郡諸軍事、西中郎將、荊州刺史,鎮守江陵(今湖北省荊州市)。不過,由於劉義隆年紀尚輕,州府事皆由司馬張邵處理。

永初元年(420年),宋武帝劉裕篡晉登位,封劉義隆為宜都王,食邑三千戶。不久加號鎮西將軍,並先後獲進督北秦州及湘州。

劉裕於永初三年(422年)死後,宋少帝劉義符即位,但因為居喪無禮,有多過失,在景平二年(424年)即因顧命大臣徐羨之、傅亮及謝晦為首發動的政變廢黜,將其幽禁並派人殺害。因義符無子,義符次弟劉義真應當繼位,然因為徐羨之認為他不宜為君,故在廢帝以前就先廢義真為庶人,後更派人殺害。廢帝後,侍中程道惠曾請改立武帝五子劉義恭,然而徐羨之屬意劉義隆,百官於是上表迎作為武帝三子劉義隆為皇帝。

時傅亮率行臺到江陵迎劉義隆入京。當時已時是七月中,江陵已聽聞少帝遇害的消息,劉義隆及一些官員都對來迎隊伍有所懷疑,不敢東下,但在王華、王曇首及到彥之的勸告下決定出發並在八月八日(八月丙申日,424年9月16日)到達建康,次日即位為帝,改元「元嘉」。

宋文帝自江陵東下起一直在提防徐羨之等人,即在東下行程上,隨行的荊州州府官員都嚴兵自衞,行臺百官都無法接近,中兵參軍朱容子更在行程數十日內一直抱刀在船艙外守衞。即位後又將親信王華及王曇首召進京內任官,更拒絕徐羨之讓當時暫鎮襄陽(今湖北襄陽市)的到彥之出任雍州刺史的建議,堅持要召其入京為中領軍,統領軍事。傅亮及謝晦亦試圖和王華等人交結,以圖安心。徐羨之及謝晦亦在元嘉二年(425年)上表歸政,讓劉義隆正式親政。不過,王華及孔甯子其時多次向劉義隆中傷徐羨之等人,劉義隆亦有了誅殺權臣的意圖,慮及謝晦當時以荊州刺史坐鎮荊州重地,於是託辭北伐及拜謁陵墓以修建船艦,其時朝廷行事異常,圖謀差點就泄露了。

元嘉三年(426年),劉義隆宣布徐羨之、傅亮及謝晦擅殺少帝及劉義真的罪行,要將徐羨之及傅亮治罪,並決定親征謝晦,命雍州刺史劉粹、南兗州刺史檀道濟及中領軍到彥之先行出兵。徐羨之聞訊自殺,傅亮被捕處死,謝晦則出兵反抗,但知檀道濟協助劉義隆討伐即惶恐不已,無計可施,最終檀道濟到後朝廷軍隊軍勢強盛,謝晦軍隊潰散,謝晦試圖逃走但被擒處死,遂消滅了三個權臣的勢力。

劉義隆殺徐羨之後,揚州刺史一職由司徒王弘出任,不過王弘卻一直試圖讓彭城王劉義康入朝他共掌朝政,以收斂當時琅邪王氏人物掌握朝廷要職的鋒芒。最終劉義康於元嘉六年(429年)得以司徒、錄尚書事身份和王弘共輔朝政,然而當時王弘常因患病而將政事推給義康處理,遂令義康漸得專掌朝政。元嘉九年(432年)王弘去世後,劉義隆更授義康揚州刺史,義康獨掌政事。

時劉義隆常常患病,政事其實都由劉義康處理,而且劉義康更衣不解帶去照料劉義隆,內廷和外朝事遂由義康所掌握。乃至元嘉十三年(436年),因應劉義隆病重,劉義康擔心一旦劉義隆去世,無人能駕馭功高才大的司空檀道濟,於是假作詔書,並在宋文帝的同意下收殺檀道濟一家及其親將。不過,劉義康自以皇帝是至親,率性而行,行事都不避嫌,沒有君臣之禮。其時劉義康親信劉湛等人更力圖想將義康推上帝位,趁義隆病重時稱應以長君繼位,甚至去儀曹處拿去東晉時晉康帝兄終弟及的資料,更去誣陷一些忠於國家,不和劉湛一夥的大臣。劉義隆病愈後知道這些事,即令兄弟之間生了嫌隙,最終劉義隆在元嘉十七年(440年)誅殺了劉湛等人,並應劉義康上表求退而讓他外調江州。隨後劉義隆將司徒、錄尚書事及揚州刺史分別授予江夏王劉義恭及尚書僕射殷景仁,然劉義恭鑑於義康被貶,雖然擔當實質宰相,行事小心謹慎,只奉行文書,卻得劉義隆安心,主相之爭以權力歸回劉義隆手中結束。

北魏在永初三年十月曾乘劉裕去世大舉南侵,奪取包括虎牢(今河南滎陽汜水鎮)、洛陽及滑臺(今河南滑县)等黃河以南土地,故劉義隆自即位以來便有收復黃河以南土地的志向。元嘉七年(430年)三月,劉義隆以到彥之為主帥,率領王仲德及兗州刺史竺靈秀率水軍至黃河,另遣段宏率八千精騎攻虎牢。到彥之軍一日只行軍約十里,到七月才到須昌(今山東東平縣西北),其時北魏以碻磝(今山東荏平西南)、滑臺、虎牢、及洛陽四鎮兵少,先後讓守將棄城北退,宋軍遂輕易奪回四鎮。然而到十月,北魏反攻,魏將安頡進攻洛陽金鏞城,守將杜驥因城池殘破且無糧食而棄守南撤;另一方面虎牢亦失陷。接著,魏將叔孫建及長孫道生等於十一月渡過黃河,到彥之見諸軍相繼敗陣,不理垣護之支援青州的諫言,南退至歷城(今山東濟南歷城區)後就燒船率軍直奔彭城,守須昌的竺靈秀於是也退,更在湖陸大敗給叔孫建。魏軍亦進攻滑臺,檀道濟雖然在十一月率軍北上救援,但次年正月起因叔孫建等人的干擾而無法支援滑臺,滑臺遂於二月失陷,檀道濟全軍撤返。北伐以失敗告終。不過後來王玄謨常常都進獻北伐策略,劉義隆聽後心動,曾對殷景仁說:「聽王玄謨說的話,令人也想在狼居胥山祭天呀。」。

元嘉二十七年(450年)-二月,北魏以步騎十萬南侵,並強攻不滿千兵的懸瓠(今河南汝南縣),守將陳憲苦戰力保不失,劉義隆遣臧質與劉康祖救援,逼退魏軍。當時義隆也命令徐兗二州刺史劉駿派兵進攻攻佔汝陽郡的魏軍,但所派的劉泰之軍卻慘敗予魏軍,泰之更戰死。魏軍在四月撤兵後,劉義隆即欲伐魏。他得到親信徐湛之、江湛及王玄謨支持,然沈慶之進諫:「步兵對陣騎兵向來處於劣勢,請放棄出征之事,而且當日檀道濟再戰無功而返,到彥之更是失利敗還。現在看王玄謨等人都比不上這兩位將軍,軍隊戰力也不及當時,這恐怕會再度戰敗,難以得志。」然劉義隆卻說:「我軍戰敗自有別的原因,這是因為檀道濟放任著敵人以圖鞏固自己地位,到彥之行軍中途病發。北虜恃著的就只是馬,夏天多雨水,河流暢通,只要派船進攻北方,那碻磝敵軍肯定會退走,滑臺守軍亦很易攻破。攻取了這兩城後送糧食慰問人民,那虎牢、洛陽人心自然不穩。等到冬天做好城間防守,待北虜騎兵過河,那就一網成擒。」

於是劉義隆堅持不聽沈慶之、太子劉劭及蕭思話勸阻,於當年(450年)七月下詔北伐,以青冀二州刺史蕭斌為六萬軍主帥,節下的王玄謨(先鋒)率沈慶之和申坦領主力進入黃河,更別遣其他四軍東西並進,大舉伐魏。不久北魏碻磝守軍就棄城,王玄謨遂攻滑臺,但強攻數月仍不能攻下,等到十月號稱百萬的北魏援軍渡過黃河,他才撤退,卻在追擊中大敗,死了萬多人。劉義隆見玄謨戰敗,魏軍一直深入,於是召還正在攻魏的各路軍隊,最終魏軍南攻至瓜步(今江蘇南京六合區瓜埠鎮),一度威脅渡江攻打建康,劉義隆唯有答應議和息兵。魏軍遂於次年自瓜步退軍,當時在彭城坐鎮的太尉劉義恭,認為碻磝不可守,就命一直守城的王玄謨退回歷城,碻磝遂失。此戰不但無功而還,且更被魏軍攻至長江,大肆燒殺擄掠,《資治通鑑》所謂「丁壯者即加斬截,嬰兒貫於槊上,盤舞以為戲。所過郡縣,赤地無餘,春燕歸,巢於林木」、「自是邑里蕭條,元嘉之政衰矣」。

元嘉二十九年(452年),劉義隆以北魏太武帝去世,命蕭思話督冀州刺史張永攻碻磝,可是自七月開始攻城起一直都無法攻破,至八月更被魏軍燒了攻城器具和軍營,蕭思話即使率兵增援,攻了十多日都沒法攻下,眼見兵糧不足,只有退兵。另一邊在攻虎牢的魯爽等知蕭思話退兵後亦撤走,北伐結束。

元嘉二十二年(445年),左衞將軍、太子詹事范曄與員外散騎侍郎孔熙先等人被揭發圖以劉義康造反,皆被誅殺,劉義隆亦因而廢劉義康為庶人。劉義隆第二次北伐失敗,令魏軍兵至瓜步,此時他憂心有人會借機擁被廢為庶人的劉義康作亂,遂於元嘉二十八年(451年)正月賜死劉義康。同時,太子劉劭將北伐失敗的罪責歸咎於當日一力支持並與持反對意見的沈慶之論戰的徐湛之及江湛,雖然劉義隆將責任歸於自己,但劉劭已經和二人極度不和。

後來,劉劭與始興王劉濬聽信女巫嚴道育,為了不再讓劉義隆知道他們做過的過失而責罵他們,就施以巫蠱,在含章殿前埋下代表劉義隆的玉雕人像。此事黃門慶國亦有參與,後來為了自保就報告給劉義隆知道。劉義隆知道後既驚訝又嘆惜,下令收捕另一同謀王鸚鵡,在其家中找到了劉劭和劉濬寫的數百張寫有咒詛之言的紙,又將那人像找到出來。劉義隆詰責二人,二人恐懼無言,只能一直道歉。劉義隆於是有了廢太子和賜死劉濬的打算,就與江湛、徐湛之及王僧綽商量;他想立建平王劉宏,徐湛之就支持女婿隨王劉誕,江湛就支持妹夫南平王劉鑠,可是久久都沒決定。王僧綽慮及機密可能泄露,勸劉義隆快作決定,但還是作不了決定。

元嘉三十年(453年)二月,劉義隆得知劉劭和劉濬還與嚴道育來往,決定實行廢太子和殺劉濬的計劃。劉義隆將此事告訴了劉濬生母潘淑妃,潘淑妃則告訴劉濬,劉劭再從劉濬口中得知,遂決定發動政變。二月二十日(3月15日)夜晚,劉劭召蕭斌及袁淑入宮,告知其計劃並表示翌日天亮就行事,蕭斌在劉劭威嚇下決定加入,堅拒的袁淑遂被殺。劉劭與蕭斌率軍在明早(3月16日)天亮時聲言受了敕命,帶著軍隊從萬春門入禁宮。那一晚,劉義隆又與徐湛之整夜討論事情,至劉劭軍隊攻入時蠟燭還亮著。劉劭齋帥張超之入殿,劉義隆舉起几桌抵抗,卻被砍斷五指,接著被殺,享年四十七歲。

劉劭隨後登位,並為劉義隆上諡號景皇帝,廟號中宗,並於三月二十日(三月癸巳日,4月14日),葬劉義隆於長寧陵。同年宋孝武帝劉駿起兵殺劉劭即位,改諡號文皇帝,廟號太祖。

劉義隆在消滅徐羨之等權臣後下詔派大使巡行四方,奏報地方官員的表現優劣,整頓吏治;又宣布一些年老、喪偶、年幼喪父及患重疾而生活困難者可向郡縣求助獲得支援,更廣開言路,歡迎人民進納有益意見和謀策。劉義隆亦多次去延賢堂聽審刑訟。元嘉十七年更下令開放禁止平民使用的山澤地區,又禁止徵老弱當兵的這些傷治害民的措施,要求各官依從法令行事。另在歷次天災時都會賑施或減免當年賦稅以撫慰人民。

劉義隆亦鼓勵農桑,元嘉八年即下詔命郡縣獎勵勤於耕作養蠶的農戶和教導正確農作方法,並將一些特別優秀的農戶上報。元嘉十七年又下令酌量減免農民欠下政府的「諸逋債」,後更於元嘉二十一年悉數免除元嘉十九年以前的欠「諸逋債」,又下令租借種子口糧給一些想參與農耕但物資缺乏的人,更賜布帛獎勵營治千畝田地的官民;元嘉二十一年夏季因連續下雨而出現水災,影響農業,劉義隆除了下令賑濟外,還在秋季命官員大力勵農民耕作米麥,又令開垦田地以備來年耕作,並於元嘉二十二年重新開墾湖熟的千頃廢田。

劉義隆重視文化建設,元嘉十五年(438年)召雷次宗在京城雞籠山(今南京市北極閣)開設「儒學館」講學,使儒學與玄學、文學、史學合稱「四學」;又於元嘉十九年(442年)下詔建國子學,待一眾冑子集合後於次年復立國子學,並於二十三年至國子學策試學生。不過,因北伐原因,劉義隆在在元嘉二十七(450年)年又廢止國子學。因陳壽所著《三國志》過於精簡,劉義隆便詔命裴松之為其作注,並於完成後親自御覽,讚道:「此為不朽矣!」。

劉義隆身高七尺五寸,博涉經史,亦擅長隸書。他喜好文儒,對文士亦十分禮待,或加以親任,甚至得劉義隆寛免罪過。

劉義隆生於京口,對京口亦留有特別感情,元嘉二十六年曾下詔以原本僑置於京口的州治北遷原地令當地不復當年繁華,從各州人民中招募數千戶人充實京口,並賜予田宅。又因懷念當時生活,命找尋當年在京口生活的官民並一一上報,去世者則酌情賞賜其子孫。

史載劉義隆儉約,不好奢侈,既曾在元嘉八年下詔「直存簡約,以應事實。內外可通共詳思,務令節儉」,他本人亦曾經因老舊的乘輦蓬蓋未壞和紫色輦席貴為由拒絕車庫令更換的建議。但他卻在元嘉二十三年修築北堤建玄武湖,甚至想在湖中建方丈、蓬萊及瀛洲三座仙山,惟因何尚之反對而作罷;同年他又在華林園修築景陽山,何尚之亦諫,認為應該給工人在盛暑休息一下,但義隆不肯,反稱他們常常曝曬,在盛暑烈日下工作不叫辛勞。

會稽長公主劉興弟是義隆嫡姐,義隆亦十分尊敬她,尤其怕她號哭,如就曾經帶著武敬皇后為劉裕造的納衣去哭罵義隆,終讓義隆不殺徐湛之。劉義康被奪相權,外調江州時,會稽長公主亦曾要求義隆不要加害義康,當時義隆亦答應,並將二人對飲中的那壺酒賜給義康。然而劉義隆於瓜步之戰後仍違背諾言賜死劉義康。

南齊的史家沈約評論宋文帝:「太祖幼年特秀,顧無保傅之嚴,而天授和敏之姿,自稟君人之德。及正位南面,歷年長久,綱維備舉,條禁明密,罰有恆科,爵無濫品。故能內清外晏,四海謐如也。昔漢氏東京常稱建武、永平故事,自茲厥後,亦每以元嘉爲言,斯固盛矣。授將遣帥,乖分閫之命,才謝光武,而遙制兵略,至於攻日戰時,莫不仰聽成旨。雖覆師喪旅,將非韓、白,而延寇慼境,抑此之由。及至言漏衾衽,難結商豎,雖禍生非慮,蓋亦有以而然也。嗚呼哀哉!」

蕭梁的史家裴子野評述文帝:「太祖寬肅宣惠,大臣光表,超越二昆,來應寶命,沈明內斷,不欲政由寧氏,克滅權逼,不使芒刺在躬,親臨朝事,率尊恭德,斟酌先王之典,強宣當時之宜,吏久其職,育孫長子,民樂其生,鮮陷刑辟,仁厚之化,既已播流,率土忻欣,無思不服……上亦蘊籍義文,思弘儒術,庠序建於國都,四學聞乎家巷,天子乃移蹕下輦以從之,束帛宴語以勸之,士莫不敦悅詩書,沐浴禮義,淑慎規矩,斐然向方……然值北虜方強,周、韓歲擾,金墉、虎牢,代失其禦,二十七年,偏師克復河南,橫挑強胡百萬之眾,匈奴遂跨彭、沛,航淮浦,設穹廬於瓜步……于時精兵猛將,嬰城而不敢鬥,謀臣智士,折撓而無可稱……我守既嚴,胡兵亦怠,且知大川所以限南北也,疲老而退,我追奔之師,橐弓裹足,系虜之民,流離道路,江淮以北蕭然矣。重以含章巫盅,始自二逆,弒帝合殿,史籍未聞,仲尼以為非一朝一夕之故,其所由來者漸矣,辨之不早辨也。元嘉之禍,其有以焉。」

唐朝的虞世南:「夫立人之道,曰仁曰义,仁有爱育之功,义有断割之用,宽猛相济,然後为善。文帝沈吟於废立之际,沦溺於嬖宠之间,当断不断,自贻其祸。孽由自作,岂命也哉。」

北宋的司馬光評論:「文帝勤於為治,子惠庶民,足為承平之良主;而不量其力,橫挑強胡,使師徒殲於河南,戎馬飲於江津。及其末路,狐疑不決,卒成子禍,豈非文有餘而武不足耶?」

南宋的辛棄疾於〈永遇樂·京口北固亭懷古〉一詞中諷喻文帝北伐:「元嘉草草,封狼居胥,贏得倉皇北顧。四十三年,望中猶記、烽火揚州路。可堪回首,佛貍祠下,一片神鴉社鼓!憑誰問:廉頗老矣,尚能飯否?」,暗喻當時南宋權臣韓侂冑的北伐失敗。

清代初期的王夫之評430年北伐:「元嘉之北伐也,文帝诛权奸,修内治,息民六年而用之,不可谓无其具;拓跋氏伐赫连,伐蠕蠕,击高车,兵疲于西北,备弛于东南,不可谓无其时;然而得地不守,瓦解蝟缩,兵歼甲弃,并淮右之地而失之,何也?将非其人也。到彦之、萧思话大溃于青、徐,(南宋孝宗之)邵弘渊、李显忠大溃于符离,一也,皆将非其人,以卒与敌者也。文帝、孝宗皆图治之英君,大有为于天下者,其命将也,非信左右佞幸之推引,如燕之任骑劫、赵之任赵葱也;所任之将,亦当时人望所归,小试有效,非若曹之任公孙彊、蜀汉之任陈祗也;意者当代有将才而莫之能用邪?然自是以后,未见有人焉,愈于彦之、思话而当时不用者,将天之吝于生材乎?非也。天生之,人主必有以鼓舞而培养之,当世之士,以人主之意指为趋,而文帝、孝宗之所信任推崇以风示天下者,皆拘葸异谨之人,谓可信以无疑,而不知其适以召败也。道不足以消逆叛之萌,智不足以驭枭雄之士,于是乎摧抑英尤而登进柔輭;则天下相戒以果敢机谋,而生人之气为之坐痿;故举世无可用之才,以保国而不足,况欲与猾虏争生死于中原乎!」

\subsubsection{元嘉}

\begin{longtable}{|>{\centering\scriptsize}m{2em}|>{\centering\scriptsize}m{1.3em}|>{\centering}m{8.8em}|}
  % \caption{秦王政}\
  \toprule
  \SimHei \normalsize 年数 & \SimHei \scriptsize 公元 & \SimHei 大事件 \tabularnewline
  % \midrule
  \endfirsthead
  \toprule
  \SimHei \normalsize 年数 & \SimHei \scriptsize 公元 & \SimHei 大事件 \tabularnewline
  \midrule
  \endhead
  \midrule
  元年 & 424 & \tabularnewline\hline
  二年 & 425 & \tabularnewline\hline
  三年 & 426 & \tabularnewline\hline
  四年 & 427 & \tabularnewline\hline
  五年 & 428 & \tabularnewline\hline
  六年 & 429 & \tabularnewline\hline
  七年 & 430 & \tabularnewline\hline
  八年 & 431 & \tabularnewline\hline
  九年 & 432 & \tabularnewline\hline
  十年 & 433 & \tabularnewline\hline
  十一年 & 434 & \tabularnewline\hline
  十二年 & 435 & \tabularnewline\hline
  十三年 & 436 & \tabularnewline\hline
  十四年 & 437 & \tabularnewline\hline
  十五年 & 438 & \tabularnewline\hline
  十六年 & 439 & \tabularnewline\hline
  十七年 & 440 & \tabularnewline\hline
  十八年 & 441 & \tabularnewline\hline
  十九年 & 442 & \tabularnewline\hline
  二十年 & 443 & \tabularnewline\hline
  二一年 & 444 & \tabularnewline\hline
  二二年 & 445 & \tabularnewline\hline
  二三年 & 446 & \tabularnewline\hline
  二四年 & 447 & \tabularnewline\hline
  二五年 & 448 & \tabularnewline\hline
  二六年 & 449 & \tabularnewline\hline
  二七年 & 450 & \tabularnewline\hline
  二八年 & 451 & \tabularnewline\hline
  二九年 & 452 & \tabularnewline\hline
  三十年 & 453 & \tabularnewline\hline
  \bottomrule
\end{longtable}


%%% Local Variables:
%%% mode: latex
%%% TeX-engine: xetex
%%% TeX-master: "../../Main"
%%% End:

%% -*- coding: utf-8 -*-
%% Time-stamp: <Chen Wang: 2019-12-20 13:42:59>

\subsection{孝武帝\tiny(453-464)}

\subsubsection{刘劭生平}

刘劭(424年-453年5月27日),字休远,彭城綏輿里(今江蘇省徐州市銅山區)人。他是中国南北朝時期南朝宋宋文帝刘义隆的长子,母為皇后袁齊媯。宋文帝晚年因劉劭與女巫嚴道育交往及行巫蠱而謀廢其太子之位,劉劭於是先發制人發起兵變弒父奪位。但即位不久即遭三弟武陵王劉駿為首的軍隊討伐,兵敗被殺,在位僅一百日。史書因劉劭殺父奪位,不用劉劭為文帝上的廟號及諡號,亦不承認劉劭為刘宋的正統皇帝。

元嘉元年(424年),刘劭出生,時正值劉義隆在服父喪,於是一直到喪期結束,於元嘉三年閏正月丙戌(426年2月28日)才正式宣布長子誕生。元嘉六年三月丁巳(429年5月14日),刘劭被立為皇太子,居於永福省。元嘉十二年(435年),劉劭出居東宮,並娶殷淳女殷氏為太子妃。劉劭愛讀史書,尤其喜歡武事,而不管是親覽宮廷事務還是接待賓客,只要他想作,宋文帝都會順從他。為保護東宮,文帝亦讓東宮兵力與羽林衞兵力相同。

元嘉二十七年(450年),宋文帝北伐,劉劭與沈慶之、蕭思話等人極力反對但不果,但戰事不利,魏軍南攻至長江北岸的瓜步,震動建康,劉劭領兵出守石頭,總統水軍。時北魏遣使求婚,包括劉劭以內群臣都認為應該准許,但一直力主北伐的江湛認為外族無信,反對和親,劉劭於是大怒,厲聲斥責他,並在眾人離去時命隨從推逼江湛,差點將他推倒。劉劭又對文帝說:「北伐敗辱,數州淪破,獨有斬江湛,可以謝天下。」但文帝以北伐乃己意為由拒絕對江湛問罪。不過此後,劉劭每次辦宴都沒邀請江湛,又常和文帝說江湛是佞人,不應親近。

文帝為提倡農耕和種桑養蠶,特意在宮內養蠶。當時有一個叫嚴道育的女巫自稱能夠通靈及使喚鬼怪,劉劭姊東陽公主在婢女王鸚鵡告知之下,向文帝假稱道育擅長養蠶而召入宮中。道育入宮後稱述服用丹藥之事,預告符瑞事後又果真看見奇異事情,劉劭及東陽公主於是都對道育能力深信不疑。文帝次子始興王劉濬一向都依附劉劭,又因二人常有過失,為了不讓文帝知道,就請嚴道育幫忙,最終道育教他們巫蠱之事,以文帝形像造一個玉雕人像,埋在含章殿前。這件事除了他們三人知道外,王鸚鵡、與王鸚鵡私通的養子奴僕陳天興以及黃門慶國都有參與,劉劭更給了陳天興一個隊主職位。東陽公主死後,王鸚鵡亦該嫁人,劉劭及劉濬為守住巫蠱秘密,就自行決定將王鸚鵡嫁給劉濬府佐沈懷遠為妾,也不報告給文帝,只稍稍和臨賀公主提及了。不過,文帝及後知陳天興任隊主,亦知天興與王鸚鵡養母子的關係,特意派人詰問劉劭有關二人之事,劉劭就答稱天興身體壯健故給其職位,而表示鸚鵡還未嫁。鸚鵡當時已嫁給沈懷遠,劉劭怕事情被揭穿,立即通報給劉濬及臨賀公主要他們都稱鸚鵡未定婚嫁;王鸚鵡亦怕陳天興會令二人私通的事曝光,於是請求劉劭殺了天興滅口。不過,天興之死令黃門慶國擔心自身安危,於是將巫蠱之事告知文帝。文帝聞訊既震驚又哀惋,立即就命人收捕王鸚鵡,並在其家中搜到數百張劉劭及劉濬所寫的紙,全都是詛咒巫蠱的文句,又挖出含章殿前的人像。嚴道育就逃亡,成功躲過搜捕之人,並易服為尼姑,匿藏在東宮,有時隨劉濬出京口,有時又住在平民張旿的家。而劉劭及劉濬面對文帝的詰責,驚懼得無法答話,只有一直道歉。元嘉二十八年(451年)至元嘉三十年(453年)幾次的天象變異,令文帝再加東宮兵眾,令東宮擁有一萬兵。

元嘉三十年(453年)二月,劉濬轉任荊州刺史,遂自京口入朝,並載著嚴道育回東宮,打算帶她同赴荊州。不過,那時就有人告發嚴道育化身尼姑,常出入劉濬府內,文帝起初不信,但派人查問下終從兩個婢女口中得悉那真是嚴道育。文帝知二子仍然和嚴道育往來十分傷心,於是命京口送二婢到來,並決定廢太子及賜死劉濬,為此與王僧綽、江湛及徐湛之商討,但久未有決定。文帝亦將決定向劉濬生母潘淑妃透露,潘淑妃就將此事告知劉濬,劉濬報告劉劭後劉劭就決定起事,於是晚晚設宴款待將士,又與心腹張超之、陳叔兒、詹叔兒及任建之籌劃。

二月二十日晚(3月15日),兩婢快將到來,劉劭假傳詔命:「魯秀謀反,汝可平明守闕,率眾入。」於是命張超之召集二千多名士兵作準備,又召各幢隊主聲稱有討伐之事。當晚,劉劭又召見了前太子中庶子蕭斌、太子左衞率袁淑、太子中舍人殷仲素及左積弩將軍王正見入宮,對他們哭著說:「主上信讒,將見罪廢。內省無過,不能受枉。明旦便當行大事,望相與勠力。」蕭斌與袁淑立即就表示反對,勸他再作考慮,劉劭聽後表現憤怒。蕭斌在驚嚇下轉為支持,但袁淑仍舊反對,惟未能讓劉劭回心轉意,劉劭接著向袁淑等人分派袴褶,又分派幾段錦布讓其縛好袴子,作出戰準備。天亮時,劉劭與蕭斌同車準備好出發,停在奉化門等袁淑,但袁淑久久不到,到後又不肯上車,劉劭遂命左右殺害袁淑,接著命部眾如同平常入朝一樣走進宮中。經萬春門時,由於違反東宮兵入宮城的規定,劉劭於是對門衞聲稱受到敕命要帶兵入宮收討,遂成功進宮。接著張超之等數十人就直入雲龍門、東中華門及齋閤,拔刀直奔上合殿。當晚文帝又與徐湛之徹夜密談,當值衞兵至此時仍然在寢,張超之就上前砍殺文帝,並殺掉徐湛之。劉劭知文帝被殺後出坐東堂,由蕭斌持刀侍衞,並派人殺害江湛;左細仗主卜天與率眾進攻劉劭,但失敗被殺。

當日劉劭就即位稱帝,寫詔道:「徐湛之、江湛弒逆無狀,吾勒兵入殿,已無所及,號惋崩衄,肝心破裂。今罪人斯得,元凶克殄,可大赦天下,改元嘉三十年為太初元年。文並賜位二等,諸科一依丁卯。」太初年號是劉劭與嚴道育所商定的,來朝的官員才數十人,劉劭就等不及要即位,即位後又稱疾退入永福省,升文帝靈柩至太極前殿。劉劭及後將先前給諸王及各處的武器回武庫,又誅殺徐湛之、江湛等人的黨羽,並封賞幫助他篡位的官員,後來更將與其有宿怨的長沙王劉瑾等人宗室殺害;又在查閱文帝巾箱時發現王僧綽亦有參與廢太子的圖謀,亦將其殺害。文帝大殮時,劉劭稱疾不敢親往,至入殮後才穿上喪服至文帝靈前,表現得痛心哀慟。然後又向四方派大使,對一眾官員求問治國之道,又減輕賦稅及減少徭役,減省出遊耗費,又分配一些田野山澤給貧民。又先後立妃殷氏為皇后,長子劉偉之為太子。

不過,江州刺史武陵王劉駿、荊州刺史南譙王劉義宣、雍州刺史臧質及會稽太守隨王劉誕等人都拒命,起兵討伐劉劭,並以劉駿為主。劉劭弒父後正值劉駿典籤董元嗣回到建康,於是就命元嗣將自己聲稱的徐湛之弒逆版本報告給劉駿,但元嗣回去後就以實情報告。劉駿一方面派元嗣奉表還都,另一方面卻謀起兵。劉劭知劉駿起兵就責問元嗣,元嗣表示出發時尚未有此事,但劉劭不信,加以拷打後元嗣仍然不招,最終被打死。劉劭亦曾密書當時與劉駿一同討蠻的沈慶之,命其殺死劉駿,但慶之就支持劉駿,反助其統兵東下。面對大軍來攻,劉劭下令中外戒嚴,又自以自己向來習武,對百官說“卿等但助我理文書,勿措意戎陳。若有寇難,吾當自出,唯恐賊不敢動爾”,並由皇后叔父司隸校尉殷沖掌文符,左衞將軍尹弘為軍隊準備衣服,由蕭斌總掌眾事。另又將劉駿及義宣諸子分別軟禁在侍中下省及太倉空屋,並打算殺害尋陽、江陵、會稽三鎮士庶官員留在建康的家眷,但在劉義恭及何尚之勸阻下改變主意,才改為下書表明不問罪。劉劭又命褚湛之守石頭,劉思考鎮東府,蕭斌及劉濬就力勸劉劭率水軍迎擊討伐軍,不過劉義恭則建議以逸待勞,劉劭取信了義恭之策,即使蕭斌力陳對方形勢佔上風,應盡快一決勝負亦未能讓劉劭改變主意。不過,其實劉劭始終都不太相信朝廷舊臣,留義恭住尚書下省外,亦軟禁義恭十二個兒子在侍中下省;另厚待王羅漢及魯秀,將兵權交給二人,大加賞賜財寶和美女以取悅二人,又天天出外慰勞將士,自督修建船艦,又焚毀秦淮河南岸,將百姓都趕到北岸。

大軍臨近,但守石頭的龐秀之卻先一步轉投劉駿陣營,大大動搖人心。四月十九日(5月12日),討伐軍前鋒已到新林,劉劭親上石頭城烽火樓觀敵。二十一日(5月14日),討伐軍進至新亭,柳元景在依山建新亭壘據險自守,而劉劭就召魯秀與王羅漢駐朱雀門,讓蕭斌率步兵、褚湛之率水軍;當時詹叔兒察知討伐軍營壘尚未建立,勸劉劭乘時進擊,但劉劭不肯。翌日,劉劭才命蕭斌率魯秀、王羅漢等精兵共萬人進攻新亭壘,劉劭亦親自登上朱雀門督戰。由於士兵都得劉劭賞賜,故都為他拼命作戰,戰事佔據上風。但就在新亭壘將被攻下時,魯秀突然收兵,柳元景抓緊機會反擊,終扭轉戰局。劉劭在蕭斌等敗後又親率心腹再戰,又遭柳元景擊敗,死傷更大,劉劭斬殺撤退者以圖遏止潰敗之勢,但失敗,劉劭只好走經朱雀門還宮。

此戰敗後,褚湛之、檀和之、魯秀及劉義恭先後叛歸劉駿,劉劭只好向神明之力求助,將蔣侯神像運到宮內,拜他為大司馬,封鍾山郡王;又以蘇侯為驃騎將軍,命南平王劉鑠寫祝文向其宣告劉駿罪狀。五月,劉劭派往抵御劉誕所領東軍的部隊在曲阿戰敗,為了遏阻他進攻,劉劭就焚毀都水西裝及左尚方,以及破壞柏崗及方山土壩,又命守家未服兵役的男丁緣秦淮河竪起舶船,並在上築上大弩作防禦,又命人以柵欄阻斷班瀆、白石等水道。其時男丁不足,甚至要動用婦女完成工事。

五月三日(5月26日),魯秀率五百人進攻大航,將之攻克,守將王羅漢酒醉中驚聞敵軍已渡河,於是棄杖投降,其餘部隊亦都隨之潰散。當晚,劉劭關閉六門拒守,在門內鑿出護城河及柵欄。不過城內混亂,已無秩序,尹弘及孟宗嗣等人出降,蕭斌知大航失守後亦命所屬軍隊解甲投降但被殺。翌日(5月27日),劉義恭登朱雀門,總領諸軍進攻宣陽門,先前劉劭召還的陳叔兒部於建陽門遠遠望見討伐軍就棄杖逃走;原本屯駐閶闔門的劉劭部隊亦逃還殿內,程天祚及譚金等人因而攻入殿內,其餘眾軍繼進,臧質亦從廣莫門進入,會師太極殿前。劉劭穿過西垣入武庫井內,但為高禽所捕。當時劉劭問高禽:「天子何在?」高禽答:「至尊近在新亭。」高禽將劉劭帶到殿前,臧質問其為何行逆,劉劭答:「先朝當見枉廢,不能作獄中囚,問計於蕭斌,斌見勸如此。」將罪責推及蕭斌,又問臧質可否代為請求劉駿流放他到遠地。臧質遂將劉劭縛在馬上,將要衞送到劉駿軍門,但到牙旗下時劉義恭率眾觀望,並詰問劉劭何以殺其十二子,劉劭亦答道這是有負於義恭。江湛妻庾氏及龐秀之亦罵劉劭,但劉劭卻大聲回罵:「汝輩復何煩爾!」劉劭四子皆被殺,劉劭對劉鑠說:「此何有哉。」接著劉劭亦於牙旗下被殺,死前嘆道:「不圖宗室一至於此。」

劉駿在較早前已即位為帝,劉劭死後亂事被平定,劉劭妻殷氏與劉劭、劉濬諸子都被賜死,其他幫助劉劭的大臣如殷沖、尹弘、王羅漢及張超之等都被殺或賜死,嚴道育及王鸚鵡都在街上被鞭殺,焚屍揚灰江上;劉劭及劉濬屍體都被棄到長江中,枭首大航,劉劭東宮住所亦被毀。

據說劉劭初生時,尚為宜都王妃的袁皇后對兒子詳細端視後就命人向劉義隆表示:「此兒形貌異常,必破國亡家,不可舉。」並要下手殺掉他。劉義隆狼狽地趕去阻止才讓劉劭得以長大。

文帝死前一天夜晚,太史曾上奏預測東方有兵突襲,建議在太極前殿列兵萬人作銷災。文帝不許,最終讓劉劭成功篡位,聞言就嘆道:「幾誤我事。」又問太史令他還有多少年壽命,太史當時回答十年,但退下後就對人稱只有十旬日,劉劭知道後大怒,將太史殺掉,最終劉劭果十旬而亡。

刘劭的年号是太初(453年二月—453年五月),共计三個月。

\subsubsection{孝武帝生平}

宋孝武帝劉駿(430年9月19日-464年7月12日),字休龍,小字道民,宋文帝劉義隆的第三子,南朝宋第五任皇帝。453年3月16日深夜,皇太子劉劭於京城建康(今南京市)行凶,殺害父皇宋文帝劉義隆,自稱皇帝;時為武陵王的劉駿在沈慶之的輔佐下,於江州(今九江市)起兵宣討。同年5月20日,於新亭(今南京市西南)即皇帝位。5月27日攻下京城,擒斬長兄劉劭、二兄劉濬。隔年(454年)2月14日改元,年號孝建;457年2月10日二度改元,年號大明。

劉駿在位期間,加強中央集權,撤除「錄尚書事」職銜,並分割州、郡以削弱藩鎮實力;誅中書令王僧達、丹陽令顏竣,討誅隨王劉誕,剷除強臣。崇禮佛教,尊奉高僧僧導,率公卿親臨瓦官寺聽宣《維摩詰經》;詔令整肅佛門,勒令不法僧人還俗;史載劉駿天性好色,臨幸不避戚誼,並有與母后路惠男亂倫之嫌疑,流傳後世。

464年7月12日,劉駿病逝於建康宮玉燭殿,享年三十五歲,在位十一年。8月27日,奉葬景寧陵。

史載劉駿其人機警聰慧,博學多聞並文采華美,讀書能七行俱下,又雄豪尚武,擅長騎射。劉駿病逝後,吏部尚書蔡興宗稱其為「守道之君」(「以道始終」);然而劉駿生性喜奢、欲求無度,晚年「尤貪財利」、不聽善諫,以致原本讚許他德行的士族,也感嘆「天下失望」;更兼大明末年,浙江大旱,通貨膨脹失控、浙江的人民餓死十分之六、七,依《宋書‧州郡志》記載之戶口推算,飢餓致死者最高可能有三十萬人。南朝梁史家裴子野總結劉駿「威可以整法,智足以勝奸,人君之略,幾將備矣。」卻也嘆道:「夫以世祖才明,少以禮度自肅,思武皇之節儉,追太祖之寬恕,則漢之文景,曾何足云!」

劉駿生於南朝宋文帝元嘉年間(430年9月19日),為宋文帝第三子。435年,年僅六歲便受封武陵王,食邑二千戶;439年,時年十歲,受詔都督湘州諸軍事、征虜將軍、湘州刺史,領石頭戍事;440年,遷使持節、都督南豫、豫、司、雍、并五州諸軍事、南豫州刺史,仍任征虜將軍,戍守石頭城;444年,加都督秦州,進號撫軍將軍;隔年(445年),時年十六歲,受詔改任都督雍、梁、南北秦四州,荊州之襄陽、竟陵、南陽、順陽、新野、隨六郡諸軍事、甯蠻校尉、雍州刺史,持節,仍任撫軍將軍。自東晉偏安江東後,劉駿為南朝第一位出鎮襄陽的皇室子弟。449年,受詔改任都督南兗、徐、兗、青、冀、幽六州、豫州之梁郡諸軍事、安北將軍、徐州刺史,持節如故,北鎮彭城。不久宋文帝又下詔加任劉駿為兗州刺史,次子始興王劉濬為南兗州刺史,因此劉駿都督南兗州的職銜當即撤銷。

450年,北魏太武帝拓跋燾率兵南侵,宋文帝詔令劉駿領兵北襲屯駐於汝陽的北魏永昌王拓跋仁。劉駿領一千五百兵馬進襲汝陽,魏兵因無防備而潰敗。但之後探得宋軍並無援軍,因而反戈一擊,宋軍大敗,士兵僅有九百人生還。5月19日,劉駿因汝陽戰敗,降號為鎮軍將軍。451年3月19日,魏軍解圍盱眙北還。4月13日,因防禦北魏入侵無功,宋文帝再下詔降劉駿為北中郎將。

452年,劉駿時年二十三歲,加封都督南兗州軍事,擔任南兗州刺史,鎮守山陽,不久改任都督江州、荊州之江夏、豫州之西陽、晉熙、新蔡四郡諸軍事、南中郎將、江州刺史,持節如故。當時江寇橫行,宋文帝派遣步兵校尉沈慶之討賊,由劉駿全權統領征討大軍。劉駿的親信顏竣,曾於彭城假托沙門僧語,散佈劉駿當為「真人」的符讖謠言,並傳至京師。宋文帝欲行加罪,卻因爆發太子劉劭詛咒皇帝的巫蠱事件,故對劉駿和顏竣暫時不予治罪。

453年3月16日深夜,劉駿長兄、皇太子劉劭趁夜帶兵入宮弒君,宋文帝遇害。劉劭稱帝,進號劉駿為征南將軍、加任散騎常侍,以示攏絡,卻矚使步兵校尉沈慶之殺害劉駿。沈慶之受命後求見劉駿,劉駿稱病不敢接見。沈慶之便闖至劉駿面前,將劉劭的手書呈遞。劉駿涕泣請求沈慶之讓自己與母親路淑媛訣別。沈慶之說:「下官受先帝厚恩,常願報德,今日之事,唯力是視,殿下是何疑之深!」劉駿聽此言,便起座再拜說:「家國安危,在於將軍。」遂由沈慶之處分內外。453年4月11日,劉駿戒嚴示眾,起兵討逆。荊州刺史南譙王劉義宣、雍州刺史臧質響應義舉。5月1日,劉駿移檄建康(今南京市);14日,冠軍將軍柳元景與劉劭大戰於新亭,劉劭敗逃;三天後,劉駿兵進江寧;18日,江夏王劉義恭來降,奉表上尊號;隔日,劉駿進駐新亭,使散騎侍郎徐爰草制即位禮儀。

453年5月20日,武陵王劉駿於新亭即皇帝位,大赦天下,時年二十四歲;27日,攻陷建康城,斬偽皇帝劉劭及二兄劉濬。

454年3月17日,南郡王劉義宣、江州刺史臧質、豫州刺史魯爽、兗州刺史徐遺寶舉兵造反。因新皇即位日淺,朝廷得報大懼。劉駿甚至想奉呈乘輿法物迎劉義宣即位,竟陵王劉誕當即阻止,說:「奈何持此座與人?」劉駿乃止。4月19日,安北司馬夏侯祖歡擊破徐遺寶;6月1日,鎮軍將軍沈慶之於曆陽之小峴大破魯爽,將其斬決;29日,劉義宣及臧質率軍攻梁山營壘,豫州刺史王玄謨派遣遊擊將軍垣護之、竟陵太守薛安都出壘迎戰,擊敗臧質。垣護之因風縱火,劉義宣及臧質大敗而逃;7月13日,臧質遭斬;8月4日,賜死劉義宣於江陵獄中。

455年8月29日,因武昌王劉渾自號楚王、擅訂年號(永光),潛越禮制,下詔將其廢為庶人,賜死。

459年,劉駿暗示有司核奏竟陵王劉誕不法,貶爵為侯,並任命垣閬為兗州刺史,以赴鎮所為名,趁機襲擊劉誕。事泄失敗,垣閬被殺。6月4日,劉誕聚眾造反,佔據廣陵城,劉駿派遣車騎大將軍沈慶之率兵平叛;9月22日,攻下廣陵,將劉誕斬首,殺光城內的三千男丁,女子賞賜給兵士。

劉駿是一個頗有作為、積極改革制度的皇帝。他加強中央集權,撤除「錄尚書事」職銜,並分割州、郡以削弱藩鎮實力。454年7月28日,因揚、荊二州地大兵多,刺史易生異志,劉駿下詔分割揚州、浙東五郡為「東揚州」,並由荊、湘、江、豫四州分割出八郡,劃歸「郢州」,荊、揚二州自此削弱;撤除「南蠻校尉」一職,戍兵移鎮建康,增強京師武備。同年(454年),劉駿因劉義宣叛亂,有意削弱諸王侯權勢,江夏王劉義恭於是奏請裁損諸王侯車服器用、樂舞制度九條,劉駿准奏後,更另有司增訂至二十四條,全面抑制藩王地位,威福獨專。宗王兄弟中只有七弟劉宏被親愛重用,455年成為宰相(458年卒)。孝武帝同時重用江東寒門沈慶之與傖荒北人柳元景,依照兩人的功績,先後提拔為三公,開啟吳興沈氏與河東柳氏攀升為南朝高門的起始之路,並開創南朝寒門、寒人以軍功升為三公的先例。

458年(大明二年),在外放顏竣並處死王僧達後,劉駿欲大權獨攬、專擅朝綱,因此除了高門蔡興宗與袁顗以外,從此不再放權給宗王兄弟與高門強族的大臣,專委任倖臣充作耳目,隱刺朝政,形成後代所謂「寒人掌機要」的政治局面,孝武帝的集權統治也被史書稱為「主威獨運,官置百司,權不外假」。倖臣當中,戴法興、巢尚之、戴明寶、徐爰四人,最有理政才幹,因此大受寵幸,事必與議。巢尚之及徐爰尤知謹慎,惟戴法興及戴明寶卻因此作威作福、納賄受貨,門庭若市,身價並達千金。戴明寶尤其驕縱,放任長子戴敬出錢搶買皇帝的御用物,甚至於劉駿出巡時,騎馬於御輦旁來回奔馳,毫無顧忌。劉駿大怒,下令處死戴敬並將戴明寶下獄,不久仍釋放,委以重任如初。而戴法興於劉子業任皇太子時即奉命侍從,後更受劉駿遺命託孤,輔佐劉子業繼位(宋前廢帝),以致宋前廢帝時有民間謠言:「戴法興為真天子,皇帝為假天子。」之語,權重若此。

劉駿生性嚴峻寡恩,對待左右侍臣,動輒屠戮;甚且自詡風流,晚年專喜戲謔大臣,各取綽號,無禮之至,惟吏部尚書蔡興宗方直嚴肅,劉駿憚怕之,不敢侵狎;平時飲食起居極盡奢華,宮殿牆柱及地板皆鋪錦繡,又嫌宮廷狹小,特命建「玉燭殿」以供享樂,並破壞其祖父、宋武帝生前所居密室,做為地基,並率大臣圍觀動工。見床頭用土作鄣,牆上掛葛燈籠、麻繩拂,侍中袁顗便稱讚宋武帝有節儉樸素之德,劉駿自以為名士派頭,瞧不起沒文化的祖父劉裕,批評說:「田舍公得此,以為過矣!」(「鄉下人能用這些東西,已經太過了!」)

劉駿生性好賭,揮霍不少,加上國家戰亂之後,中央府庫空虛、無錢可使,便效法桓玄手段,以賭博斂財。詔命凡各州刺史及二千石官員,卸職還都時須獻奉財物,限期繳納。其後更召入宮中賭博作樂,賺盡地方官於其任上所積錢財,方准離去。這種收稅辦法被後任的宋、齊皇帝沿用並發揚光大,直接強逼刺史「獻奉」,省略掉賭博這種相對體面的手法。;劉駿晚年喜好飲酒,常飲至深夜,隔日起床洗漱完畢後,便繼續喝至大醉,整日嗜睡。然而有奏疏馳至,便立刻整理好儀容,毫無醉態。宮中內外都佩服他的機神明肅,不敢偷懶懈怠。

大明七年(463年)底至八年(464年),浙江等地因為劇烈旱災,造成嚴重的大饑荒,浙江十分之六的戶口餓死逃散。宋朝史家司馬光因此批評劉駿,說他晚年好酒奢靡,以致原本強盛的劉宋,在他執政末年中衰。

454年,劉駿召幸南郡王劉義宣(六叔)的幾個女兒,劉義宣於是憎恨劉駿,隨後在江州刺史臧質的慫恿下,起兵造反。造反失敗,劉義宣遭誅。劉駿可能便秘密納娶其中一位堂妹(劉駿為避人耳目,冊封其為殷淑儀),並與其生下第八子劉子鸞等五子一女,但也有說法認為殷淑儀並非劉氏女。

史載劉駿與母親路太后有亂倫之嫌疑。南朝人沈約所著《宋書》之記載較為含蓄,內文如下:「上於閨房之內,禮敬甚寡,有所御幸,或留止太后房內,故民間喧然,咸有醜聲。宮掖事秘,莫能辨也。」——《宋書‧列傳第一‧后妃傳》

《宋書》指劉駿常於路太后所居顯陽殿中臨幸宮女,因停留時間過久,以致民間謠傳其間有不可告人之事。《宋書》作者沈約並無否認,只模稜兩可地表示:「宮掖事秘,莫能辨也。」

然而由北朝人魏收所著的《魏書》就沒有顧忌,直接指涉劉駿與其母亂倫:「駿淫亂無度,蒸其母路氏,穢汙之聲,布於甌越。」——《魏書‧列傳第八十五‧島夷劉裕傳》

魏收還記述劉駿天性好色、狎褻無度,以致其兒子、宋前廢帝劉子業即位後,指著劉駿的畫像罵:「此渠大好色,不擇尊卑!」

但也有人認為記載不實。唐朝史家劉知幾在其著作《史通》中辯誣說:「沈氏著書,好誣先代,於晉則故造奇說,在宋則多出謗言,前史所載,已譏其謬矣。而魏收黨附北朝,尤苦南國,承其詭妄,重以加諸。遂云馬睿出於牛金,劉駿上淫路氏。可謂助桀為虐,幸人之災。」

464年7月12日,劉駿病逝於玉燭殿,享年三十五歲,在位十一年。皇太子劉子業繼位,是為宋前廢帝。8月27日,奉葬位於丹陽郡秣陵縣岩山(今南京市江寧區秣陵鎮)的景寧陵,予諡「孝武皇帝」,廟號「世祖」。


\subsubsection{孝建}

\begin{longtable}{|>{\centering\scriptsize}m{2em}|>{\centering\scriptsize}m{1.3em}|>{\centering}m{8.8em}|}
  % \caption{秦王政}\
  \toprule
  \SimHei \normalsize 年数 & \SimHei \scriptsize 公元 & \SimHei 大事件 \tabularnewline
  % \midrule
  \endfirsthead
  \toprule
  \SimHei \normalsize 年数 & \SimHei \scriptsize 公元 & \SimHei 大事件 \tabularnewline
  \midrule
  \endhead
  \midrule
  元年 & 454 & \tabularnewline\hline
  二年 & 455 & \tabularnewline\hline
  三年 & 456 & \tabularnewline
  \bottomrule
\end{longtable}

\subsubsection{大明}

\begin{longtable}{|>{\centering\scriptsize}m{2em}|>{\centering\scriptsize}m{1.3em}|>{\centering}m{8.8em}|}
  % \caption{秦王政}\
  \toprule
  \SimHei \normalsize 年数 & \SimHei \scriptsize 公元 & \SimHei 大事件 \tabularnewline
  % \midrule
  \endfirsthead
  \toprule
  \SimHei \normalsize 年数 & \SimHei \scriptsize 公元 & \SimHei 大事件 \tabularnewline
  \midrule
  \endhead
  \midrule
  元年 & 457 & \tabularnewline\hline
  二年 & 458 & \tabularnewline\hline
  三年 & 459 & \tabularnewline\hline
  四年 & 460 & \tabularnewline\hline
  五年 & 461 & \tabularnewline\hline
  六年 & 462 & \tabularnewline\hline
  七年 & 463 & \tabularnewline\hline
  八年 & 464 & \tabularnewline
  \bottomrule
\end{longtable}


%%% Local Variables:
%%% mode: latex
%%% TeX-engine: xetex
%%% TeX-master: "../../Main"
%%% End:

%% -*- coding: utf-8 -*-
%% Time-stamp: <Chen Wang: 2019-12-20 13:53:07>

\subsection{前废帝\tiny(464-465)}

\subsubsection{生平}

刘子业(449年2月25日-466年1月1日),小字法师,中國歷史南北朝時期南朝宋皇帝,史稱「前廢帝」。他是宋孝武帝刘骏长子,生母為文穆皇后王憲嫄。年号“永光”、“景和”。宋前廢帝以皇太子身份即位,但即位之初受制於掌權大臣而難以專政,遂於即位一年後就先將主政大臣戴法興誅殺,接著又將圖謀廢立的三名顧命大臣殺害,其中更殘忍肢解了叔祖父江夏王劉義恭。此後前廢帝肆意行事,荒淫无道,做了很多殘暴甚至亂倫的行為,約半年後就在阮佃夫等人策劃下,被主衣壽寂之刺殺。

劉子業於元嘉二十六年正月十七日(449年2月25日)出生,四年後就發生了太子劉劭弒宋文帝奪位的事件,因為孝武帝起兵討伐劉劭,劉子業被劉劭囚於侍中下省。同年,孝武帝即位,於翌年孝建元年(454年)立了子業為皇太子。不過,子業一直居於永福省,在大明二年(458年)才出居東宮。子業在東宮多有犯錯,而孝武帝亦寵愛殷淑儀以及和她生下的皇子劉子鸞,於是一度有了廢子業,立子鸞的想法,但時為侍中的袁顗稱讚子業好學,天天進步,終也保住了其太子之位。

大明八年閏五月廿三日(464年7月12日),孝武帝去世,同日子業以皇太子繼位,是為宋前廢帝。孝武帝死前指定了江夏王劉義恭、柳元景、顏師伯、沈慶之及王玄謨五人為顧命大臣,分掌朝事以及軍旅之事。不過,前廢帝即位後朝事其實都繼續由孝武帝寵臣越騎校尉戴法興及中書通事舍人巢尚之掌握,義恭等雖錄尚書事仍只守空名。前廢帝即位後不久獲尊為皇太后的生母王憲嫄病重,遂派人召廢帝前來,但廢帝卻說:「病人房間裡有很多鬼,太可怕了,這怎麼能去呢?」竟拒絕探望母親,不久太后便過世。而前廢帝的命令和活動此時亦受戴法興所約束,意願常常被法興壓下,法興甚至多次對廢帝說:「你這樣的作為,想成為營陽王嗎?」這令廢帝很不滿,於是與痛恨戴法興的宦官華願兒勾結誣陷法興,終於永光元年八月初一(465年9月6日)賜死了戴法興。

前廢帝又為了削弱時任尚書僕射的顏師伯的權力,故意重設左右僕射,以王彧為右僕射,更加奪其兼丹陽尹之職,令師伯深感不安。而前廢帝日漸顯露的狂悖行徑亦令柳元景、劉義恭等人十分憂心,於是義恭與元景、師伯等人陰謀廢帝而立義恭,但久未有決定,又嘗試尋求沈慶之支持,但慶之卻向廢帝告發圖謀。永光元年八月十三日(465年9月18日),廢帝親領禁軍宿衞去收捕柳元景,就地將其殺害;又領兵到義恭府第殺害義恭,更下令肢解義恭,甚至將義恭眼晴拿出來浸在蜜糖中,稱之為「鬼目粽」。二人皆被夷滅三族,顏師伯、劉德願等亦被誅殺。

沈慶之因與義恭並不親厚,又與師伯有私憾,遂告發了圖謀,廢帝亦以沈慶之為太尉以褒賞他。袁顗當日在孝武帝面前保廢帝太子之位,廢帝本亦感其恩德,加上沈慶之亦念在袁顗提拔之恩,袁顗遂得以在義恭等人被殺後入為吏部尚書,與慶之、徐爰領選事。然而,很快袁顗就因不合廢帝心意而獲罪,白衣領職,袁顗在恐懼之下自求外任,終獲授雍州刺史,遠赴襄陽。而留在朝中的沈慶之盡心對廢帝的荒唐行為作出規勸,也令廢帝很不滿。廢帝後來將姑姑新蔡公主劉英媚納於後宮,向外謊稱她是謝貴嬪,宣稱公主已死並以一個婢女的屍體冒充,送到公主丈夫何邁那裏供治喪用。何邁本亦已受猜忌,早有廢立的準備,打算趁廢帝出遊時下手,但圖謀外泄,景和元年十一月,何邁又遭廢帝親自領兵誅殺。殺何邁前,廢帝知沈慶之一定會來反對,於是命人封鎖清豁諸橋阻止其前來,年已八十的慶之無法進見後回府,廢帝更派了與慶之有過節的沈攸之送藥賜死他。

新安王劉子鸞一度危及廢帝太子之位,廢帝在誅除義恭等人後開始對其進行報復,景和元年九月十一日(465年10月16日),就下令賜死子鸞,同為殷淑儀所生的十二皇女以及劉子師都一同被賜死,並開挖殷淑儀的墓穴,又怪罪寫了《宋孝武宣貴妃誄》的謝莊,甚至想掘開父親陵墓景寧陵,只因太史勸阻而不成事,不過仍然用糞便弄污陵墓。另一方面,前廢帝亦忌憚一眾叔父威脅他的地位,其中九叔義陽王劉昶早在孝武一朝就有謀反的傳言,到廢帝在位年間就更盛,廢帝亦常對旁人稱他即位以來未試過戒嚴,有所不快。劉昶在義恭等人被殺後上表入朝,廢帝就向陪使者入都的劉昶典籤籧法生宣稱劉昶與義恭合謀反叛,入朝正好,但又斥責法生沒有通報劉昶謀反的訊息。法生聞言恐懼,於是立即逃到彭城告知劉昶,而廢帝就以此為由北討,於九月己酉親征彭城,並宣布內外戒嚴。劉昶試圖起兵抵抗但無人支持,只好逃到北魏。

剩下諸叔,廢帝將他們都囚於殿內毆打欺凌,其中他最忌憚較年長的十一叔湘東王劉彧、十二叔建安王劉休仁及十三叔山陽王劉休祐,因他們都很肥壯而命人用竹籠載著他們量度體重,最重的劉彧被稱為「豬王」,休仁及休祐分別獲得「殺王」及「賊王」之號,並時常命他們跟著自己。才能差劣的八叔東海王劉褘也被稱為「驢王」,只有年紀尚輕的桂陽王劉休範及巴陵王劉休若過得好點。廢帝曾經脫光劉彧,將他放到坑中,並將飯菜都倒在木槽中混合,讓坑中的劉彧像豬一樣到木槽進食,以作娛樂;又常想殺害三王,但因劉休仁用其計策取悅廢帝,廢帝將就一直沒有殺他們。不過,廢帝卻屢次逼姦宮中妃主,例如命身邊官員侍從當著休仁面前逼姦休仁生母楊太妃,又曾威逼南平王妃江氏就範,在她堅拒後殺掉了她的三個兒子南平王劉敬猷、廬陽王劉敬先及南安縣侯劉敬淵。因著文帝及孝武帝在兄弟中皆排名第三,得入繼帝位,廢帝對三弟晉安王劉子勛亦很猜忌,而何邁的廢立圖謀都是以子勛取代廢帝。何邁失敗後,廢帝乘此指控子勛與何邁謀反,派了使者賜死子勛,以鄧琬為首的子勛屬官們最終決定起兵抗命,在十一月十九日於尋陽宣布戒嚴。

前廢帝表現無道,蔡興宗更曾經向在軍中有威望的沈慶之及王玄謨明言起事推翻廢帝,又曾向掌禁軍的右衞將軍劉道隆暗示,但都遭對方拒絕。相反,前廢帝以美女財帛等東西賜給宗越、譚金、童太一及沈攸之等將領,讓他們甘心為前廢帝服務,為其爪牙。而湘東世子師阮佃夫見劉彧被囚於殿內,常面臨被殺危機,就與王道隆、李道兒、禁中將領柳光世等人以及淳于文祖、繆方盛等前廢帝身邊近臣密謀廢立。景和元年十一月二十九日,前廢帝打算出巡荊湘二州,並欲在出發前將劉彧等三王殺害,阮佃夫在前廢帝早上出華林園時將圖謀密告主衣壽寂之、細鎧主姜產之等人,獲得響應,其中防守華林閤的隊主樊僧整也加入了。當晚入夜後,廢帝在竹林堂前與巫師射鬼,壽寂之就領著姜產之等人衝進去行刺廢帝,廢帝見寂之衝過來就用箭射他,但沒有命中,於是逃跑,但被寂之追上,被殺,得年十七歲。

史載前廢帝幼而狷急,故任太子期間屢遭孝武帝指責,如孝武帝西巡時命廢帝參侍侯起居,就因為字跡差而被罵,甚至被孝武帝指他「素都懈怠,狷戾日甚,何以固乃爾邪!」廢帝即位後初亦受制於戴法興等人,但自法興等被殺後,就沒有人敢阻遏廢帝行事,很多大臣都被打,人心騷動。

廢帝雖然多有猜忌逼害宗室的舉動,但對於同胞所生的劉子尚及山陰公主劉楚玉卻相當親近,經常一同行動,子尚性情亦有如廢帝那樣。廢帝曾應公主的要求賜其面首三十,並命當時的美男子尚書吏部郎褚淵侍候公主十天。

不過廢帝年輕時都有好學一面,故也對古事有一定認識,所作的《世祖誄》及一些雜篇都不乏有文采的地方,又曾仿效曹操設立發丘中郎將及摸金校尉兩職。

廢帝在母親病重時不肯探望,母親死後卻在其夢中出現,並說:「汝不仁不孝,本無人君之相,子尚愚悖如此,亦非運祚所及。孝武險虐滅道,怨結人神,兒子雖多,並無天命;大命所歸,應還文帝之子。」

廢帝曾在華林園竹林堂命宮女們裸身追逐以供自己取樂,其中一名宮女不肯,廢帝就殺了她。不久,廢帝卻夢見後堂有一女子罵他,廢帝醒來後大怒,遂命人在宮中找出一個和夢中女子相貌相似的宮人,又將她殺死。就在當晚,這個宮人就在廢帝夢中出現,說已經將被枉殺的事告知上帝。後巫師宣稱竹林堂有鬼,才有廢帝前往射鬼,遭壽寂之刺殺的事。

\subsubsection{永光}

\begin{longtable}{|>{\centering\scriptsize}m{2em}|>{\centering\scriptsize}m{1.3em}|>{\centering}m{8.8em}|}
  % \caption{秦王政}\
  \toprule
  \SimHei \normalsize 年数 & \SimHei \scriptsize 公元 & \SimHei 大事件 \tabularnewline
  % \midrule
  \endfirsthead
  \toprule
  \SimHei \normalsize 年数 & \SimHei \scriptsize 公元 & \SimHei 大事件 \tabularnewline
  \midrule
  \endhead
  \midrule
  元年 & 465 & \tabularnewline
  \bottomrule
\end{longtable}

\subsubsection{景和}

\begin{longtable}{|>{\centering\scriptsize}m{2em}|>{\centering\scriptsize}m{1.3em}|>{\centering}m{8.8em}|}
  % \caption{秦王政}\
  \toprule
  \SimHei \normalsize 年数 & \SimHei \scriptsize 公元 & \SimHei 大事件 \tabularnewline
  % \midrule
  \endfirsthead
  \toprule
  \SimHei \normalsize 年数 & \SimHei \scriptsize 公元 & \SimHei 大事件 \tabularnewline
  \midrule
  \endhead
  \midrule
  元年 & 465 & \tabularnewline
  \bottomrule
\end{longtable}


%%% Local Variables:
%%% mode: latex
%%% TeX-engine: xetex
%%% TeX-master: "../../Main"
%%% End:

%% -*- coding: utf-8 -*-
%% Time-stamp: <Chen Wang: 2019-12-20 13:44:50>

\subsection{明帝\tiny(465-472)}

\subsubsection{生平}

宋明帝劉彧(439年12月9日-472年5月10日),字休炳,小字榮期,南朝宋第七任皇帝。劉彧生於元嘉年間,為宋文帝劉義隆第十一子,先後受封淮陽王、湘東王。宋前廢帝劉子業即位,顧慮諸叔威脅皇位,趁劉彧入朝時將其拘留殿中,並因劉彧體胖而封其為「豬王」,大肆羞辱,且屢次欲加殺害,都因始安王劉休仁諂媚化解,才保全性命。劉子業遭壽寂之殺害後,劉休仁便奉迎劉彧即皇帝位,改元泰始,大赦天下。

劉彧在位六年半,執政前期眾親王及方鎮相繼叛變,朝廷頻繁動武平亂,國力逐漸耗損。北魏也趁機侵略,佔領山東、淮北等地區,北朝國力自此超越南朝;劉彧為防範宋孝武帝劉駿諸子奪取皇位,殺盡諸姪子,致使劉駿絕後;晚年尤多忌諱,文書奏折不得出現諱字,犯禁者一律誅殺。

472年5月10日,劉彧逝世,享年三十四歲,庙号太宗,谥号明帝,奉葬高寧陵。

史載劉彧個性寬和仁慈,儀容端雅,喜好文學。即位後雖然四方反抗但用人不疑,能使將士效忠不貳。然而晚年好猜忌,對待皇族及侍臣動輒殘忍刑戮;國家連年征伐,國庫空虛,而劉彧卻奢侈無度,致使「天下騷然,民不堪命」,劉宋國運自此衰敗。

劉彧生於南朝宋文帝元嘉十六年十月戊寅(439年12月9日),九歲時受封淮陽王,食邑二千戶。452年,改封湘東王。劉彧三哥、宋孝武帝劉駿即位後,歷任郡太守、中護軍、侍中、衛尉、游擊將軍、左衞將軍、都官尚書、領軍將軍等職銜,並獲賜鼓吹一部。453年,劉彧生母沈容姬逝世,劉彧時年十四歲,由路太后撫養於宮中,特受寵愛,時常侍奉路太后醫藥,也因此為劉駿所親愛,不招致猜忌。宋前廢帝劉子業繼位後,任命劉彧為州刺史,都督州郡軍事,並得以本號開府儀同三司。

宋前廢帝劉子業即位後荒淫無道,殺戮群臣,並恐諸叔覬覦皇位,欲加殺盡。劉彧於景和年間入朝,遭拘留宮中,百般毆打凌辱。劉彧與始安王劉休仁、山陽王劉休祐皆體型肥胖,被劉子業封為「豬王」、「殺王」、「賊王」,並將三人用竹籠囚禁。劉子業又命人掘地為坑,灌滿泥水,再以木槽盛飯,並用雜食攪和後置於坑前,命劉彧裸體於泥坑中以口對木槽中就食,戲謔為豬。劉彧曾因抗拒羞辱而惹怒劉子業,劉子業命將其裸體後用竹杖綁住四肢抬付太官,說:「即日屠豬。」劉休仁在旁笑說:「豬今日未應死。」劉子業問何故,劉休仁答說:「待皇太子生,殺豬取其肝肺。」劉子業聽後怒火漸息,命交付廷尉,劉彧才逃過死劫。466年,劉子業欲南遊荊州及湘州,決定明日殺害劉彧等諸叔父後,即行出發。劉彧遂與心腹阮佃夫、李道兒等共謀弒君。1月1日夜,阮佃夫與李道兒暗中結交宮中侍臣壽寂之,於華林園將劉子業殺害。劉子業死後,劉休仁隨即奉迎劉彧入宮即位,並令人備皇帝羽儀。由於事起倉促,劉彧半途失落鞋子,跛著走入西堂,仍戴著象徵臣子的烏紗帽,劉休仁讓人給其戴上白紗帽後,便擁劉彧登上御座召見文武百官,接受歡呼禮拜,凡事以「令書」頒布施行。隔天(1月2日),劉彧下令殺掉劉子業的同母次弟劉子尚,以絕後患。

泰始元年十二月丙寅(466年1月9日),劉彧於宮中太極前殿登基為帝,大赦天下。

465年底,宋孝武帝劉駿第三子、晉安王劉子勛為反抗劉子業謀害己命,在鄧琬等人輔佐下,於江州起兵叛亂。劉彧弒君自立後,授姪子劉子勛官爵遭拒。劉子勛甚至在鄧琬的主導下傳檄天下,改討劉彧。466年2月7日,鄧琬、袁顗等奉年僅十歲的劉子勛於尋陽城登極稱帝,年號「義嘉」,另立政府。江州的義嘉政權得到幾乎全國的承認與響應,南朝各州郡皆向劉子勛上表稱臣,改用義嘉年號,並向尋陽奉貢。劉彧所統治區域僅限京師建康(今江蘇省南京市)附近的丹陽、淮南等郡百餘里土地而已,形勢極為嚴峻。

劉彧聞變後隨即任命劉休仁為征討大都督,統帥全軍,王玄謨為副手。任用吳喜、江方興等為東路軍將領,討平會稽、吳、吳興、晉州等東南各州郡,俘虜劉駿第六子、尋陽王劉子房;任用劉勔、張永、蕭道成等為北路軍將領,擊敗殷琰、薛安都等敵對將領,抵擋住北方的攻勢。任命沈攸之、張興世等為西路軍將領討伐劉子勛的尋陽政權,擊敗袁顗、劉胡等人,攻入尋陽,捕斬敵對天子劉子勛,義嘉政權滅亡。隨後宋軍陸續平定江南及淮南各地,「義嘉之難」平息。

由於劉彧登基時,其諸弟(宋文帝劉義隆子嗣)皆在京師,多擁戴兄長劉彧為帝;而劉子勛起兵地方,方鎮都督則多為劉子勛的兄弟(宋孝武帝劉駿子嗣),皆起兵支持劉子勛的義嘉政權。南朝宋即形成文帝系諸王與孝武帝系諸王的內戰局面。劉彧鑑於此,於戰事平定後接受劉休仁的建議,將劉駿在世諸子皆诛殺殆盡,劉駿二十八子自此滅絕。

劉彧於平定叛亂後欲逞兵威,命張永及沈攸之率領重兵,往迎已於義嘉之難後投降的徐州刺史薛安都。薛安都恐劉彧趁機圖己,便向北魏輸誠,乞師自救,汝南太守常珍奇也舉懸瓠城降魏。467年初,北魏派遣尉元、孔伯恭領兵救徐州,另派拓跋石、張窮奇領兵救懸瓠,兗州刺史畢眾敬望風迎降。魏將尉元隨後於呂梁一帶大敗宋將張永及沈攸之,宋軍幾乎全軍覆沒,張永、沈攸之隻身逃回江南,徐、兗二州淪陷;467年2月,青州刺史沈文秀、冀州刺史崔道固投降北魏,旋即又於4月歸降劉宋。北魏遂派遣長孫陵、慕容白曜往攻青州,劉彧命沈攸之領兵救援,卻於睢清口遭魏將孔伯恭擊敗,退守淮陰。青州與冀州待援不至,被圍攻數年,先後降魏,青、冀二州也淪陷。

南朝宋本地處江南,國狹民脊,自此再失四州,國力更形衰弱;再加上戰亂不斷,劉彧為獎賞有功將士,大肆封賞封官,造成國庫空虛、士族制度嚴重破壞,削弱南朝宋的執政根基,北朝國力從此超越南朝。

劉彧晚年害怕諸弟在他死後奪取太子劉昱的皇位,於是接受倖臣王道隆與阮佃夫的建議,大殺立過軍功的諸弟,只有劉休範因為人才凡弱而留下未殺。王道隆與阮佃夫掌權後擅用威權、官以賄成,富逾公室。劉彧同時殺害可能會不利於太子的重要大臣,如功臣武將壽寂之、吳喜與高門名士王景文(皇后王貞風之兄、劉彧的大舅子),結果造成劉昱繼位後中央和地方軍鎮互相猜忌、攻伐的政治亂象,使得武將蕭道成因此崛起,最後篡宋建齊。

472年,宋明帝死,太子劉昱繼立,宋明帝遺詔命蔡興宗、袁粲、褚淵、劉勔、沈攸之五人託孤顧命大臣,分別掌控內外重區,另外命令蕭道成為衛尉,參掌機要。其中遺詔雖任命袁粲、褚淵在中央秉政,但實際上接受宋明帝秘密遺命,就近輔佐新帝劉昱,掌控宮中內外大權的人物,是宋明帝最親信的側近權倖——王道隆與阮佃夫二人。

史書大多記載,劉彧晚年失去了生育能力,所以他的兒子們都是借腹生子取來的,他把諸弟新生的男嬰抱為自己的兒子,然後殺掉男嬰的生母。但是史家呂思勉認為這是《宋書》作者沈約,為了迎合當時南齊皇帝所捏造的誣蔑之詞,不足採信,而《南史》與《資治通鑑》則是沿用沈約的說法。呂思勉認為宋明帝生前因為猜忌諸弟而狠心殺弟、流放諸姪,不可能殺其父而養其子、流其兄而立其弟。曾懷疑《宋書》等史料的記載,認為宋明帝的皇后王貞風既然有兩個女兒,說明宋明帝可以生育,因此《宋書》應該是為了強化南齊的合法性,故意加工偽造史料,並被後人延用。

毛泽东在阅读南北朝的史书关于刘彧的传记中,写下了:(登基)“可谓奇矣”。

《宋書》記載劉彧:「少而和令,風姿端雅……好讀書,愛文義……及即大位,四方反叛,以寬仁待物,諸軍帥有父兄子弟同逆者,並授以禁兵,委任不易,故眾為之用,莫不盡力。平定天下,逆黨多被全,其有才能者,並見授用,有如舊臣。才學之士,多蒙引進,參侍文籍,應對左右」、「末年好鬼神,多忌諱,言語文書,有禍敗凶喪及疑似之言應回避者,數百千品,有犯必加罪戮」、「泰始、泰豫之際,更忍虐好殺,左右失旨忤意,往往有斮刳斷截者。時經略淮、泗,軍旅不息,荒弊積久,府藏空竭。內外百官,並日料祿俸;而上奢費過度,務為彫侈。每所造制,必為正御三十副,御次、副又各三十,須一物輒造九十枚,天下騷然,民不堪命……親近讒慝,剪落皇枝,宋氏之業,自此衰矣」

沈約評論劉彧:「太宗因易隙之情,據已行之典,剪落洪枝,願不待慮。既而本根無庇,幼主孤立,神器以勢弱傾移,靈命隨樂推回改。斯蓋履霜有漸,堅冰自至,所從來遠也」

北宋的司馬光評論:「(明)帝猜忍奢侈,宋道益衰」、「夫以孝武之驕淫、明帝之猜忍,得保首領以沒於牖下,幸矣,其何後之有?」

蕭梁的史家裴子野評論:「景和(劉子業)申之以淫虐,太宗易之以昏縱,師旅薦興,邊鄙蹙迫,人懷苟且,朝無紀綱,內寵方議其安,外物已睹其敗矣。」

\subsubsection{泰始}

\begin{longtable}{|>{\centering\scriptsize}m{2em}|>{\centering\scriptsize}m{1.3em}|>{\centering}m{8.8em}|}
  % \caption{秦王政}\
  \toprule
  \SimHei \normalsize 年数 & \SimHei \scriptsize 公元 & \SimHei 大事件 \tabularnewline
  % \midrule
  \endfirsthead
  \toprule
  \SimHei \normalsize 年数 & \SimHei \scriptsize 公元 & \SimHei 大事件 \tabularnewline
  \midrule
  \endhead
  \midrule
  元年 & 465 & \tabularnewline\hline
  二年 & 466 & \tabularnewline\hline
  三年 & 467 & \tabularnewline\hline
  四年 & 468 & \tabularnewline\hline
  五年 & 469 & \tabularnewline\hline
  六年 & 470 & \tabularnewline\hline
  七年 & 471 & \tabularnewline
  \bottomrule
\end{longtable}

\subsubsection{泰豫}

\begin{longtable}{|>{\centering\scriptsize}m{2em}|>{\centering\scriptsize}m{1.3em}|>{\centering}m{8.8em}|}
  % \caption{秦王政}\
  \toprule
  \SimHei \normalsize 年数 & \SimHei \scriptsize 公元 & \SimHei 大事件 \tabularnewline
  % \midrule
  \endfirsthead
  \toprule
  \SimHei \normalsize 年数 & \SimHei \scriptsize 公元 & \SimHei 大事件 \tabularnewline
  \midrule
  \endhead
  \midrule
  元年 & 472 & \tabularnewline
  \bottomrule
\end{longtable}


%%% Local Variables:
%%% mode: latex
%%% TeX-engine: xetex
%%% TeX-master: "../../Main"
%%% End:

%% -*- coding: utf-8 -*-
%% Time-stamp: <Chen Wang: 2019-12-20 13:53:24>

\subsection{后废帝\tiny(472-477)}

\subsubsection{生平}

劉\xpinyin*{昱}(463年3月1日-477年8月1日),字德融,小字慧震,是劉宋第八任皇帝,史稱「後廢帝」,宋明帝長子,母是貴妃陳妙登。在位五年暴戾荒誕,令朝野憂心不已,雖經歷過兩次宗室反亂仍未有改善,反倒更為放肆,終為楊玉夫等人所弒。

劉昱在大明七年正月辛丑日(463年3月1日)出生於衞尉府。宋明帝即位後,在消滅反對自己的晉安王劉子勛政權後,於十月戊寅日(466年11月17日)冊立劉昱為皇太子。翌年才正式取名為「昱」。劉昱五、六歲時才讀書,雖然他有過目不忘的本事,無論做金銀器飾還是衣帽都很優秀;即使未曾學過吹篪,拿到手竟也能吹奏。可是他卻不愛學習,只愛玩樂,當時主管他的官員無法制約,只好向明帝報告,明帝也只命令陳貴妃嚴加督促。泰始六年(470年)劉昱正式出居東宮,制訂了太子元會朝賀之禮及袞冕九章衣,並娶了出身濟陽江氏的江簡珪為太子妃。

泰豫元年四月己亥日(472年5月10日),明帝去世,遺詔以尚書令袁粲、護軍將軍褚淵、尚書右僕射劉勔、征西將軍荊州刺史蔡興宗及安西將軍郢州刺史沈攸之五人為顧命大臣。翌日(5月11日)劉昱正式繼位,但朝政實權其實一直都掌握在明帝倖臣阮佃夫、王道隆和楊運長等人手中,在大臣們和太后王貞風阻遏下,即位之初未能放肆而行。

元徽二年(474年)發生了叔父桂陽王劉休範縱兵進犯建康的事件,顧命大臣劉勔及權臣王道隆戰死,但亂事終在右衞將軍蕭道成指揮下獲平定。同年十一月,劉昱加元服,但此後劉昱就又見變態,其在東宮時已有作的隨意動手打人以及赤腳蹲坐的無禮行為故態復萌,眾人都無法制約他了。元徽三年(475年)秋冬之間劉昱曾多次出行,生母陳太妃則多次乘車跟著看顧他,但他卻愈來愈放肆,太妃也無能為力了。劉昱出行每每都丟低部隊,只帶著身邊隨從就四處去,一直到日落才回來。當時朝野對皇帝行為如斯都相當失望,反而都希望年長而又禮遇士人的建平王劉景素能夠入繼大統;不過陳太妃外戚集團以及阮佃夫等權臣卻憂心這樣會破壞他們的利益,故對景素處處防範。終在元徽四年(476年),劉景素於京口起兵,佃夫等人已作預備,藉蕭道成等人將之消滅。可是,景素被消滅後更讓劉昱變本加厲,竟去到每日都外出的程度,每天都和身邊隨從解僧智及張五兒互相追逐,時而夜出晨歸,時而晨出夜歸,跟著他的人都帶著鋋矛傷害經過的行人和牲畜,百姓不堪其滋擾乾脆日夜都閉戶,街上幾乎都沒行人了。廢帝又更加暴力,不再穿齊衣冠,反常常穿方便活動的小袴褶,隨便就動手打人;又為數十根大棍都各改名號,身邊一定帶著針椎、鑿子和鋸等刑具。

元徽五年(477年),阮佃夫眼見皇帝的行為,決定行廢立,他看準劉昱出遊愛丟低部隊的特點而聯結直閤將軍申伯宗、步兵校尉朱幼及于天寶,要趁他出遊時召其部隊回建康,據城反叛,接著抓住他將之廢掉,改立安成王劉準。不過,那次劉昱沒有如原定北出江乘,于天寶就將圖謀向劉昱告發,劉昱遂將佃夫等人誅殺。佃夫心腹張羊當時逃跑,但還是被抓住,劉昱竟親自駕車在承明門將他輾死。不久,劉昱又忌與阮佃夫交好的散騎常侍杜幼文等人,一次出遊時在幼文府外聽到傳出的音樂聲後決意殺掉他,連同司徒左長史沈勃及游擊將軍孫超之都在劉昱率領宿衞軍下被劉昱親手殺害。杜幼文兄長長水校尉杜叔文在玄武湖北被捕,劉昱就自己執矟騎馬,親自殺死叔文。其殺人取樂的沉溺程度更到只要一日不殺人就悶悶不樂的情況。於是百官都人人自危,朝不保夕。

雖然蕭道成在劉昱即位後先後平定劉休範及劉景素的起事,功勳和名聲都愈來愈大,但劉昱卻更猜忌他。劉昱曾經帶數十人突擊蕭道成的居所,當時蕭道成因暑熱而赤膊臥睡,劉昱命道成站著,然後將道成的腹部當箭靶,拉弓就要射,只因王天恩巧言勸說下才改以無箭頭的箭射。但及後劉昱仍想殺他,命人用木頭刻了道成的身形,在其腹畫箭靶,供自己和身邊隨從射擊,更曾襲擊時道成所在的領軍將軍府,想逼他出來接著殺害,但道成不動,劉昱無可奈何,但仍時時想手殺道成;陳太妃看不過眼,出言責罵,劉昱才收斂下來。

不過,時以「四貴」當政蕭道成因此此時已圖廢立,秘密連絡同為四貴之一的司徒袁粲、褚淵,向他們表示廢立之願。當時袁粲認為劉昱所為是年幼導致,反對廢立,使得蕭道成無法成事。蕭道成遂另結直閤將軍王敬則,而王敬則亦歸款道成,並與劉昱身邊的楊玉夫、陳奉伯等二十五人聯結,伺機行事。元徽五年七月七日(477年8月1日),劉昱再度出行,如常丟下部隊先走,期間張五兒的馬墮進湖中,劉昱一怒之下命人將這匹馬抓到明光亭前,親自將牠殺死宰割,及後又和隨從們玩羌人胡伎小樂,在山崗上鬥跳高,乘露車到青園尼寺。再晚點,劉昱到了新安寺偷狗,再到曇度道人處把狗煮了下酒,到了晚上才醉著回宮中。楊玉夫原本也算得劉昱親信,但劉昱卻突然憎惡了他,更向人說:「明日當殺小子,取肝肺。」這令楊玉夫很恐懼。就在這晚,劉昱命令玉夫等織女星經過時通報他,接著就和內人們穿針,穿完後不勝醉意睡了在仁壽殿東阿氊幄之中。由於廢帝出入無定,宮省不管日夜都是開著門,但人們害怕被被突然發怒的劉昱波及,都不敢出,故宮禁內外都沒有聯絡。玉夫等到二更時確定劉昱已熟睡,與楊萬年進帳以廢帝防身刀殺了他,享年十五歲。楊玉夫等人將劉昱的頭顱割下來,交給王敬則運送到蕭道成府前,大聲敲門通知劉昱之死,但蕭道成卻認為外頭是劉昱派來的軍隊,為了騙他開門而假稱皇上已死,堅持不開門。王敬則無可奈何,只好把劉昱的頭顱越牆丟進府內,蕭道成確認頭顱是劉昱本人之後,這才騎馬直衝皇宮,眾人知道劉昱被殺後都大呼萬歲。死後的劉昱被廢為蒼梧王。

雖然劉昱是宋明帝與貴妃陳妙登的長子,但是由於陳妙登曾經為李道兒的侍妾,所以劉昱的身世也一直被質疑。“民中皆呼废帝为李氏子。废帝后每自称李将军,或自谓李统”。

劉昱葬在丹阳郡秣陵县郊坛西。

\subsubsection{元徽}

\begin{longtable}{|>{\centering\scriptsize}m{2em}|>{\centering\scriptsize}m{1.3em}|>{\centering}m{8.8em}|}
  % \caption{秦王政}\
  \toprule
  \SimHei \normalsize 年数 & \SimHei \scriptsize 公元 & \SimHei 大事件 \tabularnewline
  % \midrule
  \endfirsthead
  \toprule
  \SimHei \normalsize 年数 & \SimHei \scriptsize 公元 & \SimHei 大事件 \tabularnewline
  \midrule
  \endhead
  \midrule
  元年 & 473 & \tabularnewline\hline
  二年 & 474 & \tabularnewline\hline
  三年 & 475 & \tabularnewline\hline
  四年 & 476 & \tabularnewline\hline
  五年 & 477 & \tabularnewline
  \bottomrule
\end{longtable}


%%% Local Variables:
%%% mode: latex
%%% TeX-engine: xetex
%%% TeX-master: "../../Main"
%%% End:

%% -*- coding: utf-8 -*-
%% Time-stamp: <Chen Wang: 2021-11-01 15:04:07>

\subsection{顺帝劉準\tiny(477-479)}

\subsubsection{生平}

宋順帝劉準(467年8月8日-479年6月23日),字仲謨,小字智觀,為劉宋的末代皇帝,為宋明帝劉彧的第三子。但《宋書》卻說宋明帝晚年陽痿,不能人道,所以劉準其實是刘彧之弟桂陽王劉休範与姬妾的儿子,由陈法容所抚养。不過這項記載被史家呂思勉駁斥,認為是《宋書》的作者沈約(南齊的史官)為了討好南齊皇帝,故意編造這樣的史料,好掩飾蕭道成篡位、弒君的罪惡。

泰始五年七月癸丑生,泰始七年封為安成王。477年,後廢帝劉昱被弒之後,劉準在蕭道成的擁立下,于元徽五年七月壬辰即位,是為宋順帝,並封蕭道成為相國、齊王;雖然劉準名義上是皇帝,但是權力都被蕭道成掌握。昇明三年(479年),蕭道成要求劉準禪讓,並且派部將王敬則率兵解送出宮。479年四月,劉準禅位與蕭道成,至此,劉宋滅亡。

蕭道成即位之後,封劉準為汝陰王,遷居丹陽並派兵監管。建元元年五月己未(479年6月23日),監視劉準的兵士聽得門外馬蹄聲雜亂,以為發生了變亂,便殺害劉準,得年12歲(虛龄13歲)。

「願後身世世勿復生天王家。」 被清初三大學者之一黃宗羲的知名政治思想著作《明夷待訪錄》首篇《原君》中引用:「昔人(指南朝宋順帝)願世世無生帝王家,⋯」便是表明若將國家當做產業看待,則全天下眾多覬覦這份產業的人是擋不住的,頂多傳個幾代,殺身之禍報應在他(國君)的子孫上。像是劉準被逼讓位所言、明思宗自殺前對公主所說「妳為何要出生在我們帝王之家?」這些話是多麼悲痛。

\subsubsection{昇明}

\begin{longtable}{|>{\centering\scriptsize}m{2em}|>{\centering\scriptsize}m{1.3em}|>{\centering}m{8.8em}|}
  % \caption{秦王政}\
  \toprule
  \SimHei \normalsize 年数 & \SimHei \scriptsize 公元 & \SimHei 大事件 \tabularnewline
  % \midrule
  \endfirsthead
  \toprule
  \SimHei \normalsize 年数 & \SimHei \scriptsize 公元 & \SimHei 大事件 \tabularnewline
  \midrule
  \endhead
  \midrule
  元年 & 477 & \tabularnewline\hline
  二年 & 478 & \tabularnewline\hline
  三年 & 479 & \tabularnewline
  \bottomrule
\end{longtable}


%%% Local Variables:
%%% mode: latex
%%% TeX-engine: xetex
%%% TeX-master: "../../Main"
%%% End:



%%% Local Variables:
%%% mode: latex
%%% TeX-engine: xetex
%%% TeX-master: "../../Main"
%%% End:
