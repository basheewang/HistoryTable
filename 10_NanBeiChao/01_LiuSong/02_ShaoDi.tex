%% -*- coding: utf-8 -*-
%% Time-stamp: <Chen Wang: 2021-11-01 15:03:14>

\subsection{少帝刘义符\tiny(422-424)}

\subsubsection{生平}

刘义符(406年-424年8月4日),小字车兵,彭城綏輿里(今江蘇省銅山縣)人。中国南北朝時期宋朝的第二位皇帝,宋武帝刘裕长子,母親是夫人張闕。永初三年(422年)即位為帝,但兩年後就因居喪行為不當而遭顧命大臣徐羨之、傅亮等廢黜,不久被殺。

刘义符於義熙二年(406年)出生,其時劉裕已經四十四歲,他對得到這個遲來的兒子亦十分高興,後來更立其為豫章公世子。義熙十二年(416年)劉裕北伐後秦時,劉義符就受任中軍將軍,監太尉留府事,留守建康留府。義熙十四年(418年),劉裕受封宋公,建宋國,又以劉義符為宋公世子。劉裕於元熙元年(419年)進封宋王,並加殊禮,劉義符成為宋王太子。至永初元年(420年)劉裕篡晉自立,劉義符亦被立為皇太子。

永初三年(422年),劉裕病重,特意召見劉義符並告誡他:「檀道濟雖然有才幹和謀略,但沒有遠大志向,絕不像他兄長檀韶那麼難駕御。徐羨之、傅亮,應該沒有異心。謝晦數次跟我出征,頗為機智,若有人懷有異心,那肯定是他了。」並以徐羨之、傅亮、謝晦及檀道濟四人為顧命大臣,又明令往後有幼主繼位都不用母后臨朝,朝事都全交給宰相。劉裕在五月癸亥日(6月26日)去世後,劉義符就於同日即位為帝。

劉義符即位後仍需守父喪,但他在期間表現無禮,常與身邊的人十分親密,遊樂無度。當時特進范泰曾寫書勸諫,但劉義符不聽。而謝晦早年見劉義符身邊的都是小人,就曾向劉裕表示劉義符不是繼承宋室的人選,徐羨之、傅亮及謝晦最終在景平二年(424年)謀廢劉義符,並召江州刺史王弘及南兗州刺史檀道濟入朝,命中書舍人邢安泰及潘盛為內應,然後讓皇太后下令廢劉義符為營陽王。那天早上,謝晦、檀道濟及徐羨之領兵自雲龍門入宮,其時潘盛已經撤去宿衞軍隊,故此無人阻攔謝晦等軍。而劉義符那時在華林園設下市場,並親身去賣物;又開了水道,前一天和身邊親近乘船高歌大叫,並遊玩到天淵池後睡在船上,到了那天早上還沒醒。士兵進宮後殺害劉義符身邊兩個侍者,更傷了其手指,接著就將其扶出東閤,沒收皇帝璽綬,終將其送到吳郡幽禁。六月癸丑日(8月4日),徐羨之命邢安泰於金昌亭弒殺劉義符,劉義符極力反抗,要逃出昌門,但遭追捕者用門閂絆倒,最終遇害,享年十九歲。

徐羨之隨後立了宜都王劉義隆為帝。至元嘉九年(432年)以劉義恭長子劉朗為南豐縣王,作為劉義符的後嗣。

\subsubsection{景平}

\begin{longtable}{|>{\centering\scriptsize}m{2em}|>{\centering\scriptsize}m{1.3em}|>{\centering}m{8.8em}|}
  % \caption{秦王政}\
  \toprule
  \SimHei \normalsize 年数 & \SimHei \scriptsize 公元 & \SimHei 大事件 \tabularnewline
  % \midrule
  \endfirsthead
  \toprule
  \SimHei \normalsize 年数 & \SimHei \scriptsize 公元 & \SimHei 大事件 \tabularnewline
  \midrule
  \endhead
  \midrule
  元年 & 423 & \tabularnewline\hline
  二年 & 424 & \tabularnewline
  \bottomrule
\end{longtable}


%%% Local Variables:
%%% mode: latex
%%% TeX-engine: xetex
%%% TeX-master: "../../Main"
%%% End:
