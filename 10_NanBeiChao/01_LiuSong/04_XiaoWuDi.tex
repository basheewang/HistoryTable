%% -*- coding: utf-8 -*-
%% Time-stamp: <Chen Wang: 2021-11-01 15:03:33>

\subsection{孝武帝刘劭\tiny(453-464)}

\subsubsection{刘劭生平}

刘劭(424年-453年5月27日),字休远,彭城綏輿里(今江蘇省徐州市銅山區)人。他是中国南北朝時期南朝宋宋文帝刘义隆的长子,母為皇后袁齊媯。宋文帝晚年因劉劭與女巫嚴道育交往及行巫蠱而謀廢其太子之位,劉劭於是先發制人發起兵變弒父奪位。但即位不久即遭三弟武陵王劉駿為首的軍隊討伐,兵敗被殺,在位僅一百日。史書因劉劭殺父奪位,不用劉劭為文帝上的廟號及諡號,亦不承認劉劭為刘宋的正統皇帝。

元嘉元年(424年),刘劭出生,時正值劉義隆在服父喪,於是一直到喪期結束,於元嘉三年閏正月丙戌(426年2月28日)才正式宣布長子誕生。元嘉六年三月丁巳(429年5月14日),刘劭被立為皇太子,居於永福省。元嘉十二年(435年),劉劭出居東宮,並娶殷淳女殷氏為太子妃。劉劭愛讀史書,尤其喜歡武事,而不管是親覽宮廷事務還是接待賓客,只要他想作,宋文帝都會順從他。為保護東宮,文帝亦讓東宮兵力與羽林衞兵力相同。

元嘉二十七年(450年),宋文帝北伐,劉劭與沈慶之、蕭思話等人極力反對但不果,但戰事不利,魏軍南攻至長江北岸的瓜步,震動建康,劉劭領兵出守石頭,總統水軍。時北魏遣使求婚,包括劉劭以內群臣都認為應該准許,但一直力主北伐的江湛認為外族無信,反對和親,劉劭於是大怒,厲聲斥責他,並在眾人離去時命隨從推逼江湛,差點將他推倒。劉劭又對文帝說:「北伐敗辱,數州淪破,獨有斬江湛,可以謝天下。」但文帝以北伐乃己意為由拒絕對江湛問罪。不過此後,劉劭每次辦宴都沒邀請江湛,又常和文帝說江湛是佞人,不應親近。

文帝為提倡農耕和種桑養蠶,特意在宮內養蠶。當時有一個叫嚴道育的女巫自稱能夠通靈及使喚鬼怪,劉劭姊東陽公主在婢女王鸚鵡告知之下,向文帝假稱道育擅長養蠶而召入宮中。道育入宮後稱述服用丹藥之事,預告符瑞事後又果真看見奇異事情,劉劭及東陽公主於是都對道育能力深信不疑。文帝次子始興王劉濬一向都依附劉劭,又因二人常有過失,為了不讓文帝知道,就請嚴道育幫忙,最終道育教他們巫蠱之事,以文帝形像造一個玉雕人像,埋在含章殿前。這件事除了他們三人知道外,王鸚鵡、與王鸚鵡私通的養子奴僕陳天興以及黃門慶國都有參與,劉劭更給了陳天興一個隊主職位。東陽公主死後,王鸚鵡亦該嫁人,劉劭及劉濬為守住巫蠱秘密,就自行決定將王鸚鵡嫁給劉濬府佐沈懷遠為妾,也不報告給文帝,只稍稍和臨賀公主提及了。不過,文帝及後知陳天興任隊主,亦知天興與王鸚鵡養母子的關係,特意派人詰問劉劭有關二人之事,劉劭就答稱天興身體壯健故給其職位,而表示鸚鵡還未嫁。鸚鵡當時已嫁給沈懷遠,劉劭怕事情被揭穿,立即通報給劉濬及臨賀公主要他們都稱鸚鵡未定婚嫁;王鸚鵡亦怕陳天興會令二人私通的事曝光,於是請求劉劭殺了天興滅口。不過,天興之死令黃門慶國擔心自身安危,於是將巫蠱之事告知文帝。文帝聞訊既震驚又哀惋,立即就命人收捕王鸚鵡,並在其家中搜到數百張劉劭及劉濬所寫的紙,全都是詛咒巫蠱的文句,又挖出含章殿前的人像。嚴道育就逃亡,成功躲過搜捕之人,並易服為尼姑,匿藏在東宮,有時隨劉濬出京口,有時又住在平民張旿的家。而劉劭及劉濬面對文帝的詰責,驚懼得無法答話,只有一直道歉。元嘉二十八年(451年)至元嘉三十年(453年)幾次的天象變異,令文帝再加東宮兵眾,令東宮擁有一萬兵。

元嘉三十年(453年)二月,劉濬轉任荊州刺史,遂自京口入朝,並載著嚴道育回東宮,打算帶她同赴荊州。不過,那時就有人告發嚴道育化身尼姑,常出入劉濬府內,文帝起初不信,但派人查問下終從兩個婢女口中得悉那真是嚴道育。文帝知二子仍然和嚴道育往來十分傷心,於是命京口送二婢到來,並決定廢太子及賜死劉濬,為此與王僧綽、江湛及徐湛之商討,但久未有決定。文帝亦將決定向劉濬生母潘淑妃透露,潘淑妃就將此事告知劉濬,劉濬報告劉劭後劉劭就決定起事,於是晚晚設宴款待將士,又與心腹張超之、陳叔兒、詹叔兒及任建之籌劃。

二月二十日晚(3月15日),兩婢快將到來,劉劭假傳詔命:「魯秀謀反,汝可平明守闕,率眾入。」於是命張超之召集二千多名士兵作準備,又召各幢隊主聲稱有討伐之事。當晚,劉劭又召見了前太子中庶子蕭斌、太子左衞率袁淑、太子中舍人殷仲素及左積弩將軍王正見入宮,對他們哭著說:「主上信讒,將見罪廢。內省無過,不能受枉。明旦便當行大事,望相與勠力。」蕭斌與袁淑立即就表示反對,勸他再作考慮,劉劭聽後表現憤怒。蕭斌在驚嚇下轉為支持,但袁淑仍舊反對,惟未能讓劉劭回心轉意,劉劭接著向袁淑等人分派袴褶,又分派幾段錦布讓其縛好袴子,作出戰準備。天亮時,劉劭與蕭斌同車準備好出發,停在奉化門等袁淑,但袁淑久久不到,到後又不肯上車,劉劭遂命左右殺害袁淑,接著命部眾如同平常入朝一樣走進宮中。經萬春門時,由於違反東宮兵入宮城的規定,劉劭於是對門衞聲稱受到敕命要帶兵入宮收討,遂成功進宮。接著張超之等數十人就直入雲龍門、東中華門及齋閤,拔刀直奔上合殿。當晚文帝又與徐湛之徹夜密談,當值衞兵至此時仍然在寢,張超之就上前砍殺文帝,並殺掉徐湛之。劉劭知文帝被殺後出坐東堂,由蕭斌持刀侍衞,並派人殺害江湛;左細仗主卜天與率眾進攻劉劭,但失敗被殺。

當日劉劭就即位稱帝,寫詔道:「徐湛之、江湛弒逆無狀,吾勒兵入殿,已無所及,號惋崩衄,肝心破裂。今罪人斯得,元凶克殄,可大赦天下,改元嘉三十年為太初元年。文並賜位二等,諸科一依丁卯。」太初年號是劉劭與嚴道育所商定的,來朝的官員才數十人,劉劭就等不及要即位,即位後又稱疾退入永福省,升文帝靈柩至太極前殿。劉劭及後將先前給諸王及各處的武器回武庫,又誅殺徐湛之、江湛等人的黨羽,並封賞幫助他篡位的官員,後來更將與其有宿怨的長沙王劉瑾等人宗室殺害;又在查閱文帝巾箱時發現王僧綽亦有參與廢太子的圖謀,亦將其殺害。文帝大殮時,劉劭稱疾不敢親往,至入殮後才穿上喪服至文帝靈前,表現得痛心哀慟。然後又向四方派大使,對一眾官員求問治國之道,又減輕賦稅及減少徭役,減省出遊耗費,又分配一些田野山澤給貧民。又先後立妃殷氏為皇后,長子劉偉之為太子。

不過,江州刺史武陵王劉駿、荊州刺史南譙王劉義宣、雍州刺史臧質及會稽太守隨王劉誕等人都拒命,起兵討伐劉劭,並以劉駿為主。劉劭弒父後正值劉駿典籤董元嗣回到建康,於是就命元嗣將自己聲稱的徐湛之弒逆版本報告給劉駿,但元嗣回去後就以實情報告。劉駿一方面派元嗣奉表還都,另一方面卻謀起兵。劉劭知劉駿起兵就責問元嗣,元嗣表示出發時尚未有此事,但劉劭不信,加以拷打後元嗣仍然不招,最終被打死。劉劭亦曾密書當時與劉駿一同討蠻的沈慶之,命其殺死劉駿,但慶之就支持劉駿,反助其統兵東下。面對大軍來攻,劉劭下令中外戒嚴,又自以自己向來習武,對百官說“卿等但助我理文書,勿措意戎陳。若有寇難,吾當自出,唯恐賊不敢動爾”,並由皇后叔父司隸校尉殷沖掌文符,左衞將軍尹弘為軍隊準備衣服,由蕭斌總掌眾事。另又將劉駿及義宣諸子分別軟禁在侍中下省及太倉空屋,並打算殺害尋陽、江陵、會稽三鎮士庶官員留在建康的家眷,但在劉義恭及何尚之勸阻下改變主意,才改為下書表明不問罪。劉劭又命褚湛之守石頭,劉思考鎮東府,蕭斌及劉濬就力勸劉劭率水軍迎擊討伐軍,不過劉義恭則建議以逸待勞,劉劭取信了義恭之策,即使蕭斌力陳對方形勢佔上風,應盡快一決勝負亦未能讓劉劭改變主意。不過,其實劉劭始終都不太相信朝廷舊臣,留義恭住尚書下省外,亦軟禁義恭十二個兒子在侍中下省;另厚待王羅漢及魯秀,將兵權交給二人,大加賞賜財寶和美女以取悅二人,又天天出外慰勞將士,自督修建船艦,又焚毀秦淮河南岸,將百姓都趕到北岸。

大軍臨近,但守石頭的龐秀之卻先一步轉投劉駿陣營,大大動搖人心。四月十九日(5月12日),討伐軍前鋒已到新林,劉劭親上石頭城烽火樓觀敵。二十一日(5月14日),討伐軍進至新亭,柳元景在依山建新亭壘據險自守,而劉劭就召魯秀與王羅漢駐朱雀門,讓蕭斌率步兵、褚湛之率水軍;當時詹叔兒察知討伐軍營壘尚未建立,勸劉劭乘時進擊,但劉劭不肯。翌日,劉劭才命蕭斌率魯秀、王羅漢等精兵共萬人進攻新亭壘,劉劭亦親自登上朱雀門督戰。由於士兵都得劉劭賞賜,故都為他拼命作戰,戰事佔據上風。但就在新亭壘將被攻下時,魯秀突然收兵,柳元景抓緊機會反擊,終扭轉戰局。劉劭在蕭斌等敗後又親率心腹再戰,又遭柳元景擊敗,死傷更大,劉劭斬殺撤退者以圖遏止潰敗之勢,但失敗,劉劭只好走經朱雀門還宮。

此戰敗後,褚湛之、檀和之、魯秀及劉義恭先後叛歸劉駿,劉劭只好向神明之力求助,將蔣侯神像運到宮內,拜他為大司馬,封鍾山郡王;又以蘇侯為驃騎將軍,命南平王劉鑠寫祝文向其宣告劉駿罪狀。五月,劉劭派往抵御劉誕所領東軍的部隊在曲阿戰敗,為了遏阻他進攻,劉劭就焚毀都水西裝及左尚方,以及破壞柏崗及方山土壩,又命守家未服兵役的男丁緣秦淮河竪起舶船,並在上築上大弩作防禦,又命人以柵欄阻斷班瀆、白石等水道。其時男丁不足,甚至要動用婦女完成工事。

五月三日(5月26日),魯秀率五百人進攻大航,將之攻克,守將王羅漢酒醉中驚聞敵軍已渡河,於是棄杖投降,其餘部隊亦都隨之潰散。當晚,劉劭關閉六門拒守,在門內鑿出護城河及柵欄。不過城內混亂,已無秩序,尹弘及孟宗嗣等人出降,蕭斌知大航失守後亦命所屬軍隊解甲投降但被殺。翌日(5月27日),劉義恭登朱雀門,總領諸軍進攻宣陽門,先前劉劭召還的陳叔兒部於建陽門遠遠望見討伐軍就棄杖逃走;原本屯駐閶闔門的劉劭部隊亦逃還殿內,程天祚及譚金等人因而攻入殿內,其餘眾軍繼進,臧質亦從廣莫門進入,會師太極殿前。劉劭穿過西垣入武庫井內,但為高禽所捕。當時劉劭問高禽:「天子何在?」高禽答:「至尊近在新亭。」高禽將劉劭帶到殿前,臧質問其為何行逆,劉劭答:「先朝當見枉廢,不能作獄中囚,問計於蕭斌,斌見勸如此。」將罪責推及蕭斌,又問臧質可否代為請求劉駿流放他到遠地。臧質遂將劉劭縛在馬上,將要衞送到劉駿軍門,但到牙旗下時劉義恭率眾觀望,並詰問劉劭何以殺其十二子,劉劭亦答道這是有負於義恭。江湛妻庾氏及龐秀之亦罵劉劭,但劉劭卻大聲回罵:「汝輩復何煩爾!」劉劭四子皆被殺,劉劭對劉鑠說:「此何有哉。」接著劉劭亦於牙旗下被殺,死前嘆道:「不圖宗室一至於此。」

劉駿在較早前已即位為帝,劉劭死後亂事被平定,劉劭妻殷氏與劉劭、劉濬諸子都被賜死,其他幫助劉劭的大臣如殷沖、尹弘、王羅漢及張超之等都被殺或賜死,嚴道育及王鸚鵡都在街上被鞭殺,焚屍揚灰江上;劉劭及劉濬屍體都被棄到長江中,枭首大航,劉劭東宮住所亦被毀。

據說劉劭初生時,尚為宜都王妃的袁皇后對兒子詳細端視後就命人向劉義隆表示:「此兒形貌異常,必破國亡家,不可舉。」並要下手殺掉他。劉義隆狼狽地趕去阻止才讓劉劭得以長大。

文帝死前一天夜晚,太史曾上奏預測東方有兵突襲,建議在太極前殿列兵萬人作銷災。文帝不許,最終讓劉劭成功篡位,聞言就嘆道:「幾誤我事。」又問太史令他還有多少年壽命,太史當時回答十年,但退下後就對人稱只有十旬日,劉劭知道後大怒,將太史殺掉,最終劉劭果十旬而亡。

刘劭的年号是太初(453年二月—453年五月),共计三個月。

\subsubsection{孝武帝生平}

宋孝武帝劉駿(430年9月19日-464年7月12日),字休龍,小字道民,宋文帝劉義隆的第三子,南朝宋第五任皇帝。453年3月16日深夜,皇太子劉劭於京城建康(今南京市)行凶,殺害父皇宋文帝劉義隆,自稱皇帝;時為武陵王的劉駿在沈慶之的輔佐下,於江州(今九江市)起兵宣討。同年5月20日,於新亭(今南京市西南)即皇帝位。5月27日攻下京城,擒斬長兄劉劭、二兄劉濬。隔年(454年)2月14日改元,年號孝建;457年2月10日二度改元,年號大明。

劉駿在位期間,加強中央集權,撤除「錄尚書事」職銜,並分割州、郡以削弱藩鎮實力;誅中書令王僧達、丹陽令顏竣,討誅隨王劉誕,剷除強臣。崇禮佛教,尊奉高僧僧導,率公卿親臨瓦官寺聽宣《維摩詰經》;詔令整肅佛門,勒令不法僧人還俗;史載劉駿天性好色,臨幸不避戚誼,並有與母后路惠男亂倫之嫌疑,流傳後世。

464年7月12日,劉駿病逝於建康宮玉燭殿,享年三十五歲,在位十一年。8月27日,奉葬景寧陵。

史載劉駿其人機警聰慧,博學多聞並文采華美,讀書能七行俱下,又雄豪尚武,擅長騎射。劉駿病逝後,吏部尚書蔡興宗稱其為「守道之君」(「以道始終」);然而劉駿生性喜奢、欲求無度,晚年「尤貪財利」、不聽善諫,以致原本讚許他德行的士族,也感嘆「天下失望」;更兼大明末年,浙江大旱,通貨膨脹失控、浙江的人民餓死十分之六、七,依《宋書‧州郡志》記載之戶口推算,飢餓致死者最高可能有三十萬人。南朝梁史家裴子野總結劉駿「威可以整法,智足以勝奸,人君之略,幾將備矣。」卻也嘆道:「夫以世祖才明,少以禮度自肅,思武皇之節儉,追太祖之寬恕,則漢之文景,曾何足云!」

劉駿生於南朝宋文帝元嘉年間(430年9月19日),為宋文帝第三子。435年,年僅六歲便受封武陵王,食邑二千戶;439年,時年十歲,受詔都督湘州諸軍事、征虜將軍、湘州刺史,領石頭戍事;440年,遷使持節、都督南豫、豫、司、雍、并五州諸軍事、南豫州刺史,仍任征虜將軍,戍守石頭城;444年,加都督秦州,進號撫軍將軍;隔年(445年),時年十六歲,受詔改任都督雍、梁、南北秦四州,荊州之襄陽、竟陵、南陽、順陽、新野、隨六郡諸軍事、甯蠻校尉、雍州刺史,持節,仍任撫軍將軍。自東晉偏安江東後,劉駿為南朝第一位出鎮襄陽的皇室子弟。449年,受詔改任都督南兗、徐、兗、青、冀、幽六州、豫州之梁郡諸軍事、安北將軍、徐州刺史,持節如故,北鎮彭城。不久宋文帝又下詔加任劉駿為兗州刺史,次子始興王劉濬為南兗州刺史,因此劉駿都督南兗州的職銜當即撤銷。

450年,北魏太武帝拓跋燾率兵南侵,宋文帝詔令劉駿領兵北襲屯駐於汝陽的北魏永昌王拓跋仁。劉駿領一千五百兵馬進襲汝陽,魏兵因無防備而潰敗。但之後探得宋軍並無援軍,因而反戈一擊,宋軍大敗,士兵僅有九百人生還。5月19日,劉駿因汝陽戰敗,降號為鎮軍將軍。451年3月19日,魏軍解圍盱眙北還。4月13日,因防禦北魏入侵無功,宋文帝再下詔降劉駿為北中郎將。

452年,劉駿時年二十三歲,加封都督南兗州軍事,擔任南兗州刺史,鎮守山陽,不久改任都督江州、荊州之江夏、豫州之西陽、晉熙、新蔡四郡諸軍事、南中郎將、江州刺史,持節如故。當時江寇橫行,宋文帝派遣步兵校尉沈慶之討賊,由劉駿全權統領征討大軍。劉駿的親信顏竣,曾於彭城假托沙門僧語,散佈劉駿當為「真人」的符讖謠言,並傳至京師。宋文帝欲行加罪,卻因爆發太子劉劭詛咒皇帝的巫蠱事件,故對劉駿和顏竣暫時不予治罪。

453年3月16日深夜,劉駿長兄、皇太子劉劭趁夜帶兵入宮弒君,宋文帝遇害。劉劭稱帝,進號劉駿為征南將軍、加任散騎常侍,以示攏絡,卻矚使步兵校尉沈慶之殺害劉駿。沈慶之受命後求見劉駿,劉駿稱病不敢接見。沈慶之便闖至劉駿面前,將劉劭的手書呈遞。劉駿涕泣請求沈慶之讓自己與母親路淑媛訣別。沈慶之說:「下官受先帝厚恩,常願報德,今日之事,唯力是視,殿下是何疑之深!」劉駿聽此言,便起座再拜說:「家國安危,在於將軍。」遂由沈慶之處分內外。453年4月11日,劉駿戒嚴示眾,起兵討逆。荊州刺史南譙王劉義宣、雍州刺史臧質響應義舉。5月1日,劉駿移檄建康(今南京市);14日,冠軍將軍柳元景與劉劭大戰於新亭,劉劭敗逃;三天後,劉駿兵進江寧;18日,江夏王劉義恭來降,奉表上尊號;隔日,劉駿進駐新亭,使散騎侍郎徐爰草制即位禮儀。

453年5月20日,武陵王劉駿於新亭即皇帝位,大赦天下,時年二十四歲;27日,攻陷建康城,斬偽皇帝劉劭及二兄劉濬。

454年3月17日,南郡王劉義宣、江州刺史臧質、豫州刺史魯爽、兗州刺史徐遺寶舉兵造反。因新皇即位日淺,朝廷得報大懼。劉駿甚至想奉呈乘輿法物迎劉義宣即位,竟陵王劉誕當即阻止,說:「奈何持此座與人?」劉駿乃止。4月19日,安北司馬夏侯祖歡擊破徐遺寶;6月1日,鎮軍將軍沈慶之於曆陽之小峴大破魯爽,將其斬決;29日,劉義宣及臧質率軍攻梁山營壘,豫州刺史王玄謨派遣遊擊將軍垣護之、竟陵太守薛安都出壘迎戰,擊敗臧質。垣護之因風縱火,劉義宣及臧質大敗而逃;7月13日,臧質遭斬;8月4日,賜死劉義宣於江陵獄中。

455年8月29日,因武昌王劉渾自號楚王、擅訂年號(永光),潛越禮制,下詔將其廢為庶人,賜死。

459年,劉駿暗示有司核奏竟陵王劉誕不法,貶爵為侯,並任命垣閬為兗州刺史,以赴鎮所為名,趁機襲擊劉誕。事泄失敗,垣閬被殺。6月4日,劉誕聚眾造反,佔據廣陵城,劉駿派遣車騎大將軍沈慶之率兵平叛;9月22日,攻下廣陵,將劉誕斬首,殺光城內的三千男丁,女子賞賜給兵士。

劉駿是一個頗有作為、積極改革制度的皇帝。他加強中央集權,撤除「錄尚書事」職銜,並分割州、郡以削弱藩鎮實力。454年7月28日,因揚、荊二州地大兵多,刺史易生異志,劉駿下詔分割揚州、浙東五郡為「東揚州」,並由荊、湘、江、豫四州分割出八郡,劃歸「郢州」,荊、揚二州自此削弱;撤除「南蠻校尉」一職,戍兵移鎮建康,增強京師武備。同年(454年),劉駿因劉義宣叛亂,有意削弱諸王侯權勢,江夏王劉義恭於是奏請裁損諸王侯車服器用、樂舞制度九條,劉駿准奏後,更另有司增訂至二十四條,全面抑制藩王地位,威福獨專。宗王兄弟中只有七弟劉宏被親愛重用,455年成為宰相(458年卒)。孝武帝同時重用江東寒門沈慶之與傖荒北人柳元景,依照兩人的功績,先後提拔為三公,開啟吳興沈氏與河東柳氏攀升為南朝高門的起始之路,並開創南朝寒門、寒人以軍功升為三公的先例。

458年(大明二年),在外放顏竣並處死王僧達後,劉駿欲大權獨攬、專擅朝綱,因此除了高門蔡興宗與袁顗以外,從此不再放權給宗王兄弟與高門強族的大臣,專委任倖臣充作耳目,隱刺朝政,形成後代所謂「寒人掌機要」的政治局面,孝武帝的集權統治也被史書稱為「主威獨運,官置百司,權不外假」。倖臣當中,戴法興、巢尚之、戴明寶、徐爰四人,最有理政才幹,因此大受寵幸,事必與議。巢尚之及徐爰尤知謹慎,惟戴法興及戴明寶卻因此作威作福、納賄受貨,門庭若市,身價並達千金。戴明寶尤其驕縱,放任長子戴敬出錢搶買皇帝的御用物,甚至於劉駿出巡時,騎馬於御輦旁來回奔馳,毫無顧忌。劉駿大怒,下令處死戴敬並將戴明寶下獄,不久仍釋放,委以重任如初。而戴法興於劉子業任皇太子時即奉命侍從,後更受劉駿遺命託孤,輔佐劉子業繼位(宋前廢帝),以致宋前廢帝時有民間謠言:「戴法興為真天子,皇帝為假天子。」之語,權重若此。

劉駿生性嚴峻寡恩,對待左右侍臣,動輒屠戮;甚且自詡風流,晚年專喜戲謔大臣,各取綽號,無禮之至,惟吏部尚書蔡興宗方直嚴肅,劉駿憚怕之,不敢侵狎;平時飲食起居極盡奢華,宮殿牆柱及地板皆鋪錦繡,又嫌宮廷狹小,特命建「玉燭殿」以供享樂,並破壞其祖父、宋武帝生前所居密室,做為地基,並率大臣圍觀動工。見床頭用土作鄣,牆上掛葛燈籠、麻繩拂,侍中袁顗便稱讚宋武帝有節儉樸素之德,劉駿自以為名士派頭,瞧不起沒文化的祖父劉裕,批評說:「田舍公得此,以為過矣!」(「鄉下人能用這些東西,已經太過了!」)

劉駿生性好賭,揮霍不少,加上國家戰亂之後,中央府庫空虛、無錢可使,便效法桓玄手段,以賭博斂財。詔命凡各州刺史及二千石官員,卸職還都時須獻奉財物,限期繳納。其後更召入宮中賭博作樂,賺盡地方官於其任上所積錢財,方准離去。這種收稅辦法被後任的宋、齊皇帝沿用並發揚光大,直接強逼刺史「獻奉」,省略掉賭博這種相對體面的手法。;劉駿晚年喜好飲酒,常飲至深夜,隔日起床洗漱完畢後,便繼續喝至大醉,整日嗜睡。然而有奏疏馳至,便立刻整理好儀容,毫無醉態。宮中內外都佩服他的機神明肅,不敢偷懶懈怠。

大明七年(463年)底至八年(464年),浙江等地因為劇烈旱災,造成嚴重的大饑荒,浙江十分之六的戶口餓死逃散。宋朝史家司馬光因此批評劉駿,說他晚年好酒奢靡,以致原本強盛的劉宋,在他執政末年中衰。

454年,劉駿召幸南郡王劉義宣(六叔)的幾個女兒,劉義宣於是憎恨劉駿,隨後在江州刺史臧質的慫恿下,起兵造反。造反失敗,劉義宣遭誅。劉駿可能便秘密納娶其中一位堂妹(劉駿為避人耳目,冊封其為殷淑儀),並與其生下第八子劉子鸞等五子一女,但也有說法認為殷淑儀並非劉氏女。

史載劉駿與母親路太后有亂倫之嫌疑。南朝人沈約所著《宋書》之記載較為含蓄,內文如下:「上於閨房之內,禮敬甚寡,有所御幸,或留止太后房內,故民間喧然,咸有醜聲。宮掖事秘,莫能辨也。」——《宋書‧列傳第一‧后妃傳》

《宋書》指劉駿常於路太后所居顯陽殿中臨幸宮女,因停留時間過久,以致民間謠傳其間有不可告人之事。《宋書》作者沈約並無否認,只模稜兩可地表示:「宮掖事秘,莫能辨也。」

然而由北朝人魏收所著的《魏書》就沒有顧忌,直接指涉劉駿與其母亂倫:「駿淫亂無度,蒸其母路氏,穢汙之聲,布於甌越。」——《魏書‧列傳第八十五‧島夷劉裕傳》

魏收還記述劉駿天性好色、狎褻無度,以致其兒子、宋前廢帝劉子業即位後,指著劉駿的畫像罵:「此渠大好色,不擇尊卑!」

但也有人認為記載不實。唐朝史家劉知幾在其著作《史通》中辯誣說:「沈氏著書,好誣先代,於晉則故造奇說,在宋則多出謗言,前史所載,已譏其謬矣。而魏收黨附北朝,尤苦南國,承其詭妄,重以加諸。遂云馬睿出於牛金,劉駿上淫路氏。可謂助桀為虐,幸人之災。」

464年7月12日,劉駿病逝於玉燭殿,享年三十五歲,在位十一年。皇太子劉子業繼位,是為宋前廢帝。8月27日,奉葬位於丹陽郡秣陵縣岩山(今南京市江寧區秣陵鎮)的景寧陵,予諡「孝武皇帝」,廟號「世祖」。


\subsubsection{孝建}

\begin{longtable}{|>{\centering\scriptsize}m{2em}|>{\centering\scriptsize}m{1.3em}|>{\centering}m{8.8em}|}
  % \caption{秦王政}\
  \toprule
  \SimHei \normalsize 年数 & \SimHei \scriptsize 公元 & \SimHei 大事件 \tabularnewline
  % \midrule
  \endfirsthead
  \toprule
  \SimHei \normalsize 年数 & \SimHei \scriptsize 公元 & \SimHei 大事件 \tabularnewline
  \midrule
  \endhead
  \midrule
  元年 & 454 & \tabularnewline\hline
  二年 & 455 & \tabularnewline\hline
  三年 & 456 & \tabularnewline
  \bottomrule
\end{longtable}

\subsubsection{大明}

\begin{longtable}{|>{\centering\scriptsize}m{2em}|>{\centering\scriptsize}m{1.3em}|>{\centering}m{8.8em}|}
  % \caption{秦王政}\
  \toprule
  \SimHei \normalsize 年数 & \SimHei \scriptsize 公元 & \SimHei 大事件 \tabularnewline
  % \midrule
  \endfirsthead
  \toprule
  \SimHei \normalsize 年数 & \SimHei \scriptsize 公元 & \SimHei 大事件 \tabularnewline
  \midrule
  \endhead
  \midrule
  元年 & 457 & \tabularnewline\hline
  二年 & 458 & \tabularnewline\hline
  三年 & 459 & \tabularnewline\hline
  四年 & 460 & \tabularnewline\hline
  五年 & 461 & \tabularnewline\hline
  六年 & 462 & \tabularnewline\hline
  七年 & 463 & \tabularnewline\hline
  八年 & 464 & \tabularnewline
  \bottomrule
\end{longtable}


%%% Local Variables:
%%% mode: latex
%%% TeX-engine: xetex
%%% TeX-master: "../../Main"
%%% End:
