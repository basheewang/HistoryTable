%% -*- coding: utf-8 -*-
%% Time-stamp: <Chen Wang: 2019-12-20 13:44:50>

\subsection{明帝\tiny(465-472)}

\subsubsection{生平}

宋明帝劉彧(439年12月9日-472年5月10日),字休炳,小字榮期,南朝宋第七任皇帝。劉彧生於元嘉年間,為宋文帝劉義隆第十一子,先後受封淮陽王、湘東王。宋前廢帝劉子業即位,顧慮諸叔威脅皇位,趁劉彧入朝時將其拘留殿中,並因劉彧體胖而封其為「豬王」,大肆羞辱,且屢次欲加殺害,都因始安王劉休仁諂媚化解,才保全性命。劉子業遭壽寂之殺害後,劉休仁便奉迎劉彧即皇帝位,改元泰始,大赦天下。

劉彧在位六年半,執政前期眾親王及方鎮相繼叛變,朝廷頻繁動武平亂,國力逐漸耗損。北魏也趁機侵略,佔領山東、淮北等地區,北朝國力自此超越南朝;劉彧為防範宋孝武帝劉駿諸子奪取皇位,殺盡諸姪子,致使劉駿絕後;晚年尤多忌諱,文書奏折不得出現諱字,犯禁者一律誅殺。

472年5月10日,劉彧逝世,享年三十四歲,庙号太宗,谥号明帝,奉葬高寧陵。

史載劉彧個性寬和仁慈,儀容端雅,喜好文學。即位後雖然四方反抗但用人不疑,能使將士效忠不貳。然而晚年好猜忌,對待皇族及侍臣動輒殘忍刑戮;國家連年征伐,國庫空虛,而劉彧卻奢侈無度,致使「天下騷然,民不堪命」,劉宋國運自此衰敗。

劉彧生於南朝宋文帝元嘉十六年十月戊寅(439年12月9日),九歲時受封淮陽王,食邑二千戶。452年,改封湘東王。劉彧三哥、宋孝武帝劉駿即位後,歷任郡太守、中護軍、侍中、衛尉、游擊將軍、左衞將軍、都官尚書、領軍將軍等職銜,並獲賜鼓吹一部。453年,劉彧生母沈容姬逝世,劉彧時年十四歲,由路太后撫養於宮中,特受寵愛,時常侍奉路太后醫藥,也因此為劉駿所親愛,不招致猜忌。宋前廢帝劉子業繼位後,任命劉彧為州刺史,都督州郡軍事,並得以本號開府儀同三司。

宋前廢帝劉子業即位後荒淫無道,殺戮群臣,並恐諸叔覬覦皇位,欲加殺盡。劉彧於景和年間入朝,遭拘留宮中,百般毆打凌辱。劉彧與始安王劉休仁、山陽王劉休祐皆體型肥胖,被劉子業封為「豬王」、「殺王」、「賊王」,並將三人用竹籠囚禁。劉子業又命人掘地為坑,灌滿泥水,再以木槽盛飯,並用雜食攪和後置於坑前,命劉彧裸體於泥坑中以口對木槽中就食,戲謔為豬。劉彧曾因抗拒羞辱而惹怒劉子業,劉子業命將其裸體後用竹杖綁住四肢抬付太官,說:「即日屠豬。」劉休仁在旁笑說:「豬今日未應死。」劉子業問何故,劉休仁答說:「待皇太子生,殺豬取其肝肺。」劉子業聽後怒火漸息,命交付廷尉,劉彧才逃過死劫。466年,劉子業欲南遊荊州及湘州,決定明日殺害劉彧等諸叔父後,即行出發。劉彧遂與心腹阮佃夫、李道兒等共謀弒君。1月1日夜,阮佃夫與李道兒暗中結交宮中侍臣壽寂之,於華林園將劉子業殺害。劉子業死後,劉休仁隨即奉迎劉彧入宮即位,並令人備皇帝羽儀。由於事起倉促,劉彧半途失落鞋子,跛著走入西堂,仍戴著象徵臣子的烏紗帽,劉休仁讓人給其戴上白紗帽後,便擁劉彧登上御座召見文武百官,接受歡呼禮拜,凡事以「令書」頒布施行。隔天(1月2日),劉彧下令殺掉劉子業的同母次弟劉子尚,以絕後患。

泰始元年十二月丙寅(466年1月9日),劉彧於宮中太極前殿登基為帝,大赦天下。

465年底,宋孝武帝劉駿第三子、晉安王劉子勛為反抗劉子業謀害己命,在鄧琬等人輔佐下,於江州起兵叛亂。劉彧弒君自立後,授姪子劉子勛官爵遭拒。劉子勛甚至在鄧琬的主導下傳檄天下,改討劉彧。466年2月7日,鄧琬、袁顗等奉年僅十歲的劉子勛於尋陽城登極稱帝,年號「義嘉」,另立政府。江州的義嘉政權得到幾乎全國的承認與響應,南朝各州郡皆向劉子勛上表稱臣,改用義嘉年號,並向尋陽奉貢。劉彧所統治區域僅限京師建康(今江蘇省南京市)附近的丹陽、淮南等郡百餘里土地而已,形勢極為嚴峻。

劉彧聞變後隨即任命劉休仁為征討大都督,統帥全軍,王玄謨為副手。任用吳喜、江方興等為東路軍將領,討平會稽、吳、吳興、晉州等東南各州郡,俘虜劉駿第六子、尋陽王劉子房;任用劉勔、張永、蕭道成等為北路軍將領,擊敗殷琰、薛安都等敵對將領,抵擋住北方的攻勢。任命沈攸之、張興世等為西路軍將領討伐劉子勛的尋陽政權,擊敗袁顗、劉胡等人,攻入尋陽,捕斬敵對天子劉子勛,義嘉政權滅亡。隨後宋軍陸續平定江南及淮南各地,「義嘉之難」平息。

由於劉彧登基時,其諸弟(宋文帝劉義隆子嗣)皆在京師,多擁戴兄長劉彧為帝;而劉子勛起兵地方,方鎮都督則多為劉子勛的兄弟(宋孝武帝劉駿子嗣),皆起兵支持劉子勛的義嘉政權。南朝宋即形成文帝系諸王與孝武帝系諸王的內戰局面。劉彧鑑於此,於戰事平定後接受劉休仁的建議,將劉駿在世諸子皆诛殺殆盡,劉駿二十八子自此滅絕。

劉彧於平定叛亂後欲逞兵威,命張永及沈攸之率領重兵,往迎已於義嘉之難後投降的徐州刺史薛安都。薛安都恐劉彧趁機圖己,便向北魏輸誠,乞師自救,汝南太守常珍奇也舉懸瓠城降魏。467年初,北魏派遣尉元、孔伯恭領兵救徐州,另派拓跋石、張窮奇領兵救懸瓠,兗州刺史畢眾敬望風迎降。魏將尉元隨後於呂梁一帶大敗宋將張永及沈攸之,宋軍幾乎全軍覆沒,張永、沈攸之隻身逃回江南,徐、兗二州淪陷;467年2月,青州刺史沈文秀、冀州刺史崔道固投降北魏,旋即又於4月歸降劉宋。北魏遂派遣長孫陵、慕容白曜往攻青州,劉彧命沈攸之領兵救援,卻於睢清口遭魏將孔伯恭擊敗,退守淮陰。青州與冀州待援不至,被圍攻數年,先後降魏,青、冀二州也淪陷。

南朝宋本地處江南,國狹民脊,自此再失四州,國力更形衰弱;再加上戰亂不斷,劉彧為獎賞有功將士,大肆封賞封官,造成國庫空虛、士族制度嚴重破壞,削弱南朝宋的執政根基,北朝國力從此超越南朝。

劉彧晚年害怕諸弟在他死後奪取太子劉昱的皇位,於是接受倖臣王道隆與阮佃夫的建議,大殺立過軍功的諸弟,只有劉休範因為人才凡弱而留下未殺。王道隆與阮佃夫掌權後擅用威權、官以賄成,富逾公室。劉彧同時殺害可能會不利於太子的重要大臣,如功臣武將壽寂之、吳喜與高門名士王景文(皇后王貞風之兄、劉彧的大舅子),結果造成劉昱繼位後中央和地方軍鎮互相猜忌、攻伐的政治亂象,使得武將蕭道成因此崛起,最後篡宋建齊。

472年,宋明帝死,太子劉昱繼立,宋明帝遺詔命蔡興宗、袁粲、褚淵、劉勔、沈攸之五人託孤顧命大臣,分別掌控內外重區,另外命令蕭道成為衛尉,參掌機要。其中遺詔雖任命袁粲、褚淵在中央秉政,但實際上接受宋明帝秘密遺命,就近輔佐新帝劉昱,掌控宮中內外大權的人物,是宋明帝最親信的側近權倖——王道隆與阮佃夫二人。

史書大多記載,劉彧晚年失去了生育能力,所以他的兒子們都是借腹生子取來的,他把諸弟新生的男嬰抱為自己的兒子,然後殺掉男嬰的生母。但是史家呂思勉認為這是《宋書》作者沈約,為了迎合當時南齊皇帝所捏造的誣蔑之詞,不足採信,而《南史》與《資治通鑑》則是沿用沈約的說法。呂思勉認為宋明帝生前因為猜忌諸弟而狠心殺弟、流放諸姪,不可能殺其父而養其子、流其兄而立其弟。曾懷疑《宋書》等史料的記載,認為宋明帝的皇后王貞風既然有兩個女兒,說明宋明帝可以生育,因此《宋書》應該是為了強化南齊的合法性,故意加工偽造史料,並被後人延用。

毛泽东在阅读南北朝的史书关于刘彧的传记中,写下了:(登基)“可谓奇矣”。

《宋書》記載劉彧:「少而和令,風姿端雅……好讀書,愛文義……及即大位,四方反叛,以寬仁待物,諸軍帥有父兄子弟同逆者,並授以禁兵,委任不易,故眾為之用,莫不盡力。平定天下,逆黨多被全,其有才能者,並見授用,有如舊臣。才學之士,多蒙引進,參侍文籍,應對左右」、「末年好鬼神,多忌諱,言語文書,有禍敗凶喪及疑似之言應回避者,數百千品,有犯必加罪戮」、「泰始、泰豫之際,更忍虐好殺,左右失旨忤意,往往有斮刳斷截者。時經略淮、泗,軍旅不息,荒弊積久,府藏空竭。內外百官,並日料祿俸;而上奢費過度,務為彫侈。每所造制,必為正御三十副,御次、副又各三十,須一物輒造九十枚,天下騷然,民不堪命……親近讒慝,剪落皇枝,宋氏之業,自此衰矣」

沈約評論劉彧:「太宗因易隙之情,據已行之典,剪落洪枝,願不待慮。既而本根無庇,幼主孤立,神器以勢弱傾移,靈命隨樂推回改。斯蓋履霜有漸,堅冰自至,所從來遠也」

北宋的司馬光評論:「(明)帝猜忍奢侈,宋道益衰」、「夫以孝武之驕淫、明帝之猜忍,得保首領以沒於牖下,幸矣,其何後之有?」

蕭梁的史家裴子野評論:「景和(劉子業)申之以淫虐,太宗易之以昏縱,師旅薦興,邊鄙蹙迫,人懷苟且,朝無紀綱,內寵方議其安,外物已睹其敗矣。」

\subsubsection{泰始}

\begin{longtable}{|>{\centering\scriptsize}m{2em}|>{\centering\scriptsize}m{1.3em}|>{\centering}m{8.8em}|}
  % \caption{秦王政}\
  \toprule
  \SimHei \normalsize 年数 & \SimHei \scriptsize 公元 & \SimHei 大事件 \tabularnewline
  % \midrule
  \endfirsthead
  \toprule
  \SimHei \normalsize 年数 & \SimHei \scriptsize 公元 & \SimHei 大事件 \tabularnewline
  \midrule
  \endhead
  \midrule
  元年 & 465 & \tabularnewline\hline
  二年 & 466 & \tabularnewline\hline
  三年 & 467 & \tabularnewline\hline
  四年 & 468 & \tabularnewline\hline
  五年 & 469 & \tabularnewline\hline
  六年 & 470 & \tabularnewline\hline
  七年 & 471 & \tabularnewline
  \bottomrule
\end{longtable}

\subsubsection{泰豫}

\begin{longtable}{|>{\centering\scriptsize}m{2em}|>{\centering\scriptsize}m{1.3em}|>{\centering}m{8.8em}|}
  % \caption{秦王政}\
  \toprule
  \SimHei \normalsize 年数 & \SimHei \scriptsize 公元 & \SimHei 大事件 \tabularnewline
  % \midrule
  \endfirsthead
  \toprule
  \SimHei \normalsize 年数 & \SimHei \scriptsize 公元 & \SimHei 大事件 \tabularnewline
  \midrule
  \endhead
  \midrule
  元年 & 472 & \tabularnewline
  \bottomrule
\end{longtable}


%%% Local Variables:
%%% mode: latex
%%% TeX-engine: xetex
%%% TeX-master: "../../Main"
%%% End:
