%% -*- coding: utf-8 -*-
%% Time-stamp: <Chen Wang: 2021-11-01 15:04:07>

\subsection{顺帝劉準\tiny(477-479)}

\subsubsection{生平}

宋順帝劉準(467年8月8日-479年6月23日),字仲謨,小字智觀,為劉宋的末代皇帝,為宋明帝劉彧的第三子。但《宋書》卻說宋明帝晚年陽痿,不能人道,所以劉準其實是刘彧之弟桂陽王劉休範与姬妾的儿子,由陈法容所抚养。不過這項記載被史家呂思勉駁斥,認為是《宋書》的作者沈約(南齊的史官)為了討好南齊皇帝,故意編造這樣的史料,好掩飾蕭道成篡位、弒君的罪惡。

泰始五年七月癸丑生,泰始七年封為安成王。477年,後廢帝劉昱被弒之後,劉準在蕭道成的擁立下,于元徽五年七月壬辰即位,是為宋順帝,並封蕭道成為相國、齊王;雖然劉準名義上是皇帝,但是權力都被蕭道成掌握。昇明三年(479年),蕭道成要求劉準禪讓,並且派部將王敬則率兵解送出宮。479年四月,劉準禅位與蕭道成,至此,劉宋滅亡。

蕭道成即位之後,封劉準為汝陰王,遷居丹陽並派兵監管。建元元年五月己未(479年6月23日),監視劉準的兵士聽得門外馬蹄聲雜亂,以為發生了變亂,便殺害劉準,得年12歲(虛龄13歲)。

「願後身世世勿復生天王家。」 被清初三大學者之一黃宗羲的知名政治思想著作《明夷待訪錄》首篇《原君》中引用:「昔人(指南朝宋順帝)願世世無生帝王家,⋯」便是表明若將國家當做產業看待,則全天下眾多覬覦這份產業的人是擋不住的,頂多傳個幾代,殺身之禍報應在他(國君)的子孫上。像是劉準被逼讓位所言、明思宗自殺前對公主所說「妳為何要出生在我們帝王之家?」這些話是多麼悲痛。

\subsubsection{昇明}

\begin{longtable}{|>{\centering\scriptsize}m{2em}|>{\centering\scriptsize}m{1.3em}|>{\centering}m{8.8em}|}
  % \caption{秦王政}\
  \toprule
  \SimHei \normalsize 年数 & \SimHei \scriptsize 公元 & \SimHei 大事件 \tabularnewline
  % \midrule
  \endfirsthead
  \toprule
  \SimHei \normalsize 年数 & \SimHei \scriptsize 公元 & \SimHei 大事件 \tabularnewline
  \midrule
  \endhead
  \midrule
  元年 & 477 & \tabularnewline\hline
  二年 & 478 & \tabularnewline\hline
  三年 & 479 & \tabularnewline
  \bottomrule
\end{longtable}


%%% Local Variables:
%%% mode: latex
%%% TeX-engine: xetex
%%% TeX-master: "../../Main"
%%% End:
