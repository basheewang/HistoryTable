%% -*- coding: utf-8 -*-
%% Time-stamp: <Chen Wang: 2021-11-01 15:03:03>

\subsection{武帝劉裕\tiny(420-422)}

\subsubsection{生平}

宋武帝劉裕(363年4月16日-422年6月26日),字德輿,小字寄奴,彭城綏輿里(今江蘇省徐州市銅山区)人,東晉末年至南北朝初期的軍事家、政治家,南北朝時期劉宋開國皇帝。早年出身十分貧寒,劉裕最初為北府將領孫無終的司馬,在孫恩之亂中展現其軍事才能,及後更發起義軍擊敗篡位的桓玄,恢復了東晉政權,並獲得了極高名望,並在不久之後掌握朝政大權。

劉裕趁南燕内讧之际而出兵滅燕,隨後又平定了盧循之亂,以及消滅了劉毅、諸葛長民及司馬休之等異己,鞏固了在東晉國內的地位。接著又乘後秦内乱而北伐,收復了洛陽及關中地區,受封宋公並得九錫,終篡奪了東晉政權,建立劉宋,正式開始了南北朝時代。

劉裕家族在早年隨晉室南渡,長居京口(今江蘇鎮江市),《宋書》說他是漢高祖劉邦的弟弟楚王劉交第二十一世孫。《魏書》則猜測其祖先可能姓項。劉裕於興寧元年三月壬寅日(363年4月16日)出生,其時家境貧苦,母親更因分娩後疾病去世,父親劉翹無力請乳母給劉裕哺乳,一度打算拋棄他,只因劉懷敬之母伸出援手,養育劉裕,才得以活下來。劉裕早年曾以賣鞋为生,但卻又常賭博樗蒲,傾盡家財,遭鄉里賤視,亦因不修品行而不為當世人們所賞識。不过,刘裕才能出眾,且有大志,當時出身琅琊王氏的王謐就十分敬重他,更曾向他說:「你應當會成為一代英雄。」。

劉裕及後從軍,最初就任北府軍將領、冠軍將軍孫無終的司馬。隆安三年(399年),孫恩起兵反抗晉朝,自海島攻下會稽,三吳各郡皆響應他,孫恩之亂由而爆發。另一北府將領、前將軍劉牢之率軍鎮壓,當時他就請了劉裕參府軍事。

當時劉裕奉命率數十人去偵察敵軍,卻遇上數千人的敵軍並發生戰鬥,雖然所帶的人大多戰死了,但劉裕仍揮動長刀抵抗,殺傷多人。劉牢之子劉敬宣派兵搜尋劉裕,見劉裕獨力抵抗,都讚歎劉裕的能力,並率軍進攻,俘殺一千多人。不久諸軍擊敗孫恩各軍,又攻下會稽郡治所山陰(今浙江紹興市),令孫恩退回海島。

次年(400年)五月,孫恩再襲會稽,殺害駐鎮會稽的謝琰,至十一月時劉牢之率軍前往才擊退孫恩。劉牢之及後命劉裕守句章(今浙江寧波市)。當時句章城小兵弱,而劉裕就常做好作戰準備。翌年(401年)二月孫恩就率眾自浹口(今浙江鎮海)進攻句章,而劉裕就身先士卒,每戰都摧其鋒銳,致令孫恩無法攻下句章,反為劉牢之所敗。三月,孫恩轉戰海鹽(今浙江海鹽縣),劉裕跟隨其進攻方向,於海鹽築城抵抗,又大敗來攻的孫恩。

孫恩後循海路至丹徒(今江蘇鎮江市丹徒區),劉裕率不足千人的部隊趕路,與孫恩同時趕至。當時劉裕軍隊疲累,丹徒守軍亦無鬥志,但面對孫恩來襲,劉裕仍能率眾大敗對方,逼其狼狽登船撤離岸上。孫恩不久轉屯郁洲(今江蘇灌雲縣東北),朝廷以劉裕為建武將軍、下邳太守,討伐孫恩,多次交戰後大破對方,令其勢力轉弱而南撤。劉裕接著追擊,又再敗孫恩,令其再度逃到海島。次年孫恩就被消滅。

元興元年(402年),驃騎大將軍司馬元顯下令討伐荊州刺史桓玄,並以劉牢之為前鋒。桓玄率軍兵臨建康時,劉裕請求進攻,但劉牢之不肯,反而想叛歸桓玄。劉裕當時與何無忌極力諫止但都無效,劉牢之終向桓玄請降,桓玄亦順利消滅司馬元顯的勢力,掌握朝政。

事後桓玄調劉牢之為會稽內史以削其軍權,劉牢之圖據廣陵(今江蘇揚州市)叛桓玄,但劉裕認為人心已去,事必不成而拒絕與劉牢之合作,最終劉牢之因失去僚屬的支持而自殺。桓脩後以劉裕為其中兵參軍,並於同年參與討伐統領孫恩餘黨的盧循、徐道覆。當時桓玄诛殺了多名北府舊將,但劉裕仍領兵討伐盧循部眾,更獲加任彭城內史。及至桓玄篡位後次年(404年),劉裕跟從桓脩入朝建康,桓玄亦十分賞識他,出遊都殷勤接引,賞賜亦甚為豐厚。當時桓玄皇后劉氏就勸桓玄除去劉裕,但桓玄仍圖借助劉裕攻略中原,拒絕加害。

早在劉牢之失敗之時,劉裕就向何無忌說:「桓玄若果守著臣子的忠節,就應與你輔助他;否則,就要與你對付他。」及至劉裕入朝後回到京口,就與何無忌、劉毅、孟昶、諸葛長民、王元德等人合謀舉兵討伐桓玄,並準備在京口、廣陵、歷陽(今安徽和縣)及建康(今江蘇南京市)四地同時起兵。元興三年二月乙卯(404年3月24日),劉裕託詞遊獵而外出募眾,終得百多人。次日(3月25日)早上起兵,何無忌殺桓脩,當時桓脩司馬刁弘率眾前來,劉裕則假稱江州刺史郭昶之已在尋陽(今江西九江市)迎晉安帝復位,桓玄更已被處決,自己只是奉密詔誅除桓氏叛黨。刁弘信以為真,劉裕不久就誅除刁弘,控制了京口。同時孟昶等亦成功控制了廣陵,並率眾至京口與劉裕會合,劉裕獲眾人推舉為盟主,總督徐州事,並於次日(3月26日)起兵進攻建康。

桓玄先派吳甫之及皇甫敷抵抗劉裕,劉裕先於江乘(今江蘇句容北)殺吳甫之,至江乘以南的羅落橋時奮力作戰,又殺皇甫敷,繼續進攻。三月己未日(3月28日),劉裕進攻覆舟山,並命弱兵登山,持著旗幟分道而行,營造四周皆有士兵,數量很多的假象;而又因桓玄守軍大多是北府軍出身,面對劉裕都沒有鬥志,劉裕於是與諸軍進攻,順利以火攻擊潰桓玄守軍,而桓玄亦棄城西逃。

劉裕於三月壬戌日(3月31日)獲王謐等人推舉為使持節、都督揚兗豫青冀幽并八州諸軍事、鎮軍將軍,徐州刺史。不久,劉裕奉武陵王司馬遵承制總百官行事。劉裕在進建康城後派諸將追擊桓玄,終於當年六月誅殺了桓玄,並讓在江陵(今湖北江陵)的晉安帝復位。然而,桓氏勢力仍在荊州盤據,並反攻江陵,直至義熙元年(405年)才再收復江陵,驅逐當地桓氏勢力,並自江陵迎晉安帝回建康,不久劉裕就還鎮丹徒。

義熙二年(406年),劉裕因功受封為豫章郡公。義熙四年正月(407年),因上年年末揚州刺史、錄尚書事王謐去世,劉裕聽從劉穆之的勸言入朝議繼任人選,終獲授侍中、車騎將軍、開府儀同三司、揚州刺史、錄尚書事、徐兗二州刺史,入掌朝政大權。

盧循、徐道覆趁劉裕領兵在外,於義熙六年(410年)起兵,進攻江州。當時朝廷急徵劉裕,而當時劉裕剛滅南燕,收到詔書就撤還建康。劉裕至山陽(今江蘇淮安市)時知江州刺史何無忌已戰死,於是加速回防建康,並於四月趕至。五月,豫州刺史劉毅大敗給盧循,盧循繼續東下,而劉裕當時就招募兵眾,修治石頭城並於當地聚兵。不過,由於劉裕急急南返,士卒多有傷病,而建康兵力亦不過千人,面對有十多萬人的盧循大軍顯得實力懸殊,然而劉裕堅決不肯接受諸葛長民及孟昶奉安帝北歸廣陵避敵的建議,決意死戰。

盧循軍到後停駐蔡洲(今江蘇江寧縣西南江中),劉裕就以木柵阻斷石頭城及淮口,修治越城(今江寧縣南)並建查浦、藥園、廷尉三個堡壘,分兵戍守以禦盧循,盧循曾分疑兵進攻白石及查浦,自率大軍進攻丹陽郡,但都沒有取勝,而且在各縣中都無法搶掠到物資,被逼於七月退兵江州。同年十月,劉裕率劉藩、檀韶、劉敬宣等人進攻盧循,並於十二月以火攻擊敗盧循船隊。盧循敗後試圖於左里(今鄱陽湖口)擋住劉裕,但劉裕率軍奮戰,盧循軍無法阻擋而大敗,盧循因而南逃廣州。劉裕早於盧循撤出蔡洲後就已派了孫處及沈田子經海路攻佔了盧循根據地番禺,盧循敗逃廣州後於義熙七年(411年)又於廣州敗於沈田子等人,終在交州被刺史杜慧度所殺。

劉裕於義熙七年(411年)班師回到建康,並受太尉、中書監職位。次年(412年)四月,劉裕以劉毅為荊州刺史。劉毅自以能力不亞於劉裕,甚不服在劉裕以下,他亦得朝中有名望人士歸心交結,故此遷鎮荊州時就將大部分豫州府屬及江州萬多人的軍隊都帶去荊州,到任後又重新調度荊州郡縣首長,更以患病為由請堂弟劉藩去做他副手。劉裕知其有異心,於是假意答允其請求,但就乘劉藩自兗州治所廣陵入朝時就指稱他與謝混圖謀不軌,將二人賜死,接著就率軍自建康出發討伐劉毅。劉裕派王鎮惡為前鋒,沿路聲言是劉藩前來去騙倒各戍和民眾,直至江陵城外五六里時才被發現,但已趕及在劉毅關閉城門前率兵入城,並在城內作戰。城中民眾知劉裕在率軍前來都十分驚恐,劉毅不敵王鎮惡,唯有出逃,並於牛牧寺自殺。劉裕隨後來到江陵,誅殺了劉毅親信郗僧施,消滅了劉毅勢力。

劉裕征劉毅時以諸葛長民守留府事,但諸葛長民見劉毅敗死,自己亦深感不安,更意圖作亂,劉裕回建康時故意拖慢進度,讓等待迎接他的諸葛長民及其他官員接連幾日都等不到劉裕。劉裕卻乘輕舟快快進城,進入了官邸東府。諸葛長民知道劉裕突然回來了,於是拜訪,劉裕暗中命壯士丁旿埋伏,故意和諸葛長民閒話家常,乘諸葛長民警覺性下降時命丁旿將其殺死,接著又誅殺了長民弟諸葛黎民等人。劉裕接著就加鎮西將軍、豫州刺史,接掌諸葛長民的原職。清除京口武將中的異己勢力之後,劉裕在412年底發動晉滅譙蜀之戰,隔年(413年)西征主將朱齡石成功滅譙蜀,使劉裕加授羽葆、鼓吹及班劍二十人。

412年征討劉毅時,劉裕以晉宗室司馬休之接任荊州刺史。司馬休之頗得當地人心,而劉裕就懷疑他有異心;在義熙十年(414年),其子司馬文思又在建康招集輕俠,令劉裕十分厭惡,司馬文思終因被揭發殺害官吏而被捕,劉裕誅殺其黨眾而免文思死,反送他到司馬休之那裏,要他親自教誨他,實質就是要司馬休之將其處死。然而,司馬休之並沒有殺文思,只是上表廢掉文思的譙王爵位,並寫信向劉裕道歉。這舉動令劉裕對其大感不滿,立刻就命江州刺史孟懷玉戒備。

義熙十一年(415年),劉裕收殺司馬休之在建康的次子司馬文寶及侄兒司馬文祖,並出兵討伐司馬休之,自加黃鉞,領荊州刺史。司馬休之則上表劉裕罪狀,派兵抵抗;當時雍州刺史魯宗之自感不被劉裕所容,故與司馬休之聯結。劉裕前鋒徐逵之初戰敗於魯軌,眾將除蒯恩外皆戰死,劉裕大怒。然而當他到時,魯軌及司馬文思率軍在懸岸峭壁上列陣,令劉裕難以登岸,胡藩當時就冒險攀登,司馬文思等竟不能抵擋,劉裕就乘對方後撤的機會登岸進攻,終擊潰司馬休之的軍隊,攻下江陵,司馬休之及魯宗之北投後秦。

劉裕在消滅司馬休之後獲劍履上殿、入朝不趨、贊拜不名的崇禮,次年(416年)正月更獲加領平北將軍、兗州刺史、都督南秦州諸軍事,至此其一人已經都督徐州、南徐、豫、南豫、兗、南兗、青、冀、幽、并、司、郢、荊、江、湘、雍、梁、益、寧、交、廣、南秦共二十二州。

在魏晋十六国时期,东晋虽偏安江南,却始终没有放弃收复中原等漢地北部地區,所以屡次发动北伐战争。后秦、南燕出于内乱而败亡;公元397年北魏军攻下中山,后燕官吏兵投降两万余人,后燕的疆域被切断为南燕和北燕二部,405年南燕又发生政变;416年姚兴卒,后秦内乱不断,镇守蒲坂和岭北的姚懿、姚恢先后率叛军进攻长安。刘裕趁后秦、南燕内乱之际,乘机出兵,并一举攻灭。这次收复中原的版图之多,是东晋历次北伐中最成功、影响最深远的一次,也是以前的多次北伐都无法与之比拟的。

義熙五年(409年),南燕皇帝慕容超因為缺乏太樂伎人,派兵侵略淮北的宿豫城(今江蘇宿遷縣東南),大掠民眾北歸。及後又派兵進攻淮北,擄去陽平和濟南兩郡太守,俘擄千多家。

劉裕因此上表北伐,並於同年四月出發。當時劉裕認為燕軍短視,不會據守大峴山(今山東臨朐縣東南)天險並堅壁清野,只會進據臨朐(今山東臨朐縣),退守都城廣固(今山東青州市),而當時南燕軍的行動亦果然如此。慕容超知晉軍過了大峴山就親自率軍到臨朐,劉裕前鋒先於巨蔑水擊退燕軍,接著攻臨朐城。晉燕兩軍於臨朐以南作戰,胡藩獻計出奇兵突襲臨朐城內,最終成功攻克,慕容超倉皇自城中逃至城南大軍那裏,而此時劉裕命軍隊進攻,大敗燕軍並斬殺其十多名大將,慕容超於是逃回廣固。劉裕接著乘勝追擊至廣固,並成功攻克其外城。慕容超據守小城抵抗,劉裕就築圍圍困廣固。劉裕圍城戰爭一直維持至次年二月才攻下廣固,並俘殺慕容超,滅了南燕。

劉裕在當日平滅南燕後就已經有攻略後秦的打算,只因盧循作亂才逼令他班師建康,而劉裕在消滅了國內主要異己後,又再重拾昔日計劃。劉裕在獲加督至二十二州後月餘,又獲加中外大都督,解徐兗二州刺史而改領司、豫二州刺史,並奉琅琊王司馬德文北伐,打著晉朝皇室旗號安撫北方漢人。至五月又加北雍州刺史。終在八月,劉裕正式自建康出兵,進軍至彭城(今江蘇徐州市)後又加北徐州刺史。十月,劉裕所派的檀道濟等進攻洛陽(今河南洛陽市),守將姚洸出降,成功收復洛陽。

次年(417年)正月,劉裕自彭城率水軍西進,進入黃河。劉裕一直進軍至潼關,命王鎮惡率軍經渭河進攻後秦都城長安(今陝西西安市),王鎮惡於渭橋大敗姚丕,姚泓所率的軍隊亦因遭姚丕敗兵踐踏而潰亂,最終姚泓於八月出降,後秦滅亡。劉裕於次月到達長安,大賞將士並誅殺歸降的後秦宗室姚璞、姚讚及其百多名宗族。

同年十一月,留守建康的劉穆之去世,當時劉裕還想以長安做基地進攻西北北涼等國,只是諸將都思鄉,大多都不想留下;劉裕向來倚重的劉穆之去世更令他覺得建康根本之地已空虛無靠,於是下了決心班師東歸。劉裕於是留了當時僅得十一歲的次子劉義真鎮守長安,並留下王鎮惡、王脩、沈田子、毛德祖等將領協助他。當地人民知道劉裕要走都向他哭訴,希望他回心轉意,然而劉裕去意已決,還是在當年十二月出發離開。

然而,劉裕走後次年,諸將內訌,沈田子殺王鎮惡,王脩殺沈田子,而劉義真又在諸將唆擺下命人殺害王脩,於是關中大亂,夏國乘機進攻關中,劉裕唯有召還劉義真,派朱齡石等代鎮長安,更指令若關中不能守下去就可放棄。最終晉軍還是撤出長安,關中地區遭夏國佔領。

義熙十四年(418年),劉裕接受相國、總百揆、揚州牧的官職,以十郡建「宋國」,受封為宋公,並接受九錫的特殊禮待。同年,劉裕命令中書侍郎王韶之與晉安帝左右侍從密謀以毒酒毒殺安帝,王韶之於是乘司馬德文因病出宮的機會下手,縊殺安帝。當時劉裕因為相信預言書說:「昌明(晉孝武帝)之後尚有二帝」,於是聲稱依照遺詔,立了司馬德文為皇帝,即晉恭帝。

元熙元年(419年),劉裕進爵為宋王,又加十郡增益宋國,令宋國包括了二十郡。年末劉裕又獲加皇帝規格的的十二旒冕、天子旌旗等一系列特殊禮待。元熙二年(420年),劉裕入輔,傅亮知劉裕想要晉恭帝禪讓帝位予他但難於啓齒,於是代為向恭帝暗示,恭帝於是在六月禪讓帝位給劉裕,東晉滅亡,劉裕即位為帝,改國號為「宋」,改元永初。劉裕在稱帝之後,為了斬草除根,還殺掉了恭帝。此行為可謂劉裕一生中一個汙點,因為其行為開啟了前朝遜位之主不得善終之先(新朝王莽之於西漢孺子嬰、曹魏文帝曹丕之於東漢獻帝劉協、西晉武帝司馬炎之於曹魏元帝曹奐,都沒有加害前朝末主),至此,南朝末主除了陳後主陳叔寶其亡國非遭逢篡位外,全都俱被新立的政權所殺。

永初三年(422年),劉裕患病,五月病重時遺命司空徐羨之、尚書僕射傅亮、領軍將軍謝晦及護軍將軍檀道濟四人為顧命大臣,輔助太子劉義符。劉裕於五月癸亥日(6月26日)去世,享年六十歲。廟號高祖,謚為武皇帝,葬在初寧陵(今江蘇南京紫金山)。

刘裕自他繼王謐以錄尚書事掌權起直至其去世,一直掌握著東晉以及南朝宋的軍政大權,曾对当时积弊已久的政治、经济状况有所整顿。

門閥士族兼併土地的行為令百姓流離失所,無法保護其產業,劉裕則一改東晉以來對這種事寬松的規管,重訂規管並展示公眾,大大抑制了門閥豪強的兼并行為。及至會稽虞氏的虞亮藏匿一千多名脫離戶籍逃亡的人,劉裕將之處死,連時任會稽內史的司馬休之也遭免官。另劉裕又針對當時門閥豪強私佔山澤,人民去砍柴釣魚都受限制的問題,禁止豪強這種行為。刁氏一族向來富有,奴客亦多,在其宗族桓玄敗死後被誅滅時,劉裕亦將刁家的資產都分發給人民,讓人們按己力取用,賑濟當時處於饑荒及戰亂中的人民。劉裕亦於義熙九年(413年)將臨沂、湖熟原屬皇后所有,用來資助其化妝品開銷的田地分配給窮人。如此削奪了世族以及皇室的私產,用來資濟人民。即位為帝後更派大使巡行四方,舉善旌賢,訪問民間疾苦。

劉裕選才用人不重門第而重其才能,故對於寒門出身的劉穆之一直予以重任,在收復建康後讓他主持政局,大改官場之風,及至在劉裕領兵在外時更讓其主掌中樞重任。劉裕在晉時見州郡推薦的很多秀才、孝廉水平都不合要求,於是上請申明舊制,以策試考核他們。至登位後更曾到延賢堂為各秀才、孝廉作策試。而曾與劉裕起兵討伐桓玄的魏順之在盧循之亂時因為不敢救援部將謝寶,反倒退卻;魏順之雖為功臣,亦是魏詠之的弟弟,但劉裕大怒之下仍將其處死,此舉亦震懾其他桓玄之役中的功臣,都聽籨其命令。

劉裕於義熙九年(413年)再度實行土斷,各地人民依界土斷,只有僑居於晉陵的徐、兗、青三州人民不受影響,而當時很多僑郡僑縣都在這次土斷中被裁撤,重新整理了全國戶籍,便利於統計藏匿人口及增加賦稅收入。永初元年(420年),劉裕更下令所有逃避戶籍的人只要在限期內自首就能獲赦,並免去他們兩年的租賦,但凡有黃籍或證明文件的人都可恢復其原籍,再次減少國內藏匿人口。

刘裕消滅劉毅後,下令嚴禁荊、江二州地方官吏滥征租税、徭役,规定租税、徭役,都以现存户口为准。凡是州、郡、县的官吏利用官府之名,占据屯田、园地獲利的,皆一律废除。劉裕即位後,下令凡宫府需要的物资都要到市場採購,照价给钱,不得向人民征调。又下令官員不可徵去人民車牛,亦不能以官威逼迫人民獻出車牛,另亦將繁多的交易稅項作出減省,便利市場商業交易。

刘裕对政治、经济的整顿,进一步打击了門閥士族的势力,改善了政治和社会状况,对劳动人民的痛苦亦有所减轻。

而劉裕在建立南朝宋後亦削弱强藩,集权中央,於是限制了荆州州府置將和官吏數額,前者不可多於二千人,後者亦不可多於一萬人;另其他州府置將及官吏數亦不分別不得多於五百人及五千人。为防止权臣擁兵,他特別下诏命不得再別置軍府,宰相領揚州刺史的話可置一千兵。而凡大臣外任要職要需軍隊防衞,或要出兵討伐,一律配以朝廷军队,事情完結後軍隊都需交回朝廷。另劉裕為防外戚亂政,下令有幼主的話都委事宰相,不需太后臨朝。

劉裕高七尺六寸,氣質奇特。

劉裕行軍法令嚴明,軍隊軍容齊整,絕不擾民。而他在軍事行動的分析亦常常精準無誤,例如伐南燕時料定燕軍不會據守大峴山抵抗,而慕容德果然否決公孫五樓守大峴的計劃。命令朱齡石征伐西蜀時亦預計敵方會猜測晉軍循內水進攻,必以重兵守涪城,於是指令要從外水進攻,改派疑兵引誘涪城重兵,以圖直取成都。最終亦正如劉裕預計那樣,朱齡石成功繞過涪城重兵,直取成都,獲得勝利。

在生活上刘裕崇节俭,不爱珍宝,不喜豪华,宫中嫔妃也少。宁州地方官曾经奉献琥珀枕,是无价之宝,他不稀罕。在出征後秦时,有人说琥珀能够治疗伤口,他就命人将它砸碎,分给将领作为治伤药。平定关中後,他得到了美女姚氏(後秦天王姚興的姪女),十分宠爱。臣下谢晦劝谏他不要因女色而荒废政务,他当晚就将姚氏送出宫去。後來劉裕進封宋公,東西堂將要放置以金塗釘釘製的局腳牀,但劉裕以節為由而改用鐵釘釘製的直腳牀。又一次廣州進貢一匹筒細布,劉裕因其過於精巧瑰麗,製作必定擾民,故此下令彈劾獻布那郡的太守,將布匹送還並下令禁止再製作這種布。劉裕因患有熱病,常常要有冰冷的物件去降溫,於是有人就獻上石床。劉裕躺上冰冷的石床,感到十分舒服,但又感木牀已經很耗人力,大石頭要磨成牀就更甚了,於是下令將石床砸毀。劉裕更加下令將自己昔日的農具收起,留給後人。其子宋文帝一次看見,得知內情後大感慚愧。而其孫宋孝武帝拆毀劉裕生前的臥室而建玉燭殿,發現牀頭上有土帳,牆上掛著葛布製的燈籠及麻製蠅拂,袁顗稱許劉裕有儉素之德,但孝武帝沒有說甚麼,只說:「老農夫有這些東西,已經過於富裕了。」

劉裕不擅文才,故劉毅曾在宴會中特地賦詩:「六國多雄士,正始出風流」特意展示其文學造詣勝過劉裕。劉裕書法亦差,曾被劉穆之規勸,並在其指示下改寫大字。

劉裕不信神祇,登位後更曾下令將民間廟宇拆毀,只有先賢以及以有勳德的人的廟祠才得豁免。劉裕去世前患病,群臣上請劉裕祈求神祇庇佑,但劉裕不接受,只派了謝方明去太廟告知祖先。

昔日劉裕曾欠下刁逵三萬錢,無力償還,被刁逵抓著,王謐則去見刁逵,並替劉裕償還欠款,劉裕才得釋放;而當時劉裕既無名聲亦貧賤,不被其他具名望人士看重,唯有王謐去與他結交。王謐後在桓玄篡位時奉天子玉璽及冊文給桓玄,在桓楚任司徒,更獲封公爵,甚為禮侍。劉裕義軍攻下建康後,王謐仍任司徒,領揚州刺史、錄尚書事,但王謐既因在桓楚任高職,甚得寵待,故很不安心,最終出奔。然而劉裕沒有向王謐問罪,並念及昔日恩情,請武陵王司馬遵追還王謐,並讓其官復原職。而昔日為其債主的刁逵,在桓楚任豫州刺史,並為桓玄收捕起義失敗的諸葛長民。他在桓玄敗後出奔,終被部下抓住,可是刁氏一族接著卻遭誅殺,只有刁聘獲赦,然而不久刁聘亦因謀反而被誅,令刁氏族滅。

傳說劉裕出生時有神光照亮室內,當晚還降甘露。

劉裕曾到京口竹林寺,並獨自躺臥在寺內講堂內。一眾僧人竟看見他上面有五色龍形物體出現,大感吃驚並告知劉裕,劉裕則十分高興起說:「僧人是不會說謊的。」

有言曲阿、丹徒有天子之氣,而劉翹的墓就在丹徒,當時一個叫孔恭的人擅長占卜墓穴吉凶,劉裕一次就在父親墓前問孔恭,孔恭就言那是不平凡的墓地。劉裕聽後更為自負。更劉裕又覺得身邊總有兩條小龍,連旁人也曾看見過,至劉裕名聲漸高時,小龍也變大了。

傳說劉裕一次去伐木砍柴,射傷了一條大蛇。翌日再去時卻聽見有杵臼搗藥的聲音,發現有幾個小童正在製藥。劉裕於是問他們為何要製藥,小童則答:「我們的王被劉寄奴射傷,所以要製藥醫治。」劉裕追問:「你們的王既有神通,為何不殺了他?」小童卻答:「劉寄奴是王者,不可以殺。」劉裕喝跑了小童,拿走他們的藥。後來一次到下邳遊玩,一個僧人向他說:「江南地區會有動亂,令此地安定的人就是你呀。」僧人又給了劉裕一些傷藥,接著就失去了蹤影。劉裕手部有傷患,一直都無法痊癒,但用了僧人的藥一次後卻痊癒了。劉裕於是視剩餘的的傷藥及當日在小童那裏的藥為珍寶,每次受了傷,用那些藥都能醫好。

盧循譏諷劉裕智窮,劉裕則以續命湯反譏盧循命不長。典出藝文類聚·卷八十七:果部下:益智。

劉裕是兩晉南北朝時期最卓越的軍事統帥之一。劉裕在不到二十年時間裡,對內平息戰亂,先後平定孫恩、盧循的叛亂,消滅桓玄、劉毅等軍事集團;對外致力於北伐,取譙蜀、伐南燕、滅後秦,從一名普通軍人成長為名垂青史的軍事統帥,取得世人矚目的成就,更徹底改變晋朝政權對征服漢地北部的塞外各民族一直處於被動的局面。北魏謀臣崔浩在評價劉裕時說:「劉裕奮起寒微,不階尺土,討滅桓玄,興復晉室,北擒慕容超,南梟盧循,所向無前,非其才之過人,安能如是乎!」崔浩亦說:「劉裕之平禍亂,司馬德宗之曹操也。」何去非在《備論》中也說:「宋武帝以英特之姿,攘袂而起,平靈寶于舊楚,定劉毅于荊豫,滅南燕于二齊,克譙縱於庸蜀,殄盧循於交廣,西執姚泓而滅後秦,蓋舉無遺策而天下憚服矣。北方之寇,獨關東之拓跋,隴北之赫連耳。方其入關,魏人雖強,不敢南指西顧以議其後。」《南史》評論說:「宋武地非齊、晉,眾無一旅,曾不浹旬,夷凶翦暴,誅內清外,功格上下。若夫樂推所歸,謳歌所集,校之魏、晉,可謂收其實矣。」

劉裕的軍事生涯,指揮無數次作戰,最大特點是以少勝多,而且作戰中常身先士卒,所以能夠贏得廣大將士尊敬。劉裕北伐是中國戰爭史上最成功的北伐之一,成就不但遠較以前東晉各次北伐高,中國歷史上僅次於朱元璋,所以辛棄疾用「金戈鐵馬,氣吞萬里如虎」的詩句來形容劉裕北伐時的氣勢。司马光叙述刘裕北伐成功后匆忙东归,关中复失时,大发感叹:「惜乎,百年之寇,千里之土,得之艰难,失之造次,使丰、鄗之都复输寇手。荀子曰:『兼并易能也,坚凝之难。』信哉。」而王夫之直指劉裕是為了急急篡位而放棄關中,說:「刘裕灭姚秦,欲留长安经略西北,不果而归,而中原遂终于沦没。史称将佐思归,裕之饰说也。王、沈、毛、傅之獨留,豈繄不有思歸之念乎?西征之士,一歲而已,非久役也。新破人國,子女玉帛足系其心,梟雄者豈必故土之安乎?固知欲留經略者,裕之初志,而造次東歸者,裕之轉念也。夫裕欲归而急于篡,固其情已。」但王夫之仍然肯定了「然使裕據關中,撫雒陽,捍拓拔嗣而營河北,拒屈丐而固秦雍,平沮渠蒙遜而收隴右,勛愈大,威愈張,晉之天下其將安往?曹丕在鄴,而漢獻遙奉以璽綬,奚必反建康以面受之於晉廷乎?蓋裕之北伐,非徒示威以逼主攘夺,而无志于中原者,青泥既败,长安失守,登高北望,慨然流涕,志欲再举,止之者謝晦、鄭鮮之也。蓋當日之貪佐命以弋利祿者,既無遠志,抑無定情,裕欲孤行其志而不得,則急遽以行篡弒,裕之初心亦絀矣。」他还稱刘裕「為功于天下,烈于曹操,而其植人才以贊成其大計,不如操遠矣。操方舉事據兗州,他務未遑,而亟于用人;逮其後而丕與叡猶多得剛直明敏之才,以匡其闕失。」显然也包括了对刘裕北伐成功的肯定。「裕起自寒微,以敢戰立功名,而雄俠自喜,與士大夫之臭味不親,故胡藩言:一談一詠,搢紳之士輻湊歸之、不如劉毅。當時在廷之士,無有為裕心腹者,孤恃一機巧汰縱之劉穆之,而又死矣;傅亮、徐羡之、謝晦,皆輕躁而無定情者也。孤危遠處于外,求以制朝廷而遙授以天下也,既不可得,且有反面相距之憂,此裕所以汔濟濡尾而僅以偏安艸竊終也。當代無才,而裕又無馭才之道也。身殂而弒奪興,況望其能相佐以成底定之功哉?曹操之所以得志于天下,而待其子始篡者,得人故也。豈徒奸雄為然乎?聖人以仁義取天下,亦視其人而已矣。」

呂思勉則認為,劉裕急急篡位的說法只是史家附會王買德的話,說:「宋武代晉,在當日,業已勢如振槁,即無關、洛之績,豈慮無成?苟其急於圖,篡平司馬休之後,逕篡可矣,何必多伐秦一舉?武帝之於異己,雖云肆意翦除,亦特其庸中佼佼者耳,反之子必尚多。劉穆之死,後路無所付託,設有竊發,得不更詒大局之憂?欲攘外者必先安內,則武帝之南歸,亦不得訾其專為私計心也。義真雖云年少,留西之精兵良將,不為不多。王鎮惡之死,在正月十四日(應為十五),而勃勃之圖長安,仍歷三時而後克,可見兵力實非不足。長安之陷,其關鍵,全在王脩之死。義真之信讒,庸非始料所及,此尤不容苟責者也。」

劉裕在對待刁逵及王謐截然不同的態度,招來了不少批評,南朝梁湘東世子蕭方等就曾言:「夫蛟龍潛伏,魚蝦褻之。是以漢高赦雍齒,魏武赦梁鵠,安可以布衣之嫌而成萬乘之隙也!今王謐為公,刁逵亡族,醻恩報怨,何其狹哉。」裴子野亦言:「刁逵,玄之爪牙;王謐,楚之上相,論逆則王重,定罪則逵輕。稚遠以舊德錄萬機,長民以宿憾夷七族,以為晉政偏頗甚矣!且神龍伏於罟網,漁者安知其靈化;霸王匿於人庶,庸夫何以悟其英雄!苟在不悟則驕之者,眾可勝怨乎?是知宋高祖之非弘亮也,同盟多貮宜乎哉!」

劉裕攻下南燕都城廣固後,因為怨恨城池久久不下,故此意圖將城內人民全部坑殺,並將其妻女賞賜給將士,只因韓範勸止才不實行,但仍然盡殺南燕王公共三千人,並抄沒萬餘人。此意圖亦招來司馬光批評:「晉自濟江以來,威靈不競,戎狄橫騖,虎噬中原。劉裕始以王師翦平東夏,不於此際旌禮賢俊,忍撫疲民,宣愷悌之風,滌殘穢之政,使群士嚮風,遺黎企踵,而更恣行屠戮以快忿心;迹其施設,曾苻、姚之不如,宜其不能蕩壹四海,成美大之業,豈非雖有智勇而無仁義使之然哉!」

王夫之在《读通鉴论》评论刘裕:“宋武兴,东灭慕容超,西灭姚泓,拓跋嗣、赫连勃勃敛迹而穴处。自刘渊称乱以来,祖逖、庾翼、桓温、谢安经营百年而无能及此。后乎此者,二萧、陈氏无尺土之展,而浸以削亡。然则永嘉以降,仅延中国生人之气者,唯刘氏耳。舉晉人坐失之中原,責宋以不蕩平,沒其撻伐之功而黜之,亦大不平矣。君天下者,道也,非勢也。如以勢而已矣,則東周之季,荊、吳、徐、越割土稱王,遂將黜周以與之一等;而嬴政統一六寓,賢于五帝、三王也遠矣。拓拔氏安得抗宋而與並肩哉?唐臣隋矣,宋臣周矣,其樂推以為正者,一天下爾。以義則假禪之名,以篡而與劉宋奚擇焉?中原喪于司馬氏之手,且愛其如綫之緒以存之;徒不念中華冠帶之區,而忍割南北為華、夷之界乎?半以委匪類而使為君,顧抑撻伐有功之主以不與唐、宋等倫哉?汉之后,唐之前,唯宋室犹可以为中国主也。”


\subsubsection{永初}

\begin{longtable}{|>{\centering\scriptsize}m{2em}|>{\centering\scriptsize}m{1.3em}|>{\centering}m{8.8em}|}
  % \caption{秦王政}\
  \toprule
  \SimHei \normalsize 年数 & \SimHei \scriptsize 公元 & \SimHei 大事件 \tabularnewline
  % \midrule
  \endfirsthead
  \toprule
  \SimHei \normalsize 年数 & \SimHei \scriptsize 公元 & \SimHei 大事件 \tabularnewline
  \midrule
  \endhead
  \midrule
  元年 & 420 & \tabularnewline\hline
  二年 & 421 & \tabularnewline\hline
  三年 & 422 & \tabularnewline
  \bottomrule
\end{longtable}


%%% Local Variables:
%%% mode: latex
%%% TeX-engine: xetex
%%% TeX-master: "../../Main"
%%% End:
