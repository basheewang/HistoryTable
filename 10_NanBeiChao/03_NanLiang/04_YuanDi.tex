%% -*- coding: utf-8 -*-
%% Time-stamp: <Chen Wang: 2019-12-23 14:07:19>

\subsection{元帝\tiny(552-554)}

\subsubsection{生平}

梁元帝萧绎(508年9月16日-555年1月27日),字世誠,梁武帝蕭衍的第七子,正式谥號為「孝元皇帝」,後世比照西漢的漢元帝和東晉的晋元帝,省「孝」字稱「梁元帝」。

萧绎於514年封湘东王,早年因病而一眼失明。547年出荊州,任荊州刺史、使持節、都督荊雍湘司郢寧梁南北秦九州諸軍事、鎮西將軍。侯景之亂時,梁武帝遣人至荊州宣讀密詔,授蕭繹为侍中、假黄钺、大都督中外诸军事、司徒承制,其余职务如故。萧绎手握强兵,却没有积极勤王。

549年梁武帝餓死台城後,蕭繹首先發兵攻滅自己的侄兒河東王蕭譽與哥哥邵陵王蕭綸,並擊退萧誉弟岳阳王襄陽都督萧詧的來犯,迫使萧詧投靠西魏;之後再命王僧辯率軍東下消滅侯景。其长子萧方等即在与萧誉作战时身亡,次子萧方诸在与侯景交战时被擒杀。萧绎弟益州刺史武陵王蕭紀亦有意出兵共讨侯景,萧绎写信阻止,称“蜀人勇悍,易动难安,弟可镇之,吾自当灭贼。”又写信称“地拟孙、刘,各安境界;情深鲁、卫,书信恒通”以为示好。

552年侯景死後,蕭繹即帝位於江陵,是為梁世祖。当时,群臣中有人建议返回旧都建康,但蕭繹没有同意。并派手下朱买臣在建康杀死侯景所废皇帝萧栋兄弟三人。

萧绎即帝位之前,蕭紀已稱帝於益州;萧纪出兵讨伐侯景,得知侯景已灭,就转为讨伐萧绎。蕭繹便派兵迎战,写信讲和,同时也请求西魏出兵袭取益州。萧纪因此遭受重创,向萧绎求和,萧绎回信拒绝称兄弟情断,最终全歼萧纪势力,但也給了西魏可趁之機,益州因此沦落敌手。萧绎将萧纪父子除宗籍改姓饕餮,并将萧纪二子萧圆照、萧圆正饿死。

554年,蕭繹给西魏权臣宇文泰写信,要求按照旧图重新划定疆界,言辞又极为傲慢。宇文泰大为不满,命令常山公于谨、大將軍楊忠、兄子大将军宇文护等将领以5万兵马进攻江陵(今湖北江陵縣)。梁元帝战败,由御史中丞王孝祀作降文。隨后,便率太子等人到西魏軍營投降。不久為萧詧以土袋悶死,江陵“阖城老幼被虏入关”。

梁元帝也是一個愛好讀書與喜好文學的君主,“四十六岁,自聚书来四十年,得书八万卷”,自稱“韜於文士,愧於武夫。”曾主編《金樓子》等書;江陵被圍城時,承聖三年十二月辛未(555年1月27日),元帝入東閤竹殿,命舍人高善寶放火焚燒圖書14萬卷,包括从建康为避兵灾而转移到江陵的8万卷书,自稱“文武之道,今夜盡矣!”“讀書萬卷,猶有今日,故焚之。”江陵焚書被視為中國的文化浩劫之一。

清朝初年的王船山評論其焚書行徑:「未有不恶其不悔不仁而归咎于读书者,曰书何负于帝哉?此非知读书者之言也。

北宋的司馬光評論:「元帝於兄弟之中,殘忍尤甚,是以雖翦兇渠而克復故業,旋踵之間,身為伏馘;豈特人心之不與哉?亦天地之所誅也。」

唐朝的虞世南:「梁元聪明伎艺,才兼文武,仗顺伐逆,克雪家冤,成功遂事,有足称者。但国难之后,伤夷未复,信强寇之甘言,袭褊心于怀楚,蕃屏宗支自为仇敌,孤远悬僻,莫与同憂,身亡祚灭,生人涂炭,举鄢、郢而弃之,良可惜也。」

陳朝的史家何之元:「世祖聰明特達,才藝兼美,詩筆之麗,罕與為匹,伎能之事,無所不該,極星象之功,窮蓍龜之妙,明筆法于馬室,不愧鄭玄,辨雲物于魯台,無慚梓慎,至於帷籌將略,朝野所推,遂乃撥亂反正,夷凶殄逆,紐地維之已絕,扶天柱之將傾,黔首蒙拯溺之恩,蒼生荷仁壽之惠,微管之力,民其戎乎?鯨鯢既誅,天下且定,早應移鑾西楚,旋駕東都,祀宗土方,清蹕宮闕,西周岳陽之敗績,信口宇文之和通,以萬乘之尊,居二境之上,夷虜乘釁,再覆皇基,率土分崩,莫知攸暨,謀之不善,乃至於斯。」

\subsubsection{承圣}

\begin{longtable}{|>{\centering\scriptsize}m{2em}|>{\centering\scriptsize}m{1.3em}|>{\centering}m{8.8em}|}
  % \caption{秦王政}\
  \toprule
  \SimHei \normalsize 年数 & \SimHei \scriptsize 公元 & \SimHei 大事件 \tabularnewline
  % \midrule
  \endfirsthead
  \toprule
  \SimHei \normalsize 年数 & \SimHei \scriptsize 公元 & \SimHei 大事件 \tabularnewline
  \midrule
  \endhead
  \midrule
  元年 & 552 & \tabularnewline\hline
  二年 & 553 & \tabularnewline\hline
  三年 & 554 & \tabularnewline\hline
  四年 & 555 & \tabularnewline
  \bottomrule
\end{longtable}


%%% Local Variables:
%%% mode: latex
%%% TeX-engine: xetex
%%% TeX-master: "../../Main"
%%% End:
