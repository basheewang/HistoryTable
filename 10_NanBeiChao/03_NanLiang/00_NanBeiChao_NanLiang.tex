%% -*- coding: utf-8 -*-
%% Time-stamp: <Chen Wang: 2019-12-23 14:18:01>


\section{南梁\tiny(502-557)}

\subsection{简介}

梁(502年-557年),又稱南梁,是中国历史上南北朝时期南朝的第三个朝代,由南齐宗室蕭衍称帝,改國號為梁,都建康(今江苏南京)。以蕭衍封地在古梁郡,故国号为梁。因为皇帝姓萧,又称萧梁。

南齐末年,皇帝萧宝卷行事荒淫,并大肆诛杀大臣,杀死尚书令萧懿并追杀其兄弟,萧懿的弟弟雍州刺史萧衍因此联合行荆州府事萧颖胄起兵,拥立荆州刺史南康王萧宝融为帝。萧衍攻入京城,萧宝卷被杀。因萧颖胄已病逝,萧衍成为萧宝融势力唯一领导人物,以太后王宝明名义加封自己为建安郡公、梁公、梁王及杀死萧宝卷的儿子、兄弟、堂兄弟,仅萧宝卷庶兄萧宝义因残疾被留作二王三恪、胞弟萧宝夤北逃而幸免(庶弟萧宝源虽未被杀,但也很快病死)。他虽迎萧宝融进京,但在萧宝融进京前即迫其禅位。

萧衍本是南齐宗室,但与皇室关系疏远,为了给自己制造登基的合法性,他不继承齐朝皇统而是以封号自建国号梁,声称自己推翻萧宝卷之举是为齐高帝、齐武帝子孙报仇,另立政权也非夺取齐高帝、齐武帝的天下。

梁武帝蕭衍於代齐即位後厲行儉約,令南梁前期國勢頗盛。然而,武帝迷信佛教,曾三次出家為僧,令朝臣須用大量金錢為他贖身。他又大建佛寺及翻譯佛經,令佛教大盛,可是佛事太過損害经济,令梁朝國勢開始衰弱。

其後东魏叛將侯景投降,武帝本欲借侯景之力北伐,侯景見南梁國勢衰弱,加上武帝出賣自己,遂有反叛之意,終於548年爆發侯景之亂。皇侄临贺王萧正德曾被过继给武帝,却未能被立为皇太子且回归本宗,心怀不满,与侯景勾结,侯景许诺拥立其为帝。侯景围攻建康,包括皇子宗室们所统领的各地兵马多观望不救,萧正德奉命抵抗时率军倒戈。侯景攻克建康外城后,立萧正德为帝。549年侯景攻克建康城,以武帝名义解散勤王军队,废杀萧正德,武帝亦被其囚禁餓死,這場亂事亦是梁朝滅亡的關鍵。

武帝死后,侯景立皇太子萧纲为傀儡简文帝,把持朝政。同时,不服从侯景的南梁地方势力彼此也互相攻伐及求援于北齐、西魏。武帝第七子湘东王萧绎攻杀萧统子河东王萧誉,迫使其弟岳阳王萧詧以襄阳降西魏,受封梁王;武帝第六子邵陵王萧纶降北齐,亦受封梁王,但因萧绎亦与北齐结盟而失去北齐支持,遭萧绎、侯景打击,最终被西魏所杀;武帝幼子武陵王萧纪据益州称帝。其他地方势力亦有被侯景所灭者。北齊和西魏相繼乘机夺取淮南和中國西南大片土地,梁朝国力急剧衰败,只能偏安長江以南。双方互有胜负,但总體来说在军事上北朝转強,南朝逐渐转弱。

551年,侯景迫简文帝禅位给武帝故太子萧统孙豫章王萧栋,又杀简文帝,同年又迫使萧栋禅位,改国号为汉。552年,萧绎灭侯景,在江陵称帝,史称梁元帝;指示收复建康的手下杀死萧栋兄弟,没有还都建康。年底,他歼灭萧纪势力,但期间他联合西魏致使益州被西魏所得。

553年,北齐出兵意图拥立武帝侄湘阴侯萧退为帝,未果。

因梁元帝与西魏交恶,555年,西魏攻克江陵,迫使梁元帝父子投降,然后杀之,在江陵立萧詧为帝;元帝诸子仅晋安王萧方智幸存,大将王僧辩等在建康拥立他为梁王,以太宰承制,准备拥立为帝,却因被北齐所败,被迫同意北齐所请,改立萧懿子萧渊明为帝,萧渊明亦应王僧辩所请,立萧方智为皇太子。另一大将陈霸先随即以王僧辩投降北齐、抛弃先帝之子为由袭杀王僧辩,迫使萧渊明禅位给萧方智,萧方智史称梁敬帝。陈霸先代表敬帝对北齐称臣,后又击败北齐,亦掌握了朝中大权。

梁敬帝时,將皇位禪讓給陈霸先,陈霸先遂改国號為陈。梁朝前后共10帝,歷時55年。

梁敬帝將皇位禪讓給陳霸先之後,梁朝仍有兩支殘餘勢力與陳朝對抗,分別成為北朝東西兩個政權的傀儡,力爭正統地位。

後梁(555-587)、又稱西梁(據江陵〔今荊州市以南〕地區);傳承三帝蕭詧、蕭巋及蕭琮:西魏及北周支持蕭衍之孫蕭詧。始於西魏恭帝拓跋廓於555年在江陵立蕭詧為梁皇帝以對抗陳霸先;之後北周繼續支持後梁。传三帝共33年,587年亡於隋朝。

東梁(557-573)、據長江中上游地區;傳承蕭莊(557-560)及王琳(560-573):北齊文宣帝高洋則扶植梁元帝孫蕭莊繼承梁朝,對抗陳霸先的篡奪,根據地郢州(今荊州市以北),據有長江中上游地區,主事者為王琳。557年起蕭莊稱帝至560年被陳朝擊敗投奔北齊。後王琳又再據壽陽抵抗至573年勢力方被陳朝消滅,前後共16年。

%% -*- coding: utf-8 -*-
%% Time-stamp: <Chen Wang: 2019-12-20 18:00:48>

\subsection{武帝\tiny(502-549)}

\subsubsection{生平}

梁武帝萧衍(464年-549年),字叔达,小字练儿。南兰陵中都里人(今江苏常州市武进区西北)。南北朝時代南梁開国皇帝,廟號高祖。

萧衍是南齐宗室,亦是蘭陵蕭氏的世家子弟,出生在秣陵(今南京),父亲蕭順之是齐高帝的族弟,封临湘县侯,官至丹阳尹。母张尚柔。萧衍少年時受過良好的儒家教育,私德頗佳、亦不太注重個人享受,是文學名士竟陵八友之一。原為權臣,在其兄長蕭懿被害後,逐漸有帝位之野心,南齐中兴二年(502年),齐和帝被迫禅位于萧衍,南梁建立,是為梁武帝。稱帝後的萧衍改善許多前朝留下的弊政,並多次主持整理經史文書。然而晚年的他多次出家,傾力資助佛教發展直接導致國庫空虛,在侯景之乱爆發後絕食而亡。梁武帝萧衍在位时间長达48年,在南北朝皇帝中名列第一。

萧衍年轻時多才多藝,學識廣博。他的政治、軍事才能,在南朝諸帝中堪稱翹楚,不在另三位開國皇帝之下。在南齊武帝永明年間,他經常在當時的文化中心、竟陵王蕭子良的西邸出入,與沈約、謝脁等人合稱「竟陵八友」,在這期間發表了許多詩作,在學術研究和文學創作上皆有所成就。《梁書》紀載他:“六藝備閑,棋登逸品,陰陽緯候,卜筮占決,並悉稱善。……草隸尺牘,騎射弓馬,莫不奇妙。”他很好學,從小就受到正統的儒家教育,“少時習周孔,弱冠窮六經”,即位之後,“雖萬機多務,猶卷不輟手,燃燭側光,常至午夜”。

齐武帝驾崩时,萧衍没有参与王融意图拥立萧子良的政变,反支持皇太孙萧昭业登基。后又助权臣萧鸾篡位,是為齊明帝。齊明帝的皇叔荆州刺史萧子隆性温和、有文才,明帝欲徵之回朝,恐其不从。萧衍说:“随王(萧子隆)虽有美名,其实能力庸劣,手下没有智谋之士,爪牙只有司马垣历生、武陵太守卞白龙,而且二人唯利是从,若以显职相诱,都会来;随王只需要折简就能召来了。”齊明帝从之,徵垣历生为太子左卫率、卞白龙为游击将军,二人果然都到任。明帝再召萧子隆为侍中、抚军将军,后杀之。

齊明帝死後,繼任的東昏侯蕭寶卷暴虐無道,爆發的亂事在各地将帥們的努力下皆被平息,當中最為得力的是蕭衍的兄長、時任雍州(今湖北省襄陽)刺史的蕭懿。永元二年(500年),蕭懿被誣告謀反,遭東昏侯賜死,由蕭衍接任雍州刺史一職。喜好樂府詩的蕭衍上任後派人搜集當地的民歌,恢復自晉朝以來就已停止的民歌搜集工作。同時他積極招兵,暗中尋找機會推翻東昏侯。他秘密的派人在襄陽大伐竹木,沉於湖底,直到一年後舉兵之時,馬上派人去湖中打撈起事先砍伐好的竹木,並讓早已召集好的數千工匠在最短時間內建造戰船,此即後世 "伐竹沉木" 的典故。

中興元年(501年),蕭衍領兵攻郢城,圍攻兩百餘日,城破,「積屍床下而寢其上,比屋皆滿。」同年十二月,蕭衍發兵攻占首都建康,改立南康王蕭寶融於江陵稱帝,是為齊和帝;東昏侯在政變中被將軍王珍國所殺。中兴二年(502年),皇太后王寶明临朝称制,之後蕭衍受齊和帝禪讓登基,改國號為天監元年(502年),是為梁武帝。

梁武帝昔日的好友沈約、范雲等世族出身的名門後人在梁朝當上宰相,與前朝重臣蕭秀等人合力推動各種改革,改正南齊時施政上的種種問題。此外,武帝登基後對樂府詩的興趣不減當年,仍參與樂府詩的創作及編修。在他的影響和提倡下,梁朝文化的發展達到了東晉以來最繁榮的階段。《南史》作者李延壽評價道:“自江左以來,年逾二百,文物之盛,獨美於茲。”

520年,梁武帝改元普通,這一年被中國歷史學家視為南朝梁發展的分水嶺。在這年開始,梁武帝開始篤信佛法,多次舍身出家。

普通八年(527年)三月八日,梁武帝第一次前往同泰寺舍身出家,三日後返回,大赦天下,改年号大通,是為大通元年(527年)。同年,隸領軍曹仲宗伐渦陽(今安徽蒙城),在關中侯陳慶之的奮鬥下梁軍大敗北魏軍、俘斬甚眾,又乘勝進擊至城父。梁武帝詔下令渦陽之地設置西徐州,並以手詔嘉勉陳慶之:「本非將種,又非豪家,觖望風雲,以至於此。可深思奇略,善克令終。開朱門而待賓,揚聲名於竹帛,豈非大丈夫哉!」

大通三年(529年)九月十五日,梁武帝第二次至同泰寺举行“四部无遮大会”,脱下帝袍,换上僧衣,舍身出家,九月十六日讲解《涅槃经》。當月二十五日群臣捐錢一億,向“三宝”祷告,请求赎回“皇帝菩萨”,二十七日萧衍還俗。

梁武帝因天象称“荧惑入南斗,天子下殿走”,就赤脚下殿以应天象。之後傳來北魏孝武帝西奔的消息,得知此事的武帝羞惭地说道:“綁著辮子的胡虜(索虏)也应天象吗?”(由於天象应於北魏,意味天意认为北魏孝武帝才是正统天子)

大同十二年(546年)四月十日,蕭衍第三次出家,此次群臣用兩億錢將其贖回;太清元年(547年),三月三日蕭衍又第四次出家,在同泰寺住了三十七天,四月十日、朝廷出資一億錢贖回武帝。郭祖深形容:“都下佛寺五百餘所,窮極宏麗。僧尼十餘萬,資產豐沃。”。此時國力日衰。

侯景原为东魏的将领,由于他与东魏丞相高澄的矛盾,于太清元年(547年)正月据河南十三州叛归西魏,但西魏宇文泰对其不信任。迫于无奈,侯景致函萧衍,许愿献出河南十三州来投奔南朝梁。萧衍接纳了侯景,并任命他为大将军,封河南王。不久,东魏攻击侯景,萧衍派姪子萧渊明支援,结果战败,萧渊明被俘。侯景败退后占据寿阳。高澄假意提出和解,意在离间侯景和梁朝。司农卿傅岐认为高澄议和是离间之计。而朱异等人则极力主张与东魏和好。萧衍不听臣下劝告,与东魏使者往来,侯景感到恐慌。

此时,侯景假托东魏名义写信给萧衍,提出用萧渊明交换侯景,萧衍居然表示接受。侯景十分气愤,遂起兵叛变。他以萧正德为内应,轻易渡过了长江,并在公元549年三月围攻建康。城中久被围困,粮食断绝,饥病困扰,人多浮肿气急,横尸满路,能登城抗击者不到四千人。南梁诸王手握重兵,却彼此猜忌按兵不动,无人讨叛。十二日,侯景攻入建康,纵兵洗劫,蕭家宗室、世族琅琊王氏、陳郡謝氏皆遭血洗,史称侯景之亂。

城陷之后,侯景的武士隨意进出皇宫、甚至佩带武器。萧衍见了很奇怪,问左右侍从,侍从说是侯丞相的卫兵。萧衍生气地喝道:“甚么丞相!不就是侯景嗎!”侯景听说了,非常生气,于是派人监视萧衍,萧衍的饮食也被侯景裁减。萧衍忧愤成疾,口苦乾渴,索蜂蜜水,未得实现,怒憤更疾。

据说梁武帝曾经在志公禅师臨終時,向其询问自己寿命。志公说;「我的墓塔倒了,陛下的大限就到了。」志公涅槃後,寺方造了木製的靈塔,梁武帝担心志公的木造靈塔不坚固,就拆除,打算重建,拆了以后不久侯景之乱就发生了。

五月,萧衍在糧食尚足的情況下(身旁有數百顆雞蛋),因激憤不已,病卒於台城內,死前猶喊著軍事戰陣時的「荷,荷」(士兵先退後進的口號),表示他反擊侯景的志願。死时86岁,葬于修陵(今江苏丹阳市陵口)。谥号武帝,庙号高祖。

梁武帝除了帝王的身分,也身為學者在經、史、詩詞、佛學等領域留下大量著述而出名。

在经学方面,他撰有《周易讲疏》、《春秋答问》、《孔子正言》等二百余卷。天监十一年(512年),又制成吉、凶、军、宾、嘉五礼,共一千余卷,八千零十九条,颁布施行。

在史学方面,他不满《汉书》等断代史的写法,因而主持编撰了六百卷的《通史》,并“躬制赞序”。命殷芸将无法入史的剩余材料(主要是异闻杂谈),编入小说。这些著作大都没有流传下来。

在文学方面,梁武帝也非常喜欢诗赋創作,现存古詩、樂府詩等诗歌有80多首。蕭衍和王融、謝朓、任昉、沈約、范雲、蕭琛、陸倕七人共稱竟陵八友,在齊永明時代的文學界頗負盛名。

在宗教方面,今日漢傳佛教的素食主義即以梁武帝為首。佛教的梁皇寶懺是他編製成的,他又著有《大般涅槃經》、《大品般若經》、《淨名經》、《三慧經》等諸經義記數百卷。在道教学说中,他把儒家的“礼法”、道家的“无”和佛教“因果报应”揉合,创立了“三教同源说”,在中国古代思想史上占有极其重要的地位。由於梁武帝對佛教流通的貢獻,寺廟时以梁武帝與其長子昭明太子合祀為護法神。

梁武帝的学问路线,是先习儒,再奉道,后入佛。少年时代是习儒阶段,“少时学周孔,弱冠穷六经”。二十岁以后,改奉道教,一直到即位为帝后,仍未捨道。《隋书·经籍志》载,“武帝弱年好事,先受道法,及即位,犹自上章”。称帝后的萧衍和道士陶弘景的关系极善,他每当遇到国家大事,经常要派人到茅山去向陶弘景请教,以致于陶弘景有“山中宰相”之称。不过,在即位后的第二个年头,即天监三年(504),萧衍就颁布了《捨事道法詔》,宣布捨道归佛。而据其《述三教诗》,则称“晚年开释卷,犹月映众星”。到晚年才开始研读佛经。这也许说明,他虽然已经颁布了事佛诏,实际上还未真正彻底放弃道教。但总的来说,颁诏以后,他是以事佛为主的。有關《捨事道法詔》的真实性在学术界存疑,但无论其真伪,萧衍的奉佛则是事实。

梁武帝对佛教的支持,表现为两大方面:一是亲身修佛,二是从各方面扶持佛教的发展。

梁武帝本人归佛后,逐渐过上了佛教徒的生活。在武帝發表《斷酒肉文》前漢傳佛教「律中無有斷肉法」(反而是與釋迦佛作對的天啟,提倡素食),蕭衍把佛教五戒中的不殺生引申為素食,颁布了《断酒肉文》,禁止僧众吃肉,自己也行素食,開啟了漢傳佛教素食的傳統,之後漢傳佛教僧團開始遵守《梵網經》規定的菩薩戒,不再食肉。对那些敢于饮酒食肉的僧侶,他以世俗的刑法治罪。他又颁布《断杀绝宗庙牺牲诏》,禁止宗庙的牺牲,这是有违儒家祭祀禮儀的,但他坚持推行。他还正式受戒,据《续高僧传》卷六记载,他于天监十八年(519)“发宏誓心,受梵網經菩萨戒”。

梁武帝晚年奉佛更甚,经常日食一餐,過午不食,所食也只是豆羹、粗饭而已。篤信佛教,由於不近女色,曾經四十年無幸后宮,最突出的奉佛行为,是多次舍身出家,先后四次舍身同泰寺,每次都是朝廷花了大量的香火钱才把他赎出来還俗。他的第四次舍身是在太清元年(547)三月,历时一个月,所花赎钱为“一亿万”,这为同泰寺带来了巨额资金。

武帝本人是可以划入“义学”一类的,他对佛经很有研究,尤重《般若经》、《涅槃经》、《法华经》等,他常常为大家讲经说法,召开各种法会,开设过千僧会、无遮大会。中大通元年(529)开设的无遮大会,参加者有道俗五万多人。他的佛教撰述,则有《摩诃般若波罗蜜经注解》(现仅存序)、《三慧经义记》(《三慧经》本是《摩诃般若经》中的《三慧品》,萧衍认为此品最重要,因而独列為《三慧经》)、《制旨大涅槃经讲疏》、《净名经义记》、《制旨大集经讲疏》、《发般若经题论义并问答》(均佚),另著有《立神明成佛义记》、《敕答臣下神灭论》、《为亮法师制涅槃经疏序》、《断酒肉文》、《述三教诗》等,均存。

武帝在哲学上对中国佛教的贡献,突出之处是把中国传统的心性论、灵魂不灭论和佛教的涅槃佛性说结合起来了,他本人是属于涅槃学派的,主张“神明成佛”,所谓“神明”,是指永恒不灭的精神实体,它是众生成佛的内在根据,“神明”也就是佛性。他又提出三教同源论,认为儒、道二教同源于佛教,老子、孔子,都是释迦牟尼的弟子,所以从这个角度来看,三教可以会通,同时,三教的社会作用也是相同的,都是教化人为善。

除了自身奉佛,萧衍还大力扶持佛教事业的发展。他非常支持外僧的译经,僧伽婆罗被他召入五处译场从事译经,所译经典,又请宝唱等人写疏,他甚至“躬临法座,笔受其文,然后乃付译人”。真谛在萧衍门下也受到礼遇,只是因为侯景之乱,真谛的译事难申。萧衍和国内法師的关系也很密切,宝亮、智藏、法云、僧旻等人,都是萧衍非常器重的。他组织僧人编撰佛教著作,编成的作品至少有十二种。他还广造伽藍,所建有大爱敬寺、智度寺、光宅寺、同泰寺等十一座,各寺铸有佛像,大爱敬寺有金铜像,智度寺的正殿铸有金像,光宅寺有丈九无量寿佛铜像,同泰寺有十方银像。

禅宗祖师菩提达摩南北朝时期来中国弘法,与梁武帝会谈。但因理念不合,话不投机,离开梁朝而北去。

在梁武帝的支持下,梁代佛教达到了南朝佛教的最盛期,他最后在侯景之乱时,饥病交加,死于寺中。但武帝之后,梁简文帝和梁元帝也都篤信佛法。

蕭衍登位天子,民望所歸,敢革時政,頗得人心,初期國家興旺繁盛,為一明君。後期太過信仰宗教,企图以佛治民,學者有此評價:一,太過慈悲,不力法治,导致官吏貪污搜刮,百官“缘饰奸谄,深害时政”,奸邪小人纷纷以正人君子的面目出现,官场风气败坏,民间疾苦,国力衰败,而民怨终于为眼光锐利的侯景所利用。二,外交失敗,不能知人,“险躁之心,暮年愈甚”,导致侯景之乱,侯景之乱彻底打击了江南的经济基础、人口基础。

錢穆於《國史大綱》云:“獨有一蕭衍老翁,儉過漢文,勤如王莽,可謂南朝一令主。”
王夫之於《讀通鑑論》亦云:“梁氏享國五十年,天下且小康焉。”

\subsubsection{天监}

\begin{longtable}{|>{\centering\scriptsize}m{2em}|>{\centering\scriptsize}m{1.3em}|>{\centering}m{8.8em}|}
  % \caption{秦王政}\
  \toprule
  \SimHei \normalsize 年数 & \SimHei \scriptsize 公元 & \SimHei 大事件 \tabularnewline
  % \midrule
  \endfirsthead
  \toprule
  \SimHei \normalsize 年数 & \SimHei \scriptsize 公元 & \SimHei 大事件 \tabularnewline
  \midrule
  \endhead
  \midrule
  元年 & 502 & \tabularnewline\hline
  二年 & 503 & \tabularnewline\hline
  三年 & 504 & \tabularnewline\hline
  四年 & 505 & \tabularnewline\hline
  五年 & 506 & \tabularnewline\hline
  六年 & 507 & \tabularnewline\hline
  七年 & 508 & \tabularnewline\hline
  八年 & 509 & \tabularnewline\hline
  九年 & 510 & \tabularnewline\hline
  十年 & 511 & \tabularnewline\hline
  十一年 & 512 & \tabularnewline\hline
  十二年 & 513 & \tabularnewline\hline
  十三年 & 514 & \tabularnewline\hline
  十四年 & 515 & \tabularnewline\hline
  十五年 & 516 & \tabularnewline\hline
  十六年 & 517 & \tabularnewline\hline
  十七年 & 518 & \tabularnewline\hline
  十八年 & 519 & \tabularnewline
  \bottomrule
\end{longtable}

\subsubsection{普通}

\begin{longtable}{|>{\centering\scriptsize}m{2em}|>{\centering\scriptsize}m{1.3em}|>{\centering}m{8.8em}|}
  % \caption{秦王政}\
  \toprule
  \SimHei \normalsize 年数 & \SimHei \scriptsize 公元 & \SimHei 大事件 \tabularnewline
  % \midrule
  \endfirsthead
  \toprule
  \SimHei \normalsize 年数 & \SimHei \scriptsize 公元 & \SimHei 大事件 \tabularnewline
  \midrule
  \endhead
  \midrule
  元年 & 520 & \tabularnewline\hline
  二年 & 521 & \tabularnewline\hline
  三年 & 522 & \tabularnewline\hline
  四年 & 523 & \tabularnewline\hline
  五年 & 524 & \tabularnewline\hline
  六年 & 525 & \tabularnewline\hline
  七年 & 526 & \tabularnewline\hline
  八年 & 527 & \tabularnewline
  \bottomrule
\end{longtable}

\subsubsection{大通}

\begin{longtable}{|>{\centering\scriptsize}m{2em}|>{\centering\scriptsize}m{1.3em}|>{\centering}m{8.8em}|}
  % \caption{秦王政}\
  \toprule
  \SimHei \normalsize 年数 & \SimHei \scriptsize 公元 & \SimHei 大事件 \tabularnewline
  % \midrule
  \endfirsthead
  \toprule
  \SimHei \normalsize 年数 & \SimHei \scriptsize 公元 & \SimHei 大事件 \tabularnewline
  \midrule
  \endhead
  \midrule
  元年 & 527 & \tabularnewline\hline
  二年 & 528 & \tabularnewline\hline
  三年 & 529 & \tabularnewline
  \bottomrule
\end{longtable}

\subsubsection{中大通}

\begin{longtable}{|>{\centering\scriptsize}m{2em}|>{\centering\scriptsize}m{1.3em}|>{\centering}m{8.8em}|}
  % \caption{秦王政}\
  \toprule
  \SimHei \normalsize 年数 & \SimHei \scriptsize 公元 & \SimHei 大事件 \tabularnewline
  % \midrule
  \endfirsthead
  \toprule
  \SimHei \normalsize 年数 & \SimHei \scriptsize 公元 & \SimHei 大事件 \tabularnewline
  \midrule
  \endhead
  \midrule
  元年 & 529 & \tabularnewline\hline
  二年 & 530 & \tabularnewline\hline
  三年 & 531 & \tabularnewline\hline
  四年 & 532 & \tabularnewline\hline
  五年 & 533 & \tabularnewline\hline
  六年 & 534 & \tabularnewline
  \bottomrule
\end{longtable}

\subsubsection{大同}

\begin{longtable}{|>{\centering\scriptsize}m{2em}|>{\centering\scriptsize}m{1.3em}|>{\centering}m{8.8em}|}
  % \caption{秦王政}\
  \toprule
  \SimHei \normalsize 年数 & \SimHei \scriptsize 公元 & \SimHei 大事件 \tabularnewline
  % \midrule
  \endfirsthead
  \toprule
  \SimHei \normalsize 年数 & \SimHei \scriptsize 公元 & \SimHei 大事件 \tabularnewline
  \midrule
  \endhead
  \midrule
  元年 & 535 & \tabularnewline\hline
  二年 & 536 & \tabularnewline\hline
  三年 & 537 & \tabularnewline\hline
  四年 & 538 & \tabularnewline\hline
  五年 & 539 & \tabularnewline\hline
  六年 & 540 & \tabularnewline\hline
  七年 & 541 & \tabularnewline\hline
  八年 & 542 & \tabularnewline\hline
  九年 & 543 & \tabularnewline\hline
  十年 & 544 & \tabularnewline\hline
  十一年 & 545 & \tabularnewline\hline
  十二年 & 546 & \tabularnewline
  \bottomrule
\end{longtable}

\subsubsection{中大同}

\begin{longtable}{|>{\centering\scriptsize}m{2em}|>{\centering\scriptsize}m{1.3em}|>{\centering}m{8.8em}|}
  % \caption{秦王政}\
  \toprule
  \SimHei \normalsize 年数 & \SimHei \scriptsize 公元 & \SimHei 大事件 \tabularnewline
  % \midrule
  \endfirsthead
  \toprule
  \SimHei \normalsize 年数 & \SimHei \scriptsize 公元 & \SimHei 大事件 \tabularnewline
  \midrule
  \endhead
  \midrule
  元年 & 546 & \tabularnewline\hline
  二年 & 547 & \tabularnewline
  \bottomrule
\end{longtable}

\subsubsection{太清}

\begin{longtable}{|>{\centering\scriptsize}m{2em}|>{\centering\scriptsize}m{1.3em}|>{\centering}m{8.8em}|}
  % \caption{秦王政}\
  \toprule
  \SimHei \normalsize 年数 & \SimHei \scriptsize 公元 & \SimHei 大事件 \tabularnewline
  % \midrule
  \endfirsthead
  \toprule
  \SimHei \normalsize 年数 & \SimHei \scriptsize 公元 & \SimHei 大事件 \tabularnewline
  \midrule
  \endhead
  \midrule
  元年 & 547 & \tabularnewline\hline
  二年 & 548 & \tabularnewline\hline
  三年 & 549 & \tabularnewline
  \bottomrule
\end{longtable}


%%% Local Variables:
%%% mode: latex
%%% TeX-engine: xetex
%%% TeX-master: "../../Main"
%%% End:

%% -*- coding: utf-8 -*-
%% Time-stamp: <Chen Wang: 2021-11-01 15:05:44>

\subsection{简文帝蕭綱\tiny(549-551)}

\subsubsection{生平}

梁簡文帝蕭綱(503年-551年),字世讚,一作世纘,小字六通,梁武帝蕭衍第三子,昭明太子蕭統的胞弟,母丁令光。

蕭綱最早封為晉安王,曾經擔任過南徐州刺史,並且曾經參與北伐;531年蕭統病故之後被封為皇太子。

548年侯景叛亂,萧纲助守台城,梁武帝因自认为年老,授权萧纲主军国大事。侯景部下仪同三司范桃棒在被俘的云旗将军陈昕劝说下图谋率所部袭杀侯景部下行台左丞王伟、部将宋子仙,再去建康投降。范桃棒写信射入建康城中,再秘密派陈昕趁夜吊绳入城。武帝大喜,但萧纲担心有诈,犹豫不决。范桃棒又派陈昕写信说:“现在仅带所领五百人,如果到城门,都自己脱甲,乞求朝廷开门赐容。事成之后,保证擒侯景。”萧纲见其恳切,愈发生疑。结果事泄,范桃棒被杀,陈昕出城接应后也被擒杀。

侯景攻陷台城後,梁武帝於549年憂憤而死,但是侯景認為目前仍然不能自立為皇帝,便擁立蕭綱為皇帝,次年改元大寶。但是蕭綱不過是侯景的傀儡。551年,侯景派人廢蕭綱為晉安王,改立豫章王蕭棟為皇帝;蕭綱被囚禁於永福省,蕭綱被廢後兩個月,被侯景派人以棉被悶死,享年49歲。

侯景事後為蕭綱上諡號曰明皇帝,廟號高宗,梁元帝在552年追諡蕭綱為簡文皇帝,廟號太宗。

蕭綱本人文學造詣很高,雅好詩賦,有大量詠物、宮體、閨怨之作,其中五言詩最多,並且與蕭子显、蕭繹、徐擒、庾肩吾等人形成宮體詩流派,萧纲是宫体诗的代表。侯景攻入建康期间,曾经“募人出烧东宫,东宫台殿遂尽。所聚百橱图籍,一皆灰烬”。


\subsubsection{大宝}

\begin{longtable}{|>{\centering\scriptsize}m{2em}|>{\centering\scriptsize}m{1.3em}|>{\centering}m{8.8em}|}
  % \caption{秦王政}\
  \toprule
  \SimHei \normalsize 年数 & \SimHei \scriptsize 公元 & \SimHei 大事件 \tabularnewline
  % \midrule
  \endfirsthead
  \toprule
  \SimHei \normalsize 年数 & \SimHei \scriptsize 公元 & \SimHei 大事件 \tabularnewline
  \midrule
  \endhead
  \midrule
  元年 & 550 & \tabularnewline\hline
  二年 & 551 & \tabularnewline
  \bottomrule
\end{longtable}


%%% Local Variables:
%%% mode: latex
%%% TeX-engine: xetex
%%% TeX-master: "../../Main"
%%% End:

%% -*- coding: utf-8 -*-
%% Time-stamp: <Chen Wang: 2021-11-01 15:06:11>

\subsection{淮陰王萧栋\tiny(551)}

\subsubsection{生平}

蕭棟(?-552年),字元吉,南朝梁朝的第三代皇帝。史稱豫章王、淮陰王。

蕭棟為昭明太子蕭統之孫,豫章王蕭歡之子。蕭統去世後,梁武帝曾經有一度想立蕭歡為皇太孫,但最後沒有,而改封蕭統三弟后来的梁簡文帝蕭綱當太子。萧欢死后,萧栋袭为豫章王。

侯景之乱期间,叛将侯景攻破梁都建康,将萧栋等在京宗室软禁。

551年,侯景被皇弟湘东王萧绎部将王僧辩所败,回到建康,图谋篡位。他以簡文帝名义下诏禅位给萧栋。当时京城一带饥荒,萧栋与王妃张氏正在园中种菜,看到来迎接他为帝的士兵,不知所措,哭着登车,被侯景立为皇帝,升武德殿。当时平地起风,将华盖吹翻直出端门,时人知道萧栋不能善终。改元天正。侯景掌权,萧栋毫无实权。他追封祖父萧统、父萧欢为皇帝,尊母王氏为皇太后,立王妃张氏为皇后。

两个半月後,因侯景所迫,萧栋加侯景殊礼、九锡,侯景不久廢蕭棟為淮陰王,並自立為漢皇帝,並將蕭棟與其弟蕭橋、蕭樛囚於密室之中。

萧绎登基为梁元帝并收復建業後,指使王僧辩杀萧栋,王僧辩拒绝。萧绎于是命宣猛将军朱买臣杀萧栋。当时,蕭棟與其弟都逃出密室,相扶而出,但仍戴着镣铐,遇到将军杜崱后才去除。弟弟们以为逃出生天,但萧栋认为未必,仍然害怕。不久,他们遇到朱买臣,朱买臣请他们上船饮酒,席未终,朱买臣所部士兵抓住他们,沉入扬子江。

\subsubsection{天正}

\begin{longtable}{|>{\centering\scriptsize}m{2em}|>{\centering\scriptsize}m{1.3em}|>{\centering}m{8.8em}|}
  % \caption{秦王政}\
  \toprule
  \SimHei \normalsize 年数 & \SimHei \scriptsize 公元 & \SimHei 大事件 \tabularnewline
  % \midrule
  \endfirsthead
  \toprule
  \SimHei \normalsize 年数 & \SimHei \scriptsize 公元 & \SimHei 大事件 \tabularnewline
  \midrule
  \endhead
  \midrule
  元年 & 551 & \tabularnewline
  \bottomrule
\end{longtable}


%%% Local Variables:
%%% mode: latex
%%% TeX-engine: xetex
%%% TeX-master: "../../Main"
%%% End:

%% -*- coding: utf-8 -*-
%% Time-stamp: <Chen Wang: 2019-12-23 14:07:19>

\subsection{元帝\tiny(552-554)}

\subsubsection{生平}

梁元帝萧绎(508年9月16日-555年1月27日),字世誠,梁武帝蕭衍的第七子,正式谥號為「孝元皇帝」,後世比照西漢的漢元帝和東晉的晋元帝,省「孝」字稱「梁元帝」。

萧绎於514年封湘东王,早年因病而一眼失明。547年出荊州,任荊州刺史、使持節、都督荊雍湘司郢寧梁南北秦九州諸軍事、鎮西將軍。侯景之亂時,梁武帝遣人至荊州宣讀密詔,授蕭繹为侍中、假黄钺、大都督中外诸军事、司徒承制,其余职务如故。萧绎手握强兵,却没有积极勤王。

549年梁武帝餓死台城後,蕭繹首先發兵攻滅自己的侄兒河東王蕭譽與哥哥邵陵王蕭綸,並擊退萧誉弟岳阳王襄陽都督萧詧的來犯,迫使萧詧投靠西魏;之後再命王僧辯率軍東下消滅侯景。其长子萧方等即在与萧誉作战时身亡,次子萧方诸在与侯景交战时被擒杀。萧绎弟益州刺史武陵王蕭紀亦有意出兵共讨侯景,萧绎写信阻止,称“蜀人勇悍,易动难安,弟可镇之,吾自当灭贼。”又写信称“地拟孙、刘,各安境界;情深鲁、卫,书信恒通”以为示好。

552年侯景死後,蕭繹即帝位於江陵,是為梁世祖。当时,群臣中有人建议返回旧都建康,但蕭繹没有同意。并派手下朱买臣在建康杀死侯景所废皇帝萧栋兄弟三人。

萧绎即帝位之前,蕭紀已稱帝於益州;萧纪出兵讨伐侯景,得知侯景已灭,就转为讨伐萧绎。蕭繹便派兵迎战,写信讲和,同时也请求西魏出兵袭取益州。萧纪因此遭受重创,向萧绎求和,萧绎回信拒绝称兄弟情断,最终全歼萧纪势力,但也給了西魏可趁之機,益州因此沦落敌手。萧绎将萧纪父子除宗籍改姓饕餮,并将萧纪二子萧圆照、萧圆正饿死。

554年,蕭繹给西魏权臣宇文泰写信,要求按照旧图重新划定疆界,言辞又极为傲慢。宇文泰大为不满,命令常山公于谨、大將軍楊忠、兄子大将军宇文护等将领以5万兵马进攻江陵(今湖北江陵縣)。梁元帝战败,由御史中丞王孝祀作降文。隨后,便率太子等人到西魏軍營投降。不久為萧詧以土袋悶死,江陵“阖城老幼被虏入关”。

梁元帝也是一個愛好讀書與喜好文學的君主,“四十六岁,自聚书来四十年,得书八万卷”,自稱“韜於文士,愧於武夫。”曾主編《金樓子》等書;江陵被圍城時,承聖三年十二月辛未(555年1月27日),元帝入東閤竹殿,命舍人高善寶放火焚燒圖書14萬卷,包括从建康为避兵灾而转移到江陵的8万卷书,自稱“文武之道,今夜盡矣!”“讀書萬卷,猶有今日,故焚之。”江陵焚書被視為中國的文化浩劫之一。

清朝初年的王船山評論其焚書行徑:「未有不恶其不悔不仁而归咎于读书者,曰书何负于帝哉?此非知读书者之言也。

北宋的司馬光評論:「元帝於兄弟之中,殘忍尤甚,是以雖翦兇渠而克復故業,旋踵之間,身為伏馘;豈特人心之不與哉?亦天地之所誅也。」

唐朝的虞世南:「梁元聪明伎艺,才兼文武,仗顺伐逆,克雪家冤,成功遂事,有足称者。但国难之后,伤夷未复,信强寇之甘言,袭褊心于怀楚,蕃屏宗支自为仇敌,孤远悬僻,莫与同憂,身亡祚灭,生人涂炭,举鄢、郢而弃之,良可惜也。」

陳朝的史家何之元:「世祖聰明特達,才藝兼美,詩筆之麗,罕與為匹,伎能之事,無所不該,極星象之功,窮蓍龜之妙,明筆法于馬室,不愧鄭玄,辨雲物于魯台,無慚梓慎,至於帷籌將略,朝野所推,遂乃撥亂反正,夷凶殄逆,紐地維之已絕,扶天柱之將傾,黔首蒙拯溺之恩,蒼生荷仁壽之惠,微管之力,民其戎乎?鯨鯢既誅,天下且定,早應移鑾西楚,旋駕東都,祀宗土方,清蹕宮闕,西周岳陽之敗績,信口宇文之和通,以萬乘之尊,居二境之上,夷虜乘釁,再覆皇基,率土分崩,莫知攸暨,謀之不善,乃至於斯。」

\subsubsection{承圣}

\begin{longtable}{|>{\centering\scriptsize}m{2em}|>{\centering\scriptsize}m{1.3em}|>{\centering}m{8.8em}|}
  % \caption{秦王政}\
  \toprule
  \SimHei \normalsize 年数 & \SimHei \scriptsize 公元 & \SimHei 大事件 \tabularnewline
  % \midrule
  \endfirsthead
  \toprule
  \SimHei \normalsize 年数 & \SimHei \scriptsize 公元 & \SimHei 大事件 \tabularnewline
  \midrule
  \endhead
  \midrule
  元年 & 552 & \tabularnewline\hline
  二年 & 553 & \tabularnewline\hline
  三年 & 554 & \tabularnewline\hline
  四年 & 555 & \tabularnewline
  \bottomrule
\end{longtable}


%%% Local Variables:
%%% mode: latex
%%% TeX-engine: xetex
%%% TeX-master: "../../Main"
%%% End:

%% -*- coding: utf-8 -*-
%% Time-stamp: <Chen Wang: 2021-11-01 15:06:27>

\subsection{闵帝蕭淵明\tiny(555)}

\subsubsection{生平}

梁閔帝蕭淵明(5世紀?-556年6月2日),字靖通,蕭懿之子,梁武帝蕭衍之姪。

蕭淵明最早封貞陽侯,後擔任豫州刺史;在侯景背叛東魏投降南梁之時,蕭衍命蕭淵明與侯景北伐攻打東魏,結果蕭淵明兵敗被俘。侯景曾伪造东魏书信称愿意放还萧渊明交换侯景,萧衍回信答应“贞阳朝至,侯景夕返”,侯景遂骂萧衍薄心肠而作乱。

後在西魏攻陷江陵殺害梁孝元帝時,王僧辩、陈霸先意图拥立元帝子萧方智为帝,立萧方智为梁王;北齊文宣帝高洋與上黨王高渙於555年送回蕭淵明,準備讓蕭淵明成為北齊支持的傀儡皇帝,王僧辩未能抵抗齐军,于是接受萧渊明,但要求立萧方智为皇太子;蕭淵明於是即帝位,改年號為天成,并立萧方智为皇太子。不久後陳霸先诛王僧辯,並且廢蕭淵明,改立蕭方智為皇帝。蕭淵明改封建安公。

之後,北齊要求南梁送回蕭淵明,陳霸先也準備將蕭淵明送還北齊。但還沒有出發蕭淵明便病故。蕭莊在北齊支持下稱帝之後,諡蕭淵明為閔皇帝。

\subsubsection{天成}

\begin{longtable}{|>{\centering\scriptsize}m{2em}|>{\centering\scriptsize}m{1.3em}|>{\centering}m{8.8em}|}
  % \caption{秦王政}\
  \toprule
  \SimHei \normalsize 年数 & \SimHei \scriptsize 公元 & \SimHei 大事件 \tabularnewline
  % \midrule
  \endfirsthead
  \toprule
  \SimHei \normalsize 年数 & \SimHei \scriptsize 公元 & \SimHei 大事件 \tabularnewline
  \midrule
  \endhead
  \midrule
  元年 & 555 & \tabularnewline
  \bottomrule
\end{longtable}


%%% Local Variables:
%%% mode: latex
%%% TeX-engine: xetex
%%% TeX-master: "../../Main"
%%% End:

%% -*- coding: utf-8 -*-
%% Time-stamp: <Chen Wang: 2019-12-23 14:18:46>

\subsection{敬帝\tiny(555-557)}

\subsubsection{生平}

梁敬帝蕭方智(543年-558年5月5日),字慧相,南梁的末代皇帝,梁元帝蕭繹的第九子。549年,蕭方智被封為興梁侯;552年被封為晉安王,553年被封為江州刺史。

當梁元帝在江陵被殺之時,555年陳霸先、王僧辯擁立蕭方智以太宰承制於建康,立为梁王,但是北齊將元帝堂兄貞陽侯蕭淵明送回之後,王僧辯因无力抵抗,又同意擁立蕭淵明為皇帝,但要求立萧方智为皇太子。萧渊明登基后,立萧方智为皇太子。陳霸先以王僧辩投降北齐、抛弃元帝子为由袭殺王僧辯,萧渊明亦退位,萧方智登基。557年,蕭方智禪位與陳霸先,南朝梁被南陳取代。

陳霸先封蕭方智為江陰王,永定二年四月乙丑(558年5月5日),陈武帝陈霸先派人杀死梁敬帝,陳霸先追諡曰敬帝,封梁武帝十弟鄱陽王蕭恢之孫蕭季卿為江陰王。

\subsubsection{绍泰}

\begin{longtable}{|>{\centering\scriptsize}m{2em}|>{\centering\scriptsize}m{1.3em}|>{\centering}m{8.8em}|}
  % \caption{秦王政}\
  \toprule
  \SimHei \normalsize 年数 & \SimHei \scriptsize 公元 & \SimHei 大事件 \tabularnewline
  % \midrule
  \endfirsthead
  \toprule
  \SimHei \normalsize 年数 & \SimHei \scriptsize 公元 & \SimHei 大事件 \tabularnewline
  \midrule
  \endhead
  \midrule
  元年 & 555 & \tabularnewline\hline
  二年 & 556 & \tabularnewline
  \bottomrule
\end{longtable}

\subsubsection{太平}

\begin{longtable}{|>{\centering\scriptsize}m{2em}|>{\centering\scriptsize}m{1.3em}|>{\centering}m{8.8em}|}
  % \caption{秦王政}\
  \toprule
  \SimHei \normalsize 年数 & \SimHei \scriptsize 公元 & \SimHei 大事件 \tabularnewline
  % \midrule
  \endfirsthead
  \toprule
  \SimHei \normalsize 年数 & \SimHei \scriptsize 公元 & \SimHei 大事件 \tabularnewline
  \midrule
  \endhead
  \midrule
  元年 & 556 & \tabularnewline\hline
  二年 & 557 & \tabularnewline
  \bottomrule
\end{longtable}

\subsection{东梁简介}

蕭莊(548年-577年),南兰陵郡兰陵县(今江苏省常州市武进区)人,梁元帝之孫,武烈世子蕭方等之子。

554年,西魏攻陷江陵、殺害梁元帝時,蕭莊只有七歲,逃匿於民家之中;之後被王琳發現,將蕭莊護送回建康。梁敬帝蕭方智即帝位之後,將蕭莊作為人質送往北齊。

557年,陳霸先廢蕭方智即帝位後,王琳等人要求北齊送還蕭莊,並使其接替南梁皇帝;蕭莊回到南朝之後,王琳立蕭莊為梁皇帝於郢州,據有長江中上游地區,是為東梁。之後蕭莊的南梁與陳霸先的陳朝便持續交戰,560年,當王琳與陳朝的侯瑱在蕪湖交戰時,北周便發兵攻打郢州,結果王琳兵敗,與蕭莊逃亡北齊。蕭莊被封为侯,又封梁王。北齐允諾幫他復興梁朝,但没能实现。齐后主高纬灭亡后,蕭莊在邺城闭气自杀。



%%% Local Variables:
%%% mode: latex
%%% TeX-engine: xetex
%%% TeX-master: "../../Main"
%%% End:

%% -*- coding: utf-8 -*-
%% Time-stamp: <Chen Wang: 2021-11-01 15:07:19>

\subsection{后梁(555-587)}

\subsubsection{简介}

後梁(555年-587年)為中國南北朝時期南梁在蕭詧稱帝後殘存的政权。國都於江陵,统治地区位于原梁朝國都建康的西边,故又稱為西梁。

在南梁末年的侯景之乱中,梁武帝孙、昭明太子萧统子岳阳王萧詧为叔父湘东王萧绎所迫投降西魏,封梁王;萧绎后来登基,史称梁元帝。554年西魏攻陷江陵后,梁元帝投降,被萧詧杀死。萧詧於555年称帝,並且對西魏稱臣;但是後梁由於國土狹小,屬地僅有江陵附近數縣八百里地,先後是西魏、北周和隋朝的附庸。但後梁也一直自居為南朝正統而與陳朝對立。而後梁也由於承續南梁的文化,而成為具有高度文化的國家。

後梁共傳宣帝蕭詧、明帝蕭巋、後主蕭琮三世。587年九月,隋文帝廢除後梁,改蕭琮為莒國公,後梁因此滅亡,存在共三十三年。然而由於蕭氏歷代事奉北周、對隋朝甚為恭謹,蕭巋之女還被隋文帝选为晋王妃即后来隋煬帝楊廣之皇后(炀愍皇后),因此在後梁廢除後,蕭氏在隋朝中央朝廷與江陵仍保有一定的政治影響力。隋末群雄之一的蕭銑即為蕭巋弟蕭巖之孫。

\subsubsection{宣帝蕭詧}

梁宣帝蕭\xpinyin*{詧}(519年-562年),字理孫,梁武帝之孫、昭明太子蕭統之第三子。為後梁(西梁)的建立者,諡號宣皇帝,廟號中宗。

萧统死后,梁武帝立萧纲为皇太子,而进封萧统诸子为王。蕭詧被封為岳陽郡王並被任命為東揚州刺史,鎮守會稽。但他们仍然因未被立储而对梁武帝怀恨。

546年十月,萧詧改任雍州刺史,鎮守襄陽。549年,侯景之乱时,蕭詧兄長湘州刺史河東王蕭譽被他們的叔父湘東王(之後的梁元帝)蕭繹攻擊,邵陵王萧纶劝阻无果,蕭詧試圖救援蕭譽兵敗,部将杜岸叛变反以五百骑攻打襄阳,被留守的蔡大宝和萧詧母龚保林守御而未果。萧詧连夜撤回襄阳,龚保林不知他兵败,以为他是敌军,直到天亮了认出他才放他入城。萧詧自知得罪萧绎,为了自保,據襄陽歸降西魏,西魏於550年三月封蕭詧為梁王。萧纶被西魏军所杀,萧詧葬之。554年冬西魏派大将于谨等攻打江陵,梁元帝開門投降,被蕭詧侮辱后以土袋悶死。

之後西魏於555年在江陵立蕭詧為梁皇帝,年號大定。萧詧部将尹德毅劝他趁西魏人贪婪多有杀伤之机,趁西魏精锐尽在此时设宴,埋伏武士杀死于谨等人,再分头命令果决勇敢的人奇袭魏军营垒,全歼魏军,抚慰江陵百姓,任命文武百官,以救命之恩获得百姓支持,还可以写信招揽王僧辩等人,着朝服、渡长江,登基称帝,继承尧、禹之业。萧詧却说:“您的这条计策,并不是不好。可是魏人待我十分宽厚,我不能违背道德。如果仓促之间依计而行,就会像邓祁侯说的那样,我不会有好下场了。”

果然,西魏除江陵附近八百里之地外,將襄陽等地皆併入西魏,並且將江陵一帶的人民財產擄掠一空,萧詧所辖只有江陵周边八百里。萧詧追悔莫及,见屋宇残破,战乱不息,为自己威略不振而感到羞耻,心中常怀忧愤,于是作《愍时赋》自抒其意。每每读到“老马伏枥,志在千里,烈士暮年,壮心不已”就扬眉举目,握腕激奋,久久叹息不止。即位八年後,562年,蕭詧在抑鬱中病故。

\subsubsection{大定}

\begin{longtable}{|>{\centering\scriptsize}m{2em}|>{\centering\scriptsize}m{1.3em}|>{\centering}m{8.8em}|}
  % \caption{秦王政}\
  \toprule
  \SimHei \normalsize 年数 & \SimHei \scriptsize 公元 & \SimHei 大事件 \tabularnewline
  % \midrule
  \endfirsthead
  \toprule
  \SimHei \normalsize 年数 & \SimHei \scriptsize 公元 & \SimHei 大事件 \tabularnewline
  \midrule
  \endhead
  \midrule
  元年 & 555 & \tabularnewline\hline
  二年 & 556 & \tabularnewline\hline
  三年 & 557 & \tabularnewline\hline
  四年 & 558 & \tabularnewline\hline
  五年 & 559 & \tabularnewline\hline
  六年 & 560 & \tabularnewline\hline
  七年 & 561 & \tabularnewline\hline
  八年 & 562 & \tabularnewline
  \bottomrule
\end{longtable}


\subsubsection{明帝蕭巋}

梁明帝蕭巋(542年-585年),字仁遠,是南北朝時代西梁的第二位君主。正式諡號為「孝明皇帝」,後世比照漢朝和西晉皇帝省略「孝」字,稱「梁明帝」。

西梁是南梁的一個分裂王朝,它的地盤主要在今天湖北襄陽、荊州地區,首都江陵(今湖北省荊州市)。蕭巋的父親蕭詧與梁元帝蕭繹不和,蕭繹繼梁帝位後,蕭詧就投靠西魏,被西魏皇帝封為梁王,在他的統治地盤內他自稱皇帝,但實際上西梁的「皇帝」在他們的領土上並沒有真正的主權,很長時間裡北朝在西梁設有江陵總管,一方面用來監督西梁的君主,另一方面這些總管擁有兵權來保護西梁不被南朝攻擊。蕭詧死後他的兒子蕭巋於562年以皇太子身份繼帝位。

蕭巋的年號是天保,他繼續他父親的政策,聯合北朝(北周)來抵抗南朝(南陳)的威脅。北周武帝宇文邕滅北齊後蕭巋親自赴長安祝賀,因此深得宇文邕的信任。隋文帝楊堅登基後再次親自赴長安祝賀,又贏得了楊堅的信任。後來蕭、楊兩家又通婚,蕭巋的一個女兒還嫁給了楊廣,後來成為隋煬帝的皇后。由於蕭、楊兩家的關係如此親密,因此後來隋將它駐扎在西梁的江陵總管撤回,使得西梁獲得了自主權。

蕭巋是一個相當有學問的皇帝,他曾著《孝經》、《周易義記》、《大小乘幽微》等十四部書。

蕭巋於天保二十四年(585年)五月逝世,諡為孝明皇帝,廟號世宗。

\subsubsection{天保}

\begin{longtable}{|>{\centering\scriptsize}m{2em}|>{\centering\scriptsize}m{1.3em}|>{\centering}m{8.8em}|}
  % \caption{秦王政}\
  \toprule
  \SimHei \normalsize 年数 & \SimHei \scriptsize 公元 & \SimHei 大事件 \tabularnewline
  % \midrule
  \endfirsthead
  \toprule
  \SimHei \normalsize 年数 & \SimHei \scriptsize 公元 & \SimHei 大事件 \tabularnewline
  \midrule
  \endhead
  \midrule
  元年 & 562 & \tabularnewline\hline
  二年 & 563 & \tabularnewline\hline
  三年 & 564 & \tabularnewline\hline
  四年 & 565 & \tabularnewline\hline
  五年 & 566 & \tabularnewline\hline
  六年 & 567 & \tabularnewline\hline
  七年 & 568 & \tabularnewline\hline
  八年 & 569 & \tabularnewline\hline
  九年 & 570 & \tabularnewline\hline
  十年 & 571 & \tabularnewline\hline
  十一年 & 572 & \tabularnewline\hline
  十二年 & 573 & \tabularnewline\hline
  十三年 & 574 & \tabularnewline\hline
  十四年 & 575 & \tabularnewline\hline
  十五年 & 576 & \tabularnewline\hline
  十六年 & 577 & \tabularnewline\hline
  十七年 & 578 & \tabularnewline\hline
  十八年 & 579 & \tabularnewline\hline
  十九年 & 580 & \tabularnewline\hline
  二十年 & 581 & \tabularnewline\hline
  二一年 & 582 & \tabularnewline\hline
  二二年 & 583 & \tabularnewline\hline
  二三年 & 584 & \tabularnewline\hline
  二四年 & 585 & \tabularnewline
  \bottomrule
\end{longtable}


\subsubsection{后主蕭琮}

蕭琮(558年-607年),字溫文,為西梁明帝蕭巋之子,西梁第三位,亦是末代皇帝。

蕭琮最早封東陽王,後被立為皇太子。蕭琮博學有才,善於弓馬,個性倜儻不羈。585年即位為西梁皇帝,改年號為廣運。蕭琮即位之後,隋文帝設立江陵總管監視蕭琮的行為;587年,因蕭琮的叔父蕭巖等人帶了一部分居民逃入陳朝,隋文帝徵召蕭琮入朝,当年10月26日廢除西梁國,蕭琮亦被廢為莒國公。西梁也因此滅亡。

蕭琮在隋朝時仍然受到器重,隋煬帝即位後,因为萧琮是自己的妻兄,对萧琮更厚待,又封蕭琮為梁公、內史令,蕭琮的親族也有不少被提拔入朝廷為官。當時有童謠說:「蕭蕭亦復起」,導致隋煬帝對蕭琮的猜忌,最後蕭琮被免職,不久後在家中過世。

蕭琮死後被贈左光祿大夫,侄萧钜续封为梁公,史书没有记载萧琮子萧铉为何没有袭爵。

隋末割據勢力之一的蕭銑,為蕭琮之堂姪,並在稱帝之後追諡蕭琮為孝靖皇帝。

\subsubsection{广运}

\begin{longtable}{|>{\centering\scriptsize}m{2em}|>{\centering\scriptsize}m{1.3em}|>{\centering}m{8.8em}|}
  % \caption{秦王政}\
  \toprule
  \SimHei \normalsize 年数 & \SimHei \scriptsize 公元 & \SimHei 大事件 \tabularnewline
  % \midrule
  \endfirsthead
  \toprule
  \SimHei \normalsize 年数 & \SimHei \scriptsize 公元 & \SimHei 大事件 \tabularnewline
  \midrule
  \endhead
  \midrule
  元年 & 586 & \tabularnewline\hline
  二年 & 587 & \tabularnewline
  \bottomrule
\end{longtable}


%%% Local Variables:
%%% mode: latex
%%% TeX-engine: xetex
%%% TeX-master: "../../Main"
%%% End:



%%% Local Variables:
%%% mode: latex
%%% TeX-engine: xetex
%%% TeX-master: "../../Main"
%%% End:
