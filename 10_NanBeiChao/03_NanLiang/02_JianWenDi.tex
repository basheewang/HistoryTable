%% -*- coding: utf-8 -*-
%% Time-stamp: <Chen Wang: 2021-11-01 15:05:44>

\subsection{简文帝蕭綱\tiny(549-551)}

\subsubsection{生平}

梁簡文帝蕭綱(503年-551年),字世讚,一作世纘,小字六通,梁武帝蕭衍第三子,昭明太子蕭統的胞弟,母丁令光。

蕭綱最早封為晉安王,曾經擔任過南徐州刺史,並且曾經參與北伐;531年蕭統病故之後被封為皇太子。

548年侯景叛亂,萧纲助守台城,梁武帝因自认为年老,授权萧纲主军国大事。侯景部下仪同三司范桃棒在被俘的云旗将军陈昕劝说下图谋率所部袭杀侯景部下行台左丞王伟、部将宋子仙,再去建康投降。范桃棒写信射入建康城中,再秘密派陈昕趁夜吊绳入城。武帝大喜,但萧纲担心有诈,犹豫不决。范桃棒又派陈昕写信说:“现在仅带所领五百人,如果到城门,都自己脱甲,乞求朝廷开门赐容。事成之后,保证擒侯景。”萧纲见其恳切,愈发生疑。结果事泄,范桃棒被杀,陈昕出城接应后也被擒杀。

侯景攻陷台城後,梁武帝於549年憂憤而死,但是侯景認為目前仍然不能自立為皇帝,便擁立蕭綱為皇帝,次年改元大寶。但是蕭綱不過是侯景的傀儡。551年,侯景派人廢蕭綱為晉安王,改立豫章王蕭棟為皇帝;蕭綱被囚禁於永福省,蕭綱被廢後兩個月,被侯景派人以棉被悶死,享年49歲。

侯景事後為蕭綱上諡號曰明皇帝,廟號高宗,梁元帝在552年追諡蕭綱為簡文皇帝,廟號太宗。

蕭綱本人文學造詣很高,雅好詩賦,有大量詠物、宮體、閨怨之作,其中五言詩最多,並且與蕭子显、蕭繹、徐擒、庾肩吾等人形成宮體詩流派,萧纲是宫体诗的代表。侯景攻入建康期间,曾经“募人出烧东宫,东宫台殿遂尽。所聚百橱图籍,一皆灰烬”。


\subsubsection{大宝}

\begin{longtable}{|>{\centering\scriptsize}m{2em}|>{\centering\scriptsize}m{1.3em}|>{\centering}m{8.8em}|}
  % \caption{秦王政}\
  \toprule
  \SimHei \normalsize 年数 & \SimHei \scriptsize 公元 & \SimHei 大事件 \tabularnewline
  % \midrule
  \endfirsthead
  \toprule
  \SimHei \normalsize 年数 & \SimHei \scriptsize 公元 & \SimHei 大事件 \tabularnewline
  \midrule
  \endhead
  \midrule
  元年 & 550 & \tabularnewline\hline
  二年 & 551 & \tabularnewline
  \bottomrule
\end{longtable}


%%% Local Variables:
%%% mode: latex
%%% TeX-engine: xetex
%%% TeX-master: "../../Main"
%%% End:
