%% -*- coding: utf-8 -*-
%% Time-stamp: <Chen Wang: 2021-11-01 15:05:38>

\subsection{武帝萧衍\tiny(502-549)}

\subsubsection{生平}

梁武帝萧衍(464年-549年),字叔达,小字练儿。南兰陵中都里人(今江苏常州市武进区西北)。南北朝時代南梁開国皇帝,廟號高祖。

萧衍是南齐宗室,亦是蘭陵蕭氏的世家子弟,出生在秣陵(今南京),父亲蕭順之是齐高帝的族弟,封临湘县侯,官至丹阳尹。母张尚柔。萧衍少年時受過良好的儒家教育,私德頗佳、亦不太注重個人享受,是文學名士竟陵八友之一。原為權臣,在其兄長蕭懿被害後,逐漸有帝位之野心,南齐中兴二年(502年),齐和帝被迫禅位于萧衍,南梁建立,是為梁武帝。稱帝後的萧衍改善許多前朝留下的弊政,並多次主持整理經史文書。然而晚年的他多次出家,傾力資助佛教發展直接導致國庫空虛,在侯景之乱爆發後絕食而亡。梁武帝萧衍在位时间長达48年,在南北朝皇帝中名列第一。

萧衍年轻時多才多藝,學識廣博。他的政治、軍事才能,在南朝諸帝中堪稱翹楚,不在另三位開國皇帝之下。在南齊武帝永明年間,他經常在當時的文化中心、竟陵王蕭子良的西邸出入,與沈約、謝脁等人合稱「竟陵八友」,在這期間發表了許多詩作,在學術研究和文學創作上皆有所成就。《梁書》紀載他:“六藝備閑,棋登逸品,陰陽緯候,卜筮占決,並悉稱善。……草隸尺牘,騎射弓馬,莫不奇妙。”他很好學,從小就受到正統的儒家教育,“少時習周孔,弱冠窮六經”,即位之後,“雖萬機多務,猶卷不輟手,燃燭側光,常至午夜”。

齐武帝驾崩时,萧衍没有参与王融意图拥立萧子良的政变,反支持皇太孙萧昭业登基。后又助权臣萧鸾篡位,是為齊明帝。齊明帝的皇叔荆州刺史萧子隆性温和、有文才,明帝欲徵之回朝,恐其不从。萧衍说:“随王(萧子隆)虽有美名,其实能力庸劣,手下没有智谋之士,爪牙只有司马垣历生、武陵太守卞白龙,而且二人唯利是从,若以显职相诱,都会来;随王只需要折简就能召来了。”齊明帝从之,徵垣历生为太子左卫率、卞白龙为游击将军,二人果然都到任。明帝再召萧子隆为侍中、抚军将军,后杀之。

齊明帝死後,繼任的東昏侯蕭寶卷暴虐無道,爆發的亂事在各地将帥們的努力下皆被平息,當中最為得力的是蕭衍的兄長、時任雍州(今湖北省襄陽)刺史的蕭懿。永元二年(500年),蕭懿被誣告謀反,遭東昏侯賜死,由蕭衍接任雍州刺史一職。喜好樂府詩的蕭衍上任後派人搜集當地的民歌,恢復自晉朝以來就已停止的民歌搜集工作。同時他積極招兵,暗中尋找機會推翻東昏侯。他秘密的派人在襄陽大伐竹木,沉於湖底,直到一年後舉兵之時,馬上派人去湖中打撈起事先砍伐好的竹木,並讓早已召集好的數千工匠在最短時間內建造戰船,此即後世 "伐竹沉木" 的典故。

中興元年(501年),蕭衍領兵攻郢城,圍攻兩百餘日,城破,「積屍床下而寢其上,比屋皆滿。」同年十二月,蕭衍發兵攻占首都建康,改立南康王蕭寶融於江陵稱帝,是為齊和帝;東昏侯在政變中被將軍王珍國所殺。中兴二年(502年),皇太后王寶明临朝称制,之後蕭衍受齊和帝禪讓登基,改國號為天監元年(502年),是為梁武帝。

梁武帝昔日的好友沈約、范雲等世族出身的名門後人在梁朝當上宰相,與前朝重臣蕭秀等人合力推動各種改革,改正南齊時施政上的種種問題。此外,武帝登基後對樂府詩的興趣不減當年,仍參與樂府詩的創作及編修。在他的影響和提倡下,梁朝文化的發展達到了東晉以來最繁榮的階段。《南史》作者李延壽評價道:“自江左以來,年逾二百,文物之盛,獨美於茲。”

520年,梁武帝改元普通,這一年被中國歷史學家視為南朝梁發展的分水嶺。在這年開始,梁武帝開始篤信佛法,多次舍身出家。

普通八年(527年)三月八日,梁武帝第一次前往同泰寺舍身出家,三日後返回,大赦天下,改年号大通,是為大通元年(527年)。同年,隸領軍曹仲宗伐渦陽(今安徽蒙城),在關中侯陳慶之的奮鬥下梁軍大敗北魏軍、俘斬甚眾,又乘勝進擊至城父。梁武帝詔下令渦陽之地設置西徐州,並以手詔嘉勉陳慶之:「本非將種,又非豪家,觖望風雲,以至於此。可深思奇略,善克令終。開朱門而待賓,揚聲名於竹帛,豈非大丈夫哉!」

大通三年(529年)九月十五日,梁武帝第二次至同泰寺举行“四部无遮大会”,脱下帝袍,换上僧衣,舍身出家,九月十六日讲解《涅槃经》。當月二十五日群臣捐錢一億,向“三宝”祷告,请求赎回“皇帝菩萨”,二十七日萧衍還俗。

梁武帝因天象称“荧惑入南斗,天子下殿走”,就赤脚下殿以应天象。之後傳來北魏孝武帝西奔的消息,得知此事的武帝羞惭地说道:“綁著辮子的胡虜(索虏)也应天象吗?”(由於天象应於北魏,意味天意认为北魏孝武帝才是正统天子)

大同十二年(546年)四月十日,蕭衍第三次出家,此次群臣用兩億錢將其贖回;太清元年(547年),三月三日蕭衍又第四次出家,在同泰寺住了三十七天,四月十日、朝廷出資一億錢贖回武帝。郭祖深形容:“都下佛寺五百餘所,窮極宏麗。僧尼十餘萬,資產豐沃。”。此時國力日衰。

侯景原为东魏的将领,由于他与东魏丞相高澄的矛盾,于太清元年(547年)正月据河南十三州叛归西魏,但西魏宇文泰对其不信任。迫于无奈,侯景致函萧衍,许愿献出河南十三州来投奔南朝梁。萧衍接纳了侯景,并任命他为大将军,封河南王。不久,东魏攻击侯景,萧衍派姪子萧渊明支援,结果战败,萧渊明被俘。侯景败退后占据寿阳。高澄假意提出和解,意在离间侯景和梁朝。司农卿傅岐认为高澄议和是离间之计。而朱异等人则极力主张与东魏和好。萧衍不听臣下劝告,与东魏使者往来,侯景感到恐慌。

此时,侯景假托东魏名义写信给萧衍,提出用萧渊明交换侯景,萧衍居然表示接受。侯景十分气愤,遂起兵叛变。他以萧正德为内应,轻易渡过了长江,并在公元549年三月围攻建康。城中久被围困,粮食断绝,饥病困扰,人多浮肿气急,横尸满路,能登城抗击者不到四千人。南梁诸王手握重兵,却彼此猜忌按兵不动,无人讨叛。十二日,侯景攻入建康,纵兵洗劫,蕭家宗室、世族琅琊王氏、陳郡謝氏皆遭血洗,史称侯景之亂。

城陷之后,侯景的武士隨意进出皇宫、甚至佩带武器。萧衍见了很奇怪,问左右侍从,侍从说是侯丞相的卫兵。萧衍生气地喝道:“甚么丞相!不就是侯景嗎!”侯景听说了,非常生气,于是派人监视萧衍,萧衍的饮食也被侯景裁减。萧衍忧愤成疾,口苦乾渴,索蜂蜜水,未得实现,怒憤更疾。

据说梁武帝曾经在志公禅师臨終時,向其询问自己寿命。志公说;「我的墓塔倒了,陛下的大限就到了。」志公涅槃後,寺方造了木製的靈塔,梁武帝担心志公的木造靈塔不坚固,就拆除,打算重建,拆了以后不久侯景之乱就发生了。

五月,萧衍在糧食尚足的情況下(身旁有數百顆雞蛋),因激憤不已,病卒於台城內,死前猶喊著軍事戰陣時的「荷,荷」(士兵先退後進的口號),表示他反擊侯景的志願。死时86岁,葬于修陵(今江苏丹阳市陵口)。谥号武帝,庙号高祖。

梁武帝除了帝王的身分,也身為學者在經、史、詩詞、佛學等領域留下大量著述而出名。

在经学方面,他撰有《周易讲疏》、《春秋答问》、《孔子正言》等二百余卷。天监十一年(512年),又制成吉、凶、军、宾、嘉五礼,共一千余卷,八千零十九条,颁布施行。

在史学方面,他不满《汉书》等断代史的写法,因而主持编撰了六百卷的《通史》,并“躬制赞序”。命殷芸将无法入史的剩余材料(主要是异闻杂谈),编入小说。这些著作大都没有流传下来。

在文学方面,梁武帝也非常喜欢诗赋創作,现存古詩、樂府詩等诗歌有80多首。蕭衍和王融、謝朓、任昉、沈約、范雲、蕭琛、陸倕七人共稱竟陵八友,在齊永明時代的文學界頗負盛名。

在宗教方面,今日漢傳佛教的素食主義即以梁武帝為首。佛教的梁皇寶懺是他編製成的,他又著有《大般涅槃經》、《大品般若經》、《淨名經》、《三慧經》等諸經義記數百卷。在道教学说中,他把儒家的“礼法”、道家的“无”和佛教“因果报应”揉合,创立了“三教同源说”,在中国古代思想史上占有极其重要的地位。由於梁武帝對佛教流通的貢獻,寺廟时以梁武帝與其長子昭明太子合祀為護法神。

梁武帝的学问路线,是先习儒,再奉道,后入佛。少年时代是习儒阶段,“少时学周孔,弱冠穷六经”。二十岁以后,改奉道教,一直到即位为帝后,仍未捨道。《隋书·经籍志》载,“武帝弱年好事,先受道法,及即位,犹自上章”。称帝后的萧衍和道士陶弘景的关系极善,他每当遇到国家大事,经常要派人到茅山去向陶弘景请教,以致于陶弘景有“山中宰相”之称。不过,在即位后的第二个年头,即天监三年(504),萧衍就颁布了《捨事道法詔》,宣布捨道归佛。而据其《述三教诗》,则称“晚年开释卷,犹月映众星”。到晚年才开始研读佛经。这也许说明,他虽然已经颁布了事佛诏,实际上还未真正彻底放弃道教。但总的来说,颁诏以后,他是以事佛为主的。有關《捨事道法詔》的真实性在学术界存疑,但无论其真伪,萧衍的奉佛则是事实。

梁武帝对佛教的支持,表现为两大方面:一是亲身修佛,二是从各方面扶持佛教的发展。

梁武帝本人归佛后,逐渐过上了佛教徒的生活。在武帝發表《斷酒肉文》前漢傳佛教「律中無有斷肉法」(反而是與釋迦佛作對的天啟,提倡素食),蕭衍把佛教五戒中的不殺生引申為素食,颁布了《断酒肉文》,禁止僧众吃肉,自己也行素食,開啟了漢傳佛教素食的傳統,之後漢傳佛教僧團開始遵守《梵網經》規定的菩薩戒,不再食肉。对那些敢于饮酒食肉的僧侶,他以世俗的刑法治罪。他又颁布《断杀绝宗庙牺牲诏》,禁止宗庙的牺牲,这是有违儒家祭祀禮儀的,但他坚持推行。他还正式受戒,据《续高僧传》卷六记载,他于天监十八年(519)“发宏誓心,受梵網經菩萨戒”。

梁武帝晚年奉佛更甚,经常日食一餐,過午不食,所食也只是豆羹、粗饭而已。篤信佛教,由於不近女色,曾經四十年無幸后宮,最突出的奉佛行为,是多次舍身出家,先后四次舍身同泰寺,每次都是朝廷花了大量的香火钱才把他赎出来還俗。他的第四次舍身是在太清元年(547)三月,历时一个月,所花赎钱为“一亿万”,这为同泰寺带来了巨额资金。

武帝本人是可以划入“义学”一类的,他对佛经很有研究,尤重《般若经》、《涅槃经》、《法华经》等,他常常为大家讲经说法,召开各种法会,开设过千僧会、无遮大会。中大通元年(529)开设的无遮大会,参加者有道俗五万多人。他的佛教撰述,则有《摩诃般若波罗蜜经注解》(现仅存序)、《三慧经义记》(《三慧经》本是《摩诃般若经》中的《三慧品》,萧衍认为此品最重要,因而独列為《三慧经》)、《制旨大涅槃经讲疏》、《净名经义记》、《制旨大集经讲疏》、《发般若经题论义并问答》(均佚),另著有《立神明成佛义记》、《敕答臣下神灭论》、《为亮法师制涅槃经疏序》、《断酒肉文》、《述三教诗》等,均存。

武帝在哲学上对中国佛教的贡献,突出之处是把中国传统的心性论、灵魂不灭论和佛教的涅槃佛性说结合起来了,他本人是属于涅槃学派的,主张“神明成佛”,所谓“神明”,是指永恒不灭的精神实体,它是众生成佛的内在根据,“神明”也就是佛性。他又提出三教同源论,认为儒、道二教同源于佛教,老子、孔子,都是释迦牟尼的弟子,所以从这个角度来看,三教可以会通,同时,三教的社会作用也是相同的,都是教化人为善。

除了自身奉佛,萧衍还大力扶持佛教事业的发展。他非常支持外僧的译经,僧伽婆罗被他召入五处译场从事译经,所译经典,又请宝唱等人写疏,他甚至“躬临法座,笔受其文,然后乃付译人”。真谛在萧衍门下也受到礼遇,只是因为侯景之乱,真谛的译事难申。萧衍和国内法師的关系也很密切,宝亮、智藏、法云、僧旻等人,都是萧衍非常器重的。他组织僧人编撰佛教著作,编成的作品至少有十二种。他还广造伽藍,所建有大爱敬寺、智度寺、光宅寺、同泰寺等十一座,各寺铸有佛像,大爱敬寺有金铜像,智度寺的正殿铸有金像,光宅寺有丈九无量寿佛铜像,同泰寺有十方银像。

禅宗祖师菩提达摩南北朝时期来中国弘法,与梁武帝会谈。但因理念不合,话不投机,离开梁朝而北去。

在梁武帝的支持下,梁代佛教达到了南朝佛教的最盛期,他最后在侯景之乱时,饥病交加,死于寺中。但武帝之后,梁简文帝和梁元帝也都篤信佛法。

蕭衍登位天子,民望所歸,敢革時政,頗得人心,初期國家興旺繁盛,為一明君。後期太過信仰宗教,企图以佛治民,學者有此評價:一,太過慈悲,不力法治,导致官吏貪污搜刮,百官“缘饰奸谄,深害时政”,奸邪小人纷纷以正人君子的面目出现,官场风气败坏,民间疾苦,国力衰败,而民怨终于为眼光锐利的侯景所利用。二,外交失敗,不能知人,“险躁之心,暮年愈甚”,导致侯景之乱,侯景之乱彻底打击了江南的经济基础、人口基础。

錢穆於《國史大綱》云:“獨有一蕭衍老翁,儉過漢文,勤如王莽,可謂南朝一令主。”
王夫之於《讀通鑑論》亦云:“梁氏享國五十年,天下且小康焉。”

\subsubsection{天监}

\begin{longtable}{|>{\centering\scriptsize}m{2em}|>{\centering\scriptsize}m{1.3em}|>{\centering}m{8.8em}|}
  % \caption{秦王政}\
  \toprule
  \SimHei \normalsize 年数 & \SimHei \scriptsize 公元 & \SimHei 大事件 \tabularnewline
  % \midrule
  \endfirsthead
  \toprule
  \SimHei \normalsize 年数 & \SimHei \scriptsize 公元 & \SimHei 大事件 \tabularnewline
  \midrule
  \endhead
  \midrule
  元年 & 502 & \tabularnewline\hline
  二年 & 503 & \tabularnewline\hline
  三年 & 504 & \tabularnewline\hline
  四年 & 505 & \tabularnewline\hline
  五年 & 506 & \tabularnewline\hline
  六年 & 507 & \tabularnewline\hline
  七年 & 508 & \tabularnewline\hline
  八年 & 509 & \tabularnewline\hline
  九年 & 510 & \tabularnewline\hline
  十年 & 511 & \tabularnewline\hline
  十一年 & 512 & \tabularnewline\hline
  十二年 & 513 & \tabularnewline\hline
  十三年 & 514 & \tabularnewline\hline
  十四年 & 515 & \tabularnewline\hline
  十五年 & 516 & \tabularnewline\hline
  十六年 & 517 & \tabularnewline\hline
  十七年 & 518 & \tabularnewline\hline
  十八年 & 519 & \tabularnewline
  \bottomrule
\end{longtable}

\subsubsection{普通}

\begin{longtable}{|>{\centering\scriptsize}m{2em}|>{\centering\scriptsize}m{1.3em}|>{\centering}m{8.8em}|}
  % \caption{秦王政}\
  \toprule
  \SimHei \normalsize 年数 & \SimHei \scriptsize 公元 & \SimHei 大事件 \tabularnewline
  % \midrule
  \endfirsthead
  \toprule
  \SimHei \normalsize 年数 & \SimHei \scriptsize 公元 & \SimHei 大事件 \tabularnewline
  \midrule
  \endhead
  \midrule
  元年 & 520 & \tabularnewline\hline
  二年 & 521 & \tabularnewline\hline
  三年 & 522 & \tabularnewline\hline
  四年 & 523 & \tabularnewline\hline
  五年 & 524 & \tabularnewline\hline
  六年 & 525 & \tabularnewline\hline
  七年 & 526 & \tabularnewline\hline
  八年 & 527 & \tabularnewline
  \bottomrule
\end{longtable}

\subsubsection{大通}

\begin{longtable}{|>{\centering\scriptsize}m{2em}|>{\centering\scriptsize}m{1.3em}|>{\centering}m{8.8em}|}
  % \caption{秦王政}\
  \toprule
  \SimHei \normalsize 年数 & \SimHei \scriptsize 公元 & \SimHei 大事件 \tabularnewline
  % \midrule
  \endfirsthead
  \toprule
  \SimHei \normalsize 年数 & \SimHei \scriptsize 公元 & \SimHei 大事件 \tabularnewline
  \midrule
  \endhead
  \midrule
  元年 & 527 & \tabularnewline\hline
  二年 & 528 & \tabularnewline\hline
  三年 & 529 & \tabularnewline
  \bottomrule
\end{longtable}

\subsubsection{中大通}

\begin{longtable}{|>{\centering\scriptsize}m{2em}|>{\centering\scriptsize}m{1.3em}|>{\centering}m{8.8em}|}
  % \caption{秦王政}\
  \toprule
  \SimHei \normalsize 年数 & \SimHei \scriptsize 公元 & \SimHei 大事件 \tabularnewline
  % \midrule
  \endfirsthead
  \toprule
  \SimHei \normalsize 年数 & \SimHei \scriptsize 公元 & \SimHei 大事件 \tabularnewline
  \midrule
  \endhead
  \midrule
  元年 & 529 & \tabularnewline\hline
  二年 & 530 & \tabularnewline\hline
  三年 & 531 & \tabularnewline\hline
  四年 & 532 & \tabularnewline\hline
  五年 & 533 & \tabularnewline\hline
  六年 & 534 & \tabularnewline
  \bottomrule
\end{longtable}

\subsubsection{大同}

\begin{longtable}{|>{\centering\scriptsize}m{2em}|>{\centering\scriptsize}m{1.3em}|>{\centering}m{8.8em}|}
  % \caption{秦王政}\
  \toprule
  \SimHei \normalsize 年数 & \SimHei \scriptsize 公元 & \SimHei 大事件 \tabularnewline
  % \midrule
  \endfirsthead
  \toprule
  \SimHei \normalsize 年数 & \SimHei \scriptsize 公元 & \SimHei 大事件 \tabularnewline
  \midrule
  \endhead
  \midrule
  元年 & 535 & \tabularnewline\hline
  二年 & 536 & \tabularnewline\hline
  三年 & 537 & \tabularnewline\hline
  四年 & 538 & \tabularnewline\hline
  五年 & 539 & \tabularnewline\hline
  六年 & 540 & \tabularnewline\hline
  七年 & 541 & \tabularnewline\hline
  八年 & 542 & \tabularnewline\hline
  九年 & 543 & \tabularnewline\hline
  十年 & 544 & \tabularnewline\hline
  十一年 & 545 & \tabularnewline\hline
  十二年 & 546 & \tabularnewline
  \bottomrule
\end{longtable}

\subsubsection{中大同}

\begin{longtable}{|>{\centering\scriptsize}m{2em}|>{\centering\scriptsize}m{1.3em}|>{\centering}m{8.8em}|}
  % \caption{秦王政}\
  \toprule
  \SimHei \normalsize 年数 & \SimHei \scriptsize 公元 & \SimHei 大事件 \tabularnewline
  % \midrule
  \endfirsthead
  \toprule
  \SimHei \normalsize 年数 & \SimHei \scriptsize 公元 & \SimHei 大事件 \tabularnewline
  \midrule
  \endhead
  \midrule
  元年 & 546 & \tabularnewline\hline
  二年 & 547 & \tabularnewline
  \bottomrule
\end{longtable}

\subsubsection{太清}

\begin{longtable}{|>{\centering\scriptsize}m{2em}|>{\centering\scriptsize}m{1.3em}|>{\centering}m{8.8em}|}
  % \caption{秦王政}\
  \toprule
  \SimHei \normalsize 年数 & \SimHei \scriptsize 公元 & \SimHei 大事件 \tabularnewline
  % \midrule
  \endfirsthead
  \toprule
  \SimHei \normalsize 年数 & \SimHei \scriptsize 公元 & \SimHei 大事件 \tabularnewline
  \midrule
  \endhead
  \midrule
  元年 & 547 & \tabularnewline\hline
  二年 & 548 & \tabularnewline\hline
  三年 & 549 & \tabularnewline
  \bottomrule
\end{longtable}


%%% Local Variables:
%%% mode: latex
%%% TeX-engine: xetex
%%% TeX-master: "../../Main"
%%% End:
