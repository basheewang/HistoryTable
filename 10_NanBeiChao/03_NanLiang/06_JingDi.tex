%% -*- coding: utf-8 -*-
%% Time-stamp: <Chen Wang: 2021-11-01 15:06:37>

\subsection{敬帝蕭方智\tiny(555-557)}

\subsubsection{生平}

梁敬帝蕭方智(543年-558年5月5日),字慧相,南梁的末代皇帝,梁元帝蕭繹的第九子。549年,蕭方智被封為興梁侯;552年被封為晉安王,553年被封為江州刺史。

當梁元帝在江陵被殺之時,555年陳霸先、王僧辯擁立蕭方智以太宰承制於建康,立为梁王,但是北齊將元帝堂兄貞陽侯蕭淵明送回之後,王僧辯因无力抵抗,又同意擁立蕭淵明為皇帝,但要求立萧方智为皇太子。萧渊明登基后,立萧方智为皇太子。陳霸先以王僧辩投降北齐、抛弃元帝子为由袭殺王僧辯,萧渊明亦退位,萧方智登基。557年,蕭方智禪位與陳霸先,南朝梁被南陳取代。

陳霸先封蕭方智為江陰王,永定二年四月乙丑(558年5月5日),陈武帝陈霸先派人杀死梁敬帝,陳霸先追諡曰敬帝,封梁武帝十弟鄱陽王蕭恢之孫蕭季卿為江陰王。

\subsubsection{绍泰}

\begin{longtable}{|>{\centering\scriptsize}m{2em}|>{\centering\scriptsize}m{1.3em}|>{\centering}m{8.8em}|}
  % \caption{秦王政}\
  \toprule
  \SimHei \normalsize 年数 & \SimHei \scriptsize 公元 & \SimHei 大事件 \tabularnewline
  % \midrule
  \endfirsthead
  \toprule
  \SimHei \normalsize 年数 & \SimHei \scriptsize 公元 & \SimHei 大事件 \tabularnewline
  \midrule
  \endhead
  \midrule
  元年 & 555 & \tabularnewline\hline
  二年 & 556 & \tabularnewline
  \bottomrule
\end{longtable}

\subsubsection{太平}

\begin{longtable}{|>{\centering\scriptsize}m{2em}|>{\centering\scriptsize}m{1.3em}|>{\centering}m{8.8em}|}
  % \caption{秦王政}\
  \toprule
  \SimHei \normalsize 年数 & \SimHei \scriptsize 公元 & \SimHei 大事件 \tabularnewline
  % \midrule
  \endfirsthead
  \toprule
  \SimHei \normalsize 年数 & \SimHei \scriptsize 公元 & \SimHei 大事件 \tabularnewline
  \midrule
  \endhead
  \midrule
  元年 & 556 & \tabularnewline\hline
  二年 & 557 & \tabularnewline
  \bottomrule
\end{longtable}

\subsection{东梁简介}

蕭莊(548年-577年),南兰陵郡兰陵县(今江苏省常州市武进区)人,梁元帝之孫,武烈世子蕭方等之子。

554年,西魏攻陷江陵、殺害梁元帝時,蕭莊只有七歲,逃匿於民家之中;之後被王琳發現,將蕭莊護送回建康。梁敬帝蕭方智即帝位之後,將蕭莊作為人質送往北齊。

557年,陳霸先廢蕭方智即帝位後,王琳等人要求北齊送還蕭莊,並使其接替南梁皇帝;蕭莊回到南朝之後,王琳立蕭莊為梁皇帝於郢州,據有長江中上游地區,是為東梁。之後蕭莊的南梁與陳霸先的陳朝便持續交戰,560年,當王琳與陳朝的侯瑱在蕪湖交戰時,北周便發兵攻打郢州,結果王琳兵敗,與蕭莊逃亡北齊。蕭莊被封为侯,又封梁王。北齐允諾幫他復興梁朝,但没能实现。齐后主高纬灭亡后,蕭莊在邺城闭气自杀。



%%% Local Variables:
%%% mode: latex
%%% TeX-engine: xetex
%%% TeX-master: "../../Main"
%%% End:
