%% -*- coding: utf-8 -*-
%% Time-stamp: <Chen Wang: 2021-11-01 15:06:11>

\subsection{淮陰王萧栋\tiny(551)}

\subsubsection{生平}

蕭棟(?-552年),字元吉,南朝梁朝的第三代皇帝。史稱豫章王、淮陰王。

蕭棟為昭明太子蕭統之孫,豫章王蕭歡之子。蕭統去世後,梁武帝曾經有一度想立蕭歡為皇太孫,但最後沒有,而改封蕭統三弟后来的梁簡文帝蕭綱當太子。萧欢死后,萧栋袭为豫章王。

侯景之乱期间,叛将侯景攻破梁都建康,将萧栋等在京宗室软禁。

551年,侯景被皇弟湘东王萧绎部将王僧辩所败,回到建康,图谋篡位。他以簡文帝名义下诏禅位给萧栋。当时京城一带饥荒,萧栋与王妃张氏正在园中种菜,看到来迎接他为帝的士兵,不知所措,哭着登车,被侯景立为皇帝,升武德殿。当时平地起风,将华盖吹翻直出端门,时人知道萧栋不能善终。改元天正。侯景掌权,萧栋毫无实权。他追封祖父萧统、父萧欢为皇帝,尊母王氏为皇太后,立王妃张氏为皇后。

两个半月後,因侯景所迫,萧栋加侯景殊礼、九锡,侯景不久廢蕭棟為淮陰王,並自立為漢皇帝,並將蕭棟與其弟蕭橋、蕭樛囚於密室之中。

萧绎登基为梁元帝并收復建業後,指使王僧辩杀萧栋,王僧辩拒绝。萧绎于是命宣猛将军朱买臣杀萧栋。当时,蕭棟與其弟都逃出密室,相扶而出,但仍戴着镣铐,遇到将军杜崱后才去除。弟弟们以为逃出生天,但萧栋认为未必,仍然害怕。不久,他们遇到朱买臣,朱买臣请他们上船饮酒,席未终,朱买臣所部士兵抓住他们,沉入扬子江。

\subsubsection{天正}

\begin{longtable}{|>{\centering\scriptsize}m{2em}|>{\centering\scriptsize}m{1.3em}|>{\centering}m{8.8em}|}
  % \caption{秦王政}\
  \toprule
  \SimHei \normalsize 年数 & \SimHei \scriptsize 公元 & \SimHei 大事件 \tabularnewline
  % \midrule
  \endfirsthead
  \toprule
  \SimHei \normalsize 年数 & \SimHei \scriptsize 公元 & \SimHei 大事件 \tabularnewline
  \midrule
  \endhead
  \midrule
  元年 & 551 & \tabularnewline
  \bottomrule
\end{longtable}


%%% Local Variables:
%%% mode: latex
%%% TeX-engine: xetex
%%% TeX-master: "../../Main"
%%% End:
