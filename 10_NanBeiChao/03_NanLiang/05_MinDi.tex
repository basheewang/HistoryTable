%% -*- coding: utf-8 -*-
%% Time-stamp: <Chen Wang: 2019-12-23 14:08:09>

\subsection{闵帝\tiny(555)}

\subsubsection{生平}

梁閔帝蕭淵明(5世紀?-556年6月2日),字靖通,蕭懿之子,梁武帝蕭衍之姪。

蕭淵明最早封貞陽侯,後擔任豫州刺史;在侯景背叛東魏投降南梁之時,蕭衍命蕭淵明與侯景北伐攻打東魏,結果蕭淵明兵敗被俘。侯景曾伪造东魏书信称愿意放还萧渊明交换侯景,萧衍回信答应“贞阳朝至,侯景夕返”,侯景遂骂萧衍薄心肠而作乱。

後在西魏攻陷江陵殺害梁孝元帝時,王僧辩、陈霸先意图拥立元帝子萧方智为帝,立萧方智为梁王;北齊文宣帝高洋與上黨王高渙於555年送回蕭淵明,準備讓蕭淵明成為北齊支持的傀儡皇帝,王僧辩未能抵抗齐军,于是接受萧渊明,但要求立萧方智为皇太子;蕭淵明於是即帝位,改年號為天成,并立萧方智为皇太子。不久後陳霸先诛王僧辯,並且廢蕭淵明,改立蕭方智為皇帝。蕭淵明改封建安公。

之後,北齊要求南梁送回蕭淵明,陳霸先也準備將蕭淵明送還北齊。但還沒有出發蕭淵明便病故。蕭莊在北齊支持下稱帝之後,諡蕭淵明為閔皇帝。

\subsubsection{天成}

\begin{longtable}{|>{\centering\scriptsize}m{2em}|>{\centering\scriptsize}m{1.3em}|>{\centering}m{8.8em}|}
  % \caption{秦王政}\
  \toprule
  \SimHei \normalsize 年数 & \SimHei \scriptsize 公元 & \SimHei 大事件 \tabularnewline
  % \midrule
  \endfirsthead
  \toprule
  \SimHei \normalsize 年数 & \SimHei \scriptsize 公元 & \SimHei 大事件 \tabularnewline
  \midrule
  \endhead
  \midrule
  元年 & 555 & \tabularnewline
  \bottomrule
\end{longtable}


%%% Local Variables:
%%% mode: latex
%%% TeX-engine: xetex
%%% TeX-master: "../../Main"
%%% End:
