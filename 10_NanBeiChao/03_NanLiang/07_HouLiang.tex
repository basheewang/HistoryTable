%% -*- coding: utf-8 -*-
%% Time-stamp: <Chen Wang: 2019-12-23 14:25:17>

\subsection{后梁(555-587)}

\subsubsection{简介}

後梁(555年-587年)為中國南北朝時期南梁在蕭詧稱帝後殘存的政权。國都於江陵,统治地区位于原梁朝國都建康的西边,故又稱為西梁。

在南梁末年的侯景之乱中,梁武帝孙、昭明太子萧统子岳阳王萧詧为叔父湘东王萧绎所迫投降西魏,封梁王;萧绎后来登基,史称梁元帝。554年西魏攻陷江陵后,梁元帝投降,被萧詧杀死。萧詧於555年称帝,並且對西魏稱臣;但是後梁由於國土狹小,屬地僅有江陵附近數縣八百里地,先後是西魏、北周和隋朝的附庸。但後梁也一直自居為南朝正統而與陳朝對立。而後梁也由於承續南梁的文化,而成為具有高度文化的國家。

後梁共傳宣帝蕭詧、明帝蕭巋、後主蕭琮三世。587年九月,隋文帝廢除後梁,改蕭琮為莒國公,後梁因此滅亡,存在共三十三年。然而由於蕭氏歷代事奉北周、對隋朝甚為恭謹,蕭巋之女還被隋文帝选为晋王妃即后来隋煬帝楊廣之皇后(炀愍皇后),因此在後梁廢除後,蕭氏在隋朝中央朝廷與江陵仍保有一定的政治影響力。隋末群雄之一的蕭銑即為蕭巋弟蕭巖之孫。

\subsubsection{宣帝}

梁宣帝蕭\xpinyin*{詧}(519年-562年),字理孫,梁武帝之孫、昭明太子蕭統之第三子。為後梁(西梁)的建立者,諡號宣皇帝,廟號中宗。

萧统死后,梁武帝立萧纲为皇太子,而进封萧统诸子为王。蕭詧被封為岳陽郡王並被任命為東揚州刺史,鎮守會稽。但他们仍然因未被立储而对梁武帝怀恨。

546年十月,萧詧改任雍州刺史,鎮守襄陽。549年,侯景之乱时,蕭詧兄長湘州刺史河東王蕭譽被他們的叔父湘東王(之後的梁元帝)蕭繹攻擊,邵陵王萧纶劝阻无果,蕭詧試圖救援蕭譽兵敗,部将杜岸叛变反以五百骑攻打襄阳,被留守的蔡大宝和萧詧母龚保林守御而未果。萧詧连夜撤回襄阳,龚保林不知他兵败,以为他是敌军,直到天亮了认出他才放他入城。萧詧自知得罪萧绎,为了自保,據襄陽歸降西魏,西魏於550年三月封蕭詧為梁王。萧纶被西魏军所杀,萧詧葬之。554年冬西魏派大将于谨等攻打江陵,梁元帝開門投降,被蕭詧侮辱后以土袋悶死。

之後西魏於555年在江陵立蕭詧為梁皇帝,年號大定。萧詧部将尹德毅劝他趁西魏人贪婪多有杀伤之机,趁西魏精锐尽在此时设宴,埋伏武士杀死于谨等人,再分头命令果决勇敢的人奇袭魏军营垒,全歼魏军,抚慰江陵百姓,任命文武百官,以救命之恩获得百姓支持,还可以写信招揽王僧辩等人,着朝服、渡长江,登基称帝,继承尧、禹之业。萧詧却说:“您的这条计策,并不是不好。可是魏人待我十分宽厚,我不能违背道德。如果仓促之间依计而行,就会像邓祁侯说的那样,我不会有好下场了。”

果然,西魏除江陵附近八百里之地外,將襄陽等地皆併入西魏,並且將江陵一帶的人民財產擄掠一空,萧詧所辖只有江陵周边八百里。萧詧追悔莫及,见屋宇残破,战乱不息,为自己威略不振而感到羞耻,心中常怀忧愤,于是作《愍时赋》自抒其意。每每读到“老马伏枥,志在千里,烈士暮年,壮心不已”就扬眉举目,握腕激奋,久久叹息不止。即位八年後,562年,蕭詧在抑鬱中病故。

\subsubsection{大定}

\begin{longtable}{|>{\centering\scriptsize}m{2em}|>{\centering\scriptsize}m{1.3em}|>{\centering}m{8.8em}|}
  % \caption{秦王政}\
  \toprule
  \SimHei \normalsize 年数 & \SimHei \scriptsize 公元 & \SimHei 大事件 \tabularnewline
  % \midrule
  \endfirsthead
  \toprule
  \SimHei \normalsize 年数 & \SimHei \scriptsize 公元 & \SimHei 大事件 \tabularnewline
  \midrule
  \endhead
  \midrule
  元年 & 555 & \tabularnewline\hline
  二年 & 556 & \tabularnewline\hline
  三年 & 557 & \tabularnewline\hline
  四年 & 558 & \tabularnewline\hline
  五年 & 559 & \tabularnewline\hline
  六年 & 560 & \tabularnewline\hline
  七年 & 561 & \tabularnewline\hline
  八年 & 562 & \tabularnewline
  \bottomrule
\end{longtable}


\subsubsection{明帝}

梁明帝蕭巋(542年-585年),字仁遠,是南北朝時代西梁的第二位君主。正式諡號為「孝明皇帝」,後世比照漢朝和西晉皇帝省略「孝」字,稱「梁明帝」。

西梁是南梁的一個分裂王朝,它的地盤主要在今天湖北襄陽、荊州地區,首都江陵(今湖北省荊州市)。蕭巋的父親蕭詧與梁元帝蕭繹不和,蕭繹繼梁帝位後,蕭詧就投靠西魏,被西魏皇帝封為梁王,在他的統治地盤內他自稱皇帝,但實際上西梁的「皇帝」在他們的領土上並沒有真正的主權,很長時間裡北朝在西梁設有江陵總管,一方面用來監督西梁的君主,另一方面這些總管擁有兵權來保護西梁不被南朝攻擊。蕭詧死後他的兒子蕭巋於562年以皇太子身份繼帝位。

蕭巋的年號是天保,他繼續他父親的政策,聯合北朝(北周)來抵抗南朝(南陳)的威脅。北周武帝宇文邕滅北齊後蕭巋親自赴長安祝賀,因此深得宇文邕的信任。隋文帝楊堅登基後再次親自赴長安祝賀,又贏得了楊堅的信任。後來蕭、楊兩家又通婚,蕭巋的一個女兒還嫁給了楊廣,後來成為隋煬帝的皇后。由於蕭、楊兩家的關係如此親密,因此後來隋將它駐扎在西梁的江陵總管撤回,使得西梁獲得了自主權。

蕭巋是一個相當有學問的皇帝,他曾著《孝經》、《周易義記》、《大小乘幽微》等十四部書。

蕭巋於天保二十四年(585年)五月逝世,諡為孝明皇帝,廟號世宗。

\subsubsection{天保}

\begin{longtable}{|>{\centering\scriptsize}m{2em}|>{\centering\scriptsize}m{1.3em}|>{\centering}m{8.8em}|}
  % \caption{秦王政}\
  \toprule
  \SimHei \normalsize 年数 & \SimHei \scriptsize 公元 & \SimHei 大事件 \tabularnewline
  % \midrule
  \endfirsthead
  \toprule
  \SimHei \normalsize 年数 & \SimHei \scriptsize 公元 & \SimHei 大事件 \tabularnewline
  \midrule
  \endhead
  \midrule
  元年 & 562 & \tabularnewline\hline
  二年 & 563 & \tabularnewline\hline
  三年 & 564 & \tabularnewline\hline
  四年 & 565 & \tabularnewline\hline
  五年 & 566 & \tabularnewline\hline
  六年 & 567 & \tabularnewline\hline
  七年 & 568 & \tabularnewline\hline
  八年 & 569 & \tabularnewline\hline
  九年 & 570 & \tabularnewline\hline
  十年 & 571 & \tabularnewline\hline
  十一年 & 572 & \tabularnewline\hline
  十二年 & 573 & \tabularnewline\hline
  十三年 & 574 & \tabularnewline\hline
  十四年 & 575 & \tabularnewline\hline
  十五年 & 576 & \tabularnewline\hline
  十六年 & 577 & \tabularnewline\hline
  十七年 & 578 & \tabularnewline\hline
  十八年 & 579 & \tabularnewline\hline
  十九年 & 580 & \tabularnewline\hline
  二十年 & 581 & \tabularnewline\hline
  二一年 & 582 & \tabularnewline\hline
  二二年 & 583 & \tabularnewline\hline
  二三年 & 584 & \tabularnewline\hline
  二四年 & 585 & \tabularnewline
  \bottomrule
\end{longtable}


\subsubsection{后主}

蕭琮(558年-607年),字溫文,為西梁明帝蕭巋之子,西梁第三位,亦是末代皇帝。

蕭琮最早封東陽王,後被立為皇太子。蕭琮博學有才,善於弓馬,個性倜儻不羈。585年即位為西梁皇帝,改年號為廣運。蕭琮即位之後,隋文帝設立江陵總管監視蕭琮的行為;587年,因蕭琮的叔父蕭巖等人帶了一部分居民逃入陳朝,隋文帝徵召蕭琮入朝,当年10月26日廢除西梁國,蕭琮亦被廢為莒國公。西梁也因此滅亡。

蕭琮在隋朝時仍然受到器重,隋煬帝即位後,因为萧琮是自己的妻兄,对萧琮更厚待,又封蕭琮為梁公、內史令,蕭琮的親族也有不少被提拔入朝廷為官。當時有童謠說:「蕭蕭亦復起」,導致隋煬帝對蕭琮的猜忌,最後蕭琮被免職,不久後在家中過世。

蕭琮死後被贈左光祿大夫,侄萧钜续封为梁公,史书没有记载萧琮子萧铉为何没有袭爵。

隋末割據勢力之一的蕭銑,為蕭琮之堂姪,並在稱帝之後追諡蕭琮為孝靖皇帝。

\subsubsection{广运}

\begin{longtable}{|>{\centering\scriptsize}m{2em}|>{\centering\scriptsize}m{1.3em}|>{\centering}m{8.8em}|}
  % \caption{秦王政}\
  \toprule
  \SimHei \normalsize 年数 & \SimHei \scriptsize 公元 & \SimHei 大事件 \tabularnewline
  % \midrule
  \endfirsthead
  \toprule
  \SimHei \normalsize 年数 & \SimHei \scriptsize 公元 & \SimHei 大事件 \tabularnewline
  \midrule
  \endhead
  \midrule
  元年 & 586 & \tabularnewline\hline
  二年 & 587 & \tabularnewline
  \bottomrule
\end{longtable}


%%% Local Variables:
%%% mode: latex
%%% TeX-engine: xetex
%%% TeX-master: "../../Main"
%%% End:
