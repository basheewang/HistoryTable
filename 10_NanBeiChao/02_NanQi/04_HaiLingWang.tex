%% -*- coding: utf-8 -*-
%% Time-stamp: <Chen Wang: 2021-11-01 15:05:05>

\subsection{海陵王蕭昭文\tiny(494)}

\subsubsection{生平}

蕭昭文(480年-494年),字季尚,南朝齊的第四任皇帝,在位僅四個月。母亲为宫人许氏。

蕭昭文為文惠太子蕭長懋的第二子,永明四年(486年)閏正月封為臨汝公,邑千五百户。初为辅国将军、济阳太守。十年(492年)正月,转持节、督南豫州诸军事、南豫州刺史,将军如故。十一年(493年),进号冠军将军。493年萧昭文兄长鬱林王蕭昭業即位後,十月封昭文為新安王。

隆昌元年(494年)闰四月,以萧昭文为扬州刺史。七月,尚书令、镇军大将军、西昌侯蕭鸞刺死萧昭业,拥立萧昭文為帝,改年號為延興,大赦。但是政事俱操於蕭鸞之手,萧昭文的生母许氏也没有得到尊封。

萧昭文刚登基,萧鸾就被任为骠骑大将军、录尚书事、扬州刺史、宣城郡公。九月,萧鸾以萧昭文之名诛杀高帝、武帝诸子,先杀司徒鄱阳王萧锵、中军大将军随郡王萧子隆、南兖州刺史安陆王萧子敬。江州刺史晋安王萧子懋起兵,萧鸾假黄钺,萧子懋败亡,萧鸾又杀湘州刺史南平王萧锐、郢州刺史晋熙王萧銶、南豫州刺史宜都王萧铿。十月,萧鸾被进为太傅,领大将军、扬州牧,加殊礼,进爵为王。萧鸾又杀中军将军桂阳王萧铄、抚军将军衡阳王萧钧、侍中秘书监江夏王萧锋、镇军将军建安王萧子真、左将军巴陵王萧子伦、司徒庐陵王萧子卿,且几乎杀死萧昭文的弟弟荆州刺史萧昭秀。这时萧鸾已掌控萧昭文的起居,有一次萧昭文想吃蒸鱼菜,太官令答没有萧鸾的命令,不给他吃。萧鸾诸子年幼,于是任兄子萧遥光、萧遥欣、萧遥昌以要职。当月,萧昭文被蕭鸞以嫡母皇太后王宝明名义以有病为由廢黜為海陵王,以东汉东海王劉彊之礼安置,给虎贲、旄头、画轮车,供奉很厚。次月,萧昭文便被蕭鸞派去看病的医生殺害,谥号恭。萧鸾以劉彊之礼厚葬,但并没有用皇帝礼。北魏趁机大举入侵。

\subsubsection{延兴}

\begin{longtable}{|>{\centering\scriptsize}m{2em}|>{\centering\scriptsize}m{1.3em}|>{\centering}m{8.8em}|}
  % \caption{秦王政}\
  \toprule
  \SimHei \normalsize 年数 & \SimHei \scriptsize 公元 & \SimHei 大事件 \tabularnewline
  % \midrule
  \endfirsthead
  \toprule
  \SimHei \normalsize 年数 & \SimHei \scriptsize 公元 & \SimHei 大事件 \tabularnewline
  \midrule
  \endhead
  \midrule
  元年 & 494 & \tabularnewline
  \bottomrule
\end{longtable}


%%% Local Variables:
%%% mode: latex
%%% TeX-engine: xetex
%%% TeX-master: "../../Main"
%%% End:
