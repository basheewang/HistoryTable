%% -*- coding: utf-8 -*-
%% Time-stamp: <Chen Wang: 2021-11-01 15:04:36>

\subsection{高帝萧道成\tiny(479-482)}

\subsubsection{生平}

齐高帝萧道成(427年-482年4月11日),字绍伯,南北朝時代,南朝第二個皇朝南齊开国皇帝。

蕭道成出自兰陵萧氏,父親蕭承之仕於劉宋為右將軍,蕭道成亦在劉宋擔任軍官,宋明帝駕崩,蕭道成以右衛將軍領衛尉的名銜,與其他數位大臣受遺詔掌機要,為輔政大臣。劉昱即位皇帝後,桂陽王兼江州刺史劉休範叛變,為蕭道成領軍所平定,權勢日隆。

但青春期的劉昱喜好弓馬,時常殺人取樂,又濫殺無辜,一次突然跑到蕭道成家中,道成是個大胖子,「方坦而晝臥,腹大如瓠」。劉昱玩心頓起,覺得這麼大的肚子是個絕好的箭靶,拿起箭來就「彎弓欲射其腹」。蕭道成苦苦哀求,左右隨從也紛紛勸解說:蕭大人肚子這麼大,這麼好的目標,一箭就射死太可惜了!以後想射就沒有了!好說歹說之下,劉昱答應去掉箭鏃才射,結果一箭射中肚臍,歡呼高歌而去。剩下嚇得一身冷汗,死裡逃生心有餘悸的權臣蕭道成。

477年,劉昱被蕭道成的黨羽楊玉夫所弒,蕭道成改立宋順帝劉準,獨攬朝政。並在同年和隔年(478年),分別消滅了忠於宋室的尚書令袁粲、荊州刺史沈攸之,宰制全國。

479年,蕭道成篡宋自立為天子,國號齊,他為政務節儉,實施檢籍政策,清查詐入士族籍貫的寒人。

與南朝宋的劉裕一樣,在位僅四年去世,終年56歲。

除了在政治上的功業,蕭道成也廣覽經學、史學書籍,善寫作文、書法和下棋。

北宋的司馬光評論他:「高帝以功名之盛,不容於昏暴之朝,逆取而順守之,亦一時之良主也。」

\subsubsection{建元}

\begin{longtable}{|>{\centering\scriptsize}m{2em}|>{\centering\scriptsize}m{1.3em}|>{\centering}m{8.8em}|}
  % \caption{秦王政}\
  \toprule
  \SimHei \normalsize 年数 & \SimHei \scriptsize 公元 & \SimHei 大事件 \tabularnewline
  % \midrule
  \endfirsthead
  \toprule
  \SimHei \normalsize 年数 & \SimHei \scriptsize 公元 & \SimHei 大事件 \tabularnewline
  \midrule
  \endhead
  \midrule
  元年 & 479 & \tabularnewline\hline
  二年 & 480 & \tabularnewline\hline
  三年 & 481 & \tabularnewline\hline
  四年 & 482 & \tabularnewline
  \bottomrule
\end{longtable}


%%% Local Variables:
%%% mode: latex
%%% TeX-engine: xetex
%%% TeX-master: "../../Main"
%%% End:
