%% -*- coding: utf-8 -*-
%% Time-stamp: <Chen Wang: 2019-12-20 14:49:17>

\subsection{鬱林王\tiny(493-494)}

\subsubsection{生平}

蕭昭業(473年-494年9月7日),字元尚,小名法身,南朝齊的第三任皇帝,文惠太子蕭長懋之長子,齐武帝之孙。

蕭昭業雖然工於隸書,美容止而獲得祖父與父親的喜愛,但是蕭昭業本人是一個陽奉陰違的人。父亲病重时,他在人前表现出悲伤得连自己的健康都受损的样子,见者无不动容,而一旦回家,即表现出喜状,还要一位杨姓巫婆诅咒他的父亲和祖父,以让自己可以尽早登极。萧长懋不久死去,萧昭业认为杨氏诅咒得力,予以赏赐。武帝对此全然不觉,立蕭昭業為皇太孫。武帝不久也病倒了,萧昭业继续在人前作悲伤状,实则兴奋,在给妻何婧英的信中写了一个大“喜”字,还在其周围写了36个小“喜”字。

齐武帝病重时,和武帝子竟陵王蕭子良相好的中书郎王融想改立萧子良为新君,取代萧昭业,被武帝堂弟西昌侯蕭鸞挫败。493年末齊武帝死後,蕭昭業即位,改年號為隆昌。同時由蕭子良與蕭鸞輔政。

萧昭业尊母王宝明为皇太后,封妻何婧英为皇后。他即位之後便原型畢露,不但濫發賞賜,又與庶母霍氏通姦,並且過著十分浪費奢靡的生活,毫無一國之君的姿態,并架空涉嫌夺位的萧子良,赐死王融,朝政都委託西昌侯蕭鸞處理。还在丧期,就恢复奏乐。宠幸偏爱中书舍人綦毋珍之、朱隆之、直将军曹道刚、周奉叔、宦官徐龙驹等人,纵容近臣公然卖官。萧昭业曾不满萧鸾专权,对徐龙驹说:“我和萧锵(萧昭业叔祖父,鄱阳王)商议杀萧鸾,萧锵不同意,只能先让萧鸾专权一阵子了。”萧鸾迫使萧昭业下令诛杀皇后何婧英的男宠杨珉及徐龙驹;将军萧谌、萧坦之秘密投靠萧鸾,将萧昭业的动向告诉他;萧鸾又设计寻罪名杀死萧昭业近臣周奉叔、杜文谦、綦毋珍之;而萧昭业却因萧子良去世而放松警惕。

萧昭业与何婧英的叔父何胤图谋杀萧鸾,但何胤不敢。萧昭业不再委萧鸾以重任。萧鸾担心生变,与萧谌、萧坦之合谋政变,由二人等人于省诛杀曹道刚、朱隆之等,亲自率兵自尚书省入云龙门進宮。萧昭业正在寿昌殿和霍氏裸体相对,闻变下令关闭宫门,派宦官登兴光楼察看。他不知萧谌、萧坦之已叛变,向萧谌求助,见萧谌杀入殿中,才知。宿卫将士准备抵抗萧谌时,萧昭业逃向爱姬徐氏房,拔剑自刺,不果,以帛缠颈,乘舆出延德殿,宿卫见状要护驾,萧昭业却没有发话,出西弄,被萧谌追上弒殺,尸体运到徐龙驹府中。萧鸾以太后名义下诏追廢蕭昭業為鬱林王,葬以亲王礼。

\subsubsection{隆昌}

\begin{longtable}{|>{\centering\scriptsize}m{2em}|>{\centering\scriptsize}m{1.3em}|>{\centering}m{8.8em}|}
  % \caption{秦王政}\
  \toprule
  \SimHei \normalsize 年数 & \SimHei \scriptsize 公元 & \SimHei 大事件 \tabularnewline
  % \midrule
  \endfirsthead
  \toprule
  \SimHei \normalsize 年数 & \SimHei \scriptsize 公元 & \SimHei 大事件 \tabularnewline
  \midrule
  \endhead
  \midrule
  元年 & 494 & \tabularnewline
  \bottomrule
\end{longtable}


%%% Local Variables:
%%% mode: latex
%%% TeX-engine: xetex
%%% TeX-master: "../../Main"
%%% End:
