%% -*- coding: utf-8 -*-
%% Time-stamp: <Chen Wang: 2021-11-01 15:05:29>

\subsection{和帝蕭寶融\tiny(501-502)}

\subsubsection{生平}

齊和帝蕭寶融(488年-502年),字智昭,南齊的末代皇帝,齊明帝蕭鸞第八子,母萧鸾原配追赠皇后刘惠端。

494年十一月被封為隨郡王,499年正月改封為南康王並任荊州刺史,駐守江陵。

501年三月,蕭衍發兵攻打蕭寶卷,並且与行荆州府事萧颖胄合谋,让萧宝融自称相国,后又立蕭寶融為皇帝,以萧宝融名义废萧宝卷为庶人,萧颖胄辅佐萧宝融守江陵,萧衍东征。萧衍手下劝他把萧宝融带到襄阳以免便宜了挟天子以令诸侯的萧颖胄,萧衍拒绝,认为此次起兵如果失败了此举就没有意义,如果成功了再寻找夺权的机会也不迟。后来萧颖胄病死,萧衍成为反对萧宝卷的唯一首领。蕭衍進入建康後,便將蕭寶融於502年接入建康,在萧宝融到来前奉废帝萧昭业的母后王宝明称制。同年,王宝明封蕭衍為梁王,不久蕭衍以王宝明名義殺害湘東王蕭寶晊兄弟,後來又殺掉齊明帝其他的兒子。不久蕭寶融便在到达建康前在姑孰被迫奉王宝明诏令禪位予蕭衍,南齊到此滅亡。

蕭衍即位後封蕭寶融為巴陵王,在姑孰建立宮室供其居住;第二天蕭寶融就被蕭衍所殺。

\subsubsection{中兴}

\begin{longtable}{|>{\centering\scriptsize}m{2em}|>{\centering\scriptsize}m{1.3em}|>{\centering}m{8.8em}|}
  % \caption{秦王政}\
  \toprule
  \SimHei \normalsize 年数 & \SimHei \scriptsize 公元 & \SimHei 大事件 \tabularnewline
  % \midrule
  \endfirsthead
  \toprule
  \SimHei \normalsize 年数 & \SimHei \scriptsize 公元 & \SimHei 大事件 \tabularnewline
  \midrule
  \endhead
  \midrule
  元年 & 501 & \tabularnewline\hline
  二年 & 502 & \tabularnewline
  \bottomrule
\end{longtable}


%%% Local Variables:
%%% mode: latex
%%% TeX-engine: xetex
%%% TeX-master: "../../Main"
%%% End:
