%% -*- coding: utf-8 -*-
%% Time-stamp: <Chen Wang: 2019-12-20 14:52:49>

\subsection{萧宝卷\tiny(498-501)}

\subsubsection{生平}

蕭寶卷(483年-501年),字智藏,原名蕭明賢,南齊的第六代皇帝,因殘暴荒淫被殺,後追封東昏侯;為齊明帝蕭鸞第二子。蕭寶卷也被認為是中國歷史上昏庸荒淫的皇帝之一。

蕭寶卷的生母劉惠端是蕭鸞的正妻,早亡,由潘妃撫養。他年少時不喜讀書,以捕老鼠為樂。498年,蕭寶卷在蕭鸞死後即位,這時蕭寶卷才十六歲,並且封潘妃之侄女潘玉兒為貴妃,潘貴妃生下一個女兒,封為公主,但公主卻百日而殤,蕭寶卷制斬衰絰杖,積旬不聽音樂,衣悉粗布。

蕭寶卷性格內向,很少說話,不喜歡跟大臣接觸,常常出宮閒逛,每次出遊都一定要拆毀民居、驅逐居民,並且興建仙華、神仙、玉壽諸殿,並且大量賞賜臣下,造成國家的財政困難。而且蕭寶卷也殺害不少的大臣,即位之後便殺害六位顧命大臣表叔右僕射江祏、侍中江祀、堂兄始安王蕭遙光(作乱事败被杀)、司空徐孝嗣、右將軍蕭坦之、舅父領軍將軍劉暄及重臣曹虎、沈文季等人。也由於蕭寶卷的昏暴,導致大将裴叔业以重镇寿阳投魏;太尉陳顯達與將軍崔慧景先後起兵叛亂,但都兵敗被殺。萧宝卷胞弟江夏王萧宝玄响应崔慧景,几乎被其拥立为帝,萧宝卷平乱后杀萧宝玄。

蕭寶卷平定叛亂之後更加昏暴,除了與潘玉奴、宦官梅蟲兒等人日夜玩樂之外,並且派人毒殺平定崔慧景叛亂最力的尚書令蕭懿,結果導致蕭懿之弟蕭衍發兵進攻建康。萧衍並且与荆州行府事萧颖胄合作,改立皇弟荆州刺史南康王蕭寶融於江陵稱帝,遥废萧宝卷为庶人,又封涪陵王;蕭寶卷就在蕭衍發兵進攻建康的動亂中,被將軍王珍國所殺。

之後蕭寶卷被追廢為庶人,有司请求追封其为零阳侯,不许,请求追封涪陵王,获准。后蕭衍又將其追降為東昏侯,谥号炀。

\subsubsection{永元}

\begin{longtable}{|>{\centering\scriptsize}m{2em}|>{\centering\scriptsize}m{1.3em}|>{\centering}m{8.8em}|}
  % \caption{秦王政}\
  \toprule
  \SimHei \normalsize 年数 & \SimHei \scriptsize 公元 & \SimHei 大事件 \tabularnewline
  % \midrule
  \endfirsthead
  \toprule
  \SimHei \normalsize 年数 & \SimHei \scriptsize 公元 & \SimHei 大事件 \tabularnewline
  \midrule
  \endhead
  \midrule
  元年 & 499 & \tabularnewline\hline
  二年 & 500 & \tabularnewline\hline
  三年 & 501 & \tabularnewline
  \bottomrule
\end{longtable}


%%% Local Variables:
%%% mode: latex
%%% TeX-engine: xetex
%%% TeX-master: "../../Main"
%%% End:
