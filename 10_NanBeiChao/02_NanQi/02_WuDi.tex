%% -*- coding: utf-8 -*-
%% Time-stamp: <Chen Wang: 2019-12-20 14:48:36>

\subsection{武帝\tiny(482-493)}

\subsubsection{生平}

齊武帝蕭\xpinyin*{賾}(440年-493年8月27日),字宣遠,齊高帝蕭道成長子,母刘智容。南齊第二任皇帝,病死時54歲,葬景安陵。年号永明。

武帝十分關心百姓疾苦,即位後,就下詔說:“比歲未稔,貧窮不少,京師二岸,多有其弊。遣中書舍人優量賑恤。”不久,再次下詔說,“水雨頻降,潮流薦滿,二岸居民,多所淹漬。遣中書舍人與兩縣官長優量賑恤。”

第二年,他又下詔酌情遣返軍中的囚徒,大赦囚犯,對於百姓中的鰥寡和貧窮之人,要加以賑濟。他提倡並獎勵農桑,災年時,還減免租稅。在位第四年,他下詔說: “揚、南徐二州,今年戶租三分二取見布,一分取錢。來歲以後,遠近諸州輸錢處,並減布直,匹準四百,依舊折半,以為永制。”

武帝還下令多辦學校,挑選有學問之人任教,以培育人們的德行。武帝以富國為先,不喜歡遊宴、奢靡之事,提倡節儉。他曾下令舉辦婚禮時不得奢侈。

齐武帝登基后,延續其父蕭道成的檢籍政策。那些被认为有假的户籍,都须退还本地,称为“却籍”。核查出本应服役纳赋而户籍上造假的,便恢复原来的户籍,继续承担赋役,称为“正籍”。檢籍政策虽然增加了赋税,却严重伤害了庶族地主的利益。

公元485年富阳唐寓之为此起兵叛乱,虽然这次叛乱被齐武帝迅速平息,但检籍的政策依然受到庶族的激烈反对。最终,在490年,齐武帝被迫妥协,宣布“却籍”无效,对因为“却籍”而被发配戍边的人民准许返归故乡,恢复刘宋升明时期户籍所注的原状。

武帝對於其後事,特意下詔說:“我識滅之後,身上著夏衣,畫天衣,純烏犀導,應諸器悉不得用寶物及織成等,唯裝復裌衣各一通。常所服身刀長短二口鐵環者,隨我入梓宮。祭敬之典,本在因心,東鄰殺牛,不如西家禴祭。我靈上慎勿以牲為祭,唯設餅、茶飲、幹飯、酒脯而已。天下貴賤,咸同此制。未山陵前,朔望設菜食。陵墓萬世所宅,意嘗恨休安陵未稱,今可用東三處地最東邊以葬我,名為景安陵。喪禮每存省約,不須煩民。百官停六時入臨,朔望祖日可依舊。諸主六宮,並不須從山陵。內殿鳳華、壽昌、耀靈三處,是吾所治制。”

齊武帝時,還與北魏通好,邊境比較安定。高帝和武帝的清明統治使江南經濟也有了一定的發展,社會也暫時安定。

大體而言,齐武帝是一个英明剛斷的君主。他基本继承了齐高帝的作风,對外崇尚节俭,努力實施富國政策。

\subsubsection{永明}

\begin{longtable}{|>{\centering\scriptsize}m{2em}|>{\centering\scriptsize}m{1.3em}|>{\centering}m{8.8em}|}
  % \caption{秦王政}\
  \toprule
  \SimHei \normalsize 年数 & \SimHei \scriptsize 公元 & \SimHei 大事件 \tabularnewline
  % \midrule
  \endfirsthead
  \toprule
  \SimHei \normalsize 年数 & \SimHei \scriptsize 公元 & \SimHei 大事件 \tabularnewline
  \midrule
  \endhead
  \midrule
  元年 & 483 & \tabularnewline\hline
  二年 & 484 & \tabularnewline\hline
  三年 & 485 & \tabularnewline\hline
  四年 & 486 & \tabularnewline\hline
  五年 & 487 & \tabularnewline\hline
  六年 & 488 & \tabularnewline\hline
  七年 & 489 & \tabularnewline\hline
  八年 & 490 & \tabularnewline\hline
  九年 & 491 & \tabularnewline\hline
  十年 & 492 & \tabularnewline\hline
  十一年 & 493 & \tabularnewline
  \bottomrule
\end{longtable}


%%% Local Variables:
%%% mode: latex
%%% TeX-engine: xetex
%%% TeX-master: "../../Main"
%%% End:
