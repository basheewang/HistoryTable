%% -*- coding: utf-8 -*-
%% Time-stamp: <Chen Wang: 2021-11-01 15:05:10>

\subsection{明帝蕭鸞\tiny(494-498)}

\subsubsection{生平}

齊明帝蕭鸞(452年-498年),字景栖,小名玄度,廟號高宗,谥明皇帝,南齊的第五任皇帝。為始安貞王蕭道生之子、齊高帝蕭道成之姪。他在494年至498年期間在位,共5年。

蕭鸞自小父母雙亡,由蕭道成撫養,蕭道成對其視若己出。宋順帝時,蕭鸞擔任安吉令,以嚴格而聞名;後遷淮南、宣城太守,輔國將軍。齊高帝時封西昌侯、任郢州刺史;齊武帝蕭賾時升任侍中,領驍騎將軍。蕭賾死前,萧鸾挫败中书郎王融改立蕭賾次子竟陵王萧子良为新君的图谋。蕭賾以蕭鸞為輔政,輔佐蕭昭業。

自從文惠太子蕭長懋於493年死後,蕭鸞便有爭奪帝位之心;蕭鸞迫使萧昭业处决近臣杨珉、徐龙驹,又寻罪名杀萧昭业近臣周奉叔、杜文谦、綦毋珍之。萧昭业曾与叔祖父萧锵合谋杀萧鸾,萧锵反对,故未果。后萧昭业又和皇后何婧英叔父何胤谋杀萧鸾,何胤不敢,但萧昭业也不再委萧鸾以重任。494年,萧鸾担心有变,与投靠自己的将领萧谌、萧坦之等发动政变,由萧谌殺蕭昭業,并以太后名义追废为鬱林王,改立其弟蕭昭文,愈发控制朝政,甚至控制了萧昭文的饮食,萧昭文曾想吃蒸鱼菜,太官令却因为没有萧鸾的许可而不给;不久萧鸾又廢蕭昭文為海陵王自立為帝。蕭鸞於494年即位後,便壓制宗室力量,並以典籤監視諸王;並且从萧昭文任期开始就大肆屠殺蕭道成、蕭賾二帝诸子,先杀年长者,临终时又杀年幼者,全都誅滅。蕭鸞任內長期深居簡出,要求節儉,停止各地向中央的進獻,並且停止不少工程。

蕭鸞晚年病重,相當尊重道教與厭勝之術,將所有的服裝都改為紅色;而且蕭鸞還特地下詔向官府徵求銀魚以為藥劑,外界才知道蕭鸞患病。498年蕭鸞病故,葬於興安陵。

《南齊書》這樣形容他:“帝明審有吏才,持法無所借,制御親幸,臣下肅清。驅使寒人不得用四幅繖,大存儉約。罷世祖所起新林苑,以地還百姓。廢文帝所起太子東田,斥賣之。永明中輿輦舟乘,悉剔取金銀還主衣庫。太官進御食,有裹蒸,帝曰:‘我食此不盡,可四片破之,餘充晚食。’而世祖掖庭中宮殿服御,一無所改。”

\subsubsection{建武}

\begin{longtable}{|>{\centering\scriptsize}m{2em}|>{\centering\scriptsize}m{1.3em}|>{\centering}m{8.8em}|}
  % \caption{秦王政}\
  \toprule
  \SimHei \normalsize 年数 & \SimHei \scriptsize 公元 & \SimHei 大事件 \tabularnewline
  % \midrule
  \endfirsthead
  \toprule
  \SimHei \normalsize 年数 & \SimHei \scriptsize 公元 & \SimHei 大事件 \tabularnewline
  \midrule
  \endhead
  \midrule
  元年 & 494 & \tabularnewline\hline
  二年 & 495 & \tabularnewline\hline
  三年 & 496 & \tabularnewline\hline
  四年 & 497 & \tabularnewline\hline
  五年 & 498 & \tabularnewline
  \bottomrule
\end{longtable}

\subsubsection{永泰}

\begin{longtable}{|>{\centering\scriptsize}m{2em}|>{\centering\scriptsize}m{1.3em}|>{\centering}m{8.8em}|}
  % \caption{秦王政}\
  \toprule
  \SimHei \normalsize 年数 & \SimHei \scriptsize 公元 & \SimHei 大事件 \tabularnewline
  % \midrule
  \endfirsthead
  \toprule
  \SimHei \normalsize 年数 & \SimHei \scriptsize 公元 & \SimHei 大事件 \tabularnewline
  \midrule
  \endhead
  \midrule
  元年 & 498 & \tabularnewline
  \bottomrule
\end{longtable}


%%% Local Variables:
%%% mode: latex
%%% TeX-engine: xetex
%%% TeX-master: "../../Main"
%%% End:
