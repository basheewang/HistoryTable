%% -*- coding: utf-8 -*-
%% Time-stamp: <Chen Wang: 2019-12-20 14:46:48>


\section{南齐\tiny(479-502)}

\subsection{简介}

齊(479年-502年)是中国历史上南北朝时期南朝第二个朝代。为蕭道成所建。史称南齐(以与北朝的北齐相区别)或萧齐。

以齐为国号,源于谶纬之说。《谶书》云:“金刀利刃齐\xpinyin*{刈}之”,意即“齐”将取代“宋”(因为南朝宋皇族为刘姓)。

南齐的开国之君萧道成是刘宋将领,在宋明帝在位期间担任右军将军。宋明帝去世后,他与尚书令袁粲共同掌管朝政。474年,萧道成平定江州刺史桂阳王刘休范的反叛,进爵为公,迁中领军将军,掌握了禁卫军,督五州军事。此时刘宋政权内鬥激烈,萧道成逐渐掌握大权。477年,后废帝刘昱在睡梦中被自己卫士所杀。萧道成立劉準继位。萧道成被封齐王。在这之后,萧道成铲除了忠于刘宋的袁粲、沈攸之等人。479年,萧道成迫使刘准禅让,刘宋灭亡,南齐建立。

萧道成崇尚节俭,反对奢靡,并以身作则,将宫殿、御用仪仗等凡用金、铜制作的器具全部用铁器替代,衣服上的玉佩、挂饰等统统取消。高帝萧道成在位时经常吊在嘴边的一句话是“使我治天下十年,当使黄金与土同价”,可见他的提倡节俭与身体力行。齐高帝提倡节俭的政策减轻了人民的负担。他也与北魏和好,维护边境安定。这使得新生的南齐政权迅速走向轨道。

482年,齐高帝萧道成去世,由长子萧赜继位,即齐武帝。当时,庶族地主为了免除所承担的赋役,往往向官吏行贿,在政府的黄籍上注入伪造的父祖爵位,改成免役免税的士族。刘宋以来,这种改注籍状,诈入仕流的庶族地主很多。萧道成在继位的第二年(480年),實施檢籍政策,专门设立校籍官和置令史,负责清查户籍。齐武帝登基后,继续其父的政策。那些被认为有假的户籍,都须退还本地,称为“却籍”。核查出本应服役纳赋而户籍上造假的,便恢复原来的户籍,继续承担赋役,称为“正籍”。檢籍政策虽然增加了赋税,却严重伤害了庶族地主的利益。

485年富阳唐寓之为此起兵叛乱,虽然这次叛乱被齐武帝迅速平息,但检籍的政策依然受到庶族的激烈反对。最终,在490年,齐武帝被迫妥协,宣布“却籍”无效,对因为“却籍”而被发配戍边的人民准许返归故乡,恢复刘宋升明时期户籍所注的原状。

尽管如此,齐武帝依然是一个英明的君主。他基本继承了齐高帝的作风,對外崇尚节俭,并且与北魏保持边界和平,使得南齐的国力大幅增强,史稱「永明之治」。

齐武帝在登基时,立长子萧长懋为皇太子。萧长懋在齐武帝在位期间去世,齐武帝选择皇太孙萧昭业作为继承人。493年,齐武帝去世,萧昭业继位。为武帝发丧之日,萧昭业刚送葬车出端门,便稱自己有病不能前去墓地。回宫后,马上召集乐工大奏胡曲表演歌舞,喇叭胡琴,声彻内外。萧昭业登基后,赏赐自己的亲信,一次赏赐就百数十万。每次看见宫中财宝,就自言自语:“我从前想你们一个也难得,看我今天怎么用你们!”他刚继位时,御库中总共有钱八亿万之巨,金银布帛不可胜数。萧昭业继位不到一年,已挥霍大半,赏赐给得意的左右、宫人。甚至打碎宫中宝物作为娱乐。他爱好斗鸡,甚至花数千钱来买斗鸡。

萧昭业贪图享受,任意賞賜,甚至寵幸宦官徐龍駒和侍衛周奉叔等人,弄權不法。辅政大臣萧鸾多次劝谏,萧昭业剛開始拒聽,後來才勉強答應蕭鸞,處死了徐龍駒和周奉叔。雖然蕭昭業一直打算铲除萧鸾,可是找不到願意支持他的宗王與大臣,只好暫時繼續忍受蕭鸞的專政。近卫军首领萧谌、萧坦之看到萧昭业私德日渐敗壞,都依附萧鸾,准备发动政变罢黜他。494年,萧鸾带兵入宫,诛杀萧昭业。萧鸾以太后的名义废萧昭业为郁林王,迎立其弟新安王萧昭文为帝。不到四个月,萧鸾废萧昭文为海陵王。萧鸾自立为帝,史称齐明帝。

齐明帝萧鸾是萧道成的侄子,在軍隊中累積很高的威望。他在武帝去世时受命辅佐萧昭业。在通过政变手段上台之后,为避免历史重演,他遂大肆诛杀齐高帝、齐武帝子孙以防后患。他性情阴险,每次誅殺宗王子孫時,都要對天焚香,痛哭流涕,表現自己是萬不得已的姿態,最后在生前杀光了高、武、文三帝诸子,还险些连高、武二帝的孙子们也杀尽。他極力崇尚节俭,但有時飲宴却頗為奢侈。他崇信道术,每次出行都要占卜吉凶。向南出行则宣称是向西行,向东出行则宣称是向北行。萧鸾晚年病重,对外却一直隐瞒病情,直到萧鸾特地下诏向官府征求银鱼以为药剂,外界才知道萧鸾患病。498年,萧鸾去世,由其子萧宝卷继承。

萧宝卷刚即位,其父留下的六個輔政大臣,分別是揚州刺史始安王蕭遙光、尚書令徐孝嗣、右僕射江祏、右將軍蕭坦之、侍中江祀、衛尉劉暄。是為「六貴」。未幾萧宝卷在一年內把「六貴」分別殺死。

萧宝卷性格讷涩,很少说话,不喜欢跟大臣接触,常常出宫闲逛,出游時常拆毁民居、驱逐居民,闹得民不聊生。后宫失火被焚,他就新造仙华、神仙、玉寿三座豪华宫殿。他又凿金为莲花,贴放于地,令宠妃潘氏行走其上,称为“步步生莲花」。

面对这种局面,始安王萧遥光、太尉江州刺史陳顯達和将军崔慧景连续发起三次兵变,试图起事推翻萧宝卷,但都被平定。萧宝卷因此更加放纵,派人毒杀平叛最力的尚書令,亦為宗室的萧懿,萧懿之弟、鎮守襄陽的萧衍遂与荆州行府事萧颖胄合谋拥戴荆州刺史南康王萧宝融为帝,即齐和帝,由萧衍起兵攻建康,于是南齐出现了两帝并立的局面。萧颖胄死后,萧衍成为义军的唯一首领。公元501年,萧衍攻陷建康,萧宝卷被将军王珍国所杀。萧衍奉萧昭业母王宝明称制,迎和帝回建康,期间以王宝明名义追废萧宝卷为东昏侯,进封自己为建安郡公、梁公、梁王,并诛杀明帝的儿子们和三个侄子,只有残疾的晋安王萧宝义和北逃的建安王萧宝夤得免(庐陵王萧宝源虽也未被杀但不久病死)。

502年,齐和帝还未到建康就被迫禅讓於萧衍。萧衍改国号为梁,是為梁武帝,南齐灭亡。后来萧衍在高帝孙萧子恪、萧子范面前将自己起兵夺位解释为替齐高帝、齐武帝子孙报仇。

527年,已成为北魏将领的萧宝夤在关中地区叛魏称帝,复建齐国,直至次年败亡。但因其在位短暂、立国时南齐亡国已久也没有掌控原属南齐的领地,他并不被视为南齐君主。

齊高帝(479年—482年在位)收集大量藝術作品。南京宮廷的肖像畫家謝赫,是當時重要的畫家。現存最早的中國繪畫理論,即出自其筆下,對後世影響甚鉅。座落於南京東方二十公里棲霞山(攝山)的棲霞寺,當地附近洞窟內有佛像源自南齊時期。

\subsection{宣帝生平}

萧承之(384年-447年),字嗣伯。刘宋南兰陵郡人 ,居晉陵縣武進縣東城里。南朝宋將領,官至右軍將軍。其子蕭道成後來篡宋自立,建立南齊,獲追諡為宣皇帝。

蕭承之少有大志,才力过人,同族的丹阳尹萧摹之及北兖州刺史萧源之都很看重他。萧承之初为建威將軍朱齡石的参军,在义熙九年(413年)隨朱齡石入蜀,滅掉譙蜀政權,並迁扬武将军、安固汶山二郡太守,任內善於撫綏。

刘宋元嘉初年,徙萧承之为武烈将军、济南太守。元嘉七年(430年),右将军到彦之北伐大败,北魏乘胜進攻南朝宋青州各郡縣,其中魏將叔孫建領兵进攻济南,萧承之率数百人拒战,击退魏军。及後魏军聚集济南城下,萧承之使偃兵开城门,用空城计。部下谏道:“贼众我寡,为何如此轻敌!”萧承之道:“今日孤悬在外守著困逼的城池已是很危急的了,若果還繼續示弱必然會被他們攻陷,只應該以強勢應付。”魏军疑有伏兵,遂引兵而去。

元嘉八年(431年),在到彥之敗後領兵北援的征南大将军檀道济在寿张(今山東東平縣西南)擊敗了叔孫健等軍,但因為缺糧而被逼在歷城(今山東濟南歷城區)撤兵,一直堅守的滑台(今山東滑縣)失去援兵之下最終失陷。青州刺史蕭思話知道濟退兵,擔憂遭到魏軍攻擊而決定棄守治所東陽城,蕭承之堅決反對卻不獲聽從;最終東陽城沒有被魏軍攻擊,但留下的大批糧草就被當地人焚毀。戰後宋文帝以萧承之有保全济南城之功,曾手書長沙王劉義欣表示打算以承之為兗州刺史,並命他將建議交給檀道濟參詳,然而因為承之與檀道濟素來沒有交情,事情就被擱下了。後轉輔國、鎮北中兵參軍及員外郎。

元嘉十年(432年),萧思话出任梁南秦二州刺史,萧承之为他的横野府司马、汉中太守。就在同年十一月,仇池王杨难当趁刺史交接間进攻汉中(今陝西漢中市),留守的原梁州刺史甄法护弃治所南城(今陝西褒中縣)逃走,萧思话那時才到襄阳(今湖北襄陽市),便停駐當地,改派萧承之率五百兵先赴梁州,又派西戎長史蕭汪之為其統率。蕭順之在路上招集兵眾,招得一千精兵前進。元嘉十一年(433年)正月,蕭順之到達磝頭 (今陝西石泉縣城),時楊難當在焚燒掠奪漢中過後領兵撤走,留下他任命的梁秦二州刺史趙溫守漢中,另有魏兴太守薛健守黄金山。蕭順之就先派陰平太守蕭坦攻陷黃金山下的鐵城戍,至二月時趙溫與薛健及仇池馮翊太守蒲早子合力進攻鐵城戍,反為蕭坦所敗; 隨後蕭承之獲得荊州刺史臨川王劉義慶派的裴方明部三千兵增援,遂將黃金戍也拿下。黃金戍是東漢末年佔據漢中的張魯所修築,南接漢中而北通驛道,兼城戍極其險要,承之據此作為基地後就向盤據漢中的趙溫等進行攻擊。趙溫逼於承之威脅而退守漢中小城,另外薛健及蒲甲子就守下桃城。承之遂與從襄陽到來的蕭思話主力部隊合力進攻,屢敗趙溫等人,其中承之曾在峨公山被呂平圍攻,但他在蕭汪之等支援下大敗對方。

楊難當派兒子楊和領萬多人增援趙溫等,並圍攻承之,至三月時和承之相持了達四十日,包圍圈厚達十多重。由於楊和援軍都穿上堅韌的犀甲,士兵近身戰時戈矛刀刃都不能刺穿,承之最終想出將長槊折成數尺長,指向敵軍後讓人在槊末用大斧鎚擊,一支就連殺數個犀甲兵。閏三月,敵軍無法抵抗之下唯有燒營逃走,承之就乘勝追擊至南城,再敗敵軍,俘殺甚多,最終收復了梁州。蕭承之亦因功獲授龍驤將軍。後蕭思話進寧朔將軍,承之亦隨轉寧朔司馬、兼漢中太守。

蕭承之後來入朝任太子屯騎校尉,隨後又調為江夏王劉義恭的司徒中兵參軍、龍驤將軍、南泰山太守 ,並封晉興縣五等男,食邑三百四十戶。及後又轉任右軍將軍。

蕭承之於元嘉二十四年(447年)去世,享年六十四。梁州人因為思念承之,在峨公山為他立庙祭祀。升明二年(478年),赠散骑常侍、金紫光禄大夫。其子萧道成建立南齐,承之獲追尊为宣皇帝,葬於永安陵。

%% -*- coding: utf-8 -*-
%% Time-stamp: <Chen Wang: 2021-11-01 15:04:36>

\subsection{高帝萧道成\tiny(479-482)}

\subsubsection{生平}

齐高帝萧道成(427年-482年4月11日),字绍伯,南北朝時代,南朝第二個皇朝南齊开国皇帝。

蕭道成出自兰陵萧氏,父親蕭承之仕於劉宋為右將軍,蕭道成亦在劉宋擔任軍官,宋明帝駕崩,蕭道成以右衛將軍領衛尉的名銜,與其他數位大臣受遺詔掌機要,為輔政大臣。劉昱即位皇帝後,桂陽王兼江州刺史劉休範叛變,為蕭道成領軍所平定,權勢日隆。

但青春期的劉昱喜好弓馬,時常殺人取樂,又濫殺無辜,一次突然跑到蕭道成家中,道成是個大胖子,「方坦而晝臥,腹大如瓠」。劉昱玩心頓起,覺得這麼大的肚子是個絕好的箭靶,拿起箭來就「彎弓欲射其腹」。蕭道成苦苦哀求,左右隨從也紛紛勸解說:蕭大人肚子這麼大,這麼好的目標,一箭就射死太可惜了!以後想射就沒有了!好說歹說之下,劉昱答應去掉箭鏃才射,結果一箭射中肚臍,歡呼高歌而去。剩下嚇得一身冷汗,死裡逃生心有餘悸的權臣蕭道成。

477年,劉昱被蕭道成的黨羽楊玉夫所弒,蕭道成改立宋順帝劉準,獨攬朝政。並在同年和隔年(478年),分別消滅了忠於宋室的尚書令袁粲、荊州刺史沈攸之,宰制全國。

479年,蕭道成篡宋自立為天子,國號齊,他為政務節儉,實施檢籍政策,清查詐入士族籍貫的寒人。

與南朝宋的劉裕一樣,在位僅四年去世,終年56歲。

除了在政治上的功業,蕭道成也廣覽經學、史學書籍,善寫作文、書法和下棋。

北宋的司馬光評論他:「高帝以功名之盛,不容於昏暴之朝,逆取而順守之,亦一時之良主也。」

\subsubsection{建元}

\begin{longtable}{|>{\centering\scriptsize}m{2em}|>{\centering\scriptsize}m{1.3em}|>{\centering}m{8.8em}|}
  % \caption{秦王政}\
  \toprule
  \SimHei \normalsize 年数 & \SimHei \scriptsize 公元 & \SimHei 大事件 \tabularnewline
  % \midrule
  \endfirsthead
  \toprule
  \SimHei \normalsize 年数 & \SimHei \scriptsize 公元 & \SimHei 大事件 \tabularnewline
  \midrule
  \endhead
  \midrule
  元年 & 479 & \tabularnewline\hline
  二年 & 480 & \tabularnewline\hline
  三年 & 481 & \tabularnewline\hline
  四年 & 482 & \tabularnewline
  \bottomrule
\end{longtable}


%%% Local Variables:
%%% mode: latex
%%% TeX-engine: xetex
%%% TeX-master: "../../Main"
%%% End:

%% -*- coding: utf-8 -*-
%% Time-stamp: <Chen Wang: 2021-11-01 15:04:48>

\subsection{武帝蕭賾\tiny(482-493)}

\subsubsection{生平}

齊武帝蕭\xpinyin*{賾}(440年-493年8月27日),字宣遠,齊高帝蕭道成長子,母刘智容。南齊第二任皇帝,病死時54歲,葬景安陵。年号永明。

武帝十分關心百姓疾苦,即位後,就下詔說:“比歲未稔,貧窮不少,京師二岸,多有其弊。遣中書舍人優量賑恤。”不久,再次下詔說,“水雨頻降,潮流薦滿,二岸居民,多所淹漬。遣中書舍人與兩縣官長優量賑恤。”

第二年,他又下詔酌情遣返軍中的囚徒,大赦囚犯,對於百姓中的鰥寡和貧窮之人,要加以賑濟。他提倡並獎勵農桑,災年時,還減免租稅。在位第四年,他下詔說: “揚、南徐二州,今年戶租三分二取見布,一分取錢。來歲以後,遠近諸州輸錢處,並減布直,匹準四百,依舊折半,以為永制。”

武帝還下令多辦學校,挑選有學問之人任教,以培育人們的德行。武帝以富國為先,不喜歡遊宴、奢靡之事,提倡節儉。他曾下令舉辦婚禮時不得奢侈。

齐武帝登基后,延續其父蕭道成的檢籍政策。那些被认为有假的户籍,都须退还本地,称为“却籍”。核查出本应服役纳赋而户籍上造假的,便恢复原来的户籍,继续承担赋役,称为“正籍”。檢籍政策虽然增加了赋税,却严重伤害了庶族地主的利益。

公元485年富阳唐寓之为此起兵叛乱,虽然这次叛乱被齐武帝迅速平息,但检籍的政策依然受到庶族的激烈反对。最终,在490年,齐武帝被迫妥协,宣布“却籍”无效,对因为“却籍”而被发配戍边的人民准许返归故乡,恢复刘宋升明时期户籍所注的原状。

武帝對於其後事,特意下詔說:“我識滅之後,身上著夏衣,畫天衣,純烏犀導,應諸器悉不得用寶物及織成等,唯裝復裌衣各一通。常所服身刀長短二口鐵環者,隨我入梓宮。祭敬之典,本在因心,東鄰殺牛,不如西家禴祭。我靈上慎勿以牲為祭,唯設餅、茶飲、幹飯、酒脯而已。天下貴賤,咸同此制。未山陵前,朔望設菜食。陵墓萬世所宅,意嘗恨休安陵未稱,今可用東三處地最東邊以葬我,名為景安陵。喪禮每存省約,不須煩民。百官停六時入臨,朔望祖日可依舊。諸主六宮,並不須從山陵。內殿鳳華、壽昌、耀靈三處,是吾所治制。”

齊武帝時,還與北魏通好,邊境比較安定。高帝和武帝的清明統治使江南經濟也有了一定的發展,社會也暫時安定。

大體而言,齐武帝是一个英明剛斷的君主。他基本继承了齐高帝的作风,對外崇尚节俭,努力實施富國政策。

\subsubsection{永明}

\begin{longtable}{|>{\centering\scriptsize}m{2em}|>{\centering\scriptsize}m{1.3em}|>{\centering}m{8.8em}|}
  % \caption{秦王政}\
  \toprule
  \SimHei \normalsize 年数 & \SimHei \scriptsize 公元 & \SimHei 大事件 \tabularnewline
  % \midrule
  \endfirsthead
  \toprule
  \SimHei \normalsize 年数 & \SimHei \scriptsize 公元 & \SimHei 大事件 \tabularnewline
  \midrule
  \endhead
  \midrule
  元年 & 483 & \tabularnewline\hline
  二年 & 484 & \tabularnewline\hline
  三年 & 485 & \tabularnewline\hline
  四年 & 486 & \tabularnewline\hline
  五年 & 487 & \tabularnewline\hline
  六年 & 488 & \tabularnewline\hline
  七年 & 489 & \tabularnewline\hline
  八年 & 490 & \tabularnewline\hline
  九年 & 491 & \tabularnewline\hline
  十年 & 492 & \tabularnewline\hline
  十一年 & 493 & \tabularnewline
  \bottomrule
\end{longtable}


%%% Local Variables:
%%% mode: latex
%%% TeX-engine: xetex
%%% TeX-master: "../../Main"
%%% End:

%% -*- coding: utf-8 -*-
%% Time-stamp: <Chen Wang: 2019-12-20 14:49:17>

\subsection{鬱林王\tiny(493-494)}

\subsubsection{生平}

蕭昭業(473年-494年9月7日),字元尚,小名法身,南朝齊的第三任皇帝,文惠太子蕭長懋之長子,齐武帝之孙。

蕭昭業雖然工於隸書,美容止而獲得祖父與父親的喜愛,但是蕭昭業本人是一個陽奉陰違的人。父亲病重时,他在人前表现出悲伤得连自己的健康都受损的样子,见者无不动容,而一旦回家,即表现出喜状,还要一位杨姓巫婆诅咒他的父亲和祖父,以让自己可以尽早登极。萧长懋不久死去,萧昭业认为杨氏诅咒得力,予以赏赐。武帝对此全然不觉,立蕭昭業為皇太孫。武帝不久也病倒了,萧昭业继续在人前作悲伤状,实则兴奋,在给妻何婧英的信中写了一个大“喜”字,还在其周围写了36个小“喜”字。

齐武帝病重时,和武帝子竟陵王蕭子良相好的中书郎王融想改立萧子良为新君,取代萧昭业,被武帝堂弟西昌侯蕭鸞挫败。493年末齊武帝死後,蕭昭業即位,改年號為隆昌。同時由蕭子良與蕭鸞輔政。

萧昭业尊母王宝明为皇太后,封妻何婧英为皇后。他即位之後便原型畢露,不但濫發賞賜,又與庶母霍氏通姦,並且過著十分浪費奢靡的生活,毫無一國之君的姿態,并架空涉嫌夺位的萧子良,赐死王融,朝政都委託西昌侯蕭鸞處理。还在丧期,就恢复奏乐。宠幸偏爱中书舍人綦毋珍之、朱隆之、直将军曹道刚、周奉叔、宦官徐龙驹等人,纵容近臣公然卖官。萧昭业曾不满萧鸾专权,对徐龙驹说:“我和萧锵(萧昭业叔祖父,鄱阳王)商议杀萧鸾,萧锵不同意,只能先让萧鸾专权一阵子了。”萧鸾迫使萧昭业下令诛杀皇后何婧英的男宠杨珉及徐龙驹;将军萧谌、萧坦之秘密投靠萧鸾,将萧昭业的动向告诉他;萧鸾又设计寻罪名杀死萧昭业近臣周奉叔、杜文谦、綦毋珍之;而萧昭业却因萧子良去世而放松警惕。

萧昭业与何婧英的叔父何胤图谋杀萧鸾,但何胤不敢。萧昭业不再委萧鸾以重任。萧鸾担心生变,与萧谌、萧坦之合谋政变,由二人等人于省诛杀曹道刚、朱隆之等,亲自率兵自尚书省入云龙门進宮。萧昭业正在寿昌殿和霍氏裸体相对,闻变下令关闭宫门,派宦官登兴光楼察看。他不知萧谌、萧坦之已叛变,向萧谌求助,见萧谌杀入殿中,才知。宿卫将士准备抵抗萧谌时,萧昭业逃向爱姬徐氏房,拔剑自刺,不果,以帛缠颈,乘舆出延德殿,宿卫见状要护驾,萧昭业却没有发话,出西弄,被萧谌追上弒殺,尸体运到徐龙驹府中。萧鸾以太后名义下诏追廢蕭昭業為鬱林王,葬以亲王礼。

\subsubsection{隆昌}

\begin{longtable}{|>{\centering\scriptsize}m{2em}|>{\centering\scriptsize}m{1.3em}|>{\centering}m{8.8em}|}
  % \caption{秦王政}\
  \toprule
  \SimHei \normalsize 年数 & \SimHei \scriptsize 公元 & \SimHei 大事件 \tabularnewline
  % \midrule
  \endfirsthead
  \toprule
  \SimHei \normalsize 年数 & \SimHei \scriptsize 公元 & \SimHei 大事件 \tabularnewline
  \midrule
  \endhead
  \midrule
  元年 & 494 & \tabularnewline
  \bottomrule
\end{longtable}


%%% Local Variables:
%%% mode: latex
%%% TeX-engine: xetex
%%% TeX-master: "../../Main"
%%% End:

%% -*- coding: utf-8 -*-
%% Time-stamp: <Chen Wang: 2021-11-01 15:05:05>

\subsection{海陵王蕭昭文\tiny(494)}

\subsubsection{生平}

蕭昭文(480年-494年),字季尚,南朝齊的第四任皇帝,在位僅四個月。母亲为宫人许氏。

蕭昭文為文惠太子蕭長懋的第二子,永明四年(486年)閏正月封為臨汝公,邑千五百户。初为辅国将军、济阳太守。十年(492年)正月,转持节、督南豫州诸军事、南豫州刺史,将军如故。十一年(493年),进号冠军将军。493年萧昭文兄长鬱林王蕭昭業即位後,十月封昭文為新安王。

隆昌元年(494年)闰四月,以萧昭文为扬州刺史。七月,尚书令、镇军大将军、西昌侯蕭鸞刺死萧昭业,拥立萧昭文為帝,改年號為延興,大赦。但是政事俱操於蕭鸞之手,萧昭文的生母许氏也没有得到尊封。

萧昭文刚登基,萧鸾就被任为骠骑大将军、录尚书事、扬州刺史、宣城郡公。九月,萧鸾以萧昭文之名诛杀高帝、武帝诸子,先杀司徒鄱阳王萧锵、中军大将军随郡王萧子隆、南兖州刺史安陆王萧子敬。江州刺史晋安王萧子懋起兵,萧鸾假黄钺,萧子懋败亡,萧鸾又杀湘州刺史南平王萧锐、郢州刺史晋熙王萧銶、南豫州刺史宜都王萧铿。十月,萧鸾被进为太傅,领大将军、扬州牧,加殊礼,进爵为王。萧鸾又杀中军将军桂阳王萧铄、抚军将军衡阳王萧钧、侍中秘书监江夏王萧锋、镇军将军建安王萧子真、左将军巴陵王萧子伦、司徒庐陵王萧子卿,且几乎杀死萧昭文的弟弟荆州刺史萧昭秀。这时萧鸾已掌控萧昭文的起居,有一次萧昭文想吃蒸鱼菜,太官令答没有萧鸾的命令,不给他吃。萧鸾诸子年幼,于是任兄子萧遥光、萧遥欣、萧遥昌以要职。当月,萧昭文被蕭鸞以嫡母皇太后王宝明名义以有病为由廢黜為海陵王,以东汉东海王劉彊之礼安置,给虎贲、旄头、画轮车,供奉很厚。次月,萧昭文便被蕭鸞派去看病的医生殺害,谥号恭。萧鸾以劉彊之礼厚葬,但并没有用皇帝礼。北魏趁机大举入侵。

\subsubsection{延兴}

\begin{longtable}{|>{\centering\scriptsize}m{2em}|>{\centering\scriptsize}m{1.3em}|>{\centering}m{8.8em}|}
  % \caption{秦王政}\
  \toprule
  \SimHei \normalsize 年数 & \SimHei \scriptsize 公元 & \SimHei 大事件 \tabularnewline
  % \midrule
  \endfirsthead
  \toprule
  \SimHei \normalsize 年数 & \SimHei \scriptsize 公元 & \SimHei 大事件 \tabularnewline
  \midrule
  \endhead
  \midrule
  元年 & 494 & \tabularnewline
  \bottomrule
\end{longtable}


%%% Local Variables:
%%% mode: latex
%%% TeX-engine: xetex
%%% TeX-master: "../../Main"
%%% End:

%% -*- coding: utf-8 -*-
%% Time-stamp: <Chen Wang: 2021-11-01 15:05:10>

\subsection{明帝蕭鸞\tiny(494-498)}

\subsubsection{生平}

齊明帝蕭鸞(452年-498年),字景栖,小名玄度,廟號高宗,谥明皇帝,南齊的第五任皇帝。為始安貞王蕭道生之子、齊高帝蕭道成之姪。他在494年至498年期間在位,共5年。

蕭鸞自小父母雙亡,由蕭道成撫養,蕭道成對其視若己出。宋順帝時,蕭鸞擔任安吉令,以嚴格而聞名;後遷淮南、宣城太守,輔國將軍。齊高帝時封西昌侯、任郢州刺史;齊武帝蕭賾時升任侍中,領驍騎將軍。蕭賾死前,萧鸾挫败中书郎王融改立蕭賾次子竟陵王萧子良为新君的图谋。蕭賾以蕭鸞為輔政,輔佐蕭昭業。

自從文惠太子蕭長懋於493年死後,蕭鸞便有爭奪帝位之心;蕭鸞迫使萧昭业处决近臣杨珉、徐龙驹,又寻罪名杀萧昭业近臣周奉叔、杜文谦、綦毋珍之。萧昭业曾与叔祖父萧锵合谋杀萧鸾,萧锵反对,故未果。后萧昭业又和皇后何婧英叔父何胤谋杀萧鸾,何胤不敢,但萧昭业也不再委萧鸾以重任。494年,萧鸾担心有变,与投靠自己的将领萧谌、萧坦之等发动政变,由萧谌殺蕭昭業,并以太后名义追废为鬱林王,改立其弟蕭昭文,愈发控制朝政,甚至控制了萧昭文的饮食,萧昭文曾想吃蒸鱼菜,太官令却因为没有萧鸾的许可而不给;不久萧鸾又廢蕭昭文為海陵王自立為帝。蕭鸞於494年即位後,便壓制宗室力量,並以典籤監視諸王;並且从萧昭文任期开始就大肆屠殺蕭道成、蕭賾二帝诸子,先杀年长者,临终时又杀年幼者,全都誅滅。蕭鸞任內長期深居簡出,要求節儉,停止各地向中央的進獻,並且停止不少工程。

蕭鸞晚年病重,相當尊重道教與厭勝之術,將所有的服裝都改為紅色;而且蕭鸞還特地下詔向官府徵求銀魚以為藥劑,外界才知道蕭鸞患病。498年蕭鸞病故,葬於興安陵。

《南齊書》這樣形容他:“帝明審有吏才,持法無所借,制御親幸,臣下肅清。驅使寒人不得用四幅繖,大存儉約。罷世祖所起新林苑,以地還百姓。廢文帝所起太子東田,斥賣之。永明中輿輦舟乘,悉剔取金銀還主衣庫。太官進御食,有裹蒸,帝曰:‘我食此不盡,可四片破之,餘充晚食。’而世祖掖庭中宮殿服御,一無所改。”

\subsubsection{建武}

\begin{longtable}{|>{\centering\scriptsize}m{2em}|>{\centering\scriptsize}m{1.3em}|>{\centering}m{8.8em}|}
  % \caption{秦王政}\
  \toprule
  \SimHei \normalsize 年数 & \SimHei \scriptsize 公元 & \SimHei 大事件 \tabularnewline
  % \midrule
  \endfirsthead
  \toprule
  \SimHei \normalsize 年数 & \SimHei \scriptsize 公元 & \SimHei 大事件 \tabularnewline
  \midrule
  \endhead
  \midrule
  元年 & 494 & \tabularnewline\hline
  二年 & 495 & \tabularnewline\hline
  三年 & 496 & \tabularnewline\hline
  四年 & 497 & \tabularnewline\hline
  五年 & 498 & \tabularnewline
  \bottomrule
\end{longtable}

\subsubsection{永泰}

\begin{longtable}{|>{\centering\scriptsize}m{2em}|>{\centering\scriptsize}m{1.3em}|>{\centering}m{8.8em}|}
  % \caption{秦王政}\
  \toprule
  \SimHei \normalsize 年数 & \SimHei \scriptsize 公元 & \SimHei 大事件 \tabularnewline
  % \midrule
  \endfirsthead
  \toprule
  \SimHei \normalsize 年数 & \SimHei \scriptsize 公元 & \SimHei 大事件 \tabularnewline
  \midrule
  \endhead
  \midrule
  元年 & 498 & \tabularnewline
  \bottomrule
\end{longtable}


%%% Local Variables:
%%% mode: latex
%%% TeX-engine: xetex
%%% TeX-master: "../../Main"
%%% End:

%% -*- coding: utf-8 -*-
%% Time-stamp: <Chen Wang: 2021-11-01 15:05:19>

\subsection{萧宝卷蕭寶卷\tiny(498-501)}

\subsubsection{生平}

蕭寶卷(483年-501年),字智藏,原名蕭明賢,南齊的第六代皇帝,因殘暴荒淫被殺,後追封東昏侯;為齊明帝蕭鸞第二子。蕭寶卷也被認為是中國歷史上昏庸荒淫的皇帝之一。

蕭寶卷的生母劉惠端是蕭鸞的正妻,早亡,由潘妃撫養。他年少時不喜讀書,以捕老鼠為樂。498年,蕭寶卷在蕭鸞死後即位,這時蕭寶卷才十六歲,並且封潘妃之侄女潘玉兒為貴妃,潘貴妃生下一個女兒,封為公主,但公主卻百日而殤,蕭寶卷制斬衰絰杖,積旬不聽音樂,衣悉粗布。

蕭寶卷性格內向,很少說話,不喜歡跟大臣接觸,常常出宮閒逛,每次出遊都一定要拆毀民居、驅逐居民,並且興建仙華、神仙、玉壽諸殿,並且大量賞賜臣下,造成國家的財政困難。而且蕭寶卷也殺害不少的大臣,即位之後便殺害六位顧命大臣表叔右僕射江祏、侍中江祀、堂兄始安王蕭遙光(作乱事败被杀)、司空徐孝嗣、右將軍蕭坦之、舅父領軍將軍劉暄及重臣曹虎、沈文季等人。也由於蕭寶卷的昏暴,導致大将裴叔业以重镇寿阳投魏;太尉陳顯達與將軍崔慧景先後起兵叛亂,但都兵敗被殺。萧宝卷胞弟江夏王萧宝玄响应崔慧景,几乎被其拥立为帝,萧宝卷平乱后杀萧宝玄。

蕭寶卷平定叛亂之後更加昏暴,除了與潘玉奴、宦官梅蟲兒等人日夜玩樂之外,並且派人毒殺平定崔慧景叛亂最力的尚書令蕭懿,結果導致蕭懿之弟蕭衍發兵進攻建康。萧衍並且与荆州行府事萧颖胄合作,改立皇弟荆州刺史南康王蕭寶融於江陵稱帝,遥废萧宝卷为庶人,又封涪陵王;蕭寶卷就在蕭衍發兵進攻建康的動亂中,被將軍王珍國所殺。

之後蕭寶卷被追廢為庶人,有司请求追封其为零阳侯,不许,请求追封涪陵王,获准。后蕭衍又將其追降為東昏侯,谥号炀。

\subsubsection{永元}

\begin{longtable}{|>{\centering\scriptsize}m{2em}|>{\centering\scriptsize}m{1.3em}|>{\centering}m{8.8em}|}
  % \caption{秦王政}\
  \toprule
  \SimHei \normalsize 年数 & \SimHei \scriptsize 公元 & \SimHei 大事件 \tabularnewline
  % \midrule
  \endfirsthead
  \toprule
  \SimHei \normalsize 年数 & \SimHei \scriptsize 公元 & \SimHei 大事件 \tabularnewline
  \midrule
  \endhead
  \midrule
  元年 & 499 & \tabularnewline\hline
  二年 & 500 & \tabularnewline\hline
  三年 & 501 & \tabularnewline
  \bottomrule
\end{longtable}


%%% Local Variables:
%%% mode: latex
%%% TeX-engine: xetex
%%% TeX-master: "../../Main"
%%% End:

%% -*- coding: utf-8 -*-
%% Time-stamp: <Chen Wang: 2019-12-20 14:53:11>

\subsection{和帝\tiny(501-502)}

\subsubsection{生平}

齊和帝蕭寶融(488年-502年),字智昭,南齊的末代皇帝,齊明帝蕭鸞第八子,母萧鸾原配追赠皇后刘惠端。

494年十一月被封為隨郡王,499年正月改封為南康王並任荊州刺史,駐守江陵。

501年三月,蕭衍發兵攻打蕭寶卷,並且与行荆州府事萧颖胄合谋,让萧宝融自称相国,后又立蕭寶融為皇帝,以萧宝融名义废萧宝卷为庶人,萧颖胄辅佐萧宝融守江陵,萧衍东征。萧衍手下劝他把萧宝融带到襄阳以免便宜了挟天子以令诸侯的萧颖胄,萧衍拒绝,认为此次起兵如果失败了此举就没有意义,如果成功了再寻找夺权的机会也不迟。后来萧颖胄病死,萧衍成为反对萧宝卷的唯一首领。蕭衍進入建康後,便將蕭寶融於502年接入建康,在萧宝融到来前奉废帝萧昭业的母后王宝明称制。同年,王宝明封蕭衍為梁王,不久蕭衍以王宝明名義殺害湘東王蕭寶晊兄弟,後來又殺掉齊明帝其他的兒子。不久蕭寶融便在到达建康前在姑孰被迫奉王宝明诏令禪位予蕭衍,南齊到此滅亡。

蕭衍即位後封蕭寶融為巴陵王,在姑孰建立宮室供其居住;第二天蕭寶融就被蕭衍所殺。

\subsubsection{中兴}

\begin{longtable}{|>{\centering\scriptsize}m{2em}|>{\centering\scriptsize}m{1.3em}|>{\centering}m{8.8em}|}
  % \caption{秦王政}\
  \toprule
  \SimHei \normalsize 年数 & \SimHei \scriptsize 公元 & \SimHei 大事件 \tabularnewline
  % \midrule
  \endfirsthead
  \toprule
  \SimHei \normalsize 年数 & \SimHei \scriptsize 公元 & \SimHei 大事件 \tabularnewline
  \midrule
  \endhead
  \midrule
  元年 & 501 & \tabularnewline\hline
  二年 & 502 & \tabularnewline
  \bottomrule
\end{longtable}


%%% Local Variables:
%%% mode: latex
%%% TeX-engine: xetex
%%% TeX-master: "../../Main"
%%% End:



%%% Local Variables:
%%% mode: latex
%%% TeX-engine: xetex
%%% TeX-master: "../../Main"
%%% End:
