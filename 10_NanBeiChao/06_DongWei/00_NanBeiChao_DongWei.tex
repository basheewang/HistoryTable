%% -*- coding: utf-8 -*-
%% Time-stamp: <Chen Wang: 2019-12-23 15:25:07>


\section{东魏\tiny(534-550)}

\subsection{简介}

東魏(534年-550年)是中國南北朝時期北方王朝之一,由鮮卑化漢人高歡擁立北魏孝文帝年僅十一歲的曾孫元善見為孝靜帝,並與宇文泰所掌控的西魏對立,建都邺,以高欢霸府「大丞相府」所在地晋阳为别都。

北魏孝武帝为对抗权臣高歡逃奔关中后,高欢另立元善见为帝,东魏開始。

東魏自始,便是霸府政治權臣高欢架空皇帝掌控整个政局。高歡於公元547年病死,其权势由長子高澄所繼承。随即发生大将侯景叛投西魏的事件,但被高澄所平定。高澄权倾人主,曾命手下崔季舒打孝静帝三拳。后来孝静帝君臣图谋高澄,事泄,高澄指责身为皇帝的孝静帝要造反,虽然当时与孝静帝和解,但很快幽禁了孝静帝并诛杀参与图谋的大臣。公元550年,當二十七歲的孝靜帝以為高澄遇刺身亡、自己可以親政時,随即被高澄之弟高洋所廢,東魏亡。東魏只有孝靜帝元善見一个皇帝,用四個年號,共十六年。東魏由北齊取代。

東魏時期的藝術創作仍以佛教為主要啟發。位於泰山的神通寺,是中國現存最古老的石造塔寺,可能建於東魏一代。此時期的佛雕較北魏時期渾厚。

%% -*- coding: utf-8 -*-
%% Time-stamp: <Chen Wang: 2021-11-01 15:13:11>

\subsection{孝静帝元善見\tiny(534-550)}

\subsubsection{生平}

魏孝靜帝元善見(524年-552年1月21日),河南郡洛阳县(今河南省洛阳市东)人,魏孝文帝元宏曾孫,清河文宣王元亶嫡子,母清河王妃胡智,南北朝時期東魏唯一一代皇帝。政权被高欢父子控制。

永熙三年(西元534年)七月,魏孝武帝從洛陽出逃,投靠在長安的宇文泰。元善见父元亶原本自以为可以被权臣高欢拥立为帝,但十月,高歡和百僚詳細商議後,決定立元善見為皇帝,承嗣魏孝明帝元诩。即位於鄴城東北,改元天平,東魏正式建立,年僅十一歲。由於年幼,由高欢輔政。

高歡善於玩弄權術,權傾朝野,令孝靜帝天天都提心吊膽,對高歡頗為畏懼。高歡雖屢敗於勁敵西魏宇文泰,但一直把持权力。而高歡死後,其子高澄承繼父職,權勢更大。孝静帝在邺城东打猎,骑马疾驰,監衛都督從後叫:「天子不要馳馬,高澄發怒。」;高澄侍飲,舉起大酒杯勸酒,孝静帝不悦说“自古无不亡之国,朕何需要這樣活着?”,被高澄令手下崔季舒打了三拳。孝静帝与荀济等密謀要刺殺高澄,但事蹟敗露,高澄带兵入宫,指责孝静帝谋反,孝静帝驳斥,高澄痛哭谢罪,与孝靜帝痛饮到深夜,但三天后就將孝靜帝幽禁于含章堂而将荀济等烹杀于市。公元549年,高洋再繼任父兄之職,他見篡魏之時機已到,於次年迫帝禪位於己,改國號「齊」,東魏亡。

北齊封元善見為中山王,邑一萬戶;上書不稱臣,答不稱詔,載天子旌旗,行魏正朔,乘五時副車;封王諸子為縣公,邑各一千戶;奉絹三萬匹,錢一千萬,粟二萬石,奴婢三百人,水碾一具,田百頃,園一所;於中山國立魏宗廟。

天保二年十二月己酉(552年1月21日),高洋灌醉元善见的妻子高皇后(太原長公主),派人用毒酒毒死了元善见,又杀死了他的三个儿子,元善见时年28岁。

天保三年二月,北齐上谥号孝靜皇帝,将元善见葬于邺县漳河以北。他的陵墓曾经被发掘。

孝靜帝文武雙全,自幼聰明,能洞悉先機,且好文學,美容儀。力能挾石獅子以逾牆,射無不中。嘉辰宴會,多命群臣賦詩,從容沉雅,有孝文風。

天平初年,考虑到各地移民尚未建立起家业,孝静帝诏令政府支付一百三十万石的粟加以救助。天平三年秋,並、肆、汾、建、晉、泰、陝、東雍、南汾等州遇到旱灾,饥民开始逃荒,直到天平四年才下诏救济,死于饥荒者众多。孝静帝时期,政府在沿海地区设置煮盐灶,滄州有灶一千四百八十四,瀛州有四百五十二灶,幽州有一百八十灶,青州有五百四十六灶,邯鄲有四个灶。每年可产盐二十萬九千七百二斛四升,国家从中得益不少。

\subsubsection{天平}

\begin{longtable}{|>{\centering\scriptsize}m{2em}|>{\centering\scriptsize}m{1.3em}|>{\centering}m{8.8em}|}
  % \caption{秦王政}\
  \toprule
  \SimHei \normalsize 年数 & \SimHei \scriptsize 公元 & \SimHei 大事件 \tabularnewline
  % \midrule
  \endfirsthead
  \toprule
  \SimHei \normalsize 年数 & \SimHei \scriptsize 公元 & \SimHei 大事件 \tabularnewline
  \midrule
  \endhead
  \midrule
  元年 & 534 & \tabularnewline\hline
  二年 & 535 & \tabularnewline\hline
  三年 & 536 & \tabularnewline\hline
  四年 & 537 & \tabularnewline
  \bottomrule
\end{longtable}

\subsubsection{元象}

\begin{longtable}{|>{\centering\scriptsize}m{2em}|>{\centering\scriptsize}m{1.3em}|>{\centering}m{8.8em}|}
  % \caption{秦王政}\
  \toprule
  \SimHei \normalsize 年数 & \SimHei \scriptsize 公元 & \SimHei 大事件 \tabularnewline
  % \midrule
  \endfirsthead
  \toprule
  \SimHei \normalsize 年数 & \SimHei \scriptsize 公元 & \SimHei 大事件 \tabularnewline
  \midrule
  \endhead
  \midrule
  元年 & 538 & \tabularnewline\hline
  二年 & 539 & \tabularnewline
  \bottomrule
\end{longtable}

\subsubsection{兴和}

\begin{longtable}{|>{\centering\scriptsize}m{2em}|>{\centering\scriptsize}m{1.3em}|>{\centering}m{8.8em}|}
  % \caption{秦王政}\
  \toprule
  \SimHei \normalsize 年数 & \SimHei \scriptsize 公元 & \SimHei 大事件 \tabularnewline
  % \midrule
  \endfirsthead
  \toprule
  \SimHei \normalsize 年数 & \SimHei \scriptsize 公元 & \SimHei 大事件 \tabularnewline
  \midrule
  \endhead
  \midrule
  元年 & 539 & \tabularnewline\hline
  二年 & 540 & \tabularnewline\hline
  三年 & 541 & \tabularnewline\hline
  四年 & 542 & \tabularnewline
  \bottomrule
\end{longtable}

\subsubsection{武定}

\begin{longtable}{|>{\centering\scriptsize}m{2em}|>{\centering\scriptsize}m{1.3em}|>{\centering}m{8.8em}|}
  % \caption{秦王政}\
  \toprule
  \SimHei \normalsize 年数 & \SimHei \scriptsize 公元 & \SimHei 大事件 \tabularnewline
  % \midrule
  \endfirsthead
  \toprule
  \SimHei \normalsize 年数 & \SimHei \scriptsize 公元 & \SimHei 大事件 \tabularnewline
  \midrule
  \endhead
  \midrule
  元年 & 543 & \tabularnewline\hline
  二年 & 544 & \tabularnewline\hline
  三年 & 545 & \tabularnewline\hline
  四年 & 546 & \tabularnewline\hline
  五年 & 547 & \tabularnewline\hline
  六年 & 548 & \tabularnewline\hline
  七年 & 549 & \tabularnewline\hline
  八年 & 550 & \tabularnewline
  \bottomrule
\end{longtable}


%%% Local Variables:
%%% mode: latex
%%% TeX-engine: xetex
%%% TeX-master: "../../Main"
%%% End:



%%% Local Variables:
%%% mode: latex
%%% TeX-engine: xetex
%%% TeX-master: "../../Main"
%%% End:
