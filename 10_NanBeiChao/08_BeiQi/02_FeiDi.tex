%% -*- coding: utf-8 -*-
%% Time-stamp: <Chen Wang: 2019-12-23 15:44:05>

\subsection{废帝\tiny(559-560)}

\subsubsection{生平}

高殷(545年-561年;在位559年—560年),字正道,北齊第二代皇帝,齊文宣帝嫡長子,母亲是昭信皇后李祖娥。

天保元年(550年),立為皇太子,時年六歲。性敏慧,但偏于柔懦,高洋对此不满,曾打算立次子太原王高紹德。

天保十年(559年),高殷即位,時年十五歲。即位後,以咸陽王斛律金為左丞相,叔父錄尚書事、常山王高演為太傅,叔父司徒、長廣王高湛為太尉,司空段韶為司徒,平陽王高淹為司空,高陽王高湜為尚書左僕射,河間王高孝琬為司州牧,侍中燕子獻為右僕射共同执政,勵精圖治,對民生極為關心,曾分命使者巡省四方,求政得失,省察風俗,問人疾苦;整頓吏治,政治清明;武官年逾六旬皆放免,軍事上淘汰老弱,留下精壯,軍力大增;下詔減徭役,使由天保朝國勢的危急有紓緩。然高演在位高权重兢兢业业之余也开始覬覦皇位,且引起了朝中反对派的不满。終於560年,太后李祖娥等人与高演等人的矛盾白热化,其六叔高演發動政變,高殷被废为濟南王。

高殷被废的时候,其祖母娄昭君命高演发誓决不伤害高殷性命,但最终高演还是虑有后患,于次年将高殷秘密殺害。高殷死時十七歲,谥号「愍悼」,没有关于子女的记载。

\subsubsection{乾明}

\begin{longtable}{|>{\centering\scriptsize}m{2em}|>{\centering\scriptsize}m{1.3em}|>{\centering}m{8.8em}|}
  % \caption{秦王政}\
  \toprule
  \SimHei \normalsize 年数 & \SimHei \scriptsize 公元 & \SimHei 大事件 \tabularnewline
  % \midrule
  \endfirsthead
  \toprule
  \SimHei \normalsize 年数 & \SimHei \scriptsize 公元 & \SimHei 大事件 \tabularnewline
  \midrule
  \endhead
  \midrule
  元年 & 560 & \tabularnewline
  \bottomrule
\end{longtable}


%%% Local Variables:
%%% mode: latex
%%% TeX-engine: xetex
%%% TeX-master: "../../Main"
%%% End:
