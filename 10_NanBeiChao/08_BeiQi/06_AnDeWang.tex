%% -*- coding: utf-8 -*-
%% Time-stamp: <Chen Wang: 2019-12-23 15:46:59>

\subsection{安德王\tiny(576)}

\subsubsection{生平}

高延宗(544年-577年),勃海郡蓨县(今河北省衡水市景县)人,追尊齐文襄帝高澄第五子,母亲为姬妾陈氏,原本是北魏广阳王的家妓。

高延宗幼时,父亲高澄在547年被杀,就被他二叔文宣帝高洋抚养,宠得不成样子,高延宗十二岁,高洋还让他骑在自己的肚腹上,在肚脐里撒尿。抱着他说:“可怜止有此一个。”高洋问高延宗想做什么王,高延宗:“想作冲天王。”高洋于是问杨愔,杨愔答道:“天下无此郡名,愿使安于德。”于是封为安德王。高延宗骄傲恣睢,时常以肮脏花样折腾臣下,及至高洋去世,继位的叔叔高演看不过,命人捉高延宗来打了一百棍,自此高延宗稍有收敛。

武成帝高湛杀河间王高孝琬,高延宗痛哭不已。576年,后主高纬南逃,高延宗留守晋阳,被部下拥立为帝,改元德昌,军心登时大振,反败为胜,几乎将北周武帝宇文邕活捉。但仅两天就因得胜后懈怠防备被卷土重来的周军打败,周军攻克东门、南门,高延宗战不利,出逃到城北民宅被追上俘获。宇文邕亲自下马抓住高延宗的手,高延宗拒绝:“我是死人,我的手怎能接触至尊!”宇文邕说:“我们是两国天子(即承认了高延宗的北齐皇帝身份),彼此之间并非有仇怨,都是老百姓而来的,我终究不会加害于您,不必害怕。”让他重新穿戴好衣帽,以礼相待。后来宇文邕又问高延宗如何夺取邺城,高延宗推辞说这不是亡国之臣所能回答的,宇文邕强迫他回答,他才说:“如果任城王高湝援救邺城,臣下不知北齐能否坚持。如果是后主高纬自己守卫邺城,那么陛下可以兵不血刃就取得胜利。”

577年周武帝带高纬等回长安,与北齐君臣一起饮酒,席间让高纬跳舞,高延宗哭得不能自己,屡次想服毒自杀,被婢女劝止。同年,周武帝称高纬勾结穆提婆谋反,赐高纬及大量北齐皇室成员自尽。高氏皇族多自陈无辜,只有高延宗捋起衣袖哭着不说话。高延宗被以椒塞口而死。次年,妻李氏收葬了他的尸体。

\subsubsection{德昌}

\begin{longtable}{|>{\centering\scriptsize}m{2em}|>{\centering\scriptsize}m{1.3em}|>{\centering}m{8.8em}|}
  % \caption{秦王政}\
  \toprule
  \SimHei \normalsize 年数 & \SimHei \scriptsize 公元 & \SimHei 大事件 \tabularnewline
  % \midrule
  \endfirsthead
  \toprule
  \SimHei \normalsize 年数 & \SimHei \scriptsize 公元 & \SimHei 大事件 \tabularnewline
  \midrule
  \endhead
  \midrule
  元年 & 576 & \tabularnewline
  \bottomrule
\end{longtable}


%%% Local Variables:
%%% mode: latex
%%% TeX-engine: xetex
%%% TeX-master: "../../Main"
%%% End:
