%% -*- coding: utf-8 -*-
%% Time-stamp: <Chen Wang: 2019-12-23 15:45:56>

\subsection{高纬\tiny(565-576)}

\subsubsection{生平}

高纬(556年5月29日-577年11月),字仁纲,南北朝时期北齐第五位皇帝(565年-577年在位),史稱「後主」,北齐武成帝高湛的嫡長子,亦是中國唯一一位無上皇。

高纬与庶兄高绰同日出生,实为高湛次子,但是因为是嫡出,故被视为长子。

高纬即位时,腐朽的北齐政权已经摇摇欲坠,他自己仍然荒淫无道,杀害兄长高绰,导致北齐军队衰弱,政治腐败,尤其最大致命伤是诛杀名将高長恭、斛律光,这使得北齐失去得以抗击北周侵略的有能将领。

577年,北周来攻打北齐,占领晋阳,齐军大败,周军不久破北齐京师邺(今河北临漳),高纬慌忙将皇位传于自己8岁的儿子高恒,然后带着幼主高恒等十余人骑马准备投降江南的陈朝。他们刚逃到青州(今山东青州)就被周军俘虏了,北齐灭亡。高纬投降后,被周武帝封温国公,不久因为被诬陷谋反,而被武帝赐死,终年22岁(《北齊書》将此事记于下年)。

讽刺的是,高纬被俘后,竟对北周武帝宇文邕要求將馮小憐归还给他。周帝說:「朕對於天下,就像脫掉鞋子一樣輕視,一個老太婆有甚麼好跟您爭的呢?」後主寵愛馮小憐,李商隱曾寫詩諷刺道:

「小憐玉體橫陳夜,已報周師入晉陽」—北齊兩首(其一)第二聯

「晋阳已陷休回顾,更请君王猎一围」—北齊兩首(其二)第二聯

这两首诗說明後主在北周入侵時仍然不理政事,荒唐、淫亂。

\subsubsection{天统}

\begin{longtable}{|>{\centering\scriptsize}m{2em}|>{\centering\scriptsize}m{1.3em}|>{\centering}m{8.8em}|}
  % \caption{秦王政}\
  \toprule
  \SimHei \normalsize 年数 & \SimHei \scriptsize 公元 & \SimHei 大事件 \tabularnewline
  % \midrule
  \endfirsthead
  \toprule
  \SimHei \normalsize 年数 & \SimHei \scriptsize 公元 & \SimHei 大事件 \tabularnewline
  \midrule
  \endhead
  \midrule
  元年 & 565 & \tabularnewline\hline
  二年 & 566 & \tabularnewline\hline
  三年 & 567 & \tabularnewline\hline
  四年 & 568 & \tabularnewline\hline
  五年 & 569 & \tabularnewline
  \bottomrule
\end{longtable}

\subsubsection{武平}

\begin{longtable}{|>{\centering\scriptsize}m{2em}|>{\centering\scriptsize}m{1.3em}|>{\centering}m{8.8em}|}
  % \caption{秦王政}\
  \toprule
  \SimHei \normalsize 年数 & \SimHei \scriptsize 公元 & \SimHei 大事件 \tabularnewline
  % \midrule
  \endfirsthead
  \toprule
  \SimHei \normalsize 年数 & \SimHei \scriptsize 公元 & \SimHei 大事件 \tabularnewline
  \midrule
  \endhead
  \midrule
  元年 & 570 & \tabularnewline\hline
  二年 & 571 & \tabularnewline\hline
  三年 & 572 & \tabularnewline\hline
  四年 & 573 & \tabularnewline\hline
  五年 & 574 & \tabularnewline\hline
  六年 & 575 & \tabularnewline\hline
  七年 & 576 & \tabularnewline
  \bottomrule
\end{longtable}

\subsubsection{隆化}

\begin{longtable}{|>{\centering\scriptsize}m{2em}|>{\centering\scriptsize}m{1.3em}|>{\centering}m{8.8em}|}
  % \caption{秦王政}\
  \toprule
  \SimHei \normalsize 年数 & \SimHei \scriptsize 公元 & \SimHei 大事件 \tabularnewline
  % \midrule
  \endfirsthead
  \toprule
  \SimHei \normalsize 年数 & \SimHei \scriptsize 公元 & \SimHei 大事件 \tabularnewline
  \midrule
  \endhead
  \midrule
  元年 & 576 & \tabularnewline
  \bottomrule
\end{longtable}


%%% Local Variables:
%%% mode: latex
%%% TeX-engine: xetex
%%% TeX-master: "../../Main"
%%% End:
