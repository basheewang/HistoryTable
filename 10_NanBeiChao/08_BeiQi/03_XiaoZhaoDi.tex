%% -*- coding: utf-8 -*-
%% Time-stamp: <Chen Wang: 2021-11-01 15:13:58>

\subsection{孝昭帝高演\tiny(560-561)}

\subsubsection{生平}

齊孝昭帝高演(535年-561年;在位560年—561年),字延安,北齐第三任皇帝。他是東魏权臣高欢第六子,文宣帝同母弟,在位一年。

高演長於政術,善於理解事情的細節;天保朝起開始干預朝政,政治經驗逐漸成熟豐富,眼見次兄齊文宣帝沉湎酒色,大臣趨炎附勢,惟高演滿臉憂愁,不時直諫。其兄文宣帝臨終時,表示必要时皇位可以相让,唯不可伤害高殷。廢帝即位,獨攬朝政。560年,高演發動政變,废高殷為濟南王。高演登上皇帝寶座,改元皇建,時年二十六歲。

高演在位期間,文治武功兼盛,『帝留心於政事,積極尋求及任用賢能為朝廷效力,政治清明;帝關心民生,輕徭薄賦,並下詔分遣大使巡省四方,觀察風俗,問人疾苦,考求得失。並親征親戎北討庫莫奚,出長城,虜奔遁,分兵致討,大獲牛馬。』在北齊28年歷史和六帝之中,只有孝昭帝稱得上是明君,可惜他在位時間不長,即位翌年,高演便因墮马事故重伤而死,在位僅兩年,終年僅27歲。

高殷被废的时候,娄昭君命儿子高演发誓决不伤害孙子高殷性命,但最终高演还是虑有后患,于次年将高殷秘密殺害。不久高演即出了意外,传说是齊文宣帝的厉鬼复仇。娄昭君亦对此深感悲愤,不肯原谅高演。為了保住兒子高百年,临终时候高演宣布废掉年幼的太子,傳位於弟弟長廣王高湛。他的谥号为孝昭皇帝,廟號肃宗。

\subsubsection{皇建}

\begin{longtable}{|>{\centering\scriptsize}m{2em}|>{\centering\scriptsize}m{1.3em}|>{\centering}m{8.8em}|}
  % \caption{秦王政}\
  \toprule
  \SimHei \normalsize 年数 & \SimHei \scriptsize 公元 & \SimHei 大事件 \tabularnewline
  % \midrule
  \endfirsthead
  \toprule
  \SimHei \normalsize 年数 & \SimHei \scriptsize 公元 & \SimHei 大事件 \tabularnewline
  \midrule
  \endhead
  \midrule
  元年 & 560 & \tabularnewline\hline
  二年 & 561 & \tabularnewline
  \bottomrule
\end{longtable}


%%% Local Variables:
%%% mode: latex
%%% TeX-engine: xetex
%%% TeX-master: "../../Main"
%%% End:
