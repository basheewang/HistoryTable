%% -*- coding: utf-8 -*-
%% Time-stamp: <Chen Wang: 2021-11-01 15:13:48>

\subsection{文宣帝高洋\tiny(550-559)}

\subsubsection{生平}

齐文宣帝高洋(526年-559年,在位550年—559年),字子進,鮮卑名侯尼干,勃海郡蓨县(今河北省衡水市景县)人。因其生于晋阳,又名晋阳乐,南北朝时期北齐开国皇帝,在位10年。他是东魏权臣高欢次子,北齐追尊文襄皇帝高澄的同母弟,鮮卑化漢人。

幼時其貌不揚,沉默寡言,其實大智若愚,聰慧過人,雖偶然被兄弟嘲笑或玩弄,但其才能甚得父親高欢欣賞。高澄被蘭京刺殺以後,高洋便牢牢地掌握了大權。東魏孝靜帝元善見只好封他為丞相、齐郡王,加九锡、殊礼。高洋不甘當傀儡皇帝的大臣,就於550年就廢掉了元善見,自立為帝,改元「天保」,建都鄴,北齊建立,年僅25歲。當年十一月,西魏宇文泰率大軍進攻剛剛建立的北齊,高洋親自率軍迎戰。宇文泰看到高洋手下的部隊軍容嚴整,嘆息道:「高歡不死矣。」隨即退軍。

在位初年,留心政務,削減州郡,整頓吏治,訓練軍隊,加強兵防,使北齊在很短的時間內強盛起來。高洋出兵進攻柔然、契丹、高句麗等國,都大獲全勝。高洋亦曾趁南朝梁遭遇侯景之乱意图拥立梁宗室萧退为梁帝,遭梁将王僧辩、陈霸先抵抗而未果,后又拥立萧渊明为梁帝并迫使王僧辩接受,但陈霸先袭杀王僧辩,废黜萧渊明,后即代梁建陈。

同時,北齊的農業、鹽鐵業、瓷器製造業都相當發達,是同陳、西魏鼎立的三個國家中最富庶的。

可是,他在即位六、七年後就腐敗起來,整日不理朝政,沉湎於酒色之中,他在都城鄴(今河南安陽)修築三台宮殿,十分豪華,動用十萬民夫,簡直是奢侈至極。

幾代北齊皇帝幾乎都有精神問題,加上高洋為人殘忍嗜殺,酗酒之後更常失去理智。

其實高洋自己也清楚酒後的荒唐行為,但無法改正。腐化的生活縮短了高洋的壽命。

天保五年(554年)八月,高洋任命尉粲為司徒、侯莫陳相為司空、清河王高岳為太保(上三公)、平陽王高淹為錄尚書事、常山王高演為尚書令、上黨王高渙為左僕射。

天保六年(555年)三月十六日,齐文宣帝高洋返回都城鄴城,封高孝珩為廣寧王、高延宗為安德王,此兩人都是哥哥高澄的兒子。

天保十年(559年),高洋驾崩(最长寿的北齐皇帝),虚龄34歲,葬於武寧陵,謚號為文宣皇帝,廟號為顯祖。后主天统初年(565年),有诏改谥景烈皇帝,庙号威宗。武平初年(570年),改回原来谥号。

高洋死後,北齊統治階級內部愈來愈混亂,最終為北周所滅。

興建高台時,曾單獨爬上最高處,居民看到紛紛膽跳心驚。並時常在街道裸露身體,儘管當時季節正處寒冬。

高洋有次喝醉酒,一氣之下說要將母親婁太后嫁給北方蠻族,母親氣着說自己怎會生出禽獸不如的兒子,高洋略為清醒,想逗母親開心,沒想到將母親摔傷。完全酒醒後,發現自己鑄成大錯,於是痛鞭自己,下決心戒酒,但是最後仍無法戒掉。

高洋曾經非常宠爱一名原為歌妓的薛嬪,容貌倾国,姿色万千,高洋和她如胶似漆、整日厮守在一起,但後來怀疑薛嬪曾与清河王高岳私通,有过暧昧关系,妒火中燒,命高岳自殺。薛嬪当时怀孕,分娩后,抽出匕首把薛嫔杀了,薛嬪遭斬殺肢解,並將頭顱置於自己衣袖裡面,回宮大宴賓客時,突然將人頭丟出,嚇的賓客四散,自己則取出薛嬪的大腿骨(髀骨)當作琵琶,邊流淚邊吟唱:“佳人難再得!”薛嫔出葬时,高洋披头散发,在车后步行跟随,大声哭号。

元魏宗室元韶,因娶高洋长姊某公主(即北魏永熙皇后),是高洋姐夫,有次高洋前去並詢問他:「為何漢朝可以中興?」元韶表示因為新朝沒把漢朝劉姓宗室殺光,於是高洋下令诛杀元魏宗室始平公元世道、东平公元景式等二十五家,其余十九家被囚禁,在东市斩杀七百二十一人,与其余所杀三千人一起投尸到漳水,元韶被囚后餓死。

高洋在位后期残暴不仁,荒淫无道,经常随意出入朝中官员府第,看见长得漂亮的女人就会色心大起,不分贵贱还是人妻,接着就是霸王硬上弓。除此之外,高洋还有观淫癖,征集坊间美女大批,弄入宫中后,然后脱个精光,命令侍从和卫士与这些女人群交,朝夕临视为乐。

\subsubsection{天保}

\begin{longtable}{|>{\centering\scriptsize}m{2em}|>{\centering\scriptsize}m{1.3em}|>{\centering}m{8.8em}|}
  % \caption{秦王政}\
  \toprule
  \SimHei \normalsize 年数 & \SimHei \scriptsize 公元 & \SimHei 大事件 \tabularnewline
  % \midrule
  \endfirsthead
  \toprule
  \SimHei \normalsize 年数 & \SimHei \scriptsize 公元 & \SimHei 大事件 \tabularnewline
  \midrule
  \endhead
  \midrule
  元年 & 550 & \tabularnewline\hline
  二年 & 551 & \tabularnewline\hline
  三年 & 552 & \tabularnewline\hline
  四年 & 553 & \tabularnewline\hline
  五年 & 554 & \tabularnewline\hline
  六年 & 555 & \tabularnewline\hline
  七年 & 556 & \tabularnewline\hline
  八年 & 557 & \tabularnewline\hline
  九年 & 558 & \tabularnewline\hline
  十年 & 559 & \tabularnewline
  \bottomrule
\end{longtable}


%%% Local Variables:
%%% mode: latex
%%% TeX-engine: xetex
%%% TeX-master: "../../Main"
%%% End:
