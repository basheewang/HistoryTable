%% -*- coding: utf-8 -*-
%% Time-stamp: <Chen Wang: 2021-11-01 15:14:48>

\subsection{幼主高桓\tiny(577)}

\subsubsection{生平}

高恒(570年-570年代577或578年)北齐最后一位皇帝,高纬兒子,母親穆黃花,後母馮小憐,史稱「幼主」。

当时北周不断进攻腐朽的北齐,齐军屡战屡败。577年正月一日,高纬禅位于自己的儿子高恒,改元“承光”,是为北齐幼主。正月廿一日,太上皇高緯再命幼主讓位給任城王高湝,他成為守國天王(或为宗国天王),但讓位的詔書未達高湝處,北齐京师邺(今河北临漳)已經沦陷,太上皇高緯與左皇后馮小憐逃離,幼主等10餘人骑马欲逃往南方的陈朝,但是刚刚走到青州(今山东青州)便被周军俘虏。

建德七年(578年)七月初二,他因被誣陷與宜州刺史穆提婆謀反而於八月被杀,得年8岁(《资治通鉴》将此事记于上年)。

\subsubsection{承光}

\begin{longtable}{|>{\centering\scriptsize}m{2em}|>{\centering\scriptsize}m{1.3em}|>{\centering}m{8.8em}|}
  % \caption{秦王政}\
  \toprule
  \SimHei \normalsize 年数 & \SimHei \scriptsize 公元 & \SimHei 大事件 \tabularnewline
  % \midrule
  \endfirsthead
  \toprule
  \SimHei \normalsize 年数 & \SimHei \scriptsize 公元 & \SimHei 大事件 \tabularnewline
  \midrule
  \endhead
  \midrule
  元年 & 577 & \tabularnewline
  \bottomrule
\end{longtable}


%%% Local Variables:
%%% mode: latex
%%% TeX-engine: xetex
%%% TeX-master: "../../Main"
%%% End:
