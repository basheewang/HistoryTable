%% -*- coding: utf-8 -*-
%% Time-stamp: <Chen Wang: 2019-12-23 15:43:45>


\section{北齐\tiny(550-577)}

\subsection{简介}

北齊(550年—577年)是中国北朝之鲜卑化汉人政权。550年6月9日(庚午年五月戊午日),由文宣帝高洋取代东魏建立,建國號齊,建元天保,遷都鄴城,以晉陽為別都。史稱北齊或後齊,以別於南齊。以皇室姓高,又稱高齊。北齊歷經文宣帝高洋、廢帝高殷、孝昭帝高演、武成帝高湛、后主高緯、幼主高恆六帝,577年被北周消滅,共享國二十八年。

北齊國勢本來頗為強盛,但由於北齊帝王多為殘暴昏庸之主,導致政治情勢混亂,國勢也日漸衰落。

后主時期,北周在北周武帝的统治下日渐兴盛,而北齐则衰落,更枉杀大将斛律光、高长恭。577年北周統一北方,北齊滅亡。北齊滅亡後,境内的士族大多遷到關中,成为北周臣民。范阳王高绍义逃奔突厥投靠他钵可汗。北齐营州刺史高宝宁不降周,奉高绍义为主继续抵抗。后来北周与突厥关系改善,580年,高绍义遭他钵可汗出卖被交给北周。

581年北周外戚楊堅篡位,建國號隋,583年消灭高宝宁势力,589年南下滅陳,結束中原自魏晉南北朝長達四百年的分裂局面。

北齐继承了东魏所控制的地区,占有今黄河下游流域的河北、河南、山东、山西以及苏北、皖北的广阔地区。同时与其并存的王朝有西魏、北周(取代西魏)、梁(含西梁、东梁)、陈(取代梁,但只占有前者部分领土)等。

北齐天保三年(552年)以后,北击库莫奚、东北逐契丹、西北破柔然,西平山胡(属匈奴族),南取淮南,势力一直延伸到长江边,这时北齐的国力达到鼎盛。北齐的农业、盐铁业、瓷器制造业都相当发达,是和与其鼎立的陈、北周三个国家中最富庶的。北齐继续推行均田制,大体上与北魏相同,但也略有变化。例如,北齐取消了受倍田的规定,不过一夫一妇的实际受田数仍相当于倍田,北魏对奴婢受田没有限制。北齐则按官品限制在300人至600人之间。另外还规定了赋税。

此外,魏收於此時編寫了《魏書》。

东魏和北齐初创之际,兵制继承北魏,兵民分离,鲜卑人为兵。在齐文宣帝时改革,军人出现汉人勇夫,但没有改变兵民、汉胡之分。

后在河清三年(564年),出现一种新的兵制,将当兵与种田结合起来,成为隋文帝改革府兵制的模板。

北齐人口盛时约2200万。

佛教及印度、中亞、西亞文化在本時期持續對藝術產生重大影響。部分中國史上最精緻的佛像座落於北齊的佛寺洞窟寺,這些佛像說明當時製作佛雕的工藝,以及北魏以來藝術風格的快速進展。一些大型陶雕源自北齊。北齊的陶器的特色包括雙色以上的釉色,白胎陶器亦於此時期發展。此時期繪畫品質極高,由太原的婁叡墓壁畫可見一斑。

北齐首都邺城繁华昌盛,布局有致,邺城之盛就在北齐时期。

%% -*- coding: utf-8 -*-
%% Time-stamp: <Chen Wang: 2019-12-23 15:42:57>

\subsection{文宣帝\tiny(550-559)}

\subsubsection{生平}

齐文宣帝高洋(526年-559年,在位550年—559年),字子進,鮮卑名侯尼干,勃海郡蓨县(今河北省衡水市景县)人。因其生于晋阳,又名晋阳乐,南北朝时期北齐开国皇帝,在位10年。他是东魏权臣高欢次子,北齐追尊文襄皇帝高澄的同母弟,鮮卑化漢人。

幼時其貌不揚,沉默寡言,其實大智若愚,聰慧過人,雖偶然被兄弟嘲笑或玩弄,但其才能甚得父親高欢欣賞。高澄被蘭京刺殺以後,高洋便牢牢地掌握了大權。東魏孝靜帝元善見只好封他為丞相、齐郡王,加九锡、殊礼。高洋不甘當傀儡皇帝的大臣,就於550年就廢掉了元善見,自立為帝,改元「天保」,建都鄴,北齊建立,年僅25歲。當年十一月,西魏宇文泰率大軍進攻剛剛建立的北齊,高洋親自率軍迎戰。宇文泰看到高洋手下的部隊軍容嚴整,嘆息道:「高歡不死矣。」隨即退軍。

在位初年,留心政務,削減州郡,整頓吏治,訓練軍隊,加強兵防,使北齊在很短的時間內強盛起來。高洋出兵進攻柔然、契丹、高句麗等國,都大獲全勝。高洋亦曾趁南朝梁遭遇侯景之乱意图拥立梁宗室萧退为梁帝,遭梁将王僧辩、陈霸先抵抗而未果,后又拥立萧渊明为梁帝并迫使王僧辩接受,但陈霸先袭杀王僧辩,废黜萧渊明,后即代梁建陈。

同時,北齊的農業、鹽鐵業、瓷器製造業都相當發達,是同陳、西魏鼎立的三個國家中最富庶的。

可是,他在即位六、七年後就腐敗起來,整日不理朝政,沉湎於酒色之中,他在都城鄴(今河南安陽)修築三台宮殿,十分豪華,動用十萬民夫,簡直是奢侈至極。

幾代北齊皇帝幾乎都有精神問題,加上高洋為人殘忍嗜殺,酗酒之後更常失去理智。

其實高洋自己也清楚酒後的荒唐行為,但無法改正。腐化的生活縮短了高洋的壽命。

天保五年(554年)八月,高洋任命尉粲為司徒、侯莫陳相為司空、清河王高岳為太保(上三公)、平陽王高淹為錄尚書事、常山王高演為尚書令、上黨王高渙為左僕射。

天保六年(555年)三月十六日,齐文宣帝高洋返回都城鄴城,封高孝珩為廣寧王、高延宗為安德王,此兩人都是哥哥高澄的兒子。

天保十年(559年),高洋驾崩(最长寿的北齐皇帝),虚龄34歲,葬於武寧陵,謚號為文宣皇帝,廟號為顯祖。后主天统初年(565年),有诏改谥景烈皇帝,庙号威宗。武平初年(570年),改回原来谥号。

高洋死後,北齊統治階級內部愈來愈混亂,最終為北周所滅。

興建高台時,曾單獨爬上最高處,居民看到紛紛膽跳心驚。並時常在街道裸露身體,儘管當時季節正處寒冬。

高洋有次喝醉酒,一氣之下說要將母親婁太后嫁給北方蠻族,母親氣着說自己怎會生出禽獸不如的兒子,高洋略為清醒,想逗母親開心,沒想到將母親摔傷。完全酒醒後,發現自己鑄成大錯,於是痛鞭自己,下決心戒酒,但是最後仍無法戒掉。

高洋曾經非常宠爱一名原為歌妓的薛嬪,容貌倾国,姿色万千,高洋和她如胶似漆、整日厮守在一起,但後來怀疑薛嬪曾与清河王高岳私通,有过暧昧关系,妒火中燒,命高岳自殺。薛嬪当时怀孕,分娩后,抽出匕首把薛嫔杀了,薛嬪遭斬殺肢解,並將頭顱置於自己衣袖裡面,回宮大宴賓客時,突然將人頭丟出,嚇的賓客四散,自己則取出薛嬪的大腿骨(髀骨)當作琵琶,邊流淚邊吟唱:“佳人難再得!”薛嫔出葬时,高洋披头散发,在车后步行跟随,大声哭号。

元魏宗室元韶,因娶高洋长姊某公主(即北魏永熙皇后),是高洋姐夫,有次高洋前去並詢問他:「為何漢朝可以中興?」元韶表示因為新朝沒把漢朝劉姓宗室殺光,於是高洋下令诛杀元魏宗室始平公元世道、东平公元景式等二十五家,其余十九家被囚禁,在东市斩杀七百二十一人,与其余所杀三千人一起投尸到漳水,元韶被囚后餓死。

高洋在位后期残暴不仁,荒淫无道,经常随意出入朝中官员府第,看见长得漂亮的女人就会色心大起,不分贵贱还是人妻,接着就是霸王硬上弓。除此之外,高洋还有观淫癖,征集坊间美女大批,弄入宫中后,然后脱个精光,命令侍从和卫士与这些女人群交,朝夕临视为乐。

\subsubsection{天保}

\begin{longtable}{|>{\centering\scriptsize}m{2em}|>{\centering\scriptsize}m{1.3em}|>{\centering}m{8.8em}|}
  % \caption{秦王政}\
  \toprule
  \SimHei \normalsize 年数 & \SimHei \scriptsize 公元 & \SimHei 大事件 \tabularnewline
  % \midrule
  \endfirsthead
  \toprule
  \SimHei \normalsize 年数 & \SimHei \scriptsize 公元 & \SimHei 大事件 \tabularnewline
  \midrule
  \endhead
  \midrule
  元年 & 550 & \tabularnewline\hline
  二年 & 551 & \tabularnewline\hline
  三年 & 552 & \tabularnewline\hline
  四年 & 553 & \tabularnewline\hline
  五年 & 554 & \tabularnewline\hline
  六年 & 555 & \tabularnewline\hline
  七年 & 556 & \tabularnewline\hline
  八年 & 557 & \tabularnewline\hline
  九年 & 558 & \tabularnewline\hline
  十年 & 559 & \tabularnewline
  \bottomrule
\end{longtable}


%%% Local Variables:
%%% mode: latex
%%% TeX-engine: xetex
%%% TeX-master: "../../Main"
%%% End:

%% -*- coding: utf-8 -*-
%% Time-stamp: <Chen Wang: 2021-11-01 15:13:54>

\subsection{废帝高殷\tiny(559-560)}

\subsubsection{生平}

高殷(545年-561年;在位559年—560年),字正道,北齊第二代皇帝,齊文宣帝嫡長子,母亲是昭信皇后李祖娥。

天保元年(550年),立為皇太子,時年六歲。性敏慧,但偏于柔懦,高洋对此不满,曾打算立次子太原王高紹德。

天保十年(559年),高殷即位,時年十五歲。即位後,以咸陽王斛律金為左丞相,叔父錄尚書事、常山王高演為太傅,叔父司徒、長廣王高湛為太尉,司空段韶為司徒,平陽王高淹為司空,高陽王高湜為尚書左僕射,河間王高孝琬為司州牧,侍中燕子獻為右僕射共同执政,勵精圖治,對民生極為關心,曾分命使者巡省四方,求政得失,省察風俗,問人疾苦;整頓吏治,政治清明;武官年逾六旬皆放免,軍事上淘汰老弱,留下精壯,軍力大增;下詔減徭役,使由天保朝國勢的危急有紓緩。然高演在位高权重兢兢业业之余也开始覬覦皇位,且引起了朝中反对派的不满。終於560年,太后李祖娥等人与高演等人的矛盾白热化,其六叔高演發動政變,高殷被废为濟南王。

高殷被废的时候,其祖母娄昭君命高演发誓决不伤害高殷性命,但最终高演还是虑有后患,于次年将高殷秘密殺害。高殷死時十七歲,谥号「愍悼」,没有关于子女的记载。

\subsubsection{乾明}

\begin{longtable}{|>{\centering\scriptsize}m{2em}|>{\centering\scriptsize}m{1.3em}|>{\centering}m{8.8em}|}
  % \caption{秦王政}\
  \toprule
  \SimHei \normalsize 年数 & \SimHei \scriptsize 公元 & \SimHei 大事件 \tabularnewline
  % \midrule
  \endfirsthead
  \toprule
  \SimHei \normalsize 年数 & \SimHei \scriptsize 公元 & \SimHei 大事件 \tabularnewline
  \midrule
  \endhead
  \midrule
  元年 & 560 & \tabularnewline
  \bottomrule
\end{longtable}


%%% Local Variables:
%%% mode: latex
%%% TeX-engine: xetex
%%% TeX-master: "../../Main"
%%% End:

%% -*- coding: utf-8 -*-
%% Time-stamp: <Chen Wang: 2021-11-01 15:13:58>

\subsection{孝昭帝高演\tiny(560-561)}

\subsubsection{生平}

齊孝昭帝高演(535年-561年;在位560年—561年),字延安,北齐第三任皇帝。他是東魏权臣高欢第六子,文宣帝同母弟,在位一年。

高演長於政術,善於理解事情的細節;天保朝起開始干預朝政,政治經驗逐漸成熟豐富,眼見次兄齊文宣帝沉湎酒色,大臣趨炎附勢,惟高演滿臉憂愁,不時直諫。其兄文宣帝臨終時,表示必要时皇位可以相让,唯不可伤害高殷。廢帝即位,獨攬朝政。560年,高演發動政變,废高殷為濟南王。高演登上皇帝寶座,改元皇建,時年二十六歲。

高演在位期間,文治武功兼盛,『帝留心於政事,積極尋求及任用賢能為朝廷效力,政治清明;帝關心民生,輕徭薄賦,並下詔分遣大使巡省四方,觀察風俗,問人疾苦,考求得失。並親征親戎北討庫莫奚,出長城,虜奔遁,分兵致討,大獲牛馬。』在北齊28年歷史和六帝之中,只有孝昭帝稱得上是明君,可惜他在位時間不長,即位翌年,高演便因墮马事故重伤而死,在位僅兩年,終年僅27歲。

高殷被废的时候,娄昭君命儿子高演发誓决不伤害孙子高殷性命,但最终高演还是虑有后患,于次年将高殷秘密殺害。不久高演即出了意外,传说是齊文宣帝的厉鬼复仇。娄昭君亦对此深感悲愤,不肯原谅高演。為了保住兒子高百年,临终时候高演宣布废掉年幼的太子,傳位於弟弟長廣王高湛。他的谥号为孝昭皇帝,廟號肃宗。

\subsubsection{皇建}

\begin{longtable}{|>{\centering\scriptsize}m{2em}|>{\centering\scriptsize}m{1.3em}|>{\centering}m{8.8em}|}
  % \caption{秦王政}\
  \toprule
  \SimHei \normalsize 年数 & \SimHei \scriptsize 公元 & \SimHei 大事件 \tabularnewline
  % \midrule
  \endfirsthead
  \toprule
  \SimHei \normalsize 年数 & \SimHei \scriptsize 公元 & \SimHei 大事件 \tabularnewline
  \midrule
  \endhead
  \midrule
  元年 & 560 & \tabularnewline\hline
  二年 & 561 & \tabularnewline
  \bottomrule
\end{longtable}


%%% Local Variables:
%%% mode: latex
%%% TeX-engine: xetex
%%% TeX-master: "../../Main"
%%% End:

%% -*- coding: utf-8 -*-
%% Time-stamp: <Chen Wang: 2019-12-23 15:45:27>

\subsection{武成帝\tiny(561-565)}

\subsubsection{生平}

齊武成帝高湛(537年-569年1月13日),小字步落稽,北齐第四位皇帝,561-565年在位,在位四年。東魏權臣高歡第九子,孝昭帝高演之同母弟。

高湛幼時亦其得父親高歡喜愛。北齊建國後,被齊文宣帝封為長廣王。高湛協助其兄高演發動政變廢黜了侄兒高殷。齊孝昭帝高演繼位後,甚為寵信他。後高演傷病兼身,臨終時為了不讓自己的兒子高百年落得與高殷一樣的命運,決定傳位於弟。561年,高湛繼位,改元太寧,是為武成帝。

武成帝昏庸无能,沉湎于美色之中,不思國事,北齊岌岌可危。565年,傳位於太子高緯,自任太上皇,繼續在幕後主政。最后也因为酒色过度而死,年僅三十二歲。年號太寧、河清,谥号武成帝,庙号世祖,葬於永平陵。

高洋皇后李祖娥其子高殷即位,但只有一年便被其叔高演所篡。高演即位為孝昭帝後,將她由皇太后降為昭信皇后,居於昭信宮。

後來高湛繼位為武成帝後,逼李皇后與之相姦。高湛恐嚇她:「如果妳敢不從,我就殺妳兒子。」李皇后因害怕而答應他,從此頗受寵愛。她懷孕的時候,兒子太原王高紹德到她的宮殿,她避不見面,高紹德便怒言:「妳當做兒子的不知道嗎?您是因為肚子大了,所以才不見兒子吧。」李皇后羞愧,等到生下一個女兒,含羞將其掐死。

高湛見女兒被害,怒不可遏,將高紹德捉到宮里,舉刀怒喝:「妳殺我的女兒,我就殺妳的兒子!」高紹德驚慌求饒,高湛又罵高紹德:「想當年我被你父親毒打,你也沒來救過我!」當場將高紹德殺死。李皇后當場大哭起來,高湛更是憤怒,將她衣服脫光,胡亂鞭打,讓她哭天喊地不已。最後將她裝在絹袋裡,也不管她鮮血淋漓,就丟到渠道裡,任水漂流,許久才甦醒。用牛車送到妙勝寺出家為尼。北齊滅亡後入關內,隋朝時才得以送還故鄉。

齐武成帝高湛在位期间昏庸无道,荒淫无度,终日沉迷于女色,再加上好大喜功,是导致北齐逐渐走入灭亡的主因。

\subsubsection{太宁}

\begin{longtable}{|>{\centering\scriptsize}m{2em}|>{\centering\scriptsize}m{1.3em}|>{\centering}m{8.8em}|}
  % \caption{秦王政}\
  \toprule
  \SimHei \normalsize 年数 & \SimHei \scriptsize 公元 & \SimHei 大事件 \tabularnewline
  % \midrule
  \endfirsthead
  \toprule
  \SimHei \normalsize 年数 & \SimHei \scriptsize 公元 & \SimHei 大事件 \tabularnewline
  \midrule
  \endhead
  \midrule
  元年 & 561 & \tabularnewline\hline
  二年 & 562 & \tabularnewline
  \bottomrule
\end{longtable}

\subsubsection{河清}

\begin{longtable}{|>{\centering\scriptsize}m{2em}|>{\centering\scriptsize}m{1.3em}|>{\centering}m{8.8em}|}
  % \caption{秦王政}\
  \toprule
  \SimHei \normalsize 年数 & \SimHei \scriptsize 公元 & \SimHei 大事件 \tabularnewline
  % \midrule
  \endfirsthead
  \toprule
  \SimHei \normalsize 年数 & \SimHei \scriptsize 公元 & \SimHei 大事件 \tabularnewline
  \midrule
  \endhead
  \midrule
  元年 & 562 & \tabularnewline\hline
  二年 & 563 & \tabularnewline\hline
  三年 & 564 & \tabularnewline\hline
  四年 & 565 & \tabularnewline
  \bottomrule
\end{longtable}


%%% Local Variables:
%%% mode: latex
%%% TeX-engine: xetex
%%% TeX-master: "../../Main"
%%% End:

%% -*- coding: utf-8 -*-
%% Time-stamp: <Chen Wang: 2021-11-01 15:14:23>

\subsection{後主高纬\tiny(565-576)}

\subsubsection{生平}

高纬(556年5月29日-577年11月),字仁纲,南北朝时期北齐第五位皇帝(565年-577年在位),史稱「後主」,北齐武成帝高湛的嫡長子,亦是中國唯一一位無上皇。

高纬与庶兄高绰同日出生,实为高湛次子,但是因为是嫡出,故被视为长子。

高纬即位时,腐朽的北齐政权已经摇摇欲坠,他自己仍然荒淫无道,杀害兄长高绰,导致北齐军队衰弱,政治腐败,尤其最大致命伤是诛杀名将高長恭、斛律光,这使得北齐失去得以抗击北周侵略的有能将领。

577年,北周来攻打北齐,占领晋阳,齐军大败,周军不久破北齐京师邺(今河北临漳),高纬慌忙将皇位传于自己8岁的儿子高恒,然后带着幼主高恒等十余人骑马准备投降江南的陈朝。他们刚逃到青州(今山东青州)就被周军俘虏了,北齐灭亡。高纬投降后,被周武帝封温国公,不久因为被诬陷谋反,而被武帝赐死,终年22岁(《北齊書》将此事记于下年)。

讽刺的是,高纬被俘后,竟对北周武帝宇文邕要求將馮小憐归还给他。周帝說:「朕對於天下,就像脫掉鞋子一樣輕視,一個老太婆有甚麼好跟您爭的呢?」後主寵愛馮小憐,李商隱曾寫詩諷刺道:

「小憐玉體橫陳夜,已報周師入晉陽」—北齊兩首(其一)第二聯

「晋阳已陷休回顾,更请君王猎一围」—北齊兩首(其二)第二聯

这两首诗說明後主在北周入侵時仍然不理政事,荒唐、淫亂。

\subsubsection{天统}

\begin{longtable}{|>{\centering\scriptsize}m{2em}|>{\centering\scriptsize}m{1.3em}|>{\centering}m{8.8em}|}
  % \caption{秦王政}\
  \toprule
  \SimHei \normalsize 年数 & \SimHei \scriptsize 公元 & \SimHei 大事件 \tabularnewline
  % \midrule
  \endfirsthead
  \toprule
  \SimHei \normalsize 年数 & \SimHei \scriptsize 公元 & \SimHei 大事件 \tabularnewline
  \midrule
  \endhead
  \midrule
  元年 & 565 & \tabularnewline\hline
  二年 & 566 & \tabularnewline\hline
  三年 & 567 & \tabularnewline\hline
  四年 & 568 & \tabularnewline\hline
  五年 & 569 & \tabularnewline
  \bottomrule
\end{longtable}

\subsubsection{武平}

\begin{longtable}{|>{\centering\scriptsize}m{2em}|>{\centering\scriptsize}m{1.3em}|>{\centering}m{8.8em}|}
  % \caption{秦王政}\
  \toprule
  \SimHei \normalsize 年数 & \SimHei \scriptsize 公元 & \SimHei 大事件 \tabularnewline
  % \midrule
  \endfirsthead
  \toprule
  \SimHei \normalsize 年数 & \SimHei \scriptsize 公元 & \SimHei 大事件 \tabularnewline
  \midrule
  \endhead
  \midrule
  元年 & 570 & \tabularnewline\hline
  二年 & 571 & \tabularnewline\hline
  三年 & 572 & \tabularnewline\hline
  四年 & 573 & \tabularnewline\hline
  五年 & 574 & \tabularnewline\hline
  六年 & 575 & \tabularnewline\hline
  七年 & 576 & \tabularnewline
  \bottomrule
\end{longtable}

\subsubsection{隆化}

\begin{longtable}{|>{\centering\scriptsize}m{2em}|>{\centering\scriptsize}m{1.3em}|>{\centering}m{8.8em}|}
  % \caption{秦王政}\
  \toprule
  \SimHei \normalsize 年数 & \SimHei \scriptsize 公元 & \SimHei 大事件 \tabularnewline
  % \midrule
  \endfirsthead
  \toprule
  \SimHei \normalsize 年数 & \SimHei \scriptsize 公元 & \SimHei 大事件 \tabularnewline
  \midrule
  \endhead
  \midrule
  元年 & 576 & \tabularnewline
  \bottomrule
\end{longtable}


%%% Local Variables:
%%% mode: latex
%%% TeX-engine: xetex
%%% TeX-master: "../../Main"
%%% End:

%% -*- coding: utf-8 -*-
%% Time-stamp: <Chen Wang: 2019-12-23 15:46:59>

\subsection{安德王\tiny(576)}

\subsubsection{生平}

高延宗(544年-577年),勃海郡蓨县(今河北省衡水市景县)人,追尊齐文襄帝高澄第五子,母亲为姬妾陈氏,原本是北魏广阳王的家妓。

高延宗幼时,父亲高澄在547年被杀,就被他二叔文宣帝高洋抚养,宠得不成样子,高延宗十二岁,高洋还让他骑在自己的肚腹上,在肚脐里撒尿。抱着他说:“可怜止有此一个。”高洋问高延宗想做什么王,高延宗:“想作冲天王。”高洋于是问杨愔,杨愔答道:“天下无此郡名,愿使安于德。”于是封为安德王。高延宗骄傲恣睢,时常以肮脏花样折腾臣下,及至高洋去世,继位的叔叔高演看不过,命人捉高延宗来打了一百棍,自此高延宗稍有收敛。

武成帝高湛杀河间王高孝琬,高延宗痛哭不已。576年,后主高纬南逃,高延宗留守晋阳,被部下拥立为帝,改元德昌,军心登时大振,反败为胜,几乎将北周武帝宇文邕活捉。但仅两天就因得胜后懈怠防备被卷土重来的周军打败,周军攻克东门、南门,高延宗战不利,出逃到城北民宅被追上俘获。宇文邕亲自下马抓住高延宗的手,高延宗拒绝:“我是死人,我的手怎能接触至尊!”宇文邕说:“我们是两国天子(即承认了高延宗的北齐皇帝身份),彼此之间并非有仇怨,都是老百姓而来的,我终究不会加害于您,不必害怕。”让他重新穿戴好衣帽,以礼相待。后来宇文邕又问高延宗如何夺取邺城,高延宗推辞说这不是亡国之臣所能回答的,宇文邕强迫他回答,他才说:“如果任城王高湝援救邺城,臣下不知北齐能否坚持。如果是后主高纬自己守卫邺城,那么陛下可以兵不血刃就取得胜利。”

577年周武帝带高纬等回长安,与北齐君臣一起饮酒,席间让高纬跳舞,高延宗哭得不能自己,屡次想服毒自杀,被婢女劝止。同年,周武帝称高纬勾结穆提婆谋反,赐高纬及大量北齐皇室成员自尽。高氏皇族多自陈无辜,只有高延宗捋起衣袖哭着不说话。高延宗被以椒塞口而死。次年,妻李氏收葬了他的尸体。

\subsubsection{德昌}

\begin{longtable}{|>{\centering\scriptsize}m{2em}|>{\centering\scriptsize}m{1.3em}|>{\centering}m{8.8em}|}
  % \caption{秦王政}\
  \toprule
  \SimHei \normalsize 年数 & \SimHei \scriptsize 公元 & \SimHei 大事件 \tabularnewline
  % \midrule
  \endfirsthead
  \toprule
  \SimHei \normalsize 年数 & \SimHei \scriptsize 公元 & \SimHei 大事件 \tabularnewline
  \midrule
  \endhead
  \midrule
  元年 & 576 & \tabularnewline
  \bottomrule
\end{longtable}


%%% Local Variables:
%%% mode: latex
%%% TeX-engine: xetex
%%% TeX-master: "../../Main"
%%% End:

%% -*- coding: utf-8 -*-
%% Time-stamp: <Chen Wang: 2021-11-01 15:14:48>

\subsection{幼主高桓\tiny(577)}

\subsubsection{生平}

高恒(570年-570年代577或578年)北齐最后一位皇帝,高纬兒子,母親穆黃花,後母馮小憐,史稱「幼主」。

当时北周不断进攻腐朽的北齐,齐军屡战屡败。577年正月一日,高纬禅位于自己的儿子高恒,改元“承光”,是为北齐幼主。正月廿一日,太上皇高緯再命幼主讓位給任城王高湝,他成為守國天王(或为宗国天王),但讓位的詔書未達高湝處,北齐京师邺(今河北临漳)已經沦陷,太上皇高緯與左皇后馮小憐逃離,幼主等10餘人骑马欲逃往南方的陈朝,但是刚刚走到青州(今山东青州)便被周军俘虏。

建德七年(578年)七月初二,他因被誣陷與宜州刺史穆提婆謀反而於八月被杀,得年8岁(《资治通鉴》将此事记于上年)。

\subsubsection{承光}

\begin{longtable}{|>{\centering\scriptsize}m{2em}|>{\centering\scriptsize}m{1.3em}|>{\centering}m{8.8em}|}
  % \caption{秦王政}\
  \toprule
  \SimHei \normalsize 年数 & \SimHei \scriptsize 公元 & \SimHei 大事件 \tabularnewline
  % \midrule
  \endfirsthead
  \toprule
  \SimHei \normalsize 年数 & \SimHei \scriptsize 公元 & \SimHei 大事件 \tabularnewline
  \midrule
  \endhead
  \midrule
  元年 & 577 & \tabularnewline
  \bottomrule
\end{longtable}


%%% Local Variables:
%%% mode: latex
%%% TeX-engine: xetex
%%% TeX-master: "../../Main"
%%% End:


\subsection{献武帝简介}

高歡(496年-547年2月25日),勃海郡蓨县人(今河北衡水市景县),鲜卑化汉族,鲜卑名賀六渾,为北魏、東魏權臣,也是北齐政权的奠基者。追尊為「神武帝」。

高欢六世祖高隱,西晋玄菟太守。高隐之子高庆,高庆之子高泰,高泰之子高湖,三世仕慕容氏。高湖的儿子高谧被流放懷朔鎮,后世居于此,家族被鮮卑化。高欢雖自称為漢人,但据史载“累世北边,故习其俗,遵同鲜卑”。

《北齐书·神武上》记载他“目有精光,長頭高顴,齒白如玉,少有人傑表。深沉有大度,輕財重士,廣結士人,為豪俠所宗。”高欢的母亲韩期姬是高樹生的正室,在他出生后不久即去世,高樹生将他交给高欢姐姐高娄斤和姐夫尉景抚养长大。在六镇起义爆发后,先后投靠杜洛周、葛荣,后来投奔爾朱荣。他向爾朱榮提出讨伐胡太后亲信郑俨、徐纥而清君侧,受爾朱榮赏识。在河阴之变后,爾朱榮掌握朝政,高欢被封为晋州刺史。

後來北魏孝莊帝殺死爾朱榮,爾朱家族起兵讨伐孝莊帝,孝莊帝战败被杀。爾朱家族立长广王元晔为帝。高歡却没有参与这次行动。后来他设法说服爾朱兆派他统帅镇压六镇之乱得到的降兵,并带领他们前往河北。

爾朱家族残暴不仁,高欢遂产生讨伐爾朱家族的想法。在此期间,爾朱兆听从慕容绍宗的建议,企图一举把高欢解决。但高欢深藏不露,使得爾朱兆与他结为兄弟,不设防备。爾朱度律废元晔,立节闵帝,封高欢为渤海王,并征其入朝。高欢清楚其中有诈,拒不接受。不久之后,高欢在信都起兵,立元朗为帝,正式讨伐爾朱氏。经过一年的战斗,高欢击败了爾朱兆、爾朱世隆、爾朱彥伯、爾朱天光、爾朱度律、爾朱仲远等人,掌握了政权。慕容绍宗归降,被高欢重用。

高欢以元朗世系疏远不是皇帝之选,有心另立皇帝。他最初有意奉戴节闵帝,派仆射魏兰根观察节闵帝为人。但节闵帝神采高明,魏兰根怕日后难制,于是与高乾兄弟及黄门侍郎崔㥄以节闵帝系尔朱氏所立,一旦奉戴则当初起兵无名为由,说服高欢废帝。高欢又因汝南王元悦是北魏孝文帝子,认为他可以继位,于是告知元悦自己有意拥立,但元悦性行轻狂,举止多有过失,高欢於是也放弃了拥立他的打算,废节闵帝和元朗,立孝文帝之孫元修為北魏孝武帝,而将节闵帝囚禁于佛寺。高歡被授大丞相、天柱大将军、太师、世袭定州刺史,增封并前十五万户,辞天柱大将军,减户五万。高欢独揽大权,使孝武帝非常不满。孝武帝联合贺拔岳试图牵制高欢的势力。高欢亲信司空高乾密奏高欢孝武帝有二心,结果被孝武帝杀掉。高欢哭着说:“天子枉害司空!”两人关系迅速恶化(亦有说高乾代表的汉人豪族势力本非高欢嫡系,其死亦有被高欢故意出卖借刀杀人的成分)。高欢命令侯莫陈悦干掉贺拔岳,并派侯景去接收贺拔岳的部队。不料,贺拔岳的部下奉戴宇文泰为主,侯景无功而返。宇文泰用为贺拔岳报仇的名义起兵,并发檄文讨伐高欢。

孝武帝終於在534年逃往關中投靠宇文泰,而高歡另立元善見為孝靜帝,遷都鄴(今河北臨漳西南),史稱東魏,由高歡任相。当年12月,宇文泰杀孝武帝,立元宝炬为帝,定都长安,史称西魏。东西魏对峙的局面形成。

天平四年(537年)春,高欢、高昂、窦泰分三路进攻西魏。窦泰进攻潼关,宇文泰故意示弱,率精锐出潼关左面的小关,攻其不备,东魏军大败,大将窦泰自杀。高欢被迫撤军。

十月,高歡率兵二十萬至蒲津(今山西永濟縣一帶)攻打西魏,志在為竇泰復仇,高歡命令高昂領兵三萬出河南。時關中大饑,宇文泰所將不滿萬人。東魏右長史薛琡提議堅守糧道,不可渡河野戰;侯景也勸高歡分成二軍,相繼而進,但高歡不接受建議。後高歡渡河至馮翊城下,西魏華州刺史王羆有備,不可攖其鋒,乃涉洛水,軍於許原西。宇文泰至渭南,徵諸州兵馬,諸將認為眾寡不敵,請求緩進,不許。宇文泰令造浮橋於渭河,軍隊備有三日糧食,以輕騎渡渭河,至沙苑(今陝西大荔南,洛、渭之間)距東魏軍僅六十里。宇文泰採用李弼的計謀,列陣於渭曲,又命將士將武器藏在蘆葦中,候聞鼓聲而起。不久,高歡遣東魏兵至,見西魏兵少人乏,於是兵馬輕敵冒進,一時行伍亂次。宇文泰遂鳴鼓擊之,于謹等六軍與之合戰,李弼率鐵騎橫擊,東魏兵潰散敗北,喪兵七萬。這時李穆獻計:「高歡膽破矣,逐之可獲。」宇文泰不聽,還軍渭南,這時所徵諸州之兵剛到前線,宇文泰命令士兵每人種樹一株,以旌武功。李弼等十二大將,以功進爵,史稱“沙苑之戰”。

公元538年,高欢部将侯景夺回洛阳金墉城,宇文泰率军救援,一开始东魏气势如虹,宇文泰战马中箭,把宇文泰甩在地上,结果宇文泰差点被俘虏。但不久后西魏军重整旗鼓,侯景被击败,高昂率军追击宇文泰,战败被斩。此战双方打平,但高欢痛失一员大将。

公元543年,高昂的哥哥高仲密以北豫州投降西魏,高欢率十万大军讨伐,宇文泰率军救援。高欢大将彭乐以数千骑兵冲入西魏北军,取得很大胜利,高欢鸣鼓进击,斩首三万余级。高欢派彭乐追击宇文泰。宇文泰狼狈不堪,向彭乐哀求:“彭将军你太傻了!今天你杀掉我,明天你还有用吗?何不还营,把我丢下的金银宝物取走呢?”彭乐闻讯便不再追击,回去跟高欢报告:“宇文泰侥幸逃跑,已经心惊胆战!”高欢听说彭乐放走大敌,气得要命,却无可奈何。

隔日,双方重整旗鼓再战。这一次西魏占了上风,东魏战败,高欢被迫撤退。宇文泰命令贺拔胜率三千兵马追击高欢,贺拔胜的兵器几乎都击到了高欢,贺拔胜边追边喊:“贺六浑,我贺拔破胡(贺拔胜的表字)今天一定宰了你!”所幸高欢部下射死贺拔胜坐骑,这才顺利脱险。高欢回军后,下令把贺拔胜留在东魏的几个儿子统统杀掉,贺拔胜郁郁而终。

武定四年(546年),高歡率十萬大軍在玉璧(山西稷山)與宇文泰交戰,西魏守将韦孝宽积极防守,高欢无懈可击。东魏苦攻玉壁五十多天,战死病死七万多人,高歡因忧愤生病,被迫撤退。西魏造謠高歡中箭病危,高歡回師途中帶病召集群臣,請斛律金高歌〈敕勒歌〉一首:“敕勒川,陰山下,天似穹廬,籠蓋四野。天蒼蒼,野茫茫,風吹草低見牛羊。”曲中高歡親自和唱,哀慟流淚。

武定五年春正月朔(547年2月6日),发生了日食,高欢说:“日食是为了我吗,死了又有什么遗憾。”正月丙午(547年2月25日),高欢向魏孝静帝启禀陈说,当日,高欢在晋阳去世,虚岁五十二,葬于义平陵(不过据《资治通鉴》记载,义平陵是高欢的衣冠冢,实陵潜葬于鼓山石窟)。

高欢之子高洋篡魏登基后,追尊高欢为太祖獻武皇帝,後主時,又改為高祖神武皇帝。

高歡善於玩弄權術,足智多謀,精通權宜之計。从他替爾朱榮出谋划策,到后来击破掌权的爾朱家族都显示了这一点。另外,高欢临终前嘱咐儿子高澄,指出侯景必然造反,但只要用慕容绍宗为将就可讨平。结果不出高欢所料。高歡用人惟才是用,為北齊立國打下了堅固的基礎。

然而,高欢野心太大,未能处理好与孝武帝的关系,致使孝武帝出奔宇文泰,最终造成东西魏对峙之局。而且,高欢控制的东魏实力虽远强于西魏,但他在戰術不及宇文泰,导致他终其一生未能统一北方。高欢亦教子无方,他身后的北齐政权暴君和昏君輩出,朝政混乱,最终被宇文氏的北周消灭。

爾朱榮認識高歡時,對高歡能讓馬乖乖站著讓他清洗,十分驚訝,高歡表示強硬手段才是唯一方法,爾朱榮對他記憶十分深刻,開始拔擢他。後來,高歡幾個兒子有次面對一團繩索難解,其中次子高洋一刀砍斷,高歡十分高興。此為「快刀斬亂麻」一語由來。

\subsection{文襄帝简介}

高澄(521年-549年),字子惠,北魏、东魏权臣高欢之子,父亲死后,任大丞相。北齐追尊为文襄帝。

高澄是高歡与娄昭君所生的長子,生来就非常聪明,对政事有独到见解,自幼深得父亲喜爱和重用。北魏中兴元年(531年),立为渤海王世子。10岁时曾独自出马为高欢招降高敖曹。11岁时以高欢特使的身份两次去洛阳朝觐孝武帝元修。中兴二年(532年),加侍中、开府仪同三司,尚孝静帝妹冯翊公主,史书中赞叹他神情俊爽,恍若成人。天平元年(534年),加使持节、尚书令、大行台、并州刺史。天平三年(536年),入辅朝政,加领左右、京畿大都督。朝臣们虽听说高澄年轻老成,有风度、有见识,但总觉得他是个少年,心里并不服气。当看到他驾驭全局,有胆略、有气魄,在朝堂上做宰相时听断如流,处理问题及时妥切,不由得个个心悦诚服。元象元年(538年),高澄兼任吏部尚书。 兴和二年(540年),高澄加大将军,领中书监,仍代理吏部尚书。北魏从崔亮开始挑选官员就论资排辈,不按才能选取。高澄废除了这一个制度,开始根据才能名望挑选官员,亲自写书征召各地有才学有名望的士子为朝廷效力。当时品德好、有本事的人,都得到了提拔重用,有的一时安排不了相应的位置,高澄就将他们召为宾客,在自己府中供养起来,有时间便与他们一起游园娱乐赋诗,使这些人各得其所,各尽所长。

自从河阴之变后,尔朱荣为了安定朝中人心,上奏滥封官爵。赠荫一事,渐渐变得杂滥无章,平庸无能的官员动辄高官厚禄,被有识之士所非议。武定年间,高澄开始纠正其过失,使得追赠褒扬渐有章法。高澄推荐铁面无私的崔暹为御史中尉,严厉打击那些无法无天的贪官污吏,尤其是窃据高位的权贵,有许多人被绳之以法。官场风气大有改观,人心为之一振。兴和三年(541年),高澄在麟趾阁和群臣编纂议定了律法《麟趾格》,并颁布天下。《麟趾格》是《北齐律》的蓝本,又是隋唐律法的直接渊源,影响一直波及后世。

在高澄的主持下,朝廷将治国的政策书于榜上,公开张贴在街头,供天下百姓自由评论,发表意见。对那些提出建议或批评时事的人,都给予优厚的待遇,即使言过其实或言辞激烈,也予以宽容,不加罪责。由于百姓的称赞,高澄的威望更加上升。在这段时期内,东魏与南方的梁朝关系比较和睦,双方的使节往来频繁。然而,为了显示各自的“国威”,东魏与南朝梁的使节都竭力在言辞、才学方面争锋,常常出现热烈辩论的场面。无论是梁使至邺城(今河北临漳),还是魏使至建康,都是如此,久而成为惯例。高澄则乐于此道,每当设宴招待梁使,高澄或者亲自到场,或者派遣属下与会。凡是东魏方面有所妙论、他都兴奋异常,为之鼓掌助威。他也因此召揽了一大批文人学士.或罗致门下,以为宾客;或推荐给朝廷,出任各级官吏。

东魏兴和三年(541年),有雀衔永安五铢置于高欢座前,高澄令百炉别铸此钱,又称“令公百炉”钱。北魏末年战乱,导致经济紊乱、货币贬值,民间私铸大量假钱。高澄在武定初年开始改革这项弊政,令人前往全国各地,将铸钱用的铜和原有的钱币收集起来,重新铸造。然而民间偷铸假钱的情况仍然屡禁不绝。因此高澄在武定六年(548年)进行新的货币改革,改用悬秤五铢。 东魏武定六年(548年)所铸永安五铢,号称“重如其文”,是一种足重货币。它的铸造是魏晋南北朝货币史上由乱到治的转折点,是后世足重货币“开皇五铢”的先驱。为促进足重货币的流通,高澄还采取了强硬的手段,《魏书·食货志》:“计百钱重一斤四两二十铢,自余皆准此为数。其京邑二市、天下州镇郡县之市,各置二称,悬于市门,私民所用之称,皆准市称以定轻重。” 由于武定六年永安五铢,曾被作为标尺,悬在市场的门上,以称量入市货币的轻重。因此在钱币学上,一般也将武定六年的永安五铢称之为“悬称五铢”。

高歡在547年死後,高澄繼任大丞相,都督中外諸軍,坐鎮晉陽。美姿容,善言笑,氣度高華,聰明過人,愛士好賢,爽直義氣。但又傲慢氣盛,性格暴烈,情慾豪侈,任性恣睢。與高欢之妾原魏廣平王妃鄭大車通姦,高歡死後,其次妻柔然(蠕蠕)的公主,按照柔然習俗,蠕蠕公主改嫁給高澄。親信崔季舒指稱薛置書的夫人元氏甚美,高澄把元氏騙到府中予以姦淫,元氏痛斥高澄是人面獸心。崔季舒將她移送廷尉府治罪,廷尉陸操以無罪釋之。

孝靜帝曾在打猎时骑马疾驰,就被监卫都督乌那罗受工伐劝止,理由是高澄会不悦;孝静帝不满高澄掌权,在被高澄举大酒杯敬酒時说“自古无不亡之国,朕活着有什么意思”,高澄就怒罵道:“朕,朕,狗腳朕!”並令中书黄门郎崔季舒打了他三拳。孝靜帝不堪忧辱,咏谢灵运之诗:“韩亡子房奋,秦帝仲连耻。本自江海人,忠义动君子。”侍讲大臣荀济和尚书祠部郎中元瑾、长秋卿刘思逸、华山王元大器、淮南王元宣洪及济北王元徽等商量,要想辦法除掉高澄,即在皇宫日夜挖掘通往城外的秘密通道。事機不密,被高澄得知,高澄马上带兵直闯进宫,直斥孝静帝图谋造反,虽然当时被孝静帝驳斥得痛哭谢罪并一起痛饮到深夜,但仅三天后,高澄就把孝静帝幽禁在含章堂,将荀济等人在市场上烹杀。549年,高澄计划夺取东魏政权,却在邺城(今河北临漳邺镇一带)被家中廚子蘭京暗杀刺死,享年僅29歲。

550年,其弟高洋正式稱帝,為北齊文宣帝。高洋追尊高澄为文襄皇帝,庙号世宗。

《北史·齐本纪上第六》:“文襄嗣膺霸道,威略昭著。内除奸逆,外拓淮夷,摈斥贪残,存情人物。而志在峻法,急于御下,于前王之德,有所未同。盖天意人心,好生恶杀,虽吉凶报应,未皆影响。总而论之,积善多庆。”


%%% Local Variables:
%%% mode: latex
%%% TeX-engine: xetex
%%% TeX-master: "../../Main"
%%% End:
