%% -*- coding: utf-8 -*-
%% Time-stamp: <Chen Wang: 2018-07-10 00:44:24>

\section{汉高祖\tiny(BC206-BC195)}

\begin{longtable}{|>{\centering\scriptsize}m{2em}|>{\centering\small}m{2em}|>{\centering}m{8.3em}|}
  % \caption{秦王政}\
  \toprule
  \SimHei \normalsize 年数 & \SimHei \normalsize 公元 & \SimHei 大事件 \tabularnewline
  % \midrule
  \endfirsthead
  \toprule
  \SimHei 年数 & \SimHei 公元 & \SimHei 大事件 \tabularnewline
  \midrule
  \endhead
  \midrule
  元年 & -206 & \begin{enumerate}
    \tiny
  \item 秦朝灭亡。
  \item 鸿门宴。
  \item 项羽建立西楚王朝,自称西楚霸王。
  \end{enumerate} \tabularnewline\hline
  二年 & -205 & \begin{enumerate}
    \tiny
  \item 彭城之战。
  \item 成皋之战。
  \item 韩信破代、赵。
  \item 韩信灭燕、齐。
  \end{enumerate} \tabularnewline\hline
  三年 & -204 & \begin{enumerate}
    \tiny
  \item 背水一战。
  \item 南越国建立。
  \item 成皋之战。
  \end{enumerate} \tabularnewline\hline
  四年 & -203 & \begin{enumerate}
    \tiny
  \item 英布封王。
  \item 张耳封王。
  \end{enumerate} \tabularnewline\hline
  五年 & -202 & \begin{enumerate}
    \tiny
  \item 十二月垓下之战,汉灭楚统一天下,汉王刘邦即皇帝位。
  \item 汉置长安县、无锡县。
  \item 七月,燕王臧荼起兵反汉。
  \item 十月,刘邦率军亲征灭燕,俘杀臧荼。刘邦立卢绾为燕王。
  \item 汉高祖册封无诸为闽越王,封国闽越,首都冶城位于今之福州。
  \end{enumerate} \tabularnewline\hline
  
  \bottomrule
\end{longtable}


%%% Local Variables:
%%% mode: latex
%%% TeX-engine: xetex
%%% TeX-master: "../Main"
%%% End:
