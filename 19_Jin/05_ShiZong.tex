%% -*- coding: utf-8 -*-
%% Time-stamp: <Chen Wang: 2019-12-26 11:17:28>

\section{世宗\tiny(1161-1189)}

\subsection{生平}

金世宗完顏雍(天輔七年三月初一甲寅日,儒略曆1123年3月29日—大定二十九年正月初二癸巳日,儒略曆1189年1月20日)),原名完顏褎(xiù、ㄒㄧㄡˋ),金朝第五位皇帝(1161年10月27日—1189年1月20日在位)。女真名乌禄,金太祖完颜阿骨打孙,海陵王完颜亮征宋时为辽东留守,后被拥立为帝,在位28年,终年67岁,葬于兴陵(今北京市房山区)。

1161年十月初八日,完颜亮率领大军渡过淮水,进兵南宋庐州。东京辽阳府发生了政变。曹国公完颜雍时任东京留守,完颜秉德以谋立葛王完颜雍之罪被杀后,完颜雍从海路献珍宝以表明他的忠诚。完颜亮命渤海人高存福为副留守,监视完颜雍的行动。契丹撒八等起义,完颜雍出兵阻击括里。完颜亮命婆速府路总管完颜谋衍(完颜娄室之子)领兵五千助战。完颜亮自辽东征调大批女真兵南下侵宋,女真兵多不愿南下。行至山东时,南征万户、曷苏馆女真猛安完颜福寿等领一万多人,中途叛变,逃回辽阳。完颜福寿与完颜谋衍等在辽阳发动政变,杀高存福,拥立完颜雍作皇帝,即金世宗。十月初八日,金世宗下诏废黜完颜亮,改元大定。完颜谋衍为右副元帅,福寿为右监军。十一月,在东京的政权,逐渐巩固。中都留守阿琐等起而响应金世宗。金世宗决定迁赴中都。十一月二十七日拂晓,完颜元宜率领将士袭击完颜亮营帐,完颜亮被乱箭射死。

金世宗即位后,首先对南宋的进攻保持守势,着手平息契丹起义,待平息契丹起义后,开始对南宋采取强硬态度,击退了南宋的隆兴北伐,并在形势占优时,在与宋孝宗和谈时做出让步,最终签署了《隆兴和议》,开启了双方四十余年的和平局面。

金世宗在内政管理上,励精图治,革除了完顏亮统治时期的很多弊政。更值得称道的是,金世宗十分朴素,不穿丝织龙袍,使金朝国库充盈,农民也过上富裕的日子,天下小康,实现了“大定盛世”的繁荣鼎盛局面,金世宗也被称为“小尧舜”。

金世宗统治时期,如移剌窩幹等各族人民纷纷起义,他为了维持统治,利用科举、学校等制度,争取汉人支持,又加强猛安谋克权力,扩大女真族占有的土地。同时多次发布有关保留女真人旧习、语言的诏令,甚或要求所有皇子必须有女真语名、所有女真官员必须通晓女真語,卫士不准讲汉语。

他死後谥号是光天兴运文德武功圣明仁孝皇帝,庙号是世宗。

元朝官修正史《金史》脱脱等的評價是:“世宗之立,虽由劝进,然天命人心之所归,虽古圣贤之君,亦不能辞也。盖自太祖以来,海内用兵,宁岁无几。重以海陵无道,赋役繁兴,盗贼满野,兵甲并起,万姓盼盼,国内骚然,老无留养之丁,幼无顾复之爱,颠危愁困,待尽朝夕。世宗久典外郡,明祸乱之故,知吏治之得失。即位五载,而南北讲好,与民休息。于是躬节俭,崇孝弟,信赏罚,重农桑,慎守令之选,严廉察之责,却任得敬分国之请,拒赵位宠郡县之献,孳孳为治,夜以继日,可谓得为君之道矣!当此之时,群臣守职,上下相安,家给人足,仓廪有余,刑部岁断死罪,或十七人,或二十人,号称“小尧舜”,此其效验也。然举贤之急,求言之切,不绝于训辞,而群臣偷安苟禄,不能将顺其美,以底大顺,惜哉!”

\subsection{大定}


\begin{longtable}{|>{\centering\scriptsize}m{2em}|>{\centering\scriptsize}m{1.3em}|>{\centering}m{8.8em}|}
  % \caption{秦王政}\
  \toprule
  \SimHei \normalsize 年数 & \SimHei \scriptsize 公元 & \SimHei 大事件 \tabularnewline
  % \midrule
  \endfirsthead
  \toprule
  \SimHei \normalsize 年数 & \SimHei \scriptsize 公元 & \SimHei 大事件 \tabularnewline
  \midrule
  \endhead
  \midrule
  元年 & 1161 & \tabularnewline\hline
  二年 & 1162 & \tabularnewline\hline
  三年 & 1163 & \tabularnewline\hline
  四年 & 1164 & \tabularnewline\hline
  五年 & 1165 & \tabularnewline\hline
  六年 & 1166 & \tabularnewline\hline
  七年 & 1167 & \tabularnewline\hline
  八年 & 1168 & \tabularnewline\hline
  九年 & 1169 & \tabularnewline\hline
  十年 & 1170 & \tabularnewline\hline
  十一年 & 1171 & \tabularnewline\hline
  十二年 & 1172 & \tabularnewline\hline
  十三年 & 1173 & \tabularnewline\hline
  十四年 & 1174 & \tabularnewline\hline
  十五年 & 1175 & \tabularnewline\hline
  十六年 & 1176 & \tabularnewline\hline
  十七年 & 1177 & \tabularnewline\hline
  十八年 & 1178 & \tabularnewline\hline
  十九年 & 1179 & \tabularnewline\hline
  二十年 & 1180 & \tabularnewline\hline
  二一年 & 1181 & \tabularnewline\hline
  二二年 & 1182 & \tabularnewline\hline
  二三年 & 1183 & \tabularnewline\hline
  二四年 & 1184 & \tabularnewline\hline
  二五年 & 1185 & \tabularnewline\hline
  二六年 & 1186 & \tabularnewline\hline
  二七年 & 1187 & \tabularnewline\hline
  二八年 & 1188 & \tabularnewline\hline
  二九年 & 1189 & \tabularnewline
  \bottomrule
\end{longtable}


%%% Local Variables:
%%% mode: latex
%%% TeX-engine: xetex
%%% TeX-master: "../Main"
%%% End:
