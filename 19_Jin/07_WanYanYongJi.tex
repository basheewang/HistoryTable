%% -*- coding: utf-8 -*-
%% Time-stamp: <Chen Wang: 2021-11-01 16:57:16>

\section{紹王完颜永济\tiny(1208-1213)}

\subsection{生平}

完颜允济(?-1213年9月11日),小字興勝,金章宗時避章宗父完顏允恭諱改為完颜永济。他是金朝第七位皇帝(1208年12月29日—1213年9月11日在位),被篡位後降封衛王,卒諡「紹王」,在位5年。

完颜允济是完顏允恭之弟,金章宗之叔,金世宗完颜雍第七子,母元妃李氏。他在金世宗大定十一年(1171年)被封薛王,同年改封禭王,先後改封潞王、韓王及衛王。章宗在泰和八年(1208年)农历十一月二十日病死,無嗣,衛王完颜允济被迎立为帝。

蒙古帝國的成吉思汗有意進攻金國,首先出兵進攻臣屬金朝的西夏,西夏向金求援,卫绍王坐視不救。西夏向蒙古屈服後,成吉思汗自大安三年(1211年)起大舉攻金,屢敗金兵。是年九月,蒙古軍逼近中都,因城防堅固兼有重兵防守,於是退兵。次年成吉思汗再次親征金國,一度包圍金西京大同府。同年契丹人耶律留哥在今吉林省境起兵反金,數月之間發展至十餘萬人。耶律留哥依附蒙古,又在迪吉腦兒(今辽宁昌图附近)擊敗六十萬金兵,金國的處境更加不妙。

衛绍王为人优柔寡断,没有安邦治国之才,只是俭约守成而已。他不善于用人,忠奸不分,最终导致杀身之祸。至寧元年(1213年)八月,蒙古軍再次逼近中都,右副元帥胡沙虎(紇石烈執中)起兵叛亂,弑卫绍王。九月,迎立完顏珣為帝,即金宣宗。胡沙虎請廢允济為庶人,詔百官三百餘人議於朝堂。太子少傅奧屯忠孝、侍讀學士蒲察思忠支持胡沙虎,但戶部尚書武都、拾遺田庭芳等三十人請降允济為王侯。胡沙虎固執前議,金宣宗不得已,乃降封允济為東海郡侯。十月,元帥右監軍朮虎高琪殺胡沙虎。

貞祐四年(1216年),金宣宗詔追復允濟為衛王,諡曰紹,後世稱他為衛绍王。

元朝官修正史《金史》脱脱等的評價是:“卫绍王政乱于内,兵败于外,其灭亡已有征矣。身弑国蹙,记注亡失,南迁后不复纪载。皇朝中统三年,翰林学士承旨王鹗有志论著,求大安、崇庆事不可得,采摭当时诏令,故金部令史窦祥年八十九,耳目聪明,能记忆旧事,从之得二十余条。司天提点张正之写灾异十六条,张承旨家手本载旧事五条,金礼部尚书杨云翼日录四十条,陈老日录三十条,藏在史馆。条件虽多,重复者三之二。惟所载李妃、完颜匡定策,独吉千家奴兵败,纥石烈执中作难,及日食、星变、地震、氛昆,不相背盭。今校其重出,删其繁杂。《章宗实录》详其前事,《宣宗实录》详其后事。又于金掌奏目女官大明居士王氏所纪,得资明夫人援玺一事,附著于篇,亦可以存其梗概云尔。”明朝官修《元史》,成吉思汗对完颜永济的評價是:“我谓中原皇帝是天上人做,此等庸懦亦为之耶?”

\subsection{大安}


\begin{longtable}{|>{\centering\scriptsize}m{2em}|>{\centering\scriptsize}m{1.3em}|>{\centering}m{8.8em}|}
  % \caption{秦王政}\
  \toprule
  \SimHei \normalsize 年数 & \SimHei \scriptsize 公元 & \SimHei 大事件 \tabularnewline
  % \midrule
  \endfirsthead
  \toprule
  \SimHei \normalsize 年数 & \SimHei \scriptsize 公元 & \SimHei 大事件 \tabularnewline
  \midrule
  \endhead
  \midrule
  元年 & 1209 & \tabularnewline\hline
  二年 & 1210 & \tabularnewline\hline
  三年 & 1211 & \tabularnewline
  \bottomrule
\end{longtable}

\subsection{崇庆}

\begin{longtable}{|>{\centering\scriptsize}m{2em}|>{\centering\scriptsize}m{1.3em}|>{\centering}m{8.8em}|}
  % \caption{秦王政}\
  \toprule
  \SimHei \normalsize 年数 & \SimHei \scriptsize 公元 & \SimHei 大事件 \tabularnewline
  % \midrule
  \endfirsthead
  \toprule
  \SimHei \normalsize 年数 & \SimHei \scriptsize 公元 & \SimHei 大事件 \tabularnewline
  \midrule
  \endhead
  \midrule
  元年 & 1212 & \tabularnewline\hline
  二年 & 1213 & \tabularnewline
  \bottomrule
\end{longtable}

\subsection{至宁}

\begin{longtable}{|>{\centering\scriptsize}m{2em}|>{\centering\scriptsize}m{1.3em}|>{\centering}m{8.8em}|}
  % \caption{秦王政}\
  \toprule
  \SimHei \normalsize 年数 & \SimHei \scriptsize 公元 & \SimHei 大事件 \tabularnewline
  % \midrule
  \endfirsthead
  \toprule
  \SimHei \normalsize 年数 & \SimHei \scriptsize 公元 & \SimHei 大事件 \tabularnewline
  \midrule
  \endhead
  \midrule
  元年 & 1213 & \tabularnewline
  \bottomrule
\end{longtable}


%%% Local Variables:
%%% mode: latex
%%% TeX-engine: xetex
%%% TeX-master: "../Main"
%%% End:
