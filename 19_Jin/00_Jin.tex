%% -*- coding: utf-8 -*-
%% Time-stamp: <Chen Wang: 2021-11-01 17:02:56>

\chapter{金\tiny(1115-1234)}

\section{简介}

金朝,国号大金(1115年-1234年),是中國歷史上由女真人建立的一個朝代。女真人原為遼朝的藩屬,女真人首領金太祖完颜阿骨打在統一女真諸部後,1115年於会宁府(今黑龙江省哈爾濱市阿城區)建都立國。大金立國後,與北宋定「海上之盟」向遼朝宣戰,於1125年灭辽,然北宋两次战辽皆败,金隨即撕毀與北宋之約,兩次南下中原,於1127年灭北宋。迁都中都時,領有華北地區以及秦嶺、淮河以北的華中地區,使南宋、西夏與漠北塔塔兒、克烈等政权和部落臣服而稱霸東亞。金朝佔據華北中原地區,以中國正統王朝自居,並逐漸以此為政權中心。因其滅北宋,金朝從意識形態上認為宋朝正朔已亡,不承認南宋為正統,並根據五行相生的原則選取生自宋朝「火」德的「土」德為王朝德運。

金世宗與金章宗時期,金朝政治文化達到最高峰,然而在金章宗中後期逐漸走下坡。金軍的戰鬥力持續下降,即使統治者施以豐厚兵餉也無法遏止。女真人與漢族的關係也一直沒有能夠找到合適的道路。金帝完顏永濟與金宣宗時期,金朝受到北方新興大蒙古國的大舉南侵,內部也昏庸內鬥,河北、山東一帶民變不斷,最終被迫南迁汴京(今河南开封)。而後為了恢復勢力又與西夏、南宋交戰,彼此消耗實力。1234年,金朝在蒙古和南宋南北夾擊之下滅亡。

1115年完顏阿骨打稱帝時對群臣說:「遼以賓鐵為號,取其堅也。賓鐵雖堅,終亦變壞,唯金不變不壞。」於是,以“大金”為國號,望其永遠不變不壞也。一说女真兴起于金水,「上京路,即海古之地,金之旧土也。国言金曰按出虎,以按出虎水源于此,故名金源,建国之号,盖取诸此。」按出虎亦曰阿之古,阿之古亦称阿古,为谓之乌古。按出虎亦日按春,“按春日金”,故国号名金,在部份文献中,“金源”因此成為金朝的代稱,亦有現代學者研究認為,「金」實為女真的漢譯,大金國意同「女真國」。

金朝作為女真族所建的新興征服王朝,其部落制度的性質濃厚。初期採取貴族合議的勃極烈制度。而後吸收遼朝與宋朝制度後,逐漸由二元政治走向單一漢法制度,使金朝的政治機制得以精簡而強大。軍事方面採行軍民合一的猛安謀克制度,其鐵騎兵與火器精銳,先後打敗許多強國。經濟方面大多繼承自宋朝,陶瓷業與煉鐵業興盛,對外貿易的榷場掌控西夏的經濟命脈。女真貴族大肆占領華北田地,奴役漢族,使得雙方的衝突加劇。當金朝國勢衰退時,漢族紛紛揭竿而起。

金朝在思想文化方面也逐漸趨向漢化,中期以後女真年輕人改漢姓、著漢服的現象普遍,金廷屢禁不止。金世宗積極倡導學習女真字、女真語,但仍無法挽回女真漢化的趨勢。雜劇與戲曲在金朝得到相當的發展,已盛行以雜劇的形式作戲。金代院本的發展,為後來元曲的雜劇打下了基礎。醫學與數學都有長足的發展,金元四大家的學說為中醫發展產生重要的影響,天元術的精進與《重修大明曆》的修編為後來元朝數學帶來重要的影響。

金朝是由東北地區的女真族所建立,人民以渔猎为生。唐朝时称為靺鞨,五代時有完颜部等部落,臣屬於渤海國。遼朝攻滅渤海國後,收編南方的女真族,稱為熟女真,北方則是生女真。遼朝晚期朝政混亂,天祚帝昏庸無能,辽廷不停的索求貢品,並且鱼肉女真百姓。1112年天祚江帝赴春州與女真各族的酋長聚會時對完颜阿骨打等酋長不敬,使完颜阿骨打有意反抗遼廷,他隨後出兵統一女真族各族。1114年完颜阿骨打向遼宣战,隨後在宁江大捷和出河店之战擊敗遼軍。阿骨打于隔年一月在“皇帝寨”(即後來的上京会宁府,今黑龙江省哈尔滨市阿城区南之白城)称帝建国,即金太祖,国号大金。遼帝天祚帝至此才重視此事,並且下令親征,但是遼軍被女真軍擊敗,同時遼朝國內發生耶律章奴與高永昌的叛亂。

金太祖建国後以辽五京为目标兵分兩路展開金滅遼之戰。天祚帝曾嘗試冊封金太祖為東懷國皇帝試圖安撫,但冊文不稱完顏旻為兄長、國號不稱大金,故他不接受冊封,繼續攻打遼朝。

1116年五月東路軍佔領東京遼陽府,1120年西路軍攻陷上京臨潢府,遼失去一半的土地。戰事期間北宋陸續派使者馬政、趙良嗣與金朝定下海上之盟,聯合攻打遼朝。1122年東路軍攻下中京大定府,天祚帝逃亡沙漠。同時西路軍也攻下西京大同府,耶律大石立耶律淳於南京析津府,即北遼。北宋也派童貫等人多次率軍北伐遼南京與燕云十六州,但均被遼軍擊潰。北宋最後請金軍攻下遼南京,北遼亡,至此遼五京均攻下。宋金雙方經過協商後,金軍給予燕云十六州部分城市,並且獲得歲幣,然而北宋最後只獲得金軍洗劫後的一堆空城。1123年金太祖去世,其弟完顏吳乞買繼位,即金太宗。金太宗繼續討伐大同一帶的遼軍。1124年正月,金太宗为了聯合西夏滅遼,把下寨以北、阴山以南的遼地割给西夏。西夏則改对金朝称藩。1125年遼天祚帝被俘,遼朝亡。而耶律大石率軍西行,於西域建立西遼。

然另有一说认为《金史》记载的金朝开国史不真实,阿骨打起兵后可能于1117年或1118年建立了“女真国”,1122年才将国号改为大金。

金朝在滅遼朝後,即有意南下滅宋朝。金太宗藉由燕云十六州的平州之變為由宣布毀約,於1125年發動金滅宋之戰。他派勃極烈完顏斜也為都元帥,完顏宗望、完顏宗翰兵分河北、山西兩路,最後會師北宋首都开封。在宋將李綱死守開封的情況下,雙方簽下宣和和議。1126年金太宗以宋廷毀約為由,再派完顏宗望、完顏宗翰兵分二路攻破开封,於隔年俘虏宋徽宗、宋钦宗等宋朝皇室北歸,史稱靖康之禍,北宋灭亡。然而北宋康王趙構因機運逃過,並於宋朝南京歸德府(今河南商丘)稱帝建國南宋,即宋高宗。金朝為了統治廣大新占領的漢地,先後建立張楚與劉齊等傀儡國以統治之,並且多次派完顏宗弼等金將率軍南征南逃江南的宋高宗。然而南宋在宋將岳飛、韓世忠與張浚的努力下,屢次使南宋轉危為安。最後金朝只好迫使南宋稱臣,並且讓西夏、高麗等國臣服以稱霸東亞。

1135年金太宗去世,由金太祖的孫子完顏亶即位,即金熙宗。當時輔佐金廷的一些功臣被稱為衍慶功臣,他們左右朝政,主要分成主戰派與主和派。金熙宗於1137年廢除劉齊,聽從主和派完顏撻懶的建議與宋高宗和投降派秦檜議和,歸還河南、陝西地給南宋。這讓主戰派完顏宗弼不滿,他於1140年率軍奪回河南、陝西地。隔年還再度南征南宋,但被岳飛與劉錡擊敗。岳飛於郾城之戰後再度逼近汴京,至此完顏宗弼也接受與南宋合談。金宋兩國在岳飛被殺後簽訂紹興和議,至此雙方邊界大致底定。金熙宗自幼受漢文化薰陶,登基後與完顏宗弼推動漢制改革,並且重用漢人。隔年派衍慶功臣的完顏宗磐、完顏宗幹和完顏宗翰三人共同總管政府機構,「並領三省事」。金朝官制已經漢化,建立以尚書省為中心的三省制。1150年金熙宗受衍慶功臣與皇后的控制,本人被過度壓抑,後期不理朝政,濫殺無辜,最後被右丞相海陵王完顏亮所殺,完顏亮自行稱帝,史書稱為海陵王。

金帝完顏亮為了攻伐南宋以統一中國地區,推行許多措施:他將首都遷至燕京,是為中都(今北京市),並且有意南遷汴京(今河南省開封市);將行政區劃重新劃分成十四路以便於管理;把驻扎上京會寧府(今黑龙江省哈尔滨市阿城区南之白城)属于金太宗、衍慶功臣的完顏宗幹與完顏宗翰管辖下的军队歸金廷管制,為金朝的中央集權制打下基礎。然而金帝完顏亮對宗室猜忌甚深,金太宗的後代差不多全被完顏亮殺盡;並且耗費鉅資,不顧部分大臣的反對,執意南征。1161年5月,金廷遣使赴宋要求重劃國界,意在尋釁,南宋也開始積極備戰。隔年金帝完顏亮率大軍由汴京兵分四路南征。東面軍分成海路跟陸路兩股,陸路軍由金帝完顏亮親自率領,自宿州(今安徽省宿州市)渡過淮水直攻和州(今安徽省和縣),海路水軍則直攻臨安(今浙江省杭州市)。西路分別自关中、河南攻向四川及湖北一帶。金東路軍渡淮水,攻陷和州準備渡江。然而東路水軍在膠西(今山東省膠州市)被宋將李寶的水軍殲滅。同時間西北契丹族叛變,鎮守東京遼陽府的葛王完顏雍自立為帝,並移居燕京(今北京市),即金世宗。金帝完顏亮遇到此情形仍然執著渡江,但是先遣部隊在采石之战被宋將虞允文擊敗,船艦也被宋軍燒毀。金帝完顏亮意圖移師揚州強渡長江,但是部下大力反對,最後發動兵變殺死金帝完顏亮。宋軍趁機收復淮南地區,此後金朝不再企圖滅亡宋朝。

1161年金世宗舉兵後昭告金帝完顏亮罪過,率軍統一華北,並不再企圖滅亡南宋。然而宋金的戰爭並沒有終止,1162年他以南宋不願稱臣為由,派主將僕散忠義進駐汴京、紇石烈志寧鎮守前線,準備奪回淮南地區。此時南宋宋孝宗意圖收復失地,派主將張浚率領李顯忠、邵弘淵率軍北伐,史稱隆興北伐。宋軍陸續收復淮北各地,但於符離的符離之戰被紇石烈志寧擊潰而止。而後南宋主和派抬頭,於1164年金軍再度南征之際求和,兩國於年底簽定合約,雙方平等對待,金朝獲得歲幣。內政方面,金世宗本身十分樸素,採取中庸穩固的方式管理朝政,提倡儒學;查問細微以激勵官吏,嚴禁貪污;對經濟採取務實的態度,並且免除不合理的賦稅,若有天災發生,立即救濟賑災。當時各族人民紛紛起義,他為了維持統治,利用科舉、辦學等制度,爭取漢族貴族支持,又加強猛安、謀克權力,擴大女真族佔有的土地。這些都使金朝的經濟、文化都得到了一定程度的恢復和發展,史稱大定之治。金世宗除了抵禦南宋北伐,還出兵威震懾西夏、高麗,使這兩國臣服金朝,被金史稱為小堯舜。1189年金世宗死後,由於太子完顏允恭早逝,故立允恭的兒子完顏璟即位,是為金章宗。

金章宗前期政治漢化甚深,文化十分發達,史稱明昌之治。金章宗不單對國內文化發展加以獎勵,他本身亦能寫得一手好字。然而金章宗過度重視文化發展,寵愛李師兒(後封元妃)以及李氏外戚,任用經童出身的胥持國管理朝政。這兩位互相協助,營利干政,使金章宗後期的政風逐漸敗壞,而黄河氾濫與改道又使金朝國勢開始衰退。此時金朝軍事逐漸荒廢。金朝「盛平日久,兵日益驕,民日益貧,而奢靡之風日盛」

北方蒙古諸部兴起。金章宗曾派兵至蒙古減丁,並且誘使互相殘殺,但成效不大,最後由成吉思汗所統一。南宋權臣韓侂冑見金朝國勢衰退,命吳璘之孙吳曦管理蜀地,準備北伐,而金廷也派仆散揆坐鎮汴京,抵禦宋軍。1206年韓侂冑發動開禧北伐,宋軍一度收復淮北地區,但是鎮守蜀地的吳曦投降金朝。八月仆散揆率軍九路南下,年底金兵直逼長江,並且圍攻襄陽。隔年吳曦被殺,四川復歸南宋,至此雙方有意議和。韓侂冑最後在金朝與南宋的要求下被殺,雙方於1208年議和,史稱嘉定和議。1208年金章宗逝世,由於他的六個兒子都在三歲前夭折,李元妃立金章宗的叔父衛王完顏永济繼位,史書稱衛紹王。

金帝完顏永濟繼位後立即清除李元妃等外戚勢力,然而本身昏庸且任用錯人,加上金朝國力衰退混亂,面對蒙古入侵時無力反抗。1206年蒙古部成吉思汗(元太祖)統一大漠南北,建立大蒙古國。當時蒙古族對金朝保持嚴重的敵意,有意脫離金廷控制,而成吉思汗也知道完顏永濟是個無能之輩,認為這是攻滅金朝的好時機。成吉思汗先攻打西夏以拆散金夏同盟,避免在伐金時被其牽制。當時西夏向金廷求援,金帝完顏永濟以鄰國遭攻打為樂而坐視不救,最後西夏向蒙古臣服,並且轉為附蒙伐金。消除後顧之憂後,成吉思汗於1210年與金廷斷交。隔年發動蒙金戰爭,於野狐嶺戰役大破丞相完顏承裕與將領完顏九斤率領的四十萬金軍,金帝完顏永濟將丞相換成擅長謀略的徒單鎰。蒙古軍隨後攻入華北並四處掠奪,最後包圍金廷首都中都(今北京市),因中都城堅而撤。1212年成吉思汗再次南征金朝,一度包圍金西京大同府。同年契丹人耶律留哥在東北叛金附蒙,並在迪吉腦兒(今辽宁昌图附近)擊敗金兵,蒙古軍再次逼近中都。1213年將領胡沙虎殺金帝完顏永濟,胡沙虎擁立金章宗的庶兄完顏珣繼位,是為金宣宗。

1213年金宣宗繼位之初,胡沙虎執金朝大權。胡沙虎威脅中都守將朮虎高琪作戰不力,最後反被其所殺。同年秋成吉思汗兵分三路攻金,他派皇子朮赤經略山西、皇弟合撒兒往河北,他與幼子拖雷往山東進攻,金廷只有中都、真定、大名等十一城未失。隔年金宣宗求和,獻黃金與岐国公主與成吉思汗,蒙金和議达成。在蒙軍撤退後,金宣宗不顧徒單鎰等反對,與朮虎高琪遷都汴京 ,僅派太子鎮守中都,引來河北軍民不安。1215年蒙古以金帝南遷為由,再度率軍攻陷中都,至此占領河北地區。同年十月,蒲鮮萬奴在遼東自立,建東真國。此時金朝龍興之地被蒲鮮萬奴與耶律留哥瓜分,山東與河北一帶都是民變的紅襖軍,金廷只能控制河南、淮北與關中一帶。當時河患泛濫成災,黃河自從金室南遷後改道,流向東南。河患的範圍非常廣泛。

金宣宗南遷之後國勢益弱,蒙古已經取代金朝稱霸東亞,由於成吉思汗與花剌子模發生糾紛而發動西征,他派木華黎統領漢地,封为“太师国王”,持續威脅金朝,金朝至此終於得以喘息。雖然金宣宗想要重振金朝,但無雄才大略,且又猜忌成性,政治上並無起色。1219年太原失守,金廷建立河北九公,封立王福、移剌眾家奴、武仙等九人為公,賜號「宣力忠臣」,打算以之堅守國土,但仍然無濟於事。金宣宗任用术虎高琪,他苛刻成性,接連南征南宋、西征西夏以擴張領土,並且持續抗击蒙古。此時金朝內政不良,軍力已衰,經過多次戰爭後又使金朝處於四面楚歌的局勢。1219年朮虎高琪被金宣宗殺死,戰事直到金宣宗駕崩以後才平息。1224年金宣宗逝世,由於其長子早逝,故由次子完顏守緒繼位,即金哀宗。

金哀宗即位後,鼓勵農業生產,與南宋、西夏和好。建立直屬中央的忠孝軍,任用完顏陳和尚等抗蒙名將,於1228年大昌原之戰(今甘肅寧縣)擊潰蒙古軍。而後金軍收復了不少土地,讓金朝起死回生。然而其盟國西夏因為之前的戰爭使得國力衰落,最後在1227年被西征回來的蒙古所滅。同年成吉思汗去世,1229年正式由三子窩闊台繼任,史稱元太宗。此後蒙古再度對付金廷,1230年窩闊台汗發動三路伐金,窩闊台汗率大軍渡黃河直攻汴京,斡陈那颜率東路軍走濟南,四弟拖雷率西路軍自漢中借宋道沿漢水攻打汴京。1232年拖雷成功迂迴至汴京,金哀宗派完顏合達、移剌浦阿率大軍阻擊於鄧州。此時窩闊台汗率大軍渡河,並派速不臺攻汴京。而完顏合達急率軍北援汴京,與拖雷率領的蒙軍於三峰山(今河南禹州市東南)發生三峰山遭遇戰,金軍精銳潰敗,名將張惠、完顏合達、完顏陳和尚與移剌浦阿先後死亡。蒙軍圍攻汴京,迫使金哀宗求和。而後金廷殺蒙古使者,蒙古再度圍攻汴京。金哀宗堅持至年底放棄,南逃歸德(今河南商丘市),汴京守將崔立投降。蒙將史天澤一路緊追不捨,金哀宗逃往蔡州,蒙軍約宋將孟珙、江海率軍與糧食聯合圍攻。1234年正月蔡州岌岌可危,金哀宗不願當亡國之君,也自认为肥胖不能骑马难以突围,將皇位傳給統帥完顏承麟,史稱金末帝。哀宗认为末帝身手矫健有将略,如果侥幸突围,可以延续国祚。蔡州城陷后,金哀宗自杀,金末帝死於亂軍中,金朝亡。金亡後,少數漢臣以遺老自居,入元不仕,如王若虛、元好問等。

金朝自金太祖立國以來,接連不斷對遼朝、宋朝、西夏與高麗發動侵略戰爭。金太宗時,金朝先後攻滅遼朝與北宋,其疆域東到混同江下游吉里迷、兀的改等族的居住地,直抵日本海;北到蒲與路(今黑龙江克东县)以北三千多里火鲁火疃谋克(今俄罗斯外兴安岭南博罗达河上游一带),西北到河套地区,與蒙古部、塔塔兒部、汪古部等大漠諸部落為鄰;西沿泰州附近界壕與西夏毗鄰。南部與南宋以秦岭淮河為界,西以大散关与宋为界。金朝疆域可分成三個部分,第一個是原遼朝統治的東北區域與漠南地區,這是金朝龍興之地,包括女真各部落的住地,還有契丹、奚、渤海以及五國部、吉里迷、兀的改等各族。金朝建國之初,對此一概搬用生女真舊制。如1116年金太祖佔有遼東京州縣以後,「詔除遼法,省稅賦,置猛安謀克一如本朝之制」。即不管是遼籍女真族、漢族、渤海族、契丹族或是奚族,全都以猛安謀克制度(即千夫百夫的制度)劃分管理。第二個是遼上京臨潢府以南,直到河北、山西等燕雲十六州,這裡居民主要是漢族,長期以來受異族統治,而金統治下之漢地亦維持漢官制度。史稱「太祖入燕,始用遼南、北面官僚制度」,就是指同時奉行女真舊制和漢制的雙重體制。第三種情況是原宋朝領地的秦嶺、淮河以北之地,主要居民也是漢族,由於新受異族統治,大多不願受金廷管制。先後設置張楚與劉齊等傀儡政權統治,最後由金廷以漢法直接管理。

金朝的行政区域採用路(府)、州、县三级管理,路为一级行政区,共有十四個路與五京,合計十九。路有都總管府,以及都轉運司、按察、鹽使等三司的監司官,五京府有五京留守,而後與府伊所掌的府合一。路府則成平行機構,下轄州、縣二級。金朝的州分為三類:節度州設節度使,防禦州設防禦使,刺史州設刺史;縣則以縣令掌治,分成七等。此外尚有部落之官,千夫長的猛安,百夫長的謀克,合稱猛安謀克;屬於邊戍之官的乣詳穩,部落墟砦之首領的移里菫。

金朝採行五京制,共有中都大興府、上京會寧府、南京開封府、北京大定府、東京遼陽府和西京大同府,其中後三個陪都就在遼的中京大定府、東京遼陽府和西京大同府的原址。金朝原都上京,1151年4月金帝完顏亮頒布詔書擴建遼燕京為中都,於1153年遷都中都,至此中都在很長的時間為金朝首都。1214年金宣宗受蒙古帝國掠奪與威脅,宣布南遷汴京。1232年三峰山之戰金軍戰敗,蒙軍圍攻汴京後,金哀宗先奔歸德(今河南商丘),最後逃到蔡州(今河南汝南)。

金朝初期全面採用遼朝的北南面官制,奉行女真舊制與漢地漢制等兩元制度。自金熙宗推行天眷新制後逐步棄用女真制,逐漸採用宋朝制度。政治體制的一元化,是金朝強大的一個重要原因。

金朝建立之初,金太祖废除部落联盟时的“国相”制,採用小組化的勃极烈制度,长官均称「勃极烈」。其中都勃極烈即皇帝、諳班勃極烈為皇儲、國論勃極烈為國相、阿買勃極烈為國相助手、昊勃極烈為第二助手,形成皇帝和少數大臣共議國事的勃極烈制度。而後又有忽魯勃極烈(第三助手),到1132年金太宗分忽魯勃極烈為左勃極烈與右勃極烈,形成皇帝、皇儲與國相五人小組。

随着金朝不断向南延伸,促使金廷改採用三省制的唐制。1123年至1138年間,金太宗在占領燕雲十六州與攻滅北宋後,推行汉官制度,與女真舊制形成兩元制度。金初的所謂「南面官」,亦即設於營州廣寧(今河北省昌黎縣)的漢地樞密院,最後遷至燕京。與此相對的「北面官」,主要指當時實行於朝廷之內的勃極烈制度。1135年金熙宗繼位後廢除勃極烈制度以鞏固皇權,全面使用宋朝、遼朝官制。他建立以尚書省為中心的三省制,以三師(太師、太傅、太保)以及三公(太尉、司徒、司空)共領三省事,地方分路、府、州、县。1138年改燕京尚書省為行台尚書省,成為中央尚書省的派出機構,結束雙元制度並存的局面,這些與三省制都是漢制改革的結果,史稱天眷新制。金帝完顏亮时廢除中书省、门下省,行政機關縮編成尚书省。

金朝制度在金世宗之後大體同宋朝制度。尚書省中,尚書令為虛設,實際上由左右丞相與平章政事掌握行政權,左右丞與參知政事為副相,其下有左右兩司郎中,分掌六部。軍事機構由都元帥府改為樞密院。設置鹽鐵部、度支部、戶部等三司理財、御史台掌糾察、諫院及審官院,其他有國史院、宣徽院、宏文院與集賢院等機關,太常、大里寺,六監、司農司、大宗正等皆依宋朝制度。

金朝滅亡的原因是史學家爭論的課題,有一部份學者認為金亡是因為漢化太徹底,也有人認為金亡是因為漢化不夠徹底。例如刘祁在《歸潛志·辨亡》認為金朝“分別蕃漢人,且不變家政,不得士大夫之心,此其所以不能長久”。儒士郝經因此要求忽必烈以金朝為榜樣,力行漢法。許衡在至元二年(1265年)向忽必烈奏上的《時務五事》:“自古立國,皆有規模。……考之前代,北方之有中夏者,必行漢法,乃可長久。故後魏、遼、金歷年最多,他不能者,皆亂亡相繼,史冊具載,昭然可考。”,同樣認為“必行漢法,乃可長久”。

女真族原臣服遼朝。遼朝晚期因受女真族建立的金朝入侵,加上朝廷內部分裂與內鬥,使遼朝有意與北宋和談。但是北宋已經與金朝建立海上之盟而共同伐遼,所以拒絕和談,最後遼朝亡於金朝。金滅遼後,又南征滅北宋,在靖康之禍後宋朝與金已成為死敵,雙方多次發生戰爭。金廷為了統治廣大的漢地而建先後建立張楚與劉齊等傀儡國,但因基礎不穩與未能有效攻滅宋高宗建立的南宋而廢除。經過金滅宋之戰、完顏宗弼南征江南、金帝完顏亮的采石之戰等戰爭,金朝都未能徹底滅亡宋朝;而南宋經過岳飛北伐、隆興北伐、開禧北伐也未能擊敗逐漸穩固中原的金朝,這些都使得金廷的方針逐漸採取以戰逼和的方式以獲取更多利益,雙方先後簽署宣和和議、紹興和議、嘉定和議等協定。這個均勢直到大蒙古國崛起後才被破壞,南宋在金朝衰退後不願稱臣納貢,雙方再度交惡。

漠北地區原為金朝的附屬地區,當地主要有乃蠻、克烈、蔑兒乞、泰赤烏、塔塔兒、蒙古與札達蘭等部落,此外還有其他部落。金朝為了穩固漠北地區,採取聯合部落去壓制想要崛起的部落,並且多次派兵減丁、掠奪,這使得部分漠北部落敵視金朝。隨著金朝的衰落,漠北部落最後於1204年由蒙古部的成吉思汗所統一,兩年後建國於漠北,國號大蒙古國。他為了報仇而不再臣服金廷,在降伏金廷盟國西夏後入侵金朝,成為金廷一大外患。

西夏在遼朝後期與其友好,在金滅遼後因金太宗同意裂地贈與西夏,使西夏轉向支持金朝。金廷穩固中原後切斷西夏與宋朝的關係,使得西夏對中原的貿易完全掌控在金朝手上,這使雙方的關係處於明和暗離的狀態。蒙古崛起後為了切斷西夏與金朝的盟約,多次攻打西夏。而金帝完顏永濟對此表示以鄰國遭攻打為樂而坐視不救,導致西夏向蒙古臣服,轉而攻打金朝。這個狀態直到夏獻宗繼位後才轉為連金抗蒙,不過兩國已無力抗蒙。

蒙古在攻陷金中都、迫使金廷南遷後,有意採取連宋滅金的方針,並有意借道南宋的方式,迂迴攻打金朝後背。金宣宗為了彌補失去河北的損失,多次對南宋、西夏發生戰爭,最後雖然互相和解,但三國彼此元氣大傷。西夏最後於1227年亡於蒙古,金朝也難逃滅國的命運。1230年窩闊台汗發動三路伐金,派拖雷經宋道迂迴攻打金南都開封。在金哀宗南逃蔡州後,蒙古邀請南宋率軍夾攻,宋廷為了報靖康之恥也願意派兵運糧助戰,最後金廷亡於蒙宋夾攻之中。

金朝對高麗王朝關係的方面,在建立金朝之前有部分女真族向高麗朝贡,被稱為东北女真。金太祖统一了女真諸部后入侵高麗。此時高丽肅宗率領的高麗軍難以抵抗金軍,最後在尹瓘的说服下讓女真軍撤退。尹瓘重建高麗軍,並在1107年成功抵禦女真入侵,並在國界修建九城。1115年金立國後不久滅了遼朝與北宋,切斷高麗與宋朝的關係,孤立了高麗。金朝與吐蕃諸部亦有一定交流。1168年八月,宋、金同時遣使到越南李朝,而李朝的態度則是禮待雙方來使,然而不令相見。

金軍大體可分為本族軍、其他族軍、州郡兵和屬國軍。前二者為主力,後二者為輔翼。最初,奴隸主、封建主都應從軍。領有漢地後,主要實行徵兵制,簽發漢族和其他少數民族為兵,稱為「簽軍」,到後期也行「募兵制」。金朝統治中原後,還仿漢制,實行發軍俸、補助等措施。對年老退役的軍官,曾設「給賞」之例。對投降的宋軍,常保留原建制,仍用漢人降將統領。金軍亦以騎兵為主,步兵次之。騎兵一兵多馬,慣於披掛重甲。各部族兵增多後,步兵數量大增。水軍規模也較大,但戰鬥力較弱。除冷兵器外,還使用火炮﹑鐵火炮﹑飛火槍等火器作戰。后来蒙古南侵之时,金军就以火器抗蒙。1232年金將赤盏合喜駐守汴京,「其守城之具有火炮名震天雷者,铁礶盛药,以火点之,炮起火发,其声如雷,闻百里外,所爇围半亩之上,火点著甲铁皆透」。

軍事機關原設有都統,後改為元帥府、樞密院等,協助皇帝統轄全軍。戰時,皇帝指定親王領兵出征,稱都元帥、左右副元帥等臨時職位。邊防軍事機構有招討司、統軍司等。金朝軍隊採用結合社會與軍事制度的猛安謀克制度,也就是百夫千夫長的制度。早在女真族時期,所有成年男子都是戰士,平時從事生產,戰爭時參加戰鬥,兵器、糧食自幾自足。分置人民約一千戶為猛安、約一百戶為謀克,謀克相當於百夫長,猛安相當於千夫長。萬戶府下轄諸猛安,猛安下轄謀克,謀克之下還有五十、十、伍等組織。兵員配置大多是一正一副,戰時副軍可以遞補正軍。兵為世襲制,兵員可以子弟替代,但不能以奴充任。

完顏阿骨打起兵反遼朝時,以三百口為一「謀克」,十謀克為一「猛安」。約二千五百人的兵力,僅用了十二年的時間,就將遼朝、北宋兩邦徹底征服。後來猛安謀克既是軍事長官,又是行政長官。隨著金朝不斷南移,猛安謀克制度與奴隸制互相適應的制度逐漸遭到破壞,“舍戎狄鞍馬之長,而從事中州浮靡之習”。女真人的日趨文弱化就是一個相當普遍的現象。金世宗時,阿魯罕任陝西路統軍使,「陝西軍籍有闕,舊例用子弟補充,而材多不堪用,阿魯罕於阿裏喜、旗鼓手內選補」。史旭有詩:“郎君坐馬臂彫弧,手撚一雙金僕姑。畢竟太平何處用,只堪粧點早行圖。”,已知“國朝兵不可用,是則詩人之憂思深矣。”。1168年朝廷從猛安謀克中遴選侍衛親軍,而“其中多不能弓矢”。最後當蒙古突騎興起後,金軍在野狐嶺戰役等大型戰役中慘敗,最後南遷汴京。然而在金哀宗時期所建立的忠孝軍,對蒙古軍仍有一定威脅。

1141年紹興議和後,自从靖康之难开始減少的人口总量得到一定程度的恢复增长。到1207年金章宗泰和七年間達到53,532,151人,被稱為人口之极盛。而當時金與南宋、西夏等人口总数据估计达到一亿三千六百万人,中國人口从1083年的一亿增加到1120年的一亿三千二百四十万。金朝的四次准确的人口统计,每户平均人口都在6人以上,金朝的户规模较大,和很多贵族以及猛安谋克户们使用大量奴仆有一定关系。

隨著金朝內地經濟的發展,不僅需要更多的土地耕作,也更需要民戶當勞力。金太祖與金太宗為統治中原,將百萬以上的女真人徙置於黃河下遊人口稠密地方,是以犧牲漢人利益的辦法去救濟女真人。在把女真族遷往新佔領地區的同時,也還繼續地把契丹族、漢族遷到金朝的內地。金軍在滅遼的作戰中,曾經擄回大批的契丹族、漢族作奴隸。後來金太祖下詔,禁止對已經投降的百姓擄掠,禁止權勢之家買貧民為奴,又規定賣身為奴者,可以用勞力相等的人贖身。但實際上,這種贖身的可能性是很少的。被迫遷徙的漢族居民,不能不大批地淪為奴隸。金廷對降附區的人民,採用強迫遷徙的辦法遷到內地。如山西州縣的居民,被大批遷到金上京以至渾河路。上京地區的居民又被遷到寧江州。平州的人民在反抗後被鎮壓,隨後與潤州、隰州、來州及遷州等四州人民被遷到東都瀋陽。這些居民艱苦不能自存,被迫賣身作奴隸,導致漢人刻骨的痛恨。女真族搶佔漢族最富庶的耕地,為了增加日益增大的生活和軍事開支,又不斷加重漢族的賦役。女真人與漢人的矛盾恰如史籍所言:「盜賊滿野,向之倚國威以重者,人視之以為血仇骨怨,必報而後已」。

女真男人的髮型是“留颅后髮”,還綁個辫子,所以被南人稱為索虜。在《北风扬沙录》与南宋的游记,也有金人留辫的说法,在金太宗天会七年還强制胡服留辫。這樣剃髮留辫的習俗為後來的滿清所繼承。金代皇帝也可掌摑大臣」。

金朝各地區的經濟發展存在很大差異。金朝建立之初,女真族尚處於漁獵農耕的混合制度,而它所控制的漢地,農業經濟早已高度發展。从金熙宗到金章宗的半个多世纪里,北方社会经济有一定程度的恢复和发展。东北地区社会经济比辽时有了较大的发展,如冶铁业有明显进步。在金世宗與金章宗期間,原來使用奴隸生產的猛安謀克戶也逐漸轉化成地主。這個轉化主要有受田、賦稅、區別平民和奴隸等,女真貴族藉此擴大土地佔有範圍。

金朝把发展农业作为军事扩张的基础,视其龍興之地東北地區为粮仓,將中原地區的生產工具和耕作技術都逐漸傳播到當時落後的今東北地區。由於鐵製農業生產工具的廣泛使用,促進農業生產的發展,農作物品種也日益增多。金初,不種穀麥,只種稷子春糧。以後農作物品種日益增多,農作物有小麥、粟、黍、稗、麻、菽類等;蔬菜類有蔥、蒜、韭、葵、芥及瓜等。金廷又鼓勵墾荒,例如規定開墾荒地或黃河灘地可以減免租稅,所以開墾農田面積有所增加。

金朝的土地制度給予女真族很大的優惠,這是漢族、契丹族與渤海族所沒有的。女真族的土地制度是一種稱為「牛具稅地」的制度,繼承氏族制度的遺風。佔地多少是以耒牛、人口為依據的,擁有眾多人口和耒牛的女真貴族自然就可以廣占田土。到金世宗大定年間,人、牛、地比例不符的情形已很普遍。

金熙宗時期開始實行的「計口授田」的制度。早在金朝統治廣大的華北地區後,有計畫的將大量的猛安謀克分散各地以鎮壓漢族,被稱為屯田軍。金廷對內遷的屯田軍戶,都按照戶口給以官田,即所謂「計口授田」。當官田不敷分配時也會大量搶佔民田。屯田軍戶分得土地以後,大多讓租給漢族耕種或是強迫漢族無償耕種。由於剝削嚴重,無人願意耕種,土地逐漸荒廢。金世宗時再派官吏到各地去「拘刷良田」,兼併土地為官田。

由於女真族屬於東北民族,其畜牧業也十分發達。金帝完顏亮時原有九个群牧所。在南征時,征调战马达56万多匹,然而因戰事大半損失,到金世宗初年仅剩下四个。金世宗開始復甦畜牧業,當時在抚州、临潢府、泰州等地设立七个群牧所。1168年起,下令保护马、牛,禁止宰杀,禁止商贾和舟车使用马匹。又规定对群牧官、群牧人等,按牲畜滋息损耗给予赏罚。经常派出官员核实牲畜数字,发现短缺就处分官吏,由放牧人赔偿。对一般民户饲养的牲畜,登记数额,按贫富造簿籍,有战事,就按籍征调,避免征调时出现贫富不均的现象。对各部族的羊和马,规定制度,禁止官府随意强取。

金朝手工业生产如陶瓷、矿冶、铸造、造纸、印刷等,歷經戰亂與復甦都有发展。女真族在建国前盛行炼铁。金朝建立后冶铁业在北方地区继续发展,著名的产铁地区有云内州、真定府、汝州鲁山、宝丰、邓州南阳等,云内州盛产一种叫做青镔铁的铁器。另外在东北地区,也开始开采煤矿。铁制工具已广泛使用。在东北广大地区内,都发现了金朝的铁器。其中有大量铁制农具,种类繁多,结构复杂,形制与中原地区相似或一致,这表明已改变了粗放的农业经营方式。1961年至1962年,在黑龙江省阿城县五道岭发现金朝中期铁矿井10余处,炼铁遗址50余处。矿井最深达40余米,有采矿、选矿等不同作业区。根据开采规模估计,从这些矿井中已采出四五十万吨铁矿石。

陶瓷业因为有辽朝、宋朝的基础也比较发达。金熙宗时,原来的北方名窑如陕西耀州窑、河南均窑、河北定州窑與磁州窑也陆续恢复生产,临汝等新兴窑址,工艺各具特色。金银业和玉器业也相当发达,近年有许多珍贵的文物出土。商业活动逐渐活跃,东北地区的金朝遗址和墓葬中,发现大量宋朝铜钱,可见与南方贸易的密切。山西稷山的竹纸和平阳的麻纸,闻名一时。刻书蔚然成风气,其雕板技术,可与南宋比美,當時雕版印刷业的中心在平阳。

由於生產經濟的恢复和发展,促使商业日益繁盛。金朝建国初年,各地的商业发展处于极不平衡的状态。當時女真的龍興之地東北地區还是“无市井,买卖不用钱,惟以物相贸易”,而金中都與開封府都是興盛的商業城市。金朝建立不少的“榷场”,与西夏和南宋进行贸易。不过宋金之间由于时战时和,榷场的贸易受到一定的影响。金朝主要向南宋输出皮革、人参、纺织品等商品,南宋向金朝输入茶、药材、丝织品等。会宁府、金中都、开封府與济南府都是当时较大的商业中心。

金中都(今北京市)在金帝完顏亮正隆间成为国都后,水陆交通发达,人口迅速增加,已經是一座貿易發達的商業重鎮,其中城北三市是商业的中心。金世宗时,开封府的相国寺仍旧每月逢三、八日开寺,商贩集中在此贸易,宣德楼门下“浮屋”中买卖者甚众。1152年,共有二十三万五千多户。以后到金章宗泰和时,又增加到一百七十四万多口。

金朝币制,钱、钞、银三种并行。金朝早期使用舊有的宋、遼錢幣,直到金、宋間第二次議和後,戰爭暫告結束。1158年金帝完顏亮首次鑄行正隆通寶小平錢。1204年金章宗鑄泰和通寶真書錢。1213年金帝完顏永濟鑄至寧元寶錢。金宣宗南遷後也鑄行貨幣。金朝滅掉北宋以後,曾扶植劉齊立國,所鑄錢幣卻清秀娟美,比一般北宋錢精整。

金朝文化在發展中已達到很高水平,它「一變五代、遼季衰陋之俗」,「大定以後,其文筆雄健,直繼北宋諸賢」。在某些方面亦非宋朝可比,啟後世文化發展之先聲。金廷推行汉化政策,從“借才异代”走向“国朝文派”,逐渐形成了不同于宋朝的独特气派、风貌,但其剽悍勇猛的崇武精神隨著金朝政權的穩固而逐漸消失,最後終至亡國。元朝时,亡金故老喜言“金以儒亡”,此說未必正確,但金人全盤漢化則是不爭的事實。刘祁说:“南渡后,诸女真世袭猛安、谋克往往好文学,喜与士大夫游。”金熙宗以下的帝王都具有相当高的汉文化素养。元代有一说:“帝王知音者五人:唐玄宗、后唐庄宗、南唐后主、宋徽宗、金章宗”。金朝中期以降,女真人改汉姓、着汉服的现象越来越普遍,朝廷屢禁不止。金世宗一向反对女真人全盘汉化,积极倡导学习女真字、女真语,但仍挽不回女真漢化的速度。接受汉文化最快、汉化程度最深的首先是女真上层貴族社会,郝经谓金朝“粲粲一代之典与唐、汉比隆,讵元魏、高齐之得厕其列也”。赵翼亦稱“金源一代文物,上掩辽而下轶元”。金代文化艺术继辽、北宋之后而不断发展,超过了辽,在北宋之后与南宋平行,构成当时中国文化发展的南北两大支。在中国文化艺术发展史中起着“上掩辽而下轶元”的作用。

金朝以儒家為統治人民的基本思想,而道家、佛教與法家亦較廣泛流傳和應用。金朝思想家討論批判兩宋理學與經義學,讓理學再度於北方興起,發揚中華思想。在學術思想方面,趙秉文被稱為「儒之正理之主」,他批評漢以來的傳注之學,充分肯定周濂溪、二程(程顥、程頤)建立的北宋理學。並且將佛教、道家與理學思想融合一體,以衛道統名於金。王若虛批評傳注之學,其弊不可勝言,肯定北宋理學」。然而他也批評北宋理學,並曾下功夫對兩宋理學注釋加以評論和褒貶,但未自成一家之言。李純甫著有《中庸集解》、《鳴道集解》,其思想先是由儒教轉向道教、最後轉向佛教,「號中國心學,西方文教」。他說:「學至於佛則無所學」,以為宋伊川諸儒「皆竊吾佛書」。為了達到以佛為主的儒、道、佛三教合一,大膽地向兩宋理學開戰。

在政治思想方面,趙秉文認為王室與列國、華與夷、中國與四境的關係都是可變的;認為有公天下之心的都稱「漢」,認為社稷與民相比,民貴而社稷輕,反對唐開元末「禍始於妃后,成於宦豎,終於藩鎮」的提法,認為禍害的根源在「明皇」。王若虛認為統一中國要講「曲直之理」。他認為歐陽修不講曲直的統一,是「曲媚本朝,妄飾主闕」。他認為國之存亡可付之天數,但不能以守忠節犯食人之罪,並且讚許司馬光對傳統正閏觀的批評,「正閏之說,吾從司馬公」。

金朝如同宋朝一樣,尊崇儒學與孔子。早在金軍進軍曲阜時,金兵意圖摧毀孔子墓,即被完顏宗翰制止。自金熙宗時開始尊孔,在金上京立孔廟,又封孔子後裔為衍聖公。雖然金帝完顏亮輕視儒學,到金世宗與金章宗時又大力尊孔崇儒,修孔廟與廟學,並且推崇《尚書》、《孟子》。

金朝初期的文學比較樸陋,文學家大多是韓昉等遼人與宋人。直到蔡珪出現,才被稱為金朝文學正傳之宗,其他尚有党懷英,其他還有趙渢、王庭筠、王寂、劉從益等。金章宗時期有名的文學家有趙秉文、楊雲翼、李純甫與元好問等,女真人中有名的有金帝完顏亮與金章宗。金帝完顏亮南下侵宋時,在揚州賦詩,有句云:「提兵百萬西湖側,立馬吳山第一峰。」海陵王立志滅宋統一,作詩言志,筆力雄健,氣象恢宏。金章宗酷愛詩詞,製作甚多,但意境只在宮中生活,近似宮體詩。在金章宗的倡導下,女真貴族官員也多學作漢詩。豫王完顏允成的詩歌,編為《樂善老人集》行世。下至猛安、謀克,也努力學詩。如猛安術虎玹、謀克烏林答爽都和漢人士大夫交遊,刻意學詩。金朝有名的文人為王若虛與元好問。王若虛著有《滹南遺老集》,擅長詩文與經史考證,初步建立了文法學和修辭學,其論史則攻擊宋祁,論詩文則尊蘇軾而抑黄庭坚,是金朝具有權威的評論家,後來潘升霄的《金石文例》即受其影響。元好問是金朝文學集大成者,著有《遺山文集》。他的《論詩絕句》30首,重在衡量作家,開後來論詩的一個重要派別。元好問的《中州集》是以詩存史,他把各地區、各族的詩人均視為中州人物,這是統一的包括各族在內的中華思想的具體反映。

雜劇與戲曲在金朝得到相當的發展,已盛行以雜劇的形式作戲。金代院本的發展,為後來元曲的雜劇打下了基礎。金章宗時期的董解元的《西廂記諸宮調》,是中國古典戲劇中一部帶典範性的劃時代傑作。他是根據唐朝元稹《鶯鶯傳》改寫,但是在思想還是藝術方面都突破傳統思想的束縛,被稱為「古今傳奇鼻祖」、「北曲之祖」。

女真文和汉文是金朝通行的官方文字,其中女真文是根据汉字改製的契丹字拼写女真语言而制成的。女真族原採用契丹字,隨著金朝的建立,完顏希尹奉金太祖之令,參考漢文與契丹文創造女真文,並且在1191年八月頒行。1165年徒單子溫參考契丹字譯本,譯成《貞觀政要》、《白氏策林》等書。金世宗時,朝廷設立譯經所,翻譯漢文經史為女真文,而後又陸續翻多多本漢文書籍。金世宗對宰相們說:「朕之所以命令翻譯五經,是要女真人知道仁義道德所在」。然而當時女真字與漢字對譯,都要先譯成契丹字,然後再轉譯。金章宗時,專設弘文院譯寫儒學經書,命學官講解。1191年罷廢契丹字,規定今後女真字直譯為漢字。但隨著漢語的通用,女真貴族多已識讀漢字。漢字書籍在女真族中廣泛流行。

金朝宗教大都主张顺从和忍耐,主要和北方漢族與異族統治者有關。无论是金代的佛教还是道教,都主张以本教义为主的佛、道、儒的三者合一,如在佛教的理论发展中有很高造诣的万松行秀和李纯甫。全真教创始人王喆,凡立会也必以三教名之,完颜璹的《全真教祖碑》:“足见其冲虚明妙,寂静圆融,不独居一教也。”王喆从三教合一的主张出发,劝人们诵《道德清静经》、《般若心经》及《孝经》等道、佛、儒三家经典。

佛教早在女真族時期即有流傳,金朝灭辽朝及北宋后,又受中原佛教的影响,对佛教的信仰更加发展。佛教如華嚴、禪、淨、密教、戒律各宗都有相當的發展。其中禪宗尤為盛行,這可說完全受了北宋佛教的影響,对金代的社会经济、政治、文化和习俗都有重要影响。。女真族佔領中原後,道詢繼承淨如在靈岩寺弘法,著有《示眾廣語》、《遊方勘辨》、《頌古唱贊》諸篇。汴梁則有佛日大弘法化,傳法弟子圓性於大定間應請主持燕京潭柘山寺,大力復興禪學,著有語錄三編行世。萬松行秀尤為金代著名禪師。傳曹洞青源一系之禪,嗣法磁州大明寺雪岩滿禪師,雖治禪學,而平時恆以《華嚴》為業。他曾在從容庵評唱天童的《頌古百則》,撰《從容錄》,為禪學名著。他兼有融貫三教的思想,常勸當時重臣耶律楚材以儒治國,以佛治心,極得楚材的稱頌,說他「得曹洞的血脈,具雲門的善巧,備臨濟的機鋒」,一時傳為佳評。

道教到金朝出現全真教、大道教和太一教等三大新興道派。全真教创始人是王喆,於1167年创建全真教,後由他的七位弟子輪流接任。全真教除了繼承了中國傳統道教思想以外,更將符錄、丹藥等思想以外的內容重新整理,為今時今日的道教奠下了根基。大道教创始人是金初刘德仁,于1142年开始传道。主張「守氣養神」,提倡自食其力,少思寡慾,不談飛升煉化,長生不老,並且把儒家思想納入自己的體系。此外,大道教有出家制度。太一教始祖萧抱珍,於1138年創建。以符籙道法為主,也有守柔弱的內煉之法。尊奉太一。太一教模仿天師道的秘傳原則,每代掌教人必須改姓「蕭」。其立教宗旨是「度群生於苦厄」,尊重人倫。

女真人信仰萨满教,它是一种包括自然崇拜、图腾、万物有灵、祖先崇拜、巫术等信仰在内的原始宗教。萨满是沟通人与神之间的中介,在重大典礼、事件和节日的祭祀时都有巫师参加,或由他们司仪。消灾治病、为人求生子女、诅咒他人遭灾致祸等,几乎都成为萨满的活动内容。

金代艺术的发展,也在各方面取得很高成就。金章宗设书画院,收集民间和南宋收藏的名画,王庭筠与秘书郎张汝方鉴定金朝所收藏书画550卷,并分别定出品第。1127年金兵攻破北宋京城汴梁,就掠奪宋廷藏畫、俘擄畫工北去。金代宮廷講求書畫名蹟的收藏,以所獲北宋和內府藏畫為基礎,復從民間徵集加以充實。金朝繪畫在漢文化影響下,比遼朝繪畫更為隆盛,特別是金世宗至金章宗時期,繪畫活動益趨活躍。金章宗善詩文書法,又愛好繪畫,他在政府秘書監下設書畫局,將藏畫加以鑑定,又效宋徽宗書體在名作上題簽鈐印。金朝還在少府監下設圖畫署,“掌圖畫鏤金匠”,當時有名的畫有虞仲文《飞骏图》、王庭筠《枯木》、张珪《神龟图》、趙霖所繪《昭陵六駿圖卷》等,其中以张瑀《文姬归汉图》为最佳。金帝完颜亮能画竹,完顏允恭画獐鹿人物,王庭筠善山水墨竹,王邦基善画人物,徐荣之善画花鸟,杜锜画鞍马。武元直、李山與王庭筠等山水竹石畫作,比起同時南宋院畫家的作品,似乎更顯出「文人」的品味。

金代書法家學自北宋書法,金章宗學宋徽宗的瘦金體,很有成就。王競擅長草隸,尤工大字,兩都宮殿榜題都是競所書。党懷英擅長篆籀,為學者所宗。趙渢擅長正、行、草書,亦工小篆,正書體兼顏、蘇,書畫雄秀,當在石曼卿上;行草書備諸家體,時人以渢配党懷英小篆,號「黨、趙」。吳激得其岳父米芾筆意,王庭筠在當時學米諸人中,造詣最深,其書法為元初巙子山諸人所不及。任詢具有多方面的才藝,書法為當時第一,《中州集》稱他:「畫高於書,書高於詩,詩高於文。」。

金代初年,女真族的樂器只有鼓、笛兩種,歌詠只有「鷓鴣」一曲,「高下長短,鷓鴣二聲而已」。進入宋境後,金軍掠取宋朝教坊的樂工、樂器、樂書,漢族的音樂融入女真族的音樂之中。金世宗設宴招待南宋與西夏使者,樂人學宋朝,但服裝不同。金朝舞蹈源自先人靺鞨的靺鞨樂,立國後基本上直接吸受自北宋舞蹈,同時也發揚女真族的樂舞文化。在戲曲方面,北宋流行的諸宮調到金朝成為主要的說唱品種。當時只有董解元的《西廂記諸宮調》和《劉知遠》流傳至今,其中《西廂記諸宮調》的出現,有著元曲初步形成的意義。

金朝的科學技術也有很大的發展。醫學方面產生許多學派,不同創新的理論與爭鳴對元朝醫學與後世的醫學有较大的影響;北方農業技術在比較落後的基礎上有迅速的發展;數學方面在金元之際發展出天元術;天文曆算方面修正大明曆使其精確;此外,建築方面也有很大的發展,興建盧溝橋、金中都、山西大同華嚴寺等建築。

从靖康之变后到蒙古时期,由於频繁的战争和暴政,加上频繁的自然灾害,导致人民生活贫苦,疾病流行,使得醫學十分活躍,被稱為新学肇兴。金朝時期發展出刘完素的火热说、张从正的攻邪说與李东垣的脾胃说。由于实践的丰富,不少医家深入研究古代的医学经典,结合各自的临床经验,自成一说,来解释前人的理论,逐渐形成了不同的流派。刘完素開創了河間學派、張元素開創了易水學派,張元素的弟子李东垣又自創了脾胃學說,這三家與元朝朱震亨的养阴说合稱金元四大家,對中医理论的發展产生重要的影響。

金朝吸收北宋的農業技術,使得東北金上京一帶的農業產量得以提升。現今考古學家還在今東北地區挖掘許多金代使用的犁鏵、瓠種等鐵製農具。當時金朝與西夏等地區有名的農書有《務本新書》、《士農必用》等農書,可惜現已失傳。當時養殖蠶桑與園藝的技術也十分發達,例如利用「牛糞覆棚」將西瓜種植於較寒冷的東北地區。

當時數學最重要的進展是天元術的發展,天元術即是古代中國建立高次方程的方法,其中「天元」相當於現在的未知數。1248年金元時期的數學家李冶在其著作《測圓海鏡》、《益古演段》中,系統地介紹了用天元術建立二次方程。金廷學習北宋建立司天監以觀測天文,當時的數學也十分發達,使得金朝士人熱中編寫曆書。金廷於1137年頒布楊級編寫的《大明曆》(與祖沖之的《大明曆》不同)。而後趙知微於1180年修編成較精確的《重修大明曆》,其精確度超過宋朝優越的曆法《紀元曆》。同時間耶律履也編出《乙未曆》,然而精確度不如《重修大明曆》。


%% -*- coding: utf-8 -*-
%% Time-stamp: <Chen Wang: 2019-12-26 11:16:58>

\section{太祖\tiny(1115-1123)}

\subsection{生平}

金太祖完顏阿骨打(1068年8月1日-1123年9月19日),漢名完顏旻,金朝開國皇帝(1115年1月28日—1123年9月19日在位)。按出虎水(今黑龍江省哈爾濱東南阿什河)女真族完顏部酋長烏骨迺之孫,劾里鉢之次子,完顏部首領。善騎射,力大過人。在位9年,終年56歲。

祖父是生女真完顔部的族長烏古廼(景祖)、父劾里鉢是烏古廼的次子。阿骨打是劾里鉢的次子)。生母是女真挐懶部首長的女兒翼簡皇后。

遼國天慶三年(1113年)十月,其兄烏雅束死,繼位女真各部落聯盟長,稱都勃極烈。天慶四年,率2500人起兵叛遼,破寧江州(今吉林省扶餘市東南)。蕭嗣先率7000精兵集結於出河店,阿骨打率兵3700乘夜奔襲,渡混同江(今松花江),大敗遼軍。天慶五年農歷正月初一(1115年1月28日),阿骨打在會寧(今黑龍江省哈爾濱市阿城區南白城)稱帝,建立大金,年號收國,改名完顏旻。天慶五年九月,攻佔黃龍府(今吉林省農安縣)城。

天輔三年(1119年),遼天祚帝冊封完顏旻為東懷國皇帝,但冊文不稱完顏旻為兄長、國號不稱大金,故他不接受冊封,繼續攻打遼國。

天輔四年(1120年),與宋朝訂攻遼計劃,攻陷遼上京臨潢府(今內蒙古自治區巴林左旗南)。天輔六年(1122年),取遼中京(今內蒙古自治區寧城縣西);是年年底,攻陷燕京(今北京市)。天輔七年(1123年)八月,返金上京(今黑龍江省哈爾濱市阿城區附近)途中病逝。他死後,在天會三年六月上諡號大聖皇帝,同年十二月改為大聖武元皇帝,廟號是太祖。皇統五年十月,增諡為應乾興運昭德定功仁明莊孝大聖武元皇帝。

2003年9月5日,北京市政府文物局發表:1980年代在北京市西南郊外的九龍山的金朝陵墓,證實是完顏阿骨打的石棺、遺骨及裝飾物。

阿骨打痛恨遼,但對宋相當和善,在建國之初就有意與宋聯合,和後來諸代金朝帝王對宋朝充滿敵對大不相同。《靖康稗史箋證》中記錄其二子完顏宗望曾說過:「太祖止我伐宋,言猶在耳」。 當宋以「海上之盟」求燕京(今北京西南)及西京(今山西大同)地,金國大臣左企弓(張覺叛金時被殺)曾勸阿骨打不要歸還「燕雲十六州」,但阿骨打還是如約歸還了「燕雲十六州」中的燕京、涿州、易州、檀州、順州、景州、薊州。其中景州雖在長城之內,但並不屬於石敬瑭割給遼的燕雲十六州之一。易州是遼統和七年(989年)夺自宋,也不算作十六州之一。莫、瀛兩州早已收復,為北宋河間府所治。這樣一來,山西、河北太行山(後明在此建內長城)以內的燕、涿、檀、順、薊、莫、瀛七州都已經歸還宋,而太行山以外的儒、媯、武、新、蔚、應、寰、朔、雲九州當時遼金尚在爭奪,金太祖也無法歸還。

和阿骨打生前相處時間較長的幾個年長兒子,如長子完顏宗幹、二子完顏宗望、四子完顏宗弼都很崇尚漢文化,這對以後金國的漢化影響很大。這也從另一個側面反映了阿骨打的喜好。

元朝官修正史《金史》脱脱等的評價是:“太祖英谟睿略,豁达大度,知人善任,人乐为用。世祖阴有取辽之志,是以兄弟相授,传及康宗,遂及太祖。临终以太祖属穆宗,其素志盖如是也。初定东京,即除去辽法,减省租税,用本国制度。辽主播越,宋纳岁币,以幽、蓟、武、朔等州与宋,而置南京于平州。宋人终不能守燕、代,卒之辽主见获,宋主被执。虽功成于天会间,而规摹运为宾自此始。金有天下百十有九年,太祖数年之间算无遗策,兵无留行,底定大业,传之子孙。嗚呼,雄哉!”

\subsection{收国}


\begin{longtable}{|>{\centering\scriptsize}m{2em}|>{\centering\scriptsize}m{1.3em}|>{\centering}m{8.8em}|}
  % \caption{秦王政}\
  \toprule
  \SimHei \normalsize 年数 & \SimHei \scriptsize 公元 & \SimHei 大事件 \tabularnewline
  % \midrule
  \endfirsthead
  \toprule
  \SimHei \normalsize 年数 & \SimHei \scriptsize 公元 & \SimHei 大事件 \tabularnewline
  \midrule
  \endhead
  \midrule
  元年 & 1115 & \tabularnewline\hline
  二年 & 1116 & \tabularnewline
  \bottomrule
\end{longtable}

\subsection{天辅}

\begin{longtable}{|>{\centering\scriptsize}m{2em}|>{\centering\scriptsize}m{1.3em}|>{\centering}m{8.8em}|}
  % \caption{秦王政}\
  \toprule
  \SimHei \normalsize 年数 & \SimHei \scriptsize 公元 & \SimHei 大事件 \tabularnewline
  % \midrule
  \endfirsthead
  \toprule
  \SimHei \normalsize 年数 & \SimHei \scriptsize 公元 & \SimHei 大事件 \tabularnewline
  \midrule
  \endhead
  \midrule
  元年 & 1117 & \tabularnewline\hline
  二年 & 1118 & \tabularnewline\hline
  三年 & 1119 & \tabularnewline\hline
  四年 & 1120 & \tabularnewline\hline
  五年 & 1121 & \tabularnewline\hline
  六年 & 1122 & \tabularnewline\hline
  七年 & 1123 & \tabularnewline
  \bottomrule
\end{longtable}


%%% Local Variables:
%%% mode: latex
%%% TeX-engine: xetex
%%% TeX-master: "../Main"
%%% End:

%% -*- coding: utf-8 -*-
%% Time-stamp: <Chen Wang: 2019-10-18 15:19:07>

\section{太宗\tiny(1123-1135)}

金太宗完顏晟(1075年11月25日-1135年2月9日),金朝第二位皇帝(1123年9月27日—1135年2月9日在位)。女真名吳乞買,金太祖之弟,身材魁梧,力大無比,能親手搏熊刺虎。在位12年,终年61岁。先后滅遼朝及北宋。

完颜吴乞买出生于1075年11月25日。天會三年二月二十日(1125年3月26日),辽天祚帝在应州被金朝将领完颜娄室等所俘,八月被解送金上京,被降为海滨王,辽朝灭亡。

天會三年(1125年)十月,发动宋金战争,令諳班勃極烈完颜斜也為都元帥,統領金軍,兵分東、西兩路,逼進北宋首都汴京,由於李綱頑強抵抗,金兵一時不能得逞,雙方訂「城下之盟」。天會四年(1126年)八月,經過半年的休整,金太宗再次命宗望、宗翰兩路軍大舉南侵,汴京再度被包圍,破郭京「六甲法」,汴京城陷。天會五年二月初六(1127年3月20日),金太宗下詔廢徽、欽二帝,貶為庶人,俘虏二帝北上,并携带掠夺来的大量财宝和皇室大臣宫女等15000人,北宋滅亡。天會六年(1128年)八月二十四日,吳乞買封宋徽宗為昏德公,宋欽宗為重昏侯,移遷五國城(今黑龍江省依蘭縣城北舊古城)。

他在位时期创建了各种典章制度,奠定金代经国规模,晚年改变兄终弟及的旧制,立太祖孙完颜亶(金熙宗)为继承人。

天會十三年正月二十五日(1135年2月9日),太宗病死於明德宮,終年六十一歲。遺體葬和陵。其后代全被海陵王完颜亮所杀,海陵王遷都後,改葬於大房山,稱金恭陵。

他死後,於天會十三年三月七日上諡號文烈皇帝,廟號太宗。皇統五年閏十一月增諡体元应运世德昭功哲惠仁圣文烈皇帝。

吴乞买與宋太祖的畫像神似,民間相傳宋太宗當年殺太祖奪位,甚至還說吳乞買是宋太祖投胎來報仇,滅了宋太宗一家,宋高宗為了統治的正統性,寧可把帝位傳回宋太祖一脈,於是以太祖後代趙眘為養子,禪以帝位。

元朝官修正史《金史》脱脱等的評價是:“天辅草创,未遑礼乐之事。太宗以斜也、宗干知国政,以宗翰、宗望总戎事。既灭辽举宋,即议礼制度,治历明时,缵以武功,述以文事,经国规摹,至是始定。在位十三年,宫室苑籞无所增益。末听大臣计,传位熙宗,使太祖世嗣不失正绪,可谓行其所甚难矣!”

\subsection{天会}


\begin{longtable}{|>{\centering\scriptsize}m{2em}|>{\centering\scriptsize}m{1.3em}|>{\centering}m{8.8em}|}
  % \caption{秦王政}\
  \toprule
  \SimHei \normalsize 年数 & \SimHei \scriptsize 公元 & \SimHei 大事件 \tabularnewline
  % \midrule
  \endfirsthead
  \toprule
  \SimHei \normalsize 年数 & \SimHei \scriptsize 公元 & \SimHei 大事件 \tabularnewline
  \midrule
  \endhead
  \midrule
  元年 & 1123 & \tabularnewline\hline
  二年 & 1124 & \tabularnewline\hline
  三年 & 1125 & \tabularnewline\hline
  四年 & 1126 & \tabularnewline\hline
  五年 & 1127 & \tabularnewline\hline
  六年 & 1128 & \tabularnewline\hline
  七年 & 1129 & \tabularnewline\hline
  八年 & 1130 & \tabularnewline\hline
  九年 & 1131 & \tabularnewline\hline
  十年 & 1132 & \tabularnewline\hline
  十一年 & 1133 & \tabularnewline\hline
  十二年 & 1134 & \tabularnewline\hline
  十三年 & 1135 & \tabularnewline\hline
  十四年 & 1136 & \tabularnewline\hline
  十五年 & 1137 & \tabularnewline
  \bottomrule
\end{longtable}


%%% Local Variables:
%%% mode: latex
%%% TeX-engine: xetex
%%% TeX-master: "../Main"
%%% End:

%% -*- coding: utf-8 -*-
%% Time-stamp: <Chen Wang: 2021-11-01 16:55:19>

\section{熙宗完顏亶\tiny(1135-1149)}

\subsection{生平}

金熙宗完顏亶(1119年8月14日-1150年1月9日),金朝第三位皇帝(1135年2月10日—1150年1月9日在位)。女真名合剌,漢名亶,是金太祖完顏阿骨打之嫡長孫,父為太祖嫡長子完顏宗峻、母為蒲察氏。生于天輔三年七月七日(1119年8月14日),卒于皇統九年十二月九日(1150年1月9日)。在位15年,终年31岁。

天輔三年(己亥年)出生,本名合剌,母親是蒲察氏,父親是完顏阿骨打的嫡長子。

天會八年,諳班勃極烈完顏杲薨逝,金太宗意久未決。天會十年,左副元帥完顏宗翰、右副元帥完顏宗輔、左監軍完顏希尹等大臣進入朝廷與完顏宗幹討論國事,稱:「諳班勃極烈虛位已久,今不早定,恐授非其人。合剌,先帝嫡孫,當立。」相與請於太宗者再三,乃從之。

天會十三年正月己巳,金太宗駕崩。

1135年2月10日,即皇帝位。不久,對外公佈、並下令公私部門皆禁止飲酒與相關娛樂,並向偽齊、高麗、夏等國派遣使節稱金朝皇帝已經即位;並詔令劉齊今後稱自己為臣,不能稱子。

天會十五年(1137年)十一月丙午,为鞏固政权,金熙宗下詔廢除伪齐,降封劉豫為蜀王,並与南宋議和。十二月戊辰,劉豫上表感謝封爵。不久,發佈詔令改明年爲天眷元年,並大赦,命韓昉、耶律紹文等人編修國史。之後命令蜀王劉豫遷徙至臨潢府。

天眷二年(1139年)正月,金、宋議和成立,南宋代替偽齊政權成為金的屬國,宋對金稱臣,金朝歸還河南、陝西。但是主戰派很快占了上風。天眷三年(1140年)五月,金熙宗詔令兀朮收復河南、陝西等地。

皇統元年(1141年),完顏宗弼再次帶兵南侵,被岳飛、韓世忠等擊退,但宋高宗急於求和,再次達成紹興和議,金朝至此控制淮河以北。

皇统五年(1145年),取消辽东汉人、渤海猛安谋克世袭的制度,逐渐将兵权转移到女真人手中,分猛安谋克为上中下三等,宗室为上等,其余次之。

熙宗廢除了太祖、太宗傳下來的勃極烈制度。完顏阿骨打庶長子完顏宗幹(同時也是熙宗的養父)崇尚漢化,在開國之初太宗任命宗幹輔助朝政制定各種制度,為女真漢化及鞏固金在华北統治打下基礎。

熙宗自幼接受漢化教育,加上養父的影響,登基後開始了漢制改革、重用漢人。太祖四子完顏宗弼(又名金兀朮)是推動漢制的重臣,熙宗授以軍政大權。天會十四年(1136年),宗磐、宗幹和宗翰三人共同總管政府機構,「並領三省事」。金朝官制此時基本漢化,建立了以尚書省為中心的三省制,以三師(太師、太傅、太保)以及三公(太尉、司徒、司空)領三省事。

勃極烈制度廢除前,女真的傳統一般是同代相傳,比如景祖烏古迺將權力傳給世祖劾里缽,然後是劾里缽的四弟肅宗頗剌淑和五弟穆宗盈歌(長子劾者和三子劾孫因為柔善而被景袓跳過),這一輪過後才是最有勢力家族的下一代,世祖劾里缽之子康宗烏雅束、太祖阿骨打、太宗吳乞買和遼王斜也。斜也一死,太宗把皇儲諳班勃極烈的位置空閒了兩年,在大家的催促下才選了一個太祖阿骨打家族的嫡長孫作皇儲。

等到熙宗繼位後,漢化的結果就是廢除了諳班勃極烈這種舊的皇儲制度,皇帝立自己的兒子作太子。這引起了本來能在太宗朝成為太子的太宗長子完顏宗磐的不滿。为免出现宋朝太祖太宗朝纷爭局面,熙宗因此對太宗子孫比較忍讓。後來宗磐還是發動了叛亂,但被平息。

宋金议和以後,宗翰、宗幹、宗弼等太祖太宗朝的老功臣相繼秉政,熙宗臨朝一般不说话。等到皇統(1148年)十月,宗弼去世,熙宗才有機會親政。但悼平皇后裴滿氏又很潑辣,干預政事,無所忌憚。加上熙宗的兩個年幼兒子,太子濟安、魏王道濟相繼在皇統三、四年去世,帝位失嗣。熙宗便徹底崩潰,開始嗜酒如命,不理朝政,濫殺無辜,更杀死了蒙古族的俺巴孩汗,朝野人心惶惶。皇統九年(1149)十月,熙宗弟族完顏元、完顏阿愣等人因受海陵王完顏亮誣告而被熙宗全數殺害,熙宗因此被孤立,也給完顏亮日後的篡位埋下了禍根。

皇統九年十二月初九丁巳日(儒略曆1150年1月9日),被右丞相海陵王完顏亮所殺,終年31歲。

天德二年(1150年)二月庚戌,被海陵王降為東昏王,葬於皇后裴滿氏墓中。貞元三年(1155年),改葬於大房山蓼香甸諸王墓群。海陵王死後,金世宗於大定元年(1161年)十一月恢復完顏亶帝號,追諡武靈皇帝,廟號閔宗,墓稱思陵。大定十九年(1179年)四月,升祔於太廟,增諡弘基纘武莊靖孝成皇帝。大定二十七年(1187年)二月,改廟號熙宗。大定二十八年(1188年),以思陵狹小,改葬於峨眉谷,仍號思陵。

元朝官修正史《金史》脱脱等的評價是:“熙宗之时,四方无事,敬礼宗室大臣,委以国政,其继体守文之治,有足观者。末年酗酒妄杀,人怀危惧。所谓前有谗而不见,后有贼而不知。驯致其祸,非一朝一夕故也。”

\subsection{天眷}


\begin{longtable}{|>{\centering\scriptsize}m{2em}|>{\centering\scriptsize}m{1.3em}|>{\centering}m{8.8em}|}
  % \caption{秦王政}\
  \toprule
  \SimHei \normalsize 年数 & \SimHei \scriptsize 公元 & \SimHei 大事件 \tabularnewline
  % \midrule
  \endfirsthead
  \toprule
  \SimHei \normalsize 年数 & \SimHei \scriptsize 公元 & \SimHei 大事件 \tabularnewline
  \midrule
  \endhead
  \midrule
  元年 & 1138 & \tabularnewline\hline
  二年 & 1139 & \tabularnewline\hline
  三年 & 1140 & \tabularnewline
  \bottomrule
\end{longtable}

\subsection{皇统}

\begin{longtable}{|>{\centering\scriptsize}m{2em}|>{\centering\scriptsize}m{1.3em}|>{\centering}m{8.8em}|}
  % \caption{秦王政}\
  \toprule
  \SimHei \normalsize 年数 & \SimHei \scriptsize 公元 & \SimHei 大事件 \tabularnewline
  % \midrule
  \endfirsthead
  \toprule
  \SimHei \normalsize 年数 & \SimHei \scriptsize 公元 & \SimHei 大事件 \tabularnewline
  \midrule
  \endhead
  \midrule
  元年 & 1141 & \tabularnewline\hline
  二年 & 1142 & \tabularnewline\hline
  三年 & 1143 & \tabularnewline\hline
  四年 & 1144 & \tabularnewline\hline
  五年 & 1145 & \tabularnewline\hline
  六年 & 1146 & \tabularnewline\hline
  七年 & 1147 & \tabularnewline\hline
  八年 & 1148 & \tabularnewline\hline
  九年 & 1149 & \tabularnewline
  \bottomrule
\end{longtable}


%%% Local Variables:
%%% mode: latex
%%% TeX-engine: xetex
%%% TeX-master: "../Main"
%%% End:

%% -*- coding: utf-8 -*-
%% Time-stamp: <Chen Wang: 2021-11-01 16:56:42>

\section{废帝完颜亮\tiny(1150-1161)}

\subsection{生平}

完顏亮(1122年2月24日-1161年12月15日),字元功,女真名迪古乃,金朝第四代皇帝(1150年1月9日-1161年12月15日),金太祖阿骨打之孙,太祖庶长子遼王完顏宗幹第二子,母大氏。

完顏亮弒金熙宗而篡位,任內遷都燕京(今北京),把金朝的政治中心遷至華北,逐步汉化,使北京自此逐漸成為中國的政治中心。因伐南宋的采石大战失利,被部下所殺。完颜亮在位12年,终年40岁。完颜亮死後,继位的金世宗将他追貶為庶人,史称海陵煬王、海陵庶人、金废帝。

完颜亮生于1122年2月24日(天辅六年正月十六丙子日天眷三年)。(1140年)十八歲時以宗室子為奉國上將軍,赴梁王完顏宗弼(兀朮)幕府任使,管理萬人,遷驃騎上將軍。皇統四年(1144年),加龍虎衛上將軍,為金國中京(位於今北京市一帶)留守,遷光祿大夫。

皇統七年(1147年)五月,召入當時的金國首都上京(今黑龍江省哈尔滨市阿城区內)為同判大宗正事,加特進。十一月,拜尚書省左丞,把持了權柄,安插自己的心腹擔任要職,其中蕭裕 成為兵部侍郎。十一月某日和熙宗談話時,談到金太祖創業艱難,完顏亮痛哭流涕,熙宗認為他很忠心。後來升職加快。第二年(1148年)六月,拜平章事。十一月,拜右丞相。1149年正月,兼都元帥。三月,拜太保、領三省事,更加八面玲瓏,和有權勢家族來往密切,結其歡心。

1149年熙宗對完顏亮突然膨脹的勢力不滿。正月,熙宗派寢殿小底大興國以宋名臣司馬光畫像及其它珍玩賜完顏亮生日禮物,悼平皇后裴滿氏也附賜禮物,結果引起熙宗不悅,罰小底大興國一百杖,追回其賜物,完顏亮知道後由此不安。四月,學士張鈞起草詔書時擅自改動,被查出處死。熙宗問是誰指使的,左丞相完顏宗賢回答說是太保完顏亮。熙宗不悅,遂貶完顏亮到汴京(今河南開封),領行台尚書省事。完顏亮路過中京時,和那裡的兵部侍郎蕭裕密謀定約而去。走到良鄉,又被熙宗召還。完顏亮不知熙宗的意圖,非常恐懼。回到上京,又恢復為平章政事。但完顏亮反意已決。

《金史》說完顏亮“為人僄急,多猜忌,殘忍任數。”當熙宗以太祖的嫡孫身份嗣位時,完顏亮認為自己是太祖長子完顏宗幹的兒子,也是太祖的孫子,所以對皇位“遂懷覬覦。”早在皇統七年(1147年),熙宗就開始胡亂發脾氣殺人,比如賜宴時因為一些小事濫殺無辜,引起朝臣的不滿。皇統八年(1148年)七月,以駙馬尚書左丞唐括辯奉職不謹,杖之。皇統九年(1149年)八月,杖平章政事完顏秉德。對熙宗不滿的人即有廢立的想法,唐括辯、秉德先和大理卿烏帶(完顏言)謀劃廢掉熙宗,而烏帶就此引入完顏亮。完顏亮與唐括辯密謀廢立,問到若廢熙宗,可以立誰繼位?唐括辯與秉德初意並不在完顏亮。唐括辯說胙王完顏常勝(完顏元)似乎可以。完顏亮再問其次是誰,唐括辯說鄧王完顏奭之子完顏阿楞可以。完顏亮反駁說阿楞不行。唐括辯反問:“公豈有意邪?”完顏亮說:“果不得已,舍我其誰!”不久完顏亮和唐括辯等旦夕密謀,引起了護衛將軍完顏特思的懷疑。特思告訴了悼平皇后裴滿氏,因此熙宗得知。熙宗發怒召唐括辯並杖之。完顏亮因此非常忌諱完顏元、完顏阿楞,並且極其討厭完顏特思。

正好當時河南有士兵孫進冒稱皇弟按察大王,而熙宗之弟只有完顏元和完顏查剌。熙宗懷疑是完顏元,派完顏特思調查,卻甚麼也沒有。完顏亮乘機誣陷,對熙宗說:“孫進反有端,不稱他人,乃稱皇弟大王。陛下弟止有常勝、查刺。特思鞫不以實,故出之矣。”熙宗以為然,派唐括辯、蕭肄拷問完顏特思,完顏特思被逼招認,完顏元於是獲罪。十月,殺完顏元,一併連完顏查刺、完顏特思、完顏阿楞以及阿楞弟完顏撻楞 一起殺掉。這樣一來,熙宗殺光了自己的親兄弟,更加孤立。

到了皇統九年(1149年)十二月,要廢熙宗的人已經結黨行事。從前因送禮一事被杖責一百的大興國,因為和完顏亮的心腹尚書省令史李老僧是親戚,於是和完顏亮結黨,當時正在伺候熙宗在寢殿內的起居生活,總是有意無意地乘夜從主事者那裡帶皇宮鑰匙回家,大家習以為常。護衛十人長僕散忽土要報答完顏亮之父完顏宗幹的舊恩,徒單阿里出虎是完顏亮的姻親。十二月初九丁巳日(儒略曆1150年1月9日),此二人值班之夜,大興國用皇宮鑰匙打開所有宮門,和完顏亮、秉德、唐括辯、烏帶、徒單貞、李老僧 至寢殿。熙宗本來常置佩刀於床上,這天夜裡大興國先取之放到床下,等到事發,熙宗求佩刀不得,遂遇弒。眾人拜完顏亮為皇帝。改皇統九年為天德元年。並假稱熙宗想要商議立皇后事宜,召眾大臣入宮,殺曹國王完顏宗敏、左丞相完顏宗賢。

贞元元年三月二十六日(1153年4月21日),完颜亮正式迁都,改燕京为中都,定名为中都大兴府,同時定北宋故都開封府為金南京,使金朝逐步汉化。

海陵王在位期間不但擴大皇帝權威,甚至於濫用權力,誅殺大臣;而且海陵王的宮廷生活相當荒淫,史載「營南京(燕京)宮殿,運一木之費至二千萬,率一車之力至五百人。宮殿之飾,遍傅黃金而後間以五彩,金屑飛空如落雪。一殿之費以億萬計,成而復毀,務極華麗。」(《金史》)。據說他讀罷柳永的《望海潮》一詞:「東南形勝,三吳都會,錢塘自古繁華……有三秋桂子,十里荷花」,「遂起投鞭渡江、立馬吳山之志」,即興題詩稱:“万里车书一混同,江南岂有别疆封? 提兵百万西湖侧,立马吴山第一峰。”(《鶴林玉露》卷一) 完顏亮曾大顏不慚地說:「吾有三志,國家大事,皆我所出,一也;帥師伐遠,執其君長問罪於前,二也;得天下絕色而妻之,三也。」而被他收入深宮而「妻之」的「天下絕色」,竟有他的堂姐妹、叔母 、舅母、外甥女、侄女以及弟媳、小姨子等等。完顏亮上台後,為了壓制皇族宗室的反抗,曾大加誅戮,諸叔及其子弟幾乎屠殺殆盡,他們的妻子、女兒,或被納為嬪妃,或被強納宮中,「命諸從姊妹皆分屬諸妃,出入禁中,與為淫亂。」昭妃阿懶,就是完顏亮的親嬸嬸,完顏亮殺死叔叔曹國王宗敏,便把阿懶納入宮中,封為昭妃。

紹興三十一年(正隆六年,1161年)出兵伐宋,進迫長江。但是東京留守曹國公完顏雍杀副留守高存福,自立為帝,是为金世宗。采石大战中了南宋江淮參軍虞允文的埋伏,退兵瓜洲渡,命令所有士兵即刻南征,軍心大亂,为部下完颜元宜所弑,享年40歲。死後追貶為「海陵王」,又追貶為「海陵庶人」,被以庶人之禮安葬。

金世宗大定二年(1162年)四月,降封為海陵郡王,諡号为煬,所以又称海陵煬王,葬於大房山鹿門谷諸王的墓地中。大定二十一年(1181年)正月,由于为海陵王所弒的金熙宗于大定十九年供入太廟,完顏亮又再被降為海陵庶人,改葬于山陵西南四十里。今北京市房山区有海陵王陵。

南宋洪邁出使金世宗後歸國,向宋高宗報告完顏亮被諡為「煬」的事。宋高宗表示,當時人們都把完顏亮比作苻堅,唯獨他認為完顏亮與隋煬帝類似。宋高宗因此認為,隋煬帝和完顏亮死在同一地方,又被加上同一諡號,乃是天意。元朝官修正史《金史》脱脱等的評價是:“海陵智足以拒諫,言足以飾非。欲爲君則弑其君,欲伐國則弑其母,欲奪人之妻則使之殺其夫。三綱絕矣,何暇他論。至于屠滅宗族,剪刈忠良,婦姑姊妹盡入嬪御。方以三十二總管之兵圖一天下,卒之戾氣感召,身由惡終,使天下後世稱無道主以海陵爲首。可不戒哉!可不戒哉!”。其中,「智足以拒諫,言足以飾非」一句,是司馬遷在《史記》中對商紂王的評語原文。《醒世恆言》中有《金海陵縱慾亡身》一篇(改編自更早的話本),將海陵王描寫為淫蕩的昏君,使得海陵王的負面形象深入人心。

\subsection{天德}


\begin{longtable}{|>{\centering\scriptsize}m{2em}|>{\centering\scriptsize}m{1.3em}|>{\centering}m{8.8em}|}
  % \caption{秦王政}\
  \toprule
  \SimHei \normalsize 年数 & \SimHei \scriptsize 公元 & \SimHei 大事件 \tabularnewline
  % \midrule
  \endfirsthead
  \toprule
  \SimHei \normalsize 年数 & \SimHei \scriptsize 公元 & \SimHei 大事件 \tabularnewline
  \midrule
  \endhead
  \midrule
  元年 & 1149 & \tabularnewline\hline
  二年 & 1150 & \tabularnewline\hline
  三年 & 1151 & \tabularnewline\hline
  四年 & 1152 & \tabularnewline\hline
  五年 & 1153 & \tabularnewline
  \bottomrule
\end{longtable}

\subsection{贞元}

\begin{longtable}{|>{\centering\scriptsize}m{2em}|>{\centering\scriptsize}m{1.3em}|>{\centering}m{8.8em}|}
  % \caption{秦王政}\
  \toprule
  \SimHei \normalsize 年数 & \SimHei \scriptsize 公元 & \SimHei 大事件 \tabularnewline
  % \midrule
  \endfirsthead
  \toprule
  \SimHei \normalsize 年数 & \SimHei \scriptsize 公元 & \SimHei 大事件 \tabularnewline
  \midrule
  \endhead
  \midrule
  元年 & 1153 & \tabularnewline\hline
  二年 & 1154 & \tabularnewline\hline
  三年 & 1155 & \tabularnewline\hline
  四年 & 1156 & \tabularnewline
  \bottomrule
\end{longtable}

\subsection{正隆}

\begin{longtable}{|>{\centering\scriptsize}m{2em}|>{\centering\scriptsize}m{1.3em}|>{\centering}m{8.8em}|}
  % \caption{秦王政}\
  \toprule
  \SimHei \normalsize 年数 & \SimHei \scriptsize 公元 & \SimHei 大事件 \tabularnewline
  % \midrule
  \endfirsthead
  \toprule
  \SimHei \normalsize 年数 & \SimHei \scriptsize 公元 & \SimHei 大事件 \tabularnewline
  \midrule
  \endhead
  \midrule
  元年 & 1156 & \tabularnewline\hline
  二年 & 1157 & \tabularnewline\hline
  三年 & 1158 & \tabularnewline\hline
  四年 & 1159 & \tabularnewline\hline
  五年 & 1160 & \tabularnewline\hline
  六年 & 1161 & \tabularnewline
  \bottomrule
\end{longtable}


%%% Local Variables:
%%% mode: latex
%%% TeX-engine: xetex
%%% TeX-master: "../Main"
%%% End:

%% -*- coding: utf-8 -*-
%% Time-stamp: <Chen Wang: 2019-12-26 11:17:28>

\section{世宗\tiny(1161-1189)}

\subsection{生平}

金世宗完顏雍(天輔七年三月初一甲寅日,儒略曆1123年3月29日—大定二十九年正月初二癸巳日,儒略曆1189年1月20日)),原名完顏褎(xiù、ㄒㄧㄡˋ),金朝第五位皇帝(1161年10月27日—1189年1月20日在位)。女真名乌禄,金太祖完颜阿骨打孙,海陵王完颜亮征宋时为辽东留守,后被拥立为帝,在位28年,终年67岁,葬于兴陵(今北京市房山区)。

1161年十月初八日,完颜亮率领大军渡过淮水,进兵南宋庐州。东京辽阳府发生了政变。曹国公完颜雍时任东京留守,完颜秉德以谋立葛王完颜雍之罪被杀后,完颜雍从海路献珍宝以表明他的忠诚。完颜亮命渤海人高存福为副留守,监视完颜雍的行动。契丹撒八等起义,完颜雍出兵阻击括里。完颜亮命婆速府路总管完颜谋衍(完颜娄室之子)领兵五千助战。完颜亮自辽东征调大批女真兵南下侵宋,女真兵多不愿南下。行至山东时,南征万户、曷苏馆女真猛安完颜福寿等领一万多人,中途叛变,逃回辽阳。完颜福寿与完颜谋衍等在辽阳发动政变,杀高存福,拥立完颜雍作皇帝,即金世宗。十月初八日,金世宗下诏废黜完颜亮,改元大定。完颜谋衍为右副元帅,福寿为右监军。十一月,在东京的政权,逐渐巩固。中都留守阿琐等起而响应金世宗。金世宗决定迁赴中都。十一月二十七日拂晓,完颜元宜率领将士袭击完颜亮营帐,完颜亮被乱箭射死。

金世宗即位后,首先对南宋的进攻保持守势,着手平息契丹起义,待平息契丹起义后,开始对南宋采取强硬态度,击退了南宋的隆兴北伐,并在形势占优时,在与宋孝宗和谈时做出让步,最终签署了《隆兴和议》,开启了双方四十余年的和平局面。

金世宗在内政管理上,励精图治,革除了完顏亮统治时期的很多弊政。更值得称道的是,金世宗十分朴素,不穿丝织龙袍,使金朝国库充盈,农民也过上富裕的日子,天下小康,实现了“大定盛世”的繁荣鼎盛局面,金世宗也被称为“小尧舜”。

金世宗统治时期,如移剌窩幹等各族人民纷纷起义,他为了维持统治,利用科举、学校等制度,争取汉人支持,又加强猛安谋克权力,扩大女真族占有的土地。同时多次发布有关保留女真人旧习、语言的诏令,甚或要求所有皇子必须有女真语名、所有女真官员必须通晓女真語,卫士不准讲汉语。

他死後谥号是光天兴运文德武功圣明仁孝皇帝,庙号是世宗。

元朝官修正史《金史》脱脱等的評價是:“世宗之立,虽由劝进,然天命人心之所归,虽古圣贤之君,亦不能辞也。盖自太祖以来,海内用兵,宁岁无几。重以海陵无道,赋役繁兴,盗贼满野,兵甲并起,万姓盼盼,国内骚然,老无留养之丁,幼无顾复之爱,颠危愁困,待尽朝夕。世宗久典外郡,明祸乱之故,知吏治之得失。即位五载,而南北讲好,与民休息。于是躬节俭,崇孝弟,信赏罚,重农桑,慎守令之选,严廉察之责,却任得敬分国之请,拒赵位宠郡县之献,孳孳为治,夜以继日,可谓得为君之道矣!当此之时,群臣守职,上下相安,家给人足,仓廪有余,刑部岁断死罪,或十七人,或二十人,号称“小尧舜”,此其效验也。然举贤之急,求言之切,不绝于训辞,而群臣偷安苟禄,不能将顺其美,以底大顺,惜哉!”

\subsection{大定}


\begin{longtable}{|>{\centering\scriptsize}m{2em}|>{\centering\scriptsize}m{1.3em}|>{\centering}m{8.8em}|}
  % \caption{秦王政}\
  \toprule
  \SimHei \normalsize 年数 & \SimHei \scriptsize 公元 & \SimHei 大事件 \tabularnewline
  % \midrule
  \endfirsthead
  \toprule
  \SimHei \normalsize 年数 & \SimHei \scriptsize 公元 & \SimHei 大事件 \tabularnewline
  \midrule
  \endhead
  \midrule
  元年 & 1161 & \tabularnewline\hline
  二年 & 1162 & \tabularnewline\hline
  三年 & 1163 & \tabularnewline\hline
  四年 & 1164 & \tabularnewline\hline
  五年 & 1165 & \tabularnewline\hline
  六年 & 1166 & \tabularnewline\hline
  七年 & 1167 & \tabularnewline\hline
  八年 & 1168 & \tabularnewline\hline
  九年 & 1169 & \tabularnewline\hline
  十年 & 1170 & \tabularnewline\hline
  十一年 & 1171 & \tabularnewline\hline
  十二年 & 1172 & \tabularnewline\hline
  十三年 & 1173 & \tabularnewline\hline
  十四年 & 1174 & \tabularnewline\hline
  十五年 & 1175 & \tabularnewline\hline
  十六年 & 1176 & \tabularnewline\hline
  十七年 & 1177 & \tabularnewline\hline
  十八年 & 1178 & \tabularnewline\hline
  十九年 & 1179 & \tabularnewline\hline
  二十年 & 1180 & \tabularnewline\hline
  二一年 & 1181 & \tabularnewline\hline
  二二年 & 1182 & \tabularnewline\hline
  二三年 & 1183 & \tabularnewline\hline
  二四年 & 1184 & \tabularnewline\hline
  二五年 & 1185 & \tabularnewline\hline
  二六年 & 1186 & \tabularnewline\hline
  二七年 & 1187 & \tabularnewline\hline
  二八年 & 1188 & \tabularnewline\hline
  二九年 & 1189 & \tabularnewline
  \bottomrule
\end{longtable}


%%% Local Variables:
%%% mode: latex
%%% TeX-engine: xetex
%%% TeX-master: "../Main"
%%% End:

%% -*- coding: utf-8 -*-
%% Time-stamp: <Chen Wang: 2021-11-01 16:56:52>

\section{章宗完顏璟\tiny(1189-1208)}

\subsection{生平}

金章宗完顏璟(1168年8月31日(农历七月二十七)-1208年12月29日),女真名麻達葛,金朝第6位皇帝(1189年1月20日—1208年12月29日在位),在位19年,享年41岁。章宗為金世宗完颜雍之嫡孙,其在位期間修訂國內律法,政治清明,世稱明昌之治。章宗統治下的金朝文化發展達至頂峰,但同時軍事能力卻也日益低下,蒙古帝國也於同時崛起。

南宋主戰派權臣韓侂胄於章宗年間北伐,但遭到金軍擊敗,簽定「嘉定和議」。1208年駕崩,叔衛紹王完顏永济繼位。

金世宗在大定初年立章宗之父完顏允恭為太子,允恭在大定二十五年(1185年)逝世後,世宗在次年立章宗為皇太孫。大定二十九年正月初二,世宗去世,章宗隨即繼位。

當時金朝立國七十五年,「禮樂刑政因遼、宋舊制,雜亂無貫,章宗即位,乃更定修正,為一代法。」章宗時期的政治尚算清明,後世稱為明昌之治。

章宗時代,国内的文化發展達至最高峰。他不單對國內文化發展加以獎勵,而他本身亦能寫得一手好字,與北宋徽宗的「瘦金體」形似。但與此同時,軍事能力卻日益低下,使屬國紛紛離異、並招引鄰國侵略。章宗整日与文人饮酒作诗,不思朝政。金朝日益腐朽衰败,漠北已失去控制。此外,黄河氾濫等各種天災相繼出現,使国力開始衰退。在位后期蒙古帝国崛起,成为了日后金覆灭的隐患。

1196年,原來從屬金朝的塔塔兒部叛離,改為歸順蒙古。南宋權臣韓侂冑見金朝開始走下坡,以為有機可乘,在1206年大舉出兵攻金,結果宋軍大敗,東線金兵渡過淮河,佔領淮南多個州縣;中線金兵攻襄陽;西線宋將吳曦以四川附金,不久事敗被殺。宋寧宗殺韓侂冑向金求和,1208年「嘉定和議」成,宋尊金為伯,增加每年歲幣至銀三十萬兩、絹三十萬匹及向金朝納「犒軍錢」三百萬兩,金朝始歸還南宋失地,維持紹興和議時的局面。

1208年12月29日,金章宗駕崩。他的六個兒子都在三歲前夭折。由於他沒有後嗣,所以由叔父衛紹王完顏永济繼位。金章宗駕崩、完顏永濟繼位後,成吉思汗知道完顏永濟是個無能之輩,所以在次年立即揮軍南下開始侵略金朝。

他死後諡號是憲天光運仁文義武神聖英孝皇帝,廟號是章宗。葬于道陵。

元朝官修正史《金史》脱脱等的評價是:“章宗在位二十年,承世宗治平日久,宇内小康,乃正礼乐,修刑法,定官制,典章文物粲然成一代治规。又数问群臣汉宣综核名实、唐代考课之法,盖欲跨辽、宋而比迹于汉、唐,亦可谓有志于治者矣!然婢宠擅朝,冢嗣未立,疏忌宗室而传授非人。向之所谓维持巩固于久远者,徒为文具,而不得为后世子孙一日之用,金源氏从此衰矣!昔扬雄氏有云:‘秦之有司负秦之法度,秦之法度负圣人之法度。’盖有以夫。”

\subsection{明昌}


\begin{longtable}{|>{\centering\scriptsize}m{2em}|>{\centering\scriptsize}m{1.3em}|>{\centering}m{8.8em}|}
  % \caption{秦王政}\
  \toprule
  \SimHei \normalsize 年数 & \SimHei \scriptsize 公元 & \SimHei 大事件 \tabularnewline
  % \midrule
  \endfirsthead
  \toprule
  \SimHei \normalsize 年数 & \SimHei \scriptsize 公元 & \SimHei 大事件 \tabularnewline
  \midrule
  \endhead
  \midrule
  元年 & 1190 & \tabularnewline\hline
  二年 & 1191 & \tabularnewline\hline
  三年 & 1192 & \tabularnewline\hline
  四年 & 1193 & \tabularnewline\hline
  五年 & 1194 & \tabularnewline\hline
  六年 & 1195 & \tabularnewline\hline
  七年 & 1196 & \tabularnewline
  \bottomrule
\end{longtable}

\subsection{承安}

\begin{longtable}{|>{\centering\scriptsize}m{2em}|>{\centering\scriptsize}m{1.3em}|>{\centering}m{8.8em}|}
  % \caption{秦王政}\
  \toprule
  \SimHei \normalsize 年数 & \SimHei \scriptsize 公元 & \SimHei 大事件 \tabularnewline
  % \midrule
  \endfirsthead
  \toprule
  \SimHei \normalsize 年数 & \SimHei \scriptsize 公元 & \SimHei 大事件 \tabularnewline
  \midrule
  \endhead
  \midrule
  元年 & 1196 & \tabularnewline\hline
  二年 & 1197 & \tabularnewline\hline
  三年 & 1198 & \tabularnewline\hline
  四年 & 1199 & \tabularnewline\hline
  五年 & 1200 & \tabularnewline
  \bottomrule
\end{longtable}

\subsection{泰和}

\begin{longtable}{|>{\centering\scriptsize}m{2em}|>{\centering\scriptsize}m{1.3em}|>{\centering}m{8.8em}|}
  % \caption{秦王政}\
  \toprule
  \SimHei \normalsize 年数 & \SimHei \scriptsize 公元 & \SimHei 大事件 \tabularnewline
  % \midrule
  \endfirsthead
  \toprule
  \SimHei \normalsize 年数 & \SimHei \scriptsize 公元 & \SimHei 大事件 \tabularnewline
  \midrule
  \endhead
  \midrule
  元年 & 1201 & \tabularnewline\hline
  二年 & 1202 & \tabularnewline\hline
  三年 & 1203 & \tabularnewline\hline
  四年 & 1204 & \tabularnewline\hline
  五年 & 1205 & \tabularnewline\hline
  六年 & 1206 & \tabularnewline\hline
  七年 & 1207 & \tabularnewline\hline
  八年 & 1208 & \tabularnewline
  \bottomrule
\end{longtable}


%%% Local Variables:
%%% mode: latex
%%% TeX-engine: xetex
%%% TeX-master: "../Main"
%%% End:

%% -*- coding: utf-8 -*-
%% Time-stamp: <Chen Wang: 2021-11-01 16:57:16>

\section{紹王完颜永济\tiny(1208-1213)}

\subsection{生平}

完颜允济(?-1213年9月11日),小字興勝,金章宗時避章宗父完顏允恭諱改為完颜永济。他是金朝第七位皇帝(1208年12月29日—1213年9月11日在位),被篡位後降封衛王,卒諡「紹王」,在位5年。

完颜允济是完顏允恭之弟,金章宗之叔,金世宗完颜雍第七子,母元妃李氏。他在金世宗大定十一年(1171年)被封薛王,同年改封禭王,先後改封潞王、韓王及衛王。章宗在泰和八年(1208年)农历十一月二十日病死,無嗣,衛王完颜允济被迎立为帝。

蒙古帝國的成吉思汗有意進攻金國,首先出兵進攻臣屬金朝的西夏,西夏向金求援,卫绍王坐視不救。西夏向蒙古屈服後,成吉思汗自大安三年(1211年)起大舉攻金,屢敗金兵。是年九月,蒙古軍逼近中都,因城防堅固兼有重兵防守,於是退兵。次年成吉思汗再次親征金國,一度包圍金西京大同府。同年契丹人耶律留哥在今吉林省境起兵反金,數月之間發展至十餘萬人。耶律留哥依附蒙古,又在迪吉腦兒(今辽宁昌图附近)擊敗六十萬金兵,金國的處境更加不妙。

衛绍王为人优柔寡断,没有安邦治国之才,只是俭约守成而已。他不善于用人,忠奸不分,最终导致杀身之祸。至寧元年(1213年)八月,蒙古軍再次逼近中都,右副元帥胡沙虎(紇石烈執中)起兵叛亂,弑卫绍王。九月,迎立完顏珣為帝,即金宣宗。胡沙虎請廢允济為庶人,詔百官三百餘人議於朝堂。太子少傅奧屯忠孝、侍讀學士蒲察思忠支持胡沙虎,但戶部尚書武都、拾遺田庭芳等三十人請降允济為王侯。胡沙虎固執前議,金宣宗不得已,乃降封允济為東海郡侯。十月,元帥右監軍朮虎高琪殺胡沙虎。

貞祐四年(1216年),金宣宗詔追復允濟為衛王,諡曰紹,後世稱他為衛绍王。

元朝官修正史《金史》脱脱等的評價是:“卫绍王政乱于内,兵败于外,其灭亡已有征矣。身弑国蹙,记注亡失,南迁后不复纪载。皇朝中统三年,翰林学士承旨王鹗有志论著,求大安、崇庆事不可得,采摭当时诏令,故金部令史窦祥年八十九,耳目聪明,能记忆旧事,从之得二十余条。司天提点张正之写灾异十六条,张承旨家手本载旧事五条,金礼部尚书杨云翼日录四十条,陈老日录三十条,藏在史馆。条件虽多,重复者三之二。惟所载李妃、完颜匡定策,独吉千家奴兵败,纥石烈执中作难,及日食、星变、地震、氛昆,不相背盭。今校其重出,删其繁杂。《章宗实录》详其前事,《宣宗实录》详其后事。又于金掌奏目女官大明居士王氏所纪,得资明夫人援玺一事,附著于篇,亦可以存其梗概云尔。”明朝官修《元史》,成吉思汗对完颜永济的評價是:“我谓中原皇帝是天上人做,此等庸懦亦为之耶?”

\subsection{大安}


\begin{longtable}{|>{\centering\scriptsize}m{2em}|>{\centering\scriptsize}m{1.3em}|>{\centering}m{8.8em}|}
  % \caption{秦王政}\
  \toprule
  \SimHei \normalsize 年数 & \SimHei \scriptsize 公元 & \SimHei 大事件 \tabularnewline
  % \midrule
  \endfirsthead
  \toprule
  \SimHei \normalsize 年数 & \SimHei \scriptsize 公元 & \SimHei 大事件 \tabularnewline
  \midrule
  \endhead
  \midrule
  元年 & 1209 & \tabularnewline\hline
  二年 & 1210 & \tabularnewline\hline
  三年 & 1211 & \tabularnewline
  \bottomrule
\end{longtable}

\subsection{崇庆}

\begin{longtable}{|>{\centering\scriptsize}m{2em}|>{\centering\scriptsize}m{1.3em}|>{\centering}m{8.8em}|}
  % \caption{秦王政}\
  \toprule
  \SimHei \normalsize 年数 & \SimHei \scriptsize 公元 & \SimHei 大事件 \tabularnewline
  % \midrule
  \endfirsthead
  \toprule
  \SimHei \normalsize 年数 & \SimHei \scriptsize 公元 & \SimHei 大事件 \tabularnewline
  \midrule
  \endhead
  \midrule
  元年 & 1212 & \tabularnewline\hline
  二年 & 1213 & \tabularnewline
  \bottomrule
\end{longtable}

\subsection{至宁}

\begin{longtable}{|>{\centering\scriptsize}m{2em}|>{\centering\scriptsize}m{1.3em}|>{\centering}m{8.8em}|}
  % \caption{秦王政}\
  \toprule
  \SimHei \normalsize 年数 & \SimHei \scriptsize 公元 & \SimHei 大事件 \tabularnewline
  % \midrule
  \endfirsthead
  \toprule
  \SimHei \normalsize 年数 & \SimHei \scriptsize 公元 & \SimHei 大事件 \tabularnewline
  \midrule
  \endhead
  \midrule
  元年 & 1213 & \tabularnewline
  \bottomrule
\end{longtable}


%%% Local Variables:
%%% mode: latex
%%% TeX-engine: xetex
%%% TeX-master: "../Main"
%%% End:

%% -*- coding: utf-8 -*-
%% Time-stamp: <Chen Wang: 2021-11-01 16:57:23>

\section{宣宗完颜珣\tiny(1213-1224)}

\subsection{生平}

金宣宗完颜珣(1163年4月18日-1224年1月14日),女真名吾睹補。金世宗完颜雍长孙,卫绍王侄,父完顏允恭,母昭華劉氏。他是金朝第八位皇帝(1213年9月22日—1224年1月14日在位),在位11年,终年61岁。

金世宗大定十八年(1178年),封溫國公,二十六年賜名珣,二十九年封豐王。承安元年,封翼王。泰和五年,改賜名從嘉,其後又改封邢王及升王。

至寧元年(1213年)八月,胡沙虎杀死卫绍王,迎立從嘉为帝,由於從嘉在河北鎮守,於是暫時以從嘉長子完顏守忠監國。九月即位,是為宣宗,以胡沙虎為太師、尚書令兼都元帥,封澤王,同月改元貞祐。閏九月,宣宗復舊名珣。十月,朮虎高琪殺胡沙虎,宣宗赦免高琪,封他為左副元帥。是年秋,蒙古軍分三路攻金,幾乎攻破所有河北郡縣,金國只有中都、真定、大名等十一城未曾失守。

貞祐二年三月,蒙金和議成,五月十八日(1214年6月27日),金宣宗逃離中都,七月金宣宗南逃到達汴京,此舉觸怒蒙古,戰爭再起。貞祐三年五月初二(1215年5月31日),中都失守,十月,蒲鮮萬奴在遼東自立。興定三年十二月(1220年初),宣宗誅高琪。

宣宗对外措施十分不当,直接导致金朝灭亡。他先向蒙古大汗成吉思汗屈辱求和,又与西夏断交,還不顧丞相徒单镒和諸多大臣等的反對,将都城由中都南迁至汴京,并且发动侵宋战争。金国三面受敌,加上內部不和,叛亂頻生,國家危在旦夕。

宣宗在元光二年十二月二十二日(1224年1月14日)去世,死後諡號是繼天興統述道勤仁英武聖孝皇帝,廟號是宣宗,葬于德陵(在今河南開封)。

元朝官修正史《金史》脱脱等的評價是:“宣宗当金源末运,虽乏拨乱反正之材,而有励精图治之志。迹其勤政忧民,中兴之业盖可期也,然而卒无成功者何哉?良由性本猜忌,崇信翙御,奖用吏胥,苛刻成风,举措失当故也。执中元恶,此岂可相者乎,顾乃怀其援立之私,自除廉陛之分,悖礼甚矣。高琪之诛执中,虽云除恶,律以《春秋》之法,岂逃赵鞅晋阳之责?既不能罪而遂相之,失之又失者也。迁汴之后,北顾大元之朝日益隆盛,智识之士孰不先知?方且狃于余威,牵制群议,南开宋衅,西启夏侮,兵力既分,功不补患。曾未数年,昔也日辟国百里,今也日蹙国里,其能济乎?再迁遂至失国,岂不重可叹哉!”

\subsection{贞祐}


\begin{longtable}{|>{\centering\scriptsize}m{2em}|>{\centering\scriptsize}m{1.3em}|>{\centering}m{8.8em}|}
  % \caption{秦王政}\
  \toprule
  \SimHei \normalsize 年数 & \SimHei \scriptsize 公元 & \SimHei 大事件 \tabularnewline
  % \midrule
  \endfirsthead
  \toprule
  \SimHei \normalsize 年数 & \SimHei \scriptsize 公元 & \SimHei 大事件 \tabularnewline
  \midrule
  \endhead
  \midrule
  元年 & 1213 & \tabularnewline\hline
  二年 & 1214 & \tabularnewline\hline
  三年 & 1215 & \tabularnewline\hline
  四年 & 1216 & \tabularnewline\hline
  五年 & 1217 & \tabularnewline
  \bottomrule
\end{longtable}

\subsection{兴定}

\begin{longtable}{|>{\centering\scriptsize}m{2em}|>{\centering\scriptsize}m{1.3em}|>{\centering}m{8.8em}|}
  % \caption{秦王政}\
  \toprule
  \SimHei \normalsize 年数 & \SimHei \scriptsize 公元 & \SimHei 大事件 \tabularnewline
  % \midrule
  \endfirsthead
  \toprule
  \SimHei \normalsize 年数 & \SimHei \scriptsize 公元 & \SimHei 大事件 \tabularnewline
  \midrule
  \endhead
  \midrule
  元年 & 1217 & \tabularnewline\hline
  二年 & 1218 & \tabularnewline\hline
  三年 & 1219 & \tabularnewline\hline
  四年 & 1220 & \tabularnewline\hline
  五年 & 1221 & \tabularnewline\hline
  六年 & 1222 & \tabularnewline
  \bottomrule
\end{longtable}

\subsection{元光}

\begin{longtable}{|>{\centering\scriptsize}m{2em}|>{\centering\scriptsize}m{1.3em}|>{\centering}m{8.8em}|}
  % \caption{秦王政}\
  \toprule
  \SimHei \normalsize 年数 & \SimHei \scriptsize 公元 & \SimHei 大事件 \tabularnewline
  % \midrule
  \endfirsthead
  \toprule
  \SimHei \normalsize 年数 & \SimHei \scriptsize 公元 & \SimHei 大事件 \tabularnewline
  \midrule
  \endhead
  \midrule
  元年 & 1222 & \tabularnewline\hline
  二年 & 1223 & \tabularnewline
  \bottomrule
\end{longtable}


%%% Local Variables:
%%% mode: latex
%%% TeX-engine: xetex
%%% TeX-master: "../Main"
%%% End:

%% -*- coding: utf-8 -*-
%% Time-stamp: <Chen Wang: 2019-12-26 11:17:50>

\section{哀宗\tiny(1224-1234)}

\subsection{生平}

金哀宗完颜守緒(1198年9月25日-1234年2月9日),金朝第九位皇帝(1224年1月15日—1234年2月9日在位),女真名寧甲速,金朝亡國之君。哀宗在位10年,国破后自缢而死,终年37岁。

哀宗生於金承安三年八月二十三日(1198年9月25日),初名守禮,是金宣宗第三子,母明惠皇后王氏。宣宗登位後,封守禮為遂王。皇太子守忠及皇太孫鏗早逝,貞祐四年(1216年)正月,立守禮為皇太子,同年賜名守緒。元光二年十二月(1224年1月),宣宗去世,守緒繼位,是為哀宗。正大元年(1224年)六月立妃徒單氏為皇后。

哀宗本是一位比较有作为的皇帝,即位後,鼓励农业生产,停止侵宋战争,与西夏修好,进行内部改革,铲除奸佞,重用抗蒙名将,收复了不少土地,使金朝呈现出一片全新的景象。可是此時的蒙古勢不可擋,正大四年(1227年)滅西夏後即全力伐金。

在天興元年(1232年)的三峰山之战中,金军主力被蒙軍消滅,金国滅亡之勢已不可免。蒙軍進圍汴京,守軍奮力抵抗,當年汴京大疫,凡五十日,從各城門運出的死者有九十餘萬人,貧不能葬者尚未包括在內。哀宗在十二月逃離汴京,北渡黃河,後奔歸德(今河南商丘),最後來到蔡州(今河南汝南),然蒙古大將史天澤一路緊追不捨,在蒲城殲滅了完顏白撒的八萬精兵。天興二年(1233年)八月,蒙古召宋兵攻破唐州(今河南唐河),哀宗欲與宋連和,派使者向宋人說:「蒙古滅國四十,以及西夏,夏亡及我,我亡必及宋。唇亡齒寒,自然之理。」宋人不許。天興三年正月己酉(儒略曆1234年2月9日),蒙宋聯軍攻破蔡州,哀宗不願做亡國之君,便把皇位传给统帅完颜承麟,自己在蔡州幽蘭軒上吊自盡。末帝完顏承麟聞知哀宗死訊,“率群臣入哭,諡曰哀宗”,“哭奠未畢,城潰。”末帝同日死于亂軍中,金亡。

金哀宗之遗骸則被宋将孟珙与蒙将塔察儿所分屍。据蒙古伊兒汗国宰相拉施特主编的《史集》载,塔察儿仅获得金哀宗的一只手。当时金哀宗的尸首被贴身的近侍烧掉,并埋于汝水之上,所以拉施特的说法值得商榷。宋朝視金哀宗為金國亡國之君,把其大部分遗骸被宋軍帶回首都臨安告太廟。宋理宗最后按洪咨夔的建議處理了金哀宗遗骸,葬于大理寺狱库。

元朝官修正史《金史》脱脱等的評價是:“金之初兴,天下莫强焉。太祖、太宗威制中国,大概欲效辽初故事,立楚立齐,委而去之,宋人不竞,遂失故物。熙宗、海陵济以虐政,中原觖望,金事几去。天厌南北之兵,挺生世宗,以仁易暴,休息斯民。是故金祚百有余年,由大定之政有以固结人心,乃克尔也。章宗志存润色,而秕政日多,诛求无艺,民力浸竭,明昌、承安盛极衰始。至于卫绍,纪纲大坏,亡征已见。宣宗南度,弃厥本根,外狃余威,连兵宋、夏,内致困惫,自速土崩。哀宗之世无足为者。皇元功德日盛,天人属心,日出爝息,理势必然。区区生聚,图存于亡,力尽乃毙,可哀也矣。虽然,在《礼》“国君死社稷”,哀宗无愧焉。”

\subsection{正大}


\begin{longtable}{|>{\centering\scriptsize}m{2em}|>{\centering\scriptsize}m{1.3em}|>{\centering}m{8.8em}|}
  % \caption{秦王政}\
  \toprule
  \SimHei \normalsize 年数 & \SimHei \scriptsize 公元 & \SimHei 大事件 \tabularnewline
  % \midrule
  \endfirsthead
  \toprule
  \SimHei \normalsize 年数 & \SimHei \scriptsize 公元 & \SimHei 大事件 \tabularnewline
  \midrule
  \endhead
  \midrule
  元年 & 1224 & \tabularnewline\hline
  二年 & 1225 & \tabularnewline\hline
  三年 & 1226 & \tabularnewline\hline
  四年 & 1227 & \tabularnewline\hline
  五年 & 1228 & \tabularnewline\hline
  六年 & 1229 & \tabularnewline\hline
  七年 & 1230 & \tabularnewline\hline
  八年 & 1231 & \tabularnewline
  \bottomrule
\end{longtable}

\subsection{开兴}

\begin{longtable}{|>{\centering\scriptsize}m{2em}|>{\centering\scriptsize}m{1.3em}|>{\centering}m{8.8em}|}
  % \caption{秦王政}\
  \toprule
  \SimHei \normalsize 年数 & \SimHei \scriptsize 公元 & \SimHei 大事件 \tabularnewline
  % \midrule
  \endfirsthead
  \toprule
  \SimHei \normalsize 年数 & \SimHei \scriptsize 公元 & \SimHei 大事件 \tabularnewline
  \midrule
  \endhead
  \midrule
  元年 & 1232 & \tabularnewline
  \bottomrule
\end{longtable}

\subsection{天兴}

\begin{longtable}{|>{\centering\scriptsize}m{2em}|>{\centering\scriptsize}m{1.3em}|>{\centering}m{8.8em}|}
  % \caption{秦王政}\
  \toprule
  \SimHei \normalsize 年数 & \SimHei \scriptsize 公元 & \SimHei 大事件 \tabularnewline
  % \midrule
  \endfirsthead
  \toprule
  \SimHei \normalsize 年数 & \SimHei \scriptsize 公元 & \SimHei 大事件 \tabularnewline
  \midrule
  \endhead
  \midrule
  元年 & 1232 & \tabularnewline\hline
  二年 & 1233 & \tabularnewline\hline
  三年 & 1234 & \tabularnewline
  \bottomrule
\end{longtable}


%%% Local Variables:
%%% mode: latex
%%% TeX-engine: xetex
%%% TeX-master: "../Main"
%%% End:

\input{19_Jin/10_Modi}


%%% Local Variables:
%%% mode: latex
%%% TeX-engine: xetex
%%% TeX-master: "../Main"
%%% End:
