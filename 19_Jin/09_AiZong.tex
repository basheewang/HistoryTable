%% -*- coding: utf-8 -*-
%% Time-stamp: <Chen Wang: 2019-10-18 15:26:05>

\section{哀宗\tiny(1224-1234)}

金哀宗完颜守緒(1198年9月25日-1234年2月9日),金朝第九位皇帝(1224年1月15日—1234年2月9日在位),女真名寧甲速,金朝亡國之君。哀宗在位10年,国破后自缢而死,终年37岁。

哀宗生於金承安三年八月二十三日(1198年9月25日),初名守禮,是金宣宗第三子,母明惠皇后王氏。宣宗登位後,封守禮為遂王。皇太子守忠及皇太孫鏗早逝,貞祐四年(1216年)正月,立守禮為皇太子,同年賜名守緒。元光二年十二月(1224年1月),宣宗去世,守緒繼位,是為哀宗。正大元年(1224年)六月立妃徒單氏為皇后。

哀宗本是一位比较有作为的皇帝,即位後,鼓励农业生产,停止侵宋战争,与西夏修好,进行内部改革,铲除奸佞,重用抗蒙名将,收复了不少土地,使金朝呈现出一片全新的景象。可是此時的蒙古勢不可擋,正大四年(1227年)滅西夏後即全力伐金。

在天興元年(1232年)的三峰山之战中,金军主力被蒙軍消滅,金国滅亡之勢已不可免。蒙軍進圍汴京,守軍奮力抵抗,當年汴京大疫,凡五十日,從各城門運出的死者有九十餘萬人,貧不能葬者尚未包括在內。哀宗在十二月逃離汴京,北渡黃河,後奔歸德(今河南商丘),最後來到蔡州(今河南汝南),然蒙古大將史天澤一路緊追不捨,在蒲城殲滅了完顏白撒的八萬精兵。天興二年(1233年)八月,蒙古召宋兵攻破唐州(今河南唐河),哀宗欲與宋連和,派使者向宋人說:「蒙古滅國四十,以及西夏,夏亡及我,我亡必及宋。唇亡齒寒,自然之理。」宋人不許。天興三年正月己酉(儒略曆1234年2月9日),蒙宋聯軍攻破蔡州,哀宗不願做亡國之君,便把皇位传给统帅完颜承麟,自己在蔡州幽蘭軒上吊自盡。末帝完顏承麟聞知哀宗死訊,“率群臣入哭,諡曰哀宗”,“哭奠未畢,城潰。”末帝同日死于亂軍中,金亡。

金哀宗之遗骸則被宋将孟珙与蒙将塔察儿所分屍。据蒙古伊兒汗国宰相拉施特主编的《史集》载,塔察儿仅获得金哀宗的一只手。当时金哀宗的尸首被贴身的近侍烧掉,并埋于汝水之上,所以拉施特的说法值得商榷。宋朝視金哀宗為金國亡國之君,把其大部分遗骸被宋軍帶回首都臨安告太廟。宋理宗最后按洪咨夔的建議處理了金哀宗遗骸,葬于大理寺狱库。

元朝官修正史《金史》脱脱等的評價是:“金之初兴,天下莫强焉。太祖、太宗威制中国,大概欲效辽初故事,立楚立齐,委而去之,宋人不竞,遂失故物。熙宗、海陵济以虐政,中原觖望,金事几去。天厌南北之兵,挺生世宗,以仁易暴,休息斯民。是故金祚百有余年,由大定之政有以固结人心,乃克尔也。章宗志存润色,而秕政日多,诛求无艺,民力浸竭,明昌、承安盛极衰始。至于卫绍,纪纲大坏,亡征已见。宣宗南度,弃厥本根,外狃余威,连兵宋、夏,内致困惫,自速土崩。哀宗之世无足为者。皇元功德日盛,天人属心,日出爝息,理势必然。区区生聚,图存于亡,力尽乃毙,可哀也矣。虽然,在《礼》“国君死社稷”,哀宗无愧焉。”

\subsection{正大}


\begin{longtable}{|>{\centering\scriptsize}m{2em}|>{\centering\scriptsize}m{1.3em}|>{\centering}m{8.8em}|}
  % \caption{秦王政}\
  \toprule
  \SimHei \normalsize 年数 & \SimHei \scriptsize 公元 & \SimHei 大事件 \tabularnewline
  % \midrule
  \endfirsthead
  \toprule
  \SimHei \normalsize 年数 & \SimHei \scriptsize 公元 & \SimHei 大事件 \tabularnewline
  \midrule
  \endhead
  \midrule
  元年 & 1224 & \tabularnewline\hline
  二年 & 1225 & \tabularnewline\hline
  三年 & 1226 & \tabularnewline\hline
  四年 & 1227 & \tabularnewline\hline
  五年 & 1228 & \tabularnewline\hline
  六年 & 1229 & \tabularnewline\hline
  七年 & 1230 & \tabularnewline\hline
  八年 & 1231 & \tabularnewline
  \bottomrule
\end{longtable}

\subsection{开兴}

\begin{longtable}{|>{\centering\scriptsize}m{2em}|>{\centering\scriptsize}m{1.3em}|>{\centering}m{8.8em}|}
  % \caption{秦王政}\
  \toprule
  \SimHei \normalsize 年数 & \SimHei \scriptsize 公元 & \SimHei 大事件 \tabularnewline
  % \midrule
  \endfirsthead
  \toprule
  \SimHei \normalsize 年数 & \SimHei \scriptsize 公元 & \SimHei 大事件 \tabularnewline
  \midrule
  \endhead
  \midrule
  元年 & 1232 & \tabularnewline
  \bottomrule
\end{longtable}

\subsection{天兴}

\begin{longtable}{|>{\centering\scriptsize}m{2em}|>{\centering\scriptsize}m{1.3em}|>{\centering}m{8.8em}|}
  % \caption{秦王政}\
  \toprule
  \SimHei \normalsize 年数 & \SimHei \scriptsize 公元 & \SimHei 大事件 \tabularnewline
  % \midrule
  \endfirsthead
  \toprule
  \SimHei \normalsize 年数 & \SimHei \scriptsize 公元 & \SimHei 大事件 \tabularnewline
  \midrule
  \endhead
  \midrule
  元年 & 1232 & \tabularnewline\hline
  二年 & 1233 & \tabularnewline\hline
  三年 & 1234 & \tabularnewline
  \bottomrule
\end{longtable}


%%% Local Variables:
%%% mode: latex
%%% TeX-engine: xetex
%%% TeX-master: "../Main"
%%% End:
