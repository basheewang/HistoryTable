%% -*- coding: utf-8 -*-
%% Time-stamp: <Chen Wang: 2021-11-01 16:54:38>

\section{太祖完顏阿骨打\tiny(1115-1123)}

\subsection{生平}

金太祖完顏阿骨打(1068年8月1日-1123年9月19日),漢名完顏旻,金朝開國皇帝(1115年1月28日—1123年9月19日在位)。按出虎水(今黑龍江省哈爾濱東南阿什河)女真族完顏部酋長烏骨迺之孫,劾里鉢之次子,完顏部首領。善騎射,力大過人。在位9年,終年56歲。

祖父是生女真完顔部的族長烏古廼(景祖)、父劾里鉢是烏古廼的次子。阿骨打是劾里鉢的次子)。生母是女真挐懶部首長的女兒翼簡皇后。

遼國天慶三年(1113年)十月,其兄烏雅束死,繼位女真各部落聯盟長,稱都勃極烈。天慶四年,率2500人起兵叛遼,破寧江州(今吉林省扶餘市東南)。蕭嗣先率7000精兵集結於出河店,阿骨打率兵3700乘夜奔襲,渡混同江(今松花江),大敗遼軍。天慶五年農歷正月初一(1115年1月28日),阿骨打在會寧(今黑龍江省哈爾濱市阿城區南白城)稱帝,建立大金,年號收國,改名完顏旻。天慶五年九月,攻佔黃龍府(今吉林省農安縣)城。

天輔三年(1119年),遼天祚帝冊封完顏旻為東懷國皇帝,但冊文不稱完顏旻為兄長、國號不稱大金,故他不接受冊封,繼續攻打遼國。

天輔四年(1120年),與宋朝訂攻遼計劃,攻陷遼上京臨潢府(今內蒙古自治區巴林左旗南)。天輔六年(1122年),取遼中京(今內蒙古自治區寧城縣西);是年年底,攻陷燕京(今北京市)。天輔七年(1123年)八月,返金上京(今黑龍江省哈爾濱市阿城區附近)途中病逝。他死後,在天會三年六月上諡號大聖皇帝,同年十二月改為大聖武元皇帝,廟號是太祖。皇統五年十月,增諡為應乾興運昭德定功仁明莊孝大聖武元皇帝。

2003年9月5日,北京市政府文物局發表:1980年代在北京市西南郊外的九龍山的金朝陵墓,證實是完顏阿骨打的石棺、遺骨及裝飾物。

阿骨打痛恨遼,但對宋相當和善,在建國之初就有意與宋聯合,和後來諸代金朝帝王對宋朝充滿敵對大不相同。《靖康稗史箋證》中記錄其二子完顏宗望曾說過:「太祖止我伐宋,言猶在耳」。 當宋以「海上之盟」求燕京(今北京西南)及西京(今山西大同)地,金國大臣左企弓(張覺叛金時被殺)曾勸阿骨打不要歸還「燕雲十六州」,但阿骨打還是如約歸還了「燕雲十六州」中的燕京、涿州、易州、檀州、順州、景州、薊州。其中景州雖在長城之內,但並不屬於石敬瑭割給遼的燕雲十六州之一。易州是遼統和七年(989年)夺自宋,也不算作十六州之一。莫、瀛兩州早已收復,為北宋河間府所治。這樣一來,山西、河北太行山(後明在此建內長城)以內的燕、涿、檀、順、薊、莫、瀛七州都已經歸還宋,而太行山以外的儒、媯、武、新、蔚、應、寰、朔、雲九州當時遼金尚在爭奪,金太祖也無法歸還。

和阿骨打生前相處時間較長的幾個年長兒子,如長子完顏宗幹、二子完顏宗望、四子完顏宗弼都很崇尚漢文化,這對以後金國的漢化影響很大。這也從另一個側面反映了阿骨打的喜好。

元朝官修正史《金史》脱脱等的評價是:“太祖英谟睿略,豁达大度,知人善任,人乐为用。世祖阴有取辽之志,是以兄弟相授,传及康宗,遂及太祖。临终以太祖属穆宗,其素志盖如是也。初定东京,即除去辽法,减省租税,用本国制度。辽主播越,宋纳岁币,以幽、蓟、武、朔等州与宋,而置南京于平州。宋人终不能守燕、代,卒之辽主见获,宋主被执。虽功成于天会间,而规摹运为宾自此始。金有天下百十有九年,太祖数年之间算无遗策,兵无留行,底定大业,传之子孙。嗚呼,雄哉!”

\subsection{收国}


\begin{longtable}{|>{\centering\scriptsize}m{2em}|>{\centering\scriptsize}m{1.3em}|>{\centering}m{8.8em}|}
  % \caption{秦王政}\
  \toprule
  \SimHei \normalsize 年数 & \SimHei \scriptsize 公元 & \SimHei 大事件 \tabularnewline
  % \midrule
  \endfirsthead
  \toprule
  \SimHei \normalsize 年数 & \SimHei \scriptsize 公元 & \SimHei 大事件 \tabularnewline
  \midrule
  \endhead
  \midrule
  元年 & 1115 & \tabularnewline\hline
  二年 & 1116 & \tabularnewline
  \bottomrule
\end{longtable}

\subsection{天辅}

\begin{longtable}{|>{\centering\scriptsize}m{2em}|>{\centering\scriptsize}m{1.3em}|>{\centering}m{8.8em}|}
  % \caption{秦王政}\
  \toprule
  \SimHei \normalsize 年数 & \SimHei \scriptsize 公元 & \SimHei 大事件 \tabularnewline
  % \midrule
  \endfirsthead
  \toprule
  \SimHei \normalsize 年数 & \SimHei \scriptsize 公元 & \SimHei 大事件 \tabularnewline
  \midrule
  \endhead
  \midrule
  元年 & 1117 & \tabularnewline\hline
  二年 & 1118 & \tabularnewline\hline
  三年 & 1119 & \tabularnewline\hline
  四年 & 1120 & \tabularnewline\hline
  五年 & 1121 & \tabularnewline\hline
  六年 & 1122 & \tabularnewline\hline
  七年 & 1123 & \tabularnewline
  \bottomrule
\end{longtable}


%%% Local Variables:
%%% mode: latex
%%% TeX-engine: xetex
%%% TeX-master: "../Main"
%%% End:
