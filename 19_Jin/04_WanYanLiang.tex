%% -*- coding: utf-8 -*-
%% Time-stamp: <Chen Wang: 2019-10-18 15:21:40>

\section{完颜亮\tiny(1150-1161)}

完顏亮(1122年2月24日-1161年12月15日),字元功,女真名迪古乃,金朝第四代皇帝(1150年1月9日-1161年12月15日),金太祖阿骨打之孙,太祖庶长子遼王完顏宗幹第二子,母大氏。

完顏亮弒金熙宗而篡位,任內遷都燕京(今北京),把金朝的政治中心遷至華北,逐步汉化,使北京自此逐漸成為中國的政治中心。因伐南宋的采石大战失利,被部下所殺。完颜亮在位12年,终年40岁。完颜亮死後,继位的金世宗将他追貶為庶人,史称海陵煬王、海陵庶人、金废帝。

完颜亮生于1122年2月24日(天辅六年正月十六丙子日天眷三年)。(1140年)十八歲時以宗室子為奉國上將軍,赴梁王完顏宗弼(兀朮)幕府任使,管理萬人,遷驃騎上將軍。皇統四年(1144年),加龍虎衛上將軍,為金國中京(位於今北京市一帶)留守,遷光祿大夫。

皇統七年(1147年)五月,召入當時的金國首都上京(今黑龍江省哈尔滨市阿城区內)為同判大宗正事,加特進。十一月,拜尚書省左丞,把持了權柄,安插自己的心腹擔任要職,其中蕭裕 成為兵部侍郎。十一月某日和熙宗談話時,談到金太祖創業艱難,完顏亮痛哭流涕,熙宗認為他很忠心。後來升職加快。第二年(1148年)六月,拜平章事。十一月,拜右丞相。1149年正月,兼都元帥。三月,拜太保、領三省事,更加八面玲瓏,和有權勢家族來往密切,結其歡心。

1149年熙宗對完顏亮突然膨脹的勢力不滿。正月,熙宗派寢殿小底大興國以宋名臣司馬光畫像及其它珍玩賜完顏亮生日禮物,悼平皇后裴滿氏也附賜禮物,結果引起熙宗不悅,罰小底大興國一百杖,追回其賜物,完顏亮知道後由此不安。四月,學士張鈞起草詔書時擅自改動,被查出處死。熙宗問是誰指使的,左丞相完顏宗賢回答說是太保完顏亮。熙宗不悅,遂貶完顏亮到汴京(今河南開封),領行台尚書省事。完顏亮路過中京時,和那裡的兵部侍郎蕭裕密謀定約而去。走到良鄉,又被熙宗召還。完顏亮不知熙宗的意圖,非常恐懼。回到上京,又恢復為平章政事。但完顏亮反意已決。

《金史》說完顏亮“為人僄急,多猜忌,殘忍任數。”當熙宗以太祖的嫡孫身份嗣位時,完顏亮認為自己是太祖長子完顏宗幹的兒子,也是太祖的孫子,所以對皇位“遂懷覬覦。”早在皇統七年(1147年),熙宗就開始胡亂發脾氣殺人,比如賜宴時因為一些小事濫殺無辜,引起朝臣的不滿。皇統八年(1148年)七月,以駙馬尚書左丞唐括辯奉職不謹,杖之。皇統九年(1149年)八月,杖平章政事完顏秉德。對熙宗不滿的人即有廢立的想法,唐括辯、秉德先和大理卿烏帶(完顏言)謀劃廢掉熙宗,而烏帶就此引入完顏亮。完顏亮與唐括辯密謀廢立,問到若廢熙宗,可以立誰繼位?唐括辯與秉德初意並不在完顏亮。唐括辯說胙王完顏常勝(完顏元)似乎可以。完顏亮再問其次是誰,唐括辯說鄧王完顏奭之子完顏阿楞可以。完顏亮反駁說阿楞不行。唐括辯反問:“公豈有意邪?”完顏亮說:“果不得已,舍我其誰!”不久完顏亮和唐括辯等旦夕密謀,引起了護衛將軍完顏特思的懷疑。特思告訴了悼平皇后裴滿氏,因此熙宗得知。熙宗發怒召唐括辯並杖之。完顏亮因此非常忌諱完顏元、完顏阿楞,並且極其討厭完顏特思。

正好當時河南有士兵孫進冒稱皇弟按察大王,而熙宗之弟只有完顏元和完顏查剌。熙宗懷疑是完顏元,派完顏特思調查,卻甚麼也沒有。完顏亮乘機誣陷,對熙宗說:“孫進反有端,不稱他人,乃稱皇弟大王。陛下弟止有常勝、查刺。特思鞫不以實,故出之矣。”熙宗以為然,派唐括辯、蕭肄拷問完顏特思,完顏特思被逼招認,完顏元於是獲罪。十月,殺完顏元,一併連完顏查刺、完顏特思、完顏阿楞以及阿楞弟完顏撻楞 一起殺掉。這樣一來,熙宗殺光了自己的親兄弟,更加孤立。

到了皇統九年(1149年)十二月,要廢熙宗的人已經結黨行事。從前因送禮一事被杖責一百的大興國,因為和完顏亮的心腹尚書省令史李老僧是親戚,於是和完顏亮結黨,當時正在伺候熙宗在寢殿內的起居生活,總是有意無意地乘夜從主事者那裡帶皇宮鑰匙回家,大家習以為常。護衛十人長僕散忽土要報答完顏亮之父完顏宗幹的舊恩,徒單阿里出虎是完顏亮的姻親。十二月初九丁巳日(儒略曆1150年1月9日),此二人值班之夜,大興國用皇宮鑰匙打開所有宮門,和完顏亮、秉德、唐括辯、烏帶、徒單貞、李老僧 至寢殿。熙宗本來常置佩刀於床上,這天夜裡大興國先取之放到床下,等到事發,熙宗求佩刀不得,遂遇弒。眾人拜完顏亮為皇帝。改皇統九年為天德元年。並假稱熙宗想要商議立皇后事宜,召眾大臣入宮,殺曹國王完顏宗敏、左丞相完顏宗賢。

贞元元年三月二十六日(1153年4月21日),完颜亮正式迁都,改燕京为中都,定名为中都大兴府,同時定北宋故都開封府為金南京,使金朝逐步汉化。

海陵王在位期間不但擴大皇帝權威,甚至於濫用權力,誅殺大臣;而且海陵王的宮廷生活相當荒淫,史載「營南京(燕京)宮殿,運一木之費至二千萬,率一車之力至五百人。宮殿之飾,遍傅黃金而後間以五彩,金屑飛空如落雪。一殿之費以億萬計,成而復毀,務極華麗。」(《金史》)。據說他讀罷柳永的《望海潮》一詞:「東南形勝,三吳都會,錢塘自古繁華……有三秋桂子,十里荷花」,「遂起投鞭渡江、立馬吳山之志」,即興題詩稱:“万里车书一混同,江南岂有别疆封? 提兵百万西湖侧,立马吴山第一峰。”(《鶴林玉露》卷一) 完顏亮曾大顏不慚地說:「吾有三志,國家大事,皆我所出,一也;帥師伐遠,執其君長問罪於前,二也;得天下絕色而妻之,三也。」而被他收入深宮而「妻之」的「天下絕色」,竟有他的堂姐妹、叔母 、舅母、外甥女、侄女以及弟媳、小姨子等等。完顏亮上台後,為了壓制皇族宗室的反抗,曾大加誅戮,諸叔及其子弟幾乎屠殺殆盡,他們的妻子、女兒,或被納為嬪妃,或被強納宮中,「命諸從姊妹皆分屬諸妃,出入禁中,與為淫亂。」昭妃阿懶,就是完顏亮的親嬸嬸,完顏亮殺死叔叔曹國王宗敏,便把阿懶納入宮中,封為昭妃。

紹興三十一年(正隆六年,1161年)出兵伐宋,進迫長江。但是東京留守曹國公完顏雍杀副留守高存福,自立為帝,是为金世宗。采石大战中了南宋江淮參軍虞允文的埋伏,退兵瓜洲渡,命令所有士兵即刻南征,軍心大亂,为部下完颜元宜所弑,享年40歲。死後追貶為「海陵王」,又追貶為「海陵庶人」,被以庶人之禮安葬。

金世宗大定二年(1162年)四月,降封為海陵郡王,諡号为煬,所以又称海陵煬王,葬於大房山鹿門谷諸王的墓地中。大定二十一年(1181年)正月,由于为海陵王所弒的金熙宗于大定十九年供入太廟,完顏亮又再被降為海陵庶人,改葬于山陵西南四十里。今北京市房山区有海陵王陵。

南宋洪邁出使金世宗後歸國,向宋高宗報告完顏亮被諡為「煬」的事。宋高宗表示,當時人們都把完顏亮比作苻堅,唯獨他認為完顏亮與隋煬帝類似。宋高宗因此認為,隋煬帝和完顏亮死在同一地方,又被加上同一諡號,乃是天意。元朝官修正史《金史》脱脱等的評價是:“海陵智足以拒諫,言足以飾非。欲爲君則弑其君,欲伐國則弑其母,欲奪人之妻則使之殺其夫。三綱絕矣,何暇他論。至于屠滅宗族,剪刈忠良,婦姑姊妹盡入嬪御。方以三十二總管之兵圖一天下,卒之戾氣感召,身由惡終,使天下後世稱無道主以海陵爲首。可不戒哉!可不戒哉!”。其中,「智足以拒諫,言足以飾非」一句,是司馬遷在《史記》中對商紂王的評語原文。《醒世恆言》中有《金海陵縱慾亡身》一篇(改編自更早的話本),將海陵王描寫為淫蕩的昏君,使得海陵王的負面形象深入人心。

\subsection{天德}


\begin{longtable}{|>{\centering\scriptsize}m{2em}|>{\centering\scriptsize}m{1.3em}|>{\centering}m{8.8em}|}
  % \caption{秦王政}\
  \toprule
  \SimHei \normalsize 年数 & \SimHei \scriptsize 公元 & \SimHei 大事件 \tabularnewline
  % \midrule
  \endfirsthead
  \toprule
  \SimHei \normalsize 年数 & \SimHei \scriptsize 公元 & \SimHei 大事件 \tabularnewline
  \midrule
  \endhead
  \midrule
  元年 & 1149 & \tabularnewline\hline
  二年 & 1150 & \tabularnewline\hline
  三年 & 1151 & \tabularnewline\hline
  四年 & 1152 & \tabularnewline\hline
  五年 & 1153 & \tabularnewline
  \bottomrule
\end{longtable}

\subsection{贞元}

\begin{longtable}{|>{\centering\scriptsize}m{2em}|>{\centering\scriptsize}m{1.3em}|>{\centering}m{8.8em}|}
  % \caption{秦王政}\
  \toprule
  \SimHei \normalsize 年数 & \SimHei \scriptsize 公元 & \SimHei 大事件 \tabularnewline
  % \midrule
  \endfirsthead
  \toprule
  \SimHei \normalsize 年数 & \SimHei \scriptsize 公元 & \SimHei 大事件 \tabularnewline
  \midrule
  \endhead
  \midrule
  元年 & 1153 & \tabularnewline\hline
  二年 & 1154 & \tabularnewline\hline
  三年 & 1155 & \tabularnewline\hline
  四年 & 1156 & \tabularnewline
  \bottomrule
\end{longtable}

\subsection{正隆}

\begin{longtable}{|>{\centering\scriptsize}m{2em}|>{\centering\scriptsize}m{1.3em}|>{\centering}m{8.8em}|}
  % \caption{秦王政}\
  \toprule
  \SimHei \normalsize 年数 & \SimHei \scriptsize 公元 & \SimHei 大事件 \tabularnewline
  % \midrule
  \endfirsthead
  \toprule
  \SimHei \normalsize 年数 & \SimHei \scriptsize 公元 & \SimHei 大事件 \tabularnewline
  \midrule
  \endhead
  \midrule
  元年 & 1156 & \tabularnewline\hline
  二年 & 1157 & \tabularnewline\hline
  三年 & 1158 & \tabularnewline\hline
  四年 & 1159 & \tabularnewline\hline
  五年 & 1160 & \tabularnewline\hline
  六年 & 1161 & \tabularnewline
  \bottomrule
\end{longtable}


%%% Local Variables:
%%% mode: latex
%%% TeX-engine: xetex
%%% TeX-master: "../Main"
%%% End:
