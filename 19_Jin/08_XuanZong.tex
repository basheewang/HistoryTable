%% -*- coding: utf-8 -*-
%% Time-stamp: <Chen Wang: 2021-11-01 16:57:23>

\section{宣宗完颜珣\tiny(1213-1224)}

\subsection{生平}

金宣宗完颜珣(1163年4月18日-1224年1月14日),女真名吾睹補。金世宗完颜雍长孙,卫绍王侄,父完顏允恭,母昭華劉氏。他是金朝第八位皇帝(1213年9月22日—1224年1月14日在位),在位11年,终年61岁。

金世宗大定十八年(1178年),封溫國公,二十六年賜名珣,二十九年封豐王。承安元年,封翼王。泰和五年,改賜名從嘉,其後又改封邢王及升王。

至寧元年(1213年)八月,胡沙虎杀死卫绍王,迎立從嘉为帝,由於從嘉在河北鎮守,於是暫時以從嘉長子完顏守忠監國。九月即位,是為宣宗,以胡沙虎為太師、尚書令兼都元帥,封澤王,同月改元貞祐。閏九月,宣宗復舊名珣。十月,朮虎高琪殺胡沙虎,宣宗赦免高琪,封他為左副元帥。是年秋,蒙古軍分三路攻金,幾乎攻破所有河北郡縣,金國只有中都、真定、大名等十一城未曾失守。

貞祐二年三月,蒙金和議成,五月十八日(1214年6月27日),金宣宗逃離中都,七月金宣宗南逃到達汴京,此舉觸怒蒙古,戰爭再起。貞祐三年五月初二(1215年5月31日),中都失守,十月,蒲鮮萬奴在遼東自立。興定三年十二月(1220年初),宣宗誅高琪。

宣宗对外措施十分不当,直接导致金朝灭亡。他先向蒙古大汗成吉思汗屈辱求和,又与西夏断交,還不顧丞相徒单镒和諸多大臣等的反對,将都城由中都南迁至汴京,并且发动侵宋战争。金国三面受敌,加上內部不和,叛亂頻生,國家危在旦夕。

宣宗在元光二年十二月二十二日(1224年1月14日)去世,死後諡號是繼天興統述道勤仁英武聖孝皇帝,廟號是宣宗,葬于德陵(在今河南開封)。

元朝官修正史《金史》脱脱等的評價是:“宣宗当金源末运,虽乏拨乱反正之材,而有励精图治之志。迹其勤政忧民,中兴之业盖可期也,然而卒无成功者何哉?良由性本猜忌,崇信翙御,奖用吏胥,苛刻成风,举措失当故也。执中元恶,此岂可相者乎,顾乃怀其援立之私,自除廉陛之分,悖礼甚矣。高琪之诛执中,虽云除恶,律以《春秋》之法,岂逃赵鞅晋阳之责?既不能罪而遂相之,失之又失者也。迁汴之后,北顾大元之朝日益隆盛,智识之士孰不先知?方且狃于余威,牵制群议,南开宋衅,西启夏侮,兵力既分,功不补患。曾未数年,昔也日辟国百里,今也日蹙国里,其能济乎?再迁遂至失国,岂不重可叹哉!”

\subsection{贞祐}


\begin{longtable}{|>{\centering\scriptsize}m{2em}|>{\centering\scriptsize}m{1.3em}|>{\centering}m{8.8em}|}
  % \caption{秦王政}\
  \toprule
  \SimHei \normalsize 年数 & \SimHei \scriptsize 公元 & \SimHei 大事件 \tabularnewline
  % \midrule
  \endfirsthead
  \toprule
  \SimHei \normalsize 年数 & \SimHei \scriptsize 公元 & \SimHei 大事件 \tabularnewline
  \midrule
  \endhead
  \midrule
  元年 & 1213 & \tabularnewline\hline
  二年 & 1214 & \tabularnewline\hline
  三年 & 1215 & \tabularnewline\hline
  四年 & 1216 & \tabularnewline\hline
  五年 & 1217 & \tabularnewline
  \bottomrule
\end{longtable}

\subsection{兴定}

\begin{longtable}{|>{\centering\scriptsize}m{2em}|>{\centering\scriptsize}m{1.3em}|>{\centering}m{8.8em}|}
  % \caption{秦王政}\
  \toprule
  \SimHei \normalsize 年数 & \SimHei \scriptsize 公元 & \SimHei 大事件 \tabularnewline
  % \midrule
  \endfirsthead
  \toprule
  \SimHei \normalsize 年数 & \SimHei \scriptsize 公元 & \SimHei 大事件 \tabularnewline
  \midrule
  \endhead
  \midrule
  元年 & 1217 & \tabularnewline\hline
  二年 & 1218 & \tabularnewline\hline
  三年 & 1219 & \tabularnewline\hline
  四年 & 1220 & \tabularnewline\hline
  五年 & 1221 & \tabularnewline\hline
  六年 & 1222 & \tabularnewline
  \bottomrule
\end{longtable}

\subsection{元光}

\begin{longtable}{|>{\centering\scriptsize}m{2em}|>{\centering\scriptsize}m{1.3em}|>{\centering}m{8.8em}|}
  % \caption{秦王政}\
  \toprule
  \SimHei \normalsize 年数 & \SimHei \scriptsize 公元 & \SimHei 大事件 \tabularnewline
  % \midrule
  \endfirsthead
  \toprule
  \SimHei \normalsize 年数 & \SimHei \scriptsize 公元 & \SimHei 大事件 \tabularnewline
  \midrule
  \endhead
  \midrule
  元年 & 1222 & \tabularnewline\hline
  二年 & 1223 & \tabularnewline
  \bottomrule
\end{longtable}


%%% Local Variables:
%%% mode: latex
%%% TeX-engine: xetex
%%% TeX-master: "../Main"
%%% End:
