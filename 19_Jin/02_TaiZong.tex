%% -*- coding: utf-8 -*-
%% Time-stamp: <Chen Wang: 2019-10-18 15:19:07>

\section{太宗\tiny(1123-1135)}

金太宗完顏晟(1075年11月25日-1135年2月9日),金朝第二位皇帝(1123年9月27日—1135年2月9日在位)。女真名吳乞買,金太祖之弟,身材魁梧,力大無比,能親手搏熊刺虎。在位12年,终年61岁。先后滅遼朝及北宋。

完颜吴乞买出生于1075年11月25日。天會三年二月二十日(1125年3月26日),辽天祚帝在应州被金朝将领完颜娄室等所俘,八月被解送金上京,被降为海滨王,辽朝灭亡。

天會三年(1125年)十月,发动宋金战争,令諳班勃極烈完颜斜也為都元帥,統領金軍,兵分東、西兩路,逼進北宋首都汴京,由於李綱頑強抵抗,金兵一時不能得逞,雙方訂「城下之盟」。天會四年(1126年)八月,經過半年的休整,金太宗再次命宗望、宗翰兩路軍大舉南侵,汴京再度被包圍,破郭京「六甲法」,汴京城陷。天會五年二月初六(1127年3月20日),金太宗下詔廢徽、欽二帝,貶為庶人,俘虏二帝北上,并携带掠夺来的大量财宝和皇室大臣宫女等15000人,北宋滅亡。天會六年(1128年)八月二十四日,吳乞買封宋徽宗為昏德公,宋欽宗為重昏侯,移遷五國城(今黑龍江省依蘭縣城北舊古城)。

他在位时期创建了各种典章制度,奠定金代经国规模,晚年改变兄终弟及的旧制,立太祖孙完颜亶(金熙宗)为继承人。

天會十三年正月二十五日(1135年2月9日),太宗病死於明德宮,終年六十一歲。遺體葬和陵。其后代全被海陵王完颜亮所杀,海陵王遷都後,改葬於大房山,稱金恭陵。

他死後,於天會十三年三月七日上諡號文烈皇帝,廟號太宗。皇統五年閏十一月增諡体元应运世德昭功哲惠仁圣文烈皇帝。

吴乞买與宋太祖的畫像神似,民間相傳宋太宗當年殺太祖奪位,甚至還說吳乞買是宋太祖投胎來報仇,滅了宋太宗一家,宋高宗為了統治的正統性,寧可把帝位傳回宋太祖一脈,於是以太祖後代趙眘為養子,禪以帝位。

元朝官修正史《金史》脱脱等的評價是:“天辅草创,未遑礼乐之事。太宗以斜也、宗干知国政,以宗翰、宗望总戎事。既灭辽举宋,即议礼制度,治历明时,缵以武功,述以文事,经国规摹,至是始定。在位十三年,宫室苑籞无所增益。末听大臣计,传位熙宗,使太祖世嗣不失正绪,可谓行其所甚难矣!”

\subsection{天会}


\begin{longtable}{|>{\centering\scriptsize}m{2em}|>{\centering\scriptsize}m{1.3em}|>{\centering}m{8.8em}|}
  % \caption{秦王政}\
  \toprule
  \SimHei \normalsize 年数 & \SimHei \scriptsize 公元 & \SimHei 大事件 \tabularnewline
  % \midrule
  \endfirsthead
  \toprule
  \SimHei \normalsize 年数 & \SimHei \scriptsize 公元 & \SimHei 大事件 \tabularnewline
  \midrule
  \endhead
  \midrule
  元年 & 1123 & \tabularnewline\hline
  二年 & 1124 & \tabularnewline\hline
  三年 & 1125 & \tabularnewline\hline
  四年 & 1126 & \tabularnewline\hline
  五年 & 1127 & \tabularnewline\hline
  六年 & 1128 & \tabularnewline\hline
  七年 & 1129 & \tabularnewline\hline
  八年 & 1130 & \tabularnewline\hline
  九年 & 1131 & \tabularnewline\hline
  十年 & 1132 & \tabularnewline\hline
  十一年 & 1133 & \tabularnewline\hline
  十二年 & 1134 & \tabularnewline\hline
  十三年 & 1135 & \tabularnewline\hline
  十四年 & 1136 & \tabularnewline\hline
  十五年 & 1137 & \tabularnewline
  \bottomrule
\end{longtable}


%%% Local Variables:
%%% mode: latex
%%% TeX-engine: xetex
%%% TeX-master: "../Main"
%%% End:
