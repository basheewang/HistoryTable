%% -*- coding: utf-8 -*-
%% Time-stamp: <Chen Wang: 2021-11-01 16:56:52>

\section{章宗完顏璟\tiny(1189-1208)}

\subsection{生平}

金章宗完顏璟(1168年8月31日(农历七月二十七)-1208年12月29日),女真名麻達葛,金朝第6位皇帝(1189年1月20日—1208年12月29日在位),在位19年,享年41岁。章宗為金世宗完颜雍之嫡孙,其在位期間修訂國內律法,政治清明,世稱明昌之治。章宗統治下的金朝文化發展達至頂峰,但同時軍事能力卻也日益低下,蒙古帝國也於同時崛起。

南宋主戰派權臣韓侂胄於章宗年間北伐,但遭到金軍擊敗,簽定「嘉定和議」。1208年駕崩,叔衛紹王完顏永济繼位。

金世宗在大定初年立章宗之父完顏允恭為太子,允恭在大定二十五年(1185年)逝世後,世宗在次年立章宗為皇太孫。大定二十九年正月初二,世宗去世,章宗隨即繼位。

當時金朝立國七十五年,「禮樂刑政因遼、宋舊制,雜亂無貫,章宗即位,乃更定修正,為一代法。」章宗時期的政治尚算清明,後世稱為明昌之治。

章宗時代,国内的文化發展達至最高峰。他不單對國內文化發展加以獎勵,而他本身亦能寫得一手好字,與北宋徽宗的「瘦金體」形似。但與此同時,軍事能力卻日益低下,使屬國紛紛離異、並招引鄰國侵略。章宗整日与文人饮酒作诗,不思朝政。金朝日益腐朽衰败,漠北已失去控制。此外,黄河氾濫等各種天災相繼出現,使国力開始衰退。在位后期蒙古帝国崛起,成为了日后金覆灭的隐患。

1196年,原來從屬金朝的塔塔兒部叛離,改為歸順蒙古。南宋權臣韓侂冑見金朝開始走下坡,以為有機可乘,在1206年大舉出兵攻金,結果宋軍大敗,東線金兵渡過淮河,佔領淮南多個州縣;中線金兵攻襄陽;西線宋將吳曦以四川附金,不久事敗被殺。宋寧宗殺韓侂冑向金求和,1208年「嘉定和議」成,宋尊金為伯,增加每年歲幣至銀三十萬兩、絹三十萬匹及向金朝納「犒軍錢」三百萬兩,金朝始歸還南宋失地,維持紹興和議時的局面。

1208年12月29日,金章宗駕崩。他的六個兒子都在三歲前夭折。由於他沒有後嗣,所以由叔父衛紹王完顏永济繼位。金章宗駕崩、完顏永濟繼位後,成吉思汗知道完顏永濟是個無能之輩,所以在次年立即揮軍南下開始侵略金朝。

他死後諡號是憲天光運仁文義武神聖英孝皇帝,廟號是章宗。葬于道陵。

元朝官修正史《金史》脱脱等的評價是:“章宗在位二十年,承世宗治平日久,宇内小康,乃正礼乐,修刑法,定官制,典章文物粲然成一代治规。又数问群臣汉宣综核名实、唐代考课之法,盖欲跨辽、宋而比迹于汉、唐,亦可谓有志于治者矣!然婢宠擅朝,冢嗣未立,疏忌宗室而传授非人。向之所谓维持巩固于久远者,徒为文具,而不得为后世子孙一日之用,金源氏从此衰矣!昔扬雄氏有云:‘秦之有司负秦之法度,秦之法度负圣人之法度。’盖有以夫。”

\subsection{明昌}


\begin{longtable}{|>{\centering\scriptsize}m{2em}|>{\centering\scriptsize}m{1.3em}|>{\centering}m{8.8em}|}
  % \caption{秦王政}\
  \toprule
  \SimHei \normalsize 年数 & \SimHei \scriptsize 公元 & \SimHei 大事件 \tabularnewline
  % \midrule
  \endfirsthead
  \toprule
  \SimHei \normalsize 年数 & \SimHei \scriptsize 公元 & \SimHei 大事件 \tabularnewline
  \midrule
  \endhead
  \midrule
  元年 & 1190 & \tabularnewline\hline
  二年 & 1191 & \tabularnewline\hline
  三年 & 1192 & \tabularnewline\hline
  四年 & 1193 & \tabularnewline\hline
  五年 & 1194 & \tabularnewline\hline
  六年 & 1195 & \tabularnewline\hline
  七年 & 1196 & \tabularnewline
  \bottomrule
\end{longtable}

\subsection{承安}

\begin{longtable}{|>{\centering\scriptsize}m{2em}|>{\centering\scriptsize}m{1.3em}|>{\centering}m{8.8em}|}
  % \caption{秦王政}\
  \toprule
  \SimHei \normalsize 年数 & \SimHei \scriptsize 公元 & \SimHei 大事件 \tabularnewline
  % \midrule
  \endfirsthead
  \toprule
  \SimHei \normalsize 年数 & \SimHei \scriptsize 公元 & \SimHei 大事件 \tabularnewline
  \midrule
  \endhead
  \midrule
  元年 & 1196 & \tabularnewline\hline
  二年 & 1197 & \tabularnewline\hline
  三年 & 1198 & \tabularnewline\hline
  四年 & 1199 & \tabularnewline\hline
  五年 & 1200 & \tabularnewline
  \bottomrule
\end{longtable}

\subsection{泰和}

\begin{longtable}{|>{\centering\scriptsize}m{2em}|>{\centering\scriptsize}m{1.3em}|>{\centering}m{8.8em}|}
  % \caption{秦王政}\
  \toprule
  \SimHei \normalsize 年数 & \SimHei \scriptsize 公元 & \SimHei 大事件 \tabularnewline
  % \midrule
  \endfirsthead
  \toprule
  \SimHei \normalsize 年数 & \SimHei \scriptsize 公元 & \SimHei 大事件 \tabularnewline
  \midrule
  \endhead
  \midrule
  元年 & 1201 & \tabularnewline\hline
  二年 & 1202 & \tabularnewline\hline
  三年 & 1203 & \tabularnewline\hline
  四年 & 1204 & \tabularnewline\hline
  五年 & 1205 & \tabularnewline\hline
  六年 & 1206 & \tabularnewline\hline
  七年 & 1207 & \tabularnewline\hline
  八年 & 1208 & \tabularnewline
  \bottomrule
\end{longtable}


%%% Local Variables:
%%% mode: latex
%%% TeX-engine: xetex
%%% TeX-master: "../Main"
%%% End:
