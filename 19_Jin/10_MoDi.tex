%% -*- coding: utf-8 -*-
%% Time-stamp: <Chen Wang: 2021-11-01 17:03:38>

\section{末帝完顏承麟\tiny(1234)}

\subsection{生平}

完顏承麟(1202年-1234年2月9日),通稱金末帝,又稱金後主,金朝末代皇帝,女真名呼敦。原為金國將領,天興三年正月己酉(儒略曆1234年2月9日),金哀宗不欲做亡國之君,遂將帝位傳予他。於下旨傳位翌日舉行即位大典,但大典未及完成蒙宋聯軍已攻入城內。完顏承麟唯有草草完成大典立刻率軍迎敵,死於亂軍之中。據史學家推測,完顏承麟在位時間不足半天,甚至可能不足三個時辰,为東亞历史上在位最短的皇帝。

完颜承麟为金世祖劾里鉢的后裔,初為金國將領,驍勇善戰、才略兼備,深受金哀宗器重。兄完颜承裔,又作完颜白撒,曾任宰相。后因完颜白撒获罪,完颜承麟与完颜白撒子狗儿都被发徐州安置。

天興元年(1232年),蒙古軍隊揮軍南下犯金。哀宗遂以“率軍抗蒙”為由,留下家眷於汴京而出奔。完顏承麟沿途不離不棄,以身保哀宗之全。縱然沿途文武百官很多都為保命而逃走,完顏承麟依然以身護主,成功護送哀宗至歸德,後又護送哀宗至蔡州(今河南汝南)。

到達蔡州後,完顏承麟建議哀宗組織中央統治,積極籌備防衛蒙古來犯的事宜。

天興三年(1234年)正月,蒙古軍揮軍直進,加上南宋軍的支援下,蔡州被圍。哀宗深知亡國之日將至,不願當亡國之君,遂下詔禪位予完顏承麟,承麟初執意推卻,後哀宗苦苦哀求,曰:“朕所以付卿者,豈得己哉;以朕肌肥,不便鞍馬。城陷之後,馳突必難。顧卿平昔以疾聞,且有將略可稱。萬一得免,使祚胤不絕,此朕之志也。”故此,完顏承麟唯有答允繼位。

翌日(1234年2月9日),傳位大典剛開始不久,即接到戰報說蒙宋聯軍已攻進城內,哀宗倉皇逃往幽蘭軒,而剛即位的末帝完顏承麟即急率軍出門迎敵,展開巷戰,唯不敵蒙宋聯軍,遂退守子城。

另一方面,哀宗逃往幽蘭軒後即自縊而死。完顏承麟收到哀宗死訊後,與百官到哀宗遺體前痛哭,谥曰哀宗。“哭奠未毕,城溃,诸禁近举火焚之。奉御绛山收哀宗骨瘗之汝水上。末帝为乱軍所殺,金亡。”

蒙宋合作滅金后,对女真人有「惟完颜一族不赦」的说法。完颜氏的人不是被杀,就是隐名埋姓以其他姓氏繁衍。据查,完颜部落后人较多的地方有安徽的肥东、福建的泉州和台湾的彰化縣,但只有为完顏宗弼之子芮王完顏亨、末帝完顏承麟守陵的甘肃省平凉市泾川县王村镇完颜村的完颜族人,才在一个偏僻的地方以完顏为姓氏繁衍下来。

\subsection{天兴}

\begin{longtable}{|>{\centering\scriptsize}m{2em}|>{\centering\scriptsize}m{1.3em}|>{\centering}m{8.8em}|}
  % \caption{秦王政}\
  \toprule
  \SimHei \normalsize 年数 & \SimHei \scriptsize 公元 & \SimHei 大事件 \tabularnewline
  % \midrule
  \endfirsthead
  \toprule
  \SimHei \normalsize 年数 & \SimHei \scriptsize 公元 & \SimHei 大事件 \tabularnewline
  \midrule
  \endhead
  \midrule
  三年 & 1234 & \tabularnewline
  \bottomrule
\end{longtable}


%%% Local Variables:
%%% mode: latex
%%% TeX-engine: xetex
%%% TeX-master: "../Main"
%%% End:
