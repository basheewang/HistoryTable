%% -*- coding: utf-8 -*-
%% Time-stamp: <Chen Wang: 2019-12-26 11:17:12>

\section{熙宗\tiny(1135-1149)}

\subsection{生平}

金熙宗完顏亶(1119年8月14日-1150年1月9日),金朝第三位皇帝(1135年2月10日—1150年1月9日在位)。女真名合剌,漢名亶,是金太祖完顏阿骨打之嫡長孫,父為太祖嫡長子完顏宗峻、母為蒲察氏。生于天輔三年七月七日(1119年8月14日),卒于皇統九年十二月九日(1150年1月9日)。在位15年,终年31岁。

天輔三年(己亥年)出生,本名合剌,母親是蒲察氏,父親是完顏阿骨打的嫡長子。

天會八年,諳班勃極烈完顏杲薨逝,金太宗意久未決。天會十年,左副元帥完顏宗翰、右副元帥完顏宗輔、左監軍完顏希尹等大臣進入朝廷與完顏宗幹討論國事,稱:「諳班勃極烈虛位已久,今不早定,恐授非其人。合剌,先帝嫡孫,當立。」相與請於太宗者再三,乃從之。

天會十三年正月己巳,金太宗駕崩。

1135年2月10日,即皇帝位。不久,對外公佈、並下令公私部門皆禁止飲酒與相關娛樂,並向偽齊、高麗、夏等國派遣使節稱金朝皇帝已經即位;並詔令劉齊今後稱自己為臣,不能稱子。

天會十五年(1137年)十一月丙午,为鞏固政权,金熙宗下詔廢除伪齐,降封劉豫為蜀王,並与南宋議和。十二月戊辰,劉豫上表感謝封爵。不久,發佈詔令改明年爲天眷元年,並大赦,命韓昉、耶律紹文等人編修國史。之後命令蜀王劉豫遷徙至臨潢府。

天眷二年(1139年)正月,金、宋議和成立,南宋代替偽齊政權成為金的屬國,宋對金稱臣,金朝歸還河南、陝西。但是主戰派很快占了上風。天眷三年(1140年)五月,金熙宗詔令兀朮收復河南、陝西等地。

皇統元年(1141年),完顏宗弼再次帶兵南侵,被岳飛、韓世忠等擊退,但宋高宗急於求和,再次達成紹興和議,金朝至此控制淮河以北。

皇统五年(1145年),取消辽东汉人、渤海猛安谋克世袭的制度,逐渐将兵权转移到女真人手中,分猛安谋克为上中下三等,宗室为上等,其余次之。

熙宗廢除了太祖、太宗傳下來的勃極烈制度。完顏阿骨打庶長子完顏宗幹(同時也是熙宗的養父)崇尚漢化,在開國之初太宗任命宗幹輔助朝政制定各種制度,為女真漢化及鞏固金在华北統治打下基礎。

熙宗自幼接受漢化教育,加上養父的影響,登基後開始了漢制改革、重用漢人。太祖四子完顏宗弼(又名金兀朮)是推動漢制的重臣,熙宗授以軍政大權。天會十四年(1136年),宗磐、宗幹和宗翰三人共同總管政府機構,「並領三省事」。金朝官制此時基本漢化,建立了以尚書省為中心的三省制,以三師(太師、太傅、太保)以及三公(太尉、司徒、司空)領三省事。

勃極烈制度廢除前,女真的傳統一般是同代相傳,比如景祖烏古迺將權力傳給世祖劾里缽,然後是劾里缽的四弟肅宗頗剌淑和五弟穆宗盈歌(長子劾者和三子劾孫因為柔善而被景袓跳過),這一輪過後才是最有勢力家族的下一代,世祖劾里缽之子康宗烏雅束、太祖阿骨打、太宗吳乞買和遼王斜也。斜也一死,太宗把皇儲諳班勃極烈的位置空閒了兩年,在大家的催促下才選了一個太祖阿骨打家族的嫡長孫作皇儲。

等到熙宗繼位後,漢化的結果就是廢除了諳班勃極烈這種舊的皇儲制度,皇帝立自己的兒子作太子。這引起了本來能在太宗朝成為太子的太宗長子完顏宗磐的不滿。为免出现宋朝太祖太宗朝纷爭局面,熙宗因此對太宗子孫比較忍讓。後來宗磐還是發動了叛亂,但被平息。

宋金议和以後,宗翰、宗幹、宗弼等太祖太宗朝的老功臣相繼秉政,熙宗臨朝一般不说话。等到皇統(1148年)十月,宗弼去世,熙宗才有機會親政。但悼平皇后裴滿氏又很潑辣,干預政事,無所忌憚。加上熙宗的兩個年幼兒子,太子濟安、魏王道濟相繼在皇統三、四年去世,帝位失嗣。熙宗便徹底崩潰,開始嗜酒如命,不理朝政,濫殺無辜,更杀死了蒙古族的俺巴孩汗,朝野人心惶惶。皇統九年(1149)十月,熙宗弟族完顏元、完顏阿愣等人因受海陵王完顏亮誣告而被熙宗全數殺害,熙宗因此被孤立,也給完顏亮日後的篡位埋下了禍根。

皇統九年十二月初九丁巳日(儒略曆1150年1月9日),被右丞相海陵王完顏亮所殺,終年31歲。

天德二年(1150年)二月庚戌,被海陵王降為東昏王,葬於皇后裴滿氏墓中。貞元三年(1155年),改葬於大房山蓼香甸諸王墓群。海陵王死後,金世宗於大定元年(1161年)十一月恢復完顏亶帝號,追諡武靈皇帝,廟號閔宗,墓稱思陵。大定十九年(1179年)四月,升祔於太廟,增諡弘基纘武莊靖孝成皇帝。大定二十七年(1187年)二月,改廟號熙宗。大定二十八年(1188年),以思陵狹小,改葬於峨眉谷,仍號思陵。

元朝官修正史《金史》脱脱等的評價是:“熙宗之时,四方无事,敬礼宗室大臣,委以国政,其继体守文之治,有足观者。末年酗酒妄杀,人怀危惧。所谓前有谗而不见,后有贼而不知。驯致其祸,非一朝一夕故也。”

\subsection{天眷}


\begin{longtable}{|>{\centering\scriptsize}m{2em}|>{\centering\scriptsize}m{1.3em}|>{\centering}m{8.8em}|}
  % \caption{秦王政}\
  \toprule
  \SimHei \normalsize 年数 & \SimHei \scriptsize 公元 & \SimHei 大事件 \tabularnewline
  % \midrule
  \endfirsthead
  \toprule
  \SimHei \normalsize 年数 & \SimHei \scriptsize 公元 & \SimHei 大事件 \tabularnewline
  \midrule
  \endhead
  \midrule
  元年 & 1138 & \tabularnewline\hline
  二年 & 1139 & \tabularnewline\hline
  三年 & 1140 & \tabularnewline
  \bottomrule
\end{longtable}

\subsection{皇统}

\begin{longtable}{|>{\centering\scriptsize}m{2em}|>{\centering\scriptsize}m{1.3em}|>{\centering}m{8.8em}|}
  % \caption{秦王政}\
  \toprule
  \SimHei \normalsize 年数 & \SimHei \scriptsize 公元 & \SimHei 大事件 \tabularnewline
  % \midrule
  \endfirsthead
  \toprule
  \SimHei \normalsize 年数 & \SimHei \scriptsize 公元 & \SimHei 大事件 \tabularnewline
  \midrule
  \endhead
  \midrule
  元年 & 1141 & \tabularnewline\hline
  二年 & 1142 & \tabularnewline\hline
  三年 & 1143 & \tabularnewline\hline
  四年 & 1144 & \tabularnewline\hline
  五年 & 1145 & \tabularnewline\hline
  六年 & 1146 & \tabularnewline\hline
  七年 & 1147 & \tabularnewline\hline
  八年 & 1148 & \tabularnewline\hline
  九年 & 1149 & \tabularnewline
  \bottomrule
\end{longtable}


%%% Local Variables:
%%% mode: latex
%%% TeX-engine: xetex
%%% TeX-master: "../Main"
%%% End:
