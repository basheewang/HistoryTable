%% -*- coding: utf-8 -*-
%% Time-stamp: <Chen Wang: 2019-10-18 15:46:02>

\section{英宗\tiny(1320-1323)}

元英宗硕德八剌,是元朝第五位皇帝,蒙古帝国第九位大汗,1320年4月19日—1323年9月4日在位,在位3年零5个月,是元仁宗之子。

1321年11月28日,群臣为硕德八剌上汉语尊号继天体道敬文仁武大昭孝皇帝。

去世后,谥号睿圣文孝皇帝,庙号英宗,蒙古语称号格坚皇帝。

延祐七年农历正月二十一日(1320年3月1日),元仁宗去世。延祐七年农历三月十一日(1320年4月19日),18岁的硕德八剌在太皇太后答己及右丞相铁木迭儿等人的扶持下,在大都大明殿登基称帝,是为元英宗,改元“至治”。英宗自幼受儒學薰陶,登基后推行“以儒治國”政策,但是前期英宗的权力受到太皇太后答己和权臣铁木迭儿的很大限制。

延祐七年农历五月十一日(1320年6月17日),元英宗任命拜住为左丞相,以遏制太皇太后和铁木迭儿的权力扩张。

至治元年农历十一月九日(1321年11月28日),群臣为元英宗上尊号继天体道敬文仁武大昭孝皇帝。

1322年10月6日右丞相铁木迭儿去世,1322年11月1日太皇太后去世 ,元英宗终于得以亲政。

至治二年农历十月二十五日(1322年12月4日),元英宗任命拜住为中书右丞相,并且不设左丞相,以拜住为唯一的丞相。在右丞相拜住、中书省平章政事张珪等的帮助下,元英宗进行改革,并实施了一些新政,比如裁减冗官,监督官员不法行为,颁布新法律,采用“助役法”以减轻人民的差役负担,等等。史称“至治改革”。

英宗在位后期,官修政書《大元圣政国朝典章》(《元典章》),内容包括元太宗六年(1234年)到元英宗至治二年(1322年)大约90年的政治、经济、军事、法律等方面官方资料,具有极高的史料价值。

至治三年农历二月十九日(1323年3月26日),元英宗颁布了继《至元新格》之后元朝第二部法律典籍—《大元通制》,一共有二千五百三十九条,其中断例七百一十七、条格千一百五十一、诏赦九十四、令类五百七十七。

元英宗曾经想把征东行省(高丽王国)郡县化,罢征东行省,改立三韩行省,完全和元朝的其他行省一个待遇,“制式如他省,诏下中书杂议”,因为集贤大学士王约说:“高丽去京师四千里,地瘠民贫,夷俗杂尚,非中原比,万一梗化,疲力治之,非幸事也,不如守祖宗旧制。”得到丞相的赞同,设立三韩行省奏议没有实行。最终高丽国祚得以存续,高丽人知道后,为王约画像带回高丽,为之立生祠,并说:“不绝国祀者,王公也。”

元英宗的新政使得元朝国势大有起色,但新政却触及到了蒙古保守贵族的利益,引起了他们的不满,而且英宗下令清除朝中铁木迭儿的势力,随着清理的扩大化,铁木迭儿的义子铁失在至治三年八月初四(1323年9月4日)趁着英宗从上都避暑结束南返大都途中,在上都以南15公里的地方南坡的刺杀了英宗及右丞相拜住等人。史称南坡之变。英宗去世时年仅21岁。

泰定元年农历二月十六日(1324年3月11日),元泰定帝为硕德八剌上谥号睿圣文孝皇帝,庙号英宗。

泰定元年农历四月八日(1324年5月1日),元泰定帝为硕德八剌上蒙古文稱号“格坚皇帝”。

明朝官修正史《元史》宋濂等的評價是:“英宗性刚明,尝以地震减膳、彻乐、避正殿,有近臣称觞以贺,问:‘何为贺?朕方修德不暇,汝为大臣,不能匡辅,反为谄耶?’斥出之。拜住进曰:‘地震乃臣等失职,宜求贤以代。’曰:‘毋多逊,此朕之过也。’尝戒群臣曰:‘卿等居高位,食厚禄,当勉力图报。苟或贫乏,朕不惜赐汝;若为不法,则必刑无赦。’八思吉思下狱,谓左右曰:‘法者,祖宗所制,非朕所得私。八思吉思虽事朕日久,今其有罪,当论如法。’尝御鹿顶殿,谓拜住曰:‘朕以幼冲,嗣承大业,锦衣玉食,何求不得。惟我祖宗栉风沐雨,戡定万方,曾有此乐邪?卿元勋之裔,当体朕至怀,毋忝尔祖。’拜住顿首对曰:‘创业惟艰,守成不易,陛下睿思及此,亿兆之福也。’又谓大臣曰:‘中书选人署事未旬日,御史台即改除之。台除者,中书亦然。今山林之下,遗逸良多,卿等不能尽心求访,惟以亲戚故旧更相引用邪?’其明断如此。然以果于刑戮,奸党畏诛,遂构大变云。”

清朝史学家邵远平《元史类编》的評價是:“册曰:三载承乾,庶务锐始;大飨躬亲,致哀尽礼;刚过鲜终,肘腋祸起;不察几先,励精徒尔。”

清朝史学家毕沅《续资治通鉴》的評價是:“帝性刚明,尝以地震,减膳,彻乐,避正殿,有近臣称觞以贺,问:‘何为贺?朕方修德不暇,汝为大臣,不能匡辅,反为谄耶?’斥出之。尝戒群臣曰:‘卿等居高位,食厚禄,当勉力图报。苟或贫乏,朕不惜赐汝;若为不法,则必刑无赦。’巴尔济苏下狱,谓左右曰:‘法者,祖宗所制,非朕所得私。巴尔济苏虽事朕日久,今有罪,当论如法。’尝御鹿顶殿,谓拜珠曰:‘朕以幼冲,嗣承大业,锦衣玉食,何求不得!惟我祖宗栉风沐雨,戡定万方,曾有此乐耶?卿元勋之裔,当体朕至怀,毋忝尔祖!’拜珠顿首谢曰:‘创业维艰,守成不易,陛下言及此,亿兆之福也。’又谓大臣曰:‘中书选人署事未旬日,御史台即改除之。台除亦然。今山林之士,遗逸良多,卿等不能尽心求访,惟以亲戚故旧更相引用耶?’其明断如此。然以果于刑戮,奸党惧诛,遂构大变云。”

清朝史学家魏源《元史新编》的評價是:“旧史谓英宗果于诛戮,奸党畏惧,遂构大变。乌乎!是何言与?以铁木迭儿之奸,不明正其诛,但疏远俾得善终于位,已为漏网,而复任用其子,曲贷其子,酿成枭獍。此失之果乎?失之不果乎?拜住于铁木迭儿引其党参政张思明自助时,或告拜住为备,拜住反以大臣不和,彼仇我报,非国家之利。及铁木迭儿死,又往哭之痛,此皆失之果乎?失之不果乎?且除奸莫要于夺兵权,乃以宿卫新兵掌于铁失之手。司徒刘夔冒卖浙田之案,真人蔡道泰杀人赇逭之案,皆奸赃巨万。拜住既平反其狱,独赦铁失不问。中书参议谏以除奸不可犹豫,犹豫恐生他变,拜住是其言而不能用。大抵安童、拜住皆木华黎之孙,木华黎用兵所过,动辄屠戮。安童从许衡受学,故其子孙皆出于宽容,以水懦救火猛,德量有余,机警不足,所谓君子之过过于厚也。乃胡粹中因旧史之言,谓英宗在位数载,除诛戮外无一善政可纪,甚至皇太后以嬖孽失势之故,郁郁而终,胡氏并指为英宗不孝祖母之罪。乌乎!其性与人殊,乃至此乎?”

清朝史学家曾廉《元书》的評價是:“论曰:英宗知赵世炎之非辜,抑亦汉昭之流亚也。然汉昭能诛燕王、上官桀,而专任霍光,英宗不能诛铁木迭儿诸权倖之徒,独任拜住也。抑考元时蒙古人横不可悉裁以法度,以拜住之世旧勋贵而不能骤正也。夫自古无无小人之朝,在振纪纲而已。自世祖好货开倖进之门,安童不能与阿合马、桑哥争,况幼沖在位乎?使霍光处此,则必射隼,于高墉藏器,儃回操刀,弗割明君贤相,胥受其祸,悲夫!”

清末民初史学家屠寄《蒙兀儿史记》的評價是:“汗性刚明,励精图治,尝御上都大安阁,见太祖、世祖遗衣,皆以缣素木绵为之,重加补缀。嗟叹良久,谓侍臣曰:‘祖宗草昧经营,服御节俭乃尔,朕焉敢顷刻忘之。’敕画《蚕麦图》于鹿顶殿,以时观之,藉知民事。一日御殿,谓拜住曰:‘朕冲龄嗣祚,锦衣玉食,何求不得。惟我祖宗节风沐雨,戡定大难,曾有此乐耶?卿元勋之裔,当体朕至怀,毋忝尔祖。’拜住顿首对曰:‘创业惟艰,守成亦不易,陛下睿思及此,亿兆之福也。’汗承延祐宽政之后,思济之以猛,御下甚严,在谅闇中。中书参议乞失监坐鬻官,刑部议法当杖,太后欲改笞,汗不可,曰:‘法者,天下之公,徇私而轻重之,非所以示民也。’卒从部议。每戒群臣曰:‘卿等居高位,食厚禄,当勉力图报。苟或贫乏,朕不惜赐汝;若为不法,则必刑无赦。’八思吉思下狱时,汗谓左右曰:‘法者,祖宗之制,非朕所得私。八思吉思虽事朕日久,今既有罪,当论如法。’其明决如此。然过信喇嘛,大起山寺,不受忠谏,饮酒逾量,有时至失常度云。”

民国官修正史《新元史》柯劭忞的評價是:“英宗诛兴圣太后幸臣失列门等,太后坐视而不能救,其严明过仁宗远甚。然蔽于铁木迭儿,既死始悟其奸,又置其逆党于肘腋之地。故南坡之祸。由于帝之失刑,非由于杀戮也。旧史所讥殆不然矣。”

\subsection{志治}

\begin{longtable}{|>{\centering\scriptsize}m{2em}|>{\centering\scriptsize}m{1.3em}|>{\centering}m{8.8em}|}
  % \caption{秦王政}\
  \toprule
  \SimHei \normalsize 年数 & \SimHei \scriptsize 公元 & \SimHei 大事件 \tabularnewline
  % \midrule
  \endfirsthead
  \toprule
  \SimHei \normalsize 年数 & \SimHei \scriptsize 公元 & \SimHei 大事件 \tabularnewline
  \midrule
  \endhead
  \midrule
  元年 & 1321 & \tabularnewline\hline
  二年 & 1322 & \tabularnewline\hline
  三年 & 1323 & \tabularnewline
  \bottomrule
\end{longtable}


%%% Local Variables:
%%% mode: latex
%%% TeX-engine: xetex
%%% TeX-master: "../Main"
%%% End:
