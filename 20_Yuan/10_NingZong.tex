%% -*- coding: utf-8 -*-
%% Time-stamp: <Chen Wang: 2019-10-18 16:13:49>

\section{宁宗\tiny(1332)}

元寧宗懿璘质班是元朝第十位皇帝,蒙古帝国第十四位大汗。元明宗次子。1332年10月23日—1332年12月14日在位,在位2个月。

他去世后,谥号冲圣嗣孝皇帝,庙号寧宗。

《元史》记载,元宁宗于泰定三年三月二十九癸酉日(1326年5月1日)生于北方草原。

至顺三年八月十二日(1332年9月2日),元文宗崩。据杂史,元文宗在死前下诏让元明宗之子继承皇位。文宗死后,把持朝政的燕铁木儿为了继续专权,就请求元文宗皇后卜答失里立她的儿子古納答剌为帝。卜答失里为了执行丈夫的遗诏,予以拒绝。由于当时元明宗的长子妥懽贴睦尔(后来的元惠宗)远在广西静江(今广西桂林),而次子懿璘质班却深得文宗宠爱,受封为鄜王,留在文宗身边。

至顺三年十月初四(1332年10月23日),卜答失里皇后遂奉文宗遗诏拥立年仅7岁的懿璘质班在大都大明殿登上皇位,是为元宁宗。因为皇帝年幼,卜答失里皇后临朝称制,成了元朝的实际统治者。

懿璘质班即位后未改元,年号仍旧是“至顺”,至顺三年十一月二十六日(1332年12月14日),元宁宗在大都病逝,年仅7岁,在位仅53天。

至元三年正月十日(1337年2月10日),元惠宗为懿璘质班上谥号冲圣嗣孝皇帝、庙号宁宗。

清朝史学家魏源《元史新编》的評價是:“乌乎!《春秋》未逾年之君称子,故子般不与闵公并立庙谥。宁宗以负扆匝月之殇,而入庙称宗,立后媲谥,无一人引大谊以匡正之,斯元代礼臣博士之陋也。修史者又踵其失而立《本纪》,斯又明臣之陋也。今以附诸《文宗本纪》之末。”

清朝史学家曾廉《元书》的評價是:“论曰:文宗杀明宗皇后,播告天下,言妥懽帖睦尔非明宗子,既出之于静江,乃立皇子阿剌忒答剌为皇太子,公私之情见矣。皇天弗佑,元良夭丧,及大惭,而爱其少子之弱,非妥懽帖睦尔不能延其祚,而不可为之辞矣。则亦曰立明宗子,一似以明其固让之初志也者。任后人之拥戴,犹武宗之孙也。惟宁宗亦弗永年而大位卒,归于向所猜忌之兄子,天也!人岂有为哉!”

清末民初史学家屠寄《蒙兀儿史记》的評價是:“鄜王之立,不再月而殇。既未逾年改元,又未有所建设,顾乃追尊上谥,立庙称宗,甚乖《春秋》鲁般书子卒之义。蒙兀君臣瞢不知经,诚无足责,而明初脩胜国之史,仍立之本纪,不加裁正,宜乎魏源讥其陋也。退附《文宗本纪》,自邵远平始。”

民国官修正史《新元史》柯劭忞的評價是:“《春秋》之义,未逾年之君称子。宁宗即位匝月而殇,乃入庙称宗;其廷臣不学如此,岂非失礼之大者哉。” 

\subsection{志顺}

\begin{longtable}{|>{\centering\scriptsize}m{2em}|>{\centering\scriptsize}m{1.3em}|>{\centering}m{8.8em}|}
  % \caption{秦王政}\
  \toprule
  \SimHei \normalsize 年数 & \SimHei \scriptsize 公元 & \SimHei 大事件 \tabularnewline
  % \midrule
  \endfirsthead
  \toprule
  \SimHei \normalsize 年数 & \SimHei \scriptsize 公元 & \SimHei 大事件 \tabularnewline
  \midrule
  \endhead
  \midrule
  三年 & 1332 & \tabularnewline
  \bottomrule
\end{longtable}


%%% Local Variables:
%%% mode: latex
%%% TeX-engine: xetex
%%% TeX-master: "../Main"
%%% End:
