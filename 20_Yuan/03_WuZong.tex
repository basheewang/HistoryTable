%% -*- coding: utf-8 -*-
%% Time-stamp: <Chen Wang: 2019-10-18 15:41:12>

\section{武宗\tiny(1307-1311)}

元武宗海山,是元朝第三位皇帝,蒙古帝国第七位大汗,在位4年,自1307年6月21日至1311年1月27日。乃元世祖之曾孫、太子真金之孫、答剌麻八剌之子、元成宗之侄。

1309年2月17日,群臣为海山上汉文尊号统天继圣钦文英武大章孝皇帝。

他去世后,謚號仁惠宣孝皇帝,廟號武宗,蒙古语称曲律皇帝。

武宗為真金次子答剌麻八剌之次子,嫡長子,1299年,海山接受元成宗的命令统兵漠北,负责同西北窝阔台汗国的君主海都和察合台汗国君主笃哇作战,多立戰功,为元朝结束和西北宗王的战争,以及1303年四大汗国全部承认元朝宗主地位做出了重要贡献。因为战功被封为懷寧王。

大德十一年正月初八(1307年2月10日),元成宗鐵穆耳病逝,儲位虛懸。成宗的伯牙吾·卜鲁罕皇后下命垂簾聽政,命安西王阿難答輔政。海山回大都奔喪,其弟愛育黎拔力八達與右丞相哈剌哈孫合謀发动政变,囚禁伯牙吾·卜鲁罕皇后和安西王阿難答,宣布擁立在外拥有重兵的海山為帝,是為元武宗,海山即位后追封其父答剌麻八剌為元順宗。

大德十一年五月二十一日(1307年6月21日),武宗在上都大安阁即位,之後处死伯牙吾·卜鲁罕皇后和阿難答,并更換了成宗时期的大臣,封其弟愛育黎拔力八達為皇太弟。在位只得四年,大興土木,建筑中都城,派军士千餘人及大量民工修建五台山華佛寺,又令喇嘛翻譯佛經,并曾想规定凡毆打西僧者截其手,罵西僧者斷其舌(但在其弟即后来的元仁宗愛育黎拔力八達劝告下取消)。

大德十一年七月十九日(1307年8月17日),元武宗下诏加封“至圣文宣王”孔子为“大成至圣文宣王”。

至大元年(1308年)五月,白蓮教被禁止。

至大元年(1308年),元武宗派遣月鲁出使钦察汗国,册封钦察汗脱脱为宁肃王。

至大二年(1309年),元朝和察合台汗国联手灭亡窝阔台汗国,元朝取得窝阔台汗国北部,察合台汗国取得窝阔台汗国南部。

至大二年(1309年)九月,为摆脱财政危机,印發至大銀鈔,导致至元钞大为贬值,從二釐到二兩分為十三等,並在各路、府、州、縣設常平倉平抑物價。將中書省宣敕、用人的權力劃歸尚書省。

至大四年正月初八日(1311年1月27日),因沉耽淫乐、酗酒过度,武宗病逝於大都玉德殿,享年三十岁,葬於起輦谷。

至大四年三月十八日(1311年4月7日),其弟愛育黎拔力八達(元仁宗)以皇太弟身份即位,廢除一切新政。

至大四年六月二十四日(1311年7月10日),元仁宗为海山上謚號仁惠宣孝皇帝,廟號武宗,蒙古语称号曲律皇帝。

至大二年正月初七日(1309年2月17日),皇太子、诸王、百官为元武宗上尊号统天继圣钦文英武大章孝皇帝。

由于日本拒绝向元朝称臣,元朝下令增加日货税收,日本不满,后来虽然减少关税,但仍然对日商检查甚严。

至大元年(1308年)日本商船焚掠庆元,官军不能敌。

至大四年(1311年)十月,以江浙省尝言:“两浙沿海濒江隘口,地接诸番,海寇出没,兼收附江南之后,三十余年,承平日久,将骄卒情,帅领不得其人,军马安量不当,乞斟酌冲要去处,迁调镇遏。“枢密院官议:“庆元与日本相接,且为倭商焚毁,宜如所请,其余迁调军马,事关机务,别议行之。”由此可见,此时元朝在东南沿海一带的军队战斗力很差(草原的军队因为世祖朝和成宗朝经常在西北作战,战斗力还可以)。

明朝官修正史《元史》宋濂等的評價是:“武宗当富有之大业,慨然欲创治改法而有为,故其封爵太盛,而遥授之官众,锡赉太隆,而泛赏之恩溥,至元、大德之政,于是稍有变更云。”

清朝史学家邵远平《元史类编》的評價是:“册曰:北藩入嗣,三宫协和;慨然创治,爵滥赏阿;貮省乱政,令教繁讹;有为何裨,变政已多。”

清朝史学家毕沅《续资治通鉴》的評價是:“帝承世祖、成宗承平之业,慨然欲创制改法;而封爵太盛,多遥授之官,锡赉太优,泛赏无节。至元、大德之政,于是乎变。”

清朝史学家魏源《元史新编》的評價是:“武宗始以怀宁王总兵漠北和林,与叛王海都劲敌对垒,屡摧其锋,中间几濒险危,披坚陷阵,威震遐荒,可谓天潢之杰出,天授之雄武矣。入绍大统,谓有宏图,而始终误听宵人,以立尚书省为营利之府,何哉?夫世祖立制,以天下大政归于中书省,任相任贤,责无旁贷。故小人欲变法,忌中书不便于己,则必别立尚书省以夺其权。阿合马、桑哥之徒相继乱政,毒流海内,是以世祖深戒前辙,不复再蹈。乃当席丰履厚之余,慨然欲变更至元、大德之旧。封爵太盛,而遥授之官多;锡赉太侈,而滥赏之卮漏。母后市恩左右,挠其恭俭,于是言利之臣迎合攘袂,以争利权。虽柄操自上,不至如阿合马、桑哥之甚,而仁心仁闻渐蔽于功利,几同于宋之熙、丰。故仁宗绍统,翻然诛殛,尽复旧章。盖变法不得其人,则不如勿薬之尚得中医也。又攷陶九仪《元氏掖庭记》,则琼岛水嬉之华,月殿霓裳之豔,亦自帝大滥其觞,而《本纪》讳之,不载一字,亦英雄酒色之通病欤!惟授受之际,坚守金匮传弟之盟,虽有内侍李邦宁,怂恿离间,帝言:‘朕志已定,汝自往东宫言之。’斯则磊落光明,胜宋太宗万万。综计始末,固不失为一代之英主焉。”

清朝史学家曾廉《元书》的評價是:“论曰:武宗擐甲临边,至登大位,宜有雄武之风,而颓然晏安,惟鞠蘖芗泽之为乐,元业自是衰矣。遂至鼎鼐充庭,名器之贱如履。而欲后人惜其敝袴,得乎?易日负且乘致寇至,武宗启之矣。”

民国史学家屠寄《蒙兀儿史记》的評價是:“海山汗滥赏淫威,非恭俭之主也。明知尚书省貮政病民,排众议而立之。更钞铸钱,将以理财,而财政愈紊,前史称其慨然欲有所为,然郊天、祀孔、亲享太庙,诸虚文外,无足纪者。惟终身远铁木迭儿,虽以母后之命,不使得预朝政。由后校之,殆有所先见矣。若乃三宫协和,始终不受谗慝。其自处骨肉之间,盖亦有道焉尔。”

民国官修正史《新元史》柯劭忞的評價是:“武宗舍其子而立仁宗。与宋宣公舍与夷而立穆公无以异。公羊子曰:朱之乱,宣公为之。然则英宗之弑,文宗之篡夺,亦帝为之欤!《春秋》贵让而不贵争,公羊子之言过矣。帝享国日浅,滥恩幸赏无一善之可书。独传位仁宗,不愧孝友。其流祚于子孙宜哉。”

\subsection{至大}

\begin{longtable}{|>{\centering\scriptsize}m{2em}|>{\centering\scriptsize}m{1.3em}|>{\centering}m{8.8em}|}
  % \caption{秦王政}\
  \toprule
  \SimHei \normalsize 年数 & \SimHei \scriptsize 公元 & \SimHei 大事件 \tabularnewline
  % \midrule
  \endfirsthead
  \toprule
  \SimHei \normalsize 年数 & \SimHei \scriptsize 公元 & \SimHei 大事件 \tabularnewline
  \midrule
  \endhead
  \midrule
  元年 & 1308 & \tabularnewline\hline
  二年 & 1309 & \tabularnewline\hline
  三年 & 1310 & \tabularnewline\hline
  四年 & 1311 & \tabularnewline
  \bottomrule
\end{longtable}


%%% Local Variables:
%%% mode: latex
%%% TeX-engine: xetex
%%% TeX-master: "../Main"
%%% End:
