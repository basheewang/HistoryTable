%% -*- coding: utf-8 -*-
%% Time-stamp: <Chen Wang: 2019-12-26 14:36:18>

\chapter{元\tiny(1271-1368)}

\section{简介}

元朝(1271年-1368年),漢語国号全稱为大元,蒙古語國號全稱大元也克蒙古兀鲁思(意为大元大蒙古國),是中國歷史上由蒙古人所建立的大一統王朝。1260年,忽必烈稱帝,自立為第五代大蒙古國大汗,後於1271年取儒士劉秉忠建議,定漢文國號為「大元」,改蒙古語国号「大蒙古国」為「大元大蒙古国」,定都於漢地大都(今北京市),建立元朝。1279年元軍徹底攻灭南宋殘餘勢力,一統中國並結束南宋與金朝南北政權对峙之局面。雖然傳統以南宋為正統王朝,由於金朝認為已繼承宋朝正統,有一說認為元朝繼承金朝正統,並選取根據五行相生順序生自金朝「土」德的「金」德為王朝德運,同時選取與金德對應的白色為王朝正色。

元朝的基础為乞颜部族的首领铁木真于1206年统一漠北诸部族后建立的大蒙古國,铁木真被称为“成吉思汗”。當時蒙古诸部受金朝统辖,然而由於金朝與西夏均走向衰落,成吉思汗先後攻打西夏與金朝,並於西元1227年8月攻滅西夏、1234年3月攻滅金朝,取得中国華北地区和黄土高原地区。同一时间,大蒙古国在西方不断扩张,先後發動三次西征,形成稱霸歐亞大陸的国家,被欧洲称为蒙古帝國(Mongol Empire)。

1259年,第四代蒙古大汗蒙哥(拖雷長子)於征伐南宋的戰爭中去世後,領有漢地、主張漢化、陪同主持对南宋战争的忽必烈(拖雷第四子)與受漠北蒙古貴族擁護的阿里不哥(拖雷第七子)為了爭奪汗位而發生战争,最後忽必烈於1264年獲勝,而蒙古帝国也宣告徹底地分裂。自元太宗窩闊臺去世以來,蒙古四大汗國先後自立,而忽必烈对于“蒙古大汗”称号的继承也没有得到蒙古诸部的一致承认。

1260年三月忽必烈召集擁護自己的部分蒙古宗王,在開平府召開忽里勒台大會,舉行例行的大汗選舉儀式,宣佈即蒙古大汗位,是為薛禪汗,漢文廟號定為世祖。忽必烈建號「中統」,意即「中原正統」。1271年,忽必烈取《周易》“乾元”之语,公佈《建國號詔》,建立汉语國號為大元,宣佈新王朝為繼承歷代中原王朝的中華正統王朝,史称元朝,忽必烈即元朝的开国皇帝,庙号元世祖。1279年元朝攻滅南宋,統治全中國地區,结束自窩闊台攻宋以来40多年的蒙宋戰爭。元世祖到元武宗期間元朝國力鼎盛時期,軍事上平定西北,但在侵略日本、东南亚诸国卻屢次失利,其中在元日戰爭战败。元中期皇位之争愈演愈烈、政治动荡不安,诸帝施政亦不甚如意。元惠宗晚期,由於怠于政事、滥发纸币导致通货膨胀、為了治理氾濫的黄河又加重徭役,最後导致1351年爆發紅巾軍起事。1368年朱元璋建立明朝後,派徐達北伐攻陷大都,元朝結束。元廷退居漠北,史称北元。北元後主天元十年(1388年)去大元国号(一说1402年元臣鬼力赤篡位建國鞑靼),北元亡。

元朝建立后,承袭了蒙古帝國在中国北方、蒙古高原以及西伯利亚的領土,蒙古帝国西征而来的土地却不在元朝统治范围之内。元朝领土經過多次擴展後,於1310年元武宗時期達到全盛,西到吐鲁番,西南包括西藏、云南及缅甸北部,东到日本海,北至都播南部與北海、鄂畢河東部,被譽稱「东尽辽左西极流沙,北逾阴山南越海表,汉唐极盛之时不及也」。元朝至元成宗时,经过一系列战争和协商,获得欽察汗國、察合台汗國、窩闊台汗國與伊兒汗國等四大汗國承认为宗主國,并且元朝皇帝为名义上的“蒙古大汗”继任者;其藩屬國涵蓋高麗與東南亞各國。

元朝在經濟方面仍以農業為主,整體生產力向前發展,尤其是邊陲地區的經濟發展最為顯著,在生产技术、垦田面积、粮食产量、水利兴修以及棉花广泛种植等方面仍然取得一定進步。蒙古人是游牧民族,草原时期以畜牧为主,经济单一,无所谓土地制度。蒙古軍在攻打華北時,出於與舊主金朝的恩怨採取報複性政策,殘酷的屠杀和劫掠带来很大的破坏。攻滅金朝后,在耶律楚材勸諫下,窩闊台汗同意復甦农业,鼓勵漢人墾殖以期長治久安。元世祖即位之后,实行些鼓励生产、安抚流民的措施。到元朝時,由於经济作物棉花不断推广種植,與棉纺织品在江南一带都比较興盛。经济作物商品性生产的发展,就使当时基本上自给自足的农村经济,在某些方面渗入商品货币经济关系。但是,由於元帝集中控制大量的手工业工匠,经营日用工艺品的生产,官营手工业特别发达,对民间手工业则有限制。

元朝對中國傳統文化的影響大過對社會經濟的影響。不同於中国历史上其他征服王朝為了提升本身文化而積極吸收中華文化,元朝皇室对于宗教兴趣浓厚,极力推崇伊斯蘭教與藏傳佛教乃至景教,对中華文化则采取与西亞文化并重的模式进行发展。在政治上,政府大量使用来自西亚的色目人,降低契丹人、汉人儒者的地位,压制南人。雖然元朝前期沒有系統性舉辦科舉。,但对儒家文化有着应有的尊重,並且將儒家推廣至邊遠地區,元朝创建了24400所各级官学,使全国平均每2600人即拥有一所学校的政绩,创造了「書院之设,莫盛于元」的历史记录 。由於士大夫文化式微,意味宋朝顯貴的傳統社會秩序已經崩潰。這使得在士大夫文化底下,屬於中下層的庶民文化反而有機會迅速的抬頭並普及。這個現象在政治方面是重用胥吏,在藝術與文學方面則是發展以庶民為對象的戲劇與藝能,其中以元曲最為興盛。

元朝的汉文国号「大元」出自《易经·乾卦》“大哉乾元,万物资始,乃统天”。1271年12月18日,忽必烈汗公布《建国号诏》,宣佈新王朝為繼承歷代中原王朝的中華正統王朝 ,國號為大元。元朝是中国历史上第一个把“大”字加于正式国号之中的大一统王朝,除此前仅统治了华北地区的辽朝和金朝等外,之前各朝的“大”字均为尊称。

元朝歷史通常可以分為兩個到三個階段:

1206年元太祖铁木真統一蒙古,立國位於漠北的蒙古草原,定國號為「大蒙古國」;到1271年元世祖忽必烈定都元大都,將國號改為大元之际,共六十五年,稱為大蒙古國時期,又稱蒙古帝國;

元世祖忽必烈定都元大都,1271年將國號改為大元後,直到1368年元惠宗出亡為止,共九十七年,才是嚴格意義上的元朝歷史;

元惠宗出亡後依舊以大元為國號,至1402年鬼力赤殺順天帝坤帖木兒去國號為止(一說1388年天元帝脱古思帖木儿被也速迭尔杀害後去國號),稱為北元時期。

去國號後稱蒙古,明廷稱韃靼。

辽朝时期,蒙古草原上的诸部归于辽朝统辖。金灭辽后,草原各部歸屬不一,汪古部等成為金朝的臣屬,而乞顏部的合不勒汗乘金军大举南下而无暇北顾之机,建立了早期的蒙古国家,即蒙兀国,此后一直侵袭金朝的边境。合不勒汗死后,俺巴孩汗成为新的大汗。由于塔塔儿人的出卖,俺巴孩汗被金朝皇帝金熙宗钉在木驴上致死,此事件埋下了蒙古对金朝复仇的种子。在金章宗死后,13世紀初,金朝在衛紹王完颜永济的統治下走向衰落,蒙古乞颜部铁木真开始了统一蒙古草原的征程。先后在克烈部首领王罕以及他的安达扎答兰部首领札木合的军事援助下,打败了蔑兒乞人,夺回了被蔑儿乞人夺取的众多部众(以及其妻孛儿帖),力量逐渐壮大。1189年,在经过激烈的争夺之后,铁木真被乞颜贵族推举为部落的可汗。然而,铁木真部族的逐渐强大,危及了援助他的札木合在蒙古草原上的地位,于是札木合联合泰赤乌等部,合兵三万余人,向铁木真发起进攻。面对来势汹汹的札木合,铁木真将自己的部众3万人组成十三翼。在战斗中铁木真暂时战败,为保存实力退至斡难河的哲列捏山峡,扼险而守。史称“十三翼之战”。札木合虽然取得战役的胜利,但札木合的暴虐受到了其所属部落首领的不满,而铁木真对部众进行笼络,故部众归心于铁木真。于是畏答儿、赤老温、术赤台、晃豁坛等族人纷纷来附。此后,铁木真力量进一步壮大。1196年,从属于金朝的蒙古部族塔塔儿部叛金,完颜永济派丞相完颜襄率军征讨。铁木真联合克烈部,以“为父亲报仇”的名义,在斡里匝河击溃了塔塔儿部,使塔塔儿一蹶不振。战后,金朝授铁木真糺军统领之职,使他可以用金朝属官名义号令蒙古部众。1200年,铁木真与王汗会于萨里川(今蒙古国克鲁伦河上游之西),大败泰赤乌与蔑儿乞的联军,首领塔里忽台等被杀。1201年,铁木真又在呼伦贝尔海剌尔河支流帖尼河之野,击败以札木合为首的塔塔儿、弘吉剌、合答斤等十一部联军,史称“帖尼河之战”。宋嘉泰二年,铁木真与王汗联军又在阔亦田击败了札木合同乃蛮、泰赤乌、塔塔儿、蔑儿乞等联军,取得了阔亦田之战的胜利。接着招降了呼伦贝尔一带的弘吉剌惕等部。至此,蒙古高原都被铁木真控制了。最后平定蒙古高原,统一蒙古各部,1206年春,蒙古贵族在斡难河(今鄂嫩河)源头召开库里尔台大会,蒙古部鐵木真得到成吉思汗稱號,建國大蒙古国(即蒙古帝國),後被尊稱元太祖。

金朝與蒙古為世仇,成吉思汗有意伐金復仇,然而西南的西夏與金朝聯盟,為了避免被西夏牽制,先後三次率軍(1205年、1207年與1209年—1210年)进攻之,迫使西夏夏襄宗稱臣。1210年成吉思汗與金斷交,隔年發動蒙金戰爭,於野狐嶺戰役大破四十萬金軍,隨後攻入華北地區並四處屠杀。1214年蒙軍包圍金朝首都中都(今北京市),金宣宗被迫求和称臣,並在蒙古退兵後遷都北宋故都汴京。隔年5月31日蒙軍南下攻佔金中都,並且獲得名相耶律楚材,這對於巩固華北地區有很大的幫助。1217年,成吉思汗為了西征花剌子模,命木华黎統領漢地,封为“太师国王”,命他持續进攻金朝。木华黎為了鞏度漢地,收降地方自衛勢力如真定史天澤、滿城張柔、東平嚴實與濟南張宏,史稱漢族四大世侯,後來他們也扶佐忽必烈建立元朝。木华黎除了對金朝的戰爭讓金朝疆域萎縮剩河南與關中地區之外,並於1231年派兵進攻高麗,使高麗退到江華島以南(即今日南韓)。

西域方面,為了建立通往西方的道路,早在1209年—1210年就讓新疆东部的畏兀儿與伊犁河谷的哈剌魯先後歸順。當金朝遷都並將要滅亡之際,中亞新興大國花剌子模在沙阿摩诃末时期崛起,該國訛答剌地方大臣海儿汗亦纳勒术前后两次屠杀蒙古商队並侮辱蒙古使臣,成吉思汗遂决心發動第一次西征。1218年蒙將哲别殺死占領西遼並稱遼帝的屈出律,攻占塔里木地區,史稱蒙古攻西遼之戰。隔年六月,成吉思汗親率蒙古主力軍十万西征花剌子模。由於沙阿摩诃末抵擋不了蒙軍攻勢,畏懼而逃,在屠杀掉花剌子模的40个城镇之后,花剌子模也於1221年亡國。成吉思汗命速不臺和哲别追殺摩诃末,摩诃末最後死於裡海。其子札蘭丁於八魯灣之戰英勇抗敵,最後南逃印度,並於1224年復國於大不里士(今伊朗西北部)。1230年,札蘭丁被蒙古将军绰儿马罕攻滅。速不臺和哲别最後于1222年从撒马尔罕出发经过今伊朗高原北部,进攻杀掠高加索三国(亚美尼亚王国、格鲁吉亚、阿塞拜疆)之后,并越太和岭(今高加索山脈),抵達欽察(位於俄南),期間攻占不少國家。於1223年的迦勒迦河之战(今乌克兰日丹诺夫市北)更是擊潰基辅罗斯諸國與钦察忽炭汗的联军,并向西进军到今乌克兰西部的德涅斯特河,折转围攻基辅后东返,并于1223年9月攻击伏尔加河中上游的河谷伏尔加保加利亚,最後渡过伏尔加河東返中亞。成吉思汗將新拓展的疆土分封給長子朮赤、次子察合台和三子窝阔台,四子拖雷領有蒙古本土,三子窩闊台成為大汗繼承人。1225年蒙古回師後,因西夏不配合西征,成吉思汗又率归師滅西夏。1227年,成吉思汗病逝,由幼子拖雷监国。

拖雷监国两年后於1229年舉辦库里尔台大会,窝阔台被推举为蒙古大汗,後尊稱元太宗。1231年窝阔台汗率軍南征金朝,並命四弟拖雷自漢中借宋道沿漢水攻打汴京,隔年拖雷在河南三峰山之战擊潰金军。1234年蒙宋联軍聯合攻破蔡州,金哀宗自杀,金朝亡。南宋雖然發起端平入洛以收復河南地,但是華北地區最後全由蒙古占領。1235年,窝阔台汗定都哈拉和林(今乌兰巴托西南)後,藉此率軍南征南宋以報復之,掠奪兩淮地區後北返。蒙古為了防止華北的漢人世侯叛變,派探馬赤軍(振戍軍)進駐漢地;進行兩次人口調查,將半數漢人分封給蒙古功臣。由於需要人才治理国家,窝阔台汗接受耶律楚材的建议,於1238年命术忽德和刘中舉辦科舉,史稱戊戌选试。这次考试录取东平杨奂等名士,為統治華北帶來不少人才,但后来以“当世或以为非便,事复中止”。

西線方面,1235年窝阔台汗命术赤长子拔都、貴由與蒙哥、速不台等第二代蒙古王子發起蒙古第二次西征,史稱拔都西征,總指揮為拔都與速不台。1236年至1242年間攻占欽察草原、基輔羅斯等各公國并进犯匈牙利、摩尔达维亚、波蘭、立陶宛大公国、摩拉维亚原南斯拉夫地区、保加利亚第二帝国、拉什卡等中東歐各國。1241年11月窩闊台汗去世,由皇后乃马真脫列哥那監國,1246年3月的库里尔台大會由其子贵由即位,後追尊稱元定宗。1247年吐蕃诸部归附大蒙古,史稱涼州會盟。1248年8月貴由汗在遠征拔都的途中去世,皇后斡兀立海迷失立孫子失烈門並監國。然而在1251年7月的大會,因為拔都與兀良哈台大力支持拖雷系的蒙哥,使得窩闊台系的失烈失去汗位。蒙哥繼承汗位,後尊稱元憲宗。

1252年蒙哥即位後推行中央集權化,在漢地、中亞與伊朗等直轄地設置行中书省,分遣拖雷系諸王分守各地,以其弟忽必烈總領漠南漢地大總督以管理漢地。忽必烈統治漢地期間任用了大批漢族幕僚和儒士,鞏固了華北地區,並且與兀良合台迂迴南滅大理,擴大南宋防線缺口。1258年高麗崔氏政權跨台,高麗成為藩屬國。同年蒙哥汗宣布兵分三路南征南宋,蒙哥汗率軍攻打四川合州(今重慶)、忽必烈攻打湖北鄂州(今武昌)、兀良合台由雲南晏当(今云南丽江北部)直攻经过安南,进攻宋广南西路而直攻荆湖南路,并兵临潭州(今長沙),三軍意圖在華中會合,再大舉下長江圍攻臨安。隔年蒙哥汗在合州的釣魚城之戰戰死,忽必烈等人停止南征,北返奪位。西線方面,蒙哥汗派其弟旭烈兀西征西亞,史稱蒙古第三次西征,1256年旭烈兀攻滅伊斯蘭教的暗殺組織木剌夷。1258年西征軍攻佔阿拔斯王朝最後領地美索不达米亚的巴格達。1260年佔領大馬士革和阿勒頗。然而當旭烈兀得知蒙哥於南征南宋時去世的消息後,立即率大軍回師爭位。留下的蒙軍也在今以色列加利利的阿音札魯特戰役敗於埃及馬木留克王朝,第三次西征結束。

蒙哥汗去世後,身在戰事的忽必烈立即與南宋和談,返回華北與留守蒙古本土的七弟阿里不哥爭奪汗位。1260年5月5日忽必烈在部分宗王和蒙漢大臣的擁立下於开平(後稱上都,今内蒙古多伦县北石别苏木)自立为蒙古皇帝(又稱蒙古大汗),年號中統。忽必烈登基后不久,阿里不哥在蒙古帝國首都哈拉和林召開库里尔台大會,被阿速台等宗王和大臣選立蒙古大汗,並獲得欽察、察合台與窩闊台汗國的支持。爭奪汗位戰爭最後於1264年8月21日由阿里不哥兵敗投降,忽必烈穩固其位。

忽必烈汗為了成為中國皇帝而推行漢法,主要內容有改元建號,1267年忽必烈汗迁都中都(今北京市),並命劉秉忠兴建中都城。1272年改中都为大都(突厥语称汗八里,帝都之意),將上都作为陪都。1271年12月18日,忽必烈汗公布《建国号诏》,採納漢人儒士劉秉忠的建議,取《易经》中“乾元”之意,宣佈新王朝為繼承歷代中原王朝的中華正統王朝 ,将國號由大蒙古国改为大元,建立元朝,即元世祖;1260年设立中书省,1263年设立樞密院,1268年设立御史台等等國家機構;設置大司農司並且提倡農業;尊孔崇儒並大力發展儒學等推行漢法的政策。然而為了保留原蒙古制度,最後形成蒙漢兩元政治。元世祖雖然於爭奪汗位戰爭獲得蒙古大汗的汗位,並且最後成為中國皇帝,但由於汗位取得不合法與崇尚漢法,使得蒙古宗室不承認忽必烈的汗位,四大汗國有三國不奉忽必烈的命令,蒙古帝國完全解體。最後引發窩闊台系的海都出兵爭奪汗位,造成漠北地區動盪不安,史稱海都之亂。

早在元世祖在與阿里不哥作戰與整頓國內之際,因為無暇對付南宋,於是派郝經對南宋提出議和。當時南宋大權由謊稱擊退蒙古軍的賈似道掌握,然而賈似道由於畏懼謊言被擊破幽禁了郝經。南宋並於1262年拉攏山東漢人世侯李璮,發起李璮叛亂。元軍平定叛亂後,元世祖斷然廢止漢人世侯,以蒙古人直接管理地方事務,並且準備南征南宋。1268年元世祖發起元滅宋之戰,首先派劉整與阿朮率軍攻打襄陽,史稱襄樊之戰。1274年元軍攻下襄陽,宋將呂文煥投降,隨後中书丞相史天泽和枢密院使伯顏率軍順漢水南下長江,目標建康。1275年降將呂文煥率元水陸聯軍於芜湖擊潰贾似道的南宋水軍,史稱丁家洲之戰。隔年元军攻陷临安(今浙江杭州),谢太后與宋恭帝投降元軍。然而陸秀夫等擁立7歲的宋端宗在福州即位,文天祥、張世傑與陳宜中等大臣持續在江西、福建與廣東等地抗元。元軍陸續攻下華南各地,1278年南宋朝廷退至廣東崖山。隔年3月,張弘範在崖山海戰攻滅南宋海軍,陸秀夫帶着8岁的小皇帝宋幼主趙昺投海而死,南宋亡。元朝统一中国地區,结束自唐朝安史之乱以来520多年的分裂局面。

在此前后,元朝曾要求周边一些国家或地区(包括日本、安南、占城、缅甸、爪哇)臣服,加入元朝的朝贡关系,但遭到拒绝,元世祖於是出兵攻打这些国家,其中以入侵日本的元日战争最为著名,因為范文虎指挥不当與颱風來襲而失敗。由於元朝廷需要賞賜大量財寶予宗室貴族,加上開支繁重,財政日漸緊張,朝臣為了財政問題發生爭執,分裂成以许衡等漢人與漢化蒙古人为首的儒臣派與以阿合馬、盧世榮與桑哥等色目人與漢人为首的理財派。儒臣派認為元廷應該節省經費、減免稅收。理財派認為南人藏有大量財物,應沒收以解決朝廷的財政問題。由於元世祖信任阿合馬,設立尚書省解決財政問題。而儒臣則以受漢化更深的太子真金為核心與阿合馬抗衡。最後阿合馬被刺殺,太子真金也因為得病而死。然而元世祖不信任儒臣派,依舊任用理財派官員來解決財政問題,導致財政惡化。

1294年元世祖駕崩後,雖然太子真金早死,但是元世祖曾賜真金的三子鐵穆爾「皇太子寶」並且讓他鎮守和林。隨後鐵穆爾在库里尔台大会中獲得重臣伯颜與玉昔帖木儿等支持,打敗真金的長子甘麻剌與次子答剌麻八剌等繼位,即元成宗。元成宗主要恪守元世祖時期的成宪,任用其侄海山(答剌麻八剌之子)鎮守和林以平定西北海都之亂,並且下令停止征討日本與安南。在內政方面专力整顿国内政治,減免江南部分賦稅。然而,由於元成宗過度賞賜,入不敷出,使國庫資財匱乏。1307年正月,元成宗駕崩,由於太子德寿早逝,左丞相阿忽台擁護皇后卜魯罕與信奉伊斯蘭教的安西王阿難答監國,並有意讓阿難答稱帝。海山之弟愛育黎拔力八達與右丞相哈剌哈孫發動大都政變。他們斬殺阿忽台,控制大都局勢,擁護率軍南下的海山稱帝,即元武宗。皇后與阿難答被元武宗斬殺,其回回部下退入西域吐魯番地區。

元武宗因愛育黎拔力八達有功,冊封他為皇太弟(即未來的元仁宗),相約武宗系與仁宗系交替稱帝,即武仁之約。元武宗時期,加封孔子为“大成至圣文宣王”,并給予孔子的家族與弟子一些稱號。為了解決元成宗時期的财政危机,元武宗設置常平倉以平抑物價,下令印製至大銀鈔,然而反而使銀鈔嚴重贬值。此外他將中書省宣敕與用人權劃歸給尚書省。1311年元武宗因沉耽淫樂、酗酒過度而逝,由皇太弟愛育黎拔力八達繼位,是為元仁宗,這次是元朝首次和平繼承帝位。

西北地區方面,早在元世祖時期,由於他的大汗之位不受四大汗國的承認,使得當時窩闊臺汗海都有意奪回蒙古汗位。海都统辖叶密立(今新疆额敏东南)一带且與欽察汗國友好。元世祖為了避免在南征南宋時被海都背刺,遂扶持八剌獲得察合台汗位以牽制海都。然而在1268年,海都、八剌和欽察汗忙哥帖木兒以元世祖過度漢化、違背祖宗成法為由,在塔拉斯河招開庫里爾台大會結盟反元。他們以海都為盟主,共同瓜分中亞行省,聯合對抗元朝與伊兒汗國,史稱海都之亂。元世祖派伯顏北上平亂,海都與新任察合台汗篤哇採用游擊戰的方式迴避決戰。1287年海都聯軍夥同鎮守遼東的東道諸王乃顏與哈丹襲擊和林(今蒙古國哈尔和林),元世祖親率大軍擊敗之,派伯颜、玉昔帖木儿與李庭平定東北乃顏,主持西北军事。1289年海都再犯和林,最後其勢力被驱出阿尔泰山以西。而哈丹於遼東高麗一帶游擊,至1292年敗亡。

元成宗即位后,任命其侄海山(後繼位為元武宗)总领漠北诸军。1301年海都聯軍被海山和晉王甘麻剌擊潰,史稱鐵堅古山之役。海都於戰後去世,其子察八儿繼位,窩闊台汗國被篤哇掌控。1303年由於篤哇被欽察汗脫脫蒙哥擊潰,就與察八兒共同派使者向元廷請和,脫脫蒙哥也向元廷請和,而伊兒汗本來就支持元廷,至此四大汗國皆承認元朝的宗主地位,雙方廣設驛路,解除封禁。不久之後,窩闊臺汗國被察合台汗篤哇與元朝元武宗先後攻滅而亡,察八兒投降元朝。

元仁宗力图改变元武宗时造成的财政枯竭、政制混乱的局面,他推行「以儒治國」政策,並且減裁冗員、加強中央集權以整頓朝政。他曾令王约将《大学衍义》譯為蒙文,赐臣下说“治天下,此一书足矣。”并将《貞觀政要》和《資治通鑒》等書摘譯為蒙文,令蒙古人與色目人誦習。1312年元仁宗将其儒师王约特拜集贤大学士并将王约“兴科举”的建议“著为令甲”,至此恢复科举制度。本次科舉以程朱理学為考试的内容,史稱延祐復科,最後录取护都答儿、张起岩等56人为进士。他还倚重汉人文臣,处死蒙丞相脱虎脱等,排除朝中异己。財政方面,仁宗取消武宗的经济措施,並且於1314年在江浙、江西、河南等地查清地方田產,史稱延祐經理。任用床兀儿统军,擊敗察合台汗王也先不花以平定西北地區。然而元仁宗未能制止太后答己干预朝政,也無力制裁備受太后重用的重臣鐵木迭兒贪赃枉法。在繼承問題方面,元仁宗以王约輔助皇太子碩德八剌,並且聽從鐵木迭兒的建議,廢除武仁之約。他将元武宗長子周王和世琜外放鎮守云南、次子圖帖睦爾放逐海南島。同年冬天,元武宗舊臣皆感憤怒而擁護和世琜叛變,最後敗走漠北,依附察合台汗國。1320年元仁宗駕崩後,皇太子碩德八剌即位為元英宗。

元英宗繼續實行元仁宗的以儒治国、加强中央集权和官僚体制的政策,並于1323年下令编成并颁布元朝正式法典——《大元通制》,共2539条,他还下令拔除權臣鐵木迭兒在朝廷的势力。然而支持鐵木迭兒的蒙古與色目保守派厭惡英宗的新政,有意發動政變。1323年鐵木迭兒的義子鐵失趁英宗去上都避暑之際,在上都以南15公里的南坡地刺殺英宗及宰相拜住等人,史稱南坡之變,仁宗系自此未能再奪得皇位。晉王甘麻剌的長子,鎮守和林的也孫鐵木兒率兵南下,殺掉行刺元英宗的叛臣並稱帝,即元泰定帝。

泰定帝召回被放逐到海南島的武宗系圖帖睦爾為懷王。泰定帝於1328年七月崩於上都,丞相倒剌沙擁立七歲的阿速吉八為帝,是為元天順帝。而鎮守大都的燕帖木兒與伯顏擁立周王和世琜於漠北、懷王圖帖睦爾於江陵,同年圖帖睦爾先至大都繼位,是為元文宗。燕帖木兒率軍攻入上都,天順帝不知所終。隔年和世琜於漠北和林稱帝,即元明宗。元文宗放棄帝位,派燕帖木兒迎元明宗繼位,並且被立為皇太子。然而燕帖木兒毒死元明宗,元文宗復位,改元天曆,史稱天曆之變。

元文宗時期大兴文治,1329年設立了奎章閣學士院,掌進講經史之書,考察歷代治亂。又令所有勛貴大臣的子孫都要到奎章閣學習。於奎章閣下設藝文監,專門負責將儒家典籍譯成蒙古文字,以及校勘。同年下令編纂《元經世大典》,兩年後修成,為元朝一部重要的記述典章制度的巨著。然而丞相燕帖木儿自恃有功,玩弄朝廷,导致朝政更加腐败。1333年元文宗去世后,为洗刷毒死元明宗的罪行,遗诏立年仅七岁的明宗次子懿璘质班为帝,是为元宁宗。但元宁宗仅在位不到两个月即去世,不久后燕帖木儿也去世。元明宗的长子妥懽贴睦尔被文宗皇后卜答失里从静江(广西桂林)召回并立为帝,是为元惠宗,又称元顺帝。元朝在十三年內,換了八個皇帝。

元惠宗(元順帝)在位之初,1335年燕帖木兒的兒子唐其勢陰謀推翻,另立元文宗義子答剌海。幸右丞相伯顏粉碎叛亂,但屬於保守派的他掌握朝政,權力盛大。他禁止漢人參政並取消科舉,這些都与元惠宗發生衝突。1340年元惠宗在伯颜之侄脱脱的帮助下,终于废黜伯颜。脫脫為相與元惠宗親政前期時,元廷推行一系列改革措施如颁行《至正条格》法規,使得革新政治,社會矛盾緩和,史稱至正新政。1343年元惠宗下令修撰《辽史》、《金史》、《宋史》三史,由右丞相脱脱(后改由阿鲁图)主持,兩年後修成。然而元惠宗後期怠於政事,以至於在1350年發生天災人禍後引來民變。

元朝后期,特别是1340年代中后期至1350年代期间,乾旱、瘟疫與水災時常發生,且自宋朝奸臣杜充挖開黃河大堤以致奪淮入海後,黄河地区水患尤其严重,若以歷代中國王朝的次數作比較,秦漢平均8.8年一次,兩宋為3.5年,元代為 1.6 年,明、清兩代均為2.8年。与此同时,元廷財政體系崩潰,通貨膨漲嚴重,不断收取各种赋税,使百姓的生活更加艰苦,使得白蓮教逐漸流行,並成為對抗元廷的勢力。早在1325年就发生過河南赵丑厮、郭菩萨领导的武裝起事。1338年江西袁州(今江西宜春)彭和尚、周子旺等白莲教徒起义失败,彭和尚逃至淮西。1350年元廷下令變更钞法,鑄造“至正通寶”錢,並大量發行新“中統元寶交钞”,導致物價迅速上漲。隔年元惠宗派賈魯治黃河,欲归故道,動用民伕十五万,士兵二万。而官吏乘機敲詐勒索,造成不滿。白莲教首领韓山童、刘福通等人決定在5月率教眾起事,但事洩,韓山童被捕殺。劉福通再立韓山童之子韓林兒殺出重圍,指韩山童为宋徽宗八世孙,打出“复宋”旗号,以紅巾为标志。其後郭子興於安徽濠州起事,芝麻李等人占領徐州,此為東系紅巾軍。西系紅巾軍方面,彭瑩玉、鄒普勝與徐壽輝在湖北蘄州起事,國號天完。紅巾軍勢力遍佈河南江北、江南、兩湖與四川等地,還有非紅巾軍的张士诚等部的起事,民变揭开元朝灭亡的序幕。於元末民變期間,士人多不屑參加叛軍,叛軍也很少利用士人。

元廷派兵镇压各地紅巾軍,丞相脫脫親自率軍南下攻陷徐州芝麻李軍,一度壓制民變軍。然而脫脫在1354年南攻高邮张士诚軍之際,被元廷大臣彈劾而功虧一簣。徐寿辉部最後分裂成兩湖的陳友諒與四川的明玉珍。兩淮郭子興的部下朱元璋於1356年以南京為根據地開始擴充地盤;1363年與據有兩湖的陳友諒作戰,最後於鄱陽湖之戰獲得勝利;1365年占領兩湖後於同年冬東進攻打據有江蘇沿海的張士誠;1367年平定張士誠後,繼續南下壓制浙江的方國珍,至此江南無一人反抗朱元璋。另外,福建於1357年至1366年間發生色目軍亂,史稱亦思巴奚兵亂。與此同時,元朝在察罕帖木兒和李思齊等率領元軍反擊北方紅巾軍,1363年北方紅巾軍最後在安豐之役中敗給降元後的張士誠,劉福通戰死,韓林兒南下投奔朱元璋,隨後被殺。朱元璋統一江南後於1367年下令北伐,他派徐达、常遇春率明軍分別攻打山東與河南,並且封鎖潼關以防止關中元軍進援中原。明軍于1368年八月攻陷元大都,元惠宗北逃,史書稱此為元朝結束之年。然而元廷仍在上都,往後史書稱之為北元。而明廷认为元惠宗顺天明命,谥号为元顺帝。

元明之際有士人奉元朝為正朔,對元朝皆有故國之情,對於張士誠則有深厚的同情,而對於农民朱元璋則多表厭惡,當時江南士人,不論是否參加張吳政權,或參加朱明政權,乃至獨立人士,都相當懷念元朝。元明之際,由於元代的漢化色彩,漢人文士的華夷之辨觀念極為淡薄,而他們又不滿朱明所為,因而呈現強烈的遺民情結。朱明統治者憑藉紅巾武裝取得政權,在當時正統士大夫看來是“取天下非其道”,難逃僭偽之名,而且元末紅巾運動還帶有濃重宗教色彩,正統士人不僅視其為“賊”、“寇”,而且視之為“妖”。正如紅巾軍於汝陽起事,時人鄭元祐作詩稱“近者汝陽妖賊起,揮刀殺人丹汝水”,1359年,朱元璋部攻杭州,時人陳基記稱“妖寇犯杭”,洪武元年,明軍克大都,戴良作詩感慨“王氣幽州歇,妖氛國步屯”。

明初,不願仕官和不願效忠新朝廷的地主文人為了逃避徵辟而採取自殺、自殘、逃往漠北、 隱居深山等方法,誓不出仕(中國古代銓選,有「身言書判」四方面標準,身體有殘疾者不能任官)。為應對元遺民對明政權的鄙夷與漠視,朱元璋設立深受後人詬病的新刑罰,宣告「士大夫不為君用」律,大規模徵辟前朝遺老、搜羅岩穴隱士,並且殺害許多不願效忠明朝以及為新朝當官的學者:「率土之濱,莫非王臣。寰中士大夫不為君用,是自外其教者,誅其身而沒其家,不為之過」,導致「才能之士,數年來倖存者百無一二,今所任率迂儒俗吏」。

而居於中原的蒙古人則大量留于中原,在明代做官或參軍,史稱「達官」和「達軍」。

1368年元廷退回蒙古草原,元惠宗退至上都,隔年又至应昌。他继续使用“大元”国号,史稱北元。當時北方除了元惠宗據有漠南漠北,關中還有元將王保保駐守甘肅定西,此外元廷還領有东北地区與雲南地區。明太祖為了占領北方,採取兵分二路,各個擊破的方式,此即第一次北伐。元惠宗戰敗后于1370年在应昌去世,元昭宗即位后北逃至漠北和林。明将冯胜夺取了甘肃地区。然而元將王保保仍然在漠北多次与明将徐达等人作戰。明太祖曾多次寫信招降,但王保保從不理会,被朱元璋稱為「當世奇男子」。1378年四月,元昭宗去世,继位的元天元帝继续和明朝对抗,屢次侵犯明境。

至於北元領有的东北地区與雲南地區方面:1371年,元朝辽阳行省平章刘益降明,明朝占領辽宁南部。然而其餘东北地区仍由元朝太尉纳哈出控制,纳哈出屯兵二十万于金山(今辽宁省昌图金山堡以北辽河南岸一带),自持畜牧丰盛,与明军对峙了十几年,多次拒绝明朝的招抚。1387年冯胜、傅友德、蓝玉等人發動第五次北伐,目标是攻占纳哈出的金山。经过多次战争,1387年10月,纳哈出投降蓝玉,明朝占領东北地区。鎮守雲南的元朝梁王把匝剌瓦尔密,在元廷退回草原后仍然繼續忠效之。1371年明太祖派湯和等人領兵平定據有四川的明玉珍,並且勸降梁王未果。1381年12月,明军攻入雲南,1382年梁王逃离昆明並自杀,隨後明军攻克大理,明軍平定雲南地區。

明太祖為了徹底掃除北元勢力,於1388年5月命蓝玉率领明军十五万發動第六次北伐。明军横跨戈壁至捕鱼儿海(今中蒙边境之貝爾湖)擊潰元军,俘虜八萬餘人,元天元帝和他的长子天保奴逃走,但是幼子地保奴被明军擒住,至此北元國勢大衰。1388年元天元帝被阿里不哥后裔也速迭尔杀害(此後去年號,一說去國號),1402年鬼力赤殺元帝坤帖木兒後去國號,明人稱為鞑靼,北元亡。

元朝的前身為蒙古帝國,1206年元太祖成吉思汗成立時領有大漠南北與林木中地區(今貝加爾湖一帶)。經由成吉思汗等蒙古諸汗的經營,以及三次西征之後,蒙古帝國東達日本海與高麗、北達貝加爾湖、南與南宋對峙、西達東歐、黑海與伊拉克地區。成吉思汗時期分疆裂土給東道諸王與西道諸王,東道諸王是成吉思汗的弟弟,大多分封於塞北東部與東北地區,從屬性很高。西道諸王是成吉思汗的兒子,獨立性很好,其中分封長子朮赤於鹹海、裏海以北的欽察草原,後由拔都成立欽察汗國;封次子察合台於錫爾河以北的西遼舊地,史稱察合台汗國;三子窩闊台分封於乃蠻舊地,後由海都建立窩闊台汗國;蒙古本部由幼子拖雷獲得,後由蒙古大汗直轄。至於又稱漢地的華北地區、阿姆河與錫爾河之間的河中地區、伊朗地區與吐蕃由蒙古大汗直轄。1252年拖雷系的蒙哥即位後,命其弟旭烈兀西征西亞,最後建立伊兒汗國,與其他西道諸王合稱四大汗國。命忽必烈經營漢地、最後南滅大理。然而蒙哥於攻宋之役去世,隨後忽必烈與阿里不哥爭位使四大汗國紛紛不受蒙古大汗管制,蒙古帝國至此分裂。

元世祖忽必烈鑒於四大汗國不服於他,於是將西亞地區大汗直轄地割讓給旭烈兀(後來建立伊兒汗國),河中地區大汗直轄地割讓給察合台汗阿魯忽,以換取他們的支持。1279年元世祖在建立元朝後南滅南宋,一統中國地區,當時的疆域是:北到西伯利亚南部,越过贝加尔湖,南到南海,西南包括今西藏、云南,西北至今新疆东部,东北至外兴安岭、鄂霍次克海、日本海,包括库页岛,总面积超过1300萬平方千米。自灭亡南宋後雖然多次對日本、緬甸與爪哇等國有所衝突,然而疆域大体趋于稳定。1309年元武宗時期,元朝和察合台汗国先後攻滅窝阔台汗国,元朝取得窝阔台汗国東部部分领土,領土達1400萬平方公里(如果北方領土延伸至北冰洋,則為2200萬平方公里)。元朝的藩屬國有高麗、緬甸、安南、占城、爪哇及钦察汗国、察合台汗国、與伊儿汗国等國。北有漠北諸部、南有南洋諸國、西有四大汗國。其中有兩個直屬的藩屬國,即高麗王朝與緬甸蒲甘王朝,分別建立征東行省與緬中行省。

西北方面,1268年窩闊台汗國的海都意圖奪回汗位而聯合欽察汗國與察合台汗國反元,史稱海都之亂。直到1304年元成宗時期,元廷與這三大汗国达成和议,並與伊兒汗國一同承認元朝的宗主地位,成為元朝的藩属国,而元朝设立的行政机构(如行中书省和宣政院)也未包括这些领土。而且元成宗并赐伊儿汗国君主刻有“真命皇帝和顺万夷之宝”等汉文印玺,實質上也承認其獨立性。到1309年元武宗時期,元朝和察合台汗国先後攻滅窝阔台汗国,於元文宗年间编纂《经世大典》时,将钦察汗国、察合台汗国與伊儿汗国作为元朝的藩属国。

元朝行政區劃大致上承襲金朝與宋朝制度,然而有兩個不同之處:元朝時的路統轄的面積減少,一路僅轄二州;元廷在路上設有行省等中书省外派單位,最後行省取代路成為一級行政區,形成行省制,这是中国历史上首次正式在全国实行行省制度。元朝行政區劃由高至低依序分為行省、路、府、州與縣,另有等同行省的宣政院辖地、歸中書省直轄的「腹裏」以及等同州的土司。

腹裏是由中書省直轄的路府,宣政院(初名总制院)辖地主管吐蕃地區。行政首長以蒙古人為主、漢人為副。每省設置丞相一員,其下有平章、左右丞相即參知政事官,名稱大略與中書省相同。元代在行省以下各行政區均設置達魯花赤作(斷事官)為地方首長,並以漢人或當地土人為副,以利蒙古人控制地方區域。每路以達魯花赤為主、總管為副各一員。而府州縣均以達魯花赤為主、尹為副。州、縣均分上中下三等,中下州改州尹為知州。土司分有宣慰使、宣撫使與安撫使,於湖廣行省境內設置十五個安撫司,又於湖廣、四川行省分至四個軍。邊區的安撫司和軍,約當內地的下州,也置達魯花赤為主,其副為地方人士。縣以下基層行政區劃設有城關的坊里制與農村的村社制。坊里制於城內分若干片,名曰隅(如東西隅、西南隅之類)。隅下設坊,置坊官、坊司。坊下設里或社,置里正、社長;有的設巷而不設里,置巷長。村社制又稱村疃制度,於縣下設鄉,置鄉長,有的改設里正。鄉之下設都,置主首。都之下設村社,社設社長。

行中书省全称为“某某等处行中书省”,简称“某某行中书省”或“某某行省”,源自金朝的行尚书省。這是基於新征服之地的文化差異太大,所以中央政府就專門設置外派單位來管轄之。由於战争等需求,行省除了負責行政之外也負責軍事,最後逐渐形成一级行政区。早在蒙古時期就設有燕京(華北漢地)、别失八里(西遼等今新疆地區)、阿母河(中亞河中地區)等三断事官或行尚書省。元朝初年的行省管辖范围很大,改变也比较频繁,主要由中书省宰执带相衔临时到某一地区负责行政或征伐事务。1260年,元世祖於國內設置十路宣撫司,次年罷之。隔年改設十路宣慰司,漸成定制,並且設置陝西四川行省。往後直到滅宋為止,大多採行宣慰司與行省並行的制度。行省大多依據西夏、大理疆域與南宋新失之地設置,稱為「中書省臣出行省事」,滅南宋将全国分为中書省直轄的腹裏、宣政院辖地與十多個行中书省,並設置專司征討外國的行省。1321年元英宗時期共設置十一個行省(包含在藩屬國高麗設置的征東行省)。至元朝末年,行省增至十五個。

腹裏:由中书省直辖首都大都附近的中心之地,約今河北、山东、山西及内蒙古部分地区。

宣政院轄地:宣政院除了管理全國佛教事務外,尚管辖吐蕃地区軍政事務,約今青海、西藏。

行中書省:元世祖至元成宗時期設有十個,陕西、辽阳、甘肃、河南江北、四川、云南、湖广、江浙、江西、岭北行中书省。

另外甘肃行省之西的哈密力(今哈密地区)、北庭都元帅府(别失八里)與火州之地不属任何行省管轄。

征討行省分布:

征宋行省:如中统和至元前期的陕西四川行省、河东行省、北京行省、山东行省、西夏中兴行省、南京河南府等路行省、云南行省、平宋战争前后的荆湖行省、江淮行省等。滅宋後定型為一般的行中書省。

征外行省:於高麗設置征東行省(又称征日本行省)、於缅甸(蒲甘王朝)設置缅中行省(又称征缅行省)、於安南(陳朝)設置交趾行省(又称安南行省)、於占城設置占城行省(蒙越戰爭失敗後撤銷)。這些都是臨時性的建置,事畢即罷。只有征東行省,到元朝中期之後,穩定成高麗王的頭銜。行省丞相分别由該國國王或遠征軍主將擔任,自辟官屬,且財賦不入都省,视作藩属国,故與其他行省性質不同。

平亂行省:元末民變時,元廷爲便於鎮壓民變軍,先後於腹裏地區的濟寧(今山東巨野)、彰德(今河南安陽)、冀寧(今山西太原)、保定、真定(今河北正定)、大同等地置中書分省。又分別設立淮南江北行省(至正十二年設於扬州)、福建行省(至正十六年設於福州,後分省泉州、建寧)、山東行省(至正十七年)、廣西行省、膠東行省(至正二十三年)和福建江西行省(至正二十六年)。

另外元末民變的群雄也設置行省以便於統治,如天完之江南行省、汴梁行省、隴蜀行省、江西行省,韓宋之江南行省、益都行省,以及朱元璋所置江西行省、湖廣行省、江淮行省、江浙行省等。

元代行省之下的政区划分十分复杂且时常变化,简单时只存在行省、府州、县三级,复杂时则会出现行省、道(宣慰司)、路(总管府)、府州、县五级的情况。这跟元代“投下封邑”制度息息相关,具体政区分级可能有:

道(宣慰司):元代的道的直接来源即宋金的道路制度。中统三年李璮之乱爆发后,元廷为监察境内汉族世侯,开始仿照宋制设立临时且辖区不定的宣慰司,此时宣慰司多数兼行省相副衔。随着中国的统一,过于庞大的行省已经无法有效处理省内事务,且也有外重内轻之嫌,故至元十五年以后,对宣慰司进行大量的改革,裁撤了宣慰使相副衔并改任行省下属,使之成为辖区固定的行省分支机构及分管区域,其辖区划分也大致与宋金的道路级政区重合。同时由于行省首府附近的地域不设宣慰司,因此产生了直属省部的路州以及分属诸道的路州,但性质上这些都属于“直隶路州”。

直隶路州与封邑型政区:元代直隶于省部或宣慰司道的路州中存在大量的投下封邑型政区,这也是造成元代行政区划层级严重混乱的主要原因。基本上,直隶省部或宣慰司道的路州政区除少数冲要繁盛之地外,都是分封予汉族世侯和蒙古宗室的投下封邑。根据其规模户口的大小,可以分为总管府路、府、州三类,其关系则可参考吴澄所云“皇元因前代郡县之制损益之。郡之大者曰路。其次曰府若州……府若州,如古次国、小国。路设总管府,如古大国之为连率”。

总管府路:总管府路的设置与宣慰司道相似,也是源于宋金的道路制度,但目的性质不尽相同。蒙古初入主中原,以四大世侯为首的汉族地方军阀向蒙廷效忠,蒙廷则依仿金代制度,授予“某路都元帅”“某路都总管”的头衔,确认其在地方的高度世袭自治权,从而建立在汉地的政权机构,是为总管府路之滥觞,此时诸路规模建制与金代诸路相仿,四大世侯为首的有力总管其辖区更大。李璮之乱爆发后,元世祖为削弱地方割据势力,不但开始设置流官监察的宣慰司道,同时也对这类具有封邑性质的总管府路进行拆分,使一路仅辖三至四府州,但并没有改变总管府路封邑的政区性质,而是把它们转封给蒙古宗室,转封过程遵从“画境之制”,尽量使一王之封自成一路。灭宋后,置路以封诸侯的制度也在旧宋属地推行,这次的划分则更加零散,甚至到了“一州自成一路”的状况。

直隶府:除了总管府路的属府属州,一些府因为地处冲要或者以一府为封邑(主要在北方)而直隶于省部或宣慰司。少数人口众多地域广大的直隶散府(如南阳府、汝宁府、归德府等)经过后世的属区调整后更辖属州。直隶府与总管府路相比数量非常稀少,并非投下封邑的主要形态。

直隶州:与直隶府相似,极少数一些地处冲要或以一州为封邑的州(主要在北方)也直隶于省部或宣慰司。比较特殊的状况是,假如一些宗王的封地只有一县(比如蒙古开国功臣畏答儿之孙忽都虎郡王的封邑阳山县)的话,该县一般会升格为直隶州(升为桂阳州)。直隶州的数目比直隶府稍多,但仍远不及总管府路。

封邑型政区与其他直隶路州的最大区别在于达鲁花赤的设置,封邑型政区的达鲁花赤最早不由中央简任,而是由封君选任,作为封君在其封邑的代理人,行使最高决策权,保证封君在封邑的利益,而为了强化中央集权,一般上实际负责路州行政的总管、知府等为朝廷选任。

统县型政区:统县型政区即直接统领县级政区的中层政区,同样分为路(实质上为总管府路之首府即总府,总府所辖县在史料中多记述为直辖于路)、府、州三类,这些政区或作为投下封邑的一部分隶属于总管府路或部分直隶府(称为属府、属州),或作为独立的封邑直隶于省部或宣慰司道。其中属府的数量非常少,主要的统县型政区依然是属州。

元朝與蒙古帝國的皇位繼承異於中國歷代王朝,採取庫力台大會推舉的制度,由王室貴族公推大家的領袖。而元朝皇帝也是兼任蒙古帝國的可汗,由於元世祖的汗位沒有經過庫力台大會的認可,使得四大汗國紛紛不奉正朔,直到元成宗方恢復宗主關係。元世祖建立元朝後,有意立真金為太子,定傳子之局,可惜真金早死而使繼承問題又浮現。元朝而後常因皇太子早死或兄弟爭位而動盪不安,中期又有武仁之約的協定,武宗系與仁宗系交替繼承皇位,然而又因元仁宗廢除協定而再度混亂。元朝的繼承問題直到元惠宗方穩定,但也進入元朝末期。元朝政治制度與金朝一樣承襲宋朝制度,採取文武分權的制度,以中書省總理政務,樞密院掌管兵權。然而元朝的中書省已成為中央最高行政機關,元朝不設置門下省,尚書省時設時不設,僅元世祖時期與元武宗時期有設置,所以門下省與尚書省的權力皆交給中書省。中書省統領六部,主持全国政务,形成明清內閣制的先驅。其組織架構繼承南宋體制,宰相的稱呼共有中書令、司統率百官與總理政務等,常以皇太子兼任。下分左右丞相,中書令缺則總領中書事務。平章政事又居次,凡軍國重事,無不參決。副相方面有左右丞、參政等。六部共有吏部、户部、礼部、兵部、刑部與工部,內有尚書、侍郎。尚书省主要负责财政事务,不过时置时废。枢密院执掌军事,御史台负责督察,與宋朝制度大致相同,然而在地方設有行中書省、行樞密院與行御史台。此外又有掌管學校的集賢院、掌管御膳的宣徽院、掌管驛傳的通政院,其他還有太常禮儀院、太史院、太醫院與將作院,略前代的九寺諸監。最後新成立的是宣政院(初名总制院),负责佛教及吐蕃(今西藏)地区军政事务,這是前代所沒有的。

元朝在推行漢人的典章制度與維護蒙古舊法之間,時常發生衝突,並且分裂成守舊派與崇漢派。早在元太祖成吉思汗攻佔漢地後,有賴耶律楚材與木華黎推行漢法以維護其典章制度。當時近臣別迭建議將汉人驱赶並把中原变成大牧场以收取財富,遭到耶律楚材的反對,他認為可用徵稅的方式獲得財富,因此保留了漢地的典章制度。他積極改變蒙古軍以往「凡攻城邑,敵以矢石相加者,即為拒命,既克,必殺之。」的作風,努力興科祟儒、整頓吏治,實為漢法推行之祖。木華黎為了便於管理漢地,也於漢族四大世侯合作,逐漸鞏固了對河北、山西等地的治理。

後來管理漢地的元世祖忽必烈也積極推動漢法,任用了大批漢族幕僚和儒士等创设典章制度,如劉秉忠、許衡和姚樞等,並提出了「行漢法」的主張。積極推動了學習漢文的熱潮。如元世祖就非常熟悉漢文典籍、禮儀制度,並能用漢文創作詩歌,並且還以法律的形式規定,太子必須學習漢文。接受儒士元好問和張德輝提議的「儒教大宗師」稱號。忽必烈最後在大都建元稱帝,創建中國式的元朝,建立了一套以傳統中國中央集權作藍本的政治體制,例如设立了三省六部和司农司等一系列专司机构,使用中原的统治机构来统治人民,任劉秉忠等人的规划建立首都大都。然而,元世祖在李璮叛亂後,對漢人的信任下降。而四大汗國以及守舊派蒙古王室都不滿元世祖行漢法的舉動,或叛變或疏遠之。元世祖晚年也漸與儒臣疏遠,任用阿合馬、盧世榮與桑哥等色目人與漢人為首的理財派,漢法最後未成為一套完整的體系。後來的元仁宗、元英宗、元文宗與元惠宗等人更是可以純熟地運用漢文進行創作。一些入居中原的蒙古貴族,羨慕漢文化,還請了儒生當家庭教師教育子女。為了學習方便還翻譯了許多漢文典籍,諸如《通鑒節要》、《論語》、《孟子》、《大學》、《中庸》、《周禮》、《春秋》、《孝經》等。但崇漢派與守舊派時常發生衝突與政變,例如南坡之變等。

在人才選用方面,元朝雖然许多制度都沿袭了宋朝,但關於科舉,元朝前期並沒有常態化的定期舉辦科舉,因此高級官僚的錄用端看與元廷關係遠近而決定,主要採取世襲、恩蔭與推舉制的方式。此外尚有循胥吏(小公務員)昇進為官僚的方式,這與宋朝制度大異。宋朝官與吏的界限分明,胥吏大多以胥吏為終,然而元朝因為缺乏科舉取才,就以推舉或考試胥吏的方式晉升為官,這打破官吏屛障,使官吏成為上下的關係。科舉選材方面,窩闊台汗聽從耶律楚材建議,召集名儒講經於東宮,率大臣子弟聽講。又置“編修所”於燕京,“經籍所”於平陽,倡導學習漢族古代文化,又在1234年設“經書國子學”,以馮誌常為總教習,命侍臣子弟 18人入學,學習漢文化。並且於1238年以術忽德和劉中舉辦戊戌選試,此次科举取士录取4030人,並且建立儒户以保護士大夫。但最後仍廢除科舉,改採推舉制度,往後於1252年與1276年兩次共入選3890儒户。元世祖忽必烈即位後,正式設立了國子學,以河南許衡為集賢大學士兼國子祭酒,親擇蒙古子弟使教之,遍學儒家經典文史,培養統治人才。1289年元世祖下诏登記江南人口户籍,次年正式施行推舉制度,此次登記成为后来户计的依据。直到1313年,提倡漢化運動的元仁宗下诏恢复科举,元仁宗恢复科举,由程钜夫、李孟、许师敬拟定元朝科举制度。1314年八月在全国的17处考场,举行乡试,1315年二月和三月相继在大都举行会试和殿试(廷试),因为是在延祐年间举行的,史称“延祐復科”,本次科舉以程朱理學為考試的內容。榜分左右兩榜,官位相同,第一名從六品,第二名以下及第二甲,皆正七品,进士三甲以下都能授正八品官员,如1238年戊戌选试的状元杨奂,1315年的乙卯科左榜状元张起岩。元朝前後共舉行過16次,選舉蒙古、色目、漢人、南人進士約 1100余人。蒙古、色目人應舉者遠遠少於漢人、南人。然而為了保障蒙古人與色目人的名額,實行難度不同的「分榜取士」,並且給蒙古人與色目人保留了超過其應舉比例的名額,這也讓蒙古與色目子弟失去了學習漢族文化的積極性和進取精神。《元统元年进士录》的记载称四等人名额相等,各25人,但读书人总数确实南人、汉人要远多于蒙古、色目,因此也有破例,如延佑首科的录取名额给左榜的要远多于右榜。雖然是聊勝於無的科舉,但在形式上已經恢復,且持續坚持下去。原來放弃科举的士子重新獲得了入仕機會,因此漢族士大夫莫不對元廷忠心耿耿。在元朝滅亡之際,捨身殉國的就有很多是科舉出身者,可見科舉復辦對懷柔漢族士大夫有一定效果。

元朝時與各國外交往來頻繁,各地派遣的使節、傳教士、商旅等絡繹不絕,其中威尼斯商人尼可羅兄弟及其子馬可波羅成為得到元朝皇帝寵信,在元朝擔任外交專使的外國人。元廷曾要求周边一些国家或地区(包括日本、安南、占城、缅甸、爪哇)臣服,接受与元朝的朝贡关系,但遭到拒绝,故派遣军队进攻攻打这些国家或地区,其中以元日戰爭最为著名,也最惨烈。

東北方面有高麗王朝與日本鎌倉幕府。高麗王朝領有朝鮮半島,之後被崔氏政權統治,高麗王變成傀儡。高麗先後臣服於遼朝與金朝,蒙古興起後與高麗共同伐金,並約為兄弟之國。1225年蒙古要求高麗向其朝貢,蒙古使節抵達義州邊境時,被高麗所害,當時蒙古忙於西征,無暇征討。1231年窩闊臺汗派撒禮塔率兵入侵高麗,崔氏政權領袖崔瑀抵禦失敗,高麗首都松都(今開城)被攻陷,史稱高麗蒙古戰爭。蒙軍設置多位達魯花赤以監督高麗政事。隔年崔瑀殺死達魯花赤,擁護高麗王高麗高宗從松都遷往江華島,並且長期抗蒙,另外三別抄軍抵抗蒙軍至1273年。然而高麗朝廷分裂成反戰的文派,與抗蒙的崔氏政權。貴由、蒙哥時又四次討伐掠奪高麗地,1258年崔氏政權被顛覆後,高麗高宗遣子稱臣,正式成為蒙古的藩屬國。1283年元世祖為了討伐日本,於高麗國設置征東行省,高麗王為行省的左丞相,內政受蒙古人控制。高麗君主從忠烈王開始娶蒙古公主為妻,高麗君主繼承人按照約定,必須在元大都以蒙古人的方式長大成人後,方可回高麗。高麗成為元朝的藩屬國後,元世祖六次遣使者要求日本朝貢,均告失敗,於是發起元日戰爭。1274年元军發動第一次侵日戰爭,,日本史書稱為“文永之役”,元廷派三萬二千餘人東征日本,最後因為颱風侵襲而傷亡慘重。1281年七月,忽必烈又發動第二次侵日戰爭,日本史書稱為“弘安之役”,由范文虎、李庭率江南軍十餘萬人,到達次能、志賀二島,因日軍積極抵抗,且元軍又遇到颱風,最後再度慘敗。通常认为台风(日本人称之为“神风”)與元軍不擅水戰是造成失败的最大原因(另一方面高麗和南宋工匠故意製作式樣錯誤的戰船)。而後元世祖又準備第三次東征,因大臣勸阻,再加上出兵安南的緣故而罷。而後元世祖多次遣使均遭日本拒絕,通使关系一直未能建立,但是元朝與日本的经济和文化交流仍然十分繁盛,来元日本人以商人與禅僧最多。元廷令沿海官司通日本国人市舶,主要港口是庆元(今寧波)。

南洋諸國有安南(陳朝)、占城與爪哇(滿者伯夷)等國。安南國據有今越南北部,於五代北宋時期獨立於中華。蒙古大汗蒙哥於1257年派兀良哈台南攻安南,蒙越戰爭爆發。越南陳太宗被蒙軍擊敗,上表稱臣,蒙哥封為安南國王,而越南陳聖宗繼位後不願向元朝稱臣。當時在安南南方還有占城國,1282年占城國王因陀羅跋摩六世遣使朝貢,元世祖因此設置荊湖占城行中書省,以阿里海牙為該行省的平章政事。由於占城王扣留元使,元世祖藉此發兵分水陸攻打占城與安南。他以唆都率水軍由廣州渡海攻打占城。隔年蒙古水軍攻下占城國王據守的木城,占城國王因陀羅跋摩六世求和,但於蒙古退軍後殺使者。1284年元世祖再派鎮南王脫歡、阿里海牙與唆都率陸軍借道安南南征占城,被時任太上皇的陳聖宗反抗而爆發戰爭。元軍大舉入侵,占領安南國都。但陳聖宗、陳興道率領的陳軍積極抵抗,並且瘟疫四竄。最後元軍於1285年撤退,途中遭安南軍襲擊,損失過半。而後1288年又南征失敗,隨後安南請和。這場戰爭至元成宗才廢止,安南與占城相繼入貢元廷。當時南洋群島諸國,也多貢於元朝。有名的有馬蘭丹(今馬六甲)、蘇木都拉(今蘇門答臘)等。1292年元世祖命亦黑迷失、史弼與高兴率福建水軍南征爪哇滿者伯夷王國,並降其鄰國葛郎(爪哇島以東),但中計受突擊,戰敗而還,以後爪哇仍然派使朝貢。此外元世祖亦派使者招降琉求國,然使者僅至澎湖而返。

西南地區有大理國、吐蕃、緬甸(蒲甘王朝)、八百媳婦國(蘭納泰王國)與暹邏。大理源自唐朝的南詔,937年由段思平滅南詔建國,占有現今雲南地區,後由高昇泰等高氏政權掌控。1252年蒙哥汗命忽必烈與兀良合台自四川迂迴南滅大理,原大理國王段氏被任為大理世襲總管。吐蕃自晚唐就走向衰退,但其境內藏傳佛教(又被汉人贬稱为喇嘛教)日漸興盛,喇嘛的勢力超過贊普(吐蕃王)的地位。1247年窩闊台汗次子闊端召請喇嘛班智達來涼州,史稱涼州會盟,此後吐蕃喇嘛與蒙古大汗形成了布施關係(詳見元朝治藏歷史)。忽必烈南征大理時,分兵伐吐蕃,喇嘛班智達與贊普同時投降,吐蕃亡。元世祖封班智達的繼承人八思巴為「帝師」,兼任總制院(後改為宣政院)院使,取得了統治烏思藏地區的權力,使西藏統治者由贊普轉為喇嘛。緬甸為唐朝的驃國,宋朝以後稱緬,國內部落稱甸,所以又稱緬甸。元朝初期緬甸為蒲甘王朝,其王朝西併阿剌干(今孟加拉灣一帶),南併勃固(今仰光以北),並進占暹羅。元世祖派使招降不從,緬甸反派軍入侵雲南,元緬戰爭爆發,而後元兵又多次進攻緬甸。1283年元世祖派軍入侵緬甸,兩年後緬甸王請和。1287年緬甸內亂,元軍乘機進攻緬甸,蒲甘城破,緬甸成為元朝的藩屬,緬甸王那羅梯訶波帝失去王位,元廷建緬中行省,而後以蒲甘國王任行省左丞相,成為元朝傀儡。1368年撣族於緬甸東部阿瓦建立阿瓦王國,首領為阿散哥。孟族建都於馬達班,1369年遷都勃固,建立勃固王朝,二王國南北交戰。撣族阿散哥挾持緬甸王,使元成宗派元軍討伐,最後迫使阿散哥派使朝貢。蘭納泰王國(元人稱八百媳婦國)位於撣族東邊的金三角,曾聯合阿散哥抵抗元軍,元廷多次討伐未果,直到元泰定帝時才內附。暹羅地區原有素可泰王朝(元人稱暹國)、大城王國(元人稱羅斛)以及其他小國。暹國曾擴張其勢力於馬來半島,元成宗後遣使進貢八次。羅斛自元世祖末年就開始進貢,並於元末時期併吞暹國等小國,統一為暹羅國。

蒙古帝國的三次西征的同時,正值羅馬教皇提倡十字軍東征西亞的伊斯蘭國家以收復耶路撒冷。由於羅馬教皇急需外援以抗衡伊斯蘭教徒,而歐洲基督教國家剛剛經歷蒙古第二次西征,再加上東西交通十分便利,紛紛派使者東行了解這個東方大國。1245年羅馬教皇曾派柏朗嘉賓經欽察汗國到和林謁見貴由汗,返國著成《柏朗嘉賓蒙古行紀》。1253年法國國王路易九世派魯布魯克以傳教為名到和林進見蒙哥汗,返國著有《魯布魯克東行紀》。1316年義大利人鄂多立克經海路至元大都,參加了元泰定帝的宮廷慶典,回國著成《鄂多立克東遊錄》,範圍遠達西藏,對元大都及宮廷的描寫較細。最著名的是義大利探險家馬可波羅,他隨經商的父親、叔父於1275年到元朝進見元世祖,直至1291年才離去。他擔任元廷官吏,歷游元朝各地,其著寫的《馬可波羅遊記》對元朝進行多角度反映,吸引歐洲人東行中國。另外元朝與非洲地區諸國也有來往,汪大淵在1330年和1337年二度飄洋過海親身經歷的南洋和西洋二百多個地方的地理、風土、物產,最後著成《島夷誌略》,影響明代初期的鄭和下西洋。

元朝军队按照親疏關係分成蒙古军、探马赤军、汉军與新附军等四個等級。蒙古軍與探马赤军主要是骑兵。汉军、新附军大多为步军,也配有部分骑兵。水军编有水军万户府、水军千户所等。炮军由炮手和制炮工匠组成,编有炮手万户府、炮手千户所,设有炮手总管等。一部分侍卫亲军中,还专置弩军千户所,管领禁卫军中的弓箭手。

蒙古军是元朝軍隊的骨幹,主要由蒙古族組成。蒙古軍早在成吉思汗統一蒙古時即創立,平时分布在草原上驻牧,战时临时招集。採用兵民合一的萬户制,按十进制编组成十户、百户、千户。只要是十五歲至七十以內的人皆服兵役,其童子稍微年長者也組成「漸丁軍」。元朝時期在汉地和江南军户中签发丁男应役。探馬赤軍又名簽軍,随着战争的发展,统治者需要一支蒙古军队长期留守被征服地区,于是从蒙古各部中“签发”了部分士兵,组成专门用于镇戍的探马赤军。自1217年木華黎討伐金朝時建立,由弘吉剌、兀魯兀、忙兀、札剌亦兒及亦乞烈思五部組成,西征花剌子模後回族、維吾爾族與突厥族等族成為探馬赤軍的一部分。探馬赤軍精於火砲與西方的回回砲,攻城力強。「下馬則屯聚牧養,上馬則備戰」。

汉军是蒙古帝國占領漢地後發民為兵,主要由金朝女真與契丹降军、早期降蒙的南宋軍、漢地的地方漢族武裝勢力與签发漢地百姓等所組成。窩闊台汗於1229年收編金朝女真與契丹降軍,在漢地民戶中大規模簽發士兵,補充漢軍兵員,將蒙古軍的編製和官稱用於漢軍系統強。各漢軍萬戶統軍人數不等,「大者五、六萬,小者不下二、三萬」。漢軍有「舊軍」與「新軍」的區別。舊軍主要指敵國降軍和地方武裝勢力,新軍指從漢地百姓簽發的新兵。元世祖忽必烈即位後,蒙元帝國的統治重心由漠北草原移到了中原漢地。元世祖對軍隊體制進行改革,逐步建成中央宿衛軍和地方鎮戍軍兩大系統,確定了元軍的編製和隸屬關係,在元朝對外戰爭中,漢軍發揮了重要的作用。新附军主要是元朝南征南宋期間收邊的降軍,又被稱為新附漢軍、南軍等。新附軍內名號繁雜,而是元廷因士兵所具不同特點而起的名稱,如券軍、手號軍與鹽軍等等。估計當時新附軍的數量在二十萬人上下,元帝將新附軍分編到元軍的侍衛軍和鎮戌軍中;或以蒙古、漢人、南人建立新的軍府,管領新附軍人。每當有戰事發生,首先調發各軍中的新附軍出征,其餘則從事屯田和工役造作。經過多年的戰爭消耗和自然減員,新附軍數量日益減少,最後式微。

元朝的防衛分宿衛和鎮戍兩大系統。宿衛軍由怯薛和侍衛親軍構成,其中怯薛軍保留自成吉思汗創立的四怯薛番直宿衛,常額在萬人以上,元朝功臣博爾忽、博爾朮、木華黎、赤老溫或其後人充任怯薛長。在戰爭中,怯薛則是全軍的中堅力量,被稱之為「也客豁勒」(大中軍);侍卫亲军则是忽必烈在华北汉人世侯的建议下所置,在初期蒙制怯薛未形成战斗力之时负责宿卫之职以及与阿里不哥争夺权力。其后,侍衛親軍用於保衛大都,衛設都指揮史或率史,隸屬於樞密院。鎮戍軍由蒙古軍和探馬赤軍守衛靠近京畿的要地,華北、陝西、四川的蒙古軍、探馬赤軍由各地的都萬戶府(都元帥府)統領,隸屬於樞密院。南方以蒙古軍、漢軍、新附軍共同駐守,防禦重點是江淮地區,隸屬於各行省。鎮戍諸軍,有警時由行樞密院統領,平時日常事務歸行省,但調遣更防等重要軍務則歸屬樞密院決定。

元朝水軍原是為了元滅宋之戰而準備,1270年命劉整建造大量水軍。襄樊之戰時元朝水軍與陸軍協同包圍襄陽,攻下後降將呂文煥又率元水軍與河岸陸軍協同於丁家洲之戰擊潰南宋水軍精銳,至此領有全部長江水域。而後張弘範又率元朝水軍(平底船)渡海南下追擊南宋海軍,最後於崖山海戰包圍殲滅之,元朝水軍在滅宋之戰有重要的功能。元朝融合了南宋和阿拉伯航海技术,使海軍技術更加成熟,然而在對外戰事中,元日戰爭與元爪戰爭均以失敗結束,而且對日戰爭兩次均被颱風所毀,只有對占城的戰役獲勝而已。

早在蒙古時期,北方人口就不斷的南逃,總人數約占北方人口的十分之一,這種現象到惠宗时都還持續發生,元廷屢禁而不能止。在大蒙古國征服金朝期間在戰地进行了大规模屠杀和掠夺。随后的瘟疫与饥荒导致東亞地區大量人口消失,其中又以金朝的華北和南宋的川陕四路十分严重。这是导致“湖广填四川”移民运动发生的重大原因[需要更好来源]。

1234年3月9日金灭亡后,華北地區約有110万户與600万人,只有1208年的金朝人口5353万的13\%。蒙古宋战争期間,南宋境内因战争总计消灭了大约1500万人口,主要集中在川陕四路地区。1279年元军完全剿灭四川的抗元勢力後,在1280年的户口调查仅为9万余户與50万余人,只有1231年蒙古入侵川陕四路地区前的4\%。大蒙古國時期有過兩次戶口統計,先有1235年窝阔台汗推行的乙未籍户,獲得華北地區如燕京(今北京)、顺天(今河北保定)等三十六路的人口資料,後有1252年蒙哥汗完成的壬子籍户,顯示華北人口略有增加。1271年元世祖建国号为大元。雖然在元成宗到元惠宗至正初年期間政治動盪不安,尽管每年也成百上千次人民起义,但社會上基本處於安定狀態,經濟大體上也是呈現增长的狀態,這些都促使人口增长,大約在惠宗至正十年(1351年)達到高峰。元惠宗至正年間(1341年-1370年)全國發生多次大規模的災荒饑饉疾病和瘟疫,最終促使紅巾軍起義爆發。红巾军起义之后又造成人口大量減少。明太祖建國後論到:「前代革命之际,肆行屠戮,违天虐民,朕实不忍。」

元代戶口統計並不是准确,无法涵盖的人口包括逃戶、因土地兼併而蔭蔽的隱戶、流民以及私属人口等。朝廷不納入戶口統計的人口包括:嶺北等处行中书省、雲南等处行中书省、西南土司地區和宣政院轄地的居民;蒙古諸王、貴族、軍將的大量私屬人口(驅口、投下戶,怯憐口、打捕鷹房人戶);獨立於州縣以外的諸色戶計(軍戶、站戶、匠戶、民屯戶、釋、道、儒戶、游食者)等。現在歷史學者只能根據史書的原始數據與他們掌握的歷史資料的來推斷,所以差異甚大,僅作參考。人口逃亡的现象很严重,如1241年,忽都虎等元籍诸路民户1,004,656户,逃户即达280,746户,占全部人户的28%。另外,隨著民族關係日益密切,往來與雜居也相當普遍。從蒙金战争时期就陸續有大批漢人被迁往蒙古草原以及天山南北、遼陽等处行中书省與雲南等处行中书省各地;蒙古與色目官員、軍戶、商人等也大量移居中原內地;雲南地區居住的蒙古人約十萬人左右;大都、上都等政治城市及杭州、泉州、鎮江等商業城市都居住許多蒙古人、畏兀儿(維吾爾祖先)、穆斯林、黨項人、女真人與契丹人等,促成民族之間經濟文化的交流。

「四等人制」:有說法認為由於蒙古人與漢人的人數比例極不平均,元廷為了保護蒙古人地位,主張蒙古至上主義,推行蒙古人、色目人(包括西域各族和西夏人)、汉人(原金朝统治下的人民)、南人(南宋统治下的漢人)等四個階級的制度,但該制度并不见于官方文告及档案。有學者認為,元廷給蒙古人與色目人極大的權利,並讓汉人與南人負擔較大的賦稅與勞役,民族压迫和阶级压迫十分沉重。尽管学术界迄今并没有发现元代有把臣民明确划分为四等的专门法令,但元廷对于各民族的不平等态度却反映在一些政策和规定中,例如汉人打死怯薛需要偿命,而怯薛打死汉人只需「断罚出征,并全征烧埋银」(原文為怯薛歹蒙古人,怯薛歹為元代一特權階級)。此外汉人做官也往往只能做副貳(雖然實際上存在很多例外情況,終元一代朝廷仍任用不少漢人為官,如史天澤、贺惟一等)。

「九儒十丐」:有說法認為「九儒十丐」是元朝的定制,顯示出在蒙古統治下儒士在社會的下等地位。此「九儒十丐」的說法來自南宋遺民謝枋得,其〈送方伯載歸三山序〉云:「滑稽之雄,以儒為戲者曰:『我大元制典,人有十等:一官、二吏;先之者,貴之也,貴之者,謂有益於國也。七匠、八娼、九儒、十丐;後之者,賤之也,賤之者,謂無益於國也。』嗟乎卑哉!介乎娼之下,丐之上者,今之儒也。」及同樣是南宋遺民的鄭思肖〈大義略序〉曰:「韃法,一官、二吏、三僧、四道、五醫、六工、七獵、八民、九儒、十丐。」但因其政治立場,並不能完全盡信,或作為元朝儒士社會地位低下的佐證。中外史學界已有學者對元代儒士的地位問題進行過深入的研究,否定了元代儒人地位低落的說法。

元代经济呈现多元格局,经济活跃发达,大致上以农业为主,有学者认为其整體生產力雖然不如宋朝,但在生产技术、垦田面积、粮食产量、水利兴修以及棉花广泛种植等方面都取得了较大发展。蒙古人原来是游牧民族,草原时期以畜牧为主,经济单一,无所谓土地制度。蒙金战争时期,大臣耶律楚材建议保留汉人的农业生产,以提供財政上的收入来源,这个建议受到铁木真的采纳。窝阔台之后,为了巩固对汉地统治,实行了一些鼓励生产、安抚流亡的措施,农业生产逐漸恢复。特别是经济作物棉花的种植不断推广,棉花及棉纺织品在江南一带种植和运销都在南宋基础上有所增加。经济作物商品性生产的发展,就使当时基本上自给自足的农村经济,在某些方面渗入了商品货币经济关系。但是,由於元帝集中控制了大量的手工业工匠,经营日用工艺品的生产,官营手工业特别发达,对民间手工业则有一定的限制。

由于蒙古对商品交换依赖較大,同时受儒家轻商思想较少,故元朝比較提倡商業,使得商品经济十分繁荣,使其成为当时世界上相當富庶的国家。而元朝的首都大都,也成為當時闻名世界的商业中心。为了适应商品交换,元朝建立起世界上最早的完全的纸币流通制度,是中国历史上第一个完全以纸币作为流通货币的朝代,然而因濫發紙幣也造成通貨膨脹。商品交流也促进了元代交通业的发展,改善了陆路、漕运,内河与海路交通。

農業方面,宋真宗时推行的占城稻在元朝時已經推廣到全國各地。农业生产继续发展,1329年,南粮北运多达三百五十多万石,说明粮食生产的丰富。这一阶段,经济作物也有较大发展,茶叶、棉花與甘蔗是重要的经济作物。江南地区早在南宋時已盛產棉花,北方陕甘一带又从西域传来了新的棉种。1289年元廷设置了浙东、江东、江西、湖广、福建等省木棉提举司,年征木棉布十万匹。1296年复定江南夏税折征木棉等物,反映出棉花种植的普遍及棉纺织业的发达。元朝水利設施以華中、華南地區比較發達。元初曾设立了都水监和河渠司,专掌水利,逐步修复了前代的水利工程。陕西三白渠工程到元朝后期仍可溉田七万余顷。所修复的浙江海塘,对保护农业生产也起了较大作用。元朝農業技術繼承宋朝,南方人民曾采用了圩田、柜田、架田、涂田、沙田、梯田等扩大耕地的种植方法,對於生产工具又有改进。关于元朝的农具,在王祯的《农书》中有不少詳細的敘述。

元世祖為了清查土地徵收賦稅曾實行過土地所有者自報田地的經理法,由於未能確實執行,1314年元仁宗又派大臣往江浙、江西、河南三地實施經理法,但實施結果仍然弊端極多,人民紛起反抗,以至仁宗不得不下詔免三省自實田租二年,最後不了了之。

元朝土地仍可分为官田和私田两种。官田主要来自宋、金的官田,两朝皇亲国戚、权贵、豪右的土地,掠夺的民田,以及经过长期战乱所形成的无主荒地。元廷把所掌握的官田一部分作为屯田,一部分赏赐王公贵族和寺院僧侣,余下的则由政府直接招民耕种,收取地租。其屯田的數量極大,遍及全國,其中以河北、河南兩省最多。其中民屯是役使汉人屯垦收租,军屯则分给各军户,强迫相当于奴隶的“驱丁”耕种。私田是蒙古贵族和汉族地主的占地以及少量自耕农所有的田地。元朝以大量土地赏赐寺院,例如1316年元仁宗曾赐给上都开元寺江浙田二百顷、华严寺百顷。元朝也有一定数量的自耕农,然而地位很低下,生活十分困苦。

元朝的畜牧政策以开辟牧场,扩大牲畜的牧养繁殖為主,尤其是孳息马群。畜牧业发展趋势不稳定,由元世祖时的盛况渐渐趋向衰退,到了元惠宗時,畜牧业的衰败更为严重,其原因最大的是自然灾害。元朝完善了养马的管道,设立太仆寺、尚乘寺、群牧都转运司和买马制度等制度。元朝在全国设立了14个官马道,所有水草丰美的地方都用来牧放马群,自上都、大都以及玉你伯牙、折连怯呆儿,周回万里,无非牧地。元朝牧场广阔,西抵流沙,北际沙漠,东及辽海,凡属地气高寒,水甘草美,无非牧养之地。当时,大漠南北和西南地区的优良牧场,庐帐而居,随水草畜牧。江南和遼東諸處亦散滿了牧場,早已打破了國馬牧於北方,往年無飼於南者的界線。內地各郡縣亦有牧場。除作為官田者以外,這些牧場的部分地段往往由奪取民田而得。

牧場分為官牧場與私人牧場。官牧場是12世紀形成的大畜群所有制的高度發展形態,也是大汗和各級蒙古貴族的財產。大汗和貴族們通過戰爭掠奪,對所屬牧民徵收貢賦,收買和沒收所謂無主牲畜等方式進行大規模的畜牧業生產。元朝諸王分地都有王府的私有牧場,安西王忙哥剌,佔領大量田地進行牧馬,又擴占旁近世業民田30萬頃為牧場。雲南王忽哥赤的王府畜馬繁多,悉縱之郊,敗民禾稼,而牧人又在農家宿食,室無寧居。1331年以河間路清池、南皮縣牧地賜斡羅思駐冬。元世祖時,東平布衣趙天麟上《太平金鏡策》,云:今王公大人之家,或占民田近於千頃,不耕不稼,謂之草場,專放孳畜。可見,當時蒙古貴族的私人牧場所佔面積之大。

嶺北行省作為元朝皇室的祖宗根本之地,为了维护诸王、贵族的利益和保持国族的强盛,元帝对这个地区给予了特别的关注。畜牧业是岭北行省的主要经济生产部门,遇有自然灾害发生,元朝就从中原调拨大量粮食、布帛进行赈济,或赐银、钞,或购买羊马分给灾民;其灾民,也常由元廷发给资粮,遣送回居本部。元帝对诸王、公主、后妃、勋臣给予巨额赏赐,其目的在于巩固贵族、官僚集团之间的团结,以维持自己的皇权统治。皇帝对蒙古本土的巨额赏赐,无形中是对这一地区畜牧业生产的投资。

元朝手工業生產也有些進步,絲織業的發展以南方為主,長江下游的絹,在產量上居於首位,超過了黃河流域。元朝的加金絲織物稱為「納石矢」金錦,當時的織金錦包括兩大類:一類是用片金法織成的,用這種方法織成的金錦,金光奪目。另一類是用圓金法織成的,牢固耐用,但其金光色彩比較暗淡。棉纺织业到宋末元初起了变化,棉花由西北和东南两路迅速传入长江中下游平原和關中平原。加上元朝在五个省区设置了木棉提举司,“责民岁输木绵(棉)十万匹”,可见长江流域的棉布产量已相当可观。但当时由于工具简陋,技术低下,成品尚比较粗糙。1295年前后,婦女黄道婆把海南岛黎族的纺织技术带到松江府的乌泥泾,提升了纺织技術,被尊称为黄娘娘。

元朝的瓷器在宋代的基础上又有进步,著名的青花瓷就是元代的新产品。青花瓷器,造型优美,色彩清新,有很高的艺术价值。造船業十分發達,还有起碇用的轮车,并已经使用罗盘针导航。元朝的印刷技术,又比宋朝更有进步。活字印刷术不断改进,陆续发明了锡活字和木活字,并用来排印蒙文和汉文书籍。自1276年以来,已使用小块铜版铸印小型的蒙文和汉文印刷品,如纸币“至元通行宝钞”。套色版印刷术应用于刻书,如中兴路刊印的无闻和尚注《金刚经》。1298年王禎用木活字来印他所纂修的《大德旌德县志》,不到一月百部齐成,其效率很高。他又发明了转轮排字架,使用简单的机械,提高排字的效率。最後他總結成《造活字印书法》。

元朝行会组织还有应付官府需索、维护同业利益的作用,其組織的内部还更日趋周密。在元朝,“和雇”及“和买”,名义上是给价的,实际上却给价很少,常成为非法需索。虽然各行会多由豪商把持,对中小户进行剥削,但是由于官府科索繁重,同业需要共同来应付官府的需求,同时官府也要利用行会来控制手工业的各个行业。

元朝透過专卖政策控制盐、酒、茶、农具、竹木等一切日用必需品的贸易,影响国内商业的发展。可是元朝幅员广阔,交通发达,所以往往鼓勵对外贸易政策,因而终元之世对外贸易颇为繁盛。元朝的对外贸易主要采取官营政策,并禁止汉人往海外经商。但实际上私商入海贸易的仍然很多,政府始终无法禁绝。元代海外贸易输出入商品,大体上与宋代相同。但奴隶贸易却有相当规模,贩运进口的有“黑厮”和“高丽奴”。

在生产发展的基础上,物资交流频繁,从而促进了商业城市的发展。元朝時临安仍改名杭州,其繁荣并不因南宋覆灭而衰退多少。由于北方人纷纷南迁,城厢内外人口更加稠密,商业繁荣。杭州是江浙行省的省会,地位重要,水陆交通便利,驿站最多,不但是南方国内商业中心,也是对外贸易的重要港口之一。江浙行中书省居各行中书省征收的商税和酒醋课的第一位,城内中外商民荟萃,住有不少埃及人和突厥人,还有古印度等国富商所建的大厦。泉州在宋元時期是東方第一大港,貨物的運輸量十分巨大,泉州的稅收僅次於前朝首都杭州。然而在元朝末年色目軍爆發亦思巴奚兵亂,導致外僑大量撤離,對外貿易中斷而衰。大都(今北京)是元朝的首都,在原来中都城的东北方建立新城,规模宏大,是全国政治、军事中心,也是陆路对外贸易和国内商业中心。达官贵人、富商大贾多在此聚居,人口稠密,城厢内外街道纵横,商肆栉比鳞次,工商业很繁荣,是世界闻名的大城市。州县以上的城市,商业比较发达的还有:

长江下游和苏浙闽等地区的建康(南京)、平江(苏州)、扬州、镇江、吴江、吴兴、绍兴、衢州、福州等城市;

长江中游地区的荆南、沙市、汉阳、襄阳、黄池、太平州、江州、隆兴等城市;

长江上游川蜀地区的成都、叙州、遂宁等城市;

沿海对外贸易城市的广州、泉州、明州、秀州、温州和江阴等等。

元朝為了加強對經濟的統制,以使用紙幣為主,鑄造錢幣比其他朝代為少。1260年元世祖發行了以絲为本位的寶鈔與以白銀或金為本位的中統鈔(中統鈔没有设定流通期限),鈔幣持有者可以按照法令比價兌換銀或金,虽然其后曾一度废除,但持续使用到元朝末期,成为元朝货币的核心的纸币。全國各路都設有兌換的機關——「平準庫」。兌換基金充足,准許兌現,兑换的时候征收两到三分的手续费(工墨鈔)。1276年由於元廷大肆搜括,增發紙幣,並將各路準備金銀運往大都,引起物價上漲,紙鈔貶值。1280年,紙幣貶值成為原來的十分之一。1287年物價已經「相去幾十餘倍」了。為了穩定物價,元廷發行「至元寶鈔」和中統鈔並行。1350年元惠宗又發行「至正寶鈔」,發行不久,貶值嚴重,物價暴漲。事實上,民間的日常交易、借貸、商品標價等多有用銀的。這時使用的白銀,主要是銀錠和元寶。

元代的賦稅依舊包括田賦、開採礦產的歲課、鹽稅等。但由於元代商業發達,商稅亦成為了政府的重要收入之一

關於元朝的田賦,《元史·食貨志一》說:「元之取民,大率以唐為法。其取於內郡者曰丁稅,曰地稅,此仿唐之租庸調也;取於江南者曰秋稅,曰夏稅,此仿唐之兩稅也。」這段話雖然並不確切,但至少說明了南北田賦制度的差異。中原田賦的徵收大概始於耶律楚材輔政以後。在這之前蒙古帝國根本沒有賦稅之制。元朝行於江南的田賦制度基本上沿用了宋代的兩稅制。

元朝人民還有一項很沉重的財政負擔,即科差,是徭役向賦稅轉化的一種形式。

元朝統治中原,對中原傳統文化的影響大過對社會經濟的影響。像遼朝、金朝與西夏等征服王朝,他們為了提升本國文化,積極的吸收中華文化,進而逐漸漢化,然而蒙元對漢文化卻不甚積極。他們主要是為了維護本身文化,同時採用西亞文化與漢文化,並且提倡蒙古至上主義,來防止被漢化。例如他們提倡藏傳佛教高過於中原的佛教與道教,在政治上大量使用色目人,儒者的地位下降以及長時間沒有舉辦科舉。由於士大夫文化式微,意味宋朝的傳統社會秩序已經崩潰。這使得在士大夫文化低下,屬於中下層的的庶民文化迅速的抬頭。這個現象在政治方面是重用胥吏,在藝術與文學方面則是發展以庶民為對象的戲劇與藝能,其中以元曲最為興盛。

元朝的思想上也是兼收並用的,他們對各種思想幾乎一視同仁,都加以承認與提倡,「三教九流,莫不崇奉」。然而元廷在一定程度上尊重儒學,特別是於宋朝形成的理學,更是尊為官學,使得理學得以北傳。元仁宗初年恢復科舉,史稱延祐復科,在「明經」、「經疑」和「經義」的考試都規定用南宋儒者朱熹等人的注釋,影響後來明朝的科舉考試皆採用朱熹注釋。理學在元朝還有一些變化,南宋時期即有調和程朱理學的朱熹與心學的陸九齡等兩家學派的思想,元代理學家大多捨棄兩派其短而綜匯所長,最後「合會朱陸」成為元代理學的重要特點。當時有名的理學家有黃震、許衡與劉因與調和朱陸學的吳澄、鄭玉與趙偕。朱學的後繼者為了配合元帝的需求,更注重在程朱理學的倫理道德學說,其道德蒙昧主義的特徵日趨明顯。從而把注意力由學問思變的道問學轉向對道德實踐的尊德性的重視,這也促成朱、陸思想的合流。元代理學的發展,也為明朝朱學與陽明心學的崛起提供某些思想的開端。

江南統一後,元朝崇尚儒學的政策有新發展,漢蒙官員上書建議興舉和重視學校,於是元政府在推廣有關儒學教育政策的同時,亦更加注意優待和勉勵儒學。從元世祖到元世宗時期,元朝的重視、勉勵學改的政策已經完備。元成宗以後,這些政策基本上得到歷代皇帝的實行。例如為了維護儒學的正常運行,元世祖於至元二十五年下聖旨:「(江淮等處)仍禁約使臣人等勿得於廟學安下,非禮騷擾」,此後元政府兩次重申這一禁令,對元朝儒學教育的正常運作起到了保護作用。另外,元朝亦實行宋朝以來的學田政策,允許學校支配學田收入。元朝政府還將儒學推廣至邊遠地區,在雲南、兩廣、海南、西部地區如原西夏政權控制的範圍和原宋朝和吐蕃的邊境地區、北部和東北地區(岭北行省和遼陽行省)建立、推廣和發展儒學。元朝的統一對儒學教育向中國邊遠地區的擴散作出了推動作用,並且取得了明顯的成績。

由於元朝由蒙古人所統治,漢族士大夫基於異族統治的考量,在蒙元初期大多分成合作派與抵抗派。合作一派是華北儒者如耶律楚材、楊奐、郝經與許衡等人。他們主張與蒙古統治者和平共存,認為華、夷並非固定不變,如果夷而進於「中國」,則「中國」之。如果蒙古統治者有德行,也可以完全入主中原。他們提倡安定社會,保護百姓,將中華的典章制度帶進蒙元,以教感化蒙古人。另一派是江南南宋遺民的儒者如謝訪、鄭思肖、王應麟、胡三省、鄧牧、馬端臨等人。他們緬懷南宋故國,為了消極抵抗元廷,採取隱遁鄉里,終生不願意出仕的方式。並且以著述書籍為業,將思想化為書中主旨。到元朝後期,由於元仁宗實行延祐復科,恢復科舉,及第者都感謝天子的恩寵,紛紛願意為元廷解憂。元朝後期國勢大墬,政治腐敗、財政困難,使得當時士大夫如趙天麟、鄭介夫、張養皓與劉基等人紛紛提出各種政治主張,或從弊端中總結經驗教訓。他們大多提倡勤政愛民、廉潔公正、任用賢才等措施。元末民變的爆發使得南方有不少士大夫出於衛身、保鄉、勤王之目的,紛紛組織義兵護國,有些士大夫甚至捨身殉國。在明朝建立後,部分元朝遺老紛紛歸隱不出。

元朝文學以元曲与小说為主,對於史學研究也十分興盛。相對的元朝的詩詞成就较少,内容比較贫乏,但文以虞集為長,詩以劉因為著。明朝王世贞说“元无文”,但叙事文學如戲曲、小说第一次有主導地位。元朝使華北誕生元曲,江南則出現以浙江為中心的文人階層,孕育出《三国演义》和《水浒传》等長篇小說,自由奔放的文人如杨维桢、倪瓚等人,在城市發放出市民文化的花朵。

元曲分成散曲與雜劇,散曲具有詩獨立生命,雜劇則具有戲劇的獨立生命。當時城市經濟興盛,元廷不重視中国文學與科舉,當時社會提倡歌舞戲曲作為大眾的娛樂品,這些都使宋、金以來的戲曲昇華為元曲。散曲是元代的新體詩,也是元代一種新的韻文形式,以抒情為主,主要給舞台上清唱的流行歌曲,可以單獨唱也可以融入歌劇內,與唐宋詩詞關係密切。;雜劇是元代的歌劇,產生於金末元初,發展和興盛於元代至元大德年間。根據《太和正音譜》中所記,大約有五百三十五本,創作十分巨大而輝煌。元朝后期,雜劇創作中心逐步南移,加強與溫州發揚的南戲的交流,到元末成為傳奇,明清時發展出崑劇和粵劇。当时散曲四大名家有关汉卿、马致远、张可久与乔吉,有名的《南呂‧一枝花》(《不伏老》)反映作者樂觀和頑強精神;《恁闌人》(《江夜》)追求文字技巧,脫離散曲特有風格;描寫景物的《水仙子》(《重觀瀑布》)雅俗兼備,以出奇制勝;其中描寫自然景物的曲子《天淨沙》(《秋思》)刻劃出一幅秋郊夕照圖,情景交融,色彩鮮明,被稱為「秋思之祖」。雜劇五大名家除了關漢卿與馬致遠之外,還有白樸、王實甫與鄭光祖,有名的作品有《竇娥冤》、《拜月亭》、《漢宮秋》、《梧桐雨》、《西廂記》與《倩女離魂》,主要表現社會與生活情況、歌頌歷史人物與事件,強調人物的情感。元曲的興盛,最後成为与漢賦、唐诗、宋词并称的中国优秀文学遗产。

元朝长篇小说源自戲曲說白的平話,這些話本最後寫成書的即是小說,以《三国演义》和《水浒传》最有名,與明朝的《西遊記》、清朝的《红楼梦 》合稱中國古典四大文学名著。《三国演义》的作者是羅貫中,敘述三國時期曹操、劉備與諸葛亮等人物,小說通篇精巧敘述謀略,雖與史實多有出入,仍譽之「中國謀略全書」;《水浒传》一般認為是施耐庵所著,而羅貫中負責整理。其內容講述梁山泊以宋江為首的綠林好漢,由被迫落草,發展壯大,直至受到朝廷招安。現存宋元平話共約八種,包括《大唐三藏取經詩話》。

元代的歷史研究也十分興盛。胡三省潛心研究歷史巨著《資治通鑑》,1286年《資治通鑒音注》全部成編,公認是對《資治通鑑》的注釋最佳者。馬端臨在歷史文獻的收集和整理方面有很深的造詣,著有《文獻通考》,記載上古至宋寧宗嘉定末年曆代典章制度的政書,十通之一。蘇天爵、歐陽玄、虞集與趙世延一同編寫的《经世大典》。脫脫主編,由歐陽玄等人編寫《遼史》、《宋史》與《金史》。元朝還有記述大蒙古國立國至窩闊台汗時期的《蒙古秘史》。

元朝的文字與語言方面,一般是通用蒙古語與漢語,然而一些说法认为入聲字最早被認為在元朝官話消失。文字通用漢文與蒙古的八思巴字。八思巴文是元世祖時由國師八思巴根據當時的吐蕃文字而制定的一種文字,用以取代標音不夠準確的粟特语蒙古文字。然而此時橫跨歐亞的蒙古帝國已經析為元朝和四大汗国:蒙古欽察汗國、察合台汗国、窝阔台汗国、伊儿汗国,因此八思巴文一直只有元朝採用,並主要用作為漢字標音符號。元朝滅亡後,仍然推行於北元,到了16世纪末期,蒙古高原的蒙古人受其他蒙古民族同化,轉而重新採用蒙古文字。

元朝與四大汗国(欽察汗國、察合台汗国、窝阔台汗国、伊儿汗国)橫跨歐亞大陸,幅員遼闊,其疆土內的种族也十分繁多,這些都使得元朝的宗教呈現多元化,各類佛教(含漢傳佛教與藏傳佛教)、道教、白蓮教等都取得了较大的发展;東西方的商旅、教士亦来往频繁,自西方傳來的伊斯兰教、基督教(含景教和天主教)與猶太教的影響力也逐漸增加。由于元朝对境内各种宗教基本採取自由放任的態度,對信仰宗教的問題採取兼容並包的政策,甚且優容禮遇之,這種环境自然有利於宗教的傳播與發展。元朝僧人有免税免役特权,致使一些不法之徒投机为僧,甚至干预诉讼,横行乡里,成为元代的一个社会问题。不过,元世祖曾在禮節上歧視伊斯蘭教,例如不尊重其宰羊方法,伊斯蘭教徒被逼吃死肉,此法令亦適用於基督教徒。元朝对宗教管制较为宽松,使得民间如白莲教、明教等藉此建立秘密组织,进行抗元起事。

各類佛教中以藏传佛教最為興盛. 藏传佛教約唐中期自吐蕃傳入唐朝,專以祈禱禁咒為事。漢傳佛教在唐武宗时遭受打擊,宋朝時只剩禪宗慢慢恢復,然敵不過道教與理學。藏传佛教中,薩迦派(花教)自窩闊台汗至元世祖期間逐漸获得蒙元朝廷的尊重。忽必烈早在攻擊吐蕃時即於薩迦派的喇嘛扮底達講和,而後扮底達的繼承人八思巴被元世祖奉為國師(後升為帝師),賜玉印,任中原法王,命統天下佛教,並兼任總制院(後改名為宣政院)使來管理吐蕃(今西藏)事務,這是以宗教領袖統治西藏地區之始。八思巴還為元朝建立八思巴文。藏传佛教在元朝皇帝的推崇下,在社會與政治上均有極高的地位。諸位元朝皇帝均受藏传佛教的戒律,藏传佛教也逐漸推廣到蒙古各部。然而皇室用於佛事之錢要占國家財政支出一半(皇帝即位前要灌顶),寺院也擁有龐大的產業,部分喇嘛也驕縱不法,危害社會。例如元世祖時,江南佛教總統喇嘛楊璉真珈喜好掘墓,曾挖掘宋朝諸陵與諸大臣墳墓百餘所;包庇平民不輸租賦者,達兩萬三千戶,其餘如奪人產業,姦污婦女等類之事,更為常見。

道教自宋朝即十分興盛,金朝與南宋時期即有全真教、太一教與大道教三派。全真教由王喆創立,主張修孝僅存一之德,然後學道。成吉思汗於西征時邀请全真教道士丘处机西行中亞,十分禮遇他,並且他掌管天下道教。丘處機後來與其弟子李志常寫成《長春真人西遊記》一書,具有重要的史料價值。大道教主張苦節危行,不妄取於人,不苟奢於自,從創教教主劉德仁五傳至酈希誠,被蒙哥冊封為太玄真人,掌管教務。太一教以傳授太一三元法籙之術為主,從創教教主蕭抱珍五傳至李居壽時,元世祖興建太一宮,並讓他居之,獲得太一掌教宗師印。然而元朝以藏传佛教為國教,元世祖曾命燒去一些「捏合不實」的道經如《老子化胡經》等,然而仍然冊封各派宗師以安撫之。

元朝的基督教(即天主教)稱為也里可溫教,唐朝時基督教的分支景教(聶斯脫里派)因唐武宗的禁止而式微,到元朝時基督教再度傳入中國。當蒙古人數度西征時,歐洲频繁发动数次十字軍東征,征伐西亞的伊斯蘭教徒,因此歐洲人有意和蒙古結盟,共抗伊斯蘭教徒。貴由汗時,羅馬教皇曾派使者到和林見貴由汗;元世祖時教廷又派方濟各會教主由海道抵大都,元世祖同意其傳教,而景教教徒分布在揚州、杭州、鎮江與泉州等地,最後分布到華北、西北與西南。然而基督教時常與佛道兩教衝突,方聶兩派也自相牽制。元朝晚期,教皇有意派主教來華整頓教務,然而主事者漠不關心,元朝滅亡後東西交通斷絕,基督教再度式微。猶太教稱為術忽或主吾,犹太人大多定居開封、杭州、大都與和林等城市。由於猶太商人擅長理財,元廷視為財政來源之一。

元朝的伊斯兰教(又稱回教)稱為木速蠻教,也是於唐武宗後式微於中國,而後流行於西域中亞各國如畏吾兒、花剌子模等國。成吉思汗西征時降服許多西域回教國家,使得伊斯兰教徒仕於蒙古朝廷甚多。由於色目人(即西域各族)擅長理財,元世祖統一中國後更任用色目人,給予極大的權力。這些都使得伊斯蘭教盛行於中國西部、雲南地區等,部分色目商人也有定居於沿海廣州、泉州、杭州與揚州等地區,漸漸形成大分散、小集中的特色,幾乎覆蓋全國。1357年至1366年間更在福建發生色目軍亂,史稱亦思巴奚兵亂。當時蒙古王公大臣也有信奉伊斯蘭教,其中安西王阿難答更是虔誠的伊斯蘭教徒。他於元成宗駕崩後擔任監國,並且很有機會繼承為皇帝。如果他擔任皇帝,可能使元朝國教改為伊斯蘭教。

元朝經濟發達,城市文化興起,又因為交通發達,東西文化交流,使得元朝藝術呈現多元化。繪畫方面,文人畫成為主流,著重個人及書法表現,風格與元代強調裝飾的宮廷繪畫迥然不同。元初趙孟頫、高克恭等人提倡復古,回歸唐朝和北宋的風格,並且將書法入畫,創造出重氣韻、輕格律,注重主觀抒情的元畫風格。元朝中晚期以黃公望、王蒙、倪瓚、吳鎮等元代四大家 為主,其中又以黄公望为冠。他們寄託清高人格的理念於繪畫上,以隱逸山水與梅、蘭、竹、菊、松、石等為象徵。黃公望創始「淺降山水」,先以水墨鉤勒皴染為基礎,加上以赭石為主色的淡彩山水畫。由於元人以較乾的筆法在紙上作畫,這不同於宋人繪於絹上。山水畫除了皴法以外,增多擦的效果,猶如中國書法一樣。為了使畫面的上方可以題上詩句,所以故意留出一角,題上自己作的詩句,使詩、書、畫三者合成一體,影響明清國畫至今。元代的花鳥,以錢選最為有名,他學習宋人趙昌的畫風,具有宋人厚重典雅的趣味。其他如趙孟、趙雍、陳琳與劉貫道等均以兼善花鳥出名。

元朝書法的核心人物是趙孟頫,他的書法深受東晉書法家王羲之的影響,所創立的楷書趙體與唐楷之歐體、顏體與柳體並稱四體,成為後代規摹的主要書體,表現為“溫潤閒雅”“秀研飄逸“的風格面貌。審美觀趨向飄逸的超然之態獲得一種精神解脫有一定聯繫。鲜于枢与赵孟頫齐名,但影响略小,尤其擅长行、草书。与他们同时代的书法家邓文原则擅长章草,是研习这种古书体不多见的名家之一。康里巙巙稍晚于赵孟頫,也以草书名世,是少数民族书法家的代表人物。  

元朝的工艺美术十分发达,在传统的工艺美术上吸收了藏族等其他民族文化,对元代工艺美术带来了新的发展。官办手工业人材荟萃,技艺精湛,生产出了大量高级手工艺品和消费品,最明显的如陶瓷工艺、织绣工艺等。元朝瓷器及漆器等實用藝術常有創新。元朝是景德镇真正驰名的时期,最著名的瓷器即为青花瓷和釉里红。受到中東文化影響,瓷器有豐富的藍白色裝飾,中東商人也會訂製大量的龍泉青瓷。元朝也完成許多佛教雕刻,其中,密宗多手佛像顯示蒙古人對尼泊爾、西藏地區藏传佛教藝術的愛好。銀器工藝家朱碧山知名的銀器的雕造技術也是在此時發展。此外元代也製作生產雕漆工藝品。

由於元朝朝廷與社會提倡思想多元,經濟發達提供可靠的物質保證,交通發達與中外交往空前活躍又為吸收世界各地科技創造條件,使得科学技术有很高的成就,主要表現在天文歷法、數學、農牧業、醫藥學與地理學等方面。中國古代的發明印刷術及火藥等出現了印刷活字盤與火銃等技術,西傳西方後促進歐洲國家的進步。波斯、阿拉伯素稱發達的天文、醫學等成就,也在元朝被大量傳至中國。由於東西貿易的興旺,西域的玉石、紡織品、食品及珍禽異獸等也源源不斷輸入中國。中外的科技交流,促進了各自的科技進步,元朝正好為這種交流提供了比以前歷代都優越的條件。

元朝在天文歷法方面十分發達,元世祖邀請阿拉伯的天文學家來華,吸收了阿拉伯天文學的技術,並且先後在上都、大都、登封等處興建天文臺與回回司天臺,設立了遠達極北南海的27處天文觀測站,在測定黃道和恒星觀測方面取得了遠超前代的突出成就。元朝有名的天文學家有郭守敬、王恂、耶律楚材、紮馬魯丁等人。耶律楚材曾編訂有《西征庚午元歷》,1267年扎马鲁丁撰进《万年历》,郭守敬等人修改曆法,以近世截元法主持編訂了《授時歷》,《授時歷》於1280年頒行,延用了400多年,是人類歷法史上的一大進步。扎马鲁丁與後來的郭守敬研制出了簡儀、仰儀、圭表、景符、闚幾、正方案、候極儀、立運儀、證理儀、定時儀、日月食儀等十幾種天文儀器,當時在天文台里工作的还有阿拉伯天文学家可马剌丁、苫思丁等人。回回司天台一直存在到元末明初,仍由回回司天监黑的儿、阿都剌、司天监丞迭里月实等修定历数。元朝數學湧現出了一批傑出數學家及其著作。如李冶及其《測圓海鏡》、《益古演段》;朱世傑及其《算學啟蒙》、《四元玉鑒》;李冶提出的天元術(即立方程的方法)及朱世傑提出的四元術(即多元高次聯立方程的解法),是具有世界性影響的新成就。算盤在元代也初具規模。

元代的農業技術主要可見於《農桑輯要》、《王禎農書》與《農桑衣食撮要》等三部書。《農桑輯要》由元廷主持編纂,全書分七卷十篇,對元及其以前的作物栽培、牲畜飼養做了總結,並保存了大量古農書資料,對推廣農牧業技術,指導農牧業生產有重要作用。《農書》為著名農學家王禎所著,,全書分“農桑通訣”、“百谷譜”、“農器圖譜”三大部分,总结了古代的农业生产经验,又介绍了当时的新技术,是继北魏贾思勰的《齐民要术》之后又一部重要的农业科学著作。王禎認為要不違農時、適時播種、因地制宜、及時施肥、興修水利才是取得農業豐收的保證,其中關於棉桑種植具有現實意義。《農桑衣食撮要》為魯明善所著,此書重在實用,按月記載農事活動,特別還涉及到遊牧生產,可補《農桑輯要》及其它古農書之不足。

醫藥學方面,史稱金元四大家中有兩位生活在蒙元時期。李杲師承張元素,強調補脾胃,創立了“補土派”,著有《脾胃論》、《傷寒會要》等。朱震亨拜羅知悌為師,發展劉完素火熱學說,主張以補陰為主,多用滋陰降火之劑,後人稱其為“滋陰派”,著有《格致余論》、《局方發揮》、《傷寒辨疑》等書。外科骨傷科方面成就更為突出,危亦林在麻醉與骨折復位手術上有創新。薩德彌實的《瑞竹堂經驗方》很註意北方的寒冷氣候及蒙古族遊牧生活實際,有不少治療骨傷及風寒濕痹的方劑,有的時至今日仍為醫家所使用。元廷太醫忽思慧的《飲膳正要》反映了當時國內各少數民族及中外人民的飲食文化交流。

地理學方面《元一統誌》的編纂、河源的探索、《輿地圖》的問世及大批遊記類著作的出版是其主要成就。《元一統誌》由政府主持,紮馬魯丁、虞應龍具體負責。該書對全國各路府州縣的建置沿革、城郭鄉鎮、山川裏至、土產風俗、古跡人物均有詳細描述,具有較高史料價值。1280年元世祖命女真人都實探求黃河河源,認為星宿海(火敦腦兒)即河源,比較接近實際。潘昂霄還據此撰成《河源誌》。道士朱思本考察了今華北、華東、中南等廣大地區地理形勢,參閱《元一統誌》等地理學著作,以“計裏劃方”法,繪制成《輿地圖》,成為元朝地理學及中國地圖史上劃時代的人物。遊記類地理學著作有耶律楚材《西遊錄》,李志常整理的《長春真人西遊記》,周達觀《真臘風土記》,汪大淵《島夷誌略》等,對元朝國內外的地理地貌、風土人情、貿易來往等頗多描繪。

元代水陸交通的發達,使中外交往範圍空前擴大。當時,東西方使臣、商旅的往來非常方便。元人形容說:“適千裏者如在戶庭,之萬裏者如出鄰家。”同時代的歐洲商人也說,從裏海沿岸城市到中國各地,沿途十分安全。這對發展中外各國之間,國內各民族之間的科技文化交流是十分有利的。元朝與中亞、西亞地區的蒙古势力保持著來往關係,东西方海运及陆路交通十分畅通,使得西方与元朝中国的交往更加频繁,技术交流更加迅速。其中主要分陸路和水路兩部分。

陸路有發達的驛道,主要遞送朝廷、郡縣的文書。早在成吉思汗時代,就在西域地區新添了許多驛站。著名的長春真人丘處機在興都庫什山覲見元太祖成吉思汗時,即曾經過這些驛站。元世祖忽必烈統一中原後,在遼闊的國土上,建立了嚴密的驛傳制度(蒙古語“站赤”),使郵驛通信十分有效地發揮效能。元朝的驛路分為三種:一稱帖裏幹道,蒙古語意為車道;二稱木憐道,蒙語意為馬道;三為納憐道,蒙語意為小道。從地區講,帖裏幹和木憐道,多用於嶺北至上都、大都間的郵驛;納憐道僅用於西北軍務,大部分驛站在今甘肅省境內,所以亦稱“甘肅納憐驛”。驛道國內可達吐蕃、大理、天山南北路、蒙古草原,國外遠及波斯、敘利亞、俄羅斯及歐洲其它地區。

水路主要指河運和海運。河運方面元代鑿通了南起鎮江、北達大都的大運河。其中從鎮江至杭州的江南運河段,從淮安經揚州入長江的揚州運河段,大體是隋代運河舊道。元世祖以郭守時擔任都水監,負責修治元大都至通州的運河(其后被忽必烈命名通惠河),再加上修建濟州河、會通河等其它幾項重大工程,這使得連接大都至杭州的京杭大運河全線貫通。海運方面,当時元朝的船只已经航行于印度洋各地,包括锡兰(今斯里兰卡)、印度、波斯湾和阿拉伯半岛,甚至达到非洲的索马里亞。威尼斯人馬可·波羅在忽必烈時期隨從他的父親和叔叔來到中國,在其口述並由魯斯蒂謙記錄的《馬可·波羅遊記》中描繪出元朝中國的繁華景象。

元代社會因思想多元化、商業經濟發達與交通便利,使得元帝國的強盛,是東亞地區的富裕大國,在歐洲人馬可波羅的遊記中,可以看出當時的盛況。隨著理學影響的下降,長期以來壓在人們心頭的封建禮教的磐石隨之鬆動,下層人民和青年男女,蔑視禮教違反封建倫理的舉動越來越多,以至王惲對宣揚禮教的做法,發出了「終無分寸之效者,徒具虛名而已」的慨嘆。孔齊言道:「浙間婦女,雖有夫在,亦如無夫,有子亦如無子,非理處事,習以成風。」。在此說明元朝社會的價值觀念在變化,說明元代文學作品出現眾多違背封建禮教的人物,有著廣泛的社會基礎。

由於元帝對科舉的輕忽,使得大批文化人失去了優越的社會地位和政治上的前途,從而也就擺脫了對政權的依附。他們作為社會的普通成員而存在,通過向社會出賣自己的智力創造謀取生活資料,因而既加強了個人的獨立意識,也加強了同一般民眾尤其是市民階層的聯繫,他們的人生觀念、審美情趣,由此發生了與以往所謂「士人」明顯不同的變化。至於蒙古族的生活方式,原本純粹是游牧民族,逐水草而居。早期社會中的婚姻以外婚、仇家禁婚、無倫理上限制為主。他們有傳統的婚禮習俗,但在統一中國後,由於蒙漢通婚,以及漢化的影響,部分也採漢禮。


%% -*- coding: utf-8 -*-
%% Time-stamp: <Chen Wang: 2021-11-01 17:05:06>

\section{大蒙古国\tiny(1206-1260)}

\subsection{太祖成吉思汗生平}

成吉思汗(1162年5月31日-1227年8月25日),即元太祖,又稱成吉思皇帝、成吉思可汗。民國以前的漢文蒙古史料中除史集及新元史本紀外都以成吉思可汗及成吉思皇帝稱呼,成吉思汗稱呼為民國自西方翻譯而來。(《元朝秘史》記載為成吉思皇帝,《蒙古秘史》漢文版是現代翻譯。)為蒙古人,蒙古帝国奠基者、政治家、军事统帅,皇帝(大蒙古国可汗)。名铁木真,满清官译为特穆津。也有其他译法忒没真,意為“鐵匠”或“鐵一般堅強的人”、“鐵人”。奇渥温·孛儿只斤氏,尼倫蒙古乞顏部人。1206年春天—1227年8月25日在位,在位22年。1206年他登基时,诸王和群臣为他上蒙语尊号成吉思合罕。

至元二年(1265年)十月,元世祖忽必烈追尊成吉思汗廟號为太祖,至元三年(1266年)十月,太庙建成,制尊谥庙号,元世祖追尊成吉思汗諡號为聖武皇帝。至大二年十二月六日(1310年1月7日),元武宗海山加上尊谥法天啟運,庙号太祖。从此之后,成吉思汗的諡號变为法天啟運聖武皇帝。 

在他众子中,最为著名的四位分别是朮赤、察合台、窩闊台和拖雷。成吉思汗分封了朮赤和察合台为国主,欽定窩闊台为继承人。1227年成吉思汗去世后,拖雷自动退出继承人的選拔,担任监国两年后,1229年,拖雷和宗王们一起拥戴自己的三哥窝阔台登基。於1232年九月,在消灭金朝军队精锐主力后,拖雷去世,1234年2月9日,蒙古帝國灭金朝,為將來忽必烈揮軍南下攻打南宋打下基礎。

成吉思汗因其作戰的殘酷性而聞名,並被許多人視為種族滅絕的統治者。 然而,他也將絲綢之路置於一個有凝聚力的政治環境之下。 這使得東北亞,西亞和基督教歐洲之間的交流和貿易相對容易,擴大了這三個地區的文化視野。

金世宗大定二年(1162年),成吉思汗生于漠北草原。成吉思汗父親為其乞顏部酋長也速该。其名字「铁木真」之由來,乃是因為在他出生時,其父也速该正好俘虜到一位屬於塔塔儿部族,名為铁木真兀格的勇士。按當時蒙古人信仰,在抓到敵對部落勇士時,如正好有嬰兒出生,該勇士的勇氣會轉移到該嬰兒身上。成吉思汗「铁木真」之名遂因此而來。传说成吉思汗出生时,手中正拿著一血块,寓意天降將掌生殺大權。成吉思汗九歲时喪父,約於1170年。

在帶铁木真去弘吉剌部娶親後回来的路上,途經塔塔兒部,也速該遭到塔塔儿部殺害(怀疑被毒死),之後乞顏部族的泰赤乌氏首領塔里忽台因不滿也速該生前的所作所為,在也速該死後對鐵木真一家進行報復,命令部眾們遷至他地,孤立铁木真一家,但铁木真一家靠著毅力艱苦的活了下去。

就在铁木真漸漸出落成一個魁梧英俊的少年時,有三次劫難卻意外地降臨到他的頭上。

第一次是:脫離他們家族的泰赤乌氏擔心铁木真長大后報仇,於是就對铁木真家進行了突襲,並且計劃將被捕的铁木真處死。铁木真靠著父親的舊部鎖兒罕失剌以及其子沈白、赤老溫,其女合答安的協助脫逃,才因此逃过了一劫。身為長子的他,要攜母和弟妹們走到不兒罕山區,逃避泰赤乌氏追捕長達數年,自此形成他剛毅忍辱性格。

第二次是:在一個風雪交加的夜晚,一幫盜賊把他家僅有的幾匹馬搶走。在與盜賊的搏斗中,铁木真被盜賊射中喉嚨。危難之際,一個名叫博爾朮的青年拔刀相助,趕跑了盜賊,奪回了馬匹,铁木真得以幸免于難。

第三次是:成年後,铁木真與孛兒帖結婚時,三姓蔑兒乞部的首领脫黑脫阿,為報其弟赤列都的未婚妻訶額侖當年被铁木真的父親也速該所搶之仇,突袭了铁木真的營帳。在混戰中,铁木真逃進了不兒罕山(今肯特山),他的妻子和異母卻變成了脫黑脫阿的俘虜。

然而,三次劫難並未擊垮铁木真,反倒增強了他的復仇心理。他發誓要奪回家裡失去的一切。铁木真深知,要想立足,必須擁有實力。於是,他把妻子嫁妝中最珍貴的“黑貂皮”獻給了當時草原上實力最雄厚的克烈部落統領王汗。利用王汗的勢力,铁木真不僅收攏了他家離散的部族,還在王汗及幼時“安答”(義兄弟)札木合的幫助下,擊敗了三姓蔑兒乞部首领脫黑脫阿、忽都父子,救出了妻子孛兒帖和異母。

自此铁木真和札木合两人一起在部落共同生活。

由于铁木真提拔一些非贵族的人为将领,引發札木合不满,最终雙方决裂。1182年,铁木真被推举成为蒙古乞颜部的可汗。

統一蒙古各部:

1190年,在铁木真的領導下,乞顏迅速發展壯大,引起札达兰部首領札木合的不滿。札木合以其弟弟绐察兒被铁木真部下所殺為藉口,糾集了13個部落三萬余人,向铁木真發起進攻。铁木真也動員了部眾十三翼(即13個部落)迎擊,即著名的十三翼之戰。铁木真雖兵敗退至斡難河畔哲列捏狹地,但萬萬沒想到獲勝的札木合卻失去了人心。戰後,因為札木合把俘虜全部處死,將俘虜分七十大鍋煮殺,史稱「七十鍋慘案」。這種慘不忍睹的場面,連其部下也“多苦其主非法”,甚至擔心起自己的命運來。相反的,寬厚仁容的铁木真贏得了人心,那些擔心自己命運的札木合的部下紛紛倒向铁木真。此戰铁木真敗而得眾,使其軍力得以迅速恢復和壯大。铁木真的部眾一下子增加了許多。1196年,塔塔儿部首领蔑兀真笑里徒反抗金朝,金朝丞相完颜襄约克烈部王汗和铁木真联合出兵进攻塔塔儿,塔塔儿部大败,蔑兀真笑里徒被杀。铁木真遂被金朝封為“札兀惕忽里”,即部落官。

主兒乞部偷襲鐵木真的後方營地,被鐵木真剿滅。1201年,泰赤乌部、塔塔儿部、蔑兒乞部等11部推舉札达兰部的札木合為“古兒汗”,联兵攻打铁木真。铁木真联合王汗,於阔亦田之戰击败札木合等十二部联軍。聯軍潰散後,鐵木真追擊並剿滅了泰赤烏部。1202年,杀死塔塔儿部首领札鄰不合並屠殺残余的塔塔儿人,憶起少年時,父親也速该遭塔塔儿所害,临命终時的遗言,遂將凡是身高超過車輪高的塔塔儿士兵、男子通通都殺光,手法殘忍震驚蒙古諸部族。

1203年,王汗將铁木真收為义子,導致桑昆跟铁木真仇恨,札木合鼓動桑昆联合王汗夹击铁木真。合蘭真沙陀之戰爆发,这是铁木真经历的最为惨烈的一仗,只剩下19人隨他敗走班朱尼河,北上贝尔湖途中陸續追隨而來的部眾也只有2千6百人。同年秋天突袭王汗驻地,三天后完全消灭克烈部。王汗逃到鄂尔浑河畔之后被乃蛮人杀死。而其子桑昆則逃到庫車,被當地人杀死。

1204年,铁木真征伐蒙古草原西边的太阳汗,於納忽崖之戰击败乃蛮大軍,太阳汗当场被杀。秋,於合剌答勒忽札兀兒擊敗蔑兒乞部首領脫黑脫阿。1205年,鐵木真於額爾齊斯河擊敗蔑兒乞和乃蠻殘部聯軍,蔑兒乞首領脫黑脫阿陣亡,其子逃往康里、欽察,乃蛮部王子屈出律則逃亡西辽。1206年,札木合被叛变的将领送到铁木真之手,札木合请死,铁木真便殺了他。爾後,铁木真統一蒙古各部。

称成吉思汗:

1206年春天,蒙古贵族们在斡难河(今鄂嫩河)源头召开大会,諸王和群臣為鐵木真上尊號“成吉思汗”,正式登基成为大蒙古国皇帝 (蒙古帝国大汗),这是蒙古帝國的開始。成吉思汗遂颁布了《成吉思汗法典》,是世界上第一套應用範圍最廣泛的成文法典,建立了一套以贵族民主為基礎的蒙古贵族共和政體制度。

威脅西夏:

蒙古分别在1205年、1207年及1209年三次入侵西夏,逼使西夏臣服。1210年,西夏向蒙古称臣,並保证派军队支持蒙古以后的军事行动,此外,西夏皇帝夏襄宗献女求和,把察合公主嫁给了成吉思汗。

征服森林部落:1207年,成吉思汗命長子朮赤征森林部落。

降葛邏祿:1210年,成吉思汗命忽必來征葛邏祿,首領阿兒思蘭汗率部降。

消滅金朝未果:1210年,成吉思汗与金朝断绝了朝贡关系(约从1195年开始)。

1211年二月,成吉思汗亲率大军入侵金朝,在1211年的野狐嶺會戰击败四十萬金軍,并在次年和第三年陆续攻破金朝河北、河东北路和山东各州县,1214年三月,金宣宗遣使向蒙古求和,送上大量黄金、丝绸、马匹,并将金卫绍王的女儿岐国公主送给成吉思汗为妻,还有童男女五百陪嫁。成吉思汗从中都撤兵。

在金朝的东北地区,1212年,契丹人耶律留哥在辽东起兵反抗金朝,并宣布归附蒙古,耶律留哥和蒙古联军打败前来征讨的六十万金朝军队,1213年,耶律留哥自称辽王,1215年春,耶律留哥攻克金朝东京(今辽宁省辽阳),并占领金朝东北大部分地区。1215年十一月耶律留哥秘密与其子耶律薛阇带着厚礼前往漠北草原朝觐成吉思汗,成吉思汗极为高兴,赐给耶律留哥金虎符,仍旧封他为辽王。

为了远离蒙古的威胁,1214年6月27日,金宣宗离开中都,遷都汴京,得知金朝皇帝离开,成吉思汗下令入侵中都,蒙古軍在1215年5月31日占领中都,金朝在黃河以北之地陸續失守。

占领中都后,成吉思汗返回蒙古草原,1217年,成吉思汗任命大将木华黎为“太师国王”,让他负责继续入侵金朝,经过木华黎和他的儿子孛鲁十年的战争,到1227年成吉思汗去世前夕,蒙古军队基本占领金朝黄河以北的所有领土,金朝的领土仅局限于河南、陕西等地(当时的黄河取道江苏北部的淮河入海)。

1217年,成吉思汗派大将速不台追击脫黑脫阿諸子忽都、合剌、赤剌溫,次年於楚河地區剿滅蔑儿乞殘部。

正當金朝危在旦夕時,中亞的花剌子模王国惹怒蒙古,成吉思汗性急,转而報仇,暂时无暇顾及继续入侵金朝。

滅西辽及花剌子模:

早在1211年春天,畏兀儿亦都護巴而朮·阿而忒·的斤便归附蒙古。至1218年春季,成吉思汗派遣的蒙古使团到达花剌子模王国,强迫摩诃末苏丹签订与蒙古的条约。条约签订后,花剌子模城市讹答剌长官杀死路过此城的一支来自蒙古的由500人穆斯林组成的商队,夺取货物,仅有一人幸免于难逃回蒙古,成吉思汗派三个使臣前往花剌子模向摩诃末交涉,结果为首者被杀,另外二人被辱,成吉思汗更加愤怒,决定入侵花剌子模。

1218年,成吉思汗派大将哲别灭西辽,杀死西辽末代皇帝屈出律,平定西域。西征花剌子模进兵路上的障碍被扫除了。

1219年六月,成吉思汗親率蒙古主力(大约十万人)向西侵略,并在中途收编了5万突厥军,1220年底,一直被蒙古军队追击的花剌子模算端摩诃末病死在宽田吉思海(今里海)中的一个名為額別思寬島(或譯為阿必思昆島,已陸沉)的小岛上,并在临死前传位札兰丁。蒙古军先后取得河中地区和呼罗珊等地,1221年,蒙古军队消滅花剌子模王国,1221年十一月,成吉思汗率军追击札兰丁一直追到申河(今印度河)岸边,札兰丁大败,仅仅率少数人渡河逃走。

当初,成吉思汗命令速不台和哲别率领二万骑兵追击向西逃亡的摩诃末,摩诃末逃入里海后,他们率領蒙古軍继续向西进发,征服了太和岭(今高加索山)一带的很多国家,然后继续向西进入欽察草原擴張。1223年,者别與速不台於迦勒迦河之战(今乌克兰日丹诺夫市北)中击溃基辅罗斯诸国王公与钦察忽炭汗的联军,然后又攻入黑海北岸的克里木半岛。

1223年底,哲别與速不台率军东返,经过也的里河(今伏尔加河的突厥名,又译亦的勒),攻入此河中游的不里阿耳,遭遇顽强抵抗后,沿河南下,经由里海,咸海之北,与成吉思汗会师东归。在东返途中,哲别病逝。

攻西夏·去世:

成吉思汗回師後幹,再次入侵西夏。1227年8月25日(农历七月十二己丑日),在蒙古軍圍困西夏首都時,成吉思汗病逝於今宁夏南部六盘山(一说灵州),享壽六十五歲。其死因至今眾說紛紜,《元史》记载:“(元太祖二十二年)秋七月壬午,不豫。己丑,崩于萨里川啥老徒之行宫。”

成吉思汗去世前向儿子们交代了灭金的計劃:“假道宋境,包抄汴京。”后来窝阔台和拖雷灭金朝,采用的就是成吉思汗的这个战略。

此前西夏末代皇帝李睍已经答应投降,成吉思汗去世后,蒙古军密不发丧,李睍开城投降后,前去参见成吉思汗,诸将托言成吉思汗有疾,不让他参见。在成吉思汗去世三天后,1227年8月28日,诸将遵照成吉思汗遗命将西夏末帝杀死,西夏灭亡。蒙古军将领察罕努力使西夏首都中興府(今宁夏銀川)避免了屠城的命运,入城安抚城内军民,城内的军民得以保全。

《元朝祕史》记载成吉思坠马跌伤。而罗马天主教教廷使节约翰·普兰诺·加宾尼在《被我们称为鞑靼的蒙古人的历史》稱成吉思汗可能是被雷电击中身亡。

据《蒙古秘史》记载,成吉思汗的遗体被葬在不兒罕山接近斡難河源頭的地方,这是他生前指定的墓地。《元史》则记载他和历代元朝皇帝都葬于起辇谷。起辇谷的具体位置不详。在今日蒙古国肯特省的不儿罕山间有一片被称为“大禁忌”的土地,为达尔扈特人世代守护,相传是成吉思汗的墓地所在。在内蒙古自治区西部的鄂尔多斯高原上,有一座蒙古包式建筑宫殿,為成吉思汗的衣冠冢,经过多次迁移後直到1954年才由湟中县的塔尔寺迁回故地伊金霍洛旗,北距包头市185公里。每年的农历三月廿一、五月十五、八月十二和十月初三,为一年四次的大祭。

有傳言認為成吉思汗可能是遭三子窩闊台毒杀,原因是当时大汗打算传位给窝阔台,但突然改变注意,欲传位给四子拖雷,窝阔台为保汗位,所以毒杀其父。《成吉思汗与今日世界之形成》关于成吉思汗之死的論述与诸多的死亡故事相反,認為成吉思汗在游牧帐篷中去世,与他在游牧帐篷中的出生情形相似,这说明他在保存其本民族传统生活方式方面非常成功;然而,他保持其自身生活方式的过程中,却改变了人类社会。他在故土安葬,没有一座陵墓,没有一座寺庙,甚至没有一块用来标示其长眠之地的小墓碑。按照蒙古人的信仰,遗体应该在静穆中离去,并不需要纪念碑,因为灵魂已经不在那里了;灵魂继续活在精神之旗中。但他的精神之旗在1937年从蒙古中部的黑尚赫山下月亮河畔的寺庙里消失了。虔诚的喇嘛们护卫几个世纪的圣物,在由当时斯大林的追随者霍尔洛·乔巴山开展的遏制蒙古文化与宗教的运动中,永远的消失了。

尊谥庙号:

至元二年十月十四日(1265年11月23日),元世祖忽必烈追尊成吉思汗廟號为太祖。

至元三年十月十八日(1266年11月16日),太庙建成,制尊谥庙号,元世祖追尊成吉思汗諡號为聖武皇帝。

至大二年十二月六日(1310年1月7日),元武宗海山加上尊谥法天啟運,庙号太祖。从此之后,成吉思汗的諡號变为法天啟運聖武皇帝。《太祖皇帝加上尊谥册文》,内容如下:

维至大二年、岁次己酉、某月、某日,孝曾孙嗣皇帝臣某,谨再拜稽首言:

{\fzk 伏以恢皇纲,廓帝纮,建万世无疆之业;铺宏休,扬伟绩,遵累朝已定之规。式当继统之元,盍有称天之诔。孝弗忘于率履,制庸谨于加崇。钦惟太祖圣武皇帝陛下,渊量圣姿,睿谋雄断,沛仁恩而济屯厄,振羁策以驭豪英。惟解衣推食于初年,见君国子民之大略。玄符颛握,诸部悉平;黄钺载麾,百城随下。裔土兼收于夏孽,余波克殄于金源。荡荡乎无能名迹,远追于汤武;灏灏尔其为训道,允协于唐虞。根深峻岳而维者四焉,囊括殊封而统之一也。

肆予小子,承此丕基。两袛见于太宫,恒僾临于端扆。祚垂鸿兮锡裕,尚期昭报之申;牒镂玉以增辉,敢缓弥文之举。谨遣某官某,奉玉册玉宝,加上尊谥曰法天启运圣武皇帝,庙号太祖。

伏惟威灵昭假,景贶潜臻,阐绎吾元,与天并久。}

称号来源:“成吉思汗”是铁木真於1206年获得的称号。“成吉思”的含义不明确,一种说法由“成”派生而来。另一种说法是来自海洋一词,代表他像海洋一样伟大。

现存的13世纪和14世纪期的众多史料以及考古文物和摩崖石刻证明,1206年成吉思汗建立大蒙古国后,可能已经拥有皇帝和大汗的双重身份。生活在草原地区的蒙古等民族用蒙古语称呼铁木真为“大汗”、“成吉思汗”;生活在西北地区的突厥和其他民族用突厥语或其他语言称铁木真为“汗”或者“可汗”;生活在漠南汉地和东北地区的契丹人、女真人、党项人等民族,在13世纪前期的时候,历经辽朝、金朝、西夏等汉化政权,大部分已经汉化,通用汉语汉字,多称铁木真为“皇帝”;而生活在漠南汉地和东北地区的汉族人则直接使用“成吉思皇帝”一词。大量历史记载资料证明,1215年成吉思汗在攻取包括金中都在内的整个幽云十六州之后,其在长城以南汉地的统治保留了一些辽、金等朝的旧俗,并且在这些区域的官方文件,直接应用了“皇帝”的尊号来指代历任大蒙古国大汗。例如:

1219年农历五月,铁木真派刘仲禄邀请长春真人丘处机前往蒙古草原的诏书中,自称为“朕”,将自己建国登基称为“践祚”。

1220年农历二月丘处机抵达燕京后,得知铁木真在中亚进行西征花剌子模的战争,觉得自己年事已高,西行太远,希望约铁木真在燕京相见,于是在三月写了一份陈情表,在陈情表中,丘处机对铁木真的称呼是“皇帝”。同年收到丘处机的陈情表后,铁木真第二次派曷剌邀请丘处机前往中亚草原的诏书中,以“成吉思皇帝”和“朕”自称。

1221年南宋使者赵珙出使大蒙古国,回来后著有《蒙鞑备录》,书中对铁木真的称呼是“成吉思皇帝”。《蒙鞑备录》中提到,铁木真在位时期,朝廷使用的金牌,带两虎相向,曰虎头金牌,上书汉字:“天赐成吉思皇帝圣旨,当便宜行事”;其次为素金牌,书:“天赐成吉思皇帝圣旨疾”。1998年,一块“圣旨金牌”发现于河北廊坊,正面刻双钩汉字:“天赐成吉思皇帝圣旨疾。”和《蒙鞑备录》所记载的素金牌上汉文完全相同;背面牌心刻双钩契丹文,其汉语意思为:“速、走马,或快马”。这块圣旨牌的发现,说明铁木真在世时,其官方中文称谓作“成吉思皇帝”。

1227年全真教道士李志常写成的《长春真人西游记》,记录了丘处机从1219年受邀西行直至1227年去世的事迹,书中对铁木真的称呼是“成吉思皇帝”,将他下的命令称为“聖旨”;书中也提到了铁木真的侍臣刘仲禄前来邀请丘处机时携带了虎头金牌,金牌上面的文字是:“如朕亲行、便宜行事”,似乎在铁木真时期,凡是针对汉地的蒙古官方文件,均把成吉思汗翻译为“成吉思皇帝”。

1232年南宋使者彭大雅随奉使到大蒙古国,使者徐霆1235年—1236年随奉使到大蒙古国,二人返回南宋后,彭大雅撰写,并由徐霆作疏,合著《黑鞑事略》,书中对铁木真的称呼是“成吉思皇帝”。

2010年,刻有多位蒙古皇帝圣旨的全真教炼神庵摩崖石刻于山东徂徕山被发现,石刻一共四方,全部以汉语白话文写就,记述了大蒙古国皇室成员历代颁发给全真教掌教的官方文牒,其中有成吉思皇帝、合罕皇帝(窝阔台)、贵由皇帝,孛罗真皇后(窝阔台之妻)、唆鲁古唐妃,以及昔列门太子、和皙太子(均为窝阔台之子)等字样,其中记叙的“甲辰年十月初八日”表明该条圣旨是乃马真后称制的1244年颁发,落款“庚戌年十二月”则表明该石刻刻于海迷失后称制的1250年。圣旨石刻以汉语写就,包含不同时期、不同蒙古大汗的圣旨记录,为大蒙古国时期在汉地以中文“皇帝”作为蒙古大汗官方尊号的有力文物证据。

至元三年(1266年)忽必烈给日本的国书中,国书开头自称“大蒙古国皇帝”,在后面的内容中,自称为“朕”,此时距离他1271年正式立国号“大元”,还有五年时间。

然而大蒙古国时期的“皇帝”,和后来元朝的“皇帝”称号有本质的不同;前者是对“蒙古大汗”的汉式翻译,而后者则是按照中原文明的传统开立的新王朝君主,其“皇帝”称号上承秦汉隋唐宋等朝代。在1259年蒙哥汗去世后,忽必烈认为自己是大蒙古国汗位的正式继承者,自立为大汗,称“大蒙古国皇帝”,并于1263年将大蒙古国的历代大汗一并列入了自己新落成的太庙中;由于最终忽必烈没能获得蒙古各部贵族认可为新一任大汗,其于1271年按照中原文明的传统,建国号“大元”,因而元朝以后官方正史一直依照庙号将成吉思汗称作“太祖”。此时的大元皇帝,与之前大蒙古国时期被称作“皇帝”的蒙古大汗有本质区别——蒙古四大汗国的独立、大蒙古国的分裂,标志着忽必烈没能正式继承“大蒙古国”大汗之位;元朝,则是其新开创的王朝。元成宗时期,經過與蒙古四大汗國協商,元朝皇帝作为整个蒙古帝国共主的身份獲得四大汗國承認,作为中国历史上最高统治者称号的“皇帝”称号和作为“大蒙古国”最高统治者称号的“大汗”称号,同时集合在了後代的元朝皇帝的身上,如同中世纪歐洲由某王國國王或某公国大公出任神聖羅馬帝國皇帝。

整个元朝时期乃至后世王朝,官修历史一直保持了元朝的传统,将大蒙古国时期与元朝时期的统治一并而论,不作区分,统一将君主称为“皇帝”。《元史》中的<太祖本纪>記載鐵木真於1206年建大蒙古国时,称其“即皇帝位于斡难河之源,诸王群臣共尊其為成吉思皇帝”。元惠宗至正五年(1345年)十一月修成的法律《至正条格》中,称铁木真为“成吉思皇帝”,将他下的命令称为“聖旨”。明初官修《元史》,书中出现过“成吉思皇帝”一词多次,从未出现过“成吉思汗”一词。1252年成书的《元朝秘史》(《蒙古秘史》),蒙文音译作“成吉思合罕”,旁注释为“太祖皇帝”。直到近代中国,《新元史》中出现了“成吉思合罕”、“成吉思可汗”等词语,原因是《新元史》完成于民初(1920年),而《史集》、《世界征服者史》等西方的史书在清朝末年传入中国,《新元史》作者柯劭忞也深受其影响。

然而对于中国以外的地区,则仍将“大蒙古国”的君主称谓记作“大汗”。关于“成吉思汗”的记载见于拉施特《史集》、志费尼《世界征服者史》等中亚史籍,这两位作者均为蒙古帝国时期伊儿汗国(位于西亚)史学家,与元朝《元史》等史书基本处于同一时代,其书可为依据。四大汗国治下以的西亚国家以及欧洲公国仅知“成吉思汗”,同一时期的中国仅知“成吉思皇帝”,可见“成吉思皇帝”一词是针对古代汉字文化圈地区特设的翻译用词;由于西亚及欧洲文字皆为表音文字,其记载最能说明,大蒙古国君主的官方称谓仍为“大汗”,而非“皇帝”。

也速該,鐵木真父親,從蔑兒乞部手中奪走訶額侖,1170年被塔塔儿部首領札鄰不合毒害。也速該死後,族人離散,令鐵木真一家被逼過著流離生活。1266年元世祖忽必烈追尊也速该为皇帝,为也速该上庙号烈祖,諡號神元皇帝。

訶額侖,鐵木真母親,1206年尊为皇太后,1266年元世祖忽必烈上谥号宣懿皇后。

金末元初长春真人丘处机,拒绝金朝皇帝和南宋皇帝的邀请,答应前往草原和铁木真相见,抵达燕京后,得知铁木真已在中亚西征花剌子模,觉得自己年事已高,西行太远,希望约铁木真在燕京相见,于是在1220年三月写了一份陈情表,在陈情表中,对铁木真的评价是:“前者南京及宋国屡召不从,今者龙庭一呼即至,何也?伏闻皇帝天赐勇智,今古绝伦,道协威灵,华夷率服。是故便欲投山窜海,不忍相违;且当冒雪冲霜,图其一见。”(南京指的是当时的金朝首都开封,1214年,金朝从中都迁都到南京开封府)

南宋使者赵珙,1221年出使大蒙古国,在燕京(原为金中都,1215年被蒙古军队攻取,1217年木华黎改名燕京,今北京市)见到主持进攻金朝的太师国王木华黎,回来后著有《蒙鞑备录》,书中的評價是:“今成吉思皇帝者,……。其人英勇果决,有度量,能容众,敬天地,重信义。”

蒙古帝国伊儿汗国史学家志费尼《世界征服者史》的評價是:“倘若那善于运筹帷幄、料敌如神的亚历山大活在成吉思汗时代,他会在使计用策方面当成吉思汗的学生,而且,在攻略城池的种种妙策中,他会发现,最好莫如盲目地跟成吉思汗走。”

明朝官修正史《元史》宋濂等的評價是:“帝深沉有大略,用兵如神,故能灭国四十,遂平西夏。其奇勋伟迹甚众,惜乎当时史官不备,或多失于纪载云。”

明朝官修皇帝实录《明太祖实录》记载,洪武二十二年(1389年)五月,明太祖朱元璋给北元阿札失里大王的信中,对成吉思汗、元太宗窝阔台、元定宗贵由、元宪宗蒙哥、元世祖忽必烈这五位在一统天下中均作出重要贡献的帝王的综合评价如下:“覆载之间,生民之众,天必择君以主之,天之道福善祸淫,始古至今,无有僣差。人君能上奉天道,勤政不贰,则福祚无期,若怠政殃民,天必改择焉。昔者,二百年前,华夷异统,势分南北,奈何宋君失政,金主不仁,天择元君起于草野,戡定朔方,抚有中夏,混一南北,逮其后嗣不君,于是天更元运,以付于朕。”

明朝官修皇帝实录《明太祖实录》记载,洪武二十二年(1389年)十二月,明太祖朱元璋给哈密国兀纳失里大王的信中,对成吉思汗和元世祖忽必烈的评价如下:“昔中国大宋皇帝主天下三百一十余年,后其子孙不能敬天爱民,故天生元朝太祖皇帝,起于漠北,凡达达、回回、诸番君长尽平定之,太祖之孙以仁德著称,为世祖皇帝,混一天下,九夷八蛮、海外番国归于一统,百年之间,其恩德孰不思慕,号令孰不畏惧,是时四方无虞,民康物阜。”

清朝史学家邵远平《元史类编》的評價是:“册曰:天造鸿图,艰难开创;浑河启源,角端呈像;芟夏蹙金,电扫莫抗;栉沭廿年,驱指四将;止杀一言,皇猷弥广。”

清朝史学家毕沅《续资治通鉴》的評價是:“太祖深沉有大略,用兵如神,故能灭国四十,遂平西夏。”

清朝史学家魏源《元史新编》的評價是:“帝深沉有大略,用兵如神,故能灭国四十,遂平夏克金,有中原三分之二。使舍其攻西域之力,以从事汴京,则不俟太宗而大业定矣。然兵行西海、北海万里之外,昆仑、月竁重译不至之区,皆马足之所躏,如出入户闼焉。天地解而雷雨作,鹍鹏运而溟海立,固鸿荒未辟之乾坤矣。”

清朝史学家曾廉《元书》的評價是:“论曰:太祖崛起三河之源,奄有汉代匈奴故地,而兼西域城郭诸国,朔方之雄盛未有及之者也。遗谋灭金,竟如其策,金亡而宋亦下矣,此非其略有大过人者乎?又明于求才,近则辽金,远则西域,仇敌之裔,俘囚之虏,皆收为爪牙腹心,厥功烂焉,何其宏也,立贤无方,太祖有之矣。羽翼盛,斯其负风也大,子孙蒙业,遂一宇宙,不亦宜乎。”

民国史学家屠寄《蒙兀儿史记》的評價是:“论曰:旧史称成吉思汗深沉有大度,用兵如神,故能灭国四十,遂平西夏,信然。独惜军锋所至,屠刿生民如鹿豕,何其暴也。及至五星聚见东南,末命谆谆,始戒杀掠,岂所谓人之将死,其言善欤!蒙兀一代,并漠北四君数之,卜世十四,卜年蕲百六十,唐宋以降,享国历数,为由蹙于是者。于戏,可以观天道矣!”

民国官修正史《新元史》柯劭忞的評價是:“天下之势,由分而合,虽阻山限海、异类殊俗,终门于统一。太祖龙兴朔漠,践夏戡金,荡平西域,师行万里,犹出入户闼之内,三代而后未尝有也。天将大九州而一中外,使太祖抉其藩、躏其途,以穷其兵力之所及,虽谓华夷之大同,肇于博尔济锦氏,可也。” 

民国史学家张振佩《成吉思汗评传》(1943年版)绪言部分的評價是:“成吉思汗之功业扩大人类之世界观——促进中西文化之交流——创造民族新文化。”

1939年,处于抗战时期的中国共产党对成吉思汗做出了高度评价。6月21日,成吉思汗灵柩西迁途中到达延安时,中共中央和各界人士二万余人夹道迎灵,并在延安十里铺搭设灵堂,举行了盛大的祭祀活动。在此次祭祀仪式上,中共中央将成吉思汗正式尊称为“世界巨人”、“世界英杰”,并首次提出“继承成吉思汗精神坚持抗战到底”的口号。延安十里铺灵堂两侧悬挂一幅对联,灵堂正上方有一横联,内容如下:

横联:世界巨人
上联:蒙漢兩大民族更親密地團結起來

下联:繼承成吉思汗精神堅持抗戰到底

灵堂前面搭建一座牌楼,悬挂“恭迎成吉思汗靈柩”匾额。代表们将灵柩迎入灵堂后,举行祭典。中共中央、毛澤東、周恩來、朱德等敬献了花圈。由陕甘宁边区政府秘书长曹力如代表党政军民学各界恭读祭文:维中华民国二十八年六月二十一日,中国共产党中央委员会代表谢觉哉、国民革命军第八路军代表滕代远、陕甘宁边区政府代表高自立,率延安党政军民学各界,谨以清酌庶馐之奠,致祭于圣武皇帝成吉思汗之灵曰:

日寇逞兵,为祸中国,不分蒙汉,如出一辙。
嚣然反共,实则残良,汉蒙各族,皆眼中钉。
乃有奸人,蠢然附敌,汉有汉奸,蒙有蒙贼。
驱除败类,整我阵容,抗战到底,大义是宏。
顽固分子,准投降派,摩擦愈凶,敌愈称快。
巩固团结,唯一方针,有破坏者,群起而攻。
元朝太祖,世界英杰,今日郊迎,河山聚色。
而今而后,五族一家,真正团结,唯敌是挝。
平等自由,共同目的,道路虽艰,在乎努力。
艰苦奋斗,共产党人,煌煌纲领,救国救民。
祖武克绳,当仁不让,太旱盼霓,国人之望。
清凉岳岳,延水汤汤,此物此志,寄在酒浆。
尚飨!

1940年3月31日,中国共产党在延安成立了“蒙古文化促进会”,4月,在延安建立了“成吉思汗纪念堂”和“蒙古文化陈列馆”,敬立成吉思汗半身塑像,并由毛澤東题写了“成吉思汗紀念堂”七个大字。在这里每年农历三月二十一日,也就是成吉思汗春季查干苏鲁克大祭之日,延安各界举行盛大的祭奠仪式,以蒙汉两种语言诵读成吉思汗祭文。1942年5月5日,蒙古文化促进会还编辑出版了《延安各界纪念成吉思汗专刊》。毛澤東和朱德分别为专刊題詞,内容如下:毛澤東題詞:團結抗戰;朱德題詞:中華民族英雄。

毛泽东在1964年3月24日,在一次听取汇报时的插话中对成吉思汗、汉高祖刘邦、明太祖朱元璋的治国能力评价如下:“可不要看不起老粗。”“知识分子是比较最没有知识的,历史上当皇帝的,有许多是知识分子,是没有出息的:隋炀帝,就是一个会做文章、诗词的人;陈后主、李后主,都是能诗善赋的人;宋徽宗,既能写诗又能绘画。一些老粗能办大事:成吉思汗,是不识字的老粗;刘邦,也不认识几个字,是老粗;朱元璋也不识字,是个放牛的。”(毛泽东举例只是为了强调“一些老粗能办大事”,并不是说成吉思汗和刘邦真的不识字,也不是说刘邦只认识几个字。事实上,成吉思汗,刘邦,朱元璋三人原本可能僅能粗通文字,但當他們身为帝王時,他们的文化水平已經达到批阅奏折和签署命令的程度,甚至能為唱和文章。刘邦和朱元璋的文化水平不必细谈,相关史书记载很多,至于成吉思汗,元初名臣耶律楚材在《玄风庆会录》一书中提到成吉思汗是可以亲自阅览文件的。)

1941年十一月三日国民政府正式宣布对日本及德国、意大利宣战前夕,蒋介石赶赴甘肃省榆中县兴隆山,对成吉思汗灵寝举行了大祭。蒙藏委员会委员长吴中信代表国民政府恭读祭文:維中華民國三十年十一月三日國防最高委員會委員長蔣中正,特派蒙藏委員會委員長吳中信,以馬羊帛酒香花之儀,致祭於成吉思汗之靈而昭告以文曰:

繄我中華,五族為家,自昔漢唐盛世,文德所被,蓋已統乎西域極於流沙,洎夫大汗崛起,武功熠耀,馬嘶弓振,風撥雲拏,縱橫帶甲,馳驟歐亞,奄有萬邦,混一書車,其天縱神武之所肇造,雖曆稽往古九有之英傑而莫之能加,比者蝦夷小醜,虺毒包藏,興戎問鼎,豕突倡狂,致我先哲之靈寢乍寧處而不遑,中正忝領全民,撻伐斯張,一心一德,慷慨騰驤,前僕後興,誓殄強梁,請聽億萬鐵馬金戈之凱奏,終將相複於伊金霍洛之故鄉,緬威靈之赫赫兮天蒼蒼,撫大漠之蕩蕩兮風泱泱,修精誠以感通兮興隆在望,萬馬胙而陳體漿兮神其來嘗。尚饗。

1957年三月十二日,蒋介石在在主持陸軍指揮參謀學校正×期開學典禮講——《軍事哲學對於一般將領的重要性》中,评价成吉思汗:“我在此還要舉出我們中國歷史中兩位最有名的勇將來作一對照,以供我們今日軍人的抉擇。這兩位勇將中的第一位,就是漢楚時代的項羽。第二位就是縱橫歐亞的成吉思汗。這二位英勇無比的名將,其平生戰績乃是眾所周知,無待詳述,可是其結果則完全不同。茲據其二人所製的歌詞的氣概與精神,就可想見膽力的強弱與事業的成敗了。當成吉思汗西征時的歌詞是:「上天與下地,俯伏嘯以齊,何物蠢小醜,而敢當馬蹄」。而項羽最後失敗時的歌詞則是:「力拔山兮氣蓋世,時不濟兮騅不逝,騅不逝兮可奈何,虞兮虞兮奈若何?」後來還有許多人評判項羽這首歌詞是悲歌慷慨,不失為英雄氣概;我以為項羽的歌詞充滿了「恐懼」「憤怒」「疑惑」的氣氛,毫無英勇鎮定與自信的心理,更沒有如克勞塞維茨所說:「在絕望中之奮鬥」的軍人精神。所以到了最後他只有在烏江自刎了事。我以為這種卑怯自殺,而不能抱定榮譽戰死的軍人,只可說是一個最無志氣的懦夫,那能配稱為勇將!故無論他過去有如何勇敢的史蹟,我們不僅不屑敬仰他,而且應在棄絕不齒之列。至於成吉思汗的這首歌詞,我認為是充滿了他自信、勇敢與鎮定的心理,誠不失為一首英勇壯烈的歌詞,正與項羽的歌詞語意完全相反,所以他成功亦自不同。因為他既有這樣一個戰勝一切的信心,自然不會再有恐懼憤怒與疑惑的心理了。所以成吉思汗,實為我們中國軍人所應該效法與崇敬的第一等模範英雄。”
中華民國總統馬英九在2009年4月16日(农历三月二十一日)“二00九年中枢致祭成陵大典”中,特派蒙藏委员会委员长高思博主祭成吉思汗。祭坛上陈放有成吉思汗的画像,摆放有鲜花、水果和糕点,点燃供烛。仪式遵循古礼。台北市国乐团演奏乐曲《万寿无疆》。身穿长袍马褂的高思博,依序向成吉思汗像献香、献花、献爵(献酒)、献帛(献哈达)。司仪宣读祭文:“马英九特派蒙藏委员会委员长高思博敬以香花清酌之仪致祭于成吉思汗之灵曰:‘维汗休烈,雄才大略。天挺英明,龙兴溯漠。……礼仪孔修,有芘其芳。神之格思,德音不忘。’”

馬英九在2010年5月4日(农历三月二十一日)蒙藏委員會上午舉辦的“99年中樞祭成吉思汗大祭”典禮中,指派蒙藏委員會委員長高思博以香花清酌儀式祭拜成吉思汗。典禮安排向成吉思汗像獻花、獻香、獻爵(獻酒)、獻帛(獻哈达),並宣讀“中華民國總統祭文”,相關司祭者皆穿著蒙古傳統服飾,儀式遵循古禮,場面莊嚴隆重,馬英九在祭文中,肯定成吉思汗“雄才大略,天挺英明,拓土開疆,威震萬國。”

馬克思在《馬克思印度史編年稿》中谈到成吉思汗时曾说:“成吉思汗戎馬倥傯,征戰終生,統一了蒙古,為中國統一而戰,祖孫三代鏖戰六七十年,其後征服民族多至720部。”

瑞典學者多桑在其《蒙古史》中對成吉思汗的一生總結分析,多桑認為為成吉思汗之成功乃由於其具有極強的貪慾以及非常之野心。多桑稱他“狂傲”地妄想征服世界,死前還囑咐其子孫完成他的事業。

英国学者莱穆在《全人类帝王成吉思汗》一书中说:“成吉思汗是比欧洲历史舞台上所有的优秀人物更大规模的征服者。他不是通常尺度能够衡量的人物。他所统率的军队的足迹不能以里数来计量,实际上只能以经纬度来衡量。”

印度总理尼赫鲁在《怎样对待世界历史》一书中说:“蒙古人在战场上取得如此伟大的胜利,这并不靠兵马之众多,而靠的是严谨的纪律、制度和可行的组织。也可以说,那些辉煌的成就来自于成吉思汗的指挥艺术。”

「卡內基全球生態研究部」:「歷史上『最環保的侵略者』。因為殺人無數,讓大片耕地恢復成為森林,讓大氣中的碳大幅減量達7億噸!」

美国西維吉尼亞大學的研究人员指出成吉思汗的成功恰逢当时1000年来最温和、最潮湿的天气,之前的1180-1190年间,蒙古曾经历严重干旱,之后的温和湿润气候有助于青草的繁茂生长,为以骑兵为主的蒙古大军的战马提供了丰富的饲料。

1999年12月的美国A+E电视网评选出过去千年影响最深远的100大人物,成吉思汗被列为第22位(在亚洲人中仅次于第17位的甘地)。

\subsection{睿宗拖雷生平}

拖雷(1191年-1232年)又译图垒,是元太祖成吉思汗的幼子,排行第四。拖雷和正妻唆鲁禾帖尼生有四子:蒙哥、忽必烈、旭烈兀、阿里不哥。据《元史》,拖雷一共有子十一人。1227年8月25日至1229年9月13日担任大蒙古国(蒙古帝国)监国,历时二年。

1213年,成吉思汗分兵伐金,拖雷从其父率领中路军,攻克宣德府,再攻德兴府。拖雷与驸马赤驹先登,拔其城。即而挥师南下,拨涿州、易州,残破河北、山东诸郡县。1219年,从成吉思汗西征,攻陷布哈拉、撒馬爾罕。1221年,分领一军进入呼罗珊境,陷马鲁、尼沙不儿,渡搠搠阑河,降也里。遂与成吉思汗合兵攻塔里寒寨。

按照蒙古习俗,幼子继承父业,而年长诸子则分析外出、自谋生计。故成吉思汗生前分封诸子,拖雷留置父母身边,继承父亲所有在斡难和怯绿连的斡耳朵、牧地及军队。成吉思汗留下的军队共有12.9万人。其中10.1万的精銳俱由拖雷继承。

1221年拖雷屠杀木鹿(今梅尔夫)城中居民,超过100萬人。除去四百个工匠之外,其余人口被屠杀殆尽,城墙被毁。从此木鹿结束了繁荣的历史。

1227年8月25日成吉思汗病逝後,由拖雷監国,称也可那颜。直至两年后在选举大汗的忽里臺時,拖雷和察合台等宗王们在1229年9月13日一起推举元太宗窩闊臺即大汗位。

1231年,与窩闊臺分道伐金,拖雷总右军自陝西鳳翔渡渭水,过宝鸡,入大散关。11月,蒙古军假道南宋境,沿汉水而下,经兴元(今陕西汉中)、洋州(今陕西洋县)在均州(今湖北均县西北)、光化(今湖北光化北)一带,渡汉水,迂迴北上入金境。1232年初与金军在均州(今河南禹县)遭遇。拖雷乘雪夜天寒(有一康里人作法)大败金将完颜合达、移剌蒲阿、完颜陳和尚於三峰山,尽歼金军精锐。此役毕,拖雷与自白坡渡河南下的窩闊臺军会合。

1232年农历九月,拖雷在北返蒙古草原途中逝世。據蒙古秘史,窩闊臺在一場重要戰爭中得了重疾,為了治愈窩闊臺,拖雷決定犧牲自己。巫師認為,窩闊臺所得的惡疾的病源是由中土的水和土之靈而來。水土之靈不滿蒙古人把中土臣民趕出中土,和不滿蒙古人令中土滿目瘡痍。若以中土的土地,動物和人作祭品,只會令窩闊臺的病情更加惡化,但若是他們願意犧牲家庭成員,窩闊臺便能好過來,於是拖雷主動飮了被詛咒的飲料後死亡。另一說法,是拖雷可能因酗酒過量而死。

1251年7月1日,元宪宗蒙哥即位,大蒙古国(蒙古帝国)皇位从窝阔台家族转入拖雷家族,元宪宗追尊父亲拖雷为皇帝,为拖雷追上尊谥庙号,庙号睿宗,谥号英武皇帝。

至元三年十月十八日(1266年11月16日),太庙建成,制尊谥庙号,元世祖忽必烈将父亲拖雷的谥号由英武皇帝改谥为景襄皇帝。 

至大二年十二月六日(1310年1月7日),元武宗海山为拖雷加上尊谥仁圣,从此之后,拖雷的谥号变为仁圣景襄皇帝。《睿宗皇帝加上尊谥册文》,内容如下:“伏以诣泰坛而请命,有称天以诔之文;荐清庙而致严,盖若昔相承之典。刚辰爰卜,遗美载扬。钦惟睿宗景襄皇帝孝友温恭,聪明浚哲。属我家肈造于朔土,佐圣祖遄征于四方。逮天讨之奉行,致皇威之远畅。金源假两河之息,天水渝通好之盟,遂移秦陇之师,爰有褒斜之举。既平南郑,顺流而东,再涉襄江,自上而下,乃眷三峰之捷,实开万世之基。唇既亡而齿亦寒,虢可伐而虞不腊。适英文之违豫,图中夏之底宁。毋作神羞,请以身代。爰俟金縢之启,已知宝祚之归。迪我后人,绍兹明命。徽称显号,虽已拟诸形容;玉检金泥,尚未遑于润色。奉玉册玉宝,加上尊谥曰仁圣景襄皇帝,庙号睿宗。伏惟端临扆座,诞受鸿名。亿万斯年,永锡繁祉。”

据《元史》,拖雷有子十一人:长子蒙哥,次子忽睹都、四子忽必烈、六子旭烈兀、七子阿里不哥、八子拨绰(不者克)、九子末哥、十子岁哥都、十一子雪别台,第三子和第五子失其名。

拖雷和正妻唆鲁禾帖尼所生的四子皆有所成,元宪宗蒙哥和元世祖忽必烈相继做过大元(大蒙古国)的帝王,蒙元皇帝由拖雷一系繼承。旭烈兀在西亚开创了伊儿汗国,1259年蒙哥去世后,阿里不哥在1260年在蒙古本土的庫力臺大會被部分王公推举即位,并和忽必烈争位达四年之久。

拖雷可考的女儿有二:赵国公主独木干,下嫁汪古部聂古得;鲁国公主也速不花,下嫁弘吉剌部斡陈。

民国官修正史《新元史》柯劭忞的评价是:“周公金縢之事,三代以后能继之者,惟拖雷一人。太宗愈,而拖雷竟卒,或为事之适然,然孝弟之至,可以感动鬼神无疑也。世俗浅薄者,乃疑其诬妄,过矣!”

\subsection{太宗窝阔台生平}

窝阔台汗(1186年11月7日-1241年12月11日),又作斡歌歹、和歌台、倭闊岱等,孛儿只斤氏,成吉思汗第三子,大蒙古国大汗。他是蒙古帝国第二位大汗,1229年9月13日—1241年12月11日在位,在位12年零3个月。他登基时接受大汗的称号,和诸汗相区别。

至元三年(1266年)十月,太庙成,元廷追尊庙号太宗,谥英文皇帝。

1229年9月13日(农历八月二十四日),窝阔台在库里尔台大会中被察合台、拖雷、铁木哥斡赤斤等宗王和大臣推举为大蒙古国大汗,管理整個蒙古帝国,有史料载诸宗王和百官为窝阔台上尊号曰木亦坚合罕(合罕为大汗的别译)。

他继承父親的遺志擴張領土,主要是繼續西征和南下中原。他在位期間成功完全征服中亞和華北。内政方面,以契丹人耶律楚材為相管理华北和中原地区,在这些地区稍微改变了战后屠城作風,保存不少金朝遺民和政治制度;同時又依耶律楚材建議,提拔漢人為官,整頓內治,安定了蒙古在華北地區的统治,使华北地区经济在戰後得到一定程度的恢复性發展,為日後忽必烈称帝滅南宋打下基礎。

灭金取中原:1229年登基的时候,大蒙古国在东亚部分的东南部大体以黄河为界,金朝领土基本上只剩下黄河以南的河南、陕西等地(当时的黄河取道江苏北部的淮河入海)。

1231年,窝阔台与其四弟拖雷分道进攻金朝,1232年初,拖雷率蒙古军在河南三峰山战胜金军,尽歼金军精锐。其后,拖雷与自白坡渡河南下的窝阔台军会合,一同北返蒙古草原,1232年农历九月,拖雷於北返途中病死之後,托雷四子忽必烈继承了他在华北地区的势力。

1232年春,蒙古军队继续南下,抵达金朝首都燕京(今北京市)附近,因此周围州县难民纷纷逃入汴京(今河南开封市),城中人口激增,而入夏后瘟疫流行,死者达九十餘万人。1232年秋,蒙古派使者入城要求金朝投降,被金朝将士所杀,蒙古军于是不再议和,击溃金朝援军,围困汴京城。

1233年2月6日(农历十二月二十六日),金哀宗和后妃们分别离开汴京,一路向南。1233年2月26日(农历正月十六日),金哀宗抵达归德(今河南商丘市),随后又出走;8月3日(农历六月二十六日),金哀宗逃到蔡州(今河南汝南县),在此地稳定下来。

1233年3月5日(农历正月二十三日),金朝汴京西面元帅崔立率军队杀死汴京的留守將領完颜奴申和完颜习捏阿不,控制全城,派使者向蒙古军统帅速不台投降。

1233年3月10日(农历正月二十八日),速不台向汴京进兵。速不台得知崔立同意投降后,因为之前进攻汴京时金人抗拒持久導致军队死伤甚多,便向窝阔台奏报建议軍隊入城后屠城泄愤。中书令耶律楚材坚决反对,他认为将士辛苦奋战为的就是土地和人民,屠城会导致得地无民,而且“奇巧之工,厚藏之家”都集中在汴京,屠城会导致一无所获,没有人民就没有人向朝廷交纳赋税,军队会白辛苦一场,最后窝阔台采纳了耶律楚材的意见,只关押了金朝宗室,其他人一概赦免。当时在汴京城中躲避兵祸的147万名居民因为耶律楚材的建议得以免于兵祸。

1233年5月29日(农历四月十九日),崔立将汴京城中的金朝宗室梁王完颜从恪、荆王完颜守纯以及其他宗室男女五百余人送到速不台军队驻地青城,速不台将他们送到漠北草原窝阔台的行銮驻跸之处,窝阔台为报祖先之仇(金熙宗当年曾将蒙古俺巴孩汗钉死在木驴上),将他们全部处死。同一天,崔立面见速不台,正式归降大蒙古国,速不台率军进入汴京,维护城中秩序,并将城中的金朝后妃和宗庙宝器也送到漠北草原窝阔台的行銮驻跸之处。

1234年2月9日(农历正月十日),大蒙古国军队與南宋军队联合攻入蔡州(今河南汝南县),金哀宗自杀,金末帝死于乱军之中,金朝灭亡。整个北方中原地区并入大蒙古国版图。

自1234年窝阔台汗灭金朝,到1368年烏哈噶圖汗 (元惠宗)逃离大都回到草原,由蒙古族建立的蒙古汗国、元帝國两政权,總共统治北方中原黄河流域长达134年。

端平入洛与蒙宋开战:1233年5月29日蒙古军队取得汴京(今河南开封市)后,继续进攻蔡州(金哀宗所在地),由于金朝军队抵抗顽强,为了减少损失,窝阔台决定联合南宋政權攻克蔡州灭亡金朝。

按照蒙宋双方协议,蒙宋联军攻克蔡州后,南宋可以取得蔡州未破前尚在金朝控制的河南土地,也就是唐、邓、蔡、颍、宿、泗、徐、邳等州(均位于河南南部)。这些州位于金朝和南宋的交界地带,属于金朝领土最南端的州。

在1234年2月9日蒙宋联军攻克蔡州灭亡金朝后,因为河南一带久经战火,田地荒芜,缺乏粮食,当时又正值冬季,天气严寒,于是把当地大部分居民暂时迁往河北一带,准备等天气转暖后将居民再陆续迁回河南,并恢复农业生产。同时军队久经战事,也需要休整,大部分军队撤到黄河以北。

宋理宗在部分大臣的怂恿下违背当初的蒙宋协议,1234年六月,宋军分二路出兵北伐,准备收复当年被金朝攻取的三京:西京河南府(今河南洛阳市)、东京开封府(今河南开封市)、南京应天府(亦称之为归德,今河南商丘市),这三京均位于河南北部,在蒙宋协议之前就已经被蒙古军队攻取,自然不属于当初蒙宋协议中灭金后南宋可以得到的领土。

由于宋军北上攻取三京发生在宋理宗端平年间,史称“端平入洛”。端平入洛揭开了蒙古与南宋对峙,连续四十余年不断战争的序幕,直到忽必烈渡过长江、灭亡南宋。

中國南宋违背蒙宋协议,大举进兵,但因为蒙古灭金后,大部分金朝军队和居民都已经撤到黄河以北,南宋军队最初进展顺利,一个月后顺利占领几乎是空城的三京。由于三京缺乏粮草,宋军携带粮草较少又缺乏后勤补给,蒙古军队又随后发起反击,宋军很快撤离三京,并撤回南宋境内。

蒙古军队随后追至原金朝和南宋的边界线一带,并向南宋边界的州县发起进攻,因为蒙古军队并不是很适合南方河流密布的地形作战,在取得一定战果后撤回中原。

自南宋违约进攻蒙古,端平入洛以后,南宋天灾人祸接连不断,国力逐渐衰弱直至灭亡。在军事上,收复三京失败,损兵折将,士气不振,将心不稳,成为南宋守边士兵面临的严重问题。

在中国北方实施“以儒治国”:1230年,有近臣别迭等人向窝阔台上奏,认为“汉人无补于国,可悉空其人以为牧地。”主张将汉人驱逐,把汉地的耕地变为牧场,耶律楚材则上奏请求均定中原地税、商税、盐、酒、铁冶、山泽之利,每年可得赋税白银50万两、帛8万匹、粟40余万石,足以支持窝阔台南征金朝的军队所需,窝阔台同意由耶律楚材试行。

1230年农历十一月,耶律楚材奏请在大蒙古国统治的黄河以北的河北、山西、山东等地(当时金朝尚未灭亡,黄河取道江苏北部的淮河入海)设立燕京等地设立十路征收课税使,并选用有名的儒士作为课税官员,得到窝阔台批准。

1231年农历八月,窝阔台到达云中(今山西大同市),十路征收课税使将当年征收到的汉地赋税簿册和金帛陈于廷中,窝阔台大悦,当日设立中书省,改侍从官名,以耶律楚材为中书令,粘合重山为左丞相,镇海为右丞相。

1235年春,窝阔台决定在哈拉和林建都城,修建万安宫;并部署伐南宋、征高丽和再次西征;1236年正月,万安宫建成。窝阔台大宴群臣,同月,窝阔台下诏发行纸币交钞。

1234年正月灭金朝后,窝阔台下诏括编汉地户籍,他接受耶律楚材的建议,以按户为单位收取赋税。由中州断事官失吉忽秃忽主持。1236年八月,括户完成,括得汉地民户110余万户。

1236年括户完成后,失吉忽秃忽主张按以往风俗在中原对诸王和有功之臣进行分封,窝阔台表示同意。耶律楚材力陈“裂土分民”的弊害,使窝阔台同意封地的官吏须朝廷任命,除常定赋役外,诸王勋臣不得擅自征敛,以限制诸王勋臣在封地的权力。

括户完成后,耶律楚材制订了中原赋税制度:每两户出丝一斤,上交朝廷,以供中央政府使用,每五户出丝一斤,以与所赐之家;先由中央政府征收,然后赐予该受封贵族,除此之外贵族不得擅加征敛。上田每亩税三升半,中田三升,下田二升,水田五升;商税三十分之一;盐每银一两四十斤。

这个赋税的定额是比较轻的,有利于当时已遭破坏的中原地区休养生息。在遇到大的灾情时,楚材还采取免征的措施。如果部分地区出现逃亡浮客,他们的赋税要由留下的主户负担,这些主户负担的赋税会重一些。此外,民户们也要负担一些随意性很大的杂泛差役。总的来说,民户们的负担还是相对比较轻的。

在耶律楚材的努力下,中原及北方的经济得到了恢复和保存。

1230年耶律楚材制定课税格,1231年收取的各种赋税中,白银为50万两,1234年灭金朝取得河南等地,赋税收入一直在增加,到了1238年,朝廷在中原汉地收取的各种赋税中,白银为110万两。丝和米等赋税也有显著增加。

1233年,为了培养蒙汉双语翻译类人材,窝阔台下诏在燕京(今北京市)建国子学,派遣蒙古人子弟18人学习汉语;汉人子弟12人,学习蒙古语和弓箭,并选儒士为教读。规定受业学生不仅要学习汉人文书,还要“兼谙匠艺,事及药材所用、彩色所出、地理州郡所纪,下至酒醴麴蘖、水银之造,饮食烹饪之制,皆欲周览旁通”。当时,全真教在燕京势力很大,儒家士大夫有很多托庇于全真教。燕京的学宫也是如此,学宫的主持者除杨惟中之外,葛志先、李志常均为当时有名的全真道士。

1233年农历四月,蒙古军队进入汴京城(今河南开封市),中书令耶律楚材向窝阔台奏请遣人入城,求孔子家族後代,得五十一代孙元措,奏袭封衍圣公,付以孔林庙地。耶律楚材又派人入汴京,挑选了大量的人才。

1233年农历六月,窝阔台下诏,以孔子五十一世孙孔元措袭封衍圣公。

1233年冬天,窝阔台敕修燕京孔子庙及浑天仪。

1236年农历三月,复修孔子庙及司天台。

1236年农历六月,耶律楚材奏请窝阔台同意后,在燕京(今北京市)建立编修所,在平阳(今山西临汾市)建立经籍所,主持经史类书籍的编纂和刊行,召儒士梁陟充长官,以王万庆、赵著副之。让他们直释九经,进讲东宫。又率大臣子孙,执经解义,使他们知道圣人之道。

1237年,窝阔台下旨蠲免孔子、孟子、颜子等儒教圣人子孙的差发杂役。

1237年,耶律楚材奏请对儒士举行科举考试,这就是1238年举行的戊戌选试,共录取4030人,皆当时的名士。

1238年,耶律楚材又支持杨惟中和姚枢在燕京建立太极书院,请赵复等人为师教授儒家的经典。南宋名士赵复的讲学,使程朱理学在北方中原地区传播开来。

戊戌选试:1234年2月9日,蒙古帝国灭金朝,夺取中原地区后,急需人才治理国家。

元太宗九年农历八月二十五日(1237年9月15日),根据中书令耶律楚材的建议,窝阔台下诏书命断事官术忽德和山西东路课税所长官刘中,历诸路考试,试诸路儒士,开科取士,并对考试内容和参加考试者的身份要求以及中选者的优厚待遇作了详细说明。

北方中原地区的诸路考试,均于1238年(戊戌年)举行,史称“戊戌选试”。

1238年的这次考试共录取东平杨奂等4030人,皆为一时名士,使得朝廷及时得到了加强统治所需要的各方面的人才。但后来“当世或以为非便,事复中止”。

直到元仁宗1313年下诏恢复科举,此时距离元太宗1238年的“戊戌选试”已经有75年,天下读书的士人至此再次获得以科举方式晋身做官的途径。

西征欧洲:1234年2月9日金朝灭亡后,由於大蒙古国與南宋接壤,使雙方的衝突日漸加劇,也拉開了雙方往後45年不斷爭戰的序幕。在南方戰線僵持不下之時,蒙古大軍的鐵蹄轉往東方的高麗,並使之臣服,西線方面,以拔都為首的欽察汗國,完全控制了罗斯,並繼續西進,佔領了除诺夫哥罗德以外俄羅斯的领土,以及波蘭和匈牙利的一部。

1241年12月11日(农历十一月八日),窝阔台因為酗酒而突然暴斃,使他的西征進程被逼中止。當時大軍正朝往維也納推進,但為了趕返參加位於蒙古的库里尔台大会而急忙撤軍,自此以後,蒙古大軍再也沒有踏足這片土地。

1241年年底,在窝阔台去世后不久,他的二哥察合台去世。

窝阔台去世后,1242年春天,皇后乃马真后开始称制,处理朝政,直到1246年8月24日窝阔台之子貴由繼任大汗為止。乃马真后临朝称制期间,朝政比较混乱,中书令耶律楚材力争而不能有效果,于1244年农历五月忧愤而死。

元朝重臣郝经在中统元年(1260年)农历八月给元世祖忽必烈的上书《立政议》中对元太宗窝阔台的評價是:“当太宗皇帝临御之时,耶律楚材为相,定税赋,立造作,榷宣课,分郡县,籍户口,理狱讼,别军民,设科举,推恩肆赦,方有志于天下,而一二不逞之人,投隙抵罅,相与排摈,百计攻讦,乘宫闱违豫之际,恣为矫诬,卒使楚材愤悒以死。”(说明:元太宗窝阔台在世之时,耶律楚材还是深受重用的,1241年元太宗去世,帝位空缺,皇后乃马真后开始临朝称制,朝政比较混乱,中书令耶律楚材力争而无效果,他于1244年忧愤而死)

明朝官修正史《元史》宋濂等的評價是:“帝有宽弘之量,忠恕之心,量时度力,举无过事,华夏富庶,羊马成群,旅不赍粮,时称治平。”

清朝史学家邵远平《元史类编》的評價是:“册曰:嗣业恢基,缵绪立制;五载灭金,十路命使;定赋崇儒,用昌厥世;仁厚恭俭,时称平治。”

清朝史学家毕沅《续资治通鉴》的評價是:“太宗性宽恕,量时度力,举无过事。境内富庶,旅不赍粮,时称治平。”

清朝史学家魏源《元史新编》的評價是:“帝有宽宏之量,淳朴之质,乘开国之运,师武臣力,继志述事,席卷西域,奄有中原。惟知诸子不材,又知宪宗之克荷,而储位不早定,致身后政擅宫闱,大业几沦,有余憾焉。”

清朝史学家曾廉《元书》的評價是:“论曰:太宗时金人已弱,然犹足阻河为固也。太宗遵遗令戡凤翔,道兴元,以达唐邓,而汴梁墟,可谓闻斯行之矣。当是时,操持国政,耶律楚材郁为时栋。然太宗之用楚材,以利也。太宗言利,楚材即以其利利天下,而纪纲粗立矣。用相违也,而相成也,岂非天哉!故开国之运,云龙风虎,非雷同也。”

清末民初史学家屠寄《蒙兀儿史记》的評價是:“论曰:财者,一国所公有也。语曰:百姓足,君孰与不足?人君以国用困乏,多取于民,然且不可。况可纵奸人异类,恣其侵夺乎?斡歌歹汗初得金,许奥都剌合蛮扑买中原银课,举国家财政大权授之贾胡之手,公利而私取之,上下交损焉。封建之制,始于自然,强并弱,众暴寡。自天子以至食采之大夫,各私其土地人民。古圣王不得以而仍之。秦汉以降,此制渐废,偶一行之,罔不召乱。自非至无识者,不轻议复也。汗括汉户,分赐诸王贵戚,其视无辜之民与奴虏奚择。彼固不知封建为何事,然斯制若行,弊且甚于封建。微耶律楚材言,纵虎豹而食人肉矣。前史称汗有宽仁之量,忠恕之心,度时量力,动无过举。迹其立站赤、选税使、试儒士、释俘囚,诏免旱蝗之租,代偿羊羔之息,固非无志于民者,惜乎不达怡体,而左右之人将顺其美者,又寡也。”

民国官修正史《新元史》柯劭忞的評價是:“太宗宽平仁恕,有人君之量。常谓即位之后,有四功、四过:灭金,立站赤,设诸路探马赤,无水处使百姓凿井,朕之四功;饮酒,括叔父斡赤斤部女子,筑围墙妨兄弟之射猎,以私撼杀功臣朵豁勒,朕之四过也。然信任奥都拉合蛮,始终不悟其奸,尤为帝知人之累云。”

\subsection{定宗貴由生平}

貴由汗(1206年-1248年4月),大蒙古国第三任大汗,孛儿只斤氏,窩闊台長子,乃馬真后所生,1246年8月24日—1248年在位,计2年。

至元三年(1266年)十月,太庙建成,追尊庙号定宗,谥简平皇帝,在宗庙中列祭于第七室,排在忽必烈之父托雷后、忽必烈之兄蒙哥前。

早年參加征伐金朝,俘虜了其親王;又曾經参与西征欧洲。蒙古帝国第三任大汗贵由、第四任大汗蒙哥,以及后来的元朝开国皇帝忽必烈,堂兄弟三人都是蒙古第二次西征时拔都的部下。

1241年12月11日,窝阔台去世,汗位虚悬,贵由的母亲乃马真脱列哥那称制,法纪混乱,很多宗王贵族滥发牌符征敛财物,唯有拖雷家族没有这样做,赢得了声誉。乃马真后欲立长子贵由为大汗,拔都与贵由不和,一直不肯参加选汗大会,后来,成吉思汗幼弟铁木哥斡赤斤也领兵来争位,帝国面临汗位争夺战和混乱的危险。拖雷的遗孀唆鲁禾帖尼决定率诸子参加忽里勒台大会,1246年8月24日,宗王大臣们拥立贵由登基,贵由成为大蒙古国大汗,“全体宗王们脱帽,解开宽腰带,把贵由扶上金王位,以汗号称呼他,到会者对新君九拜表示归顺,在帐外的藩王及外国使臣等也同时跪拜称贺。”

贵由登基后,虽然本人很有权威,但是因沉湎酒色、手足痉挛,并没有什么作为,且不理政事,多委于下臣。

1248年春,貴由親率大軍西征拔都,至橫相乙兒(今新疆青河縣東南)病死。一說被拔都系势力毒殺。

1246年8月24日至1248年4月20日在位,在位仅一年零八个月。

和罗马教宗的交往:贵由在位期间和罗马教宗有交往。歐洲諸國傳言蒙古大汗信仰基督教,因此教宗诺森四世派遣若望·柏郎嘉宾出使,希望勸說蒙古大汗不要傷害基督徒,同時要他深入了解蒙古人的風土民情、作戰方式等。1245年4月16日从法国里昂出发,途经神圣罗马帝国、波兰王国和基辅罗斯等国(他于1246年2月3日离开基辅)。1246年4月4日,他在伏尔加河下游的萨莱(今伏尔加河下游阿斯特拉罕附近)受到钦察汗拔都的接见。拔都派他去蒙古草原见大汗,他经过讹答剌、伊犁河下游、叶密立河—翻越阿尔泰山,向东抵达蒙古草原。

1246年7月22日,他抵达距离哈拉和林只有半天路程的地方,选举大汗的忽里勒台大会正在此召开。他目睹了1246年8月24日贵由的当选,并留下了对贵由的生动描述:“在他当选时,约有四十,最多四十五岁。他是中等身材,非常聪明.极为精明,举止极为严肃庄重。从来没有看见他放声大笑,或者是寻欢作乐。” 最後他未能说服贵由皈依天主教,得到贵由的回信后,于1246年11月13日离开蒙古草原,向西踏上归途,经伏尔加河下游的拔都驻地返回西方,1247年9月5日他到达拔都驻地,又经基辅返回西方。

凉州会盟与吐蕃归附:1247年,吐蕃诸部宗教界领袖萨迦班智达·贡嘎坚赞(简称萨班)同大蒙古国皇子西凉王阔端(贵由之弟,窝阔台之子,成吉思汗之孙)在凉州(今中国甘肃武威市)议定了吐蕃归附的条件,其中包括呈献图册,交纳贡物,接受派官设治,吐蕃地区纳入大蒙古国(蒙古帝国)治下,史称“凉州会盟”。

窝阔台家族的衰落:根据《新元史》记载,1248年农历三月(1248年4月),贵由以养病为名带兵西巡,途中病逝于横相乙儿(今新疆青河东南),距離別失八里一天路程。

贵由死后,其遗孀斡兀立海迷失临朝称制,由於贵由与拔都早年不和,拔都拒絕奔喪。为了对抗窝阔台家族,拔都以长支宗王的身份遣使邀请宗王、大臣到他在中亚草原的驻地召开忽里台,商议推举新大汗。窝阔台系和察合台系的宗王们多数拒绝前往,海迷失后只派大臣八剌为代表到会。唆鲁禾帖尼则命长子蒙哥率诸弟及家臣应召前往。

1250年,庫力臺大會在中亚地区拔都的驻地召开,拔都在会上极力称赞蒙哥能力出众,又有西征大功,应当即位,并指出贵由之立违背了窝阔台遗命(窝阔台遗命失烈门即位),窝阔台后人无继承汗位的资格。大会通过了拔都的提议,推举蒙哥为大汗。窝阔台、察合台两家拒不承认,唆鲁禾帖尼和蒙哥又遣使邀集各支宗王到斡难河畔召开忽里台,拔都派其弟别儿哥率大军随同蒙哥前往斡难河畔,但窝阔台、察合台两家很多宗王仍不肯应召,大会拖延了很长时间。

由于蒙哥的母亲唆鲁禾帖尼的威望甚高,并且善于笼络宗王贵族,多数宗王大臣最终应召前来,1251年农历六月在蒙古草原斡难河畔举行庫力臺大會,元宪宗元年农历六月十一日(1251年7月1日),宗王大臣们共同拥戴蒙哥即大汗位。此后,为了巩固汗位,唆鲁禾帖尼在镇压反对者时毫不留情,并亲自下令处死贵由的皇后斡兀立海迷失。

自此汗位繼承,便由窝阔台家族转移到了拖雷家族,皇族内部的分裂,为后来大蒙古国的彻底分裂,埋下伏筆。

明朝官修正史《元史》宋濂等的評價是:“三年戊申春三月,帝崩于横相乙儿之地。……是岁大旱,河水尽涸,野草自焚,牛马十死八九,人不聊生。诸王及各部又遣使于燕京迤南诸郡,征求货财、弓矢、鞍辔之物,或于西域回鹘索取珠玑,或于海东楼取鹰鹘,驲骑络绎,昼夜不绝,民力益困。然自壬寅以来,法度不一,内外离心,而太宗之政衰矣。”

清朝史学家毕沅《续资治通鉴》的評價是:“自太宗皇后称制以来,法度不一,内外离心。至是国内大旱,河内尽涸,野草自焚,牛马死者十八九,人不聊生。诸王及各部,又遣使于诸郡征求货财,或于西蕃、回鹘索取珠玑,或于东海搜取鹰、鹘、驿骑络绎,昼夜不绝,民力益困。皇后立库春子实勒们听政,诸王大臣多不服。”

清朝史学家魏源《元史新编》的評價是:“连岁大旱,河水尽涸,野草自焚,牛马十死八九,人不聊生。诸王及各部又遣使于燕京迤南诸郡,征求货财,或于西域、回鹘索取珠玑,或于海东搜取鹰鹘,驿骑不绝,内外离心,故无可纪。然自太祖崩后,拖雷监国者一年,太宗崩后,六皇后称制者四年,定宗之后,皇后临朝者又几四年,前后凡九载无君而国不乱,卒能创业垂统,上竝漢、唐者,则皆宗王宿将维持拱卫,根干蟠据之力。”

清朝史学家曾廉《元书》的評價是:“论曰:定宗之世,事多缺漏,而前史曰:‘ 帝崩之岁大旱,河水尽涸,野草自焚,牛马十死八九,人不聊生。诸王及各部又遣使于燕京迤南诸部,征求货财、弓矢、鞍辔,或于西域回鹘索取珠玑,海东索取鹰鹘,驿骑络绎,昼夜不绝,民力益困。然自壬寅以来,法度不一,内外离心,而太宗之政衰矣。’其言壬寅,盖以昭慈皇后称制时言之也。夫定宗即位时,年四十矣,而不能辑诸王侯大将,纪解威亵,此太宗之不谋付以匕图者乎?然在于汉亦孝惠之亚也。惟无良臣为之辅弼,而宗藩党羽遂成,以夺皇阼。炎异之丛,兴其足信耶?而失烈门则太宗遗诏所立也。前史复曰:定宗崩后,三岁无君。蒙哥之党之不欲以为君,非蒙古之无君也。窜之北陲,并逐太宗皇后而弑定宗皇后,可不谓之逆哉!自是而太宗子孙亦不欲以蒙哥兄弟为君,逮于海都,而中原震矣。”

中華民国史学家屠寄《蒙兀儿史记》的評價是:“汗严重有威,临御未久,不及设施,惟乃蛮真可敦称制时,威福下移,汗既亲政,纲纪粗立,君权复尊,自幼多疾,成吉思汗尝命亦鲁王之祖忽鲁扎克为之主膳。中年性好酒色,手足有拘挛之病,在位之日,常以疾不视事,事多决于大臣镇海、合答二人云。”

中華民国官修正史《新元史》柯劭忞的評價是:“史臣曰:定宗诛奥部拉合蛮,用镇海、耶律铸,赏罚之明,非太宗所及。又乃马真皇后之弊政,皆为帝所铲革。旧史不详考其事,谓前人之业自帝而衰,诬莫其矣。” 

\subsection{宪宗蒙哥生平}

蒙哥汗(1209年1月10日-1259年8月11日),大蒙古国第四任大汗,也是大蒙古国分裂前最後一個受普遍承認的大汗。他是成吉思汗幼子拖雷的长子、窝阔台的养子,由窝阔台的昂灰皇后抚养长大。

1251年7月1日登基,在位8年零2个月。其间长期主持对南宋、大理的战争,为其弟忽必烈最终建立元朝奠定坚实基础。至元三年(1266年)十月,太庙成,元廷追尊蒙哥庙号为宪宗,谥桓肃皇帝 。

潜邸岁月:1209年1月10日(农历戊辰年十二月三日),蒙哥生于漠北草原,是成吉思汗之孙,拖雷的长子,拖雷正妻唆鲁禾帖尼所生的嫡长子(元世祖忽必烈是嫡次子,旭烈兀是嫡三子,阿里不哥是嫡四子)。窝阔台汗即位之前,以蒙哥为养子,让昂灰皇后抚育蒙哥,并在他长大后,为他娶火鲁剌部女子火里差为妃、分给他部民。至1232年拖雷去世后,蒙哥才回去继承拖雷的封地。蒙哥多次跟随窝阔台参加征伐,屡立奇功。蒙哥沉默寡言、不好侈靡,喜歡打獵。1235年,蒙哥参加第二次蒙古西征,與拔都、貴由西征欧洲的不里阿耳、欽察、斡羅思等地,屢立戰功,在里海附近,活捉钦察首领八赤蛮。

拖雷家族争得大汗之位:1248年农历三月贵由汗去世后,由皇后斡兀立海迷失临朝称制;由於与贵由早年不和,拔都(铁木真长子术赤之子)拒絕奔喪。为了对抗窝阔台家族,拔都以长支宗王的身份遣使邀请宗王、大臣到他的驻地(在中亚草原)召开忽里台(蒙古的军政会议),商议推举新大汗。窝阔台系和察合台系的宗王们多数拒绝前往,贵由汗的皇后斡兀立海迷失只派大臣八剌为代表與会。托雷之妻唆鲁禾帖尼则命长子蒙哥率诸弟及家臣应召前往。

1250年,忽里台大会在拔都的驻地(中亚地区)召开,拔都在会上极力称赞蒙哥能力出众,又有西征大功,应当即位,并指出贵由之立违背了窝阔台遗命(窝阔台遗命失烈门即位),窝阔台后人不当有继承汗位的资格。大会通过了拔都的提议,推举蒙哥为大汗。窝阔台、察合台两家拒不承认,唆鲁禾帖尼和蒙哥又遣使邀集各支宗王到斡难河畔召开忽里台,拔都派其弟别儿哥率大军随同蒙哥前往斡难河畔,但窝阔台、察合台两家的很多宗王仍不肯应召,大会拖延了很长时间。

由于唆鲁禾帖尼威望甚高,并且善于笼络宗王贵族,最终多数宗王大臣应召前来,1251年农历六月在蒙古草原斡难河畔举行忽里台,宗王大臣们于7月1日(农历六月十一日)共同拥戴蒙哥登基,蒙哥成为大蒙古国大汗;蒙哥即位的当日,尊母亲唆鲁禾帖尼为皇太后。此后,为了巩固汗位,皇太后唆鲁禾帖尼镇压反对者毫不留情,并亲自下令处死贵由汗的皇后斡兀立海迷失。

自此“大汗”之位的繼承,便由窝阔台家族转移到了拖雷家族,为后来大蒙古国分裂埋下伏筆。

1251年7月1日,蒙哥即位后,窩闊台系諸宗王拒絕承認,被蒙哥率兵鎮壓;蒙哥又以其弟忽必烈统領漠南漢地軍政事務,同时指挥向南(东亚)、向西(西亚)两个方向的征服战争。

征服大理:1252年农历六月,命弟忽必烈南征大理国,次月,忽必烈率军出发。1253年农历八月,忽必烈军至陕西,开始进攻位于今云南等地的大理国。1254年1月2日(元宪宗三年农历十二月十二日),忽必烈攻克大理城,大理国王段兴智投降,大理国灭亡,并入大蒙古国版图。1256年,段兴智前往漠北和林觐见蒙哥汗,被任命為大理總管,子孙世襲。

从1254年大蒙古国忽必烈奉命灭大理国、大理国王战败投降,到1382年驻守云南的元朝梁王把匝剌瓦尔密兵败自杀、元朝大理总管段世战败归降明军,蒙古族建立的政权统治云南地区长达128年。

远征西亚:元宪宗三年(1253年)六月,蒙哥命弟旭烈兀率大军十万西征。旭烈兀的西征军从漠北草原出发,1256年大军渡过阿姆河后所向披靡,先攻灭波斯南部的卢尔人政权,1256年攻灭位于波斯西部的木剌夷国(阿萨辛派),1258年灭亡巴格达的阿拔斯王朝,1260年3月1日,灭亡叙利亚的阿尤布王朝,并派兵攻占了小亚细亚大部分地区。

攻占叙利亚后,旭烈兀西征军兵锋抵达今天地中海东岸的的巴勒斯坦地区,即将与埃及的马木留克王朝交战,此时旭烈兀得到使者带来的帝国最高统治者蒙哥在四川去世的消息,于是只派先锋怯的不花率不到一万军队驻守叙利亚,自己率大军开始东返。1260年9月3日,埃及马木留克王朝趁着旭烈兀攻率主力东返,攻占叙利亚,杀怯的不花,旭烈兀愤怒至极,本想率军继续西征,但此时他和钦察汗国的别儿哥汗因为争夺阿塞拜疆爆发了战争,只好结束西征。

旭烈兀东返途中得到忽必烈和阿里不哥争位的消息,于是留在西亚,自据一方,并宣布支持忽必烈,后来被忽必烈封为“伊儿汗”,西亚的伊儿汗国从此建立。

征伐南宋:1258年,蒙哥、其弟忽必烈和大将兀良合台分三路大举进攻南宋。1258年农历七月,蒙哥亲率主力进攻四川,所向披靡,攻克四川北部大部分地区,直到1259年初在合州(今重庆合川区)釣魚城下攻势受阻,战事胶着数月,蒙哥死前最终未能完成此次战役;而蒙哥死后,忽必烈得知忽里台大会选举阿里不哥即位,匆匆率军赶回漠北争夺汗位,对南宋的征伐计划暂时搁置。

蒙哥的去世原因,至今史学界尚无明确结论。主要有以下几说:

战争中受伤不治身亡:《合州志》記載,1259年8月11日(农历七月二十一日),蒙哥在合州钓鱼山一役,被南宋軍投石機的巨石打中,六天後傷重而亡。《馬可波羅游記》和明萬歷《合州志》则记载蒙哥在攻打合州时被釣魚城守城武器矢石擊中而重傷后去世。翦伯赞主编的《中国史纲要》采取了这种说法,书:“蒙古军因军中痢疾盛行,死伤极多,蒙哥汗又为宋军的飞矢射中身死”。《古今圖書集成》中的《釣魚城記》则记載:“炮風所震,因成疾。班師至愁軍山,病甚……次過金劍山溫湯峽(今重慶市北碚北溫泉)而歿”,謝士元在《遊釣魚山詩序》亦說蒙哥是“炮風致疾”而死。

病逝:《元史》则称天气多雨,蒙哥身体不适,于农历七月癸亥日死在钓鱼山 蒙古帝国伊儿汗国宰相拉施特的《史集》也推斷當時正值酷暑季节,军中痢疾流行,蒙哥亦染病身亡。畢沅在《續資治通鑑》稱蒙哥死於痢疾。

其他说法有:黃震的《古今紀要逸編》認為蒙哥因為屢攻合州釣魚城不克,致憂憤死;《海屯紀年》說是落水死。

据传蒙哥臨終前留下遺言,將來若攻下釣魚城,必屠殺全部軍民百姓;然而此事《元史》、《新元史》、《史集》均无记载(此三本史书记载蒙哥病逝,和钓鱼城的战斗无关)。後來釣魚城於1279年投降時,忽必烈赦免了所有軍民。

蒙哥的去世,对当时的蒙古帝国政局乃至世界格局都有極大的影響:蒙哥去世导致了旭烈兀统帅的第三次蒙古西征被迫中止;随后爆发了其弟忽必烈与阿里不哥争夺汗位之战,最终导致大蒙古国(蒙古帝国)的分裂。

法国国王路易九世派遣传教士卢布鲁克前往东方觐见蒙古大汗商讨传教和结盟对抗阿拉伯人事宜。卢布鲁克于1253年从地中海东岸阿克拉城(今以色列海法北)出发,于1253年5月7日离开君士坦丁堡,一路东行,渡过黑海,秋天到达伏尔加河畔,谒见拔都汗。拔都认为自己无权准许他在蒙古人中传教,便派他去东方觐见大汗蒙哥。卢布鲁克觐见拔都后,留下了对拔都的生动描述:“拔都坐在一金色的高椅上,或者说坐在像床一样大小的王位上,须上三级才能登上宝座,他的一个妻子坐在他旁边。其余的人坐他的右边和这位妻子的左边。”

1253年12月,卢布鲁克到达哈拉和林南部蒙哥冬季营地。1254年1月4日觐见蒙哥,并留下了对蒙哥的生动描述:“我们被领入帐殿,当挂在门前的毛毡卷起时,我们走进去,唱起赞美诗。整个帐幕的内壁全都以金布覆盖着。在帐幕中央,有一个小炉,里面用树枝、苦艾草的根和牛粪生着火。大汗坐在一张小床上,穿着一件皮袍,皮袍像海豹皮一样有光泽。他中等身材,约莫45岁,鼻子扁平。大汗吩咐给我们一些米酒,像白葡萄酒一样清澈甜润。然后,他又命拿来许多种猎鹰,把它们放在他的拳头上,观赏了好一会。此后他吩咐我们说话。他有一位聂思托里安教(景教)徒作为他的译员。”

1254年4月5日,随同蒙哥来到大蒙古国首都哈拉和林。8月18日带着蒙哥致路易九世的国书西归,信中写道:“这是长生天的命令。天上只有一个上帝,地上只有一个君主,即天子成吉思汗。”蒙哥以长生天以及它在地上的代表“大汗”的名义命令法兰西国王承认是他的属臣。

他于1255年回到地中海东岸。一年后,他用拉丁文写成的出使报告交给路易九世,即《东方行记》,又称《卢布鲁克游记》。

小亚美尼亚国王海屯一世于1244年归附大蒙古国,成为属国。1254年春,海屯一世遵从拔都汗之命亲自前往蒙古草原觐见大汗蒙哥。他与随臣一路东行,5月至拔都营帐(伏尔加河下游)谒见,然后继续东行,9月13日到达蒙哥汗廷(哈拉和林)朝见、献贡,得到蒙哥颁赐的诏书;“诏书上盖有蒙哥的御玺,不许人欺凌他及他的国家。还给他一纸敕令,允许各地教堂拥有自治权。”在哈拉和林停留50天后,他离开汗廷西还。

返回途中在中亚河中地区觐见蒙哥汗之弟弟旭烈兀,行程8个月,1255年7月返抵小亚美尼亚。回国后撰写《海屯行纪》。

父亲:拖雷,1227年—1229年帝位空缺时担任大蒙古国监国,1232年去世。《元史·睿宗本纪》载,蒙哥即位后追尊拖雷为皇帝,为拖雷上庙号睿宗、谥号英武皇帝,1266年忽必烈改谥其为景襄皇帝,1310年元武宗海山加谥为仁圣景襄皇帝。(元朝由忽必烈建立于1271年;然而在元朝建立之前,随着蒙古对金国、西夏等沿袭了中原礼制的王朝的征服,蒙古在中国地区的统治也受到了汉文化的影响,包括任用契丹人、汉人为官,尊重儒学等,为逝者上庙号、谥号等,或是出于对汉文化的吸收,而非意味着大蒙古国时期的“大汗”等同于元朝时期的“皇帝”。元史所载宗室,在忽必烈的同辈以及先辈中,除了宗庙里奉祭的重要祖先被予以“追赠尊谥”,均没有汉式的封号;而忽必烈建立元朝之后,宗室贵族才渐渐有了诸如“鲁国公主”之类的汉式封号,可见“大蒙古国”与“元朝”实为两个政权,不过后者宣称对前者继承耳。)

母亲:唆鲁禾帖尼,是蒙哥,忽必烈,旭烈兀,阿里不哥四人的生母,1251年蒙哥汗即位后尊其为皇太后,1252年去世。1266年元世祖忽必烈为其上谥号庄圣皇后,1310年元武宗海山加谥为显懿庄圣皇后。她的四个儿子皆曾称汗称帝,被后世史学家尊称为“四帝之母”。

元朝重臣郝经在中统元年(1260年)农历八月给元世祖忽必烈的上书《立政议》中对元宪宗蒙哥的評價是:“先皇帝初践宝位,皆以为致治之主,不世出也。既而下令鸠括符玺,督察邮传,遣使四出,究核徭赋,以来民瘼,污吏滥官,黜责殆遍,其愿治之心亦切也。惜其授任皆前日害民之尤者,旧弊未去,新弊复生,其为烦扰,又益剧甚,而致治之几又失也。”

明朝官修正史《元史》宋濂等的評價是:“帝刚明雄毅,沉断而寡言,不乐燕饮,不好侈靡,虽后妃不许之过制。初,太宗朝,群臣擅权,政出多门。至是,凡有诏旨,帝必亲起草,更易数四,然后行之。御群臣甚严,尝谕旨曰:‘尔辈若得朕奖谕之言,即志气骄逸,志气骄逸,而灾祸有不随至者乎?尔辈其戒之。’性喜畋猎,自谓遵祖宗之法,不蹈袭他国所为。然酷信巫觋卜筮之术,凡行事必谨叩之,殆无虚日,终不自厌也。”

清朝史学家邵远平《元史类编》的評價是:“册曰:天象知祥,众心戴主;遐辟西南,深入中土;未究厥勳,亦振乃武;友弟因心,终昌时绪。”

清朝史学家毕沅《续资治通鉴》的評價是:“宪宗沉断寡言,不乐宴饮,不好侈靡,虽后妃亦不许之过制。初,定宗朝,群臣擅权,政出多门,帝即位,凡有诏旨,必亲起草,更易数四,然后行之。御群臣甚严,尝曰:‘尔辈每得朕奖谕之言,即志气骄逸。志气骄逸,而灾祸有不随至者乎?尔辈其戒之!’性喜畋猎,自谓遵祖宗之法,不蹈袭他国所为。然酷信巫觋、卜筮之术,凡行事必谨叩之,殆无虚日。”

清朝史学家魏源《元史新编》的評價是:“帝早亲军旅,刚明沉断,威著中外。即位以后,不乐燕饮,不好侈靡,虽后妃不许之过制。初,太宗崩后,旷纪无君,黄裳御统,政出多门,阿柄几于旁落。至是,凡有诏旨,帝必亲起草,更易数四,然后行之。御臣下甚严,尝谓:‘臣下奖谕太过,即志气骄溢,过咎随之,是害之也。’承开国师武臣力之后,西平印度,南并大理,东取巴蜀,所向无敌。惟遵其国俗,喜田猎,信巫觋卜筮,是其小蔽。使太宗即世,早承大业,则伐宋之役,不俟末年而南北混一矣。天未既宋,暑雨老师,景命不延,故大勳重集于世祖皇帝。”

清朝史学家曾廉《元书》的評價是:“论曰:宪宗之立,有遗议焉。前史袭《元史》旧文,未为允也。史又称宪宗能辑士卒,皇子阿速歹猎骑伤稼,责之,复挞其近侍。卒拔民葱,即斩以徇。在蒙古治军可谓肃矣。夫古今称强汉、弱宋,然王坚以孤城罢卒,抗毳旃之劲族,卒乃师老解退。虽宪宗不晏驾,庸必克乎?盖自平金以来,中汉人之习,锦衣玉食,肌骨疏懈。故金以是亡,而元人兵势亦自是遂稍衰矣。历观史策,暾欲谷之言,有以哉!”

民国史学家屠寄《蒙兀儿史记》的評價是:“汗刚明雄毅,沉断而寡言,不乐燕饮,不好侈靡,虽后妃不许逾制。尝有西域商胡献水晶盆,珍珠伞等物,价值银三万余锭,汗曰:‘今百姓疲弊,所急者钱耳。朕独有此何为?’却之。赛典赤以为言,乃稍偿其值,且禁嗣后勿献。初,古余克汗朝群臣擅权,政出多门。至是,凡有诏旨,汗必亲起草,更易数四,然后行之。御群下甚严,尝谕旨曰:‘汝曹若得朕奖谕,即志气骄逸,志气骄逸,灾祸有不随至者乎?汝曹戒之。’性喜畋猎,自谓遵祖宗之法,不蹈袭他国所为。然酷信巫觋卜筮之术,凡行事必谨叩之,殆无虚日,终不自厌也。”

民国官修正史《新元史》柯劭忞的評價是:“帝沉断寡言,不喜侈靡。太宗朝群臣擅权,政出多门。至是,凡诏令皆帝手书,更易数四,然后行之。御群臣甚严,尝谕左右曰:“汝辈得朕奖谕,即志气骄逸,灾祸有不立至者乎?汝辈其戒之。”然酷信巫觋卜笨之术,凡行事必谨叩之无虚日,终不自厌也。史臣曰:“宪宗聪明果毅,内修政事,外辟土地,亲总六师,壁于坚城之下,虽天未厌宋,赍志而殂,抑亦不世之英主矣。然帝天资凉薄,猜嫌骨肉,失烈门诸王既宥之而复诛之。拉施特有言:蒙古之内乱,自此而萌,隳成吉思汗睦族田本这训。呜呼,知言哉!”


%%% Local Variables:
%%% mode: latex
%%% TeX-engine: xetex
%%% TeX-master: "../Main"
%%% End:

%% -*- coding: utf-8 -*-
%% Time-stamp: <Chen Wang: 2019-10-18 15:36:16>

\section{世祖\tiny(1260-1294)}

元世祖忽必烈,清代乾隆晚期乾隆帝命改譯为呼必赉。孛儿只斤氏,為父親拖雷的第四子,母親唆鲁禾帖尼的第二子,蒙古帝国大汗,元王朝的建立者。

1260年5月5日在自己的弟弟旭烈兀的支持和封地属臣的拥立下,自立为大蒙古国大汗,称大蒙古国皇帝。1271年12月18日,忽必烈改国号为“大元”,建立元朝,成为元朝首任皇帝。忽必烈于1260年5月5日至1276年2月4日自立为汗期间实际统治中国北方及蒙古高原地区属于蒙古大汗的直辖领地,于1271年12月18日至1294年2月18日作为元朝皇帝统治中国,前后在位34年,作为全中国皇帝在位18年。

1276年2月4日,元军攻入南宋行都临安,宋恭帝奉上传国玉玺和降表,元朝成为全国性政权,但南宋遗臣建立小朝廷继续抗元。1279年3月19日,南宋海上政权残余的最后一支抵抗力量被消灭,元朝统一全中国。

1276年2月4日,宋恭帝在降表中为忽必烈上尊号大元仁明神武皇帝。1284年1月24日,群臣为忽必烈上尊号宪天述道仁文义武大光孝皇帝。

去世后,获諡號聖德神功文武皇帝,廟號世祖,蒙古語尊號薛禪皇帝。

成吉思汗十年八月二十八日(1215年9月23日),忽必烈生于漠北草原。忽必烈是成吉思汗第四子拖雷與正妻唆鲁禾帖尼所生的嫡次子(蒙哥是嫡长子,旭烈兀是嫡三子,阿里不哥是嫡四子)。忽必烈长大后,“仁明英睿,事太后至孝,尤善抚下。”忽必烈年少有大志、重视汉地的治理,早在1244年,年轻的忽必烈便招揽了搜罗了各方的文人、儒生、旧臣等,形成了一个属于自己的幕僚团

1251年7月1日(农历辛亥年六月十一日),忽必烈長兄蒙哥经忽里台选举成为大蒙古国大汗(于1264年被忽必烈追尊为元宪宗),即位后不久即任命忽必烈負責總領漠南漢地事務。忽必烈设置金莲川幕府,并在这段时间内任用了大批漢族幕僚和儒士,如劉秉忠、許衡、姚樞、郝经、张文谦、窦默、趙璧等等,并提出了“行汉法”的主张。儒士元好问和张德辉还请求忽必烈接受“儒教大宗师”的称号,忽必烈悦而受之。忽必烈尊崇儒学,“圣度优宏,开白炳烺,好儒术,喜衣冠,崇礼让。”

1252年六月,忽必烈前往草原觐见蒙哥汗,奉命率军征云南地区的大理国,为继续进攻南宋作跳板。1253年八月,忽必烈率军从陕西出发,于1254年1月2日(农历十二月十二日)攻克大理城,國王段兴智投降,大理国灭,云南地区并入大蒙古国版图。1256年,段兴智前往漠北和林皇宫觐见蒙哥,被蒙哥任命為大理總管,子孙世襲。从1254年忽必烈奉蒙哥之命灭大理国,到1382年驻守云南的元朝梁王把匝剌瓦尔密兵败自杀、大理总管段世战败归降明军,蒙古族建立的政权统治云南地区长达128年。

1256年夏天,蒙哥以南宋扣押蒙古使者为理由,對南宋宣戰,并布置了三路大军,亲自率领西路军,以忽必烈为中路军统帅。忽必烈率军抵达河南汝南,继续向南宋进发,并派命杨惟中、郝经宣抚江淮。1259年9月3日(农历八月十五日),忽必烈统领中路军渡过淮河,攻入南宋境内,随后一路向南,在湖北开辟新的战场,进攻长江中游的鄂州。

1259年8月11日,蒙哥在四川合州钓鱼山病逝。1259年9月19日,在四川的忽必烈异母弟末哥派来的使者向忽必烈宣布蒙哥去世的消息,并请忽必烈北归参与忽里台大会,以便争取汗位继承权。忽必烈则认为“吾奉命南来,岂可无功遽还?”于是进攻南宋,并多次获胜,后来,忽必烈的正妻察必派使者密报,阿里不哥已经派阿蓝答兒在开平附近调兵,脱里赤在燕京附近征集民兵,催促忽必烈早日北还。1259年11月17日,儒臣郝经上《班师议》,陈述必须立即退兵的理由,坚定了忽必烈退兵北返的决心。

忽必烈声称要进攻南宋首都临安,留大将继续对鄂州的围攻,增加对南宋的军事压力,元宪宗九年闰十一月二日(1259年12月17日),南宋丞相贾似道派使者请和,约定南宋割地求和,并且送岁币,忽必烈于是在当日撤兵北返,元宪宗九年闰十一月二十日(1260年1月4日),忽必烈率军抵达燕京(今北京市),解散了脱里赤征集的民兵,“民心大悦”。忽必烈率军在燕京近郊驻扎,度过整个冬天,并积极和诸王联络,准备在1260年春天召开庫力台大會,举行登基大典。

庚申年三月二十四日(1260年5月5日),忽必烈在部分宗王和大臣擁立下于自己的封地开平(后称上都,今内蒙古多伦县北石别苏木)自立为“大蒙古国皇帝”(即蒙古帝国大汗的汉语称谓),庚申年四月四日(1260年5月15日),忽必烈发布称帝的即位诏书《皇帝登宝位诏》,在诏书中,他自称为“朕”,称他的哥哥元宪宗蒙哥(1251—1259年在位)为“先皇”。

中统元年五月十九日(1260年6月29日),忽必烈发布《中统建元诏》,正式建年号“中统”。

庚申年(1260年)农历四月,其弟阿里不哥在哈拉和林城西按坦河被部分宗王和大臣拥立為大蒙古国大汗。幼弟阿里不哥與忽必烈為此發動戰爭爭奪汗位,双方战争时断时续,一共持续了四年之久。忽必烈于庚申年三月二十四日(1260年5月5日)自立为汗,又称汉文的“皇帝”,以招揽汉族知识分子归心,一部分汉族知识分子果然对此表示赞许,赞美忽必烈“既以正立,一时豪杰云从景附,全制本国,奄有中夏,挟辅辽右、白霫、乐浪、玄菟、秽貊、朝鲜,面左燕云、常代,控引西夏、秦陇、吐蕃、云南,则玉烛金瓯,未为玷缺。藩墙不穴,根本强固,倍半于金源,五倍于契丹。”

1260年忽必烈称帝后,控制了漠南草原,以及原金朝和西夏故地,吐蕃,云南,西域东部等地区,对阿里不哥实施经济控制。阿里不哥控制的则是漠北草原和西域西北部地区,面对匮乏的物资最终无以为继。1264年忽必烈最终迫使阿里不哥投降,完全控制蒙古帝国的东部、原本属于大汗直辖领地的大部分地区。阿里不哥归降忽必烈后,忽必烈赦免了他和跟随的诸王,只是处死了他的众多谋臣。。阿里不哥失败后郁郁寡欢,于1266年去世。

1264年8月21日(忽必烈中统五年七月二十八日)阿里不哥投降后,忽必烈实际管辖的政治版圖包括(古今地名对照):中原地区(位于长城以南、秦岭淮河以北)、东北地区(包括整个黑龙江流域)、朝鲜半岛北部、漠南漠北蒙古草原全境(内蒙古和外蒙古地区),西伯利亚南部地区、西域大部分地区(今新疆東部和南部)、吐蕃地区(包括今青海、西藏、四川西部等地)、以及云南地区等地。

至元元年八月十六日(1264年9月7日),忽必烈发布《至元改元诏》,取《易经》“至哉坤元”之义,改“中统五年”为“至元元年”。

庚申年四月初一日(1260年5月12日),忽必烈立中书省,以中书省为最高行政机关,行使宰相职权,以王文統为平章政事,张文谦为中书左丞。

中统四年五月六日(1263年6月13日),忽必烈立枢密院,以枢密院为中央最高军事管理机关,以燕王真金守中书令,兼判枢密院事。

至元元年(1264年),忽必烈立总制院,以总制院统领全国宗教事务并管辖吐蕃地区,以国师八思巴领之。至元二十五年(1288年),尚书省右丞相桑哥认为总制院职责重大,故向忽必烈奏请根据唐朝时期在宣政殿接待吐蕃使者的缘故,改名为宣政院。忽必烈同意,并任命桑哥和脱因为宣政院使。

至元五年七月四日(1268年8月13日),忽必烈立御史台,以御史台为最高监察机关,以右丞相塔察兒为御史大夫,以張雄飛为侍御史。

至元八年十一月十五日(1271年12月18日),因劉秉忠之勸,忽必烈发布《建国号诏》,取《易经》“大哉乾元”之义,建立“大元”国号,其自身亦从大蒙古国皇帝(大汗)变为大元皇帝,元朝正式建立。

元军延续自1268年秋天以来的攻势继续围困襄阳,将襄阳和樊城分隔开来,至元十年正月九日(1273年1月29日),在回回炮的助攻下,元军将领阿里海牙攻克樊城,襄阳彻底成为孤城,元世祖降诏谕襄阳守将吕文焕,阿里海牙亲自到城下劝降吕文焕,保证吕文焕和城中军民的安全,吕文焕犹疑未决。于是阿里海牙和吕文焕折箭为誓担保,吕文焕感泣,至元十年二月二十四日(1273年3月14日),吕文焕和儿子出城投降,归顺元朝。元军经过接近五年時间包围,最终取得襄阳。但是以後的进展则相当顺利。

至元十一年六月十五日(1274年7月20日),忽必烈向行中书省及蒙古、汉军万户千户军士发布问罪于宋的诏书《兴师征南诏》。

至元十一年(1274年)农历七月,忽必烈发布《下江南檄》,派伯颜统率大军讨伐南宋,并告诫伯颜要学习曹彬不杀平江南。伯颜后来取临安,的确做到了忽必烈的要求。

至元十三年正月十八日(1276年2月4日),伯颜率领大军攻陷南宋首都临安(今杭州),宋恭帝派遣使者给元军统帅伯颜奉上传国玉玺和降表,在降表中宋恭帝为忽必烈上尊号大元仁明神武皇帝,元军俘虏5岁的宋恭帝和谢太皇太后,以及南宋宗室和大臣,灭南宋。

至元十三年二月十一日(1276年2月27日),忽必烈发布《归附安民诏》,诏谕江南一带新附府州司县官吏士民军卒人等,稳定江南社会秩序,安定江南士人和百姓之心。

逃离临安的部分大臣陆秀夫等人,先后扶持宋端宗,宋帝昺,建立海上流亡政权,在东南沿海一带继续和元军对抗。至元十六年二月六日(1279年3月19日),在厓山海战中,元军将领张弘范击败南宋海军,南宋丞相陆秀夫挟8岁的小皇帝“宋帝昺”跳海而死,不少後宮和大臣亦相繼跳海自殺。《宋史》記載七日後,十餘萬具屍體浮海。南宋残余的最后一支抵抗力量选择了惨烈的终结,至此,元朝统一海内,结束了中国自安史之乱以来520多年的分裂局面。

1281年3月20日,忽必烈愛妻察必皇后去世。1286年1月5日,皇太子真金去世,连续几年的时间里,爱妻和爱子的先后去世,使忽必烈悲痛不已。此外,忽必烈晚年飽受肥胖與痛風病痛之苦。過度飲酒也损害了他的健康。

至元三十一年正月二十二日(1294年2月18日),忽必烈於大都皇宮紫檀殿去世,享壽七十九岁,在位三十五年。忽必烈葬于起辇谷。

忽必烈去世后,在顾命大臣伯颜等人的拥戴下,其孙铁穆耳于1294年5月10日在上都继承皇位,是为元成宗。1303年,元成宗与西北诸王达成和议,西北的四大汗国重新承认元朝的宗主国地位。

因为1260年忽必烈和阿里不哥争位导致蒙古帝国表面上维持统一,实际上已经分裂,帝国西部為四大汗国实际控制,而帝国东部為忽必烈实际控制。趁着忽必烈和阿里不哥的内战,西北地区的钦察汗国、察合台汗国、窝阔台汗国纷纷自立,此时尚在西亚进行西征的旭烈兀也准备自帝一方,不论忽必烈还是阿里不哥都只得到一部分宗王支持,没有召开成吉思汗四子嫡系后裔参加的「忽里勒臺」(決定繼承人的大會),忽必烈不被广泛承认,于是,忽必烈将大汗在西亚的直辖地(阿姆河以西直到埃及边境)封给旭烈兀换取旭烈兀的支持,旭烈兀建立伊儿汗国(其实旭烈兀留在西亚,忽必烈也没办法,但忽必烈给了旭烈兀统治的合法性)。忽必烈将大汗在中亚的直辖地(阿尔泰山以西直到阿姆河的农耕和城郭地区)封给察合台汗阿鲁忽换取阿鲁忽的支持。而钦察汗国早在元定宗贵由和元宪宗蒙哥统治时期已经取得实际上基本独立的地位。

1264年8月21日,阿里不哥向忽必烈投降。胜利之后忽必烈立即向各系兀鲁思派去急使,召他们东赴蒙古草原,重新召开忽里台大会。忽必烈重开忽里台的目的,是因为考虑到中统元年三月二十四日仓促即位于开平,没有四大兀鲁思的代表参加,不符合成吉思汗的扎撒(蒙古语“军律”、“法规”之意),故而准备依照传统惯例,在祖先发祥地斡难---怯绿涟之域召开由各系宗王参加的忽里台,重新确立自己的大汗地位,并借这次大会扼制帝国分裂的趋势。

钦察汗别儿哥、察合台汗阿鲁忽和伊兒汗旭烈兀(忽必烈之弟)一致同意东来赴会。元世祖也向窝阔台汗海都派去了急使,但海都拒绝前来。当然,这次原定于至元四年(1267年)召开的忽里台没能如约举行,主要是因为各汗国之间随后爆发战争,以及在此后一年多时间里原本同意参加忽里台的阿鲁忽、旭烈兀、别儿哥三位汗王先后去世(旭烈兀1265年去世,别儿哥、阿鲁忽1266年去世,他们不可能参加1267年的忽里台)。但窝阔台汗海都的抗命已经明白无误地表明了分裂意图,忽必烈声称的大汗之位未获公认,成吉思汗及窝阔台汗创立的蒙古帝国处于分崩离析的边缘。

1269年,钦察汗国、窝阔台汗国与察合台汗国召开塔剌思忽里台,达成了协议,共同反对拖雷家族控制的大汗直辖地(即忽必烈的实际控制区)和伊儿汗国(旭烈兀家族控制区,忽必烈的唯一支持者),并协议划分了各自在阿姆河以北地区的势力范围。塔剌思大会标志着大蒙古国的实质分裂和解体,从此察合台汗国和窝阔台汗国脱离了大蒙古国,与掌控蒙古帝国东部的拖雷系家族分头發展。察合台汗国和窝阔台汗国对此后数十年中亚和西亚历史的发展产生了深远的影响。

窝阔台汗海都一直和忽必烈敵對,企图确立自己为大汗之位的继承人。终元世祖忽必烈一朝,元朝和窝阔台汗国、察合台汗国征战不休,直到元成宗时期才彻底解决西北问题。

大蒙古国时期的历任大汗,虽然经由对辽、金故地的征服,与汉文明一直有接触,也往往对汉文化表示接纳,蒙古贵族却大多数反对建立一个汉式的政府;忽必烈对其在汉地的领地则相当重视,并且花费了时间去了解汉人的治国思想和儒家文化,最终以自己的领地开平为中心,建立起了一个汉式的行政中心,其后忽必烈在试图争取整个蒙古帝国统治权的同时,一直没有放弃尝试让汉人接受他作为一个中国皇帝,并为此做了一系列汉化努力。

忽必烈赢取汉人接受其统治的第一个措施便是效仿汉人的典章制度,将“大蒙古国”的历史和皇族“汉化”,其中一个显著做法就是建立太庙,按照中原王朝的传统为大蒙古国的历任大汗确立庙号,追尊谥号。

中统四年(1263年)农历三月,忽必烈下诏在燕京(后来改称大都)建立太庙。至元元年(1264年)十月,初定太庙七室神主。至元二年农历十月十四日(1265年11月23日),忽必烈祭祀太庙,为皇祖成吉思汗上庙号太祖。至元三年(1266年)九月,太庙始作八室神主。十月,太庙建成。丞相安童、伯颜建议制定尊谥庙号,忽必烈命平章政事趙璧等集议,制尊谥庙号,定为八室,为大蒙古国的前四位帝王成吉思汗、窝阔台(元太宗)、贵由(元定宗)、蒙哥(元宪宗)上庙号和谥号,为他们的皇后上谥号;并追尊也速该、术赤、察合台三人为皇帝,也为他们上庙号和谥号,并为拖雷(已经于1251年被追尊为皇帝)改谥号为景襄皇帝,并将他们四人的正妻追谥为皇后,也上谥号。太庙八室,这八位和他们的妻子的神主各居一室。这些做法有效地吸引了汉族谋士和儒生参与忽必烈的新政权,《剑桥中国史——辽宋夏金元》认为,这一系列做法极大地帮助忽必烈巩固了蒙古族政权在汉地的统治。

蒙古帝国的首都,大汗的汗庭处于蒙古高原上的和林哈拉。忽必烈掌控蒙古帝国东部以后,逐步建立了两都制,并最终定都大都,将政权的统治中心移到了汉地文化更加发达的地区,有利于取得汉族谋士和蒙古贵族之间的平衡。

1215年5月31日,成吉思汗率大军攻克金中都(今北京市)。1217年,太师、国王木华黎改中都为燕京。燕京即为后来两都制中的中都。

1256年,忽必烈命刘秉忠在开平(今中国内蒙古自治区锡林郭勒盟正蓝旗多伦县西北闪电河畔)建立王府,忽必烈在此建立了著名的“金莲川幕府”。中统四年五月九日(1263年6月16日),忽必烈下诏升开平府为上都。

中统五年八月十四日(1264年9月5日),忽必烈发布《建国都诏》,改燕京(今北京市)为中都,定为陪都,两都制正式形成。

至元四年正月三十日(1267年2月25日),忽必烈由上都迁都到中都,定中都为首都,忽必烈迁都中都后,居住于中都城外的金代离宫——大宁宫内,并随即在中都的东北部,以大宁宫所在的琼华岛为中心开始了新宫殿和都城的规划兴建工作,上都成为陪都。

至元九年二月三日(1272年3月4日),忽必烈将中都改名为大都(突厥语称汗八里,帝都之意),元大都包括南城(金中都旧城)和北城(元大都新城),两者的城墙“仅隔一水”。

至元十一年正月初一(1274年2月9日),宫阙告成,元世祖忽必烈首次在大都皇宫正殿大明殿举行朝会,接受皇太子、诸王、百官以及高丽国王王禃所派使节的朝贺。

至元二十八年五月二十一日(1291年6月18日),忽必烈下诏颁布元朝第一部全国性的法律典籍《至元新格》。

忽必烈统一中国后,元朝疆域空前辽阔,远超汉唐盛世。“自封建变为郡县,有天下者,汉、隋、唐、宋为盛,然幅员之广,咸不逮元。汉梗于北狄,隋不能服东夷,唐患在西戎,宋患常在西北。若元,则起朔漠,并西域,平西夏,灭女真,臣高丽,定南诏,遂下江南,而天下为一,故其地北逾阴山,西极流沙,东尽辽左,南越海表。盖汉东西九千三百二里,南北一万三千三百六十八里,唐东西九千五百一十一里,南北一万六千九百一十八里,元东南所至不下汉、唐,而西北则过之,有难以里数限者矣。”

元朝不仅在疆域面积上远迈汉唐,而且在东北、西北、西南等边疆地区的控制程度上也远超汉唐盛世。“盖岭北、辽阳与甘肃、四川、云南、湖广之边,唐所谓羁縻之州,往往在是,今皆赋役之,比于内地;而高丽守东藩,执臣礼惟谨,亦古所未见。”

元世祖至元十七年(1280年)元朝的疆域范围:东北至外兴安岭、鄂霍次克海、日本海,包括库页岛,并到达朝鲜半岛中部的铁岭和慈悲岭一带,北到西伯利亚南部(谭其骧版地图认为北到北冰洋),到达贝加尔湖以北的鄂毕河和叶尼塞河上游地区,西北至今新疆大部分地区,西南包括今西藏、云南、以及缅甸北部,南到南海,东南到达东海中的澎湖列岛。

在灭南宋前后,元政府曾要求周边一些国家或地区(包括日本、安南、占城、缅甸、爪哇、琉求國)臣服,接受与元朝的朝贡关系,但遭到拒绝,故派遣军队进攻攻打这些国家或地区,例如緬甸蒲甘王朝拒絕朝貢,元軍入侵蒲甘並攻破蒲甘城,令缅甸臣服於元朝。其中以入侵日本国最为著名,也最惨烈。

忽必烈在位时期和中亚的察合台汗国,窝阔台汗国多次交战,双方互有胜负,1289年,窝阔台汗国夺取元朝控制下的新疆南部塔里木盆地大部分地区,元朝只控制塔里木盆地东部的且末、焉耆等地区。终忽必烈一朝,元朝始终控制新疆北部的别失八里(今乌鲁木齐东北)一带和新疆东部的吐鲁番、哈密等地。

对日战争 至元十一年(1274年)元军發動第一次侵日戰爭,日本史書稱之為“文永之役”,以三萬二千餘人,東征日本。至元十八年(1281年)七月,忽必烈又發動第二次侵日戰爭,史稱“弘安之役”,由范文虎、李庭率江南軍十餘萬人,到達次能、志賀二島,卻碰到颱風,溺死近半。通常认为台风(日本人称之为“神风”)是这两次征日造成失败的最大原因。亦有观点认为,忽必烈担心归附军的忠诚,故而借东征日本而一举消除隐患。

元朝重臣郝经在中统元年(1260年)农历四月奉元世祖忽必烈之命出使南宋南北议和,在九月到达南宋后被扣留软禁于真州15年,直到至元十二年(1275年)农历二月才被南宋送归元朝境内,他在被软禁期间十余次给南宋君臣上书,希望元宋缔结和约,均无任何回复。郝经在中统元年(1260年)农历十一月给南宋两淮制置使李庭芝的书信《再与宋国两淮制置使书》中对元世祖忽必烈的評價是:“今主上应期开运,资赋英明,喜衣冠,崇礼乐,乐贤下士,甚得中土之心,久为诸王推戴。稽诸气数,观其德度,汉高帝、唐太宗、魏孝文之流也。” (“汉高帝”指的是汉太祖刘邦,“太祖”为庙号,“高帝”为谥号,《史记》中常谓“高祖”,因此人多以为其庙号为高祖,其实乃庙号谥号混称。“唐太宗”指的是李世民。“魏孝文”指的是北魏孝文帝拓跋宏。)

元朝重臣郝经在中统二年(1261年)给南宋丞相贾似道的第三封书信《复与宋国丞相论本朝兵乱书》中对元世祖忽必烈的評價是:“夫主上之立,固其所也。太母有与贤之意,先帝无立子之诏。主上虽在潜邸,久符人望,而又以亲则尊,以德则厚,以功则大,以理则顺,爱养中国,宽仁爱人,乐贤下士,甚得夷夏之心,有汉、唐英主之风。加以地广众盛,将猛兵强,神断威灵,风蜚雷厉,其为天下主无疑也。”

明朝官修正史《元史》宋濂等的評價是:“世祖度量弘广,知人善任使,信用儒术,用能以夏变夷,立经陈纪,所以为一代之制者,规模宏远矣。”

明朝官修正史《元史》宋濂等的評價是:“世称元之治以至元、大德为首。……。故终世祖之世,家给人足。”

明朝官修皇帝实录《明太祖实录》记载,明太祖朱元璋在洪武七年八月初一日(1374年9月7日),亲自前往南京历代帝王庙祭祀三皇、五帝、夏禹王、商汤王、周武王、汉太祖、汉光武帝、隋文帝、唐太宗、宋太祖、元世祖一共十七位帝王,其中对元世祖忽必烈的祝文是:“惟神昔自朔土,来主中国,治安之盛,生餋之繁,功被人民者矣。夫何传及后世不遵前训,怠政致乱,天下云扰,莫能拯救。元璋本元之农民,遭时多艰,悯烝黎于涂炭,建义聚兵,图以保全生灵,初无黄屋左纛之意,岂期天佑人助,来归者众,事不能已,取天下于群雄之手,六师北征,遂定于一。乃不揆菲德,继承正统,此天命人心所致,非智力所能。且自古立君,在乎安民,所以唐虞择人禅授,汤武用兵征伐,因时制宜,其理昭然。神灵在天不昧,想自知之。今念历代帝王开基创业、有功德于民者,乃于京师肇新庙宇,列序圣像,每岁祀以春、秋仲月,永为常典,礼奠之初,谨奉牲醴致祭,伏惟神鉴。尚享!”

明朝官修皇帝实录《明太祖实录》记载,洪武二十二年(1389年)十二月,明太祖朱元璋给北元兀纳失里大王的信中,对元太祖和元世祖的评价如下:“昔中国大宋皇帝主天下三百一十余年,后其子孙不能敬天爱民,故天生元朝太祖皇帝,起于漠北,凡达达、回回、诸番君长尽平定之,太祖之孙以仁德著称,为世祖皇帝,混一天下,九夷八蛮、海外番国归于一统,百年之间,其恩德孰不思慕,号令孰不畏惧,是时四方无虞,民康物阜。”

邵远平《元史类编》的評價是:“册曰:遂辟雄图,混一中外;德威所指,无远弗届;建号立制,垂模一代;崇儒察奸,旋用旋败;英明克断,用无祗悔。”

叶子奇《草木子》卷三上: “元朝自世祖(忽必烈)混一之后,天下治平者六、七十年,轻刑薄赋,兵革罕用;生者有养,死者有葬;行旅万里,宿泊如家,诚所谓盛也亦!”

毕沅《续资治通鉴》的評價是:“帝度量恢廓,知人善任使,故能混一区宇,扩前古所未有。惟以亟于财用,中间为阿哈玛特、卢世荣、僧格所蔽,卒能知其罪而正之。立纲陈纪,殷然欲被以文德,规模亦已弘远矣。”(“阿哈玛特”指的是阿合马,“僧格”指的是桑哥,不同的人对他们的名字进行汉语音译时,有一定差别。)

魏源《元史新编》的評價是:“论曰:元之初入中国,震荡飘突,惟以杀伐攻虏为事,不知法度纪纲为何物,其去突厥、回纥者无几。及世祖兴,始延揽姚枢、窦默、刘秉忠、许衡之徒,以汉法治中夏,变夷为华,立纲陈纪,遂乃并吞东南,中外一统。加以享国长久,垂统创业,轶遼、金而媲漢、唐,赫矣哉!且其天性宽宏,包帡无外。阿里不哥及海都、笃哇诸王,皆亲犯乘舆。对垒血战,力屈势穷,一朝归命,则皆以太祖子孙,大朝会于上都,恩礼宴赉如初。当南北锋焰血战之余,或离间以侍郎张天悦通宋而不信。敕南儒被掠卖为奴者,官赎为民。所获宋商、宋谍私入境者,皆纵遣之而不诛。置榷场于樊城,通宋互市,弛沿边军器之禁。其长驾远驭如是。宋幼主母子至通州,命大宴十日,小宴十日,然后赴上都。除弘吉剌皇后厚待之事别详《皇后传》外,其母子在江南庄田,听为世业。其后文宗时市故全太后田为大承天寺永业,市故瀛国公田为大翔龙寺永业,直至顺帝末,始夺和尚赵完普之田归官,直与元相终始。宋之宗室如福王与芮等,随宋主来归,授平原郡公,其家赀在江南者,取至京赐之。此外宗室多类此。即奸民冒称赵氏作乱者,从不以累及宋后,其优礼亡国也如是。思创业艰难,移漠北和林青草丛植殿隅,俾后世无忘草地。又留所御裘带于大安阁以示子孙。武宗至大中尝诣阁中发故箧阅之,则皆大练之服。西域贾胡屡献牙忽大珠,价值数万而不受。宫闱肃穆,无豔宠奇闻。至元八年,平滦路昌黎县民生男,夜中有光,或奏请除之,帝曰:‘何幸天生一好人,奈何反生妒忌!’命有司加恩养。伯颜伐宋,谆谆命以曹彬取江南不戮一人为法。其俭慈也又如是,非命世天纵而何?惟功利之习不能自胜于中,故日本、爪哇之师远覆于海岛,王、阿、桑、卢掊克之臣相仍于覆辙,盖质有余而学不足欤!”(“王、阿、桑、卢”指的分别是王文统、阿合马、桑哥、卢世荣。四人均为元世祖朝不同时期的理财大臣。)

曾廉《元书》的評價是:“论曰:世祖崇儒重道,而特进言利之臣,三进三乱而讫不悟,岂非其明有所蔽耶?然其不欲剥民亦审矣。殆以为自我作则,将上下均足,堪为后世经制也。呜呼!以世祖之仁,乘开国之运,而言利之弊,若此,然则利其有可言者耶?至其任中书枢密而重台纲,法纪立矣。国治民安是在知人哉!”

中華民国史学家屠寄《蒙兀儿史记》的評價是:“汗目有威稜,而度量弘广,知人善任,群下畏而怀之,虽生长漠北,中年分藩用兵,多在汉地,知非汉法不足治汉民。故即位后,引用儒臣,参决大政,诸所设施,一变祖父诸兄武断之风,渐开文明之治。惟志勤远略,平宋之后,不知息民,东兴日本之役,南起占城、交趾、缅甸、爪哇之师,北御海都、昔里吉、乃颜之乱。而又盛作宫室,造寺观,干戈土木,岁月不休。国用既匮,乃亟于理财,中间颇为阿合马、卢世荣、桑哥之徒所蔽,虽知其罪而正之,闾阎受患已深矣。”

中華民国官修正史《新元史》柯劭忞的評價是:“唐太宗承隋季之乱,魏徵劝以行王道、敦教化。封德彝驳之曰:‘书生不知时务,听其虚论,必误国家。’太宗黜德彝而用徵,卒致贞观之治。蒙古之兴,无异于匈奴、突厥。至世祖独崇儒向学,召姚枢、许衡、窦默等敷陈仁义道德之说,岂非所谓书生之虚论者哉?然践阼之后,混壹南北,纪纲法度灿然明备,致治之隆,庶几贞观。由此言之,时儿今古,治无夷夏,未有舍先王之道,而能保世长民者也。至于日本之役,弃师十万犹图再举;阿合马已败,复用桑哥;以世祖之仁明,而吝于改过。如此,不能不为之叹息焉。”

\subsection{中统}

\begin{longtable}{|>{\centering\scriptsize}m{2em}|>{\centering\scriptsize}m{1.3em}|>{\centering}m{8.8em}|}
  % \caption{秦王政}\
  \toprule
  \SimHei \normalsize 年数 & \SimHei \scriptsize 公元 & \SimHei 大事件 \tabularnewline
  % \midrule
  \endfirsthead
  \toprule
  \SimHei \normalsize 年数 & \SimHei \scriptsize 公元 & \SimHei 大事件 \tabularnewline
  \midrule
  \endhead
  \midrule
  元年 & 1260 & \tabularnewline\hline
  二年 & 1261 & \tabularnewline\hline
  三年 & 1262 & \tabularnewline\hline
  四年 & 1263 & \tabularnewline\hline
  五年 & 1264 & \tabularnewline
  \bottomrule
\end{longtable}

\subsection{至元}

\begin{longtable}{|>{\centering\scriptsize}m{2em}|>{\centering\scriptsize}m{1.3em}|>{\centering}m{8.8em}|}
  % \caption{秦王政}\
  \toprule
  \SimHei \normalsize 年数 & \SimHei \scriptsize 公元 & \SimHei 大事件 \tabularnewline
  % \midrule
  \endfirsthead
  \toprule
  \SimHei \normalsize 年数 & \SimHei \scriptsize 公元 & \SimHei 大事件 \tabularnewline
  \midrule
  \endhead
  \midrule
  元年 & 1264 & \tabularnewline\hline
  二年 & 1265 & \tabularnewline\hline
  三年 & 1266 & \tabularnewline\hline
  四年 & 1267 & \tabularnewline\hline
  五年 & 1268 & \tabularnewline\hline
  六年 & 1269 & \tabularnewline\hline
  七年 & 1270 & \tabularnewline\hline
  八年 & 1271 & \tabularnewline\hline
  九年 & 1272 & \tabularnewline\hline
  十年 & 1273 & \tabularnewline\hline
  十一年 & 1274 & \tabularnewline\hline
  十二年 & 1275 & \tabularnewline\hline
  十三年 & 1276 & \tabularnewline\hline
  十四年 & 1277 & \tabularnewline\hline
  十五年 & 1278 & \tabularnewline\hline
  十六年 & 1279 & \tabularnewline\hline
  十七年 & 1280 & \tabularnewline\hline
  十八年 & 1281 & \tabularnewline\hline
  十九年 & 1282 & \tabularnewline\hline
  二十年 & 1283 & \tabularnewline\hline
  二一年 & 1284 & \tabularnewline\hline
  二二年 & 1285 & \tabularnewline\hline
  二三年 & 1286 & \tabularnewline\hline
  二四年 & 1287 & \tabularnewline\hline
  二五年 & 1288 & \tabularnewline\hline
  二六年 & 1289 & \tabularnewline\hline
  二七年 & 1290 & \tabularnewline\hline
  二八年 & 1291 & \tabularnewline\hline
  二九年 & 1292 & \tabularnewline\hline
  三十年 & 1293 & \tabularnewline\hline
  三一年 & 1294 & \tabularnewline
  \bottomrule
\end{longtable}


%%% Local Variables:
%%% mode: latex
%%% TeX-engine: xetex
%%% TeX-master: "../Main"
%%% End:

%% -*- coding: utf-8 -*-
%% Time-stamp: <Chen Wang: 2019-12-26 14:53:44>

\section{成宗\tiny(1294-1307)}

\subsection{生平}

元成宗铁穆耳,是元朝第二位皇帝,蒙古帝国第六位大汗,1294年5月10日—1307年2月10日在位,在位14年。元世祖孙、太子真金第三子。清乾隆帝命改譯遼、金、元三史中的音譯專名,改譯特穆爾,今日學界已無人使用。

他去世后,谥号钦明广孝皇帝,庙号成宗,蒙古語号完澤篤可汗。

至元二十二年农历十二月十日(1286年1月5日),皇太子真金去世,元世祖欲立真金次子答剌麻八剌為皇太子,但1292年答剌麻八剌因病去世。至元三十年(1293年)真金三子铁穆耳受皇太子宝,总兵镇守漠北和林。至元三十一年农历正月二十二日(1294年2月18日),元世祖忽必烈去世,被封為晉王的真金長子甘麻剌決定要繼續鎮撫北方,铁穆耳得以在其母阔阔真與大臣伯顏等人的支持下,於至元三十一年农历四月十四日(1294年5月10日)在上都大安阁即位,是为元成宗。

铁穆耳即位後停止对外战争,罷征日本、安南,专力整顿国内军政,減免江南部分賦稅。並推行限制诸王势力、新编律令等措施,使社会矛盾暂时缓和。

在位期间基本维持守成局面,但滥增赏赐,入不敷出,国库资财匮乏,「向之所儲,散之殆盡」,中统钞迅速贬值。曾发兵征讨八百媳妇(在今泰国北部),引起云南、贵州地区动乱。晚年患病,委任皇后卜鲁罕和色目人大臣,朝政日渐衰败。

大德九年六月初五(1305年6月27日),元成宗冊立皇子德寿為皇太子,元成宗有数子,只有德寿皇太子为伯牙吾·卜鲁罕皇后所生。 同年十二月十八日(1306年1月3日),德寿因病去世。德寿去世後,成宗在生前未再立皇太子。

大德十一年农历正月初八日(1307年2月10日),成宗在大都玉德殿病逝,享年42岁 , 在位14年。

晚年患病,委任皇后卜鲁罕和色目人大臣,朝政日渐衰败。铁穆耳后继无人,埋下了元朝中期皇位争夺战的隐患。庙号成宗,谥号钦明广孝皇帝。蒙古汗号完泽笃可汗。

大德十一年九月十一日(1307年10月7日),元武宗为铁穆耳上谥号钦明广孝皇帝,庙号成宗,蒙古语称号完澤篤皇帝。

大德五年(1301年)秋,元军与窝阔台汗国的海都和察合台汗国的笃哇会战于金山附近的铁坚古山。元军先败海都。笃哇后至,两军再战。双方互有胜负,但都受到重创。海都、笃哇在会战中负伤,海都于1302年去世。

钦察汗国的东部藩属术赤长子斡儿答家族白帐汗封地原先与大汗的直辖地相连。窝阔台汗国的海都兴起后,隔断了元朝与术赤家族领地的直接联系。与海都接壤的白帐汗系宗王古亦鲁克为争夺汗位,投靠海都、笃哇。古亦鲁克的对手伯颜汗曾遣使元朝,要求双方联合作战。元朝的军队攻击海都,从谦州深入钦察汗国控制下的亦必儿·失必儿之地(今俄罗斯鄂毕河中游地区)。

大德六年(1302年),钦察汗国脱脱汗和白帐汗伯颜汗出兵2万,与元成宗的军队联合进攻笃哇和察八儿。此后钦察汗国承认元朝的宗主地位,长期与元朝维持友好关系。

1301年的铁坚古山之战对于元与西北宗藩的关系有决定性的影响,1302年海都去世,到了大德七年(1303年),笃哇扶立察八儿为窝阔台兀鲁思汗。笃哇暗中向元朝驻守在哈剌和林边境的安西王阿难答派出使臣,向元成宗表示臣服,请求朝廷罢兵。成宗同意约和。获得元廷支持后,笃哇与察八儿等聚会,到会诸王一致认识到,与朝廷进行长达数十年的战争是“自伤祖宗之业”。

大德七年(1303年)秋,笃哇以及海都之子察八儿约和使臣到达元廷。元廷与西北诸王达成和议,西北诸王承认元朝的宗主地位,设驿路,开关塞。自从1260年忽必烈与阿里不哥争位以来,元朝西北边境的战火终于基本平息,元朝的宗主地位得到四大汗国的正式承认。

接着,他们又联合遣使到伊儿汗国、钦察汗国王庭,大德八年(1304年)秋,伊儿汗完者都在木干草原会见钦察汗脱脱的使臣,西北四大汗国彼此之间的约和也至此完成,整个蒙古帝国境内再次迎来了和平。

明朝官修正史《元史》宋濂等的評價是:“成宗承天下混壹之后,垂拱而治,可谓善于守成者矣。惟其末年,连岁寝疾,凡国家政事,内则决于宫壸,外则委于宰臣;然其不致于废坠者,则以去世祖为未远,成宪具在故也。”

明朝官修正史《元史》宋濂等的評價是:“世称元之治以至元、大德为首。……。故终世祖之世,家给人足。……。大德之治,几于至元。”

清朝史家邵远平《元史类编》的評價是:“册曰:豢业以治,垂拱用成;中年奋武,启衅南征;末婴寝疾,壼柄廼萌;赖斯贤辅,镇侧弭倾。”

清朝史家毕沅《续资治通鉴》的評價是:“帝承世祖混一之后,善于守成;惟末年连岁寝疾,凡国家政事,内则决于宫壼,外则委于宰臣,幸去世祖未远,守其成宪,不至废坠。”

清朝史家曾廉《元书》的評價是:“论曰:成宗号为能守法度,而为病虐,前星弗耀,牝鸡司晨,而内难作矣。然非成宗之过也,成宗早任合剌合孙,资为羽翼,自古未有贤人在位而乱其国者也。股肱之寄,要在忠良,唐宗之言,信夫!”

民国史家屠寄《蒙兀儿史记》的評價是:“始汗为太孙时,好饮无节。忽必烈汗常戒之,不悛。以此受杖者三次,忽必烈汗至命医官监其饮食。有近侍司太孙节沐者,私置酒于盥器,代水以进,忽必烈汗闻之,大怒,谪戍其人远方,杀之于道。汗既登极,深以前事为非,力自节饮。其勇于改过如此。汗仁惠聪睿,承天下混一之后,信用老成,垂拱而治。一革至元中叶以来聚敛之政,冗设之官。约束诸王、妃、主、驸马扰民,禁滥请赏赐。性又谦冲,不好虚誉。群臣、皇后一再请上徽号,卒不允。可谓守成之令主矣。虽晚婴末疾,政出中宫,而举错无大过失。固由委任贤相之效,亦未始非内助之得人也。”

民国私修正史《新元史》柯劭忞的評價是:“成宗席前人之业,因其成法而损益之,析薪克荷,帝无使焉。晚年寝疾,不早决计计传位武宗,使易世之后,亲贵相夷,祸延母后。悲夫!以天子之尊,而不能保其妃匹,岂非后世之殷鉴哉。”

\subsection{元贞}

\begin{longtable}{|>{\centering\scriptsize}m{2em}|>{\centering\scriptsize}m{1.3em}|>{\centering}m{8.8em}|}
  % \caption{秦王政}\
  \toprule
  \SimHei \normalsize 年数 & \SimHei \scriptsize 公元 & \SimHei 大事件 \tabularnewline
  % \midrule
  \endfirsthead
  \toprule
  \SimHei \normalsize 年数 & \SimHei \scriptsize 公元 & \SimHei 大事件 \tabularnewline
  \midrule
  \endhead
  \midrule
  元年 & 1295 & \tabularnewline\hline
  二年 & 1296 & \tabularnewline\hline
  三年 & 1297 & \tabularnewline
  \bottomrule
\end{longtable}

\subsection{大德}

\begin{longtable}{|>{\centering\scriptsize}m{2em}|>{\centering\scriptsize}m{1.3em}|>{\centering}m{8.8em}|}
  % \caption{秦王政}\
  \toprule
  \SimHei \normalsize 年数 & \SimHei \scriptsize 公元 & \SimHei 大事件 \tabularnewline
  % \midrule
  \endfirsthead
  \toprule
  \SimHei \normalsize 年数 & \SimHei \scriptsize 公元 & \SimHei 大事件 \tabularnewline
  \midrule
  \endhead
  \midrule
  元年 & 1297 & \tabularnewline\hline
  二年 & 1298 & \tabularnewline\hline
  三年 & 1299 & \tabularnewline\hline
  四年 & 1300 & \tabularnewline\hline
  五年 & 1301 & \tabularnewline\hline
  六年 & 1302 & \tabularnewline\hline
  七年 & 1303 & \tabularnewline\hline
  八年 & 1304 & \tabularnewline\hline
  九年 & 1305 & \tabularnewline\hline
  十年 & 1306 & \tabularnewline\hline
  十一年 & 1307 & \tabularnewline
  \bottomrule
\end{longtable}


%%% Local Variables:
%%% mode: latex
%%% TeX-engine: xetex
%%% TeX-master: "../Main"
%%% End:

%% -*- coding: utf-8 -*-
%% Time-stamp: <Chen Wang: 2019-10-18 15:41:12>

\section{武宗\tiny(1307-1311)}

元武宗海山,是元朝第三位皇帝,蒙古帝国第七位大汗,在位4年,自1307年6月21日至1311年1月27日。乃元世祖之曾孫、太子真金之孫、答剌麻八剌之子、元成宗之侄。

1309年2月17日,群臣为海山上汉文尊号统天继圣钦文英武大章孝皇帝。

他去世后,謚號仁惠宣孝皇帝,廟號武宗,蒙古语称曲律皇帝。

武宗為真金次子答剌麻八剌之次子,嫡長子,1299年,海山接受元成宗的命令统兵漠北,负责同西北窝阔台汗国的君主海都和察合台汗国君主笃哇作战,多立戰功,为元朝结束和西北宗王的战争,以及1303年四大汗国全部承认元朝宗主地位做出了重要贡献。因为战功被封为懷寧王。

大德十一年正月初八(1307年2月10日),元成宗鐵穆耳病逝,儲位虛懸。成宗的伯牙吾·卜鲁罕皇后下命垂簾聽政,命安西王阿難答輔政。海山回大都奔喪,其弟愛育黎拔力八達與右丞相哈剌哈孫合謀发动政变,囚禁伯牙吾·卜鲁罕皇后和安西王阿難答,宣布擁立在外拥有重兵的海山為帝,是為元武宗,海山即位后追封其父答剌麻八剌為元順宗。

大德十一年五月二十一日(1307年6月21日),武宗在上都大安阁即位,之後处死伯牙吾·卜鲁罕皇后和阿難答,并更換了成宗时期的大臣,封其弟愛育黎拔力八達為皇太弟。在位只得四年,大興土木,建筑中都城,派军士千餘人及大量民工修建五台山華佛寺,又令喇嘛翻譯佛經,并曾想规定凡毆打西僧者截其手,罵西僧者斷其舌(但在其弟即后来的元仁宗愛育黎拔力八達劝告下取消)。

大德十一年七月十九日(1307年8月17日),元武宗下诏加封“至圣文宣王”孔子为“大成至圣文宣王”。

至大元年(1308年)五月,白蓮教被禁止。

至大元年(1308年),元武宗派遣月鲁出使钦察汗国,册封钦察汗脱脱为宁肃王。

至大二年(1309年),元朝和察合台汗国联手灭亡窝阔台汗国,元朝取得窝阔台汗国北部,察合台汗国取得窝阔台汗国南部。

至大二年(1309年)九月,为摆脱财政危机,印發至大銀鈔,导致至元钞大为贬值,從二釐到二兩分為十三等,並在各路、府、州、縣設常平倉平抑物價。將中書省宣敕、用人的權力劃歸尚書省。

至大四年正月初八日(1311年1月27日),因沉耽淫乐、酗酒过度,武宗病逝於大都玉德殿,享年三十岁,葬於起輦谷。

至大四年三月十八日(1311年4月7日),其弟愛育黎拔力八達(元仁宗)以皇太弟身份即位,廢除一切新政。

至大四年六月二十四日(1311年7月10日),元仁宗为海山上謚號仁惠宣孝皇帝,廟號武宗,蒙古语称号曲律皇帝。

至大二年正月初七日(1309年2月17日),皇太子、诸王、百官为元武宗上尊号统天继圣钦文英武大章孝皇帝。

由于日本拒绝向元朝称臣,元朝下令增加日货税收,日本不满,后来虽然减少关税,但仍然对日商检查甚严。

至大元年(1308年)日本商船焚掠庆元,官军不能敌。

至大四年(1311年)十月,以江浙省尝言:“两浙沿海濒江隘口,地接诸番,海寇出没,兼收附江南之后,三十余年,承平日久,将骄卒情,帅领不得其人,军马安量不当,乞斟酌冲要去处,迁调镇遏。“枢密院官议:“庆元与日本相接,且为倭商焚毁,宜如所请,其余迁调军马,事关机务,别议行之。”由此可见,此时元朝在东南沿海一带的军队战斗力很差(草原的军队因为世祖朝和成宗朝经常在西北作战,战斗力还可以)。

明朝官修正史《元史》宋濂等的評價是:“武宗当富有之大业,慨然欲创治改法而有为,故其封爵太盛,而遥授之官众,锡赉太隆,而泛赏之恩溥,至元、大德之政,于是稍有变更云。”

清朝史学家邵远平《元史类编》的評價是:“册曰:北藩入嗣,三宫协和;慨然创治,爵滥赏阿;貮省乱政,令教繁讹;有为何裨,变政已多。”

清朝史学家毕沅《续资治通鉴》的評價是:“帝承世祖、成宗承平之业,慨然欲创制改法;而封爵太盛,多遥授之官,锡赉太优,泛赏无节。至元、大德之政,于是乎变。”

清朝史学家魏源《元史新编》的評價是:“武宗始以怀宁王总兵漠北和林,与叛王海都劲敌对垒,屡摧其锋,中间几濒险危,披坚陷阵,威震遐荒,可谓天潢之杰出,天授之雄武矣。入绍大统,谓有宏图,而始终误听宵人,以立尚书省为营利之府,何哉?夫世祖立制,以天下大政归于中书省,任相任贤,责无旁贷。故小人欲变法,忌中书不便于己,则必别立尚书省以夺其权。阿合马、桑哥之徒相继乱政,毒流海内,是以世祖深戒前辙,不复再蹈。乃当席丰履厚之余,慨然欲变更至元、大德之旧。封爵太盛,而遥授之官多;锡赉太侈,而滥赏之卮漏。母后市恩左右,挠其恭俭,于是言利之臣迎合攘袂,以争利权。虽柄操自上,不至如阿合马、桑哥之甚,而仁心仁闻渐蔽于功利,几同于宋之熙、丰。故仁宗绍统,翻然诛殛,尽复旧章。盖变法不得其人,则不如勿薬之尚得中医也。又攷陶九仪《元氏掖庭记》,则琼岛水嬉之华,月殿霓裳之豔,亦自帝大滥其觞,而《本纪》讳之,不载一字,亦英雄酒色之通病欤!惟授受之际,坚守金匮传弟之盟,虽有内侍李邦宁,怂恿离间,帝言:‘朕志已定,汝自往东宫言之。’斯则磊落光明,胜宋太宗万万。综计始末,固不失为一代之英主焉。”

清朝史学家曾廉《元书》的評價是:“论曰:武宗擐甲临边,至登大位,宜有雄武之风,而颓然晏安,惟鞠蘖芗泽之为乐,元业自是衰矣。遂至鼎鼐充庭,名器之贱如履。而欲后人惜其敝袴,得乎?易日负且乘致寇至,武宗启之矣。”

民国史学家屠寄《蒙兀儿史记》的評價是:“海山汗滥赏淫威,非恭俭之主也。明知尚书省貮政病民,排众议而立之。更钞铸钱,将以理财,而财政愈紊,前史称其慨然欲有所为,然郊天、祀孔、亲享太庙,诸虚文外,无足纪者。惟终身远铁木迭儿,虽以母后之命,不使得预朝政。由后校之,殆有所先见矣。若乃三宫协和,始终不受谗慝。其自处骨肉之间,盖亦有道焉尔。”

民国官修正史《新元史》柯劭忞的評價是:“武宗舍其子而立仁宗。与宋宣公舍与夷而立穆公无以异。公羊子曰:朱之乱,宣公为之。然则英宗之弑,文宗之篡夺,亦帝为之欤!《春秋》贵让而不贵争,公羊子之言过矣。帝享国日浅,滥恩幸赏无一善之可书。独传位仁宗,不愧孝友。其流祚于子孙宜哉。”

\subsection{至大}

\begin{longtable}{|>{\centering\scriptsize}m{2em}|>{\centering\scriptsize}m{1.3em}|>{\centering}m{8.8em}|}
  % \caption{秦王政}\
  \toprule
  \SimHei \normalsize 年数 & \SimHei \scriptsize 公元 & \SimHei 大事件 \tabularnewline
  % \midrule
  \endfirsthead
  \toprule
  \SimHei \normalsize 年数 & \SimHei \scriptsize 公元 & \SimHei 大事件 \tabularnewline
  \midrule
  \endhead
  \midrule
  元年 & 1308 & \tabularnewline\hline
  二年 & 1309 & \tabularnewline\hline
  三年 & 1310 & \tabularnewline\hline
  四年 & 1311 & \tabularnewline
  \bottomrule
\end{longtable}


%%% Local Variables:
%%% mode: latex
%%% TeX-engine: xetex
%%% TeX-master: "../Main"
%%% End:

%% -*- coding: utf-8 -*-
%% Time-stamp: <Chen Wang: 2019-12-26 14:53:56>

\section{仁宗\tiny(1311-1320)}

\subsection{生平}

元仁宗愛育黎拔力八達是元朝第四位皇帝,蒙古帝国第八位大汗,1311年4月7日—1320年3月1日在位,一共在位9年。清代乾隆晚期乾隆帝命改譯遼、金、元三史中的音譯專名,改譯阿裕爾巴里巴特喇,今日學界已無人使用。

早年助兄长海山即位,被海山立为皇太子(元朝的皇位继承人一律称皇太子),相约兄终弟及,叔侄相传。后嗣位,年號皇慶、延祐。

他去世后,諡號聖文欽孝皇帝,廟號仁宗,蒙古語稱號普顏篤皇帝,又譯巴顏圖可汗。

至大四年正月初八日(1311年1月27日),元武宗病逝。至大四年三月十八日(1311年4月7日),元仁宗在大都大明殿即位。

仁宗自幼熟讀儒籍,傾心釋典。他从十几岁起就师从著名儒士李孟,儒家的伦理和政治观念对他有很强的影响。 他在登基称帝之前,先后在身边任用的有王约、赵孟頫、张养浩等汉儒和很多艺术家以及翻译家和散曲作家。

仁宗不仅能够读、写汉文,还能鉴赏中国书法与绘画,此外他还非常熟悉儒家学说和中国历史。

仁宗下诏下令將《貞觀政要》、《帝範》、《資治通鑒》和儒家经典《尚书》、《大学衍义》等書翻译成蒙古文并刊行天下,令蒙古人、色目人誦習。 仁宗支持下刊行天下的汉文著作包括:儒家经典《孝经》、《烈女傳》、《春秋纂例》、《辨疑》、《微旨》以及元朝官修农书《农桑辑要》。

1234年,蒙古帝国灭金朝、控制中原地区后需大量人才治理国家,根据中书令耶律楚材的建议,1237年秋八月,元太宗窝阔台下诏开科取士。诸路考试,均于1238年(戊戌年)举行,史称“戊戌选试”。这次考试共录取东平杨奂等4030人,皆为一时名士,朝廷得到了需要的各方面的人才。但后来“当世或以为非便,事复中止”。

后来的定宗(贵由)、宪宗(蒙哥)、世祖、成宗、武宗等朝,朝廷多次讨论恢复科举,但因为多种原因,一直没能实现。

皇庆改元(1312年)仁宗将其儒师王约特拜集贤大学士,并将王约“兴科举”的建议“著为令甲”(《元史》列传第六十五王约)。 皇庆二年(1313年)农历十月,仁宗要求中书省议行科举。中书省官员建议只设德行明经一科取士,仁宗同意。

皇庆二年农历十一月十八日(1313年12月6日),元仁宗下诏恢复科举,以朱熹集注的《四书》为所有科举考试者的指定用书,并以朱熹和其他宋儒注释的《五经》为汉人科举考试者增试科目的指定用书。,

这一变化最终确定了程朱理学在今后600年里的国家正统学说地位,因为后来的明清两朝的科举取士基本沿袭元朝的科举制度及其实施办法,并在其基础上进一步加以发展、充实和完善。

元仁宗1313年下诏恢复科举距离元太宗窝阔台1238年的“戊戌选试”已经有75年,天下读书的士人至此再次获得以科举方式晉身做官的途徑,方便了不同社会阶层之间的流动,缓和了社会矛盾。

中书省对于乡试、会试(“会试”之名亦始见于金朝)、殿试的举行时间,每次考试的录取人数、考试内容、考官来源、各行省的乡试录取名额分配、考试过程中的考场纪律等都做了详细的规定。

乡试,每三年一次,都是在八月二十日举行,全国共在17个省级区域设17处乡试科场,按照不同的地方的人口和民族进行名额分配,从赴试者中选300名合格者次年二月到大都参加会试。值得注意的是,高麗王朝所在的征东行省也有乡试科场,并在300名乡试中选者中有3人的名额。

延祐元年(1314年)农历八月二十日,全国举行乡试,一共录取三百人。

延祐二年(1315年)农历二月初一日,三百名乡试合格者在大都举行会试第一场,初三日第二场,初五日第三场,取中选者一百人。

延祐二年(1315年)农历三月七日,一百名会试中选者在大都皇宫举行殿试(廷试),最终录取护都答儿、张起岩等五十六人为进士。

1238年的“戊戌选试”之后,科举考试中断了75年,元仁宗延祐年间恢复科举取士,史稱“延祐復科”。

从元仁宗1315年开科取士到1368年元惠宗逃离大都、元朝灭亡为止,科举每三年一次,元朝一共举行了16次科举考试,考中进士的共计1139人(中间因为因为元惠宗时期丞相伯颜擅权,执意废科举,1336年科举和1339年科举停办。)国子学积分及格生员参加廷试录取正副榜284人,总计为1423人。

延祐元年(1314年),元仁宗下诏在江浙、江西、河南等三行省地进行田产登记,清查田亩,以增加国家税收,但是当1314年农历十月经理正式实行时,由于官吏的上下其手导致的执行不力,很多富民通过贿赂官吏隐瞒田产,很多贫苦农民和有田富民则被官吏乱加亩数,广大农民深受其害,最终导致1315年江西赣州蔡五九起义,虽然两个月中就被平定,但是元仁宗迫于形势,不得不停止经理,并减免所查出的漏隐田亩租税。「延祐经理」以失败告终。

元仁宗即位后,“以格例条画有关于风纪者,类集成书,”编修成一部专门的监察法规《风宪宏纲》。 并命监察御史马祖常作《风宪宏纲序》。

元惠宗至元二年(1336年),在增订《风宪宏纲》的基础上,将有关御史台典章制度汇编为《宪台通纪》。

至大四年(1311年)三月元仁宗即位不久,允中书所奏,“择耆旧之贤、明练之士,时则若中书右丞伯杭、平章政事商议中书刘正等,由开创以来政制法程可著为令者,类集折衷,以示所司,”分为制诏、条格、断例三部分:此外将介于《条格》、《断例》之间的内容编成成别类。

延祐三年(1316年)五月,书成。书成之后,又命“枢密、御史、翰林、国史、集贤之臣相与正是,凡经八年而是事未克果。”

至治三年二月十九日(1323年3月26日),元英宗最终审定,命名《大元通制》,颁行天下。全书共88卷,2539条。

《大元通制》是继《至元新格》之后元朝的第二部法典,现在只有条格的一部分(22卷,653条)流传下来,称为《通制条格》。

在位期間,减裁冗员,整顿朝政,推行“以儒治國”政策。又出兵西北,击败察合台后王也先不花。

元朝历代皇帝中,仁宗是对元朝较有贡献和有一番作为的其中一位(其他几位較有作為的分别是元世祖、元成宗、元英宗和元文宗)。

仁宗后将武宗之長子和世㻋徙居云南,立自己兒子碩德八剌为皇太子,打破叔侄相传的誓約。這個做法導致後來元朝長達二十年的政治混亂及宮廷鬥爭。

根据史实,仁宗生平好酒,延祐七年正月二十一日(1320年3月1日),元仁宗在大都光天宫病逝,享年三十六岁,他的逝世可能和喝酒伤身有关系。

延祐七年八月初十日(1320年9月12日),元英宗为父亲愛育黎拔力八達上諡號聖文欽孝皇帝,廟號仁宗,蒙古語稱號普顏篤皇帝。

明朝官修正史《元史》宋濂等的評價是:“仁宗天性慈孝,聪明恭俭,通达儒术,妙悟释典,尝曰:‘明心见性,佛教为深;修身治国,儒道为切。’又曰:‘儒者可尚,以能维持三纲五常之道也。’平居服御质素,澹然无欲,不事游畋,不喜征伐,不崇货利。事皇太后,终身不违颜色;待宗戚勋旧,始终以礼。大臣亲老,时加恩赉;太官进膳,必分赐贵近。有司奏大辟,每惨恻移时。其孜孜为治,一遵世祖之成宪云。”

清朝史学家邵远平《元史类编》的評價是:“册曰:立极电扫,稗政悉除;设科辍猎,屏言利徒;澹然无欲,十年罔渝;是惟令主,信史用书。”

清朝史学家毕沅《续资治通鉴》的評價是:“帝天性恭俭,通达儒术,兼晓释典,每曰:‘明心见性,佛教为深;修身治国,儒道为大。’在位十年,不事游畋,不喜征伐,尊贤重士,待宗戚勋旧,始终有礼。有司奏大辟,每惨恻移时。其孜孜为治,一遵世祖成宪云。”

清朝史学家魏源《元史新编》的評價是:“武仁授受之际,无可议者,仁宗初政,首革尚书省敝政,在位九年,仁心仁闻,恭俭慈厚,有汉文帝之风。惟武宗初约,由帝传位己子和世㻋而后及于英宗。及武宗崩,仁宗立,乃出封和世㻋于云南,而立子硕德八剌为太子。虽迫于皇太后之命,而已不守初约矣。和世㻋不之云南而举兵赴漠北,又不予以总兵和林之任,于是英宗被弑而泰定以晋王入绍大统,武宗旧臣燕帖木儿不服,遂于泰定殂后迎立周王于漠北,迎立怀王于江陵。怀王先立,周王后至,岂肯让于兄,于是弑之于中途,而国乱者数世。使当初即立周王,何至于此。至铁木迭儿奸贪不法,已经言官列款弹劾,而犹碍于皇太后,不敢质问,遂贻英宗以奸党谋逆之祸,不得谓非仁宗贻谋不臧有以致之也。”

清朝史学家曾廉《元书》的評價是:“论曰:元代科举之议久矣,至延祐而后行之,何其难乎?夫元代文学之盛,亦不须科举也。然儒风以振矣。天下啧啧以盛事归之。仁宗不亦宜乎?”

清末民初史学家屠寄《蒙兀儿史记》的評價是:“汗事兴圣太后。终身不违颜色,手勘内难,迎奉海山汗,退处东宫,不矜不伐,及海山汗升遐,哀恸不已。居丧再逾月,而后践阼。其孝友盖天性也。通达儒术,妙悟释典,尝曰:‘明心见性,佛教为深;修身治国,儒道为切。’又曰:‘儒者可尚,以能维持三纲五常之道也。’居东宫日,即有志兴学,以铁穆耳汗朝建国子监未成,趋台臣奏毕其功。既即位,一再增广国子生额,行科举取士之法。又尝遣使四方,旁求经籍。得秘笈,辄识以小玉印,命近侍掌之。承旨忽都鲁都儿迷失、刘赓进宋儒真德秀《大学衍义》,汗觉而善之,谓侍臣曰:‘治天下此一书足矣。’命翰林学士阿邻铁木儿并《贞观政要》皆译以国语,与图象《孝经》、《列女传》同刊印,以赐蒙兀、色目诸臣。平居服御,质素澹然,无欲不事游畋,不喜征伐,不崇货利,不受虚誉。待宗戚勋旧始终以礼,太官进膳,必分赐贵近;有司奏大辟,每惨恻移时。尝谓札鲁忽赤买闾曰:‘札鲁忽赤,人命所系,其详阅狱辞,事无大小,必谋诸同寮,疑不能决,与省台臣集议以闻。’又顾谓侍臣曰:‘卿等以朕居帝位为安耶?朕惟太祖创业艰难,世祖混一不易,兢业守成,恒惧不能当天心,绳祖武,使万方百姓各得其所,朕念虑在兹,卿等固不知也。’其孜孜为治,一遵忽必烈汗成宪。 惟饮酒无度,或其短祚之由欤。”

民国官修正史《新元史》柯劭忞的評價是:“仁宗孝慈恭俭,不迩声色,不殖货利。侍宗戚勋旧,始终以礼,大臣亲老,时加恩赍。有司奏大辟,辄恻怛移时,晋宁侯甲兄弟五人,俱坐法死,帝悯之,宥一人以养其父母。崇尚儒学,兴科举之法,得士为多,可谓元之令主矣。然受制母后,嬖幸之臣见权用事,虽稔知其恶,犹曲贷之。常问右丞相阿散曰:‘卿日行何事。’对曰:‘臣等奉行诏旨而已。’帝曰:‘祖宗遣训,朝廷大法,卿辈犹不遵守,况朕之诏旨乎。’其切责宰相如此。有君而无臣,惜哉!”

\subsection{皇庆}

\begin{longtable}{|>{\centering\scriptsize}m{2em}|>{\centering\scriptsize}m{1.3em}|>{\centering}m{8.8em}|}
  % \caption{秦王政}\
  \toprule
  \SimHei \normalsize 年数 & \SimHei \scriptsize 公元 & \SimHei 大事件 \tabularnewline
  % \midrule
  \endfirsthead
  \toprule
  \SimHei \normalsize 年数 & \SimHei \scriptsize 公元 & \SimHei 大事件 \tabularnewline
  \midrule
  \endhead
  \midrule
  元年 & 1312 & \tabularnewline\hline
  二年 & 1313 & \tabularnewline
  \bottomrule
\end{longtable}

\subsection{延祐}

\begin{longtable}{|>{\centering\scriptsize}m{2em}|>{\centering\scriptsize}m{1.3em}|>{\centering}m{8.8em}|}
  % \caption{秦王政}\
  \toprule
  \SimHei \normalsize 年数 & \SimHei \scriptsize 公元 & \SimHei 大事件 \tabularnewline
  % \midrule
  \endfirsthead
  \toprule
  \SimHei \normalsize 年数 & \SimHei \scriptsize 公元 & \SimHei 大事件 \tabularnewline
  \midrule
  \endhead
  \midrule
  元年 & 1314 & \tabularnewline\hline
  二年 & 1315 & \tabularnewline\hline
  三年 & 1316 & \tabularnewline\hline
  四年 & 1317 & \tabularnewline\hline
  五年 & 1318 & \tabularnewline\hline
  六年 & 1319 & \tabularnewline\hline
  七年 & 1320 & \tabularnewline
  \bottomrule
\end{longtable}


%%% Local Variables:
%%% mode: latex
%%% TeX-engine: xetex
%%% TeX-master: "../Main"
%%% End:

%% -*- coding: utf-8 -*-
%% Time-stamp: <Chen Wang: 2019-10-18 15:46:02>

\section{英宗\tiny(1320-1323)}

元英宗硕德八剌,是元朝第五位皇帝,蒙古帝国第九位大汗,1320年4月19日—1323年9月4日在位,在位3年零5个月,是元仁宗之子。

1321年11月28日,群臣为硕德八剌上汉语尊号继天体道敬文仁武大昭孝皇帝。

去世后,谥号睿圣文孝皇帝,庙号英宗,蒙古语称号格坚皇帝。

延祐七年农历正月二十一日(1320年3月1日),元仁宗去世。延祐七年农历三月十一日(1320年4月19日),18岁的硕德八剌在太皇太后答己及右丞相铁木迭儿等人的扶持下,在大都大明殿登基称帝,是为元英宗,改元“至治”。英宗自幼受儒學薰陶,登基后推行“以儒治國”政策,但是前期英宗的权力受到太皇太后答己和权臣铁木迭儿的很大限制。

延祐七年农历五月十一日(1320年6月17日),元英宗任命拜住为左丞相,以遏制太皇太后和铁木迭儿的权力扩张。

至治元年农历十一月九日(1321年11月28日),群臣为元英宗上尊号继天体道敬文仁武大昭孝皇帝。

1322年10月6日右丞相铁木迭儿去世,1322年11月1日太皇太后去世 ,元英宗终于得以亲政。

至治二年农历十月二十五日(1322年12月4日),元英宗任命拜住为中书右丞相,并且不设左丞相,以拜住为唯一的丞相。在右丞相拜住、中书省平章政事张珪等的帮助下,元英宗进行改革,并实施了一些新政,比如裁减冗官,监督官员不法行为,颁布新法律,采用“助役法”以减轻人民的差役负担,等等。史称“至治改革”。

英宗在位后期,官修政書《大元圣政国朝典章》(《元典章》),内容包括元太宗六年(1234年)到元英宗至治二年(1322年)大约90年的政治、经济、军事、法律等方面官方资料,具有极高的史料价值。

至治三年农历二月十九日(1323年3月26日),元英宗颁布了继《至元新格》之后元朝第二部法律典籍—《大元通制》,一共有二千五百三十九条,其中断例七百一十七、条格千一百五十一、诏赦九十四、令类五百七十七。

元英宗曾经想把征东行省(高丽王国)郡县化,罢征东行省,改立三韩行省,完全和元朝的其他行省一个待遇,“制式如他省,诏下中书杂议”,因为集贤大学士王约说:“高丽去京师四千里,地瘠民贫,夷俗杂尚,非中原比,万一梗化,疲力治之,非幸事也,不如守祖宗旧制。”得到丞相的赞同,设立三韩行省奏议没有实行。最终高丽国祚得以存续,高丽人知道后,为王约画像带回高丽,为之立生祠,并说:“不绝国祀者,王公也。”

元英宗的新政使得元朝国势大有起色,但新政却触及到了蒙古保守贵族的利益,引起了他们的不满,而且英宗下令清除朝中铁木迭儿的势力,随着清理的扩大化,铁木迭儿的义子铁失在至治三年八月初四(1323年9月4日)趁着英宗从上都避暑结束南返大都途中,在上都以南15公里的地方南坡的刺杀了英宗及右丞相拜住等人。史称南坡之变。英宗去世时年仅21岁。

泰定元年农历二月十六日(1324年3月11日),元泰定帝为硕德八剌上谥号睿圣文孝皇帝,庙号英宗。

泰定元年农历四月八日(1324年5月1日),元泰定帝为硕德八剌上蒙古文稱号“格坚皇帝”。

明朝官修正史《元史》宋濂等的評價是:“英宗性刚明,尝以地震减膳、彻乐、避正殿,有近臣称觞以贺,问:‘何为贺?朕方修德不暇,汝为大臣,不能匡辅,反为谄耶?’斥出之。拜住进曰:‘地震乃臣等失职,宜求贤以代。’曰:‘毋多逊,此朕之过也。’尝戒群臣曰:‘卿等居高位,食厚禄,当勉力图报。苟或贫乏,朕不惜赐汝;若为不法,则必刑无赦。’八思吉思下狱,谓左右曰:‘法者,祖宗所制,非朕所得私。八思吉思虽事朕日久,今其有罪,当论如法。’尝御鹿顶殿,谓拜住曰:‘朕以幼冲,嗣承大业,锦衣玉食,何求不得。惟我祖宗栉风沐雨,戡定万方,曾有此乐邪?卿元勋之裔,当体朕至怀,毋忝尔祖。’拜住顿首对曰:‘创业惟艰,守成不易,陛下睿思及此,亿兆之福也。’又谓大臣曰:‘中书选人署事未旬日,御史台即改除之。台除者,中书亦然。今山林之下,遗逸良多,卿等不能尽心求访,惟以亲戚故旧更相引用邪?’其明断如此。然以果于刑戮,奸党畏诛,遂构大变云。”

清朝史学家邵远平《元史类编》的評價是:“册曰:三载承乾,庶务锐始;大飨躬亲,致哀尽礼;刚过鲜终,肘腋祸起;不察几先,励精徒尔。”

清朝史学家毕沅《续资治通鉴》的評價是:“帝性刚明,尝以地震,减膳,彻乐,避正殿,有近臣称觞以贺,问:‘何为贺?朕方修德不暇,汝为大臣,不能匡辅,反为谄耶?’斥出之。尝戒群臣曰:‘卿等居高位,食厚禄,当勉力图报。苟或贫乏,朕不惜赐汝;若为不法,则必刑无赦。’巴尔济苏下狱,谓左右曰:‘法者,祖宗所制,非朕所得私。巴尔济苏虽事朕日久,今有罪,当论如法。’尝御鹿顶殿,谓拜珠曰:‘朕以幼冲,嗣承大业,锦衣玉食,何求不得!惟我祖宗栉风沐雨,戡定万方,曾有此乐耶?卿元勋之裔,当体朕至怀,毋忝尔祖!’拜珠顿首谢曰:‘创业维艰,守成不易,陛下言及此,亿兆之福也。’又谓大臣曰:‘中书选人署事未旬日,御史台即改除之。台除亦然。今山林之士,遗逸良多,卿等不能尽心求访,惟以亲戚故旧更相引用耶?’其明断如此。然以果于刑戮,奸党惧诛,遂构大变云。”

清朝史学家魏源《元史新编》的評價是:“旧史谓英宗果于诛戮,奸党畏惧,遂构大变。乌乎!是何言与?以铁木迭儿之奸,不明正其诛,但疏远俾得善终于位,已为漏网,而复任用其子,曲贷其子,酿成枭獍。此失之果乎?失之不果乎?拜住于铁木迭儿引其党参政张思明自助时,或告拜住为备,拜住反以大臣不和,彼仇我报,非国家之利。及铁木迭儿死,又往哭之痛,此皆失之果乎?失之不果乎?且除奸莫要于夺兵权,乃以宿卫新兵掌于铁失之手。司徒刘夔冒卖浙田之案,真人蔡道泰杀人赇逭之案,皆奸赃巨万。拜住既平反其狱,独赦铁失不问。中书参议谏以除奸不可犹豫,犹豫恐生他变,拜住是其言而不能用。大抵安童、拜住皆木华黎之孙,木华黎用兵所过,动辄屠戮。安童从许衡受学,故其子孙皆出于宽容,以水懦救火猛,德量有余,机警不足,所谓君子之过过于厚也。乃胡粹中因旧史之言,谓英宗在位数载,除诛戮外无一善政可纪,甚至皇太后以嬖孽失势之故,郁郁而终,胡氏并指为英宗不孝祖母之罪。乌乎!其性与人殊,乃至此乎?”

清朝史学家曾廉《元书》的評價是:“论曰:英宗知赵世炎之非辜,抑亦汉昭之流亚也。然汉昭能诛燕王、上官桀,而专任霍光,英宗不能诛铁木迭儿诸权倖之徒,独任拜住也。抑考元时蒙古人横不可悉裁以法度,以拜住之世旧勋贵而不能骤正也。夫自古无无小人之朝,在振纪纲而已。自世祖好货开倖进之门,安童不能与阿合马、桑哥争,况幼沖在位乎?使霍光处此,则必射隼,于高墉藏器,儃回操刀,弗割明君贤相,胥受其祸,悲夫!”

清末民初史学家屠寄《蒙兀儿史记》的評價是:“汗性刚明,励精图治,尝御上都大安阁,见太祖、世祖遗衣,皆以缣素木绵为之,重加补缀。嗟叹良久,谓侍臣曰:‘祖宗草昧经营,服御节俭乃尔,朕焉敢顷刻忘之。’敕画《蚕麦图》于鹿顶殿,以时观之,藉知民事。一日御殿,谓拜住曰:‘朕冲龄嗣祚,锦衣玉食,何求不得。惟我祖宗节风沐雨,戡定大难,曾有此乐耶?卿元勋之裔,当体朕至怀,毋忝尔祖。’拜住顿首对曰:‘创业惟艰,守成亦不易,陛下睿思及此,亿兆之福也。’汗承延祐宽政之后,思济之以猛,御下甚严,在谅闇中。中书参议乞失监坐鬻官,刑部议法当杖,太后欲改笞,汗不可,曰:‘法者,天下之公,徇私而轻重之,非所以示民也。’卒从部议。每戒群臣曰:‘卿等居高位,食厚禄,当勉力图报。苟或贫乏,朕不惜赐汝;若为不法,则必刑无赦。’八思吉思下狱时,汗谓左右曰:‘法者,祖宗之制,非朕所得私。八思吉思虽事朕日久,今既有罪,当论如法。’其明决如此。然过信喇嘛,大起山寺,不受忠谏,饮酒逾量,有时至失常度云。”

民国官修正史《新元史》柯劭忞的評價是:“英宗诛兴圣太后幸臣失列门等,太后坐视而不能救,其严明过仁宗远甚。然蔽于铁木迭儿,既死始悟其奸,又置其逆党于肘腋之地。故南坡之祸。由于帝之失刑,非由于杀戮也。旧史所讥殆不然矣。”

\subsection{志治}

\begin{longtable}{|>{\centering\scriptsize}m{2em}|>{\centering\scriptsize}m{1.3em}|>{\centering}m{8.8em}|}
  % \caption{秦王政}\
  \toprule
  \SimHei \normalsize 年数 & \SimHei \scriptsize 公元 & \SimHei 大事件 \tabularnewline
  % \midrule
  \endfirsthead
  \toprule
  \SimHei \normalsize 年数 & \SimHei \scriptsize 公元 & \SimHei 大事件 \tabularnewline
  \midrule
  \endhead
  \midrule
  元年 & 1321 & \tabularnewline\hline
  二年 & 1322 & \tabularnewline\hline
  三年 & 1323 & \tabularnewline
  \bottomrule
\end{longtable}


%%% Local Variables:
%%% mode: latex
%%% TeX-engine: xetex
%%% TeX-master: "../Main"
%%% End:

%% -*- coding: utf-8 -*-
%% Time-stamp: <Chen Wang: 2019-10-18 15:52:31>

\section{泰定帝\tiny(1323-1328)}

元泰定帝也孙铁木儿是元朝第六位皇帝,蒙古帝国第十位大汗,在位5年,自1323年10月4日至1328年8月15日。清代乾隆晚期乾隆帝命改譯遼、金、元三史中的音譯專名,改譯伊蘇特穆爾,今日學界已無人使用。

他去世后不久,叔父之孫元文宗打敗其子元天顺帝,亦使他沒被授與谥号和庙号,因此历史上以其年号称之为泰定帝。

关于泰定帝的出生年,《元史》中的说法互相矛盾,在《元史·泰定帝一》中称“至元十三年十月二十九日,帝生于晋邸。”至元十三年是1276年,但在《元史·泰定帝二》中又说“庚午,帝崩,寿三十六”,按这个说法他应该是1293年(至元三十年)出生的。很可能作者误把“三十”写成了“十三”。泰定帝“生于晋邸”,而1292年甘麻剌被封为晉王,而且1328年他的长子阿剌吉八當時只有8岁,所以泰定帝应该是生于1293年。他的父亲甘麻剌是元世祖太子真金的長子,1292年被封为晉王,出镇嶺北。1302年甘麻剌死后也孙铁木儿袭晋王位。

至治三年(1323年)三月也孙铁木儿在元英宗附近的亲信向他告密说英宗将对也孙铁木儿不利。同年八月二日,也孙铁木儿获得英宗将被刺杀、自己将被迎立为皇帝的消息。

至治三年八月初四(1323年9月4日),铁木迭儿的义子铁失趁着元英宗从上都避暑结束南返大都途中,在上都以南15公里的地方南坡的刺杀了元英宗及右丞相拜住等人。史称南坡之变。

元英宗被刺后也孙铁木儿果然被擁立为皇帝,至治三年九月初四日(1323年10月4日),也孙铁木儿在漠北草原的龙居河(今克鲁伦河)河畔登基称帝。虽然也孙铁木儿是知情人,但他登基后就下令将刺杀英宗的人都處死了。

至治三年十一月十三日(1323年12月11日),泰定帝到达大都。1323年12月17日,泰定帝在大都大明殿接受诸王和百官朝贺。

至治三年十二月十一日(1324年1月7日),泰定帝追尊其父亲甘麻剌為皇帝,为甘麻剌上庙号显宗,汉文谥号光圣仁孝皇帝;追尊其母亲普顏怯里迷失为皇后,为普顏怯里迷失上谥号宣懿淑圣皇后。

泰定元年三月二十日(1324年4月14日),泰定帝立八八罕氏为皇后,立阿速吉八为太子。

从1325年开始,泰定帝因国库收入少于支出,开始减少国家支出。七月,他下令不允许汉人收藏和携带兵器。

泰定二年九月初一日(1325年10月8日),泰定帝改革全国的行政区划,将全国划分为18个道,分别为:两浙道、江东道、江西道、福建道、江南道、湖广道、河南道、江北道、燕南道、山东道、河东道、陕西道、山北道、辽东道、云南道、甘肃道、四川道、京畿道。

泰定帝还下达了一系列命令禁止和尚和道士购买民间的土地,克制僧院的过分富有。

在泰定帝统治期间,广西、四川、湖南、云南等少数民族地区经常爆发反抗元朝统治的暴乱,泰定帝一般使用软硬兼施的手段来平息这些暴乱。但从整体来说整个国家基本上比较安宁。

致和元年七月初十日(1328年8月15日),元泰定帝在上都病逝,享年36岁。

元泰定帝七月去世后,九月,他的儿子元天顺帝在上都登基,改元天顺,九月十三日,元武宗之子元文宗在大都登基,改元天历,双方交战一个月,最终以元文宗获胜告终,元天顺帝失败后下落不明,不知所终。

也孙铁木儿无庙号和谥号,故以年号史称为泰定帝。

明朝宋濂等官修正史《元史》的評價是:“泰定之世,灾异数见,君臣之间,亦未见其引咎责躬之实,然能知守祖宗之法以行,天下无事,号称治平,兹其所以为足称也。”

清朝史学家邵远平《元史类编》的評價是:“册曰:长子世嫡,嗣统允宜;武仁先立,泽承人思;忽焉不世,电灭云移;或曰南坡,其蛮与知;故史具在,其又谁欺?”

清朝史学家毕沅《续资治通鉴》的評價是:“帝在位,灾异数见,然能守祖宗之法,天下号称治平。”

清朝史学家魏源《元史新编》的評價是:“一代统绪之传,有正统即有公论,岂一时私意所能傎倒磔裂者哉!世祖明孝太子早卒,皇孙成宗立,追谥裕宗。成宗本裕宗第三子,其同母二兄,一为晋王甘麻剌,一为怀王答剌麻八剌,本无嫡庶,而晋邸居长。成宗崩后无嗣,晋王之子泰定帝即可嗣立,乃因仁宗自怀庆入,先靖内难,迎立其兄怀宁王于漠北,是为武宗。所谓先入关者王之,非晋王子不当立而必立怀王子也。及再传至英宗遇弑,晋王复出自漠北入靖内难,讨贼嗣位,是为泰定。与武、仁之事相埒,非武、仁有功宗社,而泰定无功也。泰定践阼,即以和林兵柄授周王使代己任,屡通朝贡。又召怀王自海南入朝京师,锡封藩国,移近江陵,屡赐金币,是泰定于文宗兄弟有德而无怨也。泰定太子册立已五载,父终子继,名正言顺,怀王、周王安得入干大统乎!若谓武、仁当日原有传位周王,嗣及英宗之约,则仁宗实背约在前,可以责仁宗,不可以责泰定也。乃文宗篡立之诏,谓泰定以旁支入继,正统遂偏,甚至诬其与贼臣铁失潜通阴谋,冒干宝位,追毁晋王显宗庙室。乌乎!以讨贼之主,而诬以通贼之罪,是何言哉!若谓武宗二子为人心所归,泰定当舍子而传侄,则何以天历颁诏至关中、至四川、至辽东,皆焚书斩使,起兵拒命,则人心归泰定之子,而不归武宗之子,明如星日。是则燕帖木儿之为逆臣,怀王之为逆立,亦明如星日。固不待鲁桓弑隐夺国,已无所逃于《春秋》之责,况欲宽其罪于中途弑逆之后哉!斯非难定之案,而数百年尚无定论。请断之,以折曲沃桓叔之徒,假托正谊者。”

清朝史学家曾廉《元书》的評價是:“论曰:周太王以国传王季,设季而无后,则泰伯之子孙遂不可以复承周祀乎?美哉晋王之让,而泰定之立,亦不可不畏之正也。上都告变,惜已无及,然大节亦明矣。故诸凶迁官非有他也,仓卒之间,形格势禁,度权力未足以制其命也。荣宠以诱之,俾喜而懈,稍缓须臾,成备而出,而疾雷不及掩耳矣。呜呼!此帝之所以为权,然岂不果哉!至后纪纲弗振,由不纳张珪、宋本之言,而乱是用长也,累受佛戒,亦梁武之俦乎?”

清末民初史学家屠寄《蒙兀儿史记》的評價是:“至元六年,诏称英宗遇害,正统遂偏,于戏!此惠宗一人之私言也。太子真金嫡子三人,泰定之父晋王甘麻剌最长,次则武仁之父答剌麻八剌,又次为成宗。成宗之立,非世祖本意也。向使储闱符玺之归,果足为大统继嗣之证,则当世祖宾天,诸王大会,成宗曷不径遵遗诏,即位梓宫之前,出受群臣之贺。顾乃迟回三月,必得晋王北面愿事之一言,而大策始定,何也?盖成宗以皇孙出抚北军时,既无王号,又未赐印,世祖用玉昔帖木儿之请,濒行仓卒,授以故太子宝,代一时行军印之用而已。非有告庙册立之礼也。晋王不让,成宗不得立。则所谓正统,宜属晋王之子孙。明史臣王祎言:武宗约继世子孙,兄弟相及。而仁宗不守宿诺,传位英宗,仍使武宗二子出居于外。及英宗遇弑,而明宗在北,文宗在南。嗣晋王于世祖为嫡长曾孙,则求所当立,舍嗣晋王谁归?旧传英宗之弑,晋邸与闻,考之宝录,不得其证。传闻之缪,殊不足信。邵阳魏氏源亦言:成宗无嗣,大统当归,泰定徒以仁宗自怀先入,靖内难而迎立武宗,所谓先入关者王之,非晋王子不当立,而必立答剌麻八剌子也。泰定能讨贼,胜于武、仁杀疑似之宗亲,非武、仁有功社稷,而泰定无功也。泰定践阼,即归周王之妃八不沙于漠北,召图帖睦尔汗于海南,既至京师,厚加赐予。封为怀王,妻以主女。初镇建康,六朝都会;及移江陵,益据上游。泰定之于怀王,有德而无怨也。阿速吉八太子册立已五载,父终子继,名正言顺,大统所在,孰得干之?若谓武宗当日原有传位周王以及英宗之约,则仁宗实背约在前,可以责仁宗,不可以责泰定也。若谓武宗二子,人心所归,泰定当舍子而传侄,何以山后、辽东、关陇、滇、蜀,先后为上都起兵,即河南、湖广,犹必执杀省官,易置郡县长吏,强之而后从。当日讴歌讼狱,不之武宗之子,而之泰定之子,明矣。然则燕帖木儿之为逆臣,怀王之为篡立。不待鲁桓弑隐,已无所逃于《春秋》之诛。况可宽其罪于旺兀察都推刃天伦之后哉!斯狱县之六百年,请断之,以折曲沃桓叔之徒,假托名义者。”(至元六年为1338年,此处至元为元惠宗年号。)

民国柯劭忞官修正史《新元史》的評價是:“孔子称叔孙昭子之不劳。泰定帝讨铁失等弑君之罪,虽叔孙昭子何以尚之。文宗篡立,欲厌天下之人心,诬蔑之辞无所不至。惜乎后世之君子,不引孔子之言,以论定其事也。”

\subsection{泰定}

\begin{longtable}{|>{\centering\scriptsize}m{2em}|>{\centering\scriptsize}m{1.3em}|>{\centering}m{8.8em}|}
  % \caption{秦王政}\
  \toprule
  \SimHei \normalsize 年数 & \SimHei \scriptsize 公元 & \SimHei 大事件 \tabularnewline
  % \midrule
  \endfirsthead
  \toprule
  \SimHei \normalsize 年数 & \SimHei \scriptsize 公元 & \SimHei 大事件 \tabularnewline
  \midrule
  \endhead
  \midrule
  元年 & 1324 & \tabularnewline\hline
  二年 & 1325 & \tabularnewline\hline
  三年 & 1326 & \tabularnewline\hline
  四年 & 1327 & \tabularnewline\hline
  五年 & 1328 & \tabularnewline
  \bottomrule
\end{longtable}

\subsection{致和}

\begin{longtable}{|>{\centering\scriptsize}m{2em}|>{\centering\scriptsize}m{1.3em}|>{\centering}m{8.8em}|}
  % \caption{秦王政}\
  \toprule
  \SimHei \normalsize 年数 & \SimHei \scriptsize 公元 & \SimHei 大事件 \tabularnewline
  % \midrule
  \endfirsthead
  \toprule
  \SimHei \normalsize 年数 & \SimHei \scriptsize 公元 & \SimHei 大事件 \tabularnewline
  \midrule
  \endhead
  \midrule
  元年 & 1328 & \tabularnewline
  \bottomrule
\end{longtable}


%%% Local Variables:
%%% mode: latex
%%% TeX-engine: xetex
%%% TeX-master: "../Main"
%%% End:

%% -*- coding: utf-8 -*-
%% Time-stamp: <Chen Wang: 2019-10-18 15:53:31>

\section{天顺帝\tiny(1328)}

元天顺帝阿剌吉八,是元朝第七位皇帝,蒙古帝国第十一位大汗,元泰定帝之子。1328年10月3日至1328年11月14日在位,在位一个月十一天。

致和元年七月初十日(1328年8月15日),元泰定帝也孙铁木儿在上都病逝,丞相倒剌沙专权自用,过了一个多月仍迟迟不立9岁的太子阿剌吉八即位。

致和元年九月十三日(1328年10月16日),知樞密院事燕帖木儿在大都(今北京)拥立元武宗之子图帖睦尔即位,改元“天历”,图帖睦尔是为元文宗。

致和元年九月,丞相倒剌沙在上都拥立太子阿剌吉八为皇帝,改元“天顺”。

上都的天顺帝朝廷由丞相倒剌沙派兵进攻大都的文宗朝廷,元文宗派燕帖木儿率军迎战,双方经过多次战争,一开始双方互有胜负,后来大都朝廷逐渐占据军事优势。

天顺元年十月十三日(1328年11月14日),大都朝廷的军队包围上都,丞相倒剌沙等大臣奉皇帝宝出降,天顺年号被元文宗废除,倒剌沙在投降一个月后被杀。

倒剌沙投降后,天顺帝下落不明,不知所终,在位大约一個月;其後他沒被授與谥号和庙号,因此历史上以其年号称之为天顺帝。

清朝史学家曾廉《元书》的評價是:“论曰:曾子以托孤寄命,临大节而不可夺,斯为君子人也。故山有猛虎,樵采不入。前史称泰定帝能守祖宗之法,故天下无事。呜呼!徒法不能以自行也,向使汉武不委裘于霍光、金日磾,而倚上官桀、桑弘羊,则孝昭岂得晏然南面?况又弗如孝昭者乎?狙于近习而不知求天下之贤以佐佑之,贵为天子,富有天下,而不能庇其妻孥,若敖之鬼佞焉咎安在哉!君子是以不多子孟,而乐道孝武之善付托也。”

\subsection{天顺}

\begin{longtable}{|>{\centering\scriptsize}m{2em}|>{\centering\scriptsize}m{1.3em}|>{\centering}m{8.8em}|}
  % \caption{秦王政}\
  \toprule
  \SimHei \normalsize 年数 & \SimHei \scriptsize 公元 & \SimHei 大事件 \tabularnewline
  % \midrule
  \endfirsthead
  \toprule
  \SimHei \normalsize 年数 & \SimHei \scriptsize 公元 & \SimHei 大事件 \tabularnewline
  \midrule
  \endhead
  \midrule
  元年 & 1328 & \tabularnewline
  \bottomrule
\end{longtable}


%%% Local Variables:
%%% mode: latex
%%% TeX-engine: xetex
%%% TeX-master: "../Main"
%%% End:

%% -*- coding: utf-8 -*-
%% Time-stamp: <Chen Wang: 2021-11-01 17:07:22>

\section{文宗图帖睦尔\tiny(1328-1332)}

\subsection{生平}

元文宗图帖睦尔,是元朝第八位皇帝,蒙古帝国第十二位大汗,两次在位,第一次在位时间为1328年10月16日—1329年4月3日;後復位,第二次在位时间为1329年9月8日—1332年9月2日,在位时间共4年,他是元武宗的次子。清代乾隆晚期乾隆帝命改譯遼、金、元三史中的音譯專名,改譯圖卜特穆爾,今日學界已無人使用。

1330年5月25日,群臣为图帖睦尔上汉语尊号钦天统圣至德诚功大文孝皇帝。

他去世后,谥号圣明元孝皇帝,庙号文宗,蒙古語称札牙篤皇帝。

致和元年七月十日(1328年8月15日),元泰定帝在上都去世。八月,在大都(今北京)的燕帖木儿等大臣决定立元武宗的长子周王和世㻋为帝,但是因为路远而先迎周王之弟怀王图帖睦尔(元文宗)。九月,在上都的倒剌沙等大臣則立太子阿速吉八为帝,是為天顺帝,并发兵攻大都。

天曆元年九月十三日(1328年10月16日),知樞密院事燕帖木儿在大都拥立图帖睦尔在大都大明殿即位称帝,并在即位诏中改致和元年为天曆元年。燕帖木儿经过多次战争,于1328年11月14日打败位于上都的天顺帝朝廷,天下安定。

元文宗采納燕帖木儿的建议,照原本的安排立自己的兄長周王和世㻋为帝,是为元明宗。1329年2月27日,元明宗在漠北草原和宁之北即位,并派遣撒迪等人前往大都通知元文宗;但直到1329年4月3日,在大都的元文宗才派遣燕铁木儿和众多官员奉皇帝宝玺前往元明宗行在所,正式让出皇位。5月5日,燕铁木儿率百官将皇帝宝玺献给元明宗。5月15日,元明宗正式立图帖睦尔为皇太子(實應為皇太弟)。8月16日,图帖睦尔受皇太子宝。8月25日,元明宗抵达元武宗时建为中都的王忽察都。8月26日,皇太子图帖睦尔入见,两兄弟会面,元明宗宴请皇太子及诸王、大臣于行殿。1329年8月30日,燕帖木儿毒死元明宗。

天曆二年八月十五日(1329年9月8日),在燕帖木儿等官员的拥戴下,元文宗于上都大安阁再次即位称帝,并发布第二次即位诏;因該年的年号是天曆,史称天曆之变。

元文宗第一次在位期间,於天历二年二月二十七日(1329年3月27日)設立了奎章閣學士院,掌進講經史之書,考察歷代治亂,又令所有勛貴大臣的子孫都要到奎章閣學習;奎章閣下設藝文監,專門負責將儒家典籍譯成蒙古文,以及校勘。同年下令編纂《經世大典》,兩年後修成,為元代一部重要的記述典章制度的巨著。元文宗第二次登基后亦大兴文治。

至顺元年五月八日(1330年5月25日),丞相燕帖木儿率文武百官及僧道、耆老,奉玉册、玉宝,为元文宗上尊号钦天统圣至德诚功大文孝皇帝。

元文宗在位期间,丞相燕帖木儿自恃有功,玩弄朝廷,元朝朝政更加腐败,国势更加衰落。文宗在位期间国内多次爆发民变,大动乱正在酝酿之中。

至顺三年八月十二日(1332年9月2日),元文宗在上都病逝,终年28岁。

元统元年十一月二十一日(1333年12月28日),侄子元順帝为图帖睦尔上谥号圣明元孝皇帝、庙号文宗,蒙古语称札牙笃皇帝。

文宗頗具漢文化修養,喜愛作詩。《宋元詩會》記載:文宗怡情詞翰,雅喜登臨。居金陵潛邸時,常屏從官,獨造鍾山冶亭,吟賞竟日,惜現存詩作僅有數首而已。又精於書畫。《元史》記載,文宗的書法受趙孟頫影響而宗晉人,落筆過人,得唐太宗晉祠碑風,遂益超旨。文宗曾命近臣房大年畫《京都萬歲山圖》,房大年以為自己火候未到而請辭。文宗於是索紙運筆,先作一稿,大年驚服,謂格法周匝停勻,雖積學專工,莫能及也。文宗的書畫作品在今日極為罕見,僅有《相馬圖》一幅。

清朝史学家邵远平《元史类编》的評價是:“册曰:应变戡乱,莫匪尔劳;玺绶虽去,太阿已操;前车所鉴,烛影斧声;从来疑案,多在弟兄。”

清朝史学家魏源《元史新编》的評價是:“元代诸帝不习汉文,凡有章奏,皆由翻译。其读汉书而不用翻译者,前惟太子真金,从王恽、王恂受学。后惟文宗潜邸,自通汉文而已。《书画谱》言,文宗在潜邸时,召画师房大年,俾图京师万岁山。大年以未至其地辞,文宗遂取笔布画位置,顷刻立就,命大年按稿图上。大年得稿敬藏之,意匠经营,虽积学专工,有所未及。始知文宗之多材多艺也。及践阼后,开奎章阁,招集儒臣,撰备《经世大典》数百卷,宏纲巨目,礼乐兵农,灿然开一代文明之治。即其声色俭澹,亦远胜武宗,此岂庸主所希及哉!使其迎立明宗之日,亦如仁宗之退处东宫,他日明宗复如武宗之传仁庙,则一代而胜事再见,虽殷人弟兄世及,何以过此!《易》曰:‘开国承家,小人勿用。’文宗之得大位也,以燕帖木儿;其得罪万世也,亦以燕帖木儿。语曰:‘治世之能臣,乱世之奸雄。’文宗之不陨于太平王手者,亦幸矣哉!”(魏源说“元代诸帝不习汉文,凡有章奏,皆由翻译。”此事并不符合历史事实,这和他了解的相关书籍不多有关。事实上,真金太子和元文宗的汉文学修养的确很高,除此之外,还有很多位元朝帝王有很高的汉文学修养。根据史料, 元世祖、元成宗、元仁宗、元英宗、元文宗、元順帝、元昭宗,均有很高的汉文化修养,其中,元世祖、元文宗、元順帝、元昭宗这四位帝王有汉文诗传世。元仁宗、元英宗、元憲宗和元文宗都受到过良好的汉学教育,都有很高的汉文学修养。

清朝史学家曾廉《元书》的評價是:“论曰:元自文宗,始亲郊祀,礼彬彬焉。尊崇圣贤之典,至是益隆,而开奎章阁以致儒臣,考文章,论治道,勤于延访,可以为文矣。然几沉而气锐,抑亦吴闾庭之流也。其言泰定帝通贼臣,阴谋冒干宝位,呜呼!文宗将毋其自道之也!兴且晋邸,日有盟书,周王可必其终为泰伯乎?文宗之深心乃以让,济其忍,然后足固其威福也,岂不险哉!生则欺人,死而犹饰,故地碎其主,春秋震夷伯之庙,所谓有隐慝者乎?”

清末民初史学家屠寄《蒙兀儿史记》的評價是:“汗旧劳于外,多艺好文。在建康潜邸时,忽忆京师万岁山,召画师房大年图之,大年以未至其地辞,汗自取笔,布画位置,顷刻立就,命大年按稿图上。大年得稿敬藏之,意匠经营,虽积学专工,有所未及。即位后首建奎章阁,御制记文,集儒臣阁中备顾问,敕编《经世大典》,保存一代制度。性爱典礼,欲革蒙兀腥膻本俗,则躬服衮冕,虔祀郊庙。又慎于用刑,行枢密院尝当云南逃军二人死罪,汗谓:‘临阵而逃,死宜也。彼非逃战,辄当以死,何视人命之易耶?’杖而流之。天历初抗命诸王大臣,临事故多诛杀,其它窜黜者,事后多蒙召还,或仍录用。至于严惩赃吏,尊信老成,节诸王驸马朝会刍粟赏赐之财,汰宿卫鹰坊饔人僧徒冗食之数。诸所设施,实一代恭俭守文之令主也。惟得国不正,隐亏天伦,且授权燕铁木儿太甚,未能大有为。”

民国官修正史《新元史》柯劭忞的評價是:“燕铁木儿挟震主之威,专权用事。文宗垂拱于上,无所可否,日与文字之士从容翰墨而已。昔汉灵帝好词赋,召乐松等待诏鸿都门,蔡邕露章极谏,斥为俳优。况区区书画之玩乎?君子以是知元祚之哀也。”

\subsection{天历}

\begin{longtable}{|>{\centering\scriptsize}m{2em}|>{\centering\scriptsize}m{1.3em}|>{\centering}m{8.8em}|}
  % \caption{秦王政}\
  \toprule
  \SimHei \normalsize 年数 & \SimHei \scriptsize 公元 & \SimHei 大事件 \tabularnewline
  % \midrule
  \endfirsthead
  \toprule
  \SimHei \normalsize 年数 & \SimHei \scriptsize 公元 & \SimHei 大事件 \tabularnewline
  \midrule
  \endhead
  \midrule
  元年 & 1328 & \tabularnewline\hline
  二年 & 1329 & \tabularnewline\hline
  三年 & 1330 & \tabularnewline
  \bottomrule
\end{longtable}

\subsection{志顺}

\begin{longtable}{|>{\centering\scriptsize}m{2em}|>{\centering\scriptsize}m{1.3em}|>{\centering}m{8.8em}|}
  % \caption{秦王政}\
  \toprule
  \SimHei \normalsize 年数 & \SimHei \scriptsize 公元 & \SimHei 大事件 \tabularnewline
  % \midrule
  \endfirsthead
  \toprule
  \SimHei \normalsize 年数 & \SimHei \scriptsize 公元 & \SimHei 大事件 \tabularnewline
  \midrule
  \endhead
  \midrule
  元年 & 1330 & \tabularnewline\hline
  二年 & 1331 & \tabularnewline\hline
  三年 & 1332 & \tabularnewline\hline
  四年 & 1333 & \tabularnewline
  \bottomrule
\end{longtable}


%%% Local Variables:
%%% mode: latex
%%% TeX-engine: xetex
%%% TeX-master: "../Main"
%%% End:

%% -*- coding: utf-8 -*-
%% Time-stamp: <Chen Wang: 2019-12-26 14:54:31>

\section{明宗\tiny(1329)}

\subsection{生平}

元明宗和世㻋,是元朝第九位皇帝,蒙古帝国第十三位大汗,1329年2月27日至1329年8月30日在位,在位185天。元武宗長子。清代乾隆晚期乾隆帝命改譯遼、金、元三史中的音譯專名,改譯和實拉,今日學界已無人使用。

他去世后,谥号翼獻景孝皇帝,庙号明宗,蒙古语称忽都篤皇帝。

1340年10月25日,元惠宗為元明宗上汉语尊号順天立道睿文智武大聖孝皇帝。

根据《元史》,天曆二年正月丙戌(儒略曆1329年2月27日),和世琜在漠北草原的和宁之北即位,继续使用年号“天曆”,是为元明宗,1329年4月3日,元文宗图帖睦尔派人将皇帝宝玺献给明宗,正式禪讓帝位,5月15日,元明宗正式立图帖睦尔为皇太子,8月16日,图帖睦尔受皇太子宝,8月25日,元明宗抵达元武宗时建为中都的王忽察都,8月26日,皇太子图帖睦尔入见,两兄弟会面,元明宗宴请皇太子及诸王、大臣于行殿。

天曆二年八月六日(1329年8月30日),元明宗和世㻋被燕帖木儿毒死,明宗去世时享年30岁。

1329年9月8日,燕帖木儿重新拥戴元文宗復辟,因为1329年的年号是天曆,史称天曆之变。

天曆二年十月十三日(1329年11月4日),元文宗为兄長和世㻋上谥号翼獻景孝皇帝,庙号明宗,蒙古文称忽都篤皇帝。

元明宗的两个儿子元宁宗懿璘质班和元惠宗妥懽帖睦尔在1332年9月2日元文宗去世后相继登基称帝。

至元六年十月四日(1340年10月25日),元惠宗给元明宗上尊号順天立道睿文智武大聖孝皇帝。

清朝史学家邵远平《元史类编》的評價是:“册曰:艰艰备尝,人望所属;何嫌何疑,推肝置腹;人心不同,天命反覆;论定千秋,此直彼曲。”

清朝史学家曾廉《元书》的評價是:“论曰:昔曹子臧、吴季札,贤者也。其君国子民也宜哉!然而义不受者,非独远情,亦知负飞及光之不厌,其欲将无以善其后也,闇哉明宗!焉有人披衮执玉,穆穆然位乎天位而肯北面俯首为人臣者乎?呜呼!此唐明皇不敢以望肃宗,父子且然,况兄弟哉!文宗盖惧北陲,复有海都、笃哇之流,托名拥戴,其言也顺而为患也。深抑亦私心,窃望周王之效法晋邸也。己则非夷,而以齐期人。不亦难乎?悠悠南行,甘咽其饵,悲夫!”

清末民初史学家屠寄《蒙兀儿史记》的評價是:“和世㻋汗年未弱冠,远逊金山,耕牧十有三年。所谓旧劳于外,知民情伪者也。观其论台纲,谕百司,斤斤于先世成宪,是殆有心救弊者乎?然以此言论风采,自曝于风尘道路之间,致令傲弟权相闻而生心,遂有旺兀察都之变。《易》曰:‘君不密,则失臣。’此之谓矣。怀抱盛意,未见设施,惜哉!”

民国官修正史《新元史》柯劭忞的評價是:“燕铁木儿立文宗,文宗固让于兄,犹仁宗之奉武宗也。明宗之弑,盖出于燕铁木儿,非文宗之本意。然与闻乎弑,是亦文宗弑之而已。”

\subsection{天历}

\begin{longtable}{|>{\centering\scriptsize}m{2em}|>{\centering\scriptsize}m{1.3em}|>{\centering}m{8.8em}|}
  % \caption{秦王政}\
  \toprule
  \SimHei \normalsize 年数 & \SimHei \scriptsize 公元 & \SimHei 大事件 \tabularnewline
  % \midrule
  \endfirsthead
  \toprule
  \SimHei \normalsize 年数 & \SimHei \scriptsize 公元 & \SimHei 大事件 \tabularnewline
  \midrule
  \endhead
  \midrule
  二年 & 1329 & \tabularnewline
  \bottomrule
\end{longtable}


%%% Local Variables:
%%% mode: latex
%%% TeX-engine: xetex
%%% TeX-master: "../Main"
%%% End:

%% -*- coding: utf-8 -*-
%% Time-stamp: <Chen Wang: 2021-11-01 17:07:58>

\section{宁宗懿璘质班\tiny(1332)}

\subsection{生平}

元寧宗懿璘质班是元朝第十位皇帝,蒙古帝国第十四位大汗。元明宗次子。1332年10月23日—1332年12月14日在位,在位2个月。

他去世后,谥号冲圣嗣孝皇帝,庙号寧宗。

《元史》记载,元宁宗于泰定三年三月二十九癸酉日(1326年5月1日)生于北方草原。

至顺三年八月十二日(1332年9月2日),元文宗崩。据杂史,元文宗在死前下诏让元明宗之子继承皇位。文宗死后,把持朝政的燕铁木儿为了继续专权,就请求元文宗皇后卜答失里立她的儿子古納答剌为帝。卜答失里为了执行丈夫的遗诏,予以拒绝。由于当时元明宗的长子妥懽贴睦尔(后来的元惠宗)远在广西静江(今广西桂林),而次子懿璘质班却深得文宗宠爱,受封为鄜王,留在文宗身边。

至顺三年十月初四(1332年10月23日),卜答失里皇后遂奉文宗遗诏拥立年仅7岁的懿璘质班在大都大明殿登上皇位,是为元宁宗。因为皇帝年幼,卜答失里皇后临朝称制,成了元朝的实际统治者。

懿璘质班即位后未改元,年号仍旧是“至顺”,至顺三年十一月二十六日(1332年12月14日),元宁宗在大都病逝,年仅7岁,在位仅53天。

至元三年正月十日(1337年2月10日),元惠宗为懿璘质班上谥号冲圣嗣孝皇帝、庙号宁宗。

清朝史学家魏源《元史新编》的評價是:“乌乎!《春秋》未逾年之君称子,故子般不与闵公并立庙谥。宁宗以负扆匝月之殇,而入庙称宗,立后媲谥,无一人引大谊以匡正之,斯元代礼臣博士之陋也。修史者又踵其失而立《本纪》,斯又明臣之陋也。今以附诸《文宗本纪》之末。”

清朝史学家曾廉《元书》的評價是:“论曰:文宗杀明宗皇后,播告天下,言妥懽帖睦尔非明宗子,既出之于静江,乃立皇子阿剌忒答剌为皇太子,公私之情见矣。皇天弗佑,元良夭丧,及大惭,而爱其少子之弱,非妥懽帖睦尔不能延其祚,而不可为之辞矣。则亦曰立明宗子,一似以明其固让之初志也者。任后人之拥戴,犹武宗之孙也。惟宁宗亦弗永年而大位卒,归于向所猜忌之兄子,天也!人岂有为哉!”

清末民初史学家屠寄《蒙兀儿史记》的評價是:“鄜王之立,不再月而殇。既未逾年改元,又未有所建设,顾乃追尊上谥,立庙称宗,甚乖《春秋》鲁般书子卒之义。蒙兀君臣瞢不知经,诚无足责,而明初脩胜国之史,仍立之本纪,不加裁正,宜乎魏源讥其陋也。退附《文宗本纪》,自邵远平始。”

民国官修正史《新元史》柯劭忞的評價是:“《春秋》之义,未逾年之君称子。宁宗即位匝月而殇,乃入庙称宗;其廷臣不学如此,岂非失礼之大者哉。” 

\subsection{志顺}

\begin{longtable}{|>{\centering\scriptsize}m{2em}|>{\centering\scriptsize}m{1.3em}|>{\centering}m{8.8em}|}
  % \caption{秦王政}\
  \toprule
  \SimHei \normalsize 年数 & \SimHei \scriptsize 公元 & \SimHei 大事件 \tabularnewline
  % \midrule
  \endfirsthead
  \toprule
  \SimHei \normalsize 年数 & \SimHei \scriptsize 公元 & \SimHei 大事件 \tabularnewline
  \midrule
  \endhead
  \midrule
  三年 & 1332 & \tabularnewline
  \bottomrule
\end{longtable}


%%% Local Variables:
%%% mode: latex
%%% TeX-engine: xetex
%%% TeX-master: "../Main"
%%% End:

\input{20_Yuan/11_HuiZOng}
%% -*- coding: utf-8 -*-
%% Time-stamp: <Chen Wang: 2021-11-01 17:09:23>

\section{北元\tiny(1368-1388)}

\subsection{简介}

北元指明朝建立并遣徐达大军攻陷元朝首都大都(汗八里)后,退居蒙古高原的原元朝宗室的政權,因国号仍叫大元,以其地处塞北,故稱“北元”[1]。 北元始于元惠宗至正二十八年(1368年,明太祖洪武元年),终于脱古思帖木儿天元十年(明朝洪武二十一年,1388年),为蒙古(明人稱鞑靼)所代替。

元惠宗至正二十八年正月初四日(1368年1月23日)明太祖建立明朝,統一南方,令徐达北伐,徐达率领的军队逼近大都,闰七月二十八日(1368年9月10日),元惠宗夜半开大都的健德门北奔,率太子愛猷識理答臘、后妃、臣僚等撤离大都,八月初二日(1368年9月14日),明军从大都的齐化门攻城而入,元朝对中国的统治结束,回到本土蒙古草原。

元惠宗撤离大都后,继续使用“大元”国号,当时高丽人叫北元。當時政治形势是除了元惠宗據有漠南漠北的蒙古本土,關中還有元將擴廓帖木兒(王保保)駐守甘肅定西,此外元廷還領有东北地区與雲南行中书省地區。

明太祖為了驱逐位于蒙古的元廷势力,採取兵分二路,各個擊破的方式,此即第一次北伐。至正二十八年八月初四日(1368年9月16日),元惠宗到达上都。至正二十九年六月十三日(1369年7月16日),明军逼近上都,元惠宗撤离上都,当天到达应昌。六月十七日(1369年7月20日),明将常遇春攻克上都。

元惠宗在上都和应昌那里曾两次组织兵力试图收复大都,但都被明军击败。至正三十年(洪武三年)四月二十八日(1370年5月23日)元惠宗因痢疾在应昌去世,享年51岁。皇太子愛猷識理答臘在应昌繼承皇位,是为元昭宗,并于1371年改元宣光。至正三十年五月十六日(1370年6月10日),明将李文忠攻克应昌,元昭宗撤至哈拉和林,并坚持抵抗明军。

擴廓帖木兒仍然在漠北多地与明将徐达等人作戰。明太祖曾多次寫信詔降,但擴廓帖木兒從不理会,被朱元璋稱為「當世奇男子」。元昭宗宣光二年(1372年)正月,徐达从雁门出发,向哈拉和林进发。三月,明将蓝玉在土拉河大败扩廓帖木儿。五月,扩廓帖木儿在草原击败明将徐达的这支明军。自此之后,明军十几年不再进攻漠北,直到1388年,蓝玉才再次进攻漠北草原。

宣光二年六月初三(1372年7月3日),明将冯胜大败元军,明朝从元朝治下收取甘肃行中书省地区。

宣光八年(1378年)四月,元昭宗去世,继位的北元后主脫古思帖木兒在1379年六月改年号为天元,继续和明军对抗,屢次侵犯明境[2]。

1371年,元朝辽阳行省平章刘益降明,明朝控制今辽宁南部。然而之外的辽阳行省地区仍由元朝太尉纳哈出控制,纳哈出屯兵二十万于金山(今辽宁省昌图金山堡以北辽河南岸一带),自恃畜牧丰盛,与明军对峙了十几年,多次拒绝明太祖的招抚。1387年,冯胜、傅友德、蓝玉等人發動第五次北伐,目标是攻占纳哈出的金山。经过多次战争,1387年10月,纳哈出投降蓝玉,明朝控制原辽阳等处行中书省的东北地区。

鎮守雲南的元朝梁王把匝剌瓦尔密,在元朝对中国的统治结束,撤到老家蒙古草原后依然繼續忠效之。1371年明太祖派湯和等人領兵攻灭據有四川的明玉珍的明夏政权,並且勸降梁王未果。1381年12月,明军的沐英和傅友德兵分二路攻入雲南,天元三年十二月二十二日(1382年1月6日),梁王把匝剌瓦尔密自杀,数月后,元朝云南大理总管段氏投降明军,明軍征服雲南地區,元朝对云南的统治结束。[2]。

1388年,蓝玉率领明军十五万發動第六次北伐,明军穿越过戈壁沙漠到达草原东部,天元十年四月十二日(1388年5月18日),蓝玉在捕鱼儿海(今贝尔湖)附近大败元军,俘虏北元后主次子地保奴及妃主五十余人、渠率三千、男女七万余,马驼牛羊十万。脱古思帖木儿和长子天保奴、知院捏怯来、丞相失烈门等数十骑逃走。

至此北元国力衰落。天元十年十月,脫古思帖木兒被也速迭尔(阿里不哥后裔)杀害,从1388年开始,蒙古不再使用年号,帝号、大元国号被废弃,北元时期结束。

在中外蒙古史学者的论著中,屡见“北元”一词,但是长期以来,对于这一史学概念的使用范畴却众说不一。争论的焦点就是“北元”是指1368-1388年这20年间的蒙古还是指1368-1635年这260多年间的蒙古。传统说法是1402年鬼力赤杀坤帖木儿汗,为北元时期结束的时间(《明史·鞑靼传》)。 关于这个问题,蔡美彪先生和曹永年先生曾作过深入探讨,认为“北元”应适用于脱古思帖木儿败亡而止,即1388年,此后大元国号已取消,仍称蒙古。

“北元”仅指大蒙古国的一个阶段,其根据是:脱古思帖木儿败亡后,蒙语文献中不再见大元国号的使用。思帖木儿败亡后,元朝传统的帝号、谥号、年号均不再见(也先汗与达延汗时期除外)。

1388年,北元皇帝、大汗脱古思帖木儿被叛臣也速迭儿弑杀。关于这位弑汗自立的也速迭儿,《华夷译语》中所载降明的蒙古知院捏怯来的奏报称是阿里不哥的子孙。这是一条很重要的史料。当年,蒙哥汗去世,镇守漠北的阿里不哥与控制中原的忽必烈发生汗位之争,结果阿里不哥失败,忽必烈做了蒙古大汗。随后迁都北京,仿汉族王朝模式定国号为“大元”,实行“汉法”,当上了元朝的皇帝。在这个时期,阿里不哥也被忽必烈杀害。忽必烈的所做所为无疑引起了阿里不哥子孙和漠北守旧的蒙古贵族的仇恨。在他们看来,大元是他们不共戴天的仇敌。有元一代,尽管忽必烈及其子孙在祖宗根本之地设立行省,实行宗王出镇制度,但这块龙飞之地却从未平静过。阿里不哥一系为首的反元斗争持续不断,这就是阿里不哥一派地方势力与元朝中央势不两立的明证。在这种心态驱使下,一旦元朝衰落,对蒙古草原的控制减弱,他们就会奋起反元。

在脱古思帖木儿败亡后的很长一段时间里,“部帅纷擎”,战乱频仍,与外界的联系基本中断。当时的明朝在捕鱼儿海战役胜利后,重点亦转向了对内部事务的处理。1398年明太祖去世,翌年太祖四子朱棣与建文帝同室操戈,是为“靖难之役”。这段时间《明实录》基本上没有关于蒙古的记载。直到朱棣“靖难”成功,当上了皇帝,才又重新开始了对北部边防的经略,致书蒙古大汗,要求“遣使往来通好,同为一家”,而此时已是1403年了。这时的蒙古大汗已是鬼力赤(他称汗在1402年左右)。当明朝方面获悉蒙古已去大元国号后,遂有明史「鬼力赤篡立,称可汗,去国号,遂称鞑靼」的误载。事实上,“去国号”的不是鬼力赤,而是也速迭儿。

在以后的蒙古历史上,大元国号仍出现,也先汗、达延汗时期即如此。但是,他们恢复大元国号的举动给汉蒙双方都带来了巨大的震动,这恰好反映出明代蒙古在大多数时期已取消了大元国号这一事实。

大元国号的废弃一定意义上意味着蒙古政权放弃了争夺中原的目标,转为立足于蒙古本身。

“北元”(1368年-1388年)仅代表一个时期的结束,其后进入《明史》所说的鞑靼时期(为明人所称,蒙方一直以蒙古自称)。但是从成吉思汗开始的“大蒙古国”政权仍然继续,鞑靼政权长期沿用元朝时代的汉制职官(如也先官职为太师淮王),至满都海夫人时才基本取消。“大蒙古国”政权延续至1635年察哈尔部为满洲的后金-清所灭亡。

故大蒙古国(1206年-1635年)依照中国名称的划分,可划为蒙古(1206年-1271年)、元朝(1271年-1368年)、北元(1368年-1388年)、鞑靼(1388年-1635年)。有时“元朝”可泛指从1206年至1368年这段时期。

\subsection{昭宗愛猷識理達臘\tiny(1370-1368)}

\subsubsection{生平}

元昭宗愛猷識理達臘,是北元的第二位君主,第十六位蒙古大汗,蒙古文称号必里克圖汗。他的在位時間是從1370年5月27日至1378年5月10日,在位8年,年號宣光。父為元順帝妥懽帖睦爾,母親是高麗貢女奇皇后。

明代王世貞《北虜始末志》稱愛猷識理達臘為“昭宗”。清代乾隆朝《蒙古世系譜》則稱愛猷識理達臘為“哲宗”。“哲宗”一說未被後人接受。

愛猷識理答臘生于元惠宗至元四年或五年的十二月二十四日。他的生母奇氏因为生育皇子,母凭子贵,至元六年(1340年)被元惠宗封为第二皇后,就是奇皇后。

至正十三年(1353年)六月,愛猷識理答臘被元惠宗(元顺帝)封为太子,他做太子之后,元朝内部党争日益激烈。愛猷識理答臘自己试图夺取帝位,提前登基,这样就造成了他和他父亲的关系紧张。至正二十四年(1364年),他的政敵將軍孛罗帖木儿帶兵闖入大都,愛猷識理答臘被迫流亡到王保保(扩廓帖木儿)的控制區太原,并以此为基地,召集各省军阀准备反攻孛罗。与此同时,元惠宗也对孛罗的专权产生不满,遂派人将其刺死,将人头送到太原,召回了愛猷識理達臘并与其和解。

至正二十八年(明朝洪武元年)闰七月二十八日(1368年9月10日),明朝军队逼近大都,元惠宗率太子愛猷識理達臘、后妃、臣僚等北走,前往上都,至正二十八年八月二日(1368年9月14日),明太祖的將軍徐達攻克大都。

至正二十八年八月四日(1368年9月16日),元惠宗和太子愛猷識理答臘等人到达上都。至正二十九年六月十三日(1369年7月16日),明军逼近上都,元惠宗和太子等人离开上都,当天到达应昌(今内蒙古克什克腾旗达里诺尔西南古城)。至正二十九年六月十七日(1369年7月20日),明军将领常遇春攻克上都。

至正三十年农历五月二日(1370年5月27日),元惠宗因痢疾去世于应昌,皇太子愛猷識理答臘在应昌繼承了皇位,並次年改元宣光。至正三十年农历五月十六日(1370年6月10日),明军将领李文忠攻克应昌,昭宗逃往和林,身边仅有一小股随从陪同,他的众多妃子以及儿子买的里八剌被明军俘虏,还有五万余元军投降明军。

宣光二年六月初三(1372年7月3日),明军将领冯胜大败元军,明朝从元朝手中取得甘肃地区。

北元在當時仍保持一定的勢力,在宣光二年(1372年)的戰事中,在王保保指挥下,元朝對於明朝贏得了一個局部勝利。

元昭宗於宣光八年(1378年5月10日)农历四月十三日逝世,在位8年,享年40岁。

元昭宗死後由弟北元后主脱古思帖木儿繼位,脱古思帖木儿年号为天元,又称为天元帝。

民国官定正史《新元史》柯劭忞的評價是:“昭宗以下,文献无徵。惟宣光八年之事,间存一二,故附载于本纪云。”

\subsubsection{宣光}

\begin{longtable}{|>{\centering\scriptsize}m{2em}|>{\centering\scriptsize}m{1.3em}|>{\centering}m{8.8em}|}
  % \caption{秦王政}\
  \toprule
  \SimHei \normalsize 年数 & \SimHei \scriptsize 公元 & \SimHei 大事件 \tabularnewline
  % \midrule
  \endfirsthead
  \toprule
  \SimHei \normalsize 年数 & \SimHei \scriptsize 公元 & \SimHei 大事件 \tabularnewline
  \midrule
  \endhead
  \midrule
  元年 & 1371 & \tabularnewline\hline
  二年 & 1372 & \tabularnewline\hline
  三年 & 1373 & \tabularnewline\hline
  四年 & 1374 & \tabularnewline\hline
  五年 & 1375 & \tabularnewline\hline
  六年 & 1376 & \tabularnewline\hline
  七年 & 1377 & \tabularnewline\hline
  八年 & 1378 & \tabularnewline\hline
  九年 & 1379 & \tabularnewline
  \bottomrule
\end{longtable}

\subsection{益宗脱古思帖木儿\tiny(1378-1388)}

\subsubsection{生平}

元天元帝脱古思帖木儿是北元第三位君主,第十七位蒙古大汗。史称北元后主,或以他的年号天元称为天元帝。或根据明朝史籍记载,他是愛猷識理達臘的弟弟。明代王世貞《北虜始末志》記載,脫古思帖木兒繼位前是益王。1378年5月13日—1388年11月1日在位,在位10年。

根据继承的次序推断,脱古思帖木儿应该就是蒙古语史料中的兀思哈勒可汗或烏薩哈爾汗。《蒙古源流》和《新元史》等史料记载他是必里克图可汗(愛猷識理達臘)的弟弟,但是这和《元史》中愛猷識理達臘弟弟早亡的记载不符。他的蒙古文称号是烏薩哈爾汗,无汉文廟號与諡號。

脱古思帖木儿的生年不详,父元順帝乌哈噶图汗。《蒙古源流》记载兀思哈勒可汗生于壬午年(1342年)。《黄史》记载他三十岁即位,照此推算生年约为1349年。但这些记载都和兀思哈勒是愛猷識理達臘(生于1339或1340年)之子的推断矛盾。

他於1378年5月即位,1379年农历六月改年号为天元。

1381年12月,明军进攻云南,天元三年十二月二十二日(1382年1月6日),镇守云南的元梁王把匝剌瓦尔密兵败自杀,天元四年闰二月二十三日(1382年4月7日),明将蓝玉、沐英攻克大理城,元朝大理总管段世投降明军,明朝平定云南,元朝在云南的统治结束。

从1254年元宪宗的皇弟忽必烈(后继位为元世祖)灭大理国,到1382年明军击败元军夺取云南,元朝统治云南地区长达128年。

1371年,元朝辽阳行省平章刘益降明,明朝占領辽宁南部。然而其餘东北地区仍由元朝太尉纳哈出控制,纳哈出屯兵二十万于金山(今辽宁省昌图金山堡以北辽河南岸一带),自持畜牧丰盛,与明军对峙了十几年,多次拒绝明朝的招抚。1387年冯胜、傅友德、蓝玉等人發動第五次北伐,目标是攻占纳哈出的金山。经过多次战争,1387年10月,纳哈出投降蓝玉,明朝占領东北地区。

天元十年四月十二日(1388年5月18日),明军将领蓝玉在捕鱼儿海(今贝尔湖)附近大败元军,俘虏脱古思帖木儿次子地保奴及妃主五十余人、渠率三千、男女七万余,马驼牛羊十万。脱古思帖木儿和长子天保奴、知院捏怯来、丞相失烈门等数十骑逃走。

1388年农历十月,脱古思帖木儿去世,次子恩克卓里克图继位。一说脱古思帖木儿遭阿里不哥後裔也速迭兒襲殺篡位。

\subsubsection{天元}

\begin{longtable}{|>{\centering\scriptsize}m{2em}|>{\centering\scriptsize}m{1.3em}|>{\centering}m{8.8em}|}
  % \caption{秦王政}\
  \toprule
  \SimHei \normalsize 年数 & \SimHei \scriptsize 公元 & \SimHei 大事件 \tabularnewline
  % \midrule
  \endfirsthead
  \toprule
  \SimHei \normalsize 年数 & \SimHei \scriptsize 公元 & \SimHei 大事件 \tabularnewline
  \midrule
  \endhead
  \midrule
  元年 & 1379 & \tabularnewline\hline
  二年 & 1380 & \tabularnewline\hline
  三年 & 1381 & \tabularnewline\hline
  四年 & 1382 & \tabularnewline\hline
  五年 & 1383 & \tabularnewline\hline
  六年 & 1384 & \tabularnewline\hline
  七年 & 1385 & \tabularnewline\hline
  八年 & 1386 & \tabularnewline\hline
  九年 & 1387 & \tabularnewline\hline
  十年 & 1388 & \tabularnewline
  \bottomrule
\end{longtable}


%%% Local Variables:
%%% mode: latex
%%% TeX-engine: xetex
%%% TeX-master: "../Main"
%%% End:



%%% Local Variables:
%%% mode: latex
%%% TeX-engine: xetex
%%% TeX-master: "../Main"
%%% End:
