%% -*- coding: utf-8 -*-
%% Time-stamp: <Chen Wang: 2021-11-01 17:05:28>

\section{成宗铁穆耳\tiny(1294-1307)}

\subsection{生平}

元成宗铁穆耳,是元朝第二位皇帝,蒙古帝国第六位大汗,1294年5月10日—1307年2月10日在位,在位14年。元世祖孙、太子真金第三子。清乾隆帝命改譯遼、金、元三史中的音譯專名,改譯特穆爾,今日學界已無人使用。

他去世后,谥号钦明广孝皇帝,庙号成宗,蒙古語号完澤篤可汗。

至元二十二年农历十二月十日(1286年1月5日),皇太子真金去世,元世祖欲立真金次子答剌麻八剌為皇太子,但1292年答剌麻八剌因病去世。至元三十年(1293年)真金三子铁穆耳受皇太子宝,总兵镇守漠北和林。至元三十一年农历正月二十二日(1294年2月18日),元世祖忽必烈去世,被封為晉王的真金長子甘麻剌決定要繼續鎮撫北方,铁穆耳得以在其母阔阔真與大臣伯顏等人的支持下,於至元三十一年农历四月十四日(1294年5月10日)在上都大安阁即位,是为元成宗。

铁穆耳即位後停止对外战争,罷征日本、安南,专力整顿国内军政,減免江南部分賦稅。並推行限制诸王势力、新编律令等措施,使社会矛盾暂时缓和。

在位期间基本维持守成局面,但滥增赏赐,入不敷出,国库资财匮乏,「向之所儲,散之殆盡」,中统钞迅速贬值。曾发兵征讨八百媳妇(在今泰国北部),引起云南、贵州地区动乱。晚年患病,委任皇后卜鲁罕和色目人大臣,朝政日渐衰败。

大德九年六月初五(1305年6月27日),元成宗冊立皇子德寿為皇太子,元成宗有数子,只有德寿皇太子为伯牙吾·卜鲁罕皇后所生。 同年十二月十八日(1306年1月3日),德寿因病去世。德寿去世後,成宗在生前未再立皇太子。

大德十一年农历正月初八日(1307年2月10日),成宗在大都玉德殿病逝,享年42岁 , 在位14年。

晚年患病,委任皇后卜鲁罕和色目人大臣,朝政日渐衰败。铁穆耳后继无人,埋下了元朝中期皇位争夺战的隐患。庙号成宗,谥号钦明广孝皇帝。蒙古汗号完泽笃可汗。

大德十一年九月十一日(1307年10月7日),元武宗为铁穆耳上谥号钦明广孝皇帝,庙号成宗,蒙古语称号完澤篤皇帝。

大德五年(1301年)秋,元军与窝阔台汗国的海都和察合台汗国的笃哇会战于金山附近的铁坚古山。元军先败海都。笃哇后至,两军再战。双方互有胜负,但都受到重创。海都、笃哇在会战中负伤,海都于1302年去世。

钦察汗国的东部藩属术赤长子斡儿答家族白帐汗封地原先与大汗的直辖地相连。窝阔台汗国的海都兴起后,隔断了元朝与术赤家族领地的直接联系。与海都接壤的白帐汗系宗王古亦鲁克为争夺汗位,投靠海都、笃哇。古亦鲁克的对手伯颜汗曾遣使元朝,要求双方联合作战。元朝的军队攻击海都,从谦州深入钦察汗国控制下的亦必儿·失必儿之地(今俄罗斯鄂毕河中游地区)。

大德六年(1302年),钦察汗国脱脱汗和白帐汗伯颜汗出兵2万,与元成宗的军队联合进攻笃哇和察八儿。此后钦察汗国承认元朝的宗主地位,长期与元朝维持友好关系。

1301年的铁坚古山之战对于元与西北宗藩的关系有决定性的影响,1302年海都去世,到了大德七年(1303年),笃哇扶立察八儿为窝阔台兀鲁思汗。笃哇暗中向元朝驻守在哈剌和林边境的安西王阿难答派出使臣,向元成宗表示臣服,请求朝廷罢兵。成宗同意约和。获得元廷支持后,笃哇与察八儿等聚会,到会诸王一致认识到,与朝廷进行长达数十年的战争是“自伤祖宗之业”。

大德七年(1303年)秋,笃哇以及海都之子察八儿约和使臣到达元廷。元廷与西北诸王达成和议,西北诸王承认元朝的宗主地位,设驿路,开关塞。自从1260年忽必烈与阿里不哥争位以来,元朝西北边境的战火终于基本平息,元朝的宗主地位得到四大汗国的正式承认。

接着,他们又联合遣使到伊儿汗国、钦察汗国王庭,大德八年(1304年)秋,伊儿汗完者都在木干草原会见钦察汗脱脱的使臣,西北四大汗国彼此之间的约和也至此完成,整个蒙古帝国境内再次迎来了和平。

明朝官修正史《元史》宋濂等的評價是:“成宗承天下混壹之后,垂拱而治,可谓善于守成者矣。惟其末年,连岁寝疾,凡国家政事,内则决于宫壸,外则委于宰臣;然其不致于废坠者,则以去世祖为未远,成宪具在故也。”

明朝官修正史《元史》宋濂等的評價是:“世称元之治以至元、大德为首。……。故终世祖之世,家给人足。……。大德之治,几于至元。”

清朝史家邵远平《元史类编》的評價是:“册曰:豢业以治,垂拱用成;中年奋武,启衅南征;末婴寝疾,壼柄廼萌;赖斯贤辅,镇侧弭倾。”

清朝史家毕沅《续资治通鉴》的評價是:“帝承世祖混一之后,善于守成;惟末年连岁寝疾,凡国家政事,内则决于宫壼,外则委于宰臣,幸去世祖未远,守其成宪,不至废坠。”

清朝史家曾廉《元书》的評價是:“论曰:成宗号为能守法度,而为病虐,前星弗耀,牝鸡司晨,而内难作矣。然非成宗之过也,成宗早任合剌合孙,资为羽翼,自古未有贤人在位而乱其国者也。股肱之寄,要在忠良,唐宗之言,信夫!”

民国史家屠寄《蒙兀儿史记》的評價是:“始汗为太孙时,好饮无节。忽必烈汗常戒之,不悛。以此受杖者三次,忽必烈汗至命医官监其饮食。有近侍司太孙节沐者,私置酒于盥器,代水以进,忽必烈汗闻之,大怒,谪戍其人远方,杀之于道。汗既登极,深以前事为非,力自节饮。其勇于改过如此。汗仁惠聪睿,承天下混一之后,信用老成,垂拱而治。一革至元中叶以来聚敛之政,冗设之官。约束诸王、妃、主、驸马扰民,禁滥请赏赐。性又谦冲,不好虚誉。群臣、皇后一再请上徽号,卒不允。可谓守成之令主矣。虽晚婴末疾,政出中宫,而举错无大过失。固由委任贤相之效,亦未始非内助之得人也。”

民国私修正史《新元史》柯劭忞的評價是:“成宗席前人之业,因其成法而损益之,析薪克荷,帝无使焉。晚年寝疾,不早决计计传位武宗,使易世之后,亲贵相夷,祸延母后。悲夫!以天子之尊,而不能保其妃匹,岂非后世之殷鉴哉。”

\subsection{元贞}

\begin{longtable}{|>{\centering\scriptsize}m{2em}|>{\centering\scriptsize}m{1.3em}|>{\centering}m{8.8em}|}
  % \caption{秦王政}\
  \toprule
  \SimHei \normalsize 年数 & \SimHei \scriptsize 公元 & \SimHei 大事件 \tabularnewline
  % \midrule
  \endfirsthead
  \toprule
  \SimHei \normalsize 年数 & \SimHei \scriptsize 公元 & \SimHei 大事件 \tabularnewline
  \midrule
  \endhead
  \midrule
  元年 & 1295 & \tabularnewline\hline
  二年 & 1296 & \tabularnewline\hline
  三年 & 1297 & \tabularnewline
  \bottomrule
\end{longtable}

\subsection{大德}

\begin{longtable}{|>{\centering\scriptsize}m{2em}|>{\centering\scriptsize}m{1.3em}|>{\centering}m{8.8em}|}
  % \caption{秦王政}\
  \toprule
  \SimHei \normalsize 年数 & \SimHei \scriptsize 公元 & \SimHei 大事件 \tabularnewline
  % \midrule
  \endfirsthead
  \toprule
  \SimHei \normalsize 年数 & \SimHei \scriptsize 公元 & \SimHei 大事件 \tabularnewline
  \midrule
  \endhead
  \midrule
  元年 & 1297 & \tabularnewline\hline
  二年 & 1298 & \tabularnewline\hline
  三年 & 1299 & \tabularnewline\hline
  四年 & 1300 & \tabularnewline\hline
  五年 & 1301 & \tabularnewline\hline
  六年 & 1302 & \tabularnewline\hline
  七年 & 1303 & \tabularnewline\hline
  八年 & 1304 & \tabularnewline\hline
  九年 & 1305 & \tabularnewline\hline
  十年 & 1306 & \tabularnewline\hline
  十一年 & 1307 & \tabularnewline
  \bottomrule
\end{longtable}


%%% Local Variables:
%%% mode: latex
%%% TeX-engine: xetex
%%% TeX-master: "../Main"
%%% End:
