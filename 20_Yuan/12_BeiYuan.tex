%% -*- coding: utf-8 -*-
%% Time-stamp: <Chen Wang: 2021-11-01 17:09:23>

\section{北元\tiny(1368-1388)}

\subsection{简介}

北元指明朝建立并遣徐达大军攻陷元朝首都大都(汗八里)后,退居蒙古高原的原元朝宗室的政權,因国号仍叫大元,以其地处塞北,故稱“北元”[1]。 北元始于元惠宗至正二十八年(1368年,明太祖洪武元年),终于脱古思帖木儿天元十年(明朝洪武二十一年,1388年),为蒙古(明人稱鞑靼)所代替。

元惠宗至正二十八年正月初四日(1368年1月23日)明太祖建立明朝,統一南方,令徐达北伐,徐达率领的军队逼近大都,闰七月二十八日(1368年9月10日),元惠宗夜半开大都的健德门北奔,率太子愛猷識理答臘、后妃、臣僚等撤离大都,八月初二日(1368年9月14日),明军从大都的齐化门攻城而入,元朝对中国的统治结束,回到本土蒙古草原。

元惠宗撤离大都后,继续使用“大元”国号,当时高丽人叫北元。當時政治形势是除了元惠宗據有漠南漠北的蒙古本土,關中還有元將擴廓帖木兒(王保保)駐守甘肅定西,此外元廷還領有东北地区與雲南行中书省地區。

明太祖為了驱逐位于蒙古的元廷势力,採取兵分二路,各個擊破的方式,此即第一次北伐。至正二十八年八月初四日(1368年9月16日),元惠宗到达上都。至正二十九年六月十三日(1369年7月16日),明军逼近上都,元惠宗撤离上都,当天到达应昌。六月十七日(1369年7月20日),明将常遇春攻克上都。

元惠宗在上都和应昌那里曾两次组织兵力试图收复大都,但都被明军击败。至正三十年(洪武三年)四月二十八日(1370年5月23日)元惠宗因痢疾在应昌去世,享年51岁。皇太子愛猷識理答臘在应昌繼承皇位,是为元昭宗,并于1371年改元宣光。至正三十年五月十六日(1370年6月10日),明将李文忠攻克应昌,元昭宗撤至哈拉和林,并坚持抵抗明军。

擴廓帖木兒仍然在漠北多地与明将徐达等人作戰。明太祖曾多次寫信詔降,但擴廓帖木兒從不理会,被朱元璋稱為「當世奇男子」。元昭宗宣光二年(1372年)正月,徐达从雁门出发,向哈拉和林进发。三月,明将蓝玉在土拉河大败扩廓帖木儿。五月,扩廓帖木儿在草原击败明将徐达的这支明军。自此之后,明军十几年不再进攻漠北,直到1388年,蓝玉才再次进攻漠北草原。

宣光二年六月初三(1372年7月3日),明将冯胜大败元军,明朝从元朝治下收取甘肃行中书省地区。

宣光八年(1378年)四月,元昭宗去世,继位的北元后主脫古思帖木兒在1379年六月改年号为天元,继续和明军对抗,屢次侵犯明境[2]。

1371年,元朝辽阳行省平章刘益降明,明朝控制今辽宁南部。然而之外的辽阳行省地区仍由元朝太尉纳哈出控制,纳哈出屯兵二十万于金山(今辽宁省昌图金山堡以北辽河南岸一带),自恃畜牧丰盛,与明军对峙了十几年,多次拒绝明太祖的招抚。1387年,冯胜、傅友德、蓝玉等人發動第五次北伐,目标是攻占纳哈出的金山。经过多次战争,1387年10月,纳哈出投降蓝玉,明朝控制原辽阳等处行中书省的东北地区。

鎮守雲南的元朝梁王把匝剌瓦尔密,在元朝对中国的统治结束,撤到老家蒙古草原后依然繼續忠效之。1371年明太祖派湯和等人領兵攻灭據有四川的明玉珍的明夏政权,並且勸降梁王未果。1381年12月,明军的沐英和傅友德兵分二路攻入雲南,天元三年十二月二十二日(1382年1月6日),梁王把匝剌瓦尔密自杀,数月后,元朝云南大理总管段氏投降明军,明軍征服雲南地區,元朝对云南的统治结束。[2]。

1388年,蓝玉率领明军十五万發動第六次北伐,明军穿越过戈壁沙漠到达草原东部,天元十年四月十二日(1388年5月18日),蓝玉在捕鱼儿海(今贝尔湖)附近大败元军,俘虏北元后主次子地保奴及妃主五十余人、渠率三千、男女七万余,马驼牛羊十万。脱古思帖木儿和长子天保奴、知院捏怯来、丞相失烈门等数十骑逃走。

至此北元国力衰落。天元十年十月,脫古思帖木兒被也速迭尔(阿里不哥后裔)杀害,从1388年开始,蒙古不再使用年号,帝号、大元国号被废弃,北元时期结束。

在中外蒙古史学者的论著中,屡见“北元”一词,但是长期以来,对于这一史学概念的使用范畴却众说不一。争论的焦点就是“北元”是指1368-1388年这20年间的蒙古还是指1368-1635年这260多年间的蒙古。传统说法是1402年鬼力赤杀坤帖木儿汗,为北元时期结束的时间(《明史·鞑靼传》)。 关于这个问题,蔡美彪先生和曹永年先生曾作过深入探讨,认为“北元”应适用于脱古思帖木儿败亡而止,即1388年,此后大元国号已取消,仍称蒙古。

“北元”仅指大蒙古国的一个阶段,其根据是:脱古思帖木儿败亡后,蒙语文献中不再见大元国号的使用。思帖木儿败亡后,元朝传统的帝号、谥号、年号均不再见(也先汗与达延汗时期除外)。

1388年,北元皇帝、大汗脱古思帖木儿被叛臣也速迭儿弑杀。关于这位弑汗自立的也速迭儿,《华夷译语》中所载降明的蒙古知院捏怯来的奏报称是阿里不哥的子孙。这是一条很重要的史料。当年,蒙哥汗去世,镇守漠北的阿里不哥与控制中原的忽必烈发生汗位之争,结果阿里不哥失败,忽必烈做了蒙古大汗。随后迁都北京,仿汉族王朝模式定国号为“大元”,实行“汉法”,当上了元朝的皇帝。在这个时期,阿里不哥也被忽必烈杀害。忽必烈的所做所为无疑引起了阿里不哥子孙和漠北守旧的蒙古贵族的仇恨。在他们看来,大元是他们不共戴天的仇敌。有元一代,尽管忽必烈及其子孙在祖宗根本之地设立行省,实行宗王出镇制度,但这块龙飞之地却从未平静过。阿里不哥一系为首的反元斗争持续不断,这就是阿里不哥一派地方势力与元朝中央势不两立的明证。在这种心态驱使下,一旦元朝衰落,对蒙古草原的控制减弱,他们就会奋起反元。

在脱古思帖木儿败亡后的很长一段时间里,“部帅纷擎”,战乱频仍,与外界的联系基本中断。当时的明朝在捕鱼儿海战役胜利后,重点亦转向了对内部事务的处理。1398年明太祖去世,翌年太祖四子朱棣与建文帝同室操戈,是为“靖难之役”。这段时间《明实录》基本上没有关于蒙古的记载。直到朱棣“靖难”成功,当上了皇帝,才又重新开始了对北部边防的经略,致书蒙古大汗,要求“遣使往来通好,同为一家”,而此时已是1403年了。这时的蒙古大汗已是鬼力赤(他称汗在1402年左右)。当明朝方面获悉蒙古已去大元国号后,遂有明史「鬼力赤篡立,称可汗,去国号,遂称鞑靼」的误载。事实上,“去国号”的不是鬼力赤,而是也速迭儿。

在以后的蒙古历史上,大元国号仍出现,也先汗、达延汗时期即如此。但是,他们恢复大元国号的举动给汉蒙双方都带来了巨大的震动,这恰好反映出明代蒙古在大多数时期已取消了大元国号这一事实。

大元国号的废弃一定意义上意味着蒙古政权放弃了争夺中原的目标,转为立足于蒙古本身。

“北元”(1368年-1388年)仅代表一个时期的结束,其后进入《明史》所说的鞑靼时期(为明人所称,蒙方一直以蒙古自称)。但是从成吉思汗开始的“大蒙古国”政权仍然继续,鞑靼政权长期沿用元朝时代的汉制职官(如也先官职为太师淮王),至满都海夫人时才基本取消。“大蒙古国”政权延续至1635年察哈尔部为满洲的后金-清所灭亡。

故大蒙古国(1206年-1635年)依照中国名称的划分,可划为蒙古(1206年-1271年)、元朝(1271年-1368年)、北元(1368年-1388年)、鞑靼(1388年-1635年)。有时“元朝”可泛指从1206年至1368年这段时期。

\subsection{昭宗愛猷識理達臘\tiny(1370-1368)}

\subsubsection{生平}

元昭宗愛猷識理達臘,是北元的第二位君主,第十六位蒙古大汗,蒙古文称号必里克圖汗。他的在位時間是從1370年5月27日至1378年5月10日,在位8年,年號宣光。父為元順帝妥懽帖睦爾,母親是高麗貢女奇皇后。

明代王世貞《北虜始末志》稱愛猷識理達臘為“昭宗”。清代乾隆朝《蒙古世系譜》則稱愛猷識理達臘為“哲宗”。“哲宗”一說未被後人接受。

愛猷識理答臘生于元惠宗至元四年或五年的十二月二十四日。他的生母奇氏因为生育皇子,母凭子贵,至元六年(1340年)被元惠宗封为第二皇后,就是奇皇后。

至正十三年(1353年)六月,愛猷識理答臘被元惠宗(元顺帝)封为太子,他做太子之后,元朝内部党争日益激烈。愛猷識理答臘自己试图夺取帝位,提前登基,这样就造成了他和他父亲的关系紧张。至正二十四年(1364年),他的政敵將軍孛罗帖木儿帶兵闖入大都,愛猷識理答臘被迫流亡到王保保(扩廓帖木儿)的控制區太原,并以此为基地,召集各省军阀准备反攻孛罗。与此同时,元惠宗也对孛罗的专权产生不满,遂派人将其刺死,将人头送到太原,召回了愛猷識理達臘并与其和解。

至正二十八年(明朝洪武元年)闰七月二十八日(1368年9月10日),明朝军队逼近大都,元惠宗率太子愛猷識理達臘、后妃、臣僚等北走,前往上都,至正二十八年八月二日(1368年9月14日),明太祖的將軍徐達攻克大都。

至正二十八年八月四日(1368年9月16日),元惠宗和太子愛猷識理答臘等人到达上都。至正二十九年六月十三日(1369年7月16日),明军逼近上都,元惠宗和太子等人离开上都,当天到达应昌(今内蒙古克什克腾旗达里诺尔西南古城)。至正二十九年六月十七日(1369年7月20日),明军将领常遇春攻克上都。

至正三十年农历五月二日(1370年5月27日),元惠宗因痢疾去世于应昌,皇太子愛猷識理答臘在应昌繼承了皇位,並次年改元宣光。至正三十年农历五月十六日(1370年6月10日),明军将领李文忠攻克应昌,昭宗逃往和林,身边仅有一小股随从陪同,他的众多妃子以及儿子买的里八剌被明军俘虏,还有五万余元军投降明军。

宣光二年六月初三(1372年7月3日),明军将领冯胜大败元军,明朝从元朝手中取得甘肃地区。

北元在當時仍保持一定的勢力,在宣光二年(1372年)的戰事中,在王保保指挥下,元朝對於明朝贏得了一個局部勝利。

元昭宗於宣光八年(1378年5月10日)农历四月十三日逝世,在位8年,享年40岁。

元昭宗死後由弟北元后主脱古思帖木儿繼位,脱古思帖木儿年号为天元,又称为天元帝。

民国官定正史《新元史》柯劭忞的評價是:“昭宗以下,文献无徵。惟宣光八年之事,间存一二,故附载于本纪云。”

\subsubsection{宣光}

\begin{longtable}{|>{\centering\scriptsize}m{2em}|>{\centering\scriptsize}m{1.3em}|>{\centering}m{8.8em}|}
  % \caption{秦王政}\
  \toprule
  \SimHei \normalsize 年数 & \SimHei \scriptsize 公元 & \SimHei 大事件 \tabularnewline
  % \midrule
  \endfirsthead
  \toprule
  \SimHei \normalsize 年数 & \SimHei \scriptsize 公元 & \SimHei 大事件 \tabularnewline
  \midrule
  \endhead
  \midrule
  元年 & 1371 & \tabularnewline\hline
  二年 & 1372 & \tabularnewline\hline
  三年 & 1373 & \tabularnewline\hline
  四年 & 1374 & \tabularnewline\hline
  五年 & 1375 & \tabularnewline\hline
  六年 & 1376 & \tabularnewline\hline
  七年 & 1377 & \tabularnewline\hline
  八年 & 1378 & \tabularnewline\hline
  九年 & 1379 & \tabularnewline
  \bottomrule
\end{longtable}

\subsection{益宗脱古思帖木儿\tiny(1378-1388)}

\subsubsection{生平}

元天元帝脱古思帖木儿是北元第三位君主,第十七位蒙古大汗。史称北元后主,或以他的年号天元称为天元帝。或根据明朝史籍记载,他是愛猷識理達臘的弟弟。明代王世貞《北虜始末志》記載,脫古思帖木兒繼位前是益王。1378年5月13日—1388年11月1日在位,在位10年。

根据继承的次序推断,脱古思帖木儿应该就是蒙古语史料中的兀思哈勒可汗或烏薩哈爾汗。《蒙古源流》和《新元史》等史料记载他是必里克图可汗(愛猷識理達臘)的弟弟,但是这和《元史》中愛猷識理達臘弟弟早亡的记载不符。他的蒙古文称号是烏薩哈爾汗,无汉文廟號与諡號。

脱古思帖木儿的生年不详,父元順帝乌哈噶图汗。《蒙古源流》记载兀思哈勒可汗生于壬午年(1342年)。《黄史》记载他三十岁即位,照此推算生年约为1349年。但这些记载都和兀思哈勒是愛猷識理達臘(生于1339或1340年)之子的推断矛盾。

他於1378年5月即位,1379年农历六月改年号为天元。

1381年12月,明军进攻云南,天元三年十二月二十二日(1382年1月6日),镇守云南的元梁王把匝剌瓦尔密兵败自杀,天元四年闰二月二十三日(1382年4月7日),明将蓝玉、沐英攻克大理城,元朝大理总管段世投降明军,明朝平定云南,元朝在云南的统治结束。

从1254年元宪宗的皇弟忽必烈(后继位为元世祖)灭大理国,到1382年明军击败元军夺取云南,元朝统治云南地区长达128年。

1371年,元朝辽阳行省平章刘益降明,明朝占領辽宁南部。然而其餘东北地区仍由元朝太尉纳哈出控制,纳哈出屯兵二十万于金山(今辽宁省昌图金山堡以北辽河南岸一带),自持畜牧丰盛,与明军对峙了十几年,多次拒绝明朝的招抚。1387年冯胜、傅友德、蓝玉等人發動第五次北伐,目标是攻占纳哈出的金山。经过多次战争,1387年10月,纳哈出投降蓝玉,明朝占領东北地区。

天元十年四月十二日(1388年5月18日),明军将领蓝玉在捕鱼儿海(今贝尔湖)附近大败元军,俘虏脱古思帖木儿次子地保奴及妃主五十余人、渠率三千、男女七万余,马驼牛羊十万。脱古思帖木儿和长子天保奴、知院捏怯来、丞相失烈门等数十骑逃走。

1388年农历十月,脱古思帖木儿去世,次子恩克卓里克图继位。一说脱古思帖木儿遭阿里不哥後裔也速迭兒襲殺篡位。

\subsubsection{天元}

\begin{longtable}{|>{\centering\scriptsize}m{2em}|>{\centering\scriptsize}m{1.3em}|>{\centering}m{8.8em}|}
  % \caption{秦王政}\
  \toprule
  \SimHei \normalsize 年数 & \SimHei \scriptsize 公元 & \SimHei 大事件 \tabularnewline
  % \midrule
  \endfirsthead
  \toprule
  \SimHei \normalsize 年数 & \SimHei \scriptsize 公元 & \SimHei 大事件 \tabularnewline
  \midrule
  \endhead
  \midrule
  元年 & 1379 & \tabularnewline\hline
  二年 & 1380 & \tabularnewline\hline
  三年 & 1381 & \tabularnewline\hline
  四年 & 1382 & \tabularnewline\hline
  五年 & 1383 & \tabularnewline\hline
  六年 & 1384 & \tabularnewline\hline
  七年 & 1385 & \tabularnewline\hline
  八年 & 1386 & \tabularnewline\hline
  九年 & 1387 & \tabularnewline\hline
  十年 & 1388 & \tabularnewline
  \bottomrule
\end{longtable}


%%% Local Variables:
%%% mode: latex
%%% TeX-engine: xetex
%%% TeX-master: "../Main"
%%% End:
