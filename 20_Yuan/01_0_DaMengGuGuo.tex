%% -*- coding: utf-8 -*-
%% Time-stamp: <Chen Wang: 2019-12-26 14:35:33>

\section{大蒙古国\tiny(1206-1260)}


\subsection{太祖生平}


成吉思汗(1162年5月31日-1227年8月25日),即元太祖,又稱成吉思皇帝、成吉思可汗。民國以前的漢文蒙古史料中除史集及新元史本紀外都以成吉思可汗及成吉思皇帝稱呼,成吉思汗稱呼為民國自西方翻譯而來。(《元朝秘史》記載為成吉思皇帝,《蒙古秘史》漢文版是現代翻譯。)為蒙古人,蒙古帝国奠基者、政治家、军事统帅,皇帝(大蒙古国可汗)。名铁木真,满清官译为特穆津。也有其他译法忒没真,意為“鐵匠”或“鐵一般堅強的人”、“鐵人”。奇渥温·孛儿只斤氏,尼倫蒙古乞顏部人。1206年春天—1227年8月25日在位,在位22年。1206年他登基时,诸王和群臣为他上蒙语尊号成吉思合罕。

至元二年(1265年)十月,元世祖忽必烈追尊成吉思汗廟號为太祖,至元三年(1266年)十月,太庙建成,制尊谥庙号,元世祖追尊成吉思汗諡號为聖武皇帝。至大二年十二月六日(1310年1月7日),元武宗海山加上尊谥法天啟運,庙号太祖。从此之后,成吉思汗的諡號变为法天啟運聖武皇帝。 

在他众子中,最为著名的四位分别是朮赤、察合台、窩闊台和拖雷。成吉思汗分封了朮赤和察合台为国主,欽定窩闊台为继承人。1227年成吉思汗去世后,拖雷自动退出继承人的選拔,担任监国两年后,1229年,拖雷和宗王们一起拥戴自己的三哥窝阔台登基。於1232年九月,在消灭金朝军队精锐主力后,拖雷去世,1234年2月9日,蒙古帝國灭金朝,為將來忽必烈揮軍南下攻打南宋打下基礎。

成吉思汗因其作戰的殘酷性而聞名,並被許多人視為種族滅絕的統治者。 然而,他也將絲綢之路置於一個有凝聚力的政治環境之下。 這使得東北亞,西亞和基督教歐洲之間的交流和貿易相對容易,擴大了這三個地區的文化視野。

金世宗大定二年(1162年),成吉思汗生于漠北草原。成吉思汗父親為其乞顏部酋長也速该。其名字「铁木真」之由來,乃是因為在他出生時,其父也速该正好俘虜到一位屬於塔塔儿部族,名為铁木真兀格的勇士。按當時蒙古人信仰,在抓到敵對部落勇士時,如正好有嬰兒出生,該勇士的勇氣會轉移到該嬰兒身上。成吉思汗「铁木真」之名遂因此而來。传说成吉思汗出生时,手中正拿著一血块,寓意天降將掌生殺大權。成吉思汗九歲时喪父,約於1170年。

在帶铁木真去弘吉剌部娶親後回来的路上,途經塔塔兒部,也速該遭到塔塔儿部殺害(怀疑被毒死),之後乞顏部族的泰赤乌氏首領塔里忽台因不滿也速該生前的所作所為,在也速該死後對鐵木真一家進行報復,命令部眾們遷至他地,孤立铁木真一家,但铁木真一家靠著毅力艱苦的活了下去。

就在铁木真漸漸出落成一個魁梧英俊的少年時,有三次劫難卻意外地降臨到他的頭上。

第一次是:脫離他們家族的泰赤乌氏擔心铁木真長大后報仇,於是就對铁木真家進行了突襲,並且計劃將被捕的铁木真處死。铁木真靠著父親的舊部鎖兒罕失剌以及其子沈白、赤老溫,其女合答安的協助脫逃,才因此逃过了一劫。身為長子的他,要攜母和弟妹們走到不兒罕山區,逃避泰赤乌氏追捕長達數年,自此形成他剛毅忍辱性格。

第二次是:在一個風雪交加的夜晚,一幫盜賊把他家僅有的幾匹馬搶走。在與盜賊的搏斗中,铁木真被盜賊射中喉嚨。危難之際,一個名叫博爾朮的青年拔刀相助,趕跑了盜賊,奪回了馬匹,铁木真得以幸免于難。

第三次是:成年後,铁木真與孛兒帖結婚時,三姓蔑兒乞部的首领脫黑脫阿,為報其弟赤列都的未婚妻訶額侖當年被铁木真的父親也速該所搶之仇,突袭了铁木真的營帳。在混戰中,铁木真逃進了不兒罕山(今肯特山),他的妻子和異母卻變成了脫黑脫阿的俘虜。

然而,三次劫難並未擊垮铁木真,反倒增強了他的復仇心理。他發誓要奪回家裡失去的一切。铁木真深知,要想立足,必須擁有實力。於是,他把妻子嫁妝中最珍貴的“黑貂皮”獻給了當時草原上實力最雄厚的克烈部落統領王汗。利用王汗的勢力,铁木真不僅收攏了他家離散的部族,還在王汗及幼時“安答”(義兄弟)札木合的幫助下,擊敗了三姓蔑兒乞部首领脫黑脫阿、忽都父子,救出了妻子孛兒帖和異母。

自此铁木真和札木合两人一起在部落共同生活。

由于铁木真提拔一些非贵族的人为将领,引發札木合不满,最终雙方决裂。1182年,铁木真被推举成为蒙古乞颜部的可汗。

統一蒙古各部:

1190年,在铁木真的領導下,乞顏迅速發展壯大,引起札达兰部首領札木合的不滿。札木合以其弟弟绐察兒被铁木真部下所殺為藉口,糾集了13個部落三萬余人,向铁木真發起進攻。铁木真也動員了部眾十三翼(即13個部落)迎擊,即著名的十三翼之戰。铁木真雖兵敗退至斡難河畔哲列捏狹地,但萬萬沒想到獲勝的札木合卻失去了人心。戰後,因為札木合把俘虜全部處死,將俘虜分七十大鍋煮殺,史稱「七十鍋慘案」。這種慘不忍睹的場面,連其部下也“多苦其主非法”,甚至擔心起自己的命運來。相反的,寬厚仁容的铁木真贏得了人心,那些擔心自己命運的札木合的部下紛紛倒向铁木真。此戰铁木真敗而得眾,使其軍力得以迅速恢復和壯大。铁木真的部眾一下子增加了許多。1196年,塔塔儿部首领蔑兀真笑里徒反抗金朝,金朝丞相完颜襄约克烈部王汗和铁木真联合出兵进攻塔塔儿,塔塔儿部大败,蔑兀真笑里徒被杀。铁木真遂被金朝封為“札兀惕忽里”,即部落官。

主兒乞部偷襲鐵木真的後方營地,被鐵木真剿滅。1201年,泰赤乌部、塔塔儿部、蔑兒乞部等11部推舉札达兰部的札木合為“古兒汗”,联兵攻打铁木真。铁木真联合王汗,於阔亦田之戰击败札木合等十二部联軍。聯軍潰散後,鐵木真追擊並剿滅了泰赤烏部。1202年,杀死塔塔儿部首领札鄰不合並屠殺残余的塔塔儿人,憶起少年時,父親也速该遭塔塔儿所害,临命终時的遗言,遂將凡是身高超過車輪高的塔塔儿士兵、男子通通都殺光,手法殘忍震驚蒙古諸部族。

1203年,王汗將铁木真收為义子,導致桑昆跟铁木真仇恨,札木合鼓動桑昆联合王汗夹击铁木真。合蘭真沙陀之戰爆发,这是铁木真经历的最为惨烈的一仗,只剩下19人隨他敗走班朱尼河[註 2],北上贝尔湖途中陸續追隨而來的部眾也只有2千6百人。同年秋天突袭王汗驻地,三天后完全消灭克烈部。王汗逃到鄂尔浑河畔之后被乃蛮人杀死。而其子桑昆則逃到庫車,被當地人杀死。

1204年,铁木真征伐蒙古草原西边的太阳汗,於納忽崖之戰击败乃蛮大軍,太阳汗当场被杀。秋,於合剌答勒忽札兀兒擊敗蔑兒乞部首領脫黑脫阿。1205年,鐵木真於額爾齊斯河擊敗蔑兒乞和乃蠻殘部聯軍,蔑兒乞首領脫黑脫阿陣亡,其子逃往康里、欽察,乃蛮部王子屈出律則逃亡西辽。1206年,札木合被叛变的将领送到铁木真之手,札木合请死,铁木真便殺了他。爾後,铁木真統一蒙古各部。

称成吉思汗:

1206年春天,蒙古贵族们在斡难河(今鄂嫩河)源头召开大会,諸王和群臣為鐵木真上尊號“成吉思汗”,正式登基成为大蒙古国皇帝 (蒙古帝国大汗),这是蒙古帝國的開始。成吉思汗遂颁布了《成吉思汗法典》,是世界上第一套應用範圍最廣泛的成文法典,建立了一套以贵族民主為基礎的蒙古贵族共和政體制度。

威脅西夏:

蒙古分别在1205年、1207年及1209年三次入侵西夏,逼使西夏臣服。1210年,西夏向蒙古称臣,並保证派军队支持蒙古以后的军事行动,此外,西夏皇帝夏襄宗献女求和,把察合公主嫁给了成吉思汗。

征服森林部落:1207年,成吉思汗命長子朮赤征森林部落。

降葛邏祿:1210年,成吉思汗命忽必來征葛邏祿,首領阿兒思蘭汗率部降。

消滅金朝未果:1210年,成吉思汗与金朝断绝了朝贡关系(约从1195年开始)。

1211年二月,成吉思汗亲率大军入侵金朝,在1211年的野狐嶺會戰击败四十萬金軍,并在次年和第三年陆续攻破金朝河北、河东北路和山东各州县,1214年三月,金宣宗遣使向蒙古求和,送上大量黄金、丝绸、马匹,并将金卫绍王的女儿岐国公主送给成吉思汗为妻,还有童男女五百陪嫁。成吉思汗从中都撤兵。

在金朝的东北地区,1212年,契丹人耶律留哥在辽东起兵反抗金朝,并宣布归附蒙古,耶律留哥和蒙古联军打败前来征讨的六十万金朝军队,1213年,耶律留哥自称辽王,1215年春,耶律留哥攻克金朝东京(今辽宁省辽阳),并占领金朝东北大部分地区。1215年十一月耶律留哥秘密与其子耶律薛阇带着厚礼前往漠北草原朝觐成吉思汗,成吉思汗极为高兴,赐给耶律留哥金虎符,仍旧封他为辽王。

为了远离蒙古的威胁,1214年6月27日,金宣宗离开中都,遷都汴京,得知金朝皇帝离开,成吉思汗下令入侵中都,蒙古軍在1215年5月31日占领中都,金朝在黃河以北之地陸續失守。

占领中都后,成吉思汗返回蒙古草原,1217年,成吉思汗任命大将木华黎为“太师国王”,让他负责继续入侵金朝,经过木华黎和他的儿子孛鲁十年的战争,到1227年成吉思汗去世前夕,蒙古军队基本占领金朝黄河以北的所有领土,金朝的领土仅局限于河南、陕西等地(当时的黄河取道江苏北部的淮河入海)。

1217年,成吉思汗派大将速不台追击脫黑脫阿諸子忽都、合剌、赤剌溫,次年於楚河地區剿滅蔑儿乞殘部。

正當金朝危在旦夕時,中亞的花剌子模王国惹怒蒙古,成吉思汗性急,转而報仇,暂时无暇顾及继续入侵金朝。

滅西辽及花剌子模:

早在1211年春天,畏兀儿亦都護巴而朮·阿而忒·的斤便归附蒙古。至1218年春季,成吉思汗派遣的蒙古使团到达花剌子模王国,强迫摩诃末苏丹签订与蒙古的条约。条约签订后,花剌子模城市讹答剌长官杀死路过此城的一支来自蒙古的由500人穆斯林组成的商队,夺取货物,仅有一人幸免于难逃回蒙古,成吉思汗派三个使臣前往花剌子模向摩诃末交涉,结果为首者被杀,另外二人被辱,成吉思汗更加愤怒,决定入侵花剌子模。

1218年,成吉思汗派大将哲别灭西辽,杀死西辽末代皇帝屈出律,平定西域。西征花剌子模进兵路上的障碍被扫除了。

1219年六月,成吉思汗親率蒙古主力(大约十万人)向西侵略,并在中途收编了5万突厥军,1220年底,一直被蒙古军队追击的花剌子模算端摩诃末病死在宽田吉思海(今里海)中的一个名為額別思寬島(或譯為阿必思昆島,已陸沉)的小岛上,并在临死前传位札兰丁。蒙古军先后取得河中地区和呼罗珊等地,1221年,蒙古军队消滅花剌子模王国,1221年十一月,成吉思汗率军追击札兰丁一直追到申河(今印度河)岸边,札兰丁大败,仅仅率少数人渡河逃走。

当初,成吉思汗命令速不台和哲别率领二万骑兵追击向西逃亡的摩诃末,摩诃末逃入里海后,他们率領蒙古軍继续向西进发,征服了太和岭(今高加索山)一带的很多国家,然后继续向西进入欽察草原擴張。1223年,者别與速不台於迦勒迦河之战(今乌克兰日丹诺夫市北)中击溃基辅罗斯诸国王公与钦察忽炭汗的联军,然后又攻入黑海北岸的克里木半岛。

1223年底,哲别與速不台率军东返,经过也的里河(今伏尔加河的突厥名,又译亦的勒),攻入此河中游的不里阿耳,遭遇顽强抵抗后,沿河南下,经由里海,咸海之北,与成吉思汗会师东归。在东返途中,哲别病逝。

攻西夏·去世:

成吉思汗回師後幹,再次入侵西夏。1227年8月25日(农历七月十二己丑日),在蒙古軍圍困西夏首都時,成吉思汗病逝於今宁夏南部六盘山(一说灵州),享壽六十五歲。其死因至今眾說紛紜,《元史》记载:“(元太祖二十二年)秋七月壬午,不豫。己丑,崩于萨里川啥老徒之行宫。”

成吉思汗去世前向儿子们交代了灭金的計劃:“假道宋境,包抄汴京。”后来窝阔台和拖雷灭金朝,采用的就是成吉思汗的这个战略。

此前西夏末代皇帝李睍已经答应投降,成吉思汗去世后,蒙古军密不发丧,李睍开城投降后,前去参见成吉思汗,诸将托言成吉思汗有疾,不让他参见。在成吉思汗去世三天后,1227年8月28日,诸将遵照成吉思汗遗命将西夏末帝杀死,西夏灭亡。蒙古军将领察罕努力使西夏首都中興府(今宁夏銀川)避免了屠城的命运,入城安抚城内军民,城内的军民得以保全。

《元朝祕史》记载成吉思坠马跌伤。而罗马天主教教廷使节约翰·普兰诺·加宾尼在《被我们称为鞑靼的蒙古人的历史》稱成吉思汗可能是被雷电击中身亡。

据《蒙古秘史》记载,成吉思汗的遗体被葬在不兒罕山接近斡難河源頭的地方,这是他生前指定的墓地。《元史》则记载他和历代元朝皇帝都葬于起辇谷。起辇谷的具体位置不详。在今日蒙古国肯特省的不儿罕山间有一片被称为“大禁忌”的土地,为达尔扈特人世代守护,相传是成吉思汗的墓地所在。在内蒙古自治区西部的鄂尔多斯高原上,有一座蒙古包式建筑宫殿,為成吉思汗的衣冠冢,经过多次迁移後直到1954年才由湟中县的塔尔寺迁回故地伊金霍洛旗,北距包头市185公里。每年的农历三月廿一、五月十五、八月十二和十月初三,为一年四次的大祭。

有傳言認為成吉思汗可能是遭三子窩闊台毒杀,原因是当时大汗打算传位给窝阔台,但突然改变注意,欲传位给四子拖雷,窝阔台为保汗位,所以毒杀其父。《成吉思汗与今日世界之形成》关于成吉思汗之死的論述与诸多的死亡故事相反,認為成吉思汗在游牧帐篷中去世,与他在游牧帐篷中的出生情形相似,这说明他在保存其本民族传统生活方式方面非常成功;然而,他保持其自身生活方式的过程中,却改变了人类社会。他在故土安葬,没有一座陵墓,没有一座寺庙,甚至没有一块用来标示其长眠之地的小墓碑。按照蒙古人的信仰,遗体应该在静穆中离去,并不需要纪念碑,因为灵魂已经不在那里了;灵魂继续活在精神之旗中。但他的精神之旗在1937年从蒙古中部的黑尚赫山下月亮河畔的寺庙里消失了。虔诚的喇嘛们护卫几个世纪的圣物,在由当时斯大林的追随者霍尔洛·乔巴山开展的遏制蒙古文化与宗教的运动中,永远的消失了。

尊谥庙号:

至元二年十月十四日(1265年11月23日),元世祖忽必烈追尊成吉思汗廟號为太祖。

至元三年十月十八日(1266年11月16日),太庙建成,制尊谥庙号,元世祖追尊成吉思汗諡號为聖武皇帝。

至大二年十二月六日(1310年1月7日),元武宗海山加上尊谥法天啟運,庙号太祖。从此之后,成吉思汗的諡號变为法天啟運聖武皇帝。《太祖皇帝加上尊谥册文》,内容如下:

维至大二年、岁次己酉、某月、某日,孝曾孙嗣皇帝臣某,谨再拜稽首言:

{\fzk 伏以恢皇纲,廓帝纮,建万世无疆之业;铺宏休,扬伟绩,遵累朝已定之规。式当继统之元,盍有称天之诔。孝弗忘于率履,制庸谨于加崇。钦惟太祖圣武皇帝陛下,渊量圣姿,睿谋雄断,沛仁恩而济屯厄,振羁策以驭豪英。惟解衣推食于初年,见君国子民之大略。玄符颛握,诸部悉平;黄钺载麾,百城随下。裔土兼收于夏孽,余波克殄于金源。荡荡乎无能名迹,远追于汤武;灏灏尔其为训道,允协于唐虞。根深峻岳而维者四焉,囊括殊封而统之一也。

肆予小子,承此丕基。两袛见于太宫,恒僾临于端扆。祚垂鸿兮锡裕,尚期昭报之申;牒镂玉以增辉,敢缓弥文之举。谨遣某官某,奉玉册玉宝,加上尊谥曰法天启运圣武皇帝,庙号太祖。

伏惟威灵昭假,景贶潜臻,阐绎吾元,与天并久。}

称号来源:“成吉思汗”是铁木真於1206年获得的称号。“成吉思”的含义不明确,一种说法由“成”派生而来。另一种说法是来自海洋一词,代表他像海洋一样伟大。

现存的13世纪和14世纪期的众多史料以及考古文物和摩崖石刻证明,1206年成吉思汗建立大蒙古国后,可能已经拥有皇帝和大汗的双重身份。生活在草原地区的蒙古等民族用蒙古语称呼铁木真为“大汗”、“成吉思汗”;生活在西北地区的突厥和其他民族用突厥语或其他语言称铁木真为“汗”或者“可汗”;生活在漠南汉地和东北地区的契丹人、女真人、党项人等民族,在13世纪前期的时候,历经辽朝、金朝、西夏等汉化政权,大部分已经汉化,通用汉语汉字,多称铁木真为“皇帝”;而生活在漠南汉地和东北地区的汉族人则直接使用“成吉思皇帝”一词。大量历史记载资料证明,1215年成吉思汗在攻取包括金中都在内的整个幽云十六州之后,其在长城以南汉地的统治保留了一些辽、金等朝的旧俗,并且在这些区域的官方文件,直接应用了“皇帝”的尊号来指代历任大蒙古国大汗。例如:

1219年农历五月,铁木真派刘仲禄邀请长春真人丘处机前往蒙古草原的诏书中,自称为“朕”,将自己建国登基称为“践祚”。

1220年农历二月丘处机抵达燕京后,得知铁木真在中亚进行西征花剌子模的战争,觉得自己年事已高,西行太远,希望约铁木真在燕京相见,于是在三月写了一份陈情表,在陈情表中,丘处机对铁木真的称呼是“皇帝”。同年收到丘处机的陈情表后,铁木真第二次派曷剌邀请丘处机前往中亚草原的诏书中,以“成吉思皇帝”和“朕”自称。

1221年南宋使者赵珙出使大蒙古国,回来后著有《蒙鞑备录》,书中对铁木真的称呼是“成吉思皇帝”。《蒙鞑备录》中提到,铁木真在位时期,朝廷使用的金牌,带两虎相向,曰虎头金牌,上书汉字:“天赐成吉思皇帝圣旨,当便宜行事”;其次为素金牌,书:“天赐成吉思皇帝圣旨疾”。1998年,一块“圣旨金牌”发现于河北廊坊,正面刻双钩汉字:“天赐成吉思皇帝圣旨疾。”和《蒙鞑备录》所记载的素金牌上汉文完全相同;背面牌心刻双钩契丹文,其汉语意思为:“速、走马,或快马”。这块圣旨牌的发现,说明铁木真在世时,其官方中文称谓作“成吉思皇帝”。

1227年全真教道士李志常写成的《长春真人西游记》,记录了丘处机从1219年受邀西行直至1227年去世的事迹,书中对铁木真的称呼是“成吉思皇帝”,将他下的命令称为“聖旨”;书中也提到了铁木真的侍臣刘仲禄前来邀请丘处机时携带了虎头金牌,金牌上面的文字是:“如朕亲行、便宜行事”,似乎在铁木真时期,凡是针对汉地的蒙古官方文件,均把成吉思汗翻译为“成吉思皇帝”。

1232年南宋使者彭大雅随奉使到大蒙古国,使者徐霆1235年—1236年随奉使到大蒙古国,二人返回南宋后,彭大雅撰写,并由徐霆作疏,合著《黑鞑事略》,书中对铁木真的称呼是“成吉思皇帝”。

2010年,刻有多位蒙古皇帝圣旨的全真教炼神庵摩崖石刻于山东徂徕山被发现,石刻一共四方,全部以汉语白话文写就,记述了大蒙古国皇室成员历代颁发给全真教掌教的官方文牒,其中有成吉思皇帝、合罕皇帝(窝阔台)、贵由皇帝,孛罗真皇后(窝阔台之妻)、唆鲁古唐妃,以及昔列门太子、和皙太子(均为窝阔台之子)等字样,其中记叙的“甲辰年十月初八日”表明该条圣旨是乃马真后称制的1244年颁发,落款“庚戌年十二月”则表明该石刻刻于海迷失后称制的1250年。圣旨石刻以汉语写就,包含不同时期、不同蒙古大汗的圣旨记录,为大蒙古国时期在汉地以中文“皇帝”作为蒙古大汗官方尊号的有力文物证据。

至元三年(1266年)忽必烈给日本的国书中,国书开头自称“大蒙古国皇帝”,在后面的内容中,自称为“朕”,此时距离他1271年正式立国号“大元”,还有五年时间。

然而大蒙古国时期的“皇帝”,和后来元朝的“皇帝”称号有本质的不同;前者是对“蒙古大汗”的汉式翻译,而后者则是按照中原文明的传统开立的新王朝君主,其“皇帝”称号上承秦汉隋唐宋等朝代。在1259年蒙哥汗去世后,忽必烈认为自己是大蒙古国汗位的正式继承者,自立为大汗,称“大蒙古国皇帝”,并于1263年将大蒙古国的历代大汗一并列入了自己新落成的太庙中;由于最终忽必烈没能获得蒙古各部贵族认可为新一任大汗,其于1271年按照中原文明的传统,建国号“大元”,因而元朝以后官方正史一直依照庙号将成吉思汗称作“太祖”。此时的大元皇帝,与之前大蒙古国时期被称作“皇帝”的蒙古大汗有本质区别——蒙古四大汗国的独立、大蒙古国的分裂,标志着忽必烈没能正式继承“大蒙古国”大汗之位;元朝,则是其新开创的王朝。元成宗时期,經過與蒙古四大汗國協商,元朝皇帝作为整个蒙古帝国共主的身份獲得四大汗國承認,作为中国历史上最高统治者称号的“皇帝”称号和作为“大蒙古国”最高统治者称号的“大汗”称号,同时集合在了後代的元朝皇帝的身上,如同中世纪歐洲由某王國國王或某公国大公出任神聖羅馬帝國皇帝。

整个元朝时期乃至后世王朝,官修历史一直保持了元朝的传统,将大蒙古国时期与元朝时期的统治一并而论,不作区分,统一将君主称为“皇帝”。《元史》中的<太祖本纪>記載鐵木真於1206年建大蒙古国时,称其“即皇帝位于斡难河之源,诸王群臣共尊其為成吉思皇帝”。元惠宗至正五年(1345年)十一月修成的法律《至正条格》中,称铁木真为“成吉思皇帝”,将他下的命令称为“聖旨”。明初官修《元史》,书中出现过“成吉思皇帝”一词多次,从未出现过“成吉思汗”一词。1252年成书的《元朝秘史》(《蒙古秘史》),蒙文音译作“成吉思合罕”,旁注释为“太祖皇帝”。直到近代中国,《新元史》中出现了“成吉思合罕”、“成吉思可汗”等词语,原因是《新元史》完成于民初(1920年),而《史集》、《世界征服者史》等西方的史书在清朝末年传入中国,《新元史》作者柯劭忞也深受其影响。

然而对于中国以外的地区,则仍将“大蒙古国”的君主称谓记作“大汗”。关于“成吉思汗”的记载见于拉施特《史集》、志费尼《世界征服者史》等中亚史籍,这两位作者均为蒙古帝国时期伊儿汗国(位于西亚)史学家,与元朝《元史》等史书基本处于同一时代,其书可为依据。四大汗国治下以的西亚国家以及欧洲公国仅知“成吉思汗”,同一时期的中国仅知“成吉思皇帝”,可见“成吉思皇帝”一词是针对古代汉字文化圈地区特设的翻译用词;由于西亚及欧洲文字皆为表音文字,其记载最能说明,大蒙古国君主的官方称谓仍为“大汗”,而非“皇帝”。

也速該,鐵木真父親,從蔑兒乞部手中奪走訶額侖,1170年被塔塔儿部首領札鄰不合毒害。也速該死後,族人離散,令鐵木真一家被逼過著流離生活。1266年元世祖忽必烈追尊也速该为皇帝,为也速该上庙号烈祖,諡號神元皇帝。

訶額侖,鐵木真母親,1206年尊为皇太后,1266年元世祖忽必烈上谥号宣懿皇后。

金末元初长春真人丘处机,拒绝金朝皇帝和南宋皇帝的邀请,答应前往草原和铁木真相见,抵达燕京后,得知铁木真已在中亚西征花剌子模,觉得自己年事已高,西行太远,希望约铁木真在燕京相见,于是在1220年三月写了一份陈情表,在陈情表中,对铁木真的评价是:“前者南京及宋国屡召不从,今者龙庭一呼即至,何也?伏闻皇帝天赐勇智,今古绝伦,道协威灵,华夷率服。是故便欲投山窜海,不忍相违;且当冒雪冲霜,图其一见。”(南京指的是当时的金朝首都开封,1214年,金朝从中都迁都到南京开封府)

南宋使者赵珙,1221年出使大蒙古国,在燕京(原为金中都,1215年被蒙古军队攻取,1217年木华黎改名燕京,今北京市)见到主持进攻金朝的太师国王木华黎,回来后著有《蒙鞑备录》,书中的評價是:“今成吉思皇帝者,……。其人英勇果决,有度量,能容众,敬天地,重信义。”

蒙古帝国伊儿汗国史学家志费尼《世界征服者史》的評價是:“倘若那善于运筹帷幄、料敌如神的亚历山大活在成吉思汗时代,他会在使计用策方面当成吉思汗的学生,而且,在攻略城池的种种妙策中,他会发现,最好莫如盲目地跟成吉思汗走。”

明朝官修正史《元史》宋濂等的評價是:“帝深沉有大略,用兵如神,故能灭国四十,遂平西夏。其奇勋伟迹甚众,惜乎当时史官不备,或多失于纪载云。”

明朝官修皇帝实录《明太祖实录》记载,洪武二十二年(1389年)五月,明太祖朱元璋给北元阿札失里大王的信中,对成吉思汗、元太宗窝阔台、元定宗贵由、元宪宗蒙哥、元世祖忽必烈这五位在一统天下中均作出重要贡献的帝王的综合评价如下:“覆载之间,生民之众,天必择君以主之,天之道福善祸淫,始古至今,无有僣差。人君能上奉天道,勤政不贰,则福祚无期,若怠政殃民,天必改择焉。昔者,二百年前,华夷异统,势分南北,奈何宋君失政,金主不仁,天择元君起于草野,戡定朔方,抚有中夏,混一南北,逮其后嗣不君,于是天更元运,以付于朕。”

明朝官修皇帝实录《明太祖实录》记载,洪武二十二年(1389年)十二月,明太祖朱元璋给哈密国兀纳失里大王的信中,对成吉思汗和元世祖忽必烈的评价如下:“昔中国大宋皇帝主天下三百一十余年,后其子孙不能敬天爱民,故天生元朝太祖皇帝,起于漠北,凡达达、回回、诸番君长尽平定之,太祖之孙以仁德著称,为世祖皇帝,混一天下,九夷八蛮、海外番国归于一统,百年之间,其恩德孰不思慕,号令孰不畏惧,是时四方无虞,民康物阜。”

清朝史学家邵远平《元史类编》的評價是:“册曰:天造鸿图,艰难开创;浑河启源,角端呈像;芟夏蹙金,电扫莫抗;栉沭廿年,驱指四将;止杀一言,皇猷弥广。”

清朝史学家毕沅《续资治通鉴》的評價是:“太祖深沉有大略,用兵如神,故能灭国四十,遂平西夏。”

清朝史学家魏源《元史新编》的評價是:“帝深沉有大略,用兵如神,故能灭国四十,遂平夏克金,有中原三分之二。使舍其攻西域之力,以从事汴京,则不俟太宗而大业定矣。然兵行西海、北海万里之外,昆仑、月竁重译不至之区,皆马足之所躏,如出入户闼焉。天地解而雷雨作,鹍鹏运而溟海立,固鸿荒未辟之乾坤矣。”

清朝史学家曾廉《元书》的評價是:“论曰:太祖崛起三河之源,奄有汉代匈奴故地,而兼西域城郭诸国,朔方之雄盛未有及之者也。遗谋灭金,竟如其策,金亡而宋亦下矣,此非其略有大过人者乎?又明于求才,近则辽金,远则西域,仇敌之裔,俘囚之虏,皆收为爪牙腹心,厥功烂焉,何其宏也,立贤无方,太祖有之矣。羽翼盛,斯其负风也大,子孙蒙业,遂一宇宙,不亦宜乎。”

民国史学家屠寄《蒙兀儿史记》的評價是:“论曰:旧史称成吉思汗深沉有大度,用兵如神,故能灭国四十,遂平西夏,信然。独惜军锋所至,屠刿生民如鹿豕,何其暴也。及至五星聚见东南,末命谆谆,始戒杀掠,岂所谓人之将死,其言善欤!蒙兀一代,并漠北四君数之,卜世十四,卜年蕲百六十,唐宋以降,享国历数,为由蹙于是者。于戏,可以观天道矣!”

民国官修正史《新元史》柯劭忞的評價是:“天下之势,由分而合,虽阻山限海、异类殊俗,终门于统一。太祖龙兴朔漠,践夏戡金,荡平西域,师行万里,犹出入户闼之内,三代而后未尝有也。天将大九州而一中外,使太祖抉其藩、躏其途,以穷其兵力之所及,虽谓华夷之大同,肇于博尔济锦氏,可也。” 

民国史学家张振佩《成吉思汗评传》(1943年版)绪言部分的評價是:“成吉思汗之功业扩大人类之世界观——促进中西文化之交流——创造民族新文化。”

1939年,处于抗战时期的中国共产党对成吉思汗做出了高度评价。6月21日,成吉思汗灵柩西迁途中到达延安时,中共中央和各界人士二万余人夹道迎灵,并在延安十里铺搭设灵堂,举行了盛大的祭祀活动。在此次祭祀仪式上,中共中央将成吉思汗正式尊称为“世界巨人”、“世界英杰”,并首次提出“继承成吉思汗精神坚持抗战到底”的口号。延安十里铺灵堂两侧悬挂一幅对联,灵堂正上方有一横联,内容如下:

横联:世界巨人
上联:蒙漢兩大民族更親密地團結起來

下联:繼承成吉思汗精神堅持抗戰到底

灵堂前面搭建一座牌楼,悬挂“恭迎成吉思汗靈柩”匾额。代表们将灵柩迎入灵堂后,举行祭典。中共中央、毛澤東、周恩來、朱德等敬献了花圈。由陕甘宁边区政府秘书长曹力如代表党政军民学各界恭读祭文:维中华民国二十八年六月二十一日,中国共产党中央委员会代表谢觉哉、国民革命军第八路军代表滕代远、陕甘宁边区政府代表高自立,率延安党政军民学各界,谨以清酌庶馐之奠,致祭于圣武皇帝成吉思汗之灵曰:

日寇逞兵,为祸中国,不分蒙汉,如出一辙。
嚣然反共,实则残良,汉蒙各族,皆眼中钉。
乃有奸人,蠢然附敌,汉有汉奸,蒙有蒙贼。
驱除败类,整我阵容,抗战到底,大义是宏。
顽固分子,准投降派,摩擦愈凶,敌愈称快。
巩固团结,唯一方针,有破坏者,群起而攻。
元朝太祖,世界英杰,今日郊迎,河山聚色。
而今而后,五族一家,真正团结,唯敌是挝。
平等自由,共同目的,道路虽艰,在乎努力。
艰苦奋斗,共产党人,煌煌纲领,救国救民。
祖武克绳,当仁不让,太旱盼霓,国人之望。
清凉岳岳,延水汤汤,此物此志,寄在酒浆。
尚飨!

1940年3月31日,中国共产党在延安成立了“蒙古文化促进会”,4月,在延安建立了“成吉思汗纪念堂”和“蒙古文化陈列馆”,敬立成吉思汗半身塑像,并由毛澤東题写了“成吉思汗紀念堂”七个大字。在这里每年农历三月二十一日,也就是成吉思汗春季查干苏鲁克大祭之日,延安各界举行盛大的祭奠仪式,以蒙汉两种语言诵读成吉思汗祭文。1942年5月5日,蒙古文化促进会还编辑出版了《延安各界纪念成吉思汗专刊》。毛澤東和朱德分别为专刊題詞,内容如下:毛澤東題詞:團結抗戰;朱德題詞:中華民族英雄。

毛泽东在1964年3月24日,在一次听取汇报时的插话中对成吉思汗、汉高祖刘邦、明太祖朱元璋的治国能力评价如下:“可不要看不起老粗。”“知识分子是比较最没有知识的,历史上当皇帝的,有许多是知识分子,是没有出息的:隋炀帝,就是一个会做文章、诗词的人;陈后主、李后主,都是能诗善赋的人;宋徽宗,既能写诗又能绘画。一些老粗能办大事:成吉思汗,是不识字的老粗;刘邦,也不认识几个字,是老粗;朱元璋也不识字,是个放牛的。”(毛泽东举例只是为了强调“一些老粗能办大事”,并不是说成吉思汗和刘邦真的不识字,也不是说刘邦只认识几个字。事实上,成吉思汗,刘邦,朱元璋三人原本可能僅能粗通文字,但當他們身为帝王時,他们的文化水平已經达到批阅奏折和签署命令的程度,甚至能為唱和文章。刘邦和朱元璋的文化水平不必细谈,相关史书记载很多,至于成吉思汗,元初名臣耶律楚材在《玄风庆会录》一书中提到成吉思汗是可以亲自阅览文件的。)

1941年十一月三日国民政府正式宣布对日本及德国、意大利宣战前夕,蒋介石赶赴甘肃省榆中县兴隆山,对成吉思汗灵寝举行了大祭。蒙藏委员会委员长吴中信代表国民政府恭读祭文:維中華民國三十年十一月三日國防最高委員會委員長蔣中正,特派蒙藏委員會委員長吳中信,以馬羊帛酒香花之儀,致祭於成吉思汗之靈而昭告以文曰:

繄我中華,五族為家,自昔漢唐盛世,文德所被,蓋已統乎西域極於流沙,洎夫大汗崛起,武功熠耀,馬嘶弓振,風撥雲拏,縱橫帶甲,馳驟歐亞,奄有萬邦,混一書車,其天縱神武之所肇造,雖曆稽往古九有之英傑而莫之能加,比者蝦夷小醜,虺毒包藏,興戎問鼎,豕突倡狂,致我先哲之靈寢乍寧處而不遑,中正忝領全民,撻伐斯張,一心一德,慷慨騰驤,前僕後興,誓殄強梁,請聽億萬鐵馬金戈之凱奏,終將相複於伊金霍洛之故鄉,緬威靈之赫赫兮天蒼蒼,撫大漠之蕩蕩兮風泱泱,修精誠以感通兮興隆在望,萬馬胙而陳體漿兮神其來嘗。尚饗。

1957年三月十二日,蒋介石在在主持陸軍指揮參謀學校正×期開學典禮講——《軍事哲學對於一般將領的重要性》中,评价成吉思汗:“我在此還要舉出我們中國歷史中兩位最有名的勇將來作一對照,以供我們今日軍人的抉擇。這兩位勇將中的第一位,就是漢楚時代的項羽。第二位就是縱橫歐亞的成吉思汗。這二位英勇無比的名將,其平生戰績乃是眾所周知,無待詳述,可是其結果則完全不同。茲據其二人所製的歌詞的氣概與精神,就可想見膽力的強弱與事業的成敗了。當成吉思汗西征時的歌詞是:「上天與下地,俯伏嘯以齊,何物蠢小醜,而敢當馬蹄」。而項羽最後失敗時的歌詞則是:「力拔山兮氣蓋世,時不濟兮騅不逝,騅不逝兮可奈何,虞兮虞兮奈若何?」後來還有許多人評判項羽這首歌詞是悲歌慷慨,不失為英雄氣概;我以為項羽的歌詞充滿了「恐懼」「憤怒」「疑惑」的氣氛,毫無英勇鎮定與自信的心理,更沒有如克勞塞維茨所說:「在絕望中之奮鬥」的軍人精神。所以到了最後他只有在烏江自刎了事。我以為這種卑怯自殺,而不能抱定榮譽戰死的軍人,只可說是一個最無志氣的懦夫,那能配稱為勇將!故無論他過去有如何勇敢的史蹟,我們不僅不屑敬仰他,而且應在棄絕不齒之列。至於成吉思汗的這首歌詞,我認為是充滿了他自信、勇敢與鎮定的心理,誠不失為一首英勇壯烈的歌詞,正與項羽的歌詞語意完全相反,所以他成功亦自不同。因為他既有這樣一個戰勝一切的信心,自然不會再有恐懼憤怒與疑惑的心理了。所以成吉思汗,實為我們中國軍人所應該效法與崇敬的第一等模範英雄。”
中華民國總統馬英九在2009年4月16日(农历三月二十一日)“二00九年中枢致祭成陵大典”中,特派蒙藏委员会委员长高思博主祭成吉思汗。祭坛上陈放有成吉思汗的画像,摆放有鲜花、水果和糕点,点燃供烛。仪式遵循古礼。台北市国乐团演奏乐曲《万寿无疆》。身穿长袍马褂的高思博,依序向成吉思汗像献香、献花、献爵(献酒)、献帛(献哈达)。司仪宣读祭文:“马英九特派蒙藏委员会委员长高思博敬以香花清酌之仪致祭于成吉思汗之灵曰:‘维汗休烈,雄才大略。天挺英明,龙兴溯漠。……礼仪孔修,有芘其芳。神之格思,德音不忘。’”

馬英九在2010年5月4日(农历三月二十一日)蒙藏委員會上午舉辦的“99年中樞祭成吉思汗大祭”典禮中,指派蒙藏委員會委員長高思博以香花清酌儀式祭拜成吉思汗。典禮安排向成吉思汗像獻花、獻香、獻爵(獻酒)、獻帛(獻哈达),並宣讀“中華民國總統祭文”,相關司祭者皆穿著蒙古傳統服飾,儀式遵循古禮,場面莊嚴隆重,馬英九在祭文中,肯定成吉思汗“雄才大略,天挺英明,拓土開疆,威震萬國。”

馬克思在《馬克思印度史編年稿》中谈到成吉思汗时曾说:“成吉思汗戎馬倥傯,征戰終生,統一了蒙古,為中國統一而戰,祖孫三代鏖戰六七十年,其後征服民族多至720部。”[來源可靠?]

瑞典學者多桑在其《蒙古史》中對成吉思汗的一生總結分析,多桑認為為成吉思汗之成功乃由於其具有極強的貪慾以及非常之野心。多桑稱他“狂傲”地妄想征服世界,死前還囑咐其子孫完成他的事業。

英国学者莱穆在《全人类帝王成吉思汗》一书中说:“成吉思汗是比欧洲历史舞台上所有的优秀人物更大规模的征服者。他不是通常尺度能够衡量的人物。他所统率的军队的足迹不能以里数来计量,实际上只能以经纬度来衡量。”

印度总理尼赫鲁在《怎样对待世界历史》一书中说:“蒙古人在战场上取得如此伟大的胜利,这并不靠兵马之众多,而靠的是严谨的纪律、制度和可行的组织。也可以说,那些辉煌的成就来自于成吉思汗的指挥艺术。”

「卡內基全球生態研究部」:「歷史上『最環保的侵略者』。因為殺人無數,讓大片耕地恢復成為森林,讓大氣中的碳大幅減量達7億噸!」

美国西維吉尼亞大學的研究人员指出成吉思汗的成功恰逢当时1000年来最温和、最潮湿的天气,之前的1180-1190年间,蒙古曾经历严重干旱,之后的温和湿润气候有助于青草的繁茂生长,为以骑兵为主的蒙古大军的战马提供了丰富的饲料。

1999年12月的美国A+E电视网评选出过去千年影响最深远的100大人物,成吉思汗被列为第22位(在亚洲人中仅次于第17位的甘地)。

\subsection{睿宗生平}

拖雷(1191年-1232年)又译图垒,是元太祖成吉思汗的幼子,排行第四。拖雷和正妻唆鲁禾帖尼生有四子:蒙哥、忽必烈、旭烈兀、阿里不哥。据《元史》,拖雷一共有子十一人。1227年8月25日至1229年9月13日担任大蒙古国(蒙古帝国)监国,历时二年。

1213年,成吉思汗分兵伐金,拖雷从其父率领中路军,攻克宣德府,再攻德兴府。拖雷与驸马赤驹先登,拔其城。即而挥师南下,拨涿州、易州,残破河北、山东诸郡县。1219年,从成吉思汗西征,攻陷布哈拉、撒馬爾罕。1221年,分领一军进入呼罗珊境,陷马鲁、尼沙不儿,渡搠搠阑河,降也里。遂与成吉思汗合兵攻塔里寒寨。

按照蒙古习俗,幼子继承父业,而年长诸子则分析外出、自谋生计。故成吉思汗生前分封诸子,拖雷留置父母身边,继承父亲所有在斡难和怯绿连的斡耳朵、牧地及军队。成吉思汗留下的军队共有12.9万人。其中10.1万的精銳俱由拖雷继承。

1221年拖雷屠杀木鹿(今梅尔夫)城中居民,超过100萬人。除去四百个工匠之外,其余人口被屠杀殆尽,城墙被毁。从此木鹿结束了繁荣的历史。

1227年8月25日成吉思汗病逝後,由拖雷監国,称也可那颜。直至两年后在选举大汗的忽里臺時,拖雷和察合台等宗王们在1229年9月13日一起推举元太宗窩闊臺即大汗位。

1231年,与窩闊臺分道伐金,拖雷总右军自陝西鳳翔渡渭水,过宝鸡,入大散关。11月,蒙古军假道南宋境,沿汉水而下,经兴元(今陕西汉中)、洋州(今陕西洋县)在均州(今湖北均县西北)、光化(今湖北光化北)一带,渡汉水,迂迴北上入金境。1232年初与金军在均州(今河南禹县)遭遇。拖雷乘雪夜天寒(有一康里人作法)大败金将完颜合达、移剌蒲阿、完颜陳和尚於三峰山,尽歼金军精锐。此役毕,拖雷与自白坡渡河南下的窩闊臺军会合。

1232年农历九月,拖雷在北返蒙古草原途中逝世。據蒙古秘史,窩闊臺在一場重要戰爭中得了重疾,為了治愈窩闊臺,拖雷決定犧牲自己。巫師認為,窩闊臺所得的惡疾的病源是由中土的水和土之靈而來。水土之靈不滿蒙古人把中土臣民趕出中土,和不滿蒙古人令中土滿目瘡痍。若以中土的土地,動物和人作祭品,只會令窩闊臺的病情更加惡化,但若是他們願意犧牲家庭成員,窩闊臺便能好過來,於是拖雷主動飮了被詛咒的飲料後死亡。另一說法,是拖雷可能因酗酒過量而死。

1251年7月1日,元宪宗蒙哥即位,大蒙古国(蒙古帝国)皇位从窝阔台家族转入拖雷家族,元宪宗追尊父亲拖雷为皇帝,为拖雷追上尊谥庙号,庙号睿宗,谥号英武皇帝。

至元三年十月十八日(1266年11月16日),太庙建成,制尊谥庙号,元世祖忽必烈将父亲拖雷的谥号由英武皇帝改谥为景襄皇帝。 

至大二年十二月六日(1310年1月7日),元武宗海山为拖雷加上尊谥仁圣,从此之后,拖雷的谥号变为仁圣景襄皇帝。《睿宗皇帝加上尊谥册文》,内容如下:“伏以诣泰坛而请命,有称天以诔之文;荐清庙而致严,盖若昔相承之典。刚辰爰卜,遗美载扬。钦惟睿宗景襄皇帝孝友温恭,聪明浚哲。属我家肈造于朔土,佐圣祖遄征于四方。逮天讨之奉行,致皇威之远畅。金源假两河之息,天水渝通好之盟,遂移秦陇之师,爰有褒斜之举。既平南郑,顺流而东,再涉襄江,自上而下,乃眷三峰之捷,实开万世之基。唇既亡而齿亦寒,虢可伐而虞不腊。适英文之违豫,图中夏之底宁。毋作神羞,请以身代。爰俟金縢之启,已知宝祚之归。迪我后人,绍兹明命。徽称显号,虽已拟诸形容;玉检金泥,尚未遑于润色。奉玉册玉宝,加上尊谥曰仁圣景襄皇帝,庙号睿宗。伏惟端临扆座,诞受鸿名。亿万斯年,永锡繁祉。”

据《元史》,拖雷有子十一人:长子蒙哥,次子忽睹都、四子忽必烈、六子旭烈兀、七子阿里不哥、八子拨绰(不者克)、九子末哥、十子岁哥都、十一子雪别台,第三子和第五子失其名。

拖雷和正妻唆鲁禾帖尼所生的四子皆有所成,元宪宗蒙哥和元世祖忽必烈相继做过大元(大蒙古国)的帝王,蒙元皇帝由拖雷一系繼承。旭烈兀在西亚开创了伊儿汗国,1259年蒙哥去世后,阿里不哥在1260年在蒙古本土的庫力臺大會被部分王公推举即位,并和忽必烈争位达四年之久。

拖雷可考的女儿有二:赵国公主独木干,下嫁汪古部聂古得;鲁国公主也速不花,下嫁弘吉剌部斡陈。

民国官修正史《新元史》柯劭忞的评价是:“周公金縢之事,三代以后能继之者,惟拖雷一人。太宗愈,而拖雷竟卒,或为事之适然,然孝弟之至,可以感动鬼神无疑也。世俗浅薄者,乃疑其诬妄,过矣!”


\subsection{太宗生平}

窝阔台汗(1186年11月7日-1241年12月11日),又作斡歌歹、和歌台、倭闊岱等,孛儿只斤氏,成吉思汗第三子,大蒙古国大汗。他是蒙古帝国第二位大汗,1229年9月13日—1241年12月11日在位,在位12年零3个月。他登基时接受大汗的称号,和诸汗相区别。

至元三年(1266年)十月,太庙成,元廷追尊庙号太宗,谥英文皇帝。

1229年9月13日(农历八月二十四日),窝阔台在库里尔台大会中被察合台、拖雷、铁木哥斡赤斤等宗王和大臣推举为大蒙古国大汗,管理整個蒙古帝国,有史料载诸宗王和百官为窝阔台上尊号曰木亦坚合罕(合罕为大汗的别译)。

他继承父親的遺志擴張領土,主要是繼續西征和南下中原。他在位期間成功完全征服中亞和華北。内政方面,以契丹人耶律楚材為相管理华北和中原地区,在这些地区稍微改变了战后屠城作風,保存不少金朝遺民和政治制度;同時又依耶律楚材建議,提拔漢人為官,整頓內治,安定了蒙古在華北地區的统治,使华北地区经济在戰後得到一定程度的恢复性發展,為日後忽必烈称帝滅南宋打下基礎。

灭金取中原:1229年登基的时候,大蒙古国在东亚部分的东南部大体以黄河为界,金朝领土基本上只剩下黄河以南的河南、陕西等地(当时的黄河取道江苏北部的淮河入海)。

1231年,窝阔台与其四弟拖雷分道进攻金朝,1232年初,拖雷率蒙古军在河南三峰山战胜金军,尽歼金军精锐。其后,拖雷与自白坡渡河南下的窝阔台军会合,一同北返蒙古草原,1232年农历九月,拖雷於北返途中病死之後,托雷四子忽必烈继承了他在华北地区的势力。

1232年春,蒙古军队继续南下,抵达金朝首都燕京(今北京市)附近,因此周围州县难民纷纷逃入汴京(今河南开封市),城中人口激增,而入夏后瘟疫流行,死者达九十餘万人。1232年秋,蒙古派使者入城要求金朝投降,被金朝将士所杀,蒙古军于是不再议和,击溃金朝援军,围困汴京城。

1233年2月6日(农历十二月二十六日),金哀宗和后妃们分别离开汴京,一路向南。1233年2月26日(农历正月十六日),金哀宗抵达归德(今河南商丘市),随后又出走;8月3日(农历六月二十六日),金哀宗逃到蔡州(今河南汝南县),在此地稳定下来。

1233年3月5日(农历正月二十三日),金朝汴京西面元帅崔立率军队杀死汴京的留守將領完颜奴申和完颜习捏阿不,控制全城,派使者向蒙古军统帅速不台投降。

1233年3月10日(农历正月二十八日),速不台向汴京进兵。速不台得知崔立同意投降后,因为之前进攻汴京时金人抗拒持久導致军队死伤甚多,便向窝阔台奏报建议軍隊入城后屠城泄愤。中书令耶律楚材坚决反对,他认为将士辛苦奋战为的就是土地和人民,屠城会导致得地无民,而且“奇巧之工,厚藏之家”都集中在汴京,屠城会导致一无所获,没有人民就没有人向朝廷交纳赋税,军队会白辛苦一场,最后窝阔台采纳了耶律楚材的意见,只关押了金朝宗室,其他人一概赦免。当时在汴京城中躲避兵祸的147万名居民因为耶律楚材的建议得以免于兵祸。

1233年5月29日(农历四月十九日),崔立将汴京城中的金朝宗室梁王完颜从恪、荆王完颜守纯以及其他宗室男女五百余人送到速不台军队驻地青城,速不台将他们送到漠北草原窝阔台的行銮驻跸之处,窝阔台为报祖先之仇(金熙宗当年曾将蒙古俺巴孩汗钉死在木驴上),将他们全部处死。同一天,崔立面见速不台,正式归降大蒙古国,速不台率军进入汴京,维护城中秩序,并将城中的金朝后妃和宗庙宝器也送到漠北草原窝阔台的行銮驻跸之处。

1234年2月9日(农历正月十日),大蒙古国军队與南宋军队联合攻入蔡州(今河南汝南县),金哀宗自杀,金末帝死于乱军之中,金朝灭亡。整个北方中原地区并入大蒙古国版图。

自1234年窝阔台汗灭金朝,到1368年烏哈噶圖汗 (元惠宗)逃离大都回到草原,由蒙古族建立的蒙古汗国、元帝國两政权,總共统治北方中原黄河流域长达134年。

端平入洛与蒙宋开战:1233年5月29日蒙古军队取得汴京(今河南开封市)后,继续进攻蔡州(金哀宗所在地),由于金朝军队抵抗顽强,为了减少损失,窝阔台决定联合南宋政權攻克蔡州灭亡金朝。

按照蒙宋双方协议,蒙宋联军攻克蔡州后,南宋可以取得蔡州未破前尚在金朝控制的河南土地,也就是唐、邓、蔡、颍、宿、泗、徐、邳等州(均位于河南南部)。这些州位于金朝和南宋的交界地带,属于金朝领土最南端的州。

在1234年2月9日蒙宋联军攻克蔡州灭亡金朝后,因为河南一带久经战火,田地荒芜,缺乏粮食,当时又正值冬季,天气严寒,于是把当地大部分居民暂时迁往河北一带,准备等天气转暖后将居民再陆续迁回河南,并恢复农业生产。同时军队久经战事,也需要休整,大部分军队撤到黄河以北。

宋理宗在部分大臣的怂恿下违背当初的蒙宋协议,1234年六月,宋军分二路出兵北伐,准备收复当年被金朝攻取的三京:西京河南府(今河南洛阳市)、东京开封府(今河南开封市)、南京应天府(亦称之为归德,今河南商丘市),这三京均位于河南北部,在蒙宋协议之前就已经被蒙古军队攻取,自然不属于当初蒙宋协议中灭金后南宋可以得到的领土。

由于宋军北上攻取三京发生在宋理宗端平年间,史称“端平入洛”。端平入洛揭开了蒙古与南宋对峙,连续四十余年不断战争的序幕,直到忽必烈渡过长江、灭亡南宋。

中國南宋违背蒙宋协议,大举进兵,但因为蒙古灭金后,大部分金朝军队和居民都已经撤到黄河以北,南宋军队最初进展顺利,一个月后顺利占领几乎是空城的三京。由于三京缺乏粮草,宋军携带粮草较少又缺乏后勤补给,蒙古军队又随后发起反击,宋军很快撤离三京,并撤回南宋境内。

蒙古军队随后追至原金朝和南宋的边界线一带,并向南宋边界的州县发起进攻,因为蒙古军队并不是很适合南方河流密布的地形作战,在取得一定战果后撤回中原。

自南宋违约进攻蒙古,端平入洛以后,南宋天灾人祸接连不断,国力逐渐衰弱直至灭亡。在军事上,收复三京失败,损兵折将,士气不振,将心不稳,成为南宋守边士兵面临的严重问题。

在中国北方实施“以儒治国”:1230年,有近臣别迭等人向窝阔台上奏,认为“汉人无补于国,可悉空其人以为牧地。”主张将汉人驱逐,把汉地的耕地变为牧场,耶律楚材则上奏请求均定中原地税、商税、盐、酒、铁冶、山泽之利,每年可得赋税白银50万两、帛8万匹、粟40余万石,足以支持窝阔台南征金朝的军队所需,窝阔台同意由耶律楚材试行。

1230年农历十一月,耶律楚材奏请在大蒙古国统治的黄河以北的河北、山西、山东等地(当时金朝尚未灭亡,黄河取道江苏北部的淮河入海)设立燕京等地设立十路征收课税使,并选用有名的儒士作为课税官员,得到窝阔台批准。

1231年农历八月,窝阔台到达云中(今山西大同市),十路征收课税使将当年征收到的汉地赋税簿册和金帛陈于廷中,窝阔台大悦,当日设立中书省,改侍从官名,以耶律楚材为中书令,粘合重山为左丞相,镇海为右丞相。

1235年春,窝阔台决定在哈拉和林建都城,修建万安宫;并部署伐南宋、征高丽和再次西征;1236年正月,万安宫建成。窝阔台大宴群臣,同月,窝阔台下诏发行纸币交钞。

1234年正月灭金朝后,窝阔台下诏括编汉地户籍,他接受耶律楚材的建议,以按户为单位收取赋税。由中州断事官失吉忽秃忽主持。1236年八月,括户完成,括得汉地民户110余万户。

1236年括户完成后,失吉忽秃忽主张按以往风俗在中原对诸王和有功之臣进行分封,窝阔台表示同意。耶律楚材力陈“裂土分民”的弊害,使窝阔台同意封地的官吏须朝廷任命,除常定赋役外,诸王勋臣不得擅自征敛,以限制诸王勋臣在封地的权力。

括户完成后,耶律楚材制订了中原赋税制度:每两户出丝一斤,上交朝廷,以供中央政府使用,每五户出丝一斤,以与所赐之家;先由中央政府征收,然后赐予该受封贵族,除此之外贵族不得擅加征敛。上田每亩税三升半,中田三升,下田二升,水田五升;商税三十分之一;盐每银一两四十斤。

这个赋税的定额是比较轻的,有利于当时已遭破坏的中原地区休养生息。在遇到大的灾情时,楚材还采取免征的措施。如果部分地区出现逃亡浮客,他们的赋税要由留下的主户负担,这些主户负担的赋税会重一些。此外,民户们也要负担一些随意性很大的杂泛差役。总的来说,民户们的负担还是相对比较轻的。

在耶律楚材的努力下,中原及北方的经济得到了恢复和保存。

1230年耶律楚材制定课税格,1231年收取的各种赋税中,白银为50万两,1234年灭金朝取得河南等地,赋税收入一直在增加,到了1238年,朝廷在中原汉地收取的各种赋税中,白银为110万两。丝和米等赋税也有显著增加。

1233年,为了培养蒙汉双语翻译类人材,窝阔台下诏在燕京(今北京市)建国子学,派遣蒙古人子弟18人学习汉语;汉人子弟12人,学习蒙古语和弓箭,并选儒士为教读。规定受业学生不仅要学习汉人文书,还要“兼谙匠艺,事及药材所用、彩色所出、地理州郡所纪,下至酒醴麴蘖、水银之造,饮食烹饪之制,皆欲周览旁通”。当时,全真教在燕京势力很大,儒家士大夫有很多托庇于全真教。燕京的学宫也是如此,学宫的主持者除杨惟中之外,葛志先、李志常均为当时有名的全真道士。

1233年农历四月,蒙古军队进入汴京城(今河南开封市),中书令耶律楚材向窝阔台奏请遣人入城,求孔子家族後代,得五十一代孙元措,奏袭封衍圣公,付以孔林庙地。耶律楚材又派人入汴京,挑选了大量的人才。

1233年农历六月,窝阔台下诏,以孔子五十一世孙孔元措袭封衍圣公。

1233年冬天,窝阔台敕修燕京孔子庙及浑天仪。

1236年农历三月,复修孔子庙及司天台。

1236年农历六月,耶律楚材奏请窝阔台同意后,在燕京(今北京市)建立编修所,在平阳(今山西临汾市)建立经籍所,主持经史类书籍的编纂和刊行,召儒士梁陟充长官,以王万庆、赵著副之。让他们直释九经,进讲东宫。又率大臣子孙,执经解义,使他们知道圣人之道。

1237年,窝阔台下旨蠲免孔子、孟子、颜子等儒教圣人子孙的差发杂役。

1237年,耶律楚材奏请对儒士举行科举考试,这就是1238年举行的戊戌选试,共录取4030人,皆当时的名士。

1238年,耶律楚材又支持杨惟中和姚枢在燕京建立太极书院,请赵复等人为师教授儒家的经典。南宋名士赵复的讲学,使程朱理学在北方中原地区传播开来。

戊戌选试:1234年2月9日,蒙古帝国灭金朝,夺取中原地区后,急需人才治理国家。

元太宗九年农历八月二十五日(1237年9月15日),根据中书令耶律楚材的建议,窝阔台下诏书命断事官术忽德和山西东路课税所长官刘中,历诸路考试,试诸路儒士,开科取士,并对考试内容和参加考试者的身份要求以及中选者的优厚待遇作了详细说明。

北方中原地区的诸路考试,均于1238年(戊戌年)举行,史称“戊戌选试”。

1238年的这次考试共录取东平杨奂等4030人,皆为一时名士,使得朝廷及时得到了加强统治所需要的各方面的人才。但后来“当世或以为非便,事复中止”。

直到元仁宗1313年下诏恢复科举,此时距离元太宗1238年的“戊戌选试”已经有75年,天下读书的士人至此再次获得以科举方式晋身做官的途径。

西征欧洲:1234年2月9日金朝灭亡后,由於大蒙古国與南宋接壤,使雙方的衝突日漸加劇,也拉開了雙方往後45年不斷爭戰的序幕。在南方戰線僵持不下之時,蒙古大軍的鐵蹄轉往東方的高麗,並使之臣服,西線方面,以拔都為首的欽察汗國,完全控制了罗斯,並繼續西進,佔領了除诺夫哥罗德以外俄羅斯的领土,以及波蘭和匈牙利的一部。

1241年12月11日(农历十一月八日),窝阔台因為酗酒而突然暴斃,使他的西征進程被逼中止。當時大軍正朝往維也納推進,但為了趕返參加位於蒙古的库里尔台大会而急忙撤軍,自此以後,蒙古大軍再也沒有踏足這片土地。

1241年年底,在窝阔台去世后不久,他的二哥察合台去世。

窝阔台去世后,1242年春天,皇后乃马真后开始称制,处理朝政,直到1246年8月24日窝阔台之子貴由繼任大汗為止。乃马真后临朝称制期间,朝政比较混乱,中书令耶律楚材力争而不能有效果,于1244年农历五月忧愤而死。

元朝重臣郝经在中统元年(1260年)农历八月给元世祖忽必烈的上书《立政议》中对元太宗窝阔台的評價是:“当太宗皇帝临御之时,耶律楚材为相,定税赋,立造作,榷宣课,分郡县,籍户口,理狱讼,别军民,设科举,推恩肆赦,方有志于天下,而一二不逞之人,投隙抵罅,相与排摈,百计攻讦,乘宫闱违豫之际,恣为矫诬,卒使楚材愤悒以死。”(说明:元太宗窝阔台在世之时,耶律楚材还是深受重用的,1241年元太宗去世,帝位空缺,皇后乃马真后开始临朝称制,朝政比较混乱,中书令耶律楚材力争而无效果,他于1244年忧愤而死)

明朝官修正史《元史》宋濂等的評價是:“帝有宽弘之量,忠恕之心,量时度力,举无过事,华夏富庶,羊马成群,旅不赍粮,时称治平。”

清朝史学家邵远平《元史类编》的評價是:“册曰:嗣业恢基,缵绪立制;五载灭金,十路命使;定赋崇儒,用昌厥世;仁厚恭俭,时称平治。”

清朝史学家毕沅《续资治通鉴》的評價是:“太宗性宽恕,量时度力,举无过事。境内富庶,旅不赍粮,时称治平。”

清朝史学家魏源《元史新编》的評價是:“帝有宽宏之量,淳朴之质,乘开国之运,师武臣力,继志述事,席卷西域,奄有中原。惟知诸子不材,又知宪宗之克荷,而储位不早定,致身后政擅宫闱,大业几沦,有余憾焉。”

清朝史学家曾廉《元书》的評價是:“论曰:太宗时金人已弱,然犹足阻河为固也。太宗遵遗令戡凤翔,道兴元,以达唐邓,而汴梁墟,可谓闻斯行之矣。当是时,操持国政,耶律楚材郁为时栋。然太宗之用楚材,以利也。太宗言利,楚材即以其利利天下,而纪纲粗立矣。用相违也,而相成也,岂非天哉!故开国之运,云龙风虎,非雷同也。”

清末民初史学家屠寄《蒙兀儿史记》的評價是:“论曰:财者,一国所公有也。语曰:百姓足,君孰与不足?人君以国用困乏,多取于民,然且不可。况可纵奸人异类,恣其侵夺乎?斡歌歹汗初得金,许奥都剌合蛮扑买中原银课,举国家财政大权授之贾胡之手,公利而私取之,上下交损焉。封建之制,始于自然,强并弱,众暴寡。自天子以至食采之大夫,各私其土地人民。古圣王不得以而仍之。秦汉以降,此制渐废,偶一行之,罔不召乱。自非至无识者,不轻议复也。汗括汉户,分赐诸王贵戚,其视无辜之民与奴虏奚择。彼固不知封建为何事,然斯制若行,弊且甚于封建。微耶律楚材言,纵虎豹而食人肉矣。前史称汗有宽仁之量,忠恕之心,度时量力,动无过举。迹其立站赤、选税使、试儒士、释俘囚,诏免旱蝗之租,代偿羊羔之息,固非无志于民者,惜乎不达怡体,而左右之人将顺其美者,又寡也。”

民国官修正史《新元史》柯劭忞的評價是:“太宗宽平仁恕,有人君之量。常谓即位之后,有四功、四过:灭金,立站赤,设诸路探马赤,无水处使百姓凿井,朕之四功;饮酒,括叔父斡赤斤部女子,筑围墙妨兄弟之射猎,以私撼杀功臣朵豁勒,朕之四过也。然信任奥都拉合蛮,始终不悟其奸,尤为帝知人之累云。”


\subsection{定宗生平}


貴由汗(1206年-1248年4月),大蒙古国第三任大汗,孛儿只斤氏,窩闊台長子,乃馬真后所生,1246年8月24日—1248年在位,计2年。

至元三年(1266年)十月,太庙建成,追尊庙号定宗,谥简平皇帝,在宗庙中列祭于第七室,排在忽必烈之父托雷后、忽必烈之兄蒙哥前。

早年參加征伐金朝,俘虜了其親王;又曾經参与西征欧洲。蒙古帝国第三任大汗贵由、第四任大汗蒙哥,以及后来的元朝开国皇帝忽必烈,堂兄弟三人都是蒙古第二次西征时拔都的部下。

1241年12月11日,窝阔台去世,汗位虚悬,贵由的母亲乃马真脱列哥那称制,法纪混乱,很多宗王贵族滥发牌符征敛财物,唯有拖雷家族没有这样做,赢得了声誉。乃马真后欲立长子贵由为大汗,拔都与贵由不和,一直不肯参加选汗大会,后来,成吉思汗幼弟铁木哥斡赤斤也领兵来争位,帝国面临汗位争夺战和混乱的危险。拖雷的遗孀唆鲁禾帖尼决定率诸子参加忽里勒台大会,1246年8月24日,宗王大臣们拥立贵由登基,贵由成为大蒙古国大汗,“全体宗王们脱帽,解开宽腰带,把贵由扶上金王位,以汗号称呼他,到会者对新君九拜表示归顺,在帐外的藩王及外国使臣等也同时跪拜称贺。”

贵由登基后,虽然本人很有权威,但是因沉湎酒色、手足痉挛,并没有什么作为,且不理政事,多委于下臣。

1248年春,貴由親率大軍西征拔都,至橫相乙兒(今新疆青河縣東南)病死。一說被拔都系势力毒殺。

1246年8月24日至1248年4月20日在位,在位仅一年零八个月。

和罗马教宗的交往:贵由在位期间和罗马教宗有交往。歐洲諸國傳言蒙古大汗信仰基督教,因此教宗诺森四世派遣若望·柏郎嘉宾出使,希望勸說蒙古大汗不要傷害基督徒,同時要他深入了解蒙古人的風土民情、作戰方式等。1245年4月16日从法国里昂出发,途经神圣罗马帝国、波兰王国和基辅罗斯等国(他于1246年2月3日离开基辅)。1246年4月4日,他在伏尔加河下游的萨莱(今伏尔加河下游阿斯特拉罕附近)受到钦察汗拔都的接见。拔都派他去蒙古草原见大汗,他经过讹答剌、伊犁河下游、叶密立河—翻越阿尔泰山,向东抵达蒙古草原。

1246年7月22日,他抵达距离哈拉和林只有半天路程的地方,选举大汗的忽里勒台大会正在此召开。他目睹了1246年8月24日贵由的当选,并留下了对贵由的生动描述:“在他当选时,约有四十,最多四十五岁。他是中等身材,非常聪明.极为精明,举止极为严肃庄重。从来没有看见他放声大笑,或者是寻欢作乐。” 最後他未能说服贵由皈依天主教,得到贵由的回信后,于1246年11月13日离开蒙古草原,向西踏上归途,经伏尔加河下游的拔都驻地返回西方,1247年9月5日他到达拔都驻地,又经基辅返回西方。

凉州会盟与吐蕃归附:1247年,吐蕃诸部宗教界领袖萨迦班智达·贡嘎坚赞(简称萨班)同大蒙古国皇子西凉王阔端(贵由之弟,窝阔台之子,成吉思汗之孙)在凉州(今中国甘肃武威市)议定了吐蕃归附的条件,其中包括呈献图册,交纳贡物,接受派官设治,吐蕃地区纳入大蒙古国(蒙古帝国)治下,史称“凉州会盟”。

窝阔台家族的衰落:根据《新元史》记载,1248年农历三月(1248年4月),贵由以养病为名带兵西巡,途中病逝于横相乙儿(今新疆青河东南),距離別失八里一天路程。

贵由死后,其遗孀斡兀立海迷失临朝称制,由於贵由与拔都早年不和,拔都拒絕奔喪。为了对抗窝阔台家族,拔都以长支宗王的身份遣使邀请宗王、大臣到他在中亚草原的驻地召开忽里台,商议推举新大汗。窝阔台系和察合台系的宗王们多数拒绝前往,海迷失后只派大臣八剌为代表到会。唆鲁禾帖尼则命长子蒙哥率诸弟及家臣应召前往。

1250年,庫力臺大會在中亚地区拔都的驻地召开,拔都在会上极力称赞蒙哥能力出众,又有西征大功,应当即位,并指出贵由之立违背了窝阔台遗命(窝阔台遗命失烈门即位),窝阔台后人无继承汗位的资格。大会通过了拔都的提议,推举蒙哥为大汗。窝阔台、察合台两家拒不承认,唆鲁禾帖尼和蒙哥又遣使邀集各支宗王到斡难河畔召开忽里台,拔都派其弟别儿哥率大军随同蒙哥前往斡难河畔,但窝阔台、察合台两家很多宗王仍不肯应召,大会拖延了很长时间。

由于蒙哥的母亲唆鲁禾帖尼的威望甚高,并且善于笼络宗王贵族,多数宗王大臣最终应召前来,1251年农历六月在蒙古草原斡难河畔举行庫力臺大會,元宪宗元年农历六月十一日(1251年7月1日),宗王大臣们共同拥戴蒙哥即大汗位。此后,为了巩固汗位,唆鲁禾帖尼在镇压反对者时毫不留情,并亲自下令处死贵由的皇后斡兀立海迷失。

自此汗位繼承,便由窝阔台家族转移到了拖雷家族,皇族内部的分裂,为后来大蒙古国的彻底分裂,埋下伏筆。

明朝官修正史《元史》宋濂等的評價是:“三年戊申春三月,帝崩于横相乙儿之地。……是岁大旱,河水尽涸,野草自焚,牛马十死八九,人不聊生。诸王及各部又遣使于燕京迤南诸郡,征求货财、弓矢、鞍辔之物,或于西域回鹘索取珠玑,或于海东楼取鹰鹘,驲骑络绎,昼夜不绝,民力益困。然自壬寅以来,法度不一,内外离心,而太宗之政衰矣。”

清朝史学家毕沅《续资治通鉴》的評價是:“自太宗皇后称制以来,法度不一,内外离心。至是国内大旱,河内尽涸,野草自焚,牛马死者十八九,人不聊生。诸王及各部,又遣使于诸郡征求货财,或于西蕃、回鹘索取珠玑,或于东海搜取鹰、鹘、驿骑络绎,昼夜不绝,民力益困。皇后立库春子实勒们听政,诸王大臣多不服。”

清朝史学家魏源《元史新编》的評價是:“连岁大旱,河水尽涸,野草自焚,牛马十死八九,人不聊生。诸王及各部又遣使于燕京迤南诸郡,征求货财,或于西域、回鹘索取珠玑,或于海东搜取鹰鹘,驿骑不绝,内外离心,故无可纪。然自太祖崩后,拖雷监国者一年,太宗崩后,六皇后称制者四年,定宗之后,皇后临朝者又几四年,前后凡九载无君而国不乱,卒能创业垂统,上竝漢、唐者,则皆宗王宿将维持拱卫,根干蟠据之力。”

清朝史学家曾廉《元书》的評價是:“论曰:定宗之世,事多缺漏,而前史曰:‘ 帝崩之岁大旱,河水尽涸,野草自焚,牛马十死八九,人不聊生。诸王及各部又遣使于燕京迤南诸部,征求货财、弓矢、鞍辔,或于西域回鹘索取珠玑,海东索取鹰鹘,驿骑络绎,昼夜不绝,民力益困。然自壬寅以来,法度不一,内外离心,而太宗之政衰矣。’其言壬寅,盖以昭慈皇后称制时言之也。夫定宗即位时,年四十矣,而不能辑诸王侯大将,纪解威亵,此太宗之不谋付以匕图者乎?然在于汉亦孝惠之亚也。惟无良臣为之辅弼,而宗藩党羽遂成,以夺皇阼。炎异之丛,兴其足信耶?而失烈门则太宗遗诏所立也。前史复曰:定宗崩后,三岁无君。蒙哥之党之不欲以为君,非蒙古之无君也。窜之北陲,并逐太宗皇后而弑定宗皇后,可不谓之逆哉!自是而太宗子孙亦不欲以蒙哥兄弟为君,逮于海都,而中原震矣。”

中華民国史学家屠寄《蒙兀儿史记》的評價是:“汗严重有威,临御未久,不及设施,惟乃蛮真可敦称制时,威福下移,汗既亲政,纲纪粗立,君权复尊,自幼多疾,成吉思汗尝命亦鲁王之祖忽鲁扎克为之主膳。中年性好酒色,手足有拘挛之病,在位之日,常以疾不视事,事多决于大臣镇海、合答二人云。”

中華民国官修正史《新元史》柯劭忞的評價是:“史臣曰:定宗诛奥部拉合蛮,用镇海、耶律铸,赏罚之明,非太宗所及。又乃马真皇后之弊政,皆为帝所铲革。旧史不详考其事,谓前人之业自帝而衰,诬莫其矣。” 


\subsection{宪宗生平}

蒙哥汗(1209年1月10日-1259年8月11日),大蒙古国第四任大汗,也是大蒙古国分裂前最後一個受普遍承認的大汗。他是成吉思汗幼子拖雷的长子、窝阔台的养子,由窝阔台的昂灰皇后抚养长大。

1251年7月1日登基,在位8年零2个月。其间长期主持对南宋、大理的战争,为其弟忽必烈最终建立元朝奠定坚实基础。至元三年(1266年)十月,太庙成,元廷追尊蒙哥庙号为宪宗,谥桓肃皇帝 。

潜邸岁月:1209年1月10日(农历戊辰年十二月三日),蒙哥生于漠北草原,是成吉思汗之孙,拖雷的长子,拖雷正妻唆鲁禾帖尼所生的嫡长子(元世祖忽必烈是嫡次子,旭烈兀是嫡三子,阿里不哥是嫡四子)。窝阔台汗即位之前,以蒙哥为养子,让昂灰皇后抚育蒙哥,并在他长大后,为他娶火鲁剌部女子火里差为妃、分给他部民。至1232年拖雷去世后,蒙哥才回去继承拖雷的封地。蒙哥多次跟随窝阔台参加征伐,屡立奇功。蒙哥沉默寡言、不好侈靡,喜歡打獵。1235年,蒙哥参加第二次蒙古西征,與拔都、貴由西征欧洲的不里阿耳、欽察、斡羅思等地,屢立戰功,在里海附近,活捉钦察首领八赤蛮。

拖雷家族争得大汗之位:1248年农历三月贵由汗去世后,由皇后斡兀立海迷失临朝称制;由於与贵由早年不和,拔都(铁木真长子术赤之子)拒絕奔喪。为了对抗窝阔台家族,拔都以长支宗王的身份遣使邀请宗王、大臣到他的驻地(在中亚草原)召开忽里台(蒙古的军政会议),商议推举新大汗。窝阔台系和察合台系的宗王们多数拒绝前往,贵由汗的皇后斡兀立海迷失只派大臣八剌为代表與会。托雷之妻唆鲁禾帖尼则命长子蒙哥率诸弟及家臣应召前往。

1250年,忽里台大会在拔都的驻地(中亚地区)召开,拔都在会上极力称赞蒙哥能力出众,又有西征大功,应当即位,并指出贵由之立违背了窝阔台遗命(窝阔台遗命失烈门即位),窝阔台后人不当有继承汗位的资格。大会通过了拔都的提议,推举蒙哥为大汗。窝阔台、察合台两家拒不承认,唆鲁禾帖尼和蒙哥又遣使邀集各支宗王到斡难河畔召开忽里台,拔都派其弟别儿哥率大军随同蒙哥前往斡难河畔,但窝阔台、察合台两家的很多宗王仍不肯应召,大会拖延了很长时间。

由于唆鲁禾帖尼威望甚高,并且善于笼络宗王贵族,最终多数宗王大臣应召前来,1251年农历六月在蒙古草原斡难河畔举行忽里台,宗王大臣们于7月1日(农历六月十一日)共同拥戴蒙哥登基,蒙哥成为大蒙古国大汗;蒙哥即位的当日,尊母亲唆鲁禾帖尼为皇太后。此后,为了巩固汗位,皇太后唆鲁禾帖尼镇压反对者毫不留情,并亲自下令处死贵由汗的皇后斡兀立海迷失。

自此“大汗”之位的繼承,便由窝阔台家族转移到了拖雷家族,为后来大蒙古国分裂埋下伏筆。

1251年7月1日,蒙哥即位后,窩闊台系諸宗王拒絕承認,被蒙哥率兵鎮壓;蒙哥又以其弟忽必烈统領漠南漢地軍政事務,同时指挥向南(东亚)、向西(西亚)两个方向的征服战争。

征服大理:1252年农历六月,命弟忽必烈南征大理国,次月,忽必烈率军出发。1253年农历八月,忽必烈军至陕西,开始进攻位于今云南等地的大理国。1254年1月2日(元宪宗三年农历十二月十二日),忽必烈攻克大理城,大理国王段兴智投降,大理国灭亡,并入大蒙古国版图。1256年,段兴智前往漠北和林觐见蒙哥汗,被任命為大理總管,子孙世襲。

从1254年大蒙古国忽必烈奉命灭大理国、大理国王战败投降,到1382年驻守云南的元朝梁王把匝剌瓦尔密兵败自杀、元朝大理总管段世战败归降明军,蒙古族建立的政权统治云南地区长达128年。

远征西亚:元宪宗三年(1253年)六月,蒙哥命弟旭烈兀率大军十万西征。旭烈兀的西征军从漠北草原出发,1256年大军渡过阿姆河后所向披靡,先攻灭波斯南部的卢尔人政权,1256年攻灭位于波斯西部的木剌夷国(阿萨辛派),1258年灭亡巴格达的阿拔斯王朝,1260年3月1日,灭亡叙利亚的阿尤布王朝,并派兵攻占了小亚细亚大部分地区。

攻占叙利亚后,旭烈兀西征军兵锋抵达今天地中海东岸的的巴勒斯坦地区,即将与埃及的马木留克王朝交战,此时旭烈兀得到使者带来的帝国最高统治者蒙哥在四川去世的消息,于是只派先锋怯的不花率不到一万军队驻守叙利亚,自己率大军开始东返。1260年9月3日,埃及马木留克王朝趁着旭烈兀攻率主力东返,攻占叙利亚,杀怯的不花,旭烈兀愤怒至极,本想率军继续西征,但此时他和钦察汗国的别儿哥汗因为争夺阿塞拜疆爆发了战争,只好结束西征。

旭烈兀东返途中得到忽必烈和阿里不哥争位的消息,于是留在西亚,自据一方,并宣布支持忽必烈,后来被忽必烈封为“伊儿汗”,西亚的伊儿汗国从此建立。

征伐南宋:1258年,蒙哥、其弟忽必烈和大将兀良合台分三路大举进攻南宋。1258年农历七月,蒙哥亲率主力进攻四川,所向披靡,攻克四川北部大部分地区,直到1259年初在合州(今重庆合川区)釣魚城下攻势受阻,战事胶着数月,蒙哥死前最终未能完成此次战役;而蒙哥死后,忽必烈得知忽里台大会选举阿里不哥即位,匆匆率军赶回漠北争夺汗位,对南宋的征伐计划暂时搁置。

蒙哥的去世原因,至今史学界尚无明确结论。主要有以下几说:

战争中受伤不治身亡:《合州志》記載,1259年8月11日(农历七月二十一日),蒙哥在合州钓鱼山一役,被南宋軍投石機的巨石打中,六天後傷重而亡。《馬可波羅游記》和明萬歷《合州志》则记载蒙哥在攻打合州时被釣魚城守城武器矢石擊中而重傷后去世。翦伯赞主编的《中国史纲要》采取了这种说法,书:“蒙古军因军中痢疾盛行,死伤极多,蒙哥汗又为宋军的飞矢射中身死”。《古今圖書集成》中的《釣魚城記》则记載:“炮風所震,因成疾。班師至愁軍山,病甚……次過金劍山溫湯峽(今重慶市北碚北溫泉)而歿”,謝士元在《遊釣魚山詩序》亦說蒙哥是“炮風致疾”而死。

病逝:《元史》则称天气多雨,蒙哥身体不适,于农历七月癸亥日死在钓鱼山 蒙古帝国伊儿汗国宰相拉施特的《史集》也推斷當時正值酷暑季节,军中痢疾流行,蒙哥亦染病身亡。畢沅在《續資治通鑑》稱蒙哥死於痢疾。

其他说法有:黃震的《古今紀要逸編》認為蒙哥因為屢攻合州釣魚城不克,致憂憤死;《海屯紀年》說是落水死。

据传蒙哥臨終前留下遺言,將來若攻下釣魚城,必屠殺全部軍民百姓;然而此事《元史》、《新元史》、《史集》均无记载(此三本史书记载蒙哥病逝,和钓鱼城的战斗无关)。後來釣魚城於1279年投降時,忽必烈赦免了所有軍民。

蒙哥的去世,对当时的蒙古帝国政局乃至世界格局都有極大的影響:蒙哥去世导致了旭烈兀统帅的第三次蒙古西征被迫中止;随后爆发了其弟忽必烈与阿里不哥争夺汗位之战,最终导致大蒙古国(蒙古帝国)的分裂。

法国国王路易九世派遣传教士卢布鲁克前往东方觐见蒙古大汗商讨传教和结盟对抗阿拉伯人事宜。卢布鲁克于1253年从地中海东岸阿克拉城(今以色列海法北)出发,于1253年5月7日离开君士坦丁堡,一路东行,渡过黑海,秋天到达伏尔加河畔,谒见拔都汗。拔都认为自己无权准许他在蒙古人中传教,便派他去东方觐见大汗蒙哥。卢布鲁克觐见拔都后,留下了对拔都的生动描述:“拔都坐在一金色的高椅上,或者说坐在像床一样大小的王位上,须上三级才能登上宝座,他的一个妻子坐在他旁边。其余的人坐他的右边和这位妻子的左边。”

1253年12月,卢布鲁克到达哈拉和林南部蒙哥冬季营地。1254年1月4日觐见蒙哥,并留下了对蒙哥的生动描述:“我们被领入帐殿,当挂在门前的毛毡卷起时,我们走进去,唱起赞美诗。整个帐幕的内壁全都以金布覆盖着。在帐幕中央,有一个小炉,里面用树枝、苦艾草的根和牛粪生着火。大汗坐在一张小床上,穿着一件皮袍,皮袍像海豹皮一样有光泽。他中等身材,约莫45岁,鼻子扁平。大汗吩咐给我们一些米酒,像白葡萄酒一样清澈甜润。然后,他又命拿来许多种猎鹰,把它们放在他的拳头上,观赏了好一会。此后他吩咐我们说话。他有一位聂思托里安教(景教)徒作为他的译员。”

1254年4月5日,随同蒙哥来到大蒙古国首都哈拉和林。8月18日带着蒙哥致路易九世的国书西归,信中写道:“这是长生天的命令。天上只有一个上帝,地上只有一个君主,即天子成吉思汗。”蒙哥以长生天以及它在地上的代表“大汗”的名义命令法兰西国王承认是他的属臣。

他于1255年回到地中海东岸。一年后,他用拉丁文写成的出使报告交给路易九世,即《东方行记》,又称《卢布鲁克游记》。

小亚美尼亚国王海屯一世于1244年归附大蒙古国,成为属国。1254年春,海屯一世遵从拔都汗之命亲自前往蒙古草原觐见大汗蒙哥。他与随臣一路东行,5月至拔都营帐(伏尔加河下游)谒见,然后继续东行,9月13日到达蒙哥汗廷(哈拉和林)朝见、献贡,得到蒙哥颁赐的诏书;“诏书上盖有蒙哥的御玺,不许人欺凌他及他的国家。还给他一纸敕令,允许各地教堂拥有自治权。”在哈拉和林停留50天后,他离开汗廷西还。

返回途中在中亚河中地区觐见蒙哥汗之弟弟旭烈兀,行程8个月,1255年7月返抵小亚美尼亚。回国后撰写《海屯行纪》。

父亲:拖雷,1227年—1229年帝位空缺时担任大蒙古国监国,1232年去世。《元史·睿宗本纪》载,蒙哥即位后追尊拖雷为皇帝,为拖雷上庙号睿宗、谥号英武皇帝,1266年忽必烈改谥其为景襄皇帝,1310年元武宗海山加谥为仁圣景襄皇帝。(元朝由忽必烈建立于1271年;然而在元朝建立之前,随着蒙古对金国、西夏等沿袭了中原礼制的王朝的征服,蒙古在中国地区的统治也受到了汉文化的影响,包括任用契丹人、汉人为官,尊重儒学等,为逝者上庙号、谥号等,或是出于对汉文化的吸收,而非意味着大蒙古国时期的“大汗”等同于元朝时期的“皇帝”。元史所载宗室,在忽必烈的同辈以及先辈中,除了宗庙里奉祭的重要祖先被予以“追赠尊谥”,均没有汉式的封号;而忽必烈建立元朝之后,宗室贵族才渐渐有了诸如“鲁国公主”之类的汉式封号,可见“大蒙古国”与“元朝”实为两个政权,不过后者宣称对前者继承耳。)

母亲:唆鲁禾帖尼,是蒙哥,忽必烈,旭烈兀,阿里不哥四人的生母,1251年蒙哥汗即位后尊其为皇太后,1252年去世。1266年元世祖忽必烈为其上谥号庄圣皇后,1310年元武宗海山加谥为显懿庄圣皇后。她的四个儿子皆曾称汗称帝,被后世史学家尊称为“四帝之母”。

元朝重臣郝经在中统元年(1260年)农历八月给元世祖忽必烈的上书《立政议》中对元宪宗蒙哥的評價是:“先皇帝初践宝位,皆以为致治之主,不世出也。既而下令鸠括符玺,督察邮传,遣使四出,究核徭赋,以来民瘼,污吏滥官,黜责殆遍,其愿治之心亦切也。惜其授任皆前日害民之尤者,旧弊未去,新弊复生,其为烦扰,又益剧甚,而致治之几又失也。”

明朝官修正史《元史》宋濂等的評價是:“帝刚明雄毅,沉断而寡言,不乐燕饮,不好侈靡,虽后妃不许之过制。初,太宗朝,群臣擅权,政出多门。至是,凡有诏旨,帝必亲起草,更易数四,然后行之。御群臣甚严,尝谕旨曰:‘尔辈若得朕奖谕之言,即志气骄逸,志气骄逸,而灾祸有不随至者乎?尔辈其戒之。’性喜畋猎,自谓遵祖宗之法,不蹈袭他国所为。然酷信巫觋卜筮之术,凡行事必谨叩之,殆无虚日,终不自厌也。”

清朝史学家邵远平《元史类编》的評價是:“册曰:天象知祥,众心戴主;遐辟西南,深入中土;未究厥勳,亦振乃武;友弟因心,终昌时绪。”

清朝史学家毕沅《续资治通鉴》的評價是:“宪宗沉断寡言,不乐宴饮,不好侈靡,虽后妃亦不许之过制。初,定宗朝,群臣擅权,政出多门,帝即位,凡有诏旨,必亲起草,更易数四,然后行之。御群臣甚严,尝曰:‘尔辈每得朕奖谕之言,即志气骄逸。志气骄逸,而灾祸有不随至者乎?尔辈其戒之!’性喜畋猎,自谓遵祖宗之法,不蹈袭他国所为。然酷信巫觋、卜筮之术,凡行事必谨叩之,殆无虚日。”

清朝史学家魏源《元史新编》的評價是:“帝早亲军旅,刚明沉断,威著中外。即位以后,不乐燕饮,不好侈靡,虽后妃不许之过制。初,太宗崩后,旷纪无君,黄裳御统,政出多门,阿柄几于旁落。至是,凡有诏旨,帝必亲起草,更易数四,然后行之。御臣下甚严,尝谓:‘臣下奖谕太过,即志气骄溢,过咎随之,是害之也。’承开国师武臣力之后,西平印度,南并大理,东取巴蜀,所向无敌。惟遵其国俗,喜田猎,信巫觋卜筮,是其小蔽。使太宗即世,早承大业,则伐宋之役,不俟末年而南北混一矣。天未既宋,暑雨老师,景命不延,故大勳重集于世祖皇帝。”

清朝史学家曾廉《元书》的評價是:“论曰:宪宗之立,有遗议焉。前史袭《元史》旧文,未为允也。史又称宪宗能辑士卒,皇子阿速歹猎骑伤稼,责之,复挞其近侍。卒拔民葱,即斩以徇。在蒙古治军可谓肃矣。夫古今称强汉、弱宋,然王坚以孤城罢卒,抗毳旃之劲族,卒乃师老解退。虽宪宗不晏驾,庸必克乎?盖自平金以来,中汉人之习,锦衣玉食,肌骨疏懈。故金以是亡,而元人兵势亦自是遂稍衰矣。历观史策,暾欲谷之言,有以哉!”

民国史学家屠寄《蒙兀儿史记》的評價是:“汗刚明雄毅,沉断而寡言,不乐燕饮,不好侈靡,虽后妃不许逾制。尝有西域商胡献水晶盆,珍珠伞等物,价值银三万余锭,汗曰:‘今百姓疲弊,所急者钱耳。朕独有此何为?’却之。赛典赤以为言,乃稍偿其值,且禁嗣后勿献。初,古余克汗朝群臣擅权,政出多门。至是,凡有诏旨,汗必亲起草,更易数四,然后行之。御群下甚严,尝谕旨曰:‘汝曹若得朕奖谕,即志气骄逸,志气骄逸,灾祸有不随至者乎?汝曹戒之。’性喜畋猎,自谓遵祖宗之法,不蹈袭他国所为。然酷信巫觋卜筮之术,凡行事必谨叩之,殆无虚日,终不自厌也。”

民国官修正史《新元史》柯劭忞的評價是:“帝沉断寡言,不喜侈靡。太宗朝群臣擅权,政出多门。至是,凡诏令皆帝手书,更易数四,然后行之。御群臣甚严,尝谕左右曰:“汝辈得朕奖谕,即志气骄逸,灾祸有不立至者乎?汝辈其戒之。”然酷信巫觋卜笨之术,凡行事必谨叩之无虚日,终不自厌也。史臣曰:“宪宗聪明果毅,内修政事,外辟土地,亲总六师,壁于坚城之下,虽天未厌宋,赍志而殂,抑亦不世之英主矣。然帝天资凉薄,猜嫌骨肉,失烈门诸王既宥之而复诛之。拉施特有言:蒙古之内乱,自此而萌,隳成吉思汗睦族田本这训。呜呼,知言哉!”


%%% Local Variables:
%%% mode: latex
%%% TeX-engine: xetex
%%% TeX-master: "../Main"
%%% End:
