%% -*- coding: utf-8 -*-
%% Time-stamp: <Chen Wang: 2021-11-01 17:07:18>

\section{天顺帝阿剌吉八\tiny(1328)}

\subsection{生平}

元天顺帝阿剌吉八,是元朝第七位皇帝,蒙古帝国第十一位大汗,元泰定帝之子。1328年10月3日至1328年11月14日在位,在位一个月十一天。

致和元年七月初十日(1328年8月15日),元泰定帝也孙铁木儿在上都病逝,丞相倒剌沙专权自用,过了一个多月仍迟迟不立9岁的太子阿剌吉八即位。

致和元年九月十三日(1328年10月16日),知樞密院事燕帖木儿在大都(今北京)拥立元武宗之子图帖睦尔即位,改元“天历”,图帖睦尔是为元文宗。

致和元年九月,丞相倒剌沙在上都拥立太子阿剌吉八为皇帝,改元“天顺”。

上都的天顺帝朝廷由丞相倒剌沙派兵进攻大都的文宗朝廷,元文宗派燕帖木儿率军迎战,双方经过多次战争,一开始双方互有胜负,后来大都朝廷逐渐占据军事优势。

天顺元年十月十三日(1328年11月14日),大都朝廷的军队包围上都,丞相倒剌沙等大臣奉皇帝宝出降,天顺年号被元文宗废除,倒剌沙在投降一个月后被杀。

倒剌沙投降后,天顺帝下落不明,不知所终,在位大约一個月;其後他沒被授與谥号和庙号,因此历史上以其年号称之为天顺帝。

清朝史学家曾廉《元书》的評價是:“论曰:曾子以托孤寄命,临大节而不可夺,斯为君子人也。故山有猛虎,樵采不入。前史称泰定帝能守祖宗之法,故天下无事。呜呼!徒法不能以自行也,向使汉武不委裘于霍光、金日磾,而倚上官桀、桑弘羊,则孝昭岂得晏然南面?况又弗如孝昭者乎?狙于近习而不知求天下之贤以佐佑之,贵为天子,富有天下,而不能庇其妻孥,若敖之鬼佞焉咎安在哉!君子是以不多子孟,而乐道孝武之善付托也。”

\subsection{天顺}

\begin{longtable}{|>{\centering\scriptsize}m{2em}|>{\centering\scriptsize}m{1.3em}|>{\centering}m{8.8em}|}
  % \caption{秦王政}\
  \toprule
  \SimHei \normalsize 年数 & \SimHei \scriptsize 公元 & \SimHei 大事件 \tabularnewline
  % \midrule
  \endfirsthead
  \toprule
  \SimHei \normalsize 年数 & \SimHei \scriptsize 公元 & \SimHei 大事件 \tabularnewline
  \midrule
  \endhead
  \midrule
  元年 & 1328 & \tabularnewline
  \bottomrule
\end{longtable}


%%% Local Variables:
%%% mode: latex
%%% TeX-engine: xetex
%%% TeX-master: "../Main"
%%% End:
