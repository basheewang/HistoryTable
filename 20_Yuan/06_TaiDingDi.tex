%% -*- coding: utf-8 -*-
%% Time-stamp: <Chen Wang: 2021-11-01 17:07:07>

\section{泰定帝也孙铁木儿\tiny(1323-1328)}

\subsection{生平}

元泰定帝也孙铁木儿是元朝第六位皇帝,蒙古帝国第十位大汗,在位5年,自1323年10月4日至1328年8月15日。清代乾隆晚期乾隆帝命改譯遼、金、元三史中的音譯專名,改譯伊蘇特穆爾,今日學界已無人使用。

他去世后不久,叔父之孫元文宗打敗其子元天顺帝,亦使他沒被授與谥号和庙号,因此历史上以其年号称之为泰定帝。

关于泰定帝的出生年,《元史》中的说法互相矛盾,在《元史·泰定帝一》中称“至元十三年十月二十九日,帝生于晋邸。”至元十三年是1276年,但在《元史·泰定帝二》中又说“庚午,帝崩,寿三十六”,按这个说法他应该是1293年(至元三十年)出生的。很可能作者误把“三十”写成了“十三”。泰定帝“生于晋邸”,而1292年甘麻剌被封为晉王,而且1328年他的长子阿剌吉八當時只有8岁,所以泰定帝应该是生于1293年。他的父亲甘麻剌是元世祖太子真金的長子,1292年被封为晉王,出镇嶺北。1302年甘麻剌死后也孙铁木儿袭晋王位。

至治三年(1323年)三月也孙铁木儿在元英宗附近的亲信向他告密说英宗将对也孙铁木儿不利。同年八月二日,也孙铁木儿获得英宗将被刺杀、自己将被迎立为皇帝的消息。

至治三年八月初四(1323年9月4日),铁木迭儿的义子铁失趁着元英宗从上都避暑结束南返大都途中,在上都以南15公里的地方南坡的刺杀了元英宗及右丞相拜住等人。史称南坡之变。

元英宗被刺后也孙铁木儿果然被擁立为皇帝,至治三年九月初四日(1323年10月4日),也孙铁木儿在漠北草原的龙居河(今克鲁伦河)河畔登基称帝。虽然也孙铁木儿是知情人,但他登基后就下令将刺杀英宗的人都處死了。

至治三年十一月十三日(1323年12月11日),泰定帝到达大都。1323年12月17日,泰定帝在大都大明殿接受诸王和百官朝贺。

至治三年十二月十一日(1324年1月7日),泰定帝追尊其父亲甘麻剌為皇帝,为甘麻剌上庙号显宗,汉文谥号光圣仁孝皇帝;追尊其母亲普顏怯里迷失为皇后,为普顏怯里迷失上谥号宣懿淑圣皇后。

泰定元年三月二十日(1324年4月14日),泰定帝立八八罕氏为皇后,立阿速吉八为太子。

从1325年开始,泰定帝因国库收入少于支出,开始减少国家支出。七月,他下令不允许汉人收藏和携带兵器。

泰定二年九月初一日(1325年10月8日),泰定帝改革全国的行政区划,将全国划分为18个道,分别为:两浙道、江东道、江西道、福建道、江南道、湖广道、河南道、江北道、燕南道、山东道、河东道、陕西道、山北道、辽东道、云南道、甘肃道、四川道、京畿道。

泰定帝还下达了一系列命令禁止和尚和道士购买民间的土地,克制僧院的过分富有。

在泰定帝统治期间,广西、四川、湖南、云南等少数民族地区经常爆发反抗元朝统治的暴乱,泰定帝一般使用软硬兼施的手段来平息这些暴乱。但从整体来说整个国家基本上比较安宁。

致和元年七月初十日(1328年8月15日),元泰定帝在上都病逝,享年36岁。

元泰定帝七月去世后,九月,他的儿子元天顺帝在上都登基,改元天顺,九月十三日,元武宗之子元文宗在大都登基,改元天历,双方交战一个月,最终以元文宗获胜告终,元天顺帝失败后下落不明,不知所终。

也孙铁木儿无庙号和谥号,故以年号史称为泰定帝。

明朝宋濂等官修正史《元史》的評價是:“泰定之世,灾异数见,君臣之间,亦未见其引咎责躬之实,然能知守祖宗之法以行,天下无事,号称治平,兹其所以为足称也。”

清朝史学家邵远平《元史类编》的評價是:“册曰:长子世嫡,嗣统允宜;武仁先立,泽承人思;忽焉不世,电灭云移;或曰南坡,其蛮与知;故史具在,其又谁欺?”

清朝史学家毕沅《续资治通鉴》的評價是:“帝在位,灾异数见,然能守祖宗之法,天下号称治平。”

清朝史学家魏源《元史新编》的評價是:“一代统绪之传,有正统即有公论,岂一时私意所能傎倒磔裂者哉!世祖明孝太子早卒,皇孙成宗立,追谥裕宗。成宗本裕宗第三子,其同母二兄,一为晋王甘麻剌,一为怀王答剌麻八剌,本无嫡庶,而晋邸居长。成宗崩后无嗣,晋王之子泰定帝即可嗣立,乃因仁宗自怀庆入,先靖内难,迎立其兄怀宁王于漠北,是为武宗。所谓先入关者王之,非晋王子不当立而必立怀王子也。及再传至英宗遇弑,晋王复出自漠北入靖内难,讨贼嗣位,是为泰定。与武、仁之事相埒,非武、仁有功宗社,而泰定无功也。泰定践阼,即以和林兵柄授周王使代己任,屡通朝贡。又召怀王自海南入朝京师,锡封藩国,移近江陵,屡赐金币,是泰定于文宗兄弟有德而无怨也。泰定太子册立已五载,父终子继,名正言顺,怀王、周王安得入干大统乎!若谓武、仁当日原有传位周王,嗣及英宗之约,则仁宗实背约在前,可以责仁宗,不可以责泰定也。乃文宗篡立之诏,谓泰定以旁支入继,正统遂偏,甚至诬其与贼臣铁失潜通阴谋,冒干宝位,追毁晋王显宗庙室。乌乎!以讨贼之主,而诬以通贼之罪,是何言哉!若谓武宗二子为人心所归,泰定当舍子而传侄,则何以天历颁诏至关中、至四川、至辽东,皆焚书斩使,起兵拒命,则人心归泰定之子,而不归武宗之子,明如星日。是则燕帖木儿之为逆臣,怀王之为逆立,亦明如星日。固不待鲁桓弑隐夺国,已无所逃于《春秋》之责,况欲宽其罪于中途弑逆之后哉!斯非难定之案,而数百年尚无定论。请断之,以折曲沃桓叔之徒,假托正谊者。”

清朝史学家曾廉《元书》的評價是:“论曰:周太王以国传王季,设季而无后,则泰伯之子孙遂不可以复承周祀乎?美哉晋王之让,而泰定之立,亦不可不畏之正也。上都告变,惜已无及,然大节亦明矣。故诸凶迁官非有他也,仓卒之间,形格势禁,度权力未足以制其命也。荣宠以诱之,俾喜而懈,稍缓须臾,成备而出,而疾雷不及掩耳矣。呜呼!此帝之所以为权,然岂不果哉!至后纪纲弗振,由不纳张珪、宋本之言,而乱是用长也,累受佛戒,亦梁武之俦乎?”

清末民初史学家屠寄《蒙兀儿史记》的評價是:“至元六年,诏称英宗遇害,正统遂偏,于戏!此惠宗一人之私言也。太子真金嫡子三人,泰定之父晋王甘麻剌最长,次则武仁之父答剌麻八剌,又次为成宗。成宗之立,非世祖本意也。向使储闱符玺之归,果足为大统继嗣之证,则当世祖宾天,诸王大会,成宗曷不径遵遗诏,即位梓宫之前,出受群臣之贺。顾乃迟回三月,必得晋王北面愿事之一言,而大策始定,何也?盖成宗以皇孙出抚北军时,既无王号,又未赐印,世祖用玉昔帖木儿之请,濒行仓卒,授以故太子宝,代一时行军印之用而已。非有告庙册立之礼也。晋王不让,成宗不得立。则所谓正统,宜属晋王之子孙。明史臣王祎言:武宗约继世子孙,兄弟相及。而仁宗不守宿诺,传位英宗,仍使武宗二子出居于外。及英宗遇弑,而明宗在北,文宗在南。嗣晋王于世祖为嫡长曾孙,则求所当立,舍嗣晋王谁归?旧传英宗之弑,晋邸与闻,考之宝录,不得其证。传闻之缪,殊不足信。邵阳魏氏源亦言:成宗无嗣,大统当归,泰定徒以仁宗自怀先入,靖内难而迎立武宗,所谓先入关者王之,非晋王子不当立,而必立答剌麻八剌子也。泰定能讨贼,胜于武、仁杀疑似之宗亲,非武、仁有功社稷,而泰定无功也。泰定践阼,即归周王之妃八不沙于漠北,召图帖睦尔汗于海南,既至京师,厚加赐予。封为怀王,妻以主女。初镇建康,六朝都会;及移江陵,益据上游。泰定之于怀王,有德而无怨也。阿速吉八太子册立已五载,父终子继,名正言顺,大统所在,孰得干之?若谓武宗当日原有传位周王以及英宗之约,则仁宗实背约在前,可以责仁宗,不可以责泰定也。若谓武宗二子,人心所归,泰定当舍子而传侄,何以山后、辽东、关陇、滇、蜀,先后为上都起兵,即河南、湖广,犹必执杀省官,易置郡县长吏,强之而后从。当日讴歌讼狱,不之武宗之子,而之泰定之子,明矣。然则燕帖木儿之为逆臣,怀王之为篡立。不待鲁桓弑隐,已无所逃于《春秋》之诛。况可宽其罪于旺兀察都推刃天伦之后哉!斯狱县之六百年,请断之,以折曲沃桓叔之徒,假托名义者。”(至元六年为1338年,此处至元为元惠宗年号。)

民国柯劭忞官修正史《新元史》的評價是:“孔子称叔孙昭子之不劳。泰定帝讨铁失等弑君之罪,虽叔孙昭子何以尚之。文宗篡立,欲厌天下之人心,诬蔑之辞无所不至。惜乎后世之君子,不引孔子之言,以论定其事也。”

\subsection{泰定}

\begin{longtable}{|>{\centering\scriptsize}m{2em}|>{\centering\scriptsize}m{1.3em}|>{\centering}m{8.8em}|}
  % \caption{秦王政}\
  \toprule
  \SimHei \normalsize 年数 & \SimHei \scriptsize 公元 & \SimHei 大事件 \tabularnewline
  % \midrule
  \endfirsthead
  \toprule
  \SimHei \normalsize 年数 & \SimHei \scriptsize 公元 & \SimHei 大事件 \tabularnewline
  \midrule
  \endhead
  \midrule
  元年 & 1324 & \tabularnewline\hline
  二年 & 1325 & \tabularnewline\hline
  三年 & 1326 & \tabularnewline\hline
  四年 & 1327 & \tabularnewline\hline
  五年 & 1328 & \tabularnewline
  \bottomrule
\end{longtable}

\subsection{致和}

\begin{longtable}{|>{\centering\scriptsize}m{2em}|>{\centering\scriptsize}m{1.3em}|>{\centering}m{8.8em}|}
  % \caption{秦王政}\
  \toprule
  \SimHei \normalsize 年数 & \SimHei \scriptsize 公元 & \SimHei 大事件 \tabularnewline
  % \midrule
  \endfirsthead
  \toprule
  \SimHei \normalsize 年数 & \SimHei \scriptsize 公元 & \SimHei 大事件 \tabularnewline
  \midrule
  \endhead
  \midrule
  元年 & 1328 & \tabularnewline
  \bottomrule
\end{longtable}


%%% Local Variables:
%%% mode: latex
%%% TeX-engine: xetex
%%% TeX-master: "../Main"
%%% End:
