%% -*- coding: utf-8 -*-
%% Time-stamp: <Chen Wang: 2019-12-26 14:54:31>

\section{明宗\tiny(1329)}

\subsection{生平}

元明宗和世㻋,是元朝第九位皇帝,蒙古帝国第十三位大汗,1329年2月27日至1329年8月30日在位,在位185天。元武宗長子。清代乾隆晚期乾隆帝命改譯遼、金、元三史中的音譯專名,改譯和實拉,今日學界已無人使用。

他去世后,谥号翼獻景孝皇帝,庙号明宗,蒙古语称忽都篤皇帝。

1340年10月25日,元惠宗為元明宗上汉语尊号順天立道睿文智武大聖孝皇帝。

根据《元史》,天曆二年正月丙戌(儒略曆1329年2月27日),和世琜在漠北草原的和宁之北即位,继续使用年号“天曆”,是为元明宗,1329年4月3日,元文宗图帖睦尔派人将皇帝宝玺献给明宗,正式禪讓帝位,5月15日,元明宗正式立图帖睦尔为皇太子,8月16日,图帖睦尔受皇太子宝,8月25日,元明宗抵达元武宗时建为中都的王忽察都,8月26日,皇太子图帖睦尔入见,两兄弟会面,元明宗宴请皇太子及诸王、大臣于行殿。

天曆二年八月六日(1329年8月30日),元明宗和世㻋被燕帖木儿毒死,明宗去世时享年30岁。

1329年9月8日,燕帖木儿重新拥戴元文宗復辟,因为1329年的年号是天曆,史称天曆之变。

天曆二年十月十三日(1329年11月4日),元文宗为兄長和世㻋上谥号翼獻景孝皇帝,庙号明宗,蒙古文称忽都篤皇帝。

元明宗的两个儿子元宁宗懿璘质班和元惠宗妥懽帖睦尔在1332年9月2日元文宗去世后相继登基称帝。

至元六年十月四日(1340年10月25日),元惠宗给元明宗上尊号順天立道睿文智武大聖孝皇帝。

清朝史学家邵远平《元史类编》的評價是:“册曰:艰艰备尝,人望所属;何嫌何疑,推肝置腹;人心不同,天命反覆;论定千秋,此直彼曲。”

清朝史学家曾廉《元书》的評價是:“论曰:昔曹子臧、吴季札,贤者也。其君国子民也宜哉!然而义不受者,非独远情,亦知负飞及光之不厌,其欲将无以善其后也,闇哉明宗!焉有人披衮执玉,穆穆然位乎天位而肯北面俯首为人臣者乎?呜呼!此唐明皇不敢以望肃宗,父子且然,况兄弟哉!文宗盖惧北陲,复有海都、笃哇之流,托名拥戴,其言也顺而为患也。深抑亦私心,窃望周王之效法晋邸也。己则非夷,而以齐期人。不亦难乎?悠悠南行,甘咽其饵,悲夫!”

清末民初史学家屠寄《蒙兀儿史记》的評價是:“和世㻋汗年未弱冠,远逊金山,耕牧十有三年。所谓旧劳于外,知民情伪者也。观其论台纲,谕百司,斤斤于先世成宪,是殆有心救弊者乎?然以此言论风采,自曝于风尘道路之间,致令傲弟权相闻而生心,遂有旺兀察都之变。《易》曰:‘君不密,则失臣。’此之谓矣。怀抱盛意,未见设施,惜哉!”

民国官修正史《新元史》柯劭忞的評價是:“燕铁木儿立文宗,文宗固让于兄,犹仁宗之奉武宗也。明宗之弑,盖出于燕铁木儿,非文宗之本意。然与闻乎弑,是亦文宗弑之而已。”

\subsection{天历}

\begin{longtable}{|>{\centering\scriptsize}m{2em}|>{\centering\scriptsize}m{1.3em}|>{\centering}m{8.8em}|}
  % \caption{秦王政}\
  \toprule
  \SimHei \normalsize 年数 & \SimHei \scriptsize 公元 & \SimHei 大事件 \tabularnewline
  % \midrule
  \endfirsthead
  \toprule
  \SimHei \normalsize 年数 & \SimHei \scriptsize 公元 & \SimHei 大事件 \tabularnewline
  \midrule
  \endhead
  \midrule
  二年 & 1329 & \tabularnewline
  \bottomrule
\end{longtable}


%%% Local Variables:
%%% mode: latex
%%% TeX-engine: xetex
%%% TeX-master: "../Main"
%%% End:
