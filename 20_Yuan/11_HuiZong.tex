%% -*- coding: utf-8 -*-
%% Time-stamp: <Chen Wang: 2019-10-18 16:19:55>

\section{惠宗\tiny(1333-1370)}

元惠宗妥懽貼睦爾,清刊《元史》、清修《續資治通鑑》改譯托歡特穆爾,元朝廟號為惠宗,蒙古語稱號烏哈噶圖汗,明朝諡號為順皇帝,又稱至正帝,庚申帝,庚申君,元朝第十一位皇帝,蒙古帝国第十五位大汗,他是元朝北逃前的最後一位皇帝,之後以他為首的北元繼續與明朝對峙。在位时间是从1333年7月19日至1370年5月27日,在位37年。

延祐七年四月十七日(1320年5月25日),生于北方草原,生父是元明宗,生母是迈来迪。

至順元年(1330年)四月,元明宗皇后八不沙被杀,妥懽帖睦尔被驱逐,首先被驱逐到高丽大青岛,後来到湖广等处行中书省静江(今桂林)。

至順三年十一月二十六日(1332年12月14日),元宁宗逝世,太后弘吉剌·卜答失里坚持弃子立侄,下令立妥懽帖睦尔为皇帝,受到左丞相欽察人燕帖木儿反对,在燕帖阻攔下,一直未能回大都即位。

至顺四年(1333年)五月左丞相燕帖木儿病死。至順四年农历六月初八(1333年7月19日)妥懽帖睦尔最终得以继位。

1333年六月,惠宗登基后不久,任命伯颜为太师、中书右丞相。元統三年(1335年)六月,欽察人燕帖木儿的儿子唐其势阴谋推翻元惠宗,另立文宗义子答剌海。右丞相伯颜粉碎唐其势叛乱。粉碎唐其势叛乱后,伯颜的势力大增,把持着朝政,史稱伯颜专权。1335年七月,惠宗被迫下诏罢除左丞相,专命伯颜为中书右丞相,伯颜开始专权,他甚至一度不把元惠宗放在眼裏。伯颜采取排挤汉人的政策,如禁止汉人参政、取消科举、不许汉人学蒙古语等,这些做法加深了汉蒙两族之间的不和,也使得元惠宗更加不满。至元六年(1340年)二月,在伯颜之侄脱脱的帮助下,元惠宗罢免并流放伯颜。至元六年(1340年)六月,废黜文宗子燕帖古思的太子地位并将之流放。至元六年(1340年)七月,燕帖古思在放逐途中被杀,从而消除了夺权隐患,控制了政局。伯颜的一系列排挤汉人的政策也全部被废除。至正十一年(1351年)四月初四日,诏命贾鲁为工部尚书、充总治河防使。征发民工15万,军士2万,兴役治黄河。贾鲁回朝,向顺帝上《河平图》。至正十三年(1353年)五月,身为中书左丞的贾鲁,突然病卒,享年57岁。


至正十一年(1351年),徐壽輝起兵,建天完朝,陳友諒投效其將領倪文俊麾下。至正十五年(1355年)2月,刘福通迎韓山童之子韓林兒為皇帝,稱小明王,国号宋,定都亳州,建元龙凤。他为枢密院平章,旋改任丞相。至正十七年(1357年)二月,龙凤将领毛贵浮海破胶州;三月,陷莱州,据益都。龙凤将领李武、崔德绕过潼关,夺七盘,进据蓝田,直趋奉元。六月,刘福通自帅一军攻汴梁,余军分三路北伐:關鐸、潘誠、沙劉二等攻怀庆,深入晋冀,白不信、大刀敖等西取关中;毛贵自山东北上。至正十七年九月,陳友諒襲殺反徐壽輝的倪文俊,自稱勤王,自稱宣慰使,起兵攻下江西諸路,連克江西、安徽、福建等地。至正十八年(1358年)二月,毛贵攻占济南。三月,毛贵北攻蓟州、漷州,进逼枣林,距大都一百二十里,战失利,退回济南。五月,刘福通攻破汴梁,自安丰(今安徽寿县)迎韩林儿,定为国都。龙凤政权中央分设六部、御史等诸官属;在山东、江南等地分设行省。至正十九年(1359年)正月,關鐸、潘誠、沙劉二东攻全宁,焚鲁王府宫阙。再破元上都,焚之。进破辽阳,入高丽境。八月,汴梁被察罕帖木儿攻破,刘福通与韩林儿退据安丰(今安徽寿县)。至正十九年,陳友諒殺天完將領趙普勝,挾徐壽輝,遷都江州(今江西九江),自立為漢王。至正二十年(1360年),陳友諒攻陷太平路,命死士刺殺徐壽輝於至太平路采石(今安徽省馬鞍山市)五通廟,隨即登基稱帝,國號漢,改元大義,以鄒普勝為太師,張必先為丞相。隨即與張士誠合攻朱元璋。朱元璋金陵應天府被圍,只好遣胡大海進攻信州,迫陳友諒回師救援,朱元璋一面離間張士誠,張按兵不動。陳朱雙方在金陵城西北的龍灣展開惡戰,不巧江水退潮,百艘巨艦擱淺,陳友諒大敗,敗走江州(今九江)。至正二十三年(1363年)二月,張士誠遣呂珍圍攻安豐,杀劉福通。朱元璋前往救援,打败呂珍,迎韓林兒到滁州。至正二十三年,陳友諒率六十萬水軍進攻朱元璋,朱以水軍二十萬親征,是為「鄱陽湖之戰」。陳友諒自恃巨艦出戰,採用炮攻,差點捕獲朱元璋。隨後,朱元璋採納郭興的建議,利用東北風而改用火攻,致使陳友諒部隊大量受損。之後朱元璋利用鄱陽湖水位降低便於小舟活動,改為分兵水路圍攻陳友諒。陳友諒突圍時起霧,陳從船艙中探頭出看,竟中流箭而死,漢軍潰敗。隨後朱元璋圍攻武昌,並盡佔湖北各地。陳友諒死後,張定邊等人在武昌立陳友諒次子陳理登基为帝,改元德壽。至正二十四年(1364年),朱元璋西吴军廖永忠部兵臨武昌城下,陳理出降。至正二十八年(1368年9月14日),明朝军队从元大都齐化门外攻城而入,蒙古退出中原,元朝对長城以南的區域統治结束。

至正二十八年闰七月二十八日(1368年9月10日),徐达率领的军队逼近大都,元惠宗夜半开大都的健德门北奔,率太子愛猷識理答臘、后妃、臣僚等逃离大都。八月初二日(1368年9月14日),明朝军队从大都的齐化门攻城而入,元朝正式退出中原,回到北方草原。八月初四日(1368年9月16日),元惠宗到达上都。

至正二十九年六月十三日(1369年7月16日),明军逼近上都,元惠宗离开上都,当天到达应昌。六月十七日(1369年7月20日),明军将领常遇春攻克上都。

元惠宗在上都和应昌那里曾两次组织兵力试图收复大都,但都被明朝军队击败。

1370年5月27日(庚戌狗年五月二日,即元惠宗至正三十年、明太祖洪武三年),惠宗因痢疾崩於应昌,享年50岁。死後得庙号惠宗,蒙古語稱“烏哈噶圖汗”。明太祖認為他“順天應人”,上諡號順皇帝。

皇太子愛猷識理答臘在应昌繼承了皇位,史稱元昭宗,并于1371年改元宣光。至正三十年五月十六日(1370年6月10日),明军将领李文忠攻克应昌,元昭宗逃往和林延续北元,繼續和明朝对抗。

至元元年(1335年)十一月,专权的右丞相伯颜使得惠宗下诏停止科举取士,因为伯颜专权到至元六年(1340年)二月,原本定于至元二年(1336年)和至元五年(1339年)在大都举行的两次科举取士都被迫停止,史称“至元废科”。

1340年二月,伯颜去职,脱脱被惠宗任命为知枢密院事;1340年十月,惠宗任命脱脱为右丞相。

至元六年(1340年)十二月,惠宗下诏恢复科举取士。至正元年(1341年)八月,全国范围内恢复乡试,至正二年(1342年),会试和殿试相继在大都举行,史称“至正复科”。

此后科举取士三年一次,至正二十六年(1366年),最后一次在大都举行会试和殿试。1368年八月元朝退回草原后,不再有科举取士。

至元二年(1336年),在增订元仁宗年间的监察法规《风宪宏纲》的基础上,将有关御史台的典章制度汇编为《宪台通纪》。

至元四年(1338年)三月,命中书平章政事阿吉剌根据《大元通制》编定条格,至元六年(1340年)七月,命翰林学士承旨腆哈、奎章阁学士崾崾等删修《大元通制》,至正五年(1345年)十一月书成,右丞相阿鲁图等入奏,请元惠宗赐名《至正条格》。这部法典共有2909条,其中包括制诏150条、条格1700条、断例1059条。

至正六年四月五日(1346年4月26日),将《至正条格》中的条格、断例两部分(2759条)颁行天下。1368年9月14日元朝退回草原后,《至正条格》逐渐失传。

2002年在韩国庆州发现元刊残本《至正条格》,包括条格12卷、断例近13卷,以及断例全部30卷的目录。其中,条格存374条,断例存426条,总数共计为800条。

1340年-1344年,脱脱第一次为相期间,以及1344年-1349年,元惠宗亲政期间,采取了一系列改革措施,以革新政治,缓和社会矛盾,史称“至正新政”。

1344年五月,脱脱因病辞职,1344年-1349年,由元惠宗亲政,至正六年(1346年),颁行法典《至正条格》,以完善法制;颁布举荐守令法,以加强廉政;下令举荐逸隐之士,以选拔人才。

脱脱第一次为相期间和元惠宗亲政前期,政治比较清明,社会矛盾有所缓和,但未能从根本上解决积弊已久的社会问题。

后期怠于政事,荒於游宴,學“行房中运气之术”,有匠材,能製金人玉女自动报时器。又造宫漏,“其精巧绝出,人谓前代所罕有”,史稱「魯班天子」。

至正十年(1350年)國内发生通货膨胀,加上为了治水(當時因黃河水災頻繁,元惠宗下令右丞相脫脫遏黄河回故道以整治水患)加重了徭役,导致至正十一年(1351年)红巾军起事,红巾军一度在1359年1月8日攻入上都,焚毁宫阙,留七日后方才离去。虽然在元朝名将察罕帖木儿的努力下,1362年元军获得很大战果,但由于叛軍的势力已经很大,朝廷内部又发生皇帝和皇太子愛猷識理答臘(即后来即位的元昭宗)两派之间的明争暗鬥,因此元惠宗无法有效地控制政局,而在外的各行省将领各行其是,不听中央统一指挥。这一切给朱元璋提供了巩固其地位的机会。

至正二十八年正月初四日(1368年1月23日)朱元璋在應天府建立明朝,統一中國南方,責令北伐,徐达率领的军队逼近大都,闰七月二十八日(1368年9月10日),元惠宗夜半开大都的健德门北奔,率太子愛猷識理答臘、后妃、臣僚等逃离大都,八月初二日(1368年9月14日),明朝军队从大都的齐化门攻城而入,元朝正式退出中原,回到蒙古草原。

八月初四日(1368年9月16日),元惠宗到达上都。至正二十九年六月十三日(1369年7月16日),明军逼近上都,元惠宗离开上都,当天到达应昌。六月十七日(1369年7月20日),明军将领常遇春攻克上都。

元惠宗在上都和应昌曾两次组织兵力试图收复大都,但都被明朝军队击败。至正三十年(洪武三年)四月二日(1370年5月27日)元惠宗因痢疾在应昌去世,享年五十一岁。

皇太子愛猷識理答臘在应昌繼承了皇位,是为元昭宗,并于1371年改元宣光。至正三十年五月十六日(1370年6月10日),明军将领李文忠攻克应昌,元昭宗逃往和林,继续延续元朝,和明朝对抗。

元惠宗即位前,朝廷太史的看法:“不可立,立则天下乱。”

清朝史学家邵远平《元史类编》的評價是:“册曰:绝人巧智,惟事荒恣;纲纪懈弛,用殄厥世;稗史所称,非明宗嗣;附会诏书,事近暧昧。”

清朝史学家曾廉《元书》的評價是:“论曰:世有畏其子之悍戾而柔之以秘密佛法者乎?昔隋炀父子相忌,至死而俱不悟,可哀也。宠妾骄子,目羸豕蹢躅之戒而忘为潜龙,至于屠戮将相,擅兴兵戎,脱脱、太平因是陨身丧家,而激孛罗、扩廓之辟,如人之有肢体,而构之伤残,雀彀未成而社稷墟矣。然以秃鲁帖木儿之言,杀合麻、雪雪,而曾不察废立之谋之出自宫闱也。则帝亦谚所谓莫知苗硕者也。犹复徘徊塞下,考终沙漠,非不幸矣。”

清末民初史学家屠寄《蒙兀儿史记》的評價是:“先是,汗居应昌,常郁郁不乐,作歌曰:‘失我大都兮,冬无宁处;失我上都兮,夏无以逭暑。惟予狂惑兮,招此大侮;堕坏先业兮,获罪二祖。死而加我恶谥兮,予妥懽帖睦尔奚辞以拒?’歌声甚哀。继之以泣。至今蒙兀人尚能按之。汗性好技巧,尝于内苑造龙船,自制模型,委供奉少监塔思不花监匠仿作。船成,首尾长百二十尺,广二十尺。前瓦簾穿廊、两暖阁,后吾殿楼子,龙身及殿宇皆五彩金涂,行时龙首口、眼、两爪及尾胥动。水手二十四人,黄袜额、服紫衫,束金荔枝带。于船之两舷各手一篙,自后宫至前宫山下海子内,往来游戏。又自制宫漏,约高六七尺,广半之,造木为匮,藏诸壼其中,运水上下,匮上设西方三圣殿,匮腰立玉女,奉时刻筹,时至辄浮水而上,左右列二金甲神,一县钟,一县钲。夜则神人自能按更而击,无分豪差,当钟钲之鸣,狮凤在侧者皆翔舞。匮之东西有日月宫,飞仙六人立宫前,遇子午时,飞仙自能耦进,度仙桥达三圣殿,已复退立如前。精巧绝伦,前代未有。汗冲龄践阼,颇能尊师重道,自诛伯颜,躬裁大政,一时有中主之目。久之昵比群小,信奉淫僧,肆意荒嬉,万几怠废,宫庭亵狎,秽德章间。遂令悍妻干外政之柄,骄子生内禅之心,奸相肆蠹国之谋,强藩成跋扈之势。九重孤立,威福下移,是非不明,赏罚不公,水旱频仍,盗贼滋起。人心既去,天命随之矣。”

民国官修正史《新元史》柯劭忞的評價是:“惠宗自以新意制宫漏,奇妙为前所未有,又晓天文灾异。至元二十二年,自气起虚后,扫太微垣,台官奏山东应大水。帝曰:‘不然,山东必陨一良将。’未几,察罕帖木儿果为田丰所杀。其精于推验如此。乃享国三十余年。帝淫湎于上,奸人植党于下,戕害忠良,隳其成功。迨盗贼四起,又专务姑息之政,縻以官爵,豢以土地,犹为虎傅翼,恣其抟噬。孟子有言:安其危,而利其灾,乐其所以亡者。呜呼,其帝之渭欤!然北走应昌,获保余年;视宋之徽、钦,辽之天祚,犹为厚幸焉。”

庚申外史认为:他不嗜酒,善画,又善观天象。性格顽劣。当时有人认为他昏愚、优柔不断。可是庚申外史认为,他并非如此,而是一个阴毒的人,幼年“头发常生虮虱,使民妪捕之,告妪曰:「是虽血食于我,我不忍杀之,不如以纸裹之,悬于屋檐下,冷杀可也。」”其问甲则曰:「乙与汝甚不许也。」问乙则曰:「甲与汝甚不许也。」及甲之力足以去乙,则谓甲曰:「乙尝欲图汝,汝何不去之也?」乙之力足以去甲,则亦如是焉。故其大臣死,则曰:「此权臣杀我也。」小民死,则曰:「此割据弄兵杀我也。」人虽至于死,未尝有归怨之者。看似善良,实际上善于挑拨他人矛盾,借刀杀人。

\subsection{至统}

\begin{longtable}{|>{\centering\scriptsize}m{2em}|>{\centering\scriptsize}m{1.3em}|>{\centering}m{8.8em}|}
  % \caption{秦王政}\
  \toprule
  \SimHei \normalsize 年数 & \SimHei \scriptsize 公元 & \SimHei 大事件 \tabularnewline
  % \midrule
  \endfirsthead
  \toprule
  \SimHei \normalsize 年数 & \SimHei \scriptsize 公元 & \SimHei 大事件 \tabularnewline
  \midrule
  \endhead
  \midrule
  元年 & 1333 & \tabularnewline\hline
  二年 & 1334 & \tabularnewline\hline
  三年 & 1335 & \tabularnewline
  \bottomrule
\end{longtable}

\subsection{至元}

\begin{longtable}{|>{\centering\scriptsize}m{2em}|>{\centering\scriptsize}m{1.3em}|>{\centering}m{8.8em}|}
  % \caption{秦王政}\
  \toprule
  \SimHei \normalsize 年数 & \SimHei \scriptsize 公元 & \SimHei 大事件 \tabularnewline
  % \midrule
  \endfirsthead
  \toprule
  \SimHei \normalsize 年数 & \SimHei \scriptsize 公元 & \SimHei 大事件 \tabularnewline
  \midrule
  \endhead
  \midrule
  元年 & 1335 & \tabularnewline\hline
  二年 & 1336 & \tabularnewline\hline
  三年 & 1337 & \tabularnewline\hline
  四年 & 1338 & \tabularnewline\hline
  五年 & 1339 & \tabularnewline\hline
  六年 & 1340 & \tabularnewline
  \bottomrule
\end{longtable}

\subsection{至正}

\begin{longtable}{|>{\centering\scriptsize}m{2em}|>{\centering\scriptsize}m{1.3em}|>{\centering}m{8.8em}|}
  % \caption{秦王政}\
  \toprule
  \SimHei \normalsize 年数 & \SimHei \scriptsize 公元 & \SimHei 大事件 \tabularnewline
  % \midrule
  \endfirsthead
  \toprule
  \SimHei \normalsize 年数 & \SimHei \scriptsize 公元 & \SimHei 大事件 \tabularnewline
  \midrule
  \endhead
  \midrule
  元年 & 1341 & \tabularnewline\hline
  二年 & 1342 & \tabularnewline\hline
  三年 & 1343 & \tabularnewline\hline
  四年 & 1344 & \tabularnewline\hline
  五年 & 1345 & \tabularnewline\hline
  六年 & 1346 & \tabularnewline\hline
  七年 & 1347 & \tabularnewline\hline
  八年 & 1348 & \tabularnewline\hline
  九年 & 1349 & \tabularnewline\hline
  十年 & 1350 & \tabularnewline\hline
  十一年 & 1351 & \tabularnewline\hline
  十二年 & 1352 & \tabularnewline\hline
  十三年 & 1353 & \tabularnewline\hline
  十四年 & 1354 & \tabularnewline\hline
  十五年 & 1355 & \tabularnewline\hline
  十六年 & 1356 & \tabularnewline\hline
  十七年 & 1357 & \tabularnewline\hline
  十八年 & 1358 & \tabularnewline\hline
  十九年 & 1359 & \tabularnewline\hline
  二十年 & 1360 & \tabularnewline\hline
  二一年 & 1361 & \tabularnewline\hline
  二二年 & 1362 & \tabularnewline\hline
  二三年 & 1363 & \tabularnewline\hline
  二四年 & 1364 & \tabularnewline\hline
  二五年 & 1365 & \tabularnewline\hline
  二六年 & 1366 & \tabularnewline\hline
  二七年 & 1367 & \tabularnewline\hline
  二八年 & 1368 & \tabularnewline\hline
  二九年 & 1369 & \tabularnewline\hline
  三十年 & 1370 & \tabularnewline
  \bottomrule
\end{longtable}


%%% Local Variables:
%%% mode: latex
%%% TeX-engine: xetex
%%% TeX-master: "../Main"
%%% End:
