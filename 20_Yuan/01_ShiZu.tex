%% -*- coding: utf-8 -*-
%% Time-stamp: <Chen Wang: 2019-12-26 14:53:33>

\section{世祖\tiny(1260-1294)}

\subsection{生平}

元世祖忽必烈,清代乾隆晚期乾隆帝命改譯为呼必赉。孛儿只斤氏,為父親拖雷的第四子,母親唆鲁禾帖尼的第二子,蒙古帝国大汗,元王朝的建立者。

1260年5月5日在自己的弟弟旭烈兀的支持和封地属臣的拥立下,自立为大蒙古国大汗,称大蒙古国皇帝。1271年12月18日,忽必烈改国号为“大元”,建立元朝,成为元朝首任皇帝。忽必烈于1260年5月5日至1276年2月4日自立为汗期间实际统治中国北方及蒙古高原地区属于蒙古大汗的直辖领地,于1271年12月18日至1294年2月18日作为元朝皇帝统治中国,前后在位34年,作为全中国皇帝在位18年。

1276年2月4日,元军攻入南宋行都临安,宋恭帝奉上传国玉玺和降表,元朝成为全国性政权,但南宋遗臣建立小朝廷继续抗元。1279年3月19日,南宋海上政权残余的最后一支抵抗力量被消灭,元朝统一全中国。

1276年2月4日,宋恭帝在降表中为忽必烈上尊号大元仁明神武皇帝。1284年1月24日,群臣为忽必烈上尊号宪天述道仁文义武大光孝皇帝。

去世后,获諡號聖德神功文武皇帝,廟號世祖,蒙古語尊號薛禪皇帝。

成吉思汗十年八月二十八日(1215年9月23日),忽必烈生于漠北草原。忽必烈是成吉思汗第四子拖雷與正妻唆鲁禾帖尼所生的嫡次子(蒙哥是嫡长子,旭烈兀是嫡三子,阿里不哥是嫡四子)。忽必烈长大后,“仁明英睿,事太后至孝,尤善抚下。”忽必烈年少有大志、重视汉地的治理,早在1244年,年轻的忽必烈便招揽了搜罗了各方的文人、儒生、旧臣等,形成了一个属于自己的幕僚团

1251年7月1日(农历辛亥年六月十一日),忽必烈長兄蒙哥经忽里台选举成为大蒙古国大汗(于1264年被忽必烈追尊为元宪宗),即位后不久即任命忽必烈負責總領漠南漢地事務。忽必烈设置金莲川幕府,并在这段时间内任用了大批漢族幕僚和儒士,如劉秉忠、許衡、姚樞、郝经、张文谦、窦默、趙璧等等,并提出了“行汉法”的主张。儒士元好问和张德辉还请求忽必烈接受“儒教大宗师”的称号,忽必烈悦而受之。忽必烈尊崇儒学,“圣度优宏,开白炳烺,好儒术,喜衣冠,崇礼让。”

1252年六月,忽必烈前往草原觐见蒙哥汗,奉命率军征云南地区的大理国,为继续进攻南宋作跳板。1253年八月,忽必烈率军从陕西出发,于1254年1月2日(农历十二月十二日)攻克大理城,國王段兴智投降,大理国灭,云南地区并入大蒙古国版图。1256年,段兴智前往漠北和林皇宫觐见蒙哥,被蒙哥任命為大理總管,子孙世襲。从1254年忽必烈奉蒙哥之命灭大理国,到1382年驻守云南的元朝梁王把匝剌瓦尔密兵败自杀、大理总管段世战败归降明军,蒙古族建立的政权统治云南地区长达128年。

1256年夏天,蒙哥以南宋扣押蒙古使者为理由,對南宋宣戰,并布置了三路大军,亲自率领西路军,以忽必烈为中路军统帅。忽必烈率军抵达河南汝南,继续向南宋进发,并派命杨惟中、郝经宣抚江淮。1259年9月3日(农历八月十五日),忽必烈统领中路军渡过淮河,攻入南宋境内,随后一路向南,在湖北开辟新的战场,进攻长江中游的鄂州。

1259年8月11日,蒙哥在四川合州钓鱼山病逝。1259年9月19日,在四川的忽必烈异母弟末哥派来的使者向忽必烈宣布蒙哥去世的消息,并请忽必烈北归参与忽里台大会,以便争取汗位继承权。忽必烈则认为“吾奉命南来,岂可无功遽还?”于是进攻南宋,并多次获胜,后来,忽必烈的正妻察必派使者密报,阿里不哥已经派阿蓝答兒在开平附近调兵,脱里赤在燕京附近征集民兵,催促忽必烈早日北还。1259年11月17日,儒臣郝经上《班师议》,陈述必须立即退兵的理由,坚定了忽必烈退兵北返的决心。

忽必烈声称要进攻南宋首都临安,留大将继续对鄂州的围攻,增加对南宋的军事压力,元宪宗九年闰十一月二日(1259年12月17日),南宋丞相贾似道派使者请和,约定南宋割地求和,并且送岁币,忽必烈于是在当日撤兵北返,元宪宗九年闰十一月二十日(1260年1月4日),忽必烈率军抵达燕京(今北京市),解散了脱里赤征集的民兵,“民心大悦”。忽必烈率军在燕京近郊驻扎,度过整个冬天,并积极和诸王联络,准备在1260年春天召开庫力台大會,举行登基大典。

庚申年三月二十四日(1260年5月5日),忽必烈在部分宗王和大臣擁立下于自己的封地开平(后称上都,今内蒙古多伦县北石别苏木)自立为“大蒙古国皇帝”(即蒙古帝国大汗的汉语称谓),庚申年四月四日(1260年5月15日),忽必烈发布称帝的即位诏书《皇帝登宝位诏》,在诏书中,他自称为“朕”,称他的哥哥元宪宗蒙哥(1251—1259年在位)为“先皇”。

中统元年五月十九日(1260年6月29日),忽必烈发布《中统建元诏》,正式建年号“中统”。

庚申年(1260年)农历四月,其弟阿里不哥在哈拉和林城西按坦河被部分宗王和大臣拥立為大蒙古国大汗。幼弟阿里不哥與忽必烈為此發動戰爭爭奪汗位,双方战争时断时续,一共持续了四年之久。忽必烈于庚申年三月二十四日(1260年5月5日)自立为汗,又称汉文的“皇帝”,以招揽汉族知识分子归心,一部分汉族知识分子果然对此表示赞许,赞美忽必烈“既以正立,一时豪杰云从景附,全制本国,奄有中夏,挟辅辽右、白霫、乐浪、玄菟、秽貊、朝鲜,面左燕云、常代,控引西夏、秦陇、吐蕃、云南,则玉烛金瓯,未为玷缺。藩墙不穴,根本强固,倍半于金源,五倍于契丹。”

1260年忽必烈称帝后,控制了漠南草原,以及原金朝和西夏故地,吐蕃,云南,西域东部等地区,对阿里不哥实施经济控制。阿里不哥控制的则是漠北草原和西域西北部地区,面对匮乏的物资最终无以为继。1264年忽必烈最终迫使阿里不哥投降,完全控制蒙古帝国的东部、原本属于大汗直辖领地的大部分地区。阿里不哥归降忽必烈后,忽必烈赦免了他和跟随的诸王,只是处死了他的众多谋臣。。阿里不哥失败后郁郁寡欢,于1266年去世。

1264年8月21日(忽必烈中统五年七月二十八日)阿里不哥投降后,忽必烈实际管辖的政治版圖包括(古今地名对照):中原地区(位于长城以南、秦岭淮河以北)、东北地区(包括整个黑龙江流域)、朝鲜半岛北部、漠南漠北蒙古草原全境(内蒙古和外蒙古地区),西伯利亚南部地区、西域大部分地区(今新疆東部和南部)、吐蕃地区(包括今青海、西藏、四川西部等地)、以及云南地区等地。

至元元年八月十六日(1264年9月7日),忽必烈发布《至元改元诏》,取《易经》“至哉坤元”之义,改“中统五年”为“至元元年”。

庚申年四月初一日(1260年5月12日),忽必烈立中书省,以中书省为最高行政机关,行使宰相职权,以王文統为平章政事,张文谦为中书左丞。

中统四年五月六日(1263年6月13日),忽必烈立枢密院,以枢密院为中央最高军事管理机关,以燕王真金守中书令,兼判枢密院事。

至元元年(1264年),忽必烈立总制院,以总制院统领全国宗教事务并管辖吐蕃地区,以国师八思巴领之。至元二十五年(1288年),尚书省右丞相桑哥认为总制院职责重大,故向忽必烈奏请根据唐朝时期在宣政殿接待吐蕃使者的缘故,改名为宣政院。忽必烈同意,并任命桑哥和脱因为宣政院使。

至元五年七月四日(1268年8月13日),忽必烈立御史台,以御史台为最高监察机关,以右丞相塔察兒为御史大夫,以張雄飛为侍御史。

至元八年十一月十五日(1271年12月18日),因劉秉忠之勸,忽必烈发布《建国号诏》,取《易经》“大哉乾元”之义,建立“大元”国号,其自身亦从大蒙古国皇帝(大汗)变为大元皇帝,元朝正式建立。

元军延续自1268年秋天以来的攻势继续围困襄阳,将襄阳和樊城分隔开来,至元十年正月九日(1273年1月29日),在回回炮的助攻下,元军将领阿里海牙攻克樊城,襄阳彻底成为孤城,元世祖降诏谕襄阳守将吕文焕,阿里海牙亲自到城下劝降吕文焕,保证吕文焕和城中军民的安全,吕文焕犹疑未决。于是阿里海牙和吕文焕折箭为誓担保,吕文焕感泣,至元十年二月二十四日(1273年3月14日),吕文焕和儿子出城投降,归顺元朝。元军经过接近五年時间包围,最终取得襄阳。但是以後的进展则相当顺利。

至元十一年六月十五日(1274年7月20日),忽必烈向行中书省及蒙古、汉军万户千户军士发布问罪于宋的诏书《兴师征南诏》。

至元十一年(1274年)农历七月,忽必烈发布《下江南檄》,派伯颜统率大军讨伐南宋,并告诫伯颜要学习曹彬不杀平江南。伯颜后来取临安,的确做到了忽必烈的要求。

至元十三年正月十八日(1276年2月4日),伯颜率领大军攻陷南宋首都临安(今杭州),宋恭帝派遣使者给元军统帅伯颜奉上传国玉玺和降表,在降表中宋恭帝为忽必烈上尊号大元仁明神武皇帝,元军俘虏5岁的宋恭帝和谢太皇太后,以及南宋宗室和大臣,灭南宋。

至元十三年二月十一日(1276年2月27日),忽必烈发布《归附安民诏》,诏谕江南一带新附府州司县官吏士民军卒人等,稳定江南社会秩序,安定江南士人和百姓之心。

逃离临安的部分大臣陆秀夫等人,先后扶持宋端宗,宋帝昺,建立海上流亡政权,在东南沿海一带继续和元军对抗。至元十六年二月六日(1279年3月19日),在厓山海战中,元军将领张弘范击败南宋海军,南宋丞相陆秀夫挟8岁的小皇帝“宋帝昺”跳海而死,不少後宮和大臣亦相繼跳海自殺。《宋史》記載七日後,十餘萬具屍體浮海。南宋残余的最后一支抵抗力量选择了惨烈的终结,至此,元朝统一海内,结束了中国自安史之乱以来520多年的分裂局面。

1281年3月20日,忽必烈愛妻察必皇后去世。1286年1月5日,皇太子真金去世,连续几年的时间里,爱妻和爱子的先后去世,使忽必烈悲痛不已。此外,忽必烈晚年飽受肥胖與痛風病痛之苦。過度飲酒也损害了他的健康。

至元三十一年正月二十二日(1294年2月18日),忽必烈於大都皇宮紫檀殿去世,享壽七十九岁,在位三十五年。忽必烈葬于起辇谷。

忽必烈去世后,在顾命大臣伯颜等人的拥戴下,其孙铁穆耳于1294年5月10日在上都继承皇位,是为元成宗。1303年,元成宗与西北诸王达成和议,西北的四大汗国重新承认元朝的宗主国地位。

因为1260年忽必烈和阿里不哥争位导致蒙古帝国表面上维持统一,实际上已经分裂,帝国西部為四大汗国实际控制,而帝国东部為忽必烈实际控制。趁着忽必烈和阿里不哥的内战,西北地区的钦察汗国、察合台汗国、窝阔台汗国纷纷自立,此时尚在西亚进行西征的旭烈兀也准备自帝一方,不论忽必烈还是阿里不哥都只得到一部分宗王支持,没有召开成吉思汗四子嫡系后裔参加的「忽里勒臺」(決定繼承人的大會),忽必烈不被广泛承认,于是,忽必烈将大汗在西亚的直辖地(阿姆河以西直到埃及边境)封给旭烈兀换取旭烈兀的支持,旭烈兀建立伊儿汗国(其实旭烈兀留在西亚,忽必烈也没办法,但忽必烈给了旭烈兀统治的合法性)。忽必烈将大汗在中亚的直辖地(阿尔泰山以西直到阿姆河的农耕和城郭地区)封给察合台汗阿鲁忽换取阿鲁忽的支持。而钦察汗国早在元定宗贵由和元宪宗蒙哥统治时期已经取得实际上基本独立的地位。

1264年8月21日,阿里不哥向忽必烈投降。胜利之后忽必烈立即向各系兀鲁思派去急使,召他们东赴蒙古草原,重新召开忽里台大会。忽必烈重开忽里台的目的,是因为考虑到中统元年三月二十四日仓促即位于开平,没有四大兀鲁思的代表参加,不符合成吉思汗的扎撒(蒙古语“军律”、“法规”之意),故而准备依照传统惯例,在祖先发祥地斡难---怯绿涟之域召开由各系宗王参加的忽里台,重新确立自己的大汗地位,并借这次大会扼制帝国分裂的趋势。

钦察汗别儿哥、察合台汗阿鲁忽和伊兒汗旭烈兀(忽必烈之弟)一致同意东来赴会。元世祖也向窝阔台汗海都派去了急使,但海都拒绝前来。当然,这次原定于至元四年(1267年)召开的忽里台没能如约举行,主要是因为各汗国之间随后爆发战争,以及在此后一年多时间里原本同意参加忽里台的阿鲁忽、旭烈兀、别儿哥三位汗王先后去世(旭烈兀1265年去世,别儿哥、阿鲁忽1266年去世,他们不可能参加1267年的忽里台)。但窝阔台汗海都的抗命已经明白无误地表明了分裂意图,忽必烈声称的大汗之位未获公认,成吉思汗及窝阔台汗创立的蒙古帝国处于分崩离析的边缘。

1269年,钦察汗国、窝阔台汗国与察合台汗国召开塔剌思忽里台,达成了协议,共同反对拖雷家族控制的大汗直辖地(即忽必烈的实际控制区)和伊儿汗国(旭烈兀家族控制区,忽必烈的唯一支持者),并协议划分了各自在阿姆河以北地区的势力范围。塔剌思大会标志着大蒙古国的实质分裂和解体,从此察合台汗国和窝阔台汗国脱离了大蒙古国,与掌控蒙古帝国东部的拖雷系家族分头發展。察合台汗国和窝阔台汗国对此后数十年中亚和西亚历史的发展产生了深远的影响。

窝阔台汗海都一直和忽必烈敵對,企图确立自己为大汗之位的继承人。终元世祖忽必烈一朝,元朝和窝阔台汗国、察合台汗国征战不休,直到元成宗时期才彻底解决西北问题。

大蒙古国时期的历任大汗,虽然经由对辽、金故地的征服,与汉文明一直有接触,也往往对汉文化表示接纳,蒙古贵族却大多数反对建立一个汉式的政府;忽必烈对其在汉地的领地则相当重视,并且花费了时间去了解汉人的治国思想和儒家文化,最终以自己的领地开平为中心,建立起了一个汉式的行政中心,其后忽必烈在试图争取整个蒙古帝国统治权的同时,一直没有放弃尝试让汉人接受他作为一个中国皇帝,并为此做了一系列汉化努力。

忽必烈赢取汉人接受其统治的第一个措施便是效仿汉人的典章制度,将“大蒙古国”的历史和皇族“汉化”,其中一个显著做法就是建立太庙,按照中原王朝的传统为大蒙古国的历任大汗确立庙号,追尊谥号。

中统四年(1263年)农历三月,忽必烈下诏在燕京(后来改称大都)建立太庙。至元元年(1264年)十月,初定太庙七室神主。至元二年农历十月十四日(1265年11月23日),忽必烈祭祀太庙,为皇祖成吉思汗上庙号太祖。至元三年(1266年)九月,太庙始作八室神主。十月,太庙建成。丞相安童、伯颜建议制定尊谥庙号,忽必烈命平章政事趙璧等集议,制尊谥庙号,定为八室,为大蒙古国的前四位帝王成吉思汗、窝阔台(元太宗)、贵由(元定宗)、蒙哥(元宪宗)上庙号和谥号,为他们的皇后上谥号;并追尊也速该、术赤、察合台三人为皇帝,也为他们上庙号和谥号,并为拖雷(已经于1251年被追尊为皇帝)改谥号为景襄皇帝,并将他们四人的正妻追谥为皇后,也上谥号。太庙八室,这八位和他们的妻子的神主各居一室。这些做法有效地吸引了汉族谋士和儒生参与忽必烈的新政权,《剑桥中国史——辽宋夏金元》认为,这一系列做法极大地帮助忽必烈巩固了蒙古族政权在汉地的统治。

蒙古帝国的首都,大汗的汗庭处于蒙古高原上的和林哈拉。忽必烈掌控蒙古帝国东部以后,逐步建立了两都制,并最终定都大都,将政权的统治中心移到了汉地文化更加发达的地区,有利于取得汉族谋士和蒙古贵族之间的平衡。

1215年5月31日,成吉思汗率大军攻克金中都(今北京市)。1217年,太师、国王木华黎改中都为燕京。燕京即为后来两都制中的中都。

1256年,忽必烈命刘秉忠在开平(今中国内蒙古自治区锡林郭勒盟正蓝旗多伦县西北闪电河畔)建立王府,忽必烈在此建立了著名的“金莲川幕府”。中统四年五月九日(1263年6月16日),忽必烈下诏升开平府为上都。

中统五年八月十四日(1264年9月5日),忽必烈发布《建国都诏》,改燕京(今北京市)为中都,定为陪都,两都制正式形成。

至元四年正月三十日(1267年2月25日),忽必烈由上都迁都到中都,定中都为首都,忽必烈迁都中都后,居住于中都城外的金代离宫——大宁宫内,并随即在中都的东北部,以大宁宫所在的琼华岛为中心开始了新宫殿和都城的规划兴建工作,上都成为陪都。

至元九年二月三日(1272年3月4日),忽必烈将中都改名为大都(突厥语称汗八里,帝都之意),元大都包括南城(金中都旧城)和北城(元大都新城),两者的城墙“仅隔一水”。

至元十一年正月初一(1274年2月9日),宫阙告成,元世祖忽必烈首次在大都皇宫正殿大明殿举行朝会,接受皇太子、诸王、百官以及高丽国王王禃所派使节的朝贺。

至元二十八年五月二十一日(1291年6月18日),忽必烈下诏颁布元朝第一部全国性的法律典籍《至元新格》。

忽必烈统一中国后,元朝疆域空前辽阔,远超汉唐盛世。“自封建变为郡县,有天下者,汉、隋、唐、宋为盛,然幅员之广,咸不逮元。汉梗于北狄,隋不能服东夷,唐患在西戎,宋患常在西北。若元,则起朔漠,并西域,平西夏,灭女真,臣高丽,定南诏,遂下江南,而天下为一,故其地北逾阴山,西极流沙,东尽辽左,南越海表。盖汉东西九千三百二里,南北一万三千三百六十八里,唐东西九千五百一十一里,南北一万六千九百一十八里,元东南所至不下汉、唐,而西北则过之,有难以里数限者矣。”

元朝不仅在疆域面积上远迈汉唐,而且在东北、西北、西南等边疆地区的控制程度上也远超汉唐盛世。“盖岭北、辽阳与甘肃、四川、云南、湖广之边,唐所谓羁縻之州,往往在是,今皆赋役之,比于内地;而高丽守东藩,执臣礼惟谨,亦古所未见。”

元世祖至元十七年(1280年)元朝的疆域范围:东北至外兴安岭、鄂霍次克海、日本海,包括库页岛,并到达朝鲜半岛中部的铁岭和慈悲岭一带,北到西伯利亚南部(谭其骧版地图认为北到北冰洋),到达贝加尔湖以北的鄂毕河和叶尼塞河上游地区,西北至今新疆大部分地区,西南包括今西藏、云南、以及缅甸北部,南到南海,东南到达东海中的澎湖列岛。

在灭南宋前后,元政府曾要求周边一些国家或地区(包括日本、安南、占城、缅甸、爪哇、琉求國)臣服,接受与元朝的朝贡关系,但遭到拒绝,故派遣军队进攻攻打这些国家或地区,例如緬甸蒲甘王朝拒絕朝貢,元軍入侵蒲甘並攻破蒲甘城,令缅甸臣服於元朝。其中以入侵日本国最为著名,也最惨烈。

忽必烈在位时期和中亚的察合台汗国,窝阔台汗国多次交战,双方互有胜负,1289年,窝阔台汗国夺取元朝控制下的新疆南部塔里木盆地大部分地区,元朝只控制塔里木盆地东部的且末、焉耆等地区。终忽必烈一朝,元朝始终控制新疆北部的别失八里(今乌鲁木齐东北)一带和新疆东部的吐鲁番、哈密等地。

对日战争 至元十一年(1274年)元军發動第一次侵日戰爭,日本史書稱之為“文永之役”,以三萬二千餘人,東征日本。至元十八年(1281年)七月,忽必烈又發動第二次侵日戰爭,史稱“弘安之役”,由范文虎、李庭率江南軍十餘萬人,到達次能、志賀二島,卻碰到颱風,溺死近半。通常认为台风(日本人称之为“神风”)是这两次征日造成失败的最大原因。亦有观点认为,忽必烈担心归附军的忠诚,故而借东征日本而一举消除隐患。

元朝重臣郝经在中统元年(1260年)农历四月奉元世祖忽必烈之命出使南宋南北议和,在九月到达南宋后被扣留软禁于真州15年,直到至元十二年(1275年)农历二月才被南宋送归元朝境内,他在被软禁期间十余次给南宋君臣上书,希望元宋缔结和约,均无任何回复。郝经在中统元年(1260年)农历十一月给南宋两淮制置使李庭芝的书信《再与宋国两淮制置使书》中对元世祖忽必烈的評價是:“今主上应期开运,资赋英明,喜衣冠,崇礼乐,乐贤下士,甚得中土之心,久为诸王推戴。稽诸气数,观其德度,汉高帝、唐太宗、魏孝文之流也。” (“汉高帝”指的是汉太祖刘邦,“太祖”为庙号,“高帝”为谥号,《史记》中常谓“高祖”,因此人多以为其庙号为高祖,其实乃庙号谥号混称。“唐太宗”指的是李世民。“魏孝文”指的是北魏孝文帝拓跋宏。)

元朝重臣郝经在中统二年(1261年)给南宋丞相贾似道的第三封书信《复与宋国丞相论本朝兵乱书》中对元世祖忽必烈的評價是:“夫主上之立,固其所也。太母有与贤之意,先帝无立子之诏。主上虽在潜邸,久符人望,而又以亲则尊,以德则厚,以功则大,以理则顺,爱养中国,宽仁爱人,乐贤下士,甚得夷夏之心,有汉、唐英主之风。加以地广众盛,将猛兵强,神断威灵,风蜚雷厉,其为天下主无疑也。”

明朝官修正史《元史》宋濂等的評價是:“世祖度量弘广,知人善任使,信用儒术,用能以夏变夷,立经陈纪,所以为一代之制者,规模宏远矣。”

明朝官修正史《元史》宋濂等的評價是:“世称元之治以至元、大德为首。……。故终世祖之世,家给人足。”

明朝官修皇帝实录《明太祖实录》记载,明太祖朱元璋在洪武七年八月初一日(1374年9月7日),亲自前往南京历代帝王庙祭祀三皇、五帝、夏禹王、商汤王、周武王、汉太祖、汉光武帝、隋文帝、唐太宗、宋太祖、元世祖一共十七位帝王,其中对元世祖忽必烈的祝文是:“惟神昔自朔土,来主中国,治安之盛,生餋之繁,功被人民者矣。夫何传及后世不遵前训,怠政致乱,天下云扰,莫能拯救。元璋本元之农民,遭时多艰,悯烝黎于涂炭,建义聚兵,图以保全生灵,初无黄屋左纛之意,岂期天佑人助,来归者众,事不能已,取天下于群雄之手,六师北征,遂定于一。乃不揆菲德,继承正统,此天命人心所致,非智力所能。且自古立君,在乎安民,所以唐虞择人禅授,汤武用兵征伐,因时制宜,其理昭然。神灵在天不昧,想自知之。今念历代帝王开基创业、有功德于民者,乃于京师肇新庙宇,列序圣像,每岁祀以春、秋仲月,永为常典,礼奠之初,谨奉牲醴致祭,伏惟神鉴。尚享!”

明朝官修皇帝实录《明太祖实录》记载,洪武二十二年(1389年)十二月,明太祖朱元璋给北元兀纳失里大王的信中,对元太祖和元世祖的评价如下:“昔中国大宋皇帝主天下三百一十余年,后其子孙不能敬天爱民,故天生元朝太祖皇帝,起于漠北,凡达达、回回、诸番君长尽平定之,太祖之孙以仁德著称,为世祖皇帝,混一天下,九夷八蛮、海外番国归于一统,百年之间,其恩德孰不思慕,号令孰不畏惧,是时四方无虞,民康物阜。”

邵远平《元史类编》的評價是:“册曰:遂辟雄图,混一中外;德威所指,无远弗届;建号立制,垂模一代;崇儒察奸,旋用旋败;英明克断,用无祗悔。”

叶子奇《草木子》卷三上: “元朝自世祖(忽必烈)混一之后,天下治平者六、七十年,轻刑薄赋,兵革罕用;生者有养,死者有葬;行旅万里,宿泊如家,诚所谓盛也亦!”

毕沅《续资治通鉴》的評價是:“帝度量恢廓,知人善任使,故能混一区宇,扩前古所未有。惟以亟于财用,中间为阿哈玛特、卢世荣、僧格所蔽,卒能知其罪而正之。立纲陈纪,殷然欲被以文德,规模亦已弘远矣。”(“阿哈玛特”指的是阿合马,“僧格”指的是桑哥,不同的人对他们的名字进行汉语音译时,有一定差别。)

魏源《元史新编》的評價是:“论曰:元之初入中国,震荡飘突,惟以杀伐攻虏为事,不知法度纪纲为何物,其去突厥、回纥者无几。及世祖兴,始延揽姚枢、窦默、刘秉忠、许衡之徒,以汉法治中夏,变夷为华,立纲陈纪,遂乃并吞东南,中外一统。加以享国长久,垂统创业,轶遼、金而媲漢、唐,赫矣哉!且其天性宽宏,包帡无外。阿里不哥及海都、笃哇诸王,皆亲犯乘舆。对垒血战,力屈势穷,一朝归命,则皆以太祖子孙,大朝会于上都,恩礼宴赉如初。当南北锋焰血战之余,或离间以侍郎张天悦通宋而不信。敕南儒被掠卖为奴者,官赎为民。所获宋商、宋谍私入境者,皆纵遣之而不诛。置榷场于樊城,通宋互市,弛沿边军器之禁。其长驾远驭如是。宋幼主母子至通州,命大宴十日,小宴十日,然后赴上都。除弘吉剌皇后厚待之事别详《皇后传》外,其母子在江南庄田,听为世业。其后文宗时市故全太后田为大承天寺永业,市故瀛国公田为大翔龙寺永业,直至顺帝末,始夺和尚赵完普之田归官,直与元相终始。宋之宗室如福王与芮等,随宋主来归,授平原郡公,其家赀在江南者,取至京赐之。此外宗室多类此。即奸民冒称赵氏作乱者,从不以累及宋后,其优礼亡国也如是。思创业艰难,移漠北和林青草丛植殿隅,俾后世无忘草地。又留所御裘带于大安阁以示子孙。武宗至大中尝诣阁中发故箧阅之,则皆大练之服。西域贾胡屡献牙忽大珠,价值数万而不受。宫闱肃穆,无豔宠奇闻。至元八年,平滦路昌黎县民生男,夜中有光,或奏请除之,帝曰:‘何幸天生一好人,奈何反生妒忌!’命有司加恩养。伯颜伐宋,谆谆命以曹彬取江南不戮一人为法。其俭慈也又如是,非命世天纵而何?惟功利之习不能自胜于中,故日本、爪哇之师远覆于海岛,王、阿、桑、卢掊克之臣相仍于覆辙,盖质有余而学不足欤!”(“王、阿、桑、卢”指的分别是王文统、阿合马、桑哥、卢世荣。四人均为元世祖朝不同时期的理财大臣。)

曾廉《元书》的評價是:“论曰:世祖崇儒重道,而特进言利之臣,三进三乱而讫不悟,岂非其明有所蔽耶?然其不欲剥民亦审矣。殆以为自我作则,将上下均足,堪为后世经制也。呜呼!以世祖之仁,乘开国之运,而言利之弊,若此,然则利其有可言者耶?至其任中书枢密而重台纲,法纪立矣。国治民安是在知人哉!”

中華民国史学家屠寄《蒙兀儿史记》的評價是:“汗目有威稜,而度量弘广,知人善任,群下畏而怀之,虽生长漠北,中年分藩用兵,多在汉地,知非汉法不足治汉民。故即位后,引用儒臣,参决大政,诸所设施,一变祖父诸兄武断之风,渐开文明之治。惟志勤远略,平宋之后,不知息民,东兴日本之役,南起占城、交趾、缅甸、爪哇之师,北御海都、昔里吉、乃颜之乱。而又盛作宫室,造寺观,干戈土木,岁月不休。国用既匮,乃亟于理财,中间颇为阿合马、卢世荣、桑哥之徒所蔽,虽知其罪而正之,闾阎受患已深矣。”

中華民国官修正史《新元史》柯劭忞的評價是:“唐太宗承隋季之乱,魏徵劝以行王道、敦教化。封德彝驳之曰:‘书生不知时务,听其虚论,必误国家。’太宗黜德彝而用徵,卒致贞观之治。蒙古之兴,无异于匈奴、突厥。至世祖独崇儒向学,召姚枢、许衡、窦默等敷陈仁义道德之说,岂非所谓书生之虚论者哉?然践阼之后,混壹南北,纪纲法度灿然明备,致治之隆,庶几贞观。由此言之,时儿今古,治无夷夏,未有舍先王之道,而能保世长民者也。至于日本之役,弃师十万犹图再举;阿合马已败,复用桑哥;以世祖之仁明,而吝于改过。如此,不能不为之叹息焉。”

\subsection{中统}

\begin{longtable}{|>{\centering\scriptsize}m{2em}|>{\centering\scriptsize}m{1.3em}|>{\centering}m{8.8em}|}
  % \caption{秦王政}\
  \toprule
  \SimHei \normalsize 年数 & \SimHei \scriptsize 公元 & \SimHei 大事件 \tabularnewline
  % \midrule
  \endfirsthead
  \toprule
  \SimHei \normalsize 年数 & \SimHei \scriptsize 公元 & \SimHei 大事件 \tabularnewline
  \midrule
  \endhead
  \midrule
  元年 & 1260 & \tabularnewline\hline
  二年 & 1261 & \tabularnewline\hline
  三年 & 1262 & \tabularnewline\hline
  四年 & 1263 & \tabularnewline\hline
  五年 & 1264 & \tabularnewline
  \bottomrule
\end{longtable}

\subsection{至元}

\begin{longtable}{|>{\centering\scriptsize}m{2em}|>{\centering\scriptsize}m{1.3em}|>{\centering}m{8.8em}|}
  % \caption{秦王政}\
  \toprule
  \SimHei \normalsize 年数 & \SimHei \scriptsize 公元 & \SimHei 大事件 \tabularnewline
  % \midrule
  \endfirsthead
  \toprule
  \SimHei \normalsize 年数 & \SimHei \scriptsize 公元 & \SimHei 大事件 \tabularnewline
  \midrule
  \endhead
  \midrule
  元年 & 1264 & \tabularnewline\hline
  二年 & 1265 & \tabularnewline\hline
  三年 & 1266 & \tabularnewline\hline
  四年 & 1267 & \tabularnewline\hline
  五年 & 1268 & \tabularnewline\hline
  六年 & 1269 & \tabularnewline\hline
  七年 & 1270 & \tabularnewline\hline
  八年 & 1271 & \tabularnewline\hline
  九年 & 1272 & \tabularnewline\hline
  十年 & 1273 & \tabularnewline\hline
  十一年 & 1274 & \tabularnewline\hline
  十二年 & 1275 & \tabularnewline\hline
  十三年 & 1276 & \tabularnewline\hline
  十四年 & 1277 & \tabularnewline\hline
  十五年 & 1278 & \tabularnewline\hline
  十六年 & 1279 & \tabularnewline\hline
  十七年 & 1280 & \tabularnewline\hline
  十八年 & 1281 & \tabularnewline\hline
  十九年 & 1282 & \tabularnewline\hline
  二十年 & 1283 & \tabularnewline\hline
  二一年 & 1284 & \tabularnewline\hline
  二二年 & 1285 & \tabularnewline\hline
  二三年 & 1286 & \tabularnewline\hline
  二四年 & 1287 & \tabularnewline\hline
  二五年 & 1288 & \tabularnewline\hline
  二六年 & 1289 & \tabularnewline\hline
  二七年 & 1290 & \tabularnewline\hline
  二八年 & 1291 & \tabularnewline\hline
  二九年 & 1292 & \tabularnewline\hline
  三十年 & 1293 & \tabularnewline\hline
  三一年 & 1294 & \tabularnewline
  \bottomrule
\end{longtable}


%%% Local Variables:
%%% mode: latex
%%% TeX-engine: xetex
%%% TeX-master: "../Main"
%%% End:
