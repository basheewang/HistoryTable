%% -*- coding: utf-8 -*-
%% Time-stamp: <Chen Wang: 2021-11-01 11:36:46>

\subsection{景帝孫休\tiny(258-264)}

\subsubsection{生平}

吴景帝孫休(235年-264年9月3日),字子烈,為孫權第六子,在父親孫權、弟孫亮後繼任為吳國第三任皇帝,在位六年。

孙休生于嘉禾四年(235年),母王夫人,13岁时,跟随谢慈和盛冲就学。

太元二年(252年)受封為琅琊王,居於虎林,當時十八歲。同年四月,大帝因患風疾病死於建業,孫休的嫡弟孫亮繼位,由太傅諸葛恪秉政,諸葛恪不欲諸王在濱江兵馬之地,遂徙孫休至丹楊郡。其後又因丹楊太守李衡數次以事侵擾孫休,孫休上書乞求徙往其他郡,孫亮遂下詔徙孫休至會稽郡。

孙休的岳母和姐姐孙鲁育被权臣孙峻冤杀,孙休害怕,将妻子朱氏送回建业,执手泣别。朱氏到建业,被孙峻遣回。

太平三年九月廿六日(258年11月9日),宗室孫綝發動政變,罷黜孫亮為會稽王,立孫休為帝,孫休三次辭讓而受,改元永安,封孫綝為丞相,孫綝五兄弟皆封侯掌禁军,權傾朝野,時為十月十八日己卯(11月30日)。

孙休先假意麻痹孙綝,将举报孙綝谋反之人,交給孙綝处置,后又加孙綝弟孙恩为侍中分其权,年末设宴请孙綝,孙綝称病不赴,孙休十多次派人去请,孙綝终于赴宴,席间想借故早退,被丁奉等擒住,孙休历数孫綝罪状斩之,灭其三族。孙綝的权臣地位继承自其堂兄权臣孙峻,孙休又将孙峻棺材削薄后重新下葬,并将孙峻、孙綝开除宗籍,称之为“故峻”“故綝”,并赦免被孙峻、孙綝所害之人。

孫休在位期間,以衛將軍濮陽興為丞相,廷尉丁密、光祿勳孟宗為左右御史大夫。布典宮省,興關軍國。

孫休崇尚文化。永安元年創設國學,置學官,立五經博士,選送吏中及將吏子弟好學者就學,为南京太学之滥觞,韋昭為首任博士祭酒。

武功方面,无甚建树,曾图先统一南方。永安七年(264年)二月,趁蜀中無主,西征巴蜀,以鎮軍将军陸抗、撫軍将军步協、征西將軍留平、建平太守盛曼,率大軍圍蜀巴東守將羅憲。魏使將軍胡烈率步騎二萬侵擾西陵,以救羅憲,陸抗等遂引軍退回吳國,最终丝毫未能夺取蜀汉故地。

同年七月,孫休病重,不能語,尚能書;同月廿四日(9月2日)大赦天下,但次日(9月3日)以三十歲英年早逝。丞相濮陽興、左將軍張布遊說朱皇后,因蜀國初亡,而交阯攜叛,國內震懼,希望立長君,所以意欲孙休亡兄孫和之子孫皓嗣位。

孫皓繼位後,於元興元年(264年)十一月,誅殺濮陽興及張布。又於甘露元年(265年)七月,逼殺孫休之妻景皇后朱氏,只於苑中小屋治喪,又送孫休四子於吳小城,再復追殺年長的孫{\fzk 𩅦}及孫{\fzk 𩃙}。

\subsubsection{永安}

\begin{longtable}{|>{\centering\scriptsize}m{2em}|>{\centering\scriptsize}m{1.3em}|>{\centering}m{8.8em}|}
  % \caption{秦王政}\
  \toprule
  \SimHei \normalsize 年数 & \SimHei \scriptsize 公元 & \SimHei 大事件 \tabularnewline
  % \midrule
  \endfirsthead
  \toprule
  \SimHei \normalsize 年数 & \SimHei \scriptsize 公元 & \SimHei 大事件 \tabularnewline
  \midrule
  \endhead
  \midrule
  元年 & 258 & \tabularnewline\hline
  二年 & 259 & \tabularnewline\hline
  三年 & 260 & \tabularnewline\hline
  四年 & 261 & \tabularnewline\hline
  五年 & 262 & \tabularnewline\hline
  六年 & 263 & \tabularnewline\hline
  七年 & 264 & \tabularnewline
  \bottomrule
\end{longtable}



%%% Local Variables:
%%% mode: latex
%%% TeX-engine: xetex
%%% TeX-master: "../../Main"
%%% End:
