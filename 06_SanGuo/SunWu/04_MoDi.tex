%% -*- coding: utf-8 -*-
%% Time-stamp: <Chen Wang: 2019-12-18 13:05:26>

\subsection{末帝\tiny(264-280)}

\subsubsection{生平}

吴末帝孙皓(243年-284年),字元宗,幼名彭祖,又字皓宗,《三国志》原名为孫晧。為廢太子孫和之子,吳大帝孫權之孫,在位十七年(264年—280年),是三国時期孫吴的第四位,同時也是最後一位皇帝。

吳景帝孙休逝世時,太子非常年幼。因當時吳國處於內憂外患之中,大臣們便合議改立較年長的孫皓即位。孫皓即位後,初期雖然英明施政並多行善舉,在西陵之戰一度挽回吳國的厄運,但中後期實行暴政並過度役使民力,加深了亡國危機。最終,吳國於280年被西晉征服,三國時代也因此終結。

孫皓並無廟號與謚號,後世史書中多將孫皓稱為吳後主、吳末帝,也有用他即位前的封號烏程侯,或是歸晉後的封號歸命侯來指代他。

孫皓出生於赤烏六年(243年),是吳主孫權三子孫和的長子。孫皓的嫡母張妃是張承的女兒,生母何姬是孫和的一名庶妃。在他出生的同一年,孫和被立為太子,直到赤烏十三年(250年)因陷入“二宮之爭”被孫權廢黜,流放到故鄣(今浙江省安吉縣)。太元二年(252年)正月,孫權又將孫和封為南陽王,孫和帶家眷移居封地長沙(郡治今湖南省長沙縣)。不久後,孫權逝世,由十歲的幼子孫亮即位。

孫亮即位後,張妃的舅舅諸葛恪秉持朝政。建興二年(253年),宗室孫峻誅殺諸葛恪後,借故民間有傳言稱諸葛恪想迎孫和即位,而剝奪了孫和的王位,並將孫和流放到新都(治今浙江省淳安縣),隨後賜死,張妃也一同死去。此時的孫皓年僅十二歲,還有三個異母弟弟,何姬為了將孫皓等人撫養長大而保全了性命。太平三年(258年),孫亮被繼承孫峻權力的孫綝廢黜,孫權的六子孫休被立為帝。

孫休即位後,將孫皓封為烏程侯,命孫皓前往封地烏程(今浙江省湖州市)。他的異母弟孫德、孫謙也分別被封為錢塘侯、永安侯。在當烏程侯期間,孫皓與烏程令萬彧相識,彼此交好。永安七年(264年,魏咸熙元年)七月,孫休托孤於丞相濮陽興後逝世。在孫休死後,濮陽興並未遵從他的意愿立太子孫𩅦為帝。當時吳國的盟國蜀漢已經滅亡,交趾一帶又發生了叛亂,大臣們考慮著擁立一位較年長的君主。已升任為左典軍的萬彧便向濮陽興和另一位權臣張布推薦孫皓,稱孫皓英明果斷,有長沙桓王孫策的風範,并且行事遵守法度。濮陽興與張布被萬彧說服,便一起勸朱太后將孫皓迎立為帝。這一年,孫皓23歲。

統治前期(264-268年):孫皓即位後,採取了一系列的舉措來鞏固自己的地位。一方面,他大行封賞,將迎立有功的丞相濮陽興,加封侍中,兼領青州牧,左將軍張布升為驃騎將軍,加封侍中,又把吳國宿將施績、丁奉升為左、右大司馬,以拉攏臣子。另一方面,他發放糧食,救濟窮人,從皇宮放出大量侍女讓她們可以婚配,並放歸宮中圈養的一些野獸,以一系列惠民政策來爭取民心。當時人們都把他稱為明主。

但一段時間後,治國有成、志得意滿的孫皓便顯露出魯莽暴躁、驕傲自滿、迷信以及好酒色的一面。此外,他還將景帝孫休的妻子朱太后貶為景皇后,追謚自己的父親孫和為文皇帝,將自己的生母何姬奉為太后,妻子滕氏立為皇后,將孫休的太子及其它三個兒子封為王爵,以加強自己繼位的合法性。當初擁立他的濮陽興、張布對孫皓的轉變感到震驚和失望,結果被萬彧秘密向孫皓揭發,孫皓將兩人處斬,並夷三族。此時,距離兩人迎立孫皓才過了四個月。之後,孫皓扶植外戚,將滕皇后的父親滕牧和何太后的弟子何洪、何蔣、何植都封為侯。甘露元年(265年,魏咸熙二年,晉泰始元年)七月,孫皓迫使前太后朱氏自殺,又軟禁了孫休的四個兒子,並殺死了其中較年長的兩人。九月,孫皓聽信術士之言(“荊州有王氣,當破揚州”),又為了防禦司馬氏軍事包夾的迫切需要,決定遷都武昌(今湖北省鄂州市)。這一年十二月,繼承司馬昭權力的司馬炎迫使曹魏禪讓,正式建立了晉王朝。

寶鼎元年(266年,晉泰始二年),出使晉國的使臣丁忠回到武昌。孫皓召集群臣宴會,因常侍王蕃酒醉失態而大怒,雖然有滕牧、留平等重臣出面為王蕃求情,但孫皓依然下令將王蕃斬首。孫皓的這一舉動令大臣們感到震驚遺憾,賀邵、陸抗後來在270年代上疏勸諫時,都有舉此事為例來指責孫皓,重臣陸凱(同年任左丞相)更是在上疏中將王蕃比為吳國的關龍逢,隱含有將孫皓比為暴君夏桀之意。當時,晉國因為才吞併蜀地不久,有意與吳國暫時維持和平。但使臣丁忠發現晉國戰備有其漏洞,勸說孫皓攻取弋阳郡(郡治今河南省潢川縣西),遭到時任鎮西大將軍的陸凱堅決反對,孫皓表面上贊同了陸凱的意見,並未出兵,但最後還是與晉國絕交了。八月,孫皓分置左、右丞相,左丞相由陸凱擔任,右丞相則安排自己的親信萬彧擔任。

起初,孫皓遷都武昌後,因土地貧乏,而孫皓施政不當處漸多,所需的供給大多要從長江下游運上來,使江東百姓頗有不滿,兒童再度傳唱孫權定都武昌時的歌謠:「寧飲建業水,不食武昌魚,寧還建業死,不止武昌居」。十月,永安山民施旦聚眾數千人起義,劫持孫皓的異母弟永安侯孫謙後向吳故都建業(今江蘇省南京市)進發,一路上不斷有人加入,到建業城外時已有數萬人之多,但還是被吳將丁固、諸葛靚擊潰,孫謙被救回。孫皓當初聽說施旦謀反的消息後,不僅不擔憂,反倒覺得這是應驗了之前術士說的“荊州有王氣,當破揚州”一事,肯定了自己遷都的決斷,命令數百人到建業城大喊“天子使荊州兵來破揚州賊”,來壓制之前的晦氣。丁固請示皓如何處理孫謙,孫皓下令將孫謙母子一起毒殺,後來他還殺了亡父孙和的嫡子即自己的另一个异母弟孫俊。十二月時,孫皓將都城遷回了建業。

寶鼎二年(267年,晉泰始三年)夏六月,孫皓下令新建更大的宮殿──昭明宮。為了昭明宮的修建,呂秩二千石以下的官吏都被派往山中督伐木料,昭明宮的修建歷時半年,工程耗資巨大,而且耽誤了農時。當時陸凱、華覈等人上疏勸止,孫皓拒絕聽從。

統治中期(268-272年):寶鼎三年(268年,晉泰始四年),孫皓開始向晉國發起攻擊。這一年,他親率大軍屯駐東關(今安徽省含山縣西南),令左大司馬施績攻江夏(今湖北省雲夢縣南),右丞相萬彧攻襄陽(今湖北省襄陽市),右大司馬丁奉、右將軍諸葛靚進攻合肥(今安徽省合肥市西),交州刺史劉俊、前部督脩則、將軍顧容等率攻擊投降晉國的交阯(郡治今越南北寧市)叛軍,但都沒有取得成功。北伐大軍被司馬望大軍所拒,兩路主力施績、丁奉分別為晉將胡烈、司馬駿所敗,而南征交阯軍隊更是被晉將楊稷大敗,劉俊、脩則戰死,顧容率殘軍退守合浦(郡治今廣西省合浦縣東北)。

建衡元年(269年,晉泰始五年),孫皓派監軍虞汜、威南將軍薛珝、蒼梧太守陶璜從荊州出發,監軍李勖、督軍徐存從建安海路出發,令兩軍在合浦會合共同剿滅交阯叛軍。此外,還派遣右大司馬丁奉再次北征,攻打谷陽(今安徽省靈壁縣)。但到了建衡二年(270年,晉泰始六年),丁奉部在渦口(今安徽省懷遠縣)一帶被晉將牽弘擊退,李勖部以道路不通為由,殺死向導馮斐後率軍無功而返。孫皓為此大怒,丁奉的向導被處死,李勖更是在被何定揭發後,同徐存被全家誅殺。不久後,何定率領五千人馬到夏口(今湖北省武漢市)打獵,吳宗室前將軍、夏口督孫秀害怕是孫皓令何定來抓自己,提前帶領家送眷數百人投奔晉國。晉武帝拜孫秀為驃騎將軍,儀同三司,封會稽公,禮遇備至。

建衡三年(271年,晉泰始七年)正月,孫皓聽信刁玄增改的讖文(“黃旗紫蓋,見於東南,終有天下者,荊、揚之君!”),認為自己是天命所歸,不顧眾人反對,用車載著自己的母親、妻子、孩子以及後宮上千人,親率大軍從牛渚(今安徽省當塗縣)西進伐晉。晉軍派司馬望率軍駐屯在壽春(今安徽省壽縣)作為防備。結果孫皓的軍隊途中被大雪所阻,士兵忍受天寒地凍的同時還要負責拉孫皓的車隊,都難以忍受這樣的勞苦,軍中漸漸出現倒戈的傳言,因此孫皓只好下令還師。孫皓還師前,右丞相萬彧與右大司馬丁奉、左將軍留平曾私下商議先自行回去,後被孫皓得知,雖然心懷不滿,但介於三人都是老臣並沒有馬上處置。當年,丁奉病逝。翌年,孫皓試圖用毒酒毒死萬彧和留平,二人卻都倖免未死,但不久後,萬彧自殺,留平愁悶而死。

在這兩年間裡,孫吳在軍事上接連取得了重大勝利,使得孫皓的自傲心大幅膨脹。先是在建衡三年(271年,晉泰始七年),南征的薛珝、虞汜、陶璜攻破交阯,擒殺晉軍守將,並收復了九真(郡治今越南清化市)、日南(郡治今越南洞海市南)兩郡,後又平定了扶嚴夷,使持續多年的交阯之亂暫告停歇。接著於鳳凰元年(272年,晉泰始八年)秋八月,陸抗成功討伐了因擔心被孫皓加害而叛投西晉的西陵督步闡,不僅成功收復了戰略要地西陵(今湖北省宜昌市),將步闡等人夷三族,並且擊退了由名將羊祜率領的五萬大軍,圍殲晉將楊肇的三萬援軍。西陵大捷之後,孫皓因為兩年內成功收復失土及大敗西晉,越發自志得意滿,更加相信自己是有上天相助,還召術士尚廣為他占卜看是否能取得天下,占卜的結果顯示他將在庚子年“青蓋入洛陽”。孫皓非常高興,從此專門謀劃統一大業,頻繁派遣軍隊襲擊晉國邊境,但都勞而無功。陸抗上疏反對孫皓的窮兵黷武,希望孫皓看清晉強吳弱的事實,建議“蹔息進取小規,以畜士民之力,觀釁伺隙”,又上疏指出西陵、建平戰略地位的重要,請求加強兩地的兵力。建平太守吾彥也憑借從長江上游漂下的大量木屑,斷定晉國將從巴蜀由水路大舉伐吳,上書孫皓請求加強防備。但孫皓不僅沒有重視這些意見,反而在鳳凰三年(274年,晉泰始十年)陸抗病逝後,將他的兵馬一分為五,交給陸抗的五個兒子分別統領。

統治後期(272-279年):軍事上取得耀人成果的同時,吳國內部卻越發不穩定。

自建衡元年(269年,晉泰始五年)左丞相陸凱病逝後,左大司馬施績、右大司馬丁奉、司空孟仁、右丞相萬彧、左將軍留平、太尉范慎、司徒丁固、大司馬陸抗等重臣在六年時間裡先後逝世,吳國有名望的舊臣死亡殆盡。當孫皓忌憚、尊重的重臣都不復存在之後,他的施政也更加殘暴,對於其他忠臣的勸諫也就不再接納容忍。大約從272年開始,他每次召集群臣宴會,都要故意讓每個人都喝得大醉,讓人在邊上專門檢舉他們的過失。甚至剝人臉皮,挖人眼珠。272年後孫皓對勸諫忠臣的容忍度也大幅下降,不惜痛下殺手以杜絕煩人的諫言:大司農樓玄因為多次直諫忤逆孫皓,被流放廣州,服毒而死;中書令賀邵也因直諫而使孫皓痛恨,當賀邵因中風不能說話,被孫皓懷疑是裝病,拷打致死;侍中韋昭因多次堅持己見,被以不聽從詔命為由處死;東觀令華覈多次上書勸諫,結果為了一些小事被免官遣返;豫章太守張俊因為給孫奮的母親掃墓,而被孫皓處以車裂極刑,並夷三族;會稽太守車浚因為開倉賑濟飢民,被懷疑收買人心而處斬;湘東太守張詠因為征稅不足,被孫皓派人斬殺;尚書熊睦對孫皓稍加勸阻,就被孫皓派人用刀環生生打死;甚至連他曾經寵信的何定、陳聲、張俶也被他處決,其中張俶受車裂之刑,陳聲更是被鋸斷頭顱而死。

相比於272年後孫皓殘暴的高壓政策,晉國都督荊州諸軍事的羊祜則對吳國展開懷柔政策。天璽元年(276年,晉泰始十二年),繼孫秀、步闡之後,吳國又一位重要將領——吳國宗室武衛將軍、京下督孫楷叛投晉國,在此前後,平虜將軍孟泰、偏將軍王嗣、威北將軍嚴聰、揚威將軍嚴整、偏將軍朱買、邵凱、夏祥、昭武將軍劉翻、厲武將軍祖始也都紛紛向晉軍投降。但孫皓絲毫沒有感受到危機的來臨。在吳國在接下来的几年裡,各地奉承他的人爭相獻上有吉祥象徵的事物,讓迷信的孫皓始終堅信自己將一統天下。

天紀三年(279年,晉泰始五年),郭馬攻殺廣州督虞授,在廣州發起叛亂。孫皓派遣滕脩、陶濬、陶璜率軍剿滅郭馬叛軍。冬十一月,晉武帝司馬炎依羊祜生前擬制的計劃,令鎮軍將軍司馬伷、安東將軍王渾、建威將軍王戎、平南將軍胡奮、鎮南大將軍杜預、龍驤將軍王濬、巴東監軍唐彬等分六路大舉伐吳。天紀四年(280年,晉泰始六年)正月,杜預、王渾兩軍分別向江陵(今湖北省荊州市)、橫江(今安徽省和縣)進軍,接連進克吳軍要塞。王渾部率先攻克尋陽(今湖北省黃梅縣西南)、賴鄉等城,屯兵橫江,距建業僅百里之遙。二月,在王濬、唐彬部和杜預部、胡奮部、王戎部的攻擊下,荊州的軍事重鎮丹陽(今湖北省秭歸縣)、西陵、荊門(今湖北省宜昌市東南)、夷道(今湖北省宜都市)、樂鄉(今湖北省松滋縣東)、江陵、江安(今湖北省公安縣)、夏口、武昌等先後失守,吳軍僅戰死或投降的都督、監軍就有十四人,牙門將、郡守一級的將領更是有一百二十多人,荊南各郡望風而降。

三月,由丞相張悌率領的吳軍精銳在版橋(今安徽省和縣)被王渾部擊潰,張悌、孫震、沈瑩全部戰死。孫皓自知滅亡在即,在給舅舅何植的信中自責道:“天匪亡吳,孤所招也。瞑目黃壤,當複何顏見四帝乎!”不久後,何植也向王渾軍投降。這時,王濬率水軍從武昌順流而下,直取建業。吳主孫皓派遣張象率水軍一萬餘人前往抵擋,但王濬大軍一到,張象便立即投降。孫皓又另遣陶濬率軍兩萬迎敵,結果士兵全部連夜逃竄。孫皓周圍數百人又請求他殺死寵臣岑昬,他不得以而被迫答應。後來,孫皓聽從光祿勛薛瑩和中書令胡沖的計策,分別遣送使節向王濬、司馬伷、王渾請降,試圖分化晉軍,未能奏效。三月壬寅日(280年5月1日),王濬率大軍進入石頭城,孫皓率太子孫瑾、魯王孫虔等二十一人出降,全家被遣送至洛陽。吳國至此滅亡。晉武帝下詔封孫皓為歸命侯。孫皓決定投降後,為了讓晉軍順利接收各地,廣發勸降書信給臣僚,信中寫道:“孤以不德,忝继先轨。处位历年,政教凶悖,遂令百姓久困涂炭,至使一朝归命有道,社稷倾覆,宗庙无主,惭愧山积,没有余罪。自惟空薄,过偷尊号,才琐质秽,任重王公,故《周易》有折鼎之诫,诗人有彼其之讥。自居宫室。仍抱笃疾,计有不足,思虑失中,多所荒替。边侧小人,因生酷虐,虐毒横流,忠顺被害。闇昧不觉,寻其壅蔽,孤负诸君,事已难图,覆水不可收也。今大晋平治四海,劳心务于擢贤,诚是英俊展节之秋也。管仲极雠,桓公用之,良、平去楚,入为汉臣,舍乱就理,非不忠也。莫以移朝改朔,用损厥志。嘉勖休尚,爱敬动静。夫复何言,投笔而已!”

五月,孫皓到達洛陽後,得到了晉武帝較為優厚的待遇。後來晉武帝大會群臣時,召孫皓進見,孫皓上前叩首請罪,晉武帝對孫皓說:“朕設此座以待卿久矣。”孫皓回應道:“臣於南方,亦設此座以待陛下。”賈充故意刁難他說:“聞君在南方鑿人目,剝人臉皮,此何等刑也?”孫皓回答說:“人臣有弒其君及姦回不忠者,則加此刑耳。”賈充曾指使手下殺害魏帝曹髦,聽了孫皓的話後羞愧不已,而孫皓本人則面不改色。晉武帝曾与王济下棋,孙皓在旁边,晉武帝对孙皓说:“何以好剥人面皮?”孙皓回应道:“见无礼于君者则剥之。”王济当时把脚伸到了棋盘下,因而孙皓讥讽王济。晋武帝有一次问孙皓:“闻南人好作《尔汝歌》,颇能为不?”孙皓正饮酒,于是举羽觞吟诵:“昔与汝为邻,今为汝做臣;上汝一杯酒,令汝寿万春!”孙皓直呼晋武帝为汝,晋武帝感到后悔,自讨没趣。太康四年十二月(284年初),孫皓在洛陽逝世,享年四十二歲,葬於河南縣界。

孫皓是中國歷史上有名的暴君,生性多疑且殘暴,設立諸多酷刑。他曾殺死或流放多名重要宗室,如殺害再从兄弟孫奉,流放从兄弟孙基、孙壹,誅殺五叔孫奮及其五子,殺死異母弟孫謙、孫俊等。對大臣,他也常常施以重刑,僅丞相一級的官員為例:除張悌在亡國之際戰死外,濮陽興被流放處死,夷三族,萬彧被譴自殺,全家遭流放;陸凱死後數年,全家被處以流放。。此外,孙皓非常迷信,常憑借運曆、望氣、卜筮、讖緯之類的原因來決定如遷都、用兵、皇后廢立等重大事件,並因此一直堅信自己將統一天下。

孫皓富有才氣,能吟詩,並有一定書法造詣。他曾在宴會上應晉武帝之邀,當場作《爾汝歌》一首:“昔與汝為鄰,今與汝為臣。上汝一杯酒,令汝壽萬春。”

同絕大多數亡國之君一樣,對孫皓的評價以負面評價為主。從孫皓受到的各種評價來看,他是一個典型的暴君形象,吳國重臣陸凱、陸抗在給孫皓的上疏中,曾多次暗示他堪比夏桀、商紂。曾擔任吳國光祿勛的薛瑩稱在孫皓當政時“昵近小人,刑罰妄加,大臣大將無所親信,人人憂恐,各不自安”。而孫皓被晉軍俘虜後,也批判自己“虐毒横流,忠顺被害。闇昧不觉,寻其壅蔽”,要求未降的吳軍儘快投降。此外,有些吳國人士曾對孫皓做出正面評價,如吳將吾彥在歸降西晉之後,在晉武帝面前力讚孫皓的英明:「吳主英俊,宰輔賢明」;而孫皓早期的好友萬彧也稱讚孫皓好學不倦、英明果斷,有孫策的風範。

在敵國眼中,272年後孫皓因殘暴所導致的吳國內部不安,已經促成了進攻吳國的最佳時機,晉臣張華曾對晉武帝說:“吳主荒淫驕虐,誅殺賢能,當今討之,可不勞而定”。晉將羊祜更是在上疏中稱:“孫皓暴虐已甚,於今可不戰而克。若皓不幸而沒,吳人更立令主,雖有百萬之眾,長江未可窺也,將為後患矣!”認為如果孫皓死掉,伐吳的難度將會大大增加。晉代史臣陳壽、孫盛在批評孫皓的暴虐以外,還指責晉武帝對孫皓的處置太過寬厚,認為像孫皓這樣的“肆行殘暴,忠諫者誅,讒諛者進,虐用其民,窮淫極侈”的禍國殃民之君,“梟首素旗,猶不足以謝冤魂”,“宜腰首分離,以謝百姓”。

同時,關於孫皓也有一些其它方面的評價,如晉臣秦秀在為滅吳的王濬請功時,曾言及孫皓對晉國用兵給晉國帶來的心理威脅,稱“以孫皓之虛名,足以驚動諸夏,每一小出,雖聖心知其垂亡,然中國輒懷惶怖。”後世的李世民則從另一個角度出發,將孫皓前期“權施恩惠之風”與王莽稱帝前“偽行仁義之道”相提並論,認為兩人的失敗就在於“有始無終”,以此得出二人都迅速覆亡的結論。唐代的朱敬則在批評孫皓的同時,還對他的權謀和才華予以一定程度的肯定。孫皓的書跡流傳到唐代,庾肩吾在他所著《書品》中则把孙皓评为「中中」,与曹操杜预并列,称“魏帝(曹操)筆墨雄贍、呉主(孫皓)體裁綿密”,書法家韋續则把他的行隸,評為「下中」品。

\subsubsection{元兴}

\begin{longtable}{|>{\centering\scriptsize}m{2em}|>{\centering\scriptsize}m{1.3em}|>{\centering}m{8.8em}|}
  % \caption{秦王政}\
  \toprule
  \SimHei \normalsize 年数 & \SimHei \scriptsize 公元 & \SimHei 大事件 \tabularnewline
  % \midrule
  \endfirsthead
  \toprule
  \SimHei \normalsize 年数 & \SimHei \scriptsize 公元 & \SimHei 大事件 \tabularnewline
  \midrule
  \endhead
  \midrule
  元年 & 264 & \tabularnewline\hline
  二年 & 265 & \tabularnewline
  \bottomrule
\end{longtable}


\subsubsection{甘露}

\begin{longtable}{|>{\centering\scriptsize}m{2em}|>{\centering\scriptsize}m{1.3em}|>{\centering}m{8.8em}|}
  % \caption{秦王政}\
  \toprule
  \SimHei \normalsize 年数 & \SimHei \scriptsize 公元 & \SimHei 大事件 \tabularnewline
  % \midrule
  \endfirsthead
  \toprule
  \SimHei \normalsize 年数 & \SimHei \scriptsize 公元 & \SimHei 大事件 \tabularnewline
  \midrule
  \endhead
  \midrule
  元年 & 265 & \tabularnewline\hline
  二年 & 266 & \tabularnewline
  \bottomrule
\end{longtable}

\subsubsection{宝鼎}

\begin{longtable}{|>{\centering\scriptsize}m{2em}|>{\centering\scriptsize}m{1.3em}|>{\centering}m{8.8em}|}
  % \caption{秦王政}\
  \toprule
  \SimHei \normalsize 年数 & \SimHei \scriptsize 公元 & \SimHei 大事件 \tabularnewline
  % \midrule
  \endfirsthead
  \toprule
  \SimHei \normalsize 年数 & \SimHei \scriptsize 公元 & \SimHei 大事件 \tabularnewline
  \midrule
  \endhead
  \midrule
  元年 & 266 & \tabularnewline\hline
  二年 & 267 & \tabularnewline\hline
  三年 & 268 & \tabularnewline\hline
  四年 & 269 & \tabularnewline
  \bottomrule
\end{longtable}

\subsubsection{建衡}

\begin{longtable}{|>{\centering\scriptsize}m{2em}|>{\centering\scriptsize}m{1.3em}|>{\centering}m{8.8em}|}
  % \caption{秦王政}\
  \toprule
  \SimHei \normalsize 年数 & \SimHei \scriptsize 公元 & \SimHei 大事件 \tabularnewline
  % \midrule
  \endfirsthead
  \toprule
  \SimHei \normalsize 年数 & \SimHei \scriptsize 公元 & \SimHei 大事件 \tabularnewline
  \midrule
  \endhead
  \midrule
  元年 & 269 & \tabularnewline\hline
  二年 & 270 & \tabularnewline\hline
  三年 & 271 & \tabularnewline
  \bottomrule
\end{longtable}

\subsubsection{凤凰}

\begin{longtable}{|>{\centering\scriptsize}m{2em}|>{\centering\scriptsize}m{1.3em}|>{\centering}m{8.8em}|}
  % \caption{秦王政}\
  \toprule
  \SimHei \normalsize 年数 & \SimHei \scriptsize 公元 & \SimHei 大事件 \tabularnewline
  % \midrule
  \endfirsthead
  \toprule
  \SimHei \normalsize 年数 & \SimHei \scriptsize 公元 & \SimHei 大事件 \tabularnewline
  \midrule
  \endhead
  \midrule
  元年 & 272 & \tabularnewline\hline
  二年 & 273 & \tabularnewline\hline
  三年 & 274 & \tabularnewline
  \bottomrule
\end{longtable}

\subsubsection{天册}

\begin{longtable}{|>{\centering\scriptsize}m{2em}|>{\centering\scriptsize}m{1.3em}|>{\centering}m{8.8em}|}
  % \caption{秦王政}\
  \toprule
  \SimHei \normalsize 年数 & \SimHei \scriptsize 公元 & \SimHei 大事件 \tabularnewline
  % \midrule
  \endfirsthead
  \toprule
  \SimHei \normalsize 年数 & \SimHei \scriptsize 公元 & \SimHei 大事件 \tabularnewline
  \midrule
  \endhead
  \midrule
  元年 & 275 & \tabularnewline\hline
  二年 & 276 & \tabularnewline
  \bottomrule
\end{longtable}

\subsubsection{天玺}

\begin{longtable}{|>{\centering\scriptsize}m{2em}|>{\centering\scriptsize}m{1.3em}|>{\centering}m{8.8em}|}
  % \caption{秦王政}\
  \toprule
  \SimHei \normalsize 年数 & \SimHei \scriptsize 公元 & \SimHei 大事件 \tabularnewline
  % \midrule
  \endfirsthead
  \toprule
  \SimHei \normalsize 年数 & \SimHei \scriptsize 公元 & \SimHei 大事件 \tabularnewline
  \midrule
  \endhead
  \midrule
  元年 & 276 & \tabularnewline
  \bottomrule
\end{longtable}

\subsubsection{天纪}

\begin{longtable}{|>{\centering\scriptsize}m{2em}|>{\centering\scriptsize}m{1.3em}|>{\centering}m{8.8em}|}
  % \caption{秦王政}\
  \toprule
  \SimHei \normalsize 年数 & \SimHei \scriptsize 公元 & \SimHei 大事件 \tabularnewline
  % \midrule
  \endfirsthead
  \toprule
  \SimHei \normalsize 年数 & \SimHei \scriptsize 公元 & \SimHei 大事件 \tabularnewline
  \midrule
  \endhead
  \midrule
  元年 & 277 & \tabularnewline\hline
  二年 & 278 & \tabularnewline\hline
  三年 & 279 & \tabularnewline\hline
  四年 & 280 & \tabularnewline
  \bottomrule
\end{longtable}


%%% Local Variables:
%%% mode: latex
%%% TeX-engine: xetex
%%% TeX-master: "../../Main"
%%% End:
