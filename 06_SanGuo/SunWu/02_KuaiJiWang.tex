%% -*- coding: utf-8 -*-
%% Time-stamp: <Chen Wang: 2019-12-18 12:49:28>

\subsection{会稽王\tiny(252-258)}

\subsubsection{生平}

孫亮(243年-260年),字子明,是中國三國時代吳國的第二代君主,在位六年(252年-258年),後世史書多稱之為吳廢帝、會稽王。

孫亮生于赤乌六年(243年),是吳大帝孫權的幼子,因此特别受到疼爱。孙亮出生的时候,他的長兄孫登、二兄孫慮早已去世。当时的皇太子为三兄孫和,后来孙和被陷害廢去太子之位。於是赤乌十三年(250年)孫權便立孫亮為皇太子,不久又立其母潘淑为皇后。

潘皇后於神凤元年(252年)被宫女所杀,同年孫權也去世,孫亮繼位,時為四月廿八日丁酉(5月23日)。

孫亮登基时年方十岁,却聪明伶俐,受到大臣的尊敬。孫亮曾欲喫酸梅,讓黃門到庫裏去取蜂蜜,蜜中有鼠屎;就召來守庫官詢問,守庫官叩頭謝罪。少帝說:“黃門從你那兒要過蜂蜜嗎?”守庫官說:“曾要過,我沒敢給他。”黃門不服。少帝讓人破開鼠屎,屎中是乾燥的,於是他大笑著對左右說:“如果鼠屎事先就在蜜中,那麽裏外都應是濕的,現在外面濕而裏面乾燥,這必定是黃門放進去的。”詰問黃門,他果然服了罪。左右之人都很震驚恐懼。

孫亮即位之初,諸葛恪、滕胤、孙峻、吕据受顾命之托輔政孙亮(孙弘本来也是顾命大臣之一,因夺权失败而被诸葛恪先行杀害),又有旧臣吕岱、丁奉等人。曹魏乘孙权驾崩之际,于建兴元年(252年)11月发动东兴之战,结果却被太傅諸葛恪為統帥的吴军大败而归。第二年諸葛恪依仗顾命之托,不顾众臣劝阻欲乘胜出兵北伐魏國,但最後因瘟疫而失敗。

铩羽而归后的诸葛恪显得愈发刚愎自用,最终招致万民所怨、众口所嫌。建兴二年(253年),孫峻利用这个机会说服孙亮,于是在宴会上發動政變,殺死諸葛恪。孫峻因功出任丞相。

孫峻为人骄矜险害,动辄使用重刑,因此招致不少人的不满,但最後反对他的人均事敗被迫自殺或處死。他与滕胤、吕据两位顾命大臣虽然关系谈不上友好,但还能够一起融洽的共事,朝廷高层因此平静了一段时间。

255年(五凤二年),孫峻帶兵與魏國於淮河一帶交戰獲勝,魏將文欽投降。次年,孫峻派遣呂據等將領進攻魏國,但孫峻在戰爭期間病逝,由從弟孫綝接掌權力。呂岱亦於是年去世。因孙綝本不是大帝所指定的顾命大臣,呂據、文欽对孙綝完全继承孫峻权力一事非常不满,要求封滕胤為丞相。孫綝沒有理會他們的訴求,改封滕胤为大司马。于是滕胤和呂據发动政變,却反遭孙綝所殺,自此五位顾命大臣已经全部亡去。另一位將領王惇密謀殺死孫綝,亦事敗被殺。

257年(太平二年),孫亮親政,他对孙綝轻视自己的态度感到非常厌恶,於是推行多項措施(如訓練少年軍)以準備推翻他。同年,魏國的諸葛誕在壽春發動叛亂,把兒子諸葛靚送到吳國做人質。孫綝派兵協助諸葛誕但最終失敗。孙綝将失败的缘由归于大都督朱异并在镬里杀害了他,其他一些參戰的將領也因為怕被孫綝殺死而投降了魏國。孙綝返回建业後,得知孙亮对他有所戒备,内心也很恐惧,于是称病不上朝并命自己兄弟把守宫门以求自保。

258年(太平三年),孫亮因孙綝不听自己指挥进军并擅杀朱异等事对孙綝不满到了极点,于是与全尚,全公主,刘承等人密谋除掉孙綝。但消息被孙綝的从姐(全尚之妻)或从外甥女(全皇后)泄露给孙綝。孙綝获悉密报后,于6月26日率先包围皇宫,以孙亮患有精神病为由强迫众臣同意将孫亮廢為會稽王,改立孫休為帝。和孫亮一起策劃政變的大臣都被孫綝殺死。群臣也因为畏惧孙綝的声势不敢多言。

260年,孫亮的封地會稽傳出謠言,說孫亮將返回建業復辟;而孫亮的侍從亦聲稱孫亮在祭祀時口出惡言。

經審判後,孫亮再被貶為侯官侯(侯官,今福建省閩侯縣)和流放,途中死去。據《三國志》記載,孫亮可能是自殺,也可能是被孫休派人毒死的。孫亮死時只有18歲。

吴国灭亡后,吴国的少府卿丹阳人戴显上表朝廷,于是迎回孙亮遗体安葬赖乡(今江苏省南京市溧水区)

\subsubsection{建兴}

\begin{longtable}{|>{\centering\scriptsize}m{2em}|>{\centering\scriptsize}m{1.3em}|>{\centering}m{8.8em}|}
  % \caption{秦王政}\
  \toprule
  \SimHei \normalsize 年数 & \SimHei \scriptsize 公元 & \SimHei 大事件 \tabularnewline
  % \midrule
  \endfirsthead
  \toprule
  \SimHei \normalsize 年数 & \SimHei \scriptsize 公元 & \SimHei 大事件 \tabularnewline
  \midrule
  \endhead
  \midrule
  元年 & 252 & \tabularnewline\hline
  二年 & 253 & \tabularnewline
  \bottomrule
\end{longtable}

\subsubsection{五凤}

\begin{longtable}{|>{\centering\scriptsize}m{2em}|>{\centering\scriptsize}m{1.3em}|>{\centering}m{8.8em}|}
  % \caption{秦王政}\
  \toprule
  \SimHei \normalsize 年数 & \SimHei \scriptsize 公元 & \SimHei 大事件 \tabularnewline
  % \midrule
  \endfirsthead
  \toprule
  \SimHei \normalsize 年数 & \SimHei \scriptsize 公元 & \SimHei 大事件 \tabularnewline
  \midrule
  \endhead
  \midrule
  元年 & 254 & \tabularnewline\hline
  二年 & 255 & \tabularnewline\hline
  三年 & 256 & \tabularnewline
  \bottomrule
\end{longtable}

\subsubsection{太平}

\begin{longtable}{|>{\centering\scriptsize}m{2em}|>{\centering\scriptsize}m{1.3em}|>{\centering}m{8.8em}|}
  % \caption{秦王政}\
  \toprule
  \SimHei \normalsize 年数 & \SimHei \scriptsize 公元 & \SimHei 大事件 \tabularnewline
  % \midrule
  \endfirsthead
  \toprule
  \SimHei \normalsize 年数 & \SimHei \scriptsize 公元 & \SimHei 大事件 \tabularnewline
  \midrule
  \endhead
  \midrule
  元年 & 256 & \tabularnewline\hline
  二年 & 257 & \tabularnewline\hline
  三年 & 258 & \tabularnewline
  \bottomrule
\end{longtable}


%%% Local Variables:
%%% mode: latex
%%% TeX-engine: xetex
%%% TeX-master: "../../Main"
%%% End:
