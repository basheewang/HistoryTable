%% -*- coding: utf-8 -*-
%% Time-stamp: <Chen Wang: 2019-12-18 10:20:45>

\subsection{大皇帝\tiny(229-252)}

\subsubsection{生平}

孫權(182年12月22日-252年5月21日),字仲謀,吴郡富春(今浙江省杭州市富阳区)人,東漢末三国時期吳[註 1]的著名政治家、戰略家,同時也是吳的締造者及建国皇帝。而在孫權稱帝之前,吳的群臣等對其稱呼為將軍或至尊。在位23年,享年69歲,諡號為大皇帝,廟號太祖。

富春孙氏是江东不顯赫的豪族,世代仕於吳。生父孫堅據傳是春秋时期军事家孫武後人,孤微發跡;孙权亦因此可能是孙武的第22代孙。

孫權生母为吳郡豪族出身的吴夫人,當初懷孕的時候,夢見月亮進去懷裡,之後生下了孫策。及後在懷孫權的時候,又夢見太陽進去懷裡。之後告訴孫堅說:“妾昔日懷著孫策的時候,夢見月亮入懷裡;如今又夢見太陽入懷裡,為什麼會這樣呢?”孫堅回答:「太陽和月亮,是陰陽的能量精氣,是極其貴象的征兆。我們的子孫大概會興家赤旺吧!」。漢光和五年五月十八日(182年7月5日),孫堅擔任下邳县丞的時候嫡次子孫權出生,其面相方頰大口,銳目有神,孫堅覺得驚奇,認為有貴氣的象相。

漢光和七年(184年),朱儁奏请孙坚担任佐军司马,孙坚随朱儁南征北战,将妻吴氏和孙权等诸子都留在九江郡寿春县。

漢中平六年(189年),汉灵帝逝世,长沙(治所在今湖南省长沙市)太守的孙坚起兵从长沙经荆州响应讨伐董卓的关东联军。当时孙权的长兄孙策已在寿春淮南一带颇有名气。其中有庐江人周瑜前来拜会,在周瑜的建议下,孙策于是携母弟搬到庐江郡舒县(今安徽省庐江县西南)。

漢初平二年(191年),孙坚奉袁术之命讨伐荆州刘表,结果中刘表手下的黄祖的埋伏身亡。孙权和家人迁居广陵郡江都。孫策託付张紘照看母弟。自孫堅死後,孫權經常跟隨兄長。孙权性格寬宏有氣度,不但仁厚而且能夠根據不同情況作出多方面判斷。他以厚恩养士而出名,其名气渐渐不输给父兄。孙策也对这个弟弟感到很惊奇,自认为不如他。每当宴请宾客时,孙策常常回头看着孙权说:“这些人,以后都会是你的将领。”

漢初平四年(193年)因孙策决定跟随袁术,就派吕范将孙权等人护送到住在曲阿的舅舅吴景那里居住。翌年,孙策击破了陆康为袁术取得了庐江郡。当时,还是扬州刺史的刘繇担心自己也会被袁氏吞并,与袁术和孙策产生嫌隙,于是将孙权堂兄孙贲和吴景驱逐出曲阿,只有孙权及其母弟弟们还留在那里,于是朱治特意将其从曲阿接到自己家里奉养卫护。孙权和母亲后来又迁至历阳县和阜陵县居住。

漢興平二年(195年)孙策渡江击败刘繇后,孙权和家人跟随着陈宝回到了曲阿居住。孙权到江東以后,与朱然胡综一起读书,结下了深厚的友谊。

漢建安元年(196年),孙权15岁的时候,由朱治举为孝廉,任阳羡县(今江苏宜兴)长,代行奉义校尉。曹操也任命严象将其举为茂才,当时已有属下周泰和潘璋。

孙策平定江东的丹阳、会稽和吴三郡后开始给汉廷进贡。建安二年(197年),汉廷派刘琬前往江东授予孙策会稽太守的职务,刘琬对人说:“我看孙家的兄弟虽然每个都才华横溢,智慧通达,都是荣华福贵不长久。只有次男孝廉,相貌高大挺拔,有大贵之表,且会是最為长寿的,你们等着瞧吧。”袁术与孙策决裂后,拉拢丹阳等六县及山贼头目祖郎,鼓动山越和自己一起共同对付孙策。当时孙策率兵前往讨山贼,仅孙权等数百人留在宣城,山贼数千人蜂拥而至,年轻的孙权在周泰的保护之下得以幸免。

漢建安四年(199年)末至次年初,孙权随同孙策征庐江太守刘勋于皖城。刘勋败逃后,又进军沙羡讨伐黄祖,与仇敌黄祖在沙羡一带展开大战,黄祖几乎全军覆没,韩唏战死,黄祖只身逃走,士卒溺死者达万人,豫章太守华歆又举城投降。平定了庐江豫章二郡。孫策與曹操交好,表面臣服於漢朝廷之下,曹操並加封孫策為吳侯,並以礼征辟孙权和孙权的弟弟孙翊到漢朝廷擔當漢臣職務,但二人均沒有前往。

漢建安五年(200年)春,孙策遭到刺殺,不選擇與自己性格極其相似,眾人看重為最適合繼任者的三弟孫翊,而是選擇性格與自己大不相同的二弟孫權。孫權對兄長的去世痛哭不已沒能親自視察政事。經過長史张昭勸說,乃除去喪服,由張昭扶上馬外出巡察軍營,于是眾人之心都歸附於孫權。

曹操見孫策已死本打算伐吳,侍御史張紘勸諫曹操不該乘人之危。曹操聽從其言,通過东汉朝廷冊封孫權为討虏将军,兼领会稽太守,以吴县为治所。

孫權剛上任,只占有会稽、吴郡、丹杨、豫章、庐陵、庐江六郡,除孫權本人為會稽太守外,其他五人都是孙策生前所任命的部將,分別是:丹陽太守吳景、豫章太守孫賁、廬陵太守孫輔、吳郡太守朱治、廬江太守李術。然而孫策死後,孫輔認為孫權沒有能力保衛江東,於是与曹操暗通打算出賣孫權,被孫權察覺後給予制裁。堂兄孙暠(孫靜長男)欲攻打会稽郡夺取政權,被虞翻阻止。這其中李术尤为不服從孫權,與梅乾、雷緒、陳蘭數萬人在集結淮水一帶騷擾破壞,孫權寫信要求李術扣留這些叛逃者。李術公開表示有德見歸,無德見叛,不應復還為由拒絕。於是孫權用計策寫信給曹操,說嚴象被殺是李術所為,所以不應該理會李術。孫權隨後與孫河、徐琨一起親征叛徒李術于皖城。皖城被孫權包圍,李術向曹操求救,但是曹操沒有到來,一切發展正如孫權所設計的一樣。城內糧盡只能用泥丸代替食糧充飢,隨即破皖城,李術被梟首,孫權迁徙城裡人及李術部将三万余人到江東,留下一座空城。

孫策平定江東的時候,曾對當地士族進行打壓、屠戮,導致孫家在本土得不到支持。孙权以张昭为師傅,並任用父兄留下的部將,以部曲私兵世襲制作為條件懷柔本土豪族,大量起用豪族子弟,穩定江東孫家政權。陆逊、徐盛、留贊、诸葛瑾、步骘、顾雍、顧徽、是仪、吕岱、朱桓、骆統等贤才良将都在这一时期加入孙权麾下。周瑜斷言他以後能成就帝王大業,并将好友鲁肃推荐给孫權认识。鲁肃则向孙权說出漢室不能復興,曹操不能一時間消滅。應該鼎足江東靜觀其變,在北方多戰亂的時候乘勢应消滅刘表,佔據长江以南建立帝業的方案。

未開發山地潛藏的山越也大規模發動叛亂,而江東許多的本地豪族士族與山越族群都有緊密聯繫,因此,在孫氏每次出征對外的時候,都給予江東內部很大的侵擾,也一直牽制著吳國數十年的對外作戰,漢建安八年(203年),豫章鄱陽縣等地山越再起,孫權即刻命征虜中郎將呂範平定鄱陽、蕩寇中郎將程普討伐樂安,派賀齊討平東冶地,建昌都尉太史慈分頭進討山越,又派別部司馬黃蓋、韓當、呂蒙等人扼守山越經常出沒的郡縣,恢復了原設縣邑,穩定了秩序。漢建安十一年(206年)又率領孫瑜,周瑜,淩統,成功討平山越麻、保二屯。

漢建安十二年(207年),自黄祖一处来降的甘寧說:「今漢已經日漸衰微,曹操為滿足自己的心,終於成了篡漢的盜賊。南荊之地,山陵地勢有利,江川流通,國的西邊的確是這樣的形勢。我已看透劉表,考慮的不夠長遠,兒子也是無能的人,不是能夠承傳基業之才。主公應當盡早規劃,不能落入曹操手上。進圖之計,先取黃祖為佳。黃祖如今年老,老邁衰退嚴重,錢財糧谷都已經缺乏,左右矇騙他,事出於錢財私利,侵要吏士的錢財,吏士心裏都憤怒。舟船戰具,廢棄也不修理,耕農懶惰,軍隊沒有法紀。如果主公現在去攻打,必定能大敗。一旦打敗黃祖軍,擊鼓行軍至西,西據楚關,大局趨勢擴張,這樣就可以逐漸進取巴蜀。」孫權贊同並採納。張昭當時就在席上坐,難言道:「吳國如今危懼,如果行軍攻打,必然招致恐慌。」甘寧回答道:「國家將蕭何的重任交給君,君留置守護卻擔心憂亂,那為什麼還要仰慕古人?」孫權對舉起酒杯附於甘寧說:「興霸,今年行軍討伐,就如這杯酒,決意託付給卿你。卿盡量提出方略,如能夠破黃祖,則是卿的功勞,不要因為張長史(張昭)之言而放棄。」出兵虏其人民而还。

漢建安十三年(208年),孙权發動江夏之戰再讨黄祖,以周瑜為都督。呂蒙隨軍出征。黃祖見孫權兵來,急派水軍都督陳就率兵反擊,呂蒙統率前鋒部隊,身先戰陣,親自斬殺陳就。擄獲其船隻、士兵。返回到孫權大軍,並引領自軍兼程趕路,水路兩路齊進。凌統先攻下城池,黃祖隻身逃竄,被孫權軍中的騎兵馮則所斬殺。此戰,孫權大獲全勝,但是劉表長子劉琦及時前來禦敵江夏北部,孫權只能有效佔領江夏郡南部區域,後将治所自吴移居至京口。

漢建安十三年(208年)秋,曹操對孫權發出以八十萬軍力會獵江東的書信,孫權打算與曹操決一死戰,但張昭等群臣勸孫權歸降,礙於豪族群臣的壓力下孫權沒有表達自己的意見,聽後離開席間換衣服,唯獨魯肅離座找孫權說要對抗曹操,孫權很高興魯肅與自己的想法一致,對張昭等人所說的感到非常失望,魯肅勸孫權召回進兵鄱陽的周瑜,並邀請劉備加盟的提議。孫權答應,隨即派魯肅到荊州打探情況。當時荆州牧刘表病死,劉表次子劉琮及其母蔡氏其舅蔡瑁因仇視刘备而投降曹操。鲁肃到荊州之前劉備被曹操打敗,荊州已經落入曹操之手,劉備南渡长江,魯肅與他相遇詢問去向,劉備打算到蒼梧投靠朋友吳巨,魯肅則說明孫權的意向和實力,邀請劉備加盟孫權共同對抗曹操,而不是投靠力弱的人。刘备很高興孫權的邀請,聽後見事態緊急隨即派诸葛亮去求見孙权。孫權故意刁難諸葛亮,藉此通過他對曹操軍的分析,去說服江東豪族及投降派,孫權聽後大悅。之後群臣商議,眾人勸孫權投降,但周瑜向孫權分析曹操與孫權兩軍的優劣勝敗,指出:「其一,曹軍背後仍有後顧之憂,西涼有馬騰、韓遂等軍閥,戰端一開,必偷襲曹軍背後。」、「其二,北方人慣習陸戰而不擅水戰,竟敢捨馬鞍而就船槳,此乃捨長就短。」、「其三,寒冬將至,曹軍兵缺衣食,馬無藁草,兵卒士氣低落。」、「其四,曹軍遠途跋涉,奔襲千里,水土不服,多生病患。」既而進步分析了曹軍的實際力量,指出來自中原的曹軍不過十五六萬,而且所得劉表新降的七八萬人,人心並不向曹。」此時只有周瑜、鲁肃坚持抗击曹操的主张,意见与孙权相合。隨即以決斷之勢拔劍砍掉桌子一角,說:「敢再有言降曹者,如同此案!」,藉此將投降派氣焰壓倒,並將一早已經準備好的三萬軍隊交給周瑜指揮。周瑜、程普分别被任命为左、右都督,魯肅為贊軍校尉輔助周瑜。孫權派周瑜抵禦曹操大軍,在赤壁与曹军相遇,周瑜大败曹操军队。周瑜等又追击到烏林破曹軍,曹操只好撤回北方,乘勝進攻荊州南郡。甘宁在夷陵城,被曹仁的部队所包围,周瑜采纳吕蒙的计策,留下凌统抗拒曹仁,用其中一半兵力驰救甘宁,南郡相持一年間,孙权為了減低周瑜們的前線壓力,親率剩餘的小量軍力军围合肥,相持合肥一个多月,聽從張紘的建議撤退。而劉備以張飛和一千人換二千人為條件向周瑜借兵,然後在孫曹交戰間乘機攻取了长沙、桂阳、武陵、零陵荊南四郡,並上表劉琦為荊州牧,領有了荊南四郡。

漢建安十四年(209年),周瑜攻破南郡。孙权以周瑜为南郡太守。刘备上表奏封孙权代理车骑将军,兼任徐州牧。孙权又招揽了滕耽、吾粲等人。當時劉琦去世,失去了荊州四郡領有權,劉備隨即向周瑜借地,周瑜分南岸給刘备,後劉備將油江口改名為公安。劉備嫌地少無法容納人馬,親自到京口見孙权借荊州數郡(南郡、长沙、桂阳、武陵、零陵)並督領荊州,周瑜、呂範提議軟禁劉備,孙权聽從魯肅所說而不採納,並借出荆州数郡於刘备,孫權暫表劉備為荊州牧,孫權以孫夫人聯姻來鞏固孙刘联盟的關係,也奠定了三国鼎立的基础。周瑜和甘寧勸孫權入蜀,孫權邀請劉備共同取益州,劉備以劉璋是同祖宗為由拒絕,並說如果孫權打劉璋自己一定阻止,如果我打劉璋的話,我必定會披髮入山林歸隱,不做攻取同宗的事。孫權不聽,並派孫瑜進攻益州,劉備阻止並不給孫瑜前進的去路,並說不能這樣做,孫權只有下令退還。

漢建安十五年(210年),孙权任命鲁肃为太守,驻守陆口,又遣步骘为交州刺史,挥师南征。吴军压境,交州各郡守无不俯首,士燮率领家族奉承节度。唯有刘表所置苍梧太守吴巨阳奉阴违,最后被步骘發現有異心,隨即斬殺。孙权自此得到交州九郡領有權,并加封臣服于自己的士燮为左将军。南海郡、郁林郡、苍梧郡則是孫權管治,交阯郡、日南郡、珠崖郡、儋耳郡、九真郡、合浦郡則是士夑獨立管治。

漢建安十六年(211年),孙权将治所迁至秣陵。次年,孙权修筑石头城,改秣陵为建业。聽聞曹操率四十萬大軍進攻,孫權打算興建水塢,部將大家都認為直接上下船就能著陸登船,建造水塢沒用,只有呂蒙認為這個塢可以給步兵快速登船進退不失的便利,於是孫權同意呂蒙看法,派呂蒙建造濡须坞作為進出濡須到巢湖的水軍防衛要塞,也是濡須之戰的重要補給據點,與日後曹魏建造的合肥新城是互相對應的防衛設施。

漢建安十八年(213年)正月,曹操親率四十萬大軍攻孫權於濡須口,孫權向劉備發出救援。當時劉備作為客將在劉璋之下,劉備以救孫權為由向劉璋借兵去救荊州關羽,劉璋對劉備猜疑只給他一半軍需和4000兵馬。劉備憤怒劉璋給物資和兵少,隨後密謀反戈偷襲了益州劉璋,劉備也沒有理會孫權求援,任由他們自生自滅。孫權知道劉備攻劉璋而不來救援,出爾反爾,大罵劉備是狡猾的傢伙竟然敢使詐。濡須戰場最後只有孫權軍獨力以七萬大軍抵擋曹操號稱的四十萬大軍,起初戰況不好。孫權出戰沒有得到收穫,濡須口的江西營被曹軍打破並俘虜都督公孫陽,而董襲趕往救援的途中遇溺去世,孫權軍霎時間頓挫。下半場戰鬥,曹操作油船在夜中親率打算襲擊洲上孫權軍。孫權親自率軍乘機反擊,驅使水軍突襲包圍了曹操,曹操軍落水溺死有數千人,俘虜敵兵人數也有三千餘人。孫權乘勝追擊,並對曹操進行多次挑釁,但是曹操受到孫權的打擊下而不敢出擊接戰。孫權見曹操堅守不出,親自督一艘船從濡須塢出擊進入曹操大軍陣地觀陣,《吳錄》記載曹操軍眾人打算射擊孫權的船,但曹操知道孫權來觀陣下令不要妄動,孫權在曹操大營饒了一圈,曹操看到孫權的膽量還有船上士兵器械嚴整,讚歎:「生子當如孫仲謀,劉景升兒子像豬狗」。隨後,孫權下令吹號回營。而《魏略》記載则是說孫權進了曹操軍陣地,曹操命部下拉箭亂射,孫權船身一則被箭矢射滿將要翻船,孫權隨即下令調轉船身擋箭,船身也因此得到平衡,孫權從容地回營。戰鬥已經一個月有餘,曹操仍然無法打敗孫權也無法攻克嚴防的濡須塢,孫權便寫信給曹操說春天水增,你快點走吧,並在另一封信寫上,你如果不死,我不安樂。孫權給曹操一個撤退的下台階,曹操收到信後,對左右說孫權不會騙我,隨即下令撤軍。

漢建安十九年(214年)五月,孙权亲征庐江治所皖城。闰月,在一天的时间内就攻破皖城,俘获庐江太守朱光及参军董和,男女数万人,将北线扩展至合肥一带。劉備得到益州,於是孙权派诸葛瑾向刘备讨还荆州各郡。刘备拒絕,並說得到涼州後再把荊州所有郡歸還(當時益州與涼州之間還有一個漢中,漢中當時是張魯的領地),孫權經過濡須和益州一事後知道劉備的推託假話,隨即派遣魯肅到益陽與關羽對峙,再派吕蒙指挥孫皎、潘璋、呂岱、鲜于丹、徐忠、孙规等领兵二万,攻取长沙、零陵、桂阳三郡。孙权住在陆口,为各路军队的指挥、调度。吕蒙军队一到,长沙、桂阳二郡全部归服,同時通過心理戰把零陵把太守誘至開城投降。最後魯肅和關羽在益陽交鋒對峙,以及談判磋商。這個時候,曹操準備攻取漢中,刘备害怕丢失益州,便派使者求和。两国因此停戰,于是以湘水為界,劉備被逼把长沙、桂阳兩個郡以東還給孫權。江夏郡是孫權攻破黃祖後一直領有並沒有借出,而長沙郡則是生前孫堅所管治的。曹操再伐吳,但是曹操大軍被甘寧100人奇襲而全軍撤退。

215年,孫權北征合肥。孙权作战勇敢,進軍時與數名部將作為先頭部隊率先到達戰場扎寨立營,因為大軍還沒有集結只有數名部將的軍隊,所以被張遼有機可乘突襲成功。而撤军时孫權亦親自與四名部將及1000人在後方穩定士氣,張遼見此率七千人偷襲,當時全軍撤出,兵力只有一千人不如張遼七千人,呂蒙、蔣欽、凌统、甘宁等在逍遥津以北被张辽所袭击,凌统等拼死保护孙权,孙权弓馬嫻熟迎擊張遼,最後骑着骏马飛躍津桥成功撤出。張遼在戰後對孫權的弓射騎術感歎,當魏軍知道這個弓騎勇將是孫權而悔恨沒有捉到他。

漢建安二十二年(216年)冬,曹操再次兴师伐吴,丹阳四郡(今安徽定量城)民帅尤突、费栈受曹操授權联合山越,聚集數萬人起兵反叛。孙权即命賀齊和陆逊进兵征讨。賀齊和陆逊大破尤突及费栈等眾,降服丹阳、吴郡、故鄣等三郡山越,得精兵数万人。曹操屯军至居巢(今安徽巢县东北)準備進軍。孫曹交戰,關羽聯絡長沙郡縣長吳碭、袁龍再次發起叛亂,孫權派鎮守陸口的魯肅前去幫助呂岱,最終平定了叛亂。217年,曹操進攻濡須口,孙权在濡須塢前方築城,但被曹操軍先鋒逼退。之後濡須战线胶着,連日暴雨水面上漲,孫權驅使水軍前進曹操軍非常惶恐,曹操下令撤退。孫權便以吕蒙为都督,與蔣欽共同擔任此戰的總指揮。呂蒙据守之前建成的城坞,并设置万张强弓硬弩,以拒曹操。结果曹军所有先鋒尚未安然立屯,便被吕蒙攻破了,曹操軍敗走退回到居巢,最後攻不下孫權而下令撤軍,曹操自己也引軍撤退。在217年濡須口之戰孫權擊退曹操之後,要著手處理揚州內部的山越問題、自己國家的利益、孫劉關係,所以對漢朝偽降主动与曹操修好,避免日後受到曹操、劉備、內部山越的三方面侵擾。曹操當時被孫權擊敗而引軍撤退,另一方劉備也進軍漢中,因此接受請降。魯肅非常後悔借出荊州給劉備,同時也怒斥劉備、關羽沒有信用,他死後呂蒙接替他在前線總指揮職務,並向孫權提出要警戒關羽,不依靠劉備獨立對抗曹操的建議,孫權經過多年獨立對抗曹操見識過劉備等人的反復態度,於是採納呂蒙提議,與關羽表面交好。

漢建安二十四年(219年),孫權有進攻合肥態勢,魏軍全部州郡的軍隊進入戒備狀態。孫權得知劉備獲得漢中後,再次派諸葛瑾向劉備索還荊州的訴求,但劉備拒絕。孫權打算與关羽以聯姻修好孫劉關係,但關羽以虎女怎能嫁犬子為由拒絕,並怒罵使者。關羽發兵圍攻襄阳曹仁,孫權打算派兵救援,關羽嫌孫權增援太慢大罵:「狢子(對東吳人的貶稱),等我滅了樊城之後回去就把你滅了。」。孫權知道關羽傲慢輕視自己,便寫信道歉。曹操派人聯絡孫權,以荊州為條件希望孫權相助,但孫權沒有馬上答應。後来關羽在樊城之戰俘虜于禁數萬降兵,把所有人送到南郡關押,但是還假借食糧不足為藉口,對吳國湘水邊境侵略而且搶奪軍需糧食。此前,孫權跟呂蒙分析局勢時,孫權想打徐州,但呂蒙認為應該打關羽的荊州,分析認為曹軍多為騎兵擅於陸戰,徐州雖然拿得下來,但也守不住。不如著手準備拿下荊州,完全控制整條長江,對外進可攻退可守,對內下游的吳國也會十分安全。此時孫權對關羽的種種作為已經難以忍受,命吕蒙至陆口實施之前商量好的計劃,並向漢朝廷申請討伐關羽,獻帝同意。十月,孙权西征关羽,以吕蒙陆逊为先锋,孫皎為殿後,孫權則潛軍一同北上。呂蒙以白衣渡江之策計,在夜半時分計破連綿不斷的烽火台屏障,然後再占据南郡,關羽被徐晃等人打敗從樊城回來,知道南郡丟失,隨後撤退到麥城駐守。孫權沒有打算殺關羽的意圖,於是派使者對關羽勸降,關羽假裝答應,在城上立旗後逃跑。孫權知道後派潘璋和朱然截擊,呂蒙當時留在南郡指揮大局,陆逊则另率军攻取宜都郡房陵等。呂蒙通過善政安撫荊州民心,把蜀漢軍家人的情況告訴給關羽軍,頓時間關羽部下失去戰意四散,關羽軍數萬人有的向孫軍投降,有的被孫軍的將軍吸納,最後在臨沮馬忠擒獲了關羽、關平等人。孫權想用關羽制衡曹劉打算再次招降關羽,左右文臣此時對主子孫權說狼子不可養,曹操當年收留關羽,如今換來遷都的惡果。孫權聽後把關羽斬首,並把首級送去給曹操,孫權則以諸侯的禮遇安葬關羽的身軀在當陽。自此荊州南北為曹、孫兩家佔有,于是孙权免除荆州百姓的所有租税。曹操向漢獻帝上表任命孙权为骠骑将军,假节兼任荆州牧,封南昌侯,同時也征召了张承、劉基等人。

漢建安二十五年(220年)年初,魏王曹操及吳大督呂蒙等名將相繼病故。11月,繼承王位的曹丕逼劉協禪讓,正式建號,是為魏文帝。孫權並沒有向曹丕投降,而是命都尉趙咨出使魏国承認曹丕的禪讓帝位並以諸侯身份向他稱臣,再將于禁等敗將送回北方,令新上任需要彰顯功名的魏帝曹丕的虛榮自負的心迅速膨脹,同時解除對孫權的戒心。孙权又派遣趙咨、陈化、冯熙、沈珩为使节,曹丕也派侍中辛毗、尚书桓阶前来东吴与之立誓结盟,曹丕冊封孙权为諸侯藩王吴王,以大将军使持节的身份监督交州,兼任荆州牧,孫權立长子孙登为王太子。當時,群臣勸孫權不應該受封吳王應該自稱九州伯、上將軍,孫權則說當年劉邦也是受封了項羽的漢王,最後還是成就了偉業。曹丕處事浮華,在他守喪期間向孫權索求雀頭香、大貝、明珠、象牙、犀角、玳瑁、孔雀、翡翠、鬥鴨、長鳴雞,吳群臣聽後說這些是珍稀貴重物勸孫權不要給,孫權則認為這只是瓦片石頭罷了,並不介意。期間在外交上一直由孫權主導,曹丕過於天真相信孫權而拒絕劉曄順江而下伐吳的建議,還几次拒绝了大臣们的伐蜀的提案。

劉備宣稱獻帝被害,於建安二十六年(221年)4月也登基稱帝。同年7月,借以關羽報仇的名義討伐孫權欲吞併江東領土發動猇亭之战,孫權自公安迁都鄂縣,改名武昌進入備戰,以六县设置武昌郡。孫權讓諸葛瑾寫信給劉備勸說他不要開戰希望和睦相處,並陳說利害分清楚敵人主次,不要上了曹魏的當,如果真要開戰他們也不會手軟,劉備不聽。孫權派周泰準備向白帝城作攻防姿態,任命陆逊为大都督,率领朱然、韓當、駱統、潘璋、孫桓等领兵前往抵抗。黃初三年(222年)六月,陆逊彻底击败蜀军。蜀军被斩杀和放下武器投降者有几万人。刘备被孫桓追至差點被擒獲,最後仅保得自身不死。當時,徐盛、潘璋、宋謙等人認為只要繼續追擊劉備,必能把他殺掉,但陸遜、朱然、駱統等認為不要追擊,他們察覺到曹丕有進攻江東的態勢。而孫權根據自己的判斷,採納陸遜等人的看法,下令不要對逃往白帝城方向的劉備展開追擊。

夷陵之戰期間,孫權一直在自己領地橫江屯兵提防曹魏,果然如陸遜等所料,曹丕派曹休襲擊孫權的領地曆陽及蕪湖,曹丕打算控制孫權,要求孫權將孫登送到魏國都城做人質,孫權知道其用意所以以藉口多次推辭,最後曹丕發覺孫權誠心不款,於是有意發兵攻打江東。夷陵之戰一結束,吴魏之间就開始有交戰態勢,曹丕派出三十萬大軍,命令曹休、张辽、臧霸出兵洞口,曹仁出兵濡须口,曹真、夏侯尚、张郃、徐晃率军围攻南郡。當時要處理揚州境內的山越問題,孙权故意示弱,謙卑上書誘騙曹丕,只是為了拖延曹丕進軍期限爭取平定山越內亂的時間,曹丕則相信他送兒子而沒有進軍,而另一方,孫權則派遣吕範、朱然、朱桓率領其他部將暗中部署。黃初三年(222年)十月,曹丕見孫權沒有送兒子來,正式開戰。於是孙权改年号为黄武,同曹魏斷絕來往。孫權命呂範率領徐盛、孫韶、全琮、賀齊等人在水路抵御曹休等,孫盛、诸葛瑾、潘璋、杨粲前往南郡增援朱然,朱桓接替周泰以濡须督的身份在濡須塢抵擋曹仁。三方面戰鬥中,曹仁被朱桓多個戰術配搭被打得大敗,曹休、張遼、臧霸則是強弩之末被徐盛、全琮、賀齊等人反擊而敗退,曹真、夏侯尚、徐晃、張郃則團團圍攻江陵,但久攻不下朱然。最終在次年三月春魏軍全部撤走,江南國境皆得安寧。另一邊,孫權收到在白帝城休養的劉備的求和信後,於十二月派遣太中大夫鄭泉出使蜀汉,蜀、吴两国自此重結盟好。

黃武二年(223年),孙权在江夏修筑山城。改用乾象历。夏四月,孙权的大臣们进劝他称帝,孙权不答应。此前,魏國令吳領地的戲口太守晉宗造反殺同僚王直,騷擾江南國境,孫權因為三方面大戰分身不下而不能馬上消滅他。六月孫權派賀齊、胡綜等人率軍平定,最終擒獲晉宗。劉備死後,諸葛亮派鄧芝向東吳再次確立聯盟關係,孫權知道諸葛亮用意,也非常器重鄧芝,所以答應修好。孫權便斷絕同曹魏來往,派辅义中郎将张温回訪蜀漢。

黃武五年(226年),孙权下令各州郡守,对百姓实行宽容安息政策。这时陆逊因驻守的地方缺粮,上表孙权,命令诸将广开农田。七月,孙权听说魏文帝曹丕去世,兴兵征讨江夏郡,围攻石阳城,卻久攻不下。江夏郡高城被孫奐攻陷。孙权任命全琮为东安郡太守,讨伐山越的反叛。孙权分交州另置广州,不久又复合为交州。

黃武七年(228年)五月,孙权命鄱阳太守周鲂以斷髮詐降,假装叛离东吴,引诱魏将曹休。爆发石亭之战,秋八月,孙权前往皖口,派征西将军陆逊率领朱桓、全琮在石亭大败曹休。

黃龍元年(229年)夏四月十三日丙申(5月23日),統治江東三十年的孙权在南郊正式登基为帝,改年号为黄龙。四月,孫權大赦改年,在南郊拜天,即皇帝位,諸葛亮派衛尉陳震去東吳祝賀孫權登皇帝位,3個月後孫權把國都從武昌遷回建業。追谥父亲孙坚为武烈皇帝,母亲吴氏为武烈皇后,長兄孙策为长沙桓王。立吴王太子孙登为皇太子。将军官吏都晋爵加赏。六月,蜀国派人前来庆贺孙权登基。孙权還禮,承認東西二帝共存,並与蜀漢使節商议平分天下。其中,豫、青、徐、幽四州属吴;兖、冀、并、凉四州歸蜀。司州的土地,以函谷关为界分属两国,雙方制定盟书,共同声讨曹叡。秋九月,孙权將都城從武昌遷到建業(今江蘇省南京市),就住在原来的府第中,不再另建新宫殿,征召上大将军陆逊辅佐太子孙登,掌管武昌事宜。

孙权即位後,曾多次派人出海。黃龍二年(230年),他派衛溫、諸葛直等航行到達夷洲;242年,他又派聶友等航行到珠崖儋耳(指現今的海南島)。

嘉禾元年(232年),孙权派遣将军周贺等航海到辽东。十二月,辽东太守公孙渊向孙权称藩。

嘉禾二年(233年),孙权派太常张弥、贺达等万人,带上金银财宝奇货异物,加上九锡,经海路送给公孙渊。举朝大臣全都规劝孙权,认为公孙渊其人不可信,对他的恩宠礼遇不要太过分。孙权一意孤行,没有接受规劝。后来公孙渊果然将张弥等杀死,以其首級並東吳賜予的金印送往曹魏邀功。孙权聞之,大感慚恨,企图亲自征讨公孙渊,尚书仆射薛综等极力谏阻,最终中止了这个计划。

嘉禾三年(234年)二月,諸葛亮再次與兵北伐。诸葛亮集中在漢中十萬大軍全部出動,木牛流馬,運糧不停,同時相約東吳東西並舉。五月,東吳出兵,七月退兵。孙权下诏放宽徭役,夏五月,孙权派遣陆逊、诸葛瑾等驻军江夏、沔口,派孙韶、张承等进军广陵、淮阳,孙权自己亲率大军进围合肥新城,爆发合肥新城之战后退兵回返,孙韶也停止进军广陵等地。秋八月,孙权任命诸葛恪为丹杨太守,讨伐山越部族。次年,孙权派吕岱领兵讨伐贼寇李桓等。

嘉禾六年(237年),孙权让群臣讨论奔丧立科、丞相顾雍奏请违法奔丧应处以死罪。此后吴县县令孟宗违法奔母丧归家,事后在武昌将自己拘禁起来听候处罚。陆逊向孙权说明孟宗的平时作为,并借机为孟宗求情,孙权于是给孟宗减刑一等,并申明下不为例,于是违法奔丧的事绝迹。

赤烏元年(238年),改年号为赤乌。当时,孙权利用吕壹打擊豪族,吕壹本性苛刻残忍,执法严酷。太子孙登屡次进谏,孙权都不采纳,大臣们于是都不敢进言。后来吕壹奸邪的罪行败露被处死,孙权自我批评,认错误,派中书郎袁礼代自己向曾經規勸但未被採納的大臣們致歉。

赤烏二年(239年),公孙渊不滿曹魏對其待遇不高,便又復叛魏國,自立为燕王,结果受到魏国司马懿攻击。公孙渊派遣使者向吴国求助。当时吴人都对公孙渊的反复无常历历在目,劝说孫權斩杀使者。唯有羊衜说:“陛下,斩首公孙渊的使者固然能讓您出口恶气,可这样做是出了匹夫的怒气,而放弃了霸王之计。臣以为,朝廷不如借此机会,出奇兵前往以观动静。如果魏国进攻公孙渊失败,那么我军远赴辽东解救,是恩结于远夷,义盖于万里;如果魏军和公孙渊相持不下,公孙渊首尾不能相顾,那我军正好进攻辽东,这样也足以让上天惩罚公孙逆贼,一雪往日之耻。”羊衜此言深得孙權赞许,于是派遣使者羊衜、郑胄、将军孙怡以海军前往辽东,击败魏国守将张持、高虑,並俘虜當地居民南還。十月,孙权派遣将军吕岱、唐咨前往剿滅少数民族的叛乱,将他们全部屠戮。

赤乌四年(241年),太子孙登去世。孙权後立三子孙和为太子。孙权听从百官封建诸子的意见,又立孙和之弟孙霸为鲁王。但孙霸始终不服孙和。遂召集手下宾客及结交诸大臣,常与围绕在孙和一侧的太子党分庭抗礼。赤乌十三年(250年),孙权决定废黜太子孙和并赐死鲁王孙霸,同时改立七子孙亮为皇太子。第二年册立孙亮之母潘氏为皇后。

太元元年(251年),冬十一月,孙权祭祀南郊回来后,就因風疾(相當於今稱中風)生病卧床。十二月,遣驿使传书召大将军诸葛恪回京,拜为太子太傅,孙权下诏省徭役、减征赋,将将国家大事交给诸葛恪管理,并修改诸多不便法令。

太元二年(252年),孙权立原太子孙和为南阳王。五子孙奋为齐王。六子孙休为琅琊王。二月,大赦,改年号为神凤。神凤元年四月廿六日(公元252年5月21日),孫權於太初宮内殿中驾崩,享壽六十九歲。滕胤与太子太傅诸葛恪、少傅孙弘、荡魏将军吕据、侍中孙峻等人一同受遗诏辅佐太子。孙权稱帝后在位23年。葬於建業蔣陵,謚大皇帝,廟號太祖。

孫權擔任家督弱冠繼承江南政權以来,對外招納人才培養部下,以懷柔策略籠絡不服從孫家的江南豪族,以白手興家統合內部對抗外壓,鞏固孫家政權在江南的地位,另一方面通過平定揚越叛亂進行強兵吸收老幼弱者補戶的政策,同時給予落後山越民提供漢文化的學習。諸侯時期的孫權不屬於任何一方,也沒有所謂的興漢滅漢的政治口號,故此可以根據時勢局勢發展,判斷哪一方有利用價值,並進行聯合的自由外交戰略獲取自己的利益。作為開創基業的帝王,孫權以出色的政治智慧及戰略判斷,深諳縱橫捭闔,最終締造一方霸業。赤壁之戰後加強控制江東,并将江東六郡扩展到揚、荆、交三州,積極開發南方的荒蕪之地,穩健控制中國東南。

孫權在數年間將國土政權安定,以適才適所為第一原則,而不以輩分、資歷、交情、名氣為優先,深知人無完人,故此不追究缺點而用其優點的用人風格,處事也是嚴罰主義者,就算親族或功臣的家族犯罪,也會給予嚴刑處分。例如選擇寒門出身的周泰為平虜將軍,與孫權為同窗的朱然则身居其下,在夷陵之戰任命资历尚浅的陸遜为大都督,許多人因是孫策舊將或者公室貴戚,一度有所不滿,但最终都心服口服,步骘虽然出身豪族,但是避難到江東而家道中落,最终竟做到丞相一职。孙权亦能主动培养部下,同時對待功臣的態度是忘其短而貴其長。孙权以顧雍為丞相而非眾人所推薦的張昭,就是因为丞相位置處理的事情多且繁重,而張昭性情剛烈固執,不遵從他的意見則會埋怨歸咎到底,到時反而對公事沒有益處。

孫權崇尚節儉,並效法大禹以卑宮為美,原本住的建業宮其實只是孙权早期的將軍府而已,一直住到赤烏十年建材腐朽,還詔令將武昌宮拆了,把木材運來建業修繕,但其實當時武昌宮也有二十八年的歷史不堪使用,這麼做的目的是節省木料避免妨礙農桑工作,由此也可知孫權對農業的重視。陆凯向孙皓劝谏时也称孙权时代“后宫列女,及诸织络,数不满百,米有畜积,货财有余”。

孫權執法嚴格,即使面對至親也是法律優先從不循私。孫輔因通敵而被流放、庶弟孫朗因違反軍令燒毀自軍軍用而被呂範送回,於是改姓丁並禁錮終身、愛子孫霸更因圖危太子,而被賜死。另一方面,孙瑜孙桓孙韶等孙氏宗室或委以重任,女则嫁于国家重臣,即使是谋反者的后代也能不计前嫌。可见孙权实际上对于同族亲戚相当重视。陈寿因此赞曰“况此诸孙,或赞兴初基,或镇据边陲,克堪厥任,不忝其荣者乎”。

对内廣納諫言,任用父兄旧部稳定局面,平定叛徒和山越,攻滅殺父仇人黃祖,吸納北方難民。在一片降曹之声时果斷与曹操一战並與刘备结盟,任用周瑜打败曹操穩定江南地盤,後來因為劉備一連串背離同盟關係的所作所為及荊州等问题而與蜀汉決裂,連本帶利奪回荊州並在夷陵之战重創劉備,最終確立了三分天下的局面。黄龙元年(229年),孫權于武昌称帝,建国为吴,孙吴建立。称帝以后他分部諸將,鎮撫山越,增設縣邑,編制戶籍,設置農官,推行軍屯與民屯;收容南遷移民,興修水利,增廣農田;親自下田採用牛耕,大幅度改良農業生產技術,大興佛教,奠定了六朝的經濟與文化基礎。

在晚年大批豪族過分插手孫權的家事,而且分黨立派,造成政局动荡不安,孫權對此非常不滿。之後孫權得知自己繼承人意向的消息外洩之後大為憤怒,並將相關人員等捉拿問罪。。後來孫權遭到豪族暗殺及背叛,所以在二宮之戰爆發後,通過部下彈劾而削弱豪族權力來鞏固政權,之後難能可貴的是孫權同時也具有認錯的勇氣,從陸遜之子陸抗的任職態度以及陸機所著《辯亡論》看來,孫陸二家情誼仍然十分深厚,陸氏對孫權亦持肯定態度,不因孫權老年冷酷而有所怨言。神凤元年(252年)夏四月,孫權在内殿驾崩,终年七十一岁。

由於孫權大力開拓海上事業並且開拓江南,因此在中國史上有非常重要的地位,然而他死後的待遇與他的功績完全不成正比,詩人曾極在其作品《吳大帝陵》中提到“四十帝中功第一,壞陵無主使人愁”,劉克莊也在《吳大帝廟》中嘆息“今人渾忘卻,江左是誰開”。

東吳的部曲私兵和世襲制度是有利有弊,執政者需要確立王朝威信的時候,有利於鞏固自己的君主地位;當執政者權威衰微的時候,威權容易被擁有私兵的部下奪取。因此孙权晚年後世評價兩極,一方認為他晚年昏庸而做出一連串錯誤導致王朝衰退;而一方則認為孫氏在江東的權力較弱,所以孫權晚年處事冷酷無情,通過制約擁有政治影響力及私兵權部曲世襲制的豪族,把權力集中在孫氏家族手上,避免如同魏國一樣被士族奪權蠶食政權的情況發生,從而確保孫氏政權現況及未來的威權地位,只是後來的權力者不能維繫這個政權發展而導致逐漸衰落。

北方戰亂,孫權也吸納南渡的北方民眾,其北方的手工技術也在江南得到發揚及應用。另外由於孫權積極擴張海上事業,並曾發兵遼東,因此江南造船業大大興盛。

首都建業原名秣陵,最初是一小縣,因孫權定都建業並開鑿運河而成為一流都市,被稱為六朝古都,現名南京。

《吳曆》曰,黃武四年,扶南諸外國來獻琉璃。這是中國最早與南海諸國交流的記載。孫權主動派出朱應與康泰出訪南海各國,先後到過林邑(今越南中南部)、扶南(今柬埔寨)、西南大海州(今南洋群島)、大秦(羅馬)、天竺(今印度),並記下各國物產以利貿易奠定了南海貿易的基礎,回國後,二人分別撰寫《扶南異物志》及《外國傳》(又稱《吳時外國傳》),之後繼續派出使節進行南國宣化,同扶南、林邑、堂明(今柬埔寨)建立關係。,這是史無前例的事情,雖然南海諸國之前已與中國有接觸,但是由官方政府主動派出官員積極尋求國際貿易,孫權卻是創舉,貿易的範圍甚至到達了羅馬,並在建業接見了羅馬商人秦論。

《江表传》中记载孙权方頤大口,眼神很有光彩。漢朝遣使者劉琬為孫策加錫命之時看見孫權,形容孫權的相貌高大挺拔。刘备和张辽都看到孙权坐着时显得很高,认为他躯干较长,中国民国学者黎东方分析,只有不需要站著伺候人,而是坐著讓人伺候的貴人才會是所謂軀幹長而雙腿短的外形,古代这被視為大貴之相,劉備被形容為手長過膝也是基於同樣的道理。

《三國演義》里孫權则被记载“碧眼紫髯,堂堂一表人才”。

在閻立本《十三帝王圖》之中,孫權為站姿,此為開國之君之意,身著的冕服應有天子十二章,在圖中有被畫出來的有「日、月、藻、火、黼」五章 ,其中日月為明,明火三章表示的是孫權振興經濟,教化百姓,讓其光明之面普照天下之意,藻則代表孫權稱帝隨時代順應天意而起,黼則表示孫權「能斷割」,這與三國志中孫權好俠養士仁而多斷的人格特質以及遇曹操來攻能拒絕臣服決心抗曹、遇劉備來攻則稱臣曹丕以保全江東等正確的重大決斷相呼應,圖中孫權手持麈尾扇,表現了他的帝王風度,為十三帝王圖中唯一持扇者,“麈”是領隊大鹿尾,魏晉以來,尚清談,手執麈尾有“領袖群倫”含意,藝文類聚亦記載司馬懿見諸葛亮乘素輿、葛巾毛扇指揮三軍,嘆諸葛亮為名士,諸葛亮在《三國演義》中也常持羽扇指揮軍隊,扇子有善戰之意,因此蘇軾在《念奴嬌》形容周瑜時亦說周瑜「羽扇綸巾」談笑間強虜灰飛煙滅,孫權在位期間,赤壁與夷陵之戰均以少勝多,甚至取荊州而兵不血刃,足見他用人正確調度有方的善戰特質,因此辛棄疾會說「天下英雄誰敵手?曹劉,生子當如孫仲謀」。

孙权性格旷达开朗,仁爱明断,喜欢供养贤才,因此很早就与父兄齐名。由于非常重视集体的力量,能毫无保留地信任臣下,甚至部下死後代為教養其孤兒贍養其妻儿及其父母。也會調解部屬糾紛,亦下诏勿杀叛逃将领的妻子子女。孙权与臣下的亲密关系也体现在称呼其表字上,甚至是对于初见的潘濬,曾與陸遜當眾對舞,又将自身所穿衣物皆赐之。对于他国贤才,孙权也毫不掩饰地表达喜爱,如诸葛亮费祎邓芝宗预等。孫盛因而稱許孫權盡心關愛部下,令其甘心為自己拼命,是東吳能夠立於江東的原因。

孙权天性活潑奔放,能言善辩,常常肆无忌惮地恶作剧、戏弄人,经常开些无关紧要的玩笑,即使是面对蜀汉来使也不例外。其本人亦参与配合部下的戏谑。

孫權的忍辱負重性格在向曹操與曹丕稱臣時一覽無遺。因此臥薪嘗膽一詞出自蘇軾的《擬孫權答曹操書》,也因為忍辱負重,所以孫權面對荊州問題時選擇與蜀國結盟,而不與劉備爭鬥,避免曹操坐收漁翁之利。三國之中,也是東吳最晚稱帝。陳壽亦提過「孫權屈身忍辱,任才尚計,有句踐之奇英,人之傑矣」,趙咨答曹丕時亦說「屈身於陛下,是其略也」。因孫權處世手段極其柔軟,曹丕也曾以嫵媚形容孫權,所以有詩歌詠孫權時說「孝廉嫵媚還能霸」。

孙权善于判断国内外人物局势。如认定魏延和杨仪会在诸葛亮死后内讧。也预见到曹魏亡国的先兆。

孙权擅长骑術和弓術,在合肥面對張遼的突襲能平安躍馬過橋,他的弓術也给张辽留下了很深的印象。

孙权有六口宝剑,分别是白虹、紫电、辟邪、流星、青冥、百里。

關於孫權嗜好,其中射猎(射虎、射雉)與好酒和开宴会派对尤其出名。每当猛兽近前,孙权总是以亲手击打为乐趣。宴会中常常对部下进行劝酒,孙权喜好冒险,如顶着大风天坐船出航,乘轻船去见曹操军队, 密令甘宁夜袭曹营等等。

孙权也喜爱读书,据其本人所言,其所涉猎内容涵盖《诗经》、《尚书》、《礼记》、《左传》、《国语》及三史(《史记》、《汉书》和《东观汉记》),惟不曾研读《周易》,孙权在书法上亦有成就,被认为擅长行书和草书。

在宗教方面,孙权早年信仰道术,与诸多方术之士交往甚密。主要人物为吴范、刘惇、赵达、姚光、介象等人。而被后世尊为道教天师的葛玄也与孙权有过交往。孙权也对当时的新宗教佛教非常开明,赤乌年间为高僧康僧会建立建初寺。

孫權因在家事上随心所欲,表现的不在乎上下尊卑而招致陳壽的批評,称其可比拟春秋时代的齐桓公,对外“有识士之明”,对内却“嫡庶不分,闺庭错乱”,最终在繼承人問題上埋下祸根,導致很长一段时间内国家都动荡不安。裴松之则意見相反,認為孫權廢掉無罪的太子,雖然是開啟禍亂的前兆,但最多只是東吳滅亡的次因而非主因,畢竟東吳滅亡已是孫權死後二十八年的事情,而且滅亡主因仍是暴君孫皓,即使孫權當時傳位於孫和,最後也是孫皓登基,國之滅亡的根本問題其實是出在為政者昏虐,並非只有孫權廢黜一事就能造成,如孫亮能保住國祚,或者孫休不早死,都不至於讓東吳滅亡。陸遜的孫子陸機更著有《辯亡論上》《辯亡論下》詳細說明東吳亡國非因蜀國滅亡,而是孫權死後的當政者用人不當。

陳壽於《三國志》认为孫策為開國奠基人但子孙未被封为王爵,孫權於義儉矣。后人据此穿凿附会,认定孙权对孙策有所怠慢,从尚书仆射存和胡综的上书可知孙权以谦虚为美德,不愿效仿汉代旧制过分尊崇皇族,就连孙权自己的皇子也不例外,如被孙权宠爱的次子孫慮也止在侯爵。另一爱子孫和在十九岁封为太子前也从未获得任何爵位。群臣请立孙权余下四子为王时也被孙权拒绝。

孫盛从国家大局方面对陈寿的看法也表示了不同意见,认为當時天下局勢尚未統一,宜正名定本貴賤疏邈,不宜給與孫策之子更高的權力與爵位製造內亂機會,此為穩定局勢之必要行為,況天倫篤愛,孫權既已將孫策宗廟立於建業,應不會刻意吝於給予地位,這明顯是為了穩定國家局勢的必要處置方式。

从实际史料出发,孙权反倒有相当多不忘旧情的举动,如孙盛所言为孙策建庙于建业并派太子祭祀。在赤乌年间再次为孙策进行厚葬,因吕范往日对其兄的帮助而对之大加溢美,以致严峻私下认为夸大其词了,直到后来才信服。

孫策臨終傳權時:「舉江東之眾,決機於兩陳(陣)之間,與天下爭衡,卿(孫權)不如我。舉賢任能,各盡其心,以保江東,我不如卿。」(《三國志·吳書·孫破虜討逆傳第一》)

曹操於濡須之戰:「生子當如孫仲謀,劉景升(劉表)兒子若豚犬耳!」(《三國志·吳書·吳主傳第二》裴松之註引《吳歷》)於孫權稱臣時「此兒欲踞吾著爐炭上邪!」(《晉書·宣帝紀第一》)

刘备:「孙车骑长上短下,其难为下,吾不可以再见之。」

关羽:「鰂子敢爾,如使樊城拔,吾不能滅汝邪!」(《三國志·蜀書·關張馬黃趙傳第六》)

周瑜:「將軍以神武雄才,兼仗父兄之烈,割據江東,地方數千里,兵精足用,英雄樂業,尚當橫行天下,為漢家除殘去穢。」(《三國志·吳書·周瑜魯肅吕蒙傳第九》)「今主人亲贤贵士,纳奇录异。」

魯肅:「将军神武命世。」「孫討虜聰明仁惠,敬賢禮士,江表英豪,咸歸附之」(《三國志·蜀書·先主傳第二》裴松之註引《江表傳》)

張紘:「自古帝王受天命的君主,雖有皇靈在上輔佐,文德傳播天下,也要靠武功顯著。要開墾種植,任賢使能,務崇寬惠,順天命去誅討,這樣不勞師眾定天下。」

陸遜:「陛下(孫權)以神武之姿,涎膺期運,破操(曹操)烏林,敗備(劉備)西陵,禽羽(關羽)荊州,斯三虜者當世雄傑,皆摧其鋒。」(《三國志·吳書·陸遜傳第十三》)

諸葛亮:「海內大亂,將軍(孫權)起兵據有江東,劉豫州亦收眾漢南,與曹操并爭天下。今操芟夷大難,略已平矣,遂破荊州,威震四海。英雄無所用武,故豫州遁逃至此。將軍量力而處之:若能以吳、越之眾與中國抗衡,不如早與之絕﹔若不能當,何不案兵束甲,北面而事之!今將軍外託服從之名,而內懷猶豫之計,事急而不斷,禍至無日矣!」(《三國志·蜀書·諸葛亮傳第五》)「權有僭逆之心久矣」(《三國志·蜀書·諸葛亮傳第五》裴松之註引《漢晉春秋》)「孫將軍可謂人主,然觀其度,能賢亮而不能盡亮,吾是以不留。」(《三國志·蜀書·諸葛亮傳第五》裴松之註引《袁子》)「孙权据有江东,已历三世,国险而民附,贤能为之用。」「议者咸以权利在鼎足,不能并力,且志望以满,无上岸之情,推此,皆似是而非也。何者?其智力不侔,故限江自保;权之不能越江,犹魏贼之不能渡汉,非力有馀而利不取也。」

司馬懿:「權之稱臣,天人之意也。」(《晉書·宣帝紀第一》)

张辽:「向有紫髯将军,长上短下,便马善射。」

程昱:「权有谋。」(《三国志·魏书 ·程郭董刘蒋刘传第十四》)

陈琳:「夫天道助顺,人道助信,事上之谓义,亲亲之谓仁。盛孝章,君也,而权诛之,孙辅,兄也,而权杀之。贼义残仁,莫斯为甚。乃神灵之逋罪,下民所同雠。辜雠之人,谓之凶贼。」(《檄吴将校部曲文》)

彭羕:「仆昔有事於诸侯,以为曹操暴虐,孙权无道,振威闇弱,其惟主公有霸王之器,可与兴业致治,故乃翻然有轻举之志。」(《三国志·卷四十·蜀书十·刘彭廖李刘魏杨传第十》)

趙咨:「聰明仁智,雄略之主也」、「納魯肅於凡品,是其聰也;拔呂蒙於行陳,是其明也;獲於禁而不害,是其仁也;取荊州而兵不血刃,是其智也;據三州虎視於天下,是其雄也;屈身於陛下(曹丕),是其略也。」(《三國志·吳書·吳主傳第二》)

贾诩:「孙权识虚实,陆议见兵势。据险守要,泛舟江湖,皆难卒谋也。用兵之道,先胜后战,量敌论将,故举无遗策。臣窃料群臣,无备、权对,雖以天威臨之,未見萬全之勢也。」(《三国志·魏书·荀彧荀攸贾诩传第十》)

邓芝:「大王命世之英。」

刘晔:「權無故求降,必內有急。權前襲殺關羽,取荊州四郡,備怒,必大興師伐之。外有強寇,眾心不安,又恐中國承其釁而伐之,故委地求降,一以卻中國之兵,二則假中國之援,以強其眾而疑敵人。權善用兵,見策知變,其計必出於此。」、「權雖有雄才,故漢驃騎將軍南昌侯耳,官輕勢卑。士民有畏中國心,不可強迫與成所謀也。不得已受其降,可進其將軍號,封十萬戶侯,不可即以為王也。夫王位,去天子一階耳,其禮秩服御相亂也。彼直為侯,江南士民未有君臣之義也。我信其偽降,就封殖之,崇其位號,定其君臣,是為虎傅翼也。權既受王位,卻蜀兵之後,外盡禮事中國,使其國內皆聞之,內為無禮以怒陛下。」(《三國志·魏書·程郭董劉蔣劉傳第十四》)

冯熙:「吴王体量聪明,善于任使。赋政施役,每事必咨。教养宾旅,亲贤爱士。赏不择怨仇,而罚必加有罪。臣下皆感恩怀德,惟忠与义。带甲百万,谷帛如山。稻田沃野,民无饥岁。所谓金城汤池,强富之国也。」

刘基:「大王以能容贤蓄众,故海内望风。」

钟繇:「顾念孙权,了更妩媚。」(《三國志·魏書·鍾繇華歆王朗傳第十三》)

刘琬:「吾观孙氏兄弟虽各才秀明达,然皆禄祚不终,惟中弟孝廉,形貌奇伟,骨体不恒,有大贵之表,年又最寿,尔试识之。」

陳壽:「孫權屈身忍辱,任才尚計,有勾踐之奇,英人之傑矣。故能自擅江表,成鼎峙之業。然性多嫌忌,果於殺戮,暨臻末年,彌以滋甚。至於讒說殄行,胤嗣廢斃,豈所謂賜厥孫謀以燕冀於者哉?其後葉陵遲,遂致覆國,未必不由此也。」(《三國志·吳書·吳主傳第二》)「割據江東,策之基兆也,而權尊祟未至,子止侯爵,於義儉矣。」(《三國志·吳書·孫破虜討逆傳第一》)

陆凯:「自昔先帝时,后宫列女,及诸织络,数不满百,米有畜积,货财有余。先帝崩后,幼、景在位,更改奢侈,不蹈先迹。」(《三国志·吴书·潘濬陆凯传第十六》)

孙楚:「吴之先主,起自荆州,遭时扰攘,播潜江表,刘备震惧,逃迹巴岷,遂依丘陵积石之固,三江五湖,浩汗无涯,假气游魂,迄于四纪,二邦合从,东西唱和,卒相扇动,拒捍中国。」

陸機:「吳桓王基之以武,太祖(孫權)成之以德,聰明睿達,懿度深遠矣。其求賢如不及,恤民如稚子,接士盡盛德之容,親仁罄丹府之愛。拔呂蒙於戎行,識潘濬於系虜。推誠信士,不恤人之我欺;量能授器,不患權之我逼。執鞭鞠躬,以重陸公之威;悉委武衛,以濟周瑜之師。卑宮菲食,以豐功臣之賞;披懷虛己,以納謨士之算。故魯肅一面而自讬,士燮蒙險而效命。高張公之德而省游田之娛,賢諸葛之言而割情欲之歡,感陸公之規而除刑政之煩,奇劉基之議而作三爵之誓,屏氣跼蹐以伺子明之疾,分滋損甘以育凌統之孤,登壇慷慨歸魯肅之功,削投惡言信子瑜之節。是以忠臣競盡其謀,志士鹹得肆力,洪規遠略,固不厭夫區區者也。故百官苟合,庶務未遑。」(《辯亡論》下)「用集我大皇帝,以奇踪袭於逸轨,叡心发乎令图,从政咨於故实,播宪稽乎遗风,而加之以笃固,申之以节俭,畴咨俊茂,好谋善断,东帛旅於丘园,旌命交于涂巷。故豪彦寻声而响臻,志士希光而影骛,异人辐辏,猛士如林。於是张昭为师傅,周瑜、陆公(陆逊)、鲁肃、吕蒙之畴入为腹心,出作股肱;甘宁、凌统、程普、贺齐、朱桓、朱然之徒奋其威,韩当、潘璋、黄盖、蒋钦、周泰之属宣其力;风雅则诸葛瑾、张承、步骘以声名光国,政事则顾雍、潘濬、吕范、吕岱以器任干职,奇伟则虞翻、陆绩、张温、张惇以讽议举正,奉使则赵咨、沈珩以敏达延誉,术数则吴范、赵达以禨祥协德,董袭、陈武杀身以卫主,骆统、刘基强谏以补过,谋无遗算,举不失策。故遂割据山川,跨制荆、吴,而与天下争衡矣。」(《辯亡論》上)

华谭:「赖先主承运,雄谋天挺,尚内倚慈母仁明之教,外杖子布廷争之忠,又有诸葛、顾、步、张、朱、陆、全之族,故能鞭笞百越,称制南州。」。「吴武烈父子皆以英杰之才,继承大业。今以陈敏凶狡,七弟顽冗,欲蹑桓王之高踪,蹈大皇之绝轨,远度诸贤,犹当未许也。」

裴松之:「孙权横废无罪之子,为兆乱。」「权愎谏违众,信渊意了,非有攻伐之规,重复之虑。宣达锡命,乃用万人,是何不爱其民,昏虐之甚乎?此役也,非惟闇塞,实为无道。」

孙盛:「盛闻国将兴,听於民;国将亡,听於神。权年老志衰,谗臣在侧,废适立庶,以妾为妻,可谓多凉德矣。而伪设符命,求福妖邪,将亡之兆,不亦显乎!」「观孙权之养士也,倾心竭思,以求其死力,泣周泰之夷,殉陈武之妾,请吕蒙之命,育凌统之孤,卑曲苦志,如此之勤也。是故虽令德无闻,仁泽(内)著,而能屈强荆吴,僭拟年岁者,抑有由也。然霸王之道,期於大者远者,是以先王建德义之基,恢信顺之宇,制经略之纲,明贵贱之序,易简而其亲可久,体全而其功可大,岂委璅近务,邀利於当年哉?语曰“虽小道,必有可观者焉,致远恐泥”,其是之谓乎!」

虞溥:「性度弘朗,仁而多断,好侠养士,始有知名,侔于父兄矣。」(《三國志·吳書·吳主傳第二》裴松之註引《江表傳》)

《荆州先德传》:“权好嘲戏以观人。”

王勃:「孙仲谋承父兄之余事,委瑜肃之良图,泣周泰之痍,请吕蒙之命,惜休穆之才不加其罪,贤子布之谏而造其门。用能南开交趾,驱五岭之卒;东届海隅,兼百越之众。地方五千里,带甲数十万。」

朱敬则:「孙仲谋藉父兄之资,负江海之固,未敢争盟上国,竞鹿中原,自守未馀,何足言也。」(《全唐文》)

徐夤:「一主参差六十年,父兄犹庆授孙权。不迎曹操真长策,终谢张昭见硕贤。建业龙盘虽可贵,武昌鱼味亦何偏。秦嬴谩作东游计,紫气黄旗岂偶然。」

司马光:「文帝承父兄之烈,师友忠贤,以成前志,赤壁之役,决策定虑,以摧大敌,非明而有勇能如是乎?奄有荆扬,薄于南海,传祚累世,宜矣。」(《历代名贤确论·卷五十七》)

苏轼:「亲射虎,看孙郎。」(《江城子·密州出猎》)「孙权勇而有谋,此不可以声势恐喝取也。」

苏辙:「吴大帝方其属任贤将,抗衡中原,曹公惮之。及其老也,贤臣死亡略尽,喜诸葛恪之劲悍,越众而付以后事。闼其用兵劳民之后,继起大役,兵折于外,既归而不能自克,将复肆志于僚友。恪既以丧其躯,而孙氏因之三世绝统,吴、越之民陷于炮烙之地,国随以亡。彼以进取之资用进取之臣,以徼一时之功可耳,至于托六尺之孤,寄千里之命,而亦属之斯人,其势必至是哉。」(《栾城后集·孙仲谋》)「今夫曹操、孙权、刘备,此三人者,皆知以其才相取,而未知以不才取人也。世之言者曰:孙不如曹,而刘不如孙。」

謝采伯:「孫權運籌於內,劉備、諸葛亮、周瑜、關侯等,合謀並智,方拒得曹操,敗之於赤壁,亦未為竒政縁。」

何去非:「权之勇决进取,无以逮其父兄,然审机察变,持保江东,于权有焉。」(《何博士备论》)

辛弃疾:「千古江山,英雄无觅,孙仲谋处。」(《永遇乐·京口北固亭怀古》)「何处望神州,满眼风光北固楼,千古兴亡多少事,悠悠。不尽长江滚滚流。 年少万兜鍪,坐断东南战未休,天下英雄谁敌手,曹刘。生子当如孙仲谋。」

吕祖谦:「孙权起于江东,拓境荆楚,北图襄阳,西图巴、蜀而不得。北敌曹操、西敌刘备,二人皆天下英雄。所用将帅,亦一时之杰。权左右胜之而后能定其国。及权国既定,曹公已死,丕、叡继世,中原有可图之衅。权之名将死丧且尽,权亦老矣。」(《吴论》)

晁补之:「吴人轻而无谋,自古记之矣。孙坚、孙策皆无王霸器。虽赖周瑜、鲁肃辈辅权嗣立,亦权稍持重,故卒建吴国也。」(《鸡肋集》)

萧常:「权承父兄之资,勇而有谋,愤曹操窃国,尝有讨贼之志;乌林之捷,亦一时之隽功。其后关羽围襄阳,降于禁,威振北方,操大惧,欲徙都以避之。权于是时,诚能与羽协力、东西并举,则操可图而汉室可兴。今乃不然,反袭杀羽以媚曹氏,不能少降意于帝室之胄,而甘心臣贼,昭烈之不能混一区夏,由此故也。他日虽有犄角之功,亦无及矣。吁,惜哉!」(《萧氏续后汉书》)

叶适:「权有地数千里,立国数十年,以力战为强,以独任为能。残民以逞,终无毫髪爱利之意,身死而其后不复振,操术使之然也。」(《习学记言·读吴志》)

元好问:「孙郎矫矫人中龙,顾盼叱咤生云风。」

郝经:「東漢之衰,孫權承父兄之烈,尊禮英賢,撫納豪右,誅黄祖,走曹操,襲關侯,遂奄有荆颺,今年出濡須,明年戰合肥,嶷然勢常北向,而以守爲攻,稱臣於魏,結援於漢,始忍勾踐之辱,終爲熊通之譖,保據江淮,奄征南海,卒與漢魏鼎峙而立,先起而後亡,非惟智勇足抗衡,亦國勢便利然也。」(《續後漢書》)

胡三省:「當方面者,當如呂岱;委人以方面者,當如孫權。」(《資治通鑒注》)

朱元璋:「君臣之间,以敬为主。敬者,礼之本也。故礼立而上下之分定,分定而名正,名正而天下治矣。孙权盖不知此,轻与臣下戏狎,狎其臣而亵其父,失君臣之礼。」(《明太祖宝训》)

罗贯中在《三國演義》有詩贊曰:「紫髯碧眼號英雄,能使臣僚肯盡忠,二十四年興大業,龍磐虎踞在江東。」

孙承恩:「仲谋强明,委任才智。听言能断,业乃鼎峙。倍义负汉,屈身事曹。传世四君,霸图亦消。」(《文简集·卷三十八》)

王夫之:「于是而知先主之知人而能任,不及仲谋远矣。」「于子瑜也、陆逊也、顾雍也、张昭也,委任之不如先主之于公,而信之也笃,岂不贤哉?」(《宋论·卷一·太祖》)

王懋竑:「至权时,张昭、张紘虽见尊礼而不复任用,昭且几不免,而翻竟以窜死,惟顾雍、潘濬辈从容讽议,得安有位。陆逊有大功,而以数直谏愤恚而卒。周瑜、鲁肃幸已早死,不与陆逊同祸,而亦恩不及嗣。有所爱重者,惟吕蒙、凌统、甘宁、周泰辈,以视策万万不逮矣。其保有江东者,以有吕蒙辈为之用,得其死力,而其不能廓大基业,窥中原者,亦以此。」(《三国志集解》)

赵翼:「至孙氏兄弟之用人,亦自有不可及者。」「以人主而自悔其过,开诚告语如此,其谁不感泣?使操当此,早挟一‘宁我负人,勿人负我’之见,而老羞成怒矣!此孙氏兄弟之用人,所谓以意气相感也。」

王鸣盛:「孙权称臣事魏已久,及黄武元年春大破蜀,刘备奔走,势愈强盛,则魏欲与盟而不受,九月魏兵来征,又卑辞上书求自改悔,乞寄命交州乃随,又改年临江拒守,彼此互有杀伤,不分胜负。十二月又通聘于蜀,乃既和于蜀,又不绝于魏,且业已改元而仍称吴王。五年令曰北虏缩窜,方外无事,乃益务农亩,称帝之举,直隐忍以至魏明帝太和三年,而后发,反覆倾危,惟利是视,用柔胜刚,阴谋狡猾,史评以勾践相比,非虚语也。」(《三国志集解》)

何焯:「老悖昏惑,吴亡不待皓而决。」

李慈铭:「三国时,魏既屡兴大狱,吴孙皓之残刑以逞,所诛名臣,如贺邵、王蕃、楼玄等尤多。少帝之诛诸葛恪、滕胤,皆逆臣专制,又当别论。惟大帝号称贤主,而太子和被废之际,群臣以直谏受诛者,如吾粲、朱据、张休、屈晃、张纯等十数人,被流者顾谭、顾承、姚信等又数人,而陈正、陈象至加族诛,吁,何其酷哉!自是宫闱之衅,未有至此者也。」(《越缦堂读书笔记》)

蔡东藩:「黄祖本无才智,而孙坚死于祖手;孙策又不能亲复父仇,命为之,势为之也。坚阻于命,策限于势;至权承父兄之业,用瑜蒙诸将,一出再出,方举黄祖而枭夷之,春秋之义大复仇,如孙仲谋者,其固不愧为令子乎?曹操谓生子至如孙仲谋,若刘景升诸儿,与豚犬等,原非虚言。」「孙权承父兄遗业,任才尚计,史谓其有勾践遗风,乃内宠相寻,晚年益愦,废长立幼,乱本已成。」(《後漢演義》)

盧弼:「竊謂有勾踐之志則可,無勾踐之志則終爲奴虜而已,南宋其已事也。仲謀操縱其間,以江東而抗衡大國承祚,方之勾踐其信然矣。」(《三國志集解》)

柏杨:「孙权是中国历史上最可爱最有人情味的皇帝之一。」

李宗吾:「他和刘备同盟,并且是郎舅之亲,忽然夺取荆州,把关羽杀了,心之黑,仿佛曹操,无奈黑不到底,跟著向蜀请和,其黑的程度,就要比曹操稍逊一点;他与曹操比肩称雄,抗不相下,忽然在曹丞相驾下称臣,脸皮之厚,仿佛刘备,无奈厚不到底,跟著与魏绝交,其厚的程度也比刘备稍逊一点。他虽是黑不如操,厚不如备,却是二者兼备,也不能不算是一个英雄。」

毛泽东:「孙权是个很能干的人。」「当今惜无孙仲谋。」(《毛泽东读古书实录》)

\subsubsection{黄武}

\begin{longtable}{|>{\centering\scriptsize}m{2em}|>{\centering\scriptsize}m{1.3em}|>{\centering}m{8.8em}|}
  % \caption{秦王政}\
  \toprule
  \SimHei \normalsize 年数 & \SimHei \scriptsize 公元 & \SimHei 大事件 \tabularnewline
  % \midrule
  \endfirsthead
  \toprule
  \SimHei \normalsize 年数 & \SimHei \scriptsize 公元 & \SimHei 大事件 \tabularnewline
  \midrule
  \endhead
  \midrule
  元年 & 222 & \tabularnewline\hline
  二年 & 223 & \tabularnewline\hline
  三年 & 224 & \tabularnewline\hline
  四年 & 225 & \tabularnewline\hline
  五年 & 226 & \tabularnewline\hline
  六年 & 227 & \tabularnewline\hline
  七年 & 228 & \tabularnewline\hline
  八年 & 229 & \tabularnewline
  \bottomrule
\end{longtable}

\subsubsection{黄龙}

\begin{longtable}{|>{\centering\scriptsize}m{2em}|>{\centering\scriptsize}m{1.3em}|>{\centering}m{8.8em}|}
  % \caption{秦王政}\
  \toprule
  \SimHei \normalsize 年数 & \SimHei \scriptsize 公元 & \SimHei 大事件 \tabularnewline
  % \midrule
  \endfirsthead
  \toprule
  \SimHei \normalsize 年数 & \SimHei \scriptsize 公元 & \SimHei 大事件 \tabularnewline
  \midrule
  \endhead
  \midrule
  元年 & 229 & \tabularnewline\hline
  二年 & 230 & \tabularnewline\hline
  三年 & 231 & \tabularnewline
  \bottomrule
\end{longtable}

\subsubsection{嘉禾}

\begin{longtable}{|>{\centering\scriptsize}m{2em}|>{\centering\scriptsize}m{1.3em}|>{\centering}m{8.8em}|}
  % \caption{秦王政}\
  \toprule
  \SimHei \normalsize 年数 & \SimHei \scriptsize 公元 & \SimHei 大事件 \tabularnewline
  % \midrule
  \endfirsthead
  \toprule
  \SimHei \normalsize 年数 & \SimHei \scriptsize 公元 & \SimHei 大事件 \tabularnewline
  \midrule
  \endhead
  \midrule
  元年 & 232 & \tabularnewline\hline
  二年 & 233 & \tabularnewline\hline
  三年 & 234 & \tabularnewline\hline
  四年 & 235 & \tabularnewline\hline
  五年 & 236 & \tabularnewline\hline
  六年 & 237 & \tabularnewline\hline
  七年 & 238 & \tabularnewline
  \bottomrule
\end{longtable}

\subsubsection{赤乌}

\begin{longtable}{|>{\centering\scriptsize}m{2em}|>{\centering\scriptsize}m{1.3em}|>{\centering}m{8.8em}|}
  % \caption{秦王政}\
  \toprule
  \SimHei \normalsize 年数 & \SimHei \scriptsize 公元 & \SimHei 大事件 \tabularnewline
  % \midrule
  \endfirsthead
  \toprule
  \SimHei \normalsize 年数 & \SimHei \scriptsize 公元 & \SimHei 大事件 \tabularnewline
  \midrule
  \endhead
  \midrule
  元年 & 238 & \tabularnewline\hline
  二年 & 239 & \tabularnewline\hline
  三年 & 240 & \tabularnewline\hline
  四年 & 241 & \tabularnewline\hline
  五年 & 242 & \tabularnewline\hline
  六年 & 243 & \tabularnewline\hline
  七年 & 244 & \tabularnewline\hline
  八年 & 245 & \tabularnewline\hline
  九年 & 246 & \tabularnewline\hline
  十年 & 247 & \tabularnewline\hline
  十一年 & 248 & \tabularnewline\hline
  十二年 & 249 & \tabularnewline\hline
  十三年 & 250 & \tabularnewline\hline
  十四年 & 251 & \tabularnewline
  \bottomrule
\end{longtable}

\subsubsection{太元}

\begin{longtable}{|>{\centering\scriptsize}m{2em}|>{\centering\scriptsize}m{1.3em}|>{\centering}m{8.8em}|}
  % \caption{秦王政}\
  \toprule
  \SimHei \normalsize 年数 & \SimHei \scriptsize 公元 & \SimHei 大事件 \tabularnewline
  % \midrule
  \endfirsthead
  \toprule
  \SimHei \normalsize 年数 & \SimHei \scriptsize 公元 & \SimHei 大事件 \tabularnewline
  \midrule
  \endhead
  \midrule
  元年 & 251 & \tabularnewline\hline
  二年 & 252 & \tabularnewline
  \bottomrule
\end{longtable}

\subsubsection{神凤}

\begin{longtable}{|>{\centering\scriptsize}m{2em}|>{\centering\scriptsize}m{1.3em}|>{\centering}m{8.8em}|}
  % \caption{秦王政}\
  \toprule
  \SimHei \normalsize 年数 & \SimHei \scriptsize 公元 & \SimHei 大事件 \tabularnewline
  % \midrule
  \endfirsthead
  \toprule
  \SimHei \normalsize 年数 & \SimHei \scriptsize 公元 & \SimHei 大事件 \tabularnewline
  \midrule
  \endhead
  \midrule
  元年 & 252 & \tabularnewline
  \bottomrule
\end{longtable}


%%% Local Variables:
%%% mode: latex
%%% TeX-engine: xetex
%%% TeX-master: "../../Main"
%%% End:
