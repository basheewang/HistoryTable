%% -*- coding: utf-8 -*-
%% Time-stamp: <Chen Wang: 2019-12-17 22:41:25>

\subsection{昭烈帝\tiny(221-223)}

\subsubsection{生平}

漢昭烈帝劉備(161年7月16日-223年6月10日),字玄德,涿郡涿縣(今河北省涿州市)人,祖籍徐州沛縣(今徐州市沛縣),亦稱漢先主,三國時代蜀漢第一位皇帝,諡號昭烈皇帝,三國志、華陽國志等稱為先主 ,繼其帝位的劉禪則被稱為「後主」,資治通鑑稱劉備父子為漢主。

劉備雖為漢景帝後代,但世系久遠,實由布衣起步而終得一方天下。

劉備是汉景帝第九子中山靖王劉勝之子刘贞的後代,而裴松之三国志注所引《典略》记载,刘备为“临邑侯枝属”。祖父名雄,父親名弘,世代皆仕於州郡,祖父劉雄曾被推舉為孝廉,官至東郡範令。世居酈亭樓桑里。

劉弘在劉備少時已逝,劉備便與母親販賣草鞋、織草蓆為業。家裡房舍的東南角的圍籬上有種植桑樹,高五丈餘,從遠處觀看像是一臺當時小車的車頂,路過的人皆訝異此樹的非凡,或說此家必當出貴人。劉備小時候與家族中年齡相近的小孩在樹下遊戲時,曾說:「吾必當乘此羽葆蓋車。」他的叔父劉子敬說:「汝勿妄語,滅吾門也!」劉備15歲時,刘备母亲要他外出求學,與同宗刘德然、遼西公孫瓚同入大儒盧植門下求學。

劉德然之父劉元起常資助劉備,所給錢物與自己兒子劉德然等同。劉元起妻罵:「各自一家,何能常爾邪!」元起答:「吾宗中有此兒,非常人也。」公孫瓚與劉備結為好友,公孫瓚較年長,劉備以兄事之。

劉備不甚樂讀書,喜歡評馬論犬、音樂、華美的衣服。身長七尺五寸(約173公分,漢時一尺約為23.1公分),垂手下膝,有一對招風大耳,不需攬鏡自照,眼可自見其耳。少說話,善於待人,喜怒不形於色。好交結豪俠義士,年輕人爭相趨附他。中山大商人張世平、蘇雙等多給與金錢資助,劉備由是得用以糾合组织部下。由於個性與行事風格酷似先祖劉邦,而被評為有高祖之風。

184年(23歲),黃巾之亂爆發,各州郡皆有人民組織義軍討伐。劉備率領耿雍、關羽、張飛、牵招及一干下屬跟隨鄒靖討伐黃巾軍,立下戰功,被任為安喜尉。後來,漢室有令:如因軍功而成為長吏的人,都要被選精汰穢,督郵到安喜要遣散劉備,劉備知道消息後,到督郵入住的驛站休息房舍求見,督郵聲稱有病不肯相見,劉備因此感到不悅,便徑直闖入房舍,將督郵綑綁,杖打二百下,然後棄官逃亡。後來,大將軍何進派都尉毌丘毅到丹楊募兵,劉備也在途中加入,到下邳時與盜賊力戰立功,任為下密縣丞,不久又辭官。

191年(30歲),刘备時任高唐令,但被盗贼击败而投奔公孫瓚,公孫瓚隨即上表,保奏劉備為別部司馬,任為平原令、平原相。劉備平原外禦賊寇,在內則屯糧分發給百姓,士以下的人,都可與他同席而坐,同簋而食,不會有所揀擇。據說郡民劉平不服從劉備的治理,唆使刺客前去暗殺。劉備毫不知情,還對刺客十分禮遇,刺客深受感動,不忍心殺害劉備,便坦露實情離去。劉備治理平原郡深得人心、相當成功。

黄巾餘黨管亥率眾軍攻打北海郡,北海相孔融被大軍所圍,情勢危急,便派太史慈突圍向劉備求救。太史慈對劉備說:「慈,東萊之鄙人也,孔北海親非骨肉,比非鄉黨,特以名志相好,有分災共患之義。今管亥暴亂,北海被圍,孤窮無援,危在旦夕。以君有仁義之名,能救人之急。故北海區區,延頸恃仰,使慈冒白刃,突重圍,從万死之中自托于君,惟君所以存之。(我太史慈只是東萊一個無名之人。北海相孔融和我並不是有著骨肉相連的親族,也稱不上是志同道合的同鄉朋友,只是他認為我有前途而看重我,所以我有為他分擔災禍、共赴患難之義理。現在管亥起兵擾境,包圍北海城,城內居民徬徨無助,危在旦夕。孔融大人聽說劉備大人有仁義之名,能救人之危難急迫。因此盼望著能得到您的幫助,命令我突破管亥兵眾的包圍,冒著萬死無生的可能,來向劉備大人求助,惟有借重您的力量能使北海城脫危。)」劉備驚訝地答道:「孔北海知世間有劉備邪!(北海相孔融居然知道世間有我劉備啊!)」便立即派三千精兵隨太史慈去北海救援。黄巾军聞知援軍至,都四散而逃,孔融逐得以解圍。後袁紹攻公孫瓚,劉備與田楷東屯齊。

193年(32岁),曹操征討徐州,徐州牧陶謙敗退,曹操在徐州大屠殺。陶謙遺使告急於田楷,田楷與劉備俱前往相救。當時劉備自有士兵千餘人及幽州烏桓攙雜胡族騎兵,又略得饑民數千人。既到,與陶謙將領曹豹屯在郯東,被曹操擊敗。後曹操因後方生事而撤退,陶謙以丹楊兵四千人給劉備,劉備遂離開田楷,依附陶謙。陶謙表劉備為豫州刺史,屯兵於小沛。

194年(33岁),陶謙病重,對別駕從事麋竺說:「非劉備不能安此州也。」陶謙死後,麋竺便率徐州人民迎劉備入主徐州,劉備未敢當。下邳陳登對劉備說:「今漢室陵遲,海內傾覆,立功立事,在於今日。彼州殷富,戶口百萬,欲屈使君撫臨州事。(現今漢室漸趨衰敗,海內傾覆,立功名、立事業,就在於今日。本州殷實富足,戶口百萬,希望屈就使君親臨撫牧本州事務。)」劉備說:「袁公路近在壽春,此君四世五公,海內所歸,君可以州與之。(袁公路就近在壽春,此人為四世代有五人為三公,海內民心所歸,你可以徐州給與他。)」陳登答:「公路驕豪,非治亂之主。今欲為使君合步騎十萬,上可以匡主濟民,成五霸之業,下可以割地守境,書功於竹帛。若使君不見聽許,登亦未敢聽使君也。(袁術驕縱橫豪,不是治理亂局之主。現在希望您使君合共步兵騎兵十萬,對上可以匡扶主上、救濟人民,成就像春秋五霸之功業;對下可以割地自守、保境安民,寫下功業於竹帛上。若不見聽使君答許,在下亦未敢聽從使君。)」北海相孔融對劉備說:「袁公路豈憂國忘家者邪?冢中枯骨,何足介意。今日之事,百姓與能,天與不取,悔不可追。(袁公路豈是因憂慮國事而忘卻家庭之人?墓中之枯骨,不足以在意。今日之事情,是百姓讓與賢能,天意讓與你而不取,後悔不可追。)」劉備遂領徐州牧。

195年(34岁),吕布被曹操打敗來投靠,劉備善待禮遇他。吕布見劉備,極為尊敬,說:「我與卿同邊地人也。布見關東起兵,欲誅董卓。布殺卓東出,關東諸將無安布者,皆欲殺布爾。(我與你同為邊地出身的人(呂布出身五原郡,劉備出身涿郡,皆屬漢朝疆界北方邊境之地)。我見關東諸侯起兵,想要誅殺董卓。後來我殺董卓向東走,關東諸將卻沒有一個安置我,更加要殺死我啊。)」請劉備於帳中坐,並令妻子行禮,酌酒飲宴,又稱呼劉備為其弟。劉備見吕布胡言亂語,表面上雖不當一回事而心裏卻對其有所戒備。最後劉備仍讓吕布屯於小沛。

建安元年(196年,35岁),袁术來攻徐州,劉備於盱眙、淮陰抵抗袁軍。曹操上表朝廷,劉備成為鎮東將軍,封為宜城亭侯。劉備與袁术相持經一個月,大戰互有勝負。吕布乘下邳之虛,趁機偷襲。下邳守將曹豹倒戈,迎接呂布,趕走張飛,佔據下邳。吕布擄獲劉備妻子,劉備轉戰海西。東漢建安二年(197年)夏天,楊奉、韓暹等賊軍在徐、揚二州之間作惡,劉備與其決戰,盡為劉備所斬首。後來劉備向吕布求和,吕布歸還其妻子。劉備遺派關羽守下邳。

劉備還軍小沛,恢復集合兵馬得萬餘人。吕布嫌惡於此,自行出兵攻打劉備,劉備兵敗走投歸順曹操。曹操厚待禮遇劉備,以其為豫州牧。劉備與曹操一同返回許都後,被任命為左將軍。劉備來投奔,曹操謀士程昱就曾警告「觀劉備有雄才而甚得眾心,終不為人下」,勸曹操趁早解決後患,但曹操認為當時是收英雄之時,不可失天下之心。

(198年37岁)春天,吕布派人攜金到河內買馬,但被劉備兵所掠取。吕布於是派高順、張遼等攻劉備,雖然曹操曾派夏侯惇前往解救,但仍敗陣,劉備妻子又被吕布所擄。十月,曹操親自東征吕布,劉備在梁國界中與曹操相遇,便合兵成功消滅吕布。劉備復得妻子,跟從曹操還師許都。曹操表劉備為左將軍,禮之愈重,出則同車,坐則同席。

漢獻帝因曹操挾天子以令諸侯,發出衣带詔令其岳父董承誅殺曹操,劉備尚未加入。一日,曹操宴請劉備,對劉備說:「今天下英雄,唯使君與操耳。本初之徒,不足數也。(當今天下英雄唯獨是你與我,袁紹這類人稱不上)」劉備聽心中一震,筷子從手中掉落。此時剛好打雷,劉備便對曹操說:「『聖人迅雷風烈必變』,良有以也。一震之威,乃可至於此也!(『即使是圣人遇见打雷也会改变表情』,確有原因。一聲雷鳴,乃可以令我變成如此!)」《華陽國志》記載當時碰巧雷聲大作,劉備便把自己的失態歸咎於雷鳴,而此事後,劉備便加入董承。不久,在南方失利的袁术想北投袁紹,劉備便向曹操借兵出擊袁术,趁机摆脱曹操的控制。曹操便派他督朱靈、路招攻擊袁术,但軍未到,袁术已病死。

199年(38歲),劉備遣朱靈、路招佔據下邳。200年(39歲),反曹事迹敗露,董承被殺。劉備便殺死徐州刺史車冑,留關羽守下邳,自己回守小沛,另一方面派遣孫乾與袁紹連合,打出對抗曹操的名目。曹操曾派劉岱、王忠領軍攻打劉備,但不克。同時,東海昌霸反叛,郡縣多投靠劉備,劉備軍再次聚起數萬人,並連同多個地方勢力一起反曹。曹操決定親自東征劉備,雖然曹軍中將領多認為袁紹才是大敵,但曹操卻覺得劉備是英傑,必要先行討伐,郭嘉亦贊同曹操。

最後劉備大敗,小沛被佔,曹操虜獲劉備妻子及生擒關羽、夏侯博。劉備逃至青州,青州刺史袁譚親自迎接,並報知其父袁紹,袁紹出鄴城200里迎接。刘备泄露曹操曾经对自己说的密言予袁绍,袁绍才知道曹操原来有针对自己的阴谋。刘备待了一個多月後,以前的部下又重新聚會。不久,曹操與袁绍於官渡交戰,汝南郡黃巾餘軍劉辟等响應袁绍叛曹,袁绍便派劉備率軍與劉辟會合。曹操派曹仁攻打汝南,劉備惟有再次還軍袁绍。當時劉備想離開袁绍,便說服袁绍應南連劉表,袁绍再次派劉備到汝南與龔都會合。曹操另派蔡陽攻擊劉備,為劉備所殺。曹操於官渡之戰大敗袁绍。

建安六年(201年40歲),曹操又出兵南擊劉備,劉備便乘機放棄汝南,入荊州投靠劉表。劉備並派麋竺、孫乾與劉表會面。劉備到達荊州,受到劉表熱情接待。劉表接納劉備後,便為他增加兵馬,屯兵於新野,守衛荊州北大門。建安七年(202年42歲),曹操與袁尚、袁譚大戰於黎陽,許昌空虛,奉劉表命令北伐曹操。曹將夏侯惇、于禁、李典等人率軍南下,劉備奉命北上迎敵。在新野北博望,劉備設好伏兵以後,便燒毁營屯假裝懼敵退卻。夏侯惇讓李典留守,自己和于禁追擊,追到博望,劉備伏兵將夏侯惇殺得大敗,曹軍損失慘重,向北退走。劉備在荊州聲望日高,引起劉表疑心劉備,處處戒備。

建安十二年(207年46歲),曹操基本統一黃河流域之後,開始北上征伐北方烏丸,刘备力勸刘表乘機袭取许都,刘表没有採纳劉備建議。

劉備在荊州幾年,知道水鏡先生就是司馬徽,便前去請教世事。司馬徽知道劉備來意,便對他說:「儒生俗士,豈識時務?識時務者為俊傑。此間自有卧龍、鳳雛。(一個儒生見識淺俗之士,豈會認識時勢事務?認識時勢事務者,是那些英俊豪傑。從此地中,有臥龍(诸葛亮)、鳳雛(龐統)。)」亮又受徐庶推薦,劉備希望徐庶引亮來見,但徐庶卻建議:「此人可就見,不可屈致也。將軍宜枉駕顧之。(此人只能前去拜謁,不可委屈他前來。將軍宜枉屈尊駕以顧望。)」

207年(46岁),劉備三顧茅廬,問計於諸葛亮:「漢室傾頹,奸臣竊命,主上蒙塵。孤不度德量力,欲信大義於天下,而智術淺短,遂用猖獗,至於今日。然志猶未已,君謂計將安出?(漢室衰敗,奸臣掌權,使天子(漢獻帝)蒙受苦難。我不自量德行與能力,欲伸張大義於天下,然而智術淺薄,時至今日,一無所成。然則志向仍未減,先生可以出謀畫策嗎?)」諸葛亮遂向他陳述三分天下之計,分析此時曹操挾天子而令諸侯,此誠不可與爭鋒;孫權據有江東,可以爲援而不可圖;又詳述荊州用武之國、戰略要地,而其主劉表不能守,此恐怕是上天賜予劉備;益州是漢高祖成就帝業之地,其主劉璋闇弱;更建議劉備等待時局有變,由荊州、益州進攻中原。這篇論說後世稱為《隆中對》,是此後數十年劉備和蜀漢基本國策。。諸葛亮剛從隆中出來,受到劉備重視,只是由於劉備與自己情好日密,就引得「關羽、張飛等不悅」,最後還是劉備出來說:「孤之有孔明,猶魚之有水也。願諸君勿復言。(我有孔明,猶如魚得到水。但願諸君勿再說。)」;關羽、張飛才作罷。劉備在荊州擴軍,諸葛亮籌措軍需,何宇度《益部談資》記載:「先主寓荊州。從南陽大姓晁氏貸錢千𦻼,以為軍需。諸葛孔明作保,券至宋猶存。」

208年(47歲),曹操南下,時劉備屯於樊城。八月劉表病卒,次子劉琮繼任荊州牧,遣使曹操舉州投降。起初劉備不知劉琮決定投降,得知時曹軍尚在宛縣,尚未到達新野,劉備連忙棄城南撤。。在南渡漢水至襄陽時,諸葛亮曾勸劉備攻劉琮奪襄陽,但劉備不忍心進攻劉表之子,沒有攻打襄陽,只是在城下駐馬高呼劉琮出來相見,只來到劉表墓前祭奠,涕泣拜辭而去。劉備一行南下,荊州官吏百姓加入,走到當陽時,人數達10餘萬,輜重數千輛,一日只能走10幾里。惟有另派關羽乘數百艘船,直到江陵。有人向劉備進言:「宜速行保江陵,今雖擁大眾,披甲者少,若曹公兵至,何以拒之?(適宜速行而保江陵,現今雖然擁有很多隨行者,但士兵很少,若曹操軍追至,如何抵抗?)」劉備答道:「夫濟大事必以人為本,今人歸吾,吾何忍棄去!(做大事必以人為本,現今人眾歸附於我,我又如何忍心離棄而去!)」

當時江陵貯有劉表的大量糧儲、器械等軍實,曹操深怕劉備先佔領江陵,就拋棄輜重,以輕軍急行到襄陽。曹操聽聞劉備軍已離開襄陽,與曹純等領五千精騎急追,一日一夜疾行三百餘里。曹軍五千輕騎奔至當陽長坂坡追上劉備一行,劉備棄妻子,與諸葛亮、張飛、趙雲等數十騎走,10餘萬眾土崩瓦解,曹軍大舉擒獲劉備人眾輜重,張飛率20騎拒後,與曹兵邊打邊退。孫權之前派出魯肅來打探消息,在當陽長坂迎堵劉備。長坂會面後,魯肅隨劉備向東南斜趨漢津,在此適逢與關羽水軍會合,渡過沔水後向江夏進發。江夏太守劉琦聞劉備軍到來,率軍前去迎接,將劉備迎到夏口。此後,魯肅返回江東覆命,劉備進至樊口,同時派諸葛亮隨魯肅出使孫權,與孫權結盟。

孫權正式任命周瑜為左都督,程普為右都督,魯肅為贊軍校尉,率三萬水軍,與諸葛亮一起溯江西上,與樊口劉備軍會合。建安十三年冬,曹操親率20餘萬大軍從江陵順江東下,討伐孙权。黃蓋便向周瑜建議說:「今寇眾我寡,難與持久。然觀操軍船艦首尾相接,可燒而走也。」十二月,孫劉聯軍在赤壁至烏林一線以火攻大破曹軍,更追至南郡,曹操敗北。曹操一到江陵,便部署征南將軍曹仁、橫野將軍徐晃守江陵,折衝將軍樂進守襄陽,曹操撤回北方。

赤壁之戰後,劉備撤出江陵戰鬥,全力占據荊州江南四郡,先上表漢帝奏請劉琦為荊州刺史,兩萬大軍南下,武陵太守金旋獻城、長沙太守韓玄迎降、桂陽太守趙範讓位、零陵太守劉度稽顙。廬江人雷緒也率部曲數萬人投效。建安十三年(208年48歲)十二月,荊州江南四郡盡為劉備所占領。劉琦死,群下推劉備為荊州牧,劉備即遣諸葛亮為軍師中郎將,督令零陵、桂陽、長沙三郡,收其租賦,以供軍實,又以關羽為襄陽太守、蕩寇將軍駐江北,張飛為宜都太守、征虜將軍在南郡,趙雲為偏將軍領桂陽太守。廖立為長沙太守,郝普為零陵太守,向朗督秭歸、夷道、巫縣、夷陵四縣軍民事。劉備治於公安。而孫權為與劉備建立更鞏固的關係,在周瑜死後便依魯肅之策將南郡、江陵借給劉備,再分部份長沙郡給他,以及確認劉備佔有武陵和桂陽兩郡,遂提出將其妹嫁予劉備,史稱孫夫人。劉備到京口見孫權,關係表現親密、寬度。時劉備擁有荊州大部份屬地,又收取荊襄名士龐統和馬良,整日操練人馬,伺機南征北伐。

以後,孫權曾派使希望與劉備一起取益州,劉備本想答應,因東吳不可能越荆州而有蜀,蜀地就可據為己有。但荊州主簿殷觀卻反對:「若為吳先驅,進未能克蜀,退為吳所乘,即事去矣。今但可然贊其伐蜀,而自說新據諸郡,未可興動,吳必不敢越我而獨取蜀。如此進退之計,可以收吳、蜀之利。(若我們為吳開路,前進未必能攻克蜀地,後退可能為吳乘虛而入,那時即大勢而去。現今但可以贊同他伐蜀,而自己推卻說剛佔據荊南諸郡,未能興兵妄動,吳必定不敢越過我境而單獨取蜀。依照此進退得宜之計謀,便可以收吳、蜀兩地之利。)」劉備依從其計,孫權果然終輟計劃。殷觀遂升遷為別駕從事。

建安十六年(211年51歲)三月,曹操下令鍾繇率軍西征漢中張魯,讓夏侯淵出河東與鍾繇相會。益州牧劉璋遙聞曹操將遺鍾繇等向漢中討張魯,內心懷有恐懼。別駕從事蜀郡張松說服劉璋稱:「曹操兵強,無敵於天下,若因張魯之資源用以攻取益州土地,誰能抵禦?」劉璋說:「我固然擔憂,而未有計。」張松說:「劉備,使君之宗室,而且是曹操之深仇,善於用兵,若使之討伐張魯,張魯必可攻破。張魯攻破,則益州強大,曹操雖來,也無能為力。」在張松出言下,益州牧劉璋採納請劉備入蜀之意見,並派軍議校尉法正為使,孟達為副,各領兵2,000人,前往荊州邀請劉備入蜀助攻張魯。劉備親自統帥進軍益州,龐統任軍師中郎將,將領黃忠、魏延、卓膺等輔助劉備。劉備與龐統一同進入益州。諸葛亮、關羽、張飛、趙雲、劉封、孟達、馬良等留在荊州。然而劉備要知道蜀中的闊狹,兵器、府庫、人馬多少及多個要害之地的遠近,便向二人請教,張松、法正都一一詳述,更畫出地圖指示山川所在,所以劉備知道益州內情。

到達涪城,劉璋親自出迎,相見甚歡。張松、法正及龐統都提議劉備可乘機殺了劉璋,當時龐統主張趁此機會,擒住劉璋。劉備以初來到蜀地,人心尚未信服,不宜輕舉妄動為由,未採納龐統建言。劉璋推薦劉備行大司馬,領司隸校尉,劉備又推薦劉璋行鎮西大將軍,領益州牧。劉璋配給劉備士兵,及督白水軍,令他攻擊張魯。劉備當時總計有三萬多人,車甲、器械、資貨甚多。但劉備卻到葭萌時,未出兵,而是樹立恩德,收買民心。

建安十七年(212年51歲)冬十月,曹操出兵攻打孫權,孫權向劉備告急,劉備對劉璋說欲還救荊州有急。劉備請求劉璋撥出兵士萬人與軍事物資。但劉璋只允諾給予四千兵馬,其餘物資僅提供一半。劉備受此激怒,忿忿說道:「我為了益州征討強敵,軍隊勤瘁,無暇休息;現今劉璋積存起財富而不用於賞功,卻希望士大夫能為他出力死戰,這又怎可能!」當時張松不知劉備用意,寫信質問:「眼看就要大事底定,為何拋下一切離去?」結果被其兄張肅據此告密,張松遭到處死,導致劉備與劉璋關係惡化。十二月,劉備與劉璋決裂。劉備依龐統提出的計謀,召白水关守将杨怀、高沛到來並將其斬殺。另外又派黃忠、卓膺率軍攻劉璋,一路佔領至涪城。劉璋連忙派出劉璝、冷苞、張任、鄧賢、中郎将吴懿等與對抗劉備,皆破败,退保绵竹,吴懿至刘备军前投降,拜为讨逆将军。刘璋后遣护军李嚴、参军费观督绵竹军拒刘备,两人陣前倒戈亦率众投降,同拜裨將軍,劉備軍勢強,分軍平定各郡縣。但劉備軍卻被雒城守將劉循阻擋攻勢。從建安十八年建安十九年,劉備圍攻雒城將近一年,龐統被流矢射中,重創身亡。張飛、趙雲、劉封等隨諸葛亮率軍入蜀,關羽留下鎮守荊州,馬良、麋芳、士仁、廖化協助關羽鎮守荆州。建安十九年(214年54歲)夏,諸葛亮入蜀援軍溯江而上。諸葛亮分兵進攻成都:張飛從墊江北上直取巴西郡治閬中,從北面攻成都;趙雲從長江西攻取江陽北上犍為郡治武陽,從南面攻成都;諸葛亮親自沿涪江取德陽,直取成都。

214年夏天(53岁),雒城終被攻破。李恢受劉備派遣到漢中與馬超交好,馬超正想離開張魯,劉備暗暗派出人馬與馬超兵眾會合,馬超率領大隊人馬開到成都城北屯駐。關羽聽說馬超歸降備,便寫信給諸葛亮,問馬超才能可與誰相比,諸葛亮回信說:「馬超文武兼備,氣概雄烈,過於常人,可稱得上一世之豪傑,是黥布、彭越一流之人物,可以與張飛相提並論,但是趕不上美髯公你超逸絕群。」劉備乘勢率漢軍進圍成都數十日。劉備派簡雍進入成都勸說劉璋投降,劉璋與簡雍「同輿而載,出城歸命」;劉璋向劉備繳械投降,益州易主,歸屬劉備。由於蜀中繁盛、安樂,劉備便設宴大慰勞士卒,又取蜀城中的金銀,分賜將士,還其谷帛。劉備皆處之顯任,盡其器能,有志之士,無不競勸,益州之民,是以大和。有議論勸劉備將成都城中房舍及城外園地桑田分賜給諸將,但趙雲反駁說:「從前漢朝大將霍去病曾說匈奴未滅,無用家為,何況現在國賊不只像匈奴只有一個,還不到可以安定下來的時候,必須等到天下的亂賊都平定之後,才可讓眾人返回家鄉去種植桑梓,回歸故土去耕作田地,這樣才是正道。益州的人民是第一次遭遇到戰爭,應該將田宅房產歸還給百姓,先讓他們安居樂業,然後才能叫他們服兵役,納錢糧,也才能得到益州的民心。」劉備便聽從趙雲的建議,有志之士便都紛紛來投。

在建安二十年(215年54歲)三月,曹操征伐漢中,七月破南鄭,十一月最終降服張魯,搶在劉備之前占有漢中。孫權以劉備已得益州,派人討還荊州,劉備答道:「須得涼州,當以荊州相與。」孫權忿恨,乃派遣呂蒙施襲,爭奪長沙、零陵、桂陽三郡。劉備率兵五萬到公安,下令關羽進軍益陽,與孫軍對峙。時曹操勢破張魯,威脅蜀地。劉備遂派使者向孫權議和,孫權派諸葛瑾答覆劉備,雙方和好。為盡快解決荊州問題,回兵保衛益州,劉備以湘水為界,將江夏、長沙、桂陽三郡劃給孫吳。南郡、零陵、武陵以西屬劉備所有,劉回軍江州。八月,孫在東線進攻合肥,曹將張遼、李典據城抵抗,擊退孫權。十一月,張魯逃遁至巴西,偏將軍黃權對劉備說:「若失漢中,則三巴不振,此為割蜀之股臂也。」又遣黃權率兵迎向張魯,但張魯已降曹操。曹操派夏侯淵、張郃屯兵漢中,數次武力侵犯巴郡邊界。劉備令張飛進兵宕渠,與張郃等於瓦口爭戰,大敗張郃等。張郃收兵還退南鄭。翌年二月,曹操留夏侯淵、張郃鎮守漢中,自己回鄴城。

建安二十三年(218年57歲),劉備採法正勸諫率軍進攻漢中。諸葛亮鎮守成都,劉備親率大軍征漢中,法正隨從參謀軍機,趙雲、黃忠、魏延、張飛、馬超、吳蘭等從征,曹操、劉備爭奪漢中之戰開始。但漢軍先頭部隊卻被曹軍打敗。劉備一路直攻漢中,進兵至陽平關與夏侯淵、張郃等曹軍對峙,為保證道路暢通,劉備派大將陳式率10餘營兵士駐紮在馬嗚閣道,曹將夏侯淵派大將徐晃襲擊陳兵,陳式軍被打敗,士兵紛紛跳入山谷,傷亡慘重。曹操下令賜徐晃節杖,並說:「此閣道,漢中之險要喉也。劉備欲斷絕外內,以取漢中。將軍一舉,克奪賊計,善之善者也。」劉備「急書發益州兵」,諸葛亮與從事楊洪商議對策,楊洪說:「漢中則益州咽喉,存亡之機會,若無漢中則無蜀矣,此家門之禍也。方今之事,男子當戰,女子當運,發兵何疑!」;諸葛亮非常看重楊洪見識,當即發兵支援漢中前線。從建安二十二年(217年)劉備出兵起,雙方在漢中僵持一年多,建安二十四年(219年)春劉備聽從法正計策,從陽平南渡沔水,依定軍山恃險安營。夏侯淵帶少數兵力爭奪定軍山營地,法正對劉備說:「可擊矣!」;劉備便命黃忠乘高鼓噪攻之,居高臨下,衝入敵陣,殺死夏侯淵。黃忠斬殺夏侯淵及曹操所置的益州刺史趙顒等。建安二十四年三月(219年58歲),曹操自長安率兵經褒斜谷趕往漢中,劉備說:「曹公雖來,無能為也,我必有漢川矣。」劉備在險處死守,不與曹軍交戰。諸葛亮親坐益州,將人力、物力及時補充到劉備軍中。夏五月,曹操引兵撤出漢中,漢中歸劉備所有。而另一方面,又遣劉封、孟達、李嚴等進攻上庸,上庸守將申耽等見曹操率軍返回中原,逐開城投降。秋七月,馬超、龐羲、射援、諸葛亮、關羽、張飛、黃忠、法正、李嚴等120人聯名上表劉備為漢中王。劉備於沔陽設置祭壇場地,陳兵列眾,群臣陪位,宣讀奏訖,自立漢中王。後還治成都。提拔魏延為都督漢中太守,坐鎮漢中。劉備於是建起館舍,修築亭障,從成都至白水關,四百餘區。關羽率軍從江陵北上,發動襄樊戰役。于禁七軍火速增援曹仁,關羽與于禁交鋒,時至八月,大雨滂沱,山洪暴發,漢水驟漲,水淹七軍,于禁束手就擒,部下幾乎全部投降,副將龐德被活捉不降,最後被關羽所殺。孫權將呂蒙白衣渡江。十月,呂蒙任征荊州大督,率兵西上,公安士仁、江陵麋芳開城投降。關羽回軍江陵途中,陸遜任右護軍、鎮西將軍屯駐夷陵,呂蒙任南郡太守駐江陵。關羽至當陽西保麥城,敗走麥城後,士兵繼續逃散,關羽身邊只剩十餘騎。十二月關羽被孫權大將潘璋部馬忠捕殺,孫權將其首級送至洛陽曹操處。孫劉聯盟正式決裂。

建安二十五年(220年59歲)正月,曹操逝世,劉備也曾派遣韓冉奉書弔唁,「並致賻贈之禮」,但最後卻失敗。三月改元延康,十月曹丕代漢稱帝。十二月,當時有謠言指漢獻帝劉協已被加害,劉備便穿喪服發喪,諡劉協為孝愍皇帝(但實際上劉協仍在世)。同年,法正、黃忠去世。

221年,群臣勸劉備登基為帝,劉備不答應,諸葛亮用耿純遊說劉秀登基故事勸劉備(光武帝劉秀登基時,同為漢室的更始帝劉玄仍在世,此後綠林軍攻破長安殺劉玄,此後劉秀建東漢),劉備才決定接受群臣擁立,四月初六在成都武擔山之南接受皇帝璽綬,改元章武。諸葛亮、許靖、黃權等人上書勸劉備即帝位,國號仍為「漢」,史稱蜀漢。四月丙午日(5月15日),大赦天下,改元章武。以諸葛亮為丞相,許靖為司徒。設置百官,建立宗廟,祭祀漢高祖以下。五月,立皇后吳氏,劉禪為皇太子。六月,以劉永為魯王,劉理為梁王。

魏文帝曹丕召集眾臣討論,侍中劉曄認為蜀漢一定要出兵攻打孫吳,理由是:「蜀雖狹弱,而備之謀欲以威武自強,勢必用眾以示其有餘。且關羽與備,義為君臣,恩猶父子;羽死不能為興軍報敵,於終始之分不足。」七月,劉備不采纳赵云等人劝告,率軍沿江而下,討伐東吳。張飛被部下暗殺。孫權先派人給蜀漢送信求和,又令諸葛亮哥哥諸葛謹致箋勸劉備息兵罷戰,劉備一概拒絕。孫權把國都從建業遷到武昌,以便指揮戰爭。起初,漢軍氣勢如虹,不過吳將陸遜採以逸待勞兵法而戰之,於章武二年(222年)大敗漢軍。陸遜大敗劉備,「殺其兵八萬餘人,備僅以身免」。劉備退至秭歸,趙雲率兵到達白帝城,巴西太守閻芝派馬忠率5千人馬隨後到達。劉備退到永安縣。孫權聽聞劉備住白帝,甚為懼怕,遣使請和。章武二年十二月,孫權派太中大夫鄭泉到白帝城見劉備,正式表示向蜀漢請和。劉備也遣太中大夫宗瑋使吳,表示贊同蜀漢、東吳兩國和好。

當劉封失掉漢中東面三郡逃回成都後,諸葛亮勸劉備除掉劉封。漢嘉郡太守黃元聽說劉備在永安病重,於章武二年十二月舉兵反叛。同年,太傅許靖、尚書令劉巴、驃騎將軍馬超先後病逝。南中越夷高定曾向新道進攻,被李嚴打退。

章武三年(223年62歲)二月,諸葛亮接到劉備詔書,帶著劉永、劉理從成都來到永安。三月,黃元又乘諸葛亮到永安見劉備之機,率軍進攻臨邛縣,火燒臨邛城。益州治中從事楊洪立即把黃元之動向報告給劉禪,劉禪派將軍陳曶、鄭綽進討黃元。陳曶、鄭綽兩人在南安峽口生擒黃元,將其押回成都正法。四月下旬,劉備對諸葛亮說:「君才十倍曹丕,必能安國,終定大事。若嗣子可輔,輔之;如其不才,君可自取。(你的才能是曹丕的十倍,必定能夠安定國家,終可成就大事。如果嗣子(劉禪)可以輔助,便輔助他;如果他沒有才幹,你可以自取其位。)」諸葛亮涕泣說:「臣敢竭股肱之力,效忠貞之節,繼之以死!(臣必定竭盡自己所有力量,報效忠貞之氣節,繼續至死為止!)」劉備又要劉禪和其他兒子「與丞相從事,事之如父。」。劉備臨終前託孤於丞相諸葛亮,尚書令李嚴為副。臨終時,與劉永說:「吾亡之後,汝兄弟父事丞相,令卿與丞相共事而已。(我死後,你們兄弟要對父親般奉事丞相(諸葛亮),你們與丞相只是共事而已。)」。四月廿四(6月10日) 劉備崩逝於永安宮,享壽六十二歲。孫權派立信都尉馮熙出使蜀漢,弔唁劉備。諸葛亮上言讚揚劉備。五月癸巳日(6月21日),遺體自永安運返成都发丧,諡為昭烈皇帝。八月,入葬惠陵。

亦有郭沫若等学者认为由于条件所限,刘备就地下葬于今奉节县,并未归葬成都。

《三国志·先主传》中并没有记载刘备庙号。李慈铭怀疑刘备庙号烈祖是由刘渊所追尊。章学诚根据《三国志·先主传》中诸葛亮宣读的遗诏,指出刘备庙号是太宗。卢弼认为章学诚的说法不足据,如果刘备庙号太宗,《三国志》本传没有不记载的道理。郭善兵则认为刘备庙号缺失不能归咎于史书记载疏漏,而是受到郑玄礼学“一祖二宗与四亲庙”七庙学说影响所致。

刘备喜怒不形于色,常以谦虚恭敬待人,深知「得人心者得天下」的道理,重視以寬仁厚德待人,與那些殘民以逞、暴虐嗜殺的軍閥判然有別,因此而爭取到了人心。刘备不怎么喜欢读书,喜欢評馬論犬、音乐和華美的衣物。

小时候,家中有棵大桑树,遙望見如同车盖,刘备與宗中小兒於樹下玩耍時說過:「吾必當乘此羽葆蓋車。(我必定會乘坐此羽飾華蓋之車。)」叔父聽到後,不禁當下斥責他:「汝勿妄語,滅吾門也。(你不要胡說,會招來滅門之禍)」

劉備在部下聲譽受損或是特殊的理由發生背叛的可能時往往站出來捍衛部下聲譽和保護部下家眷,徐庶母被抓,庶淚崩辭別劉備、糜芳背叛,劉備對愧疚的糜竺說兄弟罪不相及、夷陵之敗黃權不得已降魏,劉備依然善待其家人「孤負黃權,權不負孤也。」

劉備由於沒有鬍鬚,因此曾被张裕取笑。有一次,劉備與劉璋於涪縣會面時,張裕時為璋從事,在一旁陪坐。由於裕的鬍鬚濃密被備嘲笑說道:“過去吾在涿縣時遇到好多姓毛的人,四方許多毛,涿縣縣令聲稱說:「許多毛(毛)繞涿(歜)而居。」,但裕立刻反唇相譏:“過去某人當上黨郡潞縣長,後來升任為涿縣縣令,其辭任歸家時,有人寫信給其,要是寫了潞縣就丟了涿縣,而寫上涿縣又失去潞縣,就寫道「潞(露)涿(歜)君」。因此裕就用此方法反譏備。讓劉備對張裕一直沒什麼好感,劉備攻漢中之前,張裕說會出師不利,但劉備仍照著既定計畫出兵。結果劉備拿下漢中,不過兩名大將吳蘭、雷銅等也在此戰中身亡,以致於劉備記恨張裕,某天,張裕私下對人說:「庚子年間(220年)會改朝換代;主公入主蜀地的九年後,也會再次失去蜀地,劉氏運氣將會消盡。」謠言亂傳,最終入劉備之耳,劉備不滿張裕散布滅亡謠言,以張裕的話語沒有應驗,把他關入獄中。諸葛亮請求劉備寬恕他,劉備只說:「芳蘭生門,不得不鉏。」於是殺了張裕,棄屍於街頭。

劉備死前告誡其子劉禪的遺詔,其言辭懇切,令人莫不動容。文中,劉備勸劉禪最重要的一句話,便是「勿以惡小而為之,勿以善小而不為。惟賢惟德,能服於人。」古人教子,常以德為根基,因為唯有賢德之人,才能服人。

刘备一生争战,乍看之下胜少败多。劉備攻打益州時,趙戩曾言:“刘备拙于用兵,每战必败。”認為劉備不會用兵,沒本事拿下益州,傅幹卻說:「劉備得人心,又有諸葛亮、關羽、張飛等人傑輔助,怎會不濟呢?」結果劉備果真取攻佔益州。在荊州依附劉表時,曾建議劉表北伐曹操,劉表不接受。劉備住荊州數年,一次與劉表飲酒時起至廁所,見大腿贅肉生,慨然流涕。還坐,劉表奇怪問起劉備,劉備說:「我戎馬半生,常常身不離鞍,大腿贅肉皆消。今天不復騎馬,大腿贅肉生。日月若馳,老年快將至矣,而功業不能建立,是以為之悲嘆。」

刘备与诸葛亮的君臣际遇,通常被史家视为君臣之典范。三顾茅庐后刘备称得到诸葛亮是“鱼之有水”。诸葛亮在刘备尚在时,就已经为丞相录尚书事假节,张飞被暗杀后又领司隶校尉,集政治实际权力于一身,这在古代是很罕见的。刘备去世时举国托孤诸葛亮,被陈寿称为“君臣之至公,古今之盛轨”。

陳壽评曰:“先主之弘毅宽厚,知人待士,盖有高祖之风,英雄之器焉。及其举国托孤于诸葛亮,而心神无贰,诚君臣之至公,古今之盛轨也。机权干略,不逮魏武,是以基宇亦狭。然折而不挠,终不为下者,抑揆彼之量必不容己,非唯竞利,且以避害云尔。”(《三國志·蜀書·先主傳第二》)、「劉備天下稱雄,一世所憚」(《三國志·吳書·陸遜傳第十三》)。尽管刘备并非西晋认为的正统政权,陈寿在三国志内仍然坚持使用同帝王本纪接近的用词,例如在刘备本传称刘备先主,称讳且不直呼其名,去世用和崩相等的殂字。这与三国时代另一位君主孙权的处理手法是不同的。这可以体现陈寿对刘备的尊重,抑或是故国情怀。

刘元起:“吾宗中有此儿,非常人也。”(《三國志·蜀書·先主傳第二》)

陈登:“雄姿杰出,有王霸之略,吾敬刘玄德。”(《三國志·魏書·桓二陳徐衛盧傳第二十二》)

袁绍:“刘玄德弘雅有信义,今徐州乐戴之,诚副所望也。”(《三國志·蜀書·先主傳第二》)

程昱:“观刘备有雄才而甚得众心,终不为人下,不如早图之。” 、“劉備有英名,關羽、張飛皆萬人敵也”(《三國志·魏書·程郭董劉蔣劉傳第十四》)

郭嘉:「备有雄才而甚得众心。张飞、关羽者,皆万人之敌也,为之死用。(郭)嘉觀之,(劉)備終不為人下,其謀未可測也。古人有言:『一日縱敵,數世之患。』宜早為之所。」(《三國志·魏書·程郭董劉蔣劉傳第十四》)

曹操:“方今收英雄时也,杀一人而失天下之心,不可。”、“夫刘备,人杰也,今不击,必为后患,将生忧寡人。”、“刘备,吾俦也。但得计少晚。”(《三國志·魏書·武帝紀第一》)“今天下英雄,唯使君与操耳。本初之徒,不足数也。”(《三國志·蜀書·先主傳第二》)

曹丕:「備不曉兵,豈有七百里營可以拒敵者乎!『苞原隰險阻而為軍者為敵所禽』,此兵忌也。孫權上事今至矣。」(《三國志·魏書·文帝紀第二》)

刘晔:「明公(曹操)以步卒五千,將誅董卓,北破袁紹,南征劉表,九州百郡,十並其八,威震天下,勢慴海外。今舉漢中,蜀人望風,破膽失守,推此而前,蜀可傳檄而定。刘备,人傑也,有度而迟,得蜀日淺,蜀人未恃也。今破漢中,蜀人震恐,其勢自傾。以公之神明,因其傾而壓之,無不克也。若小緩之,諸葛亮明於治而為相,關羽、張飛勇冠三軍而為將,蜀民既定,據險守要,則不可犯矣。今不取,必為後憂。」、「蜀雖狹弱,而備之謀欲以威武自強,勢必用眾以示其有餘。且關羽與備,義為君臣,恩猶父子。羽死不能為興軍報敵,於終始之分不足。」(《三國志·魏書·程郭董劉蔣劉傳第十四》)

贾诩:「吳、蜀雖蕞爾小國,依阻山水,有雄才,諸葛亮善治國,孫權識虛實,陸議見兵勢,據險守要,汎舟江湖,皆難卒謀也。用兵之道,先勝後戰,量敵論將,故舉無遺策。臣竊料群臣,無備、權對,雖以天威臨之,未見萬全之勢也。昔舜舞干戚而有苗服,臣以為當今宜先文後武。」(《三國志·魏書·荀彧荀攸賈詡傳第十》)

孫盛:“刘备雄才,處必亡之地,告急於吳,而獲奔助,無緣復顧望江渚而懷後計。”(《三國志·蜀書·先主傳第二》)

诸葛亮:“刘公雄才盖世,据有荆土,莫不归德,天人去就,已可知矣。”(《三國志·蜀書·董劉馬陳董呂傳第九》)“刘豫州王室之胄,英才盖世,众士仰慕,若水之归海,若事之不济,此乃天也,安能復为之下乎!”(《三國志·蜀書·諸葛亮傳第五》)

关羽:“吾受劉將軍厚恩,誓以共死,不可背之。”(《三國志·蜀書·關張馬黃趙傳第六》)

趙戩:“刘备其不济乎?拙于用兵,每战则败,奔亡不暇,何以图人?”(《三國志·蜀書·先主傳第二》)

傅幹:“刘备宽仁有度,能得人死力。諸葛亮達治知變,正而有謀,而為之相;張飛、關羽勇而有義,皆萬人之敵,而為之將;此三人者,皆人傑也。以備之略,三傑佐之,何為不濟也?”(《三國志·蜀書·先主傳第二》)

孙权:“非刘豫州莫可以当曹操者。”(《三國志·蜀書·諸葛亮傳第五》)「猾虜乃敢挾詐!」(《三國志·吳書·周瑜魯肅吕蒙傳第九》)

周瑜:“刘备以枭雄之姿,而有關羽、張飛熊虎之將,必非久屈為人用者。”(《三國志·吳書·周瑜魯肅吕蒙傳第九》)

陸遜:「備干天常,不守窟穴,而敢自送……尋備前後行軍,多敗少成,推此论之,不足为戚。」、「備是猾虜,更嘗事多」、「劉備天下知名,曹操所憚,今在境界,此强对也。」、「斯三虏者(曹操、劉備、關羽)当世雄杰,皆摧其锋。」(《三國志·吳書·陸遜傳第十三》)

张松:“刘豫州,使君之宗室而曹公之深雠也,善用兵,若使之讨鲁,鲁必破。鲁破,则益州强,曹公虽来,无能为也。”「劉豫州,使君之肺腑,可與交通。」「今州中諸將龐羲、李異等皆恃功驕豪,欲有外意,不得豫州(劉備),則敵攻其外,民攻其內,必敗之道也。」

刘巴:“备,雄人也,入必为害,不可内也。”

彭羕:“仆昔有事於诸侯,以为曹操暴虐,孙权无道,振威闇弱,其惟主公有霸王之器,可与兴业致治,故乃翻然有轻举之志。”(《三国志·卷四十·蜀书十·刘彭廖李刘魏杨传第十》)

锺会:“益州先主以命世英才,兴兵朔野,困踬冀、徐之郊,制命绍、布之手,太祖拯而济之,与隆大好。”

杨戏的《季汉辅臣赞》中赞昭烈皇帝:“皇帝遗植,爰滋八方,别自中山,灵精是锺,顺期挺生,杰起龙骧。始于燕、代,伯豫君荆,吴、越凭赖,望风请盟,挟巴跨蜀,庸汉以并。乾坤复秩,宗祀惟宁,蹑基履迹,播德芳声。华夏思美,西伯其音,开庆来世,历载攸兴。”

诸葛亮上言於刘禅曰:“伏惟大行皇帝迈仁树德,覆焘无疆,昊天不吊,寝疾弥留,今月二十四日奄忽升遐,臣妾号咷,若丧考妣。乃顾遗诏,事惟大宗,动容损益;百寮发哀,满三日除服,到葬期復如礼;其郡国太守、相、都尉、县令长,三日便除服。臣亮亲受敕戒,震畏神灵,不敢有违。臣请宣下奉行。”(《三國志·蜀書·先主傳第二》)

裴潜:“使居中國,能亂人,不能為治。若乘邊守險,足為一方之主。”(《世說新語·識鑒第七》)(《三國志·魏書·和常楊杜趙裴傳第二十三》)

吕布:“是儿最叵信者。”(《三国志·卷七·魏书七·吕布臧洪传》)

吕布诸将:“备数反覆难养,宜早图之。”(《三国志·卷32》注引王沈《魏书》)

习凿齿曰:“先主虽颠沛险难而信义愈明,势偪事危而言不失道。追景升之顾,则情感三军;恋赴义之士,则甘与同败。观其所以结物情者,岂徒投醪抚寒含蓼问疾而已哉!其终济大业,不亦宜乎!”(《三國志·蜀書·先主傳第二》)

常璩曰:「先主名微人鮮,而能龍興鳳舉,伯豫君徐,假翼荊楚,翻飛梁、益之地,克胤漢祚,而吳、魏與之鼎峙。非英才名世,孰克如之!」(《華陽國志·劉先主志》)

裴松之:「漢武用虛罔之言,滅李陵之家,劉主拒憲司所執,宥黃權之室,二主得失縣(懸)邈遠矣。《詩》云『樂只君子,保艾爾後』,其劉主之謂也。」(裴松之注《三國志·蜀書·黃李呂馬王張傳第十三》)

张辅:“刘备威而有恩,勇而有义,宽宏而大略”(《藝文類聚卷二十二》)

朱敬则:“蜀先主抱英济之器,无角逐之材。远窜荆蛮,畏曹公之神武;奄有庸蜀,乘刘璋之政衰。国小人夷,风颓俗陋。”(《全唐文》)

杜甫:“蜀主窥吴幸三峡,崩年亦在永安宫。翠华想像空山里,玉殿虚无野寺中。古庙杉松巢水鹤,岁时伏腊走村翁。武侯祠堂常邻近,一体君臣祭祀同。”

劉禹錫:“天地英雄氣,千秋尚凜然。勢分三足鼎,業复五銖錢。得相能開國,生兒不像賢。淒涼蜀故妓,來舞魏宮前。”

王勃:“以先主之宽仁得众,张飞、关羽万人之敌,诸葛孔明管、乐之俦,左提右挈,以取天下,庶几有济矣。然而丧师失律,败不旋踵。奔波谦、瓒之间,羁旅袁、曹之手,岂拙于用武,将遇非常敌乎?”

司馬光:「昭烈之漢,雖云中山靖王之後,而族屬疏遠,不能紀其世數名位,亦猶宋高祖稱楚元王後,南唐烈祖稱吳王恪後,是非難辨,故不敢以光武及晉元帝為比,使得紹漢氏之遺統也。」(《資治通鑑·第六十九卷·魏紀一》)

苏洵:“项籍有取天下之才,而无取天下之虑;曹操有取天下之虑,而无取天下之量;玄德有取天下之量,而无取天下之才。”

苏辙:“世之言者曰:孙不如曹,而刘不如孙。刘备唯智短而勇不足,故有所不若于二人者,而不知因其所不足以求胜,则亦已惑矣。盖刘备之才,近似于高祖,而不知所以用之之术。昔高祖之所以自用其才者,其道有三焉耳:先据势胜之地,以示天下之形;广收信、越出奇之将,以自辅其所不逮;有果锐刚猛之气而不用,以深折项籍猖狂之势。此三事者,三国之君,其才皆无有能行之者。独一刘备近之而未至,其中犹有翘然自喜之心,欲为椎鲁而不能纯,欲为果锐而不能达,二者交战于中,而未有所定。是故所为而不成,所欲而不遂。弃天下而入巴蜀,则非地也;用诸葛孔明治国之才,而当纷纭征伐之冲,则非将也;不忍忿忿之心,犯其所短,而自将以攻人,则是其气不足尚也。嗟夫!方其奔走于二袁之间,困于吕布而狼狈于荆州,百败而其志不折,不可谓无高祖之风矣,而终不知所以自用之方。”

謝采伯:「孫權運籌於內,劉備、諸葛亮、周瑜、關侯等,合謀並智,方拒得曹操,敗之於赤壁,亦未為竒政縁。」(《密齋筆記·卷二》)

何去非:“方其豪杰并起,而备已与之周旋于中原矣。始得徐州而吕布夺之,中得豫州而曹公夺之,晚得荆州而孙权夺之。备将兴复刘氏之大业,其志未尝一日而忘中州也。然卒无以暂寓其足,委而西入者,有曹操、孙权之兵轧之也。”

萧常:“昭烈父子以帝室支属,介在一隅,而正位号,尚数十年,由先汉至是,垂祀五百,三代以还,葢未之有。人主之结人心,其效廼尔,有大物者,庸可忽诸。”(《萧氏续后汉书》)

郝经:“汉得天统,莽簒而在光武,操窃而在昭烈。魏吴虽僣,犹夫吴楚也。昭烈天资仁厚,宇量(阙)毅,岿然一世之雄。以兴复汉室为己任,崎岖百折,偾而益坚。颠沛之际,信义逾明。故能终系景命,信大义于天下。任贤使能,洒落诚尽,使诸葛亮以死自效。复见三代君臣,高、光为不亡矣。国贼未讨,境土未复,而偾军崩殂,哀哉!”(《郝氏续后汉书》)

陶宗仪:“备又非人望之所归。周瑜以枭雄目之,刘巴以谁人视之,司马懿以诈力鄙之,孙权以猾虏呼之。”(《南村辍耕录》卷二十五)

杨璟:“昔据蜀最盛者,莫如汉昭烈。且以诸葛武侯佐之,综核官守,训练士卒,财用不足,皆取之南诏。然犹朝不谋夕,仅能自保。”

孙承恩:“贤矣昭烈,宽厚弘毅。崎岖立国,仗信履义。推诚任贤,肝胆孚契。顾命数词,可训后世。”(《文简集·卷三十八》)

王夫之:“刘先主以汉室之裔,保蜀土,奉宗祧,任贤图治,民用乂安,尚矣。”(《宋论·卷一·太祖》)

毛泽东:“刘备的优点主要于是善于用人,善于团结各方人士。而缺点则表现在两个方面:一是好感情用事;二是不能区分主次矛盾。”

趙翼:「關、張、趙雲自少結契,終身奉以周旋,即羈旅奔逃,寄人籬下,無寸土可以立業,而數人者患難相隨,別無貳志,此固數人者之忠義,而備亦必有深結其隱微而不可解者矣。」(《廿二史劄記·卷七》)

歷史學家、《三國史話》作者呂思勉認為,如其通觀前後,則劉備急於併吞劉璋,實在是失敗之遠因。倘使劉備老實一些,替劉璋北攻張魯,這是可以攻下;張魯既下,而馬超、韓遂等還未全敗,彼此聯合,以擾中原,曹操倒難於對付;劉備心計太工,不肯北攻張魯,而要反噬劉璋,以至替曹操騰出平定關中和涼州之時間,而且仍給以削平張魯之機。然而本可聯合涼州諸將共擾關中,卻變做獨當大敵。伐吳之役,劉備因為是能做一番事業,意志必較堅定,理智必較細密,斷不會輕易動於感情;況且感情必是動於當時,時間稍久,感情就漸漸衰退,理智就漸漸清醒。然其禍根,亦因急於要取益州,以致對於荊州不能兼顧之故;所以心計過工,有時也會成為失敗原因,真個閲歷多之人,倒覺得凡事還是少用機謀,依著正義而行好。

《劉備傳》作者張作耀認為,劉備人生道路危機四伏、滿途坎坷。這是一個戰鬥歷程:起步、挫折、爬起、再挫,發展至立足一方。撇開劉備政治動機不談,折而不撓、敗不氣餒、終不為下,為憧憬之目標而奮鬥不懈,始終如一。劉備與關羽、張飛一經結義,終身不易。與下士同席而坐,無所簡擇;善待部下,士卒感恩,願為驅使。

\subsubsection{章武}

\begin{longtable}{|>{\centering\scriptsize}m{2em}|>{\centering\scriptsize}m{1.3em}|>{\centering}m{8.8em}|}
  % \caption{秦王政}\
  \toprule
  \SimHei \normalsize 年数 & \SimHei \scriptsize 公元 & \SimHei 大事件 \tabularnewline
  % \midrule
  \endfirsthead
  \toprule
  \SimHei \normalsize 年数 & \SimHei \scriptsize 公元 & \SimHei 大事件 \tabularnewline
  \midrule
  \endhead
  \midrule
  元年 & 221 & \tabularnewline\hline
  二年 & 222 & \tabularnewline\hline
  三年 & 223 & \tabularnewline
  \bottomrule
\end{longtable}


%%% Local Variables:
%%% mode: latex
%%% TeX-engine: xetex
%%% TeX-master: "../../Main"
%%% End:
