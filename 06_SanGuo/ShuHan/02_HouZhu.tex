%% -*- coding: utf-8 -*-
%% Time-stamp: <Chen Wang: 2019-12-17 22:43:48>

\subsection{后主\tiny(223-263)}

\subsubsection{生平}

劉禪(207年-271年),字公嗣,又字升之。蜀漢昭烈帝劉備之子,蜀漢最後一位皇帝,史學家称蜀漢後主,223年—263年在位,歷時四十一年,是三國在位最長之皇帝。

据《三国志》记载,劉禪由刘备的妾室甘夫人所生,是刘备三位庶子中最为年长的。

212年(建安十七年),刘备入蜀,孙权派人接回孫夫人,孫夫人想将五歲的刘禅一并带走,诸葛亮派遣赵云夺回。

刘禅继位初期确实听从父亲的遗命,放权于丞相诸葛亮处理军政大事,“政事无巨细,咸决于亮”。

延熙元年(公元238年),詔命蔣琬應嚴整治軍,率各軍屯紮漢中,等東吳行動,兩國構成東西犄角之勢,伺機伐魏。

劉禪始“乃自摄国事”,由蔣琬、費禕、董允等人主政,修养生息,积蓄力量后从长计议再北伐的政策。劉禪對於寵臣陳祗與宦官黃皓也頗為寵信,姜維畏懼黃皓,只得擁兵屯墾汉中的沓中(今甘肃甘南藏族自治州迭部)。

景耀六年(公元263年),姜維上表後主:「聽聞鐘會治兵關中,欲規畫進一步拓取土地之意,宜一併派遺張翼、廖化督率各軍,分別護陽安關口、陰平橋頭,以防患於未然」,黃皓徵求鬼巫信息,謂敵人終究不會自來,而劉禪也信了鬼巫,滿朝文武竟沒有一人知曉。

最后邓艾偷渡阴平大军压境,刘禅與群臣商議如何抵禦,決定派諸葛瞻領兵迎戰,但諸葛瞻戰敗。最後,刘禅接受谯周的建议,在农历十一月向曹魏投降。劉禪派太僕蔣顯至劍閣,傳令姜維等部投降,蜀軍悲憤不已,紛紛拔刀砍石。邓艾承制拜刘禅为骠骑将军。

蜀漢亡后,刘禅移居魏国都城洛阳,封為安乐县公(常璩则作北巫县安乐乡公)。某日司马昭设宴款待刘禅,囑咐演奏蜀乐曲,并以歌舞助兴时,蜀漢旧臣们想起亡国之痛,个个掩面或低頭流泪。獨刘禅怡然自若,不為悲傷。司马昭见到,便问刘禅:“安樂公是否思念蜀?”刘禅答道:“此間樂,不思蜀也。”他的旧臣郤正闻此言,趁上廁所時对他说:“陛下,下次如司马昭若再问同一件事,您就先注視著宮殿的上方,接著閉上眼睛一陣子,最後張開雙眼,很認真地說:‘先人坟墓,远在蜀地,我没有一天不想念啊!’这样,司马昭就能让陛下回蜀了。”刘禅听后,牢记在心。酒至半酣,司马昭又问同样的问题,刘禅赶忙把郤正教他的学了一遍。司马昭听了,即回以:“咦,这话怎么像是郤正说的?”刘禅大感惊奇道:“你怎麼知道呀!”司马昭及左右大臣哈哈大笑。司马昭见刘禅如此老实忠懇,从此再也不怀疑他,刘禅就这样在洛阳度过餘生,也是乐不思蜀一詞的典故。

西晉晉武帝泰始七年(271年),刘禅去世,諡刘禅為思公。

刘禅太子刘璿在钟会之乱中丧生,按次序应该立次子刘瑶为继承人,但刘禅偏爱六子刘恂,立刘恂为继承人,旧臣文立劝谏,不听,于是刘恂袭为安乐公。

西晉末年,刘渊起事,國號為漢,即汉赵政权,追諡刘禅為孝怀皇帝,但其子孙皆已被灭族,而刘渊是匈奴血统,与刘禅并无直接血缘关系。

刘备在遗诏中说:「射君(射援)到,说丞相叹卿(即刘禅)智量,甚大增修,过于所望,审能如此,吾复何忧!勉之,勉之!」

諸葛亮在與杜微書中評價後主說:「朝廷年方十八,天資仁敏,愛德下士。」

蜀郡太守王崇論後主曰:「昔世祖内資神武之大才、外拔四屯之奇將、猶勤而獲濟。然乃登天衢、車不輟駕、坐不安席。非淵明弘鑒、則中興之業何容易哉。後主庸常之君、雖有一亮之經緯、内無胥附之謀、外無爪牙之將、焉可包括天下也。」“邓艾以疲兵二万溢出江油。姜维举十万之师,案道南归,艾易成禽。禽艾已讫,复还拒会,则蜀之存亡未可量也。乃回道之巴,远至五城。使艾轻进,径及成都。兵分家灭,己自招之。然以钟会之知略,称为子房;姜维陷之莫至,克揵筹斥相应优劣。惜哉!”(華陽國志)

司马昭:「人之无情,乃可至於是乎!虽使诸葛亮在,不能辅之久全,而况姜维邪?」

陳壽於《三国志》:“后主任贤相则为循理之君,惑阉竖则为昬闇之后,传曰‘素丝无常,唯所染之’,信矣哉!礼,国君继体,逾年改元,而章武之三年,则革称建兴,考之古义,体理为违。又国不置史,注记无官,是以行事多遗,灾异靡书。诸葛亮虽达于为政,凡此之类,犹有未周焉。然经载十二而年名不易,军旅屡兴而赦不妄下,不亦卓乎!自亮没后,兹制渐亏,优劣著矣!”、認為劉禪是「素絲無常,唯所染之」,早年得諸葛亮輔助,所以「任賢相則為循理之君」;但後來寵信黃皓,敗壞政事,卻是「惑閹豎則為昏闇之后」。但與暴虐好殺的孫皓相比,劉禪要更為善於處理政務且與大臣們保持著良好的互動。

薛珝:“主暗而不知其过,臣下容身以求免罪,入其朝不闻正言,经其野民有菜色。”

晉朝張華問李密:「安樂公(劉禪)何如?」密曰:「可次齊桓。」華問其故,對曰:「齊桓得管仲而霸,用豎刁而蟲流。安樂公得諸葛亮而抗魏,任黃皓而喪國,是知成敗一也。」(晉書‧李密傳)

裴松之为《三国志·三少帝纪》作注,在评论郭修刺杀费祎事时,称刘禅为“凡下之主”。

孫盛:“刘禅暗弱,无猜险之性。”“禅虽庸主,实无桀、纣之酷,战虽屡北,未有土崩之乱,纵不能君臣固守,背城借一,自可退次东鄙以思后图。”,認為劉禪是「庸主」。

李特:“刘禅有如此江山而降于人,岂非庸才?”(華陽國志)

常璩:“主非中兴之器。”(華陽國志)

张璠:“刘禅懦弱,心无害戾。”

朱敬则:“若乃投井求生,横奔畏死,面缚请罪,膝行待刑,是其谋也。马上唱无愁之歌,侍宴索达摩之曲,刘禅不思陇蜀,叔宝绝无心肝,对贾充以不忠之词,和晋帝以邻国之咏,是其才也。纵黄皓,嬖岑昏,宠高壤,狎江总,是其任也。剥面凿眼,孙皓之刑;弃亲即雠,高纬之志。其馀细故,不可殚论。听吾子之悬衡,任夫人之明镜。”(《全唐文》)

陈世崇:“孔明之子瞻、孙尚战死,张飞之孙遵,赵云次子广亦战死,北平王谌哭于昭烈庙,先杀妻子乃自杀,魏以蜀宫人赐将士,李昭仪不辱自杀。禅不特愧于将士,亦且愧于妇人矣。”

俞德邻:“禅以暗弱之资,而又惑于阉竖,使无此谶,其能与魏争乎?”

郑玉:“孔明盖社稷之臣也,今刘禅昏愚暗弱,纵使伊尹阿衡、周公辅相,亦必危亡而后已,虽百孔明,如之何哉?”“孔明既死,刘禅卒就擒缚。及其入魏,屈辱百端,略无愧耻。岂惟刘氏之宗社不嗣,遂使高祖、光武含羞地下,抱憾无穷。”

王夫之:「後主失德而亡,非失險也,恃險也,恃則未有不失者也。君恃之而棄德,將恃之而棄謀,士卒恃之而棄勇。伏弩飛石,恃以卻敵;危石叢薄,恃以全身;無致死之心,一失其恃,則匍伏奔竄之恐後;扼以於蹊徑,而淩峭壁以下攻,則首尾不相顧而潰。故謂後主信巫言而失陰平之守以亡國,非也。陰平守,而亙數百里之山厓谿谷,皆可度越,陰平一旅,亦贅疣而已。李特過劍閣而歎劉禪之不能守,艸竅之智,乘晉亂以茍延爾。譙縱、王建、孟知祥、明玉珍蹶然而起,熸然而滅,恃險愈甚,其亡愈速矣。」《讀通鑒論·卷十》

罗贯中:“祈哀请命拜征尘,盖为当时宠乱臣。五十四州王霸业,等闲抛弃属他人。”“魏兵数万入川来,后主偷生失自裁。黄皓终存欺国意,姜维空负济时才。全忠义士心何烈,守节王孙志可哀。昭烈经营良不易,一朝功业顿成灰。”

潘时彤:“可惜三分鼎,空怜六尺孤。大权归宦竖,强敌问神巫。斫石军心愤,回天将胆粗。山头曾学射,一矢报仇无。”

《三國志》盧弼集解引周壽昌說:「五丈原头大星夜陨,至千载下犹有余恫。廖公渊、李正方俱为武侯贬退,侯死皆痛泣而卒。李邈何人敢为此疏,直是全无心肝。使非后主之明断,则谗慝生心,乘间构衅,恐唐魏元成仆碑之祸,明张太岳籍没之惨,不待死肉寒而君心早变矣。见疏生怒,立正刑诛,君子谓后主之贤,于是乎不可及。」「(樂不思蜀一事)恐傳聞失實,不則養晦以自全耳。」

清朝方苞《望溪先生文集》中有〈蜀漢後主論〉一文,論曰:「亡國之君若劉後主者,其為世詬歷也久矣,而有合乎聖人之道一焉,則任賢勿貳是也。其奉先主之遗命也,一以国事推之孔明而己不与,世犹曰以师保受寄托,威望信于国人,故不敢贰也。然孔明既殁,而奉其遗言以任蒋琬、董允者,一如受命于先主。及琬与允殁,然后以军事属姜维,而维亦孔明所识任也。夫孔明之殁,其年乃五十有四耳。使天假之年而得乘司马氏君臣之瑕衅,虽北定中原可也。即琬与允不相继以殁,亦长保蜀汉可也。然则蜀之亡,会汉祚之当终耳,岂后主有必亡之道哉!嗚呼!使置後主他行而獨舉其任孔明以衡君德,則太甲、成王當之有愧色矣!」

蔡东藩:“成都虽危,尚堪背城借一,后主宁从谯周,不从北地王谌,面缚出降,坐丧蜀土,是咎在后主。”

魏国史书《魏略》中记载,刘禅在刘备于徐州被曹操攻打时与家人走失,因而被人口贩子拐卖,到了汉中,被一个叫做刘括的人收养。后来刘备入蜀之后,一名簡姓將軍(疑為簡雍)到汉中出使,刘禅找到他并讲解儿时故事,記得父親字玄德,证明自己的确是刘备儿子。张鲁于是下令把刘禅还给刘备,刘备才把他立为继承人。间接来讲,若这个记载为真,赵云在当阳救刘禅以及拦江截阿斗都是蜀汉编造的故事。然而裴松之根据《三国志》的说法对这个记载提出质疑,指出年齡上並不符合,后世也多采信裴松之。


\subsubsection{建兴}

\begin{longtable}{|>{\centering\scriptsize}m{2em}|>{\centering\scriptsize}m{1.3em}|>{\centering}m{8.8em}|}
  % \caption{秦王政}\
  \toprule
  \SimHei \normalsize 年数 & \SimHei \scriptsize 公元 & \SimHei 大事件 \tabularnewline
  % \midrule
  \endfirsthead
  \toprule
  \SimHei \normalsize 年数 & \SimHei \scriptsize 公元 & \SimHei 大事件 \tabularnewline
  \midrule
  \endhead
  \midrule
  元年 & 223 & \tabularnewline\hline
  二年 & 224 & \tabularnewline\hline
  三年 & 225 & \tabularnewline\hline
  四年 & 226 & \tabularnewline\hline
  五年 & 227 & \tabularnewline\hline
  六年 & 228 & \tabularnewline\hline
  七年 & 229 & \tabularnewline\hline
  八年 & 230 & \tabularnewline\hline
  九年 & 231 & \tabularnewline\hline
  十年 & 232 & \tabularnewline\hline
  十一年 & 233 & \tabularnewline\hline
  十二年 & 234 & \tabularnewline\hline
  十三年 & 235 & \tabularnewline\hline
  十四年 & 236 & \tabularnewline\hline
  十五年 & 237 & \tabularnewline
  \bottomrule
\end{longtable}

\subsubsection{延熙}

\begin{longtable}{|>{\centering\scriptsize}m{2em}|>{\centering\scriptsize}m{1.3em}|>{\centering}m{8.8em}|}
  % \caption{秦王政}\
  \toprule
  \SimHei \normalsize 年数 & \SimHei \scriptsize 公元 & \SimHei 大事件 \tabularnewline
  % \midrule
  \endfirsthead
  \toprule
  \SimHei \normalsize 年数 & \SimHei \scriptsize 公元 & \SimHei 大事件 \tabularnewline
  \midrule
  \endhead
  \midrule
  元年 & 238 & \tabularnewline\hline
  二年 & 239 & \tabularnewline\hline
  三年 & 240 & \tabularnewline\hline
  四年 & 241 & \tabularnewline\hline
  五年 & 242 & \tabularnewline\hline
  六年 & 243 & \tabularnewline\hline
  七年 & 244 & \tabularnewline\hline
  八年 & 245 & \tabularnewline\hline
  九年 & 246 & \tabularnewline\hline
  十年 & 247 & \tabularnewline\hline
  十一年 & 248 & \tabularnewline\hline
  十二年 & 249 & \tabularnewline\hline
  十三年 & 250 & \tabularnewline\hline
  十四年 & 251 & \tabularnewline\hline
  十五年 & 252 & \tabularnewline\hline
  十六年 & 253 & \tabularnewline\hline
  十七年 & 254 & \tabularnewline\hline
  十八年 & 255 & \tabularnewline\hline
  十九年 & 256 & \tabularnewline\hline
  二十年 & 257 & \tabularnewline
  \bottomrule
\end{longtable}

\subsubsection{景耀}

\begin{longtable}{|>{\centering\scriptsize}m{2em}|>{\centering\scriptsize}m{1.3em}|>{\centering}m{8.8em}|}
  % \caption{秦王政}\
  \toprule
  \SimHei \normalsize 年数 & \SimHei \scriptsize 公元 & \SimHei 大事件 \tabularnewline
  % \midrule
  \endfirsthead
  \toprule
  \SimHei \normalsize 年数 & \SimHei \scriptsize 公元 & \SimHei 大事件 \tabularnewline
  \midrule
  \endhead
  \midrule
  元年 & 258 & \tabularnewline\hline
  二年 & 259 & \tabularnewline\hline
  三年 & 260 & \tabularnewline\hline
  四年 & 261 & \tabularnewline\hline
  五年 & 262 & \tabularnewline\hline
  六年 & 263 & \tabularnewline
  \bottomrule
\end{longtable}

\subsubsection{炎兴}

\begin{longtable}{|>{\centering\scriptsize}m{2em}|>{\centering\scriptsize}m{1.3em}|>{\centering}m{8.8em}|}
  % \caption{秦王政}\
  \toprule
  \SimHei \normalsize 年数 & \SimHei \scriptsize 公元 & \SimHei 大事件 \tabularnewline
  % \midrule
  \endfirsthead
  \toprule
  \SimHei \normalsize 年数 & \SimHei \scriptsize 公元 & \SimHei 大事件 \tabularnewline
  \midrule
  \endhead
  \midrule
  元年 & 263 & \tabularnewline
  \bottomrule
\end{longtable}


%%% Local Variables:
%%% mode: latex
%%% TeX-engine: xetex
%%% TeX-master: "../../Main"
%%% End:
