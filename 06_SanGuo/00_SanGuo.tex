%% -*- coding: utf-8 -*-
%% Time-stamp: <Chen Wang: 2019-12-17 21:27:09>

\chapter{三国\tiny(220-280)}

\section{简介}

三國(狹義220年-280年,廣義184年/190年/208年-280年。)是中國歷史一段三個國家並立的時期。一般認為是從建安元年起算。三國是指曹魏、蜀漢及孫吳。東漢末年戰爭不斷,使得人口急劇下降,經濟嚴重損害,因此三國皆重視經濟發展,加上戰爭帶來的需求,各種技術都有許多進步。

東漢末年,漢廷因黃巾之亂、北宫伯玉之乱、黑山军起义、王芬谋废灵帝、张举张纯叛乱、外戚宦官火拼等一系列事件而动荡不安。184年漢靈帝時期,以張角三兄弟為首爆发黃巾之亂。為鎮壓黃巾,一方面放權到州牧、太守,一方面縱容地主組織私人武裝,對抗黃巾。董卓亂政並與關東諸勢力對抗後遷都長安,使得朝廷威信喪失,地方长官演变为独立军阀割據混戰。其中曹操擁護逃回洛陽的漢獻帝,遷都許。他擊敗多股勢力,最後在200年的官渡之戰擊敗北方最大勢力袁紹,大致掌控中國北方。曹操以優勢兵力南征荊州,但在208年冬天的赤壁之戰被孫權和劉備聯軍擊敗,形成三國鼎立的雛型。220年曹操病逝,其子曹丕迫漢獻帝禪讓稱帝,國號「魏」,史稱曹魏,至此東漢滅亡,正式進入三國時期。隔年以益州為主的劉備也以漢室宗親的身份稱帝,國號續為「漢」,史稱蜀漢。劉備與孫權在赤壁之戰後積極拓展勢力,為了荊州問題多次發生糾紛與戰爭,最後劉備在夷陵之戰戰敗,孫權獲得整個荊州南部。劉備病死後,輔佐其子劉禪的諸葛亮於同年再與孫權恢復同盟。據有揚州、荊州及交州等地的孫權遲至229年正式稱帝,國號「吳」,史稱孫吳。此後三國局勢主要為蜀吳同盟對抗曹魏,各國疆域變化不大。而曹魏朝廷漸漸地被司馬氏掌控。263年司馬昭為建立軍功準備篡位,出兵伐蜀,蜀漢亡。兩年後司馬昭病死,其子司馬炎廢魏元帝自立,國號為「晉」,史稱西晉,曹魏亡。西晉最後於280年發起晉滅吳之戰,滅亡孫吳,統一中國。至此三國時期結束,進入晉朝。

三國時期人才輩出,後世常常追思當時風雲人物。陳壽所著、裴松之作注的《三國志》,是二十四史中評價最高的「前四史」之一,成為研究三國歷史的重點書籍。而羅貫中結合歷史、民間故事,撰寫的《三國演義》章回小說,成為中國四大名著之一。三國豐富的內涵深入人心。

東漢在漢和帝後,因為種種因素而走向衰亡:皇帝大多在年幼即位,所以政權多由外戚掌控。皇帝成年後為了奪權而尋求宦官的支持,讓宦官勢力掌控朝廷。這種外戚與宦官的對峙即戚宦之爭,使朝廷陷入循環內鬥。到了漢桓帝與漢靈帝時期,士大夫不滿當時掌權的宦官敗亂朝政,紛紛上書抗議,但這兩次的抗議均被宦官與皇帝鎮壓,史稱黨錮之禍。在地方上,各地豪强地主兼併土地,壓榨百姓,人民苦不堪言。加上天災接續不斷,百姓紛紛揭竿起事,成為群雄割據的導火線。

184年,太平道首領張角及兄弟張梁和張寶率数十万信徒發起民變,史稱黃巾之亂。亂事范围迅速扩大,很快发展成全国性的战乱。漢靈帝赦免因黨錮之禍被囚的士大夫以團結內部,命何進佈防京師雒陽,以皇甫嵩、盧植及朱儁等率軍出兵,並令地方州郡和豪強地主募軍協助鎮壓。雖然黃巾軍很快被擊潰,但是朝廷依舊貪汙混亂,民變餘部散佈各地,順勢占山成寇,局勢尚未穩定。為了要穩定局勢,188年漢靈帝採納劉焉建議,讓監察各郡的刺史擁有地方軍政的權力,加強對各郡的控管。並且将部分刺史升为州牧,由劉姓宗室或重臣擔任。這個措施使州正式成为一级行政区,有利于鎮壓各地叛亂。然而當朝廷的威信減弱時,掌握地方實權的州牧及刺史就會割據一方,不受朝廷指揮。例如益州牧劉焉為割據巴蜀,藉由五斗米道首領張魯佔領漢中,切斷與朝廷的溝通。為了穩定國家而制定的州牧制度,反而將東漢推往群雄割據局面。

189年漢靈帝去世,戚宦之爭又起。蹇硕等宦官意圖殺害外戚何进,改立太子刘辩的弟弟陳留王刘协為帝。最後劉辯顺利繼立,即漢少帝。何进為了剷除以張讓為首的十常侍及其他宦官,他與袁紹等士大夫聯手,還命涼州董卓與并州丁原帶兵增援。宦官們先发制人,殺死何進。袁紹等人為替何进报仇,率軍入宮,殺死十常侍等宦官二千餘人,宦官勢力徹底潰敗,戚宦之争就此终结。然而,遭鄭泰和盧植反對的董卓已率軍入援,朝政大權順勢被他奪取。

董卓為了掌權,開始剷除反對者,手段残暴,引起了诸多不满。他促使呂布殺死掌管都城禁軍的丁原奪得其軍隊。袁紹、袁術與曹操等人陸續逃離雒陽,董卓分別封他們渤海太守、後將軍和驍騎將軍以安撫。最後廢除并杀死漢少帝,改立劉協為帝,即漢獻帝。190年,東郡太守橋瑁假借京師三公名義向各地發檄文,陳述董卓惡行,联络各地州牧、刺史及太守讨伐董卓。關東諸郡紛紛舉兵,以袁绍為盟主對抗董卓,史稱董卓討伐戰。董卓面對關東聯軍的軍勢,又擔心背後白波軍可能截斷通往長安的後路,最後不聽從朝廷大臣建議而遷都長安。他挾持漢獻帝,强遷居民到長安,還火燒舊都雒陽,四出掠奪,殺害或免職替袁紹說話的大臣。而關東聯軍也不敢與之決戰,消極的對抗董卓,只有孫堅、曹操二人敢與董卓軍對戰,孫堅還攻入雒陽。董卓遷都後,關東聯軍紛紛解散,如袁紹、袁術等勢力開始擴充版圖,割據一方,開始進入群雄混戰時期。

董卓遷都後自封太师,大封子弟為將為侯,興建郿塢為基地,濫殺大臣,並以王允維持朝廷。漢廷重臣王允、黃琬與士孫瑞合謀,聯繫董卓部下吕布,於192年刺杀董卓成功,其族人亦被屠灭殆尽。然而王允沒有安撫董卓軍餘部,還殺害蔡邕,與呂布不合。董卓属下李傕逃出長安後,听从谋士贾诩「奉國家以正天下」之策,夥同黨羽郭汜、樊稠、張濟等人攻入长安。王允被殺,吕布也兵败逃亡。李傕等人挟持漢帝,专政四年。其間凉州马腾與韩遂等也率軍逼近长安,但被郭汜、樊稠及李傕侄子李利於长平擊潰。195年,李傕與郭汜等人發生內鬥,分別挾持漢獻帝與大臣,长安陷入一片战乱。7月,漢獻帝在張濟、楊定、楊奉與董承等將協助下開啟東歸雒陽之路。李傕與郭汜後悔讓漢帝東歸,聯合追擊漢獻帝。張濟與董承不合,還與李郭合兵攻擊朝廷軍。朝廷軍一直東逃,董承與楊奉還引白波軍、南匈奴右賢王阻擊李郭聯軍。朝廷軍於大陽(今山西平陸東北)獲河內太守張揚與河東太守王邑支援,並與李郭等人講和。196年漢獻帝終於回到成為廢墟的雒陽。關東諸勢力得知此消息,袁紹不聽沮授「挾天子以令諸侯」之策,扶持漢帝。曹操則聽從毛玠「奉天子以令不臣」之策,迎奉漢獻帝到許都(即許昌)。隔年,郭汜卻被自己的部将伍習杀死。198年,曹操派裴茂率领段煨等关中诸将讨伐李傕,李傕被诛杀,至此關中初定。

董卓討伐戰結束後,地方諸勢力對東漢皇帝的安危已經不理會,陸續發展各自的勢力。原董卓部下张济因军中缺粮,途径荆州南阳掠夺,在攻打穰城时陣亡,其軍隊由侄子張繡繼承。張繡被荊州牧劉表安置在宛城,以聯手抵禦曹操。孫堅在攻打劉表據有的襄陽時戰死,其子孫策投奔袁術後向他借兵,帶領孫堅舊部於196年到199年間在江東四處征戰,擊敗揚州刺史劉繇。最後孫策領有江東六郡,與劉表對峙,並等待時機北上中原。197年袁術於壽春稱帝,國號「仲家」,最後被曹操及劉備攻滅。194年益州牧劉焉病死,其子劉璋接任,與漢中的張魯決裂,兩方對峙。馬騰、韓遂等人則於涼州、雍州一帶各自發展勢力。公孫瓚擊敗劉虞後雄踞幽州,最後被袁紹滅亡。東遷的呂布先是奪取曹操的兗州,被擊敗後先附劉備,接著奪其領地徐州,最後被曹操於下邳抓获处死。劉備继任病故的陶谦成为徐州牧,但兩度因為呂布及曹操而失去徐州,不得不依附河北袁紹,後又逃到汝南试图建立勢力。

此时期各势力中成绩最突出的是袁紹與曹操,袁紹先用計佔據韩馥的冀州,繼而打敗田楷、臧洪、公孫瓚等人,掌握青、冀、幽、并四州,雄霸河北,氣勢強勁。曹操四處征戰,收編黃巾軍餘部男女老少,择其精锐组成了著名的“青州军”,幾經轉折,控制了兗州。曹操奉立東行的漢獻帝於許昌后,藉由朝廷名義來討伐各地諸侯;先後破袁術、滅呂布、降張繡、逐劉備。勢力發展成兗、豫、徐三州、部份司隸、雍州等中原地區。由於袁曹雙方的勢力持續壯大,最後發生了決戰。

此時天下局勢分為河北袁紹、遼東公孫度、中原曹操、揚州孫策、交州士燮、荊州劉表、益州劉璋、漢中張魯及涼州的馬騰、韓遂等人。支持袁紹的孫策於200年進攻曹操的廣陵,但遭到陳登頑抗而被擊退。孫策在同年4月遇刺身亡後,繼位的孫權改與曹操和睦。而袁紹見曹操日益壯大,決定率軍南下決戰,史稱官渡之戰。他先後派大將顏良進攻白馬(今河南滑縣)及文醜進攻延津,但相繼被殺。其後,袁紹親自領兵,進軍陽武,而曹操也回兵官渡,深溝高壘,兩軍的對峙長達半年之久。最後曹操於烏巢之戰夜襲焚燒袁軍的糧倉,袁軍軍心大變而潰敗。此戰成為曹操控制北方的重要戰役。

201年袁紹也再率軍於倉亭再戰,又敗給曹軍,最後於隔年病死。曹操趁機擊敗汝南劉備,劉備則逃到荊州去依附劉表。繼承袁紹領地的三子袁尚與其長兄青州袁譚於203年內訌。曹操趁機與袁譚結姻親,並於隔年攻陷袁尚的根據地鄴城。而鄴城成為曹操的主要據點。袁尚與次兄袁熙投奔支持袁家的外族烏桓。隔年曹操陸續攻滅袁譚、幷州高幹。207年曹操以輕兵襲擊烏桓,將領張遼斬殺其首領蹋頓,袁氏兄弟逃到公孫康處。公孫康因懼怕曹操而殺死袁尚、袁熙,並且向曹操歸順。至此曹操大致上掌控了河北與中原地區,並且籌備南下。

孫權在周瑜及張昭的輔佐下,穩定揚州局勢,並發兵攻下江夏,斬劉表將領黃祖。曹操聞孫權勢力漸盛,劉表勢弱,於208年率大军快速南下,意圖立即奪取荆州。而荊州牧劉表於此時病死,其次子、尚未成年的劉琮即位后立刻後向曹操投降。原来依附刘表的刘备率军緊急撤離驻地新野,逃向江陵,逃亡途中在當陽被追上的曹軍擊潰(此即長坂之戰),後獲劉表長子劉琦的接應乘船至夏口。孫權派部下魯肅与劉備聯繫,劉備也派遣諸葛亮出使到江東。雙方經過討論後決定結盟抗曹,成立的孫劉聯軍共約5萬,以周瑜、程普為正副都督。此時曹操所率北方軍與荊州降軍聲稱100萬,事實約15萬至23萬,雙方於烏林(曹軍)、赤壁(孫劉聯軍)隔长江對峙。最後周瑜利用地形風,以火計协同劉備大败曹軍,曹操退回北方,並任曹仁守江陵。此戰役史稱赤壁之戰,大败的曹操从此失去了統一天下的機會,這場戰役也促使三國鼎立的雛型形成。

戰後,孫權與劉備展開反攻。揚州方面,孫權自己率軍攻打曹操領地合肥(第一次合肥之戰),派張昭進攻九江的當塗。守將劉馥與蔣濟的規劃使得孫權屢攻不下合肥而退,張昭的攻勢也失敗。荊州方面,孫權跟劉備在反攻過程出現競奪荊州的現象。周瑜率軍攻擊曹仁固守的江陵,並請劉備派關羽侵擾漢津阻斷江陵糧道。最後周瑜用了約一年的時間成功攻克江陵,史稱江陵之戰。而劉備與劉琦攻打荊南四郡,四郡皆降,並駐守公安(今湖北公安)。孫權為向劉備拉近關係,將其妹嫁給劉備。其後,周瑜曾想出兵攻打益州,達成「竟長江所極」的目標,但他到巴丘後不久即病逝。最後,孫權聽從魯肅的建議,将南郡借給刘备,從而得以聯合抵禦曹操。

此時益州牧劉璋為了防備張魯及曹操的入侵,不顧大臣反對邀請劉備入蜀。劉備率軍與龐統、黃忠、魏延等人入蜀,並不斷的收買人心。在212年劉備與劉璋決裂,展開益州之戰。劉備先後降伏吳懿及李嚴軍團,大軍直逼成都,但被張任堅守的雒城堵住。在劉備受阻的同時,諸葛亮開始率張飛及趙雲入蜀援助,並以關羽鎮守荊州。之後劉備雖然成功攻下雒城,但龐統於攻城時中流箭而死。214年劉備、諸葛亮等聯軍包圍成都,成功逼使劉璋投降,佔領益州。同時,曹操南下攻擊孫權,最後雙方不分胜负。

曹操在南方失利后,出軍西征漢中的張魯,激起馬騰之子馬超與韓遂等涼州、雍州軍閥集結反抗。211年馬超、韓遂的關中聯軍攻下長安,但不久於潼關之戰被曹操以反间计擊潰。曹操進一步將涼州、雍州收為領地。馬超先投靠张鲁,最後投奔劉備。215年曹操率軍迫降張魯,獲得漢中,但他不願趁勢南攻蜀地,派夏侯淵、張郃等人防守漢中。216年曹操就任魏王,加九锡,開始擁有分立於東漢的軍事政治機制。次年,劉備進攻漢中,發動漢中之戰。雙方互有勝負,但漢中主將夏侯淵中計轻敌被劉備將領黃忠于定军山斬殺。雖然曹操率援軍入援漢中,最後因為軍糧不足而退敗,其領地上庸亦歸降劉備。劉備佔領汉中後以魏延为太守,並於219年自封為漢中王。

江東方面,早在劉備獲得益州後,孫權要求劉備歸還荊州,而劉備希望取下涼州後再還。孫權就派呂蒙襲取長沙等三郡,劉備亦率軍東下支援關羽,雙方對峙於湘江。而後劉備得悉曹操得漢中,將危害蜀地安全,便與孫權和談平分荊州,但雙方的關係已惡化。孫權趁曹操西征張魯之際率大軍攻擊合肥,發動第二次合肥之戰,最後被張遼、李典與樂進等人擊潰。魯肅去世後呂蒙繼任其位,他認為不應再依靠劉備抵禦曹操,与孙权、陆逊等策划奪取荊州。219年,劉備佔領漢中並自封漢中王後,因為呂蒙的麻痺戰術,荊州守將關羽忽視江東守軍,於同年率主力軍队北上进攻曹操。關羽於樊城之戰围曹仁於樊城、俘于禁、斬龐德,威震华夏,使曹操有意遷離許都。而後曹操聽從司馬懿建議拉攏孫權對付關羽。在此期间,发生了关羽为解决于禁降兵的粮食问题,抢夺孙权米粮的事情,进一步坚定了孙权袭取荆州的打算。由于糜芳士仁不满关羽,不战而降,关羽布置的东线防御完全失效,使得呂蒙夺得江陵等地,招降武陵,公安與零陵等地区,最後於麥城擒获關羽并处决,孙刘联盟正式破裂。關羽被擊潰後,劉備的上庸守将孟达投降曹魏,至此劉備勢力完全退出荊州。

220年曹操去世,嗣子曹丕继魏王位後,同年12月10日迫使汉献帝禅让帝位給他,國號「魏」,遷都洛阳,史稱曹魏,至此東漢正式滅亡,進入三國時期。隔年221年,劉備在成都稱帝,以漢室宗親身份繼續使用國號「漢」,史稱蜀漢,随后以報關羽之仇為由東征孫權。而孫權则因此向曹魏稱臣,曹丕封孫權為吳王,加九錫。孫權派出接替呂蒙的陸遜迎戰刘备率领的蜀军,222年蜀军慘敗于夷陵、猇亭,史称夷陵之战,不久刘备病死於白帝城。孙权宣布自立年号,不再尊用曹魏的年号。223年,劉禪繼任帝位,軍政大權由諸葛亮掌握。曹丕则與孫權翻臉,兵分三路南下,為孫權所阻而退。諸葛亮於同年向孫權聯繫後,雙方再度結盟,从此三国局勢再度成為东吳蜀汉同盟對抗曹魏,未再发生变化。229年孫權正式稱帝,國號「吳」,史稱孫吳。另外,早在210年交州領主士燮即歸順孫權,226年吳將呂岱實質佔領交州,士燮之子士徽投降。

蜀汉南中地區在劉備去世後發生叛亂,諸葛亮以心戰為方針平定南方之亂,並提出不留兵、不運糧作管理,令南中大致安定;雖然後來仍有零星動亂,但都被馬忠、張嶷等平定。而後諸葛亮轉向對付曹魏,在227年至234年間與曹魏爆發六次戰爭,其中有五次北伐,一次抵禦魏將曹真的南征,史稱諸葛亮北伐。诸葛亮屢次擊敗魏軍,奪取武都、陰平,殺魏將张郃、王雙。但是司马懿採取坚守战略加上蜀道补给困难,諸葛亮最終未能攻克长安,在五丈原逝世。而魏延與楊儀因互鬥而先後去世,令蜀漢失去兩個人才。蜀漢政局由蔣琬、費禕與董允維持,並暫停大規模北伐。蔣費二人相繼去世後,姜維對曹魏展開北伐,但都没能取得明显的戰略效果,並消耗了国力。姜维過於專注北伐,在大臣董允去世後,劉禪寵信的宦官黃皓及陳祗敗壞朝政。等到姜維對付黃皓時反被迫害,只好遷鎮沓中以避禍。

孫吳為了聯合蜀漢,派將領陸遜、諸葛瑾等人於合肥、襄陽、江夏與曹魏對抗,但成果不大。吳帝孫權任用顧雍為相十九年,使吳國大治。諸葛恪等人成功讓山越歸順,安定後方,增加了人口與軍隊。然而孫權到了晚年發生不少失誤,為孫吳的未來帶來隱憂。他先是不聽大臣勸諫,誤信遼東公孫淵會歸降,最後軍馬錢糧被其併吞。淮泗集團和江東集團的利益之爭引發張溫、暨豔事件與呂壹亂政。更嚴重的是二宮之爭,由於原太子孫登的去世,使得孫權鍾愛的孫霸與繼任為太子的孫和發生鬥爭,諸大臣亦分成派系分別支持。最後孫權廢孫和殺孫霸,選擇年幼的孫亮繼承皇位,並誅殺流放一些大臣,其中名將陸遜就在這場鬥爭中死去。252年孫權逝世後,太傅諸葛恪輔佐吳帝孫亮,然而諸葛恪因北伐失利而罪責眾人,大失人心,不久被孫峻和吳帝孫亮所殺。孫峻與從弟孫綝掌握大權後並行恐怖統治,大臣世家牽連身死者多,吳帝孫亮亦被孫綝廢除。吳帝孫休繼位後,便與大將丁奉聯手將孫綝誅殺,但此時國政已江河日下。

曹魏主要戰爭都是抗衡蜀漢與孫吳的攻擊,在魏帝曹丕去去世後由曹真、曹休、司馬懿及陳群四人輔佐魏帝曹叡,而張郃和滿寵都是一方大將。這些將領守衛著魏國,其中以司馬懿最為卓越,他成功抵禦蜀漢北伐,並討於遼東之戰攻滅叛變的公孫淵。在曹叡死後,同為託孤大臣的曹爽與司馬懿發生權力鬥爭。最後司馬懿在249年發動政變,史稱高平陵之變,曹爽及其黨羽被滅族,魏國朝政為司馬懿父子掌握。其子司馬師、司馬昭相繼掌權,展開外除方鎮內廢魏帝的行動。當時守衛曹魏東方的重鎮壽春發生三次反抗司馬氏的舉兵,分別是王淩、毌丘儉與文欽、諸葛誕等三次叛亂,史稱壽春三叛。除王凌外的叛軍雖然獲得孫吳的援軍,最後仍被司馬氏擊潰。司馬氏專政期間,支持魏帝的將領與大臣有的反对司马氏事败,有的自危,于是或被殺害或逃亡至蜀吳二國,而司馬昭在殺害魏帝曹髦後,因已徹底清除異己而準備篡位稱帝。

司馬昭為了建立赫赫軍功,做好篡奪準備,於263年趁蜀漢姜維避居沓中之際發動魏滅蜀之戰。他派鍾會、鄧艾、諸葛緒分兵三路南下漢中,但蜀漢將領姜維即時趕回劍閣,擋住魏將鍾會的攻勢。而鄧艾走陰平小路進逼成都,蜀將諸葛瞻在綿竹戰敗而死,至此劉禪开城投降,蜀漢亡。其後姜维力圖復國,诈降鍾會,并促使鍾會诬陷鄧艾成功将其囚禁。钟会意圖占据西川叛變,但因魏軍將士多有不服,後被衛瓘、胡烈所平。鍾會、鄧艾與姜維在這次動亂中都被殺。當時孫吳也意圖攻入蜀地,但被蜀漢舊將羅憲擋在巴西郡而失敗。

孫吳在孫休去世後,因為太子年幼,濮陽興和張布鑑於國家動盪不安,改立廢太子孫和之子孫皓為帝。然而孫皓繼位後不修內政,殘暴濫刑,濫殺大臣並任用岑昏等奸臣,百姓生活苦不堪言。濮陽興和張布不久也被殺。但孫皓尚能以陸凱為相、以陸抗鎮守荊州江陵,維持吳國軍政。265年司馬昭去世後,其子司馬炎奪取曹魏政權,定都洛陽,國號晉,史稱西晉。司馬炎稱帝後開始籌備伐吳,派王濬於益州大造船艦,以羊祜鎮守襄陽與鎮守江陵的吳將陸抗對峙。羊祜廣施恩惠,採用「各保分界,無求細利」的方針,使荊州百姓的民心被動搖。然而,當西陵太守步闡叛降西晉、羊祜率軍南下支援之際,陸抗仍擊退晉軍並瓦解叛軍,成功地守住荊州地區。嶺南地區亦發生兩次叛亂,264年孙皓即位後,交州向曹魏投降。兩年後吳軍意圖奪回但被晋将毛炅擊敗。269年孫皓以虞汜、陶璜及李勖等人分陸海兩路會師合浦,至271年方奪回交州。279年,修允部屬郭马於廣州(約今廣東省及廣西省)叛變,孫皓先後派滕修、陶浚、陶璜等多方圍剿方平定。同年晉軍率大軍南征,吳國岌岌可危。

孫吳重臣陸凱及陸抗相繼去世後,晉將羊祜提議伐吳,但遭賈充反對而作罷。279年西北之亂始平,王濬、杜預上書司馬炎,認為是時候伐吳,賈充、荀勗等認為西北未定而反對。最後司馬炎決定於該年11月大舉進攻吳國,史稱晉滅吳之戰。他以賈充為大都督,上游王濬唐彬軍、中游杜預胡奮王戎軍、下游王渾司馬伷軍多路並進。280年1月孫皓急任丞相張悌率沈瑩、孫震渡江抵禦王渾軍,但皆戰敗而亡。而王濬軍沿長江配合其他晉軍攻下西陵、江陵、武昌及尋陽等地,杜預也奪下荊州南部。3月,孫皓見晉軍已包圍建業,認為大勢已去而投降。陶璜、滕修等见吴已亡,也都对晋表示归附。孫吳滅亡,西晉統一天下。至此三國時期結束,進入晉朝時期。

三國分為曹魏、蜀漢及孫吳三國。這三國大致繼承東漢的疆域及政區制度,為州、郡、縣三級制。州設刺史或州牧。郡設太守。縣大者置令,小者置長。郡制方面:曹魏河南郡治洛陽,為國都所在,稱河南尹。蜀漢蜀郡治成都,為國都所在。曹魏又設王國,置相,與郡同等;孫吳丹陽郡治建業,為國都所在。另外孫吳於毗陵(今江蘇常州)設有典農校尉,管轄三縣,等同郡。孫吳在一些轄區遼闊的郡下設都尉,冠以東、西、南、北部之名,並有駐所和領縣,其中有不少在后期正式成为郡。縣制方面:曹魏有公國、侯國、伯國、子國、男國之封,蜀漢和孫吳則為侯國,這些皆相當於縣。孫吳又在丹陽郡設有一些都尉,皆相當於縣。

曹魏的疆域主要是在曹操時即大幅發展,至曹丕稱帝建國後定型,約佔有整個華北地區。大致上北至山西、河北及遼東,與南匈奴、鮮卑及高句麗相鄰;東至黃海。東南與孫吳對峙於長江淮河一帶及漢江長江一帶,以壽春、襄陽為重鎮;西至甘肅,與河西鮮卑、羌及氐相鄰。西南與蜀漢對峙於秦嶺、河西一帶,以長安為重鎮。在立國後原有87郡及十二州,有:司隶、徐州、青州、豫州、冀州、并州、幽州、兗州、涼州、雍州、荆州(佔東漢荊州北部)、揚州(佔東漢揚州北部)。曹魏於西域設置管轄海頭(今新疆羅布泊西)的西域長史和管轄高昌的戊己校尉。221年孫權稱藩後,曹魏讓孫權領有荊州牧,將荊揚等孫權勢力則定為荊州,曹魏原直轄的荊州北部改稱為郢州。雙方決裂後曹魏復改郢州為荊州。220年至226年,分隴右置秦州,最後併入雍州。滅蜀漢後分益州置梁州,共增加兩州。

蜀漢為劉備所建,他直到赤壁之戰後才在諸葛亮協助下,由荊州南部開始發展。其勢力一度涵蓋荊州(佔東漢荊州西部)、益州及漢中。立國前後與孫吳發生多次戰爭並損失荊州,於諸葛亮南定南中後獲得雲南一帶疆域,至此漸漸穩定。疆域範圍:北方與曹魏對峙於秦嶺,漢中為重鎮;東與孫吳相鄰於三峽,巴西為重鎮;西南至岷江、南中,與羌、氐及南蠻相鄰。蜀汉共有22郡、僅益州一州。於益州下設庲降都督,治味縣(今雲南曲靖),專轄南中。

孫吳的疆域在孫策時即擁有大部分的揚州。孫權在赤壁之戰後陸續獲得荊州西部、交州,並在擊敗關羽後獲得整個荊州南部。至孫權稱帝後疆域方穩定下來。孫吳北與曹魏對峙在長江淮河一帶及漢江長江一帶,以建業、江陵為重鎮;西與蜀漢相鄰於三峽,西陵為重鎮;東及南至東海南海,其中最南達現在越南的中部。孫吳原有32郡及三州:荆州、揚州、交州。於226年設置廣州,後併入交州,至264年再復設廣州,共增加一州。

三國時期的政治制度和東漢稍有不同,以曹魏改革较多,後來西晉也大多繼承其制度。曹操是因為朝廷大權集中於尚書台才得以掌握大權,魏文帝曹丕為了避免尚書台(行政機關)權力過大,正式分離出中書監(決策機關),政治制度開始走向三省六部制。另外,又新設置移動式的行尚書台,稱為行台制。由尚書台分出部份官員來隨皇帝移動辦事。地方制度方面,出現類似軍區司令的都督制,其中「都督中外诸军事」掌握中央軍政權力,司馬氏三世即皆以此職務掌握曹魏朝廷。孫吳也設有中書令與都督中外诸军事。三國均設有類似御史台的監察機關。

曹操鑑於東漢弊政,用人不重虛德,反對「阿黨比周」,採用「唯才是舉」的方式,並在先後提出三次求賢令。曹操以人為貴,任之以智力,御之以法術,運用到政治上大大改善在漢末戚、宦干政下,用人唯親,政治黑暗的局面。220年曹魏建立,曹操之子魏文帝曹丕接受陳群等提议的九品中正制作為拔選人才的制度,以取代漢代的察舉制度。该制度的主要內容是在地方委任地方士紳為中正官,由中正官以家世、道德、才能為標準評定各地方人士。按這些标准評定出來的结果,會呈上中央作為对人才授官的依據。这一制度由于完全取决于中正官(通常由世族擔任)的喜好或利益,幾乎使世族完全垄断官职。這為西晋世族政治打下基礎,形成「上品無寒門、下品無世族」的局面,直到隋朝才被科舉制度取代。曹魏集團的人才可分為數個部分,核心成員是曹氏夏侯氏宗族勢力,如曹仁、曹洪、夏侯惇、夏侯淵等等。第二部分是漢末大量社會名士,例如荀彧、荀攸、鍾繇、陳群、司馬懿、華歆、王朗等等,這些人才一部分目標是平亂安定漢室,另一部分則是協助曹室篡漢立國。第三部分則是劉氏皇族成員,如劉曄、劉放、劉馥等,大多放棄本身正統包袱,擁護魏室。最後一部分,也是人數最多的,不論出身或是敵將、只要有一技之長的人才,曹操都會重用,如于禁、樂進、張遼、龐德等等。曹魏集團最後形成兩個政治派別,即世族出身的汝穎集團和將帥曹氏夏侯氏為主的譙沛集團,在曹操時期尚共同支撐擁護。但是繼承權之爭,使得曹操長子曹丕與汝穎集團關係較好,其中與陳群、司馬懿、吳質、朱鑠關係最好,合稱四友,最後削弱譙沛集團勢力。

蜀漢前期由诸葛亮以丞相总掌军政,諸葛亮死後不再設置。政事改由尚書令掌握,軍權则以大司马掌军事行政,大将军为最高军事统帅。諸葛亮入蜀後即提倡治實精神,即「治實不治名」。對南中採取攻心為上,取得「夷漢粗安」的效果。對孫吳採取實質外交,為諸葛亮北伐解除「東顧之患」。依法治国,「先理強、後理弱」,打壓豪強安撫百姓,提倡法度規範,約制官職,嚴格遵從權制,廣開誠心,公平行事。人才的部分有三種,第一是隨劉備起家的關羽張飛,第二是在劉備發展過程不斷加入的人物或士大夫,如諸葛亮、龐統與法正等,構成中堅,第三則是蜀中原劉璋的部下,如吳懿、許靖、李嚴等。政律方面,《蜀科》即由諸葛亮、法正、劉巴、李嚴及伊籍等人共同編列。劉禪執政後,政策多由諸葛亮所主持。他在朝內制定規範,訓誡大臣;而朝外風氣清廉,人心不亂。即使连年与魏国交战,蜀汉的经济并未受太大负累,有“亮之治蜀,田畴辟,仓廪实,器械利,蓄积饶,朝会不华,路无醉人”的评论。诸葛亮在世与去世后都得到蜀汉旧地百姓的怀念,其治国能力与效果为当世与后世极为称道。諸葛亮死後,蔣琬、費禕、董允、姜維等都繼續諸葛亮的政策;後來劉禪寵信宦官黃皓,朝政開始變壞。儘管如此,到蜀漢滅亡為止,地方的政風仍算清廉。

孫吳也以丞相掌握政事,為常設之職,有議政參政之權。孫亮繼位時年幼,丞相由宗室孙峻、孫綝先後掌控,廢立君王,權勢盛大。軍權以大司马掌军事行政,大将军、上大将军为最高军事统帅,其中都督中外诸军事的权任尤重。其治國方針大致以限江自保與施德緩刑為主,政治制度大致上跟東漢相近。其政權受南渡的江北世族张昭、周瑜及鲁肃等和居住江南的吳姓世族如丞相顧雍和名將陸遜、陸抗輔佐支持。在農業方面設有復客制來免除部份佃戶課役,實際上減輕地主負擔,開西晉蔭親制佃戶無課役之先聲。雖然孫權在顧雍協助下興修水利,江南獲得開發。但孫權在繼承人之事沒處理好,使得後來的政局不穩。孫權去世之後,朝政後來被權臣孙峻、孫綝等人挾持。吳景帝孫休去世後,大臣認為太子年幼,擁立年紀較長的孫皓為帝。但他卻是一個殘虐和好酒色的君主,進而導致孫吳亡國。

三國之間的外交關係,大多是蜀漢與孫吳同盟對抗曹魏,取得三國互相制衡。吳蜀聯盟是諸葛亮依據隆中對策制定下來的策略,此聯盟經歷建立、破壞、保持三個階段。曹操南征荊州之際,魯肅就勸孫權與劉備結盟,並會劉備於當陽長阪,又對諸葛亮說我是諸葛瑾好友。諸葛亮認為江東無法獨自與中原抗衡,孫權也不會屈服於曹操,認定孫、劉定須聯盟抗曹。周瑜也認為曹軍數量雖多,但兵疲將疑,不需畏懼。最後,孫劉聯盟於赤壁之戰擊敗曹操。破壞階段:劉備佔領荊、益二州後,孫權趁關羽北伐襄樊之際,派呂蒙襲取荊州,孫劉聯盟破裂。而後曹丕稱帝,孫權甚至願意成為曹魏藩屬,受封為吳王。孫權又任用陳化、馮熙、沈珩與曹魏外交,使得曹丕在軍事上猶豫不決。而後劉備伐吳,被陸遜於夷陵之戰擊敗。曹丕至此才以藉口發兵三路南征孫權,但都被孫權沿長江抵禦強敵。最後是保持階段:劉備去世後,諸葛亮馬上派鄧芝東去與孫權重修好,孫權聽從鄧芝建言後,就自絕曹魏,與蜀漢聯合,並且派張溫與蜀漢和好。之後諸葛亮又派費禕、陳震與吳通好。到孫權稱帝後,雙方甚至協議平分中原,這些都體現了諸葛亮治實精神。

三國與外族互動的部分,或聯合攻擊敵人,或攻滅以除後患與補充人口。而匈奴、鮮卑、羯、羌及氐等族也陸續遷居中原,到西晉時涵蓋了關隴、并州及幽州等地區。東北方面,有高句麗、沃沮、東扶餘及三韓、百濟。204年在遼東割據的公孫康率軍攻破高句麗王都,迫使新王伊夷模東遷至國內城。246年,曹魏毌丘儉率軍擊敗高句麗。公孫康置帶方郡後與百濟聯姻,之後百濟併帶方郡而立國。日本的邪馬台國興起後,遣使納貢曹魏,魏明帝封邪馬台國女主卑彌呼為親魏倭王。魏晉以來天山以北及蒙古草原的民族主要有烏孫、堅昆、敕勒、丁零、呼揭、匈奴、鮮卑及烏桓等族。東漢之後匈奴分為南北;51年,南匈奴大多徙居在幷州中部的汾水一帶。188年单于于扶罗趁中原內亂之际率軍入侵。202年南匈奴歸附曹操後,曹操將南匈奴分成五部,每部立帥長,並派漢人監督。烏桓族長蹋頓與袁紹結盟,並獲得了單于的封號。205年,曹操擊敗袁尚,袁尚與袁熙兄弟逃至蹋頓處。而後曹操率精銳遠征烏桓,於白狼山斬殺蹋頓,降服烏桓。鮮卑在東漢末期由檀石槐統一,屢次入侵東漢,其死後鮮卑分裂為東部、中部及西部鮮卑。西部鮮卑軻比能重整鮮卑後兩度入侵曹魏,並響應諸葛亮攻魏。235年,曹魏幽州刺史王雄遣刺客將他暗殺,其勢瓦解。

西部方面,西羌於三國時期開始遷居中原,分佈於中國中部的山地地區。當時河西諸羌和武都、陰平的羌族分別歸附曹魏及蜀漢。這兩國相互攻伐時都徵召羌族參加作戰。氐族方面,在東漢末期,興國氐王阿貴與百頃氐王千萬各擁部落,後為曹操所破。曹操恐劉備取武都以逼進關中,乃遷其人五萬餘落於扶風、天水等郡。曹魏初,又有武都氐部歸附內徙。當時西域地區有鄯善、高昌、焉耆、龜茲及于闐等國。魏文帝派官員管理西域地區,加強與西域各國聯繫,然而影響力不大。魏文帝還於229年封大月氏王波調為親魏大月氏王。西南方面,225年蜀漢丞相諸葛亮率軍平定南中之亂,降伏南蠻(西南夷)族長孟獲,並設置庲降總督管轄。往後雖有叛變發生,但皆不大。此時期在南方共有三大蠻族,分別是分佈巴郡、江陵及淮水一帶的廩君蠻;分佈武陵、長沙一帶的槃瓠蠻,又稱傒人;分布在巴郡閬中一帶的板楯蠻,又稱賨人。夷陵之戰時蜀漢也曾遣馬良聯絡武陵的槃瓠蠻共討孫吳。

孫吳內部還有山越,其為據守江南山地各族人的總稱。他們自給自足,且與曹魏聯繫,孫吳屢次征討皆難以根除。234年諸葛恪使用堅壁清野的戰術圍山三年,降伏山越,並收編其精壯為軍隊。在嶺南地區還有俚人,範圍涵蓋孫吳廣州、交州及蜀漢益州南部。孫權也展開海上的發展,他派使臣朱應、康泰泛海到夷洲(可能為現今台灣或琉球)、亶洲補充人口、到遼東、朝鮮半島、林邑(今越南南部)、扶南(今柬埔寨)和南洋群島等地溝通聯繫,這些都擴大孫吳在海外的影響力。大秦商人和林邑使臣也曾到達吳都建業。

三国军事制度大部分沿用汉制,但是又有几个重大变革,產生許多制度以及部曲的興盛。世兵制起源自漢末的質任制,當時軍閥為避免士兵逃散,將其家屬集中管理,形成軍戶。由於長年戰亂,最後出現專司作戰的「軍戶」、「士家」,子承父業,甚至祖孫三代都為兵。而且年老之後也不能退役,改為從事後勤運輸方面的工作。世兵制是对于东汉募兵制、徵兵制并行制度的重要变革,并逐渐取代了前者,保持了很多势力稳定的作战力量。內軍外軍制度與都督制:内军或中军负责政治中心的治安防卫,外军负责边境、军事要地的守卫,并从事屯田,互不统属。中军的领导為中护军、中领军,除了掌握中军還要负责各级军事将领的选拔工作。由於中军為「都督中外諸軍事」,掌握軍隊中樞,往往成為權臣夺权的重要途径之一。曹魏將分軍隊為中军、外军和州郡军(地方军队)。而蜀汉與孫吴也有分外军为中、前、左、右、後五軍。戰區依都督制可加設「都督諸州軍事」,其軍政和一,多跨越州郡。例如,曹魏設有雍涼都督和揚州都督。蜀漢設有漢中都督、永安都督和庲降都督。孫吳也於西陵、江陵、巴丘、交州及广州等地設立都督。部曲在漢代本是軍隊編制的名稱,後泛指私人統率的軍隊。東漢末期戰亂連年,許多苦於戰亂的農民都去請求武裝的世族豪強保護。而世族大姓為聚眾自保而收編農民,敵人入侵時為部曲而作戰,平時則為佃客從事生產。後來大力發展成為私家軍隊。其中孫吳實行世襲領兵制,合法化的使將領與士兵建立世代的隸屬關係。

在三國各軍特性方面,魏軍主要區分為步軍、騎軍和水軍,此外還設有虎豹騎、烏桓騎兵等精銳軍隊。但是當時的馬鐙較少,比較珍貴。在前期,兵員靠募集、徵發及強制降俘和少數民族為兵等。到後期,逐漸形成世兵制,並成為主要集兵方式。為了兵源穩定,曹魏實施军户、民户严格分离,除其子世代为兵外,军户的妻女也只嫁军户,保证其繁衍。

吳軍以舟師為主,步兵次之。孫吳水軍發達,在濡須口(今安徽巢縣東南)和西陵(今湖北宜昌)設有水軍基地,在侯官(今福建閩侯)設有造船廠。其所造名為「長安」、「飛雲」、「蓋海」等樓船,皆有五層,可載3,000名士兵。272年晉武帝升王濬為益州刺史,並密命其於四川組建樓船,以滅東吳,其所造之船,最大的可載2,000多人,且能在船上馳馬往來。孫吳的精銳軍隊有車下虎士、丹陽青巾軍與交州義士等,還有設有山越兵、蠻兵、夷兵等少數民族部隊。由于比较特殊的社会政治环境,孫吴除了有世兵制外,还有世襲領兵制。各将领所領軍隊算是其部曲,軍隊除了服从中央指挥参与战役,但还要为其将领提供其它耕种杂役等。在將領死后軍隊須繼續聽令將領之子或其弟等繼承者。

蜀軍以步兵為主,騎兵次之。蜀漢亦編有少數民族部隊,主要有賨兵、叟兵、青羌兵等,當中以「無當飛軍」最出名。在武器裝備方面,蜀漢比秦漢時有所發展。兵源方面,蜀汉實施世兵制,由于人口远远少于其它两国,也實施徵兵制來補充兵源。诸葛亮發明八陣圖以利立營練兵。在補給方面设计出木牛流马以利山地運輸。他還制造出一弩发十矢的連弩,並以此編制成「連弩士」。

黃巾之亂后,中原地區發生天災饑荒,以至出現人吃人事件。董卓掌權後,放縱士兵淫略妇女,剽虏资物。在面對關東聯合軍逼近下,強遷洛陽數百萬人民到長安,還焚燒宮廟官府居家,二百里內無復孑遺。以至於民怨載道,人口數大減。曹操征徐州時,坑殺男女數十萬人,雞犬無餘,泗水為之不流。當時李傕據關中,三輔尚還數十萬戶,但是李傕出兵掠奪,加上飢荒,人民在兩年自相食殺略盡。益州的刘焉、刘璋及荆州的刘表鎮壓叛亂,扬州因為孙策等人的戰爭,使得人口數都減少。

當時的人民朝三個方向流動:由關中西遷至涼州或是南遷至益州、沿漢水遷移至荊州,各約十萬戶。由中原地區往東北遷移至冀州或幽州,再遷至遼東。鮮卑和烏桓也因為這波流民而壯大。最後也是最大一股,是由中原地區遷移至徐州彭城,再南遷至江南地區。當時「是時四方賢士大夫避地江南者甚眾」,孫吳立國的基礎即建立在此上。例如:魯肅、諸葛瑾、呂蒙、張昭及徐盛等人就是此次南渡的中原士族之一。

自三國鼎立局勢漸漸形成後,人民轉而因統治者或戰爭而被迫遷移。曹操攻擊張魯時及攻下後,共遷部份的川東漢中居民入關中。曹丕建都洛陽後,遷冀州五万户士家以实河南。魏灭蜀后遷蜀人三万家至洛阳和关中。劉備領有益州,多次迁民于成都平原。諸葛亮第一次北伐失敗後,也遷隴西居民以實漢中。孫權在早期即擊敗江夏太守黄祖,虏掠男女数万口。他建國後為了提昇人口數,平定山越並以其「羸者充戶,強者補兵」,並且騷擾淮南來獲得人口。

以下表格可知人口銳減趨勢。由東漢晚期到西晉統一全國,雖然時間儘隔125年,但人口只有東漢人口峰值的35.3\%。至此戶口一蹶不起,至到隋文帝在位时方漸復甦。另外值得注意的是人口高度的军事化,當時三國控制的人口還有兵戶、吏戶、屯田戶等。例如曹操早在創建時期即推行屯田制。蜀汉人口雖只有九十萬,但是却有十萬多的軍隊,佔總人口十分之一。而屯田户数量之大,对当时社会经济的恢复和发展起着决定性作用。

東漢末年,因為天災戰亂,社會受到破壞,使得經濟衰退,大量農地荒廢。部份豪強世族紛紛率領族人,建立塢堡以自衛。在其周圍從事生產活動後,漸漸成為自給自足的莊園制度。塢堡和莊園制度都影響後來魏晉南北朝的經濟模式。由於東漢朝廷的崩潰,無人重鑄磨損不堪的銅錢,加上大量私錢出現。到三國鼎立後,新發行的銅錢未能廣泛通行,只好正式以布帛穀栗等實物為主要貨幣。

曹魏、蜀漢、孫吳三國當中,以曹魏人口最多,墾荒的面積最廣,這正是當時三國中以曹魏實力最強的原因。曹魏推行屯田制,組織流民耕種官田。使得稍加恢復社會秩序,增強曹魏實力。曹魏重視農業的另一實證是其大興水利,其工程的規模和數量在三國中首屈一指。如關中一帶闢建渠道,興修水庫,一舉改造了三千多頃鹽鹼地,所獲使國庫大為充實。再如曹魏在河南的水利工程,其成果使糧食產量倍增,但《三國食貨志》也指出這些水利工程許多缺乏規劃,僅能收短期效果。曹魏建置大型官營手工業作坊,發展手工業生產。鄴、洛陽等貿易城市,商業經濟發達,和海外有貿易往來。此外造船業、陶瓷業、絲織業、製鹽業等等也都十分發達。值得注意的是曹魏一直無法擺脫實物交易的經濟模式,少數幾次的貨幣改革嘗試都以失敗收場,這可能與其國土內缺乏大規模的銅礦礦山作為基礎有關。

蜀漢土地肥沃,物產豐饒,東漢末年遭受的戰亂也較中原為輕。劉備入蜀後,巴蜀地區財政混亂,劉巴提出鑄直百錢,平衡物價,解決問題。當中五銖錢與直百錢並用,為犍為郡所鑄,從中知道蜀鑄錢不只在一地,而蜀錢終三國一代也一直是蜀國重要的輸出品,甚至連魏國都大量流入跟通行。諸葛亮又派人整修和護理都江堰,保障農業灌溉。蜀漢的手工業以鹽、鐵和織錦業等最為發達。張華《博物志》提到諸葛亮發展蜀鹽,利用天然氣,大幅提高蜀鹽產值。左思《蜀都賦》中提到「闤闠之裏,伎巧之家。百室離房,機杼相和。」,所以蜀錦能遠銷吳、魏二國,諸葛亮亦認為蜀錦為支持國家的重要物資。 而南中金、銀、丹、漆、耕牛、戰馬等貢品,令蜀漢軍費有所供給,國家富裕。至蜀漢亡時,官府仍有金、銀各二千斤。首都成都也是當時的商業都市之一,《蜀都賦》提到「市廛所會,萬商之淵;列隧百里,羅肆巨千;財貨山積,纖麗星繁。」。

孫吳所處的江南,社會經濟起步較晚,在三國時還是人口稀薄之地。然而由於這裡戰亂較少,使得北方人民大量遷居,帶來先進生產技術和勞動力。孫權登位後設置農官,實行屯田制,江南地區的農業生產和社會經濟得到發展。紡織業方面,江南以產麻布出名,豫章郡(今江西南昌)的雞鳴布名傳千里。三吳出產「八蠶之綿」,諸暨、永安一帶所産絲的質量很高。冶鑄業以武昌(今湖北鄂州)爲最發達,孫權曾在開採銅礦,打造兵器。由於地處江南及海邊,吳國在造船和鹽業都相當發達,在海鹽(今浙江海鹽)、沙中(今江蘇常熟)設官員,來管理這兩地的鹽業生産。孫吳在建安郡(今福建福州)設典船校尉,海船南抵南海、北達遼東。海上貿易亦有所興起,孫吳的商業都市以建業(今江苏南京)、吳郡(今江苏苏州)、番禺(今广东广州)為主,其中番禺以國外貿易為主。

漢晉之際的學術思想發生劇烈的變動,主要受傳統思想的變化與政治鬥爭有關,前者成份居大。由尚交遊、重品藻,反動而變為循名責實,歸於申韓。因尚名務虛偽反動而為自然、率直,歸於老莊。由於東漢晚期政治敗壞,局勢混亂。曹操與諸葛亮採用名家或法家的思想來恢復社會秩序。曹操提倡信賞必罰,主張法治。提出「用人唯才」的觀念打破以門第或名教的標準。諸葛亮也提倡法治觀念,入蜀後修明法制,執法公平。提出「治國之要,務在舉賢」的主張以任才適用。他也重視軍法,如街亭之戰馬謖違反軍令而被斬,他也自貶三等。漢末魏初的名法思想為此後魏晉玄學思潮提供了基礎,使名士基於政治黑暗將焦點由名法的具體問題轉向玄學的抽象思辨。

經學方面,漢末鄭玄之經學已甚受推崇。然而在魏晉之世,王肅繼承父學而註經,其對經學的見解與鄭玄不同,遂有鄭、王兩派互相駁難。西晉篡魏後,晉武帝司馬炎為王肅外孫,王學遂被立為官學,黜鄭申王,王學成為一時宗主。然儒家經學已經陷於窮究章句之僵化境地,日益不能適應現實的需要。一些士人開始回歸傳統文化,研究道家思想。玄學應運而生。

玄學是魏晉南北朝時期最突出的思想。《老子》、《莊子》和《周易》是主要研究對象,合稱三玄。玄學家好談玄理,不談俗事,稱為清談。曹魏後期正始年間,名士何晏、王弼尚談老、易,王弼注釋《老子》並以老子思想解釋《周易》,從而開啟玄學清談之門。主張萬物皆生於無,以無(道的原理)為本,有(表象)生於無。進而,提出『名教出於自然』,名教與道的關係,是子與母的關係。魏末晉初,司馬氏已經權傾天下,黨同伐異,篡魏之心昭然若揭。一部分士人既不願意與司馬氏合流,又無力改變現實,於是生求道出世之意。提倡老莊思想的自然真性,鄙視司馬氏以儒家名教束縛世人的虛偽。以阮籍、嵇康等竹林七賢為代表,他們把焦點由思想理論轉移到人生問題上。當時司馬氏以崇尚名教自飾。嵇康提出『越名教而任自然』,嵇阮等人認為儒教禮法壓抑自然真性,強調『心與善遇』而回歸真誠無偽的人性。他們不拘禮俗,甚至放浪形骸,以此不賢行為拒絕為司馬氏徵用。嵇康被殺後,竹林玄學陷於沉寂。西晉武帝死後,晉惠帝時期,朝綱紊亂,政治凶險黑暗。玄學重新興起。

而蜀漢繼承東漢儒學道統,蜀漢劉禪時期,諸葛亮奏請冊封甘夫人為漢帝劉備皇后,就是按照儒家禮制來做。劉備、諸葛亮對蜀地的儒學、儒生都是尊重的。任用杜微、周群、杜瓊、孟光等等人才,或為做官,或為儒林校尉、典學校尉、勸學從事,有的被任命為太子家令、太子僕、太子庶子。蜀漢的政治指導思想,和東漢一樣,都是儒。至於諸葛亮推崇法家,但並未放棄德政,儒家也要刑法,但以德政為最終理想。

三國文學中以曹魏文學最盛,分為前期的建安文學及後期正始文學,其中建安文學反對靡弱詩風,被後人稱為「建安風骨」或「漢魏風骨」。這是因為自曹操等人熱愛文學,各地文士紛紛吸附。建安文學代表人物為「三曹」及「建安七子」。其他的文學家還有邯鄲淳、蔡琰、繁欽、路粹、丁儀、楊修、荀緯等。曹操具有沉雄豪邁的氣概,古樸蒼涼的風格,著有《短歌行》、《步出夏門行》、《讓縣自明本志令》等文。曹丕及曹植才華洋溢,曹丕著有文學評論《典論》,導致文學開始自覺發展,加上他本身亦從事文學創作,擅寫七言詩,故亦躋身「三曹」之列。曹植具浪漫氣質,著有《洛神賦》等文。建安七子與蔡琰、楊修等人關心現實,面向人生。他們的作品反映了漢末以來的社會變故和人民所遭受的苦難,例如蔡琰的《胡笳十八拍》。

正始文學時期,由於當時政治形勢受司馬氏操控,文人備受壓抑,難以直接面對現實。當代的作家有竹林七賢的嵇康、阮籍及何晏、夏侯玄、王弼等「正始名士」。司馬懿在高平陵政變擊潰曹爽等皇室勢力,至此司馬氏掌握魏室。而司馬師、司馬昭對反對派採取高壓政策,使得正始文人大多寒蟬不敢作為,轉而通老莊,好玄學。對於社會現實,不如建安作家那樣執著,持比較沖淡的態度。然而嵇康的散文和阮籍的《詠懷詩》尚繼承「建安風骨」,敢於面對司馬氏政權,其文學都有鮮明的特色。《文心雕龍》提到「正始明道,詩雜仙心。何晏之徒,率多浮淺。惟嵇志清峻,阮旨遙深,故能標焉。」說明了阮籍和嵇康皆為正始文學的代表詩人。

孫吳作家有張紘、薛綜、華覈、韋昭等。張紘為孫權長史,與建安七子中的孔融、陳琳等友善。薛綜為江東名儒,居孫權太子師傅之位。華覈則是孫吳末年作家。蜀漢作家有諸葛亮、郤正、秦宓、陳壽等。諸葛亮作為一代政治家,他的作品有《出師表》等。其文彩雖不如他人豔麗,然而內容淺易,情意真切,感人肺腑,表露出他北伐的決心。秦宓所寫的五言詩《遠遊》,是蜀漢流傳下來唯一可靠的詩篇。蜀中亦多有學者為書作注,如:許慈、孟光、尹默、李譔等,蜀漢後期有譙周、郤正都醉心於文學,譙周更寫下了《仇國論》討論過度征戰的缺點,及郤正以依照先代的儒士,借文表達意見的《釋譏》。 東漢末年亦有研究纖圖、術數的學者,如:任安、周舒,之後出現了周群、杜瓊等人。

三國時期有名的史學家有王沈、魚豢、韋昭及陳壽。王沈的《魏書》被史學家劉知幾評為「其書多為時諱,殊非實錄」,這跟他親附司馬氏勢力,打壓魏帝曹髦有關,故該書的參考價值也相對較低。韋昭善寫史,著有〈吳鼓吹曲十二曲〉,內容為整部孫吳發展史,與繆襲的〈魏鼓吹曲十二曲〉南北相對。他又著有《吳書》55卷等。陳壽編寫的《三國志》為前四史之一。他參考《吳書》及魚豢撰寫的《魏略》等資料,採三國並述的方式,創新紀傳體史書的寫作模式。雖仍有不足之處,但實為研究三國歷史不可或缺史籍之一。

本時期為佛教與道教的發展時期。由於天災人禍不斷,人民紛紛尋求宗教慰藉心靈,使得能夠逐漸發展。南中諸夷族的原始宗教,具有很濃厚的巫風。其性質是神話崇拜,具有多神、崇拜自然的特點。在西南地區有長遠的歷史,形成早期的原始宗教。

東漢民間流行黃老之學,張角建立的太平道和張道陵建立的五斗米道,都是道教的雛型,到西晉時則稱為天師道。張角的太平道,在道術方面較重「守一」。以《太平經》為主要經典,又稱《太平青領書》。內容龐雜,「其言以陰陽五行為家,而多巫覡雜語」。其社會思想既有維護統治階級利益的部分,也有呼籲公平、同情貧苦人民的部分。張角擁有廣大教眾後,於東漢末期率其弟張梁、張寶與部屬張曼成發起「黃巾之亂」,最後被東漢朝廷擊敗而漸漸式微。張道陵於漢順帝時入四川鶴鳴山,造作符書,創建五斗米道。該教可能是黃老之學與當地宗教的融合,符文大多源至巴蜀巫術。五斗米道與太平道教理教義基本相同,事奉黃老之學。張魯使教內祭酒誦習《老子五千文》,《道德經》成為主要經典之一。《老子想爾注》反映早期道教對《老子五千文》的解釋。經其子張衡、其孫張魯的傳播,流行於四川與漢中一帶。張魯投降曹操後,五斗米道由巴、漢流傳到江南一帶。

佛教早在西漢末期即傳入中國,但當時儒學興盛,發展不大,至三國後方有發展。不同於兩漢時傳入中國的小乘佛教,東漢後期源於印度的大乘佛教受貴霜帝國影響而傳播四周。西域受其影響,于闐、龜茲等地佛教興盛。之後又有天竺曇柯迦羅、安息曇諦和康居康僧鎧等僧侶到洛陽翻譯經典,將大乘佛教傳至中國。曇柯迦羅推廣戒律,這是中國僧侶有戒律受戒之始,後世以其爲律宗的始祖。曇諦所譯的《曇無德(法藏)羯磨》受朱士行等人戒守,一般以此爲中國僧侶出家之始。由於當時經文翻譯未善,朱士行為求原經研讀,於260年自雍州出發至于闐,成為首位西行求法的中國僧侶。他寫得《大品般若》的梵本,後由弟子於282年送回洛陽,最後由竺叔蘭譯成《放光般若經》。發展方面,在東漢末期笮融曾於江東大興佛寺。三國時期的佛教重鎮,北方以洛陽為主,南方則為建業。曹魏魏明帝大興佛寺,曹植也喜讀佛經,並創作梵唄。孫吳方面,當支謙、康僧會先後入吳,受孫權推崇並支持發展。孫皓稱帝時,本要毀壞佛寺,因康僧會說法感化,終而放棄。在蜀漢,佛教不是很興盛,規模不大。

三國在藝術方面,孫吳有很多擅長各種藝術的名士,時人稱為吳國八絕。有吳範、劉惇、趙達、嚴武、皇象、曹不興、宋壽和鄭嫗等人。例如嚴武擅下圍棋,同輩中無人能勝,有「棋聖」之稱。至於曹不興則擅繪畫、皇象則擅書法。

東漢末期動亂不堪,許多畫作被破壞或遺失,造成損失。佛教的發展,開始出現以佛教為題材的繪畫。三國時期的繪畫,因政治動蕩、社會混亂而沒有取得更大的成就。三國之前,繪畫主要屬於「百工之苑」的技術性職業,尚未藝術化,在本時期開始出現現實題材的內容,亦是由禮教宣傳過渡到宗教宣傳的時期。畫家也由黃河流域的中原地區轉移到長江流域。當時有名的畫家有曹不興、吳王趙夫人,其他擅長繪畫的有桓範、楊修、魏帝曹髦、諸葛瞻等人。孫吳曹不興,擅長寫生與繪佛畫,譽稱「畫佛之祖」。他曾把五十尺絹連在一起,畫一人像,心明手快,運筆而成。孫吳吳王趙夫人,是趙達之妹,善於書法山水繪畫,時人譽為「針絕」。她為孫權繪各國山川地形圖,實開山水畫之首。漢末楊修相傳有《西京圖》等畫。曹魏桓范擅長丹青,魏帝曹髦繪畫人物史實。蜀漢諸葛亮父子亦工書畫。

書法藝術興起於東漢末期。從三國到西晉,隸書仍是官方通行的書體,當時的碑刻大都用隸書寫成。曹魏碑文書體方正、氣度莊嚴,少有生趣。孫吳的著名碑刻有《天發神讖碑》、《禪國山碑》、《谷朗碑》等。其中《天發神讖碑》以圓馭方,勢險局寬,氣勢雄偉奇恣。本時期主要的書法家有張芝、張昶、韋誕、鍾繇及皇象等人。張芝擅章草,並創新出今草。出名的作品有《冠軍帖》、《今欲歸帖》等。張昶為張芝季弟,擅長章草與隸書。韋誕總結書法經驗,著有《筆經》。鐘繇《宣示表》、《薦季直表》等作品為楷書經典之作。皇象擅小篆、隸書,尤精章草。流傳作品有《急就章》、《文武將隊帖》及《天發神讖碑》等。诸葛亮亦长于书法,有《远涉帖》(现今流传版为王羲之临摹)存世。

機械學方面:馬鈞是曹魏陝西扶風(今陝西興平縣)人,知名發明家。他擅長機械應用,提昇生產量,製作出水轉百戲和指南車,榮獲「天下之名巧」的美譽。他改良漢代的織綾機,使織出花紋具立體感,能與蜀錦相媲美。改良漢末畢嵐的龍骨車,發明出龍骨水車來灌溉較高位的農田。現在部份梯田仍在使用。他還將發石車改造成輪轉式發石車,提昇拋擊量與速度。諸葛亮為了方便在山地棧道運輸,發明「木牛流馬」。其構造歷代文獻有異,學者一般認定木牛為四輪車及流馬為獨輪車,目前未有最終定論。他發明可以連續發射十箭的連弩,又稱「元戎」。

刘徽是曹魏数学家,山东淄博淄川人。他自幼對數學有興趣,學習中國古代數學的重典《九章算術》。年長後於曹魏景元四年(263年)著有《九章算術注》,藉由自己的註解,使其容易了解。之後劉徽又著作《九章算術注》的第十卷,即《重差》。唐代將《重差》從《九章》分離出來,單獨成書,按第一題「今有望海島」,取名為《海島算經》,是《算經十書》之一。劉徽運用二次、三次、四次測望法,是測量學歷史上領先的創造,使中國測量學達到登峰造極的地步。劉徽另著有《魯史欹器圖》、《九章重差圖》等。

医学方面,有名的有華佗、張仲景和皇甫謐。華佗醫術精湛,擅長外科手術。他與董奉、張仲景被史書稱為「建安三神醫」。華佗可能是最早使用麻醉剂「麻沸散」进行外科手术的医者。張仲景鑑於當代動亂頻繁,疫病流行,致力研究疾病,參考各家書籍寫出《傷寒雜病論》,该书序言中有提到自己从医的动力之一便是其家族中有过半的人死于伤寒等疾病。該書集兩漢醫經、經方二派的大成,是中醫史上第一部理法方藥具備的經典,喻嘉言稱此書:「為眾方之宗、群方之祖」。後世奉其為「醫聖」。皇甫謐自幼家貧,學習廢寢忘食,淡於名利而不願任官。他對針灸深入研究,將晉代之前各種經脈理論與針灸方法整理成《針灸甲乙經》,該書成為後世針灸學的範典。他還著有《寒食散論》,魏晉之後服食寒食散逐漸的流行起來。

關於其他技術,天文學方面,有先後擔任孫吳與西晉太史令的陳卓。他收集各派資訊,完善中國星官體制,並繪製星圖,為後世所沿用。裴秀的「制圖六體」在中國地圖史上占有重要的位置。蒲元擅長鍛鍊鐵器,他在斜谷(今陕西省眉县西南)为诸葛亮製刀。其刀能劈开装满铁珠的竹筒,誉为神刀。由於孫吳位於江南地區,水路發達,造船技術發達。其戰船有的上下五層,有的還能容納士兵三千人。蜀漢盛産井鹽,利用當地的天然氣來煮鹽,提昇了產能。


%% -*- coding: utf-8 -*-
%% Time-stamp: <Chen Wang: 2019-10-15 11:08:14>


\section{曹魏\tiny(220-265)}

%% -*- coding: utf-8 -*-
%% Time-stamp: <Chen Wang: 2018-07-10 20:33:47>

\subsection{文帝\tiny(220-226)}

\subsubsection{黄初}

\begin{longtable}{|>{\centering\scriptsize}m{2em}|>{\centering\scriptsize}m{1.3em}|>{\centering}m{8.8em}|}
  % \caption{秦王政}\
  \toprule
  \SimHei \normalsize 年数 & \SimHei \scriptsize 公元 & \SimHei 大事件 \tabularnewline
  % \midrule
  \endfirsthead
  \toprule
  \SimHei \normalsize 年数 & \SimHei \scriptsize 公元 & \SimHei 大事件 \tabularnewline
  \midrule
  \endhead
  \midrule
  元年 & 220 & \tabularnewline\hline
  二年 & 221 & \tabularnewline\hline
  三年 & 222 & \tabularnewline\hline
  四年 & 223 & \tabularnewline\hline
  五年 & 224 & \tabularnewline\hline
  六年 & 225 & \tabularnewline\hline
  七年 & 226 & \tabularnewline
  \bottomrule
\end{longtable}


%%% Local Variables:
%%% mode: latex
%%% TeX-engine: xetex
%%% TeX-master: "../../Main"
%%% End:

%% -*- coding: utf-8 -*-
%% Time-stamp: <Chen Wang: 2018-07-10 20:44:37>

\subsection{明帝\tiny(226-239)}

\subsubsection{太和}

\begin{longtable}{|>{\centering\scriptsize}m{2em}|>{\centering\scriptsize}m{1.3em}|>{\centering}m{8.8em}|}
  % \caption{秦王政}\
  \toprule
  \SimHei \normalsize 年数 & \SimHei \scriptsize 公元 & \SimHei 大事件 \tabularnewline
  % \midrule
  \endfirsthead
  \toprule
  \SimHei \normalsize 年数 & \SimHei \scriptsize 公元 & \SimHei 大事件 \tabularnewline
  \midrule
  \endhead
  \midrule
  元年 & 227 & \tabularnewline\hline
  二年 & 228 & \tabularnewline\hline
  三年 & 229 & \tabularnewline\hline
  四年 & 230 & \tabularnewline\hline
  五年 & 231 & \tabularnewline\hline
  六年 & 232 & \tabularnewline\hline
  七年 & 233 & \tabularnewline
  \bottomrule
\end{longtable}

\subsubsection{青龙}

\begin{longtable}{|>{\centering\scriptsize}m{2em}|>{\centering\scriptsize}m{1.3em}|>{\centering}m{8.8em}|}
  % \caption{秦王政}\
  \toprule
  \SimHei \normalsize 年数 & \SimHei \scriptsize 公元 & \SimHei 大事件 \tabularnewline
  % \midrule
  \endfirsthead
  \toprule
  \SimHei \normalsize 年数 & \SimHei \scriptsize 公元 & \SimHei 大事件 \tabularnewline
  \midrule
  \endhead
  \midrule
  元年 & 233 & \tabularnewline\hline
  二年 & 234 & \tabularnewline\hline
  三年 & 235 & \tabularnewline\hline
  四年 & 236 & \tabularnewline\hline
  五年 & 237 & \tabularnewline
  \bottomrule
\end{longtable}

\subsubsection{景初}

\begin{longtable}{|>{\centering\scriptsize}m{2em}|>{\centering\scriptsize}m{1.3em}|>{\centering}m{8.8em}|}
  % \caption{秦王政}\
  \toprule
  \SimHei \normalsize 年数 & \SimHei \scriptsize 公元 & \SimHei 大事件 \tabularnewline
  % \midrule
  \endfirsthead
  \toprule
  \SimHei \normalsize 年数 & \SimHei \scriptsize 公元 & \SimHei 大事件 \tabularnewline
  \midrule
  \endhead
  \midrule
  元年 & 237 & \tabularnewline\hline
  二年 & 238 & \tabularnewline\hline
  三年 & 239 & \tabularnewline
  \bottomrule
\end{longtable}


%%% Local Variables:
%%% mode: latex
%%% TeX-engine: xetex
%%% TeX-master: "../../Main"
%%% End:

%% -*- coding: utf-8 -*-
%% Time-stamp: <Chen Wang: 2018-07-10 20:48:48>

\subsection{曹芳\tiny(239-254)}

\subsubsection{正始}

\begin{longtable}{|>{\centering\scriptsize}m{2em}|>{\centering\scriptsize}m{1.3em}|>{\centering}m{8.8em}|}
  % \caption{秦王政}\
  \toprule
  \SimHei \normalsize 年数 & \SimHei \scriptsize 公元 & \SimHei 大事件 \tabularnewline
  % \midrule
  \endfirsthead
  \toprule
  \SimHei \normalsize 年数 & \SimHei \scriptsize 公元 & \SimHei 大事件 \tabularnewline
  \midrule
  \endhead
  \midrule
  元年 & 240 & \tabularnewline\hline
  二年 & 241 & \tabularnewline\hline
  三年 & 242 & \tabularnewline\hline
  四年 & 243 & \tabularnewline\hline
  五年 & 244 & \tabularnewline\hline
  六年 & 245 & \tabularnewline\hline
  七年 & 246 & \tabularnewline\hline
  八年 & 247 & \tabularnewline\hline
  九年 & 248 & \tabularnewline\hline
  十年 & 249 & \tabularnewline
  \bottomrule
\end{longtable}

\subsubsection{嘉平}

\begin{longtable}{|>{\centering\scriptsize}m{2em}|>{\centering\scriptsize}m{1.3em}|>{\centering}m{8.8em}|}
  % \caption{秦王政}\
  \toprule
  \SimHei \normalsize 年数 & \SimHei \scriptsize 公元 & \SimHei 大事件 \tabularnewline
  % \midrule
  \endfirsthead
  \toprule
  \SimHei \normalsize 年数 & \SimHei \scriptsize 公元 & \SimHei 大事件 \tabularnewline
  \midrule
  \endhead
  \midrule
  元年 & 249 & \tabularnewline\hline
  二年 & 250 & \tabularnewline\hline
  三年 & 251 & \tabularnewline\hline
  四年 & 252 & \tabularnewline\hline
  五年 & 253 & \tabularnewline\hline
  六年 & 254 & \tabularnewline
  \bottomrule
\end{longtable}


%%% Local Variables:
%%% mode: latex
%%% TeX-engine: xetex
%%% TeX-master: "../../Main"
%%% End:

%% -*- coding: utf-8 -*-
%% Time-stamp: <Chen Wang: 2018-07-10 20:50:14>

\subsection{曹髦\tiny(254-260)}

\subsubsection{正元}

\begin{longtable}{|>{\centering\scriptsize}m{2em}|>{\centering\scriptsize}m{1.3em}|>{\centering}m{8.8em}|}
  % \caption{秦王政}\
  \toprule
  \SimHei \normalsize 年数 & \SimHei \scriptsize 公元 & \SimHei 大事件 \tabularnewline
  % \midrule
  \endfirsthead
  \toprule
  \SimHei \normalsize 年数 & \SimHei \scriptsize 公元 & \SimHei 大事件 \tabularnewline
  \midrule
  \endhead
  \midrule
  元年 & 254 & \tabularnewline\hline
  二年 & 255 & \tabularnewline\hline
  三年 & 256 & \tabularnewline
  \bottomrule
\end{longtable}

\subsubsection{甘露}

\begin{longtable}{|>{\centering\scriptsize}m{2em}|>{\centering\scriptsize}m{1.3em}|>{\centering}m{8.8em}|}
  % \caption{秦王政}\
  \toprule
  \SimHei \normalsize 年数 & \SimHei \scriptsize 公元 & \SimHei 大事件 \tabularnewline
  % \midrule
  \endfirsthead
  \toprule
  \SimHei \normalsize 年数 & \SimHei \scriptsize 公元 & \SimHei 大事件 \tabularnewline
  \midrule
  \endhead
  \midrule
  元年 & 256 & \tabularnewline\hline
  二年 & 257 & \tabularnewline\hline
  三年 & 258 & \tabularnewline\hline
  四年 & 259 & \tabularnewline\hline
  五年 & 260 & \tabularnewline
  \bottomrule
\end{longtable}


%%% Local Variables:
%%% mode: latex
%%% TeX-engine: xetex
%%% TeX-master: "../../Main"
%%% End:

%% -*- coding: utf-8 -*-
%% Time-stamp: <Chen Wang: 2018-07-10 20:51:45>

\subsection{元帝\tiny(260-265)}

\subsubsection{景元}

\begin{longtable}{|>{\centering\scriptsize}m{2em}|>{\centering\scriptsize}m{1.3em}|>{\centering}m{8.8em}|}
  % \caption{秦王政}\
  \toprule
  \SimHei \normalsize 年数 & \SimHei \scriptsize 公元 & \SimHei 大事件 \tabularnewline
  % \midrule
  \endfirsthead
  \toprule
  \SimHei \normalsize 年数 & \SimHei \scriptsize 公元 & \SimHei 大事件 \tabularnewline
  \midrule
  \endhead
  \midrule
  元年 & 260 & \tabularnewline\hline
  二年 & 261 & \tabularnewline\hline
  三年 & 262 & \tabularnewline\hline
  四年 & 263 & \tabularnewline\hline
  五年 & 264 & \tabularnewline
  \bottomrule
\end{longtable}

\subsubsection{咸熙}

\begin{longtable}{|>{\centering\scriptsize}m{2em}|>{\centering\scriptsize}m{1.3em}|>{\centering}m{8.8em}|}
  % \caption{秦王政}\
  \toprule
  \SimHei \normalsize 年数 & \SimHei \scriptsize 公元 & \SimHei 大事件 \tabularnewline
  % \midrule
  \endfirsthead
  \toprule
  \SimHei \normalsize 年数 & \SimHei \scriptsize 公元 & \SimHei 大事件 \tabularnewline
  \midrule
  \endhead
  \midrule
  元年 & 264 & \tabularnewline\hline
  二年 & 265 & \tabularnewline
  \bottomrule
\end{longtable}


%%% Local Variables:
%%% mode: latex
%%% TeX-engine: xetex
%%% TeX-master: "../../Main"
%%% End:


%%% Local Variables:
%%% mode: latex
%%% TeX-engine: xetex
%%% TeX-master: "../../Main"
%%% End:

%% -*- coding: utf-8 -*-
%% Time-stamp: <Chen Wang: 2019-12-17 22:35:45>


\section{蜀汉\tiny(221-263)}

\subsection{简介}

漢(221年-263年,又稱蜀漢)為中国历史上三國時期西南方的一個政權。於221年由昭烈帝稱帝開始,至263年曹魏攻入蜀地,後主投降為終,共經過43年,二帝統治。漢昭烈帝劉備、漢丞相武鄉侯諸葛亮的統治下,政治清明,為大漢復興、北伐奠定基礎。

刘备以延续汉代劉氏皇室政权,称国号为“漢”,有時自稱「季漢」。不過,魏晉政權皆不承認漢政權承繼漢室、國號為「漢」,而因其主要領土古稱蜀地,而稱之為「蜀」,「蜀」遂成為其俗稱。由於漢為曹魏所滅亡,晉又取代魏國,所以《三國志》作者陳壽(漢出身,漢滅亡後仕西晉)為保持政治正確,以「蜀」稱呼其國號,而不使用正式國號「漢」;同為曾出仕漢的李密在《陳情表》中稱呼蜀漢為「偽朝」;資治通鑑稱其為「漢」。後來歷史為将其區別于西漢和東漢,稱劉備政權為「蜀漢」、「季漢」。

東漢末年,群雄割據,漢景帝後裔劉備先佔據荊州,再從攻取劉璋治的益州,又自曹操手中奪取漢中,於219年自封漢中王。220年,孫權進攻荊州,殺死守將關羽,使刘备元气大伤。同年曹丕逼漢獻帝禪讓,篡代東漢,221年劉備於成都稱帝,設立高廟,合祭漢朝皇帝,以汉室宗亲的身份承继漢祚,國號仍为「漢」,稱「東漢」為「中漢」,而刘备称帝后这段时期稱為「季漢」。

同年,劉備以為關羽報仇的名义,發兵討伐孫權,意图夺回荆州,但卻不幸大將張飞在戰爭前被部下張達、范彊殺害,後來更于222年夏被陸遜在夷陵之戰中打敗,最終撤退到白帝城。劉備於223年四月駕崩,諡號為昭烈帝。太子劉禪繼位,由託孤大臣諸葛亮、李严扶助朝政。诸葛亮立即与东吴修好,恢复了联吴抗曹的政策,双方从此再无互相争战。

225年,諸葛亮平定南中多郡的叛乱,并利用降服了南中少数民族部落,削弱李严的势力,解決蜀漢的後方問題。蜀汉此后的三十多年历史中,内外几乎只有对曹魏作战一个焦点,小有出现魏吴两国政变、叛乱等情况。

228年,諸葛亮率領大軍出漢中,開始第一次北伐曹魏,却在街亭战役中失败,并不得不依法处斩对此负有重大责任的參軍馬謖。之后诸葛亮继续北伐,但多次因补给线太长、粮草不济被迫撤军,致使北伐始终无法获得重大成效,其中在建威之戰後進佔原屬曹魏的武都、阴平两个郡。234年,諸葛亮於第五次北伐中病故於五丈原。

諸葛亮死后劉禪开始自摄国政,蜀汉由蔣琬、費禕、董允等接手执掌朝政。大将军蔣琬数次派出姜維等率军对曹魏进攻。246年蔣琬死后費禕掌权,不主张过多军事进攻。同年董允去世,刘禅开始寵信宦官黃皓和寵臣陳祗,令朝政開始變壞。費禕253年遇刺身亡。

大将姜維在247年至262年不斷的北伐,甚至一度年年大規模征戰,严重消耗蜀汉国力,人民也困苦不堪。

姜維讨厌宦官黄皓擅权,曾上书刘禅要求处死黃皓,黄皓得知后也预谋废姜维立阎宇,同时朝中大臣诸葛瞻、董厥等也對姜維多次伐魏但收效甚微感到反感,上书刘禅要求召还姜维为益州刺史,夺其兵权。姜維惟有避居陇西沓中屯田,內外產生嚴重分歧,汉中门户大开。而當時曹魏實質控制者司馬昭決定伐蜀,

263年八月司馬昭派征西將軍鄧艾、中護軍諸葛緒和鎮西將軍鍾會率三路南下,開始魏滅蜀之戰。漢中被破,鍾會軍雖被从沓中赶回来的姜維擋於劍閣,但鄧艾率軍偷襲涪城(今绵阳市),蜀漢江油守將馬邈見魏軍突然出現,投降魏軍。鄧艾繼續進攻,擊敗迎戰的衛將軍諸葛瞻。十一月,劉禪接受譙周意見,帶領文武百官出降,蜀漢正式滅亡,但姜維詐降鍾會,打算利用鍾會野心造成魏军内耗再杀死钟会夺取军权复国,但因事敗,死於亂軍之中。

蜀漢主要佔領益州,共分為二十二郡,擁有一百三十一個縣國,劉備於建安十三年(208年)赤壁之戰後始得荊州的南郡部分及长沙、武陵等荊南4郡。建安十九年(214年)入蜀取得益州,明年與孫權議定平分荊州,219年劉備於漢中之戰打敗曹操得汉中郡。至建安二十四年(219年)時,計此時共轄有2州21郡1屬國。同年孫吳擊取荊州而喪失南郡、武陵郡、零陵郡3郡,诸葛亮北伐曹魏取得武都郡、陰平郡2郡,分割設置數郡。蜀漢滅亡以前,計轄有1州22郡。

蜀漢政律《蜀科》由諸葛亮、法正、劉巴、李嚴、伊籍所編列。後來,劉備逝世,繼位的劉禪幼小,政策多由諸葛亮所主持。在朝內制定八務、七戒、六恐、五懼,訓誡大臣;而朝外風氣清廉,法家思想治蜀地,人心不亂,使蜀中政事、民事都能成功進行。

當時實行安撫百姓,展示法度規範,約制官職,嚴格遵從權制,廣開誠心,公平行事。做到盡忠益的人雖有錯,必定賞賜;犯法怠慢的人雖是親屬之人,但都懲罰。如能順從懲罰,雖重罪仍會得釋;巧言令色的人,雖輕罪仍會受重罰。而且刑政雖嚴峻,但都無人怨恨。

诸葛亮的治国取得了很大的成果,即使是连年对曹魏作战,在诸葛亮在世时,蜀汉的经济仍然得到了较大的发展。袁準评价为“亮之治蜀,田畴辟,仓廪实,器械利,蓄积饶,朝会不华,路无醉人。”正因为诸葛亮清廉而公正,并能使得百姓安居乐业,生活富足稳定,百姓对诸葛亮极为爱戴,陈寿称之为“至今梁、益之民,咨述亮者,言犹在耳,虽甘棠之咏召公,郑人之歌子产,无以远譬也。”

諸葛亮死後,蔣琬、費禕、董允等都繼續諸葛亮的政策;不過後來劉禪寵信寵臣陳祗、宦官黃皓,並開始相信鬼神之說,令朝政漸下;儘管如此,至蜀漢滅亡為止,國家政風仍算清廉,官吏約四萬人。

蜀漢政治中,在劉備、諸葛亮的經營下,人才多能發揮己用,最初起兵的麾下、荊襄舊部、蜀中降將,在劉備、諸葛亮等人的協調下得以事才任用,荊州舊部與劉璋下屬降將;有能者多能位居要職。 如法正成為僅次於諸葛亮的謀士,蜀中舊將的李嚴;劉備稱帝後任尚書令;在劉備逝世後出鎮江州,後更遷為驃騎將軍,其後由於延誤軍機而招致流放,然其子李豐仍受諸葛亮任用,並未因此有所偏廢。又如吳懿,劉備定蜀後任討逆將軍;稱帝後遷為關中都督;建興八年任左將軍,諸葛亮過世後,陞至車騎將軍,並擔任漢中的防務總指揮,其族弟吳班,官位常僅次於吳懿,後主世,任驃騎將軍。黃權於劉備領益州牧時任治中從事。又如董和、李恢等亦受劉備、諸葛亮重用。諸葛亮治事期間,不僅提拔荊州舊部的蒋琬、魏延、费祎、楊儀等,亦大加重用蜀中人才;文如董和、董允父子,武如王平、張嶷、張翼、馬忠等如是,縱如降將的姜维,亦受到諸葛亮的重用,其後更成為蜀漢後期北伐的重要將領。

其人才可謂兼容並蓄,不論荊襄舊部、起兵之初的麾下、蜀中舊將、魏國降將等,皆能依照其能力擔任相應的職務。蜀漢後期亦有不少人才,武將如霍峻之子霍弋、句扶、柳隱、羅憲等,皆為一時之選。

蜀漢將士,在劉備在位最盛時期(未失荊州時)約有十六萬至二十萬之間。至蜀亡國,仍有將士十萬二千人。

蜀漢的對外戰爭多向曹魏發動,前期最著名是諸葛亮北伐與後期最著名姜維北伐。

然而姜維的北伐卻間接導致蜀漢壓力加劇,國力日漸下滑,人心厭戰,以致蜀國重要據點缺乏士兵固守。

蜀漢內部政局較東吳及曹魏平和,除夷陵之戰及諸葛亮南征之外,鮮有內亂或向東吳方面發起攻擊。

214年,劉備入蜀後,巴蜀地區財政混亂,劉巴提出鑄直百錢,平衡物價,解決問題。當中五銖錢與直百錢並用,錢面有鑄字「直百五銖」、背有「好右有為」,為犍為郡所鑄,從中知道蜀鑄錢不只在一地。

而蜀漢的收入有田租,但暫未有實例;鹽鐵對蜀漢有不錯的利潤;而南中金、銀、丹、漆、耕牛、戰馬、蜀錦等貢品,令蜀漢軍費有所供給,國家富裕,為諸葛亮北伐提供物資;另有其他收入沒有記錄。另一方面,支出包官俸、軍糧、賞賜等,至蜀漢亡時,官府仍有金、銀各二千斤。

劉備入蜀後,藉著戰國時代李冰所開的都江堰所提供的充足水利,不斷增加耕地的灌溉面積;後至諸葛亮北伐時,為了防護都江堰而任命的督堰官人數已有一千二百人之多,足見其灌溉農業規模之大。諸葛亮在保山市法寶山下設有諸葛堰。在漢中增築山河堰,成都重築九里堤。蜀漢由於地形上山地較多,除了較主要的灌溉農地以外,幾乎沒有興修水利、屯田的必要,遠不及魏、吳,雖然有督農的官職,但只設在與魏國接壤的前線漢中郡。蜀中雖不缺糧,但受限於地理限制,在與魏國作戰時補給線往往較魏國長;諸葛亮第五次北伐時,曾在魏地屯田,只為解決運糧問題。而至姜維時,他亦在沓中種麥,但主要作用是避開黃皓。至蜀漢亡時,官府仍有四十多萬斛米糧。

三國中,蜀漢在絲織業最為興盛,以蜀郡為盛產地,稱為「蜀錦」。蜀漢設立專門的錦官製造蜀錦,甚至諸葛亮北伐的經濟來源都嚴重依賴蜀錦,此外蜀錦亦多用以外交禮物及賞賜。漢亡之時,官府藏有錦、綺、彩、絹,各二十萬匹,在當時是十分驚人的數目。井鹽也是蜀地特產,為蜀漢其中一項重要工業。另有開發天然氣。三國時期的商業不太發達,一是由於生產量減少,人民多以物易物;二是由於金屬貨幣不流通;三則是由於割據的局面,商人不能遠行。當中,魏、漢因對立而沒有發生貿易;與東吳則貿易頗多;蜀漢因地處西南,所以鮮有對外域進行貿易。

蜀漢全盛時期擁有三十多萬戶(未失荊州時),人口約一百萬,為三國中最少。 汉昭烈帝章武元年(221年),在籍户口分别为二十万户与九十万人,经诸葛亮治蜀,至蜀亡时(263年)共有1082000人 ,其中户数二十八万, 民数九十四万, 带甲将士十万二千, 官吏四万, 當中蜀郡擁有戶口最多。

因地處著名產茶的西南地區,所以飲茶的風氣甚盛,足以代酒,因此飲酒之風不及魏、吳。民間亦有拜祭鬼神,諸葛亮死後,劉禪初期不與立廟,百姓仍在路上祭祀。朝廷中,蜀漢後期,劉禪聽信黃皓、陳祗巫鬼之說,最後間接令蜀漢被曹魏入侵。

%% -*- coding: utf-8 -*-
%% Time-stamp: <Chen Wang: 2021-11-01 11:36:07>

\subsection{昭烈帝劉備\tiny(221-223)}

\subsubsection{生平}

漢昭烈帝劉備(161年7月16日-223年6月10日),字玄德,涿郡涿縣(今河北省涿州市)人,祖籍徐州沛縣(今徐州市沛縣),亦稱漢先主,三國時代蜀漢第一位皇帝,諡號昭烈皇帝,三國志、華陽國志等稱為先主 ,繼其帝位的劉禪則被稱為「後主」,資治通鑑稱劉備父子為漢主。

劉備雖為漢景帝後代,但世系久遠,實由布衣起步而終得一方天下。

劉備是汉景帝第九子中山靖王劉勝之子刘贞的後代,而裴松之三国志注所引《典略》记载,刘备为“临邑侯枝属”。祖父名雄,父親名弘,世代皆仕於州郡,祖父劉雄曾被推舉為孝廉,官至東郡範令。世居酈亭樓桑里。

劉弘在劉備少時已逝,劉備便與母親販賣草鞋、織草蓆為業。家裡房舍的東南角的圍籬上有種植桑樹,高五丈餘,從遠處觀看像是一臺當時小車的車頂,路過的人皆訝異此樹的非凡,或說此家必當出貴人。劉備小時候與家族中年齡相近的小孩在樹下遊戲時,曾說:「吾必當乘此羽葆蓋車。」他的叔父劉子敬說:「汝勿妄語,滅吾門也!」劉備15歲時,刘备母亲要他外出求學,與同宗刘德然、遼西公孫瓚同入大儒盧植門下求學。

劉德然之父劉元起常資助劉備,所給錢物與自己兒子劉德然等同。劉元起妻罵:「各自一家,何能常爾邪!」元起答:「吾宗中有此兒,非常人也。」公孫瓚與劉備結為好友,公孫瓚較年長,劉備以兄事之。

劉備不甚樂讀書,喜歡評馬論犬、音樂、華美的衣服。身長七尺五寸(約173公分,漢時一尺約為23.1公分),垂手下膝,有一對招風大耳,不需攬鏡自照,眼可自見其耳。少說話,善於待人,喜怒不形於色。好交結豪俠義士,年輕人爭相趨附他。中山大商人張世平、蘇雙等多給與金錢資助,劉備由是得用以糾合组织部下。由於個性與行事風格酷似先祖劉邦,而被評為有高祖之風。

184年(23歲),黃巾之亂爆發,各州郡皆有人民組織義軍討伐。劉備率領耿雍、關羽、張飛、牵招及一干下屬跟隨鄒靖討伐黃巾軍,立下戰功,被任為安喜尉。後來,漢室有令:如因軍功而成為長吏的人,都要被選精汰穢,督郵到安喜要遣散劉備,劉備知道消息後,到督郵入住的驛站休息房舍求見,督郵聲稱有病不肯相見,劉備因此感到不悅,便徑直闖入房舍,將督郵綑綁,杖打二百下,然後棄官逃亡。後來,大將軍何進派都尉毌丘毅到丹楊募兵,劉備也在途中加入,到下邳時與盜賊力戰立功,任為下密縣丞,不久又辭官。

191年(30歲),刘备時任高唐令,但被盗贼击败而投奔公孫瓚,公孫瓚隨即上表,保奏劉備為別部司馬,任為平原令、平原相。劉備平原外禦賊寇,在內則屯糧分發給百姓,士以下的人,都可與他同席而坐,同簋而食,不會有所揀擇。據說郡民劉平不服從劉備的治理,唆使刺客前去暗殺。劉備毫不知情,還對刺客十分禮遇,刺客深受感動,不忍心殺害劉備,便坦露實情離去。劉備治理平原郡深得人心、相當成功。

黄巾餘黨管亥率眾軍攻打北海郡,北海相孔融被大軍所圍,情勢危急,便派太史慈突圍向劉備求救。太史慈對劉備說:「慈,東萊之鄙人也,孔北海親非骨肉,比非鄉黨,特以名志相好,有分災共患之義。今管亥暴亂,北海被圍,孤窮無援,危在旦夕。以君有仁義之名,能救人之急。故北海區區,延頸恃仰,使慈冒白刃,突重圍,從万死之中自托于君,惟君所以存之。(我太史慈只是東萊一個無名之人。北海相孔融和我並不是有著骨肉相連的親族,也稱不上是志同道合的同鄉朋友,只是他認為我有前途而看重我,所以我有為他分擔災禍、共赴患難之義理。現在管亥起兵擾境,包圍北海城,城內居民徬徨無助,危在旦夕。孔融大人聽說劉備大人有仁義之名,能救人之危難急迫。因此盼望著能得到您的幫助,命令我突破管亥兵眾的包圍,冒著萬死無生的可能,來向劉備大人求助,惟有借重您的力量能使北海城脫危。)」劉備驚訝地答道:「孔北海知世間有劉備邪!(北海相孔融居然知道世間有我劉備啊!)」便立即派三千精兵隨太史慈去北海救援。黄巾军聞知援軍至,都四散而逃,孔融逐得以解圍。後袁紹攻公孫瓚,劉備與田楷東屯齊。

193年(32岁),曹操征討徐州,徐州牧陶謙敗退,曹操在徐州大屠殺。陶謙遺使告急於田楷,田楷與劉備俱前往相救。當時劉備自有士兵千餘人及幽州烏桓攙雜胡族騎兵,又略得饑民數千人。既到,與陶謙將領曹豹屯在郯東,被曹操擊敗。後曹操因後方生事而撤退,陶謙以丹楊兵四千人給劉備,劉備遂離開田楷,依附陶謙。陶謙表劉備為豫州刺史,屯兵於小沛。

194年(33岁),陶謙病重,對別駕從事麋竺說:「非劉備不能安此州也。」陶謙死後,麋竺便率徐州人民迎劉備入主徐州,劉備未敢當。下邳陳登對劉備說:「今漢室陵遲,海內傾覆,立功立事,在於今日。彼州殷富,戶口百萬,欲屈使君撫臨州事。(現今漢室漸趨衰敗,海內傾覆,立功名、立事業,就在於今日。本州殷實富足,戶口百萬,希望屈就使君親臨撫牧本州事務。)」劉備說:「袁公路近在壽春,此君四世五公,海內所歸,君可以州與之。(袁公路就近在壽春,此人為四世代有五人為三公,海內民心所歸,你可以徐州給與他。)」陳登答:「公路驕豪,非治亂之主。今欲為使君合步騎十萬,上可以匡主濟民,成五霸之業,下可以割地守境,書功於竹帛。若使君不見聽許,登亦未敢聽使君也。(袁術驕縱橫豪,不是治理亂局之主。現在希望您使君合共步兵騎兵十萬,對上可以匡扶主上、救濟人民,成就像春秋五霸之功業;對下可以割地自守、保境安民,寫下功業於竹帛上。若不見聽使君答許,在下亦未敢聽從使君。)」北海相孔融對劉備說:「袁公路豈憂國忘家者邪?冢中枯骨,何足介意。今日之事,百姓與能,天與不取,悔不可追。(袁公路豈是因憂慮國事而忘卻家庭之人?墓中之枯骨,不足以在意。今日之事情,是百姓讓與賢能,天意讓與你而不取,後悔不可追。)」劉備遂領徐州牧。

195年(34岁),吕布被曹操打敗來投靠,劉備善待禮遇他。吕布見劉備,極為尊敬,說:「我與卿同邊地人也。布見關東起兵,欲誅董卓。布殺卓東出,關東諸將無安布者,皆欲殺布爾。(我與你同為邊地出身的人(呂布出身五原郡,劉備出身涿郡,皆屬漢朝疆界北方邊境之地)。我見關東諸侯起兵,想要誅殺董卓。後來我殺董卓向東走,關東諸將卻沒有一個安置我,更加要殺死我啊。)」請劉備於帳中坐,並令妻子行禮,酌酒飲宴,又稱呼劉備為其弟。劉備見吕布胡言亂語,表面上雖不當一回事而心裏卻對其有所戒備。最後劉備仍讓吕布屯於小沛。

建安元年(196年,35岁),袁术來攻徐州,劉備於盱眙、淮陰抵抗袁軍。曹操上表朝廷,劉備成為鎮東將軍,封為宜城亭侯。劉備與袁术相持經一個月,大戰互有勝負。吕布乘下邳之虛,趁機偷襲。下邳守將曹豹倒戈,迎接呂布,趕走張飛,佔據下邳。吕布擄獲劉備妻子,劉備轉戰海西。東漢建安二年(197年)夏天,楊奉、韓暹等賊軍在徐、揚二州之間作惡,劉備與其決戰,盡為劉備所斬首。後來劉備向吕布求和,吕布歸還其妻子。劉備遺派關羽守下邳。

劉備還軍小沛,恢復集合兵馬得萬餘人。吕布嫌惡於此,自行出兵攻打劉備,劉備兵敗走投歸順曹操。曹操厚待禮遇劉備,以其為豫州牧。劉備與曹操一同返回許都後,被任命為左將軍。劉備來投奔,曹操謀士程昱就曾警告「觀劉備有雄才而甚得眾心,終不為人下」,勸曹操趁早解決後患,但曹操認為當時是收英雄之時,不可失天下之心。

(198年37岁)春天,吕布派人攜金到河內買馬,但被劉備兵所掠取。吕布於是派高順、張遼等攻劉備,雖然曹操曾派夏侯惇前往解救,但仍敗陣,劉備妻子又被吕布所擄。十月,曹操親自東征吕布,劉備在梁國界中與曹操相遇,便合兵成功消滅吕布。劉備復得妻子,跟從曹操還師許都。曹操表劉備為左將軍,禮之愈重,出則同車,坐則同席。

漢獻帝因曹操挾天子以令諸侯,發出衣带詔令其岳父董承誅殺曹操,劉備尚未加入。一日,曹操宴請劉備,對劉備說:「今天下英雄,唯使君與操耳。本初之徒,不足數也。(當今天下英雄唯獨是你與我,袁紹這類人稱不上)」劉備聽心中一震,筷子從手中掉落。此時剛好打雷,劉備便對曹操說:「『聖人迅雷風烈必變』,良有以也。一震之威,乃可至於此也!(『即使是圣人遇见打雷也会改变表情』,確有原因。一聲雷鳴,乃可以令我變成如此!)」《華陽國志》記載當時碰巧雷聲大作,劉備便把自己的失態歸咎於雷鳴,而此事後,劉備便加入董承。不久,在南方失利的袁术想北投袁紹,劉備便向曹操借兵出擊袁术,趁机摆脱曹操的控制。曹操便派他督朱靈、路招攻擊袁术,但軍未到,袁术已病死。

199年(38歲),劉備遣朱靈、路招佔據下邳。200年(39歲),反曹事迹敗露,董承被殺。劉備便殺死徐州刺史車冑,留關羽守下邳,自己回守小沛,另一方面派遣孫乾與袁紹連合,打出對抗曹操的名目。曹操曾派劉岱、王忠領軍攻打劉備,但不克。同時,東海昌霸反叛,郡縣多投靠劉備,劉備軍再次聚起數萬人,並連同多個地方勢力一起反曹。曹操決定親自東征劉備,雖然曹軍中將領多認為袁紹才是大敵,但曹操卻覺得劉備是英傑,必要先行討伐,郭嘉亦贊同曹操。

最後劉備大敗,小沛被佔,曹操虜獲劉備妻子及生擒關羽、夏侯博。劉備逃至青州,青州刺史袁譚親自迎接,並報知其父袁紹,袁紹出鄴城200里迎接。刘备泄露曹操曾经对自己说的密言予袁绍,袁绍才知道曹操原来有针对自己的阴谋。刘备待了一個多月後,以前的部下又重新聚會。不久,曹操與袁绍於官渡交戰,汝南郡黃巾餘軍劉辟等响應袁绍叛曹,袁绍便派劉備率軍與劉辟會合。曹操派曹仁攻打汝南,劉備惟有再次還軍袁绍。當時劉備想離開袁绍,便說服袁绍應南連劉表,袁绍再次派劉備到汝南與龔都會合。曹操另派蔡陽攻擊劉備,為劉備所殺。曹操於官渡之戰大敗袁绍。

建安六年(201年40歲),曹操又出兵南擊劉備,劉備便乘機放棄汝南,入荊州投靠劉表。劉備並派麋竺、孫乾與劉表會面。劉備到達荊州,受到劉表熱情接待。劉表接納劉備後,便為他增加兵馬,屯兵於新野,守衛荊州北大門。建安七年(202年42歲),曹操與袁尚、袁譚大戰於黎陽,許昌空虛,奉劉表命令北伐曹操。曹將夏侯惇、于禁、李典等人率軍南下,劉備奉命北上迎敵。在新野北博望,劉備設好伏兵以後,便燒毁營屯假裝懼敵退卻。夏侯惇讓李典留守,自己和于禁追擊,追到博望,劉備伏兵將夏侯惇殺得大敗,曹軍損失慘重,向北退走。劉備在荊州聲望日高,引起劉表疑心劉備,處處戒備。

建安十二年(207年46歲),曹操基本統一黃河流域之後,開始北上征伐北方烏丸,刘备力勸刘表乘機袭取许都,刘表没有採纳劉備建議。

劉備在荊州幾年,知道水鏡先生就是司馬徽,便前去請教世事。司馬徽知道劉備來意,便對他說:「儒生俗士,豈識時務?識時務者為俊傑。此間自有卧龍、鳳雛。(一個儒生見識淺俗之士,豈會認識時勢事務?認識時勢事務者,是那些英俊豪傑。從此地中,有臥龍(诸葛亮)、鳳雛(龐統)。)」亮又受徐庶推薦,劉備希望徐庶引亮來見,但徐庶卻建議:「此人可就見,不可屈致也。將軍宜枉駕顧之。(此人只能前去拜謁,不可委屈他前來。將軍宜枉屈尊駕以顧望。)」

207年(46岁),劉備三顧茅廬,問計於諸葛亮:「漢室傾頹,奸臣竊命,主上蒙塵。孤不度德量力,欲信大義於天下,而智術淺短,遂用猖獗,至於今日。然志猶未已,君謂計將安出?(漢室衰敗,奸臣掌權,使天子(漢獻帝)蒙受苦難。我不自量德行與能力,欲伸張大義於天下,然而智術淺薄,時至今日,一無所成。然則志向仍未減,先生可以出謀畫策嗎?)」諸葛亮遂向他陳述三分天下之計,分析此時曹操挾天子而令諸侯,此誠不可與爭鋒;孫權據有江東,可以爲援而不可圖;又詳述荊州用武之國、戰略要地,而其主劉表不能守,此恐怕是上天賜予劉備;益州是漢高祖成就帝業之地,其主劉璋闇弱;更建議劉備等待時局有變,由荊州、益州進攻中原。這篇論說後世稱為《隆中對》,是此後數十年劉備和蜀漢基本國策。。諸葛亮剛從隆中出來,受到劉備重視,只是由於劉備與自己情好日密,就引得「關羽、張飛等不悅」,最後還是劉備出來說:「孤之有孔明,猶魚之有水也。願諸君勿復言。(我有孔明,猶如魚得到水。但願諸君勿再說。)」;關羽、張飛才作罷。劉備在荊州擴軍,諸葛亮籌措軍需,何宇度《益部談資》記載:「先主寓荊州。從南陽大姓晁氏貸錢千𦻼,以為軍需。諸葛孔明作保,券至宋猶存。」

208年(47歲),曹操南下,時劉備屯於樊城。八月劉表病卒,次子劉琮繼任荊州牧,遣使曹操舉州投降。起初劉備不知劉琮決定投降,得知時曹軍尚在宛縣,尚未到達新野,劉備連忙棄城南撤。。在南渡漢水至襄陽時,諸葛亮曾勸劉備攻劉琮奪襄陽,但劉備不忍心進攻劉表之子,沒有攻打襄陽,只是在城下駐馬高呼劉琮出來相見,只來到劉表墓前祭奠,涕泣拜辭而去。劉備一行南下,荊州官吏百姓加入,走到當陽時,人數達10餘萬,輜重數千輛,一日只能走10幾里。惟有另派關羽乘數百艘船,直到江陵。有人向劉備進言:「宜速行保江陵,今雖擁大眾,披甲者少,若曹公兵至,何以拒之?(適宜速行而保江陵,現今雖然擁有很多隨行者,但士兵很少,若曹操軍追至,如何抵抗?)」劉備答道:「夫濟大事必以人為本,今人歸吾,吾何忍棄去!(做大事必以人為本,現今人眾歸附於我,我又如何忍心離棄而去!)」

當時江陵貯有劉表的大量糧儲、器械等軍實,曹操深怕劉備先佔領江陵,就拋棄輜重,以輕軍急行到襄陽。曹操聽聞劉備軍已離開襄陽,與曹純等領五千精騎急追,一日一夜疾行三百餘里。曹軍五千輕騎奔至當陽長坂坡追上劉備一行,劉備棄妻子,與諸葛亮、張飛、趙雲等數十騎走,10餘萬眾土崩瓦解,曹軍大舉擒獲劉備人眾輜重,張飛率20騎拒後,與曹兵邊打邊退。孫權之前派出魯肅來打探消息,在當陽長坂迎堵劉備。長坂會面後,魯肅隨劉備向東南斜趨漢津,在此適逢與關羽水軍會合,渡過沔水後向江夏進發。江夏太守劉琦聞劉備軍到來,率軍前去迎接,將劉備迎到夏口。此後,魯肅返回江東覆命,劉備進至樊口,同時派諸葛亮隨魯肅出使孫權,與孫權結盟。

孫權正式任命周瑜為左都督,程普為右都督,魯肅為贊軍校尉,率三萬水軍,與諸葛亮一起溯江西上,與樊口劉備軍會合。建安十三年冬,曹操親率20餘萬大軍從江陵順江東下,討伐孙权。黃蓋便向周瑜建議說:「今寇眾我寡,難與持久。然觀操軍船艦首尾相接,可燒而走也。」十二月,孫劉聯軍在赤壁至烏林一線以火攻大破曹軍,更追至南郡,曹操敗北。曹操一到江陵,便部署征南將軍曹仁、橫野將軍徐晃守江陵,折衝將軍樂進守襄陽,曹操撤回北方。

赤壁之戰後,劉備撤出江陵戰鬥,全力占據荊州江南四郡,先上表漢帝奏請劉琦為荊州刺史,兩萬大軍南下,武陵太守金旋獻城、長沙太守韓玄迎降、桂陽太守趙範讓位、零陵太守劉度稽顙。廬江人雷緒也率部曲數萬人投效。建安十三年(208年48歲)十二月,荊州江南四郡盡為劉備所占領。劉琦死,群下推劉備為荊州牧,劉備即遣諸葛亮為軍師中郎將,督令零陵、桂陽、長沙三郡,收其租賦,以供軍實,又以關羽為襄陽太守、蕩寇將軍駐江北,張飛為宜都太守、征虜將軍在南郡,趙雲為偏將軍領桂陽太守。廖立為長沙太守,郝普為零陵太守,向朗督秭歸、夷道、巫縣、夷陵四縣軍民事。劉備治於公安。而孫權為與劉備建立更鞏固的關係,在周瑜死後便依魯肅之策將南郡、江陵借給劉備,再分部份長沙郡給他,以及確認劉備佔有武陵和桂陽兩郡,遂提出將其妹嫁予劉備,史稱孫夫人。劉備到京口見孫權,關係表現親密、寬度。時劉備擁有荊州大部份屬地,又收取荊襄名士龐統和馬良,整日操練人馬,伺機南征北伐。

以後,孫權曾派使希望與劉備一起取益州,劉備本想答應,因東吳不可能越荆州而有蜀,蜀地就可據為己有。但荊州主簿殷觀卻反對:「若為吳先驅,進未能克蜀,退為吳所乘,即事去矣。今但可然贊其伐蜀,而自說新據諸郡,未可興動,吳必不敢越我而獨取蜀。如此進退之計,可以收吳、蜀之利。(若我們為吳開路,前進未必能攻克蜀地,後退可能為吳乘虛而入,那時即大勢而去。現今但可以贊同他伐蜀,而自己推卻說剛佔據荊南諸郡,未能興兵妄動,吳必定不敢越過我境而單獨取蜀。依照此進退得宜之計謀,便可以收吳、蜀兩地之利。)」劉備依從其計,孫權果然終輟計劃。殷觀遂升遷為別駕從事。

建安十六年(211年51歲)三月,曹操下令鍾繇率軍西征漢中張魯,讓夏侯淵出河東與鍾繇相會。益州牧劉璋遙聞曹操將遺鍾繇等向漢中討張魯,內心懷有恐懼。別駕從事蜀郡張松說服劉璋稱:「曹操兵強,無敵於天下,若因張魯之資源用以攻取益州土地,誰能抵禦?」劉璋說:「我固然擔憂,而未有計。」張松說:「劉備,使君之宗室,而且是曹操之深仇,善於用兵,若使之討伐張魯,張魯必可攻破。張魯攻破,則益州強大,曹操雖來,也無能為力。」在張松出言下,益州牧劉璋採納請劉備入蜀之意見,並派軍議校尉法正為使,孟達為副,各領兵2,000人,前往荊州邀請劉備入蜀助攻張魯。劉備親自統帥進軍益州,龐統任軍師中郎將,將領黃忠、魏延、卓膺等輔助劉備。劉備與龐統一同進入益州。諸葛亮、關羽、張飛、趙雲、劉封、孟達、馬良等留在荊州。然而劉備要知道蜀中的闊狹,兵器、府庫、人馬多少及多個要害之地的遠近,便向二人請教,張松、法正都一一詳述,更畫出地圖指示山川所在,所以劉備知道益州內情。

到達涪城,劉璋親自出迎,相見甚歡。張松、法正及龐統都提議劉備可乘機殺了劉璋,當時龐統主張趁此機會,擒住劉璋。劉備以初來到蜀地,人心尚未信服,不宜輕舉妄動為由,未採納龐統建言。劉璋推薦劉備行大司馬,領司隸校尉,劉備又推薦劉璋行鎮西大將軍,領益州牧。劉璋配給劉備士兵,及督白水軍,令他攻擊張魯。劉備當時總計有三萬多人,車甲、器械、資貨甚多。但劉備卻到葭萌時,未出兵,而是樹立恩德,收買民心。

建安十七年(212年51歲)冬十月,曹操出兵攻打孫權,孫權向劉備告急,劉備對劉璋說欲還救荊州有急。劉備請求劉璋撥出兵士萬人與軍事物資。但劉璋只允諾給予四千兵馬,其餘物資僅提供一半。劉備受此激怒,忿忿說道:「我為了益州征討強敵,軍隊勤瘁,無暇休息;現今劉璋積存起財富而不用於賞功,卻希望士大夫能為他出力死戰,這又怎可能!」當時張松不知劉備用意,寫信質問:「眼看就要大事底定,為何拋下一切離去?」結果被其兄張肅據此告密,張松遭到處死,導致劉備與劉璋關係惡化。十二月,劉備與劉璋決裂。劉備依龐統提出的計謀,召白水关守将杨怀、高沛到來並將其斬殺。另外又派黃忠、卓膺率軍攻劉璋,一路佔領至涪城。劉璋連忙派出劉璝、冷苞、張任、鄧賢、中郎将吴懿等與對抗劉備,皆破败,退保绵竹,吴懿至刘备军前投降,拜为讨逆将军。刘璋后遣护军李嚴、参军费观督绵竹军拒刘备,两人陣前倒戈亦率众投降,同拜裨將軍,劉備軍勢強,分軍平定各郡縣。但劉備軍卻被雒城守將劉循阻擋攻勢。從建安十八年建安十九年,劉備圍攻雒城將近一年,龐統被流矢射中,重創身亡。張飛、趙雲、劉封等隨諸葛亮率軍入蜀,關羽留下鎮守荊州,馬良、麋芳、士仁、廖化協助關羽鎮守荆州。建安十九年(214年54歲)夏,諸葛亮入蜀援軍溯江而上。諸葛亮分兵進攻成都:張飛從墊江北上直取巴西郡治閬中,從北面攻成都;趙雲從長江西攻取江陽北上犍為郡治武陽,從南面攻成都;諸葛亮親自沿涪江取德陽,直取成都。

214年夏天(53岁),雒城終被攻破。李恢受劉備派遣到漢中與馬超交好,馬超正想離開張魯,劉備暗暗派出人馬與馬超兵眾會合,馬超率領大隊人馬開到成都城北屯駐。關羽聽說馬超歸降備,便寫信給諸葛亮,問馬超才能可與誰相比,諸葛亮回信說:「馬超文武兼備,氣概雄烈,過於常人,可稱得上一世之豪傑,是黥布、彭越一流之人物,可以與張飛相提並論,但是趕不上美髯公你超逸絕群。」劉備乘勢率漢軍進圍成都數十日。劉備派簡雍進入成都勸說劉璋投降,劉璋與簡雍「同輿而載,出城歸命」;劉璋向劉備繳械投降,益州易主,歸屬劉備。由於蜀中繁盛、安樂,劉備便設宴大慰勞士卒,又取蜀城中的金銀,分賜將士,還其谷帛。劉備皆處之顯任,盡其器能,有志之士,無不競勸,益州之民,是以大和。有議論勸劉備將成都城中房舍及城外園地桑田分賜給諸將,但趙雲反駁說:「從前漢朝大將霍去病曾說匈奴未滅,無用家為,何況現在國賊不只像匈奴只有一個,還不到可以安定下來的時候,必須等到天下的亂賊都平定之後,才可讓眾人返回家鄉去種植桑梓,回歸故土去耕作田地,這樣才是正道。益州的人民是第一次遭遇到戰爭,應該將田宅房產歸還給百姓,先讓他們安居樂業,然後才能叫他們服兵役,納錢糧,也才能得到益州的民心。」劉備便聽從趙雲的建議,有志之士便都紛紛來投。

在建安二十年(215年54歲)三月,曹操征伐漢中,七月破南鄭,十一月最終降服張魯,搶在劉備之前占有漢中。孫權以劉備已得益州,派人討還荊州,劉備答道:「須得涼州,當以荊州相與。」孫權忿恨,乃派遣呂蒙施襲,爭奪長沙、零陵、桂陽三郡。劉備率兵五萬到公安,下令關羽進軍益陽,與孫軍對峙。時曹操勢破張魯,威脅蜀地。劉備遂派使者向孫權議和,孫權派諸葛瑾答覆劉備,雙方和好。為盡快解決荊州問題,回兵保衛益州,劉備以湘水為界,將江夏、長沙、桂陽三郡劃給孫吳。南郡、零陵、武陵以西屬劉備所有,劉回軍江州。八月,孫在東線進攻合肥,曹將張遼、李典據城抵抗,擊退孫權。十一月,張魯逃遁至巴西,偏將軍黃權對劉備說:「若失漢中,則三巴不振,此為割蜀之股臂也。」又遣黃權率兵迎向張魯,但張魯已降曹操。曹操派夏侯淵、張郃屯兵漢中,數次武力侵犯巴郡邊界。劉備令張飛進兵宕渠,與張郃等於瓦口爭戰,大敗張郃等。張郃收兵還退南鄭。翌年二月,曹操留夏侯淵、張郃鎮守漢中,自己回鄴城。

建安二十三年(218年57歲),劉備採法正勸諫率軍進攻漢中。諸葛亮鎮守成都,劉備親率大軍征漢中,法正隨從參謀軍機,趙雲、黃忠、魏延、張飛、馬超、吳蘭等從征,曹操、劉備爭奪漢中之戰開始。但漢軍先頭部隊卻被曹軍打敗。劉備一路直攻漢中,進兵至陽平關與夏侯淵、張郃等曹軍對峙,為保證道路暢通,劉備派大將陳式率10餘營兵士駐紮在馬嗚閣道,曹將夏侯淵派大將徐晃襲擊陳兵,陳式軍被打敗,士兵紛紛跳入山谷,傷亡慘重。曹操下令賜徐晃節杖,並說:「此閣道,漢中之險要喉也。劉備欲斷絕外內,以取漢中。將軍一舉,克奪賊計,善之善者也。」劉備「急書發益州兵」,諸葛亮與從事楊洪商議對策,楊洪說:「漢中則益州咽喉,存亡之機會,若無漢中則無蜀矣,此家門之禍也。方今之事,男子當戰,女子當運,發兵何疑!」;諸葛亮非常看重楊洪見識,當即發兵支援漢中前線。從建安二十二年(217年)劉備出兵起,雙方在漢中僵持一年多,建安二十四年(219年)春劉備聽從法正計策,從陽平南渡沔水,依定軍山恃險安營。夏侯淵帶少數兵力爭奪定軍山營地,法正對劉備說:「可擊矣!」;劉備便命黃忠乘高鼓噪攻之,居高臨下,衝入敵陣,殺死夏侯淵。黃忠斬殺夏侯淵及曹操所置的益州刺史趙顒等。建安二十四年三月(219年58歲),曹操自長安率兵經褒斜谷趕往漢中,劉備說:「曹公雖來,無能為也,我必有漢川矣。」劉備在險處死守,不與曹軍交戰。諸葛亮親坐益州,將人力、物力及時補充到劉備軍中。夏五月,曹操引兵撤出漢中,漢中歸劉備所有。而另一方面,又遣劉封、孟達、李嚴等進攻上庸,上庸守將申耽等見曹操率軍返回中原,逐開城投降。秋七月,馬超、龐羲、射援、諸葛亮、關羽、張飛、黃忠、法正、李嚴等120人聯名上表劉備為漢中王。劉備於沔陽設置祭壇場地,陳兵列眾,群臣陪位,宣讀奏訖,自立漢中王。後還治成都。提拔魏延為都督漢中太守,坐鎮漢中。劉備於是建起館舍,修築亭障,從成都至白水關,四百餘區。關羽率軍從江陵北上,發動襄樊戰役。于禁七軍火速增援曹仁,關羽與于禁交鋒,時至八月,大雨滂沱,山洪暴發,漢水驟漲,水淹七軍,于禁束手就擒,部下幾乎全部投降,副將龐德被活捉不降,最後被關羽所殺。孫權將呂蒙白衣渡江。十月,呂蒙任征荊州大督,率兵西上,公安士仁、江陵麋芳開城投降。關羽回軍江陵途中,陸遜任右護軍、鎮西將軍屯駐夷陵,呂蒙任南郡太守駐江陵。關羽至當陽西保麥城,敗走麥城後,士兵繼續逃散,關羽身邊只剩十餘騎。十二月關羽被孫權大將潘璋部馬忠捕殺,孫權將其首級送至洛陽曹操處。孫劉聯盟正式決裂。

建安二十五年(220年59歲)正月,曹操逝世,劉備也曾派遣韓冉奉書弔唁,「並致賻贈之禮」,但最後卻失敗。三月改元延康,十月曹丕代漢稱帝。十二月,當時有謠言指漢獻帝劉協已被加害,劉備便穿喪服發喪,諡劉協為孝愍皇帝(但實際上劉協仍在世)。同年,法正、黃忠去世。

221年,群臣勸劉備登基為帝,劉備不答應,諸葛亮用耿純遊說劉秀登基故事勸劉備(光武帝劉秀登基時,同為漢室的更始帝劉玄仍在世,此後綠林軍攻破長安殺劉玄,此後劉秀建東漢),劉備才決定接受群臣擁立,四月初六在成都武擔山之南接受皇帝璽綬,改元章武。諸葛亮、許靖、黃權等人上書勸劉備即帝位,國號仍為「漢」,史稱蜀漢。四月丙午日(5月15日),大赦天下,改元章武。以諸葛亮為丞相,許靖為司徒。設置百官,建立宗廟,祭祀漢高祖以下。五月,立皇后吳氏,劉禪為皇太子。六月,以劉永為魯王,劉理為梁王。

魏文帝曹丕召集眾臣討論,侍中劉曄認為蜀漢一定要出兵攻打孫吳,理由是:「蜀雖狹弱,而備之謀欲以威武自強,勢必用眾以示其有餘。且關羽與備,義為君臣,恩猶父子;羽死不能為興軍報敵,於終始之分不足。」七月,劉備不采纳赵云等人劝告,率軍沿江而下,討伐東吳。張飛被部下暗殺。孫權先派人給蜀漢送信求和,又令諸葛亮哥哥諸葛謹致箋勸劉備息兵罷戰,劉備一概拒絕。孫權把國都從建業遷到武昌,以便指揮戰爭。起初,漢軍氣勢如虹,不過吳將陸遜採以逸待勞兵法而戰之,於章武二年(222年)大敗漢軍。陸遜大敗劉備,「殺其兵八萬餘人,備僅以身免」。劉備退至秭歸,趙雲率兵到達白帝城,巴西太守閻芝派馬忠率5千人馬隨後到達。劉備退到永安縣。孫權聽聞劉備住白帝,甚為懼怕,遣使請和。章武二年十二月,孫權派太中大夫鄭泉到白帝城見劉備,正式表示向蜀漢請和。劉備也遣太中大夫宗瑋使吳,表示贊同蜀漢、東吳兩國和好。

當劉封失掉漢中東面三郡逃回成都後,諸葛亮勸劉備除掉劉封。漢嘉郡太守黃元聽說劉備在永安病重,於章武二年十二月舉兵反叛。同年,太傅許靖、尚書令劉巴、驃騎將軍馬超先後病逝。南中越夷高定曾向新道進攻,被李嚴打退。

章武三年(223年62歲)二月,諸葛亮接到劉備詔書,帶著劉永、劉理從成都來到永安。三月,黃元又乘諸葛亮到永安見劉備之機,率軍進攻臨邛縣,火燒臨邛城。益州治中從事楊洪立即把黃元之動向報告給劉禪,劉禪派將軍陳曶、鄭綽進討黃元。陳曶、鄭綽兩人在南安峽口生擒黃元,將其押回成都正法。四月下旬,劉備對諸葛亮說:「君才十倍曹丕,必能安國,終定大事。若嗣子可輔,輔之;如其不才,君可自取。(你的才能是曹丕的十倍,必定能夠安定國家,終可成就大事。如果嗣子(劉禪)可以輔助,便輔助他;如果他沒有才幹,你可以自取其位。)」諸葛亮涕泣說:「臣敢竭股肱之力,效忠貞之節,繼之以死!(臣必定竭盡自己所有力量,報效忠貞之氣節,繼續至死為止!)」劉備又要劉禪和其他兒子「與丞相從事,事之如父。」。劉備臨終前託孤於丞相諸葛亮,尚書令李嚴為副。臨終時,與劉永說:「吾亡之後,汝兄弟父事丞相,令卿與丞相共事而已。(我死後,你們兄弟要對父親般奉事丞相(諸葛亮),你們與丞相只是共事而已。)」。四月廿四(6月10日) 劉備崩逝於永安宮,享壽六十二歲。孫權派立信都尉馮熙出使蜀漢,弔唁劉備。諸葛亮上言讚揚劉備。五月癸巳日(6月21日),遺體自永安運返成都发丧,諡為昭烈皇帝。八月,入葬惠陵。

亦有郭沫若等学者认为由于条件所限,刘备就地下葬于今奉节县,并未归葬成都。

《三国志·先主传》中并没有记载刘备庙号。李慈铭怀疑刘备庙号烈祖是由刘渊所追尊。章学诚根据《三国志·先主传》中诸葛亮宣读的遗诏,指出刘备庙号是太宗。卢弼认为章学诚的说法不足据,如果刘备庙号太宗,《三国志》本传没有不记载的道理。郭善兵则认为刘备庙号缺失不能归咎于史书记载疏漏,而是受到郑玄礼学“一祖二宗与四亲庙”七庙学说影响所致。

刘备喜怒不形于色,常以谦虚恭敬待人,深知「得人心者得天下」的道理,重視以寬仁厚德待人,與那些殘民以逞、暴虐嗜殺的軍閥判然有別,因此而爭取到了人心。刘备不怎么喜欢读书,喜欢評馬論犬、音乐和華美的衣物。

小时候,家中有棵大桑树,遙望見如同车盖,刘备與宗中小兒於樹下玩耍時說過:「吾必當乘此羽葆蓋車。(我必定會乘坐此羽飾華蓋之車。)」叔父聽到後,不禁當下斥責他:「汝勿妄語,滅吾門也。(你不要胡說,會招來滅門之禍)」

劉備在部下聲譽受損或是特殊的理由發生背叛的可能時往往站出來捍衛部下聲譽和保護部下家眷,徐庶母被抓,庶淚崩辭別劉備、糜芳背叛,劉備對愧疚的糜竺說兄弟罪不相及、夷陵之敗黃權不得已降魏,劉備依然善待其家人「孤負黃權,權不負孤也。」

劉備由於沒有鬍鬚,因此曾被张裕取笑。有一次,劉備與劉璋於涪縣會面時,張裕時為璋從事,在一旁陪坐。由於裕的鬍鬚濃密被備嘲笑說道:“過去吾在涿縣時遇到好多姓毛的人,四方許多毛,涿縣縣令聲稱說:「許多毛(毛)繞涿(歜)而居。」,但裕立刻反唇相譏:“過去某人當上黨郡潞縣長,後來升任為涿縣縣令,其辭任歸家時,有人寫信給其,要是寫了潞縣就丟了涿縣,而寫上涿縣又失去潞縣,就寫道「潞(露)涿(歜)君」。因此裕就用此方法反譏備。讓劉備對張裕一直沒什麼好感,劉備攻漢中之前,張裕說會出師不利,但劉備仍照著既定計畫出兵。結果劉備拿下漢中,不過兩名大將吳蘭、雷銅等也在此戰中身亡,以致於劉備記恨張裕,某天,張裕私下對人說:「庚子年間(220年)會改朝換代;主公入主蜀地的九年後,也會再次失去蜀地,劉氏運氣將會消盡。」謠言亂傳,最終入劉備之耳,劉備不滿張裕散布滅亡謠言,以張裕的話語沒有應驗,把他關入獄中。諸葛亮請求劉備寬恕他,劉備只說:「芳蘭生門,不得不鉏。」於是殺了張裕,棄屍於街頭。

劉備死前告誡其子劉禪的遺詔,其言辭懇切,令人莫不動容。文中,劉備勸劉禪最重要的一句話,便是「勿以惡小而為之,勿以善小而不為。惟賢惟德,能服於人。」古人教子,常以德為根基,因為唯有賢德之人,才能服人。

刘备一生争战,乍看之下胜少败多。劉備攻打益州時,趙戩曾言:“刘备拙于用兵,每战必败。”認為劉備不會用兵,沒本事拿下益州,傅幹卻說:「劉備得人心,又有諸葛亮、關羽、張飛等人傑輔助,怎會不濟呢?」結果劉備果真取攻佔益州。在荊州依附劉表時,曾建議劉表北伐曹操,劉表不接受。劉備住荊州數年,一次與劉表飲酒時起至廁所,見大腿贅肉生,慨然流涕。還坐,劉表奇怪問起劉備,劉備說:「我戎馬半生,常常身不離鞍,大腿贅肉皆消。今天不復騎馬,大腿贅肉生。日月若馳,老年快將至矣,而功業不能建立,是以為之悲嘆。」

刘备与诸葛亮的君臣际遇,通常被史家视为君臣之典范。三顾茅庐后刘备称得到诸葛亮是“鱼之有水”。诸葛亮在刘备尚在时,就已经为丞相录尚书事假节,张飞被暗杀后又领司隶校尉,集政治实际权力于一身,这在古代是很罕见的。刘备去世时举国托孤诸葛亮,被陈寿称为“君臣之至公,古今之盛轨”。

陳壽评曰:“先主之弘毅宽厚,知人待士,盖有高祖之风,英雄之器焉。及其举国托孤于诸葛亮,而心神无贰,诚君臣之至公,古今之盛轨也。机权干略,不逮魏武,是以基宇亦狭。然折而不挠,终不为下者,抑揆彼之量必不容己,非唯竞利,且以避害云尔。”(《三國志·蜀書·先主傳第二》)、「劉備天下稱雄,一世所憚」(《三國志·吳書·陸遜傳第十三》)。尽管刘备并非西晋认为的正统政权,陈寿在三国志内仍然坚持使用同帝王本纪接近的用词,例如在刘备本传称刘备先主,称讳且不直呼其名,去世用和崩相等的殂字。这与三国时代另一位君主孙权的处理手法是不同的。这可以体现陈寿对刘备的尊重,抑或是故国情怀。

刘元起:“吾宗中有此儿,非常人也。”(《三國志·蜀書·先主傳第二》)

陈登:“雄姿杰出,有王霸之略,吾敬刘玄德。”(《三國志·魏書·桓二陳徐衛盧傳第二十二》)

袁绍:“刘玄德弘雅有信义,今徐州乐戴之,诚副所望也。”(《三國志·蜀書·先主傳第二》)

程昱:“观刘备有雄才而甚得众心,终不为人下,不如早图之。” 、“劉備有英名,關羽、張飛皆萬人敵也”(《三國志·魏書·程郭董劉蔣劉傳第十四》)

郭嘉:「备有雄才而甚得众心。张飞、关羽者,皆万人之敌也,为之死用。(郭)嘉觀之,(劉)備終不為人下,其謀未可測也。古人有言:『一日縱敵,數世之患。』宜早為之所。」(《三國志·魏書·程郭董劉蔣劉傳第十四》)

曹操:“方今收英雄时也,杀一人而失天下之心,不可。”、“夫刘备,人杰也,今不击,必为后患,将生忧寡人。”、“刘备,吾俦也。但得计少晚。”(《三國志·魏書·武帝紀第一》)“今天下英雄,唯使君与操耳。本初之徒,不足数也。”(《三國志·蜀書·先主傳第二》)

曹丕:「備不曉兵,豈有七百里營可以拒敵者乎!『苞原隰險阻而為軍者為敵所禽』,此兵忌也。孫權上事今至矣。」(《三國志·魏書·文帝紀第二》)

刘晔:「明公(曹操)以步卒五千,將誅董卓,北破袁紹,南征劉表,九州百郡,十並其八,威震天下,勢慴海外。今舉漢中,蜀人望風,破膽失守,推此而前,蜀可傳檄而定。刘备,人傑也,有度而迟,得蜀日淺,蜀人未恃也。今破漢中,蜀人震恐,其勢自傾。以公之神明,因其傾而壓之,無不克也。若小緩之,諸葛亮明於治而為相,關羽、張飛勇冠三軍而為將,蜀民既定,據險守要,則不可犯矣。今不取,必為後憂。」、「蜀雖狹弱,而備之謀欲以威武自強,勢必用眾以示其有餘。且關羽與備,義為君臣,恩猶父子。羽死不能為興軍報敵,於終始之分不足。」(《三國志·魏書·程郭董劉蔣劉傳第十四》)

贾诩:「吳、蜀雖蕞爾小國,依阻山水,有雄才,諸葛亮善治國,孫權識虛實,陸議見兵勢,據險守要,汎舟江湖,皆難卒謀也。用兵之道,先勝後戰,量敵論將,故舉無遺策。臣竊料群臣,無備、權對,雖以天威臨之,未見萬全之勢也。昔舜舞干戚而有苗服,臣以為當今宜先文後武。」(《三國志·魏書·荀彧荀攸賈詡傳第十》)

孫盛:“刘备雄才,處必亡之地,告急於吳,而獲奔助,無緣復顧望江渚而懷後計。”(《三國志·蜀書·先主傳第二》)

诸葛亮:“刘公雄才盖世,据有荆土,莫不归德,天人去就,已可知矣。”(《三國志·蜀書·董劉馬陳董呂傳第九》)“刘豫州王室之胄,英才盖世,众士仰慕,若水之归海,若事之不济,此乃天也,安能復为之下乎!”(《三國志·蜀書·諸葛亮傳第五》)

关羽:“吾受劉將軍厚恩,誓以共死,不可背之。”(《三國志·蜀書·關張馬黃趙傳第六》)

趙戩:“刘备其不济乎?拙于用兵,每战则败,奔亡不暇,何以图人?”(《三國志·蜀書·先主傳第二》)

傅幹:“刘备宽仁有度,能得人死力。諸葛亮達治知變,正而有謀,而為之相;張飛、關羽勇而有義,皆萬人之敵,而為之將;此三人者,皆人傑也。以備之略,三傑佐之,何為不濟也?”(《三國志·蜀書·先主傳第二》)

孙权:“非刘豫州莫可以当曹操者。”(《三國志·蜀書·諸葛亮傳第五》)「猾虜乃敢挾詐!」(《三國志·吳書·周瑜魯肅吕蒙傳第九》)

周瑜:“刘备以枭雄之姿,而有關羽、張飛熊虎之將,必非久屈為人用者。”(《三國志·吳書·周瑜魯肅吕蒙傳第九》)

陸遜:「備干天常,不守窟穴,而敢自送……尋備前後行軍,多敗少成,推此论之,不足为戚。」、「備是猾虜,更嘗事多」、「劉備天下知名,曹操所憚,今在境界,此强对也。」、「斯三虏者(曹操、劉備、關羽)当世雄杰,皆摧其锋。」(《三國志·吳書·陸遜傳第十三》)

张松:“刘豫州,使君之宗室而曹公之深雠也,善用兵,若使之讨鲁,鲁必破。鲁破,则益州强,曹公虽来,无能为也。”「劉豫州,使君之肺腑,可與交通。」「今州中諸將龐羲、李異等皆恃功驕豪,欲有外意,不得豫州(劉備),則敵攻其外,民攻其內,必敗之道也。」

刘巴:“备,雄人也,入必为害,不可内也。”

彭羕:“仆昔有事於诸侯,以为曹操暴虐,孙权无道,振威闇弱,其惟主公有霸王之器,可与兴业致治,故乃翻然有轻举之志。”(《三国志·卷四十·蜀书十·刘彭廖李刘魏杨传第十》)

锺会:“益州先主以命世英才,兴兵朔野,困踬冀、徐之郊,制命绍、布之手,太祖拯而济之,与隆大好。”

杨戏的《季汉辅臣赞》中赞昭烈皇帝:“皇帝遗植,爰滋八方,别自中山,灵精是锺,顺期挺生,杰起龙骧。始于燕、代,伯豫君荆,吴、越凭赖,望风请盟,挟巴跨蜀,庸汉以并。乾坤复秩,宗祀惟宁,蹑基履迹,播德芳声。华夏思美,西伯其音,开庆来世,历载攸兴。”

诸葛亮上言於刘禅曰:“伏惟大行皇帝迈仁树德,覆焘无疆,昊天不吊,寝疾弥留,今月二十四日奄忽升遐,臣妾号咷,若丧考妣。乃顾遗诏,事惟大宗,动容损益;百寮发哀,满三日除服,到葬期復如礼;其郡国太守、相、都尉、县令长,三日便除服。臣亮亲受敕戒,震畏神灵,不敢有违。臣请宣下奉行。”(《三國志·蜀書·先主傳第二》)

裴潜:“使居中國,能亂人,不能為治。若乘邊守險,足為一方之主。”(《世說新語·識鑒第七》)(《三國志·魏書·和常楊杜趙裴傳第二十三》)

吕布:“是儿最叵信者。”(《三国志·卷七·魏书七·吕布臧洪传》)

吕布诸将:“备数反覆难养,宜早图之。”(《三国志·卷32》注引王沈《魏书》)

习凿齿曰:“先主虽颠沛险难而信义愈明,势偪事危而言不失道。追景升之顾,则情感三军;恋赴义之士,则甘与同败。观其所以结物情者,岂徒投醪抚寒含蓼问疾而已哉!其终济大业,不亦宜乎!”(《三國志·蜀書·先主傳第二》)

常璩曰:「先主名微人鮮,而能龍興鳳舉,伯豫君徐,假翼荊楚,翻飛梁、益之地,克胤漢祚,而吳、魏與之鼎峙。非英才名世,孰克如之!」(《華陽國志·劉先主志》)

裴松之:「漢武用虛罔之言,滅李陵之家,劉主拒憲司所執,宥黃權之室,二主得失縣(懸)邈遠矣。《詩》云『樂只君子,保艾爾後』,其劉主之謂也。」(裴松之注《三國志·蜀書·黃李呂馬王張傳第十三》)

张辅:“刘备威而有恩,勇而有义,宽宏而大略”(《藝文類聚卷二十二》)

朱敬则:“蜀先主抱英济之器,无角逐之材。远窜荆蛮,畏曹公之神武;奄有庸蜀,乘刘璋之政衰。国小人夷,风颓俗陋。”(《全唐文》)

杜甫:“蜀主窥吴幸三峡,崩年亦在永安宫。翠华想像空山里,玉殿虚无野寺中。古庙杉松巢水鹤,岁时伏腊走村翁。武侯祠堂常邻近,一体君臣祭祀同。”

劉禹錫:“天地英雄氣,千秋尚凜然。勢分三足鼎,業复五銖錢。得相能開國,生兒不像賢。淒涼蜀故妓,來舞魏宮前。”

王勃:“以先主之宽仁得众,张飞、关羽万人之敌,诸葛孔明管、乐之俦,左提右挈,以取天下,庶几有济矣。然而丧师失律,败不旋踵。奔波谦、瓒之间,羁旅袁、曹之手,岂拙于用武,将遇非常敌乎?”

司馬光:「昭烈之漢,雖云中山靖王之後,而族屬疏遠,不能紀其世數名位,亦猶宋高祖稱楚元王後,南唐烈祖稱吳王恪後,是非難辨,故不敢以光武及晉元帝為比,使得紹漢氏之遺統也。」(《資治通鑑·第六十九卷·魏紀一》)

苏洵:“项籍有取天下之才,而无取天下之虑;曹操有取天下之虑,而无取天下之量;玄德有取天下之量,而无取天下之才。”

苏辙:“世之言者曰:孙不如曹,而刘不如孙。刘备唯智短而勇不足,故有所不若于二人者,而不知因其所不足以求胜,则亦已惑矣。盖刘备之才,近似于高祖,而不知所以用之之术。昔高祖之所以自用其才者,其道有三焉耳:先据势胜之地,以示天下之形;广收信、越出奇之将,以自辅其所不逮;有果锐刚猛之气而不用,以深折项籍猖狂之势。此三事者,三国之君,其才皆无有能行之者。独一刘备近之而未至,其中犹有翘然自喜之心,欲为椎鲁而不能纯,欲为果锐而不能达,二者交战于中,而未有所定。是故所为而不成,所欲而不遂。弃天下而入巴蜀,则非地也;用诸葛孔明治国之才,而当纷纭征伐之冲,则非将也;不忍忿忿之心,犯其所短,而自将以攻人,则是其气不足尚也。嗟夫!方其奔走于二袁之间,困于吕布而狼狈于荆州,百败而其志不折,不可谓无高祖之风矣,而终不知所以自用之方。”

謝采伯:「孫權運籌於內,劉備、諸葛亮、周瑜、關侯等,合謀並智,方拒得曹操,敗之於赤壁,亦未為竒政縁。」(《密齋筆記·卷二》)

何去非:“方其豪杰并起,而备已与之周旋于中原矣。始得徐州而吕布夺之,中得豫州而曹公夺之,晚得荆州而孙权夺之。备将兴复刘氏之大业,其志未尝一日而忘中州也。然卒无以暂寓其足,委而西入者,有曹操、孙权之兵轧之也。”

萧常:“昭烈父子以帝室支属,介在一隅,而正位号,尚数十年,由先汉至是,垂祀五百,三代以还,葢未之有。人主之结人心,其效廼尔,有大物者,庸可忽诸。”(《萧氏续后汉书》)

郝经:“汉得天统,莽簒而在光武,操窃而在昭烈。魏吴虽僣,犹夫吴楚也。昭烈天资仁厚,宇量(阙)毅,岿然一世之雄。以兴复汉室为己任,崎岖百折,偾而益坚。颠沛之际,信义逾明。故能终系景命,信大义于天下。任贤使能,洒落诚尽,使诸葛亮以死自效。复见三代君臣,高、光为不亡矣。国贼未讨,境土未复,而偾军崩殂,哀哉!”(《郝氏续后汉书》)

陶宗仪:“备又非人望之所归。周瑜以枭雄目之,刘巴以谁人视之,司马懿以诈力鄙之,孙权以猾虏呼之。”(《南村辍耕录》卷二十五)

杨璟:“昔据蜀最盛者,莫如汉昭烈。且以诸葛武侯佐之,综核官守,训练士卒,财用不足,皆取之南诏。然犹朝不谋夕,仅能自保。”

孙承恩:“贤矣昭烈,宽厚弘毅。崎岖立国,仗信履义。推诚任贤,肝胆孚契。顾命数词,可训后世。”(《文简集·卷三十八》)

王夫之:“刘先主以汉室之裔,保蜀土,奉宗祧,任贤图治,民用乂安,尚矣。”(《宋论·卷一·太祖》)

毛泽东:“刘备的优点主要于是善于用人,善于团结各方人士。而缺点则表现在两个方面:一是好感情用事;二是不能区分主次矛盾。”

趙翼:「關、張、趙雲自少結契,終身奉以周旋,即羈旅奔逃,寄人籬下,無寸土可以立業,而數人者患難相隨,別無貳志,此固數人者之忠義,而備亦必有深結其隱微而不可解者矣。」(《廿二史劄記·卷七》)

歷史學家、《三國史話》作者呂思勉認為,如其通觀前後,則劉備急於併吞劉璋,實在是失敗之遠因。倘使劉備老實一些,替劉璋北攻張魯,這是可以攻下;張魯既下,而馬超、韓遂等還未全敗,彼此聯合,以擾中原,曹操倒難於對付;劉備心計太工,不肯北攻張魯,而要反噬劉璋,以至替曹操騰出平定關中和涼州之時間,而且仍給以削平張魯之機。然而本可聯合涼州諸將共擾關中,卻變做獨當大敵。伐吳之役,劉備因為是能做一番事業,意志必較堅定,理智必較細密,斷不會輕易動於感情;況且感情必是動於當時,時間稍久,感情就漸漸衰退,理智就漸漸清醒。然其禍根,亦因急於要取益州,以致對於荊州不能兼顧之故;所以心計過工,有時也會成為失敗原因,真個閲歷多之人,倒覺得凡事還是少用機謀,依著正義而行好。

《劉備傳》作者張作耀認為,劉備人生道路危機四伏、滿途坎坷。這是一個戰鬥歷程:起步、挫折、爬起、再挫,發展至立足一方。撇開劉備政治動機不談,折而不撓、敗不氣餒、終不為下,為憧憬之目標而奮鬥不懈,始終如一。劉備與關羽、張飛一經結義,終身不易。與下士同席而坐,無所簡擇;善待部下,士卒感恩,願為驅使。

\subsubsection{章武}

\begin{longtable}{|>{\centering\scriptsize}m{2em}|>{\centering\scriptsize}m{1.3em}|>{\centering}m{8.8em}|}
  % \caption{秦王政}\
  \toprule
  \SimHei \normalsize 年数 & \SimHei \scriptsize 公元 & \SimHei 大事件 \tabularnewline
  % \midrule
  \endfirsthead
  \toprule
  \SimHei \normalsize 年数 & \SimHei \scriptsize 公元 & \SimHei 大事件 \tabularnewline
  \midrule
  \endhead
  \midrule
  元年 & 221 & \tabularnewline\hline
  二年 & 222 & \tabularnewline\hline
  三年 & 223 & \tabularnewline
  \bottomrule
\end{longtable}


%%% Local Variables:
%%% mode: latex
%%% TeX-engine: xetex
%%% TeX-master: "../../Main"
%%% End:

%% -*- coding: utf-8 -*-
%% Time-stamp: <Chen Wang: 2021-11-01 11:36:18>

\subsection{后主劉禪\tiny(223-263)}

\subsubsection{生平}

劉禪(207年-271年),字公嗣,又字升之。蜀漢昭烈帝劉備之子,蜀漢最後一位皇帝,史學家称蜀漢後主,223年—263年在位,歷時四十一年,是三國在位最長之皇帝。

据《三国志》记载,劉禪由刘备的妾室甘夫人所生,是刘备三位庶子中最为年长的。

212年(建安十七年),刘备入蜀,孙权派人接回孫夫人,孫夫人想将五歲的刘禅一并带走,诸葛亮派遣赵云夺回。

刘禅继位初期确实听从父亲的遗命,放权于丞相诸葛亮处理军政大事,“政事无巨细,咸决于亮”。

延熙元年(公元238年),詔命蔣琬應嚴整治軍,率各軍屯紮漢中,等東吳行動,兩國構成東西犄角之勢,伺機伐魏。

劉禪始“乃自摄国事”,由蔣琬、費禕、董允等人主政,修养生息,积蓄力量后从长计议再北伐的政策。劉禪對於寵臣陳祗與宦官黃皓也頗為寵信,姜維畏懼黃皓,只得擁兵屯墾汉中的沓中(今甘肃甘南藏族自治州迭部)。

景耀六年(公元263年),姜維上表後主:「聽聞鐘會治兵關中,欲規畫進一步拓取土地之意,宜一併派遺張翼、廖化督率各軍,分別護陽安關口、陰平橋頭,以防患於未然」,黃皓徵求鬼巫信息,謂敵人終究不會自來,而劉禪也信了鬼巫,滿朝文武竟沒有一人知曉。

最后邓艾偷渡阴平大军压境,刘禅與群臣商議如何抵禦,決定派諸葛瞻領兵迎戰,但諸葛瞻戰敗。最後,刘禅接受谯周的建议,在农历十一月向曹魏投降。劉禪派太僕蔣顯至劍閣,傳令姜維等部投降,蜀軍悲憤不已,紛紛拔刀砍石。邓艾承制拜刘禅为骠骑将军。

蜀漢亡后,刘禅移居魏国都城洛阳,封為安乐县公(常璩则作北巫县安乐乡公)。某日司马昭设宴款待刘禅,囑咐演奏蜀乐曲,并以歌舞助兴时,蜀漢旧臣们想起亡国之痛,个个掩面或低頭流泪。獨刘禅怡然自若,不為悲傷。司马昭见到,便问刘禅:“安樂公是否思念蜀?”刘禅答道:“此間樂,不思蜀也。”他的旧臣郤正闻此言,趁上廁所時对他说:“陛下,下次如司马昭若再问同一件事,您就先注視著宮殿的上方,接著閉上眼睛一陣子,最後張開雙眼,很認真地說:‘先人坟墓,远在蜀地,我没有一天不想念啊!’这样,司马昭就能让陛下回蜀了。”刘禅听后,牢记在心。酒至半酣,司马昭又问同样的问题,刘禅赶忙把郤正教他的学了一遍。司马昭听了,即回以:“咦,这话怎么像是郤正说的?”刘禅大感惊奇道:“你怎麼知道呀!”司马昭及左右大臣哈哈大笑。司马昭见刘禅如此老实忠懇,从此再也不怀疑他,刘禅就这样在洛阳度过餘生,也是乐不思蜀一詞的典故。

西晉晉武帝泰始七年(271年),刘禅去世,諡刘禅為思公。

刘禅太子刘璿在钟会之乱中丧生,按次序应该立次子刘瑶为继承人,但刘禅偏爱六子刘恂,立刘恂为继承人,旧臣文立劝谏,不听,于是刘恂袭为安乐公。

西晉末年,刘渊起事,國號為漢,即汉赵政权,追諡刘禅為孝怀皇帝,但其子孙皆已被灭族,而刘渊是匈奴血统,与刘禅并无直接血缘关系。

刘备在遗诏中说:「射君(射援)到,说丞相叹卿(即刘禅)智量,甚大增修,过于所望,审能如此,吾复何忧!勉之,勉之!」

諸葛亮在與杜微書中評價後主說:「朝廷年方十八,天資仁敏,愛德下士。」

蜀郡太守王崇論後主曰:「昔世祖内資神武之大才、外拔四屯之奇將、猶勤而獲濟。然乃登天衢、車不輟駕、坐不安席。非淵明弘鑒、則中興之業何容易哉。後主庸常之君、雖有一亮之經緯、内無胥附之謀、外無爪牙之將、焉可包括天下也。」“邓艾以疲兵二万溢出江油。姜维举十万之师,案道南归,艾易成禽。禽艾已讫,复还拒会,则蜀之存亡未可量也。乃回道之巴,远至五城。使艾轻进,径及成都。兵分家灭,己自招之。然以钟会之知略,称为子房;姜维陷之莫至,克揵筹斥相应优劣。惜哉!”(華陽國志)

司马昭:「人之无情,乃可至於是乎!虽使诸葛亮在,不能辅之久全,而况姜维邪?」

陳壽於《三国志》:“后主任贤相则为循理之君,惑阉竖则为昬闇之后,传曰‘素丝无常,唯所染之’,信矣哉!礼,国君继体,逾年改元,而章武之三年,则革称建兴,考之古义,体理为违。又国不置史,注记无官,是以行事多遗,灾异靡书。诸葛亮虽达于为政,凡此之类,犹有未周焉。然经载十二而年名不易,军旅屡兴而赦不妄下,不亦卓乎!自亮没后,兹制渐亏,优劣著矣!”、認為劉禪是「素絲無常,唯所染之」,早年得諸葛亮輔助,所以「任賢相則為循理之君」;但後來寵信黃皓,敗壞政事,卻是「惑閹豎則為昏闇之后」。但與暴虐好殺的孫皓相比,劉禪要更為善於處理政務且與大臣們保持著良好的互動。

薛珝:“主暗而不知其过,臣下容身以求免罪,入其朝不闻正言,经其野民有菜色。”

晉朝張華問李密:「安樂公(劉禪)何如?」密曰:「可次齊桓。」華問其故,對曰:「齊桓得管仲而霸,用豎刁而蟲流。安樂公得諸葛亮而抗魏,任黃皓而喪國,是知成敗一也。」(晉書‧李密傳)

裴松之为《三国志·三少帝纪》作注,在评论郭修刺杀费祎事时,称刘禅为“凡下之主”。

孫盛:“刘禅暗弱,无猜险之性。”“禅虽庸主,实无桀、纣之酷,战虽屡北,未有土崩之乱,纵不能君臣固守,背城借一,自可退次东鄙以思后图。”,認為劉禪是「庸主」。

李特:“刘禅有如此江山而降于人,岂非庸才?”(華陽國志)

常璩:“主非中兴之器。”(華陽國志)

张璠:“刘禅懦弱,心无害戾。”

朱敬则:“若乃投井求生,横奔畏死,面缚请罪,膝行待刑,是其谋也。马上唱无愁之歌,侍宴索达摩之曲,刘禅不思陇蜀,叔宝绝无心肝,对贾充以不忠之词,和晋帝以邻国之咏,是其才也。纵黄皓,嬖岑昏,宠高壤,狎江总,是其任也。剥面凿眼,孙皓之刑;弃亲即雠,高纬之志。其馀细故,不可殚论。听吾子之悬衡,任夫人之明镜。”(《全唐文》)

陈世崇:“孔明之子瞻、孙尚战死,张飞之孙遵,赵云次子广亦战死,北平王谌哭于昭烈庙,先杀妻子乃自杀,魏以蜀宫人赐将士,李昭仪不辱自杀。禅不特愧于将士,亦且愧于妇人矣。”

俞德邻:“禅以暗弱之资,而又惑于阉竖,使无此谶,其能与魏争乎?”

郑玉:“孔明盖社稷之臣也,今刘禅昏愚暗弱,纵使伊尹阿衡、周公辅相,亦必危亡而后已,虽百孔明,如之何哉?”“孔明既死,刘禅卒就擒缚。及其入魏,屈辱百端,略无愧耻。岂惟刘氏之宗社不嗣,遂使高祖、光武含羞地下,抱憾无穷。”

王夫之:「後主失德而亡,非失險也,恃險也,恃則未有不失者也。君恃之而棄德,將恃之而棄謀,士卒恃之而棄勇。伏弩飛石,恃以卻敵;危石叢薄,恃以全身;無致死之心,一失其恃,則匍伏奔竄之恐後;扼以於蹊徑,而淩峭壁以下攻,則首尾不相顧而潰。故謂後主信巫言而失陰平之守以亡國,非也。陰平守,而亙數百里之山厓谿谷,皆可度越,陰平一旅,亦贅疣而已。李特過劍閣而歎劉禪之不能守,艸竅之智,乘晉亂以茍延爾。譙縱、王建、孟知祥、明玉珍蹶然而起,熸然而滅,恃險愈甚,其亡愈速矣。」《讀通鑒論·卷十》

罗贯中:“祈哀请命拜征尘,盖为当时宠乱臣。五十四州王霸业,等闲抛弃属他人。”“魏兵数万入川来,后主偷生失自裁。黄皓终存欺国意,姜维空负济时才。全忠义士心何烈,守节王孙志可哀。昭烈经营良不易,一朝功业顿成灰。”

潘时彤:“可惜三分鼎,空怜六尺孤。大权归宦竖,强敌问神巫。斫石军心愤,回天将胆粗。山头曾学射,一矢报仇无。”

《三國志》盧弼集解引周壽昌說:「五丈原头大星夜陨,至千载下犹有余恫。廖公渊、李正方俱为武侯贬退,侯死皆痛泣而卒。李邈何人敢为此疏,直是全无心肝。使非后主之明断,则谗慝生心,乘间构衅,恐唐魏元成仆碑之祸,明张太岳籍没之惨,不待死肉寒而君心早变矣。见疏生怒,立正刑诛,君子谓后主之贤,于是乎不可及。」「(樂不思蜀一事)恐傳聞失實,不則養晦以自全耳。」

清朝方苞《望溪先生文集》中有〈蜀漢後主論〉一文,論曰:「亡國之君若劉後主者,其為世詬歷也久矣,而有合乎聖人之道一焉,則任賢勿貳是也。其奉先主之遗命也,一以国事推之孔明而己不与,世犹曰以师保受寄托,威望信于国人,故不敢贰也。然孔明既殁,而奉其遗言以任蒋琬、董允者,一如受命于先主。及琬与允殁,然后以军事属姜维,而维亦孔明所识任也。夫孔明之殁,其年乃五十有四耳。使天假之年而得乘司马氏君臣之瑕衅,虽北定中原可也。即琬与允不相继以殁,亦长保蜀汉可也。然则蜀之亡,会汉祚之当终耳,岂后主有必亡之道哉!嗚呼!使置後主他行而獨舉其任孔明以衡君德,則太甲、成王當之有愧色矣!」

蔡东藩:“成都虽危,尚堪背城借一,后主宁从谯周,不从北地王谌,面缚出降,坐丧蜀土,是咎在后主。”

魏国史书《魏略》中记载,刘禅在刘备于徐州被曹操攻打时与家人走失,因而被人口贩子拐卖,到了汉中,被一个叫做刘括的人收养。后来刘备入蜀之后,一名簡姓將軍(疑為簡雍)到汉中出使,刘禅找到他并讲解儿时故事,記得父親字玄德,证明自己的确是刘备儿子。张鲁于是下令把刘禅还给刘备,刘备才把他立为继承人。间接来讲,若这个记载为真,赵云在当阳救刘禅以及拦江截阿斗都是蜀汉编造的故事。然而裴松之根据《三国志》的说法对这个记载提出质疑,指出年齡上並不符合,后世也多采信裴松之。


\subsubsection{建兴}

\begin{longtable}{|>{\centering\scriptsize}m{2em}|>{\centering\scriptsize}m{1.3em}|>{\centering}m{8.8em}|}
  % \caption{秦王政}\
  \toprule
  \SimHei \normalsize 年数 & \SimHei \scriptsize 公元 & \SimHei 大事件 \tabularnewline
  % \midrule
  \endfirsthead
  \toprule
  \SimHei \normalsize 年数 & \SimHei \scriptsize 公元 & \SimHei 大事件 \tabularnewline
  \midrule
  \endhead
  \midrule
  元年 & 223 & \tabularnewline\hline
  二年 & 224 & \tabularnewline\hline
  三年 & 225 & \tabularnewline\hline
  四年 & 226 & \tabularnewline\hline
  五年 & 227 & \tabularnewline\hline
  六年 & 228 & \tabularnewline\hline
  七年 & 229 & \tabularnewline\hline
  八年 & 230 & \tabularnewline\hline
  九年 & 231 & \tabularnewline\hline
  十年 & 232 & \tabularnewline\hline
  十一年 & 233 & \tabularnewline\hline
  十二年 & 234 & \tabularnewline\hline
  十三年 & 235 & \tabularnewline\hline
  十四年 & 236 & \tabularnewline\hline
  十五年 & 237 & \tabularnewline
  \bottomrule
\end{longtable}

\subsubsection{延熙}

\begin{longtable}{|>{\centering\scriptsize}m{2em}|>{\centering\scriptsize}m{1.3em}|>{\centering}m{8.8em}|}
  % \caption{秦王政}\
  \toprule
  \SimHei \normalsize 年数 & \SimHei \scriptsize 公元 & \SimHei 大事件 \tabularnewline
  % \midrule
  \endfirsthead
  \toprule
  \SimHei \normalsize 年数 & \SimHei \scriptsize 公元 & \SimHei 大事件 \tabularnewline
  \midrule
  \endhead
  \midrule
  元年 & 238 & \tabularnewline\hline
  二年 & 239 & \tabularnewline\hline
  三年 & 240 & \tabularnewline\hline
  四年 & 241 & \tabularnewline\hline
  五年 & 242 & \tabularnewline\hline
  六年 & 243 & \tabularnewline\hline
  七年 & 244 & \tabularnewline\hline
  八年 & 245 & \tabularnewline\hline
  九年 & 246 & \tabularnewline\hline
  十年 & 247 & \tabularnewline\hline
  十一年 & 248 & \tabularnewline\hline
  十二年 & 249 & \tabularnewline\hline
  十三年 & 250 & \tabularnewline\hline
  十四年 & 251 & \tabularnewline\hline
  十五年 & 252 & \tabularnewline\hline
  十六年 & 253 & \tabularnewline\hline
  十七年 & 254 & \tabularnewline\hline
  十八年 & 255 & \tabularnewline\hline
  十九年 & 256 & \tabularnewline\hline
  二十年 & 257 & \tabularnewline
  \bottomrule
\end{longtable}

\subsubsection{景耀}

\begin{longtable}{|>{\centering\scriptsize}m{2em}|>{\centering\scriptsize}m{1.3em}|>{\centering}m{8.8em}|}
  % \caption{秦王政}\
  \toprule
  \SimHei \normalsize 年数 & \SimHei \scriptsize 公元 & \SimHei 大事件 \tabularnewline
  % \midrule
  \endfirsthead
  \toprule
  \SimHei \normalsize 年数 & \SimHei \scriptsize 公元 & \SimHei 大事件 \tabularnewline
  \midrule
  \endhead
  \midrule
  元年 & 258 & \tabularnewline\hline
  二年 & 259 & \tabularnewline\hline
  三年 & 260 & \tabularnewline\hline
  四年 & 261 & \tabularnewline\hline
  五年 & 262 & \tabularnewline\hline
  六年 & 263 & \tabularnewline
  \bottomrule
\end{longtable}

\subsubsection{炎兴}

\begin{longtable}{|>{\centering\scriptsize}m{2em}|>{\centering\scriptsize}m{1.3em}|>{\centering}m{8.8em}|}
  % \caption{秦王政}\
  \toprule
  \SimHei \normalsize 年数 & \SimHei \scriptsize 公元 & \SimHei 大事件 \tabularnewline
  % \midrule
  \endfirsthead
  \toprule
  \SimHei \normalsize 年数 & \SimHei \scriptsize 公元 & \SimHei 大事件 \tabularnewline
  \midrule
  \endhead
  \midrule
  元年 & 263 & \tabularnewline
  \bottomrule
\end{longtable}


%%% Local Variables:
%%% mode: latex
%%% TeX-engine: xetex
%%% TeX-master: "../../Main"
%%% End:


%%% Local Variables:
%%% mode: latex
%%% TeX-engine: xetex
%%% TeX-master: "../../Main"
%%% End:

%% -*- coding: utf-8 -*-
%% Time-stamp: <Chen Wang: 2019-12-18 10:23:25>


\section{孙吴\tiny(229-280)}

%% -*- coding: utf-8 -*-
%% Time-stamp: <Chen Wang: 2021-11-01 11:36:31>

\subsection{大皇帝孫權\tiny(229-252)}

\subsubsection{生平}

孫權(182年12月22日-252年5月21日),字仲謀,吴郡富春(今浙江省杭州市富阳区)人,東漢末三国時期吳[註 1]的著名政治家、戰略家,同時也是吳的締造者及建国皇帝。而在孫權稱帝之前,吳的群臣等對其稱呼為將軍或至尊。在位23年,享年69歲,諡號為大皇帝,廟號太祖。

富春孙氏是江东不顯赫的豪族,世代仕於吳。生父孫堅據傳是春秋时期军事家孫武後人,孤微發跡;孙权亦因此可能是孙武的第22代孙。

孫權生母为吳郡豪族出身的吴夫人,當初懷孕的時候,夢見月亮進去懷裡,之後生下了孫策。及後在懷孫權的時候,又夢見太陽進去懷裡。之後告訴孫堅說:“妾昔日懷著孫策的時候,夢見月亮入懷裡;如今又夢見太陽入懷裡,為什麼會這樣呢?”孫堅回答:「太陽和月亮,是陰陽的能量精氣,是極其貴象的征兆。我們的子孫大概會興家赤旺吧!」。漢光和五年五月十八日(182年7月5日),孫堅擔任下邳县丞的時候嫡次子孫權出生,其面相方頰大口,銳目有神,孫堅覺得驚奇,認為有貴氣的象相。

漢光和七年(184年),朱儁奏请孙坚担任佐军司马,孙坚随朱儁南征北战,将妻吴氏和孙权等诸子都留在九江郡寿春县。

漢中平六年(189年),汉灵帝逝世,长沙(治所在今湖南省长沙市)太守的孙坚起兵从长沙经荆州响应讨伐董卓的关东联军。当时孙权的长兄孙策已在寿春淮南一带颇有名气。其中有庐江人周瑜前来拜会,在周瑜的建议下,孙策于是携母弟搬到庐江郡舒县(今安徽省庐江县西南)。

漢初平二年(191年),孙坚奉袁术之命讨伐荆州刘表,结果中刘表手下的黄祖的埋伏身亡。孙权和家人迁居广陵郡江都。孫策託付张紘照看母弟。自孫堅死後,孫權經常跟隨兄長。孙权性格寬宏有氣度,不但仁厚而且能夠根據不同情況作出多方面判斷。他以厚恩养士而出名,其名气渐渐不输给父兄。孙策也对这个弟弟感到很惊奇,自认为不如他。每当宴请宾客时,孙策常常回头看着孙权说:“这些人,以后都会是你的将领。”

漢初平四年(193年)因孙策决定跟随袁术,就派吕范将孙权等人护送到住在曲阿的舅舅吴景那里居住。翌年,孙策击破了陆康为袁术取得了庐江郡。当时,还是扬州刺史的刘繇担心自己也会被袁氏吞并,与袁术和孙策产生嫌隙,于是将孙权堂兄孙贲和吴景驱逐出曲阿,只有孙权及其母弟弟们还留在那里,于是朱治特意将其从曲阿接到自己家里奉养卫护。孙权和母亲后来又迁至历阳县和阜陵县居住。

漢興平二年(195年)孙策渡江击败刘繇后,孙权和家人跟随着陈宝回到了曲阿居住。孙权到江東以后,与朱然胡综一起读书,结下了深厚的友谊。

漢建安元年(196年),孙权15岁的时候,由朱治举为孝廉,任阳羡县(今江苏宜兴)长,代行奉义校尉。曹操也任命严象将其举为茂才,当时已有属下周泰和潘璋。

孙策平定江东的丹阳、会稽和吴三郡后开始给汉廷进贡。建安二年(197年),汉廷派刘琬前往江东授予孙策会稽太守的职务,刘琬对人说:“我看孙家的兄弟虽然每个都才华横溢,智慧通达,都是荣华福贵不长久。只有次男孝廉,相貌高大挺拔,有大贵之表,且会是最為长寿的,你们等着瞧吧。”袁术与孙策决裂后,拉拢丹阳等六县及山贼头目祖郎,鼓动山越和自己一起共同对付孙策。当时孙策率兵前往讨山贼,仅孙权等数百人留在宣城,山贼数千人蜂拥而至,年轻的孙权在周泰的保护之下得以幸免。

漢建安四年(199年)末至次年初,孙权随同孙策征庐江太守刘勋于皖城。刘勋败逃后,又进军沙羡讨伐黄祖,与仇敌黄祖在沙羡一带展开大战,黄祖几乎全军覆没,韩唏战死,黄祖只身逃走,士卒溺死者达万人,豫章太守华歆又举城投降。平定了庐江豫章二郡。孫策與曹操交好,表面臣服於漢朝廷之下,曹操並加封孫策為吳侯,並以礼征辟孙权和孙权的弟弟孙翊到漢朝廷擔當漢臣職務,但二人均沒有前往。

漢建安五年(200年)春,孙策遭到刺殺,不選擇與自己性格極其相似,眾人看重為最適合繼任者的三弟孫翊,而是選擇性格與自己大不相同的二弟孫權。孫權對兄長的去世痛哭不已沒能親自視察政事。經過長史张昭勸說,乃除去喪服,由張昭扶上馬外出巡察軍營,于是眾人之心都歸附於孫權。

曹操見孫策已死本打算伐吳,侍御史張紘勸諫曹操不該乘人之危。曹操聽從其言,通過东汉朝廷冊封孫權为討虏将军,兼领会稽太守,以吴县为治所。

孫權剛上任,只占有会稽、吴郡、丹杨、豫章、庐陵、庐江六郡,除孫權本人為會稽太守外,其他五人都是孙策生前所任命的部將,分別是:丹陽太守吳景、豫章太守孫賁、廬陵太守孫輔、吳郡太守朱治、廬江太守李術。然而孫策死後,孫輔認為孫權沒有能力保衛江東,於是与曹操暗通打算出賣孫權,被孫權察覺後給予制裁。堂兄孙暠(孫靜長男)欲攻打会稽郡夺取政權,被虞翻阻止。這其中李术尤为不服從孫權,與梅乾、雷緒、陳蘭數萬人在集結淮水一帶騷擾破壞,孫權寫信要求李術扣留這些叛逃者。李術公開表示有德見歸,無德見叛,不應復還為由拒絕。於是孫權用計策寫信給曹操,說嚴象被殺是李術所為,所以不應該理會李術。孫權隨後與孫河、徐琨一起親征叛徒李術于皖城。皖城被孫權包圍,李術向曹操求救,但是曹操沒有到來,一切發展正如孫權所設計的一樣。城內糧盡只能用泥丸代替食糧充飢,隨即破皖城,李術被梟首,孫權迁徙城裡人及李術部将三万余人到江東,留下一座空城。

孫策平定江東的時候,曾對當地士族進行打壓、屠戮,導致孫家在本土得不到支持。孙权以张昭为師傅,並任用父兄留下的部將,以部曲私兵世襲制作為條件懷柔本土豪族,大量起用豪族子弟,穩定江東孫家政權。陆逊、徐盛、留贊、诸葛瑾、步骘、顾雍、顧徽、是仪、吕岱、朱桓、骆統等贤才良将都在这一时期加入孙权麾下。周瑜斷言他以後能成就帝王大業,并将好友鲁肃推荐给孫權认识。鲁肃则向孙权說出漢室不能復興,曹操不能一時間消滅。應該鼎足江東靜觀其變,在北方多戰亂的時候乘勢应消滅刘表,佔據长江以南建立帝業的方案。

未開發山地潛藏的山越也大規模發動叛亂,而江東許多的本地豪族士族與山越族群都有緊密聯繫,因此,在孫氏每次出征對外的時候,都給予江東內部很大的侵擾,也一直牽制著吳國數十年的對外作戰,漢建安八年(203年),豫章鄱陽縣等地山越再起,孫權即刻命征虜中郎將呂範平定鄱陽、蕩寇中郎將程普討伐樂安,派賀齊討平東冶地,建昌都尉太史慈分頭進討山越,又派別部司馬黃蓋、韓當、呂蒙等人扼守山越經常出沒的郡縣,恢復了原設縣邑,穩定了秩序。漢建安十一年(206年)又率領孫瑜,周瑜,淩統,成功討平山越麻、保二屯。

漢建安十二年(207年),自黄祖一处来降的甘寧說:「今漢已經日漸衰微,曹操為滿足自己的心,終於成了篡漢的盜賊。南荊之地,山陵地勢有利,江川流通,國的西邊的確是這樣的形勢。我已看透劉表,考慮的不夠長遠,兒子也是無能的人,不是能夠承傳基業之才。主公應當盡早規劃,不能落入曹操手上。進圖之計,先取黃祖為佳。黃祖如今年老,老邁衰退嚴重,錢財糧谷都已經缺乏,左右矇騙他,事出於錢財私利,侵要吏士的錢財,吏士心裏都憤怒。舟船戰具,廢棄也不修理,耕農懶惰,軍隊沒有法紀。如果主公現在去攻打,必定能大敗。一旦打敗黃祖軍,擊鼓行軍至西,西據楚關,大局趨勢擴張,這樣就可以逐漸進取巴蜀。」孫權贊同並採納。張昭當時就在席上坐,難言道:「吳國如今危懼,如果行軍攻打,必然招致恐慌。」甘寧回答道:「國家將蕭何的重任交給君,君留置守護卻擔心憂亂,那為什麼還要仰慕古人?」孫權對舉起酒杯附於甘寧說:「興霸,今年行軍討伐,就如這杯酒,決意託付給卿你。卿盡量提出方略,如能夠破黃祖,則是卿的功勞,不要因為張長史(張昭)之言而放棄。」出兵虏其人民而还。

漢建安十三年(208年),孙权發動江夏之戰再讨黄祖,以周瑜為都督。呂蒙隨軍出征。黃祖見孫權兵來,急派水軍都督陳就率兵反擊,呂蒙統率前鋒部隊,身先戰陣,親自斬殺陳就。擄獲其船隻、士兵。返回到孫權大軍,並引領自軍兼程趕路,水路兩路齊進。凌統先攻下城池,黃祖隻身逃竄,被孫權軍中的騎兵馮則所斬殺。此戰,孫權大獲全勝,但是劉表長子劉琦及時前來禦敵江夏北部,孫權只能有效佔領江夏郡南部區域,後将治所自吴移居至京口。

漢建安十三年(208年)秋,曹操對孫權發出以八十萬軍力會獵江東的書信,孫權打算與曹操決一死戰,但張昭等群臣勸孫權歸降,礙於豪族群臣的壓力下孫權沒有表達自己的意見,聽後離開席間換衣服,唯獨魯肅離座找孫權說要對抗曹操,孫權很高興魯肅與自己的想法一致,對張昭等人所說的感到非常失望,魯肅勸孫權召回進兵鄱陽的周瑜,並邀請劉備加盟的提議。孫權答應,隨即派魯肅到荊州打探情況。當時荆州牧刘表病死,劉表次子劉琮及其母蔡氏其舅蔡瑁因仇視刘备而投降曹操。鲁肃到荊州之前劉備被曹操打敗,荊州已經落入曹操之手,劉備南渡长江,魯肅與他相遇詢問去向,劉備打算到蒼梧投靠朋友吳巨,魯肅則說明孫權的意向和實力,邀請劉備加盟孫權共同對抗曹操,而不是投靠力弱的人。刘备很高興孫權的邀請,聽後見事態緊急隨即派诸葛亮去求見孙权。孫權故意刁難諸葛亮,藉此通過他對曹操軍的分析,去說服江東豪族及投降派,孫權聽後大悅。之後群臣商議,眾人勸孫權投降,但周瑜向孫權分析曹操與孫權兩軍的優劣勝敗,指出:「其一,曹軍背後仍有後顧之憂,西涼有馬騰、韓遂等軍閥,戰端一開,必偷襲曹軍背後。」、「其二,北方人慣習陸戰而不擅水戰,竟敢捨馬鞍而就船槳,此乃捨長就短。」、「其三,寒冬將至,曹軍兵缺衣食,馬無藁草,兵卒士氣低落。」、「其四,曹軍遠途跋涉,奔襲千里,水土不服,多生病患。」既而進步分析了曹軍的實際力量,指出來自中原的曹軍不過十五六萬,而且所得劉表新降的七八萬人,人心並不向曹。」此時只有周瑜、鲁肃坚持抗击曹操的主张,意见与孙权相合。隨即以決斷之勢拔劍砍掉桌子一角,說:「敢再有言降曹者,如同此案!」,藉此將投降派氣焰壓倒,並將一早已經準備好的三萬軍隊交給周瑜指揮。周瑜、程普分别被任命为左、右都督,魯肅為贊軍校尉輔助周瑜。孫權派周瑜抵禦曹操大軍,在赤壁与曹军相遇,周瑜大败曹操军队。周瑜等又追击到烏林破曹軍,曹操只好撤回北方,乘勝進攻荊州南郡。甘宁在夷陵城,被曹仁的部队所包围,周瑜采纳吕蒙的计策,留下凌统抗拒曹仁,用其中一半兵力驰救甘宁,南郡相持一年間,孙权為了減低周瑜們的前線壓力,親率剩餘的小量軍力军围合肥,相持合肥一个多月,聽從張紘的建議撤退。而劉備以張飛和一千人換二千人為條件向周瑜借兵,然後在孫曹交戰間乘機攻取了长沙、桂阳、武陵、零陵荊南四郡,並上表劉琦為荊州牧,領有了荊南四郡。

漢建安十四年(209年),周瑜攻破南郡。孙权以周瑜为南郡太守。刘备上表奏封孙权代理车骑将军,兼任徐州牧。孙权又招揽了滕耽、吾粲等人。當時劉琦去世,失去了荊州四郡領有權,劉備隨即向周瑜借地,周瑜分南岸給刘备,後劉備將油江口改名為公安。劉備嫌地少無法容納人馬,親自到京口見孙权借荊州數郡(南郡、长沙、桂阳、武陵、零陵)並督領荊州,周瑜、呂範提議軟禁劉備,孙权聽從魯肅所說而不採納,並借出荆州数郡於刘备,孫權暫表劉備為荊州牧,孫權以孫夫人聯姻來鞏固孙刘联盟的關係,也奠定了三国鼎立的基础。周瑜和甘寧勸孫權入蜀,孫權邀請劉備共同取益州,劉備以劉璋是同祖宗為由拒絕,並說如果孫權打劉璋自己一定阻止,如果我打劉璋的話,我必定會披髮入山林歸隱,不做攻取同宗的事。孫權不聽,並派孫瑜進攻益州,劉備阻止並不給孫瑜前進的去路,並說不能這樣做,孫權只有下令退還。

漢建安十五年(210年),孙权任命鲁肃为太守,驻守陆口,又遣步骘为交州刺史,挥师南征。吴军压境,交州各郡守无不俯首,士燮率领家族奉承节度。唯有刘表所置苍梧太守吴巨阳奉阴违,最后被步骘發現有異心,隨即斬殺。孙权自此得到交州九郡領有權,并加封臣服于自己的士燮为左将军。南海郡、郁林郡、苍梧郡則是孫權管治,交阯郡、日南郡、珠崖郡、儋耳郡、九真郡、合浦郡則是士夑獨立管治。

漢建安十六年(211年),孙权将治所迁至秣陵。次年,孙权修筑石头城,改秣陵为建业。聽聞曹操率四十萬大軍進攻,孫權打算興建水塢,部將大家都認為直接上下船就能著陸登船,建造水塢沒用,只有呂蒙認為這個塢可以給步兵快速登船進退不失的便利,於是孫權同意呂蒙看法,派呂蒙建造濡须坞作為進出濡須到巢湖的水軍防衛要塞,也是濡須之戰的重要補給據點,與日後曹魏建造的合肥新城是互相對應的防衛設施。

漢建安十八年(213年)正月,曹操親率四十萬大軍攻孫權於濡須口,孫權向劉備發出救援。當時劉備作為客將在劉璋之下,劉備以救孫權為由向劉璋借兵去救荊州關羽,劉璋對劉備猜疑只給他一半軍需和4000兵馬。劉備憤怒劉璋給物資和兵少,隨後密謀反戈偷襲了益州劉璋,劉備也沒有理會孫權求援,任由他們自生自滅。孫權知道劉備攻劉璋而不來救援,出爾反爾,大罵劉備是狡猾的傢伙竟然敢使詐。濡須戰場最後只有孫權軍獨力以七萬大軍抵擋曹操號稱的四十萬大軍,起初戰況不好。孫權出戰沒有得到收穫,濡須口的江西營被曹軍打破並俘虜都督公孫陽,而董襲趕往救援的途中遇溺去世,孫權軍霎時間頓挫。下半場戰鬥,曹操作油船在夜中親率打算襲擊洲上孫權軍。孫權親自率軍乘機反擊,驅使水軍突襲包圍了曹操,曹操軍落水溺死有數千人,俘虜敵兵人數也有三千餘人。孫權乘勝追擊,並對曹操進行多次挑釁,但是曹操受到孫權的打擊下而不敢出擊接戰。孫權見曹操堅守不出,親自督一艘船從濡須塢出擊進入曹操大軍陣地觀陣,《吳錄》記載曹操軍眾人打算射擊孫權的船,但曹操知道孫權來觀陣下令不要妄動,孫權在曹操大營饒了一圈,曹操看到孫權的膽量還有船上士兵器械嚴整,讚歎:「生子當如孫仲謀,劉景升兒子像豬狗」。隨後,孫權下令吹號回營。而《魏略》記載则是說孫權進了曹操軍陣地,曹操命部下拉箭亂射,孫權船身一則被箭矢射滿將要翻船,孫權隨即下令調轉船身擋箭,船身也因此得到平衡,孫權從容地回營。戰鬥已經一個月有餘,曹操仍然無法打敗孫權也無法攻克嚴防的濡須塢,孫權便寫信給曹操說春天水增,你快點走吧,並在另一封信寫上,你如果不死,我不安樂。孫權給曹操一個撤退的下台階,曹操收到信後,對左右說孫權不會騙我,隨即下令撤軍。

漢建安十九年(214年)五月,孙权亲征庐江治所皖城。闰月,在一天的时间内就攻破皖城,俘获庐江太守朱光及参军董和,男女数万人,将北线扩展至合肥一带。劉備得到益州,於是孙权派诸葛瑾向刘备讨还荆州各郡。刘备拒絕,並說得到涼州後再把荊州所有郡歸還(當時益州與涼州之間還有一個漢中,漢中當時是張魯的領地),孫權經過濡須和益州一事後知道劉備的推託假話,隨即派遣魯肅到益陽與關羽對峙,再派吕蒙指挥孫皎、潘璋、呂岱、鲜于丹、徐忠、孙规等领兵二万,攻取长沙、零陵、桂阳三郡。孙权住在陆口,为各路军队的指挥、调度。吕蒙军队一到,长沙、桂阳二郡全部归服,同時通過心理戰把零陵把太守誘至開城投降。最後魯肅和關羽在益陽交鋒對峙,以及談判磋商。這個時候,曹操準備攻取漢中,刘备害怕丢失益州,便派使者求和。两国因此停戰,于是以湘水為界,劉備被逼把长沙、桂阳兩個郡以東還給孫權。江夏郡是孫權攻破黃祖後一直領有並沒有借出,而長沙郡則是生前孫堅所管治的。曹操再伐吳,但是曹操大軍被甘寧100人奇襲而全軍撤退。

215年,孫權北征合肥。孙权作战勇敢,進軍時與數名部將作為先頭部隊率先到達戰場扎寨立營,因為大軍還沒有集結只有數名部將的軍隊,所以被張遼有機可乘突襲成功。而撤军时孫權亦親自與四名部將及1000人在後方穩定士氣,張遼見此率七千人偷襲,當時全軍撤出,兵力只有一千人不如張遼七千人,呂蒙、蔣欽、凌统、甘宁等在逍遥津以北被张辽所袭击,凌统等拼死保护孙权,孙权弓馬嫻熟迎擊張遼,最後骑着骏马飛躍津桥成功撤出。張遼在戰後對孫權的弓射騎術感歎,當魏軍知道這個弓騎勇將是孫權而悔恨沒有捉到他。

漢建安二十二年(216年)冬,曹操再次兴师伐吴,丹阳四郡(今安徽定量城)民帅尤突、费栈受曹操授權联合山越,聚集數萬人起兵反叛。孙权即命賀齊和陆逊进兵征讨。賀齊和陆逊大破尤突及费栈等眾,降服丹阳、吴郡、故鄣等三郡山越,得精兵数万人。曹操屯军至居巢(今安徽巢县东北)準備進軍。孫曹交戰,關羽聯絡長沙郡縣長吳碭、袁龍再次發起叛亂,孫權派鎮守陸口的魯肅前去幫助呂岱,最終平定了叛亂。217年,曹操進攻濡須口,孙权在濡須塢前方築城,但被曹操軍先鋒逼退。之後濡須战线胶着,連日暴雨水面上漲,孫權驅使水軍前進曹操軍非常惶恐,曹操下令撤退。孫權便以吕蒙为都督,與蔣欽共同擔任此戰的總指揮。呂蒙据守之前建成的城坞,并设置万张强弓硬弩,以拒曹操。结果曹军所有先鋒尚未安然立屯,便被吕蒙攻破了,曹操軍敗走退回到居巢,最後攻不下孫權而下令撤軍,曹操自己也引軍撤退。在217年濡須口之戰孫權擊退曹操之後,要著手處理揚州內部的山越問題、自己國家的利益、孫劉關係,所以對漢朝偽降主动与曹操修好,避免日後受到曹操、劉備、內部山越的三方面侵擾。曹操當時被孫權擊敗而引軍撤退,另一方劉備也進軍漢中,因此接受請降。魯肅非常後悔借出荊州給劉備,同時也怒斥劉備、關羽沒有信用,他死後呂蒙接替他在前線總指揮職務,並向孫權提出要警戒關羽,不依靠劉備獨立對抗曹操的建議,孫權經過多年獨立對抗曹操見識過劉備等人的反復態度,於是採納呂蒙提議,與關羽表面交好。

漢建安二十四年(219年),孫權有進攻合肥態勢,魏軍全部州郡的軍隊進入戒備狀態。孫權得知劉備獲得漢中後,再次派諸葛瑾向劉備索還荊州的訴求,但劉備拒絕。孫權打算與关羽以聯姻修好孫劉關係,但關羽以虎女怎能嫁犬子為由拒絕,並怒罵使者。關羽發兵圍攻襄阳曹仁,孫權打算派兵救援,關羽嫌孫權增援太慢大罵:「狢子(對東吳人的貶稱),等我滅了樊城之後回去就把你滅了。」。孫權知道關羽傲慢輕視自己,便寫信道歉。曹操派人聯絡孫權,以荊州為條件希望孫權相助,但孫權沒有馬上答應。後来關羽在樊城之戰俘虜于禁數萬降兵,把所有人送到南郡關押,但是還假借食糧不足為藉口,對吳國湘水邊境侵略而且搶奪軍需糧食。此前,孫權跟呂蒙分析局勢時,孫權想打徐州,但呂蒙認為應該打關羽的荊州,分析認為曹軍多為騎兵擅於陸戰,徐州雖然拿得下來,但也守不住。不如著手準備拿下荊州,完全控制整條長江,對外進可攻退可守,對內下游的吳國也會十分安全。此時孫權對關羽的種種作為已經難以忍受,命吕蒙至陆口實施之前商量好的計劃,並向漢朝廷申請討伐關羽,獻帝同意。十月,孙权西征关羽,以吕蒙陆逊为先锋,孫皎為殿後,孫權則潛軍一同北上。呂蒙以白衣渡江之策計,在夜半時分計破連綿不斷的烽火台屏障,然後再占据南郡,關羽被徐晃等人打敗從樊城回來,知道南郡丟失,隨後撤退到麥城駐守。孫權沒有打算殺關羽的意圖,於是派使者對關羽勸降,關羽假裝答應,在城上立旗後逃跑。孫權知道後派潘璋和朱然截擊,呂蒙當時留在南郡指揮大局,陆逊则另率军攻取宜都郡房陵等。呂蒙通過善政安撫荊州民心,把蜀漢軍家人的情況告訴給關羽軍,頓時間關羽部下失去戰意四散,關羽軍數萬人有的向孫軍投降,有的被孫軍的將軍吸納,最後在臨沮馬忠擒獲了關羽、關平等人。孫權想用關羽制衡曹劉打算再次招降關羽,左右文臣此時對主子孫權說狼子不可養,曹操當年收留關羽,如今換來遷都的惡果。孫權聽後把關羽斬首,並把首級送去給曹操,孫權則以諸侯的禮遇安葬關羽的身軀在當陽。自此荊州南北為曹、孫兩家佔有,于是孙权免除荆州百姓的所有租税。曹操向漢獻帝上表任命孙权为骠骑将军,假节兼任荆州牧,封南昌侯,同時也征召了张承、劉基等人。

漢建安二十五年(220年)年初,魏王曹操及吳大督呂蒙等名將相繼病故。11月,繼承王位的曹丕逼劉協禪讓,正式建號,是為魏文帝。孫權並沒有向曹丕投降,而是命都尉趙咨出使魏国承認曹丕的禪讓帝位並以諸侯身份向他稱臣,再將于禁等敗將送回北方,令新上任需要彰顯功名的魏帝曹丕的虛榮自負的心迅速膨脹,同時解除對孫權的戒心。孙权又派遣趙咨、陈化、冯熙、沈珩为使节,曹丕也派侍中辛毗、尚书桓阶前来东吴与之立誓结盟,曹丕冊封孙权为諸侯藩王吴王,以大将军使持节的身份监督交州,兼任荆州牧,孫權立长子孙登为王太子。當時,群臣勸孫權不應該受封吳王應該自稱九州伯、上將軍,孫權則說當年劉邦也是受封了項羽的漢王,最後還是成就了偉業。曹丕處事浮華,在他守喪期間向孫權索求雀頭香、大貝、明珠、象牙、犀角、玳瑁、孔雀、翡翠、鬥鴨、長鳴雞,吳群臣聽後說這些是珍稀貴重物勸孫權不要給,孫權則認為這只是瓦片石頭罷了,並不介意。期間在外交上一直由孫權主導,曹丕過於天真相信孫權而拒絕劉曄順江而下伐吳的建議,還几次拒绝了大臣们的伐蜀的提案。

劉備宣稱獻帝被害,於建安二十六年(221年)4月也登基稱帝。同年7月,借以關羽報仇的名義討伐孫權欲吞併江東領土發動猇亭之战,孫權自公安迁都鄂縣,改名武昌進入備戰,以六县设置武昌郡。孫權讓諸葛瑾寫信給劉備勸說他不要開戰希望和睦相處,並陳說利害分清楚敵人主次,不要上了曹魏的當,如果真要開戰他們也不會手軟,劉備不聽。孫權派周泰準備向白帝城作攻防姿態,任命陆逊为大都督,率领朱然、韓當、駱統、潘璋、孫桓等领兵前往抵抗。黃初三年(222年)六月,陆逊彻底击败蜀军。蜀军被斩杀和放下武器投降者有几万人。刘备被孫桓追至差點被擒獲,最後仅保得自身不死。當時,徐盛、潘璋、宋謙等人認為只要繼續追擊劉備,必能把他殺掉,但陸遜、朱然、駱統等認為不要追擊,他們察覺到曹丕有進攻江東的態勢。而孫權根據自己的判斷,採納陸遜等人的看法,下令不要對逃往白帝城方向的劉備展開追擊。

夷陵之戰期間,孫權一直在自己領地橫江屯兵提防曹魏,果然如陸遜等所料,曹丕派曹休襲擊孫權的領地曆陽及蕪湖,曹丕打算控制孫權,要求孫權將孫登送到魏國都城做人質,孫權知道其用意所以以藉口多次推辭,最後曹丕發覺孫權誠心不款,於是有意發兵攻打江東。夷陵之戰一結束,吴魏之间就開始有交戰態勢,曹丕派出三十萬大軍,命令曹休、张辽、臧霸出兵洞口,曹仁出兵濡须口,曹真、夏侯尚、张郃、徐晃率军围攻南郡。當時要處理揚州境內的山越問題,孙权故意示弱,謙卑上書誘騙曹丕,只是為了拖延曹丕進軍期限爭取平定山越內亂的時間,曹丕則相信他送兒子而沒有進軍,而另一方,孫權則派遣吕範、朱然、朱桓率領其他部將暗中部署。黃初三年(222年)十月,曹丕見孫權沒有送兒子來,正式開戰。於是孙权改年号为黄武,同曹魏斷絕來往。孫權命呂範率領徐盛、孫韶、全琮、賀齊等人在水路抵御曹休等,孫盛、诸葛瑾、潘璋、杨粲前往南郡增援朱然,朱桓接替周泰以濡须督的身份在濡須塢抵擋曹仁。三方面戰鬥中,曹仁被朱桓多個戰術配搭被打得大敗,曹休、張遼、臧霸則是強弩之末被徐盛、全琮、賀齊等人反擊而敗退,曹真、夏侯尚、徐晃、張郃則團團圍攻江陵,但久攻不下朱然。最終在次年三月春魏軍全部撤走,江南國境皆得安寧。另一邊,孫權收到在白帝城休養的劉備的求和信後,於十二月派遣太中大夫鄭泉出使蜀汉,蜀、吴两国自此重結盟好。

黃武二年(223年),孙权在江夏修筑山城。改用乾象历。夏四月,孙权的大臣们进劝他称帝,孙权不答应。此前,魏國令吳領地的戲口太守晉宗造反殺同僚王直,騷擾江南國境,孫權因為三方面大戰分身不下而不能馬上消滅他。六月孫權派賀齊、胡綜等人率軍平定,最終擒獲晉宗。劉備死後,諸葛亮派鄧芝向東吳再次確立聯盟關係,孫權知道諸葛亮用意,也非常器重鄧芝,所以答應修好。孫權便斷絕同曹魏來往,派辅义中郎将张温回訪蜀漢。

黃武五年(226年),孙权下令各州郡守,对百姓实行宽容安息政策。这时陆逊因驻守的地方缺粮,上表孙权,命令诸将广开农田。七月,孙权听说魏文帝曹丕去世,兴兵征讨江夏郡,围攻石阳城,卻久攻不下。江夏郡高城被孫奐攻陷。孙权任命全琮为东安郡太守,讨伐山越的反叛。孙权分交州另置广州,不久又复合为交州。

黃武七年(228年)五月,孙权命鄱阳太守周鲂以斷髮詐降,假装叛离东吴,引诱魏将曹休。爆发石亭之战,秋八月,孙权前往皖口,派征西将军陆逊率领朱桓、全琮在石亭大败曹休。

黃龍元年(229年)夏四月十三日丙申(5月23日),統治江東三十年的孙权在南郊正式登基为帝,改年号为黄龙。四月,孫權大赦改年,在南郊拜天,即皇帝位,諸葛亮派衛尉陳震去東吳祝賀孫權登皇帝位,3個月後孫權把國都從武昌遷回建業。追谥父亲孙坚为武烈皇帝,母亲吴氏为武烈皇后,長兄孙策为长沙桓王。立吴王太子孙登为皇太子。将军官吏都晋爵加赏。六月,蜀国派人前来庆贺孙权登基。孙权還禮,承認東西二帝共存,並与蜀漢使節商议平分天下。其中,豫、青、徐、幽四州属吴;兖、冀、并、凉四州歸蜀。司州的土地,以函谷关为界分属两国,雙方制定盟书,共同声讨曹叡。秋九月,孙权將都城從武昌遷到建業(今江蘇省南京市),就住在原来的府第中,不再另建新宫殿,征召上大将军陆逊辅佐太子孙登,掌管武昌事宜。

孙权即位後,曾多次派人出海。黃龍二年(230年),他派衛溫、諸葛直等航行到達夷洲;242年,他又派聶友等航行到珠崖儋耳(指現今的海南島)。

嘉禾元年(232年),孙权派遣将军周贺等航海到辽东。十二月,辽东太守公孙渊向孙权称藩。

嘉禾二年(233年),孙权派太常张弥、贺达等万人,带上金银财宝奇货异物,加上九锡,经海路送给公孙渊。举朝大臣全都规劝孙权,认为公孙渊其人不可信,对他的恩宠礼遇不要太过分。孙权一意孤行,没有接受规劝。后来公孙渊果然将张弥等杀死,以其首級並東吳賜予的金印送往曹魏邀功。孙权聞之,大感慚恨,企图亲自征讨公孙渊,尚书仆射薛综等极力谏阻,最终中止了这个计划。

嘉禾三年(234年)二月,諸葛亮再次與兵北伐。诸葛亮集中在漢中十萬大軍全部出動,木牛流馬,運糧不停,同時相約東吳東西並舉。五月,東吳出兵,七月退兵。孙权下诏放宽徭役,夏五月,孙权派遣陆逊、诸葛瑾等驻军江夏、沔口,派孙韶、张承等进军广陵、淮阳,孙权自己亲率大军进围合肥新城,爆发合肥新城之战后退兵回返,孙韶也停止进军广陵等地。秋八月,孙权任命诸葛恪为丹杨太守,讨伐山越部族。次年,孙权派吕岱领兵讨伐贼寇李桓等。

嘉禾六年(237年),孙权让群臣讨论奔丧立科、丞相顾雍奏请违法奔丧应处以死罪。此后吴县县令孟宗违法奔母丧归家,事后在武昌将自己拘禁起来听候处罚。陆逊向孙权说明孟宗的平时作为,并借机为孟宗求情,孙权于是给孟宗减刑一等,并申明下不为例,于是违法奔丧的事绝迹。

赤烏元年(238年),改年号为赤乌。当时,孙权利用吕壹打擊豪族,吕壹本性苛刻残忍,执法严酷。太子孙登屡次进谏,孙权都不采纳,大臣们于是都不敢进言。后来吕壹奸邪的罪行败露被处死,孙权自我批评,认错误,派中书郎袁礼代自己向曾經規勸但未被採納的大臣們致歉。

赤烏二年(239年),公孙渊不滿曹魏對其待遇不高,便又復叛魏國,自立为燕王,结果受到魏国司马懿攻击。公孙渊派遣使者向吴国求助。当时吴人都对公孙渊的反复无常历历在目,劝说孫權斩杀使者。唯有羊衜说:“陛下,斩首公孙渊的使者固然能讓您出口恶气,可这样做是出了匹夫的怒气,而放弃了霸王之计。臣以为,朝廷不如借此机会,出奇兵前往以观动静。如果魏国进攻公孙渊失败,那么我军远赴辽东解救,是恩结于远夷,义盖于万里;如果魏军和公孙渊相持不下,公孙渊首尾不能相顾,那我军正好进攻辽东,这样也足以让上天惩罚公孙逆贼,一雪往日之耻。”羊衜此言深得孙權赞许,于是派遣使者羊衜、郑胄、将军孙怡以海军前往辽东,击败魏国守将张持、高虑,並俘虜當地居民南還。十月,孙权派遣将军吕岱、唐咨前往剿滅少数民族的叛乱,将他们全部屠戮。

赤乌四年(241年),太子孙登去世。孙权後立三子孙和为太子。孙权听从百官封建诸子的意见,又立孙和之弟孙霸为鲁王。但孙霸始终不服孙和。遂召集手下宾客及结交诸大臣,常与围绕在孙和一侧的太子党分庭抗礼。赤乌十三年(250年),孙权决定废黜太子孙和并赐死鲁王孙霸,同时改立七子孙亮为皇太子。第二年册立孙亮之母潘氏为皇后。

太元元年(251年),冬十一月,孙权祭祀南郊回来后,就因風疾(相當於今稱中風)生病卧床。十二月,遣驿使传书召大将军诸葛恪回京,拜为太子太傅,孙权下诏省徭役、减征赋,将将国家大事交给诸葛恪管理,并修改诸多不便法令。

太元二年(252年),孙权立原太子孙和为南阳王。五子孙奋为齐王。六子孙休为琅琊王。二月,大赦,改年号为神凤。神凤元年四月廿六日(公元252年5月21日),孫權於太初宮内殿中驾崩,享壽六十九歲。滕胤与太子太傅诸葛恪、少傅孙弘、荡魏将军吕据、侍中孙峻等人一同受遗诏辅佐太子。孙权稱帝后在位23年。葬於建業蔣陵,謚大皇帝,廟號太祖。

孫權擔任家督弱冠繼承江南政權以来,對外招納人才培養部下,以懷柔策略籠絡不服從孫家的江南豪族,以白手興家統合內部對抗外壓,鞏固孫家政權在江南的地位,另一方面通過平定揚越叛亂進行強兵吸收老幼弱者補戶的政策,同時給予落後山越民提供漢文化的學習。諸侯時期的孫權不屬於任何一方,也沒有所謂的興漢滅漢的政治口號,故此可以根據時勢局勢發展,判斷哪一方有利用價值,並進行聯合的自由外交戰略獲取自己的利益。作為開創基業的帝王,孫權以出色的政治智慧及戰略判斷,深諳縱橫捭闔,最終締造一方霸業。赤壁之戰後加強控制江東,并将江東六郡扩展到揚、荆、交三州,積極開發南方的荒蕪之地,穩健控制中國東南。

孫權在數年間將國土政權安定,以適才適所為第一原則,而不以輩分、資歷、交情、名氣為優先,深知人無完人,故此不追究缺點而用其優點的用人風格,處事也是嚴罰主義者,就算親族或功臣的家族犯罪,也會給予嚴刑處分。例如選擇寒門出身的周泰為平虜將軍,與孫權為同窗的朱然则身居其下,在夷陵之戰任命资历尚浅的陸遜为大都督,許多人因是孫策舊將或者公室貴戚,一度有所不滿,但最终都心服口服,步骘虽然出身豪族,但是避難到江東而家道中落,最终竟做到丞相一职。孙权亦能主动培养部下,同時對待功臣的態度是忘其短而貴其長。孙权以顧雍為丞相而非眾人所推薦的張昭,就是因为丞相位置處理的事情多且繁重,而張昭性情剛烈固執,不遵從他的意見則會埋怨歸咎到底,到時反而對公事沒有益處。

孫權崇尚節儉,並效法大禹以卑宮為美,原本住的建業宮其實只是孙权早期的將軍府而已,一直住到赤烏十年建材腐朽,還詔令將武昌宮拆了,把木材運來建業修繕,但其實當時武昌宮也有二十八年的歷史不堪使用,這麼做的目的是節省木料避免妨礙農桑工作,由此也可知孫權對農業的重視。陆凯向孙皓劝谏时也称孙权时代“后宫列女,及诸织络,数不满百,米有畜积,货财有余”。

孫權執法嚴格,即使面對至親也是法律優先從不循私。孫輔因通敵而被流放、庶弟孫朗因違反軍令燒毀自軍軍用而被呂範送回,於是改姓丁並禁錮終身、愛子孫霸更因圖危太子,而被賜死。另一方面,孙瑜孙桓孙韶等孙氏宗室或委以重任,女则嫁于国家重臣,即使是谋反者的后代也能不计前嫌。可见孙权实际上对于同族亲戚相当重视。陈寿因此赞曰“况此诸孙,或赞兴初基,或镇据边陲,克堪厥任,不忝其荣者乎”。

对内廣納諫言,任用父兄旧部稳定局面,平定叛徒和山越,攻滅殺父仇人黃祖,吸納北方難民。在一片降曹之声时果斷与曹操一战並與刘备结盟,任用周瑜打败曹操穩定江南地盤,後來因為劉備一連串背離同盟關係的所作所為及荊州等问题而與蜀汉決裂,連本帶利奪回荊州並在夷陵之战重創劉備,最終確立了三分天下的局面。黄龙元年(229年),孫權于武昌称帝,建国为吴,孙吴建立。称帝以后他分部諸將,鎮撫山越,增設縣邑,編制戶籍,設置農官,推行軍屯與民屯;收容南遷移民,興修水利,增廣農田;親自下田採用牛耕,大幅度改良農業生產技術,大興佛教,奠定了六朝的經濟與文化基礎。

在晚年大批豪族過分插手孫權的家事,而且分黨立派,造成政局动荡不安,孫權對此非常不滿。之後孫權得知自己繼承人意向的消息外洩之後大為憤怒,並將相關人員等捉拿問罪。。後來孫權遭到豪族暗殺及背叛,所以在二宮之戰爆發後,通過部下彈劾而削弱豪族權力來鞏固政權,之後難能可貴的是孫權同時也具有認錯的勇氣,從陸遜之子陸抗的任職態度以及陸機所著《辯亡論》看來,孫陸二家情誼仍然十分深厚,陸氏對孫權亦持肯定態度,不因孫權老年冷酷而有所怨言。神凤元年(252年)夏四月,孫權在内殿驾崩,终年七十一岁。

由於孫權大力開拓海上事業並且開拓江南,因此在中國史上有非常重要的地位,然而他死後的待遇與他的功績完全不成正比,詩人曾極在其作品《吳大帝陵》中提到“四十帝中功第一,壞陵無主使人愁”,劉克莊也在《吳大帝廟》中嘆息“今人渾忘卻,江左是誰開”。

東吳的部曲私兵和世襲制度是有利有弊,執政者需要確立王朝威信的時候,有利於鞏固自己的君主地位;當執政者權威衰微的時候,威權容易被擁有私兵的部下奪取。因此孙权晚年後世評價兩極,一方認為他晚年昏庸而做出一連串錯誤導致王朝衰退;而一方則認為孫氏在江東的權力較弱,所以孫權晚年處事冷酷無情,通過制約擁有政治影響力及私兵權部曲世襲制的豪族,把權力集中在孫氏家族手上,避免如同魏國一樣被士族奪權蠶食政權的情況發生,從而確保孫氏政權現況及未來的威權地位,只是後來的權力者不能維繫這個政權發展而導致逐漸衰落。

北方戰亂,孫權也吸納南渡的北方民眾,其北方的手工技術也在江南得到發揚及應用。另外由於孫權積極擴張海上事業,並曾發兵遼東,因此江南造船業大大興盛。

首都建業原名秣陵,最初是一小縣,因孫權定都建業並開鑿運河而成為一流都市,被稱為六朝古都,現名南京。

《吳曆》曰,黃武四年,扶南諸外國來獻琉璃。這是中國最早與南海諸國交流的記載。孫權主動派出朱應與康泰出訪南海各國,先後到過林邑(今越南中南部)、扶南(今柬埔寨)、西南大海州(今南洋群島)、大秦(羅馬)、天竺(今印度),並記下各國物產以利貿易奠定了南海貿易的基礎,回國後,二人分別撰寫《扶南異物志》及《外國傳》(又稱《吳時外國傳》),之後繼續派出使節進行南國宣化,同扶南、林邑、堂明(今柬埔寨)建立關係。,這是史無前例的事情,雖然南海諸國之前已與中國有接觸,但是由官方政府主動派出官員積極尋求國際貿易,孫權卻是創舉,貿易的範圍甚至到達了羅馬,並在建業接見了羅馬商人秦論。

《江表传》中记载孙权方頤大口,眼神很有光彩。漢朝遣使者劉琬為孫策加錫命之時看見孫權,形容孫權的相貌高大挺拔。刘备和张辽都看到孙权坐着时显得很高,认为他躯干较长,中国民国学者黎东方分析,只有不需要站著伺候人,而是坐著讓人伺候的貴人才會是所謂軀幹長而雙腿短的外形,古代这被視為大貴之相,劉備被形容為手長過膝也是基於同樣的道理。

《三國演義》里孫權则被记载“碧眼紫髯,堂堂一表人才”。

在閻立本《十三帝王圖》之中,孫權為站姿,此為開國之君之意,身著的冕服應有天子十二章,在圖中有被畫出來的有「日、月、藻、火、黼」五章 ,其中日月為明,明火三章表示的是孫權振興經濟,教化百姓,讓其光明之面普照天下之意,藻則代表孫權稱帝隨時代順應天意而起,黼則表示孫權「能斷割」,這與三國志中孫權好俠養士仁而多斷的人格特質以及遇曹操來攻能拒絕臣服決心抗曹、遇劉備來攻則稱臣曹丕以保全江東等正確的重大決斷相呼應,圖中孫權手持麈尾扇,表現了他的帝王風度,為十三帝王圖中唯一持扇者,“麈”是領隊大鹿尾,魏晉以來,尚清談,手執麈尾有“領袖群倫”含意,藝文類聚亦記載司馬懿見諸葛亮乘素輿、葛巾毛扇指揮三軍,嘆諸葛亮為名士,諸葛亮在《三國演義》中也常持羽扇指揮軍隊,扇子有善戰之意,因此蘇軾在《念奴嬌》形容周瑜時亦說周瑜「羽扇綸巾」談笑間強虜灰飛煙滅,孫權在位期間,赤壁與夷陵之戰均以少勝多,甚至取荊州而兵不血刃,足見他用人正確調度有方的善戰特質,因此辛棄疾會說「天下英雄誰敵手?曹劉,生子當如孫仲謀」。

孙权性格旷达开朗,仁爱明断,喜欢供养贤才,因此很早就与父兄齐名。由于非常重视集体的力量,能毫无保留地信任臣下,甚至部下死後代為教養其孤兒贍養其妻儿及其父母。也會調解部屬糾紛,亦下诏勿杀叛逃将领的妻子子女。孙权与臣下的亲密关系也体现在称呼其表字上,甚至是对于初见的潘濬,曾與陸遜當眾對舞,又将自身所穿衣物皆赐之。对于他国贤才,孙权也毫不掩饰地表达喜爱,如诸葛亮费祎邓芝宗预等。孫盛因而稱許孫權盡心關愛部下,令其甘心為自己拼命,是東吳能夠立於江東的原因。

孙权天性活潑奔放,能言善辩,常常肆无忌惮地恶作剧、戏弄人,经常开些无关紧要的玩笑,即使是面对蜀汉来使也不例外。其本人亦参与配合部下的戏谑。

孫權的忍辱負重性格在向曹操與曹丕稱臣時一覽無遺。因此臥薪嘗膽一詞出自蘇軾的《擬孫權答曹操書》,也因為忍辱負重,所以孫權面對荊州問題時選擇與蜀國結盟,而不與劉備爭鬥,避免曹操坐收漁翁之利。三國之中,也是東吳最晚稱帝。陳壽亦提過「孫權屈身忍辱,任才尚計,有句踐之奇英,人之傑矣」,趙咨答曹丕時亦說「屈身於陛下,是其略也」。因孫權處世手段極其柔軟,曹丕也曾以嫵媚形容孫權,所以有詩歌詠孫權時說「孝廉嫵媚還能霸」。

孙权善于判断国内外人物局势。如认定魏延和杨仪会在诸葛亮死后内讧。也预见到曹魏亡国的先兆。

孙权擅长骑術和弓術,在合肥面對張遼的突襲能平安躍馬過橋,他的弓術也给张辽留下了很深的印象。

孙权有六口宝剑,分别是白虹、紫电、辟邪、流星、青冥、百里。

關於孫權嗜好,其中射猎(射虎、射雉)與好酒和开宴会派对尤其出名。每当猛兽近前,孙权总是以亲手击打为乐趣。宴会中常常对部下进行劝酒,孙权喜好冒险,如顶着大风天坐船出航,乘轻船去见曹操军队, 密令甘宁夜袭曹营等等。

孙权也喜爱读书,据其本人所言,其所涉猎内容涵盖《诗经》、《尚书》、《礼记》、《左传》、《国语》及三史(《史记》、《汉书》和《东观汉记》),惟不曾研读《周易》,孙权在书法上亦有成就,被认为擅长行书和草书。

在宗教方面,孙权早年信仰道术,与诸多方术之士交往甚密。主要人物为吴范、刘惇、赵达、姚光、介象等人。而被后世尊为道教天师的葛玄也与孙权有过交往。孙权也对当时的新宗教佛教非常开明,赤乌年间为高僧康僧会建立建初寺。

孫權因在家事上随心所欲,表现的不在乎上下尊卑而招致陳壽的批評,称其可比拟春秋时代的齐桓公,对外“有识士之明”,对内却“嫡庶不分,闺庭错乱”,最终在繼承人問題上埋下祸根,導致很长一段时间内国家都动荡不安。裴松之则意見相反,認為孫權廢掉無罪的太子,雖然是開啟禍亂的前兆,但最多只是東吳滅亡的次因而非主因,畢竟東吳滅亡已是孫權死後二十八年的事情,而且滅亡主因仍是暴君孫皓,即使孫權當時傳位於孫和,最後也是孫皓登基,國之滅亡的根本問題其實是出在為政者昏虐,並非只有孫權廢黜一事就能造成,如孫亮能保住國祚,或者孫休不早死,都不至於讓東吳滅亡。陸遜的孫子陸機更著有《辯亡論上》《辯亡論下》詳細說明東吳亡國非因蜀國滅亡,而是孫權死後的當政者用人不當。

陳壽於《三國志》认为孫策為開國奠基人但子孙未被封为王爵,孫權於義儉矣。后人据此穿凿附会,认定孙权对孙策有所怠慢,从尚书仆射存和胡综的上书可知孙权以谦虚为美德,不愿效仿汉代旧制过分尊崇皇族,就连孙权自己的皇子也不例外,如被孙权宠爱的次子孫慮也止在侯爵。另一爱子孫和在十九岁封为太子前也从未获得任何爵位。群臣请立孙权余下四子为王时也被孙权拒绝。

孫盛从国家大局方面对陈寿的看法也表示了不同意见,认为當時天下局勢尚未統一,宜正名定本貴賤疏邈,不宜給與孫策之子更高的權力與爵位製造內亂機會,此為穩定局勢之必要行為,況天倫篤愛,孫權既已將孫策宗廟立於建業,應不會刻意吝於給予地位,這明顯是為了穩定國家局勢的必要處置方式。

从实际史料出发,孙权反倒有相当多不忘旧情的举动,如孙盛所言为孙策建庙于建业并派太子祭祀。在赤乌年间再次为孙策进行厚葬,因吕范往日对其兄的帮助而对之大加溢美,以致严峻私下认为夸大其词了,直到后来才信服。

孫策臨終傳權時:「舉江東之眾,決機於兩陳(陣)之間,與天下爭衡,卿(孫權)不如我。舉賢任能,各盡其心,以保江東,我不如卿。」(《三國志·吳書·孫破虜討逆傳第一》)

曹操於濡須之戰:「生子當如孫仲謀,劉景升(劉表)兒子若豚犬耳!」(《三國志·吳書·吳主傳第二》裴松之註引《吳歷》)於孫權稱臣時「此兒欲踞吾著爐炭上邪!」(《晉書·宣帝紀第一》)

刘备:「孙车骑长上短下,其难为下,吾不可以再见之。」

关羽:「鰂子敢爾,如使樊城拔,吾不能滅汝邪!」(《三國志·蜀書·關張馬黃趙傳第六》)

周瑜:「將軍以神武雄才,兼仗父兄之烈,割據江東,地方數千里,兵精足用,英雄樂業,尚當橫行天下,為漢家除殘去穢。」(《三國志·吳書·周瑜魯肅吕蒙傳第九》)「今主人亲贤贵士,纳奇录异。」

魯肅:「将军神武命世。」「孫討虜聰明仁惠,敬賢禮士,江表英豪,咸歸附之」(《三國志·蜀書·先主傳第二》裴松之註引《江表傳》)

張紘:「自古帝王受天命的君主,雖有皇靈在上輔佐,文德傳播天下,也要靠武功顯著。要開墾種植,任賢使能,務崇寬惠,順天命去誅討,這樣不勞師眾定天下。」

陸遜:「陛下(孫權)以神武之姿,涎膺期運,破操(曹操)烏林,敗備(劉備)西陵,禽羽(關羽)荊州,斯三虜者當世雄傑,皆摧其鋒。」(《三國志·吳書·陸遜傳第十三》)

諸葛亮:「海內大亂,將軍(孫權)起兵據有江東,劉豫州亦收眾漢南,與曹操并爭天下。今操芟夷大難,略已平矣,遂破荊州,威震四海。英雄無所用武,故豫州遁逃至此。將軍量力而處之:若能以吳、越之眾與中國抗衡,不如早與之絕﹔若不能當,何不案兵束甲,北面而事之!今將軍外託服從之名,而內懷猶豫之計,事急而不斷,禍至無日矣!」(《三國志·蜀書·諸葛亮傳第五》)「權有僭逆之心久矣」(《三國志·蜀書·諸葛亮傳第五》裴松之註引《漢晉春秋》)「孫將軍可謂人主,然觀其度,能賢亮而不能盡亮,吾是以不留。」(《三國志·蜀書·諸葛亮傳第五》裴松之註引《袁子》)「孙权据有江东,已历三世,国险而民附,贤能为之用。」「议者咸以权利在鼎足,不能并力,且志望以满,无上岸之情,推此,皆似是而非也。何者?其智力不侔,故限江自保;权之不能越江,犹魏贼之不能渡汉,非力有馀而利不取也。」

司馬懿:「權之稱臣,天人之意也。」(《晉書·宣帝紀第一》)

张辽:「向有紫髯将军,长上短下,便马善射。」

程昱:「权有谋。」(《三国志·魏书 ·程郭董刘蒋刘传第十四》)

陈琳:「夫天道助顺,人道助信,事上之谓义,亲亲之谓仁。盛孝章,君也,而权诛之,孙辅,兄也,而权杀之。贼义残仁,莫斯为甚。乃神灵之逋罪,下民所同雠。辜雠之人,谓之凶贼。」(《檄吴将校部曲文》)

彭羕:「仆昔有事於诸侯,以为曹操暴虐,孙权无道,振威闇弱,其惟主公有霸王之器,可与兴业致治,故乃翻然有轻举之志。」(《三国志·卷四十·蜀书十·刘彭廖李刘魏杨传第十》)

趙咨:「聰明仁智,雄略之主也」、「納魯肅於凡品,是其聰也;拔呂蒙於行陳,是其明也;獲於禁而不害,是其仁也;取荊州而兵不血刃,是其智也;據三州虎視於天下,是其雄也;屈身於陛下(曹丕),是其略也。」(《三國志·吳書·吳主傳第二》)

贾诩:「孙权识虚实,陆议见兵势。据险守要,泛舟江湖,皆难卒谋也。用兵之道,先胜后战,量敌论将,故举无遗策。臣窃料群臣,无备、权对,雖以天威臨之,未見萬全之勢也。」(《三国志·魏书·荀彧荀攸贾诩传第十》)

邓芝:「大王命世之英。」

刘晔:「權無故求降,必內有急。權前襲殺關羽,取荊州四郡,備怒,必大興師伐之。外有強寇,眾心不安,又恐中國承其釁而伐之,故委地求降,一以卻中國之兵,二則假中國之援,以強其眾而疑敵人。權善用兵,見策知變,其計必出於此。」、「權雖有雄才,故漢驃騎將軍南昌侯耳,官輕勢卑。士民有畏中國心,不可強迫與成所謀也。不得已受其降,可進其將軍號,封十萬戶侯,不可即以為王也。夫王位,去天子一階耳,其禮秩服御相亂也。彼直為侯,江南士民未有君臣之義也。我信其偽降,就封殖之,崇其位號,定其君臣,是為虎傅翼也。權既受王位,卻蜀兵之後,外盡禮事中國,使其國內皆聞之,內為無禮以怒陛下。」(《三國志·魏書·程郭董劉蔣劉傳第十四》)

冯熙:「吴王体量聪明,善于任使。赋政施役,每事必咨。教养宾旅,亲贤爱士。赏不择怨仇,而罚必加有罪。臣下皆感恩怀德,惟忠与义。带甲百万,谷帛如山。稻田沃野,民无饥岁。所谓金城汤池,强富之国也。」

刘基:「大王以能容贤蓄众,故海内望风。」

钟繇:「顾念孙权,了更妩媚。」(《三國志·魏書·鍾繇華歆王朗傳第十三》)

刘琬:「吾观孙氏兄弟虽各才秀明达,然皆禄祚不终,惟中弟孝廉,形貌奇伟,骨体不恒,有大贵之表,年又最寿,尔试识之。」

陳壽:「孫權屈身忍辱,任才尚計,有勾踐之奇,英人之傑矣。故能自擅江表,成鼎峙之業。然性多嫌忌,果於殺戮,暨臻末年,彌以滋甚。至於讒說殄行,胤嗣廢斃,豈所謂賜厥孫謀以燕冀於者哉?其後葉陵遲,遂致覆國,未必不由此也。」(《三國志·吳書·吳主傳第二》)「割據江東,策之基兆也,而權尊祟未至,子止侯爵,於義儉矣。」(《三國志·吳書·孫破虜討逆傳第一》)

陆凯:「自昔先帝时,后宫列女,及诸织络,数不满百,米有畜积,货财有余。先帝崩后,幼、景在位,更改奢侈,不蹈先迹。」(《三国志·吴书·潘濬陆凯传第十六》)

孙楚:「吴之先主,起自荆州,遭时扰攘,播潜江表,刘备震惧,逃迹巴岷,遂依丘陵积石之固,三江五湖,浩汗无涯,假气游魂,迄于四纪,二邦合从,东西唱和,卒相扇动,拒捍中国。」

陸機:「吳桓王基之以武,太祖(孫權)成之以德,聰明睿達,懿度深遠矣。其求賢如不及,恤民如稚子,接士盡盛德之容,親仁罄丹府之愛。拔呂蒙於戎行,識潘濬於系虜。推誠信士,不恤人之我欺;量能授器,不患權之我逼。執鞭鞠躬,以重陸公之威;悉委武衛,以濟周瑜之師。卑宮菲食,以豐功臣之賞;披懷虛己,以納謨士之算。故魯肅一面而自讬,士燮蒙險而效命。高張公之德而省游田之娛,賢諸葛之言而割情欲之歡,感陸公之規而除刑政之煩,奇劉基之議而作三爵之誓,屏氣跼蹐以伺子明之疾,分滋損甘以育凌統之孤,登壇慷慨歸魯肅之功,削投惡言信子瑜之節。是以忠臣競盡其謀,志士鹹得肆力,洪規遠略,固不厭夫區區者也。故百官苟合,庶務未遑。」(《辯亡論》下)「用集我大皇帝,以奇踪袭於逸轨,叡心发乎令图,从政咨於故实,播宪稽乎遗风,而加之以笃固,申之以节俭,畴咨俊茂,好谋善断,东帛旅於丘园,旌命交于涂巷。故豪彦寻声而响臻,志士希光而影骛,异人辐辏,猛士如林。於是张昭为师傅,周瑜、陆公(陆逊)、鲁肃、吕蒙之畴入为腹心,出作股肱;甘宁、凌统、程普、贺齐、朱桓、朱然之徒奋其威,韩当、潘璋、黄盖、蒋钦、周泰之属宣其力;风雅则诸葛瑾、张承、步骘以声名光国,政事则顾雍、潘濬、吕范、吕岱以器任干职,奇伟则虞翻、陆绩、张温、张惇以讽议举正,奉使则赵咨、沈珩以敏达延誉,术数则吴范、赵达以禨祥协德,董袭、陈武杀身以卫主,骆统、刘基强谏以补过,谋无遗算,举不失策。故遂割据山川,跨制荆、吴,而与天下争衡矣。」(《辯亡論》上)

华谭:「赖先主承运,雄谋天挺,尚内倚慈母仁明之教,外杖子布廷争之忠,又有诸葛、顾、步、张、朱、陆、全之族,故能鞭笞百越,称制南州。」。「吴武烈父子皆以英杰之才,继承大业。今以陈敏凶狡,七弟顽冗,欲蹑桓王之高踪,蹈大皇之绝轨,远度诸贤,犹当未许也。」

裴松之:「孙权横废无罪之子,为兆乱。」「权愎谏违众,信渊意了,非有攻伐之规,重复之虑。宣达锡命,乃用万人,是何不爱其民,昏虐之甚乎?此役也,非惟闇塞,实为无道。」

孙盛:「盛闻国将兴,听於民;国将亡,听於神。权年老志衰,谗臣在侧,废适立庶,以妾为妻,可谓多凉德矣。而伪设符命,求福妖邪,将亡之兆,不亦显乎!」「观孙权之养士也,倾心竭思,以求其死力,泣周泰之夷,殉陈武之妾,请吕蒙之命,育凌统之孤,卑曲苦志,如此之勤也。是故虽令德无闻,仁泽(内)著,而能屈强荆吴,僭拟年岁者,抑有由也。然霸王之道,期於大者远者,是以先王建德义之基,恢信顺之宇,制经略之纲,明贵贱之序,易简而其亲可久,体全而其功可大,岂委璅近务,邀利於当年哉?语曰“虽小道,必有可观者焉,致远恐泥”,其是之谓乎!」

虞溥:「性度弘朗,仁而多断,好侠养士,始有知名,侔于父兄矣。」(《三國志·吳書·吳主傳第二》裴松之註引《江表傳》)

《荆州先德传》:“权好嘲戏以观人。”

王勃:「孙仲谋承父兄之余事,委瑜肃之良图,泣周泰之痍,请吕蒙之命,惜休穆之才不加其罪,贤子布之谏而造其门。用能南开交趾,驱五岭之卒;东届海隅,兼百越之众。地方五千里,带甲数十万。」

朱敬则:「孙仲谋藉父兄之资,负江海之固,未敢争盟上国,竞鹿中原,自守未馀,何足言也。」(《全唐文》)

徐夤:「一主参差六十年,父兄犹庆授孙权。不迎曹操真长策,终谢张昭见硕贤。建业龙盘虽可贵,武昌鱼味亦何偏。秦嬴谩作东游计,紫气黄旗岂偶然。」

司马光:「文帝承父兄之烈,师友忠贤,以成前志,赤壁之役,决策定虑,以摧大敌,非明而有勇能如是乎?奄有荆扬,薄于南海,传祚累世,宜矣。」(《历代名贤确论·卷五十七》)

苏轼:「亲射虎,看孙郎。」(《江城子·密州出猎》)「孙权勇而有谋,此不可以声势恐喝取也。」

苏辙:「吴大帝方其属任贤将,抗衡中原,曹公惮之。及其老也,贤臣死亡略尽,喜诸葛恪之劲悍,越众而付以后事。闼其用兵劳民之后,继起大役,兵折于外,既归而不能自克,将复肆志于僚友。恪既以丧其躯,而孙氏因之三世绝统,吴、越之民陷于炮烙之地,国随以亡。彼以进取之资用进取之臣,以徼一时之功可耳,至于托六尺之孤,寄千里之命,而亦属之斯人,其势必至是哉。」(《栾城后集·孙仲谋》)「今夫曹操、孙权、刘备,此三人者,皆知以其才相取,而未知以不才取人也。世之言者曰:孙不如曹,而刘不如孙。」

謝采伯:「孫權運籌於內,劉備、諸葛亮、周瑜、關侯等,合謀並智,方拒得曹操,敗之於赤壁,亦未為竒政縁。」

何去非:「权之勇决进取,无以逮其父兄,然审机察变,持保江东,于权有焉。」(《何博士备论》)

辛弃疾:「千古江山,英雄无觅,孙仲谋处。」(《永遇乐·京口北固亭怀古》)「何处望神州,满眼风光北固楼,千古兴亡多少事,悠悠。不尽长江滚滚流。 年少万兜鍪,坐断东南战未休,天下英雄谁敌手,曹刘。生子当如孙仲谋。」

吕祖谦:「孙权起于江东,拓境荆楚,北图襄阳,西图巴、蜀而不得。北敌曹操、西敌刘备,二人皆天下英雄。所用将帅,亦一时之杰。权左右胜之而后能定其国。及权国既定,曹公已死,丕、叡继世,中原有可图之衅。权之名将死丧且尽,权亦老矣。」(《吴论》)

晁补之:「吴人轻而无谋,自古记之矣。孙坚、孙策皆无王霸器。虽赖周瑜、鲁肃辈辅权嗣立,亦权稍持重,故卒建吴国也。」(《鸡肋集》)

萧常:「权承父兄之资,勇而有谋,愤曹操窃国,尝有讨贼之志;乌林之捷,亦一时之隽功。其后关羽围襄阳,降于禁,威振北方,操大惧,欲徙都以避之。权于是时,诚能与羽协力、东西并举,则操可图而汉室可兴。今乃不然,反袭杀羽以媚曹氏,不能少降意于帝室之胄,而甘心臣贼,昭烈之不能混一区夏,由此故也。他日虽有犄角之功,亦无及矣。吁,惜哉!」(《萧氏续后汉书》)

叶适:「权有地数千里,立国数十年,以力战为强,以独任为能。残民以逞,终无毫髪爱利之意,身死而其后不复振,操术使之然也。」(《习学记言·读吴志》)

元好问:「孙郎矫矫人中龙,顾盼叱咤生云风。」

郝经:「東漢之衰,孫權承父兄之烈,尊禮英賢,撫納豪右,誅黄祖,走曹操,襲關侯,遂奄有荆颺,今年出濡須,明年戰合肥,嶷然勢常北向,而以守爲攻,稱臣於魏,結援於漢,始忍勾踐之辱,終爲熊通之譖,保據江淮,奄征南海,卒與漢魏鼎峙而立,先起而後亡,非惟智勇足抗衡,亦國勢便利然也。」(《續後漢書》)

胡三省:「當方面者,當如呂岱;委人以方面者,當如孫權。」(《資治通鑒注》)

朱元璋:「君臣之间,以敬为主。敬者,礼之本也。故礼立而上下之分定,分定而名正,名正而天下治矣。孙权盖不知此,轻与臣下戏狎,狎其臣而亵其父,失君臣之礼。」(《明太祖宝训》)

罗贯中在《三國演義》有詩贊曰:「紫髯碧眼號英雄,能使臣僚肯盡忠,二十四年興大業,龍磐虎踞在江東。」

孙承恩:「仲谋强明,委任才智。听言能断,业乃鼎峙。倍义负汉,屈身事曹。传世四君,霸图亦消。」(《文简集·卷三十八》)

王夫之:「于是而知先主之知人而能任,不及仲谋远矣。」「于子瑜也、陆逊也、顾雍也、张昭也,委任之不如先主之于公,而信之也笃,岂不贤哉?」(《宋论·卷一·太祖》)

王懋竑:「至权时,张昭、张紘虽见尊礼而不复任用,昭且几不免,而翻竟以窜死,惟顾雍、潘濬辈从容讽议,得安有位。陆逊有大功,而以数直谏愤恚而卒。周瑜、鲁肃幸已早死,不与陆逊同祸,而亦恩不及嗣。有所爱重者,惟吕蒙、凌统、甘宁、周泰辈,以视策万万不逮矣。其保有江东者,以有吕蒙辈为之用,得其死力,而其不能廓大基业,窥中原者,亦以此。」(《三国志集解》)

赵翼:「至孙氏兄弟之用人,亦自有不可及者。」「以人主而自悔其过,开诚告语如此,其谁不感泣?使操当此,早挟一‘宁我负人,勿人负我’之见,而老羞成怒矣!此孙氏兄弟之用人,所谓以意气相感也。」

王鸣盛:「孙权称臣事魏已久,及黄武元年春大破蜀,刘备奔走,势愈强盛,则魏欲与盟而不受,九月魏兵来征,又卑辞上书求自改悔,乞寄命交州乃随,又改年临江拒守,彼此互有杀伤,不分胜负。十二月又通聘于蜀,乃既和于蜀,又不绝于魏,且业已改元而仍称吴王。五年令曰北虏缩窜,方外无事,乃益务农亩,称帝之举,直隐忍以至魏明帝太和三年,而后发,反覆倾危,惟利是视,用柔胜刚,阴谋狡猾,史评以勾践相比,非虚语也。」(《三国志集解》)

何焯:「老悖昏惑,吴亡不待皓而决。」

李慈铭:「三国时,魏既屡兴大狱,吴孙皓之残刑以逞,所诛名臣,如贺邵、王蕃、楼玄等尤多。少帝之诛诸葛恪、滕胤,皆逆臣专制,又当别论。惟大帝号称贤主,而太子和被废之际,群臣以直谏受诛者,如吾粲、朱据、张休、屈晃、张纯等十数人,被流者顾谭、顾承、姚信等又数人,而陈正、陈象至加族诛,吁,何其酷哉!自是宫闱之衅,未有至此者也。」(《越缦堂读书笔记》)

蔡东藩:「黄祖本无才智,而孙坚死于祖手;孙策又不能亲复父仇,命为之,势为之也。坚阻于命,策限于势;至权承父兄之业,用瑜蒙诸将,一出再出,方举黄祖而枭夷之,春秋之义大复仇,如孙仲谋者,其固不愧为令子乎?曹操谓生子至如孙仲谋,若刘景升诸儿,与豚犬等,原非虚言。」「孙权承父兄遗业,任才尚计,史谓其有勾践遗风,乃内宠相寻,晚年益愦,废长立幼,乱本已成。」(《後漢演義》)

盧弼:「竊謂有勾踐之志則可,無勾踐之志則終爲奴虜而已,南宋其已事也。仲謀操縱其間,以江東而抗衡大國承祚,方之勾踐其信然矣。」(《三國志集解》)

柏杨:「孙权是中国历史上最可爱最有人情味的皇帝之一。」

李宗吾:「他和刘备同盟,并且是郎舅之亲,忽然夺取荆州,把关羽杀了,心之黑,仿佛曹操,无奈黑不到底,跟著向蜀请和,其黑的程度,就要比曹操稍逊一点;他与曹操比肩称雄,抗不相下,忽然在曹丞相驾下称臣,脸皮之厚,仿佛刘备,无奈厚不到底,跟著与魏绝交,其厚的程度也比刘备稍逊一点。他虽是黑不如操,厚不如备,却是二者兼备,也不能不算是一个英雄。」

毛泽东:「孙权是个很能干的人。」「当今惜无孙仲谋。」(《毛泽东读古书实录》)

\subsubsection{黄武}

\begin{longtable}{|>{\centering\scriptsize}m{2em}|>{\centering\scriptsize}m{1.3em}|>{\centering}m{8.8em}|}
  % \caption{秦王政}\
  \toprule
  \SimHei \normalsize 年数 & \SimHei \scriptsize 公元 & \SimHei 大事件 \tabularnewline
  % \midrule
  \endfirsthead
  \toprule
  \SimHei \normalsize 年数 & \SimHei \scriptsize 公元 & \SimHei 大事件 \tabularnewline
  \midrule
  \endhead
  \midrule
  元年 & 222 & \tabularnewline\hline
  二年 & 223 & \tabularnewline\hline
  三年 & 224 & \tabularnewline\hline
  四年 & 225 & \tabularnewline\hline
  五年 & 226 & \tabularnewline\hline
  六年 & 227 & \tabularnewline\hline
  七年 & 228 & \tabularnewline\hline
  八年 & 229 & \tabularnewline
  \bottomrule
\end{longtable}

\subsubsection{黄龙}

\begin{longtable}{|>{\centering\scriptsize}m{2em}|>{\centering\scriptsize}m{1.3em}|>{\centering}m{8.8em}|}
  % \caption{秦王政}\
  \toprule
  \SimHei \normalsize 年数 & \SimHei \scriptsize 公元 & \SimHei 大事件 \tabularnewline
  % \midrule
  \endfirsthead
  \toprule
  \SimHei \normalsize 年数 & \SimHei \scriptsize 公元 & \SimHei 大事件 \tabularnewline
  \midrule
  \endhead
  \midrule
  元年 & 229 & \tabularnewline\hline
  二年 & 230 & \tabularnewline\hline
  三年 & 231 & \tabularnewline
  \bottomrule
\end{longtable}

\subsubsection{嘉禾}

\begin{longtable}{|>{\centering\scriptsize}m{2em}|>{\centering\scriptsize}m{1.3em}|>{\centering}m{8.8em}|}
  % \caption{秦王政}\
  \toprule
  \SimHei \normalsize 年数 & \SimHei \scriptsize 公元 & \SimHei 大事件 \tabularnewline
  % \midrule
  \endfirsthead
  \toprule
  \SimHei \normalsize 年数 & \SimHei \scriptsize 公元 & \SimHei 大事件 \tabularnewline
  \midrule
  \endhead
  \midrule
  元年 & 232 & \tabularnewline\hline
  二年 & 233 & \tabularnewline\hline
  三年 & 234 & \tabularnewline\hline
  四年 & 235 & \tabularnewline\hline
  五年 & 236 & \tabularnewline\hline
  六年 & 237 & \tabularnewline\hline
  七年 & 238 & \tabularnewline
  \bottomrule
\end{longtable}

\subsubsection{赤乌}

\begin{longtable}{|>{\centering\scriptsize}m{2em}|>{\centering\scriptsize}m{1.3em}|>{\centering}m{8.8em}|}
  % \caption{秦王政}\
  \toprule
  \SimHei \normalsize 年数 & \SimHei \scriptsize 公元 & \SimHei 大事件 \tabularnewline
  % \midrule
  \endfirsthead
  \toprule
  \SimHei \normalsize 年数 & \SimHei \scriptsize 公元 & \SimHei 大事件 \tabularnewline
  \midrule
  \endhead
  \midrule
  元年 & 238 & \tabularnewline\hline
  二年 & 239 & \tabularnewline\hline
  三年 & 240 & \tabularnewline\hline
  四年 & 241 & \tabularnewline\hline
  五年 & 242 & \tabularnewline\hline
  六年 & 243 & \tabularnewline\hline
  七年 & 244 & \tabularnewline\hline
  八年 & 245 & \tabularnewline\hline
  九年 & 246 & \tabularnewline\hline
  十年 & 247 & \tabularnewline\hline
  十一年 & 248 & \tabularnewline\hline
  十二年 & 249 & \tabularnewline\hline
  十三年 & 250 & \tabularnewline\hline
  十四年 & 251 & \tabularnewline
  \bottomrule
\end{longtable}

\subsubsection{太元}

\begin{longtable}{|>{\centering\scriptsize}m{2em}|>{\centering\scriptsize}m{1.3em}|>{\centering}m{8.8em}|}
  % \caption{秦王政}\
  \toprule
  \SimHei \normalsize 年数 & \SimHei \scriptsize 公元 & \SimHei 大事件 \tabularnewline
  % \midrule
  \endfirsthead
  \toprule
  \SimHei \normalsize 年数 & \SimHei \scriptsize 公元 & \SimHei 大事件 \tabularnewline
  \midrule
  \endhead
  \midrule
  元年 & 251 & \tabularnewline\hline
  二年 & 252 & \tabularnewline
  \bottomrule
\end{longtable}

\subsubsection{神凤}

\begin{longtable}{|>{\centering\scriptsize}m{2em}|>{\centering\scriptsize}m{1.3em}|>{\centering}m{8.8em}|}
  % \caption{秦王政}\
  \toprule
  \SimHei \normalsize 年数 & \SimHei \scriptsize 公元 & \SimHei 大事件 \tabularnewline
  % \midrule
  \endfirsthead
  \toprule
  \SimHei \normalsize 年数 & \SimHei \scriptsize 公元 & \SimHei 大事件 \tabularnewline
  \midrule
  \endhead
  \midrule
  元年 & 252 & \tabularnewline
  \bottomrule
\end{longtable}


%%% Local Variables:
%%% mode: latex
%%% TeX-engine: xetex
%%% TeX-master: "../../Main"
%%% End:

%% -*- coding: utf-8 -*-
%% Time-stamp: <Chen Wang: 2021-11-01 11:36:41>

\subsection{会稽王孫亮\tiny(252-258)}

\subsubsection{生平}

孫亮(243年-260年),字子明,是中國三國時代吳國的第二代君主,在位六年(252年-258年),後世史書多稱之為吳廢帝、會稽王。

孫亮生于赤乌六年(243年),是吳大帝孫權的幼子,因此特别受到疼爱。孙亮出生的时候,他的長兄孫登、二兄孫慮早已去世。当时的皇太子为三兄孫和,后来孙和被陷害廢去太子之位。於是赤乌十三年(250年)孫權便立孫亮為皇太子,不久又立其母潘淑为皇后。

潘皇后於神凤元年(252年)被宫女所杀,同年孫權也去世,孫亮繼位,時為四月廿八日丁酉(5月23日)。

孫亮登基时年方十岁,却聪明伶俐,受到大臣的尊敬。孫亮曾欲喫酸梅,讓黃門到庫裏去取蜂蜜,蜜中有鼠屎;就召來守庫官詢問,守庫官叩頭謝罪。少帝說:“黃門從你那兒要過蜂蜜嗎?”守庫官說:“曾要過,我沒敢給他。”黃門不服。少帝讓人破開鼠屎,屎中是乾燥的,於是他大笑著對左右說:“如果鼠屎事先就在蜜中,那麽裏外都應是濕的,現在外面濕而裏面乾燥,這必定是黃門放進去的。”詰問黃門,他果然服了罪。左右之人都很震驚恐懼。

孫亮即位之初,諸葛恪、滕胤、孙峻、吕据受顾命之托輔政孙亮(孙弘本来也是顾命大臣之一,因夺权失败而被诸葛恪先行杀害),又有旧臣吕岱、丁奉等人。曹魏乘孙权驾崩之际,于建兴元年(252年)11月发动东兴之战,结果却被太傅諸葛恪為統帥的吴军大败而归。第二年諸葛恪依仗顾命之托,不顾众臣劝阻欲乘胜出兵北伐魏國,但最後因瘟疫而失敗。

铩羽而归后的诸葛恪显得愈发刚愎自用,最终招致万民所怨、众口所嫌。建兴二年(253年),孫峻利用这个机会说服孙亮,于是在宴会上發動政變,殺死諸葛恪。孫峻因功出任丞相。

孫峻为人骄矜险害,动辄使用重刑,因此招致不少人的不满,但最後反对他的人均事敗被迫自殺或處死。他与滕胤、吕据两位顾命大臣虽然关系谈不上友好,但还能够一起融洽的共事,朝廷高层因此平静了一段时间。

255年(五凤二年),孫峻帶兵與魏國於淮河一帶交戰獲勝,魏將文欽投降。次年,孫峻派遣呂據等將領進攻魏國,但孫峻在戰爭期間病逝,由從弟孫綝接掌權力。呂岱亦於是年去世。因孙綝本不是大帝所指定的顾命大臣,呂據、文欽对孙綝完全继承孫峻权力一事非常不满,要求封滕胤為丞相。孫綝沒有理會他們的訴求,改封滕胤为大司马。于是滕胤和呂據发动政變,却反遭孙綝所殺,自此五位顾命大臣已经全部亡去。另一位將領王惇密謀殺死孫綝,亦事敗被殺。

257年(太平二年),孫亮親政,他对孙綝轻视自己的态度感到非常厌恶,於是推行多項措施(如訓練少年軍)以準備推翻他。同年,魏國的諸葛誕在壽春發動叛亂,把兒子諸葛靚送到吳國做人質。孫綝派兵協助諸葛誕但最終失敗。孙綝将失败的缘由归于大都督朱异并在镬里杀害了他,其他一些參戰的將領也因為怕被孫綝殺死而投降了魏國。孙綝返回建业後,得知孙亮对他有所戒备,内心也很恐惧,于是称病不上朝并命自己兄弟把守宫门以求自保。

258年(太平三年),孫亮因孙綝不听自己指挥进军并擅杀朱异等事对孙綝不满到了极点,于是与全尚,全公主,刘承等人密谋除掉孙綝。但消息被孙綝的从姐(全尚之妻)或从外甥女(全皇后)泄露给孙綝。孙綝获悉密报后,于6月26日率先包围皇宫,以孙亮患有精神病为由强迫众臣同意将孫亮廢為會稽王,改立孫休為帝。和孫亮一起策劃政變的大臣都被孫綝殺死。群臣也因为畏惧孙綝的声势不敢多言。

260年,孫亮的封地會稽傳出謠言,說孫亮將返回建業復辟;而孫亮的侍從亦聲稱孫亮在祭祀時口出惡言。

經審判後,孫亮再被貶為侯官侯(侯官,今福建省閩侯縣)和流放,途中死去。據《三國志》記載,孫亮可能是自殺,也可能是被孫休派人毒死的。孫亮死時只有18歲。

吴国灭亡后,吴国的少府卿丹阳人戴显上表朝廷,于是迎回孙亮遗体安葬赖乡(今江苏省南京市溧水区)

\subsubsection{建兴}

\begin{longtable}{|>{\centering\scriptsize}m{2em}|>{\centering\scriptsize}m{1.3em}|>{\centering}m{8.8em}|}
  % \caption{秦王政}\
  \toprule
  \SimHei \normalsize 年数 & \SimHei \scriptsize 公元 & \SimHei 大事件 \tabularnewline
  % \midrule
  \endfirsthead
  \toprule
  \SimHei \normalsize 年数 & \SimHei \scriptsize 公元 & \SimHei 大事件 \tabularnewline
  \midrule
  \endhead
  \midrule
  元年 & 252 & \tabularnewline\hline
  二年 & 253 & \tabularnewline
  \bottomrule
\end{longtable}

\subsubsection{五凤}

\begin{longtable}{|>{\centering\scriptsize}m{2em}|>{\centering\scriptsize}m{1.3em}|>{\centering}m{8.8em}|}
  % \caption{秦王政}\
  \toprule
  \SimHei \normalsize 年数 & \SimHei \scriptsize 公元 & \SimHei 大事件 \tabularnewline
  % \midrule
  \endfirsthead
  \toprule
  \SimHei \normalsize 年数 & \SimHei \scriptsize 公元 & \SimHei 大事件 \tabularnewline
  \midrule
  \endhead
  \midrule
  元年 & 254 & \tabularnewline\hline
  二年 & 255 & \tabularnewline\hline
  三年 & 256 & \tabularnewline
  \bottomrule
\end{longtable}

\subsubsection{太平}

\begin{longtable}{|>{\centering\scriptsize}m{2em}|>{\centering\scriptsize}m{1.3em}|>{\centering}m{8.8em}|}
  % \caption{秦王政}\
  \toprule
  \SimHei \normalsize 年数 & \SimHei \scriptsize 公元 & \SimHei 大事件 \tabularnewline
  % \midrule
  \endfirsthead
  \toprule
  \SimHei \normalsize 年数 & \SimHei \scriptsize 公元 & \SimHei 大事件 \tabularnewline
  \midrule
  \endhead
  \midrule
  元年 & 256 & \tabularnewline\hline
  二年 & 257 & \tabularnewline\hline
  三年 & 258 & \tabularnewline
  \bottomrule
\end{longtable}


%%% Local Variables:
%%% mode: latex
%%% TeX-engine: xetex
%%% TeX-master: "../../Main"
%%% End:

%% -*- coding: utf-8 -*-
%% Time-stamp: <Chen Wang: 2021-11-01 11:36:46>

\subsection{景帝孫休\tiny(258-264)}

\subsubsection{生平}

吴景帝孫休(235年-264年9月3日),字子烈,為孫權第六子,在父親孫權、弟孫亮後繼任為吳國第三任皇帝,在位六年。

孙休生于嘉禾四年(235年),母王夫人,13岁时,跟随谢慈和盛冲就学。

太元二年(252年)受封為琅琊王,居於虎林,當時十八歲。同年四月,大帝因患風疾病死於建業,孫休的嫡弟孫亮繼位,由太傅諸葛恪秉政,諸葛恪不欲諸王在濱江兵馬之地,遂徙孫休至丹楊郡。其後又因丹楊太守李衡數次以事侵擾孫休,孫休上書乞求徙往其他郡,孫亮遂下詔徙孫休至會稽郡。

孙休的岳母和姐姐孙鲁育被权臣孙峻冤杀,孙休害怕,将妻子朱氏送回建业,执手泣别。朱氏到建业,被孙峻遣回。

太平三年九月廿六日(258年11月9日),宗室孫綝發動政變,罷黜孫亮為會稽王,立孫休為帝,孫休三次辭讓而受,改元永安,封孫綝為丞相,孫綝五兄弟皆封侯掌禁军,權傾朝野,時為十月十八日己卯(11月30日)。

孙休先假意麻痹孙綝,将举报孙綝谋反之人,交給孙綝处置,后又加孙綝弟孙恩为侍中分其权,年末设宴请孙綝,孙綝称病不赴,孙休十多次派人去请,孙綝终于赴宴,席间想借故早退,被丁奉等擒住,孙休历数孫綝罪状斩之,灭其三族。孙綝的权臣地位继承自其堂兄权臣孙峻,孙休又将孙峻棺材削薄后重新下葬,并将孙峻、孙綝开除宗籍,称之为“故峻”“故綝”,并赦免被孙峻、孙綝所害之人。

孫休在位期間,以衛將軍濮陽興為丞相,廷尉丁密、光祿勳孟宗為左右御史大夫。布典宮省,興關軍國。

孫休崇尚文化。永安元年創設國學,置學官,立五經博士,選送吏中及將吏子弟好學者就學,为南京太学之滥觞,韋昭為首任博士祭酒。

武功方面,无甚建树,曾图先统一南方。永安七年(264年)二月,趁蜀中無主,西征巴蜀,以鎮軍将军陸抗、撫軍将军步協、征西將軍留平、建平太守盛曼,率大軍圍蜀巴東守將羅憲。魏使將軍胡烈率步騎二萬侵擾西陵,以救羅憲,陸抗等遂引軍退回吳國,最终丝毫未能夺取蜀汉故地。

同年七月,孫休病重,不能語,尚能書;同月廿四日(9月2日)大赦天下,但次日(9月3日)以三十歲英年早逝。丞相濮陽興、左將軍張布遊說朱皇后,因蜀國初亡,而交阯攜叛,國內震懼,希望立長君,所以意欲孙休亡兄孫和之子孫皓嗣位。

孫皓繼位後,於元興元年(264年)十一月,誅殺濮陽興及張布。又於甘露元年(265年)七月,逼殺孫休之妻景皇后朱氏,只於苑中小屋治喪,又送孫休四子於吳小城,再復追殺年長的孫{\fzk 𩅦}及孫{\fzk 𩃙}。

\subsubsection{永安}

\begin{longtable}{|>{\centering\scriptsize}m{2em}|>{\centering\scriptsize}m{1.3em}|>{\centering}m{8.8em}|}
  % \caption{秦王政}\
  \toprule
  \SimHei \normalsize 年数 & \SimHei \scriptsize 公元 & \SimHei 大事件 \tabularnewline
  % \midrule
  \endfirsthead
  \toprule
  \SimHei \normalsize 年数 & \SimHei \scriptsize 公元 & \SimHei 大事件 \tabularnewline
  \midrule
  \endhead
  \midrule
  元年 & 258 & \tabularnewline\hline
  二年 & 259 & \tabularnewline\hline
  三年 & 260 & \tabularnewline\hline
  四年 & 261 & \tabularnewline\hline
  五年 & 262 & \tabularnewline\hline
  六年 & 263 & \tabularnewline\hline
  七年 & 264 & \tabularnewline
  \bottomrule
\end{longtable}



%%% Local Variables:
%%% mode: latex
%%% TeX-engine: xetex
%%% TeX-master: "../../Main"
%%% End:

%% -*- coding: utf-8 -*-
%% Time-stamp: <Chen Wang: 2019-12-18 13:05:26>

\subsection{末帝\tiny(264-280)}

\subsubsection{生平}

吴末帝孙皓(243年-284年),字元宗,幼名彭祖,又字皓宗,《三国志》原名为孫晧。為廢太子孫和之子,吳大帝孫權之孫,在位十七年(264年—280年),是三国時期孫吴的第四位,同時也是最後一位皇帝。

吳景帝孙休逝世時,太子非常年幼。因當時吳國處於內憂外患之中,大臣們便合議改立較年長的孫皓即位。孫皓即位後,初期雖然英明施政並多行善舉,在西陵之戰一度挽回吳國的厄運,但中後期實行暴政並過度役使民力,加深了亡國危機。最終,吳國於280年被西晉征服,三國時代也因此終結。

孫皓並無廟號與謚號,後世史書中多將孫皓稱為吳後主、吳末帝,也有用他即位前的封號烏程侯,或是歸晉後的封號歸命侯來指代他。

孫皓出生於赤烏六年(243年),是吳主孫權三子孫和的長子。孫皓的嫡母張妃是張承的女兒,生母何姬是孫和的一名庶妃。在他出生的同一年,孫和被立為太子,直到赤烏十三年(250年)因陷入“二宮之爭”被孫權廢黜,流放到故鄣(今浙江省安吉縣)。太元二年(252年)正月,孫權又將孫和封為南陽王,孫和帶家眷移居封地長沙(郡治今湖南省長沙縣)。不久後,孫權逝世,由十歲的幼子孫亮即位。

孫亮即位後,張妃的舅舅諸葛恪秉持朝政。建興二年(253年),宗室孫峻誅殺諸葛恪後,借故民間有傳言稱諸葛恪想迎孫和即位,而剝奪了孫和的王位,並將孫和流放到新都(治今浙江省淳安縣),隨後賜死,張妃也一同死去。此時的孫皓年僅十二歲,還有三個異母弟弟,何姬為了將孫皓等人撫養長大而保全了性命。太平三年(258年),孫亮被繼承孫峻權力的孫綝廢黜,孫權的六子孫休被立為帝。

孫休即位後,將孫皓封為烏程侯,命孫皓前往封地烏程(今浙江省湖州市)。他的異母弟孫德、孫謙也分別被封為錢塘侯、永安侯。在當烏程侯期間,孫皓與烏程令萬彧相識,彼此交好。永安七年(264年,魏咸熙元年)七月,孫休托孤於丞相濮陽興後逝世。在孫休死後,濮陽興並未遵從他的意愿立太子孫𩅦為帝。當時吳國的盟國蜀漢已經滅亡,交趾一帶又發生了叛亂,大臣們考慮著擁立一位較年長的君主。已升任為左典軍的萬彧便向濮陽興和另一位權臣張布推薦孫皓,稱孫皓英明果斷,有長沙桓王孫策的風範,并且行事遵守法度。濮陽興與張布被萬彧說服,便一起勸朱太后將孫皓迎立為帝。這一年,孫皓23歲。

統治前期(264-268年):孫皓即位後,採取了一系列的舉措來鞏固自己的地位。一方面,他大行封賞,將迎立有功的丞相濮陽興,加封侍中,兼領青州牧,左將軍張布升為驃騎將軍,加封侍中,又把吳國宿將施績、丁奉升為左、右大司馬,以拉攏臣子。另一方面,他發放糧食,救濟窮人,從皇宮放出大量侍女讓她們可以婚配,並放歸宮中圈養的一些野獸,以一系列惠民政策來爭取民心。當時人們都把他稱為明主。

但一段時間後,治國有成、志得意滿的孫皓便顯露出魯莽暴躁、驕傲自滿、迷信以及好酒色的一面。此外,他還將景帝孫休的妻子朱太后貶為景皇后,追謚自己的父親孫和為文皇帝,將自己的生母何姬奉為太后,妻子滕氏立為皇后,將孫休的太子及其它三個兒子封為王爵,以加強自己繼位的合法性。當初擁立他的濮陽興、張布對孫皓的轉變感到震驚和失望,結果被萬彧秘密向孫皓揭發,孫皓將兩人處斬,並夷三族。此時,距離兩人迎立孫皓才過了四個月。之後,孫皓扶植外戚,將滕皇后的父親滕牧和何太后的弟子何洪、何蔣、何植都封為侯。甘露元年(265年,魏咸熙二年,晉泰始元年)七月,孫皓迫使前太后朱氏自殺,又軟禁了孫休的四個兒子,並殺死了其中較年長的兩人。九月,孫皓聽信術士之言(“荊州有王氣,當破揚州”),又為了防禦司馬氏軍事包夾的迫切需要,決定遷都武昌(今湖北省鄂州市)。這一年十二月,繼承司馬昭權力的司馬炎迫使曹魏禪讓,正式建立了晉王朝。

寶鼎元年(266年,晉泰始二年),出使晉國的使臣丁忠回到武昌。孫皓召集群臣宴會,因常侍王蕃酒醉失態而大怒,雖然有滕牧、留平等重臣出面為王蕃求情,但孫皓依然下令將王蕃斬首。孫皓的這一舉動令大臣們感到震驚遺憾,賀邵、陸抗後來在270年代上疏勸諫時,都有舉此事為例來指責孫皓,重臣陸凱(同年任左丞相)更是在上疏中將王蕃比為吳國的關龍逢,隱含有將孫皓比為暴君夏桀之意。當時,晉國因為才吞併蜀地不久,有意與吳國暫時維持和平。但使臣丁忠發現晉國戰備有其漏洞,勸說孫皓攻取弋阳郡(郡治今河南省潢川縣西),遭到時任鎮西大將軍的陸凱堅決反對,孫皓表面上贊同了陸凱的意見,並未出兵,但最後還是與晉國絕交了。八月,孫皓分置左、右丞相,左丞相由陸凱擔任,右丞相則安排自己的親信萬彧擔任。

起初,孫皓遷都武昌後,因土地貧乏,而孫皓施政不當處漸多,所需的供給大多要從長江下游運上來,使江東百姓頗有不滿,兒童再度傳唱孫權定都武昌時的歌謠:「寧飲建業水,不食武昌魚,寧還建業死,不止武昌居」。十月,永安山民施旦聚眾數千人起義,劫持孫皓的異母弟永安侯孫謙後向吳故都建業(今江蘇省南京市)進發,一路上不斷有人加入,到建業城外時已有數萬人之多,但還是被吳將丁固、諸葛靚擊潰,孫謙被救回。孫皓當初聽說施旦謀反的消息後,不僅不擔憂,反倒覺得這是應驗了之前術士說的“荊州有王氣,當破揚州”一事,肯定了自己遷都的決斷,命令數百人到建業城大喊“天子使荊州兵來破揚州賊”,來壓制之前的晦氣。丁固請示皓如何處理孫謙,孫皓下令將孫謙母子一起毒殺,後來他還殺了亡父孙和的嫡子即自己的另一个异母弟孫俊。十二月時,孫皓將都城遷回了建業。

寶鼎二年(267年,晉泰始三年)夏六月,孫皓下令新建更大的宮殿──昭明宮。為了昭明宮的修建,呂秩二千石以下的官吏都被派往山中督伐木料,昭明宮的修建歷時半年,工程耗資巨大,而且耽誤了農時。當時陸凱、華覈等人上疏勸止,孫皓拒絕聽從。

統治中期(268-272年):寶鼎三年(268年,晉泰始四年),孫皓開始向晉國發起攻擊。這一年,他親率大軍屯駐東關(今安徽省含山縣西南),令左大司馬施績攻江夏(今湖北省雲夢縣南),右丞相萬彧攻襄陽(今湖北省襄陽市),右大司馬丁奉、右將軍諸葛靚進攻合肥(今安徽省合肥市西),交州刺史劉俊、前部督脩則、將軍顧容等率攻擊投降晉國的交阯(郡治今越南北寧市)叛軍,但都沒有取得成功。北伐大軍被司馬望大軍所拒,兩路主力施績、丁奉分別為晉將胡烈、司馬駿所敗,而南征交阯軍隊更是被晉將楊稷大敗,劉俊、脩則戰死,顧容率殘軍退守合浦(郡治今廣西省合浦縣東北)。

建衡元年(269年,晉泰始五年),孫皓派監軍虞汜、威南將軍薛珝、蒼梧太守陶璜從荊州出發,監軍李勖、督軍徐存從建安海路出發,令兩軍在合浦會合共同剿滅交阯叛軍。此外,還派遣右大司馬丁奉再次北征,攻打谷陽(今安徽省靈壁縣)。但到了建衡二年(270年,晉泰始六年),丁奉部在渦口(今安徽省懷遠縣)一帶被晉將牽弘擊退,李勖部以道路不通為由,殺死向導馮斐後率軍無功而返。孫皓為此大怒,丁奉的向導被處死,李勖更是在被何定揭發後,同徐存被全家誅殺。不久後,何定率領五千人馬到夏口(今湖北省武漢市)打獵,吳宗室前將軍、夏口督孫秀害怕是孫皓令何定來抓自己,提前帶領家送眷數百人投奔晉國。晉武帝拜孫秀為驃騎將軍,儀同三司,封會稽公,禮遇備至。

建衡三年(271年,晉泰始七年)正月,孫皓聽信刁玄增改的讖文(“黃旗紫蓋,見於東南,終有天下者,荊、揚之君!”),認為自己是天命所歸,不顧眾人反對,用車載著自己的母親、妻子、孩子以及後宮上千人,親率大軍從牛渚(今安徽省當塗縣)西進伐晉。晉軍派司馬望率軍駐屯在壽春(今安徽省壽縣)作為防備。結果孫皓的軍隊途中被大雪所阻,士兵忍受天寒地凍的同時還要負責拉孫皓的車隊,都難以忍受這樣的勞苦,軍中漸漸出現倒戈的傳言,因此孫皓只好下令還師。孫皓還師前,右丞相萬彧與右大司馬丁奉、左將軍留平曾私下商議先自行回去,後被孫皓得知,雖然心懷不滿,但介於三人都是老臣並沒有馬上處置。當年,丁奉病逝。翌年,孫皓試圖用毒酒毒死萬彧和留平,二人卻都倖免未死,但不久後,萬彧自殺,留平愁悶而死。

在這兩年間裡,孫吳在軍事上接連取得了重大勝利,使得孫皓的自傲心大幅膨脹。先是在建衡三年(271年,晉泰始七年),南征的薛珝、虞汜、陶璜攻破交阯,擒殺晉軍守將,並收復了九真(郡治今越南清化市)、日南(郡治今越南洞海市南)兩郡,後又平定了扶嚴夷,使持續多年的交阯之亂暫告停歇。接著於鳳凰元年(272年,晉泰始八年)秋八月,陸抗成功討伐了因擔心被孫皓加害而叛投西晉的西陵督步闡,不僅成功收復了戰略要地西陵(今湖北省宜昌市),將步闡等人夷三族,並且擊退了由名將羊祜率領的五萬大軍,圍殲晉將楊肇的三萬援軍。西陵大捷之後,孫皓因為兩年內成功收復失土及大敗西晉,越發自志得意滿,更加相信自己是有上天相助,還召術士尚廣為他占卜看是否能取得天下,占卜的結果顯示他將在庚子年“青蓋入洛陽”。孫皓非常高興,從此專門謀劃統一大業,頻繁派遣軍隊襲擊晉國邊境,但都勞而無功。陸抗上疏反對孫皓的窮兵黷武,希望孫皓看清晉強吳弱的事實,建議“蹔息進取小規,以畜士民之力,觀釁伺隙”,又上疏指出西陵、建平戰略地位的重要,請求加強兩地的兵力。建平太守吾彥也憑借從長江上游漂下的大量木屑,斷定晉國將從巴蜀由水路大舉伐吳,上書孫皓請求加強防備。但孫皓不僅沒有重視這些意見,反而在鳳凰三年(274年,晉泰始十年)陸抗病逝後,將他的兵馬一分為五,交給陸抗的五個兒子分別統領。

統治後期(272-279年):軍事上取得耀人成果的同時,吳國內部卻越發不穩定。

自建衡元年(269年,晉泰始五年)左丞相陸凱病逝後,左大司馬施績、右大司馬丁奉、司空孟仁、右丞相萬彧、左將軍留平、太尉范慎、司徒丁固、大司馬陸抗等重臣在六年時間裡先後逝世,吳國有名望的舊臣死亡殆盡。當孫皓忌憚、尊重的重臣都不復存在之後,他的施政也更加殘暴,對於其他忠臣的勸諫也就不再接納容忍。大約從272年開始,他每次召集群臣宴會,都要故意讓每個人都喝得大醉,讓人在邊上專門檢舉他們的過失。甚至剝人臉皮,挖人眼珠。272年後孫皓對勸諫忠臣的容忍度也大幅下降,不惜痛下殺手以杜絕煩人的諫言:大司農樓玄因為多次直諫忤逆孫皓,被流放廣州,服毒而死;中書令賀邵也因直諫而使孫皓痛恨,當賀邵因中風不能說話,被孫皓懷疑是裝病,拷打致死;侍中韋昭因多次堅持己見,被以不聽從詔命為由處死;東觀令華覈多次上書勸諫,結果為了一些小事被免官遣返;豫章太守張俊因為給孫奮的母親掃墓,而被孫皓處以車裂極刑,並夷三族;會稽太守車浚因為開倉賑濟飢民,被懷疑收買人心而處斬;湘東太守張詠因為征稅不足,被孫皓派人斬殺;尚書熊睦對孫皓稍加勸阻,就被孫皓派人用刀環生生打死;甚至連他曾經寵信的何定、陳聲、張俶也被他處決,其中張俶受車裂之刑,陳聲更是被鋸斷頭顱而死。

相比於272年後孫皓殘暴的高壓政策,晉國都督荊州諸軍事的羊祜則對吳國展開懷柔政策。天璽元年(276年,晉泰始十二年),繼孫秀、步闡之後,吳國又一位重要將領——吳國宗室武衛將軍、京下督孫楷叛投晉國,在此前後,平虜將軍孟泰、偏將軍王嗣、威北將軍嚴聰、揚威將軍嚴整、偏將軍朱買、邵凱、夏祥、昭武將軍劉翻、厲武將軍祖始也都紛紛向晉軍投降。但孫皓絲毫沒有感受到危機的來臨。在吳國在接下来的几年裡,各地奉承他的人爭相獻上有吉祥象徵的事物,讓迷信的孫皓始終堅信自己將一統天下。

天紀三年(279年,晉泰始五年),郭馬攻殺廣州督虞授,在廣州發起叛亂。孫皓派遣滕脩、陶濬、陶璜率軍剿滅郭馬叛軍。冬十一月,晉武帝司馬炎依羊祜生前擬制的計劃,令鎮軍將軍司馬伷、安東將軍王渾、建威將軍王戎、平南將軍胡奮、鎮南大將軍杜預、龍驤將軍王濬、巴東監軍唐彬等分六路大舉伐吳。天紀四年(280年,晉泰始六年)正月,杜預、王渾兩軍分別向江陵(今湖北省荊州市)、橫江(今安徽省和縣)進軍,接連進克吳軍要塞。王渾部率先攻克尋陽(今湖北省黃梅縣西南)、賴鄉等城,屯兵橫江,距建業僅百里之遙。二月,在王濬、唐彬部和杜預部、胡奮部、王戎部的攻擊下,荊州的軍事重鎮丹陽(今湖北省秭歸縣)、西陵、荊門(今湖北省宜昌市東南)、夷道(今湖北省宜都市)、樂鄉(今湖北省松滋縣東)、江陵、江安(今湖北省公安縣)、夏口、武昌等先後失守,吳軍僅戰死或投降的都督、監軍就有十四人,牙門將、郡守一級的將領更是有一百二十多人,荊南各郡望風而降。

三月,由丞相張悌率領的吳軍精銳在版橋(今安徽省和縣)被王渾部擊潰,張悌、孫震、沈瑩全部戰死。孫皓自知滅亡在即,在給舅舅何植的信中自責道:“天匪亡吳,孤所招也。瞑目黃壤,當複何顏見四帝乎!”不久後,何植也向王渾軍投降。這時,王濬率水軍從武昌順流而下,直取建業。吳主孫皓派遣張象率水軍一萬餘人前往抵擋,但王濬大軍一到,張象便立即投降。孫皓又另遣陶濬率軍兩萬迎敵,結果士兵全部連夜逃竄。孫皓周圍數百人又請求他殺死寵臣岑昬,他不得以而被迫答應。後來,孫皓聽從光祿勛薛瑩和中書令胡沖的計策,分別遣送使節向王濬、司馬伷、王渾請降,試圖分化晉軍,未能奏效。三月壬寅日(280年5月1日),王濬率大軍進入石頭城,孫皓率太子孫瑾、魯王孫虔等二十一人出降,全家被遣送至洛陽。吳國至此滅亡。晉武帝下詔封孫皓為歸命侯。孫皓決定投降後,為了讓晉軍順利接收各地,廣發勸降書信給臣僚,信中寫道:“孤以不德,忝继先轨。处位历年,政教凶悖,遂令百姓久困涂炭,至使一朝归命有道,社稷倾覆,宗庙无主,惭愧山积,没有余罪。自惟空薄,过偷尊号,才琐质秽,任重王公,故《周易》有折鼎之诫,诗人有彼其之讥。自居宫室。仍抱笃疾,计有不足,思虑失中,多所荒替。边侧小人,因生酷虐,虐毒横流,忠顺被害。闇昧不觉,寻其壅蔽,孤负诸君,事已难图,覆水不可收也。今大晋平治四海,劳心务于擢贤,诚是英俊展节之秋也。管仲极雠,桓公用之,良、平去楚,入为汉臣,舍乱就理,非不忠也。莫以移朝改朔,用损厥志。嘉勖休尚,爱敬动静。夫复何言,投笔而已!”

五月,孫皓到達洛陽後,得到了晉武帝較為優厚的待遇。後來晉武帝大會群臣時,召孫皓進見,孫皓上前叩首請罪,晉武帝對孫皓說:“朕設此座以待卿久矣。”孫皓回應道:“臣於南方,亦設此座以待陛下。”賈充故意刁難他說:“聞君在南方鑿人目,剝人臉皮,此何等刑也?”孫皓回答說:“人臣有弒其君及姦回不忠者,則加此刑耳。”賈充曾指使手下殺害魏帝曹髦,聽了孫皓的話後羞愧不已,而孫皓本人則面不改色。晉武帝曾与王济下棋,孙皓在旁边,晉武帝对孙皓说:“何以好剥人面皮?”孙皓回应道:“见无礼于君者则剥之。”王济当时把脚伸到了棋盘下,因而孙皓讥讽王济。晋武帝有一次问孙皓:“闻南人好作《尔汝歌》,颇能为不?”孙皓正饮酒,于是举羽觞吟诵:“昔与汝为邻,今为汝做臣;上汝一杯酒,令汝寿万春!”孙皓直呼晋武帝为汝,晋武帝感到后悔,自讨没趣。太康四年十二月(284年初),孫皓在洛陽逝世,享年四十二歲,葬於河南縣界。

孫皓是中國歷史上有名的暴君,生性多疑且殘暴,設立諸多酷刑。他曾殺死或流放多名重要宗室,如殺害再从兄弟孫奉,流放从兄弟孙基、孙壹,誅殺五叔孫奮及其五子,殺死異母弟孫謙、孫俊等。對大臣,他也常常施以重刑,僅丞相一級的官員為例:除張悌在亡國之際戰死外,濮陽興被流放處死,夷三族,萬彧被譴自殺,全家遭流放;陸凱死後數年,全家被處以流放。。此外,孙皓非常迷信,常憑借運曆、望氣、卜筮、讖緯之類的原因來決定如遷都、用兵、皇后廢立等重大事件,並因此一直堅信自己將統一天下。

孫皓富有才氣,能吟詩,並有一定書法造詣。他曾在宴會上應晉武帝之邀,當場作《爾汝歌》一首:“昔與汝為鄰,今與汝為臣。上汝一杯酒,令汝壽萬春。”

同絕大多數亡國之君一樣,對孫皓的評價以負面評價為主。從孫皓受到的各種評價來看,他是一個典型的暴君形象,吳國重臣陸凱、陸抗在給孫皓的上疏中,曾多次暗示他堪比夏桀、商紂。曾擔任吳國光祿勛的薛瑩稱在孫皓當政時“昵近小人,刑罰妄加,大臣大將無所親信,人人憂恐,各不自安”。而孫皓被晉軍俘虜後,也批判自己“虐毒横流,忠顺被害。闇昧不觉,寻其壅蔽”,要求未降的吳軍儘快投降。此外,有些吳國人士曾對孫皓做出正面評價,如吳將吾彥在歸降西晉之後,在晉武帝面前力讚孫皓的英明:「吳主英俊,宰輔賢明」;而孫皓早期的好友萬彧也稱讚孫皓好學不倦、英明果斷,有孫策的風範。

在敵國眼中,272年後孫皓因殘暴所導致的吳國內部不安,已經促成了進攻吳國的最佳時機,晉臣張華曾對晉武帝說:“吳主荒淫驕虐,誅殺賢能,當今討之,可不勞而定”。晉將羊祜更是在上疏中稱:“孫皓暴虐已甚,於今可不戰而克。若皓不幸而沒,吳人更立令主,雖有百萬之眾,長江未可窺也,將為後患矣!”認為如果孫皓死掉,伐吳的難度將會大大增加。晉代史臣陳壽、孫盛在批評孫皓的暴虐以外,還指責晉武帝對孫皓的處置太過寬厚,認為像孫皓這樣的“肆行殘暴,忠諫者誅,讒諛者進,虐用其民,窮淫極侈”的禍國殃民之君,“梟首素旗,猶不足以謝冤魂”,“宜腰首分離,以謝百姓”。

同時,關於孫皓也有一些其它方面的評價,如晉臣秦秀在為滅吳的王濬請功時,曾言及孫皓對晉國用兵給晉國帶來的心理威脅,稱“以孫皓之虛名,足以驚動諸夏,每一小出,雖聖心知其垂亡,然中國輒懷惶怖。”後世的李世民則從另一個角度出發,將孫皓前期“權施恩惠之風”與王莽稱帝前“偽行仁義之道”相提並論,認為兩人的失敗就在於“有始無終”,以此得出二人都迅速覆亡的結論。唐代的朱敬則在批評孫皓的同時,還對他的權謀和才華予以一定程度的肯定。孫皓的書跡流傳到唐代,庾肩吾在他所著《書品》中则把孙皓评为「中中」,与曹操杜预并列,称“魏帝(曹操)筆墨雄贍、呉主(孫皓)體裁綿密”,書法家韋續则把他的行隸,評為「下中」品。

\subsubsection{元兴}

\begin{longtable}{|>{\centering\scriptsize}m{2em}|>{\centering\scriptsize}m{1.3em}|>{\centering}m{8.8em}|}
  % \caption{秦王政}\
  \toprule
  \SimHei \normalsize 年数 & \SimHei \scriptsize 公元 & \SimHei 大事件 \tabularnewline
  % \midrule
  \endfirsthead
  \toprule
  \SimHei \normalsize 年数 & \SimHei \scriptsize 公元 & \SimHei 大事件 \tabularnewline
  \midrule
  \endhead
  \midrule
  元年 & 264 & \tabularnewline\hline
  二年 & 265 & \tabularnewline
  \bottomrule
\end{longtable}


\subsubsection{甘露}

\begin{longtable}{|>{\centering\scriptsize}m{2em}|>{\centering\scriptsize}m{1.3em}|>{\centering}m{8.8em}|}
  % \caption{秦王政}\
  \toprule
  \SimHei \normalsize 年数 & \SimHei \scriptsize 公元 & \SimHei 大事件 \tabularnewline
  % \midrule
  \endfirsthead
  \toprule
  \SimHei \normalsize 年数 & \SimHei \scriptsize 公元 & \SimHei 大事件 \tabularnewline
  \midrule
  \endhead
  \midrule
  元年 & 265 & \tabularnewline\hline
  二年 & 266 & \tabularnewline
  \bottomrule
\end{longtable}

\subsubsection{宝鼎}

\begin{longtable}{|>{\centering\scriptsize}m{2em}|>{\centering\scriptsize}m{1.3em}|>{\centering}m{8.8em}|}
  % \caption{秦王政}\
  \toprule
  \SimHei \normalsize 年数 & \SimHei \scriptsize 公元 & \SimHei 大事件 \tabularnewline
  % \midrule
  \endfirsthead
  \toprule
  \SimHei \normalsize 年数 & \SimHei \scriptsize 公元 & \SimHei 大事件 \tabularnewline
  \midrule
  \endhead
  \midrule
  元年 & 266 & \tabularnewline\hline
  二年 & 267 & \tabularnewline\hline
  三年 & 268 & \tabularnewline\hline
  四年 & 269 & \tabularnewline
  \bottomrule
\end{longtable}

\subsubsection{建衡}

\begin{longtable}{|>{\centering\scriptsize}m{2em}|>{\centering\scriptsize}m{1.3em}|>{\centering}m{8.8em}|}
  % \caption{秦王政}\
  \toprule
  \SimHei \normalsize 年数 & \SimHei \scriptsize 公元 & \SimHei 大事件 \tabularnewline
  % \midrule
  \endfirsthead
  \toprule
  \SimHei \normalsize 年数 & \SimHei \scriptsize 公元 & \SimHei 大事件 \tabularnewline
  \midrule
  \endhead
  \midrule
  元年 & 269 & \tabularnewline\hline
  二年 & 270 & \tabularnewline\hline
  三年 & 271 & \tabularnewline
  \bottomrule
\end{longtable}

\subsubsection{凤凰}

\begin{longtable}{|>{\centering\scriptsize}m{2em}|>{\centering\scriptsize}m{1.3em}|>{\centering}m{8.8em}|}
  % \caption{秦王政}\
  \toprule
  \SimHei \normalsize 年数 & \SimHei \scriptsize 公元 & \SimHei 大事件 \tabularnewline
  % \midrule
  \endfirsthead
  \toprule
  \SimHei \normalsize 年数 & \SimHei \scriptsize 公元 & \SimHei 大事件 \tabularnewline
  \midrule
  \endhead
  \midrule
  元年 & 272 & \tabularnewline\hline
  二年 & 273 & \tabularnewline\hline
  三年 & 274 & \tabularnewline
  \bottomrule
\end{longtable}

\subsubsection{天册}

\begin{longtable}{|>{\centering\scriptsize}m{2em}|>{\centering\scriptsize}m{1.3em}|>{\centering}m{8.8em}|}
  % \caption{秦王政}\
  \toprule
  \SimHei \normalsize 年数 & \SimHei \scriptsize 公元 & \SimHei 大事件 \tabularnewline
  % \midrule
  \endfirsthead
  \toprule
  \SimHei \normalsize 年数 & \SimHei \scriptsize 公元 & \SimHei 大事件 \tabularnewline
  \midrule
  \endhead
  \midrule
  元年 & 275 & \tabularnewline\hline
  二年 & 276 & \tabularnewline
  \bottomrule
\end{longtable}

\subsubsection{天玺}

\begin{longtable}{|>{\centering\scriptsize}m{2em}|>{\centering\scriptsize}m{1.3em}|>{\centering}m{8.8em}|}
  % \caption{秦王政}\
  \toprule
  \SimHei \normalsize 年数 & \SimHei \scriptsize 公元 & \SimHei 大事件 \tabularnewline
  % \midrule
  \endfirsthead
  \toprule
  \SimHei \normalsize 年数 & \SimHei \scriptsize 公元 & \SimHei 大事件 \tabularnewline
  \midrule
  \endhead
  \midrule
  元年 & 276 & \tabularnewline
  \bottomrule
\end{longtable}

\subsubsection{天纪}

\begin{longtable}{|>{\centering\scriptsize}m{2em}|>{\centering\scriptsize}m{1.3em}|>{\centering}m{8.8em}|}
  % \caption{秦王政}\
  \toprule
  \SimHei \normalsize 年数 & \SimHei \scriptsize 公元 & \SimHei 大事件 \tabularnewline
  % \midrule
  \endfirsthead
  \toprule
  \SimHei \normalsize 年数 & \SimHei \scriptsize 公元 & \SimHei 大事件 \tabularnewline
  \midrule
  \endhead
  \midrule
  元年 & 277 & \tabularnewline\hline
  二年 & 278 & \tabularnewline\hline
  三年 & 279 & \tabularnewline\hline
  四年 & 280 & \tabularnewline
  \bottomrule
\end{longtable}


%%% Local Variables:
%%% mode: latex
%%% TeX-engine: xetex
%%% TeX-master: "../../Main"
%%% End:


%%% Local Variables:
%%% mode: latex
%%% TeX-engine: xetex
%%% TeX-master: "../../Main"
%%% End:


%%% Local Variables:
%%% mode: latex
%%% TeX-engine: xetex
%%% TeX-master: "../Main"
%%% End:
