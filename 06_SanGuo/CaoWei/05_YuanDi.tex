%% -*- coding: utf-8 -*-
%% Time-stamp: <Chen Wang: 2021-11-01 11:35:58>

\subsection{元帝曹奂\tiny(260-265)}

\subsubsection{生平}

魏元帝曹奂(246年-302年),原名璜,字景明,是曹魏最后一個皇帝,260年到266年在位。

曹奂是燕王曹宇之子,曹操之孙,初封常道乡公。

甘露五年(260年)魏帝曹髦试图从司马氏手中夺回权力失敗,兵敗被杀,丞相司马昭派儿子使持节行中护军中垒将军司马炎迎立曹璜为皇帝,改名曹奂,入继为堂兄魏明帝曹叡的儿子。曹奂实际上毫无权力,在大臣和军队中也没有任何势力,完全是司马昭的傀儡。

曹奂在位期间,263年曹魏大将邓艾和鍾会伐蜀汉,蜀汉灭亡。伐蜀期间,丞相司马昭以伐蜀之功被晋升为晋公,相国,加九锡。

蜀汉灭亡没多久,丞相司马昭又进爵晋王,不久去世,司马炎于咸熙二年(265年)篡位,魏亡。

魏亡後,曹奐被封為陳留王,並遷居邺城。出城時,太傅司馬孚握著他的手說:「我到死都是大魏的忠臣。」

晋武帝司马炎仿漢獻帝劉協例,允许曹奐仍保有皇帝仪仗、用皇家礼仪祭祖、不以臣下自称。

晉惠帝太安元年(302年)曹奐死於許昌,享年五十七歲,以皇帝礼下葬,谥号元皇帝。

曹奐的後人沒有在官方的紀錄出現,而史料中亦未记载他有沒有後人。由於他離世時正值八王之亂,很多王公士族在當時均為亂軍所殺,並且紀錄可能在那段期間被燒毀。東晉以降,陳留王由曹操玄孫曹勱及其後裔承襲,但無法得知他们和曹奐的關係。邺城遗址附近一直有相传为曹奂墓的土墩,后来考古发掘证实并非曹奂墓。


\subsubsection{景元}

\begin{longtable}{|>{\centering\scriptsize}m{2em}|>{\centering\scriptsize}m{1.3em}|>{\centering}m{8.8em}|}
  % \caption{秦王政}\
  \toprule
  \SimHei \normalsize 年数 & \SimHei \scriptsize 公元 & \SimHei 大事件 \tabularnewline
  % \midrule
  \endfirsthead
  \toprule
  \SimHei \normalsize 年数 & \SimHei \scriptsize 公元 & \SimHei 大事件 \tabularnewline
  \midrule
  \endhead
  \midrule
  元年 & 260 & \tabularnewline\hline
  二年 & 261 & \tabularnewline\hline
  三年 & 262 & \tabularnewline\hline
  四年 & 263 & \tabularnewline\hline
  五年 & 264 & \tabularnewline
  \bottomrule
\end{longtable}

\subsubsection{咸熙}

\begin{longtable}{|>{\centering\scriptsize}m{2em}|>{\centering\scriptsize}m{1.3em}|>{\centering}m{8.8em}|}
  % \caption{秦王政}\
  \toprule
  \SimHei \normalsize 年数 & \SimHei \scriptsize 公元 & \SimHei 大事件 \tabularnewline
  % \midrule
  \endfirsthead
  \toprule
  \SimHei \normalsize 年数 & \SimHei \scriptsize 公元 & \SimHei 大事件 \tabularnewline
  \midrule
  \endhead
  \midrule
  元年 & 264 & \tabularnewline\hline
  二年 & 265 & \tabularnewline
  \bottomrule
\end{longtable}


%%% Local Variables:
%%% mode: latex
%%% TeX-engine: xetex
%%% TeX-master: "../../Main"
%%% End:
