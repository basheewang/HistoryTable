%% -*- coding: utf-8 -*-
%% Time-stamp: <Chen Wang: 2019-12-17 22:28:54>

\subsection{少帝\tiny(239-254)}

\subsubsection{生平}

曹芳(232年-274年),字蘭卿,魏明帝養子,是三国時曹魏第三代皇帝,在位15年。在《三國志》中與曹髦、曹奐合稱三少帝。

曹芳是曹魏在位最久的君主,但也是曹魏第一位傀儡皇帝,實權先後由曹爽、司馬懿和司馬師掌握。

魏明帝的亲生儿子全部夭折,曹芳是其养子。但自小在宮中成長的曹芳,不知出自何宗室,出生來歷一概不詳。《魏氏春秋》記載,曹芳應該是任城王曹楷之子。

魏青龙三年,封为齐王,239年被立为太子,同年登基即帝位,年僅八歲。

第二年,即240年,改年号为正始,由大将军曹爽和太傅司马懿共同辅政。曹爽主政时,架空司马懿、隔离郭太后、起用新人。

魏正始十年(249年),司马懿發動政變,起兵控制雒邑,史稱高平陵之變。71歲的司马懿族滅曹爽,从此由司馬氏独掌军国大权,同年改元为嘉平。

251年,司马懿死后,長子司馬師受命總攬朝政,曹芳聯合李豐、張緝、夏侯玄等意圖罷黜司馬師,欲改立夏侯玄為大將軍,然而三人被司馬師搜出“衣帶詔”,夏侯玄、李豐、張緝受腰斬滅族,並且盡滅李氏、夏侯氏、張氏。张缉女张皇后也被废位。

254年,姜维进攻陇右,司马昭从许昌征还京城洛阳,准备前去讨伐姜维。曹芳前往平乐观慰问司马昭部队,中领军和大臣想趁着曹芳和司马昭会面的机会杀死司马昭,夺其军队讨伐司马师,但曹芳不敢这么做,不過图谋杀司马昭一事卻為司马师所得知,司马师於是决定废帝,派郭太后从父长水校尉郭芝告知郭太后。当时曹芳正与太后相对而坐,郭芝对曹芳说:“大将军想废了陛下,立彭城王曹据。”曹芳起身而去,太后也不悦,但郭芝说司马师决心已下,勒兵于外。太后说想见司马师,為郭芝所拒。郭太后只得取玺绶给郭芝,郭芝再给司马师。司马师以太后令告知群臣,由群臣上奏郭太后,謂曹芳年长不亲政、沉迷女色、废弃讲学、弃辱儒士、与优人、保林等淫乱作乐,并弹打进谏的清商令狐景、清商丞庞熙,乃至用烧铁重伤令狐景身体、太后丧母时不尽礼等罪,请依霍光故事废曹芳。曹芳因此被废,當年22歲,仍按登基前为齐王,另立明帝侄曹髦为帝。曹芳被迁居西宫,乘王车,泣别太后,从太极殿南出宫,数十大臣相送。在联署上奏废帝时,署名第一的太尉司马孚很悲伤,哭得尤其多。

此后,司马师、司马昭相继掌权。

266年,晋朝立国,曹芳被封为邵陵县公,274年逝世,享年43歲。谥号厉,故又称邵陵厉公。

曹芳登基后,为避讳他的名字“芳”,而将位于洛阳北宫的皇宫花园“芳林苑”改名为“华林苑”。后来南北朝分裂,建业、长安的皇宫花园都因循旧制命名为华林苑。

滕胤:“曹芳暗劣,而政在私门,彼之民臣,固有离心。”(《资治通鉴·卷七十六》)

毌丘俭:“懿每叹说齐王自堪人主,君臣之义定。奉事以来十有五载,始欲归政,按行武库,诏问禁兵不得妄出。师自知奸慝,人神所不祐,矫废君主,加之以罪。”(《三国志·魏书·王毌丘诸葛邓锺传第二十八》)

郭太后废曹芳诏令:“皇帝芳春秋已长,不亲万机,耽淫内宠,沈漫女德,日延倡优,纵其丑谑;迎六宫家人留止内房,毁人伦之叙,乱男女之节;恭孝日亏,悖慠滋甚,不可以承天绪,奉宗庙。”(《三国志·卷四·魏书四·三少帝纪第四》)

陈寿:“古者以天下为公,唯贤是与。后代世位,立子以适;若适嗣不继,则宜取旁亲明德,若汉之文、宣者,斯不易之常准也。明帝既不能然,情系私爱,抚养婴孩,传以大器,托付不专,必参枝族,终于曹爽诛夷,齐王替位。”(《三国志·卷四·魏书四·三少帝纪第四》)

梁章钜:“齐王临御之初,即罢宫室工作,免官奴婢六十以上为良人,出内府金银销冶以供军用;二年通《论语》,五年通《尚书》,七年通《礼记》,三祀孔子,以颜子配;良法美政,史不绝书。”(《三国志集解》)

何焯:“芳临御数载,非若昌邑始征,若果君德有阙,播恶于众,师何难执以为辞?今称太后之令,发床第之私,有以知其为诬矣。”(《三国志集解》)

\subsubsection{正始}

\begin{longtable}{|>{\centering\scriptsize}m{2em}|>{\centering\scriptsize}m{1.3em}|>{\centering}m{8.8em}|}
  % \caption{秦王政}\
  \toprule
  \SimHei \normalsize 年数 & \SimHei \scriptsize 公元 & \SimHei 大事件 \tabularnewline
  % \midrule
  \endfirsthead
  \toprule
  \SimHei \normalsize 年数 & \SimHei \scriptsize 公元 & \SimHei 大事件 \tabularnewline
  \midrule
  \endhead
  \midrule
  元年 & 240 & \tabularnewline\hline
  二年 & 241 & \tabularnewline\hline
  三年 & 242 & \tabularnewline\hline
  四年 & 243 & \tabularnewline\hline
  五年 & 244 & \tabularnewline\hline
  六年 & 245 & \tabularnewline\hline
  七年 & 246 & \tabularnewline\hline
  八年 & 247 & \tabularnewline\hline
  九年 & 248 & \tabularnewline\hline
  十年 & 249 & \tabularnewline
  \bottomrule
\end{longtable}

\subsubsection{嘉平}

\begin{longtable}{|>{\centering\scriptsize}m{2em}|>{\centering\scriptsize}m{1.3em}|>{\centering}m{8.8em}|}
  % \caption{秦王政}\
  \toprule
  \SimHei \normalsize 年数 & \SimHei \scriptsize 公元 & \SimHei 大事件 \tabularnewline
  % \midrule
  \endfirsthead
  \toprule
  \SimHei \normalsize 年数 & \SimHei \scriptsize 公元 & \SimHei 大事件 \tabularnewline
  \midrule
  \endhead
  \midrule
  元年 & 249 & \tabularnewline\hline
  二年 & 250 & \tabularnewline\hline
  三年 & 251 & \tabularnewline\hline
  四年 & 252 & \tabularnewline\hline
  五年 & 253 & \tabularnewline\hline
  六年 & 254 & \tabularnewline
  \bottomrule
\end{longtable}


%%% Local Variables:
%%% mode: latex
%%% TeX-engine: xetex
%%% TeX-master: "../../Main"
%%% End:
