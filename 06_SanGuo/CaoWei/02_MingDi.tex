%% -*- coding: utf-8 -*-
%% Time-stamp: <Chen Wang: 2019-12-17 22:25:50>

\subsection{明帝\tiny(226-239)}

\subsubsection{生平}

魏明帝曹叡(204年?-239年1月22日),字元仲,豫州沛国谯县(今安徽省亳州市)人。三国时期曹魏第二任皇帝(226年至239年在位)。魏文帝曹丕长子,母为文昭甄皇后。

黄初三年(222年),曹叡封平原王,黄初七年(226年)五月,魏文帝病重,立曹叡为皇太子,即位于洛阳。曹叡在位期间指挥曹真、司马懿等人成功防御吴、蜀的多次攻伐,并且平定鲜卑,攻灭公孙渊,设置律博士制度,重视狱讼审理,与尚书陈群等人制《魏律》十八篇,是古代法典编纂史上的重大进步。魏明帝在军事,政治和文化方面都颇有建树,但在统治后期大兴土木,广采众女,因此留下负面影响。

景初三年(239年),曹叡病逝于洛阳,时年三十五岁,庙号烈祖,谥号明帝,葬于高平陵。曹叡能诗文,与曹操、曹丕并称魏氏“三祖”,原有集,已散佚,后人辑有其散文二卷、乐府诗十余首。自從曹叡崩後,曹爽掌权,魏帝自此淪爲傀儡。再後曹爽被司馬懿發動高平陵之变斬殺,魏國大權完全落入司馬氏家族手中。

叡生于建安九年(204年?),母亲是文昭甄皇后,甄氏初为幽州刺史袁熙妻子,曹操打败袁绍后,被魏文帝曹丕所纳,甚为得宠,生有曹叡和东乡公主。曹叡从小才智出眾、聰明特異。祖父曹操对此十分惊喜而倍加喜爱,常令他伴随左右。在朝会宴席上,也经常叫他与侍中近臣并列。曹操曾经评价道:“我的家族基业有了你就可以继承三代了。”曹叡好学多识,尤其留意研究律法。

建安二十一年(217年),曹操封魏王,同年东征孙权,曹叡及妹妹东乡公主离开母亲甄氏,与祖母卞夫人,父亲曹丕一起随征江东。

延康元年(220年),曹操病逝,其父曹丕继位魏王,同年五月,十五岁的曹叡被封为武德侯,曹丕作《以侍中郑称为武德侯傅令》,亲自诏令时任侍中的笃学大儒郑称为曹叡的师傅,教授他经学,以此明志。

黄初二年(221年),曹叡被封为齐公,同年八月,其母甄氏因为怨言而被曹丕赐死,葬于邺城,曹叡因为母亲获罪,降为平原侯。黄初三年(222年)三月,曹丕又复其爵位,晋封为平原王。

生母被赐死,曹叡一併受罚被貶为平原侯。起初其父魏文帝认为曹叡先前既有不满,便想立徐姬所生的京兆王曹礼为嗣,因此久不立太子。这期间,曹叡府中来往的家臣官吏、师长、友伴,一律只取品行正直的人充任,互相匡扶、勉励矫正,与卫臻私交甚好,经常一起讨论朝事和书籍。曹丕也曾询问卫臻关于曹叡的情况,卫瑧只是称赞他明理而有德行,不言其他。

据《魏末传》记载,曹叡一次随曹丕狩猎,见到母子两鹿。文帝射杀鹿母,命令曹叡射杀子鹿,曹叡说:“陛下已经杀掉母鹿,臣实在不忍心再杀掉它的孩子。”说完哭泣不已。文帝于是放下弓箭,深感惊奇,而确定立曹叡为太子的心意。

黄初三年(222年)三月,曹叡升为平原王,后来曹丕下诏将其过继给郭皇后为子,进一步确定嫡长子的地位。然而曹叡因其母非善终,内心愤愤不平,后来才开始恭敬地侍奉嫡母,每日早晚都往皇后宫中定省问安,郭皇后也因自己无子,对曹叡慈爱有加。除了曹丕为曹叡诏令郑称为师,平原王府中还配置高堂隆为平原王傅,毌丘俭、何曾、吉茂等一干人等为文学属官。黄初四年,曹丕为曹叡聘河内世家大族虞氏为平原王妃,又选河内毛氏入宫,曹叡十分宠爱,出入都与其同乘舆辇。

226年,五月十六日丙辰(6月28日)魏文帝病危,立平原王曹叡為太子,召曹真、曹休、陳群、司馬懿,并受遗诏辅佐嗣主。十七日丁巳(6月29日)崩于嘉福殿。曹叡繼位,是為魏明帝。

明帝登基后首先必须对抗内外敌人的攻击,226年八月孙权攻江夏和襄阳、227年孟达反、到234年为止诸葛亮五次进攻曹魏、234年孙权攻合肥;明帝重用滿寵處理这些内外战争,亦重用曹真、張郃、司马懿等名將与诸葛亮作战。235年诸葛亮死后,魏蜀边境上的情况有所减缓,明帝开始在洛阳大建宫殿,常用人力、物力,大臣楊阜、高堂隆等對此一再勸練,明帝雖多未採納,但也不因此問罪臣屬。同年,他将养子曹芳封为齐王。

237年,聽從高堂隆的建議,發布《景初暦》,是歲將青龍五年春三月改為景初元年夏四月,同年辽东公孙渊造反,自立为燕王,明帝令司马懿攻辽东,司马懿遂带兵四万,和夏侯霸等人出征辽东,大破燕军,杀公孙渊,成功收复辽东。

从238年冬开始,魏明帝的健康开始恶化。239年初,魏明帝病重,曹叡本意让燕王曹宇为大将军,曹献、曹爽、曹肇、秦朗共同辅政,但曹宇一直不接受。于是曹叡单独召见刘放、孙资到其床边问话,问道“燕王为何一直不接受大将军的安排?”刘放和孙资回答:“燕王实在是自己知道不堪大任所以推辞”,曹叡又问:“曹爽可以代曹宇为大将军么?”刘放和孙资表示赞同,同时又多次强调应该迅速召见太尉司马懿来辅助朝纲,曹叡答应并令刘放起诏书。刘放、孙资退下之后,曹叡的想法突然改变,宣诏让司马懿不要入宫,过一段时间曹叡见到刘放、孙资说:“我同意召见司马懿,但是曹肇等人却让我不要这样做,差点坏了我的大事!”于是再次起草诏书,命曹爽、刘放、孙资一同接受诏令,同时免去曹宇、夏侯献、曹肇、秦朗等人的官职。

景初三年正月初一(239年1月22日),司马懿率师从辽东回到河内郡驻扎。明帝传令把他急招入卧室,拉着他的手嘱咐说:“终于等到你来,现在把后事托付给您,和大将军曹爽共佐曹芳。我在死前能见到你,也没什么遗憾的了。”又把齐王曹芳和秦王招来,嘱托司馬懿照顾。当天,明帝驾崩于洛阳宫嘉福殿,年仅三十五岁,《三国志》作三十六岁。。《三国志》载魏明帝崩于嘉福殿,《魏書》载他崩于九龙前殿。正月廿七癸丑日(2月17日),葬高平陵。

曹叡生母甄氏原为袁绍次子袁熙婦,據《三國志》卷三《明帝紀》記載魏明帝去世時“時年三十六”。裴松之在《三國志注》中計算曹叡年齡時,認為甄氏在建安九年(204年)八月曹操攻占邺城后才為曹丕所納,到景初三年正月初一丁亥日(239年1月22日)曹叡死時最多也只有虚岁三十五歲,不能計為三十六歲。由此如卢弼等人就認為曹叡或許是袁熙之子,陳壽故意對於年龄曲笔。

魏明帝曾經下令由盧毓來推舉官吏,並要求不要只看候選人的名氣,而要看他們的品行與能力。明帝表示,名氣就像是圖畫的餅一樣,根本無用,無法充飢。這就是「畫餅充飢」的典故。

陈寿:“明帝沉毅断识,任心而行,盖有君人之至概焉。于时百姓彫弊,四海分崩,不先聿修显祖,阐拓洪基,而遽追秦皇、汉武,宫馆是营,格之远猷,其殆疾乎!”(《三国志·魏书·明帝纪第三》)

孙权:“及操子丕,桀逆遗丑,荐作奸回,偷取天位,而叡么麽,寻丕凶迹,阻兵盗土,未伏厥诛。”(《三国志·卷四十七·吴书二·吴主传第二》)

孙盛:“魏明帝天资秀出,立发垂地,口吃少言,而沉毅好断。初,诸公受遗辅导,帝皆以方任处之,政自己出。而优礼大臣,开容善直,虽犯颜极谏,无所摧戮,其君人之量如此之伟也。然不思建德垂风,不固维城之基,至使大权偏据,社稷无卫,悲夫!”

刘晔:“秦始皇、汉孝武之俦,才具微不及耳。”(《世说新语》)

陆逊:“选用忠良,宽刑罚,布恩惠,薄赋省役,以悦民心,其患更深於操时。”(《三国志·吴书·张顾诸葛步传第七》)

钟会:“烈祖明皇帝奕世重光,恢拓洪业。”(《三国志·魏书·王毌丘诸葛邓锺传第二十八》)

阎缵:“及至明帝,因母得罪,废为平原侯,为置家臣庶子,师友文学,皆取正人,共相匡矫。兢兢慎罚,事父以孝,父没,事母以谨,闻于天下,于今称之。”(《上书理湣怀太子(司马遹)之冤》)

王沈:“好学多识,特留意于法理。”(《魏书》)

裴松之:“魏明帝一时明主。”

司马光:“汉主寿常慕汉武,魏明之为人。”(《资治通鉴·卷第九十六》)“帝沈毅明敏,任心而行,料简功能,屏绝浮伪。行师动众,论决大事,谋臣将相,咸服帝之大略。性特强识,虽左右小臣,官簿性行,名迹所履,及其父兄子弟,一经耳目,终不遗忘。”(《资治通鉴·卷第七十四》)

郭威:“汉高祖为义帝发丧,魏明帝正禅陵尊号,一时达礼,千古所称。”(《全唐文·卷一百二十三》)

胡应麟:“诗未有三世传者,既传而且煊赫,仅曹氏操、丕、睿耳。”(《诗薮》)

吕思勉:“魏文帝本来无甚才略。死后,儿子明帝继立,荒淫奢侈,朝政更坏。”(《中国通史:后汉的分裂和三国》)

蔡东藩:“曹叡奢淫无度,违理蔑伦,种种荒谬,俱足亡国,而反得平定辽东,擒斩公孙渊父子,是所谓天夺之鉴,而益其疾也。”(《后汉演义》)

马植杰:“综观曹叡之行事,优缺点各占一半,其优点是善为军计、明察断狱、比较能容人直谏。曹叡在容受直言、不杀谏臣方面,在古代封建君主中是少见的,这算是他的特色。曹叡的最大缺点是奢淫过度,还有一个重要的失误,则在确定继承人和辅政大臣方面。”(《魏的政治与司马氏专政》)

\subsubsection{太和}

\begin{longtable}{|>{\centering\scriptsize}m{2em}|>{\centering\scriptsize}m{1.3em}|>{\centering}m{8.8em}|}
  % \caption{秦王政}\
  \toprule
  \SimHei \normalsize 年数 & \SimHei \scriptsize 公元 & \SimHei 大事件 \tabularnewline
  % \midrule
  \endfirsthead
  \toprule
  \SimHei \normalsize 年数 & \SimHei \scriptsize 公元 & \SimHei 大事件 \tabularnewline
  \midrule
  \endhead
  \midrule
  元年 & 227 & \tabularnewline\hline
  二年 & 228 & \tabularnewline\hline
  三年 & 229 & \tabularnewline\hline
  四年 & 230 & \tabularnewline\hline
  五年 & 231 & \tabularnewline\hline
  六年 & 232 & \tabularnewline\hline
  七年 & 233 & \tabularnewline
  \bottomrule
\end{longtable}

\subsubsection{青龙}

\begin{longtable}{|>{\centering\scriptsize}m{2em}|>{\centering\scriptsize}m{1.3em}|>{\centering}m{8.8em}|}
  % \caption{秦王政}\
  \toprule
  \SimHei \normalsize 年数 & \SimHei \scriptsize 公元 & \SimHei 大事件 \tabularnewline
  % \midrule
  \endfirsthead
  \toprule
  \SimHei \normalsize 年数 & \SimHei \scriptsize 公元 & \SimHei 大事件 \tabularnewline
  \midrule
  \endhead
  \midrule
  元年 & 233 & \tabularnewline\hline
  二年 & 234 & \tabularnewline\hline
  三年 & 235 & \tabularnewline\hline
  四年 & 236 & \tabularnewline\hline
  五年 & 237 & \tabularnewline
  \bottomrule
\end{longtable}

\subsubsection{景初}

\begin{longtable}{|>{\centering\scriptsize}m{2em}|>{\centering\scriptsize}m{1.3em}|>{\centering}m{8.8em}|}
  % \caption{秦王政}\
  \toprule
  \SimHei \normalsize 年数 & \SimHei \scriptsize 公元 & \SimHei 大事件 \tabularnewline
  % \midrule
  \endfirsthead
  \toprule
  \SimHei \normalsize 年数 & \SimHei \scriptsize 公元 & \SimHei 大事件 \tabularnewline
  \midrule
  \endhead
  \midrule
  元年 & 237 & \tabularnewline\hline
  二年 & 238 & \tabularnewline\hline
  三年 & 239 & \tabularnewline
  \bottomrule
\end{longtable}


%%% Local Variables:
%%% mode: latex
%%% TeX-engine: xetex
%%% TeX-master: "../../Main"
%%% End:
