%% -*- coding: utf-8 -*-
%% Time-stamp: <Chen Wang: 2021-11-01 11:35:53>

\subsection{高贵乡公曹髦\tiny(254-260)}

\subsubsection{生平}

曹髦(241年11月15日-260年6月2日),字彥士,三國時期曹魏第四代皇帝(254年-260年在位),曹霖之子,魏文帝曹丕孙。在《三國志》中與曹芳、曹奐合稱三少帝。

正始五年(244年),封郯县高貴鄉公。

254年,司馬師廢掉時年23歲曹芳的皇位,打算立曹操之子彭城王曹据为皇帝。郭太后指出曹据於礼是她(丈夫)的叔叔,曹据若继皇位,則太后之位將免除,因此易魏文帝曹丕之孫高贵乡公曹髦为帝,作为其无嗣的伯父曹叡魏明帝的后嗣,改元「正元」;曹髦當年12歲,實權先後由司馬師和司馬昭掌握。

司馬師曾問鍾會曹髦的能力,鍾會回答:「文同陳思,武類太祖」(文采如同陳思王曹植,武可比太祖曹操)。司马师说若果然如此,是社稷之幸。但隨著小皇帝長大,曹髦對專權的司馬昭日益不滿。不久他寫了一首《黃龍歌》,司馬昭發現,起了戒心。

甘露五年(260年)五月曹髦召見王沈、王經、王業等三人,憤慨說道:「司馬昭之心,路人皆知也!朕不能坐受廢辱,今日當與卿等自出討之。」不顧郭太后及眾臣的反對,率領宮人三百餘人討伐司馬昭。王沈與王業先行,向司馬昭通風報信。司馬昭旋即派兵入宮鎮壓,雙方在宮內東止車門相遇。中護軍賈充在南闕下,率軍迎戰曹髦。賈充命令成濟殺曹髦,成濟一劍刺穿曹髦胸膛,曹髦斷氣,當場死在車上,血流滿地,仅虚岁20歲,史稱司馬昭弒君。

隨後,司馬昭假借郭太后的名義下詔“高贵乡公悖逆不道,自陷大祸,依汉昌邑王罪废故事,以民礼葬”。但在司马昭叔父司马孚请求下,以王礼下葬曹髦于洛阳西北三十里瀍涧之滨,仅下车数乘、不设旌旐,百姓相聚而观,说:“这就是前日所杀的天子。”有人甚至掩面而泣,悲伤不能自已。由於輿論憤憤不平,司馬昭遂在事發後的20天將弒君罪狀全推給成濟,称成济违背自己命令杀死曹髦,以「大逆不道」罪誅殺成濟一族,斷成氏一脈。司馬昭後立曹奐為曹魏皇帝。

曹髦擅長寫詩文。他的繪畫藝術也相當不錯,是一個善於琴棋書畫的才子。


\subsubsection{正元}

\begin{longtable}{|>{\centering\scriptsize}m{2em}|>{\centering\scriptsize}m{1.3em}|>{\centering}m{8.8em}|}
  % \caption{秦王政}\
  \toprule
  \SimHei \normalsize 年数 & \SimHei \scriptsize 公元 & \SimHei 大事件 \tabularnewline
  % \midrule
  \endfirsthead
  \toprule
  \SimHei \normalsize 年数 & \SimHei \scriptsize 公元 & \SimHei 大事件 \tabularnewline
  \midrule
  \endhead
  \midrule
  元年 & 254 & \tabularnewline\hline
  二年 & 255 & \tabularnewline\hline
  三年 & 256 & \tabularnewline
  \bottomrule
\end{longtable}

\subsubsection{甘露}

\begin{longtable}{|>{\centering\scriptsize}m{2em}|>{\centering\scriptsize}m{1.3em}|>{\centering}m{8.8em}|}
  % \caption{秦王政}\
  \toprule
  \SimHei \normalsize 年数 & \SimHei \scriptsize 公元 & \SimHei 大事件 \tabularnewline
  % \midrule
  \endfirsthead
  \toprule
  \SimHei \normalsize 年数 & \SimHei \scriptsize 公元 & \SimHei 大事件 \tabularnewline
  \midrule
  \endhead
  \midrule
  元年 & 256 & \tabularnewline\hline
  二年 & 257 & \tabularnewline\hline
  三年 & 258 & \tabularnewline\hline
  四年 & 259 & \tabularnewline\hline
  五年 & 260 & \tabularnewline
  \bottomrule
\end{longtable}


%%% Local Variables:
%%% mode: latex
%%% TeX-engine: xetex
%%% TeX-master: "../../Main"
%%% End:
