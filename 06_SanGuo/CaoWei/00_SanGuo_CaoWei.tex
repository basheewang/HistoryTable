%% -*- coding: utf-8 -*-
%% Time-stamp: <Chen Wang: 2019-12-17 22:30:30>


\section{曹魏\tiny(220-265)}

\subsection{简介}

魏(220年12月10日-266年2月8日,史称曹魏、魏朝)是中國歷史上東漢末年三國之中据有北方及中原的政權。始於220年曹丕逼迫漢獻帝劉協禪讓帝位,篡漢為魏,因承繼漢朝,故具法統地位。至266年魏又被司馬炎篡奪,改號為晉。

曹操受封魏公時,治所在東漢時期魏郡所在地的鄴,因此漢獻帝封他為「魏公」建立诸侯国——魏国,且如同汉朝初期诸侯王制度可以设置丞相以下百官,之后又进封「魏王」并以卞氏为魏国王妃,以曹操之女为魏国公主,后来曹操之子曹丕篡漢时便以「魏」為國號。又因为是曹氏政权,故史稱「曹魏」,以區別於其他名「魏」的政權。

魏是三國時期最為強大,領土最遼闊的政權,灭蜀汉前疆域達到近300萬平方千米[來源請求]。263年,魏軍攻滅蜀汉,同年佔領廣州,至此曹魏疆域達到全盛,约400萬平方千米。由於曹魏盤踞中原,所以人口也是三国中最多。期間最重要的政治改革有陳群的九品中正制,對魏晉時代之政治產生深遠影響。

曹魏未立國前,東漢已進入群雄割據的時代。建安元年(196年)曹操迎漢獻帝到許昌,挾天子以令諸侯,迫令各地割據勢力必須遵奉曹操,而軍事方面則選編精銳,組成一批強大的騎兵勁旅——虎豹騎,在平定中原的重要戰役中屢次建功,在掌握軍政的發展下開啟魏晉南北朝霸府政治的先河,為後來的篡漢立魏打下了基礎。200年的官渡之戰後控制了中原北方地區大部分,並準備南下一舉攻佔全國。

但由於曹軍未適應到南方的地理環境,在水土不服,天氣陡轉等因素下,於在208年的赤壁之戰中敗於孫權與劉備的南方聯軍,退守北方。但由於曹軍所傷多為原劉表麾下水軍與降軍,因此孫劉勢力亦無法撼動曹操政權。由於曹操年事已高,於是終其一生只控制了中原一帶。其後漢獻帝封曹操為魏王,打破了漢高祖所訂的白馬之盟,220年,曹操去世後由次子曹丕襲位。

曹丕逼迫漢獻帝禪讓,篡漢為帝,定國號魏,並定都洛陽(或称曹魏五都),曹魏正式建立。曹操雖未稱帝,但曹丕稱帝后追尊他為魏武帝(廟號太祖)。漢中王劉備於221年在成都自立為帝,國號漢,史稱蜀漢。最後孫權也於229年在建業稱帝國號吳,至此三國正式形成。

魏文帝曹丕稱帝七年就去世。曹丕死後,長子曹叡即位,是為魏明帝。曹魏朝廷此時分為兩大派,一是以曹真和曹休為主的曹氏一族、二是以司馬懿世家和賈逵世家為主的新勢力,日漸形成嚴重的對立,埋下日後的高平陵之變的種子。而夏侯惇和夏侯渊死後,夏侯世家人才能力沒落,漸漸遠離權勢。導致司馬世家勢力龐大,夏侯世家無法抗衡,日後高平陵之變的發生後,夏侯世家遭流放邊境,或投降蜀漢。

曹操出身寒族,且與閹宦有關,不以儒學為務,與當時的豪族、士大夫不同。曹操曾下「求才三令」,強調重才不重德,並以法家之術為治,要摧破豪族的儒學。曹操為一代梟雄,不僅得到眾多寒族人才支持,也得到部份豪族士大夫支持,如荀彧、荀攸。荀彧更為曹操引進不少士大夫階層的人才。官渡之戰,曹勝袁敗,士大夫豪族不得不暫時忍耐屈服,卻伺機恢復。終於他們支持出身士族的司馬懿,向曹氏奪回政權。

曹魏都是在與蜀漢、孫吳的戰事中度過,如蜀漢的諸葛亮發動多次北伐攻打魏國,曹叡多次力拒來犯守護國土。太傅司馬懿在諸葛亮北伐戰事中立下不少戰功,在曹魏的政治地位漸漸提升,直至高平陵之變,司馬懿利用兵變,剷除了曹家宗室的曹爽,导致夏侯霸投靠蜀漢,后其子司马师在政治斗争中又铲除了夏侯家的夏侯玄,司馬氏家族權傾全朝成為新的霸府,司马师、司馬昭兄弟成為朝廷中最有權勢的朝臣,能擅自廢立皇帝。曹魏立国以来宗室受到限制,大部分实际上被囚禁在邺城,故无法反抗司马氏。驻守扬州的地方军事长官王凌(司马懿时期)、毌丘俭(司马师时期)、诸葛诞(司马昭时期)先后起兵反抗司马氏,后两者还联合孙吴为外援,但皆败亡。魏帝曹髦不甘司馬氏威脅自己帝位,亲自攻打司馬昭,司马昭命令親信賈充派成濟弒害曹髦,事後僅成濟被處死,而司馬氏家族則沒受牽連,因此曹魏於此時名存實亡。

隨著蜀漢國力日下,263年魏國司馬氏展開攻漢計劃,派遣鍾會、鄧艾、諸葛緒等等攻漢,結果漢後主劉禪出降,蜀漢亡國,隨後司馬昭便平定由鍾會、姜維、劉璿等等蜀漢殘黨與部份鍾會勢力所發起的鍾會之亂。後司馬昭死,其子司馬炎襲晉王、相國位,於265年篡魏自立,國號晉,曹魏滅亡。

曹奂及后人被封为陈留王,在晋朝受到很高的待遇,陈留王国历经4朝,经西晋、东晋、刘宋,一直传至南朝南齐,国祚之长在历史上实为罕见。

曹魏於曹操死前,兵力約30-45萬,曹操死後,曹丕濫兵用戰,興建修房,死了不少的精兵,據當時的史書記載,兵力約13-16萬,為曹魏的低落時期,到了曹叡統治,由於推動多子政策,兵力大增到40-47萬,到司馬家族主政後,約到60-65萬,為最高峰。

曹魏雖然是以軍事起家,但曹魏一族在文學上具有相當成就,如曹操和其子曹丕和曹植都擅於寫詩,時稱三曹,後世又稱建安文學。後期君主也頗有藝術造詣,如曹操其孫曹叡擅長詩賦;曹髦擅長詩文、繪畫,被譽為才子。

魏、漢、吳三國中以曹操最重視農業(用毛玠兵農合一之策),其中以曹魏人口最多,屯田墾荒的面積最廣。

曹魏重視農業的另一實證,是其大力興修水利,其工程的規模和數量在三國中首屈一指。如233年,關中一帶闢建渠道,興修水庫,一舉改造三千多頃鹽鹼地,所獲使國庫大為充實。再如曹魏在河南的水利工程,其成果使糧食產量倍增。

諸葛亮北伐時期對曹魏的經濟影響甚是巨大,從辛毗、楊阜的奏章中,不止一次提到諸葛亮北伐造成的經濟困難。

曹魏对藩王监视防范,藩王们求为百姓而不可得,后大多被囚禁于邺城。曹叡及后来的执政大臣曹爽先后拒绝曹植、曹冏任用宗室的建议,终于使得后来皇权旁落司马氏之手时曹魏诸王几乎无力反抗。

司马炎建立晋朝后,立即解除对曹氏族人的禁锢,同时因为吸取曹魏的教训而大封宗室为实权藩王委以重任,导致了后来的八王之乱。

%% -*- coding: utf-8 -*-
%% Time-stamp: <Chen Wang: 2021-11-01 11:40:57>

\subsection{文帝曹丕\tiny(220-226)}

\subsubsection{生平}

魏文帝曹丕(187年-226年6月29日),字子桓,沛国譙县(今属安徽亳州)人。三国時期曹魏開國皇帝,曹操的嫡长子,之後繼承父親的魏王封號與丞相的大權,最終東漢皇帝汉獻帝禪讓於其,曹丕登基後改國號為魏,史称曹魏,226年駕崩,諡文皇帝。

除軍政以外,曹丕自幼好文學,於詩、賦、文學皆有成就,尤擅長於五言詩,與其父曹操及其弟曹植並稱三曹,今存《魏文帝集》二卷。另外,曹丕著有《典論》,當中的《論文》是中國文學史上第一部有系統的文學批評專論作品。與父親曹操、其子曹叡並稱「魏氏三祖」。

187年冬天,曹丕生於沛国譙县(今属安徽亳州)。曹丕文武双全,六岁懂得射箭,八歲就能提筆為文和学会骑射。曹丕好击剑,博覽古今經传,通晓諸子百家学说。

197年,長兄曹昂、曹丕隨父出征宛城,曹昂與大將典韋、堂兄曹安民一同戰死於宛城,曹丕则幸运的骑马逃走。正室丁夫人因養子曹昂之死怪罪曹操而與曹操離異,生母卞夫人被扶為正室,原為庶長子的曹丕也就取代長兄曹昂成了嫡長子。

200年,曹丕跟隨曹操參加官渡之戰。

211年2月12日(建安十六年正月辛巳日),被任命為五官中郎將、副丞相。

212-213年,曹丕參加濡须口之战。

217年,在夺嫡之争中击败弟弟曹植,被立為魏王世子,正式成为曹操的继承人。

220年正月二十三日,曹操逝世,曹丕繼任丞相、魏王。十月十三日乙卯(11月25日)曹丕篡漢,逼迫汉献帝刘協禪讓帝位,廿九日辛未(12月11日)正式登基。自謂:“舜、禹之事,吾知之矣。”(一些史書記載舜、禹通过禅让成为天子,而曹丕通過逼迫漢帝禪讓帝位,他說這句話的隱含意義是堯舜禹的禪讓不過也是跟他一樣以逼迫的方式進行的假禪讓罷了,即認同另一些史書中所記載的“堯舜禹非禪讓說”)

221年八月,孙权遣使奏章臣屬。十九日丁巳(9月23日),文帝遣太常邢貞封孫權為吳王。同年四月,漢中王刘备称帝建立蜀汉。

221年,夫人甄氏卒。

222年,孙权不聽從曹丕要求遣子到魏國作為人質,經過多次周旋,曹丕認為孫權數次背叛而三路伐吳,但最終以大敗收場。

224年八月与225年八月,發生廣陵事變,同樣以失敗告終。

226年,五月十六日丙辰(6月28日)文帝病危,立平原王曹叡為太子,召曹真、曹休、陳群、司馬懿,并受遗诏辅佐嗣主。十七日丁巳(6月29日)崩于嘉福殿,终年四十岁。曹叡繼位,是為魏明帝。六月九日戊寅(7月20日),葬首陽陵。

曹操去世后,曹丕继任丞相、魏王、冀州牧,改建安二十五年为延康元年。同年十月,逼迫汉献帝禅位,篡汉称帝,国号为魏,改元黄初,定都洛阳。

改革选官制度,采纳陈群建议,实行九品中正制。易中天認為,九品中正制使门阀士族的政治特权得到确立和巩固,得到他们对曹魏政权的支持。

限制宦官、外戚权力:颁令“其宦人为官者不得过诸署令”“群臣不得奏事太后,后族之家不得当辅政之任,又不得横受茅土之爵”,保证了魏始终没有因为宦官、外戚干政造成政治危机,但因為曹丕曹叡父子過於依賴外臣司馬懿,日後司馬懿發動高平陵之變,司馬家掌握曹魏大權,其孫司馬炎更是代魏建晉。

削夺藩王权力:曹魏藩王的封地时常变更,没有治权和兵权,举动受到严格监视,形同囹圄。这个政策虽然吸取了汉朝诸侯国作乱的教训,却留下隐患,导致曹氏夏侯氏宗亲势单力薄,日后无力阻止外臣夺权。

重视文教:黄初二年(221年),下令人口十万以上的郡国每年察举孝廉一人,如有特别优秀的人才,可以不受户口限制。黄初五年,封孔子后人孔羡为宗圣侯,重修孔庙,在各地大兴儒学,立太学,置五经课试之法,设立春秋穀梁博士。在短期内使封建正统文化复兴。

恢复社会生产:除禁令,轻关税,禁止私仇,广议轻刑,与民休养,使北方地区重现安定繁荣局面。聽任典農治生,民屯的成效受到影響,出現弊端。在貨幣政策上,雖曾於登基時發行錢幣,但卻遭到失敗;之後更因穀物價格高騰,罷除了五銖錢(漢錢),自此之後終曹魏一代「以物易物」反成為北方主要的經濟型態。

黃初前期,東漢末諸侯孫權曾向魏稱臣,接受吳王封號。經過多次斡旋,魏吳最終走向敵對。期間曹丕三次親征孫吳均無功而返。

在位期間,任用曹真擊退鮮卑,和匈奴、氐、羌等外族修好,平定邊患。

曹丕心胸狹窄記恨薄情。在年少時曾向堂叔曹洪借錢不成懷恨在心,稱帝後不顧開國功臣元勳及血緣情誼直接栽贓罪名將曹洪下獄,卞太后得知後逼郭皇后求情才讓曹洪免於牢獄之災,但是仍然被曹丕削爵沒收財產貶為庶人。

曹丕非常愛打獵,但在曹操當政時為了爭儲而故意樹立良好形象,就接受崔琰建議停止打獵。曹丕稱帝后,已無爭儲壓力而不聽鮑勳和戴淩停止打獵的忠諫,曹丕因自己最愛的娛樂遭反對非常惱怒,而對戴淩處以比死刑低一等的处罰。而鮑勳的父親鮑信是曹操早年起兵的救命恩人,在曹丕對吳出兵廣陵的時候作出勸諫,曹丕不聽找理由把他強硬處死。

在與曹植鬥爭儲中,曹丕對於曹植的表現沒有任何對策,而且優柔不斷,只能聽從自己派系的人贏取繼承權。在曹操死後稱帝,藉故曹植治理不善削權並進行十數次地方遷徙,丁儀是曹植派,也在稱帝之後將丁家全族處刑。

夏侯尚因為寵愛妾侍而不愛正妻(曹丕的妹妹),曹丕把妾侍處刑,導致夏侯尚精神衰弱至死。

曹休守喪期間不吃肉,下令強逼他吃肉,曹休變得傷心消瘦。

曹丕寵愛郭氏,甄氏對曹丕多次抱怨,於是曹丕把她處死。

曹丕得知自己封為太子,高興而且得意忘形,辛憲英知道後認為魏國國運不會長久。

在曹操死後守喪期間,向孫權索要貴重珍品享樂,對孫權親征時駕馭副車及龍舟等奢華玩意。對外又被年長老練的孫權當棋子耍,不聽群臣勸告在夷陵之戰偷襲孫權,認為這樣做不合禮數;當孫權擊敗劉備後,曹丕假借孫權造反名義發兵進攻孫權打算從中得益,但什麼都得不到。曹丕要求孫權所做的全部得不到所願,於是惱羞成怒三路伐吳脅逼孫權,同時煽動江東內部作亂,最後均以敗退告終。孫權徹底拋棄曹魏,曹丕非常憤怒,於是發動兩次大規模伐吳都是無功而返,最後一次差點被孫權部將擒獲。

力行简葬。曹丕在《终制》中表示,寿陵因山为体,不封树,不立庙,不造园邑神道,不含珠玉,敛以时衣,陶器陪葬。曹丕提出“夫葬也者,藏也,欲人之不得见也。骨无痛痒之知,冢非栖神之宅”“自古及今,未有不亡之国,亦无不掘之墓也”,深受当时社会风气和他父亲曹操的影响。曹丕死后,按《终制》葬于首阳陵。

曹丕詩歌形式多樣,而以五、七言為長,語言通俗,具有民歌精神;手法則委婉細緻,回環往復,是描寫男女愛情和遊子思婦題材的箇中能手。 代表曹丕诗歌最高成就的《燕歌行》,据考写于建安十二年曹操北征三郡乌桓期间,采用乐府体裁,开创性地以句句用韵的七言诗形式写作,是现存最早最完整的七言诗。《燕歌行》从“思妇”的角度,反映了东汉末年战乱流离的现状,表达出被迫分离的男女内心的怨愤和惆怅。全诗用词不加雕琢,音节婉约,情致流转,被明朝王夫之盛赞“倾情,倾度,倾色,倾声,古今无两”。

曹丕的一些为后人称道的作品都在担任五官中郎将至魏太子期间所作,他的诗歌细腻清越,缠绵悱恻,缺乏曹操、曹植的慷慨之气,后世对他的评价不如“三曹”中的另外两人。

學者葉嘉瑩在《葉嘉瑩說漢魏六朝詩》裡,列舉鍾嶸《詩品》、劉勰《文心雕龍》和王夫之《薑齋詩話》對曹丕的評價。《詩品》將曹丕排在中品,認為他的詩不及弟弟曹植,原因是曹丕詩「率皆鄙直如偶語」(「偶語」,即兩個普通人在講話),反觀曹植則是「骨氣奇高,詞采華茂。情兼雅怨,體被文質,粲溢今古,卓爾不群」。《文心雕龍》(才略篇)說曹丕「魏文之才,洋洋清绮,旧谈抑之,谓去植千里......子桓慮詳而力緩,故不競於先鳴」,與曹植「思捷而才俊」不同,又謂「俗情抑揚,雷同一響,遂令文帝以位尊減才,思王以勢窘益價,未為篤論也」,世人都同情曹植的處境,曹丕是兄弟爭位的勝方,人們也因此忽略他文章的美妙。明末清初,王夫之在《薑齋詩話》裡直言:「實則子桓天才駿發,豈子建所能壓倒耶?」,可谓為曹丕文學成就「平反」的宣言。葉嘉瑩說,曹丕是一位「理性詩人」,有節制有反省,「以感與韻勝」。

诗歌笔法影响着其赋的风貌,诗体之赋为其赋作品的特点,具有明显的标志性意义。曹丕所创作的二十八篇赋作,其中有序者共有十六篇。内容上来看以抒情和咏物为主,体制方面一改汉大赋之鸿篇巨幅,成为短小精悼的行情小赋。内容以真情的笔触触摸到社会现实,并将个体的喜怒哀乐带入赋中。

曹丕的《典论‧论文》是中国最早的文学理论与批评著作,写于曹丕为魏王太子时,文中要点有:以班固和傅毅為例,說明「文人相輕」和「家有弊帚,享之千金」的做文學家的不自見己身的缺點,只看到別人的小缺失就加以嘲諷,對於別人的優點卻視而不見。评价孔融、陈琳、王粲、徐干、阮瑀、应玚、刘桢的文风和得失,“建安七子”的说法来源于此提出“文以气为主,气之清浊有体,不可力强而致”,认为作家的气质决定作品的风格,肯定文学的历史价值,“盖文章,经国之大业,不朽之盛事”。

魯迅在《魏晉風度及文章與藥及酒之關係》中稱「他(曹丕)說詩賦不必寓教訓,反對當時那些寓教訓於詩賦的見解,用近代的文學眼光來看,曹丕的一個時代可說是『文學的自覺時代』,或如近代所說是為藝術而藝術的一派。」

曹丕善写妇女题材的作品,其诗歌有《寡妇诗》与之赋作品皆有机行一致的情感行发。《寡妇赋》、《出妇赋》,更多地表达对下层社会妇女哀悯同情的情怀。

曹丕是邺下文人集团的实际领袖,对建安文学的精神架构起到关键作用,由此形成的“建安风骨”对后世文学产生了深远影响。曹丕命令刘劭、王象、缪袭等人编纂中国第一部类书《皇览》,开官方组织编纂类书的先河。《登台赋》是一篇歌咏铜雀台华美壮丽的小赋,笔触清新细腻,在描写景色时做到了 “写物图貌,蔚似雕画”。《典論‧論文》開創了文學批評的風氣,為中國文學批評之祖。《燕歌行》则是中国文学史上第一首完整的七言诗,此对后世七言诗的创作有很大影响。《校猎赋》是一篇很完整的赋作品,在赋中曹丕运用笔墨不多,不过三百来字就将田猎盛况尽数描绘出来。《列异传》是魏文帝曹丕所写的一部志怪小说集,属于文学艺术,据唐代魏征等人撰写的《隋书·经籍志》记载,它作为现存最早的一部描写鬼类故事的志怪小说,对后世鬼魅小说的描写有着巨大的影响。

诸葛亮:“曹丕篡弑,自立为帝,是犹土龙刍狗之有名也。”(《三国志·卷四十二·蜀书十二·杜周杜许孟来尹李谯郤传第十二》)

孙权:“及操子丕,桀逆遗丑,荐作奸回,偷取天位,而叡么麽,寻丕凶迹,阻兵盗土,未伏厥诛。”(《三国志·卷四十七·吴书二·吴主传第二》)

曹植:「祥惟圣贤,歧嶷幼龄。研几六典,学不过庭;潜心无妄,抗志清冥。才秀藻朗,如玉之莹。」(《曹集诠评》卷十《文帝诔》)

桓阶:“仁冠群子,名昭海内,仁圣达节,天下莫不闻。”(《三国志·卷二十二·桓阶传》)

卞兰:「研精典籍,留意篇章,览照幽微,才不世出,禀聪睿之绝性,体明达之殊风,慈孝发于自然,仁恕洽於无外。是以武夫怀恩,文士归德。窃见所作典论,及诸赋颂,逸句烂然,沈思泉涌,华藻云浮,听之忘味,奉读无倦。」(《艺文类聚》十六《赞述太子赋》)

陸遜:「曹丕大合士眾。外托助國討備,內實有奸心。」(《三國志·吳書·陸遜傳第十三》)

張悌:「曹操虽功盖中夏,民畏其威而不怀其德也。丕、睿承之,刑繁役重,东西驱驰,无有宁岁。」(《资治通鉴·卷七十八·魏纪十·元皇帝下》)

陈寿:“文帝天资文藻,下笔成章,博闻强识,才艺兼该;若加之旷大之度,励以公平之诚,迈志存道,克广德心,则古之贤主,何远之有哉!”(《三国志·魏书·文帝纪》)

《晉書·禮志上》:「大魏三聖相承,以成帝業。武皇帝肇建洪基,撥亂夷險,為魏太祖。文皇帝繼天革命,應期受禪,為魏高祖。上集成大命,清定華夏,興制禮樂,宜為魏烈祖。于太祖廟北為二祧,其左為文帝廟,號曰高祖昭祧,其右擬明帝,號曰烈祖穆祧。三祖之廟,萬世不毀。其餘四廟,親盡迭遷,一如周後稷、文武廟祧之禮。」

阎缵:“魏文帝惧于见废,夙夜自祗,竟能自全。”(《晋书·卷四十八· 阎缵传》)

刘渊:“黄巾海沸于九州,群阉毒流于四海,董卓因之肆其猖勃,曹操父子凶逆相寻。”(《晋书·卷一百一·载记第一》)

李班:“观周景王太子晋、魏太子丕、吴太子孙登,文章鉴识,超然卓绝,未尝不有惭色。何古贤之高朗,后人之莫逮也!”(《晋书·卷一百二十一·载记第二十一》)

葛洪:「自建安之后,魏之武文,送终之制,务在俭薄,此则墨子之道,有可行矣。」(《抱朴子外篇》)

刘勰:「魏文之才,洋洋清绮,旧谈抑,之谓去植千里。然子建思捷而才俊,诗丽而表逸;子桓虑详而力援,故不竞于先鸣。而乐府清越,《典论》辩要,选用短长,亦无懵焉。但俗情抑扬,雷同一响,遂令文帝以位尊减才,思王以势窘益价,未为笃论也。」(《文心雕龙·才略第四十七》)

垣荣祖:「昔曹操、曹丕上马横槊,下马谈论,此于天下可不负饮矣!」(《南齐书 卷二十八 列传第九》)

释道恒:「光武尚能纵严陵之心,魏文全管宁之操,折至尊之高怀,遂匹夫之微志。」(《释文纪》卷八)

沈约:「自魏氏膺命,主爱雕虫,家弃章句,人重异术。又选贤进士,不本乡闾,铨衡之寄,任归台阁。」(《宋书》卷五十五)

萧统:「不追子晋,而事似洛滨之游;多愧子桓,而兴同漳川之赏。漾舟玄圃,必集应、阮之俦;徐轮博望,亦招龙渊之侣。」(《答湘东王求文集及书》)

颜之推:「自昔天子而有才华者,唯汉武、魏太祖、文帝、明帝、宋孝武帝,皆负世议,非懿德之君也。」(《颜氏家训》卷四)

《陈思王庙碑》:「魏高祖文皇帝,绍即四海,光泽五都,负彰魈茫朝宗万国,允文允武,庶绩咸熙,正践升平,时称宁晏。」(《全隋文·卷二十九》)

王勃:「文帝富裕春秋,光应禅让,临朝恭俭,博览坟典,文质彬彬,庶几君子者矣。」(《全唐文·卷一百八十二》)

郝处俊:「昔魏文帝著令,虽有幼主,不许皇后临朝,所以杜祸乱之萌也。」(《资治通鉴 卷第二百二》)

刘知几:「文帝临戎不武,为国好奢,忍害贤良,疏忌骨肉。」(《史通·探赜第二十七》)

李隆基:「叹节气之循环,美君臣之相乐,凡百在会,咸可赋诗,五言纪其日端,七韵成其大数,岂独汉武之殿盛,朝士之连章,魏文之台壮,辞人之并作云尔。」(《端午三殿宴群臣探得神字并序》)

张说:「周文王之为太子也,崇礼不倦;魏文帝之在青宫也,好古无怠,博览史籍,激扬令闻,取高前代,垂名不朽。」(《张燕公集》)

王锴:「文帝八岁能属文,博览古今,贯穿经史。及居帝位,益尚谦和。坐不废书,手不释卷。」(《全唐文 卷八百九十》)

范仲淹:「魏文帝宠立郭妃,谮杀甄后,被发塞口而葬,终有反报之殃。」(《范文正集 卷十五》睦州谢上表)

黄庭坚:「盖世英雄不自知,暮年初志各参差。南阳陇底卧龙日,北固樽前失箸时。霸主三分割天下,宗臣十倍胜曹丕。寒炉夜发尘书读,似覆输筹一局棋。」(《山谷诗集注》)

苏辙:「臣伏观历代帝王,如汉武,魏文,唐徳、文、宣三宗皆工于诗骚杂文,与一时文士比长絜大。至于经纶当世,讲论利害,以文墨尽天下事,则皆不足以仰望先帝之万一。」(《栾城集卷四十七》进御集表)

司馬光:「于禁將數萬眾,敗不能死,生降於敵,既而復歸。文帝廢之可也,殺之可也,乃畫陵屋以辱之,斯為不君矣!」

陈亮:「至于欲使当时累息之民得阔步高谈无危惧之心,未尝不为之三复也,于是时吴蜀争帝,中国庶几乎息肩矣,是以在位七年而谥曰文也。」(《龙川集·卷七》)

耶律楚材:「仲谋服孟徳,孔明倍曹丕。唯晋成全统,平吴混八维。」(《湛然居士集》)

郝经:「丕特负赃胠箧之盗。操死丕直取,自以为可也,乃从容禅让,自以为舜、禹复出,其自欺也甚矣!且轻薄佻靡,未除贵骄公子之习,不矜细行,隳败礼律,刻薄骨肉,自戕本根,乱亡基兆,已在于是。孔明谓为土龙刍狗,宜哉!」(《续后汉书 卷二十六》)

曹丕假借禪讓,逼漢獻帝劉協退位备受非议。直到明朝末年,复社领袖张溥在《汉魏六朝百三家集》的《魏文帝集题辞》中首次对曹丕德政品行作全面评价。张溥写到:“魏王帝业无足称,惟令宦人为官,不得过诸署令。诏群臣不得奏事太后;后族不得常辅政任,石室金策,可宝万世。彼亲见汉室炎隆,女主中人手扑灭之,麦秀黍离,伤心目。霸朝初创,力更旧辄,至待山阳公以不死,礼遇汉老臣杨彪不夺其志,盛德之事,非孟德所及……甄后塘上,陈王豆歌,损德非一。崇华首阳,有余恨焉。”他认为曹丕的施政有可取之处,礼遇汉室君臣,可见并非无德,但杀妻、害弟罪名昭著。

胡应麟:「诗未有三世传者,既传而且煊赫,仅曹氏操、丕、睿耳。然白马名存钟《品》,则彪当亦能诗。又任城武力绝人,仓舒智慧出众。阿瞒何徳,挺育多才?生子如此,孙仲谋辈讵足道哉!」(《诗薮》)

方孝孺:「汉高祖、魏文帝皆中才之主,非有圣智之度,高祖犹能不杀子婴,文帝犹能奉山阳终其身。曾谓武王圣人而忍其君至此乎?吾决知其不然矣。」(《逊志斋集卷四》)

谭嗣同:「若夫汉武帝命所忠求相如遗书,魏文帝诏天下上孔融文章,渐昭风轨,犹无集名。自荀况诸集,编题后人,张融玉海,标目己意,乃始波颓雾靡,不可胜遏。」(《谭嗣同全集》)

王世贞:「自三代而后,人主文章之美,无过于汉武帝、魏文帝者。」(《艺苑疤言》)

王夫之:「曹子建鋪排整飾,立階級以賺人升堂,用此致諸趨赴之客,容易成名,伸紙揮毫, 雷同一律。子桓精思逸韻,以絕人攀躋,故人不樂從,反為所掩。子建以是壓倒阿兄,奪其名譽。實則子桓天才駿發,豈子建所能壓倒耶?曹子建之於子桓,有仙凡之隔, 而人稱子建,不知有子桓,俗論大抵如此。」(《姜齋詩話》)

章太炎:「今人皆谓汉代经学最盛,三国已衰,然魏文廓清谶纬之功,岂可少哉!文帝虽好为文,似词章家一流,所作《典论》,《隋志》归入儒家。纬书非儒家言,乃阴阴家言,故文帝诏书未引一语。岂可仅以词章家目之!」(《经学略说》)

刘师培:「魏文与汉不同者,盖有四焉,书檄之文,骋词以张势,一也。」(《中古文学史论汉魏之际文学变迁》)

魯迅:「他(曹丕)說詩賦不必寓教訓,反對當時那些寓教訓於詩賦的見解,用近代的文學眼光來看,曹丕的一個時代可說是『文學的自覺時代』,或如近代所說是為藝術而藝術的一派。」(《魏晉風度及文章與藥及酒之關係》)

毛泽东:「曹丕也是他(曹操)儿子,也有些才华,但远不如曹操。曹丕在政治上也平庸,可他后来做了皇帝,是魏文帝。历史上所称的‘建安文学’,实际就是集中于他们父子的周围。一家两代人都有才华、有名气,在历史上也不多见哪!」

郭沫若:「曹丕在政治见解上也比乃弟高明得多,而在政治家的风度上有时还可以说是胜过他的父亲。如令宦人为官不得过诸署,禁母后预政,取士不限年资但纠其实、轻刑罚、薄赋税、禁复仇、禁淫祀、罢墓祭、诏营寿陵力求俭朴等等,处处都表示着他是一位旧式的明君典型。」(《郭沫若全集·历史篇·第四卷》)

马植杰:「从曹丕的政治设施来看,也有些不错的。拿曹丕与其他封建帝王相比,尚属中等偏上者。」

叶嘉莹:「魏文帝在即位后,曾下了息兵诏,下了薄税诏,下了轻刑诏。他实在是一个很有理想的皇帝,希望能够把天下治理得更好。但是很可惜,他只做了七年的皇帝就死了,死的时候只有四十岁。」(《汉魏六朝诗讲录》)

曹丕当五官中郎将时,有一次宴請,曹丕问相士朱建平自己的寿命,朱建平說:「您的寿命是八十岁,四十岁时会有小灾难,希望您多加小心。」曹丕果然四十而终,死前认为朱建平的占卜结果是昼夜相加计算的,自己的生命快要结束。(《三国志‧魏志二十九‧方技传》)

曹丕当五官中郎将时,曹操曾经赏赐他百辟刀。

曹丕善擊劍騎射,好博弈彈棋,在《典論》的自敘中更自詡其非凡箭藝,能「左右射」,可謂文武兼備。

曹丕喜爱葡萄和葡萄酒。

魏文帝诏群臣曰:“中国珍果甚多,且复为说蒲萄。当其朱夏涉秋,尚有余暑,醉酒宿醒,掩露而食。甘而不涓,脆而不酸,冷而不寒,味长汁多,除烦解渴。又酿以为酒,甘于曲糵,善醉而易醒。道之固已流涎咽唾,况亲食之邪?南方有桔,酢正裂人牙,时有甜耳。远方之果,宁有匹之者?”(一说出自《与吴监书》)

《与群臣诏》:“南方有龙眼、荔枝,宁比西国葡萄、石蜜乎!」

曹丕与曹植争夺太子之位,后来曹丕得立,曾经喜极失态,抱着辛毗的颈说:“辛君您知道我有多么喜悦吗?”辛毗事后将曹丕的表现告诉女儿辛宪英,时年二十多岁的宪英便感叹地说:“太子是代替君王主理宗庙社稷的人物。代君王行事不可以不怀着忧虑之心,主持国家大事亦不可以不保持戒惧之心,在应该忧戚的时候竟然表现得如此喜悦,又怎会长久呢?魏国又怎能昌盛?”

甄后塘上。文昭皇后甄氏原为袁绍次子袁熙之妻,建安九年城破被俘,曹丕纳之。後曹丕称帝,眷寵文德皇后郭氏,甄氏失宠口出怨言,觸怒曹丕。曹丕下令赐死甄氏,殓时“被发覆面,以糠塞口”。

陈王豆歌。陈思王曹植曾与曹丕争储,曹丕称帝,曹植备受迫害,屡次迁封。《世说新语‧文学》中,曹丕命令曹植七步成诗,否则问罪,曹植才思敏捷,逃过一劫,即著名的《七步诗》故事。

誅殺忠臣鮑勳,鮑勳為曹操大將鮑信之子,為人剛正不阿,卻因私人恩怨遭曹丕殺害。鮑勳被曹丕處死後二十日,曹丕也暴斃逝世。(三國志·魏志十二·鮑勳傳)

曹丕曾下詔給征南將軍夏侯尚說:「你是朕的心腹重將,應當給予你特別的任命(指征南將軍一職)。希望你可以廣施恩德足以令死者享用,實行惠愛令人終身難忘。你可以獨攬威權,擅行賞罰,有殺人或活人的權力。」後來,夏侯尚將這道詔書出示給蔣濟看。蔣濟回到朝廷,曹丕問他:「你在各地聽到和看到的社會風氣、教化都是怎麼樣的?」,蔣濟回答道:「沒有其他善行,只聽到了亡國之語。」,曹丕臉上露出憤怒的表情,問蔣濟這是什麼原因,蔣濟以曹丕頒給夏侯尚的詔書作答:「『作威作福』是《尚書》中明明白白告誡臣子不能做的事情,『天子無戲言』,古人對此非常慎重。請陛下認真考慮。」,曹丕這才明白了蔣濟的意思,下令將頒給夏侯尚的詔書追回。在詔書中提到的「作威作福」,意為獨攬威權,擅行賞罰。這句成語的最早出處是《尚書·洪範》中的「惟闢作福,惟闢作威,惟辟玉食。臣無有作福作威玉食。」,按照這句成語最早的解釋,是只有帝王才能行使的權力,而曹丕卻出現了重大筆誤。難怪蔣濟毫不客氣予以指責,曹丕也只能低頭認錯,乖乖地將詔書收回了。

\subsubsection{黄初}

\begin{longtable}{|>{\centering\scriptsize}m{2em}|>{\centering\scriptsize}m{1.3em}|>{\centering}m{8.8em}|}
  % \caption{秦王政}\
  \toprule
  \SimHei \normalsize 年数 & \SimHei \scriptsize 公元 & \SimHei 大事件 \tabularnewline
  % \midrule
  \endfirsthead
  \toprule
  \SimHei \normalsize 年数 & \SimHei \scriptsize 公元 & \SimHei 大事件 \tabularnewline
  \midrule
  \endhead
  \midrule
  元年 & 220 & \tabularnewline\hline
  二年 & 221 & \tabularnewline\hline
  三年 & 222 & \tabularnewline\hline
  四年 & 223 & \tabularnewline\hline
  五年 & 224 & \tabularnewline\hline
  六年 & 225 & \tabularnewline\hline
  七年 & 226 & \tabularnewline
  \bottomrule
\end{longtable}


%%% Local Variables:
%%% mode: latex
%%% TeX-engine: xetex
%%% TeX-master: "../../Main"
%%% End:

%% -*- coding: utf-8 -*-
%% Time-stamp: <Chen Wang: 2019-12-17 22:25:50>

\subsection{明帝\tiny(226-239)}

\subsubsection{生平}

魏明帝曹叡(204年?-239年1月22日),字元仲,豫州沛国谯县(今安徽省亳州市)人。三国时期曹魏第二任皇帝(226年至239年在位)。魏文帝曹丕长子,母为文昭甄皇后。

黄初三年(222年),曹叡封平原王,黄初七年(226年)五月,魏文帝病重,立曹叡为皇太子,即位于洛阳。曹叡在位期间指挥曹真、司马懿等人成功防御吴、蜀的多次攻伐,并且平定鲜卑,攻灭公孙渊,设置律博士制度,重视狱讼审理,与尚书陈群等人制《魏律》十八篇,是古代法典编纂史上的重大进步。魏明帝在军事,政治和文化方面都颇有建树,但在统治后期大兴土木,广采众女,因此留下负面影响。

景初三年(239年),曹叡病逝于洛阳,时年三十五岁,庙号烈祖,谥号明帝,葬于高平陵。曹叡能诗文,与曹操、曹丕并称魏氏“三祖”,原有集,已散佚,后人辑有其散文二卷、乐府诗十余首。自從曹叡崩後,曹爽掌权,魏帝自此淪爲傀儡。再後曹爽被司馬懿發動高平陵之变斬殺,魏國大權完全落入司馬氏家族手中。

叡生于建安九年(204年?),母亲是文昭甄皇后,甄氏初为幽州刺史袁熙妻子,曹操打败袁绍后,被魏文帝曹丕所纳,甚为得宠,生有曹叡和东乡公主。曹叡从小才智出眾、聰明特異。祖父曹操对此十分惊喜而倍加喜爱,常令他伴随左右。在朝会宴席上,也经常叫他与侍中近臣并列。曹操曾经评价道:“我的家族基业有了你就可以继承三代了。”曹叡好学多识,尤其留意研究律法。

建安二十一年(217年),曹操封魏王,同年东征孙权,曹叡及妹妹东乡公主离开母亲甄氏,与祖母卞夫人,父亲曹丕一起随征江东。

延康元年(220年),曹操病逝,其父曹丕继位魏王,同年五月,十五岁的曹叡被封为武德侯,曹丕作《以侍中郑称为武德侯傅令》,亲自诏令时任侍中的笃学大儒郑称为曹叡的师傅,教授他经学,以此明志。

黄初二年(221年),曹叡被封为齐公,同年八月,其母甄氏因为怨言而被曹丕赐死,葬于邺城,曹叡因为母亲获罪,降为平原侯。黄初三年(222年)三月,曹丕又复其爵位,晋封为平原王。

生母被赐死,曹叡一併受罚被貶为平原侯。起初其父魏文帝认为曹叡先前既有不满,便想立徐姬所生的京兆王曹礼为嗣,因此久不立太子。这期间,曹叡府中来往的家臣官吏、师长、友伴,一律只取品行正直的人充任,互相匡扶、勉励矫正,与卫臻私交甚好,经常一起讨论朝事和书籍。曹丕也曾询问卫臻关于曹叡的情况,卫瑧只是称赞他明理而有德行,不言其他。

据《魏末传》记载,曹叡一次随曹丕狩猎,见到母子两鹿。文帝射杀鹿母,命令曹叡射杀子鹿,曹叡说:“陛下已经杀掉母鹿,臣实在不忍心再杀掉它的孩子。”说完哭泣不已。文帝于是放下弓箭,深感惊奇,而确定立曹叡为太子的心意。

黄初三年(222年)三月,曹叡升为平原王,后来曹丕下诏将其过继给郭皇后为子,进一步确定嫡长子的地位。然而曹叡因其母非善终,内心愤愤不平,后来才开始恭敬地侍奉嫡母,每日早晚都往皇后宫中定省问安,郭皇后也因自己无子,对曹叡慈爱有加。除了曹丕为曹叡诏令郑称为师,平原王府中还配置高堂隆为平原王傅,毌丘俭、何曾、吉茂等一干人等为文学属官。黄初四年,曹丕为曹叡聘河内世家大族虞氏为平原王妃,又选河内毛氏入宫,曹叡十分宠爱,出入都与其同乘舆辇。

226年,五月十六日丙辰(6月28日)魏文帝病危,立平原王曹叡為太子,召曹真、曹休、陳群、司馬懿,并受遗诏辅佐嗣主。十七日丁巳(6月29日)崩于嘉福殿。曹叡繼位,是為魏明帝。

明帝登基后首先必须对抗内外敌人的攻击,226年八月孙权攻江夏和襄阳、227年孟达反、到234年为止诸葛亮五次进攻曹魏、234年孙权攻合肥;明帝重用滿寵處理这些内外战争,亦重用曹真、張郃、司马懿等名將与诸葛亮作战。235年诸葛亮死后,魏蜀边境上的情况有所减缓,明帝开始在洛阳大建宫殿,常用人力、物力,大臣楊阜、高堂隆等對此一再勸練,明帝雖多未採納,但也不因此問罪臣屬。同年,他将养子曹芳封为齐王。

237年,聽從高堂隆的建議,發布《景初暦》,是歲將青龍五年春三月改為景初元年夏四月,同年辽东公孙渊造反,自立为燕王,明帝令司马懿攻辽东,司马懿遂带兵四万,和夏侯霸等人出征辽东,大破燕军,杀公孙渊,成功收复辽东。

从238年冬开始,魏明帝的健康开始恶化。239年初,魏明帝病重,曹叡本意让燕王曹宇为大将军,曹献、曹爽、曹肇、秦朗共同辅政,但曹宇一直不接受。于是曹叡单独召见刘放、孙资到其床边问话,问道“燕王为何一直不接受大将军的安排?”刘放和孙资回答:“燕王实在是自己知道不堪大任所以推辞”,曹叡又问:“曹爽可以代曹宇为大将军么?”刘放和孙资表示赞同,同时又多次强调应该迅速召见太尉司马懿来辅助朝纲,曹叡答应并令刘放起诏书。刘放、孙资退下之后,曹叡的想法突然改变,宣诏让司马懿不要入宫,过一段时间曹叡见到刘放、孙资说:“我同意召见司马懿,但是曹肇等人却让我不要这样做,差点坏了我的大事!”于是再次起草诏书,命曹爽、刘放、孙资一同接受诏令,同时免去曹宇、夏侯献、曹肇、秦朗等人的官职。

景初三年正月初一(239年1月22日),司马懿率师从辽东回到河内郡驻扎。明帝传令把他急招入卧室,拉着他的手嘱咐说:“终于等到你来,现在把后事托付给您,和大将军曹爽共佐曹芳。我在死前能见到你,也没什么遗憾的了。”又把齐王曹芳和秦王招来,嘱托司馬懿照顾。当天,明帝驾崩于洛阳宫嘉福殿,年仅三十五岁,《三国志》作三十六岁。。《三国志》载魏明帝崩于嘉福殿,《魏書》载他崩于九龙前殿。正月廿七癸丑日(2月17日),葬高平陵。

曹叡生母甄氏原为袁绍次子袁熙婦,據《三國志》卷三《明帝紀》記載魏明帝去世時“時年三十六”。裴松之在《三國志注》中計算曹叡年齡時,認為甄氏在建安九年(204年)八月曹操攻占邺城后才為曹丕所納,到景初三年正月初一丁亥日(239年1月22日)曹叡死時最多也只有虚岁三十五歲,不能計為三十六歲。由此如卢弼等人就認為曹叡或許是袁熙之子,陳壽故意對於年龄曲笔。

魏明帝曾經下令由盧毓來推舉官吏,並要求不要只看候選人的名氣,而要看他們的品行與能力。明帝表示,名氣就像是圖畫的餅一樣,根本無用,無法充飢。這就是「畫餅充飢」的典故。

陈寿:“明帝沉毅断识,任心而行,盖有君人之至概焉。于时百姓彫弊,四海分崩,不先聿修显祖,阐拓洪基,而遽追秦皇、汉武,宫馆是营,格之远猷,其殆疾乎!”(《三国志·魏书·明帝纪第三》)

孙权:“及操子丕,桀逆遗丑,荐作奸回,偷取天位,而叡么麽,寻丕凶迹,阻兵盗土,未伏厥诛。”(《三国志·卷四十七·吴书二·吴主传第二》)

孙盛:“魏明帝天资秀出,立发垂地,口吃少言,而沉毅好断。初,诸公受遗辅导,帝皆以方任处之,政自己出。而优礼大臣,开容善直,虽犯颜极谏,无所摧戮,其君人之量如此之伟也。然不思建德垂风,不固维城之基,至使大权偏据,社稷无卫,悲夫!”

刘晔:“秦始皇、汉孝武之俦,才具微不及耳。”(《世说新语》)

陆逊:“选用忠良,宽刑罚,布恩惠,薄赋省役,以悦民心,其患更深於操时。”(《三国志·吴书·张顾诸葛步传第七》)

钟会:“烈祖明皇帝奕世重光,恢拓洪业。”(《三国志·魏书·王毌丘诸葛邓锺传第二十八》)

阎缵:“及至明帝,因母得罪,废为平原侯,为置家臣庶子,师友文学,皆取正人,共相匡矫。兢兢慎罚,事父以孝,父没,事母以谨,闻于天下,于今称之。”(《上书理湣怀太子(司马遹)之冤》)

王沈:“好学多识,特留意于法理。”(《魏书》)

裴松之:“魏明帝一时明主。”

司马光:“汉主寿常慕汉武,魏明之为人。”(《资治通鉴·卷第九十六》)“帝沈毅明敏,任心而行,料简功能,屏绝浮伪。行师动众,论决大事,谋臣将相,咸服帝之大略。性特强识,虽左右小臣,官簿性行,名迹所履,及其父兄子弟,一经耳目,终不遗忘。”(《资治通鉴·卷第七十四》)

郭威:“汉高祖为义帝发丧,魏明帝正禅陵尊号,一时达礼,千古所称。”(《全唐文·卷一百二十三》)

胡应麟:“诗未有三世传者,既传而且煊赫,仅曹氏操、丕、睿耳。”(《诗薮》)

吕思勉:“魏文帝本来无甚才略。死后,儿子明帝继立,荒淫奢侈,朝政更坏。”(《中国通史:后汉的分裂和三国》)

蔡东藩:“曹叡奢淫无度,违理蔑伦,种种荒谬,俱足亡国,而反得平定辽东,擒斩公孙渊父子,是所谓天夺之鉴,而益其疾也。”(《后汉演义》)

马植杰:“综观曹叡之行事,优缺点各占一半,其优点是善为军计、明察断狱、比较能容人直谏。曹叡在容受直言、不杀谏臣方面,在古代封建君主中是少见的,这算是他的特色。曹叡的最大缺点是奢淫过度,还有一个重要的失误,则在确定继承人和辅政大臣方面。”(《魏的政治与司马氏专政》)

\subsubsection{太和}

\begin{longtable}{|>{\centering\scriptsize}m{2em}|>{\centering\scriptsize}m{1.3em}|>{\centering}m{8.8em}|}
  % \caption{秦王政}\
  \toprule
  \SimHei \normalsize 年数 & \SimHei \scriptsize 公元 & \SimHei 大事件 \tabularnewline
  % \midrule
  \endfirsthead
  \toprule
  \SimHei \normalsize 年数 & \SimHei \scriptsize 公元 & \SimHei 大事件 \tabularnewline
  \midrule
  \endhead
  \midrule
  元年 & 227 & \tabularnewline\hline
  二年 & 228 & \tabularnewline\hline
  三年 & 229 & \tabularnewline\hline
  四年 & 230 & \tabularnewline\hline
  五年 & 231 & \tabularnewline\hline
  六年 & 232 & \tabularnewline\hline
  七年 & 233 & \tabularnewline
  \bottomrule
\end{longtable}

\subsubsection{青龙}

\begin{longtable}{|>{\centering\scriptsize}m{2em}|>{\centering\scriptsize}m{1.3em}|>{\centering}m{8.8em}|}
  % \caption{秦王政}\
  \toprule
  \SimHei \normalsize 年数 & \SimHei \scriptsize 公元 & \SimHei 大事件 \tabularnewline
  % \midrule
  \endfirsthead
  \toprule
  \SimHei \normalsize 年数 & \SimHei \scriptsize 公元 & \SimHei 大事件 \tabularnewline
  \midrule
  \endhead
  \midrule
  元年 & 233 & \tabularnewline\hline
  二年 & 234 & \tabularnewline\hline
  三年 & 235 & \tabularnewline\hline
  四年 & 236 & \tabularnewline\hline
  五年 & 237 & \tabularnewline
  \bottomrule
\end{longtable}

\subsubsection{景初}

\begin{longtable}{|>{\centering\scriptsize}m{2em}|>{\centering\scriptsize}m{1.3em}|>{\centering}m{8.8em}|}
  % \caption{秦王政}\
  \toprule
  \SimHei \normalsize 年数 & \SimHei \scriptsize 公元 & \SimHei 大事件 \tabularnewline
  % \midrule
  \endfirsthead
  \toprule
  \SimHei \normalsize 年数 & \SimHei \scriptsize 公元 & \SimHei 大事件 \tabularnewline
  \midrule
  \endhead
  \midrule
  元年 & 237 & \tabularnewline\hline
  二年 & 238 & \tabularnewline\hline
  三年 & 239 & \tabularnewline
  \bottomrule
\end{longtable}


%%% Local Variables:
%%% mode: latex
%%% TeX-engine: xetex
%%% TeX-master: "../../Main"
%%% End:

%% -*- coding: utf-8 -*-
%% Time-stamp: <Chen Wang: 2021-11-01 11:35:44>

\subsection{少帝曹芳\tiny(239-254)}

\subsubsection{生平}

曹芳(232年-274年),字蘭卿,魏明帝養子,是三国時曹魏第三代皇帝,在位15年。在《三國志》中與曹髦、曹奐合稱三少帝。

曹芳是曹魏在位最久的君主,但也是曹魏第一位傀儡皇帝,實權先後由曹爽、司馬懿和司馬師掌握。

魏明帝的亲生儿子全部夭折,曹芳是其养子。但自小在宮中成長的曹芳,不知出自何宗室,出生來歷一概不詳。《魏氏春秋》記載,曹芳應該是任城王曹楷之子。

魏青龙三年,封为齐王,239年被立为太子,同年登基即帝位,年僅八歲。

第二年,即240年,改年号为正始,由大将军曹爽和太傅司马懿共同辅政。曹爽主政时,架空司马懿、隔离郭太后、起用新人。

魏正始十年(249年),司马懿發動政變,起兵控制雒邑,史稱高平陵之變。71歲的司马懿族滅曹爽,从此由司馬氏独掌军国大权,同年改元为嘉平。

251年,司马懿死后,長子司馬師受命總攬朝政,曹芳聯合李豐、張緝、夏侯玄等意圖罷黜司馬師,欲改立夏侯玄為大將軍,然而三人被司馬師搜出“衣帶詔”,夏侯玄、李豐、張緝受腰斬滅族,並且盡滅李氏、夏侯氏、張氏。张缉女张皇后也被废位。

254年,姜维进攻陇右,司马昭从许昌征还京城洛阳,准备前去讨伐姜维。曹芳前往平乐观慰问司马昭部队,中领军和大臣想趁着曹芳和司马昭会面的机会杀死司马昭,夺其军队讨伐司马师,但曹芳不敢这么做,不過图谋杀司马昭一事卻為司马师所得知,司马师於是决定废帝,派郭太后从父长水校尉郭芝告知郭太后。当时曹芳正与太后相对而坐,郭芝对曹芳说:“大将军想废了陛下,立彭城王曹据。”曹芳起身而去,太后也不悦,但郭芝说司马师决心已下,勒兵于外。太后说想见司马师,為郭芝所拒。郭太后只得取玺绶给郭芝,郭芝再给司马师。司马师以太后令告知群臣,由群臣上奏郭太后,謂曹芳年长不亲政、沉迷女色、废弃讲学、弃辱儒士、与优人、保林等淫乱作乐,并弹打进谏的清商令狐景、清商丞庞熙,乃至用烧铁重伤令狐景身体、太后丧母时不尽礼等罪,请依霍光故事废曹芳。曹芳因此被废,當年22歲,仍按登基前为齐王,另立明帝侄曹髦为帝。曹芳被迁居西宫,乘王车,泣别太后,从太极殿南出宫,数十大臣相送。在联署上奏废帝时,署名第一的太尉司马孚很悲伤,哭得尤其多。

此后,司马师、司马昭相继掌权。

266年,晋朝立国,曹芳被封为邵陵县公,274年逝世,享年43歲。谥号厉,故又称邵陵厉公。

曹芳登基后,为避讳他的名字“芳”,而将位于洛阳北宫的皇宫花园“芳林苑”改名为“华林苑”。后来南北朝分裂,建业、长安的皇宫花园都因循旧制命名为华林苑。

滕胤:“曹芳暗劣,而政在私门,彼之民臣,固有离心。”(《资治通鉴·卷七十六》)

毌丘俭:“懿每叹说齐王自堪人主,君臣之义定。奉事以来十有五载,始欲归政,按行武库,诏问禁兵不得妄出。师自知奸慝,人神所不祐,矫废君主,加之以罪。”(《三国志·魏书·王毌丘诸葛邓锺传第二十八》)

郭太后废曹芳诏令:“皇帝芳春秋已长,不亲万机,耽淫内宠,沈漫女德,日延倡优,纵其丑谑;迎六宫家人留止内房,毁人伦之叙,乱男女之节;恭孝日亏,悖慠滋甚,不可以承天绪,奉宗庙。”(《三国志·卷四·魏书四·三少帝纪第四》)

陈寿:“古者以天下为公,唯贤是与。后代世位,立子以适;若适嗣不继,则宜取旁亲明德,若汉之文、宣者,斯不易之常准也。明帝既不能然,情系私爱,抚养婴孩,传以大器,托付不专,必参枝族,终于曹爽诛夷,齐王替位。”(《三国志·卷四·魏书四·三少帝纪第四》)

梁章钜:“齐王临御之初,即罢宫室工作,免官奴婢六十以上为良人,出内府金银销冶以供军用;二年通《论语》,五年通《尚书》,七年通《礼记》,三祀孔子,以颜子配;良法美政,史不绝书。”(《三国志集解》)

何焯:“芳临御数载,非若昌邑始征,若果君德有阙,播恶于众,师何难执以为辞?今称太后之令,发床第之私,有以知其为诬矣。”(《三国志集解》)

\subsubsection{正始}

\begin{longtable}{|>{\centering\scriptsize}m{2em}|>{\centering\scriptsize}m{1.3em}|>{\centering}m{8.8em}|}
  % \caption{秦王政}\
  \toprule
  \SimHei \normalsize 年数 & \SimHei \scriptsize 公元 & \SimHei 大事件 \tabularnewline
  % \midrule
  \endfirsthead
  \toprule
  \SimHei \normalsize 年数 & \SimHei \scriptsize 公元 & \SimHei 大事件 \tabularnewline
  \midrule
  \endhead
  \midrule
  元年 & 240 & \tabularnewline\hline
  二年 & 241 & \tabularnewline\hline
  三年 & 242 & \tabularnewline\hline
  四年 & 243 & \tabularnewline\hline
  五年 & 244 & \tabularnewline\hline
  六年 & 245 & \tabularnewline\hline
  七年 & 246 & \tabularnewline\hline
  八年 & 247 & \tabularnewline\hline
  九年 & 248 & \tabularnewline\hline
  十年 & 249 & \tabularnewline
  \bottomrule
\end{longtable}

\subsubsection{嘉平}

\begin{longtable}{|>{\centering\scriptsize}m{2em}|>{\centering\scriptsize}m{1.3em}|>{\centering}m{8.8em}|}
  % \caption{秦王政}\
  \toprule
  \SimHei \normalsize 年数 & \SimHei \scriptsize 公元 & \SimHei 大事件 \tabularnewline
  % \midrule
  \endfirsthead
  \toprule
  \SimHei \normalsize 年数 & \SimHei \scriptsize 公元 & \SimHei 大事件 \tabularnewline
  \midrule
  \endhead
  \midrule
  元年 & 249 & \tabularnewline\hline
  二年 & 250 & \tabularnewline\hline
  三年 & 251 & \tabularnewline\hline
  四年 & 252 & \tabularnewline\hline
  五年 & 253 & \tabularnewline\hline
  六年 & 254 & \tabularnewline
  \bottomrule
\end{longtable}


%%% Local Variables:
%%% mode: latex
%%% TeX-engine: xetex
%%% TeX-master: "../../Main"
%%% End:

%% -*- coding: utf-8 -*-
%% Time-stamp: <Chen Wang: 2019-12-17 22:30:20>

\subsection{高贵乡公\tiny(254-260)}

\subsubsection{生平}

曹髦(241年11月15日-260年6月2日),字彥士,三國時期曹魏第四代皇帝(254年-260年在位),曹霖之子,魏文帝曹丕孙。在《三國志》中與曹芳、曹奐合稱三少帝。

正始五年(244年),封郯县高貴鄉公。

254年,司馬師廢掉時年23歲曹芳的皇位,打算立曹操之子彭城王曹据为皇帝。郭太后指出曹据於礼是她(丈夫)的叔叔,曹据若继皇位,則太后之位將免除,因此易魏文帝曹丕之孫高贵乡公曹髦为帝,作为其无嗣的伯父曹叡魏明帝的后嗣,改元「正元」;曹髦當年12歲,實權先後由司馬師和司馬昭掌握。

司馬師曾問鍾會曹髦的能力,鍾會回答:「文同陳思,武類太祖」(文采如同陳思王曹植,武可比太祖曹操)。司马师说若果然如此,是社稷之幸。但隨著小皇帝長大,曹髦對專權的司馬昭日益不滿。不久他寫了一首《黃龍歌》,司馬昭發現,起了戒心。

甘露五年(260年)五月曹髦召見王沈、王經、王業等三人,憤慨說道:「司馬昭之心,路人皆知也!朕不能坐受廢辱,今日當與卿等自出討之。」不顧郭太后及眾臣的反對,率領宮人三百餘人討伐司馬昭。王沈與王業先行,向司馬昭通風報信。司馬昭旋即派兵入宮鎮壓,雙方在宮內東止車門相遇。中護軍賈充在南闕下,率軍迎戰曹髦。賈充命令成濟殺曹髦,成濟一劍刺穿曹髦胸膛,曹髦斷氣,當場死在車上,血流滿地,仅虚岁20歲,史稱司馬昭弒君。

隨後,司馬昭假借郭太后的名義下詔“高贵乡公悖逆不道,自陷大祸,依汉昌邑王罪废故事,以民礼葬”。但在司马昭叔父司马孚请求下,以王礼下葬曹髦于洛阳西北三十里瀍涧之滨,仅下车数乘、不设旌旐,百姓相聚而观,说:“这就是前日所杀的天子。”有人甚至掩面而泣,悲伤不能自已。由於輿論憤憤不平,司馬昭遂在事發後的20天將弒君罪狀全推給成濟,称成济违背自己命令杀死曹髦,以「大逆不道」罪誅殺成濟一族,斷成氏一脈。司馬昭後立曹奐為曹魏皇帝。

曹髦擅長寫詩文。他的繪畫藝術也相當不錯,是一個善於琴棋書畫的才子。


\subsubsection{正元}

\begin{longtable}{|>{\centering\scriptsize}m{2em}|>{\centering\scriptsize}m{1.3em}|>{\centering}m{8.8em}|}
  % \caption{秦王政}\
  \toprule
  \SimHei \normalsize 年数 & \SimHei \scriptsize 公元 & \SimHei 大事件 \tabularnewline
  % \midrule
  \endfirsthead
  \toprule
  \SimHei \normalsize 年数 & \SimHei \scriptsize 公元 & \SimHei 大事件 \tabularnewline
  \midrule
  \endhead
  \midrule
  元年 & 254 & \tabularnewline\hline
  二年 & 255 & \tabularnewline\hline
  三年 & 256 & \tabularnewline
  \bottomrule
\end{longtable}

\subsubsection{甘露}

\begin{longtable}{|>{\centering\scriptsize}m{2em}|>{\centering\scriptsize}m{1.3em}|>{\centering}m{8.8em}|}
  % \caption{秦王政}\
  \toprule
  \SimHei \normalsize 年数 & \SimHei \scriptsize 公元 & \SimHei 大事件 \tabularnewline
  % \midrule
  \endfirsthead
  \toprule
  \SimHei \normalsize 年数 & \SimHei \scriptsize 公元 & \SimHei 大事件 \tabularnewline
  \midrule
  \endhead
  \midrule
  元年 & 256 & \tabularnewline\hline
  二年 & 257 & \tabularnewline\hline
  三年 & 258 & \tabularnewline\hline
  四年 & 259 & \tabularnewline\hline
  五年 & 260 & \tabularnewline
  \bottomrule
\end{longtable}


%%% Local Variables:
%%% mode: latex
%%% TeX-engine: xetex
%%% TeX-master: "../../Main"
%%% End:

%% -*- coding: utf-8 -*-
%% Time-stamp: <Chen Wang: 2021-11-01 11:35:58>

\subsection{元帝曹奂\tiny(260-265)}

\subsubsection{生平}

魏元帝曹奂(246年-302年),原名璜,字景明,是曹魏最后一個皇帝,260年到266年在位。

曹奂是燕王曹宇之子,曹操之孙,初封常道乡公。

甘露五年(260年)魏帝曹髦试图从司马氏手中夺回权力失敗,兵敗被杀,丞相司马昭派儿子使持节行中护军中垒将军司马炎迎立曹璜为皇帝,改名曹奂,入继为堂兄魏明帝曹叡的儿子。曹奂实际上毫无权力,在大臣和军队中也没有任何势力,完全是司马昭的傀儡。

曹奂在位期间,263年曹魏大将邓艾和鍾会伐蜀汉,蜀汉灭亡。伐蜀期间,丞相司马昭以伐蜀之功被晋升为晋公,相国,加九锡。

蜀汉灭亡没多久,丞相司马昭又进爵晋王,不久去世,司马炎于咸熙二年(265年)篡位,魏亡。

魏亡後,曹奐被封為陳留王,並遷居邺城。出城時,太傅司馬孚握著他的手說:「我到死都是大魏的忠臣。」

晋武帝司马炎仿漢獻帝劉協例,允许曹奐仍保有皇帝仪仗、用皇家礼仪祭祖、不以臣下自称。

晉惠帝太安元年(302年)曹奐死於許昌,享年五十七歲,以皇帝礼下葬,谥号元皇帝。

曹奐的後人沒有在官方的紀錄出現,而史料中亦未记载他有沒有後人。由於他離世時正值八王之亂,很多王公士族在當時均為亂軍所殺,並且紀錄可能在那段期間被燒毀。東晉以降,陳留王由曹操玄孫曹勱及其後裔承襲,但無法得知他们和曹奐的關係。邺城遗址附近一直有相传为曹奂墓的土墩,后来考古发掘证实并非曹奂墓。


\subsubsection{景元}

\begin{longtable}{|>{\centering\scriptsize}m{2em}|>{\centering\scriptsize}m{1.3em}|>{\centering}m{8.8em}|}
  % \caption{秦王政}\
  \toprule
  \SimHei \normalsize 年数 & \SimHei \scriptsize 公元 & \SimHei 大事件 \tabularnewline
  % \midrule
  \endfirsthead
  \toprule
  \SimHei \normalsize 年数 & \SimHei \scriptsize 公元 & \SimHei 大事件 \tabularnewline
  \midrule
  \endhead
  \midrule
  元年 & 260 & \tabularnewline\hline
  二年 & 261 & \tabularnewline\hline
  三年 & 262 & \tabularnewline\hline
  四年 & 263 & \tabularnewline\hline
  五年 & 264 & \tabularnewline
  \bottomrule
\end{longtable}

\subsubsection{咸熙}

\begin{longtable}{|>{\centering\scriptsize}m{2em}|>{\centering\scriptsize}m{1.3em}|>{\centering}m{8.8em}|}
  % \caption{秦王政}\
  \toprule
  \SimHei \normalsize 年数 & \SimHei \scriptsize 公元 & \SimHei 大事件 \tabularnewline
  % \midrule
  \endfirsthead
  \toprule
  \SimHei \normalsize 年数 & \SimHei \scriptsize 公元 & \SimHei 大事件 \tabularnewline
  \midrule
  \endhead
  \midrule
  元年 & 264 & \tabularnewline\hline
  二年 & 265 & \tabularnewline
  \bottomrule
\end{longtable}


%%% Local Variables:
%%% mode: latex
%%% TeX-engine: xetex
%%% TeX-master: "../../Main"
%%% End:


%%% Local Variables:
%%% mode: latex
%%% TeX-engine: xetex
%%% TeX-master: "../../Main"
%%% End:
