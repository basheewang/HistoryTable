%% -*- coding: utf-8 -*-
%% Time-stamp: <Chen Wang: 2021-11-01 11:42:25>


\section{曹魏\tiny(220-265)}

\subsection{简介}

魏(220年12月10日-266年2月8日,史称曹魏、魏朝)是中國歷史上東漢末年三國之中据有北方及中原的政權。始於220年曹丕逼迫漢獻帝劉協禪讓帝位,篡漢為魏,因承繼漢朝,故具法統地位。至266年魏又被司馬炎篡奪,改號為晉。

曹操受封魏公時,治所在東漢時期魏郡所在地的鄴,因此漢獻帝封他為「魏公」建立诸侯国——魏国,且如同汉朝初期诸侯王制度可以设置丞相以下百官,之后又进封「魏王」并以卞氏为魏国王妃,以曹操之女为魏国公主,后来曹操之子曹丕篡漢时便以「魏」為國號。又因为是曹氏政权,故史稱「曹魏」,以區別於其他名「魏」的政權。

魏是三國時期最為強大,領土最遼闊的政權,灭蜀汉前疆域達到近300萬平方千米[來源請求]。263年,魏軍攻滅蜀汉,同年佔領廣州,至此曹魏疆域達到全盛,约400萬平方千米。由於曹魏盤踞中原,所以人口也是三国中最多。期間最重要的政治改革有陳群的九品中正制,對魏晉時代之政治產生深遠影響。

曹魏未立國前,東漢已進入群雄割據的時代。建安元年(196年)曹操迎漢獻帝到許昌,挾天子以令諸侯,迫令各地割據勢力必須遵奉曹操,而軍事方面則選編精銳,組成一批強大的騎兵勁旅——虎豹騎,在平定中原的重要戰役中屢次建功,在掌握軍政的發展下開啟魏晉南北朝霸府政治的先河,為後來的篡漢立魏打下了基礎。200年的官渡之戰後控制了中原北方地區大部分,並準備南下一舉攻佔全國。

但由於曹軍未適應到南方的地理環境,在水土不服,天氣陡轉等因素下,於在208年的赤壁之戰中敗於孫權與劉備的南方聯軍,退守北方。但由於曹軍所傷多為原劉表麾下水軍與降軍,因此孫劉勢力亦無法撼動曹操政權。由於曹操年事已高,於是終其一生只控制了中原一帶。其後漢獻帝封曹操為魏王,打破了漢高祖所訂的白馬之盟,220年,曹操去世後由次子曹丕襲位。

曹丕逼迫漢獻帝禪讓,篡漢為帝,定國號魏,並定都洛陽(或称曹魏五都),曹魏正式建立。曹操雖未稱帝,但曹丕稱帝后追尊他為魏武帝(廟號太祖)。漢中王劉備於221年在成都自立為帝,國號漢,史稱蜀漢。最後孫權也於229年在建業稱帝國號吳,至此三國正式形成。

魏文帝曹丕稱帝七年就去世。曹丕死後,長子曹叡即位,是為魏明帝。曹魏朝廷此時分為兩大派,一是以曹真和曹休為主的曹氏一族、二是以司馬懿世家和賈逵世家為主的新勢力,日漸形成嚴重的對立,埋下日後的高平陵之變的種子。而夏侯惇和夏侯渊死後,夏侯世家人才能力沒落,漸漸遠離權勢。導致司馬世家勢力龐大,夏侯世家無法抗衡,日後高平陵之變的發生後,夏侯世家遭流放邊境,或投降蜀漢。

曹操出身寒族,且與閹宦有關,不以儒學為務,與當時的豪族、士大夫不同。曹操曾下「求才三令」,強調重才不重德,並以法家之術為治,要摧破豪族的儒學。曹操為一代梟雄,不僅得到眾多寒族人才支持,也得到部份豪族士大夫支持,如荀彧、荀攸。荀彧更為曹操引進不少士大夫階層的人才。官渡之戰,曹勝袁敗,士大夫豪族不得不暫時忍耐屈服,卻伺機恢復。終於他們支持出身士族的司馬懿,向曹氏奪回政權。

曹魏都是在與蜀漢、孫吳的戰事中度過,如蜀漢的諸葛亮發動多次北伐攻打魏國,曹叡多次力拒來犯守護國土。太傅司馬懿在諸葛亮北伐戰事中立下不少戰功,在曹魏的政治地位漸漸提升,直至高平陵之變,司馬懿利用兵變,剷除了曹家宗室的曹爽,导致夏侯霸投靠蜀漢,后其子司马师在政治斗争中又铲除了夏侯家的夏侯玄,司馬氏家族權傾全朝成為新的霸府,司马师、司馬昭兄弟成為朝廷中最有權勢的朝臣,能擅自廢立皇帝。曹魏立国以来宗室受到限制,大部分实际上被囚禁在邺城,故无法反抗司马氏。驻守扬州的地方军事长官王凌(司马懿时期)、毌丘俭(司马师时期)、诸葛诞(司马昭时期)先后起兵反抗司马氏,后两者还联合孙吴为外援,但皆败亡。魏帝曹髦不甘司馬氏威脅自己帝位,亲自攻打司馬昭,司马昭命令親信賈充派成濟弒害曹髦,事後僅成濟被處死,而司馬氏家族則沒受牽連,因此曹魏於此時名存實亡。

隨著蜀漢國力日下,263年魏國司馬氏展開攻漢計劃,派遣鍾會、鄧艾、諸葛緒等等攻漢,結果漢後主劉禪出降,蜀漢亡國,隨後司馬昭便平定由鍾會、姜維、劉璿等等蜀漢殘黨與部份鍾會勢力所發起的鍾會之亂。後司馬昭死,其子司馬炎襲晉王、相國位,於265年篡魏自立,國號晉,曹魏滅亡。

曹奂及后人被封为陈留王,在晋朝受到很高的待遇,陈留王国历经4朝,经西晋、东晋、刘宋,一直传至南朝南齐,国祚之长在历史上实为罕见。

曹魏於曹操死前,兵力約30-45萬,曹操死後,曹丕濫兵用戰,興建修房,死了不少的精兵,據當時的史書記載,兵力約13-16萬,為曹魏的低落時期,到了曹叡統治,由於推動多子政策,兵力大增到40-47萬,到司馬家族主政後,約到60-65萬,為最高峰。

曹魏雖然是以軍事起家,但曹魏一族在文學上具有相當成就,如曹操和其子曹丕和曹植都擅於寫詩,時稱三曹,後世又稱建安文學。後期君主也頗有藝術造詣,如曹操其孫曹叡擅長詩賦;曹髦擅長詩文、繪畫,被譽為才子。

魏、漢、吳三國中以曹操最重視農業(用毛玠兵農合一之策),其中以曹魏人口最多,屯田墾荒的面積最廣。

曹魏重視農業的另一實證,是其大力興修水利,其工程的規模和數量在三國中首屈一指。如233年,關中一帶闢建渠道,興修水庫,一舉改造三千多頃鹽鹼地,所獲使國庫大為充實。再如曹魏在河南的水利工程,其成果使糧食產量倍增。

諸葛亮北伐時期對曹魏的經濟影響甚是巨大,從辛毗、楊阜的奏章中,不止一次提到諸葛亮北伐造成的經濟困難。

曹魏对藩王监视防范,藩王们求为百姓而不可得,后大多被囚禁于邺城。曹叡及后来的执政大臣曹爽先后拒绝曹植、曹冏任用宗室的建议,终于使得后来皇权旁落司马氏之手时曹魏诸王几乎无力反抗。

司马炎建立晋朝后,立即解除对曹氏族人的禁锢,同时因为吸取曹魏的教训而大封宗室为实权藩王委以重任,导致了后来的八王之乱。

\subsection{武帝曹操生平}

曹操(155年-220年3月15日),字孟德,小名吉利,小字阿瞞,沛國譙縣(今安徽亳州)人。東漢末年著名軍事家、政治家、文學家和詩人,三国时代曹魏奠基者和主要缔造者。曹操於建安年間權傾天下,在世时官至丞相,爵至魏王,谥号武王。其子曹丕受禅稱帝後,追尊其為武皇帝,廟號太祖。

东汉永寿元年(155年),曹操出生于沛国谯县(今安徽亳州)的一个宦官家族,养祖父是宦官曹腾,历侍四代天子,汉桓帝时封为费亭侯。父親曹嵩是曹騰養子,汉灵帝时官至太尉。《三國志》中記載曹操远祖是漢代初期的相国曹参,但是裴松之注曰:“嵩,夏侯氏之子,夏侯惇之叔父。太祖于惇为从父兄弟”;曹操身世眾說紛紜。

曹操少年时机警过人,通权谋机变,十岁时曾经在家乡击退鳄鱼。但行为放荡不羁(如曾和袁绍偷新娘),不为世人看重。只有橋玄、何顒、李瓒和王俊认为曹操是非常之人,将来一定会安定天下。当时曹操还默默无闻,橋玄建議曹操去結交当时的名士許劭,以提高名望。于是曹操就去拜访许劭,向他询问说:“我是怎样的人?”许劭鄙视曹操的为人,不肯回答,曹操找到机会威胁许劭,许劭不得已,给曹操做出了“君清平之奸贼,乱世之英雄。”的评价(《異同雜語》作:“治世之能臣,乱世之奸雄”),曹操聽罷大笑,並逐渐知名。建安七年(202年)曹操北征,路过桥玄之墓,下令祭祀桥玄,并且写了悼文。

曹操早年就喜爱武藝同時也很有才華,曾经潜入中常侍张让家,被张让发觉后,手舞著戟越墙逃出,全身而退。又博览群书,尤其喜欢兵法,曾经抄录古代诸家兵法韬略,还有注释《孙子兵法》的《孫子略解》;是為《孫子兵法》最早的註釋本。

初入官場(174年-189年):曹操年轻时是名诸生,熹平三年(174年),二十岁的曹操察举孝廉成为郎官,不久司马芳推荐曹操为洛陽北部尉,但曹操想担任洛阳令,不过负责人事的选部尚书梁鹄并没有遵从曹操的意愿。曹操上任洛阳北部尉数月,宦官蹇硕的叔叔违禁夜行,被曹操依律棒杀。这使曹操得罪宦官集团,可是曹操是依法而行,这些人又无法中伤诋毁曹操,只好转而称赞他,举荐他去担任地方官。177年,曹操被任命为頓丘令,第二年,曹操因堂妹夫滁強侯宋奇被宦官誅殺,受到牽連,被免去官職。其後,在洛陽無事可做,回到家鄉譙縣閒居。

180年,曹操又被朝廷征召,任命为議郎。此前(168年),大将军窦武、太傅陈蕃谋劃诛杀宦官,不料反为阉党所害。曹操上书陈述窦武等人为官正直而遭陷害,致使奸邪之徒满朝,而忠良之人却得不到重用的情形,言辞恳切,没有被漢靈帝采纳。而后,曹操又多次上书进谏,雖偶有成效,但是東漢朝政日益腐败,曹操知道无法匡正。

中平元年(184年)二月,太平道首领张角聚众起义,黄巾之乱爆发,朝廷任命曹操为騎都尉,前往颍川郡镇压叛乱。由於鎮壓黃巾軍有功,升任濟南相,任職後罷免貪污官員約八成,並且嚴令禁止當時盛行的宗教迷信。据说因为曹操当政素称严明,济南郡一帶作乱之徒听说曹操要来了,都纷纷潜逃到别的郡县。曹操被任命为東郡太守,但是曹操没有就任,自称在担任濟南相期间的行为得罪十常侍和地方豪强,害怕引起报复,称病回乡。当时天下纷乱,先是发生了冀州刺史王芬联合南阳许攸、沛国周旌等地方豪强,谋划废黜灵帝立合肥侯的事件。王芬等人曾经希望曹操加入他们,但被曹操拒绝,后来王芬事败自杀。接着,又有西北金城郡(今甘肃兰州)的边章、韩遂杀死刺史和太守,率兵十余万反叛朝廷。

188年,汉灵帝组建西园军,曹操被起用为典军校尉,并派往家乡募兵,结果中途士兵叛乱,袭击曹操,曹操负伤逃脱。189年,在位22年的汉灵帝驾崩,终年34岁,太子刘辩登基,何太后临朝听政。大将军何进想趁宦官失势之机诛灭阉党,但是没有取得太后的支持。于是何进便召时任并州刺史的董卓进京,胁迫太后同意。然而此举打草惊蛇,董卓尚未抵達京城,何进已经被宦官下手谋杀,随后宦官十常侍也被袁绍等人诛杀。同年九月,董卓入京,诛杀丁原,逼走袁绍,执掌朝政。废少帝刘辩为弘农王,改立其弟陈留王刘协为皇帝,是为漢獻帝,京城陷入混乱。为了稳定局面,董卓想拉拢曹操,上表奏请曹操为骁骑校尉。但是,曹操沒有接受董卓所封的官职,害怕惹禍上身,更名改姓,潜逃出洛阳。曹操逃亡到成皋,投奔故人吕伯奢家,吕伯奢不在,其子招待曹操。曹操听见有食器声音,以为吕伯奢儿子要抓捕自己献给官府,乾脆将吕伯奢五个儿子和吕家2名宾客全部杀死。事後曹操發現自己是誤殺,於是說了句“宁我负人,毋人负我!”來安慰自己。曹操殺死呂伯奢兒子離開后,道逢二人,容貌威武,曹操避讓。二人笑著對曹操說:“感覺你很害怕,為什麼呀?”曹操覺得詫異,把剛才殺人的事全盤托出。临别前,曹操解佩刀送給二人并說:“以此表吾丹心,愿二贤慎勿言。”曹操繼續向東逃亡,經過中牟縣時,被亭長懷疑是逃亡的罪犯,於是將曹操抓去見縣令。鄉野中有人認得這人就是曹操,就拜託縣令把曹操給放走。

讨伐董卓(190年-191年):189年十二月,在回到陈留郡之后,曹操散盡家財徵募鄉勇,甚至親自和刀匠一同製作武器,豪強衛茲也傾家財助之,曹操率先於己吾揭竿举義,讨伐董卓。初平元年(190年)正月,董卓討伐戰开始,勃海太守袁绍、后将军袁术、冀州牧韩馥、豫州刺史孔伷、河內太守王匡、兗州刺史刘岱、陈留太守张邈、东郡太守桥瑁、山阳太守袁遗、济北国相鲍信等地方势力,举兵反董,群雄推舉袁紹為盟主。曹操则行使奋武将军之职。

此次戰役中,曹操跟隨諸將駐紮酸棗,當時天下英雄豪傑,都以袁紹馬首是瞻,只有鮑信認為曹操是撥亂反正的雄才。而鮑信的推算也沒有錯,在董卓焚燒洛陽,挾持漢獻帝與百官遷都長安之後,袁紹等各軍將領都畏懼董卓而按兵不動,誰也不敢先發動攻擊,唯有曹操單獨引兵西行。張邈派部將衛茲率領一部份兵力跟隨曹操,曹操打算奪取成皋,在抵達滎陽汴水時,和董卓部將徐榮遭遇,大敗,士卒死傷眾多。戰中,曹操被流矢射中,所騎戰馬受傷倒下,幸而曹洪把自己的坐騎讓給曹操,且步行隨侍保護,才能乘夜逃走。也是因為這戰,徐榮發現曹操以這麼少的兵力,都能奮戰一整天,判斷酸棗不易攻下,因此也向後撤退。

曹操回到酸棗,看見各軍累積起來也有十餘萬,但諸將每天歡宴飲酒,沒人圖謀進取,於是提出建議:袁紹從河內發兵,進逼孟津。酸棗諸將,據守成皋,控制敖倉,封鎖轘轅關、太谷關,掌握險要。袁術攻擊丹水、析,直入武關,威脅三輔。全軍興築高大堅固的堡壘,不和董卓軍團正面衝突,只派出游擊部隊騷擾,完全控制關東,從而立於不敗之地,等待董卓軍內部發生變化。然後諸將不能接受這項部署,曹操因為士兵多死在滎陽之戰,只好跟司馬夏侯惇等,前往揚州募兵,揚州刺史陳溫、丹陽太守周昕給予曹操四千餘人。行軍至龍亢,士卒叛變,趁夜焚燒曹操營帳,曹操親手斬殺數十人,才出營帳,沒有參與叛變的僅剩五百餘人。又行軍到銍、建平一帶,集結殘兵一千餘人,投奔袁绍,前来河內駐守。

然而不久後,酸棗大營糧秣告盡,各軍星拔營散,同時內鬥又起,劉岱和橋瑁交惡,橋瑁被殺。袁紹和韓馥又謀立幽州牧劉虞為帝,想拉拢曹操,遭到曹操拒绝。联军無暇顧及董卓之事,於是反董卓聯盟解散,除了曹操、孫堅、王匡之外,其餘諸軍皆沒有和涼州軍團交戰,只是各懷鬼胎,保存實力。曹操聽從了鮑信的建議,想避開袁紹新佔的冀州,往黃河南邊的兗州進行發展。这时黑山军于毒、白绕、眭固等十余万众攻略魏郡、东郡,东郡太守王肱无法抵抗,曹操趁机让袁绍让自己前往东郡,并顺利的在濮阳击败白绕。袁绍于是上表曹操为东郡太守,屯駐東武陽。

兖州時期(191年-194年):初平三年(192年),曹操駐軍頓丘,黑山賊餘毒等趁機攻擊東武陽,曹操率軍直向于毒的本營西山,于毒得到消息,解除對東武陽的進攻。曹操趁勝進抵內黃,大破眭固及南匈奴單于於夫羅等,東郡遂告安定。

夏,青州的百萬黃巾大軍入侵兖州。兖州刺史劉岱不听济北相鲍信劝阻,与黄巾军交战,结果被杀,舉州驚恐。曹操部屬陳宮前往昌邑,向別駕、治中推薦曹操繼任。鮑信心中亦有同樣想法,於是與州吏萬潛等到东郡迎接太守曹操,推举他担任兗州牧。後來與黃巾軍戰於壽張以東,初期失利,後曹操即起補救,加強訓練,賞罰嚴格,又不斷使用奇兵詭計,晝夜進攻,終於逼退黃巾軍。然而鮑信在亂軍中戰死,曹操重金尋鮑信屍體不得,只好雕刻其木像安葬祭拜。之後,曹操追擊黃巾直到濟北國,黃巾軍眼見退無可退,遂全體投降,其中有士兵三十餘萬人,眷屬老幼約一百萬人。曹操遴選其精銳,組成大名鼎鼎的「青州兵」。

與此同時,漢獻帝下詔使金尚為兗州刺史,曹操在金尚要赴任之際,率軍攻擊,金尚只好投奔袁術。

當時,袁紹和袁術反目成仇,雙方拉結人馬,袁紹找上了荊州劉表,而袁術則找上公孫瓚、陶謙。作为袁绍盟友的曹操,幫助袁紹打敗屯駐高唐的劉備,屯駐平原的單經,屯駐發干的陶謙等。

初平四年(193年),曹操屯軍鄄城。此時劉表截斷了盤踞在南陽的袁術其糧道,袁術受不了如此壓力,率軍北上,打算進佔兗州。聯絡了黑山賊餘黨以及於夫羅,屯駐封丘,並派遣將領劉詳駐紮匡亭。曹操初兵攻擊劉詳,袁術來救,為曹操所敗,曹操乘勝包圍封丘,還沒包圍完成,袁術突圍,退駐襄邑,曹操追擊,連戰連勝,最後袁術退入九江郡(位在揚州),曹操因此回軍定陶。(匡亭之戰)

秋,陶謙攻打兖州東部泰山郡,使得曹操父親曹嵩被殺。对于曹嵩之死,存在爭議。為此事,曹操開始攻擊徐州牧陶謙,并得到袁绍相助。曹操打到彭城,陶謙不敵退保郯縣,曹操在徐州大屠殺。《後漢書》對此事描錄「凡殺男女數十萬人,雞犬無餘,泗水為之不流」。

興平元年(194年),曹操以荀彧、程昱留守,再攻陶謙,進逼東海、琅邪,並在剡縣大破前來幫忙陶謙的劉備,嚇得陶謙差點逃回丹陽。就在此時,張邈因為恐懼曹操將為袁紹殺死自己,而陳宮則因為邊讓被殺而心生恐懼,於是兩人聯合從事中郎許汜、王楷以及張邈弟張超等,奉迎呂布成為兗州牧。呂布佔據濮陽,兗州郡縣大部分都響應,等曹操回軍時,根據地只剩下鄄城、范、東阿三縣。

曹操和呂布交戰於濮陽,濮陽豪門田氏作為內應,大開城門迎接曹操。曹操焚燒東門,誓言絕不退出,但被呂布反攻大敗,出奔。呂布的一名軍官,抓住曹操,卻不認識曹操,反而問曹操在哪裡,曹操回答:「騎馬逃走那個就是曹操。」軍官相信而放掉曹操,曹操才因此逃過一劫。

曹操回營後,再度組織進攻,雙方再戰於濮陽,對峙一百餘天,雙方糧秣用盡,於是各自撤退。在這危急時刻,曹操曾想投奔袁紹,但被程昱勸阻,曹操這才打消念頭,但还是亲自前往袁绍处,借了五千士兵回兖州繼續和呂布作戰。

興平二年(195年),曹操在鉅野大敗呂布部將薛蘭、李封等,呂布與陳宮捲土重來,雙方遭遇,曹操兵全出去收割小麥,只剩一千餘人,於是曹操設伏,呂布發現敵人單薄,輕裝突擊,曹操伏兵出擊,大敗呂布,乘勝攻取定陶,分兵收復諸縣,呂布逃奔徐州。後,曹操圍雍丘,城破,夷張超三族。就在此時,漢獻帝封曹操為兗州牧。曹操受封兗州牧后,向漢獻帝進貢梨、椑、枣各二箱。曹操在兗州鏖戰四年,總算得到了朝廷的承認,也紮實地站穩了其第一個領地。

奉天子以令不臣(195年-197年):195年七月,李傕与郭汜为了争权夺利,内斗不断,汉献帝趁机逃出长安,进驻安邑。建安元年(196年),曹操听从谋士荀彧、毛玠建议,前往洛阳,迎接皇帝。

雖然汉献帝(或其掌权之臣)对曹操仍有疑虑,但是曹操势力击破黄巾军,表現出關心社稷的忠心。由於京師洛阳被董卓破壞,殘破不堪,漢室於八月庚午日(10月7日)遷都至颍川郡许县。曹操於十一月丙戌日(197年1月1日)被任命为司空,封武平侯,仍领兖州牧,开始掌控东汉朝廷的军政大权。

早在兖州时期,曹操就开始打算和袁绍决裂。曹操控制汉献帝后,让献帝下诏书谴责袁绍地广兵多而树立党羽,不闻勤王之师而但擅相讨伐。袁绍不服,上书辩解。曹操为了安抚冀州牧袁绍,上表汉献帝,封其为太尉。袁绍又不肯位列曹操之下,甚至被曹操背盟举动所激怒,说;“曹操有几次都快没命了,我每次都救他,现在曹操忘恩负义,居然挟天子以令我吗?!”。最后曹操让汉献帝封袁绍为大将军。袁绍眼见曹操控制皇帝,借口许都潮湿,洛阳残破,要曹操迁都靠近袁绍控制区的鄄城,曹操不同意。

征讨四方(197年-199年):197年,曹操征討南阳郡的張繡,張繡舉眾投降,之後因为曹操納张绣叔父張濟之妻邹氏,張繡對這件事感到十分痛恨,曹操得知後密謀要殺害張繡。由於計畫洩漏,張繡襲擊曹操,曹操在長子曹昂、姪子曹安民與校尉典韋殿后下逃亡,曹昂、曹安民與典韋陣亡。此後,曹操又兩度攻擊張繡,都沒有徹底擊破。两年后,張繡接受謀士賈詡的建議,向曹操投降,曹操才取得對荊州北部的控制,並且消除了許都南面的威脅。

198年十二月,曹操用荀攸、郭嘉的計策,開決泗、沂二河之水灌入下邳,最後生擒处死呂布、陳宮与高顺,收降张辽,把徐州納入勢力範圍。199年,曹操派史涣、曹仁、于禁和徐晃击破张杨旧部眭固,取得河内郡,把势力范围扩张到黄河以北。同年六月,穷途末路的袁术病死于寿春,其部下向曹操投降,淮南之地尽归曹操之手。

官渡之战与统一北方(200年-207年):200年正月,车骑将军董承与左将军刘备、长水校尉种辑、将军吴子兰、王子服等人自称收到汉献帝的衣带诏,密谋诛杀曹操,事情败露后被曹操杀害,夷灭三族,怀孕的董贵人也被绞杀。在外领兵的刘备听说后斩杀车胄,统领徐州。不久,曹操攻下徐州,刘备逃到冀州投奔袁绍。

二月開始,曹操和袁绍展開一系列的會戰,史稱官渡之戰。谋士沮授建议袁绍采取持久战略不被采纳,而许攸建议袁绍派兵袭击许都亦不被接受。十月,战事處入僵局之時,袁绍谋士许攸因为袁绍和他多次不和,许攸家属因为犯罪被袁绍处罚,许攸对袁绍怀恨在心于是投奔曹操,向曹操献策,偷袭袁绍的粮草囤积地烏巢。曹操采纳,因而扭轉了戰局。张郃向袁绍建议派大军救援乌巢,但是袁绍只派轻骑去救援。乌巢守将淳于琼对曹军未先加以防范,结果乐进率军攻陷乌巢,烧掉袁绍所有的军粮,俘虏斩杀淳于琼。乌巢沦陷之后袁绍兵败逃回邺城,张郃、高览投降曹操,沮授因为拒绝投降而被处死。202年五月,袁绍去世,他最喜欢的三子袁尚继承大将军、冀州牧之位,与兄长袁谭内斗不断。

204年,曹操趁袁氏兄弟内斗的机会,发兵攻下邺城,诛杀审配,自领冀州牧,把自己的据点北迁到冀州邺城,政令军队此后皆从此出,只是在许县留有个别官吏監視汉献帝。205年攻下青州,诛杀袁谭与郭图。206年,平定并州高干的叛乱。207年,曹操征讨乌桓,诛杀蹋顿 ,谋士郭嘉病死在行军途中。袁尚、袁熙兵败后逃往辽东,被太守公孙康所杀。至此,曹操经过七年的战争,徹底消灭袁氏集团,基本统一中國北方。

曹操出身寒族,而且與閹宦有關,雖然深通詩文,但是不以儒家經學為務,與當時服膺經學的經學、士大夫不同。曹操曾下「求才三令」,強調重才不重德,並以法家之術為治,要摧破豪族的儒學。曹操為一代梟雄,不僅得到眾多寒族人才支持,也得到部份經學士大夫支持,如荀彧、荀攸、钟繇,荀彧更為曹操引進不少士大夫階層的人才。

赤壁之战与关中之战(208年-212年):建安十三年六月癸巳日(208年7月9日),曹操恢复三公制度,被任命為丞相。七月,曹操亲统大军10餘万南征荆州,企图先灭刘表,再顺长江东进,击败孙权。八月,荆州牧刘表病亡,次子刘琮请降。九月,刘备在長坂坡被曹军重创,逃往江夏,派遣诸葛亮出使柴桑,与孙权联合。十二月,江东名将周瑜火烧乌林,曹操敗於孫權和劉備聯軍,损失惨重,逃回北方,三國鼎立的雛型初步形成。209年,孙权率军攻打合肥,却中计退兵。周瑜占领江陵与夷陵,守将曹仁、徐晃、乐进等人逃往襄阳。

211年三月,曹操為用兵關中,藉口要討伐漢中張魯,遣曹仁、夏侯淵等統率大軍與钟繇會師於關中,此舉引起起關中諸侯的惊疑,馬超等十部起兵聯合反曹,曹操依賈詡離間之計,引起馬超、韓遂等人相互猜疑,一舉擊潰關中聯軍,馬超等人各自走還涼州。十月,曹操进军安定,杨秋投降。曹操率军撤回,命令夏侯渊督众将继续西征。隨後,馬超在隴西捲土重來,先後攻下隴西各地,但是最後復奪涼州未成,兵敗逃奔漢中。曹军在数年之内逐马超、破韩遂、灭宋建、横扫羌、氐,虎步关右,凉州地区基本平定。

封公称王(212年-216年):建安十七年(212年),汉献帝准许曹操“参拜不名、入朝不趋、剑履上殿”,如汉丞相萧何故事。董昭等人推舉曹操為魏公,荀彧以忠於漢室的立場提出反對。曹操答應荀彧永不做魏公,但是因此對荀彧不悅,不久荀彧忧愤去世。曹操起兵号称四十万,亲自南征孙权。次年(213年)正月,曹军进军濡须口(今安徽巢湖东南),攻破孙权设在江北的营寨,生擒公孙阳。孙权亲率军七万,前至濡须口抵御曹军,相持月余,各无所获。曹操见孙权军容严整,自己难以取胜,遂撤军北还。五月丙申日(6月16日),汉献帝册封曹操為魏公,其领地广及魏郡、河东郡、河内郡等十个郡国,超过西汉初年的刘姓宗室藩王,更加违背“七国之乱”和推恩令后诸侯封地不得超过一郡的汉制。

建安十九年三月癸未日(214年3月30日),漢獻帝使曹操的魏公位在諸侯王上,改授金璽,赤紱、遠遊冠。伏皇后数年前曾经写信给父亲伏完,要他秘密图划铲除曹操,伏完直到去世都不敢动手。后来事情败露,曹操命令御史大夫郗虑与尚书令华歆一起统兵入宫逮捕伏皇后。伏皇后紧闭门户,披发赤脚藏匿于墙壁之中,被华歆伸手牵出,监禁于掖庭暴室里幽禁而死,所生的两位皇子被鸩酒毒杀,伏氏宗族一百多人亦被处死,曹操之女曹节被立为皇后。

215年,曹操进攻漢中,太守張魯投降。曹操收降張魯後,取得漢中屬地,但是劉備得悉曹操攻降漢中,早晚要南下伐蜀,便和孫權以湘水为界平分荊州,回師益州。此時曹操没有接受刘晔的建议,未能趁刘备未站住脚跟之时攻蜀,便班師回朝。

建安廿一年四月甲午日(216年5月29日),曹操被封为魏王,加九锡,立曹丕为世子,公然違反漢高祖所訂「非劉氏而王,天下共擊之」的白馬之盟。次年(217年)僭天子禮,設天子旌旗,戴天子旒冕,出入得稱警蹕,並作泮宮。十月,再授賜十王冠、二綵帶,乘金根車,駕六馬,設五時副車。他名義上雖仍為漢臣,實際上掌握等同於皇帝的權力和待遇,權傾朝野,漢朝已经名存實亡。曹操任命夏侯淵為征西將軍、曹仁為征南將軍,欲取荊蜀之地。

南征孙权与汉中之战(215年-219年):215年,曹操打算報復孫權的皖城打擊,隨即率軍伐吳,可惜以失敗告終。孫權率领訛稱十萬大軍進攻合肥,曹操当时刚刚拿下汉中,不能亲自前往征战,便命令合肥守將張遼、樂進、李典阻擋進攻,孙权最终攻不下撤兵。216年,曹操拘留南匈奴单于呼厨泉,派右贤王去卑监国,将南匈奴分成五部,分别安置在朔方、并州、幽州一带,其中左部帅刘豹就是十六国汉赵政权创建者刘渊的父亲。

216年冬,曹操再次率军攻打濡须口陣取居巢,217年開始進攻逼退了正在濡須口築新城的孫權,但後來孫權作出反擊,把曹操軍擊退回居巢,雙方進入了膠著階段。當時連日大雨水漲,孫權驅使水軍令魏軍將士不安,曹操當時無法打敗嚴防的孫權,也未能穿越長江巢湖,看見形勢不利便下令撤軍,征戰時及歸途中大軍受瘟疫侵襲死傷者眾多。戰後,孙权派都尉徐详以重結姻親為條件向漢朝廷请降,曹操則對徐詳表示自己想跨越長江與孫權一起在江東狩獵的意願,但徐詳認為這樣只會令江東震蕩,委婉拒絕了曹操想乘機進入長江天險的意圖,曹操聽後大笑,隨即接受孫權的請降並結為姻親。

从217年末起,劉備率軍大舉進攻漢中阳平关,漢中之戰爆發。218年七月,曹操亲率大军坐镇長安,同时边塞硝烟再起,曹彰、田豫北征,大破乌桓、鲜卑联军。镇守漢中的夏侯淵與劉備相峙一年,曹軍守將夏侯淵、張郃、徐晃多次擊退劉備軍猛烈攻勢。

219年正月,劉備親自領軍和黃忠分進合擊,於定軍山斬殺征西將軍夏侯淵。至此漢中為劉備取得,同年三月曹操親自揮軍欲奪回,一度召集抽調鎮守北方的曹彰二十萬大軍增援,但是都為劉備所敗,曹軍無功而返,劉備便派劉封、黃忠、趙雲等將晝夜不停攻擊曹軍。至五月曹操便撤退至長安,且身体已感觉不适。劉備攻下房陵,派劉封順沔水攻佔上庸。相傳此戰為「雞肋」一词的出处。

樊城之战(219年):219年七月,劉備在漢中自立為漢中王,封關羽為前將軍。關羽起荊襄之兵大舉北伐襄樊,進一步圍困曹軍大將曹仁、滿寵的殘軍於樊城,史稱樊城之戰。曹操派左將軍于禁援救,適逢漢水暴漲,淹沒于禁七軍,漢軍乘勢以水軍攻打,活捉于禁。于禁向關羽投降,龐德被俘虜後不降遭斬,關羽並另遣軍隊包圍襄陽,一時之間威震華夏。當時曹操治下許多州郡的叛軍早已受關羽遙控。

同年十月,曹操来到洛阳,欲遷都以避其鋒芒,司馬懿、蔣濟等人勸阻,認為孫權必然不願看到關羽坐大。孫權果然自請襲擊關羽後方,曹操並召集駐守合肥與孫權對峙的張遼軍隊、在漢中監視巴蜀的徐晃軍隊等,並且親自由洛陽領軍往樊城救援。

曹操又命人把孫權偷襲荊州的消息用箭射到關羽和樊城守將曹仁處,曹軍士氣大振,而關羽進退失據。最先抵達樊城的徐晃軍,乘著大水稍退,對圍城的關羽軍展開攻擊。曹仁終於突圍而出,與徐晃軍一同擊退關羽。同年十二月,往南退軍的關羽被佔領江陵的孫權俘虏后處斬,孫權將關羽的首級送到洛阳,曹操以諸侯之禮安葬,襄樊戰役結束。

去世(220年):219年冬天,孙权上书称臣,陈说天命,劝曹操称帝。曹操把孙权来书给群臣观看,陈群、夏侯惇和司马懿等人都劝曹操簒位。曹操却不想簒汉自立,他说:「若天命在吾,吾为周文王矣。」周文王自己並未除滅殷商,到了其子周武王才克殷。暗示希望由自己的儿子曹丕来取代汉朝,建立新政权。

建安二十五年正月廿三日庚子(220年3月15日),曹操病逝於洛阳,享年66歲,謚号武王。

曹操临死前留下《遗令》,提倡薄葬。二月廿一日丁卯(4月11日),曹操被安葬于邺城西郊的高陵,與西門豹祠相近。

曹操去世后,世子曹丕嗣魏王、丞相、冀州牧之位。不久,夏侯惇、程昱等人也先后去世。同年十月廿九日(12月11日),曹丕代汉,迫使汉献帝退位禅让,建立曹魏,年号黄初,定都洛阳。封刘协为山阳公,追尊曹操為太祖武皇帝。

曹操詩歌在表現形式上往往有所創新,如「薤露行」、「蒿里行」,古辭都是雜言,各曲僅為四句,曹操則改用五言來寫,各十六句。五言詩以外,又長於四言詩。

《蒿里行》原是雜言,曹操卻以五言重寫,非常成功。四言詩方面,本自《詩經》之後已見衰落,少有佳作,但曹操卻繼承了《國風》和《小雅》的傳統,反映現實,抒發情感。例如:《短歌行》、《步出夏門行》等均是四言詩之佳作,使四言詩重生而再放異彩。

此外,曹操還有不少其他文章傳世,例如《请追增郭嘉封邑表》、《讓縣自明本志令》、《与王修书》、《祀故太尉桥玄文》等,文字質樸,感情流露,流暢率真。

曹操用舊調舊題,描寫新內容。漢樂府詩多著重塑造客觀人物形象,曹操的樂府詩卻突破詩人自我形象;漢樂府詩以敘事為主,曹操的樂府詩卻以抒情為主。他沒有形式上模擬樂府,而是學習民歌反映現實創作精神,用舊曲作詞,既具有民歌的特色,而又富有自己的創造性。

曹操善於以詩歌抒寫政治理想和抱負,雄心壯志,詩中充滿奮發進取的精神。部分詩中則雜有思憂難忘、人生朝露的消極情緒,還有宿命思想,又寫了一些遊仙詩。

曹操詩內容大致有三種:反映漢末動亂的現實、統一天下的理想和頑強的進取精神、以及抒發憂思難忘的消極情緒。

漢末大亂,曹操又南征北討,接觸的社會面非常廣大,故多有親身經驗和體會如《蒿里行》謂漢末戰亂的慘象,見百姓悲慘之餘又見詩人傷時憫亂的感情。故後人謂曹操樂府「漢末實錄,真詩史也。」

曹操對天下具有野心,懷有統一之雄圖,《短歌行》有謂「周公吐哺,天下歸心。」可資明證。其進取之心亦可見出,如《龜雖壽》言之「老驥伏櫪,志在千里。」言己雖至晚年仍不棄雄心壯志。

一代梟雄,縱風光一世,亦有星落殞滅之時。曹操對此也感到無能為力,只有作詩感歎,無可奈何。如《短歌行》中「譬如朝露,去日苦多」的感傷,《秋胡行》之低沈情緒,《陌上桑》等遊仙作品中都可見他的消極情緒。

曹操的詩,極受樂府影響,現存的詩脫胎自漢樂府民歌。這些詩歌雖用樂府舊題,卻不因襲古人詩意,自辟新蹊,不受束縛,而是体现了汉乐府「感於哀樂,緣事而發」的精神。例如:《薤露行》、《蒿里行》原是輓歌,曹操卻以之憫時悼亂。《步出夏門行》原是感歎人生無常,須及時行樂的曲調,曹操卻以之抒述一統天下的抱負及北征歸來所見的壯景。可見曹操富有創新精神的民歌,開啟了建安文學的新風,也影響到後來的杜甫、白居易等人。

曹操詩語言多古樸質直,少華美詞藻;情調悲壯,激昂慷慨;音調昂揚,氣魄雄偉;形象鮮明,善用比興。

曹操詩文辭簡樸,直抒襟懷,慷慨悲涼而沉鬱雄健,華美辭藻並不常見,惟形象鮮明,如《觀滄海》一詩:「东临碣石,以观沧海。水何澹澹,山岛竦峙。树木丛生,百草丰茂。秋風蕭瑟,洪波湧起,日月之行,若出其中,星漢燦爛,若出其裡。」寥寥數筆,即能以遼闊的滄海景象,表現詩人胸襟,不加潤飾。

《世說新語》容止第十四注引《魏氏春秋》:武王姿貌短小,而神明英发。《世說新語》容止第十四:魏武将见匈奴使,自以形陋,不足雄远国,使崔季珪代,帝自捉刀立床头。既毕,令间谍问曰:「魏王何如?」匈奴使答曰:「魏王雅望非常,然床头捉刀人,此乃英雄也。」魏武闻之,追杀此使。曹操小字阿瞒。《说文解字》卷五目部:瞒,平目也。

曹操为人聰明能干,但也十分狡猾。起初名声不显,可是也有很多人看好曹操的才華,如陳壽的「抑可謂非常之人,超世之傑矣」及許劭的「子治世之能臣,亂世之奸雄」已明顯闡述曹操的才能非比常人。也有許多人瞧不起曹操的品德,孫盛曰:魏武於是失政刑矣。易稱「明折庶獄」,傳有「舉直措枉」,庶獄明則國無怨民,枉直當則民無不服,未有徵青蠅之浮聲,信浸潤之譖訴,可以允釐四海,惟清緝熙者也。昔者漢高獄蕭何,出復相之,玠之一責,永見擯放,二主度量,豈不殊哉 !

曹操性格严厉,掾属办公如果不合其意,常常被他杖責。而其中唯有何夔经常带着毒药,决心如被杖責,宁可自鴆而死也不受侮辱,何夔才终究没有遭受杖刑

曹操生性猜忌,得罪過他的人,幾乎都被他殺死,例如:崔琰、許攸、婁圭、孔融、楊修、華佗、邊讓、桓邵、劉勳等人,趙彥欲親近漢獻帝者亦被殺死。即使沒犯錯只要威脅到曹操,曹操亦殺之,神童周不疑便是最好的例子。張繡兵變複投曹營後八年便病故,不然一般認為張繡若活得夠久早晚遭曹操清算。谋士荀彧由于反对曹操称魏公也被其嫉恨冷落。同时曹操又派下属卢洪、赵达二人担任抚军都尉负责监视军人,大家对卢洪、赵达恨得要死。

曹操的性格是有兩面性的,從《讓縣自明本志令》中可以看出曹操有政治智慧,也有性情。這樣一份有重要政治意義的綱領性文件卻用了非常樸實的語言風格,以及他的遺囑中很少提及他的政治生涯,很大篇幅都是安排瑣碎的家務事,雖說蘇東坡曾對此評價“平生奸伪,死见真性”,但是“惟大英雄能本色,是真名士自风流”,可見他的性情。另一方面,他也是奸猾的,他以《讓縣自明本志令》表明對漢朝的忠心,但實際行動卻渾然不同,在相同的實例中可以看出曹操的不同性格。

曹操深通兵法,在戰略、戰術方面都能應付裕如,常用計略來應付一系列的群雄戰爭來取勝,曹操甚還為孫武(孫子)所著作的《孫子兵法》做過註釋。

曹操擅長武藝、劍術,曾在兵變時,用剑杀死数十個亂兵才脫身。甚至有一次偷進張讓家時被發現,曹操手揮舞戟才得以逃跑,曹操也有抱負著野心的態度來面對亂世,如其迎接劉協代表其掌控了漢朝大勢,使到漢獻帝劉協也沒有權力。曹操也是殘忍之者,時常屠城,所以曹操殺人亦不手軟,坑殺士卒、214年殺害皇后,所生的兩位皇子亦以毒酒毒殺,伏氏宗族有百多人亦被處死、有孕在身的董貴人也未能幸免。曹操更過度勞役人民,以致有时爆發起义。

曹操妻妾眾多,不過娶納方面並不是毫無一個標準。並其中出名者多自他处改嫁而来。收降張繡時,收了張繡伯母入側室,引來張繡不快,曹操得知後於是想殺害張繡,但是由於計畫洩漏,引起張繡兵變,其長子曹昂,侄兒曹安民以及典韋白白地犧牲。曹操曾许诺将秦宜禄前妻杜夫人赠与关羽,但见其美色后自纳之,關羽因此心中不安。雖說好色,但曹操納妾實際上是有所標準,綜觀曹操所收的妻妾,不是寡婦,就是別人休離的前妻,所以曹操才不齒呂布染指有夫之妇的行为。曹操也很疼爱妻子所带来继子,并不因为非自己所出就有所忌讳曹操死前也留下遗言,要求善待他的夫人们。

曹操生性节俭,不好华丽。《魏晋世语》记载,曹植之妻崔氏(崔琰侄女)就因穿着过于华丽的衣服违反了禁令,回家后就被曹操賜死了。

曹操不相信鬼神,在担任济南相期间捣毁城阳景王刘章祠,并且認為坟墓也终将被人盗掘,所以极力提倡丧葬从简,一改汉代奢华之风。曹操曾設立發丘中郎將、摸金校尉等職,專門盜墓掘墳以賺取軍費,行事風格非常乾脆實際。然而此舉乃失德之行,連袁绍的幕僚陈琳起草讨曹之「檄文」中亦把曹操公然發掘漢梁孝王墓列為其罪行之一。

曹操喜欢吃鱼,在他的《四时食制》中提及不少鱼类。在一场宴会中也说过“今日高会,珍羞略备,所少吴松江鲈鱼耳。”

曹操的父亲曹嵩被宦官曹腾收养,其本来身份一直存在争议。《三国志》作者陈寿记载「莫能审其生出本末。」刘宋裴松之《三国志注》中引用的《曹瞒传》和郭颁《世语》则记载曹嵩本姓夏侯,是夏侯惇的叔父。

對于曹嵩出自夏侯氏的记载,何焯提出夏侯惇的儿子夏侯楙娶了曹操的女儿清河公主,夏侯渊的儿子夏侯衡也娶了曹家的女子,所以这种说法是敌对方东吴的传闻,不可采信。而潘眉、林国赞、姚范和赵一清则认为陈寿将夏侯惇、夏侯渊、曹仁、曹洪、曹休、曹真、夏侯尚放在同一个列传中,正隐寓夏侯氏是曹魏的宗室,曹操是夏侯氏的子孙,赵一清还指出曹操把女儿嫁给夏侯楙大概是想掩盖自己的出身,非常地奸诈,何焯据此辩证曹操不是夏侯氏的子孙完全是颠倒事实。恽敬则认为曹操虽然阴险狡猾,也不应该做出近亲通婚之事。曹氏与夏侯氏世代通婚,而夏侯惇、夏侯渊和曹仁、曹洪、曹休、曹真等是曹魏开国元勋,他们死后,曹爽与夏侯玄陆续被杀,大权归于司马氏,所以陈寿将夏侯氏与曹氏合传,让后人看到曹魏兴衰的缘由,这是陈寿写史书定下的史学法规。洪亮吉猜测陈寿大概是因为当时世传曹操是夏侯氏的子孙,所以在评论中特别注明夏侯氏和曹氏世代通婚,以表明此说的错误,洪亮吉还认为将《曹瞒传》和《世语》当做信史的人都是不善于读史书的。刘咸炘认为即便曹嵩是夏侯氏的子孙,他的后裔也未必不能与夏侯氏通婚,因为两家已经是不同族了,陈矫就是如此。如果曹嵩为夏侯氏的子孙不是丑事,没必要避讳,曹嵩是宦官养子人所共知,曹氏家族也没对此事避讳,不避讳养子而避讳出自夏侯氏是不近人情的,所以此说不足信。刘咸炘认为恽敬所给出的曹氏、夏侯氏合传的解释合理,陈寿评论中曹氏合夏侯氏世代通婚就是他这样立传的理由,洪亮吉所说陈寿意在辨明流言的是非反而曲解了陈寿的意思。李景星认为“莫能审其生出本末”是陈寿揭露曹操家世的丑闻。

吴金华总结各家观点,指出陈寿“莫能审其生出本末”是一种曲笔,他还提出曹嵩为夏侯氏的三个证据:

《三国志注·吴主传》中引《魏略》记载了孙权写给浩周的书信,当中有“今子当入侍,而未有妃耦,昔君念之,以为可上连缀宗室若夏侯氏”,此时孙权向曹魏称臣,魏臣浩周以为孙权之子可以如同夏侯氏一样和曹魏宗室连结在一起,这已证明曹嵩出自夏侯氏并非敌对方的传闻。

《三国志·文帝纪》记载夏侯惇去世的时候,裴松之引用《魏书》“王素服幸邺东城门发哀”,又引孙盛的评价“在礼,天子哭同姓於宗庙门之外。哭於城门,失其所也。”孙盛是东晋时人,以“良史”著称,他的这项评价以曹丕和夏侯惇为同姓,证明曹嵩出自夏侯氏这一点在孙盛时代仍为人所共知。

1974年至1979年安徽亳县城南出土了曹氏墓砖,刻辞有“夏侯右”。對于夏侯氏和曹氏世代通婚之事,周寿昌指出陈矫原为刘氏子孙,后成为舅舅家养子改姓陈,又娶了刘颂的女儿,刘颂与陈矫是近亲,曹操因爱惜陈矫的才华,为他周全,特别下令禁止诽谤此事。周寿昌认为曹操禁止人们议论同姓通婚,也是为自己的私事提供方便。吴金华也提出曹魏时期同姓通婚毫不奇怪,甚至有同母兄妹结为夫妇的情况,如《三国志注·曹爽传》引《魏末传》记载曹操义子何晏就娶了同母妹妹金乡公主。吴金华指出只要知道这一点,就会对曹嵩出自夏侯氏没有任何疑问。此后朱子彦和韩昇仍旧以《曹瞒传》和《世语》不可信,夏侯楙、夏侯衡、夏侯尚娶曹氏女来论证曹操不是夏侯氏的后裔。

橋玄:「今天下將亂,安生民者,其在君乎!」

許劭:“君清平之奸贼,乱世之英雄。”

何颙:“汉室将亡,安天下者,必此人也!”

陳宮:“今天下分裂而州無主;曹東郡(曹操),命世之才也,若迎以牧州,必寧生民。”

袁紹:「曹操當死數矣,我輒救存之,今乃背恩,挾天子以令我乎!」

袁術欲稱帝時曾推辭:「曹公尚在,未可也。」

劉表:「今天下大亂,未知所定,曹公擁天子都許,君為我觀其釁。」

呂布:「明公(曹操)所患不過於布,今已服矣,天下不足憂。明公將步,令布將騎,則天下不足定也。」(《三國志·魏書·呂布臧洪傳第七》)

于禁:“且公聪明,谮诉何缘!”(《三國志·魏書·張樂于張徐傳第十七》)

荀彧:「將軍(曹操)本以兗州首事,平山東之難,百姓無不歸心悅服。」

郭嘉:“真吾主也。”“公奉順以率天下;公糾之以猛而上下知制;公外易簡而內機明,用人無疑,為才所宜,不問遠近;公策得輒行,應變無窮;公以至心待人,推誠而行,不為虛美,以檢率下,與有功者無所吝,士之忠正遠見而有實者皆願為用;公於目前小事,時有所忽,至於大事,與四海接,恩之所加,皆過其望,雖所不見,慮之所周,無不濟也;公御下以道,浸潤不行;公所是進之以禮,所不是正之以法;公以少克眾,用兵如神,軍人恃之,敵人畏之。”

董昭:「將軍(曹操)興義兵以誅暴亂,入朝天子,輔翼王室,此五伯之功也。」

田豐:“曹公善用兵,变化无方,众虽少,未可轻也,不如以久持之。将军据山河之固,拥四州之众,外结英雄,内脩农战,然后简其精锐,分为奇兵,乘虚迭出,以扰河南,救右则击其左,救左则击其右,使敌疲於奔命,民不得安业;我未劳而彼已困,不及二年,可坐克也。今释庙胜之策,而决成败於一战,若不如志,悔无及也。”

劉備:“今指與吾為水火者,曹操也,操以急,吾以寬;操以暴,吾以仁;操以譎,吾以忠;每與操反,事乃可成耳。今以小故而失信義於天下者,吾所不取也。”「惟獨曹操,久未梟除,侵擅國權,恣心極亂。」(《三國志·蜀書·先主傳第二》)

關羽:「吾極知曹公待我厚,然吾受劉將軍厚恩,誓以共死,不可背之。吾終不留,吾要當立效以報曹公乃去。」(《三國志·蜀書·關張馬黃趙傳第六》)

孫權:“老賊欲廢漢自立久矣,陡忌二袁、呂布、劉表與孤耳。今數雄已滅,惟孤尚存,孤與老賊,勢不兩立。”(《三國志·吳書·周瑜魯肅吕蒙傳第九》)「其惟殺伐小為過差,離間人骨肉以為酷耳,御將自古少有。」

周瑜:「操雖托名漢相,其實漢賊也。」(《三國志·吳書·周瑜魯肅吕蒙傳第九》)

魯肅:“今之曹操,猶昔項羽,將軍何由得為桓文乎?肅竊料之,漢室不可復興,曹操不可卒除。為將軍計,惟有鼎足江東,以觀天下之釁。”「彼曹公者,實嚴敵也」(裴松之註引《魏書》及《九州春秋》)「曹公威力實重」(裴松之註引《漢晉春秋》)(《三國志·吳書·周瑜魯肅吕蒙傳第九》)

陸遜:「斯三虏者(曹操、劉備、關羽)当世雄杰,皆摧其锋。」(《三國志·吳書·陸遜傳第十三》)

韩嵩:“豪杰并争,两雄相持,天下之重,在於将军。将军若欲有为,起乘其弊可也;若不然,固将择所从。将军拥十万之众,安坐而观望。夫见贤而不能助,请和而不得,此两怨必集於将军,将军不得中立矣。夫以曹公之明哲,天下贤俊皆归之,其势必举袁绍,然后称兵以向江汉,恐将军不能御也。故为将军计者,不若举州以附曹公,曹公必重德将军;长享福祚,垂之后嗣,此万全之策也。”

諸葛亮:「曹操智計,殊絕於人,其用兵也,仿佛孫、吳。」「曹操五攻昌霸不下,四越巢湖不成,任用李服而李服圖之,委夏侯而夏侯敗亡,先帝每稱操爲能,猶有此失」(《三國志·蜀書·諸葛亮傳第五》)

王沈:「太祖御軍三十餘年,手不舍書。書則講武策,夜則思經傳。登高必賦,及造新詩,被之管弦,皆成樂章。」(《魏書》)

陳琳为袁紹所作檄文:「歷觀古今書籍所載,貪殘虐烈無道之臣,於操為甚。」

李瓒:「时将乱矣,天下英雄无过曹操。」

鲍信:「夫略不世出,能总英雄以拨乱反正者,君也。」

凉茂:「曹公忧国家之危败,愍百姓之苦毒,率义兵为天下诛残贼,功高而德广,可谓无二矣。」

陈寿:「汉末,天下大乱,雄豪并起,而袁绍虎踞四州,强盛莫敌。太祖武皇帝运筹演谋,鞭挞宇内,揽申不害、商鞅之法术,该白起、韩信之奇策,官方授材,各因其器,矫情任算,不念旧恶,终能总御皇机,克成洪业者,惟其明略最优也。抑可谓非常之人,超世之杰矣。」(《三国志·魏书·武帝纪》)

陈寿:“初,太祖性忌,有所不堪者,鲁国孔融、南阳许攸、娄圭,皆以恃旧不虔见诛。而琰最为世所痛惜,至今冤之。”

崔鴻《前凉錄》曰:「張茂謂馬岌曰:『劉曜自古可誰等輩也?』」岌謂曰:『曹孟德之流。』茂默然。岌曰:『孟德公族也,劉曜戎狄,難易不同。曜殆過之。』茂曰:『曜可方呂布、關羽,而云孟德不及,豈不過哉?』岌曰:「孟德挾天子,令諸侯,仗大義,討不庭;曜一卒胡人,用烏合之眾,而能建威成大逆,天下莫之當,其不優歟!』茂曰:『天生胡以滅中國,殆不可以人事論也。』」

孙楚:「太祖承运,神武应期,征讨暴乱,克宁区夏;协建灵符,天命既集,遂廓弘基,奄有魏域。」

裴松之:「魏太祖雖機變無方,略不世出,安有以數千之兵,而得逾時相抗者哉?」

陸機:「曹氏雖功濟諸華,虐亦深矣,其民怨矣。」(《辨亡論》)

潘安:「魏武赫以霆震,奉义辞以伐叛,彼虽众其焉用,故制胜于庙算。」

刘渊:「大丈夫当为汉高、魏武,呼韩邪何足效哉!」

王导:「昔魏武,达政之主也;荀文若,功臣之最也。」

垣荣祖:「昔曹操、曹丕上马横槊,下马谈论,此于天下可不负饮矣!」

钟嵘:「曹公古直,甚有悲凉之句。」

張輔:“武帝為張繡所困,挺身逃遁,以喪二子也;然其忌克,安忍無親:董公仁賈文和,恆以佯愚自免;荀文若楊德祖之徒;多見賊害;行兵三十餘年,無不親征;功臣謀士,曾無列土之封;仁愛不加親戚;惠澤不流百姓。”(《藝文類聚卷二十二》)

張悌:「曹操雖功蓋中夏,威震四海;崇詐杖術,征伐無已!民畏其威而不懷其德也。」

张鼎:「君不见汉家失统三灵变,魏武争雄六龙战。荡海吞江制中国,回天运斗应南面。隐隐都城紫陌开,迢迢分野黄星见。流年不驻漳河水,明月俄终邺国宴。文章犹入管弦新,帷座空销狐兔尘。可惜望陵歌舞处,松风四面暮愁人。」

张说:「君不见魏武草创争天禄,群雄睚眦相驰逐。昼携壮士破坚阵,夜接词人赋华屋。都邑缭绕西山阳,桑榆汗漫漳河曲。城郭为墟人代改,但有西园明月在。邺傍高冢多贵臣,娥眉曼睩共灰尘。试上铜台歌舞处,唯有秋风愁杀人。」

王勃:「魏武用兵,仿佛孙吴。临敌制奇,鲜有丧败,故能东禽狡布,北走强袁,破黄巾于寿张,斩眭固于射犬。援戈北指,蹋顿悬颅;拥旆南临,刘琮束手。振威烈而清中夏,挟天子以令诸侯,信超然之雄杰矣。」

魏元忠:「魏武之纲神冠绝,犹依法孙、吴,假有项籍之气,袁绍之基,而皆泯智任情,终以破灭,何况复出其下哉!」

朱敬则:「观曹公明锐权略,神变不穷,兵折而意不衰,在危而听不惑,临事决机,举无遗悔,近古以来,未之有也。」;「昔魏太祖兵锋无敌,神机独行,大战五十六,九州静七八,百姓与能,天下慕德,犹且翼戴弱主,尊奖汉室。」

赵蕤:「运筹演谋,鞭挞宇内,北破袁绍,南虏刘琮,东举公孙康,西夷张鲁,九州百郡,十并其八,志绩未究,中世而殒。」

穆修:「惟帝之雄,使天济其勇尚延数年之位,岂强吴、庸蜀之不平!」

石勒:「大丈夫行事,當磊磊落落,如日月皎然,終不能如曹孟德、司馬仲達父子,欺他孤兒寡婦,狐媚以取天下也。」(《晉書·載記第五·石勒下》)

崔浩:「劉裕平逆亂,司馬德宗之曹操也。」(《資治通鑑·卷一百一十八·晉紀四十》)

習鑿齒:「昔齊桓公一矜其功而叛者九國,曹操暫自驕伐而天下三分,皆勤之於數十年之內而棄之於俯仰之頃,豈不惜乎!是以君子勞謙日昃,慮以下人,功高而居之以上,勢尊而守之以卑。情近於物,故雖貴而人不厭其重;德洽群生,故業廣而天下愈欣其慶。夫然,故能有其富貴,保其功業,隆顯當時,傳福百世,何驕矜之有哉!君子是以知曹操之不能遂兼天下者也。」(《漢晉春秋》)

李世民對曹操用兵才能評價:「临危制变,料敌设奇,一将之智有余,万乘之才不足。」(《資治通鑑/卷197》)「帝以雄武之姿,當艱難之運,棟梁之任,同乎曩時,匡正之功,異於往代。觀沈溺而不拯,視顛覆而不持,乖徇國之情,有無君之跡。既而三分肇慶,黃星之應久彰;卜主啟期,真人之運斯屬。其天意也,豈人事乎!」(《全唐文·卷十·祭魏太祖文》),又對曹操品德評價:「朕常以魏武帝多詭詐,深鄙其為人。」(《貞觀政要》)

虞世南:「曹公兵机智算,殆难与敌,故能肇迹开基,居中作相,实有英雄之才矣。然谲诡不常,雄猜多忌,至于杀伏后,鸩荀或,诛孔融,戮崔琰,娄生毙于一言,桓邵劳于下拜,弃德任刑,其虐已甚,坐论西伯,实非其人。许劭所谓『治世之能臣,乱世之奸雄』,斯言为当。」(《长短经》卷二)

劉知幾:「賊殺母后,幽迫主上,罪百田常,禍千王莽。」(《史通•探賾篇》)

元稹:「劉虞不敢作天子,曹瞞篡亂從此始」(《董逃行》)

苏洵:「项籍有取天下之才,而无取天下之虑;曹操有取天下之虑,而无取天下之量;玄德有取天下之量,而无取天下之才。」

蘇軾:「世之稱人豪者,才氣各有高卑,然皆以臨難不懼,談笑就死為雄。操以病亡,子孫滿前,而咿嬰涕泣,留連妾婦,分香賣履,區處衣物,平生奸偽,死見真性。世以成敗論人物,故操得在英雄之列。而公見謂才疏意廣,豈不悲哉!操平生畏劉備,而備以公知天下有己為喜,天若胙漢,公使備,備誅操無難也。」(《孔北海贊》)

王安石:「青山為浪入漳州,銅雀台西八九丘。螻蟻往還空壟畝,麒麟埋沒幾春秋。功名蓋世知誰是,氣力回天到此休。何必地中餘故物,魏公諸子分衣裘。」

司馬光:「知人善任,難眩以偽。識拔奇才,不拘微賤;隨能任使,皆獲其用。與敵對陣,意思安閒,如不欲戰然;及至決機乘勝,氣勢盈溢。勳勞宜賞,不吝千金;無功望施,分毫不與。用法峻急,有犯必戮,或對之流涕,然終無所赦。雅性節儉,不好華麗。故能芟刈群雄,幾平海內。」(《資治通鑒》)

何去非:「曹公逡巡独以其智起而应之,奋盈万之旅,北摧袁绍而定燕、冀;合三县之众,东擒吕布而收济衮;蹙袁术于淮左,彷徨无归,遂以奔死。而曹公智画之出,常若有余,而不少困。彼之所谓势与勇者,一旦溃败,皆不胜支。然后天下始服曹公之为无敌,而以袁、吕为不足恃也。至于彼之任势与力,及夫各挟智勇之不全者,亦皆知曹公之独以智强而未易敌也,故常内惮而共蹙之。」;「言兵无若孙武,用兵无若韩信、曹公。」

元好問:「曹劉坐嘯虎生風,四海無人角兩雄。」(《論詩絕句》)

朱熹:「曹操作詩必說周公,如云:『山不厭高,水不厭深;周公吐哺,天下歸心!』又,苦寒行云:『悲彼東山詩。』他也是做得箇賊起,不惟竊國之柄,和聖人之法也竊了!」(《朱子語類‧論文下》)

胡三省:「操蓋已棄武都而不有矣。諸氐散居秦川,苻氏亂華自此始。」

洪皓:「长笑袁本初,妄意清君侧。垂头返官渡,奇祸怜幕客。曹公走熙尚,气欲陵韩白。欺取计已成,军容漫辉赫。跨漳筑大城,劳民屈群策,北虽破乌丸,南亦困赤壁。八荒思并吞,二国尽勍敌。四陵寄遗恨,讲武存陈迹。雉堞逐尘飞,浊流深莫测。回首铜雀台,鼓吹喧黾蝈。」

钟惺:「邺则邺城水漳水,定有异人从此起。雄谋韵事与文心,君臣兄弟而父子。英雄未有俗胸中,出没岂随人眼底?功首罪魁非两人,遗臭流芳本一身。文章有神霸有气,岂能苟尔化为群?横流筑台距太行,气与理势相低昂。安有斯人不作逆,小不为霸大不王?霸王降作儿女鸣,无可奈何中不平。向帐明知非有益,分香未可谓无情。呜呼!古人作事无巨细,寂寞豪华皆有意。书生轻议冢中人,冢中笑尔书生气!」

張溥:「究其(曹操)初,一名孝廉也……孟德奮跳,當塗大振,易漢而魏,雖附會曹參,難洗宗恥……孟德御軍三十餘年,手不捨書,兼草書亞崔、張,音樂比桓、蔡,圍棋埒王、郭;復好養性,解方藥。周公所謂多材多藝,孟德誠有之。使彼不稱王謀篡,獲與周旋,畫講武策,夜論經傳;或登高賦詩,被之管絃。又觀其射飛鳥,擒猛獸,殆可終身忘老,乃竟甘心作賊者,謂時不我容耳。漢末名人,文有孔融,武有呂布,孟德實兼其長;此兩人不死,殺孟德有餘。《述志》一令,似乎欺人,未嘗不抽序心腹,慨當以慷也。」(《漢魏六朝百三家集·魏武帝集題辭》)

羅貫中:“雄哉魏太祖,天下扫狼烟。动静皆存智,高低善用贤。长驱百万众,亲注《十三篇》。豪杰同时起,谁人敢赠鞭?”(《三国志通俗演义》)

蔡東藩:“曹操為亂世奸雄,乘機逐鹿,智略過人。袁紹袁術諸徒,皆不足與操比,遑論一張繡乎?乃宛城既下,遽為一孀婦所迷,流連忘返,幾至身死繡手,坐隳前功。董卓之死也,釁由婦人﹔操之不死於婦人之手,蓋亦僅耳!(《後漢演義》)

陳祚明:「孟德天分甚高,因緣所至,成此功業。」

黃摩西:「魏武雄才大略,草創英雄中,亦當占上座;雖好用權謀,然從古英雄,豈有全不用權謀而成事者?」

鲁迅:「曹操是一个很有本事的人,至少是一个英雄。我虽不是曹操一党,但无论如何,总是非常佩服他。」

毛泽东:「曹操是了不起的政治家、军事家,也是个了不起的诗人…曹操统一中国北方,创立魏国。他改革了东汉的许多恶政,抑制豪强,发展生产,实行屯田制,还督促开荒,推行法治,提倡节俭,使遭受大破坏的社会开始稳定、恢复、发展。」;「大雨落幽燕,白浪滔天,秦皇岛外打鱼船。一片汪洋都不见,知向谁边?往事越千年,魏武挥鞭,东临碣石有遗篇。萧瑟秋风今又是,换了人间。」(《浪淘沙·北戴河》)

范文澜:「他是拨乱世的英雄,所以表现在文学上,悲凉慷慨,气魄雄豪。」

费正清、崔瑞德:“给予汉王朝的致命一击却留给了中国历史上最引人注目的人物之一的曹操。曹操出身微贱,是大诗人、大战略家,也是现实主义的政治思想家;他反对儒家的礼仪和道德束缚。”(《剑桥中国秦汉史》)

%% -*- coding: utf-8 -*-
%% Time-stamp: <Chen Wang: 2021-11-01 11:40:57>

\subsection{文帝曹丕\tiny(220-226)}

\subsubsection{生平}

魏文帝曹丕(187年-226年6月29日),字子桓,沛国譙县(今属安徽亳州)人。三国時期曹魏開國皇帝,曹操的嫡长子,之後繼承父親的魏王封號與丞相的大權,最終東漢皇帝汉獻帝禪讓於其,曹丕登基後改國號為魏,史称曹魏,226年駕崩,諡文皇帝。

除軍政以外,曹丕自幼好文學,於詩、賦、文學皆有成就,尤擅長於五言詩,與其父曹操及其弟曹植並稱三曹,今存《魏文帝集》二卷。另外,曹丕著有《典論》,當中的《論文》是中國文學史上第一部有系統的文學批評專論作品。與父親曹操、其子曹叡並稱「魏氏三祖」。

187年冬天,曹丕生於沛国譙县(今属安徽亳州)。曹丕文武双全,六岁懂得射箭,八歲就能提筆為文和学会骑射。曹丕好击剑,博覽古今經传,通晓諸子百家学说。

197年,長兄曹昂、曹丕隨父出征宛城,曹昂與大將典韋、堂兄曹安民一同戰死於宛城,曹丕则幸运的骑马逃走。正室丁夫人因養子曹昂之死怪罪曹操而與曹操離異,生母卞夫人被扶為正室,原為庶長子的曹丕也就取代長兄曹昂成了嫡長子。

200年,曹丕跟隨曹操參加官渡之戰。

211年2月12日(建安十六年正月辛巳日),被任命為五官中郎將、副丞相。

212-213年,曹丕參加濡须口之战。

217年,在夺嫡之争中击败弟弟曹植,被立為魏王世子,正式成为曹操的继承人。

220年正月二十三日,曹操逝世,曹丕繼任丞相、魏王。十月十三日乙卯(11月25日)曹丕篡漢,逼迫汉献帝刘協禪讓帝位,廿九日辛未(12月11日)正式登基。自謂:“舜、禹之事,吾知之矣。”(一些史書記載舜、禹通过禅让成为天子,而曹丕通過逼迫漢帝禪讓帝位,他說這句話的隱含意義是堯舜禹的禪讓不過也是跟他一樣以逼迫的方式進行的假禪讓罷了,即認同另一些史書中所記載的“堯舜禹非禪讓說”)

221年八月,孙权遣使奏章臣屬。十九日丁巳(9月23日),文帝遣太常邢貞封孫權為吳王。同年四月,漢中王刘备称帝建立蜀汉。

221年,夫人甄氏卒。

222年,孙权不聽從曹丕要求遣子到魏國作為人質,經過多次周旋,曹丕認為孫權數次背叛而三路伐吳,但最終以大敗收場。

224年八月与225年八月,發生廣陵事變,同樣以失敗告終。

226年,五月十六日丙辰(6月28日)文帝病危,立平原王曹叡為太子,召曹真、曹休、陳群、司馬懿,并受遗诏辅佐嗣主。十七日丁巳(6月29日)崩于嘉福殿,终年四十岁。曹叡繼位,是為魏明帝。六月九日戊寅(7月20日),葬首陽陵。

曹操去世后,曹丕继任丞相、魏王、冀州牧,改建安二十五年为延康元年。同年十月,逼迫汉献帝禅位,篡汉称帝,国号为魏,改元黄初,定都洛阳。

改革选官制度,采纳陈群建议,实行九品中正制。易中天認為,九品中正制使门阀士族的政治特权得到确立和巩固,得到他们对曹魏政权的支持。

限制宦官、外戚权力:颁令“其宦人为官者不得过诸署令”“群臣不得奏事太后,后族之家不得当辅政之任,又不得横受茅土之爵”,保证了魏始终没有因为宦官、外戚干政造成政治危机,但因為曹丕曹叡父子過於依賴外臣司馬懿,日後司馬懿發動高平陵之變,司馬家掌握曹魏大權,其孫司馬炎更是代魏建晉。

削夺藩王权力:曹魏藩王的封地时常变更,没有治权和兵权,举动受到严格监视,形同囹圄。这个政策虽然吸取了汉朝诸侯国作乱的教训,却留下隐患,导致曹氏夏侯氏宗亲势单力薄,日后无力阻止外臣夺权。

重视文教:黄初二年(221年),下令人口十万以上的郡国每年察举孝廉一人,如有特别优秀的人才,可以不受户口限制。黄初五年,封孔子后人孔羡为宗圣侯,重修孔庙,在各地大兴儒学,立太学,置五经课试之法,设立春秋穀梁博士。在短期内使封建正统文化复兴。

恢复社会生产:除禁令,轻关税,禁止私仇,广议轻刑,与民休养,使北方地区重现安定繁荣局面。聽任典農治生,民屯的成效受到影響,出現弊端。在貨幣政策上,雖曾於登基時發行錢幣,但卻遭到失敗;之後更因穀物價格高騰,罷除了五銖錢(漢錢),自此之後終曹魏一代「以物易物」反成為北方主要的經濟型態。

黃初前期,東漢末諸侯孫權曾向魏稱臣,接受吳王封號。經過多次斡旋,魏吳最終走向敵對。期間曹丕三次親征孫吳均無功而返。

在位期間,任用曹真擊退鮮卑,和匈奴、氐、羌等外族修好,平定邊患。

曹丕心胸狹窄記恨薄情。在年少時曾向堂叔曹洪借錢不成懷恨在心,稱帝後不顧開國功臣元勳及血緣情誼直接栽贓罪名將曹洪下獄,卞太后得知後逼郭皇后求情才讓曹洪免於牢獄之災,但是仍然被曹丕削爵沒收財產貶為庶人。

曹丕非常愛打獵,但在曹操當政時為了爭儲而故意樹立良好形象,就接受崔琰建議停止打獵。曹丕稱帝后,已無爭儲壓力而不聽鮑勳和戴淩停止打獵的忠諫,曹丕因自己最愛的娛樂遭反對非常惱怒,而對戴淩處以比死刑低一等的处罰。而鮑勳的父親鮑信是曹操早年起兵的救命恩人,在曹丕對吳出兵廣陵的時候作出勸諫,曹丕不聽找理由把他強硬處死。

在與曹植鬥爭儲中,曹丕對於曹植的表現沒有任何對策,而且優柔不斷,只能聽從自己派系的人贏取繼承權。在曹操死後稱帝,藉故曹植治理不善削權並進行十數次地方遷徙,丁儀是曹植派,也在稱帝之後將丁家全族處刑。

夏侯尚因為寵愛妾侍而不愛正妻(曹丕的妹妹),曹丕把妾侍處刑,導致夏侯尚精神衰弱至死。

曹休守喪期間不吃肉,下令強逼他吃肉,曹休變得傷心消瘦。

曹丕寵愛郭氏,甄氏對曹丕多次抱怨,於是曹丕把她處死。

曹丕得知自己封為太子,高興而且得意忘形,辛憲英知道後認為魏國國運不會長久。

在曹操死後守喪期間,向孫權索要貴重珍品享樂,對孫權親征時駕馭副車及龍舟等奢華玩意。對外又被年長老練的孫權當棋子耍,不聽群臣勸告在夷陵之戰偷襲孫權,認為這樣做不合禮數;當孫權擊敗劉備後,曹丕假借孫權造反名義發兵進攻孫權打算從中得益,但什麼都得不到。曹丕要求孫權所做的全部得不到所願,於是惱羞成怒三路伐吳脅逼孫權,同時煽動江東內部作亂,最後均以敗退告終。孫權徹底拋棄曹魏,曹丕非常憤怒,於是發動兩次大規模伐吳都是無功而返,最後一次差點被孫權部將擒獲。

力行简葬。曹丕在《终制》中表示,寿陵因山为体,不封树,不立庙,不造园邑神道,不含珠玉,敛以时衣,陶器陪葬。曹丕提出“夫葬也者,藏也,欲人之不得见也。骨无痛痒之知,冢非栖神之宅”“自古及今,未有不亡之国,亦无不掘之墓也”,深受当时社会风气和他父亲曹操的影响。曹丕死后,按《终制》葬于首阳陵。

曹丕詩歌形式多樣,而以五、七言為長,語言通俗,具有民歌精神;手法則委婉細緻,回環往復,是描寫男女愛情和遊子思婦題材的箇中能手。 代表曹丕诗歌最高成就的《燕歌行》,据考写于建安十二年曹操北征三郡乌桓期间,采用乐府体裁,开创性地以句句用韵的七言诗形式写作,是现存最早最完整的七言诗。《燕歌行》从“思妇”的角度,反映了东汉末年战乱流离的现状,表达出被迫分离的男女内心的怨愤和惆怅。全诗用词不加雕琢,音节婉约,情致流转,被明朝王夫之盛赞“倾情,倾度,倾色,倾声,古今无两”。

曹丕的一些为后人称道的作品都在担任五官中郎将至魏太子期间所作,他的诗歌细腻清越,缠绵悱恻,缺乏曹操、曹植的慷慨之气,后世对他的评价不如“三曹”中的另外两人。

學者葉嘉瑩在《葉嘉瑩說漢魏六朝詩》裡,列舉鍾嶸《詩品》、劉勰《文心雕龍》和王夫之《薑齋詩話》對曹丕的評價。《詩品》將曹丕排在中品,認為他的詩不及弟弟曹植,原因是曹丕詩「率皆鄙直如偶語」(「偶語」,即兩個普通人在講話),反觀曹植則是「骨氣奇高,詞采華茂。情兼雅怨,體被文質,粲溢今古,卓爾不群」。《文心雕龍》(才略篇)說曹丕「魏文之才,洋洋清绮,旧谈抑之,谓去植千里......子桓慮詳而力緩,故不競於先鳴」,與曹植「思捷而才俊」不同,又謂「俗情抑揚,雷同一響,遂令文帝以位尊減才,思王以勢窘益價,未為篤論也」,世人都同情曹植的處境,曹丕是兄弟爭位的勝方,人們也因此忽略他文章的美妙。明末清初,王夫之在《薑齋詩話》裡直言:「實則子桓天才駿發,豈子建所能壓倒耶?」,可谓為曹丕文學成就「平反」的宣言。葉嘉瑩說,曹丕是一位「理性詩人」,有節制有反省,「以感與韻勝」。

诗歌笔法影响着其赋的风貌,诗体之赋为其赋作品的特点,具有明显的标志性意义。曹丕所创作的二十八篇赋作,其中有序者共有十六篇。内容上来看以抒情和咏物为主,体制方面一改汉大赋之鸿篇巨幅,成为短小精悼的行情小赋。内容以真情的笔触触摸到社会现实,并将个体的喜怒哀乐带入赋中。

曹丕的《典论‧论文》是中国最早的文学理论与批评著作,写于曹丕为魏王太子时,文中要点有:以班固和傅毅為例,說明「文人相輕」和「家有弊帚,享之千金」的做文學家的不自見己身的缺點,只看到別人的小缺失就加以嘲諷,對於別人的優點卻視而不見。评价孔融、陈琳、王粲、徐干、阮瑀、应玚、刘桢的文风和得失,“建安七子”的说法来源于此提出“文以气为主,气之清浊有体,不可力强而致”,认为作家的气质决定作品的风格,肯定文学的历史价值,“盖文章,经国之大业,不朽之盛事”。

魯迅在《魏晉風度及文章與藥及酒之關係》中稱「他(曹丕)說詩賦不必寓教訓,反對當時那些寓教訓於詩賦的見解,用近代的文學眼光來看,曹丕的一個時代可說是『文學的自覺時代』,或如近代所說是為藝術而藝術的一派。」

曹丕善写妇女题材的作品,其诗歌有《寡妇诗》与之赋作品皆有机行一致的情感行发。《寡妇赋》、《出妇赋》,更多地表达对下层社会妇女哀悯同情的情怀。

曹丕是邺下文人集团的实际领袖,对建安文学的精神架构起到关键作用,由此形成的“建安风骨”对后世文学产生了深远影响。曹丕命令刘劭、王象、缪袭等人编纂中国第一部类书《皇览》,开官方组织编纂类书的先河。《登台赋》是一篇歌咏铜雀台华美壮丽的小赋,笔触清新细腻,在描写景色时做到了 “写物图貌,蔚似雕画”。《典論‧論文》開創了文學批評的風氣,為中國文學批評之祖。《燕歌行》则是中国文学史上第一首完整的七言诗,此对后世七言诗的创作有很大影响。《校猎赋》是一篇很完整的赋作品,在赋中曹丕运用笔墨不多,不过三百来字就将田猎盛况尽数描绘出来。《列异传》是魏文帝曹丕所写的一部志怪小说集,属于文学艺术,据唐代魏征等人撰写的《隋书·经籍志》记载,它作为现存最早的一部描写鬼类故事的志怪小说,对后世鬼魅小说的描写有着巨大的影响。

诸葛亮:“曹丕篡弑,自立为帝,是犹土龙刍狗之有名也。”(《三国志·卷四十二·蜀书十二·杜周杜许孟来尹李谯郤传第十二》)

孙权:“及操子丕,桀逆遗丑,荐作奸回,偷取天位,而叡么麽,寻丕凶迹,阻兵盗土,未伏厥诛。”(《三国志·卷四十七·吴书二·吴主传第二》)

曹植:「祥惟圣贤,歧嶷幼龄。研几六典,学不过庭;潜心无妄,抗志清冥。才秀藻朗,如玉之莹。」(《曹集诠评》卷十《文帝诔》)

桓阶:“仁冠群子,名昭海内,仁圣达节,天下莫不闻。”(《三国志·卷二十二·桓阶传》)

卞兰:「研精典籍,留意篇章,览照幽微,才不世出,禀聪睿之绝性,体明达之殊风,慈孝发于自然,仁恕洽於无外。是以武夫怀恩,文士归德。窃见所作典论,及诸赋颂,逸句烂然,沈思泉涌,华藻云浮,听之忘味,奉读无倦。」(《艺文类聚》十六《赞述太子赋》)

陸遜:「曹丕大合士眾。外托助國討備,內實有奸心。」(《三國志·吳書·陸遜傳第十三》)

張悌:「曹操虽功盖中夏,民畏其威而不怀其德也。丕、睿承之,刑繁役重,东西驱驰,无有宁岁。」(《资治通鉴·卷七十八·魏纪十·元皇帝下》)

陈寿:“文帝天资文藻,下笔成章,博闻强识,才艺兼该;若加之旷大之度,励以公平之诚,迈志存道,克广德心,则古之贤主,何远之有哉!”(《三国志·魏书·文帝纪》)

《晉書·禮志上》:「大魏三聖相承,以成帝業。武皇帝肇建洪基,撥亂夷險,為魏太祖。文皇帝繼天革命,應期受禪,為魏高祖。上集成大命,清定華夏,興制禮樂,宜為魏烈祖。于太祖廟北為二祧,其左為文帝廟,號曰高祖昭祧,其右擬明帝,號曰烈祖穆祧。三祖之廟,萬世不毀。其餘四廟,親盡迭遷,一如周後稷、文武廟祧之禮。」

阎缵:“魏文帝惧于见废,夙夜自祗,竟能自全。”(《晋书·卷四十八· 阎缵传》)

刘渊:“黄巾海沸于九州,群阉毒流于四海,董卓因之肆其猖勃,曹操父子凶逆相寻。”(《晋书·卷一百一·载记第一》)

李班:“观周景王太子晋、魏太子丕、吴太子孙登,文章鉴识,超然卓绝,未尝不有惭色。何古贤之高朗,后人之莫逮也!”(《晋书·卷一百二十一·载记第二十一》)

葛洪:「自建安之后,魏之武文,送终之制,务在俭薄,此则墨子之道,有可行矣。」(《抱朴子外篇》)

刘勰:「魏文之才,洋洋清绮,旧谈抑,之谓去植千里。然子建思捷而才俊,诗丽而表逸;子桓虑详而力援,故不竞于先鸣。而乐府清越,《典论》辩要,选用短长,亦无懵焉。但俗情抑扬,雷同一响,遂令文帝以位尊减才,思王以势窘益价,未为笃论也。」(《文心雕龙·才略第四十七》)

垣荣祖:「昔曹操、曹丕上马横槊,下马谈论,此于天下可不负饮矣!」(《南齐书 卷二十八 列传第九》)

释道恒:「光武尚能纵严陵之心,魏文全管宁之操,折至尊之高怀,遂匹夫之微志。」(《释文纪》卷八)

沈约:「自魏氏膺命,主爱雕虫,家弃章句,人重异术。又选贤进士,不本乡闾,铨衡之寄,任归台阁。」(《宋书》卷五十五)

萧统:「不追子晋,而事似洛滨之游;多愧子桓,而兴同漳川之赏。漾舟玄圃,必集应、阮之俦;徐轮博望,亦招龙渊之侣。」(《答湘东王求文集及书》)

颜之推:「自昔天子而有才华者,唯汉武、魏太祖、文帝、明帝、宋孝武帝,皆负世议,非懿德之君也。」(《颜氏家训》卷四)

《陈思王庙碑》:「魏高祖文皇帝,绍即四海,光泽五都,负彰魈茫朝宗万国,允文允武,庶绩咸熙,正践升平,时称宁晏。」(《全隋文·卷二十九》)

王勃:「文帝富裕春秋,光应禅让,临朝恭俭,博览坟典,文质彬彬,庶几君子者矣。」(《全唐文·卷一百八十二》)

郝处俊:「昔魏文帝著令,虽有幼主,不许皇后临朝,所以杜祸乱之萌也。」(《资治通鉴 卷第二百二》)

刘知几:「文帝临戎不武,为国好奢,忍害贤良,疏忌骨肉。」(《史通·探赜第二十七》)

李隆基:「叹节气之循环,美君臣之相乐,凡百在会,咸可赋诗,五言纪其日端,七韵成其大数,岂独汉武之殿盛,朝士之连章,魏文之台壮,辞人之并作云尔。」(《端午三殿宴群臣探得神字并序》)

张说:「周文王之为太子也,崇礼不倦;魏文帝之在青宫也,好古无怠,博览史籍,激扬令闻,取高前代,垂名不朽。」(《张燕公集》)

王锴:「文帝八岁能属文,博览古今,贯穿经史。及居帝位,益尚谦和。坐不废书,手不释卷。」(《全唐文 卷八百九十》)

范仲淹:「魏文帝宠立郭妃,谮杀甄后,被发塞口而葬,终有反报之殃。」(《范文正集 卷十五》睦州谢上表)

黄庭坚:「盖世英雄不自知,暮年初志各参差。南阳陇底卧龙日,北固樽前失箸时。霸主三分割天下,宗臣十倍胜曹丕。寒炉夜发尘书读,似覆输筹一局棋。」(《山谷诗集注》)

苏辙:「臣伏观历代帝王,如汉武,魏文,唐徳、文、宣三宗皆工于诗骚杂文,与一时文士比长絜大。至于经纶当世,讲论利害,以文墨尽天下事,则皆不足以仰望先帝之万一。」(《栾城集卷四十七》进御集表)

司馬光:「于禁將數萬眾,敗不能死,生降於敵,既而復歸。文帝廢之可也,殺之可也,乃畫陵屋以辱之,斯為不君矣!」

陈亮:「至于欲使当时累息之民得阔步高谈无危惧之心,未尝不为之三复也,于是时吴蜀争帝,中国庶几乎息肩矣,是以在位七年而谥曰文也。」(《龙川集·卷七》)

耶律楚材:「仲谋服孟徳,孔明倍曹丕。唯晋成全统,平吴混八维。」(《湛然居士集》)

郝经:「丕特负赃胠箧之盗。操死丕直取,自以为可也,乃从容禅让,自以为舜、禹复出,其自欺也甚矣!且轻薄佻靡,未除贵骄公子之习,不矜细行,隳败礼律,刻薄骨肉,自戕本根,乱亡基兆,已在于是。孔明谓为土龙刍狗,宜哉!」(《续后汉书 卷二十六》)

曹丕假借禪讓,逼漢獻帝劉協退位备受非议。直到明朝末年,复社领袖张溥在《汉魏六朝百三家集》的《魏文帝集题辞》中首次对曹丕德政品行作全面评价。张溥写到:“魏王帝业无足称,惟令宦人为官,不得过诸署令。诏群臣不得奏事太后;后族不得常辅政任,石室金策,可宝万世。彼亲见汉室炎隆,女主中人手扑灭之,麦秀黍离,伤心目。霸朝初创,力更旧辄,至待山阳公以不死,礼遇汉老臣杨彪不夺其志,盛德之事,非孟德所及……甄后塘上,陈王豆歌,损德非一。崇华首阳,有余恨焉。”他认为曹丕的施政有可取之处,礼遇汉室君臣,可见并非无德,但杀妻、害弟罪名昭著。

胡应麟:「诗未有三世传者,既传而且煊赫,仅曹氏操、丕、睿耳。然白马名存钟《品》,则彪当亦能诗。又任城武力绝人,仓舒智慧出众。阿瞒何徳,挺育多才?生子如此,孙仲谋辈讵足道哉!」(《诗薮》)

方孝孺:「汉高祖、魏文帝皆中才之主,非有圣智之度,高祖犹能不杀子婴,文帝犹能奉山阳终其身。曾谓武王圣人而忍其君至此乎?吾决知其不然矣。」(《逊志斋集卷四》)

谭嗣同:「若夫汉武帝命所忠求相如遗书,魏文帝诏天下上孔融文章,渐昭风轨,犹无集名。自荀况诸集,编题后人,张融玉海,标目己意,乃始波颓雾靡,不可胜遏。」(《谭嗣同全集》)

王世贞:「自三代而后,人主文章之美,无过于汉武帝、魏文帝者。」(《艺苑疤言》)

王夫之:「曹子建鋪排整飾,立階級以賺人升堂,用此致諸趨赴之客,容易成名,伸紙揮毫, 雷同一律。子桓精思逸韻,以絕人攀躋,故人不樂從,反為所掩。子建以是壓倒阿兄,奪其名譽。實則子桓天才駿發,豈子建所能壓倒耶?曹子建之於子桓,有仙凡之隔, 而人稱子建,不知有子桓,俗論大抵如此。」(《姜齋詩話》)

章太炎:「今人皆谓汉代经学最盛,三国已衰,然魏文廓清谶纬之功,岂可少哉!文帝虽好为文,似词章家一流,所作《典论》,《隋志》归入儒家。纬书非儒家言,乃阴阴家言,故文帝诏书未引一语。岂可仅以词章家目之!」(《经学略说》)

刘师培:「魏文与汉不同者,盖有四焉,书檄之文,骋词以张势,一也。」(《中古文学史论汉魏之际文学变迁》)

魯迅:「他(曹丕)說詩賦不必寓教訓,反對當時那些寓教訓於詩賦的見解,用近代的文學眼光來看,曹丕的一個時代可說是『文學的自覺時代』,或如近代所說是為藝術而藝術的一派。」(《魏晉風度及文章與藥及酒之關係》)

毛泽东:「曹丕也是他(曹操)儿子,也有些才华,但远不如曹操。曹丕在政治上也平庸,可他后来做了皇帝,是魏文帝。历史上所称的‘建安文学’,实际就是集中于他们父子的周围。一家两代人都有才华、有名气,在历史上也不多见哪!」

郭沫若:「曹丕在政治见解上也比乃弟高明得多,而在政治家的风度上有时还可以说是胜过他的父亲。如令宦人为官不得过诸署,禁母后预政,取士不限年资但纠其实、轻刑罚、薄赋税、禁复仇、禁淫祀、罢墓祭、诏营寿陵力求俭朴等等,处处都表示着他是一位旧式的明君典型。」(《郭沫若全集·历史篇·第四卷》)

马植杰:「从曹丕的政治设施来看,也有些不错的。拿曹丕与其他封建帝王相比,尚属中等偏上者。」

叶嘉莹:「魏文帝在即位后,曾下了息兵诏,下了薄税诏,下了轻刑诏。他实在是一个很有理想的皇帝,希望能够把天下治理得更好。但是很可惜,他只做了七年的皇帝就死了,死的时候只有四十岁。」(《汉魏六朝诗讲录》)

曹丕当五官中郎将时,有一次宴請,曹丕问相士朱建平自己的寿命,朱建平說:「您的寿命是八十岁,四十岁时会有小灾难,希望您多加小心。」曹丕果然四十而终,死前认为朱建平的占卜结果是昼夜相加计算的,自己的生命快要结束。(《三国志‧魏志二十九‧方技传》)

曹丕当五官中郎将时,曹操曾经赏赐他百辟刀。

曹丕善擊劍騎射,好博弈彈棋,在《典論》的自敘中更自詡其非凡箭藝,能「左右射」,可謂文武兼備。

曹丕喜爱葡萄和葡萄酒。

魏文帝诏群臣曰:“中国珍果甚多,且复为说蒲萄。当其朱夏涉秋,尚有余暑,醉酒宿醒,掩露而食。甘而不涓,脆而不酸,冷而不寒,味长汁多,除烦解渴。又酿以为酒,甘于曲糵,善醉而易醒。道之固已流涎咽唾,况亲食之邪?南方有桔,酢正裂人牙,时有甜耳。远方之果,宁有匹之者?”(一说出自《与吴监书》)

《与群臣诏》:“南方有龙眼、荔枝,宁比西国葡萄、石蜜乎!」

曹丕与曹植争夺太子之位,后来曹丕得立,曾经喜极失态,抱着辛毗的颈说:“辛君您知道我有多么喜悦吗?”辛毗事后将曹丕的表现告诉女儿辛宪英,时年二十多岁的宪英便感叹地说:“太子是代替君王主理宗庙社稷的人物。代君王行事不可以不怀着忧虑之心,主持国家大事亦不可以不保持戒惧之心,在应该忧戚的时候竟然表现得如此喜悦,又怎会长久呢?魏国又怎能昌盛?”

甄后塘上。文昭皇后甄氏原为袁绍次子袁熙之妻,建安九年城破被俘,曹丕纳之。後曹丕称帝,眷寵文德皇后郭氏,甄氏失宠口出怨言,觸怒曹丕。曹丕下令赐死甄氏,殓时“被发覆面,以糠塞口”。

陈王豆歌。陈思王曹植曾与曹丕争储,曹丕称帝,曹植备受迫害,屡次迁封。《世说新语‧文学》中,曹丕命令曹植七步成诗,否则问罪,曹植才思敏捷,逃过一劫,即著名的《七步诗》故事。

誅殺忠臣鮑勳,鮑勳為曹操大將鮑信之子,為人剛正不阿,卻因私人恩怨遭曹丕殺害。鮑勳被曹丕處死後二十日,曹丕也暴斃逝世。(三國志·魏志十二·鮑勳傳)

曹丕曾下詔給征南將軍夏侯尚說:「你是朕的心腹重將,應當給予你特別的任命(指征南將軍一職)。希望你可以廣施恩德足以令死者享用,實行惠愛令人終身難忘。你可以獨攬威權,擅行賞罰,有殺人或活人的權力。」後來,夏侯尚將這道詔書出示給蔣濟看。蔣濟回到朝廷,曹丕問他:「你在各地聽到和看到的社會風氣、教化都是怎麼樣的?」,蔣濟回答道:「沒有其他善行,只聽到了亡國之語。」,曹丕臉上露出憤怒的表情,問蔣濟這是什麼原因,蔣濟以曹丕頒給夏侯尚的詔書作答:「『作威作福』是《尚書》中明明白白告誡臣子不能做的事情,『天子無戲言』,古人對此非常慎重。請陛下認真考慮。」,曹丕這才明白了蔣濟的意思,下令將頒給夏侯尚的詔書追回。在詔書中提到的「作威作福」,意為獨攬威權,擅行賞罰。這句成語的最早出處是《尚書·洪範》中的「惟闢作福,惟闢作威,惟辟玉食。臣無有作福作威玉食。」,按照這句成語最早的解釋,是只有帝王才能行使的權力,而曹丕卻出現了重大筆誤。難怪蔣濟毫不客氣予以指責,曹丕也只能低頭認錯,乖乖地將詔書收回了。

\subsubsection{黄初}

\begin{longtable}{|>{\centering\scriptsize}m{2em}|>{\centering\scriptsize}m{1.3em}|>{\centering}m{8.8em}|}
  % \caption{秦王政}\
  \toprule
  \SimHei \normalsize 年数 & \SimHei \scriptsize 公元 & \SimHei 大事件 \tabularnewline
  % \midrule
  \endfirsthead
  \toprule
  \SimHei \normalsize 年数 & \SimHei \scriptsize 公元 & \SimHei 大事件 \tabularnewline
  \midrule
  \endhead
  \midrule
  元年 & 220 & \tabularnewline\hline
  二年 & 221 & \tabularnewline\hline
  三年 & 222 & \tabularnewline\hline
  四年 & 223 & \tabularnewline\hline
  五年 & 224 & \tabularnewline\hline
  六年 & 225 & \tabularnewline\hline
  七年 & 226 & \tabularnewline
  \bottomrule
\end{longtable}


%%% Local Variables:
%%% mode: latex
%%% TeX-engine: xetex
%%% TeX-master: "../../Main"
%%% End:

%% -*- coding: utf-8 -*-
%% Time-stamp: <Chen Wang: 2019-12-17 22:25:50>

\subsection{明帝\tiny(226-239)}

\subsubsection{生平}

魏明帝曹叡(204年?-239年1月22日),字元仲,豫州沛国谯县(今安徽省亳州市)人。三国时期曹魏第二任皇帝(226年至239年在位)。魏文帝曹丕长子,母为文昭甄皇后。

黄初三年(222年),曹叡封平原王,黄初七年(226年)五月,魏文帝病重,立曹叡为皇太子,即位于洛阳。曹叡在位期间指挥曹真、司马懿等人成功防御吴、蜀的多次攻伐,并且平定鲜卑,攻灭公孙渊,设置律博士制度,重视狱讼审理,与尚书陈群等人制《魏律》十八篇,是古代法典编纂史上的重大进步。魏明帝在军事,政治和文化方面都颇有建树,但在统治后期大兴土木,广采众女,因此留下负面影响。

景初三年(239年),曹叡病逝于洛阳,时年三十五岁,庙号烈祖,谥号明帝,葬于高平陵。曹叡能诗文,与曹操、曹丕并称魏氏“三祖”,原有集,已散佚,后人辑有其散文二卷、乐府诗十余首。自從曹叡崩後,曹爽掌权,魏帝自此淪爲傀儡。再後曹爽被司馬懿發動高平陵之变斬殺,魏國大權完全落入司馬氏家族手中。

叡生于建安九年(204年?),母亲是文昭甄皇后,甄氏初为幽州刺史袁熙妻子,曹操打败袁绍后,被魏文帝曹丕所纳,甚为得宠,生有曹叡和东乡公主。曹叡从小才智出眾、聰明特異。祖父曹操对此十分惊喜而倍加喜爱,常令他伴随左右。在朝会宴席上,也经常叫他与侍中近臣并列。曹操曾经评价道:“我的家族基业有了你就可以继承三代了。”曹叡好学多识,尤其留意研究律法。

建安二十一年(217年),曹操封魏王,同年东征孙权,曹叡及妹妹东乡公主离开母亲甄氏,与祖母卞夫人,父亲曹丕一起随征江东。

延康元年(220年),曹操病逝,其父曹丕继位魏王,同年五月,十五岁的曹叡被封为武德侯,曹丕作《以侍中郑称为武德侯傅令》,亲自诏令时任侍中的笃学大儒郑称为曹叡的师傅,教授他经学,以此明志。

黄初二年(221年),曹叡被封为齐公,同年八月,其母甄氏因为怨言而被曹丕赐死,葬于邺城,曹叡因为母亲获罪,降为平原侯。黄初三年(222年)三月,曹丕又复其爵位,晋封为平原王。

生母被赐死,曹叡一併受罚被貶为平原侯。起初其父魏文帝认为曹叡先前既有不满,便想立徐姬所生的京兆王曹礼为嗣,因此久不立太子。这期间,曹叡府中来往的家臣官吏、师长、友伴,一律只取品行正直的人充任,互相匡扶、勉励矫正,与卫臻私交甚好,经常一起讨论朝事和书籍。曹丕也曾询问卫臻关于曹叡的情况,卫瑧只是称赞他明理而有德行,不言其他。

据《魏末传》记载,曹叡一次随曹丕狩猎,见到母子两鹿。文帝射杀鹿母,命令曹叡射杀子鹿,曹叡说:“陛下已经杀掉母鹿,臣实在不忍心再杀掉它的孩子。”说完哭泣不已。文帝于是放下弓箭,深感惊奇,而确定立曹叡为太子的心意。

黄初三年(222年)三月,曹叡升为平原王,后来曹丕下诏将其过继给郭皇后为子,进一步确定嫡长子的地位。然而曹叡因其母非善终,内心愤愤不平,后来才开始恭敬地侍奉嫡母,每日早晚都往皇后宫中定省问安,郭皇后也因自己无子,对曹叡慈爱有加。除了曹丕为曹叡诏令郑称为师,平原王府中还配置高堂隆为平原王傅,毌丘俭、何曾、吉茂等一干人等为文学属官。黄初四年,曹丕为曹叡聘河内世家大族虞氏为平原王妃,又选河内毛氏入宫,曹叡十分宠爱,出入都与其同乘舆辇。

226年,五月十六日丙辰(6月28日)魏文帝病危,立平原王曹叡為太子,召曹真、曹休、陳群、司馬懿,并受遗诏辅佐嗣主。十七日丁巳(6月29日)崩于嘉福殿。曹叡繼位,是為魏明帝。

明帝登基后首先必须对抗内外敌人的攻击,226年八月孙权攻江夏和襄阳、227年孟达反、到234年为止诸葛亮五次进攻曹魏、234年孙权攻合肥;明帝重用滿寵處理这些内外战争,亦重用曹真、張郃、司马懿等名將与诸葛亮作战。235年诸葛亮死后,魏蜀边境上的情况有所减缓,明帝开始在洛阳大建宫殿,常用人力、物力,大臣楊阜、高堂隆等對此一再勸練,明帝雖多未採納,但也不因此問罪臣屬。同年,他将养子曹芳封为齐王。

237年,聽從高堂隆的建議,發布《景初暦》,是歲將青龍五年春三月改為景初元年夏四月,同年辽东公孙渊造反,自立为燕王,明帝令司马懿攻辽东,司马懿遂带兵四万,和夏侯霸等人出征辽东,大破燕军,杀公孙渊,成功收复辽东。

从238年冬开始,魏明帝的健康开始恶化。239年初,魏明帝病重,曹叡本意让燕王曹宇为大将军,曹献、曹爽、曹肇、秦朗共同辅政,但曹宇一直不接受。于是曹叡单独召见刘放、孙资到其床边问话,问道“燕王为何一直不接受大将军的安排?”刘放和孙资回答:“燕王实在是自己知道不堪大任所以推辞”,曹叡又问:“曹爽可以代曹宇为大将军么?”刘放和孙资表示赞同,同时又多次强调应该迅速召见太尉司马懿来辅助朝纲,曹叡答应并令刘放起诏书。刘放、孙资退下之后,曹叡的想法突然改变,宣诏让司马懿不要入宫,过一段时间曹叡见到刘放、孙资说:“我同意召见司马懿,但是曹肇等人却让我不要这样做,差点坏了我的大事!”于是再次起草诏书,命曹爽、刘放、孙资一同接受诏令,同时免去曹宇、夏侯献、曹肇、秦朗等人的官职。

景初三年正月初一(239年1月22日),司马懿率师从辽东回到河内郡驻扎。明帝传令把他急招入卧室,拉着他的手嘱咐说:“终于等到你来,现在把后事托付给您,和大将军曹爽共佐曹芳。我在死前能见到你,也没什么遗憾的了。”又把齐王曹芳和秦王招来,嘱托司馬懿照顾。当天,明帝驾崩于洛阳宫嘉福殿,年仅三十五岁,《三国志》作三十六岁。。《三国志》载魏明帝崩于嘉福殿,《魏書》载他崩于九龙前殿。正月廿七癸丑日(2月17日),葬高平陵。

曹叡生母甄氏原为袁绍次子袁熙婦,據《三國志》卷三《明帝紀》記載魏明帝去世時“時年三十六”。裴松之在《三國志注》中計算曹叡年齡時,認為甄氏在建安九年(204年)八月曹操攻占邺城后才為曹丕所納,到景初三年正月初一丁亥日(239年1月22日)曹叡死時最多也只有虚岁三十五歲,不能計為三十六歲。由此如卢弼等人就認為曹叡或許是袁熙之子,陳壽故意對於年龄曲笔。

魏明帝曾經下令由盧毓來推舉官吏,並要求不要只看候選人的名氣,而要看他們的品行與能力。明帝表示,名氣就像是圖畫的餅一樣,根本無用,無法充飢。這就是「畫餅充飢」的典故。

陈寿:“明帝沉毅断识,任心而行,盖有君人之至概焉。于时百姓彫弊,四海分崩,不先聿修显祖,阐拓洪基,而遽追秦皇、汉武,宫馆是营,格之远猷,其殆疾乎!”(《三国志·魏书·明帝纪第三》)

孙权:“及操子丕,桀逆遗丑,荐作奸回,偷取天位,而叡么麽,寻丕凶迹,阻兵盗土,未伏厥诛。”(《三国志·卷四十七·吴书二·吴主传第二》)

孙盛:“魏明帝天资秀出,立发垂地,口吃少言,而沉毅好断。初,诸公受遗辅导,帝皆以方任处之,政自己出。而优礼大臣,开容善直,虽犯颜极谏,无所摧戮,其君人之量如此之伟也。然不思建德垂风,不固维城之基,至使大权偏据,社稷无卫,悲夫!”

刘晔:“秦始皇、汉孝武之俦,才具微不及耳。”(《世说新语》)

陆逊:“选用忠良,宽刑罚,布恩惠,薄赋省役,以悦民心,其患更深於操时。”(《三国志·吴书·张顾诸葛步传第七》)

钟会:“烈祖明皇帝奕世重光,恢拓洪业。”(《三国志·魏书·王毌丘诸葛邓锺传第二十八》)

阎缵:“及至明帝,因母得罪,废为平原侯,为置家臣庶子,师友文学,皆取正人,共相匡矫。兢兢慎罚,事父以孝,父没,事母以谨,闻于天下,于今称之。”(《上书理湣怀太子(司马遹)之冤》)

王沈:“好学多识,特留意于法理。”(《魏书》)

裴松之:“魏明帝一时明主。”

司马光:“汉主寿常慕汉武,魏明之为人。”(《资治通鉴·卷第九十六》)“帝沈毅明敏,任心而行,料简功能,屏绝浮伪。行师动众,论决大事,谋臣将相,咸服帝之大略。性特强识,虽左右小臣,官簿性行,名迹所履,及其父兄子弟,一经耳目,终不遗忘。”(《资治通鉴·卷第七十四》)

郭威:“汉高祖为义帝发丧,魏明帝正禅陵尊号,一时达礼,千古所称。”(《全唐文·卷一百二十三》)

胡应麟:“诗未有三世传者,既传而且煊赫,仅曹氏操、丕、睿耳。”(《诗薮》)

吕思勉:“魏文帝本来无甚才略。死后,儿子明帝继立,荒淫奢侈,朝政更坏。”(《中国通史:后汉的分裂和三国》)

蔡东藩:“曹叡奢淫无度,违理蔑伦,种种荒谬,俱足亡国,而反得平定辽东,擒斩公孙渊父子,是所谓天夺之鉴,而益其疾也。”(《后汉演义》)

马植杰:“综观曹叡之行事,优缺点各占一半,其优点是善为军计、明察断狱、比较能容人直谏。曹叡在容受直言、不杀谏臣方面,在古代封建君主中是少见的,这算是他的特色。曹叡的最大缺点是奢淫过度,还有一个重要的失误,则在确定继承人和辅政大臣方面。”(《魏的政治与司马氏专政》)

\subsubsection{太和}

\begin{longtable}{|>{\centering\scriptsize}m{2em}|>{\centering\scriptsize}m{1.3em}|>{\centering}m{8.8em}|}
  % \caption{秦王政}\
  \toprule
  \SimHei \normalsize 年数 & \SimHei \scriptsize 公元 & \SimHei 大事件 \tabularnewline
  % \midrule
  \endfirsthead
  \toprule
  \SimHei \normalsize 年数 & \SimHei \scriptsize 公元 & \SimHei 大事件 \tabularnewline
  \midrule
  \endhead
  \midrule
  元年 & 227 & \tabularnewline\hline
  二年 & 228 & \tabularnewline\hline
  三年 & 229 & \tabularnewline\hline
  四年 & 230 & \tabularnewline\hline
  五年 & 231 & \tabularnewline\hline
  六年 & 232 & \tabularnewline\hline
  七年 & 233 & \tabularnewline
  \bottomrule
\end{longtable}

\subsubsection{青龙}

\begin{longtable}{|>{\centering\scriptsize}m{2em}|>{\centering\scriptsize}m{1.3em}|>{\centering}m{8.8em}|}
  % \caption{秦王政}\
  \toprule
  \SimHei \normalsize 年数 & \SimHei \scriptsize 公元 & \SimHei 大事件 \tabularnewline
  % \midrule
  \endfirsthead
  \toprule
  \SimHei \normalsize 年数 & \SimHei \scriptsize 公元 & \SimHei 大事件 \tabularnewline
  \midrule
  \endhead
  \midrule
  元年 & 233 & \tabularnewline\hline
  二年 & 234 & \tabularnewline\hline
  三年 & 235 & \tabularnewline\hline
  四年 & 236 & \tabularnewline\hline
  五年 & 237 & \tabularnewline
  \bottomrule
\end{longtable}

\subsubsection{景初}

\begin{longtable}{|>{\centering\scriptsize}m{2em}|>{\centering\scriptsize}m{1.3em}|>{\centering}m{8.8em}|}
  % \caption{秦王政}\
  \toprule
  \SimHei \normalsize 年数 & \SimHei \scriptsize 公元 & \SimHei 大事件 \tabularnewline
  % \midrule
  \endfirsthead
  \toprule
  \SimHei \normalsize 年数 & \SimHei \scriptsize 公元 & \SimHei 大事件 \tabularnewline
  \midrule
  \endhead
  \midrule
  元年 & 237 & \tabularnewline\hline
  二年 & 238 & \tabularnewline\hline
  三年 & 239 & \tabularnewline
  \bottomrule
\end{longtable}


%%% Local Variables:
%%% mode: latex
%%% TeX-engine: xetex
%%% TeX-master: "../../Main"
%%% End:

%% -*- coding: utf-8 -*-
%% Time-stamp: <Chen Wang: 2021-11-01 11:35:44>

\subsection{少帝曹芳\tiny(239-254)}

\subsubsection{生平}

曹芳(232年-274年),字蘭卿,魏明帝養子,是三国時曹魏第三代皇帝,在位15年。在《三國志》中與曹髦、曹奐合稱三少帝。

曹芳是曹魏在位最久的君主,但也是曹魏第一位傀儡皇帝,實權先後由曹爽、司馬懿和司馬師掌握。

魏明帝的亲生儿子全部夭折,曹芳是其养子。但自小在宮中成長的曹芳,不知出自何宗室,出生來歷一概不詳。《魏氏春秋》記載,曹芳應該是任城王曹楷之子。

魏青龙三年,封为齐王,239年被立为太子,同年登基即帝位,年僅八歲。

第二年,即240年,改年号为正始,由大将军曹爽和太傅司马懿共同辅政。曹爽主政时,架空司马懿、隔离郭太后、起用新人。

魏正始十年(249年),司马懿發動政變,起兵控制雒邑,史稱高平陵之變。71歲的司马懿族滅曹爽,从此由司馬氏独掌军国大权,同年改元为嘉平。

251年,司马懿死后,長子司馬師受命總攬朝政,曹芳聯合李豐、張緝、夏侯玄等意圖罷黜司馬師,欲改立夏侯玄為大將軍,然而三人被司馬師搜出“衣帶詔”,夏侯玄、李豐、張緝受腰斬滅族,並且盡滅李氏、夏侯氏、張氏。张缉女张皇后也被废位。

254年,姜维进攻陇右,司马昭从许昌征还京城洛阳,准备前去讨伐姜维。曹芳前往平乐观慰问司马昭部队,中领军和大臣想趁着曹芳和司马昭会面的机会杀死司马昭,夺其军队讨伐司马师,但曹芳不敢这么做,不過图谋杀司马昭一事卻為司马师所得知,司马师於是决定废帝,派郭太后从父长水校尉郭芝告知郭太后。当时曹芳正与太后相对而坐,郭芝对曹芳说:“大将军想废了陛下,立彭城王曹据。”曹芳起身而去,太后也不悦,但郭芝说司马师决心已下,勒兵于外。太后说想见司马师,為郭芝所拒。郭太后只得取玺绶给郭芝,郭芝再给司马师。司马师以太后令告知群臣,由群臣上奏郭太后,謂曹芳年长不亲政、沉迷女色、废弃讲学、弃辱儒士、与优人、保林等淫乱作乐,并弹打进谏的清商令狐景、清商丞庞熙,乃至用烧铁重伤令狐景身体、太后丧母时不尽礼等罪,请依霍光故事废曹芳。曹芳因此被废,當年22歲,仍按登基前为齐王,另立明帝侄曹髦为帝。曹芳被迁居西宫,乘王车,泣别太后,从太极殿南出宫,数十大臣相送。在联署上奏废帝时,署名第一的太尉司马孚很悲伤,哭得尤其多。

此后,司马师、司马昭相继掌权。

266年,晋朝立国,曹芳被封为邵陵县公,274年逝世,享年43歲。谥号厉,故又称邵陵厉公。

曹芳登基后,为避讳他的名字“芳”,而将位于洛阳北宫的皇宫花园“芳林苑”改名为“华林苑”。后来南北朝分裂,建业、长安的皇宫花园都因循旧制命名为华林苑。

滕胤:“曹芳暗劣,而政在私门,彼之民臣,固有离心。”(《资治通鉴·卷七十六》)

毌丘俭:“懿每叹说齐王自堪人主,君臣之义定。奉事以来十有五载,始欲归政,按行武库,诏问禁兵不得妄出。师自知奸慝,人神所不祐,矫废君主,加之以罪。”(《三国志·魏书·王毌丘诸葛邓锺传第二十八》)

郭太后废曹芳诏令:“皇帝芳春秋已长,不亲万机,耽淫内宠,沈漫女德,日延倡优,纵其丑谑;迎六宫家人留止内房,毁人伦之叙,乱男女之节;恭孝日亏,悖慠滋甚,不可以承天绪,奉宗庙。”(《三国志·卷四·魏书四·三少帝纪第四》)

陈寿:“古者以天下为公,唯贤是与。后代世位,立子以适;若适嗣不继,则宜取旁亲明德,若汉之文、宣者,斯不易之常准也。明帝既不能然,情系私爱,抚养婴孩,传以大器,托付不专,必参枝族,终于曹爽诛夷,齐王替位。”(《三国志·卷四·魏书四·三少帝纪第四》)

梁章钜:“齐王临御之初,即罢宫室工作,免官奴婢六十以上为良人,出内府金银销冶以供军用;二年通《论语》,五年通《尚书》,七年通《礼记》,三祀孔子,以颜子配;良法美政,史不绝书。”(《三国志集解》)

何焯:“芳临御数载,非若昌邑始征,若果君德有阙,播恶于众,师何难执以为辞?今称太后之令,发床第之私,有以知其为诬矣。”(《三国志集解》)

\subsubsection{正始}

\begin{longtable}{|>{\centering\scriptsize}m{2em}|>{\centering\scriptsize}m{1.3em}|>{\centering}m{8.8em}|}
  % \caption{秦王政}\
  \toprule
  \SimHei \normalsize 年数 & \SimHei \scriptsize 公元 & \SimHei 大事件 \tabularnewline
  % \midrule
  \endfirsthead
  \toprule
  \SimHei \normalsize 年数 & \SimHei \scriptsize 公元 & \SimHei 大事件 \tabularnewline
  \midrule
  \endhead
  \midrule
  元年 & 240 & \tabularnewline\hline
  二年 & 241 & \tabularnewline\hline
  三年 & 242 & \tabularnewline\hline
  四年 & 243 & \tabularnewline\hline
  五年 & 244 & \tabularnewline\hline
  六年 & 245 & \tabularnewline\hline
  七年 & 246 & \tabularnewline\hline
  八年 & 247 & \tabularnewline\hline
  九年 & 248 & \tabularnewline\hline
  十年 & 249 & \tabularnewline
  \bottomrule
\end{longtable}

\subsubsection{嘉平}

\begin{longtable}{|>{\centering\scriptsize}m{2em}|>{\centering\scriptsize}m{1.3em}|>{\centering}m{8.8em}|}
  % \caption{秦王政}\
  \toprule
  \SimHei \normalsize 年数 & \SimHei \scriptsize 公元 & \SimHei 大事件 \tabularnewline
  % \midrule
  \endfirsthead
  \toprule
  \SimHei \normalsize 年数 & \SimHei \scriptsize 公元 & \SimHei 大事件 \tabularnewline
  \midrule
  \endhead
  \midrule
  元年 & 249 & \tabularnewline\hline
  二年 & 250 & \tabularnewline\hline
  三年 & 251 & \tabularnewline\hline
  四年 & 252 & \tabularnewline\hline
  五年 & 253 & \tabularnewline\hline
  六年 & 254 & \tabularnewline
  \bottomrule
\end{longtable}


%%% Local Variables:
%%% mode: latex
%%% TeX-engine: xetex
%%% TeX-master: "../../Main"
%%% End:

%% -*- coding: utf-8 -*-
%% Time-stamp: <Chen Wang: 2019-12-17 22:30:20>

\subsection{高贵乡公\tiny(254-260)}

\subsubsection{生平}

曹髦(241年11月15日-260年6月2日),字彥士,三國時期曹魏第四代皇帝(254年-260年在位),曹霖之子,魏文帝曹丕孙。在《三國志》中與曹芳、曹奐合稱三少帝。

正始五年(244年),封郯县高貴鄉公。

254年,司馬師廢掉時年23歲曹芳的皇位,打算立曹操之子彭城王曹据为皇帝。郭太后指出曹据於礼是她(丈夫)的叔叔,曹据若继皇位,則太后之位將免除,因此易魏文帝曹丕之孫高贵乡公曹髦为帝,作为其无嗣的伯父曹叡魏明帝的后嗣,改元「正元」;曹髦當年12歲,實權先後由司馬師和司馬昭掌握。

司馬師曾問鍾會曹髦的能力,鍾會回答:「文同陳思,武類太祖」(文采如同陳思王曹植,武可比太祖曹操)。司马师说若果然如此,是社稷之幸。但隨著小皇帝長大,曹髦對專權的司馬昭日益不滿。不久他寫了一首《黃龍歌》,司馬昭發現,起了戒心。

甘露五年(260年)五月曹髦召見王沈、王經、王業等三人,憤慨說道:「司馬昭之心,路人皆知也!朕不能坐受廢辱,今日當與卿等自出討之。」不顧郭太后及眾臣的反對,率領宮人三百餘人討伐司馬昭。王沈與王業先行,向司馬昭通風報信。司馬昭旋即派兵入宮鎮壓,雙方在宮內東止車門相遇。中護軍賈充在南闕下,率軍迎戰曹髦。賈充命令成濟殺曹髦,成濟一劍刺穿曹髦胸膛,曹髦斷氣,當場死在車上,血流滿地,仅虚岁20歲,史稱司馬昭弒君。

隨後,司馬昭假借郭太后的名義下詔“高贵乡公悖逆不道,自陷大祸,依汉昌邑王罪废故事,以民礼葬”。但在司马昭叔父司马孚请求下,以王礼下葬曹髦于洛阳西北三十里瀍涧之滨,仅下车数乘、不设旌旐,百姓相聚而观,说:“这就是前日所杀的天子。”有人甚至掩面而泣,悲伤不能自已。由於輿論憤憤不平,司馬昭遂在事發後的20天將弒君罪狀全推給成濟,称成济违背自己命令杀死曹髦,以「大逆不道」罪誅殺成濟一族,斷成氏一脈。司馬昭後立曹奐為曹魏皇帝。

曹髦擅長寫詩文。他的繪畫藝術也相當不錯,是一個善於琴棋書畫的才子。


\subsubsection{正元}

\begin{longtable}{|>{\centering\scriptsize}m{2em}|>{\centering\scriptsize}m{1.3em}|>{\centering}m{8.8em}|}
  % \caption{秦王政}\
  \toprule
  \SimHei \normalsize 年数 & \SimHei \scriptsize 公元 & \SimHei 大事件 \tabularnewline
  % \midrule
  \endfirsthead
  \toprule
  \SimHei \normalsize 年数 & \SimHei \scriptsize 公元 & \SimHei 大事件 \tabularnewline
  \midrule
  \endhead
  \midrule
  元年 & 254 & \tabularnewline\hline
  二年 & 255 & \tabularnewline\hline
  三年 & 256 & \tabularnewline
  \bottomrule
\end{longtable}

\subsubsection{甘露}

\begin{longtable}{|>{\centering\scriptsize}m{2em}|>{\centering\scriptsize}m{1.3em}|>{\centering}m{8.8em}|}
  % \caption{秦王政}\
  \toprule
  \SimHei \normalsize 年数 & \SimHei \scriptsize 公元 & \SimHei 大事件 \tabularnewline
  % \midrule
  \endfirsthead
  \toprule
  \SimHei \normalsize 年数 & \SimHei \scriptsize 公元 & \SimHei 大事件 \tabularnewline
  \midrule
  \endhead
  \midrule
  元年 & 256 & \tabularnewline\hline
  二年 & 257 & \tabularnewline\hline
  三年 & 258 & \tabularnewline\hline
  四年 & 259 & \tabularnewline\hline
  五年 & 260 & \tabularnewline
  \bottomrule
\end{longtable}


%%% Local Variables:
%%% mode: latex
%%% TeX-engine: xetex
%%% TeX-master: "../../Main"
%%% End:

%% -*- coding: utf-8 -*-
%% Time-stamp: <Chen Wang: 2021-11-01 11:35:58>

\subsection{元帝曹奂\tiny(260-265)}

\subsubsection{生平}

魏元帝曹奂(246年-302年),原名璜,字景明,是曹魏最后一個皇帝,260年到266年在位。

曹奂是燕王曹宇之子,曹操之孙,初封常道乡公。

甘露五年(260年)魏帝曹髦试图从司马氏手中夺回权力失敗,兵敗被杀,丞相司马昭派儿子使持节行中护军中垒将军司马炎迎立曹璜为皇帝,改名曹奂,入继为堂兄魏明帝曹叡的儿子。曹奂实际上毫无权力,在大臣和军队中也没有任何势力,完全是司马昭的傀儡。

曹奂在位期间,263年曹魏大将邓艾和鍾会伐蜀汉,蜀汉灭亡。伐蜀期间,丞相司马昭以伐蜀之功被晋升为晋公,相国,加九锡。

蜀汉灭亡没多久,丞相司马昭又进爵晋王,不久去世,司马炎于咸熙二年(265年)篡位,魏亡。

魏亡後,曹奐被封為陳留王,並遷居邺城。出城時,太傅司馬孚握著他的手說:「我到死都是大魏的忠臣。」

晋武帝司马炎仿漢獻帝劉協例,允许曹奐仍保有皇帝仪仗、用皇家礼仪祭祖、不以臣下自称。

晉惠帝太安元年(302年)曹奐死於許昌,享年五十七歲,以皇帝礼下葬,谥号元皇帝。

曹奐的後人沒有在官方的紀錄出現,而史料中亦未记载他有沒有後人。由於他離世時正值八王之亂,很多王公士族在當時均為亂軍所殺,並且紀錄可能在那段期間被燒毀。東晉以降,陳留王由曹操玄孫曹勱及其後裔承襲,但無法得知他们和曹奐的關係。邺城遗址附近一直有相传为曹奂墓的土墩,后来考古发掘证实并非曹奂墓。


\subsubsection{景元}

\begin{longtable}{|>{\centering\scriptsize}m{2em}|>{\centering\scriptsize}m{1.3em}|>{\centering}m{8.8em}|}
  % \caption{秦王政}\
  \toprule
  \SimHei \normalsize 年数 & \SimHei \scriptsize 公元 & \SimHei 大事件 \tabularnewline
  % \midrule
  \endfirsthead
  \toprule
  \SimHei \normalsize 年数 & \SimHei \scriptsize 公元 & \SimHei 大事件 \tabularnewline
  \midrule
  \endhead
  \midrule
  元年 & 260 & \tabularnewline\hline
  二年 & 261 & \tabularnewline\hline
  三年 & 262 & \tabularnewline\hline
  四年 & 263 & \tabularnewline\hline
  五年 & 264 & \tabularnewline
  \bottomrule
\end{longtable}

\subsubsection{咸熙}

\begin{longtable}{|>{\centering\scriptsize}m{2em}|>{\centering\scriptsize}m{1.3em}|>{\centering}m{8.8em}|}
  % \caption{秦王政}\
  \toprule
  \SimHei \normalsize 年数 & \SimHei \scriptsize 公元 & \SimHei 大事件 \tabularnewline
  % \midrule
  \endfirsthead
  \toprule
  \SimHei \normalsize 年数 & \SimHei \scriptsize 公元 & \SimHei 大事件 \tabularnewline
  \midrule
  \endhead
  \midrule
  元年 & 264 & \tabularnewline\hline
  二年 & 265 & \tabularnewline
  \bottomrule
\end{longtable}


%%% Local Variables:
%%% mode: latex
%%% TeX-engine: xetex
%%% TeX-master: "../../Main"
%%% End:


%%% Local Variables:
%%% mode: latex
%%% TeX-engine: xetex
%%% TeX-master: "../../Main"
%%% End:
