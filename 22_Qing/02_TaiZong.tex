%% -*- coding: utf-8 -*-
%% Time-stamp: <Chen Wang: 2019-12-26 21:59:16>

\section{太宗\tiny(1626-1643)}

\subsubsection{生平}

1635年,皇太極打败林丹汗,令其遁逃至大草滩(今甘肃境),取得傳國玉璽(原為元朝所有)。漠南蒙古各部向後金臣服,為其上尊號博格达汗。后金的第二代大汗崇德元年四月十一乙酉日(1636年5月15日),皇太极改国号为“大清”,改元崇德,皇太极是1637年,皇太極率軍親自征討不服從後金統治的朝鮮,迫使朝鮮向其臣服;從此朝鮮成為清朝的藩屬。此後,朝鮮的親明派勢力被剷除,大清開始專心進攻明朝。

崇德六年即崇祯十四年(1641年)七月,帶病急援松錦之戰,史载“上行急,鼻衄不止,承以椀”,马不停蹄,昼夜兼行五百餘里。在松山大敗明軍,生俘洪承疇,並且令其投降,大大打擊了明軍的士氣。《清太宗实录》记载:“是役也,计斩杀敌众五万三千七百八十三,获马七千四百四十匹、骆驼六十六、甲胄九千三百四十六副。明兵自杏山,南至塔山,赴海死者甚众,所弃马匹、甲胄以数万计。海中浮尸漂荡,多如雁鹜。”此役为后来清朝灭明征服天下立下基础。《清史稿·太宗本纪》评价:“允文允武,内修政事,外勤讨伐,用兵如神,所向有功。”

崇德八年八月初九日(1643年9月21日)晚間十點皇太極崩逝於瀋陽故宮清寧宮東暖閣內,享年五十岁。安葬于沈阳清昭陵(今沈阳市北陵公园北)。由於死前未立繼承人,其弟睿親王多爾袞與長子豪格爭位不下,彼此陳兵示威。最終多爾袞獨排眾議,擁立莊妃的六歲兒子福臨,是為清世祖,改元順治。

後來順治帝諡皇太極為文皇帝,廟號太宗,統稱太宗文皇帝。

皇太極在一方面重用投奔後金的漢族官員為自己的智囊團;而在另一方面,皇太極多次强调国语骑射,是防止满洲人受到“服汉人衣冠,尽忘本国语言”薰染(《清太宗实录》 卷三四 崇德二年四月丁酉),危及满洲民族政权的长远存在;为此,皇太极反复告戒满洲贵族,应恪守满洲衣冠和善于骑射的风俗习惯云云,还多次下“上谕”强调这一点。

\subsection{崇德}

\begin{longtable}{|>{\centering\scriptsize}m{2em}|>{\centering\scriptsize}m{1.3em}|>{\centering}m{8.8em}|}
  % \caption{秦王政}\
  \toprule
  \SimHei \normalsize 年数 & \SimHei \scriptsize 公元 & \SimHei 大事件 \tabularnewline
  % \midrule
  \endfirsthead
  \toprule
  \SimHei \normalsize 年数 & \SimHei \scriptsize 公元 & \SimHei 大事件 \tabularnewline
  \midrule
  \endhead
  \midrule
  元年 & 1636 & \tabularnewline\hline
  二年 & 1637 & \tabularnewline\hline
  三年 & 1638 & \tabularnewline\hline
  四年 & 1639 & \tabularnewline\hline
  五年 & 1640 & \tabularnewline\hline
  六年 & 1641 & \tabularnewline\hline
  七年 & 1642 & \tabularnewline\hline
  八年 & 1643 & \tabularnewline
  \bottomrule
\end{longtable}


%%% Local Variables:
%%% mode: latex
%%% TeX-engine: xetex
%%% TeX-master: "../Main"
%%% End:
