%% -*- coding: utf-8 -*-
%% Time-stamp: <Chen Wang: 2019-12-26 21:57:50>

\chapter{清\tiny(1636-1912)}

\section{简介}

清朝(1616年2月17日、1636年5月15日或1644年6月5日-1912年2月12日),正式國号為大清,對外使用中國、大清国、大清帝國、中華大清國等名稱,又称满清(蒙古語:манж Чин)、皇清,是中国历史上由滿族建立的一個大一統朝代,亦為中國歷史上最後一個王朝,统治者为起源於明代建州女真的爱新觉罗氏。

满人源自女真,皇族愛新覺羅氏為建州女真一部,隸屬明朝建州卫管轄之部眾。建州卫是明朝在中國东北设立的一个卫所,屬於明朝邊防行政单位,曾隶属于奴儿干都司管辖,而爱新觉罗氏世代為建州左衛都指揮使。1616年,女真族人努尔哈赤在今中国东北地区建国称汗,建立大金(史稱後金),定都赫图阿拉,又稱為“兴京”(今辽宁新宾)。1636年,努尔哈赤的繼承者皇太极在盛京(今瀋陽)称帝,定国号为「大清」,當時其領土僅止於中國東北及漠南地區,但已對退守長城以南的明朝造成重大威脅。1644年,大順李自成率军攻陷北京,明朝灭亡。同年,吳三桂部等原明朝殘餘軍隊為對抗李自成而歸降清軍,由此清軍進入山海关内,在擊敗農民軍後遷都北京,並開始大規模南下。其后的数十年时间内,清朝陆续消灭华北殘餘明朝勢力、李自成的大順军、張獻忠的大西國、南明和明鄭等势力,统一中国全境。歷經康熙、雍正及乾隆三帝,清朝的綜合國力及經濟文化逐步得到恢復和發展,统治着辽阔的領土及藩屬國,史稱康雍乾盛世,是清朝發展的高峰時期,亦是中國歷史上最輝煌的時期之一。

清朝政治制度基本上沿襲明朝,其最高決策單位隨皇帝的授權而變動,例如軍機處、總理衙門等,除提升行政效率外,也使皇帝能充分掌權。清中期文字獄興盛,若有疑似反清復明的運動與散播被認為不利皇帝的消息,往往會引來冤獄,牽連多人受害。軍事方面,清朝在初期建樹巨大,原先以旗人的八旗軍為核心精銳,龐大的綠營為主力,而後期逐漸以綠營和地方團練如湘軍、淮軍為支柱。清朝領土在歷史上具有重要意義,清初期所進行的陸上與海上擴張,以及對邊疆地區的強而有力的施政,奠定了現代中國所繼承的基本版圖與統治。清极盛时可達1310万平方公里之巨,在中國歷史上僅次於元朝,即使在清末也維持1130萬平方公里。政治穩定、廣泛種植新作物與賦稅制度的改變,使得中國人口最後突破以往的平均值,達到四億左右。國內與國外的貿易提升,帶動經濟農業與手工業的發展。

外交接觸方面,除了與周邊東亞國家有往來,當時正值航海時代的歐洲人,直接透過海路來與中國貿易和傳教活動,當時主要集中於廣州,不過西方列強在18世紀左右憑藉著工業化的突破,開始大幅拉開國力差距,鴉片戰爭開啟中國近代歷史,使中國由東亞的中心變成列強環伺的國家。列強迫使清廷簽訂不平等條約,以武力獲得在華利益。清朝在抵抗外侮與內憂的同時,也一直處於改革派與守舊派拉鋸的局面。在列强入侵的同时西方科學與文化亦引入中國,讓清朝發起一連串的改革與革命,如洋務運動,促使中國文化的成長與革新。然而甲午戰爭的失敗使改革的努力受到沉重打擊,并使中國國際地位大為降低,列強加速劃分勢力範圍。而其後的維新運動也隨守舊派抵制而告終。在義和團運動排外反列強失敗、引來八國聯軍侵華後,清廷也推動清末新政,虽取得一些成效,但部分內容讓许多立憲派知識分子失望,轉而支持革命。1911年辛亥革命爆發,1912年1月1日中華民國在南京正式成立,同年宣統帝(溥儀)於2月12日宣布退位,清朝正式滅亡。清朝從後金時期算起,共經歷12位皇帝,13个年號(包含太祖的天命和太宗的天聰),國祚長296年,又有滿清十三皇朝之稱;如果自1636年皇太極改國號為清開始算起,共經歷11位皇帝,12个年號。國祚長276年。如果自1644年遷都北京入主中原以來則有10帝,歷時268年。

1616年努爾哈赤建立後金政權,1636年皇太極改國號为「大清」,入关后成爲元朝以來中國歷史上第三個(也是最后一個)把“大”字加入正式國號之中的大一統皇朝。皇太極改国号的原因,未有史料明確記載,有說法認為是為了掩蓋女真曾臣服於明的歷史以鼓舞士氣;「清」之國號,或云是金的諧音,而且滿人尚青,加水字邊以符合五德終始說,用水免去朱明(朱是指明朝皇帝的姓氏,明是明朝國號)之「火」;另有一种观点认为“大清”这一国号并非来自汉语,而是满语中的一个蒙古借词「ᠳᠠᠢᠼᠢᠨ」(Daičin),原意为“善战者,戰士”,故“大清国”的意思是“上国”或“善战之国”;但此论述存有争议,有学者认为清朝统治者在入关后逐渐淡忘了该词的原始含义。

清朝以前“中國”的詞義基本上為地理、民族、政治區域和文化意義,而現代國體意義上的“中國”,直至1689年9月7日《中俄尼布楚條約》簽訂後才首次正式出現於外交文件上。

自入主中原之后,清朝皇帝正式以“中国”稱呼全境。1689年,清朝康熙帝在与俄国签订的具有现代国际法水准的边界条约——《尼布楚条约》中,首次将“中国”作为正式國家名稱使用,与“俄羅斯”相对,該「中國」指包括蒙古以及中國東北等地在內的整個清帝國。

中西史學家如黃興濤博士及新清史學派學者、東亞史學家歐立德(Mark C. Elliott)認為,康雍乾之後的中國,是被清朝皇帝、滿人、漢人等其他族群共同認同並加以再造過的中國。現代「中國」的概念,正是來自於清朝所塑造的中國觀。

15世紀初期,位於東北亞的女真族大致分成三部分,其中以建州女真最為強大。建州女真源自於遼金時代生女真完顏部的附屬五國部,居於牡丹江與松花江匯流地方,在元代組成為五個女真萬戶府,元末因受野人女真及朝鮮滋擾不斷,以及明朝的招撫等因素,其殘餘部族選擇歸附明朝。明朝将其部遷至原渤海地(今綏芬河流域)設置為他們新的聚居地。

1403年明朝依據原渤海建州的地名稱呼為建州女真,並設置了衛所這一地方軍事行政機構,冊封阿哈出(賜名李承善)為建州卫指揮使,猛哥帖木兒為建州衛左都督。其後1416年又建立建州左衛,以猛哥帖木兒為指揮使,並賜姓童。建州衛早期歸屬奴兒干都司,奴兒干都司廢除後改屬遼東都司。

猛哥帖木兒在被野人女真所殺後,其弟凡察與子董山被迫率眾南移,最後定居赫圖阿拉(今辽宁新賓),併入建州衛內。1442年,明朝又從建州左衛分立出右衛,以凡察領導右衛、董山領導左衛,形成建州三衛,三衛首領也是世襲制,但須經明朝政府認可後方生效。

由於建州三衛對明朝過度干預女真產生不滿,因而逐漸不遵守朝廷命令。而此時女真各部也因嫌隙已四分五裂。1467年明朝聯合朝鮮削弱建州三衛(即成化犁庭),並且於遼東邊界興建長城。明朝萬曆初年,董山的後代覺昌安與其子塔克世偕同明朝遼東總兵李成梁,以建州右衛王杲叛亂為由攻滅王杲與其子阿台,然而覺昌安同其子塔克世在入城勸降叛明的阿台時發生混戰,覺昌安父子也在混戰中被明軍誤殺死亡。1586年明廷襲封塔克世之子努爾哈赤為指揮使,世襲建州衛作為補償。

努爾哈赤深覺被明朝背叛,以祖、父遺留的十三副遺甲崛起,統一建州女真後陸續併吞女真各部,並與漠南蒙古友好。

建州女真勢力日盛,1595年明朝授予努爾哈赤龍虎將軍的稱號,其勢力更加強大。1603年努爾哈赤在赫圖阿拉築城,兩年後致遼撫趙楫、總兵李成梁的呈文中說:“我奴兒哈赤收管我建州國之人,看守朝廷九百五十余里邊疆”,以守疆名義索要更高權利,地位仍與過去相同,聲勢則已不同以往。一直到1616年,努爾哈赤在建立八旗制度後於赫圖阿拉(後稱興京)稱汗立國,即後金汗国。兩年後他以「七大恨」為由起兵反明。檄文中儼然以“北朝”自居。努爾哈赤在1619年的薩爾滸之戰中,擊敗楊鎬指揮的明軍、朝鮮與葉赫聯軍;接連佔領瀋陽、遼陽、撫順等遼東城市,戰無不勝的他更堅定了入主中原之志,隨後戰事集中於遼西地區。1626年,在与明袁崇煥交战的寧遠戰役中受挫,数月后逝世。第八子皇太極歷經權力鬥爭後繼位。

皇太極即位之后,针对努尔哈赤时期的社会矛盾进行一系列改革,史称“天聪新政”。皇太极改文馆为内国史院、内秘书院、内弘文院,这是清朝内阁的雏形。还继续完善和扩大蒙古八旗、汉军八旗,设立理藩院管辖蒙古等地事务。1634年,將都城瀋陽易名「盛京」,更改「女真」族名為「滿洲」。1635年,多爾袞於征伐漠南蒙古時,聲稱得到元順帝離開中原時帶走的傳國玉璽 ,皇太極親率其子與諸官出城迎接,拜天行禮。1636年,漠南诸部尊皇太極为“博格达彻辰汗”,皇太極在盛京称帝,改年号为崇德,改國號為大清。當時明朝在關外的勢力尚有袁崇煥守備的錦州、寧遠與松山等三城。皇太極為繞道避開此防線,首先穩定根據地。他先脅迫明朝求和未果,隨後成功降伏西邊蒙古察哈爾部和東邊朝鮮。接著,皇太極經察哈爾繞道入侵明朝首都北京。最後崇禎帝(有人他认为中了反間計),处决援救北京的袁崇煥,史稱己巳之變。這種藉由繞道入侵的作法後來又執行五次,與明朝內部的流寇一同消耗明朝的經濟力。1642年,清軍於松錦之戰擊潰明軍並收降洪承疇等人,奪取明朝在關外的所有堡壘,明朝防線移至山海關及長城沿線。1643年皇太極病死,第九子福臨繼位,是為順治帝,由其叔多爾袞攝政。

明朝崇禎末年,民變勢力(史稱“流寇”)如李自成於陝西西安建國大順,張獻忠於四川成都建國大西。1644年李自成經河南、山西順利的攻入北京,明思宗在煤山上吊自殺,明亡。同年山海關守將吳三桂不願投降大順。面對李自成的順軍,吳三桂引清兵入關,於一片石戰役擊敗順軍,史稱甲申之變。李自成放棄北京,率軍退回陝西。清朝攝政王多爾袞成功迎順治帝入關,順治在北京天壇圓丘行祭天之禮,祝文宣布受天命、建王朝,名稱仍用大清國號,並將首都遷到北京。同一時間,明朝馬士英擁護福王在明朝陪都南京应天府稱帝,即弘光帝,南明成立。然而弘光朝因為黨爭與宦官之亂而混亂分裂。

多爾袞先派阿濟格、吳三桂與多鐸、孔有德分陝北、河南二路攻打陝西李自成,李自成最後於湖北滅亡;派豪格攻滅四川張獻忠,其餘部投降南明以抗清。多爾袞接著對付位於江南、內鬥分裂的南明諸勢力。1645年,多鐸率清軍攻破史可法駐守的揚州,弘光帝逃至蕪湖被逮,送到北京殺害。明朝魯王朱以海與唐王隆武帝分別在浙江與福建建立勢力,然而雙方不和,不久被清軍各個擊破,擁護隆武帝的鄭芝龍也宣佈投降。之後桂王於廣東的肇慶即位为永曆帝,期間瞿式耜、李定國、鄭成功及其他明將先後收復華南各省,最後因為距離互相難以照應,內部又發生叛變而節節敗退。1661年,清軍攻入雲南,逃亡緬甸的永曆帝最後被吳三桂殺死,南明亡。此時只剩下臺灣的明鄭(鄭成功勢力)和緬甸果敢的明軍,清朝基本佔領明朝全部領地。由於華南反清勢力較大,清帝冊封吳三桂、耿仲明與尚可喜為王以鎮守雲南、廣東與福建等地,史稱三藩。

多爾袞在清軍攻入关内後推行剃髮易服政策,導致关内汉人對清廷態度驟變,甚至極力反抗,如「入关之初,严禁杀掠,故中原人士无不悦服,及有剃头之举,民皆愤怒,或见我人泣而言曰,我以何罪独为此剃头乎?」,而明朝官員左懋第亦言「我頭可斷、髮不可斷,我早辦一死矣。」。清軍對反抗者進行鎮壓,重要事件如揚州十日、嘉定三屠與庚寅之劫等。清廷制定的圈地和投充政策使人民放棄土地,增加更多流民。為此又制定嚴禁奴僕逃亡的逃人法,激化京畿地區的民族與階層矛盾。

但後續清廷下令停止這些政策,並實行獎勵墾荒、減免捐稅的政策。並且正式開科取士,追尊崇禎帝與明朝忠臣。

1661年,順治帝英年早逝,其子8歲的玄燁即位,即康熙帝,由索尼、遏必隆、蘇克薩哈與鰲拜四大臣輔政。康熙帝於繼位之初即運用計謀消滅跋扈的權臣鰲拜以穩固皇權。三藩勢力如吳三桂、耿精忠與尚之信等涵蓋全國之半,他們先後請求撤藩以試探清廷。當時部分大臣擔憂三藩叛變而反對。最後,孝莊太皇太后與康熙帝無懼三藩而同意撤除。這使得三藩與陝西王輔臣、廣西孫延齡與台灣明鄭的鄭經聯合發動三藩之亂。在這九年期間,反清勢力遍及華中、華南,吳三桂更於後期稱帝建國周。然而清軍採取積極防禦,進軍陝西、江西以切割叛軍。加上吳三桂沒有積極北伐,反清聯軍因佔領地與吳三桂過度干涉而發生糾紛,最後王輔臣、耿繼忠與尚之信先後投降,佔領福建沿岸的鄭經被擊敗。1681年清軍攻入雲南,繼承吳三桂之位的吳世璠在昆明自殺,三藩之亂最終在1681年被完全撲滅,国家遭受较大的损失,在四川、云南以及江西等地有不少人被殺害,从许多记载来看,在三藩之亂时,清軍,叛軍,土匪等曾造成屠杀平民事件,不仅仅在川滇,其他相关地区也遭受类似的厄运,比如江西百姓遭受屠戮的数量就不少,“剿洗,玉石难分,老幼死于锋镝,妇子悉为俘囚,白骨遍野,民无噍类”。同年,鄭經之子鄭克塽繼位,明鄭因內亂不断导致不少將領降清。清朝派明鄭降將施琅率領水師攻打臺灣。施琅佔領澎湖,逼近東寧(今台灣台南),鄭克塽率領大臣降清,至此明鄭亡。

康熙帝平定三藩後,清朝進入康雍乾時期,這段時期是清朝發展的高峰時期,史學界通常稱為“康雍乾盛世”。康熙帝為政寬仁,留心民間疾苦,在他親政不久後,便宣佈停止圈地,放寬墾荒地的免稅年限。他還著手整頓吏治,恢復京察、大計等考核制度。受康熙帝的「滋生人丁,永不加賦」(攤丁入畝)政策以及外來農作物的引進等影響,清朝人口大大提升。他先後任用靳輔和于成龍治理黃河與大運河,得到很大的成績。在他六次南巡期間,考察民情習俗之外,更是親自監督河工。康熙中期以後,因戰亂而遭到嚴重破壞的手工業逐步得到恢復和發展。為安定社會秩序,他頒行十六條聖諭,要地方人士循循告誡鄉民。他又派心腹包衣(即家奴)如曹寅、李熙等人打探地方物價、人民收入與官紳不軌之事,並以密褶奏報。此即密摺制度的萌芽,到雍正時期趨於完善。康熙帝重視對漢族士大夫的優遇,他多次舉辦博學鴻儒科,創建南書房制度,並且向來華傳教士學習西方科學與文化。

清初蒙古分為四大部。其中準噶爾汗國與俄羅斯沙皇國友好,其可汗噶爾丹先滅領葉爾羌汗國與青海和碩特,又佔領喀爾喀蒙古,喀爾喀三部南下投靠清朝。康熙帝首先派薩布素於雅克薩戰役驅除入侵黑龍江的沙俄軍隊,與沙俄所簽訂的《尼布楚條約》以確立東北疆界並獲得沙俄的中立。接著於1690年至1697年間烏蘭布通之戰與三征噶爾丹使噶爾丹戰死,進行多倫會盟以將喀爾喀蒙古納入直接統治。青藏地區的和碩特汗國協助黃教達賴五世擊敗紅教統一全藏,之後分裂成青海與西藏和碩特。達賴六世時,藏區政事交由第巴(理事大臣)桑結嘉錯管理,他聯合準噶爾對抗西藏和碩特的拉藏汗,拉藏汗先下手殺桑結嘉錯並廢除達賴六世。1717年噶爾丹的侄子策妄阿拉布坦入侵西藏,攻殺拉藏汗,並且佔領拉薩。清軍多次被準軍擊敗,最後於1720年由胤禵率軍驅除成功,協助達賴七世入藏,以拉藏汗舊臣管理藏區。

康熙晚期,由於官員薪資過低以及法律過寬,導致官吏貪污,吏治敗壞,并发生南山案文字獄事件,到雍正與乾隆時期这种情况加重。康熙帝本來按照中國立嫡立長的傳統封胤礽為太子,由於太子本身的素質問題及其在朝中結黨而廢太子,使得諸皇子為皇位互相結黨傾軋。故太子一度復立,但康熙帝仍無法容忍其結黨而廢除。最終在1722年臨終時傳位於胤禛,即雍正帝。

雍正帝獲得隆科多的協助繼位,賴年羹堯平定青海亂事以穩固政局,然而後來因故賜死年羹堯、幽禁隆科多。雍正帝在位時期,針對康熙時期的弊端採取補救措施,以延續康雍乾盛世。他設置軍機處加強皇權,廢殺與他對立的王公並削弱親王勢力。注重皇子教育,採取秘密立儲制度以防止康熙晚年諸皇子爭位的局面再度發生。將丁銀併入地賦,減輕無地貧民的負擔。廢止賤民政策,令世代受到奴役且地位低賤的賤戶被解放。為解決地方貪腐問題使火耗歸公,耗羨費用改由中央政府計算;設置養廉銀以提高地方官員的薪水。

對外方面,雍正初年青海親王羅卜藏丹津意圖復興和碩特汗國而亂,隔年年羹堯與岳鍾琪等人平定。為此雍正帝佔領部分西康地區,又在西寧與拉薩分置辦事大臣與駐藏大臣以管理青藏地區。聽從鄂爾泰建議推行改土歸流,廢除具自治性質的土司,以地方官管理少數民族。將喀爾喀蒙古併入清朝;於1727年與俄羅斯帝國簽訂恰克圖條約,確立塞北疆界。1729年聽從張廷玉建議,以傅爾丹與岳鍾琪兵分二路於科布多對抗準噶爾汗噶爾丹策零,最後於和通泊之戰戰敗。1732年噶爾丹策零東征喀爾喀蒙古,兵至杭愛山,被喀爾喀親王策棱擊敗。1734年清準和談,以阿爾泰山為界,西北大致和平。

雍正帝勤於政事,自詡「以勤先天下」、「朝乾夕惕」。他在位期間的奏摺大多由他親自批改,軍機處的諭旨也由他再三修改。他所親信的內外臣僚如張廷玉、鄂爾泰、田文鏡與李衛等人也都以幹練、刻覆著稱。他所派遣的特務遍即天下以監控地方事務,密摺制度至此完善,然而屢興文字獄打壓異己。1735年雍正帝於工作時去世,其子弘曆繼位,即乾隆帝。

乾隆帝繼位之初,獲得張廷玉與鄂爾泰的協助,穩定政局。他以「寬猛相濟」理念施政,介於康熙帝的仁厚與雍正帝的嚴苛之間。人口不斷增加使乾隆末年突破三億大關,約佔當時世界人口的三分之一。江南與廣東等地的絲織業與棉織業都很發達,景德鎮的瓷器都達到歷史高峰。與此同時,銀號亦開始在山西出現。中國的國庫庫存亦從雍正十三年(1735年)的34,530,485銀兩上升至乾隆三十九年(1774年)的73,905,610銀兩。然而乾隆晚期多從寬厚,寵信貪官和珅,官員腐化使政治大壞;六次下江南以蠲賦恩賞、巡視河工、觀民察吏、加恩士紳、培植士族、閱兵祭陵,有學者認為供張過盛,擾民有餘的批評。美國哈佛大學教授歐立德指出乾隆六次南巡誠然耗資巨大,但相對於當時國庫收入而言尚在可承受的範圍之內。

乾隆帝鴻講學術,然而由於限制過多,所得人才不如康熙詞科。此時期有許多書籍出版,例如《續三通》、《皇朝三通》與《大清會典》等史書;著名小說《紅樓夢》、《聊齋誌異》和《儒林外史》等;1773年更下令考據補遺,編纂《四庫全書》,與《古今圖書集成》成為全世界最龐大的類書,這些都成為盛世的文化標誌。然而為維護統治却嚴厲控制思想,編書期間藉機對不符其思想的書籍進行禁毀與秘藏。此外大興文字獄使如戴名世等人被株連殺害或者流放。有學者認為這些都讓文人思想受到嚴厲阻礙,遲滯文化的發展,另有學者認為此舉扼殺中国人的思想活力。梁啟超則认为,清朝二百多年,對文化發展有相當程度的貢獻,是「中國之文藝復興時代」。清朝盛行的輯佚學亦救亡不少大量早已失傳的中國古籍。另外,史學家郭成康、林铁钧指出清代有些包括「反滿」內容的書籍多次在作者沒被追究的情況下合法出版,例如王夫之的《讀通鑑論》和顧炎武的詩文集。在清初年間的思想界、學術界,都相當活躍。康熙規定:“凡舊刻文卷,有國諱勿禁;其清、明、夷、虜等字,則在史館奉上諭,無避忌者”,表現出比歷代封建統治者都較為開明和寬容的態度,史學家喻大华指出不應該將清朝查禁「反清」言论与「文字狱」混为一谈,因為号召推翻现政权的言论不屬於文字问题,而是政治问题。

西方傳教士將中國文化介紹給歐洲人,引發18世紀“中國風”热潮。歐洲人追崇中國文化、思想與藝術,在1769年更有人寫道:「中國比歐洲本身的某些地區還要知名」。佩雷菲特筆下與乾隆帝不歡而散的英國特使馬戛爾尼認為清朝已經衰落,然而馬戛爾尼在其日記著作中寫道:「中國政府的行政機制和權力是如此的有組織和高效,有條件能夠迅即排除萬難,創造任何成就」,馬戛爾尼訪華團的成員之一愛尼斯·安德遜亦對當時期的清朝有相當的正面評價。

對外方面,乾隆十年(1745年)準噶爾汗噶爾丹策零去世,國內諸子爭位。乾隆十七年(1752年)冬,达瓦齐袭夺准噶尔汗国汗位,阿睦爾撒納在随后的内斗中被击败,不得不归附清廷。乾隆帝乘機於乾隆二十年(1755年)派其為引導,以定北将军班第率軍平定準噶爾,攻下伊犁。而後阿睦爾撒納想要成為新一代準噶爾之主,由於沒有獲得乾隆帝支持而叛變。乾隆帝派兆惠西征,阿睦爾撒納战败逃亡哈萨克汗国,后因为哈萨克汗归降于乾隆,又逃亡帝俄,乾隆以《尼布楚条约》规定之中俄引渡逃人的条款要求俄方引渡,而阿睦爾撒納本人已于乾隆二十二年(1757年)出天花病死在托波尔斯克附近的库杜斯克酒厂。随着阿睦爾撒納的死去,天山北路遂告平定,準噶爾亡,其族在乾隆的屠杀令下慘遭滅絕。然而在天山南路,脫離準噶爾統治的回部領袖大小和卓兄弟起兵反清,史稱大小和卓之亂。其領袖布拉尼敦(大和卓)與霍集占(小和卓)佔據喀什噶爾與葉爾羌,意圖自立。乾隆二十三年(1758年)乾隆帝再命兆惠西征,兆惠率輕軍渡沙漠圍攻葉爾羌(今新疆莎車),反被包圍於黑水營。隔年清將富德率軍解圍,兆惠與富德最終攻滅大小和卓,並讓帕米爾高原以西的中亞各國成為藩屬國。乾隆末年,尼泊爾的廓爾喀王國兩次入侵西藏。1793年清廷派福康安與海蘭察領兵入藏,擊退廓爾喀入侵,不丹與哲孟雄(今錫金)亦為藩屬國,加強駐藏大臣的權力。

西南方面,乾隆初年派張廣泗平定貴州苗民之亂,隨後清軍前往平定大渡河上游的大小金川(今四川金川縣與小金縣)動亂,史稱大小金川之役。1747年到1749年期間發生大金川之戰,清軍於此吃盡苦頭。1771年第二次金川之戰爆發,大小金川的諾木與僧桑格均叛,清將溫福戰死,阿桂歷經多次作戰,直到1776年方平定。期間緬甸貢榜王朝與清朝爆發清緬戰爭,清軍四次進攻皆失敗。1769年乾隆帝派傅恆、阿桂入緬未果,雙方最後停戰。1784年暹羅卻克里王朝派使朝貢,1788年緬甸為應付暹羅威脅,也派使朝貢。1789年安南發生西山朝統一後黎朝、鄭主與廣南國的事件。清軍入安南擊敗西山朝,護送黎帝黎愍帝復位,但途中遭西山軍的伏擊而敗,史稱清越戰爭。西山朝阮惠遣使向清廷謝罪,清廷封為安南国王。

乾隆期間清朝疆域達1300萬餘平方公里,東方的朝鮮與琉球國也早就成為藩屬國之一。但只有德川幕府統治的日本處於鎖國時期,與清朝來往甚少。乾隆帝以“十全武功”自譽,他平定準噶爾與回疆大小和卓之乱,使四川、貴州等地繼續改土歸流,然而其餘戰事皆小題大作使國庫嚴重損耗,讓清朝國力衰退,全国范围内开始爆发民变。乾隆時期的戶部存銀最高達8,000萬兩,常年保持在6,000-7,000萬兩左右,足以應付政府的各項日常開支、重大工程、戰爭,而雖然清朝的賦稅較為輕,且於康乾時期多次對外用兵、大興各項工程,但每年國家財政都會有餘,國庫儲備逐年上升。當時人口暴增與鄉村土地兼併嚴重,使得許多農民失去土地;加上貪官和珅等官員腐敗,於乾隆晚期到嘉慶時期陸續爆發民變。白蓮教於1770年代舉兵,後來又於1796年爆發川楚教亂,八年後被清軍鎮壓,領袖王三槐被處死。台灣天地會領袖林爽文於1787年發動林爽文事件,歷時一年多。在乾隆年間,平定大小金川之亂、消滅準噶爾汗国等各威脅,將新疆正式重新納入中國版圖,並且頒佈被視為西藏屬於中國領土的最有力的證據《欽定藏內善後章程》二十九条,加強中央政府對西藏的管治,最終奠定現代中國的版圖。

1795年,乾隆帝照其誓言禪位於子顒琰,即嘉慶帝。乾隆雖為太上皇,但依然「訓政」至1799年去世,嘉慶帝方得以親政。然而嘉慶帝未能解決弊端,清朝继續走向衰退。

嘉慶帝在當太子時痛恨貪官和珅,親政後將其賜死,並抄收其家產。然而嘉慶並沒有借此全面整頓政風,加上地方出現賣官以平衡開支的現象,使得貪污腐敗的風氣更加擴大,加重地方人民的負擔。另外還有河道與漕運淤塞的難題。針對乾隆時期過度開銷的弊端,嘉慶帝提倡節儉,縮減朝廷與宗室的開支,把貧窮的旗民送到關外開墾。然而,最後因為朝野強烈的反彈聲浪而妥協。此時八旗兵與綠營軍紀腐敗不可堪用,只能靠地方地主勢力的團練平定亂事,而後期更由此形成湘軍與淮軍等地方軍。當時民亂不斷,有白蓮教的川楚教亂、東南有海盜侵襲,華北又有天理教之亂。道光之後又有太平天國之亂、捻亂以及甘陝回變與雲南回變,再加上鴉片戰爭等外患,一度使清廷搖搖欲墜。

1820年,嘉慶帝崩,旻寧繼位,即道光帝。此時,朝廷暮氣沉沉,滿朝文武只知迎合貪污謊報。道光帝提倡儉樸,所穿龍袍是宮內舊料所製,滿朝文武故意在朝服補丁,以示簡樸。大臣奏章大多報喜不報憂。曹振鏞是當時第一重臣,奉行「多磕頭,少說話」哲學。繼起的穆彰阿,人稱「在位二十年,亦愛才,亦不大貪,惟性巧佞,以欺罔矇蔽為務」。鴉片戰爭時,前方將帥不斷撒謊,敗將奕山竟被欽命交部優敘。道光時期稍可稱善的政績是陶澍改革鹽法,成功的防止商販壟斷。

19世紀上半葉,西方各國為使通商正常化,多次派使者前往中國協商。然而清政府以天朝上国自居,不願與西方各國平起平坐,屢次不了了之。當時不列顛帝國對中國茶葉與絲綢的需求龐大,對華貿易成逆差狀態。為此,英國將成癮劑鴉片大量輸入中國以改善本身經濟。1838年鴉片猛增到四萬零二百箱,人民健康被削弱,清朝經濟发生通貨膨脹,國力也持續衰退。道光帝為解決此弊端,派林則徐到外贸口岸廣州宣佈禁菸,此即虎門銷煙。為此,1840年中英两国爆發鴉片戰爭,清军戰敗後和英國簽訂第一個不平等條約——《南京條約》,開啟中國近代史。當時道光帝與耆英不了解國際法,認為給予英人貿易之便以換取國家長存,所以割讓許多影響甚遠的權力。清朝後期被迫和各國簽訂不平等條約,除割地、開港、賠款之外,還讓外國派駐軍隊於首都,中國主權逐漸流失。

1850年,道光帝崩,子奕詝繼位,即咸豐帝。西方各國迫使清政府開港通商,加上地方官吏地主兼併土地,使得傳統農村經濟受到破壞。各地乘機紛紛起事,其中華北以捻亂為主,華中華南以洪秀全的太平天國與雲南杜文秀、馬如龍的雲南回變為主。洪秀全改造基督教教義,1851年於廣西金田起義,聯和天地會、三合會北伐。兩年後攻陷並定都江寧,並且發動兩次西征;不久又發動北伐,最遠達天津近郊。後來曾國藩、左宗棠與李鴻章為保護儒家文化,紛紛組織湘軍與淮軍抵抗太平天國。太平天國發生天京事變後國力衰退,部分勢力轉入捻軍。太平天國最後於1864年被湘軍、淮軍以及外國人組成的常勝軍、常捷軍圍攻之下而亡。此期間英國與法國因為和清廷修約不成,趁中國發生內亂之際,於1858年發動英法聯軍之役。清軍於八里橋之戰戰敗,聯軍攻陷北京,圓明園、清漪園等處被焚掠,簽訂《天津條約》及《北京條約》。同時帝俄以调停有功逼清廷簽訂《璦琿條約》,取走外東北領地。1864年帝俄強迫清廷訂立《勘分西北界約記》,割佔外西北。面對內外交迫的局面,清廷為使國力恢復而發起自強運動。

1861年,咸豐帝崩,其六歲之子載淳繼位,即同治帝。咸豐帝本任命肅順等八大臣贊襄政務,兩宮太后與恭親王奕訢發動辛酉政變,兩宮垂簾聽政,最後由兩宮之一的慈禧太后獲得實權。被稱為洋務派的奕訢與部分漢臣在消滅太平軍時認識到西方的船堅炮利,並且鑒於兩次鴉片戰爭的失敗,以「師夷長技以制夷」、中體西用為方針展開自強運動(又稱洋務運動)。當時總理各國事務衙門與隨後的北洋通商大臣負責對外關係與自強運動的策劃與推行,先後引入國外科學技術,建立現代銀行體系、現代郵政體系、鋪設鐵路、架設電報網。建立翻譯機構同文館、新式教育(新學),培訓技術人才並派遣留學生到欧美日等先进工业国家,培育出唐紹儀與詹天佑等人才。開設礦業、建立輪船招商局、江南製造總局與漢陽兵工廠等製造工廠與兵工廠,同時也建立新式陸軍與北洋艦隊等海軍。洋務運動使得中國社會出現較安定的局面,史稱“同治中興”。其間太平天國於1864年滅亡。1865年,僧格林沁的滿蒙騎兵(八旗兵)中捻軍埋伏後全殲,賴洋務派左宗棠與李鴻章分別滅西、東捻,捻亂到1868年為止。1862年至1878年間,左宗棠先後平定陝甘回變,平定新疆回亂,並收回伊犁。雲南回變也於1867年由馬如龍投降清朝岑毓英,以及1872年杜文秀自殺而止。西方各國的租借地也將西方思想帶入中國,推動中國革命與民主制度的發展。1875年,同治帝去世,慈禧太后立載湉為帝,即光緒帝。

對外方面,1884年,清朝和法國為越南(安南)主權爆發中法戰爭。清朝失去藩屬國越南,越南成為法國殖民地,台灣也宣布建省。1885年英國入侵緬甸,清朝駐英公使曾紀澤向英國抗議無效,隔年被迫簽訂《中英緬甸條約》,承認緬甸為英國所有。日本在明治維新後國力大增,1872年日本強迫清朝藩國琉球改屬日本,清朝拒不承認,中日交惡。1894年為朝鮮主權清朝和日本發生甲午戰爭,兩個推行西化運動的亞洲國家的战争最後以清军落敗而告终。戰後簽《馬關條約》,清朝割讓台灣和澎湖,失去藩屬國朝鮮和租界。洋務派李鴻章建立的北洋艦隊全面瓦解,也宣告自強運動最終失敗。

甲午戰爭後,維新派康有為與梁啟超於1895年公車上書光緒帝,要求深入改革政府架構、教育、經濟體制與軍事制度等多個層面,期望清廷從制度面革新。1898年光緒帝在康有為的幫助下實施維新運動(戊戌變法),然而由於做法和態度過於激進而激起舊有保守派和原本的中立群體的反抗,康有為的弟弟康廣仁評道:「伯兄規模太廣,志氣太銳,包攬太多,同志太孤,舉行太大。當地排者,忌者、擠者、謗者盈衢塞巷,而上又無權,安能有成?」,導致原本支持變革的慈禧太后以「聽信逆臣蠱惑,改變祖宗成法」為由軟禁光緒帝,處決譚嗣同、康有溥等多人。由於維持103天就結束,被稱為「百日維新」。

1896年清廷為連俄制日,簽訂《中俄密約》。後來密約泄露,外國鑒於清朝已無力自衛,紛紛劃分在中國的勢力範圍以維護為各自利益,而美利堅合眾國提出門戶開放政策以平衡列強在華勢力。中國長期受列強欺辱,使地方產生義和團之類仇洋排外的民族主義團體。慈禧太后藉此排外而發生義和團事變,義和團屠殺洋人、姦淫婦女、搶奪店舖、破壞各國使館、燒毀與西洋有關的東西。慈禧太后不理會各國抗議,更曾半正式向十一國宣戰,引發八國聯軍報復。北京被聯軍佔領,劫殺擄掠。慈禧太后率光緒皇帝西逃西安。1901年簽訂《辛丑和約》,清廷賠償重款,列強派兵駐守北京一帶、劃定租借地和勢力範圍,加深中国的半殖民化。1904年日俄兩國更因在東北的利益冲突爆發日俄戰爭。義和團事變時,李鴻章、張之洞、劉坤一、袁世凱等東南各行省之總督巡撫為保護華中華南,自行宣布中立,不服從朝廷對外一律宣戰的敕命(即東南自保);從此清廷權威低落,地方各省自主性提高。

清朝於太平天国战争、甲午战争、庚子国变後國勢大墜,各界人士及知識分子莫不提出各種方法拯救中國,主要分成立憲派與革命派兩種主要改革路線。立憲派主張效仿英德等國實行君主立憲制,而革命派則堅決主張推翻帝制,實行共和制。

1901年,立憲派康有為、梁啟超等推動立憲運動,梁啟超發表《立憲法議》,希望讓光緒帝成為立憲君主。而慈禧太后為挽清朝衰落危局,有意效仿歐日的改革而推行清末新政。新政除了推行君主立憲外,還有諸如建立清朝新軍、廢除科舉、整頓財政等一系列改革。而革命派對清廷的改革失望,他們鼓勵推翻清朝,建立中華共和。1894年,孫文等於夏威夷檀香山建立興中會;1904年,黃興等於長沙成立的華興會;1904年,蔡元培等於上海成立光復會;此外,還有其他革命團體。1905年,孫文在日本聯合興中會、華興會、光復會,成立中國同盟會,並提出「驅除韃虜、恢復中華、創立民國、平均地權」綱領。革命派聯合舊有反清勢力如三合會、洪門等,在華南地區發起十次起事,並將勢力滲入華中、華南的清朝新軍。

當時立憲派與革命派為改革方式發生爭執,起初立憲派佔上風,清廷也承諾實行立憲。1907年清廷籌設資政院,預備立憲,並籌備在各省開辦諮議局。1908年7月頒布《各省諮議局章程及議員選舉章程》,命令各省在一年之內成立諮議局。同年頒布《欽定憲法大綱》,以確立君主立憲制政體,成立代議會。在立憲派成員的請願下,清廷宣佈把預備立憲縮短三年,預定在1913年召開國會。同年光緒帝與慈禧太后皆去世,溥儀繼位,即宣統帝,其父載灃擔任監國攝政王。1911年5月清廷組成由慶親王奕劻領導的「責任內閣」,這是中國歷史上首次君主立憲。不過,該內閣中的很多成員為皇族身份,故被稱為「皇族內閣」,引發立憲派的不滿和失望,很多轉向於革命派合作。

1911年5月,四川等地爆發保路運動,清廷急派新軍入川鎮壓,湖北空虛。10月,湖广总督瑞澄斬殺彭楚藩、劉堯澄、楊洪聖等三個革命黨士兵,參加文學社與共進會等反清團體的士兵,人心惶惶,兩名革命分子金品臣、程定國夜間與排長陶啟聖齟齬,一怒之下,射殺排長,發起武昌起義,南方各省隨後紛紛宣佈獨立,是為辛亥革命。

清廷任命北洋新軍統帥袁世凱為內閣總理大臣,成立內閣並統領清朝的北洋軍。袁世凱一方面於陽夏戰爭壓迫革命軍,另一方面卻暗中與革命黨人談判,形成南北議和,1912年1月1日中華民國政府於江寧成立,改稱南京,孫文在南京就任臨時大總統,1月26日清室優待條件達成,孫文也承諾只要袁世凱贊成清帝退位,自己即讓位於袁世凱,由袁世凱出任民國大總統,袁世凱與革命黨人的意見達成一致。

在袁世凱授意下,段祺瑞等五十位北洋軍將領,發佈了《北洋五十將乞共和電》,要求宣統退位。段祺瑞不久又發《乞共和第二電》,以發動兵變要脅朝廷。2月12日,隆裕太后代表宣統帝溥仪,頒布退位詔書,將權力交給中華民國政府,清朝滅亡,標誌著中國两千多年來的君主制度正式結束。隨後孫文讓位予袁世凱,南北統一。3月6日,南京參議院正式決議同意袁世凱在北京就職,袁世凱就任大總統,定都北京。

因正統觀使然,清亡時不少漢臣如郑孝胥等依舊忠於大清,終身以滿清遺老自居,甚至有人捨身殉國。後來,1917年張勳組織辮子軍,於北京擁護宣統帝溥仪,復辟清朝(史稱張勳復辟),但只持續12天而終。1924年,冯玉祥驅逐溥仪,溥仪和他的随从只好由紫禁城移往天津租界居住。

清朝發源於東北地區,努爾哈赤在與明朝決裂前,已領有建州之地,自稱“收管我建州國之人,看守朝廷九百五十余里邊疆”。經努爾哈赤與皇太極時期的發展後領有今東北地區、外東北地區與內蒙古地區。1644年多爾袞偕同順治帝率軍入關,隨後指揮清軍占領全明朝領地,統一中原,領有內地十八省。

1661年,南明亡。然而當時尚有以吳三桂為首,鎮守華南的三藩;以及奉明朝為正朔,領有台灣台南、澎湖的明鄭。三藩之亂與施琅攻臺後,康熙帝完全掌控華南地區與台灣西部及澎湖地區。此時準噶爾汗國的可汗噶爾丹與俄羅斯沙皇國(沙俄)友好,噶爾丹南征青海和碩特,東征喀爾喀蒙古。而沙俄為在遠東尋找出海口,向東移民侵略黑龍江上游。康熙帝先是於雅克薩戰役擊敗俄軍,與其劃定邊疆;之後率軍三征噶爾丹,協助喀爾喀蒙古收復其領土。喀爾喀蒙古其后在多伦会盟后併入清朝,外蒙古地區正式歸清朝所有。1727年雍正帝與帝俄簽訂恰克圖條約,確立塞北疆界。

1717年準噶爾汗國新可汗策妄阿拉布坦入侵青藏地區,滅和碩特汗國,並且佔領拉薩。清軍多次被準軍擊敗,最後於1720年由胤禵率軍驅除成功,協助第七世達賴喇嘛入藏,以拉藏汗舊臣管理藏區,這是清朝經營青海、西藏地區之始。雍正時期,平定青海親王羅卜藏丹津之亂後,雍正帝又在西寧與拉薩分置辦事大臣與駐藏大臣以控管青藏地區。

新疆地區方面,1755年乾隆帝乘準噶爾汗國噶爾丹策零去世的機會,派將領率軍西征,軍勢直達準國首都伊犁。在平定阿睦爾撒納之亂與大小和卓之亂後徹底掌控準噶爾地區與回疆,並且獲得帕米爾高原以西諸國的朝貢。

康熙年的《尼布楚條約》和雍正年的《恰克图界约》後清朝与俄罗斯帝国确定了北部边界,乾隆時期灭亡准噶尔後清朝的疆域最為穩定,因此一般將乾隆時期疆域定為淸朝的最大範圍:東北與俄羅斯帝國(帝俄)分界額爾古納河、格爾必齊河與外興安嶺,這條疆線直到鄂霍次克海與庫頁島。正北與帝俄分界薩彥嶺、沙畢納依嶺、恰克圖與額爾古納河。西北與哈薩克汗國等西北藩屬國分界薩彥嶺、齋桑泊、阿拉湖、伊塞克湖、巴爾喀什湖至帕米爾高原。西南與印度的蒙兀兒帝國、喜馬拉雅山諸國家分界喜馬拉雅山至野人山,正南大致上與現今中華人民共和國與東南亞國家的分界相同,但清朝尚獲得緬甸北部的南坎、江心坡等地。東與日本、琉球國分界日本海與東海,與朝鮮王朝沿圖們江、鴨綠江分界,清朝還領有台灣、澎湖、海南及南海的南海諸島(時稱千里石塘、萬里長沙、曾母暗沙)。

進入19世紀,由於清朝的衰落,列強於鴉片戰爭後以不平等條約掠奪許多領土與藩屬國。俄羅斯帝國藉由1858年璦琿條約與1860年北京條約獲得外東北,包括庫頁島等地。1900年趁八國聯軍的機會又強佔黑龍江以北的江東六十四屯。1864年藉由中俄勘分西北界約記與1881年的伊犁條約獲得外西北,並且陸續佔領中亞諸藩屬國。19世紀末大博弈時期英俄兩國簽訂英俄協定,私自劃分帕米爾地區。不列顛帝國藉由1842年的南京條約、1860年的北京條約與1898年的展拓香港界址專條獲得現今香港地區,並且侵占藩屬國緬甸與喜馬拉雅山諸國家。法蘭西第三共和國於中法戰爭占領藩屬國安南、南掌。葡萄牙帝國於中葡和好通商條約永居管理澳門。明末时萨摩藩已对藩屬國琉球國实施以军事占领为后盾的遥控统治,而日本帝國明治時代更於1872年將其正式吞并,並於1895年的馬關條約獲得台灣與澎湖列島,並強迫清朝放棄藩屬國朝鮮,而原先亦被割讓的遼東半島則因三國干涉而重回清朝之手,朝鮮後來被日本吞併。甲午戰爭後,列強認為清朝無自衛能力,為自身利益劃分在中國勢力範圍,使得重要港口如旅順、大連被帝俄與後來的日本領有、威海衛被英國領有、胶澳被德國領有、廣州灣被法國領有等。部分清朝末期建立的租界到1945年二战结束后中國才得以收回主權。

清朝版圖遼闊,民族眾多,在行政區劃上也「因時順地、變通斟酌」。在漢族地區沿用明朝舊制,實行“省—府—縣”三級制。在東北地區,為滿洲八旗制、漢人“省—府—縣”三級制與漁獵部落的“姓長制”並行。在藩部地區則因俗而治,並根據中央集權統治的需要加以改革:蒙古實行“旗盟制”、“札薩克制”;西藏實行“宗谿制”,新疆回部實行伯克制。全国分为内地十八省、五个驻防将军辖区、两个办事大臣辖区共二十五个一级行政区域和内蒙古等旗、盟。

清末,在列強蠶食鯨吞的形勢下,邊疆各地依靠舊有的行政體制已無法維持有效的統治。光緒年間,新疆、奉天、吉林、黑龍江相繼建省,實行與內地相同的行政體制。蒙古、西藏也有建省之議,但在清朝滅亡之前未能實行。光緒三十四年(1908年),清朝分為二十二省,以及西藏、外蒙古、內蒙古、青海等邊疆地方。

清代山海關以內、長城以南的漢族地區被稱為「內地」、「關內」或「漢地」,又继承明代「两直十三省」的称谓合称「直省」。內地的行政區劃承襲明代「省—府(州)—縣」的層級體制。一級政區為省,又俗称“行省”,本布政使司,但随着承宣布政使逐渐沦为督抚的属官,“布政使司”的名称逐渐被“省”取代,至嘉庆朝《一统志》编纂时“省”已成为一级政区的正式称呼。二級政區為府、直隸州。府管轄的州(散州、屬州)不再領縣,形成單式的三級制。清代初年,原為臨時差官的巡撫取代布政使,成為一省的長官。在一些民族雜居之處及戰略要地,設置新型政區「廳」,分為省直轄的直隸廳和府轄的散廳。少數直隸廳下轄縣。

明代承宣布政使司、提刑按察使司派出的差官「道員」,在清代也保留下來。道員的統轄區域是「道」,介於省與府之間,有分巡道、分守道、糧儲道、鹽法道、兵備道等名目。清初的道並不是行政區,道員亦無品級。乾隆以後,定道員秩品為正四品,分巡道、分守道的職權也漸趨一致。有的道下直接領縣。有人認為清末的道實際上已成為省、府之間的一級政區,之後北洋政府更有廢省置道之計畫,後因被國民政府取代而未實施。清朝行政區劃層級為:

行省:
在行省設置方面,清代基本沿襲明代所置的兩京與十三布政使司,设山東、山西、河南、陝西、浙江、福建、江西、廣東、廣西、湖廣、四川、雲南、貴州。順治元年(1644年)定鼎北京,以盛京為留都。二年(1645年)改北直隸為直隸省,改南直隸為江南省。康熙三年(1664年),分湖廣為湖北、湖南二省。康熙六年(1667年),江南省正式分為江蘇、安徽二省。康熙七年,設立甘肅省,自此形成所謂「內地十八省」的格局。

光緒十一年(1885年),分福建省臺灣府置臺灣省。兩年後臺灣正式建省,稱「福建臺灣省」。光緒二十一年(1895年),因甲午戰爭戰敗,臺灣省被割讓予日本。光緒三十年十二月(1905年1月),分江蘇江寧、淮安、揚州、徐州四府及通州、海州二直隸州置江淮省,旋即裁撤。此後至清末,內地仍為十八省,與東三省、新疆省合為二十二省。

各省以总督巡抚为长貳,掌管一省军政大权及吏僚考察,号曰“封疆大吏”。乾隆直省辖区确定后,计有辖省总督八员,除直隶、四川两督辖一省,两江总督辖三省外,余均辖两省,而山西、山东、河南三省不设总督;辖省巡抚十五员,直隶、四川、甘肃三省以总督兼巡抚事。清季新疆、东北设省,又新设巡抚每省各一员,奉天、吉林、黑龙江三省巡抚由新设东三省总督统辖,新疆巡抚由陕甘总督统辖,同时内地八督全部兼领驻省巡抚,至是计有辖省总督九员,辖省巡抚十四员。

督抚以下,以布政使(俗称“藩台”)与按察使(俗称“臬台”)各置官司(俗称“藩司”“臬司”),分管一省行政与司法,雍正后又有提督学政一员开衙建署,负责管理教育事务,以上三员均受督抚节制。原则上每省三使均各设一员,唯江苏省民事繁重,分设江宁苏州两藩司,分管省事。

清末新政,针对省级政区实施现代化改革,其中江苏等总督驻省不复设巡抚,而以总督兼巡抚事;撤消咸同以来各省新设的新式财务机关,统归藩司属下的度支公所;按察使改提法使,遵循司法审判与司法行政分离的原则,专管司法行政与监督,审判等权分归各级审判厅和检察厅;学政改提学使,强化其教育管理职能以适应新式教育;新设交涉使,专门负责与外国通商交往事宜;每省于藩司外另设巡警道及劝业道,分管一省民政警务与农工商业事务。唯东三省因系新政时初建,不徇故例,故无藩司及巡警道,而设度支使与民政使,东三省当时发展程度低,实业事务轻,故亦无劝业道缺。

各省军队虽为督抚节制,但八旗驻防军不在其列。驻防八旗由各省驻防将军统领,直接向皇帝负责。

清代的府、州、廳、縣,按照「衝、繁、疲、難」的考語分為不同等次。考語字數越多,地位就越重要。一般以四字俱全者為「最要缺」,三字者(衝繁難、衝疲難、繁疲難)為「要缺」,二字者(衝繁、繁難、繁疲、疲難、衝難、衝疲)為「中缺」、一字或無字者為「簡缺」。

土司:雲南、貴州、廣西、四川、湖南、湖北、甘肅等省設有土司,分為宣慰司、宣撫司、招討司、安撫司和長官司(長官為武職),與土府、土州、土縣(長官為文職)。土司的長官以當地各族頭人充任,可以世襲,由朝廷或地方官府頒給印信,歸所在地方之督撫、駐紮大臣管轄。宣慰等司的長官隸屬於兵部、土知府、土知州等官隸屬於吏部。雍正年間,雲南、貴州、廣西等省的土司開始改行流官制,史稱“改土歸流”。光緒、宣統之際,趙爾豐出任川滇邊務大臣,四川西部的藏人土司、西藏東部的宗也開始改土歸流。

東三省:中國東北為清朝「龍興之地」。順治年間入關後,以駐防八旗留守盛京瀋陽。康熙至乾隆年間,逐漸形成三個相當於行省的將軍轄區:盛京、吉林和黑龍江。將軍之下設專城副都統分駐各城,並管理各城的臨近地區。副都統下有總管統領各旗。在漢民聚居之處,置府、州、縣、廳,如同內地。居於黑龍江、嫩江中上游的巴爾虎、達斡爾、索倫(鄂溫克)、鄂倫春、錫伯等族,編入八旗,由布特哈總管、呼倫貝爾總管管轄。黑龍江、烏蘇里江下游及庫頁島的赫哲、費雅喀、庫頁、奇楞等漁獵部落則分設姓長、鄉長,由三姓副都統管轄。

光緒末年的甲午戰爭、八國聯軍之役與日俄戰爭嚴重動搖清朝在東北地區的統治,迫使其廢除滿洲的旗民分治制度,設立行省。光緒三十三年(1907年),廢除盛京、吉林、黑龍江三地將軍衙門,改設奉天省、吉林省、黑龍江省;隨後裁撤各城副都統、總管,改為府、廳、州、縣。宣統三年(1911年),奉天省領八府、八廳、六州、三十三縣;吉林省領十一府、一州、五廳、十八縣;黑龍江省領七府、六廳、一州、七縣。

藩部:清代蒙古、西藏、青海、新疆与黑龙江布特哈(达呼尔、索伦、鄂伦春等族)被称为藩部,由理藩院管理。

明清之際,蒙古分為眾多部落(蒙古語稱為“艾馬克”),部落首領為“部長”(鄂拓克)或“汗”。清太宗時,依照滿洲八旗的組織形式,將蒙古各部落編為旗,是為蒙古的基本行政單位,其長官為札薩克或總管。旗下設“佐”(蘇木),相當於鄉。自此蒙古各部落被納入統一的行政體系之中。在地域上,蒙古地區大致分為察哈爾、內蒙古、西套蒙古、外蒙古(包括土謝圖汗部、賽音諾顏部、車臣汗部、札薩克圖汗部)、科布多與唐努烏梁海。

清代蒙古又分為內屬蒙古與外藩蒙古。內屬蒙古包括八旗察哈爾、歸化城土默特、唐努烏梁海、阿爾泰烏梁海等部,各旗由朝廷派遣官員(一般為總管)治理,與內地的州、縣無異。外藩蒙古各旗則由當地的世襲札薩克管理,處於半自治狀態。在外藩蒙古,以若干旗合為一盟,設正、副盟長,掌管會盟事宜,對各旗札薩克進行監管。清代的盟是監察機構,並不能視為一級政區。

外藩蒙古又按其歸附清朝的先後分為內札薩克蒙古與外札薩克蒙古。內札薩克蒙古又被稱為內蒙古,於天命至康熙初年陸續歸附清朝。乾隆以後定為二十四部,共四十九旗,設六盟。內札薩克各旗不但政治地位很高,還保留一定的兵權。康熙中期以後歸附清朝的各部落稱為外札薩克蒙古,包括漠北的喀爾喀四部、西套蒙古二旗、青海蒙古各部、科布多各札薩克旗、新疆舊土爾扈特部及中路和碩特部。外札薩克各旗無兵權,隸屬於當地的將軍、都統、駐紮大臣(西套蒙古二旗除外)。其中喀爾喀四部後來演變為外蒙古。

新疆:清代新疆分為天山北路的準部和天山南路的回部,統屬於伊犁將軍。其中的蒙古遊牧地區實行盟旗制。維吾爾、布魯特、塔吉克等族地區則實行伯克制。蒙古舊土爾扈特部與中路和碩特部設立旗、盟:舊土爾扈特部為南北東西四路烏訥恩素珠克圖盟,和碩特部為巴圖塞特奇勒圖盟。準部地方設烏魯木齊都統,統轄烏魯木齊(迪化州)、庫爾喀喇烏蘇、吐魯番、哈密、古城、巴里坤(鎮西府)等城。其中迪化州、鎮西府由新疆與甘肅省雙重管轄。塔爾巴哈臺由塔爾巴哈臺參贊大臣管轄。伊犁及其以西地方由伊犁將軍、領隊大臣管理。回部設總理回疆事務參贊大臣(一般為喀什噶爾參贊大臣),統轄喀什噶爾、葉爾羌、和闐、阿克蘇、烏什、庫車、喀喇沙爾等城。光緒十年(1884年),新疆建省,實行與內地相同的府、廳、州、縣體制。

青海:清代的青海不包括今西寧、海東、黃南以及部分青海省邊緣地區。統轄青海地方的官員為西寧辦事大臣,常駐西寧(當時屬甘肅省)。青海大致以黃河為界,分為青海蒙古和玉樹等四十族土司。黃河以北主要為蒙古人,有和碩特、輝特、綽羅斯(準噶爾)、土爾扈特、喀爾喀五大部落。雍正三年(1725年),編青海蒙古為二十七旗,後增至二十九旗,由西寧辦事大臣主持會盟。另有察漢諾門罕牧地,實際上單獨為一喇嘛旗。道光三年(1823年),分黃河以北二十四旗為左、右翼二盟,每盟設正、副盟長各一人。黃河以南主要為藏人,設有四十個土司,其中以玉樹土司最大,故稱玉樹等四十族土司。土司以下有土千戶、土百戶。嘉慶、道光年間,藏人不斷越過黃河向北遷徙,形成環青海湖一帶的環海八族。

西藏:西藏在清代又稱“唐古忒”、“圖伯特”,分為衛、喀木(康)、藏、阿里四部,以及霍爾三十九族地區。西藏地方的行政長官為駐藏大臣,駐喇薩,會同達賴喇嘛、班禪額爾德尼辦理藏內政務。其政令由噶廈(西藏官府)執行。西藏的基層政區是宗,大致相當於內地的縣,但規模很小。一些貴族、寺廟的莊園領地稱為谿卡,地位比宗低或者平級。宗的長官為“宗本”,谿的長官為“谿堆”,一般由噶廈委派,也有的由特定寺廟委任。後藏札什倫布附近的幾個宗,由班禪直接管理。今那曲地區、昌都地區北部的各部落統稱霍爾三十九族,簡稱三十九族,為蒙古人後裔,由駐藏大臣的屬員夷情章京管轄。駐紮於達木(今當雄)的達木蒙古八旗,每旗設一佐領,不設總管,直屬於駐藏大臣。

清朝的政治體制基本上沿襲明朝制度,但略有不同。官員等級分「九品十八級」,每等有「正」、「從」之別。不在十八級別以內的叫做未入流,在級別上附於從九品。清朝制定內國史院、內秘書院與內宏文院等內三院為內閣,作為中央最高決策機關。設大學士滿、漢各二人,協辦大學士滿、漢各一人,學士滿六人、漢四人,下轄中央執行機關六部。內閣的實際權力比明朝小,實際掌握權力的機關會隨時代不同而改變。後金時期,議政王大臣會議是皇帝與王公贵族讨论國事之處。1631年皇太极為了中央集权,仿明朝制度设立六部與內閣以分議政王大臣會議的權力。入關之後,1677年康熙帝設立南書房,削弱議政王大臣會議權力,同時將外朝內閣的某些職能移歸內廷,實施高度集權。雍正帝為了西征準噶爾準備設置軍需處,雍正十年改稱軍機處。軍機處機構精簡,行政效率高,能迅速處理軍國大事,進一步加強君主專制主義中央集權。鴉片戰爭之後,為推行自強運動,先後於1861年與1870年成立總理各國事務衙門與北洋通商大臣,負責對外關係與自強運動的策劃與推行,成為自強運動期間最高行政機關。八國聯軍之後,1911年5月18日清廷宣佈廢除軍機處,仿西方國家與日本實行內閣制,內閣總理大臣和諸大臣組成的內閣成為最高行政機關。

清初康熙帝一方面則通過各種手段限制滿洲貴族的權力,如剝奪各旗王公干預旗務的權力,破除“軍功勳舊諸王”統兵征伐的傳統,削弱議政王大臣會議的政治影響等;另一方面提出要建立由皇帝個人獨裁的專制政體,“天下大權當統於一”,“天下大小事務,皆朕一身親理,無可旁貸。若將要務分任於人,則斷不可行”,亦表示:「天下大事,皆朕一人独任」,康熙要掌管「用人之權」,以阻止朋黨的形成,免得鰲拜掌權時期「結黨專權」和「罔上行私」的情況再度發生,也為了防止不同派別黨派之間互相鬥爭。康熙帝確立的君主專制制度主要包括三方面:「用人之權」、「獎懲之權」由皇帝親自控制,不許臣僚干預;通過特務統治、密奏制度,對臣僚實行嚴密的監督和防範;反對朋黨,嚴防臣僚結黨對抗皇權。

在明代,文人結社超出了文學和學術的領域而成為政治上的一支重要力量。清朝統治認為前朝文人團體的龐大和干政是明亡的重要原因之一。有見及此,清初統治者吸取前朝教訓,於順治九年(1652年)下令“生員不許糾黨多人,立盟結社”和“所作文字不許妄行刊刻,違者聽提調官治罪”。康有為對此作出評論,指出中國數千年來的黨派皆為君主所深惡,又以漢代黨錮之禍,唐代清流,宋代元祐黨籍碑和明朝東林黨為例。近代日本學者宗方小太郎評論:「(清朝)建國之初便多方預防弊政,防止禍亂於未萌狀態,其中如以法令嚴禁組織會黨,故在三百年之治世中黨禍頗少者即係此故。」

康熙帝確立的君主專制原則,被雍正、乾隆二帝繼承和發展,雍正帝的專制體現在他私派特務人員監視全國各地地方長官一切活動,許多地方官的私生活,連家裡的瑣事都瞞不過他。

學者錢穆從傳統“夷夏之辨”與近代民族主義相結合的角度強調清朝的“斷裂性”,在界定清朝的性質時以漢人文明的尺度衡量其價值的優劣,認為是滿人只有接受漢人的先進文化才能步入文明的境界,才具有延續前代王朝正統的資格,是近代史學中“漢化論”的表現。錢穆的論證基於傳統漢化的歷史觀,對滿人的統治評價負面。他引據革命家鄒容的《革命軍》的内容認為中国由漢唐等朝代的“士人政权”在清代变为“部族政权”,認為蒙古人和满洲人变为中国内的特权阶级或特殊分子。亦認為清朝政权始终是维护和偏袒满洲人,须满洲人在后拥护,才能控制牢固,以及清朝的政治,制度的意义很少,法术的意义多,批評清朝政府发布最高命令的手续,比如他認為“寄信上谕”是清朝特有的,不按照中国向来的程序,而是直接由皇帝军机处寄给受命令的人,旁人都不知道。他亦認為,清朝在政治上还限制发言权、结社和出版自由。在清朝,除了六部尚书和侍郎可以向皇帝讲话,其他的不论什么人都不许向皇帝讲话,而一直得到中央重视的翰林院等,向来他们可以向政府讲话的,但是到了清朝也不准专折言事。在地方上,只有总督、巡抚、藩台、臬台可以向政府讲话,其他的府县均不可,又認為在明代的「布衣」也可以直接向皇帝讲话。他又批評在地方上,清朝也不允许民间有公开发言权。在顺治五年立在府学、县学明伦堂里的卧碑就足可以证明。在当时府学、县学都有明伦堂,清朝在每一个明伦堂里都设置一个石碑,而这个石碑不是竖立的,而是横躺在那里,所以叫「卧碑」。在卧碑上有三条禁令:第一,生员不得言事;第二,不得立盟结社;第三,不得刊刻文字。然而史學家徐復觀指出錢穆的「士人政权」說並不正確,因為政府的性質必須就權力根源之地是由誰來運用而言。他亦指出通過《二十四史》一直到現代,都證明凡是站在平民的立場進入到仕途的人地位愈高,與皇帝愈接近,命運性的困擾、艱難必定來自專制的機構與專制的觀念。他批評錢穆對專制下的必然產物例如「外戚宦官」和漢代統治者的暴行視而不見,以及把中國「歷史中成千上萬的殘酷地帝王專制的實例置之不顧」,且根據《報任安書》,凡是皇帝親自交下與皇帝自己有關的案子,承辦的官吏決不敢問是非。

新清史學派認為,滿人採取的政治制度在明代的基礎下有所創新。比如軍機處就從帶有臨時性的純粹軍事咨詢組織轉變成了一種常規的政治治理機構,由此提高了統治效率。密折制度的建立完全改變了君臣之間相互溝通的傳統方式,使得君主控制臣下的能力大大增強。八旗駐防各地使漢人人口佔絕大多數的城市染上了頗為濃厚的異族色彩。內務府的設置與運行,嚴格了宮廷內部的禮儀規範,與明代的內廷制度有了本質的區別,宦官外戚干政的現象也由此完全絕跡。

西方傳教士如南懷仁等人記載康熙經常親身到各地巡視,以便了解百姓的生活情況和官吏們的施政狀況,亦會允許「最卑賤的工匠和農夫」接近自己,諭令衛兵們不許阻止百姓靠近,康熙會向百姓提出各種問題,包括詢問百姓對當地的官吏的滿意度,以便對官員作出獎勵或處分。另外,在清朝敢言且未被追究的學者有反對專制思想的袁枚、著書批評君權的唐甄、一道反朝廷的黃宗羲以及顧炎武等學者。

南書房於康熙十六年(1677年)設立,起初是康熙帝為了與翰林院詞臣們研討學問,吟詩作畫而設。因在乾清宮西南角特辟房舍故名南書房。由於南書房「非崇班貴檁、上所親信者不得入」,所以它完全是由皇帝嚴密控制的一個核心機要機構,隨時承旨出詔行令,這使南書房「權勢日崇」。南書房地位的提高,是康熙帝削弱議政王大臣會議權力,同時將外朝內閣的某些職能移歸內廷,實施高度集權的重要步驟。

軍機處原稱軍需處,歷來被視為清朝的最高決策部門。雍正八年(1730年),清軍在西北與準噶爾激戰,為及時處理軍報雍正皇帝始立,雍正十年改稱軍機處。軍機大臣以下設章京等官,從六部員司和內閣中書里選用。章京的任務是繕寫諭旨、記載檔案、查核奏議,作軍機大臣的輔助人員。章京也是滿、漢人員各兩班,每班八人,各設一領班。章京參與機要,草擬聖旨,俗稱「小軍機」。

乾隆皇帝即位後服孝,安排數位「總理事務王大臣」進入軍機處,故改名總理事務處。乾隆二年(1737年)乾隆服滿親政,總理事務王大臣等自請罷職,恢復軍機處名稱,自此遂成定製,軍機處成為直接對皇帝負責的核心權力機構,滿洲議政王大臣會議的地位更被削弱至幾乎可忽略不計,政治權力全部掌握在皇帝手中,成為清代中央集權制度的頂峰。

《清史稿》記載,乾隆時軍機處雖然只有兩名漢人,但漢人的地位都很高,張廷玉是太保、大學士、三等伯,徐本是太子太保、大學士,高於除了鄂爾泰之外的所有同僚。至於鄂爾泰的地位之所以穩居軍機大臣之首,則與他在“改土歸流”、“混一華夷”過程中曾立下的功業恰成正比。到了清季,軍機處仍不改諸族合作之傳統,吳郁生、榮慶和世續等軍機大臣,在國勢陵夷的光緒與宣統時期,依然在默契地合作。

宣統三年(1911年)四月初十清廷宣佈成立「責任內閣」,軍機處廢止。

宣統三年四月初十(1911年5月18日),清政府宣佈廢除軍機處,實行內閣制,任命內閣總理大臣和諸大臣組成內閣。由慶親王奕劻組成中國歷史上第一個現代意義上的责任內閣。由於閣員中過半數為皇族,時人譏之為「皇族內閣」。當時的內閣學士李家駒指出:当时中国唯一具有宪法性质的《钦定宪法大纲》并没有规定皇族不能组织内阁;日本宪法也没有类似规定(李家驹曾充驻日公使和出使日本考察宪政大臣);奕励内阁只是暂行阁制,具有过渡性质。

中國歷史學博士李细珠指出,与其说奕勘内阁是因皇族亲贵太多,不如说是因清朝皇族亲贵为满族,满汉矛盾才是问题的焦点。該內閣在辛亥革命後倒台,由袁世凱組成的新內閣所取代。

与汉地地方行政制度一样,清朝中央执行机关基本沿袭明朝体制,只有少量机构调整,大致上可以总结为七部院五寺察院两府。

七部院包括六部(吏部、戶部、禮部、兵部、刑部、工部)与理藩院,為清朝最高執行機關,各部長官稱尚書,副長官稱侍郎,以前尚書均由滿人擔任,順治元年(1644年)規定尚書及侍郎滿、漢各一,只有新设的理藩院因为与汉地事务无关而多涉及旗务,始终不设汉缺。

清朝以前的政治,因为政治的公开性和六部尚书是全国行政首长的關係,由外部送到內部的公事,都是先送到六部;而皇帝拿出来的公事,六部也一定要得先看,例如有關教育的公事一定要经过礼部,而不能由皇帝私下决定,到了清朝卻非如此。清朝的六部虽然沿袭明朝,但是清朝的六部的权力不如前朝,六部尚书更不能直接对部下发命令,而六部尚书也不是行政首长。六部的权限权力集中到皇帝手里,同时還有满漢之分,有一个汉人尚书,就必须有一个满洲尚书,并且始终以满尚书为主。

五寺包括大理寺、太常寺、光祿寺、太僕寺、鴻臚寺。大理寺與刑部和都察院合為三法司,其職權與今日之最高法院相似。大理寺的首長稱為大理寺卿,也是九卿之一。其餘四個寺的卿職權較低。太常寺負責祭祀;太僕寺管理馬匹;光祿寺負責壽宴;鴻臚寺負責接待外賓。

都察院是最高监察机关,架构基本沿袭明制,以左院察京畿,右院刺外藩(故直省督抚均领右都御史或右副都御史衔),所不同者,随着君主专制的高度加强,都察院的谏諍职能遭到空前削弱。因为同样的原因,明朝具有批驳权和言官职能的六科也只余下分察六部的监察职能,故于雍正年间被并入都察左院。为加强监督管理,凡天下文武官员,都要定期进行考察。规定三载考绩,以定升降奖惩。京官叫“京察”,外官叫“大计”。对武官的考察,每五年一次,称为“军政”,由兵部主持。但是,不论“京察”、“大计”还是“军政”,在实行中都是瞻徇情面,弊端丛生。后来更成为故事具文,走走过场而已。

內廷事務方面,鑑於明朝太監亂國,清朝皇帝獨創內務府以管理宮禁事務。其成員由內務府三旗(正黃旗、鑲黃旗、正白旗)的15個包衣佐領、18個旗鼓佐領、兩個朝鮮佐領、一個回子佐領和30個內管領的包衣及太監組成,其機構組織兼容清初內務府和十三衙門兩種制度的內容和特點,並最終形成以七司三院為主幹兼轄其他40餘衙門的龐大的宮廷服務機構。

宗室管理方面,清朝仍设宗人府管理宗室觉罗事务,但与明朝宗人府人浮于事只用于优待亲王的状况不同,由于八旗制度的存在,数量庞大的宗室觉罗成为清廷的核心军事力量,管理他们的宗人府也成为重要的实权部门。

清末新政中,对此前的制度进行大量的改革,此前的七部院被改革为十一部,长官(国务大臣)均为责任内阁阁僚;大理寺改组为大理院,根据司法独立原则不再是刑部的复审机关,而是全国最高审判机关,与最高监察机关都察院同为全国最高司法机关;下四寺进行省革而归入新官司,内务宗人两府尽管保有旧时职权,但随着军制改革,权力也大为下降。

清朝武装力量主要有八旗軍、綠營、地方義勇與團練、湘軍、淮軍與清末新軍。

八旗制度是清朝特有的一種組織形式和軍事制度,是清朝軍隊之核心。原先採取军政合一、兵民合一的方式。入關後專門以兵为业,世代为兵。包括旗下士兵和戶籍被編制在八旗軍隊中的家庭成員,由各地八旗駐防將軍或都統管轄。1601年,努尔哈赤将建州女真分为四旗。1615年时扩建为八旗,八旗制度至此成形。皇太极在征服漠南蒙古察哈尔部,以及收降明朝降将後,又建立起蒙古八旗与汉军八旗。儘管八旗有滿洲、蒙古、漢軍之分,但他們都是基於同一套制度之下,因此差異不大。旗人擁有一定的社會地位,絕大多數情況下終身不變,所屬旗籍亦基本世代固定。旗人因戰功而獲得的職位可以世代承襲,例如,每旗下屬的眾佐領通常都是世襲職位。旗人居住地大多是固定的。在各地的駐防軍(分佈在如杭州、成都等大城市)更設立「八旗駐防城」(俗稱「滿城」)供旗人居住,與平民所居住之地相隔離。旗人不得務農或經營工商業,每月錢糧由朝廷供給,號稱「旱澇保收」的「鐵桿莊稼」。旗人不受當地總督、巡撫管轄,犯罪時由特定機關審理。

绿营是順治帝入关后招降明軍、招募漢人組織的軍隊,以協助少量的八旗兵鎮守廣大的疆土。當時由八旗軍守備京師、華北地區與各地要衝,綠營守備華中與華南地區。華南更交由三藩鎮守,以壓制當地反清勢力。绿营以绿旗为标志,以营为单位,兵制繼承明朝,編有標、協、營及汛。綠營由漢人統帥,最上位的提督統領一省綠營,受文官總督、巡撫所節制,各省兵力大小不一,由萬餘到六七萬不等;提督之下為總兵,管轄一镇兵力,约几千人至一万五千人。直属兵力镇标由参将统领,约千人至兩三千人不等。再下面的為副將,管理一協兵力,約數千人左右。副將以下就是參將、遊擊、都司與守備,統轄一營兵力,兵員數量各有不同。最下面的為千總與把總,負責統領一汛,也就是一個駐地。士兵為世兵制,父死則子繼。將兵由兵部直接統轄,將領無法直接統兵,有效地防止軍人擁兵自重。隨著八旗軍的腐化,綠營的重要性就日益加強,例如三藩之乱时即以绿营为主力。乾隆嘉庆两朝,绿营总兵六十余万,成为军事主力。然而由於太平已久,绿营本身隨著种种弊端而逐渐腐化。乾隆帝阅兵時,所见已是“射箭,箭虚发;驰马,人堕地”。1796年川楚教乱時,綠營已无力對付擅長游擊戰的白蓮教徒,部分将领甚至屠杀平民以换战功。到了鴉片戰爭和太平天國之亂時,綠營上陣一觸即潰,作戰主力也改交由湘軍、淮軍等地方團練負責。同治年間多次裁減綠營,綠營的重要性逐漸減弱,清末新軍成立後綠營同名存實亡。至民國初年,綠營被改編為警察性質的地方治安衛戍部隊,成為民國時期警察的濫觴。

義勇與團練於川楚教乱後逐渐成為清朝武装力量之一,由於清軍不善游击战,所以鼓勵地方建立義勇與團練協助清軍鎮壓,1799年清廷正式同意组建團練。太平天國之亂與捻亂時,由於清廷的正规军腐敗無能,且不善游击战,地方官员曾国藩整合江忠源、胡林翼與罗泽南義勇,成立湘軍。湘軍作戰能力強,屢次擊敗太平軍。監視太平天國天京的江南大营被攻破后,湘軍成為清廷唯一抵禦太平軍的力量。1860年曾国藩的门生李鸿章于安徽一带建立淮军。平定捻亂时,僧格林沁率領的八旗軍中捻軍埋伏全滅,後來有賴淮军才平定之。當時如豫军、东军、滇军、川军等義勇也陸續建立起来。這些地方軍成為清朝晚期平定內亂、抵禦外侮的主要力量。然而不管是湘軍或淮軍皆以「兵隨將轉,兵為將有」為方針,與後來的北洋軍形成聽命於個人的軍閥勢力,這個作風深刻影響著民國軍事。

當時湘軍與淮軍採用西方新式槍炮,火力強大。而由外国人協助建立的常胜军、常捷軍更是讓曾李等将领印象深刻,使他们意识到西方军事技术的重要性。例如李鸿章目睹常胜军用4个小时即攻破太仓城,事後写信给曾国藩,宣称“若火器能与西洋相埒,平中国有余,敌外国亦无不足”,這成為自強運動的起因之一。為建立现代化清軍,洋務派聘請外國教官來訓練八旗軍、绿营和守衛首都的神机营,一些兵工厂也建立起来。然而淮军的地位仍然不可動搖,例如發生天津教案时,尽管守衛首都的神机营已有三万之众,清廷仍然调集淮军来加强京师的防务。

自強運動隨中法戰爭與甲午戰爭而失敗,而清廷守舊派利用义和团抵禦西方列強的策略也隨八國聯軍而落幕。八國聯軍之後,清政府決定實施改革,即「清末新政」。為建立現代化清軍,早在維新運動時即建議成立一支現代化的陸海軍,組織團練並建立保甲制度。清末新政時,袁世凯在华北组建新建陆军(即北洋軍),张之洞在南方组建自强军。1904年清廷正式建立由36个镇组成的常备军的计划,而绿营在1901年即开始裁减。同时取消武举,在各省建立武备学堂,以培养新式军官。负责军事改革的中央机构是1903年成立的练兵处,它在1906年被并入新立的陆军部,其尚书和左右侍郎都是满人。清廷试图削弱地方軍閥力量,1907年袁世凯和张之洞就在明升暗调中被剥夺军权。1908年宣統帝的摄政王载沣决定进一步加强对军队的控制,在1909年的一道上谕中,他宣布自己(代表年幼的皇帝)对军队行使最高统帅权,他还把自己的兄弟任命为海军处和军咨府的管理大臣。到清帝国灭亡前夕,其陆军可以号称100万,但大概只有60万战斗人员,其中只有17.5万人是现代化的正规军。並非所有新軍均效忠清廷,部分是暗中支持革命軍的。参谋机制上,在1907年即仿照西制成立参谋部门军咨处,隶属陆军部之下。为把军事管理和军事指挥分开,1911年决定把它升格为独立于陆军部外的军咨府。

清朝视水师为陆军之辅。加之满洲以骑射为本,故不善水战。入关初期,在对抗郑成功等海上抗清势力时,往往力不从心。1636年皇太极征满洲瓦尔喀部,即开始造战船。1651年顺治帝令沿江沿海各省循明制,各设水师,此为清朝水师之始。内河防务以长江为主体,沿岸各设水师。海防上,清朝為封鎖明郑的經濟力量,實施海禁。即使在平定明郑后,仍受海禁影响,水师多以防御为主,缺乏攻击性战舰。嘉庆时由于东南沿海海盗泛滥,就有学者开始注意海防,如湖南的严如煜写有《海防辑要》。鸦片战争后在面对西方炮舰时,清朝水师一战即溃的事实讓魏源、郑復光等人意识到东西方的差距,紛紛提倡建立現代化海軍。太平天国兴起时,英國協助清朝建立中英聯合指揮的阿思本舰队,然而指挥权的問題使得艦隊解散。

直到自強運動时,清朝才有新建海军的动作。為建立船艦自製能力,1866年清廷在福州马尾成立总理船政事务衙门,以沈葆祯为船政大臣。同年,李鸿章要求其江南制造局建造炮舰。1868年8月,第一艘中国制造的蒸汽军舰,“恬吉”号下水。然而自製船艦与外国艦隊相比較差也較貴,李鸿章等官员仍然从国外購艦為主。其中最有名的即是由德國建造定远與镇远,這兩艘是北洋艦隊的主力艦。人事上,早在1867年即建立福州船政学堂以培養海军军官,1872年和1876年分別派使团前往海外学习。沈葆桢和丁日昌离开後,福州船政局开始衰落。1880年李鸿章在天津成立天津水师学堂,张之洞在广州成立水陆师学堂(1887年),曾国荃在南京开办南洋水师学堂(1890年)。1885年10月清政府宣布成立海军衙门,以醇亲王为总理大臣。

清朝先後建立四支舰队:受北洋大臣节制的北洋艦隊,受南洋大臣节制的南洋舰队,受福州船政局节制的福建水师,受两广总督节制的广东水师。其中北洋艦隊在當時被評論為世界第八、亞洲第一的海軍艦隊。然而四只舰队資源獨立,互不统属,也不互相合作。财政上,1891年慈禧太后挪用海軍預算於興建颐和园。到1890年后,守衛黃海、東海的北洋艦隊即已“停购船械”。李鸿章也称“自光绪十四年(1888年)后,并未添购一船。操演虽勤,战舰过少”。隨後的中法戰爭南洋水师、福建水师遭受重创,甲午战争中北洋水师全军覆没,這也标志自強運動的失败。隨後旅順、大連、威海卫、胶澳與廣州灣等海军基地相继丧失,八國聯軍後大沽等地沿海砲台又被列強下令摧毀,清朝已无海防可言。1909年,清廷决定成立海军处,并将残余的战舰重编成巡洋和长江两舰队。1910年改海军处为海军部,力图重振海军。

清朝統治者根據實地情況的差異採取不同的政策。在中原地區基本沿襲明代的統治方式,包括開科舉等,以贏得漢族知識分子的支持,並根據清朝的實際情況實行旗民分治;在邊疆地區則採取加派駐防大臣與當地貴族共同治理。

清朝統治合法性建立的基礎與前朝有所不同,清朝統治在鞏固滿洲自我認同的同時,兼容其他族群的信仰和習俗,使之擁有遠超前代的疆域和領土。清朝皇帝本身擁有各種政治與宗教頭銜,具備不同文化象征意義的多維品格,體現出對各類臣民復雜多樣之宗教信仰的認可。因為清朝對不同地域和族群的宗教信仰采取了更多的包容政策,使各種異質文化因素能夠共存。

清初入关之后有「六大弊政」之说,剃髮(或薙髮)、易服、圈地、佔房(侵佔房舍)、投充(搶掠漢人為奴隸)、逋逃(逃人法),延續時間最長的,是逃人法。順治七年六月,廣西巡撫郭肇基等人因為「擅帶逃人五十三名」,被處死。清初曾頒令諭:一、八旗制度移入關內,全族皆兵。二、鼓勵滿人入關。三、圈地,使近畿五百里內全屬旗人所有。四、禁止旗漢通婚;禁止滿人自由擇業。弊政中的投充和逋逃皆為圈地所造成的直接結果。康熙帝親政後即立即下令永遠停止圈地,並逐步放寬對逃人的禁令並最終裁撤督捕衙門。隨後康熙開始採取一系列與民休息的政策。

清軍入關之前,為易於辨識順逆,就已要求被征服或投效的漢人改變髮式。順治元年(1644年),多爾袞率清軍入關。山海关之战後,多爾袞下令沿途州縣官員按滿人風俗,剃頭留辮。清軍驅逐李自成,定鼎北京,漢人強烈反對剃髮,降清之漢族官員剃髮者亦寥寥無幾。多爾袞見滿清統治尚未穩固,便下旨收回成命,命「天下臣民照舊束髮,悉從其便」。明朝降臣孫之獬卻全家主動剃髮迎降,更令妻子不再纏足,並上疏標榜「臣妻放足獨先,閤家剃髮效滿制」,得授禮部左侍郎,兼翰林院侍讀學士。清初筆記《研堂見聞雜記》稱,孫之獬入朝後,列於滿班,滿臣認為他是漢人而不受;歸入漢班,漢臣又因為他從滿俗而不容,孫之獬於是羞憤上疏,稱「陛下平定中國,萬里鼎新,而衣冠束髮之制,獨存漢舊,此乃陛下從中國,非中國從陛下也」,言辭激烈。順治二年(1645年)五月,大順與南明弘光政權相繼被清軍摧毀,多爾袞認為大局已定,於六月重新下剃髮令。七月,又下令「衣冠皆宜遵本朝之制」,規定清軍所到之處,成年男子無論官民,限十日內盡行剃頭,削髮垂辮,不從者斬,以恫嚇抵抗軍民。江南地區不少人反抗剃髮令,嘉定三屠等亦由此引發。

當時一些在華傳教士曾描述過當時一些城市的屠城情況。20世紀的法蘭西學院院士阿蘭·佩雷菲特認為:「建國後的最初幾年,整批整批的百姓遭到屠殺。強迫留辮子引起騷亂,結果都被殺害而倒臥在血泊之中。」

被指帶有「傳統漢化的歷史觀」和「近代反滿興漢民族主義」觀點,且對滿人統治者的統治評價負面的錢穆認為:「清人又想讨好民众,又存心压迫知识分子,他们只需要有服服帖帖的官员,不许有正正大大的人,结果造成了政治上的奴性、平庸、敷衍、腐败、没精神」。

清朝統治者為能使自己的王朝更長久,按歷代漢人王朝的傳統開設科舉,大力尊崇儒學,從中選拔統治精英以贏得漢族知識分子的支持。早在後金時期,努爾哈赤和皇太極就曾起用範文程、寧完我等漢八旗人士。崇德年間,又先後招降洪承疇、吳三桂、孔有德、尚可喜、耿仲明及其統領的漢族軍隊。後來,這些前明將領在消滅南方反清勢力的過程中起到重要作用。另外,康熙年間帶兵攻克台灣的水師將領施琅也是前明降將中為清朝立功的代表人物之一。晚清時期,漢族官員逐漸成為清王朝的中流砥柱,有虎門銷煙的林則徐;還有在消滅太平天國和捻軍中立下大功的曾國藩、李鴻章、左宗棠;又如在洋務運動中起到關鍵作用的張之洞、劉坤一;實行新式練軍的袁世凱等。

清太宗皇太極即汗位後改變努爾哈赤對漢人的政策,釋放掠奪來的漢人奴隸,編莊別居,將加入漢軍八旗的明朝官員或後金提拔的漢人官員來管理。1633年皇太極下令從所屬的滿洲八旗的漢人壯丁中每十名抽出一名,組成一旗漢軍,這是皇太極組成漢軍的開始,並成為漢八旗的前身。

隨著軍隊的發展,崇德二年(1637年),即皇太極稱帝改後金為大清的第二年,又分漢軍旗為兩旗。又過五年,崇德七年(1642年),把漢軍擴為八旗。至此,漢軍八旗正式出現,成為清朝三軍之一。所使用的旗幟和滿洲、蒙古相一致,即正黃、鑲黃,正白、鑲白,正紅、鑲紅,正藍、鑲藍。由於漢軍編成八旗,所有旗下成員都是旗人,也稱漢軍旗人。漢八旗中的原漢人後代與滿人同樣享受世襲待遇,亦有很多世襲佐領之職。漢軍旗人在司法上和滿洲旗人一樣,與民人同罪不同罰。乾隆年間,由於財政原因,漢軍八旗曾大量出旗為民,僅剩在後金時期便跟隨滿洲統治者漢軍勳舊之後。雍正元年(1723年),汉军与汉人家奴壮丁共计44万余人,约占当时八旗人丁总额的72\%。至出旗后嘉庆元年(1796年)的再度统计,已降至总人口的43\%,可见汉军出旗之规模是巨大的。到宣統末年,漢軍旗人共有21596人,約佔旗人總數的6\%。

一般認為,清朝統治者在保持滿族優先前提下,很大程度上採取漢化政策。但一些研究遼金元清史的日本學者認為,清和遼金元一樣屬於中國的征服王朝,漢化深度和速度均遜於北魏等滲透王朝;清室只推行對自己有利的漢化措施,並儘可能保留本族文化,而非全盤漢化。

所有施政文書都以滿漢兩種文字發佈。自康熙起大力推行以儒學為代表的漢文化,漢傳統經典成為包括皇帝在內的滿族人必修課。滿族皇帝也納有漢族嬪妃,詳見滿漢通婚。儘管滿漢通婚的現象早已普遍存在,不過真正解除滿漢通婚禁令,是直到1902年清末新政才完全落實。到乾隆中期,滿人幾乎全部以漢語為母語,滿文漸漸成為僅用於官方歷史記載用的純書面文字,並在使用中逐步為漢文所取代。部分史學者認為,正因滿人自動漢化才沒有在短時期之內覆滅,甚至反被漢人奴化。若無法漢化,則如南北朝的胡族政權一樣,無法吸收漢族先進的文化而滅亡。支持儒化說者則認為,清朝皇帝只是有選擇尊儒,儒家的一些思想清朝皇帝也沒有完全接受,而儒家只是漢文化的內容之一,漢文化不僅僅只包括文字和儒家,還有衣冠、風俗禮儀、各種宗教信仰等 。

土司制度是在唐宋時期羈縻州縣制的基礎上發展而成的,其實質是「以土官治土民」,承認各少數民族的世襲首領地位,給予其官職頭銜,以進行間接統治,朝廷中央的敕詔實際上並沒有能夠得到真正的貫徹。有些土官以世襲故,恣肆虐殺百姓,為患邊境,「漢民被其摧殘,夷人受其荼毒。」。

康雍乾盛世时期,国力强盛,中央政府已经有足够的力量加强对少数民族地区的统治。雍正四年(1726年),鄂尔泰大力推行改土归流政策,即由中央政府选派有一定任期的流官直接管理少数民族地区的政务,“改流之法,计擒为上策,兵剿为下策,令其投献为上策,敕令投献为下策。”,“制苗之法,固应恩威并用”。

乾隆後期學者魏源通過《聖武記》的編寫,認同清朝所代表的正統地位。至於地方上的士大夫們,還透過他們編寫於乾隆、道光和光緒等不同時期的《鳳凰廳志》,逐步確證了民間對國家及其民族平等政策的認同。日本東亞史學家安部健夫指出,改土歸流是一個借苗族的漢化,證明「夷性華化」能夠實現的「活廣告」,中國史學家王柯指出,至遲到道光皇帝在位的十九世紀前半,在清朝的帝國構造中,西南部的非漢民族地區就已經被完全當作「內地」來對待。

清朝皇帝強調“满汉一体”,在實行一些政策時會考慮到要平衡各族的心理,例如在康熙晚年的內閣大學士中經常在五至六人中保持一兩個南方人的名額,令南北地主共同參政。在康熙二十年(1681年)的內閣新加入的成員當中,有兩名滿人,四名漢人,而在四名漢人當中南人北人各半。此外,康熙亦重點選拔升遷較快的漢族士大夫,這些士大夫同時是內閣的候補成員,亦容許有「反清」思想的學者嚴繩孫擔任官職。有學者指出,“满汉一体”實際上是以「首崇满洲」為前提,清朝統治者的滿漢畛域觀念始終根深蒂固。作为统治族群和八旗军队中的主要组成部分,满洲人尤其被清朝历代君主视为国家根本、朝廷柱石。满洲将士为清朝定鼎中原、以及之后平三藩、灭准部等战役中立下汗马功劳。故终清一代,“首崇滿洲”(又称“满洲根本”)是清朝的既定國策。雍正帝曾明言,“惟望尔等习为善人,如宗室内有一善人,满洲内亦有一善人,朕必先用宗室;满洲内有一善人,汉军内亦有一善人,朕必用满洲;推之汉军、汉人皆然,苟宗室不及滿洲,則朕定用滿洲矣。”。

清朝时期,八旗子弟(以滿洲為主,也包含蒙漢八旗子弟)在政治或生活领域主要在教育、科考、补缺、律法、生活待遇等方面享有一定特权待遇。清廷特为宗室子弟特设宗学,觉罗子弟有觉罗学,普通八旗子弟有咸安宫官学等八旗官学,内务府子弟有景山官学等。在文武科举之外,还提供笔帖式、翻译士、皇帝侍卫等方式供旗人子弟进入仕途。高层官职也一向有旗缺與汉缺之分。旗人可酌任汉缺,反之除极个别情况外,理论上是不可能的。能力不足以登科做官的旗人可以从参与最基本的挑选八旗兵丁做起,旗人子弟中的未成年者还可以参与类似于预备役制度的“养育兵”选拔,按月可得一定薪资。清代旗汉亦不同刑,若正身旗人犯充軍、流刑罪者有免發遣以枷號代替的特權。满洲人的司法權也獨立於汉人之外。例如,駐防旗人觸法不归当地督抚管制,而由该地区驻防将军、都统负责。京旗子弟则由步军都统衙门负责处理、宗室则由宗人府全权裁决。清廷还分拨旗地和营房给八旗子弟居住生活,不必承担任何赋税。旗地和营房受国家保护,不得私自买卖。清廷在全国各处八旗驻防地实施旗民分治,駐防地俗稱“满城”,專供兵丁居住,非旗人不得随意出入满城。

东北满洲故地无满城之分,清初设置柳条边,防止汉人及外藩蒙古进入“龙兴之地”。然而在康熙、乾隆、嘉慶年間有多次漢人移居中國東北地區的記載:「其吉林宁古塔、伯都讷、阿勒楚喀、拉林等地方,乾隆二十七年定例不准无籍流民居住。及三十四年,吉林将军傅良奏:“阿勒楚喀、拉林地方流民二百四十二户,请限一年尽行驱逐。”上曰:“流寓既在定例之前,应准入籍垦种,一例安插,俾无失所。”嘉庆中,郭尔罗斯复有内地新来流民二千三百三十户,吉林有千四百五十九户,长春有六千九百五十三户,均经将军奏令入册安置。其山东民人徙居口外者,在康熙五十一年已有十万馀人。圣祖谕:“嗣后山东民人有到口外及由口外回山东者,应查明年貌籍贯,造册稽查,互相对覈。”其后直隶、山西民人亦多有出口者。」

诸多优待政策的初衷主要是为了保证兵源、加强八旗的军事职能。同时,这也导致满族受到束缚,居所不能远离本佐领之所在;八旗兵役也使得许多人在各大战争中战死疆场,一定程度上阻碍满族人口的发展;经济方面,满族也过于依赖八旗制度,清廷除兵差外,仅允许旗人在所属旗地务农,这使得以京旗为主、已经适应城居生活的满族在清朝中期开始出现生计问题。几代皇帝都曾尝试解决八旗生计问题,但终因不肯放任旗人自行谋生而均告失败。此外,东北满族因保持八旗兵农合一的习俗,始终没有产生严重的生计问题。自乾隆末年,清朝开始走向衰落,并且在之后一系列与外国侵略者的战争中接连失败,陷入内忧外困。这期间,在与汉人的交流中,满族逐渐接受汉文化,被视为立国根本的国语骑射遭到废弛。清末民初时,仅有黑龙江齐齐哈尔和瑷珲一带还有满语使用者。1911年,辛亥革命爆发,清帝逊位,民国建立,“首崇满洲”之国策也随之寿终正寝。

美國达特茅斯学院歷史學教授柯嬌燕(英语:Pamela Kyle Crossley)認為,在清帝國統治者所構建的天下秩序觀中,皇權的表達具有「共主性」(Simultaneities),清帝國成功地將幾種不同的統治方式糅入皇權之中,並在不同的地域空間和價值體系中發揮不同的作用,在這樣的天下秩序觀之下,談論「滿族中心觀」並沒有意義,清朝皇帝在絕大多數時間裡僅僅把滿洲人看作是多民族帝國的一份子,認為一個真正的帝國不是隸屬於某一個文化的,而是超越文化的存在,帝國的皇權也是如此。她又說,乾隆皇帝作為清帝國的最高統治者,擁有征服者、家族首領、神權領袖、道德典範、律令制定者、軍事統帥、文化藝術贊助者等多重身份,這多種身份既相互聯繫又具有混雜性,但是又集中統一於乾隆帝一身,乾隆帝不僅僅是滿洲人的大汗,更是全天下的共主。

清朝統治者對滿洲民族意識的梳理和重塑有重要的政治層面的考慮,清代邊疆的少數民族主要是通過對「滿洲」的認同來體認中華「大一統」,故有「崇滿洲以安藩部」,從而有效聯繫「大一統」政治格局的切實需要。「崇滿」所針對的主要是日漸興起的蒙古和回疆勢力的挑戰,及其所觸發的「胡虜無百年之運」的思想異動,這種異動在清朝國內和屬國朝鮮有所反映,雍正時期出使清朝的朝鮮使臣回國後給朝鮮國王的上疏中說:「自古夷狄之主中國,非有仁義德禮, 服天下之心而臣之也。華夷雜處,禍變層生,苟無聖人之應期,則漠北諸種,必將因其衰而代之。蓋今胡運之窮,不十數年可決,而蒙古强盛,異時呑倂,必至之理也。」這種情況在朝隆時期進一步深化,並且從準噶爾蒙古和大小和卓之亂可見。在這種歷史背景下,清朝統治者通過對滿洲部眾的精神整合與「國語騎射」傳統的張揚,威服和結合邊疆地區。

清朝在外藩蒙古地區建立盟旗制和札薩克制,對蒙古部落採取因俗而治、多封眾建的政策。旗是分解原來的部落而組成。每盟設盟長、副盟長各一人,掌管盟務。盟長先由各旗會盟時,從旗長即札薩克中推選。後來改為清朝理藩院開列盟內札薩克,由皇帝任命。其外每盟各設備兵札薩克一人,管理軍務。有的盟還設幫辦一二人,協理盟務。旗是軍政合一的地方政權機構,每旗設旗長一人,即札薩克,掌全旗要務,可以世襲。又設協理台吉襄贊旗務。其屬有管旗章京、副章京及參領、佐領、驍騎校等。旗盟官員多是原蒙古各部落的貴族,並被冊封為札薩克親王、郡王、貝勒、貝子、公、台吉等爵位。另外,清朝統治者在一些方面較為優待蒙古人,只有蒙古人可得到親王封號的待遇。

盟是由各部定期會盟而形成的機構,主要職能是監督各旗札薩克。若干相鄰的旗為一盟,盟有盟長,由朝廷直接任命,多選旗長中勢力大、威望高、與中樞關係親密者任之。盟為監察區,不屬行政單位。當時主要有哲裡木、昭烏達、錫林郭勒等盟。各盟旗直接對朝廷負責,受理藩院的管理。另外,在內蒙古地區設熱河都統、察哈爾都統和綏遠副將軍,率軍駐防要地,以加強軍事控制。但各都統、將軍不干涉行政事務。如科爾沁部一類可以自治,察哈爾與土默特則被取締。

清朝對外札薩克蒙古盟旗的管轄,中央有理藩院的典屬、柔遠清吏司等機構,地方上有駐紮大臣。定邊左副將軍即烏里雅蘇台將軍,為漠北蒙古地區的最高官員,下設烏里雅蘇台參贊大臣二人,與將軍共同管轄喀爾喀諸部盟旗。科布多參贊大臣及幫辦大臣管轄杜爾伯特、輝特、新土爾扈特等盟旗及札哈沁、阿明特、烏梁海等旗。庫倫辦事大臣掌中俄交涉事務,其屬恰克圖辦事司員等人,負責監督中俄貿易。

清王朝統一蒙古各部後,對蒙古的統治策略是,既要使其不再成為朔方邊患的勢力,又要籠絡當地領袖們統治當地人民,使蒙古成為清政府統治全國的一支重要軍事力量和清帝國北部疆域不設防的屏障。包括在蒙古大力扶植推廣藏傳佛教,有效的收服人心,維護蒙古地區安定局面。蒙古人一向信藏傳佛教中的黃教,滿人一直重視籠絡大喇嘛,如哲布尊丹巴與章嘉呼圖克圖;與此同時,蒙古八旗亦成為清朝軍隊的一支生力軍,在征討噶爾丹的過程中曾發揮過重要作用。外藩蒙古旗下的軍事力量(不屬八旗)也曾發揮重要作用,如清中期的喀爾喀的超勇亲王策棱和晚清時的博多勒噶台親王僧格林沁的部隊。

清代西藏地方官府為噶廈。清朝將西藏納入版圖後為加強對西藏的治理曾採取一系列重大措施,建立政教分離的制度。當時清政府的治藏政策有設置駐藏大臣,訂立治藏章程;派駐官兵,整頓藏軍;設立台站,釐定疆域等。乾隆十六年(1751年),清政府平息頗羅鼐之子珠爾默特那木札勒叛亂後決定廢除札薩克郡王監政的權力,設立噶廈衙門,由駐藏大臣與達賴喇嘛共同領導。噶廈為總管藏務衙門,設三品官噶倫四人。下設商上,為分管財政的機構,除以噶倫一人管理外,設四品仔琫(孜本)三人,商卓特巴二人。還有專掌糧務的葉爾倉巴、管理拉薩城的朗仔轄、掌刑名的協爾幫、掌馬廠的達琫及第巴等四至七品的各種名目官員。後藏也設四品商卓特巴、葉爾倉巴,五品達琫等官員,掌管相應的政務。武官則有四至七品的戴琫(代本)、如琫、甲琫、定琫,從幾人至百多人。凡前後藏皆有營寨,按其地理險易和大中小,各設邊營官及營官,總計一百六十餘人。佛教在西藏盛行,喇嘛很多,有的喇嘛在噶廈、商上任職,而僧官又分國師、禪師、札薩克大喇嘛、札薩克喇嘛、大喇嘛、副喇嘛等,專掌教事。

乾隆五十八年(1793年)清政府出兵打敗入侵西藏的廓爾喀(尼泊爾)後頒行《藏內善後章程》二十九條,對西藏的宗教事務、外事、軍事、行政和司法等做出詳細的規定,并加强駐藏大臣的权力。駐藏大臣与達賴、班禪的地位是平等的,而達賴與班禪之間則互為師徒。駐藏大臣作為清政府的代表,可直接向皇帝上奏。達賴、班禪上奏事宜必須通過駐藏大臣轉奏。此外,達賴、班禪及以下呼圖克圖十八人、沙布隆十二人等活佛轉世,稱為「呼畢勒罕」,即奔巴金瓶掣簽,均由駐藏大臣監督。

清朝治藏期間,清政府振興西藏經濟的措施有改革烏拉、租賦、錢法、貿易制度;活躍民族貿易;創報、興學、發展農牧工礦業和加強交通、郵電事業的開發等。清末時為防止英國殖民者對西藏的滲透,川滇邊務大臣趙爾豐等決定在川邊藏區進行改土歸流、建置州縣等,以繼續加強中央政府對西藏地區的管理。宣統二年(1910年)川軍入藏後還曾計劃將改土歸流擴大到整個藏區並鞏固對藏南地區的控制,但因次年(1911年)武昌起義的爆發而作罷。

天山山脈將新疆分成天山南北兩個區域。天山南路的大部分地區為當今維吾爾族的祖先所居,亦稱為回部。由於哈密、吐魯番率先歸服,被封為回部札薩克。乾隆平定大小和卓之亂之後,新歸附的地區不設札薩克,實行伯克制。伯克原來是回部的酋長,經清朝重新任命,按職責和品級稱「某某伯克」,共三十餘名目。最高的為阿奇木伯克,掌綜一城回務,三品至六品,其次為伊什罕伯克,掌贊理回務。四品至六品。其餘分掌地畝、田糧、稅務的,大抵四品至七品。在清朝所封的札薩克郡王和諸伯克之上,清朝還派駐伊犁將軍,掌天山南北最高軍政大權,下設參贊大臣一人輔之。又設塔爾巴哈台參贊大臣,喀什噶爾參贊大臣及幫辦大臣,葉爾羌辦事大臣及幫辦大臣,和闐、阿克蘇、烏什、庫車、喀喇沙爾辦事大臣等。乾隆末年以後,內地漢人和回民開始遷往新疆(以天山北路為主),把大片不用的牧場變為戶屯。到十九世紀初,這些定居北疆的移民已達數十萬人。在迪化(烏魯木齊),1808年時各縣的民戶農田數量已達到1775年的十倍。

從乾隆後期開始,以沈垚、張穆、龔自珍等為代表的學者,均紛紛關心邊疆事務,為國家獻計獻策。龔自珍大倡「回人皆內地人也」,無所謂「華夷之別」,並上疏安西北策,將新疆等同內地,主張「疆其土,子其民,以遂將千萬年而無尺寸可議棄之地,所由中外一家,與前史迥異也」。

顺治四年(1647年),《大清律例》编修完成。《大清律》基本上承袭《明律》的内容。后经康熙、雍正两朝屡次增删,并于雍正五年公布。但清朝最经常起作用的是例,而不是律。胡林翼說:「《大清律》易遵,而例難盡悉。」,胥吏都諳熟例案,常可執例以壓制長官。清廷对各少数民族地区还有各种特订的法律,如对蒙古族有“蒙古律”,对维吾爾族有“回律”,对藏族有“番律”等等。

清朝皇帝為打壓漢人反清復明運動與防止散播不利皇帝的消息,屢興文字獄以控制士大夫的思想。文字獄之案件常是無中生有,小人造謠所為。較大規模的文字獄甚至可以牽連多人受害。柳诒徵稱“前代文人受祸之烈,殆未有若清代者。故雍乾以来,志节之士,荡然无存。……稍一不慎,祸且不测。”。顺治四年(1647年),发生第一起文字狱“函可案”。一位法号函可的和尚因藏有“逆书”《变记》而被逮捕,后来流放到沈阳。顺治末年又发生莊廷鑨明史案,并驚動朝廷中的輔政大臣鰲拜等人。清朝諸例文字獄中,有名的有康熙時期的南山案、雍正時期的查嗣庭試題案和呂留良案等。

《劍橋中國史》評價:「清代文字獄中禁止的大多數作品一直被保存下來,而大多數遺失的作品不在被禁之列。這可能是直到今天在許多國家看到的現象的又一種說明。一本被列入禁書名單的書,被認為有特殊價值,從而被小心地保存下來。禁令實際上是最有效的廣告形式。」。

清朝在近代以前並沒有正式的外交機構,因当时清廷一向以天朝上国自居,不愿承認与四周國家的平等關係。清朝的外交按照對象的不同,分由禮部、理藩院、內務府與公行制度負擔外交事務。六部的禮部負責對日本、朝鮮、琉球與東南亞各國外交或朝貢事務,以維繫朝貢體制。理藩院負責交涉東亞內陸如內外蒙古、準噶爾、西藏、俄羅斯帝國等事務,主要防止邊患形成。其编制与六部基本相同,官员大多由满族、蒙古族人担任,汉人只能担任堂主事、校正官等少数官职。內務府除管理本身內廷事務,也管理歐洲來華傳教士、宗教使節團的事務以及國外貿易的傳運徵收特別稅。公行制度負責西洋各國如葡萄牙、荷蘭、英國等貿易關係(在清朝來看仍為朝貢),限制於廣州一地,又稱廣州制度。

鴉片戰爭開啟中國近代史,清朝對外關係轉向被歧視。由於缺乏正式外交機構,為西方國家不滿,在《天津條約》就有要求外國公使進駐北京,這使得中國正式開始面對新的外交形勢。1861年,總理各國事務衙門成立,專門負責對外關係;然而,其地位逐漸被1870年成立的北洋通商大臣所取代。直到1901年的清末新政,將總理衙門改制為外務部後,才得以統一負責對外事務。

清朝的藩屬國方面,早在皇太極與康熙時期就有朝鮮與琉球國。到乾隆時期擴充到東南亞地區的安南(即越南)、南掌(琅勃拉邦王国,地屬今寮國)、暹羅(今泰國)、緬甸以及南洋群島的蘭芳共和國(柬埔寨被安南與暹羅瓜分,呂宋與蘇祿於西班牙統治菲律賓群島後相繼消失);西南喜馬拉雅山有廓爾喀(尼泊爾)、哲孟雄(錫金)、不丹等國;中亞地區有哈薩克汗國、布魯特汗國、浩罕汗國、布哈拉汗國、愛烏罕(今阿富汗)、巴達克山、乾竺特與拉達克等國。

清朝建立伊始,清政府為禁止和截斷東南沿海的反清勢力與據守台灣的鄭成功部的聯繫,以鞏固新朝的統治,曾五次頒布禁海令,並三次頒布「遷海令」,禁止人民出海貿易。1683年清軍攻佔台灣後,康熙接受東南沿海的官員請求,停止清初的海禁政策,并在“粤东之澳门(一说广州)、福建之漳州府、浙江之宁波府、江南之云台山”分别设立粤海关、闽海关、浙海关、江海关作为管理对外贸易和征收关税的机构。江浙闽粤四大海关总领各自所在省的所有海关口岸,通常下辖十几至几十个海关口岸。但當時西方殖民者的殖民地已遍佈中國周圍,其殖民步伐引起了清廷警惕,康熙甚至曾口諭大臣們:「海外如西洋等國,千百年後中國恐受其累,此朕逆料之言」。

乾隆二十二年(1757年),乾隆帝以海防重地规范外商活动为理由,谕令西洋商人只可以在广东通商。乾隆谕令“本年来船虽已照上年则例办理,而明岁赴浙之船,必当严行禁绝。……此地向非洋船聚集之所,将来只许在广东收泊交易,不得再赴宁波。如或再来,必令原船返棹至广,不准入浙江海口。豫令粤关,传谕该商等知悉。……令行文该国番商,遍谕番商。嗣后口岸定于广东,不得再赴浙省”这一上谕只是让“外洋红毛等国番船”、“番商”只在广东通商,不得再赴浙江等地,而不是一些资料中所说的关闭江、浙、闽三海关,更不是“广州一口通商”。

鴉片戰爭前美國和英國兩個航海大國的船舶總噸位的總和一度遜於清朝,當時中國沿海商船總數約在9,000至將近10,000艘之間,約有150萬噸。加上其他種類的船舶,全國總有大小江海船舶20多萬艘,共計400多萬噸。而在1814年,英國全國有大小21,500多艘船,共240萬噸;美國在1809年全國有船舶135萬噸。

18世紀,歐洲各國普遍流行中國風尚,当時歐洲人對中國普遍持正面和嚮往的態度,例如被譽為「法蘭西思想之父」的法國啟蒙時代思想家伏爾泰就曾高度讚揚當時在位的乾隆帝及中國的政治和文化。不過與此同時亦存在不同的聲音,佩雷菲特指出1792年外交失敗的英使馬戛爾尼回程路上寫的紀事中就包含批判性的看法。然而馬戛爾尼在他的日記《A Journal of an Embassy from the King of Great Britain to the Emperor of China》裡對清朝的平民百姓的日常生活、官員對下屬的態度、統治者的儀態以至整體社會和國家的法律制度有多處正面評價。

有學者指出,在乾隆以後,清朝開始實行「閉關鎖國」政策,一開始是四口通商,到後來只有廣州開放對外通商,且由十三行壟斷其進出貿易。當時西洋的科技發展蓬勃,漸漸地超越奧斯曼帝國為首的伊斯蘭世界和以清朝為首的東方世界。

但也有學者指出,「閉關鎖國」實際上是西方侵略者強加在清朝頭上的貶詞,反映出當時對華虎視眈眈的西方國家不顧事實反誣清朝排外,又指出就算是當時西方各國的口岸也只容許本國船隻進出,本國的進口貨物只容許本國船或原產國船裝運,稱之為「保護政策」,但又同時強迫其他國家洞開國門,任由他們自由離去和壟斷,是為雙重標準。此外,英使馬戛爾尼曾向清朝提出六項要求,當中包括:要求英國貨船能到浙江、天津等地收泊;要求在北京設立商行;要求在珠舟山佔一島嶼,以便英國人居住和收存貨物;要求在廣州城劃一地方居住英國人,或者居住澳門之人出入自便;要求准許英商從廣東內河航行澳門,貨物不納稅或少納稅;要求確定關稅條例。

乾隆帝隨之復書批駁英國使臣的要求,有學者認為信中雖有妄自尊大的一面,但一些史學家往往斷章獨引「天朝物產豐盈,無所不有,原不藉外夷貨物以通有無」這句話證明清朝「閉關鎖國」,對英國侵犯中國領土完整及關稅自主的六項要求避而不談。清廷為了防止澳門被霸佔的情況重演而限制英國只能在廣州一口通商,其他國家仍然可到四口通商,且並無任何限制。廣州海關以外的江、浙、閩三個海關依然對外開放。美國東亞史學家歐立德也指出乾隆帝並非如過去所想的對外界一無所知,乾隆不僅熟悉西方地理,同時也清楚歐洲法、俄兩國內部的情勢,且清朝政府也認識到英國在印度和廣州的影響力。

清朝在近代以前与他国簽定的條約較少,主要有《尼布楚條約》與《恰克圖條約》,均為與俄國簽定。

自19世紀以來,經歷一系列失敗之後,清廷在列強的威迫下,前後被迫簽定許多不平等條約。據統計,中國近代簽定的不平等條約共有343個,其中四十多個條約影響較大。清朝在西方國家的威逼下通過開放租界口岸,允許外國人來華經商等割地手段來達到和解。致此中國開始向近代過渡,清廷在被迫打開國門的同時也喪失中國大量領土的管轄權。甲午戰爭後,列強鑒於清朝失去自衛能力,紛紛劃分在中國勢力範圍,中國變成半殖民地。由於受到西方乃至日本的侵略以致割地賠款,中國人的國際形象和地位也驟然下降。但與此同時也激發自強運動等改革措施使西方的科技、文化以及民主憲政思想傳入中國,並為隨後的辛亥革命提供發展契機。

清朝與中國歷史上的其它朝代一樣,本來並無法定的國旗與國歌。近代以後,隨著西方國家用武力打開清朝國門,清朝逐漸引入西方國家的一些概念,其中就包括國旗與國歌。晚清重臣李鴻章在同西方國家談判、簽約、通商、互派外交人員等外交活動中,看到西方列國莊嚴懸掛國旗,而中國卻無旗可掛,深感有失「天朝威儀」。於是上奏慈禧太后,提出在外交場合中需要有代表中國的旗幟,請求頒制國旗。1888年(光緒十四年),清政府認定「黃底藍龍戲紅珠圖」(即俗稱的「黃龍旗」)為大清國旗。這是中國歷史上正式確立的第一面國旗。19世紀後期至20世紀初,清朝曾先後使用《普天樂》、《李中堂樂》、《頌龍旗》作為其半官方國歌或代國歌。1911年,清政府將《鞏金甌》定為正式國歌。不過由於辛亥革命的爆發,《鞏金甌》後來沒有流行開來。

古代中國只有「戶籍」制度而無明確的「國籍」規定,是以「戶籍」管理制度實現「國籍」管理功能。為了能夠在保護海外華人、華僑方面能夠採取法律依據,以及加強海外華人及華僑對大一統中國的認同,清政府於1909年3月28日頒布針對荷蘭國籍法的「出生地主義」、採用「血統主義」原則的《大清國籍條例》,也就是中國第一部國籍法,條例中的《固有籍》部分規定:

第一條:凡左列人等不論是否生於中國地方均屬中國國籍:生而父為中國人者、生於父死後而父死時為中國人者、母為中國人而父無可考或無國籍者。

第二條:若父母均無可考或均無國籍而生於中國地方者亦屬中國國籍。其生地並無可考而在中國地方發現之棄童同。

由於受到當時「大民族」主義的影響,清政府以「血統主義」而不以「居住地主義」的原則來確立國籍法,清政府「獨采折衷主義中注重血脈系之辦法」,其「血脈」亦包括中國各個民族如滿、漢、回、蒙等民族,「統轄於中國中華大『血脈』之中之意」與「大民族」主義主張的「國內本屬部之諸族,以對國外之諸族」觀念一致 。旅美華裔史學家何炳棣表示,是滿族創造一個包括滿、漢、蒙、回、藏和西南少數民族的多民族國家,而為五族共和的中華民國張本。

明末清初,因為流寇擾亂、清兵入塞、入關戰爭與三藩之亂的關係造成人民生命與財產的損失。而飢饉、瘟疫使得中國人口又一次的急速下降。史學家葛劍雄認為明清之際人口的跌幅估計可達40\%,從崇禎元年(1628年)以來平均每年下降19\%,至順治末年達到谷底。另一種說法則認為,人口隱匿數量遠大於人口損失數量,而真正人口損失最大的時期就是入關戰爭的戰亂時期,以及各種的滅絕性屠殺如張獻忠屠川。康熙二十年(1681年)後,清廷平定三藩之亂並佔領臺灣,經過康雍乾盛世獲得長期的休養生息,人口得以迅速增加。清初人口數量未明確,史學家姜濤估算康熙十九年(1680年)前後,人口增長到1億;趙文林推估在康熙二十四年(1685年)超過1億。到乾隆時期,全國人口正式突破2億,到鴉片戰爭前夕的道光十三年(1833年)又猛增到4億。清朝人口的增長一反中國人口過去的波浪式增長型態,呈現斜線上升。19世紀時,清朝因為太平天國的起事、捻亂與回亂等戰亂損失不少人口;光緒年間又發生不少天災,光緒三年山西、陝西發生旱災,因飢荒與暴亂而死的人達一千萬以上。最後加上海外移民風氣日盛,因此到清朝滅亡時,中國人口維持在4億3千萬多人,與道光年間的人口數差不多。

清朝的人口擴張和流動規模在長期性和常規化等方面在中國歷史上前所未有,而且清政府在大部分情況下都鼓勵內地人往外移民。 學者認為,清朝鼓勵內地人移民,讓漢文化推廣到非漢民族地區,是為了「移民實邊」,發展經濟,保衛疆土,是「為了中國本身的利益,也為了較為落後的邊疆地區人民的前途」。

在鴉片戰爭前後,中國內地各省約有七百至八百萬人遷移到邊疆向海島地區,形成了一股股由中原向東南、西南、西北、北、東北四面八方輻射狀的移民浪潮。

由於明末清初長年的戰亂與屠殺,產生許多真空地區,而後又因為人口大量提升,使部份省份人口過剩,這些都帶動移民潮。例如明末領有四川的流寇張獻忠,他於1646年兵敗退出成都時,在四川進行空前的燒殺破壞:40萬人的成都只剩下20戶居民,人口從至少三百萬一度銳減到只有八萬人。後來清廷推動以湖廣、陝西等各省人口填補四川地區,史稱第二次湖廣填四川,移民四川的趨勢歷經一個世紀。

清廷為保護其發祥地中國東北地區,於奉天(今遼寧省)設立柳條邊以限制漢人及外藩蒙古向盛京地區及長白山地區移民,但仍「封而不禁」,對越邊時而佯作不知,並通過流放犯人,招撫流民及移旗等方式進行移民實邊。而乾隆年間則開始嚴厲封禁東北地區,直到1792年,由於發生旱災,清政府公開放鬆禁令,允許並鼓勵災民前往長城及柳條邊外的東蒙及東北地區謀生,以分流難民潮。該措施隨即引發了規模空前的難民遷徙,東北三省、特別是柳條邊沿線地區從此開始大量接收關內移民。1792年旱災後的10餘年間,清廷即在東北柳條邊沿線地區新建四個行政單元以管理移民,包括長春、昌圖、伯都訥和新民,大凌河東岸、養息牧廠、拉林、雙城等官墾聚落也分佈在附近。

據統計,1780年東北人口約95萬,至1820年猛增至247萬人,較1780年增長1.6倍,年均增長率24.2‰,增長人口中大部分來自移民(約100萬),其中吉林省接收移民30萬人,移民增加的趨勢極為迅猛。1873年,清廷考量俄國與日本有意染指東北,故撤除柳條邊,並且在日俄戰爭後完全開放移民。

清朝初期,清朝統治者限制漢人與蒙古人混雜。但到了康熙年間攤丁入畝後,清廷放開封禁,引起大量民眾遷徙到長城口外漠南之地定居謀生的浪潮,史稱「走西口」。到了二十世紀初,清延開始以各類優惠政策大力鼓勵漢人移民內外蒙古,原因之一是為了反制當時俄國人在內蒙古東部的殖民計劃。由於意識到漢人大量移民最終可能會導致蒙古人自身被同化或全面漢化,加上漢商的詐欺手法以及清朝統治者的政策轉變,一些蒙古人因此懷恨在心並發起叛變。然而那些叛變基本上是地區性的叛變,而非組織嚴密的「全區性暴動」。

自雍正朝大規模地實行改土歸流政策以後,內地與西南廣西和雲南邊疆地區之間的交流被加強,地域壁壘被打破,內地人迅速移往西南邊疆地區,自清初至道光中葉,約有200-350多萬人遷入了西南滇桂地區。

康熙二十三年(1685)以後,清朝向前往台灣的大陸人民實行限額發給照票。雍正十年(1732)後規定渡台者有田產生業足以謀生,均可允許攜眷入籍,從此渡台人數急升。乾隆五十五年(1790),清政府為了減少私渡偷渡而決定多處官渡。自清初至道光中葉,共有150多萬人從中國大陸遷入了台灣。1895年臺灣割讓日本,住民根據《馬關條約》第5款,在同年5月8日住民去就決定日決定去留,最終99.75\%臺灣人選擇成為日本國籍,而中國大陸渡臺亦隨之中斷。

鑑於新疆戰略地位的重要性,清朝在統一新疆前後,對治理新疆的方針開展過深入討論,並確定向新疆移徒中國內地人口以充實農業勞動人口的方針,又制定一系列組織移民出關的具體措施,因而出現內地人戶移民新疆的熱潮。自清初至道光中葉,新疆地區約遷入了五十萬人。

福建與廣東各省因為山多人狹,又靠海,許多人口移民海外。台灣早在荷西統治時期、明鄭時期就獲得閩南、粵東的移民,約有十餘萬人。清初因為防止如朱一貴事件等動亂發生,曾嚴格限制移民台灣。同治末年發生牡丹社事件,日軍一度侵台,這使得清廷積極開放移民台灣。到台灣割日前夕,已經有三百數十萬的移民人口。早在十五、六世紀,閩粵人民就時常移民泰國、馬來西亞與印尼等東南亞地區,這些海外華人還建立蘭芳共和國。鴉片戰爭之後更多華人移民海外,主要以東南亞地區、美國西部、加勒比海群島為主。清朝滅亡時,海外已有七百萬的華僑。

康熙後期,經過長時期的休養生息,社會已日趨安定,但人丁與地畝的載冊數增加遲緩。一方面由於土地與人口的清查不夠徹底,再者也由於地主以多報少之故,貧民迫於賦役的繁重而相率逃亡,人丁的統計並不確實。康熙帝為確實掌握人口數,於康熙五十一年(1712年),下詔「盛世滋生人丁永不加賦」,以康熙五十年(1711年)在冊人丁數作為全國徵收丁銀的固定總額,以後新增者為「盛世滋生人丁」,從中央到地方均不得隨著人口的增加而增稅。但除補不易,弊端又無法避免。所以採取攤丁入地政策,廢除人頭稅(丁稅),併入土地稅內。這使得無產者沒有納稅負擔,而地主的負擔增加,對於清朝人口的持續增加、減緩土地兼併、以及促進工商業的發展有一定的作用。

錢穆認為“地丁摊粮”只收田租,不收人丁税的办法是「清朝统治者自己夸许的『仁政』。但是实际上,这一规定,并不算的是仁政,从中国历史讲,两税制度,早把丁税摊派到地租了」。中国科学院地理科学与资源研究所的學者說,地丁摊粮改革政策也給當時及其後的土地墾闢帶來了重大影響,雖然這是一次賦稅改革,似乎與墾荒並不相干,但它卻有效放鬆了對貧苦無地農民的約束,使人們可以自由流遷,為異地墾荒提供了豐富的勞力資源。歷史學者張硏評價,攤丁入畝在一定程度上改變了賦役不均的情況,增加了地主的賦役,減輕了無地少地農民的負擔,減少了戶口隱漏,穩定了社會秩序,促進了生產發展,以及改善了國家的財政狀況。

乾隆六年(1741年),戶部有感於人口的增長,有必要對人口登記制度加以檢討,並徹底改變戶口統計與管理制度,以掌握人口真實情形。然而卻遭到廷臣蘇霖渤等人們的反對,他們認為實施人口普查對維護統治沒有實質意義,各省戶口殷繁,「若每歳清查,誠多紛擾」。到乾隆三十九年(1774年),湖北東部發生災害,由於賑濟的人數超過地方登載的戶口總數,經過清查發現,有些縣份在每年上報人口數都含混交代。乾隆帝大怒,要各省督撫全面展開人口清查。隔年,保甲嚴格執行人口普查制度,共增加43,534,131人,此後全國各省人口數較以往更接近實際人口數。人口查報也成為保甲的一項重要職責。無論在乾隆四十年(1776年)前後,人口統計都限於各省。而且京師順天府、八旗、黑龍江、新疆、蒙古、西藏、台灣、雲貴川廣地區居住的少數民族等並未列入戶口統計中,不管何時見於官方記載的人口均低於實際人口數。葛劍雄以為乾隆四十一年至道光三十年的戶口統計數基本上是較可靠的。

清代社會上七十歲以上的老年人口占總人口的比例約百分之二。清代法令對於老人給予免稅和減刑,獨子犯罪可因親老而留養,而官員可因自己年老告休,也可因父母年老而申請終養。此外,清代老人在一些場合中獲政府的賞賜、禮遇和優待。貧窮的老人也是慈善機構救濟的對象之一。


清朝人口呈現爆發式增長,乾隆時期已達三億,相對應的,需要更加以多種方式提升糧食作物的產量。清朝採取開墾荒地、移民邊區及推廣新作物,改進種植技術等方式以提高生產量。由於國內與國外的貿易提升,經濟農業也相對發達。手工業方面改工匠的徭役制為代稅役制。產業以紡織和瓷器業為重,棉織業超越絲織業,瓷器以琺瑯畫在瓷胎上,江西景德鎮為瓷器制造业中心。清朝商業發達,分成十大商幫。其中晉商、徽商支配中國的金融業,閩商、潮商掌握海外貿易。清朝曾實施海禁政策,直到消滅明鄭統治台灣後,才宣佈展界開海,沿海對外貿易開始活躍,而貨幣方面則採銀銅雙本位制。康熙晚期曾為防止民變,一度推行禁礦政策,在一定程度上阻礙工商業的發展。

清朝人口在中国古代历史上为最高,其国内生产毛额(GDP)总量所占的世界比例在中国近三千年历史上也是最高的,据英国经济学家安格斯·麦迪森的研究,按照1990年美元价格计算,1820年清朝GDP总量为2286亿美元,占世界GDP总量的32.9\%,中国人均GDP为600美元,当时经过第一次工业革命的英国人均GDP为1,706美元。据他研究,即便被认为是中国历史上经济最繁荣时代的宋朝,其GDP总量为265.5亿美元,才占世界经济总量的22.7\%,宋代中国的人均GDP在450美元,略低于阿拉伯帝国阿拔斯王朝的人均GDP(621美元);这两个地区皆超过当时西欧人均GDP(427美元)。这里仅表明购买力平价,与所谓财政收入是不同的概念,英国财政年收入在1830年代至1840年代在5000万英镑以下;不过,清朝GDP数值在1840年前凌驾于欧洲之上,这一说法基本得到普遍认同,但部分中国学者如刘逖仍认为麦迪森高估了中国历史上的GDP总量和人均GDP。因此,刘逖对麦迪森1600至1840年数据做了调整,认为1820年中国人均GDP在325美元,而非麦迪森说的600美元。

清代的土地仍可分为官田和民田两大类。清朝入關後,1644年順治帝頒布圈地令。有主與無主地被滿人圈占,统称“官庄”。大量农民不得不弃家逃亡,或者沦为新主人的奴仆。圈地主要執行三次,以北京附近的顺天、保定、永平、河间四府最為突出,直到1685年康熙帝宣布廢止而終。至於全國其他原明朝皇室或地主的土地,清廷稱其為“更名田”,分配給無地農民使用,或是被新地主霸占。据统计,这种土地的总数不下二十多万顷。清代也拥有不少屯田,屯区多在新疆、漠南等边疆地带。清帝推行令民墾荒的政策。使得華北、華中地區先後著令准墾,一些邊疆如新疆、青海、海南、台灣、漠南等地區於清朝中葉先後實行開墾政策,而東北地區直到清朝後期才准許大量漢人前往開墾。

清代農業亦是歷史畝產量最高的一個歷史時期,秦漢時中國的畝產量為264市斤/市畝,唐代是334市斤/市畝,清以前畝產量最高是明代,為346市斤/市畝,清代的畝產量達到了374市斤/市畝,分別比漢代增加了41.6%,比唐代增代11.9%,比明的畝產量高了8%」,清代所編著的農書數量為之前所有中國朝代總和的2.09倍。另外,在清代的農書中,蠶桑類的農書共155部,而清以前所編寫的蠶桑書有4部,反映了清代蠶桑生產和蠶桑技術發展的程度。

清初,在康熙時期進行的多項水利興修。明末清初,黃河、淮河下游堵塞,京杭大運河也受阻塞。康熙帝時大力修治黃河,任靳輔為河道總督,採用疏導和築堤的辦法將黃、淮故道逐漸修復,使這一帶的農業生產在一段較長的時間裡減少水患的威脅。1713年康熙帝成功修浚位於北京的永定河,使舊河兩岸的「斥鹵」變為膏腴良田。另外,雍正時修築江浙海塘也是保護農田的水利工程。清朝的耕地面積於康熙時期逐漸提升。江南、湖广與四川等地的土地比中原地肥沃許多,湖广更有“湖广熟,天下足”之譽稱。

由於清朝人口成長超過可耕地發展速度,如何維持龐大人口有賴占城稻與一些新的糧食作物。占城稻在中國有一段長期的發展時間,到明清時期發展成五十日到三十日即可收穫的品種,使得二次收穫,甚至三次收穫變成可能。此外早熟稻耐旱,可在高原或山坡地種植。從宋朝初期到清朝道光年間,稻米產量以及耕種面積都增加一倍。一些從美洲引進的糧食作物也開發許多原先不擅種植的地形,以提高糧食生產面積。例如比較乾旱的高原有賴玉米與甘薯,更加崎嶇的山地則依靠馬鈴薯。到嘉慶年間,這些高原都種滿新一代的糧食作物。而河川沿岸的沙地則大量種植花生,約18世紀到19世紀才由南方推廣到北方。

清朝糧食產量遠超以往的歷史時期,康熙二十四年,全國共有耕地六億畝,到乾隆帝去世,全國耕地約為10.5億畝,全國糧食產量則迅速增至2040億斤。當時隨英國馬戛爾尼使團來中國的巴羅估計,中國的糧食收穫率高於英國,麥子的收穫率為15:1,而當時的歐洲,糧食收穫率居首位的英國也僅僅為10:1。

法國漢學家謝和耐認為:「中國農業於18世紀達到其發展的最高水平。由於該國的農業技術、作物品種的多樣化和單位面積的產量,其農業看來是近代農業科學出現以前歷史上最科學和最發達者。」。

清政府在各省設有常平倉,儲藏谷物以應付緊急需要,其幅度遠遠超過前朝。同時在全國設有災害監測網,任何地方遇上災害,政府便會利用附近常平倉的糧食來賑濟災民,以致清朝在鴉片戰爭之前從沒出現過嚴重饑荒。康熙年間的外國傳教士亦對清朝的治災手法有所讚揚。

乾隆帝多次蠲除國家賦稅錢糧,賑災救濟費用,在乾隆二十年之前達到2,500萬兩以上。乾隆十一年、三十五年、四十三年、五十五年共四次普免全國共計1.2億兩的賦稅錢糧,次數高於康熙年間的一次。

清朝的經濟作物种植面积也逐漸擴大,促进商品经济的活跃。棉花在清朝已是十分重要的经济作物,其產地遍及全国,其中江苏、浙江、河北、河南、湖北、山东等地都是著名的产棉区,甚至连农业发展较晚的奉天,也成了外输地区之一。產棉量以河北保定一带,长江中下游的松江、太仓與通州一带,以及上海等地最大。烟草原产地是美洲,明中叶以后开始传入中國,最早的种植地区是福建。种烟草获利很高,重要產地以陕南汉中、城固,山东兖州,湖南衡阳等地為主。湖南的衡烟、陕西的蒲城烟、北京的油丝烟、山西的青烟、云南的兰花烟、甘肃酒泉的水烟(又名西尖),均负盛名。甘蔗產地以華中、華南為主,江南、四川與台湾等地的制糖业非常发达。蠶桑業以江蘇浙江的蘇州、湖州、嘉興、杭州和廣東的廣州最為發達,已成為當地農民的重要生產活動。

清朝的手工業在康熙中期以後逐步得到恢復和發展。至乾隆年間,江寧、蘇州、杭州、佛山、廣州等地的絲織業都很發達。江南的棉織業、景德鎮的瓷器都達到歷史高峰。手工業分成官營與民營,由于工匠实行以银代役,所以顺治二年就下令废除工匠制度,官营缺乏必要的工匠而逐渐衰落。民间手工业興盛,例如云南民间炼铜場十分發達。苏杭一带民间丝织中已有不少具有专门技术的人,站在一定的地方等待雇用。

瓷器制作技术改进,產量也大幅提升。例如江西景德镇瓷窑所烧造的御瓷产量在雍正六年(1782年)时,一年之中生产十数万件御器。玻璃制造有较大的进步,清宫玻璃厂能生产透明玻璃和多达十五种以上的单色不透明玻璃,造型也丰腴美观。丝织技巧也有新的提高,出产的重要提花品种有妆花纱、妆花缎、妆花绢等。广东的「女儿葛」是广东增城的少女用一种葛藤的丝织成,質量極優。当时的棉织业以松江最为发达,技术最好,而染色、踹布业则以芜湖、苏州为最先進。

清朝劳动者与雇主之间的关系,主要是通过买与卖来体现的。不仅全部劳动成果全归雇主,而且在人身上也很少自由。在这些行业中,劳动者的工资是“按件而计”的;而且按照工匠技术的高低和工作的繁简論定工價。劳动者所得的工资,已经是根据劳动的熟练程度来规定。劳动者也比过去有较多的自由。例如苏州丝织业作坊中的劳动者,“倡众歇作”,要求增加工价,可以“另投别户”,追寻较好的待遇。

在明清時期的農業和手工業進一步發展的基礎上商業也很發達,商品貨幣經濟空前活躍。由於農業中商品性生產擴大,農產品越來越多地變為商品,出現許多專門化的經濟作物地區,為手工業生產提供原料,或者直接供應消費者。例如養蠶地區為調劑桑葉的供需,出現專賣桑葉的「青桑行」和「葉市」。一些經濟作物如蔗糖行銷國內外,茶葉於十八世紀輸出激增。糧食作物除大量供給城市居民食用外,還有不小的部分用於釀酒、油和豆製品加工等。這些產品自然都是為供應市場而生產的。

清代城市工商業者的地位相對改善,明代以來匠人對國家人身依附的“匠籍”制度隨之瓦解。國家對民營瓷窯、紡織工場及採礦等進一步放寬限制。大小城市各類作坊林立,蘇杭的絲織,松江的棉紡織,景德鎮的製瓷,佛山的鑄鐵等業名揚天下。

商品性生產的發展,商品流通範圍的擴大,促使一些新的工商業市鎮的興起和發展,例如漢口鎮和朱仙鎮就是位處交通樞鈕點而興起,而佛山鎮和景德鎮專司生產如絲綢、瓷器等高價值產品的城鎮。至嘉慶年間,這四鎮並稱為「四大名鎮」。其他興起的尚有于江泾、震泽镇等等。许多重要城市如北京、苏州、江寧(今南京)等地,也更趋发达。例如北京的居民已不下百万,一切生活所需,都从商业渠道取得,不能一日无贸易。當時尚流行“天下四聚”的说法:“天下有四聚,北则京师(北京),南则佛山,东则苏州,西则汉口。然东海之滨,苏州而外,更有芜湖、扬州、江宁、杭州以分其势,西则惟汉口耳。”

農業與手工業的發展為商業繁榮奠定基礎,揚州、蘇州、南京、杭州、廣州、佛山、漢口、北京,成為全國八大商業城市。中小城市星羅棋佈,取得比前朝更大的成就。

與此同時,金融業與貿易業發達,商人分成十大商幫。其中晉商、徽商支配清朝的金融業,閩商、潮商掌握海外貿易。廣州的行商與揚州的鹽商都是最闊氣的商人,山西商人掌控全國銀號。

清朝货币大体上採白銀與銅錢並用的銀銅雙本位制,大数用银,小数用钱,但银的地位更见重要。因海外貿易發達,白銀大量從國外輸入,康雍乾盛世流通的外国银元除西班牙银元外,还有葡萄牙银元、威尼斯银元、荷兰银元、法国银元等。鸦片战争前后,需要固定形式的银币出现,正式使用机器铸造银币则是鸦片战争以后的事。鴉片戰爭前,由於英國將大量鴉片銷入中國,導致中國白銀大量外流,需要更多的銅錢才能換取白銀。由於白銀是百姓納稅的固定貨幣,這帶動通貨膨脹,嚴重惡化經濟。使得曾經於1651年顺治帝发行纸币,到1853年咸豐帝又发行大清宝钞與户部官票等纸币,以穩定清朝經濟。

清廷在初期平定明鄭前对于民间海外贸易厉行海禁政策;对于外国来华贸易,仍沿袭明代的朝贡制度加以控制。最初与清朝发生朝贡关系的,主要还是南洋和东南亚诸国,但有许多限制,如贡期和随贡贸易的监视等都作严格的规定。对于西方国家来华商船的限制就更严。只许它们停泊澳门,与澳门商人进行贸易,每年来华贸易的大小船只,不得超过二十五只。1685年才允许外商到前述口岸通商。

清廷平定明鄭後放宽海禁,在“粤东之澳门(一说广州)、福建之漳州府、浙江之宁波府、江南之云台山”分别设立粤海关、闽海关、浙海关、江海关作为管理对外贸易和征收关税的机构。江浙闽粤四大海关总领各自所在省的所有海关口岸,通常下辖十几至几十个海关口岸,並准许外商在指定口岸通商,逐步建立一套管理外商来华贸易的制度,主要有公行制度和商馆制度。浙江、福建與廣東地區盛行海外貿易,人民時常與日本、琉球、東南亞各國及葡萄牙、西班牙與荷蘭等西洋各國展開貿易。到18世紀還有英國、法國與美國參與其中,當中英國幾乎獨佔對華貿易。西洋各國與日俱增的需要清朝的絲綢、茶葉與甘蔗,然而清朝對西洋事物需求不大,使得中國對外貿易呈現大幅出超的情形。大量銀元流入中國,增加貨幣流通量,刺激物價上漲,促進商業繁榮。在此期間,中國沿海以泉州、漳州、廈門、福州與廣州先後崛起,成為貿易大城,操控對外國際貿易。漢學家杜赫德認為,在清朝國內貿易的極盛時期,整個歐洲的總貿易量也不能與中國抗衡。

據西方文獻記載,清朝時期的中國各省被比喻為歐洲諸國,各自擁有自己豐富且多種多樣的特產進行貿易,如湖廣省和江西省專門向所有不產大米的省份提供大米;浙江省出產最優質的絲綢;江南盛盛產漆料、墨水,以及各種有趣的小作坊;雲南省、陝西省、山西省出產鐵、銅還有其他各類金屬,還富有馬、騾和毛皮生意等等;福建產糖和最好的茶葉;四川盛產植物、藥物、像大黃等等,而且都傾向於聯盟保護的形式,在所有的城市裡也一樣。清朝官員在商業界裡都擁有自己的股份/分成,同時亦惠及平民百姓。清朝的市集亦相當繁華,中外商家貿易往來頻繁,外國商人對中國商人的誠實也有深刻的良好印象。

清朝在康熙年間部分主要与近鄰國家或地區貿易的進出口貨品如下:清朝從日本出口的商品:各種藥材、白糖、水牛、各類刺繡、木料類等;清朝從日本進口的商品:珍珠、紫銅棒、日本刀具、花紋裝飾的紙張、瓷器、日式工藝品、金等;清朝與馬尼拉的交易商品:大量的絲綢,各類刺繡,地毯,茶葉,瓷器,各種藥材等;清朝從巴達維亞(今印尼雅加達)進口的商品:銀、各種香料、玳瑁、木材、瑪瑙玉石、琥珀、歐洲布料等;清朝向歐洲出口的商品:上等藥材、各色各樣的茶葉、金色棉線、麝香、珠寶玉石、水銀、瓷器等。

另外,當時期清朝與歐洲貿易最為重要的商品為日式工藝品、瓷器以及各類絲綢;清朝亦有將從歐洲進口的布匹轉銷至日本。

乾隆年間雖有10年的「南洋海禁」和在乾隆二十二年(1757年)禁止西洋商船前往閩、浙、江三海關貿易的阻礙和影響,但中國的海外貿易並未因此停頓或萎縮,而是不斷地向前發展,其規模和貿易總值超越前代,在清朝乾隆十年期間,四港貿易額總值達到36,571,777兩,是前朝最高時期的三十五倍,僅廣州一地,貿易額就是前朝全部貿易最高額的十餘倍。18世紀時期中國海外貿易的鼎盛還為荷蘭東印度公司帶來危機感。

乾隆二十二年(1757年),由于外商频年不断的掠夺和违法行为,清廷只保留广州一地作為「番商」如英国、荷兰、葡萄牙、西班牙等西洋商人的通商出口處,而江、浙、閩三个海关在乾隆、嘉庆和道光期間雖有所限制,但在某程度上亦有继续正常履行其管理对外贸易的职能。到十九世紀,英國在印度種植鴉片,並且大量銷往中國。這使得中國對外貿易逆轉為入超。鴉片的問題引爆鴉片戰爭,中國戰敗後門戶大開。南京條約不但開放廈門、上海、寧波、福州、廣州等五口通商口岸給外國人。隨後陸續的不平等條約使外國人大量來華投資,並且建立租界,加速對清貿易。

清朝自康熙至乾隆,雖然总体上對於佛教(藏传佛教除外)較為冷淡,不過對於穩定社會仍有一定的幫助。乾隆:「彼為僧為道,亦不過營生之術耳。窮老孤獨,多賴以存活。其勸善戒惡,化導愚頑,亦不無小補。」清朝皇帝多與僧人來往,順治帝先後與名僧憨璞性聰、玉林通琇、茚溪行森和木陳道沁等互相交流,順治本人曾削髮打算出家,他所寵愛的董鄂妃在他的影響下也棲心禪學。再如康熙在外出巡,每往名山巨剎,為之題字撰碑。雍正喜讀《金剛經》,也多與佛教徒往來,選編語錄,儼然以禪門宗匠自居。

不過佛教無限制的發展,對統治者也有不利之處,如果太多人民出家,政府徵稅的對象就會減少,寺院上層兼併土地,發展寺院經濟,就會加強土地集中的程度,激化社會矛盾,一些犯法的人,往往藏身寺廟作為躲避懲罰的手段,某些「聚眾為『匪』之案」,甚至「多由『奸邪』僧道主謀,「平時『煽惑』愚民,日漸釀成大案。」,因此清朝一方面保護佛教,另一方面又對之加以限制。清朝限制佛教的辦法主要有三種,設置僧官、實行度牒制度與不許擅造寺廟。佛教在限制下仍有一定影響力。佛教各派,除了禪宗還算盛行之外,其精神日趨世俗化,宗風也隨之衰落。另外如淨土宗、律宗、也僅能保持典型。乾隆時,曾禁止各地建立新寺院,民間出家為僧者也受限制。士大夫雖喜談佛學,但只是談論佛理,沒有興隆佛教的意願。

清朝為籠絡內外蒙古、青海、西藏外族,優禮和尊崇藏传佛教(亦称喇嘛教),順治八年,特於北京建造一大喇嘛廟,度喇嘛一百餘人,皆內府諸旗王公屬下滿州人。雍正帝曾得喇嘛之助繼位,之後以其潛邸改建為雍和宮大喇嘛廟,成為北京也是當時中國最大之喇嘛廟。雍正五年,又為蒙古之哲布尊丹巴呼圖克圖,建大喇嘛廟,發帑銀十萬兩之多。並尊哲布尊丹巴為喀爾喀大喇嘛,其地位與西藏之達賴、班禪鼎立為三,後世稱為「活佛」。乾隆帝曾經把藏传佛教作為解決現實社會矛盾的方法不過也以理藩院來控管其發展。藏传佛教雖然表面上受到君王的禮遇,不過事實上不像元明前兩代如此興旺,乾隆末年發生川楚教亂,使藏传佛教漸走衰落之途。

道教在宋朝最為盛行,之後元明兩代對其仍為優遇,到了清朝,帝室雖然也信奉道教,但不如前代之盛行,清朝對道教的政策與對待其他宗教一樣,既保護又加以控制。清朝的道教在明朝衰落的基礎上,進一步走向衰落。其主要原因如下:

滿族統治者對道教素無信仰,他們在入關前尊奉藏傳佛教格魯派,乾隆時又宣布黃教為國教。

清朝在思想上更加重視程朱理學,宣揚「忠」「孝」觀念,以束縛人們思想,麻痺人們鬥志。他們收羅一些理學家,如李光地等,纂修性理精義等書,頒行天下。而這些理學家則對釋、道兩教進行攻擊。

基督教傳入中國後,也使部分群眾改信其教。

民間秘密宗教的發展,這些民間宗教往往是正統道教的流衍,因此道教實際上慢慢演變成這些民間宗教。

衰落最根本原因是時代發生急劇的變化。商品經濟的發展,生產水平的提高,世界交流的頻繁,科學技術的傳入和進步,新文化、新思想的興起,這些衰落的表形,使其在政治上的地位日益下隆;另外道教教義學說陳陳相因,已停滯不前;道教上層人物日益腐化,失去群眾;道教也與儒佛兩教教日趨融合,使其自身的特點日益消退。

由於正統的道教逐漸衰落,促成民間宗教的崛起。其為道教的變種與流衍。因為他們被統治者視為「邪教」,而只能秘密傳播,但傳播的範圍卻很廣泛這些的民間宗教,名目不下數十百種。

民間宗教的思想內容,大量抄襲佛、道、儒等各家的教義,但也有不同之處。其中最多宣傳的東西,是關於彌勒等神佛下凡和劫變的觀念,以及關於「真空家鄉,無生父母」的信仰。民間宗教的所有這些宣傳,無疑是一種封建迷信,表達人民的不滿和抗議,它給人們以安慰和希望,在一定程度上反映人們要求改變現實願望,因此容易被人民所接受,可以成為組織,號召貧苦群眾參加起義的工作。有些民間宗教,還有明確的「反清復明」思想,如清茶門教宣傳,「清朝已盡,四正文佛落在王門;胡人盡,何人登基;日月復來屬大明,牛八元來是土星。」這些民間宗教如白蓮教與天地會,在清朝中葉以後,發動許多起義活動,如:川楚教亂、林爽文事件等等,對清朝國力造成很大的損傷。

從元朝開始,西域伊斯蘭教教徒大量來到中國各地,穆斯林居住於甘肅、陝西、四川、山西、直隸、廣東、雲南等省。清廷對伊斯蘭教採取放任的態度,尊重他們的信仰,用他們的法律來處理紛爭,但是比起元、明前兩代來說較為沒落,清朝防制回人的法律極嚴,內地各省,回人犯法,判罪較一般犯人為重,凡罪當流徒,一般人民可申請存留養親者,回民卻不得申請。在回疆,清朝每於重要所在分建漢、回二城,限制回民的自由,並禁止漢回通婚,在公文書中,則民、回別稱,表示將回民排斥於一般平民之外。

由於受到清朝及漢人的壓制,在清朝統治期間,清朝末年時,甘肅、陝西、雲南這三省曾發動叛亂。康熙順治年間,甘州回民丁國棟、米喇印起兵造反,乾隆年間的大小和卓之亂,嘉慶年間張格爾據浩罕叛,道光同治年間的陝甘回變與雲南回變一時俱起。

滿清入關後,湯若望、南懷仁等教士,先後被任命為欽天監主要官员,他們利用職務之便來傳教,雖然一度受到康熙曆獄的打擊,不過隨著康熙帝開始親政後,翻案成功重新執掌欽天監來繼續傳教,康熙帝對於天文曆算、火炮之學很有興趣,曾叫傳教士徐日昇、白晉等人輪流進講。並以他們擔任通譯及處理外交事務。如:徐日昇、張誠隨索額圖參加中俄尼布楚條約談判,充當翻譯和參謀。清朝對定居中原的西方傳教士採取禮遇態度。其中,順治帝特別倚重德國耶穌會士湯若望,並尊其為「瑪法」(滿語「爺爺」的意思)。在隨後的一百多年前,欽天監皆由耶穌會士掌管。

耶穌會傳教士對於中國原有的風俗習慣,抱持容忍態度。教徒祭天、祭祖、祭孔者盛行,雖然與教義互相衝突,但都以默忍。不過到了17世纪末至18世纪初,天主教内部发生“礼仪之争”。依照道明會傳教士的指控,罗马教宗下令禁止的传教士使用耶稣会的中文词汇“天”和“上帝” 来称呼天主,也禁止中国信徒祭拜祖先與孔子。这与当初意大利传教士利玛窦以及其后的传教士在中国传教时所采取的本土化政策截然相反。清政府对此十分不满,认为这样做有违中国敬孔祭祖的传统。康熙帝于1700年批示说:“敬孔敬祖为敬爱先人和先师的表示,并非宗教迷信”。

双方争持不下,最后清廷下令必須遵循「利玛窦规矩」传教,不然就不准傳教,逐出中國,是為「禁教令」。1722年,雍正帝徹底推行禁教令,使得清朝初年西方基督教在中国传教被终止,到了道光帝時,連欽天監也不任用傳教士。不過清朝皇帝對於教禁並沒有徹底禁止,嘉慶年間(1807年),新教教士英國人馬禮遜,曾藉工作之便私下在廣州進行傳教的工作。

鴉片戰爭後,清廷雖並未正式撤銷禁令,但基督教的傳教自由已經由不平等條約獲得確認,於是歐美各地的基督教教士在西方列強的庇護下進行宗教活動,基督宗教傳播更為迅速。除了傳教之外,設立醫院和學校,對於中國文化和社會的演進,發生巨大的影響。不過有些西方傳教士擁有種族優越感,以爭議手法傳教,如不理會傳統社會階級之分、強占土地(強行索回雍正禁教時遭沒收的土地房產)、袒護教徒干涉司法審判、下令教徒不得分攤並參與地方集體祭祀活動、直接要求北京政府撤換省級官員,甚至一些犯罪之人也藉由信教取得司法保護,引起中國人的反感,因此民教衝突不斷;加上民間以訛傳訛的負面謠傳,最終導致許多民眾則憤而紛紛起來焚燒教堂,驅逐或甚至殺害傳教士,收回被侵占的土地財物。從1856年至1889年先後發生的教案多達三百多起,著名的有1870年的天津教案,1900年的義和團之亂期間,有數萬民中西方傳教士與基督教信徒慘遭殺害。

清朝統治中原後,推行的漢化政策比其他征服王朝還要深,然而清室也盡可能保留本族文化,並且維持本身文化與漢文化的平衡 。清初以来,所有施政文书都以漢文、滿文两种文字发布。自康熙起大力推行以儒学为代表的汉文化,汉传统经典成为包括皇帝在内的满族人必修课。到乾隆中期,满人几乎全部以汉语为母语,满文渐渐成为仅用于官方历史记载用的纯书面文字。到19世纪,官方文件中的满文已基本为汉文所取代。然而儒家的一些思想清朝皇帝没有完全接受。

在18世紀康乾盛世期間,歐洲前往中國的傳教士們將當時中國圖景呈現給歐洲人,而後引發在17世紀末至18世紀末的100餘年間,甚至直到19世紀初,歐洲吹起中國風。無論是在物質、文化還是政治制度方面,歐洲都對中國極為追捧,以至於在1769年曾有歐洲人寫道:「中國比歐洲本身的某些地區還要知名。」對中國風的狂熱追逐曾經是當時歐洲社會的普遍時尚。這種時尚滲透到歐洲人生活的各個層面,如日用物品、家居裝飾、園林建築等。1735年,法國神父翻譯並發表法文版《趙氏孤兒》後,造成非常轟動的中國戲劇熱。西方對當時的中國也存在負面的聲音,認為中國朝廷過於獨裁與專權。乾隆末年英國派遣馬戛爾尼出使清朝,在佩雷菲特筆下的馬戛爾尼本人認為:「人民生活在最為卑鄙的暴政之下,生活在怕挨竹板的恐懼之中,所以人們膽怯、骯髒並且殘酷」,而馬戛爾尼本人的日記卻如此記載:「中國政府的行政機制和權力是如此的有組織和高效,有條件能夠迅即排除萬難,創造任何成就」。馬戛爾尼访华团成员之一的爱尼斯·安德逊卻如此評價:「杀头案在中国是非常少见的。关于这问题,我甚为注意而且好奇地去打听,一有机会就向人探问,我不只问过一个人,有好几个人,至少有70岁高龄的老人,他们从未见过或听到过有杀头处刑的事……比较轻的刑事案,在这人口非常多、商业又发达的国内也不常有」;「走过的乡村(北京郊区)前后每1英里路上的人数足以充塞我们英国最大的市镇」;「这城市(廣州城)的街道一般是15英尺到20英尺宽,用宽大的石板铺砌,房屋超出一层的很少,用木材和砖建筑。商店的正面大门之上有漂亮的阳台,因而门前形成一街檐,用各种油漆装修得很美丽」;“……这个马车队伍停歇在一个大市镇内,镇名“吉阳府”。说它是人口稠密,则我又用了这冗繁的语词,这语词可以同样应用于整个帝国,如每个村庄、市镇、城市;不,每一条河流和河流的两旁也充满了人。在这国家里,在我们所经过的地方,人口是极为众多而且是到处是那么多:我们走过的乡村前后每1英里路上的人数足以充塞我们英国最大的市镇,道路两旁不少别墅田庄散布在田野之间,大为增色,也足以证明其富裕”;「……不能不對這位偉大、顯赫、聰明、慈善的中國皇帝致以崇高敬意。他治理中國60年之久,按他的百姓的普通呼聲,他對他們的康樂與興旺從未忘懷。在他管理司法方面的情況是:他保護他的百姓中最低微的人」。

清朝學術興盛,文人學者對明朝以前各朝代的種種學術都加以鑽研、演繹而重加闡釋,集歷代之大成,梁啟超稱清朝為中國的「文藝復興時代」。鑑於晚明政治腐敗、內憂外患不斷,宋明理學流於空泛虛偽,致使清初學者多留心經世致用的學問。明朝亡於流寇、清朝定鼎中原後,一時學者痛定思痛,排斥空談心性的宋明理學與陽明學,推究各朝代治亂興衰的軌跡,提出種種改造政治與振興社會的方案,使清初學術思想呈現實用主義的風氣,發展出實事求是的考據學。

考據學又稱為「樸學」,強調客觀實踐,有疑問時求證,具有科學精神。考據學專研訓詁、音韻和校勘等。而其治學遠宗兩漢的經師,有異於宋明理學,故又稱為「漢學」。以顧炎武、黃宗羲、王夫之並為明末清初三大儒,與方以智、朱舜水等人並稱清初五大師,顏元也是這一時期的大師。顧炎武提倡「經學即理學」,提出以「實學」代替宋明理學,要學者直接研習六經。提倡“天下興亡,匹夫有責”,著有《日知錄》、《音學五書》等,其學說發展成乾嘉學派。黃宗羲有「中國思想啟蒙之父」之譽稱,著有《明儒學案》、《宋元學案》,是中國學術史之祖。他保護陽明學,排斥宋明理學,力主誠意慎獨之說,蔚為浙東學派。王夫之強調實際行動是知識的基礎,認為歷史發展具有規律性,是「理勢相成」。其思想發展成船山學,後人編為《船山遺書》。

以民為天下之主的思想於明末清初亦有所流行,例如生活在明末又经历清初时期的黃宗羲和顧炎武、王夫之提倡民權,所著的《明夷待訪錄》攻擊君主專制體制,提倡天下為主,君為客的觀點,倍受清末革命黨的推崇。部分學者認爲黃宗羲的思想是近代民主主義的思想,有西方學者稱黃宗羲為「中國自由主義的先驅」。清初思想家唐甄所著《潛書》描述:「清興五十年來,四海之內,日益困窮,農空、工空、市空、仕空(值得一提的是,清初五十年期間正處於氣候最為異常之時,當時世界上多處亦發生罕見的大饑荒)。」,並指出皇帝是一切罪惡的根源,認為「自秦以來,凡帝王者皆賊也。」

被當時正純學者(儒生)斥為「名教罪人」、「喪心病狂」、「人可戮而書可焚」的袁枚追求自由個性,反對專制思想和理學,亦貶斥漢學,他亦針對清廷統一人心風俗政策說「物之不齊,物之情也,天亦不能做主,而況於人乎?」,但袁枚從來沒被追究過「害義傷教」之罪,在當時亦生活得頗為順暢,名傾一世,令人羡慕。

清代中期的考據學崇尚研究歷史典籍,對中國歷史從天文地理到金石銘文無一不反覆考證。當時分成吳、皖兩派。吳派以惠棟父子、段玉裁、王引之與王念孙為主,以「博學好古」為宗旨,恪守儒家法則;皖派以戴震為首,以「實事求是」、「無徵不信」為宗旨。他們“毕注於名物训诂之考订,所成就亦超出前儒之上”。桐城派健將姚鼐提倡“義理、考據、詞章,三者不可偏廢。”道光與咸豐年間,曾國藩又把经济与义理、考据、词章并列。然而考據學到後來過分重視瑣碎事物的探究,為學問而學問,知古不知今。當時章學誠提出「六經皆史」,注重六經蘊含的義理,並使用於當代政治上,意圖矯正此歪風。鴉片戰爭後,西學大量流入中國,考據學逐漸式微。

明末清初,隨著歐洲耶穌會傳教士來華,西學輸入中國,對於當時的學風由浮虛轉為務實,也是有相當的激勵作用。他們將西方科技介紹給中國人,擴大其知識領域,使中國的學術思想添增不少新成分。

當時有才華的傳教士被皇帝欣賞和重用,西方先進的科學技術也被推崇和應用。而在民間,民人與西方傳教士能夠互相交遊,西學在社會中得以自由傳播,由康乾皇帝敕輯的叢書《古今圖書集成》和《四庫全書》亦收錄傳入中國的西方科學技術,當中《四庫全書》收錄了24種西方傳教士的著述。

康熙帝亦曾經委派傳教士閔明我(Domingo Fernández Navarrete)返回歐洲招募人才,希望增進中西方科技文化交流。《四庫全書總目》以及乾隆帝亦對西方技術作出較高的評價:「西洋之學,以測量步算為第一,而奇器次之。奇器之中,水法尤切於民用,視他器之徒矜工巧,為耳目之玩者又殊,固講水利者所必資也」;「歐羅巴人天文推算之密,工匠製作之巧,實逾前古」。

鴉片戰爭之後,大量西方科技與思想帶動中國近代化革新。此時學者如龔自珍、魏源與康有為等人繼承章學誠的說法,並進一步要求改革祖宗的法制,來應付內憂外患的局勢。龔自珍講求經世之務,志存改革,追求「更法」。魏源的《海國圖志》主張「師夷長技以制夷」,馮桂芬的《校邠廬抗議》主張「以中國之倫常名教為原本,輔以諸國富強之術」。康有為與梁啟超主張君主立憲。他們吸收來自西方的知識,先後推動自強運動與維新運動,這一波改革風潮最後引發清末新政與辛亥革命。

清朝文學多元發展,兼容並包歷代之文學特色。明朝以前的文學發展多表現在聲韻、格律、句法、結構的因襲或創變;清朝承接各代文學成果,先後形成許多學派,將各種在明朝以前已式微的文體重新復興,並繼明末進一步發展各類小說、戲曲;另外,因不同地區、民族互動而呈現出語言風格多樣化之文學面貌,於古體詩、近體詩、駢體文、散文、賦、詞、曲、小說、戲曲皆然。由於語言轉變較微妙,往往被人忽視,造成清朝文學缺乏明顯特徵與創造力的一般印象。整體而言,清代文學面向相當複雜多樣,但質量上也良莠不齊。

清朝前期出現風格率真、浪漫的小品文,以張岱、李漁與袁枚為主;又有侯方域、魏禧、汪琬合稱「清初散文三大家」。但是他們的文風不受道學學者支持,這些學者發起復興唐宋文風的古文運動,此即桐城派。創始人方苞與劉大櫆、姚鼐有「桐城三祖」之稱。姚鼐是桐城派的集大成者,他的古文主張,在提倡「義理(內容合理)、考據(材料確切)、詞章(文辭精美),三者不可偏廢。」講究義法,提倡義理,要求語言雅潔,反對俚俗。後來曾國藩發展成湘鄉派,惲敬、張惠言發展成陽湖派。

清朝的詩風甚盛,以帝王、宗室為首,官方大力提倡詩學,自清聖祖以後諸帝主導官修《御定全唐詩》、《御選唐詩》、《御選宋金元明四朝詩》、《御定全金詩》、《御定佩文齋詠物詩選》、《御定歷代題畫詩類》、《御選唐宋詩醇》、《欽定熙朝雅頌集》、《御定千叟宴詩》、《欽定千叟宴詩》、《欽定重舉千叟宴詩》、《上書房消寒詩錄》、《三元詩附三元喜宴詩》、《御定歷代賦彙》以及各代皇帝之《御製詩集》,如清高宗酷愛作詩,一生作《御製詩》五集,共計十餘萬首,每作一首詩便令詞臣注釋,若詞臣不得內容原委則准許其回家查閱典籍,多羅安郡王瑪爾渾選宗室王公詩為《宸萼集》。皇帝也將詩詠作為聯繫、攏絡官員的方式。

在清代,寫詩的女性越來越多,且詩的創作者皆來自各行各業。清代是一個文學收藏和批評的時期,許多現代流行版本的中國古典詩歌都是通過清朝詩集傳播的,如《全唐詩》和《唐詩三百首》等。

清初詩家首推錢謙益、吳偉業與王士禎;康熙中後期,江南地區出現王式丹、吳廷楨、宮鴻曆、徐昂發、錢名世、張大受、管棆、吳士玉、顧嗣立、李必恆、蔣廷錫、繆沅、王圖炳、徐永宣、郭元𨥤合稱「江左十五子」。乾隆時期袁枚、蔣士銓與趙翼並稱江左三大家,同時黃景仁與鄭板橋也以詩聞名。嘉慶、道光年間文人廣結詩社,京師與揚州風氣最盛,以消寒詩社最知名,代表人物有顧蓴、夏修恕、程恩澤、陶澍、朱珔、吳椿、梁章鉅、潘曾沂、胡承珙、李彥章、劉嗣綰、周之琦、林則徐、徐寶善、卓秉恬。被稱為「詩界革命」的詩歌改良運動產生於維新運動,其代表有黃遵憲的以写作反映时代的社会詩,其餘如譚嗣同、唐才常、康有為、蔣智由、丘逢甲、夏曾佑均有作品存世。於清末又發展出同光體,代表作家陳三立、陳衍、沈曾植等,且延續到辛亥革命後。清朝詩論學說分成沈德潛的格調說、王士禎的神韻說、袁枚的性靈說與翁方剛的肌理說。

詞興起於隋唐的「燕樂」,兩宋發展達高峰,至元朝衰微,延續至明朝則趨近消亡;清初詞學振興繁盛,康熙年間納蘭性德與朱彝尊、陳維崧並稱「清詞三大家」,隨後產生由陳維崧為代表的陽羨詞派、朱彝尊為代表的浙西詞派,詞學蔚為風潮。萬樹整理詞調輯成《詞律》,於清詞頗有影響力;康熙末,清聖祖敕命王奕清等編成《御定詞譜》,為詞調格律的集大成鉅作,影響層面最廣。乾隆、嘉慶朝,常州詞派起而代之,反對浙西詞派的「清空之弊」,代表人物有張惠言、張琦、惲敬、黃景仁、李兆洛、丁履恒、錢季重、陸繼輅、左輔、董士錫、周濟、劉嗣綰、劉逢祿、譚獻、莊棫、宋翔鳳、謝章鋌、馮煦、陳廷焯、王鵬運、鄭文焯、況周頤、朱祖謀等人,著名詞人輩出,持續到清末民初。清朝因此被稱為詞的「極盛時期」,「號稱詞學中興」,「作家之盛,直比兩宋」,門戶派別各具風采,婉約、豪放都各自重現、盛行。

清朝小說傑出者眾,曹雪芹等著《紅樓夢》不仅為四大名著之一,由于其对社会百态和众多人物全面精确的写实描绘和丰富的艺术魅力而被普遍认为是中国古典小说的巅峰之作。蒲松龄以志怪内容反映社会面貌的短篇小说集《聊齋誌異》。吴敬梓所著的虽结构松散但足称伟大讽刺小说的《儒林外史》;以及在《儒林外史》的影响下,以《老殘遊記》为代表的揭发官场丑态的譴責小說均有很大影響。

乾隆三十八年至四十九年(1773年-1784年)編纂的《四庫全書》是中國古代最大的一部官修書,也是中國古代最大的一部叢書,分經、史、子、集四部,故名四庫。據文津閣藏本,該書共收錄古籍三千五百零三(3503)種、七萬九千三百三十七(79337)卷、裝訂成三萬六千餘冊,保存豐富的文獻。「四庫」之名,源於初唐,初唐官方藏書分為經史子集四個書庫,號稱「四部庫書」,或「四庫之書」。經史子集四分法是古代圖書分類的主要方法,它基本上囊括古代所有圖書,故稱 「全書」。清代乾隆初年,學者周永年提出「儒藏說」,主張把儒家著作集中在一起,供人借閱。這是編纂《四庫全書》的社會基礎。《〈四庫全書〉總目提要》又是一部重要的目錄學著作。

《四庫全書》收錄27種西方傳教士的著述,包括西洋的數學、天文、儀器及機械等方面的著作,被收錄的書籍包括有《泰西水法》、《西儒耳目資》、《坤輿圖說》、《乾坤體義》等。

四庫全書編撰過程中,清朝統治者有诏令禁止鼓吹反政府以及违背儒家道德伦常的野史、笔记、文集等诸多书籍收入《四库全书》,并取締非議清朝統治者的著作,史稱四庫禁書。但四庫禁書中極少數遭到徹底焚毀,絕大部分是為官方藏禁。。史學家季羡林說,乾隆编修四庫全書的初衷虽是「寓禁于征」,但客观上整理、保存了一大批重要典籍,开创了中国书目学,确立了汉学在社会文化中的主导地位,具有无与伦比的文献价值、史料价值、文物价值与版本价值。史學家龚书铎亦說:“乾隆年间《四库全书》的编纂,为华夏文明的延续做出了不可磨灭的贡献,为后人保留了许许多多珍贵的书籍和资料;但是,在编纂过程中,也免不了发生许多令人遗憾的事情。”。鲁迅在批評明清以及民國文學時也對《四库全书》的编纂过程作出批評:“现在不说别的,单看雍正乾隆两朝的对于中国人著作的手段,就足够令人惊心动魄。全毁,抽毁,剜去之类也且不说,最阴险的是删改了古书的内容。乾隆朝的纂修《四库全书》,是许多人颂为一代之盛业的,但他们却不但捣乱了古书的格式,还修改了古人的文章;不但藏之内廷,还颁之文风较盛之处,使天下士子阅读,永不会觉得我们中国的作者里面,也曾经有过很有些骨气的人……清朝的考据家有人說過,“明人好刻古書而古書亡”,因為他們妄行校改。我以為這之后,則清人纂修《四庫全書》而古書亡,因為他們變亂舊式,刪改原文;今人(民國)標點古書而古書亡,因為他們亂點一通,佛頭著糞:這是古書的水火兵虫以外的三大厄。 ” 

清朝在編纂《四庫全書》時亦救亡不少大量早已失傳的中國古籍,梁啟超對此評論道:「此二百餘年間總可命為中國之『文藝復興時代』,特其興也,漸而非頓耳……吾輩尤有一事當感謝清儒者,曰輯佚。書籍經久必漸散亡,取各史藝文、經籍等志校其存佚易見也。膚蕪之作,存亡固無足輕重;名著失墜,則國民之遺產損焉。乾隆中修《四庫全書》,其書之采自《永樂大典》者以百計,實開輯佚之先聲。此後茲業日昌,自周秦諸子,漢人經注,魏晉六朝逸史逸集,苟有片語留存,無不搜羅最錄。其取材則唐宋間數種大類書,如《藝文類聚》、《初學記》、《太平御覽》等最多,而諸經註疏及他書,凡可搜者無不遍。當時學者從事此業者甚多,不備舉。而馬國翰之《玉函山房輯佚書》,分經史子三部,集所輯至數百種,他可推矣。遂使《漢志》諸書、《隋唐志》久稱已佚者,今乃累累現於吾輩之藏書目錄中,雖復片鱗碎羽,而受賜則既多矣。」

書法方面,晚明的帖學在清初仍然發達,姜英、張照、劉墉、王文治、梁同書與翁方綱等人在刻尊傳統的時候,力圖表現出新面貌,或以淡墨書寫,或改變章法結構等。但由於帖學未有很好地加以清理而逐漸頹勢。隨著金石考證學的發展,清朝書法多從碑體入手,成為清朝書壇的主流。有名的有翁方綱、劉墉、何紹基與趙之謙。到康有為大力張揚碑學,碑學作為一種與帖學相抗衡的書學系統而存在。清代的陶藝發展出繁複的不透明釉上彩陶器以及素色陶器兩種風格迥異的風格。

京劇被称为中國的“國粹”,起源於明朝的崑曲和京腔,形成於乾隆、嘉慶年間。京劇之名始見於清光緒二年(1876年)的《申報》,歷史上曾有皮黃、二黃、黃腔、京調、京戲、平劇、國劇等稱謂。乾隆五十五年(1790年)四大徽班進京後與北京劇壇的崑曲、漢劇、弋陽、亂彈等劇種經過五、六十年的融匯,衍變而成,是中國最大戲曲劇種。其劇目之豐富、表演藝術家之多、劇團之多、觀眾之多、影響之深均為全國之冠。京劇是綜合性表演藝術。集唱(歌唱)、念(念白)、做(表演)、打(武打)、舞(舞蹈)為一體、通過程式的表演手段敘演故事,刻劃人物,表達「喜、怒、哀、樂、驚、恐、悲」思想感情。角色可分為:生(男人)、旦(女人)、淨(男人)、丑(男、女皆有)四大行當。人物有忠奸之分,美醜之分、善惡之分。形象鮮明、栩栩如生。

清朝建築比前世變化不多,除了規模宏偉之外,作為中國建築特色之一的斗拱日趨虛飾纖麗,幾乎失去原來用途。北京紫禁城有許多大型色彩豐富的磚石建築。歷代帝陵無寢,自明太祖開始方有明孝陵。清朝分別建有位於遼寧瀋陽的盛京三陵、河北遵化的清東陵與河北易縣的清西陵。清代園林藝術以圓明園為代表,融合江南名園佳景與歐洲義大利樓房花園,被外國傳教士譽為「萬園之園」。清朝提倡藏傳佛教,分別於奉天、北京與五台山興建大喇嘛廟。康熙帝也於熱河承德興建仿西藏布达拉宫的承德避暑山莊,供遊獵避暑的住所。

清朝畫壇由文人畫佔主導地位,山水畫科和水墨寫意畫法盛行,更多畫家追求筆墨情趣。清代山水畫家有名的有「正統派」的四王(王時敏、王翬、王鑒和王原祁)、吳歷與惲壽平,合稱「清初六大家」。其中惲壽平創造不用墨線勾勒的沒骨花卉畫法,承自北宋徐崇嗣之沒骨法,又加入創意,蔚為清代花卉畫宗師,頗為後人所效仿,形成以惲壽平、鄒一桂為首的常州畫派。然而正統派的繪畫與元明兩朝相比,其水準水平一般,大致上總不脫臨摹的陋習。不過清初繪畫仍有翻新出奇、流于怪異之處,比如清初四僧的「遺民派」畫家(八大山人、石濤、漸江與髡殘)以及「金陵八家」的龔賢、樊圻、高岑、鄒結、吳宏、葉欣、胡慥和謝遜等人;雍乾之際以金農、鄭板橋為首的揚州八怪。清朝的宮廷畫院以義大利的郎世寧、最著名。受到西洋畫的影響,清宮廷中的畫家如焦秉貞、冷枚等人受西洋畫影響。清末時期,任伯年、吳昌碩、居廉的仕女花鳥畫及楊柳青、桃花塢和民間年畫如《蓮生貴子》、《魚躍龍門》等對後人也有很大影響。

整體而言,清朝的科學技術和同時代的西方國家相比較為落後,在晚期差距更為明顯。清初統治者和部分中國學者積極與西方傳教士進行中西文化技術交流,統治者亦會將習得的新知識實際地運用在治國方面,然而後來由於禁教的原因導致人們對西學日益模糊和隔閡。清代科學技術的落後是中國貧困積弱的重要原因之一,但在某些科學技術領域中比前也有明顯進展,尤其是在數學、天文、曆法等學科方面得到空前發展,而精於此道的學者也是英才輩出且人數眾多,從遠古到清朝這段期間,有關學者的數量在清代占約44\%。清代學者亦對古代技術和文獻的復興和修復也作出貢獻。

清軍入關後,湯若望、南懷仁等教士來華傳教,帶來西方科學與技術。他們先後被任命為欽天監。康熙帝對於天文曆算,火炮之學很有興趣,曾令白晉、德瑪諾等人,測繪全國地圖,歷時十年而成,康熙帝命名為《皇輿全覽圖》,它是中國第一部用經緯度測繪的地圖。順治帝多次向湯若望學習天文、曆法、宗教等知識,以及治國之策。不久湯若望成為「欽天監」的負責人,掌管國家天文历法事务。在隨後的一百多年前,「欽天監」皆由耶穌會士掌管。由於需要新的曆法,清政府遂下令根據湯若望所著的《西洋新法曆書》,制定新曆法並頒行全國,名為時憲曆。另一受西方影响较大的是地图测绘学。康雍乾時期,国家统一,版图巩固,始绘制全国和各地的地图,派人到各处实地测量。外国传教士雷孝思、杜德美和清朝学者何国宗、明安图等参加这项工作,采用西方经纬度定位和梯形投影法,所制地图居当时世界水平的前列。在測繪全國地圖的過程中,康熙時期編成的數學巨著《數理精蘊》亦發揮實際作用。

從順治到乾隆期間,西方傳教士在皇帝的要求下製造以及添加不少天文儀器如赤道經緯、黃道經經緯、簡平儀等,同時有相關天文學術著作出現,亦改善和編製較為先進的曆法。西方的物理學知識也從明朝末年起一直在中國傳播,康熙年間的學者戴震就写有关于阿基米德定律的作品,傳教士南懷仁著有《熙朝定案》等介紹各種工程技術的作品外,亦著有《驗氣圖說》和《形性理推》等對中國的物理學界產生一定影響的介紹西方光學知識的書籍,並在一定程度下讓一些學者啟迪進而研究光學,例如在康熙年間寫有《鏡史》一書的孫云球以及在十九世紀前期寫有《鏡鏡詅痴》的鄭復光;在對待西洋器物方面,清代有不少學者、匠師和科學家如黃履莊、黃履、孫玄球等對西方「奇器」有一定的研究和高度仿製,西方「奇器」在一定程度上亦推動中國的物理實驗和機械製造的發展。

由於清政府對天文曆法的重視,民間的天文學研究也很活躍,主要代表人物有著有《曉庵新法》和《五星行度解》的王錫闡等。

為平定三藩之亂,康熙特命比利時傳教士南懷仁製造適應南方地形特點和便於戰場上機動使用的火砲。南懷仁“依洋式鑄造新炮”,並進呈《神威圖說》一書,介紹西方的製炮理論和方法。在康熙十四年(1675)至康熙末年四十餘年間,僅中央政府就督造或改制神威無敵大將軍、金龍炮、制勝將軍、威遠將軍等各型火砲近千尊。不但數量多,而且種類也不少,乾隆二十一年(1756年)頒行的《欽定工部則例造火器式》載有各種火砲共85種,同年的《皇朝禮器圖式》中鳥槍,紅衣炮,子母炮這三樣火器成為製式武器。其他的著作分別有薛熙撰的《練閱火器陣紀》,沈善蒸撰的《火器真訣解證》,王達權、王韜同撰的《火器略說》,薛鳳祚撰的《中西火法》,陳暘撰的砲規圖說以及董祖修撰的《炮法撮要》等。清朝軍隊的火器裝備率也超過明朝,直到兩次鴉片戰爭期間和太平天國戰爭初期,清軍的主要火器是鳥槍和各種生鐵、青銅鑄造的火砲,在道光、咸豐年間又裝備了兩人抬用的抬炮和抬槍,清朝軍隊火器的裝備率達到50-60%。

清朝初期在政權穩固及穩定後容許漢族民眾持有冷兵器,但仍然警惕火器尤其是重型火器的流存。雖然清朝在法律上嚴禁民間私製火器,但在實行上卻沒有也難以嚴加管制,終清一代火器在民間幾乎隨處可見,製造和銷售火器在民間已有相當的規模,出於自衛、捕獵、遊戲等原因有不少平民都擁有了火器,甚至在京城也有鐵匠私造火器售賣。此外,清朝希望士紳掌管的民間武力協助國家維持基層社會的秩序和協助保衛國土,加上亦需顧及民間狩獵和自衛的需要,故清政府在一定條件下容許民間武器合法存在。中國歷史上的歷代王朝都不會樂見民間有大量武器,特別是不願看到有精利武器在民間流傳,但基於上述原因,清朝對民間火器的政策經常陷於「允許、鼓勵」與「禁止、控制」的兩難處境。

農業方面,清代有《授時通考》、《廣群芳譜》、《補農書》等著作,詳細論述各種作物的栽種和農業生產技術。清代亦是中國传统农学的高度发展时期,中国历代所编著的农书共714部,其中清以前的二千一百多年间编著的农书为231部,清代267年,编著的农书为483部,为清代以前农书数量的2.09倍,清朝治理及研究蝗蟲的技術亦相對比較發達。

清初至鼎盛時期,醫藥學進步所表現在很多方面,基本上是明朝醫藥盛況的延續。如對經典著作的研究、本草學、方劑學、診斷治療學、醫案整理等,均較明朝更成熟。各家學派的紛爭也逐漸緩和,大多醫家能採各家之長折衷於臨床。但也不乏固守《內經》、《難經》、《傷寒論》,而批評金元以後一切新說的醫學,這與當時考據學盛行不無關係。清朝中醫藥學最重要的成就,就是關於急性傳染病的研究,它已形成一個新的系統,即溫病學說。這一學說的出現,雖然是基於歷代醫家的有關成就上,但清代溫病學派在中醫發展史上的貢獻,仍然是相當顯著的,它並不亞於东汉張仲景著《傷寒論》,金朝劉完素創河間學派。

康熙年間有不少精通醫術的傳教士如張誠和白晉等向中國傳授西方醫學,並且被容許在入職朝廷、建立試驗室以傳授解剖學知識,康熙對此深感興趣,甚至曾經在一次研探解剖學時得病。法國傳教士巴多明(Dominicus Parrenin)用滿文翻譯人體解剖學方面的著作並名之為《欽定格體全錄》,巴多明和白晉也在康熙的支持下翻譯出有關人體血液循環的著作,並且在北京傳播相關知識。西方傳入的醫學知識和理論亦引起了當時中國醫學界人士的注意,例如清初劉獻廷研究過人物圖說等西方醫學著作,乾隆年間著有《醫林改錯》的醫學家王清任亦十分重視解剖學:「著書不明臟腑,豈不是痴人說夢,治病不明臟腑,何異於盲子夜行」。

清朝中葉後,西學的影響不像清初僅侷限於個別傳教士,西方科技的刺激顯然變得十分具有影響力。尤其是西方國家有意識地把醫藥作為實現他們宗教目的、掠奪目的的手段,所以西方醫學對中國的滲透變得比清初那時更為明顯。那時中國人民也有吸收外來醫藥學的需求,於是中西醫匯的主張應運而生。這種新的思想既有解放中醫藥學家保守思想的一面,也有壓抑對傳統中醫藥學繼承和發展的一面。

到了清末民初,由於西方醫學的強勢輸入,以其診斷、手術器械的先進、服藥的便利、診間及病房的明亮乾淨,對當時的傳統中醫造成了前所未有的生存與競爭的壓力。然而由於中國人根深蒂固的醫療習慣、中醫的實際療效以及西醫人數尚少不能普及於全國等原因,當時的中醫並未如日本明治維新之後的漢醫般幾乎絕跡。

中國古代的數學成就曾名列世界前茅,到明代衰落下來,古算幾成絕學。明末,西算傳入中國,從徐光啟翻譯《幾何原本》前六捲起,直到康熙時編成《數理精蘊》(《數理精蘊》是明末清初西算輸入時期一部帶有總結性的數學巨著,被稱之為當時中國最高水平的數學百科全書),這是中國歷史上第一次西算輸入時期,雍正以後到鴉片戰爭以前,又為「古算復興時期」。介紹西算和復興古算構成清前期數學發展的兩大內容。清初的曆法大辯論,新法以計算精確戰勝舊法,這件事使當時知識界對數學重視起來。康熙又聘請傳教士徐日昇、白晉、張誠、安多等入宮,講授幾何、代數、天文、物理等科學知識,這就推動數學的蓬勃發展,出現方中通、梅文鼎、梅轂成、明安圖、王元啟、董祐诚、項名達等著名數學家。

清朝學者亦重新發掘出在古代已長期失傳的大量數學著作,例如《四庫全書》的編纂工作者之一的戴震就從《永樂大典》中發現和整理出久已失傳的許多古典算書。如《海島算經》、《五經算術》、《周髀算經》、《九章算術》、《孫子算經》、《五曹算經》以及《夏侯陽算經》。他又從南宋刻本的毛扆影抄本中抄輯出《張丘建算經》和《輯古算經》兩種,連同明刻本的《數術記遺》共計十種。這十部算經於乾隆三十八年由孔繼涵刻入《微波榭叢書》,正式題名為《算經十書》。戴震還從《永樂大典》中抄輯出宋秦九韶的《數書九章》及楊輝的各種算書,令漢唐以來數學成就的結晶重現。

清朝的數學人才輩出且著作繁多,大約有五百人貢獻一千多種數學著作,超過以往任何一個朝代;但因受乾嘉漢學的影響,多集中在對古算的整理、註釋方面。在若干領域內,清代學者也作出創造性的貢獻。如陳世仁發展宋元以來垛積術的研究,即高階等差級數求和的方法;焦循註釋《九章算術》,提出加減乘除的交換律;還有汪萊和李銳繼承宋代天元術和四元術,發展方程論的研究,對方程根的性質以及根和係數的關係等進行探討。

建築學方面取得很高的成就,宮殿、園林、寺廟、宅宇、城垣的建築,盛極一時。或雄偉莊嚴,或富麗典雅,彩繪藻飾,光彩照人,庭院草木,錯落有致。著名匠師梁九、雷發達均有高超的設計和施工技藝。外國傳教士蔣友仁、王致誠等帶來西方的建築技術,設計圓明園內西洋樓、大水法等建築群。

清朝末年,中國的交通事業有所發展,詹天佑是中國第一位傑出的鐵路工程師,他主持修建的京張鐵路工程之艱巨是當時世界鐵路史上罕見的。詹天佑克服一道道難關,創造性地設計出「人」字形軌道,減緩坡度,降低造價,比原計劃提前兩年完工。詹天佑修建京張鐵路期間,厘定各種鐵路工程標準,並上書政府要求全國採用。中國現在仍然使用的4尺8寸半標準軌、鄭氏自動掛鈎等等都是出自詹天佑的提議。此外詹天佑亦著重鐵路人才的培訓,制定工程師升轉章程,對工程人員的考核和要求作出明文規定,並且定明工程師薪酬與考核成績掛鉤。京張鐵路培訓不少中國的工程人員,詹天佑所製定的考核章程亦成為其他中國鐵路的模仿對象。


%% -*- coding: utf-8 -*-
%% Time-stamp: <Chen Wang: 2018-07-12 22:07:49>

\section{后金\tiny(1616-1636)}

\subsection{努尔哈赤\tiny(1616-1626)}

\subsubsection{天命}

\begin{longtable}{|>{\centering\scriptsize}m{2em}|>{\centering\scriptsize}m{1.3em}|>{\centering}m{8.8em}|}
  % \caption{秦王政}\
  \toprule
  \SimHei \normalsize 年数 & \SimHei \scriptsize 公元 & \SimHei 大事件 \tabularnewline
  % \midrule
  \endfirsthead
  \toprule
  \SimHei \normalsize 年数 & \SimHei \scriptsize 公元 & \SimHei 大事件 \tabularnewline
  \midrule
  \endhead
  \midrule
  元年 & 1616 & \tabularnewline\hline
  二年 & 1617 & \tabularnewline\hline
  三年 & 1618 & \tabularnewline\hline
  四年 & 1619 & \tabularnewline\hline
  五年 & 1620 & \tabularnewline\hline
  六年 & 1621 & \tabularnewline\hline
  七年 & 1622 & \tabularnewline\hline
  八年 & 1623 & \tabularnewline\hline
  九年 & 1624 & \tabularnewline\hline
  十年 & 1625 & \tabularnewline\hline
  十一年 & 1626 & \tabularnewline
  \bottomrule
\end{longtable}

\subsection{皇太极\tiny(1626-1636)}

\subsubsection{天聪}

\begin{longtable}{|>{\centering\scriptsize}m{2em}|>{\centering\scriptsize}m{1.3em}|>{\centering}m{8.8em}|}
  % \caption{秦王政}\
  \toprule
  \SimHei \normalsize 年数 & \SimHei \scriptsize 公元 & \SimHei 大事件 \tabularnewline
  % \midrule
  \endfirsthead
  \toprule
  \SimHei \normalsize 年数 & \SimHei \scriptsize 公元 & \SimHei 大事件 \tabularnewline
  \midrule
  \endhead
  \midrule
  元年 & 1627 & \tabularnewline\hline
  二年 & 1628 & \tabularnewline\hline
  三年 & 1629 & \tabularnewline\hline
  四年 & 1630 & \tabularnewline\hline
  五年 & 1631 & \tabularnewline\hline
  六年 & 1632 & \tabularnewline\hline
  七年 & 1633 & \tabularnewline\hline
  八年 & 1634 & \tabularnewline\hline
  九年 & 1635 & \tabularnewline\hline
  十年 & 1636 & \tabularnewline
  \bottomrule
\end{longtable}


%%% Local Variables:
%%% mode: latex
%%% TeX-engine: xetex
%%% TeX-master: "../Main"
%%% End:

%% -*- coding: utf-8 -*-
%% Time-stamp: <Chen Wang: 2021-11-01 17:20:22>

\section{太宗皇太极\tiny(1626-1643)}

\subsubsection{生平}

1635年,皇太極打败林丹汗,令其遁逃至大草滩(今甘肃境),取得傳國玉璽(原為元朝所有)。漠南蒙古各部向後金臣服,為其上尊號博格达汗。后金的第二代大汗崇德元年四月十一乙酉日(1636年5月15日),皇太极改国号为“大清”,改元崇德,皇太极是1637年,皇太極率軍親自征討不服從後金統治的朝鮮,迫使朝鮮向其臣服;從此朝鮮成為清朝的藩屬。此後,朝鮮的親明派勢力被剷除,大清開始專心進攻明朝。

崇德六年即崇祯十四年(1641年)七月,帶病急援松錦之戰,史载“上行急,鼻衄不止,承以椀”,马不停蹄,昼夜兼行五百餘里。在松山大敗明軍,生俘洪承疇,並且令其投降,大大打擊了明軍的士氣。《清太宗实录》记载:“是役也,计斩杀敌众五万三千七百八十三,获马七千四百四十匹、骆驼六十六、甲胄九千三百四十六副。明兵自杏山,南至塔山,赴海死者甚众,所弃马匹、甲胄以数万计。海中浮尸漂荡,多如雁鹜。”此役为后来清朝灭明征服天下立下基础。《清史稿·太宗本纪》评价:“允文允武,内修政事,外勤讨伐,用兵如神,所向有功。”

崇德八年八月初九日(1643年9月21日)晚間十點皇太極崩逝於瀋陽故宮清寧宮東暖閣內,享年五十岁。安葬于沈阳清昭陵(今沈阳市北陵公园北)。由於死前未立繼承人,其弟睿親王多爾袞與長子豪格爭位不下,彼此陳兵示威。最終多爾袞獨排眾議,擁立莊妃的六歲兒子福臨,是為清世祖,改元順治。

後來順治帝諡皇太極為文皇帝,廟號太宗,統稱太宗文皇帝。

皇太極在一方面重用投奔後金的漢族官員為自己的智囊團;而在另一方面,皇太極多次强调国语骑射,是防止满洲人受到“服汉人衣冠,尽忘本国语言”薰染(《清太宗实录》 卷三四 崇德二年四月丁酉),危及满洲民族政权的长远存在;为此,皇太极反复告戒满洲贵族,应恪守满洲衣冠和善于骑射的风俗习惯云云,还多次下“上谕”强调这一点。

\subsection{崇德}

\begin{longtable}{|>{\centering\scriptsize}m{2em}|>{\centering\scriptsize}m{1.3em}|>{\centering}m{8.8em}|}
  % \caption{秦王政}\
  \toprule
  \SimHei \normalsize 年数 & \SimHei \scriptsize 公元 & \SimHei 大事件 \tabularnewline
  % \midrule
  \endfirsthead
  \toprule
  \SimHei \normalsize 年数 & \SimHei \scriptsize 公元 & \SimHei 大事件 \tabularnewline
  \midrule
  \endhead
  \midrule
  元年 & 1636 & \tabularnewline\hline
  二年 & 1637 & \tabularnewline\hline
  三年 & 1638 & \tabularnewline\hline
  四年 & 1639 & \tabularnewline\hline
  五年 & 1640 & \tabularnewline\hline
  六年 & 1641 & \tabularnewline\hline
  七年 & 1642 & \tabularnewline\hline
  八年 & 1643 & \tabularnewline
  \bottomrule
\end{longtable}


%%% Local Variables:
%%% mode: latex
%%% TeX-engine: xetex
%%% TeX-master: "../Main"
%%% End:

%% -*- coding: utf-8 -*-
%% Time-stamp: <Chen Wang: 2021-11-01 17:20:50>

\section{世祖順治帝福临\tiny(1643-1661)}

\subsection{生平}

順治帝(1638年3月15日-1661年2月5日),名福临,姓爱新觉罗氏,清朝第2位皇帝,清朝自入关以来的首位皇帝,1643年10月8日至1661年2月5日在位,在位18年。议政王大臣会议于1643年9月,推举五岁的福临承袭其父皇太极帝位,同时命努尔哈赤第十四子睿亲王多尔衮和努尔哈赤之侄郑亲王济尔哈朗二人助小皇帝辅理国政。

自1643年至1650年,政治权力主要掌握在多尔衮手里。在多尔衮的领导下,清朝征服明朝的大部分故土,深入西南省份追剿南明政权,在激烈的反对中,建立一系列被清代皇帝所沿袭的政策,如1645年颁布“剃发令”。多尔衮于1650年12月31日死后,13歲的顺治皇帝开始亲政。顺治皇帝试图打击腐败,整顿吏治,削弱满洲贵族的政治影响力,但最终结果成败参半。在位期間,顺治帝面临着大明遗民的复明抵抗,不过至1661年,清军已将大清帝国最后的对手,南明遺臣郑成功和永历皇帝朱由榔击败,郑成功和朱由榔分别于次年病死和被擒杀。顺治皇帝在22岁时因感染高度流行的天花去世,其皇位由已从天花中幸免于难的皇三子玄烨承袭,后者即康熙帝,在位24年。由于顺治年间的历史文献流传相对较少,加上史書為突顯康熙帝的功績,因此这段时期同整个清朝历史相比显得较为鲜为人知。

顺治帝死后受供奉于太庙,庙号「世祖」,谥号「体天隆运定统建极英睿钦文显武大德弘功至仁纯孝章皇帝」,统称世祖章皇帝,葬于清东陵的孝陵。


14世纪,数支女真部落生活在大明(1368年–1644年)东北疆域,即现代被称为中国东北或“满洲”的地区明太祖時,為壓抑北元殘餘勢力,於是東北設立遠東指揮使司,控制女真部的各個部落。

明成祖永樂年間(1403年-1424年),在东北疆域置奴兒干都司等衛所管理當地,其中建州女真一部最為強大,明政府先後將建州女真分成三個衛,總稱「建州三衛」,其後,建州女真首领努尔哈赤(1559年–1626年)经过三十余年的征抚,完成對女真各部的統一。

努尔哈赤最重要的一项改革,是将松散的女真诸部的力量凝聚在黄、白、红、蓝四色旗之下,此后,又在原有四色旗基础上再增镶黄、镶白、镶红、镶蓝四旗,形成八旗。此社会军事组织制度是为八旗制度。努尔哈赤将旗主交由子侄担任。在1612年左右,努尔哈赤為使其部族人与其他支觉罗部族人相区別,及與曾统治中国北方的女真王朝大金(1115年–1234年)扯上關係,故將部族名變更為爱新觉罗氏(意為“黄金般高贵神圣的觉罗一族”)。

1616年,努尔哈赤宣布叛明自立國號,史稱后金,建元天命。爾後,努爾哈赤繼而攻打原大明領土的辽东大多数主要地區,其軍隊所向披靡,直到1626年1月,努尔哈赤在宁远攻城之时,被驻守该地的明军指挥官袁崇焕,以不久前收购的葡萄牙人的红夷大炮击败。努尔哈赤可能在宁远之战中受了致命伤,因而在战后数月逝世。

努尔哈赤之子皇太极(1592年–1643年)继续致力于其父的大业:他把权力集于自己之手,仿效大明政治制度,并完善和拓展八旗制度,在原有满洲八旗的基础上增设蒙古八旗和汉军八旗。1629年,他率军入侵北京郊区,在此期间俘获了知道如何铸造红夷大炮的汉人工匠。1635年,皇太极改称女真为“滿洲”,1636年,他又将国号“后金”改为“大清”。在松锦之战后的1643年,明朝已经在财政破产、瘟疫肆虐以及大饥荒导致的明末农民战争等致命危机之中摇摇欲坠,大清准备展开对大明的最后一击。

清世祖福临出生于1638年3月15日。

崇德七年 (1642年) 十二月初二日,皇太極率諸王貝勒及文武大臣行獵於葉赫地方。同月十二日,到達噶哈嶺。聖汗之五歲幼子方喀拉章京射殺一狍。學者楊珍在《順治朝滿文檔案札記》認為方喀拉即為福臨的原名或乳名,章京即為方喀拉在此次隨皇太極行獵時,臨時得到的職位。

1643年9月21日,生前未指定儲君的皇太极殯天,雏鹰般的大清面临着可能出现的严重分裂危机。数名皇位争夺者——努尔哈赤的次子兼在世的长子和硕礼亲王代善、努尔哈赤第十四子和硕睿亲王多尔衮和第十五子和硕豫亲王多铎(两人为同母所出)以及皇太极之长子和硕肃亲王豪格——开始逐鹿皇位。皇太极的弟弟多铎、多罗武英郡王阿济格及多尔衮(31岁)掌有正白及镶白旗,代善(60岁)掌有两红旗,而豪格(34岁)则获得其父两黄旗的支持。

议政王大臣会议着手议立新帝,此會議直到军机处在18世纪20年代出现以前一直是满清的主要决策机构。许多亲王、贝勒主张多尔衮这个久经考验的军事将领成为新皇帝,但多尔衮拒绝为帝,而是坚持让皇太极的一个儿子承袭父位。

会议接受多尔衮的具有权势的主张,继续让皇太极的后裔继承大统。此时除豪格外,皇太极的儿子中,尚有叶布舒、硕塞、高塞、常舒、韬塞、福临、博穆博果尔七人。其中叶布舒、高塞、常舒、韬塞四人中,有三人年长于福临,但皆生母地位低微,无法越过福临、博穆博果尔继承皇位。而硕塞的生母叶赫那拉氏则早被皇太极赐给大臣,博穆博果尔则年幼于福临。最终商议决定立皇太极第九子福临承袭父位为新皇帝,但亦决定立和硕郑亲王济尔哈朗(努尔哈赤之侄,他掌有镶蓝旗)和多尔衮作这个五岁孩子的摄政。1643年10月8日,福临正式登上大清皇帝位;定年号为“顺治”。由于记载顺治年间的文献语焉不详,所以这段时期同整个清朝历史相比显得较为鲜为人知。

济尔哈朗是一位骁勇善战、受人尊敬的将领,但看起来对多尔衮已很快就抓到手中的日常行政事务毫无兴趣。1644年2月17日,济尔哈朗召集内三院、六部、都察院和理藩院的官员,向他们宣布:“嗣后,凡各衙办理事务或有应白于我二王者,或有记档者,皆先启知睿亲王档子,书名亦宜先书睿亲王名,其坐立班次及行礼仪,注俱照前例行。”此后在同年5月6日,豪格暗中动摇摄政统治的阴谋暴露。豪格的党羽全部被处死,豪格本人被褫夺亲王爵位。多尔衮在此后不久,以自己的支持者接替取代了豪格的拥护者(大多来自黄旗),从而掌控了两白旗以外的旗。至1644年6月初期,他已牢牢地把清政府及其军政大权掌握在自己手中。

1644年初期,正当多尔衮与其谋士苦思钻研如何攻大明之时,民變逼近北京。同年4月24日,民變领袖李自成攻破明都城墙,促使崇祯皇帝朱由检在紫禁城后的万岁山歪脖树上自缢身亡。多尔衮的汉人谋士洪承畴和范文程闻讯,敦促滿洲亲王抓住此机遇,给大明报仇雪恨,进而为大清夺取天命。驻扎在长城东端山海关的大明总兵吴三桂,是多尔衮同北京之间的最后一道障碍。此时他正被满洲人与李自成军间的武力夹得左右为难,吴三桂请求多尔衮帮助他驱逐土匪,恢复大明。当多尔衮要求吴三桂替大清效力之时,吴三桂除了接受之外别无选择。清兵因此得到了吴三桂的精兵的辅助,后同李自成军进行一片石之战,在多尔衮最终选择用骑兵介入此战斗前,吴三桂的精兵就已和李自成军交战了数小时。5月27日,大清取得此战的决定性胜利。战败的李自成军在北京洗劫数日,直至6月4日携带着所能带走的财物离京。

6月5日,被叛军之手肆虐了六周的北京市民,派出了一批士绅及官吏迎接他们将要来到的解放者。可当他们见到的是骑着马、把前额头发剃光并自称摄政王的满洲人多尔衮,而不是大明皇太子朱慈烺及其保护者平西伯吴三桂时,吃了一大惊。在此场动乱之中,多尔衮将自己安置在武英殿,后者是李自成在6月3日火烧大内后,唯一未被损坏的建筑。旗军们被命令不许抢劫;他们的纪律约束使统治过渡到大清“出奇地顺利”。然而在同时,多尔衮却声称他是为报复大明而来。他下令将大明皇族(包括大明末代皇帝朱由检的后裔)及其拥护者全部处决。

6月7日,进城仅两天的多尔衮向首都的官员发布谕告。该谕告向官员们保证,如果本地居民剃发易服并且接受归降,那么他们则可以官复旧职。可是在此谕告发布后的三周内,北京爆发数场农民起义,威胁大清控制首都地区。面对威胁,多尔衮不得不将此谕告废除。

1644年10月19日,多尔衮在北京大门迎接福临。10月30日,六岁的福临被带到北京南郊的天坛祭拜天地。11月8日,福临的登基仪式正式举行。同日,年幼的皇帝将多尔衮的功绩同周公进行比较,后者为古时一个受人尊敬的摄政。在登基仪式上,多尔衮的官衔由“摄政王”升为“叔父摄政王”。满语“叔父”(ecike)在此表示高于亲王的一级身份。三天后,多尔衮的摄政同事济尔哈朗的官衔由“摄政王”降为“辅政叔王”。多尔衮在1645年6月发布仪注规定,今后所有公文均应书写“皇叔父摄政王”称呼他,这使得多尔衮距离皇帝权威仅剩一步之遥。最终多尔衮在1648年更凌驾于小皇帝之上,称“皇父摄政王”。

多尔衮进入大清新首都后的最初的一个命令是,将北京北部全部腾出,然后把它分给旗人。两黄旗分得荣耀的宫殿北部,其次,东部为两白旗,西部为两红旗,南部为两蓝旗。八旗的此种布局,是为了使京城同满洲在征服中原前的故乡保持一致。此种布局“按照罗盘的指针指向,给颜色不同的旗人分配在一个固定的地理位置。”尽管大清为了加快过渡而减免税收,推迟大型建筑建造计划。但到了1648年,新来的旗人与共同生活的汉人百姓间仍有敌意。而首都以外的农业用地则全部被清军圈占。昔日的地主,现在却成了给外居旗人地主支付租金的佃户。这种土地用途的转变导致了“数十年的中断和苦难。”

在1646年,多尔衮还下令重建选任政府官员的科举考试。从那时起,他们效仿大明,每三年定期举行一次科举。同年,大清统治下的第一次殿试举行,大多数报考者为北方汉人,他们被提问如何使满汉同心合志。1649年,考试询问“联满汉为一体,使之同心合力,欢然无间,何道而可?”在1660年确定减少中额前,顺治朝下每届会试的考中人数的平均为大清最高(“得到了汉人更多的支持”)。

晚清的一幅描绘1645年5月扬州十日的版画。多尔衮的弟弟多铎为镇慑南方不服的汉人而进行了这场大屠杀,画上总共有有九位死者。十九世纪晚期,这场大屠杀被反清复明革命者用以激发汉族人群的反满情绪。
晚清的一幅描绘1645年5月扬州十日的版画。多尔衮的弟弟多铎为镇慑南方不服的汉人而进行了这场大屠杀,画上总共有有九位死者。十九世纪晚期,这场大屠杀被反清复明革命者用以激发汉族人群的反满情绪。
多尔衮被历史学家不同地称为“大清征服的优秀策划者”和“满洲洪业的首席建筑师”,大清在他的统治下,征服了中原大部分地区,并将“南明”的势力范围推到了遥远的中国西南地区。李自成从北京逃到西安,并在后者重建指挥部。多尔衮在同年夏、秋将河北、山东抗清起义镇压后,派遣军队进入西安(陕西省)主要城市搜寻李自成。1645年2月,在清军的压力下,李自成被迫离开了西安。他被杀了——无论是死于自己之手,还是被当地村民疑以为劫盗而误杀——1645年9月后,他在几个省份中消失了。

1644年6月,福王朱由崧于长江中下游以南的江南富饶的商农区建立大明弘光政权。1645年4月初,大清从新占领的西安出发,准备向那里发起进攻,南明政权的党派之争和不计其数的逃叛,阻碍了其有效抵抗能力的增强。1645年5月初,数支清军席卷南方,随手夺取了徐州淮河以北的主要城市。此后不久,他们向南明北部防线的主要城市——扬州——拥去。史可法面对包围,勇敢地反抗。5月20日,遭受一周炮轰的扬州被满洲人攻破,史可法依旧拒绝投降。多尔衮的弟弟多铎遂下令屠杀扬州全城人民。作为目的,这场大屠杀作为恐吓江南其他城市降服于大清。紧接着南京在6月16日,即最后的防卫者使多铎保证不会伤人后,钱谦益开城而降。大清在不久俘获了大明皇帝(他在翌年被处决于北京),并迅速夺取了江南包括苏州杭州的主要城市;至1645年7月初,大清与南明之间的边界被推到南方的钱塘江。

江南刚有了表面上的平静后,多尔衮便在1645年7月21日发布了一个最不合时宜的告示,他命令所有的成年男人剃去他们前额的头发,将他们的头发按照满洲人的髡髮辫式编扎起来。不服从告示者将被处以死刑。对于满洲人来讲,此象征着屈服的政策,有助于他们分清敌我。不过,在汉人官员和文人看来,新发型是一种奇耻大辱(因为它有悖于孔门弟子关于保持身体完整的指导)。而对于普通百姓来说,剃发如同丧失他们的生殖能力(英语:virility)。由于剃发令逼使社会的各个阶层的汉人联合起来反抗大清统治,所以极大地阻碍了大清的征服。在1645年8月24日和9月22日,前明将领李成栋分别对嘉定和松江反抗的人民进行屠杀。而江阴还同约一万名清军进行了八十三天的对抗。当城门最终在1645年10月9日被攻破时,降清明将刘良佐对全城人进行屠杀,这场屠杀造成了七万四千至十万不等的人的死亡。这些大屠杀结束了长江中下游的反清武装抵抗。有几个忠诚的勤王者成了隐士,并希望着清军败溃。虽然他们退出了世界,但至少象征着在继续反抗外族统治。

南京沦陷后,两支明宗室建立了两个新的南明政权:一个是以福建沿岸附近为中心隆武皇帝唐王朱聿键——明太祖朱元璋的九世孙——而另一个是浙江附近的“监国”鲁王朱以海。但由于雙方彼此不服,無法聯合抗清,不但無法反攻滿清,也導致喪失維持政權的機會,造成漢人政權走向衰亡。1646年7月,贝勒博洛领导的新的南方军事活动使鲁王的浙江朝廷陷入混乱状态,继而向隆武政权发起进攻。朱聿键于10月6日在汀州(福建西部)被俘,即刻处死。他的养子国姓爷郑成功则随他的船队逃往台湾岛。11月,江西剩余的忠明抵抗中心崩溃,整个江西降清。

1646年末,广州出现了两个新的大明皇帝:一个是年号为绍武的朱聿键之弟唐王朱聿𨮁,另一个为年号为永历的桂王朱由榔。由于朝服不够,此后绍武政权所任命的官员不得不向本地伶人购买戏袍。两支南明政权彼此残杀,直到1647年1月20日,李成栋率领的一支小规模清兵组成的先头部队开进广州,处死了朱聿𨮁,迫使永历朝廷逃往广西南宁。然而,李成栋于1648年5月起兵抗清,与江西的前明将领金声桓并发起义,帮助朱由榔夺回了中国南方的绝大部分地区。但南明的复兴只是昙花一现。清军于1649年和1650年重新征服湖广中部(今河北和湖南)、江西和广东。朱由榔再度逃亡。最后,1650年11月24日,尚可喜所统率的清军攻占广州,杀死七万多人。

同时,1646年10月,豪格(福临长兄,于1643年继承斗争中失去继承权)所统率的清军抵达四川,任务是摧毁张献忠领导的大西国。1647年2月2日,张献忠与清军在川中西充附近作战时被杀。1646年末抗清势力进一步向北蔓延,由一个穆斯林将领米喇印领导的武装力量反抗大清对甘州(甘肃)的统治。另一名穆斯林丁国栋很快加入了他的抗清运动。他们以恢复大明为号召,攻克了甘肃的数个城镇,其中包括省会兰州在内。这些起义者愿意同非穆斯林的汉人进行合作,这表明他们不是仅仅被宗教所驱使。1648年,米喇印战死于水泉(今甘肃永昌水泉子村),丁国栋则被孟乔芳俘获并被多尔衮下令处决,至1650年,造成了大量人员伤亡的穆斯林起义运动被粉碎。

1650年12月31日,多尔衮在狩猎途中意外死亡,引发了一段激烈的派系斗争,开辟了深层次政治改革之路。由于多尔衮的支持者在朝廷上仍具影响,所以多尔衮的丧礼依帝礼,多尔衮死后获追尊为皇帝,谥号懋德修道广业定功安民立政诚敬义皇帝,庙号成宗。然而,在1651年1月中旬的同一天,多尔衮的前部将吴拜统率下的数名白旗军官为防范多尔衮的胞兄阿济格自立为新摄政而将其逮捕;之后,吴拜让福临任命自己及他的几位追随者为各部尚书,准备接管政府。

同时,于1647年被褫夺摄政头衔的济尔哈朗,获得了对多尔衮统治心怀不满的旗官的支持。济尔哈朗为了巩固直属皇帝的两黄旗(前两旗自清太宗开始直属皇帝)对自己的支持,争取白旗支持者,赋予正黄、镶黄、正白三旗一个新名称:上三旗(此三旗自此由皇帝直接统辖)。于1661年成为玄烨的辅政大臣的鳌拜和苏克萨哈,是给予济尔哈朗支持的旗官,济尔哈朗以指定他们参加议政王大臣会议作为回报。

1651年2月1日,济尔哈朗宣布即将13岁的福临亲政。摄政正式废止。济尔哈朗此后展开攻势。1651年3月12日,他控告多尔衮僭越皇权:多尔衮被判有罪,他获得的追尊被剥夺。济尔哈朗继续肃清多尔衮集团前成员,为上三旗中越来越多的支持者升官晋爵,所以到了1652年,多尔衮的前支持者或是被杀,或是被有效的从政府中清除。

諭吏部:“邇來有司貪污成習,皆因總督、巡撫不能倡率,日甚一日。國家紀綱,首重廉吏。若任意妄為,不思愛養百姓,致令失所,殊違朕心。總督、巡撫任大責重,全在舉劾得當,使有司知所勸懲。今所舉者多屬冒濫,所劾者以微員塞責,大貪大惡,每多徇縱,何禆民生?何補吏治?爾部須秉公詳察奏聞,如有此等惡習,定當從重治罪不貸。部院堂官係各司楷模,尤當正身潔操砥礪自愛,殫心盡職,以不負朕惓惓用人求治之意。其京堂大小員缺,亦著選擇有才望堪用者,不得循資挨轉。以後內外官,各宜洗心滌慮,勤守職業,不得仍蹈前弊,自取罪戾!” —— 《大清世祖章皇帝實錄》卷五十四

福临仅仅亲政两个月后,便于1651年4月7日发布谕告,宣布他将肃清官场腐败。该谕告引起文人间的派系之争,令福临沮丧无比,至死也无可奈何。福临的最初的一项行动是罢免大学士冯铨。冯铨为北方汉人,先前曾于1645年受弹劾,但摄政王多尔衮仍准其任职如故。福临以陈名夏取代冯铨。陈名夏是个有影响力的南方汉人,同南方文人集团关系良好。陈名夏尽管曾于1651年受控以权谋私,但旋于1653年官复原职,旋即成为皇上的亲密的私人顾问。陈名夏甚至获准可以像昔日的明代内阁大学士那样起草诏书。同于1653年,福临决定召回声名狼藉的冯铨。皇帝如此行事,本意是想让南北汉人官员在朝廷上势均力敌,从而平息派系冲突。然而,冯铨回归后,派系之争反而激化,令皇帝始料未及。在1653年和1654年的数次朝议中,南方人形成反对北方人与满洲人的阵营。1654年4月,陈名夏向北方汉人官员宁完我建议,清廷应恢复明代衣冠,宁完我旋即向皇帝揭发此事,并指控陈名夏干犯有包括贪污受贿、裙带关系、结党营私和僭越皇权在内的各种罪行。1654年4月27日,陈名夏被绞死。

1657年11月,北京顺天省试的一场重大作弊丑闻爆出。八名江南考生贿赂了京城的主考官,希望能得到更高的名次。七名主考官以受贿的罪名被处以死刑,数百人被判处贬谪流放和没收财产。这场丑闻很快蔓延到了南京会试,揭露了官僚制的腐败和以权谋私,许多坚持正统观念的北人官员将之归因为南方文人小团体的存在和经典学问的衰落。

福临在他短暂的统治期间,鼓励汉人入仕,恢复了许多多尔衮摄政期间废止或排斥的中原王朝制度。他和大学士(诸如陈名夏,见上文)谈论历史、经典和政治,他周围聚集了一批新人,诸如能讲一口流利满语的北方年轻汉人王熙。福临于1652年颁布的《六谕》是玄烨1670年颁布的《圣谕》的前身,后者是一部“正统儒家思想的梗概”,用于指示百姓遵守孝道和法律。顺治帝用中原王朝的一些体制改革清朝制度,于1658年恢复了翰林院和内阁。这两个机构承袭明代模式,进一步削弱满洲贵族的权力,这使得深深困扰晚明的党争问题死灰复燃成为可能。

为了削弱内务府和满洲贵族的权力,1653年7月,福临设立十三衙门,后者虽由满洲人监督,但由汉族宦官而非满洲包衣阿哈掌控。宦官在多尔衮摄政期间受严格的限制,但小皇帝用他们来制衡像皇太后和皇叔济尔哈朗这样的实权派人物的影响。至1650年代后期,宦官的权力变大:他们处理关键的政治和经济问题,就官员任命提出建议,甚至负责起草诏令。由于宦官削弱了官僚集团与皇帝间的联系,满汉官员担心困扰晚明的宦官擅权局面会重现。尽管皇帝尝试限制宦官权力,他最宠爱的宦官吴良辅还是于1658年陷入腐败丑闻,吴良辅于1650年代早期帮助他肃清多尔衮集团。但吴良辅收受贿赂仅仅受到谴责,未能平息宦官权力膨胀引发的满洲贵族的怒火。。福临死后不久,1661年3月,鳌拜和另外三位辅政大臣将十三衙门裁撤,吴良辅被处决。

1646年,博洛率清军进入福州,发现来自琉球国和安南的使节和马尼拉的西班牙人。这些朝贡使团前来拜见已倒台的南明隆武皇帝朱聿键,而后者此时已被押送至京,最终,这些使者听从清廷命令辞归。最后残存的南明抵抗势力从与安南接壤的云南撤离后,琉球王尚质于1649年首次向大清派出朝贡使团,暹罗和安南分别于1652年和1661年向大清派遣朝贡使团。

同于1646年,统治吐鲁番的一名莫卧儿王公苏丹阿布·穆罕默德·海基汗派遣一支使团,请求恢复因明亡而中断的与华贸易。使节团虽未受邀请便来到中国,但大清准其请求,允许其在北京和兰州进行朝贡贸易。但该协议因1646年一场席卷中国西北的穆斯林起义(参见前文“征服中国”末段)而中断。大清与资助反政府武装的哈密和吐鲁番的朝贡贸易最终于1656年恢复。不过在1655年,清廷宣布来自吐鲁番的朝贡使节每五年才能接受一次回赐。

1651年,小皇帝邀请藏传佛教格鲁派领袖第五世达赖喇嘛访问北京,后者不久以前在蒙古和硕特部首领固始汗的军事帮助下,成为西藏的宗教统治者和世俗统治者。尽管满洲对藏传佛教的支持和保护至少始于弩尔哈齐治下的1621年,但此次邀请背后仍有政治原因。即西藏正在成为大清西部一个强大的政治实体,达赖喇嘛对蒙古部落具有影响力,而其中一些蒙古部落并未屈从于大清。为了迎接这位“活佛”的到来,福临下令在紫禁城西北边北海琼华岛的昆仑山上建造了一座白塔,其位置就在以前薛禅汗宫殿的遗址上。经过多次邀请和外交往来,西藏领袖拿定主意,接受会见大清皇帝,1653年1月14日,达赖喇嘛抵达北京。达赖喇嘛日后将此行访问的场面雕刻在拉萨的布达拉宫,后者于1645年开始建造。

与此同时,在满洲人故乡北部,探险家瓦西里·波亚尔科夫(1643–1646)和叶罗菲·哈巴罗夫(1649–1653)越过罗刹国的山谷来到了黑龙江流域。1653年,莫斯科召回哈巴罗夫,委派奥努夫里·斯捷潘诺夫接替他,斯捷潘诺夫掌握了哈巴罗夫的哥薩克军队指挥权。斯捷潘诺夫南下进入松花江,强迫当地原住居民诸如达斡尔人和久切尔人交纳“牙薩克”(毛皮税),但遭到抗拒。因为满洲当地民族已向顺治皇帝朝贡。1654年,斯捷潘诺夫击败从宁古塔被派遣去调查罗刹计划的小规模的满洲军队。1655年,另一名清军指挥官蒙古人明安达礼在黑龙江流域的呼玛要塞击败斯捷潘诺夫军,但这还不足以追捕罗刹人。不过在1658年,满洲将领沙尔虎达率四十余艘船向斯捷潘诺夫发起进攻,罗刹人大多数被击毙或生俘。经过此役,黑龙江流域哥萨克地带已无太大冲突,但大清和罗刹的边境冲突则持续了下去,直至1689年《尼布楚条约》签订,固定了罗刹和大清之间的边界。

尽管大清在多尔衮的领导下成功将南明推到华南,但大明遗民尚未死心。1652年8月初,正在保护朱由榔的张献忠前部下李定国,从大清手中夺回桂林。一月之内,广西清将大多向南明投降。此后两年,尽管对湖广和广东的军事行动偶尔成功,但李定国未能夺取重要城市。1653年,清廷命洪承畴负责夺回西南地区。洪承畴驻扎长沙,耐心地建立起自己的军力;惟在1658年底,营养充足、物资供应良好的清军分多路向桂州和云南进军。1659年1月末,铎尼率清军攻陷云南府,朱由榔逃入邻近的缅甸,后者此时正由東吁王朝国王莽平德勒统治。此后南明末代皇帝一直留在缅甸,直到1662年被1644年4月降满的前明将领吴三桂俘获并处决。

郑成功在1646年成為明绍宗朱聿键義子,賜姓朱,故稱國姓爺,1655年由明昭宗朱由榔封为延平王,亦是他继续捍卫南明的原因。1659年,正当福临准备举行一场特殊的考试来庆祝他辉煌的统治和西南战役的胜利时,郑成功率领全副武装的船队驶向长江,从大清手中夺取了几座城市,进而围攻江寧(今江蘇南京)。当郑成功围攻江寧的消息传入皇帝耳中时,他就大发雷霆,据说一怒之下用剑劈了宝座。但南京的威胁最终解除,郑成功被清兵击退,被迫求助于东南沿海的福建省。迫于清军的压力,郑成功于1661年4月攻擊由荷蘭東印度公司佔領的台湾島,并死于同年夏天。他的子孙依然自稱為延平王,继续在台灣反抗大清统治,直至1683年顺治帝之子康熙帝派遣降將施琅佔領该岛。

順治帝于1651年亲政后,他的母亲昭圣慈寿皇太后安排儿子娶她的侄女额尔德尼布木巴,但福临废黜第一任皇后。次年,昭圣慈寿皇太后另为儿子安排了一场同蒙古科尔沁部的婚姻,这次她将自己的侄孙女阿拉坦琪琪格嫁给福临。尽管福临同样不喜欢他的第二任皇后(后世习以谥号称之为孝惠章皇后),但未能废黜皇后,皇后也未有生育。约在1656年,福临开始宠幸董鄂妃,据說当时的耶稣会记述,董鄂妃是一位满洲军人的妻子。她于1657年生下一子(皇四子)。皇帝想立他为继承人,但这个孩子未及命名便于1658年初夭折。

順治帝是位开明的皇帝,不仅在天文学和科技问题上,而且在处理国事和宗教问题时都向一位来自神圣罗马帝国科隆的日耳曼耶稣会教士汤若望请教。1644年末,多尔衮为制定一部尽可能精确的新历法而任用汤若望,因为他的日蚀预报比那些清廷天文学家的预报更精确。多尔衮死后,汤若望同小皇帝建立了私人友谊,福临用满语称他为“爷爷”。在他们关系最亲密的1656年和1657年,福临常常驾临他的府中,和他交谈到深夜。他被免除叩头礼,在北京获得建造教堂的土地,甚至被允许收养一个儿子(因为福临担心汤若望没有继承人),但自1657年以后,福临开始崇信佛教禅宗,汤若望试图使清帝信仰天主教的努力最终未能成功。

順治帝亲政后,发愤学习,熟练地掌握了汉语,能够欣赏中国艺术如书法和戏曲。反清知识分子顾炎武和万寿祺的一位密友归庄所作《万古愁曲》是福临最喜欢的文章之一。福临“极富感情,重情钟情,至其极处”,他还能成段的引用背诵援引《西厢记》。

大清皇帝自順治帝开始以「中國」自居,並且在對外条约和外交文件中称清为“中国”。1689年,也就是康熙二十八年,中俄尼布楚条约上第一次在国际法的层面上确立了“中国”的概念。

順治帝最宠爱的妃子皇贵妃董鄂氏因丧子之痛,于1660年9月猝死。福临为此悲痛欲绝,沮丧数月,直至他于1661年2月2日染上天花。1661年2月4日,福临急召礼部侍郎兼翰林院掌院学士王熙(福临的知己)和原内阁学士麻勒吉到自己身边,口述遗诏。同日,7岁的皇三子玄烨可能因为从天花中幸存下来而获立为皇太子。皇帝于1661年2月5日崩于紫禁城内的养心殿,年僅二十二岁。

满族人对天花病毒没有免疫,一旦感染天花,几乎只能等死,所以他们对天花的恐惧甚于其他任何疾病。1622年,他们建立一个机构,用于研究天花病例,隔离患者避免传染。在天花流行之时,皇室成员为保护自己免受感染,定期进入避痘所。福临之所以感染如此可怕的疾病,是因为他年轻,而且居住于附近有传染源的大城市。而事实上,根据记载,在顺治年间,至少有九次天花在北京爆发,每次爆发,都迫使福临搬到保护区。保护区为北京南部的狩猎场南苑,此前多尔衮已于17世纪40年代在那里建立一所避痘所。尽管有这样的预防措施——例如规定迫使感染天花的汉族居民搬出城市——但順治最終仍死於天花。

“奉天承运皇帝诏曰:朕以涼德承嗣丕基,十八年於茲矣。自親政以來,紀綱法度,用人行政,不能仰法太祖、太宗謨烈,因循悠忽,苟且目前,且漸習漢俗,於淳樸舊制,日有更張,以致國治未臻,民生未遂,是朕之罪一也。朕自弱齡即遇皇考太宗皇帝上賓,教訓撫養,惟聖母皇太后慈育是依,隆恩罔極,高厚莫酬。惟朝夕趨承,冀盡孝養,今不幸子道不終,誠悃未遂,是朕之罪一也。皇考賓天時,朕止六歲,不能服衰絰行三年喪,終天抱恨惟侍奉皇太后順志承顏,且冀萬年之後,庶盡子職,少抒前憾,今永違膝下,反上厪聖母哀痛,是朕之罪一也。宗室諸王、貝勒等,皆係太祖、太宗子孫,為國藩翰,理宜優遇,以示展親。朕於諸王貝勒等,晉接既疏,恩惠復鮮,以致情誼暌隔,友愛之道未周,是朕之罪一也。滿洲諸臣,或歷世竭忠,或累年効力,宜加倚託,盡厥猷為,朕不能信任,有才莫展。且明季失國,多由偏用文臣,朕不以為戒,而委任漢官,即部院印信,間亦令漢官掌管,以致滿臣無心任事,精力懈弛,是朕之罪一也。朕夙性好高,不能虛己延納,於用人之際,務求其德與己相侔,未能隨材器使,以致每歎乏人,若舍短錄長,則人有微技,亦獲見用,豈遂至於舉世無材,是朕之罪一也。設官分職,惟德是用,進退黜陟,不可忽視。朕於廷臣中,有明知其不肖,不即罷斥,仍復優容姑息如劉正宗者,偏私躁忌,朕已洞悉於心,乃容其久任政地,誠可謂見賢而不能舉,見不肖而不能退,是朕之罪一也。國用浩繁,兵餉不足,而金花錢糧,盡給宮中之費,未嘗節省發施,及度支告匱,每令會議,諸王大臣,未能別有奇策,祇議裁減俸祿,以贍軍餉,厚己薄人,益上損下,是朕之罪一也。經營殿宇,造作器具,務極精工,求為前代後人之所不及,無益之地,糜費甚多,乃不自省察,罔體民艱,是朕之罪一也。端敬皇后於皇太后克盡孝道,輔佐朕躬,內政聿修。朕仰奉慈綸,追念賢淑,喪祭典禮過從優厚,不能以禮止情,諸事踰濫不經,是朕之罪一也。 祖宗創業未嘗任用中官,且明朝亡國亦因委用宦寺。朕明知其弊,不以為戒,設立內十三衙門,委用任使與明無異,以致營私作弊,更踰往時,是朕之罪一也。朕性耽閒靜,常圖安逸,燕處深宮,御朝絕少,以致與廷臣接見稀疏,上下情誼否塞,是朕之罪一也。人之行事孰能無過?在朕日御萬幾豈能一無違錯?惟肯聽言納諫則有過必知。朕每自恃聰明,不能聽言納諫。古云:‘良賈深藏若虛,君子盛德容貌若愚。’朕於斯言大相違背,以致臣工緘默,不肯盡言,是朕之罪一也。朕既知有過,每日剋責生悔,乃徒尚虛文,未能省改,以致過端日積,愆戾愈多,是朕之罪一也。太祖、太宗創垂基業,所關至重,元良儲嗣,不可久虛。朕子玄烨,佟氏妃所生,年八歲,岐嶷穎慧,克承宗祧,茲立為皇太子,即遵典制持服二十七日,釋服,即皇帝位。特命內大臣索尼、蘇克薩哈、遏必隆、鰲拜為輔臣,伊等皆勳舊重臣,朕以腹心寄託,其勉矢忠藎,保翊冲主,佐理政務。布告中外,咸使聞知。”—— 《清世祖遗诏》

2月5日夜间,順治帝的遗诏颁示天下,特命索尼、苏克萨哈、遏必隆和鳌拜四人为了他年幼的儿子的辅政大臣,此四人都曾于多尔衮死后帮助济尔哈朗肃清朝廷上的多尔衮势力。很难确定福临是否确实任命四位满洲贵族为辅政大臣,因为福临的遗诏显然被昭圣慈寿皇太后和此四人所篡改。順治帝在遗诏中表示,他在施政之中偏向任用汉族大臣而且疏远了满洲官员(自己过分信用宦官,袒护汉官),忽视了满洲亲贵和满洲传统,对皇贵妃的精神投入超过了对自己的母亲。尽管福临在位时经常发布罪己诏,但这份遗诏中所谴责的政策自他亲政以来对清政府至关重要。被称为鳌拜辅政的1661年末至1669年间,该遗诏给了四位辅政大臣“皇权外披”,使他们的亲满政策得到支持。

由于朝廷没有明确宣布順治帝的死因,很快便流言四起。坊间传言福临其实未死,而是因为对爱妃之死过于悲痛或是四位获任为辅政大臣的满洲贵族发动了政变,他退位隐居佛教寺院,匿名为僧。因为順治帝于17世纪50年代成了佛教禅宗的狂热追随者,甚至让僧人进入皇宫,这些流言似乎不那么令人难以置信。中国现代历史学家认为福临出家之谜是清初三大疑案之一。但一位僧人记录说1661年2月初皇帝因感染天花而健康严重受损,而在皇帝的葬礼上有一名妃子和一名侍卫为其殉葬,由此来看福临之死应该並非假象。

福临的遗体被安放在紫禁城,受到为时27天的哀悼,1661年3月3日,一支规模宏大的行进列队将福临的遗体运送至景山(紫禁城北部的一个小丘), 之后大量贵重物品在葬礼上被烧掉。距离葬礼仅两年后的1663年,福临的遗体被运到他最后的安息之地。与当时的满洲习俗相同,福临的遗体在火化后安葬。他的骨灰安葬在北京东北方的昌瑞山,后来通常称为清东陵。他的陵墓孝陵是建在那里的第一座陵墓。

以順治帝的名义公布的遗诏表示,他对自己放弃满洲传统深表歉意,这一表示赋予了四辅政大臣实行本土主义政策的权力。鳌拜和其他三位辅政大臣援引遗诏,迅速革除了十三衙门。在此后的几年里,他们提升了满洲人及其包衣阿哈掌管的内务府的权力,革除翰林院,规定只有满洲人和蒙古人才能参加议政王大臣会议。辅政大臣还向大清治下的汉人推行强硬政策:他们发动文字狱处决了江南富庶地区的十余人,并以拖欠税收的罪名对该地区的数千人处以刑罚;他们强迫东南沿海地区人口从该地迁出,以便截断郑成功的子孙统治的台湾东宁王国的粮食供给。

玄烨于1669年设法囚禁鳌拜后,撤销了辅政大臣的许多政策。他恢复了父亲所青睐的机构,包括使汉族官员在政府中获得重要发言权的内阁。他还平定了三藩之乱。内战(1673年–1681年)使清人的忠心一度受到考验,但清军最终占得上风。当胜利成为定局时,1679年玄烨为吸引前明遗臣出仕清廷,而举行了特别考试博学鸿儒科。中试者被邀请参与编写官修《明史》。叛乱于1681年被平定,同年,玄烨开始倡导使用人痘接种为皇家儿童预防天花。郑氏家族在台湾建立的的东宁王国于1683年倒台后,清政权完成了统一天下的事业。在多尔衮、福临和玄烨奠定的体制基础上,清朝成为一个疆域辽阔、文化灿烂的强大帝国,被誉为“世界上最成功的帝国之一”。然而具有讽刺意义的是,正是康熙皇帝的赫赫武功带来的长时间的“满洲和平”,使大清面对19世纪列强武装侵略之时毫无准备。

\subsection{顺治}

\begin{longtable}{|>{\centering\scriptsize}m{2em}|>{\centering\scriptsize}m{1.3em}|>{\centering}m{8.8em}|}
  % \caption{秦王政}\
  \toprule
  \SimHei \normalsize 年数 & \SimHei \scriptsize 公元 & \SimHei 大事件 \tabularnewline
  % \midrule
  \endfirsthead
  \toprule
  \SimHei \normalsize 年数 & \SimHei \scriptsize 公元 & \SimHei 大事件 \tabularnewline
  \midrule
  \endhead
  \midrule
  元年 & 1644 & \tabularnewline\hline
  二年 & 1645 & \tabularnewline\hline
  三年 & 1646 & \tabularnewline\hline
  四年 & 1647 & \tabularnewline\hline
  五年 & 1648 & \tabularnewline\hline
  六年 & 1649 & \tabularnewline\hline
  七年 & 1650 & \tabularnewline\hline
  八年 & 1651 & \tabularnewline\hline
  九年 & 1652 & \tabularnewline\hline
  十年 & 1653 & \tabularnewline\hline
  十一年 & 1654 & \tabularnewline\hline
  十二年 & 1655 & \tabularnewline\hline
  十三年 & 1656 & \tabularnewline\hline
  十四年 & 1657 & \tabularnewline\hline
  十五年 & 1658 & \tabularnewline\hline
  十六年 & 1659 & \tabularnewline\hline
  十七年 & 1660 & \tabularnewline\hline
  十八年 & 1661 & \tabularnewline
  \bottomrule
\end{longtable}


%%% Local Variables:
%%% mode: latex
%%% TeX-engine: xetex
%%% TeX-master: "../Main"
%%% End:

%% -*- coding: utf-8 -*-
%% Time-stamp: <Chen Wang: 2021-11-01 17:21:13>

\section{圣祖康熙帝玄烨\tiny(1661-1722)}

\subsection{生平}

康熙帝(1654年5月4日-1722年12月20日),名玄烨,爱新觉罗氏,清朝第3位皇帝,清朝自入关以来的第2位皇帝,1661年2月5日至1722年12月20日在位,年号「康熙」。

康熙帝于順治十一年農曆甲午年三月十八巳時生於北京紫禁城景仁宫。康熙帝幼年继位,朝政不得不交付给辅政大臣。少年时期的康熙帝在智擒权臣鳌拜后,开始勤政。其在位期间,注意缓和阶级矛盾,采取轻徭薄赋与民生息的农业政策,重视农耕,发展经济,改革税收,疏通漕运。同时还对三藩、明郑、噶尔丹等各地反清势力大规模用兵,对沙俄签订条约确保黑龙江流域和广大东北地区的控制,实现清朝的国土完整和统一。康熙帝努力调节满族与汉、蒙、藏等族的关系,尊崇儒学,开博学鸿儒科笼络汉族士大夫;实行“多伦会盟”安抚蒙古各部,下令编修《理藩院则例》,确定巩固边疆的统治方针;册封五世班禅为“班禅额尔德尼”,派兵入藏驱逐入侵西藏的准噶尔汗国。还开海设关,发展内外贸易,重用海外传教士,学习西方近代科学。此间,使中国社会出现“天下粗安,四海承平”相对稳定的局面,为开启百余年的康雍乾盛世奠定了夯实基础。

但是,晚年的康熙帝沉浸于前半生的丰功伟业之中,不再锐意进取,开始倦于政务,标榜仁政而放松对吏治的治理,甚至出現吏治废弛、败坏的现象,从而暴露出许多社会问题,而废太子事件造成的夺嫡之争也对清朝政治产生了不良影响。

康熙六十一年十一月十三崩于北京畅春园清溪书屋,终年68岁。死后庙号圣祖、谥号合天弘運文武睿哲恭儉寬裕孝敬誠信功德大成仁皇帝,通称聖祖仁皇帝,葬于清东陵中的景陵。康熙帝在位六十一年零十个月,是中國歷史上在位時間最長的皇帝。

順治十一年三月十八日(1654年5月4日),玄燁出生於紫禁城景仁宮內,是順治帝的第三子,母親為孝康章皇后佟佳氏。順治帝病篤前沒有冊立過皇太子(祖父皇太極生前亦不預先冊立皇太子)。顺治十八年正月初六(1661年2月4日),顺治帝早逝,时年仅23岁。

順治帝染上瘟疫天花傳染病第3天時,接受湯若望的建議,因幼年玄燁曾出過天花具有免疫力,以口述遺詔的形式立玄燁為皇太子。顺治十八年正月初七(1661年2月5日)玄燁登基時,只有八歲,次年正月(1662年2月)改元:康熙。因康熙皇帝尚年幼,順治的遗诏同时指派四大臣辅政大臣索尼、苏克萨哈、遏必隆、鰲拜,輔治康熙皇帝。

康熙六年(1667年)六月,首辅索尼病故。七月初七(8月25日),十四岁的康熙帝正式亲政,在太和殿受贺,赦天下。但亲政仅十天后,鰲拜即擅杀同为辅政大臣的苏克萨哈,数天后与遏必隆一起进位一等公,实际政局并不受康熙帝直接掌控。

少年的康熙在挫败了政治对手鳌拜之后亲政。随即便宣布停止圈地,放宽垦荒地的免税年限。他还着手整顿吏治,恢复了京察、大计等考核制度。为了防止被臣下蒙蔽欺骗,康熙还亲自出京巡视,了解民情吏治。其中最著名的是六次南巡,此外还有三次东巡、一次西巡,以及数百次巡查京畿和蒙古,此举极大的促进了康熙对民情的了解,他还亲自巡视黄河河道,督察河工,并下令整修永定河河道。

康熙是清朝历史上在位时间最长的皇帝(後代的乾隆帝因崇敬康熙而刻意禪讓)。康熙坐镇北京取得了对三藩、沙俄的战争胜利,消滅在台湾的明鄭政权,另一方面,康熙创立“多伦会盟”取代战争,联络蒙古各部;意图以条约确保清朝政府在黑龙江的领土控制。文治武功取得巨大成绩的康熙帝,群臣一再商议给他上尊号,康熙多次表示“断不受此虚名”,这在历朝帝王中十分罕见。

康熙晚年懈怠无为,曾说“多一事不如少一事”,“政宽事省”,“凡事不可深究者极多”,不能严禁浮费和规银,宽纵州县火耗和亏空。同时他还標榜仁政,對官吏盡量以寬鬆待之,導致出現吏治废弛,官場贪污,国库亏空,“大小官员,怠玩成习,徇庇尤甚”,个别地区出现暴动和骚乱,统治秩序奏出了不和谐音符。盛世处于衰微的現象,给继任者雍正帝留下许多隐患。更有甚者指出清朝衰亡,病在康熙。

康熙四十九年(1710年),御史参劾户部堂官希福纳等侵贪户部内仓银六十四万余两,牽連的官吏多達一百一十二人。康熙说“朕反复思之,终夜不寐,若将伊等审问,获罪之人甚多矣”。最後只把希福納革職,其餘官吏則勒限賠款。康熙末年社会矛盾日趋激化,有江苏无锡县人劉三因县令李牧残酷成性,聚數百人於山中反抗,後被捕。

康熙的皇太子两立两废,彻底暴露出嫡长子皇位继承制度的种种弊端,储位之争的时间之长,卷入者之多,波及面之广,以及对皇朝及皇帝本人影响之大,无不超出前代。

康熙六十一年十一月十三日(1722年12月20日),康熙皇帝崩逝于大清順天府(今北京市)暢春園清溪书屋內,享壽六十八岁,結束了長達六十一年的統治。当时八爷党支持的十四阿哥胤禵远在西北,四阿哥胤禛留京。康熙近臣步军统领隆科多奉康熙帝遺詔,命皇四子胤禛继承皇位,是为雍正皇帝,为康熙帝上庙号圣祖,谥号合天弘运文武睿哲恭俭宽裕孝敬诚信功德大成仁皇帝,安葬于清景陵。

康熙十三年(1674年),康熙立皇后所生的一歲的皇次子胤礽為太子,並親自撫養。但數十年後由於太子本身的質素問題及其在朝中結黨而決定廢嫡。廢太子後,眾皇子覬覦皇位,矛盾更加尖銳,故太子廢而復立,但康熙仍無法容忍其結黨,三年後再廢太子。康熙六十一年臨終時決定傳位給皇四子胤禛。

目前理由眾說紛紜:有人認為康熙是希望精明幹練的胤禛能大力改革康熙末年的寬縱積弊,也有人認為康熙是因為鍾愛胤禛之子弘曆(未來的乾隆帝)而傳位於胤禛。還有傳說是顧命大臣隆科多和胤禛矯篡遺詔,在十字上加一劃、下加一勾,「十」字變成「于」字,故有「傳位十四皇子胤禵」竄改為「傳位于四皇子胤禛」之傳說;但按清宮祕檔分析,康熙帝的遺詔是由滿、漢、蒙三種語文並列寫成,「傳位十四皇子胤禵」改為「傳位于四皇子胤禛」之傳說符合漢字書寫邏輯,但卻無法符合滿文及蒙文書寫邏輯,且遺詔全文並未出現「傳位于」之類的語句。

然則傳位奪嫡之說,或因雍正推行攤丁入畝、官紳一體當差納糧之新政、打擊貪腐權貴、重用張廷玉、李衛、田文鏡等漢人,而引來失勢滿人權貴之蓄意誣陷。康熙皇帝豈能將九門提督授予不可信賴之人任之,又豈會不知隆科多與雍正之關係而造成眾皇子傳位紛爭?由此而論,康熙讓隆科多任九門提督,正是意欲傳位於雍親王,並加以保護的實證之一。

康熙傳位雍正之徵兆:徵兆一:「康熙六十年正月,命皇四子雍親王胤禛、皇十二子貝子胤祹、世子弘晟以御極六十年,告祭永陵、福陵、昭陵。」康熙登基一甲子六十年之重大祭告先祖非同一般,派遣雍親王胤禛主持,豈能不具備重大意義?為何不是派遣支持皇十四子胤禵、皇八子胤禩、皇九子胤禟、皇十子胤䄉或是皇三子胤祉。徵兆二:康熙御極六十年派雍親王胤禛祭祖此舉,讓廢太子胤礽之師王掞看出端倪,故於三月「大學士王掞密奏請建儲,至是監察御史陶彝、任坪、范長發等人曾疏請建儲,帝不悅,並掞切責之。諸王、大臣奏請治大學士王掞罪,帝赦不治。」這亦可視為康熙安排接班人的佈署跡象之一,畢竟皇十四子胤禵尚且領兵在西北,一旦提早公佈,易生事端。徵兆三:「五月壬戌,命撫遠大將軍胤禎移駐甘州。以年羹堯總督四川陝西,色爾圖署四川巡撫。」康熙以皇四子雍親王胤禛之親信年羹堯箝制皇十四子胤禵的軍後補給已然成形。徵兆四:康熙六十一年四月,「命撫遠大將軍胤禎復往軍前。十月,命雍親王胤禛率弘昇、延信、孫渣濟、隆科多、查弼納、吳爾台察閱京師通州倉廒。」康熙指示由雍親王胤禛親率隆科多、查弼納等眾多京師王公重臣,竟然只為「察閱京師通州倉廒」,已有不尋常跡象。徵兆五:「十一月帝不豫,駐蹕暢春園。命皇四子胤禛恭代祀天。」康熙駕崩前祀天仍然未派皇三子胤祉、皇八子胤禩、皇九子胤禟、皇十子胤䄉代祀,更未召皇十四子胤禵返京,此時康熙意欲傳位於雍親王皇四子胤禛已然十分明顯。

曾在國立故宮博物院展出的康熙皇帝遺詔上並無「傳位于四皇子胤禛」,而是寫著:“雍親王皇四子胤禛,人品貴重,深肖朕躬,必能克承大統,著繼朕登基即皇帝位,即遵典制持服。二十七日釋服,佈告中外,咸使聞知。”

康熙八年(1669年),康熙帝时常召集小內監在宫中作「布庫」之戏,不过在五月十六日(6月14日)鰲拜进见时,突然下令以大不敬之罪,命少年們将其逮捕。大臣商议鳌拜大罪三十条,请求將他滅族,康熙帝念鳌拜曾救過祖父皇太極的功劳,赦其死罪,改為拘禁,但诛杀了鳌拜的很多弟侄亲随及党羽。仅存的另一辅政大臣遏必隆因为长期勾结鳌拜,被削去太师、一等公。康熙帝由此完全奪回朝廷大權,開始真正親政的階段。

康熙勤政,坚持每日御临乾清门会见朝臣处理政务,居住在畅春园、热河行宫以及在出巡途中仍听政不惜。黎明时分,部院大臣,起居注官员到位,各部院衙门依次奏事,皇帝与内阁大臣商决裁断。《起居注》中详细记载了康熙皇帝御门听政现场办公的场景内容。康熙帝晚年还通过赵凤诏贪污案来抑制汉官。

1677年,康熙帝開始了整治黃河工程。到1684年,歷時七年的整治黃河工程完成。在康熙五十六年(1717年),出現各地豐收,無災可免的情況。康熙在晚年亦繼續減免天下賦稅,蠲免全國各地省份的錢糧,免除多處地區的欠賦。多種措施令到各地的農民都能夠休養生息,也防止了地方官吏中飽私囊和橫徵暴斂。

康熙帝為了箝制反清復明的活動而致力於打敗明鄭王朝。拿下臺灣之后,康熙开放了海禁,并设立了四个通商口岸。

1673年,因为康熙帝决定削藩,导致平西王吴三桂起兵反清,其他二藩相繼響應,整个天下为之一动。三藩势力一时不可阻挡,清廷失去江南半壁江山。而康熙帝在孝庄太后的支持下,沉着应对,积极调兵遣将,三藩之亂最終在1681年被完全撲滅,而国家遭受了较大的损失,在四川、云南以及江西等地有不少人被殺害。

1683年(康熙二十二年),时宪历五月,康熙採納了安溪大學士李光地的意見,授明鄭降將施琅為福建水师提督,时宪历八月丙辰,福建水师提督施琅攻克台湾,郑克塽和刘国轩等投降。

康熙年間,由於戰爭連年不絕,平定三藩之亂以及抵禦外來侵略的需要大量製造火器,無論是造炮規模、數量、種類,還是火砲的性能和製造技術,都達到了前所未有的水平。同時,清朝所造的大小銅、鐵炮達905門之多,而其中半數以上由南懷仁負責設計監造,就質量而言,其「工藝之精湛,造型之美觀,炮體之堅固,均為後朝所莫及」。康熙三十五年(1696年),在對準噶爾部噶爾丹的昭莫多之戰中,發揮了重要作用。

清朝初年一時間湧現出許多熱心武器裝備、致力於引進和仿造西方火器的技術專家。如戴梓就是一位在中國最早製造出具有較高射擊速度的管形火器專家,這種火器稱為“連珠火銃”。戴梓仿鑄技術比南懷仁更為高超,亦成功地仿造了沖天炮“南懷仁謂沖天炮出其國,造之一年不成。上命先生造,八日成,上大悅,率群臣親試之,即封炮為威遠將軍,鐫治法官名,以示不朽。沖天炮,子在母腹,母送子出,從天而下,片片碎裂,銳不可當。後征噶爾靼,以三砲墜其營,遂大捷”。文獻記載的“連珠火銃”與故宮所藏的一支康熙年間外國進獻的火槍十分相似,然而在因为冲天炮事件中得罪了南怀仁,被诬陷“私通东洋”,康熙将戴梓流放到了盛京(今沈阳)。

乌兰布通之战后,康熙帝更加重视在战争中发挥火器的战斗威力,使火器营成为清军八旗兵的新的战斗编成。清军最早装备火器的是汉军八旗,随着战事频繁,满洲、蒙古八旗亦迅速装备了火器。至康熙二十二年,在每旗专设一营操练鸟枪。康熙三十年始选满洲、蒙古习火器之兵组建火器营。设鸟枪护军、鸟枪马甲和炮甲三种营兵,满洲、蒙古八旗每佐领下设鸟枪护军3人,鸟枪马甲4人,炮甲1人,共7395人。由於西方經典彈道理論在戰鬥人員中逐漸普及,火器命中率的提高,極大地提高了火力武器的殺傷力。因此,火器在康熙以後不僅成為八旗的主要武器裝備,而且清軍還產生了更專門的火器營的戰鬥編成,完全改變了清軍以騎射為主的傳統作戰方式。

康熙崇尚儒学,尤其是程朱理学。他曾多次举办博学鸿儒科,创建了南书房制度,并亲临曲阜拜谒孔庙。康熙还组织编辑与出版了《康熙字典》、《古今图书集成》、《曆象考成》、《数理精蕴》、《康熙永年历法》、《康熙皇舆全览图》等图书、历法和地图。

康熙對於宗教基本上是寬容的,不僅僅是漢傳佛教,或者滿洲的藏傳佛教、薩滿教信仰,还褒封道教白云观方丈王常月,并皈依于门下。他甚至也時常聽天主教傳教士講道。直到他发现罗马教廷试图干预中國政治,并且皇子信仰基督后以此作为争权夺利的工具,遂开始有所抵制天主教,即中國禮儀之爭。

康熙也利用戴南山(戴名世)的南山案文字獄事件,株連甚多,來抑制漢族士大夫的反叛思想,甚至桐城派文家方苞都差點遭斬首。

康熙是中国历史上少有的重视自然科学的皇帝,对西方文化也十分感兴趣,自身具有相当高的科学素养,向来华传教士学习代数、几何、天文、医学等方面的知识,并颇有著述。例如:曾从南怀仁学习欧几里得《几何原本》並且每天听讲。后来又学习西方的测量、天文、物理和医学等知识,并在宫中设置了研究化学和药学的实验室。康熙因南怀仁督造火炮方面的功绩,一直对他优礼有加,而南怀仁等西方传教士也促进了伽利略的弹道理论在中国的传播。

康熙除了學習西方科技之外還會應用實踐,其最突出的是用科學方法和西方儀器繪製全國地圖。康熙亦會利用巡行和出兵之便,實地測量,吸取經驗。在康熙四十六年(1707年)委任耶穌會傳教士雷孝思、白晉、社德美及中國學者何國楝、明安圖等人走遍各省,運用當時最先進的經緯圖法、三角測量法、梯形投影技術等在全國大規模實地測量,並於康熙五十七年(1718年)繪製成《康熙皇輿全覽圖》,其作被稱為在當時世界地理學的最高成就,英國李約瑟亦稱之為不但是亞洲當時所有的地圖中最好的一幅,而且比當時的所有歐洲地圖都要好、更精確。

康熙還以巡視之便訪求民間的有才之士,例如將在數學方面有很大成就的梅毂成調進宮中培養深造。梅毂成亦通過學習西方數學知識,重新令在明朝被廢棄的中國古代數學受到重視。

由於康熙帝是中國歷代帝王中最重视科学、最提倡科学和最精通科学的人,故後代有很多评判和標籤加在他身上,他被視為有重大贡献的「科学家皇帝」,或被視為是「窒塞民智」的「罪魁祸首」。有學者及歷史學家認為,清朝中後期國力開始遠遠落後於西方,跟康熙晚年墨守成规和缺乏创新有關,故他应当为中国科技的落后状况负责任;此外,亦有學者認為,康熙由于自身的局限性,对當時的科学内容采取又用之又防之的手段,他又担心先进的西方科技一旦传开,将会极大的动摇以骑射起家的满清的统治,另外,康熙亦被批評阻礙了中國火器的發展。

此外,由於传教士们所宣扬的基督宗教教義与中国的传统文化观念之有很大的差异和分歧,故西學受當時中国各阶层保守人士竭力反对,清初保守派官员楊光先就強調「宁可使华夏无好历法,不可使中国有西洋人」,對傳播西學的傳教士表示不滿。面對士大夫的不滿情緒以及罗马教廷對中国文化礼俗的傲慢,作為中华文化正统的最高代表,康熙特意对理学名臣李光地、熊赐履等说:“汝等知西洋人渐渐作怪乎?将孔夫子亦骂了。予所以好待他者,不过是用其技艺耳。历算之学果然好。你们通是读书人,见外面地方官与知道理者,可俱道朕意。”希望借助他們剖白他为何使用传教士及其底线所在。與批評西學為「奇技淫巧」的守旧派官僚不同,願意学习和提倡西学的康熙对西学采取较开明的态度。

康熙对国家的治理中对“汉学”传统的学习与推崇,从各方面接受并正确执行汉族政策,充分正视和运用“汉家”的传统意识,为开创鼎盛局面打下基础。但是康熙作为“天下之主”,为了维护清朝的根本利益,极力标榜“满汉一体”。但是,受本民族利益的驱使和民族情感的困扰,他往往自觉或不自觉地陷入偏徇满洲的境地,在噶礼和张伯行互参案中体现出来。

1690年至1697年多次擊敗准噶尔部噶尔丹,史稱三征噶爾丹。在雅克萨战役,康熙派遣黑龙江将军萨布素成功驱逐沙俄对黑龙江流域的侵略,收復了雅克薩城(舊稱阿爾巴津;現俄罗斯联邦斯科沃罗季诺)和尼布楚城(现俄罗斯联邦涅尔琴斯克)。他在京师东北的热河营建了避暑山莊,将其作为蒙古、西藏、哈萨克等部王公贵族觐见的场所,为清朝大肆的修建皇家园林开辟了先河。

亦有史學家指出,康熙會欣賞和重用有才華的傳教士,西方先進的科學技術也被推崇和應用。康熙曾經委派傳教士閔明我(Domingo Fernández Navarrete)返回歐洲招募人才,希望增進中西方科技文化交流。而民間與西方傳教士能夠互相交遊,西學在社會中得以自由傳播,亦指出分別由康乾皇帝敕輯的叢書-《古今圖書集成》和《四庫全書》亦收錄了傳入中國的西方科學技術。

据传教士张诚(J. F. Gerbillon)的日记記載,康熙為了保護傳教士不被其他官員陷害而不准他們在有汉人和蒙古人的衙门裏翻译任何科学文献。18世纪康熙末期,因罗马教廷發出禁止中国人教徒祭祖的禁令而引发礼仪之争,促使清廷反制并下令“自今以后,若不遵利玛窦规矩,断不准在中国住,必逐回去”。

中俄开始正式接触是在康熙帝时期,签订了《尼布楚条约》以后,两国贸易逐渐繁荣。1715年,俄国传教士首次来华,加强了两国经济、文化之间的交流。康熙晚年,因为俄商来华人数众多;更重要的是俄方一些行为违背了康熙关于安全、和平的原则,因而使中俄关系形势逆转。

然而有文獻記載指出,在清朝康熙年間,原本閉關鎖國的中國逐漸向外界開放,並維持著國內、近鄰貿易以及歐洲貿易。甚至說「全歐洲的貿易量都無法跟巨大的中國貿易量相比」,並且形容中國的各個省就相當於歐洲的各個王國,它們各自擁有自己豐富且多種多樣的特產進行貿易,而且這傾向於聯盟保護的形式,在所有的城市裡也一樣,以至官員們在商業界裡都擁有自己的股份/分成,他們當中有部分人會將他們的金錢委託給值得信任的人打理以保證他們的資產在商業往來中取得成果,連平民百姓也可以從商業貿易中得益 。同時記載了清朝市集的繁華程度和中外商家的貿易情況,又稱中國商人在交易時都很誠實可靠,跟日本、巴達維亞(今印尼雅加達)、馬尼拉以及歐洲也有貿易來往。《全球通史》裡亦指出,康熙時期中國的對外貿易急劇膨脹且發展快速,大量的茶葉、絲綢、棉布、瓷器和漆器經廣州口岸運往歐洲銷售。

华裔日籍学者杨启樵说:“康熙宽大,乾隆疏阔,要不是雍正的整饬,清朝恐早衰亡。”

英國籍史學家史景迁批评康熙有三:一是皇位继位的纠葛进退失据;二是康熙虽喜爱西学,任用耶稣会士,并允传教,但对西方并不信任,因而有礼仪之争以及导致雍正禁教;三是康熙以轻徭薄赋自豪,以此彰显盛世,但其永不加赋的政策按耕地面积缴固定税金,与人口无关,于是人口虽增,亦不加赋,为康熙的继承者造成财政困难。

法國人白晉:「康熙皇帝經常到各地巡視,以便了解百姓的生活情況和官吏們的施政狀況。在這樣的觀察時,即便最卑賤的工匠和農夫,皇帝也允許他們接近自己,並用非常親切溫和的態度詢問他們,這常常使得普通百姓至為感動。康熙皇帝會經常向百姓提出各種問題,​​並且他一定要問到的問題是他們對當地的官吏是否滿意。如果百姓普遍傾訴對某個官員的不滿,康熙皇帝會將他撤職。但是如果百姓讚揚到某個官員,他卻並不一定僅僅因此就提升那個官員。」;「康熙皇帝的孝順和感恩是如此罕見,他因此獲得了舉國百姓的尊敬和擁戴。」。白晉亦提到康熙對賑濟災區與安撫饑民的手法:「我們在北京的其中兩年,我們親眼目睹了以下這些確證的事實。當時,兩三個省遭受了大旱災,造成農業嚴重欠收。康熙皇帝為此深為憂慮,他免除了這些省份的賦稅,並設立常平倉進行賑恤,但仍不能滿足災區的需要,於是,他又向災情最為嚴重的地區調撥了大量的糧食和巨額款項。為了進一步賑濟災區的窮人,康熙皇帝又採取了捐官的政策,允許富人中有學識的人,如果能夠通過做官資格的考試、證明他確有才幹,並向災區捐獻一定數目的糧食,便可買到一個相應的官職。當時,為了尋求生路,大量的窮人紛紛湧入北京,康熙皇帝下令把這些人全都招用於六部官署的建築工程,從而找到一個既幫助了窮人又使他們對社會有所貢獻的辦法。並且,這個辦法也有利於安撫饑民,防止他們因走投無路而發生動亂。」

比利時人南懷仁:「(康熙帝)親切地接近老百姓,力圖讓所有人都能看見自己,就像在北京時的慣例一樣,他諭令衛兵們不許阻止百姓靠近。所有的百姓,不管男女,都以為他們的皇帝是從天而降的,他們的目光中充滿異常的喜悅。為一睹聖容,他們不惜遠涉跑來此地,因為,對他們來講,皇帝親臨此地是從不曾有過的事情。皇帝也非常高興於臣民們赤誠的感情表露,他盡力撤去一切尊嚴的誇飾,讓百姓們靠近,以此向臣民展示祖先傳下來的樸質精神。」

康熙帝幼年继位,立志“为治天下而学”,终身好学不倦,同时勤习骑射,弓马娴熟,体格健壮。其中,刻苦的学习精神和良好的读书方法对他治国理政具有不可替代的作用。康熙从少年时代开始直到晚年,对古代书家作品的学习都不曾间断。《石渠宝笈》和《佩文斋书画谱》著录了较多康熙对古代书迹的题跋。

康熙帝也是一位重视自然科学、精通医道的养生家,相传,八宝豆腐和康熙帝也有渊源。但是康熙晚年多病缠身,还患有高脂血症,这多少与他的饮食失衡有关。

\subsection{康熙}

\begin{longtable}{|>{\centering\scriptsize}m{2em}|>{\centering\scriptsize}m{1.3em}|>{\centering}m{8.8em}|}
  % \caption{秦王政}\
  \toprule
  \SimHei \normalsize 年数 & \SimHei \scriptsize 公元 & \SimHei 大事件 \tabularnewline
  % \midrule
  \endfirsthead
  \toprule
  \SimHei \normalsize 年数 & \SimHei \scriptsize 公元 & \SimHei 大事件 \tabularnewline
  \midrule
  \endhead
  \midrule
  元年 & 1662 & \tabularnewline\hline
  二年 & 1663 & \tabularnewline\hline
  三年 & 1664 & \tabularnewline\hline
  四年 & 1665 & \tabularnewline\hline
  五年 & 1666 & \tabularnewline\hline
  六年 & 1667 & \tabularnewline\hline
  七年 & 1668 & \tabularnewline\hline
  八年 & 1669 & \tabularnewline\hline
  九年 & 1670 & \tabularnewline\hline
  十年 & 1671 & \tabularnewline\hline
  十一年 & 1672 & \tabularnewline\hline
  十二年 & 1673 & \tabularnewline\hline
  十三年 & 1674 & \tabularnewline\hline
  十四年 & 1675 & \tabularnewline\hline
  十五年 & 1676 & \tabularnewline\hline
  十六年 & 1677 & \tabularnewline\hline
  十七年 & 1678 & \tabularnewline\hline
  十八年 & 1679 & \tabularnewline\hline
  十九年 & 1680 & \tabularnewline\hline
  二十年 & 1681 & \tabularnewline\hline
  二一年 & 1682 & \tabularnewline\hline
  二二年 & 1683 & \tabularnewline\hline
  二三年 & 1684 & \tabularnewline\hline
  二四年 & 1685 & \tabularnewline\hline
  二五年 & 1686 & \tabularnewline\hline
  二六年 & 1687 & \tabularnewline\hline
  二七年 & 1688 & \tabularnewline\hline
  二八年 & 1689 & \tabularnewline\hline
  二九年 & 1690 & \tabularnewline\hline
  三十年 & 1691 & \tabularnewline\hline
  三一年 & 1692 & \tabularnewline\hline
  三二年 & 1693 & \tabularnewline\hline
  三三年 & 1694 & \tabularnewline\hline
  三四年 & 1695 & \tabularnewline\hline
  三五年 & 1696 & \tabularnewline\hline
  三六年 & 1697 & \tabularnewline\hline
  三七年 & 1698 & \tabularnewline\hline
  三八年 & 1699 & \tabularnewline\hline
  三九年 & 1700 & \tabularnewline\hline
  四十年 & 1701 & \tabularnewline\hline
  四一年 & 1702 & \tabularnewline\hline
  四二年 & 1703 & \tabularnewline\hline
  四三年 & 1704 & \tabularnewline\hline
  四四年 & 1705 & \tabularnewline\hline
  四五年 & 1706 & \tabularnewline\hline
  四六年 & 1707 & \tabularnewline\hline
  四七年 & 1708 & \tabularnewline\hline
  四八年 & 1709 & \tabularnewline\hline
  四九年 & 1710 & \tabularnewline\hline
  五十年 & 1711 & \tabularnewline\hline
  五一年 & 1712 & \tabularnewline\hline
  五二年 & 1713 & \tabularnewline\hline
  五三年 & 1714 & \tabularnewline\hline
  五四年 & 1715 & \tabularnewline\hline
  五五年 & 1716 & \tabularnewline\hline
  五六年 & 1717 & \tabularnewline\hline
  五七年 & 1718 & \tabularnewline\hline
  五八年 & 1719 & \tabularnewline\hline
  五九年 & 1720 & \tabularnewline\hline
  六十年 & 1721 & \tabularnewline\hline
  六一年 & 1722 & \tabularnewline
  \bottomrule
\end{longtable}


%%% Local Variables:
%%% mode: latex
%%% TeX-engine: xetex
%%% TeX-master: "../Main"
%%% End:

%% -*- coding: utf-8 -*-
%% Time-stamp: <Chen Wang: 2019-10-22 10:39:07>

\section{世宗\tiny(1722-1735)}

雍正帝(1678年12月13日-1735年10月8日),爱新觉罗氏,名胤禛,法号破塵居士、圓明居士,是清朝自入关以来的第三位皇帝,1722年12月20日至1735年10月7日在位,年号「雍正」。死后庙号世宗,谥号敬天昌运建中表正文武英明宽仁信毅大孝至诚宪皇帝,通称世宗宪皇帝。

雍正帝乃康熙帝第四子,於1722年12月27日登基(農曆康熙六十一年十一月二十日)。在位期间对内政民生有着诸多改革,例如在中央设置軍機處和密折制度来加強皇权,在地方上推行摊丁入亩、火耗歸公、改土归流、打擊貪腐的王公官吏和廢除賤籍等一系列政策来推动清朝经济和国力进一步增加,对外则通过对俄国谈判确定蒙古北部边疆,同时平定青海,在西藏设置驻藏大臣等对统一多民族有着重大贡献,还对康雍乾盛世的延续具有承上启下的重大作用。

胤禛于康熙十七年十月三十日(1678年12月13日)寅時出生于故宫永和宫。由於其生母乌雅氏出身低微,沒有撫育的資格,此外,清初時後宮也不允許生母撫育自己的兒子,因此胤禛满月后由孝懿仁皇后佟佳氏撫養,視其為養母。康熙帝曾评价幼年的胤禛“喜怒不定”,后经胤禛请求,于康熙四十一年(1702年)撤此考语。因胤禛性情急躁剛烈,父皇康熙用“戒急用忍”训喻他。胤禛早年随康熙巡历四方。

康熙三十七年(1698年)三月,康熙帝第一次賜給胤禛爵位,封為多羅貝勒。

康熙四十六年(1707年),康熙賜皇家園林圓明園給貝勒胤禛,十一月,胤禛恭請康熙幸(圓明園)進宴用膳(1707年至1722年,康熙帝總共了去圓明園12次)。

康熙四十七年(1708年)夏,康熙第一次罷黜皇太子允礽。

康熙四十八年(1709年),康熙复立允礽为太子。同年也升多羅貝勒胤禛爵位為和硕雍亲王。

康熙五十年(1711年)八月,胤禛妾室典儀之女藩邸格格鈕祜祿氏(熹妃)生下雍亲王胤禛第四子弘曆,即後來的乾隆帝。

康熙五十一年(1712年)康熙再次废黜允礽,自此不再立皇太子。争夺储位斗争由明转暗,更加激烈。胤禩因争位意图过于明显,被康熙斥责、疏远。胤禛沈迷釋教,有時崇信道教,到諸宮觀禮拜天尊真人圖像,與道士們研究金丹之學,与诸兄弟维持和气,自称“天下第一闲人”,暗中与隆科多与年羹尧交往,加强自己的势力集团。

康熙五十三年(1714年),朝鮮國王所派的使臣回國後,向朝鮮國王表明大清康熙皇帝當時的意旨:「(胤禛二哥)允礽之子弘皙颇贤,难于废立(太子)允礽」;或康熙五十六年(1717年),亦表明康熙皇帝當時意旨:「弘皙甚贤,故不忍立他子,而尙尔贬处允礽矣」。

康熙六十一年(1722年),胤禛第四子弘曆垂髫之年(12歲時),康熙幸胤禛的圓明園進宴用膳。(乾隆上位後,《高宗純皇帝實錄》記載了,康熙因為乾隆弘曆之故在圓明園進宴用膳,康熙連稱弘曆生母為有福之人;但是康熙時期在自己的《聖祖仁皇帝實錄》上,並未給乾隆弘曆母子記載任何很喜愛他們的歷史,也未給乾隆弘曆母子賜冊世子或福晉以作為獎賞)

康熙六十一年(1722年)十一月,胤禛登基,為雍正皇帝。胤禛二哥允礽,第二子弘皙是胤禛上位後第一位晉升王爵(多羅理郡王)的侄輩。朝鲜使臣向朝鮮国王稟報大清國皇宫盛传:「(康熙臨命終遺言):允礽第二子弘皙朕所鍾愛,其特封为和碩親王」爵位、又有「康熙皇帝既封允礽之子弘皙为王,雍正以在邸时宫室、服御、金银、臧获及王府官属,一倂移给」。這在本國和它國等諸多史料確實有明確記載的。而康熙遺命要預備給弘曆冊封王爵在朝鮮國並沒有提及。

雍正特別在宫中抚养允礽的幼龄兒子:弘㬙、弘皖、永璥,收为雍正帝的养子(這也是乾隆弘曆上位後親口承認的)。

三月,雍正皇帝親生皇子皇女中,只追封側福晉李氏所生已成年皇二女爵位:和碩格格(郡主)为和硕怀恪公主(康熙雍正的未成年子女一律不封爵位,皇子女有超過18歲的才有尊爵)。

雍正元年(1723年)八月,雍正於乾清宮召諸王、滿漢大臣入見,面喻曰:「建儲一事,理宜夙定。去年十一月之事,倉卒之間,一言而定。聖祖仁皇帝神聖,非朕所及」。命群臣皆退,仍留這四人總理事務王大臣:允禩、允祥、隆科多、马齐 ,以康熙旨意不立皇太子,将密封遗诏收藏於乾清宮最高之處(亦是大清歷朝皇帝最早秘定的太子人選)

雍正元年(1723年)九月二十日卯时,雍正以康熙遗命,分家理郡王弘晳距京城二十里的郑各家庄王府,亦下令弘皙携福晋、以及子弟一起迁至皇城外的郑各家庄,命人以礼相待弘晳及其眷属,以隆重礼数安排至距皇城外二十里的郑各家庄藏身定居,亦命令以多羅郡王禮數儀仗相送,並命數千位兵丁家臣奴僕保護弘皙的鄭各家莊王府。而弘晳之父允礽因有罪因此仍被禁锢於皇城内咸安宫。雍正帝十分关心弘晳,弘皙亦於奏摺中稱呼本是叔父的雍正皇為:“皇父”,与弘晳关系融洽。

雍正元年(1723年)十一月,適逢康熙忌辰,雍正命皇四子弘曆祭景陵。

雍正二年(1724年)五月,雍正諭旨:「二阿哥允礽奏曰:臣蒙皇上种种施恩甚厚,臣心实深感激。又训弘皙,你若能一心竭诚效力,以事君父,方为令子,此皆二阿哥允礽至诚由衷之言」。十二月,允礽病故后,雍正追封允礽和碩理亲王,謚號,曰:密。而且,雍正還特別賜弘皙之生母李佳氏為允礽的側福晉,令弘皙盡心孝養李佳氏。並且讓允礽各妻妾,皆能豐衣足食,以終餘年。

雍正四年(1726年),雍正皇帝給大學士鄂爾泰御筆朱批中有提道:『朕之關心(你),勝朕頑劣之子』。雍正八年(1730年)又說:『皇子皆中庸之資,朕弟侄輩亦乏卓越之才』。

雍正八年(1730年),雍正遵照康熙皇帝臨命終時遺言,冊封允礽的第二子弘皙承襲其生父允礽的爵位:和碩理親王(雍正皇帝剩下18歲皇子兩位:弘曆、弘晝,但還未冊封爵位)

雍正十一年(1733年)正月,皇子只剩兩位時,雍正諭宗人府:「朕幼弟(18歲)胤秘,秉心忠厚賦性和平素為皇考(康熙)之所鍾愛,數年以來在宮中讀書學識亦漸增長,朕心嘉悅著封親王。皇四子弘曆(21歲)、皇五子弘晝(21歲),年歲俱已二十外,亦著封為親王,所有一切典禮著照例舉行」。(弘曆最受康熙鍾愛,連多羅郡王、貝勒爵位都封不到)

雍正十三年(1735年)八月,雍正皇帝于圓明園病重,宝亲王弘曆和亲王弘晝朝夕侍侧。晚上八點,大學士鄂尔泰、大學士张廷玉至雍正寢室,恭捧上御笔亲书曰:『命皇四子宝亲王弘曆为皇太子即皇帝位』。夜子時,雍正躺在病床上立弘曆為太子後,在圓明園駕崩,時年五十八歲。匾額下宣讀密封遺詔,喻旨:「寶親王皇四子(弘曆),……聖祖康熙帝於諸孫之中,最為鍾愛,撫養宮中,恩逾常格……雍正元年八月朕於乾清宮召諸王、滿漢大臣入見,面諭以建儲一事,親書諭旨,加以密封,收藏於乾清宮最高之處,即立弘曆為皇太子之旨也。其後仍封(爵位)和硕寶親王者(饋贈大寶給弘曆),蓋令備位藩封,諳習政事,以增廣識見……著繼朕登極,即皇帝位……俾皇太子弘曆成一代之令主……,與和親王(弘晝)同氣至親,實為一體…大學士張廷玉器量純全,抒誠供職,其纂修《聖祖仁皇帝實錄》宣力獨多;大學士鄂爾泰誌秉忠貞,才優經濟,…此二人者朕可保其始終不渝。」皇太子弘曆登基,是為乾隆帝。以雍正駕崩前遺命囑託封乾隆皇帝生母熹妃鈕鈷祿氏為皇太后(欠缺冊封熹貴妃和裕妃的金冊或金印,《世宗憲皇帝實錄》亦未載冊文)。封和親王弘晝之母裕妃耿氏皇貴太妃。兩名撫育乾隆為皇子時的慈母(愨惠皇貴妃)佟佳氏及(惇怡皇貴妃)瓜爾佳氏,雍正本不封她們太妃,乾隆最後晉封她們皇貴太妃。

乾隆帝以西北軍事底定撤除軍機處,軍機處改設總理事務處並兼理軍機事務,總理事務王大臣以這四人:大學士鄂爾泰、大學士張廷玉、莊親王允祿、果親王允禮,原兼任軍機大臣鄂爾泰、張廷玉改在總理事務處。(乾隆2年,准總理事務王大臣解職,復設軍機處,乾隆以總理事務王大臣…入值軍機處)。

雍正十三年(1735年)九月,奉乾隆諭旨:「理密親王允礽之子弘㬙、弘皖、永璥因年尚幼穉蒙雍正垂慈恩養,仍住宮中,年已長成,雍正原欲賜宅另居尚未降旨,茲朕仰體聖慈為籌畫久遠之計,其應加封王爵,著總理事務王大臣會同內務府定議」。

雍正十三年(1735年)十月,總理事務鄂爾泰恭擬上崇慶皇太后的尊號

乾隆三年(1738年)二月,乾隆叔父果亲王允禮薨,乾隆命六弟弘曕過繼允禮子嗣,且協助弘曕袭果亲王爵。(雍正時期乾隆三哥弘時獲罪,過繼阿其那允禩子嗣)

所以,雍正的子女只剩下兩名:乾隆弘曆與弘晝。

乾隆四年(1739年)十月,理親王弘皙因突然在乾隆上位後,有了謀反皇帝等罪名,因此弘皙永久被革除親王爵。乾隆四十八年,乾隆還特別另外編撰《钦定古今储贰金鉴》歷史史籍,奉乾隆帝諭旨,記載以下歷史:「弘皙縱欲敗度,不克幹蠱,年亦不永。使相繼嗣立,不數年間連遭變故,豈我大清宗社臣民之福乎?是以皇祖康熙有鑒於茲,自理密親王既廢不復建儲,迨我皇祖康熙龍馭上賓,傳位雍正紹登大寶,十三年勵精圖治中外肅清...雍正元年,即親書朕名,緘藏於乾清宮正大光明匾內,又另書密封匣,常以隨身。至雍正十三年八月,雍正升遐,朕同爾時大臣等敬謹啟視,傳位於朕之御筆,復取出內府緘盒密記...」。

康熙六十一年(1722年)年十一月初七(12月14日),康熙聖祖駕崩前宣詔嗣位於畅春园,皇四子雍亲王胤禛继皇帝位,是为雍正帝。康熙帝死时,多人包括多位阿哥都知道康熙傳位雍正然後隆科多一人传遗诏由雍正继位。治丧期间,隆科多提督九门、卫戍京师。隆科多是皇贵妃佟佳氏的弟弟。雍正繼位,任命康熙皇八子允禩、皇十三子允祥、馬齊和隆科多總理事務。

雍正十三年八月二十二日(1735年10月7日),雍正因工作過勞累,在批閱奏章時崩逝於圓明園,享年五十七歲。廟號世宗,諡號憲皇帝,安葬於清泰陵。命其四子弘曆登基繼位。

军机处:雍正八年,新首創立軍機處,當時主要为了緊急应对西北军情,协助辦理皇帝处理对准噶尔用兵的各種軍務。而军机处设有军机大臣,从大学士、尚书、侍郎以及皇亲国戚中担任。 議政王大臣會議與軍機大臣在雍正時期,依然是並存的,並且雙方職責各不盡相同,共同點皆需要處理軍務。只是1792年乾隆當政時,废除了议政王大臣会议,乾隆以军机处為主要專一事權。例如雍正時期的首席軍機大臣:怡親王允祥、大學士鄂爾泰。
密摺制:雍正还在中央进一步完善密摺制度来監視臣民。
清除兄弟:雍正二年四月,明詔訓飭康熙帝皇八子,令王公大臣察其善惡;削康熙帝皇十子爵永遠拘禁之;十二月,康熙帝前廢太子死。雍正三年二月,諭示康熙帝皇八子罪狀;四年正月除宗籍,易名“阿其那”(滿語罵人的話,意義眾說紛紜,有「馱負罪過」、「驅趕犬隻」、「冷凍的魚」等眾說),九月死。雍正三年二月,諭示康熙帝皇九子罪狀,八月革爵;四年五月改名“塞思黑”(意为「顫抖」,也有人說是「刺傷人的野豬」),八月死。雍正三年二月,諭示康熙皇十子胤誐罪狀。雍正二年七月,命同母弟、康熙帝皇十四子胤禵守陵;三年二月,諭示其罪狀,十二月降爵;四年五月禁錮。雍正六年六月,康熙皇三子胤祉因罪降爵;八年二月復親王爵,五月因康熙皇十三子之喪時「遲到早散,面無戚容」而削爵拘禁。
雍正帝戎装像
雍正帝戎装像
整顿吏治
康熙帝在位晚年,对下属过度宽纵,导致大清吏治腐败,官风松懈。雍正帝登基后第一项任务就是整顿吏治。一方面雍正帝告诫官员,在给总督的上谕中说:“今之居官者,钓誉以为名,肥家以为实,而曰‘名实兼收’,不知所谓名实者果何谓也”,登极一周年时又说到:“朕缵承丕基,时刻以吏治兵民为念”。另一方面完善监督体系,紧抓思想反腐,并注重官员、民众的思想道德教化,树立反腐典范。

在整饬吏治的同时又打击朋党势力,他看到朋党之间各抒政见,妄议朝政,扰乱君父视听,妨碍坚持既定的政策,认为“朋党最为恶习”,因此宣称“将唐宋元明积染之习尽行洗涤”,“务期振数百年颓风,以端治化之本”。

改善祕密立储制度,即皇帝在位时不公开宣布太子,而将写有继承人名单的一式两份诏书分别置于乾清宫“正大光明”匾额后和皇帝身边,待皇帝去世后,宣诏大臣共同拆启传位诏书,确立新君。这样使得皇位继承辦法制度化,也在一定程度上避免康熙帝晚年诸皇子互相倾轧的局面。

雍正初年,重用年羹尧和隆科多。隆科多为吏部尚书、步军统领,兼理藩院,赐太子太保衔,被雍正尊称为“舅舅”。显赫异常,但未过几年,即被雍正整肃。雍正三年七月削隆科多太保銜;雍正四年正月削職;雍正五年十月廷臣上四十二罪款,下獄,永遠禁錮;雍正七年六月,死於禁所。其較為寵信的四位臣工:李衛(江苏人)、田文鏡(福建人)、張廷玉(安徽人)、鄂爾泰;李卫和张廷玉為漢人,田文镜为汉军的旗人,以民族分,漢族佔了四分之三,足見雍正確實了解也重用漢人。雍正四年十二月,河南、陜西、四川均攤丁銀入地併徵;謝濟世劾田文鏡,被褫職,發赴阿爾泰軍前效力,陸生柟亦以黨援同時遭遣。

清兵入关以后,国家承平日久,军备废弛。而作为大清军队主力的八旗兵也是丧失斗志,特别是在旗的八旗子弟,每日游手好闲,贪图享乐。雍正帝对于此情此景对八旗旗务进行了一些整顿,例如:给那些无所事事的旗人分的土地和农具,让其自力更生,派遣八旗子弟去前线参战等。

九子奪嫡、胤禔软禁、年羹堯案、曾静吕留良案、隆科多案、谢济世案、陆生楠案、屈大均案。

年羹尧先后被任命为川陕总督、抚远大将军,赴青海征讨厄鲁特罗卜藏丹津叛乱。

雍正元年三月,封年羹堯三等公;四月命康熙皇十四子留護康熙帝遺體;五月,生母仁壽皇太后死;八月,密封立四子弘曆之上諭於正大光明匾後;十月授年羹堯撫遠大將軍。雍正二年三月,平定青海,進年羹堯為一等公。成为实际的西北王。雍正三年三月,下詔斥責年羹尧,四月調為杭州將軍,六月削太保銜,七月黜為閑散旗員,十二月廷臣上九十二罪款,賜死,斬其子年富。

康熙末年吏治松弛,贪污成风,加上诸王皇族同官僚结党营私,致使财政经济从中央到地方混乱不堪,“积弊甚大”。仅户部就亏空白银二百多万两。面对如此局面雍正帝在稅制上推動“摊丁入畝”,“火耗歸公”,“官紳一體當差納糧”等一系列改革。

同时,设会考府,清查亏空。推广养廉银制度,养廉银不但是一项经济政策,同时也是清朝前期整顿封建制度的一项综合改革措施。

雍正帝在位期间还对科举制度实行了一系列改革,例如:创考差先例,改革选派考官制度;变更考的试内容和重点;增设考试科目,考生的资格限制有所放宽;还创行“朝考”、翻译翰林 “大考”等复试制,变通一试而定终身的制度;调整用人政策,数途并用,以抑科甲。这些措施的实行力剔积弊的施政作风。

雍正兴起文字狱以打击年羹尧和隆科多两人势力(汪景祺案和钱名世案)。雍正三年十一月,年貴妃死;十二月斬《西征隨筆》作者汪景祺。雍正四年三月,錢名世以曾投詩年羹堯獲罪,雍正親書「名教罪人」懸其家門,又命文臣作詩文刺惡他。对于隆、年的死因,有人指出是由于年、隆位重之后过于骄奢、行为不检,加上结党营私,触犯了皇權的大忌,为雍正所不容。但雍正早年过于宠信放纵,随后又残酷打击,被史学家所批评。另有人与雍正得位傳說联系起来,认为隆、年参与此事,知道太多而被「兔死狗烹」。雍正四年九月,查嗣庭以謗訕下獄,五年五月死,戮屍。

據正史記載:雍正七年五月,曾靜供稱因讀呂留良書而有謀反;十年十二月,治呂留良罪,與兒子呂葆中、門人嚴鴻逵一同戮屍,斬另一兒子呂毅中與門生沈在寬。

从清代史料中可以看出,雍正帝主张民族平等,尊重民族习惯;反对民族歧视和限制国家干预;保护民族生态,禁止过度需索。对促进民族融合,化解民族矛盾和维护清朝的统一多民族有着重要贡献,也使清代的民族统治达到历史最高水平。

雍正七年九月,頒行《大义觉迷录》。在书中雍正帝梳理对华夷、正统、君臣、封建等问题,论述了他自己所谓的民族“大一统”观。

雍正二年(1724年)设置西宁办事大臣,办事大臣衙门最初设于察罕托洛亥(青海湖东南),后改驻西宁,故乾隆以后又称为西宁办事大臣。

雍正五年,在西藏设置驻藏大臣,加强对西藏的控制。

廢除西南少數民族原本的土司制度,改用朝廷分發的流官,史稱“改土歸流”,派遣官吏統治,加強對少數民族的統治及同化。

海禁问题上,开始严格执行海禁,后来考虑到闽地百姓生计困难,同意适当开禁;雍正二年降旨准廣東人移民台灣,但对外洋回来的人民仍有戒心。雍正严禁中国商人出海经商,海设置各种障碍,并说道"海禁宁严毋宽,余无善策"。在沿海各省的要求下,虽放宽海禁,但仍加以限制盘剥。尤其对久住外国的华侨商贩和劳工,“逾期不归,甘心流移外方,无可悯惜,不许其复回内地”。
社会
雍正帝在位期间还实施“废除贱籍”一项改革。雍正帝下令为賤民开豁为民,编入正户,准許置產定居、考試,宣示廢除賤民階級,但影響有限,未能改變社會大眾的歧視風氣,賤民仍然存在,如福州疍民群體較明顯存續到清末,及所謂發功臣暨披甲家爲辛者庫。

努尔哈赤和皇太极的陵墓位于沈阳的盛京三陵。清入关后,从顺治帝、康熙帝都安葬到北京东边的遵化县马兰峪皇家陵园,即清东陵。雍正帝另选北京西边的易县开辟自己的陵墓,即清西陵。

《清史稿》:圣祖政尚宽仁,世宗以严明继之。论者比於汉之文、景。独孔怀之谊,疑於未笃。然淮南暴伉,有自取之咎,不尽出於文帝之寡恩也。帝研求治道,尤患下吏之疲困。有近臣言州县所入多,宜釐剔。斥之曰:“尔未为州县,恶知州县之难?”至哉言乎,可谓知政要矣!

《清世宗实录》:天表奇伟,隆准颀身,双耳半垂,目光炯照,音吐洪亮,举止端凝。......幼耽书诗,博览弗倦,精究理学之原,旁彻性宗之旨。天章濬发,立就万言。书法遒雄,妙兼众体。毎筹度事理,评骘人才,因端竟委,烛照如神。韬略机宜,皆所洞悉。。

李氏朝鲜君臣受儒家正统华夷之辨观念的影响对清国以及清国皇帝的态度多持批评态度,甚至有妖魔化倾向。朝鲜人毫无忌讳地记录康熙帝的“雌雄眼”容貌,还认为雍正帝贪财爱银。但是朝鲜使臣李樴于雍正元年回国后,向朝鲜国王报告,亲见雍正“气象英发,语言洪亮”。

英國歷史學家史景遷認為:雍正的父親康熙為政寬鬆,執政末期受儲立之爭所擾且出現典型長壽帝王的統治能力退化現象,雍正即位之初的滿清實已浮現官僚組織膨大腐敗、農民生活水準惡化的危機;由於雍正即位時正處於政治歷練、精神與人格上的成熟階段(45歲),因此得以精準的分析問題並有魄力的作出應對。他的改革同時包含力行整頓與和現實的妥協(如火耗歸公與養廉銀)。雖然史學家黃仁宇認為雍正未能瞭解與解決明清兩代作為內歛式王朝的根本問題,但滿清得以建立起一套繼續運行百年以上仍大致有效的統治體制,而未淪為「立國百年而亡」的異族王朝,此當歸功於雍正一朝的改革。

英国人濮兰德·白克好司评价雍正:“控御之才,文章之美,亦令人赞扬不值。而批臣下之折,尤有趣味,所降谕旨,洋洋数千言,倚笔立就,事理洞明,可谓非常之才矣”。

杨珍认为雍正是一位善于观察与思考者。其思想的敏锐性以及思维广度与深度,都超过允禩、允禵等人。

中國歷史學家钱穆認為:雍正帝是有名的专制,他私派的特务人员监视全国各地地方长官一切活动,许多地方官的私生活,连家里的琐事都瞒不过他,虽然雍正帝精明,但仍是独裁的本质。此外,雍正帝在平定外患之後,唯恐国内发生政变,于是使計把功高权重的大臣统统清除。他把過去與其爭位的两个兄弟——胤禩、胤禟以种种罪名逮捕拘禁,并将为他策划取得帝位的人处死,比如年羹尧和隆科多。

雍正即位经过至今也是一个解不开的谜。从雍正年间时,对雍正继位的谈论便不绝于耳。歷史記載雍正在康熙駕崩當晚連續觐見兩次之多,後康熙便身亡。主张篡位说的学者中,有的认为康熙去世过于突然,未来得及留下任何传位遗诏,而雍正和隆科多等合谋抢占了先机;有的认为康熙生前两立两废太子,对立储君一事劳心伤神,直到临终前才属意皇十四子为储君。按照正统继位说学者观点,如果没有实在的证据证明其他皇子为康熙所属意,雍正的即位是有理由的。

康熙帝傳位雍正帝之徵兆:徵兆一:「康熙六十年正月,命皇四子雍親王胤禛、皇十二子貝子胤祹、世子弘晟以御極六十年,告祭永陵、福陵、昭陵。」康熙登基一甲子六十年之重大祭告先祖非同一般,派遣雍親王胤禛主持,豈能不具備重大意義?為何不是派遣支持皇十四子胤禵的皇八子胤禩、皇九子胤禟、皇十子胤䄉?或是皇三子胤祉?徵兆二:康熙御極六十年派雍親王胤禛祭祖此舉,讓廢太子胤礽之師王掞看出端倪,故於三月「大學士王掞密奏請建儲,至是監察御史陶彝、任坪、范長發等人曾疏請建儲,帝不悅,並掞切責之。諸王、大臣奏請治大學士王掞罪,帝赦不治。」這亦可視為康熙安排接班人的佈署跡象之一,畢竟皇十四子胤禵尚且領兵在西北,一旦提早公佈,易生事端。徵兆三:「五月壬戌,命撫遠大將軍胤禵移駐甘州。以年羹堯總督四川陝西,色爾圖署四川巡撫。」康熙以皇四子雍親王胤禛之親信年羹堯箝制皇十四子胤禵的軍後補給已然成形。徵兆四:康熙六十一年四月,「命撫遠大將軍胤禵復往軍前。十月,命雍親王胤禛率弘昇、延信、孫渣齊、隆科多、查弼納、吳爾台察閱京師通州倉廒。」康熙指示由雍親王胤禛親率隆科多、查弼納等眾多京師王公重臣,竟然只為「察閱京師通州倉廒」,已有不尋常跡象。徵兆五:「十一月帝不豫,駐蹕暢春園。命皇四子胤禛恭代祀天。」康熙駕崩前祀天仍然未派皇三子胤祉、皇八子胤禩、皇九子胤禟、皇十子胤䄉代祀,更未召皇十四子胤禵返京,此時康熙意欲傳位於雍親王皇四子胤禛已然十分明顯。雍正元年三月,封年羹堯三等公;四月命康熙皇十四子留護康熙帝遺體;五月,生母仁壽皇太后死;八月,密封立四子弘曆之上諭於正大光明匾後(遺詔:「寶親王皇四子,……著繼朕登極,即皇帝位……俾皇太子弘曆成一代之令主。」);十月授年羹堯撫遠大將軍。雍正帝继位后,对其兄弟手段颇为毒辣,用各种方式进行迫害。雍正二年四月,明詔訓飭康熙帝皇八子,令王公大臣察其善惡;削康熙帝皇十子爵永遠拘禁之;十二月,康熙帝前廢太子胤礽死。

康熙帝皇八子胤禩先是被安抚封为廉亲王。雍正三年二月,諭示康熙帝皇八子罪狀;四年正月除宗籍,易名“阿其那”(滿語罵人的話,意義眾說紛紜,有「馱負罪過」、「驅趕犬隻」、「冷凍的魚」等眾說),九月死。康熙帝皇九子胤禟发往西宁。雍正三年二月,諭示康熙帝皇九子罪狀,八月革爵;四年五月改名“塞思黑”(意为「顫抖」,也有人說是「刺傷人的野豬」),八月死。雍正三年二月,諭示康熙皇十子胤誐罪狀;後被圈禁。雍正二年七月,命同母弟、康熙帝皇十四子胤禵守陵;三年二月,諭示其罪狀,十二月降爵;四年五月禁錮。雍正六年六月,康熙皇三子胤祉因罪降爵;八年二月復親王爵,五月因康熙皇十三子之喪時「遲到早散,面無戚容」而削爵拘禁。皇十二子胤祹被降爵。

然而「手段毒辣」之說,有人反對,並舉出了皇五子胤祺封至恆親王、皇七子胤祐封至淳親王、皇十二子胤祹封為履郡王、皇十三子胤祥封至和碩怡親王、皇十五子胤禑封多羅愉郡王、皇十六子胤祿承襲莊親王、皇十七子胤禮果親王、皇二十子胤禕封多羅貝勒、皇二十一子胤禧封貝勒、皇二十二子胤祜封固山貝子、皇二十三子胤祁封鎮國公等,諸多無利害關係的兄弟,得到封賞。

雍正突然的离世,史书不记载其去世原因,引起人们的疑惑。

病死:有人认为雍正帝“是中风死去的”。暗杀:民间流行的说法是,吕留良的后人吕四娘,为报仇,砍去雍正的头。丹药中毒:近年来由于对清代的档案进行了大量研究,许多史学工作者认为,雍正吃丹药中毒致死也有很大可能,而乾隆帝即位后,马上将圆明园内的炼丹道士和民间术士全部赶出。

雍正帝篤信佛教,熱衷藏傳佛教、漢傳佛教,與密宗的章嘉活佛交往密切;雍正也研究禪宗,精通《金剛經》,並著作佛學書籍數部,為章嘉活佛認可其參透三關,成為中國佛教史上唯一一位自認為已覺悟的皇帝。雍正帝也喜歡道教,常常服食道士的金丹。雍正元年重申禁止天主教,史稱雍正禁教。

雍正皇帝委託宮廷畫師郎世寧,創作一幅《雍正行樂圖》(現存於北京故宮博物院),顯示雍正喜愛打扮成不同年代的各式人物,後世人稱他為“近代Cosplay始祖”。

早期未即位前(九子奪嫡時期),就曾委託畫師給自己家人畫《春耕圖》進獻給康熙皇帝以表明無爭位之心,後來的乾隆皇帝也有相似的喜好。

\subsection{雍正}

\begin{longtable}{|>{\centering\scriptsize}m{2em}|>{\centering\scriptsize}m{1.3em}|>{\centering}m{8.8em}|}
  % \caption{秦王政}\
  \toprule
  \SimHei \normalsize 年数 & \SimHei \scriptsize 公元 & \SimHei 大事件 \tabularnewline
  % \midrule
  \endfirsthead
  \toprule
  \SimHei \normalsize 年数 & \SimHei \scriptsize 公元 & \SimHei 大事件 \tabularnewline
  \midrule
  \endhead
  \midrule
  元年 & 1723 & \tabularnewline\hline
  二年 & 1724 & \tabularnewline\hline
  三年 & 1725 & \tabularnewline\hline
  四年 & 1726 & \tabularnewline\hline
  五年 & 1727 & \tabularnewline\hline
  六年 & 1728 & \tabularnewline\hline
  七年 & 1729 & \tabularnewline\hline
  八年 & 1730 & \tabularnewline\hline
  九年 & 1731 & \tabularnewline\hline
  十年 & 1732 & \tabularnewline\hline
  十一年 & 1733 & \tabularnewline\hline
  十二年 & 1734 & \tabularnewline\hline
  十三年 & 1735 & \tabularnewline
  \bottomrule
\end{longtable}


%%% Local Variables:
%%% mode: latex
%%% TeX-engine: xetex
%%% TeX-master: "../Main"
%%% End:

%% -*- coding: utf-8 -*-
%% Time-stamp: <Chen Wang: 2019-10-22 10:42:07>

\section{高宗\tiny(1736-1795)}

乾隆帝(1711年9月25日-1799年2月7日),爱新觉罗氏,名弘曆,是清朝自入关以來的第四位皇帝,1735年10月18日—1796年2月9日在位,年号「乾隆」。西藏方面尊其為「文殊皇帝」。死后廟號高宗,諡號簡稱純皇帝,葬清東陵中的裕陵。

乾隆帝是满洲镶黄旗人,为雍正帝第四子,生於康熙五十年八月十三日(1711年9月25日)子時。登基於雍正十三年(1735年10月18日),在位至乾隆六十年(1735—1796年)。因其继位之时有在位时间不越祖父康熙帝之誓言,故而禅位于其子颙琰(即後來的嘉庆帝)。此时的乾隆虽为太上皇,但依然“训政”,在宫内仍然沿用乾隆年号,成為事實上的最高统治者,直至驾崩於嘉庆四年正月初三日(1799年2月7日)辰刻,享壽89岁,是中國歷史上最長壽的皇帝以及中国历史上實際掌權(執政)時間最长的皇帝(合共64年)。

弘历於康熙五十年八月十三日(1711年9月25日)出生,為雍正帝胤禛第四子,幼名「元寿」。当时,其父胤禛为雍亲王,生母为藩邸格格钮祜禄氏(孝聖憲皇后)。弘历生于雍王府东书院“如意室”。他被认为是雍正帝诸子中最有才干的一位,自小甚得其祖父康熙帝与父亲喜愛,虽然祖孙真正相处的时间並不长,但康熙帝曾為其慎择良师,进行多方面教育。在许多记载中也显示康熙帝对这个孙子十分疼愛。一些清史学家认为正因為康熙帝认为弘历在为人处事的方式上与自己极为相像,在十数岁时就精于武术,並对艺术创作十分著迷,所以才传位于其父,以便将來能传位与弘历。

康熙六十一年(1722年)十一月十三日,康熙帝驾崩前宣诏嗣位。二十日其父登基,是为雍正帝。第二年即雍正元年(1723年),雍正帝秘密建储亲书逾臾旨,密封遗诏藏于正大光明匾额后。同年三月,雍正亲生儿女中,只追封皇二女爵位為和碩怀恪公主,雍正帝尚活着的兒子弘時、弘曆、弘晝、福惠、福沛均未封爵位(其生母分別是齐妃、熹妃钮祜禄氏、纯愨皇贵妃、敦肃皇贵妃)。

雍正元年(1723年)十一月,雍正命皇四子弘历祭景陵。

雍正二年(1724年)十一月,適逢康熙忌辰,雍正命皇四子弘历祭景陵。

雍正四年(1726年)五月,適逄孝恭仁皇后三周忌辰,雍正帝欲要亲往祭陵。王大臣等,以圣躬素畏炎暑,万几已极劳苦,又触热往返五六百里,洵非所宜,且二麦登场,一路夫役祗候,不免耽误农功,因合词恳请停止。雍正皇帝勉从所请,因此命皇四子弘历前往行礼。

雍正四年(1726年)十一月,雍正下令,遣官祭永陵、福陵、昭陵、昭西陵、孝陵、孝东陵。命皇四子弘历祭景陵。

雍正四年(1726年)十二月,命皇四子弘历、庄亲王允禄,视马武疾。雍正谕,曰:「马武抱病危笃,闻之深为凄恻,马武事我皇考康熙五十余年,朝夕侍奉不离左右、恪恭谨慎,事事能仰体圣心」。馬武病故後。命皇四子弘历、怡亲王允祥、庄亲王允禄、及左翼四旗部院大臣、一等侍卫,往奠故内大臣马武茶酒。

雍正五年(1727年),皇四子弘历娶妻富察氏。完婚後,弘历由紫禁城的毓庆宮移居乾西二所(日后改名為重华宮)。

雍正六年(1728年),他的一位妾室富察氏為他生下了第一个孩子长子永璜。

雍正九年(1731年)八月,大学士忠达公、抚远大将军马尔赛起程,命皇四子弘历告祭奉先殿。王以下官员,俱至西长安门外送行。

雍正十年(1732年)春正月,享太庙,命皇四子弘历行礼。

雍正十一年(1733年),雍正皇帝第一次封皇子爵位(那时雍正帝儿子只剩下兩位:熹妃钮祜禄氏子弘历与裕妃子弘晝):弘历封和碩宝亲王爵位、弘昼封和硕和亲王爵位。住地獲赐名「乐善堂」,未外设王府。

雍正十二年(1734年)夏四月,享太庙,命皇四子宝亲王弘历行礼。

雍正十三年(1735年)五月。雍正命果亲王允礼、皇四子宝亲王弘历、皇五子和亲王弘昼、及大学士鄂尔泰、张廷玉、户部尚书公庆复、礼部尚书魏廷珍、刑部尚书宪德、张照、工部尚书徐本、正红旗汉军都统李禧、正黄旗汉军都统甘国璧、仓场侍郎吕耀曾,俱办理苗疆事务。

由于皇四子弘历行事恩威並施,手段宽猛相济,雍正帝指派他钦差出京办事,以及参与西北準部用兵西南改土归流的決策。在雍正後期,使自己渐得到了父亲恩宠。

雍正十三年(1735年)八月二十三日(10月8日),其父雍正帝駕崩,享年五十七歲。宣讀遺詔:“寶親王皇四子(弘曆),……聖祖康熙皇帝於諸孫之中,最為鍾愛,撫養宮中,恩逾常格……與和親王(弘晝)同氣至親,實為一體……俾皇太子弘曆成一代之令主。”寶親王皇四子弘曆登基,是為乾隆帝。乾隆弘曆以雍正駕崩前遺命尊生母熹貴妃鈕鈷祿氏為崇慶皇太后。封和親王弘昼之母裕妃耿氏為皇貴太妃。九月,曾撫養過弘曆為皇子時的慈母愨惠皇貴妃、惇怡皇貴妃各加封號晉封太妃(雍正不立兩位前朝的皇考太妃之位)。十一月,大学士等议奏崇慶皇太后父四品典仪官凌柱封一等承恩公、母為一品夫人。

同時遺詔命莊親王允祿、果親王允禮、大學士鄂爾泰、張廷玉為輔政大臣,輔佐新君處理政務。

乾隆十年(1745年) 正月,魏貴人詔封為令嬪。(即,後來嘉慶帝的生母)

乾隆上位後,命人編纂《國朝官史》。 收錄以下雍正皇帝諭旨:

雍正元年正月,上諭:諸皇子入學之日,與師傅豫備杌子四張,高桌四張,將書籍筆硯表裏安設桌上。皇子行禮時,爾等力勸其受禮,如不肯受,皇子向座一揖,以師儒之禮相敬。如此則皇子知隆重師傅,師傅等得以盡心教導,此古禮也。朕為藩王時,在府中亦如此行。至桌張飯菜,爾等照例用心預備。

雍正八年三月,上諭:諭總管太監傳與各處首領太監知悉:阿哥現居宮內,年已長成,爾等不可趨奉,亦不可得罪,並不許向阿哥處往來行走。即阿哥下太監亦不許與爾等所屬太監飲酒、下棋、鬥骨牌、說閑話。除趙進朝、靳進忠、趙運祥、楊進朝四人奉旨行走,不必攔阻外,其餘各處首領太監,嚴加曉諭,小心遵行,不可日久懈怠。嗣後如有玩法之人,經朕察出,係宮內太監,治宮內總管之罪;係圓明園太監,治圓明園總管之罪。

乾隆帝即位後,以「宽猛相济」理念施政,先後平定新疆、蒙古,还使四川、贵州等地继续改土归流,人口不断增加,在乾隆末年时突破三亿大关,约占当时世界人口的三分之一。乾隆三十八年(1773年)下令编纂《四庫全書》,歷時9年成書,是当时世界上最为庞大的百科全书。统治期间与康熙、雍正二朝合稱「康雍乾盛世」。

同時,乾隆為了打擊朋黨以及加強對人民主要是漢人的思想控制,大興文字獄,並藉此焚書箝制漢人反清思想的傳播。郭成康指出,乾隆查辦禁書目的就是要徹底消滅部分漢人中的反滿思想;然而,乾隆當時民族矛盾和鬥爭的情況已經逐漸緩和、並且在漢族臣民已承認清朝對全國統治的情況下,乾隆將民族矛盾和鬥爭的嚴重性誇大,在有關文字獄和禁書的決定中作錯誤估計,並且表現得過度敏感。此外,在乾隆時期的文字獄,針對的並非只有漢族,犧牲者中亦有滿族如鄂昌。

中期以後,乾隆多次下江南,有安撫百姓,檢閱軍隊,視察水利,增加科舉以及免除税收之舉。

乾隆五十一年十一月廿六日(1787年1月16日),台灣爆發林爽文事件,滿清雖利用台灣閩客之間的族群對立,但戰事曠日廢時,要至福康安率大兵登陸後,方於四個月內鎮壓此亂。並將林爽文凌遲斬首,女眷發放邊疆做奴,十五歲以下男童連坐犯被押解至北京閹割。

乾隆五十八年(1793年),英國遣使喬治·馬戛爾尼于乾隆83歲時到中國尋求駐節,但雙方出現與乾隆皇帝會面採「單膝下跪」(英方主張)或「三跪九叩」(中方主張)的禮儀之爭,最后以“单膝下跪”而为礼。

喬治·馬戛爾尼在回國後向英国议会写出报告:“中国是一艘破旧的大船,150年来,它之所以没有倾覆,是因为幸运的遇见了极为谨慎的船长。一旦赶上昏庸的船长,这艘大船随时就可能沉没。中国根本就没有现代的军事工业,中国的军事实力比英国差三到四个世纪”。而在馬戛爾尼的日記中卻有以下記載:“中国工业虽有数种,远出吾欧人之上,然以全体而论,化学上及医学上之知识,实处于极幼稚之地位。”,又稱:「中國政府的行政機制和權力是如此的有組織和高效,有條件能夠迅即排除萬難,創造任何成就。」。

為了打擊腐败之風,乾隆鼓励人們秘密向他汇报官员们的可疑行为,收受贿赂、欺诈、任人唯亲、滥用职权和瞒报等,例如福建大獄案,至于控诉的真假则由皇帝决定,在其统治初期坚定了惩治贪腐的决心,下令任何案件只要涉赃额超过一千两,案犯就将斩立决;然而到了乾隆统治的後半期,官员贪污这一严重问题再次出现,到了晚期每隔几年就会爆出一些重大案件及弹劾案,當時年迈的乾隆已经没有初時的魄力去严惩官员们的渎职行为,有學者指出:「從乾隆看来,在这些欺诈行为中也存在一些积极因素,其中之一便是所有被没收的贪官污吏的家产都流入了乾隆的腰包,大大增加了他的财富。而财政赤字和粮食亏空则由那些被免官员的继任者负责。另一个积极因素是满、汉官员都卷入了这种犯罪,这样乾隆就无须担心存在汉官通过腐败来故意破坏国家政治体制的阴谋。但是,看到那些本应更加效忠皇帝的满洲官员同样也在做着有损皇帝统治之事时,乾隆也会感到不太舒服。不过,好在还有一些值得依靠的、公正廉明的官员让乾隆感到些许安心,这些人对乾隆总是以诚相待,不收受贿赂,不会为了一己私利而欺君罔上。他们之中多数是满洲人,包括阿桂和傅恒」。

乾隆五十年之後,睡眠减少,“寅初已懒睡,寅正无不醒。”,左眼视力下降,年过七十之后,“昨日之事,今日辄忘;早间所行,晚或不省。”

乾隆25歲登基時表示過若蒙眷佑,得在位六十年,即當傳位嗣子,不敢上同皇祖紀元六十一載之數。因此在乾隆六十年九月初三日(1795年10月15日)85歲的乾隆將皇位傳予十五子顒琰(嘉庆帝),自稱太上皇,但軍國大事及用人皆由乾隆躬親指教,嘉慶帝朝夕敬聆訓聽;宫中仍用“乾隆”年号,中国第一历史档案馆藏《万岁爷进药底簿》封皮上书“乾隆六十四年”。嘉慶四年正月初三日(1799年2月7日),乾隆太上皇駕崩於北京紫禁城養心殿內,享壽八十八歲,結束了長達六十三年又四個月的統治。廟號清高宗,諡號純皇帝。死後與其愛妻孝賢純皇后合葬於清裕陵。

嘉庆四年二月二十一日,总管张进喜传旨交如意馆绘画太上皇帝圣容一轴,大边上花纹按照安佑宫供奉的圣祖和世宗圣容挂轴上的大边花纹式样绘画。圆明园四十景之鸿慈永祜的主体建筑安佑宫殿內便供奉他的画像。嘉庆四年四月的圆明园文开稱,安佑宫供奉高宗纯皇帝圣容,照例供献用宫香饼一觔、小黑芸香五两和小白芸香五两。

1928年,乾隆去世近一百三十年後,軍閥孫殿英看準了乾隆帝陵墓及慈禧太后陵墓的珍貴財寶,藉演習之名,率其部下盜掘乾隆帝及慈禧太后之陵墓。士兵為得棺內珠寶,將乾隆梓棺劈開並大肆搜掠,乾隆帝后遺骸四散在地,情況奇慘;及後溥儀派人前往收拾,亦只能找回部份遺骸,勉強砌回主體,並將帝后遺骸合葬一棺,重新行葬。

乾隆帝好詩、書、畫,作品極多,作詩多達四萬首(38630首)。其作品多采用「御題」做题跋。紫禁城宮殿內絕大部份的匾額,楹聯,亦是出自其御筆。乾隆有在宮中收藏的名家書畫上題詩用印的嗜好,被認為有一定的史料價值,但这种行为也破坏了原作品的艺术价值。

乾隆五十七年,乾隆親自撰寫成《十全武功記》,自詡「十全老人」。命人以滿、漢、蒙、藏四種文字刻碑,昭示其武功。「十全武功」指“平準噶爾為二,定回部為一,掃金川為二,靖台灣為一,降緬甸、安南各一,即今二次受廓爾喀降,合為十”。

在各种民间传说中,乾隆帝被描绘成风流天子。民国初年,就盛行香妃的传说。至今,关于香妃以及乾隆帝与平民女子的爱情故事为主题的各类文学、戏剧、影视作品,络绎不绝。另外在大臣中,乾隆帝对傅恒之子福康安最为优待。民国后,多传说福康安为他与傅恒妻的私生子,但黄一农等学者已考证此说不确。

民间对乾隆帝六次南巡亦多有演绎,或称之“乾隆下江南”。当代广告中,声称乾隆帝在南巡过程中曾品尝过某种美食的例子不胜枚举。

坊間野史指乾隆之母为汉人(非汉军旗),甚至有出自漢人之家(海寧陳家),並非雍正帝親生子之說,故認為乾隆为漢族后裔,但未被完全證實。乾隆帝的六次南巡亦被认为是去浙江海宁探望亲生父母陈世倌夫妇。金庸所著的武俠小說《書劍恩仇錄》正是以此傳說為藍本。

另外,關於乾隆出生之處也有爭議,一說在雍親王府(雍和宮),另一說則是在承德避暑山莊獅子園,而且避暑山莊一說是由嘉慶皇帝親口提起,這也是野史會傳出乾隆是由避暑山莊漢人宮女所生的原因。

\subsection{乾隆}

\begin{longtable}{|>{\centering\scriptsize}m{2em}|>{\centering\scriptsize}m{1.3em}|>{\centering}m{8.8em}|}
  % \caption{秦王政}\
  \toprule
  \SimHei \normalsize 年数 & \SimHei \scriptsize 公元 & \SimHei 大事件 \tabularnewline
  % \midrule
  \endfirsthead
  \toprule
  \SimHei \normalsize 年数 & \SimHei \scriptsize 公元 & \SimHei 大事件 \tabularnewline
  \midrule
  \endhead
  \midrule
  元年 & 1736 & \tabularnewline\hline
  二年 & 1737 & \tabularnewline\hline
  三年 & 1738 & \tabularnewline\hline
  四年 & 1739 & \tabularnewline\hline
  五年 & 1740 & \tabularnewline\hline
  六年 & 1741 & \tabularnewline\hline
  七年 & 1742 & \tabularnewline\hline
  八年 & 1743 & \tabularnewline\hline
  九年 & 1744 & \tabularnewline\hline
  十年 & 1745 & \tabularnewline\hline
  十一年 & 1746 & \tabularnewline\hline
  十二年 & 1747 & \tabularnewline\hline
  十三年 & 1748 & \tabularnewline\hline
  十四年 & 1749 & \tabularnewline\hline
  十五年 & 1750 & \tabularnewline\hline
  十六年 & 1751 & \tabularnewline\hline
  十七年 & 1752 & \tabularnewline\hline
  十八年 & 1753 & \tabularnewline\hline
  十九年 & 1754 & \tabularnewline\hline
  二十年 & 1755 & \tabularnewline\hline
  二一年 & 1756 & \tabularnewline\hline
  二二年 & 1757 & \tabularnewline\hline
  二三年 & 1758 & \tabularnewline\hline
  二四年 & 1759 & \tabularnewline\hline
  二五年 & 1760 & \tabularnewline\hline
  二六年 & 1761 & \tabularnewline\hline
  二七年 & 1762 & \tabularnewline\hline
  二八年 & 1763 & \tabularnewline\hline
  二九年 & 1764 & \tabularnewline\hline
  三十年 & 1765 & \tabularnewline\hline
  三一年 & 1766 & \tabularnewline\hline
  三二年 & 1767 & \tabularnewline\hline
  三三年 & 1768 & \tabularnewline\hline
  三四年 & 1769 & \tabularnewline\hline
  三五年 & 1770 & \tabularnewline\hline
  三六年 & 1771 & \tabularnewline\hline
  三七年 & 1772 & \tabularnewline\hline
  三八年 & 1773 & \tabularnewline\hline
  三九年 & 1774 & \tabularnewline\hline
  四十年 & 1775 & \tabularnewline\hline
  四一年 & 1776 & \tabularnewline\hline
  四二年 & 1777 & \tabularnewline\hline
  四三年 & 1778 & \tabularnewline\hline
  四四年 & 1779 & \tabularnewline\hline
  四五年 & 1780 & \tabularnewline\hline
  四六年 & 1781 & \tabularnewline\hline
  四七年 & 1782 & \tabularnewline\hline
  四八年 & 1783 & \tabularnewline\hline
  四九年 & 1784 & \tabularnewline\hline
  五十年 & 1785 & \tabularnewline\hline
  五一年 & 1786 & \tabularnewline\hline
  五二年 & 1787 & \tabularnewline\hline
  五三年 & 1788 & \tabularnewline\hline
  五四年 & 1789 & \tabularnewline\hline
  五五年 & 1790 & \tabularnewline\hline
  五六年 & 1791 & \tabularnewline\hline
  五七年 & 1792 & \tabularnewline\hline
  五八年 & 1793 & \tabularnewline\hline
  五九年 & 1794 & \tabularnewline\hline
  六十年 & 1795 & \tabularnewline
  \bottomrule
\end{longtable}


%%% Local Variables:
%%% mode: latex
%%% TeX-engine: xetex
%%% TeX-master: "../Main"
%%% End:

%% -*- coding: utf-8 -*-
%% Time-stamp: <Chen Wang: 2018-07-12 22:18:34>

\section{仁宗\tiny(1795-1820)}

\subsection{嘉庆}

\begin{longtable}{|>{\centering\scriptsize}m{2em}|>{\centering\scriptsize}m{1.3em}|>{\centering}m{8.8em}|}
  % \caption{秦王政}\
  \toprule
  \SimHei \normalsize 年数 & \SimHei \scriptsize 公元 & \SimHei 大事件 \tabularnewline
  % \midrule
  \endfirsthead
  \toprule
  \SimHei \normalsize 年数 & \SimHei \scriptsize 公元 & \SimHei 大事件 \tabularnewline
  \midrule
  \endhead
  \midrule
  元年 & 1796 & \tabularnewline\hline
  二年 & 1797 & \tabularnewline\hline
  三年 & 1798 & \tabularnewline\hline
  四年 & 1799 & \tabularnewline\hline
  五年 & 1800 & \tabularnewline\hline
  六年 & 1801 & \tabularnewline\hline
  七年 & 1802 & \tabularnewline\hline
  八年 & 1803 & \tabularnewline\hline
  九年 & 1804 & \tabularnewline\hline
  十年 & 1805 & \tabularnewline\hline
  十一年 & 1806 & \tabularnewline\hline
  十二年 & 1807 & \tabularnewline\hline
  十三年 & 1808 & \tabularnewline\hline
  十四年 & 1809 & \tabularnewline\hline
  十五年 & 1810 & \tabularnewline\hline
  十六年 & 1811 & \tabularnewline\hline
  十七年 & 1812 & \tabularnewline\hline
  十八年 & 1813 & \tabularnewline\hline
  十九年 & 1814 & \tabularnewline\hline
  二十年 & 1815 & \tabularnewline\hline
  二一年 & 1816 & \tabularnewline\hline
  二二年 & 1817 & \tabularnewline\hline
  二三年 & 1818 & \tabularnewline\hline
  二四年 & 1819 & \tabularnewline\hline
  二五年 & 1820 & \tabularnewline
  \bottomrule
\end{longtable}


%%% Local Variables:
%%% mode: latex
%%% TeX-engine: xetex
%%% TeX-master: "../Main"
%%% End:

%% -*- coding: utf-8 -*-
%% Time-stamp: <Chen Wang: 2021-11-01 17:22:41>

\section{宣宗道光帝旻寧\tiny(1821-1850)}

\subsection{生平}

道光帝(1782年9月16日-1850年2月26日),名旻寧,爱新觉罗氏,是清朝自入关以来的第六位皇帝,1820年10月3日至1850年2月26日在位,年号「道光」。西藏方面尊為「文殊皇帝」。

道光帝乃嘉庆帝次子,生母為孝淑睿皇后喜塔腊氏。原名綿寧,即位後為避免他人避諱麻煩而改名旻寧。他是清朝历史上僅有一位以嫡長子身分繼承皇位的皇帝。死後廟號宣宗,諡號成皇帝,葬于清西陵中的慕陵。

道光皇帝原名绵宁,1782年9月16日(乾隆四十七年八月初十日)出生於紫禁城擷芳殿。他出生时,父亲嘉庆帝颙琰尚为普通的皇子,母亲喜塔拉氏为颙琰福晋(嫡妻)。绵宁出生之前,嘉庆帝长子已夭折。绵宁成为他实际上的长子。绵宁從小即十分聰明,據說是祖父乾隆帝最疼愛的孫子,乾隆五十六年,賜黃馬褂,賞戴雙眼花翎。

乾隆帝执政的后期,父亲颙琰被立太子,乾隆帝禪讓,1796年(嘉庆元年)颙琰登基,同年,绵宁娶妻鈕祜祿氏。

1799年(嘉庆四年四月),嘉庆帝依照秘密建儲制,立绵宁其为太子。嘉庆十三年(1808年),他的第一个孩子奕纬出生。

道光帝在繼位之前,其騎射武功在嘉慶帝諸子裡相當聞名,亦習得一手好槍法。嘉庆十八年(1813年),這年旻寧33歲,因天理教癸酉之变,取出宮中禁用的鳥銃,連殺二敵的英勇表现,封为「智亲王」,所執的鳥銃也被封為“威烈”。根據中央研究院歷史語言研究所內閣大庫檔案038280號,嘉慶25年封智親王、皇太子同時接任皇帝。

嘉慶二十五年(1820年)七月十八日,嘉慶帝到熱河木蘭圍獵,命皇次子智親王綿寧、皇四子瑞親王綿忻隨駕。這年他35歲,“身體豐腴,精神強固”。二十四日,到達熱河行宮,“聖躬不豫”。當天,嘉慶帝到城隍廟燒香,又到永佑宮行禮,二十五日,病情嚴重,當夜崩逝,死因不明,據今日推測,嘉慶帝死亡的原因可能是年逾花甲,身體肥胖,天氣暑熱,旅途勞頓,誘發腦中風或心臟病而暴斃。旻寧繼位,得以禧恩為代表的宗室之建議和認同,又得到皇太后的中宮懿旨和皇弟瑞親王綿忻的贊同,最主要是有軍機大臣等開啟鐍匣的密諭。

绵宁继位後,免眾兄弟避諱改名比照父親顒琰改名,所以改名旻宁,定年號為道光。即位时正值鸦片氾濫,道光帝为挽救国家财政危机,也主张禁烟,多次下诏禁鸦片进口,禁止自种自制。之後鴉片戰爭爆發,由于道光帝战守无策,時和時戰,再加上武器装备上的差距,清朝战败于英国,并与英人簽訂近代中国的首條不平等條約──《南京條約》,割讓香港岛及開放五口通商。

道光年間,推行三項改革措施:漕糧海運、改綱鹽法為票鹽法、允許開採礦產。

道光三十年正月十五曰(1850年2月26日),道光帝因堅持為其繼母恭慈皇太后守靈,以致生病,在圓明園九洲清晏殿駕崩,享壽六十八歲。安葬于清慕陵(今河北省易县西)。

咸丰元年正月初六日,库掌祥麟持来报单一件,内开咸丰元年正月初四日总管内务府大臣面奉谕旨,恭绘热河绥成殿宣宗成皇帝圣容,著沈振麟于正月底吉日敬谨恭绘。咸豐帝要求圣容要盘膝坐,前方设置放着《易经》首页乾卦的書案。

历史学家孟森认为:“宣宗之庸暗,亦为清朝入关以来所未有。”称这时期为「嘉道中衰」。

蔡東藩:「徒齊其末,未揣其本,省衣減膳之為,治家有餘,治國不足。」

大事年表:乾隆四十七年八月初十日,綿寧在擷芳殿出生。嘉慶元年,以钮祜禄氏為嫡福晋。嘉慶十八年九月,封為智親王。嘉慶二十五年七月,仁宗去世,綿寧繼位,更名旻寧。道光八年,平定在西域新疆地區西南部為期八年的張格爾之亂,嚴禁新疆與支持白山派和卓張格爾反清的浩罕汗國通商。道光十一年,浩罕汗國遣使議和進貢。道光十二年,准許西域新疆地區與浩罕汗國重開貿易。道光十八年閏四月,黃爵滋奏請「將內地吸食鴉片者俱罪死」。十一月命林則徐為欽差大臣,赴廣東查禁鴉片。道光十九年四月廿二日,虎門銷煙開始。道光二十年五月二十九日,英艦封鎖廣州珠江口,鴉片戰爭正式開始。英艦北上,六月攻陷浙江定海,七月抵達天津附近,其後返回廣東。九月林則徐被革職。琦善與英方全權代表義律商議和約,十二月義律單方面公佈《穿鼻草約》。同年,位於喀什米爾地區東南部的拉達克王國,面臨錫克帝國查謨拉者古拉卜·辛格派遣左拉瓦爾·辛格兵團(主帥全名左拉瓦爾·辛格·卡赫盧里拉)進攻的亡國危機,遣使向大清駐藏大臣求援遭拒。道光二十一年正月,英軍佔領香港。道光帝不承認《穿鼻草約》,二月琦善被革職,押京審理。五月,錫克帝國屬地查謨-克什米爾地區多格拉人的左拉瓦爾·辛格·卡赫盧里拉兵團趁併吞拉達克王國的滅國威勢,進攻清屬西藏阿里地區,爆發森巴戰爭(藏人稱多格拉人為森巴)。道光二十二年七月,英軍兵臨南京,清廷同意議和,《南京條約》立。冬季,西藏阿里地區西北部爆發的森巴戰爭,以錫克帝國多格拉兵團主帥左拉瓦爾·辛格·卡赫盧里拉和駐藏清軍交戰陣亡、餘眾敗走告終。道光二十三年八月,《中英五口通商章程》立。道光二十六年正月,正式解除對天主教的禁令。道光二十七年,平定在西域新疆地區西南部爆發的七和卓之亂。道光三十年正月,道光帝在圓明園去世。

\subsection{道光}

\begin{longtable}{|>{\centering\scriptsize}m{2em}|>{\centering\scriptsize}m{1.3em}|>{\centering}m{8.8em}|}
  % \caption{秦王政}\
  \toprule
  \SimHei \normalsize 年数 & \SimHei \scriptsize 公元 & \SimHei 大事件 \tabularnewline
  % \midrule
  \endfirsthead
  \toprule
  \SimHei \normalsize 年数 & \SimHei \scriptsize 公元 & \SimHei 大事件 \tabularnewline
  \midrule
  \endhead
  \midrule
  元年 & 1821 & \tabularnewline\hline
  二年 & 1822 & \tabularnewline\hline
  三年 & 1823 & \tabularnewline\hline
  四年 & 1824 & \tabularnewline\hline
  五年 & 1825 & \tabularnewline\hline
  六年 & 1826 & \tabularnewline\hline
  七年 & 1827 & \tabularnewline\hline
  八年 & 1828 & \tabularnewline\hline
  九年 & 1829 & \tabularnewline\hline
  十年 & 1830 & \tabularnewline\hline
  十一年 & 1831 & \tabularnewline\hline
  十二年 & 1832 & \tabularnewline\hline
  十三年 & 1833 & \tabularnewline\hline
  十四年 & 1834 & \tabularnewline\hline
  十五年 & 1835 & \tabularnewline\hline
  十六年 & 1836 & \tabularnewline\hline
  十七年 & 1837 & \tabularnewline\hline
  十八年 & 1838 & \tabularnewline\hline
  十九年 & 1839 & \tabularnewline\hline
  二十年 & 1840 & \tabularnewline\hline
  二一年 & 1841 & \tabularnewline\hline
  二二年 & 1842 & \tabularnewline\hline
  二三年 & 1843 & \tabularnewline\hline
  二四年 & 1844 & \tabularnewline\hline
  二五年 & 1845 & \tabularnewline\hline
  二六年 & 1846 & \tabularnewline\hline
  二七年 & 1847 & \tabularnewline\hline
  二八年 & 1848 & \tabularnewline\hline
  二九年 & 1849 & \tabularnewline\hline
  三十年 & 1850 & \tabularnewline
  \bottomrule
\end{longtable}


%%% Local Variables:
%%% mode: latex
%%% TeX-engine: xetex
%%% TeX-master: "../Main"
%%% End:

%% -*- coding: utf-8 -*-
%% Time-stamp: <Chen Wang: 2021-11-01 17:23:03>

\section{文宗咸豐帝奕詝\tiny(1850-1861)}

\subsection{生平}

咸豐帝(1831年7月17日-1861年8月22日),爱新觉罗氏,名奕詝,號且樂道人,是清朝自入关以来的第九位皇帝,1850年3月9日至1861年8月22日在位,年号「咸豐」。西藏方面尊為「文殊皇帝」。

咸豐帝是道光帝第四子,生母為孝全成皇后钮祜禄氏,誕於北京圆明园澄静斋,养母為孝静成皇后博尔济吉特氏。20岁登基,1861年崩于承德避暑山庄烟波致爽殿,享年30岁。死后庙号文宗,谥号簡稱显皇帝,葬于清东陵中的定陵。他也是清朝最後一位掌握實權的皇帝與最後一位儲位密建的皇帝。

1831年7月17日(道光十一年六月初九日),咸丰帝生于北京圆明园之澄静斋。时道光帝前三个儿子都已去世,咸丰帝出生后即为在世的皇长子。二十六年,按照秘密立储制度,被道光帝立为储君。三十年正月丁未,道光帝驾崩前,宣召大臣开启鐍匣,立为皇太子。

咸豐帝即位後便勤於政事,广开言路、明詔求賢,先後將有损国家利益的穆彰阿和耆英革職,大手笔地对朝政颇有改革。但此時的大清帝国內憂外患不斷,先後爆发太平天国运动以及第二次鸦片战争。

在第二次鸦片战争中,俄国西伯利亚总督穆拉维约夫迫使黑龙江将军奕山签订《瑷珲条约》,割去黑龙江以北、外兴安岭以南原属清朝的领土约60万平方公里,咸丰帝拒绝承认该条约。随后英法联军进攻北京,咸丰帝下诏对英国法国宣战:“兵家胜败何常,该国兵远来即有数万,未可当我中国人民千百之一,其能经几战乎?”最后圆明园、清漪园等被焚掠,以签定一系列不平等条约收场。

咸丰十一年(1861年)七月十七日,咸丰帝崩于行在避暑山庄烟波致爽殿。皇长子同治帝继位。同时依遺詔,由肃顺、载垣、端华、景寿、穆荫、匡源、杜翰、焦祐瀛为顾命八大臣,肅順為首,輔導皇帝施政。在咸丰灵柩启程返京期间,東太后慈安、西太后慈禧、恭親王奕訢、醇亲王奕譞四人联合發動辛酉政变,醇亲王奕譞亲自抓捕肃顺,随之八大臣非死即貶。同时,政府随即由慈安、慈禧两宫听政,咸丰帝生前的輔政遺命宣告廢止。

根據内务府档案的记载,如意馆画士沈振麟曾在同治年间绘制了两幅先帝咸丰帝圣容,这两幅圣容均先画稿,呈览后再进一步绘制。和硕恭亲王奕訢向两宫皇太后和同治帝呈览的两幅墨稿分别为便衣墨稿一件及道装配山洞景致墨稿一件。

清朝皇帝的評價中,咸豐帝的爭議最大。咸丰爱看戏,爱唱戏,即使到热河行在唱戏,“着升平署三拔至热河”,也表現得乐不思蜀。咸丰一朝,财政十分困难,要镇压太平天国,对付英法联军,财源枯竭,“户部因军兴财匮,行钞,置宝钞处,行大钱,置官钱总局,分领其事”,钞票大量发行,造成通货膨胀,“官民交累,徒滋弊窦”。

咸丰帝“任賢擢才,洞觀肆應”,在面對太平天国运动與“三千年未有之政局”的內憂外患中,咸丰指揮若定,重用汉族大臣曾国藩人等组织团练来对付太平天国,咸丰帝颇思除弊求治,提拔行事果断的肃顺,并支持肃顺等人在朝政上推行的改革。為後來的同光中興打下良好基礎。但也因為採納肃顺對英法兩國強硬的做法,導致引發英法聯軍之役。

咸丰帝临终前对朝政事宜不合理的安排和权力制衡,使朝臣和后宫在有关朝政和国事方面展开逐鹿,间接导致慈安太后、慈禧太后、恭親王奕訢、醇亲王奕譞联手,打倒了顾命八大臣,最後慈禧掌权近半个世纪,从而也被有些史家流派认为咸丰帝没有安排好善后事宜,致使后宫干政近半个世纪的局面。

英、法等國要求清廷能讓英法兩國在北京設置駐京公使,新任公使到任時能觐見皇帝,但咸丰帝不接受。英國也要求清廷開放中國貿易,咸丰帝也拒絕,雖然天津條約簽訂後,咸豐帝默許了英、法等國的要求,但又對英、法等國的公使刁難(主因是採納肃顺對英法兩國強硬的做法)。咸丰帝本人的守舊,間接導致英法聯軍之役清朝的慘敗,使清朝失去首都北京,圓明園也遭聯軍焚毀。

太平天国运动與英法聯軍之役,使咸丰帝的執政遭受打擊,逃往熱河行宮後就病逝了,太平天国运动與英法聯軍之役的善後工作,直到慈安、慈禧两宫听政時期才結束。

大事年表:道光十一年六月初九日,奕詝出生。后受教于杜受田。道光三十年正月,宣宗去世,奕詝繼位。是年太平天國起事。咸豐三年二月,太平軍攻佔江寧,定都在此,改名天京。九月,太平天國北伐軍逼近天津。是年曾國藩建湘軍。咸豐五年四月,李開芳被俘,太平天國北伐軍覆沒。咸豐六年八月,天京事變發生。九月,「亞羅號事件」發生。咸豐八年四月,與俄國簽訂《璦琿條約》。五月,先後與俄、美、英、法四國簽訂《中俄天津條約》、《中美天津條約》、《中英天津條約》及《中法天津條約》。十月,太平軍取得三河大捷。咸豐九年五月,清大沽守軍擊退英、法艦隊。咸豐十年七月,英法聯軍攻佔天津和大沽一帶。八月,八里橋和大沽口相繼被攻佔,咸豐帝逃往承德,亦不足兩個月,英法聯軍火燒圓明園,進佔北京。十月,《中英北京條約》及《中法北京條約》立。十一月,《中俄北京條約》立。十二月總理各國事務衙門成立。咸豐十一年七月十七日,咸豐皇帝駕崩於承德避暑山庄烟波致爽殿,年三十。其子载淳年仅六岁,继承大统,是為同治帝。咸丰委派载垣、端华、景寿、肃顺、穆荫、匡源、杜翰、焦祐瀛为辅政八大臣,辅助小皇帝。

\subsection{咸丰}

\begin{longtable}{|>{\centering\scriptsize}m{2em}|>{\centering\scriptsize}m{1.3em}|>{\centering}m{8.8em}|}
  % \caption{秦王政}\
  \toprule
  \SimHei \normalsize 年数 & \SimHei \scriptsize 公元 & \SimHei 大事件 \tabularnewline
  % \midrule
  \endfirsthead
  \toprule
  \SimHei \normalsize 年数 & \SimHei \scriptsize 公元 & \SimHei 大事件 \tabularnewline
  \midrule
  \endhead
  \midrule
  元年 & 1851 & \tabularnewline\hline
  二年 & 1852 & \tabularnewline\hline
  三年 & 1853 & \tabularnewline\hline
  四年 & 1854 & \tabularnewline\hline
  五年 & 1855 & \tabularnewline\hline
  六年 & 1856 & \tabularnewline\hline
  七年 & 1857 & \tabularnewline\hline
  八年 & 1858 & \tabularnewline\hline
  九年 & 1859 & \tabularnewline\hline
  十年 & 1860 & \tabularnewline\hline
  十一年 & 1861 & \tabularnewline
  \bottomrule
\end{longtable}


%%% Local Variables:
%%% mode: latex
%%% TeX-engine: xetex
%%% TeX-master: "../Main"
%%% End:

%% -*- coding: utf-8 -*-
%% Time-stamp: <Chen Wang: 2018-07-12 22:21:05>

\section{穆宗\tiny(1861-1875)}

\subsection{同治}

\begin{longtable}{|>{\centering\scriptsize}m{2em}|>{\centering\scriptsize}m{1.3em}|>{\centering}m{8.8em}|}
  % \caption{秦王政}\
  \toprule
  \SimHei \normalsize 年数 & \SimHei \scriptsize 公元 & \SimHei 大事件 \tabularnewline
  % \midrule
  \endfirsthead
  \toprule
  \SimHei \normalsize 年数 & \SimHei \scriptsize 公元 & \SimHei 大事件 \tabularnewline
  \midrule
  \endhead
  \midrule
  元年 & 1862 & \tabularnewline\hline
  二年 & 1863 & \tabularnewline\hline
  三年 & 1864 & \tabularnewline\hline
  四年 & 1865 & \tabularnewline\hline
  五年 & 1866 & \tabularnewline\hline
  六年 & 1867 & \tabularnewline\hline
  七年 & 1868 & \tabularnewline\hline
  八年 & 1869 & \tabularnewline\hline
  九年 & 1870 & \tabularnewline\hline
  十年 & 1871 & \tabularnewline\hline
  十一年 & 1872 & \tabularnewline\hline
  十二年 & 1873 & \tabularnewline\hline
  十三年 & 1874 & \tabularnewline
  \bottomrule
\end{longtable}


%%% Local Variables:
%%% mode: latex
%%% TeX-engine: xetex
%%% TeX-master: "../Main"
%%% End:

%% -*- coding: utf-8 -*-
%% Time-stamp: <Chen Wang: 2019-12-26 22:00:54>

\section{德宗\tiny(1875-1908)}

\subsection{生平}

光緒帝(1871年8月14日-1908年11月14日),名載湉,爱新觉罗氏,是清朝自入关以来的第九位皇帝,同時是中國最後一位有正式諡號及正式廟號的皇帝,1875年2月25日至1908年11月14日在位,年号「光绪」。

光緒帝是醇贤亲王奕譞次子,也是道光帝之孙。同治帝駕崩後,他以三歲沖齡過繼給咸丰帝,因而繼承了皇位。他在幼年時由慈安太后及慈禧太后兩宮聽政。在位期间,历经甲午战争和戊戌變法。1898年戊戌變法失败后,被慈禧太后禁闭在中南海瀛台。1908年,慈禧死之前一日,光绪帝在中南海瀛台死於砒霜中毒。死后庙号德宗,谥号景皇帝,葬于清西陵中的崇陵。

他为前一位皇帝——同治帝的堂弟兼表弟。其父为宣宗(道光帝)第七子醇贤亲王奕譞,生母嫡福晋那拉氏为慈禧太后之妹。因穆宗(同治帝)为文宗(咸丰帝)独子,又早死无後,慈禧太后便以和自己血缘最近的载湉,過繼于咸丰帝,登基为帝,名义上继承咸丰帝而非同治帝的皇位。


四岁即位,主少國疑,大臣未附,兩宮太后姑允王大臣所請,依《太后垂簾章程》十一條,垂簾聽政。

光緒七年(1881年),慈安太后駕崩,慈禧太后獨自垂簾聽政。

光緒九年(1883年),中法戰爭爆發,翌年簽定《中法新約》。

光绪帝一生受慈禧太后的控制,自小由翁同龢做他的老師,但慈禧太后規定翁同龢只能教孝經,更被李連英監視。朝廷大权在成年(1890年,20歲)後,仍掌握在慈禧太后手中。

光绪十四年(1888年),光绪帝已十八岁(虚岁),而在达到他这个年龄之前,包括同治帝在内的幼年继位的清帝均已完成大婚并亲政。六月己亥,慈禧太后颁布懿旨,明年正月举行皇帝大婚典礼。婚礼完成,光绪帝即应亲裁大政。壬寅,再颁懿旨,明年二月初三日归政。慈禧太后选择自己的侄女亦是光绪帝的表姐叶赫那拉氏为其皇后。

光绪十五年正月丁卯,御史屠仁守请求慈禧太后在归政后继续批阅奏折,被斥“乖谬”。癸酉,清廷如期举行大婚典礼。此次大婚相比同治帝大婚,花费较少。二月戊寅,慈禧太后颂懿旨斥责吴大澂要求尊崇光绪帝的生父醇亲王奕譞的请求,并出示醇亲王在光绪元年的奏折,表明醇亲王的忠心。己卯,慈禧太后正式归政,宣告十九岁的光绪帝亲政。慈禧太后勉允禮親王領班軍機大臣世鐸等王大臣所請,於皇帝親政後再訓政數年。中外同辭,再三瀝懇,慈禧不得不依據《訓政細則》開始訓政,不需垂簾亦無須議政王引見大臣,其餘細則與《垂簾章程》略同,实际大权仍掌握在慈禧太后手中。此後,光緒帝逐渐建立了以翁同龢、汪鸣銮、孙家鼐、文廷式、志锐等为骨干的帝党,影響光緒親政後的作為。

光绪二十年(1894年)甲午战争爆发,坚决主战,但光緒指揮不當,加上翁同龢、李鴻章之間内斗严重,導致清朝战败,次年在《马关条约》上签字用玺。

自甲午战败后,光緒帝锐意变法革新,“不做亡国之君”,1898年頒布《明定國是詔》,表明變革決心。在慈禧的默許下,于1898年起用康有为、梁启超等推行新政,并以谭嗣同、楊銳、林旭、劉光第等四军卿架空原有的军机大臣,但受到保守派的反对。光緒在軍事上,陸軍改練洋操,為掌握軍事召袁世凱來京,下旨進行一系列整頓:
國家振興庶政,兼採西法,誠以為民立政,中西所同,而西法可補我所未及。…...今將變法之意,佈告天下,使百姓咸喻朕心,共知其君之可恃,上下同心,以成新政,以強中國,朕不勝厚望焉。

光緒二十四年(1898年)8、9月间,由于變法操之過急,坊间盛传慈禧太后有以借“天津阅兵”废弑光绪帝的阴谋。光緒帝懼怕變法失敗,聽信康有為的意見,打算不經過慈禧太后同意,亲自提拔候补侍郎袁世凯,以新式陸軍发兵,杀慈禧提拔的直隶总督兼北洋大臣荣禄,围颐和园(慈禧所居)。慈禧得知消息,立刻從颐和园返回紫禁城,發動政變幽禁光緒帝,戊戌變法宣布失敗,軍機處譚嗣同、楊銳、林旭、劉光第四軍卿以及楊深秀、康廣仁等六維新派人士被捕處死,康有為、梁啟超流亡日本,光緒帝被慈禧幽禁在三面环水的南海瀛台,对外则宣布光绪帝生病,由慈禧训政。从戊戌年(1898年)四月二十三日光绪帝下「明定国是诏」起,到政变发生的八月六日为止(西历6月11日至9月21日),整个变法维新历时不过103天,故称百日维新。

戊戌政變後,慈禧太后宣布訓政,架空光緒。光緒二十五年(己亥年)十二月二十四日(公元1900年1月24日),慈禧太后欲废光绪帝,挑选载漪之子溥儁入宫,成為光緒的義子,是为己亥立储,由於這是廢立皇帝的先兆,上海電報局總辦經元善領銜,與馬裕藻、章太炎、丁惠康、沈藎,唐才常、經亨頤、蔡元培、黃炎培等聯名抗議,且各国公使都同情光緒,否認此事的合法性,導致慈禧失敗。慈禧遂不斷召外國西醫入宮探視“上疾”。

溥儁之父载漪等权贵利用刚刚兴起的义和团排洋情绪,招引义和团进京,发生庚子事变,光绪帝與慈禧太后共同参加决定是否向八国联军宣战的御前会议,光緒表達反對與八國聯軍開戰,但他已沒有親政的權力。[來源請求]八国联军攻入北京,慈禧挟光緒帝逃至西安,並殺害珍妃。次年签定辛丑条约(庚子事变赔款)後才回北京。此后处境稍有改善,但仍被慈禧软禁。八國聯軍之後,慈禧太后啟動第三波的政治變革,稱為清末新政(或稱庚子新政)。

1908年11月14日,光緒帝逝於瀛台,比慈禧早一日驾崩,得年37歲。

清代官方文獻和宮廷檔案記載光緒帝為病死。但光緒帝在慈禧死前一日晏驾,時間過於巧合,外界對其死因歷來有諸多揣測。许多野史、宫廷回忆录包括溥仪均指出光绪帝是被人下毒所害,但对凶手的推测各不相同。中華民國成立之後,據光緒帝的御醫透露,皇帝生前的確身體並不非常健康,主因是長時間不見天日、身體欠運動、心情不佳導致飲食不正常,卻也無病重之跡象。1980年,整理崇陵光绪帝遗骨时“未发现外伤及中毒迹象”,结合官方档案上的说法,自然病死一说在当时一度成为学术界主流观点。直到2008年对清西陵光緒遺體的頭髮、遺骨、衣服及墓內外環境樣品进行检测分析后,证实光绪帝是砒霜中毒死亡。

光绪辞世时尚没有陵墓,一直到1913年(民国二年)才葬入中国最後一座帝陵——河北易县清西陵中的崇陵。1938年曾被盗。

《崇陵傳信錄》和《清稗類鈔》兩書指出:慈禧太后病危期間,曾猶豫對光緒帝要如何處置,遂以自己不久人世的消息透露給光緒帝知道,惟其近侍回報,帝曾微露喜色,故慈禧決意自己病終前,帝須先於自己命終,以免皇帝有再度親政、否定慈禧生前之佈局的可能。

清室後裔、書法家啟功指出,其曾祖父、時任禮部尚書的溥良曾親眼看到太監從病重的慈禧宮中傳出一個蓋碗,稱“是老佛爺賞給萬歲爺的塌喇”。“塌喇”在滿語中是酸奶之意。此前從未聽說過光緒帝有任何急症大病。送後不久,就由隆裕皇后的太監小德張(張蘭德)向太醫院正堂宣布光緒皇帝駕崩了。隨後樂壽宮才哭聲四起,宣布太后已死。慈禧與光緒素有嫌隙,況且當時慈禧已處於彌留之際,此時派人給軟禁中的皇帝贈食,極不尋常。啟功認為,慈禧可能先於光緒帝病死,但祕不發喪,直到確認光緒帝死亡後才對外公布死訊。

央視主任編輯鍾里滿依檢驗結果及史料記載認為,慈禧自戊戌政變以後就陰謀廢黜及弑害光緒,更擔心光緒會在自己死後復位翻案,所以才會在病危之時下毒手。

曾留洋并担任慈禧的御前女官的裕德龄在其英文版自述《瀛台泣血记》中提出,应是慈禧指使李莲英下手。

稱光緒帝为袁世凯所弑者认为,袁负恩反戈,陷光绪帝于万劫不复,光绪帝在瀛台,“日书‘项城’(袁世凯别号“袁项城”)名以志其愤”。袁既知光绪帝对其深恶痛绝,则不能不惧太后死而帝独生,故加以谋害(见于光緒侄、末帝溥仪所著《我的前半生》及其他)。但鍾里滿认为,当时除了慈禧太后外,并无其他人具备指使人对皇帝下毒的能力。袁世凯亦難以接近光绪帝。

虽然清政府公布的死亡日期是清曆十月二十一日(西曆1908年11月14日),但喻大华认为清政府推迟了光绪帝的死亡日期,光绪帝的起居注史官恽毓鼎在他的《崇陵传信录》中回忆,在此前两天的十月十九日,太监成群结队地出宫剃头,并毫不避讳地说皇上驾崩,因为国丧期间,服喪不允許理髮,所以抢在发布之前剃头。由此看来,朝廷發布的光绪帝死亡时间不準,在溥仪入宫之前光绪就已经死了,慈禧从容布置之后,确信可以掌控全局的时候才向天下公布皇帝已駕崩。日本电报也称光绪皇帝死于11月12日夜。

多數主流史學者(梁啟超、楊天石等)認為,光緒帝是清朝歷代皇帝之中較能接受新式制度的開明君主。甲午戰爭時主戰,不欲割讓台澎,展現其想保疆衛土的決心。但無奈其一生從未掌握實權,可謂心有餘而力不足。論及維新變法之失敗,亦當歸於改革操之過切,如王照稱光緒帝為瑣事一日罷去六堂官,致懷塔布之妻入宮向太后哭訴。在重大改革中,一百天內制定如此多的諭令,就算慈禧太后沒有最終加以阻撓,以當時清廷內外交困的情實,也難以逆睹預料成敗。唯帝師翁同龢稱光緒幼承孝貞庭訓,近臣惲毓鼎稱某大員入蜀,慈禧唯絮問,而光緒帝不發一語,臨陛見,忽然曰:“西藏事,其要在蜀,勞費心。”及至,果然,而入民國,告同僚:“帝有先見之明”。惲毓鼎對宣統而作《崇陵傳信錄》。

光緒帝自親政到戊戌政變,並沒法建立自己的班底,除了師傅翁同龢是自己的親信外,其餘皆是過去慈禧培養的官員,再加上官員任命權與國家大政,皆需要請示慈禧才能定奪,因此光緒帝親政後並沒有太大權力,導致在戊戌變法時,光緒帝想積極培養自己的親信班底,反而遭致朝內的反對,再加上用人不當、操之過急,最後失敗。

重用翁同龢、文廷式等人是光緒帝親政最大的失誤。翁同龢雖是光緒帝師傅,但性格過於守舊,得罪不少洋務派,並對慈安、慈禧兩宮太后推動的洋務運動嗤之以鼻。由於光緒帝親政後並沒有自己的班底,只有信任翁同龢。另一方面,翁同龢為刁難李鴻章的北洋艦隊,不惜上奏光緒帝禁止海軍外購軍火(《請停購船械裁減勇營折》),致使海軍失去了申請專項資金用於艦和、武備更新的途徑,導致北洋艦隊無法更新武器,在甲午戰爭全面戰敗。在甲午戰爭的指揮中,光緒帝指揮不當,導致甲午戰爭節節失敗,後期又與慈禧發生指揮上的衝突,等到慈禧接手指揮權時,只剩下與大日本帝國求和一途。

戊戌變法聽信康有為的策略,導致原本與慈禧已有默契的光緒帝,開始與慈禧產生摩擦。而康有為在翁同龢被罷官後,取代其位置成為光緒帝最親近的大臣,因康有為誤事,反而讓光緒帝苦吞戊戌變法的敗果。

光緒帝親自接見由親近大臣張蔭桓推薦的大日本帝國總理大臣伊藤博文,並接受其提供的改革方針。

光緒帝為了擺脫慈禧的控制,急於自己有班底,在戊戌變法中,並沒有看清時局的演變,再加上康有為、梁啟超的激進作法,導致一向默許光緒帝主導戊戌變法的慈禧太后終於忍無可忍,發動政變並囚禁光緒帝。[來源請求]

光緒帝相較於前一位同治帝,處理政務過於急躁,且用人不當,只聽信翁同龢、康有為之言,進而導致甲午戰爭、戊戌變法的失敗。

敘述光緒事蹟的書籍,《清史稿·德宗本紀》及清宮檔案自是第一信史,此外有清末民初惲毓鼎的《崇陵傳信錄》和清室遠支德齡的《瀛台泣血記》等。關於戊戌變法的資料則有梁啟超的《戊戌政變記》和袁世凱的《戊戌日記》。近年來有《走向共和》《戊戌風雲》等影視,亦頗有影響。

惲毓鼎(1862-1917)《崇陵傳信錄》「緬維先帝御宇,不為不久。幼而提攜,長而禁制,終於損其天年。無母子之親,無夫婦、昆季之愛,無臣下侍從宴遊暇豫之樂。平世齊民之福,且有勝於一人之尊者。毓鼎侍左右,近且久,天顏戚戚,常若不愉,未嘗一日展容舒氣也。」

\subsection{光绪}

\begin{longtable}{|>{\centering\scriptsize}m{2em}|>{\centering\scriptsize}m{1.3em}|>{\centering}m{8.8em}|}
  % \caption{秦王政}\
  \toprule
  \SimHei \normalsize 年数 & \SimHei \scriptsize 公元 & \SimHei 大事件 \tabularnewline
  % \midrule
  \endfirsthead
  \toprule
  \SimHei \normalsize 年数 & \SimHei \scriptsize 公元 & \SimHei 大事件 \tabularnewline
  \midrule
  \endhead
  \midrule
  元年 & 1875 & \tabularnewline\hline
  二年 & 1876 & \tabularnewline\hline
  三年 & 1877 & \tabularnewline\hline
  四年 & 1878 & \tabularnewline\hline
  五年 & 1879 & \tabularnewline\hline
  六年 & 1880 & \tabularnewline\hline
  七年 & 1881 & \tabularnewline\hline
  八年 & 1882 & \tabularnewline\hline
  九年 & 1883 & \tabularnewline\hline
  十年 & 1884 & \tabularnewline\hline
  十一年 & 1885 & \tabularnewline\hline
  十二年 & 1886 & \tabularnewline\hline
  十三年 & 1887 & \tabularnewline\hline
  十四年 & 1888 & \tabularnewline\hline
  十五年 & 1889 & \tabularnewline\hline
  十六年 & 1890 & \tabularnewline\hline
  十七年 & 1891 & \tabularnewline\hline
  十八年 & 1892 & \tabularnewline\hline
  十九年 & 1893 & \tabularnewline\hline
  二十年 & 1894 & \tabularnewline\hline
  二一年 & 1895 & \tabularnewline\hline
  二二年 & 1896 & \tabularnewline\hline
  二三年 & 1897 & \tabularnewline\hline
  二四年 & 1898 & \tabularnewline\hline
  二五年 & 1899 & \tabularnewline\hline
  二六年 & 1900 & \tabularnewline\hline
  二七年 & 1901 & \tabularnewline\hline
  二八年 & 1902 & \tabularnewline\hline
  二九年 & 1903 & \tabularnewline\hline
  三十年 & 1904 & \tabularnewline\hline
  三一年 & 1905 & \tabularnewline\hline
  三二年 & 1906 & \tabularnewline\hline
  三三年 & 1907 & \tabularnewline\hline
  三四年 & 1908 & \tabularnewline
  \bottomrule
\end{longtable}


%%% Local Variables:
%%% mode: latex
%%% TeX-engine: xetex
%%% TeX-master: "../Main"
%%% End:

%% -*- coding: utf-8 -*-
%% Time-stamp: <Chen Wang: 2018-07-12 22:22:42>

\section{溥仪\tiny(1909-1912)}

\subsection{宣统}

\begin{longtable}{|>{\centering\scriptsize}m{2em}|>{\centering\scriptsize}m{1.3em}|>{\centering}m{8.8em}|}
  % \caption{秦王政}\
  \toprule
  \SimHei \normalsize 年数 & \SimHei \scriptsize 公元 & \SimHei 大事件 \tabularnewline
  % \midrule
  \endfirsthead
  \toprule
  \SimHei \normalsize 年数 & \SimHei \scriptsize 公元 & \SimHei 大事件 \tabularnewline
  \midrule
  \endhead
  \midrule
  元年 & 1909 & \tabularnewline\hline
  二年 & 1910 & \tabularnewline\hline
  三年 & 1911 & \tabularnewline\hline
  四年 & 1912 & \tabularnewline
  \bottomrule
\end{longtable}


%%% Local Variables:
%%% mode: latex
%%% TeX-engine: xetex
%%% TeX-master: "../Main"
%%% End:



%%% Local Variables:
%%% mode: latex
%%% TeX-engine: xetex
%%% TeX-master: "../Main"
%%% End:
