%% -*- coding: utf-8 -*-
%% Time-stamp: <Chen Wang: 2019-12-26 22:00:17>

\section{世宗\tiny(1722-1735)}

\subsection{生平}

雍正帝(1678年12月13日-1735年10月8日),爱新觉罗氏,名胤禛,法号破塵居士、圓明居士,是清朝自入关以来的第三位皇帝,1722年12月20日至1735年10月7日在位,年号「雍正」。死后庙号世宗,谥号敬天昌运建中表正文武英明宽仁信毅大孝至诚宪皇帝,通称世宗宪皇帝。

雍正帝乃康熙帝第四子,於1722年12月27日登基(農曆康熙六十一年十一月二十日)。在位期间对内政民生有着诸多改革,例如在中央设置軍機處和密折制度来加強皇权,在地方上推行摊丁入亩、火耗歸公、改土归流、打擊貪腐的王公官吏和廢除賤籍等一系列政策来推动清朝经济和国力进一步增加,对外则通过对俄国谈判确定蒙古北部边疆,同时平定青海,在西藏设置驻藏大臣等对统一多民族有着重大贡献,还对康雍乾盛世的延续具有承上启下的重大作用。

胤禛于康熙十七年十月三十日(1678年12月13日)寅時出生于故宫永和宫。由於其生母乌雅氏出身低微,沒有撫育的資格,此外,清初時後宮也不允許生母撫育自己的兒子,因此胤禛满月后由孝懿仁皇后佟佳氏撫養,視其為養母。康熙帝曾评价幼年的胤禛“喜怒不定”,后经胤禛请求,于康熙四十一年(1702年)撤此考语。因胤禛性情急躁剛烈,父皇康熙用“戒急用忍”训喻他。胤禛早年随康熙巡历四方。

康熙三十七年(1698年)三月,康熙帝第一次賜給胤禛爵位,封為多羅貝勒。

康熙四十六年(1707年),康熙賜皇家園林圓明園給貝勒胤禛,十一月,胤禛恭請康熙幸(圓明園)進宴用膳(1707年至1722年,康熙帝總共了去圓明園12次)。

康熙四十七年(1708年)夏,康熙第一次罷黜皇太子允礽。

康熙四十八年(1709年),康熙复立允礽为太子。同年也升多羅貝勒胤禛爵位為和硕雍亲王。

康熙五十年(1711年)八月,胤禛妾室典儀之女藩邸格格鈕祜祿氏(熹妃)生下雍亲王胤禛第四子弘曆,即後來的乾隆帝。

康熙五十一年(1712年)康熙再次废黜允礽,自此不再立皇太子。争夺储位斗争由明转暗,更加激烈。胤禩因争位意图过于明显,被康熙斥责、疏远。胤禛沈迷釋教,有時崇信道教,到諸宮觀禮拜天尊真人圖像,與道士們研究金丹之學,与诸兄弟维持和气,自称“天下第一闲人”,暗中与隆科多与年羹尧交往,加强自己的势力集团。

康熙五十三年(1714年),朝鮮國王所派的使臣回國後,向朝鮮國王表明大清康熙皇帝當時的意旨:「(胤禛二哥)允礽之子弘皙颇贤,难于废立(太子)允礽」;或康熙五十六年(1717年),亦表明康熙皇帝當時意旨:「弘皙甚贤,故不忍立他子,而尙尔贬处允礽矣」。

康熙六十一年(1722年),胤禛第四子弘曆垂髫之年(12歲時),康熙幸胤禛的圓明園進宴用膳。(乾隆上位後,《高宗純皇帝實錄》記載了,康熙因為乾隆弘曆之故在圓明園進宴用膳,康熙連稱弘曆生母為有福之人;但是康熙時期在自己的《聖祖仁皇帝實錄》上,並未給乾隆弘曆母子記載任何很喜愛他們的歷史,也未給乾隆弘曆母子賜冊世子或福晉以作為獎賞)

康熙六十一年(1722年)十一月,胤禛登基,為雍正皇帝。胤禛二哥允礽,第二子弘皙是胤禛上位後第一位晉升王爵(多羅理郡王)的侄輩。朝鲜使臣向朝鮮国王稟報大清國皇宫盛传:「(康熙臨命終遺言):允礽第二子弘皙朕所鍾愛,其特封为和碩親王」爵位、又有「康熙皇帝既封允礽之子弘皙为王,雍正以在邸时宫室、服御、金银、臧获及王府官属,一倂移给」。這在本國和它國等諸多史料確實有明確記載的。而康熙遺命要預備給弘曆冊封王爵在朝鮮國並沒有提及。

雍正特別在宫中抚养允礽的幼龄兒子:弘㬙、弘皖、永璥,收为雍正帝的养子(這也是乾隆弘曆上位後親口承認的)。

三月,雍正皇帝親生皇子皇女中,只追封側福晉李氏所生已成年皇二女爵位:和碩格格(郡主)为和硕怀恪公主(康熙雍正的未成年子女一律不封爵位,皇子女有超過18歲的才有尊爵)。

雍正元年(1723年)八月,雍正於乾清宮召諸王、滿漢大臣入見,面喻曰:「建儲一事,理宜夙定。去年十一月之事,倉卒之間,一言而定。聖祖仁皇帝神聖,非朕所及」。命群臣皆退,仍留這四人總理事務王大臣:允禩、允祥、隆科多、马齐 ,以康熙旨意不立皇太子,将密封遗诏收藏於乾清宮最高之處(亦是大清歷朝皇帝最早秘定的太子人選)

雍正元年(1723年)九月二十日卯时,雍正以康熙遗命,分家理郡王弘晳距京城二十里的郑各家庄王府,亦下令弘皙携福晋、以及子弟一起迁至皇城外的郑各家庄,命人以礼相待弘晳及其眷属,以隆重礼数安排至距皇城外二十里的郑各家庄藏身定居,亦命令以多羅郡王禮數儀仗相送,並命數千位兵丁家臣奴僕保護弘皙的鄭各家莊王府。而弘晳之父允礽因有罪因此仍被禁锢於皇城内咸安宫。雍正帝十分关心弘晳,弘皙亦於奏摺中稱呼本是叔父的雍正皇為:“皇父”,与弘晳关系融洽。

雍正元年(1723年)十一月,適逢康熙忌辰,雍正命皇四子弘曆祭景陵。

雍正二年(1724年)五月,雍正諭旨:「二阿哥允礽奏曰:臣蒙皇上种种施恩甚厚,臣心实深感激。又训弘皙,你若能一心竭诚效力,以事君父,方为令子,此皆二阿哥允礽至诚由衷之言」。十二月,允礽病故后,雍正追封允礽和碩理亲王,謚號,曰:密。而且,雍正還特別賜弘皙之生母李佳氏為允礽的側福晉,令弘皙盡心孝養李佳氏。並且讓允礽各妻妾,皆能豐衣足食,以終餘年。

雍正四年(1726年),雍正皇帝給大學士鄂爾泰御筆朱批中有提道:『朕之關心(你),勝朕頑劣之子』。雍正八年(1730年)又說:『皇子皆中庸之資,朕弟侄輩亦乏卓越之才』。

雍正八年(1730年),雍正遵照康熙皇帝臨命終時遺言,冊封允礽的第二子弘皙承襲其生父允礽的爵位:和碩理親王(雍正皇帝剩下18歲皇子兩位:弘曆、弘晝,但還未冊封爵位)

雍正十一年(1733年)正月,皇子只剩兩位時,雍正諭宗人府:「朕幼弟(18歲)胤秘,秉心忠厚賦性和平素為皇考(康熙)之所鍾愛,數年以來在宮中讀書學識亦漸增長,朕心嘉悅著封親王。皇四子弘曆(21歲)、皇五子弘晝(21歲),年歲俱已二十外,亦著封為親王,所有一切典禮著照例舉行」。(弘曆最受康熙鍾愛,連多羅郡王、貝勒爵位都封不到)

雍正十三年(1735年)八月,雍正皇帝于圓明園病重,宝亲王弘曆和亲王弘晝朝夕侍侧。晚上八點,大學士鄂尔泰、大學士张廷玉至雍正寢室,恭捧上御笔亲书曰:『命皇四子宝亲王弘曆为皇太子即皇帝位』。夜子時,雍正躺在病床上立弘曆為太子後,在圓明園駕崩,時年五十八歲。匾額下宣讀密封遺詔,喻旨:「寶親王皇四子(弘曆),……聖祖康熙帝於諸孫之中,最為鍾愛,撫養宮中,恩逾常格……雍正元年八月朕於乾清宮召諸王、滿漢大臣入見,面諭以建儲一事,親書諭旨,加以密封,收藏於乾清宮最高之處,即立弘曆為皇太子之旨也。其後仍封(爵位)和硕寶親王者(饋贈大寶給弘曆),蓋令備位藩封,諳習政事,以增廣識見……著繼朕登極,即皇帝位……俾皇太子弘曆成一代之令主……,與和親王(弘晝)同氣至親,實為一體…大學士張廷玉器量純全,抒誠供職,其纂修《聖祖仁皇帝實錄》宣力獨多;大學士鄂爾泰誌秉忠貞,才優經濟,…此二人者朕可保其始終不渝。」皇太子弘曆登基,是為乾隆帝。以雍正駕崩前遺命囑託封乾隆皇帝生母熹妃鈕鈷祿氏為皇太后(欠缺冊封熹貴妃和裕妃的金冊或金印,《世宗憲皇帝實錄》亦未載冊文)。封和親王弘晝之母裕妃耿氏皇貴太妃。兩名撫育乾隆為皇子時的慈母(愨惠皇貴妃)佟佳氏及(惇怡皇貴妃)瓜爾佳氏,雍正本不封她們太妃,乾隆最後晉封她們皇貴太妃。

乾隆帝以西北軍事底定撤除軍機處,軍機處改設總理事務處並兼理軍機事務,總理事務王大臣以這四人:大學士鄂爾泰、大學士張廷玉、莊親王允祿、果親王允禮,原兼任軍機大臣鄂爾泰、張廷玉改在總理事務處。(乾隆2年,准總理事務王大臣解職,復設軍機處,乾隆以總理事務王大臣…入值軍機處)。

雍正十三年(1735年)九月,奉乾隆諭旨:「理密親王允礽之子弘㬙、弘皖、永璥因年尚幼穉蒙雍正垂慈恩養,仍住宮中,年已長成,雍正原欲賜宅另居尚未降旨,茲朕仰體聖慈為籌畫久遠之計,其應加封王爵,著總理事務王大臣會同內務府定議」。

雍正十三年(1735年)十月,總理事務鄂爾泰恭擬上崇慶皇太后的尊號

乾隆三年(1738年)二月,乾隆叔父果亲王允禮薨,乾隆命六弟弘曕過繼允禮子嗣,且協助弘曕袭果亲王爵。(雍正時期乾隆三哥弘時獲罪,過繼阿其那允禩子嗣)

所以,雍正的子女只剩下兩名:乾隆弘曆與弘晝。

乾隆四年(1739年)十月,理親王弘皙因突然在乾隆上位後,有了謀反皇帝等罪名,因此弘皙永久被革除親王爵。乾隆四十八年,乾隆還特別另外編撰《钦定古今储贰金鉴》歷史史籍,奉乾隆帝諭旨,記載以下歷史:「弘皙縱欲敗度,不克幹蠱,年亦不永。使相繼嗣立,不數年間連遭變故,豈我大清宗社臣民之福乎?是以皇祖康熙有鑒於茲,自理密親王既廢不復建儲,迨我皇祖康熙龍馭上賓,傳位雍正紹登大寶,十三年勵精圖治中外肅清...雍正元年,即親書朕名,緘藏於乾清宮正大光明匾內,又另書密封匣,常以隨身。至雍正十三年八月,雍正升遐,朕同爾時大臣等敬謹啟視,傳位於朕之御筆,復取出內府緘盒密記...」。

康熙六十一年(1722年)年十一月初七(12月14日),康熙聖祖駕崩前宣詔嗣位於畅春园,皇四子雍亲王胤禛继皇帝位,是为雍正帝。康熙帝死时,多人包括多位阿哥都知道康熙傳位雍正然後隆科多一人传遗诏由雍正继位。治丧期间,隆科多提督九门、卫戍京师。隆科多是皇贵妃佟佳氏的弟弟。雍正繼位,任命康熙皇八子允禩、皇十三子允祥、馬齊和隆科多總理事務。

雍正十三年八月二十二日(1735年10月7日),雍正因工作過勞累,在批閱奏章時崩逝於圓明園,享年五十七歲。廟號世宗,諡號憲皇帝,安葬於清泰陵。命其四子弘曆登基繼位。

军机处:雍正八年,新首創立軍機處,當時主要为了緊急应对西北军情,协助辦理皇帝处理对准噶尔用兵的各種軍務。而军机处设有军机大臣,从大学士、尚书、侍郎以及皇亲国戚中担任。 議政王大臣會議與軍機大臣在雍正時期,依然是並存的,並且雙方職責各不盡相同,共同點皆需要處理軍務。只是1792年乾隆當政時,废除了议政王大臣会议,乾隆以军机处為主要專一事權。例如雍正時期的首席軍機大臣:怡親王允祥、大學士鄂爾泰。
密摺制:雍正还在中央进一步完善密摺制度来監視臣民。
清除兄弟:雍正二年四月,明詔訓飭康熙帝皇八子,令王公大臣察其善惡;削康熙帝皇十子爵永遠拘禁之;十二月,康熙帝前廢太子死。雍正三年二月,諭示康熙帝皇八子罪狀;四年正月除宗籍,易名“阿其那”(滿語罵人的話,意義眾說紛紜,有「馱負罪過」、「驅趕犬隻」、「冷凍的魚」等眾說),九月死。雍正三年二月,諭示康熙帝皇九子罪狀,八月革爵;四年五月改名“塞思黑”(意为「顫抖」,也有人說是「刺傷人的野豬」),八月死。雍正三年二月,諭示康熙皇十子胤誐罪狀。雍正二年七月,命同母弟、康熙帝皇十四子胤禵守陵;三年二月,諭示其罪狀,十二月降爵;四年五月禁錮。雍正六年六月,康熙皇三子胤祉因罪降爵;八年二月復親王爵,五月因康熙皇十三子之喪時「遲到早散,面無戚容」而削爵拘禁。
雍正帝戎装像
雍正帝戎装像
整顿吏治
康熙帝在位晚年,对下属过度宽纵,导致大清吏治腐败,官风松懈。雍正帝登基后第一项任务就是整顿吏治。一方面雍正帝告诫官员,在给总督的上谕中说:“今之居官者,钓誉以为名,肥家以为实,而曰‘名实兼收’,不知所谓名实者果何谓也”,登极一周年时又说到:“朕缵承丕基,时刻以吏治兵民为念”。另一方面完善监督体系,紧抓思想反腐,并注重官员、民众的思想道德教化,树立反腐典范。

在整饬吏治的同时又打击朋党势力,他看到朋党之间各抒政见,妄议朝政,扰乱君父视听,妨碍坚持既定的政策,认为“朋党最为恶习”,因此宣称“将唐宋元明积染之习尽行洗涤”,“务期振数百年颓风,以端治化之本”。

改善祕密立储制度,即皇帝在位时不公开宣布太子,而将写有继承人名单的一式两份诏书分别置于乾清宫“正大光明”匾额后和皇帝身边,待皇帝去世后,宣诏大臣共同拆启传位诏书,确立新君。这样使得皇位继承辦法制度化,也在一定程度上避免康熙帝晚年诸皇子互相倾轧的局面。

雍正初年,重用年羹尧和隆科多。隆科多为吏部尚书、步军统领,兼理藩院,赐太子太保衔,被雍正尊称为“舅舅”。显赫异常,但未过几年,即被雍正整肃。雍正三年七月削隆科多太保銜;雍正四年正月削職;雍正五年十月廷臣上四十二罪款,下獄,永遠禁錮;雍正七年六月,死於禁所。其較為寵信的四位臣工:李衛(江苏人)、田文鏡(福建人)、張廷玉(安徽人)、鄂爾泰;李卫和张廷玉為漢人,田文镜为汉军的旗人,以民族分,漢族佔了四分之三,足見雍正確實了解也重用漢人。雍正四年十二月,河南、陜西、四川均攤丁銀入地併徵;謝濟世劾田文鏡,被褫職,發赴阿爾泰軍前效力,陸生柟亦以黨援同時遭遣。

清兵入关以后,国家承平日久,军备废弛。而作为大清军队主力的八旗兵也是丧失斗志,特别是在旗的八旗子弟,每日游手好闲,贪图享乐。雍正帝对于此情此景对八旗旗务进行了一些整顿,例如:给那些无所事事的旗人分的土地和农具,让其自力更生,派遣八旗子弟去前线参战等。

九子奪嫡、胤禔软禁、年羹堯案、曾静吕留良案、隆科多案、谢济世案、陆生楠案、屈大均案。

年羹尧先后被任命为川陕总督、抚远大将军,赴青海征讨厄鲁特罗卜藏丹津叛乱。

雍正元年三月,封年羹堯三等公;四月命康熙皇十四子留護康熙帝遺體;五月,生母仁壽皇太后死;八月,密封立四子弘曆之上諭於正大光明匾後;十月授年羹堯撫遠大將軍。雍正二年三月,平定青海,進年羹堯為一等公。成为实际的西北王。雍正三年三月,下詔斥責年羹尧,四月調為杭州將軍,六月削太保銜,七月黜為閑散旗員,十二月廷臣上九十二罪款,賜死,斬其子年富。

康熙末年吏治松弛,贪污成风,加上诸王皇族同官僚结党营私,致使财政经济从中央到地方混乱不堪,“积弊甚大”。仅户部就亏空白银二百多万两。面对如此局面雍正帝在稅制上推動“摊丁入畝”,“火耗歸公”,“官紳一體當差納糧”等一系列改革。

同时,设会考府,清查亏空。推广养廉银制度,养廉银不但是一项经济政策,同时也是清朝前期整顿封建制度的一项综合改革措施。

雍正帝在位期间还对科举制度实行了一系列改革,例如:创考差先例,改革选派考官制度;变更考的试内容和重点;增设考试科目,考生的资格限制有所放宽;还创行“朝考”、翻译翰林 “大考”等复试制,变通一试而定终身的制度;调整用人政策,数途并用,以抑科甲。这些措施的实行力剔积弊的施政作风。

雍正兴起文字狱以打击年羹尧和隆科多两人势力(汪景祺案和钱名世案)。雍正三年十一月,年貴妃死;十二月斬《西征隨筆》作者汪景祺。雍正四年三月,錢名世以曾投詩年羹堯獲罪,雍正親書「名教罪人」懸其家門,又命文臣作詩文刺惡他。对于隆、年的死因,有人指出是由于年、隆位重之后过于骄奢、行为不检,加上结党营私,触犯了皇權的大忌,为雍正所不容。但雍正早年过于宠信放纵,随后又残酷打击,被史学家所批评。另有人与雍正得位傳說联系起来,认为隆、年参与此事,知道太多而被「兔死狗烹」。雍正四年九月,查嗣庭以謗訕下獄,五年五月死,戮屍。

據正史記載:雍正七年五月,曾靜供稱因讀呂留良書而有謀反;十年十二月,治呂留良罪,與兒子呂葆中、門人嚴鴻逵一同戮屍,斬另一兒子呂毅中與門生沈在寬。

从清代史料中可以看出,雍正帝主张民族平等,尊重民族习惯;反对民族歧视和限制国家干预;保护民族生态,禁止过度需索。对促进民族融合,化解民族矛盾和维护清朝的统一多民族有着重要贡献,也使清代的民族统治达到历史最高水平。

雍正七年九月,頒行《大义觉迷录》。在书中雍正帝梳理对华夷、正统、君臣、封建等问题,论述了他自己所谓的民族“大一统”观。

雍正二年(1724年)设置西宁办事大臣,办事大臣衙门最初设于察罕托洛亥(青海湖东南),后改驻西宁,故乾隆以后又称为西宁办事大臣。

雍正五年,在西藏设置驻藏大臣,加强对西藏的控制。

廢除西南少數民族原本的土司制度,改用朝廷分發的流官,史稱“改土歸流”,派遣官吏統治,加強對少數民族的統治及同化。

海禁问题上,开始严格执行海禁,后来考虑到闽地百姓生计困难,同意适当开禁;雍正二年降旨准廣東人移民台灣,但对外洋回来的人民仍有戒心。雍正严禁中国商人出海经商,海设置各种障碍,并说道"海禁宁严毋宽,余无善策"。在沿海各省的要求下,虽放宽海禁,但仍加以限制盘剥。尤其对久住外国的华侨商贩和劳工,“逾期不归,甘心流移外方,无可悯惜,不许其复回内地”。
社会
雍正帝在位期间还实施“废除贱籍”一项改革。雍正帝下令为賤民开豁为民,编入正户,准許置產定居、考試,宣示廢除賤民階級,但影響有限,未能改變社會大眾的歧視風氣,賤民仍然存在,如福州疍民群體較明顯存續到清末,及所謂發功臣暨披甲家爲辛者庫。

努尔哈赤和皇太极的陵墓位于沈阳的盛京三陵。清入关后,从顺治帝、康熙帝都安葬到北京东边的遵化县马兰峪皇家陵园,即清东陵。雍正帝另选北京西边的易县开辟自己的陵墓,即清西陵。

《清史稿》:圣祖政尚宽仁,世宗以严明继之。论者比於汉之文、景。独孔怀之谊,疑於未笃。然淮南暴伉,有自取之咎,不尽出於文帝之寡恩也。帝研求治道,尤患下吏之疲困。有近臣言州县所入多,宜釐剔。斥之曰:“尔未为州县,恶知州县之难?”至哉言乎,可谓知政要矣!

《清世宗实录》:天表奇伟,隆准颀身,双耳半垂,目光炯照,音吐洪亮,举止端凝。......幼耽书诗,博览弗倦,精究理学之原,旁彻性宗之旨。天章濬发,立就万言。书法遒雄,妙兼众体。毎筹度事理,评骘人才,因端竟委,烛照如神。韬略机宜,皆所洞悉。。

李氏朝鲜君臣受儒家正统华夷之辨观念的影响对清国以及清国皇帝的态度多持批评态度,甚至有妖魔化倾向。朝鲜人毫无忌讳地记录康熙帝的“雌雄眼”容貌,还认为雍正帝贪财爱银。但是朝鲜使臣李樴于雍正元年回国后,向朝鲜国王报告,亲见雍正“气象英发,语言洪亮”。

英國歷史學家史景遷認為:雍正的父親康熙為政寬鬆,執政末期受儲立之爭所擾且出現典型長壽帝王的統治能力退化現象,雍正即位之初的滿清實已浮現官僚組織膨大腐敗、農民生活水準惡化的危機;由於雍正即位時正處於政治歷練、精神與人格上的成熟階段(45歲),因此得以精準的分析問題並有魄力的作出應對。他的改革同時包含力行整頓與和現實的妥協(如火耗歸公與養廉銀)。雖然史學家黃仁宇認為雍正未能瞭解與解決明清兩代作為內歛式王朝的根本問題,但滿清得以建立起一套繼續運行百年以上仍大致有效的統治體制,而未淪為「立國百年而亡」的異族王朝,此當歸功於雍正一朝的改革。

英国人濮兰德·白克好司评价雍正:“控御之才,文章之美,亦令人赞扬不值。而批臣下之折,尤有趣味,所降谕旨,洋洋数千言,倚笔立就,事理洞明,可谓非常之才矣”。

杨珍认为雍正是一位善于观察与思考者。其思想的敏锐性以及思维广度与深度,都超过允禩、允禵等人。

中國歷史學家钱穆認為:雍正帝是有名的专制,他私派的特务人员监视全国各地地方长官一切活动,许多地方官的私生活,连家里的琐事都瞒不过他,虽然雍正帝精明,但仍是独裁的本质。此外,雍正帝在平定外患之後,唯恐国内发生政变,于是使計把功高权重的大臣统统清除。他把過去與其爭位的两个兄弟——胤禩、胤禟以种种罪名逮捕拘禁,并将为他策划取得帝位的人处死,比如年羹尧和隆科多。

雍正即位经过至今也是一个解不开的谜。从雍正年间时,对雍正继位的谈论便不绝于耳。歷史記載雍正在康熙駕崩當晚連續觐見兩次之多,後康熙便身亡。主张篡位说的学者中,有的认为康熙去世过于突然,未来得及留下任何传位遗诏,而雍正和隆科多等合谋抢占了先机;有的认为康熙生前两立两废太子,对立储君一事劳心伤神,直到临终前才属意皇十四子为储君。按照正统继位说学者观点,如果没有实在的证据证明其他皇子为康熙所属意,雍正的即位是有理由的。

康熙帝傳位雍正帝之徵兆:徵兆一:「康熙六十年正月,命皇四子雍親王胤禛、皇十二子貝子胤祹、世子弘晟以御極六十年,告祭永陵、福陵、昭陵。」康熙登基一甲子六十年之重大祭告先祖非同一般,派遣雍親王胤禛主持,豈能不具備重大意義?為何不是派遣支持皇十四子胤禵的皇八子胤禩、皇九子胤禟、皇十子胤䄉?或是皇三子胤祉?徵兆二:康熙御極六十年派雍親王胤禛祭祖此舉,讓廢太子胤礽之師王掞看出端倪,故於三月「大學士王掞密奏請建儲,至是監察御史陶彝、任坪、范長發等人曾疏請建儲,帝不悅,並掞切責之。諸王、大臣奏請治大學士王掞罪,帝赦不治。」這亦可視為康熙安排接班人的佈署跡象之一,畢竟皇十四子胤禵尚且領兵在西北,一旦提早公佈,易生事端。徵兆三:「五月壬戌,命撫遠大將軍胤禵移駐甘州。以年羹堯總督四川陝西,色爾圖署四川巡撫。」康熙以皇四子雍親王胤禛之親信年羹堯箝制皇十四子胤禵的軍後補給已然成形。徵兆四:康熙六十一年四月,「命撫遠大將軍胤禵復往軍前。十月,命雍親王胤禛率弘昇、延信、孫渣齊、隆科多、查弼納、吳爾台察閱京師通州倉廒。」康熙指示由雍親王胤禛親率隆科多、查弼納等眾多京師王公重臣,竟然只為「察閱京師通州倉廒」,已有不尋常跡象。徵兆五:「十一月帝不豫,駐蹕暢春園。命皇四子胤禛恭代祀天。」康熙駕崩前祀天仍然未派皇三子胤祉、皇八子胤禩、皇九子胤禟、皇十子胤䄉代祀,更未召皇十四子胤禵返京,此時康熙意欲傳位於雍親王皇四子胤禛已然十分明顯。雍正元年三月,封年羹堯三等公;四月命康熙皇十四子留護康熙帝遺體;五月,生母仁壽皇太后死;八月,密封立四子弘曆之上諭於正大光明匾後(遺詔:「寶親王皇四子,……著繼朕登極,即皇帝位……俾皇太子弘曆成一代之令主。」);十月授年羹堯撫遠大將軍。雍正帝继位后,对其兄弟手段颇为毒辣,用各种方式进行迫害。雍正二年四月,明詔訓飭康熙帝皇八子,令王公大臣察其善惡;削康熙帝皇十子爵永遠拘禁之;十二月,康熙帝前廢太子胤礽死。

康熙帝皇八子胤禩先是被安抚封为廉亲王。雍正三年二月,諭示康熙帝皇八子罪狀;四年正月除宗籍,易名“阿其那”(滿語罵人的話,意義眾說紛紜,有「馱負罪過」、「驅趕犬隻」、「冷凍的魚」等眾說),九月死。康熙帝皇九子胤禟发往西宁。雍正三年二月,諭示康熙帝皇九子罪狀,八月革爵;四年五月改名“塞思黑”(意为「顫抖」,也有人說是「刺傷人的野豬」),八月死。雍正三年二月,諭示康熙皇十子胤誐罪狀;後被圈禁。雍正二年七月,命同母弟、康熙帝皇十四子胤禵守陵;三年二月,諭示其罪狀,十二月降爵;四年五月禁錮。雍正六年六月,康熙皇三子胤祉因罪降爵;八年二月復親王爵,五月因康熙皇十三子之喪時「遲到早散,面無戚容」而削爵拘禁。皇十二子胤祹被降爵。

然而「手段毒辣」之說,有人反對,並舉出了皇五子胤祺封至恆親王、皇七子胤祐封至淳親王、皇十二子胤祹封為履郡王、皇十三子胤祥封至和碩怡親王、皇十五子胤禑封多羅愉郡王、皇十六子胤祿承襲莊親王、皇十七子胤禮果親王、皇二十子胤禕封多羅貝勒、皇二十一子胤禧封貝勒、皇二十二子胤祜封固山貝子、皇二十三子胤祁封鎮國公等,諸多無利害關係的兄弟,得到封賞。

雍正突然的离世,史书不记载其去世原因,引起人们的疑惑。

病死:有人认为雍正帝“是中风死去的”。暗杀:民间流行的说法是,吕留良的后人吕四娘,为报仇,砍去雍正的头。丹药中毒:近年来由于对清代的档案进行了大量研究,许多史学工作者认为,雍正吃丹药中毒致死也有很大可能,而乾隆帝即位后,马上将圆明园内的炼丹道士和民间术士全部赶出。

雍正帝篤信佛教,熱衷藏傳佛教、漢傳佛教,與密宗的章嘉活佛交往密切;雍正也研究禪宗,精通《金剛經》,並著作佛學書籍數部,為章嘉活佛認可其參透三關,成為中國佛教史上唯一一位自認為已覺悟的皇帝。雍正帝也喜歡道教,常常服食道士的金丹。雍正元年重申禁止天主教,史稱雍正禁教。

雍正皇帝委託宮廷畫師郎世寧,創作一幅《雍正行樂圖》(現存於北京故宮博物院),顯示雍正喜愛打扮成不同年代的各式人物,後世人稱他為“近代Cosplay始祖”。

早期未即位前(九子奪嫡時期),就曾委託畫師給自己家人畫《春耕圖》進獻給康熙皇帝以表明無爭位之心,後來的乾隆皇帝也有相似的喜好。

\subsection{雍正}

\begin{longtable}{|>{\centering\scriptsize}m{2em}|>{\centering\scriptsize}m{1.3em}|>{\centering}m{8.8em}|}
  % \caption{秦王政}\
  \toprule
  \SimHei \normalsize 年数 & \SimHei \scriptsize 公元 & \SimHei 大事件 \tabularnewline
  % \midrule
  \endfirsthead
  \toprule
  \SimHei \normalsize 年数 & \SimHei \scriptsize 公元 & \SimHei 大事件 \tabularnewline
  \midrule
  \endhead
  \midrule
  元年 & 1723 & \tabularnewline\hline
  二年 & 1724 & \tabularnewline\hline
  三年 & 1725 & \tabularnewline\hline
  四年 & 1726 & \tabularnewline\hline
  五年 & 1727 & \tabularnewline\hline
  六年 & 1728 & \tabularnewline\hline
  七年 & 1729 & \tabularnewline\hline
  八年 & 1730 & \tabularnewline\hline
  九年 & 1731 & \tabularnewline\hline
  十年 & 1732 & \tabularnewline\hline
  十一年 & 1733 & \tabularnewline\hline
  十二年 & 1734 & \tabularnewline\hline
  十三年 & 1735 & \tabularnewline
  \bottomrule
\end{longtable}


%%% Local Variables:
%%% mode: latex
%%% TeX-engine: xetex
%%% TeX-master: "../Main"
%%% End:
