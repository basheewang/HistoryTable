%% -*- coding: utf-8 -*-
%% Time-stamp: <Chen Wang: 2021-11-01 17:23:03>

\section{文宗咸豐帝奕詝\tiny(1850-1861)}

\subsection{生平}

咸豐帝(1831年7月17日-1861年8月22日),爱新觉罗氏,名奕詝,號且樂道人,是清朝自入关以来的第九位皇帝,1850年3月9日至1861年8月22日在位,年号「咸豐」。西藏方面尊為「文殊皇帝」。

咸豐帝是道光帝第四子,生母為孝全成皇后钮祜禄氏,誕於北京圆明园澄静斋,养母為孝静成皇后博尔济吉特氏。20岁登基,1861年崩于承德避暑山庄烟波致爽殿,享年30岁。死后庙号文宗,谥号簡稱显皇帝,葬于清东陵中的定陵。他也是清朝最後一位掌握實權的皇帝與最後一位儲位密建的皇帝。

1831年7月17日(道光十一年六月初九日),咸丰帝生于北京圆明园之澄静斋。时道光帝前三个儿子都已去世,咸丰帝出生后即为在世的皇长子。二十六年,按照秘密立储制度,被道光帝立为储君。三十年正月丁未,道光帝驾崩前,宣召大臣开启鐍匣,立为皇太子。

咸豐帝即位後便勤於政事,广开言路、明詔求賢,先後將有损国家利益的穆彰阿和耆英革職,大手笔地对朝政颇有改革。但此時的大清帝国內憂外患不斷,先後爆发太平天国运动以及第二次鸦片战争。

在第二次鸦片战争中,俄国西伯利亚总督穆拉维约夫迫使黑龙江将军奕山签订《瑷珲条约》,割去黑龙江以北、外兴安岭以南原属清朝的领土约60万平方公里,咸丰帝拒绝承认该条约。随后英法联军进攻北京,咸丰帝下诏对英国法国宣战:“兵家胜败何常,该国兵远来即有数万,未可当我中国人民千百之一,其能经几战乎?”最后圆明园、清漪园等被焚掠,以签定一系列不平等条约收场。

咸丰十一年(1861年)七月十七日,咸丰帝崩于行在避暑山庄烟波致爽殿。皇长子同治帝继位。同时依遺詔,由肃顺、载垣、端华、景寿、穆荫、匡源、杜翰、焦祐瀛为顾命八大臣,肅順為首,輔導皇帝施政。在咸丰灵柩启程返京期间,東太后慈安、西太后慈禧、恭親王奕訢、醇亲王奕譞四人联合發動辛酉政变,醇亲王奕譞亲自抓捕肃顺,随之八大臣非死即貶。同时,政府随即由慈安、慈禧两宫听政,咸丰帝生前的輔政遺命宣告廢止。

根據内务府档案的记载,如意馆画士沈振麟曾在同治年间绘制了两幅先帝咸丰帝圣容,这两幅圣容均先画稿,呈览后再进一步绘制。和硕恭亲王奕訢向两宫皇太后和同治帝呈览的两幅墨稿分别为便衣墨稿一件及道装配山洞景致墨稿一件。

清朝皇帝的評價中,咸豐帝的爭議最大。咸丰爱看戏,爱唱戏,即使到热河行在唱戏,“着升平署三拔至热河”,也表現得乐不思蜀。咸丰一朝,财政十分困难,要镇压太平天国,对付英法联军,财源枯竭,“户部因军兴财匮,行钞,置宝钞处,行大钱,置官钱总局,分领其事”,钞票大量发行,造成通货膨胀,“官民交累,徒滋弊窦”。

咸丰帝“任賢擢才,洞觀肆應”,在面對太平天国运动與“三千年未有之政局”的內憂外患中,咸丰指揮若定,重用汉族大臣曾国藩人等组织团练来对付太平天国,咸丰帝颇思除弊求治,提拔行事果断的肃顺,并支持肃顺等人在朝政上推行的改革。為後來的同光中興打下良好基礎。但也因為採納肃顺對英法兩國強硬的做法,導致引發英法聯軍之役。

咸丰帝临终前对朝政事宜不合理的安排和权力制衡,使朝臣和后宫在有关朝政和国事方面展开逐鹿,间接导致慈安太后、慈禧太后、恭親王奕訢、醇亲王奕譞联手,打倒了顾命八大臣,最後慈禧掌权近半个世纪,从而也被有些史家流派认为咸丰帝没有安排好善后事宜,致使后宫干政近半个世纪的局面。

英、法等國要求清廷能讓英法兩國在北京設置駐京公使,新任公使到任時能觐見皇帝,但咸丰帝不接受。英國也要求清廷開放中國貿易,咸丰帝也拒絕,雖然天津條約簽訂後,咸豐帝默許了英、法等國的要求,但又對英、法等國的公使刁難(主因是採納肃顺對英法兩國強硬的做法)。咸丰帝本人的守舊,間接導致英法聯軍之役清朝的慘敗,使清朝失去首都北京,圓明園也遭聯軍焚毀。

太平天国运动與英法聯軍之役,使咸丰帝的執政遭受打擊,逃往熱河行宮後就病逝了,太平天国运动與英法聯軍之役的善後工作,直到慈安、慈禧两宫听政時期才結束。

大事年表:道光十一年六月初九日,奕詝出生。后受教于杜受田。道光三十年正月,宣宗去世,奕詝繼位。是年太平天國起事。咸豐三年二月,太平軍攻佔江寧,定都在此,改名天京。九月,太平天國北伐軍逼近天津。是年曾國藩建湘軍。咸豐五年四月,李開芳被俘,太平天國北伐軍覆沒。咸豐六年八月,天京事變發生。九月,「亞羅號事件」發生。咸豐八年四月,與俄國簽訂《璦琿條約》。五月,先後與俄、美、英、法四國簽訂《中俄天津條約》、《中美天津條約》、《中英天津條約》及《中法天津條約》。十月,太平軍取得三河大捷。咸豐九年五月,清大沽守軍擊退英、法艦隊。咸豐十年七月,英法聯軍攻佔天津和大沽一帶。八月,八里橋和大沽口相繼被攻佔,咸豐帝逃往承德,亦不足兩個月,英法聯軍火燒圓明園,進佔北京。十月,《中英北京條約》及《中法北京條約》立。十一月,《中俄北京條約》立。十二月總理各國事務衙門成立。咸豐十一年七月十七日,咸豐皇帝駕崩於承德避暑山庄烟波致爽殿,年三十。其子载淳年仅六岁,继承大统,是為同治帝。咸丰委派载垣、端华、景寿、肃顺、穆荫、匡源、杜翰、焦祐瀛为辅政八大臣,辅助小皇帝。

\subsection{咸丰}

\begin{longtable}{|>{\centering\scriptsize}m{2em}|>{\centering\scriptsize}m{1.3em}|>{\centering}m{8.8em}|}
  % \caption{秦王政}\
  \toprule
  \SimHei \normalsize 年数 & \SimHei \scriptsize 公元 & \SimHei 大事件 \tabularnewline
  % \midrule
  \endfirsthead
  \toprule
  \SimHei \normalsize 年数 & \SimHei \scriptsize 公元 & \SimHei 大事件 \tabularnewline
  \midrule
  \endhead
  \midrule
  元年 & 1851 & \tabularnewline\hline
  二年 & 1852 & \tabularnewline\hline
  三年 & 1853 & \tabularnewline\hline
  四年 & 1854 & \tabularnewline\hline
  五年 & 1855 & \tabularnewline\hline
  六年 & 1856 & \tabularnewline\hline
  七年 & 1857 & \tabularnewline\hline
  八年 & 1858 & \tabularnewline\hline
  九年 & 1859 & \tabularnewline\hline
  十年 & 1860 & \tabularnewline\hline
  十一年 & 1861 & \tabularnewline
  \bottomrule
\end{longtable}


%%% Local Variables:
%%% mode: latex
%%% TeX-engine: xetex
%%% TeX-master: "../Main"
%%% End:
