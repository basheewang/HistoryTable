%% -*- coding: utf-8 -*-
%% Time-stamp: <Chen Wang: 2021-11-01 17:21:13>

\section{圣祖康熙帝玄烨\tiny(1661-1722)}

\subsection{生平}

康熙帝(1654年5月4日-1722年12月20日),名玄烨,爱新觉罗氏,清朝第3位皇帝,清朝自入关以来的第2位皇帝,1661年2月5日至1722年12月20日在位,年号「康熙」。

康熙帝于順治十一年農曆甲午年三月十八巳時生於北京紫禁城景仁宫。康熙帝幼年继位,朝政不得不交付给辅政大臣。少年时期的康熙帝在智擒权臣鳌拜后,开始勤政。其在位期间,注意缓和阶级矛盾,采取轻徭薄赋与民生息的农业政策,重视农耕,发展经济,改革税收,疏通漕运。同时还对三藩、明郑、噶尔丹等各地反清势力大规模用兵,对沙俄签订条约确保黑龙江流域和广大东北地区的控制,实现清朝的国土完整和统一。康熙帝努力调节满族与汉、蒙、藏等族的关系,尊崇儒学,开博学鸿儒科笼络汉族士大夫;实行“多伦会盟”安抚蒙古各部,下令编修《理藩院则例》,确定巩固边疆的统治方针;册封五世班禅为“班禅额尔德尼”,派兵入藏驱逐入侵西藏的准噶尔汗国。还开海设关,发展内外贸易,重用海外传教士,学习西方近代科学。此间,使中国社会出现“天下粗安,四海承平”相对稳定的局面,为开启百余年的康雍乾盛世奠定了夯实基础。

但是,晚年的康熙帝沉浸于前半生的丰功伟业之中,不再锐意进取,开始倦于政务,标榜仁政而放松对吏治的治理,甚至出現吏治废弛、败坏的现象,从而暴露出许多社会问题,而废太子事件造成的夺嫡之争也对清朝政治产生了不良影响。

康熙六十一年十一月十三崩于北京畅春园清溪书屋,终年68岁。死后庙号圣祖、谥号合天弘運文武睿哲恭儉寬裕孝敬誠信功德大成仁皇帝,通称聖祖仁皇帝,葬于清东陵中的景陵。康熙帝在位六十一年零十个月,是中國歷史上在位時間最長的皇帝。

順治十一年三月十八日(1654年5月4日),玄燁出生於紫禁城景仁宮內,是順治帝的第三子,母親為孝康章皇后佟佳氏。順治帝病篤前沒有冊立過皇太子(祖父皇太極生前亦不預先冊立皇太子)。顺治十八年正月初六(1661年2月4日),顺治帝早逝,时年仅23岁。

順治帝染上瘟疫天花傳染病第3天時,接受湯若望的建議,因幼年玄燁曾出過天花具有免疫力,以口述遺詔的形式立玄燁為皇太子。顺治十八年正月初七(1661年2月5日)玄燁登基時,只有八歲,次年正月(1662年2月)改元:康熙。因康熙皇帝尚年幼,順治的遗诏同时指派四大臣辅政大臣索尼、苏克萨哈、遏必隆、鰲拜,輔治康熙皇帝。

康熙六年(1667年)六月,首辅索尼病故。七月初七(8月25日),十四岁的康熙帝正式亲政,在太和殿受贺,赦天下。但亲政仅十天后,鰲拜即擅杀同为辅政大臣的苏克萨哈,数天后与遏必隆一起进位一等公,实际政局并不受康熙帝直接掌控。

少年的康熙在挫败了政治对手鳌拜之后亲政。随即便宣布停止圈地,放宽垦荒地的免税年限。他还着手整顿吏治,恢复了京察、大计等考核制度。为了防止被臣下蒙蔽欺骗,康熙还亲自出京巡视,了解民情吏治。其中最著名的是六次南巡,此外还有三次东巡、一次西巡,以及数百次巡查京畿和蒙古,此举极大的促进了康熙对民情的了解,他还亲自巡视黄河河道,督察河工,并下令整修永定河河道。

康熙是清朝历史上在位时间最长的皇帝(後代的乾隆帝因崇敬康熙而刻意禪讓)。康熙坐镇北京取得了对三藩、沙俄的战争胜利,消滅在台湾的明鄭政权,另一方面,康熙创立“多伦会盟”取代战争,联络蒙古各部;意图以条约确保清朝政府在黑龙江的领土控制。文治武功取得巨大成绩的康熙帝,群臣一再商议给他上尊号,康熙多次表示“断不受此虚名”,这在历朝帝王中十分罕见。

康熙晚年懈怠无为,曾说“多一事不如少一事”,“政宽事省”,“凡事不可深究者极多”,不能严禁浮费和规银,宽纵州县火耗和亏空。同时他还標榜仁政,對官吏盡量以寬鬆待之,導致出現吏治废弛,官場贪污,国库亏空,“大小官员,怠玩成习,徇庇尤甚”,个别地区出现暴动和骚乱,统治秩序奏出了不和谐音符。盛世处于衰微的現象,给继任者雍正帝留下许多隐患。更有甚者指出清朝衰亡,病在康熙。

康熙四十九年(1710年),御史参劾户部堂官希福纳等侵贪户部内仓银六十四万余两,牽連的官吏多達一百一十二人。康熙说“朕反复思之,终夜不寐,若将伊等审问,获罪之人甚多矣”。最後只把希福納革職,其餘官吏則勒限賠款。康熙末年社会矛盾日趋激化,有江苏无锡县人劉三因县令李牧残酷成性,聚數百人於山中反抗,後被捕。

康熙的皇太子两立两废,彻底暴露出嫡长子皇位继承制度的种种弊端,储位之争的时间之长,卷入者之多,波及面之广,以及对皇朝及皇帝本人影响之大,无不超出前代。

康熙六十一年十一月十三日(1722年12月20日),康熙皇帝崩逝于大清順天府(今北京市)暢春園清溪书屋內,享壽六十八岁,結束了長達六十一年的統治。当时八爷党支持的十四阿哥胤禵远在西北,四阿哥胤禛留京。康熙近臣步军统领隆科多奉康熙帝遺詔,命皇四子胤禛继承皇位,是为雍正皇帝,为康熙帝上庙号圣祖,谥号合天弘运文武睿哲恭俭宽裕孝敬诚信功德大成仁皇帝,安葬于清景陵。

康熙十三年(1674年),康熙立皇后所生的一歲的皇次子胤礽為太子,並親自撫養。但數十年後由於太子本身的質素問題及其在朝中結黨而決定廢嫡。廢太子後,眾皇子覬覦皇位,矛盾更加尖銳,故太子廢而復立,但康熙仍無法容忍其結黨,三年後再廢太子。康熙六十一年臨終時決定傳位給皇四子胤禛。

目前理由眾說紛紜:有人認為康熙是希望精明幹練的胤禛能大力改革康熙末年的寬縱積弊,也有人認為康熙是因為鍾愛胤禛之子弘曆(未來的乾隆帝)而傳位於胤禛。還有傳說是顧命大臣隆科多和胤禛矯篡遺詔,在十字上加一劃、下加一勾,「十」字變成「于」字,故有「傳位十四皇子胤禵」竄改為「傳位于四皇子胤禛」之傳說;但按清宮祕檔分析,康熙帝的遺詔是由滿、漢、蒙三種語文並列寫成,「傳位十四皇子胤禵」改為「傳位于四皇子胤禛」之傳說符合漢字書寫邏輯,但卻無法符合滿文及蒙文書寫邏輯,且遺詔全文並未出現「傳位于」之類的語句。

然則傳位奪嫡之說,或因雍正推行攤丁入畝、官紳一體當差納糧之新政、打擊貪腐權貴、重用張廷玉、李衛、田文鏡等漢人,而引來失勢滿人權貴之蓄意誣陷。康熙皇帝豈能將九門提督授予不可信賴之人任之,又豈會不知隆科多與雍正之關係而造成眾皇子傳位紛爭?由此而論,康熙讓隆科多任九門提督,正是意欲傳位於雍親王,並加以保護的實證之一。

康熙傳位雍正之徵兆:徵兆一:「康熙六十年正月,命皇四子雍親王胤禛、皇十二子貝子胤祹、世子弘晟以御極六十年,告祭永陵、福陵、昭陵。」康熙登基一甲子六十年之重大祭告先祖非同一般,派遣雍親王胤禛主持,豈能不具備重大意義?為何不是派遣支持皇十四子胤禵、皇八子胤禩、皇九子胤禟、皇十子胤䄉或是皇三子胤祉。徵兆二:康熙御極六十年派雍親王胤禛祭祖此舉,讓廢太子胤礽之師王掞看出端倪,故於三月「大學士王掞密奏請建儲,至是監察御史陶彝、任坪、范長發等人曾疏請建儲,帝不悅,並掞切責之。諸王、大臣奏請治大學士王掞罪,帝赦不治。」這亦可視為康熙安排接班人的佈署跡象之一,畢竟皇十四子胤禵尚且領兵在西北,一旦提早公佈,易生事端。徵兆三:「五月壬戌,命撫遠大將軍胤禎移駐甘州。以年羹堯總督四川陝西,色爾圖署四川巡撫。」康熙以皇四子雍親王胤禛之親信年羹堯箝制皇十四子胤禵的軍後補給已然成形。徵兆四:康熙六十一年四月,「命撫遠大將軍胤禎復往軍前。十月,命雍親王胤禛率弘昇、延信、孫渣濟、隆科多、查弼納、吳爾台察閱京師通州倉廒。」康熙指示由雍親王胤禛親率隆科多、查弼納等眾多京師王公重臣,竟然只為「察閱京師通州倉廒」,已有不尋常跡象。徵兆五:「十一月帝不豫,駐蹕暢春園。命皇四子胤禛恭代祀天。」康熙駕崩前祀天仍然未派皇三子胤祉、皇八子胤禩、皇九子胤禟、皇十子胤䄉代祀,更未召皇十四子胤禵返京,此時康熙意欲傳位於雍親王皇四子胤禛已然十分明顯。

曾在國立故宮博物院展出的康熙皇帝遺詔上並無「傳位于四皇子胤禛」,而是寫著:“雍親王皇四子胤禛,人品貴重,深肖朕躬,必能克承大統,著繼朕登基即皇帝位,即遵典制持服。二十七日釋服,佈告中外,咸使聞知。”

康熙八年(1669年),康熙帝时常召集小內監在宫中作「布庫」之戏,不过在五月十六日(6月14日)鰲拜进见时,突然下令以大不敬之罪,命少年們将其逮捕。大臣商议鳌拜大罪三十条,请求將他滅族,康熙帝念鳌拜曾救過祖父皇太極的功劳,赦其死罪,改為拘禁,但诛杀了鳌拜的很多弟侄亲随及党羽。仅存的另一辅政大臣遏必隆因为长期勾结鳌拜,被削去太师、一等公。康熙帝由此完全奪回朝廷大權,開始真正親政的階段。

康熙勤政,坚持每日御临乾清门会见朝臣处理政务,居住在畅春园、热河行宫以及在出巡途中仍听政不惜。黎明时分,部院大臣,起居注官员到位,各部院衙门依次奏事,皇帝与内阁大臣商决裁断。《起居注》中详细记载了康熙皇帝御门听政现场办公的场景内容。康熙帝晚年还通过赵凤诏贪污案来抑制汉官。

1677年,康熙帝開始了整治黃河工程。到1684年,歷時七年的整治黃河工程完成。在康熙五十六年(1717年),出現各地豐收,無災可免的情況。康熙在晚年亦繼續減免天下賦稅,蠲免全國各地省份的錢糧,免除多處地區的欠賦。多種措施令到各地的農民都能夠休養生息,也防止了地方官吏中飽私囊和橫徵暴斂。

康熙帝為了箝制反清復明的活動而致力於打敗明鄭王朝。拿下臺灣之后,康熙开放了海禁,并设立了四个通商口岸。

1673年,因为康熙帝决定削藩,导致平西王吴三桂起兵反清,其他二藩相繼響應,整个天下为之一动。三藩势力一时不可阻挡,清廷失去江南半壁江山。而康熙帝在孝庄太后的支持下,沉着应对,积极调兵遣将,三藩之亂最終在1681年被完全撲滅,而国家遭受了较大的损失,在四川、云南以及江西等地有不少人被殺害。

1683年(康熙二十二年),时宪历五月,康熙採納了安溪大學士李光地的意見,授明鄭降將施琅為福建水师提督,时宪历八月丙辰,福建水师提督施琅攻克台湾,郑克塽和刘国轩等投降。

康熙年間,由於戰爭連年不絕,平定三藩之亂以及抵禦外來侵略的需要大量製造火器,無論是造炮規模、數量、種類,還是火砲的性能和製造技術,都達到了前所未有的水平。同時,清朝所造的大小銅、鐵炮達905門之多,而其中半數以上由南懷仁負責設計監造,就質量而言,其「工藝之精湛,造型之美觀,炮體之堅固,均為後朝所莫及」。康熙三十五年(1696年),在對準噶爾部噶爾丹的昭莫多之戰中,發揮了重要作用。

清朝初年一時間湧現出許多熱心武器裝備、致力於引進和仿造西方火器的技術專家。如戴梓就是一位在中國最早製造出具有較高射擊速度的管形火器專家,這種火器稱為“連珠火銃”。戴梓仿鑄技術比南懷仁更為高超,亦成功地仿造了沖天炮“南懷仁謂沖天炮出其國,造之一年不成。上命先生造,八日成,上大悅,率群臣親試之,即封炮為威遠將軍,鐫治法官名,以示不朽。沖天炮,子在母腹,母送子出,從天而下,片片碎裂,銳不可當。後征噶爾靼,以三砲墜其營,遂大捷”。文獻記載的“連珠火銃”與故宮所藏的一支康熙年間外國進獻的火槍十分相似,然而在因为冲天炮事件中得罪了南怀仁,被诬陷“私通东洋”,康熙将戴梓流放到了盛京(今沈阳)。

乌兰布通之战后,康熙帝更加重视在战争中发挥火器的战斗威力,使火器营成为清军八旗兵的新的战斗编成。清军最早装备火器的是汉军八旗,随着战事频繁,满洲、蒙古八旗亦迅速装备了火器。至康熙二十二年,在每旗专设一营操练鸟枪。康熙三十年始选满洲、蒙古习火器之兵组建火器营。设鸟枪护军、鸟枪马甲和炮甲三种营兵,满洲、蒙古八旗每佐领下设鸟枪护军3人,鸟枪马甲4人,炮甲1人,共7395人。由於西方經典彈道理論在戰鬥人員中逐漸普及,火器命中率的提高,極大地提高了火力武器的殺傷力。因此,火器在康熙以後不僅成為八旗的主要武器裝備,而且清軍還產生了更專門的火器營的戰鬥編成,完全改變了清軍以騎射為主的傳統作戰方式。

康熙崇尚儒学,尤其是程朱理学。他曾多次举办博学鸿儒科,创建了南书房制度,并亲临曲阜拜谒孔庙。康熙还组织编辑与出版了《康熙字典》、《古今图书集成》、《曆象考成》、《数理精蕴》、《康熙永年历法》、《康熙皇舆全览图》等图书、历法和地图。

康熙對於宗教基本上是寬容的,不僅僅是漢傳佛教,或者滿洲的藏傳佛教、薩滿教信仰,还褒封道教白云观方丈王常月,并皈依于门下。他甚至也時常聽天主教傳教士講道。直到他发现罗马教廷试图干预中國政治,并且皇子信仰基督后以此作为争权夺利的工具,遂开始有所抵制天主教,即中國禮儀之爭。

康熙也利用戴南山(戴名世)的南山案文字獄事件,株連甚多,來抑制漢族士大夫的反叛思想,甚至桐城派文家方苞都差點遭斬首。

康熙是中国历史上少有的重视自然科学的皇帝,对西方文化也十分感兴趣,自身具有相当高的科学素养,向来华传教士学习代数、几何、天文、医学等方面的知识,并颇有著述。例如:曾从南怀仁学习欧几里得《几何原本》並且每天听讲。后来又学习西方的测量、天文、物理和医学等知识,并在宫中设置了研究化学和药学的实验室。康熙因南怀仁督造火炮方面的功绩,一直对他优礼有加,而南怀仁等西方传教士也促进了伽利略的弹道理论在中国的传播。

康熙除了學習西方科技之外還會應用實踐,其最突出的是用科學方法和西方儀器繪製全國地圖。康熙亦會利用巡行和出兵之便,實地測量,吸取經驗。在康熙四十六年(1707年)委任耶穌會傳教士雷孝思、白晉、社德美及中國學者何國楝、明安圖等人走遍各省,運用當時最先進的經緯圖法、三角測量法、梯形投影技術等在全國大規模實地測量,並於康熙五十七年(1718年)繪製成《康熙皇輿全覽圖》,其作被稱為在當時世界地理學的最高成就,英國李約瑟亦稱之為不但是亞洲當時所有的地圖中最好的一幅,而且比當時的所有歐洲地圖都要好、更精確。

康熙還以巡視之便訪求民間的有才之士,例如將在數學方面有很大成就的梅毂成調進宮中培養深造。梅毂成亦通過學習西方數學知識,重新令在明朝被廢棄的中國古代數學受到重視。

由於康熙帝是中國歷代帝王中最重视科学、最提倡科学和最精通科学的人,故後代有很多评判和標籤加在他身上,他被視為有重大贡献的「科学家皇帝」,或被視為是「窒塞民智」的「罪魁祸首」。有學者及歷史學家認為,清朝中後期國力開始遠遠落後於西方,跟康熙晚年墨守成规和缺乏创新有關,故他应当为中国科技的落后状况负责任;此外,亦有學者認為,康熙由于自身的局限性,对當時的科学内容采取又用之又防之的手段,他又担心先进的西方科技一旦传开,将会极大的动摇以骑射起家的满清的统治,另外,康熙亦被批評阻礙了中國火器的發展。

此外,由於传教士们所宣扬的基督宗教教義与中国的传统文化观念之有很大的差异和分歧,故西學受當時中国各阶层保守人士竭力反对,清初保守派官员楊光先就強調「宁可使华夏无好历法,不可使中国有西洋人」,對傳播西學的傳教士表示不滿。面對士大夫的不滿情緒以及罗马教廷對中国文化礼俗的傲慢,作為中华文化正统的最高代表,康熙特意对理学名臣李光地、熊赐履等说:“汝等知西洋人渐渐作怪乎?将孔夫子亦骂了。予所以好待他者,不过是用其技艺耳。历算之学果然好。你们通是读书人,见外面地方官与知道理者,可俱道朕意。”希望借助他們剖白他为何使用传教士及其底线所在。與批評西學為「奇技淫巧」的守旧派官僚不同,願意学习和提倡西学的康熙对西学采取较开明的态度。

康熙对国家的治理中对“汉学”传统的学习与推崇,从各方面接受并正确执行汉族政策,充分正视和运用“汉家”的传统意识,为开创鼎盛局面打下基础。但是康熙作为“天下之主”,为了维护清朝的根本利益,极力标榜“满汉一体”。但是,受本民族利益的驱使和民族情感的困扰,他往往自觉或不自觉地陷入偏徇满洲的境地,在噶礼和张伯行互参案中体现出来。

1690年至1697年多次擊敗准噶尔部噶尔丹,史稱三征噶爾丹。在雅克萨战役,康熙派遣黑龙江将军萨布素成功驱逐沙俄对黑龙江流域的侵略,收復了雅克薩城(舊稱阿爾巴津;現俄罗斯联邦斯科沃罗季诺)和尼布楚城(现俄罗斯联邦涅尔琴斯克)。他在京师东北的热河营建了避暑山莊,将其作为蒙古、西藏、哈萨克等部王公贵族觐见的场所,为清朝大肆的修建皇家园林开辟了先河。

亦有史學家指出,康熙會欣賞和重用有才華的傳教士,西方先進的科學技術也被推崇和應用。康熙曾經委派傳教士閔明我(Domingo Fernández Navarrete)返回歐洲招募人才,希望增進中西方科技文化交流。而民間與西方傳教士能夠互相交遊,西學在社會中得以自由傳播,亦指出分別由康乾皇帝敕輯的叢書-《古今圖書集成》和《四庫全書》亦收錄了傳入中國的西方科學技術。

据传教士张诚(J. F. Gerbillon)的日记記載,康熙為了保護傳教士不被其他官員陷害而不准他們在有汉人和蒙古人的衙门裏翻译任何科学文献。18世纪康熙末期,因罗马教廷發出禁止中国人教徒祭祖的禁令而引发礼仪之争,促使清廷反制并下令“自今以后,若不遵利玛窦规矩,断不准在中国住,必逐回去”。

中俄开始正式接触是在康熙帝时期,签订了《尼布楚条约》以后,两国贸易逐渐繁荣。1715年,俄国传教士首次来华,加强了两国经济、文化之间的交流。康熙晚年,因为俄商来华人数众多;更重要的是俄方一些行为违背了康熙关于安全、和平的原则,因而使中俄关系形势逆转。

然而有文獻記載指出,在清朝康熙年間,原本閉關鎖國的中國逐漸向外界開放,並維持著國內、近鄰貿易以及歐洲貿易。甚至說「全歐洲的貿易量都無法跟巨大的中國貿易量相比」,並且形容中國的各個省就相當於歐洲的各個王國,它們各自擁有自己豐富且多種多樣的特產進行貿易,而且這傾向於聯盟保護的形式,在所有的城市裡也一樣,以至官員們在商業界裡都擁有自己的股份/分成,他們當中有部分人會將他們的金錢委託給值得信任的人打理以保證他們的資產在商業往來中取得成果,連平民百姓也可以從商業貿易中得益 。同時記載了清朝市集的繁華程度和中外商家的貿易情況,又稱中國商人在交易時都很誠實可靠,跟日本、巴達維亞(今印尼雅加達)、馬尼拉以及歐洲也有貿易來往。《全球通史》裡亦指出,康熙時期中國的對外貿易急劇膨脹且發展快速,大量的茶葉、絲綢、棉布、瓷器和漆器經廣州口岸運往歐洲銷售。

华裔日籍学者杨启樵说:“康熙宽大,乾隆疏阔,要不是雍正的整饬,清朝恐早衰亡。”

英國籍史學家史景迁批评康熙有三:一是皇位继位的纠葛进退失据;二是康熙虽喜爱西学,任用耶稣会士,并允传教,但对西方并不信任,因而有礼仪之争以及导致雍正禁教;三是康熙以轻徭薄赋自豪,以此彰显盛世,但其永不加赋的政策按耕地面积缴固定税金,与人口无关,于是人口虽增,亦不加赋,为康熙的继承者造成财政困难。

法國人白晉:「康熙皇帝經常到各地巡視,以便了解百姓的生活情況和官吏們的施政狀況。在這樣的觀察時,即便最卑賤的工匠和農夫,皇帝也允許他們接近自己,並用非常親切溫和的態度詢問他們,這常常使得普通百姓至為感動。康熙皇帝會經常向百姓提出各種問題,​​並且他一定要問到的問題是他們對當地的官吏是否滿意。如果百姓普遍傾訴對某個官員的不滿,康熙皇帝會將他撤職。但是如果百姓讚揚到某個官員,他卻並不一定僅僅因此就提升那個官員。」;「康熙皇帝的孝順和感恩是如此罕見,他因此獲得了舉國百姓的尊敬和擁戴。」。白晉亦提到康熙對賑濟災區與安撫饑民的手法:「我們在北京的其中兩年,我們親眼目睹了以下這些確證的事實。當時,兩三個省遭受了大旱災,造成農業嚴重欠收。康熙皇帝為此深為憂慮,他免除了這些省份的賦稅,並設立常平倉進行賑恤,但仍不能滿足災區的需要,於是,他又向災情最為嚴重的地區調撥了大量的糧食和巨額款項。為了進一步賑濟災區的窮人,康熙皇帝又採取了捐官的政策,允許富人中有學識的人,如果能夠通過做官資格的考試、證明他確有才幹,並向災區捐獻一定數目的糧食,便可買到一個相應的官職。當時,為了尋求生路,大量的窮人紛紛湧入北京,康熙皇帝下令把這些人全都招用於六部官署的建築工程,從而找到一個既幫助了窮人又使他們對社會有所貢獻的辦法。並且,這個辦法也有利於安撫饑民,防止他們因走投無路而發生動亂。」

比利時人南懷仁:「(康熙帝)親切地接近老百姓,力圖讓所有人都能看見自己,就像在北京時的慣例一樣,他諭令衛兵們不許阻止百姓靠近。所有的百姓,不管男女,都以為他們的皇帝是從天而降的,他們的目光中充滿異常的喜悅。為一睹聖容,他們不惜遠涉跑來此地,因為,對他們來講,皇帝親臨此地是從不曾有過的事情。皇帝也非常高興於臣民們赤誠的感情表露,他盡力撤去一切尊嚴的誇飾,讓百姓們靠近,以此向臣民展示祖先傳下來的樸質精神。」

康熙帝幼年继位,立志“为治天下而学”,终身好学不倦,同时勤习骑射,弓马娴熟,体格健壮。其中,刻苦的学习精神和良好的读书方法对他治国理政具有不可替代的作用。康熙从少年时代开始直到晚年,对古代书家作品的学习都不曾间断。《石渠宝笈》和《佩文斋书画谱》著录了较多康熙对古代书迹的题跋。

康熙帝也是一位重视自然科学、精通医道的养生家,相传,八宝豆腐和康熙帝也有渊源。但是康熙晚年多病缠身,还患有高脂血症,这多少与他的饮食失衡有关。

\subsection{康熙}

\begin{longtable}{|>{\centering\scriptsize}m{2em}|>{\centering\scriptsize}m{1.3em}|>{\centering}m{8.8em}|}
  % \caption{秦王政}\
  \toprule
  \SimHei \normalsize 年数 & \SimHei \scriptsize 公元 & \SimHei 大事件 \tabularnewline
  % \midrule
  \endfirsthead
  \toprule
  \SimHei \normalsize 年数 & \SimHei \scriptsize 公元 & \SimHei 大事件 \tabularnewline
  \midrule
  \endhead
  \midrule
  元年 & 1662 & \tabularnewline\hline
  二年 & 1663 & \tabularnewline\hline
  三年 & 1664 & \tabularnewline\hline
  四年 & 1665 & \tabularnewline\hline
  五年 & 1666 & \tabularnewline\hline
  六年 & 1667 & \tabularnewline\hline
  七年 & 1668 & \tabularnewline\hline
  八年 & 1669 & \tabularnewline\hline
  九年 & 1670 & \tabularnewline\hline
  十年 & 1671 & \tabularnewline\hline
  十一年 & 1672 & \tabularnewline\hline
  十二年 & 1673 & \tabularnewline\hline
  十三年 & 1674 & \tabularnewline\hline
  十四年 & 1675 & \tabularnewline\hline
  十五年 & 1676 & \tabularnewline\hline
  十六年 & 1677 & \tabularnewline\hline
  十七年 & 1678 & \tabularnewline\hline
  十八年 & 1679 & \tabularnewline\hline
  十九年 & 1680 & \tabularnewline\hline
  二十年 & 1681 & \tabularnewline\hline
  二一年 & 1682 & \tabularnewline\hline
  二二年 & 1683 & \tabularnewline\hline
  二三年 & 1684 & \tabularnewline\hline
  二四年 & 1685 & \tabularnewline\hline
  二五年 & 1686 & \tabularnewline\hline
  二六年 & 1687 & \tabularnewline\hline
  二七年 & 1688 & \tabularnewline\hline
  二八年 & 1689 & \tabularnewline\hline
  二九年 & 1690 & \tabularnewline\hline
  三十年 & 1691 & \tabularnewline\hline
  三一年 & 1692 & \tabularnewline\hline
  三二年 & 1693 & \tabularnewline\hline
  三三年 & 1694 & \tabularnewline\hline
  三四年 & 1695 & \tabularnewline\hline
  三五年 & 1696 & \tabularnewline\hline
  三六年 & 1697 & \tabularnewline\hline
  三七年 & 1698 & \tabularnewline\hline
  三八年 & 1699 & \tabularnewline\hline
  三九年 & 1700 & \tabularnewline\hline
  四十年 & 1701 & \tabularnewline\hline
  四一年 & 1702 & \tabularnewline\hline
  四二年 & 1703 & \tabularnewline\hline
  四三年 & 1704 & \tabularnewline\hline
  四四年 & 1705 & \tabularnewline\hline
  四五年 & 1706 & \tabularnewline\hline
  四六年 & 1707 & \tabularnewline\hline
  四七年 & 1708 & \tabularnewline\hline
  四八年 & 1709 & \tabularnewline\hline
  四九年 & 1710 & \tabularnewline\hline
  五十年 & 1711 & \tabularnewline\hline
  五一年 & 1712 & \tabularnewline\hline
  五二年 & 1713 & \tabularnewline\hline
  五三年 & 1714 & \tabularnewline\hline
  五四年 & 1715 & \tabularnewline\hline
  五五年 & 1716 & \tabularnewline\hline
  五六年 & 1717 & \tabularnewline\hline
  五七年 & 1718 & \tabularnewline\hline
  五八年 & 1719 & \tabularnewline\hline
  五九年 & 1720 & \tabularnewline\hline
  六十年 & 1721 & \tabularnewline\hline
  六一年 & 1722 & \tabularnewline
  \bottomrule
\end{longtable}


%%% Local Variables:
%%% mode: latex
%%% TeX-engine: xetex
%%% TeX-master: "../Main"
%%% End:
