%% -*- coding: utf-8 -*-
%% Time-stamp: <Chen Wang: 2019-10-22 10:42:07>

\section{高宗\tiny(1736-1795)}

乾隆帝(1711年9月25日-1799年2月7日),爱新觉罗氏,名弘曆,是清朝自入关以來的第四位皇帝,1735年10月18日—1796年2月9日在位,年号「乾隆」。西藏方面尊其為「文殊皇帝」。死后廟號高宗,諡號簡稱純皇帝,葬清東陵中的裕陵。

乾隆帝是满洲镶黄旗人,为雍正帝第四子,生於康熙五十年八月十三日(1711年9月25日)子時。登基於雍正十三年(1735年10月18日),在位至乾隆六十年(1735—1796年)。因其继位之时有在位时间不越祖父康熙帝之誓言,故而禅位于其子颙琰(即後來的嘉庆帝)。此时的乾隆虽为太上皇,但依然“训政”,在宫内仍然沿用乾隆年号,成為事實上的最高统治者,直至驾崩於嘉庆四年正月初三日(1799年2月7日)辰刻,享壽89岁,是中國歷史上最長壽的皇帝以及中国历史上實際掌權(執政)時間最长的皇帝(合共64年)。

弘历於康熙五十年八月十三日(1711年9月25日)出生,為雍正帝胤禛第四子,幼名「元寿」。当时,其父胤禛为雍亲王,生母为藩邸格格钮祜禄氏(孝聖憲皇后)。弘历生于雍王府东书院“如意室”。他被认为是雍正帝诸子中最有才干的一位,自小甚得其祖父康熙帝与父亲喜愛,虽然祖孙真正相处的时间並不长,但康熙帝曾為其慎择良师,进行多方面教育。在许多记载中也显示康熙帝对这个孙子十分疼愛。一些清史学家认为正因為康熙帝认为弘历在为人处事的方式上与自己极为相像,在十数岁时就精于武术,並对艺术创作十分著迷,所以才传位于其父,以便将來能传位与弘历。

康熙六十一年(1722年)十一月十三日,康熙帝驾崩前宣诏嗣位。二十日其父登基,是为雍正帝。第二年即雍正元年(1723年),雍正帝秘密建储亲书逾臾旨,密封遗诏藏于正大光明匾额后。同年三月,雍正亲生儿女中,只追封皇二女爵位為和碩怀恪公主,雍正帝尚活着的兒子弘時、弘曆、弘晝、福惠、福沛均未封爵位(其生母分別是齐妃、熹妃钮祜禄氏、纯愨皇贵妃、敦肃皇贵妃)。

雍正元年(1723年)十一月,雍正命皇四子弘历祭景陵。

雍正二年(1724年)十一月,適逢康熙忌辰,雍正命皇四子弘历祭景陵。

雍正四年(1726年)五月,適逄孝恭仁皇后三周忌辰,雍正帝欲要亲往祭陵。王大臣等,以圣躬素畏炎暑,万几已极劳苦,又触热往返五六百里,洵非所宜,且二麦登场,一路夫役祗候,不免耽误农功,因合词恳请停止。雍正皇帝勉从所请,因此命皇四子弘历前往行礼。

雍正四年(1726年)十一月,雍正下令,遣官祭永陵、福陵、昭陵、昭西陵、孝陵、孝东陵。命皇四子弘历祭景陵。

雍正四年(1726年)十二月,命皇四子弘历、庄亲王允禄,视马武疾。雍正谕,曰:「马武抱病危笃,闻之深为凄恻,马武事我皇考康熙五十余年,朝夕侍奉不离左右、恪恭谨慎,事事能仰体圣心」。馬武病故後。命皇四子弘历、怡亲王允祥、庄亲王允禄、及左翼四旗部院大臣、一等侍卫,往奠故内大臣马武茶酒。

雍正五年(1727年),皇四子弘历娶妻富察氏。完婚後,弘历由紫禁城的毓庆宮移居乾西二所(日后改名為重华宮)。

雍正六年(1728年),他的一位妾室富察氏為他生下了第一个孩子长子永璜。

雍正九年(1731年)八月,大学士忠达公、抚远大将军马尔赛起程,命皇四子弘历告祭奉先殿。王以下官员,俱至西长安门外送行。

雍正十年(1732年)春正月,享太庙,命皇四子弘历行礼。

雍正十一年(1733年),雍正皇帝第一次封皇子爵位(那时雍正帝儿子只剩下兩位:熹妃钮祜禄氏子弘历与裕妃子弘晝):弘历封和碩宝亲王爵位、弘昼封和硕和亲王爵位。住地獲赐名「乐善堂」,未外设王府。

雍正十二年(1734年)夏四月,享太庙,命皇四子宝亲王弘历行礼。

雍正十三年(1735年)五月。雍正命果亲王允礼、皇四子宝亲王弘历、皇五子和亲王弘昼、及大学士鄂尔泰、张廷玉、户部尚书公庆复、礼部尚书魏廷珍、刑部尚书宪德、张照、工部尚书徐本、正红旗汉军都统李禧、正黄旗汉军都统甘国璧、仓场侍郎吕耀曾,俱办理苗疆事务。

由于皇四子弘历行事恩威並施,手段宽猛相济,雍正帝指派他钦差出京办事,以及参与西北準部用兵西南改土归流的決策。在雍正後期,使自己渐得到了父亲恩宠。

雍正十三年(1735年)八月二十三日(10月8日),其父雍正帝駕崩,享年五十七歲。宣讀遺詔:“寶親王皇四子(弘曆),……聖祖康熙皇帝於諸孫之中,最為鍾愛,撫養宮中,恩逾常格……與和親王(弘晝)同氣至親,實為一體……俾皇太子弘曆成一代之令主。”寶親王皇四子弘曆登基,是為乾隆帝。乾隆弘曆以雍正駕崩前遺命尊生母熹貴妃鈕鈷祿氏為崇慶皇太后。封和親王弘昼之母裕妃耿氏為皇貴太妃。九月,曾撫養過弘曆為皇子時的慈母愨惠皇貴妃、惇怡皇貴妃各加封號晉封太妃(雍正不立兩位前朝的皇考太妃之位)。十一月,大学士等议奏崇慶皇太后父四品典仪官凌柱封一等承恩公、母為一品夫人。

同時遺詔命莊親王允祿、果親王允禮、大學士鄂爾泰、張廷玉為輔政大臣,輔佐新君處理政務。

乾隆十年(1745年) 正月,魏貴人詔封為令嬪。(即,後來嘉慶帝的生母)

乾隆上位後,命人編纂《國朝官史》。 收錄以下雍正皇帝諭旨:

雍正元年正月,上諭:諸皇子入學之日,與師傅豫備杌子四張,高桌四張,將書籍筆硯表裏安設桌上。皇子行禮時,爾等力勸其受禮,如不肯受,皇子向座一揖,以師儒之禮相敬。如此則皇子知隆重師傅,師傅等得以盡心教導,此古禮也。朕為藩王時,在府中亦如此行。至桌張飯菜,爾等照例用心預備。

雍正八年三月,上諭:諭總管太監傳與各處首領太監知悉:阿哥現居宮內,年已長成,爾等不可趨奉,亦不可得罪,並不許向阿哥處往來行走。即阿哥下太監亦不許與爾等所屬太監飲酒、下棋、鬥骨牌、說閑話。除趙進朝、靳進忠、趙運祥、楊進朝四人奉旨行走,不必攔阻外,其餘各處首領太監,嚴加曉諭,小心遵行,不可日久懈怠。嗣後如有玩法之人,經朕察出,係宮內太監,治宮內總管之罪;係圓明園太監,治圓明園總管之罪。

乾隆帝即位後,以「宽猛相济」理念施政,先後平定新疆、蒙古,还使四川、贵州等地继续改土归流,人口不断增加,在乾隆末年时突破三亿大关,约占当时世界人口的三分之一。乾隆三十八年(1773年)下令编纂《四庫全書》,歷時9年成書,是当时世界上最为庞大的百科全书。统治期间与康熙、雍正二朝合稱「康雍乾盛世」。

同時,乾隆為了打擊朋黨以及加強對人民主要是漢人的思想控制,大興文字獄,並藉此焚書箝制漢人反清思想的傳播。郭成康指出,乾隆查辦禁書目的就是要徹底消滅部分漢人中的反滿思想;然而,乾隆當時民族矛盾和鬥爭的情況已經逐漸緩和、並且在漢族臣民已承認清朝對全國統治的情況下,乾隆將民族矛盾和鬥爭的嚴重性誇大,在有關文字獄和禁書的決定中作錯誤估計,並且表現得過度敏感。此外,在乾隆時期的文字獄,針對的並非只有漢族,犧牲者中亦有滿族如鄂昌。

中期以後,乾隆多次下江南,有安撫百姓,檢閱軍隊,視察水利,增加科舉以及免除税收之舉。

乾隆五十一年十一月廿六日(1787年1月16日),台灣爆發林爽文事件,滿清雖利用台灣閩客之間的族群對立,但戰事曠日廢時,要至福康安率大兵登陸後,方於四個月內鎮壓此亂。並將林爽文凌遲斬首,女眷發放邊疆做奴,十五歲以下男童連坐犯被押解至北京閹割。

乾隆五十八年(1793年),英國遣使喬治·馬戛爾尼于乾隆83歲時到中國尋求駐節,但雙方出現與乾隆皇帝會面採「單膝下跪」(英方主張)或「三跪九叩」(中方主張)的禮儀之爭,最后以“单膝下跪”而为礼。

喬治·馬戛爾尼在回國後向英国议会写出报告:“中国是一艘破旧的大船,150年来,它之所以没有倾覆,是因为幸运的遇见了极为谨慎的船长。一旦赶上昏庸的船长,这艘大船随时就可能沉没。中国根本就没有现代的军事工业,中国的军事实力比英国差三到四个世纪”。而在馬戛爾尼的日記中卻有以下記載:“中国工业虽有数种,远出吾欧人之上,然以全体而论,化学上及医学上之知识,实处于极幼稚之地位。”,又稱:「中國政府的行政機制和權力是如此的有組織和高效,有條件能夠迅即排除萬難,創造任何成就。」。

為了打擊腐败之風,乾隆鼓励人們秘密向他汇报官员们的可疑行为,收受贿赂、欺诈、任人唯亲、滥用职权和瞒报等,例如福建大獄案,至于控诉的真假则由皇帝决定,在其统治初期坚定了惩治贪腐的决心,下令任何案件只要涉赃额超过一千两,案犯就将斩立决;然而到了乾隆统治的後半期,官员贪污这一严重问题再次出现,到了晚期每隔几年就会爆出一些重大案件及弹劾案,當時年迈的乾隆已经没有初時的魄力去严惩官员们的渎职行为,有學者指出:「從乾隆看来,在这些欺诈行为中也存在一些积极因素,其中之一便是所有被没收的贪官污吏的家产都流入了乾隆的腰包,大大增加了他的财富。而财政赤字和粮食亏空则由那些被免官员的继任者负责。另一个积极因素是满、汉官员都卷入了这种犯罪,这样乾隆就无须担心存在汉官通过腐败来故意破坏国家政治体制的阴谋。但是,看到那些本应更加效忠皇帝的满洲官员同样也在做着有损皇帝统治之事时,乾隆也会感到不太舒服。不过,好在还有一些值得依靠的、公正廉明的官员让乾隆感到些许安心,这些人对乾隆总是以诚相待,不收受贿赂,不会为了一己私利而欺君罔上。他们之中多数是满洲人,包括阿桂和傅恒」。

乾隆五十年之後,睡眠减少,“寅初已懒睡,寅正无不醒。”,左眼视力下降,年过七十之后,“昨日之事,今日辄忘;早间所行,晚或不省。”

乾隆25歲登基時表示過若蒙眷佑,得在位六十年,即當傳位嗣子,不敢上同皇祖紀元六十一載之數。因此在乾隆六十年九月初三日(1795年10月15日)85歲的乾隆將皇位傳予十五子顒琰(嘉庆帝),自稱太上皇,但軍國大事及用人皆由乾隆躬親指教,嘉慶帝朝夕敬聆訓聽;宫中仍用“乾隆”年号,中国第一历史档案馆藏《万岁爷进药底簿》封皮上书“乾隆六十四年”。嘉慶四年正月初三日(1799年2月7日),乾隆太上皇駕崩於北京紫禁城養心殿內,享壽八十八歲,結束了長達六十三年又四個月的統治。廟號清高宗,諡號純皇帝。死後與其愛妻孝賢純皇后合葬於清裕陵。

嘉庆四年二月二十一日,总管张进喜传旨交如意馆绘画太上皇帝圣容一轴,大边上花纹按照安佑宫供奉的圣祖和世宗圣容挂轴上的大边花纹式样绘画。圆明园四十景之鸿慈永祜的主体建筑安佑宫殿內便供奉他的画像。嘉庆四年四月的圆明园文开稱,安佑宫供奉高宗纯皇帝圣容,照例供献用宫香饼一觔、小黑芸香五两和小白芸香五两。

1928年,乾隆去世近一百三十年後,軍閥孫殿英看準了乾隆帝陵墓及慈禧太后陵墓的珍貴財寶,藉演習之名,率其部下盜掘乾隆帝及慈禧太后之陵墓。士兵為得棺內珠寶,將乾隆梓棺劈開並大肆搜掠,乾隆帝后遺骸四散在地,情況奇慘;及後溥儀派人前往收拾,亦只能找回部份遺骸,勉強砌回主體,並將帝后遺骸合葬一棺,重新行葬。

乾隆帝好詩、書、畫,作品極多,作詩多達四萬首(38630首)。其作品多采用「御題」做题跋。紫禁城宮殿內絕大部份的匾額,楹聯,亦是出自其御筆。乾隆有在宮中收藏的名家書畫上題詩用印的嗜好,被認為有一定的史料價值,但这种行为也破坏了原作品的艺术价值。

乾隆五十七年,乾隆親自撰寫成《十全武功記》,自詡「十全老人」。命人以滿、漢、蒙、藏四種文字刻碑,昭示其武功。「十全武功」指“平準噶爾為二,定回部為一,掃金川為二,靖台灣為一,降緬甸、安南各一,即今二次受廓爾喀降,合為十”。

在各种民间传说中,乾隆帝被描绘成风流天子。民国初年,就盛行香妃的传说。至今,关于香妃以及乾隆帝与平民女子的爱情故事为主题的各类文学、戏剧、影视作品,络绎不绝。另外在大臣中,乾隆帝对傅恒之子福康安最为优待。民国后,多传说福康安为他与傅恒妻的私生子,但黄一农等学者已考证此说不确。

民间对乾隆帝六次南巡亦多有演绎,或称之“乾隆下江南”。当代广告中,声称乾隆帝在南巡过程中曾品尝过某种美食的例子不胜枚举。

坊間野史指乾隆之母为汉人(非汉军旗),甚至有出自漢人之家(海寧陳家),並非雍正帝親生子之說,故認為乾隆为漢族后裔,但未被完全證實。乾隆帝的六次南巡亦被认为是去浙江海宁探望亲生父母陈世倌夫妇。金庸所著的武俠小說《書劍恩仇錄》正是以此傳說為藍本。

另外,關於乾隆出生之處也有爭議,一說在雍親王府(雍和宮),另一說則是在承德避暑山莊獅子園,而且避暑山莊一說是由嘉慶皇帝親口提起,這也是野史會傳出乾隆是由避暑山莊漢人宮女所生的原因。

\subsection{乾隆}

\begin{longtable}{|>{\centering\scriptsize}m{2em}|>{\centering\scriptsize}m{1.3em}|>{\centering}m{8.8em}|}
  % \caption{秦王政}\
  \toprule
  \SimHei \normalsize 年数 & \SimHei \scriptsize 公元 & \SimHei 大事件 \tabularnewline
  % \midrule
  \endfirsthead
  \toprule
  \SimHei \normalsize 年数 & \SimHei \scriptsize 公元 & \SimHei 大事件 \tabularnewline
  \midrule
  \endhead
  \midrule
  元年 & 1736 & \tabularnewline\hline
  二年 & 1737 & \tabularnewline\hline
  三年 & 1738 & \tabularnewline\hline
  四年 & 1739 & \tabularnewline\hline
  五年 & 1740 & \tabularnewline\hline
  六年 & 1741 & \tabularnewline\hline
  七年 & 1742 & \tabularnewline\hline
  八年 & 1743 & \tabularnewline\hline
  九年 & 1744 & \tabularnewline\hline
  十年 & 1745 & \tabularnewline\hline
  十一年 & 1746 & \tabularnewline\hline
  十二年 & 1747 & \tabularnewline\hline
  十三年 & 1748 & \tabularnewline\hline
  十四年 & 1749 & \tabularnewline\hline
  十五年 & 1750 & \tabularnewline\hline
  十六年 & 1751 & \tabularnewline\hline
  十七年 & 1752 & \tabularnewline\hline
  十八年 & 1753 & \tabularnewline\hline
  十九年 & 1754 & \tabularnewline\hline
  二十年 & 1755 & \tabularnewline\hline
  二一年 & 1756 & \tabularnewline\hline
  二二年 & 1757 & \tabularnewline\hline
  二三年 & 1758 & \tabularnewline\hline
  二四年 & 1759 & \tabularnewline\hline
  二五年 & 1760 & \tabularnewline\hline
  二六年 & 1761 & \tabularnewline\hline
  二七年 & 1762 & \tabularnewline\hline
  二八年 & 1763 & \tabularnewline\hline
  二九年 & 1764 & \tabularnewline\hline
  三十年 & 1765 & \tabularnewline\hline
  三一年 & 1766 & \tabularnewline\hline
  三二年 & 1767 & \tabularnewline\hline
  三三年 & 1768 & \tabularnewline\hline
  三四年 & 1769 & \tabularnewline\hline
  三五年 & 1770 & \tabularnewline\hline
  三六年 & 1771 & \tabularnewline\hline
  三七年 & 1772 & \tabularnewline\hline
  三八年 & 1773 & \tabularnewline\hline
  三九年 & 1774 & \tabularnewline\hline
  四十年 & 1775 & \tabularnewline\hline
  四一年 & 1776 & \tabularnewline\hline
  四二年 & 1777 & \tabularnewline\hline
  四三年 & 1778 & \tabularnewline\hline
  四四年 & 1779 & \tabularnewline\hline
  四五年 & 1780 & \tabularnewline\hline
  四六年 & 1781 & \tabularnewline\hline
  四七年 & 1782 & \tabularnewline\hline
  四八年 & 1783 & \tabularnewline\hline
  四九年 & 1784 & \tabularnewline\hline
  五十年 & 1785 & \tabularnewline\hline
  五一年 & 1786 & \tabularnewline\hline
  五二年 & 1787 & \tabularnewline\hline
  五三年 & 1788 & \tabularnewline\hline
  五四年 & 1789 & \tabularnewline\hline
  五五年 & 1790 & \tabularnewline\hline
  五六年 & 1791 & \tabularnewline\hline
  五七年 & 1792 & \tabularnewline\hline
  五八年 & 1793 & \tabularnewline\hline
  五九年 & 1794 & \tabularnewline\hline
  六十年 & 1795 & \tabularnewline
  \bottomrule
\end{longtable}


%%% Local Variables:
%%% mode: latex
%%% TeX-engine: xetex
%%% TeX-master: "../Main"
%%% End:
