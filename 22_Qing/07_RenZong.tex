%% -*- coding: utf-8 -*-
%% Time-stamp: <Chen Wang: 2019-10-22 10:44:56>

\section{仁宗\tiny(1795-1820)}

嘉慶帝(1760年11月14日-1820年9月2日)爱新觉罗氏,名顒琰,是清朝自入关以来的第五位皇帝,1796年2月9日至1820年9月2日在位,年号「嘉慶」。西藏方面尊為「文殊皇帝」。死後廟號仁宗,諡號睿皇帝,葬于清西陵中的昌陵。

嘉慶帝乃乾隆帝第十五子,原名永琰,乾隆六十年立为皇太子,為避免他人避諱麻煩而改名颙琰。1796年2月9日接受父親乾隆帝禪位而成為皇帝,但乾隆帝仍以太上皇身份「訓政」。1799年2月7日乾隆帝逝世後,嘉慶帝才得以掌握實權。

1760年11月14日(乾隆二十五年十月初六日)顒琰出生,初名永琰,是乾隆帝第十五子,母魏佳氏。乾隆三十八年冬至,以秘密建储制,立颙琰为大清皇太子。乾隆五十四年,封永琰為嘉親王。清代以秘密建儲制指定皇位繼承人,乾隆連兩次建儲,惜指定人選均早早去世;到乾隆晚年才第三次立儲,為十五子永琰。窮通寶鑑上記載嘉慶的出生時辰為丑時。不過,其書另外也提供一個可能的時辰即戌時。

乾隆六十年(1795年)九月辛亥,乾隆帝在勤政殿,召皇子、皇孙、王公大臣等入见,宣示十五子永琰为皇太子,第二年传位于他,是为嘉庆元年,並改皇太子名為顒琰。而太子顒琰及王公大臣等到相继上奏请求等到“寿跻期颐”(乾隆帝一百岁时),再举行归政典礼。

虽最终乾隆帝禪位顒琰,但最初四年,乾隆帝仍以太上皇名義掌朝;至嘉慶四年(1799年)乾隆駕崩,顒琰開始親政,是為嘉慶帝,時已40歲。

嘉慶四年(1799年)正月,乾隆帝去世,嘉慶帝親政僅五天即逮捕乾隆朝大權臣和珅,迅即下詔宣佈其二十大罪,將和珅賜死,抄沒其家產。親政第十五天,就將和珅一黨全部打倒。政府歲入七千萬兩白銀,而和珅以二十年之閣臣,其所蓄當一國十五年歲入半額而強,時人謂「和珅跌倒,嘉慶吃飽」。

嘉慶帝對貪污深惡痛絕,翰林院侍講梁同書「恭錄嘉慶七年御制罵廷臣詩」:「滿朝文武著錦袍,閭閻與朕無分毫;一杯美酒千人血,數碗肥羹萬姓膏。人淚落時天淚落,笑聲高處哭聲高;牛羊付與豺狼牧,負盡皇恩為爾曹。」但他卻拿不出治貪的辦法,他的治貪方式僅針對和珅一人,不肯擴大掃蕩層級,以致於收效有限,更無以改變朝廷全面性的腐化,尤其到了其末期更發生嘉慶兵部大印丟失案遷出一系列朝廷腐化真相,朝野震動。

嘉慶元年(1796年)十月,四川達州爆發徐天德、王登廷領導的起義,東鄉(今宣漢)爆發冷天祿、王三槐領導的起義,以及太平孫賜俸、龍紹周等人領導的起義,史稱川楚陝農民大起義。

嘉慶四年(1799年),白蓮教首領王三槐在北京受審時的供詞提到「官逼民反」,嘉慶知道後受到很大震動。

嘉慶八年(1803年),爆發陳德在紫禁城門口行刺嘉慶案。

嘉慶十八年(1813年)九月十五日,發生了天理教民攻入皇宮事件。林清率二百名天理教徒在宦官內應下進攻紫禁城,進至隆宗門方被包圍擊敗。當時,嘉慶在熱河避暑山莊回京的途中,不在宮內。不成軍的平民,武裝進攻皇城,為唐宋明以來從未見之事。

嘉慶元年正月,是年「會計天下民數二万万七千五百六十六萬二千四十四名口」(275,662,044人)。
嘉慶十年,是年「會計天下民數三万万三千二百一十八萬一千四百三名口」(332,181,403人)。正月,颙琰诏内务府大臣严行约束太监的权力。嘉慶二十年,是年「會計天下民數三万万二千六百五十七萬四千八百九十五名口」(326,574,895人)。

1802年底,安南新国王阮福映请求改国号为“南越”。惟古代南越国包括两广和越南北部,引起嘉庆帝警惕,改赐名“越南”,沿用至今。

俄国沙皇保罗一世时,“就想惩罚一下这个高傲的邻居(指中国),只是他的逝世使军事准备停了下来”。1803年11月,沙俄枢密院致函清朝理藩院俄国沙皇亚历山大一世已经即位了,想派出使团祝贺嘉庆登基。次年3月,嘉庆帝称俄人“言辞极为和顺,用意亦然颇为诚恳”,“应即准如所请办理”。1805年9月,俄国使者戈罗夫金率242人到伊尔库斯克。清俄礼仪纠纷,直到12月底抵达外蒙古库伦,库伦办事大臣云登道尔济举办盛大宴会招待。宴会前,戈罗夫金拒绝向清朝皇帝香案行三跪九叩礼,双方不欢而散。三十多天争辩后,库伦办事大臣收嘉庆帝圣旨,“将该大使妥为护送回国,并将从该大使所收取之全部贡品一并交还”。1806年2月,戈罗夫金被迫率团离开。他在边境逗留了几个月。直到8月,接沙皇旨意,返回莫斯科。俄国一方面对清朝继续交涉,另一方面准备对清战争。戈罗夫金“坚决要求将整个阿穆尔河左岸归还俄国”,“只要她(俄国)愿意,就能够把自己的陆军和海军派到这个国家的国土上,无论是从地理位置,还是从国力来看”。但这时“在西方拿破仑的桂冠引开了我们对东方的注意”。

1805年农历四月,查禁洋人刻书传教。五月,诏内务府大臣管理西洋堂,未能严加稽查,任令传教,下部议处,其经卷检查销毁,习教佟兰等获罪。1808年农历七月,英国称帮助葡萄牙防御法国侵占澳门,保护英国贸易,派兵船9艘入侵,九月到广东香山鸡颈洋面,英军300人登岸,占据澳门三巴寺、东西炮台等,又驾舢板3艘驶入省河,至省城外十三行停泊,要求在澳门居住。两广总督吴熊光令他们回黄浦候旨。嘉庆帝指示吴熊光严加诘责,命其驶离。英军不动,清军封锁水路,断绝粮食,英军在十月间撤离。颙琰以吴熊光表现怯懦,免总督职务,戍伊犁;广东巡抚孙玉庭革职,颙琰谕示加强澳门炮台。1809年,嘉庆帝指示百龄:英吉利“素性强横诡诈”,“于本年该国货船到时,先期留心侦探,如再敢多带夷兵欲图进口,即行调集官兵相机堵剿。”1810年,农历二月,下诏令各督抚断鸦片来源。1811年农历七月,禁西洋人在内地居住,禁人民接触天主教。1814年底批准两广总督蒋攸铦主张的严禁农民为洋人服役,洋行不得私盖西式房屋以及清查商欠等。1816年,英使阿美士德访华,双方礼仪之争,由于赶路紧急,载有官服与国书的车辆未抵达,路途劳累,阿美士德坚持休息。负责觐见的官员向嘉庆帝谎称英使生病。嘉庆帝大怒,取消觐见,下令驱逐使团,不要贡品国书,次日赏了使团一些礼物,收了“贡品”,送上敕谕一道,拒绝英国提出的建立外交关系、开辟通商口岸、割让浙江沿海岛屿要求。

終嘉慶一朝,雖“宵旰勤勞,曾無一日稍紓聖慮”,但貪污問題沒有解決,剪除和珅後卻未能斬草除根,反倒更加嚴重。這時期還爆發了白蓮教、天理教等民變,社會衝突激化,鴉片流入中國、八旗的生計問題、錢糧的虧空、河道漕運的難題,清朝國勢日非。清廷只能傾全力平定叛亂。嘉慶在天理教起義平定後,頒布「罪己诏」,史稱這時期為「嘉道中衰」。

嘉慶二十五年(1820年)秋季,嘉慶帝木蘭秋獮(秋季打獵)。在到達熱河避暑山莊的次日,即嘉慶二十五年七月二十五日(1820年9月2日),嘉慶帝因天氣暑熱,旅途勞頓,可能併發心血管病或腦中風而猝死。卒年六十一歲。據說嘉慶帝是清朝在位體型最胖皇帝。

道光元年三月二十三日(1821年4月23日),嘉慶葬于昌陵(清西陵)。

終嘉慶一朝,除了賜死和珅、處斬郑源鹴、處絞富綱、平定白蓮教川楚教亂、平反並停止一些文字獄、主持為被毒殺清官李毓昌伸冤,如此之外,明顯的政績實在不多,只是一位勤政圖治的守成君主,個性循规蹈矩。他曾在《勤政愛民論》寫道:“勤政本來是為了愛民。有實心而後才有實政,有實政才能給百姓以實惠。……內外大臣應該在勤政的同時,實心實意的為百姓辦事才是……”然此期間黃河再度決堤,然而人謀不臧,用心做事的數名官員被貪官汙吏陷於不義,好官員或被罷職,或降職,甚至鞠躬盡瘁都未能獲得重視,災民大起。

乾隆晚年好大喜功,重用和珅,造成吏治敗壞,加上白莲教起义,在嘉庆时期达到高潮,嘉慶帝雖有心整治國家,接连發布整饬吏治的谕旨,但性情优柔寡断,对弊政多是惩而不杀,戒而不绝,最後政令不出紫禁城,無力解決其皇父統治期間晚年社會的矛盾,僅保持大清的盛世一陣子,吃老本的問題未解決,貪污更加嚴重,從此清朝進入了嘉道中衰。嘉慶帝一生勤政,也雅好戲劇,洪亮吉上疏指責他“恐退朝之後,俳優近習之人,熒惑聖听者不少”。嘉庆元年(1796年)正月,刚当皇帝的颙琰连看了十八天戏。当时乾隆虽退位,但仍大权独揽,颙琰无事可做。

阎崇年:嘉庆的悲剧在于:认为天下的问题都是由于和珅不好、百官不好造成的,而没有从自身找责任,也没有从制度挖根源。嘉庆在25年的皇帝生涯中,虽一件一件地解决乾隆盛世留下的危机,却又一步一步地陷入更深的危机。乾隆朝盛世下的危机,到嘉庆朝更加深重。

张宏杰:从亲政初期的伟大,到谢幕时的尴尬,嘉庆的滑落曲线如此令人叹息。他二十多年的统治,前面连着“康乾盛世”,紧接其后的,则是“鸦片战争”。正是在嘉庆皇帝的统治下,大清王朝完成了走向万劫不复的衰败的关键几步。

\subsection{嘉庆}

\begin{longtable}{|>{\centering\scriptsize}m{2em}|>{\centering\scriptsize}m{1.3em}|>{\centering}m{8.8em}|}
  % \caption{秦王政}\
  \toprule
  \SimHei \normalsize 年数 & \SimHei \scriptsize 公元 & \SimHei 大事件 \tabularnewline
  % \midrule
  \endfirsthead
  \toprule
  \SimHei \normalsize 年数 & \SimHei \scriptsize 公元 & \SimHei 大事件 \tabularnewline
  \midrule
  \endhead
  \midrule
  元年 & 1796 & \tabularnewline\hline
  二年 & 1797 & \tabularnewline\hline
  三年 & 1798 & \tabularnewline\hline
  四年 & 1799 & \tabularnewline\hline
  五年 & 1800 & \tabularnewline\hline
  六年 & 1801 & \tabularnewline\hline
  七年 & 1802 & \tabularnewline\hline
  八年 & 1803 & \tabularnewline\hline
  九年 & 1804 & \tabularnewline\hline
  十年 & 1805 & \tabularnewline\hline
  十一年 & 1806 & \tabularnewline\hline
  十二年 & 1807 & \tabularnewline\hline
  十三年 & 1808 & \tabularnewline\hline
  十四年 & 1809 & \tabularnewline\hline
  十五年 & 1810 & \tabularnewline\hline
  十六年 & 1811 & \tabularnewline\hline
  十七年 & 1812 & \tabularnewline\hline
  十八年 & 1813 & \tabularnewline\hline
  十九年 & 1814 & \tabularnewline\hline
  二十年 & 1815 & \tabularnewline\hline
  二一年 & 1816 & \tabularnewline\hline
  二二年 & 1817 & \tabularnewline\hline
  二三年 & 1818 & \tabularnewline\hline
  二四年 & 1819 & \tabularnewline\hline
  二五年 & 1820 & \tabularnewline
  \bottomrule
\end{longtable}


%%% Local Variables:
%%% mode: latex
%%% TeX-engine: xetex
%%% TeX-master: "../Main"
%%% End:
