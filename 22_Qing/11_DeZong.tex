%% -*- coding: utf-8 -*-
%% Time-stamp: <Chen Wang: 2021-11-01 17:25:48>

\section{德宗光緒帝載湉\tiny(1875-1908)}

\subsection{生平}

光緒帝(1871年8月14日-1908年11月14日),名載湉,爱新觉罗氏,是清朝自入关以来的第九位皇帝,同時是中國最後一位有正式諡號及正式廟號的皇帝,1875年2月25日至1908年11月14日在位,年号「光绪」。

光緒帝是醇贤亲王奕譞次子,也是道光帝之孙。同治帝駕崩後,他以三歲沖齡過繼給咸丰帝,因而繼承了皇位。他在幼年時由慈安太后及慈禧太后兩宮聽政。在位期间,历经甲午战争和戊戌變法。1898年戊戌變法失败后,被慈禧太后禁闭在中南海瀛台。1908年,慈禧死之前一日,光绪帝在中南海瀛台死於砒霜中毒。死后庙号德宗,谥号景皇帝,葬于清西陵中的崇陵。

他为前一位皇帝——同治帝的堂弟兼表弟。其父为宣宗(道光帝)第七子醇贤亲王奕譞,生母嫡福晋那拉氏为慈禧太后之妹。因穆宗(同治帝)为文宗(咸丰帝)独子,又早死无後,慈禧太后便以和自己血缘最近的载湉,過繼于咸丰帝,登基为帝,名义上继承咸丰帝而非同治帝的皇位。


四岁即位,主少國疑,大臣未附,兩宮太后姑允王大臣所請,依《太后垂簾章程》十一條,垂簾聽政。

光緒七年(1881年),慈安太后駕崩,慈禧太后獨自垂簾聽政。

光緒九年(1883年),中法戰爭爆發,翌年簽定《中法新約》。

光绪帝一生受慈禧太后的控制,自小由翁同龢做他的老師,但慈禧太后規定翁同龢只能教孝經,更被李連英監視。朝廷大权在成年(1890年,20歲)後,仍掌握在慈禧太后手中。

光绪十四年(1888年),光绪帝已十八岁(虚岁),而在达到他这个年龄之前,包括同治帝在内的幼年继位的清帝均已完成大婚并亲政。六月己亥,慈禧太后颁布懿旨,明年正月举行皇帝大婚典礼。婚礼完成,光绪帝即应亲裁大政。壬寅,再颁懿旨,明年二月初三日归政。慈禧太后选择自己的侄女亦是光绪帝的表姐叶赫那拉氏为其皇后。

光绪十五年正月丁卯,御史屠仁守请求慈禧太后在归政后继续批阅奏折,被斥“乖谬”。癸酉,清廷如期举行大婚典礼。此次大婚相比同治帝大婚,花费较少。二月戊寅,慈禧太后颂懿旨斥责吴大澂要求尊崇光绪帝的生父醇亲王奕譞的请求,并出示醇亲王在光绪元年的奏折,表明醇亲王的忠心。己卯,慈禧太后正式归政,宣告十九岁的光绪帝亲政。慈禧太后勉允禮親王領班軍機大臣世鐸等王大臣所請,於皇帝親政後再訓政數年。中外同辭,再三瀝懇,慈禧不得不依據《訓政細則》開始訓政,不需垂簾亦無須議政王引見大臣,其餘細則與《垂簾章程》略同,实际大权仍掌握在慈禧太后手中。此後,光緒帝逐渐建立了以翁同龢、汪鸣銮、孙家鼐、文廷式、志锐等为骨干的帝党,影響光緒親政後的作為。

光绪二十年(1894年)甲午战争爆发,坚决主战,但光緒指揮不當,加上翁同龢、李鴻章之間内斗严重,導致清朝战败,次年在《马关条约》上签字用玺。

自甲午战败后,光緒帝锐意变法革新,“不做亡国之君”,1898年頒布《明定國是詔》,表明變革決心。在慈禧的默許下,于1898年起用康有为、梁启超等推行新政,并以谭嗣同、楊銳、林旭、劉光第等四军卿架空原有的军机大臣,但受到保守派的反对。光緒在軍事上,陸軍改練洋操,為掌握軍事召袁世凱來京,下旨進行一系列整頓:
國家振興庶政,兼採西法,誠以為民立政,中西所同,而西法可補我所未及。…...今將變法之意,佈告天下,使百姓咸喻朕心,共知其君之可恃,上下同心,以成新政,以強中國,朕不勝厚望焉。

光緒二十四年(1898年)8、9月间,由于變法操之過急,坊间盛传慈禧太后有以借“天津阅兵”废弑光绪帝的阴谋。光緒帝懼怕變法失敗,聽信康有為的意見,打算不經過慈禧太后同意,亲自提拔候补侍郎袁世凯,以新式陸軍发兵,杀慈禧提拔的直隶总督兼北洋大臣荣禄,围颐和园(慈禧所居)。慈禧得知消息,立刻從颐和园返回紫禁城,發動政變幽禁光緒帝,戊戌變法宣布失敗,軍機處譚嗣同、楊銳、林旭、劉光第四軍卿以及楊深秀、康廣仁等六維新派人士被捕處死,康有為、梁啟超流亡日本,光緒帝被慈禧幽禁在三面环水的南海瀛台,对外则宣布光绪帝生病,由慈禧训政。从戊戌年(1898年)四月二十三日光绪帝下「明定国是诏」起,到政变发生的八月六日为止(西历6月11日至9月21日),整个变法维新历时不过103天,故称百日维新。

戊戌政變後,慈禧太后宣布訓政,架空光緒。光緒二十五年(己亥年)十二月二十四日(公元1900年1月24日),慈禧太后欲废光绪帝,挑选载漪之子溥儁入宫,成為光緒的義子,是为己亥立储,由於這是廢立皇帝的先兆,上海電報局總辦經元善領銜,與馬裕藻、章太炎、丁惠康、沈藎,唐才常、經亨頤、蔡元培、黃炎培等聯名抗議,且各国公使都同情光緒,否認此事的合法性,導致慈禧失敗。慈禧遂不斷召外國西醫入宮探視“上疾”。

溥儁之父载漪等权贵利用刚刚兴起的义和团排洋情绪,招引义和团进京,发生庚子事变,光绪帝與慈禧太后共同参加决定是否向八国联军宣战的御前会议,光緒表達反對與八國聯軍開戰,但他已沒有親政的權力。[來源請求]八国联军攻入北京,慈禧挟光緒帝逃至西安,並殺害珍妃。次年签定辛丑条约(庚子事变赔款)後才回北京。此后处境稍有改善,但仍被慈禧软禁。八國聯軍之後,慈禧太后啟動第三波的政治變革,稱為清末新政(或稱庚子新政)。

1908年11月14日,光緒帝逝於瀛台,比慈禧早一日驾崩,得年37歲。

清代官方文獻和宮廷檔案記載光緒帝為病死。但光緒帝在慈禧死前一日晏驾,時間過於巧合,外界對其死因歷來有諸多揣測。许多野史、宫廷回忆录包括溥仪均指出光绪帝是被人下毒所害,但对凶手的推测各不相同。中華民國成立之後,據光緒帝的御醫透露,皇帝生前的確身體並不非常健康,主因是長時間不見天日、身體欠運動、心情不佳導致飲食不正常,卻也無病重之跡象。1980年,整理崇陵光绪帝遗骨时“未发现外伤及中毒迹象”,结合官方档案上的说法,自然病死一说在当时一度成为学术界主流观点。直到2008年对清西陵光緒遺體的頭髮、遺骨、衣服及墓內外環境樣品进行检测分析后,证实光绪帝是砒霜中毒死亡。

光绪辞世时尚没有陵墓,一直到1913年(民国二年)才葬入中国最後一座帝陵——河北易县清西陵中的崇陵。1938年曾被盗。

《崇陵傳信錄》和《清稗類鈔》兩書指出:慈禧太后病危期間,曾猶豫對光緒帝要如何處置,遂以自己不久人世的消息透露給光緒帝知道,惟其近侍回報,帝曾微露喜色,故慈禧決意自己病終前,帝須先於自己命終,以免皇帝有再度親政、否定慈禧生前之佈局的可能。

清室後裔、書法家啟功指出,其曾祖父、時任禮部尚書的溥良曾親眼看到太監從病重的慈禧宮中傳出一個蓋碗,稱“是老佛爺賞給萬歲爺的塌喇”。“塌喇”在滿語中是酸奶之意。此前從未聽說過光緒帝有任何急症大病。送後不久,就由隆裕皇后的太監小德張(張蘭德)向太醫院正堂宣布光緒皇帝駕崩了。隨後樂壽宮才哭聲四起,宣布太后已死。慈禧與光緒素有嫌隙,況且當時慈禧已處於彌留之際,此時派人給軟禁中的皇帝贈食,極不尋常。啟功認為,慈禧可能先於光緒帝病死,但祕不發喪,直到確認光緒帝死亡後才對外公布死訊。

央視主任編輯鍾里滿依檢驗結果及史料記載認為,慈禧自戊戌政變以後就陰謀廢黜及弑害光緒,更擔心光緒會在自己死後復位翻案,所以才會在病危之時下毒手。

曾留洋并担任慈禧的御前女官的裕德龄在其英文版自述《瀛台泣血记》中提出,应是慈禧指使李莲英下手。

稱光緒帝为袁世凯所弑者认为,袁负恩反戈,陷光绪帝于万劫不复,光绪帝在瀛台,“日书‘项城’(袁世凯别号“袁项城”)名以志其愤”。袁既知光绪帝对其深恶痛绝,则不能不惧太后死而帝独生,故加以谋害(见于光緒侄、末帝溥仪所著《我的前半生》及其他)。但鍾里滿认为,当时除了慈禧太后外,并无其他人具备指使人对皇帝下毒的能力。袁世凯亦難以接近光绪帝。

虽然清政府公布的死亡日期是清曆十月二十一日(西曆1908年11月14日),但喻大华认为清政府推迟了光绪帝的死亡日期,光绪帝的起居注史官恽毓鼎在他的《崇陵传信录》中回忆,在此前两天的十月十九日,太监成群结队地出宫剃头,并毫不避讳地说皇上驾崩,因为国丧期间,服喪不允許理髮,所以抢在发布之前剃头。由此看来,朝廷發布的光绪帝死亡时间不準,在溥仪入宫之前光绪就已经死了,慈禧从容布置之后,确信可以掌控全局的时候才向天下公布皇帝已駕崩。日本电报也称光绪皇帝死于11月12日夜。

多數主流史學者(梁啟超、楊天石等)認為,光緒帝是清朝歷代皇帝之中較能接受新式制度的開明君主。甲午戰爭時主戰,不欲割讓台澎,展現其想保疆衛土的決心。但無奈其一生從未掌握實權,可謂心有餘而力不足。論及維新變法之失敗,亦當歸於改革操之過切,如王照稱光緒帝為瑣事一日罷去六堂官,致懷塔布之妻入宮向太后哭訴。在重大改革中,一百天內制定如此多的諭令,就算慈禧太后沒有最終加以阻撓,以當時清廷內外交困的情實,也難以逆睹預料成敗。唯帝師翁同龢稱光緒幼承孝貞庭訓,近臣惲毓鼎稱某大員入蜀,慈禧唯絮問,而光緒帝不發一語,臨陛見,忽然曰:“西藏事,其要在蜀,勞費心。”及至,果然,而入民國,告同僚:“帝有先見之明”。惲毓鼎對宣統而作《崇陵傳信錄》。

光緒帝自親政到戊戌政變,並沒法建立自己的班底,除了師傅翁同龢是自己的親信外,其餘皆是過去慈禧培養的官員,再加上官員任命權與國家大政,皆需要請示慈禧才能定奪,因此光緒帝親政後並沒有太大權力,導致在戊戌變法時,光緒帝想積極培養自己的親信班底,反而遭致朝內的反對,再加上用人不當、操之過急,最後失敗。

重用翁同龢、文廷式等人是光緒帝親政最大的失誤。翁同龢雖是光緒帝師傅,但性格過於守舊,得罪不少洋務派,並對慈安、慈禧兩宮太后推動的洋務運動嗤之以鼻。由於光緒帝親政後並沒有自己的班底,只有信任翁同龢。另一方面,翁同龢為刁難李鴻章的北洋艦隊,不惜上奏光緒帝禁止海軍外購軍火(《請停購船械裁減勇營折》),致使海軍失去了申請專項資金用於艦和、武備更新的途徑,導致北洋艦隊無法更新武器,在甲午戰爭全面戰敗。在甲午戰爭的指揮中,光緒帝指揮不當,導致甲午戰爭節節失敗,後期又與慈禧發生指揮上的衝突,等到慈禧接手指揮權時,只剩下與大日本帝國求和一途。

戊戌變法聽信康有為的策略,導致原本與慈禧已有默契的光緒帝,開始與慈禧產生摩擦。而康有為在翁同龢被罷官後,取代其位置成為光緒帝最親近的大臣,因康有為誤事,反而讓光緒帝苦吞戊戌變法的敗果。

光緒帝親自接見由親近大臣張蔭桓推薦的大日本帝國總理大臣伊藤博文,並接受其提供的改革方針。

光緒帝為了擺脫慈禧的控制,急於自己有班底,在戊戌變法中,並沒有看清時局的演變,再加上康有為、梁啟超的激進作法,導致一向默許光緒帝主導戊戌變法的慈禧太后終於忍無可忍,發動政變並囚禁光緒帝。[來源請求]

光緒帝相較於前一位同治帝,處理政務過於急躁,且用人不當,只聽信翁同龢、康有為之言,進而導致甲午戰爭、戊戌變法的失敗。

敘述光緒事蹟的書籍,《清史稿·德宗本紀》及清宮檔案自是第一信史,此外有清末民初惲毓鼎的《崇陵傳信錄》和清室遠支德齡的《瀛台泣血記》等。關於戊戌變法的資料則有梁啟超的《戊戌政變記》和袁世凱的《戊戌日記》。近年來有《走向共和》《戊戌風雲》等影視,亦頗有影響。

惲毓鼎(1862-1917)《崇陵傳信錄》「緬維先帝御宇,不為不久。幼而提攜,長而禁制,終於損其天年。無母子之親,無夫婦、昆季之愛,無臣下侍從宴遊暇豫之樂。平世齊民之福,且有勝於一人之尊者。毓鼎侍左右,近且久,天顏戚戚,常若不愉,未嘗一日展容舒氣也。」

\subsection{光绪}

\begin{longtable}{|>{\centering\scriptsize}m{2em}|>{\centering\scriptsize}m{1.3em}|>{\centering}m{8.8em}|}
  % \caption{秦王政}\
  \toprule
  \SimHei \normalsize 年数 & \SimHei \scriptsize 公元 & \SimHei 大事件 \tabularnewline
  % \midrule
  \endfirsthead
  \toprule
  \SimHei \normalsize 年数 & \SimHei \scriptsize 公元 & \SimHei 大事件 \tabularnewline
  \midrule
  \endhead
  \midrule
  元年 & 1875 & \tabularnewline\hline
  二年 & 1876 & \tabularnewline\hline
  三年 & 1877 & \tabularnewline\hline
  四年 & 1878 & \tabularnewline\hline
  五年 & 1879 & \tabularnewline\hline
  六年 & 1880 & \tabularnewline\hline
  七年 & 1881 & \tabularnewline\hline
  八年 & 1882 & \tabularnewline\hline
  九年 & 1883 & \tabularnewline\hline
  十年 & 1884 & \tabularnewline\hline
  十一年 & 1885 & \tabularnewline\hline
  十二年 & 1886 & \tabularnewline\hline
  十三年 & 1887 & \tabularnewline\hline
  十四年 & 1888 & \tabularnewline\hline
  十五年 & 1889 & \tabularnewline\hline
  十六年 & 1890 & \tabularnewline\hline
  十七年 & 1891 & \tabularnewline\hline
  十八年 & 1892 & \tabularnewline\hline
  十九年 & 1893 & \tabularnewline\hline
  二十年 & 1894 & \tabularnewline\hline
  二一年 & 1895 & \tabularnewline\hline
  二二年 & 1896 & \tabularnewline\hline
  二三年 & 1897 & \tabularnewline\hline
  二四年 & 1898 & \tabularnewline\hline
  二五年 & 1899 & \tabularnewline\hline
  二六年 & 1900 & \tabularnewline\hline
  二七年 & 1901 & \tabularnewline\hline
  二八年 & 1902 & \tabularnewline\hline
  二九年 & 1903 & \tabularnewline\hline
  三十年 & 1904 & \tabularnewline\hline
  三一年 & 1905 & \tabularnewline\hline
  三二年 & 1906 & \tabularnewline\hline
  三三年 & 1907 & \tabularnewline\hline
  三四年 & 1908 & \tabularnewline
  \bottomrule
\end{longtable}


%%% Local Variables:
%%% mode: latex
%%% TeX-engine: xetex
%%% TeX-master: "../Main"
%%% End:
