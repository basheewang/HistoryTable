%% -*- coding: utf-8 -*-
%% Time-stamp: <Chen Wang: 2019-12-26 22:00:48>

\section{穆宗\tiny(1861-1875)}

\subsection{生平}

同治帝(1856年4月27日-1875年1月12日),名载淳,爱新觉罗氏,是清朝自入关以来的第十位皇帝,1861年11月11日至1875年1月12日在位,年号「同治」。西藏方面尊為「文殊皇帝」。

同治帝是咸丰帝长子,5歲(虛歲六歲)登基,原設年號為「祺祥」,隨後不及一年發生辛酉政變,最終由其嫡母慈安太后與生母慈禧太后共同垂簾聽政(史稱「兩宮聽政」),並改設年號為「同治」。1875年驾崩,時年僅18週歲。死後廟號穆宗,諡號毅皇帝,葬于清東陵中的惠陵。

咸豐六年三月二十三日,生於儲秀宮,生母叶赫那拉氏懿嫔,即后来的慈禧太后。八年,其异母弟弟憫郡王早夭,其后载淳一直是咸丰帝唯一存活的儿子。咸丰十一年,载淳开始上学,由编修李鸿藻授读。

咸丰十一年七月,咸丰帝弥留之际,立皇长子载淳为皇太子,任命肃顺等八人赞襄政务,稱顧命八大臣。咸丰帝死後,载淳成为继任皇帝。嫡母皇后钮祜禄氏和生母懿贵妃那拉氏并尊为皇太后。此时,顧命八大臣主政,他们为载淳选定年號——祺祥。不久,两宫皇太后與他的叔叔们——恭親王奕訢、醇郡王奕譞等人共同發動辛酉政变,扳倒八大臣。

咸丰十一年九月,上两宫太后徽号,称慈安、慈禧。冬十月庚申,下诏改年号祺祥为同治,以誌“同歸于治”、“君臣同治”、“同于顺治”(出自《春秋》,或譯“母子同治天下”)的垂簾聽政。甲子,载淳在北京紫禁城太和殿登基,颁诏天下,以第二年为同治元年,故称同治帝。十一月乙酉朔,嫡母慈安太后、生母慈禧太后在养心殿正式垂帘听政。登基时,同治帝年仅五岁,故其后一直由慈安太后、慈禧太后垂帘听政,史称兩宮聽政。

同治元年春正月,同治帝在慈宁宫率王公大臣向两宫太后行礼,自己则在乾清宫受贺,此后每年亦如此。二月乙卯,懿旨同治帝在弘德殿入学读书,祁寯藻、翁心存授读。

同治八年,清朝政府就已开始为同治帝大婚作准备。三月己亥,懿旨,大婚典礼,力崇节俭。此前,两位幼年继位的清帝——顺治帝、康熙帝均在十四岁(虚岁)完成大婚,并亲政。此时的同治帝已到达他们的年龄,但完成婚礼则是三年后。

同治十一年(1872年)初,清宫选秀,从秀女中为同治帝选定一后四妃。九月十五日(10月16日),举行大婚典礼,正式迎娶皇后阿鲁特氏。

同治十二年春正月乙巳,两宫太后以亲政届期,颁布懿旨,鼓励同治帝“祇承家法,讲求用人行政,毋荒典学”。勉廷臣及中外臣工“公忠尽职,宏济艰难”。丙午,同治帝亲政,下诏“恪遵慈训,敬天法祖,勤政爱民”。

親政时,同治帝年方十八岁。在位期間,歐洲列強未有入侵,而太平天國亦已經被消滅,清室亦興辦洋務,頗有發憤圖強之心。此段時期被稱為同治中興。

同治十三年十月己亥,因同治帝有病,命李鸿藻代阅奏章。十一月,命恭亲王奕訢处理批答清文摺件。己酉,命内外奏折呈两宫太后披览。十二月初五日(1875年1月12日),同治帝崩於皇宮養心殿,年仅18岁,为清朝寿命最短的皇帝。同治无后,慈禧即挑出咸丰之弟奕譞之子载湉入嗣大宗为帝,是为德宗(光绪帝)。光绪帝亦无子而逝,以溥仪继承帝位,兼祧两房。

據正史記載,同治帝是死於天花。相同紀錄亦出現於《翁同龢日記》,說同治帝得了天花,導致毒熱內陷,最終“走馬牙疳”而死。

但在民間傳说稱同治死于梅毒。或说同治帝婚后獨宿乾清宮,在內監和寵臣載澂引導下經常微服私行,常到崇文門外的酒肆、戲館及花巷尋花問柳。野史記載:“伶人小六如、春眉,娼小鳳輩,皆邀幸。”又有人推薦他一些黃色小說,“小說淫詞,祕戲圖冊,帝益沉迷”。據《清宮遺聞》記載,“同治到私娼處,致染梅毒”。而《清朝野史大观》卷一《清宫遗闻》中说:“孝哲后,崇绮之女,端庄贞静,美而有德,帝甚爱之,以格于慈禧之威,不能相款洽,慈禧又强其爱所不爱之妃(指凤秀之女淑慎皇贵妃),帝遂于家庭无乐趣矣,乃出而纵淫,……专觅内城之私卖淫者取乐焉。……久之毒发,始犹不觉,继而见于面,盎于背。”“太医知为淫毒,而不敢言,遂以治痘药治之,不效”。1923年蕭一山的《清代通史》再三強調了同治帝是死於梅毒。台灣作家高陽长篇巨著《慈禧全传》认定是梅毒。御醫李德立的曾孫李鎮和李志綏分別撰文稱,祖上口傳秘聞,同治帝死于梅毒。慈禧聽到李德立的診斷結果之後,強迫他宣布是天花。李鎮表示“同治梅毒溃烂后,流脓不止,奇臭难闻,曾祖父(李德立)每日必须亲自为他清洗敷药,一个多月来受到强烈恶臭刺激,从此失去了嗅觉”。 目前則以天花梅毒兩種說法最為大宗。

同治帝親政時間短暫,期間最大的事件是牡丹社事件,當時日本明治政府不滿琉球漁民遭生番殺害,因此藉口出兵幫琉球漁民報仇,引起清廷的注意,由於此事件發生在同治帝親政後,同治帝特別關注此一事件,並不定期向慈安、慈禧兩宮太后匯報牡丹社事件處理進度,一方面同治帝調度得宜,並派船政大臣沈葆楨為欽差大臣,以巡閱為名來台,主持台灣海防及對各國的外交事務。另外派唐定奎率領的淮軍十三營六千五百人赴台,安定台灣,使局勢發生變化,日本明治政府不得不請英國公使威妥瑪調停,並簽訂北京專約,牡丹社事件告一段落。

北京專約導致琉球國被日本明治政府納為日本領土,同治帝對於北京專約中的第一條:「日本國此次所辦,原為保民義舉起見,中國不指以為不是。」默認,間接導致琉球國被日本併吞。

牡丹社事件後,同治帝對於台灣的治理轉為積極,增設府縣,並對台灣東部及生番地區以「開山撫番」進行開發、平定及台灣東西部越嶺古道,防止外國勢力以生番問題進犯台灣。

同治帝相較於其繼任者光緒帝,雖然身邊沒有自己的親信班底,不過同治帝用人唯才,不計較官員是兩宮太后任用的官員(在處理牡丹社事件時),與光緒帝急於培養身邊的班底,而導致甲午戰爭、戊戌變法的失敗,形成強烈對比。

同治帝也是清朝自乾隆帝以來,首位接見外國使臣的大清皇帝。

同治帝親政期間,發生一件震驚中外的楊乃武與小白菜案,上海媒體《申報》還對此做詳細的報導。

同治帝親政後,為了展現對慈安、慈禧兩宮太后的孝心,開始進行圓明園的修復工程,希望讓兩宮太后能住進圓明園安享天年,但因為所修造的費用太高,遭到群臣的反對,最後在慈禧的干預下,圓明園的修復工程也就停止了。

同治帝親政後對於朝政興趣缺缺,國政大事並不有效處理,圓明園修復工程又遲遲被刁難,甚至因為不滿恭親王奕訢等王公大臣對他的節制,一度想免去這些王公大臣的職務,震動朝野,後來在慈禧的調停下,王公大臣才免於被罷官。[來源請求]

大事年表:清咸豐六年三月二十三日,載淳在北京紫禁城儲秀宮出生。清咸豐十一年七月,咸豐帝去世,年僅六歲的載淳登基,依照咸豐帝遺詔,由肅順等八位大臣輔政。九月兩宮太后與恭親王奕訢發動「辛酉政變」,八大臣等被奕訢與慈禧奪權。清同治三年六月,清軍攻陷太平天國首都天京。清同治四年四月,科爾沁親王僧格林沁為捻軍所殺。清同治六年十二月,東捻軍被平定。清同治七年七月,西捻軍主力被平定。清同治九年七月,兩江總督馬新貽被刺殺。清同治十一年九月,迎娶皇后阿魯特氏(孝哲毅皇后)。清同治十二年正月,親政,同年同治陝甘回亂及雲南回亂大致平定。清同治十三年六月,臺灣牡丹社事件爆發,與日本在北京訂立北京專約。清同治十三年九月,發生震驚中外的楊乃武與小白菜案。清同治十三年十二月,同治帝崩,得年19歲。

\subsection{同治}

\begin{longtable}{|>{\centering\scriptsize}m{2em}|>{\centering\scriptsize}m{1.3em}|>{\centering}m{8.8em}|}
  % \caption{秦王政}\
  \toprule
  \SimHei \normalsize 年数 & \SimHei \scriptsize 公元 & \SimHei 大事件 \tabularnewline
  % \midrule
  \endfirsthead
  \toprule
  \SimHei \normalsize 年数 & \SimHei \scriptsize 公元 & \SimHei 大事件 \tabularnewline
  \midrule
  \endhead
  \midrule
  元年 & 1862 & \tabularnewline\hline
  二年 & 1863 & \tabularnewline\hline
  三年 & 1864 & \tabularnewline\hline
  四年 & 1865 & \tabularnewline\hline
  五年 & 1866 & \tabularnewline\hline
  六年 & 1867 & \tabularnewline\hline
  七年 & 1868 & \tabularnewline\hline
  八年 & 1869 & \tabularnewline\hline
  九年 & 1870 & \tabularnewline\hline
  十年 & 1871 & \tabularnewline\hline
  十一年 & 1872 & \tabularnewline\hline
  十二年 & 1873 & \tabularnewline\hline
  十三年 & 1874 & \tabularnewline
  \bottomrule
\end{longtable}


%%% Local Variables:
%%% mode: latex
%%% TeX-engine: xetex
%%% TeX-master: "../Main"
%%% End:
