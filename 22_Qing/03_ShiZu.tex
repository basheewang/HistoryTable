%% -*- coding: utf-8 -*-
%% Time-stamp: <Chen Wang: 2019-10-22 10:24:47>

\section{世祖\tiny(1643-1661)}

順治帝(1638年3月15日-1661年2月5日),名福临,姓爱新觉罗氏,清朝第2位皇帝,清朝自入关以来的首位皇帝,1643年10月8日至1661年2月5日在位,在位18年。议政王大臣会议于1643年9月,推举五岁的福临承袭其父皇太极帝位,同时命努尔哈赤第十四子睿亲王多尔衮和努尔哈赤之侄郑亲王济尔哈朗二人助小皇帝辅理国政。

自1643年至1650年,政治权力主要掌握在多尔衮手里。在多尔衮的领导下,清朝征服明朝的大部分故土,深入西南省份追剿南明政权,在激烈的反对中,建立一系列被清代皇帝所沿袭的政策,如1645年颁布“剃发令”。多尔衮于1650年12月31日死后,13歲的顺治皇帝开始亲政。顺治皇帝试图打击腐败,整顿吏治,削弱满洲贵族的政治影响力,但最终结果成败参半。在位期間,顺治帝面临着大明遗民的复明抵抗,不过至1661年,清军已将大清帝国最后的对手,南明遺臣郑成功和永历皇帝朱由榔击败,郑成功和朱由榔分别于次年病死和被擒杀。顺治皇帝在22岁时因感染高度流行的天花去世,其皇位由已从天花中幸免于难的皇三子玄烨承袭,后者即康熙帝,在位24年。由于顺治年间的历史文献流传相对较少,加上史書為突顯康熙帝的功績,因此这段时期同整个清朝历史相比显得较为鲜为人知。

顺治帝死后受供奉于太庙,庙号「世祖」,谥号「体天隆运定统建极英睿钦文显武大德弘功至仁纯孝章皇帝」,统称世祖章皇帝,葬于清东陵的孝陵。


14世纪,数支女真部落生活在大明(1368年–1644年)东北疆域,即现代被称为中国东北或“满洲”的地区明太祖時,為壓抑北元殘餘勢力,於是東北設立遠東指揮使司,控制女真部的各個部落。

明成祖永樂年間(1403年-1424年),在东北疆域置奴兒干都司等衛所管理當地,其中建州女真一部最為強大,明政府先後將建州女真分成三個衛,總稱「建州三衛」,其後,建州女真首领努尔哈赤(1559年–1626年)经过三十余年的征抚,完成對女真各部的統一。

努尔哈赤最重要的一项改革,是将松散的女真诸部的力量凝聚在黄、白、红、蓝四色旗之下,此后,又在原有四色旗基础上再增镶黄、镶白、镶红、镶蓝四旗,形成八旗。此社会军事组织制度是为八旗制度。努尔哈赤将旗主交由子侄担任。在1612年左右,努尔哈赤為使其部族人与其他支觉罗部族人相区別,及與曾统治中国北方的女真王朝大金(1115年–1234年)扯上關係,故將部族名變更為爱新觉罗氏(意為“黄金般高贵神圣的觉罗一族”)。

1616年,努尔哈赤宣布叛明自立國號,史稱后金,建元天命。爾後,努爾哈赤繼而攻打原大明領土的辽东大多数主要地區,其軍隊所向披靡,直到1626年1月,努尔哈赤在宁远攻城之时,被驻守该地的明军指挥官袁崇焕,以不久前收购的葡萄牙人的红夷大炮击败。努尔哈赤可能在宁远之战中受了致命伤,因而在战后数月逝世。

努尔哈赤之子皇太极(1592年–1643年)继续致力于其父的大业:他把权力集于自己之手,仿效大明政治制度,并完善和拓展八旗制度,在原有满洲八旗的基础上增设蒙古八旗和汉军八旗。1629年,他率军入侵北京郊区,在此期间俘获了知道如何铸造红夷大炮的汉人工匠。1635年,皇太极改称女真为“滿洲”,1636年,他又将国号“后金”改为“大清”。在松锦之战后的1643年,明朝已经在财政破产、瘟疫肆虐以及大饥荒导致的明末农民战争等致命危机之中摇摇欲坠,大清准备展开对大明的最后一击。

清世祖福临出生于1638年3月15日。

崇德七年 (1642年) 十二月初二日,皇太極率諸王貝勒及文武大臣行獵於葉赫地方。同月十二日,到達噶哈嶺。聖汗之五歲幼子方喀拉章京射殺一狍。學者楊珍在《順治朝滿文檔案札記》認為方喀拉即為福臨的原名或乳名,章京即為方喀拉在此次隨皇太極行獵時,臨時得到的職位。

1643年9月21日,生前未指定儲君的皇太极殯天,雏鹰般的大清面临着可能出现的严重分裂危机。数名皇位争夺者——努尔哈赤的次子兼在世的长子和硕礼亲王代善、努尔哈赤第十四子和硕睿亲王多尔衮和第十五子和硕豫亲王多铎(两人为同母所出)以及皇太极之长子和硕肃亲王豪格——开始逐鹿皇位。皇太极的弟弟多铎、多罗武英郡王阿济格及多尔衮(31岁)掌有正白及镶白旗,代善(60岁)掌有两红旗,而豪格(34岁)则获得其父两黄旗的支持。

议政王大臣会议着手议立新帝,此會議直到军机处在18世纪20年代出现以前一直是满清的主要决策机构。许多亲王、贝勒主张多尔衮这个久经考验的军事将领成为新皇帝,但多尔衮拒绝为帝,而是坚持让皇太极的一个儿子承袭父位。

会议接受多尔衮的具有权势的主张,继续让皇太极的后裔继承大统。此时除豪格外,皇太极的儿子中,尚有叶布舒、硕塞、高塞、常舒、韬塞、福临、博穆博果尔七人。其中叶布舒、高塞、常舒、韬塞四人中,有三人年长于福临,但皆生母地位低微,无法越过福临、博穆博果尔继承皇位。而硕塞的生母叶赫那拉氏则早被皇太极赐给大臣,博穆博果尔则年幼于福临。最终商议决定立皇太极第九子福临承袭父位为新皇帝,但亦决定立和硕郑亲王济尔哈朗(努尔哈赤之侄,他掌有镶蓝旗)和多尔衮作这个五岁孩子的摄政。1643年10月8日,福临正式登上大清皇帝位;定年号为“顺治”。由于记载顺治年间的文献语焉不详,所以这段时期同整个清朝历史相比显得较为鲜为人知。

济尔哈朗是一位骁勇善战、受人尊敬的将领,但看起来对多尔衮已很快就抓到手中的日常行政事务毫无兴趣。1644年2月17日,济尔哈朗召集内三院、六部、都察院和理藩院的官员,向他们宣布:“嗣后,凡各衙办理事务或有应白于我二王者,或有记档者,皆先启知睿亲王档子,书名亦宜先书睿亲王名,其坐立班次及行礼仪,注俱照前例行。”此后在同年5月6日,豪格暗中动摇摄政统治的阴谋暴露。豪格的党羽全部被处死,豪格本人被褫夺亲王爵位。多尔衮在此后不久,以自己的支持者接替取代了豪格的拥护者(大多来自黄旗),从而掌控了两白旗以外的旗。至1644年6月初期,他已牢牢地把清政府及其军政大权掌握在自己手中。

1644年初期,正当多尔衮与其谋士苦思钻研如何攻大明之时,民變逼近北京。同年4月24日,民變领袖李自成攻破明都城墙,促使崇祯皇帝朱由检在紫禁城后的万岁山歪脖树上自缢身亡。多尔衮的汉人谋士洪承畴和范文程闻讯,敦促滿洲亲王抓住此机遇,给大明报仇雪恨,进而为大清夺取天命。驻扎在长城东端山海关的大明总兵吴三桂,是多尔衮同北京之间的最后一道障碍。此时他正被满洲人与李自成军间的武力夹得左右为难,吴三桂请求多尔衮帮助他驱逐土匪,恢复大明。当多尔衮要求吴三桂替大清效力之时,吴三桂除了接受之外别无选择。清兵因此得到了吴三桂的精兵的辅助,后同李自成军进行一片石之战,在多尔衮最终选择用骑兵介入此战斗前,吴三桂的精兵就已和李自成军交战了数小时。5月27日,大清取得此战的决定性胜利。战败的李自成军在北京洗劫数日,直至6月4日携带着所能带走的财物离京。

6月5日,被叛军之手肆虐了六周的北京市民,派出了一批士绅及官吏迎接他们将要来到的解放者。可当他们见到的是骑着马、把前额头发剃光并自称摄政王的满洲人多尔衮,而不是大明皇太子朱慈烺及其保护者平西伯吴三桂时,吃了一大惊。在此场动乱之中,多尔衮将自己安置在武英殿,后者是李自成在6月3日火烧大内后,唯一未被损坏的建筑。旗军们被命令不许抢劫;他们的纪律约束使统治过渡到大清“出奇地顺利”。然而在同时,多尔衮却声称他是为报复大明而来。他下令将大明皇族(包括大明末代皇帝朱由检的后裔)及其拥护者全部处决。

6月7日,进城仅两天的多尔衮向首都的官员发布谕告。该谕告向官员们保证,如果本地居民剃发易服并且接受归降,那么他们则可以官复旧职。可是在此谕告发布后的三周内,北京爆发数场农民起义,威胁大清控制首都地区。面对威胁,多尔衮不得不将此谕告废除。

1644年10月19日,多尔衮在北京大门迎接福临。10月30日,六岁的福临被带到北京南郊的天坛祭拜天地。11月8日,福临的登基仪式正式举行。同日,年幼的皇帝将多尔衮的功绩同周公进行比较,后者为古时一个受人尊敬的摄政。在登基仪式上,多尔衮的官衔由“摄政王”升为“叔父摄政王”。满语“叔父”(ecike)在此表示高于亲王的一级身份。三天后,多尔衮的摄政同事济尔哈朗的官衔由“摄政王”降为“辅政叔王”。多尔衮在1645年6月发布仪注规定,今后所有公文均应书写“皇叔父摄政王”称呼他,这使得多尔衮距离皇帝权威仅剩一步之遥。最终多尔衮在1648年更凌驾于小皇帝之上,称“皇父摄政王”。

多尔衮进入大清新首都后的最初的一个命令是,将北京北部全部腾出,然后把它分给旗人。两黄旗分得荣耀的宫殿北部,其次,东部为两白旗,西部为两红旗,南部为两蓝旗。八旗的此种布局,是为了使京城同满洲在征服中原前的故乡保持一致。此种布局“按照罗盘的指针指向,给颜色不同的旗人分配在一个固定的地理位置。”尽管大清为了加快过渡而减免税收,推迟大型建筑建造计划。但到了1648年,新来的旗人与共同生活的汉人百姓间仍有敌意。而首都以外的农业用地则全部被清军圈占。昔日的地主,现在却成了给外居旗人地主支付租金的佃户。这种土地用途的转变导致了“数十年的中断和苦难。”

在1646年,多尔衮还下令重建选任政府官员的科举考试。从那时起,他们效仿大明,每三年定期举行一次科举。同年,大清统治下的第一次殿试举行,大多数报考者为北方汉人,他们被提问如何使满汉同心合志。1649年,考试询问“联满汉为一体,使之同心合力,欢然无间,何道而可?”在1660年确定减少中额前,顺治朝下每届会试的考中人数的平均为大清最高(“得到了汉人更多的支持”)。

晚清的一幅描绘1645年5月扬州十日的版画。多尔衮的弟弟多铎为镇慑南方不服的汉人而进行了这场大屠杀,画上总共有有九位死者。十九世纪晚期,这场大屠杀被反清复明革命者用以激发汉族人群的反满情绪。
晚清的一幅描绘1645年5月扬州十日的版画。多尔衮的弟弟多铎为镇慑南方不服的汉人而进行了这场大屠杀,画上总共有有九位死者。十九世纪晚期,这场大屠杀被反清复明革命者用以激发汉族人群的反满情绪。
多尔衮被历史学家不同地称为“大清征服的优秀策划者”和“满洲洪业的首席建筑师”,大清在他的统治下,征服了中原大部分地区,并将“南明”的势力范围推到了遥远的中国西南地区。李自成从北京逃到西安,并在后者重建指挥部。多尔衮在同年夏、秋将河北、山东抗清起义镇压后,派遣军队进入西安(陕西省)主要城市搜寻李自成。1645年2月,在清军的压力下,李自成被迫离开了西安。他被杀了——无论是死于自己之手,还是被当地村民疑以为劫盗而误杀——1645年9月后,他在几个省份中消失了。

1644年6月,福王朱由崧于长江中下游以南的江南富饶的商农区建立大明弘光政权。1645年4月初,大清从新占领的西安出发,准备向那里发起进攻,南明政权的党派之争和不计其数的逃叛,阻碍了其有效抵抗能力的增强。1645年5月初,数支清军席卷南方,随手夺取了徐州淮河以北的主要城市。此后不久,他们向南明北部防线的主要城市——扬州——拥去。史可法面对包围,勇敢地反抗。5月20日,遭受一周炮轰的扬州被满洲人攻破,史可法依旧拒绝投降。多尔衮的弟弟多铎遂下令屠杀扬州全城人民。作为目的,这场大屠杀作为恐吓江南其他城市降服于大清。紧接着南京在6月16日,即最后的防卫者使多铎保证不会伤人后,钱谦益开城而降。大清在不久俘获了大明皇帝(他在翌年被处决于北京),并迅速夺取了江南包括苏州杭州的主要城市;至1645年7月初,大清与南明之间的边界被推到南方的钱塘江。

江南刚有了表面上的平静后,多尔衮便在1645年7月21日发布了一个最不合时宜的告示,他命令所有的成年男人剃去他们前额的头发,将他们的头发按照满洲人的髡髮辫式编扎起来。不服从告示者将被处以死刑。对于满洲人来讲,此象征着屈服的政策,有助于他们分清敌我。不过,在汉人官员和文人看来,新发型是一种奇耻大辱(因为它有悖于孔门弟子关于保持身体完整的指导)。而对于普通百姓来说,剃发如同丧失他们的生殖能力(英语:virility)。由于剃发令逼使社会的各个阶层的汉人联合起来反抗大清统治,所以极大地阻碍了大清的征服。在1645年8月24日和9月22日,前明将领李成栋分别对嘉定和松江反抗的人民进行屠杀。而江阴还同约一万名清军进行了八十三天的对抗。当城门最终在1645年10月9日被攻破时,降清明将刘良佐对全城人进行屠杀,这场屠杀造成了七万四千至十万不等的人的死亡。这些大屠杀结束了长江中下游的反清武装抵抗。有几个忠诚的勤王者成了隐士,并希望着清军败溃。虽然他们退出了世界,但至少象征着在继续反抗外族统治。

南京沦陷后,两支明宗室建立了两个新的南明政权:一个是以福建沿岸附近为中心隆武皇帝唐王朱聿键——明太祖朱元璋的九世孙——而另一个是浙江附近的“监国”鲁王朱以海。但由于雙方彼此不服,無法聯合抗清,不但無法反攻滿清,也導致喪失維持政權的機會,造成漢人政權走向衰亡。1646年7月,贝勒博洛领导的新的南方军事活动使鲁王的浙江朝廷陷入混乱状态,继而向隆武政权发起进攻。朱聿键于10月6日在汀州(福建西部)被俘,即刻处死。他的养子国姓爷郑成功则随他的船队逃往台湾岛。11月,江西剩余的忠明抵抗中心崩溃,整个江西降清。

1646年末,广州出现了两个新的大明皇帝:一个是年号为绍武的朱聿键之弟唐王朱聿𨮁,另一个为年号为永历的桂王朱由榔。由于朝服不够,此后绍武政权所任命的官员不得不向本地伶人购买戏袍。两支南明政权彼此残杀,直到1647年1月20日,李成栋率领的一支小规模清兵组成的先头部队开进广州,处死了朱聿𨮁,迫使永历朝廷逃往广西南宁。然而,李成栋于1648年5月起兵抗清,与江西的前明将领金声桓并发起义,帮助朱由榔夺回了中国南方的绝大部分地区。但南明的复兴只是昙花一现。清军于1649年和1650年重新征服湖广中部(今河北和湖南)、江西和广东。朱由榔再度逃亡。最后,1650年11月24日,尚可喜所统率的清军攻占广州,杀死七万多人。

同时,1646年10月,豪格(福临长兄,于1643年继承斗争中失去继承权)所统率的清军抵达四川,任务是摧毁张献忠领导的大西国。1647年2月2日,张献忠与清军在川中西充附近作战时被杀。1646年末抗清势力进一步向北蔓延,由一个穆斯林将领米喇印领导的武装力量反抗大清对甘州(甘肃)的统治。另一名穆斯林丁国栋很快加入了他的抗清运动。他们以恢复大明为号召,攻克了甘肃的数个城镇,其中包括省会兰州在内。这些起义者愿意同非穆斯林的汉人进行合作,这表明他们不是仅仅被宗教所驱使。1648年,米喇印战死于水泉(今甘肃永昌水泉子村),丁国栋则被孟乔芳俘获并被多尔衮下令处决,至1650年,造成了大量人员伤亡的穆斯林起义运动被粉碎。

1650年12月31日,多尔衮在狩猎途中意外死亡,引发了一段激烈的派系斗争,开辟了深层次政治改革之路。由于多尔衮的支持者在朝廷上仍具影响,所以多尔衮的丧礼依帝礼,多尔衮死后获追尊为皇帝,谥号懋德修道广业定功安民立政诚敬义皇帝,庙号成宗。然而,在1651年1月中旬的同一天,多尔衮的前部将吴拜统率下的数名白旗军官为防范多尔衮的胞兄阿济格自立为新摄政而将其逮捕;之后,吴拜让福临任命自己及他的几位追随者为各部尚书,准备接管政府。

同时,于1647年被褫夺摄政头衔的济尔哈朗,获得了对多尔衮统治心怀不满的旗官的支持。济尔哈朗为了巩固直属皇帝的两黄旗(前两旗自清太宗开始直属皇帝)对自己的支持,争取白旗支持者,赋予正黄、镶黄、正白三旗一个新名称:上三旗(此三旗自此由皇帝直接统辖)。于1661年成为玄烨的辅政大臣的鳌拜和苏克萨哈,是给予济尔哈朗支持的旗官,济尔哈朗以指定他们参加议政王大臣会议作为回报。

1651年2月1日,济尔哈朗宣布即将13岁的福临亲政。摄政正式废止。济尔哈朗此后展开攻势。1651年3月12日,他控告多尔衮僭越皇权:多尔衮被判有罪,他获得的追尊被剥夺。济尔哈朗继续肃清多尔衮集团前成员,为上三旗中越来越多的支持者升官晋爵,所以到了1652年,多尔衮的前支持者或是被杀,或是被有效的从政府中清除。

諭吏部:“邇來有司貪污成習,皆因總督、巡撫不能倡率,日甚一日。國家紀綱,首重廉吏。若任意妄為,不思愛養百姓,致令失所,殊違朕心。總督、巡撫任大責重,全在舉劾得當,使有司知所勸懲。今所舉者多屬冒濫,所劾者以微員塞責,大貪大惡,每多徇縱,何禆民生?何補吏治?爾部須秉公詳察奏聞,如有此等惡習,定當從重治罪不貸。部院堂官係各司楷模,尤當正身潔操砥礪自愛,殫心盡職,以不負朕惓惓用人求治之意。其京堂大小員缺,亦著選擇有才望堪用者,不得循資挨轉。以後內外官,各宜洗心滌慮,勤守職業,不得仍蹈前弊,自取罪戾!” —— 《大清世祖章皇帝實錄》卷五十四

福临仅仅亲政两个月后,便于1651年4月7日发布谕告,宣布他将肃清官场腐败。该谕告引起文人间的派系之争,令福临沮丧无比,至死也无可奈何。福临的最初的一项行动是罢免大学士冯铨。冯铨为北方汉人,先前曾于1645年受弹劾,但摄政王多尔衮仍准其任职如故。福临以陈名夏取代冯铨。陈名夏是个有影响力的南方汉人,同南方文人集团关系良好。陈名夏尽管曾于1651年受控以权谋私,但旋于1653年官复原职,旋即成为皇上的亲密的私人顾问。陈名夏甚至获准可以像昔日的明代内阁大学士那样起草诏书。同于1653年,福临决定召回声名狼藉的冯铨。皇帝如此行事,本意是想让南北汉人官员在朝廷上势均力敌,从而平息派系冲突。然而,冯铨回归后,派系之争反而激化,令皇帝始料未及。在1653年和1654年的数次朝议中,南方人形成反对北方人与满洲人的阵营。1654年4月,陈名夏向北方汉人官员宁完我建议,清廷应恢复明代衣冠,宁完我旋即向皇帝揭发此事,并指控陈名夏干犯有包括贪污受贿、裙带关系、结党营私和僭越皇权在内的各种罪行。1654年4月27日,陈名夏被绞死。

1657年11月,北京顺天省试的一场重大作弊丑闻爆出。八名江南考生贿赂了京城的主考官,希望能得到更高的名次。七名主考官以受贿的罪名被处以死刑,数百人被判处贬谪流放和没收财产。这场丑闻很快蔓延到了南京会试,揭露了官僚制的腐败和以权谋私,许多坚持正统观念的北人官员将之归因为南方文人小团体的存在和经典学问的衰落。

福临在他短暂的统治期间,鼓励汉人入仕,恢复了许多多尔衮摄政期间废止或排斥的中原王朝制度。他和大学士(诸如陈名夏,见上文)谈论历史、经典和政治,他周围聚集了一批新人,诸如能讲一口流利满语的北方年轻汉人王熙。福临于1652年颁布的《六谕》是玄烨1670年颁布的《圣谕》的前身,后者是一部“正统儒家思想的梗概”,用于指示百姓遵守孝道和法律。顺治帝用中原王朝的一些体制改革清朝制度,于1658年恢复了翰林院和内阁。这两个机构承袭明代模式,进一步削弱满洲贵族的权力,这使得深深困扰晚明的党争问题死灰复燃成为可能。

为了削弱内务府和满洲贵族的权力,1653年7月,福临设立十三衙门,后者虽由满洲人监督,但由汉族宦官而非满洲包衣阿哈掌控。宦官在多尔衮摄政期间受严格的限制,但小皇帝用他们来制衡像皇太后和皇叔济尔哈朗这样的实权派人物的影响。至1650年代后期,宦官的权力变大:他们处理关键的政治和经济问题,就官员任命提出建议,甚至负责起草诏令。由于宦官削弱了官僚集团与皇帝间的联系,满汉官员担心困扰晚明的宦官擅权局面会重现。尽管皇帝尝试限制宦官权力,他最宠爱的宦官吴良辅还是于1658年陷入腐败丑闻,吴良辅于1650年代早期帮助他肃清多尔衮集团。但吴良辅收受贿赂仅仅受到谴责,未能平息宦官权力膨胀引发的满洲贵族的怒火。。福临死后不久,1661年3月,鳌拜和另外三位辅政大臣将十三衙门裁撤,吴良辅被处决。

1646年,博洛率清军进入福州,发现来自琉球国和安南的使节和马尼拉的西班牙人。这些朝贡使团前来拜见已倒台的南明隆武皇帝朱聿键,而后者此时已被押送至京,最终,这些使者听从清廷命令辞归。最后残存的南明抵抗势力从与安南接壤的云南撤离后,琉球王尚质于1649年首次向大清派出朝贡使团,暹罗和安南分别于1652年和1661年向大清派遣朝贡使团。

同于1646年,统治吐鲁番的一名莫卧儿王公苏丹阿布·穆罕默德·海基汗派遣一支使团,请求恢复因明亡而中断的与华贸易。使节团虽未受邀请便来到中国,但大清准其请求,允许其在北京和兰州进行朝贡贸易。但该协议因1646年一场席卷中国西北的穆斯林起义(参见前文“征服中国”末段)而中断。大清与资助反政府武装的哈密和吐鲁番的朝贡贸易最终于1656年恢复。不过在1655年,清廷宣布来自吐鲁番的朝贡使节每五年才能接受一次回赐。

1651年,小皇帝邀请藏传佛教格鲁派领袖第五世达赖喇嘛访问北京,后者不久以前在蒙古和硕特部首领固始汗的军事帮助下,成为西藏的宗教统治者和世俗统治者。尽管满洲对藏传佛教的支持和保护至少始于弩尔哈齐治下的1621年,但此次邀请背后仍有政治原因。即西藏正在成为大清西部一个强大的政治实体,达赖喇嘛对蒙古部落具有影响力,而其中一些蒙古部落并未屈从于大清。为了迎接这位“活佛”的到来,福临下令在紫禁城西北边北海琼华岛的昆仑山上建造了一座白塔,其位置就在以前薛禅汗宫殿的遗址上。经过多次邀请和外交往来,西藏领袖拿定主意,接受会见大清皇帝,1653年1月14日,达赖喇嘛抵达北京。达赖喇嘛日后将此行访问的场面雕刻在拉萨的布达拉宫,后者于1645年开始建造。

与此同时,在满洲人故乡北部,探险家瓦西里·波亚尔科夫(1643–1646)和叶罗菲·哈巴罗夫(1649–1653)越过罗刹国的山谷来到了黑龙江流域。1653年,莫斯科召回哈巴罗夫,委派奥努夫里·斯捷潘诺夫接替他,斯捷潘诺夫掌握了哈巴罗夫的哥薩克军队指挥权。斯捷潘诺夫南下进入松花江,强迫当地原住居民诸如达斡尔人和久切尔人交纳“牙薩克”(毛皮税),但遭到抗拒。因为满洲当地民族已向顺治皇帝朝贡。1654年,斯捷潘诺夫击败从宁古塔被派遣去调查罗刹计划的小规模的满洲军队。1655年,另一名清军指挥官蒙古人明安达礼在黑龙江流域的呼玛要塞击败斯捷潘诺夫军,但这还不足以追捕罗刹人。不过在1658年,满洲将领沙尔虎达率四十余艘船向斯捷潘诺夫发起进攻,罗刹人大多数被击毙或生俘。经过此役,黑龙江流域哥萨克地带已无太大冲突,但大清和罗刹的边境冲突则持续了下去,直至1689年《尼布楚条约》签订,固定了罗刹和大清之间的边界。

尽管大清在多尔衮的领导下成功将南明推到华南,但大明遗民尚未死心。1652年8月初,正在保护朱由榔的张献忠前部下李定国,从大清手中夺回桂林。一月之内,广西清将大多向南明投降。此后两年,尽管对湖广和广东的军事行动偶尔成功,但李定国未能夺取重要城市。1653年,清廷命洪承畴负责夺回西南地区。洪承畴驻扎长沙,耐心地建立起自己的军力;惟在1658年底,营养充足、物资供应良好的清军分多路向桂州和云南进军。1659年1月末,铎尼率清军攻陷云南府,朱由榔逃入邻近的缅甸,后者此时正由東吁王朝国王莽平德勒统治。此后南明末代皇帝一直留在缅甸,直到1662年被1644年4月降满的前明将领吴三桂俘获并处决。

郑成功在1646年成為明绍宗朱聿键義子,賜姓朱,故稱國姓爺,1655年由明昭宗朱由榔封为延平王,亦是他继续捍卫南明的原因。1659年,正当福临准备举行一场特殊的考试来庆祝他辉煌的统治和西南战役的胜利时,郑成功率领全副武装的船队驶向长江,从大清手中夺取了几座城市,进而围攻江寧(今江蘇南京)。当郑成功围攻江寧的消息传入皇帝耳中时,他就大发雷霆,据说一怒之下用剑劈了宝座。但南京的威胁最终解除,郑成功被清兵击退,被迫求助于东南沿海的福建省。迫于清军的压力,郑成功于1661年4月攻擊由荷蘭東印度公司佔領的台湾島,并死于同年夏天。他的子孙依然自稱為延平王,继续在台灣反抗大清统治,直至1683年顺治帝之子康熙帝派遣降將施琅佔領该岛。

順治帝于1651年亲政后,他的母亲昭圣慈寿皇太后安排儿子娶她的侄女额尔德尼布木巴,但福临废黜第一任皇后。次年,昭圣慈寿皇太后另为儿子安排了一场同蒙古科尔沁部的婚姻,这次她将自己的侄孙女阿拉坦琪琪格嫁给福临。尽管福临同样不喜欢他的第二任皇后(后世习以谥号称之为孝惠章皇后),但未能废黜皇后,皇后也未有生育。约在1656年,福临开始宠幸董鄂妃,据說当时的耶稣会记述,董鄂妃是一位满洲军人的妻子。她于1657年生下一子(皇四子)。皇帝想立他为继承人,但这个孩子未及命名便于1658年初夭折。

順治帝是位开明的皇帝,不仅在天文学和科技问题上,而且在处理国事和宗教问题时都向一位来自神圣罗马帝国科隆的日耳曼耶稣会教士汤若望请教。1644年末,多尔衮为制定一部尽可能精确的新历法而任用汤若望,因为他的日蚀预报比那些清廷天文学家的预报更精确。多尔衮死后,汤若望同小皇帝建立了私人友谊,福临用满语称他为“爷爷”。在他们关系最亲密的1656年和1657年,福临常常驾临他的府中,和他交谈到深夜。他被免除叩头礼,在北京获得建造教堂的土地,甚至被允许收养一个儿子(因为福临担心汤若望没有继承人),但自1657年以后,福临开始崇信佛教禅宗,汤若望试图使清帝信仰天主教的努力最终未能成功。

順治帝亲政后,发愤学习,熟练地掌握了汉语,能够欣赏中国艺术如书法和戏曲。反清知识分子顾炎武和万寿祺的一位密友归庄所作《万古愁曲》是福临最喜欢的文章之一。福临“极富感情,重情钟情,至其极处”,他还能成段的引用背诵援引《西厢记》。

大清皇帝自順治帝开始以「中國」自居,並且在對外条约和外交文件中称清为“中国”。1689年,也就是康熙二十八年,中俄尼布楚条约上第一次在国际法的层面上确立了“中国”的概念。

順治帝最宠爱的妃子皇贵妃董鄂氏因丧子之痛,于1660年9月猝死。福临为此悲痛欲绝,沮丧数月,直至他于1661年2月2日染上天花。1661年2月4日,福临急召礼部侍郎兼翰林院掌院学士王熙(福临的知己)和原内阁学士麻勒吉到自己身边,口述遗诏。同日,7岁的皇三子玄烨可能因为从天花中幸存下来而获立为皇太子。皇帝于1661年2月5日崩于紫禁城内的养心殿,年僅二十二岁。

满族人对天花病毒没有免疫,一旦感染天花,几乎只能等死,所以他们对天花的恐惧甚于其他任何疾病。1622年,他们建立一个机构,用于研究天花病例,隔离患者避免传染。在天花流行之时,皇室成员为保护自己免受感染,定期进入避痘所。福临之所以感染如此可怕的疾病,是因为他年轻,而且居住于附近有传染源的大城市。而事实上,根据记载,在顺治年间,至少有九次天花在北京爆发,每次爆发,都迫使福临搬到保护区。保护区为北京南部的狩猎场南苑,此前多尔衮已于17世纪40年代在那里建立一所避痘所。尽管有这样的预防措施——例如规定迫使感染天花的汉族居民搬出城市——但順治最終仍死於天花。

“奉天承运皇帝诏曰:朕以涼德承嗣丕基,十八年於茲矣。自親政以來,紀綱法度,用人行政,不能仰法太祖、太宗謨烈,因循悠忽,苟且目前,且漸習漢俗,於淳樸舊制,日有更張,以致國治未臻,民生未遂,是朕之罪一也。朕自弱齡即遇皇考太宗皇帝上賓,教訓撫養,惟聖母皇太后慈育是依,隆恩罔極,高厚莫酬。惟朝夕趨承,冀盡孝養,今不幸子道不終,誠悃未遂,是朕之罪一也。皇考賓天時,朕止六歲,不能服衰絰行三年喪,終天抱恨惟侍奉皇太后順志承顏,且冀萬年之後,庶盡子職,少抒前憾,今永違膝下,反上厪聖母哀痛,是朕之罪一也。宗室諸王、貝勒等,皆係太祖、太宗子孫,為國藩翰,理宜優遇,以示展親。朕於諸王貝勒等,晉接既疏,恩惠復鮮,以致情誼暌隔,友愛之道未周,是朕之罪一也。滿洲諸臣,或歷世竭忠,或累年効力,宜加倚託,盡厥猷為,朕不能信任,有才莫展。且明季失國,多由偏用文臣,朕不以為戒,而委任漢官,即部院印信,間亦令漢官掌管,以致滿臣無心任事,精力懈弛,是朕之罪一也。朕夙性好高,不能虛己延納,於用人之際,務求其德與己相侔,未能隨材器使,以致每歎乏人,若舍短錄長,則人有微技,亦獲見用,豈遂至於舉世無材,是朕之罪一也。設官分職,惟德是用,進退黜陟,不可忽視。朕於廷臣中,有明知其不肖,不即罷斥,仍復優容姑息如劉正宗者,偏私躁忌,朕已洞悉於心,乃容其久任政地,誠可謂見賢而不能舉,見不肖而不能退,是朕之罪一也。國用浩繁,兵餉不足,而金花錢糧,盡給宮中之費,未嘗節省發施,及度支告匱,每令會議,諸王大臣,未能別有奇策,祇議裁減俸祿,以贍軍餉,厚己薄人,益上損下,是朕之罪一也。經營殿宇,造作器具,務極精工,求為前代後人之所不及,無益之地,糜費甚多,乃不自省察,罔體民艱,是朕之罪一也。端敬皇后於皇太后克盡孝道,輔佐朕躬,內政聿修。朕仰奉慈綸,追念賢淑,喪祭典禮過從優厚,不能以禮止情,諸事踰濫不經,是朕之罪一也。 祖宗創業未嘗任用中官,且明朝亡國亦因委用宦寺。朕明知其弊,不以為戒,設立內十三衙門,委用任使與明無異,以致營私作弊,更踰往時,是朕之罪一也。朕性耽閒靜,常圖安逸,燕處深宮,御朝絕少,以致與廷臣接見稀疏,上下情誼否塞,是朕之罪一也。人之行事孰能無過?在朕日御萬幾豈能一無違錯?惟肯聽言納諫則有過必知。朕每自恃聰明,不能聽言納諫。古云:‘良賈深藏若虛,君子盛德容貌若愚。’朕於斯言大相違背,以致臣工緘默,不肯盡言,是朕之罪一也。朕既知有過,每日剋責生悔,乃徒尚虛文,未能省改,以致過端日積,愆戾愈多,是朕之罪一也。太祖、太宗創垂基業,所關至重,元良儲嗣,不可久虛。朕子玄烨,佟氏妃所生,年八歲,岐嶷穎慧,克承宗祧,茲立為皇太子,即遵典制持服二十七日,釋服,即皇帝位。特命內大臣索尼、蘇克薩哈、遏必隆、鰲拜為輔臣,伊等皆勳舊重臣,朕以腹心寄託,其勉矢忠藎,保翊冲主,佐理政務。布告中外,咸使聞知。”—— 《清世祖遗诏》

2月5日夜间,順治帝的遗诏颁示天下,特命索尼、苏克萨哈、遏必隆和鳌拜四人为了他年幼的儿子的辅政大臣,此四人都曾于多尔衮死后帮助济尔哈朗肃清朝廷上的多尔衮势力。很难确定福临是否确实任命四位满洲贵族为辅政大臣,因为福临的遗诏显然被昭圣慈寿皇太后和此四人所篡改。順治帝在遗诏中表示,他在施政之中偏向任用汉族大臣而且疏远了满洲官员(自己过分信用宦官,袒护汉官),忽视了满洲亲贵和满洲传统,对皇贵妃的精神投入超过了对自己的母亲。尽管福临在位时经常发布罪己诏,但这份遗诏中所谴责的政策自他亲政以来对清政府至关重要。被称为鳌拜辅政的1661年末至1669年间,该遗诏给了四位辅政大臣“皇权外披”,使他们的亲满政策得到支持。

由于朝廷没有明确宣布順治帝的死因,很快便流言四起。坊间传言福临其实未死,而是因为对爱妃之死过于悲痛或是四位获任为辅政大臣的满洲贵族发动了政变,他退位隐居佛教寺院,匿名为僧。因为順治帝于17世纪50年代成了佛教禅宗的狂热追随者,甚至让僧人进入皇宫,这些流言似乎不那么令人难以置信。中国现代历史学家认为福临出家之谜是清初三大疑案之一。但一位僧人记录说1661年2月初皇帝因感染天花而健康严重受损,而在皇帝的葬礼上有一名妃子和一名侍卫为其殉葬,由此来看福临之死应该並非假象。

福临的遗体被安放在紫禁城,受到为时27天的哀悼,1661年3月3日,一支规模宏大的行进列队将福临的遗体运送至景山(紫禁城北部的一个小丘), 之后大量贵重物品在葬礼上被烧掉。距离葬礼仅两年后的1663年,福临的遗体被运到他最后的安息之地。与当时的满洲习俗相同,福临的遗体在火化后安葬。他的骨灰安葬在北京东北方的昌瑞山,后来通常称为清东陵。他的陵墓孝陵是建在那里的第一座陵墓。

以順治帝的名义公布的遗诏表示,他对自己放弃满洲传统深表歉意,这一表示赋予了四辅政大臣实行本土主义政策的权力。鳌拜和其他三位辅政大臣援引遗诏,迅速革除了十三衙门。在此后的几年里,他们提升了满洲人及其包衣阿哈掌管的内务府的权力,革除翰林院,规定只有满洲人和蒙古人才能参加议政王大臣会议。辅政大臣还向大清治下的汉人推行强硬政策:他们发动文字狱处决了江南富庶地区的十余人,并以拖欠税收的罪名对该地区的数千人处以刑罚;他们强迫东南沿海地区人口从该地迁出,以便截断郑成功的子孙统治的台湾东宁王国的粮食供给。

玄烨于1669年设法囚禁鳌拜后,撤销了辅政大臣的许多政策。他恢复了父亲所青睐的机构,包括使汉族官员在政府中获得重要发言权的内阁。他还平定了三藩之乱。内战(1673年–1681年)使清人的忠心一度受到考验,但清军最终占得上风。当胜利成为定局时,1679年玄烨为吸引前明遗臣出仕清廷,而举行了特别考试博学鸿儒科。中试者被邀请参与编写官修《明史》。叛乱于1681年被平定,同年,玄烨开始倡导使用人痘接种为皇家儿童预防天花。郑氏家族在台湾建立的的东宁王国于1683年倒台后,清政权完成了统一天下的事业。在多尔衮、福临和玄烨奠定的体制基础上,清朝成为一个疆域辽阔、文化灿烂的强大帝国,被誉为“世界上最成功的帝国之一”。然而具有讽刺意义的是,正是康熙皇帝的赫赫武功带来的长时间的“满洲和平”,使大清面对19世纪列强武装侵略之时毫无准备。

\subsection{顺治}

\begin{longtable}{|>{\centering\scriptsize}m{2em}|>{\centering\scriptsize}m{1.3em}|>{\centering}m{8.8em}|}
  % \caption{秦王政}\
  \toprule
  \SimHei \normalsize 年数 & \SimHei \scriptsize 公元 & \SimHei 大事件 \tabularnewline
  % \midrule
  \endfirsthead
  \toprule
  \SimHei \normalsize 年数 & \SimHei \scriptsize 公元 & \SimHei 大事件 \tabularnewline
  \midrule
  \endhead
  \midrule
  元年 & 1644 & \tabularnewline\hline
  二年 & 1645 & \tabularnewline\hline
  三年 & 1646 & \tabularnewline\hline
  四年 & 1647 & \tabularnewline\hline
  五年 & 1648 & \tabularnewline\hline
  六年 & 1649 & \tabularnewline\hline
  七年 & 1650 & \tabularnewline\hline
  八年 & 1651 & \tabularnewline\hline
  九年 & 1652 & \tabularnewline\hline
  十年 & 1653 & \tabularnewline\hline
  十一年 & 1654 & \tabularnewline\hline
  十二年 & 1655 & \tabularnewline\hline
  十三年 & 1656 & \tabularnewline\hline
  十四年 & 1657 & \tabularnewline\hline
  十五年 & 1658 & \tabularnewline\hline
  十六年 & 1659 & \tabularnewline\hline
  十七年 & 1660 & \tabularnewline\hline
  十八年 & 1661 & \tabularnewline
  \bottomrule
\end{longtable}


%%% Local Variables:
%%% mode: latex
%%% TeX-engine: xetex
%%% TeX-master: "../Main"
%%% End:
