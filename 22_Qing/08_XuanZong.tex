%% -*- coding: utf-8 -*-
%% Time-stamp: <Chen Wang: 2021-11-01 17:22:41>

\section{宣宗道光帝旻寧\tiny(1821-1850)}

\subsection{生平}

道光帝(1782年9月16日-1850年2月26日),名旻寧,爱新觉罗氏,是清朝自入关以来的第六位皇帝,1820年10月3日至1850年2月26日在位,年号「道光」。西藏方面尊為「文殊皇帝」。

道光帝乃嘉庆帝次子,生母為孝淑睿皇后喜塔腊氏。原名綿寧,即位後為避免他人避諱麻煩而改名旻寧。他是清朝历史上僅有一位以嫡長子身分繼承皇位的皇帝。死後廟號宣宗,諡號成皇帝,葬于清西陵中的慕陵。

道光皇帝原名绵宁,1782年9月16日(乾隆四十七年八月初十日)出生於紫禁城擷芳殿。他出生时,父亲嘉庆帝颙琰尚为普通的皇子,母亲喜塔拉氏为颙琰福晋(嫡妻)。绵宁出生之前,嘉庆帝长子已夭折。绵宁成为他实际上的长子。绵宁從小即十分聰明,據說是祖父乾隆帝最疼愛的孫子,乾隆五十六年,賜黃馬褂,賞戴雙眼花翎。

乾隆帝执政的后期,父亲颙琰被立太子,乾隆帝禪讓,1796年(嘉庆元年)颙琰登基,同年,绵宁娶妻鈕祜祿氏。

1799年(嘉庆四年四月),嘉庆帝依照秘密建儲制,立绵宁其为太子。嘉庆十三年(1808年),他的第一个孩子奕纬出生。

道光帝在繼位之前,其騎射武功在嘉慶帝諸子裡相當聞名,亦習得一手好槍法。嘉庆十八年(1813年),這年旻寧33歲,因天理教癸酉之变,取出宮中禁用的鳥銃,連殺二敵的英勇表现,封为「智亲王」,所執的鳥銃也被封為“威烈”。根據中央研究院歷史語言研究所內閣大庫檔案038280號,嘉慶25年封智親王、皇太子同時接任皇帝。

嘉慶二十五年(1820年)七月十八日,嘉慶帝到熱河木蘭圍獵,命皇次子智親王綿寧、皇四子瑞親王綿忻隨駕。這年他35歲,“身體豐腴,精神強固”。二十四日,到達熱河行宮,“聖躬不豫”。當天,嘉慶帝到城隍廟燒香,又到永佑宮行禮,二十五日,病情嚴重,當夜崩逝,死因不明,據今日推測,嘉慶帝死亡的原因可能是年逾花甲,身體肥胖,天氣暑熱,旅途勞頓,誘發腦中風或心臟病而暴斃。旻寧繼位,得以禧恩為代表的宗室之建議和認同,又得到皇太后的中宮懿旨和皇弟瑞親王綿忻的贊同,最主要是有軍機大臣等開啟鐍匣的密諭。

绵宁继位後,免眾兄弟避諱改名比照父親顒琰改名,所以改名旻宁,定年號為道光。即位时正值鸦片氾濫,道光帝为挽救国家财政危机,也主张禁烟,多次下诏禁鸦片进口,禁止自种自制。之後鴉片戰爭爆發,由于道光帝战守无策,時和時戰,再加上武器装备上的差距,清朝战败于英国,并与英人簽訂近代中国的首條不平等條約──《南京條約》,割讓香港岛及開放五口通商。

道光年間,推行三項改革措施:漕糧海運、改綱鹽法為票鹽法、允許開採礦產。

道光三十年正月十五曰(1850年2月26日),道光帝因堅持為其繼母恭慈皇太后守靈,以致生病,在圓明園九洲清晏殿駕崩,享壽六十八歲。安葬于清慕陵(今河北省易县西)。

咸丰元年正月初六日,库掌祥麟持来报单一件,内开咸丰元年正月初四日总管内务府大臣面奉谕旨,恭绘热河绥成殿宣宗成皇帝圣容,著沈振麟于正月底吉日敬谨恭绘。咸豐帝要求圣容要盘膝坐,前方设置放着《易经》首页乾卦的書案。

历史学家孟森认为:“宣宗之庸暗,亦为清朝入关以来所未有。”称这时期为「嘉道中衰」。

蔡東藩:「徒齊其末,未揣其本,省衣減膳之為,治家有餘,治國不足。」

大事年表:乾隆四十七年八月初十日,綿寧在擷芳殿出生。嘉慶元年,以钮祜禄氏為嫡福晋。嘉慶十八年九月,封為智親王。嘉慶二十五年七月,仁宗去世,綿寧繼位,更名旻寧。道光八年,平定在西域新疆地區西南部為期八年的張格爾之亂,嚴禁新疆與支持白山派和卓張格爾反清的浩罕汗國通商。道光十一年,浩罕汗國遣使議和進貢。道光十二年,准許西域新疆地區與浩罕汗國重開貿易。道光十八年閏四月,黃爵滋奏請「將內地吸食鴉片者俱罪死」。十一月命林則徐為欽差大臣,赴廣東查禁鴉片。道光十九年四月廿二日,虎門銷煙開始。道光二十年五月二十九日,英艦封鎖廣州珠江口,鴉片戰爭正式開始。英艦北上,六月攻陷浙江定海,七月抵達天津附近,其後返回廣東。九月林則徐被革職。琦善與英方全權代表義律商議和約,十二月義律單方面公佈《穿鼻草約》。同年,位於喀什米爾地區東南部的拉達克王國,面臨錫克帝國查謨拉者古拉卜·辛格派遣左拉瓦爾·辛格兵團(主帥全名左拉瓦爾·辛格·卡赫盧里拉)進攻的亡國危機,遣使向大清駐藏大臣求援遭拒。道光二十一年正月,英軍佔領香港。道光帝不承認《穿鼻草約》,二月琦善被革職,押京審理。五月,錫克帝國屬地查謨-克什米爾地區多格拉人的左拉瓦爾·辛格·卡赫盧里拉兵團趁併吞拉達克王國的滅國威勢,進攻清屬西藏阿里地區,爆發森巴戰爭(藏人稱多格拉人為森巴)。道光二十二年七月,英軍兵臨南京,清廷同意議和,《南京條約》立。冬季,西藏阿里地區西北部爆發的森巴戰爭,以錫克帝國多格拉兵團主帥左拉瓦爾·辛格·卡赫盧里拉和駐藏清軍交戰陣亡、餘眾敗走告終。道光二十三年八月,《中英五口通商章程》立。道光二十六年正月,正式解除對天主教的禁令。道光二十七年,平定在西域新疆地區西南部爆發的七和卓之亂。道光三十年正月,道光帝在圓明園去世。

\subsection{道光}

\begin{longtable}{|>{\centering\scriptsize}m{2em}|>{\centering\scriptsize}m{1.3em}|>{\centering}m{8.8em}|}
  % \caption{秦王政}\
  \toprule
  \SimHei \normalsize 年数 & \SimHei \scriptsize 公元 & \SimHei 大事件 \tabularnewline
  % \midrule
  \endfirsthead
  \toprule
  \SimHei \normalsize 年数 & \SimHei \scriptsize 公元 & \SimHei 大事件 \tabularnewline
  \midrule
  \endhead
  \midrule
  元年 & 1821 & \tabularnewline\hline
  二年 & 1822 & \tabularnewline\hline
  三年 & 1823 & \tabularnewline\hline
  四年 & 1824 & \tabularnewline\hline
  五年 & 1825 & \tabularnewline\hline
  六年 & 1826 & \tabularnewline\hline
  七年 & 1827 & \tabularnewline\hline
  八年 & 1828 & \tabularnewline\hline
  九年 & 1829 & \tabularnewline\hline
  十年 & 1830 & \tabularnewline\hline
  十一年 & 1831 & \tabularnewline\hline
  十二年 & 1832 & \tabularnewline\hline
  十三年 & 1833 & \tabularnewline\hline
  十四年 & 1834 & \tabularnewline\hline
  十五年 & 1835 & \tabularnewline\hline
  十六年 & 1836 & \tabularnewline\hline
  十七年 & 1837 & \tabularnewline\hline
  十八年 & 1838 & \tabularnewline\hline
  十九年 & 1839 & \tabularnewline\hline
  二十年 & 1840 & \tabularnewline\hline
  二一年 & 1841 & \tabularnewline\hline
  二二年 & 1842 & \tabularnewline\hline
  二三年 & 1843 & \tabularnewline\hline
  二四年 & 1844 & \tabularnewline\hline
  二五年 & 1845 & \tabularnewline\hline
  二六年 & 1846 & \tabularnewline\hline
  二七年 & 1847 & \tabularnewline\hline
  二八年 & 1848 & \tabularnewline\hline
  二九年 & 1849 & \tabularnewline\hline
  三十年 & 1850 & \tabularnewline
  \bottomrule
\end{longtable}


%%% Local Variables:
%%% mode: latex
%%% TeX-engine: xetex
%%% TeX-master: "../Main"
%%% End:
