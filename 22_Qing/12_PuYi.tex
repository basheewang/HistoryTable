%% -*- coding: utf-8 -*-
%% Time-stamp: <Chen Wang: 2019-12-26 22:01:02>

\section{溥仪\tiny(1909-1912)}

\subsection{生平}

溥儀(1906年2月7日-1967年10月17日),愛新覺羅氏,乳名午格,字耀之,號浩然,英語名「亨利」(Henry),西藏方面尊為「文殊皇帝」,年號「宣統」,後世通稱「宣統皇帝」。

溥儀正式登基時年僅3歲,其實權由父親攝政王載灃掌握。辛亥革命以後,溥儀被袁世凱強逼退位,故亦有「清遜帝」或「清廢帝」之稱。1917年,溥儀在張勳等人的支持和幫助下,曾短暫復辟但最終失敗。1934年,溥儀在日本支持和幫助下登基成為满洲国皇帝,年號「康德」,故又得名「康德皇帝」。

溥儀是清朝第十二位君主、清兵入關以來的第十位皇帝,是清朝最後一任皇帝和公認的「中國最後一位皇帝」即「末代皇帝」,亦是全世界唯一登基三次和退位三次的皇帝。

溥儀的祖父為道光帝七子、咸豐帝之弟奕譞,外祖父為榮祿,生父載灃为光绪帝之弟,后继承醇親王爵位,母嫡福晋幼兰。光绪三十四年冬(公元1908年),光绪帝载湉病重,慈禧太后下令将溥仪养育在宫中。不久光绪帝去世,慈禧太后命溥仪继承皇统,过继于同治帝载淳,同时兼承光绪帝之祧,一人兼祧两房。

光绪三十二年春正月十四日(1906年2月7日),溥儀出生在北京醇親王府(今北京市西城区后海北沿44号)。祖父是道光皇帝旻寧七子咸豐皇帝奕詝之弟奕譞,外祖父是榮祿,父親是載灃為光緒皇帝載湉之弟,後繼承醇親王爵位,母親是蘇完瓜爾佳·幼蘭。

載灃生了四個兒子,依次為溥儀、二子溥傑(1907年-1994年)、三子溥倛(1915年-1918年)、四子溥任(1918年-2015年)與七個女兒,依次為長女韞媖(1909年-1925年)、次女韞龢(1911年-2001年)、三女韞穎(1913年-1992年)、四女韞嫻(1914年-2003年)、五女韞馨(1917年-1998年)、六女韞娛(1919年-1982年)、七女韞歡(1921年-2004年)。

光緒三十四年冬季(1908年),光緒皇帝載湉患重病,11月13日三歲的溥儀被抱入皇城,慈禧太后命令將溥儀養育在宮中,11月14日光緒皇帝載湉病逝,慈禧太后命令溥儀繼承登基成為皇帝,過繼於同治帝,同時兼承光緒皇帝載湉之宗祧,11月15日慈禧太后駕崩,享壽七十三歲(《清史稿·本紀二十五·宣統皇帝本紀》:三十四年冬十月壬申,德宗疾大漸,太皇太后命教養宮內。癸酉,德宗崩,奉太皇太后懿旨,入承大統,為嗣皇帝,嗣穆宗,兼承大行皇帝之祧,時年三歲)。溥仪小时候曾被太监欺负,所以导致他出现畸形性格。

1908年12月2日,光绪皇帝與慈禧太后相繼去世後,朝廷眾大臣奉慈禧太皇太后遺囑,溥儀在紫禁城太和殿內登基成為大清王朝第十二任皇帝,年號宣統,年僅三歲,其父載灃擔任監國攝政王。

1911年武昌起義成功,大清帝國各行省各自宣布獨立,脫離大清帝國政府管轄和控制,但大清帝國政府依然管轄和控制北京市附近省份,並派遣袁世凱使用北洋陸軍攻打革命黨人。南方革命軍與袁世凱商定若能成功逼使溥儀退位和結束大清政權,便讓他成為中華民國大總統,是為南北議和。袁世凱便一面好言相勸,一面威逼利誘溥儀退位。

1912年2月12日(宣統三年十二月戊午),隆裕皇太后以大清帝國皇太后名義宣布《退位詔書》,溥儀正式退位,結束大清帝國自順治皇帝福臨入主中原以来268年的統治。

溥儀退位之後,因為中華民國北洋政府的《清室優待條件》而能繼續居住在紫禁城內,並保留「大清帝國」國號(只限在紫禁城內),溥儀依然保留宦官,宮女在紫禁城內供其使喚。

1917年7月1日,北洋政府陸軍定武上將軍,安徽省督軍張勳協同陳寶琛、王士珍、陳光遠、康有為、劉廷琛、沈曾植和勞乃宣等人發動政變,宣佈大清帝國復辟。張勳復辟大清帝國的行動遭到共和派系勢力的反對與攻擊,在北洋政府各界壓力和不滿之下,復辟行動僅12天便宣告失敗,溥儀也第二次宣布退位,結束大清帝國政權。

1919年,英國蘇格蘭人莊士敦前往北京紫禁城,擔任溥儀帝師,教授并指導溥儀學習英語、數學、世界歷史和地理。溥儀和莊士敦二人關係良好感情深厚,傳為佳話,為人津津樂道。溥儀因此對現代世界眼界大開,开闊了國際視野,增加了西方基本知識。溥儀剪掉自己的髮辮并穿著西服,此舉遭到陳寶琛,鄭孝胥等傳統保守派人士的不滿和批評。

1924年10月23日,馮玉祥、胡景翼和孫岳發動北京政變。11月5日,馮玉祥突然帶領軍隊包圍紫禁城。鹿鍾麟奉大總統黃郛之命令,帶著《修正清室優待條件》宣言文件,與李石曾和張璧帶領軍隊佔領紫禁城,使用武力要求溥儀簽署宣統皇帝退位聲明,取消宣統皇帝尊稱和命令溥儀離開紫禁城,如果溥儀拒絕答應其要求,馮玉祥威脅會使用多門火炮射擊紫禁城。溥儀為保護紫禁城免遭破壞,別無選擇只能無奈地答應其要求,馮玉祥限溥儀二天時間內收拾個人物品離開紫禁城。

溥儀離開紫禁城之後,先前往父親載灃的宅邸醇王府居住,並由國民軍進行保護(實際上是監視)。1925年2月24日,溥儀在鄭孝胥,陳寶琛和日本人的協助下,裝扮成商人經東交民巷日本大使館至使館前方的火車站乘車逃往天津市,溥儀先後居住於天津市日租界的張園和靜園。

1928年6月,孙殿英以“军事演习”的旗号,对清东陵当中的裕陵和菩陀峪定东陵进行大规模盗掘,并将其中部分盗取的宝物贿赂宋美龄、孔祥熙等人,案件查办最终不了了之。该事件史称“东陵事件”,国民政府不追究孙殿英的责任,导致溥仪和国民政府完全决裂,这是溥仪和日本人合作的重要原因之一。

1931年9月18日滿洲事變發生後,前大清帝國穆爾哈齊的後裔熙洽,趁東北邊防軍駐吉林副司令官、吉林省政府主席張作相參加母親葬禮不在吉林市城內之機,命令士兵開啟城門,向日本關東軍投降。熙洽給溥儀發信邀請其前往祖宗發祥地,復辟大清帝國,「救民於水火」,在日本的支持下,先擁有滿洲,再圖關內。以任職吉林省政府主席的熙洽為首的前大清帝國貴族向日本人提出迎接溥儀前来滿洲,建立君主制國家。日本關東軍也早已認定溥儀是適合的新國家(滿洲國)君主人選。

1931年11月8日,土肥原賢二製造了「天津事件」,將溥儀從其日租界的住所秘密帶出,溥儀經大沽街,營口市,旅順口區,最後再前往撫順市。 1932年2月16日,日本關東軍召集張景惠、熙洽、馬占山、臧式毅、謝介石、于沖漢、趙欣伯和袁金鎧在瀋陽市大和旅館召開「東北政務會議」,會議由日本關東軍司令官本庄繁主持。東北政務會議決定迎接溥儀成為滿洲國執政,並分配了與會者在滿洲國政權中的職位,其中板垣徵四郎任職奉天特務機關長,為滿洲國軍政部最高顧問。 18日,發布《滿洲國獨立宣言》:「從即日起宣佈滿蒙地區同中國中央政府脫離關係,根據滿蒙居民的自由選擇與呼籲,滿蒙地區從此實行完全獨立,成立完全獨立自主之政府。 」23日,板垣徵四郎在撫順與溥儀會面,告知溥儀出任滿洲國執政。原本以為能夠重新成為皇帝的溥儀儘管對於日本人所安排的「執政」職位甚為失望,但最後還是欣然接受。

1932年3月1日,日本在滿洲地區正式成立滿洲國。3月9日,溥儀在長春市吉長道尹公署道台衙門大堂舉行就職典禮儀式,宣布就任滿洲國執政。

1934年3月1日,溥儀正式登基成為皇帝,年號康德,又被稱為康德皇帝。日本昭和天皇為表慎重其事,在溥儀登基典禮的時候,贈送一輛凱迪拉克豪華都鐸8C型轎車(Cadillac Deluxe Tudor Limousine 8C)。車首前方,車體後方和車輪中央都鑲有滿洲國國徽,以表示對溥儀登基成為滿洲國康德皇帝的祝賀。溥儀雖然名義上貴為滿洲國康德皇帝,但實際上所有重大權力和決定都要得到日本關東軍的批准才可以實行。而滿洲國康德皇帝只是个象徵性的頭銜,实为傀儡君主。

1935年(康德二年、昭和十年)4月6日,在日本关东军的授意下溥仪以满洲国皇帝的身份首次访问日本国首都东京,受到日方高规格接待。1940年(康德七年、昭和十五年)6月26日,溥仪第二次访问东京,日本昭和天皇裕仁亲自迎接。

據美國《歷史》雜誌報導,1940年,溥儀秘密聯繫薩爾瓦多外交代表團人員,希望能逃亡薩爾瓦多,擺脫日本人控制。薩爾瓦多外交代表團人員返國後,將溥儀的意願報告給薩爾瓦多總統馬丁內斯。正好馬丁內斯是一個神秘主義者,認為自己與溥儀都是螞蟻轉世,他曾對薩爾瓦多外交代表團人員說:“殺死一隻螞蟻,比殺死一個人罪行嚴重得多!”他認為溥儀前往薩爾瓦多是上天的安排,便不顧與日本關係惡化的危險,亳不猶豫地答應了溥儀的請求。

1941年10月,又有薩爾瓦多外交代表團人員到達新京特別市(今吉林省長春市),溥儀把逃亡薩爾瓦多的計劃告訴了一名滿洲國禁衛隊軍官,打算讓滿洲國禁衛隊護送自己前往薩爾瓦多大使館,然後再裝扮成大使館職員逃離滿洲國。沒想到的是,滿洲國禁衛隊早被日本關東軍收買,那名禁衛隊軍官向日本關東軍告密,溥儀逃亡計劃完全失敗。日本陸軍參謀本部立即派出憲兵隊,將薩爾瓦多外交代表團人員驅逐,關閉薩爾瓦多駐滿洲國大使館和數間薩爾瓦多駐滿洲國貿易公司以作懲罰,從此薩爾瓦多中斷與日本的外交結盟關係。日本關東軍人員前往滿洲國宮內府向溥儀提出威胁性交涉和斥责。

1945年8月9日,蘇聯開始八月風暴行動。蘇聯紅軍迅速打敗了駐守在中國東北的日本關東軍。 11日晚上,溥儀,溥傑,嵯峨浩和其他親屬在日本關東軍士兵挟持下在新京東站登上火車展開逃亡行動。 13日到達臨江市大栗子街,停留數日觀察最新戰爭局勢来決定是否要前往鴨綠江大橋進入朝鮮半島境內。15日,裕仁天皇宣布日本投降。 17日晚上,溥儀在大栗子溝宣讀滿洲國皇帝退位詔書和取消滿洲國康德皇帝尊稱,宣告滿洲國正式滅亡。之後,溥儀,溥傑,嵯峨浩和其他親屬乘坐火車前往通化市,然后在瀋陽東塔機場乘坐日本關東軍飛機欲逃亡日本。

1945年8月19日,溥儀、溥傑、嵯峨浩和其他親屬在瀋陽東塔機場乘坐日本關東軍飛機準備逃亡日本的時侯,被蘇聯紅軍空降傘兵逮捕,溥儀等人被蘇聯士兵扣留在通遼市至8月20日(有一說法是8月21日),然後被蘇聯空軍飛機運送到俄羅斯赤塔一號軍用機場,囚禁於莫洛可夫卡30號特別監獄直至11月初。之後被囚禁在伯力45號特別監獄直至1946年春季。溥儀在伯力45號特別監獄內受到優厚的待遇,令其多次向蘇聯政府表示願意申請在蘇聯永久居留,並申請加入蘇聯共產黨,但有推測認為這可能是溥儀害怕日後被追究責任,故而申請在蘇聯永久居留。溥儀在蘇聯囚禁期間,曾經作為證人出席遠東國際軍事法庭 。溥儀在遠東國際軍事法庭出任證人的時候,聲稱自己在就任滿洲國康德皇帝期間,完全為日本關東軍所控制,自己也是身不由己的,也沒有滿洲國康德皇帝作為最高元首的最大實際決策權力和指揮權力。但是,被轉交給中華人民共和國政府後,溥儀承認由於懼怕日後被中國政府追究,作證時將部分責任推卸給日本方面(含如何到達內滿洲),在部分涉及雙方責任的地方皆有所保留。

1950年8月1日,溥儀與滿洲國其他263名戰犯在綏芬河由蘇聯移交給中華人民共和國,後被送往撫順戰犯管理所接受為期10年的勞動改造和思想教育。溥儀的囚犯編號是981。

1959年12月4日,中華人民共和國最高人民法院遵照國家主席劉少奇特赦令,特赦首批戰爭罪犯:132。特赦令說:「溥儀關押已經滿十年。在關押期間,經過勞動改造和思想教育,已經有確實改惡從善的表現,符合特赦令第一條的規定,予以釋放。」溥儀被特赦後離開撫順戰犯管理所:132。從此,溥儀成為中華人民共和國公民。

1960年3月,溥儀被分配到中国科学院北京植物园任職植物護理員和售票員。1964年1月1日,溥儀加入政協全國委員會,任職文化歷史資料研究委員會專員。

1967年10月17日,溥儀因患腎癌,在北京大學人民醫院病逝,終年61歲。

溥儀的遺體依據中華人民共和國的有關法規火化,溥儀的骨灰放置在八寶山革命公墓。1995年,溥儀的遺孀李淑賢將溥儀的骨灰葬於北京市西南120公里的河北省易縣華龍皇家陵園,溥儀墓在清西陵附近。

溥儀的親弟弟溥傑後來與昭和天皇的表妹嵯峨浩結婚,滿洲國的帝位繼承法規定以溥傑繼承沒有子嗣的溥儀。

溥儀的異母弟弟溥任(1918-2015)取漢名金友之,曾居住於中國大陸直到2015年去世。金友之曾于2006年就溥儀的肖像權和隱私權提起訴訟。金友之聲稱,“中國最後的帝王世家展”嚴重侵犯溥儀的肖像權,同時對死者親屬造成巨大的精神侵害,侵犯原告對溥儀肖像的使用權。金友之的上訴最終被駁回。

\subsection{宣统}

\begin{longtable}{|>{\centering\scriptsize}m{2em}|>{\centering\scriptsize}m{1.3em}|>{\centering}m{8.8em}|}
  % \caption{秦王政}\
  \toprule
  \SimHei \normalsize 年数 & \SimHei \scriptsize 公元 & \SimHei 大事件 \tabularnewline
  % \midrule
  \endfirsthead
  \toprule
  \SimHei \normalsize 年数 & \SimHei \scriptsize 公元 & \SimHei 大事件 \tabularnewline
  \midrule
  \endhead
  \midrule
  元年 & 1909 & \tabularnewline\hline
  二年 & 1910 & \tabularnewline\hline
  三年 & 1911 & \tabularnewline\hline
  四年 & 1912 & \tabularnewline
  \bottomrule
\end{longtable}


%%% Local Variables:
%%% mode: latex
%%% TeX-engine: xetex
%%% TeX-master: "../Main"
%%% End:
