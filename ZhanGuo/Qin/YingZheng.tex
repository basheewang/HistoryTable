%% -*- coding: utf-8 -*-
%% Time-stamp: <Chen Wang: 2018-07-10 01:37:02>

\subsection{赢政{\tiny(BC246-BC221)}}


% \centering
\begin{longtable}{|>{\centering\scriptsize}m{2em}|>{\centering\small}m{2em}|>{\centering}m{8.3em}|}
  % \caption{秦王政}\\
  \toprule
  \SimHei \normalsize 年数 & \SimHei \normalsize 公元 & \SimHei 大事件 \tabularnewline
  % \midrule
  \endfirsthead
  \toprule
  \SimHei \normalsize 年数 & \SimHei \normalsize 公元 & \SimHei 大事件 \tabularnewline
  \midrule
  \endhead
  \midrule
  一年\\\threept 齐王建19年\\赵孝成王20年\\魏安僖王31年\\韩桓惠王27年\\楚考烈王17年\\卫元君6年\\燕王喜9年 & -246 & \begin{enumerate}
    \tiny
  \item 秦王\CJKunderline{政}即位。
  \item 韩国人\CJKunderline{郑国}\footnote{郑国,战国时期韩国卓越的水利专家,出生于韩国都城新郑(现在河南省新郑市)。成年后,郑国曾任韩国管理水利事务的水工(官名),参与过治理荥泽水患以及整修鸿沟之渠等水利工程。后来被韩王派去秦国修建水利工事,从而“疲秦”,而郑国渠修建之后,关中成为天下粮仓,赢得了“天府之国”的美名。虽然郑国作为间谍不成功,但是作为一名卓越的水利专家,治理水患,改变了关中农业区的面貌,使八百里秦川成为富饶之乡。郑国渠和都江堰、灵渠并称为秦代三大水利工程。}始建郑国渠。
  \item 晋阳被秦所占领,发生暴动,\CJKunderline{蒙骜}\footnote{蒙骜(?—公元前240年),《战国策》作蒙傲,战国末期秦国著名将领。蒙骜本是齐国人,后来投靠秦国,官至上卿。蒙骜历仕秦昭襄王、秦孝文王、秦庄襄王、秦始皇四朝,数次率军出征,屡立战功。先后夺取韩国十余座城池、赵国三十余座城池、魏国五十余座城池,使秦国得以设立三川郡和东郡,并让秦国疆域与齐国相接,对韩国、魏国形成三面包围之势,为日后秦始皇统一六国打下坚定的基础。公元前240年,蒙骜去世,时年七十多岁。其子蒙武、其孙蒙恬、蒙毅都是秦国名将。}击定之。
  \end{enumerate} \tabularnewline\hline
  二年 & -245 & \begin{enumerate}
    \tiny
  \item 赵孝成王逝世,悼襄王即位。
  \item \CJKunderline{廉颇}攻繁阳。
  \item 秦将\CJKunderline{麃公}\footnote{麃(biāo)公,姓名、生卒年不详,战国时期秦国的将军。秦王政元年(公元前246年),秦王政即位,麃公与蒙骜、王齮同为将军。秦王政二年(公元前245年),麃公率军攻打卷城,斩首三万人。}攻伐魏国卷地(今河南原阳西北),斩首三万。
  \item 吕不韦得河间十城。
  \item 鲁仲连\footnote{鲁仲连(约前305~前245)战国时名士。亦称鲁连。今茌平人。善于出谋划策,常周游各国,为其排难解纷。赵孝王九年,秦军围困赵国国都邯郸。迫于压力,魏王派使臣劝赵王尊秦为帝,赵王犹豫不决。鲁仲连以利害说赵、魏两国联合抗秦。两国接受其主张,秦军以此撤军。20余年后,燕将攻占齐国的聊城。齐派田单收复聊城却久攻不下,双方损兵折将,死伤严重。鲁仲连闻之赶来,书写了一封义正辞言的书信,射入城中,燕将读后,忧虑、惧怕,遂拔剑自刎,于是齐军轻而易举攻下聊城。赵、齐诸国大臣皆欲奏上为其封官嘉赏。他一一推辞,退而隐居。《汉书·艺文志》载有《鲁仲连子》14篇,今佚。}逝世。
  \end{enumerate} \tabularnewline\hline
  三年 & -244 & \begin{enumerate}
    \tiny
  \item 赵边将\CJKunderline{李牧}\footnote{李牧(?-公元前229年),嬴姓,李氏,名牧,战国时期赵国柏仁(今河北省邢台市隆尧县)人,战国时期的赵国名将、军事家,与白起、王翦、廉颇并称“战国四大名将”。战国末期,李牧是赵国赖以支撑危局的唯一良将,素有“李牧死,赵国亡”之称。}率军大规模反击匈奴,斩杀匈奴10余万骑兵。
  \item 燕王\CJKunderline{喜}使太子\CJKunderline{丹}入质于秦。
  \item \CJKunderline{蒙骜}攻韩十三城,攻魏氏畼、有诡。
  \end{enumerate} \tabularnewline\hline
  四年 & -243 & \begin{enumerate}
    \tiny
  \item 信陵君\CJKunderline{魏无忌}\footnote{魏无忌(?—前243年),即信陵君,魏国公子,与春申君黄歇、孟尝君田文、平原君赵胜并称为“战国四公子”。是战国时期魏国著名的军事家、政治家,魏昭王少子、魏安釐王的异母弟。公元前276年,被封于信陵(河南宁陵县),所以后世皆称其为信陵君。}逝世。
  \item 佛教入中国。
  \item 赵将\CJKunderline{李牧}攻燕,拔武遂、方城。
  \end{enumerate} \tabularnewline\hline
  五年 & -242 & \begin{enumerate}
    \tiny
  \item 燕军十万犯赵,\CJKunderline{庞锾}率军抵之。
  \item 将军\CJKunderline{蒙骜}攻打魏国,夺取了二十个城邑。
  \end{enumerate} \tabularnewline\hline
  六年 & -241 & \begin{enumerate}
    \tiny
  \item 楚国向东迁都到寿春,改其名为郢。
  \item 秦继续攻魏,占领了魏地朝歌(今河南淇县)及卫濮阳(今河南濮阳西南),以濮阳为东郡治所。
  \item 赵庞煖率赵、楚、魏、燕、韩五国兵攻秦,至蕞。
  \end{enumerate} \tabularnewline\hline
  七年 & -240 & \begin{enumerate}
    \tiny
  \item 秦置濮阳县,属东郡,并定其为东郡治所。
  \item \CJKunderline{蒙骜}逝世\footnote{秦王政七年(BC240年),赵悼襄王六年秦长安君及大将军蒙骜率军十万攻赵,赵将庞煖领军十万御之,斩杀秦军大半,射杀蒙骜。赵国顿时国威大震。}。
  \item \CJKunderline{陆贾}\footnote{陆贾(约公元前240~前170),汉初思想家,政治家,楚人。早年随刘邦平定天下,口才极佳,常出使诸侯。刘邦即帝位后,他受命出使南越,说服尉佗接受汉朝赐予的南越王印,称臣奉汉约,被任为太中大夫。刘邦即位之初,重武力,轻诗书,以“居马上得天下”自矜,他乃建议重视儒学,“行仁义,法先圣”,提出“逆取顺守,文武并用”的统治方略,遂受命总结秦朝灭亡及历史上国家成败的经验教训,共著文12篇,每奏一篇,高祖无不称善,故名其书为《新语》。}出生。
  \item 彗星光出东方,见北方,五月见西方。
  \end{enumerate} \tabularnewline\hline
  八年 & -239 & \begin{enumerate}
    \tiny
  \item 北扶余王国建立。
  \item \CJKunderwave{吕氏春秋}编成。
  \item 长安君\CJKunderline{成蟜}\footnote{成蟜,嬴姓,生卒年不详,战国末年秦国公子,秦庄襄王之子,秦王政之弟,后在屯留叛秦降赵,史称“成蟜之乱”,叛乱平定后,其部军吏皆因连坐被斩首处死,屯留百姓被流放,成蟜亡命到赵国,被授予封地饶。}死。
  \end{enumerate} \tabularnewline\hline
  九年 & -238 & \begin{enumerate}
    \tiny
  \item \CJKunderline{荀子}\footnote{荀子(约公元前313-前238)名况,字卿,后避汉宣帝讳,改称孙卿。战国时期赵国猗氏(今山西新绛)人,著名思想家、文学家、政治家,儒家学派代表人物,时人尊称“荀卿”。曾三次出齐国稷下学宫的祭酒,后为楚兰陵(今山东兰陵)令。《史记·荀卿列传》记录了他的生平。荀子于五十年始来游学于齐,至襄王时代“最为老师”,“三为祭酒”。后来被逸而适楚,春申君以为兰陵令,春申君死而荀卿废,家居兰陵,在此期间,他曾入秦,称秦国“治之至也”。又到过赵国与临武君议兵于赵孝成王面前。最后老死于楚国。}逝世。
  \item 嬴政亲政。
  \item 嫪毐叛乱\footnote{秦王政命相国昌平君、昌文君领咸阳士卒平息叛乱,两军战于咸阳。秦王下令:“凡有战功的均拜爵厚赏,宦官参战的也拜爵一级。”叛军数百人被杀死,嫪毐的军队大败,与死党仓皇逃亡。秦王令谕全国:“生擒嫪毐者赐钱百万,杀死嫪毐者赐钱五十万。”嫪毐及其死党被一网打尽,秦王车裂嫪毐,灭其三族。}。
  \item 秦伐魏,取垣、浦。
  \item 楚春申君\CJKunderline{黄歇}\footnote{黄歇(前314年-前238年),黄国(今河南省潢川县)人,楚国大臣,曾任楚相。黄歇游学博闻,善辩。楚考烈王元年(公元前262年),以黄歇为相,赐其淮河以北十二县,封为春申君。与魏国信陵君魏无忌、赵国平原君赵胜、齐国孟尝君田文并称为“战国四公子”。}亡。
  \end{enumerate} \tabularnewline\hline
  十年 & -237 & \begin{enumerate}
    \tiny
  \item 齐王\CJKunderline{建}去拜会秦王\CJKunderline{嬴政}。
  \item \CJKunderline{吕不韦}免相。
  \item \CJKunderline{嬴政}下令驱除异邦客卿,\CJKunderline{李斯}\footnote{李斯(约公元前284年—公元前208年),李氏,名斯,字通古。战国末期楚国上蔡(今河南省驻马店市上蔡县芦冈乡李斯楼村)人。秦代著名的政治家、文学家和书法家。}上书劝秦始皇收回逐客令。
  \end{enumerate} \tabularnewline\hline
  十一年 & -236 & \begin{enumerate}
    \tiny
  \item 郑国渠建成。
  \item 秦攻赵,赵攻燕\footnote{公元前236年,秦乘攻取赵的阏与、橑阳、邺、安阳等城,后又大举攻赵,遭到顽强抵抗。赵虽两次打败秦军,但兵力耗损殆尽。秦国西出太行山,突袭赵国邯郸拉开了统一战的的序幕。 赵国和燕国激战正酣,他想将秦国造成的领土损失在燕国身上补回来。这时秦国乘虚而入。赵国急忙命令大将李牧率军南下应敌。}。
  \end{enumerate} \tabularnewline\hline
  十二年 & -235 & \begin{enumerate}
    \tiny
  \item 秦攻楚国\footnote{秦继攻赵之后,即命辛梧率四郡兵,会同魏国,对楚国发起攻击。}。
  \item \CJKunderline{吕不韦}卒\footnote{因嫪毐集团叛乱事受牵连,被免除相邦职务,出居河南封地。不久,秦王政下令将其流放至蜀地(今四川),不韦忧惧交加,于是在三川郡(今河南洛阳)自鸩而亡。}。
  \end{enumerate} \tabularnewline\hline
  十三年 & -234 & \begin{enumerate}
    \tiny
  \item 秦攻赵\footnote{公元前234年,秦再度向赵南部进攻。桓龁避开正面渡河,改由漳河下游渡河迂回赵扈辄军的侧后,攻击邯郸东南的平阳。两军于平阳展开交战,赵军被击破,被斩10万人,赵将扈辄阵亡。赵王启用北部边疆名将李牧为统帅。李牧军曾歼灭匈奴入侵军10万之众,威震边疆,战斗力最强。李牧率军回赵,立即同秦桓龁军交战于宜安肥下地区,给秦军几乎全军覆灭的沉重打击,只有统帅桓龁带领少数护卫突围逃走。}。
  \item \CJKunderline{韩非}\footnote{韩非(约前281年-前233年),生活于战国末期时期的韩国(今属河南省新郑市)的思想家,为中国古代著名法家思想的代表人物,认为应该要“法”、“术”、“势”三者并重,是法家的集大成者。韩非出身韩国公族,与李斯均是荀子学生,后因其学识渊博,被秦始皇召唤入秦,正欲重用,却遭到妒忌的同窗李斯害死,在韩非死后,秦始皇在韩非的思想指引下,完成统一六国的帝业。韩非其学出于荀子,源于儒家,而成为法家,又推究老子思想,归本于道家。司马迁指出韩非喜好“刑名法术”且归本于道家的“黄老之学”,一套由“道”、“法”共同完善的政治统治理论。}作为韩国的使臣来到秦国,上书秦王,劝其先伐赵而缓伐韩。
  \end{enumerate} \tabularnewline\hline
  十四年 & -233 & \begin{enumerate}
    \tiny
  \item \CJKunderline{韩非子}卒。
  \item 燕抗秦\footnote{公元前233年,秦将樊於期叛逃至燕国后,太子丹的师傅鞠武害怕秦国以此借口攻燕,便策划送樊於期到头曼那里,利用熟悉秦国虚实的樊於期结连匈奴攻秦。可惜性急的太子丹等不得这种长远之计凑效,他决定派出荆轲刺杀自己的童年好友嬴政,为了能够解除嬴政的戒备,荆轲提出要携带两样礼物:樊於期的人头和燕国督亢地图(割地求和)。嬴政在逃过刺杀威胁后更以迅雷不及掩耳之势统一六国。}。
  \item 赵将\CJKunderline{李牧}大败秦将\CJKunderline{桓齮}\footnote{桓齮(yǐ)(?-前227年),战国末年秦国将军。杨宽的《战国史》认为桓齮就是樊於期。始皇十一年(前237年),桓齮与王翦和杨端和攻赵,取邺九城。秦始皇十四年,也就是赵王迁二年(前233年),桓齮从上党越太行山进攻赵的赤丽、宜安(石家庄东南),与赵将李牧战于肥下(宜安东北),为李牧所败,逃至燕国(《战国策》说是战败被杀,《资治通鉴》记载“秦师败绩,桓齮奔还”)后无相关记载。}于肥。
  \end{enumerate} \tabularnewline\hline
  十五年 & -232 & \begin{enumerate}
    \tiny
  \item \CJKunderline{项羽}出生。
  \item 太子\CJKunderline{丹}回燕。
  \end{enumerate} \tabularnewline\hline
  十六年 & -231 & \begin{enumerate}
    \tiny
  \item 秦攻韩。
  \item 魏献丽邑。
  \item 赵国地震。
  \item \CJKunderline{韩信}出生。
  \end{enumerate} \tabularnewline\hline
  十七年 & -230 & \begin{enumerate}
    \tiny
  \item 韩国灭亡。
  \end{enumerate} \tabularnewline\hline
  十八年 & -229 & \begin{enumerate}
    \tiny
  \item 秦攻赵国。
  \item \CJKunderline{李牧}被杀。
  \end{enumerate} \tabularnewline\hline
  十九年 & -228 & \begin{enumerate}
    \tiny
  \item 秦破赵得和氏璧。
  \item 赵国灭亡。
  \end{enumerate} \tabularnewline\hline
  二十年 & -227 & \begin{enumerate}
    \tiny
  \item \CJKunderline{荆轲}刺秦王。
  \item \CJKunderline{王翦}、\CJKunderline{辛胜}在易水西败燕、代联军。
  \end{enumerate} \tabularnewline\hline
  二一年 & -226 & \begin{enumerate}
    \tiny
  \item 秦军攻燕都。
  \item 秦攻蓟城。
  \end{enumerate} \tabularnewline\hline
  二二年 & -225 & \begin{enumerate}
    \tiny
  \item 魏国灭亡。
  \item 秦置砀郡,立浚仪(大梁)、启封两县。
  \end{enumerate} \tabularnewline\hline
  二三年 & -224 & \begin{enumerate}
    \tiny
  \item 秦楚之战。
  \item 秦置修武县。
  \end{enumerate} \tabularnewline\hline
  二四年 & -223 & \begin{enumerate}
    \tiny
  \item 楚将\CJKunderline{项燕}自杀。
  \item 秦灭楚。
  \end{enumerate} \tabularnewline\hline
  二五年 & -222 & \begin{enumerate}
    \tiny
  \item 秦灭代。
  \item 秦灭燕。
  \end{enumerate} \tabularnewline
  \bottomrule
\end{longtable}

%%% Local Variables:
%%% mode: latex
%%% TeX-engine: xetex
%%% TeX-master: "../../Main"
%%% End:
