%% -*- coding: utf-8 -*-
%% Time-stamp: <Chen Wang: 2018-07-16 22:54:57>

\subsection{烈王{\tiny(BC375-BC369)}}

周烈王(?-前369年),又称周夷烈王,姓姬,名喜,中国东周君主,在位7年。他是周安王之子。周烈王在位期间,秦献公迁都栎阳(今陕西省临潼市),开启秦国强盛的序幕。周烈王五年(庚戌,前371年),秦献公发兵攻占韩国六座城市。烈王六年(前370年)齐威王朝见周天子,威王贤名更盛。

% \centering
\begin{longtable}{|>{\centering\scriptsize}m{2em}|>{\centering\scriptsize}m{1.3em}|>{\centering}m{8.8em}|}
  % \caption{秦王政}\\
  \toprule
  \SimHei \normalsize 年数 & \SimHei \scriptsize 公元 & \SimHei 大事件 \tabularnewline
  % \midrule
  \endfirsthead
  \toprule
  \SimHei \normalsize 年数 & \SimHei \scriptsize 公元 & \SimHei 大事件 \tabularnewline
  \midrule
  \endhead
  \midrule
  元年 & -375 & \begin{enumerate}
    \tiny
  \item 日有食之。
  \item 韓滅鄭,因徙都之\footnote{韓本都平陽,其地屬漢之河東郡;中間徙都陽翟。鄭都新鄭,其地屬漢之河南郡。鄭桓公始封於鄭,其地屬漢之京兆;後滅虢、鄶而國於溱、洧之間,故曰新鄭,«左傳»鄭莊公所謂「吾先君新邑於此」是也。今韓旣滅鄭,自陽翟徙都之。韓旣都鄭,故時人亦謂韓王爲鄭王,考之«戰國策»、«韓非子»可見。}。
  \item 趙敬侯薨,子成侯種立。
  \end{enumerate} \tabularnewline\hline
  二年 & -374 & \tiny \kaiti 无记载 \tabularnewline\hline
  三年 & -373 & \begin{enumerate}
    \tiny
  \item 燕敗齊師於林狐。
  \item 魯伐齊,入陽關。
  \item 魏伐齊,至博陵。
  \item 燕僖公薨,子桓公立。
  \item 宋休公薨,子辟公立。
  \item 衞愼公\footnote{«諡法»︰敏以敬曰愼。«戴記»︰思慮深遠曰愼。}薨,子聲公訓立。
  \end{enumerate} \tabularnewline\hline
  四年 & -372 & \begin{enumerate}
    \tiny
  \item 趙伐衞,取都鄙\footnote{«周禮»︰太宰以八則治都鄙。«註»云︰都之所居曰鄙。都鄙,卿大夫之采邑。蓋周之制,四縣爲都,方四十里,一千六百井,積一萬四千四百夫;五酇爲鄙,鄙五百家也。此時衞國褊小,若都鄙七十三,以成周之制率之,其地廣矣,盡衞之提封,未必能及此數也。更俟博考。}七十三。
  \item 魏敗趙師于北藺。
  \end{enumerate} \tabularnewline\hline
  五年 & -371 & \begin{enumerate}
    \tiny
  \item 魏伐楚,取魯陽。
  \item 韓嚴遂弑哀侯,國人立其子懿侯\footnote{哀侯任命韩廆当宰相,但对严遂却更亲信。韩廆跟严遂之间,结仇至深,已不可解,互相想置对方于死地。严遂雇请杀手行刺韩廆。韩廆急奔哀侯身旁,哀侯为了保护他,把他抱住。然而杀手并不停止,仍刺杀韩廆;刀锋所及,哀侯也中刃而亡。(《战国策》认为聂政杀侠累和严遂杀哀侯是一件事,《史记》认为是两件事,《资治通鉴》根据《史记》。然而,二十六年间,韩国政府发生两次重大凶案,一次杀宰相,一次除了杀宰相外,还顺手杀了国君,太过突出。所以司马光对此并不敢十分肯定,在给刘道原信中,也曾表示他的怀疑。)}。
  \item 魏武侯薨,不立太子,子罃與公中緩爭立,國內亂。
  \end{enumerate} \tabularnewline\hline
  六年 & -370 & \begin{enumerate}
    \tiny
  \item 齊威王來朝。是時周室微弱,諸侯莫朝,而齊獨朝之,天下以此益賢威王。
  \item 趙伐齊,至鄄。
  \item 魏敗趙師于懷。
  \item 齊威王奖卽墨大夫,惩阿大夫,羣臣聳懼,莫敢飾詐,務盡其情,齊國大治,強於天下\footnote{齐国国君(四任)田因齐把即墨(山东省平度市东南)城主(大夫)召到首府临淄(山东省淄博市东临淄镇),对他说:“自从命你前去即墨,我每天都接到控告你的报告。然而我派人去即墨秘密调查,发现你开荒辟田,农作物遍野,人民生活富庶,官员清廉,齐国东部,得到平安。你之所以口碑不好,我了解,是你没有巴结我左右那些当权派而已。”于是,增加他一万户人家的封邑,作为奖励。又把阿邑(山东省东阿县)城主(大夫)召到首府临淄,对他说:“自从命你前去阿邑,我几乎每天都听到对你的赞扬。可是,我派人去阿邑秘密调查,发现完全不是那么回事,那里田野荒芜,农民贫困。前些时,赵国攻击鄄城(山东省鄄城县),你不率军救援。卫国占领薛陵(山东省阳谷县东北,薛陵跟阿邑之间航空距离不到十千米),你却假装不知道。我了解,我所听到的那些捧你场的话,都是你拿钱买来的。”于是下令把阿邑城主以及平常赞扬阿邑城主的一批官员,全都用大锅烹杀。全国大为震动,官员悚然戒惧,不敢再弄玄虚,大家改变态度,认真做事。齐国大治,成为强国。}。
  \item 楚肅王薨,無子,立其弟良夫,是爲宣王。
  \item 宋辟公薨,子剔成立。
  \end{enumerate} \tabularnewline\hline \newpage
  七年 & -369 & \begin{enumerate}
    \tiny
  \item 日有食之。
  \item 王崩,弟扁(音篇)立,是爲顯王。
  \item 魏大夫王錯出奔韓,韩懿侯乃與趙成侯合兵伐魏\footnote{魏国(首府安邑【山西省夏县】)内乱(参考前三七一年),已历时三年,国务官(大夫)王错,投奔韩国(首府新郑【河南省新郑县】)。韩国国务官(大夫)公孙颀,向国君(五任懿侯)韩若山建议说:“魏国已经腐烂,亡在眉睫,我们应该把它吞并。”韩若山遂跟赵国(首府邯郸【河北省邯郸市】)国君(四任成侯)赵种结盟,联合攻击魏国,在浊泽(山西省永济县西,与安邑航空距离五十千米)会战,魏军大败,韩、赵联军遂包围魏国首府安邑(山西省夏县)。赵种主张:“杀掉魏罃,立公中缓当魏国国君,割一部分士地给我们,我们就退兵。”韩若山说:“杀掉魏罃,我们落得一个残暴的名声。割让土地,又落得一个贪心的名声。不如把魏国一分为二,分成两个国家,使他们二人都当国君。魏国一分为二之后,就跟宋国、卫国一样,成了一个小国,我们就可永远摆脱魏国的压力。”赵种不同意,韩若山大不高兴,在夜晚撤军而去。赵种人单势孤,也只好撤军而去。魏罃遂乘机袭杀他的对头公中缓,继任国君(三任)。}。
  \end{enumerate} \tabularnewline
  \bottomrule
\end{longtable}

%%% Local Variables:
%%% mode: latex
%%% TeX-engine: xetex
%%% TeX-master: "../../Main"
%%% End:
