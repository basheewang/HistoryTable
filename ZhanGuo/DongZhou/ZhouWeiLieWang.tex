%% -*- coding: utf-8 -*-
%% Time-stamp: <Chen Wang: 2018-07-13 01:01:27>

\subsection{威烈王{\tiny(BC425-BC402)}}


% \centering
\begin{longtable}{|>{\centering\scriptsize}m{2em}|>{\centering\scriptsize}m{1.3em}|>{\centering}m{8.8em}|}
  % \caption{秦王政}\\
  \toprule
  \SimHei \normalsize 年数 & \SimHei \scriptsize 公元 & \SimHei 大事件 \tabularnewline
  % \midrule
  \endfirsthead
  \toprule
  \SimHei \normalsize 年数 & \SimHei \scriptsize 公元 & \SimHei 大事件 \tabularnewline
  \midrule
  \endhead
  \midrule
  二三年 & -403 & \begin{enumerate}
    \tiny
  \item 命晉大夫魏斯\footnote{魏文侯(?-前396年),安邑(今山西夏县)人。中国战国时魏国统治者。姬姓,魏氏,名斯。周贞定王二十四年(前445年)继魏桓子位,周威烈王二年(前424年)称侯改元,威烈王二十三年(前403年)与韩、赵两家一起被周威烈王册封为诸侯,是为三家分晋,周安王六年(前396年)卒。}、趙籍\footnote{赵烈侯(?-前400年),是中国战国时期赵国的君主,原名赵籍,赵献侯之子。在位时用公仲连、牛畜、荀欣、徐越等人,为政待以仁义,约以王道。}、韓虔\footnote{韩景侯(?-前400年),名虔,韩武子之子。}爲諸侯\footnote{晋国(首府新田【山西省侯马市】)长期以来,在魏、赵、韩三大家族控制之下,国君不过空拥虚名,只在形式上,看起来晋国仍是一个完整的独立封国。本年(前四〇三年),周王国(首都洛阳【河南省洛阳市白马寺东】)国王(三十八任威烈王)姬午,下令擢升三大家族族长,亦即晋国三位国务官(大夫):魏斯当魏国(首府安邑【山西省夏县】)国君、赵籍当赵国(首府晋阳【山西省太原市】)国君、韩虔当韩国(首府平阳【山西省临汾市】)国君。晋国被三国瓜分后,只剩下一小片国土。}。
  \item 魏文侯使乐羊\footnote{乐羊,中山国人,战国时魏国的大将。是乐毅先祖。}伐中山\footnote{中山国,姬姓,春秋战国时白狄的一支——鲜虞仿照东周各诸侯国于公元前507年建立的国家,位于今河北省中部太行山东麓一带,中山国当时位于赵国和燕国之间,都于顾,后迁都于灵寿(今中国河北省灵寿县),因城中有山得国名。},克之。
  \item 吴起\footnote{吴起(前440年-前381年),中国战国初期军事家、政治家、改革家,兵家代表人物。卫国左氏(今山东省定陶县,一说山东省曹县东北)人。}杀妻以求为鲁将,大破齐师。
  \item 燕愍公\footnote{燕国(首府蓟城【北京市】)国君(三十四任)。}薨,子僖公立。
  \end{enumerate} \tabularnewline\hline
  二四年 & -402 & \tabularnewline
  \bottomrule
\end{longtable}

%%% Local Variables:
%%% mode: latex
%%% TeX-engine: xetex
%%% TeX-master: "../../Main"
%%% End:
