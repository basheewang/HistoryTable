%% -*- coding: utf-8 -*-
%% Time-stamp: <Chen Wang: 2018-07-16 15:39:17>

\subsection{元安王{\tiny(BC401-BC376)}}

周安王姬骄(?—前376年),姬姓,名骄,华夏族,周威烈王之子,威烈王死后继位,在位26年,病死。葬处不明。在位时封齐国大夫田和为齐侯,是谓“田氏代齐”。

% \centering
\begin{longtable}{|>{\centering\scriptsize}m{2em}|>{\centering\scriptsize}m{1.3em}|>{\centering}m{8.8em}|}
  % \caption{秦王政}\\
  \toprule
  \SimHei \normalsize 年数 & \SimHei \scriptsize 公元 & \SimHei 大事件 \tabularnewline
  % \midrule
  \endfirsthead
  \toprule
  \SimHei \normalsize 年数 & \SimHei \scriptsize 公元 & \SimHei 大事件 \tabularnewline
  \midrule
  \endhead
  \midrule
  元年 & -401 & \begin{enumerate}
    \tiny
  \item 秦伐魏,至陽孤\footnote{秦国(首府雍县【陕西省凤翔县】)进攻魏国(首府安邑【山西省夏县】),大军进抵阳孤(山西省垣曲县东南)。}。
  \end{enumerate} \tabularnewline\hline
  二年 & -400 & \begin{enumerate}
    \tiny
  \item 魏、韓、趙伐楚,至桑丘\footnote{魏国(首府安邑【山西省夏县】)、韩国(首府平阳【山西省临汾市】)、赵国(首府晋阳【山西省太原市】),联合攻击楚王国(首都郢都【湖北省江陵县】),大军进抵桑丘(《史记》作乘丘【山东省兖州市西北】)。}。
  \item 鄭圍韓陽翟\footnote{郑国(首府新郑【河南省新郑县】)围攻韩国所属的阳翟(河南省禹州市)。}。
  \item 韓景侯薨,子烈侯取立。
  \item 趙烈侯薨,國人立其弟武侯。
  \item 秦簡公薨,子惠\footnote{«諡法»︰愛民好與曰惠。}公立。
  \end{enumerate} \tabularnewline\hline
  三年 & -399 & \begin{enumerate}
    \tiny
  \item 王子定奔晉。
  \item 虢山崩,壅河\footnote{虢山(河南省三门峡市西)发生崩塌,土石坠入黄河,河水壅塞。}。
  \end{enumerate} \tabularnewline\hline
  四年 & -398 & \begin{enumerate}
    \tiny
  \item 楚圍鄭。鄭人殺其相駟子陽\footnote{郑国十一任国君穆公姬兰的儿子姬腓,别名子驷。古人往往用祖父的名字最后一个字作自己这一支派的姓。这位驷子阳,姓驷,名子阳,也是郑国贵族。}。
  \end{enumerate} \tabularnewline\hline
  五年 & -397 & \begin{enumerate}
    \tiny
  \item 日有食之。
  \item 三月,盜殺韓相俠累\footnote{侠累跟濮阳(河南省濮阳市)人严仲子之间,有难解的怨毒,严仲子听说轵邑(河南省济源市东南)人聂政,勇猛过人,备了黄金二千四百两(百镒),送给聂政的母亲,作为祝寿礼物,请聂政代他报仇。聂政拒绝,说:“娘亲在堂,要我奉养,我不能轻言牺牲。”稍后,娘亲逝世,聂政才接受这项委托。当暗杀行动开始时,侠累正在宰相府主持会报,警卫森严。聂政像闪电一样,突击而入,在众人惊愕中,举刀直刺侠累的咽喉,侠累立即死亡。聂政自知难以逃生,咬紧牙关,用利刃自行毁容,脸皮全被割破,又自挖双眼,再自刺腹部自杀,肠出满地。韩国政府把尸首拖到市场,公开示众,要求市人辨识刺客身份。聂政的姐姐聂荌听到消息,赶到首府平阳(山西省临汾市),抚尸哀哭说:“他就是轵邑深井里(济通市东南十五千米)的聂政,只因为我这个姐姐尚在人间,恐怕连累我,才忍心重重的自我毁灭。弟弟啊,我怎么会贪生怕死,使你埋没英名?”就在尸旁,自杀殉难。}。
  \end{enumerate} \tabularnewline\hline
  六年 & -396 & \begin{enumerate}
    \tiny
  \item 鄭駟子陽之黨弑繻公\footnote{繻者,«諡法»所不載。},而立其弟乙,是爲康公\footnote{郑国(首府新郑【河南省新郑县】)故宰相(相)驷子阳的残余党羽,击杀国君(二十七任)繻公姬贻,拥立他的弟弟姬乙继位(二十八任),是为康公。}。
  \item 宋悼公薨,子休公田立\footnote{宋国(首府睢阳【河南省商丘县】)国君(三十一任悼公)宋购由逝世,子宋田继位(三十二任),是为休公。武王封微子啓於宋,唐宋州之睢陽縣是也。自微子二十七世至悼公,名購由。休,亦«諡法»所不載。}。
  \end{enumerate} \tabularnewline\hline
  七年 & -395 & \tabularnewline\hline
  八年 & -394 & \begin{enumerate}
    \tiny
  \item 齊\footnote{武王封太公於齊,唐青州之臨淄是也。«括地志»曰︰天齊水在臨淄東南十五里。«封禪書»曰︰齊之所以爲齊者,以天齊。是年,康公貸之十一年。自太公至康公二十九世。}伐魯\footnote{成王封伯禽於魯,唐兗州之曲阜是也。是年,穆公之十六年。自伯禽至穆公凡二十八世。},取最\footnote{山东省曲阜市东南}。
  \item 鄭負黍\footnote{負黍山在陽城縣西南二十七里,或云在西南三十五里。}叛,復歸韓\footnote{前四〇七年,郑国攻击韩国,占领负黍城。}。
  \end{enumerate} \tabularnewline\hline
  九年 & -393 & \tabularnewline\hline
  十年 & -392 & \tabularnewline\hline
  十一年 & -391 & \tabularnewline\hline
  十二年 & -390 & \tabularnewline\hline
  十三年 & -389 & \tabularnewline\hline
  十四年 & -388 & \tabularnewline\hline
  十五年 & -387 & \tabularnewline\hline
  十六年 & -386 & \tabularnewline\hline
  十七年 & -385 & \tabularnewline\hline
  十八年 & -384 & \tabularnewline\hline
  十九年 & -383 & \tabularnewline\hline
  二十年 & -382 & \tabularnewline\hline
  二一年 & -381 & \tabularnewline\hline
  二二年 & -380 & \tabularnewline\hline
  二三年 & -379 & \tabularnewline\hline
  二四年 & -378 & \tabularnewline\hline
  二五年 & -377 & \tabularnewline\hline
  二六年 & -376 & \tabularnewline
  \bottomrule
\end{longtable}

%%% Local Variables:
%%% mode: latex
%%% TeX-engine: xetex
%%% TeX-master: "../../Main"
%%% End:
