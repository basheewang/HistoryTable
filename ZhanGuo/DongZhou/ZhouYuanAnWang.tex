%% -*- coding: utf-8 -*-
%% Time-stamp: <Chen Wang: 2018-07-16 22:51:28>

\subsection{元安王{\tiny(BC401-BC376)}}

周安王姬骄(?—前376年),姬姓,名骄,华夏族,周威烈王之子,威烈王死后继位,在位26年,病死。葬处不明。在位时封齐国大夫田和为齐侯,是谓“田氏代齐”。

% \centering
\begin{longtable}{|>{\centering\scriptsize}m{2em}|>{\centering\scriptsize}m{1.3em}|>{\centering}m{8.8em}|}
  % \caption{秦王政}\\
  \toprule
  \SimHei \normalsize 年数 & \SimHei \scriptsize 公元 & \SimHei 大事件 \tabularnewline
  % \midrule
  \endfirsthead
  \toprule
  \SimHei \normalsize 年数 & \SimHei \scriptsize 公元 & \SimHei 大事件 \tabularnewline
  \midrule
  \endhead
  \midrule
  元年 & -401 & \begin{enumerate}
    \tiny
  \item 秦伐魏,至陽孤\footnote{秦国(首府雍县【陕西省凤翔县】)进攻魏国(首府安邑【山西省夏县】),大军进抵阳孤(山西省垣曲县东南)。}。
  \end{enumerate} \tabularnewline\hline
  二年 & -400 & \begin{enumerate}
    \tiny
  \item 魏、韓、趙伐楚,至桑丘\footnote{魏国(首府安邑【山西省夏县】)、韩国(首府平阳【山西省临汾市】)、赵国(首府晋阳【山西省太原市】),联合攻击楚王国(首都郢都【湖北省江陵县】),大军进抵桑丘(《史记》作乘丘【山东省兖州市西北】)。}。
  \item 鄭圍韓陽翟\footnote{郑国(首府新郑【河南省新郑县】)围攻韩国所属的阳翟(河南省禹州市)。}。
  \item 韓景侯薨,子烈侯取立。
  \item 趙烈侯薨,國人立其弟武侯。
  \item 秦簡公薨,子惠\footnote{«諡法»︰愛民好與曰惠。}公立。
  \end{enumerate} \tabularnewline\hline
  三年 & -399 & \begin{enumerate}
    \tiny
  \item 王子定奔晉。
  \item 虢山崩,壅河\footnote{虢山(河南省三门峡市西)发生崩塌,土石坠入黄河,河水壅塞。}。
  \end{enumerate} \tabularnewline\hline
  四年 & -398 & \begin{enumerate}
    \tiny
  \item 楚圍鄭。鄭人殺其相駟子陽\footnote{郑国十一任国君穆公姬兰的儿子姬腓,别名子驷。古人往往用祖父的名字最后一个字作自己这一支派的姓。这位驷子阳,姓驷,名子阳,也是郑国贵族。}。
  \end{enumerate} \tabularnewline\hline
  五年 & -397 & \begin{enumerate}
    \tiny
  \item 日有食之。
  \item 三月,盜殺韓相俠累\footnote{侠累跟濮阳(河南省濮阳市)人严仲子之间,有难解的怨毒,严仲子听说轵邑(河南省济源市东南)人聂政,勇猛过人,备了黄金二千四百两(百镒),送给聂政的母亲,作为祝寿礼物,请聂政代他报仇。聂政拒绝,说:“娘亲在堂,要我奉养,我不能轻言牺牲。”稍后,娘亲逝世,聂政才接受这项委托。当暗杀行动开始时,侠累正在宰相府主持会报,警卫森严。聂政像闪电一样,突击而入,在众人惊愕中,举刀直刺侠累的咽喉,侠累立即死亡。聂政自知难以逃生,咬紧牙关,用利刃自行毁容,脸皮全被割破,又自挖双眼,再自刺腹部自杀,肠出满地。韩国政府把尸首拖到市场,公开示众,要求市人辨识刺客身份。聂政的姐姐聂荌听到消息,赶到首府平阳(山西省临汾市),抚尸哀哭说:“他就是轵邑深井里(济通市东南十五千米)的聂政,只因为我这个姐姐尚在人间,恐怕连累我,才忍心重重的自我毁灭。弟弟啊,我怎么会贪生怕死,使你埋没英名?”就在尸旁,自杀殉难。}。
  \end{enumerate} \tabularnewline\hline
  六年 & -396 & \begin{enumerate}
    \tiny
  \item 鄭駟子陽之黨弑繻公\footnote{繻者,«諡法»所不載。},而立其弟乙,是爲康公\footnote{郑国(首府新郑【河南省新郑县】)故宰相(相)驷子阳的残余党羽,击杀国君(二十七任)繻公姬贻,拥立他的弟弟姬乙继位(二十八任),是为康公。}。
  \item 宋悼公薨,子休公田立\footnote{宋国(首府睢阳【河南省商丘县】)国君(三十一任悼公)宋购由逝世,子宋田继位(三十二任),是为休公。武王封微子啓於宋,唐宋州之睢陽縣是也。自微子二十七世至悼公,名購由。休,亦«諡法»所不載。}。
  \end{enumerate} \tabularnewline\hline
  七年 & -395 & \tiny \kaiti 无记载 \tabularnewline\hline
  八年 & -394 & \begin{enumerate}
    \tiny
  \item 齊\footnote{武王封太公於齊,唐青州之臨淄是也。«括地志»曰︰天齊水在臨淄東南十五里。«封禪書»曰︰齊之所以爲齊者,以天齊。是年,康公貸之十一年。自太公至康公二十九世。}伐魯\footnote{成王封伯禽於魯,唐兗州之曲阜是也。是年,穆公之十六年。自伯禽至穆公凡二十八世。},取最\footnote{山东省曲阜市东南}。
  \item 鄭負黍\footnote{負黍山在陽城縣西南二十七里,或云在西南三十五里。}叛,復歸韓\footnote{前四〇七年,郑国攻击韩国,占领负黍城。}。
  \end{enumerate} \tabularnewline\hline
  九年 & -393 & \begin{enumerate}
    \tiny
  \item 魏伐鄭。
  \item 晉烈公\footnote{周成王封弟叔虞於唐。«括地志»曰︰故唐城在幷州晉陽縣北二里,堯所築也。«都城記»曰︰唐叔虞之子燮父徙居晉水旁,今幷州理故唐城,卽燮父初徙之處;其城南半入州城中。«毛詩譜»曰︰燮父以堯墟南有晉水,改曰晉侯。自唐叔至烈公三十七世。烈公,名止。«諡法»︰慈惠愛親曰孝。}薨,子孝公傾立。
  \end{enumerate} \tabularnewline\hline
  十年 & -392 & \tiny \kaiti 无记载\tabularnewline\hline
  十一年 & -391 & \begin{enumerate}
    \tiny
  \item 秦伐韓宜陽,取六邑\footnote{班«志»,宜陽縣屬弘農郡。«史記正義»曰︰宜陽縣故城,在河南府福昌縣東十四里,故韓城是也。此邑卽«周禮»「四井爲邑」之邑。}。
  \item 齊田和\footnote{田常生襄子盤,盤生莊子白,白生太公和。此序齊田氏之世也。田常,卽«左傳»陳成子恆也。溫公避仁廟諱,改「恆」曰「常」。自陳公子完奔齊,五世至常得政。«諡法»︰勝敵志強曰莊。}遷齊康公於海上,使食一城,以奉其先祀。
  \end{enumerate} \tabularnewline\hline
  十二年 & -390 & \begin{enumerate}
    \tiny
  \item 秦、晉戰于武城\footnote{晋国【首府新田】自被瓜分后,连本身生存都有问题,已无力作任何战争。可能是和魏国【首府安邑·山西省夏县】,或韩国【首府平阳·山西省临汾市】会战。}。
  \item 齊伐魏,取襄陽。
  \item 魯敗齊師于平陸。
  \end{enumerate} \tabularnewline\hline
  十三年 & -389 & \begin{enumerate}
    \tiny
  \item 秦侵晉。
  \item 齊田和會\footnote{孔穎達曰︰諸侯未及期而相見曰遇。會者,謂及期之禮,旣及期,又至所期之地。}魏文侯、楚人、衞人于濁澤,求爲諸侯。魏文侯爲之請於王及諸侯,王許之。
  \end{enumerate} \tabularnewline\hline
  十四年 & -388 & \begin{enumerate}
    \tiny
  \item 齊田和逝世,子田剡继位。
  \end{enumerate} \tabularnewline\hline
  十五年 & -387 & \begin{enumerate}
    \tiny
  \item 秦伐蜀\footnote{«譜記»普[疑衍]云︰蜀之先,肇自人皇之際。黃帝子昌意娶蜀山氏女,生帝俈。旣立,封其支庶於蜀,歷虞、夏、商、周。周衰,先稱王者蠶叢。余據武王伐紂,庸、蜀諸國皆會于牧野。孔安國曰︰蜀,叟也,春秋之時不與中國通。班«志»,南鄭縣屬漢中郡,唐爲梁州治所。},取南鄭。
  \item 魏文侯薨,太子擊立,是爲武侯。魏置相,相田文\footnote{魏击任命田文担任宰相。吴起不高兴,对田文说:“我想跟你讨论一下你我对于国家的贡献,你以为如何?”田文说:“当然可以。”吴起说:“指挥武装部队,官兵们愿意牺牲性命,使敌国惊惧,不敢打我们的主意,你比我怎么样?”田文说:“我不如你。”吴起说:“使政府的功能充分发挥,使全国人民安居乐业、国库充实、社会富庶,你比我怎么样?”田文说:“我不如你。”吴起说:“防卫西河(潼关以北的黄河),秦国不敢向东侵略。而韩国(首府平阳【山西省临汾市】)与赵国(首府晋阳【山西省太原市】),不敢不对我们唯命是听,你比我怎么样?”田文说:“我不如你。”吴起说:“这三项重要大事,你都不如我,可是官位却比我高,那为什么?”田文说:“当君王年纪还小,有权势的重要官员互相猜忌,随时可能发动政变,民心恐慌。这个时候,宰相位置,应该属于你?还是属于我?”吴起沉默很久,抱歉说:“我承认,应该属于你。”}。
  \item 秦惠公薨,子出公\footnote{出,非諡也;以其失國出死,故曰出公。}立。
  \item 趙武侯薨,國人復立烈侯之太子章,是爲敬侯\footnote{«諡法»︰夙夜警戒曰敬。}。
  \item 韓烈侯薨,子文侯立。
  \end{enumerate} \tabularnewline\hline
  十六年 & -386 & \begin{enumerate}
    \tiny
  \item 趙公子朝作亂,奔魏;與魏襲邯鄲,不克\footnote{本年【前三八六年】,赵国首府自晋阳迁邯郸,赵朝当是利用迁府之际,发动政变。}。
  \end{enumerate} \tabularnewline\hline
  十七年 & -385 & \begin{enumerate}
    \tiny
  \item 秦庶長\footnote{後秦制爵,一級曰公士,二上造,三簪裊,四不更,五大夫,六官大夫,七公大夫,八公乘,九五大夫,十左庶長,十一右庶長,十二左更,十三中更,十四右更,十五少上造,十六大上造,十七駟車庶長,十八大庶長,十九關內侯,二十徹侯。師古曰︰庶長,言衆列之長。}改逆獻公\footnote{威烈王十一年秦靈公卒,子獻公師隰不得立,立靈公季父悼子,是爲簡公。出子,簡公之孫也。今庶長改迎獻公而殺出子。}于河西而立之;殺出子及其母,沈之淵旁。
  \item 齐伐魯。
  \item 韓伐鄭,取陽城;伐宋,執宋公。
  \end{enumerate} \tabularnewline\hline
  十八年 & -384 & \tiny \kaiti 无记载 \tabularnewline\hline
  十九年 & -383 & \begin{enumerate}
    \tiny
  \item 魏敗趙師于兔臺。
  \end{enumerate} \tabularnewline\hline
  二十年 & -382 & \begin{enumerate}
    \tiny
  \item 日有食之,旣\footnote{旣,盡也}。
  \end{enumerate} \tabularnewline\hline
  二一年 & -381 & \begin{enumerate}
    \tiny
  \item 楚悼王薨。貴戚大臣作亂,攻吳起;起走之王尸而伏之。擊起之徒因射刺起,並中王尸。旣葬,肅\footnote{«諡法»︰剛德克就曰肅;執心決斷曰肅。}王卽位,使令尹盡誅爲亂者;坐起夷宗者七十餘家。
  \end{enumerate} \tabularnewline\hline
  二二年 & -380 & \begin{enumerate}
    \tiny
  \item 齊伐燕,取桑丘。
  \item 魏、韓、趙伐齊,至桑丘。
  \end{enumerate} \tabularnewline\hline
  二三年 & -379 & \begin{enumerate}
    \tiny
  \item 趙襲衞\footnote{成王封康叔於衞,居河、淇之間,故殷墟也。至懿公爲狄所滅,東徙度河。文公徙居楚丘,遂國於濮陽。是年,愼公頹之三十五年。自康叔至愼公凡三十二世。},不克。
  \item 齊康公薨,無子,田氏遂幷齊而有之。姜氏至此滅矣。
  \end{enumerate} \tabularnewline\hline
  二四年 & -378 & \begin{enumerate}
    \tiny
  \item 狄\footnote{漢之中山、上黨、西河、上郡,自春秋以來,狄皆居之,此亦其種也。«水經»︰澮水出河東絳縣東澮山,西過絳縣南,又西南過虒祁宮南,又西南至王橋,入汾水。«括地志»︰澮山在絳州翼城縣東北。}敗魏師于澮。
  \item 魏、韓、趙伐齊,至靈丘。
  \item 晉孝公薨,子靖公\footnote{«諡法»︰柔衆安民曰靖;又,恭己鮮言曰靖。}俱酒立。
  \item 齐国(首府临淄)国君(二任)田剡逝世,子田午继位(三任),是为桓公。
  \end{enumerate} \tabularnewline\hline
  二五年 & -377 & \begin{enumerate}
    \tiny
  \item 蜀伐楚,取茲方(四川省奉节县)。
  \item 子思论卫\footnote{卫国(首府濮阳【河南省濮阳市】),孔伋(子思)向卫国国君(四十一任慎公)卫颓,推荐苟变,说:“他的才干可以指挥五百辆战车作战。”卫颓说:“我知道他的军事才能,但苟变曾经当过税务员,有次平白吃了民家两个鸡蛋,品德上有瑕疵。”孔伋说:“政府任用官吏,跟建筑师选择木材一样,取其所长,弃其所短。巨木高耸云际,几个人都合抱不住,却有几尺朽烂,优秀的建筑师不会不用它。现在,我们正处在大混战时代,应该积极物色英雄豪杰,却为了两个鸡蛋,丧失一员大将,这话可别让别国听见才好。”卫颓再三致谢说:“我接受你的指教。”卫颓做了一项错误的决定,全体官员却一致赞扬那决定非常正确。孔伋对公丘懿子说:“我看你们卫国,真是君不像君,臣不像臣。”(“君不君,臣不臣”,《论语》引齐国【首府临淄·山东省淄博市东临淄镇】国君【二十六任景公】姜杵臼的话。)公丘懿子说:“怎么会糟到这种程度?”孔伋说:“领袖人物经常的自以为是,大家就不敢贡献自己的意见。做对了而自以为是,还会排斥众人的智慧。何况做错了而仍自以为是,硬教大家赞扬,那简直是鼓励邪恶。不问事情的是非,而只一味喜欢听悦耳的声音,可以说绝顶糊涂。不管那是不是合理,而只努力露出忠贞嘴脸,满口顺调,那就是马屁精。君主昏庸、官员谄媚,而高高坐在人民头上,人民绝对不会认同。如果一直这样下去,国家必亡。”孔伋告诉卫颓说:“你的国家,恐怕将要没落了。”卫颓说:“什么原因?”孔伋说:“当然有原因,领袖说一句话,自以为是,官员们没有一个人敢指出他的错误;官员们说一句话,自以为是,民间没有一个人敢指出他的错误。领袖和官员,都自以为英明盖世,属下的小官小民也同声赞扬他们果然是真的英明盖世。马屁精就有福了,指出君王错误的人一定大祸临久。如此这般,有益于国家的善政,怎能产生?《诗经》说:‘都说自己是圣贤,谁分辨乌鸦的雌雄?’听起来好像就是指的你们。”}。
  \item 魯穆公薨,子共公奮立\footnote{«諡法»︰布德就義曰穆;中情見貌曰穆;尊賢敬讓曰共;旣過能改曰共;執事堅固曰共。}。
  \item 韓文侯薨,子哀侯立。
  \end{enumerate} \tabularnewline\hline
  二六年 & -376 & \begin{enumerate}
    \tiny
  \item 王崩,子烈王喜立。
  \item 魏、韓、趙共廢晉靖公爲家人而分其地。唐叔不祀矣。
  \end{enumerate} \tabularnewline
  \bottomrule
\end{longtable}

%%% Local Variables:
%%% mode: latex
%%% TeX-engine: xetex
%%% TeX-master: "../../Main"
%%% End:
