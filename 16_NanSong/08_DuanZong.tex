%% -*- coding: utf-8 -*-
%% Time-stamp: <Chen Wang: 2019-12-26 10:51:22>

\section{端宗\tiny(1276-1278)}

\subsection{生平}

宋端宗赵昰(1269年7月10日-1278年5月8日),南宋第八位皇帝(1276年6月14日—1278年5月8日在位),在位3年,得年10岁,廟號端宗,諡號「裕文昭武愍孝皇帝」或「孝恭仁裕懿聖濬文英武勤政皇帝」,又有史稱宋帝昰。他是宋度宗的五子,宋恭帝兄,曾被封为建国公、吉王、益王等。

宋恭帝德祐二年正月十八日(1276年2月4日)元军攻克临安时,宋恭帝和謝太皇太后相繼被俘。赵昰和母亲杨淑妃和弟弟赵昺由国舅杨亮节等护卫,出逃福建,定行都於福州濂浦平山福地,改年號景炎,行宮為平山閣(當時时值战乱,哀鸿遍野,宋军撤离此地时,曾开仓济民,当地人民甚感其恩,元军佔领福州时,当地人民遂将平山阁改名为泰山宫,祭祀南宋高宗赵构及入闽的益、广二王。左右列的是文臣武将:文天祥、陆秀夫、陈宜中、张世杰。当地泰山宫便塑这些神像,实是回避元代的查禁,以泰山宫作掩护,泰山宫現存完好)。

赵昰登基前被封为「天下兵马都元帅」。1276年6月14日即位,改元景炎,时年只有7岁。虽然朝臣陆秀夫等坚持抗元,力图恢复宋朝,但在元军的紧紧追击下,端宗只能由大将张世杰护卫登船入海,东逃西避,疲于奔命。他曾逃到南澳岛上,在岛上海滩上开挖的宋井至今仍存,之后又逃到香港的九龙城一帶,現存的宋王臺和侯王廟都是为纪念宋端宗而建。

景炎三年(1278年)3月,端宗在元将刘深追逐下避入广州對開海面,不料座船倾覆,端宗遇溺被左右救起,因此染病。因元军追兵逼近,又不得不浮海逃往碙洲(今香港大屿山)。不到10岁的小皇帝屡受颠簸,又惊病交加,于几个月后(1278年5月8日)在碙洲去世,葬于永福陵(今香港大嶼山東涌黃龍坑)。

根據宋王臺花園《九龙宋皇台遗址碑记》记载,昰、昺二帝南逃期間,「有金夫人墓,相传为杨太后女,晋国公主,先溺于水,至是铸金身以葬者」,葬于今九龙城区,人称金夫人墓,后来在该址兴建了圣三一堂,金夫人墓隨之湮沒。另一方面,碑記也記載端宗曾經於九龍城白鶴山行朝,及以一塊大石為御座,後人稱此為「交椅石」,惟該石現在已經不知所終。

元朝官修正史《宋史》脱脱等的評價是:“宋之亡征,已非一日。历数有归,真主御世,而宋之遗臣,区区奉二王为海上之谋,可谓不知天命也已。然人臣忠于所事而至于斯,其亦可悲也夫!”


\subsection{景炎}


\begin{longtable}{|>{\centering\scriptsize}m{2em}|>{\centering\scriptsize}m{1.3em}|>{\centering}m{8.8em}|}
  % \caption{秦王政}\
  \toprule
  \SimHei \normalsize 年数 & \SimHei \scriptsize 公元 & \SimHei 大事件 \tabularnewline
  % \midrule
  \endfirsthead
  \toprule
  \SimHei \normalsize 年数 & \SimHei \scriptsize 公元 & \SimHei 大事件 \tabularnewline
  \midrule
  \endhead
  \midrule
  元年 & 1276 & \tabularnewline\hline
  二年 & 1277 & \tabularnewline\hline
  三年 & 1278 & \tabularnewline
  \bottomrule
\end{longtable}



%%% Local Variables:
%%% mode: latex
%%% TeX-engine: xetex
%%% TeX-master: "../Main"
%%% End:
