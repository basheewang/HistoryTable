%% -*- coding: utf-8 -*-
%% Time-stamp: <Chen Wang: 2021-11-01 15:56:52>

\section{恭帝趙㬎\tiny(1274-1276)}

\subsection{生平}

宋恭帝趙㬎(1271年11月2日-1323年5月31日),南宋第七代皇帝(1274年8月12日—1276年2月4日在位),在位2年,得年53岁,宋度宗六子。他是全皇后所生,是宋端宗赵昰之弟,宋末帝赵昺兄,即位前封嘉国公、左卫上将军等,宋端宗为兄弟上尊号孝恭懿圣皇帝,无庙号。1276年2月領宋室投降元朝后,封为「瀛國公」,之後又被迫剃髮出家,最後被賜死。

《宋史本纪第四十七》記載:“瀛国公名㬎,度宗皇帝子也,母曰全皇后,咸淳七年(1271年)九月己丑,生于临安府之大内。”宋咸淳十年(1274年)七月,宋度宗驾崩,年三岁的趙㬎在丞相贾似道的扶持下登基做皇帝,是为宋恭帝,改明年為德祐元年。祖母謝太皇太后、母亲全太后垂帘听政。但军国大权依然在贾似道之手。

當時元军已控制中國北方和西南,在取得南下最重要通道襄陽城的控制權之後,渡过长江向南宋都城临安(今杭州)进发。謝太皇太后一面在全国通令“勤王”,一面向元軍乞和。勢如破竹的元軍在擊破各地的防線,相繼降服了長江中游諸州。1275年,賈似道率領的13萬大軍在蕪湖與元軍對戰大敗。不久,謝太皇太后和宋恭帝在輿論壓力下貶贾似道,不过为时已晚,宋朝亡國在即。同年年中,元軍已經佔領了江東(今日的江蘇省南部)大半的領土。

1276年1月18日伯顏率領的元军兵临临安。南宋朝廷遣陸秀夫求和稱侄不成,只好向元军投降。正月十八日(1276年2月4日),謝太皇太后抱着五岁的小皇帝宋恭帝㬎出趙城,派遣监察御史杨应奎向元軍献上传国玺投降,南宋滅亡。南宋残余势力在福建、广东抗元。

南宋滅亡後,宋恭帝曾徙居元大都、上都、烏斯藏、甘州(一說還有謙州,今俄罗斯图瓦共和国境内)等地,是中國歷史上遊歷最遠的一位漢人皇帝。

恭帝降元後,元將巴延促其北上入覲。帝於至元十三年(宋德祐二年,1276年)丙子三月從臨安啓程,前往上都。太皇太后謝氏以疾留内。太后全氏、隆國夫人黃氏(度宗母、恭帝祖母)、榮王赵与芮(理宗弟、度宗父、恭帝祖父)、沂王趙乃猷、樞密院參知政事高應松、謝堂、知臨安府翁仲德及汪元量等朝臣、宮人隨同北上(見劉一清錢塘遺事)。渡江後,宋將李庭芝、苗再成等謀奪駕,不克。五月,過大都,赴上都。丙申,見元世祖忽必烈於上都大安殿。忽必烈封恭帝为瀛國公,妻以公主,詔優待之,使居大都;福王趙與芮受封平原郡公(汪元量《水雲集》湖州歌八十一:“福王又拜平原郡,幼主新封瀛國公”)。

1279年3月19日,陸秀夫擕年仅八岁的小皇帝赵昺在崖山蹈海自盡,南宋最終全面灭亡。

至元十九年(1282年),中書省奏請徙瀛國公居上都,詔許之。後元仁宗延祐中,隨高麗國王王璋入朝的高麗人權漢功,見瀛國公故宅尚存,作《瀛國公第盆梅》詠之。

忽必烈欲保全亡宋宗室。至元二十五年(1288年)十月詔遣瀛國公趙㬎入吐蕃習梵書、西蕃字經(一說瀛國公自求入吐蕃學佛法)。十二月啓程,由脫思麻(今青海省海南藏族自治州一帶)入烏思藏,駐錫薩斯迦大寺,號木波講師。[來源請求]他在萨迦寺出家,取藏文法名“却季仁钦”(ཆོས་ཀྱི་རིན་ཆེན་)。藏族人尊称他为“蛮子拉尊”(སྨན་རྩེ་ལྷ་བཙུན་);“蛮子”是蒙古人对宋人的称谓,“拉尊”是藏语对出家王族的尊称,汉译合尊。後為薩斯迦大寺住持。嘗取漢藏佛經互譯比勘,校訂異文。

元英宗至治三年(1323年)四月(據釋念常《佛祖歷代通載》),賜瀛國公趙㬎死於河西(今甘肃河西走廊张掖)。明初僧人釋無慍《山庵雜錄》云:“瀛國公為僧後,至英宗朝,適興吟詩云:「寄語林和靖,梅開几度花。黄金臺上客,無復得還家。」諜者以其意在諷動江南人心,聞之於上,收斬之。既而上悔,出內帑黃金,詔江南善書僧儒集燕京,書大藏经云。”案:陶宗儀《輟耕錄》引此詩,作:“寄語林和靖。梅花幾度開。黃金臺下客,應是不歸來。”並云“此宋幼主在京師所作也”。明人瞿佑《歸田詩話》引作:“黃金臺上客,底事又思家。歸問林和靖,寒梅幾度花。”謂瀛國公以此詩贈汪元量。藏文史書則謂其罪名是以讲经为名,聚衆謀反。元史英宗紀可旁證瀛國公卒年:“至治三年四月壬戌朔,敕天下諸司命僧誦經十萬部。……敕京師萬安、慶壽、聖安、普慶四寺,揚子江金山寺、五台山萬聖祐國寺,做水陸佛事七晝夜。”按《元史》卷四十二顺帝纪五,至正十二年五月,“河南诸处群盗,辄引亡宋故号以为口实,宜以瀛国公子和尚赵完普及亲属徙沙州安置,禁勿与人交通。”可见,赵完普并未被赐死,且有一定数量的家眷。

後世盛傳宋恭帝為元惠宗妥懽帖睦爾之生父。元文宗曾佈告中外,引元惠宗乳母夫之言,稱元明宗在漠北時,素謂太子(妥懽帖睦爾)非己子,遂徙於高麗,後遷靜江。元末明初人權衡撰《庚申外史》,謂瀛國公駐錫甘州山寺(元時稱十字寺,即張掖大佛寺)時,封地位於汪古部舊地及居延一帶的趙王曾以一回回女子與之(即順帝生母邁來迪)。延祐七年四月,回回女生一男子。時值元武宗長子周王和世琜(即位後為元明宗)流亡西北,過甘州山寺,見瀛國公幼子,“大喜,因求為子,並其母載以歸”。明代以後,此說遂成確論。至清代,《欽定四庫全書總目提要》認為此說乃宋遺民僞造,明人“附會而盛傳之”,“覈以事實,渺無可據,實為荒誕之尤,非信史也。”近時學者有謂瀛國公在移駐甘州之前,可能居於謙州吉利吉思地界(今葉尼塞河上游)。當時周王和世琜自陝西至嶺北過金山(阿尔泰山),流亡於察合臺後王封地,地理上與謙州接近。

“司马迁论秦、赵世系同出伯益。夫稷、契、伯益其子孙皆有天下,至于运祚短长,亦系其功德之厚薄焉。赵宋虽起于用武,功成治定之后,以仁传家,视秦宜有间矣。然仁之敝失于弱,即文之敝失于僿也。中世有欲自强,以革其敝,用乖其方,驯致棼扰。建炎而后,土宇分裂,犹能六主百五十年而后亡,岂非礼义足以维持君子之志,恩惠足以固结黎庶之心欤?瀛国四岁即位,而天兵渡江,六岁而群臣奉之入朝。汉刘向言:“孔子论《诗》至‘殷士肤敏,裸将于京。’喟然叹曰:大哉天命,善不可不传于后嗣,是以富贵无常。”至哉言乎!我皇元之平宋也,吴越之民,市不易肆。世祖皇帝命征南之帅,辄以宋祖戒曹彬勿杀之言训之。《书》曰:“大哉王言,一哉王心。”我元一天下之本,其在于兹。”—— 元朝官修正史《宋史》脱脱等的評價


\subsection{德祐}


\begin{longtable}{|>{\centering\scriptsize}m{2em}|>{\centering\scriptsize}m{1.3em}|>{\centering}m{8.8em}|}
  % \caption{秦王政}\
  \toprule
  \SimHei \normalsize 年数 & \SimHei \scriptsize 公元 & \SimHei 大事件 \tabularnewline
  % \midrule
  \endfirsthead
  \toprule
  \SimHei \normalsize 年数 & \SimHei \scriptsize 公元 & \SimHei 大事件 \tabularnewline
  \midrule
  \endhead
  \midrule
  元年 & 1275 & \tabularnewline\hline
  二年 & 1276 & \tabularnewline
  \bottomrule
\end{longtable}



%%% Local Variables:
%%% mode: latex
%%% TeX-engine: xetex
%%% TeX-master: "../Main"
%%% End:
