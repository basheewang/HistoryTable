%% -*- coding: utf-8 -*-
%% Time-stamp: <Chen Wang: 2021-11-01 15:59:23>

\section{高宗趙構\tiny(1127-1162)}

\subsection{生平}

宋高宗趙構(1107年6月12日-1187年11月9日),字德基,宋朝第十位皇帝、南宋第一代皇帝(1127年6月12日-1162年7月24日在位),在位35年。北宋皇帝宋徽宗第九子,宋欽宗之弟;曾獲封為「康王」。

在位初期因為眼見金朝強勢,為了保持江山,起用主戰派李綱、岳飛等等。但恐懼將领权力过大,為了強化中央集權,採取求和政策,終於1141年(紹興十一年)達成紹興和議,重用主和派黃潛善、汪伯彥、王倫、秦檜等人,並處死岳飛,罷免李綱、張浚、韓世忠等主戰派大臣。雖然宋高宗之稱臣決策導致南宋偏安之局面,卻成功鞏固了南宋在中國南方的統治,並與金朝形成南北對峙之局面。

大觀元年四月二十七日(1107年5月21日)宋高宗趙構於汴京出生,是宋徽宗第九子,徽宗時被封為康王。

靖康元年春(1126年),金兵圍困汴京,並要求宋以親王、宰相各一人為人質,才肯與宋和談,宋欽宗派趙構以親王身份在金營中為人質,後因金人懷疑其宗室身份,要求更換,故得以回宋。正當趙構獲釋返汴京途中,金兵再次南侵,最初宋欽宗命他往河北召集兵馬勤王,後來金人發現趙構原來是真正親王,忿怒不已,要求宋朝安排趙構為使,才肯再議和,欽宗於是改派他出使金營求和。趙構前往金營時途經河北磁州(今屬河北),被守將宗澤勸阻留下,得以免遭金兵俘虜。此时金兵已跟踪到康王所在,知相州汪伯彦请康王入相州。

靖康元年闰十一月,欽宗命康王为河北兵马大元帅。閏十一月丙辰日(1127年1月9日),金兵攻破汴京開封府,造成「靖康之難」。北宋灭亡。十二月初一壬戌日(1127年1月15日)康王趙構在河北相州開河北兵馬大元帥幕府。趙構自己為河北兵馬大元帥,陳亨伯為元帥,汪伯彥、宗澤為副元帥。有兵万人,分为五军南下。渡河,次大名府。宗泽请直取汴梁。康王从耿南仲及伯彦意见,欲移军东平。十二月乙亥,康王到魏博,庚寅至东平府。

靖康二年二月庚辰,康王如济州。时兵已八万。黄潜善时归之。四月庚辰,康王发济州,趣應天(今河南商丘),刘光世以所部来会。癸未至南京。

此間有過一段插曲:趙構在磁州時,曾由宗澤陪同拜謁了城北崔府君廟,當地稱之為“應王祠”。該廟位於通往邢、洺州的驛道側旁,當時此處“民如山擁”,眾多百姓因為擔心康王取道繼續北行,而聚集在廟宇周圍,號呼勸諫。進入祠廟後,康王抽籤詩,卜得“吉”之籤,廟吏抬應王神輿、擁廟中神馬,請康王乘歸館舍。紛亂中,力主使金的王雲被殺,趙構則留了下來,並於次日返回相州。此事件後卻成為南宋官私記載中極力渲染的“崔府君顯聖”、“泥馬渡康王”故事的緣起;此亦為趙構將來引作為應天登基即位正統性之證明。)

靖康二年三、四月間,徽、欽二帝被金軍虜掠北去。

靖康二年五月初一庚寅日(1127年6月12日),趙構在南京應天府(今河南商丘)登基為帝,改元「建炎」。建炎改元後,宋高宗遙尊被擄到金國的其母親韋氏為「宣和皇后」,封自己的外祖父韋安道為郡王,親屬三十人均任官職。並且從此不斷派遣使者到金國求和要迎韋氏回南宋。

建炎元年十月丁巳初一日,宋高宗离南京南下扬州。癸未到达扬州;金人听闻后,决计大举南伐。建炎三年一月韩世忠在沐阳溃军,金军快速南下。至金数百骑兵到扬州西北之天长。壬子,金人破天长军。赵构得内侍探报,即穿盔甲乘马出门,出走扬州,而百官宰相不知。高宗渡江至京口。再次镇江;至甲寅再次长州;乙卯次无锡;丙辰次平江府;壬戌至杭州。而次月金兵并未过江。

建炎三年三月,因禁军将领对人事安排等不满,发生苗刘兵变,宋高宗被迫禅让皇位于皇子。四月,高宗在勤王大军的进发下,复辟。复辟后举行仁宗法度,录用元祐党人,多所改易政策。四月,丁卯,赵构发杭州前往江宁(建康),以谋恢复。

宋高宗被金兵追殺,一度在海上飄泊,至紹興八年(1138年)正式定都於臨安(今浙江杭州),建炎南渡完成。

紹興七年(1137年),高宗生父宋徽宗的死訊傳到南宋。『帝號慟,諭輔臣曰:「宣和皇后春秋高,朕思之不遑甯處,屈己請和,正為此耳。」(高宗號哭,對大臣說:「我母親宣和皇后年歲已經大了,我思念她到了坐不安的地步,我委屈自己向金國求和,正是為了這事。」)翰林學士朱震引用唐德宗李適的事,請高宗遙尊韋氏為皇太后,宋高宗聽從。

紹興八年(1138年),在宋使王倫的成功外交下,金朝撤銷偽齊,把包含東京開封等三京(東京、西京、南京)之地的河南、陝西歸還給南宋,但高宗生母韋太后尚未歸還。

紹興十年(1140年),金朝撕毀協約,重新攻佔陝西、河南之地。金軍主帥完顏宗弼(兀朮)先在開封正南的順昌敗於劉錡所部的「八字軍」,再於開封西南的郾城和穎昌,在女真精銳部隊所拿手的騎兵對陣中兩次敗於岳飛的岳家軍,只在開封東南面的淮西亳州、宿州一帶戰勝了宋軍中最弱的張俊一軍,在宋高宗以「十二道金牌」召回岳家軍前,金軍已被壓縮到開封東部和北部。

紹興十一年(1141年)二月,金熙宗對南宋示好,將死去的宋徽宗追封為天水郡王,將在押的宋欽宗封為天水郡公。第一提高了級別,原來封徽宗為二品昏德公,追封郡王升為一品,原來封欽宗為三品重昏侯,現封公爵升為二品。第二是去掉了原封號中的污侮含義。第三是以趙姓天水族望之郡作為封號,以示尊重。同時,在宋軍中最強大的岳家軍根本未參戰的情況下,完顏宗弼的金國最精銳的部隊又在淮西柘皋先敗於張俊部下楊沂中和劉錡的聯軍,後來雖然因為張俊搶功調走劉錡,完顏宗弼在濠州勝宋軍中最弱的張俊一軍,但由於韓世忠軍和岳家軍趕到,完顏宗弼不得不退軍北上。

四月下旬,宋高宗解除了岳飛、韓世忠、劉錡、楊沂中、張俊等大將的兵權,為《紹興和議》做好了準備。十月,南宋派魏良臣赴金,提出要議和。

十一月,金國派蕭毅、邢具瞻為審議使,隨魏良臣回南宋,提出議和條件。此時高宗生母韋氏託人將一封信送到趙構手裏。「洪皓在燕,求得(韋)后書,遣李微持歸。帝大喜曰:「遣使百輩,不如一書。」遂加(李)微官。金人遣蕭毅、邢具瞻來議和,帝曰:『朕有天下,而養不及親。徽宗無及矣!今立誓信,當明言歸我(韋)太后,朕不恥和。不然,朕不憚用兵!』(『我擁有天下,但卻不能贍養親人,我父親徽宗已經死了!現在我發誓,我要公開要求金國歸還我母親韋太后,我不以議和為恥。不然的話,我不怕向金國用兵!』),蕭毅等還,帝又語之曰:『(韋)太后果還,自當謹守誓約。如其未也,雖有誓約,徒為虛文。』」(「如果我母親韋太后果然能回南宋,自當謹守我們訂的和議誓約。如果回不來,有和議誓約也是一紙空文。」)當月,《紹興和議》最後的書面內容即達成。

十二月末除夕夜(1142年1月27日),宋高宗殺害岳飛與其子岳雲、部將張憲於臨安(今杭州),據《宋史》載這是為了滿足完顏宗弼為《紹興和議》所設的前提以防止岳飛的十萬岳家軍攻入黃河以北。

至此,高宗以稱臣賠款,割讓從前被岳飛收復的唐州、鄧州以及商州、秦州的大半為代價,簽定紹興和議。宋金東以淮河,西以大散關為界,南宋正式放棄上次和約所獲得的陝西、河南領土。宋高宗也立刻成功地迎回生母韋氏。《宋史·高宗本紀》記載:紹興十二年(1142年)夏四月丁卯(5月1日),「(韋)皇太后偕梓宮(徽宗靈柩)發五國城,金遣完顏宗賢護送梓宮,高居安護送皇太后」。按照當時信息的傳遞方式,岳飛於紹興十一年除夕夜(1142年1月27日)被殺,南宋使節立刻於紹興十二年(1142年)正月帶著正式照函從岳飛被殺的臨安(今杭州)去金國禁錮宋欽宗和韋氏的五國城(今黑龍江哈爾濱市依蘭縣依蘭鎮五國城村)接人,韋氏四月丁卯(5月1日)即啟程回宋,八月壬午(9月13日),韋氏到達宋都臨安。從正月初一到八月壬午,除了用時在行程腳力上,沒有絲毫拖延。韋氏離開五國城前,曾答應欽宗回南方後努力營救欽宗回去,但高宗可能考慮到自己已經不育而絕後,不希望有生育能力的兄長欽宗回來爭奪皇位繼承權,所以欽宗就永遠被留在北方。

紹兴和议達成后,秦桧专权弄政长達十五年,高宗一方面对秦桧放任,另一方面,处处对秦桧提防。秦桧死后,高宗始打擊秦桧餘党。

紹興三十一年(1161年),《紹興和議》被金朝皇帝完顏亮撕毀,金兵再次南侵,是为采石之战,宋军以少勝多擊退金兵。

紹興三十二年六月十一日(1162年7月24日),高宗以「倦勤」想多休養為由,禪讓於養子建王趙眘,是為宋孝宗,終結了宋太宗一脈自976年起長達186年的統治,回歸宋太祖一脈,直至南宋滅亡。

宋高宗本有一子趙旉,但因苗劉兵變受到驚嚇而病逝,得年僅兩歲。而據說高宗建炎南渡後也因為兵亂而驚嚇過度,患有陽痿,不能人道,之後未能再生下任何子女,故須在宋室子姪中選出皇位繼任人。身為宋太宗後裔的宋高宗,之所以立宋太祖的後裔趙眘為繼承人,一來宋太宗的後裔大多在靖康之難被金人虜去,另外根據《宋史》的記載,傳說是因為宋太祖顯靈託夢,野史記載高宗被宋太祖託夢稱「自從你的祖先攝用計謀,佔據我的位置很久了,以至於如今天下寥落的局面,是時候把位置還給我了。」故宋高宗過繼太祖八世孫作為養子,並立為太子;宋史中也有相似的記載,但稱孟太后被託夢。雖然是禪讓,主要決定权还是在高宗,尤其在議和問題上。宋孝宗趙眘登基後馬上為岳飛平反和肅清秦檜餘黨,身為太上皇的高宗並未阻撓,而且退位後的高宗,與君臨天下的孝宗關係相當好,父慈子孝。

淳熙十四年十月初八日(1187年11月9日),宋高宗去世,享壽八十歲,孝宗悲痛不已,持續守喪三年後,也自行退位。

宋高宗同其父宋徽宗一樣,頗有藝術天份,是傑出的書法家;自言「……凡五十年間,非大利害相仿,未始一日舍筆墨」,初學黃庭堅,後改學米芾,至終以追摹魏晉法度和王羲之、王獻之父子,流傳有《賜岳飛手敕》及《真草嵇康養生論書卷》。元朝書法家趙孟頫早年即以宋高宗書法為榜樣。

宋高宗與金朝議和,穩固南宋對中國南方的統治,議和一說在於經濟因素,所謂三軍未動,糧草先行;野蠻民族是以燒殺擄掠為錢糧來源,文明國家卻是打仗燒錢。宋高宗若不先安內,只怕民變四起,連半壁江山都沒了;歲幣議和,可緩和兩國關係,讓國家有喘息的機會;另一方面,宋高宗可以掌握軍權,壓制將領對軍隊的影響力。

《續資治通鑒》中:「康王入,毅然請行,曰:“敵必欲親王出質,臣為宗社大計,豈應辭避!”欽宗立,改元靖康,人拆其字,謂“十二月立康王”也。資性郎悟,好學強記,日誦千餘言,挽弓至一石五斗。」其他含有關於宋高宗節儉、不迷信祥瑞、不好女色、潛心治國、文才武德具備等描述。

宋高宗為保住皇位,在位初期不惜創造傳說,使天下人相信其正當正統地位,以掩飾自己「銜命出和,已作潛身之計;提師入衛,反為護己之資。忍視父兄甘為俘虜」。因金兵追擊而貪生怕死地逃命,故被後世戲稱為「逃跑皇帝」。及後他定都臨安後,為求偏安,保持半壁江山的統治,不惜把岳飛等主战派大臣殺害,以與金朝達成和議,成为后世評價的重要污点。

當時詩人林升在宿新住宿徐公店,在牆上提詩《題臨安邸》諷刺當朝的統治者曰:山外青山樓外樓,西湖歌舞幾時休?暖風熏得遊人醉,直把杭州作汴州!

元朝官修正史《宋史》脱脱等的評價是:“昔夏后氏传五世而后羿篡,少康复立而祀夏;周传九世而厉王死于彘,宣王复立而继周;汉传十有一世而新莽窃位,光武复立而兴汉;晋传四世有怀、愍之祸,元帝正位于建邺;唐传六世有安、史之难,肃宗即位于灵武;宋传九世而徽、钦陷于金,高宗缵图于南京:六君者,史皆称为中兴,而有异同焉。夏经羿、浞,周历共和,汉间新室、更始,晋、唐、宋则岁月相续者也。萧王、琅琊皆出疏属,少康、宣王、肃宗、高宗则父子相承者也。至于克复旧物,则晋元与宋高宗视四君者有余责焉。高宗恭俭仁厚,以之继体守文则有余,以之拨乱反正则非其才也。况时危势逼,兵弱财匮,而事之难处又有甚于数君者乎?君子于此,盖亦有悯高宗之心,而重伤其所遭之不幸也。然当其初立,因四方勤王之师,内相李纲,外任宗泽,天下之事宜无不可为者。顾乃播迁穷僻,重以苗、刘群盗之乱,权宜立国,确虖艰哉。其始惑于汪、黄,其终制于奸桧,恬堕猥懦,坐失事机。甚而赵鼎、张浚相继窜斥,岳飞父子竟死于大功垂成之秋。一时有志之士,为之扼腕切齿。帝方偷安忍耻,匿怨忘亲,卒不免于来世之诮,悲夫!”

明末清初儒者王夫之在《宋論》一書中如此評價高宗:“高宗之畏女真也,窜身而不耻,屈膝而无惭,直不可谓有生人之气矣。乃考其言动,察其志趣,固非周赧、晋惠之比也。何以如是其馁也?李纲之言,非不知信也;宗泽之忠,非不知任也;韩世忠、岳飞之功,非不知赏也;吴敏、李棁、耿南仲、李邦彦主和以误钦宗之罪,非不知贬也。而忘亲释怨,包羞丧节,乃至陈东、欧阳澈拂众怒而骈诛于市,视李纲如仇仇,以释女直之恨。是岂汪、黄二竖子之能取必于高宗哉?且高宗亦终见其奸而斥之矣。抑主张屈辱者,非但汪、黄也。张浚、赵鼎力主战者,而首施两端,前却无定,抑不敢昌言和议之非。则自李纲、宗泽而外,能不以避寇求和为必不可者,一二冗散敢言之士而止。以时势度之,于斯时也,诚有旦夕不保之势,迟回葸畏,固有不足深责者焉。苟非汉光武之识量,足以屡败而不挠,则外竞者中必枵,况其不足以竞者乎?高宗为质于虏廷,熏灼于剽悍凶疾之气,俯身自顾,固非其敌。已而追帝者,滨海而至明州,追隆祐太后者,薄岭而至皂口,去之不速,则相胥为俘而已。君不自保,臣不能保其君,震慑无聊,中人之恒也。亢言者恶足以振之哉? ”

清高宗乾隆帝于乾隆五十五年得玄孙,为庆贺五代同堂,特地御制诗一首:“古稀六帝三登八,所鄙宋梁所慕元,惟至元称一代杰,逊乾隆看五世孙”,意即年过古稀(70岁)的皇帝只有6个(包括汉武帝、唐玄宗、明太祖),其中只有三个活过了80岁,即梁武帝、宋高宗、元世祖,乾隆帝只敬仰元世祖忽必烈,而鄙夷梁武帝和宋高宗。而梁武帝和宋高宗皆是偏安南方的开国皇帝。

現代王曾瑜批評宋高宗,違反宋太祖「不誅大臣、言官」之誓約,殺上書言事之陳東與歐陽澈,以鉗制天下異議之口,卻竊取了「中興之主」之美譽,另外也將岳飛等主战派大臣殺害,是宋朝歷代皇帝中,唯一一位違反宋太祖祖訓的皇帝,為後人所唾罵。

\subsection{建炎}


\begin{longtable}{|>{\centering\scriptsize}m{2em}|>{\centering\scriptsize}m{1.3em}|>{\centering}m{8.8em}|}
  % \caption{秦王政}\
  \toprule
  \SimHei \normalsize 年数 & \SimHei \scriptsize 公元 & \SimHei 大事件 \tabularnewline
  % \midrule
  \endfirsthead
  \toprule
  \SimHei \normalsize 年数 & \SimHei \scriptsize 公元 & \SimHei 大事件 \tabularnewline
  \midrule
  \endhead
  \midrule
  元年 & 1127 & \tabularnewline\hline
  二年 & 1128 & \tabularnewline\hline
  三年 & 1129 & \tabularnewline\hline
  四年 & 1130 & \tabularnewline
  \bottomrule
\end{longtable}

\subsection{绍兴}

\begin{longtable}{|>{\centering\scriptsize}m{2em}|>{\centering\scriptsize}m{1.3em}|>{\centering}m{8.8em}|}
  % \caption{秦王政}\
  \toprule
  \SimHei \normalsize 年数 & \SimHei \scriptsize 公元 & \SimHei 大事件 \tabularnewline
  % \midrule
  \endfirsthead
  \toprule
  \SimHei \normalsize 年数 & \SimHei \scriptsize 公元 & \SimHei 大事件 \tabularnewline
  \midrule
  \endhead
  \midrule
  元年 & 1131 & \tabularnewline\hline
  二年 & 1132 & \tabularnewline\hline
  三年 & 1133 & \tabularnewline\hline
  四年 & 1134 & \tabularnewline\hline
  五年 & 1135 & \tabularnewline\hline
  六年 & 1136 & \tabularnewline\hline
  七年 & 1137 & \tabularnewline\hline
  八年 & 1138 & \tabularnewline\hline
  九年 & 1139 & \tabularnewline\hline
  十年 & 1140 & \tabularnewline\hline
  十一年 & 1141 & \tabularnewline\hline
  十二年 & 1142 & \tabularnewline\hline
  十三年 & 1143 & \tabularnewline\hline
  十四年 & 1144 & \tabularnewline\hline
  十五年 & 1145 & \tabularnewline\hline
  十六年 & 1146 & \tabularnewline\hline
  十七年 & 1147 & \tabularnewline\hline
  十八年 & 1148 & \tabularnewline\hline
  十九年 & 1149 & \tabularnewline\hline
  二十年 & 1150 & \tabularnewline\hline
  二一年 & 1151 & \tabularnewline\hline
  二二年 & 1152 & \tabularnewline\hline
  二三年 & 1153 & \tabularnewline\hline
  二四年 & 1154 & \tabularnewline\hline
  二五年 & 1155 & \tabularnewline\hline
  二六年 & 1156 & \tabularnewline\hline
  二七年 & 1157 & \tabularnewline\hline
  二八年 & 1158 & \tabularnewline\hline
  二九年 & 1159 & \tabularnewline\hline
  三十年 & 1160 & \tabularnewline\hline
  三一年 & 1161 & \tabularnewline\hline
  三二年 & 1162 & \tabularnewline
  \bottomrule
\end{longtable}


\section{元懿太子赵旉生平}

元懿太子赵\xpinyin*{旉}(1127年7月23日-1129年7月28日),是南宋高宗唯一的親生儿子,潘贤妃所生。

建炎元年六月十三辛未日(1127年7月23日)生于南京应天府(今河南商丘),九月十二己亥日(10月19日)拜检校少保、集庆军节度使,封魏国公。

建炎三年(1129年),金人入侵淮南,三月初五癸未日(3月26日),高宗在扬州遇上苗傅、刘正彥作乱,逼迫高宗退位,拥立魏国公赵旉为傀儡皇帝,十一己丑日(4月1日)改年號為明受,是為苗劉兵變。宰相张浚等闻知,举兵向苗傅、劉正彦二人问罪。四月初一戊申日(4月20日),高宗復位。初三庚戌日(4月22日),苗傅、劉正彦二人兵败逃走,年号改回建炎。

高宗于四月二十丁卯日(5月9日)迁往临安府(今浙江杭州),四月二十五壬申日(5月14日),趙旉被册立为皇太子。五月初八乙酉日(5月27日)随父亲一起抵达建康(今江苏南京)。

赵旉立为太子后不久就生病,宫人不小心踢到金炉发出声响,太子受到驚吓,随后病情转剧,最后于同年七月十一丁亥日(7月28日)去世,得年仅2岁。谥号为元懿太子。

靖康南渡及苗刘兵变後,懷疑宋高宗患有陽萎,之後並未生育任何的亲生子女。

元懿太子赵旉的年号明受(1129年三月-四月),南宋使用这个年号共2个月。



%%% Local Variables:
%%% mode: latex
%%% TeX-engine: xetex
%%% TeX-master: "../Main"
%%% End:
