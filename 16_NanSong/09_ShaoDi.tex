%% -*- coding: utf-8 -*-
%% Time-stamp: <Chen Wang: 2021-11-01 15:57:27>

\section{赵昺趙昺\tiny(1278-1279)}

\subsection{生平}

宋少帝趙昺(1272年2月12日-1279年3月19日),宋朝末代皇帝(南宋第九位,1278年5月10日—1279年3月19日在位)在位313天,得年7岁。宋朝第十五位皇帝宋度宗趙禥的七子,母俞修容。前任皇帝宋端宗趙昰的異母弟。先後封為永國公、信王、廣王等。

宋恭帝德祐二年正月十八日(1276年2月4日),南宋首都臨安被伯顏率領的元朝大軍佔領,5歲的小皇帝宋恭帝和謝太皇太后相繼被俘。宋恭帝的兩個異母兄弟益王趙昰和廣王趙昺,在國舅楊亮節、朝臣陸秀夫、張世傑、陳宜中和文天祥等人的護衛下南逃。在金華,趙昰被封為天下兵馬都元帥,趙昺為副元帥,晉為衛王。1276年6月14日,剛滿7歲的趙昰在福州即皇帝位,是為宋端宗,改元“景炎”,奉母楊淑妃為皇太后。

一心想對宋朝皇室斬草除根的元軍統帥伯顏,對宋端宗的南宋小朝廷窮追不捨。景炎三年(1278年),宋端宗崩,以至軍心渙散,無心戀戰。當時陸秀夫在碙州(即今日湛江市硇洲島),改碙州為祥龍縣,擁立趙昺為皇帝,改元“祥興”,仍奉端宗母楊淑妃為太后(赵昺母俞修容下落无考,亦未有被尊为太后之记载),並逃往新會崖山避難。

元朝命大將張弘範大舉進攻崖山的趙昺小朝廷。事實上,當時的宋軍還未到岸,一行人還在海上。宋軍水師在張世傑的指揮下進行頑抗,在崖門海域裡與元朝軍隊交戰,史稱崖山戰役,這場戰役關係到南宋小朝廷的存亡。結果,宋軍全軍覆沒。1279年3月19日,丞相陸秀夫見無法脫逃,便背著這位剛滿8歲的趙昺跳海殉國,楊太后亦投海殉國,宋朝正式宣告滅亡。

深圳赤灣(現屬南山區)有宋少帝陵,據說是少帝遺骸漂至赤灣附近,由僧人發現,從其身穿龍袍看出是宋少帝,於是把他葬於此。1984年,蛇口工業區和香港趙氏宗親會出資,修葺並擴建了宋少帝陵(或稱祥慶陵),現為深圳重點文物保護單位。

香港有紀念兩位宋末皇帝逃難的地方宋王臺公園,附近有「金夫人墓」,相傳為宋端宗生母楊太后之墓。由於該址後來興建聖三一堂,「金夫人墓」也隨之湮沒。

此外宋王臺公園附近以前曾建有以其命名的宋街、帝街、昺街。二次大戰期間,日軍在1942年3月擴建啟德機場,招募了數千名工人炸毀了古蹟宋王臺。戰後港英政府為重建機場,剷平了宋王臺餘下的「聖山」部份。

元朝官修正史《宋史》脱脱等的評價是:“宋之亡征,已非一日。歷数有归,真主御世,而宋之遗臣,区区奉二王为海上之谋,可谓不知天命也已。然人臣忠于所事而至于斯,其亦可悲也夫!”

\subsection{祥兴}


\begin{longtable}{|>{\centering\scriptsize}m{2em}|>{\centering\scriptsize}m{1.3em}|>{\centering}m{8.8em}|}
  % \caption{秦王政}\
  \toprule
  \SimHei \normalsize 年数 & \SimHei \scriptsize 公元 & \SimHei 大事件 \tabularnewline
  % \midrule
  \endfirsthead
  \toprule
  \SimHei \normalsize 年数 & \SimHei \scriptsize 公元 & \SimHei 大事件 \tabularnewline
  \midrule
  \endhead
  \midrule
  元年 & 1278 & \tabularnewline\hline
  二年 & 1279 & \tabularnewline
  \bottomrule
\end{longtable}



%%% Local Variables:
%%% mode: latex
%%% TeX-engine: xetex
%%% TeX-master: "../Main"
%%% End:
