%% -*- coding: utf-8 -*-
%% Time-stamp: <Chen Wang: 2019-12-26 10:44:41>

\section{孝宗\tiny(1162-1189)}

\subsection{生平}

宋孝宗趙眘(1127年11月27日-1194年6月28日、眘,「慎」異體字,shèn),南宋第二位皇帝(1162年7月24日—1189年2月18日在位),曾名伯琮、瑗、玮,字元瑰,一字元永,他是宋太祖之幼子趙德芳的後裔,宋高宗養子,宋太祖的子孫。自宋朝的帝位落入宋太宗的子孫之手後,恢復由開國皇帝宋太祖的後裔繼承,時隔186年。孝宗是南宋较有作为的皇帝,同時期的皇帝是金朝的金世宗。

绍兴三十二年(1162年),高宗禪位予赵眘,是为宋孝宗,宋孝宗在位27年,期间与金達成隆興和议。淳熙十六年(1189年)孝宗逊位,讓位予兒子宋光宗趙惇。绍熙五年(1194年)孝宗病逝,终年67岁。葬于永阜陵。

建炎元年十月廿二出生。六世祖為太祖趙匡胤的四子秦康惠王趙德芳。父亲為遠支宗親趙子偁,後封為秀安僖王。

由於高宗的唯一兒子元懿太子夭折後再沒有嗣子,所以只好從宗室中選擇繼承人,有野史稱宋高宗受宋太祖託夢,稱「汝祖自攝謀,據我位久,至於天下寥落,是當還我位。」故高宗過繼了太祖七世孫作為養子,並立為太子;《宋史》中也有相似的記載,但為孟太后受託夢。

宋高宗在苗劉兵變後,患上陽痿而不育,又因為民間傳说多年的宋太宗燭影斧聲,強奪宋太祖之位,且金太宗完顏吳乞買與宋太祖的畫像神似,吳乞買是宋太祖投胎(一說金太祖完顏阿骨打是宋太祖投胎),要來報仇,滅了宋太宗一家,導致北宋亡國及失去半壁江山,加上為了阻止欽宗后代及部分宋太宗後裔继承,且有見金朝同样因太祖系及太宗系的皇位继承導致自相殘殺,故高宗在有生之年将帝位傳承給宋太祖的後代,以阻止相關情況发生。

绍兴二年五月,高宗于太祖系宗室中选了赵伯琮养于宫中。绍兴三年(1133年)二月为和州防御使,赐名赵瑗,改贵州防御使。五年(1135年)五月,用左仆射赵鼎议,在宫中设立书院教育他,建成后以书院为资善堂。孝宗读书强记,天资特异。诏授保庆军节度使,封建国公。因传言金朝欲在北宋旧都开封拥立钦宗太子赵谌为傀儡皇帝挑战高宗法统,大将岳飞密请高宗立赵瑗为皇太子以让金人的图谋落空,高宗也安排岳飞与赵瑗相见,岳飞感叹赵瑗是兴复宋室之主。

十二年(1142年)正月,赵瑗封普安郡王。此后,宋高宗另一位养子赵璩亦封郡王。两人官属礼制相同,号为东、西府。太子多年未定,内外颇以为疑。三十年(1160年)二月立为皇子,改名赵玮,进封建王。赵璩则称“皇侄”,两人名分终定。

绍兴三十二年(1162年)五月立为皇太子,改名赵眘。六月,高宗禅位,孝宗登基,定年号隆兴,立志光复中原,收复河山,遂将岳飞平反並追封为鄂国公,肅清秦桧餘黨,并且命令老将张浚北伐中原,但在符离遭遇金军突襲大败。接着金军趁胜追击,南宋军队损失惨重,此後雙方互有勝負,但金無法越過長江,宋亦未能渡黃河。宋孝宗被迫于隆兴二年(1164年)和金国金世宗签订“隆兴和议”,次年改元“乾道”。

乾道年间,由于没有战事的干扰,加上高宗除对金和战外較少干政,孝宗专心理政,百姓富裕,五谷丰登,太平安乐,一改高宗朝时贪腐的局面。由于宋孝宗治国有方,所以使南宋出现乾淳之治(乾道,淳熙)的小康局面。

宋孝宗有一批宠臣,如曾觌、龙大渊、张说等人,与宦官梁珂等“相与盘结”,“士大夫无耻者争附之”,被列入《佞幸传》。

淳熙十四年(1187年)十月,高宗死於德壽宮中,孝宗聽聞後失聲痛哭,兩天不能進食,又表示要服喪三年。孝宗为了服丧,让太子赵惇参预政事。淳熙十六年二月初二(1189年2月18日)又禅讓帝位予太子,太子即位,是为宋光宗。守孝三年后,孝宗退位,自称太上皇,闲居慈福宮,後改名重华殿,继续为高宗服丧。

光宗由於皇后李鳳娘挑撥,与孝宗不睦,长期不去探望孝宗,孝宗为此闷闷不乐而起病。最终在宋光宗绍熙五年六月初九(1194年6月28日),宋孝宗逝于临安重华殿內,享壽六十七歲。

元朝官修正史《宋史》脱脱等的評價是:“高宗以公天下之心,择太祖之后而立之,乃得孝宗之贤,聪明英毅,卓然为南渡诸帝之称首,可谓难矣哉。即位之初,锐志恢复,符离邂逅失利,重违高宗之命,不轻出师,又值金世宗之立,金国平治,无衅可乘,然易表称书,改臣称侄,减去岁币,以定邻好,金人易宋之心,至是亦寝异于前日矣。故世宗每戒群臣积钱谷,谨边备,必曰:『吾恐宋人之和,终不可恃。』盖亦忌帝之将有为也。天厌南北之兵,欲休民生,故帝用兵之意弗遂而终焉。然自古人君起自外藩,入继大统,而能尽宫庭之孝,未有若帝。其间父子怡愉,同享高寿,亦无有及之者。终丧三年,又能却群臣之请而力行之。宋之庙号,若仁宗之为『仁』,孝宗之为『孝』,其无愧焉,其无愧焉!”
傅樂成《中國通史》:「孝宗也是南宋有數的賢君,他伐金雖然失敗,但在外交上與金人力爭,終於除去對金的臣屬關係,不能說是毫無收穫。約成之後,他仍主備禦金人,無晏安之意,假若不是金世宗在位,金勢極強,他必不甘於信守這個不平等條約。內政方面,也沒有多大缺失。他以遠支宗室,繼承大統,對高宗能竭盡孝道,這一點也是值得稱道的。」


\subsection{隆兴}


\begin{longtable}{|>{\centering\scriptsize}m{2em}|>{\centering\scriptsize}m{1.3em}|>{\centering}m{8.8em}|}
  % \caption{秦王政}\
  \toprule
  \SimHei \normalsize 年数 & \SimHei \scriptsize 公元 & \SimHei 大事件 \tabularnewline
  % \midrule
  \endfirsthead
  \toprule
  \SimHei \normalsize 年数 & \SimHei \scriptsize 公元 & \SimHei 大事件 \tabularnewline
  \midrule
  \endhead
  \midrule
  元年 & 1163 & \tabularnewline\hline
  二年 & 1164 & \tabularnewline
  \bottomrule
\end{longtable}

\subsection{乾道}

\begin{longtable}{|>{\centering\scriptsize}m{2em}|>{\centering\scriptsize}m{1.3em}|>{\centering}m{8.8em}|}
  % \caption{秦王政}\
  \toprule
  \SimHei \normalsize 年数 & \SimHei \scriptsize 公元 & \SimHei 大事件 \tabularnewline
  % \midrule
  \endfirsthead
  \toprule
  \SimHei \normalsize 年数 & \SimHei \scriptsize 公元 & \SimHei 大事件 \tabularnewline
  \midrule
  \endhead
  \midrule
  元年 & 1165 & \tabularnewline\hline
  二年 & 1166 & \tabularnewline\hline
  三年 & 1167 & \tabularnewline\hline
  四年 & 1168 & \tabularnewline\hline
  五年 & 1169 & \tabularnewline\hline
  六年 & 1170 & \tabularnewline\hline
  七年 & 1171 & \tabularnewline\hline
  八年 & 1172 & \tabularnewline\hline
  九年 & 1173 & \tabularnewline
  \bottomrule
\end{longtable}

\subsection{淳熙}

\begin{longtable}{|>{\centering\scriptsize}m{2em}|>{\centering\scriptsize}m{1.3em}|>{\centering}m{8.8em}|}
  % \caption{秦王政}\
  \toprule
  \SimHei \normalsize 年数 & \SimHei \scriptsize 公元 & \SimHei 大事件 \tabularnewline
  % \midrule
  \endfirsthead
  \toprule
  \SimHei \normalsize 年数 & \SimHei \scriptsize 公元 & \SimHei 大事件 \tabularnewline
  \midrule
  \endhead
  \midrule
  元年 & 1174 & \tabularnewline\hline
  二年 & 1175 & \tabularnewline\hline
  三年 & 1176 & \tabularnewline\hline
  四年 & 1177 & \tabularnewline\hline
  五年 & 1178 & \tabularnewline\hline
  六年 & 1179 & \tabularnewline\hline
  七年 & 1180 & \tabularnewline\hline
  八年 & 1181 & \tabularnewline\hline
  九年 & 1182 & \tabularnewline\hline
  十年 & 1183 & \tabularnewline\hline
  十一年 & 1184 & \tabularnewline\hline
  十二年 & 1185 & \tabularnewline\hline
  十三年 & 1186 & \tabularnewline\hline
  十四年 & 1187 & \tabularnewline\hline
  十五年 & 1188 & \tabularnewline\hline
  十六年 & 1189 & \tabularnewline
  \bottomrule
\end{longtable}



%%% Local Variables:
%%% mode: latex
%%% TeX-engine: xetex
%%% TeX-master: "../Main"
%%% End:
