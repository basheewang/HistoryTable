%% -*- coding: utf-8 -*-
%% Time-stamp: <Chen Wang: 2021-11-01 15:56:07>

\section{光宗赵惇\tiny(1189-1194)}

\subsection{生平}

宋光宗赵惇(1147年9月30日-1200年9月17日),南宋第三位皇帝(1189年2月18日—1194年7月24日在位),宋孝宗第三子。42岁時受孝宗禅位而登基(亦是最老即位的宋帝),由於皇后李鳳娘的挑撥,與父親宋孝宗失和,趙汝愚、韓仛冑等大臣不滿,宋孝宗死後,在隆慈太皇太后的支持下,光宗被迫內禪大寶予其子宋寧宗,史稱宋光宗內禪。退位六年後駕崩,享年53岁,谥号「循道宪仁明功茂德温文顺武圣哲慈孝皇帝」。

绍兴十七年(1147年)九月乙丑生,郭皇后之子,二十年(1150年)赐名赵惇。宋孝宗即位,封恭王,封邑恭州(今重慶市)。

乾道七年(1171年)二月立为皇太子,當時孝宗因趙惇性格與自己相似,故立為太子,與此同時,孝宗也擔憂太子的政治能力,但是淳熙十六年(1189年)二月,孝宗依然禪位予光宗,自稱「壽皇」,以明年为绍熙元年。

光宗是宋朝皇帝中比较平庸的一位。他登基時42岁,並不算年老,卻体弱多病。心理上,也没有安邦治国之才,听取奸臣谗言,罢免辛弃疾等主战派大臣,又让当时著名的妒妇、心狠手辣的皇后李鳳娘干政,自己對朝政的掌握力不斷下降。

紹熙四年,光宗將兩浙東路馬步軍總管姜特立召至杭州行在,遭到丞相留正的強烈反對,留正隨後出走,直到當年十一月,光宗將姜特立遣回本官,留正方才回朝。但光宗在位的五年時間,基本延續了高宗、孝宗兩朝賑濟災民的政策,也能對一些官吏瀆職的現象予以打擊,如文思院監常良孫因貪污,不僅自身受到貶黜,薦舉其人的宰相周必大也遭到了降爵處分。經濟方面,淮河一帶在光宗朝繼續嚴格實行紙鈔第一政策,一切銀錢都不得進入兩淮地區。湖北一部份地區在紹熙後期開放了鐵錢。

光宗由於皇后李鳳娘的挑撥,素來与孝宗不和,宋孝宗退位后,自稱「壽皇」,閒居宮中,皇后李鳳娘教唆光宗,光宗长期不去探望。李皇后謀殺了有孕在身的黃貴妃,光宗知道之後,又生病了,壽皇送來了藥丸,但李皇后卻造謠說那是毒藥,離間光宗與壽皇父子關係。有一次,光宗宴請大臣,大臣請光宗探望壽皇,李皇后知道了以後立即阻止。绍熙五年(1194年),壽皇得病,宋光宗不去探望壽皇,也不讓他人探望,最终朝臣拒绝见光宗,集体前往朝见壽皇,光宗才被迫妥协。

壽皇病逝时,光宗不服丧。壽皇的喪禮需要身為長子的宋光宗主持,光宗也不主持,大臣們都極為不滿。樞密使趙汝愚则聯合趙彥逾、葉適、徐誼、郭杲、韓仛冑等人策畫政變,由隆慈太皇太后(憲聖慈烈皇后)的外甥韓仛冑請隆慈太皇太后垂簾聽政,又以防不测,命殿帅郭杲,步帅阎仲连夜带兵控制南皇宫南北,傅昌朝暗中制造黄袍,趙汝愚兵谏逼迫光宗退位。使得尚未成为太子的赵扩,就受禪稱帝,是为宋宁宗,光宗闲居临安行在寿康宫,自称“太上皇”。

庆元六年八月初八(1200年9月17日),憂鬱而駕崩,葬于永崇陵(今浙江绍兴东南35里处宝山)。

宋光宗在太子位已经很长時間,却仍不见宋孝宗传位给他。一次他前往重华宫侍奉孝宗,顺便向孝宗试探道:“有人给兒臣送来了染胡须的药,兒臣却不敢用。”孝宗清楚儿子的用意,就说:“我正要向天下显示你的老成,要染鬍鬚的药幹甚麼呢?”

一次,光宗的黄贵妃病了,御医用了许多药也不见效果。光宗无奈只好张榜求医。一位江湖郎中揭榜进宫,建议以山楂和冰糖煎熬服用。贵妃按照郎中的方法服用,果然就如期痊愈了。此后,该方法传入民间,逐渐演变成今天的冰糖葫芦。

元朝官修正史《宋史》脱脱等的評價是:“光宗幼有令闻,向用儒雅。逮其即位,总权纲,屏嬖幸,薄赋缓刑,见于绍熙初政,宜若可取。及夫宫闱妒悍,内不能制,惊忧致疾。自是政治日昏,孝养日怠,而乾、淳之业衰焉。”


\subsection{绍熙}


\begin{longtable}{|>{\centering\scriptsize}m{2em}|>{\centering\scriptsize}m{1.3em}|>{\centering}m{8.8em}|}
  % \caption{秦王政}\
  \toprule
  \SimHei \normalsize 年数 & \SimHei \scriptsize 公元 & \SimHei 大事件 \tabularnewline
  % \midrule
  \endfirsthead
  \toprule
  \SimHei \normalsize 年数 & \SimHei \scriptsize 公元 & \SimHei 大事件 \tabularnewline
  \midrule
  \endhead
  \midrule
  元年 & 1190 & \tabularnewline\hline
  二年 & 1191 & \tabularnewline\hline
  三年 & 1192 & \tabularnewline\hline
  四年 & 1193 & \tabularnewline\hline
  五年 & 1194 & \tabularnewline
  \bottomrule
\end{longtable}



%%% Local Variables:
%%% mode: latex
%%% TeX-engine: xetex
%%% TeX-master: "../Main"
%%% End:
