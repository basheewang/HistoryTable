%% -*- coding: utf-8 -*-
%% Time-stamp: <Chen Wang: 2019-12-26 10:48:39>

\section{理宗\tiny(1224-1264)}

\subsection{生平}

宋理宗趙昀(1205年1月26日-1264年11月16日),原名赵与莒,後賜名赵贵诚,宋太祖次子燕懿王趙德昭九世孫,宋宁宗太子趙竑與宰相史彌遠不睦,1224年寧宗駕崩後,彌遠矯詔立貴誠為帝,是為宋理宗,改名赵昀,是南宋的第五位皇帝(1224年9月17日—1264年11月16日在位),在位40年,享年59岁。

宋理宗本名趙與莒,本不是皇子,而只是宋宁宗的远房堂侄。他是宋太祖赵匡胤的十世孫,是趙匡胤次子燕懿王趙德昭的後人,但由於宋帝位一向並非由趙德昭這一脈後人繼承,至趙與莒父親趙希瓐這一代已與皇室血緣十分疏遠,而趙希瓐在生時並沒有任何封爵,僅官至山陰尉,生活與平民無異,趙與莒也因此在平民家庭出生及成長。趙與莒七歲時,父趙希瓐逝世,生母全氏帶他及弟趙與芮返娘家,三母子在全氏在紹興當保長的兄長家寄居,一直到趙與莒十六歲。

宋寧宗因八名親生子皆幼年夭折,故立趙德昭後裔趙詢為太子,趙詢於廿八歲時英年早逝,寧宗改立沂王赵竑為太子,沂王王位於是懸空,寧宗命宰相史彌遠找尋品行端正的宗室過繼給沂王,繼承王位,而史彌遠將此任務交了其幕僚余天錫。余天錫回鄉應考科舉,途經紹興遇著大雨,在全保長家中避雨,於是認識了趙與莒兄弟。余天錫知他們為趙氏宗族,也覺得兄弟二人行為得體,認為是合適人選繼承沂王,故向史彌遠推薦。史彌遠接兩兄弟往臨安親自考量,也認為兄長趙與莒為繼承沂王的合適人選,故於嘉定十四年(1221年)將趙與莒賜名贵诚,繼承沂王王位。

太子赵竑一向不滿史彌遠專權,聲言繼位後立即要貶史彌遠到海南島去,史彌遠決心另立新君。嘉定十七年(1224年),宋宁宗駕崩,史弥远聯同楊皇后假傳寧宗遺詔,太子赵竑廢為濟王,藩封霅川。以沂王趙貴誠為養子,賜名趙昀,是為宋理宗。

由於宋理宗是史彌遠一手擁立,登基后史弥远繼續專權,早已成年的理宗对政务完全不能过问,一直到1233年史彌遠死后,理宗才开始正式亲政。理宗一親政就任用洪咨夔等人做監察御史,彈劾了史彌遠一黨的「三凶」梁成大、李知孝、莫澤。而被史彌遠排斥的真德秀、魏了翁則被召入朝。當時紙幣的發行量超過三億貫,通貨膨脹,物價飛漲。宋廷停止發行新幣,回收部分舊幣,並動用庫存黃金十萬兩、白銀數百萬兩平抑物價。1234年南宋联蒙灭金,但不久蒙古入侵,南宋为防止蒙军南下而軍費陡升,宋廷不得不大量發行貨幣以緩解財政壓力。最終經濟整頓破產。

宋理宗一直希望使理學成為正統官學,早在寶慶三年(1227年)就追封朱熹為信國公。端平更化後,朱熹和理學大師周敦頤、程顥、程頤、張載都先後被入祀孔廟。淳祐元年(1241年)理宗又分別加封周敦頤為汝南伯、程顥為河南伯、程頤為伊陽伯、張載為噤伯。景定二年(1261年)理宗排定的入祀孔廟的名單包括:司馬光、周敦頤、程顥、程頤、張載、朱熹、邵雍、張栻、呂祖謙。其中除司馬光外,剩下的都是理學代表人物。

理宗的政治改革也要解決大量冗官,這是通過控制考中進士的人數和嚴格升遷制度辦到的。從端平元年開始,平均每次科考的中進士人數為四百五十人,而不是以前的平均每次六百人。理宗又規定無論在首都的朝官還是在外地的地方官都不得私薦官員,沒有擔任過州縣地方官员的人不能進入朝廷做郎官,已經當上郎官的必須外放,補上州縣地方官這一任。但這些措施流於表面,未能根本解決問題。

晚年,宋理宗对政治不感兴趣,将国家大事交给他的丞相处理,先后有吴潜、丁大全、范光瑞等,其中任用賈似道。

宋理宗晚年盡情女色,三宫六院已满足不了他的私欲。善于奉迎的内侍董宋臣看到了,便在一次元宵佳节,董宋臣为宋理宗找来临安名妓唐安安入宫淫乐。唐安安姿色豔美,能歌善舞。宋理宗一見非常喜愛,便把她留在宫里,日夜宠幸。唐安安仗着宋理宗的宠爱,过起了豪奢的生活,家中的用具上到妆盒酒具,下到水盆火箱,都是用金银制成的;帐幔茵褥,也都是绫罗锦绣;珍奇宝玩,更是不计其数。除了唐安安之外,宋理宗还经常召一些歌妓舞女进宫,起居郎牟子才上书劝诫宋理宗:“此举坏了陛下三十年自修之操!”宋理宗却让人转告牟子才、不得告知他人,以免有损皇帝的形象。姚勉以唐玄宗、杨贵妃、高力士为例劝告宋理宗,宋理宗竟然恬不知耻地回答:“朕虽不德,未如明皇之甚也(自嘲不如唐玄宗厲害)。”

阎贵妃是宋理宗晚年最宠爱的妃子,姿色妖媚,以美色受寵愛,初封婉容。淳祐九年(1240)九月,宋理宗封阎氏为贵妃,當時贾贵妃去世,留下一個六歲的女兒瑞國公主,而阎贵妃沒有懷孕,生下子女,宋理宗便將瑞國公主交由阎贵妃撫養。宋理宗為了表示對阎贵妃的寵愛,对她赏赐无數,阎贵妃想修建一座功德寺,宋理宗不惜动用国库,耗費巨資,破天荒地派遣吏卒到各州县搜集木材,为其修功德寺,闹得老百姓不得安宁,为了求得合适的的梁柱,竟想砍去灵隐寺前的晋代古松。幸好灵隐寺住持僧元肇,写了一首诗:“不为栽松种茯苓,只缘山色四时青。老僧不许移松去,留与西湖作画屏。”這才保住古松,这座功德寺前后花了三年才建成,耗费极大,修得比自家祖宗的功德寺还要富丽堂皇,當時人称為“赛灵隐寺”。

後來阎贵妃在理宗的宠爱下,權勢大增,不可一世,骄横放肆,恃宠弄权,一些投机钻营的小人,走她的门路。其中,周汉国公主下嫁,马天骥绞尽脑汁送了一份别出心裁的大礼,得到宋理宗的欢心,与丁大全同时被任命为执政,所以阎贵妃又與马天骥、丁大全、人稱“董閻羅”的董宋臣等奸臣内外勾結,狼狽為奸,沆瀣一氣,阎贵妃、马天骥、丁大全、董閻羅,史稱“阎马丁董”,恃寵亂政,结党营私,排除异己,陷害忠良,迫逐宰相董槐,引起很多忠臣不满,當時,有人在朝門上題八個大字:“閻馬丁當,國勢將亡。”這對男女又與贾似道明争暗斗,打擊迫害,紊亂朝政,民怨沸騰,“阎马丁董”等四人又強奪民田,招權納賄,作惡多端,無所不爲。

1259年,忽必烈攻鄂州,右丞相贾似道向忽必烈称臣并许诺将长江以北的土地完全割让给以蒙元,后因蒙哥在釣魚城战死,为争夺帝位,忽必烈不得不退回北方,南宋才化险为夷。贾似道因此谎报军情邀功,博得宋理宗信任,而对曾经卖国的许诺只字不提。

宋理宗在位四十年后,1264年逝世於臨安。由於宋理宗无子,所以立他的侄子赵禥为太子,是为宋度宗。

宋亡後,元朝西藏藏傳佛教僧人楊璉真珈盜掘南宋六陵,見宋理宗屍身保存完好,將屍體倒掛在樹上三天,結果流出水銀,又以理宗頭蓋骨奉給帝師八思巴為飲器,是為骷髏碗。躯干则火化。明初,明太祖得知此事,“嘆息久之”,怒斥道:“南宋的皇帝们没有很大的道德操守问题,跟蒙元也没有世仇。蒙元已然是趁人之危攻取南宋的,为什么还要如此残酷?”,派人找回理宗的頭顱,洪武二年(1369年)以帝王禮葬於應天府(江蘇南京),次年又將理宗的頭骨歸葬到紹興永穆陵舊址。

元朝官修正史《宋史》脱脱等的評價是:“理宗享国久长,与仁宗同。然仁宗之世,贤相相继。理宗四十年之间,若李宗勉、崔与之、吴潜之贤,皆弗究于用;而史弥远、丁大全、贾似道窃弄威福,与相始终。治效之不及庆历、嘉祐,宜也。蔡州之役,幸依大朝以定夹攻之策,及函守绪遗骨,俘宰臣天纲,归献庙社,亦可以刷会稽之耻,复齐襄之仇矣。顾乃贪地弃盟,入洛之师,事衅随起,兵连祸结,境土日蹙。郝经来使,似道讳言其纳币请和,蒙蔽抑塞,拘留不报,自速灭亡。吁,可惜哉!由其中年嗜欲既多,怠于政事,权移奸臣,经筵性命之讲,徒资虚谈,固无益也。虽然,宋嘉定以来,正邪贸乱,国是靡定,自帝继统,首黜王安石孔庙从祀,升濂、洛九儒,表章朱熹《四书》,丕变士习,视前朝奸党之碑、伪学之禁,岂不大有径庭也哉!身当季运,弗获大效,后世有以理学复古帝王之治者,考论匡直辅翼之功,实自帝始焉。庙号曰"理",其殆庶乎!”

明末清初思想家王夫之的评价是:“理宗无君人之才,而犹有君人之度。”

李贽:“理宗是个得失相半之主。”


\subsection{宝庆}


\begin{longtable}{|>{\centering\scriptsize}m{2em}|>{\centering\scriptsize}m{1.3em}|>{\centering}m{8.8em}|}
  % \caption{秦王政}\
  \toprule
  \SimHei \normalsize 年数 & \SimHei \scriptsize 公元 & \SimHei 大事件 \tabularnewline
  % \midrule
  \endfirsthead
  \toprule
  \SimHei \normalsize 年数 & \SimHei \scriptsize 公元 & \SimHei 大事件 \tabularnewline
  \midrule
  \endhead
  \midrule
  元年 & 1225 & \tabularnewline\hline
  二年 & 1226 & \tabularnewline\hline
  三年 & 1227 & \tabularnewline
  \bottomrule
\end{longtable}

\subsection{绍定}

\begin{longtable}{|>{\centering\scriptsize}m{2em}|>{\centering\scriptsize}m{1.3em}|>{\centering}m{8.8em}|}
  % \caption{秦王政}\
  \toprule
  \SimHei \normalsize 年数 & \SimHei \scriptsize 公元 & \SimHei 大事件 \tabularnewline
  % \midrule
  \endfirsthead
  \toprule
  \SimHei \normalsize 年数 & \SimHei \scriptsize 公元 & \SimHei 大事件 \tabularnewline
  \midrule
  \endhead
  \midrule
  元年 & 1228 & \tabularnewline\hline
  二年 & 1229 & \tabularnewline\hline
  三年 & 1230 & \tabularnewline\hline
  四年 & 1231 & \tabularnewline\hline
  五年 & 1232 & \tabularnewline\hline
  六年 & 1233 & \tabularnewline
  \bottomrule
\end{longtable}

\subsection{端平}

\begin{longtable}{|>{\centering\scriptsize}m{2em}|>{\centering\scriptsize}m{1.3em}|>{\centering}m{8.8em}|}
  % \caption{秦王政}\
  \toprule
  \SimHei \normalsize 年数 & \SimHei \scriptsize 公元 & \SimHei 大事件 \tabularnewline
  % \midrule
  \endfirsthead
  \toprule
  \SimHei \normalsize 年数 & \SimHei \scriptsize 公元 & \SimHei 大事件 \tabularnewline
  \midrule
  \endhead
  \midrule
  元年 & 1234 & \tabularnewline\hline
  二年 & 1235 & \tabularnewline\hline
  三年 & 1236 & \tabularnewline
  \bottomrule
\end{longtable}

\subsection{嘉熙}

\begin{longtable}{|>{\centering\scriptsize}m{2em}|>{\centering\scriptsize}m{1.3em}|>{\centering}m{8.8em}|}
  % \caption{秦王政}\
  \toprule
  \SimHei \normalsize 年数 & \SimHei \scriptsize 公元 & \SimHei 大事件 \tabularnewline
  % \midrule
  \endfirsthead
  \toprule
  \SimHei \normalsize 年数 & \SimHei \scriptsize 公元 & \SimHei 大事件 \tabularnewline
  \midrule
  \endhead
  \midrule
  元年 & 1237 & \tabularnewline\hline
  二年 & 1238 & \tabularnewline\hline
  三年 & 1239 & \tabularnewline\hline
  四年 & 1240 & \tabularnewline
  \bottomrule
\end{longtable}

\subsection{淳祐}

\begin{longtable}{|>{\centering\scriptsize}m{2em}|>{\centering\scriptsize}m{1.3em}|>{\centering}m{8.8em}|}
  % \caption{秦王政}\
  \toprule
  \SimHei \normalsize 年数 & \SimHei \scriptsize 公元 & \SimHei 大事件 \tabularnewline
  % \midrule
  \endfirsthead
  \toprule
  \SimHei \normalsize 年数 & \SimHei \scriptsize 公元 & \SimHei 大事件 \tabularnewline
  \midrule
  \endhead
  \midrule
  元年 & 241 & \tabularnewline\hline
  二年 & 242 & \tabularnewline\hline
  三年 & 243 & \tabularnewline\hline
  四年 & 244 & \tabularnewline\hline
  五年 & 245 & \tabularnewline\hline
  六年 & 246 & \tabularnewline\hline
  七年 & 247 & \tabularnewline\hline
  八年 & 248 & \tabularnewline\hline
  九年 & 249 & \tabularnewline\hline
  十年 & 250 & \tabularnewline\hline
  十一年 & 251 & \tabularnewline\hline
  十二年 & 252 & \tabularnewline
  \bottomrule
\end{longtable}

\subsection{宝祐}

\begin{longtable}{|>{\centering\scriptsize}m{2em}|>{\centering\scriptsize}m{1.3em}|>{\centering}m{8.8em}|}
  % \caption{秦王政}\
  \toprule
  \SimHei \normalsize 年数 & \SimHei \scriptsize 公元 & \SimHei 大事件 \tabularnewline
  % \midrule
  \endfirsthead
  \toprule
  \SimHei \normalsize 年数 & \SimHei \scriptsize 公元 & \SimHei 大事件 \tabularnewline
  \midrule
  \endhead
  \midrule
  元年 & 1253 & \tabularnewline\hline
  二年 & 1254 & \tabularnewline\hline
  三年 & 1255 & \tabularnewline\hline
  四年 & 1256 & \tabularnewline\hline
  五年 & 1257 & \tabularnewline\hline
  六年 & 1258 & \tabularnewline
  \bottomrule
\end{longtable}

\subsection{开庆}

\begin{longtable}{|>{\centering\scriptsize}m{2em}|>{\centering\scriptsize}m{1.3em}|>{\centering}m{8.8em}|}
  % \caption{秦王政}\
  \toprule
  \SimHei \normalsize 年数 & \SimHei \scriptsize 公元 & \SimHei 大事件 \tabularnewline
  % \midrule
  \endfirsthead
  \toprule
  \SimHei \normalsize 年数 & \SimHei \scriptsize 公元 & \SimHei 大事件 \tabularnewline
  \midrule
  \endhead
  \midrule
  元年 & 1259 & \tabularnewline
  \bottomrule
\end{longtable}

\subsection{景定}

\begin{longtable}{|>{\centering\scriptsize}m{2em}|>{\centering\scriptsize}m{1.3em}|>{\centering}m{8.8em}|}
  % \caption{秦王政}\
  \toprule
  \SimHei \normalsize 年数 & \SimHei \scriptsize 公元 & \SimHei 大事件 \tabularnewline
  % \midrule
  \endfirsthead
  \toprule
  \SimHei \normalsize 年数 & \SimHei \scriptsize 公元 & \SimHei 大事件 \tabularnewline
  \midrule
  \endhead
  \midrule
  元年 & 1260 & \tabularnewline\hline
  二年 & 1261 & \tabularnewline\hline
  三年 & 1262 & \tabularnewline\hline
  四年 & 1263 & \tabularnewline\hline
  五年 & 1264 & \tabularnewline
  \bottomrule
\end{longtable}



%%% Local Variables:
%%% mode: latex
%%% TeX-engine: xetex
%%% TeX-master: "../Main"
%%% End:
