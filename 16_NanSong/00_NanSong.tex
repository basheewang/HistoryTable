%% -*- coding: utf-8 -*-
%% Time-stamp: <Chen Wang: 2019-12-26 10:39:48>

\chapter{南宋\tiny(1127-1279)}

\section{简介}

南宋(1127年6月12日-1279年3月19日)是宋朝的其中一个時期,與北宋合稱「兩宋」。北宋以开封被金人攻入及佔領而结束,1127年6月12日,宋徽宗第九子康王赵构南逃至南京应天府(今河南商丘)即位,是为宋高宗,改元建炎。因宋朝的五行德運為「火」,改元「建炎」意味著重建王朝的火德。

因以临安(今浙江杭州)為都城,史称南宋,以别于北宋。绍兴和议后,大部分時間与金朝东沿淮水(今淮河),西以大散关为界。南宋与金朝、西辽、大理国、西夏、吐蕃及13世纪初兴起的蒙古帝国/元朝为并存政权,直至1279年亡于元朝,共152年。

由于未能收复中原,南宋的统治范围被迫限于秦岭淮河線以南地区,與金国长期军事对峙,但是亦刺激了南宋發展经济、手工业、航運贸易、武器制造及科技。

金人領袖完顏阿骨打建國號大金。宋徽宗认为有机可乘,便派使者向金提出「联金灭辽」。宋攻燕京,但无功而返。金攻陷临潢府,辽亡。宋付上巨额赎款给金,以换取燕京等地。然而,在這次軍事合作中,金看穿北宋的疲弱,借口北宋收容金叛将与辽遗臣,分兵南下,趋汴梁。钦宗即位,与金人和议,金人解兵北归。次年,即靖康二年(1127年),金人南下,攻陷汴梁(京)(今开封),掳走两宗北去,史称「靖康之变」,北宋灭亡。

1127年,金朝从汴京撤军,立张邦昌为偽楚皇帝。徽宗唯一免于被俘的儿子康王赵构拥兵在外,张邦昌无力抗衡,以孟太后之名,下诏立其为帝。靖康二年五月初一(1127年6月12日),赵构在南京应天府(又稱歸德軍,金称之为归德府,今河南商丘)正式即位,重建宋朝,是为宋高宗,即年改元建炎。當時宋朝仍控制河南、關中。高宗重用主戰派,以李綱為相、宗澤鎮守汴梁,曾多次大敗金兵,令局面稍為穩定。但是,高宗沒有對抗金朝的决心,罷免了李綱等人,亦屡次拒绝宗泽要他回汴梁主持大局的请求。不久高宗南逃揚州。燕人赵恭冒称高宗弟信王赵榛,聚兵五马山抗金,寻求高宗支持,高宗亦名为支持实予制约(曾在听闻其欲入京祭祖后,一反常态地表态要回京),导致其最终失败。宗澤亦憂憤而死。高宗派去接手河北的杜充更令河北义军全面崩溃。

建炎二年(1128年),金完顏宗弼又繼續大舉南侵,宋高宗南逃至杭州,升杭州為臨安府,以備日後建都。1129年,南宋发生废黜高宗、改立皇子魏国公赵旉为帝的苗刘兵变,旋即被平定。同年秋,完顏兀朮繼續揮軍南下,高宗乘船出海避難。翌年春,金軍回師,宋將韓世忠率八千精兵,堵塞五萬金軍在黃天蕩四十八日,史稱黃天蕩之戰。1130年秋,金專注進攻關中,金國又立劉豫為帝,國號齊(偽齊),以加強對黃河以南地區之統治[需要更好来源]。

建炎三年(1129年)至紹興八年(1138年),建康(今南京)曾短暫地是南宋之實際首都。绍兴八年(1138)正月,宋高宗离开建康,定都临安(今杭州)。終南宋一朝,建康依然為留都,因建康太鄰近淮河,为免金人南侵迅即攻破,宋高宗定都臨安,故都開封始終視為正統首都。

期间,绍兴七年(1137年)发生的淮西宋军投金的淮西兵变令南宋军心大挫,也令南宋抗金的姿态转向保守。

宋高宗由於害怕軍人戰勝回朝會專橫、難以控制;而且亦擔心欽宗被金人拥立挑战自己的皇统,所以在1138年任秦檜為相,向金求和。秦檜削去抗金將領韓世忠兵權,1138年宋金初次議和,南宋向金稱臣,但取回包含開封府的河南、陝西之地,可說是外交一大勝利。

1140年,金朝撕毀協議,金兵分三路南侵,重佔河南、陝西等地,宋軍在許多抗金名將指揮下,取得輝煌戰果。尤其是岳飛在郾城與金將完颜宗弼會戰,力挫金兵,乘機進兵朱仙鎮,收復了黃河以南一帶,與開封只有四十五里。後來,高宗先召回北進諸軍,再以十二面金牌下令岳飛班師回朝。后他察觉岳飞北伐之成效,再下诏命其暂停班师继续北伐,但为时已晚。1142年1月,宋高宗以莫須有罪名殺害了岳飛父子,完全收奪諸大將兵權。宋與金簽訂《紹興和議》,宋高宗以向金朝納貢稱臣為代價,換取双方停战偏安東南半壁江山。

比建国初期相比,稳定后的南宋不再控制河北、河南、山东、关中等地,疆域大幅缩水。

1149年,金海陵王完颜亮发动政变,自立为帝。经过12年的准备,1161年(绍兴三十一年)5月完颜亮南侵,企图灭宋。宰相陈康伯主导抗金大计,危难之际,要求高宗下诏亲征建康(今南京)。10月,宋将李宝率舟师以火攻全歼金水军于胶西陈家岛(或唐岛,又称陈家岛海战)。11月,宋中书舍人虞允文于采石(建康西三四十公里,今马鞍山)以建康守军败金兵,完颜亮移兵瓜洲(今江苏扬州),宋军刘琦、杨存中在镇江(建康东七八十公里)严密防守。完颜亮渡江失利,11月下旬,为部将所弑。南宋取胜。双方动员的兵力在70万至80万。

自完顏亮南侵失敗後,南宋北伐的聲音高漲。宋高宗決定禪位於養子趙昚。是為孝宗,趙昚是宋太祖趙匡胤的次子秦王趙德芳的六世孫,自此宋朝的帝位經186年後由太宗的子孫轉回到太祖的子孫的手上。

在1164年12月,張浚隆興北伐失敗後,簽訂《隆兴和议》(又名《乾道和議》),把原本向金稱臣改為叔姪關係,金為叔,宋為侄,金改詔表為國書,歲貢改為歲幣,減少貢獻,割讓秦州及商州,維持原疆界。绢貢獻由25萬減至20萬,歲幣減至20萬銀兩。

宋孝宗起用虞允文、周必大等人,朝政較為安定。宋高宗雖然作為太上皇,但在幕後對宋孝宗施加壓力反对北伐,促成和议,維持與金的和平局面。1187年,當了二十五年太上皇的宋高宗去世,在守孝三年後,宋孝宗決定讓位給皇子宋光宗趙惇,退位為太上皇。

權臣專權是終南宋一朝的特點,南宋自秦檜後,權臣一直在南宋發揮重要的角色,皇帝通過讓權臣掌握一定的權力,同時籍權臣來控制武將,以掌握軍權。

宋光宗即位後,由於體弱多病,而皇后李氏恃光宗生性懦弱,任何事要取決於她。光宗更賜李家家廟、追封及授與李家官爵。加上光宗在李氏的影響下,對孝宗的情況不聞不問,喪禮幾乎無人主持。於是樞密使趙汝愚借光宗手諭「曆事歲久,念欲退閑」八字,在高宗吳太后的外甥韓侂胄的聯絡下請吳太后出面宣佈光宗退位。立光宗子嘉王趙擴繼位。是為宋寧宗,稱為紹熙內禪。

宋寧宗信任韓侂胄,韓侂胄排擠趙汝愚等人專斷朝政,而且又引發慶元黨禁,排擠朱熹等人。在1206年,韓侂胄北伐,後來被金擊退。在1208年,簽訂《嘉定和议》。兩國改為叔侄關係,宋由貢獻歲幣及绢由20萬增至30萬,宋賠償300萬軍費,獻上韓侂胄首級。金放弃占领的大散关、濠州。

宋寧宗在史彌遠協助下,誅殺韓侂胄,自此宋寧宗起用史彌遠,史彌遠在寧宗駕崩時因知曉皇子濟國公竑(实为养子)對自己不滿,矯詔擁立宋理宗,也因此更加掌握大權,理宗登基後將所有國家事物都交給史彌遠,自己對政務幾乎不過問,史彌遠專政二十六年,南宋政治日漸腐敗。但史弥远也支持李全等抗金势力。理宗親政後,把政事交給丁大全處理,後來更信任賈似道,種下南宋滅亡的遠因。

1214年七月,金已遭受蒙古打擊,被逼由燕京遷都至開封,宋寧宗接納真德秀奏议,决定从此罷金國歲幣。1217年四月,为了扩大疆土以弥补被蒙古侵占的地域,金以宋不再纳岁币为名,出兵南侵,南宋则与蒙古協議联手擊退金军。

1231年,蒙軍欲借道宋境,繞道攻金後方,宋拒絕,蒙古拖雷強行通過。蒙軍在1232年三峰山之戰消滅金國主力部隊,金國實際上已無力回天。十二月,宋理宗看到金滅亡在即,無屏蔽之用,就同意聯合蒙古滅金,雙方協議黃河以南歸宋,以北土地歸蒙古。1233年,金哀宗逃至蔡州。南宋再攻下金唐州等地。九月,哀宗向宋理宗說明「唇亡齒寒」之理,提議聯手抗蒙,宋朝不許,繼續伐金。1234年正月,金国蔡州被蒙宋联军攻陷,金哀宗自缢,金灭亡。南宋與蒙古接壤,在失去金國作為屏障后,却面臨比金更強大的蒙古南下威脅。

早在窝阔台三年(1231年)蒙军就入侵南宋之川陕四路所辖的汉中地区,以借道攻金。在窝阔台七年(1235年)开始全面入侵宋,蒙軍首次南侵,被擊退。蒙軍並不甘心失敗,於次年9月和第三年兩次分三道南侵,其前部幾乎接近長江北岸。由於宋軍奮勇作戰並擁有優勢水軍,打敗蒙軍,再一次挫敗蒙軍渡江南下的企圖。而後,南宋軍民又在抗蒙將領孟珙、余玠、赵葵、杜杲、曹友闻、张钰、向士璧、曹世雄等人的指揮下,多次擊敗蒙軍,使其不得不繞道吐蕃(1240年)、大理(1252年)而行。1239年宋軍從蒙古手上收復襄陽等地。1245年蒙古軍越過淮河南侵宋境。

蒙哥汗八年(1259年)蒙古大汗蒙哥攻戰合州,7月在釣魚城之戰受宋軍流矢所傷,因而死於軍中。其弟忽必烈正於鄂州與宋軍交戰,聽到消息後,立即撤軍以便奪取大汗之位,南宋賈似道派人與忽必烈議和,以保太平。賈似道回京後,無恥地隱瞞自己與蒙古割地議和一事,在其姐賈貴妃協助下,使理宗罷免宰相丁大全,任命賈似道執政。賈似道討好邊境守將,隱瞞割地議和一事,更擁立皇姪建安郡王繼位,是為度宗,賈似道權勢更盛。

1260年,忽必烈返回北方自立為汗之後,派遣使者郝經赴南宋與賈似道繼續談和,但是被賈扣押在真州(今江蘇儀徵)。1267年(至元四年),忽必烈下令攻打南宋重鎮襄陽,是為襄樊之戰。宋軍利用漢水將資源源源不絕送入城內,才能堅守城池。守將呂文德及呂文煥堅守城池六年,賈似道派了范文虎及李庭芝援助,但兩者之間不和。賈似道封鎖了所有蒙古南侵消息,皇帝並不知此事。1271年,忽必烈在中原建立大元帝國。在1272年,張順、張貴兩兄弟的義兵曾血戰元軍。在1273年,樊城失守,襄陽城破,宋軍繼續巷戰,呂文煥最終投降,六年的襄陽保衛戰結束。度宗在蒙古大軍南侵的情況下,得病駕崩。1275年(恭帝德祐元年)賈似道在太皇太后的壓力下,不得不率兵親征,但賈似道拋棄其統領的13萬精兵乘小船逃走,南宋軍隊大敗(丁家洲之戰),結果在朝野壓力下,賈似道被貶,在木棉庵被會稽縣尉鄭虎臣殺害於廁中。

元軍南侵過程中,宋人拼死抵抗,池州趙卯發,饒州唐震、江萬里相繼殉國,太皇太后謝氏遂下哀痛之詔,號召天下勤王,張世傑、文天祥、李芾率兵入援。

1276年2月4日(至元十三年正月十八)元軍攻佔南宋都城臨安(今杭州),俘5歲的南宋皇帝恭帝,南宋大勢已去。但是,南宋殘餘勢力陸秀夫、文天祥和張世傑等人連續擁立了兩個幼小皇帝(端宗、幼主),成立小朝廷,元軍對小皇帝窮追不捨,不斷逃亡至南方,端宗落水得病逝世,陸秀夫在碙洲梅尉(今香港梅窩)另立幼主,逃至新會至南海一帶。文天祥在海豐兵敗被俘,走投無路的南宋殘餘勢力終於在至元十六年二月初六(1279年3月19日)厓山海戰失敗而宋軍全軍覆沒,陸秀夫徹底絕望,在安撫幼主之後,將國璽綁在幼主身上,背著他跳海自盡。而張世傑在厓山戰敗後本欲突圍,卻得知陸秀夫已背負幼主跳海自盡,自己也無力回天,最後因颶風翻船而溺死。自此趙宋宗室結束在中國320年的統治,而南宋153年在南方的偏安統治亦終結。

南宋疆域比北宋時更小。主要是北面疆土被金朝奪去,形成金宋南北对峙,是中國第二個南北朝時期。1139年(紹興九年)宋金第一次和議,雙方確定以改道後的黃河為界,後來金人毀約,出兵取河南、陝西。1141年,宋金議定以淮河為界。第二年又將西部界線調整至大散關(今陝西寶雞市西南)及今秦嶺以南。以後雖有局部變動,基本穩定在秦嶺淮河線。南部和西南邊界無甚變化。

南宋繼承了北宋「強幹弱枝」政策,加強中央集權,在中央地方權力、官僚機構、司法、軍權等方面加強中央集權的一系列措施,以維護國家內部統一、社會穩定和發展經濟。用人制度方面,南宋繼承了北宋「皇帝與士大夫共治天下」之時代。南宋時,取士更是不受出身門第限制,只要不是重刑罪犯,即使是工商、雜類、僧道、農民,甚至是殺豬宰牛的屠戶,都可以應試授官,南宋的科舉登第者多數為平民。

两宋时期是封建社会思想文化环境最为宽松的时期,客观上对经济、社会、文化发展起到了积极的促进作用。

宋金對峙時期,南方經濟明顯超過了北方。主要原因是:南宋初期,金軍雖然多次南攻,但金軍少有渡過長江,南方所受戰爭侵擾較少;大量北方宋人不願在金國統治下生活,移居到南方,使南方人口大為增加。宋朝确立了“农商并重”国策,采取了惠商、恤商政策措施,使社会各阶层纷纷从事商业经营,商品经济呈现出划时代的发展变化,宋代商业已被视同农业,均为创造社会财富的源泉,“士、农、工、商皆百姓之本业”成为社会共识,使宋朝商人的社会地位得到前所未有的提高。

宋朝期間,中國科技得到了空前發展。在南宋時期,大量數學以及科學教材開始廣泛推出。南宋人可以利用水力水車來製作一個有三層樓高之計時器(水運儀象台);首次測量降雨量、降雪量.在北宋時期開始,且在南宋時期興盛的浮橋(橋下是可以浮在水上的東西,多為充氣品,必要時可為高大船隻讓路,由徽宗發明);且拱橋技術出現等。韓信點兵等數學問題也得到了完整的解釋;圓周率的精細度被提高,李约瑟把宋代称为“伟大的代数学家的时代”,认为“中国的代数学在宋代达到最高峰”。沈括著書夢溪筆談,首次進行諧振現象的實驗,領先西方數個世紀。南宋手工業,包括紡織業、造船業、制瓷業、造紙業、印刷業和火器製造業都有較大的發展。

中国历史上的重要发明,一半以上都出现在宋朝,宋代的不少科技发明不仅在中国科技史及世界科技史上也号称第一。火药和火药武器的大规模使用和推广始自南宋。近代的枪炮就是在南宋出现的原始的管形火器基础上发展起来的,南宋还广泛使用威力巨大的火炮作战,反映了南宋火器制造技术的巨大进步。

南宋農業發展顯著,主要包括興建農田水利、大量開墾圩田、推廣農作物優良品種、擴大經濟作物種植面積等方面。水稻種植更加普遍,而棉花種植也從北宋時局限在福建、兩廣一帶擴大到長江淮河流域。

南宋时期,农作物单位面积产量比唐代提高了两三倍,有学者甚至将宋代农作物单位面积产量的大幅提高称为“农业革命”。“苏湖熟,天下足”這句谚语就出现在南宋。

随着棉花种植的推广,到南宋末年,江南一带较为普遍地纺织棉布了。根据南宋诗人艾可叔的《木棉诗》可以看出,当时已经有了纺车、弹弓、织机等工具。南宋的纺织业达到了较高的水平。纺织业规模和技术都大大超过了同时代的女真金國。

南宋地处江南,交通运输多用船只,因而造船业较为发达。明州、泉州、广州等地都是当时的造船中心,能制造大型海船,造船业得到空前发展,如漕船、商船、游船等数量庞大,打造奇巧兼富有创造性 ;海船所采用的多根桅杆,为前代所无 ;战船种类众多,在抗金和抗蒙的战争中发挥了重要作 用。

许多官窑随着一起迁到南方。如著名的修内司官窑设于临安鳳凰山下。景德镇已经发展为全国著名的制瓷业中心,产品销售各地,所烧瓷器极其精美,有“饶玉”之称。

南宋時期由於文化事業發展,印刷業和造紙業都很興盛。當時官府、民間都從事書籍印刷。臨安、福建和四川是印刷業中心。臨安國子監所出版之圖書,稱「監本」,印刷技術高。四川和福建亦有不少書坊。造紙方面,在紙的品種、質和量都有顯著進步,成都、臨安、徽州、池州、平江、建陽等地都是產紙之地。

南宋时期,纸币大量流通,逐渐代替铜钱成为主要交换手段。南宋的纸币分为“交子”和“会子”。交子主要在四川地区使用,会子则分为“东南会子”、“两淮会子”和“湖北会子”三种。不过,南宋后期因为大量发行纸币,造成货币贬值,物价飞涨。

南宋海外贸易繁荣表现在对外贸易港口众多。广州、泉州、临安、明州(浙江宁 波) 等大型海港相继兴起,与外洋通商的港口近20个。北起淮南/东海,中经杭州湾和福、漳、泉金三角,南到广州湾和琼州海峡的南宋万余里海岸线上兴起了一大批港口城镇,这种盛况不仅唐代未见,就是明清亦未能再现。与南宋有外贸关系的国家和地区在60个以上,南宋的出口商品附加值高,不但外贸范围扩大、出口商品数量增加,而且进口商品以原材料与初级制品为主,而出口商品则以手工业制成品为主,附加值高。

宋金两国在淮河设置称作“榷场”的贸易市场。除了榷场,民间私下交易也较多。

由於絲綢之路受西夏所阻隔,西夏在南宋立國時取得了河湟地區(今青海東部),陸上貿易停止。

所有貿易幾乎是經由海上絲綢之路。由於歲幣支出龐大,南宋皇朝內部稅收繁重。經濟幾乎一面倒在與西方貿易之上,阿拉伯商人往來日漸增多,促成海上貿易之繁華。

宋朝成立後不久在各地設置市舶司,與高麗、發展貿易。

宋朝成立後不久在各地設置市舶司,與日本發展貿易。日本則在大宰府監督下進行貿易,建有鴻臚館(接待、交易的機關)。但中日間並無正式的外交貿易,僅限於私人貿易,宋商多行至博多與越前敦賀作交易。

辛弃疾的爱国诗词流传较广。宋理宗时代,朱熹道学(又称程朱理學)得以兴盛。

此時期藝術風格不受外來文化的影響,主要承襲古老的傳統。山水畫仍是此時期重要的繪畫類別,山水畫家有馬遠、夏圭,他們所描繪的是地方山水,飄渺柔和的景致,與北宋山水畫家所畫的險峻山水景致形成對比。此派畫風出自宋高宗的畫院,一般稱「馬夏」,與此時期禪僧生動自然的草草逸筆亦形成強烈對比。

近代的中国文化,其实皆脱胎于两宋文化。著名史学家邓广铭指出: “宋代文化发展所能达到的高度,在从十世纪后半期到十三世纪中叶这一历史时期内,是居于全世界的领先地位的。”


%% -*- coding: utf-8 -*-
%% Time-stamp: <Chen Wang: 2021-11-01 15:59:23>

\section{高宗趙構\tiny(1127-1162)}

\subsection{生平}

宋高宗趙構(1107年6月12日-1187年11月9日),字德基,宋朝第十位皇帝、南宋第一代皇帝(1127年6月12日-1162年7月24日在位),在位35年。北宋皇帝宋徽宗第九子,宋欽宗之弟;曾獲封為「康王」。

在位初期因為眼見金朝強勢,為了保持江山,起用主戰派李綱、岳飛等等。但恐懼將领权力过大,為了強化中央集權,採取求和政策,終於1141年(紹興十一年)達成紹興和議,重用主和派黃潛善、汪伯彥、王倫、秦檜等人,並處死岳飛,罷免李綱、張浚、韓世忠等主戰派大臣。雖然宋高宗之稱臣決策導致南宋偏安之局面,卻成功鞏固了南宋在中國南方的統治,並與金朝形成南北對峙之局面。

大觀元年四月二十七日(1107年5月21日)宋高宗趙構於汴京出生,是宋徽宗第九子,徽宗時被封為康王。

靖康元年春(1126年),金兵圍困汴京,並要求宋以親王、宰相各一人為人質,才肯與宋和談,宋欽宗派趙構以親王身份在金營中為人質,後因金人懷疑其宗室身份,要求更換,故得以回宋。正當趙構獲釋返汴京途中,金兵再次南侵,最初宋欽宗命他往河北召集兵馬勤王,後來金人發現趙構原來是真正親王,忿怒不已,要求宋朝安排趙構為使,才肯再議和,欽宗於是改派他出使金營求和。趙構前往金營時途經河北磁州(今屬河北),被守將宗澤勸阻留下,得以免遭金兵俘虜。此时金兵已跟踪到康王所在,知相州汪伯彦请康王入相州。

靖康元年闰十一月,欽宗命康王为河北兵马大元帅。閏十一月丙辰日(1127年1月9日),金兵攻破汴京開封府,造成「靖康之難」。北宋灭亡。十二月初一壬戌日(1127年1月15日)康王趙構在河北相州開河北兵馬大元帥幕府。趙構自己為河北兵馬大元帥,陳亨伯為元帥,汪伯彥、宗澤為副元帥。有兵万人,分为五军南下。渡河,次大名府。宗泽请直取汴梁。康王从耿南仲及伯彦意见,欲移军东平。十二月乙亥,康王到魏博,庚寅至东平府。

靖康二年二月庚辰,康王如济州。时兵已八万。黄潜善时归之。四月庚辰,康王发济州,趣應天(今河南商丘),刘光世以所部来会。癸未至南京。

此間有過一段插曲:趙構在磁州時,曾由宗澤陪同拜謁了城北崔府君廟,當地稱之為“應王祠”。該廟位於通往邢、洺州的驛道側旁,當時此處“民如山擁”,眾多百姓因為擔心康王取道繼續北行,而聚集在廟宇周圍,號呼勸諫。進入祠廟後,康王抽籤詩,卜得“吉”之籤,廟吏抬應王神輿、擁廟中神馬,請康王乘歸館舍。紛亂中,力主使金的王雲被殺,趙構則留了下來,並於次日返回相州。此事件後卻成為南宋官私記載中極力渲染的“崔府君顯聖”、“泥馬渡康王”故事的緣起;此亦為趙構將來引作為應天登基即位正統性之證明。)

靖康二年三、四月間,徽、欽二帝被金軍虜掠北去。

靖康二年五月初一庚寅日(1127年6月12日),趙構在南京應天府(今河南商丘)登基為帝,改元「建炎」。建炎改元後,宋高宗遙尊被擄到金國的其母親韋氏為「宣和皇后」,封自己的外祖父韋安道為郡王,親屬三十人均任官職。並且從此不斷派遣使者到金國求和要迎韋氏回南宋。

建炎元年十月丁巳初一日,宋高宗离南京南下扬州。癸未到达扬州;金人听闻后,决计大举南伐。建炎三年一月韩世忠在沐阳溃军,金军快速南下。至金数百骑兵到扬州西北之天长。壬子,金人破天长军。赵构得内侍探报,即穿盔甲乘马出门,出走扬州,而百官宰相不知。高宗渡江至京口。再次镇江;至甲寅再次长州;乙卯次无锡;丙辰次平江府;壬戌至杭州。而次月金兵并未过江。

建炎三年三月,因禁军将领对人事安排等不满,发生苗刘兵变,宋高宗被迫禅让皇位于皇子。四月,高宗在勤王大军的进发下,复辟。复辟后举行仁宗法度,录用元祐党人,多所改易政策。四月,丁卯,赵构发杭州前往江宁(建康),以谋恢复。

宋高宗被金兵追殺,一度在海上飄泊,至紹興八年(1138年)正式定都於臨安(今浙江杭州),建炎南渡完成。

紹興七年(1137年),高宗生父宋徽宗的死訊傳到南宋。『帝號慟,諭輔臣曰:「宣和皇后春秋高,朕思之不遑甯處,屈己請和,正為此耳。」(高宗號哭,對大臣說:「我母親宣和皇后年歲已經大了,我思念她到了坐不安的地步,我委屈自己向金國求和,正是為了這事。」)翰林學士朱震引用唐德宗李適的事,請高宗遙尊韋氏為皇太后,宋高宗聽從。

紹興八年(1138年),在宋使王倫的成功外交下,金朝撤銷偽齊,把包含東京開封等三京(東京、西京、南京)之地的河南、陝西歸還給南宋,但高宗生母韋太后尚未歸還。

紹興十年(1140年),金朝撕毀協約,重新攻佔陝西、河南之地。金軍主帥完顏宗弼(兀朮)先在開封正南的順昌敗於劉錡所部的「八字軍」,再於開封西南的郾城和穎昌,在女真精銳部隊所拿手的騎兵對陣中兩次敗於岳飛的岳家軍,只在開封東南面的淮西亳州、宿州一帶戰勝了宋軍中最弱的張俊一軍,在宋高宗以「十二道金牌」召回岳家軍前,金軍已被壓縮到開封東部和北部。

紹興十一年(1141年)二月,金熙宗對南宋示好,將死去的宋徽宗追封為天水郡王,將在押的宋欽宗封為天水郡公。第一提高了級別,原來封徽宗為二品昏德公,追封郡王升為一品,原來封欽宗為三品重昏侯,現封公爵升為二品。第二是去掉了原封號中的污侮含義。第三是以趙姓天水族望之郡作為封號,以示尊重。同時,在宋軍中最強大的岳家軍根本未參戰的情況下,完顏宗弼的金國最精銳的部隊又在淮西柘皋先敗於張俊部下楊沂中和劉錡的聯軍,後來雖然因為張俊搶功調走劉錡,完顏宗弼在濠州勝宋軍中最弱的張俊一軍,但由於韓世忠軍和岳家軍趕到,完顏宗弼不得不退軍北上。

四月下旬,宋高宗解除了岳飛、韓世忠、劉錡、楊沂中、張俊等大將的兵權,為《紹興和議》做好了準備。十月,南宋派魏良臣赴金,提出要議和。

十一月,金國派蕭毅、邢具瞻為審議使,隨魏良臣回南宋,提出議和條件。此時高宗生母韋氏託人將一封信送到趙構手裏。「洪皓在燕,求得(韋)后書,遣李微持歸。帝大喜曰:「遣使百輩,不如一書。」遂加(李)微官。金人遣蕭毅、邢具瞻來議和,帝曰:『朕有天下,而養不及親。徽宗無及矣!今立誓信,當明言歸我(韋)太后,朕不恥和。不然,朕不憚用兵!』(『我擁有天下,但卻不能贍養親人,我父親徽宗已經死了!現在我發誓,我要公開要求金國歸還我母親韋太后,我不以議和為恥。不然的話,我不怕向金國用兵!』),蕭毅等還,帝又語之曰:『(韋)太后果還,自當謹守誓約。如其未也,雖有誓約,徒為虛文。』」(「如果我母親韋太后果然能回南宋,自當謹守我們訂的和議誓約。如果回不來,有和議誓約也是一紙空文。」)當月,《紹興和議》最後的書面內容即達成。

十二月末除夕夜(1142年1月27日),宋高宗殺害岳飛與其子岳雲、部將張憲於臨安(今杭州),據《宋史》載這是為了滿足完顏宗弼為《紹興和議》所設的前提以防止岳飛的十萬岳家軍攻入黃河以北。

至此,高宗以稱臣賠款,割讓從前被岳飛收復的唐州、鄧州以及商州、秦州的大半為代價,簽定紹興和議。宋金東以淮河,西以大散關為界,南宋正式放棄上次和約所獲得的陝西、河南領土。宋高宗也立刻成功地迎回生母韋氏。《宋史·高宗本紀》記載:紹興十二年(1142年)夏四月丁卯(5月1日),「(韋)皇太后偕梓宮(徽宗靈柩)發五國城,金遣完顏宗賢護送梓宮,高居安護送皇太后」。按照當時信息的傳遞方式,岳飛於紹興十一年除夕夜(1142年1月27日)被殺,南宋使節立刻於紹興十二年(1142年)正月帶著正式照函從岳飛被殺的臨安(今杭州)去金國禁錮宋欽宗和韋氏的五國城(今黑龍江哈爾濱市依蘭縣依蘭鎮五國城村)接人,韋氏四月丁卯(5月1日)即啟程回宋,八月壬午(9月13日),韋氏到達宋都臨安。從正月初一到八月壬午,除了用時在行程腳力上,沒有絲毫拖延。韋氏離開五國城前,曾答應欽宗回南方後努力營救欽宗回去,但高宗可能考慮到自己已經不育而絕後,不希望有生育能力的兄長欽宗回來爭奪皇位繼承權,所以欽宗就永遠被留在北方。

紹兴和议達成后,秦桧专权弄政长達十五年,高宗一方面对秦桧放任,另一方面,处处对秦桧提防。秦桧死后,高宗始打擊秦桧餘党。

紹興三十一年(1161年),《紹興和議》被金朝皇帝完顏亮撕毀,金兵再次南侵,是为采石之战,宋军以少勝多擊退金兵。

紹興三十二年六月十一日(1162年7月24日),高宗以「倦勤」想多休養為由,禪讓於養子建王趙眘,是為宋孝宗,終結了宋太宗一脈自976年起長達186年的統治,回歸宋太祖一脈,直至南宋滅亡。

宋高宗本有一子趙旉,但因苗劉兵變受到驚嚇而病逝,得年僅兩歲。而據說高宗建炎南渡後也因為兵亂而驚嚇過度,患有陽痿,不能人道,之後未能再生下任何子女,故須在宋室子姪中選出皇位繼任人。身為宋太宗後裔的宋高宗,之所以立宋太祖的後裔趙眘為繼承人,一來宋太宗的後裔大多在靖康之難被金人虜去,另外根據《宋史》的記載,傳說是因為宋太祖顯靈託夢,野史記載高宗被宋太祖託夢稱「自從你的祖先攝用計謀,佔據我的位置很久了,以至於如今天下寥落的局面,是時候把位置還給我了。」故宋高宗過繼太祖八世孫作為養子,並立為太子;宋史中也有相似的記載,但稱孟太后被託夢。雖然是禪讓,主要決定权还是在高宗,尤其在議和問題上。宋孝宗趙眘登基後馬上為岳飛平反和肅清秦檜餘黨,身為太上皇的高宗並未阻撓,而且退位後的高宗,與君臨天下的孝宗關係相當好,父慈子孝。

淳熙十四年十月初八日(1187年11月9日),宋高宗去世,享壽八十歲,孝宗悲痛不已,持續守喪三年後,也自行退位。

宋高宗同其父宋徽宗一樣,頗有藝術天份,是傑出的書法家;自言「……凡五十年間,非大利害相仿,未始一日舍筆墨」,初學黃庭堅,後改學米芾,至終以追摹魏晉法度和王羲之、王獻之父子,流傳有《賜岳飛手敕》及《真草嵇康養生論書卷》。元朝書法家趙孟頫早年即以宋高宗書法為榜樣。

宋高宗與金朝議和,穩固南宋對中國南方的統治,議和一說在於經濟因素,所謂三軍未動,糧草先行;野蠻民族是以燒殺擄掠為錢糧來源,文明國家卻是打仗燒錢。宋高宗若不先安內,只怕民變四起,連半壁江山都沒了;歲幣議和,可緩和兩國關係,讓國家有喘息的機會;另一方面,宋高宗可以掌握軍權,壓制將領對軍隊的影響力。

《續資治通鑒》中:「康王入,毅然請行,曰:“敵必欲親王出質,臣為宗社大計,豈應辭避!”欽宗立,改元靖康,人拆其字,謂“十二月立康王”也。資性郎悟,好學強記,日誦千餘言,挽弓至一石五斗。」其他含有關於宋高宗節儉、不迷信祥瑞、不好女色、潛心治國、文才武德具備等描述。

宋高宗為保住皇位,在位初期不惜創造傳說,使天下人相信其正當正統地位,以掩飾自己「銜命出和,已作潛身之計;提師入衛,反為護己之資。忍視父兄甘為俘虜」。因金兵追擊而貪生怕死地逃命,故被後世戲稱為「逃跑皇帝」。及後他定都臨安後,為求偏安,保持半壁江山的統治,不惜把岳飛等主战派大臣殺害,以與金朝達成和議,成为后世評價的重要污点。

當時詩人林升在宿新住宿徐公店,在牆上提詩《題臨安邸》諷刺當朝的統治者曰:山外青山樓外樓,西湖歌舞幾時休?暖風熏得遊人醉,直把杭州作汴州!

元朝官修正史《宋史》脱脱等的評價是:“昔夏后氏传五世而后羿篡,少康复立而祀夏;周传九世而厉王死于彘,宣王复立而继周;汉传十有一世而新莽窃位,光武复立而兴汉;晋传四世有怀、愍之祸,元帝正位于建邺;唐传六世有安、史之难,肃宗即位于灵武;宋传九世而徽、钦陷于金,高宗缵图于南京:六君者,史皆称为中兴,而有异同焉。夏经羿、浞,周历共和,汉间新室、更始,晋、唐、宋则岁月相续者也。萧王、琅琊皆出疏属,少康、宣王、肃宗、高宗则父子相承者也。至于克复旧物,则晋元与宋高宗视四君者有余责焉。高宗恭俭仁厚,以之继体守文则有余,以之拨乱反正则非其才也。况时危势逼,兵弱财匮,而事之难处又有甚于数君者乎?君子于此,盖亦有悯高宗之心,而重伤其所遭之不幸也。然当其初立,因四方勤王之师,内相李纲,外任宗泽,天下之事宜无不可为者。顾乃播迁穷僻,重以苗、刘群盗之乱,权宜立国,确虖艰哉。其始惑于汪、黄,其终制于奸桧,恬堕猥懦,坐失事机。甚而赵鼎、张浚相继窜斥,岳飞父子竟死于大功垂成之秋。一时有志之士,为之扼腕切齿。帝方偷安忍耻,匿怨忘亲,卒不免于来世之诮,悲夫!”

明末清初儒者王夫之在《宋論》一書中如此評價高宗:“高宗之畏女真也,窜身而不耻,屈膝而无惭,直不可谓有生人之气矣。乃考其言动,察其志趣,固非周赧、晋惠之比也。何以如是其馁也?李纲之言,非不知信也;宗泽之忠,非不知任也;韩世忠、岳飞之功,非不知赏也;吴敏、李棁、耿南仲、李邦彦主和以误钦宗之罪,非不知贬也。而忘亲释怨,包羞丧节,乃至陈东、欧阳澈拂众怒而骈诛于市,视李纲如仇仇,以释女直之恨。是岂汪、黄二竖子之能取必于高宗哉?且高宗亦终见其奸而斥之矣。抑主张屈辱者,非但汪、黄也。张浚、赵鼎力主战者,而首施两端,前却无定,抑不敢昌言和议之非。则自李纲、宗泽而外,能不以避寇求和为必不可者,一二冗散敢言之士而止。以时势度之,于斯时也,诚有旦夕不保之势,迟回葸畏,固有不足深责者焉。苟非汉光武之识量,足以屡败而不挠,则外竞者中必枵,况其不足以竞者乎?高宗为质于虏廷,熏灼于剽悍凶疾之气,俯身自顾,固非其敌。已而追帝者,滨海而至明州,追隆祐太后者,薄岭而至皂口,去之不速,则相胥为俘而已。君不自保,臣不能保其君,震慑无聊,中人之恒也。亢言者恶足以振之哉? ”

清高宗乾隆帝于乾隆五十五年得玄孙,为庆贺五代同堂,特地御制诗一首:“古稀六帝三登八,所鄙宋梁所慕元,惟至元称一代杰,逊乾隆看五世孙”,意即年过古稀(70岁)的皇帝只有6个(包括汉武帝、唐玄宗、明太祖),其中只有三个活过了80岁,即梁武帝、宋高宗、元世祖,乾隆帝只敬仰元世祖忽必烈,而鄙夷梁武帝和宋高宗。而梁武帝和宋高宗皆是偏安南方的开国皇帝。

現代王曾瑜批評宋高宗,違反宋太祖「不誅大臣、言官」之誓約,殺上書言事之陳東與歐陽澈,以鉗制天下異議之口,卻竊取了「中興之主」之美譽,另外也將岳飛等主战派大臣殺害,是宋朝歷代皇帝中,唯一一位違反宋太祖祖訓的皇帝,為後人所唾罵。

\subsection{建炎}


\begin{longtable}{|>{\centering\scriptsize}m{2em}|>{\centering\scriptsize}m{1.3em}|>{\centering}m{8.8em}|}
  % \caption{秦王政}\
  \toprule
  \SimHei \normalsize 年数 & \SimHei \scriptsize 公元 & \SimHei 大事件 \tabularnewline
  % \midrule
  \endfirsthead
  \toprule
  \SimHei \normalsize 年数 & \SimHei \scriptsize 公元 & \SimHei 大事件 \tabularnewline
  \midrule
  \endhead
  \midrule
  元年 & 1127 & \tabularnewline\hline
  二年 & 1128 & \tabularnewline\hline
  三年 & 1129 & \tabularnewline\hline
  四年 & 1130 & \tabularnewline
  \bottomrule
\end{longtable}

\subsection{绍兴}

\begin{longtable}{|>{\centering\scriptsize}m{2em}|>{\centering\scriptsize}m{1.3em}|>{\centering}m{8.8em}|}
  % \caption{秦王政}\
  \toprule
  \SimHei \normalsize 年数 & \SimHei \scriptsize 公元 & \SimHei 大事件 \tabularnewline
  % \midrule
  \endfirsthead
  \toprule
  \SimHei \normalsize 年数 & \SimHei \scriptsize 公元 & \SimHei 大事件 \tabularnewline
  \midrule
  \endhead
  \midrule
  元年 & 1131 & \tabularnewline\hline
  二年 & 1132 & \tabularnewline\hline
  三年 & 1133 & \tabularnewline\hline
  四年 & 1134 & \tabularnewline\hline
  五年 & 1135 & \tabularnewline\hline
  六年 & 1136 & \tabularnewline\hline
  七年 & 1137 & \tabularnewline\hline
  八年 & 1138 & \tabularnewline\hline
  九年 & 1139 & \tabularnewline\hline
  十年 & 1140 & \tabularnewline\hline
  十一年 & 1141 & \tabularnewline\hline
  十二年 & 1142 & \tabularnewline\hline
  十三年 & 1143 & \tabularnewline\hline
  十四年 & 1144 & \tabularnewline\hline
  十五年 & 1145 & \tabularnewline\hline
  十六年 & 1146 & \tabularnewline\hline
  十七年 & 1147 & \tabularnewline\hline
  十八年 & 1148 & \tabularnewline\hline
  十九年 & 1149 & \tabularnewline\hline
  二十年 & 1150 & \tabularnewline\hline
  二一年 & 1151 & \tabularnewline\hline
  二二年 & 1152 & \tabularnewline\hline
  二三年 & 1153 & \tabularnewline\hline
  二四年 & 1154 & \tabularnewline\hline
  二五年 & 1155 & \tabularnewline\hline
  二六年 & 1156 & \tabularnewline\hline
  二七年 & 1157 & \tabularnewline\hline
  二八年 & 1158 & \tabularnewline\hline
  二九年 & 1159 & \tabularnewline\hline
  三十年 & 1160 & \tabularnewline\hline
  三一年 & 1161 & \tabularnewline\hline
  三二年 & 1162 & \tabularnewline
  \bottomrule
\end{longtable}


\section{元懿太子赵旉生平}

元懿太子赵\xpinyin*{旉}(1127年7月23日-1129年7月28日),是南宋高宗唯一的親生儿子,潘贤妃所生。

建炎元年六月十三辛未日(1127年7月23日)生于南京应天府(今河南商丘),九月十二己亥日(10月19日)拜检校少保、集庆军节度使,封魏国公。

建炎三年(1129年),金人入侵淮南,三月初五癸未日(3月26日),高宗在扬州遇上苗傅、刘正彥作乱,逼迫高宗退位,拥立魏国公赵旉为傀儡皇帝,十一己丑日(4月1日)改年號為明受,是為苗劉兵變。宰相张浚等闻知,举兵向苗傅、劉正彦二人问罪。四月初一戊申日(4月20日),高宗復位。初三庚戌日(4月22日),苗傅、劉正彦二人兵败逃走,年号改回建炎。

高宗于四月二十丁卯日(5月9日)迁往临安府(今浙江杭州),四月二十五壬申日(5月14日),趙旉被册立为皇太子。五月初八乙酉日(5月27日)随父亲一起抵达建康(今江苏南京)。

赵旉立为太子后不久就生病,宫人不小心踢到金炉发出声响,太子受到驚吓,随后病情转剧,最后于同年七月十一丁亥日(7月28日)去世,得年仅2岁。谥号为元懿太子。

靖康南渡及苗刘兵变後,懷疑宋高宗患有陽萎,之後並未生育任何的亲生子女。

元懿太子赵旉的年号明受(1129年三月-四月),南宋使用这个年号共2个月。



%%% Local Variables:
%%% mode: latex
%%% TeX-engine: xetex
%%% TeX-master: "../Main"
%%% End:

%% -*- coding: utf-8 -*-
%% Time-stamp: <Chen Wang: 2021-11-01 15:56:00>

\section{孝宗趙眘\tiny(1162-1189)}

\subsection{生平}

宋孝宗趙眘(1127年11月27日-1194年6月28日、眘,「慎」異體字,shèn),南宋第二位皇帝(1162年7月24日—1189年2月18日在位),曾名伯琮、瑗、玮,字元瑰,一字元永,他是宋太祖之幼子趙德芳的後裔,宋高宗養子,宋太祖的子孫。自宋朝的帝位落入宋太宗的子孫之手後,恢復由開國皇帝宋太祖的後裔繼承,時隔186年。孝宗是南宋较有作为的皇帝,同時期的皇帝是金朝的金世宗。

绍兴三十二年(1162年),高宗禪位予赵眘,是为宋孝宗,宋孝宗在位27年,期间与金達成隆興和议。淳熙十六年(1189年)孝宗逊位,讓位予兒子宋光宗趙惇。绍熙五年(1194年)孝宗病逝,终年67岁。葬于永阜陵。

建炎元年十月廿二出生。六世祖為太祖趙匡胤的四子秦康惠王趙德芳。父亲為遠支宗親趙子偁,後封為秀安僖王。

由於高宗的唯一兒子元懿太子夭折後再沒有嗣子,所以只好從宗室中選擇繼承人,有野史稱宋高宗受宋太祖託夢,稱「汝祖自攝謀,據我位久,至於天下寥落,是當還我位。」故高宗過繼了太祖七世孫作為養子,並立為太子;《宋史》中也有相似的記載,但為孟太后受託夢。

宋高宗在苗劉兵變後,患上陽痿而不育,又因為民間傳说多年的宋太宗燭影斧聲,強奪宋太祖之位,且金太宗完顏吳乞買與宋太祖的畫像神似,吳乞買是宋太祖投胎(一說金太祖完顏阿骨打是宋太祖投胎),要來報仇,滅了宋太宗一家,導致北宋亡國及失去半壁江山,加上為了阻止欽宗后代及部分宋太宗後裔继承,且有見金朝同样因太祖系及太宗系的皇位继承導致自相殘殺,故高宗在有生之年将帝位傳承給宋太祖的後代,以阻止相關情況发生。

绍兴二年五月,高宗于太祖系宗室中选了赵伯琮养于宫中。绍兴三年(1133年)二月为和州防御使,赐名赵瑗,改贵州防御使。五年(1135年)五月,用左仆射赵鼎议,在宫中设立书院教育他,建成后以书院为资善堂。孝宗读书强记,天资特异。诏授保庆军节度使,封建国公。因传言金朝欲在北宋旧都开封拥立钦宗太子赵谌为傀儡皇帝挑战高宗法统,大将岳飞密请高宗立赵瑗为皇太子以让金人的图谋落空,高宗也安排岳飞与赵瑗相见,岳飞感叹赵瑗是兴复宋室之主。

十二年(1142年)正月,赵瑗封普安郡王。此后,宋高宗另一位养子赵璩亦封郡王。两人官属礼制相同,号为东、西府。太子多年未定,内外颇以为疑。三十年(1160年)二月立为皇子,改名赵玮,进封建王。赵璩则称“皇侄”,两人名分终定。

绍兴三十二年(1162年)五月立为皇太子,改名赵眘。六月,高宗禅位,孝宗登基,定年号隆兴,立志光复中原,收复河山,遂将岳飞平反並追封为鄂国公,肅清秦桧餘黨,并且命令老将张浚北伐中原,但在符离遭遇金军突襲大败。接着金军趁胜追击,南宋军队损失惨重,此後雙方互有勝負,但金無法越過長江,宋亦未能渡黃河。宋孝宗被迫于隆兴二年(1164年)和金国金世宗签订“隆兴和议”,次年改元“乾道”。

乾道年间,由于没有战事的干扰,加上高宗除对金和战外較少干政,孝宗专心理政,百姓富裕,五谷丰登,太平安乐,一改高宗朝时贪腐的局面。由于宋孝宗治国有方,所以使南宋出现乾淳之治(乾道,淳熙)的小康局面。

宋孝宗有一批宠臣,如曾觌、龙大渊、张说等人,与宦官梁珂等“相与盘结”,“士大夫无耻者争附之”,被列入《佞幸传》。

淳熙十四年(1187年)十月,高宗死於德壽宮中,孝宗聽聞後失聲痛哭,兩天不能進食,又表示要服喪三年。孝宗为了服丧,让太子赵惇参预政事。淳熙十六年二月初二(1189年2月18日)又禅讓帝位予太子,太子即位,是为宋光宗。守孝三年后,孝宗退位,自称太上皇,闲居慈福宮,後改名重华殿,继续为高宗服丧。

光宗由於皇后李鳳娘挑撥,与孝宗不睦,长期不去探望孝宗,孝宗为此闷闷不乐而起病。最终在宋光宗绍熙五年六月初九(1194年6月28日),宋孝宗逝于临安重华殿內,享壽六十七歲。

元朝官修正史《宋史》脱脱等的評價是:“高宗以公天下之心,择太祖之后而立之,乃得孝宗之贤,聪明英毅,卓然为南渡诸帝之称首,可谓难矣哉。即位之初,锐志恢复,符离邂逅失利,重违高宗之命,不轻出师,又值金世宗之立,金国平治,无衅可乘,然易表称书,改臣称侄,减去岁币,以定邻好,金人易宋之心,至是亦寝异于前日矣。故世宗每戒群臣积钱谷,谨边备,必曰:『吾恐宋人之和,终不可恃。』盖亦忌帝之将有为也。天厌南北之兵,欲休民生,故帝用兵之意弗遂而终焉。然自古人君起自外藩,入继大统,而能尽宫庭之孝,未有若帝。其间父子怡愉,同享高寿,亦无有及之者。终丧三年,又能却群臣之请而力行之。宋之庙号,若仁宗之为『仁』,孝宗之为『孝』,其无愧焉,其无愧焉!”
傅樂成《中國通史》:「孝宗也是南宋有數的賢君,他伐金雖然失敗,但在外交上與金人力爭,終於除去對金的臣屬關係,不能說是毫無收穫。約成之後,他仍主備禦金人,無晏安之意,假若不是金世宗在位,金勢極強,他必不甘於信守這個不平等條約。內政方面,也沒有多大缺失。他以遠支宗室,繼承大統,對高宗能竭盡孝道,這一點也是值得稱道的。」


\subsection{隆兴}


\begin{longtable}{|>{\centering\scriptsize}m{2em}|>{\centering\scriptsize}m{1.3em}|>{\centering}m{8.8em}|}
  % \caption{秦王政}\
  \toprule
  \SimHei \normalsize 年数 & \SimHei \scriptsize 公元 & \SimHei 大事件 \tabularnewline
  % \midrule
  \endfirsthead
  \toprule
  \SimHei \normalsize 年数 & \SimHei \scriptsize 公元 & \SimHei 大事件 \tabularnewline
  \midrule
  \endhead
  \midrule
  元年 & 1163 & \tabularnewline\hline
  二年 & 1164 & \tabularnewline
  \bottomrule
\end{longtable}

\subsection{乾道}

\begin{longtable}{|>{\centering\scriptsize}m{2em}|>{\centering\scriptsize}m{1.3em}|>{\centering}m{8.8em}|}
  % \caption{秦王政}\
  \toprule
  \SimHei \normalsize 年数 & \SimHei \scriptsize 公元 & \SimHei 大事件 \tabularnewline
  % \midrule
  \endfirsthead
  \toprule
  \SimHei \normalsize 年数 & \SimHei \scriptsize 公元 & \SimHei 大事件 \tabularnewline
  \midrule
  \endhead
  \midrule
  元年 & 1165 & \tabularnewline\hline
  二年 & 1166 & \tabularnewline\hline
  三年 & 1167 & \tabularnewline\hline
  四年 & 1168 & \tabularnewline\hline
  五年 & 1169 & \tabularnewline\hline
  六年 & 1170 & \tabularnewline\hline
  七年 & 1171 & \tabularnewline\hline
  八年 & 1172 & \tabularnewline\hline
  九年 & 1173 & \tabularnewline
  \bottomrule
\end{longtable}

\subsection{淳熙}

\begin{longtable}{|>{\centering\scriptsize}m{2em}|>{\centering\scriptsize}m{1.3em}|>{\centering}m{8.8em}|}
  % \caption{秦王政}\
  \toprule
  \SimHei \normalsize 年数 & \SimHei \scriptsize 公元 & \SimHei 大事件 \tabularnewline
  % \midrule
  \endfirsthead
  \toprule
  \SimHei \normalsize 年数 & \SimHei \scriptsize 公元 & \SimHei 大事件 \tabularnewline
  \midrule
  \endhead
  \midrule
  元年 & 1174 & \tabularnewline\hline
  二年 & 1175 & \tabularnewline\hline
  三年 & 1176 & \tabularnewline\hline
  四年 & 1177 & \tabularnewline\hline
  五年 & 1178 & \tabularnewline\hline
  六年 & 1179 & \tabularnewline\hline
  七年 & 1180 & \tabularnewline\hline
  八年 & 1181 & \tabularnewline\hline
  九年 & 1182 & \tabularnewline\hline
  十年 & 1183 & \tabularnewline\hline
  十一年 & 1184 & \tabularnewline\hline
  十二年 & 1185 & \tabularnewline\hline
  十三年 & 1186 & \tabularnewline\hline
  十四年 & 1187 & \tabularnewline\hline
  十五年 & 1188 & \tabularnewline\hline
  十六年 & 1189 & \tabularnewline
  \bottomrule
\end{longtable}



%%% Local Variables:
%%% mode: latex
%%% TeX-engine: xetex
%%% TeX-master: "../Main"
%%% End:

%% -*- coding: utf-8 -*-
%% Time-stamp: <Chen Wang: 2021-11-01 15:56:07>

\section{光宗赵惇\tiny(1189-1194)}

\subsection{生平}

宋光宗赵惇(1147年9月30日-1200年9月17日),南宋第三位皇帝(1189年2月18日—1194年7月24日在位),宋孝宗第三子。42岁時受孝宗禅位而登基(亦是最老即位的宋帝),由於皇后李鳳娘的挑撥,與父親宋孝宗失和,趙汝愚、韓仛冑等大臣不滿,宋孝宗死後,在隆慈太皇太后的支持下,光宗被迫內禪大寶予其子宋寧宗,史稱宋光宗內禪。退位六年後駕崩,享年53岁,谥号「循道宪仁明功茂德温文顺武圣哲慈孝皇帝」。

绍兴十七年(1147年)九月乙丑生,郭皇后之子,二十年(1150年)赐名赵惇。宋孝宗即位,封恭王,封邑恭州(今重慶市)。

乾道七年(1171年)二月立为皇太子,當時孝宗因趙惇性格與自己相似,故立為太子,與此同時,孝宗也擔憂太子的政治能力,但是淳熙十六年(1189年)二月,孝宗依然禪位予光宗,自稱「壽皇」,以明年为绍熙元年。

光宗是宋朝皇帝中比较平庸的一位。他登基時42岁,並不算年老,卻体弱多病。心理上,也没有安邦治国之才,听取奸臣谗言,罢免辛弃疾等主战派大臣,又让当时著名的妒妇、心狠手辣的皇后李鳳娘干政,自己對朝政的掌握力不斷下降。

紹熙四年,光宗將兩浙東路馬步軍總管姜特立召至杭州行在,遭到丞相留正的強烈反對,留正隨後出走,直到當年十一月,光宗將姜特立遣回本官,留正方才回朝。但光宗在位的五年時間,基本延續了高宗、孝宗兩朝賑濟災民的政策,也能對一些官吏瀆職的現象予以打擊,如文思院監常良孫因貪污,不僅自身受到貶黜,薦舉其人的宰相周必大也遭到了降爵處分。經濟方面,淮河一帶在光宗朝繼續嚴格實行紙鈔第一政策,一切銀錢都不得進入兩淮地區。湖北一部份地區在紹熙後期開放了鐵錢。

光宗由於皇后李鳳娘的挑撥,素來与孝宗不和,宋孝宗退位后,自稱「壽皇」,閒居宮中,皇后李鳳娘教唆光宗,光宗长期不去探望。李皇后謀殺了有孕在身的黃貴妃,光宗知道之後,又生病了,壽皇送來了藥丸,但李皇后卻造謠說那是毒藥,離間光宗與壽皇父子關係。有一次,光宗宴請大臣,大臣請光宗探望壽皇,李皇后知道了以後立即阻止。绍熙五年(1194年),壽皇得病,宋光宗不去探望壽皇,也不讓他人探望,最终朝臣拒绝见光宗,集体前往朝见壽皇,光宗才被迫妥协。

壽皇病逝时,光宗不服丧。壽皇的喪禮需要身為長子的宋光宗主持,光宗也不主持,大臣們都極為不滿。樞密使趙汝愚则聯合趙彥逾、葉適、徐誼、郭杲、韓仛冑等人策畫政變,由隆慈太皇太后(憲聖慈烈皇后)的外甥韓仛冑請隆慈太皇太后垂簾聽政,又以防不测,命殿帅郭杲,步帅阎仲连夜带兵控制南皇宫南北,傅昌朝暗中制造黄袍,趙汝愚兵谏逼迫光宗退位。使得尚未成为太子的赵扩,就受禪稱帝,是为宋宁宗,光宗闲居临安行在寿康宫,自称“太上皇”。

庆元六年八月初八(1200年9月17日),憂鬱而駕崩,葬于永崇陵(今浙江绍兴东南35里处宝山)。

宋光宗在太子位已经很长時間,却仍不见宋孝宗传位给他。一次他前往重华宫侍奉孝宗,顺便向孝宗试探道:“有人给兒臣送来了染胡须的药,兒臣却不敢用。”孝宗清楚儿子的用意,就说:“我正要向天下显示你的老成,要染鬍鬚的药幹甚麼呢?”

一次,光宗的黄贵妃病了,御医用了许多药也不见效果。光宗无奈只好张榜求医。一位江湖郎中揭榜进宫,建议以山楂和冰糖煎熬服用。贵妃按照郎中的方法服用,果然就如期痊愈了。此后,该方法传入民间,逐渐演变成今天的冰糖葫芦。

元朝官修正史《宋史》脱脱等的評價是:“光宗幼有令闻,向用儒雅。逮其即位,总权纲,屏嬖幸,薄赋缓刑,见于绍熙初政,宜若可取。及夫宫闱妒悍,内不能制,惊忧致疾。自是政治日昏,孝养日怠,而乾、淳之业衰焉。”


\subsection{绍熙}


\begin{longtable}{|>{\centering\scriptsize}m{2em}|>{\centering\scriptsize}m{1.3em}|>{\centering}m{8.8em}|}
  % \caption{秦王政}\
  \toprule
  \SimHei \normalsize 年数 & \SimHei \scriptsize 公元 & \SimHei 大事件 \tabularnewline
  % \midrule
  \endfirsthead
  \toprule
  \SimHei \normalsize 年数 & \SimHei \scriptsize 公元 & \SimHei 大事件 \tabularnewline
  \midrule
  \endhead
  \midrule
  元年 & 1190 & \tabularnewline\hline
  二年 & 1191 & \tabularnewline\hline
  三年 & 1192 & \tabularnewline\hline
  四年 & 1193 & \tabularnewline\hline
  五年 & 1194 & \tabularnewline
  \bottomrule
\end{longtable}



%%% Local Variables:
%%% mode: latex
%%% TeX-engine: xetex
%%% TeX-master: "../Main"
%%% End:

%% -*- coding: utf-8 -*-
%% Time-stamp: <Chen Wang: 2019-12-26 10:46:25>

\section{宁宗\tiny(1194-1224)}

\subsection{生平}

宋寧宗趙擴(1168年11月18日-1224年9月18日),南宋第四位皇帝(1194年7月24日—1224年9月18日在位)在位30年,享年56岁,宋光宗之次子,李鳳娘所生。寧宗本人頗為好學,即位初年召朱熹入宮講學,受朱熹影響很深,但寧宗政治能力并不十分出色。

寧宗在位前期,太師韓侂胄打壓理學,在韓侂胄死后,官方恢复了理学地位。縱觀宋宁宗时期,大規模宋金戰爭發生了兩次,第一次是開禧初年韓侂胄伐金,最終不能戰勝金國,從而簽訂了嘉定和議。第二次宋金戰爭從嘉定十年開始一直持續到嘉定十四年三月,戰爭波及了長江上游至下游所有地區,最終宋金都沒能獲勝。

宋宁宗生于乾道四年(1168年)十月丙午,五年(1169年)五月赐名赵扩。淳熙五年(1178年)十月封英国公,十二年(1185年)三月封平阳郡王,十六年(1189年)三月进封嘉王。绍熙五年(1194年)为太子,不久紹熙內禪受禪位,其父宋光宗稱為太上皇。紹熙內禪名义上是光宗自行禅位,实际上是赵汝愚、趙彥逾、葉適、徐誼、韓侂冑等人,在獲得隆慈太皇太后支持所造成的宫廷政变,廢除宋光宗,改立宋宁宗。

宋宁宗继位后,宗室宰相赵汝愚与外戚韩侂胄不和,兩者互相争斗。最后韩侂胄使用「宗室不得為宰執」的祖宗家法,让宋宁宗罢免了赵汝愚,并且将其所提倡的理学称为伪学,对理学家造成了打击,造成庆元党禁。这个政策一直维持到1202年,韓侂冑後悔和葉適建言才解除禁制。韩从此成为南宋举足轻重的人物。他的地位和权力远高出一般的宰相。

宋宁宗在南宋统治初期由于韩侂胄的作用对金朝持对抗态度,他追封岳飞为鄂王,剥夺秦桧的所有封职。1206年,韩侂胄北伐战败后宋宁宗改变了政策。1207年11月,其皇后杨氏与史弥远一起秘密策划利用韩侂胄战败的机会谋杀了韩侂胄,并且将韩侂胄的首级送往金朝谢罪。1208年,在史弥远的操纵下,宋宁宗与金朝达成了嘉定和议,向金朝皇帝称伯,自己称侄,將原來的二十萬兩歲幣增至三十萬兩,絹三十萬匹,並且支付金國軍費三百萬兩。韩侂胄死后,史弥远成为了宋宁宗的宰相兼枢密使,独揽大政,同時史弥远恢复了秦桧的王爵和官职。

宋宁宗先后有9个儿子,但是在未成年時就夭折,以後的繼承人問題都有史彌遠介入與操控。曾以養子趙詢為太子,但早死(1207年至1220年在位)。後立养子赵竑为太子(1221年至1224年在位),但是因为赵竑对史弥远专权不满,因此宋宁宗死后史弥远矯詔立赵昀为皇帝。

宋宁宗於嘉定十七年八月初三(1224年9月17日)崩於福寧殿,次年三月葬於永茂陵。

宋寧宗十分重視台諫,但由於他缺乏一定的辨別是非的能力,使得這一政治體制成為權臣控制自己的工具。韓侂胄死後,宋寧宗進行了革除韓侂胄弊政的政治更化。歷史上稱為“嘉定更化”。這些措施包括廣開言路、修正國史、清洗韓黨、平凡昭雪等。但由於寧宗用人失誤,使得更化效果並不理想,適得其反。

1199年五月,寧宗頒佈由楊忠輔創制的新曆法,並賜名為《統天曆》。該曆法是宋代頒行的18種曆法中最精確的,領先西方《格里曆》383年。1202年,宋寧宗令大臣謝深甫等人編纂的《慶元條法事類》成書,並於次年(1203年)正式下詔頒行。

北伐诏书曰:“天道好还,盖中国有必伸之理,人心助顺,虽匹夫无不报之仇。朕丕承万世之基,追述三朝之志。蠢兹逆虏,犹托要盟,朘生灵之资,奉溪壑之欲,此非出于得已,彼乃谓之当然。衣冠遗黎,虐视均于草芥;骨肉同姓,吞噬剧于豺狼。兼别境之侵陵,重连年之水旱,流移罔恤,盗贼恣行。边陲第谨于周防,文牒屡形于恐胁。自处大国,如临小邦,迹其不恭,如务容忍。曾故态之弗改,谓皇朝之可欺,军入塞而公肆创残,使来庭而敢为桀鹜。洎行李之继迁,复慢词之见加,含垢纳污,在人情而已极。声罪致讨,属故运之将倾。兵出有名,师直为壮,况志士仁人挺身而竟节,而谋臣猛将投袂以立功。西北二百州之豪杰,怀旧而愿归;东南七十载之遗黎,久郁而思奋。闻鼓旗之电举,想怒气之飚驰。噫!齐君复仇,上通九世,唐宗刷耻,卒报百王。矧乎家国之仇,接乎月日之近,夙宵是悼,涕泗无从。将勉辑于大勋,必允资于众力。言乎远,言乎迩,孰无中义之心?为人子,为人臣,当念愤。益砺执干之勇,式对在天之灵,庶几中黎旧业之再光,庸示永世宏纲之犹在。布告中外,明体至怀。”

宋史并没有记载宁宗去世是身患何病,而野史《东南纪闻》则记载宁宗病重,史弥远急于拥立理宗即位,于是奉上金丹百粒,宁宗服用后不久就去世。

宋史记载宁宗在即位前曾力请护送高宗的灵柩去会稽下葬,路上见到百姓在田间艰难劳作的场景,感慨地对身边的人说:“平时居住在宫中,哪里知道劳动的辛苦!”此外在个人生活上,宁宗也力行节俭,穿戴也较为朴素,使用的酒器都是以锡代银。有一年元宵夜,他独自对着蜡烛清坐,一个宦官见此劝他设宴过节,宁宗以外间百姓无饭可吃而拒绝。宁宗还曾游幸聚景园,晚上回宫的时候,临安的百姓争相观看,都想一睹天子之容。不幸的是有人被践踏踩死,宁宗得知此事后十分后悔,从此再也不出宫游玩了。

宁宗常让两个小太监背着两扇屏风作为他的前导,走到哪里都要跟随。屏风用白纸作底,边上糊着青纸,上写着“少饮酒,怕吐;少食生冷,怕痛。”当大臣让他喝酒或吃生冷食物时,他就指指屏风加以拒绝。就算饮酒也不超过三杯。

嘉泰年间,宁宗有意前往西湖泛舟游赏。有个叫张巨济的大臣上书劝谏宁宗道:“慈懿皇后的陵寝近在湖滨,陛下出游,难免要鼓乐一番,岂不是要惊动先人的在天之灵吗?”宁宗认为他说的很有道理,不仅升了他的俸禄,还把游船都沉到湖底,以表示自己不再游湖的决心。

\subsection{庆元}


\begin{longtable}{|>{\centering\scriptsize}m{2em}|>{\centering\scriptsize}m{1.3em}|>{\centering}m{8.8em}|}
  % \caption{秦王政}\
  \toprule
  \SimHei \normalsize 年数 & \SimHei \scriptsize 公元 & \SimHei 大事件 \tabularnewline
  % \midrule
  \endfirsthead
  \toprule
  \SimHei \normalsize 年数 & \SimHei \scriptsize 公元 & \SimHei 大事件 \tabularnewline
  \midrule
  \endhead
  \midrule
  元年 & 1195 & \tabularnewline\hline
  二年 & 1196 & \tabularnewline\hline
  三年 & 1197 & \tabularnewline\hline
  四年 & 1198 & \tabularnewline\hline
  五年 & 1199 & \tabularnewline\hline
  六年 & 1200 & \tabularnewline
  \bottomrule
\end{longtable}

\subsection{嘉泰}

\begin{longtable}{|>{\centering\scriptsize}m{2em}|>{\centering\scriptsize}m{1.3em}|>{\centering}m{8.8em}|}
  % \caption{秦王政}\
  \toprule
  \SimHei \normalsize 年数 & \SimHei \scriptsize 公元 & \SimHei 大事件 \tabularnewline
  % \midrule
  \endfirsthead
  \toprule
  \SimHei \normalsize 年数 & \SimHei \scriptsize 公元 & \SimHei 大事件 \tabularnewline
  \midrule
  \endhead
  \midrule
  元年 & 1201 & \tabularnewline\hline
  二年 & 1202 & \tabularnewline\hline
  三年 & 1203 & \tabularnewline\hline
  四年 & 1204 & \tabularnewline
  \bottomrule
\end{longtable}

\subsection{开禧}

\begin{longtable}{|>{\centering\scriptsize}m{2em}|>{\centering\scriptsize}m{1.3em}|>{\centering}m{8.8em}|}
  % \caption{秦王政}\
  \toprule
  \SimHei \normalsize 年数 & \SimHei \scriptsize 公元 & \SimHei 大事件 \tabularnewline
  % \midrule
  \endfirsthead
  \toprule
  \SimHei \normalsize 年数 & \SimHei \scriptsize 公元 & \SimHei 大事件 \tabularnewline
  \midrule
  \endhead
  \midrule
  元年 & 1205 & \tabularnewline\hline
  二年 & 1206 & \tabularnewline\hline
  三年 & 1207 & \tabularnewline
  \bottomrule
\end{longtable}

\subsection{嘉定}

\begin{longtable}{|>{\centering\scriptsize}m{2em}|>{\centering\scriptsize}m{1.3em}|>{\centering}m{8.8em}|}
  % \caption{秦王政}\
  \toprule
  \SimHei \normalsize 年数 & \SimHei \scriptsize 公元 & \SimHei 大事件 \tabularnewline
  % \midrule
  \endfirsthead
  \toprule
  \SimHei \normalsize 年数 & \SimHei \scriptsize 公元 & \SimHei 大事件 \tabularnewline
  \midrule
  \endhead
  \midrule
  元年 & 1208 & \tabularnewline\hline
  二年 & 1209 & \tabularnewline\hline
  三年 & 1210 & \tabularnewline\hline
  四年 & 1211 & \tabularnewline\hline
  五年 & 1212 & \tabularnewline\hline
  六年 & 1213 & \tabularnewline\hline
  七年 & 1214 & \tabularnewline\hline
  八年 & 1215 & \tabularnewline\hline
  九年 & 1216 & \tabularnewline\hline
  十年 & 1217 & \tabularnewline\hline
  十一年 & 1218 & \tabularnewline\hline
  十二年 & 1219 & \tabularnewline\hline
  十三年 & 1220 & \tabularnewline\hline
  十四年 & 1221 & \tabularnewline\hline
  十五年 & 1222 & \tabularnewline\hline
  十六年 & 1223 & \tabularnewline\hline
  十七年 & 1224 & \tabularnewline
  \bottomrule
\end{longtable}



%%% Local Variables:
%%% mode: latex
%%% TeX-engine: xetex
%%% TeX-master: "../Main"
%%% End:

%% -*- coding: utf-8 -*-
%% Time-stamp: <Chen Wang: 2019-12-26 10:48:39>

\section{理宗\tiny(1224-1264)}

\subsection{生平}

宋理宗趙昀(1205年1月26日-1264年11月16日),原名赵与莒,後賜名赵贵诚,宋太祖次子燕懿王趙德昭九世孫,宋宁宗太子趙竑與宰相史彌遠不睦,1224年寧宗駕崩後,彌遠矯詔立貴誠為帝,是為宋理宗,改名赵昀,是南宋的第五位皇帝(1224年9月17日—1264年11月16日在位),在位40年,享年59岁。

宋理宗本名趙與莒,本不是皇子,而只是宋宁宗的远房堂侄。他是宋太祖赵匡胤的十世孫,是趙匡胤次子燕懿王趙德昭的後人,但由於宋帝位一向並非由趙德昭這一脈後人繼承,至趙與莒父親趙希瓐這一代已與皇室血緣十分疏遠,而趙希瓐在生時並沒有任何封爵,僅官至山陰尉,生活與平民無異,趙與莒也因此在平民家庭出生及成長。趙與莒七歲時,父趙希瓐逝世,生母全氏帶他及弟趙與芮返娘家,三母子在全氏在紹興當保長的兄長家寄居,一直到趙與莒十六歲。

宋寧宗因八名親生子皆幼年夭折,故立趙德昭後裔趙詢為太子,趙詢於廿八歲時英年早逝,寧宗改立沂王赵竑為太子,沂王王位於是懸空,寧宗命宰相史彌遠找尋品行端正的宗室過繼給沂王,繼承王位,而史彌遠將此任務交了其幕僚余天錫。余天錫回鄉應考科舉,途經紹興遇著大雨,在全保長家中避雨,於是認識了趙與莒兄弟。余天錫知他們為趙氏宗族,也覺得兄弟二人行為得體,認為是合適人選繼承沂王,故向史彌遠推薦。史彌遠接兩兄弟往臨安親自考量,也認為兄長趙與莒為繼承沂王的合適人選,故於嘉定十四年(1221年)將趙與莒賜名贵诚,繼承沂王王位。

太子赵竑一向不滿史彌遠專權,聲言繼位後立即要貶史彌遠到海南島去,史彌遠決心另立新君。嘉定十七年(1224年),宋宁宗駕崩,史弥远聯同楊皇后假傳寧宗遺詔,太子赵竑廢為濟王,藩封霅川。以沂王趙貴誠為養子,賜名趙昀,是為宋理宗。

由於宋理宗是史彌遠一手擁立,登基后史弥远繼續專權,早已成年的理宗对政务完全不能过问,一直到1233年史彌遠死后,理宗才开始正式亲政。理宗一親政就任用洪咨夔等人做監察御史,彈劾了史彌遠一黨的「三凶」梁成大、李知孝、莫澤。而被史彌遠排斥的真德秀、魏了翁則被召入朝。當時紙幣的發行量超過三億貫,通貨膨脹,物價飛漲。宋廷停止發行新幣,回收部分舊幣,並動用庫存黃金十萬兩、白銀數百萬兩平抑物價。1234年南宋联蒙灭金,但不久蒙古入侵,南宋为防止蒙军南下而軍費陡升,宋廷不得不大量發行貨幣以緩解財政壓力。最終經濟整頓破產。

宋理宗一直希望使理學成為正統官學,早在寶慶三年(1227年)就追封朱熹為信國公。端平更化後,朱熹和理學大師周敦頤、程顥、程頤、張載都先後被入祀孔廟。淳祐元年(1241年)理宗又分別加封周敦頤為汝南伯、程顥為河南伯、程頤為伊陽伯、張載為噤伯。景定二年(1261年)理宗排定的入祀孔廟的名單包括:司馬光、周敦頤、程顥、程頤、張載、朱熹、邵雍、張栻、呂祖謙。其中除司馬光外,剩下的都是理學代表人物。

理宗的政治改革也要解決大量冗官,這是通過控制考中進士的人數和嚴格升遷制度辦到的。從端平元年開始,平均每次科考的中進士人數為四百五十人,而不是以前的平均每次六百人。理宗又規定無論在首都的朝官還是在外地的地方官都不得私薦官員,沒有擔任過州縣地方官员的人不能進入朝廷做郎官,已經當上郎官的必須外放,補上州縣地方官這一任。但這些措施流於表面,未能根本解決問題。

晚年,宋理宗对政治不感兴趣,将国家大事交给他的丞相处理,先后有吴潜、丁大全、范光瑞等,其中任用賈似道。

宋理宗晚年盡情女色,三宫六院已满足不了他的私欲。善于奉迎的内侍董宋臣看到了,便在一次元宵佳节,董宋臣为宋理宗找来临安名妓唐安安入宫淫乐。唐安安姿色豔美,能歌善舞。宋理宗一見非常喜愛,便把她留在宫里,日夜宠幸。唐安安仗着宋理宗的宠爱,过起了豪奢的生活,家中的用具上到妆盒酒具,下到水盆火箱,都是用金银制成的;帐幔茵褥,也都是绫罗锦绣;珍奇宝玩,更是不计其数。除了唐安安之外,宋理宗还经常召一些歌妓舞女进宫,起居郎牟子才上书劝诫宋理宗:“此举坏了陛下三十年自修之操!”宋理宗却让人转告牟子才、不得告知他人,以免有损皇帝的形象。姚勉以唐玄宗、杨贵妃、高力士为例劝告宋理宗,宋理宗竟然恬不知耻地回答:“朕虽不德,未如明皇之甚也(自嘲不如唐玄宗厲害)。”

阎贵妃是宋理宗晚年最宠爱的妃子,姿色妖媚,以美色受寵愛,初封婉容。淳祐九年(1240)九月,宋理宗封阎氏为贵妃,當時贾贵妃去世,留下一個六歲的女兒瑞國公主,而阎贵妃沒有懷孕,生下子女,宋理宗便將瑞國公主交由阎贵妃撫養。宋理宗為了表示對阎贵妃的寵愛,对她赏赐无數,阎贵妃想修建一座功德寺,宋理宗不惜动用国库,耗費巨資,破天荒地派遣吏卒到各州县搜集木材,为其修功德寺,闹得老百姓不得安宁,为了求得合适的的梁柱,竟想砍去灵隐寺前的晋代古松。幸好灵隐寺住持僧元肇,写了一首诗:“不为栽松种茯苓,只缘山色四时青。老僧不许移松去,留与西湖作画屏。”這才保住古松,这座功德寺前后花了三年才建成,耗费极大,修得比自家祖宗的功德寺还要富丽堂皇,當時人称為“赛灵隐寺”。

後來阎贵妃在理宗的宠爱下,權勢大增,不可一世,骄横放肆,恃宠弄权,一些投机钻营的小人,走她的门路。其中,周汉国公主下嫁,马天骥绞尽脑汁送了一份别出心裁的大礼,得到宋理宗的欢心,与丁大全同时被任命为执政,所以阎贵妃又與马天骥、丁大全、人稱“董閻羅”的董宋臣等奸臣内外勾結,狼狽為奸,沆瀣一氣,阎贵妃、马天骥、丁大全、董閻羅,史稱“阎马丁董”,恃寵亂政,结党营私,排除异己,陷害忠良,迫逐宰相董槐,引起很多忠臣不满,當時,有人在朝門上題八個大字:“閻馬丁當,國勢將亡。”這對男女又與贾似道明争暗斗,打擊迫害,紊亂朝政,民怨沸騰,“阎马丁董”等四人又強奪民田,招權納賄,作惡多端,無所不爲。

1259年,忽必烈攻鄂州,右丞相贾似道向忽必烈称臣并许诺将长江以北的土地完全割让给以蒙元,后因蒙哥在釣魚城战死,为争夺帝位,忽必烈不得不退回北方,南宋才化险为夷。贾似道因此谎报军情邀功,博得宋理宗信任,而对曾经卖国的许诺只字不提。

宋理宗在位四十年后,1264年逝世於臨安。由於宋理宗无子,所以立他的侄子赵禥为太子,是为宋度宗。

宋亡後,元朝西藏藏傳佛教僧人楊璉真珈盜掘南宋六陵,見宋理宗屍身保存完好,將屍體倒掛在樹上三天,結果流出水銀,又以理宗頭蓋骨奉給帝師八思巴為飲器,是為骷髏碗。躯干则火化。明初,明太祖得知此事,“嘆息久之”,怒斥道:“南宋的皇帝们没有很大的道德操守问题,跟蒙元也没有世仇。蒙元已然是趁人之危攻取南宋的,为什么还要如此残酷?”,派人找回理宗的頭顱,洪武二年(1369年)以帝王禮葬於應天府(江蘇南京),次年又將理宗的頭骨歸葬到紹興永穆陵舊址。

元朝官修正史《宋史》脱脱等的評價是:“理宗享国久长,与仁宗同。然仁宗之世,贤相相继。理宗四十年之间,若李宗勉、崔与之、吴潜之贤,皆弗究于用;而史弥远、丁大全、贾似道窃弄威福,与相始终。治效之不及庆历、嘉祐,宜也。蔡州之役,幸依大朝以定夹攻之策,及函守绪遗骨,俘宰臣天纲,归献庙社,亦可以刷会稽之耻,复齐襄之仇矣。顾乃贪地弃盟,入洛之师,事衅随起,兵连祸结,境土日蹙。郝经来使,似道讳言其纳币请和,蒙蔽抑塞,拘留不报,自速灭亡。吁,可惜哉!由其中年嗜欲既多,怠于政事,权移奸臣,经筵性命之讲,徒资虚谈,固无益也。虽然,宋嘉定以来,正邪贸乱,国是靡定,自帝继统,首黜王安石孔庙从祀,升濂、洛九儒,表章朱熹《四书》,丕变士习,视前朝奸党之碑、伪学之禁,岂不大有径庭也哉!身当季运,弗获大效,后世有以理学复古帝王之治者,考论匡直辅翼之功,实自帝始焉。庙号曰"理",其殆庶乎!”

明末清初思想家王夫之的评价是:“理宗无君人之才,而犹有君人之度。”

李贽:“理宗是个得失相半之主。”


\subsection{宝庆}


\begin{longtable}{|>{\centering\scriptsize}m{2em}|>{\centering\scriptsize}m{1.3em}|>{\centering}m{8.8em}|}
  % \caption{秦王政}\
  \toprule
  \SimHei \normalsize 年数 & \SimHei \scriptsize 公元 & \SimHei 大事件 \tabularnewline
  % \midrule
  \endfirsthead
  \toprule
  \SimHei \normalsize 年数 & \SimHei \scriptsize 公元 & \SimHei 大事件 \tabularnewline
  \midrule
  \endhead
  \midrule
  元年 & 1225 & \tabularnewline\hline
  二年 & 1226 & \tabularnewline\hline
  三年 & 1227 & \tabularnewline
  \bottomrule
\end{longtable}

\subsection{绍定}

\begin{longtable}{|>{\centering\scriptsize}m{2em}|>{\centering\scriptsize}m{1.3em}|>{\centering}m{8.8em}|}
  % \caption{秦王政}\
  \toprule
  \SimHei \normalsize 年数 & \SimHei \scriptsize 公元 & \SimHei 大事件 \tabularnewline
  % \midrule
  \endfirsthead
  \toprule
  \SimHei \normalsize 年数 & \SimHei \scriptsize 公元 & \SimHei 大事件 \tabularnewline
  \midrule
  \endhead
  \midrule
  元年 & 1228 & \tabularnewline\hline
  二年 & 1229 & \tabularnewline\hline
  三年 & 1230 & \tabularnewline\hline
  四年 & 1231 & \tabularnewline\hline
  五年 & 1232 & \tabularnewline\hline
  六年 & 1233 & \tabularnewline
  \bottomrule
\end{longtable}

\subsection{端平}

\begin{longtable}{|>{\centering\scriptsize}m{2em}|>{\centering\scriptsize}m{1.3em}|>{\centering}m{8.8em}|}
  % \caption{秦王政}\
  \toprule
  \SimHei \normalsize 年数 & \SimHei \scriptsize 公元 & \SimHei 大事件 \tabularnewline
  % \midrule
  \endfirsthead
  \toprule
  \SimHei \normalsize 年数 & \SimHei \scriptsize 公元 & \SimHei 大事件 \tabularnewline
  \midrule
  \endhead
  \midrule
  元年 & 1234 & \tabularnewline\hline
  二年 & 1235 & \tabularnewline\hline
  三年 & 1236 & \tabularnewline
  \bottomrule
\end{longtable}

\subsection{嘉熙}

\begin{longtable}{|>{\centering\scriptsize}m{2em}|>{\centering\scriptsize}m{1.3em}|>{\centering}m{8.8em}|}
  % \caption{秦王政}\
  \toprule
  \SimHei \normalsize 年数 & \SimHei \scriptsize 公元 & \SimHei 大事件 \tabularnewline
  % \midrule
  \endfirsthead
  \toprule
  \SimHei \normalsize 年数 & \SimHei \scriptsize 公元 & \SimHei 大事件 \tabularnewline
  \midrule
  \endhead
  \midrule
  元年 & 1237 & \tabularnewline\hline
  二年 & 1238 & \tabularnewline\hline
  三年 & 1239 & \tabularnewline\hline
  四年 & 1240 & \tabularnewline
  \bottomrule
\end{longtable}

\subsection{淳祐}

\begin{longtable}{|>{\centering\scriptsize}m{2em}|>{\centering\scriptsize}m{1.3em}|>{\centering}m{8.8em}|}
  % \caption{秦王政}\
  \toprule
  \SimHei \normalsize 年数 & \SimHei \scriptsize 公元 & \SimHei 大事件 \tabularnewline
  % \midrule
  \endfirsthead
  \toprule
  \SimHei \normalsize 年数 & \SimHei \scriptsize 公元 & \SimHei 大事件 \tabularnewline
  \midrule
  \endhead
  \midrule
  元年 & 241 & \tabularnewline\hline
  二年 & 242 & \tabularnewline\hline
  三年 & 243 & \tabularnewline\hline
  四年 & 244 & \tabularnewline\hline
  五年 & 245 & \tabularnewline\hline
  六年 & 246 & \tabularnewline\hline
  七年 & 247 & \tabularnewline\hline
  八年 & 248 & \tabularnewline\hline
  九年 & 249 & \tabularnewline\hline
  十年 & 250 & \tabularnewline\hline
  十一年 & 251 & \tabularnewline\hline
  十二年 & 252 & \tabularnewline
  \bottomrule
\end{longtable}

\subsection{宝祐}

\begin{longtable}{|>{\centering\scriptsize}m{2em}|>{\centering\scriptsize}m{1.3em}|>{\centering}m{8.8em}|}
  % \caption{秦王政}\
  \toprule
  \SimHei \normalsize 年数 & \SimHei \scriptsize 公元 & \SimHei 大事件 \tabularnewline
  % \midrule
  \endfirsthead
  \toprule
  \SimHei \normalsize 年数 & \SimHei \scriptsize 公元 & \SimHei 大事件 \tabularnewline
  \midrule
  \endhead
  \midrule
  元年 & 1253 & \tabularnewline\hline
  二年 & 1254 & \tabularnewline\hline
  三年 & 1255 & \tabularnewline\hline
  四年 & 1256 & \tabularnewline\hline
  五年 & 1257 & \tabularnewline\hline
  六年 & 1258 & \tabularnewline
  \bottomrule
\end{longtable}

\subsection{开庆}

\begin{longtable}{|>{\centering\scriptsize}m{2em}|>{\centering\scriptsize}m{1.3em}|>{\centering}m{8.8em}|}
  % \caption{秦王政}\
  \toprule
  \SimHei \normalsize 年数 & \SimHei \scriptsize 公元 & \SimHei 大事件 \tabularnewline
  % \midrule
  \endfirsthead
  \toprule
  \SimHei \normalsize 年数 & \SimHei \scriptsize 公元 & \SimHei 大事件 \tabularnewline
  \midrule
  \endhead
  \midrule
  元年 & 1259 & \tabularnewline
  \bottomrule
\end{longtable}

\subsection{景定}

\begin{longtable}{|>{\centering\scriptsize}m{2em}|>{\centering\scriptsize}m{1.3em}|>{\centering}m{8.8em}|}
  % \caption{秦王政}\
  \toprule
  \SimHei \normalsize 年数 & \SimHei \scriptsize 公元 & \SimHei 大事件 \tabularnewline
  % \midrule
  \endfirsthead
  \toprule
  \SimHei \normalsize 年数 & \SimHei \scriptsize 公元 & \SimHei 大事件 \tabularnewline
  \midrule
  \endhead
  \midrule
  元年 & 1260 & \tabularnewline\hline
  二年 & 1261 & \tabularnewline\hline
  三年 & 1262 & \tabularnewline\hline
  四年 & 1263 & \tabularnewline\hline
  五年 & 1264 & \tabularnewline
  \bottomrule
\end{longtable}



%%% Local Variables:
%%% mode: latex
%%% TeX-engine: xetex
%%% TeX-master: "../Main"
%%% End:

%% -*- coding: utf-8 -*-
%% Time-stamp: <Chen Wang: 2019-12-26 10:49:24>

\section{度宗\tiny(1264-1274)}

\subsection{生平}

宋度宗赵禥(1240年5月2日-1274年8月12日),曾赐名趙孟启、赵孜,1253年立為皇子,賜名禥,是南宋第六位皇帝(1264年11月16日-1274年8月12日在位),宋太祖的第十一世孫,宋理宗養子。生父為榮王趙與芮,生母隆国夫人黄定喜。在位10年,享年34歲,死后葬于永绍陵,谥号为「端文明武景孝皇帝」。

嘉熙四年(1240年)四月九日,赵禥生于绍兴府的荣王府邸,出生时的名字未被记录下来。生母黄定黄是父亲赵与芮亡妻李氏的陪嫁女,因身份卑微,她在怀孕时曾服堕胎药,以致赵禥出生后,手脚皆软,七岁才能说话。《宋史》对此事有所回避,仅称“资识内慧,七岁始言,言必合度,理宗奇之。”

伯父宋理宗子嗣不多,兩名兒子又在幼年夭折,故須在宗室另尋繼任人。宗室中,趙禥是与宋理宗血缘关系最近的人。宋理宗有意以趙禥为继承人。淳祐六年(1246年)十月己丑,伯父赐名孟启,以皇侄授贵州刺史,入内小学。七年正月乙卯,授宜州观察使,就王邸训习。九年正月乙巳,授庆远军节度使,封益国公。十一年正月壬戌,改赐名孜,进封建安郡王。宝祐元年(1253年)正月庚辰,下诏立为皇子,改赐名禥。癸未,授崇庆军节度使、开府仪同三司,进封永嘉郡王。二年十月癸酉,进封忠王。十一月壬寅,加元服,赐字邦寿。五年十月庚子,授镇南、遂安军节度使。景定元年(1260年)六月壬寅,立赵禥为皇太子,赐字长源。次年,其妻永嘉郡夫人全氏被立为皇太子妃。

最初,不少大臣反對立他為太子,立太子之前,大臣吴潜曾密奏,称“臣无弥远之材,忠王无陛下之福”此语激怒宋理宗。宋理宗在立趙禥為太子后,對他刻意栽培,管教甚严。

景定五年冬十月,在位四十年的理宗駕崩,趙禥即位,是為度宗,参考先代咸平 (宋真宗年號) 、淳熙 (宋孝宗年號) 年間国家安定盛世的歷史,改年号为咸淳。

度宗即位时,金朝已经灭亡三十年,北方元朝军队大举南下,国难当头,他却将军国大权交给奸臣贾似道,南宋政治十分腐败黑暗,人民生活十分困苦。度宗甫即位,丞相贾似道便私自往绍兴养老,度宗亲自手书劝他归朝多次,贾似道方才答应。即位之初,度宗便行幸太学,鼓励学问。咸淳二年,民间叶李、萧至二人指责贾似道专权乱政,但度宗反而更加信任贾似道。咸淳三年正月,度宗行幸辟雍同时授予贾似道太师,叶梦鼎右丞相,留梦炎枢密。八月,他写信鼓励边防士卒,其中提到自己经常感染风寒并且从没停止过。十一月,因为极度担心蒙古的侵略,召见武官朱禩并询问边防情况。咸淳五年正月,任命江万里、马廷鸾为左右丞相。咸淳六年三月,度宗下诏禁止奢侈,崇尚节俭。

北方元军多次出兵进攻南宋,南宋朝廷虽腐朽,但是广大军民的抵抗,使得元军不得不撤回,無法攻破长江防缘。度宗即位后,元军猛攻襄陽。然而贾似道密而不报,还说已经取胜(实际上一直僵持),度宗在完全不予以查問的情況下竟对此言深信不疑。最后,元军于咸淳九年(1273年)初攻破围攻5年的襄陽。宋度宗闻知顿时昏倒,之後不思振作,終日借酒浇愁。咸淳十年 (1274年) 七月,度宗因酒色过度而死。死后兩年,元军攻破臨安。

元朝官修正史《宋史》脱脱等的評價是:“宋至理宗,疆宇日蹙,贾似道执国命。度宗继统,虽无大失德,而拱手权奸,衰敝寝甚。考其当时事势,非有雄才睿略之主,岂能振起其坠绪哉!历数有归,宋祚寻讫,亡国不于其身,幸矣。”


\subsection{咸淳}


\begin{longtable}{|>{\centering\scriptsize}m{2em}|>{\centering\scriptsize}m{1.3em}|>{\centering}m{8.8em}|}
  % \caption{秦王政}\
  \toprule
  \SimHei \normalsize 年数 & \SimHei \scriptsize 公元 & \SimHei 大事件 \tabularnewline
  % \midrule
  \endfirsthead
  \toprule
  \SimHei \normalsize 年数 & \SimHei \scriptsize 公元 & \SimHei 大事件 \tabularnewline
  \midrule
  \endhead
  \midrule
  元年 & 1265 & \tabularnewline\hline
  二年 & 1266 & \tabularnewline\hline
  三年 & 1267 & \tabularnewline\hline
  四年 & 1268 & \tabularnewline\hline
  五年 & 1269 & \tabularnewline\hline
  六年 & 1270 & \tabularnewline\hline
  七年 & 1271 & \tabularnewline\hline
  八年 & 1272 & \tabularnewline\hline
  九年 & 1273 & \tabularnewline\hline
  十年 & 1274 & \tabularnewline
  \bottomrule
\end{longtable}



%%% Local Variables:
%%% mode: latex
%%% TeX-engine: xetex
%%% TeX-master: "../Main"
%%% End:

%% -*- coding: utf-8 -*-
%% Time-stamp: <Chen Wang: 2021-11-01 15:56:52>

\section{恭帝趙㬎\tiny(1274-1276)}

\subsection{生平}

宋恭帝趙㬎(1271年11月2日-1323年5月31日),南宋第七代皇帝(1274年8月12日—1276年2月4日在位),在位2年,得年53岁,宋度宗六子。他是全皇后所生,是宋端宗赵昰之弟,宋末帝赵昺兄,即位前封嘉国公、左卫上将军等,宋端宗为兄弟上尊号孝恭懿圣皇帝,无庙号。1276年2月領宋室投降元朝后,封为「瀛國公」,之後又被迫剃髮出家,最後被賜死。

《宋史本纪第四十七》記載:“瀛国公名㬎,度宗皇帝子也,母曰全皇后,咸淳七年(1271年)九月己丑,生于临安府之大内。”宋咸淳十年(1274年)七月,宋度宗驾崩,年三岁的趙㬎在丞相贾似道的扶持下登基做皇帝,是为宋恭帝,改明年為德祐元年。祖母謝太皇太后、母亲全太后垂帘听政。但军国大权依然在贾似道之手。

當時元军已控制中國北方和西南,在取得南下最重要通道襄陽城的控制權之後,渡过长江向南宋都城临安(今杭州)进发。謝太皇太后一面在全国通令“勤王”,一面向元軍乞和。勢如破竹的元軍在擊破各地的防線,相繼降服了長江中游諸州。1275年,賈似道率領的13萬大軍在蕪湖與元軍對戰大敗。不久,謝太皇太后和宋恭帝在輿論壓力下貶贾似道,不过为时已晚,宋朝亡國在即。同年年中,元軍已經佔領了江東(今日的江蘇省南部)大半的領土。

1276年1月18日伯顏率領的元军兵临临安。南宋朝廷遣陸秀夫求和稱侄不成,只好向元军投降。正月十八日(1276年2月4日),謝太皇太后抱着五岁的小皇帝宋恭帝㬎出趙城,派遣监察御史杨应奎向元軍献上传国玺投降,南宋滅亡。南宋残余势力在福建、广东抗元。

南宋滅亡後,宋恭帝曾徙居元大都、上都、烏斯藏、甘州(一說還有謙州,今俄罗斯图瓦共和国境内)等地,是中國歷史上遊歷最遠的一位漢人皇帝。

恭帝降元後,元將巴延促其北上入覲。帝於至元十三年(宋德祐二年,1276年)丙子三月從臨安啓程,前往上都。太皇太后謝氏以疾留内。太后全氏、隆國夫人黃氏(度宗母、恭帝祖母)、榮王赵与芮(理宗弟、度宗父、恭帝祖父)、沂王趙乃猷、樞密院參知政事高應松、謝堂、知臨安府翁仲德及汪元量等朝臣、宮人隨同北上(見劉一清錢塘遺事)。渡江後,宋將李庭芝、苗再成等謀奪駕,不克。五月,過大都,赴上都。丙申,見元世祖忽必烈於上都大安殿。忽必烈封恭帝为瀛國公,妻以公主,詔優待之,使居大都;福王趙與芮受封平原郡公(汪元量《水雲集》湖州歌八十一:“福王又拜平原郡,幼主新封瀛國公”)。

1279年3月19日,陸秀夫擕年仅八岁的小皇帝赵昺在崖山蹈海自盡,南宋最終全面灭亡。

至元十九年(1282年),中書省奏請徙瀛國公居上都,詔許之。後元仁宗延祐中,隨高麗國王王璋入朝的高麗人權漢功,見瀛國公故宅尚存,作《瀛國公第盆梅》詠之。

忽必烈欲保全亡宋宗室。至元二十五年(1288年)十月詔遣瀛國公趙㬎入吐蕃習梵書、西蕃字經(一說瀛國公自求入吐蕃學佛法)。十二月啓程,由脫思麻(今青海省海南藏族自治州一帶)入烏思藏,駐錫薩斯迦大寺,號木波講師。[來源請求]他在萨迦寺出家,取藏文法名“却季仁钦”(ཆོས་ཀྱི་རིན་ཆེན་)。藏族人尊称他为“蛮子拉尊”(སྨན་རྩེ་ལྷ་བཙུན་);“蛮子”是蒙古人对宋人的称谓,“拉尊”是藏语对出家王族的尊称,汉译合尊。後為薩斯迦大寺住持。嘗取漢藏佛經互譯比勘,校訂異文。

元英宗至治三年(1323年)四月(據釋念常《佛祖歷代通載》),賜瀛國公趙㬎死於河西(今甘肃河西走廊张掖)。明初僧人釋無慍《山庵雜錄》云:“瀛國公為僧後,至英宗朝,適興吟詩云:「寄語林和靖,梅開几度花。黄金臺上客,無復得還家。」諜者以其意在諷動江南人心,聞之於上,收斬之。既而上悔,出內帑黃金,詔江南善書僧儒集燕京,書大藏经云。”案:陶宗儀《輟耕錄》引此詩,作:“寄語林和靖。梅花幾度開。黃金臺下客,應是不歸來。”並云“此宋幼主在京師所作也”。明人瞿佑《歸田詩話》引作:“黃金臺上客,底事又思家。歸問林和靖,寒梅幾度花。”謂瀛國公以此詩贈汪元量。藏文史書則謂其罪名是以讲经为名,聚衆謀反。元史英宗紀可旁證瀛國公卒年:“至治三年四月壬戌朔,敕天下諸司命僧誦經十萬部。……敕京師萬安、慶壽、聖安、普慶四寺,揚子江金山寺、五台山萬聖祐國寺,做水陸佛事七晝夜。”按《元史》卷四十二顺帝纪五,至正十二年五月,“河南诸处群盗,辄引亡宋故号以为口实,宜以瀛国公子和尚赵完普及亲属徙沙州安置,禁勿与人交通。”可见,赵完普并未被赐死,且有一定数量的家眷。

後世盛傳宋恭帝為元惠宗妥懽帖睦爾之生父。元文宗曾佈告中外,引元惠宗乳母夫之言,稱元明宗在漠北時,素謂太子(妥懽帖睦爾)非己子,遂徙於高麗,後遷靜江。元末明初人權衡撰《庚申外史》,謂瀛國公駐錫甘州山寺(元時稱十字寺,即張掖大佛寺)時,封地位於汪古部舊地及居延一帶的趙王曾以一回回女子與之(即順帝生母邁來迪)。延祐七年四月,回回女生一男子。時值元武宗長子周王和世琜(即位後為元明宗)流亡西北,過甘州山寺,見瀛國公幼子,“大喜,因求為子,並其母載以歸”。明代以後,此說遂成確論。至清代,《欽定四庫全書總目提要》認為此說乃宋遺民僞造,明人“附會而盛傳之”,“覈以事實,渺無可據,實為荒誕之尤,非信史也。”近時學者有謂瀛國公在移駐甘州之前,可能居於謙州吉利吉思地界(今葉尼塞河上游)。當時周王和世琜自陝西至嶺北過金山(阿尔泰山),流亡於察合臺後王封地,地理上與謙州接近。

“司马迁论秦、赵世系同出伯益。夫稷、契、伯益其子孙皆有天下,至于运祚短长,亦系其功德之厚薄焉。赵宋虽起于用武,功成治定之后,以仁传家,视秦宜有间矣。然仁之敝失于弱,即文之敝失于僿也。中世有欲自强,以革其敝,用乖其方,驯致棼扰。建炎而后,土宇分裂,犹能六主百五十年而后亡,岂非礼义足以维持君子之志,恩惠足以固结黎庶之心欤?瀛国四岁即位,而天兵渡江,六岁而群臣奉之入朝。汉刘向言:“孔子论《诗》至‘殷士肤敏,裸将于京。’喟然叹曰:大哉天命,善不可不传于后嗣,是以富贵无常。”至哉言乎!我皇元之平宋也,吴越之民,市不易肆。世祖皇帝命征南之帅,辄以宋祖戒曹彬勿杀之言训之。《书》曰:“大哉王言,一哉王心。”我元一天下之本,其在于兹。”—— 元朝官修正史《宋史》脱脱等的評價


\subsection{德祐}


\begin{longtable}{|>{\centering\scriptsize}m{2em}|>{\centering\scriptsize}m{1.3em}|>{\centering}m{8.8em}|}
  % \caption{秦王政}\
  \toprule
  \SimHei \normalsize 年数 & \SimHei \scriptsize 公元 & \SimHei 大事件 \tabularnewline
  % \midrule
  \endfirsthead
  \toprule
  \SimHei \normalsize 年数 & \SimHei \scriptsize 公元 & \SimHei 大事件 \tabularnewline
  \midrule
  \endhead
  \midrule
  元年 & 1275 & \tabularnewline\hline
  二年 & 1276 & \tabularnewline
  \bottomrule
\end{longtable}



%%% Local Variables:
%%% mode: latex
%%% TeX-engine: xetex
%%% TeX-master: "../Main"
%%% End:

%% -*- coding: utf-8 -*-
%% Time-stamp: <Chen Wang: 2021-11-01 15:57:14>

\section{端宗赵昰\tiny(1276-1278)}

\subsection{生平}

宋端宗赵\xpinyin*{昰}(1269年7月10日-1278年5月8日),南宋第八位皇帝(1276年6月14日—1278年5月8日在位),在位3年,得年10岁,廟號端宗,諡號「裕文昭武愍孝皇帝」或「孝恭仁裕懿聖濬文英武勤政皇帝」,又有史稱宋帝昰。他是宋度宗的五子,宋恭帝兄,曾被封为建国公、吉王、益王等。

宋恭帝德祐二年正月十八日(1276年2月4日)元军攻克临安时,宋恭帝和謝太皇太后相繼被俘。赵昰和母亲杨淑妃和弟弟赵昺由国舅杨亮节等护卫,出逃福建,定行都於福州濂浦平山福地,改年號景炎,行宮為平山閣(當時时值战乱,哀鸿遍野,宋军撤离此地时,曾开仓济民,当地人民甚感其恩,元军佔领福州时,当地人民遂将平山阁改名为泰山宫,祭祀南宋高宗赵构及入闽的益、广二王。左右列的是文臣武将:文天祥、陆秀夫、陈宜中、张世杰。当地泰山宫便塑这些神像,实是回避元代的查禁,以泰山宫作掩护,泰山宫現存完好)。

赵昰登基前被封为「天下兵马都元帅」。1276年6月14日即位,改元景炎,时年只有7岁。虽然朝臣陆秀夫等坚持抗元,力图恢复宋朝,但在元军的紧紧追击下,端宗只能由大将张世杰护卫登船入海,东逃西避,疲于奔命。他曾逃到南澳岛上,在岛上海滩上开挖的宋井至今仍存,之后又逃到香港的九龙城一帶,現存的宋王臺和侯王廟都是为纪念宋端宗而建。

景炎三年(1278年)3月,端宗在元将刘深追逐下避入广州對開海面,不料座船倾覆,端宗遇溺被左右救起,因此染病。因元军追兵逼近,又不得不浮海逃往碙洲(今香港大屿山)。不到10岁的小皇帝屡受颠簸,又惊病交加,于几个月后(1278年5月8日)在碙洲去世,葬于永福陵(今香港大嶼山東涌黃龍坑)。

根據宋王臺花園《九龙宋皇台遗址碑记》记载,昰、昺二帝南逃期間,「有金夫人墓,相传为杨太后女,晋国公主,先溺于水,至是铸金身以葬者」,葬于今九龙城区,人称金夫人墓,后来在该址兴建了圣三一堂,金夫人墓隨之湮沒。另一方面,碑記也記載端宗曾經於九龍城白鶴山行朝,及以一塊大石為御座,後人稱此為「交椅石」,惟該石現在已經不知所終。

元朝官修正史《宋史》脱脱等的評價是:“宋之亡征,已非一日。历数有归,真主御世,而宋之遗臣,区区奉二王为海上之谋,可谓不知天命也已。然人臣忠于所事而至于斯,其亦可悲也夫!”


\subsection{景炎}


\begin{longtable}{|>{\centering\scriptsize}m{2em}|>{\centering\scriptsize}m{1.3em}|>{\centering}m{8.8em}|}
  % \caption{秦王政}\
  \toprule
  \SimHei \normalsize 年数 & \SimHei \scriptsize 公元 & \SimHei 大事件 \tabularnewline
  % \midrule
  \endfirsthead
  \toprule
  \SimHei \normalsize 年数 & \SimHei \scriptsize 公元 & \SimHei 大事件 \tabularnewline
  \midrule
  \endhead
  \midrule
  元年 & 1276 & \tabularnewline\hline
  二年 & 1277 & \tabularnewline\hline
  三年 & 1278 & \tabularnewline
  \bottomrule
\end{longtable}



%%% Local Variables:
%%% mode: latex
%%% TeX-engine: xetex
%%% TeX-master: "../Main"
%%% End:

%% -*- coding: utf-8 -*-
%% Time-stamp: <Chen Wang: 2021-11-01 15:57:27>

\section{赵昺趙昺\tiny(1278-1279)}

\subsection{生平}

宋少帝趙昺(1272年2月12日-1279年3月19日),宋朝末代皇帝(南宋第九位,1278年5月10日—1279年3月19日在位)在位313天,得年7岁。宋朝第十五位皇帝宋度宗趙禥的七子,母俞修容。前任皇帝宋端宗趙昰的異母弟。先後封為永國公、信王、廣王等。

宋恭帝德祐二年正月十八日(1276年2月4日),南宋首都臨安被伯顏率領的元朝大軍佔領,5歲的小皇帝宋恭帝和謝太皇太后相繼被俘。宋恭帝的兩個異母兄弟益王趙昰和廣王趙昺,在國舅楊亮節、朝臣陸秀夫、張世傑、陳宜中和文天祥等人的護衛下南逃。在金華,趙昰被封為天下兵馬都元帥,趙昺為副元帥,晉為衛王。1276年6月14日,剛滿7歲的趙昰在福州即皇帝位,是為宋端宗,改元“景炎”,奉母楊淑妃為皇太后。

一心想對宋朝皇室斬草除根的元軍統帥伯顏,對宋端宗的南宋小朝廷窮追不捨。景炎三年(1278年),宋端宗崩,以至軍心渙散,無心戀戰。當時陸秀夫在碙州(即今日湛江市硇洲島),改碙州為祥龍縣,擁立趙昺為皇帝,改元“祥興”,仍奉端宗母楊淑妃為太后(赵昺母俞修容下落无考,亦未有被尊为太后之记载),並逃往新會崖山避難。

元朝命大將張弘範大舉進攻崖山的趙昺小朝廷。事實上,當時的宋軍還未到岸,一行人還在海上。宋軍水師在張世傑的指揮下進行頑抗,在崖門海域裡與元朝軍隊交戰,史稱崖山戰役,這場戰役關係到南宋小朝廷的存亡。結果,宋軍全軍覆沒。1279年3月19日,丞相陸秀夫見無法脫逃,便背著這位剛滿8歲的趙昺跳海殉國,楊太后亦投海殉國,宋朝正式宣告滅亡。

深圳赤灣(現屬南山區)有宋少帝陵,據說是少帝遺骸漂至赤灣附近,由僧人發現,從其身穿龍袍看出是宋少帝,於是把他葬於此。1984年,蛇口工業區和香港趙氏宗親會出資,修葺並擴建了宋少帝陵(或稱祥慶陵),現為深圳重點文物保護單位。

香港有紀念兩位宋末皇帝逃難的地方宋王臺公園,附近有「金夫人墓」,相傳為宋端宗生母楊太后之墓。由於該址後來興建聖三一堂,「金夫人墓」也隨之湮沒。

此外宋王臺公園附近以前曾建有以其命名的宋街、帝街、昺街。二次大戰期間,日軍在1942年3月擴建啟德機場,招募了數千名工人炸毀了古蹟宋王臺。戰後港英政府為重建機場,剷平了宋王臺餘下的「聖山」部份。

元朝官修正史《宋史》脱脱等的評價是:“宋之亡征,已非一日。歷数有归,真主御世,而宋之遗臣,区区奉二王为海上之谋,可谓不知天命也已。然人臣忠于所事而至于斯,其亦可悲也夫!”

\subsection{祥兴}


\begin{longtable}{|>{\centering\scriptsize}m{2em}|>{\centering\scriptsize}m{1.3em}|>{\centering}m{8.8em}|}
  % \caption{秦王政}\
  \toprule
  \SimHei \normalsize 年数 & \SimHei \scriptsize 公元 & \SimHei 大事件 \tabularnewline
  % \midrule
  \endfirsthead
  \toprule
  \SimHei \normalsize 年数 & \SimHei \scriptsize 公元 & \SimHei 大事件 \tabularnewline
  \midrule
  \endhead
  \midrule
  元年 & 1278 & \tabularnewline\hline
  二年 & 1279 & \tabularnewline
  \bottomrule
\end{longtable}



%%% Local Variables:
%%% mode: latex
%%% TeX-engine: xetex
%%% TeX-master: "../Main"
%%% End:


%%% Local Variables:
%%% mode: latex
%%% TeX-engine: xetex
%%% TeX-master: "../Main"
%%% End:
