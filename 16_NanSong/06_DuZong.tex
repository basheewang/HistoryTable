%% -*- coding: utf-8 -*-
%% Time-stamp: <Chen Wang: 2019-12-26 10:49:24>

\section{度宗\tiny(1264-1274)}

\subsection{生平}

宋度宗赵禥(1240年5月2日-1274年8月12日),曾赐名趙孟启、赵孜,1253年立為皇子,賜名禥,是南宋第六位皇帝(1264年11月16日-1274年8月12日在位),宋太祖的第十一世孫,宋理宗養子。生父為榮王趙與芮,生母隆国夫人黄定喜。在位10年,享年34歲,死后葬于永绍陵,谥号为「端文明武景孝皇帝」。

嘉熙四年(1240年)四月九日,赵禥生于绍兴府的荣王府邸,出生时的名字未被记录下来。生母黄定黄是父亲赵与芮亡妻李氏的陪嫁女,因身份卑微,她在怀孕时曾服堕胎药,以致赵禥出生后,手脚皆软,七岁才能说话。《宋史》对此事有所回避,仅称“资识内慧,七岁始言,言必合度,理宗奇之。”

伯父宋理宗子嗣不多,兩名兒子又在幼年夭折,故須在宗室另尋繼任人。宗室中,趙禥是与宋理宗血缘关系最近的人。宋理宗有意以趙禥为继承人。淳祐六年(1246年)十月己丑,伯父赐名孟启,以皇侄授贵州刺史,入内小学。七年正月乙卯,授宜州观察使,就王邸训习。九年正月乙巳,授庆远军节度使,封益国公。十一年正月壬戌,改赐名孜,进封建安郡王。宝祐元年(1253年)正月庚辰,下诏立为皇子,改赐名禥。癸未,授崇庆军节度使、开府仪同三司,进封永嘉郡王。二年十月癸酉,进封忠王。十一月壬寅,加元服,赐字邦寿。五年十月庚子,授镇南、遂安军节度使。景定元年(1260年)六月壬寅,立赵禥为皇太子,赐字长源。次年,其妻永嘉郡夫人全氏被立为皇太子妃。

最初,不少大臣反對立他為太子,立太子之前,大臣吴潜曾密奏,称“臣无弥远之材,忠王无陛下之福”此语激怒宋理宗。宋理宗在立趙禥為太子后,對他刻意栽培,管教甚严。

景定五年冬十月,在位四十年的理宗駕崩,趙禥即位,是為度宗,参考先代咸平 (宋真宗年號) 、淳熙 (宋孝宗年號) 年間国家安定盛世的歷史,改年号为咸淳。

度宗即位时,金朝已经灭亡三十年,北方元朝军队大举南下,国难当头,他却将军国大权交给奸臣贾似道,南宋政治十分腐败黑暗,人民生活十分困苦。度宗甫即位,丞相贾似道便私自往绍兴养老,度宗亲自手书劝他归朝多次,贾似道方才答应。即位之初,度宗便行幸太学,鼓励学问。咸淳二年,民间叶李、萧至二人指责贾似道专权乱政,但度宗反而更加信任贾似道。咸淳三年正月,度宗行幸辟雍同时授予贾似道太师,叶梦鼎右丞相,留梦炎枢密。八月,他写信鼓励边防士卒,其中提到自己经常感染风寒并且从没停止过。十一月,因为极度担心蒙古的侵略,召见武官朱禩并询问边防情况。咸淳五年正月,任命江万里、马廷鸾为左右丞相。咸淳六年三月,度宗下诏禁止奢侈,崇尚节俭。

北方元军多次出兵进攻南宋,南宋朝廷虽腐朽,但是广大军民的抵抗,使得元军不得不撤回,無法攻破长江防缘。度宗即位后,元军猛攻襄陽。然而贾似道密而不报,还说已经取胜(实际上一直僵持),度宗在完全不予以查問的情況下竟对此言深信不疑。最后,元军于咸淳九年(1273年)初攻破围攻5年的襄陽。宋度宗闻知顿时昏倒,之後不思振作,終日借酒浇愁。咸淳十年 (1274年) 七月,度宗因酒色过度而死。死后兩年,元军攻破臨安。

元朝官修正史《宋史》脱脱等的評價是:“宋至理宗,疆宇日蹙,贾似道执国命。度宗继统,虽无大失德,而拱手权奸,衰敝寝甚。考其当时事势,非有雄才睿略之主,岂能振起其坠绪哉!历数有归,宋祚寻讫,亡国不于其身,幸矣。”


\subsection{咸淳}


\begin{longtable}{|>{\centering\scriptsize}m{2em}|>{\centering\scriptsize}m{1.3em}|>{\centering}m{8.8em}|}
  % \caption{秦王政}\
  \toprule
  \SimHei \normalsize 年数 & \SimHei \scriptsize 公元 & \SimHei 大事件 \tabularnewline
  % \midrule
  \endfirsthead
  \toprule
  \SimHei \normalsize 年数 & \SimHei \scriptsize 公元 & \SimHei 大事件 \tabularnewline
  \midrule
  \endhead
  \midrule
  元年 & 1265 & \tabularnewline\hline
  二年 & 1266 & \tabularnewline\hline
  三年 & 1267 & \tabularnewline\hline
  四年 & 1268 & \tabularnewline\hline
  五年 & 1269 & \tabularnewline\hline
  六年 & 1270 & \tabularnewline\hline
  七年 & 1271 & \tabularnewline\hline
  八年 & 1272 & \tabularnewline\hline
  九年 & 1273 & \tabularnewline\hline
  十年 & 1274 & \tabularnewline
  \bottomrule
\end{longtable}



%%% Local Variables:
%%% mode: latex
%%% TeX-engine: xetex
%%% TeX-master: "../Main"
%%% End:
