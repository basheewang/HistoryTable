%% -*- coding: utf-8 -*-
%% Time-stamp: <Chen Wang: 2019-12-26 10:46:25>

\section{宁宗\tiny(1194-1224)}

\subsection{生平}

宋寧宗趙擴(1168年11月18日-1224年9月18日),南宋第四位皇帝(1194年7月24日—1224年9月18日在位)在位30年,享年56岁,宋光宗之次子,李鳳娘所生。寧宗本人頗為好學,即位初年召朱熹入宮講學,受朱熹影響很深,但寧宗政治能力并不十分出色。

寧宗在位前期,太師韓侂胄打壓理學,在韓侂胄死后,官方恢复了理学地位。縱觀宋宁宗时期,大規模宋金戰爭發生了兩次,第一次是開禧初年韓侂胄伐金,最終不能戰勝金國,從而簽訂了嘉定和議。第二次宋金戰爭從嘉定十年開始一直持續到嘉定十四年三月,戰爭波及了長江上游至下游所有地區,最終宋金都沒能獲勝。

宋宁宗生于乾道四年(1168年)十月丙午,五年(1169年)五月赐名赵扩。淳熙五年(1178年)十月封英国公,十二年(1185年)三月封平阳郡王,十六年(1189年)三月进封嘉王。绍熙五年(1194年)为太子,不久紹熙內禪受禪位,其父宋光宗稱為太上皇。紹熙內禪名义上是光宗自行禅位,实际上是赵汝愚、趙彥逾、葉適、徐誼、韓侂冑等人,在獲得隆慈太皇太后支持所造成的宫廷政变,廢除宋光宗,改立宋宁宗。

宋宁宗继位后,宗室宰相赵汝愚与外戚韩侂胄不和,兩者互相争斗。最后韩侂胄使用「宗室不得為宰執」的祖宗家法,让宋宁宗罢免了赵汝愚,并且将其所提倡的理学称为伪学,对理学家造成了打击,造成庆元党禁。这个政策一直维持到1202年,韓侂冑後悔和葉適建言才解除禁制。韩从此成为南宋举足轻重的人物。他的地位和权力远高出一般的宰相。

宋宁宗在南宋统治初期由于韩侂胄的作用对金朝持对抗态度,他追封岳飞为鄂王,剥夺秦桧的所有封职。1206年,韩侂胄北伐战败后宋宁宗改变了政策。1207年11月,其皇后杨氏与史弥远一起秘密策划利用韩侂胄战败的机会谋杀了韩侂胄,并且将韩侂胄的首级送往金朝谢罪。1208年,在史弥远的操纵下,宋宁宗与金朝达成了嘉定和议,向金朝皇帝称伯,自己称侄,將原來的二十萬兩歲幣增至三十萬兩,絹三十萬匹,並且支付金國軍費三百萬兩。韩侂胄死后,史弥远成为了宋宁宗的宰相兼枢密使,独揽大政,同時史弥远恢复了秦桧的王爵和官职。

宋宁宗先后有9个儿子,但是在未成年時就夭折,以後的繼承人問題都有史彌遠介入與操控。曾以養子趙詢為太子,但早死(1207年至1220年在位)。後立养子赵竑为太子(1221年至1224年在位),但是因为赵竑对史弥远专权不满,因此宋宁宗死后史弥远矯詔立赵昀为皇帝。

宋宁宗於嘉定十七年八月初三(1224年9月17日)崩於福寧殿,次年三月葬於永茂陵。

宋寧宗十分重視台諫,但由於他缺乏一定的辨別是非的能力,使得這一政治體制成為權臣控制自己的工具。韓侂胄死後,宋寧宗進行了革除韓侂胄弊政的政治更化。歷史上稱為“嘉定更化”。這些措施包括廣開言路、修正國史、清洗韓黨、平凡昭雪等。但由於寧宗用人失誤,使得更化效果並不理想,適得其反。

1199年五月,寧宗頒佈由楊忠輔創制的新曆法,並賜名為《統天曆》。該曆法是宋代頒行的18種曆法中最精確的,領先西方《格里曆》383年。1202年,宋寧宗令大臣謝深甫等人編纂的《慶元條法事類》成書,並於次年(1203年)正式下詔頒行。

北伐诏书曰:“天道好还,盖中国有必伸之理,人心助顺,虽匹夫无不报之仇。朕丕承万世之基,追述三朝之志。蠢兹逆虏,犹托要盟,朘生灵之资,奉溪壑之欲,此非出于得已,彼乃谓之当然。衣冠遗黎,虐视均于草芥;骨肉同姓,吞噬剧于豺狼。兼别境之侵陵,重连年之水旱,流移罔恤,盗贼恣行。边陲第谨于周防,文牒屡形于恐胁。自处大国,如临小邦,迹其不恭,如务容忍。曾故态之弗改,谓皇朝之可欺,军入塞而公肆创残,使来庭而敢为桀鹜。洎行李之继迁,复慢词之见加,含垢纳污,在人情而已极。声罪致讨,属故运之将倾。兵出有名,师直为壮,况志士仁人挺身而竟节,而谋臣猛将投袂以立功。西北二百州之豪杰,怀旧而愿归;东南七十载之遗黎,久郁而思奋。闻鼓旗之电举,想怒气之飚驰。噫!齐君复仇,上通九世,唐宗刷耻,卒报百王。矧乎家国之仇,接乎月日之近,夙宵是悼,涕泗无从。将勉辑于大勋,必允资于众力。言乎远,言乎迩,孰无中义之心?为人子,为人臣,当念愤。益砺执干之勇,式对在天之灵,庶几中黎旧业之再光,庸示永世宏纲之犹在。布告中外,明体至怀。”

宋史并没有记载宁宗去世是身患何病,而野史《东南纪闻》则记载宁宗病重,史弥远急于拥立理宗即位,于是奉上金丹百粒,宁宗服用后不久就去世。

宋史记载宁宗在即位前曾力请护送高宗的灵柩去会稽下葬,路上见到百姓在田间艰难劳作的场景,感慨地对身边的人说:“平时居住在宫中,哪里知道劳动的辛苦!”此外在个人生活上,宁宗也力行节俭,穿戴也较为朴素,使用的酒器都是以锡代银。有一年元宵夜,他独自对着蜡烛清坐,一个宦官见此劝他设宴过节,宁宗以外间百姓无饭可吃而拒绝。宁宗还曾游幸聚景园,晚上回宫的时候,临安的百姓争相观看,都想一睹天子之容。不幸的是有人被践踏踩死,宁宗得知此事后十分后悔,从此再也不出宫游玩了。

宁宗常让两个小太监背着两扇屏风作为他的前导,走到哪里都要跟随。屏风用白纸作底,边上糊着青纸,上写着“少饮酒,怕吐;少食生冷,怕痛。”当大臣让他喝酒或吃生冷食物时,他就指指屏风加以拒绝。就算饮酒也不超过三杯。

嘉泰年间,宁宗有意前往西湖泛舟游赏。有个叫张巨济的大臣上书劝谏宁宗道:“慈懿皇后的陵寝近在湖滨,陛下出游,难免要鼓乐一番,岂不是要惊动先人的在天之灵吗?”宁宗认为他说的很有道理,不仅升了他的俸禄,还把游船都沉到湖底,以表示自己不再游湖的决心。

\subsection{庆元}


\begin{longtable}{|>{\centering\scriptsize}m{2em}|>{\centering\scriptsize}m{1.3em}|>{\centering}m{8.8em}|}
  % \caption{秦王政}\
  \toprule
  \SimHei \normalsize 年数 & \SimHei \scriptsize 公元 & \SimHei 大事件 \tabularnewline
  % \midrule
  \endfirsthead
  \toprule
  \SimHei \normalsize 年数 & \SimHei \scriptsize 公元 & \SimHei 大事件 \tabularnewline
  \midrule
  \endhead
  \midrule
  元年 & 1195 & \tabularnewline\hline
  二年 & 1196 & \tabularnewline\hline
  三年 & 1197 & \tabularnewline\hline
  四年 & 1198 & \tabularnewline\hline
  五年 & 1199 & \tabularnewline\hline
  六年 & 1200 & \tabularnewline
  \bottomrule
\end{longtable}

\subsection{嘉泰}

\begin{longtable}{|>{\centering\scriptsize}m{2em}|>{\centering\scriptsize}m{1.3em}|>{\centering}m{8.8em}|}
  % \caption{秦王政}\
  \toprule
  \SimHei \normalsize 年数 & \SimHei \scriptsize 公元 & \SimHei 大事件 \tabularnewline
  % \midrule
  \endfirsthead
  \toprule
  \SimHei \normalsize 年数 & \SimHei \scriptsize 公元 & \SimHei 大事件 \tabularnewline
  \midrule
  \endhead
  \midrule
  元年 & 1201 & \tabularnewline\hline
  二年 & 1202 & \tabularnewline\hline
  三年 & 1203 & \tabularnewline\hline
  四年 & 1204 & \tabularnewline
  \bottomrule
\end{longtable}

\subsection{开禧}

\begin{longtable}{|>{\centering\scriptsize}m{2em}|>{\centering\scriptsize}m{1.3em}|>{\centering}m{8.8em}|}
  % \caption{秦王政}\
  \toprule
  \SimHei \normalsize 年数 & \SimHei \scriptsize 公元 & \SimHei 大事件 \tabularnewline
  % \midrule
  \endfirsthead
  \toprule
  \SimHei \normalsize 年数 & \SimHei \scriptsize 公元 & \SimHei 大事件 \tabularnewline
  \midrule
  \endhead
  \midrule
  元年 & 1205 & \tabularnewline\hline
  二年 & 1206 & \tabularnewline\hline
  三年 & 1207 & \tabularnewline
  \bottomrule
\end{longtable}

\subsection{嘉定}

\begin{longtable}{|>{\centering\scriptsize}m{2em}|>{\centering\scriptsize}m{1.3em}|>{\centering}m{8.8em}|}
  % \caption{秦王政}\
  \toprule
  \SimHei \normalsize 年数 & \SimHei \scriptsize 公元 & \SimHei 大事件 \tabularnewline
  % \midrule
  \endfirsthead
  \toprule
  \SimHei \normalsize 年数 & \SimHei \scriptsize 公元 & \SimHei 大事件 \tabularnewline
  \midrule
  \endhead
  \midrule
  元年 & 1208 & \tabularnewline\hline
  二年 & 1209 & \tabularnewline\hline
  三年 & 1210 & \tabularnewline\hline
  四年 & 1211 & \tabularnewline\hline
  五年 & 1212 & \tabularnewline\hline
  六年 & 1213 & \tabularnewline\hline
  七年 & 1214 & \tabularnewline\hline
  八年 & 1215 & \tabularnewline\hline
  九年 & 1216 & \tabularnewline\hline
  十年 & 1217 & \tabularnewline\hline
  十一年 & 1218 & \tabularnewline\hline
  十二年 & 1219 & \tabularnewline\hline
  十三年 & 1220 & \tabularnewline\hline
  十四年 & 1221 & \tabularnewline\hline
  十五年 & 1222 & \tabularnewline\hline
  十六年 & 1223 & \tabularnewline\hline
  十七年 & 1224 & \tabularnewline
  \bottomrule
\end{longtable}



%%% Local Variables:
%%% mode: latex
%%% TeX-engine: xetex
%%% TeX-master: "../Main"
%%% End:
