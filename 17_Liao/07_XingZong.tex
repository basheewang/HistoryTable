%% -*- coding: utf-8 -*-
%% Time-stamp: <Chen Wang: 2021-11-01 16:02:08>

\section{兴宗耶律宗真\tiny(1031-1055)}

\subsection{生平}

遼興宗耶律宗真(1016年4月3日-1055年8月28日),契丹第七位皇帝(1031年6月25日-1055年8月28日在位),契丹名只骨。在位24年,享年40歲,謚孝章皇帝。他是遼聖宗的長子,母乃宮女蕭耨斤。

耶律宗真生于1016年,其后,由辽圣宗的皇后蕭菩薩哥抚养。《辽史》记耶律宗真为“圣宗长子”,实际上辽圣宗第六子耶律宗愿的生年是1008年或1009年,耶律宗真和同母弟耶律宗元应该是辽圣宗最年幼的两个儿子。虽然年幼,但与其他四位皇子相比,只有耶律宗真兄弟的生母蕭耨斤出身于契丹萧氏,其他皇子的生母出身于汉族或不详。

太平元年(1021年),耶律宗真被冊立為太子,太平十年(1030年)六月判北南院枢密院事。太平十一年(1031年6月25日)夏六月己卯,辽圣宗逝世,同时,耶律宗真繼承皇位,改元景福。興宗繼位後,其母順聖元妃蕭耨斤自立為皇太后攝政,並把聖宗的齊天皇后迫死。並重用了在聖宗時代被裁示永不錄用的貪官污吏以及其娘家的人。

景福二年十一月,興宗上太后尊號為法天應運仁德章聖皇太后(法天太后),而興宗被群臣上尊號為文武仁聖昭孝皇帝,改元重熙。

重熙三年(1034年),法天太后企圖廢掉興宗,改立次子宗元(遼史作重元),重元告訴其兄興宗,興宗發動政變,迫法天太后「躬守慶陵」。大殺太后親信。七月,興宗親政。

興宗在位時,遼國勢已日益衰落。而有興宗一朝,奸佞當權,政治腐敗,百姓困苦,軍隊衰弱。面對日益衰落的國勢,興宗連年征戰,多次征伐西夏;逼迫宋朝多交納歲幣,反而使遼內部百姓怨聲載道,民不聊生。興宗還迷信佛教,窮途奢極。興宗曾與其弟宗元賭博,一連輸了幾個城池。

他對自己的弟弟宗元非常感激,一次酒醉時答應百年之後傳位給宗元,其子耶律洪基(後來的遼道宗)也未曾封為皇太子,只封為天下兵馬大元帥而已。種下了道宗繼位後,宗元父子企圖謀奪帝位的惡果。

重熙二十四年八月初四日(1055年8月28日),興宗駕崩。

元朝官修正史《辽史》脱脱等的評價是:“兴宗即位,年十有六矣,不能先尊母后而尊其母,以致临朝专政,贼杀不辜,又不能以礼几谏,使齐天死于弑逆,有亏王者之孝,惜哉!若夫大行在殡,饮酒博鞠,叠见简书。及其谒遗像而哀恸,受宋吊而衰绖,所为若出二人。何为其然欤?至于感富弼之言而申南宋之好,许谅祚之盟而罢西夏之兵,边鄙不耸,政治内修,亲策进士,大修条制,下至士庶,得陈便宜,则求治之志切矣。于时左右大臣,曾不闻一贤之进,一事之谏,欲庶几古帝王之风,其可得乎?虽然,圣宗而下,可谓贤君矣。 ”

\subsection{景福}

\begin{longtable}{|>{\centering\scriptsize}m{2em}|>{\centering\scriptsize}m{1.3em}|>{\centering}m{8.8em}|}
  % \caption{秦王政}\
  \toprule
  \SimHei \normalsize 年数 & \SimHei \scriptsize 公元 & \SimHei 大事件 \tabularnewline
  % \midrule
  \endfirsthead
  \toprule
  \SimHei \normalsize 年数 & \SimHei \scriptsize 公元 & \SimHei 大事件 \tabularnewline
  \midrule
  \endhead
  \midrule
  元年 & 1031 & \tabularnewline\hline
  二年 & 1032 & \tabularnewline
  \bottomrule
\end{longtable}

\subsection{重熙}

\begin{longtable}{|>{\centering\scriptsize}m{2em}|>{\centering\scriptsize}m{1.3em}|>{\centering}m{8.8em}|}
  % \caption{秦王政}\
  \toprule
  \SimHei \normalsize 年数 & \SimHei \scriptsize 公元 & \SimHei 大事件 \tabularnewline
  % \midrule
  \endfirsthead
  \toprule
  \SimHei \normalsize 年数 & \SimHei \scriptsize 公元 & \SimHei 大事件 \tabularnewline
  \midrule
  \endhead
  \midrule
  元年 & 1032 & \tabularnewline\hline
  二年 & 1033 & \tabularnewline\hline
  三年 & 1034 & \tabularnewline\hline
  四年 & 1035 & \tabularnewline\hline
  五年 & 1036 & \tabularnewline\hline
  六年 & 1037 & \tabularnewline\hline
  七年 & 1038 & \tabularnewline\hline
  八年 & 1039 & \tabularnewline\hline
  九年 & 1040 & \tabularnewline\hline
  十年 & 1041 & \tabularnewline\hline
  十一年 & 1042 & \tabularnewline\hline
  十二年 & 1043 & \tabularnewline\hline
  十三年 & 1044 & \tabularnewline\hline
  十四年 & 1045 & \tabularnewline\hline
  十五年 & 1046 & \tabularnewline\hline
  十六年 & 1047 & \tabularnewline\hline
  十七年 & 1048 & \tabularnewline\hline
  十八年 & 1049 & \tabularnewline\hline
  十九年 & 1050 & \tabularnewline\hline
  二十年 & 1051 & \tabularnewline\hline
  二一年 & 1052 & \tabularnewline\hline
  二二年 & 1053 & \tabularnewline\hline
  二三年 & 1054 & \tabularnewline\hline
  二四年 & 1055 & \tabularnewline
  \bottomrule
\end{longtable}


%%% Local Variables:
%%% mode: latex
%%% TeX-engine: xetex
%%% TeX-master: "../Main"
%%% End:
