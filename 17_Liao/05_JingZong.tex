%% -*- coding: utf-8 -*-
%% Time-stamp: <Chen Wang: 2019-12-26 10:57:54>

\section{景宗\tiny(969-982)}

\subsection{生平}

遼景宗耶律賢(948年9月1日-982年10月13日),字賢寧,遼朝第五位皇帝(969年3月13日-982年10月13日在位),在位13年,遼世宗的次子,其母為懷節皇后蕭氏。

在遼世宗在位時的政變中,耶律賢險而被殺,後來得人所救。951年,其父辽世宗被刺身亡,堂叔耶律璟即位,是为辽穆宗。969年3月12日,堂叔遼穆宗被弑,次日,耶律賢被推舉為帝,尊號天贊皇帝,改元為保寧。

耶律賢从小惊吓过度,体弱多病,皇后萧绰(953年-1009年,小字燕燕,原姓拔黎氏)则成了辽国政治军事的参与者。景宗在位時復回登聞鼓院,令百姓有申冤之地,又寬減刑法,對百姓加以安撫。

後來,景宗於乾亨四年九月廿四日(即982年10月13日)死於現今的山西省大同市,享年三十五歲,葬於乾陵,位于今辽宁省北镇市。

元朝官修正史《辽史》脱脱等的評價是:“辽兴六十馀年,神册、会同之间,日不暇给;天禄、应历之君,不令其终;保宁而来,人人望治。以景宗之资,任人不疑,信赏必罚,若可与有为也。而竭国之力以助河东,破军杀将,无救灭亡。虽一取偿于宋,得不偿失。知匡嗣之罪,数而不罚;善郭袭之谏,纳而不用;沙门昭敏以左道乱德,宠以侍中。不亦惑乎!”

\subsection{保宁}

\begin{longtable}{|>{\centering\scriptsize}m{2em}|>{\centering\scriptsize}m{1.3em}|>{\centering}m{8.8em}|}
  % \caption{秦王政}\
  \toprule
  \SimHei \normalsize 年数 & \SimHei \scriptsize 公元 & \SimHei 大事件 \tabularnewline
  % \midrule
  \endfirsthead
  \toprule
  \SimHei \normalsize 年数 & \SimHei \scriptsize 公元 & \SimHei 大事件 \tabularnewline
  \midrule
  \endhead
  \midrule
  元年 & 969 & \tabularnewline\hline
  二年 & 970 & \tabularnewline\hline
  三年 & 971 & \tabularnewline\hline
  四年 & 972 & \tabularnewline\hline
  五年 & 973 & \tabularnewline\hline
  六年 & 974 & \tabularnewline\hline
  七年 & 975 & \tabularnewline\hline
  八年 & 976 & \tabularnewline\hline
  九年 & 977 & \tabularnewline\hline
  十年 & 978 & \tabularnewline\hline
  十一年 & 979 & \tabularnewline
  \bottomrule
\end{longtable}

\subsection{乾亨}

\begin{longtable}{|>{\centering\scriptsize}m{2em}|>{\centering\scriptsize}m{1.3em}|>{\centering}m{8.8em}|}
  % \caption{秦王政}\
  \toprule
  \SimHei \normalsize 年数 & \SimHei \scriptsize 公元 & \SimHei 大事件 \tabularnewline
  % \midrule
  \endfirsthead
  \toprule
  \SimHei \normalsize 年数 & \SimHei \scriptsize 公元 & \SimHei 大事件 \tabularnewline
  \midrule
  \endhead
  \midrule
  元年 & 979 & \tabularnewline\hline
  二年 & 980 & \tabularnewline\hline
  三年 & 981 & \tabularnewline\hline
  四年 & 982 & \tabularnewline\hline
  五年 & 983 & \tabularnewline
  \bottomrule
\end{longtable}



%%% Local Variables:
%%% mode: latex
%%% TeX-engine: xetex
%%% TeX-master: "../Main"
%%% End:
