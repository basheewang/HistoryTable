%% -*- coding: utf-8 -*-
%% Time-stamp: <Chen Wang: 2019-10-15 16:18:25>

\section{世宗\tiny(947-951)}

遼世宗耶律阮(919年1月29日-951年10月7日),中国遼朝第三位皇帝(947年5月16日-951年10月7日在位),在位4年。契丹迭剌部霞濑益石烈乡耶律里(今中国内蒙古阿鲁科尔沁旗东)人,姓耶律,汉文名阮,契丹文名兀欲(又名隈欲、烏雲),他是大契丹国(後改称大辽国)皇太子、人皇王、東丹国王、遼義宗让国皇帝(追尊,未即位)耶律倍的長子、太祖耶律阿保機的长孙、太宗耶律德光之侄。

阿保机死后,世宗耶律阮之父人皇王耶律倍在权力斗争中失利,未能即位为帝,耶律阮遂失去继承皇位的权利。人皇王後来愤而投奔后唐,终于客死他乡。耶律阮则留在国内,后随叔父太宗耶律德光南征後晋。太宗在北归途中病逝后,耶律阮被随军将领拥立为帝,是为辽世宗。但世宗即位后发生多起夺权事变,统治活动被严重干扰,最终遇刺身亡,在位仅四年有余,其堂弟耶律璟继位,是为辽穆宗。

世宗虽在辽代诸帝中享国最短,却是一位有作为的皇帝。受其父耶律倍的影响,世宗在位期间推崇汉文化,推广中原制度,在世宗之孙圣宗时最后完成,促进了辽国社会的发展。

契丹神册三年,耶律阮出生,他的父亲是契丹开国皇帝耶律阿保机的长子耶律倍,母亲是耶律倍之妃萧氏(死后追谥柔贞皇后)。祖父阿保机死后,父亲耶律倍在权力斗争中失利,不得立为皇帝,耶律阮也就失去了继承皇位的权利。耶律倍后来愤而渡海投奔后唐,终于於932年末被後唐末帝李從珂殺害,客死他乡。耶律阮则留在契丹国内,其叔父太宗耶律德光爱之如己出。契丹会同九年、后晋开运三年(946年),太宗以后晋皇帝石重贵不肯称臣为由大举入侵中原,耶律阮随行军中。第二年(947年)契丹军入后晋国都东京开封府(今河南省开封市),晋帝石重贵投降,后晋灭亡。太宗改国号为“大辽”,改元大同,封耶律阮为永康王。

太宗滅後晉后在北归途中逝世,耶律阮發兵奪取南京析津府(今北京),並在随军将领拥戴下自立為皇帝,在上京(今內蒙古巴林左旗)的蕭太后述律平派其子耶律李胡在南京北部的泰德泉交戰,大敗。經過大臣耶律屋質的勸阻,太后才同意耶律阮當皇帝。世宗時任用賢臣耶律屋質,進行一系列改革,將太宗時的南面官和北面官合併,成立南北樞密院,廢南、北大王,後來南北樞密院合併,形成一個樞密院。這些改革使遼朝從部落聯盟形式進入中央集權,這些都是與遼世宗的改革分不開的。但是世宗好酒色,喜愛打獵。他晚年更是任用奸佞,大興封賞降殺,導致朝政不修,政治腐敗。遼天祿五年(951年)9月,世宗協助北漢攻後周,行軍至歸化(今內蒙古呼和浩特)的祥古山,由於其他部隊未到,所以駐紮在火神澱。其間喝酒、打人、打獵,眾將很是不滿。晚上,一直有篡位之心的耶律察割將遼世宗耶律阮殺死於夢鄉。耶律阮死時年僅34歲,在位4年。其諡號為孝和莊憲皇帝,廟號世宗。

\subsection{天禄}

\begin{longtable}{|>{\centering\scriptsize}m{2em}|>{\centering\scriptsize}m{1.3em}|>{\centering}m{8.8em}|}
  % \caption{秦王政}\
  \toprule
  \SimHei \normalsize 年数 & \SimHei \scriptsize 公元 & \SimHei 大事件 \tabularnewline
  % \midrule
  \endfirsthead
  \toprule
  \SimHei \normalsize 年数 & \SimHei \scriptsize 公元 & \SimHei 大事件 \tabularnewline
  \midrule
  \endhead
  \midrule
  元年 & 947 & \tabularnewline\hline
  二年 & 948 & \tabularnewline\hline
  三年 & 949 & \tabularnewline\hline
  四年 & 950 & \tabularnewline\hline
  五年 & 951 & \tabularnewline
  \bottomrule
\end{longtable}



%%% Local Variables:
%%% mode: latex
%%% TeX-engine: xetex
%%% TeX-master: "../Main"
%%% End:
