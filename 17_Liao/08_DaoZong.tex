%% -*- coding: utf-8 -*-
%% Time-stamp: <Chen Wang: 2019-10-15 16:29:50>

\section{道宗\tiny(1055-1101)}

遼道宗耶律洪基(1032年9月14日-1101年2月12日),契丹及遼朝第八位皇帝(1055年8月28日-1101年2月12日在位),在位長達46年,僅次於遼聖宗。他是遼興宗的長子,契丹名查剌。

重熙二十四年八月初四(1055年8月28日),興宗駕崩,即位於柩前。改元清寧。

道宗繼位後,封皇叔宗元為皇太叔,清寧二年又加天下兵馬大元帥。四年又賜金券等,極盡榮寵。但宗元始終有謀奪帝位的意圖,在清寧九年(1063年)七月,宗元聽從兒子的勸說,發動叛亂,自立為帝,未幾被道宗所平,宗元自盡。史稱灤河之亂。

咸雍二年(1066年),辽道宗把国号“契丹”改为“大辽”。

他在位期間,遼政治腐敗,國勢逐漸衰落。道宗並沒有進行改革圖新,而且本人也腐朽奢侈,這時地主官僚急劇兼併土地,百姓痛苦不堪,怨聲載道。道宗還重用耶律乙辛等奸佞,自己不理朝政,導致他聽信乙辛的讒言,相信皇后蕭觀音與伶官趙惟一通姦而賜死皇后,史稱十香詞冤案。而同時乙辛為防太子耶律濬登基對自己不利(因為道宗只有皇太子這個兒子),故陷害皇太子謀反,殺害了皇太子。

後來,一位姓李的婦女向道宗進「挾穀歌」,道宗才把皇太子的兒女接進宮,大康五年(1079年)七月,耶律乙辛乘道宗遊獵的時候謀害皇孫,道宗接納大臣的勸諫,命皇孫一同秋獵,才化解乙辛的陰謀。

大康九年,道宗追封故太子為昭懷太子,以天子禮改葬。同年十月,耶律乙辛企圖帶私藏武器到宋朝避難,事發,被誅。

道宗篤信佛教,在位期間曾大修佛寺、佛塔。遼的腐朽統治引起了各族人民的不滿,其間被遼統治者壓迫的女真族開始興起,最終成為遼的掘墓人。

寿昌七年正月十三日(1101年2月12日),遼道宗去世,終年70歲。

元朝官修正史《辽史》脱脱等的評價是:“道宗初即位,求直言,访治道,劝农兴学,救灾恤患,粲然可观。及夫谤讪之令既行,告讦之赏日重。群邪并兴,谗巧竞进。贼及骨肉,皇基浸危。众正沦胥,诸部反侧,甲兵之用,无宁岁矣。一岁而饭僧三十六万,一日而祝发三千。徒勤小惠,蔑计大本,尚足与论治哉? ”

\subsection{清宁}

\begin{longtable}{|>{\centering\scriptsize}m{2em}|>{\centering\scriptsize}m{1.3em}|>{\centering}m{8.8em}|}
  % \caption{秦王政}\
  \toprule
  \SimHei \normalsize 年数 & \SimHei \scriptsize 公元 & \SimHei 大事件 \tabularnewline
  % \midrule
  \endfirsthead
  \toprule
  \SimHei \normalsize 年数 & \SimHei \scriptsize 公元 & \SimHei 大事件 \tabularnewline
  \midrule
  \endhead
  \midrule
  元年 & 1055 & \tabularnewline\hline
  二年 & 1056 & \tabularnewline\hline
  三年 & 1057 & \tabularnewline\hline
  四年 & 1058 & \tabularnewline\hline
  五年 & 1059 & \tabularnewline\hline
  六年 & 1060 & \tabularnewline\hline
  七年 & 1061 & \tabularnewline\hline
  八年 & 1062 & \tabularnewline\hline
  九年 & 1063 & \tabularnewline\hline
  十年 & 1064 & \tabularnewline
  \bottomrule
\end{longtable}

\subsection{咸雍}

\begin{longtable}{|>{\centering\scriptsize}m{2em}|>{\centering\scriptsize}m{1.3em}|>{\centering}m{8.8em}|}
  % \caption{秦王政}\
  \toprule
  \SimHei \normalsize 年数 & \SimHei \scriptsize 公元 & \SimHei 大事件 \tabularnewline
  % \midrule
  \endfirsthead
  \toprule
  \SimHei \normalsize 年数 & \SimHei \scriptsize 公元 & \SimHei 大事件 \tabularnewline
  \midrule
  \endhead
  \midrule
  元年 & 1065 & \tabularnewline\hline
  二年 & 1066 & \tabularnewline\hline
  三年 & 1067 & \tabularnewline\hline
  四年 & 1068 & \tabularnewline\hline
  五年 & 1069 & \tabularnewline\hline
  六年 & 1070 & \tabularnewline\hline
  七年 & 1071 & \tabularnewline\hline
  八年 & 1072 & \tabularnewline\hline
  九年 & 1073 & \tabularnewline\hline
  十年 & 1074 & \tabularnewline
  \bottomrule
\end{longtable}

\subsection{大康}

\begin{longtable}{|>{\centering\scriptsize}m{2em}|>{\centering\scriptsize}m{1.3em}|>{\centering}m{8.8em}|}
  % \caption{秦王政}\
  \toprule
  \SimHei \normalsize 年数 & \SimHei \scriptsize 公元 & \SimHei 大事件 \tabularnewline
  % \midrule
  \endfirsthead
  \toprule
  \SimHei \normalsize 年数 & \SimHei \scriptsize 公元 & \SimHei 大事件 \tabularnewline
  \midrule
  \endhead
  \midrule
  元年 & 1075 & \tabularnewline\hline
  二年 & 1076 & \tabularnewline\hline
  三年 & 1077 & \tabularnewline\hline
  四年 & 1078 & \tabularnewline\hline
  五年 & 1079 & \tabularnewline\hline
  六年 & 1080 & \tabularnewline\hline
  七年 & 1081 & \tabularnewline\hline
  八年 & 1082 & \tabularnewline\hline
  九年 & 1083 & \tabularnewline\hline
  十年 & 1084 & \tabularnewline
  \bottomrule
\end{longtable}

\subsection{大安}

\begin{longtable}{|>{\centering\scriptsize}m{2em}|>{\centering\scriptsize}m{1.3em}|>{\centering}m{8.8em}|}
  % \caption{秦王政}\
  \toprule
  \SimHei \normalsize 年数 & \SimHei \scriptsize 公元 & \SimHei 大事件 \tabularnewline
  % \midrule
  \endfirsthead
  \toprule
  \SimHei \normalsize 年数 & \SimHei \scriptsize 公元 & \SimHei 大事件 \tabularnewline
  \midrule
  \endhead
  \midrule
  元年 & 1085 & \tabularnewline\hline
  二年 & 1086 & \tabularnewline\hline
  三年 & 1087 & \tabularnewline\hline
  四年 & 1088 & \tabularnewline\hline
  五年 & 1089 & \tabularnewline\hline
  六年 & 1090 & \tabularnewline\hline
  七年 & 1091 & \tabularnewline\hline
  八年 & 1092 & \tabularnewline\hline
  九年 & 1093 & \tabularnewline\hline
  十年 & 1094 & \tabularnewline
  \bottomrule
\end{longtable}

\subsection{寿昌}

\begin{longtable}{|>{\centering\scriptsize}m{2em}|>{\centering\scriptsize}m{1.3em}|>{\centering}m{8.8em}|}
  % \caption{秦王政}\
  \toprule
  \SimHei \normalsize 年数 & \SimHei \scriptsize 公元 & \SimHei 大事件 \tabularnewline
  % \midrule
  \endfirsthead
  \toprule
  \SimHei \normalsize 年数 & \SimHei \scriptsize 公元 & \SimHei 大事件 \tabularnewline
  \midrule
  \endhead
  \midrule
  元年 & 1095 & \tabularnewline\hline
  二年 & 1096 & \tabularnewline\hline
  三年 & 1097 & \tabularnewline\hline
  四年 & 1098 & \tabularnewline\hline
  五年 & 1099 & \tabularnewline\hline
  六年 & 1100 & \tabularnewline\hline
  七年 & 1101 & \tabularnewline
  \bottomrule
\end{longtable}


%%% Local Variables:
%%% mode: latex
%%% TeX-engine: xetex
%%% TeX-master: "../Main"
%%% End:
