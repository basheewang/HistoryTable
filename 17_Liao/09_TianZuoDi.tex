%% -*- coding: utf-8 -*-
%% Time-stamp: <Chen Wang: 2019-12-26 10:58:23>

\section{天祚帝\tiny(1101-1125)}

\subsection{生平}

遼天祚帝耶律延禧(1075年6月5日-1128年或1156年),字延宁,小名阿果,是遼國西遷前的最后一位皇帝,他的统治时间是从1101年2月12日至1125年3月26日,在位24年。

天祚帝是辽道宗的孙子,他的父亲是道宗的太子耶律濬,母亲是貞順皇后萧氏。六岁时他被封为梁王,九岁时封为燕国王。

寿昌七年正月十三日(1101年2月12日),道宗崩,临死前立耶律延禧为继承人,耶律延禧奉遗诏即皇帝位于柩前。延禧以「天祚皇帝」作為自己的尊號。二月壬辰改元乾統。

天祚帝继位后西夏崇宗因受到北宋攻击一再向辽求援,并求尚天祚帝女公主为妻,最后天祚帝于1105年将一个族女封为公主嫁给了夏崇宗,并派使者赴宋,劝宋对西夏罢兵。

1112年二月丁酉天祚帝赴春州,召集附近的女真族酋长来朝,宴席中醉酒后令女真酋长为他跳舞,只有完颜阿骨打不肯。天祚帝不以为意,但从此完颜阿骨打与遼國之间不和。从九月开始完颜阿骨打不再奉诏,并开始对其他不服从自己的女真部落用兵。1114年春,完颜阿骨打正式起兵反辽。一开始天祚帝不将阿骨打当作大威胁,但是1114年天祚帝所有派去镇压阿骨打的军队全部被战败。

1115年天祚帝終於开始觉察到女真的威胁勢力,下令亲征,但是辽军到处被女真打败,与此同时遼國国内也发生叛乱,耶律章奴在上京临潢府叛乱,虽然这场叛乱很快就被平定,但是这场叛乱分裂了遼國内部。此后位于原渤海国的东京辽阳府也发生叛乱自立。这场叛乱一直到1116年四月才被平定。但是在五月女真就借机占领了辽阳和瀋州。1117年女真攻春州,辽军不战自败。这年完颜阿骨打称帝,建立金朝。

1120年金攻克上京臨潢府,留守降。到1121年辽已经失去了其疆域之半。而遼國内部又发生了因为皇位继承问题而爆发的内乱,1122年天祚帝杀了自己的长子耶律敖卢斡,这使得更多的辽國军人感到不安而投靠金朝。四月,金攻克辽西京大同府。由于战场上消息不通,遼國内部又以为天祚帝在前线阵亡或被围,于是在臨潢立耶律淳为皇帝,进一步扩大了遼國内部的混乱。而遼國的大臣也各不自保,有的与北宋大臣童贯通气打算投降宋朝的,有的则想投降金朝。十一月居庸关失守,十二月辽南京被攻破。1123年正月上京叛金。

到1124年天祚帝已经失去了遼國的大部分土地而退出漠外,他的儿子和家属大多数被杀或被俘,虽然他还打算重新守護燕州和云州,但是实际上他已经没有多少希望了。保大五年二月二十日(1125年3月26日)天祚帝在应州为金人完颜娄室等所俘,八月被解送金上京,被降为海滨王。金太宗天會六年(1128年)病死。金皇統元年(1141年),改封豫王。皇統五年(1145年),葬於乾陵旁。

《大宋宣和遺事》則記載南宋紹興二十六年(金朝正隆元年,1156年)六月,金朝皇帝完顏亮命令56歲的宋欽宗趙桓和81歲的耶律延禧去比賽馬球,趙桓中途從馬上跌下來,被馬亂踐而死,耶律延禧則因善騎術,企圖縱馬衝出重圍逃命,結果被金人以亂箭射死。

元朝官修正史《辽史》脱脱等的評價是:“辽起朔野,兵甲之盛,鼓行皞外,席卷河朔,树晋植汉,何其壮欤?太祖、太宗乘百战之势,辑新造之邦,英谋睿略,可谓远矣。虽以世宗中才,穆宗残暴,连遘弑逆,而神器不摇。盖由祖宗威令犹足以震叠其国人也。圣宗以来,内修政治,外拓疆宇,既而申固邻好,四境乂安。维侍二百余年之基,有自来矣。降臻天祚,既丁末运,又觖人望,崇信奸回,自椓国本,群下离心。金兵一集,内难先作,废立之谋,叛亡之迹,相继蜂起。驯致土崩瓦解,不可复支,良可哀也!耶律与萧,世为甥舅,义同休戚,奉先挟私灭公,首祸构难,一至于斯。天祚穷蹙,始悟奉先误己,不几晚乎!淳、雅里所谓名不正,言不顺,事不成者也。大石苟延,彼善于此,亦几何哉?”

\subsection{乾统}

\begin{longtable}{|>{\centering\scriptsize}m{2em}|>{\centering\scriptsize}m{1.3em}|>{\centering}m{8.8em}|}
  % \caption{秦王政}\
  \toprule
  \SimHei \normalsize 年数 & \SimHei \scriptsize 公元 & \SimHei 大事件 \tabularnewline
  % \midrule
  \endfirsthead
  \toprule
  \SimHei \normalsize 年数 & \SimHei \scriptsize 公元 & \SimHei 大事件 \tabularnewline
  \midrule
  \endhead
  \midrule
  元年 & 1101 & \tabularnewline\hline
  二年 & 1102 & \tabularnewline\hline
  三年 & 1103 & \tabularnewline\hline
  四年 & 1104 & \tabularnewline\hline
  五年 & 1105 & \tabularnewline\hline
  六年 & 1106 & \tabularnewline\hline
  七年 & 1107 & \tabularnewline\hline
  八年 & 1108 & \tabularnewline\hline
  九年 & 1109 & \tabularnewline\hline
  十年 & 1110 & \tabularnewline
  \bottomrule
\end{longtable}

\subsection{天庆}

\begin{longtable}{|>{\centering\scriptsize}m{2em}|>{\centering\scriptsize}m{1.3em}|>{\centering}m{8.8em}|}
  % \caption{秦王政}\
  \toprule
  \SimHei \normalsize 年数 & \SimHei \scriptsize 公元 & \SimHei 大事件 \tabularnewline
  % \midrule
  \endfirsthead
  \toprule
  \SimHei \normalsize 年数 & \SimHei \scriptsize 公元 & \SimHei 大事件 \tabularnewline
  \midrule
  \endhead
  \midrule
  元年 & 1111 & \tabularnewline\hline
  二年 & 1112 & \tabularnewline\hline
  三年 & 1113 & \tabularnewline\hline
  四年 & 1114 & \tabularnewline\hline
  五年 & 1115 & \tabularnewline\hline
  六年 & 1116 & \tabularnewline\hline
  七年 & 1117 & \tabularnewline\hline
  八年 & 1118 & \tabularnewline\hline
  九年 & 1119 & \tabularnewline\hline
  十年 & 1120 & \tabularnewline
  \bottomrule
\end{longtable}

\subsection{保大}

\begin{longtable}{|>{\centering\scriptsize}m{2em}|>{\centering\scriptsize}m{1.3em}|>{\centering}m{8.8em}|}
  % \caption{秦王政}\
  \toprule
  \SimHei \normalsize 年数 & \SimHei \scriptsize 公元 & \SimHei 大事件 \tabularnewline
  % \midrule
  \endfirsthead
  \toprule
  \SimHei \normalsize 年数 & \SimHei \scriptsize 公元 & \SimHei 大事件 \tabularnewline
  \midrule
  \endhead
  \midrule
  元年 & 1121 & \tabularnewline\hline
  二年 & 1122 & \tabularnewline\hline
  三年 & 1123 & \tabularnewline\hline
  四年 & 1124 & \tabularnewline\hline
  五年 & 1125 & \tabularnewline
  \bottomrule
\end{longtable}


%%% Local Variables:
%%% mode: latex
%%% TeX-engine: xetex
%%% TeX-master: "../Main"
%%% End:
