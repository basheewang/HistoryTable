%% -*- coding: utf-8 -*-
%% Time-stamp: <Chen Wang: 2021-11-01 16:12:23>

\section{西辽\tiny(1124-1218)}

\subsection{简介}

西辽(1124年-1218年),又称喀喇契丹,是契丹人耶律大石建立的国家。耶律大石原本效力于辽天祚帝,在辽朝即将灭亡之际出奔。1124年,耶律大石称王,到达可敦城(今蒙古国布尔干省青托罗盖古回鹘城)建立根据地。1132年,在叶密立(今新疆维吾尔自治区额敏县)称“菊儿汗”,西辽帝国正式建立。随后耶律大石向新疆、蒙古高原、中亚及西亚地区扩张,建都于虎思斡鲁朵(今吉尔吉斯斯坦托克玛克东南布拉纳)。在1141年的卡特万之战,击败塞尔柱帝国联军,成为中亚霸主,将威名远播至欧洲。高昌回鹘、西喀喇汗国、东喀喇汗国及花剌子模先后臣服于强盛期的西辽。耶律大石死后,历经萧塔不烟、耶律夷列、耶律普速完三代君主后,到耶律直鲁古时期,由于长期对外战争,使西辽的国力走向衰落,最终被屈出律篡国。蒙古帝国崛起后,1218年,西辽被蒙古帝国灭亡。

\subsection{德宗耶律大石\tiny(1124-1143)}

\subsubsection{生平}

遼德宗耶律大石(1094年-1143年),又称大石林牙或林牙大石。字重德,契丹人。西辽開國皇帝,庙号德宗,在位20年。

耶律大石早年效力于辽天祚帝,辽天祚帝出奔后,耶律大石参与拥立耶律淳和萧德妃,在北宋、金朝两面夹击的情况下,积极维持风雨飘摇的北辽,两次率军以少胜多击败北宋的进攻。北辽灭亡后,耶律大石投奔天祚帝,在辽朝即将灭亡之际出奔。1124年,耶律大石称遼王建號延慶,到达可敦城(今蒙古国布尔干省青托罗盖古回鹘城)建立根据地。1132年,在叶密立(今新疆维吾尔自治区额敏县)称“菊儿汗”,西辽帝国正式建立。随后耶律大石向新疆、蒙古高原、中亚及西亚地区扩张,建都于虎思斡鲁朵(今吉尔吉斯斯坦托克玛克东南布拉纳)。在1141年的卡特万之战,击败塞尔柱帝国联军,成为中亚霸主,将威名远播至欧洲。高昌回鹘、西喀喇汗国、东喀喇汗国及花剌子模先后臣服于强盛期的西辽。1143年,耶律大石去世。

耶律大石在军事、政治和外交上都有成就,欧洲得知其西征的事迹,流传着祭司王约翰的传说。耶律大石的名字也成为西辽帝国的代称,在耶律大石去世后多年,很多国家仍用“大石”称呼西辽的后代君主。

耶律大石是辽朝开国君主耶律阿保机的八世孙,精通契丹语和汉语,擅长弓马骑射。1115年,耶律大石中进士入翰林,初为翰林应奉,不久累迁翰林承旨。根据辽朝的科举制度,殿试头名才有入翰林应奉的资格。因契丹语称翰林为林牙,耶律大石又被称为大石林牙或林牙大石。后历任泰州、祥州刺史,辽兴军节度使。

历经200多年统治的辽朝国力逐渐走向衰弱,取而代之的是女真族建立的金朝。在金军势如破竹的攻击下,辽朝节节败退。1122年,金军攻克辽中京大定府和泽州,辽天祚帝如惊弓之鸟,从居庸关至鸳鸯泺(今河北省张北县安固里淖)到白水泺(今内蒙古自治区乌兰察布市察右前旗黄旗海),再到女古底仓,一路仓皇逃跑至夹山(今内蒙古自治区武川县附近)。数日后,宰相李处温与南京(即燕京,今北京市西南)都统萧干、耶律大石等拥立秦晋国王耶律淳为帝,建立北辽。耶律大石被视为肱骨之臣,官至太师。

1120年,一心想收复燕云十六州的北宋与金朝缔结了海上之盟,约定南北夹击辽朝。1122年5月,宋徽宗得知金朝大举进攻的消息后,任命童贯为宣抚使,蔡攸为副使,率军15万巡边,伺机收复燕云十六州。耶律淳委派耶律大石为西南路都统,牛栏监军萧遏鲁为副将,率领奚、契丹骑兵2000,驻扎于涿州新城县(今河北省高碑店市)防备。

宋军裨将杨可世听闻燕地百姓早有归宋之心,如果宋军到达,燕人必定箪食壶浆迎接,便率轻骑数千奇袭燕京,但7月1日在兰沟甸遭到耶律大石军的掩杀,大败而归。耶律淳得知消息后,又增兵3万。耶律大石率军渡过白沟河,4日与宋军东路统制种师道隔河对峙。战前,杨可世派赵明持黄榜旗前往耶律大石的营帐劝降,耶律大石毁旗怒骂:“无多言,有死而已。”话语未完,辽军矢石如雨。耶律大石指挥骑兵从西部浅滩处渡河,分左右两翼包抄宋军,宋军大败,杨可世中铁蒺藜负伤。次日,驻扎于范村(今河北省涿州市西南)的宋军西路统制辛兴宗的部队也遭到四军大王萧干的围攻。

7月8日,种师道下令撤兵,耶律大石得知消息后,率轻骑追击至古城,双方交战,宋军大乱,种师道几乎不能脱逃。宋军一路逃奔至雄州,辽军一路跟随,童贯禁止宋军入城,契丹人斥责北宋背弃澶渊之盟,挑起战争。正逢此日北风大雨冰雹交加,宋军一败再败,阵亡者不计其数,种师道也因燕京之战的失利遭到童贯的弹劾,责令致仕。

7月11日,耶律大石在涿州召见北宋使者马扩,责问他辽朝与北宋通好百年,现今北宋为何率军前来抢夺辽朝的领土。马扩以“宋不取怕金来取”作答辩。耶律大石斥责马扩,说西夏屡次派使者唆使辽朝进攻北宋,但辽朝不肯见利忘义,将表章封存后交给北宋,如今北宋只听信了女真人的一句话,便于辽朝兵戈相见。耶律大石又质问马扩既为使者,为何与叛将刘宗吉有联系,并让他转告童贯,如果两国想和好仍可交好,如果不愿和好便可提兵来战,不要在天热时打仗使士兵受苦。

1122年7月29日,耶律淳病死,其妻萧德妃临朝称制。宰相李处温南通童贯,想纳土降宋,北联络金朝作为内应,事发后被处死。李处温死后,北辽的军政事务由太师耶律大石和四军大王萧干掌控。

北宋得知耶律淳去世的消息后,在太宰王黼的倡议下,再次兴兵攻打北辽。8月29日,宋徽宗下诏集结各道兵20万,以刘延庆为都统制,于10月在三关(草桥关、益津关、瓦桥关)汇合。10月25日,北辽都管押常胜军、涿州留守郭药师叛降北宋。11月19日,刘延庆、何灌、郭药师等率军从雄州出发,进入新城县;刘光世、杨可世从安肃州(今河北省徐水县安肃镇)出发,进入易州,两军于涿州汇合,共50万。耶律大石和萧干统帅的北辽军不足2万人,在泸沟河部署。宋辽两军隔河对峙,双方曾战于料石冈,但未分胜负。11月24日,郭药师率军6000奇袭燕京,入外城。契丹守军拼力死战,而宋军毫无军纪,饮酒后到处奸淫掳掠。萧德妃秘遣使者召耶律大石、萧干军,昼夜疾行,自南暗门入城,宋军大败,仅百余骑得以逃脱。29日,泸沟河北面四处火起,宋军以为辽军将至,烧营落荒而逃。逃兵自相践踏,坠落山涧者不计其数,丢弃的军需物资绵延数百里。

北辽刚刚击退南方的宋军,北方的金军又再次逼近。萧德妃曾五次上表金朝,请求立秦王耶律定为帝,称臣求和,金太祖不许。萧德妃只好派精兵防守居庸关,但金兵到来时,居庸关城墙倒塌,士兵多被压死,其余守军不战而溃。萧德妃闻讯后连夜逃离燕京,声称御敌,实为出奔。萧德妃、耶律大石、萧干等经古北口(今北京市密云县古北口镇),向东逃至松亭关(今河北省宽城满族自治县西南),但因去往何处,发生争执。萧干主张去奚王府立国,而耶律大石则主张投奔天祚帝。驸马都尉萧勃迭反对耶律大石的意见,被耶律大石下令斩首。耶律大石又传令军中,有异议者斩。于是北辽军兵分两路,萧干率领奚、渤海军前往奚王府,耶律大石挟持萧德妃去夹山投奔天祚帝。萧干到达奚王府后,自立为帝,国号大奚,半年后败亡。耶律大石与萧德妃率军7000,于1123年3月至夹山。天祚帝因耶律淳被立之事杀萧德妃及外甥耶律常哥。天祚帝又质问耶律大石为何擅立耶律淳,耶律大石指出天祚帝以辽朝全国国力不能抵御金朝的进攻,弃国而逃,致使生灵涂炭。耶律淳为辽太祖子孙,立其为帝保社稷远胜于投降金朝。在耶律大石的辩解下,天祚帝下令赦免其余众人。

耶律大石在辽天祚帝帐下任都统一职,1123年,率辽军进攻奉圣州,驻军于龙门山东二十五里处。金朝都统完颜斡鲁派完颜照立、完颜娄室、马和尚等率军攻打,耶律大石战败被完颜娄室俘虏,所部投降。完颜宗望用绳子绑着耶律大石,强迫他作为向导,率军袭击了天祚帝位于青冢泺(今内蒙古自治区呼和浩特市南)的大营,俘获了天祚帝之子秦王耶律定、许王耶律寧和嫔妃、公主、从臣多人,获取辎重车万余辆,只有梁王耶律雅里和天祚帝长女趁乱逃出。耶律大石因作为向导有功,免其罪并特受金太祖降诏奖谕。金太祖还十分欣赏耶律大石的仪表俊美,为人聪辩,特赐予其妻子。同年9月,耶律大石跟随金朝西征,带领家眷自金营逃出,率领一支部队投奔天祚帝。关于耶律大石在金营中的生活,《契丹国志》记载耶律大石投降金朝后与粘罕不和,粘罕想杀掉耶律大石,耶律大石带着五个儿子夜间逃脱,但把妻子留在金营中。粘罕将耶律大石的妻子赐给部落中地位最低贱的人,但他的妻子坚贞不屈,最后被粘罕射杀,但此段资料真实性待考。

1124年,在得到耶律大石的部队和阴山室韦首领毛割石的援助后,辽天祚帝认为反攻的时机已经来临,决定亲自出兵收复燕州、云州地区。耶律大石认为金军气盛,应当养精蓄锐,不能贸然出击,天祚帝不听,坚持出兵。耶律大石知道天祚帝无法完成复兴辽朝的大业,又害怕得到天祚帝的猜忌,于是杀掉萧乙薛和坡里括后自立为王,率领铁骑200出奔。耶律大石走后,辽天祚帝虽然取得一些战役的胜利,但不久便被金朝所败。1125年,辽天祚帝在投奔西夏的途中被俘,辽朝灭亡。

耶律大石率军从夹山出发,北行三日渡过黑水(爱毕哈河),途中遇到白鞑靼人首领床古儿,床古儿给予耶律大石四百匹马,二十头骆驼,若干只羊的援助。耶律大石一路向西北,于1124年到达可敦城,召集威武、崇德、会蕃、新、大林、紫河、驼等七个军州的长官和大黄室韦、敌剌、王纪剌、茶赤剌、也喜、鼻古德、尼剌、达剌乖、达密里、密儿纪、合主、乌古里、阻蔔、普速完、唐古、忽母思、奚的、纠而毕十八个部族的首领举行大会。在大会上,耶律大石慷慨激昂地指出先祖创建辽朝的艰难以及由于金朝对于辽朝侵略,造成天祚帝流亡在外、生灵涂炭,号召各军州和部族驱逐仇敌,复兴大辽。由于可敦城是辽朝的西北边防重镇,边防军队不得随意征调,军队在战乱中得以保存,并且此地还拥有可骑乘的战马数十万匹。耶律大石安置官吏,整顿兵马,磨砺武器,得到精兵万余人。

耶律大石在可敦城建立根据地后,积攒实力,不断派使者联络白鞑靼人、西夏以及北宋,从外交上孤立金朝。1125年夏,西夏联络耶律大石攻取金朝的山西诸郡。同年末,耶律大石派使者联络北宋,提议合力攻打金朝。1127年,白鞑靼人与耶律大石通好,拒绝将马匹卖给金朝。金太宗派使者问罪,双方关系紧张。1129年,耶律大石率军攻取了金朝的北方二营。次年,金太宗派耶律余睹、石家奴、拔离速征讨耶律大石,但由于诸部落不同意出兵,大军行进至兀纳水后收兵。

经过休整,耶律大石的军事实力得到壮大。1130年3月,耶律大石以青牛、白马祭告天地、列祖,准备西征。耶律大石先派使者送信给高昌回鹘首领毕勒哥,阐明两国先代的友好并要求借道去大食。毕勒哥得到书信后,迎接耶律大石至宫邸大宴三日,临行前毕勒哥亲自护送耶律大石出境,赠送耶律大石马匹六百、骆驼数百、羊三千只作为礼物,并约定交出人质,作为耶律大石的附庸国。

耶律大石率军离开高昌回鹘,进入吉尔吉斯境内,遭到了当地的抵抗,但双方未发生大规模的战争。耶律大石率军继续西进,到达叶密立。大军所到之处望风披靡,获取骆驼、牛、马、羊等辎重无数。1131年春,金朝统帅粘罕及耶律余睹率领云中、燕、云州汉军、金军1万人攻打耶律大石的根据地可敦城,但遭到失败。耶律大石到达叶密立后,虽然与高昌回鹘发生过摩擦,但基本得到了当地突厥部族的支持,户数达到4万。1132年,耶律大石在新建成的叶密立正式称“菊儿汗”,群臣又尊汉号为“天祐皇帝”,建元延庆,追尊祖父为元皇帝,祖母为宣义皇后,册封元妃萧氏为昭德皇后,西辽帝国正式建立。

西辽帝国建立后,耶律大石开始酝酿向周边地区扩张。1132年,耶律大石亲率大军向南进发,高昌回鹘再次臣服于西辽。随后耶律大石率军越过天山,沿塔里木盆地北向西推进,与东喀喇汗国发生冲突。西辽军被东喀喇汗国阿尔斯兰汗阿赫马德·伊本·哈桑的军队击败,大将阿勒·阿瓦尔被俘,损失惨重。耶律大石撤军后向七河地区进发,收编了当地的契丹人和突厥人,共16000帐,使西辽军队的人数增加了一倍。耶律大石率军驻扎于西辽与东喀喇汗国边境地区,等待时机准备反攻。

1132年,阿赫马德·伊本·哈桑去世,其子伊卜拉欣二世继任。伊卜拉欣二世软弱无能,原本臣属于东喀喇汗国的葛逻禄和康里人趁机袭击他的部属和牲畜,进行劫掠。伊卜拉欣二世不能控制住国内的局势,于是派使者请求耶律大石进入八剌沙衮(今吉尔吉斯斯坦托克馬克東)接管他的国家,使他“摆脱这尘世的烦恼”。耶律大石接到请求后,率军进入东喀喇汗国首都八剌沙衮,“登上那不费分文的宝座”。耶律大石将伊卜拉欣二世降为伊列克·突厥蛮(意为突厥王),保留了他对喀什噶尔(今新疆维吾尔自治区喀什市)、和田地区的控制,东喀喇汗国成为西辽的附庸。由于八剌沙衮附近是可耕可牧的肥沃地区,耶律大石决定建都于此,将八剌沙衮改名为虎思斡耳朵(意为强而有力的宫帐),并改元康国。耶律大石随后又派军队战胜了吉尔吉斯人,征服了别失八里(今新疆维吾尔自治区吉木萨尔县境内),康里人不久也臣服于西辽。

1134年4月,耶律大石任命六院司大王萧斡里剌为兵马都元帅,敌剌部前同知枢密院事萧查剌阿不为副元帅,茶赤剌部秃鲁耶律燕山为都部署,护卫耶律铁哥为都监,率军7万征讨金朝。在战前的誓师大会上,耶律大石用白马青牛祭天,指出先祖创业艰难,是由于后代君主耽于享乐致使社稷倾覆。中亚并非久居之地,应当荣归故里,复兴大辽。他又劝谕萧斡里剌要与士卒同甘共苦,赏罚分明。作战时要选择水草丰富处扎营,谨慎用兵。但由于西辽与金朝两国相隔遥远,西辽军队行进万里一无所获,兵马损失惨重,不得不撤军回国。另据《三朝北盟会编》记载,1135年,耶律大石再次率军攻打金朝,金熙宗派粘罕迎战。金军进入沙漠后与西辽军征战三昼夜不分胜败,但金军粮草断绝,人马冻死很多,加上本为契丹人的副将临阵倒戈,致使粘罕大败而归。但此段史料的真实性待考。

自1137年起,耶律大石开始了第二次扩张。1137年,耶律大石率军向察赤(今乌兹别克斯坦塔什干)、费尔干纳盆地及泽拉夫尚河流域进兵。同年5至6月,在忽毡(今塔吉克斯坦苦盏)遭到了西喀喇汗国可汗马赫穆德·伊本·穆海默德的抵抗。西喀喇汗国战败,马赫穆德败逃回撒马尔罕。这次战败使马黑木二世的臣民感到震惊、惊恐和沮丧,但耶律大石并没有继续进兵。1141年,西喀喇汗国与葛逻禄人爆发冲突,马赫穆德向宗主国塞尔柱帝国求援。塞尔柱苏丹桑贾尔动员伊斯兰诸国参战,集中了呼罗珊、锡斯坦、伽色尼、马赞德兰、古尔等国的军队近10万人,单单阅兵就耗费了半年时间。同年7月,桑贾尔率军渡过阿姆河,进入河中地区,葛逻禄人急忙派使者向耶律大石求救。

耶律大石写信给桑贾尔替葛逻禄人说情,但桑贾尔十分傲慢的回信命令耶律大石加入伊斯兰教,并称自己的军队能用箭截断敌人的须发。当耶律大石听完桑贾尔的使者读完书信后,下令拔下他的一撮胡须,然后给他一根针让他当场示范,使者不能做到。耶律大石说既然针不能截断胡须,那那个人又怎么能用箭折断须发呢?于是下令进兵,双方在撒马尔罕以北的卡特万草原对峙,西辽的军队中有契丹人、突厥人、汉人和蒙古人。耶律大石观察了战场的地形后,让军队背靠达尔加姆峡谷安营。两军于1141年9月9日展开会战,战前耶律大石指出桑贾尔的联军人多少谋,如果全力进攻,他们就会首尾不顾。耶律大石派六院司大王萧斡里剌、招讨副使耶律松山等率兵2500攻打联军右翼,枢密副使萧剌阿不、招讨使耶律术薛等率兵2500攻打其左翼,耶律大石亲率部队攻打中军;桑贾尔的联军右翼是埃米尔库马吉,左翼是锡斯坦埃米尔胡马希,他自己亲率中军,有战斗经验的老兵负责殿后。

在战场上,锡斯坦贵族作战英勇,但西辽军队中的葛逻禄人发挥了重要的作用,迫使桑贾尔的联军败逃。桑贾尔和马赫穆德逃奔至泰尔梅兹,桑贾尔的妻子、左、右翼统帅和伊斯兰法学家胡萨姆·奥玛尔·伊本·阿布杜·阿齐兹·伊本·马扎·布哈里均被俘虏。桑贾尔的联军损失惨重,仅达尔加姆峡谷就装下1万名死者。《辽史》记载塞尔柱帝国联军的阵亡者横尸数十里。卡特万之战后,塞尔柱帝国的势力退出河中地区,西辽成为中亚霸主。耶律大石随后率军进入撒马尔罕,立马赫穆德之弟伊卜拉欣·伊本·穆海默德为桃花石汗,继续让其统治西喀喇汗国。 他还下令处死布哈拉的伊斯兰教教长胡沙穆丁·倭玛尔,任命阿尔普·的斤统治该地。随后派大将额儿布思(一说即萧斡里剌)出兵花剌子模,在该地烧杀抢掠,迫使花剌子模沙阿阿拉丁·阿比兹向西辽臣服并且每年缴纳价值3万金第纳尔的货物和牲畜。耶律大石在撒马尔罕驻扎90天后,至起儿漫(今乌兹别克斯坦卡尼梅赫镇)巡行后班师返回虎思斡耳朵。

1143年,耶律大石去世,在位20年,庙号德宗。因耶律大石之子耶律夷列年幼,遗诏命皇后萧塔不烟临朝称制,改元咸清,称感天皇后。

耶律大石的西征事迹被传到欧洲,正逢第二次十字军东征,于是在欧洲流传着东方世界有一位神秘的祭司王约翰,是基督教的捍卫者。俄语、阿拉伯语、拉丁语和古英语中中国的发音类似于“契丹”,都是受耶律大石西征的影响。而耶律大石的名字也成了西辽帝国的代称,在耶律大石死后,金、西夏、南宋等国家对西辽的后代君主皆称为“大石”。

耶律大石凭借卓越的军事、政治、外交才能,在伊斯兰世界建立了幅员辽阔的西辽帝国,将辽朝的国祚延续了近百年,他为东西方文化、经济方面的交流作出了积极的贡献。东西方史学家对于耶律大石的成就多有赞誉:穆斯林史学家朱兹贾尼评价耶律大石:是一位公正的君主,因为公正和才干而受到崇敬;耶律楚材评价耶律大石:颇尚文教,西域人至今思之。拉施特称赞耶律大石:是一个有智慧而又有才干的人。他有条不紊地从这些地区上把队伍召集到身边,占领了整个突厥斯坦地区,(从而)获得了古儿汗,即伟大的君主的称号。清代史学家谭宗浚评价耶律大石:德宗起自词臣,兼通藩俗,削平各部,殄定诸藩,意其典章制度必可多采。

\subsubsection{延庆}


\begin{longtable}{|>{\centering\scriptsize}m{2em}|>{\centering\scriptsize}m{1.3em}|>{\centering}m{8.8em}|}
  % \caption{秦王政}\
  \toprule
  \SimHei \normalsize 年数 & \SimHei \scriptsize 公元 & \SimHei 大事件 \tabularnewline
  % \midrule
  \endfirsthead
  \toprule
  \SimHei \normalsize 年数 & \SimHei \scriptsize 公元 & \SimHei 大事件 \tabularnewline
  \midrule
  \endhead
  \midrule
  元年 & 1124 & \tabularnewline\hline
  二年 & 1125 & \tabularnewline\hline
  三年 & 1126 & \tabularnewline\hline
  四年 & 1127 & \tabularnewline\hline
  五年 & 1128 & \tabularnewline\hline
  六年 & 1129 & \tabularnewline\hline
  七年 & 1130 & \tabularnewline\hline
  八年 & 1131 & \tabularnewline\hline
  九年 & 1132 & \tabularnewline\hline
  十年 & 1133 & \tabularnewline
  \bottomrule
\end{longtable}

\subsubsection{康国}

\begin{longtable}{|>{\centering\scriptsize}m{2em}|>{\centering\scriptsize}m{1.3em}|>{\centering}m{8.8em}|}
  % \caption{秦王政}\
  \toprule
  \SimHei \normalsize 年数 & \SimHei \scriptsize 公元 & \SimHei 大事件 \tabularnewline
  % \midrule
  \endfirsthead
  \toprule
  \SimHei \normalsize 年数 & \SimHei \scriptsize 公元 & \SimHei 大事件 \tabularnewline
  \midrule
  \endhead
  \midrule
  元年 & 1134 & \tabularnewline\hline
  二年 & 1135 & \tabularnewline\hline
  三年 & 1136 & \tabularnewline\hline
  四年 & 1137 & \tabularnewline\hline
  五年 & 1138 & \tabularnewline\hline
  六年 & 1139 & \tabularnewline\hline
  七年 & 1140 & \tabularnewline\hline
  八年 & 1141 & \tabularnewline\hline
  九年 & 1142 & \tabularnewline\hline
  十年 & 1143 & \tabularnewline
  \bottomrule
\end{longtable}


\subsection{萧塔不烟\tiny(1143-1150)}

\subsubsection{生平}

萧塔不烟,生卒年不详,西辽开国君主遼德宗的皇后,德宗死後稱制,執政7年。

1143年,耶律大石去世后,其子耶律夷列年幼,遗诏命皇后萧塔不烟临朝称制,改元咸清,称感天皇后。

1144年,金熙宗得知耶律大石去世的消息後,派使者粘割韩奴前往劝降西辽。粘割韩奴進入西遼國境後,正好遇上外出打獵的萧塔不烟。見到萧塔不烟後,粘割韩奴不但沒有下马跪拜,反而讓她下马接诏。萧塔不烟於是命人将粘割韩奴拉下马,讓他跪下。粘割韩奴痛骂不止,斥責其為反賊,侮辱上国使臣。萧塔不烟发怒,派人将其杀死。

執政七年後,萧塔不烟退位。耶律夷列親政,改年號為紹興。

一说萧塔不烟与耶律大石在叶密立(今新疆维吾尔自治区额敏县)称菊儿汗时册封的昭德皇后萧氏为同一人;也有观点认为二者并非同一人。

\subsubsection{咸清}

\begin{longtable}{|>{\centering\scriptsize}m{2em}|>{\centering\scriptsize}m{1.3em}|>{\centering}m{8.8em}|}
  % \caption{秦王政}\
  \toprule
  \SimHei \normalsize 年数 & \SimHei \scriptsize 公元 & \SimHei 大事件 \tabularnewline
  % \midrule
  \endfirsthead
  \toprule
  \SimHei \normalsize 年数 & \SimHei \scriptsize 公元 & \SimHei 大事件 \tabularnewline
  \midrule
  \endhead
  \midrule
  元年 & 1144 & \tabularnewline\hline
  二年 & 1145 & \tabularnewline\hline
  三年 & 1146 & \tabularnewline\hline
  四年 & 1147 & \tabularnewline\hline
  五年 & 1148 & \tabularnewline\hline
  六年 & 1149 & \tabularnewline\hline
  七年 & 1150 & \tabularnewline
  \bottomrule
\end{longtable}

\subsection{仁宗耶律夷列\tiny(1150-1163)}

\subsubsection{生平}

辽仁宗耶律夷列(?-1163年),耶律大石和萧塔不烟之子,耶律普速完之兄,西遼第二任君主,在位13年。

耶律大石去世后,其子耶律夷列年幼,遗诏命皇后萧塔不烟临朝称制,改元咸清,称感天皇后。萧塔不烟在位7年后,还政于子耶律夷列。

1150年,耶律夷列即位,改元绍兴。耶律夷列在位期间普查首都虎思斡耳朵内畿18岁以上成年男子的人口,共84500户[a]。1156年,西喀喇汗国大汗伊卜拉欣三世与葛逻禄军队长官艾亚尔伯克发生冲突,双方在饥饿草原发生战争,伊卜拉欣三世战败被暴尸荒野,其子阿里·本·哈桑继任,称恰格雷汗。恰格雷汗随后对葛逻禄人展开报复,杀死其首领比古汗。葛逻禄的拉钦伯克和比古汗之子向花剌子模求助,而恰格雷汗则向西辽求援。耶律夷列派东喀喇汗国土库曼王伊卜拉欣·本·阿赫马德率军1万前去救援,双方隔粟特河对峙。经撒马尔罕的宗教人士调节,双方签订合约,恰格雷汗恢复了葛逻禄首领的军事职务,双方撤军(後來西遼把葛邏祿人安置在阿力麻里單獨管理)。

耶律夷列在位13年,于1163年去世,庙号仁宗。由于其子年幼,遗诏命其妹耶律普速完临朝称制,改元崇福,称承天太后。

\subsubsection{绍兴}


\begin{longtable}{|>{\centering\scriptsize}m{2em}|>{\centering\scriptsize}m{1.3em}|>{\centering}m{8.8em}|}
  % \caption{秦王政}\
  \toprule
  \SimHei \normalsize 年数 & \SimHei \scriptsize 公元 & \SimHei 大事件 \tabularnewline
  % \midrule
  \endfirsthead
  \toprule
  \SimHei \normalsize 年数 & \SimHei \scriptsize 公元 & \SimHei 大事件 \tabularnewline
  \midrule
  \endhead
  \midrule
  元年 & 1151 & \tabularnewline\hline
  二年 & 1152 & \tabularnewline\hline
  三年 & 1153 & \tabularnewline\hline
  四年 & 1154 & \tabularnewline\hline
  五年 & 1155 & \tabularnewline\hline
  六年 & 1156 & \tabularnewline\hline
  七年 & 1157 & \tabularnewline\hline
  八年 & 1158 & \tabularnewline\hline
  九年 & 1159 & \tabularnewline\hline
  十年 & 1160 & \tabularnewline\hline
  十一年 & 1161 & \tabularnewline\hline
  十二年 & 1162 & \tabularnewline\hline
  十三年 & 1163 & \tabularnewline
  \bottomrule
\end{longtable}

\subsection{承天太后耶律普速完\tiny(1163-1177)}

\subsubsection{生平}

耶律普速完(?-1177年)是遼仁宗耶律夷列的妹妹,為西遼第四任統治者。仁宗在1163年死後,其子尚年幼,遺詔由其妹耶律普速完權理國事,臨朝稱制,並改元崇福,號承天太后。

她的丈夫是蕭朵魯不。她與丈夫之弟樸古只沙里私通,把丈夫改為東平王,後來又殺了他。崇福十四年(1177年),蕭朵魯不之父斡里剌以兵圍其宮,射殺普速完及樸古只沙里。仁宗子耶律直魯古即位,改元天禧,是為西遼末主。

中國歷朝的臨朝稱制,皆為皇太后、皇后在君主因故無法理朝時的一種權宜之計(也有例外,如武則天曾與唐高宗並稱二聖並臨朝聽政),但耶律普速完是中國歷史上唯一一個以先朝公主與當朝君主姑母的身分臨朝稱制者,可為前無古人、後無來者。耶律普速完稱制時,又更改年號,在實質意義上已經得到等同君王的待遇和地位。

\subsubsection{崇福}

\begin{longtable}{|>{\centering\scriptsize}m{2em}|>{\centering\scriptsize}m{1.3em}|>{\centering}m{8.8em}|}
  % \caption{秦王政}\
  \toprule
  \SimHei \normalsize 年数 & \SimHei \scriptsize 公元 & \SimHei 大事件 \tabularnewline
  % \midrule
  \endfirsthead
  \toprule
  \SimHei \normalsize 年数 & \SimHei \scriptsize 公元 & \SimHei 大事件 \tabularnewline
  \midrule
  \endhead
  \midrule
  元年 & 1164 & \tabularnewline\hline
  二年 & 1165 & \tabularnewline\hline
  三年 & 1166 & \tabularnewline\hline
  四年 & 1167 & \tabularnewline\hline
  五年 & 1168 & \tabularnewline\hline
  六年 & 1169 & \tabularnewline\hline
  七年 & 1170 & \tabularnewline\hline
  八年 & 1171 & \tabularnewline\hline
  九年 & 1172 & \tabularnewline\hline
  十年 & 1173 & \tabularnewline\hline
  十一年 & 1174 & \tabularnewline\hline
  十二年 & 1175 & \tabularnewline\hline
  十三年 & 1176 & \tabularnewline\hline
  十四年 & 1177 & \tabularnewline
  \bottomrule
\end{longtable}

\subsection{天禧帝耶律直鲁古\tiny(1177-1211)}

\subsubsection{生平}

耶律直魯古(12世纪-1213年),是西遼皇帝耶律夷列的次子。姑姑耶律普速完在崇福十四年(1177年)被殺,耶律直魯古即位,改元天禧,史称天禧帝。

乃蠻王子屈出律于1208年流亡至西辽,天禧帝耶律直鲁古不仅信任他还将女儿嫁给他。天禧三十四年(1211年),屈出律以伏兵八千擒直魯古,強迫天禧帝直鲁古讓位,尊他為太上皇,皇后為皇太后。1213年,天禧帝直魯古去世。1218年,蒙古攻西遼,殺屈出律,西遼亡。

遼史(卷三十 本紀第三十):“仁宗次子直魯古即位,改元天禧,在位三十四年,天禧帝。時秋出獵,乃蠻王屈出律以伏兵八千擒之,而據其位。遂襲遼衣冠,尊直魯古為太上皇,皇后為皇太后,朝夕問起居,以侍終焉。直魯古死,遼絕。”

\subsubsection{天禧}

\begin{longtable}{|>{\centering\scriptsize}m{2em}|>{\centering\scriptsize}m{1.3em}|>{\centering}m{8.8em}|}
  % \caption{秦王政}\
  \toprule
  \SimHei \normalsize 年数 & \SimHei \scriptsize 公元 & \SimHei 大事件 \tabularnewline
  % \midrule
  \endfirsthead
  \toprule
  \SimHei \normalsize 年数 & \SimHei \scriptsize 公元 & \SimHei 大事件 \tabularnewline
  \midrule
  \endhead
  \midrule
  元年 & 1178 & \tabularnewline\hline
  二年 & 1179 & \tabularnewline\hline
  三年 & 1180 & \tabularnewline\hline
  四年 & 1181 & \tabularnewline\hline
  五年 & 1182 & \tabularnewline\hline
  六年 & 1183 & \tabularnewline\hline
  七年 & 1184 & \tabularnewline\hline
  八年 & 1185 & \tabularnewline\hline
  九年 & 1186 & \tabularnewline\hline
  十年 & 1187 & \tabularnewline\hline
  十一年 & 1188 & \tabularnewline\hline
  十二年 & 1189 & \tabularnewline\hline
  十三年 & 1190 & \tabularnewline\hline
  十四年 & 1191 & \tabularnewline\hline
  十五年 & 1192 & \tabularnewline\hline
  十六年 & 1193 & \tabularnewline\hline
  十七年 & 1194 & \tabularnewline\hline
  十八年 & 1195 & \tabularnewline\hline
  十九年 & 1196 & \tabularnewline\hline
  二十年 & 1197 & \tabularnewline\hline
  二一年 & 1198 & \tabularnewline\hline
  二二年 & 1199 & \tabularnewline\hline
  二三年 & 1200 & \tabularnewline\hline
  二四年 & 1201 & \tabularnewline\hline
  二五年 & 1202 & \tabularnewline\hline
  二六年 & 1203 & \tabularnewline\hline
  二七年 & 1204 & \tabularnewline\hline
  二八年 & 1205 & \tabularnewline\hline
  二九年 & 1206 & \tabularnewline\hline
  三十年 & 1207 & \tabularnewline\hline
  三一年 & 1208 & \tabularnewline\hline
  三二年 & 1209 & \tabularnewline\hline
  三三年 & 1210 & \tabularnewline\hline
  三四年 & 1211 & \tabularnewline
  \bottomrule
\end{longtable}

\subsection{末帝屈出律\tiny{1211-1218}}

\subsubsection{生平}

屈出律是乃蠻太陽汗之子,1204年,成吉思汗攻滅乃蠻部,太陽汗戰死。屈出律投奔其叔父不亦鲁黑汗。不亦鲁黑汗死後,屈出律又聯合蔑儿乞首领脱黑脱阿對抗成吉思汗。1208年,屈出律和脱黑脱阿在也儿的石河(今额尔齐斯河)上游被成吉思汗击败,脱黑脱阿戰死。屈出律逃奔至别失八里(今新疆维吾尔自治区吉木萨尔县境内),又抵达苦叉(今新疆维吾尔自治区库车县),他的部队缺乏給養又没有粮食,一路上纷纷散去。屈出律隨後投奔西遼,侍奉於菊儿汗耶律直鲁古。《史集》记载屈出律觐见耶律直鲁古时化妆成一名马夫,这身装扮触怒了耶律直鲁古的大臣,但得到了耶律直鲁古正妻菊儿别速的女儿渾忽公主的賞識,三天後渾忽公主便嫁给了屈出律。

随着西遼国力的衰落,附庸国高昌回鹘、花剌子模和西喀喇汗国纷纷背叛西遼,屈出律便向耶律直鲁古建议自己返回叶密立(今新疆维吾尔自治区额敏县)、海押立(今哈萨克斯坦塔尔迪库尔干)、别失八里地区召集乃蠻舊部,帮助耶律直鲁古鎮壓叛乱。耶律直鲁古封屈出律为可汗,并赠送他很多禮物。屈出律收集失散的乃蠻人,组成军隊,劫掠七河地区。同时派使者聯絡花剌子模沙阿阿拉乌丁·摩诃末,雙方约定誰先奪取西遼就占有它的土地。屈出律先擊败了西遼的军隊,劫掠了耶律直鲁古位于乌兹根的府库,隨後又進攻西遼首都虎思斡耳朵(今吉尔吉斯斯坦托克玛克东南布拉纳),但在真兀赤被耶律直鲁古擊败。屈出律返回葉密立,圖謀再次進攻。

1210年,西遼國内發生西遼軍隊燒殺劫掠首都虎思斡耳朵的事件,由于宰相马合木·太处理不當,致使軍隊纷纷離散。屈出律得知此消息后,于次年秋率軍8千“像雲中的閃電一樣”襲擊了正在外出打獵的耶律直鲁古,奪取了皇位。屈出律惺惺作態,尊耶律直鲁古为太上皇,皇后为皇太后,早晚问候他们的衣食起居。1213年,耶律直鲁古在憤恨中死去。

东喀喇汗国土库曼王穆罕默德三世起兵反抗西辽的统治,遭到耶律直鲁古的镇压,穆罕默德三世被俘。屈出律篡位后,释放了穆罕默德三世,将其送回喀什噶尔,但他不受当地贵族的欢迎,入城时被刺死于城门洞中。由于喀什噶尔不肯归附屈出律,屈出律每逢秋收时节派兵烧毁他们的庄稼。三、四年后,当地百姓因为饥荒不得已而归顺。在占领喀什噶尔后,屈出律下令在每家每户派驻一名士兵,这些士兵毫无军纪,到处烧杀抢掠。屈出律随后又派兵征服了和田。阿力麻里(今新疆维吾尔自治区霍城县一带)汗不札兒不肯服从屈出律,屈出律多次派军队征讨无果,最终趁不札兒出獵時將其擒殺。

屈出律篡位后虽然没有更改西辽的国号和政治制度,但他一改前任西辽君主的宗教自由政策,转而实行宗教迫害。屈出律原本信奉景教,后在浑忽公主的劝说下改信佛教。他用强制手段强迫西辽当地的穆斯林和基督徒改信佛教,穿戴契丹人的服装,这引起了当地人民的强烈不满。屈出律征服和田后,下令召集当地的伊斯兰教阿訇讨论教义,教长阿訇阿剌丁·摩诃末极力维护伊斯兰教,屈出律命人将其严刑拷打,强迫他改教,摩诃末不从,被屈出律下令钉死于清真寺的大门上。

1218年,成吉思汗派哲别、曷思麦里率2萬蒙古軍攻打屈出律,屈出律闻讯带领随从从喀什噶尔逃跑。哲别进入喀什噶尔后宣布宗教自由,城中居民開始對屈出律展开报复,大肆屠杀屈出律的军队。屈出律逃至巴达克山(今阿富汗巴达赫尚省),在瓦罕河谷东部的达拉兹峡谷迷路。由于当地山路崎岖难行,哲别向当地猎户许诺以屈出律随身携带的财物为条件,抓捕屈出律。屈出律被俘获后交予哲别,哲别将屈出律斩首后,命曷思麦里拿着他的首级传示于喀什噶尔、押儿牵(今新疆维吾尔自治区莎车县)、斡端(今新疆维吾尔自治区和田市)等城,城中将领皆率部投降,西辽彻底被蒙古帝国所征服。

\subsubsection{天禧}

\begin{longtable}{|>{\centering\scriptsize}m{2em}|>{\centering\scriptsize}m{1.3em}|>{\centering}m{8.8em}|}
  % \caption{秦王政}\
  \toprule
  \SimHei \normalsize 年数 & \SimHei \scriptsize 公元 & \SimHei 大事件 \tabularnewline
  % \midrule
  \endfirsthead
  \toprule
  \SimHei \normalsize 年数 & \SimHei \scriptsize 公元 & \SimHei 大事件 \tabularnewline
  \midrule
  \endhead
  \midrule
  三四年 & 1211 & \tabularnewline\hline
  三五年 & 1212 & \tabularnewline\hline
  三六年 & 1213 & \tabularnewline\hline
  三七年 & 1214 & \tabularnewline\hline
  三八年 & 1215 & \tabularnewline\hline
  三九年 & 1216 & \tabularnewline\hline
  四十年 & 1217 & \tabularnewline\hline
  四一年 & 1218 & \tabularnewline
  \bottomrule
\end{longtable}

%%% Local Variables:
%%% mode: latex
%%% TeX-engine: xetex
%%% TeX-master: "../Main"
%%% End:
