%% -*- coding: utf-8 -*-
%% Time-stamp: <Chen Wang: 2019-12-26 10:56:57>

\section{太祖\tiny(916-926)}

\subsection{生平}

辽太祖耶律阿保机(872年-926年9月6日),清輯本《旧五代史》改譯安巴堅,汉名耶律亿,是大契丹國的第一位皇帝(916年3月17日-926年9月6日在位),在位10年。

《辽史·后妃传》记载:“太祖慕汉高皇帝,故耶律氏兼称刘氏;以乙室、拔里比萧相国,遂为萧氏”。《辽史·国语解》记载:“耶律和萧两个姓,以汉字书者曰刘、萧,以契丹字书者曰移喇、石抹”。《金史·国语解》记载:“移喇曰刘,石抹曰萧”。

耶律阿保机的前辈是契丹迭剌部的酋长和军事首领(夷里堇),为耶律撒剌的的长子,母萧岩母斤。耶律是其氏族名。他本人于901年被立为军事首领(夷里堇兼任于越),后不久被选为酋长。他以武力征服契丹附近的地区,掠虏了许多汉人和其他人。907年2月27日他被选为部落联盟的首领,连任九年。任用汉人,采纳他们的建议,决定要将这种三年一次的选举制度改为世袭的制度。為了鞏固統治,史載遼太祖初元,韓廷徽助其正君臣,定名分。廢除三年一次的選汗制度造成諸弟之亂,後來叛亂被平定。

公元915年,耶律阿保机出征室韦得胜回国,但被迫交出汗位,但他在在滦河边建设了一座仿幽州式的汉城。耶律阿保机后伏杀了他的敌人,吞并了契丹的各个部落。916年3月17日,耶律阿保机登基称皇帝,立国号“契丹”,建立“大契丹国”(947年2月24日,辽太宗耶律德光改国号为“大辽”),建年号为神册。此外他还令人建立自己的契丹文。

耶律阿保机建国后继续进攻其周围的民族或政权,渤海国、室韦和奚分别被他消灭。926年9月6日去世于扶余城,终年55岁。

耶律阿保機將其母親、祖母、曾祖母、高祖母家族的姓氏拔里氏、乙室氏賜姓蕭氏。相傳是因為他本人羨慕蕭何輔助劉邦的典故。耶律阿保機的皇后名述律平,其子耶律德光即位後,亦將述律氏賜姓蕭氏。故蕭氏有遼朝后族之稱。阿保機汉名姓刘名亿,長子耶律突欲汉名劉倍。

\subsection{神册}


\begin{longtable}{|>{\centering\scriptsize}m{2em}|>{\centering\scriptsize}m{1.3em}|>{\centering}m{8.8em}|}
  % \caption{秦王政}\
  \toprule
  \SimHei \normalsize 年数 & \SimHei \scriptsize 公元 & \SimHei 大事件 \tabularnewline
  % \midrule
  \endfirsthead
  \toprule
  \SimHei \normalsize 年数 & \SimHei \scriptsize 公元 & \SimHei 大事件 \tabularnewline
  \midrule
  \endhead
  \midrule
  元年 & 916 & \tabularnewline\hline
  二年 & 917 & \tabularnewline\hline
  三年 & 918 & \tabularnewline\hline
  四年 & 919 & \tabularnewline\hline
  五年 & 920 & \tabularnewline\hline
  六年 & 921 & \tabularnewline\hline
  七年 & 922 & \tabularnewline
  \bottomrule
\end{longtable}

\subsection{天赞}

\begin{longtable}{|>{\centering\scriptsize}m{2em}|>{\centering\scriptsize}m{1.3em}|>{\centering}m{8.8em}|}
  % \caption{秦王政}\
  \toprule
  \SimHei \normalsize 年数 & \SimHei \scriptsize 公元 & \SimHei 大事件 \tabularnewline
  % \midrule
  \endfirsthead
  \toprule
  \SimHei \normalsize 年数 & \SimHei \scriptsize 公元 & \SimHei 大事件 \tabularnewline
  \midrule
  \endhead
  \midrule
  元年 & 922 & \tabularnewline\hline
  二年 & 923 & \tabularnewline\hline
  三年 & 924 & \tabularnewline\hline
  四年 & 925 & \tabularnewline\hline
  五年 & 926 & \tabularnewline
  \bottomrule
\end{longtable}

\subsection{天显}

\begin{longtable}{|>{\centering\scriptsize}m{2em}|>{\centering\scriptsize}m{1.3em}|>{\centering}m{8.8em}|}
  % \caption{秦王政}\
  \toprule
  \SimHei \normalsize 年数 & \SimHei \scriptsize 公元 & \SimHei 大事件 \tabularnewline
  % \midrule
  \endfirsthead
  \toprule
  \SimHei \normalsize 年数 & \SimHei \scriptsize 公元 & \SimHei 大事件 \tabularnewline
  \midrule
  \endhead
  \midrule
  元年 & 926 & \tabularnewline
  \bottomrule
\end{longtable}


%%% Local Variables:
%%% mode: latex
%%% TeX-engine: xetex
%%% TeX-master: "../Main"
%%% End:
