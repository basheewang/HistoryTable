%% -*- coding: utf-8 -*-
%% Time-stamp: <Chen Wang: 2019-12-26 10:57:29>

\section{太宗\tiny(927-947)}


\subsection{生平}

遼太宗耶律德光(902年11月25日-947年5月15日),大契丹國第二位皇帝(927年12月11日至947年5月15日在位),在位20年。字德谨,契丹名耶律尧骨,辽太祖耶律阿保机次子。947年2月24日,辽太宗耶律德光将国号由“大契丹国”改为“大辽”,成为遼朝首位皇帝。

辽太祖天赞元年(922年),被任命为天下兵马大元帅,随同太祖参加了一系列征服战争,尤其是在南征幽州、西征吐谷浑、回鹘期间,战功卓著。天显元年(926年),又随同太祖灭渤海国,作为前锋攻克渤海首都忽汗城。

天显元年七月二十七日(926年9月6日)辽太祖死后,述律后称制,耶律德光总揽朝政,927年12月11日,在述律后的支持下即位。天显六年(930年),割据原渤海国疆域的东丹王耶律倍南逃后唐,耶律德光统一了契丹。

天显十一年(936年),后唐河东节度使石敬瑭以称子、割让燕云十六州为条件,乞求耶律德光出兵助其反对后唐。耶律德光遂亲率5万骑兵,在晋阳城下击败后唐军,册立石敬塘为后晋皇帝。其后,更率军南下上党,助石敬塘灭后唐。

割取燕云十六州后,耶律德光采取“因俗而治”的统治方式,实行南北两面官制度,分治汉人和契丹。又改幽州为南京、云州为西京,将燕云十六州建设成为进一步南下的基地。

会同四年(944年),后晋出帝石重贵即位,拒不称臣。耶律德光于是率军南下。会同九年十二月十六日(947年1月10日),耶律德光率军攻入后晋首都东京汴梁(今河南开封),俘虏后晋出帝石重贵,灭后晋。

会同十年正月初一(947年1月25日),耶律德光以中原皇帝的仪仗进入东京汴梁,在崇元殿接受百官朝贺。大同元年二月初一(947年2月24日),耶律德光在东京皇宫下诏将国号“大契丹国”改为“大辽”,改会同十年为大同元年,升镇州为中京。

大同元年四月初一(947年4月24日),因遼人實施的「打草穀」物資掠奪政策導致中原反抗不断,无法巩固统治,耶律德光被迫离开东京汴梁,引军北返,在临城县(今河北省临城县)得熱疾。四月二十二日(947年5月15日),在栾城县殺胡林(今河北石家庄市欒城区西北)病逝,遼人将耶律德光的尸体破腹,取出内脏,装入几斗盐,带回北方,時人稱為「帝羓」,即「皇帝醃肉」之意。

\subsection{天显}

\begin{longtable}{|>{\centering\scriptsize}m{2em}|>{\centering\scriptsize}m{1.3em}|>{\centering}m{8.8em}|}
  % \caption{秦王政}\
  \toprule
  \SimHei \normalsize 年数 & \SimHei \scriptsize 公元 & \SimHei 大事件 \tabularnewline
  % \midrule
  \endfirsthead
  \toprule
  \SimHei \normalsize 年数 & \SimHei \scriptsize 公元 & \SimHei 大事件 \tabularnewline
  \midrule
  \endhead
  \midrule
  二年 & 927 & \tabularnewline\hline
  三年 & 928 & \tabularnewline\hline
  四年 & 929 & \tabularnewline\hline
  五年 & 930 & \tabularnewline\hline
  六年 & 931 & \tabularnewline\hline
  七年 & 932 & \tabularnewline\hline
  八年 & 933 & \tabularnewline\hline
  九年 & 934 & \tabularnewline\hline
  十年 & 935 & \tabularnewline\hline
  十一年 & 936 & \tabularnewline\hline
  十二年 & 937 & \tabularnewline\hline
  十三年 & 938 & \tabularnewline
  \bottomrule
\end{longtable}


\subsection{会同}


\begin{longtable}{|>{\centering\scriptsize}m{2em}|>{\centering\scriptsize}m{1.3em}|>{\centering}m{8.8em}|}
  % \caption{秦王政}\
  \toprule
  \SimHei \normalsize 年数 & \SimHei \scriptsize 公元 & \SimHei 大事件 \tabularnewline
  % \midrule
  \endfirsthead
  \toprule
  \SimHei \normalsize 年数 & \SimHei \scriptsize 公元 & \SimHei 大事件 \tabularnewline
  \midrule
  \endhead
  \midrule
  元年 & 938 & \tabularnewline\hline
  二年 & 939 & \tabularnewline\hline
  三年 & 940 & \tabularnewline\hline
  四年 & 941 & \tabularnewline\hline
  五年 & 942 & \tabularnewline\hline
  六年 & 943 & \tabularnewline\hline
  七年 & 944 & \tabularnewline\hline
  八年 & 945 & \tabularnewline\hline
  九年 & 946 & \tabularnewline\hline
  十年 & 947 & \tabularnewline
  \bottomrule
\end{longtable}

\subsection{大同}

\begin{longtable}{|>{\centering\scriptsize}m{2em}|>{\centering\scriptsize}m{1.3em}|>{\centering}m{8.8em}|}
  % \caption{秦王政}\
  \toprule
  \SimHei \normalsize 年数 & \SimHei \scriptsize 公元 & \SimHei 大事件 \tabularnewline
  % \midrule
  \endfirsthead
  \toprule
  \SimHei \normalsize 年数 & \SimHei \scriptsize 公元 & \SimHei 大事件 \tabularnewline
  \midrule
  \endhead
  \midrule
  元年 & 947 & \tabularnewline
  \bottomrule
\end{longtable}


%%% Local Variables:
%%% mode: latex
%%% TeX-engine: xetex
%%% TeX-master: "../Main"
%%% End:
