%% -*- coding: utf-8 -*-
%% Time-stamp: <Chen Wang: 2019-12-26 10:55:40>

\chapter{辽\tiny(916-1218)}

\section{简介}

遼朝(907年-1125年),國號大遼,契丹文又稱大契丹國,是中國歷史上由契丹人建立的一個朝代,國祚210年。

契丹族原為唐朝臣屬(松漠都督府),唐朝末年,首領耶律阿保機吞併了契丹各個部落後,於916年稱帝建國“契丹”。918年定都臨潢府(今內蒙古巴林左旗南)。契丹屢次南下中原,946年阿保機之子耶律德光攻滅後晉後確定國號為「大遼」,983年改為“契丹”,1066年改為“大遼”,直到1125年3月26日為金朝所滅為止。除了遼朝之外,契丹族尚建立相關國家。1122年,天祚帝北逃夾山,耶律淳於遼南京被立為帝,史稱北遼。遼朝滅亡後,耶律大石西遷到中亞楚河流域,1132年建立西遼。1211年西遼被屈出律篡位,1218年被蒙古帝國所滅。

史學界對「契丹」含義最廣為接受的說法是鑌鐵或刀劍之意。後來改國名為“遼”也是“鐵”的意思,同時“遼”也是契丹人發祥地遼水的名字,以示不忘本之意。又因與南方的中原政權長期對峙,而稱“北朝”,而稱中原王朝為“南朝”。遼朝926年滅渤海國,938年據燕雲十六州,後滅後晉,自居為繼承後晉的中原正統,即使之後退回北方。依據五行德運說的五行相生規律,後晉的「木」德之後為「水」德,因此遼朝以水為德運,並相應以黑色為正色。

遼朝全盛時期疆域東到日本海,西至阿爾泰山,北到額爾古納河、大興安嶺一帶,南到河北省南部的白溝河。契丹族本是遊牧民族,遼朝皇帝使農牧業共同發展繁榮,各得其所,建立獨特的、比較完整的管理體制。遼朝將重心放在民族發祥地,為了保持民族性將遊牧民族(契丹人)與農業民族(漢人)分開統治,主張因俗而治,開創出兩院制的政治體制。並且創造契丹文字,保存自己的文化。此外,吸收渤海國、五代、北宋、西夏及西域各國的文化,促進遼朝政治、經濟和文化各個方面發展。遼朝的軍事力量與影響力涵蓋西域地區,因此在唐朝滅亡後中亞、西亞與東歐等地區常將遼朝(契丹,英語作Cathay或Khitan)當做中國(俄語作Китай)。

契丹源於鮮卑,即為東胡後裔,北魏道武帝時出現,當時聚居于遼水上游一帶,自稱青牛白馬之後。648年唐太宗在契丹人領地設置松漠都督府,酋長任都督並賜李姓。契丹在660年唐高宗時反叛自立,並與李唐脫離關係,開元年間再度歸附,安史之亂後大唐國力衰微,契丹時而复叛,松漠都督府逐漸空殼化。晚唐時契丹迭刺部的首領耶律阿保機崛起並征服各部,取代痕德堇可汗後於907年即可汗位。他先後鎮壓了契丹貴族的叛亂和征服奚、室韋、黠嘎斯、阻蔔等部落,並且握有蒙古地區的產鹽區,在軍事與經濟方面都十分強盛。915年耶律阿保機出征室韋得勝回國,但被迫交出汗位,不久他在灤河邊建設了一座仿幽州的城敦。916年3月17日耶律阿保機建立契丹國,即遼太祖。

遼太祖掠奪中原的人口,收留因河北戰爭的流民,在草原上按照中原風格建立城敦以安置他們。並且任用韓延徽、韓知古、康默記與盧文進等漢人為佐命功臣。918年遼太祖建皇都臨潢府(今內蒙古巴林左旗南)。兩年後創建契丹大字並推行之。在軍事方面,他於925年東征渤海國,於舊地建立東丹國以統治渤海遺民,冊立皇太子耶律倍為東丹王。遼太祖一直有南征中原的意圖,然而於攻滅渤海後的隔年,在回師途中病倒,最後逝世。其妻述律平宣佈攝政,以次子耶律德光總攬朝政,屠殺政敵數百人以穩定政權。927年,耶律德光在述律平的支持下即位,即遼太宗。930年,東丹王耶律倍南逃後唐,遼太宗統一了契丹。

936年後唐發生內亂,河東節度使石敬瑭以自稱兒皇帝、割讓燕雲十六州為條件,請求遼太宗支援攻打後唐。遼太宗遂親率5萬騎兵,於晉陽、洛陽等地擊敗後唐軍,最後協助石敬塘攻滅後唐,石敬塘得以建國後晉。契丹國獲得燕雲十六州後,將燕雲十六州建設成為進一步南下的基地。為了統治當地漢族,遼太宗採取「因俗而治」的統治方式,實行分治漢人和契丹人、南北兩面官的兩院制。並且改幽州為南京、雲州為西京。

944年後晉出帝即位,他不願向契丹臣服,上表稱孫不稱臣。遼太宗趁機率軍南下。947年,契丹軍攻克後晉首都開封,後晉亡,遼太宗改國號為大遼。(另一說指契丹早於會同元年(公元938年)改國號為大遼)雖然遼太宗有長久經營中原的意圖,然而因縱兵掠奪民財,以及不讓諸位節度使返回鎮地,招來中原人民的反抗。四月,遼太宗被迫引軍北返,最後在河北欒城病逝。947年位於中原的耶律吼等將領擁立耶律阮為帝,是為遼世宗。在上京臨潢府的太后述律平想讓其子耶律李胡繼承帝位,不同意耶律阮稱帝。太后派耶律李胡與耶律阮在遼南京北部的泰德泉交戰,最後由耶律阮打贏這場戰爭。在經過大臣耶律屋質的勸阻之下,太后才認同耶律阮的帝位。

遼世宗任用賢臣耶律屋質,進行一系列改革,將遼太宗時的南面官和北面官合併,成立南北樞密院,廢南、北大王。後來南北樞密院合併,形成一個樞密院。遼世宗的改革使遼朝從部落聯盟演進為中央集權。遼世宗在位期間,一直不忘佔領中原的期望,多次對中原用兵。然而遼世宗好酒色,喜愛打獵。晚年更是任用奸佞,大興封賞降殺,導致朝政不修,政治腐敗。951年,遼世宗協助北漢攻打後周,行軍至歸化 (今內蒙古呼和浩特)的祥古山時,由於其他部隊未到,先行駐紮在火神澱(今河北宣化西)。其間喝酒、打人、打獵,眾將很是不滿。最後被耶律察割殺死於夢鄉中。

951年耶律察割在火神淀(今河北宣化)發動政變,殺遼世宗並自行稱帝,遼太宗之長子耶律璟和耶律屋質等率兵殺死耶律察割後,被立為帝,即遼穆宗。遼穆宗雖討厭女色,而無所出,但卻經常酗酒,天亮才睡,中午方醒,因此長時期不理朝政,國人稱之為「睡王」。遼穆宗前期,朝廷內部不穩,離心離德,大臣經常發生叛亂或是南奔中原的事件:952年六月,蕭眉古得欲叛遼南奔後周,陰謀敗露,被殺。七月,政事令耶律婁國、林牙耶律敵烈等謀亂被捕後伏誅。953年十月,耶律李胡之子耶律宛等人謀反,事情被察覺後被捕。960年七月,政事令耶律壽遠、太保楚阿不等人謀反,事敗伏誅。十月,耶律李胡之子耶律喜隱謀反,事敗被捕,因供詞牽涉耶律李胡,耶律李胡入獄而死。

由於政局動盪不安,迫使遼穆宗停止了遼太宗、遼世宗一貫執行的南伐中原政策,以恢復因長期戰事而消損的國力,與南唐、北漢聯合對抗遂漸強盛的後周。959年後周發動北伐,遼朝寧州(今河北青縣)刺史王洪舉城投降。周軍隨後攻克益津關(今河北霸州)、瓦橋關,莫州、瀛州刺史劉楚信、高彥暉也舉城投降。當時後周世宗欲一鼓作氣,直取幽州,遼穆宗甚至有意放棄燕雲十六州。最後後周世宗因為重病而南返,莫州、瀛州歸後周領有,而遼軍加強防禦,不敢南下。由於遼穆宗本人喜好殺戮,經常親手殺人。同時又愛好打獵到「竟月不視朝」,最後於969年二月被侍人所弒。耶律賢被推舉為帝,即遼景宗,改元為保寧。

遼景宗勤於政事,重用賢臣如室昉、郭襲,使遼朝出現一陣清明。由於遼景宗體弱多病,有時無法上朝,軍國大事多由皇后蕭綽協助處理。遼景宗對遼穆宗時謀反的皇族採比較寬鬆的政策,因而謀亂者少,朝廷比較穩定。遼景宗對外政策仍採不主動南伐中原、僅援北漢的方針。遼景宗前期,遼朝與北宋聘史往還,互賀節日。宋太宗趙光義統一江南後,於979年親征北漢,遼朝派數萬兵支援北漢。三月,遼軍在白馬嶺(今山西盂縣)與宋軍交戰戰敗,遼將耶律敵烈等人戰死。六月,北漢主劉繼元降宋。遼朝只能全力固守幽薊。宋太宗乘勝圍攻幽州,遼朝派耶律沙、耶律休哥、耶律斜軫等率軍與宋軍會戰於高梁河(今北京西直門外),史稱高梁河之戰。遼軍最後擊潰宋軍,宋太宗僅以身免,此後宋遼兩國進入了相持狀態。

982年遼景宗病逝,遼聖宗繼位,尊蕭綽為皇太后,並由蕭太后攝政。當時蕭太后30歲,遼聖宗12歲,而蕭太后之父蕭思溫於970年被害,無嗣,使得蕭太后也沒有外戚可以依靠。而諸王宗室二百餘人擁兵自重,控制朝廷,對蕭太后及遼聖宗構成了莫大的威脅。蕭太后先重用大臣耶律斜軫、韓德讓參決大政,南面軍事委派給耶律休哥,撤換一批大臣,並下令諸王不得相互宴請,要求他們無事不出門,並設法解除他們的兵權。在這些行動後,遼聖宗和蕭太后的地位才穩定下來。蕭太后攝政二十七年,傳聞曾改嫁給韓德讓。在她執政期間進行改革,並且勵精圖治,注重農桑,興修水利,減少賦稅,整頓吏治,訓練軍隊,使遼朝百姓富裕,國勢強盛。1009年遼聖宗親政後,遼朝已進入鼎盛,基本上延續蕭太后執政時的遼朝風貌,反對嚴刑峻法,並且防止貪汙事件。在文教方面,遼聖宗實行科舉,編修佛經,佛教極為盛行。在位其間四方征戰,對宋戰爭屢屢獲勝,俘獲號稱楊無敵的宋朝名將楊繼業。

北宋立國之初即有意要收復燕雲十六州,先後於979年、986年兩度北伐,皆為遼軍所擊敗。遼聖宗為了防止高麗與北宋結盟,進而威脅遼朝東部,於993年發動高麗契丹戰爭以降服高麗,於1009年的東征時最遠攻入高麗開城。之後為瞭解決遼宋之間的長期對抗,以及避免契丹貴族威脅皇權,蕭太后與遼聖宗於1004年親率大軍深入宋境。宋真宗畏敵,欲遷都南逃,因宰相寇準堅持而親至澶州(今河南濮陽)督戰。宋軍士氣大振,擊敗遼軍前鋒,遼將蕭撻凜戰死。遼軍恐腹背受敵,提出和約。主和的宋真宗於次年初與遼訂立和約,協定宋每年貢遼歲幣銀十萬兩、絹二十萬匹,雙方各守疆界,互不騷擾,成為兄弟之邦,此即澶淵之盟,從此兩朝和好達一百二十年之久。之後遼聖宗結好西夏,而西夏也搖擺於宋、遼之間以圖存,形成遼宋西夏三國鼎立的局勢。

1031年遼聖宗去世,長子耶律宗真即位,即遼興宗。遼興宗其生母蕭耨斤(即法天太后)自立為皇太后並攝政,並派人杀死遼興宗的養母齊天皇后蕭菩薩哥。法天太后重用在遼聖宗時代被裁示永不錄用的貪官汙吏以及其娘家的人。遼興宗因無權而不能救,母子因此結怨。法天太后對遼興宗並不信任,打算改立次子耶律宗元(即耶律重元)為帝。耶律宗元把這一事告訴興宗。遼興宗怒不可遏,於1034年用武力廢除法天太后,迫法天太后“躬守慶陵”,大殺太后親信。七月,遼興宗親政,修建陵園安葬齊天皇后。而後,把法天太后接回來,並與她保持十里的距離,以防不測。興宗母子的感情裂痕始終沒有填平。

遼興宗在位時,遼朝國勢已日益衰落。而有遼興宗一朝,奸佞當權,政治腐敗,百姓困苦,軍隊衰弱。面對日益衰落的國勢,遼興宗連年征戰,多次征伐西夏;逼迫宋朝多交納歲幣。但是這些反而使遼朝百姓怨聲載道,民不聊生。遼興宗還迷信佛教,穷奢极欲。遼興宗曾與其弟耶律宗元賭博,一連輸了幾個城池。他對自己的弟弟耶律宗元非常感激,一次酒醉時答應百年之後傳位給耶律宗元。其子耶律洪基(即為後來的遼道宗),也未曾封為皇太子,只封為天下兵馬大元帥而已。這種下了遼道宗繼位後,耶律宗元父子企圖謀奪帝位的惡果。

宋夏戰爭後北宋內外交困之際,使得遼朝趁機侵宋。在徵求張儉的意見後,一面派其弟耶律宗元和蕭惠在邊境製造欲攻宋的虛張聲勢,一面派蕭特末(漢名蕭英)和劉六符於1042年正月去宋朝索要瓦橋關南十縣地。宋朝派富弼與遼方使節談判,此即重熙增幣。雙方於九月達成協議,在澶淵之盟規定贈遼歲幣基礎中,再增加增歲幣銀十萬兩、絹十萬匹以了結這次索地之爭。遼興宗還派耶律仁先和劉六符再次使宋爭得一個“納”字,即歲幣是宋方納給遼方的,不是贈送的。宋仁宗也委曲求全予以應允,而條件是遼朝須逼西夏與宋朝和談。因此,在遼宋和好之後,為答應宋朝要求,遼夏關係惡化並發生遼夏戰爭。遼興宗兩次親征西夏,均遭失敗,而西夏最後願意分別向遼和宋稱臣進貢。

遼道宗繼位後,1063年七月耶律宗元聽從兒子的勸說,發動叛亂,自立為帝,不久被遼道宗所平,耶律宗元自盡,史稱灤河之亂。遼道宗在位期間,遼政治腐敗,國勢逐漸衰落。道宗並沒有進行改革圖新,而且本人也腐朽奢侈,這時地主官僚急劇兼併土地,百姓痛苦不堪,怨聲載道。遼道宗重用耶律乙辛等奸佞,自己不理朝政,並聽信耶律乙辛的讒言,相信皇后蕭觀音與伶官趙惟一通姦而賜死皇后。而同時耶律乙辛為防太子登基對自己不利,故陷害皇太子耶律濬,並將其殺害,史稱十香詞冤案。後來,一位姓李的婦女向遼道宗進「挾穀歌」遼道宗才把皇太子的兒女接進宮。1079年七月,耶律乙辛乘遼道宗遊獵的時候意圖謀害皇孫耶律延禧,遼道宗接納大臣的勸諫,命皇孫一同秋獵,才化解耶律乙辛的陰謀。大康九年,遼道宗追封故太子為昭懷太子,以天子禮改葬。同年十月,耶律乙辛企圖帶私藏武器到宋朝避難,事敗被誅。1101年正月,遼道宗去世,皇孫耶律延禧繼位,即天祚帝。當時西夏夏崇宗因受到北宋攻擊一再向遼求援,並求天祚帝女尚公主為妻。最後天祚帝於1105年將一個族女耶律南仙提升為公主嫁給夏崇宗,並派使者赴宋,勸北宋對西夏和談。

1112年二月十日天祚帝赴春州,召集附近女真族的酋長來朝,宴席中醉酒後令諸位酋長為他跳舞,只有完顏阿骨打不肯。天祚帝不以為意,但從此完顏阿骨打與遼朝之間不和。九月,完顏阿骨打不再奉詔,並開始對其他不服從他的女真部落用兵。1114年春,完顏阿骨打正式起兵反遼。一開始天祚帝並未將完顏阿骨打當作一個重大威脅,但是所有他派去鎮壓完顏阿骨打的軍隊全部戰敗。1115年天祚帝為瞭解決女真的威脅,下令親征,但是遼軍到處被女真軍擊敗,完顏阿骨打也自稱皇帝,建立金朝,即金太祖。遼朝於同年發生內亂,耶律章奴在遼上京叛亂,雖然這場叛亂很快就被平定,但是分裂了遼朝內部。此後位於原渤海國的東京也發生高永昌叛亂自立,這場叛亂一直到1116年四月才被平定。五月女真藉機佔領了遼東京和瀋州。1117年女真攻春州,遼軍不戰自敗。

1120年金軍攻克遼上京,守將蕭撻不也投降,到1121年遼朝已經失去一半的領土。遼將統伊都等人到咸州(今遼寧開原)請降,天祚帝逃到鴛鴦濼(今河北赤城),奔向遼西京。金軍追擊,天祚帝又逃到伊蘇部。而內部又發生因為皇位繼承問題而爆發的內亂,最後天祚帝殺他的長子耶律敖魯斡而結束,但是這使得更多的遼兵投降金朝。1122年正月,金軍攻克遼中京,天祚帝被金兵所迫,流亡夾山。

由於位於遼南京的耶律大石與李處溫等人不知天祚帝去向,他們擁立耶律淳為帝,即天錫帝,史稱北遼。天錫帝降天祚帝為湘陰王,並遣大使奉表於金朝,乞為附庸。可是事未完成,他就病死,妻遼德妃稱制,改年號為德興。此時遼臣李處溫父子覺得前景不妙,打算向南私通北宋的童貫,欲劫持遼德妃納土於宋。向北私通金人,作金的內應。後她發現他們罪行而賜死之。當年十一月,遼德妃五次上表給金朝,只要允許立耶律定為遼帝,其他條件均答應。金人不許,她只好派兵死守居庸關,十一月居庸關失守,十二月遼南京被攻破。遼德妃帶著隨從的官員投靠天祚帝,天祚帝誅殺她。

1123年正月,在上京的回離保(蕭幹)自立,號奚國皇帝,八月平定。1124年,天祚帝已經失去了遼朝的大部分土地,他的兒子和家屬大多數被殺或被俘,天祚帝退出漠外,準備投奔西夏。1125年3月26日,天祚帝在應州被為金人完顏婁室等所俘,遼朝亡。八月天祚帝被解送金上京(今黑龍江阿城),金太宗封為海濱王。1128年,天祚帝病故,遺臣蕭朮者對故主行人臣之禮。

此後,遼朝貴族耶律大石在西北召集殘部,控制了蒙古高原和新疆東部一帶。1130年,由於受到金兵的壓迫,耶律大石決定放棄蒙古高原,率部西征。1132年,耶律大石在葉密立(今新疆額敏)稱帝,史稱西遼(西方稱為黑契丹或哈剌契丹),首都虎思斡魯朵(今吉爾吉斯托克馬克東南布拉納城)。西遼曾一度擴張到中亞,成為中亞強國。1143年,在耶律大石死後,西遼經歷蕭塔不煙、耶律夷列、耶律普速完、耶律直魯古與屈出律的統治。最後1218年被成吉思汗的蒙古軍隊滅亡,立國凡87年。

1212年,遼朝宗室耶律留哥在隆安(今吉林農安)、韓州(今吉林梨樹)一帶起軍反抗金朝,並且受到蒙古帝國的庇護。隔年三月,耶律留哥稱王,國號遼,史稱東遼。1216年初,耶律留哥之弟耶律廝不叛變,在澄州稱帝,史稱後遼。耶律廝不不久被部下所殺,眾推耶律乞奴為監國。同年秋,木華黎率蒙古軍東下,耶律乞奴等不敵,率九萬契丹族越過鴨綠江進入高麗境內。不久契丹諸貴族自相殘殺,後遼最後於1220年滅亡。耶律留哥建國後依然歸附蒙古帝國,成為其藩屬,1270年元世祖撤藩,東遼正式滅亡。

遼朝初期的疆域在今遼河流域上游一帶,在遼太祖及遼太宗時期不斷對外擴張,遼太祖時征服奚(今河北北部)、烏古、黑車子室韋(今內蒙古東部呼倫湖東南)、韃靼、回鶻與渤海國。938年遼太宗時取得燕雲十六州,並一度佔有中原。1005年遼聖宗與北宋簽定澶淵之盟,最後確定了與宋的邊界。遼朝全盛時,疆域東北至今庫頁島,北至蒙古國中部的色楞格河、石勒喀河一帶,西到阿爾泰山,南部至今天津市的海河、河北省霸縣、山西省雁門關一線與北宋交界,與當時統治中原的宋朝相對峙,形成南北朝對峙之勢。

遼朝於契丹國時期領有八部,建立遼國後的行政區劃為道、府(州)、縣三級。共有5京、6府,156州(軍、城),309縣。道有五個,每個道有一個政治中心,稱為京,並以京的名稱來命名道。道下設府、州、軍、城4種政區,為同一級別。

遼朝政治的核心是因俗而治,以該文化的典章制度統治該族人民,這個特色在行政區劃也看得出來。在契丹部落時期就征服鄰近的奚族,於當地依舊立奚王,建立自己的政府機構。契丹國時期攻滅渤海國,為了便於統治渤海人民,於當地建立東丹國,沿襲渤海國行政體制。東丹國最後被廢,改為中臺省。在佔領燕雲十六州後,也在當地也沿襲後唐行政體制以便於統治當地漢人。

而頭下軍州是遼朝一種特殊建置。契丹貴族將所俘掠的人口,建立州、軍安置,督迫其為主人勞作。遼諸王、外戚、大臣所領有的頭下軍州可建城郭,其餘只能有自己的頭下寨堡。頭下軍州多設在潢河流域契丹住地。俘戶主要是河北、山西的漢人和東北地區的渤海人。頭下州縣名稱,常採用俘戶原籍州縣名稱,如俘衛州民,建衛州;俘三河縣民,建三河縣;俘密雲民,建密雲縣等。頭下軍州的制度到遼聖宗時期逐漸廢除。

遼朝如同宋朝,也有五京制度,主要是為控制因戰爭獲的土地而設置的,或是因為爭奪一地而設置的前進基地。這些先後成立的五京為上京臨潢府(今內蒙古林東)、因控制奚領地而設置的中京大定府(今內蒙古寧城)、因為渤海遺民設置的東京遼陽府(今遼寧遼陽)、因為燕雲十六州而設置控制漢地的南京析津府(今北京)與監視西夏的西京大同府(今山西大同)。五京中,只有上京是首都,其他均是陪都。然而遼中京至澶淵之盟後,其政治作用加強,地位直逼上京的首都地位。

捺缽,即「行在」、「營盤」,為遼帝的行宮。遼朝雖以上京臨潢府作為首都,但其政治核心在捺缽。這是因為契丹族轉徙不定、車馬為家的特性,決定了皇帝的巡狩制。一切重大政治問題均在捺缽隨時決定,是處理政務的行政中心。每年又「四時巡守」,「四時各有行在之所,謂之捺缽」。皇帝在遊獵地區設的行帳,以區別於皇都的宮帳。因氣候、自然條件的制約,四時各有捺缽之地。

遼太宗時,取燕雲十六州後,其國土包括長城以南的廣大地區,為保持契丹族的騎射善戰傳統的經濟生活,仍然過著「轉徙隨時,車馬為家」的生活。正如《遼史》中記載的「遼國盡有大漠,浸包長城之境,因宜為治,秋冬違寒,春夏避暑,隨水草就畋漁,歲以為常」,四時各有行在之所,在這種特殊經濟、政治、文化背景下,在契丹的管理體制上,逐漸形成了一套縣有鮮明遊牧契丹民族獨特特點的四時捺缽制度。契丹皇帝四時巡行的宮帳(也稱牙帳),即春捺缽、夏捺缽、秋捺缽、冬捺缽。

由於遼朝屬於多民族國家,其政治體制融合契丹體制與唐宋體制而形成南北院制。南北院制分成北面官制和南面官制,以「本族之制治契丹,以漢制待漢人」,借此保護契丹固有文化與政治體制。北面官治宮帳、部族、屬國之政,南面官治漢人州縣、租賦、軍馬之事,因俗而治。

北面官制中,北南樞密院是遼朝最高官制,北樞密院掌管全國軍政,類似唐朝的兵部;南樞密院掌管銓選、丁賦等政。北樞密院管轄契丹族在內的少數民族,南樞密院管轄漢族以及州、郡、縣。樞密院下還設北南宰相府,北、南宰相都由皇族耶律氏和后族蕭氏所把持。此外還有管理契丹或漢族軍民之事的北南大王院、管理北南院御前祗應之的北南宣徽院、管理皇室教育的大內惕隱司、管理刑獄的夷離畢院、管理文翰之事的大林牙院與管理禮儀的敵烈麻都司等。

南面官制的官名及職掌沿襲唐朝制度,並參照五代和宋朝的官制。以太尉、司徒、司空為三公;太師、太傅、太保為三師。在其下設有中書省、門下省、尚書省等三省。其下有六部與大理寺。還有御史台、翰林院(又稱南面林牙)、國史院、太常寺以及諸監、衛等。官有實授、遙授之分。職事官與散官及階、勳、憲銜、封爵、食邑戶數等配套。遼代官名多有契丹語官名,如林牙即翰林,惕隱掌管皇族政教,夷離畢掌管刑獄,乙裡免為誥命夫人等。而朝廷重要職位都掌握在契丹人手中,尤其是帝系和外戚手中。

遼朝的法律因俗而治,使用雙軌制度,基本原則以國制治契丹,以漢制待漢人。契丹人採屬人主義,漢人採屬地主義。早期有民族岐視,契丹制度較為寬鬆,而漢地由於繼承歷代法律,法條較為綿密。遼聖宗時契丹人法也用漢律來斷,這反映漢人地位的提升。而皇帝往往隨意殺人,無法無天,遼穆宗尤甚。

契丹族原臣服唐朝,被唐朝設立為松漠都督府。於晚唐五代時建立契丹國獨立,並且屢次入侵河北地區。五代後唐末年,遼太宗接受石敬瑭的請求,協助他建立後晉取代後唐,以獲得燕雲十六州與後晉的臣服。不久又南征中原,滅後晉以建立遼朝。至此遼朝與中原的外交關係首度轉為遼朝居上,中原臣服的狀態。之後遼朝衰退,後周與北宋為了燕雲十六州又相繼北伐,雙方恢復對峙的局面。遼朝採取防禦策略,並且扶持北漢對抗中原的北伐,屢次抵禦中原的進攻。直到遼聖宗時,經過充分準備之後,再度發動南征,率遼軍直逼北宋的澶州。最後雙方訂立澶淵之盟,遼朝與北宋建立大致上平等的外交關係,長達120年,雙方並且加強經濟和貿易往來。

1042年遼興宗乘宋夏戰爭後北宋內外交困之際,率重兵陳列遼宋邊界,並派蕭特末(漢名蕭英)和劉六符去宋朝索要瓦橋關南十縣地。宋朝派富弼與遼方使節談判,雙方於九月達成協議,此即重熙增幣。最後增加增歲幣銀十萬兩、絹十萬匹以了結這次索地之爭。遼興宗還派耶律仁先和劉六符再次使宋爭得一個「納」字,即歲幣是宋方納給遼方的,不是贈送的。宋臣富弼建議宋仁宗答應要求,並且要求遼朝約束西夏作為條件以破壞遼與西夏的關係,最後使遼興宗兩次親征西夏,勞民傷災。遼朝晚期因受女真族建立的金朝入侵,加上朝廷內部分裂與內鬥,使遼朝有意與北宋和談。但是北宋已經與金朝建立海上之盟而共同伐遼,所以拒絕和談,最後遼朝亡於金朝。

遼朝於926年滅渤海國後與高麗接觸。942年送給高麗50匹駱駝,但遭高麗太祖拒絕。遼使被放逐到孤島,所送駱駝也都被餓死。至此遼朝多次襲擾高麗邊界,993年,遼聖宗率大軍越過鴨綠江入侵高麗。最後雙方和談,在高麗同意斷絕與宋的聯盟後,遼聖宗率軍北返,雙方建立友好的睦鄰關係。1009年高麗發生軍變。遼聖宗趁機入侵高麗,最後在攻下開城後北返。1018年,遼朝率大軍再度東征高麗。但不敵高麗軍隊。雙方之後談和,以後遼朝再也沒有入侵高麗。

遼朝與西北諸國保持著較為友好和睦的往來。遼朝西境的主要鄰國西夏,長期以來,一直與遼朝保有朝貢和聯姻關係。一度為遼藩屬,被稱為甥舅之邦。遼朝與西域諸國的關係也源遠流長。早在遼太祖耶律阿保機時,就曾經率軍西征,使西域諸國相繼臣服。統和年間,王太妃出師西域,1003年建可敦城,作為西北的邊防重鎮,經過多年的經營,使遼朝的勢力範圍涵蓋漠南、漠北與西域之地。遼朝政府對這些降附的部落屬國,均採取「因遷種落,內置三部」的羈縻政策,使的這些國家互相監督,皆不願背叛遼朝。這些都使蔥嶺以東的甘州回鶻、西州回鶻與蔥嶺以西的喀喇汗國,基本上都是親附遼朝,其與北宋的關係較疏。此外,西亞的波斯與大食(伽色尼王朝)在遼初也相繼道使來通好。天贊二年,波斯使來,其明年大食使來。大食國王遣使為王子請婚,未允。次年,復遣使請婚,遼聖宗以宗室之女嫁之。

因此,在唐朝滅亡之後,西域、西亞與東歐地區皆將遼朝(契丹)作為中國的代表稱謂。中亞和西亞的伊斯蘭兵書中,還將中國傳過去的火藥與火器稱為「契丹花」、「契丹火箭」等。直到今日,俄羅斯民族的語言和文字當中,也依舊以契丹作為中國的稱呼。

遼朝的軍隊,平時約在二十萬至三十萬左右。契丹是遊牧民族,善於騎射,平時放牧漁獵,既是生產經濟活動,也是軍事練習,有戰爭很快即可集合成軍。由於全民皆兵,遼朝所能動員的兵力在總人口當中,比例很高,為164萬2800人。由於保留著原始部族的痕跡,並處於由奴隸制向封建制迅速轉化的歷史階段,軍事制度初期多與本民族社會制度合為一體,進入長城以南地區後,既保有本民族特色,又逐步接受漢族影響,具有民族融合的特點。遼朝皇帝親掌最高兵權。下設北南樞密院。北樞密院為最高軍事行政機構,一般由契丹人主管﹔南樞密院亦稱漢人樞密院,掌漢人兵馬之政,因而出現一個朝廷兩種軍事體制並存的局面。  

遼朝兵制分為宮帳軍、部族軍、京州軍和屬國軍。宮帳軍,即皮室軍,徵集直屬皇帝的著帳戶壯丁組成,是契丹族親軍,供宿衛和征戰,「以行營為宮,選諸部豪健千人,置腹心部」。部族軍,主要由契丹以外的部族壯丁組成,供守衛四邊。以上兩種部隊是遼軍的主力。京州軍,亦稱五州鄉軍,徵集五京道各州縣的漢族、渤海族等的壯丁組成。屬國軍,由臣屬國壯丁組成。後兩種部隊為輔助兵力。遼初,貴族男子人人服兵役,年齡在15~50歲之間的列籍正軍,兵器、戰馬自備。並且時常派遣掠奪周邊物資,時稱打草穀。遼軍以騎兵為主,主要武器是弓箭和刀槍。後期從宋朝傳入拋石機,編有炮手軍。

遼朝軍制十分重要的一點便是所謂的斡魯朵制度,即宮衛制度。斡魯朵意為宮帳或宮殿之意,這是直屬於遼國皇帝及太后的禁衛,另外皇室貴族或受皇帝特別恩寵的大臣也有自己的斡魯朵。斡魯朵制度對加強皇權,維護耶律氏的統治有相當重要的作用。當主人去世後,斡魯朵的人員就變成主人陵墓的守衛者。遼朝共計有十二宮一府。而當代皇帝的斡魯朵出巡時,所有前朝的斡魯朵守衛都要隨行出動當守衛者,所以越後代皇帝的出巡規模就越大。

契丹族本是遊牧民族,原本是「畋漁以食、皮毛以衣、馬逐水草、人仰湩酩」。遊牧民族經濟上的弱點,在契丹立國之前大致上解決。以人為方式在遊牧地區內營造綠洲,再將農耕民族移居其中。契丹人從事農業、手工業,都是由遼太祖的祖父、父親以及伯父等傳入契丹,又傳授紡織。遼在各地均設群牧使司以管理官有的牲畜。遼朝皇帝使農牧業共同發展繁榮,各得其所,建立獨特的、比較完整的管理體制。

遼朝境內農作物品種齊全,既有粟、麥、稻、穄等糧食作物,也有蔬菜瓜果。他們借鑒和學習中原的農業技術,引進作物品種,還從回鶻引進了西瓜、回鶻豆等瓜果品種,結合北方氣候特點形成了一套獨特的作物栽培技術。遼朝的土地有公田和私田兩類。在沿邊設置的屯田自然是公田。募民耕種的在官閒田也是公田,百姓領種十年以後,要對朝廷繳納租賦。至於所說的「占田置業入稅」則是私田了。估計屯田多集中在北部沿邊,私田則多在遼國南境。在契丹的漢人依然是以男耕女織的方法維持家庭收入。同時,契丹將戰爭中俘掠的漢人,安置在契丹腹心地區,建立許多頭下軍州。除少部分需上繳,其餘收入皆歸頭下主所有。遼廷為了鼓勵人民開闢荒地,立例若成功開闢農地可免租賦十年,形成契丹特有的農牧混合經濟。遇到兵荒、歲饑之年,也要減、免賦稅,991年1月遼聖宗時期,「詔免三京諸道租稅,仍罷括田」。1075年9月遼道宗時期,「以南京饑,免租稅一年,仍出錢粟振之」。遼朝從事農業生產的居民被編入州縣,包括擁有少量土地的自耕農和靠租種地主土地為生的佃戶。他們無論經濟地位如何,都是具有自由民身分的國家編戶,並承擔著國家的賦役負擔。寺廟的佃戶多是貴族、官僚隨同土地一起轉贈的,是既向國家納稅又向寺廟交租的另一種形式的稅戶。

遼朝的畜牧業十分發達,契丹人的牧業經濟得到了較大發展。牧業是契丹等部落民的生活來源,也是遼朝所以武力強盛、所向克捷的物質條件。當時陰山以北至臚朐河,土河、潢水至撻魯河、額爾古納河流域,歷來有優良的牧場。契丹各部和屬部中的阻卜、烏古、敵烈、回鶻、黨項等,主要從事遊牧業。羊、馬是契丹等遊牧民的主要生活資料:乳肉是食品,皮毛為衣被,馬、駱駝則是重要的交通工具。戰爭和射獵活動中馬匹又是不可缺少的裝備。因此,“蕃漢人戶亦以牧養多少為高下”。阿保機之妻述律氏曾自豪地說:“我有西樓羊馬之富,其樂不可勝窮也”。羊、馬也是遼朝向契丹諸部和西北、東北屬國、屬部徵收的賦稅和貢品,是遼朝的重要經濟來源,因而受到統治集團的重視。遊牧的契丹人,編入相應的部落和石烈,在部落首領的管理下,在部落的分地上從事牧業生產,承擔著部落和國家的賦役負擔,沒有朝廷和部落首領的允許,不能隨意脫離本部。他們是牧區的勞動者、牧業生產的主要承擔者,是部落貴族的屬民。

遼代的冶鐵業發達,發掘出土鐵制的農業工具、炊具、馬具、手工工具可與中原的產品相媲美。遼東是遼朝產鐵要地,促進遼朝冶鐵業的發展。初期,曾以橫帳和大族奴隸置曷術石烈,從事冶煉。“曷術”,即契丹語“鐵”。曷術石烈在聖宗時因戶口繁息和生產關係的變化,改編為部,仍以鐵為賦。遼在手山、三黜古斯和柳濕河分置三冶。其中手山為今遼寧省鞍山市的首山,這裡的礦冶史最晚當起自遼代。

遼代陶藝受唐代影響,墓葬出土文物則顯示部分宋代器皿及其他器皿自國外輸入,但金、銀器製作亦採用唐、宋的金屬打製和鍍金技術。遼瓷在中國陶瓷發展史上佔有重要地位,瓷器的造型可分為中原式和契丹式兩類,中原式仿造中原的風格燒造,有碗、盤、杯、碟、盂、盒、壺、瓶等,契丹式則仿造本族習慣使用的皮制、木制等容器樣式燒造,器類有瓶、壺、盤、碟,造型獨具一格。缸瓦窯村窯是一處目前所知遼代最大的古瓷窯遺址,可生產白釉、單釉和三彩釉瓷以及宮廷所用的官窯器物。遼代的鎏金、鎏銀、染織、造馬具、制瓷以及造紙等手工業門類齊全,工藝精湛。契丹鞍與端硯、蜀錦、定瓷更被北宋《袖中锦》評比為「天下第一」。陳國公主與駙馬墓、耶律羽之墓等貴族墓葬出土的精美金銀器都反映出契丹獨特的民族特色和高度的工藝技術水準。如在內蒙古翁牛特旗廣德公鄉遼墓出土的雙猴綠釉雞冠壺和龍首綠釉雞冠壺就是仿契丹族皮囊容器的模式,在壺體側邊作出仿皮革縫製的痕跡,此類壺是契丹民族特有的生活器皿。


隨著農、牧、手工業的發展,交換逐漸頻繁,商業活動也日益活躍。早期,遼太祖在炭山北建羊城,“起榷務以通諸道市易”。後版圖擴大,建置完善,經濟成分增加,範圍擴大,商業也有了進一步的發展。遼五京相繼建成後,都成了遼朝的重要商業城市。遼朝與周邊各政權、各民族、國家的經濟往來多以朝貢和互市的方式進行。由於商業的發展,遼朝境內也出現了富有的商人階層,他們或經商於五京、州縣,或來往於遼、五代諸國或宋朝,有的甚至成為代表遼朝辦理交涉的使臣,如遼太宗時的回圖使喬榮經商於後晉,又為遼朝商業貿易的代表,並可作為使臣與後晉交涉政務。西京歸化州的韓師訓也是富甲一方的商人。

遼代物價甚低,雖有鹽酒之稅,但各地稅率並不一致。商業貿易的繁榮促進了貨幣經濟的發展。據文獻記載,耶律阿保機之父撒剌的時,已開始鑄造貨幣。然而貨幣使用量不多,遼世宗時,上京還處在交易無錢而用布的狀態。各地都用不同貨幣,如聖宗以前所鑄的遼錢極少,聖宗之後稍微多了一些,但在流通貨幣中,所佔數量仍甚少,不及百分之二,主要的是宋錢,其次是唐及五代及其他朝代的錢;在對外交易方面,遼主要與宋和西夏等通過邊境上的榷場進行互補性的交易。另外與日本、高麗、阿拔斯王朝、基輔公國和喀喇汗國也有貿易往來。

遼朝吸收許多漢文化與渤海國文化。滅渤海後,渤海遺民大量聚居于遼上京、遼東京一帶的州縣,較先進的渤海文化對遼文化有較為廣泛的影響。據漢地幽雲十六州到後來和宋朝的頻繁交往,無論是戰爭還是和平時期的榷場貿易,漢文化對於遼朝的影響都是巨大的。由於大量漢文書籍的翻譯,將中原人民的科學技術、文學、史學成就等介紹到了草原地區,帶動和促進了遊牧民族草原文化的發展。遼朝皇室和契丹貴族多仰慕漢文化,如遼的開國皇帝遼太祖崇拜孔子,先後於上京建國子監,府、州、縣設學,以傳授儒家學說,又建立孔子廟;遼聖宗常閱讀《貞觀政要》、道宗愛看《論語》等;遼道宗時,契丹以“諸夏”自稱,道宗又說“吾修文物,彬彬不異中華。”教育方面實行設學養士和科舉取士。

遼朝文人既用契丹語言文字創作,也大量用漢語文寫作。他們的作品有詩、詞、歌、賦、文、章奏、書簡等各種體裁,有述懷、戒喻、諷諫、敘事等各種題材。作者包括帝后、宗室、群臣、諸部人和著帳郎君子弟。契丹的詩詞既有氣勢磅礴之句,也有清新優美之詞。遼興宗也善為詩文,1050年宋使趙概至遼,遼興宗于席上請概賦《信誓如山河詩》。在遼朝諸帝中,遼道宗文學修養最高,善詩賦,作品清新雅麗,意境深遠。有《題李儼黃菊賦》。宗室東丹王耶律倍有《樂田園詩》、《海上詩》。耶律國留、耶律資宗、耶律昭兄弟三人皆善屬文、工辭章,耶律國留有《兔賦》、《寤寐歌》;耶律資宗出使高麗被留期間,“每懷君親,輒有著述”,後編為《西亭集》;耶律昭因事被流放西北部,致書招討使蕭撻凜,陳安邊之策,詞旨皆可稱。遼道宗的皇后蕭觀音《諫獵疏》、《回心院》和應制詩《君臣同志華夷同風》表達關心社稷安危、致主澤民的政治理想。流傳至今的遼人作品除王鼎的《焚椒錄》外,還有寺公大師的《醉義歌》。《醉義歌》是使用契丹語創作,有金朝耶律履的譯文,只是契丹文原作和耶律履譯文已經失傳,今有耶律履的兒子耶律楚材的漢譯本傳世。

在書目方面,遼設國史院,專修整歷史,設官監修國士、史館學士、修國史等,曾撰寫起居注、日曆、實錄二十卷、國史,又把不少漢人書籍翻譯為遼朝文字,如《五代史》等。當中,遼代所寫的實錄成為元朝脫脫等所編寫的《遼史》主要材料之一。

語言文字方面,漢語與契丹語都是通行的,不少文書都是以這兩種語言寫就。遼代還出現了為佛教信眾學習佛經而編纂的漢字字典《龍龕手鏡》。契丹文是遼代為記錄契丹語而參照漢字創製的文字,分契丹大字和契丹小字兩種形式。但現時已缺少類似的文獻。契丹大字相傳於920年由遼太祖下令耶律突呂不和耶律魯不古參照漢字創制,應有三千餘字;契丹小字由遼太祖弟耶律迭剌參考回鶻文對大字加以改變而成。小字為拼音文字,約五百個發音符號。契丹小字較大字簡便,原字雖少,卻能把契丹語全部貫通。契丹族創字表現出強烈的民族自覺,對其他民族也有不少影響,例如西夏創造党項文字、金朝創造女真文字、元朝創造八思巴文字。契丹字的通行直到1191年金朝金章宗廢除為止。有观点认为契丹人受印章雕刻启发,参照雕版印刷术优缺点发明了契丹大字石活字印刷,比毕昇的泥活字早150年。

遼朝的宗教以佛教和薩滿教為主,此外也崇拜契丹祖先和民間信仰。民族信仰有木葉山崇拜、天地崇拜與拜日神、拜山神等。木葉山崇拜源自契丹始祖出現與契丹八部興起的傳說,帶有薩滿教的文化背景。契丹族於木葉山(今內蒙古西拉木倫河與老哈河合流處)興建契丹祖廟以祭拜始祖,最後發展成遼朝皇室的柴冊儀。

遼朝佛教基本上繼承盛唐的教學佛教。早在唐朝唐武宗發動滅佛事件時因為河北諸藩鎮不聽從,大量僧侶與佛教文物流向河北地區,使得當地佛教文化蓬逢發展。902年龍化州建開教寺,為佛教北傳契丹的起始點。918年遼上京又建佛寺,佛教逐漸為契丹人所信仰和崇尚。926年遼朝滅渤海國後,俘渤海僧人崇文等57人至上京,又建天雄寺。此後,諸京和各州縣也相繼修建寺廟。938年遼朝領有燕雲十六州後,此地逐漸發展成佛教文化重心,到遼朝晚期「僧侶、佛寺之數冠北方」。遼太宗等遼朝皇帝也採取保護佛教政策,尊崇佛教,佛教大盛。遼興宗時覺華島海雲寺僧人海山(郎思孝)與遼興宗關係甚好。遼道宗曾以詩贊譽法均:“行高峰頂松千尺,戒凈天心月一輪。”隨著佛教的傳播,由皇帝下令,寺廟校勘、雕印佛經和個人寫經,集資刻經、印經等活動十分活躍。從山西應縣木塔佛像中發現的丹藏、佛經及佛畫,河北豐潤天寶寺塔發現的佛經,內蒙古巴林右旗釋迦佛舍利塔中發現的佛經,堪稱佛教藝術瑰寶。遼朝完成以《大般若經》為首的主要佛教石刻,於遼興宗時期出版的《契丹大藏經》,其地位僅次於宋朝宋太祖時期開版《蜀版大藏經》,在佛典史上佔有重要地位。

道教和道家思想對契丹人也產生了一定影響。遼初,以各種方式進入草原的漢人中,就有一些道教信仰者。如上京有天長觀,中京有通天觀,一些州城也多有道士和道觀。某些契丹上層和契丹部民也信仰道教。遼聖宗對“道釋二教,皆洞其旨”,其弟耶律隆裕更是個虔誠的道教信徒。某些上層道士同佛教上層一樣受到皇帝的禮遇。遼聖宗曾予道士馮若谷加官太子中允。道教的傳播也帶動了道家經典的研究,遼初道士劉海蟾著有《還丹破迷歌》和《還金篇》,耶律倍譯有《陰符經》,遼聖宗時於闐張文寶曾進《內丹書》,寺公大師的《醉義歌》中也雜有道教思想。

遼朝也有通行伊斯蘭教,主要經由位於西域、已經伊斯蘭化的喀喇汗國東傳而來。996年入仕遼廷的阿拉伯學者納蘇魯丁即在遼南京(今北京)興建牛街禮拜寺。後來的西遼遼帝對伊斯蘭教採取的寬容伏待政策,使伊斯蘭教持續在西域發展。

遼朝繪畫作品具有很高的藝術價值,契丹人善畫草原風光和騎射人物,遼朝湧現出不少卓有成就的畫家,創作了大量優秀的繪畫作品。耶律倍和著名畫家胡瑰、胡虔父子所畫多入北宋內府,被譽為“神品”。耶律倍畫的《射騎圖》、胡環的《出獵圖》、無名氏的《丹楓呦鹿圖》、《秋林群鹿圖》等名畫,均為曠世珍寶。此外比較有名的尚有:耶律防曾兩次使宋,見宋仁宗,“陛辭,僅一見,”即臨摹如真容。蕭瀜據《繪事備考》雲:“好讀書,親翰墨,尤善丹青……”。虞仲文據《圖繪寶鑒》記載他善畫人馬,墨竹學文湖州(文同)。其它還有契丹族耶律題子、秦晉國妃蕭氏,以及漢族陳升、常思言與吳九州等人也皆以善畫稱。

雕塑作品刀法遒勁,栩栩如生。建築藝術主要體現在佛塔和佛寺。山西省靈丘覺山寺西塔院中的覺山寺塔、北京市天寧寺塔、遼寧省遼陽白塔、海城析木城金塔造型美觀,是遼代最流行的密簷塔中的傑出代表作品。天津薊縣獨樂寺的觀音閣兼唐代和宋朝之長處,雄健壯麗。

遼朝用契丹文字刻製的石刻。契丹文石刻用契丹大字和小字刻制,一般分為紀功碑、建廟記、哀冊文、墓誌銘、題記等類。契丹大字石刻如:《遼太祖紀功碑》(殘)、《大遼大橫帳蘭陵郡夫人建靜安寺碑》、《耶律延寧墓誌》、《蕭孝忠墓誌銘》、《故太師銘石記》與《北大王墓誌》等。其中《北大王墓誌》(又作《耶律萬辛墓誌》)是契丹大字石刻中字體最工整的一件,講述耶律萬辛的事蹟,本墓誌使用遼代契丹大字、漢字刻印。由於刻字工整,字數較多,有利於契丹大字的解讀。

遼朝散樂受唐朝和五代後晉影響極深,在此基礎上與契丹族民間藝術相融合,建立起的一種類似宮廷音樂的形式。《遼史》中有記載,其演奏樂器有:觱篥、簫、笛、笙、琵琶、五弦、箜篌、箏、方響、枝鼓、第二鼓、第三鼓、腰鼓、大鼓與拍板等。散樂由12人組成,是一支完整的表演隊伍。樂隊呈兩排,前排第三人下,有一低矮的舞蹈者,隨著節拍翩翩起舞。

遼朝在科學技術方面也取得一些成就。遼朝的醫藥久負盛名,遼朝醫生直魯古撰有《脈訣》與《針灸書》,其中的治療方法至今仍應用在臨床實踐中。當時具有屍體防腐保存的技術,文惟簡所著的《虜廷事實》、《新五代史•四夷附錄》等文獻都記載契丹人用香藥、鹽、礬等保存屍體的方法。1981年在內蒙古察右前旗豪欠營遼墓中發現有保存比較完整的乾屍。

遼朝的天文曆法繼承五代曆法,並略有改進。遼朝原使用後晉馬重元的調元曆,995年行用遼刺史賈俊的大明曆。遼朝皇帝十分重視天象觀測,時人並將天象與政事相連繫。1971年在河北省宣化遼墓發現的彩繪星圖繪有二十八宿、黃道十二宮。1989年在宣化遼墓又發現兩幅星圖,除與前圖略同外,並有十二生肖,均作人形,從中可以得知遼朝天文學已達到很高的水準。

契丹民族的社會與風俗,本不同於漢人。遼朝在統治漢人的燕雲十六州地區,則同於中原;在北方的契丹人,則依舊俗生活;介於混雜地區,就呈現兩種混合型態。契丹人的儀俗很多,如拜日儀、柴冊儀、再生儀、祭山、射鬼箭等。特別的生活方式就是四時的「捺缽」,遼朝皇帝帶領百官的中央政權,隨著一年四時,到各地巡狩,其宮帳的所在地就是「捺缽」。其他還有「頭魚宴」、「頭鵝宴」等生活習慣。契丹飲食文化因地制宜,有蜜餞、果脯等,是用蜜蠟浸漬水果而成,以利保存。清朝東北仍有以歐李(野果)“漬以餳蜜”之俗,今日北京特產果脯,也是與契丹人的“蜜漬山果”“蜜曬山果”之類一脈相承。

在日常生活上,契丹人具有北亞民族傳統,以羊狐皮裘居多。而貴族官吏則以貂裘為主,並且穿絲絹服飾,所配戴的裝飾也比較多。飲酒食肉為普遍現象,居住以帳幕為主,也有居住在宮室。摔跤、擊鞠(踢球)、射柳、射兔節、下圍棋和雙陸等均是遼人的業餘活動。關於節令風俗,遼漢皆有,仍以契丹舊俗為主。例如元旦日,以弱米和白羊髓為餅。正月七日為人日,食煎餅,稱為「薰天餅」。其他尚有中和、上巳、端午、夏至、中元、中秋、重九、冬至等,都是直接或間接從中原傳入的,節日風俗大體相同。然而也有一些節令,名稱雖同,卻保留了契丹固有的風俗和儀式。


%% -*- coding: utf-8 -*-
%% Time-stamp: <Chen Wang: 2019-10-15 16:08:29>

\section{太祖\tiny(916-926)}

辽太祖耶律阿保机(872年-926年9月6日),清輯本《旧五代史》改譯安巴堅,汉名耶律亿,是大契丹國的第一位皇帝(916年3月17日-926年9月6日在位),在位10年。

《辽史·后妃传》记载:“太祖慕汉高皇帝,故耶律氏兼称刘氏;以乙室、拔里比萧相国,遂为萧氏”。《辽史·国语解》记载:“耶律和萧两个姓,以汉字书者曰刘、萧,以契丹字书者曰移喇、石抹”。《金史·国语解》记载:“移喇曰刘,石抹曰萧”。

耶律阿保机的前辈是契丹迭剌部的酋长和军事首领(夷里堇),为耶律撒剌的的长子,母萧岩母斤。耶律是其氏族名。他本人于901年被立为军事首领(夷里堇兼任于越),后不久被选为酋长。他以武力征服契丹附近的地区,掠虏了许多汉人和其他人。907年2月27日他被选为部落联盟的首领,连任九年。任用汉人,采纳他们的建议,决定要将这种三年一次的选举制度改为世袭的制度。為了鞏固統治,史載遼太祖初元,韓廷徽助其正君臣,定名分。廢除三年一次的選汗制度造成諸弟之亂,後來叛亂被平定。

公元915年,耶律阿保机出征室韦得胜回国,但被迫交出汗位,但他在在滦河边建设了一座仿幽州式的汉城。耶律阿保机后伏杀了他的敌人,吞并了契丹的各个部落。916年3月17日,耶律阿保机登基称皇帝,立国号“契丹”,建立“大契丹国”(947年2月24日,辽太宗耶律德光改国号为“大辽”),建年号为神册。此外他还令人建立自己的契丹文。

耶律阿保机建国后继续进攻其周围的民族或政权,渤海国、室韦和奚分别被他消灭。926年9月6日去世于扶余城,终年55岁。

耶律阿保機將其母親、祖母、曾祖母、高祖母家族的姓氏拔里氏、乙室氏賜姓蕭氏。相傳是因為他本人羨慕蕭何輔助劉邦的典故。耶律阿保機的皇后名述律平,其子耶律德光即位後,亦將述律氏賜姓蕭氏。故蕭氏有遼朝后族之稱。阿保機汉名姓刘名亿,長子耶律突欲汉名劉倍。

\subsection{神册}


\begin{longtable}{|>{\centering\scriptsize}m{2em}|>{\centering\scriptsize}m{1.3em}|>{\centering}m{8.8em}|}
  % \caption{秦王政}\
  \toprule
  \SimHei \normalsize 年数 & \SimHei \scriptsize 公元 & \SimHei 大事件 \tabularnewline
  % \midrule
  \endfirsthead
  \toprule
  \SimHei \normalsize 年数 & \SimHei \scriptsize 公元 & \SimHei 大事件 \tabularnewline
  \midrule
  \endhead
  \midrule
  元年 & 916 & \tabularnewline\hline
  二年 & 917 & \tabularnewline\hline
  三年 & 918 & \tabularnewline\hline
  四年 & 919 & \tabularnewline\hline
  五年 & 920 & \tabularnewline\hline
  六年 & 921 & \tabularnewline\hline
  七年 & 922 & \tabularnewline
  \bottomrule
\end{longtable}

\subsection{天赞}

\begin{longtable}{|>{\centering\scriptsize}m{2em}|>{\centering\scriptsize}m{1.3em}|>{\centering}m{8.8em}|}
  % \caption{秦王政}\
  \toprule
  \SimHei \normalsize 年数 & \SimHei \scriptsize 公元 & \SimHei 大事件 \tabularnewline
  % \midrule
  \endfirsthead
  \toprule
  \SimHei \normalsize 年数 & \SimHei \scriptsize 公元 & \SimHei 大事件 \tabularnewline
  \midrule
  \endhead
  \midrule
  元年 & 922 & \tabularnewline\hline
  二年 & 923 & \tabularnewline\hline
  三年 & 924 & \tabularnewline\hline
  四年 & 925 & \tabularnewline\hline
  五年 & 926 & \tabularnewline
  \bottomrule
\end{longtable}

\subsection{天显}

\begin{longtable}{|>{\centering\scriptsize}m{2em}|>{\centering\scriptsize}m{1.3em}|>{\centering}m{8.8em}|}
  % \caption{秦王政}\
  \toprule
  \SimHei \normalsize 年数 & \SimHei \scriptsize 公元 & \SimHei 大事件 \tabularnewline
  % \midrule
  \endfirsthead
  \toprule
  \SimHei \normalsize 年数 & \SimHei \scriptsize 公元 & \SimHei 大事件 \tabularnewline
  \midrule
  \endhead
  \midrule
  元年 & 926 & \tabularnewline
  \bottomrule
\end{longtable}


%%% Local Variables:
%%% mode: latex
%%% TeX-engine: xetex
%%% TeX-master: "../Main"
%%% End:

%% -*- coding: utf-8 -*-
%% Time-stamp: <Chen Wang: 2019-12-26 10:57:29>

\section{太宗\tiny(927-947)}


\subsection{生平}

遼太宗耶律德光(902年11月25日-947年5月15日),大契丹國第二位皇帝(927年12月11日至947年5月15日在位),在位20年。字德谨,契丹名耶律尧骨,辽太祖耶律阿保机次子。947年2月24日,辽太宗耶律德光将国号由“大契丹国”改为“大辽”,成为遼朝首位皇帝。

辽太祖天赞元年(922年),被任命为天下兵马大元帅,随同太祖参加了一系列征服战争,尤其是在南征幽州、西征吐谷浑、回鹘期间,战功卓著。天显元年(926年),又随同太祖灭渤海国,作为前锋攻克渤海首都忽汗城。

天显元年七月二十七日(926年9月6日)辽太祖死后,述律后称制,耶律德光总揽朝政,927年12月11日,在述律后的支持下即位。天显六年(930年),割据原渤海国疆域的东丹王耶律倍南逃后唐,耶律德光统一了契丹。

天显十一年(936年),后唐河东节度使石敬瑭以称子、割让燕云十六州为条件,乞求耶律德光出兵助其反对后唐。耶律德光遂亲率5万骑兵,在晋阳城下击败后唐军,册立石敬塘为后晋皇帝。其后,更率军南下上党,助石敬塘灭后唐。

割取燕云十六州后,耶律德光采取“因俗而治”的统治方式,实行南北两面官制度,分治汉人和契丹。又改幽州为南京、云州为西京,将燕云十六州建设成为进一步南下的基地。

会同四年(944年),后晋出帝石重贵即位,拒不称臣。耶律德光于是率军南下。会同九年十二月十六日(947年1月10日),耶律德光率军攻入后晋首都东京汴梁(今河南开封),俘虏后晋出帝石重贵,灭后晋。

会同十年正月初一(947年1月25日),耶律德光以中原皇帝的仪仗进入东京汴梁,在崇元殿接受百官朝贺。大同元年二月初一(947年2月24日),耶律德光在东京皇宫下诏将国号“大契丹国”改为“大辽”,改会同十年为大同元年,升镇州为中京。

大同元年四月初一(947年4月24日),因遼人實施的「打草穀」物資掠奪政策導致中原反抗不断,无法巩固统治,耶律德光被迫离开东京汴梁,引军北返,在临城县(今河北省临城县)得熱疾。四月二十二日(947年5月15日),在栾城县殺胡林(今河北石家庄市欒城区西北)病逝,遼人将耶律德光的尸体破腹,取出内脏,装入几斗盐,带回北方,時人稱為「帝羓」,即「皇帝醃肉」之意。

\subsection{天显}

\begin{longtable}{|>{\centering\scriptsize}m{2em}|>{\centering\scriptsize}m{1.3em}|>{\centering}m{8.8em}|}
  % \caption{秦王政}\
  \toprule
  \SimHei \normalsize 年数 & \SimHei \scriptsize 公元 & \SimHei 大事件 \tabularnewline
  % \midrule
  \endfirsthead
  \toprule
  \SimHei \normalsize 年数 & \SimHei \scriptsize 公元 & \SimHei 大事件 \tabularnewline
  \midrule
  \endhead
  \midrule
  二年 & 927 & \tabularnewline\hline
  三年 & 928 & \tabularnewline\hline
  四年 & 929 & \tabularnewline\hline
  五年 & 930 & \tabularnewline\hline
  六年 & 931 & \tabularnewline\hline
  七年 & 932 & \tabularnewline\hline
  八年 & 933 & \tabularnewline\hline
  九年 & 934 & \tabularnewline\hline
  十年 & 935 & \tabularnewline\hline
  十一年 & 936 & \tabularnewline\hline
  十二年 & 937 & \tabularnewline\hline
  十三年 & 938 & \tabularnewline
  \bottomrule
\end{longtable}


\subsection{会同}


\begin{longtable}{|>{\centering\scriptsize}m{2em}|>{\centering\scriptsize}m{1.3em}|>{\centering}m{8.8em}|}
  % \caption{秦王政}\
  \toprule
  \SimHei \normalsize 年数 & \SimHei \scriptsize 公元 & \SimHei 大事件 \tabularnewline
  % \midrule
  \endfirsthead
  \toprule
  \SimHei \normalsize 年数 & \SimHei \scriptsize 公元 & \SimHei 大事件 \tabularnewline
  \midrule
  \endhead
  \midrule
  元年 & 938 & \tabularnewline\hline
  二年 & 939 & \tabularnewline\hline
  三年 & 940 & \tabularnewline\hline
  四年 & 941 & \tabularnewline\hline
  五年 & 942 & \tabularnewline\hline
  六年 & 943 & \tabularnewline\hline
  七年 & 944 & \tabularnewline\hline
  八年 & 945 & \tabularnewline\hline
  九年 & 946 & \tabularnewline\hline
  十年 & 947 & \tabularnewline
  \bottomrule
\end{longtable}

\subsection{大同}

\begin{longtable}{|>{\centering\scriptsize}m{2em}|>{\centering\scriptsize}m{1.3em}|>{\centering}m{8.8em}|}
  % \caption{秦王政}\
  \toprule
  \SimHei \normalsize 年数 & \SimHei \scriptsize 公元 & \SimHei 大事件 \tabularnewline
  % \midrule
  \endfirsthead
  \toprule
  \SimHei \normalsize 年数 & \SimHei \scriptsize 公元 & \SimHei 大事件 \tabularnewline
  \midrule
  \endhead
  \midrule
  元年 & 947 & \tabularnewline
  \bottomrule
\end{longtable}


%%% Local Variables:
%%% mode: latex
%%% TeX-engine: xetex
%%% TeX-master: "../Main"
%%% End:

%% -*- coding: utf-8 -*-
%% Time-stamp: <Chen Wang: 2019-10-15 16:18:25>

\section{世宗\tiny(947-951)}

遼世宗耶律阮(919年1月29日-951年10月7日),中国遼朝第三位皇帝(947年5月16日-951年10月7日在位),在位4年。契丹迭剌部霞濑益石烈乡耶律里(今中国内蒙古阿鲁科尔沁旗东)人,姓耶律,汉文名阮,契丹文名兀欲(又名隈欲、烏雲),他是大契丹国(後改称大辽国)皇太子、人皇王、東丹国王、遼義宗让国皇帝(追尊,未即位)耶律倍的長子、太祖耶律阿保機的长孙、太宗耶律德光之侄。

阿保机死后,世宗耶律阮之父人皇王耶律倍在权力斗争中失利,未能即位为帝,耶律阮遂失去继承皇位的权利。人皇王後来愤而投奔后唐,终于客死他乡。耶律阮则留在国内,后随叔父太宗耶律德光南征後晋。太宗在北归途中病逝后,耶律阮被随军将领拥立为帝,是为辽世宗。但世宗即位后发生多起夺权事变,统治活动被严重干扰,最终遇刺身亡,在位仅四年有余,其堂弟耶律璟继位,是为辽穆宗。

世宗虽在辽代诸帝中享国最短,却是一位有作为的皇帝。受其父耶律倍的影响,世宗在位期间推崇汉文化,推广中原制度,在世宗之孙圣宗时最后完成,促进了辽国社会的发展。

契丹神册三年,耶律阮出生,他的父亲是契丹开国皇帝耶律阿保机的长子耶律倍,母亲是耶律倍之妃萧氏(死后追谥柔贞皇后)。祖父阿保机死后,父亲耶律倍在权力斗争中失利,不得立为皇帝,耶律阮也就失去了继承皇位的权利。耶律倍后来愤而渡海投奔后唐,终于於932年末被後唐末帝李從珂殺害,客死他乡。耶律阮则留在契丹国内,其叔父太宗耶律德光爱之如己出。契丹会同九年、后晋开运三年(946年),太宗以后晋皇帝石重贵不肯称臣为由大举入侵中原,耶律阮随行军中。第二年(947年)契丹军入后晋国都东京开封府(今河南省开封市),晋帝石重贵投降,后晋灭亡。太宗改国号为“大辽”,改元大同,封耶律阮为永康王。

太宗滅後晉后在北归途中逝世,耶律阮發兵奪取南京析津府(今北京),並在随军将领拥戴下自立為皇帝,在上京(今內蒙古巴林左旗)的蕭太后述律平派其子耶律李胡在南京北部的泰德泉交戰,大敗。經過大臣耶律屋質的勸阻,太后才同意耶律阮當皇帝。世宗時任用賢臣耶律屋質,進行一系列改革,將太宗時的南面官和北面官合併,成立南北樞密院,廢南、北大王,後來南北樞密院合併,形成一個樞密院。這些改革使遼朝從部落聯盟形式進入中央集權,這些都是與遼世宗的改革分不開的。但是世宗好酒色,喜愛打獵。他晚年更是任用奸佞,大興封賞降殺,導致朝政不修,政治腐敗。遼天祿五年(951年)9月,世宗協助北漢攻後周,行軍至歸化(今內蒙古呼和浩特)的祥古山,由於其他部隊未到,所以駐紮在火神澱。其間喝酒、打人、打獵,眾將很是不滿。晚上,一直有篡位之心的耶律察割將遼世宗耶律阮殺死於夢鄉。耶律阮死時年僅34歲,在位4年。其諡號為孝和莊憲皇帝,廟號世宗。

\subsection{天禄}

\begin{longtable}{|>{\centering\scriptsize}m{2em}|>{\centering\scriptsize}m{1.3em}|>{\centering}m{8.8em}|}
  % \caption{秦王政}\
  \toprule
  \SimHei \normalsize 年数 & \SimHei \scriptsize 公元 & \SimHei 大事件 \tabularnewline
  % \midrule
  \endfirsthead
  \toprule
  \SimHei \normalsize 年数 & \SimHei \scriptsize 公元 & \SimHei 大事件 \tabularnewline
  \midrule
  \endhead
  \midrule
  元年 & 947 & \tabularnewline\hline
  二年 & 948 & \tabularnewline\hline
  三年 & 949 & \tabularnewline\hline
  四年 & 950 & \tabularnewline\hline
  五年 & 951 & \tabularnewline
  \bottomrule
\end{longtable}



%%% Local Variables:
%%% mode: latex
%%% TeX-engine: xetex
%%% TeX-master: "../Main"
%%% End:

%% -*- coding: utf-8 -*-
%% Time-stamp: <Chen Wang: 2021-11-01 16:01:26>

\section{穆宗耶律璟\tiny(951-969)}

\subsection{生平}

遼穆宗耶律璟(931年9月19日-969年3月12日),一说名耶律明,小字述律,遼朝第四位皇帝(951年10月11日-969年3月12日在位),在位18年,是為遼太宗之長子,其母為靖安皇后萧温。

於會同二年(939年)三月被封為壽安王。妻子萧氏。於天祿五年九月初八日(951年10月11日)火神淀之乱后,被立為帝,尊稱天順皇帝,改年号應曆。

遼穆宗雖討厭女色,而無所出,但卻經常酗酒,天亮才睡,中午方醒,因此長時期不理朝政,人稱之為「睡王」。另外,穆宗又好殺,經常親手殺人。同時,他又愛好打獵而「竟月不視朝」。

不過,遼穆宗也曾有才華之士可破格提拔,年老或是無能官員可增俸歸鄉,以免在其位而不謀其政的做法。

應曆十九年二月廿二日(969年3月12日),遼穆宗被侍人所弒,享年三十九歲,死後遼景宗繼位。

元朝官修正史《辽史》脱脱等的評價是:“穆宗在位十八年,知女巫妖妄见诛,谕臣下滥刑切谏,非不明也。而荒耽于酒,畋猎无厌。侦鹅失期,加炮烙铁梳之刑;获鸭甚欢,除鹰坊刺面之令。赏罚无章,朝政不视,而嗜杀不已。变起肘腋,宜哉!”

\subsection{应历}

\begin{longtable}{|>{\centering\scriptsize}m{2em}|>{\centering\scriptsize}m{1.3em}|>{\centering}m{8.8em}|}
  % \caption{秦王政}\
  \toprule
  \SimHei \normalsize 年数 & \SimHei \scriptsize 公元 & \SimHei 大事件 \tabularnewline
  % \midrule
  \endfirsthead
  \toprule
  \SimHei \normalsize 年数 & \SimHei \scriptsize 公元 & \SimHei 大事件 \tabularnewline
  \midrule
  \endhead
  \midrule
  元年 & 951 & \tabularnewline\hline
  二年 & 952 & \tabularnewline\hline
  三年 & 953 & \tabularnewline\hline
  四年 & 954 & \tabularnewline\hline
  五年 & 955 & \tabularnewline\hline
  六年 & 956 & \tabularnewline\hline
  七年 & 957 & \tabularnewline\hline
  八年 & 958 & \tabularnewline\hline
  九年 & 959 & \tabularnewline\hline
  十年 & 960 & \tabularnewline\hline
  十一年 & 961 & \tabularnewline\hline
  十二年 & 962 & \tabularnewline\hline
  十三年 & 963 & \tabularnewline\hline
  十四年 & 964 & \tabularnewline\hline
  十五年 & 965 & \tabularnewline\hline
  十六年 & 966 & \tabularnewline\hline
  十七年 & 967 & \tabularnewline\hline
  十八年 & 968 & \tabularnewline\hline
  十九年 & 969 & \tabularnewline
  \bottomrule
\end{longtable}



%%% Local Variables:
%%% mode: latex
%%% TeX-engine: xetex
%%% TeX-master: "../Main"
%%% End:

%% -*- coding: utf-8 -*-
%% Time-stamp: <Chen Wang: 2021-11-01 16:01:53>

\section{景宗耶律賢\tiny(969-982)}

\subsection{生平}

遼景宗耶律賢(948年9月1日-982年10月13日),字賢寧,遼朝第五位皇帝(969年3月13日-982年10月13日在位),在位13年,遼世宗的次子,其母為懷節皇后蕭氏。

在遼世宗在位時的政變中,耶律賢險而被殺,後來得人所救。951年,其父辽世宗被刺身亡,堂叔耶律璟即位,是为辽穆宗。969年3月12日,堂叔遼穆宗被弑,次日,耶律賢被推舉為帝,尊號天贊皇帝,改元為保寧。

耶律賢从小惊吓过度,体弱多病,皇后萧绰(953年-1009年,小字燕燕,原姓拔黎氏)则成了辽国政治军事的参与者。景宗在位時復回登聞鼓院,令百姓有申冤之地,又寬減刑法,對百姓加以安撫。

後來,景宗於乾亨四年九月廿四日(即982年10月13日)死於現今的山西省大同市,享年三十五歲,葬於乾陵,位于今辽宁省北镇市。

元朝官修正史《辽史》脱脱等的評價是:“辽兴六十馀年,神册、会同之间,日不暇给;天禄、应历之君,不令其终;保宁而来,人人望治。以景宗之资,任人不疑,信赏必罚,若可与有为也。而竭国之力以助河东,破军杀将,无救灭亡。虽一取偿于宋,得不偿失。知匡嗣之罪,数而不罚;善郭袭之谏,纳而不用;沙门昭敏以左道乱德,宠以侍中。不亦惑乎!”

\subsection{保宁}

\begin{longtable}{|>{\centering\scriptsize}m{2em}|>{\centering\scriptsize}m{1.3em}|>{\centering}m{8.8em}|}
  % \caption{秦王政}\
  \toprule
  \SimHei \normalsize 年数 & \SimHei \scriptsize 公元 & \SimHei 大事件 \tabularnewline
  % \midrule
  \endfirsthead
  \toprule
  \SimHei \normalsize 年数 & \SimHei \scriptsize 公元 & \SimHei 大事件 \tabularnewline
  \midrule
  \endhead
  \midrule
  元年 & 969 & \tabularnewline\hline
  二年 & 970 & \tabularnewline\hline
  三年 & 971 & \tabularnewline\hline
  四年 & 972 & \tabularnewline\hline
  五年 & 973 & \tabularnewline\hline
  六年 & 974 & \tabularnewline\hline
  七年 & 975 & \tabularnewline\hline
  八年 & 976 & \tabularnewline\hline
  九年 & 977 & \tabularnewline\hline
  十年 & 978 & \tabularnewline\hline
  十一年 & 979 & \tabularnewline
  \bottomrule
\end{longtable}

\subsection{乾亨}

\begin{longtable}{|>{\centering\scriptsize}m{2em}|>{\centering\scriptsize}m{1.3em}|>{\centering}m{8.8em}|}
  % \caption{秦王政}\
  \toprule
  \SimHei \normalsize 年数 & \SimHei \scriptsize 公元 & \SimHei 大事件 \tabularnewline
  % \midrule
  \endfirsthead
  \toprule
  \SimHei \normalsize 年数 & \SimHei \scriptsize 公元 & \SimHei 大事件 \tabularnewline
  \midrule
  \endhead
  \midrule
  元年 & 979 & \tabularnewline\hline
  二年 & 980 & \tabularnewline\hline
  三年 & 981 & \tabularnewline\hline
  四年 & 982 & \tabularnewline\hline
  五年 & 983 & \tabularnewline
  \bottomrule
\end{longtable}



%%% Local Variables:
%%% mode: latex
%%% TeX-engine: xetex
%%% TeX-master: "../Main"
%%% End:

%% -*- coding: utf-8 -*-
%% Time-stamp: <Chen Wang: 2021-11-01 16:02:03>

\section{圣宗耶律隆緒\tiny(982-1031)}

\subsection{生平}

遼聖宗耶律隆緒(972年1月16日-1031年6月25日),遼朝第六位皇帝(982年10月14日-1031年6月25日在位),契丹名文殊奴。是遼在位最長的皇帝,在位49年。遼景宗長子,母皇后萧绰。

辽圣宗即位前曾被封為梁王。乾亨四年(982年)九月壬子(10月13日),遼景宗去世,次日,耶律隆绪登基,即辽圣宗。

他即位時,年12歲,太后蕭綽執政。983年改元統和,并将国号“大辽”改为“大契丹”。统和四年(986年),立皇后萧氏。蕭太后執政期間,進行了改革,並且勵精圖治,注重農桑,興修水利,減少賦稅,整頓吏治,訓練軍隊,使百姓富裕,國勢強盛。統和二十二年(1004年)遼聖宗与宋真宗達成澶淵之盟。

統和二十七年(1009年)聖宗全面親政後,遼朝(契丹)已進入鼎盛,基本上延續蕭太后執政時的遼朝風貌,並且還反對嚴刑峻法,不給貪官可乘之機。在位其間四方征戰,進入遼朝疆域的頂峰。

晚年时,辽圣宗迷信佛教,窮途奢侈,遼國勢走向下坡路。遼聖宗死於太平十一年六月初三日(1031年6月25日),終年61歲,葬於庆云山。謚號為文武大孝宣肅景皇帝。

元朝官修正史《辽史》脱脱等的評價是:“圣宗幼冲嗣位,政出慈闱。及宋人二道来攻,亲御甲胄,一举而复燕、云,破信、彬,再举而躏河、朔,不亦伟欤!既而侈心一启,佳兵不祥,东有茶、陀之败,西有甘州之丧,此狃于常胜之过也。然其践阼四十九年,理冤滞,举才行,察贪残,抑奢僣,录死事之子孙,振诸部之贫乏,责迎合不忠之罪,却高丽女乐之归。辽之诸帝,在位长久,令名无穷,其唯圣宗乎!”

\subsection{统合}

\begin{longtable}{|>{\centering\scriptsize}m{2em}|>{\centering\scriptsize}m{1.3em}|>{\centering}m{8.8em}|}
  % \caption{秦王政}\
  \toprule
  \SimHei \normalsize 年数 & \SimHei \scriptsize 公元 & \SimHei 大事件 \tabularnewline
  % \midrule
  \endfirsthead
  \toprule
  \SimHei \normalsize 年数 & \SimHei \scriptsize 公元 & \SimHei 大事件 \tabularnewline
  \midrule
  \endhead
  \midrule
  元年 & 983 & \tabularnewline\hline
  二年 & 984 & \tabularnewline\hline
  三年 & 985 & \tabularnewline\hline
  四年 & 986 & \tabularnewline\hline
  五年 & 987 & \tabularnewline\hline
  六年 & 988 & \tabularnewline\hline
  七年 & 989 & \tabularnewline\hline
  八年 & 990 & \tabularnewline\hline
  九年 & 991 & \tabularnewline\hline
  十年 & 992 & \tabularnewline\hline
  十一年 & 993 & \tabularnewline\hline
  十二年 & 994 & \tabularnewline\hline
  十三年 & 995 & \tabularnewline\hline
  十四年 & 996 & \tabularnewline\hline
  十五年 & 997 & \tabularnewline\hline
  十六年 & 998 & \tabularnewline\hline
  十七年 & 999 & \tabularnewline\hline
  十八年 & 1000 & \tabularnewline\hline
  十九年 & 1001 & \tabularnewline\hline
  二十年 & 1002 & \tabularnewline\hline
  二一年 & 1003 & \tabularnewline\hline
  二二年 & 1004 & \tabularnewline\hline
  二三年 & 1005 & \tabularnewline\hline
  二四年 & 1006 & \tabularnewline\hline
  二五年 & 1007 & \tabularnewline\hline
  二六年 & 1008 & \tabularnewline\hline
  二七年 & 1009 & \tabularnewline\hline
  二八年 & 1010 & \tabularnewline\hline
  二九年 & 1011 & \tabularnewline\hline
  三十年 & 1012 & \tabularnewline
  \bottomrule
\end{longtable}

\subsection{开泰}

\begin{longtable}{|>{\centering\scriptsize}m{2em}|>{\centering\scriptsize}m{1.3em}|>{\centering}m{8.8em}|}
  % \caption{秦王政}\
  \toprule
  \SimHei \normalsize 年数 & \SimHei \scriptsize 公元 & \SimHei 大事件 \tabularnewline
  % \midrule
  \endfirsthead
  \toprule
  \SimHei \normalsize 年数 & \SimHei \scriptsize 公元 & \SimHei 大事件 \tabularnewline
  \midrule
  \endhead
  \midrule
  元年 & 1012 & \tabularnewline\hline
  二年 & 1013 & \tabularnewline\hline
  三年 & 1014 & \tabularnewline\hline
  四年 & 1015 & \tabularnewline\hline
  五年 & 1016 & \tabularnewline\hline
  六年 & 1017 & \tabularnewline\hline
  七年 & 1018 & \tabularnewline\hline
  八年 & 1019 & \tabularnewline\hline
  九年 & 1020 & \tabularnewline\hline
  十年 & 1021 & \tabularnewline
  \bottomrule
\end{longtable}

\subsection{太平}

\begin{longtable}{|>{\centering\scriptsize}m{2em}|>{\centering\scriptsize}m{1.3em}|>{\centering}m{8.8em}|}
  % \caption{秦王政}\
  \toprule
  \SimHei \normalsize 年数 & \SimHei \scriptsize 公元 & \SimHei 大事件 \tabularnewline
  % \midrule
  \endfirsthead
  \toprule
  \SimHei \normalsize 年数 & \SimHei \scriptsize 公元 & \SimHei 大事件 \tabularnewline
  \midrule
  \endhead
  \midrule
  元年 & 1021 & \tabularnewline\hline
  二年 & 1022 & \tabularnewline\hline
  三年 & 1023 & \tabularnewline\hline
  四年 & 1024 & \tabularnewline\hline
  五年 & 1025 & \tabularnewline\hline
  六年 & 1026 & \tabularnewline\hline
  七年 & 1027 & \tabularnewline\hline
  八年 & 1028 & \tabularnewline\hline
  九年 & 1029 & \tabularnewline\hline
  十年 & 1030 & \tabularnewline\hline
  十一年 & 1031 & \tabularnewline
  \bottomrule
\end{longtable}



%%% Local Variables:
%%% mode: latex
%%% TeX-engine: xetex
%%% TeX-master: "../Main"
%%% End:

%% -*- coding: utf-8 -*-
%% Time-stamp: <Chen Wang: 2021-11-01 16:02:08>

\section{兴宗耶律宗真\tiny(1031-1055)}

\subsection{生平}

遼興宗耶律宗真(1016年4月3日-1055年8月28日),契丹第七位皇帝(1031年6月25日-1055年8月28日在位),契丹名只骨。在位24年,享年40歲,謚孝章皇帝。他是遼聖宗的長子,母乃宮女蕭耨斤。

耶律宗真生于1016年,其后,由辽圣宗的皇后蕭菩薩哥抚养。《辽史》记耶律宗真为“圣宗长子”,实际上辽圣宗第六子耶律宗愿的生年是1008年或1009年,耶律宗真和同母弟耶律宗元应该是辽圣宗最年幼的两个儿子。虽然年幼,但与其他四位皇子相比,只有耶律宗真兄弟的生母蕭耨斤出身于契丹萧氏,其他皇子的生母出身于汉族或不详。

太平元年(1021年),耶律宗真被冊立為太子,太平十年(1030年)六月判北南院枢密院事。太平十一年(1031年6月25日)夏六月己卯,辽圣宗逝世,同时,耶律宗真繼承皇位,改元景福。興宗繼位後,其母順聖元妃蕭耨斤自立為皇太后攝政,並把聖宗的齊天皇后迫死。並重用了在聖宗時代被裁示永不錄用的貪官污吏以及其娘家的人。

景福二年十一月,興宗上太后尊號為法天應運仁德章聖皇太后(法天太后),而興宗被群臣上尊號為文武仁聖昭孝皇帝,改元重熙。

重熙三年(1034年),法天太后企圖廢掉興宗,改立次子宗元(遼史作重元),重元告訴其兄興宗,興宗發動政變,迫法天太后「躬守慶陵」。大殺太后親信。七月,興宗親政。

興宗在位時,遼國勢已日益衰落。而有興宗一朝,奸佞當權,政治腐敗,百姓困苦,軍隊衰弱。面對日益衰落的國勢,興宗連年征戰,多次征伐西夏;逼迫宋朝多交納歲幣,反而使遼內部百姓怨聲載道,民不聊生。興宗還迷信佛教,窮途奢極。興宗曾與其弟宗元賭博,一連輸了幾個城池。

他對自己的弟弟宗元非常感激,一次酒醉時答應百年之後傳位給宗元,其子耶律洪基(後來的遼道宗)也未曾封為皇太子,只封為天下兵馬大元帥而已。種下了道宗繼位後,宗元父子企圖謀奪帝位的惡果。

重熙二十四年八月初四日(1055年8月28日),興宗駕崩。

元朝官修正史《辽史》脱脱等的評價是:“兴宗即位,年十有六矣,不能先尊母后而尊其母,以致临朝专政,贼杀不辜,又不能以礼几谏,使齐天死于弑逆,有亏王者之孝,惜哉!若夫大行在殡,饮酒博鞠,叠见简书。及其谒遗像而哀恸,受宋吊而衰绖,所为若出二人。何为其然欤?至于感富弼之言而申南宋之好,许谅祚之盟而罢西夏之兵,边鄙不耸,政治内修,亲策进士,大修条制,下至士庶,得陈便宜,则求治之志切矣。于时左右大臣,曾不闻一贤之进,一事之谏,欲庶几古帝王之风,其可得乎?虽然,圣宗而下,可谓贤君矣。 ”

\subsection{景福}

\begin{longtable}{|>{\centering\scriptsize}m{2em}|>{\centering\scriptsize}m{1.3em}|>{\centering}m{8.8em}|}
  % \caption{秦王政}\
  \toprule
  \SimHei \normalsize 年数 & \SimHei \scriptsize 公元 & \SimHei 大事件 \tabularnewline
  % \midrule
  \endfirsthead
  \toprule
  \SimHei \normalsize 年数 & \SimHei \scriptsize 公元 & \SimHei 大事件 \tabularnewline
  \midrule
  \endhead
  \midrule
  元年 & 1031 & \tabularnewline\hline
  二年 & 1032 & \tabularnewline
  \bottomrule
\end{longtable}

\subsection{重熙}

\begin{longtable}{|>{\centering\scriptsize}m{2em}|>{\centering\scriptsize}m{1.3em}|>{\centering}m{8.8em}|}
  % \caption{秦王政}\
  \toprule
  \SimHei \normalsize 年数 & \SimHei \scriptsize 公元 & \SimHei 大事件 \tabularnewline
  % \midrule
  \endfirsthead
  \toprule
  \SimHei \normalsize 年数 & \SimHei \scriptsize 公元 & \SimHei 大事件 \tabularnewline
  \midrule
  \endhead
  \midrule
  元年 & 1032 & \tabularnewline\hline
  二年 & 1033 & \tabularnewline\hline
  三年 & 1034 & \tabularnewline\hline
  四年 & 1035 & \tabularnewline\hline
  五年 & 1036 & \tabularnewline\hline
  六年 & 1037 & \tabularnewline\hline
  七年 & 1038 & \tabularnewline\hline
  八年 & 1039 & \tabularnewline\hline
  九年 & 1040 & \tabularnewline\hline
  十年 & 1041 & \tabularnewline\hline
  十一年 & 1042 & \tabularnewline\hline
  十二年 & 1043 & \tabularnewline\hline
  十三年 & 1044 & \tabularnewline\hline
  十四年 & 1045 & \tabularnewline\hline
  十五年 & 1046 & \tabularnewline\hline
  十六年 & 1047 & \tabularnewline\hline
  十七年 & 1048 & \tabularnewline\hline
  十八年 & 1049 & \tabularnewline\hline
  十九年 & 1050 & \tabularnewline\hline
  二十年 & 1051 & \tabularnewline\hline
  二一年 & 1052 & \tabularnewline\hline
  二二年 & 1053 & \tabularnewline\hline
  二三年 & 1054 & \tabularnewline\hline
  二四年 & 1055 & \tabularnewline
  \bottomrule
\end{longtable}


%%% Local Variables:
%%% mode: latex
%%% TeX-engine: xetex
%%% TeX-master: "../Main"
%%% End:

%% -*- coding: utf-8 -*-
%% Time-stamp: <Chen Wang: 2019-12-26 10:58:15>

\section{道宗\tiny(1055-1101)}

\subsection{生平}

遼道宗耶律洪基(1032年9月14日-1101年2月12日),契丹及遼朝第八位皇帝(1055年8月28日-1101年2月12日在位),在位長達46年,僅次於遼聖宗。他是遼興宗的長子,契丹名查剌。

重熙二十四年八月初四(1055年8月28日),興宗駕崩,即位於柩前。改元清寧。

道宗繼位後,封皇叔宗元為皇太叔,清寧二年又加天下兵馬大元帥。四年又賜金券等,極盡榮寵。但宗元始終有謀奪帝位的意圖,在清寧九年(1063年)七月,宗元聽從兒子的勸說,發動叛亂,自立為帝,未幾被道宗所平,宗元自盡。史稱灤河之亂。

咸雍二年(1066年),辽道宗把国号“契丹”改为“大辽”。

他在位期間,遼政治腐敗,國勢逐漸衰落。道宗並沒有進行改革圖新,而且本人也腐朽奢侈,這時地主官僚急劇兼併土地,百姓痛苦不堪,怨聲載道。道宗還重用耶律乙辛等奸佞,自己不理朝政,導致他聽信乙辛的讒言,相信皇后蕭觀音與伶官趙惟一通姦而賜死皇后,史稱十香詞冤案。而同時乙辛為防太子耶律濬登基對自己不利(因為道宗只有皇太子這個兒子),故陷害皇太子謀反,殺害了皇太子。

後來,一位姓李的婦女向道宗進「挾穀歌」,道宗才把皇太子的兒女接進宮,大康五年(1079年)七月,耶律乙辛乘道宗遊獵的時候謀害皇孫,道宗接納大臣的勸諫,命皇孫一同秋獵,才化解乙辛的陰謀。

大康九年,道宗追封故太子為昭懷太子,以天子禮改葬。同年十月,耶律乙辛企圖帶私藏武器到宋朝避難,事發,被誅。

道宗篤信佛教,在位期間曾大修佛寺、佛塔。遼的腐朽統治引起了各族人民的不滿,其間被遼統治者壓迫的女真族開始興起,最終成為遼的掘墓人。

寿昌七年正月十三日(1101年2月12日),遼道宗去世,終年70歲。

元朝官修正史《辽史》脱脱等的評價是:“道宗初即位,求直言,访治道,劝农兴学,救灾恤患,粲然可观。及夫谤讪之令既行,告讦之赏日重。群邪并兴,谗巧竞进。贼及骨肉,皇基浸危。众正沦胥,诸部反侧,甲兵之用,无宁岁矣。一岁而饭僧三十六万,一日而祝发三千。徒勤小惠,蔑计大本,尚足与论治哉? ”

\subsection{清宁}

\begin{longtable}{|>{\centering\scriptsize}m{2em}|>{\centering\scriptsize}m{1.3em}|>{\centering}m{8.8em}|}
  % \caption{秦王政}\
  \toprule
  \SimHei \normalsize 年数 & \SimHei \scriptsize 公元 & \SimHei 大事件 \tabularnewline
  % \midrule
  \endfirsthead
  \toprule
  \SimHei \normalsize 年数 & \SimHei \scriptsize 公元 & \SimHei 大事件 \tabularnewline
  \midrule
  \endhead
  \midrule
  元年 & 1055 & \tabularnewline\hline
  二年 & 1056 & \tabularnewline\hline
  三年 & 1057 & \tabularnewline\hline
  四年 & 1058 & \tabularnewline\hline
  五年 & 1059 & \tabularnewline\hline
  六年 & 1060 & \tabularnewline\hline
  七年 & 1061 & \tabularnewline\hline
  八年 & 1062 & \tabularnewline\hline
  九年 & 1063 & \tabularnewline\hline
  十年 & 1064 & \tabularnewline
  \bottomrule
\end{longtable}

\subsection{咸雍}

\begin{longtable}{|>{\centering\scriptsize}m{2em}|>{\centering\scriptsize}m{1.3em}|>{\centering}m{8.8em}|}
  % \caption{秦王政}\
  \toprule
  \SimHei \normalsize 年数 & \SimHei \scriptsize 公元 & \SimHei 大事件 \tabularnewline
  % \midrule
  \endfirsthead
  \toprule
  \SimHei \normalsize 年数 & \SimHei \scriptsize 公元 & \SimHei 大事件 \tabularnewline
  \midrule
  \endhead
  \midrule
  元年 & 1065 & \tabularnewline\hline
  二年 & 1066 & \tabularnewline\hline
  三年 & 1067 & \tabularnewline\hline
  四年 & 1068 & \tabularnewline\hline
  五年 & 1069 & \tabularnewline\hline
  六年 & 1070 & \tabularnewline\hline
  七年 & 1071 & \tabularnewline\hline
  八年 & 1072 & \tabularnewline\hline
  九年 & 1073 & \tabularnewline\hline
  十年 & 1074 & \tabularnewline
  \bottomrule
\end{longtable}

\subsection{大康}

\begin{longtable}{|>{\centering\scriptsize}m{2em}|>{\centering\scriptsize}m{1.3em}|>{\centering}m{8.8em}|}
  % \caption{秦王政}\
  \toprule
  \SimHei \normalsize 年数 & \SimHei \scriptsize 公元 & \SimHei 大事件 \tabularnewline
  % \midrule
  \endfirsthead
  \toprule
  \SimHei \normalsize 年数 & \SimHei \scriptsize 公元 & \SimHei 大事件 \tabularnewline
  \midrule
  \endhead
  \midrule
  元年 & 1075 & \tabularnewline\hline
  二年 & 1076 & \tabularnewline\hline
  三年 & 1077 & \tabularnewline\hline
  四年 & 1078 & \tabularnewline\hline
  五年 & 1079 & \tabularnewline\hline
  六年 & 1080 & \tabularnewline\hline
  七年 & 1081 & \tabularnewline\hline
  八年 & 1082 & \tabularnewline\hline
  九年 & 1083 & \tabularnewline\hline
  十年 & 1084 & \tabularnewline
  \bottomrule
\end{longtable}

\subsection{大安}

\begin{longtable}{|>{\centering\scriptsize}m{2em}|>{\centering\scriptsize}m{1.3em}|>{\centering}m{8.8em}|}
  % \caption{秦王政}\
  \toprule
  \SimHei \normalsize 年数 & \SimHei \scriptsize 公元 & \SimHei 大事件 \tabularnewline
  % \midrule
  \endfirsthead
  \toprule
  \SimHei \normalsize 年数 & \SimHei \scriptsize 公元 & \SimHei 大事件 \tabularnewline
  \midrule
  \endhead
  \midrule
  元年 & 1085 & \tabularnewline\hline
  二年 & 1086 & \tabularnewline\hline
  三年 & 1087 & \tabularnewline\hline
  四年 & 1088 & \tabularnewline\hline
  五年 & 1089 & \tabularnewline\hline
  六年 & 1090 & \tabularnewline\hline
  七年 & 1091 & \tabularnewline\hline
  八年 & 1092 & \tabularnewline\hline
  九年 & 1093 & \tabularnewline\hline
  十年 & 1094 & \tabularnewline
  \bottomrule
\end{longtable}

\subsection{寿昌}

\begin{longtable}{|>{\centering\scriptsize}m{2em}|>{\centering\scriptsize}m{1.3em}|>{\centering}m{8.8em}|}
  % \caption{秦王政}\
  \toprule
  \SimHei \normalsize 年数 & \SimHei \scriptsize 公元 & \SimHei 大事件 \tabularnewline
  % \midrule
  \endfirsthead
  \toprule
  \SimHei \normalsize 年数 & \SimHei \scriptsize 公元 & \SimHei 大事件 \tabularnewline
  \midrule
  \endhead
  \midrule
  元年 & 1095 & \tabularnewline\hline
  二年 & 1096 & \tabularnewline\hline
  三年 & 1097 & \tabularnewline\hline
  四年 & 1098 & \tabularnewline\hline
  五年 & 1099 & \tabularnewline\hline
  六年 & 1100 & \tabularnewline\hline
  七年 & 1101 & \tabularnewline
  \bottomrule
\end{longtable}


%%% Local Variables:
%%% mode: latex
%%% TeX-engine: xetex
%%% TeX-master: "../Main"
%%% End:

%% -*- coding: utf-8 -*-
%% Time-stamp: <Chen Wang: 2019-10-15 16:31:37>

\section{天祚帝\tiny(1101-1125)}

遼天祚帝耶律延禧(1075年6月5日-1128年或1156年),字延宁,小名阿果,是遼國西遷前的最后一位皇帝,他的统治时间是从1101年2月12日至1125年3月26日,在位24年。

天祚帝是辽道宗的孙子,他的父亲是道宗的太子耶律濬,母亲是貞順皇后萧氏。六岁时他被封为梁王,九岁时封为燕国王。

寿昌七年正月十三日(1101年2月12日),道宗崩,临死前立耶律延禧为继承人,耶律延禧奉遗诏即皇帝位于柩前。延禧以「天祚皇帝」作為自己的尊號。二月壬辰改元乾統。

天祚帝继位后西夏崇宗因受到北宋攻击一再向辽求援,并求尚天祚帝女公主为妻,最后天祚帝于1105年将一个族女封为公主嫁给了夏崇宗,并派使者赴宋,劝宋对西夏罢兵。

1112年二月丁酉天祚帝赴春州,召集附近的女真族酋长来朝,宴席中醉酒后令女真酋长为他跳舞,只有完颜阿骨打不肯。天祚帝不以为意,但从此完颜阿骨打与遼國之间不和。从九月开始完颜阿骨打不再奉诏,并开始对其他不服从自己的女真部落用兵。1114年春,完颜阿骨打正式起兵反辽。一开始天祚帝不将阿骨打当作大威胁,但是1114年天祚帝所有派去镇压阿骨打的军队全部被战败。

1115年天祚帝終於开始觉察到女真的威胁勢力,下令亲征,但是辽军到处被女真打败,与此同时遼國国内也发生叛乱,耶律章奴在上京临潢府叛乱,虽然这场叛乱很快就被平定,但是这场叛乱分裂了遼國内部。此后位于原渤海国的东京辽阳府也发生叛乱自立。这场叛乱一直到1116年四月才被平定。但是在五月女真就借机占领了辽阳和瀋州。1117年女真攻春州,辽军不战自败。这年完颜阿骨打称帝,建立金朝。

1120年金攻克上京臨潢府,留守降。到1121年辽已经失去了其疆域之半。而遼國内部又发生了因为皇位继承问题而爆发的内乱,1122年天祚帝杀了自己的长子耶律敖卢斡,这使得更多的辽國军人感到不安而投靠金朝。四月,金攻克辽西京大同府。由于战场上消息不通,遼國内部又以为天祚帝在前线阵亡或被围,于是在臨潢立耶律淳为皇帝,进一步扩大了遼國内部的混乱。而遼國的大臣也各不自保,有的与北宋大臣童贯通气打算投降宋朝的,有的则想投降金朝。十一月居庸关失守,十二月辽南京被攻破。1123年正月上京叛金。

到1124年天祚帝已经失去了遼國的大部分土地而退出漠外,他的儿子和家属大多数被杀或被俘,虽然他还打算重新守護燕州和云州,但是实际上他已经没有多少希望了。保大五年二月二十日(1125年3月26日)天祚帝在应州为金人完颜娄室等所俘,八月被解送金上京,被降为海滨王。金太宗天會六年(1128年)病死。金皇統元年(1141年),改封豫王。皇統五年(1145年),葬於乾陵旁。

《大宋宣和遺事》則記載南宋紹興二十六年(金朝正隆元年,1156年)六月,金朝皇帝完顏亮命令56歲的宋欽宗趙桓和81歲的耶律延禧去比賽馬球,趙桓中途從馬上跌下來,被馬亂踐而死,耶律延禧則因善騎術,企圖縱馬衝出重圍逃命,結果被金人以亂箭射死。

元朝官修正史《辽史》脱脱等的評價是:“辽起朔野,兵甲之盛,鼓行皞外,席卷河朔,树晋植汉,何其壮欤?太祖、太宗乘百战之势,辑新造之邦,英谋睿略,可谓远矣。虽以世宗中才,穆宗残暴,连遘弑逆,而神器不摇。盖由祖宗威令犹足以震叠其国人也。圣宗以来,内修政治,外拓疆宇,既而申固邻好,四境乂安。维侍二百余年之基,有自来矣。降臻天祚,既丁末运,又觖人望,崇信奸回,自椓国本,群下离心。金兵一集,内难先作,废立之谋,叛亡之迹,相继蜂起。驯致土崩瓦解,不可复支,良可哀也!耶律与萧,世为甥舅,义同休戚,奉先挟私灭公,首祸构难,一至于斯。天祚穷蹙,始悟奉先误己,不几晚乎!淳、雅里所谓名不正,言不顺,事不成者也。大石苟延,彼善于此,亦几何哉?”

\subsection{乾统}

\begin{longtable}{|>{\centering\scriptsize}m{2em}|>{\centering\scriptsize}m{1.3em}|>{\centering}m{8.8em}|}
  % \caption{秦王政}\
  \toprule
  \SimHei \normalsize 年数 & \SimHei \scriptsize 公元 & \SimHei 大事件 \tabularnewline
  % \midrule
  \endfirsthead
  \toprule
  \SimHei \normalsize 年数 & \SimHei \scriptsize 公元 & \SimHei 大事件 \tabularnewline
  \midrule
  \endhead
  \midrule
  元年 & 1101 & \tabularnewline\hline
  二年 & 1102 & \tabularnewline\hline
  三年 & 1103 & \tabularnewline\hline
  四年 & 1104 & \tabularnewline\hline
  五年 & 1105 & \tabularnewline\hline
  六年 & 1106 & \tabularnewline\hline
  七年 & 1107 & \tabularnewline\hline
  八年 & 1108 & \tabularnewline\hline
  九年 & 1109 & \tabularnewline\hline
  十年 & 1110 & \tabularnewline
  \bottomrule
\end{longtable}

\subsection{天庆}

\begin{longtable}{|>{\centering\scriptsize}m{2em}|>{\centering\scriptsize}m{1.3em}|>{\centering}m{8.8em}|}
  % \caption{秦王政}\
  \toprule
  \SimHei \normalsize 年数 & \SimHei \scriptsize 公元 & \SimHei 大事件 \tabularnewline
  % \midrule
  \endfirsthead
  \toprule
  \SimHei \normalsize 年数 & \SimHei \scriptsize 公元 & \SimHei 大事件 \tabularnewline
  \midrule
  \endhead
  \midrule
  元年 & 1111 & \tabularnewline\hline
  二年 & 1112 & \tabularnewline\hline
  三年 & 1113 & \tabularnewline\hline
  四年 & 1114 & \tabularnewline\hline
  五年 & 1115 & \tabularnewline\hline
  六年 & 1116 & \tabularnewline\hline
  七年 & 1117 & \tabularnewline\hline
  八年 & 1118 & \tabularnewline\hline
  九年 & 1119 & \tabularnewline\hline
  十年 & 1120 & \tabularnewline
  \bottomrule
\end{longtable}

\subsection{保大}

\begin{longtable}{|>{\centering\scriptsize}m{2em}|>{\centering\scriptsize}m{1.3em}|>{\centering}m{8.8em}|}
  % \caption{秦王政}\
  \toprule
  \SimHei \normalsize 年数 & \SimHei \scriptsize 公元 & \SimHei 大事件 \tabularnewline
  % \midrule
  \endfirsthead
  \toprule
  \SimHei \normalsize 年数 & \SimHei \scriptsize 公元 & \SimHei 大事件 \tabularnewline
  \midrule
  \endhead
  \midrule
  元年 & 1121 & \tabularnewline\hline
  二年 & 1122 & \tabularnewline\hline
  三年 & 1123 & \tabularnewline\hline
  四年 & 1124 & \tabularnewline\hline
  五年 & 1125 & \tabularnewline
  \bottomrule
\end{longtable}


%%% Local Variables:
%%% mode: latex
%%% TeX-engine: xetex
%%% TeX-master: "../Main"
%%% End:

%% -*- coding: utf-8 -*-
%% Time-stamp: <Chen Wang: 2019-10-15 16:34:10>

\section{北辽\tiny(1122)}

北遼,於1122年3月立國,是時辽朝天祚帝被金兵所迫,流亡夹山,耶律淳在燕京被耶律大石等人擁立為君主,是為北遼的開始。1122年6月24日,耶律淳病死,德妃蕭普賢女以皇太后身份攝政,期间击退宋朝进攻(宣和北伐)。1123年2月2日,金朝攻佔燕京,蕭德妃和耶律大石投奔天祚帝,北遼滅亡,國祚不足一年。後來,萧德妃因為謀反而被殺,但耶律大石卻得到赦免。

\subsection{宣宗\tiny(1122)}

遼宣宗耶律淳(1063年-1122年),小字涅里,是北遼開國皇帝,為遼兴宗第四子宋魏國王耶律和鲁斡之子。淳一出生就由其祖母遼興宗的仁懿皇后撫養,長大成人之後,好文學。遼道宗太子耶律濬被殺害之後,遼道宗曾打算立侄子淳為嗣,後罷,封北平郡王,出為彰聖等軍節度使。

天祚帝即位。乾統元年(1101年)封耶律和鲁斡為天下兵馬大元帅,此意味著有皇位的繼承權,封淳為鄭王。乾統三年(1103年)封耶律和鲁斡為皇太叔,進封淳為越國王。乾統六年(1106年),拜為南府宰相,創議制訂兩府禮儀,進封為魏國王。乾統十年(1110年),耶律和鲁斡去世,淳襲南京留職,冬夏入朝,寵冠諸王。

天慶五年(1115年),耶律章奴謀反,打算迎立耶律淳為帝。耶律淳不從。次年(1116年)六月,耶律淳進封秦晉國王,拜都元帥,賜金券,免漢拜禮,不名。

保大二年(1122年)正月,金軍攻克遼中京,天祚帝被金兵所迫,流亡夾山。奚王回離保和林牙耶律大石援引唐肅宗靈武稱帝的例子,勸說耶律淳稱帝。三月,淳即皇帝位,百官上尊號為天錫皇帝,改年號建福元年,遥降天祚皇帝为湘阴王,封妻蕭普賢女為德妃,並遣使奉表于金國,乞为附庸。

六月,耶律淳事未完成就病死,終年六十歲。百官上諡号孝章皇帝,庙号宣宗,葬燕京西部的香山永安陵。

\subsubsection{建福}


\begin{longtable}{|>{\centering\scriptsize}m{2em}|>{\centering\scriptsize}m{1.3em}|>{\centering}m{8.8em}|}
  % \caption{秦王政}\
  \toprule
  \SimHei \normalsize 年数 & \SimHei \scriptsize 公元 & \SimHei 大事件 \tabularnewline
  % \midrule
  \endfirsthead
  \toprule
  \SimHei \normalsize 年数 & \SimHei \scriptsize 公元 & \SimHei 大事件 \tabularnewline
  \midrule
  \endhead
  \midrule
  元年 & 1122 & \tabularnewline
  \bottomrule
\end{longtable}

\subsection{萧普贤女\tiny(1122)}

蕭普賢女(?-1123年),為北遼宣宗耶律淳的德妃,宣宗遺詔立天祚帝耶律延禧第五子耶律定為皇帝,但他在天祚帝身邊,不在燕京,只能遙立。德妃被立為皇太后,稱制,改建福元年為德興元年。

此時大臣李處溫父子覺得前景不妙,打算向南私通宋的童貫,欲劫持德妃納土於宋。向北私通金人,作金的內應。後她發現他私通宋、金的罪行把他拘捕並賜死。

當年十一月,德妃五次上表給金朝,只要允許立耶律定為北遼皇帝,其他條件均答應,金人不許,她只好派兵把守居庸關,沒能守住,金兵直奔燕京。德妃帶著隨從的官員投靠天祚帝,天祚帝將她誅殺。

\subsubsection{德兴}

\begin{longtable}{|>{\centering\scriptsize}m{2em}|>{\centering\scriptsize}m{1.3em}|>{\centering}m{8.8em}|}
  % \caption{秦王政}\
  \toprule
  \SimHei \normalsize 年数 & \SimHei \scriptsize 公元 & \SimHei 大事件 \tabularnewline
  % \midrule
  \endfirsthead
  \toprule
  \SimHei \normalsize 年数 & \SimHei \scriptsize 公元 & \SimHei 大事件 \tabularnewline
  \midrule
  \endhead
  \midrule
  元年 & 1122 & \tabularnewline
  \bottomrule
\end{longtable}



%%% Local Variables:
%%% mode: latex
%%% TeX-engine: xetex
%%% TeX-master: "../Main"
%%% End:

%% -*- coding: utf-8 -*-
%% Time-stamp: <Chen Wang: 2021-11-01 16:12:23>

\section{西辽\tiny(1124-1218)}

\subsection{简介}

西辽(1124年-1218年),又称喀喇契丹,是契丹人耶律大石建立的国家。耶律大石原本效力于辽天祚帝,在辽朝即将灭亡之际出奔。1124年,耶律大石称王,到达可敦城(今蒙古国布尔干省青托罗盖古回鹘城)建立根据地。1132年,在叶密立(今新疆维吾尔自治区额敏县)称“菊儿汗”,西辽帝国正式建立。随后耶律大石向新疆、蒙古高原、中亚及西亚地区扩张,建都于虎思斡鲁朵(今吉尔吉斯斯坦托克玛克东南布拉纳)。在1141年的卡特万之战,击败塞尔柱帝国联军,成为中亚霸主,将威名远播至欧洲。高昌回鹘、西喀喇汗国、东喀喇汗国及花剌子模先后臣服于强盛期的西辽。耶律大石死后,历经萧塔不烟、耶律夷列、耶律普速完三代君主后,到耶律直鲁古时期,由于长期对外战争,使西辽的国力走向衰落,最终被屈出律篡国。蒙古帝国崛起后,1218年,西辽被蒙古帝国灭亡。

\subsection{德宗耶律大石\tiny(1124-1143)}

\subsubsection{生平}

遼德宗耶律大石(1094年-1143年),又称大石林牙或林牙大石。字重德,契丹人。西辽開國皇帝,庙号德宗,在位20年。

耶律大石早年效力于辽天祚帝,辽天祚帝出奔后,耶律大石参与拥立耶律淳和萧德妃,在北宋、金朝两面夹击的情况下,积极维持风雨飘摇的北辽,两次率军以少胜多击败北宋的进攻。北辽灭亡后,耶律大石投奔天祚帝,在辽朝即将灭亡之际出奔。1124年,耶律大石称遼王建號延慶,到达可敦城(今蒙古国布尔干省青托罗盖古回鹘城)建立根据地。1132年,在叶密立(今新疆维吾尔自治区额敏县)称“菊儿汗”,西辽帝国正式建立。随后耶律大石向新疆、蒙古高原、中亚及西亚地区扩张,建都于虎思斡鲁朵(今吉尔吉斯斯坦托克玛克东南布拉纳)。在1141年的卡特万之战,击败塞尔柱帝国联军,成为中亚霸主,将威名远播至欧洲。高昌回鹘、西喀喇汗国、东喀喇汗国及花剌子模先后臣服于强盛期的西辽。1143年,耶律大石去世。

耶律大石在军事、政治和外交上都有成就,欧洲得知其西征的事迹,流传着祭司王约翰的传说。耶律大石的名字也成为西辽帝国的代称,在耶律大石去世后多年,很多国家仍用“大石”称呼西辽的后代君主。

耶律大石是辽朝开国君主耶律阿保机的八世孙,精通契丹语和汉语,擅长弓马骑射。1115年,耶律大石中进士入翰林,初为翰林应奉,不久累迁翰林承旨。根据辽朝的科举制度,殿试头名才有入翰林应奉的资格。因契丹语称翰林为林牙,耶律大石又被称为大石林牙或林牙大石。后历任泰州、祥州刺史,辽兴军节度使。

历经200多年统治的辽朝国力逐渐走向衰弱,取而代之的是女真族建立的金朝。在金军势如破竹的攻击下,辽朝节节败退。1122年,金军攻克辽中京大定府和泽州,辽天祚帝如惊弓之鸟,从居庸关至鸳鸯泺(今河北省张北县安固里淖)到白水泺(今内蒙古自治区乌兰察布市察右前旗黄旗海),再到女古底仓,一路仓皇逃跑至夹山(今内蒙古自治区武川县附近)。数日后,宰相李处温与南京(即燕京,今北京市西南)都统萧干、耶律大石等拥立秦晋国王耶律淳为帝,建立北辽。耶律大石被视为肱骨之臣,官至太师。

1120年,一心想收复燕云十六州的北宋与金朝缔结了海上之盟,约定南北夹击辽朝。1122年5月,宋徽宗得知金朝大举进攻的消息后,任命童贯为宣抚使,蔡攸为副使,率军15万巡边,伺机收复燕云十六州。耶律淳委派耶律大石为西南路都统,牛栏监军萧遏鲁为副将,率领奚、契丹骑兵2000,驻扎于涿州新城县(今河北省高碑店市)防备。

宋军裨将杨可世听闻燕地百姓早有归宋之心,如果宋军到达,燕人必定箪食壶浆迎接,便率轻骑数千奇袭燕京,但7月1日在兰沟甸遭到耶律大石军的掩杀,大败而归。耶律淳得知消息后,又增兵3万。耶律大石率军渡过白沟河,4日与宋军东路统制种师道隔河对峙。战前,杨可世派赵明持黄榜旗前往耶律大石的营帐劝降,耶律大石毁旗怒骂:“无多言,有死而已。”话语未完,辽军矢石如雨。耶律大石指挥骑兵从西部浅滩处渡河,分左右两翼包抄宋军,宋军大败,杨可世中铁蒺藜负伤。次日,驻扎于范村(今河北省涿州市西南)的宋军西路统制辛兴宗的部队也遭到四军大王萧干的围攻。

7月8日,种师道下令撤兵,耶律大石得知消息后,率轻骑追击至古城,双方交战,宋军大乱,种师道几乎不能脱逃。宋军一路逃奔至雄州,辽军一路跟随,童贯禁止宋军入城,契丹人斥责北宋背弃澶渊之盟,挑起战争。正逢此日北风大雨冰雹交加,宋军一败再败,阵亡者不计其数,种师道也因燕京之战的失利遭到童贯的弹劾,责令致仕。

7月11日,耶律大石在涿州召见北宋使者马扩,责问他辽朝与北宋通好百年,现今北宋为何率军前来抢夺辽朝的领土。马扩以“宋不取怕金来取”作答辩。耶律大石斥责马扩,说西夏屡次派使者唆使辽朝进攻北宋,但辽朝不肯见利忘义,将表章封存后交给北宋,如今北宋只听信了女真人的一句话,便于辽朝兵戈相见。耶律大石又质问马扩既为使者,为何与叛将刘宗吉有联系,并让他转告童贯,如果两国想和好仍可交好,如果不愿和好便可提兵来战,不要在天热时打仗使士兵受苦。

1122年7月29日,耶律淳病死,其妻萧德妃临朝称制。宰相李处温南通童贯,想纳土降宋,北联络金朝作为内应,事发后被处死。李处温死后,北辽的军政事务由太师耶律大石和四军大王萧干掌控。

北宋得知耶律淳去世的消息后,在太宰王黼的倡议下,再次兴兵攻打北辽。8月29日,宋徽宗下诏集结各道兵20万,以刘延庆为都统制,于10月在三关(草桥关、益津关、瓦桥关)汇合。10月25日,北辽都管押常胜军、涿州留守郭药师叛降北宋。11月19日,刘延庆、何灌、郭药师等率军从雄州出发,进入新城县;刘光世、杨可世从安肃州(今河北省徐水县安肃镇)出发,进入易州,两军于涿州汇合,共50万。耶律大石和萧干统帅的北辽军不足2万人,在泸沟河部署。宋辽两军隔河对峙,双方曾战于料石冈,但未分胜负。11月24日,郭药师率军6000奇袭燕京,入外城。契丹守军拼力死战,而宋军毫无军纪,饮酒后到处奸淫掳掠。萧德妃秘遣使者召耶律大石、萧干军,昼夜疾行,自南暗门入城,宋军大败,仅百余骑得以逃脱。29日,泸沟河北面四处火起,宋军以为辽军将至,烧营落荒而逃。逃兵自相践踏,坠落山涧者不计其数,丢弃的军需物资绵延数百里。

北辽刚刚击退南方的宋军,北方的金军又再次逼近。萧德妃曾五次上表金朝,请求立秦王耶律定为帝,称臣求和,金太祖不许。萧德妃只好派精兵防守居庸关,但金兵到来时,居庸关城墙倒塌,士兵多被压死,其余守军不战而溃。萧德妃闻讯后连夜逃离燕京,声称御敌,实为出奔。萧德妃、耶律大石、萧干等经古北口(今北京市密云县古北口镇),向东逃至松亭关(今河北省宽城满族自治县西南),但因去往何处,发生争执。萧干主张去奚王府立国,而耶律大石则主张投奔天祚帝。驸马都尉萧勃迭反对耶律大石的意见,被耶律大石下令斩首。耶律大石又传令军中,有异议者斩。于是北辽军兵分两路,萧干率领奚、渤海军前往奚王府,耶律大石挟持萧德妃去夹山投奔天祚帝。萧干到达奚王府后,自立为帝,国号大奚,半年后败亡。耶律大石与萧德妃率军7000,于1123年3月至夹山。天祚帝因耶律淳被立之事杀萧德妃及外甥耶律常哥。天祚帝又质问耶律大石为何擅立耶律淳,耶律大石指出天祚帝以辽朝全国国力不能抵御金朝的进攻,弃国而逃,致使生灵涂炭。耶律淳为辽太祖子孙,立其为帝保社稷远胜于投降金朝。在耶律大石的辩解下,天祚帝下令赦免其余众人。

耶律大石在辽天祚帝帐下任都统一职,1123年,率辽军进攻奉圣州,驻军于龙门山东二十五里处。金朝都统完颜斡鲁派完颜照立、完颜娄室、马和尚等率军攻打,耶律大石战败被完颜娄室俘虏,所部投降。完颜宗望用绳子绑着耶律大石,强迫他作为向导,率军袭击了天祚帝位于青冢泺(今内蒙古自治区呼和浩特市南)的大营,俘获了天祚帝之子秦王耶律定、许王耶律寧和嫔妃、公主、从臣多人,获取辎重车万余辆,只有梁王耶律雅里和天祚帝长女趁乱逃出。耶律大石因作为向导有功,免其罪并特受金太祖降诏奖谕。金太祖还十分欣赏耶律大石的仪表俊美,为人聪辩,特赐予其妻子。同年9月,耶律大石跟随金朝西征,带领家眷自金营逃出,率领一支部队投奔天祚帝。关于耶律大石在金营中的生活,《契丹国志》记载耶律大石投降金朝后与粘罕不和,粘罕想杀掉耶律大石,耶律大石带着五个儿子夜间逃脱,但把妻子留在金营中。粘罕将耶律大石的妻子赐给部落中地位最低贱的人,但他的妻子坚贞不屈,最后被粘罕射杀,但此段资料真实性待考。

1124年,在得到耶律大石的部队和阴山室韦首领毛割石的援助后,辽天祚帝认为反攻的时机已经来临,决定亲自出兵收复燕州、云州地区。耶律大石认为金军气盛,应当养精蓄锐,不能贸然出击,天祚帝不听,坚持出兵。耶律大石知道天祚帝无法完成复兴辽朝的大业,又害怕得到天祚帝的猜忌,于是杀掉萧乙薛和坡里括后自立为王,率领铁骑200出奔。耶律大石走后,辽天祚帝虽然取得一些战役的胜利,但不久便被金朝所败。1125年,辽天祚帝在投奔西夏的途中被俘,辽朝灭亡。

耶律大石率军从夹山出发,北行三日渡过黑水(爱毕哈河),途中遇到白鞑靼人首领床古儿,床古儿给予耶律大石四百匹马,二十头骆驼,若干只羊的援助。耶律大石一路向西北,于1124年到达可敦城,召集威武、崇德、会蕃、新、大林、紫河、驼等七个军州的长官和大黄室韦、敌剌、王纪剌、茶赤剌、也喜、鼻古德、尼剌、达剌乖、达密里、密儿纪、合主、乌古里、阻蔔、普速完、唐古、忽母思、奚的、纠而毕十八个部族的首领举行大会。在大会上,耶律大石慷慨激昂地指出先祖创建辽朝的艰难以及由于金朝对于辽朝侵略,造成天祚帝流亡在外、生灵涂炭,号召各军州和部族驱逐仇敌,复兴大辽。由于可敦城是辽朝的西北边防重镇,边防军队不得随意征调,军队在战乱中得以保存,并且此地还拥有可骑乘的战马数十万匹。耶律大石安置官吏,整顿兵马,磨砺武器,得到精兵万余人。

耶律大石在可敦城建立根据地后,积攒实力,不断派使者联络白鞑靼人、西夏以及北宋,从外交上孤立金朝。1125年夏,西夏联络耶律大石攻取金朝的山西诸郡。同年末,耶律大石派使者联络北宋,提议合力攻打金朝。1127年,白鞑靼人与耶律大石通好,拒绝将马匹卖给金朝。金太宗派使者问罪,双方关系紧张。1129年,耶律大石率军攻取了金朝的北方二营。次年,金太宗派耶律余睹、石家奴、拔离速征讨耶律大石,但由于诸部落不同意出兵,大军行进至兀纳水后收兵。

经过休整,耶律大石的军事实力得到壮大。1130年3月,耶律大石以青牛、白马祭告天地、列祖,准备西征。耶律大石先派使者送信给高昌回鹘首领毕勒哥,阐明两国先代的友好并要求借道去大食。毕勒哥得到书信后,迎接耶律大石至宫邸大宴三日,临行前毕勒哥亲自护送耶律大石出境,赠送耶律大石马匹六百、骆驼数百、羊三千只作为礼物,并约定交出人质,作为耶律大石的附庸国。

耶律大石率军离开高昌回鹘,进入吉尔吉斯境内,遭到了当地的抵抗,但双方未发生大规模的战争。耶律大石率军继续西进,到达叶密立。大军所到之处望风披靡,获取骆驼、牛、马、羊等辎重无数。1131年春,金朝统帅粘罕及耶律余睹率领云中、燕、云州汉军、金军1万人攻打耶律大石的根据地可敦城,但遭到失败。耶律大石到达叶密立后,虽然与高昌回鹘发生过摩擦,但基本得到了当地突厥部族的支持,户数达到4万。1132年,耶律大石在新建成的叶密立正式称“菊儿汗”,群臣又尊汉号为“天祐皇帝”,建元延庆,追尊祖父为元皇帝,祖母为宣义皇后,册封元妃萧氏为昭德皇后,西辽帝国正式建立。

西辽帝国建立后,耶律大石开始酝酿向周边地区扩张。1132年,耶律大石亲率大军向南进发,高昌回鹘再次臣服于西辽。随后耶律大石率军越过天山,沿塔里木盆地北向西推进,与东喀喇汗国发生冲突。西辽军被东喀喇汗国阿尔斯兰汗阿赫马德·伊本·哈桑的军队击败,大将阿勒·阿瓦尔被俘,损失惨重。耶律大石撤军后向七河地区进发,收编了当地的契丹人和突厥人,共16000帐,使西辽军队的人数增加了一倍。耶律大石率军驻扎于西辽与东喀喇汗国边境地区,等待时机准备反攻。

1132年,阿赫马德·伊本·哈桑去世,其子伊卜拉欣二世继任。伊卜拉欣二世软弱无能,原本臣属于东喀喇汗国的葛逻禄和康里人趁机袭击他的部属和牲畜,进行劫掠。伊卜拉欣二世不能控制住国内的局势,于是派使者请求耶律大石进入八剌沙衮(今吉尔吉斯斯坦托克馬克東)接管他的国家,使他“摆脱这尘世的烦恼”。耶律大石接到请求后,率军进入东喀喇汗国首都八剌沙衮,“登上那不费分文的宝座”。耶律大石将伊卜拉欣二世降为伊列克·突厥蛮(意为突厥王),保留了他对喀什噶尔(今新疆维吾尔自治区喀什市)、和田地区的控制,东喀喇汗国成为西辽的附庸。由于八剌沙衮附近是可耕可牧的肥沃地区,耶律大石决定建都于此,将八剌沙衮改名为虎思斡耳朵(意为强而有力的宫帐),并改元康国。耶律大石随后又派军队战胜了吉尔吉斯人,征服了别失八里(今新疆维吾尔自治区吉木萨尔县境内),康里人不久也臣服于西辽。

1134年4月,耶律大石任命六院司大王萧斡里剌为兵马都元帅,敌剌部前同知枢密院事萧查剌阿不为副元帅,茶赤剌部秃鲁耶律燕山为都部署,护卫耶律铁哥为都监,率军7万征讨金朝。在战前的誓师大会上,耶律大石用白马青牛祭天,指出先祖创业艰难,是由于后代君主耽于享乐致使社稷倾覆。中亚并非久居之地,应当荣归故里,复兴大辽。他又劝谕萧斡里剌要与士卒同甘共苦,赏罚分明。作战时要选择水草丰富处扎营,谨慎用兵。但由于西辽与金朝两国相隔遥远,西辽军队行进万里一无所获,兵马损失惨重,不得不撤军回国。另据《三朝北盟会编》记载,1135年,耶律大石再次率军攻打金朝,金熙宗派粘罕迎战。金军进入沙漠后与西辽军征战三昼夜不分胜败,但金军粮草断绝,人马冻死很多,加上本为契丹人的副将临阵倒戈,致使粘罕大败而归。但此段史料的真实性待考。

自1137年起,耶律大石开始了第二次扩张。1137年,耶律大石率军向察赤(今乌兹别克斯坦塔什干)、费尔干纳盆地及泽拉夫尚河流域进兵。同年5至6月,在忽毡(今塔吉克斯坦苦盏)遭到了西喀喇汗国可汗马赫穆德·伊本·穆海默德的抵抗。西喀喇汗国战败,马赫穆德败逃回撒马尔罕。这次战败使马黑木二世的臣民感到震惊、惊恐和沮丧,但耶律大石并没有继续进兵。1141年,西喀喇汗国与葛逻禄人爆发冲突,马赫穆德向宗主国塞尔柱帝国求援。塞尔柱苏丹桑贾尔动员伊斯兰诸国参战,集中了呼罗珊、锡斯坦、伽色尼、马赞德兰、古尔等国的军队近10万人,单单阅兵就耗费了半年时间。同年7月,桑贾尔率军渡过阿姆河,进入河中地区,葛逻禄人急忙派使者向耶律大石求救。

耶律大石写信给桑贾尔替葛逻禄人说情,但桑贾尔十分傲慢的回信命令耶律大石加入伊斯兰教,并称自己的军队能用箭截断敌人的须发。当耶律大石听完桑贾尔的使者读完书信后,下令拔下他的一撮胡须,然后给他一根针让他当场示范,使者不能做到。耶律大石说既然针不能截断胡须,那那个人又怎么能用箭折断须发呢?于是下令进兵,双方在撒马尔罕以北的卡特万草原对峙,西辽的军队中有契丹人、突厥人、汉人和蒙古人。耶律大石观察了战场的地形后,让军队背靠达尔加姆峡谷安营。两军于1141年9月9日展开会战,战前耶律大石指出桑贾尔的联军人多少谋,如果全力进攻,他们就会首尾不顾。耶律大石派六院司大王萧斡里剌、招讨副使耶律松山等率兵2500攻打联军右翼,枢密副使萧剌阿不、招讨使耶律术薛等率兵2500攻打其左翼,耶律大石亲率部队攻打中军;桑贾尔的联军右翼是埃米尔库马吉,左翼是锡斯坦埃米尔胡马希,他自己亲率中军,有战斗经验的老兵负责殿后。

在战场上,锡斯坦贵族作战英勇,但西辽军队中的葛逻禄人发挥了重要的作用,迫使桑贾尔的联军败逃。桑贾尔和马赫穆德逃奔至泰尔梅兹,桑贾尔的妻子、左、右翼统帅和伊斯兰法学家胡萨姆·奥玛尔·伊本·阿布杜·阿齐兹·伊本·马扎·布哈里均被俘虏。桑贾尔的联军损失惨重,仅达尔加姆峡谷就装下1万名死者。《辽史》记载塞尔柱帝国联军的阵亡者横尸数十里。卡特万之战后,塞尔柱帝国的势力退出河中地区,西辽成为中亚霸主。耶律大石随后率军进入撒马尔罕,立马赫穆德之弟伊卜拉欣·伊本·穆海默德为桃花石汗,继续让其统治西喀喇汗国。 他还下令处死布哈拉的伊斯兰教教长胡沙穆丁·倭玛尔,任命阿尔普·的斤统治该地。随后派大将额儿布思(一说即萧斡里剌)出兵花剌子模,在该地烧杀抢掠,迫使花剌子模沙阿阿拉丁·阿比兹向西辽臣服并且每年缴纳价值3万金第纳尔的货物和牲畜。耶律大石在撒马尔罕驻扎90天后,至起儿漫(今乌兹别克斯坦卡尼梅赫镇)巡行后班师返回虎思斡耳朵。

1143年,耶律大石去世,在位20年,庙号德宗。因耶律大石之子耶律夷列年幼,遗诏命皇后萧塔不烟临朝称制,改元咸清,称感天皇后。

耶律大石的西征事迹被传到欧洲,正逢第二次十字军东征,于是在欧洲流传着东方世界有一位神秘的祭司王约翰,是基督教的捍卫者。俄语、阿拉伯语、拉丁语和古英语中中国的发音类似于“契丹”,都是受耶律大石西征的影响。而耶律大石的名字也成了西辽帝国的代称,在耶律大石死后,金、西夏、南宋等国家对西辽的后代君主皆称为“大石”。

耶律大石凭借卓越的军事、政治、外交才能,在伊斯兰世界建立了幅员辽阔的西辽帝国,将辽朝的国祚延续了近百年,他为东西方文化、经济方面的交流作出了积极的贡献。东西方史学家对于耶律大石的成就多有赞誉:穆斯林史学家朱兹贾尼评价耶律大石:是一位公正的君主,因为公正和才干而受到崇敬;耶律楚材评价耶律大石:颇尚文教,西域人至今思之。拉施特称赞耶律大石:是一个有智慧而又有才干的人。他有条不紊地从这些地区上把队伍召集到身边,占领了整个突厥斯坦地区,(从而)获得了古儿汗,即伟大的君主的称号。清代史学家谭宗浚评价耶律大石:德宗起自词臣,兼通藩俗,削平各部,殄定诸藩,意其典章制度必可多采。

\subsubsection{延庆}


\begin{longtable}{|>{\centering\scriptsize}m{2em}|>{\centering\scriptsize}m{1.3em}|>{\centering}m{8.8em}|}
  % \caption{秦王政}\
  \toprule
  \SimHei \normalsize 年数 & \SimHei \scriptsize 公元 & \SimHei 大事件 \tabularnewline
  % \midrule
  \endfirsthead
  \toprule
  \SimHei \normalsize 年数 & \SimHei \scriptsize 公元 & \SimHei 大事件 \tabularnewline
  \midrule
  \endhead
  \midrule
  元年 & 1124 & \tabularnewline\hline
  二年 & 1125 & \tabularnewline\hline
  三年 & 1126 & \tabularnewline\hline
  四年 & 1127 & \tabularnewline\hline
  五年 & 1128 & \tabularnewline\hline
  六年 & 1129 & \tabularnewline\hline
  七年 & 1130 & \tabularnewline\hline
  八年 & 1131 & \tabularnewline\hline
  九年 & 1132 & \tabularnewline\hline
  十年 & 1133 & \tabularnewline
  \bottomrule
\end{longtable}

\subsubsection{康国}

\begin{longtable}{|>{\centering\scriptsize}m{2em}|>{\centering\scriptsize}m{1.3em}|>{\centering}m{8.8em}|}
  % \caption{秦王政}\
  \toprule
  \SimHei \normalsize 年数 & \SimHei \scriptsize 公元 & \SimHei 大事件 \tabularnewline
  % \midrule
  \endfirsthead
  \toprule
  \SimHei \normalsize 年数 & \SimHei \scriptsize 公元 & \SimHei 大事件 \tabularnewline
  \midrule
  \endhead
  \midrule
  元年 & 1134 & \tabularnewline\hline
  二年 & 1135 & \tabularnewline\hline
  三年 & 1136 & \tabularnewline\hline
  四年 & 1137 & \tabularnewline\hline
  五年 & 1138 & \tabularnewline\hline
  六年 & 1139 & \tabularnewline\hline
  七年 & 1140 & \tabularnewline\hline
  八年 & 1141 & \tabularnewline\hline
  九年 & 1142 & \tabularnewline\hline
  十年 & 1143 & \tabularnewline
  \bottomrule
\end{longtable}


\subsection{萧塔不烟\tiny(1143-1150)}

\subsubsection{生平}

萧塔不烟,生卒年不详,西辽开国君主遼德宗的皇后,德宗死後稱制,執政7年。

1143年,耶律大石去世后,其子耶律夷列年幼,遗诏命皇后萧塔不烟临朝称制,改元咸清,称感天皇后。

1144年,金熙宗得知耶律大石去世的消息後,派使者粘割韩奴前往劝降西辽。粘割韩奴進入西遼國境後,正好遇上外出打獵的萧塔不烟。見到萧塔不烟後,粘割韩奴不但沒有下马跪拜,反而讓她下马接诏。萧塔不烟於是命人将粘割韩奴拉下马,讓他跪下。粘割韩奴痛骂不止,斥責其為反賊,侮辱上国使臣。萧塔不烟发怒,派人将其杀死。

執政七年後,萧塔不烟退位。耶律夷列親政,改年號為紹興。

一说萧塔不烟与耶律大石在叶密立(今新疆维吾尔自治区额敏县)称菊儿汗时册封的昭德皇后萧氏为同一人;也有观点认为二者并非同一人。

\subsubsection{咸清}

\begin{longtable}{|>{\centering\scriptsize}m{2em}|>{\centering\scriptsize}m{1.3em}|>{\centering}m{8.8em}|}
  % \caption{秦王政}\
  \toprule
  \SimHei \normalsize 年数 & \SimHei \scriptsize 公元 & \SimHei 大事件 \tabularnewline
  % \midrule
  \endfirsthead
  \toprule
  \SimHei \normalsize 年数 & \SimHei \scriptsize 公元 & \SimHei 大事件 \tabularnewline
  \midrule
  \endhead
  \midrule
  元年 & 1144 & \tabularnewline\hline
  二年 & 1145 & \tabularnewline\hline
  三年 & 1146 & \tabularnewline\hline
  四年 & 1147 & \tabularnewline\hline
  五年 & 1148 & \tabularnewline\hline
  六年 & 1149 & \tabularnewline\hline
  七年 & 1150 & \tabularnewline
  \bottomrule
\end{longtable}

\subsection{仁宗耶律夷列\tiny(1150-1163)}

\subsubsection{生平}

辽仁宗耶律夷列(?-1163年),耶律大石和萧塔不烟之子,耶律普速完之兄,西遼第二任君主,在位13年。

耶律大石去世后,其子耶律夷列年幼,遗诏命皇后萧塔不烟临朝称制,改元咸清,称感天皇后。萧塔不烟在位7年后,还政于子耶律夷列。

1150年,耶律夷列即位,改元绍兴。耶律夷列在位期间普查首都虎思斡耳朵内畿18岁以上成年男子的人口,共84500户[a]。1156年,西喀喇汗国大汗伊卜拉欣三世与葛逻禄军队长官艾亚尔伯克发生冲突,双方在饥饿草原发生战争,伊卜拉欣三世战败被暴尸荒野,其子阿里·本·哈桑继任,称恰格雷汗。恰格雷汗随后对葛逻禄人展开报复,杀死其首领比古汗。葛逻禄的拉钦伯克和比古汗之子向花剌子模求助,而恰格雷汗则向西辽求援。耶律夷列派东喀喇汗国土库曼王伊卜拉欣·本·阿赫马德率军1万前去救援,双方隔粟特河对峙。经撒马尔罕的宗教人士调节,双方签订合约,恰格雷汗恢复了葛逻禄首领的军事职务,双方撤军(後來西遼把葛邏祿人安置在阿力麻里單獨管理)。

耶律夷列在位13年,于1163年去世,庙号仁宗。由于其子年幼,遗诏命其妹耶律普速完临朝称制,改元崇福,称承天太后。

\subsubsection{绍兴}


\begin{longtable}{|>{\centering\scriptsize}m{2em}|>{\centering\scriptsize}m{1.3em}|>{\centering}m{8.8em}|}
  % \caption{秦王政}\
  \toprule
  \SimHei \normalsize 年数 & \SimHei \scriptsize 公元 & \SimHei 大事件 \tabularnewline
  % \midrule
  \endfirsthead
  \toprule
  \SimHei \normalsize 年数 & \SimHei \scriptsize 公元 & \SimHei 大事件 \tabularnewline
  \midrule
  \endhead
  \midrule
  元年 & 1151 & \tabularnewline\hline
  二年 & 1152 & \tabularnewline\hline
  三年 & 1153 & \tabularnewline\hline
  四年 & 1154 & \tabularnewline\hline
  五年 & 1155 & \tabularnewline\hline
  六年 & 1156 & \tabularnewline\hline
  七年 & 1157 & \tabularnewline\hline
  八年 & 1158 & \tabularnewline\hline
  九年 & 1159 & \tabularnewline\hline
  十年 & 1160 & \tabularnewline\hline
  十一年 & 1161 & \tabularnewline\hline
  十二年 & 1162 & \tabularnewline\hline
  十三年 & 1163 & \tabularnewline
  \bottomrule
\end{longtable}

\subsection{承天太后耶律普速完\tiny(1163-1177)}

\subsubsection{生平}

耶律普速完(?-1177年)是遼仁宗耶律夷列的妹妹,為西遼第四任統治者。仁宗在1163年死後,其子尚年幼,遺詔由其妹耶律普速完權理國事,臨朝稱制,並改元崇福,號承天太后。

她的丈夫是蕭朵魯不。她與丈夫之弟樸古只沙里私通,把丈夫改為東平王,後來又殺了他。崇福十四年(1177年),蕭朵魯不之父斡里剌以兵圍其宮,射殺普速完及樸古只沙里。仁宗子耶律直魯古即位,改元天禧,是為西遼末主。

中國歷朝的臨朝稱制,皆為皇太后、皇后在君主因故無法理朝時的一種權宜之計(也有例外,如武則天曾與唐高宗並稱二聖並臨朝聽政),但耶律普速完是中國歷史上唯一一個以先朝公主與當朝君主姑母的身分臨朝稱制者,可為前無古人、後無來者。耶律普速完稱制時,又更改年號,在實質意義上已經得到等同君王的待遇和地位。

\subsubsection{崇福}

\begin{longtable}{|>{\centering\scriptsize}m{2em}|>{\centering\scriptsize}m{1.3em}|>{\centering}m{8.8em}|}
  % \caption{秦王政}\
  \toprule
  \SimHei \normalsize 年数 & \SimHei \scriptsize 公元 & \SimHei 大事件 \tabularnewline
  % \midrule
  \endfirsthead
  \toprule
  \SimHei \normalsize 年数 & \SimHei \scriptsize 公元 & \SimHei 大事件 \tabularnewline
  \midrule
  \endhead
  \midrule
  元年 & 1164 & \tabularnewline\hline
  二年 & 1165 & \tabularnewline\hline
  三年 & 1166 & \tabularnewline\hline
  四年 & 1167 & \tabularnewline\hline
  五年 & 1168 & \tabularnewline\hline
  六年 & 1169 & \tabularnewline\hline
  七年 & 1170 & \tabularnewline\hline
  八年 & 1171 & \tabularnewline\hline
  九年 & 1172 & \tabularnewline\hline
  十年 & 1173 & \tabularnewline\hline
  十一年 & 1174 & \tabularnewline\hline
  十二年 & 1175 & \tabularnewline\hline
  十三年 & 1176 & \tabularnewline\hline
  十四年 & 1177 & \tabularnewline
  \bottomrule
\end{longtable}

\subsection{天禧帝耶律直鲁古\tiny(1177-1211)}

\subsubsection{生平}

耶律直魯古(12世纪-1213年),是西遼皇帝耶律夷列的次子。姑姑耶律普速完在崇福十四年(1177年)被殺,耶律直魯古即位,改元天禧,史称天禧帝。

乃蠻王子屈出律于1208年流亡至西辽,天禧帝耶律直鲁古不仅信任他还将女儿嫁给他。天禧三十四年(1211年),屈出律以伏兵八千擒直魯古,強迫天禧帝直鲁古讓位,尊他為太上皇,皇后為皇太后。1213年,天禧帝直魯古去世。1218年,蒙古攻西遼,殺屈出律,西遼亡。

遼史(卷三十 本紀第三十):“仁宗次子直魯古即位,改元天禧,在位三十四年,天禧帝。時秋出獵,乃蠻王屈出律以伏兵八千擒之,而據其位。遂襲遼衣冠,尊直魯古為太上皇,皇后為皇太后,朝夕問起居,以侍終焉。直魯古死,遼絕。”

\subsubsection{天禧}

\begin{longtable}{|>{\centering\scriptsize}m{2em}|>{\centering\scriptsize}m{1.3em}|>{\centering}m{8.8em}|}
  % \caption{秦王政}\
  \toprule
  \SimHei \normalsize 年数 & \SimHei \scriptsize 公元 & \SimHei 大事件 \tabularnewline
  % \midrule
  \endfirsthead
  \toprule
  \SimHei \normalsize 年数 & \SimHei \scriptsize 公元 & \SimHei 大事件 \tabularnewline
  \midrule
  \endhead
  \midrule
  元年 & 1178 & \tabularnewline\hline
  二年 & 1179 & \tabularnewline\hline
  三年 & 1180 & \tabularnewline\hline
  四年 & 1181 & \tabularnewline\hline
  五年 & 1182 & \tabularnewline\hline
  六年 & 1183 & \tabularnewline\hline
  七年 & 1184 & \tabularnewline\hline
  八年 & 1185 & \tabularnewline\hline
  九年 & 1186 & \tabularnewline\hline
  十年 & 1187 & \tabularnewline\hline
  十一年 & 1188 & \tabularnewline\hline
  十二年 & 1189 & \tabularnewline\hline
  十三年 & 1190 & \tabularnewline\hline
  十四年 & 1191 & \tabularnewline\hline
  十五年 & 1192 & \tabularnewline\hline
  十六年 & 1193 & \tabularnewline\hline
  十七年 & 1194 & \tabularnewline\hline
  十八年 & 1195 & \tabularnewline\hline
  十九年 & 1196 & \tabularnewline\hline
  二十年 & 1197 & \tabularnewline\hline
  二一年 & 1198 & \tabularnewline\hline
  二二年 & 1199 & \tabularnewline\hline
  二三年 & 1200 & \tabularnewline\hline
  二四年 & 1201 & \tabularnewline\hline
  二五年 & 1202 & \tabularnewline\hline
  二六年 & 1203 & \tabularnewline\hline
  二七年 & 1204 & \tabularnewline\hline
  二八年 & 1205 & \tabularnewline\hline
  二九年 & 1206 & \tabularnewline\hline
  三十年 & 1207 & \tabularnewline\hline
  三一年 & 1208 & \tabularnewline\hline
  三二年 & 1209 & \tabularnewline\hline
  三三年 & 1210 & \tabularnewline\hline
  三四年 & 1211 & \tabularnewline
  \bottomrule
\end{longtable}

\subsection{末帝屈出律\tiny{1211-1218}}

\subsubsection{生平}

屈出律是乃蠻太陽汗之子,1204年,成吉思汗攻滅乃蠻部,太陽汗戰死。屈出律投奔其叔父不亦鲁黑汗。不亦鲁黑汗死後,屈出律又聯合蔑儿乞首领脱黑脱阿對抗成吉思汗。1208年,屈出律和脱黑脱阿在也儿的石河(今额尔齐斯河)上游被成吉思汗击败,脱黑脱阿戰死。屈出律逃奔至别失八里(今新疆维吾尔自治区吉木萨尔县境内),又抵达苦叉(今新疆维吾尔自治区库车县),他的部队缺乏給養又没有粮食,一路上纷纷散去。屈出律隨後投奔西遼,侍奉於菊儿汗耶律直鲁古。《史集》记载屈出律觐见耶律直鲁古时化妆成一名马夫,这身装扮触怒了耶律直鲁古的大臣,但得到了耶律直鲁古正妻菊儿别速的女儿渾忽公主的賞識,三天後渾忽公主便嫁给了屈出律。

随着西遼国力的衰落,附庸国高昌回鹘、花剌子模和西喀喇汗国纷纷背叛西遼,屈出律便向耶律直鲁古建议自己返回叶密立(今新疆维吾尔自治区额敏县)、海押立(今哈萨克斯坦塔尔迪库尔干)、别失八里地区召集乃蠻舊部,帮助耶律直鲁古鎮壓叛乱。耶律直鲁古封屈出律为可汗,并赠送他很多禮物。屈出律收集失散的乃蠻人,组成军隊,劫掠七河地区。同时派使者聯絡花剌子模沙阿阿拉乌丁·摩诃末,雙方约定誰先奪取西遼就占有它的土地。屈出律先擊败了西遼的军隊,劫掠了耶律直鲁古位于乌兹根的府库,隨後又進攻西遼首都虎思斡耳朵(今吉尔吉斯斯坦托克玛克东南布拉纳),但在真兀赤被耶律直鲁古擊败。屈出律返回葉密立,圖謀再次進攻。

1210年,西遼國内發生西遼軍隊燒殺劫掠首都虎思斡耳朵的事件,由于宰相马合木·太处理不當,致使軍隊纷纷離散。屈出律得知此消息后,于次年秋率軍8千“像雲中的閃電一樣”襲擊了正在外出打獵的耶律直鲁古,奪取了皇位。屈出律惺惺作態,尊耶律直鲁古为太上皇,皇后为皇太后,早晚问候他们的衣食起居。1213年,耶律直鲁古在憤恨中死去。

东喀喇汗国土库曼王穆罕默德三世起兵反抗西辽的统治,遭到耶律直鲁古的镇压,穆罕默德三世被俘。屈出律篡位后,释放了穆罕默德三世,将其送回喀什噶尔,但他不受当地贵族的欢迎,入城时被刺死于城门洞中。由于喀什噶尔不肯归附屈出律,屈出律每逢秋收时节派兵烧毁他们的庄稼。三、四年后,当地百姓因为饥荒不得已而归顺。在占领喀什噶尔后,屈出律下令在每家每户派驻一名士兵,这些士兵毫无军纪,到处烧杀抢掠。屈出律随后又派兵征服了和田。阿力麻里(今新疆维吾尔自治区霍城县一带)汗不札兒不肯服从屈出律,屈出律多次派军队征讨无果,最终趁不札兒出獵時將其擒殺。

屈出律篡位后虽然没有更改西辽的国号和政治制度,但他一改前任西辽君主的宗教自由政策,转而实行宗教迫害。屈出律原本信奉景教,后在浑忽公主的劝说下改信佛教。他用强制手段强迫西辽当地的穆斯林和基督徒改信佛教,穿戴契丹人的服装,这引起了当地人民的强烈不满。屈出律征服和田后,下令召集当地的伊斯兰教阿訇讨论教义,教长阿訇阿剌丁·摩诃末极力维护伊斯兰教,屈出律命人将其严刑拷打,强迫他改教,摩诃末不从,被屈出律下令钉死于清真寺的大门上。

1218年,成吉思汗派哲别、曷思麦里率2萬蒙古軍攻打屈出律,屈出律闻讯带领随从从喀什噶尔逃跑。哲别进入喀什噶尔后宣布宗教自由,城中居民開始對屈出律展开报复,大肆屠杀屈出律的军队。屈出律逃至巴达克山(今阿富汗巴达赫尚省),在瓦罕河谷东部的达拉兹峡谷迷路。由于当地山路崎岖难行,哲别向当地猎户许诺以屈出律随身携带的财物为条件,抓捕屈出律。屈出律被俘获后交予哲别,哲别将屈出律斩首后,命曷思麦里拿着他的首级传示于喀什噶尔、押儿牵(今新疆维吾尔自治区莎车县)、斡端(今新疆维吾尔自治区和田市)等城,城中将领皆率部投降,西辽彻底被蒙古帝国所征服。

\subsubsection{天禧}

\begin{longtable}{|>{\centering\scriptsize}m{2em}|>{\centering\scriptsize}m{1.3em}|>{\centering}m{8.8em}|}
  % \caption{秦王政}\
  \toprule
  \SimHei \normalsize 年数 & \SimHei \scriptsize 公元 & \SimHei 大事件 \tabularnewline
  % \midrule
  \endfirsthead
  \toprule
  \SimHei \normalsize 年数 & \SimHei \scriptsize 公元 & \SimHei 大事件 \tabularnewline
  \midrule
  \endhead
  \midrule
  三四年 & 1211 & \tabularnewline\hline
  三五年 & 1212 & \tabularnewline\hline
  三六年 & 1213 & \tabularnewline\hline
  三七年 & 1214 & \tabularnewline\hline
  三八年 & 1215 & \tabularnewline\hline
  三九年 & 1216 & \tabularnewline\hline
  四十年 & 1217 & \tabularnewline\hline
  四一年 & 1218 & \tabularnewline
  \bottomrule
\end{longtable}

%%% Local Variables:
%%% mode: latex
%%% TeX-engine: xetex
%%% TeX-master: "../Main"
%%% End:


%%% Local Variables:
%%% mode: latex
%%% TeX-engine: xetex
%%% TeX-master: "../Main"
%%% End:
