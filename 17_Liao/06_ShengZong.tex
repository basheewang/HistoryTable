%% -*- coding: utf-8 -*-
%% Time-stamp: <Chen Wang: 2019-12-26 10:58:03>

\section{圣宗\tiny(982-1031)}

\subsection{生平}

遼聖宗耶律隆緒(972年1月16日-1031年6月25日),遼朝第六位皇帝(982年10月14日-1031年6月25日在位),契丹名文殊奴。是遼在位最長的皇帝,在位49年。遼景宗長子,母皇后萧绰。

辽圣宗即位前曾被封為梁王。乾亨四年(982年)九月壬子(10月13日),遼景宗去世,次日,耶律隆绪登基,即辽圣宗。

他即位時,年12歲,太后蕭綽執政。983年改元統和,并将国号“大辽”改为“大契丹”。统和四年(986年),立皇后萧氏。蕭太后執政期間,進行了改革,並且勵精圖治,注重農桑,興修水利,減少賦稅,整頓吏治,訓練軍隊,使百姓富裕,國勢強盛。統和二十二年(1004年)遼聖宗与宋真宗達成澶淵之盟。

統和二十七年(1009年)聖宗全面親政後,遼朝(契丹)已進入鼎盛,基本上延續蕭太后執政時的遼朝風貌,並且還反對嚴刑峻法,不給貪官可乘之機。在位其間四方征戰,進入遼朝疆域的頂峰。

晚年时,辽圣宗迷信佛教,窮途奢侈,遼國勢走向下坡路。遼聖宗死於太平十一年六月初三日(1031年6月25日),終年61歲,葬於庆云山。謚號為文武大孝宣肅景皇帝。

元朝官修正史《辽史》脱脱等的評價是:“圣宗幼冲嗣位,政出慈闱。及宋人二道来攻,亲御甲胄,一举而复燕、云,破信、彬,再举而躏河、朔,不亦伟欤!既而侈心一启,佳兵不祥,东有茶、陀之败,西有甘州之丧,此狃于常胜之过也。然其践阼四十九年,理冤滞,举才行,察贪残,抑奢僣,录死事之子孙,振诸部之贫乏,责迎合不忠之罪,却高丽女乐之归。辽之诸帝,在位长久,令名无穷,其唯圣宗乎!”

\subsection{统合}

\begin{longtable}{|>{\centering\scriptsize}m{2em}|>{\centering\scriptsize}m{1.3em}|>{\centering}m{8.8em}|}
  % \caption{秦王政}\
  \toprule
  \SimHei \normalsize 年数 & \SimHei \scriptsize 公元 & \SimHei 大事件 \tabularnewline
  % \midrule
  \endfirsthead
  \toprule
  \SimHei \normalsize 年数 & \SimHei \scriptsize 公元 & \SimHei 大事件 \tabularnewline
  \midrule
  \endhead
  \midrule
  元年 & 983 & \tabularnewline\hline
  二年 & 984 & \tabularnewline\hline
  三年 & 985 & \tabularnewline\hline
  四年 & 986 & \tabularnewline\hline
  五年 & 987 & \tabularnewline\hline
  六年 & 988 & \tabularnewline\hline
  七年 & 989 & \tabularnewline\hline
  八年 & 990 & \tabularnewline\hline
  九年 & 991 & \tabularnewline\hline
  十年 & 992 & \tabularnewline\hline
  十一年 & 993 & \tabularnewline\hline
  十二年 & 994 & \tabularnewline\hline
  十三年 & 995 & \tabularnewline\hline
  十四年 & 996 & \tabularnewline\hline
  十五年 & 997 & \tabularnewline\hline
  十六年 & 998 & \tabularnewline\hline
  十七年 & 999 & \tabularnewline\hline
  十八年 & 1000 & \tabularnewline\hline
  十九年 & 1001 & \tabularnewline\hline
  二十年 & 1002 & \tabularnewline\hline
  二一年 & 1003 & \tabularnewline\hline
  二二年 & 1004 & \tabularnewline\hline
  二三年 & 1005 & \tabularnewline\hline
  二四年 & 1006 & \tabularnewline\hline
  二五年 & 1007 & \tabularnewline\hline
  二六年 & 1008 & \tabularnewline\hline
  二七年 & 1009 & \tabularnewline\hline
  二八年 & 1010 & \tabularnewline\hline
  二九年 & 1011 & \tabularnewline\hline
  三十年 & 1012 & \tabularnewline
  \bottomrule
\end{longtable}

\subsection{开泰}

\begin{longtable}{|>{\centering\scriptsize}m{2em}|>{\centering\scriptsize}m{1.3em}|>{\centering}m{8.8em}|}
  % \caption{秦王政}\
  \toprule
  \SimHei \normalsize 年数 & \SimHei \scriptsize 公元 & \SimHei 大事件 \tabularnewline
  % \midrule
  \endfirsthead
  \toprule
  \SimHei \normalsize 年数 & \SimHei \scriptsize 公元 & \SimHei 大事件 \tabularnewline
  \midrule
  \endhead
  \midrule
  元年 & 1012 & \tabularnewline\hline
  二年 & 1013 & \tabularnewline\hline
  三年 & 1014 & \tabularnewline\hline
  四年 & 1015 & \tabularnewline\hline
  五年 & 1016 & \tabularnewline\hline
  六年 & 1017 & \tabularnewline\hline
  七年 & 1018 & \tabularnewline\hline
  八年 & 1019 & \tabularnewline\hline
  九年 & 1020 & \tabularnewline\hline
  十年 & 1021 & \tabularnewline
  \bottomrule
\end{longtable}

\subsection{太平}

\begin{longtable}{|>{\centering\scriptsize}m{2em}|>{\centering\scriptsize}m{1.3em}|>{\centering}m{8.8em}|}
  % \caption{秦王政}\
  \toprule
  \SimHei \normalsize 年数 & \SimHei \scriptsize 公元 & \SimHei 大事件 \tabularnewline
  % \midrule
  \endfirsthead
  \toprule
  \SimHei \normalsize 年数 & \SimHei \scriptsize 公元 & \SimHei 大事件 \tabularnewline
  \midrule
  \endhead
  \midrule
  元年 & 1021 & \tabularnewline\hline
  二年 & 1022 & \tabularnewline\hline
  三年 & 1023 & \tabularnewline\hline
  四年 & 1024 & \tabularnewline\hline
  五年 & 1025 & \tabularnewline\hline
  六年 & 1026 & \tabularnewline\hline
  七年 & 1027 & \tabularnewline\hline
  八年 & 1028 & \tabularnewline\hline
  九年 & 1029 & \tabularnewline\hline
  十年 & 1030 & \tabularnewline\hline
  十一年 & 1031 & \tabularnewline
  \bottomrule
\end{longtable}



%%% Local Variables:
%%% mode: latex
%%% TeX-engine: xetex
%%% TeX-master: "../Main"
%%% End:
