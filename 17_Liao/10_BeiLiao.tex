%% -*- coding: utf-8 -*-
%% Time-stamp: <Chen Wang: 2019-10-15 16:34:10>

\section{北辽\tiny(1122)}

北遼,於1122年3月立國,是時辽朝天祚帝被金兵所迫,流亡夹山,耶律淳在燕京被耶律大石等人擁立為君主,是為北遼的開始。1122年6月24日,耶律淳病死,德妃蕭普賢女以皇太后身份攝政,期间击退宋朝进攻(宣和北伐)。1123年2月2日,金朝攻佔燕京,蕭德妃和耶律大石投奔天祚帝,北遼滅亡,國祚不足一年。後來,萧德妃因為謀反而被殺,但耶律大石卻得到赦免。

\subsection{宣宗\tiny(1122)}

遼宣宗耶律淳(1063年-1122年),小字涅里,是北遼開國皇帝,為遼兴宗第四子宋魏國王耶律和鲁斡之子。淳一出生就由其祖母遼興宗的仁懿皇后撫養,長大成人之後,好文學。遼道宗太子耶律濬被殺害之後,遼道宗曾打算立侄子淳為嗣,後罷,封北平郡王,出為彰聖等軍節度使。

天祚帝即位。乾統元年(1101年)封耶律和鲁斡為天下兵馬大元帅,此意味著有皇位的繼承權,封淳為鄭王。乾統三年(1103年)封耶律和鲁斡為皇太叔,進封淳為越國王。乾統六年(1106年),拜為南府宰相,創議制訂兩府禮儀,進封為魏國王。乾統十年(1110年),耶律和鲁斡去世,淳襲南京留職,冬夏入朝,寵冠諸王。

天慶五年(1115年),耶律章奴謀反,打算迎立耶律淳為帝。耶律淳不從。次年(1116年)六月,耶律淳進封秦晉國王,拜都元帥,賜金券,免漢拜禮,不名。

保大二年(1122年)正月,金軍攻克遼中京,天祚帝被金兵所迫,流亡夾山。奚王回離保和林牙耶律大石援引唐肅宗靈武稱帝的例子,勸說耶律淳稱帝。三月,淳即皇帝位,百官上尊號為天錫皇帝,改年號建福元年,遥降天祚皇帝为湘阴王,封妻蕭普賢女為德妃,並遣使奉表于金國,乞为附庸。

六月,耶律淳事未完成就病死,終年六十歲。百官上諡号孝章皇帝,庙号宣宗,葬燕京西部的香山永安陵。

\subsubsection{建福}


\begin{longtable}{|>{\centering\scriptsize}m{2em}|>{\centering\scriptsize}m{1.3em}|>{\centering}m{8.8em}|}
  % \caption{秦王政}\
  \toprule
  \SimHei \normalsize 年数 & \SimHei \scriptsize 公元 & \SimHei 大事件 \tabularnewline
  % \midrule
  \endfirsthead
  \toprule
  \SimHei \normalsize 年数 & \SimHei \scriptsize 公元 & \SimHei 大事件 \tabularnewline
  \midrule
  \endhead
  \midrule
  元年 & 1122 & \tabularnewline
  \bottomrule
\end{longtable}

\subsection{萧普贤女\tiny(1122)}

蕭普賢女(?-1123年),為北遼宣宗耶律淳的德妃,宣宗遺詔立天祚帝耶律延禧第五子耶律定為皇帝,但他在天祚帝身邊,不在燕京,只能遙立。德妃被立為皇太后,稱制,改建福元年為德興元年。

此時大臣李處溫父子覺得前景不妙,打算向南私通宋的童貫,欲劫持德妃納土於宋。向北私通金人,作金的內應。後她發現他私通宋、金的罪行把他拘捕並賜死。

當年十一月,德妃五次上表給金朝,只要允許立耶律定為北遼皇帝,其他條件均答應,金人不許,她只好派兵把守居庸關,沒能守住,金兵直奔燕京。德妃帶著隨從的官員投靠天祚帝,天祚帝將她誅殺。

\subsubsection{德兴}

\begin{longtable}{|>{\centering\scriptsize}m{2em}|>{\centering\scriptsize}m{1.3em}|>{\centering}m{8.8em}|}
  % \caption{秦王政}\
  \toprule
  \SimHei \normalsize 年数 & \SimHei \scriptsize 公元 & \SimHei 大事件 \tabularnewline
  % \midrule
  \endfirsthead
  \toprule
  \SimHei \normalsize 年数 & \SimHei \scriptsize 公元 & \SimHei 大事件 \tabularnewline
  \midrule
  \endhead
  \midrule
  元年 & 1122 & \tabularnewline
  \bottomrule
\end{longtable}



%%% Local Variables:
%%% mode: latex
%%% TeX-engine: xetex
%%% TeX-master: "../Main"
%%% End:
