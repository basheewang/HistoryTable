%% -*- coding: utf-8 -*-
%% Time-stamp: <Chen Wang: 2018-10-30 16:26:11>

\subsection{魏晋南北朝}

\begin{longtable}{|>{\centering\namefont\heiti}m{2em}|>{\centering\tiny}m{3.0em}|>{\xzfont\kaiti}m{7em}|}
  % \caption{秦王政}\
  \toprule
  \SimHei \normalsize 姓名 & \SimHei \normalsize 异名 & \SimHei \normalsize \hspace{2.5em}小传 \tabularnewline
  % \midrule
  \endfirsthead
  \toprule
  \SimHei \normalsize 姓名 & \SimHei \normalsize 异名 & \SimHei \normalsize \hspace{2.5em}小传 \tabularnewline 
  \midrule
  \endhead
  \midrule
  张协 & \begin{description}
  \item[字] 景阳
  \item[号] 
  \item[谥] 
  \item[尊] 
  \item[生] 河北安平
  \end{description} & 张协(?~307?),字景阳。 西晋文学家,安平(今属河北省)人。父亲张收,蜀郡太守。张协少有俊才,与兄长张载齐名。曾任公府掾、秘书郎、华阳令等职。永宁元年(301年),为成都王、征北将军司马颖的从事中郎,后迁中书侍郎,转河间内史,治郡清简。惠帝末年,天下纷乱,他辞官隐居,以吟咏自娱。永嘉初,复征为黄门侍郎,托病不就。后逝于家。与其兄张载、其弟张亢,都是西晋著名的文学家,时称“三张”。 \tabularnewline\hline
  潘岳 & \begin{description}
  \item[字] 安仁
  \item[号] 
  \item[谥] 
  \item[尊] 
  \item[生] 河南中牟
  \end{description} & 潘安(公元247年―公元300年),即潘岳,字安仁。河南中牟人。西晋著名文学家、政治家,潘安之名始于杜甫《花底》诗“恐是潘安县,堪留卫玠车。”后世遂以潘安称焉。美姿仪,少以才名闻世,他性轻躁,趋于世利,与石崇等谄事贾谧,每候其出,辄望尘而拜。与石崇、陆机、刘琨、左思等并为“贾谧二十四友”,潘安为首。潘安被誉为“古代第一美男”。潘岳在文学上与陆机并称“潘江陆海”,钟嵘《诗品》称“陆才如海,潘才如江”,王勃《滕王阁序》“请洒潘江,各倾陆海云尔。” \tabularnewline\hline
  潘尼 & \begin{description}
  \item[字] 正叔
  \item[号] 
  \item[谥] 
  \item[尊] 
  \item[生] 荥阳中牟
  \end{description} & 潘尼(约250~约311年),字正叔,荥阳中牟人(在今河南城关镇大潘庄),西晋文学家。祖父潘勖,中国东汉东海相。父亲潘满,平原内史。潘岳之侄,少有才,与潘岳俱以文章知名,并称“两潘”。潘尼生情稳静恬淡,不与人争利,安心研读,专志著述。 \tabularnewline\hline
  陆机 & \begin{description}
  \item[字] 士衡
  \item[号] 
  \item[谥] 
  \item[尊] 
  \item[生] 江苏苏州
  \end{description} & 陆机(261年-303年),字士衡,吴郡吴县(今江苏苏州)人。西晋著名文学家、书法家。出身吴郡陆氏,为孙吴丞相陆逊之孙、大司马陆抗第四子,与其弟陆云合称“二陆”,又与顾荣、陆云并称“洛阳三俊”。陆机在孙吴时曾任牙门将,吴亡后出仕西晋,太康十年(289年),陆机兄弟来到洛阳,文才倾动一时,受太常张华赏识,此后名气大振。时有“二陆入洛,三张减价”之说。历任太傅祭酒、吴国郎中令、著作郎等职,与贾谧等结为“金谷二十四友”。 \tabularnewline\hline
  陆云 & \begin{description}
  \item[字] 士龙
  \item[号] 
  \item[谥] 
  \item[尊] 陆清河
  \item[生] 江苏苏州
  \end{description} & 陆云(262年-303年),字士龙,吴郡吴县(今江苏苏州)人,西晋官员、文学家,东吴丞相陆逊之孙,大司马陆抗第五子。与其兄陆机合称“二陆”,曾任清河内史,故世称“陆清河”。陆机死于“八王之乱”而被夷三族后,陆云也为之牵连入狱。尽管许多人上疏司马颖请求不要株连陆云,但他最终还是遇害了。时年四十二岁,无子,生有二女。由门生故吏迎葬于清河。 \tabularnewline\hline
  左思 & \begin{description}
  \item[字] 泰冲
  \item[号] 
  \item[谥] 
  \item[尊] 
  \item[生] 山东淄博
  \end{description} & 左思(约250~305),字泰冲,齐国临淄(今山东淄博)人。西晋著名文学家,其《三都赋》颇被当时称颂,造成“洛阳纸贵”。另外,其《咏史诗》《娇女诗》也很有名。其诗文语言质朴凝练。后人辑有《左太冲集》。 \tabularnewline\hline
  卢谌 & \begin{description}
  \item[字] 子谅
  \item[号] 
  \item[谥] 
  \item[尊] 
  \item[生] 河北涿县
  \end{description} & 卢谌(284─351),字子谅,范阳涿(今属河北涿县)人,晋代文学家。曹魏司空卢毓曾孙。西晋卫尉卿卢珽之孙,尚书卢志长子。晋朝历任司空主簿、从事中郎、幽州别驾。后赵、冉魏时官至侍中、中书监。卢谌最初担任太尉椽。311年,洛阳失陷,随父北依刘琨,途中被刘粲所掳。312年,辗转归于姨父刘琨,受到青睐。315年,刘琨为司空,任卢谌为主簿,继转任从事中郎。316年,并州失守,随刘琨投奔幽州刺史段匹磾,匹磾以卢谌幽州别驾。318年,刘琨为匹磾所拘。期间,卢谌与刘琨以诗相互赠答,写有《答刘琨诗二首》《赠刘琨诗二十首》。刘琨被害,卢谌前往辽西依附段末波。朝廷不敢为其吊祭,后卢谌等上表申理,文旨甚是切恳。石虎攻取辽西后,进入后赵,历任中书侍郎、国子祭酒、侍中、中书监等职。350年,冉闵诛石氏、灭后赵,卢谌在冉魏任中书监,后在襄国遇害。时年67岁。卢谌为人清敏、才思敏捷,喜读老庄,又善于写文章。他著有《祭法》《庄子注》及文集十卷,其中有些诗篇流传至今。 \tabularnewline\hline
  孙绰 & \begin{description}
  \item[字] 兴公
  \item[号] 
  \item[谥] 
  \item[尊] 
  \item[生] 山西平遥
  \end{description} & 孙绰(314—371),字兴公,东晋玄言诗人。中都(今山西平遥)人,后迁会稽(今浙江绍兴)。曾任临海章安令,在任时写过著名的《天台山赋》。其善书博学,是参加王羲之兰亭修禊的诗人和书法家。 \tabularnewline\hline
    颜延之 & \begin{description}
    \item[字] 延年
    \item[号] 
    \item[谥] 
    \item[尊] 
    \item[生] 山东临沂
    \end{description} & 颜延之(384~456年),字延年,南朝宋文学家。琅邪临沂(今山东临沂)人。曾祖含,右光禄大夫。祖约,零陵太守。父显,护军司马。少孤贫,居陋室,好读书,无所不览,文章之美,冠绝当时,与谢灵运并称“颜谢”。 \tabularnewline\hline
    谢灵运 & \begin{description}
    \item[字] 灵运
    \item[号] 
    \item[谥] 
    \item[尊] 谢客
    \item[生] 河南太康
    \end{description} & 谢灵运(385年—433年),原名公义,字灵运,以字行于世,小名客儿,世称谢客。南北朝时期杰出的诗人、文学家、旅行家、道家。谢灵运出身陈郡谢氏,祖籍陈郡阳夏(今河南太康县),生于会稽始宁(今绍兴市嵊州市三界镇)。为东晋名将谢玄之孙、秘书郎谢瑍之子。东晋时世袭为康乐公,世称谢康乐。曾出任大司马行军参军、抚军将军记室参军、太尉参军等职。刘宋代晋后,降封康乐侯,历任永嘉太守、秘书监、临川内史,元嘉十年(433年)被宋文帝刘义隆以“叛逆”罪名杀害,年四十九。谢灵运少即好学,博览群书,工诗善文。其诗与颜延之齐名,并称“颜谢”,开创了中国文学史上的山水诗派,他还兼通史学,擅书法,曾翻译外来佛经,并奉诏撰《晋书》。明人辑有《谢康乐集》。 \tabularnewline\hline
    鲍照 & \begin{description}
    \item[字] 明远
    \item[号] 
    \item[谥] 
    \item[尊] 
    \item[生] 山工临沂
    \end{description} & 鲍照(414年-466年),字明远,东海郡人(今属山东临沂市兰陵县长城镇),中国南朝宋杰出的文学家、诗人。宋元嘉中,临川王刘义庆“招聚文学之士,近远必至”,鲍照以辞章之美而被看重,遂引为“佐史国臣”。元嘉十六年因献诗而被宋文帝用为中书令、秣稜令。大明五年出任前军参军,故世称“鲍参军”。泰始二年刘子顼起兵反明帝失败,鲍照死于乱军中。鲍照与颜延之、谢灵运同为宋元嘉时代的著名诗人,合称“元嘉三大家”,其诗歌注意描写山水,讲究对仗和辞藻。他长于乐府诗,其七言诗对唐代诗歌的发展起了重要作用。世称“元嘉体”,现有《鲍参军集》传世。鲍照和庾信合称“南照北信”。 \tabularnewline\hline
    谢朓 & \begin{description}
    \item[字] 玄晖
    \item[号] 
    \item[谥] 
    \item[尊] 
    \item[生] 河南太康
    \end{description} & 谢朓(464—499),字玄晖,汉族,陈郡阳夏(今河南太康县)人。南朝齐杰出的山水诗人,出身高门士族,与“大谢”谢灵运同族,世称“小谢”。19岁解褐豫章王太尉行参军。永明五年(487),与竟陵王萧子良西邸之游,初任其功曹、文学,为“竟陵八友”之一。永明九年(491),随随王萧子隆至荆州,十一年还京,为骠骑咨议、领记室。建武二年(495),出为宣城太守。两年后,复返京为中书郎。之后,又出为南东海太守,寻迁尚书吏部郎,又称谢宣城、谢吏部。东昏侯永元元年(499)遭始安王萧遥光诬陷,死狱中,时年36岁。曾与沈约等共创“永明体”。今存诗二百余首,多描写自然景物,间亦直抒怀抱,诗风清新秀丽,圆美流转,善于发端,时有佳句;又平仄协调,对偶工整,开启唐代律绝之先河。 \tabularnewline\hline
    江淹 & \begin{description}
    \item[字] 文通
    \item[号] 
    \item[谥] 
    \item[尊] 
    \item[生]河南商丘
    \end{description} & 江淹(444年—505年),字文通,南朝著名政治家、文学家,历仕三朝,宋州济阳考城(今河南省商丘市民权县程庄镇江集村)人。江淹少时孤贫好学,六岁能诗。文章华著,十三岁丧父。二十岁左右在新安王刘子鸾幕下任职,开始其政治生涯,齐高帝闻其才,召授尚书驾部郎,骠骑参军事;明帝时为御史中丞,先后弹劾中书令谢朏等人;武帝时任骠骑将军兼尚书左丞,历仕南朝宋、齐、梁三代。 \tabularnewline\hline
    丘迟 & \begin{description}
    \item[字] 希范
    \item[号] 
    \item[谥] 
    \item[尊] 
    \item[生] 浙江湖州
    \end{description} & 丘迟(464年-508年),字希范,中国南朝文学家,吴兴乌程(今属浙江省湖州市)人。父丘灵鞠,南齐太中大夫,亦为当时知名文人。丘迟八岁能文,初仕南齐,官至殿中郎、车骑录事参军。后投入萧衍幕中,为其所重,其后萧衍代齐为帝建立南梁的一应劝进文书均为丘迟所作。天监四年(505年)随萧宏北伐,为其记室,以一封《与陈伯之书》成功招降投奔北魏的原南齐将领陈伯之来降,后历任永嘉太守、拜中书郎,再升任司徒从事中郎。天监七年,以四十五岁卒于官。 \tabularnewline\hline
    钟嵘 & \begin{description}
    \item[字] 仲伟
    \item[号] 
    \item[谥] 
    \item[尊] 
    \item[生] 不详
    \end{description} & 钟嵘(约468—约518), 中国南朝文学批评家。字仲伟。颍川长社(今河南许昌长葛市)人。齐代官至司徒行参军。入梁,历任中军临川王行参军、西中郎将晋安王记室。梁武帝天监十二年(513)以后,仿汉代“九品论人,七略裁士”的著作先例,写成诗歌评论专著《诗品》。以五言诗为主,全书将两汉至梁作家122人,分为上、中、下三品进行评论,故名为《诗品》。《隋书·经籍志》著录此书,书名为《诗评》,这是因为除品第之外,还就作品评论其优劣。后以《诗品》定名。在《诗品》中,钟嵘提倡风力,反对玄言;主张音韵自然和谐,反对人为的声病说;主张“直寻”,反对用典,提出了一套比较系统的诗歌品评的标准。钟嵘(约468—约518), 中国南朝文学批评家。字仲伟。颍川长社(今河南许昌长葛市)人。齐代官至司徒行参军。入梁,历任中军临川王行参军、西中郎将晋安王记室。梁武帝天监十二年(513)以后,仿汉代“九品论人,七略裁士”的著作先例,写成诗歌评论专著《诗品》。以五言诗为主,全书将两汉至梁作家122人,分为上、中、下三品进行评论,故名为《诗品》。《隋书·经籍志》著录此书,书名为《诗评》,这是因为除品第之外,还就作品评论其优劣。后以《诗品》定名。在《诗品》中,钟嵘提倡风力,反对玄言;主张音韵自然和谐,反对人为的声病说;主张“直寻”,反对用典,提出了一套比较系统的诗歌品评的标准。 \tabularnewline\hline

  \bottomrule
\end{longtable}


%%% Local Variables:
%%% mode: latex
%%% TeX-engine: xetex
%%% TeX-master: "../../Main"
%%% End:
