%% -*- coding: utf-8 -*-
%% Time-stamp: <Chen Wang: 2018-07-21 20:29:31>

\subsection{清}

\begin{longtable}{|>{\centering\namefont\heiti}m{2em}|>{\centering\tiny}m{3.0em}|>{\xzfont\kaiti}m{7.3em}|}
  % \caption{秦王政}\
  \toprule
  \SimHei \normalsize 姓名 & \SimHei \normalsize 异名 & \SimHei \normalsize \hspace{2.5em}小传 \tabularnewline
  % \midrule
  \endfirsthead
  \toprule
  \SimHei \normalsize 姓名 & \SimHei \normalsize 异名 & \SimHei \normalsize \hspace{2.5em}小传 \tabularnewline 
  \midrule
  \endhead
  \midrule
  吴伟业 & \begin{description}
  \item[字] 骏公
  \item[号] 梅村
  \item[谥] 
  \item[尊] 
  \item[生] 江苏昆山
  \end{description} & 吴伟业(1609年6月21日-1672年1月23日),字骏公,号梅村,祖籍南直隶苏州府昆山县(今江苏省昆山市),祖父始迁居太仓州(今江苏省苏州市太仓市),明末清初著名诗人、政治人物,长于七言歌行,初学“长庆体”,后自成新吟,后人称之为“梅村体”。与钱谦益、龚鼎孳并称为江左三大家。吴伟业著有《梅村家藏稿》、《梅村诗馀》,传奇《秣陵春》,杂剧《通天台》、《临春阁》,史料《绥寇纪略》等。其诗情深文丽,宫商和谐,敷衍成长篇七言,蔚然可观,在清朝被称为“本朝词家之领袖”。 \tabularnewline\hline
  钱谦益 & \begin{description}
  \item[字] 受之
  \item[号] 木斋\\绛云楼主
  \item[谥] 
  \item[尊] 虞山\\宗伯
  \item[生] 江苏常熟
  \end{description} & 钱谦益(1582年10月22日-1664年6月17日),字受之,号牧斋,晚号绛云楼主人、蒙叟、东涧老人,又因其住址而称虞山、因其职位而称宗伯,直隶常熟县(今江苏省苏州市常熟市)人。作为明末清初时期文学领域的集大成者,钱谦益领导这一时期的文坛长达五十年。在政治上钱视为东林党或复社人士。明朝时四次出仕,官至礼部尚书。后在南京降清,任礼部侍郎五个月,被视作“贰臣”。辞官后投入反清复明运动,为遗民义士接纳,更成为联络东南与西南抗清复明势力的总枢纽。后钱谦益的诗文被乾隆帝下诏禁毁。陈寅恪认为其是“复国之英雄”,“应恕其前此失节之愆,而嘉其后来赎罪之意,始可称为平心之论”,并称钱与其妻柳如是的诗文足以“表彰我民族独立之精神,自由之思想”。钱谦益学问渊博,反对竟陵派“尖新”、“鬼趣”的文风,倡言“情真”、“情至”,主张具“独至之性,旁出之情,偏诣之学”。 \tabularnewline\hline
  李渔 & \begin{description}
  \item[字] 谪凡
  \item[号] 笠翁\\蟹仙
  \item[谥] 
  \item[尊] 
  \item[生] 浙江兰溪
  \end{description} & 李渔(1611年9月13日-1680年2月12日),初名仙侣,后改名渔,字谪凡,号笠翁,后人常称之蟹仙。明末清初文学家、戏曲家,曾经评定《四大奇书》,祖籍浙江省兰溪县(今浙江省兰溪市)夏李村,后来祖父随“兰溪帮”到了江苏如皋做种药材生意。著有《凰求凤》、《玉搔头》等戏剧,《肉蒲团》、《觉世名言十二楼》、《无声戏》、《连城璧》等小说,与《闲情偶寄》等书。 \tabularnewline\hline
  冯班 & \begin{description}
  \item[字] 定远
  \item[号] 钝吟老人
  \item[谥] 
  \item[尊] 
  \item[生] 江苏常熟
  \end{description} & 冯班(1602年-1671年),字定远,号钝吟老人,江苏常熟人。生于万历三十年(1602)。早年为诸生。从钱谦益学诗,与兄冯舒齐名,称为“海虞二冯”。明亡后不仕,常常就座中恸哭,人称其为“二痴”。冯班是虞山诗派的重要人物,钱谦益称冯班之诗“沈酣六代,出入于义山、牧之、庭筠之间”,论诗讲究“无字无来历气”,反对严羽《沧浪诗话》的妙悟说,其《钝吟杂录》卷五《严氏纠谬》专驳严说。吴乔推崇贺裳、冯班,称《载酒园诗话》、《钝吟杂录》与自己的《围炉诗话》为“谈诗三绝”,书中多引贺、冯之语。康熙十年(1671)卒。赵执信尝谒其墓,写“私淑门人”刺焚冢前。有《钝吟集》、《钝吟杂录》、《钝吟书要》和《钝吟诗文稿》等。 \tabularnewline\hline
  贺裳 & \begin{description}
  \item[字] 黄公
  \item[号] 白凤词人\\檗斋
  \item[谥] 
  \item[尊] 
  \item[生] 丹阳
  \end{description} & 贺裳,字黄公,号檗斋,别号白凤词人。丹阳人。生卒年不详。明末入太学,崇祯二年加入复社。入清为诸生。工于词,长于批评,“于诗有深得,而又能详读宋人之诗,持论至当。”吴乔推崇贺裳与冯班,称贺的《载酒园诗话》、冯的《钝吟杂录》与自著《围炉诗话》为“谈诗三绝”,书中多引贺、冯之语。著有《载酒园诗话》三卷、《红牙词》、《史折》等。 \tabularnewline\hline
  吴乔 & \begin{description}
  \item[字] 修龄
  \item[号] 
  \item[谥] 
  \item[尊] 
  \item[生] 江苏常熟
  \end{description} & 吴乔(1611~1695),原名殳,字修龄,江南太仓(今属江苏)人,入赘昆山。明崇祯十一年诸生,寻被斥;字不详,生卒年不详,属蜀汉至成汉期间,蜀车骑将军吴壹之孙。有《载酒园诗话》、《古宫词》、《托物草》、《好山诗》、《舒拂集》等。 \tabularnewline\hline
  王夫之 & \begin{description}
  \item[字] 而农
  \item[号] 姜斋
  \item[谥] 
  \item[尊] 船山先生
  \item[生] 湖南衡阳
  \end{description} & 王夫之(1619年-1692年,即万历四十七年-康熙三十一年),湖广衡阳县人,杰出的思想家、哲学家、明末清初大儒。字而农,号姜斋、又号夕堂,或署一瓢道人、双髻外史,自署船山病叟、南岳遗民,晚年隐居于石船山麓,世称遂称船山先生,主要著作有《周易外传》、《读通鉴论》等,后汇编为《船山遗书》。与顾炎武、黄宗羲并称明清之际三大思想家。王夫之生前著有《周易外传》、《黄书》、《尚书引义》、《永历实录》、《春秋世论》、《噩梦》、《读通鉴论》、《宋论》等书。 \tabularnewline\hline
  邓汉仪 & \begin{description}
  \item[字] 孝威
  \item[号] 旧山
  \item[谥] 
  \item[尊] 
  \item[生] 江苏苏州
  \end{description} & 邓汉仪(1617年-1689年),字孝威,号旧山,别号旧山农、钵叟。江南苏州府吴县洞庭琦里人。邓旭之弟。清顺治元年(1644年),迁居泰州,不仕清,与吴梅村、龚鼎孳友好,早负诗名,有《题息夫人庙》诗: “千古艰难惟一死,伤心岂独息夫人。”。曾纂有《江南通志》。康熙十八年(1679年),召试博学鸿儒,不第,以年老授中书舍人。著有《淮阴集》、《官梅集》、《过岭集》、《甬东集》、《濠梁集》、《燕薹集》、《被征集》、《慎墨堂笔记》一卷,《诗观》四集,《箫楼集》等。 \tabularnewline\hline
  周在浚 & \begin{description}
  \item[字] 雪客
  \item[号] 犁庄\\仓谷
  \item[谥] 
  \item[尊] 
  \item[生] 河南开封
  \end{description} & 清藏书家。字雪客,号梨庄,一号苍谷,又号耐龛。祥符(今河南开封)人。约公元一六七五年前后在世,周亮工之子。和著名藏书家黄虞稷合编纂目录《征刻唐宋秘本书目》1卷、附《考证》1卷。《征刻书启五先生事略》1卷。著有《云烟过眼录》、《晋碑》、《南唐书注》、《大梁守城志》、《黎庄集》、《遗谷集》、《天发神谶碑考》、《秋水轩集》等。 \tabularnewline\hline
  毛奇龄 & \begin{description}
  \item[字] 大可
  \item[号] 西河
  \item[谥] 
  \item[尊] 
  \item[生] 浙江萧山
  \end{description} & 毛奇龄(1629年10月28日-1713年),字大可,又字于一,号西河,又号河右、初晴、晚晴。浙江萧山人。明末清初经学家、文学家。毛奇龄之文章,“纵横博辨,傲睨一世”,[4]他反对朱子学,他的弟子收集其旧文编撰《四书改错》以攻击朱熹《四书集注》。清初《四库全书》收录其著作二十八种,见于《存目》的三十五种,为《四库全书》中个人著作被收录最多的一位。 \tabularnewline\hline
  邹祗谟 & \begin{description}
  \item[字] 訏士
  \item[号] 程村
  \item[谥] 
  \item[尊] 
  \item[生] 江南武进
  \end{description} & 邹祇(zhǐ)谟(1627年-1670年),字𬣙士,号程村,江南武进人。清朝文学家。同进士出身。天启七年(1627年)出生。读书过目不忘,顺治十五年(1658年)登戊戌科孙承恩榜进士,顺治十八年以逋粮案黜职,遂不复仕。著有《丽农词》二卷,与王士祯《衍波词》、彭孙遹《延露词》并称“三名家词”。工于诗,与陈维崧、黄永、 董以宁号“毗陵四子”。又与王士祯编《倚声初集》,收集一千九百余首,于清初词风影响甚巨。 康熙九年卒。此外著有《远志斋集》。 \tabularnewline\hline
  王士祯 & \begin{description}
  \item[字] 贻上
  \item[号] 阮亭\\渔洋山人
  \item[谥] 文简
  \item[尊] 王渔洋
  \item[生] 山东桓台
  \end{description} & 王士禛(1634年9月17日-1711年6月26日),赐名士祯,小名豫孙,字贻上,号阮亭,别号渔洋山人,人称王渔洋,谥文简。山东新城(今山东桓台)人,清代著名文人,进士出身,康熙年间官至刑部尚书。工诗文,勤著述,著作有《渔洋山人精华录》、《池北偶谈》等五百余种。渔洋与长兄王士禄、二兄王士禧、三兄王士祜皆有诗名。其一生著述达500余种,作诗4000余首,主要有《渔洋山人精华录》、《蚕尾集》、《池北偶谈》、《香祖笔记》、《居易录》、《古夫于亭杂录》、《分甘余话》、《渔洋文略》、《渔洋诗集》、《带经堂集》、《感旧集》等。作中间有明季入清之家事。 \tabularnewline\hline
  邵长蘅 & \begin{description}
  \item[字] 子湘
  \item[号] 青门山人
  \item[谥] 
  \item[尊] 
  \item[生] 江苏常州
  \end{description} & 邵长蘅(1637年-1704年),字子湘,号青门山人,江苏武进人。生于明思宗崇祯十年(1637年),读书一目数行,十岁补诸生,康熙中曾应博学鸿词科。江苏巡抚宋荦聘致幕中。善写文章,为王士禛、汪琬所称道,主张为文必多读书[1]。卒于清圣祖康熙四十三年(1704年)。著有《青门集》、《八大山人传》。 \tabularnewline\hline
  李光地 & \begin{description}
  \item[字] 晋卿
  \item[号] 厚庵\\榕村
  \item[谥] 
  \item[尊] 安溪先生
  \item[生] 福建泉州
  \end{description} & 李光地(1642年-1718年),字晋卿,号厚庵,又号榕村,福建泉州安溪湖头人,闽南人。清圣祖康熙九年(1670年)登进士第五名,官至直隶巡抚、吏部尚书、文渊阁大学士。1681年并推保荐施琅领军,结束明郑;是清初著名的政治人物与理学家。同时代的学者尊称为“安溪先生”,或“安溪李相国”。李光地研究理学,倡言礼乐,实行海禁措施,导致近海百里无人烟,限制了农耕渔矿多种产业的发展,对康熙中年的决策有决定性的影响。晚年的李光地仍大受康熙宠信,出任吏部尚书、文渊阁大学士等职。康熙称他“谨慎清勤,始终一节,学问渊博。朕知之最真,知朕亦无过光地者”。太子允礽被废后,李光地开始辅助后来的雍正帝。雍正帝称李光地为“一代之完人”。 \tabularnewline\hline
  阎若璩 & \begin{description}
  \item[字] 百诗
  \item[号] 潜丘
  \item[谥] 
  \item[尊] 
  \item[生] 山西太原
  \end{description} & 阎若璩(1636年-1704年)字百诗,号潜丘。清初经学家、学者。山西太原人。一生勤奋著书,著有《尚书古文疏证》、《四书释地》、《潜邱札记》、《困学记闻注》、《孟子生逐年月考》、《眷西堂集》等。又曾为顾炎武《日知录》订正错误。其中《尚书古文疏证》八卷,引经据典,确定《古文尚书》为东晋梅赜所伪著。 \tabularnewline\hline
  赵执信 & \begin{description}
  \item[字] 伸符
  \item[号] 秋谷\\饴山老人
  \item[谥] 
  \item[尊] 
  \item[生] 山东淄博
  \end{description} & 赵执信[shēn](1662~1744)清代诗人、诗论家、书法家。字伸符,号秋谷,晚号饴山老人、知如老人。山东省淄博市博山人。十四岁中秀才,十七岁中举人,十八岁中进士,后任右春坊右赞善兼翰林院检讨。二十八岁因佟皇后丧葬期间观看洪升所作《长生殿》戏剧,被劾革职。此后五十年间,终身不仕,徜徉林壑。赵执信为王士祯甥婿,然论诗与其异趣,强调“文意为主,言语为役”。所作诗文深沉峭拔,亦不乏反映民生疾苦的篇目。赵执信的著作已经刊行的有《饴山诗集》十九卷,《饴山文集》十二卷,《诗余》一卷,《谈龙录》一卷,《声调谱》一卷,《礼俗权衡》两卷等。 \tabularnewline\hline
  沈德潜 & \begin{description}
  \item[字] 碻士
  \item[号] 归愚
  \item[谥] 
  \item[尊] 
  \item[生] 江苏苏州
  \end{description} & 沈德潜(1673年-1769年),字碻士(碻读音què),号归愚,江苏苏州人,清代政治人物、诗人。他在诗歌理论方面主张格调说,反对钱谦益之后的重视宋元诗的风潮,也与袁枚的性灵说相对立。编有《唐宋八家文读本》,另外他所编辑的隋代以前古诗选集《古诗源》、唐诗选集《唐诗别裁》、唐明清诗选集《国朝诗别裁集》代表了他的诗歌创作观念,广受欢迎。 \tabularnewline\hline
  王琦 & \begin{description}
  \item[字] 琢崖
  \item[号] 
  \item[谥] 
  \item[尊] 
  \item[生] 浙江杭州
  \end{description} & 王琦,字琢崖,清代钱塘人,乾隆时期的有名学者。曾注《李太白文集》三十六卷、《李长吉歌诗汇解》五卷,并帮助赵殿成注释《王右丞集》中的佛教典故。 \tabularnewline\hline
  袁枚 & \begin{description}
  \item[字] 子才
  \item[号] 简斋\\随园老人
  \item[谥] 
  \item[尊] 
  \item[生] 浙江杭州
  \end{description} & 袁枚(1716年-1797年),清代诗人,散文家。字子才,号简斋,别号随园老人,时称随园先生,浙江钱塘县(今浙江杭州)人,祖籍浙江慈谿[1][2],年廿四中进士,曾官溧水、江浦、沭阳、江宁等地知县,不到卅八岁即辞官还乡,致仕之后因投资地产有道,家财万贯。袁枚擅长诗、赋、制艺,能写骈文、小品文、笔记,乾隆时期为诗坛盟主,又为“清代骈文八大家”、“江右三大家”之一,文笔亦与大学士直隶纪昀齐名,时称“南袁北纪”。其喜好广泛,甚至编写食谱、志怪小说,著有《小仓山房文集》、《随园诗话》、《子不语》、《祭妹文》等。书信亦有名,其《小仓山房尺牍》与许葭村《秋水轩尺牍》、龚未斋《雪鸿轩尺牍》,人称“清代三大尺牍”。袁枚生平喜称人善、奖掖士类,也提倡女性文学,广收女弟子。不喜理学、汉学,追求自由,反对统一思想,他说“物之不齐,物之情也,天亦不能做主,而况于人乎?”,故被当时的许多文人严厉批判,袁枚依然悠哉度日,在文坛享有盛名。 \tabularnewline\hline
  纪昀 & \begin{description}
  \item[字] 晓岚
  \item[号] 石云
  \item[谥] 文达
  \item[尊] 
  \item[生] 河北献县
  \end{description} & 纪昀(雍正2年六月十五日-嘉庆10年二月十四日,即1724年7月26日-1805年3月14日),字晓岚,又字春帆,晚号石云,又号观弈道人、孤石老人、河间才子,在文学作品、通俗评论中,常被称为纪晓岚。清代直隶献县(今河北献县)人,乾隆年间的著名学者,政治人物。官至礼部尚书、协办大学士,曾任《四库全书》总纂修官。卒谥文达。纪昀文采超群,与同时代江南的袁枚齐名,时称“北纪南袁”。纪昀反对理学[2],《阅微草堂笔记》和《四库全书总目提要》中有相当深刻的反映。 \tabularnewline\hline
  张惠言 & \begin{description}
  \item[字] 皋文
  \item[号] 
  \item[谥] 
  \item[尊] 
  \item[生] 江苏常州
  \end{description} & 张惠言(1761年-1802年),原名一鸣,字皋文,江苏武进(今常州)人,清代政治人物,经学家、词学家。生于清高宗乾隆二十六年(1761年),幼年贫困[1],清仁宗嘉庆四年(1799年)中进士,授庶吉士,充实录馆纂修官,卒于嘉庆七年(1802年)。著有《茗柯文》五卷。张惠言提出“比兴寄托”,主张“意内言外”,人称常州词派始祖。 \tabularnewline\hline
  周济 & \begin{description}
  \item[字] 介存\\保绪
  \item[号] 未斋
  \item[谥] 
  \item[尊] 
  \item[生] 江苏宜兴
  \end{description} & 周济(1781年-1839年),清朝词人及词论家。字保绪,一字介存,号未斋,晚号止庵。江苏荆溪(今江苏宜兴)人。周济是董士锡的弟子,继承了张惠言的词论传统,一般被称为常州词派的集大成者。他论词强调寄托;自作词意旨较为隐晦。著有《味隽斋词》、《词辨》、《介存斋论词杂著》、《晋略》,编有《宋四家词选》。 \tabularnewline\hline
  康有为 & \begin{description}
  \item[字] 广厦
  \item[号] 长素
  \item[谥] 
  \item[尊] 康南海
  \item[生] 广东南海
  \end{description} & 康有为(1858年3月19日-1927年3月31日),清末维新变法派主要发起者,原名祖诒,字广厦,号长素,又号明夷、更生、西樵山人、游存叟、天游化人,广东省南海县丹灶苏村人,人称康南海,光绪廿一年(1895年)进士,曾与弟子梁启超合作戊戌变法,变法失败后,被慈禧太后通缉而出逃。1912年宣统退位后,康有为继续反对共和,1917年曾与张勋合作,发动兵变,拥立宣统帝,是为辫军复辟,但十二日之内就被段祺瑞讨平。1927年在一场宴会后病逝,被质疑是政敌下毒。康有为的理想和政治主张主要在他撰写的《大同书》中得到体现。 \tabularnewline\hline
  王国维 & \begin{description}
  \item[字] 静安
  \item[号] 观堂
  \item[谥] 忠悫
  \item[尊] 
  \item[生] 浙江杭州
  \end{description} & 王国维(1877年12月3日-1927年6月2日),字静安,又字伯隅,晚号观堂(甲骨四堂之一),谥忠悫(què)。浙江杭州府海宁人,国学大师。王国维与梁启超、陈寅恪、和赵元任号称清华国学研究院的“四大导师”。中国新学术的开拓者,连接中西美学的大家,在文学、美学、史学、哲学、金石学、甲骨文、考古学等领域成就卓著。甲骨四堂之一。王国维精通英文、德文、日文,使他在研究宋元戏曲史时独树一帜,成为用西方文学原理批评中国旧文学的第一人。陈寅恪认为王国维的学术成就“几若无涯岸之可望、辙迹之可寻”。著述甚丰,有《海宁王静安先生遗书》、《红楼梦评论》、《宋元戏曲考》、《人间词话》、《观堂集林》、《古史新证》、《曲录》、《殷周制度论》、《流沙坠简》等62种。 \tabularnewline
  \bottomrule
\end{longtable}


%%% Local Variables:
%%% mode: latex
%%% TeX-engine: xetex
%%% TeX-master: "../../Main"
%%% End:
