%% -*- coding: utf-8 -*-
%% Time-stamp: <Chen Wang: 2018-10-30 16:26:19>

\subsection{先秦}

\begin{longtable}{|>{\centering\namefont\heiti}m{2em}|>{\centering\tiny}m{3.0em}|>{\xzfont\kaiti}m{7em}|}
    % \caption{秦王政}\
    \toprule
    \SimHei \normalsize 姓名 & \SimHei \normalsize 异名 & \SimHei \normalsize \hspace{2.5em}小传 \tabularnewline
    % \midrule
    \endfirsthead
    \toprule
    \SimHei \normalsize 姓名 & \SimHei \normalsize 异名 & \SimHei \normalsize \hspace{2.5em}小传 \tabularnewline 
    \midrule
    \endhead
    \midrule
    屈平 & \begin{description}
    \item[字] 屈原
    \item[号] 屈子
    \item[谥] 
    \item[尊] 三闾大夫
    \item[生] 楚国
    \end{description} & 屈原(约前340年-约前278年6月6日),芈姓,屈氏,名平,字原,楚国人(今湖北秭归),是古帝高阳氏的后裔,其自作词曰:“帝高阳之苗裔兮,朕皇考曰伯庸。”,其先祖屈瑕受楚武王封于屈地,因以屈为氏,名平。屈,昭,景为楚国大姓,官拜左徒,左徒多以贵族近臣任之,左徒任务有四 “议国事”、“出号令”、“接遇宾客”、“应对诸侯”。 \tabularnewline
    \bottomrule
\end{longtable}


%%% Local Variables:
%%% mode: latex
%%% TeX-engine: xetex
%%% TeX-master: "../../Main"
%%% End:
