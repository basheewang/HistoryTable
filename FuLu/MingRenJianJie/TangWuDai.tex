%% -*- coding: utf-8 -*-
%% Time-stamp: <Chen Wang: 2018-10-30 16:26:04>

\subsection{唐五代}

\begin{longtable}{|>{\centering\namefont\heiti}m{2em}|>{\centering\tiny}m{3.0em}|>{\xzfont\kaiti}m{7em}|}
  % \caption{秦王政}\
  \toprule
  \SimHei \normalsize 姓名 & \SimHei \normalsize 异名 & \SimHei \normalsize \hspace{2.5em}小传 \tabularnewline
  % \midrule
  \endfirsthead
  \toprule
  \SimHei \normalsize 姓名 & \SimHei \normalsize 异名 & \SimHei \normalsize \hspace{2.5em}小传 \tabularnewline 
  \midrule
  \endhead
  \midrule
    宋之问 & \begin{description}
    \item[字] 延清
    \item[号] 
    \item[谥] 
    \item[尊] 宋考功
    \item[生] 山西汾阳
    \end{description} & 宋之问(约656 — 约712),字延清,名少连,汉族,汾州隰城人(今山西汾阳市)人,初唐时期的诗人,与沈佺期并称“沈宋”。与陈子昂、卢藏用、司马承祯、王适、毕构、李白、孟浩然、王维、贺知章称为仙宗十友。 \tabularnewline\hline
    沈佺期 & \begin{description}
    \item[字] 云卿
    \item[号] 
    \item[谥] 
    \item[尊] 
    \item[生] 安阳
    \end{description} & 沈佺期(约656 — 约715),字云卿,相州内黄(今安阳市内黄县)人,祖籍吴兴(今浙江湖州)。唐代诗人。与宋之问齐名,称“ 沈宋 ”。善属文,尤长七言之作。擢进士第。长安中,累迁通事舍人,预修《三教珠英》,转考功郎给事中。坐交张易之,流驩州。稍迁台州录事参军。神龙中,召见,拜起居郎,修文馆直学士,历中书舍人,太子少詹事。开元初卒。建安后,讫江左,诗律屡变,至沈约、庾信,以音韵相婉附,属对精密,及沈佺期与宋之问,尤加靡丽。回忌声病,约句准篇,如锦绣成文,学者宗之,号为“沈宋”。语曰:苏李居前,沈宋比肩。有集十卷,今编诗三卷。 \tabularnewline\hline
    陈子昂 & \begin{description}
    \item[字] 伯玉
    \item[号] 
    \item[谥] 
    \item[尊] 陈拾遗
    \item[生] 四川射洪
    \end{description} & 陈子昂(公元659~公元702),字伯玉,梓州射洪(今四川省遂宁市射洪县)人,唐代诗人,初唐诗文革新人物之一。因曾任右拾遗,后世称陈拾遗。青少年时轻财好施,慷慨任侠,24岁举进士,以上书论政得到女皇武则天重视,授麟台正字。后升右拾遗,直言敢谏,曾因“逆党”反对武后而株连下狱。在26岁、36岁时两次从军边塞,对边防颇有些远见。38岁(圣历元年698)时,因父老解官回乡,不久父死。陈子昂居丧期间,权臣武三思指使射洪县令段简罗织罪名,加以迫害,冤死狱中。   其存诗共100多首,其诗风骨峥嵘,寓意深远,苍劲有力。其中最有代表性的有组诗《感遇》38首,《蓟丘览古》7首和《登幽州台歌》、《登泽州城北楼宴》等。 \tabularnewline\hline
  张说 & \begin{description}
  \item[字] 道济\\说之
  \item[号] 
  \item[谥] 
  \item[尊] 张燕公
  \item[生] 河北涿州
  \end{description} & 张说(667年-730年),字道济,一字说之,原籍范阳(今河北涿州市),世居河东(今山西永济),后徙洛阳。唐玄宗宰相,封燕国公。擅长文学,当时朝廷重要辞章多出其手,尤长于碑文墓志,与许国公苏颋齐名,并称“燕许大手笔”。 \tabularnewline\hline
  李颀 & \begin{description}
  \item[字] 
  \item[号] 
  \item[谥] 
  \item[尊] 
  \item[生] 河北赵县
  \end{description} & 李颀(690年-751年),唐代赵郡(今河北赵县)人,后长居颖阳(今河南登封),唐代诗人。李颀出身于唐朝士族赵郡李氏,常服饵丹砂,“甚有好颜色”,因结识富豪轻薄子弟,倾财破产。后立志刻苦读书,隐居颍阳苦读十年,于唐玄宗开元二十三年(735年)贾季邻榜进士及第,曾为新乡县尉,始终未得迁调,天宝十载前即辞官归隐。余事不详。李颀性格超脱,厌薄世俗,以写诗称著,与诗人王维、王昌龄、高适等来往密切。他的诗秀丽雄浑,内容与体裁颇为广泛。又以五言、七言歌行和七言律诗见长,清代王士祯评:“盛唐七言诗,老杜外,王维、李颀、岑参耳。”。他尤以边塞诗著称,格调雄浑奔放,慷慨激昂。李颀的代表作有《古从军行》、《古意》、《塞下曲》、《听董大弹胡笳兼寄语房给事》。著有《李颀集》。《全唐诗》录其诗3卷,共127首。 \tabularnewline\hline
  王昌龄 & \begin{description}
  \item[字] 少伯
  \item[号] 
  \item[谥] 
  \item[尊] 王江宁\\龙标
  \item[生] 山西太原
  \end{description} & 王昌龄(698年-756年),字少伯,山西太原人,盛唐著名边塞诗人。他的诗和高适、王之涣齐名,因其善写场面雄阔的边塞诗,而有“诗家天子”(或作“诗家夫子”)、 “七绝圣手”、“开天圣手”、“诗天子”的美誉。世称“王江宁”。著有文集六卷,今编诗四卷。代表作有《从军行七首》、《出塞》、《闺怨》等。 \tabularnewline\hline
  崔颢 & \begin{description}
  \item[字] 
  \item[号] 
  \item[谥] 
  \item[尊] 崔思勋\\崔郎中
  \item[生] 河南开封
  \end{description} & 崔颢(约704年-754年)是中国唐朝诗人,汴州(今河南开封)人。开元十一年(723年)中进士,开元二十九年 ,担任扶沟县尉,官位一直不显,后游历天下。天宝九载前后曾任监察御史,官至司勋员外郎。天宝十三载卒。现存诗仅四十二首,最有名的一首莫过于《黄鹤楼》,乃千古绝唱。少年时作的诗多写闺情,流于浮艳,后历边塞,诗风变得雄浑奔放、风骨凛然。崔颢四处游历,吟诗甚勤,其友人笑他吟诗吟得人也瘦(非子病如此,乃苦吟诗瘦耳)。明人辑有《崔颢集》。 \tabularnewline\hline
  高适 & \begin{description}
  \item[字] 达夫
  \item[号] 
  \item[谥] 
  \item[尊] 高常侍
  \item[生] 河北景县
  \end{description} & 高适(706年-765年2月17日),字达夫,沧州渤海人(今河北景县)。唐朝边塞诗人,诗词语言质朴,风格雄浑,与岑参并称“高岑”。 \tabularnewline\hline
  刘长卿 & \begin{description}
  \item[字] 文房
  \item[号] 
  \item[谥] 
  \item[尊] 刘随州
  \item[生] 安徽宣城
  \end{description} & 刘长卿(?-约790年),字文房,宣城(今属安徽)人,郡望河间(今属河北),唐代诗人。年轻时在嵩山读书,唐玄宗开元进士,曾任监察御史,常因性情刚烈而冒犯他人,至德三年(乾元元年,758年)正月,摄海盐令。因事由苏州长洲尉贬为潘州南巴(今广东电白县)尉,代宗时任转运使判官,知淮西、鄂岳转运留后,大历年间,又因得罪了鄂岳观察使吴仲孺,被诬为贪赃,贬为睦州(今浙江淳安)司马。终官随州(今湖北随县)刺史,世称“刘随州”。贞元元年,淮西节度使李希烈割据随州称王,时局动荡,刘长卿离开随州,晚年流寓江州,曾入淮南节度使幕。约卒于贞元六年。 \tabularnewline\hline
  岑参 & \begin{description}
  \item[字] 
  \item[号] 
  \item[谥] 
  \item[尊] 岑嘉州
  \item[生] 荆州江陵
  \end{description} & 岑参(715年-770年),荆州江陵县人,郡望南阳,唐朝诗人,宰相岑文本曾孙,边塞诗代表人物,与高适并称高岑。曾任嘉州(今四川省乐山市)刺史,后人因称“岑嘉州”。 \tabularnewline\hline
  元结 & \begin{description}
  \item[字] 次山
  \item[号] 漫郎\\猗玕子
  \item[谥] 
  \item[尊] 
  \item[生] 河南鲁山
  \end{description} & 元结(723年-772年5月26日),字次山,号漫郎、猗玕子,河南鲁山人。唐朝进士、官员。有《元次山集》。 \tabularnewline\hline
  钱起 & \begin{description}
  \item[字] 仲文
  \item[号] 
  \item[谥] 
  \item[尊] 员外郎
  \item[生] 浙江湖州
  \end{description} & 钱起(710年-782年),字仲文,吴兴(今浙江湖州)人。唐代诗人,诗风清奇,与郎士元、司空曙、李益、李端、卢纶、李嘉祐等称大历十才子。 \tabularnewline\hline
  李泌 & \begin{description}
  \item[字] 长源
  \item[号] 
  \item[谥] 
  \item[尊] 
  \item[生] 京兆
  \end{description} & 李泌(722年-789年),字长源,唐朝宰相,京兆人,祖籍辽东襄平。李泌是西魏八柱国李弼的六代孙,父亲李承休是吴房县令,娶汝南周氏为妻,聚书两万余卷,并告诫子孙不得卖书。李泌幼居长安,七岁能文,张九龄奇之,玄宗召令供奉东宫,写诗讽刺杨国忠,有“青青东门柳,岁晏复憔悴。”之句,隐居颍阳。肃宗时,参预军国大议,拜银青光禄大夫,隐居衡山(今湖南省),修练道教,刘昫说:“居相位而从事鬼神,乃见狂妄浮薄之踪。”代宗时,召为翰林学士,不久因得罪权臣元载,被代宗外放为杭州刺史以避祸。德宗时,元载失势,复召回朝廷并授散骑常侍。贞元中,拜中书侍郎平章事,封邺县侯。李泌以虚诞自任,辅佐四朝天子。贞元五年(789年)三月,辞世。有文集二十卷。 \tabularnewline\hline
  司空曙 & \begin{description}
  \item[字] 文明
  \item[号] 
  \item[谥] 
  \item[尊] 
  \item[生] 河北永年
  \end{description} & 司空曙(720年-790年),字文明,或作文初。广平郡(治所在今河北省永年县东南)人。唐代官员、诗人。中年因安史之乱避居南方,数年后北归长安,曾中进士,曾任洛阳主簿,后任左拾遗,因事贬长林(今湖北荆门西北)县丞 。贞元年间在剑南节度使韦皋下作幕府,官检校水部郎中。终官虞部郎中。余事不可确考,生卒年亦不详。司空曙经历安史之乱,他的诗作以写自然景色和乡情旅思为主,擅长五律,共有集三卷,是大历十才子之一。 \tabularnewline\hline
  戴叔伦 & \begin{description}
  \item[字] 幼公\\次公
  \item[号] 
  \item[谥] 
  \item[尊] 戴容州
  \item[生] 江苏常州
  \end{description} & 戴叔伦(732年-789年),字幼公,一字次公,润州金坛南瑶村(今属江苏常州市)人,唐朝著名诗人。远祖戴安道。戴于唐代宗广德初年任秘书省正字,大历元年(766年),在户部尚书充诸道盐铁使刘晏幕下任职,经刘晏表奏,授监察御史衔。唐德宗建中初年出任东阳县令,后又入江西观察使幕府,授大理寺司直衔,兴元元年(784年)出任抚州刺史,贞元元年(785年)十一月撰有《贺平贼赦表》,授吏部郎中衔。贞元二年(786)辞官还乡,四年出任容州刺史兼容管经略使,贞元五年,卒于任所。 \tabularnewline\hline
  韦应物 & \begin{description}
  \item[字] 
  \item[号] 
  \item[谥] 
  \item[尊] 韦左司\\韦苏州
  \item[生] 陕西西安
  \end{description} & 韦应物(737年-791年),京兆郡杜陵县(今陕西省西安市长安区)人。唐代诗人。《唐诗三百首》收录韦应物诗12首。 \tabularnewline\hline
  李益 & \begin{description}
  \item[字] 君虞
  \item[号] 
  \item[谥] 
  \item[尊] 
  \item[生] 河南郑州
  \end{description} & 李益(746年-829年),字君虞,郑州人,祖籍陇西狄道(今甘肃省临洮县),中唐诗人,以边塞诗作名世,擅长绝句,尤其工于七绝。李益出自陇西李氏姑臧房,是唐朝给事、赠兵部尚书李亶的曾孙,虞部郎中李成绩的孙子,大理司直、赠太子少师李存的儿子,擅长写作诗歌,成名于贞元末年,与唐朝宗室大郑王房“诗鬼”之称的李贺齐名。年轻时的他颇负文名,每写成一篇诗作,宫中都会有乐工名伶争相出价,希望买下他的作品,编排乐曲,让皇帝欣赏。李益所创作的《征人》、《早行》等名篇,更被当时人绘成图赞,流传天下。可是李益为人多疑善妒,相当执著,对于妻妾的德操管治非常严苛,不许妻妾与帮闲坊众接触,因此世人都戏称那些善妒者是患上“李益疾”,更有“妒痴尚书李十郎”之语。 \tabularnewline\hline
  张籍 & \begin{description}
  \item[字] 文昌
  \item[号] 诗肠
  \item[谥] 
  \item[尊] 张水部\\张司业
  \item[生] 江苏苏州
  \end{description} & 张籍(约767年—约830年),唐朝诗人。字文昌,又称“诗肠”。原籍吴郡(今江苏苏州),后迁居和州乌江(今安徽和县)。唐德宗贞元十四年北游,经孟郊介绍,在汴州(今河南开封)认识韩愈。贞元十五年进士。历任太常寺太祝,因患目疾,自称“草色遥看近却无”,孟郊称他为“穷瞎张太祝”。元和十一年,转任国子监助教,目疾稍愈。迁秘书郎。藩镇李师道仰慕张籍的学识,想网罗入幕,张籍婉拒,写了一首〈节妇吟〉寄给了李司徒。长庆元年(821年),韩愈荐为国子博士,历任水部员外郎、主客郎中,终国子司业。时称“张水部”或“张司业”。因其出身贫寒,官职低微,能较多地接触社会底层的民众,故其所作乐府诗多批判社会,同情百姓的遭遇,颇为白居易等人所推崇,白居易称赞为“尤工乐府诗,举代少其伦”。与王建齐名,号称“张王乐府”。《彦周诗话》论道:“张籍,乐府、宫辞皆杰出”。其与白居易,孟郊等所作的诗歌被称为“元和体”。著有《张司业集》,编为五卷。南唐张洎收集其诗400多首编为《木铎集》12卷。明代嘉靖万历间刻本《唐张司业诗集》8卷,收诗450多首。 \tabularnewline\hline
  刘禹锡 & \begin{description}
  \item[字] 梦得
  \item[号] 
  \item[谥] 
  \item[尊] 刘宾客\\刘尚书
  \item[生] 河南洛阳
  \end{description} & 刘禹锡(772年-842年),河南洛阳人,字梦得,祖先来自北方,自言出于中山(今河北省定州市)(又自称“家本荥上,籍占洛阳”)。唐朝著名诗人,中唐文学的代表人物之一。因曾任太子宾客,故称刘宾客,晚年曾加检校礼部尚书、秘书监等虚衔,故又称秘书刘尚书。 \tabularnewline\hline
  李绅 & \begin{description}
  \item[字] 公垂
  \item[号] 
  \item[谥] 
  \item[尊] 
  \item[生] 安徽亳州
  \end{description} & 李绅(772年-846年),字公垂,中唐诗人。亳州(今属安徽)人,生于乌程(今浙江湖州),长于润州无锡(今属江苏)。李绅生于唐大历七年(772年),曾祖父李敬玄,祖父定李守一籍安徽亳州。父李晤,历任金坛、乌程(今浙江吴兴)、晋陵(今江苏常州)等县令,携家来无锡,定居梅里抵陀里(今江苏无锡东亭长大厦村)。15岁时读书于惠山。与元稹、白居易共倡“新乐府” 诗体,史称“新乐府运动”。元和元年(806年)进士,补国子监助教。润州观察使李锜聘为从事,不随其叛乱,拜右拾遗。元和七年担任校书郎。历官翰林学士,转任右补阙,与李德裕、元稹同时号“三俊”,后卷入牛李党争。长庆元年(821)三月,改为司勋员外郎、知制诰。二年二月,破格升任中书舍人,入中书省。长庆四年(824年)李党失势,受李逢吉排挤被贬为端州(今广东肇庆)司马,宝历元年(825年)改任江州(今江西九江市)刺史,不久迁滁州、寿州刺史,又改授太子宾客分司东都。太和七年,李德裕为相,任浙东观察使,开成元年(836年)任河南尹,历任汴州刺史、宣武军节度使、宋亳汴颖观察使。开成五年(840年)任淮南节度使。不久入京拜相,官至尚书右仆射门下侍郎,封赵国公。 \tabularnewline\hline
  孟郊 & \begin{description}
  \item[字] 东野
  \item[号] 
  \item[谥] 
  \item[尊] 
  \item[生] 浙江德清
  \end{description} & 孟郊(751年-814年),字东野,唐朝湖州武康(今浙江德清)人。现存诗歌500多首,以短篇的五言古诗最多,没有一首律诗。代表作有<游子吟>。祖籍平昌(今山东临邑东北)。先世居洛阳(今属河南),孟郊早年生活贫困,曾游历湖北、湖南、广西等地,无所遇合,屡试不第。贞元中张建封镇守徐州时,孟郊曾往谒见。46岁(一说45岁),始登进士第,有诗《登科后》:“昔日龌龊不足夸,今朝放荡思无涯;春风得意马蹄疾,一日看尽长安花(成语“走马看花”由来)。”。然后东归,旅游汴州(今河南开封)、越州(今浙江绍兴)。贞元十七年(801年),任为溧阳尉。在任不事曹务,常以作诗为乐,被罚半俸。韩愈称他为“酸寒溧阳尉”。元和元年(806年),河南尹郑余庆奏为河南水陆转运从事,试协律郎,定居洛阳。元和三年(808年)为检校兵部尚书,兼东都留守。60岁时,因母死去官。九年三月,郑余庆转任山南西道节度使,镇守兴元,又奏孟郊为参谋、试大理评事。郊应邀前往,到阌乡(今河南灵宝),不幸以暴病去世,孟郊的朋友韩愈等人凑了100贯为他营葬,郑余庆派人送300贯,“为遗孀永久之赖”。张籍私谥为贞曜先生。 \tabularnewline\hline
  贾岛 & \begin{description}
  \item[字] 浪仙\\阆仙
  \item[号] 
  \item[谥] 
  \item[尊] 
  \item[生] 河北涿州
  \end{description} & 贾岛(779年-843年),字浪仙(亦作阆仙),范阳(今河北省涿州市)人,唐朝诗人,与韩愈同时。贾岛贫寒,曾经做过和尚,法号无本。元和五年(810年)冬,至长安,见张籍。据说在洛阳的时候后因当时有命令禁止和尚午后外出,贾岛做诗发牢骚,被韩愈发现其才华。后受教于韩愈,并还俗参加科举,但累举不中第。元和十四年(819年),韩愈抵潮州(今广东潮州),致信贾岛,贾岛作《寄韩潮州愈》诗给韩愈。长庆二年(822年)举进士,以“僻涩之才无所用”。唐文宗的时候被排挤,贬做长江主簿。唐武宗会昌年初由普州司仓参军改任司户,未任病逝。《新唐书》将贾岛附名于《韩愈传》之后。 \tabularnewline\hline
  卢仝 & \begin{description}
  \item[字] 
  \item[号] 玉川子
  \item[谥] 
  \item[尊] 
  \item[生] 河南济源
  \end{description} & 卢仝(795年-835年),自号玉川子,河南济源人,中国唐朝中期诗人。诗风奇诡险怪,人称“卢仝体”,有《玉川子诗集》传世。后为韩愈赏识,韩愈的《月蚀诗效玉川子作》是对卢仝《月蚀诗》进行繁删,体现他对卢仝体的推崇。后卢仝迁居洛阳。元和六年,卢仝在洛阳里仁坊购宅。他好饮茶,一首《走笔谢孟谏议寄新茶》人称“玉川茶歌”,与陆羽茶经齐名。 \tabularnewline\hline
  杜牧 & \begin{description}
  \item[字] 牧之
  \item[号] 樊川
  \item[谥] 
  \item[尊] 杜紫薇
  \item[生] 陕西西安
  \end{description} & 杜牧(803年-852年),字牧之,号樊川,京兆府万年县(今陕西省西安市)人。晚唐著名诗人和古文家。擅长长篇五言古诗和七律。曾任中书舍人(中书省别名紫微省),人称杜紫微。其诗英发俊爽,为文尤纵横奥衍,多切经世之务,在晚唐成就颇高,时人称其为“小杜”,以别于杜甫;又与李商隐齐名,人称“小李杜”。 \tabularnewline\hline
  陆龟蒙 & \begin{description}
  \item[字] 鲁望
  \item[号] 江湖散人\\甫里先生\\天随子
  \item[谥] 
  \item[尊] 
  \item[生] 江苏苏州
  \end{description} & 陆龟蒙(?-881年),字鲁望,唐朝苏州吴县(今属江苏)人,自号江湖散人、甫里先生,又号天随子。陆元方七世孙,其父陆宾虞曾任御史之职。开成元年(836)前后出生,进士不第,曾在湖州、苏州从事幕僚。随湖州刺史张博游历,后来回到了故乡苏州甫里(今江苏吴县东南甪直镇),过着隐居耕读的生活,自号天随子;由于甫里地低下,常苦水潦,乃至饥馑,著有《耒耜经》,是一本农学书;喜爱品茗,在顾渚山下辟一茶园,耕读之余,则喜好垂钓。与皮日休为友,时常在一起游山玩水,饮酒吟诗,世称“皮陆”,二人唱和之作编为《松陵集》十卷。 \tabularnewline\hline
  司空图 & \begin{description}
  \item[字] 表圣
  \item[号] 
  \item[谥] 
  \item[尊] 
  \item[生] 山西永济
  \end{description} & 司空图(837年-908年),字表圣,中国唐朝末年诗人、文学评论家,河中郡虞乡(今山西省永济县)人。司空图早年为王凝赏识,在其推荐下于唐懿宗咸通10年(869年)中进士,后为报恩,放弃在朝中为官的机会,长期居于王凝幕府中。878年,被任命为光禄寺主簿,分司洛阳。在洛阳期间得到卢携的赏识,后卢携回朝复相,司空图被任命为礼部员外郎,不久升任郎中。唐僖宗广明元年(880年),黄巢入长安,司空图拒绝其招揽,逃往凤翔投奔唐僖宗,被任命为知制诰、中书舍人。次年,唐僖宗迁往宝鸡,司空图与其失散,回乡隐居中条山王官谷。唐昭宗及宰相朱温屡次征召其为侍郎、尚书等职,他均坚辞不受,最后接受了宰相柳璨的要求为官,却故意装作衰老的样子,在朝堂上失手坠落笏板,得以放还本乡中条山。907年,朱温废去唐哀帝,建立后梁,次年又将哀帝刺杀。司空图闻信后,绝食而死。 \tabularnewline\hline
  郑谷 & \begin{description}
  \item[字] 守愚
  \item[号] 
  \item[谥] 
  \item[尊] 郑鹧鸪\\郑都官
  \item[生] 江西宜春
  \end{description} & 郑谷(849年-911年),字守愚,江西袁州(今宜春)人,唐代诗人。父郑史为永州刺史。七岁能诗,光启三年进士,官右拾遗,历都官郎中。生逢乱世,际遇坎坷。郑谷与许棠、任涛等九人时相唱和,时称“芳林十哲”,“尝赋鹧鸪,警绝”,故有“郑鹧鸪”的称号。郑谷隐居仰山,有一诗僧齐己以一首《早梅》诗求教,郑谷将诗中“前村深雪里,昨夜数枝开”的“数枝”改为“一枝”,齐己下拜,当时士子又称郑谷称为齐己的“一字之师”。乾宁三年(896年),昭宗避难华州,郑谷亦赴华州,“寓居云台道舍”,因而自称诗集为《云台编》。其作品有《云台编》三卷、《宜阳集》三卷以及《国风正诀》一卷等。 \tabularnewline\hline
  韦庄 & \begin{description}
  \item[字] 端己
  \item[号] 
  \item[谥] 文靖
  \item[尊] 
  \item[生] 陕西西安
  \end{description} & 韦庄(836年-910年),字端己,京兆杜陵(今陕西省西安市)人。晚唐政治家,诗人。广明元年(880年)韦庄在长安应举,黄巢攻占长安以后,与弟妹失散,浪迹天涯。中和三年(883年)三月,在洛阳写有长篇歌行《秦妇吟》。昭宗乾宁元年(894年)进士,曾任校书郎、左补阙等职。乾宁四年(897年),李询为两川宣谕和协使,聘用他为判官。在四川时为王建掌书记,蜀开国制度皆庄所定,官至吏部尚书,同平章事,武成三年(910年)八月,卒于成都花林坊。葬白沙之阳。谥文靖。韦庄是唐朝花间派词人,词风清丽,与温庭筠并称“温韦”。韦庄的弟弟韦蔼所编之《浣花词》流传。《花间集》收四十八首。杨慎《升庵外集》评韦庄词“明白如画,蕴情深至”。况周颐《蕙风词话》称他“尤能运密如疏、寓浓于淡,花间群贤,殆鲜其匹”。近人王国维谓之“骨秀也”,评价更在温庭筠之上。 \tabularnewline\hline
  冯延巳 & \begin{description}
  \item[字] 正中
  \item[号] 
  \item[谥] 忠肃
  \item[尊] 
  \item[生] 广陵
  \end{description} & 冯延巳(903年-960年),原名冯延嗣,是五代时词人,广陵人。字正中,一说名延己,但支持延巳的较多。南唐时官至宰相,是南唐中主李璟的老师。他是南唐吏部尚书冯令额的儿子,弟弟冯延鲁亦是著名文人。死后谥号忠肃,有《阳春集》传世。冯延巳词风清丽,善写离情别绪,有很高的艺术成就,对李煜影响很大。冯延巳、李煜被认为直接影响了北宋以来的词风。有“吹皱一池春水”名句。 \tabularnewline\hline
  毛文锡 & \begin{description}
  \item[字] 平珪
  \item[号] 
  \item[谥] 
  \item[尊] 毛司徒
  \item[生] 不详
  \end{description} & 毛文锡,字平珪,五代十国时期前蜀国人。曾任翰林学士承旨、礼部尚书。曾经力谏前蜀帝王建不要决坝淹没江陵,挽救无数百姓性命。 \tabularnewline\hline
  顾夐 & \begin{description}
  \item[字] 
  \item[号] 
  \item[谥] 
  \item[尊] 顾太尉
  \item[生] 不详
  \end{description} & 顾敻(xìong),五代词人。生卒年、籍贯及字号均不详。前蜀王建通正(916)时,以小臣给事内廷,见秃鹫翔摩诃池上,作诗刺之,几遭不测之祸。后擢茂州刺史。入后蜀,累官至太尉。顾夐能诗善词。 《花间集》收其词55首,全部写男女艳情。 \tabularnewline\hline

  \bottomrule
\end{longtable}


%%% Local Variables:
%%% mode: latex
%%% TeX-engine: xetex
%%% TeX-master: "../../Main"
%%% End:
