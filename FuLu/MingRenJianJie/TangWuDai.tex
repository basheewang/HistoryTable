%% -*- coding: utf-8 -*-
%% Time-stamp: <Chen Wang: 2018-07-18 14:17:59>

\subsection{唐五代}

\begin{longtable}{|>{\centering\namefont\heiti}m{2em}|>{\centering\tiny}m{3.0em}|>{\xzfont\kaiti}m{7.3em}|}
  % \caption{秦王政}\
  \toprule
  \SimHei \normalsize 姓名 & \SimHei \normalsize 异名 & \SimHei \normalsize \hspace{2.5em}小传 \tabularnewline
  % \midrule
  \endfirsthead
  \toprule
  \SimHei \normalsize 姓名 & \SimHei \normalsize 异名 & \SimHei \normalsize \hspace{2.5em}小传 \tabularnewline 
  \midrule
  \endhead
  \midrule
    宋之问 & \begin{description}
    \item[字] 延清
    \item[号] 
    \item[谥] 
    \item[尊] 宋考功
    \item[生] 山西汾阳
    \end{description} & 宋之问(约656 — 约712),字延清,名少连,汉族,汾州隰城人(今山西汾阳市)人,初唐时期的诗人,与沈佺期并称“沈宋”。与陈子昂、卢藏用、司马承祯、王适、毕构、李白、孟浩然、王维、贺知章称为仙宗十友。 \tabularnewline\hline
    沈佺期 & \begin{description}
    \item[字] 云卿
    \item[号] 
    \item[谥] 
    \item[尊] 
    \item[生] 安阳
    \end{description} & 沈佺期(约656 — 约715),字云卿,相州内黄(今安阳市内黄县)人,祖籍吴兴(今浙江湖州)。唐代诗人。与宋之问齐名,称“ 沈宋 ”。善属文,尤长七言之作。擢进士第。长安中,累迁通事舍人,预修《三教珠英》,转考功郎给事中。坐交张易之,流驩州。稍迁台州录事参军。神龙中,召见,拜起居郎,修文馆直学士,历中书舍人,太子少詹事。开元初卒。建安后,讫江左,诗律屡变,至沈约、庾信,以音韵相婉附,属对精密,及沈佺期与宋之问,尤加靡丽。回忌声病,约句准篇,如锦绣成文,学者宗之,号为“沈宋”。语曰:苏李居前,沈宋比肩。有集十卷,今编诗三卷。 \tabularnewline\hline
    陈子昂 & \begin{description}
    \item[字] 伯玉
    \item[号] 
    \item[谥] 
    \item[尊] 陈拾遗
    \item[生] 四川射洪
    \end{description} & 陈子昂(公元659~公元702),字伯玉,梓州射洪(今四川省遂宁市射洪县)人,唐代诗人,初唐诗文革新人物之一。因曾任右拾遗,后世称陈拾遗。青少年时轻财好施,慷慨任侠,24岁举进士,以上书论政得到女皇武则天重视,授麟台正字。后升右拾遗,直言敢谏,曾因“逆党”反对武后而株连下狱。在26岁、36岁时两次从军边塞,对边防颇有些远见。38岁(圣历元年698)时,因父老解官回乡,不久父死。陈子昂居丧期间,权臣武三思指使射洪县令段简罗织罪名,加以迫害,冤死狱中。 [1]  其存诗共100多首,其诗风骨峥嵘,寓意深远,苍劲有力。其中最有代表性的有组诗《感遇》38首,《蓟丘览古》7首和《登幽州台歌》、《登泽州城北楼宴》等。 \tabularnewline\hline
  
  \bottomrule
\end{longtable}


%%% Local Variables:
%%% mode: latex
%%% TeX-engine: xetex
%%% TeX-master: "../../Main"
%%% End:
