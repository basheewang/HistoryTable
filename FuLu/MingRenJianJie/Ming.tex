%% -*- coding: utf-8 -*-
%% Time-stamp: <Chen Wang: 2018-07-21 19:33:56>

\subsection{明}

\begin{longtable}{|>{\centering\namefont\heiti}m{2em}|>{\centering\tiny}m{3.0em}|>{\xzfont\kaiti}m{7.3em}|}
  % \caption{秦王政}\
  \toprule
  \SimHei \normalsize 姓名 & \SimHei \normalsize 异名 & \SimHei \normalsize \hspace{2.5em}小传 \tabularnewline
  % \midrule
  \endfirsthead
  \toprule
  \SimHei \normalsize 姓名 & \SimHei \normalsize 异名 & \SimHei \normalsize \hspace{2.5em}小传 \tabularnewline 
  \midrule
  \endhead
  \midrule
  高棅 & \begin{description}
  \item[字] 廷礼\\彦恢
  \item[号] 漫士
  \item[谥] 
  \item[尊] 
  \item[生] 福建长乐
  \end{description} & 高棅(1350年-1423年),又名廷礼,字彦恢,号漫士,福建长乐人。闽中十才子之一。高棅为明朝初年研究唐诗的重要学者,所著的《唐诗品汇》为明初诗歌复古的里程碑,也是中国文学的重要评论著作。高棅著有《啸台集》、《水天清气集》、《唐诗品汇》、《唐诗拾遗》、《唐诗正声》。 \tabularnewline\hline
  高启 & \begin{description}
  \item[字] 季迪
  \item[号] 青丘子
  \item[谥] 
  \item[尊] 
  \item[生] 江苏苏州
  \end{description} & 高启(1336年-1373年,37岁),字季迪,号青丘,元末明初平江路(明改苏州府)长洲县(今江苏省苏州市)人,明初十才子之一。和宋濂、刘基合称“明初诗文三大家”。因得罪明太祖,以魏观案累文字狱,处腰斩。高启有诗才,其诗清新超拔,雄健豪迈,尤擅长于七言歌行,《四库全书总目提要》称:“拟汉魏似汉魏,拟六朝似六朝,拟唐似唐,拟宋似宋,凡古人所长,无不兼之。”与杨基、张羽、徐贲合称“吴中四杰”。景泰元年(1450年),徐庸搜集《缶鸣集》等遗篇,编为《高太史大全集》18卷。 \tabularnewline\hline
  陈献章 & \begin{description}
  \item[字] 公甫
  \item[号] 实斋
  \item[谥] 
  \item[尊] 白沙先生
  \item[生] 广东江门
  \end{description} & 陈献章(1428年-1500年),字公甫,号实斋,广东新会县会城都会乡(今江门市新会区会城街道)人,后迁居白沙乡,世称白沙先生。明代著名的书法家、诗人、教育家、思想家,为岭南学派创始人。是岭南唯一诏准从祀孔庙的学者,有“岭南第一人”、“广东第一大儒”的称誉。曾自制以新会圭峰山长成的硬朗的茅草为材料的茅龙笔,字体苍劲有力,别具风格。陈献章遭逢明朝中叶的乱象,历经王振弄权(1435年)、土木之变(1449年)、明英宗夺门之变复辟(1457年)等社会动乱。一生清贫,都御史邓廷缵曾令番禺县每月给他米一石,陈拒不接受,说自己“有田二顷,耕之足矣”。又有按察使花了巨金买园林豪宅送他,他亦不受。陈献章的入学法门是“以静为主”,“端坐澄心,于静中养出端倪。”献章创立了岭南第一个颇具影响的学术流派——岭南学派。其弟子有湛若水、梁储、李承箕、林缉熙、张廷实、贺钦、陈茂烈、容一之、罗服周、潘汉、叶宏、谢佑、林廷{\fzk 𤩽}等。 \tabularnewline\hline
  唐寅 & \begin{description}
  \item[字] 伯虎\\子谓
  \item[号] 六如居士\\桃花庵主
  \item[谥] 
  \item[尊] 
  \item[生] 江苏苏州
  \end{description} & 唐寅(1470年3月6日-1524年1月4日),明代著名画家、文学家。字伯虎,又字子畏,以字行,号六如居士、桃花庵主、逃禅仙吏等,直隶吴县人,吴中四才子之一。在画史上又与沈周、文徵明、仇英合称“明四家”或“吴门四家”。民间有很多关于唐伯虎的传说,最为人熟悉的《唐伯虎点秋香》曾多次被改编成戏剧,以及拍成电视剧及电影,也宣传、加深了唐伯虎在民间的形象。唐寅出生于世商家庭,有一妹一弟,父亲唐广德,经营一家唐记酒店。唐寅作品以山水画、人物画闻名于世,其创作的多幅春宫图也为他个人添加了“风流才子”的名声。 \tabularnewline\hline
  沈周 & \begin{description}
  \item[字] 启南
  \item[号] 石田\\白石翁\\玉田生
  \item[谥] 
  \item[尊] 
  \item[生] 江苏苏州
  \end{description} & 沈周(1427年-1509年)字启南、号石田、白石翁、玉田生、有竹居主人等,明朝画家,吴门画派的创始人,明四家之一,长洲(今属江苏苏州市)人。沈周的书画流传很广,真伪混杂,较难分辨。文征明因此称他为飘然世外的“神仙中人”。 \tabularnewline\hline
  李梦阳 & \begin{description}
  \item[字] 恩赐
  \item[号] 空同子
  \item[谥] 
  \item[尊] 
  \item[生] 甘肃庆阳
  \end{description} & 李梦阳(1472年-1529年),又名献吉,字恩赐,号空同子,祖籍河南扶沟,出生于陕西庆阳(今甘肃),后又还归故里。明朝政治人物,文坛前七子之一。著有《空同集》。李梦阳为明朝初年研究唐诗的重要学者,乐府﹑歌行有相当成就,郭卓茂评论道:“有明一代研究唐诗的重要学者,中国古代文坛上胆大包天的诗人。”王维祯认为:“七言律自杜甫以后﹐善用顿挫倒插之法﹐惟梦阳一人。”他主要贡献在于诗歌理论批评,他所提出的“古体学习汉魏,近体学唐诗”的观念,相当具有指标性。他还提出“真诗乃在民间”的观点。他与何景明“倡导复古﹐文自西京﹑诗自中唐而下﹐一切吐弃。操觚谈艺之士﹐翁然宗之”。 \tabularnewline\hline
  李东阳 & \begin{description}
  \item[字] 宾之
  \item[号] 西涯
  \item[谥] 文正
  \item[尊] 
  \item[生] 湖南茶陵
  \end{description} & 李东阳(1447年-1516年),字宾之,号西涯,谥文正,明朝中叶重臣,文学家,书法家,茶陵诗派的核心人物。湖广茶陵县(今湖南茶陵)人,金吾左卫军籍。李东阳入阁多年,在朝时间长,地位高,不仅自己才学渊博,又能奖励后学,推荐隽才,因此不少文学之士都围聚在他周围,形成了一个颇有影响的诗人派别。李东阳也就在明中期一度领导文坛。因而《明史》中写道:“弘治时,宰相李东阳主文柄,天下翕然宗之。” \tabularnewline\hline
  何景明 & \begin{description}
  \item[字] 仲默
  \item[号] 白坡\\大复山人
  \item[谥] 
  \item[尊] 
  \item[生] 河南信阳
  \end{description} & 何景明(1483年-1521年),字仲默,号白坡,又号大复山人。河南信阳人。明朝作家。明朝文学家前七子之一,官至陕西提学副使。何景明工诗古文,与李梦阳皆提倡复古之学,天下从之,文体一变。在“七子”中,地位仅次于李梦阳,“天下语诗文,必并称何、李”(《明史‧何景明传》)。他也主张文宗秦、汉,古诗宗汉、魏,近体诗宗盛唐。 \tabularnewline\hline
  杨慎 & \begin{description}
  \item[字] 用修
  \item[号] 升庵
  \item[谥] 文宪
  \item[尊] 
  \item[生] 四川新都
  \end{description} & 杨慎(1488年12月8日-1559年8月8日),字用修,号升庵,别号博南山人、博南戍史,谥文宪,四川新都县(今成都市新都区马家镇升庵村)人,祖籍江西庐陵,为内阁首辅杨廷和之子,正德年间状元,官至翰林院修撰。大礼议事件中,因率领百官在左顺门求世宗改变皇考,而遭贬云南,终老于戍地。现成都市新都区仍存有其私家园林升庵桂湖。杨慎与解缙、徐渭合称“明朝三才子”。主要著作有《滇程记》、《丹铅总录》、《丹铅杂录》、《南诏野史》、《古音猎要》、《全蜀艺文志》、《春秋地名考》等。 \tabularnewline\hline
  薛蕙 & \begin{description}
  \item[字] 君采
  \item[号] 西原
  \item[谥] 
  \item[尊] 
  \item[生] 亳州
  \end{description} & 薛蕙(1489年-1541年),字君采,号西原,直隶亳州人,明朝政治人物。正德九年(1514年)登甲戌科进士,授刑部主事。明武宗南巡之争中,因进谏劝阻而受杖夺俸,随后引疾归乡。此后起用恢复官职,改吏部,历任吏部考功司郎中。嘉靖二年(1523年)大礼议事件中,廷臣数次进谏,薛蕙亦上疏劝阻。世宗读后大怒,夺俸三月,此后因事诬陷而归乡。薛蕙一生著有《西原集》10卷,《补遗》1卷,《五经杂录》、《大宁斋日录》五卷、《老子集解》、《庄子注》、《考功集》、《约言》和《西原遗书》二卷。 \tabularnewline\hline
  李攀龙 & \begin{description}
  \item[字] 于鳞
  \item[号] 沧溟
  \item[谥] 
  \item[尊] 
  \item[生] 山东济南
  \end{description} & 李攀龙(1514年-1570年),字于鳞,号沧溟,山东历城(今济南)人,明朝官员、文学家,“后七子”之首。是明朝知名作家,也是知名的文学评论家。他对于秦汉文学抱甚高的评价,并对唐诗颇多与他学者不同的贬抑看法。他所著的《答冯通书》就提到:“秦汉以后无文矣”。著有《沧溟集》。 \tabularnewline\hline
  王世贞 & \begin{description}
  \item[字] 元美
  \item[号] 凤洲\\弇州山人
  \item[谥] 
  \item[尊] 
  \item[生] 江苏太仓
  \end{description} & 王世贞(1526年-1590年),字元美,号凤洲,又号弇州山人,直隶太仓州(今江苏太仓)人,明朝文学家、史学家。“后七子”之一。世贞早年与李攀龙同为“后七子”领袖。攀龙死后,他独主诗坛二十年。“一时士大夫及山人、词客、衲子、羽流,莫不奔走门下。片言褒赏,声价骤起”。善诗,尤擅律,绝,倡导文学复古运动,有“文必秦汉,诗必盛唐”的主张。有《弇州山人四部稿》一百七十四卷、《弇山堂别集》一百卷(多载史事杂考)、《艺苑卮言》十二卷(南北曲源流与评论)、《鸣凤记》(剧本,以批严嵩为题材。王世贞之父被严嵩陷害死,作品大斥严氏罪行。)、《史乘考误》传世。不少学者认为《金瓶梅》作者兰陵笑笑生的真实身份便是王世贞。 \tabularnewline\hline
  王世懋 & \begin{description}
  \item[字] 敬美
  \item[号] 麟州\\损斋\\墙东生
  \item[谥] 
  \item[尊] 
  \item[生] 江苏太仓
  \end{description} & 王世懋(1536年-1588年),字敬美,号麟州,又号损斋,或曰墙东生。直隶太仓(今属江苏省)人。明朝政治人物。南京刑部尚书、史学家王世贞之弟。著有《王仪部集》、《二酋委谭摘录》、《名山游记》、《奉常集词》、《窥天外乘》、《艺圃撷余》等。《明史》附其传于王世贞传后。 \tabularnewline\hline
  胡应麟 & \begin{description}
  \item[字] 元瑞
  \item[号] 少室山人\\石羊生
  \item[谥] 
  \item[尊] 
  \item[生] 浙江金华
  \end{description} & 胡应麟(1551年-1602年),字元瑞,一字明瑞,号“少室山人”,又号“石羊生”,浙江金华兰溪人。他的《四部正伪》一书,上承宋濂的“诸子辩”,扩大检讨重要的古书,为古书辨伪。古书辨伪工作早发于刘知几、柳宗元,由胡应麟与姚际恒等续作。另著有《诗薮》、《华阳博议》、《九流绪论》、《经籍会通》、《史书占毕》、《庄岳委谈》、《唐同姓名录》、《二酉山房歌》、《少室山房笔丛》等。 \tabularnewline\hline
  钟惺 & \begin{description}
  \item[字] 伯敬
  \item[号] 退谷
  \item[谥] 
  \item[尊] 
  \item[生] 湖北天门
  \end{description} & 钟惺(1574—1625), 明代文学家。字伯敬,号退谷,湖广竟陵(今湖北天门市)人。万历三十八年(1610)进士。曾任工部主事,万历四十四年(1616)与林古度登泰山。后官至福建提学佥事。不久辞官归乡,闭户读书,晚年入寺院。其为人严冷,不喜接俗客,由此得谢人事,研读史书。他与同里谭元春共选《唐诗归》和《古诗归》(见《诗归》),名扬一时,形成“竟陵派”,世称“钟谭”。 \tabularnewline\hline
  谭元春 & \begin{description}
  \item[字] 友夏
  \item[号] 鹄湾\\蓑翁
  \item[谥] 
  \item[尊] 
  \item[生] 湖北天门
  \end{description} & 谭元春(1586~1637),湖广竟陵(今湖北天门市)人,字友夏,号鹄湾,别号蓑翁。明代文学家,天启间乡试第一,与同里钟惺同为“竟陵派”创始人,论文重视性灵,反对摹古,提倡幽深孤峭的风格,所作亦流于僻奥冷涩,有《谭友夏合集》。复社兴起后,他又加入了复社,被列为“复社四十八友”之一。 \tabularnewline\hline
  钱澄之 & \begin{description}
  \item[字] 幼光
  \item[号] 田间\\西顽道人
  \item[谥] 
  \item[尊] 
  \item[生] 安徽桐城
  \end{description} & 钱澄之(1612年-1693年)是一名明朝末年的诗人、官员。安徽桐城人。初名秉镫,字幼光;后改名澄之,字饮光,号田间,又号西顽道人。自小随父读书,十一岁能写文章,崇祯时中秀才。明崇祯初年,与方以智、孙临、方文、周岐等人成立诗社泽社。曾参加抗清活动,兵败后游历于江浙一带著述。王夫之推崇他“诗体整健”。著有《田间集》、《田间诗集》、《田间文集》、《藏山阁集》等。 \tabularnewline
  \bottomrule
\end{longtable}


%%% Local Variables:
%%% mode: latex
%%% TeX-engine: xetex
%%% TeX-master: "../../Main"
%%% End:
