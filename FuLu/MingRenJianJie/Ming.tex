%% -*- coding: utf-8 -*-
%% Time-stamp: <Chen Wang: 2018-07-20 00:18:24>

\subsection{明}

\begin{longtable}{|>{\centering\namefont\heiti}m{2em}|>{\centering\tiny}m{3.0em}|>{\xzfont\kaiti}m{7.3em}|}
  % \caption{秦王政}\
  \toprule
  \SimHei \normalsize 姓名 & \SimHei \normalsize 异名 & \SimHei \normalsize \hspace{2.5em}小传 \tabularnewline
  % \midrule
  \endfirsthead
  \toprule
  \SimHei \normalsize 姓名 & \SimHei \normalsize 异名 & \SimHei \normalsize \hspace{2.5em}小传 \tabularnewline 
  \midrule
  \endhead
  \midrule
  高棅 & \begin{description}
  \item[字] 廷礼\\彦恢
  \item[号] 漫士
  \item[谥] 
  \item[尊] 
  \item[生] 福建长乐
  \end{description} & 高棅(1350年-1423年),又名廷礼,字彦恢,号漫士,福建长乐人。闽中十才子之一。高棅为明朝初年研究唐诗的重要学者,所著的《唐诗品汇》为明初诗歌复古的里程碑,也是中国文学的重要评论著作。高棅著有《啸台集》、《水天清气集》、《唐诗品汇》、《唐诗拾遗》、《唐诗正声》。 \tabularnewline\hline
  高启 & \begin{description}
  \item[字] 季迪
  \item[号] 青丘子
  \item[谥] 
  \item[尊] 
  \item[生] 江苏苏州
  \end{description} & 高启(1336年-1373年,37岁),字季迪,号青丘,元末明初平江路(明改苏州府)长洲县(今江苏省苏州市)人,明初十才子之一。和宋濂、刘基合称“明初诗文三大家”。因得罪明太祖,以魏观案累文字狱,处腰斩。高启有诗才,其诗清新超拔,雄健豪迈,尤擅长于七言歌行,《四库全书总目提要》称:“拟汉魏似汉魏,拟六朝似六朝,拟唐似唐,拟宋似宋,凡古人所长,无不兼之。”与杨基、张羽、徐贲合称“吴中四杰”。景泰元年(1450年),徐庸搜集《缶鸣集》等遗篇,编为《高太史大全集》18卷。 \tabularnewline\hline
  
  \bottomrule
\end{longtable}


%%% Local Variables:
%%% mode: latex
%%% TeX-engine: xetex
%%% TeX-master: "../../Main"
%%% End:
