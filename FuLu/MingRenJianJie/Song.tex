%% -*- coding: utf-8 -*-
%% Time-stamp: <Chen Wang: 2018-07-19 00:19:09>

\subsection{南北两宋}

\begin{longtable}{|>{\centering\namefont\heiti}m{2em}|>{\centering\tiny}m{3.0em}|>{\xzfont\kaiti}m{7.3em}|}
  % \caption{秦王政}\
  \toprule
  \SimHei \normalsize 姓名 & \SimHei \normalsize 异名 & \SimHei \normalsize \hspace{2.5em}小传 \tabularnewline
  % \midrule
  \endfirsthead
  \toprule
  \SimHei \normalsize 姓名 & \SimHei \normalsize 异名 & \SimHei \normalsize \hspace{2.5em}小传 \tabularnewline 
  \midrule
  \endhead
  \midrule
  王禹偁 & \begin{description}
  \item[字] 元之
  \item[号] 
  \item[谥] 
  \item[尊] 王黄州
  \item[生] 山东菏泽
  \end{description} & 王禹偁(954年-1001年),字元之,济州钜野(今山东菏泽市钜野县)人,北宋文学家。王禹偁出身清寒,家庭世代务农。从小发愤求学,五岁便能够写诗。宋太宗太平兴国八年(983年)中进士,最初担任成武县主簿。他对仕途充满抱负,曾在《吾志》诗中表白:“吾生非不辰,吾志复不卑,致君望尧舜,学业根孔姬”。端拱元年(988年),他被召见入京,担任右拾遗、直史馆。他旋即进谏,以《端拱箴》来批评皇宫的奢侈生活。后来历任左司谏、知制诰、翰林学士。为人刚直,敢直言进谏,誓言要“兼磨断佞剑,拟树直言旗”。曾三次被贬职:于淳化二年(991年),一贬商州,于至道元年,二贬滁州,于咸平元年(998年),三贬黄州。故有“王黄州”之称。谪居黄州期间,以骈散相间之"黄州新建小竹楼记"抒发虽遭贬谪却心地坦荡,具达观旷逸之胸怀。 \tabularnewline\hline
  寇准 & \begin{description}
  \item[字] 平仲
  \item[号] 
  \item[谥] 忠愍
  \item[尊] 寇莱公
  \item[生] 陕西渭南
  \end{description} & 寇准(961年-1023年10月24日),北宋名相。字平仲,华州下邽(今陕西渭南)人。寇凖系春秋司寇苏氏裔孙,其父寇湘,后晋开运中,很有学问,应辟为魏王府记室参军。寇准出生于山西太谷县,年少时期豪爽嗜酒,性格大方,喜欢在家里大摆筵席。[注 1] 。皇祐四年,诏翰林学士孙抃撰神道碑,帝为篆其首曰“旌忠”。寇凖善诗能文,七绝尤有韵味,今传《寇忠湣诗集》三卷。 \tabularnewline\hline
  丁谓 & \begin{description}
  \item[字] 谓之\\公言
  \item[号] 
  \item[谥] 
  \item[尊] 丁晋公
  \item[生] 江苏苏州
  \end{description} & 丁谓(966年-1037年),字谓之,后更字公言,北宋时期苏州长州(今江苏苏州)人。善言谈,喜欢作诗,于图书、博奕、音律无一不精。出自寇准门下,太宗淳化三年(992年)进士,授大理寺评事、通判饶州事。真宗咸平初除三司户部判官,大中祥符初,权三司使。咸平五年(1012年)任户部侍郎。官至参政知事。当政后极力排斥寇准,乾兴元年(1022年)二月,再贬寇准为雷州司户参军。丁谓同党雷允恭因先帝陵寝工程事故,坐“擅移皇堂”罪,丁谓受牵连,贬为太子太保。后以“丁谓前后欺罔”罪,被贬崖州(今海南省琼山县)司户参军。以秘书省监致仕归里。景祐四年(1037年)病卒。著有《丁谓集》8卷、《虎丘集》50卷、《刀笔集》2卷、《青衿集》3卷、《知命集》1卷,皆佚。所著《天香传》,则是中国最早系统性针对沉香尤其是海南沉香所写的专著,书中记载沉香自古便为人所用,最早用于祭天礼地的场合,焚沉香祝祷。丁谓实地考察海南沉香,提出了四名十二状的分类法,被后世多人藉鉴。 \tabularnewline\hline
  陈尧佐 & \begin{description}
  \item[字] 希元
  \item[号] 
  \item[谥] 文惠
  \item[尊] 
  \item[生] 四川南充
  \end{description} & 陈尧佐(963年—1044年),字希元,阆州阆中新井县(今四川省南充市南部县)人。北宋大臣,官至同中书门下平章事、集贤殿大学士,太子太师致仕,追赠司空兼侍中,谥号“文惠”。 陈尧佐,字希元,号知余子。阆州阆中人。北宋大臣、水利专家、书法家、诗人。左谏议大夫陈省华次子,枢密使陈尧叟之弟。陈尧佐与长兄陈尧叟、弟陈尧咨皆中状元。端拱元年(988年),陈尧佐进士及第,授魏县、中牟县尉。咸平初年,任潮州通判。历官翰林学士、枢密副使、参知政事。宋仁宗时官至宰相,景祐四年(1037年),拜同中书门下平章事。康定元年(1040年),以太子太师致仕。庆历四年(1044年),陈尧佐去世,年八十二,赠司空兼侍中,谥号“文惠”。陈尧佐明吏事,工书法,喜欢写特大的隶书字,著有《潮阳编》、《野庐编》、《遣兴集》、《愚邱集》等。 \tabularnewline\hline
  林逋 & \begin{description}
  \item[字] 君复
  \item[号] 
  \item[谥] 和靖先生
  \item[尊] 
  \item[生] 浙江杭州
  \end{description} & 林逋(967年或968年─1028年),汉族,北宋诗人。字君复,后人称为和靖先生,钱塘人(今浙江杭州)。出生于儒学世家,恬淡好古,早年曾游历于江淮等地,隐居于西湖孤山,终身不仕,未娶妻,与梅花、仙鹤作伴,称为“梅妻鹤子”。宋真宗闻其名,赐粟帛,诏长吏岁时劳问。性孤高自好,喜恬淡,不趋名利,自谓:“然吾志之所适,非室家也,非功名富贵也,只觉青山绿水与我情相宜。”林逋善为诗,其词澄浃峭特,多奇句。其诗大都反映隐居生活,描写梅花尤其入神,苏轼高度赞扬林逋之诗、书及人品,并诗跋其书:“诗如东野不言寒,书似留台差少肉。”宋仁宗天圣六年(1028年)去世,享寿六十二岁,仁宗赐谥“和靖先生”。留有《林和靖诗集》。宋代桑世昌著有《林逋传》。 \tabularnewline\hline
  杨亿 & \begin{description}
  \item[字] 大年
  \item[号] 
  \item[谥] 文
  \item[尊] 杨文公
  \item[生] 福建浦城
  \end{description} & 杨亿(974年-1020年),字大年,人称杨文公。建州浦城(今属福建浦城县)人。北宋文学家。自幼是个神童,博览强记,太宗雍熙元年(984年),十一岁受宋太宗召试,授秘书省正字(掌管图书秘籍的次长),淳化三年(992年)赐进士及第,迁光禄寺丞。淳化四年,直集贤院。至道二年(996年)迁著作佐郎。大中祥符六年(1013年)以太常少卿分司西京。天禧二年(1018年)拜工部侍郎。官至工部侍郎。以“秉清节”自许,“性特刚劲寡合”,为“忠清鲠亮之士”。又好谈禅。又好写诗,善于西昆体,朱熹评之为“巧中犹有混成底意思,便巧得来不觉”,与刘筠、钱惟演等诗歌唱和,其编著《西昆酬唱集》,收录十七位诗人作品,共250首,多言学李商隐,不喜杜工部诗,谓为村夫子。杨亿曾为翰林学士兼史馆修撰。长于典章制度,真宗即位初,曾参预修《太宗实录》,咸平元年(998年)书成,景德二年(1005年)与王钦若主修《册府元龟》。在政治上支持丞相寇准抵抗辽兵入侵。又反对宋真宗大兴土木。卒谥文,故称杨文公。著作多佚,今存《武夷新集》20卷。《宋史》卷三○五有传。清人全祖望有《杨文公论》。 \tabularnewline\hline

  \bottomrule
\end{longtable}


%%% Local Variables:
%%% mode: latex
%%% TeX-engine: xetex
%%% TeX-master: "../../Main"
%%% End:
