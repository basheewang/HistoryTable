%% -*- coding: utf-8 -*-
%% Time-stamp: <Chen Wang: 2018-10-30 16:25:48>

\subsection{秦汉}

\begin{longtable}{|>{\centering\namefont\heiti}m{2em}|>{\centering\tiny}m{3.0em}|>{\xzfont\kaiti}m{7em}|}
    % \caption{秦王政}\
    \toprule
    \SimHei \normalsize 姓名 & \SimHei \normalsize 异名 & \SimHei \normalsize \hspace{2.5em}小传 \tabularnewline
    % \midrule
    \endfirsthead
    \toprule
    \SimHei \normalsize 姓名 & \SimHei \normalsize 异名 & \SimHei \normalsize \hspace{2.5em}小传 \tabularnewline 
    \midrule
    \endhead
    \midrule
    司马迁 & \begin{description}
    \item[字] 子长
    \item[号] 
    \item[谥] 
    \item[尊] 太史公
    \item[生] 龙门
    \end{description} & 司马迁(前145年(景帝五年)-约前86年(昭帝始元元年)),字子长,左冯翊夏阳(今山西河津)人(一说陕西韩城人),是中国西汉时期著名的史学家和文学家。司马迁所撰写的《史记》被公认为是中国史书的典范,首创的纪传体撰史方法为后来历代正史所传承,被后世尊称为史迁,又因曾任太史令,故自称太史公。 \tabularnewline\hline
    班固 & \begin{description}
    \item[字] 孟坚
    \item[号] 
    \item[谥] 
    \item[尊] 
    \item[生] 陕西咸阳
    \end{description} & 班固(东汉光武帝建武十(公元32)年-东汉和帝永元四(公元92)年),字孟坚,扶风安陵(今陕西咸阳)人,东汉史学家班彪之子,东汉历史学家,《汉书》的作者。 \tabularnewline\hline
    张衡 & \begin{description}
    \item[字] 平子
    \item[号] 
    \item[谥] 
    \item[尊] 
    \item[生] 南阳西鄂
    \end{description} &  张衡(78年-139年),字平子,南阳西鄂人,东汉士大夫、天文学家、地理学家、数学家、科学家、发明家及文学家,官至太史令、侍中、尚书。张衡一生成就不凡,曾制作以水力推动的浑天仪、发明能够探测震源方向的地动仪和指南车、发现月蚀的原因、绘制记录2,500颗星体的星图、计算圆周率准确至小数点后一个位、解释和确立浑天说的宇宙论;在文学方面,他创作了《二京赋》及《归田赋》等辞赋名篇,拓展了汉赋的文体与题材,被列为“汉赋四大家”之一。他开创了七言古诗的诗歌体裁,对中华文化有巨大贡献。张衡为备受尊崇的伟大科学家,成就与西方同时期的托勒密媲美。此外,他的地位也被现代天文学界所肯定。\tabularnewline\hline
    王粲 & \begin{description}
    \item[字] 仲宣
    \item[号] 
    \item[谥] 
    \item[尊] 
    \item[生] 山东微山
    \end{description} & 王粲(177年-217年2月17日),字仲宣,东汉山阳高平(今山东省微山县)人。擅长辞赋,建安七子之一,被誉为“七子之冠冕”。少有才名,为著名学者蔡邕所赏识。初平二年(192年),因关中骚乱,前往荆州依靠刘表,客居荆州十余年,有志不伸,心怀颇郁郁。建安十三年(208年),曹操南征荆州,不久,刘表病逝,其子刘琮举州投降,王粲也归曹操,深得曹氏父子信赖,赐爵关内侯。建安十八年(213年),魏王国建立,王粲任侍中。建安二十二年(216年),王粲随曹操南征孙权,于北还途中病逝,终年四十一岁。王粲善属文,其诗赋为建安七子之冠,又与曹植并称“曹王”。著《英雄记》,《三国志》记王粲著诗、赋、论、议近60篇,《隋书·经籍志》著录有文集十一卷。明人张溥辑有《王侍中集》。 \tabularnewline\hline

    \bottomrule
\end{longtable}


%%% Local Variables:
%%% mode: latex
%%% TeX-engine: xetex
%%% TeX-master: "../../Main"
%%% End:
