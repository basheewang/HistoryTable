%% -*- coding: utf-8 -*-
%% Time-stamp: <Chen Wang: 2018-07-20 00:07:51>

\subsection{辽金元}

\begin{longtable}{|>{\centering\namefont\heiti}m{2em}|>{\centering\tiny}m{3.0em}|>{\xzfont\kaiti}m{7.3em}|}
 % \caption{秦王政}\
 \toprule
 \SimHei \normalsize 姓名 & \SimHei \normalsize 异名 & \SimHei \normalsize \hspace{2.5em}小传 \tabularnewline
 % \midrule
 \endfirsthead
 \toprule
 \SimHei \normalsize 姓名 & \SimHei \normalsize 异名 & \SimHei \normalsize \hspace{2.5em}小传 \tabularnewline 
 \midrule
 \endhead
 \midrule
 元好问 & \begin{description}
 \item[字] 裕之
 \item[号] 遗山
 \item[谥] 
 \item[尊] 遗山先生
 \item[生] 山西忻州
 \end{description} & 元好(hào)问(1190年8月10日—1257年10月12日),字裕之,号遗山,世称遗山先生。太原秀容(今山西忻州)人。金末至大蒙古国时期著名文学家、历史学家。元好问自幼聪慧,有“神童”之誉。金宣宗兴定五年(1221年),元好问进士及第。正大元年(1224年),又以宏词科登第后,授权国史院编修,官至知制诰。金朝灭亡后,元好问被囚数年。晚年重回故乡,隐居不仕,于家中潜心著述。元宪宗七年(1257年),元好问逝世,年六十八。元好问是宋金对峙时期北方文学的主要代表、文坛盟主,又是金元之际在文学上承前启后的桥梁,被尊为“北方文雄”、“一代文宗”。他擅作诗、文、词、曲。其中以诗作成就最高,其“丧乱诗”尤为有名;其词为金代一朝之冠,可与两宋名家媲美;其散曲虽传世不多,但当时影响很大,有倡导之功。有《元遗山先生全集》、《中州集》。 \tabularnewline\hline
 方回 & \begin{description}
 \item[字] 万里\\渊甫
 \item[号] 虚谷\\紫阳山人
 \item[谥] 
 \item[尊] 
 \item[生] 安徽歙县
 \end{description} & 方回(1227年-1307年),字万里,一字渊甫,号虚谷,别号紫阳山人,徽州歙县(今属安徽)人。长期寓居钱塘,与宋遗民往来。元成宗大德十一年(1307年)卒。有《桐江集》六十五卷。另有《瀛奎律髓》49卷,其中本集卷二四《送丘子正以能书入都……》阿谀元廷为“今日朝廷贞观同”,为周密《癸辛杂识》别集卷上所深诋。 \tabularnewline\hline
 范梈 & \begin{description}
 \item[字] 亨父\\德机
 \item[号] 
 \item[谥] 
 \item[尊] 
 \item[生] 江西樟树
 \end{description} & 范梈(pēng)(1272年~1330年),字亨父,一名德机,清江(今江西樟树)人。元代诗人,生于宋度宗咸淳八年(1272年),幼孤贫,过目成诵,作文师宗颜延年、谢灵运,大德十一年(1307),至京师,在中丞董士选家担任家教。被荐为左卫教授,历官海南海北道廉访司照磨、翰林应奉、福建闽海道知事等,官至翰林院编修,以疾归里。工于诗,同时代的虞集称范椁诗“如唐临晋帖”,与虞集、杨载、揭傒斯齐名,被誉为“元诗四大家”之一。天历二年(1329年),以母病辞归,不久母卒。天历三年(1330年)范梈亦卒。人称“文白先生”。著有《木天禁语》、《诗学禁脔》。 \tabularnewline\hline
 乔吉 & \begin{description}
 \item[字] 梦符
 \item[号] 笙鹤翁\\惺惺道人
 \item[谥] 
 \item[尊] 
 \item[生] 山西太原
 \end{description} & 乔吉(1280年-1345年),又名乔吉甫,字梦符,号笙鹤翁,又号惺惺道人。中国元朝杂剧(元曲)家、散曲作家。乔吉为山西太原人,寓居杭州。他与张可久并称双璧。 \tabularnewline\hline
 杨维桢 & \begin{description}
 \item[字] 廉夫
 \item[号] 铁崖\\东维子
 \item[谥] 
 \item[尊] 
 \item[生] 浙江绍兴
 \end{description} & 杨维桢(1296年-1370年),又作维祯,字廉夫,号铁崖、东维子会稽(今浙江绍兴)人。元末明初政治人物。杨维翰之弟。成宗元贞二年(1296年)生,少时读书于铁崖山,其父杨宏在铁崖山麓筑楼,楼上藏书万卷,周围种数百株梅树,将梯子撤去,令其专心攻读,杨维桢苦读五年,每日用辘皿传递食物。泰定四年(1327年)中进士,授天台县尹,杭州四务提举。维桢为人倔强,诗文奇诡,喜做翻案文章,如《炮烙辞》一诗支持纣王。又以拟古乐府见称于时,是当时诗坛领袖,因“诗名擅一时,号铁崖体”,独领风骚。元末天下大乱,维桢避寓富春江一带,张士诚屡召不仕,迁苏州、松江等地,隐居不出,和文人“笔墨纵横,铅粉狼藉”,沉溺声色。与陆居仁、钱惟善被称为“元末三高士”。著有《东维子文集》、《铁崖先生古乐府》等。 \tabularnewline\hline
 萨都剌 & \begin{description}
 \item[字] 天锡
 \item[号] 直斋
 \item[谥] 
 \item[尊] 
 \item[生] 蒙古
 \end{description} & 萨都剌,又作萨都拉,(1272年或1300年-1355年),字天锡,庵号直斋,元代著名诗人、画家、书法家。蒙古化色目人(一说回回人)。出身将门,但据其《溪行中秋玩月》诗自序,幼年家贫。早年科举不顺,以经商为业。到泰定四年(1327年)才中进士,一生只做过一些卑微的官职,包括京口录事司达鲁花赤、江南行御史台掾史、燕南河北道肃政廉访司照磨、闽海福建道肃政廉访司知事、燕南河北道肃政廉访司经历等职等。为官清廉,有政绩,不趋炎附势,因得罪权贵而被贬。 \tabularnewline\hline
 赵汸 & \begin{description}
 \item[字] 子常
 \item[号] 
 \item[谥] 
 \item[尊] 东山先生
 \item[生] 安徽休宁
 \end{description} & 赵汸(1319年-1369年),字子常。安徽休宁人。生于元仁宗延佑六年(1319年),读朱子《四书》,多所疑难,乃尽取朱子书读之。师事黄泽,专攻《春秋》《易》象之学。后复从临川虞集游,获闻吴澄之学,思想可见于《对江右六君子策略》,主张“澄心默坐,涵养本源,以为致思之地”,而后“凡所得于师之指及文字奥义有未通者,必用向上功夫以求之”。赵汸生于乱世,淡泊名利,隐居著述,作“东山精舍”以奉母,学者称东山先生,邑人建商山书院,聘赵汸、朱升为书院山长。洪武二年(1369年)召修《元史》,完成初稿159卷。半年后乞还东山。未几,以病卒。著有《葬书问对》、《东山存槁》、《周易文诠》、《春秋集传》等。《明史·儒林传》有传。 \tabularnewline
 \bottomrule
\end{longtable}


%%% Local Variables:
%%% mode: latex
%%% TeX-engine: xetex
%%% TeX-master: "../../Main"
%%% End:
