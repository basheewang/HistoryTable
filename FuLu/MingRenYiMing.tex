%% -*- coding: utf-8 -*-
%% Time-stamp: <Chen Wang: 2018-07-18 09:04:31>

\section{名人异名录}

常见于各种诗话、词话,多以名人之字、号、尊称、谥号等,外加小传。

\begin{longtable}{|>{\centering\namefont\heiti}m{2em}|>{\centering\tiny}m{3.0em}|>{\xzfont\kaiti}m{7.3em}|}
    % \caption{秦王政}\
    \toprule
    \SimHei \normalsize 姓名 & \SimHei \normalsize 异名 & \SimHei \normalsize \hspace{2.5em}小传 \tabularnewline
    % \midrule
    \endfirsthead
    \toprule
    \SimHei \normalsize 姓名 & \SimHei \normalsize 异名 & \SimHei \normalsize \hspace{2.5em}小传 \tabularnewline 
    \midrule
    \endhead
    \midrule
    屈平 & \begin{description}
    \item[字] 屈原
    \item[号] 屈子
    \item[谥] 
    \item[尊] 三闾大夫
    \item[生] 楚国
    \end{description} & 屈原(约前340年-约前278年6月6日),芈姓,屈氏,名平,字原,楚国人(今湖北秭归),是古帝高阳氏的后裔,其自作词曰:“帝高阳之苗裔兮,朕皇考曰伯庸。”,其先祖屈瑕受楚武王封于屈地,因以屈为氏,名平。屈,昭,景为楚国大姓,官拜左徒,左徒多以贵族近臣任之,左徒任务有四 “议国事”、“出号令”、“接遇宾客”、“应对诸侯”。 \tabularnewline\hline
    司马迁 & \begin{description}
    \item[字] 子长
    \item[号] 
    \item[谥] 
    \item[尊] 太史公
    \item[生] 龙门
    \end{description} & 司马迁(前145年(景帝五年)-约前86年(昭帝始元元年)),字子长,左冯翊夏阳(今山西河津)人(一说陕西韩城人),是中国西汉时期著名的史学家和文学家。司马迁所撰写的《史记》被公认为是中国史书的典范,首创的纪传体撰史方法为后来历代正史所传承,被后世尊称为史迁,又因曾任太史令,故自称太史公。 \tabularnewline\hline
    班固 & \begin{description}
    \item[字] 孟坚
    \item[号] 
    \item[谥] 
    \item[尊] 
    \item[生] 陕西咸阳
    \end{description} & 班固(东汉光武帝建武十(公元32)年-东汉和帝永元四(公元92)年),字孟坚,扶风安陵(今陕西咸阳)人,东汉史学家班彪之子,东汉历史学家,《汉书》的作者。 \tabularnewline\hline
    张衡 & \begin{description}
    \item[字] 平子
    \item[号] 
    \item[谥] 
    \item[尊] 
    \item[生] 南阳西鄂
    \end{description} &  张衡(78年-139年),字平子,南阳西鄂人,东汉士大夫、天文学家、地理学家、数学家、科学家、发明家及文学家,官至太史令、侍中、尚书。张衡一生成就不凡,曾制作以水力推动的浑天仪、发明能够探测震源方向的地动仪和指南车、发现月蚀的原因、绘制记录2,500颗星体的星图、计算圆周率准确至小数点后一个位、解释和确立浑天说的宇宙论;在文学方面,他创作了《二京赋》及《归田赋》等辞赋名篇,拓展了汉赋的文体与题材,被列为“汉赋四大家”之一。他开创了七言古诗的诗歌体裁,对中华文化有巨大贡献。张衡为备受尊崇的伟大科学家,成就与西方同时期的托勒密媲美。此外,他的地位也被现代天文学界所肯定。\tabularnewline\hline

    \bottomrule
\end{longtable}


%%% Local Variables:
%%% mode: latex
%%% TeX-engine: xetex
%%% TeX-master: "../Main"
%%% End:
