%% -*- coding: utf-8 -*-
%% Time-stamp: <Chen Wang: 2019-12-26 10:11:06>

\subsection{武信王\tiny(924-929)}

\subsubsection{生平}

楚武信王高季兴(858年-929年),原名高季昌,因避后唐庄宗李存勗祖父李国昌的名讳,改为季兴,字贻孙,陕州峡石(今河南三门峡东南)人,五代十國时期荆南建立者(楚王),以江陵一城周旋于中原、诸侯之间,长于纵横之术。自称东魏司徒高昂之后。

高季兴幼年为汴州商人李七郎(李七郎后为朱全忠义子,改名朱友让)家奴,后为朱友让收为义子,改姓为朱,由此得入军门,为朱全忠亲随牙将,因破凤翔救唐昭宗有功,被唐昭宗授“迎銮毅勇功臣”之号,迁宋州刺史。又随朱全忠扫荡青州,累功升颍州防御使,并復姓高氏。907年朱全忠称帝,建立后梁,授其为荆南节度使。但由于战乱,本来荆南所辖八州只有江陵一城为季昌所领。开平二年,与雷彦恭、马殷、杨行密鏖战,互有胜负。加同中书门下平章事。开平四年,大败马楚于油口。后梁末帝朱友贞乾化三年(913年),被册封为渤海王。是年九月,季昌私造战舰五百艘,招募亡命之徒,与杨吴、前蜀交好,后梁对其失去控制。乾化四年(914年)春正月,季昌試圖攻取前蜀的夔、万、忠、涪四州,結果大敗而還。贞明三年,季昌修筑堤坝以防长江水患,荆南人称高氏堤。龙德元年,季昌命都指挥使倪可福修江陵外城。

至后唐灭梁,高季兴向后唐称臣,携300骑入朝觐见,升中书令。在唐时建议李存勗攻取前蜀,李存勗认为十分有道理。但险些为后唐所留,季兴丢弃辎重和随从,连夜赶路才侥幸逃脱。回到江陵后,季兴认为李存勗耽于声色,定不能长久,于是修缮城池,招纳梁军旧部,以备万全之策。后唐庄宗同光二年(924年),后唐帝李存勗为笼络其心,册封其为南平王。随即与后唐一同攻蜀,不克而还。而蜀国仍被灭亡。消息传至荆南,季兴为之前给后唐出计而感到十分懊悔。高季兴在荆南,常截留各国供品,或也为讨得赐物向诸国称臣,反复无常,时称「高赖子」。

其后后唐灭前蜀,未几,又逢后唐的鄴都之變,李存勗被杀,天成元年,季兴趁机向李嗣源要回了夔、忠、万、归、峡五州。高季兴截获蜀地入朝贡物,又厚颜向后唐索地,妄图扩大地盘,后唐明宗李嗣源怒其无耻,罢其官爵,发湖南和蜀地两地兵马来征,荆南不敌,辖地日蹙,求兵于南吴。后因江南雨季,粮草不济,后唐罢兵,方才逃此一劫,此后归顺南吴,得封秦王。后唐明宗天成三年(928年),马楚兴师进攻江陵,季兴不敌。九月,在白田击败楚军。然而后唐又兴兵来攻。十二月十五日(阳历为929年1月28日)高季兴病死,时年七十一。葬于江陵城西的龙山乡。其子高从诲继位,重新向后唐称臣,因此后唐始追封高季兴为楚王,谥武信(楚武信王),故南平又被称为北楚。

\subsubsection{同光}

\begin{longtable}{|>{\centering\scriptsize}m{2em}|>{\centering\scriptsize}m{1.3em}|>{\centering}m{8.8em}|}
  % \caption{秦王政}\
  \toprule
  \SimHei \normalsize 年数 & \SimHei \scriptsize 公元 & \SimHei 大事件 \tabularnewline
  % \midrule
  \endfirsthead
  \toprule
  \SimHei \normalsize 年数 & \SimHei \scriptsize 公元 & \SimHei 大事件 \tabularnewline
  \midrule
  \endhead
  \midrule
  元年 & 924 & \tabularnewline\hline
  二年 & 925 & \tabularnewline\hline
  三年 & 926 & \tabularnewline
  \bottomrule
\end{longtable}

\subsubsection{天成}

\begin{longtable}{|>{\centering\scriptsize}m{2em}|>{\centering\scriptsize}m{1.3em}|>{\centering}m{8.8em}|}
  % \caption{秦王政}\
  \toprule
  \SimHei \normalsize 年数 & \SimHei \scriptsize 公元 & \SimHei 大事件 \tabularnewline
  % \midrule
  \endfirsthead
  \toprule
  \SimHei \normalsize 年数 & \SimHei \scriptsize 公元 & \SimHei 大事件 \tabularnewline
  \midrule
  \endhead
  \midrule
  元年 & 926 & \tabularnewline\hline
  二年 & 927 & \tabularnewline\hline
  三年 & 928 & \tabularnewline
  \bottomrule
\end{longtable}

\subsubsection{乾贞}

\begin{longtable}{|>{\centering\scriptsize}m{2em}|>{\centering\scriptsize}m{1.3em}|>{\centering}m{8.8em}|}
  % \caption{秦王政}\
  \toprule
  \SimHei \normalsize 年数 & \SimHei \scriptsize 公元 & \SimHei 大事件 \tabularnewline
  % \midrule
  \endfirsthead
  \toprule
  \SimHei \normalsize 年数 & \SimHei \scriptsize 公元 & \SimHei 大事件 \tabularnewline
  \midrule
  \endhead
  \midrule
  元年 & 928 & \tabularnewline
  \bottomrule
\end{longtable}



%%% Local Variables:
%%% mode: latex
%%% TeX-engine: xetex
%%% TeX-master: "../../Main"
%%% End:
