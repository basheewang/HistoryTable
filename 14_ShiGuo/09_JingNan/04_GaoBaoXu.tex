%% -*- coding: utf-8 -*-
%% Time-stamp: <Chen Wang: 2019-12-26 10:12:04>

\subsection{高保勗\tiny(960-962)}

\subsubsection{生平}

高保勗(924年-962年),字省躬,五代時期荊南君主(荊南節度使)。為高從誨之第十子,高保融之弟。高保勗幼時為高從誨所喜愛,高從誨因事盛怒,見到高保勗必釋然而笑,是故百姓稱之為「萬事休」。

宋太祖建隆元年(960年),高保融因病去世,其子高繼沖年紀尚小,因此遺命高保勗繼位,總判內外軍馬事。不久,即為宋任命為荊南節度使。

高保勗少時多病,體態瘦弱,但頗有治事之才。然而繼位後,放縱荒淫而沒有節制,白天召娼妓至官府,而挑選強壯的士兵,命其隨便調戲淫謔,然後自己再和姬妾一同觀賞做為娛樂。又喜歡營造亭台樓閣,花費人力物力無數,而不理國政,人民都很不滿。

建隆三年(962年),高保勗因病去世,被宋朝贈侍中。遺命其姪即高保融之子高繼沖嗣位。

高保勗去世後數月,南平即為宋所滅,有附會者即以其綽號「萬事休」為預兆。

\subsubsection{建隆}

\begin{longtable}{|>{\centering\scriptsize}m{2em}|>{\centering\scriptsize}m{1.3em}|>{\centering}m{8.8em}|}
  % \caption{秦王政}\
  \toprule
  \SimHei \normalsize 年数 & \SimHei \scriptsize 公元 & \SimHei 大事件 \tabularnewline
  % \midrule
  \endfirsthead
  \toprule
  \SimHei \normalsize 年数 & \SimHei \scriptsize 公元 & \SimHei 大事件 \tabularnewline
  \midrule
  \endhead
  \midrule
  元年 & 960 & \tabularnewline\hline
  二年 & 961 & \tabularnewline\hline
  三年 & 962 & \tabularnewline
  \bottomrule
\end{longtable}




%%% Local Variables:
%%% mode: latex
%%% TeX-engine: xetex
%%% TeX-master: "../../Main"
%%% End:
