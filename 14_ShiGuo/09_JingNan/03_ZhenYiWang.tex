%% -*- coding: utf-8 -*-
%% Time-stamp: <Chen Wang: 2021-11-01 15:50:00>

\subsection{贞懿王高保融\tiny(948-960)}

\subsubsection{生平}

南平貞懿王高保融(920年-960年),字德長,五代時期荊南君主(南平王)。為高從誨之第三子。

後漢隱帝劉承祐乾祐元年(948年),南平王高從誨去世,高保融繼位。不久,即被後漢任命為荊南節度使、同平章事、兼侍中。後周太祖郭威廣順元年(951年),被封為渤海郡王。顯德元年(954年),再被進封為南平王。

高保融個性遲鈍緩慢,沒有什麼才能,無論事情大小,皆委由其弟高保勗決定。宋太祖趙匡胤建隆元年(960年),宋朝建立後,高保融愈發感到恐懼,因此一年之間三次進貢。同年,因病去世,贈太尉,諡貞懿王。其子高繼沖年紀尚小,因此遺命高保勗繼位。

\subsubsection{乾佑}

\begin{longtable}{|>{\centering\scriptsize}m{2em}|>{\centering\scriptsize}m{1.3em}|>{\centering}m{8.8em}|}
  % \caption{秦王政}\
  \toprule
  \SimHei \normalsize 年数 & \SimHei \scriptsize 公元 & \SimHei 大事件 \tabularnewline
  % \midrule
  \endfirsthead
  \toprule
  \SimHei \normalsize 年数 & \SimHei \scriptsize 公元 & \SimHei 大事件 \tabularnewline
  \midrule
  \endhead
  \midrule
  元年 & 948 & \tabularnewline\hline
  二年 & 949 & \tabularnewline\hline
  三年 & 950 & \tabularnewline
  \bottomrule
\end{longtable}

\subsubsection{广顺}

\begin{longtable}{|>{\centering\scriptsize}m{2em}|>{\centering\scriptsize}m{1.3em}|>{\centering}m{8.8em}|}
  % \caption{秦王政}\
  \toprule
  \SimHei \normalsize 年数 & \SimHei \scriptsize 公元 & \SimHei 大事件 \tabularnewline
  % \midrule
  \endfirsthead
  \toprule
  \SimHei \normalsize 年数 & \SimHei \scriptsize 公元 & \SimHei 大事件 \tabularnewline
  \midrule
  \endhead
  \midrule
  元年 & 951 & \tabularnewline\hline
  二年 & 952 & \tabularnewline\hline
  三年 & 953 & \tabularnewline
  \bottomrule
\end{longtable}

\subsubsection{显德}

\begin{longtable}{|>{\centering\scriptsize}m{2em}|>{\centering\scriptsize}m{1.3em}|>{\centering}m{8.8em}|}
  % \caption{秦王政}\
  \toprule
  \SimHei \normalsize 年数 & \SimHei \scriptsize 公元 & \SimHei 大事件 \tabularnewline
  % \midrule
  \endfirsthead
  \toprule
  \SimHei \normalsize 年数 & \SimHei \scriptsize 公元 & \SimHei 大事件 \tabularnewline
  \midrule
  \endhead
  \midrule
  元年 & 954 & \tabularnewline\hline
  二年 & 955 & \tabularnewline\hline
  三年 & 956 & \tabularnewline\hline
  四年 & 957 & \tabularnewline\hline
  五年 & 958 & \tabularnewline\hline
  六年 & 959 & \tabularnewline\hline
  七年 & 960 & \tabularnewline
  \bottomrule
\end{longtable}



%%% Local Variables:
%%% mode: latex
%%% TeX-engine: xetex
%%% TeX-master: "../../Main"
%%% End:
