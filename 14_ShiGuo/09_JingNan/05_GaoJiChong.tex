%% -*- coding: utf-8 -*-
%% Time-stamp: <Chen Wang: 2019-12-26 10:12:19>

\subsection{高继冲\tiny(962-963)}

\subsubsection{生平}

高繼沖(943年-973年),字成和(一作字贊平),五代十國荊南政權末期君主(荊南節度使)。為高保融之長子,高保勗之姪。

宋太祖趙匡胤建隆三年(962年),高保勗因病去世,遺命高繼沖權判內外軍馬事以繼其位。後來高繼沖亦被宋任命為荊南節度使。

同年,湖南的武平節度使周行逢亦去世,年僅11歲的周保權嗣位,而境內大將張文表叛變,周保權向宋朝求援。建隆四年(963年),宋軍討伐張文表,假道荊南,發生荊湖之戰,趁機控制南平都城江陵(今湖北江陵)城巷,高繼沖只得納地以歸,南平亡。

南平亡後,宋一度仍任命高繼沖為荊南節度使。不久,高繼沖舉族歸朝,被改命為武寧節度使(約在今江蘇、安徽一帶)。

開寶六年(973年)高繼沖於武寧節度使任內去世,贈侍中。高繼沖鎮守彭門(今江蘇徐州),政事委諸僚佐,然有德政,因此被百姓請求留葬當地,但不被宋太祖允許。

\subsubsection{建隆}

\begin{longtable}{|>{\centering\scriptsize}m{2em}|>{\centering\scriptsize}m{1.3em}|>{\centering}m{8.8em}|}
  % \caption{秦王政}\
  \toprule
  \SimHei \normalsize 年数 & \SimHei \scriptsize 公元 & \SimHei 大事件 \tabularnewline
  % \midrule
  \endfirsthead
  \toprule
  \SimHei \normalsize 年数 & \SimHei \scriptsize 公元 & \SimHei 大事件 \tabularnewline
  \midrule
  \endhead
  \midrule
  元年 & 962 & \tabularnewline\hline
  二年 & 963 & \tabularnewline
  \bottomrule
\end{longtable}




%%% Local Variables:
%%% mode: latex
%%% TeX-engine: xetex
%%% TeX-master: "../../Main"
%%% End:
