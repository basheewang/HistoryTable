%% -*- coding: utf-8 -*-
%% Time-stamp: <Chen Wang: 2019-12-26 10:10:54>


\section{荆南\tiny(924-963)}

\subsection{简介}

荆南(924年-963年),又称南平、北楚,是五代时十国之一,但值得注意的是,荊南雖曰「十國」之一,但自始自終並未真正稱帝,事實上僅能算是一割據政權。高季兴所建。统治范围包括今湖北的江陵、公安一带。

后梁开平元年(907年)高季兴任荆南节度使。当时荆南所辖的10州为邻道侵夺,只有江陵一城。高季兴到任后,招集流亡,民渐复业,又收用一些文武官作辅佐,暗中准备割据。后唐同光二年(924年)封为南平王,建都荆州(今湖北荆州市荆州区)[來源請求],史称「南平」。又以方镇名为「荆南」,后世以此称之。

荆南虽地狭兵弱,但却是南北的交通要冲。其时南汉、闽、楚皆向后梁称臣,而每年贡奉均假道于荆南;因此高季兴便邀留使者,劫其财物。至南汉、闽、楚各称帝后,高氏对南北称帝诸国,一概上表称臣,以获取赏赐和维持商贸往来,由是被诸国视为“高赖子”。据有今湖北江陵、公安一带,建都荆州(今湖北江陵)。[來源請求]或后唐灭前蜀以后,高季兴得到了归、峡二州。他本欲夺取夔、忠、万等州,终不敌后唐而作罢。929年,高季兴死后,其子高从诲继位,重新向后唐称臣,因此后唐明宗始追封高季兴为楚王,谥武信(楚武信王),故南平又被称为北楚。后经高保融、高保勗,直到第五主高继冲,于宋太祖建隆四年荊湖之戰(963年)戰敗,纳地归降。

由高季兴公元907年担任节度使至荆南963年亡国,前后历五十七年。

%% -*- coding: utf-8 -*-
%% Time-stamp: <Chen Wang: 2021-11-01 15:49:20>

\subsection{武信王高季兴\tiny(924-929)}

\subsubsection{生平}

楚武信王高季兴(858年-929年),原名高季昌,因避后唐庄宗李存勗祖父李国昌的名讳,改为季兴,字贻孙,陕州峡石(今河南三门峡东南)人,五代十國时期荆南建立者(楚王),以江陵一城周旋于中原、诸侯之间,长于纵横之术。自称东魏司徒高昂之后。

高季兴幼年为汴州商人李七郎(李七郎后为朱全忠义子,改名朱友让)家奴,后为朱友让收为义子,改姓为朱,由此得入军门,为朱全忠亲随牙将,因破凤翔救唐昭宗有功,被唐昭宗授“迎銮毅勇功臣”之号,迁宋州刺史。又随朱全忠扫荡青州,累功升颍州防御使,并復姓高氏。907年朱全忠称帝,建立后梁,授其为荆南节度使。但由于战乱,本来荆南所辖八州只有江陵一城为季昌所领。开平二年,与雷彦恭、马殷、杨行密鏖战,互有胜负。加同中书门下平章事。开平四年,大败马楚于油口。后梁末帝朱友贞乾化三年(913年),被册封为渤海王。是年九月,季昌私造战舰五百艘,招募亡命之徒,与杨吴、前蜀交好,后梁对其失去控制。乾化四年(914年)春正月,季昌試圖攻取前蜀的夔、万、忠、涪四州,結果大敗而還。贞明三年,季昌修筑堤坝以防长江水患,荆南人称高氏堤。龙德元年,季昌命都指挥使倪可福修江陵外城。

至后唐灭梁,高季兴向后唐称臣,携300骑入朝觐见,升中书令。在唐时建议李存勗攻取前蜀,李存勗认为十分有道理。但险些为后唐所留,季兴丢弃辎重和随从,连夜赶路才侥幸逃脱。回到江陵后,季兴认为李存勗耽于声色,定不能长久,于是修缮城池,招纳梁军旧部,以备万全之策。后唐庄宗同光二年(924年),后唐帝李存勗为笼络其心,册封其为南平王。随即与后唐一同攻蜀,不克而还。而蜀国仍被灭亡。消息传至荆南,季兴为之前给后唐出计而感到十分懊悔。高季兴在荆南,常截留各国供品,或也为讨得赐物向诸国称臣,反复无常,时称「高赖子」。

其后后唐灭前蜀,未几,又逢后唐的鄴都之變,李存勗被杀,天成元年,季兴趁机向李嗣源要回了夔、忠、万、归、峡五州。高季兴截获蜀地入朝贡物,又厚颜向后唐索地,妄图扩大地盘,后唐明宗李嗣源怒其无耻,罢其官爵,发湖南和蜀地两地兵马来征,荆南不敌,辖地日蹙,求兵于南吴。后因江南雨季,粮草不济,后唐罢兵,方才逃此一劫,此后归顺南吴,得封秦王。后唐明宗天成三年(928年),马楚兴师进攻江陵,季兴不敌。九月,在白田击败楚军。然而后唐又兴兵来攻。十二月十五日(阳历为929年1月28日)高季兴病死,时年七十一。葬于江陵城西的龙山乡。其子高从诲继位,重新向后唐称臣,因此后唐始追封高季兴为楚王,谥武信(楚武信王),故南平又被称为北楚。

\subsubsection{同光}

\begin{longtable}{|>{\centering\scriptsize}m{2em}|>{\centering\scriptsize}m{1.3em}|>{\centering}m{8.8em}|}
  % \caption{秦王政}\
  \toprule
  \SimHei \normalsize 年数 & \SimHei \scriptsize 公元 & \SimHei 大事件 \tabularnewline
  % \midrule
  \endfirsthead
  \toprule
  \SimHei \normalsize 年数 & \SimHei \scriptsize 公元 & \SimHei 大事件 \tabularnewline
  \midrule
  \endhead
  \midrule
  元年 & 924 & \tabularnewline\hline
  二年 & 925 & \tabularnewline\hline
  三年 & 926 & \tabularnewline
  \bottomrule
\end{longtable}

\subsubsection{天成}

\begin{longtable}{|>{\centering\scriptsize}m{2em}|>{\centering\scriptsize}m{1.3em}|>{\centering}m{8.8em}|}
  % \caption{秦王政}\
  \toprule
  \SimHei \normalsize 年数 & \SimHei \scriptsize 公元 & \SimHei 大事件 \tabularnewline
  % \midrule
  \endfirsthead
  \toprule
  \SimHei \normalsize 年数 & \SimHei \scriptsize 公元 & \SimHei 大事件 \tabularnewline
  \midrule
  \endhead
  \midrule
  元年 & 926 & \tabularnewline\hline
  二年 & 927 & \tabularnewline\hline
  三年 & 928 & \tabularnewline
  \bottomrule
\end{longtable}

\subsubsection{乾贞}

\begin{longtable}{|>{\centering\scriptsize}m{2em}|>{\centering\scriptsize}m{1.3em}|>{\centering}m{8.8em}|}
  % \caption{秦王政}\
  \toprule
  \SimHei \normalsize 年数 & \SimHei \scriptsize 公元 & \SimHei 大事件 \tabularnewline
  % \midrule
  \endfirsthead
  \toprule
  \SimHei \normalsize 年数 & \SimHei \scriptsize 公元 & \SimHei 大事件 \tabularnewline
  \midrule
  \endhead
  \midrule
  元年 & 928 & \tabularnewline
  \bottomrule
\end{longtable}



%%% Local Variables:
%%% mode: latex
%%% TeX-engine: xetex
%%% TeX-master: "../../Main"
%%% End:

%% -*- coding: utf-8 -*-
%% Time-stamp: <Chen Wang: 2021-11-01 15:49:28>

\subsection{文献王高從誨\tiny(928-948)}

\subsubsection{生平}

高從誨(891年-948年),字遵聖,五代時期荊南君主(南平王)。為高季興之長子。曾仕於後梁中央政府,高季興為荊南節度使時,告歸其父,被高季興任命為馬步軍都指揮使、行軍司馬。

後唐明宗李嗣源天成三年(928年)十二月十五日(929年1月28日),被後唐冊封為南平王的高季興去世,高從誨嗣位。由於高季興在位末期曾與後唐決裂,並向南吳稱臣,而唐強吳弱、唐近吳遠,因此高從誨嗣位後,回歸向後唐稱臣,為後唐任命為荊南節度使,兼侍中。

長興三年(932年),被封為渤海王。後唐閔帝李從厚應順元年(934年)被改封南平王。

高從誨與部屬聊天,談到鄰國楚王馬希範非常奢侈。高從誨說:「像馬王那樣,可以說是大丈夫了」。孫光憲對答說:「天子諸侯,禮有等差。馬希範那種乳臭兒驕奢僭越,圖一時之樂,不謀遠慮,沒多久就要滅亡了,又有什麼好羨慕的!」高從誨久而悟,說:「你說的是」。之後,高從誨對梁震說:「我自己反省平生花費,實在太多了」。於是捨棄各種玩樂,以讀書作為消遣,省刑薄賦,南平國於是十分安寧。

荊南(南平)地狹兵弱,但因位處交通要道,每年各地區向中原政權的進貢,只要經過荊南,高季興、高從誨父子就會款待使者,掠奪財物,等到對方加以款待或讚賞,就把財物歸還,而且還會覺得這種行為很丟臉。後來後唐、後晉、遼國、後漢先後據有中原,南漢、閩國、南吳、南唐、後蜀皆稱帝,高從誨為求賞賜向他們都稱臣,所以各國都叫他們為「高賴子」或是「高無賴」。

後漢隱帝劉承祐乾祐元年(948年),高從誨去世,贈尚書令,諡文獻王。其子高保融繼位。

\subsubsection{乾贞}

\begin{longtable}{|>{\centering\scriptsize}m{2em}|>{\centering\scriptsize}m{1.3em}|>{\centering}m{8.8em}|}
  % \caption{秦王政}\
  \toprule
  \SimHei \normalsize 年数 & \SimHei \scriptsize 公元 & \SimHei 大事件 \tabularnewline
  % \midrule
  \endfirsthead
  \toprule
  \SimHei \normalsize 年数 & \SimHei \scriptsize 公元 & \SimHei 大事件 \tabularnewline
  \midrule
  \endhead
  \midrule
  元年 & 929 & \tabularnewline
  \bottomrule
\end{longtable}

\subsubsection{天成}

\begin{longtable}{|>{\centering\scriptsize}m{2em}|>{\centering\scriptsize}m{1.3em}|>{\centering}m{8.8em}|}
  % \caption{秦王政}\
  \toprule
  \SimHei \normalsize 年数 & \SimHei \scriptsize 公元 & \SimHei 大事件 \tabularnewline
  % \midrule
  \endfirsthead
  \toprule
  \SimHei \normalsize 年数 & \SimHei \scriptsize 公元 & \SimHei 大事件 \tabularnewline
  \midrule
  \endhead
  \midrule
  元年 & 929 & \tabularnewline\hline
  二年 & 930 & \tabularnewline
  \bottomrule
\end{longtable}

\subsubsection{长兴}

\begin{longtable}{|>{\centering\scriptsize}m{2em}|>{\centering\scriptsize}m{1.3em}|>{\centering}m{8.8em}|}
  % \caption{秦王政}\
  \toprule
  \SimHei \normalsize 年数 & \SimHei \scriptsize 公元 & \SimHei 大事件 \tabularnewline
  % \midrule
  \endfirsthead
  \toprule
  \SimHei \normalsize 年数 & \SimHei \scriptsize 公元 & \SimHei 大事件 \tabularnewline
  \midrule
  \endhead
  \midrule
  元年 & 930 & \tabularnewline\hline
  二年 & 931 & \tabularnewline\hline
  三年 & 932 & \tabularnewline\hline
  四年 & 933 & \tabularnewline
  \bottomrule
\end{longtable}

\subsubsection{应顺}

\begin{longtable}{|>{\centering\scriptsize}m{2em}|>{\centering\scriptsize}m{1.3em}|>{\centering}m{8.8em}|}
  % \caption{秦王政}\
  \toprule
  \SimHei \normalsize 年数 & \SimHei \scriptsize 公元 & \SimHei 大事件 \tabularnewline
  % \midrule
  \endfirsthead
  \toprule
  \SimHei \normalsize 年数 & \SimHei \scriptsize 公元 & \SimHei 大事件 \tabularnewline
  \midrule
  \endhead
  \midrule
  元年 & 934 & \tabularnewline
  \bottomrule
\end{longtable}

\subsubsection{清泰}

\begin{longtable}{|>{\centering\scriptsize}m{2em}|>{\centering\scriptsize}m{1.3em}|>{\centering}m{8.8em}|}
  % \caption{秦王政}\
  \toprule
  \SimHei \normalsize 年数 & \SimHei \scriptsize 公元 & \SimHei 大事件 \tabularnewline
  % \midrule
  \endfirsthead
  \toprule
  \SimHei \normalsize 年数 & \SimHei \scriptsize 公元 & \SimHei 大事件 \tabularnewline
  \midrule
  \endhead
  \midrule
  元年 & 934 & \tabularnewline\hline
  二年 & 935 & \tabularnewline\hline
  三年 & 936 & \tabularnewline
  \bottomrule
\end{longtable}

\subsubsection{天福}

\begin{longtable}{|>{\centering\scriptsize}m{2em}|>{\centering\scriptsize}m{1.3em}|>{\centering}m{8.8em}|}
  % \caption{秦王政}\
  \toprule
  \SimHei \normalsize 年数 & \SimHei \scriptsize 公元 & \SimHei 大事件 \tabularnewline
  % \midrule
  \endfirsthead
  \toprule
  \SimHei \normalsize 年数 & \SimHei \scriptsize 公元 & \SimHei 大事件 \tabularnewline
  \midrule
  \endhead
  \midrule
  元年 & 936 & \tabularnewline\hline
  二年 & 937 & \tabularnewline\hline
  三年 & 938 & \tabularnewline\hline
  四年 & 939 & \tabularnewline\hline
  五年 & 940 & \tabularnewline\hline
  六年 & 941 & \tabularnewline\hline
  七年 & 942 & \tabularnewline\hline
  八年 & 943 & \tabularnewline\hline
  九年 & 944 & \tabularnewline
  \bottomrule
\end{longtable}

\subsubsection{开运}

\begin{longtable}{|>{\centering\scriptsize}m{2em}|>{\centering\scriptsize}m{1.3em}|>{\centering}m{8.8em}|}
  % \caption{秦王政}\
  \toprule
  \SimHei \normalsize 年数 & \SimHei \scriptsize 公元 & \SimHei 大事件 \tabularnewline
  % \midrule
  \endfirsthead
  \toprule
  \SimHei \normalsize 年数 & \SimHei \scriptsize 公元 & \SimHei 大事件 \tabularnewline
  \midrule
  \endhead
  \midrule
  元年 & 944 & \tabularnewline\hline
  二年 & 945 & \tabularnewline\hline
  三年 & 946 & \tabularnewline
  \bottomrule
\end{longtable}

\subsubsection{天复}

\begin{longtable}{|>{\centering\scriptsize}m{2em}|>{\centering\scriptsize}m{1.3em}|>{\centering}m{8.8em}|}
  % \caption{秦王政}\
  \toprule
  \SimHei \normalsize 年数 & \SimHei \scriptsize 公元 & \SimHei 大事件 \tabularnewline
  % \midrule
  \endfirsthead
  \toprule
  \SimHei \normalsize 年数 & \SimHei \scriptsize 公元 & \SimHei 大事件 \tabularnewline
  \midrule
  \endhead
  \midrule
  元年 & 947 & \tabularnewline
  \bottomrule
\end{longtable}

\subsubsection{乾佑}

\begin{longtable}{|>{\centering\scriptsize}m{2em}|>{\centering\scriptsize}m{1.3em}|>{\centering}m{8.8em}|}
  % \caption{秦王政}\
  \toprule
  \SimHei \normalsize 年数 & \SimHei \scriptsize 公元 & \SimHei 大事件 \tabularnewline
  % \midrule
  \endfirsthead
  \toprule
  \SimHei \normalsize 年数 & \SimHei \scriptsize 公元 & \SimHei 大事件 \tabularnewline
  \midrule
  \endhead
  \midrule
  元年 & 948 & \tabularnewline
  \bottomrule
\end{longtable}



%%% Local Variables:
%%% mode: latex
%%% TeX-engine: xetex
%%% TeX-master: "../../Main"
%%% End:

%% -*- coding: utf-8 -*-
%% Time-stamp: <Chen Wang: 2019-12-26 10:11:49>

\subsection{贞懿王\tiny(948-960)}

\subsubsection{生平}

南平貞懿王高保融(920年-960年),字德長,五代時期荊南君主(南平王)。為高從誨之第三子。

後漢隱帝劉承祐乾祐元年(948年),南平王高從誨去世,高保融繼位。不久,即被後漢任命為荊南節度使、同平章事、兼侍中。後周太祖郭威廣順元年(951年),被封為渤海郡王。顯德元年(954年),再被進封為南平王。

高保融個性遲鈍緩慢,沒有什麼才能,無論事情大小,皆委由其弟高保勗決定。宋太祖趙匡胤建隆元年(960年),宋朝建立後,高保融愈發感到恐懼,因此一年之間三次進貢。同年,因病去世,贈太尉,諡貞懿王。其子高繼沖年紀尚小,因此遺命高保勗繼位。

\subsubsection{乾佑}

\begin{longtable}{|>{\centering\scriptsize}m{2em}|>{\centering\scriptsize}m{1.3em}|>{\centering}m{8.8em}|}
  % \caption{秦王政}\
  \toprule
  \SimHei \normalsize 年数 & \SimHei \scriptsize 公元 & \SimHei 大事件 \tabularnewline
  % \midrule
  \endfirsthead
  \toprule
  \SimHei \normalsize 年数 & \SimHei \scriptsize 公元 & \SimHei 大事件 \tabularnewline
  \midrule
  \endhead
  \midrule
  元年 & 948 & \tabularnewline\hline
  二年 & 949 & \tabularnewline\hline
  三年 & 950 & \tabularnewline
  \bottomrule
\end{longtable}

\subsubsection{广顺}

\begin{longtable}{|>{\centering\scriptsize}m{2em}|>{\centering\scriptsize}m{1.3em}|>{\centering}m{8.8em}|}
  % \caption{秦王政}\
  \toprule
  \SimHei \normalsize 年数 & \SimHei \scriptsize 公元 & \SimHei 大事件 \tabularnewline
  % \midrule
  \endfirsthead
  \toprule
  \SimHei \normalsize 年数 & \SimHei \scriptsize 公元 & \SimHei 大事件 \tabularnewline
  \midrule
  \endhead
  \midrule
  元年 & 951 & \tabularnewline\hline
  二年 & 952 & \tabularnewline\hline
  三年 & 953 & \tabularnewline
  \bottomrule
\end{longtable}

\subsubsection{显德}

\begin{longtable}{|>{\centering\scriptsize}m{2em}|>{\centering\scriptsize}m{1.3em}|>{\centering}m{8.8em}|}
  % \caption{秦王政}\
  \toprule
  \SimHei \normalsize 年数 & \SimHei \scriptsize 公元 & \SimHei 大事件 \tabularnewline
  % \midrule
  \endfirsthead
  \toprule
  \SimHei \normalsize 年数 & \SimHei \scriptsize 公元 & \SimHei 大事件 \tabularnewline
  \midrule
  \endhead
  \midrule
  元年 & 954 & \tabularnewline\hline
  二年 & 955 & \tabularnewline\hline
  三年 & 956 & \tabularnewline\hline
  四年 & 957 & \tabularnewline\hline
  五年 & 958 & \tabularnewline\hline
  六年 & 959 & \tabularnewline\hline
  七年 & 960 & \tabularnewline
  \bottomrule
\end{longtable}



%%% Local Variables:
%%% mode: latex
%%% TeX-engine: xetex
%%% TeX-master: "../../Main"
%%% End:

%% -*- coding: utf-8 -*-
%% Time-stamp: <Chen Wang: 2019-12-26 10:12:04>

\subsection{高保勗\tiny(960-962)}

\subsubsection{生平}

高保勗(924年-962年),字省躬,五代時期荊南君主(荊南節度使)。為高從誨之第十子,高保融之弟。高保勗幼時為高從誨所喜愛,高從誨因事盛怒,見到高保勗必釋然而笑,是故百姓稱之為「萬事休」。

宋太祖建隆元年(960年),高保融因病去世,其子高繼沖年紀尚小,因此遺命高保勗繼位,總判內外軍馬事。不久,即為宋任命為荊南節度使。

高保勗少時多病,體態瘦弱,但頗有治事之才。然而繼位後,放縱荒淫而沒有節制,白天召娼妓至官府,而挑選強壯的士兵,命其隨便調戲淫謔,然後自己再和姬妾一同觀賞做為娛樂。又喜歡營造亭台樓閣,花費人力物力無數,而不理國政,人民都很不滿。

建隆三年(962年),高保勗因病去世,被宋朝贈侍中。遺命其姪即高保融之子高繼沖嗣位。

高保勗去世後數月,南平即為宋所滅,有附會者即以其綽號「萬事休」為預兆。

\subsubsection{建隆}

\begin{longtable}{|>{\centering\scriptsize}m{2em}|>{\centering\scriptsize}m{1.3em}|>{\centering}m{8.8em}|}
  % \caption{秦王政}\
  \toprule
  \SimHei \normalsize 年数 & \SimHei \scriptsize 公元 & \SimHei 大事件 \tabularnewline
  % \midrule
  \endfirsthead
  \toprule
  \SimHei \normalsize 年数 & \SimHei \scriptsize 公元 & \SimHei 大事件 \tabularnewline
  \midrule
  \endhead
  \midrule
  元年 & 960 & \tabularnewline\hline
  二年 & 961 & \tabularnewline\hline
  三年 & 962 & \tabularnewline
  \bottomrule
\end{longtable}




%%% Local Variables:
%%% mode: latex
%%% TeX-engine: xetex
%%% TeX-master: "../../Main"
%%% End:

%% -*- coding: utf-8 -*-
%% Time-stamp: <Chen Wang: 2019-12-26 10:12:19>

\subsection{高继冲\tiny(962-963)}

\subsubsection{生平}

高繼沖(943年-973年),字成和(一作字贊平),五代十國荊南政權末期君主(荊南節度使)。為高保融之長子,高保勗之姪。

宋太祖趙匡胤建隆三年(962年),高保勗因病去世,遺命高繼沖權判內外軍馬事以繼其位。後來高繼沖亦被宋任命為荊南節度使。

同年,湖南的武平節度使周行逢亦去世,年僅11歲的周保權嗣位,而境內大將張文表叛變,周保權向宋朝求援。建隆四年(963年),宋軍討伐張文表,假道荊南,發生荊湖之戰,趁機控制南平都城江陵(今湖北江陵)城巷,高繼沖只得納地以歸,南平亡。

南平亡後,宋一度仍任命高繼沖為荊南節度使。不久,高繼沖舉族歸朝,被改命為武寧節度使(約在今江蘇、安徽一帶)。

開寶六年(973年)高繼沖於武寧節度使任內去世,贈侍中。高繼沖鎮守彭門(今江蘇徐州),政事委諸僚佐,然有德政,因此被百姓請求留葬當地,但不被宋太祖允許。

\subsubsection{建隆}

\begin{longtable}{|>{\centering\scriptsize}m{2em}|>{\centering\scriptsize}m{1.3em}|>{\centering}m{8.8em}|}
  % \caption{秦王政}\
  \toprule
  \SimHei \normalsize 年数 & \SimHei \scriptsize 公元 & \SimHei 大事件 \tabularnewline
  % \midrule
  \endfirsthead
  \toprule
  \SimHei \normalsize 年数 & \SimHei \scriptsize 公元 & \SimHei 大事件 \tabularnewline
  \midrule
  \endhead
  \midrule
  元年 & 962 & \tabularnewline\hline
  二年 & 963 & \tabularnewline
  \bottomrule
\end{longtable}




%%% Local Variables:
%%% mode: latex
%%% TeX-engine: xetex
%%% TeX-master: "../../Main"
%%% End:



%%% Local Variables:
%%% mode: latex
%%% TeX-engine: xetex
%%% TeX-master: "../../Main"
%%% End:
