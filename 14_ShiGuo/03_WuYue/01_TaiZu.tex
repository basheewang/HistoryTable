%% -*- coding: utf-8 -*-
%% Time-stamp: <Chen Wang: 2019-12-25 10:50:52>

\subsection{太祖\tiny(907-932)}

\subsubsection{生平}

吴越太祖钱\xpinyin*{镠}(852年3月10日-932年5月6日),字具美(一作巨美),浙江杭州临安(今临安区)人。五代十国时期吴越國開國國王。

唐末跟从石镜镇将军董昌镇压農民反抗軍,任镇海节度使,乾宁年间击败董昌,占有两浙十三州,后梁开平初年被封为吴越王。在位期间,曾征用民工,修建钱塘江海塘,又在太湖流域,普造堰闸,以时蓄洪,不畏旱涝,并建立水网圩区的维修制度,有利于这一地区的农业经济。

由於吳越國小力弱,又同鄰近的吳、閩政權不和,投靠中原王朝,不斷遣使進貢以求庇護。先臣服後梁,又臣服後唐。后唐明宗時因惹怒樞密使安重誨,被削去官職,安重誨死後又恢復。長興三年(932年)病死,葬安國縣(现临安区)衣錦鄉茅山。庙号太祖,諡號武肃王。

唐朝大中六年二月十六日,钱镠生于临安县石镜乡大官山下的临水里钱坞垅。父亲钱宽,母亲水丘氏。一家以农耕打渔为生。传说钱镠出生时突现红光,且相貌奇丑,父亲本欲弃之,但因其祖母怜惜,最后得以保全性命,因此钱鏐小名“婆留”(“阿婆留其命”之义)。

钱鏐自幼不喜诗文,偏好习武,常与邻里诸小儿戏于里中大木之下,指挥群儿为队伍,号令颇有法(钱鏐即位后将此树封为“将军木”。钱鏐在16岁的时候就弃学贩盐。当时私贩盐料是官府严厉禁止的,但由于利润极高,因此钱鏐铤而走险,在杭州、越州(今绍兴)、宣州等地贩卖私盐和粮食。这段贩卖私盐的经历,练就了钱鏐体魄和胆略,也为他日后发展提供了充足的经济基础。

17岁开始,钱鏐苦练硬弓长矛,并读些《孙子兵法》,史书称其“善射与槊,稍通图纬诸书”。到21岁时,他在石镜镇充当“义兵”,并将小名“钱婆留”改为大名“钱鏐”(其为金字辈,并取“留”字音,故改“鏐”)。由于钱镠武艺高强,受到石镜镇指挥使董昌重用,经过平定王郢、朱直管、曹师雄、王知新等叛乱之后,逐渐提拔为偏将、副指挥使、兵马使、镇海军右副使等职。

879年(唐僖宗乾符六年)七月,黄巢起义军进犯临安。钱鏐以少敌多,巧妙运用伏击和虚张声势等战术,阻吓了黄巢军的进攻。880年,唐朝内乱四起,为保护地方安定,董昌、钱鏐联合各县民团,建立“八都军”(临安县“石镜都”、余杭县“清平都”、於潜县“於潜都”、盐官县“盐官都”、新城县“武安都”、唐山县“唐山都”、富阳县“富春都”和龙泉县“龙泉都”),次年,钱鏐授“都知兵马使”,并注意团结各都力量和下层头目,还将其弟钱銶、钱镒、钱铧、钱镖,以及儿子钱元璙、钱元瓘等人安插到部队中担任将领,从而将八都军逐渐培养成坚强的嫡系部队。

唐末、五代时期所称“两浙十四州”,包括现在浙江全境和江苏长江以南部分地区。七五八年,江南东道下属的浙江东道 和浙江西道 共有十四州,其中除去润州和常州,再加上福建的福州和临安县的安国衣锦军,共为一军十三州,号称“十四州”,便是钱鏐创立的吴越国的大致范围。

自讨伐王郢起,钱鏐身经百战,先后与刘汉宏、董昌等地方主要军阀作战,最终平定了两浙范围内的敌对势力,建立了巩固的地方割据政权。

882年7月起,占据浙东的义胜军节度使刘汉宏发兵西进,欲并吞浙西。董昌、钱鏐率八都军在钱塘江边御敌。由于出奇制胜,加上利用江上夜雾遮掩,钱鏐突袭敌营,获得首胜。之后,又在江干、富阳、诸暨、萧山西陵等地屡败刘军。最后,刘汉宏亲自督战,率十万大军与钱鏐在萧山西陵一带决战,结果被钱镠击溃,刘汉宏本人易装成屠户逃跑。这一次西陵大捷,是钱鏐取得的第一次重大战果,据说,从此钱鏐将西陵改名为西兴至今(现钱江三桥又名“西兴大桥”)。

此后,刘汉宏仍不断骚扰浙西,导致董昌和钱鏐决心彻底平定浙东之患。886年10起,钱鏐仅用了2个月左右的时间,就率军攻克越州,并将潜逃被捕的刘汉宏斩于会稽街市。此后,钱鏐为杭州刺史,董昌升任浙东观察使、检校太尉、陇西郡王等职。

董昌其人昏庸残暴,野心日增,随后就即位称帝,国号大越罗平,改元顺天。895年2月,唐朝封钱鏐为浙东招讨使,令其讨伐董昌。但钱鏐起初感念董昌提携之恩,犹豫不决,但董昌却联合淮南杨行密偷袭苏州、杭州,最终使得钱鏐下定决心,攻克越州。董昌在被押付杭州途中,心存惭愧,投江自杀。从此,钱鏐基本控制两浙,并于896年10月,被授为镇海、镇东军节度使,加检校太尉,兼中书令。

897年8月,鉴于钱鏐招讨董昌有功,唐昭宗特赐金书铁券于他,免其本人九死或子孙三死。这件钱镠铁券后经宋代陆游、明代刘基等人为其写跋,还呈宋太宗、宋仁宗、宋神宗、明太祖、明成祖和清高宗等七位帝王御览。900年,为了表彰钱王的功绩,唐王朝派人取钱鏐画像,悬于凌烟阁。

钱鏐在平定了两浙内部的敌对势力后,基本停止了大规模的征讨。但由于三面受敌,仍经历了多次边境保卫战,有时还将战斗延伸至江西的信州(今上饶)和虔州(今赣州)等地。其主要对手就是淮南军阀杨行密和内部的“徐许之乱”。

钱鏐和杨行密的关系时而友好,时而敌对,体现出五代十国乱世的特点。双方的冲突共持续了三十年,其间钱曾出兵援助杨擒斩孙儒、安仁义等叛逆,并正式通婚,但也因董昌之战等发生过激烈的战斗。最后通过两次衣锦军保卫战和一次浪山江水战,才结束了双方的敌对状态。从此两浙地区进入休养生息的安定建设阶段。

902年,钱鏐刚被封为越王不久,其部下的徐绾和许再思起兵叛变,使钱鏐大伤元气。最后钱鏐支付了二十万缗犒军钱,并派两个儿子作为人质,才使得叛军撤兵。这次内乱后,钱鏐吸取了教训,治国更为谨慎。

904年被封为吴王;907年,后梁封钱鏐为吴越王,吴越国自此创建。龙德三年(923年),钱镠被册封为吴越国王,吴越建立王国体制。他改府署为朝廷,设置丞相、侍郎等百官,一切礼制皆按照君主的规格。

结束了与周边敌对势力的战争后,钱鏐开始转向对内的大规模经济和文化建设。唐大顺元年(890年)钱鏐开始着手建设杭州城。先后建造了夹城、罗城和子城。杭州罗城筑于唐景福元年(892年)七月,筑城时发动余杭、盐官、新城、唐山、富阳、龙泉“八都兵”,及紫溪、保城、龙通、三泉、三镇,合计“十三都兵”二十余万人。城区范围广袤七十里,四至分别是:南到六和塔;东至侯潮门和艮山门一线;北达武林门;西临涌金门和清波门一带,设朝天门、龙山门、竹车门、南土门、北土门、盐桥门、西关门(涵水门)、北关门、宝德门共十门。天宝三年(910年)又扩杭州城,凤凰山柳浦隋唐所筑子城被改造为府城,南为通越门,北为只门,子城内大修台馆,有天册堂(即王位之所)、天宠堂(即位、理政之所)、思政堂、功臣堂(寝宫)、握发殿、咸宁院、义和院、碧波亭、虚白堂、八会亭、都会堂、蓬莱阁、直仪门(设厅)、青史楼、天长楼、玉华楼、瑞萼园等建筑。钱鏐筑杭州城,在客观上为杭州成为日后南宋的都城打下了基础,南宋临安宫城即原吴越王宫。

钱王还在城内开凿水井(据说杭州的百井坊巷原有99眼,就开凿于此时),建设钱塘江堤,为杭州的饮水淡化问题做出了很大贡献。此外,钱鏐及其继承者崇信佛教,前后修建了不少寺院佛塔,使杭州在当时就有“佛国”之称。其中著名的灵隐寺、净慈寺、昭庆寺等寺院,以及雷峰塔、六和塔、保俶塔、闸口白塔和临安功臣塔等都是在吴越国时期兴建或扩建的。

钱鏐在内政建设上的主要成就体现在修筑海塘和疏浚内湖上。910年起,钱鏐上书后梁朝廷,指出“目击平原沃野,尽成江水汪洋,虽值干戈扰攘之后,即兴筑塘修堤之举。”,并开始着手修筑钱塘江沿岸石塘。由于钱江潮汛,工程进展困难,后钱鏐以竹器填以巨石,才奠定了基础。当时修筑的石塘,从六和塔一直到艮山门,长33万8593丈。此外,钱鏐还重点抓了疏浚西湖、太湖和鉴湖等工作。当时他设置了7000名撩湖兵,专门从事西湖的开浚工作 后代的苏轼也是在参考了钱鏐治湖的经验上,才开始大规模疏浚西湖。

然而据欧阳修《五代史》吴越世家所称,吳越自錢鏐時起,賦稅繁苛,小至雞、魚、雞卵、雞雛,也要納稅。貧民欠稅被捉到官府,按各稅欠數多少定笞數,往往積至笞數十以至百餘(一說五百余),民尤不勝其苦。於杭州建造「地上天宮」,耗盡民財民力。

钱鏐做節度使時,有人獻詩,詩中有「一條江水檻前流」句,「前流」與「钱鏐」是諧音,钱鏐認為獻詩人諷刺自己,於是暗殺此人。羅隱聲名大,曾作詩譏笑钱鏐出身寒家,錢鏐卻欣然不怒。錢鏐留心收買名士,皮日休(當是黃巢失敗後,逃來依靠錢鏐)、羅隱、胡嶽等都得到優待,自己也學吟詠,與名士唱和。天宝三年十月钱鏐巡视故乡衣锦军,置酒宴请父老,赏八十岁以上者金樽,百岁以上者玉樽,又作《还乡歌》:“三节还乡兮挂锦衣,碧天朗朗兮爱日晖。功臣道上兮列旌旗,父老远来兮相追随。家山乡眷兮会时稀,今朝设宴兮觥散飞。斗牛无孛兮民无欺,吴越一王兮驷马归”。父老不解其意,钱鏐复用吴语为歌:“你辈见侬底欢喜,则是一般滋味子,长在我侬心底里”,举座叫笑振席。

由于钱鏐在其晚年坚持保境安民政策,不参与军阀混战,而且对内统治相对廉洁清明,使得这一时期杭州的发展超越了中原地区的许多大城市,成为东南地区的经济中心。

后唐长兴三年(932年)三月己酉,錢鏐薨于临安王府正寝,年八十一岁,在位四十一年。葬安国县衣锦乡茅山,建庙于东府。后唐赐谥号武肃,吴越国上庙号太祖。

錢鏐累事三朝,唐、后梁、后唐屡加封号,累赐启圣匡运同德功臣、定乱安国启圣昌运同德守道戴功臣、淮南镇海镇东等军节度使、淮南浙江东西等道管内观察处置、充淮南四面都统营田安抚、兼两浙盐铁制置发运等使、天下兵马都元帅、开府仪同三司、尚父、检校太师、尚书令、兼中书令、上柱国、吴越国王,赐剑履上殿、诏书不名,食邑一万五千户。

欧阳修《五代史》称吴越“有改元而无称帝之事”。吴越国从908年(后梁开平二年)至913年(后梁乾化三年),曾用天宝年号;924年(后唐同光二年)至931年(后唐长兴二年)用宝大、宝正年号,皆仅行于吴越国中。

后世一般对钱氏评价较高,认为他促进了地方经济发展,保障了民众安居乐业的局面。主要有:“时维五纪乱何如?史册闲观亦皱眉。是地却逢钱节度,民间无事看花嬉!”——北宋·赵抃

“钱立国,置营田数千人于松江,辟土而耕,…民老死无他缠累,且完国归朝,不杀一人,则其功德大矣!”—— 明·朱国桢

史书载钱鏐性俭朴,衣衾杂用细布,常膳用瓷器、漆器。除夕子夜与子孙宴于府城内,未鼓数曲而令罢宴,称“闻者以我为长夜之歌”。其寝居之殿名为“握发殿”,取周公“一沐三握发”典故。

欧阳修在《新五代史·吴越世家》中谴责钱氏严刑酷法。而宋代别史《丹铅录》称,欧阳修为推官时,昵一妓,比而为忠懿王之子钱惟演得去,欧阳修深衔之,后作《五代史》时乃诬以钱氏诸王“重敛虐民”之语,以公报私。钱世昭撰《钱氏私志》也稱歐陽修是挾怨報复。

目前在西湖南岸,建有钱王祠,供后人瞻仰钱王业绩。

\subsubsection{天祐}

\begin{longtable}{|>{\centering\scriptsize}m{2em}|>{\centering\scriptsize}m{1.3em}|>{\centering}m{8.8em}|}
  % \caption{秦王政}\
  \toprule
  \SimHei \normalsize 年数 & \SimHei \scriptsize 公元 & \SimHei 大事件 \tabularnewline
  % \midrule
  \endfirsthead
  \toprule
  \SimHei \normalsize 年数 & \SimHei \scriptsize 公元 & \SimHei 大事件 \tabularnewline
  \midrule
  \endhead
  \midrule
  元年 & 907 & \tabularnewline
  \bottomrule
\end{longtable}

\subsubsection{天宝}

\begin{longtable}{|>{\centering\scriptsize}m{2em}|>{\centering\scriptsize}m{1.3em}|>{\centering}m{8.8em}|}
  % \caption{秦王政}\
  \toprule
  \SimHei \normalsize 年数 & \SimHei \scriptsize 公元 & \SimHei 大事件 \tabularnewline
  % \midrule
  \endfirsthead
  \toprule
  \SimHei \normalsize 年数 & \SimHei \scriptsize 公元 & \SimHei 大事件 \tabularnewline
  \midrule
  \endhead
  \midrule
  元年 & 908 & \tabularnewline\hline
  二年 & 909 & \tabularnewline\hline
  三年 & 910 & \tabularnewline\hline
  四年 & 911 & \tabularnewline\hline
  五年 & 912 & \tabularnewline
  \bottomrule
\end{longtable}

\subsubsection{凤历}

\begin{longtable}{|>{\centering\scriptsize}m{2em}|>{\centering\scriptsize}m{1.3em}|>{\centering}m{8.8em}|}
  % \caption{秦王政}\
  \toprule
  \SimHei \normalsize 年数 & \SimHei \scriptsize 公元 & \SimHei 大事件 \tabularnewline
  % \midrule
  \endfirsthead
  \toprule
  \SimHei \normalsize 年数 & \SimHei \scriptsize 公元 & \SimHei 大事件 \tabularnewline
  \midrule
  \endhead
  \midrule
  元年 & 913 & \tabularnewline
  \bottomrule
\end{longtable}

\subsubsection{乾化}

\begin{longtable}{|>{\centering\scriptsize}m{2em}|>{\centering\scriptsize}m{1.3em}|>{\centering}m{8.8em}|}
  % \caption{秦王政}\
  \toprule
  \SimHei \normalsize 年数 & \SimHei \scriptsize 公元 & \SimHei 大事件 \tabularnewline
  % \midrule
  \endfirsthead
  \toprule
  \SimHei \normalsize 年数 & \SimHei \scriptsize 公元 & \SimHei 大事件 \tabularnewline
  \midrule
  \endhead
  \midrule
  元年 & 913 & \tabularnewline\hline
  二年 & 914 & \tabularnewline\hline
  三年 & 915 & \tabularnewline
  \bottomrule
\end{longtable}

\subsubsection{贞明}

\begin{longtable}{|>{\centering\scriptsize}m{2em}|>{\centering\scriptsize}m{1.3em}|>{\centering}m{8.8em}|}
  % \caption{秦王政}\
  \toprule
  \SimHei \normalsize 年数 & \SimHei \scriptsize 公元 & \SimHei 大事件 \tabularnewline
  % \midrule
  \endfirsthead
  \toprule
  \SimHei \normalsize 年数 & \SimHei \scriptsize 公元 & \SimHei 大事件 \tabularnewline
  \midrule
  \endhead
  \midrule
  元年 & 915 & \tabularnewline\hline
  二年 & 916 & \tabularnewline\hline
  三年 & 917 & \tabularnewline\hline
  四年 & 918 & \tabularnewline\hline
  五年 & 919 & \tabularnewline\hline
  六年 & 920 & \tabularnewline\hline
  七年 & 921 & \tabularnewline
  \bottomrule
\end{longtable}

\subsubsection{龙德}

\begin{longtable}{|>{\centering\scriptsize}m{2em}|>{\centering\scriptsize}m{1.3em}|>{\centering}m{8.8em}|}
  % \caption{秦王政}\
  \toprule
  \SimHei \normalsize 年数 & \SimHei \scriptsize 公元 & \SimHei 大事件 \tabularnewline
  % \midrule
  \endfirsthead
  \toprule
  \SimHei \normalsize 年数 & \SimHei \scriptsize 公元 & \SimHei 大事件 \tabularnewline
  \midrule
  \endhead
  \midrule
  元年 & 921 & \tabularnewline\hline
  二年 & 922 & \tabularnewline\hline
  三年 & 923 & \tabularnewline
  \bottomrule
\end{longtable}

\subsubsection{宝大}

\begin{longtable}{|>{\centering\scriptsize}m{2em}|>{\centering\scriptsize}m{1.3em}|>{\centering}m{8.8em}|}
  % \caption{秦王政}\
  \toprule
  \SimHei \normalsize 年数 & \SimHei \scriptsize 公元 & \SimHei 大事件 \tabularnewline
  % \midrule
  \endfirsthead
  \toprule
  \SimHei \normalsize 年数 & \SimHei \scriptsize 公元 & \SimHei 大事件 \tabularnewline
  \midrule
  \endhead
  \midrule
  元年 & 924 & \tabularnewline\hline
  二年 & 925 & \tabularnewline
  \bottomrule
\end{longtable}

\subsubsection{宝正}

\begin{longtable}{|>{\centering\scriptsize}m{2em}|>{\centering\scriptsize}m{1.3em}|>{\centering}m{8.8em}|}
  % \caption{秦王政}\
  \toprule
  \SimHei \normalsize 年数 & \SimHei \scriptsize 公元 & \SimHei 大事件 \tabularnewline
  % \midrule
  \endfirsthead
  \toprule
  \SimHei \normalsize 年数 & \SimHei \scriptsize 公元 & \SimHei 大事件 \tabularnewline
  \midrule
  \endhead
  \midrule
  元年 & 926 & \tabularnewline\hline
  二年 & 927 & \tabularnewline\hline
  三年 & 928 & \tabularnewline\hline
  四年 & 929 & \tabularnewline\hline
  五年 & 930 & \tabularnewline\hline
  六年 & 931 & \tabularnewline
  \bottomrule
\end{longtable}




%%% Local Variables:
%%% mode: latex
%%% TeX-engine: xetex
%%% TeX-master: "../../Main"
%%% End:
