%% -*- coding: utf-8 -*-
%% Time-stamp: <Chen Wang: 2019-12-25 10:28:42>


\section{吴越\tiny(907-978)}

\subsection{简介}

吳越(907-978)是五代十國時期的十國之一,由錢鏐在公元907年所建。都城為錢塘(杭州)。強盛時擁有十三州疆域,約為現今浙江全省、江蘇東南部和福建東北部。吳越國共有五位君主,傳國七十一年,末主錢弘俶於公元978年獻土入宋。

吳越國前身與基礎可一直上溯至唐末大混亂時期杭州地方的鄉兵集團杭州八都。

錢鏐本為石鏡都的副將,助主將董昌取得杭州、擊敗浙東觀察使劉漢宏的侵略後,董昌將杭州刺史一職以及杭州八都集團的大部分讓給了錢鏐,是錢鏐獲得獨立地盤之始。以下表列出此政權之擴張事件及領域變動。

893年,錢鏐為唐鎮海節度使。 907年被後梁封為吳越王。

975年援北宋滅南唐,978年吳越末代國王錢俶為了避免戰亂主動献土并入北宋。

886年,浙西鎮海軍兵變,錢鏐以平亂為名出兵攻陷常州及潤州。

887年,消滅佔據蘇州的徐約。

891年,孫儒亂江南,與楊行密、錢鏐爭奪常、潤、蘇三州。潤、常最終為楊行密所據。

896年,為唐朝討伐稱帝的越州董昌,平之;過程中楊行密援董昌,攻陷蘇州。

896年,湖州刺史李師悅病逝,部下都將聯合趕走其子李繼徽、歸附錢鏐。

898年,克復蘇州。分蘇州嘉興為秀州。陸續與浙東諸州勢力交戰,或在該州土豪病逝後收服之。

945年,出兵援助閩國抵抗南唐,閩國部將李仁達以福州歸附。

吴越国采取保境安民的政策,经济繁荣,渔盐桑蚕之利甲于江南;文士荟萃,人才济济,文艺也著称于世。由于吴国阻隔陆路,因此吴越朝贡中原王朝多经登、莱海路,海上交通发达,与後百濟、新罗、日本的海上贸易和文化交流频繁。

吳越國的水利在十國中是最著名的。錢鏐設撩湖軍,開浚錢塘湖,得其遊覽、灌溉兩利,又引湖水為湧金池,與運河相通。此外,在唐末時期,錢塘江口地區因海潮襲擊,“自秦望山東南十八堡,數千萬畝田地悉成江面,民不堪命”。後梁開平四年/吳越天寶三年(910年),錢鏐動員大批勞力,修築“捍海石塘”。用木樁把裝滿石塊的巨大石籠固定在江邊,形成堅固的海堤,保護了江邊農田不再受潮水侵蝕。並且由於石塘具有蓄水作用,使得江邊農田得獲灌溉之利。由是“錢塘富庶盛於東南”。

錢鏐還在太湖地區設“撩水軍”四部、七八千人,專門負責浚湖、築堤、疏濬河浦,使得蘇州、嘉興、長洲等地得享灌溉之利。此外錢氏還修建武義縣的長安堰,受益農田上萬頃;東府的鑑湖,餘杭縣的上湖、下湖、北湖,諸暨的完浦,慈溪的慈濟湖,明州的南湖,鄞縣的廣德湖、東錢湖、它山堰,也都是重要的灌溉水源。吳越境內田塘眾多,土地膏腴,有“近澤知田美”之語。

錢鏐在位時,即鼓勵擴大墾田,下令“荒田任開,不起稅額”。由是“境內無棄田”,歲熟豐稔,民間五十錢可糴白米一石。兩浙又為著名桑麻產地,湖州顧渚山出產著名的“紫筍茶”,天福七年(942年)忠獻王錢弘佐一次就向後晉進貢二萬五千斤之多。

吳越國的手工業高度發達,官府生產的各色繡金錦緞綾絹不僅供王宮之需,還大量進貢中原王朝。吳越國的陶瓷業也相當興盛,主要的陶瓷器生產場地是越州餘姚上林湖的越州窯,此外還在處州龍泉、上虞窯前寺等地設立官窯。吳越生產的“秘色瓷”昔日為錢氏內用,大臣非有功不得賜,故名。其工藝細膩,胎骨均勻,底部光潔,為吳越進貢及海外貿易的主要物資之一。溫州出產的蠲紙,潔白堅滑,專門供應官府。

佛教是吴越国文化的重要组成部分,历代吴越国王均笃信佛教,吴越境内佛寺林立。忠懿王时期,境内佛塔达到8万4千座,有名的如今天杭州的保俶塔、雷峰塔。受吴越国的佛教氛围影响,杭州的雕版印刷业异常发达,仅忠懿王钱俶(钱弘俶)时期印刷的经书、佛像就达六十六万两千卷之多。宋代杭州印刷业居全国第一,即缘于此故。

吴越与南汉、新罗、后百济、高丽、日本、琉球等国通商于海上。吴越向海外出口瓷器、锦缎、绫绢,进口苏木、乳香、沉香、龙脑、玳瑁、珍珠、日本椤木、铜器、扇子等货物,甚至包括大食的猛火油。吴越与中原内地的贸易最初通过楚国和荆南,后吴国占领江西全境,吴越的北方贸易改走海路,在登州、莱州、青州等地登陆,经陆路至汴梁、洛阳。杭州与明州是最重要的两座港口,都城杭州“舟楫辐辏,望之不见首尾”。

吳越國在錢氏家族治理下,政治上比較安定,對外謹事中原王朝,奉正朔,歲時進貢;內無楊吳、馬楚、南漢的兄弟相殘之禍,保境安民,社會繁榮,經濟富裕。

司馬光《資治通鑑》載,吳越忠獻王錢弘佐年十四即位,問倉吏“今蓄積幾何?”答曰“十年”,錢弘佐曰“軍食足矣,可以寬吾民”,於是命境內免稅三年。明朝朱國楨則評價吳越政治稱“錢立國,置營田數千人於松江,闢土而耕,…民老死無他纏累,且完國歸朝,不殺一人,則其功德大矣!”

然而北宋歐陽修《五代史·吳越世家》評價吳越國則稱,自錢鏐時起,賦稅繁苛,小至雞、魚、雞卵、雞雛,也要納稅。貧民欠稅被捉到官府,按各稅欠數多少定笞數,往往積至笞數十以至百餘(一說五百餘),民尤不勝其苦。而宋代即有論者稱歐陽修此舉為挾私怨於褒貶之間。

吳越曾向後唐進貢萬壽節金器、連花金食器、盤龍鳳錦織成紅羅袍、金排方盤龍帶禦衣、通犀瑞象腰帶、紅地龍鳳錦被、錦綺、越綾、吳綾、異紋綾、金條紗、絹布、綿布、秘色瓷器、銀妝花櫚木櫥子、龍鳳紗紋櫥、紅藤龍鳳箱、佛頭螺子青、山螺子青、菩薩石蟹子、白龍腦、大茶、腦源茶等物。向後晉進貢銀、絹、絲、細甲、弓弩、箭、扇子、靴履、細酒、細紙、蘇木、乾薑、秘色瓷、乳香、啟聖節金大排方坐龍腰帶、禦衣等物。向後漢進貢禦衣、通犀帶、戲龍金帶、金器、銀器、兵仗、綾絹、茶、香、藥物、鞍履、海味。向後周進貢禦衣、銀、綾、絹、龍舟、天祿舟。

宋朝開國後,吳越國日益向其進貢無算,宋太祖曾說:「這些我遲早都要拿的,哪需要獻來呢?」吳越常向宋朝進貢赭黃犀角、龍鳳龜魚、仙人鰲山寶樹等物。進貢通犀寶帶七十餘條,玉帶二十條,紫金獅子帶一條,塗金銀香龍一座,及珊瑚樹十棵、高三尺五寸,皆稀世之寶。又向宋進貢金飾玳瑁器皿一千五百餘件,水晶瑪瑙玉器四千餘件,金銀器及秘色瓷十四萬餘件,金銀飾龍鳳舟二百餘艘,銀飾器物七十萬件,金九萬五千餘兩,銀一百一十萬餘兩,繡金盤龍鳳錦緞衣料數万匹,白龍腦二百餘斤。至於珍珠、玳瑁、琥珀、乳香、沉香、龍涎香、蘇木、貢茶、御酒、綾絹、海味等物無數。而宋朝舉朝文武及宮中內官亦多有饋贈。以十三州之力,供大國歡心,吳越國力以是漸貧。

吳越最大的敵國是其西北的楊吳政權,吳越與其攻伐多年,919年無錫​​之戰後應吳國齊王徐溫要求,兩國修和。但楊吳的繼承者南唐也是吳越發展的主要競爭對手。為保護吳越國,歷代國王尊奉後梁、後唐、後晉、後漢、後周、宋朝六個中原王朝,事大主義,抵抗兩淮,保境安民。在宋滅南唐的戰役中,吳越出兵援宋。此外吳越國亦曾和南唐共同攻打閩國,佔領福州。

935年,吳越與日本的國交始開。次年,藤原忠平向吳越王送抵國書,建立兩國良好關係。940年藤原仲平、947年藤原實賴、953年藤原師輔相繼向吳越王遞交國書。957年吳越王回送黃金。

%% -*- coding: utf-8 -*-
%% Time-stamp: <Chen Wang: 2021-11-01 15:40:07>

\subsection{太祖钱镠\tiny(907-932)}

\subsubsection{生平}

吴越太祖钱\xpinyin*{镠}(852年3月10日-932年5月6日),字具美(一作巨美),浙江杭州临安(今临安区)人。五代十国时期吴越國開國國王。

唐末跟从石镜镇将军董昌镇压農民反抗軍,任镇海节度使,乾宁年间击败董昌,占有两浙十三州,后梁开平初年被封为吴越王。在位期间,曾征用民工,修建钱塘江海塘,又在太湖流域,普造堰闸,以时蓄洪,不畏旱涝,并建立水网圩区的维修制度,有利于这一地区的农业经济。

由於吳越國小力弱,又同鄰近的吳、閩政權不和,投靠中原王朝,不斷遣使進貢以求庇護。先臣服後梁,又臣服後唐。后唐明宗時因惹怒樞密使安重誨,被削去官職,安重誨死後又恢復。長興三年(932年)病死,葬安國縣(现临安区)衣錦鄉茅山。庙号太祖,諡號武肃王。

唐朝大中六年二月十六日,钱镠生于临安县石镜乡大官山下的临水里钱坞垅。父亲钱宽,母亲水丘氏。一家以农耕打渔为生。传说钱镠出生时突现红光,且相貌奇丑,父亲本欲弃之,但因其祖母怜惜,最后得以保全性命,因此钱鏐小名“婆留”(“阿婆留其命”之义)。

钱鏐自幼不喜诗文,偏好习武,常与邻里诸小儿戏于里中大木之下,指挥群儿为队伍,号令颇有法(钱鏐即位后将此树封为“将军木”。钱鏐在16岁的时候就弃学贩盐。当时私贩盐料是官府严厉禁止的,但由于利润极高,因此钱鏐铤而走险,在杭州、越州(今绍兴)、宣州等地贩卖私盐和粮食。这段贩卖私盐的经历,练就了钱鏐体魄和胆略,也为他日后发展提供了充足的经济基础。

17岁开始,钱鏐苦练硬弓长矛,并读些《孙子兵法》,史书称其“善射与槊,稍通图纬诸书”。到21岁时,他在石镜镇充当“义兵”,并将小名“钱婆留”改为大名“钱鏐”(其为金字辈,并取“留”字音,故改“鏐”)。由于钱镠武艺高强,受到石镜镇指挥使董昌重用,经过平定王郢、朱直管、曹师雄、王知新等叛乱之后,逐渐提拔为偏将、副指挥使、兵马使、镇海军右副使等职。

879年(唐僖宗乾符六年)七月,黄巢起义军进犯临安。钱鏐以少敌多,巧妙运用伏击和虚张声势等战术,阻吓了黄巢军的进攻。880年,唐朝内乱四起,为保护地方安定,董昌、钱鏐联合各县民团,建立“八都军”(临安县“石镜都”、余杭县“清平都”、於潜县“於潜都”、盐官县“盐官都”、新城县“武安都”、唐山县“唐山都”、富阳县“富春都”和龙泉县“龙泉都”),次年,钱鏐授“都知兵马使”,并注意团结各都力量和下层头目,还将其弟钱銶、钱镒、钱铧、钱镖,以及儿子钱元璙、钱元瓘等人安插到部队中担任将领,从而将八都军逐渐培养成坚强的嫡系部队。

唐末、五代时期所称“两浙十四州”,包括现在浙江全境和江苏长江以南部分地区。七五八年,江南东道下属的浙江东道 和浙江西道 共有十四州,其中除去润州和常州,再加上福建的福州和临安县的安国衣锦军,共为一军十三州,号称“十四州”,便是钱鏐创立的吴越国的大致范围。

自讨伐王郢起,钱鏐身经百战,先后与刘汉宏、董昌等地方主要军阀作战,最终平定了两浙范围内的敌对势力,建立了巩固的地方割据政权。

882年7月起,占据浙东的义胜军节度使刘汉宏发兵西进,欲并吞浙西。董昌、钱鏐率八都军在钱塘江边御敌。由于出奇制胜,加上利用江上夜雾遮掩,钱鏐突袭敌营,获得首胜。之后,又在江干、富阳、诸暨、萧山西陵等地屡败刘军。最后,刘汉宏亲自督战,率十万大军与钱鏐在萧山西陵一带决战,结果被钱镠击溃,刘汉宏本人易装成屠户逃跑。这一次西陵大捷,是钱鏐取得的第一次重大战果,据说,从此钱鏐将西陵改名为西兴至今(现钱江三桥又名“西兴大桥”)。

此后,刘汉宏仍不断骚扰浙西,导致董昌和钱鏐决心彻底平定浙东之患。886年10起,钱鏐仅用了2个月左右的时间,就率军攻克越州,并将潜逃被捕的刘汉宏斩于会稽街市。此后,钱鏐为杭州刺史,董昌升任浙东观察使、检校太尉、陇西郡王等职。

董昌其人昏庸残暴,野心日增,随后就即位称帝,国号大越罗平,改元顺天。895年2月,唐朝封钱鏐为浙东招讨使,令其讨伐董昌。但钱鏐起初感念董昌提携之恩,犹豫不决,但董昌却联合淮南杨行密偷袭苏州、杭州,最终使得钱鏐下定决心,攻克越州。董昌在被押付杭州途中,心存惭愧,投江自杀。从此,钱鏐基本控制两浙,并于896年10月,被授为镇海、镇东军节度使,加检校太尉,兼中书令。

897年8月,鉴于钱鏐招讨董昌有功,唐昭宗特赐金书铁券于他,免其本人九死或子孙三死。这件钱镠铁券后经宋代陆游、明代刘基等人为其写跋,还呈宋太宗、宋仁宗、宋神宗、明太祖、明成祖和清高宗等七位帝王御览。900年,为了表彰钱王的功绩,唐王朝派人取钱鏐画像,悬于凌烟阁。

钱鏐在平定了两浙内部的敌对势力后,基本停止了大规模的征讨。但由于三面受敌,仍经历了多次边境保卫战,有时还将战斗延伸至江西的信州(今上饶)和虔州(今赣州)等地。其主要对手就是淮南军阀杨行密和内部的“徐许之乱”。

钱鏐和杨行密的关系时而友好,时而敌对,体现出五代十国乱世的特点。双方的冲突共持续了三十年,其间钱曾出兵援助杨擒斩孙儒、安仁义等叛逆,并正式通婚,但也因董昌之战等发生过激烈的战斗。最后通过两次衣锦军保卫战和一次浪山江水战,才结束了双方的敌对状态。从此两浙地区进入休养生息的安定建设阶段。

902年,钱鏐刚被封为越王不久,其部下的徐绾和许再思起兵叛变,使钱鏐大伤元气。最后钱鏐支付了二十万缗犒军钱,并派两个儿子作为人质,才使得叛军撤兵。这次内乱后,钱鏐吸取了教训,治国更为谨慎。

904年被封为吴王;907年,后梁封钱鏐为吴越王,吴越国自此创建。龙德三年(923年),钱镠被册封为吴越国王,吴越建立王国体制。他改府署为朝廷,设置丞相、侍郎等百官,一切礼制皆按照君主的规格。

结束了与周边敌对势力的战争后,钱鏐开始转向对内的大规模经济和文化建设。唐大顺元年(890年)钱鏐开始着手建设杭州城。先后建造了夹城、罗城和子城。杭州罗城筑于唐景福元年(892年)七月,筑城时发动余杭、盐官、新城、唐山、富阳、龙泉“八都兵”,及紫溪、保城、龙通、三泉、三镇,合计“十三都兵”二十余万人。城区范围广袤七十里,四至分别是:南到六和塔;东至侯潮门和艮山门一线;北达武林门;西临涌金门和清波门一带,设朝天门、龙山门、竹车门、南土门、北土门、盐桥门、西关门(涵水门)、北关门、宝德门共十门。天宝三年(910年)又扩杭州城,凤凰山柳浦隋唐所筑子城被改造为府城,南为通越门,北为只门,子城内大修台馆,有天册堂(即王位之所)、天宠堂(即位、理政之所)、思政堂、功臣堂(寝宫)、握发殿、咸宁院、义和院、碧波亭、虚白堂、八会亭、都会堂、蓬莱阁、直仪门(设厅)、青史楼、天长楼、玉华楼、瑞萼园等建筑。钱鏐筑杭州城,在客观上为杭州成为日后南宋的都城打下了基础,南宋临安宫城即原吴越王宫。

钱王还在城内开凿水井(据说杭州的百井坊巷原有99眼,就开凿于此时),建设钱塘江堤,为杭州的饮水淡化问题做出了很大贡献。此外,钱鏐及其继承者崇信佛教,前后修建了不少寺院佛塔,使杭州在当时就有“佛国”之称。其中著名的灵隐寺、净慈寺、昭庆寺等寺院,以及雷峰塔、六和塔、保俶塔、闸口白塔和临安功臣塔等都是在吴越国时期兴建或扩建的。

钱鏐在内政建设上的主要成就体现在修筑海塘和疏浚内湖上。910年起,钱鏐上书后梁朝廷,指出“目击平原沃野,尽成江水汪洋,虽值干戈扰攘之后,即兴筑塘修堤之举。”,并开始着手修筑钱塘江沿岸石塘。由于钱江潮汛,工程进展困难,后钱鏐以竹器填以巨石,才奠定了基础。当时修筑的石塘,从六和塔一直到艮山门,长33万8593丈。此外,钱鏐还重点抓了疏浚西湖、太湖和鉴湖等工作。当时他设置了7000名撩湖兵,专门从事西湖的开浚工作 后代的苏轼也是在参考了钱鏐治湖的经验上,才开始大规模疏浚西湖。

然而据欧阳修《五代史》吴越世家所称,吳越自錢鏐時起,賦稅繁苛,小至雞、魚、雞卵、雞雛,也要納稅。貧民欠稅被捉到官府,按各稅欠數多少定笞數,往往積至笞數十以至百餘(一說五百余),民尤不勝其苦。於杭州建造「地上天宮」,耗盡民財民力。

钱鏐做節度使時,有人獻詩,詩中有「一條江水檻前流」句,「前流」與「钱鏐」是諧音,钱鏐認為獻詩人諷刺自己,於是暗殺此人。羅隱聲名大,曾作詩譏笑钱鏐出身寒家,錢鏐卻欣然不怒。錢鏐留心收買名士,皮日休(當是黃巢失敗後,逃來依靠錢鏐)、羅隱、胡嶽等都得到優待,自己也學吟詠,與名士唱和。天宝三年十月钱鏐巡视故乡衣锦军,置酒宴请父老,赏八十岁以上者金樽,百岁以上者玉樽,又作《还乡歌》:“三节还乡兮挂锦衣,碧天朗朗兮爱日晖。功臣道上兮列旌旗,父老远来兮相追随。家山乡眷兮会时稀,今朝设宴兮觥散飞。斗牛无孛兮民无欺,吴越一王兮驷马归”。父老不解其意,钱鏐复用吴语为歌:“你辈见侬底欢喜,则是一般滋味子,长在我侬心底里”,举座叫笑振席。

由于钱鏐在其晚年坚持保境安民政策,不参与军阀混战,而且对内统治相对廉洁清明,使得这一时期杭州的发展超越了中原地区的许多大城市,成为东南地区的经济中心。

后唐长兴三年(932年)三月己酉,錢鏐薨于临安王府正寝,年八十一岁,在位四十一年。葬安国县衣锦乡茅山,建庙于东府。后唐赐谥号武肃,吴越国上庙号太祖。

錢鏐累事三朝,唐、后梁、后唐屡加封号,累赐启圣匡运同德功臣、定乱安国启圣昌运同德守道戴功臣、淮南镇海镇东等军节度使、淮南浙江东西等道管内观察处置、充淮南四面都统营田安抚、兼两浙盐铁制置发运等使、天下兵马都元帅、开府仪同三司、尚父、检校太师、尚书令、兼中书令、上柱国、吴越国王,赐剑履上殿、诏书不名,食邑一万五千户。

欧阳修《五代史》称吴越“有改元而无称帝之事”。吴越国从908年(后梁开平二年)至913年(后梁乾化三年),曾用天宝年号;924年(后唐同光二年)至931年(后唐长兴二年)用宝大、宝正年号,皆仅行于吴越国中。

后世一般对钱氏评价较高,认为他促进了地方经济发展,保障了民众安居乐业的局面。主要有:“时维五纪乱何如?史册闲观亦皱眉。是地却逢钱节度,民间无事看花嬉!”——北宋·赵抃

“钱立国,置营田数千人于松江,辟土而耕,…民老死无他缠累,且完国归朝,不杀一人,则其功德大矣!”—— 明·朱国桢

史书载钱鏐性俭朴,衣衾杂用细布,常膳用瓷器、漆器。除夕子夜与子孙宴于府城内,未鼓数曲而令罢宴,称“闻者以我为长夜之歌”。其寝居之殿名为“握发殿”,取周公“一沐三握发”典故。

欧阳修在《新五代史·吴越世家》中谴责钱氏严刑酷法。而宋代别史《丹铅录》称,欧阳修为推官时,昵一妓,比而为忠懿王之子钱惟演得去,欧阳修深衔之,后作《五代史》时乃诬以钱氏诸王“重敛虐民”之语,以公报私。钱世昭撰《钱氏私志》也稱歐陽修是挾怨報复。

目前在西湖南岸,建有钱王祠,供后人瞻仰钱王业绩。

\subsubsection{天祐}

\begin{longtable}{|>{\centering\scriptsize}m{2em}|>{\centering\scriptsize}m{1.3em}|>{\centering}m{8.8em}|}
  % \caption{秦王政}\
  \toprule
  \SimHei \normalsize 年数 & \SimHei \scriptsize 公元 & \SimHei 大事件 \tabularnewline
  % \midrule
  \endfirsthead
  \toprule
  \SimHei \normalsize 年数 & \SimHei \scriptsize 公元 & \SimHei 大事件 \tabularnewline
  \midrule
  \endhead
  \midrule
  元年 & 907 & \tabularnewline
  \bottomrule
\end{longtable}

\subsubsection{天宝}

\begin{longtable}{|>{\centering\scriptsize}m{2em}|>{\centering\scriptsize}m{1.3em}|>{\centering}m{8.8em}|}
  % \caption{秦王政}\
  \toprule
  \SimHei \normalsize 年数 & \SimHei \scriptsize 公元 & \SimHei 大事件 \tabularnewline
  % \midrule
  \endfirsthead
  \toprule
  \SimHei \normalsize 年数 & \SimHei \scriptsize 公元 & \SimHei 大事件 \tabularnewline
  \midrule
  \endhead
  \midrule
  元年 & 908 & \tabularnewline\hline
  二年 & 909 & \tabularnewline\hline
  三年 & 910 & \tabularnewline\hline
  四年 & 911 & \tabularnewline\hline
  五年 & 912 & \tabularnewline
  \bottomrule
\end{longtable}

\subsubsection{凤历}

\begin{longtable}{|>{\centering\scriptsize}m{2em}|>{\centering\scriptsize}m{1.3em}|>{\centering}m{8.8em}|}
  % \caption{秦王政}\
  \toprule
  \SimHei \normalsize 年数 & \SimHei \scriptsize 公元 & \SimHei 大事件 \tabularnewline
  % \midrule
  \endfirsthead
  \toprule
  \SimHei \normalsize 年数 & \SimHei \scriptsize 公元 & \SimHei 大事件 \tabularnewline
  \midrule
  \endhead
  \midrule
  元年 & 913 & \tabularnewline
  \bottomrule
\end{longtable}

\subsubsection{乾化}

\begin{longtable}{|>{\centering\scriptsize}m{2em}|>{\centering\scriptsize}m{1.3em}|>{\centering}m{8.8em}|}
  % \caption{秦王政}\
  \toprule
  \SimHei \normalsize 年数 & \SimHei \scriptsize 公元 & \SimHei 大事件 \tabularnewline
  % \midrule
  \endfirsthead
  \toprule
  \SimHei \normalsize 年数 & \SimHei \scriptsize 公元 & \SimHei 大事件 \tabularnewline
  \midrule
  \endhead
  \midrule
  元年 & 913 & \tabularnewline\hline
  二年 & 914 & \tabularnewline\hline
  三年 & 915 & \tabularnewline
  \bottomrule
\end{longtable}

\subsubsection{贞明}

\begin{longtable}{|>{\centering\scriptsize}m{2em}|>{\centering\scriptsize}m{1.3em}|>{\centering}m{8.8em}|}
  % \caption{秦王政}\
  \toprule
  \SimHei \normalsize 年数 & \SimHei \scriptsize 公元 & \SimHei 大事件 \tabularnewline
  % \midrule
  \endfirsthead
  \toprule
  \SimHei \normalsize 年数 & \SimHei \scriptsize 公元 & \SimHei 大事件 \tabularnewline
  \midrule
  \endhead
  \midrule
  元年 & 915 & \tabularnewline\hline
  二年 & 916 & \tabularnewline\hline
  三年 & 917 & \tabularnewline\hline
  四年 & 918 & \tabularnewline\hline
  五年 & 919 & \tabularnewline\hline
  六年 & 920 & \tabularnewline\hline
  七年 & 921 & \tabularnewline
  \bottomrule
\end{longtable}

\subsubsection{龙德}

\begin{longtable}{|>{\centering\scriptsize}m{2em}|>{\centering\scriptsize}m{1.3em}|>{\centering}m{8.8em}|}
  % \caption{秦王政}\
  \toprule
  \SimHei \normalsize 年数 & \SimHei \scriptsize 公元 & \SimHei 大事件 \tabularnewline
  % \midrule
  \endfirsthead
  \toprule
  \SimHei \normalsize 年数 & \SimHei \scriptsize 公元 & \SimHei 大事件 \tabularnewline
  \midrule
  \endhead
  \midrule
  元年 & 921 & \tabularnewline\hline
  二年 & 922 & \tabularnewline\hline
  三年 & 923 & \tabularnewline
  \bottomrule
\end{longtable}

\subsubsection{宝大}

\begin{longtable}{|>{\centering\scriptsize}m{2em}|>{\centering\scriptsize}m{1.3em}|>{\centering}m{8.8em}|}
  % \caption{秦王政}\
  \toprule
  \SimHei \normalsize 年数 & \SimHei \scriptsize 公元 & \SimHei 大事件 \tabularnewline
  % \midrule
  \endfirsthead
  \toprule
  \SimHei \normalsize 年数 & \SimHei \scriptsize 公元 & \SimHei 大事件 \tabularnewline
  \midrule
  \endhead
  \midrule
  元年 & 924 & \tabularnewline\hline
  二年 & 925 & \tabularnewline
  \bottomrule
\end{longtable}

\subsubsection{宝正}

\begin{longtable}{|>{\centering\scriptsize}m{2em}|>{\centering\scriptsize}m{1.3em}|>{\centering}m{8.8em}|}
  % \caption{秦王政}\
  \toprule
  \SimHei \normalsize 年数 & \SimHei \scriptsize 公元 & \SimHei 大事件 \tabularnewline
  % \midrule
  \endfirsthead
  \toprule
  \SimHei \normalsize 年数 & \SimHei \scriptsize 公元 & \SimHei 大事件 \tabularnewline
  \midrule
  \endhead
  \midrule
  元年 & 926 & \tabularnewline\hline
  二年 & 927 & \tabularnewline\hline
  三年 & 928 & \tabularnewline\hline
  四年 & 929 & \tabularnewline\hline
  五年 & 930 & \tabularnewline\hline
  六年 & 931 & \tabularnewline
  \bottomrule
\end{longtable}




%%% Local Variables:
%%% mode: latex
%%% TeX-engine: xetex
%%% TeX-master: "../../Main"
%%% End:

%% -*- coding: utf-8 -*-
%% Time-stamp: <Chen Wang: 2019-12-26 09:39:23>

\subsection{世宗\tiny(932-941)}

\subsubsection{生平}

吳越世宗錢元瓘(887年-941年),字明寶,原名錢傳瓘,五代時期吳越國君主,是吳越國建立者錢鏐之子。

錢傳瓘为吴越武肃王錢鏐第七子,唐朝光启三年十一月十二日生于杭州东院,母妃陈氏。

唐昭宗天復二年(902年),寧國節度使田頵攻擊時為鎮海節度使(地處今浙江杭州)的錢鏐,將回師時,要求錢鏐以一子為質,並將女兒嫁其子。錢鏐諸子皆不願去,只有錢傳瓘自願前往,錢鏐因之稱奇。後來田頵敗死,錢傳瓘得以復歸杭州。等到長大成人後,率吳越軍爭戰各地,頗有戰功。贞明五年三月,吴越国应同盟後梁的邀请,进攻杨吴。传瓘任诸军都指挥使,帅战舰五百艘进攻。于狼山江大败吴军,进击常州。吴国实权者徐温亲自拒战,傳瓘力不能及,爲其所败。

吳錢鏐欲立儲,諸兄钱传懿、钱传璙、钱传璟皆相讓。越寶正七年(932年)錢鏐去世,錢傳瓘繼立,改名錢元瓘,不稱王,使用後唐長興年號。後唐明宗李嗣源長興四年(933年),為後唐封吳王。明年(934年)改封錢傳瓘为吳越王。後晉天福二年(937年)四月,後晉高祖石敬瑭進封錢元瓘為兴邦保运崇德志道功臣、天下兵马副元帅、镇海镇东等军节度使、浙江东西等道管内观察处置使兼两浙盐铁制置使、开府仪同三司、检校太师、守中书令、杭州越州大都督府长史、上柱国、食邑一万五千户实封一千五百户、吳越國國王,赐天下兵马副元帅金印。錢傳瓘于四月甲午在杭州行即位礼,吳越再度開國。是年十一月后晋赐吴越国王金册。

天福五年(940年),闽国内乱,王延政在建州起兵,钱元瓘派兵四万支持王延政,然而吴越军到达建州后王延羲与王延政已经休兵,吴越将领仰仁銓不肯班师,王延政惧怕,倒戈攻击,吴越死伤惨重。随后在同年,初次设立秀州(辖嘉兴、海盐、华亭、崇德四县,包括现今嘉兴和上海城区)和新昌县。

天福六年(941年),吳越王宮丽春院失火,延及内城,宮室府庫幾乎完全燒燬,錢元瓘逃到何處,火即蔓延何處。受此驚嚇,錢元瓘因而發瘋,迁居于杭州城东北的瑶台院(原为錢元瓘为孝献世子钱弘僔营建的世子府)。八月辛亥,錢元瓘在瑶台院綵云堂去世,终年五十五岁,在位十年。后晋赠諡庄穆,后改文穆。吴越上廟號世宗 。葬于今浙江萧山龙山南。子錢弘佐繼位。

\subsubsection{长兴}

\begin{longtable}{|>{\centering\scriptsize}m{2em}|>{\centering\scriptsize}m{1.3em}|>{\centering}m{8.8em}|}
  % \caption{秦王政}\
  \toprule
  \SimHei \normalsize 年数 & \SimHei \scriptsize 公元 & \SimHei 大事件 \tabularnewline
  % \midrule
  \endfirsthead
  \toprule
  \SimHei \normalsize 年数 & \SimHei \scriptsize 公元 & \SimHei 大事件 \tabularnewline
  \midrule
  \endhead
  \midrule
  元年 & 932 & \tabularnewline\hline
  二年 & 933 & \tabularnewline
  \bottomrule
\end{longtable}

\subsubsection{应顺}

\begin{longtable}{|>{\centering\scriptsize}m{2em}|>{\centering\scriptsize}m{1.3em}|>{\centering}m{8.8em}|}
  % \caption{秦王政}\
  \toprule
  \SimHei \normalsize 年数 & \SimHei \scriptsize 公元 & \SimHei 大事件 \tabularnewline
  % \midrule
  \endfirsthead
  \toprule
  \SimHei \normalsize 年数 & \SimHei \scriptsize 公元 & \SimHei 大事件 \tabularnewline
  \midrule
  \endhead
  \midrule
  元年 & 934 & \tabularnewline
  \bottomrule
\end{longtable}

\subsubsection{清泰}

\begin{longtable}{|>{\centering\scriptsize}m{2em}|>{\centering\scriptsize}m{1.3em}|>{\centering}m{8.8em}|}
  % \caption{秦王政}\
  \toprule
  \SimHei \normalsize 年数 & \SimHei \scriptsize 公元 & \SimHei 大事件 \tabularnewline
  % \midrule
  \endfirsthead
  \toprule
  \SimHei \normalsize 年数 & \SimHei \scriptsize 公元 & \SimHei 大事件 \tabularnewline
  \midrule
  \endhead
  \midrule
  元年 & 934 & \tabularnewline\hline
  二年 & 935 & \tabularnewline\hline
  三年 & 936 & \tabularnewline
  \bottomrule
\end{longtable}

\subsubsection{天福}

\begin{longtable}{|>{\centering\scriptsize}m{2em}|>{\centering\scriptsize}m{1.3em}|>{\centering}m{8.8em}|}
  % \caption{秦王政}\
  \toprule
  \SimHei \normalsize 年数 & \SimHei \scriptsize 公元 & \SimHei 大事件 \tabularnewline
  % \midrule
  \endfirsthead
  \toprule
  \SimHei \normalsize 年数 & \SimHei \scriptsize 公元 & \SimHei 大事件 \tabularnewline
  \midrule
  \endhead
  \midrule
  元年 & 936 & \tabularnewline\hline
  二年 & 937 & \tabularnewline\hline
  三年 & 938 & \tabularnewline\hline
  四年 & 939 & \tabularnewline\hline
  五年 & 940 & \tabularnewline\hline
  六年 & 941 & \tabularnewline
  \bottomrule
\end{longtable}



%%% Local Variables:
%%% mode: latex
%%% TeX-engine: xetex
%%% TeX-master: "../../Main"
%%% End:

%% -*- coding: utf-8 -*-
%% Time-stamp: <Chen Wang: 2021-11-01 15:40:37>

\subsection{成宗錢弘佐\tiny(941-947)}

\subsubsection{生平}

吳越成宗錢弘佐(928年-947年6月22日),字元祐,五代時期吳越國君主。

錢弘佐為吴越文穆王錢元瓘第六子,存世第二子。宝正三年七月二十六日生于杭州吴越王宫功臣堂,母许氏。

錢元瓘初以五子钱弘僔为世子,并为其建世子府于杭州城东北。一日钱弘僔与钱弘佐博彩于王宫青史楼,世子称“君王方为我营府署,愿与若博之”。骰四掷,钱弘佐得六赤色,世子失色。钱弘佐从容称“五哥入府,弘佐当将符印之命”,世子变色,投骰盘于楼下而去。天福五年钱弘僔薨,追赠孝献世子(世子府改为瑶台院),而钱弘佐得封镇海、镇东节度副使、检校太傅。

後晉天福六年(941年),錢元瓘去世,錢弘佐于当年九月庚申即位于王宫倦居堂。十一月,後晉封以镇国大将军、右金吾卫上将军、员外置同正员、领镇海镇东等军节度使、检校太师、兼中书令、吳越國王,食邑一万户,实封一千户。天福七年赐保邦宣化忠正戴功臣,加食邑七千户。天福八年赐吴越国王玉册。

後晉開運二年(945年),闽国内乱,钱弘佐派軍與南唐瓜分閩國,佔領福州。

錢弘佐喜好讀書,性情溫順,很會做詩。即位後,因尚年幼,無力控制下屬的驕橫,又曾寵信諂媚之人,然而終能摘奸發伏,亦不失果斷。

後漢天福十二年(遼國會同十年,947年)六月乙卯,錢弘佐去世于王宫咸宁院西堂,终年二十岁,在位七年。后漢赠諡忠獻王。吴越上廟號成宗。因其子尚年幼,故由其弟錢弘倧繼位。

\subsubsection{天福}

\begin{longtable}{|>{\centering\scriptsize}m{2em}|>{\centering\scriptsize}m{1.3em}|>{\centering}m{8.8em}|}
  % \caption{秦王政}\
  \toprule
  \SimHei \normalsize 年数 & \SimHei \scriptsize 公元 & \SimHei 大事件 \tabularnewline
  % \midrule
  \endfirsthead
  \toprule
  \SimHei \normalsize 年数 & \SimHei \scriptsize 公元 & \SimHei 大事件 \tabularnewline
  \midrule
  \endhead
  \midrule
  元年 & 941 & \tabularnewline\hline
  二年 & 942 & \tabularnewline\hline
  三年 & 943 & \tabularnewline\hline
  四年 & 944 & \tabularnewline
  \bottomrule
\end{longtable}

\subsubsection{开运}

\begin{longtable}{|>{\centering\scriptsize}m{2em}|>{\centering\scriptsize}m{1.3em}|>{\centering}m{8.8em}|}
  % \caption{秦王政}\
  \toprule
  \SimHei \normalsize 年数 & \SimHei \scriptsize 公元 & \SimHei 大事件 \tabularnewline
  % \midrule
  \endfirsthead
  \toprule
  \SimHei \normalsize 年数 & \SimHei \scriptsize 公元 & \SimHei 大事件 \tabularnewline
  \midrule
  \endhead
  \midrule
  元年 & 944 & \tabularnewline\hline
  二年 & 945 & \tabularnewline\hline
  三年 & 946 & \tabularnewline
  \bottomrule
\end{longtable}



%%% Local Variables:
%%% mode: latex
%%% TeX-engine: xetex
%%% TeX-master: "../../Main"
%%% End:

\input{14_ShiGuo/03_WuYue/04_ZhongXunWang}
%% -*- coding: utf-8 -*-
%% Time-stamp: <Chen Wang: 2019-12-26 09:41:36>

\subsection{钱弘俶\tiny(949-978)}

\subsubsection{生平}

钱俶(929年9月29日-988年10月7日)本名弘俶,因犯宋宣祖赵弘殷名讳,入宋後避諱,只称钱俶。字文德,小字虎子,五代十國時期吳越國文穆王錢元瓘第九子,吳越最后一位國王。

钱弘俶为吴越文穆王錢元瓘第九子,宝正四年八月二十四日(后唐天成四年,929年9月29日)生于杭州吴越王宫功臣堂,生母吴氏。累授内牙诸军指挥使、检校司空、检校太尉。开运四年出镇台州。忠逊王錢弘倧即位后,召钱弘俶返回杭州为同参相府事,居于南邸。

後漢天福十二年十二月三十(陽曆948年2月12日),吴越將領胡进思趁錢弘倧夜宴將吏时發動政變,錢弘倧被軟禁,錢弘俶被胡進思迎立為吳越王。乾祐元年正月乙卯,钱弘俶即位于杭州吴越王宫天宠堂。乾祐二年十月,后汉册封钱弘俶为匡圣广运同德保定功臣、东南面兵马都元帅、镇海镇东等军节度使、浙江东西等道管内观察处置使兼两浙盐铁制置发运营田等使、开府仪同三司、检校太师、兼中书令、杭州越州大都督、上柱国、吴越国王,食邑一万户、实封一千户。

钱弘俶嗣位三十餘年,期间恭事後漢、后周和北宋。后周广顺元年(951年)后周加封钱弘俶诸道兵马都元帅,加食邑一千户、实封三百户,翌年封天下兵马都元帅,加食邑二千户、实封五百户,改授推诚保德安邦致理忠正功臣。

后周显德三年(956年)正月,钱弘俶奉后周世宗诏,出兵攻南唐。显德五年四月,杭州城南失火,延烧内城,官府庐舍夷为平地,钱弘俶出居都城驿。后因大火即将烧及镇国仓,乃命官兵伐林木,火势方绝。尽管火灾惨重,吴越仍在当月向后周进贡绫、绢各二万匹,银子一万两;当年七月又向后周进贡银五千两、绢一万匹、龙舟一艘、天禄舟一艘,皆饰以银;十一月又进贡贺正钱一千贯、绢一千匹。周恭帝加封钱弘俶为崇仁昭德宣忠保庆扶天亮功臣。

宋建隆元年(960年)正月,后周殿前都点检赵匡胤發動陳橋兵變称帝,改国号为宋,遣使宣谕吴越。当年三月,钱弘俶改名为钱俶,以避赵匡胤之父赵弘殷名讳。建隆元年四月,赵匡胤封钱俶为天下兵马大元帅,加食邑一千户,实封五百户。乾德元年钱俶向宋朝进贡犀角、象牙各十株,香药十五万斤,珍珠、玳瑁器皿数百件,宋朝改赐钱俶为承家保国宣德守道忠贞恭顺忠臣,加食邑一千户、实封四百户。乾德二年加封食邑一千户、实封四百户。当年十一月宋朝伐后蜀,钱俶派孙承佑率军与宋会师。

开宝元年(968年),宋朝重新封钱俶为吴越国王,加食邑一千户、实封一百户,十二月再加封三千九百户,实封三百户。开宝四年加食邑二千户、实封六百户,改赐钱俶开吴镇海崇文耀武宣德守道功臣。开宝七年(974年)七月,宋太祖诏钱俶出兵协助宋朝攻打江南国(南唐),钱俶率军亲征,攻打常州城。翌年五月,宋朝赐钱俶守太师、尚书令,加食邑两千户、实封九百户。

吴越对宋谨遵事大之礼,世子钱惟濬曾四次入使宋朝,宋太祖亦屡赐钱俶御衣、剑佩、玉带、玉鞍、金银器、锦绣、金锁甲、御酒、马、羊、驼等物,及生辰礼物。

开宝九年(976年)正月,钱俶携王妃孙氏、世子钱惟濬自杭州出发,前往汴京觐见宋太祖,宋太祖命在礼贤宅为钱俶营建府第。钱俶贡通犀玉带、宝玉金器五千余件、上酒一千瓶、银十六万两、绢十一万匹、乳香五万斤,宋太祖赏赐钱俶黄金照匣、黄金钞锣、金二千两、银三万两、绢二万匹,赐以剑履上殿、诏书不名之礼。宋太祖宴钱俶于迎春苑,并许以“尽我一世”及“誓不杀钱王”之语。及钱俶辞行,宋太祖赏赐锦衣、玉带、玉鞍、玳瑁鞭,及金银锦彩二十余万、银装兵器八百余件,又赐王妃金器三百两、衣料二千匹、银二千两。又给钱俶黄袱一件,嘱曰“途中密视”。钱俶中途开袱检视,皆宋朝诸臣劝说扣留钱俶的奏章。当年五月、十一月,宋朝又两次加封钱俶食邑八千户、实封两千户。

开宝九年十月,宋太祖在宫中斧聲燭影而驾崩,其弟晋王赵光义即位,为宋太宗。太平兴国二年(977年,宋太宗即位,當年改元)三月封钱俶为尚书令、兼中书令、天下兵马大元帅。

太平兴国三年二月,钱俶自杭州出发,再次入宋朝觐。宋太宗赐宴于长春殿,命南唐后主违命侯李煜、南汉末帝恩赦侯劉鋹陪座。钱俶上吴越军甲器物名册,又乞辞天下兵马大元帅,宋太宗不许。五月乙酉,随同朝觐的吴越丞相崔仁冀劝钱俶上表纳土,否则祸患立至。钱俶遂当即上奏,献吴越国十三州、一军、八十六县、户五十五万六百八十、兵一十一万五千三十六于宋。

宋太宗升扬州为淮海国,虚封钱俶为淮海国王,食邑一万户、实封一千户,仍充天下兵马大元帅、守太师、尚书令、兼中书令,授宁淮镇海崇文耀武宣德守道功臣,赐剑履上殿。王世子钱惟濬为节度使兼侍中,其余各子亦授节度使、团练使、刺史等官。吴越幕僚宰相以下拜官者两千五百余人。

钱俶献土后居于东京礼贤宅,屡被宋太宗召入宫中赐宴、击球,并多次赏赐金银器、水晶、玛瑙、珊瑚、珍珠、龙涎香、贡茶、银、钱、绢等物,加食邑二万户、实封两千户。太平兴国四年,钱俶随宋太宗征北汉。雍熙元年改封汉南国王,加食邑两千户、实封两百户。

雍熙四年宋太宗改封钱俶为武胜军节度使、南阳国王,出居南阳,赐玉带、金唾壶;旋又加封为许王,加食邑一万户、实封两千户,赐安时镇国崇文耀武宣德守道功臣。端拱元年改封钱俶为邓王,加食邑一万户、实封三千户。

端拱元年八月二十四日(988年10月7日),钱俶六十大寿,宋太宗遣皇城使李惠、河州团练使王继恩至南阳,赐钱俶生辰礼物。钱俶与使者宴饮极欢。当天傍晚,钱俶在南阳住宅西轩命左右读《唐书》,又令子孙颂诗,忽然因风眩(脑卒中)发作,而于四漏时薨逝。或有怀疑其被毒杀者。

钱俶去世时的封号为安时镇国崇文耀武宣德守道中正功臣、武胜军节度使、开府仪同三司、守太师、尚书令兼中书令、使持节邓州诸军事、行邓州刺史、上柱国、邓王、食邑九万七千户、实封一万六千九百户、赐剑履上殿、诏书不名。宋太宗为其废朝七日,追封秦国王,赐谥号忠懿,葬于洛阳贤相里陶公原。宋真宗时,特诏追赠钱俶为尚父。有司请以礼贤宅为司天监,真宗不许。

钱俶好吟詠,自編其詩爲《政本集》,陶穀爲序,共十卷,今存一首“宫中作”。

\subsubsection{乾佑}

\begin{longtable}{|>{\centering\scriptsize}m{2em}|>{\centering\scriptsize}m{1.3em}|>{\centering}m{8.8em}|}
  % \caption{秦王政}\
  \toprule
  \SimHei \normalsize 年数 & \SimHei \scriptsize 公元 & \SimHei 大事件 \tabularnewline
  % \midrule
  \endfirsthead
  \toprule
  \SimHei \normalsize 年数 & \SimHei \scriptsize 公元 & \SimHei 大事件 \tabularnewline
  \midrule
  \endhead
  \midrule
  元年 & 948 & \tabularnewline\hline
  二年 & 949 & \tabularnewline\hline
  三年 & 950 & \tabularnewline
  \bottomrule
\end{longtable}

\subsubsection{广顺}

\begin{longtable}{|>{\centering\scriptsize}m{2em}|>{\centering\scriptsize}m{1.3em}|>{\centering}m{8.8em}|}
  % \caption{秦王政}\
  \toprule
  \SimHei \normalsize 年数 & \SimHei \scriptsize 公元 & \SimHei 大事件 \tabularnewline
  % \midrule
  \endfirsthead
  \toprule
  \SimHei \normalsize 年数 & \SimHei \scriptsize 公元 & \SimHei 大事件 \tabularnewline
  \midrule
  \endhead
  \midrule
  元年 & 951 & \tabularnewline\hline
  二年 & 952 & \tabularnewline\hline
  三年 & 953 & \tabularnewline
  \bottomrule
\end{longtable}

\subsubsection{显德}

\begin{longtable}{|>{\centering\scriptsize}m{2em}|>{\centering\scriptsize}m{1.3em}|>{\centering}m{8.8em}|}
  % \caption{秦王政}\
  \toprule
  \SimHei \normalsize 年数 & \SimHei \scriptsize 公元 & \SimHei 大事件 \tabularnewline
  % \midrule
  \endfirsthead
  \toprule
  \SimHei \normalsize 年数 & \SimHei \scriptsize 公元 & \SimHei 大事件 \tabularnewline
  \midrule
  \endhead
  \midrule
  元年 & 954 & \tabularnewline\hline
  二年 & 955 & \tabularnewline\hline
  三年 & 956 & \tabularnewline\hline
  四年 & 957 & \tabularnewline\hline
  五年 & 958 & \tabularnewline\hline
  六年 & 959 & \tabularnewline\hline
  七年 & 960 & \tabularnewline
  \bottomrule
\end{longtable}

\subsubsection{建隆}

\begin{longtable}{|>{\centering\scriptsize}m{2em}|>{\centering\scriptsize}m{1.3em}|>{\centering}m{8.8em}|}
  % \caption{秦王政}\
  \toprule
  \SimHei \normalsize 年数 & \SimHei \scriptsize 公元 & \SimHei 大事件 \tabularnewline
  % \midrule
  \endfirsthead
  \toprule
  \SimHei \normalsize 年数 & \SimHei \scriptsize 公元 & \SimHei 大事件 \tabularnewline
  \midrule
  \endhead
  \midrule
  元年 & 960 & \tabularnewline\hline
  二年 & 961 & \tabularnewline\hline
  三年 & 962 & \tabularnewline\hline
  四年 & 963 & \tabularnewline
  \bottomrule
\end{longtable}

\subsubsection{乾德}

\begin{longtable}{|>{\centering\scriptsize}m{2em}|>{\centering\scriptsize}m{1.3em}|>{\centering}m{8.8em}|}
  % \caption{秦王政}\
  \toprule
  \SimHei \normalsize 年数 & \SimHei \scriptsize 公元 & \SimHei 大事件 \tabularnewline
  % \midrule
  \endfirsthead
  \toprule
  \SimHei \normalsize 年数 & \SimHei \scriptsize 公元 & \SimHei 大事件 \tabularnewline
  \midrule
  \endhead
  \midrule
  元年 & 963 & \tabularnewline\hline
  二年 & 964 & \tabularnewline\hline
  三年 & 965 & \tabularnewline\hline
  四年 & 966 & \tabularnewline\hline
  五年 & 967 & \tabularnewline\hline
  六年 & 968 & \tabularnewline
  \bottomrule
\end{longtable}

\subsubsection{开宝}

\begin{longtable}{|>{\centering\scriptsize}m{2em}|>{\centering\scriptsize}m{1.3em}|>{\centering}m{8.8em}|}
  % \caption{秦王政}\
  \toprule
  \SimHei \normalsize 年数 & \SimHei \scriptsize 公元 & \SimHei 大事件 \tabularnewline
  % \midrule
  \endfirsthead
  \toprule
  \SimHei \normalsize 年数 & \SimHei \scriptsize 公元 & \SimHei 大事件 \tabularnewline
  \midrule
  \endhead
  \midrule
  元年 & 968 & \tabularnewline\hline
  二年 & 969 & \tabularnewline\hline
  三年 & 970 & \tabularnewline\hline
  四年 & 971 & \tabularnewline\hline
  五年 & 972 & \tabularnewline\hline
  六年 & 973 & \tabularnewline\hline
  七年 & 974 & \tabularnewline\hline
  八年 & 975 & \tabularnewline\hline
  九年 & 976 & \tabularnewline
  \bottomrule
\end{longtable}

\subsubsection{太平兴国}

\begin{longtable}{|>{\centering\scriptsize}m{2em}|>{\centering\scriptsize}m{1.3em}|>{\centering}m{8.8em}|}
  % \caption{秦王政}\
  \toprule
  \SimHei \normalsize 年数 & \SimHei \scriptsize 公元 & \SimHei 大事件 \tabularnewline
  % \midrule
  \endfirsthead
  \toprule
  \SimHei \normalsize 年数 & \SimHei \scriptsize 公元 & \SimHei 大事件 \tabularnewline
  \midrule
  \endhead
  \midrule
  元年 & 976 & \tabularnewline\hline
  二年 & 977 & \tabularnewline\hline
  三年 & 978 & \tabularnewline
  \bottomrule
\end{longtable}


%%% Local Variables:
%%% mode: latex
%%% TeX-engine: xetex
%%% TeX-master: "../../Main"
%%% End:



%%% Local Variables:
%%% mode: latex
%%% TeX-engine: xetex
%%% TeX-master: "../../Main"
%%% End:
