%% -*- coding: utf-8 -*-
%% Time-stamp: <Chen Wang: 2019-12-26 09:41:36>

\subsection{钱弘俶\tiny(949-978)}

\subsubsection{生平}

钱俶(929年9月29日-988年10月7日)本名弘俶,因犯宋宣祖赵弘殷名讳,入宋後避諱,只称钱俶。字文德,小字虎子,五代十國時期吳越國文穆王錢元瓘第九子,吳越最后一位國王。

钱弘俶为吴越文穆王錢元瓘第九子,宝正四年八月二十四日(后唐天成四年,929年9月29日)生于杭州吴越王宫功臣堂,生母吴氏。累授内牙诸军指挥使、检校司空、检校太尉。开运四年出镇台州。忠逊王錢弘倧即位后,召钱弘俶返回杭州为同参相府事,居于南邸。

後漢天福十二年十二月三十(陽曆948年2月12日),吴越將領胡进思趁錢弘倧夜宴將吏时發動政變,錢弘倧被軟禁,錢弘俶被胡進思迎立為吳越王。乾祐元年正月乙卯,钱弘俶即位于杭州吴越王宫天宠堂。乾祐二年十月,后汉册封钱弘俶为匡圣广运同德保定功臣、东南面兵马都元帅、镇海镇东等军节度使、浙江东西等道管内观察处置使兼两浙盐铁制置发运营田等使、开府仪同三司、检校太师、兼中书令、杭州越州大都督、上柱国、吴越国王,食邑一万户、实封一千户。

钱弘俶嗣位三十餘年,期间恭事後漢、后周和北宋。后周广顺元年(951年)后周加封钱弘俶诸道兵马都元帅,加食邑一千户、实封三百户,翌年封天下兵马都元帅,加食邑二千户、实封五百户,改授推诚保德安邦致理忠正功臣。

后周显德三年(956年)正月,钱弘俶奉后周世宗诏,出兵攻南唐。显德五年四月,杭州城南失火,延烧内城,官府庐舍夷为平地,钱弘俶出居都城驿。后因大火即将烧及镇国仓,乃命官兵伐林木,火势方绝。尽管火灾惨重,吴越仍在当月向后周进贡绫、绢各二万匹,银子一万两;当年七月又向后周进贡银五千两、绢一万匹、龙舟一艘、天禄舟一艘,皆饰以银;十一月又进贡贺正钱一千贯、绢一千匹。周恭帝加封钱弘俶为崇仁昭德宣忠保庆扶天亮功臣。

宋建隆元年(960年)正月,后周殿前都点检赵匡胤發動陳橋兵變称帝,改国号为宋,遣使宣谕吴越。当年三月,钱弘俶改名为钱俶,以避赵匡胤之父赵弘殷名讳。建隆元年四月,赵匡胤封钱俶为天下兵马大元帅,加食邑一千户,实封五百户。乾德元年钱俶向宋朝进贡犀角、象牙各十株,香药十五万斤,珍珠、玳瑁器皿数百件,宋朝改赐钱俶为承家保国宣德守道忠贞恭顺忠臣,加食邑一千户、实封四百户。乾德二年加封食邑一千户、实封四百户。当年十一月宋朝伐后蜀,钱俶派孙承佑率军与宋会师。

开宝元年(968年),宋朝重新封钱俶为吴越国王,加食邑一千户、实封一百户,十二月再加封三千九百户,实封三百户。开宝四年加食邑二千户、实封六百户,改赐钱俶开吴镇海崇文耀武宣德守道功臣。开宝七年(974年)七月,宋太祖诏钱俶出兵协助宋朝攻打江南国(南唐),钱俶率军亲征,攻打常州城。翌年五月,宋朝赐钱俶守太师、尚书令,加食邑两千户、实封九百户。

吴越对宋谨遵事大之礼,世子钱惟濬曾四次入使宋朝,宋太祖亦屡赐钱俶御衣、剑佩、玉带、玉鞍、金银器、锦绣、金锁甲、御酒、马、羊、驼等物,及生辰礼物。

开宝九年(976年)正月,钱俶携王妃孙氏、世子钱惟濬自杭州出发,前往汴京觐见宋太祖,宋太祖命在礼贤宅为钱俶营建府第。钱俶贡通犀玉带、宝玉金器五千余件、上酒一千瓶、银十六万两、绢十一万匹、乳香五万斤,宋太祖赏赐钱俶黄金照匣、黄金钞锣、金二千两、银三万两、绢二万匹,赐以剑履上殿、诏书不名之礼。宋太祖宴钱俶于迎春苑,并许以“尽我一世”及“誓不杀钱王”之语。及钱俶辞行,宋太祖赏赐锦衣、玉带、玉鞍、玳瑁鞭,及金银锦彩二十余万、银装兵器八百余件,又赐王妃金器三百两、衣料二千匹、银二千两。又给钱俶黄袱一件,嘱曰“途中密视”。钱俶中途开袱检视,皆宋朝诸臣劝说扣留钱俶的奏章。当年五月、十一月,宋朝又两次加封钱俶食邑八千户、实封两千户。

开宝九年十月,宋太祖在宫中斧聲燭影而驾崩,其弟晋王赵光义即位,为宋太宗。太平兴国二年(977年,宋太宗即位,當年改元)三月封钱俶为尚书令、兼中书令、天下兵马大元帅。

太平兴国三年二月,钱俶自杭州出发,再次入宋朝觐。宋太宗赐宴于长春殿,命南唐后主违命侯李煜、南汉末帝恩赦侯劉鋹陪座。钱俶上吴越军甲器物名册,又乞辞天下兵马大元帅,宋太宗不许。五月乙酉,随同朝觐的吴越丞相崔仁冀劝钱俶上表纳土,否则祸患立至。钱俶遂当即上奏,献吴越国十三州、一军、八十六县、户五十五万六百八十、兵一十一万五千三十六于宋。

宋太宗升扬州为淮海国,虚封钱俶为淮海国王,食邑一万户、实封一千户,仍充天下兵马大元帅、守太师、尚书令、兼中书令,授宁淮镇海崇文耀武宣德守道功臣,赐剑履上殿。王世子钱惟濬为节度使兼侍中,其余各子亦授节度使、团练使、刺史等官。吴越幕僚宰相以下拜官者两千五百余人。

钱俶献土后居于东京礼贤宅,屡被宋太宗召入宫中赐宴、击球,并多次赏赐金银器、水晶、玛瑙、珊瑚、珍珠、龙涎香、贡茶、银、钱、绢等物,加食邑二万户、实封两千户。太平兴国四年,钱俶随宋太宗征北汉。雍熙元年改封汉南国王,加食邑两千户、实封两百户。

雍熙四年宋太宗改封钱俶为武胜军节度使、南阳国王,出居南阳,赐玉带、金唾壶;旋又加封为许王,加食邑一万户、实封两千户,赐安时镇国崇文耀武宣德守道功臣。端拱元年改封钱俶为邓王,加食邑一万户、实封三千户。

端拱元年八月二十四日(988年10月7日),钱俶六十大寿,宋太宗遣皇城使李惠、河州团练使王继恩至南阳,赐钱俶生辰礼物。钱俶与使者宴饮极欢。当天傍晚,钱俶在南阳住宅西轩命左右读《唐书》,又令子孙颂诗,忽然因风眩(脑卒中)发作,而于四漏时薨逝。或有怀疑其被毒杀者。

钱俶去世时的封号为安时镇国崇文耀武宣德守道中正功臣、武胜军节度使、开府仪同三司、守太师、尚书令兼中书令、使持节邓州诸军事、行邓州刺史、上柱国、邓王、食邑九万七千户、实封一万六千九百户、赐剑履上殿、诏书不名。宋太宗为其废朝七日,追封秦国王,赐谥号忠懿,葬于洛阳贤相里陶公原。宋真宗时,特诏追赠钱俶为尚父。有司请以礼贤宅为司天监,真宗不许。

钱俶好吟詠,自編其詩爲《政本集》,陶穀爲序,共十卷,今存一首“宫中作”。

\subsubsection{乾佑}

\begin{longtable}{|>{\centering\scriptsize}m{2em}|>{\centering\scriptsize}m{1.3em}|>{\centering}m{8.8em}|}
  % \caption{秦王政}\
  \toprule
  \SimHei \normalsize 年数 & \SimHei \scriptsize 公元 & \SimHei 大事件 \tabularnewline
  % \midrule
  \endfirsthead
  \toprule
  \SimHei \normalsize 年数 & \SimHei \scriptsize 公元 & \SimHei 大事件 \tabularnewline
  \midrule
  \endhead
  \midrule
  元年 & 948 & \tabularnewline\hline
  二年 & 949 & \tabularnewline\hline
  三年 & 950 & \tabularnewline
  \bottomrule
\end{longtable}

\subsubsection{广顺}

\begin{longtable}{|>{\centering\scriptsize}m{2em}|>{\centering\scriptsize}m{1.3em}|>{\centering}m{8.8em}|}
  % \caption{秦王政}\
  \toprule
  \SimHei \normalsize 年数 & \SimHei \scriptsize 公元 & \SimHei 大事件 \tabularnewline
  % \midrule
  \endfirsthead
  \toprule
  \SimHei \normalsize 年数 & \SimHei \scriptsize 公元 & \SimHei 大事件 \tabularnewline
  \midrule
  \endhead
  \midrule
  元年 & 951 & \tabularnewline\hline
  二年 & 952 & \tabularnewline\hline
  三年 & 953 & \tabularnewline
  \bottomrule
\end{longtable}

\subsubsection{显德}

\begin{longtable}{|>{\centering\scriptsize}m{2em}|>{\centering\scriptsize}m{1.3em}|>{\centering}m{8.8em}|}
  % \caption{秦王政}\
  \toprule
  \SimHei \normalsize 年数 & \SimHei \scriptsize 公元 & \SimHei 大事件 \tabularnewline
  % \midrule
  \endfirsthead
  \toprule
  \SimHei \normalsize 年数 & \SimHei \scriptsize 公元 & \SimHei 大事件 \tabularnewline
  \midrule
  \endhead
  \midrule
  元年 & 954 & \tabularnewline\hline
  二年 & 955 & \tabularnewline\hline
  三年 & 956 & \tabularnewline\hline
  四年 & 957 & \tabularnewline\hline
  五年 & 958 & \tabularnewline\hline
  六年 & 959 & \tabularnewline\hline
  七年 & 960 & \tabularnewline
  \bottomrule
\end{longtable}

\subsubsection{建隆}

\begin{longtable}{|>{\centering\scriptsize}m{2em}|>{\centering\scriptsize}m{1.3em}|>{\centering}m{8.8em}|}
  % \caption{秦王政}\
  \toprule
  \SimHei \normalsize 年数 & \SimHei \scriptsize 公元 & \SimHei 大事件 \tabularnewline
  % \midrule
  \endfirsthead
  \toprule
  \SimHei \normalsize 年数 & \SimHei \scriptsize 公元 & \SimHei 大事件 \tabularnewline
  \midrule
  \endhead
  \midrule
  元年 & 960 & \tabularnewline\hline
  二年 & 961 & \tabularnewline\hline
  三年 & 962 & \tabularnewline\hline
  四年 & 963 & \tabularnewline
  \bottomrule
\end{longtable}

\subsubsection{乾德}

\begin{longtable}{|>{\centering\scriptsize}m{2em}|>{\centering\scriptsize}m{1.3em}|>{\centering}m{8.8em}|}
  % \caption{秦王政}\
  \toprule
  \SimHei \normalsize 年数 & \SimHei \scriptsize 公元 & \SimHei 大事件 \tabularnewline
  % \midrule
  \endfirsthead
  \toprule
  \SimHei \normalsize 年数 & \SimHei \scriptsize 公元 & \SimHei 大事件 \tabularnewline
  \midrule
  \endhead
  \midrule
  元年 & 963 & \tabularnewline\hline
  二年 & 964 & \tabularnewline\hline
  三年 & 965 & \tabularnewline\hline
  四年 & 966 & \tabularnewline\hline
  五年 & 967 & \tabularnewline\hline
  六年 & 968 & \tabularnewline
  \bottomrule
\end{longtable}

\subsubsection{开宝}

\begin{longtable}{|>{\centering\scriptsize}m{2em}|>{\centering\scriptsize}m{1.3em}|>{\centering}m{8.8em}|}
  % \caption{秦王政}\
  \toprule
  \SimHei \normalsize 年数 & \SimHei \scriptsize 公元 & \SimHei 大事件 \tabularnewline
  % \midrule
  \endfirsthead
  \toprule
  \SimHei \normalsize 年数 & \SimHei \scriptsize 公元 & \SimHei 大事件 \tabularnewline
  \midrule
  \endhead
  \midrule
  元年 & 968 & \tabularnewline\hline
  二年 & 969 & \tabularnewline\hline
  三年 & 970 & \tabularnewline\hline
  四年 & 971 & \tabularnewline\hline
  五年 & 972 & \tabularnewline\hline
  六年 & 973 & \tabularnewline\hline
  七年 & 974 & \tabularnewline\hline
  八年 & 975 & \tabularnewline\hline
  九年 & 976 & \tabularnewline
  \bottomrule
\end{longtable}

\subsubsection{太平兴国}

\begin{longtable}{|>{\centering\scriptsize}m{2em}|>{\centering\scriptsize}m{1.3em}|>{\centering}m{8.8em}|}
  % \caption{秦王政}\
  \toprule
  \SimHei \normalsize 年数 & \SimHei \scriptsize 公元 & \SimHei 大事件 \tabularnewline
  % \midrule
  \endfirsthead
  \toprule
  \SimHei \normalsize 年数 & \SimHei \scriptsize 公元 & \SimHei 大事件 \tabularnewline
  \midrule
  \endhead
  \midrule
  元年 & 976 & \tabularnewline\hline
  二年 & 977 & \tabularnewline\hline
  三年 & 978 & \tabularnewline
  \bottomrule
\end{longtable}


%%% Local Variables:
%%% mode: latex
%%% TeX-engine: xetex
%%% TeX-master: "../../Main"
%%% End:
