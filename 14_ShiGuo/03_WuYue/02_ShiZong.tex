%% -*- coding: utf-8 -*-
%% Time-stamp: <Chen Wang: 2019-12-26 09:39:23>

\subsection{世宗\tiny(932-941)}

\subsubsection{生平}

吳越世宗錢元瓘(887年-941年),字明寶,原名錢傳瓘,五代時期吳越國君主,是吳越國建立者錢鏐之子。

錢傳瓘为吴越武肃王錢鏐第七子,唐朝光启三年十一月十二日生于杭州东院,母妃陈氏。

唐昭宗天復二年(902年),寧國節度使田頵攻擊時為鎮海節度使(地處今浙江杭州)的錢鏐,將回師時,要求錢鏐以一子為質,並將女兒嫁其子。錢鏐諸子皆不願去,只有錢傳瓘自願前往,錢鏐因之稱奇。後來田頵敗死,錢傳瓘得以復歸杭州。等到長大成人後,率吳越軍爭戰各地,頗有戰功。贞明五年三月,吴越国应同盟後梁的邀请,进攻杨吴。传瓘任诸军都指挥使,帅战舰五百艘进攻。于狼山江大败吴军,进击常州。吴国实权者徐温亲自拒战,傳瓘力不能及,爲其所败。

吳錢鏐欲立儲,諸兄钱传懿、钱传璙、钱传璟皆相讓。越寶正七年(932年)錢鏐去世,錢傳瓘繼立,改名錢元瓘,不稱王,使用後唐長興年號。後唐明宗李嗣源長興四年(933年),為後唐封吳王。明年(934年)改封錢傳瓘为吳越王。後晉天福二年(937年)四月,後晉高祖石敬瑭進封錢元瓘為兴邦保运崇德志道功臣、天下兵马副元帅、镇海镇东等军节度使、浙江东西等道管内观察处置使兼两浙盐铁制置使、开府仪同三司、检校太师、守中书令、杭州越州大都督府长史、上柱国、食邑一万五千户实封一千五百户、吳越國國王,赐天下兵马副元帅金印。錢傳瓘于四月甲午在杭州行即位礼,吳越再度開國。是年十一月后晋赐吴越国王金册。

天福五年(940年),闽国内乱,王延政在建州起兵,钱元瓘派兵四万支持王延政,然而吴越军到达建州后王延羲与王延政已经休兵,吴越将领仰仁銓不肯班师,王延政惧怕,倒戈攻击,吴越死伤惨重。随后在同年,初次设立秀州(辖嘉兴、海盐、华亭、崇德四县,包括现今嘉兴和上海城区)和新昌县。

天福六年(941年),吳越王宮丽春院失火,延及内城,宮室府庫幾乎完全燒燬,錢元瓘逃到何處,火即蔓延何處。受此驚嚇,錢元瓘因而發瘋,迁居于杭州城东北的瑶台院(原为錢元瓘为孝献世子钱弘僔营建的世子府)。八月辛亥,錢元瓘在瑶台院綵云堂去世,终年五十五岁,在位十年。后晋赠諡庄穆,后改文穆。吴越上廟號世宗 。葬于今浙江萧山龙山南。子錢弘佐繼位。

\subsubsection{长兴}

\begin{longtable}{|>{\centering\scriptsize}m{2em}|>{\centering\scriptsize}m{1.3em}|>{\centering}m{8.8em}|}
  % \caption{秦王政}\
  \toprule
  \SimHei \normalsize 年数 & \SimHei \scriptsize 公元 & \SimHei 大事件 \tabularnewline
  % \midrule
  \endfirsthead
  \toprule
  \SimHei \normalsize 年数 & \SimHei \scriptsize 公元 & \SimHei 大事件 \tabularnewline
  \midrule
  \endhead
  \midrule
  元年 & 932 & \tabularnewline\hline
  二年 & 933 & \tabularnewline
  \bottomrule
\end{longtable}

\subsubsection{应顺}

\begin{longtable}{|>{\centering\scriptsize}m{2em}|>{\centering\scriptsize}m{1.3em}|>{\centering}m{8.8em}|}
  % \caption{秦王政}\
  \toprule
  \SimHei \normalsize 年数 & \SimHei \scriptsize 公元 & \SimHei 大事件 \tabularnewline
  % \midrule
  \endfirsthead
  \toprule
  \SimHei \normalsize 年数 & \SimHei \scriptsize 公元 & \SimHei 大事件 \tabularnewline
  \midrule
  \endhead
  \midrule
  元年 & 934 & \tabularnewline
  \bottomrule
\end{longtable}

\subsubsection{清泰}

\begin{longtable}{|>{\centering\scriptsize}m{2em}|>{\centering\scriptsize}m{1.3em}|>{\centering}m{8.8em}|}
  % \caption{秦王政}\
  \toprule
  \SimHei \normalsize 年数 & \SimHei \scriptsize 公元 & \SimHei 大事件 \tabularnewline
  % \midrule
  \endfirsthead
  \toprule
  \SimHei \normalsize 年数 & \SimHei \scriptsize 公元 & \SimHei 大事件 \tabularnewline
  \midrule
  \endhead
  \midrule
  元年 & 934 & \tabularnewline\hline
  二年 & 935 & \tabularnewline\hline
  三年 & 936 & \tabularnewline
  \bottomrule
\end{longtable}

\subsubsection{天福}

\begin{longtable}{|>{\centering\scriptsize}m{2em}|>{\centering\scriptsize}m{1.3em}|>{\centering}m{8.8em}|}
  % \caption{秦王政}\
  \toprule
  \SimHei \normalsize 年数 & \SimHei \scriptsize 公元 & \SimHei 大事件 \tabularnewline
  % \midrule
  \endfirsthead
  \toprule
  \SimHei \normalsize 年数 & \SimHei \scriptsize 公元 & \SimHei 大事件 \tabularnewline
  \midrule
  \endhead
  \midrule
  元年 & 936 & \tabularnewline\hline
  二年 & 937 & \tabularnewline\hline
  三年 & 938 & \tabularnewline\hline
  四年 & 939 & \tabularnewline\hline
  五年 & 940 & \tabularnewline\hline
  六年 & 941 & \tabularnewline
  \bottomrule
\end{longtable}



%%% Local Variables:
%%% mode: latex
%%% TeX-engine: xetex
%%% TeX-master: "../../Main"
%%% End:
