%% -*- coding: utf-8 -*-
%% Time-stamp: <Chen Wang: 2021-11-01 15:40:37>

\subsection{成宗錢弘佐\tiny(941-947)}

\subsubsection{生平}

吳越成宗錢弘佐(928年-947年6月22日),字元祐,五代時期吳越國君主。

錢弘佐為吴越文穆王錢元瓘第六子,存世第二子。宝正三年七月二十六日生于杭州吴越王宫功臣堂,母许氏。

錢元瓘初以五子钱弘僔为世子,并为其建世子府于杭州城东北。一日钱弘僔与钱弘佐博彩于王宫青史楼,世子称“君王方为我营府署,愿与若博之”。骰四掷,钱弘佐得六赤色,世子失色。钱弘佐从容称“五哥入府,弘佐当将符印之命”,世子变色,投骰盘于楼下而去。天福五年钱弘僔薨,追赠孝献世子(世子府改为瑶台院),而钱弘佐得封镇海、镇东节度副使、检校太傅。

後晉天福六年(941年),錢元瓘去世,錢弘佐于当年九月庚申即位于王宫倦居堂。十一月,後晉封以镇国大将军、右金吾卫上将军、员外置同正员、领镇海镇东等军节度使、检校太师、兼中书令、吳越國王,食邑一万户,实封一千户。天福七年赐保邦宣化忠正戴功臣,加食邑七千户。天福八年赐吴越国王玉册。

後晉開運二年(945年),闽国内乱,钱弘佐派軍與南唐瓜分閩國,佔領福州。

錢弘佐喜好讀書,性情溫順,很會做詩。即位後,因尚年幼,無力控制下屬的驕橫,又曾寵信諂媚之人,然而終能摘奸發伏,亦不失果斷。

後漢天福十二年(遼國會同十年,947年)六月乙卯,錢弘佐去世于王宫咸宁院西堂,终年二十岁,在位七年。后漢赠諡忠獻王。吴越上廟號成宗。因其子尚年幼,故由其弟錢弘倧繼位。

\subsubsection{天福}

\begin{longtable}{|>{\centering\scriptsize}m{2em}|>{\centering\scriptsize}m{1.3em}|>{\centering}m{8.8em}|}
  % \caption{秦王政}\
  \toprule
  \SimHei \normalsize 年数 & \SimHei \scriptsize 公元 & \SimHei 大事件 \tabularnewline
  % \midrule
  \endfirsthead
  \toprule
  \SimHei \normalsize 年数 & \SimHei \scriptsize 公元 & \SimHei 大事件 \tabularnewline
  \midrule
  \endhead
  \midrule
  元年 & 941 & \tabularnewline\hline
  二年 & 942 & \tabularnewline\hline
  三年 & 943 & \tabularnewline\hline
  四年 & 944 & \tabularnewline
  \bottomrule
\end{longtable}

\subsubsection{开运}

\begin{longtable}{|>{\centering\scriptsize}m{2em}|>{\centering\scriptsize}m{1.3em}|>{\centering}m{8.8em}|}
  % \caption{秦王政}\
  \toprule
  \SimHei \normalsize 年数 & \SimHei \scriptsize 公元 & \SimHei 大事件 \tabularnewline
  % \midrule
  \endfirsthead
  \toprule
  \SimHei \normalsize 年数 & \SimHei \scriptsize 公元 & \SimHei 大事件 \tabularnewline
  \midrule
  \endhead
  \midrule
  元年 & 944 & \tabularnewline\hline
  二年 & 945 & \tabularnewline\hline
  三年 & 946 & \tabularnewline
  \bottomrule
\end{longtable}



%%% Local Variables:
%%% mode: latex
%%% TeX-engine: xetex
%%% TeX-master: "../../Main"
%%% End:
