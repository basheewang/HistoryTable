%% -*- coding: utf-8 -*-
%% Time-stamp: <Chen Wang: 2019-12-26 09:40:32>

\subsection{忠逊王\tiny(947)}

\subsubsection{生平}

吳越忠遜王錢弘倧(928年-971年),字隆道,五代時期吳越國君主。

錢弘倧為文穆王錢元瓘第七子,忠献王錢弘佐之弟,孝献世子钱弘僔同母弟。诞生时,其父梦人献黄金一箧,故幼名万金。

後漢天福十二年(遼國會同十年,947年),錢弘佐去世,子尚年幼,因此在遗诏中命弟錢弘倧繼立。天福十二年六月丙寅,即王位于杭州吴越王宫天册堂。當時遼太宗耶律德光滅後晉,佔據中原,於是錢弘倧向其稱臣;不久遼軍退去,復對後漢稱臣,奉其正朔。

先前忠献王錢弘佐在位時,諸將驕橫,雖然擅權者旋遭誅殺,然而對下屬還是頗為寬大;而錢弘倧個性嚴厲堅定,等到繼位後,急欲改變這種情形,因此極力抑制將領。即位后不久在碧波亭检阅水師,內牙統軍使胡进思进谏说颁赏太厚,钱弘倧怒,掷笔于水中。胡进思因害怕被剷除,遂先發制人。

後漢天福十二年(947年)十二月三十日(陽曆為948年2月12日),錢弘倧在王宫中夜宴诸将。胡进思怀疑王将图己,于是率内牙亲兵戎服入宫,發動政變。錢弘倧被軟禁于义和院,胡進思假传钱弘倧命令,称钱弘倧中风,并迎錢弘倧之異母弟錢弘俶于私第,将其策立为王。

钱弘俶即位后,迁钱弘倧于太祖錢鏐故里衣锦軍,派匡武都头薛温保护,并嘱咐薛温:「自己没有杀兄的意思,一旦傳來类似的命令,必须拚死拒绝。」胡进思屡次请求钱弘俶杀钱弘倧,钱弘俶都拒绝,胡进思又假传王命要薛温杀钱弘倧,薛温也拒绝;胡进思自己派刺客方安等二人持兵器翻墙去杀钱弘倧,钱弘倧发现后闭门呼救,薛温率军赶来在庭院击杀方安二人。雖然胡進思不久後即病逝,但钱弘倧還是繼續被軟禁。

後周廣順元年(951年),錢弘俶把錢弘倧遷至東府越州(今浙江紹興),並為其興築宮室,以东府官物为供给。在西寝殿后的卧龙山为钱弘倧开辟花园,遍植花木。遇良辰美景,钱弘倧穿道士服,拥妓乐,旦暮登山赏景。每年元夜张灯于山谷,用油数千斤;七夕在山顶以绫罗结为彩楼,钱弘倧登山击鼓,声达于外,官吏报之,钱弘俶都不追究。以後每年逢年過節時的贈禮都非常豐厚。

北宋建立后,吴越国臣服北宋,为宋之先祖趙弘殷避讳,钱弘倧改名钱倧。宋太祖開寶年間,錢倧因病去世,享年四十四歲,以王禮葬之,赠諡忠遜王(一作諡讓王)。

\subsubsection{天福}

\begin{longtable}{|>{\centering\scriptsize}m{2em}|>{\centering\scriptsize}m{1.3em}|>{\centering}m{8.8em}|}
  % \caption{秦王政}\
  \toprule
  \SimHei \normalsize 年数 & \SimHei \scriptsize 公元 & \SimHei 大事件 \tabularnewline
  % \midrule
  \endfirsthead
  \toprule
  \SimHei \normalsize 年数 & \SimHei \scriptsize 公元 & \SimHei 大事件 \tabularnewline
  \midrule
  \endhead
  \midrule
  元年 & 947 & \tabularnewline
  \bottomrule
\end{longtable}


%%% Local Variables:
%%% mode: latex
%%% TeX-engine: xetex
%%% TeX-master: "../../Main"
%%% End:
