%% -*- coding: utf-8 -*-
%% Time-stamp: <Chen Wang: 2019-12-24 17:47:12>

\subsection{武帝\tiny(902-905)}

\subsubsection{生平}

吳孝武王杨行密(852年-905年),字化源,原名行愍,庐州合肥(今安徽合肥长丰)人,唐朝末年著名政治家、军事家,五代十国時期吴国政權奠定者。唐乾宁二年(895年)进同中书门下平章事、弘农郡王。天复二年(902)进中书令、封吴王,天佑二年(905)病死,唐谥武忠王,吴国武义年间改谥孝武王,其子杨溥稱帝时,追尊其为武皇帝,庙号太祖。

杨行密原为庐州牙将,中和三年(883)拜庐州刺史,归淮南节度使高骈。886年,高骈賜名為杨行密。

中和五年(885)毕师铎反高骈,召宣歙观察使秦彦助战,高骈向行密求救,行密尚未赶到,毕师铎已俘高骈,行密一到,大败毕师铎,秦彥一气之下,杀死高骈,行密占领扬州,毕师铎投奔秦宗权部下孙儒,孙儒杀毕师铎并吞并其军队,发兵围攻扬州,欲一举消灭杨行密,将扬州收归己有。

行密采谋士袁袭建议,放弃扬州,先退守庐州(今安徽合肥),后攻克宣州(今安徽宣城)。龙纪元年(889)拜宣州观察使。行密據有宣州後,趁势向东、南、西三个方向发展,占领苏州、常州、润州(今江苏镇江)、滁州、和州(今安徽和县)等地,势力急剧扩大,领地包括了现在的安徽、江苏、浙江和江西、湖北等省部分地区。

景福元年(892)取楚州(今江苏淮安)、杨行密的发展,使占有扬州的孙儒受到三面包围,孙儒杀向宣州。行密击溃孙儒,并当众將之斬殺。复入扬州,进淮南节度使。

此后行密又出兵扩大地盘,将淮河以南和长江以东大片领土都纳入自己势力范围,为后来楊吳疆土基本上定型。乾宁二年(895)进同中书门下平章事、弘农郡王。

行密为扩大势力范围,乘蔡州四面行营都统朱温忙于对兖州、郓州(治今山东郓城)用兵之机,主动出击作战。攻取濠州(治今安徽凤阳)、寿州(治今安徽寿县),袭占涟水。又通过对依附于朱温的州县主动出击作战,拓展地盘,阻遏朱温插足淮南,为稳定和发展自己势力创造条件。

乾宁四年(897)宣武节度使朱温大举南侵,行密亲战,先集中自己的精锐主力攻击东边庞师古部,命朱瑾掘开淮河河堤,用水大淹庞师古部,同时遣朱瑾、张训领兵击败庞师古部于清口(今江苏淮安),宣武军损失惨重,大败而归,庞师古阵亡,葛从周逃回。朱温此后即无力南下,此后数十年间,南北遂成分裂之局。

吴越王钱鏐派兵攻打行密,兵進苏州。行密命周本禦敵,卻作战失利,失苏州。行密经过充分准备,派李神福进攻钱鏐,于杭州大败钱鏐军队并活捉其大将顾全武。经过长期混战,行密在江淮一带立足。

天复二年(902)进中书令、封吴王。天复三年(903年),行密遣李神福击破武昌节度使杜洪于君山(今湖南岳阳)。朱温败于行密后,向东进攻王师范,王师范求救于行密。行密于是年四月遣王茂章领兵出征,六月王茂章击破朱温军队,朱温再不敢对江淮用兵。八月至十二月,宁国节度使田頵与润州团练使安仁义起兵反行密,行密令李神福、臺濛、王茂章等将击溃田頵部于芜湖,广德、黄池(今安徽当涂)、宣州(治今安徽宣城)等地,田頵亡于宣州战场。令王茂章击破安仁义于润州并斩于广陵(今江苏扬州)。后行密诈瞎诛杀朱延寿叛逆势力。

天佑二年(905)十一月庚辰(二十六)日(12月24日)吴王杨行密病逝,其子杨渥继立。唐朝谥武忠王,吴国武义年间改谥孝武王,杨溥即帝位时追尊其为武皇帝,庙号太祖。

行密为政颇能选拔贤才,招集流散,轻徭薄赋,劝课农桑,使江淮一带社会经济在战争的间隙有较大恢复。

大唐衰变后,藩鎮割據,諸侯并起。行密在江淮地区举起割据大旗,强力遏止中原軍閥朱温南进步伐,成功避免全国更大范围动乱。经略淮南过程中,其政治方略、经济措施和军事思想,对五代十国及其后来社会产生深远影响。其奠基之吴国,初步实现由藩镇向王国转型,继而,南方割据势力与北方中原政权并存局面得以实现。

政治上,行密为後代的南唐奠定经济文化基础,开启唐宋之交政治整合和经济文化中心南渐先河,原因在于一仍吴旧的南唐是南方最为重要的割据政权,中国古代经济与文化中心的初步南移实际上是在以南唐为龙头、以吴越和马楚等政权为呼应的统治区域内实现的,这个时期是唐宋之交社会分野的标点,为后来社会的强劲发展提供了前瞻、新鲜的要素。杨行密经略江淮,实为十国第一人。


\subsubsection{天复}

\begin{longtable}{|>{\centering\scriptsize}m{2em}|>{\centering\scriptsize}m{1.3em}|>{\centering}m{8.8em}|}
  % \caption{秦王政}\
  \toprule
  \SimHei \normalsize 年数 & \SimHei \scriptsize 公元 & \SimHei 大事件 \tabularnewline
  % \midrule
  \endfirsthead
  \toprule
  \SimHei \normalsize 年数 & \SimHei \scriptsize 公元 & \SimHei 大事件 \tabularnewline
  \midrule
  \endhead
  \midrule
  元年 & 902 & \tabularnewline\hline
  二年 & 903 & \tabularnewline\hline
  三年 & 904 & \tabularnewline
  \bottomrule
\end{longtable}

\subsubsection{天祐}

\begin{longtable}{|>{\centering\scriptsize}m{2em}|>{\centering\scriptsize}m{1.3em}|>{\centering}m{8.8em}|}
  % \caption{秦王政}\
  \toprule
  \SimHei \normalsize 年数 & \SimHei \scriptsize 公元 & \SimHei 大事件 \tabularnewline
  % \midrule
  \endfirsthead
  \toprule
  \SimHei \normalsize 年数 & \SimHei \scriptsize 公元 & \SimHei 大事件 \tabularnewline
  \midrule
  \endhead
  \midrule
  元年 & 904 & \tabularnewline\hline
  二年 & 905 & \tabularnewline
  \bottomrule
\end{longtable}



%%% Local Variables:
%%% mode: latex
%%% TeX-engine: xetex
%%% TeX-master: "../../Main"
%%% End:
