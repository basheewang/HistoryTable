%% -*- coding: utf-8 -*-
%% Time-stamp: <Chen Wang: 2019-12-24 17:42:35>


\section{吴\tiny(902-937)}

\subsection{简介}

吴(902年-937年)是五代时十国之一,为杨行密所建,又称杨吴、南吴、弘農、淮南。

唐昭宗景福元年(892年)杨行密为唐淮南节度使,据扬州。天复二年(902年)封为吴王。建都广陵(即扬州),称江都府。杨行密注意招集拒絕朱溫統治的唐人,奖励农桑,使江淮一带社会经济有所恢复。

吴最强盛时,有今江苏、安徽、江西和湖北等省的一部分。唐哀帝天祐二年(905年)杨行密去世,其子杨渥继位,仅称弘农郡王,在唐朝灭亡后不承认后梁,仍用天祐年号,兼并镇南军,但不久遭牙将张颢、徐温夺权杀害,不久徐温杀张颢,以摄政身份掌握吴国大权。天祐七年(910年),弘农郡王杨隆演复称吴王,919年,称吴国王,建年号,以示脱离唐朝体系,以徐温为大丞相。徐温死后,大权落入其养子宰相和继任摄政徐知誥之手。顺义七年(927年),吴国王楊溥称帝,其实是为徐知诰篡位称帝做准备。吴天祚三年(937年),徐知诰杀死图谋反抗的历阳郡公杨濛,迫使杨溥禅让,建立南唐。吴国共历4主,36年,但大部分时间杨氏受徐氏控制。

楊吳的统治地区包括今江苏、安徽、江西、湖北等一部分。

%% -*- coding: utf-8 -*-
%% Time-stamp: <Chen Wang: 2021-11-01 15:38:27>

\subsection{武帝杨行密\tiny(902-905)}

\subsubsection{生平}

吳孝武王杨行密(852年-905年),字化源,原名行愍,庐州合肥(今安徽合肥长丰)人,唐朝末年著名政治家、军事家,五代十国時期吴国政權奠定者。唐乾宁二年(895年)进同中书门下平章事、弘农郡王。天复二年(902)进中书令、封吴王,天佑二年(905)病死,唐谥武忠王,吴国武义年间改谥孝武王,其子杨溥稱帝时,追尊其为武皇帝,庙号太祖。

杨行密原为庐州牙将,中和三年(883)拜庐州刺史,归淮南节度使高骈。886年,高骈賜名為杨行密。

中和五年(885)毕师铎反高骈,召宣歙观察使秦彦助战,高骈向行密求救,行密尚未赶到,毕师铎已俘高骈,行密一到,大败毕师铎,秦彥一气之下,杀死高骈,行密占领扬州,毕师铎投奔秦宗权部下孙儒,孙儒杀毕师铎并吞并其军队,发兵围攻扬州,欲一举消灭杨行密,将扬州收归己有。

行密采谋士袁袭建议,放弃扬州,先退守庐州(今安徽合肥),后攻克宣州(今安徽宣城)。龙纪元年(889)拜宣州观察使。行密據有宣州後,趁势向东、南、西三个方向发展,占领苏州、常州、润州(今江苏镇江)、滁州、和州(今安徽和县)等地,势力急剧扩大,领地包括了现在的安徽、江苏、浙江和江西、湖北等省部分地区。

景福元年(892)取楚州(今江苏淮安)、杨行密的发展,使占有扬州的孙儒受到三面包围,孙儒杀向宣州。行密击溃孙儒,并当众將之斬殺。复入扬州,进淮南节度使。

此后行密又出兵扩大地盘,将淮河以南和长江以东大片领土都纳入自己势力范围,为后来楊吳疆土基本上定型。乾宁二年(895)进同中书门下平章事、弘农郡王。

行密为扩大势力范围,乘蔡州四面行营都统朱温忙于对兖州、郓州(治今山东郓城)用兵之机,主动出击作战。攻取濠州(治今安徽凤阳)、寿州(治今安徽寿县),袭占涟水。又通过对依附于朱温的州县主动出击作战,拓展地盘,阻遏朱温插足淮南,为稳定和发展自己势力创造条件。

乾宁四年(897)宣武节度使朱温大举南侵,行密亲战,先集中自己的精锐主力攻击东边庞师古部,命朱瑾掘开淮河河堤,用水大淹庞师古部,同时遣朱瑾、张训领兵击败庞师古部于清口(今江苏淮安),宣武军损失惨重,大败而归,庞师古阵亡,葛从周逃回。朱温此后即无力南下,此后数十年间,南北遂成分裂之局。

吴越王钱鏐派兵攻打行密,兵進苏州。行密命周本禦敵,卻作战失利,失苏州。行密经过充分准备,派李神福进攻钱鏐,于杭州大败钱鏐军队并活捉其大将顾全武。经过长期混战,行密在江淮一带立足。

天复二年(902)进中书令、封吴王。天复三年(903年),行密遣李神福击破武昌节度使杜洪于君山(今湖南岳阳)。朱温败于行密后,向东进攻王师范,王师范求救于行密。行密于是年四月遣王茂章领兵出征,六月王茂章击破朱温军队,朱温再不敢对江淮用兵。八月至十二月,宁国节度使田頵与润州团练使安仁义起兵反行密,行密令李神福、臺濛、王茂章等将击溃田頵部于芜湖,广德、黄池(今安徽当涂)、宣州(治今安徽宣城)等地,田頵亡于宣州战场。令王茂章击破安仁义于润州并斩于广陵(今江苏扬州)。后行密诈瞎诛杀朱延寿叛逆势力。

天佑二年(905)十一月庚辰(二十六)日(12月24日)吴王杨行密病逝,其子杨渥继立。唐朝谥武忠王,吴国武义年间改谥孝武王,杨溥即帝位时追尊其为武皇帝,庙号太祖。

行密为政颇能选拔贤才,招集流散,轻徭薄赋,劝课农桑,使江淮一带社会经济在战争的间隙有较大恢复。

大唐衰变后,藩鎮割據,諸侯并起。行密在江淮地区举起割据大旗,强力遏止中原軍閥朱温南进步伐,成功避免全国更大范围动乱。经略淮南过程中,其政治方略、经济措施和军事思想,对五代十国及其后来社会产生深远影响。其奠基之吴国,初步实现由藩镇向王国转型,继而,南方割据势力与北方中原政权并存局面得以实现。

政治上,行密为後代的南唐奠定经济文化基础,开启唐宋之交政治整合和经济文化中心南渐先河,原因在于一仍吴旧的南唐是南方最为重要的割据政权,中国古代经济与文化中心的初步南移实际上是在以南唐为龙头、以吴越和马楚等政权为呼应的统治区域内实现的,这个时期是唐宋之交社会分野的标点,为后来社会的强劲发展提供了前瞻、新鲜的要素。杨行密经略江淮,实为十国第一人。


\subsubsection{天复}

\begin{longtable}{|>{\centering\scriptsize}m{2em}|>{\centering\scriptsize}m{1.3em}|>{\centering}m{8.8em}|}
  % \caption{秦王政}\
  \toprule
  \SimHei \normalsize 年数 & \SimHei \scriptsize 公元 & \SimHei 大事件 \tabularnewline
  % \midrule
  \endfirsthead
  \toprule
  \SimHei \normalsize 年数 & \SimHei \scriptsize 公元 & \SimHei 大事件 \tabularnewline
  \midrule
  \endhead
  \midrule
  元年 & 902 & \tabularnewline\hline
  二年 & 903 & \tabularnewline\hline
  三年 & 904 & \tabularnewline
  \bottomrule
\end{longtable}

\subsubsection{天祐}

\begin{longtable}{|>{\centering\scriptsize}m{2em}|>{\centering\scriptsize}m{1.3em}|>{\centering}m{8.8em}|}
  % \caption{秦王政}\
  \toprule
  \SimHei \normalsize 年数 & \SimHei \scriptsize 公元 & \SimHei 大事件 \tabularnewline
  % \midrule
  \endfirsthead
  \toprule
  \SimHei \normalsize 年数 & \SimHei \scriptsize 公元 & \SimHei 大事件 \tabularnewline
  \midrule
  \endhead
  \midrule
  元年 & 904 & \tabularnewline\hline
  二年 & 905 & \tabularnewline
  \bottomrule
\end{longtable}



%%% Local Variables:
%%% mode: latex
%%% TeX-engine: xetex
%%% TeX-master: "../../Main"
%%% End:

%% -*- coding: utf-8 -*-
%% Time-stamp: <Chen Wang: 2019-12-24 17:47:48>

\subsection{杨渥\tiny(905-908)}

\subsubsection{生平}

楊渥(886年-908年6月9日),字奉天(《九国志》作承天),五代十國時期南吳君主,但從未稱「吳王」,南吳太祖楊行密長子。

楊行密在位時,任牙內諸軍使,楊行密晚年病重後被任命為宣州觀察使。杨行密临终欲召回杨渥以传位,节度判官周隐认为杨渥不务正业且好酒,反对其继业,建议杨行密将领地托管给庐州刺史刘威,等杨行密诸子年长后刘威自然会还政。但在左牙(衙)指揮使張顥、右牙(衙)指揮使徐溫等劝说下,杨行密仍然决定传位杨渥。徐温和幕僚严可求得知周隐没有发出召回杨渥的牒文,夺取牒文将其发出,杨渥回到扬州军部。

唐哀帝天祐二年(905年),楊行密過世,楊渥嗣位,為宣諭使李儼承制授為淮南節度使、東南諸道行營都統、兼侍中、弘農郡王。

楊渥喜好遊玩作樂,居丧当中燃十围之烛以击毬,一烛费钱数万。又常单骑出游,左右莫知所之。天祐三年(906年)杨渥任命的西南行营都招讨使秦裴吞并镇南军,杨渥愈发骄傲,杀死周隐,致使将佐不自安。张颢、徐温屢勸,杨渥不聽,说:“你们认为我不才,为什么不杀了我,自己坐我的位子!”其親信又不斷欺壓元勳舊臣,將領們頗感不安。楊渥为修建毬场,将扬州牙城中的亲军悉数迁出,張顥、徐溫二人因此无所忌惮。他们让杨渥从宣州带来的指挥使朱思勍、范思从、陈璠帮助秦裴平定镇南军,又诬陷三将谋反,派别将陈祐前去秦裴帐中处死三人。杨渥因而想杀死张颢、徐温,但天祐四年(907年),張顥、徐溫抢先發動兵變,露刃入宫,以铁挝击杀楊渥亲信数十人。此后诸将与張顥、徐溫意见不同者,辄被杀,二人遂控制軍政。楊渥大權盡失。

天祐五年(908年)五月戊寅,張顥、徐溫遣亲信纪祥、陈晖、黎璠、孙殷等人入子城,弑楊渥于寝室。杨渥说:“你们要是杀了张颢、徐温,我让你们做刺史。”很多刺客都被说动,但纪祥仍将杨渥缢死。杨渥终年二十三岁,张颢、徐温对外声称暴卒。死後諡威王(弘農威王);楊隆演登南吳國王位時,改諡景王(南吳景王),廟號烈祖;楊溥登南吳帝位時,再改諡景皇帝(南吳景帝)。楊渥雖被認為是南吳君主之一,惟其在位時尚未稱吳王。

杨溥还封兄子南昌公杨珙为建安王。杨溥的三个哥哥中,二哥杨隆演仅有一子见于《十国春秋》,三哥杨濛尚在人世且一并被封为常山王。故杨珙很可能是杨渥之子。杨溥禅位后,杨珙降为公。

其弟楊溥即皇帝位時,追尊楊渥為烈祖景皇帝,陵墓号绍陵。绍陵地望不详,应在其父杨行密墓附近。


\subsubsection{天祐}

\begin{longtable}{|>{\centering\scriptsize}m{2em}|>{\centering\scriptsize}m{1.3em}|>{\centering}m{8.8em}|}
  % \caption{秦王政}\
  \toprule
  \SimHei \normalsize 年数 & \SimHei \scriptsize 公元 & \SimHei 大事件 \tabularnewline
  % \midrule
  \endfirsthead
  \toprule
  \SimHei \normalsize 年数 & \SimHei \scriptsize 公元 & \SimHei 大事件 \tabularnewline
  \midrule
  \endhead
  \midrule
  元年 & 905 & \tabularnewline\hline
  二年 & 906 & \tabularnewline\hline
  三年 & 907 & \tabularnewline\hline
  四年 & 908 & \tabularnewline
  \bottomrule
\end{longtable}


%%% Local Variables:
%%% mode: latex
%%% TeX-engine: xetex
%%% TeX-master: "../../Main"
%%% End:

%% -*- coding: utf-8 -*-
%% Time-stamp: <Chen Wang: 2021-11-01 15:39:25>

\subsection{宣帝楊隆演\tiny(908-920)}

\subsubsection{生平}

吳宣王楊隆演(897年-920年),字鴻源,原名楊瀛,又名楊渭,中国五代時期南吳君主,孝武王次子,楊渥之弟。

天祐五年(908年),弘農王楊渥為張顥、徐溫所殺。張顥欲自立,而徐温力主立楊隆演。徐溫尋殺張顥,因此專權。雖楊隆演不久為宣諭使李儼承制授為淮南節度使、東南諸道行營都統、同平章事、弘農郡王,然而大權仍掌握在徐溫之手。天祐七年(910年),再為岐王李茂貞承制加中書令,並繼承楊行密吳王之位。

楊隆演個性穩重恭順,对于徐溫父子專權不會顯露出不平之色,因此徐溫也很放心。但因大权旁落,杨隆演建立吳國後並不快樂。徐温长子徐知训骄横恣肆,常侮弄杨隆演。在看戏时一时兴起要杨隆演和他一起演戏,自己演参军,让杨隆演扮作他的僮奴,扎着小辫子,穿着破衣服拿着帽子跟在后面。徐知训又与杨隆演泛舟于河,杨隆演比他先登岸,他就用弹子打杨隆演,被杨隆演随卒挡下才未中。一次在禅智寺赏花喝酒时,徐知训借酒意谩骂杨隆演,其悖慢之状竟将杨隆演吓哭。徐知训还因追赶杨隆演不及,就打死杨隆演的亲吏。

徐知训的种种所为,其父徐温都不知道。副都统朱瑾设计杀死徐知训,提首入宫见杨隆演,杨隆演不但没有振作,反而连称与自己无关,朱瑾最终被徐温部下逼死,徐温养子徐知诰代徐知训执掌杨吴国政。於是杨隆演放縱自己飲酒,而很少吃東西,因此生病臥床。

天祐十六年(919年),徐温奉楊隆演即吳國國王位,改元武義,建宗庙社稷,置百官如天子之制。南吳自是斷絕與唐朝的法統关系。徐温受封为大丞相、都督中外诸军事、诸道都统、镇海宁国节度使、守太尉、兼中书令、东海郡王。其养子徐知诰为左仆射、参知政事、同平章事、领江州观察使、奉化军节度使。

南吳武義二年(920年)五月,楊隆演疾寝,临终时召大丞相徐温入宫,试探其意称“蜀先主谓武侯‘嗣子不才,君宜自取。’”徐温正色称“吾果有意取之,当在诛張顥之初,岂至今日!使杨氏无男,有女亦当立之。敢妄言者斩!”楊隆演去世,諡宣王,徐温因杨隆演三弟杨濛年长且与自己不和,迎其四弟丹阳公楊溥繼位。杨溥后称帝,改諡杨隆演为宣皇帝,廟號高祖。

其弟楊溥即皇帝位時,追尊楊隆演為高祖宣皇帝,陵墓号肃陵。肃陵地望不详,应在其父杨行密墓附近。


\subsubsection{天祐}

\begin{longtable}{|>{\centering\scriptsize}m{2em}|>{\centering\scriptsize}m{1.3em}|>{\centering}m{8.8em}|}
  % \caption{秦王政}\
  \toprule
  \SimHei \normalsize 年数 & \SimHei \scriptsize 公元 & \SimHei 大事件 \tabularnewline
  % \midrule
  \endfirsthead
  \toprule
  \SimHei \normalsize 年数 & \SimHei \scriptsize 公元 & \SimHei 大事件 \tabularnewline
  \midrule
  \endhead
  \midrule
  元年 & 908 & \tabularnewline\hline
  二年 & 909 & \tabularnewline\hline
  三年 & 910 & \tabularnewline\hline
  四年 & 911 & \tabularnewline\hline
  五年 & 912 & \tabularnewline\hline
  六年 & 913 & \tabularnewline\hline
  七年 & 914 & \tabularnewline\hline
  八年 & 915 & \tabularnewline\hline
  九年 & 916 & \tabularnewline\hline
  十年 & 917 & \tabularnewline\hline
  十一年 & 918 & \tabularnewline\hline
  十二年 & 919 & \tabularnewline
  \bottomrule
\end{longtable}

\subsubsection{武义}

\begin{longtable}{|>{\centering\scriptsize}m{2em}|>{\centering\scriptsize}m{1.3em}|>{\centering}m{8.8em}|}
  % \caption{秦王政}\
  \toprule
  \SimHei \normalsize 年数 & \SimHei \scriptsize 公元 & \SimHei 大事件 \tabularnewline
  % \midrule
  \endfirsthead
  \toprule
  \SimHei \normalsize 年数 & \SimHei \scriptsize 公元 & \SimHei 大事件 \tabularnewline
  \midrule
  \endhead
  \midrule
  元年 & 919 & \tabularnewline\hline
  二年 & 920 & \tabularnewline\hline
  三年 & 921 & \tabularnewline
  \bottomrule
\end{longtable}


%%% Local Variables:
%%% mode: latex
%%% TeX-engine: xetex
%%% TeX-master: "../../Main"
%%% End:

%% -*- coding: utf-8 -*-
%% Time-stamp: <Chen Wang: 2021-11-01 15:39:32>

\subsection{睿帝楊溥\tiny(920-937)}

\subsubsection{生平}

吳睿帝楊溥(900年-938年),五代時期南吳君主,楊行密四子,母王氏。楊渥、楊隆演之弟,南吳唯一正式稱帝的君主(先前僅稱王)。

杨隆演称吴国王时,封杨溥为丹阳郡公。吳武義二年(920年)楊隆演去世,因其三弟杨濛年长且不为权臣徐温所喜,楊溥為徐溫所迎繼吳國王位,明年(921年),改元順義。順義七年(927年),即皇帝位,改年號乾貞。乾貞三年(929年)改元大和。大和七年(935年),再改元天祚。

南吳於楊隆演及楊溥在位時,軍政大權皆操之在徐溫、徐知誥父子之中,之所以即國王位、帝位,只是為徐氏父子篡位稱帝之準備而已。

天祚元年,楊溥加中书令徐知誥为尚父、太师、大丞相、天下兵马大元帅,进封齐王,以昇州、润州、宣州、池州、歙州、常州、江州、饶州、信州、海州为齐国。徐知誥置百官,以金陵府为西都。

天祚三年(937年)正月,徐知誥建齐国,立宗庙、社稷,改金陵府为江宁府,子城称宫城,厅堂曰殿,册王妃为王后,世子为王太子,太妃为王太后。置左右丞相、百官如天子之制。当年十月乙酉,楊溥讓位予徐知誥,南吳亡。

楊溥被徐知誥上尊號為高尚思玄弘古讓皇帝,安置于江都宫殿居住,其宗庙、正朔、乘舆、服御、均从吴国旧制,宫殿名称则从道教仙经中取名。楊溥在宫中多穿羽衣,习辟谷之术。

南唐昇元二年(938年),徐知誥(改名李昪)改润州牙城为丹杨宫,迁楊溥于其中,以严兵守护之。当年十一月辛丑,有使者来丹杨宫,楊溥方颂佛经于楼上,使者趋前,楊溥以香炉掷之,俄而去世,终年三十八岁。李昪废朝二十七日,追諡楊溥為睿皇帝。

昇元六年,南唐听宋齐丘之谋尽迁杨吴宗室于泰州,号“永宁宫”,守卫甚严,不使与外人通婚,久而男女自为婚配。后周显德三年,周世宗征淮南,下诏安抚杨氏子孙。南唐元宗李璟遣园苑使尹廷范将杨氏宗族迁置京口。尹廷范杀楊溥二弟及男口六十余人,携妇女渡江。李璟怒曰“小人以不义之名累我”,下令腰斩尹廷范于市。后来宋齐丘也失势被逼自杀,临死感叹这是自己献计幽禁杨溥一族的报应。


\subsubsection{顺义}

\begin{longtable}{|>{\centering\scriptsize}m{2em}|>{\centering\scriptsize}m{1.3em}|>{\centering}m{8.8em}|}
  % \caption{秦王政}\
  \toprule
  \SimHei \normalsize 年数 & \SimHei \scriptsize 公元 & \SimHei 大事件 \tabularnewline
  % \midrule
  \endfirsthead
  \toprule
  \SimHei \normalsize 年数 & \SimHei \scriptsize 公元 & \SimHei 大事件 \tabularnewline
  \midrule
  \endhead
  \midrule
  元年 & 921 & \tabularnewline\hline
  二年 & 922 & \tabularnewline\hline
  三年 & 923 & \tabularnewline\hline
  四年 & 924 & \tabularnewline\hline
  五年 & 925 & \tabularnewline\hline
  六年 & 926 & \tabularnewline\hline
  七年 & 927 & \tabularnewline
  \bottomrule
\end{longtable}

\subsubsection{乾贞}

\begin{longtable}{|>{\centering\scriptsize}m{2em}|>{\centering\scriptsize}m{1.3em}|>{\centering}m{8.8em}|}
  % \caption{秦王政}\
  \toprule
  \SimHei \normalsize 年数 & \SimHei \scriptsize 公元 & \SimHei 大事件 \tabularnewline
  % \midrule
  \endfirsthead
  \toprule
  \SimHei \normalsize 年数 & \SimHei \scriptsize 公元 & \SimHei 大事件 \tabularnewline
  \midrule
  \endhead
  \midrule
  元年 & 927 & \tabularnewline\hline
  二年 & 928 & \tabularnewline\hline
  三年 & 929 & \tabularnewline
  \bottomrule
\end{longtable}

\subsubsection{大和}

\begin{longtable}{|>{\centering\scriptsize}m{2em}|>{\centering\scriptsize}m{1.3em}|>{\centering}m{8.8em}|}
  % \caption{秦王政}\
  \toprule
  \SimHei \normalsize 年数 & \SimHei \scriptsize 公元 & \SimHei 大事件 \tabularnewline
  % \midrule
  \endfirsthead
  \toprule
  \SimHei \normalsize 年数 & \SimHei \scriptsize 公元 & \SimHei 大事件 \tabularnewline
  \midrule
  \endhead
  \midrule
  元年 & 929 & \tabularnewline\hline
  二年 & 930 & \tabularnewline\hline
  三年 & 931 & \tabularnewline\hline
  四年 & 932 & \tabularnewline\hline
  五年 & 933 & \tabularnewline\hline
  六年 & 934 & \tabularnewline\hline
  七年 & 935 & \tabularnewline
  \bottomrule
\end{longtable}

\subsubsection{天祚}

\begin{longtable}{|>{\centering\scriptsize}m{2em}|>{\centering\scriptsize}m{1.3em}|>{\centering}m{8.8em}|}
  % \caption{秦王政}\
  \toprule
  \SimHei \normalsize 年数 & \SimHei \scriptsize 公元 & \SimHei 大事件 \tabularnewline
  % \midrule
  \endfirsthead
  \toprule
  \SimHei \normalsize 年数 & \SimHei \scriptsize 公元 & \SimHei 大事件 \tabularnewline
  \midrule
  \endhead
  \midrule
  元年 & 935 & \tabularnewline\hline
  二年 & 936 & \tabularnewline\hline
  三年 & 937 & \tabularnewline
  \bottomrule
\end{longtable}


%%% Local Variables:
%%% mode: latex
%%% TeX-engine: xetex
%%% TeX-master: "../../Main"
%%% End:



%%% Local Variables:
%%% mode: latex
%%% TeX-engine: xetex
%%% TeX-master: "../../Main"
%%% End:
