%% -*- coding: utf-8 -*-
%% Time-stamp: <Chen Wang: 2021-11-01 15:39:25>

\subsection{宣帝楊隆演\tiny(908-920)}

\subsubsection{生平}

吳宣王楊隆演(897年-920年),字鴻源,原名楊瀛,又名楊渭,中国五代時期南吳君主,孝武王次子,楊渥之弟。

天祐五年(908年),弘農王楊渥為張顥、徐溫所殺。張顥欲自立,而徐温力主立楊隆演。徐溫尋殺張顥,因此專權。雖楊隆演不久為宣諭使李儼承制授為淮南節度使、東南諸道行營都統、同平章事、弘農郡王,然而大權仍掌握在徐溫之手。天祐七年(910年),再為岐王李茂貞承制加中書令,並繼承楊行密吳王之位。

楊隆演個性穩重恭順,对于徐溫父子專權不會顯露出不平之色,因此徐溫也很放心。但因大权旁落,杨隆演建立吳國後並不快樂。徐温长子徐知训骄横恣肆,常侮弄杨隆演。在看戏时一时兴起要杨隆演和他一起演戏,自己演参军,让杨隆演扮作他的僮奴,扎着小辫子,穿着破衣服拿着帽子跟在后面。徐知训又与杨隆演泛舟于河,杨隆演比他先登岸,他就用弹子打杨隆演,被杨隆演随卒挡下才未中。一次在禅智寺赏花喝酒时,徐知训借酒意谩骂杨隆演,其悖慢之状竟将杨隆演吓哭。徐知训还因追赶杨隆演不及,就打死杨隆演的亲吏。

徐知训的种种所为,其父徐温都不知道。副都统朱瑾设计杀死徐知训,提首入宫见杨隆演,杨隆演不但没有振作,反而连称与自己无关,朱瑾最终被徐温部下逼死,徐温养子徐知诰代徐知训执掌杨吴国政。於是杨隆演放縱自己飲酒,而很少吃東西,因此生病臥床。

天祐十六年(919年),徐温奉楊隆演即吳國國王位,改元武義,建宗庙社稷,置百官如天子之制。南吳自是斷絕與唐朝的法統关系。徐温受封为大丞相、都督中外诸军事、诸道都统、镇海宁国节度使、守太尉、兼中书令、东海郡王。其养子徐知诰为左仆射、参知政事、同平章事、领江州观察使、奉化军节度使。

南吳武義二年(920年)五月,楊隆演疾寝,临终时召大丞相徐温入宫,试探其意称“蜀先主谓武侯‘嗣子不才,君宜自取。’”徐温正色称“吾果有意取之,当在诛張顥之初,岂至今日!使杨氏无男,有女亦当立之。敢妄言者斩!”楊隆演去世,諡宣王,徐温因杨隆演三弟杨濛年长且与自己不和,迎其四弟丹阳公楊溥繼位。杨溥后称帝,改諡杨隆演为宣皇帝,廟號高祖。

其弟楊溥即皇帝位時,追尊楊隆演為高祖宣皇帝,陵墓号肃陵。肃陵地望不详,应在其父杨行密墓附近。


\subsubsection{天祐}

\begin{longtable}{|>{\centering\scriptsize}m{2em}|>{\centering\scriptsize}m{1.3em}|>{\centering}m{8.8em}|}
  % \caption{秦王政}\
  \toprule
  \SimHei \normalsize 年数 & \SimHei \scriptsize 公元 & \SimHei 大事件 \tabularnewline
  % \midrule
  \endfirsthead
  \toprule
  \SimHei \normalsize 年数 & \SimHei \scriptsize 公元 & \SimHei 大事件 \tabularnewline
  \midrule
  \endhead
  \midrule
  元年 & 908 & \tabularnewline\hline
  二年 & 909 & \tabularnewline\hline
  三年 & 910 & \tabularnewline\hline
  四年 & 911 & \tabularnewline\hline
  五年 & 912 & \tabularnewline\hline
  六年 & 913 & \tabularnewline\hline
  七年 & 914 & \tabularnewline\hline
  八年 & 915 & \tabularnewline\hline
  九年 & 916 & \tabularnewline\hline
  十年 & 917 & \tabularnewline\hline
  十一年 & 918 & \tabularnewline\hline
  十二年 & 919 & \tabularnewline
  \bottomrule
\end{longtable}

\subsubsection{武义}

\begin{longtable}{|>{\centering\scriptsize}m{2em}|>{\centering\scriptsize}m{1.3em}|>{\centering}m{8.8em}|}
  % \caption{秦王政}\
  \toprule
  \SimHei \normalsize 年数 & \SimHei \scriptsize 公元 & \SimHei 大事件 \tabularnewline
  % \midrule
  \endfirsthead
  \toprule
  \SimHei \normalsize 年数 & \SimHei \scriptsize 公元 & \SimHei 大事件 \tabularnewline
  \midrule
  \endhead
  \midrule
  元年 & 919 & \tabularnewline\hline
  二年 & 920 & \tabularnewline\hline
  三年 & 921 & \tabularnewline
  \bottomrule
\end{longtable}


%%% Local Variables:
%%% mode: latex
%%% TeX-engine: xetex
%%% TeX-master: "../../Main"
%%% End:
