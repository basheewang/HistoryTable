%% -*- coding: utf-8 -*-
%% Time-stamp: <Chen Wang: 2021-11-01 15:38:52>

\subsection{景帝杨渥\tiny(905-908)}

\subsubsection{生平}

楊渥(886年-908年6月9日),字奉天(《九国志》作承天),五代十國時期南吳君主,但從未稱「吳王」,南吳太祖楊行密長子。

楊行密在位時,任牙內諸軍使,楊行密晚年病重後被任命為宣州觀察使。杨行密临终欲召回杨渥以传位,节度判官周隐认为杨渥不务正业且好酒,反对其继业,建议杨行密将领地托管给庐州刺史刘威,等杨行密诸子年长后刘威自然会还政。但在左牙(衙)指揮使張顥、右牙(衙)指揮使徐溫等劝说下,杨行密仍然决定传位杨渥。徐温和幕僚严可求得知周隐没有发出召回杨渥的牒文,夺取牒文将其发出,杨渥回到扬州军部。

唐哀帝天祐二年(905年),楊行密過世,楊渥嗣位,為宣諭使李儼承制授為淮南節度使、東南諸道行營都統、兼侍中、弘農郡王。

楊渥喜好遊玩作樂,居丧当中燃十围之烛以击毬,一烛费钱数万。又常单骑出游,左右莫知所之。天祐三年(906年)杨渥任命的西南行营都招讨使秦裴吞并镇南军,杨渥愈发骄傲,杀死周隐,致使将佐不自安。张颢、徐温屢勸,杨渥不聽,说:“你们认为我不才,为什么不杀了我,自己坐我的位子!”其親信又不斷欺壓元勳舊臣,將領們頗感不安。楊渥为修建毬场,将扬州牙城中的亲军悉数迁出,張顥、徐溫二人因此无所忌惮。他们让杨渥从宣州带来的指挥使朱思勍、范思从、陈璠帮助秦裴平定镇南军,又诬陷三将谋反,派别将陈祐前去秦裴帐中处死三人。杨渥因而想杀死张颢、徐温,但天祐四年(907年),張顥、徐溫抢先發動兵變,露刃入宫,以铁挝击杀楊渥亲信数十人。此后诸将与張顥、徐溫意见不同者,辄被杀,二人遂控制軍政。楊渥大權盡失。

天祐五年(908年)五月戊寅,張顥、徐溫遣亲信纪祥、陈晖、黎璠、孙殷等人入子城,弑楊渥于寝室。杨渥说:“你们要是杀了张颢、徐温,我让你们做刺史。”很多刺客都被说动,但纪祥仍将杨渥缢死。杨渥终年二十三岁,张颢、徐温对外声称暴卒。死後諡威王(弘農威王);楊隆演登南吳國王位時,改諡景王(南吳景王),廟號烈祖;楊溥登南吳帝位時,再改諡景皇帝(南吳景帝)。楊渥雖被認為是南吳君主之一,惟其在位時尚未稱吳王。

杨溥还封兄子南昌公杨珙为建安王。杨溥的三个哥哥中,二哥杨隆演仅有一子见于《十国春秋》,三哥杨濛尚在人世且一并被封为常山王。故杨珙很可能是杨渥之子。杨溥禅位后,杨珙降为公。

其弟楊溥即皇帝位時,追尊楊渥為烈祖景皇帝,陵墓号绍陵。绍陵地望不详,应在其父杨行密墓附近。


\subsubsection{天祐}

\begin{longtable}{|>{\centering\scriptsize}m{2em}|>{\centering\scriptsize}m{1.3em}|>{\centering}m{8.8em}|}
  % \caption{秦王政}\
  \toprule
  \SimHei \normalsize 年数 & \SimHei \scriptsize 公元 & \SimHei 大事件 \tabularnewline
  % \midrule
  \endfirsthead
  \toprule
  \SimHei \normalsize 年数 & \SimHei \scriptsize 公元 & \SimHei 大事件 \tabularnewline
  \midrule
  \endhead
  \midrule
  元年 & 905 & \tabularnewline\hline
  二年 & 906 & \tabularnewline\hline
  三年 & 907 & \tabularnewline\hline
  四年 & 908 & \tabularnewline
  \bottomrule
\end{longtable}


%%% Local Variables:
%%% mode: latex
%%% TeX-engine: xetex
%%% TeX-master: "../../Main"
%%% End:
