%% -*- coding: utf-8 -*-
%% Time-stamp: <Chen Wang: 2019-12-24 17:48:52>

\subsection{睿帝\tiny(920-937)}

\subsubsection{生平}

吳睿帝楊溥(900年-938年),五代時期南吳君主,楊行密四子,母王氏。楊渥、楊隆演之弟,南吳唯一正式稱帝的君主(先前僅稱王)。

杨隆演称吴国王时,封杨溥为丹阳郡公。吳武義二年(920年)楊隆演去世,因其三弟杨濛年长且不为权臣徐温所喜,楊溥為徐溫所迎繼吳國王位,明年(921年),改元順義。順義七年(927年),即皇帝位,改年號乾貞。乾貞三年(929年)改元大和。大和七年(935年),再改元天祚。

南吳於楊隆演及楊溥在位時,軍政大權皆操之在徐溫、徐知誥父子之中,之所以即國王位、帝位,只是為徐氏父子篡位稱帝之準備而已。

天祚元年,楊溥加中书令徐知誥为尚父、太师、大丞相、天下兵马大元帅,进封齐王,以昇州、润州、宣州、池州、歙州、常州、江州、饶州、信州、海州为齐国。徐知誥置百官,以金陵府为西都。

天祚三年(937年)正月,徐知誥建齐国,立宗庙、社稷,改金陵府为江宁府,子城称宫城,厅堂曰殿,册王妃为王后,世子为王太子,太妃为王太后。置左右丞相、百官如天子之制。当年十月乙酉,楊溥讓位予徐知誥,南吳亡。

楊溥被徐知誥上尊號為高尚思玄弘古讓皇帝,安置于江都宫殿居住,其宗庙、正朔、乘舆、服御、均从吴国旧制,宫殿名称则从道教仙经中取名。楊溥在宫中多穿羽衣,习辟谷之术。

南唐昇元二年(938年),徐知誥(改名李昪)改润州牙城为丹杨宫,迁楊溥于其中,以严兵守护之。当年十一月辛丑,有使者来丹杨宫,楊溥方颂佛经于楼上,使者趋前,楊溥以香炉掷之,俄而去世,终年三十八岁。李昪废朝二十七日,追諡楊溥為睿皇帝。

昇元六年,南唐听宋齐丘之谋尽迁杨吴宗室于泰州,号“永宁宫”,守卫甚严,不使与外人通婚,久而男女自为婚配。后周显德三年,周世宗征淮南,下诏安抚杨氏子孙。南唐元宗李璟遣园苑使尹廷范将杨氏宗族迁置京口。尹廷范杀楊溥二弟及男口六十余人,携妇女渡江。李璟怒曰“小人以不义之名累我”,下令腰斩尹廷范于市。后来宋齐丘也失势被逼自杀,临死感叹这是自己献计幽禁杨溥一族的报应。


\subsubsection{顺义}

\begin{longtable}{|>{\centering\scriptsize}m{2em}|>{\centering\scriptsize}m{1.3em}|>{\centering}m{8.8em}|}
  % \caption{秦王政}\
  \toprule
  \SimHei \normalsize 年数 & \SimHei \scriptsize 公元 & \SimHei 大事件 \tabularnewline
  % \midrule
  \endfirsthead
  \toprule
  \SimHei \normalsize 年数 & \SimHei \scriptsize 公元 & \SimHei 大事件 \tabularnewline
  \midrule
  \endhead
  \midrule
  元年 & 921 & \tabularnewline\hline
  二年 & 922 & \tabularnewline\hline
  三年 & 923 & \tabularnewline\hline
  四年 & 924 & \tabularnewline\hline
  五年 & 925 & \tabularnewline\hline
  六年 & 926 & \tabularnewline\hline
  七年 & 927 & \tabularnewline
  \bottomrule
\end{longtable}

\subsubsection{乾贞}

\begin{longtable}{|>{\centering\scriptsize}m{2em}|>{\centering\scriptsize}m{1.3em}|>{\centering}m{8.8em}|}
  % \caption{秦王政}\
  \toprule
  \SimHei \normalsize 年数 & \SimHei \scriptsize 公元 & \SimHei 大事件 \tabularnewline
  % \midrule
  \endfirsthead
  \toprule
  \SimHei \normalsize 年数 & \SimHei \scriptsize 公元 & \SimHei 大事件 \tabularnewline
  \midrule
  \endhead
  \midrule
  元年 & 927 & \tabularnewline\hline
  二年 & 928 & \tabularnewline\hline
  三年 & 929 & \tabularnewline
  \bottomrule
\end{longtable}

\subsubsection{大和}

\begin{longtable}{|>{\centering\scriptsize}m{2em}|>{\centering\scriptsize}m{1.3em}|>{\centering}m{8.8em}|}
  % \caption{秦王政}\
  \toprule
  \SimHei \normalsize 年数 & \SimHei \scriptsize 公元 & \SimHei 大事件 \tabularnewline
  % \midrule
  \endfirsthead
  \toprule
  \SimHei \normalsize 年数 & \SimHei \scriptsize 公元 & \SimHei 大事件 \tabularnewline
  \midrule
  \endhead
  \midrule
  元年 & 929 & \tabularnewline\hline
  二年 & 930 & \tabularnewline\hline
  三年 & 931 & \tabularnewline\hline
  四年 & 932 & \tabularnewline\hline
  五年 & 933 & \tabularnewline\hline
  六年 & 934 & \tabularnewline\hline
  七年 & 935 & \tabularnewline
  \bottomrule
\end{longtable}

\subsubsection{天祚}

\begin{longtable}{|>{\centering\scriptsize}m{2em}|>{\centering\scriptsize}m{1.3em}|>{\centering}m{8.8em}|}
  % \caption{秦王政}\
  \toprule
  \SimHei \normalsize 年数 & \SimHei \scriptsize 公元 & \SimHei 大事件 \tabularnewline
  % \midrule
  \endfirsthead
  \toprule
  \SimHei \normalsize 年数 & \SimHei \scriptsize 公元 & \SimHei 大事件 \tabularnewline
  \midrule
  \endhead
  \midrule
  元年 & 935 & \tabularnewline\hline
  二年 & 936 & \tabularnewline\hline
  三年 & 937 & \tabularnewline
  \bottomrule
\end{longtable}


%%% Local Variables:
%%% mode: latex
%%% TeX-engine: xetex
%%% TeX-master: "../../Main"
%%% End:
