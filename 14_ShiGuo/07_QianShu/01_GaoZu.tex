%% -*- coding: utf-8 -*-
%% Time-stamp: <Chen Wang: 2021-11-01 15:46:42>

\subsection{高祖王建\tiny(907-918)}

\subsubsection{生平}

前蜀高祖王建(847年2月26日-918年7月11日),字光圖,五代十國时期前蜀開國皇帝(907年—918年在位),许州舞阳(今河南舞阳)人。

少年时为无赖,以屠牛驴和贩私盐为业,乡里称为“贼王八”,黄巢起事时期投效唐朝军队,隶属忠武军。长安沦陷时他奋不顾身地护驾,號為“隨駕五都”,为忠武八都的都将之一,被唐僖宗封为西川节度使、壁州刺史,十軍觀軍容使田令孜也收他為養子。僖宗還長安後,升為御林軍宿衛將領。光啟二年(886年),僖宗又逃往興元(今陝西漢中),任命王建為“清道使”,以后他向四方发展势力。

大顺二年(891年)以精兵二千奔往成都,為陳敬瑄所阻,王建攻破鹿頭關,取漢州,攻彭州,大敗陳敬瑄五萬兵,不久攻占成都,陳敬瑄與田令孜開門出降,据西川,杀陈敬瑄、田令孜,接著又降黔南節度使王建肇,殺東川節度使顧彥暉、降武定節度使拓拔思敬。897年占有东川梓(今四川三台)、渝(今重庆)诸州,遂有有兩川兼三峽之地。902年取得山南西道控制權。天復三年(903年),唐昭宗又封他为蜀王,遂成为当时最大的割据势力。次年,朱温挟持唐昭宗迁都洛阳,改元天祐,王建不承认,继续使用天復年号。

唐哀帝天祐四年(907年)唐亡,后王建因不服后梁而自立为皇帝,国号“大蜀”,史称“前蜀”,定都成都,当年沿用唐朝天復年号,908年建年号“武成”。在位12年。在位时期,励精图治,注重农桑,兴修水利,扩张疆土,实行“与民休息”的政策,蜀中大治。死后谥号神武圣文孝德明惠皇帝,庙号高祖,葬于成都的永陵(今成都市西延线永陵路)。

\subsubsection{天复}

\begin{longtable}{|>{\centering\scriptsize}m{2em}|>{\centering\scriptsize}m{1.3em}|>{\centering}m{8.8em}|}
  % \caption{秦王政}\
  \toprule
  \SimHei \normalsize 年数 & \SimHei \scriptsize 公元 & \SimHei 大事件 \tabularnewline
  % \midrule
  \endfirsthead
  \toprule
  \SimHei \normalsize 年数 & \SimHei \scriptsize 公元 & \SimHei 大事件 \tabularnewline
  \midrule
  \endhead
  \midrule
  元年 & 907 & \tabularnewline
  \bottomrule
\end{longtable}

\subsubsection{武成}

\begin{longtable}{|>{\centering\scriptsize}m{2em}|>{\centering\scriptsize}m{1.3em}|>{\centering}m{8.8em}|}
  % \caption{秦王政}\
  \toprule
  \SimHei \normalsize 年数 & \SimHei \scriptsize 公元 & \SimHei 大事件 \tabularnewline
  % \midrule
  \endfirsthead
  \toprule
  \SimHei \normalsize 年数 & \SimHei \scriptsize 公元 & \SimHei 大事件 \tabularnewline
  \midrule
  \endhead
  \midrule
  元年 & 908 & \tabularnewline\hline
  二年 & 909 & \tabularnewline\hline
  三年 & 910 & \tabularnewline
  \bottomrule
\end{longtable}

\subsubsection{通正}

\begin{longtable}{|>{\centering\scriptsize}m{2em}|>{\centering\scriptsize}m{1.3em}|>{\centering}m{8.8em}|}
  % \caption{秦王政}\
  \toprule
  \SimHei \normalsize 年数 & \SimHei \scriptsize 公元 & \SimHei 大事件 \tabularnewline
  % \midrule
  \endfirsthead
  \toprule
  \SimHei \normalsize 年数 & \SimHei \scriptsize 公元 & \SimHei 大事件 \tabularnewline
  \midrule
  \endhead
  \midrule
  元年 & 916 & \tabularnewline
  \bottomrule
\end{longtable}


\subsubsection{天汉}

\begin{longtable}{|>{\centering\scriptsize}m{2em}|>{\centering\scriptsize}m{1.3em}|>{\centering}m{8.8em}|}
  % \caption{秦王政}\
  \toprule
  \SimHei \normalsize 年数 & \SimHei \scriptsize 公元 & \SimHei 大事件 \tabularnewline
  % \midrule
  \endfirsthead
  \toprule
  \SimHei \normalsize 年数 & \SimHei \scriptsize 公元 & \SimHei 大事件 \tabularnewline
  \midrule
  \endhead
  \midrule
  元年 & 917 & \tabularnewline
  \bottomrule
\end{longtable}


\subsubsection{光天}

\begin{longtable}{|>{\centering\scriptsize}m{2em}|>{\centering\scriptsize}m{1.3em}|>{\centering}m{8.8em}|}
  % \caption{秦王政}\
  \toprule
  \SimHei \normalsize 年数 & \SimHei \scriptsize 公元 & \SimHei 大事件 \tabularnewline
  % \midrule
  \endfirsthead
  \toprule
  \SimHei \normalsize 年数 & \SimHei \scriptsize 公元 & \SimHei 大事件 \tabularnewline
  \midrule
  \endhead
  \midrule
  元年 & 918 & \tabularnewline
  \bottomrule
\end{longtable}


%%% Local Variables:
%%% mode: latex
%%% TeX-engine: xetex
%%% TeX-master: "../../Main"
%%% End:
