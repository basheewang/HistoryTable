%% -*- coding: utf-8 -*-
%% Time-stamp: <Chen Wang: 2019-12-26 10:04:32>


\section{前蜀\tiny(903-925)}

\subsection{简介}

前蜀(907年—925年),是中国五代十国时期由王建建立的政权,十国之一。前蜀疆域辽阔,东控荆襄,南通南诏,西达维州(今四川理县),北过秦州(今甘肃天水),占领了今天四川、湖北、陕西以及甘肃大部,重庆、贵州全部以及云南部分地区,方圆数千里。

前蜀是唐朝的“蜀王”、西川节度使王建在成都建立的,早在891年,王建就开始统辖全川。903年,唐昭宗封王建为蜀王。唐哀帝天祐四年(907年),王建不服后梁统治,建国号“蜀”,史称“前蜀”,定都成都。917年,行刘备在成都称汉故事,改国号为汉,次年又恢复国号蜀。

王氏父子共统治两川35年。前蜀初年,王建励精图治,开拓疆土,兴修水利,注重农桑,实行“与民休息”的政策。在没有战争的情况下,在拥有沃地千里、丰饶五谷的成都平原的情况下,前蜀的经济、文化、军事大大的发展,成为了当时的一个强国。可是王建死后,继承人王衍奢侈无度,残暴昏庸,后唐趁机伐蜀,蜀军溃败,成都沦陷,前蜀灭亡。

今日成都前蜀永陵(王建墓)規模頗可觀。

前蜀建國後,典章制度主要由宰相韋莊制定。韋莊熟習唐代制度,因此前蜀的政制充滿唐代遺風。前蜀宰相是同平章事,大多由中書侍郎門下侍郎兼領。唐代以中書令為宰相,地位崇高,前蜀的中書令則多以宗室兼任,並非專員。

前蜀樞密使亦是機要官職,自韋莊之後掌握朝廷大權。唐代內樞密使以宦官擔任,前蜀起初則以士人擔任,後來擔心將領不受控制,也起用宦官擔任樞密使,朝政漸壞。此外,大學士亦參與朝政。

宮廷中,內飛龍使一職掌握禁軍,往往干預朝政。地方上掌握兵權的,有節度使、團練使、觀察使等,屬下有判官和掌書記。地方官制中,州有刺史,下有參軍;縣有縣令,下有主簿,大體上和唐代相同。


%% -*- coding: utf-8 -*-
%% Time-stamp: <Chen Wang: 2019-12-26 10:05:05>

\subsection{高祖\tiny(907-918)}

\subsubsection{生平}

前蜀高祖王建(847年2月26日-918年7月11日),字光圖,五代十國时期前蜀開國皇帝(907年—918年在位),许州舞阳(今河南舞阳)人。

少年时为无赖,以屠牛驴和贩私盐为业,乡里称为“贼王八”,黄巢起事时期投效唐朝军队,隶属忠武军。长安沦陷时他奋不顾身地护驾,號為“隨駕五都”,为忠武八都的都将之一,被唐僖宗封为西川节度使、壁州刺史,十軍觀軍容使田令孜也收他為養子。僖宗還長安後,升為御林軍宿衛將領。光啟二年(886年),僖宗又逃往興元(今陝西漢中),任命王建為“清道使”,以后他向四方发展势力。

大顺二年(891年)以精兵二千奔往成都,為陳敬瑄所阻,王建攻破鹿頭關,取漢州,攻彭州,大敗陳敬瑄五萬兵,不久攻占成都,陳敬瑄與田令孜開門出降,据西川,杀陈敬瑄、田令孜,接著又降黔南節度使王建肇,殺東川節度使顧彥暉、降武定節度使拓拔思敬。897年占有东川梓(今四川三台)、渝(今重庆)诸州,遂有有兩川兼三峽之地。902年取得山南西道控制權。天復三年(903年),唐昭宗又封他为蜀王,遂成为当时最大的割据势力。次年,朱温挟持唐昭宗迁都洛阳,改元天祐,王建不承认,继续使用天復年号。

唐哀帝天祐四年(907年)唐亡,后王建因不服后梁而自立为皇帝,国号“大蜀”,史称“前蜀”,定都成都,当年沿用唐朝天復年号,908年建年号“武成”。在位12年。在位时期,励精图治,注重农桑,兴修水利,扩张疆土,实行“与民休息”的政策,蜀中大治。死后谥号神武圣文孝德明惠皇帝,庙号高祖,葬于成都的永陵(今成都市西延线永陵路)。

\subsubsection{天复}

\begin{longtable}{|>{\centering\scriptsize}m{2em}|>{\centering\scriptsize}m{1.3em}|>{\centering}m{8.8em}|}
  % \caption{秦王政}\
  \toprule
  \SimHei \normalsize 年数 & \SimHei \scriptsize 公元 & \SimHei 大事件 \tabularnewline
  % \midrule
  \endfirsthead
  \toprule
  \SimHei \normalsize 年数 & \SimHei \scriptsize 公元 & \SimHei 大事件 \tabularnewline
  \midrule
  \endhead
  \midrule
  元年 & 907 & \tabularnewline
  \bottomrule
\end{longtable}

\subsubsection{武成}

\begin{longtable}{|>{\centering\scriptsize}m{2em}|>{\centering\scriptsize}m{1.3em}|>{\centering}m{8.8em}|}
  % \caption{秦王政}\
  \toprule
  \SimHei \normalsize 年数 & \SimHei \scriptsize 公元 & \SimHei 大事件 \tabularnewline
  % \midrule
  \endfirsthead
  \toprule
  \SimHei \normalsize 年数 & \SimHei \scriptsize 公元 & \SimHei 大事件 \tabularnewline
  \midrule
  \endhead
  \midrule
  元年 & 908 & \tabularnewline\hline
  二年 & 909 & \tabularnewline\hline
  三年 & 910 & \tabularnewline
  \bottomrule
\end{longtable}

\subsubsection{通正}

\begin{longtable}{|>{\centering\scriptsize}m{2em}|>{\centering\scriptsize}m{1.3em}|>{\centering}m{8.8em}|}
  % \caption{秦王政}\
  \toprule
  \SimHei \normalsize 年数 & \SimHei \scriptsize 公元 & \SimHei 大事件 \tabularnewline
  % \midrule
  \endfirsthead
  \toprule
  \SimHei \normalsize 年数 & \SimHei \scriptsize 公元 & \SimHei 大事件 \tabularnewline
  \midrule
  \endhead
  \midrule
  元年 & 916 & \tabularnewline
  \bottomrule
\end{longtable}


\subsubsection{天汉}

\begin{longtable}{|>{\centering\scriptsize}m{2em}|>{\centering\scriptsize}m{1.3em}|>{\centering}m{8.8em}|}
  % \caption{秦王政}\
  \toprule
  \SimHei \normalsize 年数 & \SimHei \scriptsize 公元 & \SimHei 大事件 \tabularnewline
  % \midrule
  \endfirsthead
  \toprule
  \SimHei \normalsize 年数 & \SimHei \scriptsize 公元 & \SimHei 大事件 \tabularnewline
  \midrule
  \endhead
  \midrule
  元年 & 917 & \tabularnewline
  \bottomrule
\end{longtable}


\subsubsection{光天}

\begin{longtable}{|>{\centering\scriptsize}m{2em}|>{\centering\scriptsize}m{1.3em}|>{\centering}m{8.8em}|}
  % \caption{秦王政}\
  \toprule
  \SimHei \normalsize 年数 & \SimHei \scriptsize 公元 & \SimHei 大事件 \tabularnewline
  % \midrule
  \endfirsthead
  \toprule
  \SimHei \normalsize 年数 & \SimHei \scriptsize 公元 & \SimHei 大事件 \tabularnewline
  \midrule
  \endhead
  \midrule
  元年 & 918 & \tabularnewline
  \bottomrule
\end{longtable}


%%% Local Variables:
%%% mode: latex
%%% TeX-engine: xetex
%%% TeX-master: "../../Main"
%%% End:

%% -*- coding: utf-8 -*-
%% Time-stamp: <Chen Wang: 2021-11-01 15:47:03>

\subsection{后主王衍\tiny(918-925)}

\subsubsection{生平}

王衍(901年8月31日-926年5月18日),本名王宗衍,字化源,前蜀末代皇帝(第二代,918年—925年在位),在位7年,史稱「後主」。前蜀灭亡后,入后唐。在押送途中全族被殺。

王衍是蜀高祖王建第11子,也是幼子,母親是徐賢妃。当初,王建因长子卫王王宗仁有病而立次子王宗懿(又名王元膺)为皇太子,但王宗懿因与王建宠臣唐道袭冲突而发动兵变,杀死唐道袭,自己也被杀。王建本意从长得像自己的三子豳王王宗辂和诸子中最贤的八子信王王宗杰中择嗣,但在徐贤妃和宰相张格等人经营下,功臣王宗侃等误以为王建中意王宗衍,都请求立王宗衍为太子,而王建也因此误以为王宗衍得众心,虽然认为他幼懦,怀疑他是否堪任,仍然立他为太子。918年,王建死,他继承了皇位,是为蜀后主。

王衍是一个十分荒淫腐朽的皇帝,他迷恋美色,宦官王承休「多以邪僻姦穢之事媚其主,主愈寵之」,对北方小朝廷后梁、后唐的攻击不闻不问,前蜀国势一天不如一天。终于在咸康元年(925年),唐庄宗派遣大军进攻前蜀,蜀军溃败,王衍养兄齐王王宗弼劫持王衍,迫使王衍举国投降,前蜀灭亡。

次年即926年,庄宗正欲对付邺都变兵时,伶人景进向莊宗进言说王衍是个祸害,应当设法翦除,莊宗便杀害了王衍及其亲族。但枢密使张居翰擅自将庄宗诛杀“王衍一行”的诏书改成诛杀“王衍一家”,使得跟随王衍的臣仆千余人得活。天成三年,在前蜀旧臣王宗寿等乞求下,後唐明宗李嗣源追封王衍順正公,以诸侯礼葬于长安南三赵屯。

\subsubsection{乾德}

\begin{longtable}{|>{\centering\scriptsize}m{2em}|>{\centering\scriptsize}m{1.3em}|>{\centering}m{8.8em}|}
  % \caption{秦王政}\
  \toprule
  \SimHei \normalsize 年数 & \SimHei \scriptsize 公元 & \SimHei 大事件 \tabularnewline
  % \midrule
  \endfirsthead
  \toprule
  \SimHei \normalsize 年数 & \SimHei \scriptsize 公元 & \SimHei 大事件 \tabularnewline
  \midrule
  \endhead
  \midrule
  元年 & 919 & \tabularnewline\hline
  二年 & 920 & \tabularnewline\hline
  三年 & 921 & \tabularnewline\hline
  四年 & 922 & \tabularnewline\hline
  五年 & 923 & \tabularnewline\hline
  六年 & 924 & \tabularnewline
  \bottomrule
\end{longtable}

\subsubsection{咸康}

\begin{longtable}{|>{\centering\scriptsize}m{2em}|>{\centering\scriptsize}m{1.3em}|>{\centering}m{8.8em}|}
  % \caption{秦王政}\
  \toprule
  \SimHei \normalsize 年数 & \SimHei \scriptsize 公元 & \SimHei 大事件 \tabularnewline
  % \midrule
  \endfirsthead
  \toprule
  \SimHei \normalsize 年数 & \SimHei \scriptsize 公元 & \SimHei 大事件 \tabularnewline
  \midrule
  \endhead
  \midrule
  元年 & 925 & \tabularnewline
  \bottomrule
\end{longtable}


%%% Local Variables:
%%% mode: latex
%%% TeX-engine: xetex
%%% TeX-master: "../../Main"
%%% End:



%%% Local Variables:
%%% mode: latex
%%% TeX-engine: xetex
%%% TeX-master: "../../Main"
%%% End:
