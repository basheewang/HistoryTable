%% -*- coding: utf-8 -*-
%% Time-stamp: <Chen Wang: 2019-12-26 09:54:02>

\subsection{惠宗\tiny(926-935)}

\subsubsection{生平}

閩惠宗王延鈞(閩東語:Uòng Iòng-gĭng,闽南语:Ông Iân-kun),(?-935年),繼位後更名王鏻(又作王璘),五代時期閩國第一代稱皇帝的君主,926年至935年在位。

王延鈞是太祖王審知的次子,也是嗣王王延翰之弟。王延翰繼位後,任王延鈞為泉州刺史,並在福州四處尋找民女,納入宮中為妾,王延鈞與建州刺史王延稟為此上書勸諫,王延翰大怒。王延稟與王延翰一向不睦,便與王延鈞聯軍攻打福州。王延稟自建州順流而下,於天成元年(926年)十二月初八(陽曆927年1月14日)先攻破福州,殺王延翰。不久王延鈞亦至,被王延稟推戴为武威留后。夏五月,後唐明宗任命王延鈞為威武軍節度使守中書令,封琅琊王。十一月,王延鈞遣使,向後唐進貢犀牛、香藥、海味。天成三年七月,後唐又遣吏部郎中裴羽、右散騎常侍陸崇,進封王延鈞為閩王。十二月,奉國節度使知建州王延稟向後唐上表稱疾,後唐任命其子王繼雄為建州刺史。

長興二年(931年),王延稟誤信王延鈞患重病,命次子王繼昇為建州留後,與王繼雄一起率水軍攻打福州。王延稟攻西門、王繼雄攻東門。王延鈞派樓船指揮使王仁達拒戰於南臺江。王仁達把士兵藏在船中,自己假裝出降。王繼雄登船撫慰,被王仁達殺死,梟首示眾於西門。王延稟正在放火攻城,見其子首級,不禁慟哭,軍心大亂。王仁達趁機發起攻擊,將其擒獲。王延鈞將王延稟斬於市,恢復其原名周彥琛,又派使者前去建州招撫其餘黨。王延稟之子王繼昇、王繼倫得知後出奔吳越。王延鈞便派弟弟都教練使王延政去鎮守建州。

當時福建僧侶眾多。王延鈞在位期間,下令丈量土地,分為三等,上等賜給僧侶,中等授予土著百姓,下等給流寓之人耕種。閩國的科舉之法模仿唐朝,但兩稅卻被加重。他喜好神仙之術,寵信左道巫者徐彥、朴盛韜、陳守元等人,還建造寶皇宮給陳守元居住,稱其為宮主。當時福州有王霸壇、煉丹井。壇旁皂莢木枯萎已久,一日突然長出枝葉來。井中又有白龜浮出。掘地,得石銘,有「王霸裔孫」之文。王延鈞便認為這在自己身上應驗了,便在壇旁建造宮殿,極盡奢華。

長興二年十二月,陳守元假借寶皇之命,建議王延鈞「避位受道,當為天子六十年。」於是王延鈞遜位給長子威武軍節度副使王繼鵬,成為道士,取道號玄錫。翌年春三月復位,要求後唐仿吴越钱镠、南楚马殷之例,封自己為尚書令。後唐不答,王延鈞遂斷絕雙方關係。

後唐長興四年(933年),黄龙现于真封宅,王延鈞下令改宅为龙跃宫,又建东华宫。当年正月,王延鈞在寶皇宮受册稱帝,國號大閩,改元龍啟,改名王鏻,立五庙,追谥王审知为太祖,封高盖山为西岳。王延鈞自知國土狹小,土地偏僻,因此謹慎與四鄰相處,境內還算安定。

王延鈞的元配是南漢清遠公主劉德秀,十分美麗,但早逝。繼室金氏賢而無寵。王延鈞相當寵愛淑妃陳金鳳,筑水晶宫于福州西湖旁,在湖中造彩舫数十,每舫置宫女二十余人,自乘大龙舟与陳金鳳同游。后又筑长春宫,与陳金鳳居于其中,每晚欢宴,燃金龙烛数百,使宫女擎金玉、玛瑙、琥珀、水晶之杯盘进馔,酒酣时裸体追逐嬉笑为乐。陳金鳳本是王審知的婢女,有才藝又十分淫蕩。因王延鈞晚年得风疾,陳金鳳遂與王延鈞的男宠歸守明、李可殷私通,閩人都很痛恨他們。

閩國永和元年(935年),陳金鳳被立為后。同年,王延鈞病重,王延鈞之子王繼鵬與皇城使李倣欲聯手提前了結陳后之勢力。李倣派兵進宮,殺死皇后陳金鳳及其黨羽。王延鈞躲到為他特製的九龍帳下,變軍刺了幾下後才出去。王延鈞重傷未死卻痛不欲生,宮女不忍見其受苦,遂殺死王延鈞。王延鈞死後,為王繼鵬諡為齊肅明孝皇帝,廟號惠宗,唯《新五代史》則作廟號太宗,諡號惠皇帝。

\subsubsection{天成}

\begin{longtable}{|>{\centering\scriptsize}m{2em}|>{\centering\scriptsize}m{1.3em}|>{\centering}m{8.8em}|}
  % \caption{秦王政}\
  \toprule
  \SimHei \normalsize 年数 & \SimHei \scriptsize 公元 & \SimHei 大事件 \tabularnewline
  % \midrule
  \endfirsthead
  \toprule
  \SimHei \normalsize 年数 & \SimHei \scriptsize 公元 & \SimHei 大事件 \tabularnewline
  \midrule
  \endhead
  \midrule
  元年 & 926 & \tabularnewline\hline
  二年 & 927 & \tabularnewline\hline
  三年 & 928 & \tabularnewline\hline
  四年 & 929 & \tabularnewline\hline
  五年 & 930 & \tabularnewline
  \bottomrule
\end{longtable}

\subsubsection{长兴}

\begin{longtable}{|>{\centering\scriptsize}m{2em}|>{\centering\scriptsize}m{1.3em}|>{\centering}m{8.8em}|}
  % \caption{秦王政}\
  \toprule
  \SimHei \normalsize 年数 & \SimHei \scriptsize 公元 & \SimHei 大事件 \tabularnewline
  % \midrule
  \endfirsthead
  \toprule
  \SimHei \normalsize 年数 & \SimHei \scriptsize 公元 & \SimHei 大事件 \tabularnewline
  \midrule
  \endhead
  \midrule
  元年 & 930 & \tabularnewline\hline
  二年 & 931 & \tabularnewline\hline
  三年 & 932 & \tabularnewline
  \bottomrule
\end{longtable}

\subsubsection{龙启}

\begin{longtable}{|>{\centering\scriptsize}m{2em}|>{\centering\scriptsize}m{1.3em}|>{\centering}m{8.8em}|}
  % \caption{秦王政}\
  \toprule
  \SimHei \normalsize 年数 & \SimHei \scriptsize 公元 & \SimHei 大事件 \tabularnewline
  % \midrule
  \endfirsthead
  \toprule
  \SimHei \normalsize 年数 & \SimHei \scriptsize 公元 & \SimHei 大事件 \tabularnewline
  \midrule
  \endhead
  \midrule
  元年 & 933 & \tabularnewline\hline
  二年 & 934 & \tabularnewline
  \bottomrule
\end{longtable}

\subsubsection{永和}

\begin{longtable}{|>{\centering\scriptsize}m{2em}|>{\centering\scriptsize}m{1.3em}|>{\centering}m{8.8em}|}
  % \caption{秦王政}\
  \toprule
  \SimHei \normalsize 年数 & \SimHei \scriptsize 公元 & \SimHei 大事件 \tabularnewline
  % \midrule
  \endfirsthead
  \toprule
  \SimHei \normalsize 年数 & \SimHei \scriptsize 公元 & \SimHei 大事件 \tabularnewline
  \midrule
  \endhead
  \midrule
  元年 & 935 & \tabularnewline\hline
  二年 & 936 & \tabularnewline
  \bottomrule
\end{longtable}


%%% Local Variables:
%%% mode: latex
%%% TeX-engine: xetex
%%% TeX-master: "../../Main"
%%% End:
