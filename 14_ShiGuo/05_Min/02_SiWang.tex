%% -*- coding: utf-8 -*-
%% Time-stamp: <Chen Wang: 2019-12-26 09:53:19>

\subsection{嗣王\tiny(926)}

\subsubsection{生平}

閩嗣王延翰(闽东语平話字:Uòng Iòng-hâng;?-927年),字子逸,五代時期閩國君主,王審知之長子。妻博陵郡夫人崔氏。

王審知在位時任威武軍節度副使。同光三年(925年),王審知去世,王延翰奉遺命繼立,權知軍府事,自稱威武留後,向後唐朝貢。汀州人陳本起兵,聚集三萬人圍攻汀州。王延翰派右軍都監柳邕等人,以二萬人前往討伐,將其斬殺。翌年春二月,後唐莊宗得知王延翰繼任之後,任命他為威武軍節度使。不久莊宗被弑,明宗繼位,改元天成,於夏五月,加其為同平章事。王延翰得知中原大亂,便有割據福建稱王之心。十月,王延翰取出《史記》,將其中的「東越列傳」的無諸一節翻出來給諸將吏看,並說:「閩,自古王國也。吾今不王,何待之有!」將吏們紛紛勸他割據自立,於是王延翰自稱大閩國王,立宮殿,置百官,威儀、文物皆擬天子制,羣下稱之曰殿下。又追諡王審知為昭武王,但仍奉後唐的正朔。

根據《十國春秋》的記載,王延翰「為人長大,美皙如玉,而好讀書、通經史」。但他個性驕傲荒淫,殘忍兇暴。《新五代史》中也記載了王延翰的荒淫,他在王審知喪服未除的時候便開始飲酒作樂。他於福州的西湖「築室十餘里,號曰水晶宮;每攜後庭游宴,從子城複道以出」。王延翰命人四處尋找民女,投入後宮作為自己的妾。其妻崔氏「陋而淫」,「延翰不能制」。這些被選入宮中的民女命運悲慘,被崔氏關押起來,「繫以大械,刻木為人手以擊頰,又以鐵錐刺之,一歲中死者八十四人」。後來崔氏大病一場,見者認為是被其害死者作祟,不久崔氏便死了。

王延翰看不起自己的兄弟,繼位才一個月,便將弟弟王延鈞貶為泉州刺史。建州刺史王延稟是王審知的養子,與王延翰一向不睦。當二人得知王延翰四處尋找民女之後,都上書勸阻,王延翰更加大怒。十二月,王延鈞與王延稟便聯手反叛,進軍福州。王延稟自建州順流而下,先至福州。福州指揮使陳陶率軍抵抗,兵敗自殺。天成元年十二月初八(陽曆927年1月14日)夜,王延稟率壯士百餘人,從西門架梯登城而入,攻取武庫,奪取兵器,直趨寢門。王延翰聞變,匿於別室,第二天早晨被王延稟擒獲。王延稟歷數其暴虐之罪,並聲稱他和崔氏一起害死了王審知,告諭百姓,斬於紫宸門之外。

王延鈞將王延翰葬於城北太平山,建太平地藏院,派丁守墓,稱為「王墓」。其地位於今福州市晉安區新店鎮,今日被福州人稱為「黃墓」。閩東語「黃」與「王」同音,因此被訛作「黃墓」。

\subsubsection{天成}

\begin{longtable}{|>{\centering\scriptsize}m{2em}|>{\centering\scriptsize}m{1.3em}|>{\centering}m{8.8em}|}
  % \caption{秦王政}\
  \toprule
  \SimHei \normalsize 年数 & \SimHei \scriptsize 公元 & \SimHei 大事件 \tabularnewline
  % \midrule
  \endfirsthead
  \toprule
  \SimHei \normalsize 年数 & \SimHei \scriptsize 公元 & \SimHei 大事件 \tabularnewline
  \midrule
  \endhead
  \midrule
  元年 & 926 & \tabularnewline
  \bottomrule
\end{longtable}



%%% Local Variables:
%%% mode: latex
%%% TeX-engine: xetex
%%% TeX-master: "../../Main"
%%% End:
