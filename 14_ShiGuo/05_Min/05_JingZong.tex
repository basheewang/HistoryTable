%% -*- coding: utf-8 -*-
%% Time-stamp: <Chen Wang: 2019-12-26 09:54:58>

\subsection{景宗\tiny(939-944)}

\subsubsection{生平}

闽景宗王延羲(福州話平话字:Uòng Iòng-hĭ;?-944年),继位後改名王曦,五代時期閩國君主,王審知之子,王延翰、王延鈞之弟,王繼鵬之叔。

王延羲於王繼鵬在位時任左僕射、同平章事,因王繼鵬猜忌宗室,遂裝瘋賣傻,因此被軟禁自宅。閩國通文四年(939年)拱宸、控鶴軍使朱文進、連重遇反,迎王延羲進宮,並殺王繼鵬,王延羲遂自稱威武節度使、閩國王,更名王曦,改元永隆,稱臣於後晉,但在國內官制就如同皇帝一樣。

然而王延羲繼位後,驕傲奢侈,荒淫無度,猜忌宗族,比王繼鵬有過之而無不及,其弟建州刺史王延政多有規勸,王延羲反而回信怒罵,又差人探聽王延政的隱私,二人因此結怨。永隆二年(940年)王延羲攻建州,開啟了閩國內戰,二人於數年爭戰中互有勝負。永隆三年(941年)七月,王延羲自稱大闽皇、威武军節度使。十月,再晋尊为大闽皇帝,加尊号睿明文廣武聖光德隆道大孝皇帝。王延政随后称帝,国号殷,闽国分裂。

由於王延羲個性一向暴虐,而朱文進、連重遇自從殺了王繼鵬後,就一直擔心為人所害,二人因此認為王延羲有加害之意,永隆六年(944年),連朱二人先下手為強,王延羲被刺殺。王延羲死後被諡為睿文廣武明聖元德隆道大孝皇帝,廟號景宗。

\subsubsection{永隆}

\begin{longtable}{|>{\centering\scriptsize}m{2em}|>{\centering\scriptsize}m{1.3em}|>{\centering}m{8.8em}|}
  % \caption{秦王政}\
  \toprule
  \SimHei \normalsize 年数 & \SimHei \scriptsize 公元 & \SimHei 大事件 \tabularnewline
  % \midrule
  \endfirsthead
  \toprule
  \SimHei \normalsize 年数 & \SimHei \scriptsize 公元 & \SimHei 大事件 \tabularnewline
  \midrule
  \endhead
  \midrule
  元年 & 939 & \tabularnewline\hline
  二年 & 940 & \tabularnewline\hline
  三年 & 941 & \tabularnewline\hline
  四年 & 942 & \tabularnewline\hline
  五年 & 943 & \tabularnewline\hline
  六年 & 944 & \tabularnewline
  \bottomrule
\end{longtable}


%%% Local Variables:
%%% mode: latex
%%% TeX-engine: xetex
%%% TeX-master: "../../Main"
%%% End:
