%% -*- coding: utf-8 -*-
%% Time-stamp: <Chen Wang: 2021-11-01 15:45:49>

\subsection{福恭懿王王延政\tiny(943-945)}

\subsubsection{朱文进生平}

朱文進,永泰(今福建永泰)人,五代時期閩國君主。閩帝王繼鵬在位時任拱宸軍使。

「拱宸都」與「控鶴都」原來都是閩太祖王審知的親軍,閩康宗王繼鵬即位後建立自己的親軍名喚「宸衛都」,而待之比拱宸、控鶴二都更厚,二都迭有怨言。朱文進並與控鶴軍使連重遇曾被王繼鵬三番四次的侮辱,二人因此十分不滿。

閩國通文四年(939年),北宮失火,連重遇奉派率軍清理火場殘餘的灰燼,工作勞苦,士卒怨懟。而連重遇又被王繼鵬懷疑參與縱火,因此率軍叛變,迎立王繼鵬之叔王延羲為帝,並殺害王繼鵬。朱文進在這次政變後,被任命為拱宸都指揮使。

朱文進與連重遇自從殺了王繼鵬後,就一直擔心為人所害,而王延羲個性一向暴虐,二人因此認為王延羲有加害之意,閩國永隆六年(944年),朱、連二人先發制人,刺殺王延羲,朱文進並被連重遇推舉,自稱閩主,殺害境內王姓皇族成員五十餘人,並放宮女出宮,停止興建中的工程,企圖與王延羲的暴政完全相反以拉攏人心。

不久,朱文進取消帝號自稱威武留後,向後晉稱臣,而後晉任命朱文進為威武節度使。後晉開運元年(944年)十二月十五日(陽曆為945年1月1日),朱文進正式被後晉出帝石重貴冊封為閩國王。

但在此時,朱、連二人的軍隊不斷被由將領留從效、陳洪進以及殷帝王延政所率領的討伐軍擊敗,情勢日漸窘迫,部下因此離心。後晉開運元年(944年)閏十二月二十九日(陽曆為945年2月14日)朱文進及連重遇被部屬林仁翰誅殺。

\subsubsection{王延政生平}

閩天德帝王延政(?-951年),五代時期閩國最後一位君主,也是殷国唯一君主,王審知的第十三子,人稱十三郎。王延翰、王延鈞、王延羲之弟,王延羲在位時任建州刺史。

由於王延羲繼位後,驕傲奢侈,荒淫無度,猜忌宗族,王延政因此多有規勸,然而王延羲反而回信怒罵,又差人探聽王延政的隱私,二人因此結怨。閩國永隆二年(940年)王延羲攻建州,開啟了閩國內戰,二人於數年爭戰中互有勝負。

永隆三年(941年)二人短暫休兵,王延政被王延羲封為富沙王,惟不久又重啟戰端。閩國永隆五年(943年),王延政自行於建州稱帝,國號殷,改元天德。王延政與王繼鵬、王延羲一樣橫徵暴斂,因此人民生活困苦。

閩國永隆六年(944年),朱文進、連重遇殺王延羲,朱文進自稱閩主。王延政遂出兵討伐,而朱文進、連重遇尋為部下所殺。天德三年(945年)諸臣請王延政還都福州並恢復閩國國號,惟當時南唐大軍已趁閩國內亂時壓境,只好任命姪兒王繼昌出鎮福州,並派黃仁諷協助王繼昌。但黃仁諷受李仁達慫恿,殺死王繼昌,延政聞之,族誅黃仁諷一家,並派張漢真領水軍討伐福州。不久,建州城陷,王延政投降,閩國亡。

王延政後來被送往南唐都城金陵,南唐帝李璟封他為羽林大將軍;南唐保大五年(947年)改封為鄱陽王;保大九年(951年),再改封為光山王,不久過世,被追贈為福王,諡號恭懿。

\subsubsection{天德}

\begin{longtable}{|>{\centering\scriptsize}m{2em}|>{\centering\scriptsize}m{1.3em}|>{\centering}m{8.8em}|}
  % \caption{秦王政}\
  \toprule
  \SimHei \normalsize 年数 & \SimHei \scriptsize 公元 & \SimHei 大事件 \tabularnewline
  % \midrule
  \endfirsthead
  \toprule
  \SimHei \normalsize 年数 & \SimHei \scriptsize 公元 & \SimHei 大事件 \tabularnewline
  \midrule
  \endhead
  \midrule
  元年 & 943 & \tabularnewline\hline
  二年 & 944 & \tabularnewline\hline
  三年 & 945 & \tabularnewline
  \bottomrule
\end{longtable}


%%% Local Variables:
%%% mode: latex
%%% TeX-engine: xetex
%%% TeX-master: "../../Main"
%%% End:
