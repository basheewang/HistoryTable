%% -*- coding: utf-8 -*-
%% Time-stamp: <Chen Wang: 2019-12-26 09:50:52>


\section{闽\tiny(909-945)}

\subsection{简介}

闽(閩東語:Mìng;閩南語:Bân;909年-945年),五代十国的十國之一,由閩太祖王審知於909年時所建立。933年,閩惠宗王廷鈞稱帝,定國號大閩,是繼閩越國後福建第二次獨立於中原政權。943年,富沙王王延政在建州自立反抗景宗王延羲,並一度將國號改為大殷。閩國內亂造成最後被南唐所滅。閩國前後共歷經6位君王的統治,享國37年;而若從王潮攻佔福州當上福建節度使開始算起,王閩皇族統治福建共長達55年。

唐昭宗景福二年(893年)王潮、王審邽、王審知兄弟攻占福州,并逐渐据有福建全地。王潮被唐朝廷授职为福建观察使,不久升为威武军节度使。乾宁四年十二月(陽曆為898年1月)王潮卒,遺命以王審知繼位。

审知受封为琅琊王,后梁太祖开平三年(909年),王审知受后梁封为闽王,都长乐(今福建省福州市)。王审知称臣后梁,交好邻国,提倡节俭,减轻稅金、勞役,以保境息民为立国方针,他还建立学校,奖励通商。在他在位期间,闽地的经济、文化都得以迅速发展。

925年王审知死后,其子王延翰即位,926年被兄弟王延禀、王延钧杀害。

926年,王延钧继任闽王。后唐长兴二年(931年)王延钧殺王延稟。

后唐长兴四年(933年)王延钧称帝,建都长乐(今福建福州),国号闽,年号龙启。

935年,王延钧在政变中为其子王继鹏所弑。继鹏继位,改名王昶。

939年,王昶被连重遇所弑,重遇立王曦(原名王延羲)为闽王。王曦在位时猜忌其弟建州刺史王延政,二人结怨,导致940年王曦进攻建州,开始了闽国内乱。941年,王曦先后称大闽皇和大闽皇帝。943年,王延政于建州称帝,改国号为殷,年号天德。

944年,朱文进、连重遇杀王曦,朱文进自立为闽王,王延政出兵讨伐,945年朱文进、连重遇被部下所杀。王延政恢复国号为“闽”。同年,南唐进攻闽国,王延政战败,闽国灭亡。

946年,闽国旧将留从效驱逐了南唐在泉州、漳州的驻军,但仍向南唐稱臣,留及后继者占有泉、漳二州直至北宋建国之后。

西元881年(中和元年),王審潮、王審邽與王審知三兄弟加入王緒的麾下,王審潮擔任軍正。885年,王審潮推翻王緒,成為軍隊的領導人。886年(光啟二年),王審潮攻克泉州,畏時任福建觀察使的陳巖威名,向其投降,陈岩上表授予王審潮泉州刺史的官職。893年,陳巖死後,王審潮命令三弟王審知與堂弟王彥復率軍攻佔福州,並逐漸控制了整個福建,因而受唐朝政府封為福建觀察使,不久又晉升威武軍節度使。898年,王審潮去世,遺命跳過二弟王審邽與其四位兒子,將節度使之職傳給三弟王審知。王審知繼承節度使後,又受唐昭宗敕封為琅琊王,不久朱溫篡位建立後梁,唐朝滅亡。

西元909年,後梁太祖敕封王審知為「閩王」,閩國正式開國,建都長樂府(今福州市)。王審知鑒於開國初期應當休息養民,因而實行黄老治術的統治方針;政治上稱臣於中原的後梁政權,並與鄰近國家政權交好、聯姻,而經濟民生上則是提倡節儉,減輕稅金、勞役。另外他還建立學校,獎勵通商,使得閩國的經濟、文化在他在位時期得以迅速發展。西元925年,王審知薨,廟號太祖,葬宣陵,諡昭武孝皇帝。

西元925年,嫡長子王延翰在太祖薨後登基,隔年自稱「大閩國王」,但仍舊稱臣於後唐中原政權。不久,後唐莊宗遭弒,中原政權內部陷入混亂,王延翰推崇閩越王騶無諸的建國事蹟,並以繼承閩越國為由建國稱王,但國家的年號依然使用後唐的天成[註 1]。王延翰個性荒淫無道、殘忍凶暴,經常為了貪圖美色而收括民女;泉州刺史王延鈞與建州刺史王延禀遂以此為藉口謀反弒君,殺死了兄長王延翰。

西元945年,王延政投降南唐後,閩國滅亡。946年,割據福州的李仁達派其弟李弘通進攻泉州,閩南當地將領留從效趁此機會罷黜了身為閩國皇族的泉州刺史王繼勳,改由自己出任,次年更佔領漳州。不久将南唐军队赶出泉漳二州。949年,南唐中主李璟不得不任命割據泉州的留從效為清源軍節度使,最後晉封晉江王,為閩南真正的統治者。南唐歸降北宋後,留從效亦請求歸屬,宋朝朝廷同意之,但清源軍後來經過留紹鎡、張漢思與陳洪進等節度使的統馭,持續割據閩南泉漳直至978年奉表正式獻出泉、漳二州,前後共歷經33年。


%% -*- coding: utf-8 -*-
%% Time-stamp: <Chen Wang: 2019-12-26 09:52:35>

\subsection{太祖\tiny(909-925)}

\subsubsection{生平}

閩太祖王審知(閩東語:Uòng Sīng-dĭ;闽南语:Ông Sím-ti;862年-925年12月30日),表字信通,一字详卿,庙号太祖(閩東語:Mìng Tái-cū;閩南語:Bân Thài-chó͘),是五代十國時期閩國開國國王,909年至925年在位。

淮南道光州固始(今河南省固始縣)人,為王恁第三子,也是王潮與王審邽之弟。出身貧苦,後在唐末民變期間,與兩位兄長一起加入王緒的軍隊,隨之轉戰福建。其兄王潮被唐昭宗任命為福建觀察使後,他也獲封觀察副使,後福建觀察使升格為威武軍節度使,898年,繼承兄位。後梁篡唐後,後梁太祖於909年冊封王審知為閩王,正式建立閩國。

王審知出身貧苦,故能節儉自處,統治福建期間省刑惜費,輕徭薄賦,與民休息,儘量避免戰爭,並與中原王朝保持朝貢關係。另一方面,他注重教育,吸納中原逃離戰亂的人才,又積極發展海外貿易,使福建的經濟和文化得到很大發展。

同光三年十二月辛未(925年12月30日),王審知在福州逝世,其長子王延翰繼位。後唐得知後賜諡忠懿。王延翰稱大閩國王之後,諡他為昭武王。王延鈞稱大閩皇帝後,再改諡號昭武孝皇帝,廟號太祖,陵號為宣陵。

因王審知三兄弟對福建發展貢獻很大,福建人尊稱王審知為「開閩尊王」、「開閩聖王」或「忠惠尊王」;尊其長兄王潮為「威武尊王」「广武尊王」、次兄王審邽為「泉安尊王」,視為鄉土神明供奉,合稱開閩三王。946年(開運三年),南唐佔領福州之後,為紀念王審知的德政,將其府邸改建為閩王祠,對他進行祭祀,即今日福州市鼓樓區的忠懿閩王祠。

據《十国春秋》記載:王審知是秦朝名将王翦,东晋王导的后代,為瑯琊王氏士族。其五代祖名王曄,為固始令。因「民愛其仁」,被當地百姓挽留,最終定居於固始。父親王恁,為當地一個富有的農民。王恁生王潮(又名王審潮)、王審邽、王審知三個兄弟。

唐僖宗在位期間,蜀地盜賊起兵叛亂。壽州(今安徽壽縣)的屠夫王緒與妹夫劉行全也起兵響應,佔據江淮一帶,自稱將軍,不久攻取光州。當時王審知的長兄王潮是固始縣長史,王潮、王審邽和王審知三兄弟以才氣知名,邑人號曰「三龍」。王緒將他們擄走後,委任王潮為軍正,主管糧草之事,得到部下的擁護。三兄弟皆受重用。根據《十國春秋》的說法,王審知「身長七尺六寸,紫色方口隆準,常乘白馬,軍中號白馬三郎,所居處恒有紫氣幕其上」,時人認為這是大貴之相。

汝南節度使秦宗權封王緒為光州刺史,要求王緒率兵一起討伐黃巢,王緒按兵不動。於是在中和五年(885年),秦宗權率兵攻打王緒,王緒率部眾逃往福建,攻汀州、漳州。

王緒軍中缺糧,下令不准部將攜帶老人孺子,違者斬首。當時王審知三兄弟奉母親董氏一起行軍,王緒命其棄母,三兄弟請求與母同死,王緒只能赦免。當時有方士對王緒預言,軍中會出現推翻他的人,因此王緒對諸將多疑猜忌,身材魁梧、才能雄傑者多被他找藉口斬殺。王潮日夜憂患,遂說服一位前鋒將軍發動兵變,將王緒擒獲捆綁之,王緒不堪其辱,自殺身亡。王潮聲稱要推戴這位將領為主,遭到推辭,但這位將軍取了一把寶劍插在地上,權作神位,讓諸將輪流跪拜,向上天祈禱,希望神明能夠指點迷津,相傳當輪到王審知跪拜的時候,寶劍從地上躍起。眾人便認為是神諭,推戴王審知為主帥。王審知將主帥之位讓給了兄長王潮,自己擔任副帥。

光啓元年八月,泉州的張延魯等人聲稱泉州刺史廖彥若貪婪殘暴,聽說王潮治軍有法,要求前來討伐。王潮便率軍圍泉州。翌年八月,攻陷泉州,殺廖彥若,據有其地。王潮歸附新上任的福建觀察使陳巖,陳巖表奏王潮為泉州刺史。王潮三兄弟治理泉州期間,「招懷離散,均賦繕兵,吏民悅之」。

大順二年(891年),陳巖病危,作書予王潮,希望他來福州授以軍政。王潮未至,陳巖即病逝。陳巖的妻舅福州護軍使范暉,自稱留後。范暉「驕侈,失眾心」,陳巖的舊將多與王潮友善,聲稱范暉可取。王潮便派從弟王彥復為都統、三弟王審知為都監,攻打福州,「彌年不下」。范暉向威勝節度使董昌求援。董昌派溫、台、婺州之兵五千人救援。王審知等人要求班師,被王潮拒絕;又請求王潮親自前來督戰,王潮回覆稱:「兵盡,益兵;將盡,益將;將盡,則吾至矣。」於是王審知等人並立攻城,最終在景福二年(893年)攻克福州。范暉棄城逃跑,被部將殺死。汀州刺史鐘全慕舉州來降,福建各地勢力紛紛歸附。乾寧年間,唐昭宗任命王潮為福建觀察使,王審知為副觀察使。

根據《十國春秋》的記載,王潮擔元帥的時候,曾請占卜師給自己的兩個弟弟算命,得到的結論是「一人勝一人」。當時王審知就在王潮身邊,渾身大汗而退。王潮在任期間執法嚴明,即便是王審知「有過」,王潮也「輙加捶楚,不以為嫌」。王審知也毫無怨色。王潮臨終之前,認為自己的兒子都沒有王審知有才能,便捨弃了自己的兒子,任命王審知為「權知軍府事」。王潮病逝後,王審知推戴次兄王審邽為泉州刺史,但王審邽認為王審知有功,於是推辭不受。王審知便自稱福建留後,上表於唐朝朝廷。光化元年(898年)春三月,被唐朝冊封為威武軍節度留後、檢校太保、刑部尚書。冬十月,又授金紫光祿大夫、尚書省右僕射、威武軍節度使。三年春二月,加同中書門下平章事;不久又改授光祿大夫、檢校司空、特進、檢校司徒。天復二年(902年),授賜武庫戟十二枝,立於私邸大門之前。天祐元年(904年)夏四月,唐朝派遣右拾遺翁承贊前往福州,加王審知為檢校太保,封琅琊郡王,食邑四千戶,實封一百戶。朱晃建立後梁以後,於開平三年(909年)封為閩王,加中書令,升福州為大都督府,正式建立閩國。後唐建立後,後唐莊宗加王審知為檢校太師守中書令。

王審知在位期間謹事四鄰,儘量地避免戰爭。開平三年(909年)時,楊吳遣使張知遠來聘,因其舉止倨慢而被王審知斬首。因此閩國與楊吳關係不佳,但在位期間兩國並未發生軍事衝突。王審知於貞明二年(916年)將女兒嫁給吳越國國王錢鏐之子錢傳珦(錢元珦)。翌年,王審知命次子王延鈞娶南漢君主劉龑(劉巖)之女。貞明四年(918年)夏六月,吳鎮南軍節度使、虔州行營招討使劉信率兵攻打虔州,百勝軍防禦使譚全播向王審知與楚王馬殷求救。王審知出兵鄠都救援,但在秋八月得知南楚戰敗後,便率軍班師。同光二年(924年)夏四月,劉巖領兵犯境,屯兵於汀、漳之境。王審知前去攻打,敗績。

經過王審知的努力,在戰亂的五代十國時期,福建相對來說比較安定,逃難的中原人相繼遷入福建。史載王審知「為人儉約,好禮下士」,「王雖據有一方,府舍卑陋,未常葺;居,恒常躡麻屢;寬刑薄賦,公私富實,境內以安」。正因為如此,招攬了不少中原名士前來投奔,其中包括唐朝學士韓偓、王淡(王溥之子)、楊沂(楊涉從弟)、徐寅(進士)等人。他也注重教育,「建學四門,以教閩士之秀者」。王審知積極發展海外貿易,招攬海外商賈,佛齊等國相繼前來朝貢。另一方面,他奉中原王朝後梁的正朔,並向後梁朝貢。當時楊吳的楊行密控制了江淮一帶,陸路朝貢路線被阻斷,王審知每年都遣使自登、萊入貢於後梁。後唐攻滅後梁後,王審知又繼續向後唐朝貢。

此外,王審知也著手擴建福州城。902年,王審知築福州外羅城四十里。905年,又築南北夾城,稱為「南北月城」,與大城合起來共計方圓二十六里四千八百丈。他也是一位佛教的虔誠供養者,在位期間曾向開元寺進獻菩薩之像,並舉行道場。

同光三年(925年)夏五月,王審知病危,命由長子威武節度副使王延翰為「權知軍府事」。冬十二月辛未薨,在位二十九年,年六十有四。葬于福州城北凤池山。長興三年,改葬莲花山,即今日晋安区新店镇斗顶村斗顶山。後唐朝廷賜諡忠懿,又賜神道碑,命張文蔚撰文。翌年,王延翰諡王審知为昭武王。王延钧即位后,追尊廟號太祖,谥号昭武孝皇帝,陵号宣陵。

王審知有八子,可是諸子积相猜忌,治兵相取。在诸多内乱纠纷中,许多王審知后人都被杀害。朱文进篡夺王氏政权时,王審知后人50余人尽被杀戮,僅存王延政一脈。

《舊五代史》:「審知起自隴畝,以至富貴。每以節儉自處,選任良吏,省刑惜費,輕徭薄斂,與民休息。三十年間,一境晏然。」

《十國春秋》:「太祖昆弟英姿傑出,號稱三龍。據有閩疆,賓賢禮士,衣冠懷之。抑亦可謂開國之雄歟?廼卒之,臣服中原,息兵養民,大指與吳越畧同,豈非度量有過人者遠哉!」

王審知有功於福建,故受福建人民崇敬。福州建有閩王祠,於市區立有閩王塑像。再如馬祖北竿鄉坂里村王家大宅內即有供奉閩王王審知,是凝聚坂里王家的宗族中心。

除以宗祠形式紀念外,更有列為神明以神廟作為信仰供奉。由於王審知喜乘白馬,並排行第三,故稱「白馬三郎」,死後被立廟奉祀,號「白馬尊王」。不過另有閩越王郢的第三子,也被福州人尊為「白馬三郎」。馬祖地區有數座白馬尊王廟,為當地民間信仰神祇之一,但其中目前僅莒光鄉東莒島福正村白馬尊王廟證實供奉為閩王王審知,為唯一供奉閩王王審知的白馬尊王廟,源自長樂沙洋鐃鈸境白馬忠懿王宮。

福建人尊稱王審知為「開閩尊王」、「開閩聖王」或「忠惠尊王」;尊審知長兄王潮為「威武尊王」、次兄王審邽為「泉安尊王」,視為鄉土神明供奉,合稱開閩三王。

\subsubsection{开平}

\begin{longtable}{|>{\centering\scriptsize}m{2em}|>{\centering\scriptsize}m{1.3em}|>{\centering}m{8.8em}|}
  % \caption{秦王政}\
  \toprule
  \SimHei \normalsize 年数 & \SimHei \scriptsize 公元 & \SimHei 大事件 \tabularnewline
  % \midrule
  \endfirsthead
  \toprule
  \SimHei \normalsize 年数 & \SimHei \scriptsize 公元 & \SimHei 大事件 \tabularnewline
  \midrule
  \endhead
  \midrule
  元年 & 909 & \tabularnewline\hline
  二年 & 910 & \tabularnewline\hline
  三年 & 911 & \tabularnewline
  \bottomrule
\end{longtable}

\subsubsection{乾化}

\begin{longtable}{|>{\centering\scriptsize}m{2em}|>{\centering\scriptsize}m{1.3em}|>{\centering}m{8.8em}|}
  % \caption{秦王政}\
  \toprule
  \SimHei \normalsize 年数 & \SimHei \scriptsize 公元 & \SimHei 大事件 \tabularnewline
  % \midrule
  \endfirsthead
  \toprule
  \SimHei \normalsize 年数 & \SimHei \scriptsize 公元 & \SimHei 大事件 \tabularnewline
  \midrule
  \endhead
  \midrule
  元年 & 911 & \tabularnewline\hline
  二年 & 912 & \tabularnewline\hline
  三年 & 913 & \tabularnewline\hline
  四年 & 914 & \tabularnewline\hline
  五年 & 915 & \tabularnewline
  \bottomrule
\end{longtable}

\subsubsection{贞明}

\begin{longtable}{|>{\centering\scriptsize}m{2em}|>{\centering\scriptsize}m{1.3em}|>{\centering}m{8.8em}|}
  % \caption{秦王政}\
  \toprule
  \SimHei \normalsize 年数 & \SimHei \scriptsize 公元 & \SimHei 大事件 \tabularnewline
  % \midrule
  \endfirsthead
  \toprule
  \SimHei \normalsize 年数 & \SimHei \scriptsize 公元 & \SimHei 大事件 \tabularnewline
  \midrule
  \endhead
  \midrule
  元年 & 915 & \tabularnewline\hline
  二年 & 916 & \tabularnewline\hline
  三年 & 917 & \tabularnewline\hline
  四年 & 918 & \tabularnewline\hline
  五年 & 919 & \tabularnewline\hline
  六年 & 920 & \tabularnewline\hline
  七年 & 921 & \tabularnewline
  \bottomrule
\end{longtable}

\subsubsection{龙德}

\begin{longtable}{|>{\centering\scriptsize}m{2em}|>{\centering\scriptsize}m{1.3em}|>{\centering}m{8.8em}|}
  % \caption{秦王政}\
  \toprule
  \SimHei \normalsize 年数 & \SimHei \scriptsize 公元 & \SimHei 大事件 \tabularnewline
  % \midrule
  \endfirsthead
  \toprule
  \SimHei \normalsize 年数 & \SimHei \scriptsize 公元 & \SimHei 大事件 \tabularnewline
  \midrule
  \endhead
  \midrule
  元年 & 921 & \tabularnewline\hline
  二年 & 922 & \tabularnewline\hline
  三年 & 923 & \tabularnewline
  \bottomrule
\end{longtable}


\subsubsection{同光}

\begin{longtable}{|>{\centering\scriptsize}m{2em}|>{\centering\scriptsize}m{1.3em}|>{\centering}m{8.8em}|}
  % \caption{秦王政}\
  \toprule
  \SimHei \normalsize 年数 & \SimHei \scriptsize 公元 & \SimHei 大事件 \tabularnewline
  % \midrule
  \endfirsthead
  \toprule
  \SimHei \normalsize 年数 & \SimHei \scriptsize 公元 & \SimHei 大事件 \tabularnewline
  \midrule
  \endhead
  \midrule
  元年 & 923 & \tabularnewline\hline
  二年 & 924 & \tabularnewline\hline
  三年 & 925 & \tabularnewline
  \bottomrule
\end{longtable}



%%% Local Variables:
%%% mode: latex
%%% TeX-engine: xetex
%%% TeX-master: "../../Main"
%%% End:

%% -*- coding: utf-8 -*-
%% Time-stamp: <Chen Wang: 2021-11-01 15:43:10>

\subsection{嗣王王延翰\tiny(926)}

\subsubsection{生平}

閩嗣王延翰(闽东语平話字:Uòng Iòng-hâng;?-927年),字子逸,五代時期閩國君主,王審知之長子。妻博陵郡夫人崔氏。

王審知在位時任威武軍節度副使。同光三年(925年),王審知去世,王延翰奉遺命繼立,權知軍府事,自稱威武留後,向後唐朝貢。汀州人陳本起兵,聚集三萬人圍攻汀州。王延翰派右軍都監柳邕等人,以二萬人前往討伐,將其斬殺。翌年春二月,後唐莊宗得知王延翰繼任之後,任命他為威武軍節度使。不久莊宗被弑,明宗繼位,改元天成,於夏五月,加其為同平章事。王延翰得知中原大亂,便有割據福建稱王之心。十月,王延翰取出《史記》,將其中的「東越列傳」的無諸一節翻出來給諸將吏看,並說:「閩,自古王國也。吾今不王,何待之有!」將吏們紛紛勸他割據自立,於是王延翰自稱大閩國王,立宮殿,置百官,威儀、文物皆擬天子制,羣下稱之曰殿下。又追諡王審知為昭武王,但仍奉後唐的正朔。

根據《十國春秋》的記載,王延翰「為人長大,美皙如玉,而好讀書、通經史」。但他個性驕傲荒淫,殘忍兇暴。《新五代史》中也記載了王延翰的荒淫,他在王審知喪服未除的時候便開始飲酒作樂。他於福州的西湖「築室十餘里,號曰水晶宮;每攜後庭游宴,從子城複道以出」。王延翰命人四處尋找民女,投入後宮作為自己的妾。其妻崔氏「陋而淫」,「延翰不能制」。這些被選入宮中的民女命運悲慘,被崔氏關押起來,「繫以大械,刻木為人手以擊頰,又以鐵錐刺之,一歲中死者八十四人」。後來崔氏大病一場,見者認為是被其害死者作祟,不久崔氏便死了。

王延翰看不起自己的兄弟,繼位才一個月,便將弟弟王延鈞貶為泉州刺史。建州刺史王延稟是王審知的養子,與王延翰一向不睦。當二人得知王延翰四處尋找民女之後,都上書勸阻,王延翰更加大怒。十二月,王延鈞與王延稟便聯手反叛,進軍福州。王延稟自建州順流而下,先至福州。福州指揮使陳陶率軍抵抗,兵敗自殺。天成元年十二月初八(陽曆927年1月14日)夜,王延稟率壯士百餘人,從西門架梯登城而入,攻取武庫,奪取兵器,直趨寢門。王延翰聞變,匿於別室,第二天早晨被王延稟擒獲。王延稟歷數其暴虐之罪,並聲稱他和崔氏一起害死了王審知,告諭百姓,斬於紫宸門之外。

王延鈞將王延翰葬於城北太平山,建太平地藏院,派丁守墓,稱為「王墓」。其地位於今福州市晉安區新店鎮,今日被福州人稱為「黃墓」。閩東語「黃」與「王」同音,因此被訛作「黃墓」。

\subsubsection{天成}

\begin{longtable}{|>{\centering\scriptsize}m{2em}|>{\centering\scriptsize}m{1.3em}|>{\centering}m{8.8em}|}
  % \caption{秦王政}\
  \toprule
  \SimHei \normalsize 年数 & \SimHei \scriptsize 公元 & \SimHei 大事件 \tabularnewline
  % \midrule
  \endfirsthead
  \toprule
  \SimHei \normalsize 年数 & \SimHei \scriptsize 公元 & \SimHei 大事件 \tabularnewline
  \midrule
  \endhead
  \midrule
  元年 & 926 & \tabularnewline
  \bottomrule
\end{longtable}



%%% Local Variables:
%%% mode: latex
%%% TeX-engine: xetex
%%% TeX-master: "../../Main"
%%% End:

%% -*- coding: utf-8 -*-
%% Time-stamp: <Chen Wang: 2021-11-01 15:43:42>

\subsection{惠宗王延鈞\tiny(926-935)}

\subsubsection{生平}

閩惠宗王延鈞(?-935年),繼位後更名王鏻(又作王璘),五代時期閩國第一代稱皇帝的君主,926年至935年在位。

王延鈞是太祖王審知的次子,也是嗣王王延翰之弟。王延翰繼位後,任王延鈞為泉州刺史,並在福州四處尋找民女,納入宮中為妾,王延鈞與建州刺史王延稟為此上書勸諫,王延翰大怒。王延稟與王延翰一向不睦,便與王延鈞聯軍攻打福州。王延稟自建州順流而下,於天成元年(926年)十二月初八(陽曆927年1月14日)先攻破福州,殺王延翰。不久王延鈞亦至,被王延稟推戴为武威留后。夏五月,後唐明宗任命王延鈞為威武軍節度使守中書令,封琅琊王。十一月,王延鈞遣使,向後唐進貢犀牛、香藥、海味。天成三年七月,後唐又遣吏部郎中裴羽、右散騎常侍陸崇,進封王延鈞為閩王。十二月,奉國節度使知建州王延稟向後唐上表稱疾,後唐任命其子王繼雄為建州刺史。

長興二年(931年),王延稟誤信王延鈞患重病,命次子王繼昇為建州留後,與王繼雄一起率水軍攻打福州。王延稟攻西門、王繼雄攻東門。王延鈞派樓船指揮使王仁達拒戰於南臺江。王仁達把士兵藏在船中,自己假裝出降。王繼雄登船撫慰,被王仁達殺死,梟首示眾於西門。王延稟正在放火攻城,見其子首級,不禁慟哭,軍心大亂。王仁達趁機發起攻擊,將其擒獲。王延鈞將王延稟斬於市,恢復其原名周彥琛,又派使者前去建州招撫其餘黨。王延稟之子王繼昇、王繼倫得知後出奔吳越。王延鈞便派弟弟都教練使王延政去鎮守建州。

當時福建僧侶眾多。王延鈞在位期間,下令丈量土地,分為三等,上等賜給僧侶,中等授予土著百姓,下等給流寓之人耕種。閩國的科舉之法模仿唐朝,但兩稅卻被加重。他喜好神仙之術,寵信左道巫者徐彥、朴盛韜、陳守元等人,還建造寶皇宮給陳守元居住,稱其為宮主。當時福州有王霸壇、煉丹井。壇旁皂莢木枯萎已久,一日突然長出枝葉來。井中又有白龜浮出。掘地,得石銘,有「王霸裔孫」之文。王延鈞便認為這在自己身上應驗了,便在壇旁建造宮殿,極盡奢華。

長興二年十二月,陳守元假借寶皇之命,建議王延鈞「避位受道,當為天子六十年。」於是王延鈞遜位給長子威武軍節度副使王繼鵬,成為道士,取道號玄錫。翌年春三月復位,要求後唐仿吴越钱镠、南楚马殷之例,封自己為尚書令。後唐不答,王延鈞遂斷絕雙方關係。

後唐長興四年(933年),黄龙现于真封宅,王延鈞下令改宅为龙跃宫,又建东华宫。当年正月,王延鈞在寶皇宮受册稱帝,國號大閩,改元龍啟,改名王鏻,立五庙,追谥王审知为太祖,封高盖山为西岳。王延鈞自知國土狹小,土地偏僻,因此謹慎與四鄰相處,境內還算安定。

王延鈞的元配是南漢清遠公主劉德秀,十分美麗,但早逝。繼室金氏賢而無寵。王延鈞相當寵愛淑妃陳金鳳,筑水晶宫于福州西湖旁,在湖中造彩舫数十,每舫置宫女二十余人,自乘大龙舟与陳金鳳同游。后又筑长春宫,与陳金鳳居于其中,每晚欢宴,燃金龙烛数百,使宫女擎金玉、玛瑙、琥珀、水晶之杯盘进馔,酒酣时裸体追逐嬉笑为乐。陳金鳳本是王審知的婢女,有才藝又十分淫蕩。因王延鈞晚年得风疾,陳金鳳遂與王延鈞的男宠歸守明、李可殷私通,閩人都很痛恨他們。

閩國永和元年(935年),陳金鳳被立為后。同年,王延鈞病重,王延鈞之子王繼鵬與皇城使李倣欲聯手提前了結陳后之勢力。李倣派兵進宮,殺死皇后陳金鳳及其黨羽。王延鈞躲到為他特製的九龍帳下,變軍刺了幾下後才出去。王延鈞重傷未死卻痛不欲生,宮女不忍見其受苦,遂殺死王延鈞。王延鈞死後,為王繼鵬諡為齊肅明孝皇帝,廟號惠宗,唯《新五代史》則作廟號太宗,諡號惠皇帝。

\subsubsection{天成}

\begin{longtable}{|>{\centering\scriptsize}m{2em}|>{\centering\scriptsize}m{1.3em}|>{\centering}m{8.8em}|}
  % \caption{秦王政}\
  \toprule
  \SimHei \normalsize 年数 & \SimHei \scriptsize 公元 & \SimHei 大事件 \tabularnewline
  % \midrule
  \endfirsthead
  \toprule
  \SimHei \normalsize 年数 & \SimHei \scriptsize 公元 & \SimHei 大事件 \tabularnewline
  \midrule
  \endhead
  \midrule
  元年 & 926 & \tabularnewline\hline
  二年 & 927 & \tabularnewline\hline
  三年 & 928 & \tabularnewline\hline
  四年 & 929 & \tabularnewline\hline
  五年 & 930 & \tabularnewline
  \bottomrule
\end{longtable}

\subsubsection{长兴}

\begin{longtable}{|>{\centering\scriptsize}m{2em}|>{\centering\scriptsize}m{1.3em}|>{\centering}m{8.8em}|}
  % \caption{秦王政}\
  \toprule
  \SimHei \normalsize 年数 & \SimHei \scriptsize 公元 & \SimHei 大事件 \tabularnewline
  % \midrule
  \endfirsthead
  \toprule
  \SimHei \normalsize 年数 & \SimHei \scriptsize 公元 & \SimHei 大事件 \tabularnewline
  \midrule
  \endhead
  \midrule
  元年 & 930 & \tabularnewline\hline
  二年 & 931 & \tabularnewline\hline
  三年 & 932 & \tabularnewline
  \bottomrule
\end{longtable}

\subsubsection{龙启}

\begin{longtable}{|>{\centering\scriptsize}m{2em}|>{\centering\scriptsize}m{1.3em}|>{\centering}m{8.8em}|}
  % \caption{秦王政}\
  \toprule
  \SimHei \normalsize 年数 & \SimHei \scriptsize 公元 & \SimHei 大事件 \tabularnewline
  % \midrule
  \endfirsthead
  \toprule
  \SimHei \normalsize 年数 & \SimHei \scriptsize 公元 & \SimHei 大事件 \tabularnewline
  \midrule
  \endhead
  \midrule
  元年 & 933 & \tabularnewline\hline
  二年 & 934 & \tabularnewline
  \bottomrule
\end{longtable}

\subsubsection{永和}

\begin{longtable}{|>{\centering\scriptsize}m{2em}|>{\centering\scriptsize}m{1.3em}|>{\centering}m{8.8em}|}
  % \caption{秦王政}\
  \toprule
  \SimHei \normalsize 年数 & \SimHei \scriptsize 公元 & \SimHei 大事件 \tabularnewline
  % \midrule
  \endfirsthead
  \toprule
  \SimHei \normalsize 年数 & \SimHei \scriptsize 公元 & \SimHei 大事件 \tabularnewline
  \midrule
  \endhead
  \midrule
  元年 & 935 & \tabularnewline\hline
  二年 & 936 & \tabularnewline
  \bottomrule
\end{longtable}


%%% Local Variables:
%%% mode: latex
%%% TeX-engine: xetex
%%% TeX-master: "../../Main"
%%% End:

%% -*- coding: utf-8 -*-
%% Time-stamp: <Chen Wang: 2021-11-01 15:43:59>

\subsection{康宗王繼鵬\tiny(935-939)}

\subsubsection{生平}

闽康宗王繼鵬(閩東語:Uòng Gié-bèng;閩南語:Ông Kè-phêng;?-939年),後改名王昶,五代時期閩國君主,王延鈞的嫡出次子,母親是南漢清遠公主劉德秀。

王繼鵬原封福王。寵妾李春鷰本為王延鈞的宮女,王繼鵬與之私通,因此向繼母陳金鳳求助,說服王延鈞將其賜給王繼鵬。

閩國永和元年(935年),王繼鵬與李倣政變,殺王延鈞、陳皇后和弟王繼韜,繼位稱帝,改名王昶,封李春鷰為賢妃。次年(936年),改元通文,再封李春鷰為皇后。

王繼鵬亦如其父,十分寵信道士陳守元,連政事亦與之商量,興建紫微宮,以水晶装饰,工程浩大,更勝于寶皇宮,又因工程繁多而費用不足,因此賣官鬻爵,橫徵暴斂。

王繼鵬個性猜忌因此屢殺宗室,其叔王延羲為避禍,遂裝瘋賣傻,被王繼鵬軟禁自宅。當時原王審知的親軍「拱宸都」、「控鶴都」因賞賜不如王繼鵬自己的親軍「宸衛都」而迭有怨言。閩國通文四年(939年),北宫失火,宫殿焚烧殆尽。拱宸、控鶴軍使朱文進、連重遇因被王繼鵬懷疑對皇宮縱火,恐懼之餘遂先發難。乱兵焚长春宮,随后迎王延羲于长春宫瓦砾中登基,並攻擊王繼鵬。王繼鵬携皇后逃往宸卫都,天明后「拱宸都」、「控鶴都」进攻「宸衛都」,后者兵败,王繼鵬自福州北门出逃,在梧桐岭為追兵所獲,與皇后李春鷰及諸子一同被其堂兄王继业所殺。王延羲隨後把王繼鵬被殺之責推到「宸衛都」身上,並追諡王繼鵬為聖神英睿文明廣武應道大弘孝皇帝,廟號康宗。

\subsubsection{通文}

\begin{longtable}{|>{\centering\scriptsize}m{2em}|>{\centering\scriptsize}m{1.3em}|>{\centering}m{8.8em}|}
  % \caption{秦王政}\
  \toprule
  \SimHei \normalsize 年数 & \SimHei \scriptsize 公元 & \SimHei 大事件 \tabularnewline
  % \midrule
  \endfirsthead
  \toprule
  \SimHei \normalsize 年数 & \SimHei \scriptsize 公元 & \SimHei 大事件 \tabularnewline
  \midrule
  \endhead
  \midrule
  元年 & 936 & \tabularnewline\hline
  二年 & 937 & \tabularnewline\hline
  三年 & 938 & \tabularnewline\hline
  四年 & 939 & \tabularnewline
  \bottomrule
\end{longtable}


%%% Local Variables:
%%% mode: latex
%%% TeX-engine: xetex
%%% TeX-master: "../../Main"
%%% End:

%% -*- coding: utf-8 -*-
%% Time-stamp: <Chen Wang: 2021-11-01 15:45:37>

\subsection{景宗王延羲\tiny(939-944)}

\subsubsection{生平}

闽景宗王延羲(?-944年),继位後改名王曦,五代時期閩國君主,王審知之子,王延翰、王延鈞之弟,王繼鵬之叔。

王延羲於王繼鵬在位時任左僕射、同平章事,因王繼鵬猜忌宗室,遂裝瘋賣傻,因此被軟禁自宅。閩國通文四年(939年)拱宸、控鶴軍使朱文進、連重遇反,迎王延羲進宮,並殺王繼鵬,王延羲遂自稱威武節度使、閩國王,更名王曦,改元永隆,稱臣於後晉,但在國內官制就如同皇帝一樣。

然而王延羲繼位後,驕傲奢侈,荒淫無度,猜忌宗族,比王繼鵬有過之而無不及,其弟建州刺史王延政多有規勸,王延羲反而回信怒罵,又差人探聽王延政的隱私,二人因此結怨。永隆二年(940年)王延羲攻建州,開啟了閩國內戰,二人於數年爭戰中互有勝負。永隆三年(941年)七月,王延羲自稱大闽皇、威武军節度使。十月,再晋尊为大闽皇帝,加尊号睿明文廣武聖光德隆道大孝皇帝。王延政随后称帝,国号殷,闽国分裂。

由於王延羲個性一向暴虐,而朱文進、連重遇自從殺了王繼鵬後,就一直擔心為人所害,二人因此認為王延羲有加害之意,永隆六年(944年),連朱二人先下手為強,王延羲被刺殺。王延羲死後被諡為睿文廣武明聖元德隆道大孝皇帝,廟號景宗。

\subsubsection{永隆}

\begin{longtable}{|>{\centering\scriptsize}m{2em}|>{\centering\scriptsize}m{1.3em}|>{\centering}m{8.8em}|}
  % \caption{秦王政}\
  \toprule
  \SimHei \normalsize 年数 & \SimHei \scriptsize 公元 & \SimHei 大事件 \tabularnewline
  % \midrule
  \endfirsthead
  \toprule
  \SimHei \normalsize 年数 & \SimHei \scriptsize 公元 & \SimHei 大事件 \tabularnewline
  \midrule
  \endhead
  \midrule
  元年 & 939 & \tabularnewline\hline
  二年 & 940 & \tabularnewline\hline
  三年 & 941 & \tabularnewline\hline
  四年 & 942 & \tabularnewline\hline
  五年 & 943 & \tabularnewline\hline
  六年 & 944 & \tabularnewline
  \bottomrule
\end{longtable}


%%% Local Variables:
%%% mode: latex
%%% TeX-engine: xetex
%%% TeX-master: "../../Main"
%%% End:

%% -*- coding: utf-8 -*-
%% Time-stamp: <Chen Wang: 2019-12-26 09:57:33>

\subsection{王延政\tiny(943-945)}

\subsubsection{朱文进生平}

朱文進(閩東語:Ciŏ Ùng-Céng;閩南語:Tsu Bûn-Tsìn;?-945年),永泰(今福建永泰)人,五代時期閩國君主。閩帝王繼鵬在位時任拱宸軍使。

「拱宸都」與「控鶴都」原來都是閩太祖王審知的親軍,閩康宗王繼鵬即位後建立自己的親軍名喚「宸衛都」,而待之比拱宸、控鶴二都更厚,二都迭有怨言。朱文進並與控鶴軍使連重遇曾被王繼鵬三番四次的侮辱,二人因此十分不滿。

閩國通文四年(939年),北宮失火,連重遇奉派率軍清理火場殘餘的灰燼,工作勞苦,士卒怨懟。而連重遇又被王繼鵬懷疑參與縱火,因此率軍叛變,迎立王繼鵬之叔王延羲為帝,並殺害王繼鵬。朱文進在這次政變後,被任命為拱宸都指揮使。

朱文進與連重遇自從殺了王繼鵬後,就一直擔心為人所害,而王延羲個性一向暴虐,二人因此認為王延羲有加害之意,閩國永隆六年(944年),朱、連二人先發制人,刺殺王延羲,朱文進並被連重遇推舉,自稱閩主,殺害境內王姓皇族成員五十餘人,並放宮女出宮,停止興建中的工程,企圖與王延羲的暴政完全相反以拉攏人心。

不久,朱文進取消帝號自稱威武留後,向後晉稱臣,而後晉任命朱文進為威武節度使。後晉開運元年(944年)十二月十五日(陽曆為945年1月1日),朱文進正式被後晉出帝石重貴冊封為閩國王。

但在此時,朱、連二人的軍隊不斷被由將領留從效、陳洪進以及殷帝王延政所率領的討伐軍擊敗,情勢日漸窘迫,部下因此離心。後晉開運元年(944年)閏十二月二十九日(陽曆為945年2月14日)朱文進及連重遇被部屬林仁翰誅殺。

\subsubsection{王延政生平}

閩天德帝王延政(閩東語:Uòng Iòng-céng;?-951年),五代時期閩國最後一位君主,也是殷国唯一君主,王審知的第十三子,人稱十三郎。王延翰、王延鈞、王延羲之弟,王延羲在位時任建州刺史。

由於王延羲繼位後,驕傲奢侈,荒淫無度,猜忌宗族,王延政因此多有規勸,然而王延羲反而回信怒罵,又差人探聽王延政的隱私,二人因此結怨。閩國永隆二年(940年)王延羲攻建州,開啟了閩國內戰,二人於數年爭戰中互有勝負。

永隆三年(941年)二人短暫休兵,王延政被王延羲封為富沙王,惟不久又重啟戰端。閩國永隆五年(943年),王延政自行於建州稱帝,國號殷,改元天德。王延政與王繼鵬、王延羲一樣橫徵暴斂,因此人民生活困苦。

閩國永隆六年(944年),朱文進、連重遇殺王延羲,朱文進自稱閩主。王延政遂出兵討伐,而朱文進、連重遇尋為部下所殺。天德三年(945年)諸臣請王延政還都福州並恢復閩國國號,惟當時南唐大軍已趁閩國內亂時壓境,只好任命姪兒王繼昌出鎮福州,並派黃仁諷協助王繼昌。但黃仁諷受李仁達慫恿,殺死王繼昌,延政聞之,族誅黃仁諷一家,並派張漢真領水軍討伐福州。不久,建州城陷,王延政投降,閩國亡。

王延政後來被送往南唐都城金陵,南唐帝李璟封他為羽林大將軍;南唐保大五年(947年)改封為鄱陽王;保大九年(951年),再改封為光山王,不久過世,被追贈為福王,諡號恭懿。

\subsubsection{天德}

\begin{longtable}{|>{\centering\scriptsize}m{2em}|>{\centering\scriptsize}m{1.3em}|>{\centering}m{8.8em}|}
  % \caption{秦王政}\
  \toprule
  \SimHei \normalsize 年数 & \SimHei \scriptsize 公元 & \SimHei 大事件 \tabularnewline
  % \midrule
  \endfirsthead
  \toprule
  \SimHei \normalsize 年数 & \SimHei \scriptsize 公元 & \SimHei 大事件 \tabularnewline
  \midrule
  \endhead
  \midrule
  元年 & 943 & \tabularnewline\hline
  二年 & 944 & \tabularnewline\hline
  三年 & 945 & \tabularnewline
  \bottomrule
\end{longtable}


%%% Local Variables:
%%% mode: latex
%%% TeX-engine: xetex
%%% TeX-master: "../../Main"
%%% End:



%%% Local Variables:
%%% mode: latex
%%% TeX-engine: xetex
%%% TeX-master: "../../Main"
%%% End:
