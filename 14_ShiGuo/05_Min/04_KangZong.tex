%% -*- coding: utf-8 -*-
%% Time-stamp: <Chen Wang: 2019-12-26 09:54:29>

\subsection{康宗\tiny(935-939)}

\subsubsection{生平}

闽康宗王繼鵬(閩東語:Uòng Gié-bèng;閩南語:Ông Kè-phêng;?-939年),後改名王昶,五代時期閩國君主,王延鈞的嫡出次子,母親是南漢清遠公主劉德秀。

王繼鵬原封福王。寵妾李春鷰本為王延鈞的宮女,王繼鵬與之私通,因此向繼母陳金鳳求助,說服王延鈞將其賜給王繼鵬。

閩國永和元年(935年),王繼鵬與李倣政變,殺王延鈞、陳皇后和弟王繼韜,繼位稱帝,改名王昶,封李春鷰為賢妃。次年(936年),改元通文,再封李春鷰為皇后。

王繼鵬亦如其父,十分寵信道士陳守元,連政事亦與之商量,興建紫微宮,以水晶装饰,工程浩大,更勝于寶皇宮,又因工程繁多而費用不足,因此賣官鬻爵,橫徵暴斂。

王繼鵬個性猜忌因此屢殺宗室,其叔王延羲為避禍,遂裝瘋賣傻,被王繼鵬軟禁自宅。當時原王審知的親軍「拱宸都」、「控鶴都」因賞賜不如王繼鵬自己的親軍「宸衛都」而迭有怨言。閩國通文四年(939年),北宫失火,宫殿焚烧殆尽。拱宸、控鶴軍使朱文進、連重遇因被王繼鵬懷疑對皇宮縱火,恐懼之餘遂先發難。乱兵焚长春宮,随后迎王延羲于长春宫瓦砾中登基,並攻擊王繼鵬。王繼鵬携皇后逃往宸卫都,天明后「拱宸都」、「控鶴都」进攻「宸衛都」,后者兵败,王繼鵬自福州北门出逃,在梧桐岭為追兵所獲,與皇后李春鷰及諸子一同被其堂兄王继业所殺。王延羲隨後把王繼鵬被殺之責推到「宸衛都」身上,並追諡王繼鵬為聖神英睿文明廣武應道大弘孝皇帝,廟號康宗。

\subsubsection{通文}

\begin{longtable}{|>{\centering\scriptsize}m{2em}|>{\centering\scriptsize}m{1.3em}|>{\centering}m{8.8em}|}
  % \caption{秦王政}\
  \toprule
  \SimHei \normalsize 年数 & \SimHei \scriptsize 公元 & \SimHei 大事件 \tabularnewline
  % \midrule
  \endfirsthead
  \toprule
  \SimHei \normalsize 年数 & \SimHei \scriptsize 公元 & \SimHei 大事件 \tabularnewline
  \midrule
  \endhead
  \midrule
  元年 & 936 & \tabularnewline\hline
  二年 & 937 & \tabularnewline\hline
  三年 & 938 & \tabularnewline\hline
  四年 & 939 & \tabularnewline
  \bottomrule
\end{longtable}


%%% Local Variables:
%%% mode: latex
%%% TeX-engine: xetex
%%% TeX-master: "../../Main"
%%% End:
