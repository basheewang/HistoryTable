%% -*- coding: utf-8 -*-
%% Time-stamp: <Chen Wang: 2019-12-26 10:02:48>

\subsection{后主\tiny(958-971)}

\subsubsection{生平}

劉\xpinyin*{鋹}(942年-980年)原名劉繼興,五代十國時期南漢末代君主,是南漢中宗劉晟之長子,原封衛王。

南漢乾和十六年(958年)劉晟去世,劉繼興繼位,改名劉鋹,改元大寶。

劉鋹不會治國,政事皆委諸宦官龔澄樞及女侍中盧瓊仙等人,女官亦任命為參政官員,其餘官員只是聊備一格而已。

劉鋹又認為群臣都有家室,會為了顧及子孫不肯盡忠,因此只信任宦官,臣屬必須自宮才會被進用,以致於一度宦官高達二萬人之多。

又相當寵愛一名波斯女子,與之淫戲於後宮,叫她「媚豬」,而自稱「蕭閒大夫」,不理政事。後來將政事又交予女巫樊胡,連龔澄樞及盧瓊仙都依附她,政事紊亂。

南漢大寶十三年(970年),宋朝派潭州防禦使潘美攻南漢。南漢舊將先前多因讒言而被殺,宗室亦遭翦除殆盡,掌兵權的只有宦官而已,城牆、護城河,都裝飾為宮殿、水塘;樓船戰艦、武器盔甲,全部腐朽。

大寶十四年(宋開寶四年,971年),宋軍節節進逼。劉鋹纵火焚毁宫殿、府库,挑選十幾艘船,滿載金銀財寶及嬪妃,準備逃亡入海;還沒出發,宦官與衛兵就盜取船舶逃走,劉鋹只好投降,南漢亡。

劉鋹歸順宋朝後,押送至东京開封府,囚于玉津园,后以帛系颈,与南汉官属一同献俘于太庙、太社。宋太祖遣吕余庆问责焚烧府库之事。劉鋹將責任完全推給龔澄樞,其曰“臣年十六僭伪号,澄枢等皆先臣旧人,每事,臣不得自由,在国时,臣是臣下,澄枢是国主。”宋太祖趙匡胤就將龔澄樞斬首,而赦免劉鋹的罪,並任命其為金紫光禄大夫、检校太保、右千牛衛大將軍、员外置同正员,封恩赦侯。

荒淫無度的劉鋹投降後,為宋太祖、宋太宗厚待,也出現不少趣事,與南唐後主李煜的國愁家恨形成強烈對比。劉鋹本人體態豐滿,眉清目秀。有巧思,亦能言善辯,曾用珠子將馬鞍串成戲龍的形狀獻予宋太祖。宋太祖因此感嘆說:「劉鋹如果能將這項技藝用在治國上,怎麼會滅亡!」劉鋹稱帝在位时,多置鴆酒,毒死臣下。降宋後,一日宋太祖乘肩舆,与随从数十人幸讲武池。从官未至,而劉鋹先至。宋太祖賜以酒,劉鋹以為要毒殺自己,大哭曰:“臣承祖父基业,违拒朝廷,劳王师讨致,罪固当诛。陛下既待臣不死,愿为大梁布衣,观太平之盛。臣未敢饮此酒。”宋太祖笑而取酒自飲,劉鋹大感慚愧。

開寶八年(975年),宋滅南唐後,將劉鋹改命左監門衛上將軍,封彭城郡公。宋太宗即帝位,再改封其為衛國公。

太平興國四年(979年),宋太宗將伐北漢劉繼元,在長春殿宴請潘美等將領。當時劉鋹與已降宋的前吳越王錢俶、前平海军節度使陳洪進都參加,劉鋹因此說:「朝廷威靈遠播,四方竊位僭主的君王,今日都在座,不久又要平定太原,劉繼元又將到達,臣率先來朝,願揮舞大棒,替陛下吶喊助威,望成為各國降王的領袖。」宋太宗因此大笑。

太平興國五年(980年),劉鋹去世,獲贈授太師,追封為南越王。由於劉鋹是南漢最後一位君主,復無諡號、廟號,史家所以習稱其為南漢後主。

\subsubsection{大宝}

\begin{longtable}{|>{\centering\scriptsize}m{2em}|>{\centering\scriptsize}m{1.3em}|>{\centering}m{8.8em}|}
  % \caption{秦王政}\
  \toprule
  \SimHei \normalsize 年数 & \SimHei \scriptsize 公元 & \SimHei 大事件 \tabularnewline
  % \midrule
  \endfirsthead
  \toprule
  \SimHei \normalsize 年数 & \SimHei \scriptsize 公元 & \SimHei 大事件 \tabularnewline
  \midrule
  \endhead
  \midrule
  元年 & 958 & \tabularnewline\hline
  二年 & 959 & \tabularnewline\hline
  三年 & 960 & \tabularnewline\hline
  四年 & 961 & \tabularnewline\hline
  五年 & 962 & \tabularnewline\hline
  六年 & 963 & \tabularnewline\hline
  七年 & 964 & \tabularnewline\hline
  八年 & 965 & \tabularnewline\hline
  九年 & 966 & \tabularnewline\hline
  十年 & 967 & \tabularnewline\hline
  十一年 & 968 & \tabularnewline\hline
  十二年 & 969 & \tabularnewline\hline
  十三年 & 970 & \tabularnewline\hline
  十四年 & 971 & \tabularnewline
  \bottomrule
\end{longtable}



%%% Local Variables:
%%% mode: latex
%%% TeX-engine: xetex
%%% TeX-master: "../../Main"
%%% End:
