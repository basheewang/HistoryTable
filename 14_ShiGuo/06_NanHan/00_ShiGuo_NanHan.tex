%% -*- coding: utf-8 -*-
%% Time-stamp: <Chen Wang: 2019-12-26 10:03:02>


\section{南汉\tiny(917-971)}

\subsection{简介}

南汉(917年-971年)是五代十国时期的地方政权之一,位于现广东、广西、海南三省及越南北部(后失)。971年为北宋所灭。

唐朝末年,刘谦在岭南的封州(今广东封开)任刺史,拥兵过万,战舰百餘。刘谦死后,其长子刘隐继承父职。唐天祐二年(905年),刘隐任清海军(岭南东道)节度使。907年,刘隐受后梁封为彭郡王,909年改封为南平王,次年又改封为南海王。刘隐死后,其弟劉龑(又名刘岩)袭封南海王。劉龑凭借父兄在岭南的基业,于后梁贞明三年(917年)在番禺(今广州,号兴王府)称帝,国号「大越」。次年,劉龑以汉朝刘氏后裔的身份改国号为「漢」,史称南汉,以别于北汉。

刘龑迷信,他非常喜欢《周易》,年号的改变,以及名字的变动,原因都是算卦所致。南汉、南唐曾经是友好邻邦,潘佑《为李后主与南汉后主书》称两国“情若弟兄,义同交契”。南漢末年政治黑暗,“作燒煮剝剔、刀山劍樹之刑,或令罪人鬥虎抵象。又賦斂煩重,人不聊生。”,邕州“民入城者,人输一钱”,“置媚川都,定其課,令入海五百尺采珠。所居宮殿以珠、玳瑁飾之。陳延壽作諸淫巧,日費數萬金。”

大有十年(937年),南汉的交州发生兵变,属将矯公羨杀死了主管官员,割据一方,另一属将吳權领兵攻打矯公羨,矯公羨便求救于劉龑,刘龑封儿子刘弘操为交王,然后领兵进攻吴权,结果被吴权打败,刘弘操阵亡。吴权从此占有了交州,吴权建立的王朝即越南吳朝,越南由此正式从中国独立出去。

乾和十四年,周世宗遣使通好,南汉不把中原王朝放在眼里,“馆接者遗茉莉,文其名曰小南强”。大宝十三年,宋太祖授意南唐后主遣使游说南汉归宋,后主派陈省躬出使南汉,游说南汉向宋称臣,刘鋹不从。不久,宋以潭州防御使潘美为贺州道行营兵马都总管,朗州团练使尹崇珂为副,发兵攻南汉。十四年二月,南汉亡。

南汉是一个商业气息浓郁的国家,“岭北行商至国都,必召示之夸其富”,“每见北人,盛夸岭海之强”。马端临谓:“宋兴,而吴、蜀、江南、荆湖、南粤,皆号富强”。钱多为主,钱少为奴视为一般的准则。

劉鋹时,日与波斯女等大宫中游宴,“无名之费,日有千万”,其官制则有特殊的规定,科举被录取者,若要做官必须先净身,也就是閹割。在劉鋹看来,百官们有家有室,有妻儿老小,肯定不能对皇上尽忠。


%% -*- coding: utf-8 -*-
%% Time-stamp: <Chen Wang: 2019-12-26 10:01:09>

\subsection{高祖\tiny(917-942)}

\subsubsection{生平}

漢高祖劉龑yǎn(889年-942年),原稱劉巖,又名刘陟。五代十国时期南汉开國皇帝,蔡州上蔡(今河南省上蔡县)人,郡望彭城(今江苏省徐州市),父亲是南汉追赠圣武皇帝刘知谦,母亲是妾段氏。

祖父劉安仁為蔡州上蔡(今河南上蔡)人,遷居福建,以經商為生,又因為生意需要,遷居嶺南,生刘知谦。唐朝末年,知谦從軍,受到當時的南海軍節度使韋宙賞識,與韋的姪女韦氏結婚,生下劉隱、劉臺兩個兒子。黃巢之亂出征有功,882年任封州刺史。

刘知谦后来又私纳小妾段氏,生下劉巖。正妻韦氏大怒,杀了段氏,但未忍伤害还是婴儿的劉巖,抱回家中和自己的两个儿子一起抚育。劉巖长大之後,聪慧又擅長武艺,且精通占卜算命之术,但又天性苛酷,每视杀人则喜,人皆以为蛟蜃化身。

知谦死後,長子劉隱擁兵自重,克肇慶、番禺(今屬广州),割据岭南地区,逐渐坐大,但劉隱向後梁稱臣納貢,後梁封為南海王,乾化元年(911年),劉隱还没来得及称帝,就因病在本郡去世,諡襄王。

劉隱有子,但劉巖篡奪其位,自命為留後,又襲位交趾节度使,其后又袭封南海王称号。劉隱死后六年(917年),劉巖在番禺称帝,建国号为「大越」, 改元为「乾亨」,定都番禺(今日廣州市)。次年,刘巖自称是汉朝皇室的后裔,为了表示自己建国是恢复昔日的汉家天下,于是又改国号为「大汉」,史称南汉。

後唐同光四年(926年),劉巖取《周易》中「飞龍在天」之意,改名为「劉龑」。據說他原先为自己改名为「劉龔」,後才自创一「龑」字。

除了劉龑及祖父劉安仁为上蔡(今河南省上蔡县)人的说法外,其侄女刘华的墓志称家族出于东晋时南渡的彭城刘氏,“而家于五羊,今为封州贺水人也”。藤田豐八认为他是大食商人后裔。藤田氏此说遭到另一位日本学者桑原骘藏的质疑。桑原认为,关于宋元祐间广州蕃坊娶宗女的刘姓人为阿拉伯人的说法,不能用以证明南汉刘氏是阿拉伯人的后裔,相反,宋元祐间广州蕃客的刘姓,可能是南汉所赐,“余谓唐代每以国姓赐外国人,此刘姓回民,或南汉刘氏赐与广州蕃客者,因为得姓之源欤?”

\subsubsection{乾亨}

\begin{longtable}{|>{\centering\scriptsize}m{2em}|>{\centering\scriptsize}m{1.3em}|>{\centering}m{8.8em}|}
  % \caption{秦王政}\
  \toprule
  \SimHei \normalsize 年数 & \SimHei \scriptsize 公元 & \SimHei 大事件 \tabularnewline
  % \midrule
  \endfirsthead
  \toprule
  \SimHei \normalsize 年数 & \SimHei \scriptsize 公元 & \SimHei 大事件 \tabularnewline
  \midrule
  \endhead
  \midrule
  元年 & 917 & \tabularnewline\hline
  二年 & 918 & \tabularnewline\hline
  三年 & 919 & \tabularnewline\hline
  四年 & 920 & \tabularnewline\hline
  五年 & 921 & \tabularnewline\hline
  六年 & 922 & \tabularnewline\hline
  七年 & 923 & \tabularnewline\hline
  八年 & 924 & \tabularnewline\hline
  九年 & 925 & \tabularnewline
  \bottomrule
\end{longtable}

\subsubsection{白龙}

\begin{longtable}{|>{\centering\scriptsize}m{2em}|>{\centering\scriptsize}m{1.3em}|>{\centering}m{8.8em}|}
  % \caption{秦王政}\
  \toprule
  \SimHei \normalsize 年数 & \SimHei \scriptsize 公元 & \SimHei 大事件 \tabularnewline
  % \midrule
  \endfirsthead
  \toprule
  \SimHei \normalsize 年数 & \SimHei \scriptsize 公元 & \SimHei 大事件 \tabularnewline
  \midrule
  \endhead
  \midrule
  元年 & 925 & \tabularnewline\hline
  二年 & 926 & \tabularnewline\hline
  三年 & 927 & \tabularnewline\hline
  四年 & 928 & \tabularnewline
  \bottomrule
\end{longtable}

\subsubsection{大有}

\begin{longtable}{|>{\centering\scriptsize}m{2em}|>{\centering\scriptsize}m{1.3em}|>{\centering}m{8.8em}|}
  % \caption{秦王政}\
  \toprule
  \SimHei \normalsize 年数 & \SimHei \scriptsize 公元 & \SimHei 大事件 \tabularnewline
  % \midrule
  \endfirsthead
  \toprule
  \SimHei \normalsize 年数 & \SimHei \scriptsize 公元 & \SimHei 大事件 \tabularnewline
  \midrule
  \endhead
  \midrule
  元年 & 928 & \tabularnewline\hline
  二年 & 929 & \tabularnewline\hline
  三年 & 930 & \tabularnewline\hline
  四年 & 931 & \tabularnewline\hline
  五年 & 932 & \tabularnewline\hline
  六年 & 933 & \tabularnewline\hline
  七年 & 934 & \tabularnewline\hline
  八年 & 935 & \tabularnewline\hline
  九年 & 936 & \tabularnewline\hline
  十年 & 937 & \tabularnewline\hline
  十一年 & 938 & \tabularnewline\hline
  十二年 & 939 & \tabularnewline\hline
  十三年 & 940 & \tabularnewline\hline
  十四年 & 941 & \tabularnewline\hline
  十五年 & 942 & \tabularnewline
  \bottomrule
\end{longtable}


%%% Local Variables:
%%% mode: latex
%%% TeX-engine: xetex
%%% TeX-master: "../../Main"
%%% End:

%% -*- coding: utf-8 -*-
%% Time-stamp: <Chen Wang: 2019-12-26 10:01:32>

\subsection{殇帝\tiny(942-943)}

\subsubsection{生平}

漢殤帝劉玢(920年-943年),原名劉弘度,五代時期南漢君主,是南漢建立者劉龑之第三子,原封賓王,後改封秦王。

南漢大有十五年(942年),南漢高祖劉龑病重,原以劉弘度及晉王劉弘熙驕傲放縱,因此欲立幼子越王劉弘昌,為臣下所力諫,乃打消此意。不久,劉龑病逝,劉弘度遂繼位,改名劉玢,並改年號光天。

同年,循州(今廣東龍川)變民共推縣吏張遇賢為主,稱中天八國王,改元永樂,一時間南漢的東方州縣多被張遇賢所攻陷。

劉玢驕傲奢侈,荒淫無度。劉龑還在停靈的時條,就肆無忌憚地奏樂飲酒;夜晚則穿著黑色的喪服外出與娼妓混在一起;又命男女赤身裸體供其觀賞;政事廢弛,因此民變愈演愈烈。而左右有忤逆其意者動輒被處死,以致於除了劉弘昌及內常侍吳懷恩外,沒有人敢勸諫,而受勸諫後又不聽。劉玢亦猜忌諸弟及官員,所以叫宦官把守宮門,進宮時要將衣服脫掉檢查,才可以入內,劉弘熙遂生政變之意。

因為知道劉玢喜歡角力,所以南漢光天二年(943年),劉弘熙命陳道庠找來力士數人並告知劉玢,劉玢聽到後大為高興,即與諸王於長春宮聚會飲宴,並觀賞角力。當晚宴會結束,劉玢大醉,劉弘熙於是命力士抓住劉玢,摧擊其前胸斃命。劉玢死後,被諡殤帝。

\subsubsection{光天}

\begin{longtable}{|>{\centering\scriptsize}m{2em}|>{\centering\scriptsize}m{1.3em}|>{\centering}m{8.8em}|}
  % \caption{秦王政}\
  \toprule
  \SimHei \normalsize 年数 & \SimHei \scriptsize 公元 & \SimHei 大事件 \tabularnewline
  % \midrule
  \endfirsthead
  \toprule
  \SimHei \normalsize 年数 & \SimHei \scriptsize 公元 & \SimHei 大事件 \tabularnewline
  \midrule
  \endhead
  \midrule
  元年 & 942 & \tabularnewline\hline
  二年 & 943 & \tabularnewline
  \bottomrule
\end{longtable}



%%% Local Variables:
%%% mode: latex
%%% TeX-engine: xetex
%%% TeX-master: "../../Main"
%%% End:

%% -*- coding: utf-8 -*-
%% Time-stamp: <Chen Wang: 2021-11-01 15:47:55>

\subsection{中宗劉晟\tiny(943-958)}

\subsubsection{生平}

漢中宗劉\xpinyin*{晟}(920年-958年),原名劉弘熙,五代十國時期南漢君主,是南漢建立者劉龑之子,劉玢之弟。

南漢帝劉玢即位後驕傲奢侈,不理政事,荒淫無道,並且猜忌諸弟,原封晉王的劉弘熙因此有政變之意。南漢光天二年(943年),劉弘熙找來力士數人表演角力,與劉玢飲宴觀賞,當晚宴會結束,劉玢大醉,劉弘熙即命力士抓住劉玢,摧擊其前胸,斃命。翌晨,越王劉弘昌率諸弟至寢殿,迎劉弘熙即皇帝位,劉弘熙繼位後,改名劉晟,改年號应乾,同年稍後又改元乾和。

然而劉晟掌握神器後,亦如其兄劉玢一樣猜忌諸弟,因此不久後就逐漸殺光其弟,並將他們的兒子殺死,女兒收入後宮。又興建離宮一千餘間,以珠寶裝飾,並設有許多殘酷的刑具,號「生地獄」。命宮女為女侍中,參與政事,由於宗室元勳幾乎剷除殆盡,當權者就是宦官、女官這些人而已。

刘晟生性荒淫暴虐,得志之后,专门用威势刑法统治下民,多诛灭旧臣以及自己的兄弟、侄子,将侄女收入后宫。数年之间,刘家被他差不多杀尽。又修造「活地狱」,大凡开水锅、铁烙床之类,无不齐备。人们犯有小的过失,就备受其刑罚之苦。到南楚的马家兄弟互动干戈时,刘晟趁此机会,派兵进攻桂林管区内各郡以及彬州、连州、梧州、贺州,都被攻克,从此全部拥有南越之地。

乾和六年(948年)劉晟派兵攻南楚,不久南楚內亂,乾和九年(951年)趁南楚為南唐所滅之際,占有南楚嶺南之地。

乾和十三年(955年),刘晟又杀其弟刘弘政,於是,刘龑的诸子被诛杀殆尽。

显德三年(956年),后周世宗柴刘晟忧虑万分。刘晟曾说过知晓占星术,同年六月在甘泉宫观天,牛女星间有月食,刘晟去對照占星之书,立即把書丢到地下,叹道:“自古以来,有谁能不死吗!”从此彻夜放纵饮酒。

乾和十六年(958年),在城北选定墓址,修建陵墓,刘晟亲自视察。同年秋去世,终年三十九岁,諡文武光聖明孝皇帝,廟號中宗,子劉鋹繼位。

\subsubsection{应乾}

\begin{longtable}{|>{\centering\scriptsize}m{2em}|>{\centering\scriptsize}m{1.3em}|>{\centering}m{8.8em}|}
  % \caption{秦王政}\
  \toprule
  \SimHei \normalsize 年数 & \SimHei \scriptsize 公元 & \SimHei 大事件 \tabularnewline
  % \midrule
  \endfirsthead
  \toprule
  \SimHei \normalsize 年数 & \SimHei \scriptsize 公元 & \SimHei 大事件 \tabularnewline
  \midrule
  \endhead
  \midrule
  元年 & 943 & \tabularnewline
  \bottomrule
\end{longtable}

\subsubsection{乾和}

\begin{longtable}{|>{\centering\scriptsize}m{2em}|>{\centering\scriptsize}m{1.3em}|>{\centering}m{8.8em}|}
  % \caption{秦王政}\
  \toprule
  \SimHei \normalsize 年数 & \SimHei \scriptsize 公元 & \SimHei 大事件 \tabularnewline
  % \midrule
  \endfirsthead
  \toprule
  \SimHei \normalsize 年数 & \SimHei \scriptsize 公元 & \SimHei 大事件 \tabularnewline
  \midrule
  \endhead
  \midrule
  元年 & 943 & \tabularnewline\hline
  二年 & 944 & \tabularnewline\hline
  三年 & 945 & \tabularnewline\hline
  四年 & 946 & \tabularnewline\hline
  五年 & 947 & \tabularnewline\hline
  六年 & 948 & \tabularnewline\hline
  七年 & 949 & \tabularnewline\hline
  八年 & 950 & \tabularnewline\hline
  九年 & 951 & \tabularnewline\hline
  十年 & 952 & \tabularnewline\hline
  十一年 & 953 & \tabularnewline\hline
  十二年 & 954 & \tabularnewline\hline
  十三年 & 955 & \tabularnewline\hline
  十四年 & 956 & \tabularnewline\hline
  十五年 & 957 & \tabularnewline\hline
  十六年 & 958 & \tabularnewline
  \bottomrule
\end{longtable}



%%% Local Variables:
%%% mode: latex
%%% TeX-engine: xetex
%%% TeX-master: "../../Main"
%%% End:

%% -*- coding: utf-8 -*-
%% Time-stamp: <Chen Wang: 2019-12-26 10:02:48>

\subsection{后主\tiny(958-971)}

\subsubsection{生平}

劉\xpinyin*{鋹}(942年-980年)原名劉繼興,五代十國時期南漢末代君主,是南漢中宗劉晟之長子,原封衛王。

南漢乾和十六年(958年)劉晟去世,劉繼興繼位,改名劉鋹,改元大寶。

劉鋹不會治國,政事皆委諸宦官龔澄樞及女侍中盧瓊仙等人,女官亦任命為參政官員,其餘官員只是聊備一格而已。

劉鋹又認為群臣都有家室,會為了顧及子孫不肯盡忠,因此只信任宦官,臣屬必須自宮才會被進用,以致於一度宦官高達二萬人之多。

又相當寵愛一名波斯女子,與之淫戲於後宮,叫她「媚豬」,而自稱「蕭閒大夫」,不理政事。後來將政事又交予女巫樊胡,連龔澄樞及盧瓊仙都依附她,政事紊亂。

南漢大寶十三年(970年),宋朝派潭州防禦使潘美攻南漢。南漢舊將先前多因讒言而被殺,宗室亦遭翦除殆盡,掌兵權的只有宦官而已,城牆、護城河,都裝飾為宮殿、水塘;樓船戰艦、武器盔甲,全部腐朽。

大寶十四年(宋開寶四年,971年),宋軍節節進逼。劉鋹纵火焚毁宫殿、府库,挑選十幾艘船,滿載金銀財寶及嬪妃,準備逃亡入海;還沒出發,宦官與衛兵就盜取船舶逃走,劉鋹只好投降,南漢亡。

劉鋹歸順宋朝後,押送至东京開封府,囚于玉津园,后以帛系颈,与南汉官属一同献俘于太庙、太社。宋太祖遣吕余庆问责焚烧府库之事。劉鋹將責任完全推給龔澄樞,其曰“臣年十六僭伪号,澄枢等皆先臣旧人,每事,臣不得自由,在国时,臣是臣下,澄枢是国主。”宋太祖趙匡胤就將龔澄樞斬首,而赦免劉鋹的罪,並任命其為金紫光禄大夫、检校太保、右千牛衛大將軍、员外置同正员,封恩赦侯。

荒淫無度的劉鋹投降後,為宋太祖、宋太宗厚待,也出現不少趣事,與南唐後主李煜的國愁家恨形成強烈對比。劉鋹本人體態豐滿,眉清目秀。有巧思,亦能言善辯,曾用珠子將馬鞍串成戲龍的形狀獻予宋太祖。宋太祖因此感嘆說:「劉鋹如果能將這項技藝用在治國上,怎麼會滅亡!」劉鋹稱帝在位时,多置鴆酒,毒死臣下。降宋後,一日宋太祖乘肩舆,与随从数十人幸讲武池。从官未至,而劉鋹先至。宋太祖賜以酒,劉鋹以為要毒殺自己,大哭曰:“臣承祖父基业,违拒朝廷,劳王师讨致,罪固当诛。陛下既待臣不死,愿为大梁布衣,观太平之盛。臣未敢饮此酒。”宋太祖笑而取酒自飲,劉鋹大感慚愧。

開寶八年(975年),宋滅南唐後,將劉鋹改命左監門衛上將軍,封彭城郡公。宋太宗即帝位,再改封其為衛國公。

太平興國四年(979年),宋太宗將伐北漢劉繼元,在長春殿宴請潘美等將領。當時劉鋹與已降宋的前吳越王錢俶、前平海军節度使陳洪進都參加,劉鋹因此說:「朝廷威靈遠播,四方竊位僭主的君王,今日都在座,不久又要平定太原,劉繼元又將到達,臣率先來朝,願揮舞大棒,替陛下吶喊助威,望成為各國降王的領袖。」宋太宗因此大笑。

太平興國五年(980年),劉鋹去世,獲贈授太師,追封為南越王。由於劉鋹是南漢最後一位君主,復無諡號、廟號,史家所以習稱其為南漢後主。

\subsubsection{大宝}

\begin{longtable}{|>{\centering\scriptsize}m{2em}|>{\centering\scriptsize}m{1.3em}|>{\centering}m{8.8em}|}
  % \caption{秦王政}\
  \toprule
  \SimHei \normalsize 年数 & \SimHei \scriptsize 公元 & \SimHei 大事件 \tabularnewline
  % \midrule
  \endfirsthead
  \toprule
  \SimHei \normalsize 年数 & \SimHei \scriptsize 公元 & \SimHei 大事件 \tabularnewline
  \midrule
  \endhead
  \midrule
  元年 & 958 & \tabularnewline\hline
  二年 & 959 & \tabularnewline\hline
  三年 & 960 & \tabularnewline\hline
  四年 & 961 & \tabularnewline\hline
  五年 & 962 & \tabularnewline\hline
  六年 & 963 & \tabularnewline\hline
  七年 & 964 & \tabularnewline\hline
  八年 & 965 & \tabularnewline\hline
  九年 & 966 & \tabularnewline\hline
  十年 & 967 & \tabularnewline\hline
  十一年 & 968 & \tabularnewline\hline
  十二年 & 969 & \tabularnewline\hline
  十三年 & 970 & \tabularnewline\hline
  十四年 & 971 & \tabularnewline
  \bottomrule
\end{longtable}



%%% Local Variables:
%%% mode: latex
%%% TeX-engine: xetex
%%% TeX-master: "../../Main"
%%% End:



%%% Local Variables:
%%% mode: latex
%%% TeX-engine: xetex
%%% TeX-master: "../../Main"
%%% End:
