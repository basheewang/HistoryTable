%% -*- coding: utf-8 -*-
%% Time-stamp: <Chen Wang: 2021-11-01 15:46:12>

\subsection{高祖劉龑\tiny(917-942)}

\subsubsection{生平}

漢高祖劉龑yǎn(889年-942年),原稱劉巖,又名刘陟。五代十国时期南汉开國皇帝,蔡州上蔡(今河南省上蔡县)人,郡望彭城(今江苏省徐州市),父亲是南汉追赠圣武皇帝刘知谦,母亲是妾段氏。

祖父劉安仁為蔡州上蔡(今河南上蔡)人,遷居福建,以經商為生,又因為生意需要,遷居嶺南,生刘知谦。唐朝末年,知谦從軍,受到當時的南海軍節度使韋宙賞識,與韋的姪女韦氏結婚,生下劉隱、劉臺兩個兒子。黃巢之亂出征有功,882年任封州刺史。

刘知谦后来又私纳小妾段氏,生下劉巖。正妻韦氏大怒,杀了段氏,但未忍伤害还是婴儿的劉巖,抱回家中和自己的两个儿子一起抚育。劉巖长大之後,聪慧又擅長武艺,且精通占卜算命之术,但又天性苛酷,每视杀人则喜,人皆以为蛟蜃化身。

知谦死後,長子劉隱擁兵自重,克肇慶、番禺(今屬广州),割据岭南地区,逐渐坐大,但劉隱向後梁稱臣納貢,後梁封為南海王,乾化元年(911年),劉隱还没来得及称帝,就因病在本郡去世,諡襄王。

劉隱有子,但劉巖篡奪其位,自命為留後,又襲位交趾节度使,其后又袭封南海王称号。劉隱死后六年(917年),劉巖在番禺称帝,建国号为「大越」, 改元为「乾亨」,定都番禺(今日廣州市)。次年,刘巖自称是汉朝皇室的后裔,为了表示自己建国是恢复昔日的汉家天下,于是又改国号为「大汉」,史称南汉。

後唐同光四年(926年),劉巖取《周易》中「飞龍在天」之意,改名为「劉龑」。據說他原先为自己改名为「劉龔」,後才自创一「龑」字。

除了劉龑及祖父劉安仁为上蔡(今河南省上蔡县)人的说法外,其侄女刘华的墓志称家族出于东晋时南渡的彭城刘氏,“而家于五羊,今为封州贺水人也”。藤田豐八认为他是大食商人后裔。藤田氏此说遭到另一位日本学者桑原骘藏的质疑。桑原认为,关于宋元祐间广州蕃坊娶宗女的刘姓人为阿拉伯人的说法,不能用以证明南汉刘氏是阿拉伯人的后裔,相反,宋元祐间广州蕃客的刘姓,可能是南汉所赐,“余谓唐代每以国姓赐外国人,此刘姓回民,或南汉刘氏赐与广州蕃客者,因为得姓之源欤?”

\subsubsection{乾亨}

\begin{longtable}{|>{\centering\scriptsize}m{2em}|>{\centering\scriptsize}m{1.3em}|>{\centering}m{8.8em}|}
  % \caption{秦王政}\
  \toprule
  \SimHei \normalsize 年数 & \SimHei \scriptsize 公元 & \SimHei 大事件 \tabularnewline
  % \midrule
  \endfirsthead
  \toprule
  \SimHei \normalsize 年数 & \SimHei \scriptsize 公元 & \SimHei 大事件 \tabularnewline
  \midrule
  \endhead
  \midrule
  元年 & 917 & \tabularnewline\hline
  二年 & 918 & \tabularnewline\hline
  三年 & 919 & \tabularnewline\hline
  四年 & 920 & \tabularnewline\hline
  五年 & 921 & \tabularnewline\hline
  六年 & 922 & \tabularnewline\hline
  七年 & 923 & \tabularnewline\hline
  八年 & 924 & \tabularnewline\hline
  九年 & 925 & \tabularnewline
  \bottomrule
\end{longtable}

\subsubsection{白龙}

\begin{longtable}{|>{\centering\scriptsize}m{2em}|>{\centering\scriptsize}m{1.3em}|>{\centering}m{8.8em}|}
  % \caption{秦王政}\
  \toprule
  \SimHei \normalsize 年数 & \SimHei \scriptsize 公元 & \SimHei 大事件 \tabularnewline
  % \midrule
  \endfirsthead
  \toprule
  \SimHei \normalsize 年数 & \SimHei \scriptsize 公元 & \SimHei 大事件 \tabularnewline
  \midrule
  \endhead
  \midrule
  元年 & 925 & \tabularnewline\hline
  二年 & 926 & \tabularnewline\hline
  三年 & 927 & \tabularnewline\hline
  四年 & 928 & \tabularnewline
  \bottomrule
\end{longtable}

\subsubsection{大有}

\begin{longtable}{|>{\centering\scriptsize}m{2em}|>{\centering\scriptsize}m{1.3em}|>{\centering}m{8.8em}|}
  % \caption{秦王政}\
  \toprule
  \SimHei \normalsize 年数 & \SimHei \scriptsize 公元 & \SimHei 大事件 \tabularnewline
  % \midrule
  \endfirsthead
  \toprule
  \SimHei \normalsize 年数 & \SimHei \scriptsize 公元 & \SimHei 大事件 \tabularnewline
  \midrule
  \endhead
  \midrule
  元年 & 928 & \tabularnewline\hline
  二年 & 929 & \tabularnewline\hline
  三年 & 930 & \tabularnewline\hline
  四年 & 931 & \tabularnewline\hline
  五年 & 932 & \tabularnewline\hline
  六年 & 933 & \tabularnewline\hline
  七年 & 934 & \tabularnewline\hline
  八年 & 935 & \tabularnewline\hline
  九年 & 936 & \tabularnewline\hline
  十年 & 937 & \tabularnewline\hline
  十一年 & 938 & \tabularnewline\hline
  十二年 & 939 & \tabularnewline\hline
  十三年 & 940 & \tabularnewline\hline
  十四年 & 941 & \tabularnewline\hline
  十五年 & 942 & \tabularnewline
  \bottomrule
\end{longtable}


%%% Local Variables:
%%% mode: latex
%%% TeX-engine: xetex
%%% TeX-master: "../../Main"
%%% End:
