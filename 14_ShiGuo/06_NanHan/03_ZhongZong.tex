%% -*- coding: utf-8 -*-
%% Time-stamp: <Chen Wang: 2019-12-26 10:02:01>

\subsection{中宗\tiny(943-958)}

\subsubsection{生平}

漢中宗劉\xpinyin*{晟}(920年-958年),原名劉弘熙,五代十國時期南漢君主,是南漢建立者劉龑之子,劉玢之弟。

南漢帝劉玢即位後驕傲奢侈,不理政事,荒淫無道,並且猜忌諸弟,原封晉王的劉弘熙因此有政變之意。南漢光天二年(943年),劉弘熙找來力士數人表演角力,與劉玢飲宴觀賞,當晚宴會結束,劉玢大醉,劉弘熙即命力士抓住劉玢,摧擊其前胸,斃命。翌晨,越王劉弘昌率諸弟至寢殿,迎劉弘熙即皇帝位,劉弘熙繼位後,改名劉晟,改年號应乾,同年稍後又改元乾和。

然而劉晟掌握神器後,亦如其兄劉玢一樣猜忌諸弟,因此不久後就逐漸殺光其弟,並將他們的兒子殺死,女兒收入後宮。又興建離宮一千餘間,以珠寶裝飾,並設有許多殘酷的刑具,號「生地獄」。命宮女為女侍中,參與政事,由於宗室元勳幾乎剷除殆盡,當權者就是宦官、女官這些人而已。

刘晟生性荒淫暴虐,得志之后,专门用威势刑法统治下民,多诛灭旧臣以及自己的兄弟、侄子,将侄女收入后宫。数年之间,刘家被他差不多杀尽。又修造「活地狱」,大凡开水锅、铁烙床之类,无不齐备。人们犯有小的过失,就备受其刑罚之苦。到南楚的马家兄弟互动干戈时,刘晟趁此机会,派兵进攻桂林管区内各郡以及彬州、连州、梧州、贺州,都被攻克,从此全部拥有南越之地。

乾和六年(948年)劉晟派兵攻南楚,不久南楚內亂,乾和九年(951年)趁南楚為南唐所滅之際,占有南楚嶺南之地。

乾和十三年(955年),刘晟又杀其弟刘弘政,於是,刘龑的诸子被诛杀殆尽。

显德三年(956年),后周世宗柴荣平定江北,刘晟这才恐慌,派人到后周朝贡,但被南楚所拦住,使者过不去,刘晟忧虑万分。刘晟曾说过知晓占星术,同年六月在甘泉宫观天,牛女星间有月食,刘晟去對照占星之书,立即把書丢到地下,叹道:“自古以来,有谁能不死吗!”从此彻夜放纵饮酒。

乾和十六年(958年),在城北选定墓址,修建陵墓,刘晟亲自视察。同年秋去世,终年三十九岁,諡文武光聖明孝皇帝,廟號中宗,子劉鋹繼位。

\subsubsection{应乾}

\begin{longtable}{|>{\centering\scriptsize}m{2em}|>{\centering\scriptsize}m{1.3em}|>{\centering}m{8.8em}|}
  % \caption{秦王政}\
  \toprule
  \SimHei \normalsize 年数 & \SimHei \scriptsize 公元 & \SimHei 大事件 \tabularnewline
  % \midrule
  \endfirsthead
  \toprule
  \SimHei \normalsize 年数 & \SimHei \scriptsize 公元 & \SimHei 大事件 \tabularnewline
  \midrule
  \endhead
  \midrule
  元年 & 943 & \tabularnewline
  \bottomrule
\end{longtable}

\subsubsection{乾和}

\begin{longtable}{|>{\centering\scriptsize}m{2em}|>{\centering\scriptsize}m{1.3em}|>{\centering}m{8.8em}|}
  % \caption{秦王政}\
  \toprule
  \SimHei \normalsize 年数 & \SimHei \scriptsize 公元 & \SimHei 大事件 \tabularnewline
  % \midrule
  \endfirsthead
  \toprule
  \SimHei \normalsize 年数 & \SimHei \scriptsize 公元 & \SimHei 大事件 \tabularnewline
  \midrule
  \endhead
  \midrule
  元年 & 943 & \tabularnewline\hline
  二年 & 944 & \tabularnewline\hline
  三年 & 945 & \tabularnewline\hline
  四年 & 946 & \tabularnewline\hline
  五年 & 947 & \tabularnewline\hline
  六年 & 948 & \tabularnewline\hline
  七年 & 949 & \tabularnewline\hline
  八年 & 950 & \tabularnewline\hline
  九年 & 951 & \tabularnewline\hline
  十年 & 952 & \tabularnewline\hline
  十一年 & 953 & \tabularnewline\hline
  十二年 & 954 & \tabularnewline\hline
  十三年 & 955 & \tabularnewline\hline
  十四年 & 956 & \tabularnewline\hline
  十五年 & 957 & \tabularnewline\hline
  十六年 & 958 & \tabularnewline
  \bottomrule
\end{longtable}



%%% Local Variables:
%%% mode: latex
%%% TeX-engine: xetex
%%% TeX-master: "../../Main"
%%% End:
