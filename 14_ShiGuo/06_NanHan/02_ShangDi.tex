%% -*- coding: utf-8 -*-
%% Time-stamp: <Chen Wang: 2019-12-26 10:01:32>

\subsection{殇帝\tiny(942-943)}

\subsubsection{生平}

漢殤帝劉玢(920年-943年),原名劉弘度,五代時期南漢君主,是南漢建立者劉龑之第三子,原封賓王,後改封秦王。

南漢大有十五年(942年),南漢高祖劉龑病重,原以劉弘度及晉王劉弘熙驕傲放縱,因此欲立幼子越王劉弘昌,為臣下所力諫,乃打消此意。不久,劉龑病逝,劉弘度遂繼位,改名劉玢,並改年號光天。

同年,循州(今廣東龍川)變民共推縣吏張遇賢為主,稱中天八國王,改元永樂,一時間南漢的東方州縣多被張遇賢所攻陷。

劉玢驕傲奢侈,荒淫無度。劉龑還在停靈的時條,就肆無忌憚地奏樂飲酒;夜晚則穿著黑色的喪服外出與娼妓混在一起;又命男女赤身裸體供其觀賞;政事廢弛,因此民變愈演愈烈。而左右有忤逆其意者動輒被處死,以致於除了劉弘昌及內常侍吳懷恩外,沒有人敢勸諫,而受勸諫後又不聽。劉玢亦猜忌諸弟及官員,所以叫宦官把守宮門,進宮時要將衣服脫掉檢查,才可以入內,劉弘熙遂生政變之意。

因為知道劉玢喜歡角力,所以南漢光天二年(943年),劉弘熙命陳道庠找來力士數人並告知劉玢,劉玢聽到後大為高興,即與諸王於長春宮聚會飲宴,並觀賞角力。當晚宴會結束,劉玢大醉,劉弘熙於是命力士抓住劉玢,摧擊其前胸斃命。劉玢死後,被諡殤帝。

\subsubsection{光天}

\begin{longtable}{|>{\centering\scriptsize}m{2em}|>{\centering\scriptsize}m{1.3em}|>{\centering}m{8.8em}|}
  % \caption{秦王政}\
  \toprule
  \SimHei \normalsize 年数 & \SimHei \scriptsize 公元 & \SimHei 大事件 \tabularnewline
  % \midrule
  \endfirsthead
  \toprule
  \SimHei \normalsize 年数 & \SimHei \scriptsize 公元 & \SimHei 大事件 \tabularnewline
  \midrule
  \endhead
  \midrule
  元年 & 942 & \tabularnewline\hline
  二年 & 943 & \tabularnewline
  \bottomrule
\end{longtable}



%%% Local Variables:
%%% mode: latex
%%% TeX-engine: xetex
%%% TeX-master: "../../Main"
%%% End:
