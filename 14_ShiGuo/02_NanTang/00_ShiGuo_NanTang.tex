%% -*- coding: utf-8 -*-
%% Time-stamp: <Chen Wang: 2019-12-24 17:53:23>


\section{南唐\tiny(937-975)}

\subsection{简介}

南唐(937年-975年)是五代十國的十國之一,定都金陵,歷時39年,有烈祖李昪、元宗李璟和後主李煜三位帝王。

南唐的成立可以追溯到吳國權臣徐溫的身上。徐溫原本是吳國(南吳)的開國功臣,後來他漸漸掌握了南吳的實權。他年老的時候,因亲子徐知训骄狂被杀、徐知询等年少能力不足,信任養子徐知誥,也漸給與他繼承人的地位。

徐溫去世後,徐知诰设计控制了徐知询,掌握了吴国的军政。937年,徐知诰代吴稱帝建國。根據《新五代史》、《舊五代史》、《南唐書》、《十國春秋》等史籍記載,徐知誥篡奪政權所建國號为齊,都金陵,号江宁府(今江苏南京),史家稱之為徐齊。939年,徐知誥回復自称的本姓李姓,並改名李昪,為了附會已滅亡的唐朝,把國號改為唐,以其位于南方,史称南唐。不過,《資治通鑑》卻記載937年徐知誥即帝位時,即以唐為國號,並不認為徐齊曾經存在過。后又称江南国。江淮地区的吴与后继的南唐国势强盛,他们采取联合北方契丹国制约中原的策略,屡次征讨周边国家壮大势力,成为中原王朝的一大威胁。

李昪称帝时期是南唐的盛世,經濟繁榮,文化昌盛。中主李璟時由於與周邊各國多次興兵,945年滅閩國、入侵楚國等使國力衰退,不但因此失去进取中原的良机,还因为策略不当等原因只占领了闽国的少部分地区,灭楚所得的土地更是都被武平军节度使刘言收复。

955年,周世宗發動后周攻南唐之战,南唐大敗,958年被迫將長江以北十四州割讓給後周,並且稱臣,奉后周年号,去帝號改稱唐国主,而周世宗称其为「江南國主」。宋朝建立后,南唐维持现状,用宋朝年号。961年,为应对宋朝巨大的军事压力迁都南昌府。后主李煜继位后,仍都江宁。

971年,宋朝改唐国主为江南国主,李煜去鸱吻,诸王降封为公。974年,後主李煜为保政权拒絕入朝,被宋讨伐,于是弃用宋年号,改以干支纪年。公元976年1月1日,宋灭南唐。李璟與李煜兩父子在中國文學上是有名的詞人。

南唐极盛时统治地区包括今江苏、安徽两省淮河以南、苏北东部、福建、江西、湖南、湖北东部。郑方坤称:“十国文物,首推南唐、西蜀。”马令《南唐书》卷十三《儒者传论》说:“五代之乱也,礼乐崩坏,文献俱亡,而儒衣书服,盛于南唐。岂斯文之未丧,而天将有所寓欤?不然,则圣王之大典,扫地尽矣。南唐累世好儒,而儒者之盛,见于载籍,灿然可观。如韩熙载之不羁,江文蔚之高才,徐锴之典赡,高越之华藻,潘佑之清逸,皆能擅价于一时,而徐铉、汤悦、张洎之徒,又足以争名于天下。其馀落落,不可胜数。故曰江左三十年间,文物有元和之风,岂虚言乎?”

%% -*- coding: utf-8 -*-
%% Time-stamp: <Chen Wang: 2019-12-24 17:54:33>

\subsection{烈祖\tiny(937-943)}

\subsubsection{生平}

唐烈祖李昪(889年1月7日-943年3月30日),字正倫,小字彭奴,五代十國時期南唐開國皇帝。海州人(今属江苏连云港),一说徐州人(今属江苏),另一说湖州安吉人(今属浙江),原稱「徐知誥」,是南吳大臣徐溫養子。

关于李昪的身世,历史上众说纷纭,莫衷一是,宋朝司马光《资治通鉴考异》就收录了四种不同的说法。

其中一种观点认为李昪是唐朝皇族后裔,持这种观点者以私修史著及杂史、稗史居多。据《十国春秋》总结,南唐灭亡后,南唐旧臣徐铉作《江南录》记录南唐历史,其中就提出李昪是唐宪宗第八子建王李恪的玄孙,释文莹《玉壶清话》采用了这种说法,李昪之孙李从浦墓志铭《宋故左龙武卫大将军李公墓志铭》也自称是建王李恪的后裔。陆游《南唐书》进一步提出了具体的世系是李恪生李超,李超生李荣,李荣生李昪,龙衮《江南野史》和马令《南唐书》世系谱与陆游书类似,但认为李超仅仅是李恪的后裔而非儿子,赵世延《南唐书序》,陈霆《唐余纪传》沿袭了这种观点;李昊《蜀后主实录》记载李昪是曾任岭南节度使的薛王李知柔之子,郑文宝《江表志》认为李昪是唐朝郑王的疏属支脉,陈彭年《江南别录》仅称之为唐之宗室,没有指明是谁的后代,《旧五代史》则记载李昪自称是唐玄宗之子永王李璘的后代。李昪自称唐朝皇室后裔,在五代十国时期就受到诟病,钱元瓘与沈韬文曾出言讽刺。

而宋代以来另有一种观点认为李昪祖先不过是平民,正史所持都是这种观点。《旧五代史》记载李昪仅仅是“自称”唐朝皇室后裔,《新五代史》同样是记载李昪“自称”建王李恪的玄孙,且称其出身微贱,而《资治通鉴》记载李昪打算以吴王李恪为祖先时曾有部下建议以郑王李元懿为祖先,李昪下令有关部门考察李恪和李元懿的后代,因为李恪的孙子李祎曾有军功,李祎的儿子李岘又做过宰相,于是李昪才以李恪为祖先,自称李岘下传五世到李昪的父亲李荣,这五世的名字大部分都是杜撰出来的,李昪又觉得唐朝经历了十九个皇帝历时三百年,怀疑自己的世系十代人太少,有关部门奏称一代人三十年,而李昪出生于唐僖宗文德年间,已经五十年了,李昪于是依从了他们。

另一种观点则见于钱俨《吴越备史》,其中称李昪之父本姓潘,因为被敌将李神福掳走而成为李神福的家奴,后徐温在李神福家见到李昪,对其十分惊异,遂请求收为养子。刘恕《十国纪年》认为李昪附会祖宗,不是唐朝宗室后裔,不过吴越与南唐是仇敌,《吴越备史》也非史实。李昪少年时就遭遇战乱成了孤儿,其祖先世系根本无法得知,李超、李志的名字都与徐温曾祖和祖父同名,完全是附会。

李昪,唐朝光启四年十二月二日(889年1月7日)生人[來源請求],小名李彭奴。父李荣性格谨厚,多游于佛寺。李彭奴六岁时其父於動亂中喪生,与伯父李球逃亡濠州。不久之后生母刘氏卒,遂托养于濠州开元寺。

乾宁二年,楊吳太祖杨行密攻濠州,得李彭奴,奇其相貌,欲收养为己子,而杨行密诸亲子以其身世微贱,不齿为兄弟。杨行密遂将李彭奴交给徐温为养子,遂改名为徐知誥。徐温妻子李氏以李彭奴与己为同姓,甚为爱护。

楊吳時期,徐知誥因功累升昇州防遏使、楼船使、昇州刺史、潤州團練使、检校司徒。徐知誥為政寬仁,又能節儉自處,獎勵農桑,因此府庫充實。當時,徐溫居昇州,並以長子徐知訓居南吳都城揚州控制南吳政權。天祐十五年(918年)徐知訓因驕傲荒淫為朱瑾所殺,徐知誥就近自潤州渡長江平變,自是徐溫乃以其为淮南节度行军副使、内外马步都军副使,代替徐知訓留揚州,日常政事皆由徐知誥處斷。

徐知誥在揚州,一反徐知訓的作為,恭敬事奉吳王楊隆演,並且謙卑對待士大夫。對待部屬寬大,生活儉僕,並以宋齊丘為謀士,改革稅制,因此國勢漸強,人心歸附。武义元年,徐知誥拜为左仆射、参知政事。顺义初年加封同平章事、领江州观察使、奉化军节度使。

南吳順義七年(927年)徐溫去世,徐知誥與徐溫親子徐知詢爭權,徐知誥趁徐知詢入朝的機會,將其扣留,自此完全掌握南吳政權。太和三年,徐知誥升为太尉、中书令、领镇海宁国诸军节度使,封东海郡王,出镇金陵。天祚元年(935年),加封尚父、太师、大丞相、天下兵马大元帅,进封齐王,以升州、润州、宣州、池州、歙州、常州、江州、饶州、信州、海州为齐国。徐知诰置百官,以金陵府为西都。

天祚三年(937年),徐知誥改名徐誥。同年,杨溥让位,南吴亡。徐誥即皇帝位,国号大齐,改年號昇元,以昇州金陵府(建康)为西都,扬州广陵府(江都)为东都。追尊徐温为太祖武皇帝。

昇元三年(939年)正月庚戌,江王徐知证、饶王徐知谔表奏,请徐誥恢復原姓,徐誥不许。正月癸亥,左丞相宋齐丘等人再次上表,乃允之。二月乙亥,徐誥自認是唐朝宗室,改国号为大唐,改徐温庙号为义祖。复李姓,初改自名为昂,犯唐文宗名讳;旋改名晃,又其与后梁太祖朱温同名,又改名为旦,犯唐睿宗庙讳。最终改名为昪。立天子七庙,以唐高祖、唐太宗、义祖徐温为不迁之祖。李昪由於家族譜系不詳,附會唐朝宗室,欲以唐朝吴王李恪为远祖,大臣奏以李恪被長孫無忌絞死,不如以郑王李元懿为祖。李昪命诸臣考二王苗裔,李恪之孙李祎有功,李祎之子李岘为宰相,遂以李恪为祖。创家谱,曰生父李荣,李荣之父李志,李志之父李超,李超之祖为李岘。其名字与官衔皆杜撰。当年三月,李昪下诏尊十世祖李恪为定宗孝静皇帝,曾祖李超为成宗孝平皇帝,祖李志为惠宗孝安皇帝,父李荣为庆宗孝德皇帝。但李昪孙李从镒墓志又认祖唐宪宗子建王李恪,未详孰是。

李昪登帝位後,改旧邸为崇德宫,正厅为光庆殿。又改东都文明殿为乾元殿、英武殿为明光殿、应乾殿为垂拱殿、朝阳殿为福昌殿、积庆宫为崇道宫;改西都崇英殿为延英殿、凝华前殿为昇元殿、后殿为雍和殿、兴祥殿为昭德殿、积庆殿为穆清殿。李昪勤於政事,並興利除弊,變更舊法。保境安民,與民休息。又與吳越和解,昇元五年吴越国大火,群臣请趁机攻打,而李昪称“奈何利人之灾!”遣使厚赠金帛慰问。吴越水灾,其民就食于南唐境内,李昪也遣官员赈济。

然而李昪崇尚道术,因服用丹藥中毒,個性變得暴躁易怒。昇元七年(943年),李昪服食方士史守冲所献“金丹”,背上生瘡,不久病情惡化,在昇元殿去世,终年五十六岁。临终前召子齐王李璟,嘱曰“德昌宫储戎器金帛七百余万,汝守成业,宜善交邻国,以保社稷。吾服金石,欲求延年,反以速死,汝宜视以为戒”;又啮齐王手指出血,称“他日北方必有事,勿忘吾言”。李璟繼位,上李昪谥号光文肅武孝神烈高皇帝,廟號烈祖,葬于永陵(后改陵号为钦陵)。

\subsubsection{昇元}

\begin{longtable}{|>{\centering\scriptsize}m{2em}|>{\centering\scriptsize}m{1.3em}|>{\centering}m{8.8em}|}
  % \caption{秦王政}\
  \toprule
  \SimHei \normalsize 年数 & \SimHei \scriptsize 公元 & \SimHei 大事件 \tabularnewline
  % \midrule
  \endfirsthead
  \toprule
  \SimHei \normalsize 年数 & \SimHei \scriptsize 公元 & \SimHei 大事件 \tabularnewline
  \midrule
  \endhead
  \midrule
  元年 & 937 & \tabularnewline\hline
  二年 & 938 & \tabularnewline\hline
  三年 & 939 & \tabularnewline\hline
  四年 & 940 & \tabularnewline\hline
  五年 & 941 & \tabularnewline\hline
  六年 & 942 & \tabularnewline\hline
  七年 & 943 & \tabularnewline
  \bottomrule
\end{longtable}


%%% Local Variables:
%%% mode: latex
%%% TeX-engine: xetex
%%% TeX-master: "../../Main"
%%% End:

%% -*- coding: utf-8 -*-
%% Time-stamp: <Chen Wang: 2019-12-25 10:22:18>

\subsection{元宗\tiny(943-961)}

\subsubsection{生平}

唐元宗李璟(916年-961年),字伯玉,原稱徐景通,南唐建立後,復本姓李,改名璟。對後周稱臣後,又為避後周信祖諱,而改名景。南唐烈祖李昪的長子。五代十国时期南唐第二位君主,因此也被称为中主、嗣主。李璟的书法頗佳,词亦有名,與其子李煜並稱「南唐二主」。其詞“小楼吹彻玉笙寒”是流芳千古的名句。作品被收入《南唐二主词》中。

昇元七年(943年)李昪過世,李璟繼位,改元保大。

李璟即位后,改变父皇李昪保守的政策,开始大规模对外用兵,消滅因繼承人爭位而內亂的马楚及闽国,他在位的大部分时間,南唐疆土最大。

李璟一心想著建功立業,但沒有治世之才。過度好大喜功的他不守父皇遺命,罷黜先皇時期的元老重臣,反而起用五個專事諂媚和自己興趣相投的佞臣──馮延巳、馮延魯、魏岑、陳覺、查文徽,史稱「南唐五鬼」。這幾人阿諛奉承、結黨營私,致使南唐政治陷入一片黑暗。

李璟改變先皇保境安民的國策,不斷地侵犯周邊國家,陸續攻滅閩、楚。雖然增加七州的新土地,但隨之而來反叛鬥爭,更令南唐疲於應付。與後周的兩次戰爭,消耗南唐大量庫存軍費,南唐戰敗,也使國力顯露頹敗之勢。

李璟奢侈无度,导致政治腐败,百姓民不聊生,怨声载道。

957年后周派兵侵入南唐,占领了南唐淮南江北的大片土地,并长驱直入到长江一带,迫近金陵,李璟只好向后周世宗柴榮称臣,去帝號,自稱唐國主,年號由原本的交泰改為後周的顯德。

961年8月12日卒,时年46岁,庙号烈宗,谥号为明道崇德文宣孝皇帝。

\subsubsection{保大}

\begin{longtable}{|>{\centering\scriptsize}m{2em}|>{\centering\scriptsize}m{1.3em}|>{\centering}m{8.8em}|}
  % \caption{秦王政}\
  \toprule
  \SimHei \normalsize 年数 & \SimHei \scriptsize 公元 & \SimHei 大事件 \tabularnewline
  % \midrule
  \endfirsthead
  \toprule
  \SimHei \normalsize 年数 & \SimHei \scriptsize 公元 & \SimHei 大事件 \tabularnewline
  \midrule
  \endhead
  \midrule
  元年 & 943 & \tabularnewline\hline
  二年 & 944 & \tabularnewline\hline
  三年 & 945 & \tabularnewline\hline
  四年 & 946 & \tabularnewline\hline
  五年 & 947 & \tabularnewline\hline
  六年 & 948 & \tabularnewline\hline
  七年 & 949 & \tabularnewline\hline
  八年 & 950 & \tabularnewline\hline
  九年 & 951 & \tabularnewline\hline
  十年 & 952 & \tabularnewline\hline
  十一年 & 953 & \tabularnewline\hline
  十二年 & 954 & \tabularnewline\hline
  十三年 & 955 & \tabularnewline\hline
  十四年 & 956 & \tabularnewline\hline
  十五年 & 957 & \tabularnewline
  \bottomrule
\end{longtable}

\subsubsection{中兴}

\begin{longtable}{|>{\centering\scriptsize}m{2em}|>{\centering\scriptsize}m{1.3em}|>{\centering}m{8.8em}|}
  % \caption{秦王政}\
  \toprule
  \SimHei \normalsize 年数 & \SimHei \scriptsize 公元 & \SimHei 大事件 \tabularnewline
  % \midrule
  \endfirsthead
  \toprule
  \SimHei \normalsize 年数 & \SimHei \scriptsize 公元 & \SimHei 大事件 \tabularnewline
  \midrule
  \endhead
  \midrule
  元年 & 958 & \tabularnewline
  \bottomrule
\end{longtable}

\subsubsection{交泰}

\begin{longtable}{|>{\centering\scriptsize}m{2em}|>{\centering\scriptsize}m{1.3em}|>{\centering}m{8.8em}|}
  % \caption{秦王政}\
  \toprule
  \SimHei \normalsize 年数 & \SimHei \scriptsize 公元 & \SimHei 大事件 \tabularnewline
  % \midrule
  \endfirsthead
  \toprule
  \SimHei \normalsize 年数 & \SimHei \scriptsize 公元 & \SimHei 大事件 \tabularnewline
  \midrule
  \endhead
  \midrule
  元年 & 958 & \tabularnewline
  \bottomrule
\end{longtable}

\subsubsection{显德}

\begin{longtable}{|>{\centering\scriptsize}m{2em}|>{\centering\scriptsize}m{1.3em}|>{\centering}m{8.8em}|}
  % \caption{秦王政}\
  \toprule
  \SimHei \normalsize 年数 & \SimHei \scriptsize 公元 & \SimHei 大事件 \tabularnewline
  % \midrule
  \endfirsthead
  \toprule
  \SimHei \normalsize 年数 & \SimHei \scriptsize 公元 & \SimHei 大事件 \tabularnewline
  \midrule
  \endhead
  \midrule
  元年 & 958 & \tabularnewline\hline
  二年 & 959 & \tabularnewline\hline
  三年 & 960 & \tabularnewline\hline
  四年 & 961 & \tabularnewline
  \bottomrule
\end{longtable}


%%% Local Variables:
%%% mode: latex
%%% TeX-engine: xetex
%%% TeX-master: "../../Main"
%%% End:

%% -*- coding: utf-8 -*-
%% Time-stamp: <Chen Wang: 2021-11-01 15:39:52>

\subsection{后主李煜\tiny(961-975)}

\subsubsection{生平}

李煜(937年8月15日-978年8月13日),或稱李後主,為南唐的末代君主,徐州人。李煜原名從嘉,字重光,號鍾山隱士、鍾峰隠者、白蓮居士、蓮峰居士等。史書描述其政治上毫无建树,李煜在南唐灭亡后被北宋俘虏,但是却成为了中国历史上首屈一指的词人,獲誉为「詞聖」、「千古詞帝」,作品千古流传。

李煜“为人仁孝,善属文,工书画,而廣顙丰额骈齿,一目重瞳子”,是南唐元宗(南唐中主)李璟的第六子。由于李璟的第二子到第五子均早死,故李煜长兄李弘冀为皇太子时,其为事实上的第二子。李弘冀“为人猜忌严刻”,时为安定公的李煜因而惶恐,不敢参与政事,每天只醉心研究典籍,以读书为乐。

959年李弘冀在毒死李景遂后不久亦死。李璟欲立李煜为太子,钟谟说“从嘉德轻志懦,又酷信释氏,非人主才。从善果敢凝重,宜为嗣。”李璟怒,将钟谟贬为国子监司,流放到饶州。封李煜为吴王、尚书令、知政事,令其住在东宫,就近學習處理政事。

宋建隆二年(961年),李璟迁都南昌并立李煜为太子、監國,令其留在金陵。六月李璟死后,李煜在金陵登基即位。李煜“性骄侈,好声色,又喜浮图,为高谈,不恤政事。”笃信佛教,“酷好浮屠,崇塔庙,度僧尼不可胜算。罢朝,辄造佛屋,易服膜拜,颇废政事。”在宫内和国内大兴宗教,甚至在军国大事上都以佛事为凭,自己每日穿袈裟诵佛经。直到宋軍临城下,李煜还在净居寺听和尚念经。[來源請求]

971年宋军灭南汉后,李煜为了表示他不对抗宋,对宋称臣,将自己的称呼改为江南国主,去鸱吻,诸王降封为公。

973年,宋太祖令李煜至汴京,李煜托病不往。974年,宋太祖遂派曹彬领军攻南唐。李煜因此弃用北宋年号,改用干支纪年。

12月,曹彬攻克金陵,南唐灭亡。李煜在位十五年,後世称李后主或南唐后主。

975年,李煜被俘后,在汴京被封为违命侯,拜左千牛卫将军。

976年,宋太祖逝世,弟赵光义继位为宋太宗,改封隴國公。嘗與金陵舊宮人書寫:「此中日夕,以淚珠洗面」。宋人笔记上說趙光義多次逼迫小周后侍寢。李煜在痛苦鬱悶中,寫下《望江南》、《子夜歌》、《虞美人》等名曲。

978年,徐铉奉宋太宗之命探视李煜,李煜對徐鉉叹息:“当初我错杀潘佑、李平,悔之不已!”徐铉退而告之,宋太宗闻之大怒。史載三年七月初七(978年8月13日),农历七夕,当李煜在其42岁生日那天与后妃们聚会,李煜卒,年四十二。一說李煜因寫“故国不堪回首月明中”、“恰似一江春水向東流”之词,宋太宗再也不能容忍,用牵机毒杀之。牽機藥或說是中藥馬錢子,其主要成分番木鳖碱有劇毒,服後會破壞中樞神經系統,全身抽搐,腳往腹部縮,頭亦彎至腹部,狀極痛苦。李煜死后,葬洛陽北邙山,小周后悲痛欲絕,不久也隨之死去。

李煜“生于深宫之中,长于妇人之手”,雖無力治國,然“性宽恕,威令不素著”,好生戒杀,性格出了名的善良,故在他死后,江南人闻之,“皆巷哭为斋”。

李煜在艺术方面具有很高的成就。劉毓盤说李后主“于富贵时能作富贵语,愁苦时能作愁苦语,无一字不真。”

李煜词本有集,已失传。现存词四十四首。其中几首前期作品或为他人所作,可以确定者仅三十八首。李煜的词的风格可以以975年被俘而分为两个时期:


李煜亡國前的詞,透插富麗奢華的宮廷生活,言詞多溫軟綺麗,卿卿我我,呈現「花間詞」氣息。根据内容可大致分为两类:一类是描写富丽堂皇的宫廷生活和风花雪月的男女情事,如《菩萨蛮》:“花明月暗籠輕霧,今宵好向郎邊去。剗袜步香階,手提金缕鞋。画堂南畔见,一晌偎人颤。奴為出来難,教君恣意憐。”又如《一斛珠》:“曉妝初過,沈檀輕注些兒個,向人微露丁香顆,一曲清歌,暫引櫻桃破。羅袖裛殘殷色可,杯深旋被香醪涴。繡床斜憑嬌無那,爛嚼紅茸,笑向檀郎唾。”

李煜亡國後,晚年的詞寫家國之恨,拓展了詞的題材,感慨既深,詞益悲壯。李煜詞最大特色,是自然真率,醇厚率真,情感真摯。喜用白描手法,通俗生動,語言精鍊而明淨洗煉,接近口語,與「花間詞」縷金刻翠,堆砌華麗詞藻的作風迥然不同。李煜后期的词由于生活的巨变,以一首首泣尽以血的绝唱,使亡国之君成为千古词坛的“南面王”(清沈雄《古今词话》语),正是“国家不幸诗家幸,话到沧桑语始工”。这些后期词作,凄凉悲壮,意境深远,为词史上承前启后的大宗师。至于其语句的清丽,音韵的和谐,更是空前绝后。如《破阵子》:“四十年来家国,三千里地山河。凤阁龙楼连霄汉,玉树琼枝作烟萝。几曾识干戈?一旦归为臣虏,沈腰潘鬓消磨。最是仓皇辞庙日,教坊犹奏别离歌。揮泪对宫娥。”《虞美人》:“春花秋月何时了,往事知多少。小楼昨夜又东风,故国不堪回首月明中。雕栏玉砌应犹在,只是朱颜改。问君能有几多愁,恰似一江春水向東流。”《浪淘沙令》:“帘外雨潺潺,春意阑珊。罗衾不耐五更寒,梦里不知身是客,一晌贪欢。独自莫凭栏,无限江山,别时容易见时难。流水落花春去也,天上人间。”

他能书善画,对其书法:陶穀《清異錄》曾云:“后主善书,作颤笔樛曲之状,遒劲如寒松霜竹,谓之‘金错刀’。作大字不事笔,卷帛书之,皆能如意,世谓‘撮襟书’。”。对其的画,宋代郭若虛的《图书见闻志》曰:“江南后主李煜,才识清赡,书画兼精。尝观所画林石、飞鸟,远过常流,高出意外。”。

歐陽修在《新五代史》中描述李煜:“煜字重光,初名從嘉,景第六子也。煜為人仁孝,善屬文,工書畫,而豐額駢齒,一目重瞳子。”

《漁隱叢話前集·西清詩話》提到宋太祖征服南唐统一中国后感叹:“李煜若以作诗词工夫治国家,岂为吾所俘也!”

近代学者王国维认为:“温飞卿之词,句秀也;韦端己之词,骨秀也;李重光之词,神秀也。”“词至李后主而眼界始大,感慨遂深,遂变伶工之词而为士大夫之词。周介存置诸温、韦之下,可谓颠倒黑白矣。”。此最后一句乃是针对周济在《介存斋论词杂著》中所道:“毛嫱、西施,天下美妇人也,严妆佳,淡妆亦佳,粗服乱头不掩国色。飞卿,严妆也;端己,淡妆也;后主,则粗服乱头矣。”王氏认为此评乃扬温、韦,抑后主。而学术界亦有观点认为,周济的本意是指李煜在词句的工整对仗等修饰方面不如温庭筠、韦庄,然而在词作的生动和流畅度方面,则前者显然更为生机勃发,浑然天成,“粗服乱头不掩国色”。

李煜词摆脱了《花间集》的浮靡,他的词不假雕饰,语言明快,形象生动,性格鲜明,用情真挚,亡国后作更是题材广阔,含意深沉,超过晚唐五代的词,不但成为宋初婉约派词的开山,也为豪放派打下基础,後世尊稱他為「詞聖」。

后代念及李煜的诗词中以清朝袁枚引《南唐雜詠》最有名:“作個才人真絕代,可憐薄命作君王。”

《宋史·潘慎修傳》記載:南唐滅亡後,一些南唐舊臣開始批評李煜為人愚昧懦弱,添油加醋地成份越來越多。宋真宗問潘慎修李煜是不是真的如此,潘慎修回答:「如果李煜真的這麼愚昧懦弱的話,他怎麼能治國十餘年?」

另一位南唐舊臣徐鉉在《大宋左千牛衛上將軍追封吴王隴西公墓誌銘》中評價李煜:「以厭兵之俗當用武之世,孔明罕應變之略,不成近功;偃王,躬仁義之行,終于亡國,道有所在,復何媿歟?」

徐铉:“王以世嫡嗣服,以古道驭民,钦若彝伦,率循先志。奉蒸尝、恭色养,必以孝;事耇老、宾大臣,必以礼。居处服御必以节,言动施舍必以时。至于荷全济之恩,谨藩国之度,勤修九贡,府无虚月,祗奉百役,知无不为。十五年间,天眷弥渥。”“精究六经,旁综百氏。常以周孔之道不可暂离,经国化民,发号施令,造次于是,始终不渝。”“酷好文辞,多所述作。一游一豫,必以颂宣。载笑载言,不忘经义。洞晓音律,精别雅郑;穷先王制作之意,审风俗淳薄之原,为文论之,以续《乐记》。所著文集三十卷,杂说百篇,味其文、知其道矣。至于弧矢之善,笔札之工,天纵多能,必造精绝。”“本以恻隐之性,仍好竺干之教。草木不杀,禽鱼咸遂。赏人之善,常若不及;掩人之过,惟恐其闻。以至法不胜奸,威不克爱。以厌兵之俗当用武之世,孔明罕应变之略,不成近功;偃王躬仁义之行,终于亡国。道有所在,复何愧欤!”

郑文宝:“后主奉竺乾之教,多不茹晕,常买禽鱼为放生。”“后主天性纯孝,孜孜儒学,虚怀接下,宾对大臣,倾奉中国,惟恐不及。但以著述勤于政事,至于书画皆尽精妙。然颇耽竺乾之教,果于自信,所以奸邪得计。排斥忠谠,土地曰削,贡举不充。越人肆谋,遂为敌国。又求援于北虏行人设谋,兵遂不解矣。”(《江表志》)

陆游:“后主天资纯孝......专以爱民为急,蠲赋息役,以裕民力。尊事中原,不惮卑屈,境内赖以少安者十有五年。”“然酷好浮屠,崇塔庙,度僧尼不可胜算。罢朝辄造佛屋,易服膜拜,以故颇废政事。兵兴之际,降御札移易将帅,大臣无知者。虽仁爱足以感其遗民,而卒不能保社稷。”(《南唐书·卷三·后主本纪第三》)

龙衮:“后主自少俊迈,喜肄儒学,工诗,能属文,晓悟音律。姿仪风雅,举止儒措,宛若士人。”(《江南野史·卷三后主、宜春王》)

陈彭年:“(后主煜)幼而好古,为文有汉魏风。”(《江南别录》)

欧阳修:“煜性骄侈,好声色,又喜浮图,为高谈,不恤政事。”

王世贞:“花间犹伤促碎,至南唐李王父子而妙矣。”(《弇州山人词评》)

胡应麟:“后主目重瞳子,乐府为宋人一代开山。盖温韦虽藻丽,而气颇伤促,意不胜辞。至此君方为当行作家,清便宛转,词家王、孟。”(《诗薮·杂篇》)

纳兰性德:“花间之词,如古玉器,贵重而不适用;宋词适用而少质重,李后主兼有其美,更饶烟水迷离之致。”(《渌水亭杂识·卷四》)

王夫之:“(李璟父子)无殃兆民,绝彝伦淫虐之巨惹。”“生聚完,文教兴,犹然彼都人士之余风也。”(《读通鉴论》)

余怀:“李重光风流才子,误作人主,至有入宋牵机之恨。其所作之词,一字一珠,非他家所能及也。”(《玉琴斋词·序》)

沈谦:“男中李后主,女中李易安,极是当行本色。”(徐釚《词苑丛谈》引语)“后主疏于治国,在词中犹不失南面王。”(沈雄《古今词话·词话》卷上引语)

郭麐:“作个才子真绝代,可怜薄命作君王。”(清代袁枚《随园诗话补遗》引郭麐《南唐杂咏》)

周济:“李后主词如生马驹,不受控捉。”“毛嫱西施,天下美妇人也。严妆佳,淡妆亦佳,粗服乱头,不掩国色。飞卿,严妆也;端己,淡妆也;后主则粗服乱头矣。”(《介存斋论词杂著》)

周之琦:“予谓重光天籁也,恐非人力所及。”

陈廷焯:“后主词思路凄惋,词场本色,不及飞卿之厚,自胜牛松卿辈。”“余尝谓后主之视飞卿,合而离者也;端己之视飞卿,离而合者也。”“李后主、晏叔原,皆非词中正声,而其词无人不爱,以其情胜也。”(《白雨斋词话·卷一》)

王鹏运:“莲峰居士(李煜)词,超逸绝伦,虚灵在骨。芝兰空谷,未足比其芳华;笙鹤瑶天,讵能方兹清怨?后起之秀,格调气韵之间,或月日至,得十一于千首。若小晏、若徽庙,其殆庶几。断代南流,嗣音阒然,盖间气所钟,以谓词中之大成者,当之无愧色矣。”(《半塘老人遣稿》)

冯煦:“词至南唐,二主作于上,正中和于下,诣微造极,得未曾有。宋初诸家,靡不祖述二主。”(《宋六十一家词选·例言》)

王国维:“温飞卿之词,句秀也;韦端己之词,骨秀也;李重光之词,神秀也。”“词至李后主而眼界始大,感慨遂深,遂变伶工之词而为士大夫之词。”“词人者,不失其赤子之心者也。故生于深宫之中,长于妇人之手,是后主为人君所短处,亦即为词人所长处。”“主观之诗人,不必多阅世,阅世愈浅,则性情愈真,李后主是也。”“尼采谓一切文字,余爱以血书者,后主之词,真所谓以血书者也。宋道君皇帝《燕山亭》词,亦略似之。然道君不过自道身世之感,后主则俨有释迦、基督担荷人类罪恶之意,其大小固不同矣。”“唐五代之词,有句而无篇;南宋名家之词,有篇而无句。有篇有句,唯李后主之作及永叔、少游、美成、稼轩数人而已。”(《人间词话》)

毛泽东:“南唐李后主虽多才多艺,但不抓政治,终于亡国。”(毛泽东评价历史人物)

柏杨:“南唐皇帝李煜先生词学的造诣,空前绝后,用在填词上的精力,远超过用在治国上。”(《浊世人间》)

叶嘉莹:“李后主的词是他对生活的敏锐而真切的体验,无论是享乐的欢愉,还是悲哀的痛苦,他都全身心的投入其间。我们有的人活过一生,既没有好好的体会过快乐,也没有好好的体验过悲哀,因为他从来没有以全部的心灵感情投注入某一件事,这是人生的遗憾。”(《唐宋名家词赏析》)

\subsubsection{显德}

\begin{longtable}{|>{\centering\scriptsize}m{2em}|>{\centering\scriptsize}m{1.3em}|>{\centering}m{8.8em}|}
  % \caption{秦王政}\
  \toprule
  \SimHei \normalsize 年数 & \SimHei \scriptsize 公元 & \SimHei 大事件 \tabularnewline
  % \midrule
  \endfirsthead
  \toprule
  \SimHei \normalsize 年数 & \SimHei \scriptsize 公元 & \SimHei 大事件 \tabularnewline
  \midrule
  \endhead
  \midrule
  元年 & 961 & \tabularnewline\hline
  二年 & 962 & \tabularnewline
  \bottomrule
\end{longtable}

\subsubsection{建隆}

\begin{longtable}{|>{\centering\scriptsize}m{2em}|>{\centering\scriptsize}m{1.3em}|>{\centering}m{8.8em}|}
  % \caption{秦王政}\
  \toprule
  \SimHei \normalsize 年数 & \SimHei \scriptsize 公元 & \SimHei 大事件 \tabularnewline
  % \midrule
  \endfirsthead
  \toprule
  \SimHei \normalsize 年数 & \SimHei \scriptsize 公元 & \SimHei 大事件 \tabularnewline
  \midrule
  \endhead
  \midrule
  元年 & 963 & \tabularnewline
  \bottomrule
\end{longtable}

\subsubsection{乾德}

\begin{longtable}{|>{\centering\scriptsize}m{2em}|>{\centering\scriptsize}m{1.3em}|>{\centering}m{8.8em}|}
  % \caption{秦王政}\
  \toprule
  \SimHei \normalsize 年数 & \SimHei \scriptsize 公元 & \SimHei 大事件 \tabularnewline
  % \midrule
  \endfirsthead
  \toprule
  \SimHei \normalsize 年数 & \SimHei \scriptsize 公元 & \SimHei 大事件 \tabularnewline
  \midrule
  \endhead
  \midrule
  元年 & 963 & \tabularnewline\hline
  二年 & 964 & \tabularnewline\hline
  三年 & 965 & \tabularnewline\hline
  四年 & 966 & \tabularnewline\hline
  五年 & 967 & \tabularnewline\hline
  六年 & 968 & \tabularnewline
  \bottomrule
\end{longtable}

\subsubsection{开宝}

\begin{longtable}{|>{\centering\scriptsize}m{2em}|>{\centering\scriptsize}m{1.3em}|>{\centering}m{8.8em}|}
  % \caption{秦王政}\
  \toprule
  \SimHei \normalsize 年数 & \SimHei \scriptsize 公元 & \SimHei 大事件 \tabularnewline
  % \midrule
  \endfirsthead
  \toprule
  \SimHei \normalsize 年数 & \SimHei \scriptsize 公元 & \SimHei 大事件 \tabularnewline
  \midrule
  \endhead
  \midrule
  元年 & 968 & \tabularnewline\hline
  二年 & 969 & \tabularnewline\hline
  三年 & 970 & \tabularnewline\hline
  四年 & 971 & \tabularnewline\hline
  五年 & 972 & \tabularnewline\hline
  六年 & 973 & \tabularnewline\hline
  七年 & 974 & \tabularnewline\hline
  八年 & 975 & \tabularnewline
  \bottomrule
\end{longtable}


%%% Local Variables:
%%% mode: latex
%%% TeX-engine: xetex
%%% TeX-master: "../../Main"
%%% End:



%%% Local Variables:
%%% mode: latex
%%% TeX-engine: xetex
%%% TeX-master: "../../Main"
%%% End:
