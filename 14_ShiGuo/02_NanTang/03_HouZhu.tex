%% -*- coding: utf-8 -*-
%% Time-stamp: <Chen Wang: 2021-11-01 15:39:52>

\subsection{后主李煜\tiny(961-975)}

\subsubsection{生平}

李煜(937年8月15日-978年8月13日),或稱李後主,為南唐的末代君主,徐州人。李煜原名從嘉,字重光,號鍾山隱士、鍾峰隠者、白蓮居士、蓮峰居士等。史書描述其政治上毫无建树,李煜在南唐灭亡后被北宋俘虏,但是却成为了中国历史上首屈一指的词人,獲誉为「詞聖」、「千古詞帝」,作品千古流传。

李煜“为人仁孝,善属文,工书画,而廣顙丰额骈齿,一目重瞳子”,是南唐元宗(南唐中主)李璟的第六子。由于李璟的第二子到第五子均早死,故李煜长兄李弘冀为皇太子时,其为事实上的第二子。李弘冀“为人猜忌严刻”,时为安定公的李煜因而惶恐,不敢参与政事,每天只醉心研究典籍,以读书为乐。

959年李弘冀在毒死李景遂后不久亦死。李璟欲立李煜为太子,钟谟说“从嘉德轻志懦,又酷信释氏,非人主才。从善果敢凝重,宜为嗣。”李璟怒,将钟谟贬为国子监司,流放到饶州。封李煜为吴王、尚书令、知政事,令其住在东宫,就近學習處理政事。

宋建隆二年(961年),李璟迁都南昌并立李煜为太子、監國,令其留在金陵。六月李璟死后,李煜在金陵登基即位。李煜“性骄侈,好声色,又喜浮图,为高谈,不恤政事。”笃信佛教,“酷好浮屠,崇塔庙,度僧尼不可胜算。罢朝,辄造佛屋,易服膜拜,颇废政事。”在宫内和国内大兴宗教,甚至在军国大事上都以佛事为凭,自己每日穿袈裟诵佛经。直到宋軍临城下,李煜还在净居寺听和尚念经。[來源請求]

971年宋军灭南汉后,李煜为了表示他不对抗宋,对宋称臣,将自己的称呼改为江南国主,去鸱吻,诸王降封为公。

973年,宋太祖令李煜至汴京,李煜托病不往。974年,宋太祖遂派曹彬领军攻南唐。李煜因此弃用北宋年号,改用干支纪年。

12月,曹彬攻克金陵,南唐灭亡。李煜在位十五年,後世称李后主或南唐后主。

975年,李煜被俘后,在汴京被封为违命侯,拜左千牛卫将军。

976年,宋太祖逝世,弟赵光义继位为宋太宗,改封隴國公。嘗與金陵舊宮人書寫:「此中日夕,以淚珠洗面」。宋人笔记上說趙光義多次逼迫小周后侍寢。李煜在痛苦鬱悶中,寫下《望江南》、《子夜歌》、《虞美人》等名曲。

978年,徐铉奉宋太宗之命探视李煜,李煜對徐鉉叹息:“当初我错杀潘佑、李平,悔之不已!”徐铉退而告之,宋太宗闻之大怒。史載三年七月初七(978年8月13日),农历七夕,当李煜在其42岁生日那天与后妃们聚会,李煜卒,年四十二。一說李煜因寫“故国不堪回首月明中”、“恰似一江春水向東流”之词,宋太宗再也不能容忍,用牵机毒杀之。牽機藥或說是中藥馬錢子,其主要成分番木鳖碱有劇毒,服後會破壞中樞神經系統,全身抽搐,腳往腹部縮,頭亦彎至腹部,狀極痛苦。李煜死后,葬洛陽北邙山,小周后悲痛欲絕,不久也隨之死去。

李煜“生于深宫之中,长于妇人之手”,雖無力治國,然“性宽恕,威令不素著”,好生戒杀,性格出了名的善良,故在他死后,江南人闻之,“皆巷哭为斋”。

李煜在艺术方面具有很高的成就。劉毓盤说李后主“于富贵时能作富贵语,愁苦时能作愁苦语,无一字不真。”

李煜词本有集,已失传。现存词四十四首。其中几首前期作品或为他人所作,可以确定者仅三十八首。李煜的词的风格可以以975年被俘而分为两个时期:


李煜亡國前的詞,透插富麗奢華的宮廷生活,言詞多溫軟綺麗,卿卿我我,呈現「花間詞」氣息。根据内容可大致分为两类:一类是描写富丽堂皇的宫廷生活和风花雪月的男女情事,如《菩萨蛮》:“花明月暗籠輕霧,今宵好向郎邊去。剗袜步香階,手提金缕鞋。画堂南畔见,一晌偎人颤。奴為出来難,教君恣意憐。”又如《一斛珠》:“曉妝初過,沈檀輕注些兒個,向人微露丁香顆,一曲清歌,暫引櫻桃破。羅袖裛殘殷色可,杯深旋被香醪涴。繡床斜憑嬌無那,爛嚼紅茸,笑向檀郎唾。”

李煜亡國後,晚年的詞寫家國之恨,拓展了詞的題材,感慨既深,詞益悲壯。李煜詞最大特色,是自然真率,醇厚率真,情感真摯。喜用白描手法,通俗生動,語言精鍊而明淨洗煉,接近口語,與「花間詞」縷金刻翠,堆砌華麗詞藻的作風迥然不同。李煜后期的词由于生活的巨变,以一首首泣尽以血的绝唱,使亡国之君成为千古词坛的“南面王”(清沈雄《古今词话》语),正是“国家不幸诗家幸,话到沧桑语始工”。这些后期词作,凄凉悲壮,意境深远,为词史上承前启后的大宗师。至于其语句的清丽,音韵的和谐,更是空前绝后。如《破阵子》:“四十年来家国,三千里地山河。凤阁龙楼连霄汉,玉树琼枝作烟萝。几曾识干戈?一旦归为臣虏,沈腰潘鬓消磨。最是仓皇辞庙日,教坊犹奏别离歌。揮泪对宫娥。”《虞美人》:“春花秋月何时了,往事知多少。小楼昨夜又东风,故国不堪回首月明中。雕栏玉砌应犹在,只是朱颜改。问君能有几多愁,恰似一江春水向東流。”《浪淘沙令》:“帘外雨潺潺,春意阑珊。罗衾不耐五更寒,梦里不知身是客,一晌贪欢。独自莫凭栏,无限江山,别时容易见时难。流水落花春去也,天上人间。”

他能书善画,对其书法:陶穀《清異錄》曾云:“后主善书,作颤笔樛曲之状,遒劲如寒松霜竹,谓之‘金错刀’。作大字不事笔,卷帛书之,皆能如意,世谓‘撮襟书’。”。对其的画,宋代郭若虛的《图书见闻志》曰:“江南后主李煜,才识清赡,书画兼精。尝观所画林石、飞鸟,远过常流,高出意外。”。

歐陽修在《新五代史》中描述李煜:“煜字重光,初名從嘉,景第六子也。煜為人仁孝,善屬文,工書畫,而豐額駢齒,一目重瞳子。”

《漁隱叢話前集·西清詩話》提到宋太祖征服南唐统一中国后感叹:“李煜若以作诗词工夫治国家,岂为吾所俘也!”

近代学者王国维认为:“温飞卿之词,句秀也;韦端己之词,骨秀也;李重光之词,神秀也。”“词至李后主而眼界始大,感慨遂深,遂变伶工之词而为士大夫之词。周介存置诸温、韦之下,可谓颠倒黑白矣。”。此最后一句乃是针对周济在《介存斋论词杂著》中所道:“毛嫱、西施,天下美妇人也,严妆佳,淡妆亦佳,粗服乱头不掩国色。飞卿,严妆也;端己,淡妆也;后主,则粗服乱头矣。”王氏认为此评乃扬温、韦,抑后主。而学术界亦有观点认为,周济的本意是指李煜在词句的工整对仗等修饰方面不如温庭筠、韦庄,然而在词作的生动和流畅度方面,则前者显然更为生机勃发,浑然天成,“粗服乱头不掩国色”。

李煜词摆脱了《花间集》的浮靡,他的词不假雕饰,语言明快,形象生动,性格鲜明,用情真挚,亡国后作更是题材广阔,含意深沉,超过晚唐五代的词,不但成为宋初婉约派词的开山,也为豪放派打下基础,後世尊稱他為「詞聖」。

后代念及李煜的诗词中以清朝袁枚引《南唐雜詠》最有名:“作個才人真絕代,可憐薄命作君王。”

《宋史·潘慎修傳》記載:南唐滅亡後,一些南唐舊臣開始批評李煜為人愚昧懦弱,添油加醋地成份越來越多。宋真宗問潘慎修李煜是不是真的如此,潘慎修回答:「如果李煜真的這麼愚昧懦弱的話,他怎麼能治國十餘年?」

另一位南唐舊臣徐鉉在《大宋左千牛衛上將軍追封吴王隴西公墓誌銘》中評價李煜:「以厭兵之俗當用武之世,孔明罕應變之略,不成近功;偃王,躬仁義之行,終于亡國,道有所在,復何媿歟?」

徐铉:“王以世嫡嗣服,以古道驭民,钦若彝伦,率循先志。奉蒸尝、恭色养,必以孝;事耇老、宾大臣,必以礼。居处服御必以节,言动施舍必以时。至于荷全济之恩,谨藩国之度,勤修九贡,府无虚月,祗奉百役,知无不为。十五年间,天眷弥渥。”“精究六经,旁综百氏。常以周孔之道不可暂离,经国化民,发号施令,造次于是,始终不渝。”“酷好文辞,多所述作。一游一豫,必以颂宣。载笑载言,不忘经义。洞晓音律,精别雅郑;穷先王制作之意,审风俗淳薄之原,为文论之,以续《乐记》。所著文集三十卷,杂说百篇,味其文、知其道矣。至于弧矢之善,笔札之工,天纵多能,必造精绝。”“本以恻隐之性,仍好竺干之教。草木不杀,禽鱼咸遂。赏人之善,常若不及;掩人之过,惟恐其闻。以至法不胜奸,威不克爱。以厌兵之俗当用武之世,孔明罕应变之略,不成近功;偃王躬仁义之行,终于亡国。道有所在,复何愧欤!”

郑文宝:“后主奉竺乾之教,多不茹晕,常买禽鱼为放生。”“后主天性纯孝,孜孜儒学,虚怀接下,宾对大臣,倾奉中国,惟恐不及。但以著述勤于政事,至于书画皆尽精妙。然颇耽竺乾之教,果于自信,所以奸邪得计。排斥忠谠,土地曰削,贡举不充。越人肆谋,遂为敌国。又求援于北虏行人设谋,兵遂不解矣。”(《江表志》)

陆游:“后主天资纯孝......专以爱民为急,蠲赋息役,以裕民力。尊事中原,不惮卑屈,境内赖以少安者十有五年。”“然酷好浮屠,崇塔庙,度僧尼不可胜算。罢朝辄造佛屋,易服膜拜,以故颇废政事。兵兴之际,降御札移易将帅,大臣无知者。虽仁爱足以感其遗民,而卒不能保社稷。”(《南唐书·卷三·后主本纪第三》)

龙衮:“后主自少俊迈,喜肄儒学,工诗,能属文,晓悟音律。姿仪风雅,举止儒措,宛若士人。”(《江南野史·卷三后主、宜春王》)

陈彭年:“(后主煜)幼而好古,为文有汉魏风。”(《江南别录》)

欧阳修:“煜性骄侈,好声色,又喜浮图,为高谈,不恤政事。”

王世贞:“花间犹伤促碎,至南唐李王父子而妙矣。”(《弇州山人词评》)

胡应麟:“后主目重瞳子,乐府为宋人一代开山。盖温韦虽藻丽,而气颇伤促,意不胜辞。至此君方为当行作家,清便宛转,词家王、孟。”(《诗薮·杂篇》)

纳兰性德:“花间之词,如古玉器,贵重而不适用;宋词适用而少质重,李后主兼有其美,更饶烟水迷离之致。”(《渌水亭杂识·卷四》)

王夫之:“(李璟父子)无殃兆民,绝彝伦淫虐之巨惹。”“生聚完,文教兴,犹然彼都人士之余风也。”(《读通鉴论》)

余怀:“李重光风流才子,误作人主,至有入宋牵机之恨。其所作之词,一字一珠,非他家所能及也。”(《玉琴斋词·序》)

沈谦:“男中李后主,女中李易安,极是当行本色。”(徐釚《词苑丛谈》引语)“后主疏于治国,在词中犹不失南面王。”(沈雄《古今词话·词话》卷上引语)

郭麐:“作个才子真绝代,可怜薄命作君王。”(清代袁枚《随园诗话补遗》引郭麐《南唐杂咏》)

周济:“李后主词如生马驹,不受控捉。”“毛嫱西施,天下美妇人也。严妆佳,淡妆亦佳,粗服乱头,不掩国色。飞卿,严妆也;端己,淡妆也;后主则粗服乱头矣。”(《介存斋论词杂著》)

周之琦:“予谓重光天籁也,恐非人力所及。”

陈廷焯:“后主词思路凄惋,词场本色,不及飞卿之厚,自胜牛松卿辈。”“余尝谓后主之视飞卿,合而离者也;端己之视飞卿,离而合者也。”“李后主、晏叔原,皆非词中正声,而其词无人不爱,以其情胜也。”(《白雨斋词话·卷一》)

王鹏运:“莲峰居士(李煜)词,超逸绝伦,虚灵在骨。芝兰空谷,未足比其芳华;笙鹤瑶天,讵能方兹清怨?后起之秀,格调气韵之间,或月日至,得十一于千首。若小晏、若徽庙,其殆庶几。断代南流,嗣音阒然,盖间气所钟,以谓词中之大成者,当之无愧色矣。”(《半塘老人遣稿》)

冯煦:“词至南唐,二主作于上,正中和于下,诣微造极,得未曾有。宋初诸家,靡不祖述二主。”(《宋六十一家词选·例言》)

王国维:“温飞卿之词,句秀也;韦端己之词,骨秀也;李重光之词,神秀也。”“词至李后主而眼界始大,感慨遂深,遂变伶工之词而为士大夫之词。”“词人者,不失其赤子之心者也。故生于深宫之中,长于妇人之手,是后主为人君所短处,亦即为词人所长处。”“主观之诗人,不必多阅世,阅世愈浅,则性情愈真,李后主是也。”“尼采谓一切文字,余爱以血书者,后主之词,真所谓以血书者也。宋道君皇帝《燕山亭》词,亦略似之。然道君不过自道身世之感,后主则俨有释迦、基督担荷人类罪恶之意,其大小固不同矣。”“唐五代之词,有句而无篇;南宋名家之词,有篇而无句。有篇有句,唯李后主之作及永叔、少游、美成、稼轩数人而已。”(《人间词话》)

毛泽东:“南唐李后主虽多才多艺,但不抓政治,终于亡国。”(毛泽东评价历史人物)

柏杨:“南唐皇帝李煜先生词学的造诣,空前绝后,用在填词上的精力,远超过用在治国上。”(《浊世人间》)

叶嘉莹:“李后主的词是他对生活的敏锐而真切的体验,无论是享乐的欢愉,还是悲哀的痛苦,他都全身心的投入其间。我们有的人活过一生,既没有好好的体会过快乐,也没有好好的体验过悲哀,因为他从来没有以全部的心灵感情投注入某一件事,这是人生的遗憾。”(《唐宋名家词赏析》)

\subsubsection{显德}

\begin{longtable}{|>{\centering\scriptsize}m{2em}|>{\centering\scriptsize}m{1.3em}|>{\centering}m{8.8em}|}
  % \caption{秦王政}\
  \toprule
  \SimHei \normalsize 年数 & \SimHei \scriptsize 公元 & \SimHei 大事件 \tabularnewline
  % \midrule
  \endfirsthead
  \toprule
  \SimHei \normalsize 年数 & \SimHei \scriptsize 公元 & \SimHei 大事件 \tabularnewline
  \midrule
  \endhead
  \midrule
  元年 & 961 & \tabularnewline\hline
  二年 & 962 & \tabularnewline
  \bottomrule
\end{longtable}

\subsubsection{建隆}

\begin{longtable}{|>{\centering\scriptsize}m{2em}|>{\centering\scriptsize}m{1.3em}|>{\centering}m{8.8em}|}
  % \caption{秦王政}\
  \toprule
  \SimHei \normalsize 年数 & \SimHei \scriptsize 公元 & \SimHei 大事件 \tabularnewline
  % \midrule
  \endfirsthead
  \toprule
  \SimHei \normalsize 年数 & \SimHei \scriptsize 公元 & \SimHei 大事件 \tabularnewline
  \midrule
  \endhead
  \midrule
  元年 & 963 & \tabularnewline
  \bottomrule
\end{longtable}

\subsubsection{乾德}

\begin{longtable}{|>{\centering\scriptsize}m{2em}|>{\centering\scriptsize}m{1.3em}|>{\centering}m{8.8em}|}
  % \caption{秦王政}\
  \toprule
  \SimHei \normalsize 年数 & \SimHei \scriptsize 公元 & \SimHei 大事件 \tabularnewline
  % \midrule
  \endfirsthead
  \toprule
  \SimHei \normalsize 年数 & \SimHei \scriptsize 公元 & \SimHei 大事件 \tabularnewline
  \midrule
  \endhead
  \midrule
  元年 & 963 & \tabularnewline\hline
  二年 & 964 & \tabularnewline\hline
  三年 & 965 & \tabularnewline\hline
  四年 & 966 & \tabularnewline\hline
  五年 & 967 & \tabularnewline\hline
  六年 & 968 & \tabularnewline
  \bottomrule
\end{longtable}

\subsubsection{开宝}

\begin{longtable}{|>{\centering\scriptsize}m{2em}|>{\centering\scriptsize}m{1.3em}|>{\centering}m{8.8em}|}
  % \caption{秦王政}\
  \toprule
  \SimHei \normalsize 年数 & \SimHei \scriptsize 公元 & \SimHei 大事件 \tabularnewline
  % \midrule
  \endfirsthead
  \toprule
  \SimHei \normalsize 年数 & \SimHei \scriptsize 公元 & \SimHei 大事件 \tabularnewline
  \midrule
  \endhead
  \midrule
  元年 & 968 & \tabularnewline\hline
  二年 & 969 & \tabularnewline\hline
  三年 & 970 & \tabularnewline\hline
  四年 & 971 & \tabularnewline\hline
  五年 & 972 & \tabularnewline\hline
  六年 & 973 & \tabularnewline\hline
  七年 & 974 & \tabularnewline\hline
  八年 & 975 & \tabularnewline
  \bottomrule
\end{longtable}


%%% Local Variables:
%%% mode: latex
%%% TeX-engine: xetex
%%% TeX-master: "../../Main"
%%% End:
