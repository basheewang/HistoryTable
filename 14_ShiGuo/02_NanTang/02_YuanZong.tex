%% -*- coding: utf-8 -*-
%% Time-stamp: <Chen Wang: 2021-11-01 15:39:47>

\subsection{元宗李璟\tiny(943-961)}

\subsubsection{生平}

唐元宗李璟(916年-961年),字伯玉,原稱徐景通,南唐建立後,復本姓李,改名璟。對後周稱臣後,又為避後周信祖諱,而改名景。南唐烈祖李昪的長子。五代十国时期南唐第二位君主,因此也被称为中主、嗣主。李璟的书法頗佳,词亦有名,與其子李煜並稱「南唐二主」。其詞“小楼吹彻玉笙寒”是流芳千古的名句。作品被收入《南唐二主词》中。

昇元七年(943年)李昪過世,李璟繼位,改元保大。

李璟即位后,改变父皇李昪保守的政策,开始大规模对外用兵,消滅因繼承人爭位而內亂的马楚及闽国,他在位的大部分时間,南唐疆土最大。

李璟一心想著建功立業,但沒有治世之才。過度好大喜功的他不守父皇遺命,罷黜先皇時期的元老重臣,反而起用五個專事諂媚和自己興趣相投的佞臣──馮延巳、馮延魯、魏岑、陳覺、查文徽,史稱「南唐五鬼」。這幾人阿諛奉承、結黨營私,致使南唐政治陷入一片黑暗。

李璟改變先皇保境安民的國策,不斷地侵犯周邊國家,陸續攻滅閩、楚。雖然增加七州的新土地,但隨之而來反叛鬥爭,更令南唐疲於應付。與後周的兩次戰爭,消耗南唐大量庫存軍費,南唐戰敗,也使國力顯露頹敗之勢。

李璟奢侈无度,导致政治腐败,百姓民不聊生,怨声载道。

957年后周派兵侵入南唐,占领了南唐淮南江北的大片土地,并长驱直入到长江一带,迫近金陵,李璟只好向后周世宗柴榮称臣,去帝號,自稱唐國主,年號由原本的交泰改為後周的顯德。

961年8月12日卒,时年46岁,庙号烈宗,谥号为明道崇德文宣孝皇帝。

\subsubsection{保大}

\begin{longtable}{|>{\centering\scriptsize}m{2em}|>{\centering\scriptsize}m{1.3em}|>{\centering}m{8.8em}|}
  % \caption{秦王政}\
  \toprule
  \SimHei \normalsize 年数 & \SimHei \scriptsize 公元 & \SimHei 大事件 \tabularnewline
  % \midrule
  \endfirsthead
  \toprule
  \SimHei \normalsize 年数 & \SimHei \scriptsize 公元 & \SimHei 大事件 \tabularnewline
  \midrule
  \endhead
  \midrule
  元年 & 943 & \tabularnewline\hline
  二年 & 944 & \tabularnewline\hline
  三年 & 945 & \tabularnewline\hline
  四年 & 946 & \tabularnewline\hline
  五年 & 947 & \tabularnewline\hline
  六年 & 948 & \tabularnewline\hline
  七年 & 949 & \tabularnewline\hline
  八年 & 950 & \tabularnewline\hline
  九年 & 951 & \tabularnewline\hline
  十年 & 952 & \tabularnewline\hline
  十一年 & 953 & \tabularnewline\hline
  十二年 & 954 & \tabularnewline\hline
  十三年 & 955 & \tabularnewline\hline
  十四年 & 956 & \tabularnewline\hline
  十五年 & 957 & \tabularnewline
  \bottomrule
\end{longtable}

\subsubsection{中兴}

\begin{longtable}{|>{\centering\scriptsize}m{2em}|>{\centering\scriptsize}m{1.3em}|>{\centering}m{8.8em}|}
  % \caption{秦王政}\
  \toprule
  \SimHei \normalsize 年数 & \SimHei \scriptsize 公元 & \SimHei 大事件 \tabularnewline
  % \midrule
  \endfirsthead
  \toprule
  \SimHei \normalsize 年数 & \SimHei \scriptsize 公元 & \SimHei 大事件 \tabularnewline
  \midrule
  \endhead
  \midrule
  元年 & 958 & \tabularnewline
  \bottomrule
\end{longtable}

\subsubsection{交泰}

\begin{longtable}{|>{\centering\scriptsize}m{2em}|>{\centering\scriptsize}m{1.3em}|>{\centering}m{8.8em}|}
  % \caption{秦王政}\
  \toprule
  \SimHei \normalsize 年数 & \SimHei \scriptsize 公元 & \SimHei 大事件 \tabularnewline
  % \midrule
  \endfirsthead
  \toprule
  \SimHei \normalsize 年数 & \SimHei \scriptsize 公元 & \SimHei 大事件 \tabularnewline
  \midrule
  \endhead
  \midrule
  元年 & 958 & \tabularnewline
  \bottomrule
\end{longtable}

\subsubsection{显德}

\begin{longtable}{|>{\centering\scriptsize}m{2em}|>{\centering\scriptsize}m{1.3em}|>{\centering}m{8.8em}|}
  % \caption{秦王政}\
  \toprule
  \SimHei \normalsize 年数 & \SimHei \scriptsize 公元 & \SimHei 大事件 \tabularnewline
  % \midrule
  \endfirsthead
  \toprule
  \SimHei \normalsize 年数 & \SimHei \scriptsize 公元 & \SimHei 大事件 \tabularnewline
  \midrule
  \endhead
  \midrule
  元年 & 958 & \tabularnewline\hline
  二年 & 959 & \tabularnewline\hline
  三年 & 960 & \tabularnewline\hline
  四年 & 961 & \tabularnewline
  \bottomrule
\end{longtable}


%%% Local Variables:
%%% mode: latex
%%% TeX-engine: xetex
%%% TeX-master: "../../Main"
%%% End:
