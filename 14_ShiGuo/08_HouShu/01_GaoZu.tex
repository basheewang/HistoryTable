%% -*- coding: utf-8 -*-
%% Time-stamp: <Chen Wang: 2021-11-01 15:48:51>

\subsection{高祖孟知祥\tiny(934-937)}

\subsubsection{生平}

後蜀高祖孟知祥(874年6月9日-934年9月7日),字保胤,邢州龙冈(今河北邢台市西南)人。后唐太祖李克用婿。他是五代十国时期后蜀開國皇帝(934年在位),在位不到1年,享壽61岁。

后唐建立后,以孟知祥为太原尹充西京副留守。

后唐灭前蜀后,任孟知祥为西川节度使。孟知祥不久即谋求独立,先和东川节度使董璋合作,击退后唐来讨伐的军队。932年,又与董璋决裂,董璋来攻打孟知祥,反被消灭。933年,孟知祥被后唐封为蜀王。934年正月,他不受后唐闵帝官爵,在成都即皇帝位,建国号“大蜀”,史称“后蜀”,改元“明德”。

孟知祥只做了7个月皇帝就病重了,臨終時,由第三子(按孟知祥夫妇墓志铭,实为第五子,孟知祥前两个儿子疑因早夭故未序齿)孟昶监国。孟知祥死后,秘不发丧,王处回与赵季良立孟昶后才发丧。孟知祥谥号为文武圣德英明孝昭烈武皇帝,庙号高祖。其墓称和陵,位于四川省成都市北郊青龙乡石岭村的磨盘山南。陵墓以青石砌筑,建22级阶梯通向墓室。墓呈圆锥形,主室高8.16米,直径6.7米,是在南方少有的具五代后蜀的北方草原建筑风格的陵墓。现已发掘。

\subsubsection{明德}

\begin{longtable}{|>{\centering\scriptsize}m{2em}|>{\centering\scriptsize}m{1.3em}|>{\centering}m{8.8em}|}
  % \caption{秦王政}\
  \toprule
  \SimHei \normalsize 年数 & \SimHei \scriptsize 公元 & \SimHei 大事件 \tabularnewline
  % \midrule
  \endfirsthead
  \toprule
  \SimHei \normalsize 年数 & \SimHei \scriptsize 公元 & \SimHei 大事件 \tabularnewline
  \midrule
  \endhead
  \midrule
  元年 & 934 & \tabularnewline\hline
  二年 & 935 & \tabularnewline\hline
  三年 & 936 & \tabularnewline\hline
  四年 & 937 & \tabularnewline
  \bottomrule
\end{longtable}



%%% Local Variables:
%%% mode: latex
%%% TeX-engine: xetex
%%% TeX-master: "../../Main"
%%% End:
