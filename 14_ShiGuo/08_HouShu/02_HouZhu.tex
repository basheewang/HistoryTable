%% -*- coding: utf-8 -*-
%% Time-stamp: <Chen Wang: 2021-11-01 15:49:00>

\subsection{后主孟昶\tiny(938-965)}

\subsubsection{生平}

孟昶chǎng(919年12月9日-965年7月12日),初名仁赞,表字保元。後蜀高祖孟知祥第三子(据孟知祥夫妇墓誌铭则为第五子,疑两位兄长因早夭未序齿),母李贵妃。後蜀末代皇帝(第二代,934年~964年在位),在位31年,享年47岁,史書作「後主」。

孟昶一般被称为后主。即位初年,即铲除桀骜不驯的宿将李仁罕,慑服傲慢的老将李肇,励精图治,衣着朴素,兴修水利,注重农桑,实行“与民休息”政策,後蜀国势强盛,将北线疆土扩张到长安。

但是他在位後期,貪圖逸樂、沉湎酒色,不思国政,生活荒淫,奢侈无度,连夜壶都用珍宝制成,称为“七宝溺器”,朝政十分腐败。[來源請求]

後蜀广政二十八年(965年),宋师在大将王全斌的指挥下以两路伐後蜀,蜀军与宋军在剑门关外进行一场大战,蜀军全军覆灭,後蜀精兵被全歼,灭亡之势已不可免了。宋军包围成都府,孟昶投降,後蜀灭亡。

孟昶被俘後被封为检校太师兼中书令、秦国公,居住在汴京。北宋乾德三年(965年),孟昶入开封七日后郁郁而终(一说被赵光义毒死),追封楚王,諡恭孝。

孟昶的寵妃花蕊夫人在亡國之後寫下了悲憤的詩句:“君王城上豎降旗,妾在深宮哪得知,十四萬人齊解甲,更無一個是男兒。”

孟昶注重吏民關係,曾頒布四句箴言,令刻於巨石上,爾後被宋太宗拿來引註於各種官箴。這四句箴言即是:「爾俸爾祿,民膏民脂;下民易虐,上天難欺!」亦是流傳至今的「官箴」之一。

因孟昶喜好音樂,善於作曲,被南管界尊為祖師爺,稱為孟府郎君。

有人認為送子張仙一神,實為花蕊夫人入趙宋後,紀念孟昶的偽稱。《金台紀聞》記載:世所傳「張仙像」者,乃蜀王孟昶挾彈圖也。初,花蕊夫人入宋宮,念其故主,偶攜此圖,遂懸於壁,且祀之謹。太祖幸而見之,致詰焉。夫人詭答之曰:「此蜀中張仙神也。祀之能令人有子。」

另說後蜀亡後,花蕊夫人或感念孟昶的百姓,以「二郎神」、「孟府郎君」等名義加以供奉。

張太華,原為最受寵之妃,深得后主宠爱,与后主孟昶同游于青城山时被霹雷震死。(王文才、王炎《蜀檮杌校箋》認為此說不確,張太華為明人小說家言。)

徐慧妃(花蕊夫人),張太華去世後最受寵之妃(五代蜀主孟昶宠愛慧妃徐氏,徐国璋的女儿,被蜀主封为慧妃,慧妃常与后主(孟昶)登楼,以龙脑末涂白扇。扇坠地,为人所得。蜀人争效其制,名曰“雪香扇”。见清吴任臣《十国春秋·後蜀·慧妃徐氏传》。涂以香料的白色扇子。宋陶谷《清异录·雪香扇》:“孟昶夏月水调龙脑末,涂白扇上,用以挥风。一夜,与花蕊夫人登楼望月,悮堕其扇,为人所得。外有效者,名雪香扇。”)

李豔娘,因献舞而入宫为妃,封为昭容,并赐其家人钱财十万。李豔娘好梳高髻,宫人皆学她以邀宠幸,也唤作“朝天髻”。《十国宫词》露台灯耀舞衣妍,一搦纤腰十万钱。进御乞颁新位号,梳将高髻学朝天。

\subsubsection{广政}

\begin{longtable}{|>{\centering\scriptsize}m{2em}|>{\centering\scriptsize}m{1.3em}|>{\centering}m{8.8em}|}
  % \caption{秦王政}\
  \toprule
  \SimHei \normalsize 年数 & \SimHei \scriptsize 公元 & \SimHei 大事件 \tabularnewline
  % \midrule
  \endfirsthead
  \toprule
  \SimHei \normalsize 年数 & \SimHei \scriptsize 公元 & \SimHei 大事件 \tabularnewline
  \midrule
  \endhead
  \midrule
  元年 & 938 & \tabularnewline\hline
  二年 & 939 & \tabularnewline\hline
  三年 & 940 & \tabularnewline\hline
  四年 & 941 & \tabularnewline\hline
  五年 & 942 & \tabularnewline\hline
  六年 & 943 & \tabularnewline\hline
  七年 & 944 & \tabularnewline\hline
  八年 & 945 & \tabularnewline\hline
  九年 & 946 & \tabularnewline\hline
  十年 & 947 & \tabularnewline\hline
  十一年 & 948 & \tabularnewline\hline
  十二年 & 949 & \tabularnewline\hline
  十三年 & 950 & \tabularnewline\hline
  十四年 & 951 & \tabularnewline\hline
  十五年 & 952 & \tabularnewline\hline
  十六年 & 953 & \tabularnewline\hline
  十七年 & 954 & \tabularnewline\hline
  十八年 & 955 & \tabularnewline\hline
  十九年 & 956 & \tabularnewline\hline
  二十年 & 957 & \tabularnewline\hline
  二一年 & 958 & \tabularnewline\hline
  二二年 & 959 & \tabularnewline\hline
  二三年 & 960 & \tabularnewline\hline
  二四年 & 961 & \tabularnewline\hline
  二五年 & 962 & \tabularnewline\hline
  二六年 & 963 & \tabularnewline\hline
  二七年 & 964 & \tabularnewline\hline
  二八年 & 965 & \tabularnewline
  \bottomrule
\end{longtable}



%%% Local Variables:
%%% mode: latex
%%% TeX-engine: xetex
%%% TeX-master: "../../Main"
%%% End:
