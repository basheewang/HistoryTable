%% -*- coding: utf-8 -*-
%% Time-stamp: <Chen Wang: 2019-12-26 09:44:15>


\section{楚\tiny(907-951)}

\subsection{简介}

楚(907年-951年)是五代十国时期的十国之一,湖南历史上唯一以湖南为中心建立的王朝。以其为马氏所建,史称马楚,又称南楚,都长沙。楚国创始人马殷,许州鄢陵(今河南省鄢陵)人。

楚全盛时,辖域包括潭、衡、永、道、郴、邵、岳、朗、澧、辰、溆、连、昭、宜、全、桂、梧、贺、蒙、富、严、柳、象、容共24州,下设武安、武平、静江等5个节镇,即今湖南省全境和广西壮族自治区大部、贵州省东部和广东省北部。楚自896年马殷受命节度使到951年楚国灭于南唐,共存世56年,在湖南历史上产生重要影响。通过战争,消灭了湖南境内割据势力,实现了湖南的统一。

马殷政权时期,政治上采取“上奉天子、下抚士民”、内靖乱军、外御强藩等政策,使百姓获得了一个相对安定的环境。经济上,采取兴修水利、奖励农桑、发展茶业、提倡纺织、通商中原等措施,使社会经济得到了较快的发展。

马殷本是唐朝末年军阀孙儒部将,孙儒败亡后,马殷助孙儒余部龙骧指挥使刘建锋夺取武安军。后刘建锋被杀,马殷被推举接手武安军。

896年,唐朝朝廷任马殷为武安军节度使,奠定了他在湖南立足的根基。

907年,后梁封马殷为楚王,都潭州,号长沙府。927年,后唐天成二年,正式册封马殷为楚国王,楚国正式成立。马殷仿效朝廷体制,改潭州为长沙府,作为国都,并在长沙城内修宫殿,置百官,建立了一个名符其实的独立王国,成为五代时期十个封建割据国家之一。

930年马殷死,马殷次子马希声继位。932年马希声死,马殷子馬希範继位。947年马希範死,马希广继位。950年马希萼攻打长沙,马希广兵败被杀。于是马希萼自立楚王。

951年11月,南唐乘马楚内乱,派大将边镐率军进入楚国,占领长沙,楚灭亡。南唐还未站稳脚跟,马殷旧将刘言又起兵击败了南唐军,继续据有湖南。947年到951年的这段争夺王位的战争,被称为众驹争槽。

952年,王进逵杀刘言控制湖南。955年,部将潘叔嗣又杀了王进逵;潭州军府事周行逢又进军朗州杀了潘叔嗣,湖南全境遂为周行逢所控制。962年,行逢死,子周保权继位,手下大将张文表起兵反叛。此时赵匡胤已发动陈桥兵变即帝位,建立宋朝。周保权一边平叛一边求救于宋,虽然在宋军到来前败杀张文表,但宋军也趁机挥军南下攻占潭州。963年,湖南完全并入宋的版图。

据史料记载,马殷“土宇既广,乃养士息民”,由于政治上采取上奉天子、下抚士民的保境息民政策,同时奉行奖励农桑、发展茶叶、倡导纺织、重视商业贸易。马楚利用湖南地处南方各政权中心的地理优势,大力发展与中原和周边的商业贸易,采取免收关税,鼓励进出口贸易,招徕各国商人。《十国春秋·楚武穆王世家》载:“是时王关市无征,四方商旅闻风辐。”

茶税为当时楚国主要税收来源,因此政府每年税收“凡百万计”。为促进茶叶的生产与贸易,马楚政权采取“令民自造茶”、“听民售茶北客”的宽松政策,让百姓自己制造茶叶“以通商旅”。同时,马楚政权全国各地设置商业货栈(回图务),组织商人收购茶叶(茶商号“八床主人”),销往中原地区的商人,换回战马和丝织品。

由于马楚政权重商政策,那时的潭州已成为南方最大的茶市,城市化水平有了较大的发展。当时手工业和矿冶业也比较发达,其时采取“命民输税者皆以帛代钱”后“民间机抒大盛”,长沙棉纺业也始于马楚时期,其时楚地已种棉,故有胡三省之“木棉,今南方多有焉。于春中作畦种之,至夏秋之交结实,至秋丰其实之外皮四裂,中踊出自如绵。土人取而纺之,织之以布,细密厚暖,宜以御冬。”矿冶业方面,楚时潭州境内丹砂矿的开采风行一时,据说州的东境山崩,“涌出丹砂,委积如丘陵”,主要用于作为涂料之用,楚王马希範丹砂涂壁,“凡用数十万斤”。

为了发展商业,马殷采纳大臣高郁的建议,铸造铅、铁钱币在境内流通。由于铅铁钱币笨重,携带不便,在南唐等国又被禁用,因此商旅出境外贸易,大都“无所用钱”,往往销货后又在楚就地购买大量产品销往外地,这样楚地境内生产的产品通过“易天下百货”而变得富饶。当时楚国的茶叶和粮食等为与周边的主要贸易产品。

楚国发行的钱币主要为小平钱的铅钱“开元通宝”和折十钱的铁钱“乾封泉宝”。另外还有铜钱“天策府宝”、“乾元重宝”等。铜钱的数量十分稀少,天策府宝为古泉五十名珍之一。


%% -*- coding: utf-8 -*-
%% Time-stamp: <Chen Wang: 2021-11-01 15:41:09>

\subsection{武穆王马殷\tiny(907-930)}

\subsubsection{生平}

楚武穆王马殷(852年-930年12月2日),字霸图,许州鄢陵(今河南鄢陵)人,五代十国时期南楚开国君王。

马殷早年家贫,以木匠为業,后投入秦宗权军中,属孙儒部下,随孙儒渡淮攻下广陵(今江苏扬州东北)。唐僖宗光启三年(887年),秦宗权派其弟秦宗衡为主将,孙儒为副将,将兵三万,南下渡过淮河,同杨行密争夺扬州。不久,孙儒杀秦宗衡,自立为帅,号“土团白条军”。

大顺二年(891年),马殷受命率军击败杨行密部将田頵,随刘建锋镇守常州,后被调往宣州(今属安徽)参与围攻杨行密。景福元年(892年),孙儒兵败,战死军中,刘建锋和马殷率残军7000人逃走。其后,马殷作为刘建锋的先锋,一路攻占洪州、潭州等城。乾寧三年(896年)四月,刘建锋为部将所杀,马殷被众将推举为主,唐朝任其为湖南留後、判湖南軍府事。光化元年(898年)又授为武安军节度使。天復三年(903年),杨行密派刘存攻打鄂州(今武昌)的杜洪,马殷派秦彦晖、许德勋以舟兵救之,但同时又派许德勋与武贞军节度使雷彦威部将欧阳思联手,趁荆南节度使成汭出救杜洪之机洗劫其军府江陵,导致成汭军心涣散,最终被杨行密部将李神福所败而自杀。杜洪败死,刘存等攻马殷,马殷便沿江布防,埋下伏兵,双方激战甚烈,刘存战死,马殷夺下岳州(今湖南岳阳)。

后梁开平元年(907年),朱温封马殷为楚王,都于潭州(今长沙),开平四年(910年)六月加封“天策上将军”。后唐灭后梁后,明宗天成二年(927年)六月又册封为楚國王,同年八月冊封使至,马殷乃建立楚国,立宮殿、置百官,以潭州为都城,改名长沙府,使用后唐年号。

马殷在位期间,采取“上奉天子,下奉士民”的策略即保境安民的政策,自其于897年占据湖南后,很少主动对外交战,之后与杨吴的几次战争也是对方先发动进攻的,对于北边的荆南,也只进行了相当有限的战争。马殷对内采取措施发展农业生产,减轻百姓的赋税,“不征商旅,由是四方商旅輻凑”。他下令百姓可以用帛代替钱交纳赋税,减少了官吏加重赋税的机会,並且促进了湖南的桑蚕业的发展。因而楚国的经济非常繁荣。

長興元年十一月十日(930年12月2日),马殷去世,時年79歲,遺命諸子兄弟相繼。后唐明宗罷朝三天,下诏赐马殷谥号武穆王。次子马希声继其位,遵从父亲的遗命,不再称楚国,而是降低规格,恢复了节度使的称号,将楚政权延续了二十一年。

\subsubsection{天成}

\begin{longtable}{|>{\centering\scriptsize}m{2em}|>{\centering\scriptsize}m{1.3em}|>{\centering}m{8.8em}|}
  % \caption{秦王政}\
  \toprule
  \SimHei \normalsize 年数 & \SimHei \scriptsize 公元 & \SimHei 大事件 \tabularnewline
  % \midrule
  \endfirsthead
  \toprule
  \SimHei \normalsize 年数 & \SimHei \scriptsize 公元 & \SimHei 大事件 \tabularnewline
  \midrule
  \endhead
  \midrule
  元年 & 927 & \tabularnewline\hline
  二年 & 928 & \tabularnewline\hline
  三年 & 929 & \tabularnewline\hline
  四年 & 930 & \tabularnewline
  \bottomrule
\end{longtable}


%%% Local Variables:
%%% mode: latex
%%% TeX-engine: xetex
%%% TeX-master: "../../Main"
%%% End:

%% -*- coding: utf-8 -*-
%% Time-stamp: <Chen Wang: 2019-12-26 09:46:16>

\subsection{衡阳王\tiny(930-932)}

\subsubsection{生平}

馬希聲(898年-932年8月15日),字若訥,五代十國時期南楚國君主,是楚王馬殷的次子,馬殷在位時任武安節度副使,为其内定的接班人。妻杨氏。

929年,马希声听信后唐和荆南离间,排挤谋主高郁,高郁因而口出怨言,马希声恼怒,没有告知马殷,就以谋反为由诛杀高郁亲党。马殷虽然为高郁之死大恸,却没有处罚马希声。後唐明宗長興元年(930年),馬殷去世,馬希聲繼立,不稱王,只稱藩鎮。後唐則任命馬希聲武安、靜江節度使,兼中書令。馬希聲聽說朱全忠喜歡吃雞,很是羨慕,因此繼位後,每天都殺五十隻雞做菜;服喪期間也沒有哀傷的表情,馬殷要下葬時,還吃了好幾盤雞肉。

長興三年(932年)馬希聲去世,弟馬希範繼立。馬希聲在位時並未稱王,只在死後被追封為衡陽王。

\subsubsection{长兴}

\begin{longtable}{|>{\centering\scriptsize}m{2em}|>{\centering\scriptsize}m{1.3em}|>{\centering}m{8.8em}|}
  % \caption{秦王政}\
  \toprule
  \SimHei \normalsize 年数 & \SimHei \scriptsize 公元 & \SimHei 大事件 \tabularnewline
  % \midrule
  \endfirsthead
  \toprule
  \SimHei \normalsize 年数 & \SimHei \scriptsize 公元 & \SimHei 大事件 \tabularnewline
  \midrule
  \endhead
  \midrule
  元年 & 930 & \tabularnewline\hline
  二年 & 931 & \tabularnewline\hline
  三年 & 932 & \tabularnewline
  \bottomrule
\end{longtable}


%%% Local Variables:
%%% mode: latex
%%% TeX-engine: xetex
%%% TeX-master: "../../Main"
%%% End:

%% -*- coding: utf-8 -*-
%% Time-stamp: <Chen Wang: 2021-11-01 15:41:28>

\subsection{文昭王馬希範\tiny(932-947)}

\subsubsection{生平}

楚文昭王馬希範(899年-947年5月30日),字寶規,五代十國時期南楚國君主,是楚王馬殷的四子,馬希聲之弟,與馬希聲同年同月同日生。

後唐明宗長興三年(932年)馬希聲去世,因之前馬殷去世時遺命兄終弟及,因此群臣迎接時任鎮南節度使的馬希範繼位。後唐則任命馬希範為武安、武平節度使,兼中書令。後唐明宗清泰元年(934年),馬希範被封為楚王,之後又被封為天策上將軍。

馬希範好學,很會做詩,然而非常奢侈,其妻彭夫人“貌陋而治家有法”,希範“惮之”。彭夫人死后,马希範“始纵声色,为长夜之饮”。門戶檻杆都用金玉裝飾,塗抹牆壁的丹砂用量數十萬斤,常與子弟及部屬在內遊玩宴會。原本楚地多產金銀,而販賣茶葉的利潤更多,因此十分富庶,但是在無節制的揮霍下,只好向人民加稅,又賣官鬻爵,規定捐錢可贖罪刑,人民困苦不堪。

後漢高祖天福十二年(947年),馬希範去世,諡文昭王,弟馬希廣繼立。

\subsubsection{长兴}

\begin{longtable}{|>{\centering\scriptsize}m{2em}|>{\centering\scriptsize}m{1.3em}|>{\centering}m{8.8em}|}
  % \caption{秦王政}\
  \toprule
  \SimHei \normalsize 年数 & \SimHei \scriptsize 公元 & \SimHei 大事件 \tabularnewline
  % \midrule
  \endfirsthead
  \toprule
  \SimHei \normalsize 年数 & \SimHei \scriptsize 公元 & \SimHei 大事件 \tabularnewline
  \midrule
  \endhead
  \midrule
  元年 & 932 & \tabularnewline\hline
  二年 & 933 & \tabularnewline
  \bottomrule
\end{longtable}

\subsubsection{应顺}

\begin{longtable}{|>{\centering\scriptsize}m{2em}|>{\centering\scriptsize}m{1.3em}|>{\centering}m{8.8em}|}
  % \caption{秦王政}\
  \toprule
  \SimHei \normalsize 年数 & \SimHei \scriptsize 公元 & \SimHei 大事件 \tabularnewline
  % \midrule
  \endfirsthead
  \toprule
  \SimHei \normalsize 年数 & \SimHei \scriptsize 公元 & \SimHei 大事件 \tabularnewline
  \midrule
  \endhead
  \midrule
  元年 & 934 & \tabularnewline
  \bottomrule
\end{longtable}

\subsubsection{清泰}

\begin{longtable}{|>{\centering\scriptsize}m{2em}|>{\centering\scriptsize}m{1.3em}|>{\centering}m{8.8em}|}
  % \caption{秦王政}\
  \toprule
  \SimHei \normalsize 年数 & \SimHei \scriptsize 公元 & \SimHei 大事件 \tabularnewline
  % \midrule
  \endfirsthead
  \toprule
  \SimHei \normalsize 年数 & \SimHei \scriptsize 公元 & \SimHei 大事件 \tabularnewline
  \midrule
  \endhead
  \midrule
  元年 & 934 & \tabularnewline\hline
  二年 & 935 & \tabularnewline\hline
  三年 & 936 & \tabularnewline
  \bottomrule
\end{longtable}

\subsubsection{天福}

\begin{longtable}{|>{\centering\scriptsize}m{2em}|>{\centering\scriptsize}m{1.3em}|>{\centering}m{8.8em}|}
  % \caption{秦王政}\
  \toprule
  \SimHei \normalsize 年数 & \SimHei \scriptsize 公元 & \SimHei 大事件 \tabularnewline
  % \midrule
  \endfirsthead
  \toprule
  \SimHei \normalsize 年数 & \SimHei \scriptsize 公元 & \SimHei 大事件 \tabularnewline
  \midrule
  \endhead
  \midrule
  元年 & 936 & \tabularnewline\hline
  二年 & 937 & \tabularnewline\hline
  三年 & 938 & \tabularnewline\hline
  四年 & 939 & \tabularnewline\hline
  五年 & 940 & \tabularnewline\hline
  六年 & 941 & \tabularnewline\hline
  七年 & 942 & \tabularnewline\hline
  八年 & 943 & \tabularnewline\hline
  九年 & 944 & \tabularnewline
  \bottomrule
\end{longtable}

\subsubsection{开运}

\begin{longtable}{|>{\centering\scriptsize}m{2em}|>{\centering\scriptsize}m{1.3em}|>{\centering}m{8.8em}|}
  % \caption{秦王政}\
  \toprule
  \SimHei \normalsize 年数 & \SimHei \scriptsize 公元 & \SimHei 大事件 \tabularnewline
  % \midrule
  \endfirsthead
  \toprule
  \SimHei \normalsize 年数 & \SimHei \scriptsize 公元 & \SimHei 大事件 \tabularnewline
  \midrule
  \endhead
  \midrule
  元年 & 944 & \tabularnewline\hline
  二年 & 945 & \tabularnewline\hline
  三年 & 946 & \tabularnewline
  \bottomrule
\end{longtable}


%%% Local Variables:
%%% mode: latex
%%% TeX-engine: xetex
%%% TeX-master: "../../Main"
%%% End:

%% -*- coding: utf-8 -*-
%% Time-stamp: <Chen Wang: 2019-12-26 09:47:28>

\subsection{马希广\tiny(947-950)}

\subsubsection{生平}

馬希廣(?-951年1月25日),字德丕,五代十國時期南楚國君主,楚王馬殷第三十五子,馬希範一母同胞之弟,個性謹慎溫順,馬希範對他疼愛有加。

後漢高祖天福十二年(947年),馬希範去世,將領排除馬希範諸弟中年齡最長的馬希萼,而擁護馬希廣繼立,後漢封馬希廣天策上將軍、楚王,因而馬希廣、希萼之弟馬希崇就以馬希廣之繼位違反父親兄終弟及的遺命挑撥馬希萼。

後漢隱帝乾祐二年(949年),時任武貞(武平)節度使的馬希萼叛,率軍南下進攻南楚都城潭州(今湖南長沙),馬希萼戰敗,馬希廣以不願傷其兄為由,放棄追擊。乾祐三年(950年)馬希萼結合蠻族軍再度攻擊馬希廣,並向南唐稱臣,請求發兵攻潭州。馬希廣派軍討伐馬希萼,大敗。馬希萼遂與蠻族軍兵圍潭州,守將許可瓊投降,潭州陷落,馬希廣夫妇被擒。马希萼为了避免后患,处死马希广。马希广临死还在背诵佛经。马希广夫人被杖杀于闹市。马希广有子藏在慈堂,后不知所终。马军指挥使李彦温与战棹指挥使刘彦瑫奉马希广其余诸子去袁州奔南唐,马希广诸子后在南唐都城金陵去世。

\subsubsection{天福}

\begin{longtable}{|>{\centering\scriptsize}m{2em}|>{\centering\scriptsize}m{1.3em}|>{\centering}m{8.8em}|}
  % \caption{秦王政}\
  \toprule
  \SimHei \normalsize 年数 & \SimHei \scriptsize 公元 & \SimHei 大事件 \tabularnewline
  % \midrule
  \endfirsthead
  \toprule
  \SimHei \normalsize 年数 & \SimHei \scriptsize 公元 & \SimHei 大事件 \tabularnewline
  \midrule
  \endhead
  \midrule
  元年 & 947 & \tabularnewline
  \bottomrule
\end{longtable}

\subsubsection{乾佑}

\begin{longtable}{|>{\centering\scriptsize}m{2em}|>{\centering\scriptsize}m{1.3em}|>{\centering}m{8.8em}|}
  % \caption{秦王政}\
  \toprule
  \SimHei \normalsize 年数 & \SimHei \scriptsize 公元 & \SimHei 大事件 \tabularnewline
  % \midrule
  \endfirsthead
  \toprule
  \SimHei \normalsize 年数 & \SimHei \scriptsize 公元 & \SimHei 大事件 \tabularnewline
  \midrule
  \endhead
  \midrule
  元年 & 948 & \tabularnewline\hline
  二年 & 949 & \tabularnewline\hline
  三年 & 950 & \tabularnewline
  \bottomrule
\end{longtable}


%%% Local Variables:
%%% mode: latex
%%% TeX-engine: xetex
%%% TeX-master: "../../Main"
%%% End:

%% -*- coding: utf-8 -*-
%% Time-stamp: <Chen Wang: 2019-12-26 09:47:50>

\subsection{恭孝王\tiny(950-951)}

\subsubsection{生平}

楚恭孝王馬希萼(900年-953年),五代十國時期南楚國君主,楚武穆王馬殷之子,馬希聲、馬希範之弟,馬希廣之兄。馬希範在位時任武貞(武平)節度使,鎮守朗州(今湖南常德)。

後漢高祖天福十二年(947年),馬希範去世,將領排除馬希範諸弟中年齡最長的馬希萼,而擁護馬希廣繼立,因而馬希廣、希萼之弟馬希崇就以馬希廣之繼位違反父親兄終弟及的遺命挑撥馬希萼。

後漢隱帝乾祐二年(949年),馬希萼叛變,率軍南下進攻南楚都城潭州(今湖南長沙),馬希萼戰敗,馬希廣以不願傷其兄為由,放棄追擊。乾祐三年(950年)馬希萼結合蠻族軍再度攻擊馬希廣,並向南唐稱臣,請求發兵攻潭州。馬希廣派軍討伐馬希萼,大敗。馬希萼遂自稱順天王,並與蠻族軍兵圍潭州,守將許可瓊投降,佔領潭州,擒馬希廣。不久,將馬希廣賜死。

馬希萼當初因認為後漢偏袒馬希廣而轉向南唐稱臣,故一改馬殷以來臣服中原的態度,未待冊封即自稱天策上將軍、武安、武平、靜江、寧遠等軍節度使、楚王。登位後,志得意滿,殺戮報復,縱酒荒淫,將事務都交給馬希崇,然而馬希崇也只是交給下屬而已,因此政事混亂,又對士卒不加賞賜,遂軍心思變。

後周太祖廣順元年(951年),王逵、周行逢首先佔據朗州,擁護馬殷長子馬希振之子馬光惠當節度使。數月後,徐威等將領兵變,擁護馬希崇為武安留後,馬希萼被囚禁於衡山縣。馬希萼抵衡山後,復被廖偃、廖匡凝、彭師暠等擁護稱衡山王。不久,南楚為南唐所滅,馬希萼被南唐任命為江南西道觀察使,仍封楚王。其後在入朝的時候,被南唐元宗李璟留下,三年後在南唐都城金陵(今江蘇南京)去世寿五十四岁,諡恭孝王。

\subsubsection{保大}

\begin{longtable}{|>{\centering\scriptsize}m{2em}|>{\centering\scriptsize}m{1.3em}|>{\centering}m{8.8em}|}
  % \caption{秦王政}\
  \toprule
  \SimHei \normalsize 年数 & \SimHei \scriptsize 公元 & \SimHei 大事件 \tabularnewline
  % \midrule
  \endfirsthead
  \toprule
  \SimHei \normalsize 年数 & \SimHei \scriptsize 公元 & \SimHei 大事件 \tabularnewline
  \midrule
  \endhead
  \midrule
  元年 & 950 & \tabularnewline\hline
  二年 & 951 & \tabularnewline
  \bottomrule
\end{longtable}


%%% Local Variables:
%%% mode: latex
%%% TeX-engine: xetex
%%% TeX-master: "../../Main"
%%% End:



%%% Local Variables:
%%% mode: latex
%%% TeX-engine: xetex
%%% TeX-master: "../../Main"
%%% End:
