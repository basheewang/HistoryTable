%% -*- coding: utf-8 -*-
%% Time-stamp: <Chen Wang: 2021-11-01 15:41:09>

\subsection{武穆王马殷\tiny(907-930)}

\subsubsection{生平}

楚武穆王马殷(852年-930年12月2日),字霸图,许州鄢陵(今河南鄢陵)人,五代十国时期南楚开国君王。

马殷早年家贫,以木匠为業,后投入秦宗权军中,属孙儒部下,随孙儒渡淮攻下广陵(今江苏扬州东北)。唐僖宗光启三年(887年),秦宗权派其弟秦宗衡为主将,孙儒为副将,将兵三万,南下渡过淮河,同杨行密争夺扬州。不久,孙儒杀秦宗衡,自立为帅,号“土团白条军”。

大顺二年(891年),马殷受命率军击败杨行密部将田頵,随刘建锋镇守常州,后被调往宣州(今属安徽)参与围攻杨行密。景福元年(892年),孙儒兵败,战死军中,刘建锋和马殷率残军7000人逃走。其后,马殷作为刘建锋的先锋,一路攻占洪州、潭州等城。乾寧三年(896年)四月,刘建锋为部将所杀,马殷被众将推举为主,唐朝任其为湖南留後、判湖南軍府事。光化元年(898年)又授为武安军节度使。天復三年(903年),杨行密派刘存攻打鄂州(今武昌)的杜洪,马殷派秦彦晖、许德勋以舟兵救之,但同时又派许德勋与武贞军节度使雷彦威部将欧阳思联手,趁荆南节度使成汭出救杜洪之机洗劫其军府江陵,导致成汭军心涣散,最终被杨行密部将李神福所败而自杀。杜洪败死,刘存等攻马殷,马殷便沿江布防,埋下伏兵,双方激战甚烈,刘存战死,马殷夺下岳州(今湖南岳阳)。

后梁开平元年(907年),朱温封马殷为楚王,都于潭州(今长沙),开平四年(910年)六月加封“天策上将军”。后唐灭后梁后,明宗天成二年(927年)六月又册封为楚國王,同年八月冊封使至,马殷乃建立楚国,立宮殿、置百官,以潭州为都城,改名长沙府,使用后唐年号。

马殷在位期间,采取“上奉天子,下奉士民”的策略即保境安民的政策,自其于897年占据湖南后,很少主动对外交战,之后与杨吴的几次战争也是对方先发动进攻的,对于北边的荆南,也只进行了相当有限的战争。马殷对内采取措施发展农业生产,减轻百姓的赋税,“不征商旅,由是四方商旅輻凑”。他下令百姓可以用帛代替钱交纳赋税,减少了官吏加重赋税的机会,並且促进了湖南的桑蚕业的发展。因而楚国的经济非常繁荣。

長興元年十一月十日(930年12月2日),马殷去世,時年79歲,遺命諸子兄弟相繼。后唐明宗罷朝三天,下诏赐马殷谥号武穆王。次子马希声继其位,遵从父亲的遗命,不再称楚国,而是降低规格,恢复了节度使的称号,将楚政权延续了二十一年。

\subsubsection{天成}

\begin{longtable}{|>{\centering\scriptsize}m{2em}|>{\centering\scriptsize}m{1.3em}|>{\centering}m{8.8em}|}
  % \caption{秦王政}\
  \toprule
  \SimHei \normalsize 年数 & \SimHei \scriptsize 公元 & \SimHei 大事件 \tabularnewline
  % \midrule
  \endfirsthead
  \toprule
  \SimHei \normalsize 年数 & \SimHei \scriptsize 公元 & \SimHei 大事件 \tabularnewline
  \midrule
  \endhead
  \midrule
  元年 & 927 & \tabularnewline\hline
  二年 & 928 & \tabularnewline\hline
  三年 & 929 & \tabularnewline\hline
  四年 & 930 & \tabularnewline
  \bottomrule
\end{longtable}


%%% Local Variables:
%%% mode: latex
%%% TeX-engine: xetex
%%% TeX-master: "../../Main"
%%% End:
