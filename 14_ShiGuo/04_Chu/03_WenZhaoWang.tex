%% -*- coding: utf-8 -*-
%% Time-stamp: <Chen Wang: 2021-11-01 15:41:28>

\subsection{文昭王馬希範\tiny(932-947)}

\subsubsection{生平}

楚文昭王馬希範(899年-947年5月30日),字寶規,五代十國時期南楚國君主,是楚王馬殷的四子,馬希聲之弟,與馬希聲同年同月同日生。

後唐明宗長興三年(932年)馬希聲去世,因之前馬殷去世時遺命兄終弟及,因此群臣迎接時任鎮南節度使的馬希範繼位。後唐則任命馬希範為武安、武平節度使,兼中書令。後唐明宗清泰元年(934年),馬希範被封為楚王,之後又被封為天策上將軍。

馬希範好學,很會做詩,然而非常奢侈,其妻彭夫人“貌陋而治家有法”,希範“惮之”。彭夫人死后,马希範“始纵声色,为长夜之饮”。門戶檻杆都用金玉裝飾,塗抹牆壁的丹砂用量數十萬斤,常與子弟及部屬在內遊玩宴會。原本楚地多產金銀,而販賣茶葉的利潤更多,因此十分富庶,但是在無節制的揮霍下,只好向人民加稅,又賣官鬻爵,規定捐錢可贖罪刑,人民困苦不堪。

後漢高祖天福十二年(947年),馬希範去世,諡文昭王,弟馬希廣繼立。

\subsubsection{长兴}

\begin{longtable}{|>{\centering\scriptsize}m{2em}|>{\centering\scriptsize}m{1.3em}|>{\centering}m{8.8em}|}
  % \caption{秦王政}\
  \toprule
  \SimHei \normalsize 年数 & \SimHei \scriptsize 公元 & \SimHei 大事件 \tabularnewline
  % \midrule
  \endfirsthead
  \toprule
  \SimHei \normalsize 年数 & \SimHei \scriptsize 公元 & \SimHei 大事件 \tabularnewline
  \midrule
  \endhead
  \midrule
  元年 & 932 & \tabularnewline\hline
  二年 & 933 & \tabularnewline
  \bottomrule
\end{longtable}

\subsubsection{应顺}

\begin{longtable}{|>{\centering\scriptsize}m{2em}|>{\centering\scriptsize}m{1.3em}|>{\centering}m{8.8em}|}
  % \caption{秦王政}\
  \toprule
  \SimHei \normalsize 年数 & \SimHei \scriptsize 公元 & \SimHei 大事件 \tabularnewline
  % \midrule
  \endfirsthead
  \toprule
  \SimHei \normalsize 年数 & \SimHei \scriptsize 公元 & \SimHei 大事件 \tabularnewline
  \midrule
  \endhead
  \midrule
  元年 & 934 & \tabularnewline
  \bottomrule
\end{longtable}

\subsubsection{清泰}

\begin{longtable}{|>{\centering\scriptsize}m{2em}|>{\centering\scriptsize}m{1.3em}|>{\centering}m{8.8em}|}
  % \caption{秦王政}\
  \toprule
  \SimHei \normalsize 年数 & \SimHei \scriptsize 公元 & \SimHei 大事件 \tabularnewline
  % \midrule
  \endfirsthead
  \toprule
  \SimHei \normalsize 年数 & \SimHei \scriptsize 公元 & \SimHei 大事件 \tabularnewline
  \midrule
  \endhead
  \midrule
  元年 & 934 & \tabularnewline\hline
  二年 & 935 & \tabularnewline\hline
  三年 & 936 & \tabularnewline
  \bottomrule
\end{longtable}

\subsubsection{天福}

\begin{longtable}{|>{\centering\scriptsize}m{2em}|>{\centering\scriptsize}m{1.3em}|>{\centering}m{8.8em}|}
  % \caption{秦王政}\
  \toprule
  \SimHei \normalsize 年数 & \SimHei \scriptsize 公元 & \SimHei 大事件 \tabularnewline
  % \midrule
  \endfirsthead
  \toprule
  \SimHei \normalsize 年数 & \SimHei \scriptsize 公元 & \SimHei 大事件 \tabularnewline
  \midrule
  \endhead
  \midrule
  元年 & 936 & \tabularnewline\hline
  二年 & 937 & \tabularnewline\hline
  三年 & 938 & \tabularnewline\hline
  四年 & 939 & \tabularnewline\hline
  五年 & 940 & \tabularnewline\hline
  六年 & 941 & \tabularnewline\hline
  七年 & 942 & \tabularnewline\hline
  八年 & 943 & \tabularnewline\hline
  九年 & 944 & \tabularnewline
  \bottomrule
\end{longtable}

\subsubsection{开运}

\begin{longtable}{|>{\centering\scriptsize}m{2em}|>{\centering\scriptsize}m{1.3em}|>{\centering}m{8.8em}|}
  % \caption{秦王政}\
  \toprule
  \SimHei \normalsize 年数 & \SimHei \scriptsize 公元 & \SimHei 大事件 \tabularnewline
  % \midrule
  \endfirsthead
  \toprule
  \SimHei \normalsize 年数 & \SimHei \scriptsize 公元 & \SimHei 大事件 \tabularnewline
  \midrule
  \endhead
  \midrule
  元年 & 944 & \tabularnewline\hline
  二年 & 945 & \tabularnewline\hline
  三年 & 946 & \tabularnewline
  \bottomrule
\end{longtable}


%%% Local Variables:
%%% mode: latex
%%% TeX-engine: xetex
%%% TeX-master: "../../Main"
%%% End:
