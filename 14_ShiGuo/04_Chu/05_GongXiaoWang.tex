%% -*- coding: utf-8 -*-
%% Time-stamp: <Chen Wang: 2019-12-26 09:47:50>

\subsection{恭孝王\tiny(950-951)}

\subsubsection{生平}

楚恭孝王馬希萼(900年-953年),五代十國時期南楚國君主,楚武穆王馬殷之子,馬希聲、馬希範之弟,馬希廣之兄。馬希範在位時任武貞(武平)節度使,鎮守朗州(今湖南常德)。

後漢高祖天福十二年(947年),馬希範去世,將領排除馬希範諸弟中年齡最長的馬希萼,而擁護馬希廣繼立,因而馬希廣、希萼之弟馬希崇就以馬希廣之繼位違反父親兄終弟及的遺命挑撥馬希萼。

後漢隱帝乾祐二年(949年),馬希萼叛變,率軍南下進攻南楚都城潭州(今湖南長沙),馬希萼戰敗,馬希廣以不願傷其兄為由,放棄追擊。乾祐三年(950年)馬希萼結合蠻族軍再度攻擊馬希廣,並向南唐稱臣,請求發兵攻潭州。馬希廣派軍討伐馬希萼,大敗。馬希萼遂自稱順天王,並與蠻族軍兵圍潭州,守將許可瓊投降,佔領潭州,擒馬希廣。不久,將馬希廣賜死。

馬希萼當初因認為後漢偏袒馬希廣而轉向南唐稱臣,故一改馬殷以來臣服中原的態度,未待冊封即自稱天策上將軍、武安、武平、靜江、寧遠等軍節度使、楚王。登位後,志得意滿,殺戮報復,縱酒荒淫,將事務都交給馬希崇,然而馬希崇也只是交給下屬而已,因此政事混亂,又對士卒不加賞賜,遂軍心思變。

後周太祖廣順元年(951年),王逵、周行逢首先佔據朗州,擁護馬殷長子馬希振之子馬光惠當節度使。數月後,徐威等將領兵變,擁護馬希崇為武安留後,馬希萼被囚禁於衡山縣。馬希萼抵衡山後,復被廖偃、廖匡凝、彭師暠等擁護稱衡山王。不久,南楚為南唐所滅,馬希萼被南唐任命為江南西道觀察使,仍封楚王。其後在入朝的時候,被南唐元宗李璟留下,三年後在南唐都城金陵(今江蘇南京)去世寿五十四岁,諡恭孝王。

\subsubsection{保大}

\begin{longtable}{|>{\centering\scriptsize}m{2em}|>{\centering\scriptsize}m{1.3em}|>{\centering}m{8.8em}|}
  % \caption{秦王政}\
  \toprule
  \SimHei \normalsize 年数 & \SimHei \scriptsize 公元 & \SimHei 大事件 \tabularnewline
  % \midrule
  \endfirsthead
  \toprule
  \SimHei \normalsize 年数 & \SimHei \scriptsize 公元 & \SimHei 大事件 \tabularnewline
  \midrule
  \endhead
  \midrule
  元年 & 950 & \tabularnewline\hline
  二年 & 951 & \tabularnewline
  \bottomrule
\end{longtable}


%%% Local Variables:
%%% mode: latex
%%% TeX-engine: xetex
%%% TeX-master: "../../Main"
%%% End:
