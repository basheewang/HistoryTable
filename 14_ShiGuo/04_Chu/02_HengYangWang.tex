%% -*- coding: utf-8 -*-
%% Time-stamp: <Chen Wang: 2019-12-26 09:46:16>

\subsection{衡阳王\tiny(930-932)}

\subsubsection{生平}

馬希聲(898年-932年8月15日),字若訥,五代十國時期南楚國君主,是楚王馬殷的次子,馬殷在位時任武安節度副使,为其内定的接班人。妻杨氏。

929年,马希声听信后唐和荆南离间,排挤谋主高郁,高郁因而口出怨言,马希声恼怒,没有告知马殷,就以谋反为由诛杀高郁亲党。马殷虽然为高郁之死大恸,却没有处罚马希声。後唐明宗長興元年(930年),馬殷去世,馬希聲繼立,不稱王,只稱藩鎮。後唐則任命馬希聲武安、靜江節度使,兼中書令。馬希聲聽說朱全忠喜歡吃雞,很是羨慕,因此繼位後,每天都殺五十隻雞做菜;服喪期間也沒有哀傷的表情,馬殷要下葬時,還吃了好幾盤雞肉。

長興三年(932年)馬希聲去世,弟馬希範繼立。馬希聲在位時並未稱王,只在死後被追封為衡陽王。

\subsubsection{长兴}

\begin{longtable}{|>{\centering\scriptsize}m{2em}|>{\centering\scriptsize}m{1.3em}|>{\centering}m{8.8em}|}
  % \caption{秦王政}\
  \toprule
  \SimHei \normalsize 年数 & \SimHei \scriptsize 公元 & \SimHei 大事件 \tabularnewline
  % \midrule
  \endfirsthead
  \toprule
  \SimHei \normalsize 年数 & \SimHei \scriptsize 公元 & \SimHei 大事件 \tabularnewline
  \midrule
  \endhead
  \midrule
  元年 & 930 & \tabularnewline\hline
  二年 & 931 & \tabularnewline\hline
  三年 & 932 & \tabularnewline
  \bottomrule
\end{longtable}


%%% Local Variables:
%%% mode: latex
%%% TeX-engine: xetex
%%% TeX-master: "../../Main"
%%% End:
