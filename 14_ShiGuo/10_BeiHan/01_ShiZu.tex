%% -*- coding: utf-8 -*-
%% Time-stamp: <Chen Wang: 2021-11-01 15:50:53>

\subsection{世祖劉旻\tiny(951-954)}

\subsubsection{生平}

漢世祖劉旻(895年-954年),並州晉陽(今山西太原)人,沙陀族,原名刘彦崇、劉崇,五代十國時期北漢開國皇帝,為後漢高祖劉知遠之弟,父劉琠。

劉崇年輕時喜歡飲酒賭博,曾經於臉上刺青從軍。劉知遠於後晉任河東節度使時,他擔任都指揮使。劉知遠建後漢之後,任太原尹,後漢隱帝劉承祐在位時任河東節度使(位於太原),鎮守河東地區。

後漢乾祐三年(950年),樞密使郭威為劉承祐逼反,進軍後漢都城大梁(今河南開封),劉承祐逃亡中為下屬郭允明所殺,郭威遂控制朝政。劉崇此時原欲舉兵南下,但聽到郭威計畫迎立劉崇之長子武寧節度使劉贇為帝,遂打消此意。然而不久郭威被黃旗加身後自登帝位,建立後周,改元廣順,並殺劉贇,當時為951年。因此劉崇隨即亦在太原(今山西省太原市)登帝位,延續後漢,改名劉旻,仍維持乾祐年號,稱乾祐四年。後世把劉崇稱帝後的政權稱作北漢。

北漢地小民貧,又以興復後漢為業,遂向遼國乞援,與遼國約為父子之國,由劉旻稱遼帝為叔,而自稱姪皇帝;遼國則封劉旻為大漢神武皇帝。北漢因遼國的援助,而與後周進行了不少戰爭,但仍勝少敗多。北漢乾祐七年(954年),趁郭威去世之際,聯合遼國南攻後周,然為後周世宗柴榮率軍敗於高平,劉旻穿著農人的衣服隨百餘騎逃走,途中一度迷路,劉旻年老力衰,差點無法支撐回到太原。

經此一役,北漢元氣大傷,無力南下,而劉旻亦憂憤成疾,不久去世,廟號世祖,次子劉承鈞繼位。

\subsubsection{乾佑}

\begin{longtable}{|>{\centering\scriptsize}m{2em}|>{\centering\scriptsize}m{1.3em}|>{\centering}m{8.8em}|}
  % \caption{秦王政}\
  \toprule
  \SimHei \normalsize 年数 & \SimHei \scriptsize 公元 & \SimHei 大事件 \tabularnewline
  % \midrule
  \endfirsthead
  \toprule
  \SimHei \normalsize 年数 & \SimHei \scriptsize 公元 & \SimHei 大事件 \tabularnewline
  \midrule
  \endhead
  \midrule
  元年 & 951 & \tabularnewline\hline
  二年 & 952 & \tabularnewline\hline
  三年 & 953 & \tabularnewline\hline
  四年 & 954 & \tabularnewline
  \bottomrule
\end{longtable}


%%% Local Variables:
%%% mode: latex
%%% TeX-engine: xetex
%%% TeX-master: "../../Main"
%%% End:
