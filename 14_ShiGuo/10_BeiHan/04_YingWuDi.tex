%% -*- coding: utf-8 -*-
%% Time-stamp: <Chen Wang: 2019-12-26 10:15:18>

\subsection{英武帝\tiny(968-979)}

\subsubsection{生平}

漢英武帝劉繼元(10世纪?-992年),本姓何,五代時期北漢君主,世祖刘旻外孙,睿宗刘钧外甥、养子。

其母刘氏為劉旻之女,因此劉鈞是他的舅父。母亲刘氏在劉繼恩之父薛釗自殺後改嫁何氏而生下劉繼元,所以劉繼恩是他的同母異父之兄,劉繼元的父母都去世以後,劉鈞收他為養子。

劉繼元在劉繼恩在位時任太原尹,北漢天會十二年(968年)劉繼恩為侯霸榮刺殺後,劉繼元被司空郭無為迎立為帝,繼位後即緩和與遼國間的緊張關係。劉繼元為人殘忍嗜殺,嫡母劉承鈞之妻郭皇后及劉旻之子皆被其所殺;亦動輒將忤逆他的臣屬滅族。

天會十三年(969年)宋太祖趙匡胤親征北漢,宋軍久攻不下而退兵,北漢收取宋軍所拋棄輜重,瀕臨枯竭的國力賴以恢復。

974年,改年號廣運。廣運六年(宋太平興國四年,979年),宋朝將南方各國併入版圖之後,再度決意北伐,由宋太宗趙光義親征,宋軍攻勢猛烈,遼國援軍亦被擊退,五月初六日劉繼元投降,北漢亡。投降後被任命為右衛上將軍,封彭城郡公。太平興國六年(981年),進封為彭城公。雍熙三年(986年),再被任命為保康軍節度使。淳化二年十二月十八日(陽曆為992年1月25日)去世,被贈中書令,追封為彭城郡王。

\subsubsection{天会}

\begin{longtable}{|>{\centering\scriptsize}m{2em}|>{\centering\scriptsize}m{1.3em}|>{\centering}m{8.8em}|}
  % \caption{秦王政}\
  \toprule
  \SimHei \normalsize 年数 & \SimHei \scriptsize 公元 & \SimHei 大事件 \tabularnewline
  % \midrule
  \endfirsthead
  \toprule
  \SimHei \normalsize 年数 & \SimHei \scriptsize 公元 & \SimHei 大事件 \tabularnewline
  \midrule
  \endhead
  \midrule
  元年 & 968 & \tabularnewline\hline
  二年 & 969 & \tabularnewline\hline
  三年 & 970 & \tabularnewline\hline
  四年 & 971 & \tabularnewline\hline
  五年 & 972 & \tabularnewline\hline
  六年 & 973 & \tabularnewline
  \bottomrule
\end{longtable}

\subsubsection{广运}

\begin{longtable}{|>{\centering\scriptsize}m{2em}|>{\centering\scriptsize}m{1.3em}|>{\centering}m{8.8em}|}
  % \caption{秦王政}\
  \toprule
  \SimHei \normalsize 年数 & \SimHei \scriptsize 公元 & \SimHei 大事件 \tabularnewline
  % \midrule
  \endfirsthead
  \toprule
  \SimHei \normalsize 年数 & \SimHei \scriptsize 公元 & \SimHei 大事件 \tabularnewline
  \midrule
  \endhead
  \midrule
  元年 & 974 & \tabularnewline\hline
  二年 & 975 & \tabularnewline\hline
  三年 & 976 & \tabularnewline\hline
  四年 & 977 & \tabularnewline\hline
  五年 & 978 & \tabularnewline\hline
  六年 & 979 & \tabularnewline
  \bottomrule
\end{longtable}


%%% Local Variables:
%%% mode: latex
%%% TeX-engine: xetex
%%% TeX-master: "../../Main"
%%% End:
