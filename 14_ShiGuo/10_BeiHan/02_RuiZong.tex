%% -*- coding: utf-8 -*-
%% Time-stamp: <Chen Wang: 2019-12-26 10:14:15>

\subsection{睿宗\tiny(954-968)}

\subsubsection{生平}

汉睿宗劉鈞(926年-968年),原名劉承鈞,五代時期北漢在位最久的君主,為劉旻之次子。

劉承鈞個性孝順恭謹,喜歡讀書,擅長書法,北漢乾祐七年(954年),劉旻去世,劉承鈞為遼國冊封為帝之後繼位,不改年號,改名劉鈞。上表於遼帝時都自稱「男」,遼帝下詔時,都稱呼他「兒皇帝」。

劉鈞繼位後,勤政愛民,禮敬士大夫,任用郭無為為相,並減少南侵,因此境內還算安定。然而劉鈞並不像其父事奉遼國之恭敬,以致在位後期遼國援助漸少。

劉鈞於957年,改元天會。天會十二年(968年)忧郁死,諡孝和皇帝,廟號睿宗,劉鈞的外甥同時也是養子的劉繼恩繼位。

\subsubsection{乾佑}

\begin{longtable}{|>{\centering\scriptsize}m{2em}|>{\centering\scriptsize}m{1.3em}|>{\centering}m{8.8em}|}
  % \caption{秦王政}\
  \toprule
  \SimHei \normalsize 年数 & \SimHei \scriptsize 公元 & \SimHei 大事件 \tabularnewline
  % \midrule
  \endfirsthead
  \toprule
  \SimHei \normalsize 年数 & \SimHei \scriptsize 公元 & \SimHei 大事件 \tabularnewline
  \midrule
  \endhead
  \midrule
  元年 & 954 & \tabularnewline\hline
  二年 & 955 & \tabularnewline\hline
  三年 & 956 & \tabularnewline
  \bottomrule
\end{longtable}

\subsubsection{天会}

\begin{longtable}{|>{\centering\scriptsize}m{2em}|>{\centering\scriptsize}m{1.3em}|>{\centering}m{8.8em}|}
  % \caption{秦王政}\
  \toprule
  \SimHei \normalsize 年数 & \SimHei \scriptsize 公元 & \SimHei 大事件 \tabularnewline
  % \midrule
  \endfirsthead
  \toprule
  \SimHei \normalsize 年数 & \SimHei \scriptsize 公元 & \SimHei 大事件 \tabularnewline
  \midrule
  \endhead
  \midrule
  元年 & 957 & \tabularnewline\hline
  二年 & 958 & \tabularnewline\hline
  三年 & 959 & \tabularnewline\hline
  四年 & 960 & \tabularnewline\hline
  五年 & 961 & \tabularnewline\hline
  六年 & 962 & \tabularnewline\hline
  七年 & 963 & \tabularnewline\hline
  八年 & 964 & \tabularnewline\hline
  九年 & 965 & \tabularnewline\hline
  十年 & 966 & \tabularnewline\hline
  十一年 & 967 & \tabularnewline\hline
  十二年 & 968 & \tabularnewline\hline
  \bottomrule
\end{longtable}


%%% Local Variables:
%%% mode: latex
%%% TeX-engine: xetex
%%% TeX-master: "../../Main"
%%% End:
