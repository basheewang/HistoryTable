%% -*- coding: utf-8 -*-
%% Time-stamp: <Chen Wang: 2021-11-01 15:51:09>

\subsection{少主劉繼恩\tiny(968)}

\subsubsection{生平}

劉繼恩(10世纪?-968年),本姓薛,五代時期北漢君主,世祖刘旻外孙,睿宗刘钧外甥、养子,史稱「少主」。

其母刘氏為劉旻之女,因此劉鈞是他的舅父;其父薛釗,本來只是士兵,早年娶劉旻之女,劉旻之兄劉知遠發跡後,薛釗就很難見到其妻,因此很不快樂,有一天乘著酒意求見,竟拿刀刺傷其妻,薛釗後來因此自盡。當時劉繼恩年紀還小,劉旻因為劉鈞無子,遂命劉鈞收養劉繼恩。

劉繼恩在劉鈞在位時任太原尹,然而他本人資質平庸,劉鈞生前也曾向郭無為說過劉繼恩不是濟世之才。北漢天會十二年(968年)劉鈞病逝,劉繼恩繼位,仍以天會為年號,不久就將郭無為的權力架空。兩個月後,劉繼恩於酒宴後被供奉官侯霸榮刺殺[1],死後無諡號及廟號,史家習稱其為少主。

\subsubsection{天会}

\begin{longtable}{|>{\centering\scriptsize}m{2em}|>{\centering\scriptsize}m{1.3em}|>{\centering}m{8.8em}|}
  % \caption{秦王政}\
  \toprule
  \SimHei \normalsize 年数 & \SimHei \scriptsize 公元 & \SimHei 大事件 \tabularnewline
  % \midrule
  \endfirsthead
  \toprule
  \SimHei \normalsize 年数 & \SimHei \scriptsize 公元 & \SimHei 大事件 \tabularnewline
  \midrule
  \endhead
  \midrule
  元年 & 968 & \tabularnewline
  \bottomrule
\end{longtable}


%%% Local Variables:
%%% mode: latex
%%% TeX-engine: xetex
%%% TeX-master: "../../Main"
%%% End:
