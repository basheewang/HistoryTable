%% -*- coding: utf-8 -*-
%% Time-stamp: <Chen Wang: 2019-12-26 10:14:55>


\section{北汉\tiny(951-979)}

\subsection{简介}

北漢(951年-979年)是中国五代十国時在今山西省北部、陕西省、河北省局部的政權,為十国之一。

951年,后汉被郭威所篡,改国号周,史称后周。郭威并废杀原本将被立为汉帝的後漢高祖劉知遠的养子,也是高祖弟鎮守晉陽的河东节度使刘崇的嫡长子刘赟。刘崇原本以为儿子将被拥立为帝而按兵不动,得知儿子死讯后在太原继位,繼承後漢,但国家疆域和地位已发生巨大变化,史学家将其定位为新政权或残余政权,为别于后汉和南方的南汉,史称北汉。又以所统治的山西古称河东,而被称为“东汉”(如欧阳修《新五代史·東漢世家》)。

统治范围包括今山西北部、陕西、河北部分地区。

为与后周抗衡,曾向辽朝请封,援后晋故事,自称侄皇帝。

北漢国兵役繁重,与后周、北宋进行多次的战争,人口锐减到只有盛唐时的八分之一。

北漢最後在979年宋太宗年間被包围,杨业归宋后,太原城内军心动摇,最终投降,宋太宗在战事中损兵折将,气愤之下将太原城平毁再引汾、晋二水灌城,给屡遭战火的北方百姓又带来严重的损失。

%% -*- coding: utf-8 -*-
%% Time-stamp: <Chen Wang: 2021-11-01 15:50:53>

\subsection{世祖劉旻\tiny(951-954)}

\subsubsection{生平}

漢世祖劉旻(895年-954年),並州晉陽(今山西太原)人,沙陀族,原名刘彦崇、劉崇,五代十國時期北漢開國皇帝,為後漢高祖劉知遠之弟,父劉琠。

劉崇年輕時喜歡飲酒賭博,曾經於臉上刺青從軍。劉知遠於後晉任河東節度使時,他擔任都指揮使。劉知遠建後漢之後,任太原尹,後漢隱帝劉承祐在位時任河東節度使(位於太原),鎮守河東地區。

後漢乾祐三年(950年),樞密使郭威為劉承祐逼反,進軍後漢都城大梁(今河南開封),劉承祐逃亡中為下屬郭允明所殺,郭威遂控制朝政。劉崇此時原欲舉兵南下,但聽到郭威計畫迎立劉崇之長子武寧節度使劉贇為帝,遂打消此意。然而不久郭威被黃旗加身後自登帝位,建立後周,改元廣順,並殺劉贇,當時為951年。因此劉崇隨即亦在太原(今山西省太原市)登帝位,延續後漢,改名劉旻,仍維持乾祐年號,稱乾祐四年。後世把劉崇稱帝後的政權稱作北漢。

北漢地小民貧,又以興復後漢為業,遂向遼國乞援,與遼國約為父子之國,由劉旻稱遼帝為叔,而自稱姪皇帝;遼國則封劉旻為大漢神武皇帝。北漢因遼國的援助,而與後周進行了不少戰爭,但仍勝少敗多。北漢乾祐七年(954年),趁郭威去世之際,聯合遼國南攻後周,然為後周世宗柴榮率軍敗於高平,劉旻穿著農人的衣服隨百餘騎逃走,途中一度迷路,劉旻年老力衰,差點無法支撐回到太原。

經此一役,北漢元氣大傷,無力南下,而劉旻亦憂憤成疾,不久去世,廟號世祖,次子劉承鈞繼位。

\subsubsection{乾佑}

\begin{longtable}{|>{\centering\scriptsize}m{2em}|>{\centering\scriptsize}m{1.3em}|>{\centering}m{8.8em}|}
  % \caption{秦王政}\
  \toprule
  \SimHei \normalsize 年数 & \SimHei \scriptsize 公元 & \SimHei 大事件 \tabularnewline
  % \midrule
  \endfirsthead
  \toprule
  \SimHei \normalsize 年数 & \SimHei \scriptsize 公元 & \SimHei 大事件 \tabularnewline
  \midrule
  \endhead
  \midrule
  元年 & 951 & \tabularnewline\hline
  二年 & 952 & \tabularnewline\hline
  三年 & 953 & \tabularnewline\hline
  四年 & 954 & \tabularnewline
  \bottomrule
\end{longtable}


%%% Local Variables:
%%% mode: latex
%%% TeX-engine: xetex
%%% TeX-master: "../../Main"
%%% End:

%% -*- coding: utf-8 -*-
%% Time-stamp: <Chen Wang: 2019-12-26 10:14:15>

\subsection{睿宗\tiny(954-968)}

\subsubsection{生平}

汉睿宗劉鈞(926年-968年),原名劉承鈞,五代時期北漢在位最久的君主,為劉旻之次子。

劉承鈞個性孝順恭謹,喜歡讀書,擅長書法,北漢乾祐七年(954年),劉旻去世,劉承鈞為遼國冊封為帝之後繼位,不改年號,改名劉鈞。上表於遼帝時都自稱「男」,遼帝下詔時,都稱呼他「兒皇帝」。

劉鈞繼位後,勤政愛民,禮敬士大夫,任用郭無為為相,並減少南侵,因此境內還算安定。然而劉鈞並不像其父事奉遼國之恭敬,以致在位後期遼國援助漸少。

劉鈞於957年,改元天會。天會十二年(968年)忧郁死,諡孝和皇帝,廟號睿宗,劉鈞的外甥同時也是養子的劉繼恩繼位。

\subsubsection{乾佑}

\begin{longtable}{|>{\centering\scriptsize}m{2em}|>{\centering\scriptsize}m{1.3em}|>{\centering}m{8.8em}|}
  % \caption{秦王政}\
  \toprule
  \SimHei \normalsize 年数 & \SimHei \scriptsize 公元 & \SimHei 大事件 \tabularnewline
  % \midrule
  \endfirsthead
  \toprule
  \SimHei \normalsize 年数 & \SimHei \scriptsize 公元 & \SimHei 大事件 \tabularnewline
  \midrule
  \endhead
  \midrule
  元年 & 954 & \tabularnewline\hline
  二年 & 955 & \tabularnewline\hline
  三年 & 956 & \tabularnewline
  \bottomrule
\end{longtable}

\subsubsection{天会}

\begin{longtable}{|>{\centering\scriptsize}m{2em}|>{\centering\scriptsize}m{1.3em}|>{\centering}m{8.8em}|}
  % \caption{秦王政}\
  \toprule
  \SimHei \normalsize 年数 & \SimHei \scriptsize 公元 & \SimHei 大事件 \tabularnewline
  % \midrule
  \endfirsthead
  \toprule
  \SimHei \normalsize 年数 & \SimHei \scriptsize 公元 & \SimHei 大事件 \tabularnewline
  \midrule
  \endhead
  \midrule
  元年 & 957 & \tabularnewline\hline
  二年 & 958 & \tabularnewline\hline
  三年 & 959 & \tabularnewline\hline
  四年 & 960 & \tabularnewline\hline
  五年 & 961 & \tabularnewline\hline
  六年 & 962 & \tabularnewline\hline
  七年 & 963 & \tabularnewline\hline
  八年 & 964 & \tabularnewline\hline
  九年 & 965 & \tabularnewline\hline
  十年 & 966 & \tabularnewline\hline
  十一年 & 967 & \tabularnewline\hline
  十二年 & 968 & \tabularnewline\hline
  \bottomrule
\end{longtable}


%%% Local Variables:
%%% mode: latex
%%% TeX-engine: xetex
%%% TeX-master: "../../Main"
%%% End:

%% -*- coding: utf-8 -*-
%% Time-stamp: <Chen Wang: 2019-12-26 10:14:40>

\subsection{少主\tiny(968)}

\subsubsection{生平}

劉繼恩(10世纪?-968年),本姓薛,五代時期北漢君主,世祖刘旻外孙,睿宗刘钧外甥、养子,史稱「少主」。

其母刘氏為劉旻之女,因此劉鈞是他的舅父;其父薛釗,本來只是士兵,早年娶劉旻之女,劉旻之兄劉知遠發跡後,薛釗就很難見到其妻,因此很不快樂,有一天乘著酒意求見,竟拿刀刺傷其妻,薛釗後來因此自盡。當時劉繼恩年紀還小,劉旻因為劉鈞無子,遂命劉鈞收養劉繼恩。

劉繼恩在劉鈞在位時任太原尹,然而他本人資質平庸,劉鈞生前也曾向郭無為說過劉繼恩不是濟世之才。北漢天會十二年(968年)劉鈞病逝,劉繼恩繼位,仍以天會為年號,不久就將郭無為的權力架空。兩個月後,劉繼恩於酒宴後被供奉官侯霸榮刺殺[1],死後無諡號及廟號,史家習稱其為少主。

\subsubsection{天会}

\begin{longtable}{|>{\centering\scriptsize}m{2em}|>{\centering\scriptsize}m{1.3em}|>{\centering}m{8.8em}|}
  % \caption{秦王政}\
  \toprule
  \SimHei \normalsize 年数 & \SimHei \scriptsize 公元 & \SimHei 大事件 \tabularnewline
  % \midrule
  \endfirsthead
  \toprule
  \SimHei \normalsize 年数 & \SimHei \scriptsize 公元 & \SimHei 大事件 \tabularnewline
  \midrule
  \endhead
  \midrule
  元年 & 968 & \tabularnewline
  \bottomrule
\end{longtable}


%%% Local Variables:
%%% mode: latex
%%% TeX-engine: xetex
%%% TeX-master: "../../Main"
%%% End:

%% -*- coding: utf-8 -*-
%% Time-stamp: <Chen Wang: 2019-12-26 10:15:18>

\subsection{英武帝\tiny(968-979)}

\subsubsection{生平}

漢英武帝劉繼元(10世纪?-992年),本姓何,五代時期北漢君主,世祖刘旻外孙,睿宗刘钧外甥、养子。

其母刘氏為劉旻之女,因此劉鈞是他的舅父。母亲刘氏在劉繼恩之父薛釗自殺後改嫁何氏而生下劉繼元,所以劉繼恩是他的同母異父之兄,劉繼元的父母都去世以後,劉鈞收他為養子。

劉繼元在劉繼恩在位時任太原尹,北漢天會十二年(968年)劉繼恩為侯霸榮刺殺後,劉繼元被司空郭無為迎立為帝,繼位後即緩和與遼國間的緊張關係。劉繼元為人殘忍嗜殺,嫡母劉承鈞之妻郭皇后及劉旻之子皆被其所殺;亦動輒將忤逆他的臣屬滅族。

天會十三年(969年)宋太祖趙匡胤親征北漢,宋軍久攻不下而退兵,北漢收取宋軍所拋棄輜重,瀕臨枯竭的國力賴以恢復。

974年,改年號廣運。廣運六年(宋太平興國四年,979年),宋朝將南方各國併入版圖之後,再度決意北伐,由宋太宗趙光義親征,宋軍攻勢猛烈,遼國援軍亦被擊退,五月初六日劉繼元投降,北漢亡。投降後被任命為右衛上將軍,封彭城郡公。太平興國六年(981年),進封為彭城公。雍熙三年(986年),再被任命為保康軍節度使。淳化二年十二月十八日(陽曆為992年1月25日)去世,被贈中書令,追封為彭城郡王。

\subsubsection{天会}

\begin{longtable}{|>{\centering\scriptsize}m{2em}|>{\centering\scriptsize}m{1.3em}|>{\centering}m{8.8em}|}
  % \caption{秦王政}\
  \toprule
  \SimHei \normalsize 年数 & \SimHei \scriptsize 公元 & \SimHei 大事件 \tabularnewline
  % \midrule
  \endfirsthead
  \toprule
  \SimHei \normalsize 年数 & \SimHei \scriptsize 公元 & \SimHei 大事件 \tabularnewline
  \midrule
  \endhead
  \midrule
  元年 & 968 & \tabularnewline\hline
  二年 & 969 & \tabularnewline\hline
  三年 & 970 & \tabularnewline\hline
  四年 & 971 & \tabularnewline\hline
  五年 & 972 & \tabularnewline\hline
  六年 & 973 & \tabularnewline
  \bottomrule
\end{longtable}

\subsubsection{广运}

\begin{longtable}{|>{\centering\scriptsize}m{2em}|>{\centering\scriptsize}m{1.3em}|>{\centering}m{8.8em}|}
  % \caption{秦王政}\
  \toprule
  \SimHei \normalsize 年数 & \SimHei \scriptsize 公元 & \SimHei 大事件 \tabularnewline
  % \midrule
  \endfirsthead
  \toprule
  \SimHei \normalsize 年数 & \SimHei \scriptsize 公元 & \SimHei 大事件 \tabularnewline
  \midrule
  \endhead
  \midrule
  元年 & 974 & \tabularnewline\hline
  二年 & 975 & \tabularnewline\hline
  三年 & 976 & \tabularnewline\hline
  四年 & 977 & \tabularnewline\hline
  五年 & 978 & \tabularnewline\hline
  六年 & 979 & \tabularnewline
  \bottomrule
\end{longtable}


%%% Local Variables:
%%% mode: latex
%%% TeX-engine: xetex
%%% TeX-master: "../../Main"
%%% End:



%%% Local Variables:
%%% mode: latex
%%% TeX-engine: xetex
%%% TeX-master: "../../Main"
%%% End:
