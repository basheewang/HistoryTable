%% -*- coding: utf-8 -*-
%% Time-stamp: <Chen Wang: 2021-11-01 16:13:57>

\section{毅宗李諒祚\tiny(1048-1067)}

\subsection{生平}

夏毅宗李諒祚(1047年3月5日-1068年1月),名諒祚,本名寧令兩岔,是西夏第二位皇帝(1048年—1068年1月在位)。夏景宗之子,生母沒藏氏,党項族人。生于1047年农历二月六日。

嘉祐八年丙辰(1063年),谅祚派使者前往宋朝,上表要求改回李姓,宋仁宗下诏谴责他,命令他遵守旧约。然而此后西夏皇室仍多用嵬名为姓氏。

1048年1月19日,景宗被殺,毅宗以一歲幼齡繼位,其母沒藏太后及其家族專權。即位次年(1049年),遼國乘景宗新喪之機,與西夏爆發第二次賀蘭山之戰,西夏大敗,損失慘重,向遼稱臣。福聖承道四年(1056年),太后被殺,舅舅沒藏訛龐執政。十二歲開始預政。奲都五年(1061年),訛龐父子密謀害他,遂殺訛龐及皇后(訛龐之女),立梁氏為皇后,親掌國政。廢行蕃禮,改用漢儀;並增設各部尚書、侍郎等多種官職,以完善中央行政體制。調整州軍,以加強對地方統治。這些措施使皇帝對軍政權力的控制得到加強。他連年對宋用兵,攻掠臨近州縣。先後收降吐蕃首領瞎氈的兒子木征和青唐吐蕃部。後期注意修好與遼、宋關係,減少戰役。拱化四年(1066年),在与北宋作战时受箭伤,拱化五年十二月(1068年1月)去世,享年僅21岁,諡号昭英皇帝。

\subsection{延嗣宁国}

\begin{longtable}{|>{\centering\scriptsize}m{2em}|>{\centering\scriptsize}m{1.3em}|>{\centering}m{8.8em}|}
  % \caption{秦王政}\
  \toprule
  \SimHei \normalsize 年数 & \SimHei \scriptsize 公元 & \SimHei 大事件 \tabularnewline
  % \midrule
  \endfirsthead
  \toprule
  \SimHei \normalsize 年数 & \SimHei \scriptsize 公元 & \SimHei 大事件 \tabularnewline
  \midrule
  \endhead
  \midrule
  元年 & 1048 & \tabularnewline
  \bottomrule
\end{longtable}

\subsection{天祐垂圣}

\begin{longtable}{|>{\centering\scriptsize}m{2em}|>{\centering\scriptsize}m{1.3em}|>{\centering}m{8.8em}|}
  % \caption{秦王政}\
  \toprule
  \SimHei \normalsize 年数 & \SimHei \scriptsize 公元 & \SimHei 大事件 \tabularnewline
  % \midrule
  \endfirsthead
  \toprule
  \SimHei \normalsize 年数 & \SimHei \scriptsize 公元 & \SimHei 大事件 \tabularnewline
  \midrule
  \endhead
  \midrule
  元年 & 1050 & \tabularnewline\hline
  二年 & 1051 & \tabularnewline\hline
  三年 & 1052 & \tabularnewline
  \bottomrule
\end{longtable}

\subsection{福圣承道}

\begin{longtable}{|>{\centering\scriptsize}m{2em}|>{\centering\scriptsize}m{1.3em}|>{\centering}m{8.8em}|}
  % \caption{秦王政}\
  \toprule
  \SimHei \normalsize 年数 & \SimHei \scriptsize 公元 & \SimHei 大事件 \tabularnewline
  % \midrule
  \endfirsthead
  \toprule
  \SimHei \normalsize 年数 & \SimHei \scriptsize 公元 & \SimHei 大事件 \tabularnewline
  \midrule
  \endhead
  \midrule
  元年 & 1053 & \tabularnewline\hline
  二年 & 1054 & \tabularnewline\hline
  三年 & 1055 & \tabularnewline\hline
  四年 & 1056 & \tabularnewline
  \bottomrule
\end{longtable}

\subsection{奲都}

\begin{longtable}{|>{\centering\scriptsize}m{2em}|>{\centering\scriptsize}m{1.3em}|>{\centering}m{8.8em}|}
  % \caption{秦王政}\
  \toprule
  \SimHei \normalsize 年数 & \SimHei \scriptsize 公元 & \SimHei 大事件 \tabularnewline
  % \midrule
  \endfirsthead
  \toprule
  \SimHei \normalsize 年数 & \SimHei \scriptsize 公元 & \SimHei 大事件 \tabularnewline
  \midrule
  \endhead
  \midrule
  元年 & 1057 & \tabularnewline\hline
  二年 & 1058 & \tabularnewline\hline
  三年 & 1059 & \tabularnewline\hline
  四年 & 1060 & \tabularnewline\hline
  五年 & 1061 & \tabularnewline\hline
  六年 & 1062 & \tabularnewline
  \bottomrule
\end{longtable}

\subsection{拱化}

\begin{longtable}{|>{\centering\scriptsize}m{2em}|>{\centering\scriptsize}m{1.3em}|>{\centering}m{8.8em}|}
  % \caption{秦王政}\
  \toprule
  \SimHei \normalsize 年数 & \SimHei \scriptsize 公元 & \SimHei 大事件 \tabularnewline
  % \midrule
  \endfirsthead
  \toprule
  \SimHei \normalsize 年数 & \SimHei \scriptsize 公元 & \SimHei 大事件 \tabularnewline
  \midrule
  \endhead
  \midrule
  元年 & 1063 & \tabularnewline\hline
  二年 & 1064 & \tabularnewline\hline
  三年 & 1065 & \tabularnewline\hline
  四年 & 1066 & \tabularnewline\hline
  五年 & 1067 & \tabularnewline
  \bottomrule
\end{longtable}


%%% Local Variables:
%%% mode: latex
%%% TeX-engine: xetex
%%% TeX-master: "../Main"
%%% End:
