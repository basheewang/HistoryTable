%% -*- coding: utf-8 -*-
%% Time-stamp: <Chen Wang: 2019-12-26 11:08:11>

\section{景宗\tiny(1032-1048)}

\subsection{生平}

夏景宗李元昊(1003年6月7日-1048年1月19日),又名趙元昊,小字嵬埋,出身党項拓跋氏,即皇帝位後,放棄唐朝賜姓李與宋朝赐姓趙,改姓嵬名氏,更名曩霄(或作曩甯、曩宁),是西夏開國皇帝(1038年11月10日-1048年1月19日在位),為李繼遷孫,李德明長子,生母衛慕氏。生于1003年农历五月五日。

李元昊少年時身型魁梧,而且勤奮好學,手不釋卷,尤好法律和兵書。通漢、蕃語言,精繪畫,多才多藝。其父在位時,他率軍不斷對外出戰,擴大勢力,野心勃勃。1032年以太子身份繼位,仍称藩於宋朝。後來為表獨立,廢唐宋分別賜李姓、趙姓,改姓嵬名,改名曩霄,自称“兀卒”(党项语天子之意),以元魏王室后裔自居,並以嚴酷手段徹底翦除守舊派。大庆三年十月十一日(1038年11月10日)自立为帝,自称世祖始文本武兴法建礼仁孝皇帝,改年号为天授礼法延祚,脱离北宋,国号“大夏”,亦称西夏,定都興慶府。

建國後命大臣野利仁榮創西夏文,大力發展西夏的文化。推動教育,創蕃學,大啟西夏文教之風。開鑿「李王渠」,以便西夏國民耕種。他重用張元等漢人。他三次分別於三川口(今陝西延安西北)、好水川(今寧夏隆德東)及定川砦(今寧夏固原西北)的戰役中大敗北宋,並於遼夏第一次賀蘭山之戰,大勝遼國,奠定西夏與遼、宋兩國并列的地位。本來有意奪取關中之地,攻占長安,但因宋軍頑強抵抗,夏軍戰敗,直搗關中之美夢就此破滅。由於戰事繁多,西夏經濟破損,遂於1044年與北宋簽訂慶曆和議,向宋稱臣,被封為夏國王。為西夏建樹良多,堪稱一代英豪。

李元昊一朝文治武功達於鼎盛,但其人亦有不足之處。在位16年(1032年继承王位起計),猜忌功臣,稍有不滿即罷或殺,反而導致日後母黨專權;另外,晚年沉湎酒色,好大喜功,导致西夏内部日益腐朽,众叛亲离。據說他下令民伕每日建一座陵墓,足足建了三百六十座,作為他的疑塚,其後竟把那批民伕統統殺掉。廢皇后野利氏、太子寧令哥,改立與太子訂親的沒移氏為新皇后,招致殺身之禍,延祚十一年正月初二(1048年1月19日),其子寧令哥趁元昊酒醉時,割其鼻子,元昊最後因失血過多而死,享年46岁,庙号景宗,諡号武烈皇帝,葬泰陵。寧令哥後來因弒父之罪被處死。

\subsection{显道}

\begin{longtable}{|>{\centering\scriptsize}m{2em}|>{\centering\scriptsize}m{1.3em}|>{\centering}m{8.8em}|}
  % \caption{秦王政}\
  \toprule
  \SimHei \normalsize 年数 & \SimHei \scriptsize 公元 & \SimHei 大事件 \tabularnewline
  % \midrule
  \endfirsthead
  \toprule
  \SimHei \normalsize 年数 & \SimHei \scriptsize 公元 & \SimHei 大事件 \tabularnewline
  \midrule
  \endhead
  \midrule
  元年 & 1032 & \tabularnewline\hline
  二年 & 1033 & \tabularnewline\hline
  三年 & 1034 & \tabularnewline
  \bottomrule
\end{longtable}

\subsection{开运}

\begin{longtable}{|>{\centering\scriptsize}m{2em}|>{\centering\scriptsize}m{1.3em}|>{\centering}m{8.8em}|}
  % \caption{秦王政}\
  \toprule
  \SimHei \normalsize 年数 & \SimHei \scriptsize 公元 & \SimHei 大事件 \tabularnewline
  % \midrule
  \endfirsthead
  \toprule
  \SimHei \normalsize 年数 & \SimHei \scriptsize 公元 & \SimHei 大事件 \tabularnewline
  \midrule
  \endhead
  \midrule
  元年 & 1034 & \tabularnewline
  \bottomrule
\end{longtable}

\subsection{广运}

\begin{longtable}{|>{\centering\scriptsize}m{2em}|>{\centering\scriptsize}m{1.3em}|>{\centering}m{8.8em}|}
  % \caption{秦王政}\
  \toprule
  \SimHei \normalsize 年数 & \SimHei \scriptsize 公元 & \SimHei 大事件 \tabularnewline
  % \midrule
  \endfirsthead
  \toprule
  \SimHei \normalsize 年数 & \SimHei \scriptsize 公元 & \SimHei 大事件 \tabularnewline
  \midrule
  \endhead
  \midrule
  元年 & 1034 & \tabularnewline\hline
  二年 & 1035 & \tabularnewline\hline
  三年 & 1036 & \tabularnewline
  \bottomrule
\end{longtable}

\subsection{大庆}

\begin{longtable}{|>{\centering\scriptsize}m{2em}|>{\centering\scriptsize}m{1.3em}|>{\centering}m{8.8em}|}
  % \caption{秦王政}\
  \toprule
  \SimHei \normalsize 年数 & \SimHei \scriptsize 公元 & \SimHei 大事件 \tabularnewline
  % \midrule
  \endfirsthead
  \toprule
  \SimHei \normalsize 年数 & \SimHei \scriptsize 公元 & \SimHei 大事件 \tabularnewline
  \midrule
  \endhead
  \midrule
  元年 & 1036 & \tabularnewline\hline
  二年 & 1037 & \tabularnewline\hline
  三年 & 1038 & \tabularnewline
  \bottomrule
\end{longtable}

\subsection{天授}

\begin{longtable}{|>{\centering\scriptsize}m{2em}|>{\centering\scriptsize}m{1.3em}|>{\centering}m{8.8em}|}
  % \caption{秦王政}\
  \toprule
  \SimHei \normalsize 年数 & \SimHei \scriptsize 公元 & \SimHei 大事件 \tabularnewline
  % \midrule
  \endfirsthead
  \toprule
  \SimHei \normalsize 年数 & \SimHei \scriptsize 公元 & \SimHei 大事件 \tabularnewline
  \midrule
  \endhead
  \midrule
  元年 & 1038 & \tabularnewline\hline
  二年 & 1039 & \tabularnewline\hline
  三年 & 1040 & \tabularnewline\hline
  四年 & 1041 & \tabularnewline\hline
  五年 & 1042 & \tabularnewline\hline
  六年 & 1043 & \tabularnewline\hline
  七年 & 1044 & \tabularnewline\hline
  八年 & 1045 & \tabularnewline\hline
  九年 & 1046 & \tabularnewline\hline
  十年 & 1047 & \tabularnewline\hline
  十一年 & 1048 & \tabularnewline
  \bottomrule
\end{longtable}


%%% Local Variables:
%%% mode: latex
%%% TeX-engine: xetex
%%% TeX-master: "../Main"
%%% End:
