%% -*- coding: utf-8 -*-
%% Time-stamp: <Chen Wang: 2021-11-01 16:15:08>

\section{末帝李睍\tiny(1226-1227)}

\subsection{生平}
李\xpinyin*{睍}(?-1227年8月28日),夏献宗侄,父清平郡王,夏獻宗病危時被推舉為帝,史稱「末帝」。

李睍在位時已為西夏滅亡的前夕,曾拒降蒙古。右丞相高良惠及各將士積極抵抗蒙古,但無奈天意造化弄人,中興府發生大地震,以致瘟疫肆虐,糧水短缺,軍民死傷過半,西夏注定滅亡了。最後於1227年农历六月投降蒙古,1227年8月25日,成吉思汗病故,蒙軍恐夏主有變,李睍开城投降后,按照成吉思汗遗嘱,在成吉思汗去世三天后,也就是寶義二年七月十五日(1227年8月28日),李睍被杀。西夏滅亡後,蒙古軍商议屠城,最終在出身党項族的蒙古軍將领察罕的极力勸諫下使中興府內的百姓避免了被屠殺的命运,察罕随后入城安撫城内軍民,使西夏遗民得以保全。


\subsection{宝义}

\begin{longtable}{|>{\centering\scriptsize}m{2em}|>{\centering\scriptsize}m{1.3em}|>{\centering}m{8.8em}|}
  % \caption{秦王政}\
  \toprule
  \SimHei \normalsize 年数 & \SimHei \scriptsize 公元 & \SimHei 大事件 \tabularnewline
  % \midrule
  \endfirsthead
  \toprule
  \SimHei \normalsize 年数 & \SimHei \scriptsize 公元 & \SimHei 大事件 \tabularnewline
  \midrule
  \endhead
  \midrule
  元年 & 1226 & \tabularnewline\hline
  二年 & 1227 & \tabularnewline
  \bottomrule
\end{longtable}


%%% Local Variables:
%%% mode: latex
%%% TeX-engine: xetex
%%% TeX-master: "../Main"
%%% End:
