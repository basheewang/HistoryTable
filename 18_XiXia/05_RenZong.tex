%% -*- coding: utf-8 -*-
%% Time-stamp: <Chen Wang: 2021-11-01 16:14:27>

\section{仁宗李仁孝\tiny(1139-1193)}

\subsection{生平}

夏仁宗李仁孝(1124年-1193年10月16日),夏崇宗次子,母為漢人,不知名。

由于其兄李仁爱先于崇宗而死,故被立为太子。1139年7月1日夏崇宗李乾顺去世,李仁孝即位,時年十六歲。在位期間結好金國,以穩定外部環境;重用文化程度較高的党項和漢族大臣主持國政;設立各級學校,以推廣教育;實行科舉,以選拔人才;尊崇儒學,大修孔廟及尊奉孔子為文宣帝;建立翰林學士院,編纂歷朝實錄;重視禮樂,修樂書《新律》;天盛年間,頒行法典《天盛年改新定律令》;尊尚佛教,供奉藏傳佛教僧人為國師,並刻印佛經多種。

乾祐元年(1170年),得金之助,處死權相任得敬,粉碎其分國陰謀。可能因為任得敬的專權跋扈,令仁孝對武官不太信任,政策多數重文輕武,導致軍備開始廢弛,戰鬥力減弱,晚夏戰爭屢戰屢敗,國家於仁宗末年開始走下坡。但總結來說,他統治期間為西夏的盛世,也是金國、南宋的盛世,三國之間戰爭甚少,因此仁孝能專心料理國家內政。各汗國羡慕西夏之強盛,紛紛朝貢。文化臻於鼎盛,為党項文化寫下光輝燦爛的一頁。

乾祐二十四年九月二十日(1193年10月16日)崩,年七十,諡聖德皇帝,廟號仁宗。

\subsection{大庆}

\begin{longtable}{|>{\centering\scriptsize}m{2em}|>{\centering\scriptsize}m{1.3em}|>{\centering}m{8.8em}|}
  % \caption{秦王政}\
  \toprule
  \SimHei \normalsize 年数 & \SimHei \scriptsize 公元 & \SimHei 大事件 \tabularnewline
  % \midrule
  \endfirsthead
  \toprule
  \SimHei \normalsize 年数 & \SimHei \scriptsize 公元 & \SimHei 大事件 \tabularnewline
  \midrule
  \endhead
  \midrule
  元年 & 1140 & \tabularnewline\hline
  二年 & 1141 & \tabularnewline\hline
  三年 & 1142 & \tabularnewline\hline
  四年 & 1143 & \tabularnewline
  \bottomrule
\end{longtable}

\subsection{人庆}

\begin{longtable}{|>{\centering\scriptsize}m{2em}|>{\centering\scriptsize}m{1.3em}|>{\centering}m{8.8em}|}
  % \caption{秦王政}\
  \toprule
  \SimHei \normalsize 年数 & \SimHei \scriptsize 公元 & \SimHei 大事件 \tabularnewline
  % \midrule
  \endfirsthead
  \toprule
  \SimHei \normalsize 年数 & \SimHei \scriptsize 公元 & \SimHei 大事件 \tabularnewline
  \midrule
  \endhead
  \midrule
  元年 & 1144 & \tabularnewline\hline
  二年 & 1145 & \tabularnewline\hline
  三年 & 1146 & \tabularnewline\hline
  四年 & 1147 & \tabularnewline\hline
  五年 & 1148 & \tabularnewline
  \bottomrule
\end{longtable}

\subsection{天盛}

\begin{longtable}{|>{\centering\scriptsize}m{2em}|>{\centering\scriptsize}m{1.3em}|>{\centering}m{8.8em}|}
  % \caption{秦王政}\
  \toprule
  \SimHei \normalsize 年数 & \SimHei \scriptsize 公元 & \SimHei 大事件 \tabularnewline
  % \midrule
  \endfirsthead
  \toprule
  \SimHei \normalsize 年数 & \SimHei \scriptsize 公元 & \SimHei 大事件 \tabularnewline
  \midrule
  \endhead
  \midrule
  元年 & 1149 & \tabularnewline\hline
  二年 & 1150 & \tabularnewline\hline
  三年 & 1151 & \tabularnewline\hline
  四年 & 1152 & \tabularnewline\hline
  五年 & 1153 & \tabularnewline\hline
  六年 & 1154 & \tabularnewline\hline
  七年 & 1155 & \tabularnewline\hline
  八年 & 1156 & \tabularnewline\hline
  九年 & 1157 & \tabularnewline\hline
  十年 & 1158 & \tabularnewline\hline
  十一年 & 1159 & \tabularnewline\hline
  十二年 & 1160 & \tabularnewline\hline
  十三年 & 1161 & \tabularnewline\hline
  十四年 & 1162 & \tabularnewline\hline
  十五年 & 1163 & \tabularnewline\hline
  十六年 & 1164 & \tabularnewline\hline
  十七年 & 1165 & \tabularnewline\hline
  十八年 & 1166 & \tabularnewline\hline
  十九年 & 1167 & \tabularnewline\hline
  二十年 & 1168 & \tabularnewline\hline
  二一年 & 1169 & \tabularnewline
  \bottomrule
\end{longtable}

\subsection{乾佑}

\begin{longtable}{|>{\centering\scriptsize}m{2em}|>{\centering\scriptsize}m{1.3em}|>{\centering}m{8.8em}|}
  % \caption{秦王政}\
  \toprule
  \SimHei \normalsize 年数 & \SimHei \scriptsize 公元 & \SimHei 大事件 \tabularnewline
  % \midrule
  \endfirsthead
  \toprule
  \SimHei \normalsize 年数 & \SimHei \scriptsize 公元 & \SimHei 大事件 \tabularnewline
  \midrule
  \endhead
  \midrule
  元年 & 1170 & \tabularnewline\hline
  二年 & 1171 & \tabularnewline\hline
  三年 & 1172 & \tabularnewline\hline
  四年 & 1173 & \tabularnewline\hline
  五年 & 1174 & \tabularnewline\hline
  六年 & 1175 & \tabularnewline\hline
  七年 & 1176 & \tabularnewline\hline
  八年 & 1177 & \tabularnewline\hline
  九年 & 1178 & \tabularnewline\hline
  十年 & 1179 & \tabularnewline\hline
  十一年 & 1180 & \tabularnewline\hline
  十二年 & 1181 & \tabularnewline\hline
  十三年 & 1182 & \tabularnewline\hline
  十四年 & 1183 & \tabularnewline\hline
  十五年 & 1184 & \tabularnewline\hline
  十六年 & 1185 & \tabularnewline\hline
  十七年 & 1186 & \tabularnewline\hline
  十八年 & 1187 & \tabularnewline\hline
  十九年 & 1188 & \tabularnewline\hline
  二十年 & 1189 & \tabularnewline\hline
  二一年 & 1190 & \tabularnewline\hline
  二二年 & 1191 & \tabularnewline\hline
  二三年 & 1192 & \tabularnewline\hline
  二四年 & 1193 & \tabularnewline
  \bottomrule
\end{longtable}


%%% Local Variables:
%%% mode: latex
%%% TeX-engine: xetex
%%% TeX-master: "../Main"
%%% End:
