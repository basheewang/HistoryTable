%% -*- coding: utf-8 -*-
%% Time-stamp: <Chen Wang: 2021-11-01 16:14:31>

\section{桓宗李純佑\tiny(1193-1206)}

\subsection{生平}

夏桓宗李純佑(1177年-1206年3月),夏仁宗子。性溫和厚實。1193年10月16日,夏仁宗李仁孝去世,李純佑即位,時年十七,改元天慶。桓宗基本上還能奉行仁宗时期的政治方針和外交政策,對内安國養民,對外附金和宋。但隨着國家的安定承平日久,統治階層開始貪圖安逸,日益腐朽墮落,从此西夏政治日益腐败,国势衰落。從盛到衰已成為西夏社會不可逆轉的趨勢。同时,桓宗統治時期,正是蒙古崛起並日漸強大的时期,來自蒙古的嚴重威脅加速了西夏由盛而衰的歷史進程。1205年,改興慶府名為中興府,取夏國中興之意,但同年蒙古第一次進攻西夏,自此迄無寧日了。1206年3月1日,李安全發動政變,桓宗被廢。不久暴卒,年僅三十。諡昭簡皇帝。

\subsection{天庆}

\begin{longtable}{|>{\centering\scriptsize}m{2em}|>{\centering\scriptsize}m{1.3em}|>{\centering}m{8.8em}|}
  % \caption{秦王政}\
  \toprule
  \SimHei \normalsize 年数 & \SimHei \scriptsize 公元 & \SimHei 大事件 \tabularnewline
  % \midrule
  \endfirsthead
  \toprule
  \SimHei \normalsize 年数 & \SimHei \scriptsize 公元 & \SimHei 大事件 \tabularnewline
  \midrule
  \endhead
  \midrule
  元年 & 1194 & \tabularnewline\hline
  二年 & 1195 & \tabularnewline\hline
  三年 & 1196 & \tabularnewline\hline
  四年 & 1197 & \tabularnewline\hline
  五年 & 1198 & \tabularnewline\hline
  六年 & 1199 & \tabularnewline\hline
  七年 & 1200 & \tabularnewline\hline
  八年 & 1201 & \tabularnewline\hline
  九年 & 1202 & \tabularnewline\hline
  十年 & 1203 & \tabularnewline\hline
  十一年 & 1204 & \tabularnewline\hline
  十二年 & 1205 & \tabularnewline\hline
  十三年 & 1206 & \tabularnewline
  \bottomrule
\end{longtable}


%%% Local Variables:
%%% mode: latex
%%% TeX-engine: xetex
%%% TeX-master: "../Main"
%%% End:
