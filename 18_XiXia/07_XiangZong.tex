%% -*- coding: utf-8 -*-
%% Time-stamp: <Chen Wang: 2021-11-01 16:14:37>

\section{襄宗李安全\tiny(1206-1211)}

\subsection{生平}

夏襄宗李安全(1170年-1211年9月13日),夏崇宗孫,其父乃夏仁宗弟越王李仁友,1196年,仁友逝,安全上書要求襲越王爵位,桓宗不許,安全被降封為鎮夷郡王,他極為不滿,於是萌生了篡奪皇位之心。1206年3月1日與桓宗母羅氏合謀,廢桓宗自立,改元應天。在位時昏庸無能,破壞金國與西夏長期的友好關係,發兵侵金,為後來一場場令夏金耗盡精兵的戰役掀起序幕,改附不斷強大起來的蒙古帝国,但這一切都沒有為西夏帶來利益和跟蒙古之友好,蒙古也以西夏作為侵略目標,西夏不斷積弱。1211年8月12日,宗室齊王李遵頊發動政變,被廢,並於一個月後不明不白地死去,去世于1211年9月13日,終年四十二。諡敬慕皇帝,廟號襄宗。

\subsection{应天}

\begin{longtable}{|>{\centering\scriptsize}m{2em}|>{\centering\scriptsize}m{1.3em}|>{\centering}m{8.8em}|}
  % \caption{秦王政}\
  \toprule
  \SimHei \normalsize 年数 & \SimHei \scriptsize 公元 & \SimHei 大事件 \tabularnewline
  % \midrule
  \endfirsthead
  \toprule
  \SimHei \normalsize 年数 & \SimHei \scriptsize 公元 & \SimHei 大事件 \tabularnewline
  \midrule
  \endhead
  \midrule
  元年 & 1206 & \tabularnewline\hline
  二年 & 1207 & \tabularnewline\hline
  三年 & 1208 & \tabularnewline\hline
  四年 & 1209 & \tabularnewline
  \bottomrule
\end{longtable}

\subsection{皇建}

\begin{longtable}{|>{\centering\scriptsize}m{2em}|>{\centering\scriptsize}m{1.3em}|>{\centering}m{8.8em}|}
  % \caption{秦王政}\
  \toprule
  \SimHei \normalsize 年数 & \SimHei \scriptsize 公元 & \SimHei 大事件 \tabularnewline
  % \midrule
  \endfirsthead
  \toprule
  \SimHei \normalsize 年数 & \SimHei \scriptsize 公元 & \SimHei 大事件 \tabularnewline
  \midrule
  \endhead
  \midrule
  元年 & 1210 & \tabularnewline\hline
  二年 & 1211 & \tabularnewline
  \bottomrule
\end{longtable}


%%% Local Variables:
%%% mode: latex
%%% TeX-engine: xetex
%%% TeX-master: "../Main"
%%% End:
