%% -*- coding: utf-8 -*-
%% Time-stamp: <Chen Wang: 2019-12-26 11:12:29>

\chapter{西夏\tiny(1038-1227)}

\subsection{简介}

西夏(1038年-1227年),国号大夏、邦泥定国或白高大夏國等,是中國歷史上由党項族建立的一個朝代。主要以党項族為主體,包括漢族、回鶻族與吐蕃族等民族在內的國家。因位於中原地區的西北方,國土佔據黃河中上游,史稱西夏。

党項族原居四川松潘高原,唐朝時遷居陕北。因平亂有功被唐帝封為夏州節度使,先後臣服於唐朝、五代諸朝與宋朝。夏州政權被北宋併吞後,由於李繼遷不願投降而再次立國,並且取得遼帝的冊封。李繼遷採取連遼抵宋的方式,陸續占領蘭州與河西走廊地區。1038年11月10日李元昊稱帝建國,即夏景宗,西夏正式立國。西夏在宋夏戰爭與遼夏戰爭中,戰況大致上取得優勢,形成三國鼎立的局面。夏景宗去世後,大權掌握在皇帝的太后與母黨手中,史稱母黨專政時期。西夏因為皇黨與母黨的對峙而內亂,北宋趁機多次伐夏。西夏抵禦成功並擊潰宋軍,但是橫山的喪失讓防線出現破洞。金朝崛起並滅遼、北宋,西夏改臣服金朝,獲得不少土地,兩國建立金夏同盟,大致上維持著和平關係。夏仁宗期間發生天災與任得敬分國事件,但經過改革後,到天盛年間出現盛世。然而漠北的蒙古帝國崛起,六次入侵西夏後拆散金夏同盟,讓西夏與金朝自相殘殺。西夏內部也多次發生弒君、內亂之事,經濟也因戰爭而趨於崩潰。最後於1227年8月28日亡於蒙古。

西夏屬於番漢聯合政治,以党項族為主導,漢族與其他族群為輔。制度由番漢兩元政治逐漸變成一元化的漢法制度。西夏的皇權備受貴族、母黨與權臣等勢力的挑戰而動盪不安。由於處於列強環視的河西走廊與河套地區,對外採取依附強者,攻擊弱者、以戰求和的外交策略。軍事手段十分靈活,配合沙漠地形,採取有利則進,不利則退,誘敵設伏、斷敵糧道的戰術;並且有铁鹞子、步跋子與潑喜等特殊兵種輔助。經濟方面以畜牧業與商業為主力,對外貿易易受中原王朝的影響,壟斷河西走廊與北宋的歲幣為西夏經濟帶來很大的幫助。

西夏是一個佛教王國,興建大量的佛塔與佛寺,以承天寺塔最有名。然而也是崇尚儒學漢法的帝國,立國前積極漢化;雖然夏景宗為了維護本身文化而提倡党項、吐蕃與回鶻文化,並且創立西夏文、立番官、建番俗等措施;但自夏毅宗到夏仁宗後,西夏已經由番漢同行轉為普遍漢化。文學方面以詩歌和諺語為主。在藝術方面於敦煌莫高窟、安西榆林窟有豐富的佛教壁畫,具有「綠壁畫」的特色。此外在雕塑、音樂與舞蹈等方面都有獨特之處。

西夏由党項族所建立。党項族是羌族的一支,帶有鮮卑的血統。唐朝時居住在四川松潘高原一帶,是唐朝的羈糜州之一。當時分有八部,以拓跋氏最為強盛。吐蕃於唐朝安史之亂後占領河西一帶,並且壓迫党項。唐高宗時期,党項首領拓跋赤辞在唐朝幫助下遷移到陝北一帶,奠定党項興起的根據地。881年占據宥州的平夏部拓跋思恭因平黄巢之亂有功,被唐僖宗封为夏州節度使,賜號定難軍。協助收復長安後又封夏国公,賜姓李,領有夏銀等地,夏州政權(正式稱呼是夏州節度使或定難軍)形成一個割據陝北的籓鎮。五代十國時,夏州政權避免介入中原各勢力間的內鬥,向五代與北漢稱臣以鞏固在陝北的勢力。然而後唐時,唐明宗意圖將延州節度使安從進與夏州節度使李彝超對調以併吞夏州政權。李彝超極力反對,成功擊退安從進率領的後唐軍,在北宋初年時累積雄厚的實力。

960年宋太祖赵匡胤建立宋朝後,夏州政權首領李彝殷向北宋稱臣,並且多次協助北宋對抗北漢。當北宋陸續平定南方各國後,宋太宗開始將目光放在北方,有意削除夏州政權。此時李氏家族反對李繼捧擔任夏州節度使。982年宋太宗招李繼捧與其族人遷居開封,命親宋的李克文繼任之,夏州政權被北宋併吞。李繼捧族弟李繼遷不願投降宋朝,率族人逃往地斤澤(今陝西橫山縣東北)抗宋。984年宋將尹憲、曹光實擊破夏軍。隔年李繼遷由弱轉強,攻破宋軍後陸續收復銀、夏等等夏州領地。990年被遼朝遼聖宗冊封為夏國王,即被追尊的夏太祖。宋廷採取以夷制夷方式,派李繼捧回任夏州,招撫李繼遷任銀州,並對此二人賜姓趙。不久李繼遷又叛,於996年擊退宋將李繼隆率領的五路大軍。在鞏固夏州領地後,一直努力西擴河西,最後於1002年第三次攻打靈州(今寧夏靈武西南)時成功攻下,改名西平府。宋朝至此無力圍堵,於隔年承認李繼遷領有夏州領地。李繼遷陸續占領涼州(今甘肅武威縣)等河西重鎮,擊退與宋朝聯手的河西涼州吐蕃六谷部。兩年後李繼遷被六谷部首領潘羅支襲擊而亡,其子李德明繼位,即被追尊的夏太宗。

李德明繼位後,因為國土快速膨脹,為了穩固國力以抵禦四方敵國,有意與宋和談;而北宋對外戰事也由擴張轉為和平,與遼朝簽署澶淵之盟後也希望西北也能穩定下來。最後雙方於1006年雙方簽署景德和議。李德明為了維護自身獨立,東和宋朝,北附遼朝,並讓太子李元昊迎娶遼朝的興平公主。對內方面,李德明定都興州(今宁夏银川东南),採取保境安民、注重生產的策略。並且請求北宋於保安軍(今陝西丹縣)設置榷场,聽許兩國貿易。此外積極西征河西,1028年派太子李元昊攻下甘州(今甘肅張掖縣),甘州回鶻首領夜落隔通順自殺,降服吐蕃六谷部首領折逋游龙钵。而後又奪肅州,降服瓜州歸義軍的曹賢順。至此夏州政權國力大盛,為日後李元昊稱帝立國建立穩固的基礎。1032年李德明去世,其子李元昊继位。

李元昊繼位後完成河西走廊的佔領,並且積極準備脫離宋朝獨立建國。他首先棄李姓,自稱嵬名氏,以北魏王室後裔自居。採楊守素的建議,以避父諱為由,改宋朝年號明道为显道,以建立自己的年號。隨後建宮殿,下禿髮令,恢復故俗,都興慶府,設文武二班,立軍名,用兵制,創造西夏文,改定禮樂等。

1038年11月10日(大庆三年十月十一日)李元昊稱帝,即夏景宗,改年号为天授礼法延祚,定都興州並改稱為興慶府,国号“大夏”,亦称西夏,至此西夏正式立國。夏景宗霸氣縱橫,脫離對北宋、遼朝的臣屬關係。為了要獨霸西方,他四處擴土,先後和宋遼開戰,為西夏武力顛峰的時候。

夏景宗於隔年採取聯遼抗宋的戰略不斷入侵宋邊境,並且要求宋朝承認西夏獨立。當時北宋在橫山山脈一帶建立不少堡壘,不過東方重鎮延州防禦薄弱,守將范雍無能。1040年夏景宗發動三川口之戰,率10萬大軍包圍延州,於三川口襲擊宋將劉平、石元孫的援軍,最後夏軍因大雪而解圍撤退。宋廷面對西夏大舉入侵,派夏竦為正使、韓琦與范仲淹為副使經略西夏。當時宋軍兵多於西夏,但是不擅野戰、補給也不易,主攻的韓琦與主守的范仲淹對此爭執不斷。1041年夏景宗發動大軍包圍宋朝西線的渭川、懷遠一帶,韓琦不聽派范仲淹建議派大將任福率大軍救援懷遠,夏景宗引誘至埋伏地好水川口襲擊,此即好水川之戰。此後宋廷轉為防禦,改派陳執中、夏竦經略,並且建立四路防線。1042年西夏謀臣張元建議避開宋防線,繞道奇襲京兆府(長安)。同年夏景宗於防線薄弱的涇原路發動定川寨之戰,於定川寨包圍殲滅宋軍,目標是長安,但另一路遇到原州景泰的阻擊而罷。宋夏戰爭一直到1044年才平息。雙方簽訂慶曆和議,宋朝承認西夏的割據地位,給予若干財物茶葉,封夏景帝為夏國主。西夏對宋稱臣,但對內依舊稱帝,實際上仍然是獨立的國家。

西夏擊敗宋朝後,自稱西朝,稱遼朝為北朝。遼朝遼興宗不滿西夏壯大,意圖再度壓服之。1043年在慶曆增幣后,遼興宗為報答宋朝,以國內西南部的党項叛附西夏為由,於隔年冬率大軍伐夏。西夏求和不成,採取堅壁清野方式擊潰遼軍。戰後四年夏景宗去世之際,遼軍又於1049年來犯。夏軍極力抵抗,最後雙方和談而止。

夏景宗建國西夏並推行中央集權,雖然鞏固了帝權,但同時與貴族的矛盾進一步加深。他獨裁專制、日益驕淫並且貪好女色。後宮之亂引來貴族衛慕氏的叛變(1034年)。又中种世衡的反間計而錯殺野利旺榮與野利遇乞,並且迷戀迎娶野利遇乞的妻子沒藏氏,生李諒祚。太子李寧林格與夏景宗因廢母(野利皇后)奪妻(沒移皇后)之仇,被沒藏氏的弟弟沒藏訛龐教唆刺殺夏景宗。夏景宗死後沒藏訛龐殺太子,立兩歲的李諒祚繼位,即夏毅宗。

夏毅宗與夏惠宗時期時,夏廷對內進一步鞏固統治,對外常與宋遼兩國處於戰爭與議和的狀態。夏毅宗繼位時年幼,由其母没藏太后與沒藏訛龐专政。當時辽朝遼興宗再度攻打西夏,最後西夏向辽朝称臣。没藏太后荒淫好色,多次勾結外人,其中李守貴與吃多己多次爭寵。最後李守貴殺太后與吃多己,事後也被沒藏訛龐所殺。沒藏訛龐又將其女許配夏毅宗以控制小皇帝。1059年夏毅宗參與政事,沒藏訛龐密謀刺殺夏毅宗,後被夏帝誅殺全家。親政後,夏毅宗娶協助他剷除沒藏訛龐的梁氏,任用梁乙埋與景詢等人。對內整治軍隊使地方軍政分立,文武官員互相牽制,提倡漢文化與技術,廢行蕃禮,改用漢儀,並於1063年改姓為李。對外方面,與宋重新劃定邊界,恢復榷場,貿易正常化。對吐蕃多次戰事,占領河湟與青海一带,於1063年招撫西域城(今甘肅定西縣)吐蕃首領禹藏花麻。夏毅宗的改革對以後各朝產生了深遠影響,然而於1066年與北宋作戰時受箭傷,兩年後去世,由其子7歲的李秉常即位,即夏惠宗。


1081年宋神宗命李憲率領五路宋軍伐夏,西夏梁太后採取誘敵深入、斷其糧道,最後只讓北宋奪得蘭州。

1082年年宋軍採取碉堡戰術,派徐禧興建永樂城,以步步逼近興慶府。西夏梁太后緊急率領30萬大軍突襲攻陷此城,北宋至此暫停伐夏,史稱永樂城之戰

由於夏惠宗年幼,由其母梁太后掌握大權,形成了以梁太后與梁乙埋為首的母黨專權。母黨大力發展其勢力,提倡番禮,重用都羅尾與罔萌訛,排擠夏景宗的弟弟嵬名浪遇等反對派。1080年,夏惠宗最後在皇族嵬名氏的協助下得以親政。夏惠宗崇尚漢法,下令以漢禮藩儀,遭到梁太后為主的保守派極力反對。對此,夏惠宗想用大臣李清策的建議,將河南地區歸還宋朝,以利用宋朝削弱外戚勢力。不料機密洩漏,梁太后殺李清策,幽禁夏惠宗。梁太后此舉引來皇黨、仁多族的叛亂,連吐蕃禹藏花麻都向宋朝請求派兵攻打梁太后。此時宋朝正值宋神宗王安石變法而國力增強,並在1071年由王韶於熙河之戰占領熙河路,對西夏右廂地區造成威脅。1081年宋神宗聽從种諤建議,趁西夏內亂之際,以李憲為總指揮發動五路伐夏,目標興慶府。梁太后採取堅壁清野策略,襲擊糧道以粉碎五路宋軍,宋軍最後只奪下蘭州。隔年年宋軍採取碉堡戰術,派徐禧興建永樂城,步步壓縮西夏在橫山的軍事空間。梁太后趁永樂城新建之初,率30萬大軍包圍攻陷,宋軍慘敗,史稱永樂城之戰。西夏雖然多次擊潰宋軍,但與宋朝貿易中斷使經濟衰退,戰事頻繁又大耗國力,人民不滿。梁太后與梁乙埋最後於1083年讓夏惠宗復位,以平和矛盾,然而夏惠宗依舊沒有掌握實權。梁乙埋去世後,政權轉由其子梁乙逋掌握。1086年夏惠宗在憂憤之下去世,由3岁儿子李乾顺即位,即夏崇宗。

此時西夏政權又落入小梁太后及梁乙逋手中。宋朝宋哲宗時期,知渭州章楶建議對西夏採取經濟制裁與碉堡作戰,其後為了實踐這套戰術,他在公元1096年於西邊的渭川修建平夏城與灵平砦,並且多次擊退夏軍。隔年宋軍攻入東邊的洪州、鹽州。1098年小梁太后偕同夏崇宗猛攻平夏城而敗,大將嵬名阿埋與妹勒都逋均被擒,史稱平夏城之戰。宋軍隨後興建西安州與天都寨,打通涇原路與熙河路,秦州變成內地。宋朝控制橫山地區後,西夏處境日益艱困。1099年在遼朝遼道宗的周旋下,宋夏再度和談,西夏向宋朝請罪而終。西夏在母黨專權的十年裡,梁乞逋依仗「梁氏一門二后」的威勢,連連發動与北宋和辽朝的战争,使西夏蒙受嚴重損失。他經常在朝廷上向眾大臣誇耀自己的功勞,認為西夏連年出征,主要就是讓宋朝屈服,只有這樣才能使西夏獲得和平。環慶之戰時,梁乙逋被小梁太后制止出征而懷恨在心。他意圖叛變,但是事機敗露。小梁太后命嵬名阿吳、仁多宗保與撒辰率兵逮捕處死。小梁太后親自專權後,為了加強對宋朝戰事,多次向遼朝請求援軍被拒。遼廷對小梁太后極度反感,認為遼夏戰爭是由她引起,而小梁太后因多次被拒也惡言相向。1099年夏崇宗親政在即,但「梁氏專恣,不許主國事」。遼朝遼道宗派使至西夏,用毒藥毒死小梁太后。至此長期的太后專政終止,西夏皇帝得以親政。

1099年夏崇宗亲政后采取依附辽朝,與北宋修和的策略,逐年減少战争。對內推廣漢文化,注重經濟,使得社会经济得到恢复和发展。宋朝宋徽宗時期,宋廷執行「紹盛開邊」政策。1114年童貫經略西夏,率領六路宋軍(包含永興、秦鳳兩路)伐夏。雖然西夏多次擊敗劉法、劉仲武與种師道等宋將,但宋軍也攻陷不少堡壘。最後西夏緊急向遼朝請求周旋,到1119年宋軍才率軍而退,夏崇宗再度向宋朝表示臣服。此時西夏國勢不如以往,而北宋也瀕臨崩潰邊緣。

1115年金朝興起,三國鼎立的局面被破壞,遼朝、北宋先後被滅,西夏經濟被金朝掌控。1123年退往漠北的辽朝遼天祚帝有意逃往西夏,金朝將領完顏宗望勸誘夏崇宗捕捉辽帝,許以下寨以北、陰山以南的遼地,並以進攻西夏為脅。夏崇宗答应条件,轉而連金滅遼,从此西夏归服金朝。1125年遼朝亡後,金朝約西夏夾攻北宋,並且給予宋地為誘餌。西夏占領天德軍、雲內等地後,1126年又被金朝強占,並且被強索河東八館之地。金朝為了補償西夏,同意占領陝西後將橫山地區歸還,但又違約。這些都讓金夏關係處於不信任的狀態。然而西夏與南宋隔絕,又讓西夏只能依賴金朝的經濟。這些都使得西夏對金朝維持鬆弛的和平,最多只有小規模的戰事。1141年金朝同意設置榷場,一度開放鐵禁。但是金世宗時不願以紡織品換取西夏的奢侈品,採取貿易緊縮的方式,到十年後才恢復正常貿易。夏崇宗於1139年去世後由其子李仁孝繼位,即夏仁宗。

夏仁宗時期是西夏的文化思想的發展到達高峰,對金朝大致上處於和平狀態。但是他重文輕武、務虛廢實的方式,使西夏軍力逐步走向衰落。宋朝降將任得敬才智均佳,陸續平定1140年夏將蕭合達叛變與隔年的山讹首領慕洧、慕濬投奔金朝之亂,備受夏仁宗重用。1143年發生大饑荒和地震,民不聊生,哆訛等人於威州、靜州與定州發動民變,夏仁宗又派任得敬平定。任得敬因被重用而野心膨脹,受晉王李察哥推薦入京。1156年李察哥去世後掌握政權,擴大私有勢力。1160年被封為楚王,出入等同皇帝。任得敬有意篡位,他以靈州為都城,1170年又迫夏仁宗給予靈州、西夏西南等領地。然而屢次不受金廷支持,任得敬與其弟任得聰等人陰謀叛亂。夏仁宗在金朝的支持下成功撲滅任黨,這個掌握政權二十年的權臣終於被拔除。夏仁宗於1143年的民變後,為了促進經濟穩定而推行改革。他改良地租和賦稅制度;發展教育,實行科舉;推崇儒術,以科舉取仕,這些措施對抑制世家大族有一定的作用;改革禮樂和法律。到天盛年間出現了盛世。1193年夏仁宗去世,子李純佑立,即夏桓宗。

夏桓宗基本上依循夏仁宗的國策,對內安國養民,推行漢法儒學,對外與金朝和好。但是,此時西夏過於安逸,軍力大大衰減。不久,北方的蒙古帝國興起,打破金宋與西夏三國鼎立的地位。夏仁宗的弟弟越王李仁友在挫敗任得敬之亂有功,去世後其子李安全上表請求表彰其父功勳與承襲王位。然而夏桓宗不但不同意,還降其為鎮夷郡王。李安全不滿,遂萌篡奪帝位之心。1206年與夏桓宗之母羅太后聯合廢夏桓宗,自立為帝,即夏襄宗。不久,夏桓宗去世。

西夏與漠北鄰國克烈部的關係十分友好,但是蒙古部的鐵木真崛起後開始威脅克烈部,成為西夏晚期的外患。1203年克烈部被鐵木真攻滅,其領袖王汗之子桑昆逃往西夏。兩年後,鐵木真率軍攻打西夏,掠奪西夏邊界城市而去。夏桓宗為擊退外患,改興慶府名為中興府,取夏國中興之意,實際上西夏反而壟罩在蒙古的威脅之下。1206年鐵木真建立蒙古帝國,即成吉思汗,後被尊稱元太祖。成吉思汗為了要攻滅敵國金朝,勢必要切斷金夏聯盟,所以西夏成為他的目標之一。隔年夏襄宗奪位不久,成吉思汗率大軍攻破西夏要塞斡羅孩城(今内蒙古乌拉特中后旗西境),因各路夏軍奮力抵抗而擊退之。1209年蒙古降服高昌回鶻,河西地區也暴露在蒙古威脅之下。蒙古第三次征夏即自河西入侵,出黑水城,圍攻斡羅孩關口。夏襄宗派其子李承禎率軍抵抗失敗,夏將高逸被俘而死。蒙軍又攻陷西壁讹答守備的斡羅孩城,直逼中興府的最後防線克夷門。夏將嵬名令公率軍伏擊蒙軍,最後仍被蒙軍擊潰。中興府被蒙軍圍困,夏襄宗派使向金朝金帝完顏永濟求救,但是金帝拒絕,還以鄰國遭攻打為樂而坐視不救。最後夏襄宗納女請和,貢獻大量物資,並且附蒙伐金。

夏襄宗附蒙伐金後,對金朝進行長達十餘年的戰爭,使雙方損失很大。國內方面,西夏百姓十分贫困,經濟生產受到破壞,军队衰弱,政治腐败。夏襄宗本身也沉湎于酒色之中,整日不理朝政。1211年齊王李遵頊發動宮庭政變,廢夏襄宗自立為帝,即夏神宗,史書稱為状元皇帝。夏神宗不顧國內大臣反對,仍然堅持附蒙抗金,金宣宗也多次反擊之。此時西夏國內社會經濟凋蔽,民變不斷。1216年因為西夏不肯幫助成吉思汗西征,次年成吉思汗率軍第四次进攻西夏。夏神宗以太子李德任守中興府,自己逃至西京靈州。最後李德任派使向蒙古和談才終戰。1223年由于夏神宗不愿做亡国之君,便让位给次子李德旺(原太子李德任被廢),即夏獻宗。此時夏廷已經認清蒙古將會滅亡西夏,夏獻宗決定採取聯金抗蒙的策略,趁成吉思汗西征時派使聯合漠北諸部落抗蒙,以便鞏固西夏北疆。當時總管漢地的蒙將孛魯(木華黎之子)察覺西夏的意圖,於1224年率軍從東面攻入西夏,攻陷銀州,夏將塔海被俘。隔年成吉思汗得勝返國,同時率軍攻打沙洲,但夏將籍辣思義極力防守。最後夏獻宗同意蒙軍條件請和,蒙軍撤退。

1226年成吉思汗以夏獻宗沒有履約及接納蒙古仇人為由,越大漠向西夏出征,此即蒙古滅西夏之戰。成吉思汗與速不台率大軍逼降黑水城(今內蒙額濟納旗),而後成吉思汗屯軍渾垂山(今甘肅酒泉北)避暑,並令速不台率別部迂迴攻入撒里畏吾兒(即黃頭回紇)與西蕃邊部(吐蕃諸部)等部。成吉思汗派忽都鐵穆兒、昔里铃部、察罕等將先後攻下肅州、甘州與沙州(1227年淪陷),大軍前進至涼州並降伏守將斡扎篑,至此河西走廊全數淪陷。夏獻宗憂患而死,由其侄南平王李睍繼位,即夏末帝。同年八月,成吉思汗率軍穿越沙漠,攻占應理(今寧夏中衛)進軍黃河九渡渡過黃河,主力逼近廢太子李德任守衛的西平府靈州。夏末帝派嵬名令公率軍與李德任會合,蒙夏雙方於凍結的黃河決戰,此役西夏軍死傷慘重,最後城陷被殺,但蒙軍也受損不少,成吉思汗駐守鹽州川(今陝西定邊花馬池)休整軍隊。1227年,成吉思汗此時將目標訂為攻佔西夏退路,並迂迴侵入金朝關中。他命蒙軍(應為蒙將阿術魯率領)包圍中興府(今寧夏銀川),並且率軍西南渡河攻下西夏積石州(今青海循化),與早已入侵西夏西南的速不台會合,陸續拿下西夏西寧(今青海西寧)、臨洮府(今甘肅臨洮)、金朝德順(今甘肅隆德)等西夏與金朝領地。同年6月,成吉思汗駐夏六盘山,又南下攻取金朝秦州(今甘肅天水),逼近關中京兆府(今陝西西安)。1227年夏末帝在中興府被圍半年後投降蒙古,西夏灭亡。成吉思汗此时已病死,但密不发丧,以免西夏反悔。而後诸将遵照成吉思汗遗命将夏末帝杀死,並且殺盡西夏宗室。而中興府百姓因蒙将察罕的勸諫而沒有被屠城。

1038年西夏立國時,疆域範圍在今宁夏,甘肃西北部、青海东北部、内蒙古西部以及陕西北部地区。东尽黄河,西至玉门,南接萧关(今宁夏同心南),北控大漠,佔地兩萬餘里。西夏東北與遼朝西京道相鄰,東面與東南面與宋朝為鄰。金朝滅遼宋後,西夏的東北、東與南都與金朝相鄰。西夏南部和西部是吐蕃諸部、黃頭回鶻與西州回鶻相鄰。國內三分之二以上是沙漠地形,水源以黃河與山上雪水形成的地下水為主。首都興慶府所在的銀川平原,西有賀蘭山作屏障,東有黃河灌溉,有「天下黃河富寧夏之稱」。

西夏是党项族建立的王朝,党项族原本定居四川松潘高原一帶。唐高宗時期受吐蕃壓迫,最後在唐廷協助下遷移到河套陝北一帶,分為平夏部與東山部,至此建立西夏的龍興之地。881年因平夏部拓跋思恭平黃巢之亂有功,被封為夏州節度使,至此正式領有銀州(今陝西米脂縣)、夏州(今陝西橫山縣)、綏州(今陝西綏德縣)、宥州(今陝西靜邊縣)與靜州(今陝西米脂縣西)等五州之地。宋朝時,宋太宗併吞夏州節度使之地。然而李繼遷不願意投降,率部四處攻擊,最後收復五州之地。攻下靈州後,將勢力擴展到黃河河套地區與河西走廊。夏景宗繼位後持續鞏固河西走廊,並且開國稱帝,疆域擴大到二十個州。而後夏景宗與宋朝征戰於橫山地區,並有意占領關中。夏景宗之後,西夏與北宋展開拉鋸戰,雙方互相占領對方的堡壘城寨,並且擴大到河煌青海地區。夏崇宗後期喪失橫山地區,一度造成危機。金朝滅遼朝與北宋後,西夏陸續收復失地,並且占領黃河前套地區。然而其勢力被金朝所侷限,領土擴張不大。到夏仁宗時期大約有22個州,這是西夏版圖最後穩固的狀態。

西夏行政區劃大體上是州(府)縣兩級,一些重點州則設府。另分左右廂十二監軍司,作為軍管區。西夏在建國前只領有五州之地。占領河套地區與河西走廊後,在夏崇宗、夏仁宗時期達到22州:河南9州、河西9州,熙秦河外4州。文獻記載比較明確的有32州。州所轄縣不多,有的就是堡壘和城鎮,其規模比不上宋朝的州。其目的只是壯大聲勢,安置親信以嚴密控制而已。升州為府的有河套的興州(興慶府、中興府)、靈州(西平府)與河西走廊的涼州(西涼府)、甘州(宣化府)等。夏州、靈州與興州相繼是西夏立國前的都城,地位十分重要。涼州控管河西走廊與河套地區的樞紐,地理位置重要。甘州所設的宣化府,負責處理吐蕃、回鶻的事務。左右廂與十二監軍司主要是夏景宗為了方便對軍隊的管理與調遣、佈防而設置的。每一監軍司都仿宋制立有軍名,規定駐地。

西夏政治是蕃漢聯合政治,党項族為主要統治民族,並且聯合漢族、吐蕃族、回鶻族共同統治。皇族注意與党項貴族的關係,以通婚與權力分享攏絡,而母黨「貴寵用事」。這些都使皇族與母黨、党項貴族之間時常發生衝突。西夏在前期即有像遼朝那樣的蕃漢官制,但是到中後期全面採用宋朝制度後,蕃官逐漸式微。

西夏的国家体制和统治方式深受儒家政治文化影响。官制自1038年夏景宗立國後確立,大體上學自宋朝制度。官分文武兩班,中書司、樞密司與三司(鹽鐵部、度支部與戶部)分別管理行政、軍事與財政。御史台管監察、開封府管理首都地區的事務,其他還有翊衛司、官計司、受納司、農田司、群牧司、飛龍苑、磨勘司、文思院、番學與漢學等機構。隔年,夏景宗仿照宋朝制度設立總理庶務的尚書令,改宋朝二十四司為十六司,分理功、倉、戶、兵、法、士六曹,使西夏官制和機構已頗具規模。到夏毅宗時又增設各部尚書、侍郎、南北宣徽使及中書、學士等官。一來職官和機構愈分愈細,二來官制改革由擴充政治軍事的官職轉向擴充社會經濟文化方面的官職。

蕃官是專由党項族擔任的官職,有一說此為爵位制度。蕃官主要是為了保持党項貴族在政權中的主導地位,非党項族不能擔任,有寧令(大王)、謨寧令(天大王)、丁盧、丁弩、素齋、祖儒、呂則、樞銘等等官稱。夏景宗增設番官後,還學習遼朝與吐蕃的一些制度,如南北面官制。西夏的蕃官制度很雜亂,夏毅宗時又增設不少官職,至今仍不太清楚其官職功能,有一說蕃官只是西夏文表示的漢官官名而已。西夏文諺語也提到「衙門官員曾幾何,要數弭藥為最多」,表明党項族當官為數不少。隨着西夏皇帝越來越崇尚漢法,改蕃禮、用漢儀,蕃官系統逐漸式微。夏崇宗以後,蕃官就在也沒出現在相關文獻中。

關於法律方面,因為西夏舊律有不明疑礙處,夏仁宗在“尚文重法”的主張下頒布《天盛改舊新定律令》,又稱《天盛律令》、《開盛律令》。主要由北王兼中書令嵬名地暴與中書、樞密院宰輔要員及中興府、殿前司、閤門司等重要官員參與編寫。該法典參考了唐朝、宋朝的法典,並且結合本國的國情、民情和軍情,使得更加切合實際。在某些方面(如畜牧業、軍制、民俗,等等)更具有本民族的特點。

西夏地理處於四戰之地,陸續要應付後唐、回鶻、吐蕃、宋朝、遼朝、金朝、西遼與蒙古的威脅與戰爭,所以外交是夏廷十分重視的環節。外交策略主要是聯合或依附強者,並且攻擊弱者、以戰求和。這些策略使自己得以不斷延續、發展。然而依附國過於強大,最後難逃滅亡之命運。

西夏早在夏州政權時期(定難軍)就奉唐朝、五代諸國與北宋為宗主國,以維持自身勢力。後來北宋併吞夏州政權,李繼遷舉兵再起。此時他採取事奉遼朝、連遼抗宋的策略,多次擊退宋軍,並且擴張勢力。並且於990年被遼朝遼聖宗冊封為夏國王。到李德明時,為了鞏固新領地,對北宋和談,於1006年簽署景德和議。然而李德明依舊維持與遼朝的關係。除了應付遼、宋的戰事外,為了稱霸河西、先後攻滅甘州回鶻、沙洲歸義軍,對抗吐蕃六谷部、唃廝囉國等,與西州回鶻為鄰。

夏景宗時正式稱帝建國,自稱邦泥定國,稱男不稱臣,並且多次入侵宋朝邊疆。宋仁宗不滿西夏獨立,派兵攻打之,至此宋夏戰爭爆發。夏景宗在三大戰役(三川口之戰、好水川之戰與定川寨之戰)戰勝宋朝後,雙方於1044年簽訂慶曆和約。宋朝給予「夏國主」名號,西夏皇帝對宋朝稱臣,但實際上西夏皇帝在國內仍以君王自稱。宋朝給與金錢、茶葉等大量物資。西夏雖然擊敗北宋,但惹來遼朝不滿,雙方發生三次戰爭(賀蘭山之戰),最後以西夏稱臣作收。而後北宋的宋神宗為了擊敗西夏,趁西夏內亂之際發動五路伐夏與永樂城之戰,最後都以西夏戰勝作收。然而西夏國力漸衰,橫山地區又被北宋占領,此後有賴遼朝周旋方能穩定宋、遼、西夏三國鼎立的關係。

金朝崛起後滅遼朝與北宋,西夏為了自保,放棄遼夏同盟,臣服於金朝。金朝包圍西夏的東方與南方,掌握西夏的經濟力,所以夏廷對金朝不敢輕舉妄動,最多只有小規模的戰事。蒙古帝國崛起後,多次入侵西夏,破壞金夏同盟。夏襄宗與夏神宗改採取聯蒙攻金的策略,多次與金朝發生戰爭,然而此為錯誤的方針。到夏獻宗時才改連金抗蒙,但不久就在蒙夏戰爭中於1227年亡國。金史稱西夏「立國二百餘年,抗衡遼、金、宋三國,偭鄉無常,視三國之勢強弱以為異同焉。」。

西夏對於回鶻、吐蕃等少數民族採取懷柔與招撫的方式,似乎比宋朝還要好。例如西使城(今甘肃定西西南)吐蕃首领禹藏花麻不愿降宋朝,又受到宋军王韶的攻掠。夏毅宗立即派兵支援,將宗女嫁給他。禹藏花麻遂把西使城及兰州献给西夏。

軍事制度是以党項部落兵制為基礎,加入宋朝制度而改良。西夏是以戰立國的國家,軍隊是賴以維生的基礎。所以實行全民皆兵的制度,平時生產,戰時作戰,軍事與社會經濟合為一。除了給予軍官和正軍很少的軍事裝備之外,其餘作戰一律自帶糧食。最小單位是「抄」,每抄由三人組成,主力一人,輔主一人,負擔一人。樞密院是西夏最高的軍事統御機構,下設諸司。指揮系統分別是統軍、行主、佐將、首領、佐首領、小首領。全國軍隊分成中央軍與地方軍等兩個系統。軍事佈防以賀蘭山、興慶府與靈州為三角防線,成為西夏作戰的核心。當大敵逼近時,邊防軍迅速回京助守;邊將吃緊時,主力軍立即機動支援,軍隊調動十分靈活。作戰時有利則進,不利則退。由於地形以沙漠、山岳為主,所以夏軍擅長誘敵設伏、斷敵糧道、集中兵力作運動戰,所以時常能以少擊多。

中央軍分為擒生軍、衛戍軍、侍衛軍、潑喜、铁鹞子、撞令郎與步跋子等。擒生軍人數約十萬,主要任務是在作战中掳掠生口作奴隶,相當遼軍的「打草穀騎」,不擔負決戰任務。衛戍軍人數約兩萬五千,部屬在興慶府周圍地區,裝備精良,所配置的副兵多達七萬,是夏軍的主力部隊。侍衛軍又號「御園內六班直」,人數約五千,是由豪族子弟中選拔善於騎射者組成的一支衛戍部隊,負責保衛皇帝安全,分三番宿衛。潑喜約兩百人,是西夏的砲兵,掌握火蔟黎的技術,又能拋射石彈。由於由騎兵施放,十分靈活。鐵鷂子約三百人,後擴充至萬人,是西夏的重甲騎兵,具有機動靈活的特點,在平地作戰具有威力,時常隨皇帝出入作戰。撞令郎,由俘獲的健壯漢族士兵擔任,作為戰事的砲灰,減少西夏軍人的損失。步跋子,西夏的步兵,擅長山區作戰,由橫山等山間部落的丁壯組成。時常與鐵鷂子聯合突擊敵軍。

地方軍部分,夏景宗將全國軍區分為左廂、右廂與十二監軍司,共有左廂神勇軍司駐銀川彌陀洞(今陝西榆林市東)、祥祐軍司駐石州、嘉寧軍司駐宥州、靜塞軍司駐韋州、西壽保泰軍司駐柔狼山北(今甘肅平川)、卓囉和南軍司駐蘭州黃河北岸喀羅川(今甘肅永登)、右廂朝順軍司駐賀蘭山克夷門(今寧夏石嘴山區)、甘州甘肅軍司駐甘州刪丹縣故地、瓜州西平軍司駐瓜州、黑水鎮燕軍司駐居延海黑水城(內蒙古額濟納旗)、白馬強鎮軍司駐婁博貝(內蒙古阿拉善左旗吉蘭泰鎮)、黑山威福軍司駐河套(內蒙古五原縣)。全盛時期「諸軍兵總計五十餘萬」,軍兵種主要是騎兵和步兵兩種。每一監軍司都仿宋制立有軍名,設有都統軍、副統軍和監軍司各一員,由皇帝任命貴族擔任。下設指揮使、教練使及左右侍禁官等數十員,由党項人和漢人分任。 

由於沒有專為記錄西夏人口的史書,使得西夏人口的統計十分模糊,現今史學界也沒有一個統一的數據。然而西夏採取全民皆兵的制度,可以由兵力數量去類推人口量。目前認為人口數的下限不低於三十萬戶,上限不超過兩百萬。據《宋史》記載,西夏具有五十萬大軍,中間相差十幾萬。今日一種論點為西夏軍具有三十七萬,並在《東都事略·西夏傳》找到「曩宵有兵十五萬八千五百人」的記載,認為此差距是指党項族的西夏軍。另一種論點是將西夏地方軍五十餘萬人加上中央軍約十九萬人,總共約七十萬左右。由於西夏採取全民皆兵的制度,人口數推算是二百萬至三百萬左右。然而還有一派論點認為此數據過於誇張,他們根據《續資治通鑒長編》與《隆平集》的記載,西夏兵力約在十五萬至十八萬,總戶數約三十萬左右。

關於西夏人口的變化,各家說法不同。根據《中国人口史》赵文林與谢淑君的版本推算,西夏人口峰值在1038年,243万人,后因西夏和辽朝、北宋陸續發生战争而不断减少,战争停止后又缓慢回升。1069年,西夏建国后的人口峰值,230万人,后又因为战争不断减少。最後在1131年至1210年年間,人口一直维持在120万左右。而《中国人口发展史》葛剑雄的版本推算,西夏人口的峰值在夏景宗超过300万。1127年后,西夏人口一直未超过300万。根據《中国人口史》(第三卷)辽宋金元时期,吴松弟的版本推算,西夏人口的峰值在1100年夏崇宗時期,大约300万人。西夏人口密度低於北宋各路與唐朝各道的人口密度,然而比唐朝的隴右道高。這是因為西夏領土大多是由沙漠組成,適居範圍不大,在加上西夏採取全民皆兵的制度,因連年戰事不斷,人口消耗不少。

西夏是一個多民族的朝代,其主要民族有党項族、漢族、回鶻族與吐蕃族等。西夏人大都身材修長高大,充分表現出党項羌人粗獷、剽悍、豪爽的民族性格。社會等級分明,以日常服飾禮儀區分。皇帝、文官與武官的服裝均有規定標準,平民百姓只准穿青綠色衣服,貴賤等級分明。冠飾也是區分等級的依據之一,皇帝氈冠、皇后龍鳳冠、命婦花杈冠,文官幞頭、武官還有各種頭冠樣式。

西夏的經濟是以畜牧業為基礎,主要以牛、羊、馬和駱駝為大宗。農產品主要有大麥、稻、蓽豆和青稞等物。藥材和一部分手工製品也特別有名。西夏在冶煉、採鹽製鹽、磚瓦、陶瓷、紡織、造紙、印刷、釀造、金銀木器製作等手工業生產也都具有一定的規模和水平。慶曆和議後,宋廷設置榷場,恢復雙方貿易往來,西夏的手工業生產和商業貿易迅速發展。夏崇宗與夏仁宗時期,西夏經濟大大的發展,四方的物品會集到首都興慶,進入西夏經濟最鼎盛的時期。

党項族是游牧民族,其農業較畜牧業晚發展,農牧並重是西夏社會經濟的特色。李繼遷時期陸續領有河套與河西走廊地區如靈州(今寧夏吳忠市)、興慶(今寧夏銀川)、涼州(今甘肅武威)和瓜州(今甘肅安西)等地後,由於這些地區豐饒五穀,「地饒五穀,尤宜稻麥」。其中興靈地區與橫山地區為西夏糧食的主要產地,其產量还可以用来救济灾民,而橫山地區的糧食時常提供給伐宋夏軍使用。西夏主要的農產品有大麥、稻、蓽豆和青稞等物,當發生戰亂或天災時只能以大麦、荜豆、青麻子等物充饥,並且等待自靈夏所運來的糧食。藥材中比較有名的有大黃、枸杞與甘草,皆是商人極力採購的重點商品之一。其他還有麝臍、羱羚角、柴胡、蓯蓉、紅花和蜜蠟等。党項族向漢族學習比較先進的耕種技術,已普遍使用鐵製農具和牛耕。西夏領地以沙漠居多,水源得來不易,所以十分重視水利設施。西夏古渠主要分布在兴州和灵州,其中以兴州的汉源渠和唐徕渠最有名。夏景宗時興修從今青銅峽至平羅的灌渠,世稱「昊王渠」或「李王渠」。在甘州、凉州一带,则利用祁连山雪水,疏浚河渠,引水灌田。在這些水源中,又以甘州境内的黑水最为著名。横山地區則以无定河、白马川等等為水源。夏仁宗時期修訂的法典《天盛改舊新定律令》中,鼓励人民开垦荒地,並規定水利灌溉事宜。

西夏的畜牧業十分發達,夏廷還設立群牧司以專屬管理。牧區分布在橫山以北和河西走廊地區,重要的牧區有夏州(今陝西靖邊北白城子)、綏州(今綏德)、銀州(今米脂西北)、鹽州(今寧夏鹽池北)與宥州(今陝西定邊東)諸州,還有鄂爾多斯高原、阿拉善和額濟納草原及河西走廊草原,都是興盛的牧區。畜類主要以牛、羊、馬和駱駝為大宗,其他還有驢、騾、豬等。馬匹可做軍事與生產用途,並且是對外的重點商品與貢品,以「党項馬」最有名。駱駝主要產於阿拉善和額濟納地區,是高原和沙漠地區的重要運輸工具。在西夏辭書《文海》中對牲畜的研究十分細緻,有關餵養、疾病、生產與品種的區分都有詳細的說明,表現出西夏人對畜牧的經驗十分豐富。除畜牧业外,狩猎业也十分興盛,主要有兔鹘、沙狐皮、犬、马等。其規模不小,例如對遼朝的貢品中,即有沙狐皮一千张。狩猎业在西夏中後期時仍然興盛,受西夏大臣所重視,西夏軍隊也時常以狩獵為軍事訓練或演習。

西夏手工业分官营民营两种,主要以官營為主。其生產目的主要是供西夏貴族使用,其次則是生產外銷。手工業門比較齊全,夏仁宗修訂的法典《天盛改舊新定律令·司序行文門》中即分類詳細。手工業以紡織、冶煉、金銀、木器製作、採鹽、釀造、陶瓷、建築、磚瓦等為主,兵器製造也較為發達。

西夏的青鹽是宋夏邊界人民最喜歡的商品,也是西夏重要的財源之一。主要產地有鹽州(今寧夏鹽池北)的烏池、白池、瓦池與細項池,河西走廊和西安州(今宁夏海原西)的鹽州與鹽山,靈州(今寧夏吳忠市)的溫泉池等等老井。所出產的青鹽味甘價賤,比宋朝的河東解鹽更具歡迎,另外西安州的碱隈川還產白鹽、紅鹽,只是質量不如青鹽。西夏青白盐除了供西夏人民食用外,主要用于同宋朝、辽朝、金朝进行官方贸易,其中运往宋关中地区最多,并以此换回大批粮食。宋廷為此禁止西夏進口青鹽,宋人只能透過走私進口,谋取暴利。

西夏的氈毯是外銷的名貴商品,其中以白駱駝毛製成的白氈於《马可波罗遊記》記載有「為世界最良之氈」的美稱。西夏矿产比较丰富,所以其兵器製造業,如神臂弓、旋風炮以及勁弩不能射入的冷鍛鎧甲均為世人稱道,特別值得一提的是「夏國劍」,鋒利無比,貴重一時,當時就為宋人所珍視。

西夏印刷業頗為發達,西夏人為了吸收漢族文化,並且維護自己的文化,用夏漢兩種文字雕印書籍。為了發展印刷業,夏廷還設置刻字司以專司出版,另外私人和學校也可能刻印書籍。刻書種類繁多,有佛經、漢學經典、文學詩書、音韻、卜筮咒文、醫學技術等等書籍,其中以佛經數量最多。如1189年夏仁宗就在大度民寺作大法會,散發蕃漢《觀彌勒上升兜率天經》十萬卷,漢《金剛普賢行誦經》、《觀音經》等五萬卷。

西夏本來沒有瓷器,主要靠掠奪宋人來獲得。慶曆和議後,西夏自漢族學得制瓷技術。夏毅宗時期開始發展制瓷業,主要以興慶為生產中心。从考古出土的陶瓷看,西夏烧制的瓷器大多以白瓷碗、白瓷盘等等為主。其瓷器技术上比不上宋瓷,但樸實凝重,形成獨具一格的西夏瓷器。

由於西夏領有絲路商業要道河西走廊,再加上國內只盛產畜牧,對於糧食、茶葉與部分手工品的需求量大,所以對外貿易是西夏經濟的命脈之一,主要分為朝貢貿易、榷場貿易與竊市(私市)。西夏國內的城市商業十分繁榮,興慶、涼州、甘州、黑水城都十分興盛。商品以糧食、布、絹帛、牲畜、肉類為大宗。西夏可以藉由掌控河西走廊以管理西域與中原的貿易往來,與北宋、遼朝、金朝、西州回鶻及吐蕃諸部有頻繁的商業貿易。由於西夏過度壟斷河西走廊,使得部分西域商人改走柴達木盆地,經鄯州(今青海西寧)沿湟水而到達宋朝的秦州(今陝西天水),史稱吐谷渾路。

西夏對中原或北亞的宗主國採取朝貢貿易,時常以駱駝或牛羊等價換取糧食、茶葉或重要物資。西夏在宋夏戰爭獲勝,於慶曆和議中,每年自宋朝獲得銀5萬兩,絹13萬匹,茶2萬斤,每年還在可以各種節日中獲得銀22000兩,絹23000匹,茶1萬斤。西夏自李繼遷叛宋附遼開始向遼朝貢,至遼天祚帝亡國,總計向遼朝貢24次。西夏也會以宋朝的茶葉與歲幣換取回鶻、吐蕃的羊隻,再轉賣給宋、遼、金等國,從中牟取暴利。由於朝貢貿易時常因為戰事而中斷,並不是很穩定。

比較大宗且穩定的貿易為榷場貿易,西夏與北宋、遼朝和金朝的邊境地帶設有共同使用的榷場進行和市,例如與宋朝制定的保安軍(今陝西志丹)、鎮戎軍(今寧夏固原)、麟州、延州等地的榷場;與遼朝在遼西京西北的天德府、雲內和雲中西北的銀瓮口、過腰帶與上石楞坡等地的榷場等。在榷市中,有固定的貿易場地和牙人評定貨色等級,由雙方官府派遣的監督、稽查人員共同管理市場,徵收稅務。貿易內容以牲畜、毛织品、药材為大宗。而“官市”以外的商品种类不受此限。金滅北宋後,由於南宋與西夏隔絕,西夏對外貿易掌握在金朝手中,經濟上不能不依賴於金朝。1141年金朝同意開放保安軍、蘭州、綏德、環州與東勝州的榷場。1172年金朝金世宗以保安軍、蘭州、綏德不產布為由關閉這些榷場,認為以紡織品換取西夏的奢侈品不划算。這使得雙方關係緊張,在夏仁宗末期戰事不斷,十年後才恢復正常貿易。最後比較大量且分散的是「竊市」(私市),也就是非正式市場與走私貿易,例如青鹽貿易即採取走私方式換取宋朝的糧食。

由於西夏商業的興盛,作為流通的貨幣也十分重要:一類是本國鑄造的西夏貨幣;另一類是從宋、金進口的貨幣。早在夏景宗時期即鑄造貨幣,各代皇帝除了夏獻宗、夏末帝之外都有鑄錢實例,夏仁宗還於1158年設立通濟監鑄錢。西夏錢幣的鑄造大都比較精美,而且書法俊逸、流暢。目前面文為西夏文的錢幣已經發現有五種,分別是「福聖寶錢」、「大安寶錢」、「貞觀寶錢」、「乾祐寶錢」以及「天慶寶錢」。

西夏文化深受漢族河隴文化及吐蕃、回鶻文化的影響。並且積極吸收漢族文化與典章制度。發展儒學,宏揚佛學,形成具儒家典章制度的佛教王國。西夏起初是游牧部落,佛教在1世纪东传凉州刺史部以后,於该区逐渐兴盛起来,在西夏建国后开始创造自己独有的佛教艺术文化。内蒙古鄂托克旗的百眼窑石窟寺,是西夏佛教壁画艺术的宝库。在额济纳旗黑水城中发现的西夏文佛经、释迦佛塔、彩塑观音像等,是荒漠的重大发现。另外西夏也大力發展敦煌莫高窟。1036年西夏攻滅归义军後,占領瓜州、沙州,領有莫高窟。从夏景宗到夏仁宗,西夏皇帝多次下令修改莫高窟,使其更加增添了几分光辉。当时莫高窟涂绿油漆,接受中原文化與畏兀儿、吐鲁番风格。此外,表现西夏文化的还有西夏文,又称蕃书。西夏设立蕃學和漢學,使西夏民族意识增强,百姓“通蕃汉字”,文化也增加了许多。值得一提,李元昊曾經頒布禿髮令,命令全國男人三天內必須禿髮,違者格殺勿論。西夏还设立蕃学和太学。史家戴锡章《西夏记》曾言:“夫西夏声明文物,诚不能与宋相匹,然观其制国书、厘官制、定新律、兴汉学、立养贤务、置博士弟子员。尊孔子为文宣帝,彬彬乎质有其文,固未尝不可与辽金比烈!”。

西夏儒學的發展是一種處在儒家影響下的官僚體制與政治文化,制度深受儒家文化影響,從李繼遷伊始至西夏末年,歷代帝王莫不學習與模仿漢制。例如李繼遷時「潛設中官,盡異羌夷之體,曲延儒士,漸行中國之風。」,李德明時 「大輦方輿,鹵薄儀衛,一如中國制。」。西夏党項世代皇親宗室,崇拜孔子,欽慕漢族文化。除了崇儒尚文,還編寫了一些融合和宣揚儒家學說的書籍,如《聖立義海》、《三才雜字》、《德行記》、《新集慈孝傳》、《新集錦合道理》、《德事要文》等。其儒學經過夏景宗、夏毅宗、夏惠宗與夏崇宗的提倡,到夏仁宗之時出現盛況。

夏景宗在建立官制的同時設立了蕃學和漢學,作為文化培養的搖籃。以博學多才的野利仁榮主持蕃學以重視蕃學,並於各州蕃學裡設置教授,進行教學。西夏大致設立了五種學校:蕃學、國學、小學、宮學、太學。西夏建立學校的目的主要是為了培養人才的需要,尊孔子為文宣帝。西夏在中後期還發展科舉制度,夏崇宗後期開始設童子科實行科舉考試,1147年夏仁宗策舉人,立唱名法,復設童子科。西夏後期基本以科舉取士選拔官吏,不論蕃漢及宗室貴族由科舉而進入仕途成為必然的途徑。

西夏崇尚漢文化,但漢文創作的文學作品傳世不多,大多以詩歌和諺語為主。詩歌有宮廷詩、宗教勸善詩、啟蒙詩、紀事詩與史詩等幾類。西夏詩歌有韻律,一般為對稱結構,通常是五言或七言體,也有多言體,每一詩句的音節數目不同。比較有名的有頌揚西夏文創製者野利仁榮的《大頌詩》。史詩性的作品《夏聖根讚歌》,內容多為民間傳說,遣詞造句帶有濃重的民謠色彩。其中開首三句:「黑頭石城漠水邊,赤面父冢白河上,高彌藥國在彼方」,被西夏學學者用來研究党項歷史源起。另外還有讚美重建太學的《新修太學歌》,具有宮廷詩的風格。夏崇宗重視文學,本人曾作《靈芝歌》與大臣王仁忠酬唱,傳為佳話。

西夏諺語對偶工整,結構嚴謹,字數多少不一,內容廣泛地反映了西夏社會的各種面向、並涉及百姓生產、風俗與宗教等內容。著名的西夏諺語集《新集錦合辭》,是由西夏人梁德養於1176年初編、1187年由王仁持補編,共有364条谚语。其內容有「谚语不熟不要说话」的记载,「千千诸人」、「万万民庶」都离不开谚语,凸顯出諺語對西夏人民的重要性。

西夏皇帝十分重視本國國史的編撰工作。斡道沖於李德明時期就掌管撰修西夏國史之職,其後代亦同。夏仁宗時設置翰林學士院,命王僉、焦景顏參照宋朝編修實錄的辦法纂修國史,負責修《李氏實錄》。1225年南院宣徽使羅世昌罷官後,撰寫《夏國世次》,可惜已失。

西夏立國前夕,夏景宗為了建議屬於本國的語言,派野利仁榮仿照漢字結構創建西夏文,於1036年頒行,又稱「國書」或「蕃書」,與周圍王朝往來表奏、文書,都使用西夏文。文字構成多採用類似漢字六書構造,但筆畫比漢字繁多。西夏文學家骨勒茂才認為西夏文和漢文的關係是「論末則殊,考本則同」。西夏文創製後,廣泛運用在歷史、法律、文學、醫學著作,鐫刻碑文,鑄造錢幣、符牌等也都使用西夏文。夏廷又設立蕃學,由野利仁榮主持,選派貴族官僚子弟翻譯漢文典籍與佛教經典等。為了翻譯漢夏文字,又有骨勒茂才於1190年所撰寫的《番汉合时掌中珠》,序言有西夏文和漢文兩種,內容相同。謂「不學番言,則豈和番人之眾;不會漢語,則豈入漢人之數。」表明本書目的在於便於西夏人和漢人互相學習對方語言,是現今研究西夏歷史的重要鑰匙。

西夏人民大致上以佛教為主要信仰,在建國之前則是以自然崇拜為主。 党項族在唐朝四川松潘地區時,就以「天」為崇拜對象。當党項族遷移到陝北之後,由自然崇拜發展到對鬼神的信仰。在建國之後,仍然崇尚多神信仰,有山神、水神、龙神、树神、土地诸神等自然神。例如夏景宗曾「自诣西凉府祠神」。夏仁宗曾在甘州黑水河边立黑水桥碑,祭告诸神,祈求保护桥梁,平息水患。除了崇拜鬼神,党項族還崇尚巫術,並且備受重視。党項族稱巫為「廝」,巫師被稱為「廝乩」,是溝通人和鬼神間的橋樑,主要負責驅鬼與占卜。在战争前實行占卜以问吉凶,於戰爭中经常施行「杀鬼招魂」的巫术。

佛教是西夏的國教,建國前後六次向宋求贖佛經,宋朝賜以《大藏經》。夏景宗在立國後,便開始用西夏文翻譯佛經。五十多年內譯出大小乘佛經820部,3579卷,滿足人民對佛教的需求。除此之外,夏景宗等歷代夏帝與太后也興建許多佛教寺廟高台寺,概括地分為興慶府—賀蘭山中心、甘州—涼州中心、敦煌—安西中心以及黑水城中心。例如有名的承天寺是應夏毅宗母后沒藏太后要求而興建,1093年更重修涼州感通塔及寺廟,隔年立「重修護國寺感通塔碑」。夏崇宗時期更在甘州建築臥佛寺。西夏朝廷大力提倡佛教,提高僧人地位,使僧人不用納稅與負擔雜役;犯罪也可減免罪刑;寺院環境也受朝廷保護。西夏後期受藏傳佛教影響的趨勢日益增大,1159年吐蕃迦瑪迦舉系教派初祖都松欽巴建立粗布寺,夏仁宗派使入藏迎奉。都松欽巴派大弟子格西藏瑣布帶經文到西夏興慶府,被夏仁宗尊為上師,並參與翻譯經文。西夏比元朝還要早設立帝師,提高藏傳佛教的地位。除帝師外,還有國師以及其它有高級職稱的僧人,在推動西夏佛教發展方面起著核心和中堅的作用。

除了佛教以外,西夏也包容其他宗教。西夏有流傳道教,例如夏景宗之子寧明就是學習道家的辟穀術而死。《文海》解釋「仙」字為「山中求道者」,「山中求長壽者」。在西夏晚期,在沙州和甘州一帶還有流傳景教和伊斯蘭教。例如《馬可波羅遊記》中記載敦煌(唐古忒省)與甘州有部分景教和伊斯蘭教徒。

西夏的藝術文化十分多元且豐富,在繪畫、書法、雕刻、舞蹈與音樂都有成就。繪畫方面,以佛教繪畫流傳至今,主要呈現在石窟與寺廟壁畫等,現今以敦煌莫高窟、安西榆林窟等最為豐富。早期學習北宋風格,後來受回鶻佛教與吐蕃藏傳佛教的壁畫藝術的影響,最後形成獨特的藝術風格。在線條採用鐵線與蘭葉描為主,輔以折蘆、蓴菜條;敷彩大量使用石綠打底,使畫面呈獨具風格的冷色調的「綠壁畫」。繪畫內容分別有佛教故事與說法、供養菩薩與人像以及洞窟裝飾圖案等,以《文殊變圖》、《普賢變圖》、《水月觀音圖》與《千手千眼观音经变圖》最為有名。此外,也可在《千手千眼觀世音像》內的《農耕圖》、《踏碓圖》、《釀酒圖》與《鍛鐵圖》中觀察到西夏社會生產和生活內容。木刻版畫方面,大多來自西夏文和漢文佛經中。在黑水城出土的大量佛畫中,有《文殊圖》、《普賢圖》、《勝三世明王曼荼羅圖》等等。內容濃抹重彩,色調深沉。而版畫《賣肉圖》和《魔鬼現世圖》,描繪生動,反映出西夏繪畫的深度。

書法在楷書多見於寫經與碑文,篆書見於碑額與官印。夏仁宗時期的翰林學士劉志直,工於書法,他用黃羊尾毫製作之筆,為時人所效法。雕塑方面十分發達,有鑄銅、石雕、磚雕、木雕、竹雕、泥塑與陶瓷等。其特點比例均衡,刀法細膩,十分寫實。泥塑以佛寺塑像為代表,多運用寫實與藝術誇張手法,刻劃現實生活的人物形象。例如夏崇宗時期修建的甘州大佛寺釋迦牟尼涅槃像、敦煌莫高窟第491窟西夏供養天女彩塑等等。其他陶瓷藝術品也是刻工精細而生動。

西夏在党項時期的樂器以琵琶、橫吹,擊缶為主,其中橫吹即竹笛。後來接受中原音樂的文化,李德明時採用宋制製樂而逐漸繁多。夏景宗建國後,革除唐宋縟節之音,「革樂之五音為一音」。1148年,夏仁宗令樂官李元儒更定音律,賜名《鼎新律》。西夏音樂十分豐富,且設有蕃漢樂人院,夏惠宗時曾招誘漢界娼婦、樂人加入樂院,戲曲如《劉知遠諸宮調》等也已經傳入西夏。西夏時期的舞蹈在碑刻和石窟壁畫中留有生動的形象,富含唐宋舞蹈與蒙古舞蹈的風格。如《涼州護國寺感應塔碑》碑額兩側的線刻舞伎,舞姿對稱,裸身赤足,執巾佩瓔,於豪放中又顯出嫵媚。榆林窟第3窟西夏壁畫中的《樂舞圖》,左右相對吸腿舞狀,姿態雄健。

西夏本身的科技比較薄弱,主要以吸收宋朝或金朝的技術為主,然而在武器鍛鍊方面有獨到之處。在天文氣象方面,主要是學習宋朝的天文與曆法。西夏人設置司天監以觀察天文,並列有分析、解釋天文的「太史」「司天」和「占者」以分析天文。在骨勒茂才的《番漢合時掌中珠·天相》中有對天文星象的詳細記載。例如將天空分為青龍(東)、白虎(西)、朱雀(南)、玄武(北)等方位,每個方位設有7個星宿。在氣象方面也有詳細的分析,例如風有和風、清風、金風、朔風、黑風、旋風;雨有膏雨、穀雨、時雨、絲雨;雲有煙雲、鶴雲、拳雲、羅雲、同雲,等等。曆法方面,西夏至1004年材自北宋獲得《儀天曆》,這是西夏第一本曆書。立國後,設“大恒曆院”的機構掌管曆法的編制和頒行。西夏曆書採用番漢合璧曆書與宋朝頒賜曆書兩類,其詳細情形仍需研究。

在醫學方面,在党項時期,醫學知識十分匱乏,百姓迷信鬼神,大多向神明求醫。在立國後,積極吸收宋朝的醫學與藥學,並且出版《治療惡瘡要論》等醫學著作。並且設有“醫人院”,在政府機構中屬“中等司”。西夏人對病理的認知大多分成血脈不通、傳染、「四大不和」(地、水、火、風)等觀點,其中四大不和是緣自藏傳佛教的說法。由於西夏本身醫學不如中原的朝代,所以一些疑難病症無法醫治,只好求助於宋朝或金朝。例如夏仁宗時,權臣任得敬患病,久治不愈。所以夏仁宗派使者向金朝請求醫療支援。夏桓宗時,其母患病,也派使至金朝求醫。這些都表示西夏醫學不如中原的朝代。

西夏武器製作十分精實,其中以夏國劍最有名,在宋朝被譽為「天下第一」。北宋文學家蘇軾曾請晁補之為其作歌,內有「試人一縷立褫魄,戲客三招森動容」。而西夏鎧甲被譽稱為堅滑光瑩,非勁弩可入,專給鐵鷂子使用。其他有名的攻城武器有名叫「對壘」的戰車、可以越壕溝而進;裝在駱駝鞍上的「旋風炮」,可以發射大石彈;以及最厲害的「神臂弓」,可以射240步至300步,「能洞重扎」。


%% -*- coding: utf-8 -*-
%% Time-stamp: <Chen Wang: 2019-12-26 11:08:11>

\section{景宗\tiny(1032-1048)}

\subsection{生平}

夏景宗李元昊(1003年6月7日-1048年1月19日),又名趙元昊,小字嵬埋,出身党項拓跋氏,即皇帝位後,放棄唐朝賜姓李與宋朝赐姓趙,改姓嵬名氏,更名曩霄(或作曩甯、曩宁),是西夏開國皇帝(1038年11月10日-1048年1月19日在位),為李繼遷孫,李德明長子,生母衛慕氏。生于1003年农历五月五日。

李元昊少年時身型魁梧,而且勤奮好學,手不釋卷,尤好法律和兵書。通漢、蕃語言,精繪畫,多才多藝。其父在位時,他率軍不斷對外出戰,擴大勢力,野心勃勃。1032年以太子身份繼位,仍称藩於宋朝。後來為表獨立,廢唐宋分別賜李姓、趙姓,改姓嵬名,改名曩霄,自称“兀卒”(党项语天子之意),以元魏王室后裔自居,並以嚴酷手段徹底翦除守舊派。大庆三年十月十一日(1038年11月10日)自立为帝,自称世祖始文本武兴法建礼仁孝皇帝,改年号为天授礼法延祚,脱离北宋,国号“大夏”,亦称西夏,定都興慶府。

建國後命大臣野利仁榮創西夏文,大力發展西夏的文化。推動教育,創蕃學,大啟西夏文教之風。開鑿「李王渠」,以便西夏國民耕種。他重用張元等漢人。他三次分別於三川口(今陝西延安西北)、好水川(今寧夏隆德東)及定川砦(今寧夏固原西北)的戰役中大敗北宋,並於遼夏第一次賀蘭山之戰,大勝遼國,奠定西夏與遼、宋兩國并列的地位。本來有意奪取關中之地,攻占長安,但因宋軍頑強抵抗,夏軍戰敗,直搗關中之美夢就此破滅。由於戰事繁多,西夏經濟破損,遂於1044年與北宋簽訂慶曆和議,向宋稱臣,被封為夏國王。為西夏建樹良多,堪稱一代英豪。

李元昊一朝文治武功達於鼎盛,但其人亦有不足之處。在位16年(1032年继承王位起計),猜忌功臣,稍有不滿即罷或殺,反而導致日後母黨專權;另外,晚年沉湎酒色,好大喜功,导致西夏内部日益腐朽,众叛亲离。據說他下令民伕每日建一座陵墓,足足建了三百六十座,作為他的疑塚,其後竟把那批民伕統統殺掉。廢皇后野利氏、太子寧令哥,改立與太子訂親的沒移氏為新皇后,招致殺身之禍,延祚十一年正月初二(1048年1月19日),其子寧令哥趁元昊酒醉時,割其鼻子,元昊最後因失血過多而死,享年46岁,庙号景宗,諡号武烈皇帝,葬泰陵。寧令哥後來因弒父之罪被處死。

\subsection{显道}

\begin{longtable}{|>{\centering\scriptsize}m{2em}|>{\centering\scriptsize}m{1.3em}|>{\centering}m{8.8em}|}
  % \caption{秦王政}\
  \toprule
  \SimHei \normalsize 年数 & \SimHei \scriptsize 公元 & \SimHei 大事件 \tabularnewline
  % \midrule
  \endfirsthead
  \toprule
  \SimHei \normalsize 年数 & \SimHei \scriptsize 公元 & \SimHei 大事件 \tabularnewline
  \midrule
  \endhead
  \midrule
  元年 & 1032 & \tabularnewline\hline
  二年 & 1033 & \tabularnewline\hline
  三年 & 1034 & \tabularnewline
  \bottomrule
\end{longtable}

\subsection{开运}

\begin{longtable}{|>{\centering\scriptsize}m{2em}|>{\centering\scriptsize}m{1.3em}|>{\centering}m{8.8em}|}
  % \caption{秦王政}\
  \toprule
  \SimHei \normalsize 年数 & \SimHei \scriptsize 公元 & \SimHei 大事件 \tabularnewline
  % \midrule
  \endfirsthead
  \toprule
  \SimHei \normalsize 年数 & \SimHei \scriptsize 公元 & \SimHei 大事件 \tabularnewline
  \midrule
  \endhead
  \midrule
  元年 & 1034 & \tabularnewline
  \bottomrule
\end{longtable}

\subsection{广运}

\begin{longtable}{|>{\centering\scriptsize}m{2em}|>{\centering\scriptsize}m{1.3em}|>{\centering}m{8.8em}|}
  % \caption{秦王政}\
  \toprule
  \SimHei \normalsize 年数 & \SimHei \scriptsize 公元 & \SimHei 大事件 \tabularnewline
  % \midrule
  \endfirsthead
  \toprule
  \SimHei \normalsize 年数 & \SimHei \scriptsize 公元 & \SimHei 大事件 \tabularnewline
  \midrule
  \endhead
  \midrule
  元年 & 1034 & \tabularnewline\hline
  二年 & 1035 & \tabularnewline\hline
  三年 & 1036 & \tabularnewline
  \bottomrule
\end{longtable}

\subsection{大庆}

\begin{longtable}{|>{\centering\scriptsize}m{2em}|>{\centering\scriptsize}m{1.3em}|>{\centering}m{8.8em}|}
  % \caption{秦王政}\
  \toprule
  \SimHei \normalsize 年数 & \SimHei \scriptsize 公元 & \SimHei 大事件 \tabularnewline
  % \midrule
  \endfirsthead
  \toprule
  \SimHei \normalsize 年数 & \SimHei \scriptsize 公元 & \SimHei 大事件 \tabularnewline
  \midrule
  \endhead
  \midrule
  元年 & 1036 & \tabularnewline\hline
  二年 & 1037 & \tabularnewline\hline
  三年 & 1038 & \tabularnewline
  \bottomrule
\end{longtable}

\subsection{天授}

\begin{longtable}{|>{\centering\scriptsize}m{2em}|>{\centering\scriptsize}m{1.3em}|>{\centering}m{8.8em}|}
  % \caption{秦王政}\
  \toprule
  \SimHei \normalsize 年数 & \SimHei \scriptsize 公元 & \SimHei 大事件 \tabularnewline
  % \midrule
  \endfirsthead
  \toprule
  \SimHei \normalsize 年数 & \SimHei \scriptsize 公元 & \SimHei 大事件 \tabularnewline
  \midrule
  \endhead
  \midrule
  元年 & 1038 & \tabularnewline\hline
  二年 & 1039 & \tabularnewline\hline
  三年 & 1040 & \tabularnewline\hline
  四年 & 1041 & \tabularnewline\hline
  五年 & 1042 & \tabularnewline\hline
  六年 & 1043 & \tabularnewline\hline
  七年 & 1044 & \tabularnewline\hline
  八年 & 1045 & \tabularnewline\hline
  九年 & 1046 & \tabularnewline\hline
  十年 & 1047 & \tabularnewline\hline
  十一年 & 1048 & \tabularnewline
  \bottomrule
\end{longtable}


%%% Local Variables:
%%% mode: latex
%%% TeX-engine: xetex
%%% TeX-master: "../Main"
%%% End:

%% -*- coding: utf-8 -*-
%% Time-stamp: <Chen Wang: 2021-11-01 16:13:57>

\section{毅宗李諒祚\tiny(1048-1067)}

\subsection{生平}

夏毅宗李諒祚(1047年3月5日-1068年1月),名諒祚,本名寧令兩岔,是西夏第二位皇帝(1048年—1068年1月在位)。夏景宗之子,生母沒藏氏,党項族人。生于1047年农历二月六日。

嘉祐八年丙辰(1063年),谅祚派使者前往宋朝,上表要求改回李姓,宋仁宗下诏谴责他,命令他遵守旧约。然而此后西夏皇室仍多用嵬名为姓氏。

1048年1月19日,景宗被殺,毅宗以一歲幼齡繼位,其母沒藏太后及其家族專權。即位次年(1049年),遼國乘景宗新喪之機,與西夏爆發第二次賀蘭山之戰,西夏大敗,損失慘重,向遼稱臣。福聖承道四年(1056年),太后被殺,舅舅沒藏訛龐執政。十二歲開始預政。奲都五年(1061年),訛龐父子密謀害他,遂殺訛龐及皇后(訛龐之女),立梁氏為皇后,親掌國政。廢行蕃禮,改用漢儀;並增設各部尚書、侍郎等多種官職,以完善中央行政體制。調整州軍,以加強對地方統治。這些措施使皇帝對軍政權力的控制得到加強。他連年對宋用兵,攻掠臨近州縣。先後收降吐蕃首領瞎氈的兒子木征和青唐吐蕃部。後期注意修好與遼、宋關係,減少戰役。拱化四年(1066年),在与北宋作战时受箭伤,拱化五年十二月(1068年1月)去世,享年僅21岁,諡号昭英皇帝。

\subsection{延嗣宁国}

\begin{longtable}{|>{\centering\scriptsize}m{2em}|>{\centering\scriptsize}m{1.3em}|>{\centering}m{8.8em}|}
  % \caption{秦王政}\
  \toprule
  \SimHei \normalsize 年数 & \SimHei \scriptsize 公元 & \SimHei 大事件 \tabularnewline
  % \midrule
  \endfirsthead
  \toprule
  \SimHei \normalsize 年数 & \SimHei \scriptsize 公元 & \SimHei 大事件 \tabularnewline
  \midrule
  \endhead
  \midrule
  元年 & 1048 & \tabularnewline
  \bottomrule
\end{longtable}

\subsection{天祐垂圣}

\begin{longtable}{|>{\centering\scriptsize}m{2em}|>{\centering\scriptsize}m{1.3em}|>{\centering}m{8.8em}|}
  % \caption{秦王政}\
  \toprule
  \SimHei \normalsize 年数 & \SimHei \scriptsize 公元 & \SimHei 大事件 \tabularnewline
  % \midrule
  \endfirsthead
  \toprule
  \SimHei \normalsize 年数 & \SimHei \scriptsize 公元 & \SimHei 大事件 \tabularnewline
  \midrule
  \endhead
  \midrule
  元年 & 1050 & \tabularnewline\hline
  二年 & 1051 & \tabularnewline\hline
  三年 & 1052 & \tabularnewline
  \bottomrule
\end{longtable}

\subsection{福圣承道}

\begin{longtable}{|>{\centering\scriptsize}m{2em}|>{\centering\scriptsize}m{1.3em}|>{\centering}m{8.8em}|}
  % \caption{秦王政}\
  \toprule
  \SimHei \normalsize 年数 & \SimHei \scriptsize 公元 & \SimHei 大事件 \tabularnewline
  % \midrule
  \endfirsthead
  \toprule
  \SimHei \normalsize 年数 & \SimHei \scriptsize 公元 & \SimHei 大事件 \tabularnewline
  \midrule
  \endhead
  \midrule
  元年 & 1053 & \tabularnewline\hline
  二年 & 1054 & \tabularnewline\hline
  三年 & 1055 & \tabularnewline\hline
  四年 & 1056 & \tabularnewline
  \bottomrule
\end{longtable}

\subsection{奲都}

\begin{longtable}{|>{\centering\scriptsize}m{2em}|>{\centering\scriptsize}m{1.3em}|>{\centering}m{8.8em}|}
  % \caption{秦王政}\
  \toprule
  \SimHei \normalsize 年数 & \SimHei \scriptsize 公元 & \SimHei 大事件 \tabularnewline
  % \midrule
  \endfirsthead
  \toprule
  \SimHei \normalsize 年数 & \SimHei \scriptsize 公元 & \SimHei 大事件 \tabularnewline
  \midrule
  \endhead
  \midrule
  元年 & 1057 & \tabularnewline\hline
  二年 & 1058 & \tabularnewline\hline
  三年 & 1059 & \tabularnewline\hline
  四年 & 1060 & \tabularnewline\hline
  五年 & 1061 & \tabularnewline\hline
  六年 & 1062 & \tabularnewline
  \bottomrule
\end{longtable}

\subsection{拱化}

\begin{longtable}{|>{\centering\scriptsize}m{2em}|>{\centering\scriptsize}m{1.3em}|>{\centering}m{8.8em}|}
  % \caption{秦王政}\
  \toprule
  \SimHei \normalsize 年数 & \SimHei \scriptsize 公元 & \SimHei 大事件 \tabularnewline
  % \midrule
  \endfirsthead
  \toprule
  \SimHei \normalsize 年数 & \SimHei \scriptsize 公元 & \SimHei 大事件 \tabularnewline
  \midrule
  \endhead
  \midrule
  元年 & 1063 & \tabularnewline\hline
  二年 & 1064 & \tabularnewline\hline
  三年 & 1065 & \tabularnewline\hline
  四年 & 1066 & \tabularnewline\hline
  五年 & 1067 & \tabularnewline
  \bottomrule
\end{longtable}


%%% Local Variables:
%%% mode: latex
%%% TeX-engine: xetex
%%% TeX-master: "../Main"
%%% End:

%% -*- coding: utf-8 -*-
%% Time-stamp: <Chen Wang: 2021-11-01 16:14:03>

\section{惠宗李秉常\tiny(1067-1086)}

\subsection{生平}

夏惠宗李秉常(1061年-1086年8月21日),西夏第三位皇帝(1068年1月-1086年8月21日在位)。父親夏毅宗,梁皇后所生。

拱化五年农历十二月(1068年1月),毅宗突然病死,英年早逝,年二十一,惠宗以七歲稚齡繼位,生母梁太后及其家族專權,執政期間沒有任何治國良策,西夏國勢積弱,北宋乘機入侵。十六歲時本能親政,但梁氏勢力很大,不能輕易翦滅,因此他仍然不能親政。後來因長期不能親政,憂憤而死,1086年农历七月十日去世,享年僅二十六岁,諡号康靖皇帝。

\subsection{乾道}

\begin{longtable}{|>{\centering\scriptsize}m{2em}|>{\centering\scriptsize}m{1.3em}|>{\centering}m{8.8em}|}
  % \caption{秦王政}\
  \toprule
  \SimHei \normalsize 年数 & \SimHei \scriptsize 公元 & \SimHei 大事件 \tabularnewline
  % \midrule
  \endfirsthead
  \toprule
  \SimHei \normalsize 年数 & \SimHei \scriptsize 公元 & \SimHei 大事件 \tabularnewline
  \midrule
  \endhead
  \midrule
  元年 & 1067 & \tabularnewline\hline
  二年 & 1068 & \tabularnewline
  \bottomrule
\end{longtable}

\subsection{天赐国庆}

\begin{longtable}{|>{\centering\scriptsize}m{2em}|>{\centering\scriptsize}m{1.3em}|>{\centering}m{8.8em}|}
  % \caption{秦王政}\
  \toprule
  \SimHei \normalsize 年数 & \SimHei \scriptsize 公元 & \SimHei 大事件 \tabularnewline
  % \midrule
  \endfirsthead
  \toprule
  \SimHei \normalsize 年数 & \SimHei \scriptsize 公元 & \SimHei 大事件 \tabularnewline
  \midrule
  \endhead
  \midrule
  元年 & 1069 & \tabularnewline\hline
  二年 & 1070 & \tabularnewline\hline
  三年 & 1071 & \tabularnewline\hline
  四年 & 1072 & \tabularnewline\hline
  五年 & 1073 & \tabularnewline\hline
  六年 & 1074 & \tabularnewline
  \bottomrule
\end{longtable}

\subsection{大安}

\begin{longtable}{|>{\centering\scriptsize}m{2em}|>{\centering\scriptsize}m{1.3em}|>{\centering}m{8.8em}|}
  % \caption{秦王政}\
  \toprule
  \SimHei \normalsize 年数 & \SimHei \scriptsize 公元 & \SimHei 大事件 \tabularnewline
  % \midrule
  \endfirsthead
  \toprule
  \SimHei \normalsize 年数 & \SimHei \scriptsize 公元 & \SimHei 大事件 \tabularnewline
  \midrule
  \endhead
  \midrule
  元年 & 1075 & \tabularnewline\hline
  二年 & 1076 & \tabularnewline\hline
  三年 & 1077 & \tabularnewline\hline
  四年 & 1078 & \tabularnewline\hline
  五年 & 1079 & \tabularnewline\hline
  六年 & 1080 & \tabularnewline\hline
  七年 & 1081 & \tabularnewline\hline
  八年 & 1082 & \tabularnewline\hline
  九年 & 1083 & \tabularnewline\hline
  十年 & 1084 & \tabularnewline\hline
  十一年 & 1085 & \tabularnewline
  \bottomrule
\end{longtable}

\subsection{天安礼定}

\begin{longtable}{|>{\centering\scriptsize}m{2em}|>{\centering\scriptsize}m{1.3em}|>{\centering}m{8.8em}|}
  % \caption{秦王政}\
  \toprule
  \SimHei \normalsize 年数 & \SimHei \scriptsize 公元 & \SimHei 大事件 \tabularnewline
  % \midrule
  \endfirsthead
  \toprule
  \SimHei \normalsize 年数 & \SimHei \scriptsize 公元 & \SimHei 大事件 \tabularnewline
  \midrule
  \endhead
  \midrule
  元年 & 1086 & \tabularnewline
  \bottomrule
\end{longtable}



%%% Local Variables:
%%% mode: latex
%%% TeX-engine: xetex
%%% TeX-master: "../Main"
%%% End:

%% -*- coding: utf-8 -*-
%% Time-stamp: <Chen Wang: 2021-11-01 16:14:22>

\section{崇宗李乾順\tiny(1086-1139)}

\subsection{生平}

夏崇宗李乾順(1083年-1139年7月1日),西夏第四位皇帝(1086年8月-1139年7月1日在位)。父惠宗李秉常,母梁皇后。

幼时祖母梁太后对他寵愛有加。1086年8月21日,父亲李秉常去世。他即位时仅3岁,梁氏专政。梁氏统治期间,西夏政治腐败,军队衰弱,北宋趁机来攻,夏军屡战屡败,自幼雄才大略的李乾顺看到了这一点,于1099年即他16岁时灭梁氏而亲政。他亲政后整顿吏治,减少赋税,注重农桑,兴修水利。

在李乾顺的励精图治下,西夏国势强盛,政治清明,社会经济得到很好的发展。另外,李乾顺的外交政策也非常巧妙。贞观初年,李乾顺多次请求辽朝下嫁公主。贞观五年(1105年)三月壬申,娶辽宗室女耶律南仙,立耶律南仙为皇后。贞观八年(1108年)六月,耶律南仙为他生下嫡长子。当时,辽朝、北宋都日益衰落,李乾顺先联辽侵宋,夺大片土地;又在辽天祚帝向西夏求救时断然拒绝,联合金朝灭辽、北宋,趁机取河西千余里之地。大德五年农历六月四日(1139年7月1日)去世。李乾顺庙号为崇宗,谥号为圣文皇帝。

\subsection{天仪治平}

\begin{longtable}{|>{\centering\scriptsize}m{2em}|>{\centering\scriptsize}m{1.3em}|>{\centering}m{8.8em}|}
  % \caption{秦王政}\
  \toprule
  \SimHei \normalsize 年数 & \SimHei \scriptsize 公元 & \SimHei 大事件 \tabularnewline
  % \midrule
  \endfirsthead
  \toprule
  \SimHei \normalsize 年数 & \SimHei \scriptsize 公元 & \SimHei 大事件 \tabularnewline
  \midrule
  \endhead
  \midrule
  元年 & 1086 & \tabularnewline\hline
  二年 & 1087 & \tabularnewline\hline
  三年 & 1088 & \tabularnewline\hline
  四年 & 1089 & \tabularnewline
  \bottomrule
\end{longtable}

\subsection{天祐民安}

\begin{longtable}{|>{\centering\scriptsize}m{2em}|>{\centering\scriptsize}m{1.3em}|>{\centering}m{8.8em}|}
  % \caption{秦王政}\
  \toprule
  \SimHei \normalsize 年数 & \SimHei \scriptsize 公元 & \SimHei 大事件 \tabularnewline
  % \midrule
  \endfirsthead
  \toprule
  \SimHei \normalsize 年数 & \SimHei \scriptsize 公元 & \SimHei 大事件 \tabularnewline
  \midrule
  \endhead
  \midrule
  元年 & 1090 & \tabularnewline\hline
  二年 & 1091 & \tabularnewline\hline
  三年 & 1092 & \tabularnewline\hline
  四年 & 1093 & \tabularnewline\hline
  五年 & 1094 & \tabularnewline\hline
  六年 & 1095 & \tabularnewline\hline
  七年 & 1096 & \tabularnewline\hline
  八年 & 1097 & \tabularnewline
  \bottomrule
\end{longtable}

\subsection{永安}

\begin{longtable}{|>{\centering\scriptsize}m{2em}|>{\centering\scriptsize}m{1.3em}|>{\centering}m{8.8em}|}
  % \caption{秦王政}\
  \toprule
  \SimHei \normalsize 年数 & \SimHei \scriptsize 公元 & \SimHei 大事件 \tabularnewline
  % \midrule
  \endfirsthead
  \toprule
  \SimHei \normalsize 年数 & \SimHei \scriptsize 公元 & \SimHei 大事件 \tabularnewline
  \midrule
  \endhead
  \midrule
  元年 & 1098 & \tabularnewline\hline
  二年 & 1099 & \tabularnewline\hline
  三年 & 1100 & \tabularnewline
  \bottomrule
\end{longtable}

\subsection{贞观}

\begin{longtable}{|>{\centering\scriptsize}m{2em}|>{\centering\scriptsize}m{1.3em}|>{\centering}m{8.8em}|}
  % \caption{秦王政}\
  \toprule
  \SimHei \normalsize 年数 & \SimHei \scriptsize 公元 & \SimHei 大事件 \tabularnewline
  % \midrule
  \endfirsthead
  \toprule
  \SimHei \normalsize 年数 & \SimHei \scriptsize 公元 & \SimHei 大事件 \tabularnewline
  \midrule
  \endhead
  \midrule
  元年 & 1101 & \tabularnewline\hline
  二年 & 1102 & \tabularnewline\hline
  三年 & 1103 & \tabularnewline\hline
  四年 & 1104 & \tabularnewline\hline
  五年 & 1105 & \tabularnewline\hline
  六年 & 1106 & \tabularnewline\hline
  七年 & 1107 & \tabularnewline\hline
  八年 & 1108 & \tabularnewline\hline
  九年 & 1109 & \tabularnewline\hline
  十年 & 1110 & \tabularnewline\hline
  十一年 & 1111 & \tabularnewline\hline
  十二年 & 1112 & \tabularnewline\hline
  十三年 & 1113 & \tabularnewline
  \bottomrule
\end{longtable}

\subsection{雍宁}

\begin{longtable}{|>{\centering\scriptsize}m{2em}|>{\centering\scriptsize}m{1.3em}|>{\centering}m{8.8em}|}
  % \caption{秦王政}\
  \toprule
  \SimHei \normalsize 年数 & \SimHei \scriptsize 公元 & \SimHei 大事件 \tabularnewline
  % \midrule
  \endfirsthead
  \toprule
  \SimHei \normalsize 年数 & \SimHei \scriptsize 公元 & \SimHei 大事件 \tabularnewline
  \midrule
  \endhead
  \midrule
  元年 & 1114 & \tabularnewline\hline
  二年 & 1115 & \tabularnewline\hline
  三年 & 1116 & \tabularnewline\hline
  四年 & 1117 & \tabularnewline\hline
  五年 & 1118 & \tabularnewline
  \bottomrule
\end{longtable}

\subsection{元德}

\begin{longtable}{|>{\centering\scriptsize}m{2em}|>{\centering\scriptsize}m{1.3em}|>{\centering}m{8.8em}|}
  % \caption{秦王政}\
  \toprule
  \SimHei \normalsize 年数 & \SimHei \scriptsize 公元 & \SimHei 大事件 \tabularnewline
  % \midrule
  \endfirsthead
  \toprule
  \SimHei \normalsize 年数 & \SimHei \scriptsize 公元 & \SimHei 大事件 \tabularnewline
  \midrule
  \endhead
  \midrule
  元年 & 1119 & \tabularnewline\hline
  二年 & 1120 & \tabularnewline\hline
  三年 & 1121 & \tabularnewline\hline
  四年 & 1122 & \tabularnewline\hline
  五年 & 1123 & \tabularnewline\hline
  六年 & 1124 & \tabularnewline\hline
  七年 & 1125 & \tabularnewline\hline
  八年 & 1126 & \tabularnewline\hline
  九年 & 1127 & \tabularnewline
  \bottomrule
\end{longtable}

\subsection{正德}

\begin{longtable}{|>{\centering\scriptsize}m{2em}|>{\centering\scriptsize}m{1.3em}|>{\centering}m{8.8em}|}
  % \caption{秦王政}\
  \toprule
  \SimHei \normalsize 年数 & \SimHei \scriptsize 公元 & \SimHei 大事件 \tabularnewline
  % \midrule
  \endfirsthead
  \toprule
  \SimHei \normalsize 年数 & \SimHei \scriptsize 公元 & \SimHei 大事件 \tabularnewline
  \midrule
  \endhead
  \midrule
  元年 & 1127 & \tabularnewline\hline
  二年 & 1128 & \tabularnewline\hline
  三年 & 1129 & \tabularnewline\hline
  四年 & 1130 & \tabularnewline\hline
  五年 & 1131 & \tabularnewline\hline
  六年 & 1132 & \tabularnewline\hline
  七年 & 1133 & \tabularnewline\hline
  八年 & 1134 & \tabularnewline
  \bottomrule
\end{longtable}

\subsection{大德}

\begin{longtable}{|>{\centering\scriptsize}m{2em}|>{\centering\scriptsize}m{1.3em}|>{\centering}m{8.8em}|}
  % \caption{秦王政}\
  \toprule
  \SimHei \normalsize 年数 & \SimHei \scriptsize 公元 & \SimHei 大事件 \tabularnewline
  % \midrule
  \endfirsthead
  \toprule
  \SimHei \normalsize 年数 & \SimHei \scriptsize 公元 & \SimHei 大事件 \tabularnewline
  \midrule
  \endhead
  \midrule
  元年 & 1135 & \tabularnewline\hline
  二年 & 1136 & \tabularnewline\hline
  三年 & 1137 & \tabularnewline\hline
  四年 & 1138 & \tabularnewline\hline
  五年 & 1139 & \tabularnewline
  \bottomrule
\end{longtable}


%%% Local Variables:
%%% mode: latex
%%% TeX-engine: xetex
%%% TeX-master: "../Main"
%%% End:

%% -*- coding: utf-8 -*-
%% Time-stamp: <Chen Wang: 2019-12-26 11:10:34>

\section{仁宗\tiny(1139-1193)}

\subsection{生平}

夏仁宗李仁孝(1124年-1193年10月16日),夏崇宗次子,母為漢人,不知名。

由于其兄李仁爱先于崇宗而死,故被立为太子。1139年7月1日夏崇宗李乾顺去世,李仁孝即位,時年十六歲。在位期間結好金國,以穩定外部環境;重用文化程度較高的党項和漢族大臣主持國政;設立各級學校,以推廣教育;實行科舉,以選拔人才;尊崇儒學,大修孔廟及尊奉孔子為文宣帝;建立翰林學士院,編纂歷朝實錄;重視禮樂,修樂書《新律》;天盛年間,頒行法典《天盛年改新定律令》;尊尚佛教,供奉藏傳佛教僧人為國師,並刻印佛經多種。

乾祐元年(1170年),得金之助,處死權相任得敬,粉碎其分國陰謀。可能因為任得敬的專權跋扈,令仁孝對武官不太信任,政策多數重文輕武,導致軍備開始廢弛,戰鬥力減弱,晚夏戰爭屢戰屢敗,國家於仁宗末年開始走下坡。但總結來說,他統治期間為西夏的盛世,也是金國、南宋的盛世,三國之間戰爭甚少,因此仁孝能專心料理國家內政。各汗國羡慕西夏之強盛,紛紛朝貢。文化臻於鼎盛,為党項文化寫下光輝燦爛的一頁。

乾祐二十四年九月二十日(1193年10月16日)崩,年七十,諡聖德皇帝,廟號仁宗。

\subsection{大庆}

\begin{longtable}{|>{\centering\scriptsize}m{2em}|>{\centering\scriptsize}m{1.3em}|>{\centering}m{8.8em}|}
  % \caption{秦王政}\
  \toprule
  \SimHei \normalsize 年数 & \SimHei \scriptsize 公元 & \SimHei 大事件 \tabularnewline
  % \midrule
  \endfirsthead
  \toprule
  \SimHei \normalsize 年数 & \SimHei \scriptsize 公元 & \SimHei 大事件 \tabularnewline
  \midrule
  \endhead
  \midrule
  元年 & 1140 & \tabularnewline\hline
  二年 & 1141 & \tabularnewline\hline
  三年 & 1142 & \tabularnewline\hline
  四年 & 1143 & \tabularnewline
  \bottomrule
\end{longtable}

\subsection{人庆}

\begin{longtable}{|>{\centering\scriptsize}m{2em}|>{\centering\scriptsize}m{1.3em}|>{\centering}m{8.8em}|}
  % \caption{秦王政}\
  \toprule
  \SimHei \normalsize 年数 & \SimHei \scriptsize 公元 & \SimHei 大事件 \tabularnewline
  % \midrule
  \endfirsthead
  \toprule
  \SimHei \normalsize 年数 & \SimHei \scriptsize 公元 & \SimHei 大事件 \tabularnewline
  \midrule
  \endhead
  \midrule
  元年 & 1144 & \tabularnewline\hline
  二年 & 1145 & \tabularnewline\hline
  三年 & 1146 & \tabularnewline\hline
  四年 & 1147 & \tabularnewline\hline
  五年 & 1148 & \tabularnewline
  \bottomrule
\end{longtable}

\subsection{天盛}

\begin{longtable}{|>{\centering\scriptsize}m{2em}|>{\centering\scriptsize}m{1.3em}|>{\centering}m{8.8em}|}
  % \caption{秦王政}\
  \toprule
  \SimHei \normalsize 年数 & \SimHei \scriptsize 公元 & \SimHei 大事件 \tabularnewline
  % \midrule
  \endfirsthead
  \toprule
  \SimHei \normalsize 年数 & \SimHei \scriptsize 公元 & \SimHei 大事件 \tabularnewline
  \midrule
  \endhead
  \midrule
  元年 & 1149 & \tabularnewline\hline
  二年 & 1150 & \tabularnewline\hline
  三年 & 1151 & \tabularnewline\hline
  四年 & 1152 & \tabularnewline\hline
  五年 & 1153 & \tabularnewline\hline
  六年 & 1154 & \tabularnewline\hline
  七年 & 1155 & \tabularnewline\hline
  八年 & 1156 & \tabularnewline\hline
  九年 & 1157 & \tabularnewline\hline
  十年 & 1158 & \tabularnewline\hline
  十一年 & 1159 & \tabularnewline\hline
  十二年 & 1160 & \tabularnewline\hline
  十三年 & 1161 & \tabularnewline\hline
  十四年 & 1162 & \tabularnewline\hline
  十五年 & 1163 & \tabularnewline\hline
  十六年 & 1164 & \tabularnewline\hline
  十七年 & 1165 & \tabularnewline\hline
  十八年 & 1166 & \tabularnewline\hline
  十九年 & 1167 & \tabularnewline\hline
  二十年 & 1168 & \tabularnewline\hline
  二一年 & 1169 & \tabularnewline
  \bottomrule
\end{longtable}

\subsection{乾佑}

\begin{longtable}{|>{\centering\scriptsize}m{2em}|>{\centering\scriptsize}m{1.3em}|>{\centering}m{8.8em}|}
  % \caption{秦王政}\
  \toprule
  \SimHei \normalsize 年数 & \SimHei \scriptsize 公元 & \SimHei 大事件 \tabularnewline
  % \midrule
  \endfirsthead
  \toprule
  \SimHei \normalsize 年数 & \SimHei \scriptsize 公元 & \SimHei 大事件 \tabularnewline
  \midrule
  \endhead
  \midrule
  元年 & 1170 & \tabularnewline\hline
  二年 & 1171 & \tabularnewline\hline
  三年 & 1172 & \tabularnewline\hline
  四年 & 1173 & \tabularnewline\hline
  五年 & 1174 & \tabularnewline\hline
  六年 & 1175 & \tabularnewline\hline
  七年 & 1176 & \tabularnewline\hline
  八年 & 1177 & \tabularnewline\hline
  九年 & 1178 & \tabularnewline\hline
  十年 & 1179 & \tabularnewline\hline
  十一年 & 1180 & \tabularnewline\hline
  十二年 & 1181 & \tabularnewline\hline
  十三年 & 1182 & \tabularnewline\hline
  十四年 & 1183 & \tabularnewline\hline
  十五年 & 1184 & \tabularnewline\hline
  十六年 & 1185 & \tabularnewline\hline
  十七年 & 1186 & \tabularnewline\hline
  十八年 & 1187 & \tabularnewline\hline
  十九年 & 1188 & \tabularnewline\hline
  二十年 & 1189 & \tabularnewline\hline
  二一年 & 1190 & \tabularnewline\hline
  二二年 & 1191 & \tabularnewline\hline
  二三年 & 1192 & \tabularnewline\hline
  二四年 & 1193 & \tabularnewline
  \bottomrule
\end{longtable}


%%% Local Variables:
%%% mode: latex
%%% TeX-engine: xetex
%%% TeX-master: "../Main"
%%% End:

%% -*- coding: utf-8 -*-
%% Time-stamp: <Chen Wang: 2021-11-01 16:14:31>

\section{桓宗李純佑\tiny(1193-1206)}

\subsection{生平}

夏桓宗李純佑(1177年-1206年3月),夏仁宗子。性溫和厚實。1193年10月16日,夏仁宗李仁孝去世,李純佑即位,時年十七,改元天慶。桓宗基本上還能奉行仁宗时期的政治方針和外交政策,對内安國養民,對外附金和宋。但隨着國家的安定承平日久,統治階層開始貪圖安逸,日益腐朽墮落,从此西夏政治日益腐败,国势衰落。從盛到衰已成為西夏社會不可逆轉的趨勢。同时,桓宗統治時期,正是蒙古崛起並日漸強大的时期,來自蒙古的嚴重威脅加速了西夏由盛而衰的歷史進程。1205年,改興慶府名為中興府,取夏國中興之意,但同年蒙古第一次進攻西夏,自此迄無寧日了。1206年3月1日,李安全發動政變,桓宗被廢。不久暴卒,年僅三十。諡昭簡皇帝。

\subsection{天庆}

\begin{longtable}{|>{\centering\scriptsize}m{2em}|>{\centering\scriptsize}m{1.3em}|>{\centering}m{8.8em}|}
  % \caption{秦王政}\
  \toprule
  \SimHei \normalsize 年数 & \SimHei \scriptsize 公元 & \SimHei 大事件 \tabularnewline
  % \midrule
  \endfirsthead
  \toprule
  \SimHei \normalsize 年数 & \SimHei \scriptsize 公元 & \SimHei 大事件 \tabularnewline
  \midrule
  \endhead
  \midrule
  元年 & 1194 & \tabularnewline\hline
  二年 & 1195 & \tabularnewline\hline
  三年 & 1196 & \tabularnewline\hline
  四年 & 1197 & \tabularnewline\hline
  五年 & 1198 & \tabularnewline\hline
  六年 & 1199 & \tabularnewline\hline
  七年 & 1200 & \tabularnewline\hline
  八年 & 1201 & \tabularnewline\hline
  九年 & 1202 & \tabularnewline\hline
  十年 & 1203 & \tabularnewline\hline
  十一年 & 1204 & \tabularnewline\hline
  十二年 & 1205 & \tabularnewline\hline
  十三年 & 1206 & \tabularnewline
  \bottomrule
\end{longtable}


%%% Local Variables:
%%% mode: latex
%%% TeX-engine: xetex
%%% TeX-master: "../Main"
%%% End:

%% -*- coding: utf-8 -*-
%% Time-stamp: <Chen Wang: 2019-12-26 11:11:10>

\section{襄宗\tiny(1206-1211)}

\subsection{生平}

夏襄宗李安全(1170年-1211年9月13日),夏崇宗孫,其父乃夏仁宗弟越王李仁友,1196年,仁友逝,安全上書要求襲越王爵位,桓宗不許,安全被降封為鎮夷郡王,他極為不滿,於是萌生了篡奪皇位之心。1206年3月1日與桓宗母羅氏合謀,廢桓宗自立,改元應天。在位時昏庸無能,破壞金國與西夏長期的友好關係,發兵侵金,為後來一場場令夏金耗盡精兵的戰役掀起序幕,改附不斷強大起來的蒙古帝国,但這一切都沒有為西夏帶來利益和跟蒙古之友好,蒙古也以西夏作為侵略目標,西夏不斷積弱。1211年8月12日,宗室齊王李遵頊發動政變,被廢,並於一個月後不明不白地死去,去世于1211年9月13日,終年四十二。諡敬慕皇帝,廟號襄宗。

\subsection{应天}

\begin{longtable}{|>{\centering\scriptsize}m{2em}|>{\centering\scriptsize}m{1.3em}|>{\centering}m{8.8em}|}
  % \caption{秦王政}\
  \toprule
  \SimHei \normalsize 年数 & \SimHei \scriptsize 公元 & \SimHei 大事件 \tabularnewline
  % \midrule
  \endfirsthead
  \toprule
  \SimHei \normalsize 年数 & \SimHei \scriptsize 公元 & \SimHei 大事件 \tabularnewline
  \midrule
  \endhead
  \midrule
  元年 & 1206 & \tabularnewline\hline
  二年 & 1207 & \tabularnewline\hline
  三年 & 1208 & \tabularnewline\hline
  四年 & 1209 & \tabularnewline
  \bottomrule
\end{longtable}

\subsection{皇建}

\begin{longtable}{|>{\centering\scriptsize}m{2em}|>{\centering\scriptsize}m{1.3em}|>{\centering}m{8.8em}|}
  % \caption{秦王政}\
  \toprule
  \SimHei \normalsize 年数 & \SimHei \scriptsize 公元 & \SimHei 大事件 \tabularnewline
  % \midrule
  \endfirsthead
  \toprule
  \SimHei \normalsize 年数 & \SimHei \scriptsize 公元 & \SimHei 大事件 \tabularnewline
  \midrule
  \endhead
  \midrule
  元年 & 1210 & \tabularnewline\hline
  二年 & 1211 & \tabularnewline
  \bottomrule
\end{longtable}


%%% Local Variables:
%%% mode: latex
%%% TeX-engine: xetex
%%% TeX-master: "../Main"
%%% End:

%% -*- coding: utf-8 -*-
%% Time-stamp: <Chen Wang: 2021-11-01 16:14:41>

\section{神宗李遵頊\tiny(1211-1223)}

\subsection{生平}

夏神宗李遵頊(1163年-1226年),西夏宗室齐国忠武王李彥宗之子。史書記載:「端重明粹,少力學,長博通群書,工隸篆」。1203年,廷试进士第一名,1211年8月12日廢襄宗自立,改元光定,成为中国历史上唯一的状元皇帝。任内全盤承襲襄宗自取滅亡的政策,繼續破壞金國與西夏關係,發兵侵金;金宣宗也不遑多讓,決定痛擊西夏。更不幸的是,夏軍軍力早已廢弛,因此不斷戰敗,反而沒有令遵頊知難而退,更激起他的野心和戰慾,繼續發動戰爭,令人民家破人亡,民怨四起,經濟嚴重破壞,國力直線下降。在位期間根本沒有想過與金國議和,雖不斷有忠良之士直言進諫,但一一被他痛罵,包括其子李德任。1223年,傳位於子李德旺,為西夏唯一的太上皇。1226年病卒,年64,諡英文皇帝,廟號神宗。

\subsection{光定}

\begin{longtable}{|>{\centering\scriptsize}m{2em}|>{\centering\scriptsize}m{1.3em}|>{\centering}m{8.8em}|}
  % \caption{秦王政}\
  \toprule
  \SimHei \normalsize 年数 & \SimHei \scriptsize 公元 & \SimHei 大事件 \tabularnewline
  % \midrule
  \endfirsthead
  \toprule
  \SimHei \normalsize 年数 & \SimHei \scriptsize 公元 & \SimHei 大事件 \tabularnewline
  \midrule
  \endhead
  \midrule
  元年 & 1211 & \tabularnewline\hline
  二年 & 1212 & \tabularnewline\hline
  三年 & 1213 & \tabularnewline\hline
  四年 & 1214 & \tabularnewline\hline
  五年 & 1215 & \tabularnewline\hline
  六年 & 1216 & \tabularnewline\hline
  七年 & 1217 & \tabularnewline\hline
  八年 & 1218 & \tabularnewline\hline
  九年 & 1219 & \tabularnewline\hline
  十年 & 1220 & \tabularnewline\hline
  十一年 & 1221 & \tabularnewline\hline
  十二年 & 1222 & \tabularnewline\hline
  十三年 & 1223 & \tabularnewline
  \bottomrule
\end{longtable}


%%% Local Variables:
%%% mode: latex
%%% TeX-engine: xetex
%%% TeX-master: "../Main"
%%% End:

%% -*- coding: utf-8 -*-
%% Time-stamp: <Chen Wang: 2019-12-26 11:11:43>

\section{献宗\tiny(1223-1226)}

\subsection{生平}

夏獻宗李德旺(1181年-1226年),夏神宗之次子。力挽面臨滅亡的西夏,但西夏經過襄宗、神宗兩朝的亡國政策,人民早已生活於水深火熱之困境中,經濟疲弊,他根本沒有回天之力。他一改前朝政策,決心與金國修好,於1225年正式和好;然金都也被蒙古包圍,金國已是泥菩薩過江,自身難保。又改國策附蒙為抗蒙,但西夏精兵早於夏金戰役中消耗殆盡,无力抵抗蒙古军,終於1226年驚憂而死,年46岁,廟號獻宗。


\subsection{乾定}

\begin{longtable}{|>{\centering\scriptsize}m{2em}|>{\centering\scriptsize}m{1.3em}|>{\centering}m{8.8em}|}
  % \caption{秦王政}\
  \toprule
  \SimHei \normalsize 年数 & \SimHei \scriptsize 公元 & \SimHei 大事件 \tabularnewline
  % \midrule
  \endfirsthead
  \toprule
  \SimHei \normalsize 年数 & \SimHei \scriptsize 公元 & \SimHei 大事件 \tabularnewline
  \midrule
  \endhead
  \midrule
  元年 & 1223 & \tabularnewline\hline
  二年 & 1224 & \tabularnewline\hline
  三年 & 1225 & \tabularnewline\hline
  四年 & 1226 & \tabularnewline
  \bottomrule
\end{longtable}


%%% Local Variables:
%%% mode: latex
%%% TeX-engine: xetex
%%% TeX-master: "../Main"
%%% End:

%% -*- coding: utf-8 -*-
%% Time-stamp: <Chen Wang: 2021-11-01 16:15:08>

\section{末帝李睍\tiny(1226-1227)}

\subsection{生平}
李\xpinyin*{睍}(?-1227年8月28日),夏献宗侄,父清平郡王,夏獻宗病危時被推舉為帝,史稱「末帝」。

李睍在位時已為西夏滅亡的前夕,曾拒降蒙古。右丞相高良惠及各將士積極抵抗蒙古,但無奈天意造化弄人,中興府發生大地震,以致瘟疫肆虐,糧水短缺,軍民死傷過半,西夏注定滅亡了。最後於1227年农历六月投降蒙古,1227年8月25日,成吉思汗病故,蒙軍恐夏主有變,李睍开城投降后,按照成吉思汗遗嘱,在成吉思汗去世三天后,也就是寶義二年七月十五日(1227年8月28日),李睍被杀。西夏滅亡後,蒙古軍商议屠城,最終在出身党項族的蒙古軍將领察罕的极力勸諫下使中興府內的百姓避免了被屠殺的命运,察罕随后入城安撫城内軍民,使西夏遗民得以保全。


\subsection{宝义}

\begin{longtable}{|>{\centering\scriptsize}m{2em}|>{\centering\scriptsize}m{1.3em}|>{\centering}m{8.8em}|}
  % \caption{秦王政}\
  \toprule
  \SimHei \normalsize 年数 & \SimHei \scriptsize 公元 & \SimHei 大事件 \tabularnewline
  % \midrule
  \endfirsthead
  \toprule
  \SimHei \normalsize 年数 & \SimHei \scriptsize 公元 & \SimHei 大事件 \tabularnewline
  \midrule
  \endhead
  \midrule
  元年 & 1226 & \tabularnewline\hline
  二年 & 1227 & \tabularnewline
  \bottomrule
\end{longtable}


%%% Local Variables:
%%% mode: latex
%%% TeX-engine: xetex
%%% TeX-master: "../Main"
%%% End:



%%% Local Variables:
%%% mode: latex
%%% TeX-engine: xetex
%%% TeX-master: "../Main"
%%% End:
