%% -*- coding: utf-8 -*-
%% Time-stamp: <Chen Wang: 2019-12-17 11:53:07>

\section{平帝\tiny(1-5)}

\subsection{生平}

漢平帝劉衎(前9年-6年2月3日),原名劉箕子,后改名衎kàn)他是西汉第十四位、亦即倒数第二位皇帝(前1年10月17日-6年2月3日在位),其正式諡號為「孝平皇帝」,後世省略「孝」字稱「漢平帝」。

劉衎父親是漢元帝之子、中山孝王劉興,母親衛姬。前任君主漢哀帝的堂弟。

前1年,漢哀帝去世後,王莽再出任大司馬,太皇太后與王莽商議帝位繼承人選,劉衎獲擁立為帝,承汉成帝嗣。當時太皇太后臨朝,外戚王氏一族當權,王莽當政,百官大都聽命於王莽。王莽的女儿是汉平帝的皇后,史称孝平皇后。

由於王莽担心平帝母衛姬外戚勢力坐大,阻止衛姬及衛氏外戚來長安。衛姬日夜哭泣,想念在長安的平帝。平帝逐漸長大,開始對王莽不滿,6年後突然死亡。年仅14岁。後王莽立劉嬰為皇太子,自己擔任“攝皇帝”。根据翟義起兵以及日后更始政权的说法是王莽毒杀平帝,但王莽予以否认。

\subsection{元始}

\begin{longtable}{|>{\centering\scriptsize}m{2em}|>{\centering\scriptsize}m{1.3em}|>{\centering}m{8.8em}|}
  % \caption{秦王政}\
  \toprule
  \SimHei \normalsize 年数 & \SimHei \scriptsize 公元 & \SimHei 大事件 \tabularnewline
  % \midrule
  \endfirsthead
  \toprule
  \SimHei \normalsize 年数 & \SimHei \scriptsize 公元 & \SimHei 大事件 \tabularnewline
  \midrule
  \endhead
  \midrule
  元年 & 1 & \tabularnewline\hline
  二年 & 2 & \tabularnewline\hline
  三年 & 3 & \tabularnewline\hline
  四年 & 4 & \tabularnewline\hline
  五年 & 5 & \tabularnewline
  \bottomrule
\end{longtable}


%%% Local Variables:
%%% mode: latex
%%% TeX-engine: xetex
%%% TeX-master: "../Main"
%%% End:
