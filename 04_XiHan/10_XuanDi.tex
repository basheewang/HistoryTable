%% -*- coding: utf-8 -*-
%% Time-stamp: <Chen Wang: 2019-12-16 14:24:52>

\section{宣帝\tiny(BC74-BC49)}

\subsection{海昏侯生平}

刘贺(?-前59年),西汉第九位皇帝(前74年7月18日至同年8月14日在位)。漢武帝孫,原為昌邑王,在漢昭帝駕崩後繼位,在位27日,被權臣霍光以行事乖戾為由廢黜,後改封為海昏侯。

劉賀袭父劉髆(漢武帝與李夫人之子)封为昌邑王(地近今山東菏泽)。

昌邑郎中令龔遂以明經為官,侍奉昌邑王劉賀。昌邑王行為舉止多不正當,龔遂為人忠厚,剛毅有大節操,對內對昌邑王勸諫,對外責備師傅卿相,引用經義,陳言禍福,以至於流淚,正言直諫而不停止。當面指責昌邑王過錯,昌邑王甚至掩起耳朵跑走,說:「郎中令善愧人。」昌邑國中皆畏懼龔遂。

昌邑王曾長久與駕車奴僕、炊事人員嘻戲共食,賞賜沒有節制。龔遂晉見昌邑王,跪著行走,流淚不止,左右侍御都淚流滿面。昌邑王問:「郎中令何為哭?」龔遂說:「臣痛社稷危也!願賜清閒竭愚。」昌邑王令左右退下,龔遂問:「大王知膠西王所以為無道亡乎?」昌邑王說:「不知也。」龔遂說:「臣聞膠西王有諛臣侯得,王所為擬於桀、紂也,得以為堯、舜也。王說其諂諛,嘗與寢處,唯得所言,以至於是。今大王親近群小,漸漬邪惡所習,存亡之機,不可不慎也。臣請選郎通經術有行義者與王起居,坐則通《詩》、《書》,立則習禮容,宜有益。」昌邑王同意他。龔遂於是挑選郎中張安等十人侍奉昌邑王。幾天之後,昌邑王卻遠離張安等人。

昌邑王在昌邑國時,曾數次出現妖怪。曾見到有白色的狗,身長三尺,沒有頭,在其頸部以下似人,頭戴方山冠。後來出現了熊,昌邑王左右都沒看見。又有大鳥飛來集聚在昌邑宮中。昌邑王得知後,非常討厭,數次以此詢問郎中令龔遂。龔遂說:「此天戒,言在側者盡冠狗也,去之則存,不去則亡矣。」之後又聽到人的聲音,說:「熊!」望去而見到大熊,左右都沒看見,昌邑王以此問龔遂,龔遂說:「熊,山野之獸,而來入宮室,王獨見之,此天戒大王,恐宮室將空,危亡象也。」昌邑王望天嘆息說:「不祥何為數來!」龔遂磕頭說:「臣不敢隱忠,數言危亡之戒,大王不說。夫國之存亡,豈在臣言哉?願王內自揆度。大王誦《詩》三百五篇,人事浹,王道備,王之所行中《詩》一篇何等也?大王位為諸侯王,行污於庶人,以存難,以亡易,宜深察之。」之後又有血弄髒了昌邑王的坐席,昌邑王問龔遂,龔遂大聲叫說:「宮空不久,祅祥數至。血者,陰憂象也。宜畏慎自省。」然而,昌邑王依然故我。

昭帝元平元年(前74年),昭帝駕崩,没有子嗣,大司馬大將軍霍光征召昌邑王主持喪禮。

璽書說:“制詔昌邑王:使行大鴻臚事少府樂成、宗正德、光祿大夫吉、中郎將利漢徵王,乘七乘傳詣長安邸。”

凌晨一點左右,用燭火照著打開璽書。當天中午,昌邑王就出發了,下午四五點抵達定陶,走了一百三十五里,侍從的馬一匹接一匹死在路上。郎中令龔遂向昌邑王勸諫,昌邑王才令郎官、謁者五十多人返回昌邑。昌邑王到濟陽,尋求鳴叫聲很長的雞,路上買合竹杖。經過弘農,讓身材高大的奴僕善用裝載衣物的車輛裝載搶來的女子。到了湖縣,使者就此事責備昌邑國相安樂,安樂告訴龔遂,龔遂進去問昌邑王,昌邑王說:「無有。」龔遂說:「即無有,何愛一善以毀行義!請收屬吏,以湔洒大王。」就揪住善,交给衛士執法。

昌邑王到霸上,大鴻臚在郊外迎接,主管車馬的騶官奉上皇帝乘坐的車子。昌邑王讓他的僕人壽成駕車,與郎中令龔遂同車。天明到了廣明東都門,龔遂說:「禮,奔喪望見國都哭。此長安東郭門也。」昌邑王說:「我嗌痛,不能哭。」到了城門,龔遂又說,昌邑王說:「城門與郭門等耳。」當快到未央宫的東門,龔遂說:「昌邑帳在是闕外馳道北,未至帳所,有南北行道,馬足未至數步,大王宜下車,鄉闕西面伏,哭盡哀止。」昌邑王說:「諾。」到了那裡,按禮儀哭喪。昌邑王接受皇帝璽印和绶帶,六月丙寅(公历7月18日),即天子位。

即位以后,昌邑王夢見蒼蠅屎堆積在西階的東面,約五六石,用大瓦覆蓋,揭開一看,是蒼蠅屎。以此問龔遂,龔遂說:「陛下,之《詩》不云乎?『營營青蠅,至於籓;愷悌君子,毋信讒言。』陛下左側讒人眾多,如是青蠅惡矣。宜進先帝大臣子孫親近以為左右。如不忍昌邑故人,信用讒諛,必有凶咎。願詭禍為福,皆放逐之。臣當先逐矣。」

昌邑國國相安樂改任長樂衛尉,龔遂見到安樂,流著眼淚說:「王立為天子,日益驕溢,諫之不復聽,今哀痛未盡,日與近臣飲食作樂,鬥虎豹,召皮軒,車九流,驅馳東西,所為悖道。古制寬,大臣有隱退,今去不得,陽狂恐知,身死為世戮,奈何?君,陛下故相,宜極諫爭。」

《漢書》卷六八《霍光傳》:“遂召丞相、御史、將軍、列侯、中二千石、大夫、博士會議未央宮。光曰:「昌邑王行昏亂,恐危社稷,如何?」群臣皆驚鄂失色,莫敢發言,但唯唯而已。田延年前,離席按劍,曰:「先帝屬將軍以幼孤,寄將軍以天下,以將軍忠賢能安劉氏也。今群下鼎沸,社稷將傾,且漢之傳謚常為孝者,以長有天下,令宗廟血食也。如令漢家絕祀,將軍雖死,何面目見先帝於地下乎?今日之議,不得旋踵。群臣後應者,臣請劍斬之。」光謝曰:「九卿責光是也。天下匈匈不安,光當受難。」於是議者皆叩頭,曰:「萬姓之命在於將軍,唯大將軍令。」光即與群臣俱見白太后,具陳昌邑王不可以承宗廟狀。”

昌邑王即位二十七日,曾与僚属密议罢黜霍光职权,但被霍光以行为淫乱而废。昌邑國群臣因涉昌邑王事而被入罪,皆被處死,死者二百多人,只有龔遂與中尉王吉因數次勸諫而得以免死,受髡刑,發配築城。蘇軾《霍光疏昌邑王之罪》析此事,「其中從官,必有謀光者,光知之,故立、廢賀,非專以淫亂故也。二百人方誅,號呼於市,曰:『當斷不斷,反受其亂。』此其有謀明矣。特其事秘密,無緣得之。著此者,亦欲後人微見其意也。」

据《漢書·霍光金日磾傳》載霍光所述的劉賀罪行:“受璽以來二十七日,使者旁午,持節詔諸官署徵發,凡千一百二十七事。”刘贺在即位27天内,就做了1,127件荒唐事情,平均一天40件。霍光以其不堪重任,與大臣奏请上官太后(霍光外孫女)下诏,于同月癸巳(公历8月14日)废黜了他,并且亲自送他回到封地昌邑,削去王号,给他食邑二千户,另赐刘贺的四个姐妹汤沐邑千户。同年9月10日,霍光尊立戾太子唯一的遺孫劉病已為帝。

元康二年,霍光寫信給山阳太守张敞:“谨备盗贼,察往来宾客。毋下所赐书。”要求当地官员密切监视刘贺。前63年汉宣帝封刘贺為海昏侯。

刘贺的儿子刘充国、刘奉亲都在袭封海昏侯之前死去,海昏侯国一度绝封,汉元帝初元三年(前46年)才复封刘贺另一个儿子刘代宗为海昏侯。

\subsection{宣帝生平}

漢宣帝劉詢(前91年-前48年1月10日),原名刘病已,字次卿,即位九年後改名询,西汉第十位皇帝(前74年9月10日—前48年1月10日在位),其正式諡號為「孝宣皇帝」,後世省略「孝」字稱「漢宣帝」。汉武帝的曾孙,戾太子刘据的长孙,史皇孙刘进的长子,生母為王翁須。

汉宣帝是汉武帝的曾孙,祖父为衞太子刘据。祖母史良娣,汉武帝元鼎四年(前113年)入为衞太子的良娣,生刘进,号史皇孙。史皇孙刘进于汉武帝太始年间(前96年—前93年)娶王翁须,生刘病已,时号称“皇曾孙”。

汉武帝征和二年(前91年),“巫蛊之祸”爆发,刘病已曾祖母卫子夫、祖父衞太子刘据、祖母史良娣、父亲史皇孙刘进、母亲王翁須均因此被杀,刚刚出生数月的刘病已也被投入大牢。由于他还是个婴儿,廷尉监丙吉在狱中挑选两位女囚趙徵卿與胡組做他的乳母,暂时免除二人刑罚。

巫蛊之狱连年不决,汉武帝后元二年(前87年),因为有人说长安狱中有天子气,武帝命令处死長安所有監獄的犯人,使者先到中都官詔獄處決犯人,後連夜趕到丙吉掌管的郡邸獄,丙吉据门不纳使者,保住了刘病已的性命。第二天使者回宮報告,武帝感慨,以為天意,就撤销了这道命令,並大赦,四歲的刘病已遇赦出狱。

出狱后的刘病已被丙吉送至祖母史良娣的娘家。史家怜其孤苦,对其照顾甚厚。不久刘病已恢复宗室身份,诏养于掖庭。是任掖庭令的张贺是衞太子刘据的故吏,哀衞太子无辜受难和皇曾孙的孤弱,对其抚养甚厚。及长,张贺教其诗书为之启蒙,后自费延请名儒東海澓中翁教授刘病已。刘病已聪颖好学,不久即通晓儒家经典。与此同时,刘病已亦喜好游侠、喜好斗鸡走马,游侠于三辅一带,结识了戴长乐等。这些民间经历都成为他日后当皇帝积累了重要的经验。

前76年,刘病已到了成家娶亲的年龄,掖庭令张贺有一孙女与刘病已年龄相仿,因此打算把她嫁与刘病已为妻。但是却遭到为人谨慎的弟弟张安世的强烈反对,他说:“曾孙乃衞太子后也,幸得平民衣食县官,足矣,勿复言予女事!”张贺不敢违逆弟弟的意思,只好为刘病已另聘属下许广汉的女儿许平君为妻。刘病已与许平君婚后感情很好,不久生下了儿子刘奭,也就是后来的汉元帝。

汉昭帝元平元年(前74年),汉昭帝驾崩,由于无嗣,大司马霍光拥立的昌邑王刘贺为帝。但是刘贺在即位的27天就被权臣霍光提请其外孙女上官皇太后废掉。在确立继任人选时,时任光祿大夫的丙吉此时向霍光推荐刘病已,元平元年秋七月庚申(前74年9月10日),刘病已入宫见上官太后,被封为阳武侯,同日登基为皇帝,承嗣汉昭帝,隔年改元本始。

宣帝由于是霍光所立,他吸取昌邑王被废的教训,初即位政事一决于光。唯立后问题上坚持己见,他与发妻许平君感情深厚,当上皇帝后许平君并没有立即被立为皇后,而是仅封为婕妤。朝臣和上官皇太后都认为应立霍光的小女儿霍成君为皇后。于是汉宣帝“诏求微时故剑”,群臣见宣帝意思坚决,于是议决立许平君为皇后。

霍光的夫人显对女儿没能当上皇后非常恼怒。本始三年时值皇后许平君有孕,霍光的夫人于是勾结女医生淳于衍将其暗杀。霍光知道后非常惊愕,但是他没有去追究自己的妻子罪行,而是利用自己的权势授意宣帝不追查此事。次年,霍成君如愿以偿成为皇后。汉宣帝对结发之妻的去世非常悲伤,这也影响了他后来对继任人的选择。后来他渐渐对时为太子的汉元帝感到不满意,并下了“乱吾家者,太子也”的评语,但始终没有废汉元帝的太子之位。

霍光属于汉武帝时的衞氏外戚集团。霍光十五岁,其兄(同父异母)霍去病回到家乡认祖归宗,把他带到长安,并因兄长的关系出任郎官,开始了漫长的仕宦生涯。

汉武帝末年巫蛊之祸,衞氏外戚遭到了毁灭性的打击,皇后卫子夫、大司馬卫青的子嗣以及衞太子一族全被族誅,但是霍氏躲过了此难。之后汉武帝渐渐明白过来,于是霍光开始受到重用。汉武帝临死前,任命霍光、金日磾、上官桀三人为辅政大臣,并以霍光为首,加封其为大司马。但是不久金日磾去世,霍光以和親拉攏上官桀,昭帝封光為丞相,开始独揽大权。

地节二年(前68年)霍光病逝,宣帝下令以帝王的规格下葬霍光,同时亦开始亲政。面对霍氏宗族的专权,汉宣帝不动声色对其予以翦除。他先是迁霍光的女婿大将军范明友为光禄勋,羽林监任胜为安定郡太守,几个月后又把霍光的姐夫张朔由給事中光祿大夫改为蜀郡太守,孙婿王汉为武威郡太守,长乐宫卫尉邓广汉为少府,这样夺取了他们的军权,扫清了霍家的外围势力。接着开始对霍家动手,改霍禹为大司马,无印绶,也就是剥夺了兵权,霍光的另一女婿赵平的兵权也被夺,空下来的职位完全由汉宣帝的外戚史、许两家子弟充任。

霍光是权力斗争的高手,但是他的儿孙却都很无能。霍光的儿子霍禹面对这种情况毫无应对之策,只是整日与霍山、霍云等哭泣。不久霍光夫人显毒杀许平君的事情开始败露。地节四年七月,大司马霍禹谋反事发,汉宣帝下令诛杀冠阳侯霍云、乐平侯霍山(两人皆为霍去病之孙)諸姊妹壻度遼將軍范明友、長信少府鄧廣漢、中郎將任勝、騎都尉趙平、長安男子馮殷等。与此同时,霍光之女霍皇后被废,于十二年后被迫自杀。

汉宣帝尚为平民之时,就对霍氏的权势有很深的了解。霍光挟专权之势,行伊尹废立天子之事,更是让汉宣帝胆颤心惊。在汉宣帝即位之初,汉宣帝拜谒高庙,霍光为骖乘(也就是駕驶车马),汉宣帝对其深为忌惮,在车上犹如芒刺在背;但是当驃骑将军张安世为骖乘时,汉宣帝体貌从容,一点不感到紧张。所以民间传说为:“威震主者不畜,霍氏之禍萌於驂乘。”

霍氏一门虽然被诛,但是汉宣帝仍然十分感念霍光的功勋,在麒麟阁十一功臣中,霍光名列第一,称“大司马大将军博陆侯,姓霍氏”,仅称官职和爵位而不道其名,以示尊重。后来又封霍光堂兄弟的后裔为博陆侯,以续霍光的祭祠。

宣帝虽然诛除霍氏一族,但是并没有废除霍光之政。他通过诏书正式肯定霍光的功绩,并且继续霍光的政策。他继续推行轻徭薄赋与民休息的政策,把皇家掌控的园囿和公田分给平民耕种,并贷给他们种子。后来又在元康元年(前65年)、元康二年(前64年)、神爵元年(前61年)和五凤四年(前54年)下令勾销百姓所贷官府的种子,如果受灾则免除他们的赋税。还设立常平仓,平抑物价,保证物价的稳定。此外汉宣帝还减少人口税(即算赋)。

汉宣帝曾生长于民间,为平民时喜欢游侠,足迹遍于三辅,因此深知吏治的重要性。他五日一听事,对官吏观其言,察其行,考试功能。他要求官吏尽职,地节三年(公元前67年)下诏说:“二千石严教吏谨视遇,毋令失职。”要求郡国长官管教和督促地方官吏,不能让他们失职。

他强调决狱宜平,特设廷平官。曾下诏说:“间者吏用法,巧文寖深,是朕之不德也。夫决狱不当,使有罪兴邪,不辜蒙戮,父子悲恨,朕甚伤之。今遣廷史与郡鞠狱,任轻禄薄,其为置廷平,秩六百石,员四人。其务平之,以称朕意。”他要求官吏奉法,元康二年(公元前64年)下诏说,“吏务平法。或擅兴徭役,饰厨传,称过使客,越职逾法,以取名誉,譬犹践薄冰以待白日,岂不殆哉!”他审察吏治,元康四年派遣大中大夫强等十二人循行天下,主要任务是“察吏治得失”;五凤四年(公元前54年)又派遣丞相、御史掾二十四人循行天下,“举冤狱,察擅为苛禁深刻者”。

反对苛政,下诏批评说:“今郡国二千石或擅为苛禁,禁民嫁娶不得具酒食相贺召”,即反对地方长官干涉民间喜庆之事。他反对欺谩,黄龙元年(公元前49年)诏责当时“上计簿,具文而已,务为欺谩,以避其课”,指令“御史察计簿,疑非实者,按之,使真伪无相乱”。

根据吏治情况,奖功罚罪。奖赏有功者,如:地节三年(公元前67年)对安抚流民有功的胶东相王成,下诏奖励,定秩中二千石,赐爵关内侯。神爵四年(公元前58 年)对治行优异的颍川太守黄霸,定秩中二千石,赐爵关内侯,黄金百斤,同时对颍川吏民也有赏赐。王成与黄霸,原秩二千石,一年得一千四百四十石,升秩中二千石,一年得二千一百六十石,增加秩俸百分之五十。责罚罪过者,如:元康二年(公元前64年)冬,本来精明能干、治理有绩的京兆尹赵广汉,因执法出了偏差,“坐贼杀不辜,鞠狱故不以实,擅斥除骑士乏军兴数罪”,而被腰斩。神爵四年(公元前58年)十一月,号称“屠伯”的河南太守严延年因酷急和诽谤之罪,弃市。

故史称宣帝之治“信赏必罚,综核名实”、“吏称其职,民安其业”。

与汉武帝劳民伤财式的连番对匈奴发动战争的方式不同,汉宣帝对匈奴的战争采用了更多的持巧,军事、政治、经济多管齐下。宣帝即位之初,汉与烏孫为了反抗匈奴侵扰,相约分头出兵击匈奴,匈奴无力抵抗而逃,损失很重。后来匈奴又遭乌孙、乌桓、丁零等族袭击,加之大雪成灾,力量大大削弱,故欲与汉和亲。于是汉边境“少事”。宣帝亲政时,正是匈奴内乱外患之日,无力侵扰汉境。为了减少对匈奴边防驻军的压力,他下令减少军屯。罢车骑将军、右将军屯兵。

匈奴内乱,出现了五个单于,各派多争取与汉和亲,或来投靠汉朝。汉为了自身的安宁,也积极应付。神爵三年(公元前59 年),匈奴日逐王先贤掸率众来降,汉封其为归德靖侯。五凤二年(公元前56 年),匈奴左大将军王定来降,封其为信成侯。同年,匈奴呼遬累单于来降,汉也封其为列侯。五凤三年(公元前55 年)三月,宣帝诏中提到:“(匈奴)诸王并自立,分为五单于,更相攻击,死者以万数,畜产大耗什八九,人民饥饿,相燔烧以求食,因大乖乱。单于阏氏子孙昆弟及呼遬累单于、名王、右伊秩訾、且渠、当户以下将众五万余人来降归义。单于称臣,使弟奉珍朝贺。正月,北边晏然,靡有兵革之事。”

汉朝此时设置西河、北地属国,以安置匈奴来降者。次年,匈奴单于向汉称臣,派遣其弟谷蠡王入侍。汉朝因边塞无寇,减戍卒十分之二。甘露元年(公元前53 年),匈奴呼韓邪單于派遣其子右贤王铢娄渠堂入侍汉廷;郅支单于也派遣其子右大将驹于利受入侍于汉。甘露二年(公元前52 年),呼韩邪单于叩五原塞,表示愿奉国珍三年正月来朝,宣帝同意,并安排接待。次年正月,呼韩邪来汉朝贺,受到盛情接待,并得到很多赏赐。这年郅支单于也遣使来汉奉献。甘露四年,呼韩邪单于、郅支单于都遣使朝献于汉,汉朝款待呼韩邪单于的使者格外有礼。黄龙元年(公元前49 年)正月,呼韩邪单于又来朝,汉朝对他礼赐如初。

宣帝初年,西羌先零部落擅自北渡湟水,侵占汉民地区。元康三年(公元前63年),西羌先零部落与各部落的酋长二百多人集会,“解仇交质”,订立盟约,打算共同侵扰汉地。宣帝闻知,问赵充国如何对策。赵充国以为,羌人各部盟约,还可能联合其他各部,应当及早准备。他建议一方面命令边兵加强战备,监视诸羌;一方面要破坏诸羌联合,探听其预谋内情。于是派遣义渠安国出使诸羌,了解其动向。

义渠前去,召集诸羌首领,杀了逆而不顺者,又调兵杀了先零羌民一千余人。西羌各部震恐,起而反抗,犯汉边塞,攻城邑,杀长吏。神爵元年(公元前61 年)春,义渠所部三千骑兵被羌人袭击,退到令居,向皇帝报告情况。宣帝当即调发兵马前往金城。以后将军赵充国、强弩将军许延寿带兵前往;又任酒泉太守辛武贤为破羌将军,与两将军并进。

赵充国到了金城,以哨兵了解敌情,派间谍宣传政策,日飨军士而不进击。西羌人见汉军坚壁固守,无法进攻,互相埋怨,发生了矛盾。辛武贤以为进军时机已到,向皇帝上书建议进兵。赵充国以为,辛武贤的建议不妥,如果冒险进兵,必然进退两难。他一再上书建议只能先击主谋者先零部落,逼其悔过而赦之,再选择良吏前去抚慰羌众。宣帝要他作详细说明。赵充国反复论说,马上进击失十二利,留兵屯田有十二便。宣帝肯定了赵充国屯田之策,于是诏令罢兵,让赵充国负责屯田。到了神爵二年(公元前60 年),羌民斩了先零大豪杨玉、犹非之首,向汉投顺,汉朝设金城属国以安置投顺的羌民。羌乱至此告一段落。

汉自張騫在前138年—前126年和前119年两次出使中亚(大宛、康居、大夏、烏孫、阿尔沙克王朝、身毒),和前104—前102年李广利两次伐大宛获胜之后,于前102年在西域的天山山脉南麓乌堡设置校尉,屯田于渠犁,将塔里木盆地的26个印欧人的城邦国置于西汉的管制之下。地节二年(公元前68年),宣帝派遣侍郎郑吉到渠犁负责屯田。郑吉通过屯田积蓄了粮食,发兵打败了车师。宣帝诏令郑吉继续在渠犁与车师屯田积粮,以管制西域,对付匈奴。匈奴得知消息,前来争夺车师之地。郑吉固守力弱,要求增援。宣帝诏令长罗侯常惠带领张掖郡、酒泉郡的骑兵前往车师北边千余里,显示汉军威武,吓得匈奴骑兵退去。车师王因得到汉军保护而不受匈奴欺压,乐于“亲汉”。稍后,郑吉又迎匈奴日逐王来汉投降。宣帝先命郑吉负责衞护鄯善西南方(南道)各国的安全,继又命其兼护车师西北方(北道)各国的安全,所以号称“都护”。宣帝还封郑吉为安远侯,这是神爵三年(公元前59 年)之事。

西域都护的幕府,设置在烏壘城(在今新疆库尔勒与轮台之间),负责处理西域三十六国事务,同时主管屯田事业。汉朝的西域都护取代了匈奴在西域的僮仆都尉,反映了汉匈政治力量在西域的消长,所以史称:“汉之号令班于西域矣,始于张骞而成于郑吉。”

由于宣帝长期在民间生活,深知民间疾苦,所以他在位时期,勤俭治国,而且还很放松人民的思想,对大臣要求严格,特别是宣帝亲政以后,汉朝的政治更加清明,社会经济更加繁荣。在亲政的二十年中,他着重于整肃吏治,加强皇权。他不但族灭了腐败的霍氏家族,而且诛杀了一些地位很高的、腐朽贪污的官员。为维护法律正常行使,宣帝设置治御史以审核廷尉量刑轻重;设廷尉平至地方鞠狱,规定郡国呈报狱囚被笞瘐死名数,重视民命之余又加强中央对地方的控制。此外宣帝又召集著名儒生在未央宫讲论五经异同,目的是为了巩固皇权、统一思想。其余如废除一些苛法,屡次蠲免田租、算赋,招抚流亡,在发展农业生产方面继续霍光的政策。对周边異族的关系,则软硬皆施。神爵元年(前60年),先零部(属西羌)与诸羌联盟并和匈奴借兵,企图对汉复仇。宣帝派后将军赵充国、弩将军許延壽出金城攻击西羌,均获胜利,留赵充国屯田。神爵二年五月(前59年),西羌杀其首领杨玉、犹非等,遂降漢。宣帝设金城属国,撤回屯田军。袭破车师。时匈奴发生内乱,呼韩邪单于于甘露三年(前51年)亲至五原郡塞上请求入朝称臣,成了汉朝的藩属,宣帝又得以完成武帝倾国之力而未完成的事业。

漢宣帝在位期间,「吏称其职,民安其业」,号称「中兴」,应该说,宣帝统治时期是西漢武力最强盛、经济最繁荣的时候,因此史书对宣帝大为赞赏,曰:“孝宣之治,信赏必罚,文治武功,可谓中兴”,算是西漢、甚至是中國歷史上,少有的中興之主。他与前任汉昭帝刘弗陵的统治并称为昭宣之治。

民国史学家吕思勉:「宣帝是个『旧劳于外』的人,颇知道民生疾苦,极其留意吏治,武帝和霍光时,用法都极严。宣帝却留意于平恕,也算西汉一个贤君。」

黄龙元年十二月甲戌日(前48年1月10日),汉宣帝去世,在位25年,享年43岁。谥号孝宣皇帝,东汉建武年間上庙号中宗。初元元年正月辛丑(前48年2月6日),葬于今天西安市东郊的杜陵。

宣帝是中國歷史上唯一一位在即位前受过牢狱之苦的大一统皇朝皇帝。宣帝改名“询”的理由是“病”、“已”两字太过常用,臣民避讳不易;或許也認為這二字有些不吉。宣帝与许皇后和霍皇后的感情纠葛是越剧《漢宮怨》的主题。其太子刘奭十分相信儒家學說,對宣帝某些專制行為不滿;宣帝在训斥其时說:“汉家自有制度,本以霸王道杂之,奈何纯用德教,用周政乎?”宣帝是西汉四位拥有庙号的皇帝之一,廟號為「中宗」,其餘三位為漢高帝(太祖)、漢文帝(太宗)、漢武帝(世宗)。

\subsection{本始}

\begin{longtable}{|>{\centering\scriptsize}m{2em}|>{\centering\scriptsize}m{1.3em}|>{\centering}m{8.8em}|}
  % \caption{秦王政}\
  \toprule
  \SimHei \normalsize 年数 & \SimHei \scriptsize 公元 & \SimHei 大事件 \tabularnewline
  % \midrule
  \endfirsthead
  \toprule
  \SimHei \normalsize 年数 & \SimHei \scriptsize 公元 & \SimHei 大事件 \tabularnewline
  \midrule
  \endhead
  \midrule
  元年 & -73 & \tabularnewline\hline
  二年 & -72 & \tabularnewline\hline
  三年 & -71 & \tabularnewline\hline
  四年 & -70 & \tabularnewline
  \bottomrule
\end{longtable}


\subsection{地节}

\begin{longtable}{|>{\centering\scriptsize}m{2em}|>{\centering\scriptsize}m{1.3em}|>{\centering}m{8.8em}|}
  % \caption{秦王政}\
  \toprule
  \SimHei \normalsize 年数 & \SimHei \scriptsize 公元 & \SimHei 大事件 \tabularnewline
  % \midrule
  \endfirsthead
  \toprule
  \SimHei \normalsize 年数 & \SimHei \scriptsize 公元 & \SimHei 大事件 \tabularnewline
  \midrule
  \endhead
  \midrule
  元年 & -69 & \tabularnewline\hline
  二年 & -68 & \tabularnewline\hline
  三年 & -67 & \tabularnewline\hline
  四年 & -66 & \tabularnewline
  \bottomrule
\end{longtable}


\subsection{元康}

\begin{longtable}{|>{\centering\scriptsize}m{2em}|>{\centering\scriptsize}m{1.3em}|>{\centering}m{8.8em}|}
  % \caption{秦王政}\
  \toprule
  \SimHei \normalsize 年数 & \SimHei \scriptsize 公元 & \SimHei 大事件 \tabularnewline
  % \midrule
  \endfirsthead
  \toprule
  \SimHei \normalsize 年数 & \SimHei \scriptsize 公元 & \SimHei 大事件 \tabularnewline
  \midrule
  \endhead
  \midrule
  元年 & -65 & \tabularnewline\hline
  二年 & -64 & \tabularnewline\hline
  三年 & -63 & \tabularnewline\hline
  四年 & -62 & \tabularnewline
  \bottomrule
\end{longtable}

\subsection{神爵}

\begin{longtable}{|>{\centering\scriptsize}m{2em}|>{\centering\scriptsize}m{1.3em}|>{\centering}m{8.8em}|}
  % \caption{秦王政}\
  \toprule
  \SimHei \normalsize 年数 & \SimHei \scriptsize 公元 & \SimHei 大事件 \tabularnewline
  % \midrule
  \endfirsthead
  \toprule
  \SimHei \normalsize 年数 & \SimHei \scriptsize 公元 & \SimHei 大事件 \tabularnewline
  \midrule
  \endhead
  \midrule
  元年 & -61 & \tabularnewline\hline
  二年 & -60 & \tabularnewline\hline
  三年 & -59 & \tabularnewline\hline
  四年 & -58 & \tabularnewline
  \bottomrule
\end{longtable}

\subsection{五凤}

\begin{longtable}{|>{\centering\scriptsize}m{2em}|>{\centering\scriptsize}m{1.3em}|>{\centering}m{8.8em}|}
  % \caption{秦王政}\
  \toprule
  \SimHei \normalsize 年数 & \SimHei \scriptsize 公元 & \SimHei 大事件 \tabularnewline
  % \midrule
  \endfirsthead
  \toprule
  \SimHei \normalsize 年数 & \SimHei \scriptsize 公元 & \SimHei 大事件 \tabularnewline
  \midrule
  \endhead
  \midrule
  元年 & -57 & \tabularnewline\hline
  二年 & -56 & \tabularnewline\hline
  三年 & -55 & \tabularnewline\hline
  四年 & -54 & \tabularnewline
  \bottomrule
\end{longtable}

\subsection{甘露}

\begin{longtable}{|>{\centering\scriptsize}m{2em}|>{\centering\scriptsize}m{1.3em}|>{\centering}m{8.8em}|}
  % \caption{秦王政}\
  \toprule
  \SimHei \normalsize 年数 & \SimHei \scriptsize 公元 & \SimHei 大事件 \tabularnewline
  % \midrule
  \endfirsthead
  \toprule
  \SimHei \normalsize 年数 & \SimHei \scriptsize 公元 & \SimHei 大事件 \tabularnewline
  \midrule
  \endhead
  \midrule
  元年 & -53 & \tabularnewline\hline
  二年 & -52 & \tabularnewline\hline
  三年 & -51 & \tabularnewline\hline
  四年 & -50 & \tabularnewline
  \bottomrule
\end{longtable}


\subsection{黄龙}

\begin{longtable}{|>{\centering\scriptsize}m{2em}|>{\centering\scriptsize}m{1.3em}|>{\centering}m{8.8em}|}
  % \caption{秦王政}\
  \toprule
  \SimHei \normalsize 年数 & \SimHei \scriptsize 公元 & \SimHei 大事件 \tabularnewline
  % \midrule
  \endfirsthead
  \toprule
  \SimHei \normalsize 年数 & \SimHei \scriptsize 公元 & \SimHei 大事件 \tabularnewline
  \midrule
  \endhead
  \midrule
  元年 & -49 & \tabularnewline
  \bottomrule
\end{longtable}


%%% Local Variables:
%%% mode: latex
%%% TeX-engine: xetex
%%% TeX-master: "../Main"
%%% End:
