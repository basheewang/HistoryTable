%% -*- coding: utf-8 -*-
%% Time-stamp: <Chen Wang: 2019-12-16 11:37:46>

\section{楚汉之争\tiny(BC206-BC203)}

\subsection{简介}


楚汉战争,或稱楚漢相爭,是秦朝滅亡后,项羽和刘邦之间为争夺统治权力而进行的战争。時間一般認定為前206年-前202年,秦朝滅亡之后開始,一直到項羽於烏江邊自刎結束。楚汉战争结束了秦末民變之后短暂的分裂局面,是继秦灭六国之后的又一次中国统一战争。

秦二世元年(公元前209年壬辰年),陳勝、吳廣在大澤鄉起義,天下群雄并起反秦。在此期間,陳勝的部將周章由陳縣略地至戲水附近,吳廣部下周文更是率軍打到離咸陽只有數十公里的戏地。同时,六國舊貴族也趁機反秦復國,武臣重建趙國、韓廣重建燕國,楚國貴族項梁、項羽殺死會稽郡守殷通,率領八千江東子弟兵加入起義,北上渡江作戰。不久,吳廣因內鬨被殺;陳勝也兵敗於滎陽,被叛變的車伕莊賈斬首。其後范增進言「楚雖三戶,亡秦必楚」,項梁就以重建楚國為口號,立楚懷王之孫熊心為王,也稱為「楚懷王」,史家稱為楚後懷王。同時,泗水亭亭長劉邦殺死沛縣縣令而造反,自稱「沛公」(沛縣首領),率部投奔項梁。

公元前208年,項梁在取得对秦軍的连胜后驕傲輕敵,宋義諫,但項梁不聽,導致定陶之戰战敗,項梁戰死。此战后,秦将章邯及王離认为楚军元气大伤,已构不成实质威胁,遂调转枪口,率秦軍40萬北上攻趙國,圍趙王歇於鉅鹿。楚後懷王見項梁兵敗,由盱眙遷都到彭城。趙國屢屢求救,於是楚後懷王派遣宋義为上将军、項羽和范增为副将,率軍6萬北上救趙,同時派劉邦進攻關中,楚後懷王還許諾「先入定關中者王之。」

公元前207年,宋义率军到达战场附近,但宋义懦弱无能,且想观秦赵相斗,逡巡逗留46天不进,贻误了宝贵的战机。項羽见此情形,果断殺死宋義,夺得军权,率軍破釜沉舟,與秦軍爆發鉅鹿之戰,項羽先派英布截斷秦軍糧道,再親率以楚軍为主的諸侯盟軍与秦军决戰。楚军视死如归,九战九捷,大败秦军,俘秦主帥王離,副帥涉間自焚,秦勇將蘇角被项羽亲自擊殺。

同時,劉邦領軍向西攻擊,先攻昌邑,在缺少攻城器具情況下,虽有彭越協助,仍然攻城不下。不久,酈食其投奔劉邦,建議他去攻擊陳留以奪取糧草,攻擊成功。劉邦率軍在中原戰鬥時,項羽正在追擊章邯,章邯當時糧食已缺,派別將司馬欣往求糧,被拒絕後陳餘又送信來,使得章邯猶豫不決。不久,項羽攻克三戶津,截斷章邯退路,章邯派偏軍於漳水南岸戰項羽,大敗。章邯再派軍戰於汙水(漳水的支流),又敗。章邯因得不到秦二世跟趙高的支援被困,后见大势已去,被迫在洹水南殷墟(今河南安阳)率秦軍20萬投降。巨鹿之战后,项羽声威大振,联军会师后,各路起义军将领被他的气势所震慑,甚至不敢与他直接对视。项羽由此被各路诸侯推举为上将军,统兵40万,继续反秦事业。

刘邦来到陈留西约30公里的开封,与秦大将赵贲大战,大破赵贲,三月,砀郡长刘邦自开封向北约100公里至白马擊敗楊熊奪取白馬,再降南陽,佔領武關,在藍田再破秦軍,逼近咸陽,趙高派人偷偷與劉邦商議平分關中被拒絕。同時,二世三年八月,趙高殺害秦二世,擁立子嬰為秦王,但子嬰九月即殺了趙高,取得大權。西元前207年11月,劉邦攻抵灞上,子嬰出降,秦帝国只经过了短暂的15年统一后便灭亡了。劉邦派軍據守函谷關,阻止項羽聯軍入關。不久,項羽以及諸侯40万大軍攻克函谷關,直入關中,駐兵新豐鴻門。此时刘邦只有10万兵马,只得将关中拱手相让。鴻門宴上,范增欲殺劉邦,但劉邦言詞甚懇切,項羽顧及與劉邦的結義之情與諸侯的威脅,放歸劉邦。

西元前206年2月,项羽尊楚后怀王为楚义帝,徙义帝于江南,都郴,後來又刺殺了義帝。西元前206年3月,项羽自封为西楚霸王,楚國疆域為梁楚九郡(今華東、江南大部分地區),建都彭城(江苏徐州),又分封十八路諸侯。刘邦被封为汉王,辖汉中、巴、蜀一带,建都南郑(今陕西汉中)。关中故秦之地三分,封章邯为雍王、司马欣为塞王、董翳为翟王。项羽分封诸侯后即罢兵裁軍,東归彭城。

齐國相田榮因拒隨從率军入关,故而没有得到分封。项羽把齐地一分为三,把齐王田市徙为胶东王,引起田荣对项羽封王方案极为不满,率軍驱逐齐王田都,并阻止齐王田市任胶东王。田市惧怕项羽,欲自行去胶东就封。田荣怒,追斩田市,自立為齊王。隨後,田荣再击杀济北王田安,兼并三齐之地。

赵国舊大将军陈馀与國相張耳在钜鹿之战結怨而去职,將兵權讓給张耳,归隐南皮。项羽分封时闻陈馀贤,与张耳一体有功,但未从入关,因此仅将南皮附近三县封与陈馀。陈馀见张耳为王,而自己功劳与张耳相当,却仅得南皮三县,对此十分不满。田荣兼并三齐,陈馀派夏说为使者使齐借兵。田榮又借兵給陳餘,令陳餘擊敗張耳,重迎赵王歇复为赵王。赵王歇感念陈馀,封陈馀为代王。赵王歇弱,因此陈馀不归代国,自稱趙國太傅,繼續輔赵,陈馀派夏说以代国國相身分,留守代国。

项羽分封时虽据梁國为己有,但梁地时有義軍首領彭越有众万余,彭越未从入关,故亦无分封,无所属。田荣除资助陈馀外,亦封彭越為齊國將軍,令其在梁地起兵破楚。项羽派萧公角击梁,彭越大破楚军。而在齐赵北方,故燕王韩广亦不愿徙王辽东,项羽所封之燕王臧荼击杀韩广,并王辽东。齊趙之反,使得楚國受到威脅,於是項羽決定北伐田榮。

汉元年八月,项羽闻刘邦袭取关中,于是杀韩王成,立亲信故吴令郑昌为韩王以拒汉。初项梁立韩公子成为韩王,张良为韩相国。项羽封诸侯,韩王成仍为韩王,但是项羽以张良从汉王刘邦,而韩王成又无功,故不许韩王成就国。项羽东归,韩王成被带至彭城。

项羽分封後数個月,形势就陷入一片混乱。项羽所确定的新秩序基本上被打破。同时项羽攻齐,向九江徵兵,英布不从,引发项羽不满,项羽数遣使责英布。

在齐地田荣兼并三齐之时,刘邦在汉中也为攻袭三秦做准备。刘邦入汉中,項羽給予刘邦三萬士兵。刘邦依张良计,入南鄭時燒毀棧道,以防被偷襲和向項羽示意無外侵的意願。

项羽帐下的执戟郎中韩信亦在此时從項羽軍中逃出,投靠劉邦,但没有被重用,仅任连敖,后坐法当斩为滕公夏侯婴所救。夏侯婴与之交談,知其有才能,向刘邦推荐韩信,刘邦拜韩信为治粟都尉。韩信自觉不能受到重用,欲离去另寻明主,萧何听闻后連夜苦追,人稱「萧何月下追韩信」故事。萧何再次向刘邦推荐韩信,刘邦拜韩信为大将军,统领三军。

公元前206年八月,劉邦用韓信的計謀,但受阻於陈仓,所幸趙衍指出一條小道。劉邦沿著這條小道進入關中,趁項羽北攻田榮時,突然出现在三秦旧将面前,在好疇擊潰章邯,最後圍困章邯於廢丘。隔年塞王司马欣、翟王董翳被迫向汉王刘邦投降。之后几个月,劉邦率領漢軍攻取隴西、北地、上郡。这样,三秦除章邯困守的废丘之外全部归汉。

而此时因项羽杀韩王成,张良间行归汉,派人遗书项羽,称“漢欲得關中,如約即止,不敢復東。”项羽以故無西意,而北擊齊。

刘邦略取关中时,九月,命令薛歐、王吸出武關,與王陵聯合,迎接劉太公和呂后於沛。十月,汉王拜韩王信为韓國太尉,令其循韩地,并许之若定韩地则拜其为韩王。韩信循韩地,下十余城,项羽所立之韩王郑昌降,汉二年十一月汉立韩王信为韩王。

汉二年十月,汉王刘邦进至陕(今河南陕县)。在汉基本平定关中之后,开始准备东进了。

公元前205年三月,劉邦東攻,迫降塞王司馬欣、翟王董翳、河南王申陽。劉邦渡臨晉,魏王豹率兵跟從,破河內,虜獲殷王司馬卬,南渡河,直抵雒陽。劉邦聞項羽命英布刺殺楚義帝,為義帝發喪,號召天下王公反抗項羽。

劉邦部将张良、陈平、韩信、吕泽、张耳、夏侯婴、樊哙以及五诸侯军,至外黄,击败楚将程处、王武,彭越率三万人归附刘邦,刘邦封彭越为魏国国相,攻打梁地,派樊哙北上攻打邹县、鲁县、薛县、瑕丘,以阻止项羽从齐国南下,向东攻打下邑、派吕泽驻守,下邑在萧县西面不远,萧县在彭城西面不远,项羽南下救援彭城必经萧县,这样,如果项羽回援彭城,吕泽可以与刘邦东西两面夹击项羽。与北路军曹参、灌婴会合,进攻砀县、萧县,攻取彭城。

项羽虽击败齐军,殺死田荣,但项羽暴虐,使齐地降而复叛。田榮之弟田橫立田榮之子田廣為齊王,繼續抗楚,项羽因此深陷齐地而无暇抽身。四月,劉邦與諸侯聯軍號稱五十六萬人,趁虚直捣楚都彭城。此时彭城由项羽军师范增守备,面对敌众我寡的不利态势,范增决定不死守彭城。数周后,在给予汉军以极大杀伤后,范增主动率大量楚军有生力量撤出彭城。

虽只得了一座空城,但以刘邦为首的各诸侯却以为他们取得了大胜,开始麻痹大意起来。入城后,联军日夜欢饮,军纪败坏,戒备松弛。項羽听闻这一情况,决定出奇兵制胜。他亲率精锐骑兵三萬人疾进,令士卒衔枚,马蹄裹布,由曲阜經胡陵,到蕭縣,於清晨攻擊漢軍,至中午大破漢軍。韓信等各路漢軍敗退至谷、泗水,被殲十餘萬人。楚軍一刻不停,繼續追殺,漢軍敗走,於靈壁東邊的睢水上被楚軍驅趕下河,汉军士兵光淹死的就有十數萬,使得睢水被屍體阻塞,河水一度断流。 此即公元前205年的彭城之戰。此战中,西楚霸王项羽充分利用汉军自以为人数众多,麻痹大意的弱点,只率三万精兵便击溃刘邦五十六万大军,是为中国乃至世界战争史中以少胜多之典范。

美中不足的是,彭城之战,项羽虽大胜,但田横亦复定三齐。

刘邦大败于彭城,劉邦收集散兵到下邑。父親劉太公、母親劉媼和妻子呂雉被楚軍擄為人質。諸侯見劉邦潰敗後,重新投奔項羽,連塞王司馬欣和翟王董翳也入楚為將。汉王刘邦以杀死张耳向赵国太傅陳餘結盟,陳餘发现汉王刘邦并没有杀张耳,赵兵退去反与汉为敌,汉王刘邦联盟顿时瓦解。时吕后兄周吕侯将兵居下邑。

张良亦至下邑,与刘邦于下邑规划下一步对策。张良说:“九江王黥布,楚梟將,與項王有郄;彭越與齊王田榮反梁地:此兩人可急使。而漢王之將獨韓信可屬大事,當一面。即欲捐之,捐之此三人,則楚可破也。”刘邦采纳张良建议。

劉邦在彭城之戰後,勢力一落千丈。刘邦在下邑收集散兵后,到達虞(今河南虞城县),派随何出使九江,随何成功游说九江王英布投汉。项羽不得不派龙且分兵攻打英布,牵制了项羽的后方,後汉军往彭城东南边的灵壁溃散,刘邦向西撤退,首先去了下邑与周吕侯吕泽会合,然后向南接应败兵,在砀县驻扎,战败后刘邦及时稳住阵脚,防止了大军继续溃散。刘邦驻守下邑时,马上吸引了项羽亲自来进攻,彭城之战一个月之后,公元前205年五月,刘邦回到关中。漢軍水攻廢丘,雍王章邯在抵抗了十個月后兵敗自殺,劉邦攻占三秦分兵滅魏,劉邦同時北上滅趙。同時,英布與龍且戰爭,不得勝利,與隨何往見劉邦,楚军尽取九江。 劉邦收取士卒,會合關中蕭何派來的援軍,加上韓信率領殘兵敗卒赶来会合,項羽率領楚军亦追击而至,劉邦指揮諸軍並且讓灌嬰率領汉军騎兵於“京县”(今河南郑州荥阳豫龙镇京襄城村附近)、“索亭”(今河南滎陽索河街道)之間擊敗項羽統領的楚軍,将楚軍击退到滎陽以东。此為京索之戰。

京索之战,汉军稳住阵脚,楚军也无力突破汉军防线进攻关中。双方从此开始在荥、成一带拉锯,战争进入相持阶段。

京索之战后,刘邦回到栎阳,进行整顿,立刘盈为太子。时关中爆发饥荒,刘邦令关中民移民至汉中、巴、蜀。接着刘邦再次来到荥阳前线。八月,汉分兵,韩信、曹参率军伐魏,後劉邦北上击赵;刘邦与韩信在襄国会合,杀赵王歇。周勃、召欧等继续平定恒山、巨鹿、燕 国,燕王臧荼降汉,项羽派楚将争夺赵国,唐厉在武城打败楚军。

韩信请刘邦封张耳为赵王,刘邦同意。

漢二年(前205年),魏王豹藉口親人有病,重返魏地,重新歸順項羽,封鎖蒲阪,起兵反對劉邦。劉邦派出酈食其遊說,不成,派韓信曹參攻打魏國。不久,魏王豹駐兵蒲阪,堵塞臨晉。曹参以代理左丞相的身份分别与韩信各率军向东攻魏国,在东张(秦汉东张县包括蒲坂、临晋关,所以魏军主力应该在临晋关一带防守“魏王盛兵蒲坂,塞临晋”)大败孙遫的军队。曹参率军大败魏王亲自率领的军队,魏豹逃跑到武垣,被汉军活捉。取平阳,得魏王母妻子,尽定魏地,凡五十二城。劉邦赐曹參食邑平阳,安邑之战后不久,韩信與张耳被派往趙地,汉三年九月(前204年),韩信與曹參先破代兵,生擒代国丞相夏说于阏与,但刘邦很快就收回军队主力。

劉邦命令韓信與張耳繼續率兵經井陘攻打趙國。趙王歇與陳餘在井陉口部署二十萬重兵,企图阻止漢軍北上。赵将军李左車建議派三萬士兵給他,截斷韓信的糧道,陳餘率軍在前線防守,使韓信进退两难,不出十日,定能斬殺韓信和張耳。但陳餘懦弱且迂腐,自恃兵精粮足,堅持正面迎战,拒絕使用李左車的計謀。韓信得知陳餘不用李左車之計,就放心率軍出井陘,並派出两千輕騎兵,命他们手持漢軍旗幟,準備在趙軍倾巢出擊後,立刻攻佔其營寨,插上漢軍旗幟。不久,韓信在绵蔓河畔設背水陣,詐敗誘敵,陈馀中计,大笑韩信不懂兵法,命趙軍猛攻背水陣。漢軍知道已无退路,人人拼死作戰,趙軍不能打敗漢軍,只好撤退,但漢軍两千騎兵已經攻佔敵營,插上漢旗,回撤的趙軍见状,一溃千里。漢軍乘勝夾擊追殺,大破趙軍,斬殺陳餘於泜水上,俘虜趙王歇、李左車。此为汉三年十月(属前204年)。

与此同时,刘邦亦亲攻赵。汉将靳歙兵出河内,击赵将贲郝于朝歌,破之。又随刘邦进击安阳以东,下七县;别将攻赵军,虏两司马,得赵军二千四百余人。接着刘邦对赵之邯郸发起进攻,破赵军,攻下邯郸。汉将靳歙破赵军于平阳,攻下邺。这样赵国悉平。刘邦与韩信在襄国会合,杀赵王歇。周勃、召欧等继续平定恒山、钜鹿、燕国,燕王臧荼降汉,项羽派楚兵攻擊赵国,唐厉在武城打败楚军。

汉灭赵之战的过程是,韩信与曹参受刘邦的命令,先破代国,杀夏说,把赵国的注意力转向赵国北部,派韩信与张耳在井陉设疑兵,利用地理优势吸引赵军主力,刘邦自己亲自率兵,趁虚直取邯郸。当赵国失去邯郸,襄国危急,陈馀进退两难,此时韩信与张耳出井陉,攻杀了陈馀。赵王歇逃到襄国,刘邦与张耳、韩信南北夹击襄国,攻破襄国会合,杀赵王歇,平定赵国。后来汉军又平定了巨鹿、常山郡,招降了燕国。韩信张耳继续平定赵国余寇,刘邦、靳歙、周勃、曹参等返回敖仓,此前英布被龙且与项声打败,与随何归汉,此时英布正式归降刘邦,这时汉营调走他旗下的兵到荥阳抵抗楚军。

韓信當時向東攻擊,尚未渡過平原津時,劉邦已經派出酈食其往齊國說齊王。韓信於是打算停止進兵。但由范陽來的蒯通認為酈食其一介書生,竟能憑三寸不爛之舌說下齊國七十多座城池,恐怕韓信的功勞比不上酈食其。於是韓信跟從他的計謀,襲擊齊國。當時齊王見劉邦已經派出酈食其,所以安心下來,把歷下的守軍撤走。 韓信率兵攻打历城,齐国叛汉,攻打楚国后方的靳歙与丁复不得不停止进攻,回到前线,刘邦派灌婴、曹参、陈武等支援韩信,攻打齐国 灌婴、曹参等到达齐国,攻下历城,齐王逃往高密,向项羽求救,项羽派龙且援齐汉四年十一月,陈武军,蔡寅军,丁复军,王周军,陈涓等汉军杀齐王田广、龙且,齐相田横自立为齐王,灌婴在嬴县打败田横,田横投奔彭越,曹参留在齐国继续平定齐国頑軍。

同時,項羽見龍且敗死,遂派武涉去遊說韓信投楚或中立,蒯通也建議韓信應該自立門戶,但全部被韓信拒絕。於是武涉離開,蒯通裝瘋逃去。韓信遂坐镇齐地

汉三年十二月,韩信与张耳留下在赵国继续跟余寇作战,刘邦返回荥阳。此前英布被随何策反,项羽派龙且項聲攻打英布,英布战败,与随何回到了荥阳。刘邦召见英布,派英布重返九江,收聚数千人归汉,刘邦也离开荥阳,从成皋南下,到宛县(今河南南阳)、叶县一带迎接英布,给英布增兵,一起回到成皋。

刘邦据守荥阳,开始修筑甬道,由敖仓运输粮食来荥阳。与项羽对峙,双方进入相持状态。

刘邦为解除荥阳相持的僵局,命令靳歙击断楚军从荥阳至襄邑的粮道,命令灌婴击断了楚军从阳武至襄邑的粮道,离开荥阳,攻打楚国后方的二号大本营:鲁城,并留下御史大夫周苛、魏王豹、韩王信、枞公等人守荥阳。刘邦与灌靳二将攻打鲁城的时候,汉三年八月,项羽猛攻荥阳,负责守荥阳的御史大夫周苛以魏王豹是反复之人,难与共守城,杀了魏豹。随后项羽便攻破了荥阳,杀了御史大夫周苛、枞公,俘虏了韩王信,又攻下了成皋。

刘邦得知荥阳已失守,便命令靳歙等攻打楚国的后方,后来靳歙攻下缯、郯、下邳,蕲、竹邑,几乎包围彭城,同时刘邦自己与灌婴回前线,在燕县打败楚将王武,在白马津打败楚将桓婴,渡过白马津,至河内,南渡黄河回到洛阳。

此时项羽攻下了成皋,进军至巩县,汉军与楚军在洛阳东边的巩县交战,楚军战败,项羽不能继续西进,楚军退至成皋,据险坚守,汉军攻之不下,一时无法夺回成皋。刘邦产生了放弃攻打成皋、退守巩县与洛阳的念头,郦食其劝刘邦不要退却,向刘邦说明敖仓的重要性,即放弃成皋与荥阳意味着放弃敖仓。郦食其说:“楚人拔荥阳,不坚守敖仓,乃引而东,令适卒分守成皋。楚军为什么要这样做?因为此时靳歙、丁复、傅宽等正在扫荡楚国的后方,项羽不得不分兵攻打他们解后方之急,故不能全力守成皋与敖仓。我方务必夺回成皋与荥阳,并坚守敖仓,取得战略上的优势,向诸侯昭示天下形势。”郦食其并自请出使齐国,劝说齐王降汉。刘邦接受了郦食其的建议。

刘邦兵败彭城,彭越亦亡其所下之城,率军北居河上,往来为汉游兵击楚,绝楚军粮道。汉三年五月(203年),彭越率军渡睢水,与楚将项声、薛公战于下邳(今江苏邳州古邳镇),大破楚军,杀薛公。

汉三年九月,刘邦采用郎中郑忠之策,派将军刘贾、卢绾将卒二万人、骑数百,由渡白马津,进入楚地佐助彭越。汉军与彭越联军烧掉楚军积聚的粮草,楚军乏食。楚军回击刘贾,刘贾坚守不出不与楚军交战,与彭越互相呼应。

接着彭越攻梁地,下睢阳、外黄等十七城。项羽无奈,只好留大司马曹咎守成皋,嘱令不与出战。然后东击彭越,迅速拿下陈留、睢阳、外黄等地。彭越敗北走谷城。

汉四年秋,项羽南走阳夏,彭越趁势下昌邑旁二十余城,得谷十余万斛送给汉军。汉五年冬十月,彭越得到刘邦立其为梁王,王睢阳以北至谷城,于是出兵助汉。十一月,汉将刘贾渡淮围寿春,楚大司马周殷叛楚,以舒屠六,举九江兵迎英布,并行屠城父,亦进至垓下。

韩信拒绝武涉游说以及蒯通建议后,自己坐镇齐国,同時劉邦命令灌婴对楚直接发起进攻。

灌婴率汉军首先进攻楚的鲁地,大破楚将薛公杲于鲁北。南下再破薛郡长,攻博阳,进军至下相,夺取取虑、僮、徐等县。接着渡过淮河,进至广陵(今江苏扬州),尽降楚之城邑。但项羽很快派项声、薛公、郯公夺回淮北。灌婴回师复渡淮,在下邳大破项声、郯公军,斩薛公,夺取下邳。接着追击楚军,破楚军于平阳(南平阳,今山东邹城市),回师还攻并占领彭城,虏楚柱国项佗降留、薛、沛、酇、蕭、相。攻苦、譙,再次俘获亚将周兰。

灌婴平淮北后,与刘邦军会师于颐乡(位于今河南鹿邑县)。

彭越、刘贾袭扰楚軍后方,使楚军补给无法保障,前方士疲粮绝。此时灌婴所部尽略楚地,夺取彭城,虏楚柱国项佗。刘邦趁机派陆贾為使者与项羽進行和谈,遭项羽拒绝。之后刘再次派出侯生出使楚國,終議和成功,即:楚、汉两家鴻溝和約,约定中分天下,双方以鸿沟为界,以东属楚,以西属汉,此稱「楚河漢界」; 并放还刘邦家人,止紛爭,罷干戈。

和約訂立后,项羽如约放还刘邦家人。刘邦封侯生为“平国君”,但侯生本人却功成隐退。

张良、陈平建议撕毁鸿沟和议,趁楚军疲师东返之机自其背后发动追歼。张、陈二人认为:“汉有天下太半,而诸侯皆附之。楚兵罢食尽,此天亡楚之时也”,建议“不如因其机而遂取之”。刘邦遂聽张良、陈平的建议,趁楚军锐气消磨殆尽的退兵路上发起追击。彭越趁项羽向南撤退到阳夏之机,攻克昌邑旁二十多个城邑,缴获谷物十多万斛,用作汉王的军粮。刘邦亦趁機率军发起追歼,於汉五年十月(按:这时以十月为岁首)击败項羽親率楚军取得阳夏(今河南太康),樊哙虏楚大将周将军卒四千人。

刘邦率领汉军追击项羽至固陵,项羽为了摆脱汉军,发动反击攻打刘邦。刘邦在各路汉军没到的情况下,为避免不必要的损失,选择高壁深垒防守战,為接下来的反击战做准备。刘邦另派刘贾南渡淮水包围寿春,刘贾很快到达,派人寻找机会招降楚大司马周殷。周殷叛变楚王,帮助刘贾攻下九江,迎着武王黥布的军队在垓下会合,断项羽向南逃后路,並共同攻打项羽。不久,灌婴、靳歙率领骑兵军团从彭城往固陵而来,刘邦亲自在固陵东边颐乡与灌婴率领的汉军铁骑会合。项羽得知灌婴、靳歙等率领汉军东来后,为防自己被包围往南退守至陈下,刘邦在灌婴、靳歙率领精锐骑兵到来后,发动反攻。汉将宣曲侯义率领骑兵和汾阳侯靳强率汉军为先锋,攻固陵楚军,便击破了楚大将钟离昧的部队,揭开了陈下之战的序幕。

陳下之战,楚方是项羽亲率的主力部队,有大将锺离眜,还有属于楚的陈公(陈县令)利几等;汉方是刘邦率领出成皋追击之军,有周勃、樊哙、靳歙等将领,加上破彭城后与刘邦会合的灌婴,还有刚来投降的楚将灵常。刘邦亲率汉军从西北方来,灌婴从东方来,对驻陈的楚军形成东西夹击合围之势。交战的结果,楚军大败,陈公利几向汉方投降,汉军大胜。

项羽战败后,率残兵败将逃跑。刘贾已策反楚大司马周殷,驻守在城父,周殷以舒县的兵力屠戮了六县,与英布一同北上攻打项羽,项羽立即调转马头,转向东南方逃跑,刘贾在城父堵截项羽,此前楚军一败再败,一无粮草,二无后援,军心战心,项羽无心恋战,逃往垓下。

刘邦的大部队迅速追上,刘贾也离开城父追击项羽,周殷与英布一同追击项羽,把项羽包围在垓下。

前203年,项羽依照鸿沟和约率兵东归。刘邦则派人赴齐招韩信,韩信坐山观虎斗,拒不与刘邦会合。汉五年十一月,项羽离开荥阳向南逃跑,刘邦率军追击,在阳夏南打败楚军,追至固陵,刘邦止步,派刘贾攻取寿春,策反项羽的大司马周殷,攻下九江,陈下大战后与英布会合包抄项羽,刘邦追项羽到陈县,与灌婴会合,大败楚军,楚将灵常、陈公利几降汉军。项羽继续逃跑,欲前往会稽。

彭越攻下昌邑二十余城,刘贾、周殷、英布攻下城父堵截项羽,项羽逃到垓下,刘邦与刘贾、周殷、英布会合。韩信看到项羽大势已去,也前往垓下与刘邦会合。彭越也来垓下,与刘邦合兵一处。公元前202年12月,10万楚军在垓下之战中被60余万汉军打败,被围在垓下(今安徽灵璧县东南)。当日夜晚,漢軍命士兵唱起楚地歌谣,楚军官兵听到四面楚歌,误以为汉军已渡江,

项羽自刎后,劉邦罷韓信兵權,令灌婴率军渡江,破吴郡长于吴下(今江苏苏州),得吴守,斬首八萬級,略定吴、豫章、会稽。复還定淮北,凡五十二縣,楚地略定。

临江王共尉(共敖之子)忠於項羽,不肯降漢。汉遣卢绾、劉賈别将攻之,久攻不下,靳歙还兵攻临江,下之,俘临江王共尉,被殺於洛阳。楚将陈公利几以陈郡降汉,不久,利幾懷疑劉邦欲誅之,於潁川反,劉邦乃率軍親征,利幾兵敗被殺。

项羽死后,楚國舊地皆向刘邦投降,只有鲁國不降,刘邦亲率大军想杀死他们,但考虑到鲁人守礼义,为君主死节,于是拿项羽的首級给魯國長老們看,鲁國的父兄才投降。楚义帝曾封项羽为鲁公,项羽死后又是鲁最后投降,于是刘邦下令以鲁公之礼葬项羽于穀城,并亲为发丧,洒泪而去。

项羽之宗族皆赦而不诛,究其前功,而各封为列侯。历时四年的楚汉战争,終以刘邦取得天下,建立西汉而告终。一年後,劉邦開始消滅異姓王。

\subsection{年表}



\begin{longtable}{|>{\centering\scriptsize}m{2em}|>{\centering\scriptsize}m{1.3em}|>{\centering}m{8.8em}|}
  % \caption{秦王政}\
  \toprule
  \SimHei \normalsize 年数 & \SimHei \scriptsize 公元 & \SimHei 大事件 \tabularnewline
  % \midrule
  \endfirsthead
  \toprule
  \SimHei \normalsize 年数 & \SimHei \scriptsize 公元 & \SimHei 大事件 \tabularnewline
  \midrule
  \endhead
  \midrule
  高祖\\元年 & -206 & \begin{enumerate}
    \tiny
  \item 秦朝灭亡。
  \item 鸿门宴。
  \item 项羽建立西楚王朝,自称西楚霸王。
  \end{enumerate} \tabularnewline\hline
  二年 & -205 & \begin{enumerate}
    \tiny
  \item 彭城之战。
  \item 成皋之战。
  \item 韩信破代、赵。
  \item 韩信灭燕、齐。
  \end{enumerate} \tabularnewline\hline
  三年 & -204 & \begin{enumerate}
    \tiny
  \item 背水一战。
  \item 南越国建立。
  \item 成皋之战。
  \end{enumerate} \tabularnewline\hline
  四年 & -203 & \begin{enumerate}
    \tiny
  \item 英布封王。
  \item 张耳封王。
  \end{enumerate} \tabularnewline
  \bottomrule
\end{longtable}


%%% Local Variables:
%%% mode: latex
%%% TeX-engine: xetex
%%% TeX-master: "../Main"
%%% End:
