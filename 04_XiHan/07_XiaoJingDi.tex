%% -*- coding: utf-8 -*-
%% Time-stamp: <Chen Wang: 2019-12-16 14:06:01>

\section{孝景帝\tiny(BC156-BC141)}

\subsection{生平}

汉景帝刘启(前188年-前141年3月9日),为西漢第六位皇帝(前157年7月14日—前141年3月9日在位),在位16年,享年48岁,其正式諡號為「孝景皇帝」,後世省略「孝」字稱「漢景帝」,景帝後元三年正月甲子(前141年3月9日)崩于未央宮,二月癸酉(3月18日)葬于阳陵(今陕西高陵县西南)。为汉文帝刘恒長子,母竇皇后。他在位期间,主要是削诸侯封地,平定七国之乱,巩固中央集权,勤俭治国,发展生产,他统治时期和他父亲文帝统治时期合称文景之治。

刘启为汉文帝刘恒长子。刘启出生时,父亲刘恒为代王,母亲窦姬为妾。前180年,呂太后駕崩,陳平、周勃等人誅滅諸呂,立刘恒为帝,是為汉文帝。嫡母代王王后早已逝世,而她所生的世子也在此时病死,刘启成为父亲刘恒事实上的长子。同年正月,刘啟以刘恒长子的身份被立为太子。其后,母亲窦姬亦被立为皇后。

刘启為太子時性格剛烈,因與吳國太子劉賢下圍棋(一說六博)發生爭執,而拿棋盤打死了劉賢,漢文帝敕命送遗體回去埋葬,到了吳國,劉賢的父親吳王劉濞大怒,說道:「天下都是劉家的,死在長安就埋在長安,何必回吳國埋葬!」遂又把遗體送回長安埋葬,以示對朝廷的不滿,從此刘濞怨恨刘启。

前157年7月6日(六月己亥),汉文帝崩于长安未央宮,7月14日(丁未),皇太子刘启即位,是为汉景帝。

景帝前元三年(前154年),御史大夫晁错建议削藩,景帝听从,引起那些早就想反叛的诸侯王们的不满,于是以吴王刘濞、楚王刘戊为首的七国之乱开始了。七國诸侯王以“诛晁错,清君侧”为藉口叛乱,欲夺天下。晁错政敌袁盎献策景帝,诛晁错以平叛乱。景帝“嘿然良久,曰:‘顾诚何如,吾不爱一人以谢天下。’”。于是有丞相青翟等一众臣子劾奏晁错。景帝“制曰:‘可’”,令中尉以上朝议事为名,誘晁错上朝,行中错道,至东市,中尉宣汉景帝刘启诏书,当场腰斩晁错。但晁错死後,七国之乱不但没有停止,反而越演越烈,占领了不少土地。景帝无可奈何,只得派太尉周亞夫、竇嬰镇压,前後三個月即平定七国之乱。

七国之乱以后,景帝开始专心打理朝政,据说景帝十分樸素,仁厚爱民。除了平定七国之乱以外,从来没有大规模用过兵,和匈奴的战争始终控制在一定的规模内,依然对匈奴采取和亲政策。

前元六年(前151年),皇后薄氏被废。第二年(前150年),废太子刘荣。同年四月,立王氏为皇后,随后立王氏的独子胶东王刘彻为太子。

景帝崇尚黃老之說,减少刑罚,减少赋税,兴修水利,提倡農業,要求人心不服的案子进行重审,以免冤狱发生。百姓在和平稳定的环境下创造了大量财富,其间百姓富裕,丰衣足食,安居乐业,天下太平安乐,一派盛世景象,与其父汉文帝统治时期并称文景之治。

景帝后元三年正月甲子(前141年3月9日),景帝崩于未央宫,遺詔賜予諸侯王與列侯駿馬兩匹、吏二千石、黃金兩斤,吏民戶百錢;又命放出一批宮人,使其歸家再嫁。景帝享年48歲,諡號孝景皇帝,无廟號,二月癸酉(3月18日),葬于阳陵。景帝崩後由皇太子刘彻即位,是为汉武帝。

司马迁在《史记》中评价:“至孝景,不复忧异姓,而晁错刻削诸侯,遂使七国俱起,合从而西乡,以诸侯太盛,而错之不以为渐也。”

班固在《汉书》中评价:“孝景遵业,五六十载之间,至于移风易俗,黎民醇厚。周云成康,汉言文景,美矣!”

唐代司馬貞在《史記索隱》中評價景帝以德待臣子、鼓勵耕作,面對吳楚之叛,引領將領翦除逆賊。但是平叛的周亞夫,受到景帝的猜忌,倉促下獄,這對於一個治理國家的明君來說,是很可惜的:『【索隱述贊】景帝即位,因脩靜默。勉人於農,率下以德。制度斯創,禮法可則。一朝吳楚,乍起凶慝。提局成釁,拒輪致惑。晁錯雖誅,梁城未克。條侯出將,追奔逐北。坐見梟黥,立翦牟賊。如何太尉,後卒下獄。惜哉明君,斯功不錄!』

\subsection{前元}

\begin{longtable}{|>{\centering\scriptsize}m{2em}|>{\centering\scriptsize}m{1.3em}|>{\centering}m{8.8em}|}
  % \caption{秦王政}\
  \toprule
  \SimHei \normalsize 年数 & \SimHei \scriptsize 公元 & \SimHei 大事件 \tabularnewline
  % \midrule
  \endfirsthead
  \toprule
  \SimHei \normalsize 年数 & \SimHei \scriptsize 公元 & \SimHei 大事件 \tabularnewline
  \midrule
  \endhead
  \midrule
  元年 & -156 & \tabularnewline\hline
  二年 & -155 & \tabularnewline\hline
  三年 & -154 & \tabularnewline\hline
  四年 & -153 & \tabularnewline\hline
  五年 & -152 & \tabularnewline\hline
  六年 & -151 & \tabularnewline\hline
  七年 & -150 & \tabularnewline
  \bottomrule
\end{longtable}


\subsection{中元}

\begin{longtable}{|>{\centering\scriptsize}m{2em}|>{\centering\scriptsize}m{1.3em}|>{\centering}m{8.8em}|}
  % \caption{秦王政}\
  \toprule
  \SimHei \normalsize 年数 & \SimHei \scriptsize 公元 & \SimHei 大事件 \tabularnewline
  % \midrule
  \endfirsthead
  \toprule
  \SimHei \normalsize 年数 & \SimHei \scriptsize 公元 & \SimHei 大事件 \tabularnewline
  \midrule
  \endhead
  \midrule
  元年 & -149 & \tabularnewline\hline
  二年 & -148 & \tabularnewline\hline
  三年 & -147 & \tabularnewline\hline
  四年 & -146 & \tabularnewline\hline
  五年 & -145 & \tabularnewline\hline
  六年 & -144 & \tabularnewline
  \bottomrule
\end{longtable}


\subsection{后元}

\begin{longtable}{|>{\centering\scriptsize}m{2em}|>{\centering\scriptsize}m{1.3em}|>{\centering}m{8.8em}|}
  % \caption{秦王政}\
  \toprule
  \SimHei \normalsize 年数 & \SimHei \scriptsize 公元 & \SimHei 大事件 \tabularnewline
  % \midrule
  \endfirsthead
  \toprule
  \SimHei \normalsize 年数 & \SimHei \scriptsize 公元 & \SimHei 大事件 \tabularnewline
  \midrule
  \endhead
  \midrule
  元年 & -143 & \tabularnewline\hline
  二年 & -142 & \tabularnewline\hline
  三年 & -141 & \tabularnewline
  \bottomrule
\end{longtable}


%%% Local Variables:
%%% mode: latex
%%% TeX-engine: xetex
%%% TeX-master: "../Main"
%%% End:
