%% -*- coding: utf-8 -*-
%% Time-stamp: <Chen Wang: 2021-10-29 17:25:34>

\section{孝文帝劉恆\tiny(BC179-BC157)}

\subsection{生平}

漢文帝劉恆(前203年-前157年7月6日),劉邦第四子,母薄姬,漢惠帝之庶弟。西漢第五位皇帝(前180年11月14日-前157年7月6日在位),在位23年,享年46歲。其廟號太宗,正式諡號為「孝文皇帝」,後世省略「孝」字稱「漢文帝」。葬於霸陵(在今陝西省西安市灞桥区白鹿原东北角)。漢文帝也是《二十四孝》中親嘗湯藥的主角。

漢王四年(前203年)漢王劉邦於成臯召幸薄夫人,有身孕,当年就生下文帝,高祖十一年春(前196年)年八岁封為代王,其為人寬容平和,在政治上保持低調。

呂后在殺害劉邦愛姬戚夫人和其子趙王劉如意後,提議代王劉恆改封趙王,然而劉恆巧妙地謙讓了,故而才能夠在呂后專權時期得以保命。

呂后專權,大封呂姓子弟為異姓王,呂產、呂祿等呂姓外戚強勢,高祖长孙齐王刘襄实力强大,率先起兵;陳平、周勃等元老、列侯和宗室劉章等人以計謀騙來呂氏外戚的兵权,夷滅諸呂,也打算把呂氏血緣掃清,要廢除漢後少帝,擁立新皇帝。

琅琊王刘泽等指刘襄的舅舅驷钧为恶人,立刘襄等于复立一个吕氏。因為劉恆時年二十四歲,是劉邦當時現存年紀最長之子,又寬厚孝順,而且劉恆之母薄氏的家族比較不強勢。陳平、周勃等大臣隨即請劉恆入長安即位,劉恆遍問代國眾臣意見,郎中令張武認為此事有詐,中尉宋昌卻覺得此為天賜良機,劉恆只好與母親薄姬商議此事,薄姬也不知如何是好,遣其弟薄昭前往長安與朝廷大臣們協商,在九月的最後一日,劉恆奔赴長安,群臣三呼萬歲,繼天子位,太僕夏侯嬰親自拘捕漢後少帝劉弘,迎接劉恆。大臣們宣稱劉弘、梁王劉泰、淮阳王刘武、恆山王刘朝等非惠帝子,废黜刘弘并将四人杀害。

漢文帝即位後,勵精圖治,興修水利,衣著樸素,廢除肉刑,使漢朝進入強盛安定的時期。當時百姓富裕,天下小康。漢文帝與其子漢景帝統治時期被合稱為文景之治。

汉文帝十年(前170年),汉文帝舅薄昭因故杀死了朝廷使者,被赐自尽。

前157年7月6日(六月己亥),漢文帝崩於長安未央宮。乙巳(7月12日),葬霸陵。

漢文帝與其兒子漢景帝統治時期合稱為文景之治,奉為賢明皇帝的典範。此外,漢文帝在位時,存在諸侯王國勢力過大及匈奴入侵中原等問題。汉文帝对待这些问题采取的是异常谨慎而且又有效的手法。对待诸侯王,文帝采取以德服人的态度,小错不纠,在中央弱势的时候成功安抚了各地蠢蠢欲动的诸侯,为后来漢景帝处理七国之乱造就了一批忠心耿耿的诸侯王和大臣。最重要的两个大动作是:安抚吴王,使得吴王在最年富力强的时候没有假借丧子之仇反叛;在齐王死后将齐国一分为七,既满足了齐王的儿子们称王的需求,为自己赢得了贤德之名,又消除了最大的一个诸侯国齐国。假如文帝的谨慎稳重的做法被一直持续下去,汉朝也就不会发生后来的七国之乱,诸侯王问题亦有希望能夠和平解决。

文帝與其妻子竇皇后,其子景帝都愛好黃老之術,假托黃帝老子思想,以道家的清靜無為為治世方法,文景兩世與民休息,輕刑罰減賦帑,文帝本身亦恭行仁孝,生活質樸簡約,是中國歷史上少有的賢君之一。

西漢末年的儒家学者劉向曾經回答漢成帝詢問,評價文帝“本修黄、老之言,不甚好儒术,其治尚清净无为,以故礼乐庠序未修,民俗未能大化,苟温饱完给,所谓治安之国也”、“(訟獄)治理不能過中宗(漢宣帝)之世”、“似不及中宗之世,不可以為昇平”,但认为文帝能礼待劝谏者,“文帝礼言事者,不伤其意,群臣无小大,至卽便从容言,上止辇听之,其言可者称善,不可者喜笑而已”。

道德方面,文帝亦曾經親自為母親薄氏嘗藥,深具孝心,在《二十四孝》裏被排在第二位。

漢文帝重用鄧通,“賞賜通巨萬以十數”,又將蜀地的銅山賜給鄧通,准許他任意鑄銅錢,史稱「鄧氏錢布天下,其富如此」。因为邓通曾經在文帝在位时得罪过太子刘启,所以文帝驾崩之后,即位的汉景帝刘启罢免邓通,并且没收其全部家产,最后邓通因为穷困而饿死在蜀地(传说是雅安)。

\subsection{前元}

\begin{longtable}{|>{\centering\scriptsize}m{2em}|>{\centering\scriptsize}m{1.3em}|>{\centering}m{8.8em}|}
  % \caption{秦王政}\
  \toprule
  \SimHei \normalsize 年数 & \SimHei \scriptsize 公元 & \SimHei 大事件 \tabularnewline
  % \midrule
  \endfirsthead
  \toprule
  \SimHei \normalsize 年数 & \SimHei \scriptsize 公元 & \SimHei 大事件 \tabularnewline
  \midrule
  \endhead
  \midrule
  元年 & -179 & \tabularnewline\hline
  二年 & -178 & \tabularnewline\hline
  三年 & -177 & \tabularnewline\hline
  四年 & -176 & \tabularnewline\hline
  五年 & -175 & \tabularnewline\hline
  六年 & -174 & \tabularnewline\hline
  七年 & -173 & \tabularnewline\hline
  八年 & -172 & \tabularnewline\hline
  九年 & -171 & \tabularnewline\hline
  十年 & -170 & \tabularnewline\hline
  十一年 & -169 & \tabularnewline\hline
  十二年 & -168 & \tabularnewline\hline
  十三年 & -167 & \tabularnewline\hline
  十四年 & -166 & \tabularnewline\hline
  十五年 & -165 & \tabularnewline\hline
  十六年 & -164 & \tabularnewline
  \bottomrule
\end{longtable}


\subsection{后元}

\begin{longtable}{|>{\centering\scriptsize}m{2em}|>{\centering\scriptsize}m{1.3em}|>{\centering}m{8.8em}|}
  % \caption{秦王政}\
  \toprule
  \SimHei \normalsize 年数 & \SimHei \scriptsize 公元 & \SimHei 大事件 \tabularnewline
  % \midrule
  \endfirsthead
  \toprule
  \SimHei \normalsize 年数 & \SimHei \scriptsize 公元 & \SimHei 大事件 \tabularnewline
  \midrule
  \endhead
  \midrule
  元年 & -163 & \tabularnewline\hline
  二年 & -162 & \tabularnewline\hline
  三年 & -161 & \tabularnewline\hline
  四年 & -160 & \tabularnewline\hline
  五年 & -159 & \tabularnewline\hline
  六年 & -158 & \tabularnewline\hline
  七年 & -157 & \tabularnewline
  \bottomrule
\end{longtable}


%%% Local Variables:
%%% mode: latex
%%% TeX-engine: xetex
%%% TeX-master: "../Main"
%%% End:
