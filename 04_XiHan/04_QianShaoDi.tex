%% -*- coding: utf-8 -*-
%% Time-stamp: <Chen Wang: 2019-12-16 13:58:50>

\section{前少帝\tiny(BC187-BC184)}

\subsection{生平}

西汉前少帝(?-前184年6月15日),名不详,前188年9月27日-前184年6月15日在位,西漢第三任皇帝,史稱「前少帝」。

历史记录中未记载其名字,“刘恭”一名可能来自近代日本历史著作中“刘某”一名的误写。 钱穆《国史大纲》作刘恭。

刘强,前187年6月6日被立为淮阳王,前183年8月或9月去世,谥怀,史称淮阳怀王。劉不疑,前187年6月6日被立为恒山王,次年去世,谥号哀,史称恒山哀王。刘弘,本名劉山,前187年6月6日封为襄城侯;后于前184年8月21日改封恒山王,并改名刘义;前183年6月20日即帝位後,改名劉弘,前180年11月14日被诛杀。劉朝,前187年6月6日封为轵侯,後于前183年6月20日改封恒山王,前180年11月14日被诛杀。劉武,前187年6月6日封为壶关侯,後于前183年8月或9月改封淮陽王,前180年11月14日被诛杀。劉太,前184年3月14日封为昌平侯;後于前181年4月1日改封济川王,前180年10月4日徙封梁王,前180年11月14日被诛杀。

汉惠帝年间,皇太后吕雉命令皇后张氏假装怀孕,收養美人所生子为己子,立為太子,並且殺死生母。前188年,漢惠帝死,太子登基,就是汉前少帝。前187年為漢前少帝元年。

前少帝登基后,其名义上的母后也是事实上的嫡母张皇后,未得称皇太后。吕雉虽为少帝祖母,仍以皇太后(非太皇太后)身分临朝听政,并分封吕姓诸王,极力扩展吕氏势力。前184年,少帝发现身世真相,扬言等年长了必要报杀母之仇,吕雉得知後,囚禁他於永巷(即宫廷監獄),對外聲稱皇帝重病,拒絕接見任何人。吕雉又對朝臣說小皇帝重病,無法治理國家,應有人接替,得到朝臣肯定。于是,少帝被廢黜,並被吕雉幽禁殺害,改立常山王劉弘為帝,即「後少帝」。

前少帝為吕后傀儡,且後陳平、周勃、灌嬰、劉章、夏侯嬰等眾大臣誅殺諸吕後,認為惠帝並無生子,包括前少帝、後少帝在内的惠帝諸子,血統都有問題,来歷不明,並非皇子(無劉氏皇族血統),故少帝不被漢朝認可為正统君主。

\subsection{年表}


\begin{longtable}{|>{\centering\scriptsize}m{2em}|>{\centering\scriptsize}m{1.3em}|>{\centering}m{8.8em}|}
  % \caption{秦王政}\
  \toprule
  \SimHei \normalsize 年数 & \SimHei \scriptsize 公元 & \SimHei 大事件 \tabularnewline
  % \midrule
  \endfirsthead
  \toprule
  \SimHei \normalsize 年数 & \SimHei \scriptsize 公元 & \SimHei 大事件 \tabularnewline
  \midrule
  \endhead
  \midrule
  元年 & -187 & \tabularnewline\hline
  二年 & -186 & \tabularnewline\hline
  三年 & -185 & \tabularnewline\hline
  四年 & -184 & \tabularnewline
  \bottomrule
\end{longtable}


%%% Local Variables:
%%% mode: latex
%%% TeX-engine: xetex
%%% TeX-master: "../Main"
%%% End:
