%% -*- coding: utf-8 -*-
%% Time-stamp: <Chen Wang: 2019-12-16 11:34:52>

\section{汉高祖\tiny(BC206-BC195)}

\subsection{生平简介}


汉高帝劉邦(前256年或前247年-前195年6月1日),字季,是中國第一位有廟號及諡號的皇帝。沛丰邑中阳里(今江蘇徐州丰县)人。秦末汉初的軍事家、政治家。刘邦从沛县起兵反秦,被萧何、曹參、樊哙等人拥立,自稱沛公,后投奔楚項梁,以“先入定关中者为王”之约,破武关,秦王子婴降。秦朝灭亡后,项羽分封入关诸侯有功者,封刘邦為汉王。楚汉战争中,劉邦擊敗項羽獲勝,統一自秦亡後的天下,於西元前202年稱帝,史称西漢,為漢朝开国皇帝,駕崩於西元前195年,在位7年,死后庙号太祖,谥号高皇帝,史稱汉高帝。太祖為其正式廟號,而史書多稱呼「漢高祖」。

刘邦出生于前256年或前247年。刘邦少年时仰慕信陵君,但未及成年,信陵君便亡故。之後信陵君舊門客張耳为外黄令,招徕门客,劉邦因而到大梁往投张耳门下為門客,並與之結交為友。不久魏国滅亡,张耳不愿事秦而改名換姓逃亡,劉邦亦返回家鄉。劉邦登帝位後,只要途經大梁,多半會祭祀信陵君;在楚漢戰爭期間,张耳为常山王,被他的故友陈馀所败,张耳为此故投奔刘邦。後劉邦在秦都咸陽服徭役時見到秦始皇出遊,發出了:「嗟乎,大丈夫當如此也」的感歎。

劉邦早年有一名外婦曹氏,生下長子劉肥,但並未正式娶妻,故劉肥並非嫡子。之後單父縣的門閥呂公因避仇而移居沛縣,县內士紳豪傑,皆往祝賀。主吏蕭何負責排定門客的座次,要求賀禮不到一千銅錢的客人,都坐在堂下。亭長刘邦認為沛縣諸官吏也沒甚麼了不起,就自稱獻「賀錢一萬」,其實一個銅錢都沒有帶來。呂公看到劉邦的面相後大吃一惊,覺得他将来定是個不凡人物,因此引入堂內就座。蕭何告訴呂公:「劉邦只會說大話,沒甚麼成就。」但吕公不以为然。劉邦坐在上賓座位後,就大聲調侃其他沛縣官吏。呂公對劉邦說:「我很會看面相,但是沒看過像你這麼相貌不凡的,我有個女兒(即呂雉),希望你願意接受她當你的妻子。」事後呂公的妻子呂媼很生氣,說:「你以前說你這個女兒很難得,一定要嫁個非常好的丈夫。沛縣縣令對你這麼好,你還不肯把女兒嫁給他,現在居然要把她嫁給劉邦?」呂公說:「這不是你婦道人家懂的事。」最后還是把女兒嫁給劉邦。呂雉後來生了漢惠帝劉盈和魯元公主。

劉邦为亭长时,經常請假回家到田裡去。有一次呂雉和孩子正在田中除草,有一老者經過討水喝,呂雉讓他喝了水,給他飯吃。老者給呂雉相面說:「夫人是天下貴人。」呂雉又讓他給兩個孩子相面,見了劉盈,說:「夫人尊貴的原因,是因為這個男孩子。」又給魯元公主相面,同樣也是富貴面相。老者走後,劉邦正巧從旁邊的房舍走來,呂雉就把剛才那老人給她們看相的情況,告訴了劉邦。劉邦問這個人在哪,呂雉說:「還不遠。」於是追上了老者,問他剛才的事,老者說:「剛才的夫人、孩子的樣貌都像您,您的相貌贵不可言。」劉邦道謝說:「真的像您說的,我一定不忘您的恩德。」等到劉邦顯貴的時候,卻再也找不到當初那位老者。

前210年十月劉邦奉命押解犯人到骊山,途中有不少人逃脫,因為當時讓犯人逃脫是重如死罪,所以劉邦索性放走所有人,劉邦也因此逃亡,當時逃犯中有十餘人願意跟隨他一同逃亡,也成為未來起義的部分勢力。一行人路遇一條大白蛇擋路,劉邦一怒之下就提劍把蛇斬殺了,突然出現一個老婦人啼哭,自稱:“我兒是白帝之子,化成白蛇躺在路上,卻被赤帝之子殺死了。”隨即消失。由於秦始皇的先祖秦襄公說自己是白帝的後裔,眾人都認為劉邦被賦予取代秦朝的天命,是為“斬蛇起義”。秦始皇常說「東南有天子氣」,乃东游欲以厭勝之。劉邦自認始皇东游是針對他,于芒碭山山澤落草為寇。

秦二世元年七月(前209年),陳勝與吳廣因為下雨誤期,害怕受到秦朝法律處罰,率領約九百名役夫在蕲县大泽乡(今安徽宿州东南刘村集)起事,是為大澤之變。当时許多郡县的仕紳杀死郡守、縣令,以响应陈胜,沛縣縣令恐懼,欲响应陈胜,于是主吏萧何、狱掾曹参等劝縣令召回刘邦。縣令答应,于是派樊哙往召刘邦。劉邦至沛,而縣令反悔,于是刘邦率約百人於沛縣城外射箭夾信,說服城內人誅殺縣令。劉邦多次推让后被众人立為沛縣縣令,自稱沛公,徵發縣中約三千子弟,攻占沛县等地。刘邦起事后,攻胡陵(江苏省沛县龙固镇)、方与,还守丰。秦泗水郡监「平」将兵围丰,为刘邦军所破。接着刘邦命雍齿守丰,亲率军攻薛,泗水郡郡守「壮」战败逃到戚(今山東微山),不久为沛公左司马曹無傷所杀。刘邦还军亢父。至方与,魏國丞相周市来攻方与(今山东鱼台),雍齿佔据丰邑,歸降周市,刘邦攻丰邑,不能攻下。

时东阳宁君、秦嘉在留邑(今沛縣東南)立景駒为王,刘邦欲往投奔,并借兵再度攻擊丰地,路上遇到了张良。章邯的偏将司马枿往北方進兵,拿下了楚地,在相地屠城,進兵砀郡。东阳宁君与沛公率军,向西迎击,在萧邑之西與司马枿交戰,不利,刘邦退回留县休整,獨自出兵击敗司马枿獲得勝利,使秦军退往砀县东边,三天攻下砀县,再以周勃先登攻克下邑,回師聚兵于留邑,接着三天攻取砀郡,得兵五、六千人,此时刘邦約有九千士卒,接連攻取下邑,还军攻丰邑,不能攻下。

时楚國起義軍首領项梁在薛城,沛公刘邦率骑兵百餘人前往跟隨,项梁給予刘邦士卒五千人、五大夫等級的将領十人,刘邦反攻丰,拔之。雍齿逃亡魏国。

在得知楚王陳勝确实已死的消息,以及秦嘉立景驹为楚王,项梁為爭取楚國父老的民心,把在外诸将召还薛城,共立楚懷王之孫熊心,亦稱「楚懷王」,史家稱楚后怀王,项梁自命為武信君。六月,章邯破杀魏王魏咎、齐王田儋于临济。七月,刘邦攻亢父。章邯包围田荣于东阿。刘邦随项梁率军前往救援,打败了章邯。刘邦及项羽继续追击秦军至城阳,屠城,驻扎在濮阳东,再次和章邯军交战,打敗了秦军。章邯军再次聚集,守濮阳、环水。刘项联军离开,去攻定陶。八月,定陶無法攻下,往西攻擊到雍丘,与秦军战,大破秦兵,五大夫曹参從刘邦攻杀三川郡郡守李由,李由地位顯赫,是秦丞相李斯之子。项羽、沛公还师攻外黄,未下。

九月,项羽、刘邦攻陈留时,闻定陶之役中章邯擊殺项梁,士卒惊恐。刘邦、项羽及吕臣徏楚怀王从盱台迁都彭城。吕臣军驻彭城东,项羽驻军彭城西,刘邦驻军砀城。

秦二世二年闰九月(前208年九月),楚后怀王迁都彭城,并将项羽、吕臣等诸军的兵卒收归自己率领,命沛公刘邦为砀郡首长,封武安侯,率领砀郡的军队。又封项羽长安侯,建國於魯,号称“鲁公”。吕臣为司徒,其父吕清为令尹。时秦兵围赵甚急,赵国数向楚王请求救兵。楚王以宋义为上将军,项羽为次将,范增为末将,北上救赵。而沛公率军西攻秦兵。怀王与将军们约定,谁先进入关中,就可以在关中称王,但当时秦兵强大,常乘胜逐北,诸将都认为西入关攻秦没有好处,都不愿去,只有项羽因为叔父项梁之死,非常有意愿入关攻秦。但楚后怀王及其他诸将都认为项羽为人不可取,为了报叔父之仇一定会屠城,不让项羽参加西征,于是沛公独领军西征。

次月,即秦二世三年十月(前208年十月,秦制以十月为岁首),刘邦率军西征,收项梁、陈胜散卒,由砀郡到达成阳,与杠里的秦军僵持,击败了王离所率之秦军,接着在成武南攻王离及东郡尉,大破之,揭开了楚军反攻的序幕,并为宋义军的前进,扫除前方障碍做了贡献。十二月,还至栗,遇到楚军的将领“刚武侯”(封号,姓名不详),刘邦夺取了“刚武侯”的部队,收编了四千余人,与魏国将军皇欣、申徒武蒲之军合攻昌邑,未下,只好绕道至高阳。同期王离败给刘邦,后来听章邯之令去围钜鹿,项羽跟诸侯联军在钜鹿大破秦军,俘获王离,招降章邯。

秦二世三年二月(前207年二月),在昌邑遇彭越,与彭越军合攻昌邑,未下。在高阳,郦食其、郦商兄弟来投奔,郦食其劝说刘邦袭陈留,掠夺秦兵的粮食,因此封郦食其为广野君,然后刘邦与郦商一起攻击开封,却攻不下,只好继续向西前进,三月刘邦来到陈留西约30公里的开封,与秦大将赵贲大战,大破赵贲 在白马、曲遇大破秦将杨熊,杨熊逃到荥阳,秦二世遣使斬殺杨熊示众。四月,刘邦攻打颍阳,并破颍阳,透过张良,又占领了韩国的轘辕关。四月刘邦拿下轘辕后,刘邦率领军由轘辕攻进了洛阳盆地。这时候,赵国的别将司马昂正想渡过黄河,进入函谷关。刘邦就向北拿下平阴,截断黄河渡口,向南进军,与秦军在洛城東方戰鬥,再次大破赵贲大军。

六月,与南阳郡守吕齮交战,攻取南阳郡,吕齮逃到宛城,刘邦想要放弃追击,但张良说如此将会腹背受敌,一定要先破宛城,刘邦于是攻击宛城,吕齮本来要自刎,他的舍人陈恢建议投降,吕齮答应了,刘邦封吕齮为“殷侯”,封陈恢食邑一千户,然后刘邦继续西进,所经过的城纷纷归顺。到了丹水的时候,秦国的高武侯戚鳃、襄阳侯王陵也在西陵投降了,刘邦于是回过头来攻打胡阳,遇到了番君吴芮的副将梅鋗,就跟梅鋗一起拿下了析县和郦县。刘邦派遣甯昌出使秦地,甯昌还没回来。这时,钜鹿之战结束,秦将章邯向项羽投降。

赵高刺杀了秦二世之后,派人向刘邦说,愿意割地给刘邦,刘邦认为有诈,而且趙高奸詐不可信,非但不答应,還處死了秦使者,同時加快攻击的脚步,用了张良计谋,派郦生与陆贾去游说、行贿武关的秦将,却乘机偷袭武关。刘邦攻武关之后,八月,刘邦在经过连续机动后,攻破秦国的东南门户,位于丹水河谷的险要武关,秦王子婴即位,隨即刺殺赵高,並发动关中所有军队并派大军据守蛲关、蓝田之战中刘邦击败秦军最后一支大军,入秦。秦廷大为震动,刘邦最后抵达霸上。子婴驾素车白马于轵道投降,刘邦反对众将的建议,不愿处死子婴,只把他俘虏而已。

汉元年十月(前207年十月),秦王子婴于轵道向刘邦投降,秦朝灭亡。史家一般以此月为「汉元年」开始。

刘邦見到秦國皇宮中富裕堂皇,想要入住秦宫中享受榮華富貴,为樊哙、张良所谏阻。于是刘邦乃下令封閉秦王的王宫府库,还军霸上,而萧何等则收了秦朝之地图、戶籍資料等。刘邦召見咸陽附近的父老、豪杰,慰勞他们说:你們忍受秦朝苛法已經很久了,然后与他们约法三章,把苛法全部廢除,并令吏人仍守旧职。同时也拒绝了秦人犒劳。刘邦此举,大得秦人民意,唯恐刘邦将来不为秦王。

有人告诉刘邦说:「秦國是六國的十倍富裕,地形易守難攻。聽說章邯投降項羽,項羽要讓章邯到關中稱王,章邯一來,你就沒得稱王了,趕快守住函谷關,不要讓诸侯進兵。」刘邦采纳提議,命人守函谷关。十一月中旬,项羽率诸侯联军进至函谷关,闻刘邦已定关中并派人守住了关口,大怒,下令黥布等攻破函谷关,十二月,至戏(今陕西西安),欲攻刘邦。而刘邦军的左司马曹无伤派人向项羽告密,项羽更怒。而项伯因与张良是好友,往刘邦军欲勸告张良离去,不要被亂軍殺死,张良因项伯求项羽罢兵。因此发生了鸿门宴故事。鸿门宴中,項羽未聽從亞父范增之計,使得劉邦逃過一劫。(有名的俗諺“項莊舞劍,意在沛公”即出於此。)

項羽進入咸陽,處死子嬰,劫掠財寶,火燒阿房宫(根據考查,應為咸陽宮),自立为西楚霸王,领有梁國、楚國九郡,定都彭城,儼然天下共主,分封群雄。項羽為了困住劉邦,违背怀王所定“先入定关中者王之”之约,假稱巴蜀亦屬關中,改立劉邦为漢王,领有巴、蜀、漢中三郡四十一县,定都南郑。然后三分關中,封章邯为雍王,司马欣为塞王,董翳为翟王,史称三秦,其余群雄各有参差。汉王刘邦对此次分封不满,想要进攻项羽,被丞相萧何阻止。

項羽立楚後懷王為天子,號稱「楚義帝」,讓義帝成為傀儡,但不久之後,命令黥布刺殺了義帝,以報義帝不遣他入關的仇怨。項羽殺義帝,加上分封無法服眾,有實力的軍人們紛紛起兵叛變,劉邦趁項羽出外平亂,暗渡陈仓,出兵關中,平定三秦,甚至一度攻佔項羽的根據地彭城。

經過楚漢之間長期的拉鋸戰,在眾多漢軍文臣、武將如曹參、灌嬰、蕭何、靳歙、周勃、郦生、張良等人的協助下,劉邦所率領的漢軍逐漸坐大。楚漢兩國協議以鴻溝為界,鴻溝以西為漢,以東為楚,互不侵犯。但是,當項羽遵守諾言退兵,並放回曾被扣為人質的劉邦的父親劉太公、妻子呂雉之後,劉邦發動偷襲。項羽在陽夏之戰一路敗退,後在陳下之戰劉邦跟項羽展開一次主力大戰(陳下是今河南淮阳)項羽大敗後逃到垓下,韓信、彭越、英布等加入漢軍對項羽的包圍網之後,參戰兵力數倍於楚軍。

前202年12月,楚軍在垓下之战中被漢軍打敗,被圍在垓下。劉邦用四面楚歌之計瓦解楚軍軍心,最後項羽走投無路,自覺無顏見江东父老,只好自刎於烏江邊。這場歷時近五年〔高帝元年(前206年)四月至五年(前203年)十二月〕的楚汉战争,项羽彻底败亡而自杀,劉邦統一天下。

前202年初(汉四年十二月),项羽败亡。

漢五年(前202年)正月,追尊长兄刘伯为武哀侯,以楚義帝无后,徙齐王韩信为楚王,王楚地。魏相国彭越定梁地,拜为梁王。

諸侯及將相們共同尊請漢王劉邦為「皇帝」。劉邦說:「我聽說皇帝這種尊號,是極其賢能的人才能享有的,空言虛語不是我能接受的,我可承擔不了皇帝的尊號。」群臣們都說:「大王從一個小老百姓的身分起兵討伐暴逆,平定四海,有功的人便分賞土地,封王封侯,如果大王不稱帝,封賞就沒有信用,我們這群人願意以生命向大王請求。」劉邦辭讓再三,實在推辭不過了,說:「諸君都認為這樣才對,那就為了國家的便利,我答應罷。」

汉五年二月甲午,劉邦於泛水之陽即皇帝位,定都洛陽(不久遷至长安),定国号為漢,史稱西汉。並立王后呂雉為皇后,立太子劉盈為皇太子,追尊母为昭灵夫人。

漢朝開國,最初仅排定十八人之位次,分别为:萧何、曹参、张敖、周勃、樊哙、郦商、奚涓、夏侯婴、灌嬰、傅寬、靳歙、王陵、陈武、王吸、薛欧、周昌、丁復、蟲達。群臣都抱怨蕭何受封第一,甚為不悅,向劉邦抱怨道:「我們像狗、像馬一樣地作戰,在戰事中出生入死,冒著生命危險,受了一大堆傷,為甚麼蕭何一個文弱書生,拿個筆、動動嘴,居然居功第一?這是甚麼道理!」劉邦說:「你們懂得狩獵的道理嗎?知道獵犬的作用嗎?今天各位將軍就好像負責追蹤獵物的狗,但是發現獵物蹤跡,並指揮獵狗的人是蕭何。而且你們都只是一個人追隨我,但蕭何是全家親族都投效我,我不能忘記他的功勞。」

群臣又認為曹參軍功無數,應列第一,但劉邦仍較屬意蕭何。關內侯鄂君於是進諫:「群臣當然沒有錯,曹參的確有野戰略地之功,但楚漢相爭五年,失軍亡眾,喪失不少士兵,但是蕭何從關中派兵來補足士兵的不足,不是蕭何,漢軍早就折損所有兵卒了。楚漢在滎陽僵持時,漢軍面臨缺糧危機,蕭何利用關中的漕運,我們才都有糧草可用。雖然蕭何沒有隨軍出征,卻守護關中,給我們支援,形同隨時陪伴陛下一般,這應是萬世之功。我認為論功應該是蕭何第一,曹參第二。」高祖曰:「你說的沒錯。」於是乃令蕭何第一,賜給錦帶,允許蕭何可以配劍,穿鞋子上殿,入朝拜見時不需要小跑步。

眾將每天辯論誰功勞大,所以十八人以外,遲遲未論功行賞,同時劉邦在宮中看到一群將領們群聚議論,他詢問張良,將領們在商討何事?張良回答道:「陛下本來是個老百姓,當上皇帝,封賞的都是蕭何、曹參這些好朋友,誅殺的都是生平討厭的人。將軍們認為天下的土地不夠封,但很有可能被翻舊帳而被您殺死,所以將領們正在討論如何造反!」此話使得劉邦大為緊張,進一步詢問當如何解決,張良問劉邦:「陛下生平最痛恨,而將領們都知道的有誰?」劉邦答:「雍齒跟我是老朋友,但以前他經常譏辱我,使我難堪,我很想殺了他。不過他多次立功,我又不忍心殺。」張良說:「請陛下立刻重賞雍齒,眾臣見到,就都會放心了。」劉邦立即照做,此事傳開後,眾臣皆認為連皇帝最痛恨的雍齒都能獲賞,想必自己也不會過差,因此打消造反念頭。

刘邦自起兵起,滅亡秦朝,打敗项羽,以及后来消滅異姓王,至即位十二年(公元前195年),共封功臣143人。高后二年(公元前186年),吕后下诏让陈平序定其他功臣名次。

婁敬勸劉邦說:「定都關中。」劉邦尚對此心有疑慮。左右的大臣都是關東地區的人,多數勸劉邦定都洛陽,他們說周朝定都洛阳,拥有天下数百年;秦朝定都关中,到秦二世就灭亡了。洛阳位居“天下之中”,便于四面八方的物资供给,而且四周群山环绕,背靠邙山,东有成皋,西有崤函,背对黄河,面向伊洛之水,土地肥沃,地势险要,形势完固,足以设险守国。

張良說:「洛陽雖有此固,其中小,不過數百里,田地薄,四面受敵,此非用武之國也。夫關中左殽函,右隴蜀,沃野千里,南有巴蜀之饒,北有胡苑之利,阻三面而守,獨以一面東制諸侯。諸侯安定,河渭漕輓天下,西給京師;諸侯有變,順流而下,足以委輸。此所謂金城千里,天府之國也,婁敬之說是也。」於是劉邦當即決定起駕往西,定都關中,並拜婁敬為郎中,賜劉姓。

劉邦稱帝後,鑒於全國新形勢,感到:「三章之法,不足以御奸。」於是令萧何參照秦朝法律:「取其宜於時者,作律九章。」蕭何在保留《秦律》六章的基礎上,補充了《戶律》、《廄律》、《興律》三章,史稱《九章律》。

漢高帝五年(前202年),天下已經統一,諸侯們在定陶共同尊推漢王劉邦爲皇帝,叔孫通負責擬定儀式禮節。當時劉邦把秦朝的那些嚴苛的儀禮法規全部取消,只是擬定了一些簡單易行的規矩。可是群臣在朝廷飲酒作樂爭論功勞,醉了有的狂呼亂叫,甚至拔出劍來砍削庭中立柱,劉邦爲這事感到頭疼。叔孫通知道皇帝愈來愈討厭這類事,就勸說道:「夫儒者難與進取,可與守成。臣原徵魯諸生,與臣弟子共起朝儀。」劉邦說:「得無難乎?」叔孫通說:「五帝異樂,三王不同禮。禮者,因時世人情為之節文者也。故夏、殷、周之禮所因損益可知者,謂不相複也。臣原頗采古禮與秦儀雜就之。」劉邦說:「可試為之,令易知,度吾所能行為之。」

于是叔孫通奉命征召了魯地儒生三十多人。魯地有兩個儒生不願走,說:「公所事者且十主,皆面諛以得親貴。今天下初定,死者未葬,傷者未起,又欲起禮樂。禮樂所由起,積德百年而後可興也。吾不忍為公所為。公所為不合古,吾不行。公往矣,無汙我!」叔孫通笑著說:「若真鄙儒也,不知時變。」

叔孫通就與征來的三十人一起向西來到都城,他們和皇帝左右有學問的侍從以及叔孫通的弟子一百多人,在郊外拉起繩子表示施禮的處所,立上茅草代表位次的尊卑進行演練。演習了一個多月,叔孫通說:「上可試觀。」皇帝視察後,讓他們向自己行禮,然後說:「吾能為此。」于是命令群臣都來學習,這時正巧是十月,能進行歲首朝會的實際排練。

漢高帝七年(前200年),長樂宮已經建成,各諸侯王及朝廷群臣都來朝拜皇帝參加歲首大典。當時的朝儀是:天亮時,由謁者掌禮,來訪者依次進入殿門。宮中設有車騎、步卒守衛,以及兵器、旗幟等。殿上傳言:「趨」,殿下郎中俠陛,陛數百人入殿。功臣、列侯、將軍及其他軍官在西列隊,向東而立;文官自丞相以下在東列隊,向西而立。大行依爵位高低宣示來賓上殿。於是皇帝乘輦出房,百官手執幟而傳警,引諸侯王以下至領六百石薪金的吏員依次奉賀。這時,自諸侯王以下,各人無不肅然起敬。禮成後開始酒會,宮內侍從坐在殿上,全部伏下,以來賓尊卑依次敬酒。九觴酒後,謁者宣佈:「罷酒」。御史在場內執法,見到不依禮儀的人便立刻把他帶走。整個酒會過程中都沒有人敢喧嘩失禮。

大典之後,劉邦非常得意地說:「吾乃今日知為皇帝之貴也。」于是授給叔孫通太常的官職,賞賜黃金500斤。隨叔孫通入京的儒生獲漢高帝封為郎,另外叔孫通把賞賜所得全數分贈隨行的儒生。

在楚汉战争中,刘邦为了换取各路實力軍人的支持,以打敗项羽,故封韩信等人为諸侯王。这样在西汉建立之初,被封的异姓王共有數人,即齐王韩信(后徙为楚王)、梁王彭越、淮南王英布、韩王信、赵王张耳、燕王臧荼(后更立卢绾)、衡山王吴芮(后改为长沙王)。

王国的封地,多者一百多城,少者三四十县,总面积比朝廷直辖郡县还要多,而且各王都拥有軍隊,行政、财政自专,名为汉臣,实为汉不能控制之独立軍閥,对朝廷造成很大威胁。

汉高祖建国称帝时已过半百,身體多病,容易疑心,尤其是畏懼那些實力強大的异姓王们,各種削藩政策使得諸侯各自造反。结果,燕王臧荼最先起兵,兵败之后被捕殺、韓信先由齊王,改為楚王,劉邦假裝出巡楚國,卻將他劫持,貶為淮陰侯,留居京城,不讓他到外地任職,而陳豨於漢十一年(前196年)在代國造反,劉邦親征。舍人樂說告韓信欲攻皇后與太子以應陳豨,呂后採納蕭何之計殺掉韓信,夷三族。彭越被呂后誣以謀逆,被酼刑;韓王信、陳豨等败后叛逃匈奴,后战败被杀;英布起兵淮南,一開始聲勢甚猛,劉邦抱著病御駕親征,並且與曹參會師,夾擊滅了英布。燕王卢绾逃入匈奴。只有长沙国因為國家太小,才勉強得以保存,直到汉惠帝時才因為絕後而国除。

高帝十二年(前195年),刘邦曾杀白马为盟,除了宗室,不許以後再立異姓諸侯王,订下誓约:“非刘氏而王者,天下共击之。”这就是历史上的“白马之盟”。

劉邦還承認閩越的當地君長亡諸為王,但事實上閩越相當於藩屬國,所以未在消滅之列。

劉邦對其他的列侯亦起疑心,萧何刻意以低价强行购买平民的房地產,自毀名譽,來使高祖對他釋疑。高祖崩逝前因重病心憂下令陈平、周勃斬殺一直忠心耿耿的好友樊哙,樊哙是高祖的連襟,還在鴻門宴上救過高祖一命,只因樊哙是吕后之妹婿,高祖担心樊哙幫助諸呂作亂。後因樊哙是劉邦的老友,又是呂后的妹婿,陈平、周勃不敢妄動,高祖驾崩,樊哙才躲过一劫。

劉邦晚年寵幸愛戚夫人而疏遠了呂后,又認為呂后所生太子劉盈(漢惠帝)過於軟弱,多次想廢黜太子而改立戚氏之子趙王刘如意。

漢高帝十二年(前195年),劉邦隨著擊敗英布的軍隊,回长安,病勢更加沈重,愈想更換太子。张良勸諫,劉邦不聽,張良就託病,不再理事。太子太傅叔孫通進谏規勸道:「古時晋献公因骊姬之故,罷黜了太子,立奚齊,晋国動亂了數十年,被天下人所笑。秦朝以不早定扶苏為太子,令赵高得以詐立胡亥,使自己的宗廟滅亡,此陛下所親見。今太子仁孝,天下人皆聞之;呂后與陛下一起憂勞辛苦,難道可以背叛她!陛下必欲廢嫡長子而立少子,臣願意先伏誅,把脖子的血灑汙宮殿之地。」劉邦說:「公不必這麼說,我只是開開玩笑。」叔孫通說:「太子是天下的『根本』,『根本』一搖,天下震動,怎麼以天下開玩笑!」劉邦說:「吾聽公言。」但劉邦只是假裝答應了他,還是想更換太子。

而御史大夫周昌更與劉邦極力爭辯,劉邦問他理由何在,周昌有口吃的毛病,再加上非常激動,也就口吃得更厲害了,他說:「臣口不能言,然臣期期(期,通「極」)知其不可。陛下雖欲廢太子,臣期期不奉詔。」劉邦聽罷,很高興地笑了。呂后因爲在東廂側耳聽到上述對話,見到周昌時,就跪謝說:「沒有君,太子幾乎要被廢了。」

呂后很驚慌,不知如何保住太子的地位。有人對呂后說:「留侯善於出計策,陛下相信、任用他。」呂后就派建成侯呂澤脅迫留侯張良說:「君長久以來都是陛下的谋臣,现在陛下打算撤換太子,怎么能高枕無憂地睡大觉呢?」張良說:「当初陛下多次是处在危急之中,才幸而采用了臣的计策。如今天下已經安定,由于自己偏爱的原因想更换太子,这些是人家骨肉之间的家務事,即使跟臣一样的一百多個人一同进谏,又有甚麼用处呢?」呂澤竭力要挾說:「為我想想辦法罷!」張良說:「这件事是很难用口舌来争辩的。陛下有招致不到的四个賢者,这四个人已经年老了,都认为陛下对人傲慢無禮,所以躲在山中,他们依道义而行,不肯做汉朝的臣子。但是陛下很敬重这四个人。现在您果真能不惜錢財,让太子写一封信,言辞谦恭,并预备可以安適的馬車,再派有口才的人恳切地聘请,他们应当会来。来了以后,把他们当作贵宾,让他们时常跟着入朝,叫陛下见到他们,那么陛下一定会感到惊异并询问他们。一问他们,陛下知道这四个人贤能,这对太子是一個帮助啊。」於是呂后讓呂澤派人攜帶太子的書信,用謙恭的言辭和豐厚的禮品,迎請這四個人。四個人來了,就住在建成侯的府第中為客。

等到安閑的時候,設置酒席,太子在旁侍侯。那四人跟著太子,他們的年齡都已八十多歲,鬂眉潔白,衣冠非常壯美奇特。劉邦感到奇怪,問道:「彼何為者?」四個人向前對答,各自說出姓名,叫東園公、角裏先生、綺裏季、夏黃公(商山四皓)。劉邦於是大驚說:「我訪求諸公已好幾年了,諸公都逃避著我,現在諸公為何自願跟隨我兒交遊呢?」四人都說:「陛下輕慢士人,擅長的就是辱罵,我們追求義理,不願受人侮辱,所以惶恐地隱居起來了。我們私下得知太子為人仁義孝順,恭敬有禮,珍愛士人,天下人無不伸長脖子想為太子效死,所以我們就來了。」劉邦說:「要麻煩諸公好好調護太子到最後了。」

四個人敬酒祝福已畢,小步快走離去。劉邦目送他們,召喚戚夫人過來,指著那四個人給她看,說道:「我是想更換太子,但他們四個人輔佐太子,太子的羽翼已經形成,難以更換了。呂后真是你的主人了。」戚夫人哭泣起來,劉邦說:「你為我跳楚國的舞,我為你唱楚國的歌。」劉邦唱道:「鴻鵠高飛,一舉千里。羽翮已就,橫絕四海。橫絕四海,當可奈何!雖有矰繳,尚安所施!」劉邦唱了幾遍,戚夫人抽泣流淚,劉邦起身離去,酒宴結束。劉邦最終沒更換太子,臨終時還在牽掛劉如意戚夫人母子。戚夫人也在劉邦死後,遭呂后拔頭髮、斬去手腳,丟進廁所。

劉邦稱帝後,將士兵都遣散回家。下令各诸侯子弟留在關中的,免除賦稅、徭役十二年,回到封國去的免除賦稅徭役六年,國家供養他們一年。汉高帝十二年(前195年)二月,他又連下2詔,公告天下,表示朝廷有意輕徭薄賦。而各郡國對朝廷貢獻過多,於是下詔規定數額,並規定進奉日期是每年的10月。漢初實行的十五稅一制(田租為收成的1/15,约为6.67\%),更是輕徭薄賦政策的明顯例證。

凡民眾因為飢餓沒有糧食,自己賣身到富戶,成為奴婢者,皆免除其奴隸身分,改為庶人。

劉邦早年放蕩不羈,輕視儒生,稱帝以后,仍認為讀書無用。儒生陸賈在劉邦面前必言《诗經》、《書經》。劉邦破口大罵說:「你老子我,騎在馬背上得了天下,怎麼要我去鑽研那些《詩》、《書》?」陸賈據理力爭地說:「在馬背上取得天下,難道也在馬背上治理天下麼?商湯和周武王,都是以武力取得天下,然後政治上順應形勢以文治守成,文治與武力並用,這才是長治久安的好方法啊。從前吳王夫差、智伯都是因窮兵黷武而使國家滅亡;秦朝也是一味使用嚴刑峻法而不知變通,最後導致秦二世的滅亡。假使秦朝統一天下之後行仁義,效法古代聖人,那陛下又怎麼取得天下呢?」劉邦雖然不太開心、有點尷尬,但還是對陸賈說:「試著為我寫下,秦國失天下的理由,我之所以得天下的理由,還有古代國家興亡的歷史。」於是命陸賈論述國家興衰存亡的征兆和原因,共寫十二篇。每寫完一篇就上奏給劉邦,劉邦無不稱贊,左右群臣皆高呼萬歲,他稱這部書爲《新語》。

後來劉邦因為平定英布叛亂回途路經山東,還準備祭品,親自祭祀了孔子。

漢高帝五年(前202年),開始興建都城長安,當時以秦朝的興樂宮為基礎,修建長樂宮,做為皇宮。漢高帝七年(前200年),長樂宮建成,劉邦從洛陽遷都長安,並在長樂宮行叔孫通所制定的漢朝朝儀。

漢高帝八年(前199年),丞相蕭何主持修建未央宮,並於長樂宮和未央宮之間修建武庫,又於長安東南修建太倉。

劉邦看到宮殿非常壯觀,很生氣的對蕭何說:「天下匈匈苦戰數歲,成敗未可知,是何治宮室過度也?」蕭何說:「天下方未定,故可因遂就宮室。且夫天子四海為家,非壯麗無以重威,且無令後世有以加也。」劉邦這才信服。

秦亡以後,漠北的匈奴乘機南下騷擾漢朝北方邊境。漢高帝六年(前201年),韓王信投降匈奴。漢高帝七年(前200年),在北伐匈奴時,劉邦打退冒顿又接受婁敬之策,以宗室女假稱長公主,遠嫁冒頓單于,開始了與匈奴的和親政策。

劉邦在討伐英布前,就已得病,之后和英布交战时,又為流矢所中,回到长安后病况危急,不欲見人,躺在皇宮之中,诏令守門武士不得讓群臣進入。群臣中如周勃、灌嬰等人都不敢進宮。這樣過了十多天,樊哙推開宮門,闖了進去,後面群臣緊緊跟隨。看到劉邦枕著一個宦官躺在床上。樊哙等人見到劉邦之後,痛哭流涕地說:「起初陛下與臣等在豐沛起兵,平定天下,多麼雄壯啊!今天下已定,又多麼的疲憊啊!而且陛下病得很重,大臣震驚恐懼,不願意接見臣等討論大事,反而獨自與跟一個宦官隔絕在這麼?而且陛下唯獨不見趙高之事麼?」於是劉邦笑著從床上起來。

之後劉邦的病情急剧惡化,呂后請了良醫為他醫治,劉邦詢問病情,醫生進言:「可以治好。」劉邦聽了不但不高興,還罵醫生說:「我不過是一個布衣,手提三尺之劍,取得天下,這不是因為天命嗎?人的命運決定於上天,就算是扁鵲來醫治我,就能把我醫好麼?」不願意繼續治療,賜給醫師五十斤黃金,令醫師離去。

之後呂后問劉邦:「陛下百歲之後,蕭相國死了,令誰代之?」劉邦說:「曹参可。」呂后問還有誰,劉邦說:「王陵可。然陵略顯憨直,陳平可以助之。陳平智謀有餘,但是難以獨當重任。周勃厚道穩重,不夠文雅,但是安定劉氏者一定是周勃,可令為太尉。」呂后再問,劉邦說:「以後的事,也不是你能知道的。」暗示呂后也無法活那麼久。

四月甲辰(前195年6月1日),劉邦駕崩於長樂宮中,呂后和審食其商量說:「那些將軍們跟先帝都出身普通編戶民,後來變成先帝的臣子,就常常流露出不滿、不服的樣子,怎麼可能要他們侍奉少主,如果不全部族滅他們,天下就安定不了。」所以呂后過了四天還不發喪,打算發兵殺盡功臣將領。有人轉告這些話給將軍酈商。酈商去會見審食其,說:「我聽說皇帝駕崩四天,還不發喪,意思是要殺掉所有大將。如果真的如此,天下就危險了。陳平、灌嬰率十萬大軍鎮守滎陽,樊噲、周勃率二十萬大軍攻下燕國和代國,如他們聽說皇帝駕崩了,將軍們都即將被遭殺戮,必定會聯兵回頭來攻關中。屆時大臣們在京師叛亂,諸侯們在外面造反,遲早要滅亡的了。」審食其把這些話轉告了呂后,於是在丁未日(6月4日)發喪,大赦天下。

五月丙寅(6月23日),葬劉邦于长陵。己巳(6月26日),太子劉盈繼位,是為漢惠帝。

汉惠帝來到太上皇廟。群臣們都說:「先帝出身為普通百姓,為世道撥亂反正,平定天下,為漢朝的太祖,功最高。」上谥号為高皇帝,庙号為太祖。又下令全國各郡國諸侯建太祖廟,每年按時祭祀。至今在江西客家地区还有汉高帝信仰,尤其是江西宁都县汉帝信仰最普遍,几乎每个村都建有汉帝庙。

漢高帝十二年(前195年),劉邦平定英布叛亂後,途中路過故鄉沛縣。在沛宮置備酒席,把老朋友和父老子弟都請來一起開懷暢飲。鄉民們還召集沛中兒童120人唱歌助興。酒到濃時,劉邦彈擊著築琴,唱起自編的楚歌:「大風起兮雲飛揚,威加海內兮歸故鄉,安得猛士兮守四方!」讓兒童們跟著學唱。劉邦邊歌邊舞,熱淚盈眶。他對沛縣父老兄弟說:「遊子悲故鄉。吾雖都關中,萬歲後吾魂魄猶樂思沛。且朕自沛公以誅暴逆,遂有天下,其以沛為朕湯沐邑,複其民,世世無有所與。」沛縣的父老兄弟天天快活飲酒,盡情歡宴,敘談往事,取笑作樂。過了十多天,劉邦決定要走了,沛縣父老堅決要劉邦多留幾日。劉邦說:「吾人眾多,父兄不能給。」於是離開沛縣。這天,沛縣城裏全空了,百姓都趕到城西來敬獻牛、酒等禮物。劉邦又停下來,搭起帳篷,痛飲三天。沛縣父兄都叩頭請求說:「沛幸得複,豐未複,唯陛下哀憐之。」劉邦說:「豐吾所生長,極不忘耳,吾特為其以雍齒故反我為魏。」但在百姓的再三請求下,劉邦才答應免除豐邑的賦稅徭役,並封沛侯劉濞為吳王。

西汉史学家司马迁在《史记》中的评价:“太史公曰:夏之政忠。忠之敝,小人以野,故殷人承之以敬。敬之敝,小人以鬼,故周人承之以文。文之敝,小人以僿,故救僿莫若以忠。三王之道若循环,终而复始。周秦之间,可谓文敝矣。秦政不改,反酷刑法,岂不缪乎?故汉兴,承敝易变,使人不倦,得天统矣。”

东汉史学家班固在《汉书》中对汉高祖刘邦一生事业的评价:“初,高祖不修文学,而性明达,好谋,能听,自监门戍卒,见之如旧。初顺民心作三章之约。天下既定,命萧何次律令,韩信申军法,张苍定章程,叔孙通制礼仪,陆贾造《新语》。又与功臣剖符作誓,丹书铁契,金匮石室,藏之宗庙。虽日不暇给,规摹弘远矣。”

“赞曰:《春秋》晋史蔡墨有言:陶唐氏既衰,其后有刘累,学扰龙,事孔甲,范氏其后也。而大夫范宣子亦曰:‘祖自虞以上为陶唐氏,在夏为御龙氏,在商为豕韦氏,在周为唐杜氏,晋主夏盟为范氏。’范氏为晋士师,鲁文公世奔秦。后归于晋,其处者为刘氏。刘向云战国时刘氏自秦获于魏。秦灭魏,迁大梁,都于丰,故周市说雍齿曰:‘丰,故梁徙也。’是以颂高祖云:‘汉帝本系,出自唐帝。降及于周,在秦作刘。涉魏而东,遂为丰公。’丰公,盖太上皇父。其迁日浅,坟墓在丰鲜焉。及高祖即位,置祠祀官,则有秦、晋、梁、荆之巫,世祠天地,缀之以祀,岂不信哉!由是推之,汉承尧运,德祚已盛,断蛇著符,旗帜上赤,协于火德,自然之应,得天统矣。”

郦食其:“收天下之兵,立诸侯之后。降城即以侯其将,得赂即以分其士,与天下同其利,豪英贤才皆乐为之用。”

魏豹:“汉王慢而侮人,骂詈诸侯髃臣如骂奴耳,非有上下礼节也。”

高起、王陵:“陛下慢而侮人,项羽仁而爱人。然陛下使人攻城略地,所降下者因以予之,与天下同利也。”

韩信:“陛下不能将兵,而善将将,此乃言之所以为陛下禽也。且陛下所谓天授,非人力也。”

陆贾:“项羽倍约,自立为西楚霸王,诸侯皆属,可谓至强。然汉王起巴、蜀,鞭笞天下,遂诛项羽,灭之。五年之间,海内平定。此非人力,天之所建也。”“皇帝继五帝、三皇之业,统理中国;中国之人以亿计,地方万里,万物殷富;政由一家,自天地剖判未始有也。”

司马迁:“然王迹之兴,起於闾巷,合从讨伐,轶於三代,乡秦之禁,适足以资贤者为驱除难耳。故愤发其所为天下雄,安在无土不王。此乃传之所谓大圣乎?”

荀悦:“高祖起于布衣之中,奋剑而取天下,不由唐虞之禅,不阶汤武之王,龙行虎变,率从风云,征乱伐暴。廓清帝宇。八载之间,海内克定,遂何天之衢。登建皇极。上古已来,书籍所载,未尝有也。非雄俊之才、宽明之略、历数所授、神祇所相、安能致功如此。”

曹植:“昔汉之初兴,高祖因暴秦而起。官由亭长,自身亡徒。招集英雄,遂诛强楚。光有天下,功齐汤武。业流后嗣,诚帝王之元勋,人君之盛事也。然而名不继德,行不纯道。寡善人之美称,鲜君子之风采。惑秦宫而不出,窘项座而不起。计失乎郦生,忿过乎韩信。太公是诰,于孝违矣。败古今之大教,伤王道之实义。身没之后,崩亡之际,果令凶妇肆鸩酷之心,嬖妾被人豕之刑。亡赵幽囚,祸殃骨肉。诸吕专权,社稷几移。凡此诸事,岂非高祖寡计浅虑以致祸乱?然彼之雄才大略,倜傥之节,信当世至豪健壮杰士也。又其枭将尽荩臣,皆古今之鲜有,历世之希睹。彼能任其才而用之,听其言而察之。故兼天下而有帝位,流巨勋而遗元功也。不然斯不免当世之妄。”

曹冏:“汉祖奋三尺之剑,驱乌集之众,五年之中,遂成帝业。自开关以来,其兴立功勋,未有若汉祖之易也。夫伐深根者难为功,摧枯朽者易为力,理势然也。”

刘邵:“若一人之身,兼有英雄,则能长世;高祖、项羽是也。”

刘渊:“大丈夫当为汉高、魏武,呼韩邪何足效哉!”

石勒:“朕若逢高皇,当北面而事之,与韩彭竞鞭而争先耳。”。脱遇光武,当并驱于中原,未知鹿死谁手。大丈夫行事当礌礌落落,如日月皎然,终不能如曹孟德、司马仲达父子,欺他孤儿寡妇,狐媚以取天下也

司马昱:“高祖则倜傥疏达,魏武则猜忌狭吝。”

李世民:“正主御邪臣,不能致理;正臣事邪主,亦不能致理。唯君臣相遇,有同鱼水,则海内可安也。昔汉高祖,田舍翁耳。提三尺剑定天下,既而规模弘远,庆流子孙者,此盖任得贤臣所致也。”

辽太祖耶律阿保机敬仰刘邦,故兼姓刘氏;又以萧何助刘,故变其母族、后族为萧氏,辽朝皇族耶律氏也兼姓刘氏。

司馬光於資治通鑑中,引用班固的評價:“初,高祖不修文學,而性明達,好謀,能聽,自監門、戍卒,見之如舊。初順民心作三章之約。天下既定,命蕭何次律、令,韓信申軍法,張蒼定章程,叔孫通制禮儀;又與功臣剖符作誓,丹書、鐵契,金匱、石室,藏之宗廟。雖日不暇給,規摹弘遠矣。”

苏轼:“予观汉高祖及光武,及唐太宗,及我太祖皇帝,能一天下者四君,皆以不嗜杀人者致之,其余杀人愈多,而天下愈乱。”

苏辙:“夫古之英雄,唯汉高帝为不可及也夫。”

何去非:“汉太祖挟其在己之智术,固无足以定天下而王之。然天下卒归之者,盖能收人之智而任之不疑也。”

范浚:“夫以高祖权略智数,揽英豪而驱御之,盖真王霸才,虽羽百辈不敌也。”

明朝官修皇帝实录《明太祖实录》记载,明太祖朱元璋在洪武七年八月初一日(1374年9月7日),亲自前往南京历代帝王庙祭祀三皇五帝、夏禹王、商汤王、周武王、汉高祖、汉光武帝、隋文帝、唐太宗、宋太祖、元世祖一共十七位帝王。

洪武二十一年(1388年)隋文帝的塑像被撤出南京历代帝王庙,其中对汉高祖刘邦的祝文是:“惟汉高祖皇帝除嬴平项,宽仁大度,威加海内,年开四百。有君天下之德而安万世之功者也。元璋以菲德荷天佑人助,君临天下,继承中国帝王正统,伏念列圣去世已远,神灵在天,万古长存,崇报之礼,多未举行,故于祭祀有阙。是用肇新庙宇于京师,列序圣像及历代开基帝王,每岁祀以春、秋仲月,永为常典。今礼奠之初,谨奉牲醴、庶品致祭,伏惟神鉴。尚享!”;“项羽南面称孤,仁义不施,而自矜功伐。高祖知其然,承以柔逊,济以宽仁,卒以胜之。”

毛泽东对刘邦的评价是「老粗出人物」、「封建皇帝里边最厉害的一個」,「他得天下一因决策对头,二因用人得当」。1964年毛泽东更借题发挥,指出:「自古以来,能干皇帝大多是老粗出身。汉朝的刘邦是封建皇帝里边最厉害的一个——刘敬劝他不要建都洛阳,要建都长安,他立刻就去长安;鸿沟划界,项羽引兵东退,他也想到长安休息,张良说,什么条约不条约、要进攻,他立刻听了张良的话,向东进;韩信要求自封假齐王,刘邦说不行,张良踢了他一脚,他立刻改口说:『要封就封真齐王,何必要假的。』」毛澤東還認為项羽非政治家,劉邦则为一位高明的政治家。

毛泽东在1964年3月24日,在一次听取汇报时的插话中对汉高祖刘邦,元太祖成吉思汗、明太祖朱元璋的治国能力评价如下:“可不要看不起老粗。”“知识分子是比较最没有知识的,历史上当皇帝的,有许多是知识分子,是没有出息的:隋炀帝,就是一个会做文章、诗词的人;陈后主、李后主,都是能诗善赋的人;宋徽宗,既能写诗又能绘画。一些老粗能办大事:成吉思汗,是不识字的老粗;刘邦,也不认识几个字,是老粗;朱元璋也不识字,是个放牛的。”(毛泽东举例只是为了强调“一些老粗能办大事”,并不是说成吉思汗和刘邦真的不识字,也不是说刘邦只认识几个字。事实上,成吉思汗,刘邦,朱元璋三人身为帝王,他们的文化水平至少达到批阅奏折和签署命令的程度。刘邦和朱元璋的文化水平不必细谈,相关史书记载很多,至于成吉思汗,元初耶律楚材在《玄风庆会录》一书中提到成吉思汗是可以亲自阅览文件的。)

英國史學家湯恩比更與之和羅馬的凱撒大帝齊名。“人类历史上最有远见、对后世影响最大的两位政治人物,一位是开创罗马帝国的恺撒,另一位便是创建大汉文明的刘邦。恺撒未能目睹罗马帝国的建立以及文明的兴起,便不幸遇刺身亡,而刘邦却亲手缔造了一个昌盛的时期,并以其极富远见的领导才能,为人类历史开创了新纪元!”

\subsection{年表}



\begin{longtable}{|>{\centering\scriptsize}m{2em}|>{\centering\scriptsize}m{1.3em}|>{\centering}m{8.8em}|}
  % \caption{秦王政}\
  \toprule
  \SimHei \normalsize 年数 & \SimHei \scriptsize 公元 & \SimHei 大事件 \tabularnewline
  % \midrule
  \endfirsthead
  \toprule
  \SimHei \normalsize 年数 & \SimHei \scriptsize 公元 & \SimHei 大事件 \tabularnewline
  \midrule
  \endhead
  \midrule
  五年 & -202 & \begin{enumerate}
    \tiny
  \item 十二月垓下之战,汉灭楚统一天下,汉王刘邦即皇帝位。
  \item 汉置长安县、无锡县。
  \item 七月,燕王臧荼起兵反汉。
  \item 十月,刘邦率军亲征灭燕,俘杀臧荼。刘邦立卢绾为燕王。
  \item 汉高祖册封无诸为闽越王,封国闽越,首都冶城位于今之福州。
  \end{enumerate} \tabularnewline\hline
  六年 & -201 & \tabularnewline\hline
  七年 & -200 & \tabularnewline\hline
  八年 & -199 & \tabularnewline\hline
  九年 & -198 & \tabularnewline\hline
  十年 & -197 & \tabularnewline\hline
  十一年 & -196 & \tabularnewline\hline
  十二年 & -195 & \tabularnewline
  \bottomrule
\end{longtable}


%%% Local Variables:
%%% mode: latex
%%% TeX-engine: xetex
%%% TeX-master: "../Main"
%%% End:
