%% -*- coding: utf-8 -*-
%% Time-stamp: <Chen Wang: 2019-12-17 12:02:29>

\section{玄汉\tiny(23-25)}

\subsection{生平}

漢更始帝刘玄(?-25年),字圣公,南阳郡蔡阳县(今湖北省枣阳市西南)人,两汉之际绿林军擁立的皇帝,或被視為西漢的最後一位皇帝。

劉玄原本是西汉宗室,是汉景帝劉啟的後代、汉光武帝刘秀的族兄。祖父為苍梧太守刘利,父刘子张,母何氏。

刘玄生年不详。起初,刘玄的弟弟被人杀害,他便结交宾客准备报仇,可不久之后宾客犯罪,刘玄便跑到平阳躲避官府的追捕。官吏抓走他的父亲刘子张,后来刘玄诈死,让人把他的灵柩运回舂陵,官吏才释放了他的父亲。随后刘玄便逃跑藏匿起来。

新莽天凤四年(17年),南方发生饥荒,百姓只得到沼泽中挖荸荠野菜度日。新市人王匡、王凤带领马武、王常、成丹等人起义。他们藏身绿林山中,不久队伍就发展到五万多人。

地皇三年(22年)七月,平林人陈牧、廖湛领导千余人起义,号平林兵,以应王匡,正在这里避难的刘玄参加平林军,担任安集掾。地皇四年(23年)正月,绿林军诸部合兵击破新莽将领甄阜、梁丘赐,遂号刘玄为更始将军。二月辛巳(3月11日),后因其为刘姓宗室,遂在淯水边被拥立为帝,大赦天下,建元更始。更始元年六月入都宛城,大封宗室诸将。他嫉刘縯、刘秀兄弟威名,便诛杀刘縯。起义军在昆阳之戰获胜后,更始帝遣王匡攻洛阳,申屠建、李松攻武关,三辅震动,各地豪强纷纷诛杀新莽的州牧郡守,用汉年号,服从更始政令。更始元年十月,定都洛阳。王莽败死后的更始二年(24年),迁都长安。

李松和趙萌進言更始帝,将功臣封王。大司馬朱鮪反对,称汉高祖的誓言非刘氏不王。更始帝采纳李松之言,以王匡为比阳王、王凤为宜城王、朱鲔为胶东王、卫尉大将军张卬为淮阳王、廷尉大将军王常为邓王,执金吾大将军廖湛为穰王、申屠建为平氏王、尚书胡殷为随王、柱天大将军李通为西平王、五威中郎将李轶为舞阴王、水衡大将军成丹为襄邑王、大司空陈牧为阴平王、骠骑大将军宋佻为颍阴王、尹尊为郾王。李松昇進丞相,和趙萌共理内政。

更始帝雖恢復漢室,但其性格迂腐,重用小人。一登基便沉湎于宫廷奢華生活,即位后将政事都委托于自己的岳父赵萌,放任其外戚专权。赤眉军进逼长安时,更始帝杀害申屠建、陈牧、成丹等起义军重要将领。更始三年(25年)他殺害西漢末代君主劉嬰。

更始三年(25年)九月,赤眉军攻入长安,更始帝单骑逃走。同年十月,投降赤眉,将玺绶送给赤眉拥立的皇帝刘盆子,自己被封为畏威侯,不久改封为长沙王,刘秀则遥降封之为淮阳王。赤眉将领张卬为绝后患,派人将更始帝缢死。

25年刘秀建立東漢後,為確立自己為光復漢室的正統,劉秀及其后代并不认定刘玄是汉朝皇帝,官修《东观汉记》直接将刘玄归为列传。张衡提出不同意见,认为光武之前,更始踐祚,号令天下,光武是继更始之后登上天子之位的,《光武帝纪》之前,应有《更始帝纪》,不過朝廷并没有采纳张衡的意见

\subsection{更始}

\begin{longtable}{|>{\centering\scriptsize}m{2em}|>{\centering\scriptsize}m{1.3em}|>{\centering}m{8.8em}|}
  % \caption{秦王政}\
  \toprule
  \SimHei \normalsize 年数 & \SimHei \scriptsize 公元 & \SimHei 大事件 \tabularnewline
  % \midrule
  \endfirsthead
  \toprule
  \SimHei \normalsize 年数 & \SimHei \scriptsize 公元 & \SimHei 大事件 \tabularnewline
  \midrule
  \endhead
  \midrule
  元年 & 23 & \tabularnewline\hline
  二年 & 24 & \tabularnewline\hline
  三年 & 25 & \tabularnewline
  \bottomrule
\end{longtable}


%%% Local Variables:
%%% mode: latex
%%% TeX-engine: xetex
%%% TeX-master: "../Main"
%%% End:
