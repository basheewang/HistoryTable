%% -*- coding: utf-8 -*-
%% Time-stamp: <Chen Wang: 2021-10-29 17:25:25>

\section{后少帝劉弘\tiny(BC183-BC180)}

\subsection{生平}

漢後少帝劉弘(?-前180年11月14日),原名山,亦曾經為义,西漢第四任皇帝,前184年6月15日-前180年11月14日在位,汉惠帝之子。漢朝大臣誅滅諸呂之后,劉弘被稱是其他人的私生子,被罷黜後旋即處死,史稱「後少帝」。

前187年四月廿八(6月5日),被封为襄成侯。前186年七月,恆山王刘不疑死后,刘山于七月廿七接封为恆山王(后世为避汉文帝刘恒讳,改称“常山王”),并改名刘义。

惠帝死后,刘义的兄长继位为西汉前少帝,惠帝母皇太后吕雉临朝称制,仍称皇太后,不称太皇太后。后吕雉废黜前少帝,于前184年6月15日(五月十一)恆山王刘义继任帝位,并改名为刘弘,即后少帝。在中国历史上,皇帝即位一般改称明年为元年,但因皇太后吕雉臨朝聽政,故未改稱元年。两少帝皆为祖母吕雉的傀儡,他们的嫡母惠帝张皇后也始终未得尊为皇太后。

前180年8月18日(七月三十),吕雉逝世,临终将吕禄女吕氏许给后少帝为后。周勃、陈平等漢朝大臣铲除諸呂,昭告天下,少帝刘弘及梁王劉泰、淮阳王刘武、恆山王刘朝等,並非漢惠帝亲生兒子,应当废黜。

朝臣选定高祖与妃子薄氏之子代王刘恒作为新皇帝(即汉文帝)迎入长安,东牟侯刘兴居和太仆汝阴侯夏侯婴入宫,直称刘弘是其他人的私生子,非刘氏血緣,不当立,夏侯婴親自驾车将刘弘送去少府禁錮,当夜,刘弘及其仍在世的兄弟劉泰、劉武、劉朝即被殺害。

因在位期间为傀儡,且后来被汉朝大臣們否认皇族身份,两少帝常不被认为西汉正统皇帝。

\subsection{年表}


\begin{longtable}{|>{\centering\scriptsize}m{2em}|>{\centering\scriptsize}m{1.3em}|>{\centering}m{8.8em}|}
  % \caption{秦王政}\
  \toprule
  \SimHei \normalsize 年数 & \SimHei \scriptsize 公元 & \SimHei 大事件 \tabularnewline
  % \midrule
  \endfirsthead
  \toprule
  \SimHei \normalsize 年数 & \SimHei \scriptsize 公元 & \SimHei 大事件 \tabularnewline
  \midrule
  \endhead
  \midrule
  元年 & -183 & \tabularnewline\hline
  二年 & -182 & \tabularnewline\hline
  三年 & -181 & \tabularnewline\hline
  四年 & -180 & \tabularnewline
  \bottomrule
\end{longtable}


%%% Local Variables:
%%% mode: latex
%%% TeX-engine: xetex
%%% TeX-master: "../Main"
%%% End:
