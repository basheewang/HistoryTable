%% -*- coding: utf-8 -*-
%% Time-stamp: <Chen Wang: 2021-10-29 17:27:48>

\section{哀帝劉欣\tiny(BC7-BC1)}

\subsection{生平}

汉哀帝劉欣(前27年-前1年8月15日),西汉的第十三位皇帝,在位7年,其正式諡號為「孝哀皇帝」,後世省略「孝」字稱「漢哀帝」。劉欣是汉元帝孙,汉成帝弟定陶恭王刘康之子,母丁氏。

汉哀帝於成帝駕崩後繼位為帝,實權掌握在祖母太皇太后傅昭儀手中。哀帝在位時期,西漢政權已經名存實亡。哀帝為重振皇權而屢次誅殺在漢成帝時代提拔的大臣,尤其是以太皇太后王政君與王莽為首的王氏家族,將曲陽侯王根免去職務趕到封國,將成都侯王況廢為庶人,並給母親丁氏(丁太后)和妻子傅皇后的親戚丁明、丁滿、傅商、鄭業等人加官晉爵。外戚王莽反對為哀帝祖母傅氏和生母丁氏上太后尊號,觸怒傅昭儀(傅太皇太后),王莽被迫辭去大司馬一職。但在哀帝死後,太皇太后王政君一派又重新得勢,傅昭儀(傅太皇太后)、丁氏(丁太后)一派遭誅殺,王莽扶植汉平帝繼位,從此西漢皇權旁落。

劉欣在三歲時繼承父親的定陶王的王位,長大後喜好文辭法律。

汉成帝在位多年仍無子,在绥和元年(前8年)立劉欣為太子,次年成帝去世,劉欣继位,成为漢哀帝。

即位後,以“郑声淫而乱乐,圣王所放”廢除採集民歌的乐府官,非鄭衛之樂者,改由他官職掌。

汉哀帝统治时期,为重振皇权而屡次诛杀王氏家族在汉成帝时代提拔的大臣,将曲阳侯王根免去职务赶到封国,将成都侯王况废为庶人,并给母亲丁太后和妻子傅皇后的亲戚丁明、丁满、傅商、郑业等人加官晋爵。外戚王莽反對為哀帝祖母傅氏和生母丁氏上太后尊號,但不成功,王莽被迫辭去大司馬一職。哀帝死後,外戚王氏一派重新得勢。

汉哀帝还是中國歷史上著名的好男風皇帝,他宠爱宫中舍人、御史董恭的儿子、美男子董贤。建平二年,哀帝拜董贤为黄门郎,至此独宠董贤一人,董贤一月之內三次升遷,震驚朝野。元壽元年晉升董賢為大司馬,當時年僅22歲的董賢就成為百官之首。某日宴席,汉哀帝對董賢笑說,想效法堯舜禪讓,遭王閎勸諫不該戲言而不悅,並將王閎趕出宴席。汉哀帝愛寵董贤之餘,甚至表示要和他合葬義陵。

有次,汉哀帝与董贤白晝相拥而眠,哀帝醒時董賢尚未醒,哀帝乃命人割裂自己衣袖而起身,以免驚醒董賢,這一段傳說即為男風典故「斷袖」之由來。

汉哀帝“雅性不好声色”,常觀賞搏擊角力。即位後身體痿痹,難以使力,晚年加劇,於元寿二年六月戊午(公元前1年8月15日)去世,得年27岁,在位7年,葬于義陵(今陕西咸阳市西北),谥号孝哀皇帝。

建平元年(公元前6年)侍中、奉車都尉劉歆,奏請立《左氏春秋》《毛詩》《逸禮》《古文尚書》等古文經於學官。哀帝命劉歆「與五經博士講論其義,諸博士或不肯置對,歆因移書太常博士」,引發兩漢經學今古文之爭。劉歆的提議,遭今文經博士激烈反對,又觸犯了執政大臣。哀帝持「歆欲廣道術」的意見,然而最終劉歆懼誅,求外放為官。

漢哀帝在位期間,中國人就和佛教有所接觸。據《魏書》等記載元壽元年(公元前2年)博士弟子景盧,從大月氏王使者伊存,受取佛經,是為佛經至中國最早傳授的紀錄。

汉哀帝去世后,因为没有儿子,其堂弟刘\xpinyin*{衎}入继大统,是为汉平帝。太皇太后王政君将汉平帝立为汉成帝的嗣子,而改立东平思王刘宇的孙子、桃乡顷侯刘宣的儿子刘成都为中山王,作为平帝之生父中山孝王刘兴的继承人。何焯认为这是孔光知道王政君怨恨汉哀帝,所以不再主张汉平帝继承汉哀帝是兄终弟及的议题,即汉平帝不算是嗣承了汉哀帝的帝位,使汉哀帝绝了后嗣。

\subsection{建平}

\begin{longtable}{|>{\centering\scriptsize}m{2em}|>{\centering\scriptsize}m{1.3em}|>{\centering}m{8.8em}|}
  % \caption{秦王政}\
  \toprule
  \SimHei \normalsize 年数 & \SimHei \scriptsize 公元 & \SimHei 大事件 \tabularnewline
  % \midrule
  \endfirsthead
  \toprule
  \SimHei \normalsize 年数 & \SimHei \scriptsize 公元 & \SimHei 大事件 \tabularnewline
  \midrule
  \endhead
  \midrule
  元年 & -6 & \tabularnewline\hline
  二年 & -5 & \tabularnewline\hline
  太初\\元将 & -5 & \tabularnewline\hline
  三年 & -4 & \tabularnewline\hline
  四年 & -3 & \tabularnewline
  \bottomrule
\end{longtable}


\subsection{元寿}

\begin{longtable}{|>{\centering\scriptsize}m{2em}|>{\centering\scriptsize}m{1.3em}|>{\centering}m{8.8em}|}
  % \caption{秦王政}\
  \toprule
  \SimHei \normalsize 年数 & \SimHei \scriptsize 公元 & \SimHei 大事件 \tabularnewline
  % \midrule
  \endfirsthead
  \toprule
  \SimHei \normalsize 年数 & \SimHei \scriptsize 公元 & \SimHei 大事件 \tabularnewline
  \midrule
  \endhead
  \midrule
  元年 & -2 & \tabularnewline\hline
  二年 & -1 & \tabularnewline
  \bottomrule
\end{longtable}


%%% Local Variables:
%%% mode: latex
%%% TeX-engine: xetex
%%% TeX-master: "../Main"
%%% End:
