%% -*- coding: utf-8 -*-
%% Time-stamp: <Chen Wang: 2019-12-16 11:49:38>

\section{孝惠帝\tiny(BC195-BC188)}

\subsection{简介}

漢惠帝劉盈(前210年-前188年9月26日),漢朝將盈避諱為满,汉朝开国皇帝漢高帝刘邦和皇后吕雉之子。西汉第二代皇帝,於前195年6月23日—前188年9月26日在位,在位7年,其正式諡號為「孝惠皇帝」,後世省略「孝」字稱「漢惠帝」,也是中国史上第一位皇帝所立的「皇太子」(扶蘇被訛稱為秦「皇太子」,事實上只是長子,秦始皇未立之。胡亥在秦始皇死后诈立为皇太子)。

惠帝在位期间,以温柔敦厚的个性,积极推行黄老学说,注重国家的休养生息和无为而治,放松文化专制政策,修筑长安城,为巩固西汉政权,安定社会,促进经济文化的发展,作出了一定的贡献。惠帝在位时朝政深受母亲呂后干预,呂后具有压制性的影响力,並成为實際统治者,因此司马迁《史记》未设孝惠本纪,反而设〈吕太后本纪〉。

由于受到有限的历史史料,加上其被呂后控制,史官为提高其異母弟汉文帝的地位,长期以来将惠帝视为一位“仁弱”君主,由司馬遷開始,古代学者稱在位时深受呂后臨朝聽政控制,对惠帝往往不太注意,且他所有的兒子都被誅滅諸呂的陳平、周勃、夏侯嬰等大臣們殺死滅口,更無子嗣可提,故只有偶爾宣揚其仁慈友愛的性格。

近世学者所著秦汉史更常直接对其略而不提,而对他的研究也屈指可数,近年有少数学者将他重新审视,稱其統治時與呂后配合,採用黃老之道,能與民休息。但另外的學者則稱,政治上建樹基本上屬於呂后的發揮,惠帝甚至不如部分东汉皇帝通過宦官與外戚大臣鬥爭而掌握实权,故评价不高。

惠帝之後的漢朝和西晉皇帝諡號中都有一「孝」字(除漢光武帝、晉武帝,不含追封(如劉禪)),故常省略。

刘盈,生于前210年,是刘邦和吕雉之子,秦时泗川郡丰邑中阳里人。刘盈有一位异母兄长刘肥,为刘邦在婚前與外妇曹氏所生。有人認為根據后来刘肥之子刘襄为刘邦“嫡长孙”的说辞,以及吕雉父亲吕公要将吕雉许配给刘邦时说的“臣有息女,愿为季箕帚妾”,曹氏可能为刘邦原配,吕雉后来因为某些原因取代曹氏成为刘邦正妻,但也有人反對,其實「箕帚妾」是謙詞,不代表當時的劉邦有正妻。

刘盈还有一名同母姐姐。吕雉有两兄吕泽、吕释之,其中吕泽日后比刘邦率先起兵,有一支独立的军队,为刘邦在楚汉战争胜利起了很大作用。吕雉的妹妹吕媭为刘邦日后的重要将领樊哙的妻子。

刘盈年幼时,刘邦为泗上亭长,家境并不殷实,据叔孙通日后对刘邦的诤谏之词,刘邦和吕雉曾过着“攻苦食啖”的生活。司马迁《史记》卷八<高祖本纪>记述的一段具有神话色彩的轶事,反映了刘盈很小时,便跟随母亲和姐姐下田工作。刘盈不到两岁,刘邦走上了反秦之路,据《史记》<高祖本纪>的一段神化记述,秦始皇在位末年,刘邦私自解纵刑徒,为躲避秦官府的追捕而亡隐于芒、砀山泽之间,吕雉为找到刘邦而颠沛流离。

前209年,秦末民變爆发,陈胜、吴广發動大澤之變,陈胜到了陈县称王,建立张楚政权。各郡县的仕紳大多杀死长官,响应陈胜。刘邦亦响应陈胜,于沛县起兵,号为沛公。刘邦不久即转战南北,刘盈和家人则都被留在丰邑,过着动荡不安的生活。

前206年,项羽灭秦,自立为西楚霸王,尊楚后怀王为义帝,分封十八诸侯王,立刘邦为汉王。此时刘盈和家人仍在丰邑,直到楚汉战争爆发,刘邦路经沛郡时,才派薛欧、王吸去寻找他们,但未能找到。

前205年,刘邦回军平定了三秦,又击项羽至彭城,项羽大败汉军。刘邦因兵败不利,乘车马急速逃去,打算经过沛县,接取家眷西行。在半路上夏侯婴遇到了刘盈和他的姐姐,就把他们收上车来。马已跑得十分疲乏,敌人又紧追在后,刘邦特别着急,有好几次用脚把两个孩子踢下车去,想扔掉他们了事,但每次都是夏侯婴下车把他们收上来,一直把他们载在车上。夏侯婴赶着车子,先是慢慢行走,等到两个吓坏了的孩子抱紧了自己的脖子之后,才驾车奔驰。刘邦为此非常生气,有十多次想要杀死夏侯婴,但最终还是逃出了险境,王陵因战事不利前来护送,侍奉刘盈和他的姐姐逃出睢水,把姐弟两人安然无恙地送到了丰邑。刘盈日后成为皇帝之后,夏侯婴作为太仆侍奉刘盈。刘盈和吕雉非常感激夏侯婴在下邑的路上救了刘盈和他的姐姐,就把紧靠在皇宫北面的一等宅第赐给他,名为“近我”,意思是说“这样可以离我最近”,以此表示对夏侯婴的格外尊宠。

刘邦没有找到父亲刘太公和妻子呂雉,审食其跟随刘太公和呂雉从小路潜行,寻找刘邦,反而碰上了楚军。楚军掳取了刘太公和呂雉,在军中作为人质。。直到前203年,项羽才放回刘太公和呂雉。

前205年7月1日,汉王刘邦立刘盈为太子,大赦罪人。命令刘盈驻守栎阳,在关中的诸侯国人都集中在栎阳守卫。丞相萧何留守关中,侍奉太子,在栎阳处理政务。。

前202年,汉军击败项羽楚军,相继灭亡楚国、齐国,取得了楚汉战争的胜利。八位异姓诸侯王和将相共同拥立汉王刘邦为皇帝,2月28日,刘邦即皇帝位,建立汉朝,尊王后吕雉为皇后,太子刘盈为皇太子。刘邦当初做汉王时,得到了定陶的美女戚姬,非常宠爱她,生下儿子刘如意。刘盈为人仁惠柔弱,刘邦认为不像自己,常想废掉他,改立戚姬的儿子刘如意为太子,因为刘如意像自己。戚姬得到宠幸,常跟随刘邦到关东,日夜啼哭,想要让自己的儿子取代刘盈做太子。吕后年纪大了,“色衰爱弛”,经常留在家中,很少见到刘邦,和刘邦越来越疏远。前198年,刘盈的大舅吕泽去世。同年,刘如意被立为赵王,此后几年,他多次险些取代了刘盈的太子地位。

刘邦想废掉刘盈的太子之位,改立戚夫人生的儿子赵王刘如意为太子。包括为刘邦所敬畏的周昌在内的很多大臣进谏劝阻,都没能改变刘邦坚定不移的想法。刘盈的母亲吕雉很惊恐,不知该怎么办。有人告诉吕雉,留候张良善于出谋划策,刘邦信任他。吕雉就派二哥建成侯吕释之请托张良说:“您一直是陛下的谋臣,现在陛下打算更换太子,您怎么能垫高枕头睡大觉呢?”张良说:“当初陛下多次处在危急之中,采用了我的计谋。如今天下安定,由于偏爱的原因想更换太子,这些至亲骨肉之间的事,即使同我一样的有一百多人进谏又有什么益处。”吕释之竭力请托张良一定得给他出个主意。张良说:“这件事是很难用口舌来争辩的。陛下不能招致而来的,天下有四人(即商山四皓)。这四人已经年老了,都认为陛下对人傲慢,所以逃避躲藏在山中,他们按照道义不肯做汉朝的臣子。但是陛下很敬重这四人。现在您果真能不惜金玉壁帛,让太子写一封信,言辞要谦恭,并预备安车,再派有口才的人恳切地聘请,他们应当会来。来了以后,把他们当作贵宾,让他们时常跟着入朝,叫陛下见到他们,那么陛下一定会感到惊异并询问他们。一问他们,陛下知道这四人贤能,那么这对太子是一种帮助。”于是吕雉让吕释之派人携带太子的书信,用谦恭的言辞和丰厚的礼品迎请这四人。四人前来后住在建成侯的府第中为客。

前198年,刘邦调太常叔孙通任太子太傅。前196年,淮南王黥布反叛,刘邦患重病,打算派刘盈率兵前往讨伐叛军。这四个人互相商议说:“我们之所以来,是为了要保全太子,太子如若率兵平叛,事情就危险了。”于是劝告吕释之说:“太子率兵出战,如立了功,那么权位也不会高过太子;如无功而返,那么从这以后就是遭受祸患了。再说跟太子一起出征的各位将领,都是曾经同陛下平定天下的猛将,如今让太子统率这些人,这和让羊指挥狼有什么两样,他们决不肯为太子卖力,太子不能建功是必定的了。我们听说‘爱其母必抱其子’,现在戚夫人日夜侍奉陛下,赵王如意常被抱在陛下面前,陛下说‘终归不能让不成器的儿子居于我的爱子之上’,显然,赵王如意取代太子的宝位是必定的了。您何不赶紧请吕皇后找机会向陛下哭诉:‘黥布是天下的猛将,很会用兵,现今的各位将领都是陛下过去的同辈,您却让太子统率这些人,这和让羊指挥狼没有两样,没有人肯为太子效力,而且如让黥布听说这个情况,就会大张旗鼓地向西进犯。陛下虽然患病,还可以勉强地乘坐辎车,躺着统辖军队,众将不敢不尽力。陛下虽然受些辛苦,为了妻儿还是要自己奋发图强一下。’”于是吕释之立即在当夜晋见吕雉,吕雉找机会向刘邦哭诉,说了四个人授意的那番话。刘邦说:“我就想到这小子本来不能派遣他,老子自己去吧。”于是刘邦亲自带兵东征,群臣留守,都送到灞上。张良患病,自己勉强支撑起来,送到曲邮,谒见刘邦说:“我本应跟从前往,但病势沉重。楚地人马迅猛敏捷,希望陛下不要跟楚地人斗个高低。”张良又趁机规劝刘邦说:“让太子做将军,监守关中的军队吧。”刘邦说:“子房虽然患病,也要勉强在卧床养病时辅佐太子。”这时叔孙通做太傅,张良任少傅之职。

前195年,刘邦随着击败黥布的军队回来,病势更加沉重,愈想更换太子。张良劝谏,刘邦不听,张良就托病不再理事。刘盈的太傅叔孙通引证古今事例进行劝说:“从前,晋献公因为宠幸骊姬的缘故废掉太子,立了奚齐,使晋国大乱几十年,被天下人耻笑。秦始皇因为不早早确定扶苏当太子,让赵高能够用欺诈伎俩立了胡亥,结果自取灭亡,这是陛下亲眼见到的事实。现在太子仁义忠孝,是天下人都知道的;吕皇后与陛下同经艰难困苦,同吃粗茶淡饭,是患难与共的夫妻怎么可以背弃她呢!陛下一定要废掉嫡长子而扶立小儿子,我宁愿先受一死,让我的一腔鲜血染红大地”,死命争保刘盈的太子之位。刘邦假装答应了他,但还是想更换太子。等到安闲的时候,设置酒席款待宾客,刘盈在旁侍侯。先前依張良建議聘來的四人跟着刘盈,他们的年龄都已八十多岁,须眉洁白,衣冠非常壮美奇特。刘邦感到奇怪,问道:“他们是干什么的?”四人向前对答,各自说出姓名,叫东园公、甪里先生、绮里季、夏黄公。刘邦于是大惊说:“我访求各位好几年了,各位都逃避着我,现在你们为何自愿跟随我儿交游呢?”四人都说:“陛下轻慢士人,喜欢骂人,我们讲求义理,不愿受辱,所以惶恐地逃躲。我们私下闻知太子为人仁义孝顺,谦恭有礼,喜爱士人,天下人没有谁不伸长脖子想为太子拼死效力的。因此我们就来了。”刘邦说:“烦劳诸位始终如一地好好调理保护太子吧。” 

四个人敬酒祝福已毕,小步快走离去。刘邦目送他们,召唤戚夫人过来,指着那四人给她看,说道:“我想更换太子,他们四人辅佐他,太子的羽翼已经形成,难以更动了。吕皇后真是你的主人了。”戚夫人哭泣起来,刘邦说:“你为我跳楚舞,我为你唱楚歌。”刘邦唱道:“鸿鹄高飞,一举千里。羽翮已就,横绝四海。横绝四海,当可奈何!虽有矰缴,尚安所施!”刘邦唱了几遍,戚夫人抽泣流泪,刘邦起身离去,酒宴结束。刘邦最终没更换太子,原本是张良招致这四个人发生了效力。刘邦死后,吕雉宴请张良以报答保太子之恩。

刘邦的将领舞阳侯樊哙因为娶了吕雉的妹妹吕媭为妻,生下儿子樊伉,因此和其他将领相比,刘邦对樊哙更为亲近。在黥布反叛的时候,刘邦一度病得很厉害,讨厌见人,他躺在宫禁之中,诏令守门人不得让群臣进去看他。群臣中如绛侯周勃、颍阴侯灌婴等人都不敢进宫。这样过了十多天,有一次樊哙推开宫门,径直闯了进去,后面群臣紧紧跟随。看到刘邦一人枕着一个宦官躺在床上。樊哙等人见到皇帝之后,痛哭流涕地说:“想当初陛下和我们一道从丰沛起兵,平定天下,那是什么样的壮举啊!而如今天下已经安定,您又是何等的疲惫不堪啊!况且您病得不轻,大臣们都惊慌失措,您又不肯接见我们这些人来讨论国家大事,难道您只想和一个宦官诀别吗?再说您难道不知道赵高作乱的往事吗?”刘邦听罢,于是笑着从床上起来。前195年,燕王卢绾谋反,刘邦命令樊哙以相国的身份去攻打燕国。这时刘邦又病得很厉害,有人诋毁樊哙和吕氏结党,皇帝假如有一天去世的话,那么樊哙就要带兵把戚氏和赵王如意这帮人全部杀死。刘邦听说之后,勃然大怒,立刻命令曲逆候陈平用车载着周勃去代替樊哙,并在军中立刻把樊哙斩首。陈平因惧怕吕雉,并没有执行刘邦的命令,而是把樊哙解赴长安。到达长安时,刘邦已经去世,吕雉就释放了樊哙,并恢复了他的爵位和封邑。陈平也因此得到了吕雉的信任。

刘邦臨終前,曾對呂后說如果相国酂候萧何去世,可以讓平阳侯曹参继任相国职位。《古文苑》之中,收录了五封刘邦给刘盈的手诏,郑重宣布刘盈为自己的继承人,反思了自己曾经所奉行的“读书无益”论,要求刘盈勤奋学习,自己写奏章,重用功臣集团,保护赵王如意。

前195年6月1日,刘邦在长乐宫驾崩。过了四天还不发布丧事消息。吕雉和审食其商量说:“那些将领先前和皇帝同为戶籍編列在册的老百姓,后来北面称臣,这些人就常常怏怏不乐,现在,又要侍奉少主,如果不全部族灭他们,天下就安定不了。”有人听到了这个话,告诉了将军郦商。郦商去见审食其,说:“我听说,皇帝已驾崩四天了,还不发布丧事,而且要杀掉所有的将领。若果真如此,天下可就危险了。陈平、灌婴率领十万大军镇守荥阳,樊哙、周勃率领二十万大军平定燕國和代國,如果他们听说皇帝驾崩了,诸将都将遭诛杀,必定把军队联合在一起,回过头来进攻关中。那时候大臣们在京師內乱,诸侯们在關外造反,漢朝滅亡的日子就可以跷著腳來等待了。”审食其进宫把这告诉了吕雉,于是就在6月4日发丧,大赦天下。

6月23日,安葬刘邦于长陵,当日(或6月26日),太子刘盈即位,来到太上皇庙。群臣都说皇帝“起徽(微)细,拨乱世反之正,平定天下,为汉太祖,功最高。”献上尊号称为高皇帝,即汉高帝。太子刘盈承袭皇帝之号,是为汉孝惠帝。又下令让各郡国诸侯都建高祖庙,每年按时祭祀。

前196年10月,刘邦在会甀击败黥布军,回京途中,经沛县时停留下来。在沛宫置备酒席,把老朋友和父老子弟都请来一起纵情畅饮。挑选沛中幼童一百二十人,教他们唱歌。酒喝得正畅快时,刘邦自己弹击着筑琴,唱起自己编的歌:“大风起兮云飞扬,威加海内兮归故乡,安得猛士兮守四方!”让儿童们跟着学唱。到了前190年,已经做了五年皇帝的刘盈想到父亲生前思念和喜欢沛县,就把沛宫定为父亲的原庙。刘邦所教过唱歌的儿童一百二十人,都让他们在原庙奏乐唱歌,以后有了缺员,就随时加以补充。

刘盈的一母同胞只有姐姐鲁元公主。除了鲁元公主,刘盈在即位时还有七个异母兄弟,他们按年龄大小分别是齐王刘肥、赵王刘如意、代王刘恒、梁王(趙王)刘恢、淮阳王刘友、淮南王刘长和燕王刘建。

刘盈即皇帝位后,尊母亲皇后吕雉为皇太后。吕太后对那些为刘邦侍寝而得宠幸的妃子如戚夫人等人非常气愤,就把她们都囚禁起来,不准出宫。而只有代王刘恒的母亲薄姬由于极少见刘邦的缘故,得以出宫,跟随儿子到代国,成为代王太后。而最终刘邦后宫的妃子只有不受宠爱被疏远的人才能平安无事。

吕雉在成为皇太后之后,就将戚夫人貶為奴隸,囚禁于永巷,剃去頭髮,穿着囚服,令其舂米。戚夫人边舂边唱:“子為王,母為虜,終日舂薄暮,常與死為伍!相離三千里,當誰使告女(汝)?(兒子為諸侯王,母親為奴隸,終日舂米到太陽落西,常常與死亡在一起!母子相離三千里,要找誰來告訴你?)”有人傳話給呂后,呂后大怒,说:“乃欲倚女(汝)子邪?(你想倚靠你儿子吗?)”于是派人召赵王如意入朝以便诛杀。此前汉高帝担心自己死后年轻的赵王难以保全,因而接受赵尧建议,徙御史大夫周昌担任赵國相以保护如意。太后的使臣到后,周昌让赵王推说身体不好,不能前往。使者往返去了三次,周昌都一直坚持不送赵王进京。于是太后很是忧虑,就派使者召周昌进京。周昌进京之后,拜见太后,太后非常生气地骂他:“难道你还不知道我非常恨戚氏吗?而你却不让赵王进京,为什么?”周昌被召进长安之后,太后又派使者召赵王,赵王动身赴京,还在半路上。刘盈仁慈,知道母亲恼恨赵王,就亲自到霸上去迎接,跟他一起回到宫中,亲自保护,跟他同吃同睡。太后想要杀赵王,却得不到机会。前195年12月的一天,刘盈在天明时出去射箭。赵王年幼,不能早起。太后得知赵王独自在家,派人拿去毒酒让他喝下。等到刘盈回到宫中,赵王已经被毒死了,此时距离赵王来到长安已有一个多月,刘如意的谥号为隐王。据《西京杂记》的一段记载,刘盈可能对毒死赵隐王的人进行惩处。赵隐王死后,淮阳王刘友被调去做赵王。

刘盈即位之初,吕雉就想增封诸吕为王。前194年夏天,刘盈与母亲可能就封诸吕为侯之事发生激烈冲突,太后于是派人砍断戚夫人的手脚,挖去眼睛,熏聋耳朵,灌了哑药,扔到廁所之中,叫她“人彘”。过了几天,太后叫刘盈观看「人彘」。刘盈看了,一经询问,才知道这是戚夫人,于是大哭起来。刘盈派人责备母亲说:“這種事不是人做得出來的,兒臣作為太后的兒子,終究無法治理天下!”刘盈因此大病一場。最终刘盈在位的七年间,仅封侯三人,无一人为诸吕。

前201年,汉高帝立庶长子刘肥为齐王,封地七十座城,百姓凡是使用齊國語言的,都归属齐王管轄。前193年,齐王和叔叔楚王刘交入京朝见刘盈。刘盈与齐王饮宴,二人行平等礼节如同家人兄弟的礼节。太后为此发怒,给齐王倒了杯毒酒,刘盈知道后,欲代替兄长饮毒酒,太后因此作罢。齐王害怕不能免祸,就用他的内史勋的计策,把城阳郡献出,做为鲁元公主的汤沐邑。太后很高兴,齐王才得以辞朝归国。前189年,刘肥去世,谥为悼惠王,史称齐悼惠王。刘盈派张良立刘肥的儿子刘襄为齐王,是为齐哀王。

刘盈即位后,相国依旧由萧何担任。前194年,废除了诸侯国设相国的法令,改命曹参为齐国丞相。曹参做齐国丞相时,齐国有七十座城邑。当时全国刚刚安定,齐悼惠王春秋鼎盛,曹参把老年人、读书人都召来,询问安抚百姓的办法。但齐国原有的那些读书人数以百计,众说纷纭,曹参不知如何决定。他听说胶西有位盖公,精研黄老学说,就派人带着厚礼把他请来。见到盖公后,盖公对曹参说,治理国家的办法贵在清靜無為,让百姓们自行安定。以此类推,把这方面的道理都讲了。曹参于是让出自己办公的正厅,让盖公住在里面。此后,曹参治理国家的要领就是采用黄老的学说,所以他当齐国丞相九年,齐国安定,人们大大地称赞他是贤明的丞相。

前193年8月16日,萧何卒。臨終前,漢惠帝詢問蕭何誰可以代替他,蕭何表示陛下最清楚。漢惠帝說:“曹參怎麼樣?”蕭何激動:“皇帝說對了!臣死無遺憾。”曹参听到蕭何去世这个消息,就告诉他的门客赶快整理行装,说:“我将要入朝当相国去了。”过了不久,朝廷派来的人果然来召曹参,曹参在9月7日被任命为相国。曹参离开时,嘱咐后任齐国丞相说:“要把齐国的監獄、市場作为寄托,要慎重对待这些行为,不要轻易干涉。”后任丞相说:“治理国家没有比这件事更重要的吗?”曹参说:“不是这样。監獄、市場这些行为,是善恶并容的,如果您严加干涉,坏人在哪里容身呢?我因此把这件事摆在前面。”曹参起初卑贱的时候,跟萧何关系很好;等到各自做了将军、相国,便有了隔阂。到汉高帝临终时,萧何向高帝推荐的贤臣只有曹参,高帝安排萧何死后,由曹参接替。曹参接替萧何做了汉朝的相国,做事情没有任何变更,一概遵循萧何制定的法度。

曹参从各郡和诸侯国中挑选一些质朴而不善文辞的厚道人,立即召来任命为丞相的属官。对官吏中那些言语文字苛求细微末节,想要一味追求声誉的人,就斥退撵走他们。曹参自己整天痛饮美酒。卿大夫以下的官吏和宾客们见曹参不理政事,上门来的人都想有言相劝。可是这些人一到,曹参就立即拿美酒给他们喝,过了一会儿,有的人想说些什么,曹参又让他们喝酒,直到喝醉后离去,始终没能够开口劝说,如此习以为常。

相国住宅的后园靠近官吏的房舍,官吏的房舍里整天饮酒歌唱,大呼小叫。曹参的随从官员们很厌恶这件事,但对此也无可奈何,于是就请曹参到后园中游玩,一起听到了那些官吏们醉酒高歌、狂呼乱叫的声音,随从官员们希望相国把他们召来加以制止。曹参反而叫人取酒陈设座席痛饮起来,并且也高歌呼叫,与那些官吏们相应和。

曹参见别人有细小的过失,总是隐瞒遮盖,因此相府中平安无事。当时曹参的儿子曹窋担任中大夫。孝惠帝刘盈埋怨曹参不理政事,觉得相国是否看不起自己,于是对曹窋说:“你回家后,试着私下随便问问你父亲说:‘高皇帝刚刚永别了群臣,陛下又很年轻,您身为相国,整天喝酒,遇事也不向陛下请示报告,根据什么考虑国家大事呢?’但这些话不要说是我告诉你的。”曹窋假日休息时回家,闲暇时陪着父亲,把刘盈的意思用自己的語氣說出,规劝曹参。曹参听了大怒,鞭打了曹窋二百下,说:“快点儿进宫侍奉陛下去,天下大事不是你应该说的。”到上朝的时候,;刘盈责备曹参说:“为什么要惩治曹窋?上次是朕让他规劝君的。”曹参脱帽谢罪说:“请陛下自己仔细考虑一下,在您和高皇帝谁圣明英武?”刘盈说:“朕怎么敢跟先帝相比呢!”曹参说:“陛下看我跟萧何,誰比較贤能?”刘盈说:“君好像不如萧何。”曹参说:“陛下说的这番话很对。高皇帝与萧何平定了天下,法令已经明确,如今陛下无为而治,我等谨守各自的职责,遵循原有的法度而不随意更改,不就行了吗?”刘盈说:“好。您休息休息吧!”

经过刘盈和曹参的这番对话,刘盈消除了对曹参的误会,君臣取得了共识。因之黄老之术兴起,并取得显著成效。司马迁对此赞叹道,孝惠皇帝在位时,百姓得以脱离战国时期的苦难,君臣都想通过无为而治来休养生息,所以孝惠帝无为而治,“天下晏然”。曹参担任汉相国三年,于前190年9月24日卒于任上。曹参卒后,百姓对他作歌唱:“萧何为法,顜若画一;曹参代之,守而勿失。载其清净,民以宁一。”

前189年12月11日,刘盈任命安国侯王陵为右丞相,曲逆侯陈平为左丞相。

刘邦驾崩,刘盈即位后,就对叔孙通说:“先帝园陵寝庙,群臣都不熟悉。”于是让叔孙通再度担任太常,让他制定宗庙的礼仪法度。此后陆续制定的汉朝诸多仪礼制度,皆为叔孙通任太常时所论定著录。刘盈自己居住在未央宫,经常要去东边的长乐宫朝拜母亲,还常有小的谒见,每次出行都要开路清道,禁止通行很是烦扰别人,于是就修了一座天桥,正好建在未央宫武库的南面。叔孙通向刘盈报告请示工作,趁机请求秘密谈话说:“陛下怎么能擅自把天桥修建在每月从高寝送衣冠出游到高庙的通道上面呢?高庙是汉太祖的所在,怎么能让后代子孙登到宗庙通道的上面行走呢?”刘盈听了大为惊恐,说要赶快毁掉它。叔孙通说:“人主不能有错误的举动。现在已经建成了,百姓全知道这件事,如果又要毁掉这座天桥,那就是显露出您有错误的举动。希望陛下在渭水北面另立一座原样的的祠庙,把高帝衣冠在每月出游时送到那里,更要增多、增广宗庙,这是大孝的根本措施。”刘盈就下诏令让有关官吏另立一座祠庙。这座另立的祠庙建造起来,就是由于天桥的缘故。刘盈曾在前190年春天到离宫出游,叔孙通说:“古的时候有春天给宗庙进献樱桃果的仪礼,现在正当樱桃成熟的季节,可以进献,希望陛下出游时,顺便采些樱桃来献给宗庙。”刘盈答应办这件事。以后汉代进献各种果品的仪礼就是由此兴盛起来的。

刘盈宠爱闳孺,与他同起同卧,公卿大臣通过闳孺去向刘盈沟通自己的说词。刘盈在位时的郎官和侍中,受到闳孺影响,都戴着用鵕璘毛装饰的帽子,系着饰有贝壳的衣带,涂脂抹粉。平原君朱建能言善辩,口才很好,同时他又刚正不阿,恪守廉洁无私的节操。家安在长安。他说话做事决不随便附和,坚持道义的原则而不肯曲从讨好,取悦于人。辟阳侯审食其品行不端正,靠阿谀奉承深得吕太后的宠爱。当时审食其很想和朱建交好,但朱建就是不肯见他。在朱建母亲去世的时候,陆贾和朱建一直很要好,所以就前去吊唁。朱建家境贫寒,连给母亲出殡送丧的钱都没有,正要去借钱来置办殡丧用品,陆贾却让朱建只管发丧,不必去借钱。然后,陆贾却到审食其家中,向他祝贺说:“平原君的母亲去世了。”审食其不解地说:“平原君的母亲死了,你祝贺我干什么?”陆贾说道:“以前你一直想和平原君交好,但是他讲究道义不和你往来,这是因为他母亲的缘故。现在他母亲已经去世,您若是赠送厚礼为他母亲送丧,那么他一定愿意为您拼死效劳。”于是审食其就给朱建送去价值一百金的厚礼。而当时的不少列侯贵人也因为辟阳侯送重礼的缘故,也送去了总值五百金的钱物。辟阳侯审食其很受吕太后宠爱,有人就在刘盈面前说他的坏话,刘盈大怒,就把审食其逮捕交给官吏审讯,并想借此机会杀掉他。吕太后感到惭愧,又不能替他说情。而大臣们大都痛恨审食其的行为,更想借此机会杀掉他。审食其很着急,就派人给平原君朱建传话,说自己想见见他。但朱建却推辞说:“您的案子现在正紧,我不敢会见您。”然后朱建请求会见刘盈的宠臣闳孺,说服他道:“皇帝宠爱您的原因,天下的人谁都知道。现在辟阳侯受宠于太后,却被逮捕入狱,满城的人都说您给说的坏话,想杀掉您。如果今天辟阳侯被陛下杀了,那么明天早上太后发了火,也会杀掉您。您为什么还不脱了上衣,光着膀子,替辟阳侯到皇帝那里求个情呢?如果皇帝听了您的话,放出辟阳侯,太后一定会非常高兴。而太后、皇帝两人都宠爱您,那么您也就会加倍富贵了。”于是闳孺非常害怕,就听从了朱建的主意,向刘盈给审食其说情,刘盈果然放出了审食其。审食其在被囚禁的时候,很想会见朱建,但是朱建却不肯见审食其,审食其认为这是背叛自己,所以对他很是恼恨。等到他被朱建成功地救出之后,才感到特别吃惊。

汉朝建立之初,经济凋敝,即使皇帝也不能凑齐四匹纯色的马,到了将相有的只能乘坐牛车。汉高帝刘邦为此颁布法令约束节俭,减轻田租赋税,实行“十五税一”制度。刘邦在位后期,因为消灭异姓王和抗击匈奴等战事,田租有所加征。刘盈即位后,下诏减少田租,恢复十五税一。这种举措减轻了农民一些负担,有助于休养生息。前191年1月,为劝勉农桑,刘盈下诏在各地选拔孝弟力田的贤者,免除其徭役负担。又在前191年4月1日趁着加冠之时,减免刑罚,下诏省去妨害吏民的法令,以减少繁扰,调动农民的生产积极性。

前194年2月,开始修筑长安城。前192年春天,征发长安六百里之内的男女民工十四万六千人筑长安城,修筑了三十天。7月,又调发诸侯王、列侯的家人奴隶二万人到长安筑城。前190年2月,再次调發长安六百里之内的男女十四万五千人修筑长安城三十天。9月,长安城最后竣工。赏赐民爵,每户一级。前189年7月,起长安西市,修敖仓。11月修成,诸侯都来京聚会,入朝祝贺。

为了使人口迅速繁殖,刘盈在前189年12月23日借兄长齐悼惠王刘肥去世的时机,下诏规定女子年龄在十五岁以上至三十岁不出嫁的,罚款五算。据东汉时期的应劭解释,汉代凡十五以上的成年人均需交人口税,每人每年一百二十钱为一算,称为“算赋”。刘盈这一时规定,实际就是强制妇女到了十五岁便要结婚生息。这对于繁衍人口和恢复、发展经济都有着显著的促进作用。

汉初,国家贫困,经济萧条,为了巩固雏鹰般的汉朝,刘邦采取了减轻钱重,以便利流通,求得商业发展的政策,结果反而造成物价飞涨、通货膨胀的局面。全国统一后,刘邦对商贾采取了一定的抑制政策,“乃令贾人不得衣丝乘车,重租税以困辱之”。刘盈在位时,为繁荣工商业,便下令进一步放宽对商贾的限制,“复弛商贾之律”。除了不准做官、商贾的人口税比平民重一倍外,其他多予废除。这对于经济的迅速恢复与发展,尤其对商品经济的活跃,起了很大作用。

在外交事务方面,汉仍采取谨慎政策,包括继续以宗室女为公主嫁匈奴单于,立闽越君驺摇为东海王,接受南越王赵佗称臣奉贡。

刘邦死后,刘盈在位时期,汉朝刚刚安定,所以匈奴显得骄傲。冒顿单于渐渐骄横起来,于是写了书信,派使者送给皇太后吕雉,说:“我是孤独无依的君主,生在潮湿的沼泽地,长在平旷的放牛放马的地方,我多次到边境上来,希望能到中原游玩一番。陛下您独立为君,也是孤独无依,单独居住。我们两个做君主的很不快乐,没有什么可以娱乐的。希望我俩能以自己擁有的,來交换到自己没有的。”吕太后看信后十分愤怒,把左丞相陈平、上将军樊哙、中郎将季布召来,商议杀掉匈奴的使者,发兵攻打匈奴。樊哙说:“臣愿意率领十万大军,到匈奴境内去横行冲击。”吕雉询问季布,季布说:“樊哙真该斩首啊!以前陈豨在代地反叛,汉兵有三十二万,樊哙是上将军,当时匈奴把高帝围困在平城,樊哙不能冲破围困。天下百姓唱道:‘平城之下亦诚苦!七日不食,不能彀弩。’现在人们吟唱的声音还在耳畔,没有断绝,受创伤的人刚能站立起来,而樊哙却要让天下震动,胡说什么要带十万大兵橫行匈奴,这是当面欺君罔上。况且这些蠻夷就好比禽兽一样,听到了他们的好话不值得高兴,听到恶语也不值得生气。”吕太后说:“那好吧。”于是便命令大谒者张释写信回报,说:“单于没有忘掉我们这破败的国家,以书信赏赐我们,我们戒慎恐懼。退朝的時日,我自己思虑,我年老气衰,头发、牙齿脱落,走路也走不稳,单于听到誇大的謠言了,不值得单于污辱自己。敝国没有什么罪过,就請單于放過我們。我有两辆御车,驾车的马八匹,奉送给您日常駕。”冒顿得到回信,又派使者来谢罪说:“我们没有听过中國的礼节,幸而得到陛下宽恕。”匈奴献上马匹,前192年春天,刘盈以宗室女为公主继续与匈奴和亲。

前213年,秦始皇在咸阳宫设宴招待群臣,博士仆射周青臣等人称颂秦始皇的武威盛德。博士齐人淳于越则进言建议恢复分封子弟的做法,他认为分封子弟能巩固中央的权利,因为商周都因为分封子弟功臣而得以长治久安。始皇帝把他们的意见下交群臣议论,丞相李斯认为分封子弟的做法不合时宜,建议除了秦国官修史书《秦纪》以为,其他各国的历史记载都予以烧毁。除了博士官负责管理的文献外,天下敢有收藏《诗》、《书》、诸子百家著作者,全部上交地方政府予以烧毁。胆敢私下讨论《诗》、《书》者,处以死刑。以古非今、收藏禁书者,诛灭其家族。而医药、占卜、种树等有实用价值的书籍,不在禁烧之列。如果有人想要学习法令,就以官吏为师。李斯的建议得到了秦始皇的采纳,成为“挟书律”。

秦朝灭亡后,刘邦刚进入关中就同关中百姓约法三章,“杀人者死,伤人及盗抵罪”,取得了关中百姓的拥护。此后,战事仍在继续,三条法令不足以稳定社会,于是相国萧何采集秦朝法令,选取其中合乎时宜的,制订了九章律。到了刘盈在位时,百姓刚免除战争的毒害,人人都想抚育儿童事奉老人。萧何、曹参任丞相,用无为之策来安定百姓,顺从他们的要求,而不加以扰乱,因此百姓衣食丰盛,刑罚使用得很少。前191年4月1日,刘盈加冠,废除了挟书律。

前189年12月,刘盈可能病入膏肓,岁余不能起。为预防东方诸侯王起兵叛乱,刘盈派太尉灌婴指挥车骑、材官镇守荥阳。前188年9月26日,刘盈在未央宮驾崩,终年22岁。呂后哭的卻不怎麼傷心,被張良兒子侍中张辟彊察覺。张辟彊對丞相說:“太后只有孝惠帝這麼一個兒子,但是皇帝駕崩,太后卻不是很難過,你知道個中緣由嗎?”丞相說:“怎麼解釋?”張辟彊說:“皇帝沒有成年兒子,太后害怕你們這些高官,你現在請太后讓呂家親戚掌握兵權和權力,這樣太后才能心安,你們也可以避禍。”丞相聽從張辟彊的計謀。太后大喜,但卻直接導致諸呂專權。呂太后又想為漢惠帝起高墳,甚至能達到在未央宮就能望見的程度。大臣勸諫,呂后不聽。東陽侯張相如勸道:“如果陵墓落成,太后陛下天天見到惠帝陵墓,不停地傷心,對此臣感到很悲痛。”太后於是停止先前的起高墳計劃。10月19日,刘盈被安葬于安陵,谥号孝惠皇帝。此后的汉朝及蜀漢皇帝,除了光武皇帝刘秀、昭烈皇帝劉備和不被認可的皇帝以外,谥号都有一个孝字。汉孝惠帝被安葬后,太子即位成为皇帝,吕雉由此临朝称制,增封诸吕为侯。

\subsection{年表}


\begin{longtable}{|>{\centering\scriptsize}m{2em}|>{\centering\scriptsize}m{1.3em}|>{\centering}m{8.8em}|}
  % \caption{秦王政}\
  \toprule
  \SimHei \normalsize 年数 & \SimHei \scriptsize 公元 & \SimHei 大事件 \tabularnewline
  % \midrule
  \endfirsthead
  \toprule
  \SimHei \normalsize 年数 & \SimHei \scriptsize 公元 & \SimHei 大事件 \tabularnewline
  \midrule
  \endhead
  \midrule
  元年 & -194 & \tabularnewline\hline
  二年 & -193 & \tabularnewline\hline
  三年 & -192 & \tabularnewline\hline
  四年 & -191 & \tabularnewline\hline
  五年 & -190 & \tabularnewline\hline
  六年 & -189 & \tabularnewline\hline
  七年 & -188 & \tabularnewline
  \bottomrule
\end{longtable}


%%% Local Variables:
%%% mode: latex
%%% TeX-engine: xetex
%%% TeX-master: "../Main"
%%% End:
