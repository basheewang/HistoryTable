%% -*- coding: utf-8 -*-
%% Time-stamp: <Chen Wang: 2019-12-17 11:59:15>

\section{新莽\tiny(9-23)}

\subsection{新朝简介}

新朝(9年-23年),又稱新莽,是中國歷史上繼西漢之後的朝代,為當時權臣王莽所建立,僅王莽一代,建都西安(即原長安)。

西漢末年,人民被豪強欺壓而急需改革,儒者信奉讖緯學說認為將改朝換代,當時漢室外戚王莽博得雅名,獲得人民與儒者的支持,使他以偽造符瑞的方式於9年1月10日篡位稱帝,國號為「新」,西漢結束。

王莽稱帝後進行許多改革,主要有改革官制、改變地名、推行王田制、禁止奴婢買賣、五均六筦(國營事業、所得稅與借貸)及改革幣制等。然而,王莽改制大多遵循《周禮》等古制,沒有明確的解決問題。新制政令繁雜,名稱不斷變動。而且朝令夕改,用人不當,改革最終失敗。17年因為天災不斷,而人民因為改革失敗而經濟破產,最後爆發新末民變,赤眉軍、綠林軍等等民變軍相繼而起。新莽軍相繼在成昌之戰、昆陽之戰慘敗。23年劉玄稱帝,即更始帝。同年绿林军攻入長安,王莽被殺,新朝亡。25年東漢光武帝劉秀脫離更始帝宣布登基。同年赤眉軍攻入長安,不久更始帝被殺。漢光武帝擊潰赤眉軍後,最後於36年一統天下。

新朝開創了中國歷史上透過篡位取得政權的先例。王莽積極推動古制,也使得古文經持續發展。而王莽的失敗代表復古思想的破滅,使得漢儒變法禪讓的政治理論至此消失,漸變帝王萬世一統的思想。東漢班固所寫的《漢書》視王莽為逆臣賊子,以至于新朝一度被称呼为“亡新”,《资治通鉴》直接把新朝归入“汉纪”。而且傳統史觀鄙棄用篡位的方式取得政權。所以後世史學家對王莽的評價皆差。直到清末之後評價才逐漸中立。

西漢自漢宣帝去世後,其政治與社會結構變動劇烈,使西漢走向滅亡。其原因主要有以下四點:元、成、哀、平等幾位皇帝或怠於政事、或軟弱無能,政權先後由宦官石顯與外戚(擔任大司馬或大將軍)王氏、傅氏等集團掌控。地方豪強與商賈再度興起,控制地方吏治與經濟,並且與中央官員密切結合。他們一方面壟斷富利,一方面兼併大量土地,以致大量百姓轉為佃農、流民或奴婢。儒家集團的政治力量超過崇尚務實的法家,最終獨佔朝政,而法家勢力衰退瓦解。最後是儒家提倡改制運動,他們加入陰陽家的五行學說,推演天變災異的現象,形成讖緯學說。並且認為王朝德衰,應該禪國讓位。漢帝孤立無援,地方劉姓諸侯國削弱,中央功臣列侯耗盡,又無能臣幹將扭轉局勢,其政權最終被外戚王莽奪取。

漢成帝繼位後,怠忽政事,喜好女色,最後死在「溫柔鄉」中。在漢成帝怠忽職守期間,國事由皇太后王政君的哥哥大司馬王鳳管理,他們開啟王氏集團執政的開端。王鳳的能力頗強,執政後廣收人才,儒法兩家人才與之合流,一致擁戴聽命;而王家兄弟分別位居要津,奠定王家不可動搖的政治勢力。前22年王鳳去世,其兄弟如王音、王商、王根先後以大司馬一職掌握朝政。形成「王鳳專權,五侯當朝」的局面。此時王家因長期安逸而浮華奢侈,但王鳳之侄王莽節儉樸實,酷好儒術,禮賢下士,漸得王鳳與皇太后王政君的重視,於前16年受封為新都侯,並於前8年擔任大司馬一職。

然而隔年漢成帝去世,其侄劉欣繼位,即漢哀帝。漢哀帝祖母傅太皇太后擅權謀且強勢,其與丁太后、外戚傅喜、丁明把持朝政。傅太皇太后與王莽不合,王莽退位而隱居新野,王氏外戚衰退。漢哀帝本身幼體弱多病,政事又被被傅太皇太后把持,所以轉而寵幸董賢(同性戀)。漢哀帝還封董賢為大司馬,並想讓帝位。這些都激起普遍反感,當時地方百姓備受地方豪強、地主欺壓,國家已是一片末世之象,民間「再受命」說法四起。此時王氏勢力尚在,王莽本人更受儒生懷念與人民的擁戴。前1年汉哀帝去世,王莽奉王太皇太后王政君之命,重返朝廷擔任大司馬。

返回政局的王莽積極推行篡位之路,他擁立年僅九歲的汉平帝為傀儡,並且陸續受封安漢公、宰衡等崇高之職。他以王舜、王邑為心腹,甄豐、甄邯、孫建為將領,平晏與劉歆為參謀。打擊何武、公孫祿等反對他的大臣與傅、丁、衛(漢平帝母家)等外戚勢力。王莽又積極施行善政,攏絡人民。民間有災害即捐錢賑災,擴充太學以徵求各地人才,甚至操弄讖緯、杜撰古史以獲取禪位的合法性。到了5年,朝中一片都是王莽勢力,王莽受封「九錫」,汉平帝十分不滿。同年,汉平帝猝死,王莽迎立宗室刘婴即位,即孺子嬰。最後藉由眾大臣以讖緯之事推舉,王莽得以對內稱「假皇帝」,對外稱「攝皇帝」。9年王莽建國「新」,即新朝,西漢滅亡。

王莽是儒家學派巨子,且以新聖自居,所以積極改制西漢末年亂象,意圖回復到儒家歌頌的夏商周三代盛世。改制的內容上從典章制度、法律與教育,下到人民習俗、經濟制度等,十分全面。雖然王莽積極的改革,但大多依據《周禮》的制度推行新政,對於西漢末年的亂象未能完全對症下藥。由於亂象未能改善,不久地方爆發叛亂,顛覆新朝。

首先他依《周禮》改官名為西周官名,例如改大司農為羲和(後為納言)、改郡太守為大尹等等。地方制度也效仿周代的封建制度,許多地名經過多次改名。由於官員和百姓無所適從,最後還是回復原名,平白增加無謂的煩惱。經濟方面,王莽於9年推行改革,內容大多與漢武帝推行的制度相似,不過王莽是依據《周禮》來制定。土地制度方面,耕地收為國有,推行王田制,限制豪強百姓只能有一定土地大小。然而這些措施與現實狀況差異過大,地方豪強不可能因為一道法令而服從;由於土地過小而不能負擔一戶生活,連人民都反對這個改革。三年後,王莽接受區博的建議取消王田制。而奴婢問題,因為王莽無意廢除奴婢制度,而又禁止自由買賣,導致豪強在黑市賤賣奴婢,這個措施最後也廢除了。在改革財政方面,為了防止商人剝削,王莽建立五均六筦與賒貸政策,以公權力平衡物價,類似後世國家社會主義政策。以五均官掌管工商業的利得稅(類似所得稅),把鹽、鐵、酒、幣制、山林川澤收歸國有,以提升國家收入。然而,這些政策多是由薛子仲、張長叔等富商大賈執行。他們以變法為名義,勾結地方官員榨取百姓,使得地方貧富更加懸殊,國家經濟更加失調。最後是貨幣改革,這是最失敗的政策。王莽依據古制,陸續推行刀貨、貝貨等新幣,到2年共有黃金、銀貨、龜寶、貝貨、錢貨、布貨等。這些數種貨幣擾亂新朝財政,到14年又盡數廢除,以致農商失業,經濟崩潰。

對外方面,王莽依據儒家大一統思想,認為世界上應該只有一個王號,所以將諸侯「王」改稱「公」,將四周屬國由「王」改為「侯」。為了宣示新朝的威德,王莽收回原漢朝發給四周各國的「璽」,換成新朝的「章」。這些措施,使漢家諸侯窮困潦倒,匈奴、高句麗、西域諸國和西南夷等君王先後拒絕臣服新朝。11年,由於匈奴不願臣服新朝,王莽發兵三十萬北伐,戰事連年不決。隔年,王莽強迫高句麗協助伐匈奴,反而使高句麗反叛,屢次侵擾東北。同時間西南夷的鉤町叛變,王莽屢次派兵都未能平定。不久西域發生內亂,王莽派王駿西征未果,反而使西域各國正式與新朝斷絕關係。王莽為了報復,又將匈奴單于改為「降奴服于」,高句麗改名「下句麗」。

王莽改革並非是依據孔孟思想的儒家學說,大部分是迷信讖緯和復古論,照搬儒學經書如《周禮》等改革,政策多迂通而不合實情。其政治純屬「書生政治」,僅關注政策的制訂方式,對於實施方式與效率並不在意,政策朝夕相改,官員百姓到最後都敷衍了事。百姓本來期盼王莽能帶領他們脫離西漢末年的亂象,但反而使百姓跌入黑暗的深淵,最後引發民變,促使新朝滅亡。

新朝執政不當引發民怨,17年全國發生蝗災、旱災而饑荒四起,農民紛起反抗叛亂,開啟新末民變。瓜田儀等人於會稽長洲(今江蘇苏州)起事。同年琅邪女子呂母因縣宰冤殺其子,率眾攻陷海曲縣,而後引兵入海為寇。在災情最嚴重的青州、徐州與荊北地區,則分別在17年於荊北興起綠林軍、18年於青徐地區興起赤眉軍,史稱赤眉、綠林起義。河北在馬適求之亂後,也陸續出現其他民變軍,有的以山川土地為號,或以軍容為號,其中以銅馬軍最強。當大臣舉報各地民變時,王莽認為至只是民賊作亂。他不願承認執政錯誤,反而遷怒大臣,並且造作「威斗」、「華蓋」以粉飾太平。為了防止州郡叛變,不許州郡擅自發兵平亂,導致亂事擴大。而中央軍紀律敗壞,四處掠奪,戰鬥力又不強,時常有數十萬大軍被義軍擊潰之事。在這些義軍中,以赤眉軍與綠林軍對局勢影響最大。

赤眉軍由樊崇所建立,18年他率飢民在莒縣起事,眾皆將眉毛染紅,即赤眉軍。亂軍由農民組成,大多不識字,組織包括地位最高的三老、其次有從事和卒史等名稱,大多延用漢朝鄉官的名稱。赤眉軍收編呂母部屬後,在泰山山區擴大勢力。21年王莽派太師犧仲景尚、更始將軍護軍王黨出兵討伐,但在隔年被赤眉軍擊潰而死。王莽再派太師王匡、更始將軍廉丹率十萬兵東征,所經之路都縱兵掠奪。關東人民都稱「寧逢赤眉,不逢太師!太師尚可,更始殺我!」。雙方爆發成昌之戰,最後王匡慘敗,廉丹被殺,赤眉軍擴張到青、徐、豫、兗等州(約今山東、河南與江蘇北部)。只有翼平連率田況率領百姓守衛青徐部分地區,一度阻擋赤眉軍入侵。

綠林軍源自荊北民變。17年荊州北部發生飢荒,王匡、王鳳率領飢民於新市(今湖北京山東北)綠林山起事,稱綠林軍。新莽荊州軍被綠林軍擊敗後,王莽遣司命將軍孔仁守豫州,派納言將軍嚴尤、秩宗將軍陳茂進入荊州平亂。隔年綠林山瘟疫爆發,王常、成丹率兵轉入南郡,稱下江兵,王匡、王鳳率兵東進新市,稱新市兵,並北上攻打宛城。途中於平林(今湖北隨縣東北)獲得陳牧、廖湛率眾加入,即平林兵。下江兵被嚴尤擊敗後,也北上南陽會合新市兵。南陽當地豪強劉縯、劉秀也舉兵響應,稱舂陵兵。23年二月,綠林聯軍擊破新莽軍甄阜、梁丘賜等將,包圍宛城,占領昆陽,史稱藍鄉之戰。綠林諸將擁護劉玄為更始將軍,最後稱帝,建元更始,史稱更始帝,即玄漢。王匡、王凤、朱鲔、刘縯等人被封将相。

王莽得知宛城被圍後,於23年五月派王邑、王尋率領四十二萬新莽軍圍攻昆陽,目標宛城。6月劉秀率軍於昆陽擊潰新莽軍,王尋敗死,王邑逃至洛陽,宛城也被劉縯攻陷,史稱昆陽之戰。四方豪傑得知後,紛紛殺掉州郡守,自稱將軍,用更始年號。戰後劉縯、劉秀人氣大增,更始帝以違抗命令為由处死劉縯。劉秀得知後親赴謝罪,不敢為劉縯服喪。更始帝心有所慚,遂拜劉秀為破虜將軍,封武信侯。更始帝派遣王匡攻洛陽,申屠建、李松攻武關,三輔震動。王莽在南郊舉行「哭天大典」,以求天救,只要民眾哭得夠哀傷,就可加官進爵。但於秋天,更始軍仍攻入长安,王莽在混乱中为商人杜吴殺死於未央宮的漸臺,新朝亡。

新朝亡後,更始帝定都洛陽,由於局勢混亂,就派劉秀巡視黃河以北。劉秀在河北勢力單薄,恰巧地方實力派王郎於河北稱帝,劉秀只能暫避鋒芒。24年初,劉秀獲得劉植、耿純擁護,聯合上谷耿況、漁陽彭寵包圍王郎,同年四月攻陷邯鄲,王郎亡,劉秀被更始帝封為蕭王。而後劉秀率吳漢、鄧禹等將領平定銅馬等河北諸民變軍,被關西人號為銅馬帝。最後在25年六月於鄗城(今河北柏鄉)即皇帝位,史稱光武帝,國號為漢,史稱東漢。

24年更始帝遷都至長安,他建國後昏庸無能,濫封諸侯,將政事委託給岳父趙萌,政治混亂,反而使人民懷念王莽。當時李軼、朱鮪自立於山東,王匡、張卬橫暴三輔。而在汝南、潁川的赤眉軍,因為糧食不足,加上更始帝分封不給國邑,於25年由樊崇和徐宣兵分二路進逼長安。進軍途中擁立劉盆子為帝,史稱赤眉漢,建元建世。更始帝得知赤眉入侵時,還殺害申屠建、陳牧、成丹等將領。同年九月,赤眉軍攻陷長安,更始帝不久被殺,玄漢亡。光武帝乘機南下洛陽,並定都之(改稱雒陽)。赤眉漢政治混亂,諸將跋扈,劉盆子與其兄練習投降的詞說。27年關中缺糧,赤眉軍引兵東歸。光武帝率軍與東歸途中的赤眉會戰於華陰,赤眉軍大敗,劉盆子與樊崇投降。

27年光武帝領有河北大部、河洛與關中等地,但天下仍然群雄割據。當時約有數個勢力,燕代之地的九原盧芳、漁陽彭寵;關東淮水的睢陽劉永、青州張步、東海董憲、魯佼彊與廬江李憲;荊州隴蜀等地有天水隗囂、河西竇融、成都公孫述、漢中延岑、南郡秦豐與夷陵田戎;以及南海鄧讓等勢力。光武帝採取先東後西的戰略,先安撫燕代勢力,於27年到30年間集中力量消滅劉永(自稱天子)為首的關東勢力與廬江的李憲(自稱天子),東方大定。而後南向征服秦豐、田戎等等荊州諸侯,諸侯殘部投奔蜀地。同時,漁陽彭寵也被人刺殺。河西竇融與南海鄧讓也於29年歸順東漢。最後西向對付受荊隴諸侯擁護的成家帝公孫述與天水隗囂、九原盧芳等。光武帝於34年平定天水隗囂,於36年由吳漢攻克成都,滅成家。隔年九原盧芳亡命入匈奴,東漢統一天下,進入東漢時期。

新朝疆域大致上與西漢相同,但在末期疆域萎縮,只新設了西海郡(郡治龍耆城,今青海民和縣)而已。遼東地區撤消了真番、臨屯二郡。在西南地區由七郡變成五郡,部分西南夷成半獨立狀態,放棄了海南島與象郡。西域諸王與新朝中斷關係,使得新朝勢力退出西域。這些疆域直到東漢前期才陸續收復領土。

新朝的行政區劃大致與西漢後期相同,但由郡縣制上加州牧,並且與分封制結合。王莽推行復古改制,亂設行政區劃,改了許多新地名。並且網羅漢宗室功臣後裔、封建官僚,改郡封國。在設置行政區劃方面,王莽修改西漢十三部,據《堯典》分成十二州,裁撤朔方、司隸部,改涼州為雍州、交趾為交州;後又據《禹貢》改爲九州。有的郡甚至五易其名,最後又恢複舊稱。地名的混亂,十分困擾人民。9年更改地方官制的名稱為古稱。14年後大規模更動,結合分封制和郡縣制,郡縣首長與受有茅土的諸侯二合一,將郡太守(新朝稱大尹)分成卒正(侯爵)、連率(伯爵)與大尹等。地方軍事單位的都尉,分成屬令(子爵)、屬長(男爵)等。

在官職的部分,14年設立州牧、部監以監督地方各郡,地位等同三公。設監,地位同上大夫,監督五郡事務。更置牧監副,秩元士,冠法冠,行事如漢刺史。。随着六队、六尉等的建立,新朝也派出监察官员对这些队、尉进行监察。17年,王莽選用能吏侯霸等分督六尉、六隊,如漢刺史,與三公士郡一人從事。

十二州:首長即州牧,新朝增設為部監、監與州牧。共有幽州、并州、冀州、青州、徐州、兗州、豫州、雍州(原涼州)、揚州、荊州、益州、交州,後改為九州。郡:首長原為太守,新朝改稱大尹、後增設為卒正、連率與大尹。縣:首長原為縣令,新朝改稱宰。

新朝官制上承西漢官制,正值王莽改制,所以新朝官制多變,官名及職責也十分複雜。自西漢居攝年間起,王莽便開始推行改制。他附會周禮官制,恢復五等爵制,濫加封賞,濫改官名,如宰衡、太阿之職。建國後,在中央置四輔、三公、四將、六監、九卿、二十七大夫、八十一元士等。四輔(太師、太傅、國師、國將)、三公(大司馬、大司徒、大司空)、四將(更始將軍、衛將軍、立國將軍、前將軍)合稱「十一公」。是9年新朝建國初年,王莽按照哀章所獻金匱內記錄所封王舜、平晏、劉歆、哀章、王邑、甄豐等等十一位高官。其中四輔位列上公,對映五嶽其中的四嶽。四輔與三公對映日月星辰。由於《書·堯典》中,有羲仲、羲叔、和仲、和叔,14年王莽將這些稱號對映到四輔,例如太師犧仲景尚。

新朝九卿與西漢九卿大為不同,王莽修改九卿成新朝用的官名,其中將水衡都尉更名為予虞。將宗伯移除,光祿勳、衛尉、太僕被列入六監,加入大司馬司允、大司徒司直、大司空司若,最後湊成九卿。每卿有大夫三人、每大夫有元士三人,合稱二十七大夫、八十一元士。六監位皆上卿,包含原光祿勳、衛尉、太僕、中尉、執金吾,但都改成新朝官名,並且新設大贅。最後設五司大夫為監察官。

10年更始將軍甄豐之子甄尋不滿父親封賞過低,作符命,言新室要如西周分陝,立二伯。以甄豐為右伯,平晏為左伯,如周召故事,王莽即從之。而後甄尋作符命欲娶黃皇室主为妻,符命案爆發,甄尋逃亡,甄豐自殺。最後甄尋、劉歆之子劉棻、劉泳,王邑弟王奇,及劉歆門人丁隆等數百人或流放、或被斬。更始將軍也被寧始將軍姚恂取代。大封宗室及功臣後裔二百人為侯。為了杜絕反新勢力,10年王莽聽從孫建建議,廢除劉姓諸侯,並將部分擁護劉姓諸侯(劉歆、劉龔、劉嘉等三十二人)賜姓王。」改定安太后號曰黃皇室主,絕之於漢。

王莽建立新朝後,為了宣示新朝的威德,派遣使者四出,東到遼東及朝鮮半島北部的玄菟、樂浪、高句麗及夫餘;南到西南邊境;西到西域。收回舊日漢朝授予外族的印綬,改受新朝的印綬,並把所封的王貶為侯,所用的璽改為章,這樣就引起西南夷鉤町王及匈奴的叛變,西域諸國也逐漸與王莽破裂關係。

在北方方面,匈奴與西漢和平約有30多年,直到新朝建立為止。王莽推行改王為侯的政策,並將「匈奴單于」稱號改為「恭奴善于」,後改為「降奴服于」。為了弱化匈奴,王莽分匈奴居地為15部,強立呼韓邪子孫十五人俱為單于(如孝單于、順單于 (助)、順單于 (登)等)。匈奴烏珠留單于因此而叛變,王莽就於11年徵發士兵三十萬人,大舉進攻匈奴。由於戰事連年不決,自宣帝以來,「數世不見煙火之警,人民熾盛,牛馬布野」的北方邊界,又變成了「北邊虛空,野有暴骨」的悲慘情況;而新朝北部的人民也因為戰亂而相聚為盜,動亂開始形成。王莽為了討伐匈奴,於12年強令高句麗、烏桓出兵,兩國皆不願而叛變,西域地區也陸續叛新投匈。新朝滅亡後,匈奴呼都而尸道皋若鞮單于認為有機可趁,扶持九原盧芳與漁陽彭寵,其中盧芳還被封為漢帝。另一方面,率軍東掠并、燕,西侵涼、朔,對當時的新成立的東漢威脅很大。東北方面,高句麗是東北強國,役屬沃沮、東濊。高句麗叛變新朝後侵擾東北各郡,新朝遼西太守田譚戰死。王莽派嚴尤出兵斬其王,但高句麗別種濊貊仍舊屢次寇邊。直到東漢初年,還入侵右北平、漁陽、上谷等幽西數郡。而烏桓與鮮卑連和,烏桓叛變新朝後投奔匈奴,在東漢中期匈奴衰退後與鮮卑瓜分漠北領地。

西域方面,到漢哀帝、漢平帝時,西域已有五十五國。王莽建立新朝後,西域諸國大多不服統領,而匈奴勢力也進入西域的塔里木盆地。13年親近匈奴的焉耆就殺西域都護但欽,投奔匈奴陣營。王莽於16年派五威將王駿、李崇與郭欽等西征西域,最後被焉耆率領姑墨、尉犁、危須等連軍擊潰,王駿被殺,西域與新朝斷絕往來。西域北道諸國淪入匈奴勢力範圍,只有位於西域南道的莎車率領南道諸國抗衡匈奴。到東漢初年,以莎車王延與其子康最支持漢朝,但漢光武帝為了要全力對內,不能支援南道諸國。不久全西域地區被匈奴占領。而西羌的部分,王莽用政治手段領有西海郡(今青海海宴附近)。到新末漢初,西羌遷入境內掠奪,隗囂招懷其酋豪,隴西數郡都成五谿羌、先零羌的勢力範圍。同時間位於四川松潘一帶的武都參狼羌也被蜀地的公孫述煽動,發起叛亂。這些羌族於35年至37年被東漢馬援所平定,到光武末年,燒當羌又崛起,成為東漢一朝的西患。在北方、西方一片叛亂之際,12年西南夷的鉤町(今雲南廣南一帶)也發生叛變,鉤町王攻殺牂柯太守周韶,跟者益州蠻夷也攻殺太守程隆。越巂、遂久、仇牛、同亭、邪豆之屬,陸續叛變。王莽屢次派兵討伐,寧始將軍廉丹率領的大軍水土不服,數十年拖延未果。而後以文齊為太守,他開墾南中,勸降西南夷,與其恢復關係。公孫述佔據蜀地後,文齊據南中不願投降。到東漢時才歸順漢光武帝。

軍事制度承襲西漢,但王莽更改官制名稱為古稱。為了平定叛亂,王莽採取「以軍領政」的方式控管地方。他命令原為文官的「七公六卿」都兼稱將軍,監督地方官吏,以便穩定地方治安。11年并州、平州發生民變,派著武將軍逯並駐守平亂。22年由於綠林軍在荊北作亂,派司命將軍孔仁駐守豫州,納言將軍嚴尤、秩宗將軍陳茂平定荊州,此為兼稱將軍的實例。以軍領政的方式還有於10年命中郎將、繡衣執法各五十五人,分別駐守大郡,監督地方。中郎將不僅涉内政,兼有對外職責。王莽所封的太子四友,就有中郎將廉丹。此外還在內置司命軍正,外設軍監十二人。20年,新末民變期間,王莽見四方盜賊多,復欲厭之,又置前後左右中大司馬之位,賜諸州牧號為大將軍,各郡的卒正、連帥、大尹為偏將軍,屬令為裨將軍,縣宰為校尉。然而部分軍人在地方胡作非爲,擾亂地方行政,以軍領政的方式還是失敗。

在建立新軍方面,王莽陸續建立豬突豨勇、理軍等新軍,但無大用。與匈奴發生戰爭的期間,王莽招募天下丁男、死罪囚、吏民奴而編成的新軍「豬突豨勇」。又令公卿以下至郡縣黃綬皆保養軍馬,多少各以秩為差。又招募自稱有竒技術可以攻匈奴的人,然而大多誇大其詞,但王莽仍拜為「理軍」,賜以車馬。此外還設立捕盜都尉以平定三輔盜賊。23年王莽拜將軍九人,皆以虎為號,號曰「九虎」,率領北軍精兵數萬人前往關東平亂。這些士兵的妻子與兒女還留在宮中當人質。以上介紹的這些新軍,除捕盜都尉外,其餘大多無用。

新朝时期没有具体的人口调查,估计17年全国有5600万人。由於王莽改制失败加上自然灾害频发和14年黄河下游改道,致使17年爆发绿林赤眉起义。之后烽火遍地,军阀割据和混战,造成黄河流域大量人口死亡,其餘为躲避战火大量向长江流域迁徙。東漢初年,江南地区人口升至全国四成,口数超过500万的有豫州、荆州、扬州與益州等四州。南方人口增长的同时,北方大部分郡国人口减少。

新朝的經濟政策有部分是遵循古制,有部分是重建西漢漢武帝時的經濟政策。立國初年,西漢末年的土地與奴婢問題依舊存在。為了穩定統治,王莽附會《周禮》上的古制,先後下令改制。針對土地被豪強強烈併購與大量貧窮人口轉為奴婢的問題,王莽建立了王田制與禁止奴婢買賣(私屬制)。王田制於9年推行,視全國土地為朝廷所有,稱為「王田」,王田不得任意買賣。恢復井田制,限定男丁八口以下之家,佔田不得超過九百畝(一井),超過的土地須分給宗族鄉鄰。如果無地者由政府授田,每夫一百畝,這是與後世均田制極為類似。針對奴婢問題,王莽推行私屬制,禁止奴婢自由買賣。然而地方大地主激烈反對土地轉讓,王莽雖然派張邯、孫陽到地方強力推行,反而使地方大亂。三年後,王莽接受區博的建議於取消王田制。而私屬制,因為王莽禁止奴婢買賣,地方豪強競相於黑市賣奴婢,使價格低落,最後也宣布廢止。

為了穩定物價、鼓勵生產、增加國家稅收與打壓商人,早在漢武帝時就向商人和工匠征税,但王莽的制度更加完整。他建立五均六筦政策、貢所得、徵荒地稅與賒貸。這是新朝在民生及財政的重要革新,也說是一種國家社會主義政策的推行。「五均」就是把鹽、鐵、酒、貨幣、山林川澤等五類收歸國有以控制經濟,平衡物價,防止商人剝削,增加國庫收入。五均官還針對漁、獵、畜牧、巫、醫以及養蠶、紡織等業,均收取所得純利的十分一,稱「貢」,即現代的所得稅。「六筦」即六管,就是前面的五均與貢所得等六項由官府管理,對每一項制定條例與處罰。此外,為了鼓勵生產,對荒地徵荒地稅,鼓勵開墾荒地。對貧窮或需資金周轉的人,給予賒貸。這些政策雖然出自好意,但推行者多是薛子仲、張長叔等大商人。這些商人到處和地方官吏勾結以榨取百姓,百姓未蒙其利,先受其害。且改革步驟太快,朝令夕改,使百姓官吏不知所從,經濟更加崩壞。

針對貨幣,王莽先後五次改幣。7年,王莽附會周代鑄大錢之說,加鑄契刀、錯刀、大錢與漢代五銖錢共為四品。9年,除大錢外的貨幣均廢除,並鑄小錢與大錢通用,並嚴禁盜鑄。隔年,另造二十八種貨幣:黃金一品、銀貨二品、龜寶四品、貝貨五品、錢貨六品、布貨十品。錢、布共為銅製,所以總稱「五物、六名、二十八品」。後因人民抵制繁雜的莽幣,改用漢五銖錢。在官府無法禁止的情況下,王莽又盡廢諸幣,改行貨幣、貨泉兩品,並於許民間鑄大錢(限期六年)。這樣反覆的改革幣制,讓新朝的經濟混亂,加速人民破產。

新朝思想上起西漢,下承東漢。西漢末期盛行讖緯學說,讖緯是神學與庸俗經學的混合物。儒生好談災異、祥瑞,常以自然現象來附會人事的禍福,後來成為王莽建立新朝的依據。早在西漢時,儒生就多信奉陰陽家「五德終始」之說,盛言「天運循環,貴賤無常」,相信「漢歷當終,新王將興」。由於社會改革的要求及天運循環的理論相結合,儒生鼓吹禅让、改元易号以更始,這些都成為王莽建立新朝时所依靠的理论。

前78年汉昭帝時,眭弘便附會董仲舒之言,認為漢帝應該尋到賢人,禪讓帝位給他,自退位為王,如同夏代堯、周代商故事,他將董仲舒半人半神的神學目的論演變為讖緯神學。汉成帝时,又有齊人甘忠可詐造天官歷、包元太平經十二卷,以言「漢家逢天地之大終,當更受命於天,天帝使真人赤精子,下敎我此道。」甘忠可傳授給重平夏賀良、容丘丁廣世、東郡郭昌等。甘忠可弟子夏賀良等對漢哀帝陈说西漢中衰,當更受命。於是漢哀帝改元太初元將,號曰『陳聖劉太平皇帝』。後因無嘉應,漢哀帝遂誅殺夏賀良等人。讖緯學說到新莽時達到高峰,王莽崇尚古制,也利用讖緯學說以取得帝位。他假借符命、祥瑞,偽造禪讓的根據,如製作了「告安漢公莽為皇帝」的石碑、「金匱神嬗」書言王莽為真天子等讖緯。

西漢末年也有人對陰陽家提出質疑,揚雄仿《論語》作《法言》,模仿《易經》作《太玄》,提出以「玄」作為宇宙萬物根源之學說,強調如實的認識自然現象,並認為「有生者必有死,有始者必有終」,駁斥了方士的學說。他主張要回復儒學五經的本來面目,為東漢注重文字本身的真實性的訓詁學開啟了先河。

新莽的覆滅代表儒學家復古思想的破滅,也使漢儒變法禪讓的政治理論至此消失,漸變帝王萬世一統的思想。先秦學術注重矯正社會的病態,建立大同世界。由於王莽新政的失敗,說明以古代禮法改革的方式不通。魏晉以後,思潮不向整體利益求答案,轉為尋求人性及生存的意義,玄學及佛學遂取代先秦諸子的思想地位。

新莽時期,王莽與劉歆等儒者提倡古文經,使古文經與今文經抗衡,即古今文之爭。王莽還於五經以外增設樂經,增加古文經博士和博士弟子的人數五人。並且擴建太學和太學生宿舍,於地方學校廣招生徒,徵求各地異才。

古今文之爭源自秦始皇焚毀經書事件,後來儒者憑記憶書寫經書,成為今文經。在西漢時於孔壁發現古經書,稱為古文經。西漢的五經(樂經已失散)博士仍以今文經為主。西漢晚期,今文經學派如劉向等人受陰陽家影響,偏向怪力亂神,到西漢末年出現讖緯學說。古文經學派則在西漢末年由巨儒劉歆(劉向之子)與王莽提倡。漢成帝時,劉向負責整理古文經,劉向去世後由劉歆繼承。劉歆最後完成編目,即《七略》,這是中國最早的目錄書,集結古代學術思想與著作的內容。劉歆在整理古籍中發現先秦時期的蝌蚪文,主要有《春秋左氏傳》、《古文尚書》、《逸禮》等等,並認為《毛詩》與其他家派不同,可列為古文。最後劉歆大力提倡古文經,並建議立古文經博士、學官,得以和今文學家抗衡,這受到今文學家的抵制,即今古文之爭。新朝成立後,王莽為上述古文立古文經博士。雖然東漢成立後古文經博士被廢,但不排斥古文經,而且民間研究風氣大盛,三國時期古文終於取代今文,成為學術正統。

王莽改定文字為新莽六書,即古文、奇字、篆書、佐書、繆篆、鳥蟲書,可分成古代文獻文字、通用文字與應用文字等。古文為孔壁經書的戰國文字,奇字是有非孔壁古文的戰國文字,均屬古代文獻文字。王莽为了拉抬古文经学的地位,所以將古文及奇字分列六书的前二位。篆書即秦朝小篆、佐書即秦朝隸書,為新莽時期的通用文字,兩個都廣泛運用,一般日常文書也是用佐書。繆篆為小篆變體,較為權威、莊重的場合使用,如銅器、印章、石刻、貨幣、瓦當等;鳥蟲書即秦體蟲書,用於旗幟和符信,與繆篆都是應用文字。

语言学研究方面,揚雄曾著《方言》,叙述西汉时代各地方言,为研究古代语言的重要资料。王莽当政後,拉拢扬雄,任他为中散大夫。揚雄還写过《劇秦美新》,指斥秦朝,美化新朝。

新朝藝術屬於漢朝藝術的一個時期,比較有特色的有印章、書法與墓畫。新莽的印章屬於秦漢印章的系統,但其工藝水平高,古代璽印無出其右。新莽印章具有自己的風格,分成繆篆與鳥蟲書兩類,在制度、印文、字數、名稱諸方面,與秦漢、魏晉南北朝的印章有較大的區別,其所達到的藝術水準堪稱秦漢印章中最大的靚點。

書法方面,《王俊幕府檔案簡》為起草正式文書的底稿,這是完全成熟的草書。而《郁平大尹馮君孺久墓題記》與《張伯升柩銘》相似,屬於繆篆。字形總體為方扁形,偏旁結構很明確地分為方、圓兩類,方形結構以直線銜接建構而成,圓形結構以曲線糾結盤繞形成,兩者構成了鮮明的對比。

西漢末年到新莽時期,墓室內繪製的壁畫的面積增大,增入了世俗生活宴樂的內容。墓例有洛陽金谷園和偃師辛村的新莽墓,金谷園墓前室穹隆頂在白地上以朱墨等色滿繪彩雲,四壁影作枋柱,以象徵木結構建築。後室頂脊及柱頭斗子間分繪日月神靈異獸。辛村墓則除日、月及辟邪畫面外,其最有名的有《壁画宴乐图》与《西王母图》。並且繪有多幅門吏、庖廚、宴飲、六博等世俗生活的畫面。

除了墓畫,自西漢宣、昭兩帝萌芽的畫像石也有充足的發展,以河南唐河出土的天鳳五年的「漢郁平大尹馮君孺人畫像石墓」為經典之作。該區畫像石的內容豐富,約有30餘。主要描述墓主生活的迎賓拜謁、馴虎騎象、樂舞雜技,反映儒家倫理道德的歷史故事,反映仙人思想的羽人、應龍、四首人面虎以及鎮墓辟邪神怪如蹶張、青龍、白虎、朱雀、鋪首銜杯等。由於主題明顯、內容豐富又質樸,而且紀年明確,所以十分被重視。

\subsection{王莽生平}

王莽(前45年-23年10月6日),字巨君,魏郡元城貴鄉(今河北邯郸大名縣東)人。新朝皇帝。西漢末年政治人物及權臣,之後篡奪皇位並自立新朝。

王莽為濟北王田安六世孫,即陳國、田齊之王裔,田家失國後,齊地的庶民卻依然稱呼田家為「王家」,日久,田家由田姓改為王姓。

王莽原籍濟南郡東平陵。汉元帝皇后王政君之侄。幼年时父親王曼去世,很快其兄也去世。王莽孝母尊嫂,生活俭朴,饱读诗书,结士,声名远播。

王莽对其身居大司马之位的伯父王凤极为恭顺。王凤临終時嘱咐王政君照顾王莽。汉成帝时,西漢陽朔三年(前22年),王莽初任黄门郎,后升为射声校尉。王莽礼贤下士,清廉俭朴,常把自己的俸禄分给门客和穷人,甚至卖掉自己车马接济穷人,深受众人爱戴。其叔父王商甚至上书愿把其封地的一部分让给王莽。永始元年(前16年)封新都侯、騎都尉及光祿大夫侍中。绥和元年(前8年)继他的三位伯、叔之后出任大司马,时年38岁。

翌年,汉成帝去世。汉哀帝继位后傅太后、丁太后及其外戚得势,王莽免官,隐居新野。其间他的二子王获杀死家奴,王莽逼其自杀,得到世人好评。

元寿元年(前2年)其回京城居住。元寿二年(前1年)汉哀帝去世,並未留下子嗣,由太皇太后王政君掌管传国玉玺,王莽任大司马,兼管軍事令及禁軍,立汉平帝,得到朝野的拥戴。元始元年(1年)王莽在推辞再三之后接受了“安汉公”的爵位,将俸禄转封两万多人。元始三年(3年)王莽的女儿成了皇后。元始四年(4年)加号宰衡,位在诸侯王公之上。大力宣扬礼乐教化,得到儒生的拥戴,被加九锡。元始五年(5年),王莽毒死汉平帝,立年仅两岁的孺子婴为皇太子,太皇太后王氏命王莽代天子朝政,称“假皇帝”或“摄皇帝”。从居攝二年(6年)翟義起兵反对王莽,有人开始不断借各种名目对王莽劝进。初始元年十一月戊辰(9年1月10日),王莽正式称帝,改国号为「新」,改长安为常安,封孺子婴为定安公。是為始建国元年。

他當上皇帝後仿照周朝的制度推行新政,屢次改變幣制,更改官制與官名,以王田制為名恢復井田制,把鹽、鐵、酒、幣制、五均賒貸及山林川澤收歸國有,不停恢复西周時代的周禮模式。「今更名天下田曰王田,奴婢曰私属,皆不得买卖”。由於政策多存在不合實情處,百姓未蒙其利,先受其害,不斷挑起天下各貴族和平民的不滿。另外措施又不合時宜,所以措施如王田制推行四年便令民怨沸騰。這可見於當時的中郎將區博所言:「井田雖聖王法,共廢久矣。」和學家馬端臨《文獻通考》所言:「廢之於寡,立之於眾,土地有列在豪強。」

此外,王莽外交政策極為不當。他將原本臣服於漢朝的匈奴、高句麗、西域諸國和西南夷等屬國統治者由原本的「王」降格為「侯」。又收回並損毀匈奴单于之印璽,改授予新匈奴單于之章;甚至將匈奴單于改為降奴服于,高句麗改名下句麗;各國因此拒絕臣服新朝,造成邊境戰争不絕。

天凤四年(17年)各地民軍纷起,有赤眉及绿林大規模的反抗。地皇四年(23年)王莽在南郊舉行哭天大典。同年,绿林军攻入长安,王莽在混乱中为商人杜吴所杀(王莽終年68歲),並被校尉公賓就斬其首,懸於宛市(今宛市镇)之中,新朝灭亡。其頭顱後來被各代收藏,直到西晉元康五年(295年)晉惠帝時,洛陽武庫大火,王莽的頭顱被焚毀。

王莽陵一说在今陕西省渭南市华阴市红岩村附近。由于年代久远,陵墓痕迹难辩。在陕西省安康、镇安、旬阳三县市交界处有王莽山。主峰海拔1560米。地势险要,有王莽墓、刘秀寨遗迹。

中國传统历史学强调忠君、家天下等理念,对王莽的評價普遍不高,一般都認為他只是一位「偽君子」,眾口一辭的千古罪人。东汉史學家班固修订的《漢書》就把王莽列作「逆臣」一類和「佞邪之材」。」而後世評價也大抵是受到了後漢時代史家所影響。事實上王莽本身是篡漢而取得帝位,而同時也是漢朝宗室所滅,從漢朝政權來看,王莽被視作「逆臣賊子」是理所當然。而他在取得帝位前的種種行徑,更被視為作為「逆臣賊子」的理據,例如他殺了漢平帝而立了孺子嬰為皇帝就為一例。

近人胡適開始為王莽平反:“王莽是一千九百年前的一個社會主義者。”他認同王莽改革中的土地國有、均產、廢奴三個大政策,“王莽受了一千九百年的冤枉,至今還沒有公平的論定。他的貴本家王安石受一時的唾罵,卻早已有人替他伸冤了。然而王莽卻是一個大政治家,他的魄力和手腕遠在王安石之上……可憐這樣一個勤勤懇懇,生性『不能無為』,要『均眾庶,抑並兼』的人,到末了竟死在斬台上,……竟沒有人替他說一句公平的話。”

但從另一角度看,王莽也是書生式政治家。王莽登位後推行之新政,大抵都是為了仿照周朝的制度推行,如屢次改變幣制、更改官制與官名、以王田制為名恢復井田制,把鹽、鐵、酒、幣制、山林川澤收歸國有,都是不停恢复西周時代的周禮模式。可是古今風俗不同,環境各異,源於古制的新法,未必一切都合時合宜。而這些新政都是違反了歷史規律,所以推行失敗,自屬歷史必然。所以這個角度看,王莽是一個事事復古,脫離現實的政治家,就正如史家錢穆所言:「王莽的政治,完全是一種書生的政治。...... 不達政情,又無賢輔,徒以文字議論政治。」

旅美歷史學家黃仁宇則指出,從王莽登位後發出的一系詔書中看到,王莽的政策根本脫離了當時的實際環境,亦缺乏適當的用人安排。他在《中國大歷史》裡語帶諷刺的評論王莽:「他盡信中國古典,真的以為金字塔可以倒砌。」

傅樂成在其著作《中國通史》中評論王莽。王莽具有超人的智力、辯才和威嚴,但也有重大的缺點,諸如過度的自信,一味的復古以及猜疑部下等。王莽的行為看來有些偽,也有些愚,但西漢的偽風並不始於王莽,他不過承襲此風而擴充之,結果以偽獲得名聲並篡位之後,得意之餘,乃至無往而不偽。他有他的政治理想,其新法是為整個西漢政治作一通盤的改革,但因缺乏政治才能又迷信復古,事事行之以偽,才會看來令人有愚的感覺。王莽是實際政治的失敗者,也是復古思想的殉道者,他在政治舞台上所表現的一切,雖然最後都歸幻滅,但實在是不平凡的。

史学家吕思勉也认为以汉朝为出发点的历史评价不公,即将王莽的优点全部用一个“伪”字掩盖。王莽本身博学,礼贤下士,孝敬母亲功显君及寡居的嫂嫂,地位越高而对人越谦虚,而且自己与自己家人的生活始终接近清贫,甚至王莽的妻子因为穿着朴素出门迎客被认为是仆佣。吕思勉认为凡是作伪之人,必然是有所图的,而王莽篡汉称帝所图达到之后却并无改变,一生作为如一,又如何能称其为伪?更重要的是,王莽改制成为中国文化的一次重大转变,在西汉及以前,凡是谈论政治的人大多对社会现状进行攻击要求改革,至东汉及以后,玄学、佛学先后兴起,都强调适应社会,而不再追求改革。王莽的行事,诸如恢复井田等,其实很大程度上代表了从先秦以来仁人志士的公意,无论成败,都应由抱有此类见解的人士共负,而不是王莽一人之责。

吕思勉进一步谈到王莽改革的历史影响,「从此以后大家都知道社会改革,不是件容易的事,无人敢作根本改革之想。“治天下不如安天下,安天下不如与天下安”,遂成为政治上的金科玉律」。

史學家韓復智認為王莽的經濟改革對解決當時的經濟問題有一定的幫助。他在《兩漢經濟問題癥結》中提到王莽推行的經濟措施「除變更幣制外,可謂都切中時弊,真正兼顧到平均地權與節制資本兩方面。」其說法是基於王莽一方面把全國土地收歸國有,平均分配給人民。另一方面,他強制有勞動能力的人從事生產,以改善農民生活。其次,他實行五均六筦,不僅防止資本家的兼併和農民遭受重利盤剝,並且扶助小商人的經營,用來救濟農民。但同時變更幣制的經濟措施付卻令通貨膨脹的情況惡化和幣制混亂,而貧窮的人民更加未能在拉闊了的貧富差距下受惠。連富裕的商人亦都破產。雖然如此,王莽的社會經濟改革仍然得到韓復智的正面評價。

哥倫比亞大學的漢學家畢漢思在崔瑞德及魯惟一主編的《劍橋中國秦漢史》的「王莽,漢之中興,後漢」一章中表示王莽如果沒有真才實學,他不可能升為攝皇帝。他雖篡漢建立新朝,孺子劉嬰受到了他不尋常的寬大,雖然被廢但沒有被殺且能過著隱居的生活。而王莽也將孫女嫁給劉嬰。在始建国元年(9年)爆發了兩次原劉氏皇室的起事,王莽很快就派員鎮壓並牢牢地控制漢室,在長安建都。他執政期間的外交、經濟、政治政策並不如傳統認爲地那樣不堪,只是沒能堅持到表現其積極影響。他的新朝的覆滅應由黃河改道帶來的災難負責,士紳和農民的騷亂中他失去了支持。

柏杨赞美他和平建国,在《中国人史纲》一书中提到:“中国历史有一个现象,每一次政权转移,都要发生一次改朝换代型的大混乱,野心家或英雄们各自握有武力,互相争夺吞噬,最后剩下的那一个,即成为儒学派所称颂为‘得国最正’的圣君,在血海中建立他的政权。王莽打破这种惯例,他跟战国时代齐国的田和一样,用和平的方式接收政权,同时也创造了一个权臣夺取宝座的程式,以后很多王朝建立,都照本宣科。西汉王朝在平静中消失,新王朝在平静中诞生,两大王朝交接之际,没有流血。……王莽是儒家学派巨子,以一个学者建立一个庞大的帝国,中国历史上仅此一次。”文人夺权,没有大面积流血,和平过渡,和平演变,殊属不易。”

王莽是汉初济北王田安的六世孙,是战国时代齐国王室的后裔,因齐人称他们为“王家”,所以后来便以“王”为氏。

\subsection{始建国}

\begin{longtable}{|>{\centering\scriptsize}m{2em}|>{\centering\scriptsize}m{1.3em}|>{\centering}m{8.8em}|}
  % \caption{秦王政}\
  \toprule
  \SimHei \normalsize 年数 & \SimHei \scriptsize 公元 & \SimHei 大事件 \tabularnewline
  % \midrule
  \endfirsthead
  \toprule
  \SimHei \normalsize 年数 & \SimHei \scriptsize 公元 & \SimHei 大事件 \tabularnewline
  \midrule
  \endhead
  \midrule
  元年 & 9 & \tabularnewline\hline
  二年 & 10 & \tabularnewline\hline
  三年 & 11 & \tabularnewline\hline
  四年 & 12 & \tabularnewline\hline
  五年 & 13 & \tabularnewline
  \bottomrule
\end{longtable}

\subsection{天凤}

\begin{longtable}{|>{\centering\scriptsize}m{2em}|>{\centering\scriptsize}m{1.3em}|>{\centering}m{8.8em}|}
  % \caption{秦王政}\
  \toprule
  \SimHei \normalsize 年数 & \SimHei \scriptsize 公元 & \SimHei 大事件 \tabularnewline
  % \midrule
  \endfirsthead
  \toprule
  \SimHei \normalsize 年数 & \SimHei \scriptsize 公元 & \SimHei 大事件 \tabularnewline
  \midrule
  \endhead
  \midrule
  元年 & 14 & \tabularnewline\hline
  二年 & 15 & \tabularnewline\hline
  三年 & 16 & \tabularnewline\hline
  四年 & 17 & \tabularnewline\hline
  五年 & 18 & \tabularnewline\hline
  六年 & 19 & \tabularnewline
  \bottomrule
\end{longtable}

\subsection{地皇}

\begin{longtable}{|>{\centering\scriptsize}m{2em}|>{\centering\scriptsize}m{1.3em}|>{\centering}m{8.8em}|}
  % \caption{秦王政}\
  \toprule
  \SimHei \normalsize 年数 & \SimHei \scriptsize 公元 & \SimHei 大事件 \tabularnewline
  % \midrule
  \endfirsthead
  \toprule
  \SimHei \normalsize 年数 & \SimHei \scriptsize 公元 & \SimHei 大事件 \tabularnewline
  \midrule
  \endhead
  \midrule
  元年 & 20 & \tabularnewline\hline
  二年 & 21 & \tabularnewline\hline
  三年 & 22 & \tabularnewline\hline
  四年 & 23 & \tabularnewline
  \bottomrule
\end{longtable}


%%% Local Variables:
%%% mode: latex
%%% TeX-engine: xetex
%%% TeX-master: "../Main"
%%% End:
