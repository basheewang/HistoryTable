%% -*- coding: utf-8 -*-
%% Time-stamp: <Chen Wang: 2021-10-29 17:25:59>

\section{武帝刘彻\tiny(BC140-BC87)}

\subsection{生平}

漢武帝刘彻(前156年7月31日-前87年3月29日),原名彘,西汉第七位皇帝。於7岁时被冊立为储君,16岁登基,在位達54年。其正式諡號為「孝武皇帝」,後世省略「孝」字稱「漢武帝」,是清圣祖以前在位最長的中國皇帝。他雄才大略,文治武功都有顯赫建树,與秦始皇被後世並稱為「秦皇漢武」,被历代史学界和政治家们評價為中國歷史上最偉大的皇帝之一。漢武帝的思想積極進取,极具前瞻性,為朝廷以至社会帶了新思維,亲政後進行了多項大刀闊斧的改革,深遠地影響著後世。

對內政策上,漢武帝用人唯才,不問出身,開創了察舉制并兴太学,以致該時期培養及出現了大量名臣良將;他又頒布《推恩令》,和平地削減了诸侯的權力及勢力,并将盐铁和铸币权收归中央;另外罷黜百家,獨尊儒術,儒学从此成為中國社會主流思想,另有首开丝绸之路、使用年号、设立刺史、加强内廷权力等开创性举措。

對外政策上,漢武帝一改漢高祖刘邦白登之围後世代朝廷奉行的和親傳統,以強勢態度積極地對付匈奴,發動第二階段漢匈戰爭,先後收復了西漢初年的多處領土,不过终其一世未能解除秦朝以來匈奴於中國西北部的威脅。

漢武帝又大幅度地开疆拓土,先後在秦朝故土吞灭了东瓯国、南越國、閩越國,并远征异域,消灭衛滿朝鮮及册封夜郎國等等,继秦朝后再次拓展了中国疆域;同時兩次派遣張騫出使西域,開闢丝绸之路,远征大宛,使汉帝国的影响力和控制力远达中亚,將帝國在民生、經濟、文化和軍事上,都推上了空前的高峰,其在位期間被稱為漢武盛世,為漢朝的極盛時期。

而漢武帝晚年穷兵黩武,對人民造成了相當大負擔。其晚年性情也變得反覆無常,而且迷信多疑,致使了巫蛊之祸的發生,為其普遍整體正面評價留下負面部份。他也对臣下擅用权力,司马迁和李陵家族都在他的命令下遭难。駕崩前兩年,漢武帝下《轮台诏》,重拾文景之治時期的與民生息的政策,為後來的昭宣中興奠定基礎。

据《史记》、《汉书》的武帝本纪以及《漢武故事》,汉武帝生于汉景帝前元年(前156年);母王氏,汉景帝中子,具体排序不详。其母王氏在怀孕时,汉景帝尚为太子。王氏梦见太阳进入她的怀中,告诉景帝后,景帝说:“此贵徵也。”刘彻还未出生,他的祖父汉文帝就逝世了。汉景帝即位后,刘彻出生,他亦是王氏唯一的儿子。一說劉徹的乳名為劉彘,根據漢武故事記載劉徹被立太子時方才改名,但此說與史書說法有出入。

前元四年(前153年),刘彻以皇子的身份被封为胶东王。同年,景帝的长子、他的异母长兄刘荣获封为太子。前元六年(前151年)秋九月,无子无宠的薄皇后被废。第二年(前150年)春正月,废栗太子刘荣为临江王;夏四月乙巳,其母王氏被立为皇后,丁巳,刘彻被立为太子。他成为太子与其母孝景王皇后和其姑母馆陶公主刘嫖有很大关系。刘嫖许诺将她的女儿陈氏嫁给当时四岁(古代按虚岁计算)的胶东王刘彻。刘彻后娶陈氏为妃,两人成婚的时间无考。

后元三年正月甲子(前141年3月9日),景帝逝世,太子刘彻即位,尊皇太后窦氏曰太皇太后,皇后王氏曰皇太后。太子妃陈氏后获封为皇后(具体时间不详)。

漢武帝建立了中朝削弱相权,巩固皇权。“中朝”又称“内朝”,由皇帝左右的亲信的近臣所构成。汉武帝时,他选用一些亲信侍从如尚书、常侍等组成宫中的决策班子,称为“中朝”或“内朝”。相对与“外朝”而言,“大司马、左右前后将军、侍中、常侍、散骑诸吏为中朝。丞相以下至六百石为外朝也”。中、外是相对皇帝居住的宫禁而言,中朝(内朝)官员享有较大的出入宫禁的自由,可随侍皇帝左右且能在宫中办公,外朝官员则无此特权。借由此来培植一批立足于宫中、与以丞相为首的原有朝臣分庭抗礼的内廷官员。重要政事,“中朝”在宫廷之内就先自作出了决策,再交由“外朝”的丞相来执行。尚书,本来是皇帝身边掌管文书员。“中朝”形成之后,尚书的地位日益重要。尚书和一般只参与宫廷议政的官员不同,由于既有官署、官属,又有具体的职司,作为皇帝的秘书机构,在“中朝”逐渐居于核心地位。

汉武帝在地方設置十三州部刺史。即完善监察制度,加強對地方的控制,打擊地方豪強。京師七郡則另外設立司隸校尉監察。汉武帝将全国地方划分为13个监察区,是为冀、兖、豫、青、徐、幽、并、凉、荆、扬、益、朔方、交趾共13州(京畿附近7郡为司隶校尉部作为一个单独的监察区)。每州派遣一名刺史,每年8月巡行所部,监察地方官员和强宗豪右,岁终至京师向御史中丞禀报。此时的刺史为监察官,秩六百石,较郡守的秩比二千石为低。

西汉初,诸侯王的爵位,封地都是由嫡长子单独继承的,其他子孙得不到尺寸之地。虽然文景两代采取了一定的削藩措施,但是到汉武帝初年,“诸侯或连城数十,地方千里,缓则骄,易为淫乱;急则阻其强而合从,谋以逆京师”,严重威胁着汉朝的中央集权。因此元朔二年正月,武帝采纳主父偃的建议,颁行“推恩令”。推恩令吸取了晁错削藩令引起七国之乱的教训,规定诸侯王除以嫡长子继承王位外,其余诸子在原封国内封侯,新封侯国不再受王国管辖,直接由各郡来管理,地位相当于县。这使得诸侯王国名义上没有任何的削藩,避免激起诸侯王武装反抗的可能。于是“藩国始分,而子弟毕侯矣”,导致封国越分越小,势力大为削弱,从此“大国不过十余城,小侯不过十余里”。

察举制為中國古代有系統選拔人才制度之濫觴,對後世影響極大。主要用于选拔官吏。它的确立是从汉武帝元光元年(公元前134年)开始的。察举制不同于以前先秦时期的世袭制和从隋唐时建立的科举制,它的主要特徵是由地方长官在辖区内随时考察、选取人才并推荐给上级或中央,经过试用考核再任命官职。察举制此后成为汉代聘用官吏的制度,有的学者曾经指出,汉武帝“初令郡国举孝廉各一人”的元光元年,是“中国学术史和中国政治史的最可纪念的一年。”

征辟制是汉武帝时推行的一种自上而下选拔官吏制度,就是征召名望显赫的人士出来做官,主要有皇帝征聘和府、州郡辟除两方面,皇帝征召称“征”,官府征召称“辟”。用以作为察举制的补充。

在中国歷史上,年号由汉武帝發明及首先使用,首個年号为建元(前140年—前135年)。此前的帝王只有年数,没有年号。據滿清趙翼的《二十二史札記》考證,年號紀年是在漢武帝十九年首創的,年號為「元狩」,并追認元狩前的年號建元、元光和元朔。《漢書》上記載說,前122年十月,漢武帝出去狩獵,捉到一隻獨角獸白麟,群臣認為這是吉祥的神物,值得紀念,建議用來記年,於是立年號為「元狩」,稱那年(前122年)為元狩元年。可是,過了六年,又在山西汾陽地方獲得一只三個腳的寶鼎,群臣又認為這是吉祥的神物,建議用來紀年,於是改年號為「元鼎」,稱那年為元鼎元年。後來,人們把這記錄年代的開始之年稱为「紀元」,改換年號叫做「改元」。此后,每次新皇帝登基,常常会改元。一般改元从下诏的第2年算起,也有一些从本年年中算起。

汉武盛世是西汉的全盛时期。

匈奴自秦末以来一直威胁中国北边,使农耕生产的受到严重影响。武帝即位之后,自前133年马邑之战起,结束漢朝初期对匈奴的和亲政策,决心设法解决匈奴的外患问题。从元光六年(前129年)开始对匈奴作战。经过卫青和霍去病等人的反击后,西汉西北边境上的威胁暂时解除。中原北边农耕经济从匈奴造成严重破坏的局面中得以恢复。匈奴在军队主力以及人畜资产受到严重损失的情况下,继续向北远遁,并有七年时间即从公元前119年至前112年漠南无王庭,不过其后匈奴又南下与羌人组织联盟攻击汉朝。而西汉军队占领从朔方至张掖、居延廷间的大片土地,设置酒泉、武威、张掖及敦煌四郡,并且命令关东地区人民移民这一地区,此举不但保障河西走廊的安全,使西部地区的得到开发,更打通了中原文化和西域文化交通的通路。

汉武帝除了北伐匈奴之外,也武力平定四方,大幅开扩领土,在西南,漢朝消灭夜郎及南越國,先后建立七個郡,使到今日的两广地区自秦朝后重新归納中國版图。而海南岛在历史上,也首次真正纳入中国的版图。在東方,他於公元前109年至前108年派兵消灭卫氏朝鲜,並且將衛氏朝鮮的國土分為四郡──樂浪郡、真番郡、臨屯郡及玄菟郡。

汉武帝派遣了張騫出使西域,张骞的两次出使打通了中原文化和西域文化交通的通路。即丝绸之路,极大促进了中國同西方经济及文化的交流。

建元元年(辛丑,公元前140年)诏举贤良方正直言极谏之士,上亲策问以古今治道。广川董仲舒上天人三策,对曰“《春秋》大一统者,天地之常经,古今之通谊也。今师异道,人异论,百家殊方,指意不同,是以上无以持一统,法制数变,下不知所守。臣愚以为诸不在六艺之科、孔子之术者,皆绝其道,勿使并进,邪辟之说灭息,然后统纪可一而法度可明,民知所从矣!”。汉武帝采用了董仲舒的建议,「罷黜百家,独尊儒术」。结束先秦以来“师异道,人异论,百家殊方”的局面,于是“令后学者有所统一”。为儒学在中国古代的特殊地位铺路,亦使到儒学成为了中國社会的基礎思想。对中國後代的政治、社會及文化等領域产生了深远的影响。但是,亦有人认为他利用儒学敦化民風,同时采用法術及刑名鞏固政府的權威,即是所谓儒表法裏。

汉武帝元朔五年,创建太学,是接受当时儒家学者董仲舒的建议。董仲舒指出,太学可以作为“教化之本原”,也就是作为教化天下的文化基地。他建议,“臣愿陛下兴太学,置明师,以养天下之士”,这样可以使国家得到未来的人才。所谓“养天下之士”,体现出太学在当时有为国家培育人才和储备人才的作用。汉武帝时期的太学,虽然规模很有限,只有几位经学博士和五十名博士弟子,但是这一文化雏形,代表着中国古代教育发展的方向。太学的成立,助长民间积极向学的风气,对于文化的传播,成为重要的推手,同时使大官僚和大富豪子嗣垄断官位的情形有所转变,一般人家子弟得以增加入仕的机会,一些出身社会下层的人才,也有机会到朝廷做官。

樂府一名本指管理音樂的官府。漢武帝在掌管雅樂的太樂官署之外,另創立樂府官署,掌管俗樂,收集民間的歌辭入樂。「采詩夜誦,有趙、代、秦、楚之謳」、「以李延年為協律都尉,多舉司馬相如等數十人造為詩賦,略論律呂,以合八音之調,作十九章之歌」。後人把樂府機關配樂演唱的詩歌,也稱樂府。

太初历是中国历史上曾经使用过的一种历法,亦是中国历史上第一部完整统一,而且有明确文字记载的历法。在天文学发展历史上具有划时代的意义。在汉武帝太初元年(前104年),由邓平、唐都、落下闳及司馬遷等根据对天象实测和长期天文纪录所制订。《太初历》的制订是中国历史上具有重要性的一次历法大改革。《太初历》的科学成就,首先在于历法计算上的精密准确。中国汉初以前,主要采用“古六历”(黄帝、颛顼、夏、殷、周、鲁)中的《颛顼历》。这个古历,计算一年为三百六十五又四分之一日,一月是二十九天又九百四十分之四百九十九。由于这种古历计算不够精密,常出现月初是无月光的朔日,但实际天空中却有圆满的月光;月中是有月光的望满之日,夜晚却并没有月亮。为了改变这种不对照的现象,司马迁主持制订《太初历》时,重新进行了反复地周密地运算和实践验证。还在于第一次计算了日月蚀发生的周期和精确计算了行星会合的周期。

指中国西汉武帝统治时期进行的币制改革。西汉自建立以来,币制混乱,郡国铸币失控又是汉景帝时期七国之乱發生的原因之一。汉武帝统治时期,由于对外征伐不断,中央财政从此前“京师之钱累巨万,贯朽而不可校”的丰盈一变而为入不敷出的困局。“而富商大贾或蹛财役贫,转榖百数,废居居邑,封君皆低首仰给。”富商大贾富可敌国,恰与窘困的中央财政形成了鲜明对比。中央政府除了靠鬻武功爵等方式快速增加财政收入外,“冶铸煑盐,财或累万金,而不佐国家之急,黎民重困。于是天子与公卿议,更钱造币以赡用,而摧浮淫并兼之徒。”增加中央财政收入,打击大商人,此即汉武帝币制改革的初衷。故汉武帝即位后,为了中央政府在经济管理和政治统治上的需要,便十分重视解决币制问题,先后进行了六次币制改革,基本解决了汉初以来一直未能解决的币制问题。一方面稳定了金融,另一方面将地方的铸币权重新统一于中央。六次改革后三官五铢的发行一举解决了困扰西汉金融多年的私铸、盗铸问题,汉武帝的币制改革至此取得了较大成功。

中央政府在盐、铁产地分别设置盐官和铁官,实行统一生产和统一销售,利润为国家所有。这项制度实施,使国家独占国计民生意义最重要的手工业和商业的利润,可以供给皇室消费以及巨额军事支出。当时,人民的赋税的负担没有增加,国家的收入大增,不但弥补财政上的赤字,并且还有羸余。不过官营盐铁却给社会经济和民众生活带来负面的影响。例如官盐价高而味苦,铁制农具粗劣不合用等。

漢武帝元封元年,桑弘羊針對「諸官各自市(購買),相與爭,物以故騰躍,而天下賦輸,或不償其僦費」的情況,在全國推行均輸法,下令各郡設均輸鹽鐵官,將上貢物品運往缺乏該類貨物的地區出售,然後在適當地區購入京師需求的物資。此法既能解決運費高昂的問題,又可調節物價。更重要的是均輸法舒緩漢武帝晚年的財政危機,桑弘羊對此曾有所讚揚:「山東被災,齊趙大饑,賴均輸之蓄,倉稟之積,戰士以奉,饑民以賑」。然而,均輸法卻被批評未能解決物價問題,「輕賈奸吏,收賤以取貴,未見準之平也」。

在經濟方面,漢武帝爲推動農業,采取了一系列措施。他在全國修了不少水利工程,例如:龍首渠,六輔渠等等,以便農田灌溉。再加上新式耕種技術的提倡,農業生產得到進一步發展。

征和元年(西元前92年)十一月,巫蠱之禍興起。丞相公孫賀之妻使用巫術詛咒及在馳道埋木偶人的事件被告發,公孫賀一家被斬殺,同時還牽連到陽石公主和皇后卫子夫所生的女兒諸邑公主。其後漢武帝又發動了三輔騎士在皇家園林進行搜查,並且在長安城中到處尋找,過了11日才收兵。征和二年七月,與太子劉據結怨的武帝寵臣江充指使胡巫,說宫中有蠱氣。武帝命令江充與按道侯韓說等入宫追查,江充誣告太子宫中埋的木人最多,又有帛書,所言不守道法。太子得知後非常恐懼,聽從少傅石德的計策,派人詐稱武帝使者捕殺江充等人。漢武帝命令丞相劉屈氂派兵擊潰太子,太子舉兵對抗。激戰五日,太子兵敗逃亡,被漢武帝所廢,被圍捕,乃自殺,滅族,唯其曾孫劉病已得親信保全。征和三年,此謀反案的根源巫蛊案真相漸明,大臣上书直言进谏,武帝感悟,下令族滅貳師將軍李廣利、丞相劉屈氂、太監蘇文、江充家族。

漢武帝將鹽鐵酒國營專賣,實行平準均輸政策,防止商人從中漁利,從而增加政府收入,達到了調節物價及防止市場壟斷的功效,但是亦造成了與民爭利的局面。商人遂將注意力轉移至土地買賣,導致土地兼併嚴重。雖然漢武帝武功極盛,但是到處征伐也造成了國庫空虛,大量人民被徵召從軍,死傷甚重,也影響了經濟發展。由於民生困苦、社會動盪不安、人民流離失所及民怨沸騰,天漢二年(前99年),齊、楚、燕、趙和南陽等地相繼爆發大規模農民起義;征和四年(前89年)漢武帝頒下了《輪台罪己詔》向人民承認自己的罪過及公開罪己詔。

汉武帝晚年得子刘弗陵,甚爱之。刘据於巫蛊之乱死後,漢武帝立刘弗陵为太子。太子即位前不久,其生母钩弋夫人被處死,避免未來再有太后涉政的現象。前88年,汉武帝命令画工画了一张《周公背成王朝诸侯图》送予霍光,意思是让霍光辅佐他的小儿刘弗陵作為未來皇帝。對此,中国史学家吕思勉对《汉书·霍光传》的此记载颇有异议,认为汉武帝於临终前杀掉刘弗陵生母是为了避免母后干政、托孤说的“立少子,君行周公之事”和画周公辅政图完全属于捏造。

前87年3月29日(二月丁卯),汉武帝驾崩於五柞宮,享年70岁。4月15日(三月甲申),葬于茂陵,谥号为孝高皇帝,庙号為世宗。

漢武帝愛好文學,為提倡辭賦的詩人。他個人的文學造詣甚高,在南北朝以前的皇帝中屬於文采一流的人物,顏之推把他歸類為曹操、曹丕一級文才的君主。明朝王世貞以為,其成就在“長卿下、子雲上”(《藝苑卮鹽》)其他存留的詩作,《瓠子歌》、《天馬歌》、《悼李夫人賦》都“壯麗鴻奇”(徐禎卿《談藝錄》),為詩詞評論家所推崇。

夏侯胜:武帝虽有攘四夷广土斥境之功,然多杀士众,竭民财力,奢泰亡度,天下虚耗,百姓流离,物故者半。蝗虫大起,赤地数千里,或人民相食,畜积至今未复。亡德泽於民,不宜为立庙乐。

桓谭:「汉武帝才质高妙,有崇先广统之规,故即位而开发大志,考合古今模范,获前圣代故事,建正朔,定制度,招选俊杰,奋扬威怒,武义四加,所征者服,兴起六艺,广进儒术,自开辟以来,惟汉家最为盛图,故显为世宗,可谓卓尔绝世之主矣。」

崔骃:昔孝武皇帝始为天子,年方十八,崇信圣道,师则先王,五六年间,号胜文、景。及后恣己,忘其前之为善。

刘歆:『孝武皇帝愍中国罢劳,无安宁之时,乃遣大将伏波、楼船之属,灭百越七郡。北攘匈奴,降昆邪之众,置五属国,起朔方,以夺其肥饶之地。东伐朝鲜,起玄菟、乐浪以断匈奴之左臂。西伐大宛,并三十六国,结乌孙,起敦煌、酒泉、张掖、武威,以隔氐羌,裂匈奴之右肩。单于孤将远遁漠北,四垂无事,斥地远境,起十馀郡。功业既定,乃封丞相为富民侯,以安天下,富实百姓,其规模可见。又招集天下贤俊,与协心同谋,兴制度,改正朔,易服色,立天地之祀。建封禅,殊官号,存周后,定诸侯之制,永无逆争之心,至今累世赖之。单于守蕃,百蛮服从,万世之基也。中兴之功,未有高焉者也。』

何去非:“孝武帝以雄才大略,承三世涵育之泽,知夫天下之势将就强而不振,所当济之以威强而抗武节之时也。”是以孝武抗其英特之气,选待习骑,择命将帅,先发而昌诛之。盖师行十年,斩刈殆尽,名王贵人俘获百数,单于捧首穷遁漠北,遂收两河之地而郡属之。刷四世之侵辱,遗后嗣之安强。”

班固:“漢承百王之弊,高祖撥亂反正,文、景務在養民,至於稽古禮文之事,猶多闕焉。孝武初立,卓然罷黜百家,表章《六經》,遂疇咨海內,舉其俊茂,與之立功。興太學,修郊祀,改正朔,定歷數,協音律,作詩樂。建封禪,禮百神,紹周後,號令文章,煥然可述,後嗣得遵洪業而有三代之風。如武帝之雄材大略,不改文、景之恭儉以濟斯民,雖《詩》、《書》所稱何有加焉!”

曹丕:“孝武帝承累世之遗业,遇中国之殷阜,府库余金钱,仓廪畜腐粟,因此有意乎灭匈奴而廓清边境矣。故即位之初,从王恢之画,设马邑之谋,自元光以迄征和四五十载之间,征匈奴四十馀:举盛馀,逾广汉,绝梓岭,封狼居胥,禅姑幕,梁北河,观兵瀚海,刈单于之旗,剿阏氏之首,探符离之窟,扫五王之庭。纳休屠昆邪之附,获祭天金人之宝。斩名王以千数,馘酋虏以万计。既穷追其散亡,又摧破其积聚,虏不暇于救死扶伤,疲困于孕重堕殒。元封初,躬秉武节,告以天子自将,惧以两越之诛,彼时号为威震匈奴矣。”

曹植:“世宗光光,文武是攘。威震百蛮,恢拓土疆。简定律历,辨修旧章。封天禅土,功越百王。”

虞世南:“汉武承六世之业,海内殷富,又有高人之资,故能总揽英雄,驾御豪杰,内兴礼乐,外开边境,制度宪章,焕然可述。方於始皇,则为优矣。”

唐太宗:「近代平一天下,拓定邊方者,惟秦皇、漢武。」

司马贞的《史记索隐》:「孝武纂极,四海承平。志尚奢丽,尤敬神明。坛开八道,接通五城。朝亲五利,夕拜文成。祭非祀典,巡乖卜征。登嵩勒岱,望景传声。迎年祀日,改历定正。疲秏中土,事彼边兵。日不暇给,人无聊生。俯观嬴政,几欲齐衡。」

司马光的《资治通鉴》:“孝武穷奢极欲,繁刑重敛,内侈宫室,外事四夷,信惑神怪,巡游无度,使百姓疲敝,起为盗贼,其所以异于秦始皇无几矣。然秦以之亡,汉以之兴者,孝武能尊王之道,知所统守,受忠直之言,恶人欺蔽,好贤不倦,诛罚严明,晚而改过,顾托得人,此其所以有亡秦之失而免亡秦之祸乎!”

李纲:“茂陵仙客,算真是,天与雄才宏略。猎取天骄驰卫霍,如使鹰鹯驱雀。战皋兰,犁庭龙碛,饮至行勋爵。中华疆盛,坐令夷狄衰弱。追想当日巡行,勒兵十万骑,横临边朔。亲总貔貅谈笑看,黠虏心惊胆落。寄语单于,两君相见,何苦逃沙漠。英风如在,卓然千古高著。”

洪迈:“汉之武帝、唐之武后,不可谓不明。”

朱熹:“武帝天资高,志向大,足以有为。末年海内虚耗,去秦始皇无几。轮台之悔,亦是天资高,方能如此。”

王夫之:「武帝之勞民甚矣,而其救饑民也為得。虛倉廥以振之,寵富民之假貸者以救之,不給,則通其變而徙荒民於朔方、新秦者七十余萬口,仰給縣官,給予產業,民喜於得生,而輕去其鄉以安新邑,邊因以實。」

王夫之:「武帝之发觉而捕弗满品者,二千石以下至小吏,主者皆死,则欲吏之弗匿盗不上闻、而以禁其窃发也,必不可得矣。……漢武有喪邦之道焉,此其一矣。」

赵翼:「仰思帝之雄才大略,正在武功。」

吴裕垂:「武帝雄才大略,非不深知征伐之劳民也,盖欲复三代之境土。削平四夷,尽去后患,而量力度德,慨然有舍我其谁之想。于是承累朝之培养,既庶且富,相时而动,战以为守,攻以为御,匈奴远遁,日以削弱。至于宣、元、成、哀,单于称臣,稽玄而朝,两汉之生灵,并受其福,庙号‘世宗’,宜哉!」

夏曾佑:“有为汉一朝之皇帝者,高祖是也;有为中国二十四朝之皇帝者,秦皇、汉武是也。”

白寿彝:“促进了经济繁荣与国家统一。”

翦伯赞:“用剑犹如用情,用情犹如用兵”。

黄仁宇:“有专制魔王的毛病。”

钱穆:“‘王莽代汉’源自汉武帝种下的恶果。”

孙中山:““秦皇汉武、元世祖、拿破仑,或数百年,数十年而斩,亦可谓有志之士矣。拿破仑兴法典,汉武帝纪赞,不言武功,又有千年之志者。”

毛泽东:“汉武帝雄才大略,开拓刘邦的业绩,晚年自知奢侈、黩武、方士之弊,下了罪己诏,不失为鼎盛之世。”

根據《史記》和《漢書》的描述,漢武帝為雙性戀。记载于史书上的佞幸(有公职或者贵族身份的男性情人)有韩嫣、李延年和韩说。《佞幸列傳》紀錄李延年“與上臥起,甚貴幸。”大臣金日磾之子亦曾經為弄儿(少年男性情人)。

《天地陰陽交歡大樂賦》、《藝文類聚·寵幸》、《情史·情外類》引述史書裡,漢武帝寵幸的韓嫣記載,視之為男風代表。

汉武帝巡游汾河,在船上和群臣飲宴,汉武帝突然对群臣说:“汉朝有六七之厄,六七四十二,汉朝传到第42代皇帝,会有当涂高取代汉朝。”群臣说:“汉朝应天受命,王朝长过商周,永世不绝,陛下为何说这种亡国之言?”汉武帝表示「只是醉言,但是自古以来没有一姓可以一直拥有天下,不过即使汉朝灭亡,不要灭亡在我父子手上就行。」

当涂高的意思是路上有很高的东西,后来的公孙述、袁术和曹丕等都用「代汉者当涂高」这句谶言为自己称帝造势。

汉武帝建元年间,汉武帝和随从微服外出打猎,麻烦事不断。一天夜晚汉武帝和随从投宿旅社,旅社主人觉得一行人来者不善,对汉武帝等人非常傲慢。旅社主人准备和门客一同杀死汉武帝等人,但是主人妻子觉得汉武帝等人气势非凡,不像强盗,于是将她丈夫灌醉,偷偷放走汉武帝等人。后来又不慎踩伤农民庄稼,引发纠纷,农民叫来县令。汉武帝自称平阳侯,县令本想拜谒,汉武帝随从却想鞭打县令。县令大怒,扣押汉武帝随从,拒绝他们离开。汉武帝不得已,向县令展示皇家身份,县令才予以放行。后来汉武帝微服外出的举动被众人得知,地方政府纷纷建立行宫招待汉武帝。汉武帝认为微服外出会扰民,干脆建立上林苑,专供皇家打猎。

\subsection{建元}

\begin{longtable}{|>{\centering\scriptsize}m{2em}|>{\centering\scriptsize}m{1.3em}|>{\centering}m{8.8em}|}
  % \caption{秦王政}\
  \toprule
  \SimHei \normalsize 年数 & \SimHei \scriptsize 公元 & \SimHei 大事件 \tabularnewline
  % \midrule
  \endfirsthead
  \toprule
  \SimHei \normalsize 年数 & \SimHei \scriptsize 公元 & \SimHei 大事件 \tabularnewline
  \midrule
  \endhead
  \midrule
  元年 & -140 & \tabularnewline\hline
  二年 & -139 & \tabularnewline\hline
  三年 & -138 & \tabularnewline\hline
  四年 & -137 & \tabularnewline\hline
  五年 & -136 & \tabularnewline\hline
  六年 & -135 & \tabularnewline
  \bottomrule
\end{longtable}


\subsection{元光}

\begin{longtable}{|>{\centering\scriptsize}m{2em}|>{\centering\scriptsize}m{1.3em}|>{\centering}m{8.8em}|}
  % \caption{秦王政}\
  \toprule
  \SimHei \normalsize 年数 & \SimHei \scriptsize 公元 & \SimHei 大事件 \tabularnewline
  % \midrule
  \endfirsthead
  \toprule
  \SimHei \normalsize 年数 & \SimHei \scriptsize 公元 & \SimHei 大事件 \tabularnewline
  \midrule
  \endhead
  \midrule
  元年 & -134 & \tabularnewline\hline
  二年 & -133 & \tabularnewline\hline
  三年 & -132 & \tabularnewline\hline
  四年 & -131 & \tabularnewline\hline
  五年 & -130 & \tabularnewline\hline
  六年 & -129 & \tabularnewline
  \bottomrule
\end{longtable}


\subsection{元朔}

\begin{longtable}{|>{\centering\scriptsize}m{2em}|>{\centering\scriptsize}m{1.3em}|>{\centering}m{8.8em}|}
  % \caption{秦王政}\
  \toprule
  \SimHei \normalsize 年数 & \SimHei \scriptsize 公元 & \SimHei 大事件 \tabularnewline
  % \midrule
  \endfirsthead
  \toprule
  \SimHei \normalsize 年数 & \SimHei \scriptsize 公元 & \SimHei 大事件 \tabularnewline
  \midrule
  \endhead
  \midrule
  元年 & -128 & \tabularnewline\hline
  二年 & -127 & \tabularnewline\hline
  三年 & -126 & \tabularnewline\hline
  四年 & -125 & \tabularnewline\hline
  五年 & -124 & \tabularnewline\hline
  六年 & -123 & \tabularnewline
  \bottomrule
\end{longtable}

\subsection{元狩}

\begin{longtable}{|>{\centering\scriptsize}m{2em}|>{\centering\scriptsize}m{1.3em}|>{\centering}m{8.8em}|}
  % \caption{秦王政}\
  \toprule
  \SimHei \normalsize 年数 & \SimHei \scriptsize 公元 & \SimHei 大事件 \tabularnewline
  % \midrule
  \endfirsthead
  \toprule
  \SimHei \normalsize 年数 & \SimHei \scriptsize 公元 & \SimHei 大事件 \tabularnewline
  \midrule
  \endhead
  \midrule
  元年 & -122 & \tabularnewline\hline
  二年 & -121 & \tabularnewline\hline
  三年 & -120 & \tabularnewline\hline
  四年 & -119 & \tabularnewline\hline
  五年 & -118 & \tabularnewline\hline
  六年 & -117 & \tabularnewline  
  \bottomrule
\end{longtable}

\subsection{元鼎}

\begin{longtable}{|>{\centering\scriptsize}m{2em}|>{\centering\scriptsize}m{1.3em}|>{\centering}m{8.8em}|}
  % \caption{秦王政}\
  \toprule
  \SimHei \normalsize 年数 & \SimHei \scriptsize 公元 & \SimHei 大事件 \tabularnewline
  % \midrule
  \endfirsthead
  \toprule
  \SimHei \normalsize 年数 & \SimHei \scriptsize 公元 & \SimHei 大事件 \tabularnewline
  \midrule
  \endhead
  \midrule
  元年 & -116 & \tabularnewline\hline
  二年 & -115 & \tabularnewline\hline
  三年 & -114 & \tabularnewline\hline
  四年 & -113 & \tabularnewline\hline
  五年 & -112 & \tabularnewline\hline
  六年 & -111 & \tabularnewline  
  \bottomrule
\end{longtable}

\subsection{元封}

\begin{longtable}{|>{\centering\scriptsize}m{2em}|>{\centering\scriptsize}m{1.3em}|>{\centering}m{8.8em}|}
  % \caption{秦王政}\
  \toprule
  \SimHei \normalsize 年数 & \SimHei \scriptsize 公元 & \SimHei 大事件 \tabularnewline
  % \midrule
  \endfirsthead
  \toprule
  \SimHei \normalsize 年数 & \SimHei \scriptsize 公元 & \SimHei 大事件 \tabularnewline
  \midrule
  \endhead
  \midrule
  元年 & -110 & \tabularnewline\hline
  二年 & -109 & \tabularnewline\hline
  三年 & -108 & \tabularnewline\hline
  四年 & -107 & \tabularnewline\hline
  五年 & -106 & \tabularnewline\hline
  六年 & -105 & \tabularnewline
  \bottomrule
\end{longtable}

\subsection{太初}

\begin{longtable}{|>{\centering\scriptsize}m{2em}|>{\centering\scriptsize}m{1.3em}|>{\centering}m{8.8em}|}
  % \caption{秦王政}\
  \toprule
  \SimHei \normalsize 年数 & \SimHei \scriptsize 公元 & \SimHei 大事件 \tabularnewline
  % \midrule
  \endfirsthead
  \toprule
  \SimHei \normalsize 年数 & \SimHei \scriptsize 公元 & \SimHei 大事件 \tabularnewline
  \midrule
  \endhead
  \midrule
  元年 & -104 & \tabularnewline\hline
  二年 & -103 & \tabularnewline\hline
  三年 & -102 & \tabularnewline\hline
  四年 & -101 & \tabularnewline
  \bottomrule
\end{longtable}

\subsection{天汉}

\begin{longtable}{|>{\centering\scriptsize}m{2em}|>{\centering\scriptsize}m{1.3em}|>{\centering}m{8.8em}|}
  % \caption{秦王政}\
  \toprule
  \SimHei \normalsize 年数 & \SimHei \scriptsize 公元 & \SimHei 大事件 \tabularnewline
  % \midrule
  \endfirsthead
  \toprule
  \SimHei \normalsize 年数 & \SimHei \scriptsize 公元 & \SimHei 大事件 \tabularnewline
  \midrule
  \endhead
  \midrule
  元年 & -100 & \tabularnewline\hline
  二年 & -99 & \tabularnewline\hline
  三年 & -98 & \tabularnewline\hline
  四年 & -97 & \tabularnewline
  \bottomrule
\end{longtable}

\subsection{太始}

\begin{longtable}{|>{\centering\scriptsize}m{2em}|>{\centering\scriptsize}m{1.3em}|>{\centering}m{8.8em}|}
  % \caption{秦王政}\
  \toprule
  \SimHei \normalsize 年数 & \SimHei \scriptsize 公元 & \SimHei 大事件 \tabularnewline
  % \midrule
  \endfirsthead
  \toprule
  \SimHei \normalsize 年数 & \SimHei \scriptsize 公元 & \SimHei 大事件 \tabularnewline
  \midrule
  \endhead
  \midrule
  元年 & -96 & \tabularnewline\hline
  二年 & -95 & \tabularnewline\hline
  三年 & -94 & \tabularnewline\hline
  四年 & -93 & \tabularnewline
  \bottomrule
\end{longtable}

\subsection{征和}

\begin{longtable}{|>{\centering\scriptsize}m{2em}|>{\centering\scriptsize}m{1.3em}|>{\centering}m{8.8em}|}
  % \caption{秦王政}\
  \toprule
  \SimHei \normalsize 年数 & \SimHei \scriptsize 公元 & \SimHei 大事件 \tabularnewline
  % \midrule
  \endfirsthead
  \toprule
  \SimHei \normalsize 年数 & \SimHei \scriptsize 公元 & \SimHei 大事件 \tabularnewline
  \midrule
  \endhead
  \midrule
  元年 & -92 & \tabularnewline\hline
  二年 & -91 & \tabularnewline\hline
  三年 & -90 & \tabularnewline\hline
  四年 & -89 & \tabularnewline
  \bottomrule
\end{longtable}

\subsection{后元}

\begin{longtable}{|>{\centering\scriptsize}m{2em}|>{\centering\scriptsize}m{1.3em}|>{\centering}m{8.8em}|}
  % \caption{秦王政}\
  \toprule
  \SimHei \normalsize 年数 & \SimHei \scriptsize 公元 & \SimHei 大事件 \tabularnewline
  % \midrule
  \endfirsthead
  \toprule
  \SimHei \normalsize 年数 & \SimHei \scriptsize 公元 & \SimHei 大事件 \tabularnewline
  \midrule
  \endhead
  \midrule
  元年 & -88 & \tabularnewline\hline
  二年 & -87 & \tabularnewline
  \bottomrule
\end{longtable}


%%% Local Variables:
%%% mode: latex
%%% TeX-engine: xetex
%%% TeX-master: "../Main"
%%% End:
