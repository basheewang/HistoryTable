%% -*- coding: utf-8 -*-
%% Time-stamp: <Chen Wang: 2019-12-17 11:54:38>

\section{刘婴\tiny(6-8)}

\subsection{生平}

劉婴(5年-25年2月),西汉末代皇太子(6年4月17日-9年1月10日為皇太子,未正式即位),号孺子,是汉宣帝的玄孙、楚孝王劉囂曾孙、广戚煬侯劉勳孫、广戚侯刘显子。

元始五年(5年),汉朝外戚王莽毒死汉平帝,次年从汉朝宗室中挑选时年2岁的刘婴。但是因為年齡太小,未正式即位,僅為「皇太子」。王莽自称“摄皇帝”,任何排場實與皇帝無異,僅在見孺子及太后時需自稱臣。為了實質控制朝政大权,完全摄政,以周公、伊尹自居,改元“居摄”。而刘婴這個皇太子則其實只是傀儡。

初始元年十一月甲子(9年1月6日),王莽將「攝皇帝」稱號改稱「假皇帝」;十一月戊辰(1月10日),王莽自称汉太祖刘邦要他做皇帝,便称帝,建国号“新”,尊太皇太后王政君为皇太后,刘婴为定安公。至此,立国二百一十一年的西汉王朝结束。

王莽将年幼的刘婴养在高墙府第之中,与外界隔绝任何联系,甚至乳母也不被允许和他讲话,导致刘婴成人后不识六畜,知识面与幼儿无异。王莽将自己的孙女、王宇的女儿王氏嫁给他做妻子。

更始二年(24年),王莽为更始帝刘玄所败,刘婴当时身在长安。平陵人方望等人依据天象认为更始帝必败,而刘婴才是天子正统,便起兵将刘婴迎至临泾,拥立为天子。更始三年(25年),更始帝派遣丞相李松進攻临泾,刘婴被杀。

\subsection{居摄}

\begin{longtable}{|>{\centering\scriptsize}m{2em}|>{\centering\scriptsize}m{1.3em}|>{\centering}m{8.8em}|}
  % \caption{秦王政}\
  \toprule
  \SimHei \normalsize 年数 & \SimHei \scriptsize 公元 & \SimHei 大事件 \tabularnewline
  % \midrule
  \endfirsthead
  \toprule
  \SimHei \normalsize 年数 & \SimHei \scriptsize 公元 & \SimHei 大事件 \tabularnewline
  \midrule
  \endhead
  \midrule
  元年 & 6 & \tabularnewline\hline
  二年 & 7 & \tabularnewline\hline
  三年\\初始 & 8 & \tabularnewline
  \bottomrule
\end{longtable}


%%% Local Variables:
%%% mode: latex
%%% TeX-engine: xetex
%%% TeX-master: "../Main"
%%% End:
