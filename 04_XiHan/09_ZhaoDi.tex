%% -*- coding: utf-8 -*-
%% Time-stamp: <Chen Wang: 2021-10-29 17:26:39>

\section{昭帝劉弗陵\tiny(BC87-BC74)}

\subsection{生平}

漢昭帝劉弗陵(前94年-前74年6月5日),西漢第八位皇帝(前87年—前74年在位),其正式諡號為「孝昭皇帝」,後世省略「孝」字稱「漢昭帝」,漢武帝幼子,母親鉤弋夫人。昭帝身高八尺二寸(約1.89米)。

據說鉤弋夫人懷孕14個月才生下劉弗陵,大臣們都以為堯帝降生,紛紛恭祝武帝。武帝老年得子,更是愛不釋手,常说像自己。巫蛊之祸兴起,征和二年(前91年),刘弗陵的长兄太子刘据兵败逃亡后自杀。

武帝駕崩前,準備立劉弗陵為太子,但是為了防止「子幼母壯」、外戚專權的事情發生,他藉故處死了鉤弋夫人,然後請得力大將霍去病的異母弟霍光為首輔、匈奴人金日磾為次輔、上官桀為佐軍以及桑弘羊為理財等四重臣來輔佐劉弗陵。武帝駕崩後,劉弗陵在重臣的擁立下登基繼位。

为了便于臣民避讳,昭帝去掉名中的“陵”字,改名刘弗,并讓臣民把弗寫成不。

昭帝登基時才8歲,平時受到鄂邑公主照顧,并從蔡義和韦贤那裡學習《詩經》。

漢昭帝和金日磾兒子金賞、金建關係十分友好,平時一起寢居,漢昭帝看到金賞繼承金日磾的侯爵,想封金建為侯,卻遭到霍光拒絕。

面對漢武帝時代的連年征戰、增加徭役,昭帝聽取重臣的建言,減少賦稅3成,進一步深化武帝晚年重新施行漢初與民休息的政策。在首輔大臣霍光的主持下,昭帝朝的百姓生活比以前富裕,四夷來朝,使漢朝出現中興穩定的局面。

霍光外孫女上官氏當上皇后,霍光想讓皇后擅寵生子,於是不讓漢昭帝親近其他宮女。

前74年6月5日(四月癸未),昭帝於未央宮暴病而崩,年僅21歲,在位13年。7月24日(六月壬申),漢昭帝葬於今天咸阳市的平陵。

昭帝无子,其侄昌邑王刘贺被立为嗣。

漢昭帝一次遊覽渭河,他的隨行大臣釣上了一頭白蛟,長三丈。漢昭帝饒有興趣,讓廚師將其醃製,味道鮮美,漢昭帝飯後回味無窮。之後卻再也沒釣到這種魚。

始元元年(前86年),漢昭帝在太液池上看見黄鹄,於是創作了黄鹄歌。同年,淋池修建,後來漢昭帝在淋池中遊樂,并讓宮人唱《淋池歌》,很是愉快。

\subsection{始元}

\begin{longtable}{|>{\centering\scriptsize}m{2em}|>{\centering\scriptsize}m{1.3em}|>{\centering}m{8.8em}|}
  % \caption{秦王政}\
  \toprule
  \SimHei \normalsize 年数 & \SimHei \scriptsize 公元 & \SimHei 大事件 \tabularnewline
  % \midrule
  \endfirsthead
  \toprule
  \SimHei \normalsize 年数 & \SimHei \scriptsize 公元 & \SimHei 大事件 \tabularnewline
  \midrule
  \endhead
  \midrule
  元年 & -86 & \tabularnewline\hline
  二年 & -85 & \tabularnewline\hline
  三年 & -84 & \tabularnewline\hline
  四年 & -83 & \tabularnewline\hline
  五年 & -82 & \tabularnewline\hline
  六年 & -81 & \tabularnewline\hline
  七年 & -80 & \tabularnewline
  \bottomrule
\end{longtable}


\subsection{元凤}

\begin{longtable}{|>{\centering\scriptsize}m{2em}|>{\centering\scriptsize}m{1.3em}|>{\centering}m{8.8em}|}
  % \caption{秦王政}\
  \toprule
  \SimHei \normalsize 年数 & \SimHei \scriptsize 公元 & \SimHei 大事件 \tabularnewline
  % \midrule
  \endfirsthead
  \toprule
  \SimHei \normalsize 年数 & \SimHei \scriptsize 公元 & \SimHei 大事件 \tabularnewline
  \midrule
  \endhead
  \midrule
  元年 & -80 & \tabularnewline\hline
  二年 & -79 & \tabularnewline\hline
  三年 & -78 & \tabularnewline\hline
  四年 & -77 & \tabularnewline\hline
  五年 & -76 & \tabularnewline\hline
  六年 & -75 & \tabularnewline
  \bottomrule
\end{longtable}


\subsection{元平}

\begin{longtable}{|>{\centering\scriptsize}m{2em}|>{\centering\scriptsize}m{1.3em}|>{\centering}m{8.8em}|}
  % \caption{秦王政}\
  \toprule
  \SimHei \normalsize 年数 & \SimHei \scriptsize 公元 & \SimHei 大事件 \tabularnewline
  % \midrule
  \endfirsthead
  \toprule
  \SimHei \normalsize 年数 & \SimHei \scriptsize 公元 & \SimHei 大事件 \tabularnewline
  \midrule
  \endhead
  \midrule
  元年 & -74 & \tabularnewline
  \bottomrule
\end{longtable}


%%% Local Variables:
%%% mode: latex
%%% TeX-engine: xetex
%%% TeX-master: "../Main"
%%% End:
