%% -*- coding: utf-8 -*-
%% Time-stamp: <Chen Wang: 2019-12-16 17:11:12>

\section{元帝\tiny(BC48-BC33)}

\subsection{生平}

汉元帝劉\xpinyin*{奭}(前75年-前33年7月8日),西汉第十一位皇帝,其正式諡號為「孝元皇帝」,後世省略「孝」字稱「漢元帝」。汉宣帝长子,生于民间,母恭哀皇后许平君。宣帝死后继位,在位16年即前49年-前33年。

前74年,父亲刘询登基,当时他年仅两岁。本始三年(前71年),母亲皇后许平君被霍成君的母亲显毒死。地節三年(前67年)八歲的刘奭被宣帝立为太子。继母霍成君则试图毒死他,但未能成功。因为他曾经向父亲宣帝进言“持刑太深,宜用儒生”,而不被宣帝所喜爱。宣帝甚至预言“乱我家者,必太子也”,但顾念他是发妻许平君的儿子而没有褫夺他的太子之位。

宣帝病死后继位,第二年(前48年)改年号为“初元”,在位时期“崇尚儒术”,多次出兵击溃匈奴。建昭三年(前36年),汉将甘延寿、陈汤诛郅支单于于康居(郅支之战)。至此,唯一反汉的匈奴单于被消灭了。汉匈百年大战于此告一段落。竟宁元年(前33年),匈奴呼韩邪单于入朝求亲。刘奭以宫女王嫱(王昭君)嫁之为妻。

此时的汉朝比较强盛,但也是衰落的起点。刘奭在位期间,純任德教,在儒臣的要求下以不符合儒家心目中的古礼为由废除郡国庙,废除迁关东豪强至关中帝陵置邑制度,豪强大地主兼并之风盛行,社会危机日益加深。又由于汉元帝过于放纵外戚、宦官,最终汉元帝皇后王政君侄子王莽代汉称帝。

竟宁元年五月壬辰(前33年7月8日),病死于长安未央宫,终年四十二岁。七月丙戌(8月31日),葬于渭陵(今陕西咸阳市东北12里处)。死后庙号高宗(后被取消),谥号孝元皇帝。次年,长子刘骜登基,是为汉成帝。

班固《漢書·卷九·元帝紀第九》:“壯大,柔仁好儒”。 “臣外祖兄弟為元帝侍中,語臣曰:元帝多材藝,善史書。鼓琴瑟,吹洞簫,自度曲,被歌聲,分刌節度,窮極幼眇。少而好儒,及即位,徵用儒生,委之以政,貢、薛、韋、匡迭為宰相。而上牽製文義,優游不斷,孝宣之業衰焉。然寬弘盡下,出於恭儉,號令溫雅,有古之風烈。”

\subsection{初元}

\begin{longtable}{|>{\centering\scriptsize}m{2em}|>{\centering\scriptsize}m{1.3em}|>{\centering}m{8.8em}|}
  % \caption{秦王政}\
  \toprule
  \SimHei \normalsize 年数 & \SimHei \scriptsize 公元 & \SimHei 大事件 \tabularnewline
  % \midrule
  \endfirsthead
  \toprule
  \SimHei \normalsize 年数 & \SimHei \scriptsize 公元 & \SimHei 大事件 \tabularnewline
  \midrule
  \endhead
  \midrule
  元年 & -48 & \tabularnewline\hline
  二年 & -47 & \tabularnewline\hline
  三年 & -46 & \tabularnewline\hline
  四年 & -45 & \tabularnewline\hline
  五年 & -44 & \tabularnewline
  \bottomrule
\end{longtable}


\subsection{永光}

\begin{longtable}{|>{\centering\scriptsize}m{2em}|>{\centering\scriptsize}m{1.3em}|>{\centering}m{8.8em}|}
  % \caption{秦王政}\
  \toprule
  \SimHei \normalsize 年数 & \SimHei \scriptsize 公元 & \SimHei 大事件 \tabularnewline
  % \midrule
  \endfirsthead
  \toprule
  \SimHei \normalsize 年数 & \SimHei \scriptsize 公元 & \SimHei 大事件 \tabularnewline
  \midrule
  \endhead
  \midrule
  元年 & -43 & \tabularnewline\hline
  二年 & -42 & \tabularnewline\hline
  三年 & -41 & \tabularnewline\hline
  四年 & -40 & \tabularnewline\hline
  五年 & -39 & \tabularnewline
  \bottomrule
\end{longtable}


\subsection{建昭}

\begin{longtable}{|>{\centering\scriptsize}m{2em}|>{\centering\scriptsize}m{1.3em}|>{\centering}m{8.8em}|}
  % \caption{秦王政}\
  \toprule
  \SimHei \normalsize 年数 & \SimHei \scriptsize 公元 & \SimHei 大事件 \tabularnewline
  % \midrule
  \endfirsthead
  \toprule
  \SimHei \normalsize 年数 & \SimHei \scriptsize 公元 & \SimHei 大事件 \tabularnewline
  \midrule
  \endhead
  \midrule
  元年 & -38 & \tabularnewline\hline
  二年 & -37 & \tabularnewline\hline
  三年 & -36 & \tabularnewline\hline
  四年 & -35 & \tabularnewline\hline
  五年 & -34 & \tabularnewline
  \bottomrule
\end{longtable}

\subsection{竟宁}

\begin{longtable}{|>{\centering\scriptsize}m{2em}|>{\centering\scriptsize}m{1.3em}|>{\centering}m{8.8em}|}
  % \caption{秦王政}\
  \toprule
  \SimHei \normalsize 年数 & \SimHei \scriptsize 公元 & \SimHei 大事件 \tabularnewline
  % \midrule
  \endfirsthead
  \toprule
  \SimHei \normalsize 年数 & \SimHei \scriptsize 公元 & \SimHei 大事件 \tabularnewline
  \midrule
  \endhead
  \midrule
  元年 & -33 & \tabularnewline
  \bottomrule
\end{longtable}


%%% Local Variables:
%%% mode: latex
%%% TeX-engine: xetex
%%% TeX-master: "../Main"
%%% End:
