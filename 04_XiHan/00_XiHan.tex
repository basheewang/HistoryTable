%% -*- coding: utf-8 -*-
%% Time-stamp: <Chen Wang: 2019-12-16 11:37:04>

\chapter{西汉\tiny(BC202-8)}

\section{简介}


西汉(前206年或前202年-9年、23-25年)是中国歷史上的一个朝代,是汉朝的一部分。前206年刘邦被西楚霸王分封为汉王而建立政權,后经过历时四年的楚汉战争,刘邦取胜后,前202年最终统一天下,在定陶县 (今属山东省)称帝,定都洛阳,国号为“汉”后都长安,项羽以“巴蜀汉中四十一县”封刘邦,以治所在汉中称“汉王”,称帝后遂以封国名为王朝名。又刘邦都城长安位于东汉时期都城洛陽之西,为加以区别,故史称“西汉”,有时又以都城与东汉时的都城洛阳的相对位置代称为“西京”。而西汉在东汉之前,因此历史上又称前者为“前汉”。诸葛亮在《出师表》中亦将西汉称呼为“先汉”。

西汉建立后,刘邦废秦苛法,实施无为而治。减轻赋稅和徭役,釋放部分奴婢,抑制富商,限制土地兼併,并且獎勵開荒,使汉初经济得到恢复。文帝、景帝时继续重视农业,盐铁等手工业及商业也有发展。

西汉初期分封诸侯王,而后地方诸侯王势力膨胀,最终导致在景帝时出现了「七国之乱」,平叛后诸侯王势力被削弱。

武帝时是西汉的鼎盛时期。国家统一货币,铸五铢钱,严禁私铸钱,铸铁业实行国家专营,实行均输法、平准法,朝廷财政收入大增。在这基础上,武帝采取了积极的对外政策。北方匈奴长期以来是汉朝主要边患,武帝发动了三次战役打击匈奴,使匈奴远徙漠北,保证了河西走廊的安全。又在西北边地屯田,修长城,筑烽燧,并派张骞出使西域,打通了汉朝通往中亚的贸易通道。丝绸之路的开辟,促进欧亚大陆文化贸易的交流。武帝还采用儒生董仲舒建议,罢黜百家独尊儒术,教授五经,使经学成为食禄的工具。当时又建立藏书库,整理图籍,使文化得到发展。史学家司马迁写下了中国第一部纪传体通史《史记》。

昭帝、宣帝时,出现外戚专政,豪强势力增强,流民增多。元帝以后,宦官和外戚更加得势,百姓起义不断出现。成帝时太后的外戚王氏控制了政权,王氏兄弟四人和侄王莽相继为大司马大将军。哀帝时,王莽企图用「再受命」的办法来解决危机,结果失势。平帝时,王莽复起,通过一系列手段来为支持自己篡位夺权做准备。平帝病死,孺子刘婴立,王莽见有机可乘,于初始元年(公元9年)废孺子婴,自立为皇帝,改国号为「新」,西汉亡。西汉一共210年。假如不算少帝、昌邑王(废帝)刘贺等一些在位时间较短且为吕雉等人的傀儡皇帝的话,西汉一共经历了11代皇帝、12代君主(因为孺子婴没有当上皇帝,仅仅当了3年的皇太子)。也有把新朝灭亡后在長安重建漢室的更始帝算作西汉最後一位皇帝,但劉秀建立的東漢並不視其為漢朝皇帝,因此歷史說法存在爭議。

西汉极盛时的疆域东、南到海,西到今巴尔喀什湖、费尔干纳盆地、葱岭一线,西南到今云南、广西以及今越南中部,北接大漠,东北至今朝鲜半岛北部。

西汉时期随着丝绸之路的开辟,对外贸易繁荣,耕地扩大,冶金和纺织技术先进,现在巩县铁生沟遗址发现的低温炒钢炉在世界冶金技术史上有着重要的地位。

西汉是中华文化的高峰,通过丝绸之路和各国往来频繁,成为当时世界上最强盛的国家之一。西汉定都长安(今陕西省西安市西北)。陵寝遍布关中,文物遗存十分丰富,显示出“汉并天下”的时代风貌。汉朝后华夏族出现了新的自称“汉族”。

秦朝實施源自商鞅变法時的苛法和高賦稅,使百姓對秦朝不滿。随后陳勝、吳廣的農民起義,各豪強和六國的舊貴族趁势割據地方,劉邦和項羽也加入其中。同時,在沛县任泗水亭長〈即掌管治安之小吏,泗水亭直属四川郡,其亭在沛县境内。〉的劉邦起义,途中遇到項梁,便征其麾下。

西元前207年10月,劉邦攻入咸陽,秦帝子嬰投降。項羽于12月攻入咸陽杀子嬰,之後殺楚義帝,自封為「西楚霸王」號令全國,举办鸿门宴分封諸侯,他將滅秦的劉邦封在蜀地,劉邦定國號為漢,建立漢朝。不久,劉邦和項羽矛盾激化,楚汉战争爆发。由于刘邦知人善用,重用蕭何、張良、陳平等謀士,启用將領韓信。在垓下之戰中,打敗項羽,項羽在烏江自刎,楚汉之战结束。

前202年,劉邦正式稱帝,建國號漢。定都洛阳后,五月在咸陽地定都长安,西汉建立。

西汉初年,由于连年战乱,经济凋敝,造成大量流民。汉初人口,與秦代相比大为减少,大城市人口尺剩下十分之二三。出现「自天子不能具钧驷,而将相或乘牛车,齐民无藏盖」(天子的騎乘找不到四隻顏色相同的馬,而將與相只能乘坐牛車)的现象。 在这种情况下,劉邦遂採取「黃老治術」、「無為而治」的理念治理国家,目的在與民修養身息。

首先,在政治上採取「郡國制」,郡县和封国并存。皇帝分封侯国和王国,其中侯国只享有封地内的税收,无军事和行政权,并受郡的管辖,而王国则拥有独立的政治和军事权力。另外,在经济對內注意興修水利,減免賦稅,组织军队复员,军队官兵复员为民。招抚流亡,令战争期间流亡山泽不着户籍的人口,各归原籍,“复故爵田宅”。释放奴婢。诏令规定:因饥饿而自卖为人奴婢者,“皆免为庶人”。这些举措恢复农业;對外則和親匈奴,維持邊區和平。

劉邦的這一系列整治措施維持几十年的和平,使汉朝国力得以恢复。但是引致一系列問題:對內的輕徭薄賦政策,令地方上一些豪強勢力日大,土地兼併现象严重;對外則匈奴背信弃义的寇掠边境日频,威脅边境。

在執行這些政策之餘,刘邦也對在楚漢相爭中有大功的異姓諸侯王起了猜忌之心。早在楚汉战争中,刘邦为了打败项羽,曾分封韩信、英布、彭越等一些重要将领为王。汉初,被封的异姓王有七个。此外,还封了功臣萧何等一百四十多人为列侯。这些异姓王的存在,对中央集权是一个严重威胁。韓信就是在這時候被蕭何用計由呂后處死,随后彭越、英布等人也相继被铲除。異姓諸侯王所剩无几,取而代之的是劉姓諸侯王,陆续分封了九个刘姓子弟为王。劉邦在白马之盟时对众人說了一句話:「非劉氏而王者,天下共擊之。」其时封国的存在,对中央集权必然是个离心力。随着经济的恢复和发展,诸王的势力日益膨胀,诸王掌握着封国内的征收赋税、任免官吏、铸造钱币等政治、经济大权。形成“尾大不掉”之势。

前195年,劉邦在討伐英布叛亂時受傷,不治身亡。

劉邦死後,呂太后漸漸掌握朝政。繼位的太子漢惠帝因受到呂后的刺激,不理政事。惠帝死後,呂太后連立兩傀儡皇帝,並漸漸削弱劉氏宗室,並封諸呂為王,掌權長達八年。

吕后虽然在高层斗争中手段残酷,但是在国事和人事方面,沿用刘邦生前的既定方针和政策。经济和社会进一步向良性发展。与此同时,吕后也开始进行恢复文化发展,她废除了秦始皇时制定的《挟书律》,允许民间学术团体存在并教授,允许和鼓励民间藏书,设立国家性质的图书机构,凡向官府献书者一律给予奖励。

汉初基本上沿用萧何所删定的《秦律》即《九章律》,虽然与秦律相比要改善许多,但是《九章律》仍然留下《秦律》中许多严苛的条款。吕后下令对这些条款进行再次厘定,废除许多条款,如族诛、连坐等条款都予以废除,同时还减轻大量刑罚的处罚标准。

呂后死後,周勃和陳平奪禁軍權,斬殺呂產等人,才清除呂氏在朝中的勢力。

呂太后死後,由於諸呂掌握兵權,而功臣更不滿呂氏專權,太尉周勃、丞相陳平施計奪取呂氏的兵權。由於漢高祖只餘下兩個兒子,朝臣以淮南王母家趙氏強橫,代王母家薄家善良,故以呂太后所立的傀儡皇帝非惠帝親生為由,迎立代王恆即帝位,是為漢文帝。

文帝时,提倡以农为本,多次发布诏令劝农,以发展农业经济。公元前167年七月,文帝下诏「除田之租税」;公元前156年六月,景帝「令田半租」,即三十税一,并成为汉朝的定制。文景时又减少地方的徭役、卫卒,停止郡国岁贡,开放山泽禁苑给贫民耕种;并颁布了赈贷鳏寡孤独的法令。这些措施的施行,减轻了农民的负担。文景之世,「流民既归,户口亦息」,粮价也大大降低,「谷至石数十钱」。

后来即位的景帝,也持續此種政策,同时还「惩恶亡秦之政,伦议务在宽厚」。施行「约法省禁」的政策,废除了一些严刑苛法,如妻孥连坐法、断残肢体的肉刑等,并减轻笞刑。所以这个时期,许多官吏断狱从轻,不求细苛,以至有「刑轻于它时而犯法者寡」、「断狱数百,几致刑措」之说。这和秦时「断狱岁以千万数」的惨景形成了鲜明的对比。

文景时期的“与民休息”政策,恢复了经济,社会民生得到改善,因此歷史上將此時期稱為「文景之治」。文景二帝都是黄老思想的推崇者,主张无为而治,休养生息。在此期间,汉朝国力得到增强。

只是,景帝在位期間,聽取晁错的意见,進行削藩,开始削夺王国的一部分土地,划归中央直接管辖。但是在执行的过程中操之过急,吴楚等七国遂于公元前154年(景帝前元三年)举兵叛乱。 吴王刘濞是这次叛乱的主谋,他联合了胶西王、楚王、赵王、济南王、菑川王及胶东王,借口“请诛晁错,以清君侧”,共同起兵。景帝派周勃的兒子太尉周亚夫等率大军前去迎击,同时又杀晁错。但刘濞继续进攻,于是景帝决心讨平叛乱。 周亚夫率领大军迎击吴楚七国叛军。下邑(安徽砀山东)一战,“吴大败,士卒多饥死叛散”。七国之乱的平定,使诸侯王势力受到致命的打击。朝廷设法剥夺各个诸侯王的權力,加强中央集权。

漢武帝在位时期是西漢全盛时期。景帝死後,太子刘彻即位,即汉武帝,一上台便改「無為而治」的治國策略,对内进一步采取了一系列加强皇权的措施, 继续打击诸侯国。

汉武帝采纳主父偃的建议,颁布「推恩令」。规定每个诸侯王除由嫡长子继承王位外,其他诸子都在王国范围内分到封地,作为侯国。从此,「大国不过十余城,小侯不过数十里」。封国越分越小,势力大为削弱。后来,又作左官之律和设附益之法。「左官律」规定,凡在诸侯王国任官者,地位低于中央任命的官吏,并不得进入中央任职。以此限制诸侯王网罗人才。「附益法」严禁封国的官吏与诸侯王串通一气,结党营私,以达到孤立诸侯王的目的。

公元前112年(元鼎五年),汉武帝以诸侯王所献祭的“酎金”成色不好或斤两不足为借口,夺爵、削地者达106人,占当时列侯的半数。至此,王、侯二等封爵制度虽然还存在,但所封王、侯只能「衣食租税」,不得过问封国的政事,封土而不治民。通过这些措施,基本上结束了汉初以来诸侯王割据的局面。

在打击诸侯的同时,汉武帝还在中央提高皇权,采取了限制丞相权力的措施。他亲自过问一切政务,令九卿不通过丞相直接向他奏事。与此同时,提拔了一批中下层官员,作为汉武帝的高级侍从和助手,替他出谋划策。这样,在朝官中有了「中朝」和「外朝」之分。由尚书、中书、侍中等组成的“中朝”成为实际的决策机关,而以丞相为首的外朝官,逐渐成为执行一般政务的机关了。中外朝的形成,显示了皇权的高度集中。

对外则派衛青、霍去病等對外十一次攻打匈奴,期间一度把匈奴驅逐至漠北地區、打通西域,開通「絲綢之路」;召開「鹽鐵會議」,其後將製鹽、制鐵和釀酒的事業收為國有。但武帝對外戰爭的代價是很大的,在武帝十一次打击匈奴後繼續對匈奴攻伐,令匈奴邊患復燃。

武帝中后期官府的稅收再次增加,同時武帝將刑罰再次加嚴,武帝後期因此發生了大规模的流民起义;開銷甚大,因此創立「均輸官」、「平準官」,與民爭利。另外,「罷黜百家,獨尊儒術」,使儒家成為了中國固有的文化潮流,对后世的影响十分深遠。自从汉武帝确定“独尊儒术”以后,儒学成了历朝历代的官学思想。

武帝晚年,由于连年的战争,再加上汉武帝本人好大喜功,十分铺张浪费,人口大量减少,财政处于崩溃边缘,如不进行改革,可能重蹈秦亡的覆辙。汉武帝及時发现了这种状况,于是改弦更张,發表了著名的輪臺之詔,這也表達了漢武帝對自己的深刻反省,重拾汉初与民休養生息的政策,國家也漸漸穩定下來,把汉朝从崩溃边缘挽救回来。后世评价漢武帝:「雖有亡秦之失卻無亡秦之禍」。

武帝末年,由于长时期的兴师暴众和严刑峻法,民众起义不断。汉武帝不得不下轮台罪己之诏,表示要发展生产,与民休息 。武帝死後,由年僅八歲的漢昭帝即位,霍光輔政,政策完全秉承武帝晚年的政策,国力进一步恢复,财政进一步好转,社会得到恢复。昭帝極為聰明,惜21歲即病死。死後,荒唐的劉賀被廢,由漢宣帝即位。

宣帝来自民间,是西汉第二位平民皇帝。宣帝性格与其祖父卫太子刘据很不相同,而与曾祖父汉武帝极为相似。年轻时受到良好的儒家诗书礼乐教育,但为人却极喜斗鸡走狗、游侠,遍游三辅之地,因此对民间的疾苦和社会弊病有极清醒认识。

宣帝上台初期,霍光操控国政,政策一如从前。霍光去世,汉宣帝开始真正主政,秉持“王霸之道杂之”的政治策略。他一方面沿续霍光时期的政策,继续减轻各种税赋和民众的负担,即所谓的“王道”;同时厉行法家政策,整顿吏治,即所谓的“霸道”。

宣帝位期間,匈奴也表示了臣服的意願,甘露三年(前51年)呼韓邪單于以臣子的身分晉見宣帝,漢朝與匈奴的百年大戰,終告落幕。在西域,设立西域都护,把西域三十六国正式纳入汉朝的疆域。

宣帝和先前的昭帝御宇期間,采取了轻徭薄赋、重视吏治、平理刑狱等政策和措施。使西汉又稳定下来。被稱為昭宣中興或昭宣之治。

西汉后期,赋役和土地兼并严重,造成自耕农破产,公元前107年,关东出现流民200万,无户籍者40万。汉宣帝时,因天灾人祸,流民更多,使西汉王朝出现危机。

汉宣帝於43歲時病殂,漢元帝即位。元帝即位以後,一反汉宣帝时期的政策,大力推行儒家空阔不切实际的政策,致使土地兼并之風盛行,吏治也开始败坏,中央集權逐漸削弱。漢元帝时期关东十一个郡国闹水灾,人民相食,漢元帝却只知打猎取乐,「驰骋干戈,纵恣于野」。

后来的漢成帝将政权交由其舅。王氏的權力愈來愈大,自王太后的親戚王鳳以來,全由王氏子姪出任大司馬大將軍,王氏在朝廷的勢力日漸鞏固。王氏得势,更「争为奢侈。赂遗珍宝,四面而至」;「狗马驰逐,大治第室」。其他公卿、近臣,也都奢侈淫逸,醉生梦死。

汉在弘农等郡设立铁官,利用刑徒冶铸铁器,在残酷的奴役下,阳朔三年(前22年),颍川(今河南禹州)铁官徒申屠圣等人发动起义,申屠圣自称将军,夺取武器,杀死官吏,历经九郡;永始三年(前14年),山阳(今山东金乡)铁官徒苏令等人发动起义,夺取武器,杀死官吏,历经十九郡国,沿途释放囚徒,杀死汉东郡太守和汝南都尉。最后被汉军镇压下失败。

公元前9年(绥和元年),王莽继任大司马大将军。漢成帝崩後,成帝皇后趙飛燕聯同太子合力排擠王氏。太子即位是為漢哀帝。把哀帝祖母傅太后及生母丁太后入主宮禁。大司馬王莽見大勢已去,向太皇太后王政君建議暫時退讓,結果王莽辭官回到新野新鄉封國。

公元前1年(元寿二年)漢哀帝死,王氏權力再冒起,王莽复任大司马,并录尚书事,操纵了汉政权。此時,王莽以君子之姿逐漸干預朝政。

一方面王莽排斥异己,另一方面采取了笼络人心的措施:如封汉宗室和功臣的后裔,对于退休的高级官员,终身食原俸的三分之一,扩充太学,增加博士和太学生名额等等。从而取得了一部分王侯、士大夫及士人的拥护。 与此同时,王莽也采取了一些改革措施。如公元2年(元始二年),郡国发生灾害时,他献田三十顷、钱百万,以分配给贫民。王莽又在长安城中建住宅二百区,让贫民居住。这些措施得到了好评。如因王莽不受新野田而上书颂其功德者竟达487000人,各地方官吏也不断向王莽献祥瑞,这样王莽实际上控制了汉朝政权。

最後,王莽弑漢平帝,廢孺子,於公元9年1月10日自立為帝,篡漢為新,建立214年的西漢王朝结束。

23年綠林軍攻入長安,弑王莽,立漢室劉玄為皇帝,改元更始,是為玄漢。但漢更始帝隨即荒政,全國隨即混亂,最終公元25年被赤眉军再次攻陷,漢更始帝被殺,长安宫室从此荒廢於戰火。

%% -*- coding: utf-8 -*-
%% Time-stamp: <Chen Wang: 2021-10-29 17:24:36>

\section{楚汉之争\tiny(BC206-BC203)}

\subsection{简介}


楚汉战争,或稱楚漢相爭,是秦朝滅亡后,项羽和刘邦之间为争夺统治权力而进行的战争。時間一般認定為前206年-前202年,秦朝滅亡之后開始,一直到項羽於烏江邊自刎結束。楚汉战争结束了秦末民變之后短暂的分裂局面,是继秦灭六国之后的又一次中国统一战争。

秦二世元年(公元前209年壬辰年),陳勝、吳廣在大澤鄉起義,天下群雄并起反秦。在此期間,陳勝的部將周章由陳縣略地至戲水附近,吳廣部下周文更是率軍打到離咸陽只有數十公里的戏地。同时,六國舊貴族也趁機反秦復國,武臣重建趙國、韓廣重建燕國,楚國貴族項梁、項羽殺死會稽郡守殷通,率領八千江東子弟兵加入起義,北上渡江作戰。不久,吳廣因內鬨被殺;陳勝也兵敗於滎陽,被叛變的車伕莊賈斬首。其後范增進言「楚雖三戶,亡秦必楚」,項梁就以重建楚國為口號,立楚懷王之孫熊心為王,也稱為「楚懷王」,史家稱為楚後懷王。同時,泗水亭亭長劉邦殺死沛縣縣令而造反,自稱「沛公」(沛縣首領),率部投奔項梁。

公元前208年,項梁在取得对秦軍的连胜后驕傲輕敵,宋義諫,但項梁不聽,導致定陶之戰战敗,項梁戰死。此战后,秦将章邯及王離认为楚军元气大伤,已构不成实质威胁,遂调转枪口,率秦軍40萬北上攻趙國,圍趙王歇於鉅鹿。楚後懷王見項梁兵敗,由盱眙遷都到彭城。趙國屢屢求救,於是楚後懷王派遣宋義为上将军、項羽和范增为副将,率軍6萬北上救趙,同時派劉邦進攻關中,楚後懷王還許諾「先入定關中者王之。」

公元前207年,宋义率军到达战场附近,但宋义懦弱无能,且想观秦赵相斗,逡巡逗留46天不进,贻误了宝贵的战机。項羽见此情形,果断殺死宋義,夺得军权,率軍破釜沉舟,與秦軍爆發鉅鹿之戰,項羽先派英布截斷秦軍糧道,再親率以楚軍为主的諸侯盟軍与秦军决戰。楚军视死如归,九战九捷,大败秦军,俘秦主帥王離,副帥涉間自焚,秦勇將蘇角被项羽亲自擊殺。

同時,劉邦領軍向西攻擊,先攻昌邑,在缺少攻城器具情況下,虽有彭越協助,仍然攻城不下。不久,酈食其投奔劉邦,建議他去攻擊陳留以奪取糧草,攻擊成功。劉邦率軍在中原戰鬥時,項羽正在追擊章邯,章邯當時糧食已缺,派別將司馬欣往求糧,被拒絕後陳餘又送信來,使得章邯猶豫不決。不久,項羽攻克三戶津,截斷章邯退路,章邯派偏軍於漳水南岸戰項羽,大敗。章邯再派軍戰於汙水(漳水的支流),又敗。章邯因得不到秦二世跟趙高的支援被困,后见大势已去,被迫在洹水南殷墟(今河南安阳)率秦軍20萬投降。巨鹿之战后,项羽声威大振,联军会师后,各路起义军将领被他的气势所震慑,甚至不敢与他直接对视。项羽由此被各路诸侯推举为上将军,统兵40万,继续反秦事业。

刘邦来到陈留西约30公里的开封,与秦大将赵贲大战,大破赵贲,三月,砀郡长刘邦自开封向北约100公里至白马擊敗楊熊奪取白馬,再降南陽,佔領武關,在藍田再破秦軍,逼近咸陽,趙高派人偷偷與劉邦商議平分關中被拒絕。同時,二世三年八月,趙高殺害秦二世,擁立子嬰為秦王,但子嬰九月即殺了趙高,取得大權。西元前207年11月,劉邦攻抵灞上,子嬰出降,秦帝国只经过了短暂的15年统一后便灭亡了。劉邦派軍據守函谷關,阻止項羽聯軍入關。不久,項羽以及諸侯40万大軍攻克函谷關,直入關中,駐兵新豐鴻門。此时刘邦只有10万兵马,只得将关中拱手相让。鴻門宴上,范增欲殺劉邦,但劉邦言詞甚懇切,項羽顧及與劉邦的結義之情與諸侯的威脅,放歸劉邦。

西元前206年2月,项羽尊楚后怀王为楚义帝,徙义帝于江南,都郴,後來又刺殺了義帝。西元前206年3月,项羽自封为西楚霸王,楚國疆域為梁楚九郡(今華東、江南大部分地區),建都彭城(江苏徐州),又分封十八路諸侯。刘邦被封为汉王,辖汉中、巴、蜀一带,建都南郑(今陕西汉中)。关中故秦之地三分,封章邯为雍王、司马欣为塞王、董翳为翟王。项羽分封诸侯后即罢兵裁軍,東归彭城。

齐國相田榮因拒隨從率军入关,故而没有得到分封。项羽把齐地一分为三,把齐王田市徙为胶东王,引起田荣对项羽封王方案极为不满,率軍驱逐齐王田都,并阻止齐王田市任胶东王。田市惧怕项羽,欲自行去胶东就封。田荣怒,追斩田市,自立為齊王。隨後,田荣再击杀济北王田安,兼并三齐之地。

赵国舊大将军陈馀与國相張耳在钜鹿之战結怨而去职,將兵權讓給张耳,归隐南皮。项羽分封时闻陈馀贤,与张耳一体有功,但未从入关,因此仅将南皮附近三县封与陈馀。陈馀见张耳为王,而自己功劳与张耳相当,却仅得南皮三县,对此十分不满。田荣兼并三齐,陈馀派夏说为使者使齐借兵。田榮又借兵給陳餘,令陳餘擊敗張耳,重迎赵王歇复为赵王。赵王歇感念陈馀,封陈馀为代王。赵王歇弱,因此陈馀不归代国,自稱趙國太傅,繼續輔赵,陈馀派夏说以代国國相身分,留守代国。

项羽分封时虽据梁國为己有,但梁地时有義軍首領彭越有众万余,彭越未从入关,故亦无分封,无所属。田荣除资助陈馀外,亦封彭越為齊國將軍,令其在梁地起兵破楚。项羽派萧公角击梁,彭越大破楚军。而在齐赵北方,故燕王韩广亦不愿徙王辽东,项羽所封之燕王臧荼击杀韩广,并王辽东。齊趙之反,使得楚國受到威脅,於是項羽決定北伐田榮。

汉元年八月,项羽闻刘邦袭取关中,于是杀韩王成,立亲信故吴令郑昌为韩王以拒汉。初项梁立韩公子成为韩王,张良为韩相国。项羽封诸侯,韩王成仍为韩王,但是项羽以张良从汉王刘邦,而韩王成又无功,故不许韩王成就国。项羽东归,韩王成被带至彭城。

项羽分封後数個月,形势就陷入一片混乱。项羽所确定的新秩序基本上被打破。同时项羽攻齐,向九江徵兵,英布不从,引发项羽不满,项羽数遣使责英布。

在齐地田荣兼并三齐之时,刘邦在汉中也为攻袭三秦做准备。刘邦入汉中,項羽給予刘邦三萬士兵。刘邦依张良计,入南鄭時燒毀棧道,以防被偷襲和向項羽示意無外侵的意願。

项羽帐下的执戟郎中韩信亦在此时從項羽軍中逃出,投靠劉邦,但没有被重用,仅任连敖,后坐法当斩为滕公夏侯婴所救。夏侯婴与之交談,知其有才能,向刘邦推荐韩信,刘邦拜韩信为治粟都尉。韩信自觉不能受到重用,欲离去另寻明主,萧何听闻后連夜苦追,人稱「萧何月下追韩信」故事。萧何再次向刘邦推荐韩信,刘邦拜韩信为大将军,统领三军。

公元前206年八月,劉邦用韓信的計謀,但受阻於陈仓,所幸趙衍指出一條小道。劉邦沿著這條小道進入關中,趁項羽北攻田榮時,突然出现在三秦旧将面前,在好疇擊潰章邯,最後圍困章邯於廢丘。隔年塞王司马欣、翟王董翳被迫向汉王刘邦投降。之后几个月,劉邦率領漢軍攻取隴西、北地、上郡。这样,三秦除章邯困守的废丘之外全部归汉。

而此时因项羽杀韩王成,张良间行归汉,派人遗书项羽,称“漢欲得關中,如約即止,不敢復東。”项羽以故無西意,而北擊齊。

刘邦略取关中时,九月,命令薛歐、王吸出武關,與王陵聯合,迎接劉太公和呂后於沛。十月,汉王拜韩王信为韓國太尉,令其循韩地,并许之若定韩地则拜其为韩王。韩信循韩地,下十余城,项羽所立之韩王郑昌降,汉二年十一月汉立韩王信为韩王。

汉二年十月,汉王刘邦进至陕(今河南陕县)。在汉基本平定关中之后,开始准备东进了。

公元前205年三月,劉邦東攻,迫降塞王司馬欣、翟王董翳、河南王申陽。劉邦渡臨晉,魏王豹率兵跟從,破河內,虜獲殷王司馬卬,南渡河,直抵雒陽。劉邦聞項羽命英布刺殺楚義帝,為義帝發喪,號召天下王公反抗項羽。

劉邦部将张良、陈平、韩信、吕泽、张耳、夏侯婴、樊哙以及五诸侯军,至外黄,击败楚将程处、王武,彭越率三万人归附刘邦,刘邦封彭越为魏国国相,攻打梁地,派樊哙北上攻打邹县、鲁县、薛县、瑕丘,以阻止项羽从齐国南下,向东攻打下邑、派吕泽驻守,下邑在萧县西面不远,萧县在彭城西面不远,项羽南下救援彭城必经萧县,这样,如果项羽回援彭城,吕泽可以与刘邦东西两面夹击项羽。与北路军曹参、灌婴会合,进攻砀县、萧县,攻取彭城。

项羽虽击败齐军,殺死田荣,但项羽暴虐,使齐地降而复叛。田榮之弟田橫立田榮之子田廣為齊王,繼續抗楚,项羽因此深陷齐地而无暇抽身。四月,劉邦與諸侯聯軍號稱五十六萬人,趁虚直捣楚都彭城。此时彭城由项羽军师范增守备,面对敌众我寡的不利态势,范增决定不死守彭城。数周后,在给予汉军以极大杀伤后,范增主动率大量楚军有生力量撤出彭城。

虽只得了一座空城,但以刘邦为首的各诸侯却以为他们取得了大胜,开始麻痹大意起来。入城后,联军日夜欢饮,军纪败坏,戒备松弛。項羽听闻这一情况,决定出奇兵制胜。他亲率精锐骑兵三萬人疾进,令士卒衔枚,马蹄裹布,由曲阜經胡陵,到蕭縣,於清晨攻擊漢軍,至中午大破漢軍。韓信等各路漢軍敗退至谷、泗水,被殲十餘萬人。楚軍一刻不停,繼續追殺,漢軍敗走,於靈壁東邊的睢水上被楚軍驅趕下河,汉军士兵光淹死的就有十數萬,使得睢水被屍體阻塞,河水一度断流。 此即公元前205年的彭城之戰。此战中,西楚霸王项羽充分利用汉军自以为人数众多,麻痹大意的弱点,只率三万精兵便击溃刘邦五十六万大军,是为中国乃至世界战争史中以少胜多之典范。

美中不足的是,彭城之战,项羽虽大胜,但田横亦复定三齐。

刘邦大败于彭城,劉邦收集散兵到下邑。父親劉太公、母親劉媼和妻子呂雉被楚軍擄為人質。諸侯見劉邦潰敗後,重新投奔項羽,連塞王司馬欣和翟王董翳也入楚為將。汉王刘邦以杀死张耳向赵国太傅陳餘結盟,陳餘发现汉王刘邦并没有杀张耳,赵兵退去反与汉为敌,汉王刘邦联盟顿时瓦解。时吕后兄周吕侯将兵居下邑。

张良亦至下邑,与刘邦于下邑规划下一步对策。张良说:“九江王黥布,楚梟將,與項王有郄;彭越與齊王田榮反梁地:此兩人可急使。而漢王之將獨韓信可屬大事,當一面。即欲捐之,捐之此三人,則楚可破也。”刘邦采纳张良建议。

劉邦在彭城之戰後,勢力一落千丈。刘邦在下邑收集散兵后,到達虞(今河南虞城县),派随何出使九江,随何成功游说九江王英布投汉。项羽不得不派龙且分兵攻打英布,牵制了项羽的后方,後汉军往彭城东南边的灵壁溃散,刘邦向西撤退,首先去了下邑与周吕侯吕泽会合,然后向南接应败兵,在砀县驻扎,战败后刘邦及时稳住阵脚,防止了大军继续溃散。刘邦驻守下邑时,马上吸引了项羽亲自来进攻,彭城之战一个月之后,公元前205年五月,刘邦回到关中。漢軍水攻廢丘,雍王章邯在抵抗了十個月后兵敗自殺,劉邦攻占三秦分兵滅魏,劉邦同時北上滅趙。同時,英布與龍且戰爭,不得勝利,與隨何往見劉邦,楚军尽取九江。 劉邦收取士卒,會合關中蕭何派來的援軍,加上韓信率領殘兵敗卒赶来会合,項羽率領楚军亦追击而至,劉邦指揮諸軍並且讓灌嬰率領汉军騎兵於“京县”(今河南郑州荥阳豫龙镇京襄城村附近)、“索亭”(今河南滎陽索河街道)之間擊敗項羽統領的楚軍,将楚軍击退到滎陽以东。此為京索之戰。

京索之战,汉军稳住阵脚,楚军也无力突破汉军防线进攻关中。双方从此开始在荥、成一带拉锯,战争进入相持阶段。

京索之战后,刘邦回到栎阳,进行整顿,立刘盈为太子。时关中爆发饥荒,刘邦令关中民移民至汉中、巴、蜀。接着刘邦再次来到荥阳前线。八月,汉分兵,韩信、曹参率军伐魏,後劉邦北上击赵;刘邦与韩信在襄国会合,杀赵王歇。周勃、召欧等继续平定恒山、巨鹿、燕 国,燕王臧荼降汉,项羽派楚将争夺赵国,唐厉在武城打败楚军。

韩信请刘邦封张耳为赵王,刘邦同意。

漢二年(前205年),魏王豹藉口親人有病,重返魏地,重新歸順項羽,封鎖蒲阪,起兵反對劉邦。劉邦派出酈食其遊說,不成,派韓信曹參攻打魏國。不久,魏王豹駐兵蒲阪,堵塞臨晉。曹参以代理左丞相的身份分别与韩信各率军向东攻魏国,在东张(秦汉东张县包括蒲坂、临晋关,所以魏军主力应该在临晋关一带防守“魏王盛兵蒲坂,塞临晋”)大败孙遫的军队。曹参率军大败魏王亲自率领的军队,魏豹逃跑到武垣,被汉军活捉。取平阳,得魏王母妻子,尽定魏地,凡五十二城。劉邦赐曹參食邑平阳,安邑之战后不久,韩信與张耳被派往趙地,汉三年九月(前204年),韩信與曹參先破代兵,生擒代国丞相夏说于阏与,但刘邦很快就收回军队主力。

劉邦命令韓信與張耳繼續率兵經井陘攻打趙國。趙王歇與陳餘在井陉口部署二十萬重兵,企图阻止漢軍北上。赵将军李左車建議派三萬士兵給他,截斷韓信的糧道,陳餘率軍在前線防守,使韓信进退两难,不出十日,定能斬殺韓信和張耳。但陳餘懦弱且迂腐,自恃兵精粮足,堅持正面迎战,拒絕使用李左車的計謀。韓信得知陳餘不用李左車之計,就放心率軍出井陘,並派出两千輕騎兵,命他们手持漢軍旗幟,準備在趙軍倾巢出擊後,立刻攻佔其營寨,插上漢軍旗幟。不久,韓信在绵蔓河畔設背水陣,詐敗誘敵,陈馀中计,大笑韩信不懂兵法,命趙軍猛攻背水陣。漢軍知道已无退路,人人拼死作戰,趙軍不能打敗漢軍,只好撤退,但漢軍两千騎兵已經攻佔敵營,插上漢旗,回撤的趙軍见状,一溃千里。漢軍乘勝夾擊追殺,大破趙軍,斬殺陳餘於泜水上,俘虜趙王歇、李左車。此为汉三年十月(属前204年)。

与此同时,刘邦亦亲攻赵。汉将靳歙兵出河内,击赵将贲郝于朝歌,破之。又随刘邦进击安阳以东,下七县;别将攻赵军,虏两司马,得赵军二千四百余人。接着刘邦对赵之邯郸发起进攻,破赵军,攻下邯郸。汉将靳歙破赵军于平阳,攻下邺。这样赵国悉平。刘邦与韩信在襄国会合,杀赵王歇。周勃、召欧等继续平定恒山、钜鹿、燕国,燕王臧荼降汉,项羽派楚兵攻擊赵国,唐厉在武城打败楚军。

汉灭赵之战的过程是,韩信与曹参受刘邦的命令,先破代国,杀夏说,把赵国的注意力转向赵国北部,派韩信与张耳在井陉设疑兵,利用地理优势吸引赵军主力,刘邦自己亲自率兵,趁虚直取邯郸。当赵国失去邯郸,襄国危急,陈馀进退两难,此时韩信与张耳出井陉,攻杀了陈馀。赵王歇逃到襄国,刘邦与张耳、韩信南北夹击襄国,攻破襄国会合,杀赵王歇,平定赵国。后来汉军又平定了巨鹿、常山郡,招降了燕国。韩信张耳继续平定赵国余寇,刘邦、靳歙、周勃、曹参等返回敖仓,此前英布被龙且与项声打败,与随何归汉,此时英布正式归降刘邦,这时汉营调走他旗下的兵到荥阳抵抗楚军。

韓信當時向東攻擊,尚未渡過平原津時,劉邦已經派出酈食其往齊國說齊王。韓信於是打算停止進兵。但由范陽來的蒯通認為酈食其一介書生,竟能憑三寸不爛之舌說下齊國七十多座城池,恐怕韓信的功勞比不上酈食其。於是韓信跟從他的計謀,襲擊齊國。當時齊王見劉邦已經派出酈食其,所以安心下來,把歷下的守軍撤走。 韓信率兵攻打历城,齐国叛汉,攻打楚国后方的靳歙与丁复不得不停止进攻,回到前线,刘邦派灌婴、曹参、陈武等支援韩信,攻打齐国 灌婴、曹参等到达齐国,攻下历城,齐王逃往高密,向项羽求救,项羽派龙且援齐汉四年十一月,陈武军,蔡寅军,丁复军,王周军,陈涓等汉军杀齐王田广、龙且,齐相田横自立为齐王,灌婴在嬴县打败田横,田横投奔彭越,曹参留在齐国继续平定齐国頑軍。

同時,項羽見龍且敗死,遂派武涉去遊說韓信投楚或中立,蒯通也建議韓信應該自立門戶,但全部被韓信拒絕。於是武涉離開,蒯通裝瘋逃去。韓信遂坐镇齐地

汉三年十二月,韩信与张耳留下在赵国继续跟余寇作战,刘邦返回荥阳。此前英布被随何策反,项羽派龙且項聲攻打英布,英布战败,与随何回到了荥阳。刘邦召见英布,派英布重返九江,收聚数千人归汉,刘邦也离开荥阳,从成皋南下,到宛县(今河南南阳)、叶县一带迎接英布,给英布增兵,一起回到成皋。

刘邦据守荥阳,开始修筑甬道,由敖仓运输粮食来荥阳。与项羽对峙,双方进入相持状态。

刘邦为解除荥阳相持的僵局,命令靳歙击断楚军从荥阳至襄邑的粮道,命令灌婴击断了楚军从阳武至襄邑的粮道,离开荥阳,攻打楚国后方的二号大本营:鲁城,并留下御史大夫周苛、魏王豹、韩王信、枞公等人守荥阳。刘邦与灌靳二将攻打鲁城的时候,汉三年八月,项羽猛攻荥阳,负责守荥阳的御史大夫周苛以魏王豹是反复之人,难与共守城,杀了魏豹。随后项羽便攻破了荥阳,杀了御史大夫周苛、枞公,俘虏了韩王信,又攻下了成皋。

刘邦得知荥阳已失守,便命令靳歙等攻打楚国的后方,后来靳歙攻下缯、郯、下邳,蕲、竹邑,几乎包围彭城,同时刘邦自己与灌婴回前线,在燕县打败楚将王武,在白马津打败楚将桓婴,渡过白马津,至河内,南渡黄河回到洛阳。

此时项羽攻下了成皋,进军至巩县,汉军与楚军在洛阳东边的巩县交战,楚军战败,项羽不能继续西进,楚军退至成皋,据险坚守,汉军攻之不下,一时无法夺回成皋。刘邦产生了放弃攻打成皋、退守巩县与洛阳的念头,郦食其劝刘邦不要退却,向刘邦说明敖仓的重要性,即放弃成皋与荥阳意味着放弃敖仓。郦食其说:“楚人拔荥阳,不坚守敖仓,乃引而东,令适卒分守成皋。楚军为什么要这样做?因为此时靳歙、丁复、傅宽等正在扫荡楚国的后方,项羽不得不分兵攻打他们解后方之急,故不能全力守成皋与敖仓。我方务必夺回成皋与荥阳,并坚守敖仓,取得战略上的优势,向诸侯昭示天下形势。”郦食其并自请出使齐国,劝说齐王降汉。刘邦接受了郦食其的建议。

刘邦兵败彭城,彭越亦亡其所下之城,率军北居河上,往来为汉游兵击楚,绝楚军粮道。汉三年五月(203年),彭越率军渡睢水,与楚将项声、薛公战于下邳(今江苏邳州古邳镇),大破楚军,杀薛公。

汉三年九月,刘邦采用郎中郑忠之策,派将军刘贾、卢绾将卒二万人、骑数百,由渡白马津,进入楚地佐助彭越。汉军与彭越联军烧掉楚军积聚的粮草,楚军乏食。楚军回击刘贾,刘贾坚守不出不与楚军交战,与彭越互相呼应。

接着彭越攻梁地,下睢阳、外黄等十七城。项羽无奈,只好留大司马曹咎守成皋,嘱令不与出战。然后东击彭越,迅速拿下陈留、睢阳、外黄等地。彭越敗北走谷城。

汉四年秋,项羽南走阳夏,彭越趁势下昌邑旁二十余城,得谷十余万斛送给汉军。汉五年冬十月,彭越得到刘邦立其为梁王,王睢阳以北至谷城,于是出兵助汉。十一月,汉将刘贾渡淮围寿春,楚大司马周殷叛楚,以舒屠六,举九江兵迎英布,并行屠城父,亦进至垓下。

韩信拒绝武涉游说以及蒯通建议后,自己坐镇齐国,同時劉邦命令灌婴对楚直接发起进攻。

灌婴率汉军首先进攻楚的鲁地,大破楚将薛公杲于鲁北。南下再破薛郡长,攻博阳,进军至下相,夺取取虑、僮、徐等县。接着渡过淮河,进至广陵(今江苏扬州),尽降楚之城邑。但项羽很快派项声、薛公、郯公夺回淮北。灌婴回师复渡淮,在下邳大破项声、郯公军,斩薛公,夺取下邳。接着追击楚军,破楚军于平阳(南平阳,今山东邹城市),回师还攻并占领彭城,虏楚柱国项佗降留、薛、沛、酇、蕭、相。攻苦、譙,再次俘获亚将周兰。

灌婴平淮北后,与刘邦军会师于颐乡(位于今河南鹿邑县)。

彭越、刘贾袭扰楚軍后方,使楚军补给无法保障,前方士疲粮绝。此时灌婴所部尽略楚地,夺取彭城,虏楚柱国项佗。刘邦趁机派陆贾為使者与项羽進行和谈,遭项羽拒绝。之后刘再次派出侯生出使楚國,終議和成功,即:楚、汉两家鴻溝和約,约定中分天下,双方以鸿沟为界,以东属楚,以西属汉,此稱「楚河漢界」; 并放还刘邦家人,止紛爭,罷干戈。

和約訂立后,项羽如约放还刘邦家人。刘邦封侯生为“平国君”,但侯生本人却功成隐退。

张良、陈平建议撕毁鸿沟和议,趁楚军疲师东返之机自其背后发动追歼。张、陈二人认为:“汉有天下太半,而诸侯皆附之。楚兵罢食尽,此天亡楚之时也”,建议“不如因其机而遂取之”。刘邦遂聽张良、陈平的建议,趁楚军锐气消磨殆尽的退兵路上发起追击。彭越趁项羽向南撤退到阳夏之机,攻克昌邑旁二十多个城邑,缴获谷物十多万斛,用作汉王的军粮。刘邦亦趁機率军发起追歼,於汉五年十月(按:这时以十月为岁首)击败項羽親率楚军取得阳夏(今河南太康),樊哙虏楚大将周将军卒四千人。

刘邦率领汉军追击项羽至固陵,项羽为了摆脱汉军,发动反击攻打刘邦。刘邦在各路汉军没到的情况下,为避免不必要的损失,选择高壁深垒防守战,為接下来的反击战做准备。刘邦另派刘贾南渡淮水包围寿春,刘贾很快到达,派人寻找机会招降楚大司马周殷。周殷叛变楚王,帮助刘贾攻下九江,迎着武王黥布的军队在垓下会合,断项羽向南逃后路,並共同攻打项羽。不久,灌婴、靳歙率领骑兵军团从彭城往固陵而来,刘邦亲自在固陵东边颐乡与灌婴率领的汉军铁骑会合。项羽得知灌婴、靳歙等率领汉军东来后,为防自己被包围往南退守至陈下,刘邦在灌婴、靳歙率领精锐骑兵到来后,发动反攻。汉将宣曲侯义率领骑兵和汾阳侯靳强率汉军为先锋,攻固陵楚军,便击破了楚大将钟离昧的部队,揭开了陈下之战的序幕。

陳下之战,楚方是项羽亲率的主力部队,有大将锺离眜,还有属于楚的陈公(陈县令)利几等;汉方是刘邦率领出成皋追击之军,有周勃、樊哙、靳歙等将领,加上破彭城后与刘邦会合的灌婴,还有刚来投降的楚将灵常。刘邦亲率汉军从西北方来,灌婴从东方来,对驻陈的楚军形成东西夹击合围之势。交战的结果,楚军大败,陈公利几向汉方投降,汉军大胜。

项羽战败后,率残兵败将逃跑。刘贾已策反楚大司马周殷,驻守在城父,周殷以舒县的兵力屠戮了六县,与英布一同北上攻打项羽,项羽立即调转马头,转向东南方逃跑,刘贾在城父堵截项羽,此前楚军一败再败,一无粮草,二无后援,军心战心,项羽无心恋战,逃往垓下。

刘邦的大部队迅速追上,刘贾也离开城父追击项羽,周殷与英布一同追击项羽,把项羽包围在垓下。

前203年,项羽依照鸿沟和约率兵东归。刘邦则派人赴齐招韩信,韩信坐山观虎斗,拒不与刘邦会合。汉五年十一月,项羽离开荥阳向南逃跑,刘邦率军追击,在阳夏南打败楚军,追至固陵,刘邦止步,派刘贾攻取寿春,策反项羽的大司马周殷,攻下九江,陈下大战后与英布会合包抄项羽,刘邦追项羽到陈县,与灌婴会合,大败楚军,楚将灵常、陈公利几降汉军。项羽继续逃跑,欲前往会稽。

彭越攻下昌邑二十余城,刘贾、周殷、英布攻下城父堵截项羽,项羽逃到垓下,刘邦与刘贾、周殷、英布会合。韩信看到项羽大势已去,也前往垓下与刘邦会合。彭越也来垓下,与刘邦合兵一处。公元前202年12月,10万楚军在垓下之战中被60余万汉军打败,被围在垓下(今安徽灵璧县东南)。当日夜晚,漢軍命士兵唱起楚地歌谣,楚军官兵听到四面楚歌,误以为汉军已渡江,

项羽自刎后,劉邦罷韓信兵權,令灌婴率军渡江,破吴郡长于吴下(今江苏苏州),得吴守,斬首八萬級,略定吴、豫章、会稽。复還定淮北,凡五十二縣,楚地略定。

临江王共尉(共敖之子)忠於項羽,不肯降漢。汉遣卢绾、劉賈别将攻之,久攻不下,靳歙还兵攻临江,下之,俘临江王共尉,被殺於洛阳。楚将陈公利几以陈郡降汉,不久,利幾懷疑劉邦欲誅之,於潁川反,劉邦乃率軍親征,利幾兵敗被殺。

项羽死后,楚國舊地皆向刘邦投降,只有鲁國不降,刘邦亲率大军想杀死他们,但考虑到鲁人守礼义,为君主死节,于是拿项羽的首級给魯國長老們看,鲁國的父兄才投降。楚义帝曾封项羽为鲁公,项羽死后又是鲁最后投降,于是刘邦下令以鲁公之礼葬项羽于穀城,并亲为发丧,洒泪而去。

项羽之宗族皆赦而不诛,究其前功,而各封为列侯。历时四年的楚汉战争,終以刘邦取得天下,建立西汉而告终。一年後,劉邦開始消滅異姓王。

\subsection{年表}

\begin{longtable}{|>{\centering\scriptsize}m{2em}|>{\centering\scriptsize}m{1.3em}|>{\centering}m{8.8em}|}
  % \caption{秦王政}\
  \toprule
  \SimHei \normalsize 年数 & \SimHei \scriptsize 公元 & \SimHei 大事件 \tabularnewline
  % \midrule
  \endfirsthead
  \toprule
  \SimHei \normalsize 年数 & \SimHei \scriptsize 公元 & \SimHei 大事件 \tabularnewline
  \midrule
  \endhead
  \midrule
  高祖\\元年 & -206 & \begin{enumerate}
    \tiny
  \item 秦朝灭亡。
  \item 鸿门宴。
  \item 项羽建立西楚王朝,自称西楚霸王。
  \end{enumerate} \tabularnewline\hline
  二年 & -205 & \begin{enumerate}
    \tiny
  \item 彭城之战。
  \item 成皋之战。
  \item 韩信破代、赵。
  \item 韩信灭燕、齐。
  \end{enumerate} \tabularnewline\hline
  三年 & -204 & \begin{enumerate}
    \tiny
  \item 背水一战。
  \item 南越国建立。
  \item 成皋之战。
  \end{enumerate} \tabularnewline\hline
  四年 & -203 & \begin{enumerate}
    \tiny
  \item 英布封王。
  \item 张耳封王。
  \end{enumerate} \tabularnewline
  \bottomrule
\end{longtable}


%%% Local Variables:
%%% mode: latex
%%% TeX-engine: xetex
%%% TeX-master: "../Main"
%%% End:

%% -*- coding: utf-8 -*-
%% Time-stamp: <Chen Wang: 2018-07-10 17:28:48>

\section{汉高祖\tiny(BC206-BC195)}

\begin{longtable}{|>{\centering\scriptsize}m{2em}|>{\centering\scriptsize}m{1.3em}|>{\centering}m{8.8em}|}
  % \caption{秦王政}\
  \toprule
  \SimHei \normalsize 年数 & \SimHei \scriptsize 公元 & \SimHei 大事件 \tabularnewline
  % \midrule
  \endfirsthead
  \toprule
  \SimHei \normalsize 年数 & \SimHei \scriptsize 公元 & \SimHei 大事件 \tabularnewline
  \midrule
  \endhead
  \midrule
  五年 & -202 & \begin{enumerate}
    \tiny
  \item 十二月垓下之战,汉灭楚统一天下,汉王刘邦即皇帝位。
  \item 汉置长安县、无锡县。
  \item 七月,燕王臧荼起兵反汉。
  \item 十月,刘邦率军亲征灭燕,俘杀臧荼。刘邦立卢绾为燕王。
  \item 汉高祖册封无诸为闽越王,封国闽越,首都冶城位于今之福州。
  \end{enumerate} \tabularnewline\hline
  六年 & -201 & \tabularnewline\hline
  七年 & -200 & \tabularnewline\hline
  八年 & -199 & \tabularnewline\hline
  九年 & -198 & \tabularnewline\hline
  十年 & -197 & \tabularnewline\hline
  十一年 & -196 & \tabularnewline\hline
  十二年 & -195 & \tabularnewline
  \bottomrule
\end{longtable}


%%% Local Variables:
%%% mode: latex
%%% TeX-engine: xetex
%%% TeX-master: "../Main"
%%% End:

%% -*- coding: utf-8 -*-
%% Time-stamp: <Chen Wang: 2019-12-16 11:49:38>

\section{孝惠帝\tiny(BC195-BC188)}

\subsection{简介}

漢惠帝劉盈(前210年-前188年9月26日),漢朝將盈避諱為满,汉朝开国皇帝漢高帝刘邦和皇后吕雉之子。西汉第二代皇帝,於前195年6月23日—前188年9月26日在位,在位7年,其正式諡號為「孝惠皇帝」,後世省略「孝」字稱「漢惠帝」,也是中国史上第一位皇帝所立的「皇太子」(扶蘇被訛稱為秦「皇太子」,事實上只是長子,秦始皇未立之。胡亥在秦始皇死后诈立为皇太子)。

惠帝在位期间,以温柔敦厚的个性,积极推行黄老学说,注重国家的休养生息和无为而治,放松文化专制政策,修筑长安城,为巩固西汉政权,安定社会,促进经济文化的发展,作出了一定的贡献。惠帝在位时朝政深受母亲呂后干预,呂后具有压制性的影响力,並成为實際统治者,因此司马迁《史记》未设孝惠本纪,反而设〈吕太后本纪〉。

由于受到有限的历史史料,加上其被呂后控制,史官为提高其異母弟汉文帝的地位,长期以来将惠帝视为一位“仁弱”君主,由司馬遷開始,古代学者稱在位时深受呂后臨朝聽政控制,对惠帝往往不太注意,且他所有的兒子都被誅滅諸呂的陳平、周勃、夏侯嬰等大臣們殺死滅口,更無子嗣可提,故只有偶爾宣揚其仁慈友愛的性格。

近世学者所著秦汉史更常直接对其略而不提,而对他的研究也屈指可数,近年有少数学者将他重新审视,稱其統治時與呂后配合,採用黃老之道,能與民休息。但另外的學者則稱,政治上建樹基本上屬於呂后的發揮,惠帝甚至不如部分东汉皇帝通過宦官與外戚大臣鬥爭而掌握实权,故评价不高。

惠帝之後的漢朝和西晉皇帝諡號中都有一「孝」字(除漢光武帝、晉武帝,不含追封(如劉禪)),故常省略。

刘盈,生于前210年,是刘邦和吕雉之子,秦时泗川郡丰邑中阳里人。刘盈有一位异母兄长刘肥,为刘邦在婚前與外妇曹氏所生。有人認為根據后来刘肥之子刘襄为刘邦“嫡长孙”的说辞,以及吕雉父亲吕公要将吕雉许配给刘邦时说的“臣有息女,愿为季箕帚妾”,曹氏可能为刘邦原配,吕雉后来因为某些原因取代曹氏成为刘邦正妻,但也有人反對,其實「箕帚妾」是謙詞,不代表當時的劉邦有正妻。

刘盈还有一名同母姐姐。吕雉有两兄吕泽、吕释之,其中吕泽日后比刘邦率先起兵,有一支独立的军队,为刘邦在楚汉战争胜利起了很大作用。吕雉的妹妹吕媭为刘邦日后的重要将领樊哙的妻子。

刘盈年幼时,刘邦为泗上亭长,家境并不殷实,据叔孙通日后对刘邦的诤谏之词,刘邦和吕雉曾过着“攻苦食啖”的生活。司马迁《史记》卷八<高祖本纪>记述的一段具有神话色彩的轶事,反映了刘盈很小时,便跟随母亲和姐姐下田工作。刘盈不到两岁,刘邦走上了反秦之路,据《史记》<高祖本纪>的一段神化记述,秦始皇在位末年,刘邦私自解纵刑徒,为躲避秦官府的追捕而亡隐于芒、砀山泽之间,吕雉为找到刘邦而颠沛流离。

前209年,秦末民變爆发,陈胜、吴广發動大澤之變,陈胜到了陈县称王,建立张楚政权。各郡县的仕紳大多杀死长官,响应陈胜。刘邦亦响应陈胜,于沛县起兵,号为沛公。刘邦不久即转战南北,刘盈和家人则都被留在丰邑,过着动荡不安的生活。

前206年,项羽灭秦,自立为西楚霸王,尊楚后怀王为义帝,分封十八诸侯王,立刘邦为汉王。此时刘盈和家人仍在丰邑,直到楚汉战争爆发,刘邦路经沛郡时,才派薛欧、王吸去寻找他们,但未能找到。

前205年,刘邦回军平定了三秦,又击项羽至彭城,项羽大败汉军。刘邦因兵败不利,乘车马急速逃去,打算经过沛县,接取家眷西行。在半路上夏侯婴遇到了刘盈和他的姐姐,就把他们收上车来。马已跑得十分疲乏,敌人又紧追在后,刘邦特别着急,有好几次用脚把两个孩子踢下车去,想扔掉他们了事,但每次都是夏侯婴下车把他们收上来,一直把他们载在车上。夏侯婴赶着车子,先是慢慢行走,等到两个吓坏了的孩子抱紧了自己的脖子之后,才驾车奔驰。刘邦为此非常生气,有十多次想要杀死夏侯婴,但最终还是逃出了险境,王陵因战事不利前来护送,侍奉刘盈和他的姐姐逃出睢水,把姐弟两人安然无恙地送到了丰邑。刘盈日后成为皇帝之后,夏侯婴作为太仆侍奉刘盈。刘盈和吕雉非常感激夏侯婴在下邑的路上救了刘盈和他的姐姐,就把紧靠在皇宫北面的一等宅第赐给他,名为“近我”,意思是说“这样可以离我最近”,以此表示对夏侯婴的格外尊宠。

刘邦没有找到父亲刘太公和妻子呂雉,审食其跟随刘太公和呂雉从小路潜行,寻找刘邦,反而碰上了楚军。楚军掳取了刘太公和呂雉,在军中作为人质。。直到前203年,项羽才放回刘太公和呂雉。

前205年7月1日,汉王刘邦立刘盈为太子,大赦罪人。命令刘盈驻守栎阳,在关中的诸侯国人都集中在栎阳守卫。丞相萧何留守关中,侍奉太子,在栎阳处理政务。。

前202年,汉军击败项羽楚军,相继灭亡楚国、齐国,取得了楚汉战争的胜利。八位异姓诸侯王和将相共同拥立汉王刘邦为皇帝,2月28日,刘邦即皇帝位,建立汉朝,尊王后吕雉为皇后,太子刘盈为皇太子。刘邦当初做汉王时,得到了定陶的美女戚姬,非常宠爱她,生下儿子刘如意。刘盈为人仁惠柔弱,刘邦认为不像自己,常想废掉他,改立戚姬的儿子刘如意为太子,因为刘如意像自己。戚姬得到宠幸,常跟随刘邦到关东,日夜啼哭,想要让自己的儿子取代刘盈做太子。吕后年纪大了,“色衰爱弛”,经常留在家中,很少见到刘邦,和刘邦越来越疏远。前198年,刘盈的大舅吕泽去世。同年,刘如意被立为赵王,此后几年,他多次险些取代了刘盈的太子地位。

刘邦想废掉刘盈的太子之位,改立戚夫人生的儿子赵王刘如意为太子。包括为刘邦所敬畏的周昌在内的很多大臣进谏劝阻,都没能改变刘邦坚定不移的想法。刘盈的母亲吕雉很惊恐,不知该怎么办。有人告诉吕雉,留候张良善于出谋划策,刘邦信任他。吕雉就派二哥建成侯吕释之请托张良说:“您一直是陛下的谋臣,现在陛下打算更换太子,您怎么能垫高枕头睡大觉呢?”张良说:“当初陛下多次处在危急之中,采用了我的计谋。如今天下安定,由于偏爱的原因想更换太子,这些至亲骨肉之间的事,即使同我一样的有一百多人进谏又有什么益处。”吕释之竭力请托张良一定得给他出个主意。张良说:“这件事是很难用口舌来争辩的。陛下不能招致而来的,天下有四人(即商山四皓)。这四人已经年老了,都认为陛下对人傲慢,所以逃避躲藏在山中,他们按照道义不肯做汉朝的臣子。但是陛下很敬重这四人。现在您果真能不惜金玉壁帛,让太子写一封信,言辞要谦恭,并预备安车,再派有口才的人恳切地聘请,他们应当会来。来了以后,把他们当作贵宾,让他们时常跟着入朝,叫陛下见到他们,那么陛下一定会感到惊异并询问他们。一问他们,陛下知道这四人贤能,那么这对太子是一种帮助。”于是吕雉让吕释之派人携带太子的书信,用谦恭的言辞和丰厚的礼品迎请这四人。四人前来后住在建成侯的府第中为客。

前198年,刘邦调太常叔孙通任太子太傅。前196年,淮南王黥布反叛,刘邦患重病,打算派刘盈率兵前往讨伐叛军。这四个人互相商议说:“我们之所以来,是为了要保全太子,太子如若率兵平叛,事情就危险了。”于是劝告吕释之说:“太子率兵出战,如立了功,那么权位也不会高过太子;如无功而返,那么从这以后就是遭受祸患了。再说跟太子一起出征的各位将领,都是曾经同陛下平定天下的猛将,如今让太子统率这些人,这和让羊指挥狼有什么两样,他们决不肯为太子卖力,太子不能建功是必定的了。我们听说‘爱其母必抱其子’,现在戚夫人日夜侍奉陛下,赵王如意常被抱在陛下面前,陛下说‘终归不能让不成器的儿子居于我的爱子之上’,显然,赵王如意取代太子的宝位是必定的了。您何不赶紧请吕皇后找机会向陛下哭诉:‘黥布是天下的猛将,很会用兵,现今的各位将领都是陛下过去的同辈,您却让太子统率这些人,这和让羊指挥狼没有两样,没有人肯为太子效力,而且如让黥布听说这个情况,就会大张旗鼓地向西进犯。陛下虽然患病,还可以勉强地乘坐辎车,躺着统辖军队,众将不敢不尽力。陛下虽然受些辛苦,为了妻儿还是要自己奋发图强一下。’”于是吕释之立即在当夜晋见吕雉,吕雉找机会向刘邦哭诉,说了四个人授意的那番话。刘邦说:“我就想到这小子本来不能派遣他,老子自己去吧。”于是刘邦亲自带兵东征,群臣留守,都送到灞上。张良患病,自己勉强支撑起来,送到曲邮,谒见刘邦说:“我本应跟从前往,但病势沉重。楚地人马迅猛敏捷,希望陛下不要跟楚地人斗个高低。”张良又趁机规劝刘邦说:“让太子做将军,监守关中的军队吧。”刘邦说:“子房虽然患病,也要勉强在卧床养病时辅佐太子。”这时叔孙通做太傅,张良任少傅之职。

前195年,刘邦随着击败黥布的军队回来,病势更加沉重,愈想更换太子。张良劝谏,刘邦不听,张良就托病不再理事。刘盈的太傅叔孙通引证古今事例进行劝说:“从前,晋献公因为宠幸骊姬的缘故废掉太子,立了奚齐,使晋国大乱几十年,被天下人耻笑。秦始皇因为不早早确定扶苏当太子,让赵高能够用欺诈伎俩立了胡亥,结果自取灭亡,这是陛下亲眼见到的事实。现在太子仁义忠孝,是天下人都知道的;吕皇后与陛下同经艰难困苦,同吃粗茶淡饭,是患难与共的夫妻怎么可以背弃她呢!陛下一定要废掉嫡长子而扶立小儿子,我宁愿先受一死,让我的一腔鲜血染红大地”,死命争保刘盈的太子之位。刘邦假装答应了他,但还是想更换太子。等到安闲的时候,设置酒席款待宾客,刘盈在旁侍侯。先前依張良建議聘來的四人跟着刘盈,他们的年龄都已八十多岁,须眉洁白,衣冠非常壮美奇特。刘邦感到奇怪,问道:“他们是干什么的?”四人向前对答,各自说出姓名,叫东园公、甪里先生、绮里季、夏黄公。刘邦于是大惊说:“我访求各位好几年了,各位都逃避着我,现在你们为何自愿跟随我儿交游呢?”四人都说:“陛下轻慢士人,喜欢骂人,我们讲求义理,不愿受辱,所以惶恐地逃躲。我们私下闻知太子为人仁义孝顺,谦恭有礼,喜爱士人,天下人没有谁不伸长脖子想为太子拼死效力的。因此我们就来了。”刘邦说:“烦劳诸位始终如一地好好调理保护太子吧。” 

四个人敬酒祝福已毕,小步快走离去。刘邦目送他们,召唤戚夫人过来,指着那四人给她看,说道:“我想更换太子,他们四人辅佐他,太子的羽翼已经形成,难以更动了。吕皇后真是你的主人了。”戚夫人哭泣起来,刘邦说:“你为我跳楚舞,我为你唱楚歌。”刘邦唱道:“鸿鹄高飞,一举千里。羽翮已就,横绝四海。横绝四海,当可奈何!虽有矰缴,尚安所施!”刘邦唱了几遍,戚夫人抽泣流泪,刘邦起身离去,酒宴结束。刘邦最终没更换太子,原本是张良招致这四个人发生了效力。刘邦死后,吕雉宴请张良以报答保太子之恩。

刘邦的将领舞阳侯樊哙因为娶了吕雉的妹妹吕媭为妻,生下儿子樊伉,因此和其他将领相比,刘邦对樊哙更为亲近。在黥布反叛的时候,刘邦一度病得很厉害,讨厌见人,他躺在宫禁之中,诏令守门人不得让群臣进去看他。群臣中如绛侯周勃、颍阴侯灌婴等人都不敢进宫。这样过了十多天,有一次樊哙推开宫门,径直闯了进去,后面群臣紧紧跟随。看到刘邦一人枕着一个宦官躺在床上。樊哙等人见到皇帝之后,痛哭流涕地说:“想当初陛下和我们一道从丰沛起兵,平定天下,那是什么样的壮举啊!而如今天下已经安定,您又是何等的疲惫不堪啊!况且您病得不轻,大臣们都惊慌失措,您又不肯接见我们这些人来讨论国家大事,难道您只想和一个宦官诀别吗?再说您难道不知道赵高作乱的往事吗?”刘邦听罢,于是笑着从床上起来。前195年,燕王卢绾谋反,刘邦命令樊哙以相国的身份去攻打燕国。这时刘邦又病得很厉害,有人诋毁樊哙和吕氏结党,皇帝假如有一天去世的话,那么樊哙就要带兵把戚氏和赵王如意这帮人全部杀死。刘邦听说之后,勃然大怒,立刻命令曲逆候陈平用车载着周勃去代替樊哙,并在军中立刻把樊哙斩首。陈平因惧怕吕雉,并没有执行刘邦的命令,而是把樊哙解赴长安。到达长安时,刘邦已经去世,吕雉就释放了樊哙,并恢复了他的爵位和封邑。陈平也因此得到了吕雉的信任。

刘邦臨終前,曾對呂后說如果相国酂候萧何去世,可以讓平阳侯曹参继任相国职位。《古文苑》之中,收录了五封刘邦给刘盈的手诏,郑重宣布刘盈为自己的继承人,反思了自己曾经所奉行的“读书无益”论,要求刘盈勤奋学习,自己写奏章,重用功臣集团,保护赵王如意。

前195年6月1日,刘邦在长乐宫驾崩。过了四天还不发布丧事消息。吕雉和审食其商量说:“那些将领先前和皇帝同为戶籍編列在册的老百姓,后来北面称臣,这些人就常常怏怏不乐,现在,又要侍奉少主,如果不全部族灭他们,天下就安定不了。”有人听到了这个话,告诉了将军郦商。郦商去见审食其,说:“我听说,皇帝已驾崩四天了,还不发布丧事,而且要杀掉所有的将领。若果真如此,天下可就危险了。陈平、灌婴率领十万大军镇守荥阳,樊哙、周勃率领二十万大军平定燕國和代國,如果他们听说皇帝驾崩了,诸将都将遭诛杀,必定把军队联合在一起,回过头来进攻关中。那时候大臣们在京師內乱,诸侯们在關外造反,漢朝滅亡的日子就可以跷著腳來等待了。”审食其进宫把这告诉了吕雉,于是就在6月4日发丧,大赦天下。

6月23日,安葬刘邦于长陵,当日(或6月26日),太子刘盈即位,来到太上皇庙。群臣都说皇帝“起徽(微)细,拨乱世反之正,平定天下,为汉太祖,功最高。”献上尊号称为高皇帝,即汉高帝。太子刘盈承袭皇帝之号,是为汉孝惠帝。又下令让各郡国诸侯都建高祖庙,每年按时祭祀。

前196年10月,刘邦在会甀击败黥布军,回京途中,经沛县时停留下来。在沛宫置备酒席,把老朋友和父老子弟都请来一起纵情畅饮。挑选沛中幼童一百二十人,教他们唱歌。酒喝得正畅快时,刘邦自己弹击着筑琴,唱起自己编的歌:“大风起兮云飞扬,威加海内兮归故乡,安得猛士兮守四方!”让儿童们跟着学唱。到了前190年,已经做了五年皇帝的刘盈想到父亲生前思念和喜欢沛县,就把沛宫定为父亲的原庙。刘邦所教过唱歌的儿童一百二十人,都让他们在原庙奏乐唱歌,以后有了缺员,就随时加以补充。

刘盈的一母同胞只有姐姐鲁元公主。除了鲁元公主,刘盈在即位时还有七个异母兄弟,他们按年龄大小分别是齐王刘肥、赵王刘如意、代王刘恒、梁王(趙王)刘恢、淮阳王刘友、淮南王刘长和燕王刘建。

刘盈即皇帝位后,尊母亲皇后吕雉为皇太后。吕太后对那些为刘邦侍寝而得宠幸的妃子如戚夫人等人非常气愤,就把她们都囚禁起来,不准出宫。而只有代王刘恒的母亲薄姬由于极少见刘邦的缘故,得以出宫,跟随儿子到代国,成为代王太后。而最终刘邦后宫的妃子只有不受宠爱被疏远的人才能平安无事。

吕雉在成为皇太后之后,就将戚夫人貶為奴隸,囚禁于永巷,剃去頭髮,穿着囚服,令其舂米。戚夫人边舂边唱:“子為王,母為虜,終日舂薄暮,常與死為伍!相離三千里,當誰使告女(汝)?(兒子為諸侯王,母親為奴隸,終日舂米到太陽落西,常常與死亡在一起!母子相離三千里,要找誰來告訴你?)”有人傳話給呂后,呂后大怒,说:“乃欲倚女(汝)子邪?(你想倚靠你儿子吗?)”于是派人召赵王如意入朝以便诛杀。此前汉高帝担心自己死后年轻的赵王难以保全,因而接受赵尧建议,徙御史大夫周昌担任赵國相以保护如意。太后的使臣到后,周昌让赵王推说身体不好,不能前往。使者往返去了三次,周昌都一直坚持不送赵王进京。于是太后很是忧虑,就派使者召周昌进京。周昌进京之后,拜见太后,太后非常生气地骂他:“难道你还不知道我非常恨戚氏吗?而你却不让赵王进京,为什么?”周昌被召进长安之后,太后又派使者召赵王,赵王动身赴京,还在半路上。刘盈仁慈,知道母亲恼恨赵王,就亲自到霸上去迎接,跟他一起回到宫中,亲自保护,跟他同吃同睡。太后想要杀赵王,却得不到机会。前195年12月的一天,刘盈在天明时出去射箭。赵王年幼,不能早起。太后得知赵王独自在家,派人拿去毒酒让他喝下。等到刘盈回到宫中,赵王已经被毒死了,此时距离赵王来到长安已有一个多月,刘如意的谥号为隐王。据《西京杂记》的一段记载,刘盈可能对毒死赵隐王的人进行惩处。赵隐王死后,淮阳王刘友被调去做赵王。

刘盈即位之初,吕雉就想增封诸吕为王。前194年夏天,刘盈与母亲可能就封诸吕为侯之事发生激烈冲突,太后于是派人砍断戚夫人的手脚,挖去眼睛,熏聋耳朵,灌了哑药,扔到廁所之中,叫她“人彘”。过了几天,太后叫刘盈观看「人彘」。刘盈看了,一经询问,才知道这是戚夫人,于是大哭起来。刘盈派人责备母亲说:“這種事不是人做得出來的,兒臣作為太后的兒子,終究無法治理天下!”刘盈因此大病一場。最终刘盈在位的七年间,仅封侯三人,无一人为诸吕。

前201年,汉高帝立庶长子刘肥为齐王,封地七十座城,百姓凡是使用齊國語言的,都归属齐王管轄。前193年,齐王和叔叔楚王刘交入京朝见刘盈。刘盈与齐王饮宴,二人行平等礼节如同家人兄弟的礼节。太后为此发怒,给齐王倒了杯毒酒,刘盈知道后,欲代替兄长饮毒酒,太后因此作罢。齐王害怕不能免祸,就用他的内史勋的计策,把城阳郡献出,做为鲁元公主的汤沐邑。太后很高兴,齐王才得以辞朝归国。前189年,刘肥去世,谥为悼惠王,史称齐悼惠王。刘盈派张良立刘肥的儿子刘襄为齐王,是为齐哀王。

刘盈即位后,相国依旧由萧何担任。前194年,废除了诸侯国设相国的法令,改命曹参为齐国丞相。曹参做齐国丞相时,齐国有七十座城邑。当时全国刚刚安定,齐悼惠王春秋鼎盛,曹参把老年人、读书人都召来,询问安抚百姓的办法。但齐国原有的那些读书人数以百计,众说纷纭,曹参不知如何决定。他听说胶西有位盖公,精研黄老学说,就派人带着厚礼把他请来。见到盖公后,盖公对曹参说,治理国家的办法贵在清靜無為,让百姓们自行安定。以此类推,把这方面的道理都讲了。曹参于是让出自己办公的正厅,让盖公住在里面。此后,曹参治理国家的要领就是采用黄老的学说,所以他当齐国丞相九年,齐国安定,人们大大地称赞他是贤明的丞相。

前193年8月16日,萧何卒。臨終前,漢惠帝詢問蕭何誰可以代替他,蕭何表示陛下最清楚。漢惠帝說:“曹參怎麼樣?”蕭何激動:“皇帝說對了!臣死無遺憾。”曹参听到蕭何去世这个消息,就告诉他的门客赶快整理行装,说:“我将要入朝当相国去了。”过了不久,朝廷派来的人果然来召曹参,曹参在9月7日被任命为相国。曹参离开时,嘱咐后任齐国丞相说:“要把齐国的監獄、市場作为寄托,要慎重对待这些行为,不要轻易干涉。”后任丞相说:“治理国家没有比这件事更重要的吗?”曹参说:“不是这样。監獄、市場这些行为,是善恶并容的,如果您严加干涉,坏人在哪里容身呢?我因此把这件事摆在前面。”曹参起初卑贱的时候,跟萧何关系很好;等到各自做了将军、相国,便有了隔阂。到汉高帝临终时,萧何向高帝推荐的贤臣只有曹参,高帝安排萧何死后,由曹参接替。曹参接替萧何做了汉朝的相国,做事情没有任何变更,一概遵循萧何制定的法度。

曹参从各郡和诸侯国中挑选一些质朴而不善文辞的厚道人,立即召来任命为丞相的属官。对官吏中那些言语文字苛求细微末节,想要一味追求声誉的人,就斥退撵走他们。曹参自己整天痛饮美酒。卿大夫以下的官吏和宾客们见曹参不理政事,上门来的人都想有言相劝。可是这些人一到,曹参就立即拿美酒给他们喝,过了一会儿,有的人想说些什么,曹参又让他们喝酒,直到喝醉后离去,始终没能够开口劝说,如此习以为常。

相国住宅的后园靠近官吏的房舍,官吏的房舍里整天饮酒歌唱,大呼小叫。曹参的随从官员们很厌恶这件事,但对此也无可奈何,于是就请曹参到后园中游玩,一起听到了那些官吏们醉酒高歌、狂呼乱叫的声音,随从官员们希望相国把他们召来加以制止。曹参反而叫人取酒陈设座席痛饮起来,并且也高歌呼叫,与那些官吏们相应和。

曹参见别人有细小的过失,总是隐瞒遮盖,因此相府中平安无事。当时曹参的儿子曹窋担任中大夫。孝惠帝刘盈埋怨曹参不理政事,觉得相国是否看不起自己,于是对曹窋说:“你回家后,试着私下随便问问你父亲说:‘高皇帝刚刚永别了群臣,陛下又很年轻,您身为相国,整天喝酒,遇事也不向陛下请示报告,根据什么考虑国家大事呢?’但这些话不要说是我告诉你的。”曹窋假日休息时回家,闲暇时陪着父亲,把刘盈的意思用自己的語氣說出,规劝曹参。曹参听了大怒,鞭打了曹窋二百下,说:“快点儿进宫侍奉陛下去,天下大事不是你应该说的。”到上朝的时候,;刘盈责备曹参说:“为什么要惩治曹窋?上次是朕让他规劝君的。”曹参脱帽谢罪说:“请陛下自己仔细考虑一下,在您和高皇帝谁圣明英武?”刘盈说:“朕怎么敢跟先帝相比呢!”曹参说:“陛下看我跟萧何,誰比較贤能?”刘盈说:“君好像不如萧何。”曹参说:“陛下说的这番话很对。高皇帝与萧何平定了天下,法令已经明确,如今陛下无为而治,我等谨守各自的职责,遵循原有的法度而不随意更改,不就行了吗?”刘盈说:“好。您休息休息吧!”

经过刘盈和曹参的这番对话,刘盈消除了对曹参的误会,君臣取得了共识。因之黄老之术兴起,并取得显著成效。司马迁对此赞叹道,孝惠皇帝在位时,百姓得以脱离战国时期的苦难,君臣都想通过无为而治来休养生息,所以孝惠帝无为而治,“天下晏然”。曹参担任汉相国三年,于前190年9月24日卒于任上。曹参卒后,百姓对他作歌唱:“萧何为法,顜若画一;曹参代之,守而勿失。载其清净,民以宁一。”

前189年12月11日,刘盈任命安国侯王陵为右丞相,曲逆侯陈平为左丞相。

刘邦驾崩,刘盈即位后,就对叔孙通说:“先帝园陵寝庙,群臣都不熟悉。”于是让叔孙通再度担任太常,让他制定宗庙的礼仪法度。此后陆续制定的汉朝诸多仪礼制度,皆为叔孙通任太常时所论定著录。刘盈自己居住在未央宫,经常要去东边的长乐宫朝拜母亲,还常有小的谒见,每次出行都要开路清道,禁止通行很是烦扰别人,于是就修了一座天桥,正好建在未央宫武库的南面。叔孙通向刘盈报告请示工作,趁机请求秘密谈话说:“陛下怎么能擅自把天桥修建在每月从高寝送衣冠出游到高庙的通道上面呢?高庙是汉太祖的所在,怎么能让后代子孙登到宗庙通道的上面行走呢?”刘盈听了大为惊恐,说要赶快毁掉它。叔孙通说:“人主不能有错误的举动。现在已经建成了,百姓全知道这件事,如果又要毁掉这座天桥,那就是显露出您有错误的举动。希望陛下在渭水北面另立一座原样的的祠庙,把高帝衣冠在每月出游时送到那里,更要增多、增广宗庙,这是大孝的根本措施。”刘盈就下诏令让有关官吏另立一座祠庙。这座另立的祠庙建造起来,就是由于天桥的缘故。刘盈曾在前190年春天到离宫出游,叔孙通说:“古的时候有春天给宗庙进献樱桃果的仪礼,现在正当樱桃成熟的季节,可以进献,希望陛下出游时,顺便采些樱桃来献给宗庙。”刘盈答应办这件事。以后汉代进献各种果品的仪礼就是由此兴盛起来的。

刘盈宠爱闳孺,与他同起同卧,公卿大臣通过闳孺去向刘盈沟通自己的说词。刘盈在位时的郎官和侍中,受到闳孺影响,都戴着用鵕璘毛装饰的帽子,系着饰有贝壳的衣带,涂脂抹粉。平原君朱建能言善辩,口才很好,同时他又刚正不阿,恪守廉洁无私的节操。家安在长安。他说话做事决不随便附和,坚持道义的原则而不肯曲从讨好,取悦于人。辟阳侯审食其品行不端正,靠阿谀奉承深得吕太后的宠爱。当时审食其很想和朱建交好,但朱建就是不肯见他。在朱建母亲去世的时候,陆贾和朱建一直很要好,所以就前去吊唁。朱建家境贫寒,连给母亲出殡送丧的钱都没有,正要去借钱来置办殡丧用品,陆贾却让朱建只管发丧,不必去借钱。然后,陆贾却到审食其家中,向他祝贺说:“平原君的母亲去世了。”审食其不解地说:“平原君的母亲死了,你祝贺我干什么?”陆贾说道:“以前你一直想和平原君交好,但是他讲究道义不和你往来,这是因为他母亲的缘故。现在他母亲已经去世,您若是赠送厚礼为他母亲送丧,那么他一定愿意为您拼死效劳。”于是审食其就给朱建送去价值一百金的厚礼。而当时的不少列侯贵人也因为辟阳侯送重礼的缘故,也送去了总值五百金的钱物。辟阳侯审食其很受吕太后宠爱,有人就在刘盈面前说他的坏话,刘盈大怒,就把审食其逮捕交给官吏审讯,并想借此机会杀掉他。吕太后感到惭愧,又不能替他说情。而大臣们大都痛恨审食其的行为,更想借此机会杀掉他。审食其很着急,就派人给平原君朱建传话,说自己想见见他。但朱建却推辞说:“您的案子现在正紧,我不敢会见您。”然后朱建请求会见刘盈的宠臣闳孺,说服他道:“皇帝宠爱您的原因,天下的人谁都知道。现在辟阳侯受宠于太后,却被逮捕入狱,满城的人都说您给说的坏话,想杀掉您。如果今天辟阳侯被陛下杀了,那么明天早上太后发了火,也会杀掉您。您为什么还不脱了上衣,光着膀子,替辟阳侯到皇帝那里求个情呢?如果皇帝听了您的话,放出辟阳侯,太后一定会非常高兴。而太后、皇帝两人都宠爱您,那么您也就会加倍富贵了。”于是闳孺非常害怕,就听从了朱建的主意,向刘盈给审食其说情,刘盈果然放出了审食其。审食其在被囚禁的时候,很想会见朱建,但是朱建却不肯见审食其,审食其认为这是背叛自己,所以对他很是恼恨。等到他被朱建成功地救出之后,才感到特别吃惊。

汉朝建立之初,经济凋敝,即使皇帝也不能凑齐四匹纯色的马,到了将相有的只能乘坐牛车。汉高帝刘邦为此颁布法令约束节俭,减轻田租赋税,实行“十五税一”制度。刘邦在位后期,因为消灭异姓王和抗击匈奴等战事,田租有所加征。刘盈即位后,下诏减少田租,恢复十五税一。这种举措减轻了农民一些负担,有助于休养生息。前191年1月,为劝勉农桑,刘盈下诏在各地选拔孝弟力田的贤者,免除其徭役负担。又在前191年4月1日趁着加冠之时,减免刑罚,下诏省去妨害吏民的法令,以减少繁扰,调动农民的生产积极性。

前194年2月,开始修筑长安城。前192年春天,征发长安六百里之内的男女民工十四万六千人筑长安城,修筑了三十天。7月,又调发诸侯王、列侯的家人奴隶二万人到长安筑城。前190年2月,再次调發长安六百里之内的男女十四万五千人修筑长安城三十天。9月,长安城最后竣工。赏赐民爵,每户一级。前189年7月,起长安西市,修敖仓。11月修成,诸侯都来京聚会,入朝祝贺。

为了使人口迅速繁殖,刘盈在前189年12月23日借兄长齐悼惠王刘肥去世的时机,下诏规定女子年龄在十五岁以上至三十岁不出嫁的,罚款五算。据东汉时期的应劭解释,汉代凡十五以上的成年人均需交人口税,每人每年一百二十钱为一算,称为“算赋”。刘盈这一时规定,实际就是强制妇女到了十五岁便要结婚生息。这对于繁衍人口和恢复、发展经济都有着显著的促进作用。

汉初,国家贫困,经济萧条,为了巩固雏鹰般的汉朝,刘邦采取了减轻钱重,以便利流通,求得商业发展的政策,结果反而造成物价飞涨、通货膨胀的局面。全国统一后,刘邦对商贾采取了一定的抑制政策,“乃令贾人不得衣丝乘车,重租税以困辱之”。刘盈在位时,为繁荣工商业,便下令进一步放宽对商贾的限制,“复弛商贾之律”。除了不准做官、商贾的人口税比平民重一倍外,其他多予废除。这对于经济的迅速恢复与发展,尤其对商品经济的活跃,起了很大作用。

在外交事务方面,汉仍采取谨慎政策,包括继续以宗室女为公主嫁匈奴单于,立闽越君驺摇为东海王,接受南越王赵佗称臣奉贡。

刘邦死后,刘盈在位时期,汉朝刚刚安定,所以匈奴显得骄傲。冒顿单于渐渐骄横起来,于是写了书信,派使者送给皇太后吕雉,说:“我是孤独无依的君主,生在潮湿的沼泽地,长在平旷的放牛放马的地方,我多次到边境上来,希望能到中原游玩一番。陛下您独立为君,也是孤独无依,单独居住。我们两个做君主的很不快乐,没有什么可以娱乐的。希望我俩能以自己擁有的,來交换到自己没有的。”吕太后看信后十分愤怒,把左丞相陈平、上将军樊哙、中郎将季布召来,商议杀掉匈奴的使者,发兵攻打匈奴。樊哙说:“臣愿意率领十万大军,到匈奴境内去横行冲击。”吕雉询问季布,季布说:“樊哙真该斩首啊!以前陈豨在代地反叛,汉兵有三十二万,樊哙是上将军,当时匈奴把高帝围困在平城,樊哙不能冲破围困。天下百姓唱道:‘平城之下亦诚苦!七日不食,不能彀弩。’现在人们吟唱的声音还在耳畔,没有断绝,受创伤的人刚能站立起来,而樊哙却要让天下震动,胡说什么要带十万大兵橫行匈奴,这是当面欺君罔上。况且这些蠻夷就好比禽兽一样,听到了他们的好话不值得高兴,听到恶语也不值得生气。”吕太后说:“那好吧。”于是便命令大谒者张释写信回报,说:“单于没有忘掉我们这破败的国家,以书信赏赐我们,我们戒慎恐懼。退朝的時日,我自己思虑,我年老气衰,头发、牙齿脱落,走路也走不稳,单于听到誇大的謠言了,不值得单于污辱自己。敝国没有什么罪过,就請單于放過我們。我有两辆御车,驾车的马八匹,奉送给您日常駕。”冒顿得到回信,又派使者来谢罪说:“我们没有听过中國的礼节,幸而得到陛下宽恕。”匈奴献上马匹,前192年春天,刘盈以宗室女为公主继续与匈奴和亲。

前213年,秦始皇在咸阳宫设宴招待群臣,博士仆射周青臣等人称颂秦始皇的武威盛德。博士齐人淳于越则进言建议恢复分封子弟的做法,他认为分封子弟能巩固中央的权利,因为商周都因为分封子弟功臣而得以长治久安。始皇帝把他们的意见下交群臣议论,丞相李斯认为分封子弟的做法不合时宜,建议除了秦国官修史书《秦纪》以为,其他各国的历史记载都予以烧毁。除了博士官负责管理的文献外,天下敢有收藏《诗》、《书》、诸子百家著作者,全部上交地方政府予以烧毁。胆敢私下讨论《诗》、《书》者,处以死刑。以古非今、收藏禁书者,诛灭其家族。而医药、占卜、种树等有实用价值的书籍,不在禁烧之列。如果有人想要学习法令,就以官吏为师。李斯的建议得到了秦始皇的采纳,成为“挟书律”。

秦朝灭亡后,刘邦刚进入关中就同关中百姓约法三章,“杀人者死,伤人及盗抵罪”,取得了关中百姓的拥护。此后,战事仍在继续,三条法令不足以稳定社会,于是相国萧何采集秦朝法令,选取其中合乎时宜的,制订了九章律。到了刘盈在位时,百姓刚免除战争的毒害,人人都想抚育儿童事奉老人。萧何、曹参任丞相,用无为之策来安定百姓,顺从他们的要求,而不加以扰乱,因此百姓衣食丰盛,刑罚使用得很少。前191年4月1日,刘盈加冠,废除了挟书律。

前189年12月,刘盈可能病入膏肓,岁余不能起。为预防东方诸侯王起兵叛乱,刘盈派太尉灌婴指挥车骑、材官镇守荥阳。前188年9月26日,刘盈在未央宮驾崩,终年22岁。呂后哭的卻不怎麼傷心,被張良兒子侍中张辟彊察覺。张辟彊對丞相說:“太后只有孝惠帝這麼一個兒子,但是皇帝駕崩,太后卻不是很難過,你知道個中緣由嗎?”丞相說:“怎麼解釋?”張辟彊說:“皇帝沒有成年兒子,太后害怕你們這些高官,你現在請太后讓呂家親戚掌握兵權和權力,這樣太后才能心安,你們也可以避禍。”丞相聽從張辟彊的計謀。太后大喜,但卻直接導致諸呂專權。呂太后又想為漢惠帝起高墳,甚至能達到在未央宮就能望見的程度。大臣勸諫,呂后不聽。東陽侯張相如勸道:“如果陵墓落成,太后陛下天天見到惠帝陵墓,不停地傷心,對此臣感到很悲痛。”太后於是停止先前的起高墳計劃。10月19日,刘盈被安葬于安陵,谥号孝惠皇帝。此后的汉朝及蜀漢皇帝,除了光武皇帝刘秀、昭烈皇帝劉備和不被認可的皇帝以外,谥号都有一个孝字。汉孝惠帝被安葬后,太子即位成为皇帝,吕雉由此临朝称制,增封诸吕为侯。

\subsection{年表}


\begin{longtable}{|>{\centering\scriptsize}m{2em}|>{\centering\scriptsize}m{1.3em}|>{\centering}m{8.8em}|}
  % \caption{秦王政}\
  \toprule
  \SimHei \normalsize 年数 & \SimHei \scriptsize 公元 & \SimHei 大事件 \tabularnewline
  % \midrule
  \endfirsthead
  \toprule
  \SimHei \normalsize 年数 & \SimHei \scriptsize 公元 & \SimHei 大事件 \tabularnewline
  \midrule
  \endhead
  \midrule
  元年 & -194 & \tabularnewline\hline
  二年 & -193 & \tabularnewline\hline
  三年 & -192 & \tabularnewline\hline
  四年 & -191 & \tabularnewline\hline
  五年 & -190 & \tabularnewline\hline
  六年 & -189 & \tabularnewline\hline
  七年 & -188 & \tabularnewline
  \bottomrule
\end{longtable}


%%% Local Variables:
%%% mode: latex
%%% TeX-engine: xetex
%%% TeX-master: "../Main"
%%% End:

%% -*- coding: utf-8 -*-
%% Time-stamp: <Chen Wang: 2018-07-10 17:28:59>

\section{前少帝\tiny(BC187-BC184)}

\begin{longtable}{|>{\centering\scriptsize}m{2em}|>{\centering\scriptsize}m{1.3em}|>{\centering}m{8.8em}|}
  % \caption{秦王政}\
  \toprule
  \SimHei \normalsize 年数 & \SimHei \scriptsize 公元 & \SimHei 大事件 \tabularnewline
  % \midrule
  \endfirsthead
  \toprule
  \SimHei \normalsize 年数 & \SimHei \scriptsize 公元 & \SimHei 大事件 \tabularnewline
  \midrule
  \endhead
  \midrule
  元年 & -187 & \tabularnewline\hline
  二年 & -186 & \tabularnewline\hline
  三年 & -185 & \tabularnewline\hline
  四年 & -184 & \tabularnewline
  \bottomrule
\end{longtable}


%%% Local Variables:
%%% mode: latex
%%% TeX-engine: xetex
%%% TeX-master: "../Main"
%%% End:

%% -*- coding: utf-8 -*-
%% Time-stamp: <Chen Wang: 2018-07-10 17:28:54>

\section{后少帝\tiny(BC183-BC180)}

\begin{longtable}{|>{\centering\scriptsize}m{2em}|>{\centering\scriptsize}m{1.3em}|>{\centering}m{8.8em}|}
  % \caption{秦王政}\
  \toprule
  \SimHei \normalsize 年数 & \SimHei \scriptsize 公元 & \SimHei 大事件 \tabularnewline
  % \midrule
  \endfirsthead
  \toprule
  \SimHei \normalsize 年数 & \SimHei \scriptsize 公元 & \SimHei 大事件 \tabularnewline
  \midrule
  \endhead
  \midrule
  元年 & -183 & \tabularnewline\hline
  二年 & -182 & \tabularnewline\hline
  三年 & -181 & \tabularnewline\hline
  四年 & -180 & \tabularnewline
  \bottomrule
\end{longtable}


%%% Local Variables:
%%% mode: latex
%%% TeX-engine: xetex
%%% TeX-master: "../Main"
%%% End:

%% -*- coding: utf-8 -*-
%% Time-stamp: <Chen Wang: 2018-07-10 17:28:15>

\section{孝文帝\tiny(BC179-BC157)}

\subsection{前元}

\begin{longtable}{|>{\centering\scriptsize}m{2em}|>{\centering\scriptsize}m{1.3em}|>{\centering}m{8.8em}|}
  % \caption{秦王政}\
  \toprule
  \SimHei \normalsize 年数 & \SimHei \scriptsize 公元 & \SimHei 大事件 \tabularnewline
  % \midrule
  \endfirsthead
  \toprule
  \SimHei \normalsize 年数 & \SimHei \scriptsize 公元 & \SimHei 大事件 \tabularnewline
  \midrule
  \endhead
  \midrule
  元年 & -179 & \tabularnewline\hline
  二年 & -178 & \tabularnewline\hline
  三年 & -177 & \tabularnewline\hline
  四年 & -176 & \tabularnewline\hline
  五年 & -175 & \tabularnewline\hline
  六年 & -174 & \tabularnewline\hline
  七年 & -173 & \tabularnewline\hline
  八年 & -172 & \tabularnewline\hline
  九年 & -171 & \tabularnewline\hline
  十年 & -170 & \tabularnewline\hline
  十一年 & -169 & \tabularnewline\hline
  十二年 & -168 & \tabularnewline\hline
  十三年 & -167 & \tabularnewline\hline
  十四年 & -166 & \tabularnewline\hline
  十五年 & -165 & \tabularnewline\hline
  十六年 & -164 & \tabularnewline
  \bottomrule
\end{longtable}


\subsection{后元}

\begin{longtable}{|>{\centering\scriptsize}m{2em}|>{\centering\scriptsize}m{1.3em}|>{\centering}m{8.8em}|}
  % \caption{秦王政}\
  \toprule
  \SimHei \normalsize 年数 & \SimHei \scriptsize 公元 & \SimHei 大事件 \tabularnewline
  % \midrule
  \endfirsthead
  \toprule
  \SimHei \normalsize 年数 & \SimHei \scriptsize 公元 & \SimHei 大事件 \tabularnewline
  \midrule
  \endhead
  \midrule
  元年 & -163 & \tabularnewline\hline
  二年 & -162 & \tabularnewline\hline
  三年 & -161 & \tabularnewline\hline
  四年 & -160 & \tabularnewline\hline
  五年 & -159 & \tabularnewline\hline
  六年 & -158 & \tabularnewline\hline
  七年 & -157 & \tabularnewline
  \bottomrule
\end{longtable}


%%% Local Variables:
%%% mode: latex
%%% TeX-engine: xetex
%%% TeX-master: "../Main"
%%% End:

%% -*- coding: utf-8 -*-
%% Time-stamp: <Chen Wang: 2019-12-16 14:06:01>

\section{孝景帝\tiny(BC156-BC141)}

\subsection{生平}

汉景帝刘启(前188年-前141年3月9日),为西漢第六位皇帝(前157年7月14日—前141年3月9日在位),在位16年,享年48岁,其正式諡號為「孝景皇帝」,後世省略「孝」字稱「漢景帝」,景帝後元三年正月甲子(前141年3月9日)崩于未央宮,二月癸酉(3月18日)葬于阳陵(今陕西高陵县西南)。为汉文帝刘恒長子,母竇皇后。他在位期间,主要是削诸侯封地,平定七国之乱,巩固中央集权,勤俭治国,发展生产,他统治时期和他父亲文帝统治时期合称文景之治。

刘启为汉文帝刘恒长子。刘启出生时,父亲刘恒为代王,母亲窦姬为妾。前180年,呂太后駕崩,陳平、周勃等人誅滅諸呂,立刘恒为帝,是為汉文帝。嫡母代王王后早已逝世,而她所生的世子也在此时病死,刘启成为父亲刘恒事实上的长子。同年正月,刘啟以刘恒长子的身份被立为太子。其后,母亲窦姬亦被立为皇后。

刘启為太子時性格剛烈,因與吳國太子劉賢下圍棋(一說六博)發生爭執,而拿棋盤打死了劉賢,漢文帝敕命送遗體回去埋葬,到了吳國,劉賢的父親吳王劉濞大怒,說道:「天下都是劉家的,死在長安就埋在長安,何必回吳國埋葬!」遂又把遗體送回長安埋葬,以示對朝廷的不滿,從此刘濞怨恨刘启。

前157年7月6日(六月己亥),汉文帝崩于长安未央宮,7月14日(丁未),皇太子刘启即位,是为汉景帝。

景帝前元三年(前154年),御史大夫晁错建议削藩,景帝听从,引起那些早就想反叛的诸侯王们的不满,于是以吴王刘濞、楚王刘戊为首的七国之乱开始了。七國诸侯王以“诛晁错,清君侧”为藉口叛乱,欲夺天下。晁错政敌袁盎献策景帝,诛晁错以平叛乱。景帝“嘿然良久,曰:‘顾诚何如,吾不爱一人以谢天下。’”。于是有丞相青翟等一众臣子劾奏晁错。景帝“制曰:‘可’”,令中尉以上朝议事为名,誘晁错上朝,行中错道,至东市,中尉宣汉景帝刘启诏书,当场腰斩晁错。但晁错死後,七国之乱不但没有停止,反而越演越烈,占领了不少土地。景帝无可奈何,只得派太尉周亞夫、竇嬰镇压,前後三個月即平定七国之乱。

七国之乱以后,景帝开始专心打理朝政,据说景帝十分樸素,仁厚爱民。除了平定七国之乱以外,从来没有大规模用过兵,和匈奴的战争始终控制在一定的规模内,依然对匈奴采取和亲政策。

前元六年(前151年),皇后薄氏被废。第二年(前150年),废太子刘荣。同年四月,立王氏为皇后,随后立王氏的独子胶东王刘彻为太子。

景帝崇尚黃老之說,减少刑罚,减少赋税,兴修水利,提倡農業,要求人心不服的案子进行重审,以免冤狱发生。百姓在和平稳定的环境下创造了大量财富,其间百姓富裕,丰衣足食,安居乐业,天下太平安乐,一派盛世景象,与其父汉文帝统治时期并称文景之治。

景帝后元三年正月甲子(前141年3月9日),景帝崩于未央宫,遺詔賜予諸侯王與列侯駿馬兩匹、吏二千石、黃金兩斤,吏民戶百錢;又命放出一批宮人,使其歸家再嫁。景帝享年48歲,諡號孝景皇帝,无廟號,二月癸酉(3月18日),葬于阳陵。景帝崩後由皇太子刘彻即位,是为汉武帝。

司马迁在《史记》中评价:“至孝景,不复忧异姓,而晁错刻削诸侯,遂使七国俱起,合从而西乡,以诸侯太盛,而错之不以为渐也。”

班固在《汉书》中评价:“孝景遵业,五六十载之间,至于移风易俗,黎民醇厚。周云成康,汉言文景,美矣!”

唐代司馬貞在《史記索隱》中評價景帝以德待臣子、鼓勵耕作,面對吳楚之叛,引領將領翦除逆賊。但是平叛的周亞夫,受到景帝的猜忌,倉促下獄,這對於一個治理國家的明君來說,是很可惜的:『【索隱述贊】景帝即位,因脩靜默。勉人於農,率下以德。制度斯創,禮法可則。一朝吳楚,乍起凶慝。提局成釁,拒輪致惑。晁錯雖誅,梁城未克。條侯出將,追奔逐北。坐見梟黥,立翦牟賊。如何太尉,後卒下獄。惜哉明君,斯功不錄!』

\subsection{前元}

\begin{longtable}{|>{\centering\scriptsize}m{2em}|>{\centering\scriptsize}m{1.3em}|>{\centering}m{8.8em}|}
  % \caption{秦王政}\
  \toprule
  \SimHei \normalsize 年数 & \SimHei \scriptsize 公元 & \SimHei 大事件 \tabularnewline
  % \midrule
  \endfirsthead
  \toprule
  \SimHei \normalsize 年数 & \SimHei \scriptsize 公元 & \SimHei 大事件 \tabularnewline
  \midrule
  \endhead
  \midrule
  元年 & -156 & \tabularnewline\hline
  二年 & -155 & \tabularnewline\hline
  三年 & -154 & \tabularnewline\hline
  四年 & -153 & \tabularnewline\hline
  五年 & -152 & \tabularnewline\hline
  六年 & -151 & \tabularnewline\hline
  七年 & -150 & \tabularnewline
  \bottomrule
\end{longtable}


\subsection{中元}

\begin{longtable}{|>{\centering\scriptsize}m{2em}|>{\centering\scriptsize}m{1.3em}|>{\centering}m{8.8em}|}
  % \caption{秦王政}\
  \toprule
  \SimHei \normalsize 年数 & \SimHei \scriptsize 公元 & \SimHei 大事件 \tabularnewline
  % \midrule
  \endfirsthead
  \toprule
  \SimHei \normalsize 年数 & \SimHei \scriptsize 公元 & \SimHei 大事件 \tabularnewline
  \midrule
  \endhead
  \midrule
  元年 & -149 & \tabularnewline\hline
  二年 & -148 & \tabularnewline\hline
  三年 & -147 & \tabularnewline\hline
  四年 & -146 & \tabularnewline\hline
  五年 & -145 & \tabularnewline\hline
  六年 & -144 & \tabularnewline
  \bottomrule
\end{longtable}


\subsection{后元}

\begin{longtable}{|>{\centering\scriptsize}m{2em}|>{\centering\scriptsize}m{1.3em}|>{\centering}m{8.8em}|}
  % \caption{秦王政}\
  \toprule
  \SimHei \normalsize 年数 & \SimHei \scriptsize 公元 & \SimHei 大事件 \tabularnewline
  % \midrule
  \endfirsthead
  \toprule
  \SimHei \normalsize 年数 & \SimHei \scriptsize 公元 & \SimHei 大事件 \tabularnewline
  \midrule
  \endhead
  \midrule
  元年 & -143 & \tabularnewline\hline
  二年 & -142 & \tabularnewline\hline
  三年 & -141 & \tabularnewline
  \bottomrule
\end{longtable}


%%% Local Variables:
%%% mode: latex
%%% TeX-engine: xetex
%%% TeX-master: "../Main"
%%% End:

%% -*- coding: utf-8 -*-
%% Time-stamp: <Chen Wang: 2019-12-16 14:11:34>

\section{武帝\tiny(BC140-BC87)}

\subsection{生平}

漢武帝刘彻(前156年7月31日-前87年3月29日),原名彘,西汉第七位皇帝。於7岁时被冊立为储君,16岁登基,在位達54年。其正式諡號為「孝武皇帝」,後世省略「孝」字稱「漢武帝」,是清圣祖以前在位最長的中國皇帝。他雄才大略,文治武功都有顯赫建树,與秦始皇被後世並稱為「秦皇漢武」,被历代史学界和政治家们評價為中國歷史上最偉大的皇帝之一。漢武帝的思想積極進取,极具前瞻性,為朝廷以至社会帶了新思維,亲政後進行了多項大刀闊斧的改革,深遠地影響著後世。

對內政策上,漢武帝用人唯才,不問出身,開創了察舉制并兴太学,以致該時期培養及出現了大量名臣良將;他又頒布《推恩令》,和平地削減了诸侯的權力及勢力,并将盐铁和铸币权收归中央;另外罷黜百家,獨尊儒術,儒学从此成為中國社會主流思想,另有首开丝绸之路、使用年号、设立刺史、加强内廷权力等开创性举措。

對外政策上,漢武帝一改漢高祖刘邦白登之围後世代朝廷奉行的和親傳統,以強勢態度積極地對付匈奴,發動第二階段漢匈戰爭,先後收復了西漢初年的多處領土,不过终其一世未能解除秦朝以來匈奴於中國西北部的威脅。

漢武帝又大幅度地开疆拓土,先後在秦朝故土吞灭了东瓯国、南越國、閩越國,并远征异域,消灭衛滿朝鮮及册封夜郎國等等,继秦朝后再次拓展了中国疆域;同時兩次派遣張騫出使西域,開闢丝绸之路,远征大宛,使汉帝国的影响力和控制力远达中亚,將帝國在民生、經濟、文化和軍事上,都推上了空前的高峰,其在位期間被稱為漢武盛世,為漢朝的極盛時期。

而漢武帝晚年穷兵黩武,對人民造成了相當大負擔。其晚年性情也變得反覆無常,而且迷信多疑,致使了巫蛊之祸的發生,為其普遍整體正面評價留下負面部份。他也对臣下擅用权力,司马迁和李陵家族都在他的命令下遭难。駕崩前兩年,漢武帝下《轮台诏》,重拾文景之治時期的與民生息的政策,為後來的昭宣中興奠定基礎。

据《史记》、《汉书》的武帝本纪以及《漢武故事》,汉武帝生于汉景帝前元年(前156年);母王氏,汉景帝中子,具体排序不详。其母王氏在怀孕时,汉景帝尚为太子。王氏梦见太阳进入她的怀中,告诉景帝后,景帝说:“此贵徵也。”刘彻还未出生,他的祖父汉文帝就逝世了。汉景帝即位后,刘彻出生,他亦是王氏唯一的儿子。一說劉徹的乳名為劉彘,根據漢武故事記載劉徹被立太子時方才改名,但此說與史書說法有出入。

前元四年(前153年),刘彻以皇子的身份被封为胶东王。同年,景帝的长子、他的异母长兄刘荣获封为太子。前元六年(前151年)秋九月,无子无宠的薄皇后被废。第二年(前150年)春正月,废栗太子刘荣为临江王;夏四月乙巳,其母王氏被立为皇后,丁巳,刘彻被立为太子。他成为太子与其母孝景王皇后和其姑母馆陶公主刘嫖有很大关系。刘嫖许诺将她的女儿陈氏嫁给当时四岁(古代按虚岁计算)的胶东王刘彻。刘彻后娶陈氏为妃,两人成婚的时间无考。

后元三年正月甲子(前141年3月9日),景帝逝世,太子刘彻即位,尊皇太后窦氏曰太皇太后,皇后王氏曰皇太后。太子妃陈氏后获封为皇后(具体时间不详)。

漢武帝建立了中朝削弱相权,巩固皇权。“中朝”又称“内朝”,由皇帝左右的亲信的近臣所构成。汉武帝时,他选用一些亲信侍从如尚书、常侍等组成宫中的决策班子,称为“中朝”或“内朝”。相对与“外朝”而言,“大司马、左右前后将军、侍中、常侍、散骑诸吏为中朝。丞相以下至六百石为外朝也”。中、外是相对皇帝居住的宫禁而言,中朝(内朝)官员享有较大的出入宫禁的自由,可随侍皇帝左右且能在宫中办公,外朝官员则无此特权。借由此来培植一批立足于宫中、与以丞相为首的原有朝臣分庭抗礼的内廷官员。重要政事,“中朝”在宫廷之内就先自作出了决策,再交由“外朝”的丞相来执行。尚书,本来是皇帝身边掌管文书员。“中朝”形成之后,尚书的地位日益重要。尚书和一般只参与宫廷议政的官员不同,由于既有官署、官属,又有具体的职司,作为皇帝的秘书机构,在“中朝”逐渐居于核心地位。

汉武帝在地方設置十三州部刺史。即完善监察制度,加強對地方的控制,打擊地方豪強。京師七郡則另外設立司隸校尉監察。汉武帝将全国地方划分为13个监察区,是为冀、兖、豫、青、徐、幽、并、凉、荆、扬、益、朔方、交趾共13州(京畿附近7郡为司隶校尉部作为一个单独的监察区)。每州派遣一名刺史,每年8月巡行所部,监察地方官员和强宗豪右,岁终至京师向御史中丞禀报。此时的刺史为监察官,秩六百石,较郡守的秩比二千石为低。

西汉初,诸侯王的爵位,封地都是由嫡长子单独继承的,其他子孙得不到尺寸之地。虽然文景两代采取了一定的削藩措施,但是到汉武帝初年,“诸侯或连城数十,地方千里,缓则骄,易为淫乱;急则阻其强而合从,谋以逆京师”,严重威胁着汉朝的中央集权。因此元朔二年正月,武帝采纳主父偃的建议,颁行“推恩令”。推恩令吸取了晁错削藩令引起七国之乱的教训,规定诸侯王除以嫡长子继承王位外,其余诸子在原封国内封侯,新封侯国不再受王国管辖,直接由各郡来管理,地位相当于县。这使得诸侯王国名义上没有任何的削藩,避免激起诸侯王武装反抗的可能。于是“藩国始分,而子弟毕侯矣”,导致封国越分越小,势力大为削弱,从此“大国不过十余城,小侯不过十余里”。

察举制為中國古代有系統選拔人才制度之濫觴,對後世影響極大。主要用于选拔官吏。它的确立是从汉武帝元光元年(公元前134年)开始的。察举制不同于以前先秦时期的世袭制和从隋唐时建立的科举制,它的主要特徵是由地方长官在辖区内随时考察、选取人才并推荐给上级或中央,经过试用考核再任命官职。察举制此后成为汉代聘用官吏的制度,有的学者曾经指出,汉武帝“初令郡国举孝廉各一人”的元光元年,是“中国学术史和中国政治史的最可纪念的一年。”

征辟制是汉武帝时推行的一种自上而下选拔官吏制度,就是征召名望显赫的人士出来做官,主要有皇帝征聘和府、州郡辟除两方面,皇帝征召称“征”,官府征召称“辟”。用以作为察举制的补充。

在中国歷史上,年号由汉武帝發明及首先使用,首個年号为建元(前140年—前135年)。此前的帝王只有年数,没有年号。據滿清趙翼的《二十二史札記》考證,年號紀年是在漢武帝十九年首創的,年號為「元狩」,并追認元狩前的年號建元、元光和元朔。《漢書》上記載說,前122年十月,漢武帝出去狩獵,捉到一隻獨角獸白麟,群臣認為這是吉祥的神物,值得紀念,建議用來記年,於是立年號為「元狩」,稱那年(前122年)為元狩元年。可是,過了六年,又在山西汾陽地方獲得一只三個腳的寶鼎,群臣又認為這是吉祥的神物,建議用來紀年,於是改年號為「元鼎」,稱那年為元鼎元年。後來,人們把這記錄年代的開始之年稱为「紀元」,改換年號叫做「改元」。此后,每次新皇帝登基,常常会改元。一般改元从下诏的第2年算起,也有一些从本年年中算起。

汉武盛世是西汉的全盛时期。

匈奴自秦末以来一直威胁中国北边,使农耕生产的受到严重影响。武帝即位之后,自前133年马邑之战起,结束漢朝初期对匈奴的和亲政策,决心设法解决匈奴的外患问题。从元光六年(前129年)开始对匈奴作战。经过卫青和霍去病等人的反击后,西汉西北边境上的威胁暂时解除。中原北边农耕经济从匈奴造成严重破坏的局面中得以恢复。匈奴在军队主力以及人畜资产受到严重损失的情况下,继续向北远遁,并有七年时间即从公元前119年至前112年漠南无王庭,不过其后匈奴又南下与羌人组织联盟攻击汉朝。而西汉军队占领从朔方至张掖、居延廷间的大片土地,设置酒泉、武威、张掖及敦煌四郡,并且命令关东地区人民移民这一地区,此举不但保障河西走廊的安全,使西部地区的得到开发,更打通了中原文化和西域文化交通的通路。

汉武帝除了北伐匈奴之外,也武力平定四方,大幅开扩领土,在西南,漢朝消灭夜郎及南越國,先后建立七個郡,使到今日的两广地区自秦朝后重新归納中國版图。而海南岛在历史上,也首次真正纳入中国的版图。在東方,他於公元前109年至前108年派兵消灭卫氏朝鲜,並且將衛氏朝鮮的國土分為四郡──樂浪郡、真番郡、臨屯郡及玄菟郡。

汉武帝派遣了張騫出使西域,张骞的两次出使打通了中原文化和西域文化交通的通路。即丝绸之路,极大促进了中國同西方经济及文化的交流。

建元元年(辛丑,公元前140年)诏举贤良方正直言极谏之士,上亲策问以古今治道。广川董仲舒上天人三策,对曰“《春秋》大一统者,天地之常经,古今之通谊也。今师异道,人异论,百家殊方,指意不同,是以上无以持一统,法制数变,下不知所守。臣愚以为诸不在六艺之科、孔子之术者,皆绝其道,勿使并进,邪辟之说灭息,然后统纪可一而法度可明,民知所从矣!”。汉武帝采用了董仲舒的建议,「罷黜百家,独尊儒术」。结束先秦以来“师异道,人异论,百家殊方”的局面,于是“令后学者有所统一”。为儒学在中国古代的特殊地位铺路,亦使到儒学成为了中國社会的基礎思想。对中國後代的政治、社會及文化等領域产生了深远的影响。但是,亦有人认为他利用儒学敦化民風,同时采用法術及刑名鞏固政府的權威,即是所谓儒表法裏。

汉武帝元朔五年,创建太学,是接受当时儒家学者董仲舒的建议。董仲舒指出,太学可以作为“教化之本原”,也就是作为教化天下的文化基地。他建议,“臣愿陛下兴太学,置明师,以养天下之士”,这样可以使国家得到未来的人才。所谓“养天下之士”,体现出太学在当时有为国家培育人才和储备人才的作用。汉武帝时期的太学,虽然规模很有限,只有几位经学博士和五十名博士弟子,但是这一文化雏形,代表着中国古代教育发展的方向。太学的成立,助长民间积极向学的风气,对于文化的传播,成为重要的推手,同时使大官僚和大富豪子嗣垄断官位的情形有所转变,一般人家子弟得以增加入仕的机会,一些出身社会下层的人才,也有机会到朝廷做官。

樂府一名本指管理音樂的官府。漢武帝在掌管雅樂的太樂官署之外,另創立樂府官署,掌管俗樂,收集民間的歌辭入樂。「采詩夜誦,有趙、代、秦、楚之謳」、「以李延年為協律都尉,多舉司馬相如等數十人造為詩賦,略論律呂,以合八音之調,作十九章之歌」。後人把樂府機關配樂演唱的詩歌,也稱樂府。

太初历是中国历史上曾经使用过的一种历法,亦是中国历史上第一部完整统一,而且有明确文字记载的历法。在天文学发展历史上具有划时代的意义。在汉武帝太初元年(前104年),由邓平、唐都、落下闳及司馬遷等根据对天象实测和长期天文纪录所制订。《太初历》的制订是中国历史上具有重要性的一次历法大改革。《太初历》的科学成就,首先在于历法计算上的精密准确。中国汉初以前,主要采用“古六历”(黄帝、颛顼、夏、殷、周、鲁)中的《颛顼历》。这个古历,计算一年为三百六十五又四分之一日,一月是二十九天又九百四十分之四百九十九。由于这种古历计算不够精密,常出现月初是无月光的朔日,但实际天空中却有圆满的月光;月中是有月光的望满之日,夜晚却并没有月亮。为了改变这种不对照的现象,司马迁主持制订《太初历》时,重新进行了反复地周密地运算和实践验证。还在于第一次计算了日月蚀发生的周期和精确计算了行星会合的周期。

指中国西汉武帝统治时期进行的币制改革。西汉自建立以来,币制混乱,郡国铸币失控又是汉景帝时期七国之乱發生的原因之一。汉武帝统治时期,由于对外征伐不断,中央财政从此前“京师之钱累巨万,贯朽而不可校”的丰盈一变而为入不敷出的困局。“而富商大贾或蹛财役贫,转榖百数,废居居邑,封君皆低首仰给。”富商大贾富可敌国,恰与窘困的中央财政形成了鲜明对比。中央政府除了靠鬻武功爵等方式快速增加财政收入外,“冶铸煑盐,财或累万金,而不佐国家之急,黎民重困。于是天子与公卿议,更钱造币以赡用,而摧浮淫并兼之徒。”增加中央财政收入,打击大商人,此即汉武帝币制改革的初衷。故汉武帝即位后,为了中央政府在经济管理和政治统治上的需要,便十分重视解决币制问题,先后进行了六次币制改革,基本解决了汉初以来一直未能解决的币制问题。一方面稳定了金融,另一方面将地方的铸币权重新统一于中央。六次改革后三官五铢的发行一举解决了困扰西汉金融多年的私铸、盗铸问题,汉武帝的币制改革至此取得了较大成功。

中央政府在盐、铁产地分别设置盐官和铁官,实行统一生产和统一销售,利润为国家所有。这项制度实施,使国家独占国计民生意义最重要的手工业和商业的利润,可以供给皇室消费以及巨额军事支出。当时,人民的赋税的负担没有增加,国家的收入大增,不但弥补财政上的赤字,并且还有羸余。不过官营盐铁却给社会经济和民众生活带来负面的影响。例如官盐价高而味苦,铁制农具粗劣不合用等。

漢武帝元封元年,桑弘羊針對「諸官各自市(購買),相與爭,物以故騰躍,而天下賦輸,或不償其僦費」的情況,在全國推行均輸法,下令各郡設均輸鹽鐵官,將上貢物品運往缺乏該類貨物的地區出售,然後在適當地區購入京師需求的物資。此法既能解決運費高昂的問題,又可調節物價。更重要的是均輸法舒緩漢武帝晚年的財政危機,桑弘羊對此曾有所讚揚:「山東被災,齊趙大饑,賴均輸之蓄,倉稟之積,戰士以奉,饑民以賑」。然而,均輸法卻被批評未能解決物價問題,「輕賈奸吏,收賤以取貴,未見準之平也」。

在經濟方面,漢武帝爲推動農業,采取了一系列措施。他在全國修了不少水利工程,例如:龍首渠,六輔渠等等,以便農田灌溉。再加上新式耕種技術的提倡,農業生產得到進一步發展。

征和元年(西元前92年)十一月,巫蠱之禍興起。丞相公孫賀之妻使用巫術詛咒及在馳道埋木偶人的事件被告發,公孫賀一家被斬殺,同時還牽連到陽石公主和皇后卫子夫所生的女兒諸邑公主。其後漢武帝又發動了三輔騎士在皇家園林進行搜查,並且在長安城中到處尋找,過了11日才收兵。征和二年七月,與太子劉據結怨的武帝寵臣江充指使胡巫,說宫中有蠱氣。武帝命令江充與按道侯韓說等入宫追查,江充誣告太子宫中埋的木人最多,又有帛書,所言不守道法。太子得知後非常恐懼,聽從少傅石德的計策,派人詐稱武帝使者捕殺江充等人。漢武帝命令丞相劉屈氂派兵擊潰太子,太子舉兵對抗。激戰五日,太子兵敗逃亡,被漢武帝所廢,被圍捕,乃自殺,滅族,唯其曾孫劉病已得親信保全。征和三年,此謀反案的根源巫蛊案真相漸明,大臣上书直言进谏,武帝感悟,下令族滅貳師將軍李廣利、丞相劉屈氂、太監蘇文、江充家族。

漢武帝將鹽鐵酒國營專賣,實行平準均輸政策,防止商人從中漁利,從而增加政府收入,達到了調節物價及防止市場壟斷的功效,但是亦造成了與民爭利的局面。商人遂將注意力轉移至土地買賣,導致土地兼併嚴重。雖然漢武帝武功極盛,但是到處征伐也造成了國庫空虛,大量人民被徵召從軍,死傷甚重,也影響了經濟發展。由於民生困苦、社會動盪不安、人民流離失所及民怨沸騰,天漢二年(前99年),齊、楚、燕、趙和南陽等地相繼爆發大規模農民起義;征和四年(前89年)漢武帝頒下了《輪台罪己詔》向人民承認自己的罪過及公開罪己詔。

汉武帝晚年得子刘弗陵,甚爱之。刘据於巫蛊之乱死後,漢武帝立刘弗陵为太子。太子即位前不久,其生母钩弋夫人被處死,避免未來再有太后涉政的現象。前88年,汉武帝命令画工画了一张《周公背成王朝诸侯图》送予霍光,意思是让霍光辅佐他的小儿刘弗陵作為未來皇帝。對此,中国史学家吕思勉对《汉书·霍光传》的此记载颇有异议,认为汉武帝於临终前杀掉刘弗陵生母是为了避免母后干政、托孤说的“立少子,君行周公之事”和画周公辅政图完全属于捏造。

前87年3月29日(二月丁卯),汉武帝驾崩於五柞宮,享年70岁。4月15日(三月甲申),葬于茂陵,谥号为孝高皇帝,庙号為世宗。

漢武帝愛好文學,為提倡辭賦的詩人。他個人的文學造詣甚高,在南北朝以前的皇帝中屬於文采一流的人物,顏之推把他歸類為曹操、曹丕一級文才的君主。明朝王世貞以為,其成就在“長卿下、子雲上”(《藝苑卮鹽》)其他存留的詩作,《瓠子歌》、《天馬歌》、《悼李夫人賦》都“壯麗鴻奇”(徐禎卿《談藝錄》),為詩詞評論家所推崇。

夏侯胜:武帝虽有攘四夷广土斥境之功,然多杀士众,竭民财力,奢泰亡度,天下虚耗,百姓流离,物故者半。蝗虫大起,赤地数千里,或人民相食,畜积至今未复。亡德泽於民,不宜为立庙乐。

桓谭:「汉武帝才质高妙,有崇先广统之规,故即位而开发大志,考合古今模范,获前圣代故事,建正朔,定制度,招选俊杰,奋扬威怒,武义四加,所征者服,兴起六艺,广进儒术,自开辟以来,惟汉家最为盛图,故显为世宗,可谓卓尔绝世之主矣。」

崔骃:昔孝武皇帝始为天子,年方十八,崇信圣道,师则先王,五六年间,号胜文、景。及后恣己,忘其前之为善。

刘歆:『孝武皇帝愍中国罢劳,无安宁之时,乃遣大将伏波、楼船之属,灭百越七郡。北攘匈奴,降昆邪之众,置五属国,起朔方,以夺其肥饶之地。东伐朝鲜,起玄菟、乐浪以断匈奴之左臂。西伐大宛,并三十六国,结乌孙,起敦煌、酒泉、张掖、武威,以隔氐羌,裂匈奴之右肩。单于孤将远遁漠北,四垂无事,斥地远境,起十馀郡。功业既定,乃封丞相为富民侯,以安天下,富实百姓,其规模可见。又招集天下贤俊,与协心同谋,兴制度,改正朔,易服色,立天地之祀。建封禅,殊官号,存周后,定诸侯之制,永无逆争之心,至今累世赖之。单于守蕃,百蛮服从,万世之基也。中兴之功,未有高焉者也。』

何去非:“孝武帝以雄才大略,承三世涵育之泽,知夫天下之势将就强而不振,所当济之以威强而抗武节之时也。”是以孝武抗其英特之气,选待习骑,择命将帅,先发而昌诛之。盖师行十年,斩刈殆尽,名王贵人俘获百数,单于捧首穷遁漠北,遂收两河之地而郡属之。刷四世之侵辱,遗后嗣之安强。”

班固:“漢承百王之弊,高祖撥亂反正,文、景務在養民,至於稽古禮文之事,猶多闕焉。孝武初立,卓然罷黜百家,表章《六經》,遂疇咨海內,舉其俊茂,與之立功。興太學,修郊祀,改正朔,定歷數,協音律,作詩樂。建封禪,禮百神,紹周後,號令文章,煥然可述,後嗣得遵洪業而有三代之風。如武帝之雄材大略,不改文、景之恭儉以濟斯民,雖《詩》、《書》所稱何有加焉!”

曹丕:“孝武帝承累世之遗业,遇中国之殷阜,府库余金钱,仓廪畜腐粟,因此有意乎灭匈奴而廓清边境矣。故即位之初,从王恢之画,设马邑之谋,自元光以迄征和四五十载之间,征匈奴四十馀:举盛馀,逾广汉,绝梓岭,封狼居胥,禅姑幕,梁北河,观兵瀚海,刈单于之旗,剿阏氏之首,探符离之窟,扫五王之庭。纳休屠昆邪之附,获祭天金人之宝。斩名王以千数,馘酋虏以万计。既穷追其散亡,又摧破其积聚,虏不暇于救死扶伤,疲困于孕重堕殒。元封初,躬秉武节,告以天子自将,惧以两越之诛,彼时号为威震匈奴矣。”

曹植:“世宗光光,文武是攘。威震百蛮,恢拓土疆。简定律历,辨修旧章。封天禅土,功越百王。”

虞世南:“汉武承六世之业,海内殷富,又有高人之资,故能总揽英雄,驾御豪杰,内兴礼乐,外开边境,制度宪章,焕然可述。方於始皇,则为优矣。”

唐太宗:「近代平一天下,拓定邊方者,惟秦皇、漢武。」

司马贞的《史记索隐》:「孝武纂极,四海承平。志尚奢丽,尤敬神明。坛开八道,接通五城。朝亲五利,夕拜文成。祭非祀典,巡乖卜征。登嵩勒岱,望景传声。迎年祀日,改历定正。疲秏中土,事彼边兵。日不暇给,人无聊生。俯观嬴政,几欲齐衡。」

司马光的《资治通鉴》:“孝武穷奢极欲,繁刑重敛,内侈宫室,外事四夷,信惑神怪,巡游无度,使百姓疲敝,起为盗贼,其所以异于秦始皇无几矣。然秦以之亡,汉以之兴者,孝武能尊王之道,知所统守,受忠直之言,恶人欺蔽,好贤不倦,诛罚严明,晚而改过,顾托得人,此其所以有亡秦之失而免亡秦之祸乎!”

李纲:“茂陵仙客,算真是,天与雄才宏略。猎取天骄驰卫霍,如使鹰鹯驱雀。战皋兰,犁庭龙碛,饮至行勋爵。中华疆盛,坐令夷狄衰弱。追想当日巡行,勒兵十万骑,横临边朔。亲总貔貅谈笑看,黠虏心惊胆落。寄语单于,两君相见,何苦逃沙漠。英风如在,卓然千古高著。”

洪迈:“汉之武帝、唐之武后,不可谓不明。”

朱熹:“武帝天资高,志向大,足以有为。末年海内虚耗,去秦始皇无几。轮台之悔,亦是天资高,方能如此。”

王夫之:「武帝之勞民甚矣,而其救饑民也為得。虛倉廥以振之,寵富民之假貸者以救之,不給,則通其變而徙荒民於朔方、新秦者七十余萬口,仰給縣官,給予產業,民喜於得生,而輕去其鄉以安新邑,邊因以實。」

王夫之:「武帝之发觉而捕弗满品者,二千石以下至小吏,主者皆死,则欲吏之弗匿盗不上闻、而以禁其窃发也,必不可得矣。……漢武有喪邦之道焉,此其一矣。」

赵翼:「仰思帝之雄才大略,正在武功。」

吴裕垂:「武帝雄才大略,非不深知征伐之劳民也,盖欲复三代之境土。削平四夷,尽去后患,而量力度德,慨然有舍我其谁之想。于是承累朝之培养,既庶且富,相时而动,战以为守,攻以为御,匈奴远遁,日以削弱。至于宣、元、成、哀,单于称臣,稽玄而朝,两汉之生灵,并受其福,庙号‘世宗’,宜哉!」

夏曾佑:“有为汉一朝之皇帝者,高祖是也;有为中国二十四朝之皇帝者,秦皇、汉武是也。”

白寿彝:“促进了经济繁荣与国家统一。”

翦伯赞:“用剑犹如用情,用情犹如用兵”。

黄仁宇:“有专制魔王的毛病。”

钱穆:“‘王莽代汉’源自汉武帝种下的恶果。”

孙中山:““秦皇汉武、元世祖、拿破仑,或数百年,数十年而斩,亦可谓有志之士矣。拿破仑兴法典,汉武帝纪赞,不言武功,又有千年之志者。”

毛泽东:“汉武帝雄才大略,开拓刘邦的业绩,晚年自知奢侈、黩武、方士之弊,下了罪己诏,不失为鼎盛之世。”

根據《史記》和《漢書》的描述,漢武帝為雙性戀。记载于史书上的佞幸(有公职或者贵族身份的男性情人)有韩嫣、李延年和韩说。《佞幸列傳》紀錄李延年“與上臥起,甚貴幸。”大臣金日磾之子亦曾經為弄儿(少年男性情人)。

《天地陰陽交歡大樂賦》、《藝文類聚·寵幸》、《情史·情外類》引述史書裡,漢武帝寵幸的韓嫣記載,視之為男風代表。

汉武帝巡游汾河,在船上和群臣飲宴,汉武帝突然对群臣说:“汉朝有六七之厄,六七四十二,汉朝传到第42代皇帝,会有当涂高取代汉朝。”群臣说:“汉朝应天受命,王朝长过商周,永世不绝,陛下为何说这种亡国之言?”汉武帝表示「只是醉言,但是自古以来没有一姓可以一直拥有天下,不过即使汉朝灭亡,不要灭亡在我父子手上就行。」

当涂高的意思是路上有很高的东西,后来的公孙述、袁术和曹丕等都用「代汉者当涂高」这句谶言为自己称帝造势。

汉武帝建元年间,汉武帝和随从微服外出打猎,麻烦事不断。一天夜晚汉武帝和随从投宿旅社,旅社主人觉得一行人来者不善,对汉武帝等人非常傲慢。旅社主人准备和门客一同杀死汉武帝等人,但是主人妻子觉得汉武帝等人气势非凡,不像强盗,于是将她丈夫灌醉,偷偷放走汉武帝等人。后来又不慎踩伤农民庄稼,引发纠纷,农民叫来县令。汉武帝自称平阳侯,县令本想拜谒,汉武帝随从却想鞭打县令。县令大怒,扣押汉武帝随从,拒绝他们离开。汉武帝不得已,向县令展示皇家身份,县令才予以放行。后来汉武帝微服外出的举动被众人得知,地方政府纷纷建立行宫招待汉武帝。汉武帝认为微服外出会扰民,干脆建立上林苑,专供皇家打猎。

\subsection{建元}

\begin{longtable}{|>{\centering\scriptsize}m{2em}|>{\centering\scriptsize}m{1.3em}|>{\centering}m{8.8em}|}
  % \caption{秦王政}\
  \toprule
  \SimHei \normalsize 年数 & \SimHei \scriptsize 公元 & \SimHei 大事件 \tabularnewline
  % \midrule
  \endfirsthead
  \toprule
  \SimHei \normalsize 年数 & \SimHei \scriptsize 公元 & \SimHei 大事件 \tabularnewline
  \midrule
  \endhead
  \midrule
  元年 & -140 & \tabularnewline\hline
  二年 & -139 & \tabularnewline\hline
  三年 & -138 & \tabularnewline\hline
  四年 & -137 & \tabularnewline\hline
  五年 & -136 & \tabularnewline\hline
  六年 & -135 & \tabularnewline
  \bottomrule
\end{longtable}


\subsection{元光}

\begin{longtable}{|>{\centering\scriptsize}m{2em}|>{\centering\scriptsize}m{1.3em}|>{\centering}m{8.8em}|}
  % \caption{秦王政}\
  \toprule
  \SimHei \normalsize 年数 & \SimHei \scriptsize 公元 & \SimHei 大事件 \tabularnewline
  % \midrule
  \endfirsthead
  \toprule
  \SimHei \normalsize 年数 & \SimHei \scriptsize 公元 & \SimHei 大事件 \tabularnewline
  \midrule
  \endhead
  \midrule
  元年 & -134 & \tabularnewline\hline
  二年 & -133 & \tabularnewline\hline
  三年 & -132 & \tabularnewline\hline
  四年 & -131 & \tabularnewline\hline
  五年 & -130 & \tabularnewline\hline
  六年 & -129 & \tabularnewline
  \bottomrule
\end{longtable}


\subsection{元朔}

\begin{longtable}{|>{\centering\scriptsize}m{2em}|>{\centering\scriptsize}m{1.3em}|>{\centering}m{8.8em}|}
  % \caption{秦王政}\
  \toprule
  \SimHei \normalsize 年数 & \SimHei \scriptsize 公元 & \SimHei 大事件 \tabularnewline
  % \midrule
  \endfirsthead
  \toprule
  \SimHei \normalsize 年数 & \SimHei \scriptsize 公元 & \SimHei 大事件 \tabularnewline
  \midrule
  \endhead
  \midrule
  元年 & -128 & \tabularnewline\hline
  二年 & -127 & \tabularnewline\hline
  三年 & -126 & \tabularnewline\hline
  四年 & -125 & \tabularnewline\hline
  五年 & -124 & \tabularnewline\hline
  六年 & -123 & \tabularnewline
  \bottomrule
\end{longtable}

\subsection{元狩}

\begin{longtable}{|>{\centering\scriptsize}m{2em}|>{\centering\scriptsize}m{1.3em}|>{\centering}m{8.8em}|}
  % \caption{秦王政}\
  \toprule
  \SimHei \normalsize 年数 & \SimHei \scriptsize 公元 & \SimHei 大事件 \tabularnewline
  % \midrule
  \endfirsthead
  \toprule
  \SimHei \normalsize 年数 & \SimHei \scriptsize 公元 & \SimHei 大事件 \tabularnewline
  \midrule
  \endhead
  \midrule
  元年 & -122 & \tabularnewline\hline
  二年 & -121 & \tabularnewline\hline
  三年 & -120 & \tabularnewline\hline
  四年 & -119 & \tabularnewline\hline
  五年 & -118 & \tabularnewline\hline
  六年 & -117 & \tabularnewline  
  \bottomrule
\end{longtable}

\subsection{元鼎}

\begin{longtable}{|>{\centering\scriptsize}m{2em}|>{\centering\scriptsize}m{1.3em}|>{\centering}m{8.8em}|}
  % \caption{秦王政}\
  \toprule
  \SimHei \normalsize 年数 & \SimHei \scriptsize 公元 & \SimHei 大事件 \tabularnewline
  % \midrule
  \endfirsthead
  \toprule
  \SimHei \normalsize 年数 & \SimHei \scriptsize 公元 & \SimHei 大事件 \tabularnewline
  \midrule
  \endhead
  \midrule
  元年 & -116 & \tabularnewline\hline
  二年 & -115 & \tabularnewline\hline
  三年 & -114 & \tabularnewline\hline
  四年 & -113 & \tabularnewline\hline
  五年 & -112 & \tabularnewline\hline
  六年 & -111 & \tabularnewline  
  \bottomrule
\end{longtable}

\subsection{元封}

\begin{longtable}{|>{\centering\scriptsize}m{2em}|>{\centering\scriptsize}m{1.3em}|>{\centering}m{8.8em}|}
  % \caption{秦王政}\
  \toprule
  \SimHei \normalsize 年数 & \SimHei \scriptsize 公元 & \SimHei 大事件 \tabularnewline
  % \midrule
  \endfirsthead
  \toprule
  \SimHei \normalsize 年数 & \SimHei \scriptsize 公元 & \SimHei 大事件 \tabularnewline
  \midrule
  \endhead
  \midrule
  元年 & -110 & \tabularnewline\hline
  二年 & -109 & \tabularnewline\hline
  三年 & -108 & \tabularnewline\hline
  四年 & -107 & \tabularnewline\hline
  五年 & -106 & \tabularnewline\hline
  六年 & -105 & \tabularnewline
  \bottomrule
\end{longtable}

\subsection{太初}

\begin{longtable}{|>{\centering\scriptsize}m{2em}|>{\centering\scriptsize}m{1.3em}|>{\centering}m{8.8em}|}
  % \caption{秦王政}\
  \toprule
  \SimHei \normalsize 年数 & \SimHei \scriptsize 公元 & \SimHei 大事件 \tabularnewline
  % \midrule
  \endfirsthead
  \toprule
  \SimHei \normalsize 年数 & \SimHei \scriptsize 公元 & \SimHei 大事件 \tabularnewline
  \midrule
  \endhead
  \midrule
  元年 & -104 & \tabularnewline\hline
  二年 & -103 & \tabularnewline\hline
  三年 & -102 & \tabularnewline\hline
  四年 & -101 & \tabularnewline
  \bottomrule
\end{longtable}

\subsection{天汉}

\begin{longtable}{|>{\centering\scriptsize}m{2em}|>{\centering\scriptsize}m{1.3em}|>{\centering}m{8.8em}|}
  % \caption{秦王政}\
  \toprule
  \SimHei \normalsize 年数 & \SimHei \scriptsize 公元 & \SimHei 大事件 \tabularnewline
  % \midrule
  \endfirsthead
  \toprule
  \SimHei \normalsize 年数 & \SimHei \scriptsize 公元 & \SimHei 大事件 \tabularnewline
  \midrule
  \endhead
  \midrule
  元年 & -100 & \tabularnewline\hline
  二年 & -99 & \tabularnewline\hline
  三年 & -98 & \tabularnewline\hline
  四年 & -97 & \tabularnewline
  \bottomrule
\end{longtable}

\subsection{太始}

\begin{longtable}{|>{\centering\scriptsize}m{2em}|>{\centering\scriptsize}m{1.3em}|>{\centering}m{8.8em}|}
  % \caption{秦王政}\
  \toprule
  \SimHei \normalsize 年数 & \SimHei \scriptsize 公元 & \SimHei 大事件 \tabularnewline
  % \midrule
  \endfirsthead
  \toprule
  \SimHei \normalsize 年数 & \SimHei \scriptsize 公元 & \SimHei 大事件 \tabularnewline
  \midrule
  \endhead
  \midrule
  元年 & -96 & \tabularnewline\hline
  二年 & -95 & \tabularnewline\hline
  三年 & -94 & \tabularnewline\hline
  四年 & -93 & \tabularnewline
  \bottomrule
\end{longtable}

\subsection{征和}

\begin{longtable}{|>{\centering\scriptsize}m{2em}|>{\centering\scriptsize}m{1.3em}|>{\centering}m{8.8em}|}
  % \caption{秦王政}\
  \toprule
  \SimHei \normalsize 年数 & \SimHei \scriptsize 公元 & \SimHei 大事件 \tabularnewline
  % \midrule
  \endfirsthead
  \toprule
  \SimHei \normalsize 年数 & \SimHei \scriptsize 公元 & \SimHei 大事件 \tabularnewline
  \midrule
  \endhead
  \midrule
  元年 & -92 & \tabularnewline\hline
  二年 & -91 & \tabularnewline\hline
  三年 & -90 & \tabularnewline\hline
  四年 & -89 & \tabularnewline
  \bottomrule
\end{longtable}

\subsection{后元}

\begin{longtable}{|>{\centering\scriptsize}m{2em}|>{\centering\scriptsize}m{1.3em}|>{\centering}m{8.8em}|}
  % \caption{秦王政}\
  \toprule
  \SimHei \normalsize 年数 & \SimHei \scriptsize 公元 & \SimHei 大事件 \tabularnewline
  % \midrule
  \endfirsthead
  \toprule
  \SimHei \normalsize 年数 & \SimHei \scriptsize 公元 & \SimHei 大事件 \tabularnewline
  \midrule
  \endhead
  \midrule
  元年 & -88 & \tabularnewline\hline
  二年 & -87 & \tabularnewline
  \bottomrule
\end{longtable}


%%% Local Variables:
%%% mode: latex
%%% TeX-engine: xetex
%%% TeX-master: "../Main"
%%% End:

%% -*- coding: utf-8 -*-
%% Time-stamp: <Chen Wang: 2021-10-29 17:26:39>

\section{昭帝劉弗陵\tiny(BC87-BC74)}

\subsection{生平}

漢昭帝劉弗陵(前94年-前74年6月5日),西漢第八位皇帝(前87年—前74年在位),其正式諡號為「孝昭皇帝」,後世省略「孝」字稱「漢昭帝」,漢武帝幼子,母親鉤弋夫人。昭帝身高八尺二寸(約1.89米)。

據說鉤弋夫人懷孕14個月才生下劉弗陵,大臣們都以為堯帝降生,紛紛恭祝武帝。武帝老年得子,更是愛不釋手,常说像自己。巫蛊之祸兴起,征和二年(前91年),刘弗陵的长兄太子刘据兵败逃亡后自杀。

武帝駕崩前,準備立劉弗陵為太子,但是為了防止「子幼母壯」、外戚專權的事情發生,他藉故處死了鉤弋夫人,然後請得力大將霍去病的異母弟霍光為首輔、匈奴人金日磾為次輔、上官桀為佐軍以及桑弘羊為理財等四重臣來輔佐劉弗陵。武帝駕崩後,劉弗陵在重臣的擁立下登基繼位。

为了便于臣民避讳,昭帝去掉名中的“陵”字,改名刘弗,并讓臣民把弗寫成不。

昭帝登基時才8歲,平時受到鄂邑公主照顧,并從蔡義和韦贤那裡學習《詩經》。

漢昭帝和金日磾兒子金賞、金建關係十分友好,平時一起寢居,漢昭帝看到金賞繼承金日磾的侯爵,想封金建為侯,卻遭到霍光拒絕。

面對漢武帝時代的連年征戰、增加徭役,昭帝聽取重臣的建言,減少賦稅3成,進一步深化武帝晚年重新施行漢初與民休息的政策。在首輔大臣霍光的主持下,昭帝朝的百姓生活比以前富裕,四夷來朝,使漢朝出現中興穩定的局面。

霍光外孫女上官氏當上皇后,霍光想讓皇后擅寵生子,於是不讓漢昭帝親近其他宮女。

前74年6月5日(四月癸未),昭帝於未央宮暴病而崩,年僅21歲,在位13年。7月24日(六月壬申),漢昭帝葬於今天咸阳市的平陵。

昭帝无子,其侄昌邑王刘贺被立为嗣。

漢昭帝一次遊覽渭河,他的隨行大臣釣上了一頭白蛟,長三丈。漢昭帝饒有興趣,讓廚師將其醃製,味道鮮美,漢昭帝飯後回味無窮。之後卻再也沒釣到這種魚。

始元元年(前86年),漢昭帝在太液池上看見黄鹄,於是創作了黄鹄歌。同年,淋池修建,後來漢昭帝在淋池中遊樂,并讓宮人唱《淋池歌》,很是愉快。

\subsection{始元}

\begin{longtable}{|>{\centering\scriptsize}m{2em}|>{\centering\scriptsize}m{1.3em}|>{\centering}m{8.8em}|}
  % \caption{秦王政}\
  \toprule
  \SimHei \normalsize 年数 & \SimHei \scriptsize 公元 & \SimHei 大事件 \tabularnewline
  % \midrule
  \endfirsthead
  \toprule
  \SimHei \normalsize 年数 & \SimHei \scriptsize 公元 & \SimHei 大事件 \tabularnewline
  \midrule
  \endhead
  \midrule
  元年 & -86 & \tabularnewline\hline
  二年 & -85 & \tabularnewline\hline
  三年 & -84 & \tabularnewline\hline
  四年 & -83 & \tabularnewline\hline
  五年 & -82 & \tabularnewline\hline
  六年 & -81 & \tabularnewline\hline
  七年 & -80 & \tabularnewline
  \bottomrule
\end{longtable}


\subsection{元凤}

\begin{longtable}{|>{\centering\scriptsize}m{2em}|>{\centering\scriptsize}m{1.3em}|>{\centering}m{8.8em}|}
  % \caption{秦王政}\
  \toprule
  \SimHei \normalsize 年数 & \SimHei \scriptsize 公元 & \SimHei 大事件 \tabularnewline
  % \midrule
  \endfirsthead
  \toprule
  \SimHei \normalsize 年数 & \SimHei \scriptsize 公元 & \SimHei 大事件 \tabularnewline
  \midrule
  \endhead
  \midrule
  元年 & -80 & \tabularnewline\hline
  二年 & -79 & \tabularnewline\hline
  三年 & -78 & \tabularnewline\hline
  四年 & -77 & \tabularnewline\hline
  五年 & -76 & \tabularnewline\hline
  六年 & -75 & \tabularnewline
  \bottomrule
\end{longtable}


\subsection{元平}

\begin{longtable}{|>{\centering\scriptsize}m{2em}|>{\centering\scriptsize}m{1.3em}|>{\centering}m{8.8em}|}
  % \caption{秦王政}\
  \toprule
  \SimHei \normalsize 年数 & \SimHei \scriptsize 公元 & \SimHei 大事件 \tabularnewline
  % \midrule
  \endfirsthead
  \toprule
  \SimHei \normalsize 年数 & \SimHei \scriptsize 公元 & \SimHei 大事件 \tabularnewline
  \midrule
  \endhead
  \midrule
  元年 & -74 & \tabularnewline
  \bottomrule
\end{longtable}


%%% Local Variables:
%%% mode: latex
%%% TeX-engine: xetex
%%% TeX-master: "../Main"
%%% End:

%% -*- coding: utf-8 -*-
%% Time-stamp: <Chen Wang: 2021-10-29 17:26:58>

\section{宣帝刘贺\tiny(BC74-BC49)}

\subsection{海昏侯生平}

宣帝刘贺(?-前59年),西汉第九位皇帝(前74年7月18日至同年8月14日在位)。漢武帝孫,原為昌邑王,在漢昭帝駕崩後繼位,在位27日,被權臣霍光以行事乖戾為由廢黜,後改封為海昏侯。

劉賀袭父劉髆(漢武帝與李夫人之子)封为昌邑王(地近今山東菏泽)。

昌邑郎中令龔遂以明經為官,侍奉昌邑王劉賀。昌邑王行為舉止多不正當,龔遂為人忠厚,剛毅有大節操,對內對昌邑王勸諫,對外責備師傅卿相,引用經義,陳言禍福,以至於流淚,正言直諫而不停止。當面指責昌邑王過錯,昌邑王甚至掩起耳朵跑走,說:「郎中令善愧人。」昌邑國中皆畏懼龔遂。

昌邑王曾長久與駕車奴僕、炊事人員嘻戲共食,賞賜沒有節制。龔遂晉見昌邑王,跪著行走,流淚不止,左右侍御都淚流滿面。昌邑王問:「郎中令何為哭?」龔遂說:「臣痛社稷危也!願賜清閒竭愚。」昌邑王令左右退下,龔遂問:「大王知膠西王所以為無道亡乎?」昌邑王說:「不知也。」龔遂說:「臣聞膠西王有諛臣侯得,王所為擬於桀、紂也,得以為堯、舜也。王說其諂諛,嘗與寢處,唯得所言,以至於是。今大王親近群小,漸漬邪惡所習,存亡之機,不可不慎也。臣請選郎通經術有行義者與王起居,坐則通《詩》、《書》,立則習禮容,宜有益。」昌邑王同意他。龔遂於是挑選郎中張安等十人侍奉昌邑王。幾天之後,昌邑王卻遠離張安等人。

昌邑王在昌邑國時,曾數次出現妖怪。曾見到有白色的狗,身長三尺,沒有頭,在其頸部以下似人,頭戴方山冠。後來出現了熊,昌邑王左右都沒看見。又有大鳥飛來集聚在昌邑宮中。昌邑王得知後,非常討厭,數次以此詢問郎中令龔遂。龔遂說:「此天戒,言在側者盡冠狗也,去之則存,不去則亡矣。」之後又聽到人的聲音,說:「熊!」望去而見到大熊,左右都沒看見,昌邑王以此問龔遂,龔遂說:「熊,山野之獸,而來入宮室,王獨見之,此天戒大王,恐宮室將空,危亡象也。」昌邑王望天嘆息說:「不祥何為數來!」龔遂磕頭說:「臣不敢隱忠,數言危亡之戒,大王不說。夫國之存亡,豈在臣言哉?願王內自揆度。大王誦《詩》三百五篇,人事浹,王道備,王之所行中《詩》一篇何等也?大王位為諸侯王,行污於庶人,以存難,以亡易,宜深察之。」之後又有血弄髒了昌邑王的坐席,昌邑王問龔遂,龔遂大聲叫說:「宮空不久,祅祥數至。血者,陰憂象也。宜畏慎自省。」然而,昌邑王依然故我。

昭帝元平元年(前74年),昭帝駕崩,没有子嗣,大司馬大將軍霍光征召昌邑王主持喪禮。

璽書說:“制詔昌邑王:使行大鴻臚事少府樂成、宗正德、光祿大夫吉、中郎將利漢徵王,乘七乘傳詣長安邸。”

凌晨一點左右,用燭火照著打開璽書。當天中午,昌邑王就出發了,下午四五點抵達定陶,走了一百三十五里,侍從的馬一匹接一匹死在路上。郎中令龔遂向昌邑王勸諫,昌邑王才令郎官、謁者五十多人返回昌邑。昌邑王到濟陽,尋求鳴叫聲很長的雞,路上買合竹杖。經過弘農,讓身材高大的奴僕善用裝載衣物的車輛裝載搶來的女子。到了湖縣,使者就此事責備昌邑國相安樂,安樂告訴龔遂,龔遂進去問昌邑王,昌邑王說:「無有。」龔遂說:「即無有,何愛一善以毀行義!請收屬吏,以湔洒大王。」就揪住善,交给衛士執法。

昌邑王到霸上,大鴻臚在郊外迎接,主管車馬的騶官奉上皇帝乘坐的車子。昌邑王讓他的僕人壽成駕車,與郎中令龔遂同車。天明到了廣明東都門,龔遂說:「禮,奔喪望見國都哭。此長安東郭門也。」昌邑王說:「我嗌痛,不能哭。」到了城門,龔遂又說,昌邑王說:「城門與郭門等耳。」當快到未央宫的東門,龔遂說:「昌邑帳在是闕外馳道北,未至帳所,有南北行道,馬足未至數步,大王宜下車,鄉闕西面伏,哭盡哀止。」昌邑王說:「諾。」到了那裡,按禮儀哭喪。昌邑王接受皇帝璽印和绶帶,六月丙寅(公历7月18日),即天子位。

即位以后,昌邑王夢見蒼蠅屎堆積在西階的東面,約五六石,用大瓦覆蓋,揭開一看,是蒼蠅屎。以此問龔遂,龔遂說:「陛下,之《詩》不云乎?『營營青蠅,至於籓;愷悌君子,毋信讒言。』陛下左側讒人眾多,如是青蠅惡矣。宜進先帝大臣子孫親近以為左右。如不忍昌邑故人,信用讒諛,必有凶咎。願詭禍為福,皆放逐之。臣當先逐矣。」

昌邑國國相安樂改任長樂衛尉,龔遂見到安樂,流著眼淚說:「王立為天子,日益驕溢,諫之不復聽,今哀痛未盡,日與近臣飲食作樂,鬥虎豹,召皮軒,車九流,驅馳東西,所為悖道。古制寬,大臣有隱退,今去不得,陽狂恐知,身死為世戮,奈何?君,陛下故相,宜極諫爭。」

《漢書》卷六八《霍光傳》:“遂召丞相、御史、將軍、列侯、中二千石、大夫、博士會議未央宮。光曰:「昌邑王行昏亂,恐危社稷,如何?」群臣皆驚鄂失色,莫敢發言,但唯唯而已。田延年前,離席按劍,曰:「先帝屬將軍以幼孤,寄將軍以天下,以將軍忠賢能安劉氏也。今群下鼎沸,社稷將傾,且漢之傳謚常為孝者,以長有天下,令宗廟血食也。如令漢家絕祀,將軍雖死,何面目見先帝於地下乎?今日之議,不得旋踵。群臣後應者,臣請劍斬之。」光謝曰:「九卿責光是也。天下匈匈不安,光當受難。」於是議者皆叩頭,曰:「萬姓之命在於將軍,唯大將軍令。」光即與群臣俱見白太后,具陳昌邑王不可以承宗廟狀。”

昌邑王即位二十七日,曾与僚属密议罢黜霍光职权,但被霍光以行为淫乱而废。昌邑國群臣因涉昌邑王事而被入罪,皆被處死,死者二百多人,只有龔遂與中尉王吉因數次勸諫而得以免死,受髡刑,發配築城。蘇軾《霍光疏昌邑王之罪》析此事,「其中從官,必有謀光者,光知之,故立、廢賀,非專以淫亂故也。二百人方誅,號呼於市,曰:『當斷不斷,反受其亂。』此其有謀明矣。特其事秘密,無緣得之。著此者,亦欲後人微見其意也。」

据《漢書·霍光金日磾傳》載霍光所述的劉賀罪行:“受璽以來二十七日,使者旁午,持節詔諸官署徵發,凡千一百二十七事。”刘贺在即位27天内,就做了1,127件荒唐事情,平均一天40件。霍光以其不堪重任,與大臣奏请上官太后(霍光外孫女)下诏,于同月癸巳(公历8月14日)废黜了他,并且亲自送他回到封地昌邑,削去王号,给他食邑二千户,另赐刘贺的四个姐妹汤沐邑千户。同年9月10日,霍光尊立戾太子唯一的遺孫劉病已為帝。

元康二年,霍光寫信給山阳太守张敞:“谨备盗贼,察往来宾客。毋下所赐书。”要求当地官员密切监视刘贺。前63年汉宣帝封刘贺為海昏侯。

刘贺的儿子刘充国、刘奉亲都在袭封海昏侯之前死去,海昏侯国一度绝封,汉元帝初元三年(前46年)才复封刘贺另一个儿子刘代宗为海昏侯。

\subsection{宣帝生平}

漢宣帝劉詢(前91年-前48年1月10日),原名刘病已,字次卿,即位九年後改名询,西汉第十位皇帝(前74年9月10日—前48年1月10日在位),其正式諡號為「孝宣皇帝」,後世省略「孝」字稱「漢宣帝」。汉武帝的曾孙,戾太子刘据的长孙,史皇孙刘进的长子,生母為王翁須。

汉宣帝是汉武帝的曾孙,祖父为衞太子刘据。祖母史良娣,汉武帝元鼎四年(前113年)入为衞太子的良娣,生刘进,号史皇孙。史皇孙刘进于汉武帝太始年间(前96年—前93年)娶王翁须,生刘病已,时号称“皇曾孙”。

汉武帝征和二年(前91年),“巫蛊之祸”爆发,刘病已曾祖母卫子夫、祖父衞太子刘据、祖母史良娣、父亲史皇孙刘进、母亲王翁須均因此被杀,刚刚出生数月的刘病已也被投入大牢。由于他还是个婴儿,廷尉监丙吉在狱中挑选两位女囚趙徵卿與胡組做他的乳母,暂时免除二人刑罚。

巫蛊之狱连年不决,汉武帝后元二年(前87年),因为有人说长安狱中有天子气,武帝命令处死長安所有監獄的犯人,使者先到中都官詔獄處決犯人,後連夜趕到丙吉掌管的郡邸獄,丙吉据门不纳使者,保住了刘病已的性命。第二天使者回宮報告,武帝感慨,以為天意,就撤销了这道命令,並大赦,四歲的刘病已遇赦出狱。

出狱后的刘病已被丙吉送至祖母史良娣的娘家。史家怜其孤苦,对其照顾甚厚。不久刘病已恢复宗室身份,诏养于掖庭。是任掖庭令的张贺是衞太子刘据的故吏,哀衞太子无辜受难和皇曾孙的孤弱,对其抚养甚厚。及长,张贺教其诗书为之启蒙,后自费延请名儒東海澓中翁教授刘病已。刘病已聪颖好学,不久即通晓儒家经典。与此同时,刘病已亦喜好游侠、喜好斗鸡走马,游侠于三辅一带,结识了戴长乐等。这些民间经历都成为他日后当皇帝积累了重要的经验。

前76年,刘病已到了成家娶亲的年龄,掖庭令张贺有一孙女与刘病已年龄相仿,因此打算把她嫁与刘病已为妻。但是却遭到为人谨慎的弟弟张安世的强烈反对,他说:“曾孙乃衞太子后也,幸得平民衣食县官,足矣,勿复言予女事!”张贺不敢违逆弟弟的意思,只好为刘病已另聘属下许广汉的女儿许平君为妻。刘病已与许平君婚后感情很好,不久生下了儿子刘奭,也就是后来的汉元帝。

汉昭帝元平元年(前74年),汉昭帝驾崩,由于无嗣,大司马霍光拥立的昌邑王刘贺为帝。但是刘贺在即位的27天就被权臣霍光提请其外孙女上官皇太后废掉。在确立继任人选时,时任光祿大夫的丙吉此时向霍光推荐刘病已,元平元年秋七月庚申(前74年9月10日),刘病已入宫见上官太后,被封为阳武侯,同日登基为皇帝,承嗣汉昭帝,隔年改元本始。

宣帝由于是霍光所立,他吸取昌邑王被废的教训,初即位政事一决于光。唯立后问题上坚持己见,他与发妻许平君感情深厚,当上皇帝后许平君并没有立即被立为皇后,而是仅封为婕妤。朝臣和上官皇太后都认为应立霍光的小女儿霍成君为皇后。于是汉宣帝“诏求微时故剑”,群臣见宣帝意思坚决,于是议决立许平君为皇后。

霍光的夫人显对女儿没能当上皇后非常恼怒。本始三年时值皇后许平君有孕,霍光的夫人于是勾结女医生淳于衍将其暗杀。霍光知道后非常惊愕,但是他没有去追究自己的妻子罪行,而是利用自己的权势授意宣帝不追查此事。次年,霍成君如愿以偿成为皇后。汉宣帝对结发之妻的去世非常悲伤,这也影响了他后来对继任人的选择。后来他渐渐对时为太子的汉元帝感到不满意,并下了“乱吾家者,太子也”的评语,但始终没有废汉元帝的太子之位。

霍光属于汉武帝时的衞氏外戚集团。霍光十五岁,其兄(同父异母)霍去病回到家乡认祖归宗,把他带到长安,并因兄长的关系出任郎官,开始了漫长的仕宦生涯。

汉武帝末年巫蛊之祸,衞氏外戚遭到了毁灭性的打击,皇后卫子夫、大司馬卫青的子嗣以及衞太子一族全被族誅,但是霍氏躲过了此难。之后汉武帝渐渐明白过来,于是霍光开始受到重用。汉武帝临死前,任命霍光、金日磾、上官桀三人为辅政大臣,并以霍光为首,加封其为大司马。但是不久金日磾去世,霍光以和親拉攏上官桀,昭帝封光為丞相,开始独揽大权。

地节二年(前68年)霍光病逝,宣帝下令以帝王的规格下葬霍光,同时亦开始亲政。面对霍氏宗族的专权,汉宣帝不动声色对其予以翦除。他先是迁霍光的女婿大将军范明友为光禄勋,羽林监任胜为安定郡太守,几个月后又把霍光的姐夫张朔由給事中光祿大夫改为蜀郡太守,孙婿王汉为武威郡太守,长乐宫卫尉邓广汉为少府,这样夺取了他们的军权,扫清了霍家的外围势力。接着开始对霍家动手,改霍禹为大司马,无印绶,也就是剥夺了兵权,霍光的另一女婿赵平的兵权也被夺,空下来的职位完全由汉宣帝的外戚史、许两家子弟充任。

霍光是权力斗争的高手,但是他的儿孙却都很无能。霍光的儿子霍禹面对这种情况毫无应对之策,只是整日与霍山、霍云等哭泣。不久霍光夫人显毒杀许平君的事情开始败露。地节四年七月,大司马霍禹谋反事发,汉宣帝下令诛杀冠阳侯霍云、乐平侯霍山(两人皆为霍去病之孙)諸姊妹壻度遼將軍范明友、長信少府鄧廣漢、中郎將任勝、騎都尉趙平、長安男子馮殷等。与此同时,霍光之女霍皇后被废,于十二年后被迫自杀。

汉宣帝尚为平民之时,就对霍氏的权势有很深的了解。霍光挟专权之势,行伊尹废立天子之事,更是让汉宣帝胆颤心惊。在汉宣帝即位之初,汉宣帝拜谒高庙,霍光为骖乘(也就是駕驶车马),汉宣帝对其深为忌惮,在车上犹如芒刺在背;但是当驃骑将军张安世为骖乘时,汉宣帝体貌从容,一点不感到紧张。所以民间传说为:“威震主者不畜,霍氏之禍萌於驂乘。”

霍氏一门虽然被诛,但是汉宣帝仍然十分感念霍光的功勋,在麒麟阁十一功臣中,霍光名列第一,称“大司马大将军博陆侯,姓霍氏”,仅称官职和爵位而不道其名,以示尊重。后来又封霍光堂兄弟的后裔为博陆侯,以续霍光的祭祠。

宣帝虽然诛除霍氏一族,但是并没有废除霍光之政。他通过诏书正式肯定霍光的功绩,并且继续霍光的政策。他继续推行轻徭薄赋与民休息的政策,把皇家掌控的园囿和公田分给平民耕种,并贷给他们种子。后来又在元康元年(前65年)、元康二年(前64年)、神爵元年(前61年)和五凤四年(前54年)下令勾销百姓所贷官府的种子,如果受灾则免除他们的赋税。还设立常平仓,平抑物价,保证物价的稳定。此外汉宣帝还减少人口税(即算赋)。

汉宣帝曾生长于民间,为平民时喜欢游侠,足迹遍于三辅,因此深知吏治的重要性。他五日一听事,对官吏观其言,察其行,考试功能。他要求官吏尽职,地节三年(公元前67年)下诏说:“二千石严教吏谨视遇,毋令失职。”要求郡国长官管教和督促地方官吏,不能让他们失职。

他强调决狱宜平,特设廷平官。曾下诏说:“间者吏用法,巧文寖深,是朕之不德也。夫决狱不当,使有罪兴邪,不辜蒙戮,父子悲恨,朕甚伤之。今遣廷史与郡鞠狱,任轻禄薄,其为置廷平,秩六百石,员四人。其务平之,以称朕意。”他要求官吏奉法,元康二年(公元前64年)下诏说,“吏务平法。或擅兴徭役,饰厨传,称过使客,越职逾法,以取名誉,譬犹践薄冰以待白日,岂不殆哉!”他审察吏治,元康四年派遣大中大夫强等十二人循行天下,主要任务是“察吏治得失”;五凤四年(公元前54年)又派遣丞相、御史掾二十四人循行天下,“举冤狱,察擅为苛禁深刻者”。

反对苛政,下诏批评说:“今郡国二千石或擅为苛禁,禁民嫁娶不得具酒食相贺召”,即反对地方长官干涉民间喜庆之事。他反对欺谩,黄龙元年(公元前49年)诏责当时“上计簿,具文而已,务为欺谩,以避其课”,指令“御史察计簿,疑非实者,按之,使真伪无相乱”。

根据吏治情况,奖功罚罪。奖赏有功者,如:地节三年(公元前67年)对安抚流民有功的胶东相王成,下诏奖励,定秩中二千石,赐爵关内侯。神爵四年(公元前58 年)对治行优异的颍川太守黄霸,定秩中二千石,赐爵关内侯,黄金百斤,同时对颍川吏民也有赏赐。王成与黄霸,原秩二千石,一年得一千四百四十石,升秩中二千石,一年得二千一百六十石,增加秩俸百分之五十。责罚罪过者,如:元康二年(公元前64年)冬,本来精明能干、治理有绩的京兆尹赵广汉,因执法出了偏差,“坐贼杀不辜,鞠狱故不以实,擅斥除骑士乏军兴数罪”,而被腰斩。神爵四年(公元前58年)十一月,号称“屠伯”的河南太守严延年因酷急和诽谤之罪,弃市。

故史称宣帝之治“信赏必罚,综核名实”、“吏称其职,民安其业”。

与汉武帝劳民伤财式的连番对匈奴发动战争的方式不同,汉宣帝对匈奴的战争采用了更多的持巧,军事、政治、经济多管齐下。宣帝即位之初,汉与烏孫为了反抗匈奴侵扰,相约分头出兵击匈奴,匈奴无力抵抗而逃,损失很重。后来匈奴又遭乌孙、乌桓、丁零等族袭击,加之大雪成灾,力量大大削弱,故欲与汉和亲。于是汉边境“少事”。宣帝亲政时,正是匈奴内乱外患之日,无力侵扰汉境。为了减少对匈奴边防驻军的压力,他下令减少军屯。罢车骑将军、右将军屯兵。

匈奴内乱,出现了五个单于,各派多争取与汉和亲,或来投靠汉朝。汉为了自身的安宁,也积极应付。神爵三年(公元前59 年),匈奴日逐王先贤掸率众来降,汉封其为归德靖侯。五凤二年(公元前56 年),匈奴左大将军王定来降,封其为信成侯。同年,匈奴呼遬累单于来降,汉也封其为列侯。五凤三年(公元前55 年)三月,宣帝诏中提到:“(匈奴)诸王并自立,分为五单于,更相攻击,死者以万数,畜产大耗什八九,人民饥饿,相燔烧以求食,因大乖乱。单于阏氏子孙昆弟及呼遬累单于、名王、右伊秩訾、且渠、当户以下将众五万余人来降归义。单于称臣,使弟奉珍朝贺。正月,北边晏然,靡有兵革之事。”

汉朝此时设置西河、北地属国,以安置匈奴来降者。次年,匈奴单于向汉称臣,派遣其弟谷蠡王入侍。汉朝因边塞无寇,减戍卒十分之二。甘露元年(公元前53 年),匈奴呼韓邪單于派遣其子右贤王铢娄渠堂入侍汉廷;郅支单于也派遣其子右大将驹于利受入侍于汉。甘露二年(公元前52 年),呼韩邪单于叩五原塞,表示愿奉国珍三年正月来朝,宣帝同意,并安排接待。次年正月,呼韩邪来汉朝贺,受到盛情接待,并得到很多赏赐。这年郅支单于也遣使来汉奉献。甘露四年,呼韩邪单于、郅支单于都遣使朝献于汉,汉朝款待呼韩邪单于的使者格外有礼。黄龙元年(公元前49 年)正月,呼韩邪单于又来朝,汉朝对他礼赐如初。

宣帝初年,西羌先零部落擅自北渡湟水,侵占汉民地区。元康三年(公元前63年),西羌先零部落与各部落的酋长二百多人集会,“解仇交质”,订立盟约,打算共同侵扰汉地。宣帝闻知,问赵充国如何对策。赵充国以为,羌人各部盟约,还可能联合其他各部,应当及早准备。他建议一方面命令边兵加强战备,监视诸羌;一方面要破坏诸羌联合,探听其预谋内情。于是派遣义渠安国出使诸羌,了解其动向。

义渠前去,召集诸羌首领,杀了逆而不顺者,又调兵杀了先零羌民一千余人。西羌各部震恐,起而反抗,犯汉边塞,攻城邑,杀长吏。神爵元年(公元前61 年)春,义渠所部三千骑兵被羌人袭击,退到令居,向皇帝报告情况。宣帝当即调发兵马前往金城。以后将军赵充国、强弩将军许延寿带兵前往;又任酒泉太守辛武贤为破羌将军,与两将军并进。

赵充国到了金城,以哨兵了解敌情,派间谍宣传政策,日飨军士而不进击。西羌人见汉军坚壁固守,无法进攻,互相埋怨,发生了矛盾。辛武贤以为进军时机已到,向皇帝上书建议进兵。赵充国以为,辛武贤的建议不妥,如果冒险进兵,必然进退两难。他一再上书建议只能先击主谋者先零部落,逼其悔过而赦之,再选择良吏前去抚慰羌众。宣帝要他作详细说明。赵充国反复论说,马上进击失十二利,留兵屯田有十二便。宣帝肯定了赵充国屯田之策,于是诏令罢兵,让赵充国负责屯田。到了神爵二年(公元前60 年),羌民斩了先零大豪杨玉、犹非之首,向汉投顺,汉朝设金城属国以安置投顺的羌民。羌乱至此告一段落。

汉自張騫在前138年—前126年和前119年两次出使中亚(大宛、康居、大夏、烏孫、阿尔沙克王朝、身毒),和前104—前102年李广利两次伐大宛获胜之后,于前102年在西域的天山山脉南麓乌堡设置校尉,屯田于渠犁,将塔里木盆地的26个印欧人的城邦国置于西汉的管制之下。地节二年(公元前68年),宣帝派遣侍郎郑吉到渠犁负责屯田。郑吉通过屯田积蓄了粮食,发兵打败了车师。宣帝诏令郑吉继续在渠犁与车师屯田积粮,以管制西域,对付匈奴。匈奴得知消息,前来争夺车师之地。郑吉固守力弱,要求增援。宣帝诏令长罗侯常惠带领张掖郡、酒泉郡的骑兵前往车师北边千余里,显示汉军威武,吓得匈奴骑兵退去。车师王因得到汉军保护而不受匈奴欺压,乐于“亲汉”。稍后,郑吉又迎匈奴日逐王来汉投降。宣帝先命郑吉负责衞护鄯善西南方(南道)各国的安全,继又命其兼护车师西北方(北道)各国的安全,所以号称“都护”。宣帝还封郑吉为安远侯,这是神爵三年(公元前59 年)之事。

西域都护的幕府,设置在烏壘城(在今新疆库尔勒与轮台之间),负责处理西域三十六国事务,同时主管屯田事业。汉朝的西域都护取代了匈奴在西域的僮仆都尉,反映了汉匈政治力量在西域的消长,所以史称:“汉之号令班于西域矣,始于张骞而成于郑吉。”

由于宣帝长期在民间生活,深知民间疾苦,所以他在位时期,勤俭治国,而且还很放松人民的思想,对大臣要求严格,特别是宣帝亲政以后,汉朝的政治更加清明,社会经济更加繁荣。在亲政的二十年中,他着重于整肃吏治,加强皇权。他不但族灭了腐败的霍氏家族,而且诛杀了一些地位很高的、腐朽贪污的官员。为维护法律正常行使,宣帝设置治御史以审核廷尉量刑轻重;设廷尉平至地方鞠狱,规定郡国呈报狱囚被笞瘐死名数,重视民命之余又加强中央对地方的控制。此外宣帝又召集著名儒生在未央宫讲论五经异同,目的是为了巩固皇权、统一思想。其余如废除一些苛法,屡次蠲免田租、算赋,招抚流亡,在发展农业生产方面继续霍光的政策。对周边異族的关系,则软硬皆施。神爵元年(前60年),先零部(属西羌)与诸羌联盟并和匈奴借兵,企图对汉复仇。宣帝派后将军赵充国、弩将军許延壽出金城攻击西羌,均获胜利,留赵充国屯田。神爵二年五月(前59年),西羌杀其首领杨玉、犹非等,遂降漢。宣帝设金城属国,撤回屯田军。袭破车师。时匈奴发生内乱,呼韩邪单于于甘露三年(前51年)亲至五原郡塞上请求入朝称臣,成了汉朝的藩属,宣帝又得以完成武帝倾国之力而未完成的事业。

漢宣帝在位期间,「吏称其职,民安其业」,号称「中兴」,应该说,宣帝统治时期是西漢武力最强盛、经济最繁荣的时候,因此史书对宣帝大为赞赏,曰:“孝宣之治,信赏必罚,文治武功,可谓中兴”,算是西漢、甚至是中國歷史上,少有的中興之主。他与前任汉昭帝刘弗陵的统治并称为昭宣之治。

民国史学家吕思勉:「宣帝是个『旧劳于外』的人,颇知道民生疾苦,极其留意吏治,武帝和霍光时,用法都极严。宣帝却留意于平恕,也算西汉一个贤君。」

黄龙元年十二月甲戌日(前48年1月10日),汉宣帝去世,在位25年,享年43岁。谥号孝宣皇帝,东汉建武年間上庙号中宗。初元元年正月辛丑(前48年2月6日),葬于今天西安市东郊的杜陵。

宣帝是中國歷史上唯一一位在即位前受过牢狱之苦的大一统皇朝皇帝。宣帝改名“询”的理由是“病”、“已”两字太过常用,臣民避讳不易;或許也認為這二字有些不吉。宣帝与许皇后和霍皇后的感情纠葛是越剧《漢宮怨》的主题。其太子刘奭十分相信儒家學說,對宣帝某些專制行為不滿;宣帝在训斥其时說:“汉家自有制度,本以霸王道杂之,奈何纯用德教,用周政乎?”宣帝是西汉四位拥有庙号的皇帝之一,廟號為「中宗」,其餘三位為漢高帝(太祖)、漢文帝(太宗)、漢武帝(世宗)。

\subsection{本始}

\begin{longtable}{|>{\centering\scriptsize}m{2em}|>{\centering\scriptsize}m{1.3em}|>{\centering}m{8.8em}|}
  % \caption{秦王政}\
  \toprule
  \SimHei \normalsize 年数 & \SimHei \scriptsize 公元 & \SimHei 大事件 \tabularnewline
  % \midrule
  \endfirsthead
  \toprule
  \SimHei \normalsize 年数 & \SimHei \scriptsize 公元 & \SimHei 大事件 \tabularnewline
  \midrule
  \endhead
  \midrule
  元年 & -73 & \tabularnewline\hline
  二年 & -72 & \tabularnewline\hline
  三年 & -71 & \tabularnewline\hline
  四年 & -70 & \tabularnewline
  \bottomrule
\end{longtable}


\subsection{地节}

\begin{longtable}{|>{\centering\scriptsize}m{2em}|>{\centering\scriptsize}m{1.3em}|>{\centering}m{8.8em}|}
  % \caption{秦王政}\
  \toprule
  \SimHei \normalsize 年数 & \SimHei \scriptsize 公元 & \SimHei 大事件 \tabularnewline
  % \midrule
  \endfirsthead
  \toprule
  \SimHei \normalsize 年数 & \SimHei \scriptsize 公元 & \SimHei 大事件 \tabularnewline
  \midrule
  \endhead
  \midrule
  元年 & -69 & \tabularnewline\hline
  二年 & -68 & \tabularnewline\hline
  三年 & -67 & \tabularnewline\hline
  四年 & -66 & \tabularnewline
  \bottomrule
\end{longtable}


\subsection{元康}

\begin{longtable}{|>{\centering\scriptsize}m{2em}|>{\centering\scriptsize}m{1.3em}|>{\centering}m{8.8em}|}
  % \caption{秦王政}\
  \toprule
  \SimHei \normalsize 年数 & \SimHei \scriptsize 公元 & \SimHei 大事件 \tabularnewline
  % \midrule
  \endfirsthead
  \toprule
  \SimHei \normalsize 年数 & \SimHei \scriptsize 公元 & \SimHei 大事件 \tabularnewline
  \midrule
  \endhead
  \midrule
  元年 & -65 & \tabularnewline\hline
  二年 & -64 & \tabularnewline\hline
  三年 & -63 & \tabularnewline\hline
  四年 & -62 & \tabularnewline
  \bottomrule
\end{longtable}

\subsection{神爵}

\begin{longtable}{|>{\centering\scriptsize}m{2em}|>{\centering\scriptsize}m{1.3em}|>{\centering}m{8.8em}|}
  % \caption{秦王政}\
  \toprule
  \SimHei \normalsize 年数 & \SimHei \scriptsize 公元 & \SimHei 大事件 \tabularnewline
  % \midrule
  \endfirsthead
  \toprule
  \SimHei \normalsize 年数 & \SimHei \scriptsize 公元 & \SimHei 大事件 \tabularnewline
  \midrule
  \endhead
  \midrule
  元年 & -61 & \tabularnewline\hline
  二年 & -60 & \tabularnewline\hline
  三年 & -59 & \tabularnewline\hline
  四年 & -58 & \tabularnewline
  \bottomrule
\end{longtable}

\subsection{五凤}

\begin{longtable}{|>{\centering\scriptsize}m{2em}|>{\centering\scriptsize}m{1.3em}|>{\centering}m{8.8em}|}
  % \caption{秦王政}\
  \toprule
  \SimHei \normalsize 年数 & \SimHei \scriptsize 公元 & \SimHei 大事件 \tabularnewline
  % \midrule
  \endfirsthead
  \toprule
  \SimHei \normalsize 年数 & \SimHei \scriptsize 公元 & \SimHei 大事件 \tabularnewline
  \midrule
  \endhead
  \midrule
  元年 & -57 & \tabularnewline\hline
  二年 & -56 & \tabularnewline\hline
  三年 & -55 & \tabularnewline\hline
  四年 & -54 & \tabularnewline
  \bottomrule
\end{longtable}

\subsection{甘露}

\begin{longtable}{|>{\centering\scriptsize}m{2em}|>{\centering\scriptsize}m{1.3em}|>{\centering}m{8.8em}|}
  % \caption{秦王政}\
  \toprule
  \SimHei \normalsize 年数 & \SimHei \scriptsize 公元 & \SimHei 大事件 \tabularnewline
  % \midrule
  \endfirsthead
  \toprule
  \SimHei \normalsize 年数 & \SimHei \scriptsize 公元 & \SimHei 大事件 \tabularnewline
  \midrule
  \endhead
  \midrule
  元年 & -53 & \tabularnewline\hline
  二年 & -52 & \tabularnewline\hline
  三年 & -51 & \tabularnewline\hline
  四年 & -50 & \tabularnewline
  \bottomrule
\end{longtable}


\subsection{黄龙}

\begin{longtable}{|>{\centering\scriptsize}m{2em}|>{\centering\scriptsize}m{1.3em}|>{\centering}m{8.8em}|}
  % \caption{秦王政}\
  \toprule
  \SimHei \normalsize 年数 & \SimHei \scriptsize 公元 & \SimHei 大事件 \tabularnewline
  % \midrule
  \endfirsthead
  \toprule
  \SimHei \normalsize 年数 & \SimHei \scriptsize 公元 & \SimHei 大事件 \tabularnewline
  \midrule
  \endhead
  \midrule
  元年 & -49 & \tabularnewline
  \bottomrule
\end{longtable}


%%% Local Variables:
%%% mode: latex
%%% TeX-engine: xetex
%%% TeX-master: "../Main"
%%% End:

%% -*- coding: utf-8 -*-
%% Time-stamp: <Chen Wang: 2019-12-16 17:11:12>

\section{元帝\tiny(BC48-BC33)}

\subsection{生平}

汉元帝劉\xpinyin*{奭}(前75年-前33年7月8日),西汉第十一位皇帝,其正式諡號為「孝元皇帝」,後世省略「孝」字稱「漢元帝」。汉宣帝长子,生于民间,母恭哀皇后许平君。宣帝死后继位,在位16年即前49年-前33年。

前74年,父亲刘询登基,当时他年仅两岁。本始三年(前71年),母亲皇后许平君被霍成君的母亲显毒死。地節三年(前67年)八歲的刘奭被宣帝立为太子。继母霍成君则试图毒死他,但未能成功。因为他曾经向父亲宣帝进言“持刑太深,宜用儒生”,而不被宣帝所喜爱。宣帝甚至预言“乱我家者,必太子也”,但顾念他是发妻许平君的儿子而没有褫夺他的太子之位。

宣帝病死后继位,第二年(前48年)改年号为“初元”,在位时期“崇尚儒术”,多次出兵击溃匈奴。建昭三年(前36年),汉将甘延寿、陈汤诛郅支单于于康居(郅支之战)。至此,唯一反汉的匈奴单于被消灭了。汉匈百年大战于此告一段落。竟宁元年(前33年),匈奴呼韩邪单于入朝求亲。刘奭以宫女王嫱(王昭君)嫁之为妻。

此时的汉朝比较强盛,但也是衰落的起点。刘奭在位期间,純任德教,在儒臣的要求下以不符合儒家心目中的古礼为由废除郡国庙,废除迁关东豪强至关中帝陵置邑制度,豪强大地主兼并之风盛行,社会危机日益加深。又由于汉元帝过于放纵外戚、宦官,最终汉元帝皇后王政君侄子王莽代汉称帝。

竟宁元年五月壬辰(前33年7月8日),病死于长安未央宫,终年四十二岁。七月丙戌(8月31日),葬于渭陵(今陕西咸阳市东北12里处)。死后庙号高宗(后被取消),谥号孝元皇帝。次年,长子刘骜登基,是为汉成帝。

班固《漢書·卷九·元帝紀第九》:“壯大,柔仁好儒”。 “臣外祖兄弟為元帝侍中,語臣曰:元帝多材藝,善史書。鼓琴瑟,吹洞簫,自度曲,被歌聲,分刌節度,窮極幼眇。少而好儒,及即位,徵用儒生,委之以政,貢、薛、韋、匡迭為宰相。而上牽製文義,優游不斷,孝宣之業衰焉。然寬弘盡下,出於恭儉,號令溫雅,有古之風烈。”

\subsection{初元}

\begin{longtable}{|>{\centering\scriptsize}m{2em}|>{\centering\scriptsize}m{1.3em}|>{\centering}m{8.8em}|}
  % \caption{秦王政}\
  \toprule
  \SimHei \normalsize 年数 & \SimHei \scriptsize 公元 & \SimHei 大事件 \tabularnewline
  % \midrule
  \endfirsthead
  \toprule
  \SimHei \normalsize 年数 & \SimHei \scriptsize 公元 & \SimHei 大事件 \tabularnewline
  \midrule
  \endhead
  \midrule
  元年 & -48 & \tabularnewline\hline
  二年 & -47 & \tabularnewline\hline
  三年 & -46 & \tabularnewline\hline
  四年 & -45 & \tabularnewline\hline
  五年 & -44 & \tabularnewline
  \bottomrule
\end{longtable}


\subsection{永光}

\begin{longtable}{|>{\centering\scriptsize}m{2em}|>{\centering\scriptsize}m{1.3em}|>{\centering}m{8.8em}|}
  % \caption{秦王政}\
  \toprule
  \SimHei \normalsize 年数 & \SimHei \scriptsize 公元 & \SimHei 大事件 \tabularnewline
  % \midrule
  \endfirsthead
  \toprule
  \SimHei \normalsize 年数 & \SimHei \scriptsize 公元 & \SimHei 大事件 \tabularnewline
  \midrule
  \endhead
  \midrule
  元年 & -43 & \tabularnewline\hline
  二年 & -42 & \tabularnewline\hline
  三年 & -41 & \tabularnewline\hline
  四年 & -40 & \tabularnewline\hline
  五年 & -39 & \tabularnewline
  \bottomrule
\end{longtable}


\subsection{建昭}

\begin{longtable}{|>{\centering\scriptsize}m{2em}|>{\centering\scriptsize}m{1.3em}|>{\centering}m{8.8em}|}
  % \caption{秦王政}\
  \toprule
  \SimHei \normalsize 年数 & \SimHei \scriptsize 公元 & \SimHei 大事件 \tabularnewline
  % \midrule
  \endfirsthead
  \toprule
  \SimHei \normalsize 年数 & \SimHei \scriptsize 公元 & \SimHei 大事件 \tabularnewline
  \midrule
  \endhead
  \midrule
  元年 & -38 & \tabularnewline\hline
  二年 & -37 & \tabularnewline\hline
  三年 & -36 & \tabularnewline\hline
  四年 & -35 & \tabularnewline\hline
  五年 & -34 & \tabularnewline
  \bottomrule
\end{longtable}

\subsection{竟宁}

\begin{longtable}{|>{\centering\scriptsize}m{2em}|>{\centering\scriptsize}m{1.3em}|>{\centering}m{8.8em}|}
  % \caption{秦王政}\
  \toprule
  \SimHei \normalsize 年数 & \SimHei \scriptsize 公元 & \SimHei 大事件 \tabularnewline
  % \midrule
  \endfirsthead
  \toprule
  \SimHei \normalsize 年数 & \SimHei \scriptsize 公元 & \SimHei 大事件 \tabularnewline
  \midrule
  \endhead
  \midrule
  元年 & -33 & \tabularnewline
  \bottomrule
\end{longtable}


%%% Local Variables:
%%% mode: latex
%%% TeX-engine: xetex
%%% TeX-master: "../Main"
%%% End:

%% -*- coding: utf-8 -*-
%% Time-stamp: <Chen Wang: 2018-07-10 19:14:14>

\section{成帝\tiny(BC33-BC7)}

\subsection{建始}

\begin{longtable}{|>{\centering\scriptsize}m{2em}|>{\centering\scriptsize}m{1.3em}|>{\centering}m{8.8em}|}
  % \caption{秦王政}\
  \toprule
  \SimHei \normalsize 年数 & \SimHei \scriptsize 公元 & \SimHei 大事件 \tabularnewline
  % \midrule
  \endfirsthead
  \toprule
  \SimHei \normalsize 年数 & \SimHei \scriptsize 公元 & \SimHei 大事件 \tabularnewline
  \midrule
  \endhead
  \midrule
  元年 & -32 & \tabularnewline\hline
  二年 & -31 & \tabularnewline\hline
  三年 & -30 & \tabularnewline\hline
  四年 & -29 & \tabularnewline
  \bottomrule
\end{longtable}


\subsection{河平}

\begin{longtable}{|>{\centering\scriptsize}m{2em}|>{\centering\scriptsize}m{1.3em}|>{\centering}m{8.8em}|}
  % \caption{秦王政}\
  \toprule
  \SimHei \normalsize 年数 & \SimHei \scriptsize 公元 & \SimHei 大事件 \tabularnewline
  % \midrule
  \endfirsthead
  \toprule
  \SimHei \normalsize 年数 & \SimHei \scriptsize 公元 & \SimHei 大事件 \tabularnewline
  \midrule
  \endhead
  \midrule
  元年 & -28 & \tabularnewline\hline
  二年 & -27 & \tabularnewline\hline
  三年 & -26 & \tabularnewline\hline
  四年 & -25 & \tabularnewline
  \bottomrule
\end{longtable}


\subsection{阳朔}

\begin{longtable}{|>{\centering\scriptsize}m{2em}|>{\centering\scriptsize}m{1.3em}|>{\centering}m{8.8em}|}
  % \caption{秦王政}\
  \toprule
  \SimHei \normalsize 年数 & \SimHei \scriptsize 公元 & \SimHei 大事件 \tabularnewline
  % \midrule
  \endfirsthead
  \toprule
  \SimHei \normalsize 年数 & \SimHei \scriptsize 公元 & \SimHei 大事件 \tabularnewline
  \midrule
  \endhead
  \midrule
  元年 & -24 & \tabularnewline\hline
  二年 & -23 & \tabularnewline\hline
  三年 & -22 & \tabularnewline\hline
  四年 & -21 & \tabularnewline
  \bottomrule
\end{longtable}


\subsection{鸿嘉}

\begin{longtable}{|>{\centering\scriptsize}m{2em}|>{\centering\scriptsize}m{1.3em}|>{\centering}m{8.8em}|}
  % \caption{秦王政}\
  \toprule
  \SimHei \normalsize 年数 & \SimHei \scriptsize 公元 & \SimHei 大事件 \tabularnewline
  % \midrule
  \endfirsthead
  \toprule
  \SimHei \normalsize 年数 & \SimHei \scriptsize 公元 & \SimHei 大事件 \tabularnewline
  \midrule
  \endhead
  \midrule
  元年 & -20 & \tabularnewline\hline
  二年 & -19 & \tabularnewline\hline
  三年 & -18 & \tabularnewline\hline
  四年 & -17 & \tabularnewline
  \bottomrule
\end{longtable}


\subsection{永始}

\begin{longtable}{|>{\centering\scriptsize}m{2em}|>{\centering\scriptsize}m{1.3em}|>{\centering}m{8.8em}|}
  % \caption{秦王政}\
  \toprule
  \SimHei \normalsize 年数 & \SimHei \scriptsize 公元 & \SimHei 大事件 \tabularnewline
  % \midrule
  \endfirsthead
  \toprule
  \SimHei \normalsize 年数 & \SimHei \scriptsize 公元 & \SimHei 大事件 \tabularnewline
  \midrule
  \endhead
  \midrule
  元年 & -16 & \tabularnewline\hline
  二年 & -15 & \tabularnewline\hline
  三年 & -14 & \tabularnewline\hline
  四年 & -13 & \tabularnewline
  \bottomrule
\end{longtable}


\subsection{元诞}

\begin{longtable}{|>{\centering\scriptsize}m{2em}|>{\centering\scriptsize}m{1.3em}|>{\centering}m{8.8em}|}
  % \caption{秦王政}\
  \toprule
  \SimHei \normalsize 年数 & \SimHei \scriptsize 公元 & \SimHei 大事件 \tabularnewline
  % \midrule
  \endfirsthead
  \toprule
  \SimHei \normalsize 年数 & \SimHei \scriptsize 公元 & \SimHei 大事件 \tabularnewline
  \midrule
  \endhead
  \midrule
  元年 & -12 & \tabularnewline\hline
  二年 & -11 & \tabularnewline\hline
  三年 & -10 & \tabularnewline\hline
  四年 & -9 & \tabularnewline
  \bottomrule
\end{longtable}

\subsection{绥和}

\begin{longtable}{|>{\centering\scriptsize}m{2em}|>{\centering\scriptsize}m{1.3em}|>{\centering}m{8.8em}|}
  % \caption{秦王政}\
  \toprule
  \SimHei \normalsize 年数 & \SimHei \scriptsize 公元 & \SimHei 大事件 \tabularnewline
  % \midrule
  \endfirsthead
  \toprule
  \SimHei \normalsize 年数 & \SimHei \scriptsize 公元 & \SimHei 大事件 \tabularnewline
  \midrule
  \endhead
  \midrule
  元年 & -8 & \tabularnewline\hline
  二年 & -7 & \tabularnewline
  \bottomrule
\end{longtable}


%%% Local Variables:
%%% mode: latex
%%% TeX-engine: xetex
%%% TeX-master: "../Main"
%%% End:

%% -*- coding: utf-8 -*-
%% Time-stamp: <Chen Wang: 2018-07-10 19:18:50>

\section{哀帝\tiny(BC7-BC1)}

\subsection{建平}

\begin{longtable}{|>{\centering\scriptsize}m{2em}|>{\centering\scriptsize}m{1.3em}|>{\centering}m{8.8em}|}
  % \caption{秦王政}\
  \toprule
  \SimHei \normalsize 年数 & \SimHei \scriptsize 公元 & \SimHei 大事件 \tabularnewline
  % \midrule
  \endfirsthead
  \toprule
  \SimHei \normalsize 年数 & \SimHei \scriptsize 公元 & \SimHei 大事件 \tabularnewline
  \midrule
  \endhead
  \midrule
  元年 & -6 & \tabularnewline\hline
  二年 & -5 & \tabularnewline\hline
  太初\\元将 & -5 & \tabularnewline\hline
  三年 & -4 & \tabularnewline\hline
  四年 & -3 & \tabularnewline
  \bottomrule
\end{longtable}


\subsection{元寿}

\begin{longtable}{|>{\centering\scriptsize}m{2em}|>{\centering\scriptsize}m{1.3em}|>{\centering}m{8.8em}|}
  % \caption{秦王政}\
  \toprule
  \SimHei \normalsize 年数 & \SimHei \scriptsize 公元 & \SimHei 大事件 \tabularnewline
  % \midrule
  \endfirsthead
  \toprule
  \SimHei \normalsize 年数 & \SimHei \scriptsize 公元 & \SimHei 大事件 \tabularnewline
  \midrule
  \endhead
  \midrule
  元年 & -2 & \tabularnewline\hline
  二年 & -1 & \tabularnewline
  \bottomrule
\end{longtable}


%%% Local Variables:
%%% mode: latex
%%% TeX-engine: xetex
%%% TeX-master: "../Main"
%%% End:

%% -*- coding: utf-8 -*-
%% Time-stamp: <Chen Wang: 2018-07-10 19:28:12>

\section{平帝\tiny(1-5)}

\subsection{元始}

\begin{longtable}{|>{\centering\scriptsize}m{2em}|>{\centering\scriptsize}m{1.3em}|>{\centering}m{8.8em}|}
  % \caption{秦王政}\
  \toprule
  \SimHei \normalsize 年数 & \SimHei \scriptsize 公元 & \SimHei 大事件 \tabularnewline
  % \midrule
  \endfirsthead
  \toprule
  \SimHei \normalsize 年数 & \SimHei \scriptsize 公元 & \SimHei 大事件 \tabularnewline
  \midrule
  \endhead
  \midrule
  元年 & 1 & \tabularnewline\hline
  二年 & 2 & \tabularnewline\hline
  三年 & 3 & \tabularnewline\hline
  四年 & 4 & \tabularnewline\hline
  五年 & 5 & \tabularnewline
  \bottomrule
\end{longtable}


%%% Local Variables:
%%% mode: latex
%%% TeX-engine: xetex
%%% TeX-master: "../Main"
%%% End:

%% -*- coding: utf-8 -*-
%% Time-stamp: <Chen Wang: 2019-12-17 11:54:38>

\section{刘婴\tiny(6-8)}

\subsection{生平}

劉婴(5年-25年2月),西汉末代皇太子(6年4月17日-9年1月10日為皇太子,未正式即位),号孺子,是汉宣帝的玄孙、楚孝王劉囂曾孙、广戚煬侯劉勳孫、广戚侯刘显子。

元始五年(5年),汉朝外戚王莽毒死汉平帝,次年从汉朝宗室中挑选时年2岁的刘婴。但是因為年齡太小,未正式即位,僅為「皇太子」。王莽自称“摄皇帝”,任何排場實與皇帝無異,僅在見孺子及太后時需自稱臣。為了實質控制朝政大权,完全摄政,以周公、伊尹自居,改元“居摄”。而刘婴這個皇太子則其實只是傀儡。

初始元年十一月甲子(9年1月6日),王莽將「攝皇帝」稱號改稱「假皇帝」;十一月戊辰(1月10日),王莽自称汉太祖刘邦要他做皇帝,便称帝,建国号“新”,尊太皇太后王政君为皇太后,刘婴为定安公。至此,立国二百一十一年的西汉王朝结束。

王莽将年幼的刘婴养在高墙府第之中,与外界隔绝任何联系,甚至乳母也不被允许和他讲话,导致刘婴成人后不识六畜,知识面与幼儿无异。王莽将自己的孙女、王宇的女儿王氏嫁给他做妻子。

更始二年(24年),王莽为更始帝刘玄所败,刘婴当时身在长安。平陵人方望等人依据天象认为更始帝必败,而刘婴才是天子正统,便起兵将刘婴迎至临泾,拥立为天子。更始三年(25年),更始帝派遣丞相李松進攻临泾,刘婴被杀。

\subsection{居摄}

\begin{longtable}{|>{\centering\scriptsize}m{2em}|>{\centering\scriptsize}m{1.3em}|>{\centering}m{8.8em}|}
  % \caption{秦王政}\
  \toprule
  \SimHei \normalsize 年数 & \SimHei \scriptsize 公元 & \SimHei 大事件 \tabularnewline
  % \midrule
  \endfirsthead
  \toprule
  \SimHei \normalsize 年数 & \SimHei \scriptsize 公元 & \SimHei 大事件 \tabularnewline
  \midrule
  \endhead
  \midrule
  元年 & 6 & \tabularnewline\hline
  二年 & 7 & \tabularnewline\hline
  三年\\初始 & 8 & \tabularnewline
  \bottomrule
\end{longtable}


%%% Local Variables:
%%% mode: latex
%%% TeX-engine: xetex
%%% TeX-master: "../Main"
%%% End:

%% -*- coding: utf-8 -*-
%% Time-stamp: <Chen Wang: 2019-12-17 11:59:15>

\section{新莽\tiny(9-23)}

\subsection{新朝简介}

新朝(9年-23年),又稱新莽,是中國歷史上繼西漢之後的朝代,為當時權臣王莽所建立,僅王莽一代,建都西安(即原長安)。

西漢末年,人民被豪強欺壓而急需改革,儒者信奉讖緯學說認為將改朝換代,當時漢室外戚王莽博得雅名,獲得人民與儒者的支持,使他以偽造符瑞的方式於9年1月10日篡位稱帝,國號為「新」,西漢結束。

王莽稱帝後進行許多改革,主要有改革官制、改變地名、推行王田制、禁止奴婢買賣、五均六筦(國營事業、所得稅與借貸)及改革幣制等。然而,王莽改制大多遵循《周禮》等古制,沒有明確的解決問題。新制政令繁雜,名稱不斷變動。而且朝令夕改,用人不當,改革最終失敗。17年因為天災不斷,而人民因為改革失敗而經濟破產,最後爆發新末民變,赤眉軍、綠林軍等等民變軍相繼而起。新莽軍相繼在成昌之戰、昆陽之戰慘敗。23年劉玄稱帝,即更始帝。同年绿林军攻入長安,王莽被殺,新朝亡。25年東漢光武帝劉秀脫離更始帝宣布登基。同年赤眉軍攻入長安,不久更始帝被殺。漢光武帝擊潰赤眉軍後,最後於36年一統天下。

新朝開創了中國歷史上透過篡位取得政權的先例。王莽積極推動古制,也使得古文經持續發展。而王莽的失敗代表復古思想的破滅,使得漢儒變法禪讓的政治理論至此消失,漸變帝王萬世一統的思想。東漢班固所寫的《漢書》視王莽為逆臣賊子,以至于新朝一度被称呼为“亡新”,《资治通鉴》直接把新朝归入“汉纪”。而且傳統史觀鄙棄用篡位的方式取得政權。所以後世史學家對王莽的評價皆差。直到清末之後評價才逐漸中立。

西漢自漢宣帝去世後,其政治與社會結構變動劇烈,使西漢走向滅亡。其原因主要有以下四點:元、成、哀、平等幾位皇帝或怠於政事、或軟弱無能,政權先後由宦官石顯與外戚(擔任大司馬或大將軍)王氏、傅氏等集團掌控。地方豪強與商賈再度興起,控制地方吏治與經濟,並且與中央官員密切結合。他們一方面壟斷富利,一方面兼併大量土地,以致大量百姓轉為佃農、流民或奴婢。儒家集團的政治力量超過崇尚務實的法家,最終獨佔朝政,而法家勢力衰退瓦解。最後是儒家提倡改制運動,他們加入陰陽家的五行學說,推演天變災異的現象,形成讖緯學說。並且認為王朝德衰,應該禪國讓位。漢帝孤立無援,地方劉姓諸侯國削弱,中央功臣列侯耗盡,又無能臣幹將扭轉局勢,其政權最終被外戚王莽奪取。

漢成帝繼位後,怠忽政事,喜好女色,最後死在「溫柔鄉」中。在漢成帝怠忽職守期間,國事由皇太后王政君的哥哥大司馬王鳳管理,他們開啟王氏集團執政的開端。王鳳的能力頗強,執政後廣收人才,儒法兩家人才與之合流,一致擁戴聽命;而王家兄弟分別位居要津,奠定王家不可動搖的政治勢力。前22年王鳳去世,其兄弟如王音、王商、王根先後以大司馬一職掌握朝政。形成「王鳳專權,五侯當朝」的局面。此時王家因長期安逸而浮華奢侈,但王鳳之侄王莽節儉樸實,酷好儒術,禮賢下士,漸得王鳳與皇太后王政君的重視,於前16年受封為新都侯,並於前8年擔任大司馬一職。

然而隔年漢成帝去世,其侄劉欣繼位,即漢哀帝。漢哀帝祖母傅太皇太后擅權謀且強勢,其與丁太后、外戚傅喜、丁明把持朝政。傅太皇太后與王莽不合,王莽退位而隱居新野,王氏外戚衰退。漢哀帝本身幼體弱多病,政事又被被傅太皇太后把持,所以轉而寵幸董賢(同性戀)。漢哀帝還封董賢為大司馬,並想讓帝位。這些都激起普遍反感,當時地方百姓備受地方豪強、地主欺壓,國家已是一片末世之象,民間「再受命」說法四起。此時王氏勢力尚在,王莽本人更受儒生懷念與人民的擁戴。前1年汉哀帝去世,王莽奉王太皇太后王政君之命,重返朝廷擔任大司馬。

返回政局的王莽積極推行篡位之路,他擁立年僅九歲的汉平帝為傀儡,並且陸續受封安漢公、宰衡等崇高之職。他以王舜、王邑為心腹,甄豐、甄邯、孫建為將領,平晏與劉歆為參謀。打擊何武、公孫祿等反對他的大臣與傅、丁、衛(漢平帝母家)等外戚勢力。王莽又積極施行善政,攏絡人民。民間有災害即捐錢賑災,擴充太學以徵求各地人才,甚至操弄讖緯、杜撰古史以獲取禪位的合法性。到了5年,朝中一片都是王莽勢力,王莽受封「九錫」,汉平帝十分不滿。同年,汉平帝猝死,王莽迎立宗室刘婴即位,即孺子嬰。最後藉由眾大臣以讖緯之事推舉,王莽得以對內稱「假皇帝」,對外稱「攝皇帝」。9年王莽建國「新」,即新朝,西漢滅亡。

王莽是儒家學派巨子,且以新聖自居,所以積極改制西漢末年亂象,意圖回復到儒家歌頌的夏商周三代盛世。改制的內容上從典章制度、法律與教育,下到人民習俗、經濟制度等,十分全面。雖然王莽積極的改革,但大多依據《周禮》的制度推行新政,對於西漢末年的亂象未能完全對症下藥。由於亂象未能改善,不久地方爆發叛亂,顛覆新朝。

首先他依《周禮》改官名為西周官名,例如改大司農為羲和(後為納言)、改郡太守為大尹等等。地方制度也效仿周代的封建制度,許多地名經過多次改名。由於官員和百姓無所適從,最後還是回復原名,平白增加無謂的煩惱。經濟方面,王莽於9年推行改革,內容大多與漢武帝推行的制度相似,不過王莽是依據《周禮》來制定。土地制度方面,耕地收為國有,推行王田制,限制豪強百姓只能有一定土地大小。然而這些措施與現實狀況差異過大,地方豪強不可能因為一道法令而服從;由於土地過小而不能負擔一戶生活,連人民都反對這個改革。三年後,王莽接受區博的建議取消王田制。而奴婢問題,因為王莽無意廢除奴婢制度,而又禁止自由買賣,導致豪強在黑市賤賣奴婢,這個措施最後也廢除了。在改革財政方面,為了防止商人剝削,王莽建立五均六筦與賒貸政策,以公權力平衡物價,類似後世國家社會主義政策。以五均官掌管工商業的利得稅(類似所得稅),把鹽、鐵、酒、幣制、山林川澤收歸國有,以提升國家收入。然而,這些政策多是由薛子仲、張長叔等富商大賈執行。他們以變法為名義,勾結地方官員榨取百姓,使得地方貧富更加懸殊,國家經濟更加失調。最後是貨幣改革,這是最失敗的政策。王莽依據古制,陸續推行刀貨、貝貨等新幣,到2年共有黃金、銀貨、龜寶、貝貨、錢貨、布貨等。這些數種貨幣擾亂新朝財政,到14年又盡數廢除,以致農商失業,經濟崩潰。

對外方面,王莽依據儒家大一統思想,認為世界上應該只有一個王號,所以將諸侯「王」改稱「公」,將四周屬國由「王」改為「侯」。為了宣示新朝的威德,王莽收回原漢朝發給四周各國的「璽」,換成新朝的「章」。這些措施,使漢家諸侯窮困潦倒,匈奴、高句麗、西域諸國和西南夷等君王先後拒絕臣服新朝。11年,由於匈奴不願臣服新朝,王莽發兵三十萬北伐,戰事連年不決。隔年,王莽強迫高句麗協助伐匈奴,反而使高句麗反叛,屢次侵擾東北。同時間西南夷的鉤町叛變,王莽屢次派兵都未能平定。不久西域發生內亂,王莽派王駿西征未果,反而使西域各國正式與新朝斷絕關係。王莽為了報復,又將匈奴單于改為「降奴服于」,高句麗改名「下句麗」。

王莽改革並非是依據孔孟思想的儒家學說,大部分是迷信讖緯和復古論,照搬儒學經書如《周禮》等改革,政策多迂通而不合實情。其政治純屬「書生政治」,僅關注政策的制訂方式,對於實施方式與效率並不在意,政策朝夕相改,官員百姓到最後都敷衍了事。百姓本來期盼王莽能帶領他們脫離西漢末年的亂象,但反而使百姓跌入黑暗的深淵,最後引發民變,促使新朝滅亡。

新朝執政不當引發民怨,17年全國發生蝗災、旱災而饑荒四起,農民紛起反抗叛亂,開啟新末民變。瓜田儀等人於會稽長洲(今江蘇苏州)起事。同年琅邪女子呂母因縣宰冤殺其子,率眾攻陷海曲縣,而後引兵入海為寇。在災情最嚴重的青州、徐州與荊北地區,則分別在17年於荊北興起綠林軍、18年於青徐地區興起赤眉軍,史稱赤眉、綠林起義。河北在馬適求之亂後,也陸續出現其他民變軍,有的以山川土地為號,或以軍容為號,其中以銅馬軍最強。當大臣舉報各地民變時,王莽認為至只是民賊作亂。他不願承認執政錯誤,反而遷怒大臣,並且造作「威斗」、「華蓋」以粉飾太平。為了防止州郡叛變,不許州郡擅自發兵平亂,導致亂事擴大。而中央軍紀律敗壞,四處掠奪,戰鬥力又不強,時常有數十萬大軍被義軍擊潰之事。在這些義軍中,以赤眉軍與綠林軍對局勢影響最大。

赤眉軍由樊崇所建立,18年他率飢民在莒縣起事,眾皆將眉毛染紅,即赤眉軍。亂軍由農民組成,大多不識字,組織包括地位最高的三老、其次有從事和卒史等名稱,大多延用漢朝鄉官的名稱。赤眉軍收編呂母部屬後,在泰山山區擴大勢力。21年王莽派太師犧仲景尚、更始將軍護軍王黨出兵討伐,但在隔年被赤眉軍擊潰而死。王莽再派太師王匡、更始將軍廉丹率十萬兵東征,所經之路都縱兵掠奪。關東人民都稱「寧逢赤眉,不逢太師!太師尚可,更始殺我!」。雙方爆發成昌之戰,最後王匡慘敗,廉丹被殺,赤眉軍擴張到青、徐、豫、兗等州(約今山東、河南與江蘇北部)。只有翼平連率田況率領百姓守衛青徐部分地區,一度阻擋赤眉軍入侵。

綠林軍源自荊北民變。17年荊州北部發生飢荒,王匡、王鳳率領飢民於新市(今湖北京山東北)綠林山起事,稱綠林軍。新莽荊州軍被綠林軍擊敗後,王莽遣司命將軍孔仁守豫州,派納言將軍嚴尤、秩宗將軍陳茂進入荊州平亂。隔年綠林山瘟疫爆發,王常、成丹率兵轉入南郡,稱下江兵,王匡、王鳳率兵東進新市,稱新市兵,並北上攻打宛城。途中於平林(今湖北隨縣東北)獲得陳牧、廖湛率眾加入,即平林兵。下江兵被嚴尤擊敗後,也北上南陽會合新市兵。南陽當地豪強劉縯、劉秀也舉兵響應,稱舂陵兵。23年二月,綠林聯軍擊破新莽軍甄阜、梁丘賜等將,包圍宛城,占領昆陽,史稱藍鄉之戰。綠林諸將擁護劉玄為更始將軍,最後稱帝,建元更始,史稱更始帝,即玄漢。王匡、王凤、朱鲔、刘縯等人被封将相。

王莽得知宛城被圍後,於23年五月派王邑、王尋率領四十二萬新莽軍圍攻昆陽,目標宛城。6月劉秀率軍於昆陽擊潰新莽軍,王尋敗死,王邑逃至洛陽,宛城也被劉縯攻陷,史稱昆陽之戰。四方豪傑得知後,紛紛殺掉州郡守,自稱將軍,用更始年號。戰後劉縯、劉秀人氣大增,更始帝以違抗命令為由处死劉縯。劉秀得知後親赴謝罪,不敢為劉縯服喪。更始帝心有所慚,遂拜劉秀為破虜將軍,封武信侯。更始帝派遣王匡攻洛陽,申屠建、李松攻武關,三輔震動。王莽在南郊舉行「哭天大典」,以求天救,只要民眾哭得夠哀傷,就可加官進爵。但於秋天,更始軍仍攻入长安,王莽在混乱中为商人杜吴殺死於未央宮的漸臺,新朝亡。

新朝亡後,更始帝定都洛陽,由於局勢混亂,就派劉秀巡視黃河以北。劉秀在河北勢力單薄,恰巧地方實力派王郎於河北稱帝,劉秀只能暫避鋒芒。24年初,劉秀獲得劉植、耿純擁護,聯合上谷耿況、漁陽彭寵包圍王郎,同年四月攻陷邯鄲,王郎亡,劉秀被更始帝封為蕭王。而後劉秀率吳漢、鄧禹等將領平定銅馬等河北諸民變軍,被關西人號為銅馬帝。最後在25年六月於鄗城(今河北柏鄉)即皇帝位,史稱光武帝,國號為漢,史稱東漢。

24年更始帝遷都至長安,他建國後昏庸無能,濫封諸侯,將政事委託給岳父趙萌,政治混亂,反而使人民懷念王莽。當時李軼、朱鮪自立於山東,王匡、張卬橫暴三輔。而在汝南、潁川的赤眉軍,因為糧食不足,加上更始帝分封不給國邑,於25年由樊崇和徐宣兵分二路進逼長安。進軍途中擁立劉盆子為帝,史稱赤眉漢,建元建世。更始帝得知赤眉入侵時,還殺害申屠建、陳牧、成丹等將領。同年九月,赤眉軍攻陷長安,更始帝不久被殺,玄漢亡。光武帝乘機南下洛陽,並定都之(改稱雒陽)。赤眉漢政治混亂,諸將跋扈,劉盆子與其兄練習投降的詞說。27年關中缺糧,赤眉軍引兵東歸。光武帝率軍與東歸途中的赤眉會戰於華陰,赤眉軍大敗,劉盆子與樊崇投降。

27年光武帝領有河北大部、河洛與關中等地,但天下仍然群雄割據。當時約有數個勢力,燕代之地的九原盧芳、漁陽彭寵;關東淮水的睢陽劉永、青州張步、東海董憲、魯佼彊與廬江李憲;荊州隴蜀等地有天水隗囂、河西竇融、成都公孫述、漢中延岑、南郡秦豐與夷陵田戎;以及南海鄧讓等勢力。光武帝採取先東後西的戰略,先安撫燕代勢力,於27年到30年間集中力量消滅劉永(自稱天子)為首的關東勢力與廬江的李憲(自稱天子),東方大定。而後南向征服秦豐、田戎等等荊州諸侯,諸侯殘部投奔蜀地。同時,漁陽彭寵也被人刺殺。河西竇融與南海鄧讓也於29年歸順東漢。最後西向對付受荊隴諸侯擁護的成家帝公孫述與天水隗囂、九原盧芳等。光武帝於34年平定天水隗囂,於36年由吳漢攻克成都,滅成家。隔年九原盧芳亡命入匈奴,東漢統一天下,進入東漢時期。

新朝疆域大致上與西漢相同,但在末期疆域萎縮,只新設了西海郡(郡治龍耆城,今青海民和縣)而已。遼東地區撤消了真番、臨屯二郡。在西南地區由七郡變成五郡,部分西南夷成半獨立狀態,放棄了海南島與象郡。西域諸王與新朝中斷關係,使得新朝勢力退出西域。這些疆域直到東漢前期才陸續收復領土。

新朝的行政區劃大致與西漢後期相同,但由郡縣制上加州牧,並且與分封制結合。王莽推行復古改制,亂設行政區劃,改了許多新地名。並且網羅漢宗室功臣後裔、封建官僚,改郡封國。在設置行政區劃方面,王莽修改西漢十三部,據《堯典》分成十二州,裁撤朔方、司隸部,改涼州為雍州、交趾為交州;後又據《禹貢》改爲九州。有的郡甚至五易其名,最後又恢複舊稱。地名的混亂,十分困擾人民。9年更改地方官制的名稱為古稱。14年後大規模更動,結合分封制和郡縣制,郡縣首長與受有茅土的諸侯二合一,將郡太守(新朝稱大尹)分成卒正(侯爵)、連率(伯爵)與大尹等。地方軍事單位的都尉,分成屬令(子爵)、屬長(男爵)等。

在官職的部分,14年設立州牧、部監以監督地方各郡,地位等同三公。設監,地位同上大夫,監督五郡事務。更置牧監副,秩元士,冠法冠,行事如漢刺史。。随着六队、六尉等的建立,新朝也派出监察官员对这些队、尉进行监察。17年,王莽選用能吏侯霸等分督六尉、六隊,如漢刺史,與三公士郡一人從事。

十二州:首長即州牧,新朝增設為部監、監與州牧。共有幽州、并州、冀州、青州、徐州、兗州、豫州、雍州(原涼州)、揚州、荊州、益州、交州,後改為九州。郡:首長原為太守,新朝改稱大尹、後增設為卒正、連率與大尹。縣:首長原為縣令,新朝改稱宰。

新朝官制上承西漢官制,正值王莽改制,所以新朝官制多變,官名及職責也十分複雜。自西漢居攝年間起,王莽便開始推行改制。他附會周禮官制,恢復五等爵制,濫加封賞,濫改官名,如宰衡、太阿之職。建國後,在中央置四輔、三公、四將、六監、九卿、二十七大夫、八十一元士等。四輔(太師、太傅、國師、國將)、三公(大司馬、大司徒、大司空)、四將(更始將軍、衛將軍、立國將軍、前將軍)合稱「十一公」。是9年新朝建國初年,王莽按照哀章所獻金匱內記錄所封王舜、平晏、劉歆、哀章、王邑、甄豐等等十一位高官。其中四輔位列上公,對映五嶽其中的四嶽。四輔與三公對映日月星辰。由於《書·堯典》中,有羲仲、羲叔、和仲、和叔,14年王莽將這些稱號對映到四輔,例如太師犧仲景尚。

新朝九卿與西漢九卿大為不同,王莽修改九卿成新朝用的官名,其中將水衡都尉更名為予虞。將宗伯移除,光祿勳、衛尉、太僕被列入六監,加入大司馬司允、大司徒司直、大司空司若,最後湊成九卿。每卿有大夫三人、每大夫有元士三人,合稱二十七大夫、八十一元士。六監位皆上卿,包含原光祿勳、衛尉、太僕、中尉、執金吾,但都改成新朝官名,並且新設大贅。最後設五司大夫為監察官。

10年更始將軍甄豐之子甄尋不滿父親封賞過低,作符命,言新室要如西周分陝,立二伯。以甄豐為右伯,平晏為左伯,如周召故事,王莽即從之。而後甄尋作符命欲娶黃皇室主为妻,符命案爆發,甄尋逃亡,甄豐自殺。最後甄尋、劉歆之子劉棻、劉泳,王邑弟王奇,及劉歆門人丁隆等數百人或流放、或被斬。更始將軍也被寧始將軍姚恂取代。大封宗室及功臣後裔二百人為侯。為了杜絕反新勢力,10年王莽聽從孫建建議,廢除劉姓諸侯,並將部分擁護劉姓諸侯(劉歆、劉龔、劉嘉等三十二人)賜姓王。」改定安太后號曰黃皇室主,絕之於漢。

王莽建立新朝後,為了宣示新朝的威德,派遣使者四出,東到遼東及朝鮮半島北部的玄菟、樂浪、高句麗及夫餘;南到西南邊境;西到西域。收回舊日漢朝授予外族的印綬,改受新朝的印綬,並把所封的王貶為侯,所用的璽改為章,這樣就引起西南夷鉤町王及匈奴的叛變,西域諸國也逐漸與王莽破裂關係。

在北方方面,匈奴與西漢和平約有30多年,直到新朝建立為止。王莽推行改王為侯的政策,並將「匈奴單于」稱號改為「恭奴善于」,後改為「降奴服于」。為了弱化匈奴,王莽分匈奴居地為15部,強立呼韓邪子孫十五人俱為單于(如孝單于、順單于 (助)、順單于 (登)等)。匈奴烏珠留單于因此而叛變,王莽就於11年徵發士兵三十萬人,大舉進攻匈奴。由於戰事連年不決,自宣帝以來,「數世不見煙火之警,人民熾盛,牛馬布野」的北方邊界,又變成了「北邊虛空,野有暴骨」的悲慘情況;而新朝北部的人民也因為戰亂而相聚為盜,動亂開始形成。王莽為了討伐匈奴,於12年強令高句麗、烏桓出兵,兩國皆不願而叛變,西域地區也陸續叛新投匈。新朝滅亡後,匈奴呼都而尸道皋若鞮單于認為有機可趁,扶持九原盧芳與漁陽彭寵,其中盧芳還被封為漢帝。另一方面,率軍東掠并、燕,西侵涼、朔,對當時的新成立的東漢威脅很大。東北方面,高句麗是東北強國,役屬沃沮、東濊。高句麗叛變新朝後侵擾東北各郡,新朝遼西太守田譚戰死。王莽派嚴尤出兵斬其王,但高句麗別種濊貊仍舊屢次寇邊。直到東漢初年,還入侵右北平、漁陽、上谷等幽西數郡。而烏桓與鮮卑連和,烏桓叛變新朝後投奔匈奴,在東漢中期匈奴衰退後與鮮卑瓜分漠北領地。

西域方面,到漢哀帝、漢平帝時,西域已有五十五國。王莽建立新朝後,西域諸國大多不服統領,而匈奴勢力也進入西域的塔里木盆地。13年親近匈奴的焉耆就殺西域都護但欽,投奔匈奴陣營。王莽於16年派五威將王駿、李崇與郭欽等西征西域,最後被焉耆率領姑墨、尉犁、危須等連軍擊潰,王駿被殺,西域與新朝斷絕往來。西域北道諸國淪入匈奴勢力範圍,只有位於西域南道的莎車率領南道諸國抗衡匈奴。到東漢初年,以莎車王延與其子康最支持漢朝,但漢光武帝為了要全力對內,不能支援南道諸國。不久全西域地區被匈奴占領。而西羌的部分,王莽用政治手段領有西海郡(今青海海宴附近)。到新末漢初,西羌遷入境內掠奪,隗囂招懷其酋豪,隴西數郡都成五谿羌、先零羌的勢力範圍。同時間位於四川松潘一帶的武都參狼羌也被蜀地的公孫述煽動,發起叛亂。這些羌族於35年至37年被東漢馬援所平定,到光武末年,燒當羌又崛起,成為東漢一朝的西患。在北方、西方一片叛亂之際,12年西南夷的鉤町(今雲南廣南一帶)也發生叛變,鉤町王攻殺牂柯太守周韶,跟者益州蠻夷也攻殺太守程隆。越巂、遂久、仇牛、同亭、邪豆之屬,陸續叛變。王莽屢次派兵討伐,寧始將軍廉丹率領的大軍水土不服,數十年拖延未果。而後以文齊為太守,他開墾南中,勸降西南夷,與其恢復關係。公孫述佔據蜀地後,文齊據南中不願投降。到東漢時才歸順漢光武帝。

軍事制度承襲西漢,但王莽更改官制名稱為古稱。為了平定叛亂,王莽採取「以軍領政」的方式控管地方。他命令原為文官的「七公六卿」都兼稱將軍,監督地方官吏,以便穩定地方治安。11年并州、平州發生民變,派著武將軍逯並駐守平亂。22年由於綠林軍在荊北作亂,派司命將軍孔仁駐守豫州,納言將軍嚴尤、秩宗將軍陳茂平定荊州,此為兼稱將軍的實例。以軍領政的方式還有於10年命中郎將、繡衣執法各五十五人,分別駐守大郡,監督地方。中郎將不僅涉内政,兼有對外職責。王莽所封的太子四友,就有中郎將廉丹。此外還在內置司命軍正,外設軍監十二人。20年,新末民變期間,王莽見四方盜賊多,復欲厭之,又置前後左右中大司馬之位,賜諸州牧號為大將軍,各郡的卒正、連帥、大尹為偏將軍,屬令為裨將軍,縣宰為校尉。然而部分軍人在地方胡作非爲,擾亂地方行政,以軍領政的方式還是失敗。

在建立新軍方面,王莽陸續建立豬突豨勇、理軍等新軍,但無大用。與匈奴發生戰爭的期間,王莽招募天下丁男、死罪囚、吏民奴而編成的新軍「豬突豨勇」。又令公卿以下至郡縣黃綬皆保養軍馬,多少各以秩為差。又招募自稱有竒技術可以攻匈奴的人,然而大多誇大其詞,但王莽仍拜為「理軍」,賜以車馬。此外還設立捕盜都尉以平定三輔盜賊。23年王莽拜將軍九人,皆以虎為號,號曰「九虎」,率領北軍精兵數萬人前往關東平亂。這些士兵的妻子與兒女還留在宮中當人質。以上介紹的這些新軍,除捕盜都尉外,其餘大多無用。

新朝时期没有具体的人口调查,估计17年全国有5600万人。由於王莽改制失败加上自然灾害频发和14年黄河下游改道,致使17年爆发绿林赤眉起义。之后烽火遍地,军阀割据和混战,造成黄河流域大量人口死亡,其餘为躲避战火大量向长江流域迁徙。東漢初年,江南地区人口升至全国四成,口数超过500万的有豫州、荆州、扬州與益州等四州。南方人口增长的同时,北方大部分郡国人口减少。

新朝的經濟政策有部分是遵循古制,有部分是重建西漢漢武帝時的經濟政策。立國初年,西漢末年的土地與奴婢問題依舊存在。為了穩定統治,王莽附會《周禮》上的古制,先後下令改制。針對土地被豪強強烈併購與大量貧窮人口轉為奴婢的問題,王莽建立了王田制與禁止奴婢買賣(私屬制)。王田制於9年推行,視全國土地為朝廷所有,稱為「王田」,王田不得任意買賣。恢復井田制,限定男丁八口以下之家,佔田不得超過九百畝(一井),超過的土地須分給宗族鄉鄰。如果無地者由政府授田,每夫一百畝,這是與後世均田制極為類似。針對奴婢問題,王莽推行私屬制,禁止奴婢自由買賣。然而地方大地主激烈反對土地轉讓,王莽雖然派張邯、孫陽到地方強力推行,反而使地方大亂。三年後,王莽接受區博的建議於取消王田制。而私屬制,因為王莽禁止奴婢買賣,地方豪強競相於黑市賣奴婢,使價格低落,最後也宣布廢止。

為了穩定物價、鼓勵生產、增加國家稅收與打壓商人,早在漢武帝時就向商人和工匠征税,但王莽的制度更加完整。他建立五均六筦政策、貢所得、徵荒地稅與賒貸。這是新朝在民生及財政的重要革新,也說是一種國家社會主義政策的推行。「五均」就是把鹽、鐵、酒、貨幣、山林川澤等五類收歸國有以控制經濟,平衡物價,防止商人剝削,增加國庫收入。五均官還針對漁、獵、畜牧、巫、醫以及養蠶、紡織等業,均收取所得純利的十分一,稱「貢」,即現代的所得稅。「六筦」即六管,就是前面的五均與貢所得等六項由官府管理,對每一項制定條例與處罰。此外,為了鼓勵生產,對荒地徵荒地稅,鼓勵開墾荒地。對貧窮或需資金周轉的人,給予賒貸。這些政策雖然出自好意,但推行者多是薛子仲、張長叔等大商人。這些商人到處和地方官吏勾結以榨取百姓,百姓未蒙其利,先受其害。且改革步驟太快,朝令夕改,使百姓官吏不知所從,經濟更加崩壞。

針對貨幣,王莽先後五次改幣。7年,王莽附會周代鑄大錢之說,加鑄契刀、錯刀、大錢與漢代五銖錢共為四品。9年,除大錢外的貨幣均廢除,並鑄小錢與大錢通用,並嚴禁盜鑄。隔年,另造二十八種貨幣:黃金一品、銀貨二品、龜寶四品、貝貨五品、錢貨六品、布貨十品。錢、布共為銅製,所以總稱「五物、六名、二十八品」。後因人民抵制繁雜的莽幣,改用漢五銖錢。在官府無法禁止的情況下,王莽又盡廢諸幣,改行貨幣、貨泉兩品,並於許民間鑄大錢(限期六年)。這樣反覆的改革幣制,讓新朝的經濟混亂,加速人民破產。

新朝思想上起西漢,下承東漢。西漢末期盛行讖緯學說,讖緯是神學與庸俗經學的混合物。儒生好談災異、祥瑞,常以自然現象來附會人事的禍福,後來成為王莽建立新朝的依據。早在西漢時,儒生就多信奉陰陽家「五德終始」之說,盛言「天運循環,貴賤無常」,相信「漢歷當終,新王將興」。由於社會改革的要求及天運循環的理論相結合,儒生鼓吹禅让、改元易号以更始,這些都成為王莽建立新朝时所依靠的理论。

前78年汉昭帝時,眭弘便附會董仲舒之言,認為漢帝應該尋到賢人,禪讓帝位給他,自退位為王,如同夏代堯、周代商故事,他將董仲舒半人半神的神學目的論演變為讖緯神學。汉成帝时,又有齊人甘忠可詐造天官歷、包元太平經十二卷,以言「漢家逢天地之大終,當更受命於天,天帝使真人赤精子,下敎我此道。」甘忠可傳授給重平夏賀良、容丘丁廣世、東郡郭昌等。甘忠可弟子夏賀良等對漢哀帝陈说西漢中衰,當更受命。於是漢哀帝改元太初元將,號曰『陳聖劉太平皇帝』。後因無嘉應,漢哀帝遂誅殺夏賀良等人。讖緯學說到新莽時達到高峰,王莽崇尚古制,也利用讖緯學說以取得帝位。他假借符命、祥瑞,偽造禪讓的根據,如製作了「告安漢公莽為皇帝」的石碑、「金匱神嬗」書言王莽為真天子等讖緯。

西漢末年也有人對陰陽家提出質疑,揚雄仿《論語》作《法言》,模仿《易經》作《太玄》,提出以「玄」作為宇宙萬物根源之學說,強調如實的認識自然現象,並認為「有生者必有死,有始者必有終」,駁斥了方士的學說。他主張要回復儒學五經的本來面目,為東漢注重文字本身的真實性的訓詁學開啟了先河。

新莽的覆滅代表儒學家復古思想的破滅,也使漢儒變法禪讓的政治理論至此消失,漸變帝王萬世一統的思想。先秦學術注重矯正社會的病態,建立大同世界。由於王莽新政的失敗,說明以古代禮法改革的方式不通。魏晉以後,思潮不向整體利益求答案,轉為尋求人性及生存的意義,玄學及佛學遂取代先秦諸子的思想地位。

新莽時期,王莽與劉歆等儒者提倡古文經,使古文經與今文經抗衡,即古今文之爭。王莽還於五經以外增設樂經,增加古文經博士和博士弟子的人數五人。並且擴建太學和太學生宿舍,於地方學校廣招生徒,徵求各地異才。

古今文之爭源自秦始皇焚毀經書事件,後來儒者憑記憶書寫經書,成為今文經。在西漢時於孔壁發現古經書,稱為古文經。西漢的五經(樂經已失散)博士仍以今文經為主。西漢晚期,今文經學派如劉向等人受陰陽家影響,偏向怪力亂神,到西漢末年出現讖緯學說。古文經學派則在西漢末年由巨儒劉歆(劉向之子)與王莽提倡。漢成帝時,劉向負責整理古文經,劉向去世後由劉歆繼承。劉歆最後完成編目,即《七略》,這是中國最早的目錄書,集結古代學術思想與著作的內容。劉歆在整理古籍中發現先秦時期的蝌蚪文,主要有《春秋左氏傳》、《古文尚書》、《逸禮》等等,並認為《毛詩》與其他家派不同,可列為古文。最後劉歆大力提倡古文經,並建議立古文經博士、學官,得以和今文學家抗衡,這受到今文學家的抵制,即今古文之爭。新朝成立後,王莽為上述古文立古文經博士。雖然東漢成立後古文經博士被廢,但不排斥古文經,而且民間研究風氣大盛,三國時期古文終於取代今文,成為學術正統。

王莽改定文字為新莽六書,即古文、奇字、篆書、佐書、繆篆、鳥蟲書,可分成古代文獻文字、通用文字與應用文字等。古文為孔壁經書的戰國文字,奇字是有非孔壁古文的戰國文字,均屬古代文獻文字。王莽为了拉抬古文经学的地位,所以將古文及奇字分列六书的前二位。篆書即秦朝小篆、佐書即秦朝隸書,為新莽時期的通用文字,兩個都廣泛運用,一般日常文書也是用佐書。繆篆為小篆變體,較為權威、莊重的場合使用,如銅器、印章、石刻、貨幣、瓦當等;鳥蟲書即秦體蟲書,用於旗幟和符信,與繆篆都是應用文字。

语言学研究方面,揚雄曾著《方言》,叙述西汉时代各地方言,为研究古代语言的重要资料。王莽当政後,拉拢扬雄,任他为中散大夫。揚雄還写过《劇秦美新》,指斥秦朝,美化新朝。

新朝藝術屬於漢朝藝術的一個時期,比較有特色的有印章、書法與墓畫。新莽的印章屬於秦漢印章的系統,但其工藝水平高,古代璽印無出其右。新莽印章具有自己的風格,分成繆篆與鳥蟲書兩類,在制度、印文、字數、名稱諸方面,與秦漢、魏晉南北朝的印章有較大的區別,其所達到的藝術水準堪稱秦漢印章中最大的靚點。

書法方面,《王俊幕府檔案簡》為起草正式文書的底稿,這是完全成熟的草書。而《郁平大尹馮君孺久墓題記》與《張伯升柩銘》相似,屬於繆篆。字形總體為方扁形,偏旁結構很明確地分為方、圓兩類,方形結構以直線銜接建構而成,圓形結構以曲線糾結盤繞形成,兩者構成了鮮明的對比。

西漢末年到新莽時期,墓室內繪製的壁畫的面積增大,增入了世俗生活宴樂的內容。墓例有洛陽金谷園和偃師辛村的新莽墓,金谷園墓前室穹隆頂在白地上以朱墨等色滿繪彩雲,四壁影作枋柱,以象徵木結構建築。後室頂脊及柱頭斗子間分繪日月神靈異獸。辛村墓則除日、月及辟邪畫面外,其最有名的有《壁画宴乐图》与《西王母图》。並且繪有多幅門吏、庖廚、宴飲、六博等世俗生活的畫面。

除了墓畫,自西漢宣、昭兩帝萌芽的畫像石也有充足的發展,以河南唐河出土的天鳳五年的「漢郁平大尹馮君孺人畫像石墓」為經典之作。該區畫像石的內容豐富,約有30餘。主要描述墓主生活的迎賓拜謁、馴虎騎象、樂舞雜技,反映儒家倫理道德的歷史故事,反映仙人思想的羽人、應龍、四首人面虎以及鎮墓辟邪神怪如蹶張、青龍、白虎、朱雀、鋪首銜杯等。由於主題明顯、內容豐富又質樸,而且紀年明確,所以十分被重視。

\subsection{王莽生平}

王莽(前45年-23年10月6日),字巨君,魏郡元城貴鄉(今河北邯郸大名縣東)人。新朝皇帝。西漢末年政治人物及權臣,之後篡奪皇位並自立新朝。

王莽為濟北王田安六世孫,即陳國、田齊之王裔,田家失國後,齊地的庶民卻依然稱呼田家為「王家」,日久,田家由田姓改為王姓。

王莽原籍濟南郡東平陵。汉元帝皇后王政君之侄。幼年时父親王曼去世,很快其兄也去世。王莽孝母尊嫂,生活俭朴,饱读诗书,结士,声名远播。

王莽对其身居大司马之位的伯父王凤极为恭顺。王凤临終時嘱咐王政君照顾王莽。汉成帝时,西漢陽朔三年(前22年),王莽初任黄门郎,后升为射声校尉。王莽礼贤下士,清廉俭朴,常把自己的俸禄分给门客和穷人,甚至卖掉自己车马接济穷人,深受众人爱戴。其叔父王商甚至上书愿把其封地的一部分让给王莽。永始元年(前16年)封新都侯、騎都尉及光祿大夫侍中。绥和元年(前8年)继他的三位伯、叔之后出任大司马,时年38岁。

翌年,汉成帝去世。汉哀帝继位后傅太后、丁太后及其外戚得势,王莽免官,隐居新野。其间他的二子王获杀死家奴,王莽逼其自杀,得到世人好评。

元寿元年(前2年)其回京城居住。元寿二年(前1年)汉哀帝去世,並未留下子嗣,由太皇太后王政君掌管传国玉玺,王莽任大司马,兼管軍事令及禁軍,立汉平帝,得到朝野的拥戴。元始元年(1年)王莽在推辞再三之后接受了“安汉公”的爵位,将俸禄转封两万多人。元始三年(3年)王莽的女儿成了皇后。元始四年(4年)加号宰衡,位在诸侯王公之上。大力宣扬礼乐教化,得到儒生的拥戴,被加九锡。元始五年(5年),王莽毒死汉平帝,立年仅两岁的孺子婴为皇太子,太皇太后王氏命王莽代天子朝政,称“假皇帝”或“摄皇帝”。从居攝二年(6年)翟義起兵反对王莽,有人开始不断借各种名目对王莽劝进。初始元年十一月戊辰(9年1月10日),王莽正式称帝,改国号为「新」,改长安为常安,封孺子婴为定安公。是為始建国元年。

他當上皇帝後仿照周朝的制度推行新政,屢次改變幣制,更改官制與官名,以王田制為名恢復井田制,把鹽、鐵、酒、幣制、五均賒貸及山林川澤收歸國有,不停恢复西周時代的周禮模式。「今更名天下田曰王田,奴婢曰私属,皆不得买卖”。由於政策多存在不合實情處,百姓未蒙其利,先受其害,不斷挑起天下各貴族和平民的不滿。另外措施又不合時宜,所以措施如王田制推行四年便令民怨沸騰。這可見於當時的中郎將區博所言:「井田雖聖王法,共廢久矣。」和學家馬端臨《文獻通考》所言:「廢之於寡,立之於眾,土地有列在豪強。」

此外,王莽外交政策極為不當。他將原本臣服於漢朝的匈奴、高句麗、西域諸國和西南夷等屬國統治者由原本的「王」降格為「侯」。又收回並損毀匈奴单于之印璽,改授予新匈奴單于之章;甚至將匈奴單于改為降奴服于,高句麗改名下句麗;各國因此拒絕臣服新朝,造成邊境戰争不絕。

天凤四年(17年)各地民軍纷起,有赤眉及绿林大規模的反抗。地皇四年(23年)王莽在南郊舉行哭天大典。同年,绿林军攻入长安,王莽在混乱中为商人杜吴所杀(王莽終年68歲),並被校尉公賓就斬其首,懸於宛市(今宛市镇)之中,新朝灭亡。其頭顱後來被各代收藏,直到西晉元康五年(295年)晉惠帝時,洛陽武庫大火,王莽的頭顱被焚毀。

王莽陵一说在今陕西省渭南市华阴市红岩村附近。由于年代久远,陵墓痕迹难辩。在陕西省安康、镇安、旬阳三县市交界处有王莽山。主峰海拔1560米。地势险要,有王莽墓、刘秀寨遗迹。

中國传统历史学强调忠君、家天下等理念,对王莽的評價普遍不高,一般都認為他只是一位「偽君子」,眾口一辭的千古罪人。东汉史學家班固修订的《漢書》就把王莽列作「逆臣」一類和「佞邪之材」。」而後世評價也大抵是受到了後漢時代史家所影響。事實上王莽本身是篡漢而取得帝位,而同時也是漢朝宗室所滅,從漢朝政權來看,王莽被視作「逆臣賊子」是理所當然。而他在取得帝位前的種種行徑,更被視為作為「逆臣賊子」的理據,例如他殺了漢平帝而立了孺子嬰為皇帝就為一例。

近人胡適開始為王莽平反:“王莽是一千九百年前的一個社會主義者。”他認同王莽改革中的土地國有、均產、廢奴三個大政策,“王莽受了一千九百年的冤枉,至今還沒有公平的論定。他的貴本家王安石受一時的唾罵,卻早已有人替他伸冤了。然而王莽卻是一個大政治家,他的魄力和手腕遠在王安石之上……可憐這樣一個勤勤懇懇,生性『不能無為』,要『均眾庶,抑並兼』的人,到末了竟死在斬台上,……竟沒有人替他說一句公平的話。”

但從另一角度看,王莽也是書生式政治家。王莽登位後推行之新政,大抵都是為了仿照周朝的制度推行,如屢次改變幣制、更改官制與官名、以王田制為名恢復井田制,把鹽、鐵、酒、幣制、山林川澤收歸國有,都是不停恢复西周時代的周禮模式。可是古今風俗不同,環境各異,源於古制的新法,未必一切都合時合宜。而這些新政都是違反了歷史規律,所以推行失敗,自屬歷史必然。所以這個角度看,王莽是一個事事復古,脫離現實的政治家,就正如史家錢穆所言:「王莽的政治,完全是一種書生的政治。...... 不達政情,又無賢輔,徒以文字議論政治。」

旅美歷史學家黃仁宇則指出,從王莽登位後發出的一系詔書中看到,王莽的政策根本脫離了當時的實際環境,亦缺乏適當的用人安排。他在《中國大歷史》裡語帶諷刺的評論王莽:「他盡信中國古典,真的以為金字塔可以倒砌。」

傅樂成在其著作《中國通史》中評論王莽。王莽具有超人的智力、辯才和威嚴,但也有重大的缺點,諸如過度的自信,一味的復古以及猜疑部下等。王莽的行為看來有些偽,也有些愚,但西漢的偽風並不始於王莽,他不過承襲此風而擴充之,結果以偽獲得名聲並篡位之後,得意之餘,乃至無往而不偽。他有他的政治理想,其新法是為整個西漢政治作一通盤的改革,但因缺乏政治才能又迷信復古,事事行之以偽,才會看來令人有愚的感覺。王莽是實際政治的失敗者,也是復古思想的殉道者,他在政治舞台上所表現的一切,雖然最後都歸幻滅,但實在是不平凡的。

史学家吕思勉也认为以汉朝为出发点的历史评价不公,即将王莽的优点全部用一个“伪”字掩盖。王莽本身博学,礼贤下士,孝敬母亲功显君及寡居的嫂嫂,地位越高而对人越谦虚,而且自己与自己家人的生活始终接近清贫,甚至王莽的妻子因为穿着朴素出门迎客被认为是仆佣。吕思勉认为凡是作伪之人,必然是有所图的,而王莽篡汉称帝所图达到之后却并无改变,一生作为如一,又如何能称其为伪?更重要的是,王莽改制成为中国文化的一次重大转变,在西汉及以前,凡是谈论政治的人大多对社会现状进行攻击要求改革,至东汉及以后,玄学、佛学先后兴起,都强调适应社会,而不再追求改革。王莽的行事,诸如恢复井田等,其实很大程度上代表了从先秦以来仁人志士的公意,无论成败,都应由抱有此类见解的人士共负,而不是王莽一人之责。

吕思勉进一步谈到王莽改革的历史影响,「从此以后大家都知道社会改革,不是件容易的事,无人敢作根本改革之想。“治天下不如安天下,安天下不如与天下安”,遂成为政治上的金科玉律」。

史學家韓復智認為王莽的經濟改革對解決當時的經濟問題有一定的幫助。他在《兩漢經濟問題癥結》中提到王莽推行的經濟措施「除變更幣制外,可謂都切中時弊,真正兼顧到平均地權與節制資本兩方面。」其說法是基於王莽一方面把全國土地收歸國有,平均分配給人民。另一方面,他強制有勞動能力的人從事生產,以改善農民生活。其次,他實行五均六筦,不僅防止資本家的兼併和農民遭受重利盤剝,並且扶助小商人的經營,用來救濟農民。但同時變更幣制的經濟措施付卻令通貨膨脹的情況惡化和幣制混亂,而貧窮的人民更加未能在拉闊了的貧富差距下受惠。連富裕的商人亦都破產。雖然如此,王莽的社會經濟改革仍然得到韓復智的正面評價。

哥倫比亞大學的漢學家畢漢思在崔瑞德及魯惟一主編的《劍橋中國秦漢史》的「王莽,漢之中興,後漢」一章中表示王莽如果沒有真才實學,他不可能升為攝皇帝。他雖篡漢建立新朝,孺子劉嬰受到了他不尋常的寬大,雖然被廢但沒有被殺且能過著隱居的生活。而王莽也將孫女嫁給劉嬰。在始建国元年(9年)爆發了兩次原劉氏皇室的起事,王莽很快就派員鎮壓並牢牢地控制漢室,在長安建都。他執政期間的外交、經濟、政治政策並不如傳統認爲地那樣不堪,只是沒能堅持到表現其積極影響。他的新朝的覆滅應由黃河改道帶來的災難負責,士紳和農民的騷亂中他失去了支持。

柏杨赞美他和平建国,在《中国人史纲》一书中提到:“中国历史有一个现象,每一次政权转移,都要发生一次改朝换代型的大混乱,野心家或英雄们各自握有武力,互相争夺吞噬,最后剩下的那一个,即成为儒学派所称颂为‘得国最正’的圣君,在血海中建立他的政权。王莽打破这种惯例,他跟战国时代齐国的田和一样,用和平的方式接收政权,同时也创造了一个权臣夺取宝座的程式,以后很多王朝建立,都照本宣科。西汉王朝在平静中消失,新王朝在平静中诞生,两大王朝交接之际,没有流血。……王莽是儒家学派巨子,以一个学者建立一个庞大的帝国,中国历史上仅此一次。”文人夺权,没有大面积流血,和平过渡,和平演变,殊属不易。”

王莽是汉初济北王田安的六世孙,是战国时代齐国王室的后裔,因齐人称他们为“王家”,所以后来便以“王”为氏。

\subsection{始建国}

\begin{longtable}{|>{\centering\scriptsize}m{2em}|>{\centering\scriptsize}m{1.3em}|>{\centering}m{8.8em}|}
  % \caption{秦王政}\
  \toprule
  \SimHei \normalsize 年数 & \SimHei \scriptsize 公元 & \SimHei 大事件 \tabularnewline
  % \midrule
  \endfirsthead
  \toprule
  \SimHei \normalsize 年数 & \SimHei \scriptsize 公元 & \SimHei 大事件 \tabularnewline
  \midrule
  \endhead
  \midrule
  元年 & 9 & \tabularnewline\hline
  二年 & 10 & \tabularnewline\hline
  三年 & 11 & \tabularnewline\hline
  四年 & 12 & \tabularnewline\hline
  五年 & 13 & \tabularnewline
  \bottomrule
\end{longtable}

\subsection{天凤}

\begin{longtable}{|>{\centering\scriptsize}m{2em}|>{\centering\scriptsize}m{1.3em}|>{\centering}m{8.8em}|}
  % \caption{秦王政}\
  \toprule
  \SimHei \normalsize 年数 & \SimHei \scriptsize 公元 & \SimHei 大事件 \tabularnewline
  % \midrule
  \endfirsthead
  \toprule
  \SimHei \normalsize 年数 & \SimHei \scriptsize 公元 & \SimHei 大事件 \tabularnewline
  \midrule
  \endhead
  \midrule
  元年 & 14 & \tabularnewline\hline
  二年 & 15 & \tabularnewline\hline
  三年 & 16 & \tabularnewline\hline
  四年 & 17 & \tabularnewline\hline
  五年 & 18 & \tabularnewline\hline
  六年 & 19 & \tabularnewline
  \bottomrule
\end{longtable}

\subsection{地皇}

\begin{longtable}{|>{\centering\scriptsize}m{2em}|>{\centering\scriptsize}m{1.3em}|>{\centering}m{8.8em}|}
  % \caption{秦王政}\
  \toprule
  \SimHei \normalsize 年数 & \SimHei \scriptsize 公元 & \SimHei 大事件 \tabularnewline
  % \midrule
  \endfirsthead
  \toprule
  \SimHei \normalsize 年数 & \SimHei \scriptsize 公元 & \SimHei 大事件 \tabularnewline
  \midrule
  \endhead
  \midrule
  元年 & 20 & \tabularnewline\hline
  二年 & 21 & \tabularnewline\hline
  三年 & 22 & \tabularnewline\hline
  四年 & 23 & \tabularnewline
  \bottomrule
\end{longtable}


%%% Local Variables:
%%% mode: latex
%%% TeX-engine: xetex
%%% TeX-master: "../Main"
%%% End:

%% -*- coding: utf-8 -*-
%% Time-stamp: <Chen Wang: 2021-10-29 17:28:53>

\section{玄汉更始帝刘玄\tiny(23-25)}

\subsection{生平}

漢更始帝刘玄(?-25年),字圣公,南阳郡蔡阳县(今湖北省枣阳市西南)人,两汉之际绿林军擁立的皇帝,或被視為西漢的最後一位皇帝。

劉玄原本是西汉宗室,是汉景帝劉啟的後代、汉光武帝刘秀的族兄。祖父為苍梧太守刘利,父刘子张,母何氏。

刘玄生年不详。起初,刘玄的弟弟被人杀害,他便结交宾客准备报仇,可不久之后宾客犯罪,刘玄便跑到平阳躲避官府的追捕。官吏抓走他的父亲刘子张,后来刘玄诈死,让人把他的灵柩运回舂陵,官吏才释放了他的父亲。随后刘玄便逃跑藏匿起来。

新莽天凤四年(17年),南方发生饥荒,百姓只得到沼泽中挖荸荠野菜度日。新市人王匡、王凤带领马武、王常、成丹等人起义。他们藏身绿林山中,不久队伍就发展到五万多人。

地皇三年(22年)七月,平林人陈牧、廖湛领导千余人起义,号平林兵,以应王匡,正在这里避难的刘玄参加平林军,担任安集掾。地皇四年(23年)正月,绿林军诸部合兵击破新莽将领甄阜、梁丘赐,遂号刘玄为更始将军。二月辛巳(3月11日),后因其为刘姓宗室,遂在淯水边被拥立为帝,大赦天下,建元更始。更始元年六月入都宛城,大封宗室诸将。他嫉刘縯、刘秀兄弟威名,便诛杀刘縯。起义军在昆阳之戰获胜后,更始帝遣王匡攻洛阳,申屠建、李松攻武关,三辅震动,各地豪强纷纷诛杀新莽的州牧郡守,用汉年号,服从更始政令。更始元年十月,定都洛阳。王莽败死后的更始二年(24年),迁都长安。

李松和趙萌進言更始帝,将功臣封王。大司馬朱鮪反对,称汉高祖的誓言非刘氏不王。更始帝采纳李松之言,以王匡为比阳王、王凤为宜城王、朱鲔为胶东王、卫尉大将军张卬为淮阳王、廷尉大将军王常为邓王,执金吾大将军廖湛为穰王、申屠建为平氏王、尚书胡殷为随王、柱天大将军李通为西平王、五威中郎将李轶为舞阴王、水衡大将军成丹为襄邑王、大司空陈牧为阴平王、骠骑大将军宋佻为颍阴王、尹尊为郾王。李松昇進丞相,和趙萌共理内政。

更始帝雖恢復漢室,但其性格迂腐,重用小人。一登基便沉湎于宫廷奢華生活,即位后将政事都委托于自己的岳父赵萌,放任其外戚专权。赤眉军进逼长安时,更始帝杀害申屠建、陈牧、成丹等起义军重要将领。更始三年(25年)他殺害西漢末代君主劉嬰。

更始三年(25年)九月,赤眉军攻入长安,更始帝单骑逃走。同年十月,投降赤眉,将玺绶送给赤眉拥立的皇帝刘盆子,自己被封为畏威侯,不久改封为长沙王,刘秀则遥降封之为淮阳王。赤眉将领张卬为绝后患,派人将更始帝缢死。

25年刘秀建立東漢後,為確立自己為光復漢室的正統,劉秀及其后代并不认定刘玄是汉朝皇帝,官修《东观汉记》直接将刘玄归为列传。张衡提出不同意见,认为光武之前,更始踐祚,号令天下,光武是继更始之后登上天子之位的,《光武帝纪》之前,应有《更始帝纪》,不過朝廷并没有采纳张衡的意见

\subsection{更始}

\begin{longtable}{|>{\centering\scriptsize}m{2em}|>{\centering\scriptsize}m{1.3em}|>{\centering}m{8.8em}|}
  % \caption{秦王政}\
  \toprule
  \SimHei \normalsize 年数 & \SimHei \scriptsize 公元 & \SimHei 大事件 \tabularnewline
  % \midrule
  \endfirsthead
  \toprule
  \SimHei \normalsize 年数 & \SimHei \scriptsize 公元 & \SimHei 大事件 \tabularnewline
  \midrule
  \endhead
  \midrule
  元年 & 23 & \tabularnewline\hline
  二年 & 24 & \tabularnewline\hline
  三年 & 25 & \tabularnewline
  \bottomrule
\end{longtable}


%%% Local Variables:
%%% mode: latex
%%% TeX-engine: xetex
%%% TeX-master: "../Main"
%%% End:


%%% Local Variables:
%%% mode: latex
%%% TeX-engine: xetex
%%% TeX-master: "../Main"
%%% End:
