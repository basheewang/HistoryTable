%% -*- coding: utf-8 -*-
%% Time-stamp: <Chen Wang: 2018-07-13 00:59:52>

\documentclass[zihao=-4]{ctexbook}
\ctexset{
  chapter = {
    name = {第,卷},
    format = \centering\Large\bfseries\heiti,
    beforeskip = 10pt,
    afterskip = 20pt,
    titleformat = \chaptertitleformat
  },
  section = {
    name = {第,章},
    number = \chinese{section},
    format = \newpage\large\heiti,
    afterskip = 10pt,
    beforeskip = 10pt,
  },
  subsection = {
    name = {第,节},
    number = \chinese{subsection},
    format = \heiti,
    afterskip = 0pt,
    beforeskip = 0pt,
  },
  subsubsection = {
    % name = {第,},
    % numbering = true,
    number = \chinese{subsubsection},
    format = \heiti,
    afterskip = 0pt,
    beforeskip = 0pt,
  }
}
\setcounter{secnumdepth}{5}

\usepackage{varwidth}
\newcommand{\chaptertitleformat}[1]{%%
  \begin{varwidth}[t]{.7\linewidth}#1\end{varwidth}}

% Some extra packages
\usepackage{listings}
\lstset{
  basicstyle=\ttfamily,
  escapeinside={||},
  mathescape=true
}

\usepackage{fancyvrb}
\newsavebox{\FVerbBox}
\newenvironment{FVerbatim}
{\VerbatimEnvironment
  \begin{center}
\begin{BVerbatim}[commandchars=\\\{\}]}
 {\end{BVerbatim}
   \end{center}}

\usepackage{color}
\usepackage[perpage,hang,flushmargin]{footmisc}

% \usepackage{enotez}

\usepackage[hidelinks]{hyperref}
\hypersetup{
  colorlinks=true,
  linkcolor=blue,
  filecolor=magenta,      
  urlcolor=cyan,
}
% \hypersetup{
%   colorlinks,
%   citecolor=black,
%   filecolor=black,
%   linkcolor=black,
%   urlcolor=black
% }

\usepackage{fancyhdr}
\fancyhf{} % clear all header and footers
% \renewcommand{\headrulewidth}{0pt} % remove the header rule
% \rfoot{\thepage}
% \pagestyle{fancy}
% \fancyhf{}% clearsall
% \fancyhead[RE,LO]{\normalsize foo}
% \rfoot{\thepage}
% \renewcommand{\footrulewidth}{-30pt}

% For Meizu Pro5
\usepackage[
    % showframe,
    % includefoot,
    paperwidth=2.75in,
    paperheight=4.9in,
    left=0.1in,
    right=0.1in,
    top=0.1in,
    bottom=0.18in,
    footskip=10pt
]
{geometry}

% For Kindle 6"
% \usepackage[
% % showframe,
% paperwidth=3.6in,
% paperheight=4.8in,
% left=0.1in,
% right=0.1in,
% top=0.1in,
% bottom=0.18in,
% % footskip=10pt
% ]
% {geometry}

\setCJKmainfont{FZLanTingSong}

\newCJKfontfamily[fzsong]\fzsong{FZLanTingSong}
\newCJKfontfamily[kaiti]\kaiti{KaiTi}
\newCJKfontfamily[hkxm]\hkxm{FZBeiWeiKaiShu-S19_GB18030}
\newCJKfontfamily[song]\pml{PMingLiU}
\newCJKfontfamily[song]\hnm{HanaMin}
\newCJKfontfamily[fzk]\fzk{FZKaiS-Extended(SIP)}
\newCJKfontfamily[HeiTi]\SimHei{SimHei}

\usepackage{booktabs}
\usepackage{array}

\usepackage{tocloft}
\renewcommand{\cftsubsecfont}{\small}

\usepackage{enumitem}
% \setlength{\parindent}{0pt}
\setlist[description]{leftmargin=\parindent,labelindent=\parindent}
\setlist[enumerate]{
  nosep,
  itemsep=0pt,
  parsep=0pt,
  topsep=3pt,
  partopsep=0pt,
  leftmargin=0.4em,
  labelsep=2pt,
  after=\vspace{-3.2em},
  before=\vspace{-1.6em},  
  }
\setlist[itemize]{
    nosep,
    itemsep=0pt,
    parsep=0pt,
    topsep=3pt,
    partopsep=0pt,
    leftmargin=2em,
    % labelsep=2pt,
    % after=\vspace{-3.2em},
    % before=\vspace{-1.6em},  
  }

\usepackage{longtable}

%% 改变页数字体大小
\renewcommand*{\thepage}{\scriptsize\arabic{page}}

%% 改变字体大小
\renewcommand{\footnotesize}{\fontsize{5pt}{6pt}\selectfont}
\newcommand{\threept}{\fontsize{3pt}{3pt}\selectfont}

%% 使每页的脚注的数字为黑色圆圈数字
\usepackage{pifont}
\makeatletter
\newcommand*{\bcircnum}[1]{%
  \expandafter\@bcircnum\csname c@#1\endcsname
}
\newcommand*{\@bcircnum}[1]{%
  \ifnum#1<1 %
  \@ctrerr
  \else
  \ifnum#1>20 %
  \@ctrerr
  \else
  \ding{\the\numexpr 181+(#1)\relax}%
  \fi
  \fi
}
\makeatother

\renewcommand*{\thefootnote}{\bcircnum{footnote}}

% \makeatletter
% \newcommand*{\circnum}[1]{%
%   \expandafter\@circnum\csname c@#1\endcsname
% }
% \newcommand*{\@circnum}[1]{%
%   \ifnum#1<1 %
%   \@ctrerr
%   \else
%   \ifnum#1>10 %
%   \@ctrerr
%   \else
%   \ding{\the\numexpr 171+(#1)\relax}%
%   \fi
%   \fi
% }
% \makeatother

% \renewcommand*{\theenumi}{\circnum{enumi}}

\setlength{\footskip}{0pt} 
\setlength{\footnotesep}{0pt}

% \usepackage{footnote}
% \makesavenoteenv{tabular}
% \makesavenoteenv{table}

%% title & author
\title{\huge 历\quad{}代\quad{}年\quad{}表}
\author{}
\date{}

\begin{document}
\clearpage\maketitle
\clearpage\tableofcontents
% \thispagestyle{empty}

\thispagestyle{empty}

\setcounter{page}{1}

\addtocontents{toc}{\protect\thispagestyle{empty}}

%% -*- coding: utf-8 -*-
%% Time-stamp: <Chen Wang: 2018-07-08 23:41:09>

\chapter{前言}

本书包括历代君王年号。

%%% Local Variables:
%%% mode: latex
%%% TeX-engine: xetex
%%% TeX-master: "../Main"
%%% End: % 前言

% %% -*- coding: utf-8 -*-
%% Time-stamp: <Chen Wang: 2018-07-12 22:49:49>

\chapter{春秋{\tiny(BC770-BC403)}}

%% -*- coding: utf-8 -*-
%% Time-stamp: <Chen Wang: 2018-07-12 22:40:18>

\section{东周}

%% -*- coding: utf-8 -*-
%% Time-stamp: <Chen Wang: 2018-07-12 22:47:18>

\subsection{平王{\tiny(BC772-BC720)}}

% \centering
\begin{longtable}{|>{\centering\scriptsize}m{2em}|>{\centering\scriptsize}m{1.3em}|>{\centering}m{8.8em}|}
  % \caption{秦王政}\\
  \toprule
  \SimHei \normalsize 年数 & \SimHei \scriptsize 公元 & \SimHei 大事件 \tabularnewline
  % \midrule
  \endfirsthead
  \toprule
  \SimHei \normalsize 年数 & \SimHei \scriptsize 公元 & \SimHei 大事件 \tabularnewline
  \midrule
  \endhead
  \midrule
  % 元年 & -770 & \tabularnewline\hline
  % 二年 & -769 & \tabularnewline\hline
  % 三年 & -768 & \tabularnewline\hline
  % 四年 & -767 & \tabularnewline\hline
  % 五年 & -766 & \tabularnewline\hline
  % 六年 & -765 & \tabularnewline\hline
  % 七年 & -764 & \tabularnewline\hline
  % 八年 & -763 & \tabularnewline\hline
  % 九年 & -762 & \tabularnewline\hline
  % 十年 & -761 & \tabularnewline\hline
  % 十一年 & -760 & \tabularnewline\hline
  % 十二年 & -759 & \tabularnewline\hline
  % 十三年 & -758 & \tabularnewline\hline
  % 十四年 & -757 & \tabularnewline\hline
  % 十五年 & -756 & \tabularnewline\hline
  % 十六年 & -755 & \tabularnewline\hline
  % 十七年 & -754 & \tabularnewline\hline
  % 十八年 & -753 & \tabularnewline\hline
  % 十九年 & -752 & \tabularnewline\hline
  % 二十年 & -751 & \tabularnewline\hline
  % 二一年 & -750 & \tabularnewline\hline
  % 二二年 & -749 & \tabularnewline\hline
  % 二三年 & -748 & \tabularnewline\hline
  % 二四年 & -747 & \tabularnewline\hline
  % 二五年 & -746 & \tabularnewline\hline
  % 二六年 & -745 & \tabularnewline\hline
  % 二七年 & -744 & \tabularnewline\hline
  % 二八年 & -743 & \tabularnewline\hline
  % 二九年 & -742 & \tabularnewline\hline
  % 三十年 & -741 & \tabularnewline\hline
  % 三一年 & -740 & \tabularnewline\hline
  % 三二年 & -739 & \tabularnewline\hline
  % 三三年 & -738 & \tabularnewline\hline
  % 三四年 & -737 & \tabularnewline\hline
  % 三五年 & -736 & \tabularnewline\hline
  % 三六年 & -735 & \tabularnewline\hline
  % 三七年 & -734 & \tabularnewline\hline
  % 三八年 & -733 & \tabularnewline\hline
  % 三九年 & -732 & \tabularnewline\hline
  % 四十年 & -731 & \tabularnewline\hline
  % 四一年 & -730 & \tabularnewline\hline
  % 四二年 & -729 & \tabularnewline\hline
  % 四三年 & -728 & \tabularnewline\hline
  % 四四年 & -727 & \tabularnewline\hline
  % 四五年 & -726 & \tabularnewline\hline
  % 四六年 & -725 & \tabularnewline\hline
  % 四七年 & -724 & \tabularnewline\hline
  % 四八年 & -723 & \tabularnewline\hline
  四九年 & -722 & \tabularnewline\hline
  五十年 & -721 & \tabularnewline\hline
  五一年 & -720 & \tabularnewline
  \bottomrule
\end{longtable}

%%% Local Variables:
%%% mode: latex
%%% TeX-engine: xetex
%%% TeX-master: "../../Main"
%%% End:

% %% -*- coding: utf-8 -*-
%% Time-stamp: <Chen Wang: 2018-07-16 22:51:28>

\subsection{元安王{\tiny(BC401-BC376)}}

周安王姬骄(?—前376年),姬姓,名骄,华夏族,周威烈王之子,威烈王死后继位,在位26年,病死。葬处不明。在位时封齐国大夫田和为齐侯,是谓“田氏代齐”。

% \centering
\begin{longtable}{|>{\centering\scriptsize}m{2em}|>{\centering\scriptsize}m{1.3em}|>{\centering}m{8.8em}|}
  % \caption{秦王政}\\
  \toprule
  \SimHei \normalsize 年数 & \SimHei \scriptsize 公元 & \SimHei 大事件 \tabularnewline
  % \midrule
  \endfirsthead
  \toprule
  \SimHei \normalsize 年数 & \SimHei \scriptsize 公元 & \SimHei 大事件 \tabularnewline
  \midrule
  \endhead
  \midrule
  元年 & -401 & \begin{enumerate}
    \tiny
  \item 秦伐魏,至陽孤\footnote{秦国(首府雍县【陕西省凤翔县】)进攻魏国(首府安邑【山西省夏县】),大军进抵阳孤(山西省垣曲县东南)。}。
  \end{enumerate} \tabularnewline\hline
  二年 & -400 & \begin{enumerate}
    \tiny
  \item 魏、韓、趙伐楚,至桑丘\footnote{魏国(首府安邑【山西省夏县】)、韩国(首府平阳【山西省临汾市】)、赵国(首府晋阳【山西省太原市】),联合攻击楚王国(首都郢都【湖北省江陵县】),大军进抵桑丘(《史记》作乘丘【山东省兖州市西北】)。}。
  \item 鄭圍韓陽翟\footnote{郑国(首府新郑【河南省新郑县】)围攻韩国所属的阳翟(河南省禹州市)。}。
  \item 韓景侯薨,子烈侯取立。
  \item 趙烈侯薨,國人立其弟武侯。
  \item 秦簡公薨,子惠\footnote{«諡法»︰愛民好與曰惠。}公立。
  \end{enumerate} \tabularnewline\hline
  三年 & -399 & \begin{enumerate}
    \tiny
  \item 王子定奔晉。
  \item 虢山崩,壅河\footnote{虢山(河南省三门峡市西)发生崩塌,土石坠入黄河,河水壅塞。}。
  \end{enumerate} \tabularnewline\hline
  四年 & -398 & \begin{enumerate}
    \tiny
  \item 楚圍鄭。鄭人殺其相駟子陽\footnote{郑国十一任国君穆公姬兰的儿子姬腓,别名子驷。古人往往用祖父的名字最后一个字作自己这一支派的姓。这位驷子阳,姓驷,名子阳,也是郑国贵族。}。
  \end{enumerate} \tabularnewline\hline
  五年 & -397 & \begin{enumerate}
    \tiny
  \item 日有食之。
  \item 三月,盜殺韓相俠累\footnote{侠累跟濮阳(河南省濮阳市)人严仲子之间,有难解的怨毒,严仲子听说轵邑(河南省济源市东南)人聂政,勇猛过人,备了黄金二千四百两(百镒),送给聂政的母亲,作为祝寿礼物,请聂政代他报仇。聂政拒绝,说:“娘亲在堂,要我奉养,我不能轻言牺牲。”稍后,娘亲逝世,聂政才接受这项委托。当暗杀行动开始时,侠累正在宰相府主持会报,警卫森严。聂政像闪电一样,突击而入,在众人惊愕中,举刀直刺侠累的咽喉,侠累立即死亡。聂政自知难以逃生,咬紧牙关,用利刃自行毁容,脸皮全被割破,又自挖双眼,再自刺腹部自杀,肠出满地。韩国政府把尸首拖到市场,公开示众,要求市人辨识刺客身份。聂政的姐姐聂荌听到消息,赶到首府平阳(山西省临汾市),抚尸哀哭说:“他就是轵邑深井里(济通市东南十五千米)的聂政,只因为我这个姐姐尚在人间,恐怕连累我,才忍心重重的自我毁灭。弟弟啊,我怎么会贪生怕死,使你埋没英名?”就在尸旁,自杀殉难。}。
  \end{enumerate} \tabularnewline\hline
  六年 & -396 & \begin{enumerate}
    \tiny
  \item 鄭駟子陽之黨弑繻公\footnote{繻者,«諡法»所不載。},而立其弟乙,是爲康公\footnote{郑国(首府新郑【河南省新郑县】)故宰相(相)驷子阳的残余党羽,击杀国君(二十七任)繻公姬贻,拥立他的弟弟姬乙继位(二十八任),是为康公。}。
  \item 宋悼公薨,子休公田立\footnote{宋国(首府睢阳【河南省商丘县】)国君(三十一任悼公)宋购由逝世,子宋田继位(三十二任),是为休公。武王封微子啓於宋,唐宋州之睢陽縣是也。自微子二十七世至悼公,名購由。休,亦«諡法»所不載。}。
  \end{enumerate} \tabularnewline\hline
  七年 & -395 & \tiny \kaiti 无记载 \tabularnewline\hline
  八年 & -394 & \begin{enumerate}
    \tiny
  \item 齊\footnote{武王封太公於齊,唐青州之臨淄是也。«括地志»曰︰天齊水在臨淄東南十五里。«封禪書»曰︰齊之所以爲齊者,以天齊。是年,康公貸之十一年。自太公至康公二十九世。}伐魯\footnote{成王封伯禽於魯,唐兗州之曲阜是也。是年,穆公之十六年。自伯禽至穆公凡二十八世。},取最\footnote{山东省曲阜市东南}。
  \item 鄭負黍\footnote{負黍山在陽城縣西南二十七里,或云在西南三十五里。}叛,復歸韓\footnote{前四〇七年,郑国攻击韩国,占领负黍城。}。
  \end{enumerate} \tabularnewline\hline
  九年 & -393 & \begin{enumerate}
    \tiny
  \item 魏伐鄭。
  \item 晉烈公\footnote{周成王封弟叔虞於唐。«括地志»曰︰故唐城在幷州晉陽縣北二里,堯所築也。«都城記»曰︰唐叔虞之子燮父徙居晉水旁,今幷州理故唐城,卽燮父初徙之處;其城南半入州城中。«毛詩譜»曰︰燮父以堯墟南有晉水,改曰晉侯。自唐叔至烈公三十七世。烈公,名止。«諡法»︰慈惠愛親曰孝。}薨,子孝公傾立。
  \end{enumerate} \tabularnewline\hline
  十年 & -392 & \tiny \kaiti 无记载\tabularnewline\hline
  十一年 & -391 & \begin{enumerate}
    \tiny
  \item 秦伐韓宜陽,取六邑\footnote{班«志»,宜陽縣屬弘農郡。«史記正義»曰︰宜陽縣故城,在河南府福昌縣東十四里,故韓城是也。此邑卽«周禮»「四井爲邑」之邑。}。
  \item 齊田和\footnote{田常生襄子盤,盤生莊子白,白生太公和。此序齊田氏之世也。田常,卽«左傳»陳成子恆也。溫公避仁廟諱,改「恆」曰「常」。自陳公子完奔齊,五世至常得政。«諡法»︰勝敵志強曰莊。}遷齊康公於海上,使食一城,以奉其先祀。
  \end{enumerate} \tabularnewline\hline
  十二年 & -390 & \begin{enumerate}
    \tiny
  \item 秦、晉戰于武城\footnote{晋国【首府新田】自被瓜分后,连本身生存都有问题,已无力作任何战争。可能是和魏国【首府安邑·山西省夏县】,或韩国【首府平阳·山西省临汾市】会战。}。
  \item 齊伐魏,取襄陽。
  \item 魯敗齊師于平陸。
  \end{enumerate} \tabularnewline\hline
  十三年 & -389 & \begin{enumerate}
    \tiny
  \item 秦侵晉。
  \item 齊田和會\footnote{孔穎達曰︰諸侯未及期而相見曰遇。會者,謂及期之禮,旣及期,又至所期之地。}魏文侯、楚人、衞人于濁澤,求爲諸侯。魏文侯爲之請於王及諸侯,王許之。
  \end{enumerate} \tabularnewline\hline
  十四年 & -388 & \begin{enumerate}
    \tiny
  \item 齊田和逝世,子田剡继位。
  \end{enumerate} \tabularnewline\hline
  十五年 & -387 & \begin{enumerate}
    \tiny
  \item 秦伐蜀\footnote{«譜記»普[疑衍]云︰蜀之先,肇自人皇之際。黃帝子昌意娶蜀山氏女,生帝俈。旣立,封其支庶於蜀,歷虞、夏、商、周。周衰,先稱王者蠶叢。余據武王伐紂,庸、蜀諸國皆會于牧野。孔安國曰︰蜀,叟也,春秋之時不與中國通。班«志»,南鄭縣屬漢中郡,唐爲梁州治所。},取南鄭。
  \item 魏文侯薨,太子擊立,是爲武侯。魏置相,相田文\footnote{魏击任命田文担任宰相。吴起不高兴,对田文说:“我想跟你讨论一下你我对于国家的贡献,你以为如何?”田文说:“当然可以。”吴起说:“指挥武装部队,官兵们愿意牺牲性命,使敌国惊惧,不敢打我们的主意,你比我怎么样?”田文说:“我不如你。”吴起说:“使政府的功能充分发挥,使全国人民安居乐业、国库充实、社会富庶,你比我怎么样?”田文说:“我不如你。”吴起说:“防卫西河(潼关以北的黄河),秦国不敢向东侵略。而韩国(首府平阳【山西省临汾市】)与赵国(首府晋阳【山西省太原市】),不敢不对我们唯命是听,你比我怎么样?”田文说:“我不如你。”吴起说:“这三项重要大事,你都不如我,可是官位却比我高,那为什么?”田文说:“当君王年纪还小,有权势的重要官员互相猜忌,随时可能发动政变,民心恐慌。这个时候,宰相位置,应该属于你?还是属于我?”吴起沉默很久,抱歉说:“我承认,应该属于你。”}。
  \item 秦惠公薨,子出公\footnote{出,非諡也;以其失國出死,故曰出公。}立。
  \item 趙武侯薨,國人復立烈侯之太子章,是爲敬侯\footnote{«諡法»︰夙夜警戒曰敬。}。
  \item 韓烈侯薨,子文侯立。
  \end{enumerate} \tabularnewline\hline
  十六年 & -386 & \begin{enumerate}
    \tiny
  \item 趙公子朝作亂,奔魏;與魏襲邯鄲,不克\footnote{本年【前三八六年】,赵国首府自晋阳迁邯郸,赵朝当是利用迁府之际,发动政变。}。
  \end{enumerate} \tabularnewline\hline
  十七年 & -385 & \begin{enumerate}
    \tiny
  \item 秦庶長\footnote{後秦制爵,一級曰公士,二上造,三簪裊,四不更,五大夫,六官大夫,七公大夫,八公乘,九五大夫,十左庶長,十一右庶長,十二左更,十三中更,十四右更,十五少上造,十六大上造,十七駟車庶長,十八大庶長,十九關內侯,二十徹侯。師古曰︰庶長,言衆列之長。}改逆獻公\footnote{威烈王十一年秦靈公卒,子獻公師隰不得立,立靈公季父悼子,是爲簡公。出子,簡公之孫也。今庶長改迎獻公而殺出子。}于河西而立之;殺出子及其母,沈之淵旁。
  \item 齐伐魯。
  \item 韓伐鄭,取陽城;伐宋,執宋公。
  \end{enumerate} \tabularnewline\hline
  十八年 & -384 & \tiny \kaiti 无记载 \tabularnewline\hline
  十九年 & -383 & \begin{enumerate}
    \tiny
  \item 魏敗趙師于兔臺。
  \end{enumerate} \tabularnewline\hline
  二十年 & -382 & \begin{enumerate}
    \tiny
  \item 日有食之,旣\footnote{旣,盡也}。
  \end{enumerate} \tabularnewline\hline
  二一年 & -381 & \begin{enumerate}
    \tiny
  \item 楚悼王薨。貴戚大臣作亂,攻吳起;起走之王尸而伏之。擊起之徒因射刺起,並中王尸。旣葬,肅\footnote{«諡法»︰剛德克就曰肅;執心決斷曰肅。}王卽位,使令尹盡誅爲亂者;坐起夷宗者七十餘家。
  \end{enumerate} \tabularnewline\hline
  二二年 & -380 & \begin{enumerate}
    \tiny
  \item 齊伐燕,取桑丘。
  \item 魏、韓、趙伐齊,至桑丘。
  \end{enumerate} \tabularnewline\hline
  二三年 & -379 & \begin{enumerate}
    \tiny
  \item 趙襲衞\footnote{成王封康叔於衞,居河、淇之間,故殷墟也。至懿公爲狄所滅,東徙度河。文公徙居楚丘,遂國於濮陽。是年,愼公頹之三十五年。自康叔至愼公凡三十二世。},不克。
  \item 齊康公薨,無子,田氏遂幷齊而有之。姜氏至此滅矣。
  \end{enumerate} \tabularnewline\hline
  二四年 & -378 & \begin{enumerate}
    \tiny
  \item 狄\footnote{漢之中山、上黨、西河、上郡,自春秋以來,狄皆居之,此亦其種也。«水經»︰澮水出河東絳縣東澮山,西過絳縣南,又西南過虒祁宮南,又西南至王橋,入汾水。«括地志»︰澮山在絳州翼城縣東北。}敗魏師于澮。
  \item 魏、韓、趙伐齊,至靈丘。
  \item 晉孝公薨,子靖公\footnote{«諡法»︰柔衆安民曰靖;又,恭己鮮言曰靖。}俱酒立。
  \item 齐国(首府临淄)国君(二任)田剡逝世,子田午继位(三任),是为桓公。
  \end{enumerate} \tabularnewline\hline
  二五年 & -377 & \begin{enumerate}
    \tiny
  \item 蜀伐楚,取茲方(四川省奉节县)。
  \item 子思论卫\footnote{卫国(首府濮阳【河南省濮阳市】),孔伋(子思)向卫国国君(四十一任慎公)卫颓,推荐苟变,说:“他的才干可以指挥五百辆战车作战。”卫颓说:“我知道他的军事才能,但苟变曾经当过税务员,有次平白吃了民家两个鸡蛋,品德上有瑕疵。”孔伋说:“政府任用官吏,跟建筑师选择木材一样,取其所长,弃其所短。巨木高耸云际,几个人都合抱不住,却有几尺朽烂,优秀的建筑师不会不用它。现在,我们正处在大混战时代,应该积极物色英雄豪杰,却为了两个鸡蛋,丧失一员大将,这话可别让别国听见才好。”卫颓再三致谢说:“我接受你的指教。”卫颓做了一项错误的决定,全体官员却一致赞扬那决定非常正确。孔伋对公丘懿子说:“我看你们卫国,真是君不像君,臣不像臣。”(“君不君,臣不臣”,《论语》引齐国【首府临淄·山东省淄博市东临淄镇】国君【二十六任景公】姜杵臼的话。)公丘懿子说:“怎么会糟到这种程度?”孔伋说:“领袖人物经常的自以为是,大家就不敢贡献自己的意见。做对了而自以为是,还会排斥众人的智慧。何况做错了而仍自以为是,硬教大家赞扬,那简直是鼓励邪恶。不问事情的是非,而只一味喜欢听悦耳的声音,可以说绝顶糊涂。不管那是不是合理,而只努力露出忠贞嘴脸,满口顺调,那就是马屁精。君主昏庸、官员谄媚,而高高坐在人民头上,人民绝对不会认同。如果一直这样下去,国家必亡。”孔伋告诉卫颓说:“你的国家,恐怕将要没落了。”卫颓说:“什么原因?”孔伋说:“当然有原因,领袖说一句话,自以为是,官员们没有一个人敢指出他的错误;官员们说一句话,自以为是,民间没有一个人敢指出他的错误。领袖和官员,都自以为英明盖世,属下的小官小民也同声赞扬他们果然是真的英明盖世。马屁精就有福了,指出君王错误的人一定大祸临久。如此这般,有益于国家的善政,怎能产生?《诗经》说:‘都说自己是圣贤,谁分辨乌鸦的雌雄?’听起来好像就是指的你们。”}。
  \item 魯穆公薨,子共公奮立\footnote{«諡法»︰布德就義曰穆;中情見貌曰穆;尊賢敬讓曰共;旣過能改曰共;執事堅固曰共。}。
  \item 韓文侯薨,子哀侯立。
  \end{enumerate} \tabularnewline\hline
  二六年 & -376 & \begin{enumerate}
    \tiny
  \item 王崩,子烈王喜立。
  \item 魏、韓、趙共廢晉靖公爲家人而分其地。唐叔不祀矣。
  \end{enumerate} \tabularnewline
  \bottomrule
\end{longtable}

%%% Local Variables:
%%% mode: latex
%%% TeX-engine: xetex
%%% TeX-master: "../../Main"
%%% End:

% %% -*- coding: utf-8 -*-
%% Time-stamp: <Chen Wang: 2018-07-10 17:30:09>

\subsection{烈王{\tiny(BC375-BC369)}}


% \centering
\begin{longtable}{|>{\centering\scriptsize}m{2em}|>{\centering\scriptsize}m{1.3em}|>{\centering}m{8.8em}|}
  % \caption{秦王政}\\
  \toprule
  \SimHei \normalsize 年数 & \SimHei \scriptsize 公元 & \SimHei 大事件 \tabularnewline
  % \midrule
  \endfirsthead
  \toprule
  \SimHei \normalsize 年数 & \SimHei \scriptsize 公元 & \SimHei 大事件 \tabularnewline
  \midrule
  \endhead
  \midrule
  元年 & -375 & \tabularnewline\hline
  二年 & -374 & \tabularnewline\hline
  三年 & -373 & \tabularnewline\hline
  四年 & -372 & \tabularnewline\hline
  五年 & -371 & \tabularnewline\hline
  六年 & -370 & \tabularnewline\hline
  七年 & -369 & \tabularnewline
  \bottomrule
\end{longtable}

%%% Local Variables:
%%% mode: latex
%%% TeX-engine: xetex
%%% TeX-master: "../../Main"
%%% End:

% %% -*- coding: utf-8 -*-
%% Time-stamp: <Chen Wang: 2018-07-10 17:30:31>

\subsection{显王{\tiny(BC368-BC321)}}


% \centering
\begin{longtable}{|>{\centering\scriptsize}m{2em}|>{\centering\scriptsize}m{1.3em}|>{\centering}m{8.8em}|}
  % \caption{秦王政}\\
  \toprule
  \SimHei \normalsize 年数 & \SimHei \scriptsize 公元 & \SimHei 大事件 \tabularnewline
  % \midrule
  \endfirsthead
  \toprule
  \SimHei \normalsize 年数 & \SimHei \scriptsize 公元 & \SimHei 大事件 \tabularnewline
  \midrule
  \endhead
  \midrule
  元年 & -368 & \tabularnewline\hline
  二年 & -367 & \tabularnewline\hline
  三年 & -366 & \tabularnewline\hline
  四年 & -365 & \tabularnewline\hline
  五年 & -364 & \tabularnewline\hline
  六年 & -363 & \tabularnewline\hline
  七年 & -362 & \tabularnewline\hline
  八年 & -361 & \tabularnewline\hline
  九年 & -360 & \tabularnewline\hline
  十年 & -359 & \tabularnewline\hline
  十一年 & -358 & \tabularnewline\hline
  十二年 & -357 & \tabularnewline\hline
  十三年 & -356 & \tabularnewline\hline
  十四年 & -355 & \tabularnewline\hline
  十五年 & -354 & \tabularnewline\hline
  十六年 & -353 & \tabularnewline\hline
  十七年 & -352 & \tabularnewline\hline
  十八年 & -351 & \tabularnewline\hline
  十九年 & -350 & \tabularnewline\hline
  二十年 & -349 & \tabularnewline\hline
  二一年 & -348 & \tabularnewline\hline
  二二年 & -347 & \tabularnewline\hline
  二三年 & -346 & \tabularnewline\hline
  二四年 & -345 & \tabularnewline\hline
  二五年 & -344 & \tabularnewline\hline
  二六年 & -343 & \tabularnewline\hline
  二七年 & -342 & \tabularnewline\hline
  二八年 & -341 & \tabularnewline\hline
  二九年 & -340 & \tabularnewline\hline
  三十年 & -339 & \tabularnewline\hline
  三一年 & -338 & \tabularnewline\hline
  三二年 & -337 & \tabularnewline\hline
  三三年 & -336 & \tabularnewline\hline
  三四年 & -335 & \tabularnewline\hline
  三五年 & -334 & \tabularnewline\hline
  三六年 & -333 & \tabularnewline\hline
  三七年 & -332 & \tabularnewline\hline
  三八年 & -331 & \tabularnewline\hline
  三九年 & -330 & \tabularnewline\hline
  四十年 & -329 & \tabularnewline\hline
  四一年 & -328 & \tabularnewline\hline
  四二年 & -327 & \tabularnewline\hline
  四三年 & -326 & \tabularnewline\hline
  四四年 & -325 & \tabularnewline\hline
  四五年 & -324 & \tabularnewline\hline
  四六年 & -323 & \tabularnewline\hline
  四七年 & -322 & \tabularnewline\hline
  四八年 & -321 & \tabularnewline
  \bottomrule
\end{longtable}

%%% Local Variables:
%%% mode: latex
%%% TeX-engine: xetex
%%% TeX-master: "../../Main"
%%% End:

% %% -*- coding: utf-8 -*-
%% Time-stamp: <Chen Wang: 2018-07-10 17:30:19>

\subsection{慎靓王{\tiny(BC320-BC315)}}


% \centering
\begin{longtable}{|>{\centering\scriptsize}m{2em}|>{\centering\scriptsize}m{1.3em}|>{\centering}m{8.8em}|}
  % \caption{秦王政}\\
  \toprule
  \SimHei \normalsize 年数 & \SimHei \scriptsize 公元 & \SimHei 大事件 \tabularnewline
  % \midrule
  \endfirsthead
  \toprule
  \SimHei \normalsize 年数 & \SimHei \scriptsize 公元 & \SimHei 大事件 \tabularnewline
  \midrule
  \endhead
  \midrule
  元年 & -320 & \tabularnewline\hline
  二年 & -319 & \tabularnewline\hline
  三年 & -318 & \tabularnewline\hline
  四年 & -317 & \tabularnewline\hline
  五年 & -316 & \tabularnewline\hline
  六年 & -315 & \tabularnewline
  \bottomrule
\end{longtable}

%%% Local Variables:
%%% mode: latex
%%% TeX-engine: xetex
%%% TeX-master: "../../Main"
%%% End:

% %% -*- coding: utf-8 -*-
%% Time-stamp: <Chen Wang: 2018-07-10 17:30:15>

\subsection{赧王{\tiny(BC314-BC256)}}


% \centering
\begin{longtable}{|>{\centering\scriptsize}m{2em}|>{\centering\scriptsize}m{1.3em}|>{\centering}m{8.8em}|}
  % \caption{秦王政}\\
  \toprule
  \SimHei \normalsize 年数 & \SimHei \scriptsize 公元 & \SimHei 大事件 \tabularnewline
  % \midrule
  \endfirsthead
  \toprule
  \SimHei \normalsize 年数 & \SimHei \scriptsize 公元 & \SimHei 大事件 \tabularnewline
  \midrule
  \endhead
  \midrule
  元年 & -314 & \tabularnewline\hline
  二年 & -313 & \tabularnewline\hline
  三年 & -312 & \tabularnewline\hline
  四年 & -311 & \tabularnewline\hline
  五年 & -310 & \tabularnewline\hline
  六年 & -309 & \tabularnewline\hline
  七年 & -308 & \tabularnewline\hline
  八年 & -307 & \tabularnewline\hline
  九年 & -306 & \tabularnewline\hline
  十年 & -305 & \tabularnewline\hline
  十一年 & -304 & \tabularnewline\hline
  十二年 & -303 & \tabularnewline\hline
  十三年 & -302 & \tabularnewline\hline
  十四年 & -301 & \tabularnewline\hline
  十五年 & -300 & \tabularnewline\hline
  十六年 & -299 & \tabularnewline\hline
  十七年 & -298 & \tabularnewline\hline
  十八年 & -297 & \tabularnewline\hline
  十九年 & -296 & \tabularnewline\hline
  二十年 & -295 & \tabularnewline\hline
  二一年 & -294 & \tabularnewline\hline
  二二年 & -293 & \tabularnewline\hline
  二三年 & -292 & \tabularnewline\hline
  二四年 & -291 & \tabularnewline\hline
  二五年 & -290 & \tabularnewline\hline
  二六年 & -289 & \tabularnewline\hline
  二七年 & -288 & \tabularnewline\hline
  二八年 & -287 & \tabularnewline\hline
  二九年 & -286 & \tabularnewline\hline
  三十年 & -285 & \tabularnewline\hline
  三一年 & -284 & \tabularnewline\hline
  三二年 & -283 & \tabularnewline\hline
  三三年 & -282 & \tabularnewline\hline
  三四年 & -281 & \tabularnewline\hline
  三五年 & -280 & \tabularnewline\hline
  三六年 & -279 & \tabularnewline\hline
  三七年 & -278 & \tabularnewline\hline
  三八年 & -277 & \tabularnewline\hline
  三九年 & -276 & \tabularnewline\hline
  四十年 & -275 & \tabularnewline\hline
  四一年 & -274 & \tabularnewline\hline
  四二年 & -273 & \tabularnewline\hline
  四三年 & -272 & \tabularnewline\hline
  四四年 & -271 & \tabularnewline\hline
  四五年 & -270 & \tabularnewline\hline
  四六年 & -269 & \tabularnewline\hline
  四七年 & -268 & \tabularnewline\hline
  四八年 & -267 & \tabularnewline\hline
  四九年 & -266 & \tabularnewline\hline
  五十年 & -265 & \tabularnewline\hline
  五一年 & -264 & \tabularnewline\hline
  五二年 & -263 & \tabularnewline\hline
  五三年 & -262 & \tabularnewline\hline
  五四年 & -261 & \tabularnewline\hline
  五五年 & -260 & \tabularnewline\hline
  五六年 & -259 & \tabularnewline\hline
  五七年 & -258 & \tabularnewline\hline
  五八年 & -257 & \tabularnewline\hline
  五九年 & -256 & \tabularnewline
  \bottomrule
\end{longtable}

%%% Local Variables:
%%% mode: latex
%%% TeX-engine: xetex
%%% TeX-master: "../../Main"
%%% End:


%%% Local Variables:
%%% mode: latex
%%% TeX-engine: xetex
%%% TeX-master: "../../Main"
%%% End:

%% -*- coding: utf-8 -*-
%% Time-stamp: <Chen Wang: 2018-07-12 23:33:00>

\section{鲁}

%% -*- coding: utf-8 -*-
%% Time-stamp: <Chen Wang: 2018-07-12 22:50:43>

\subsection{隐公{\tiny(BC722-BC712)}}

% \centering
\begin{longtable}{|>{\centering\scriptsize}m{2em}|>{\centering\scriptsize}m{1.3em}|>{\centering}m{8.8em}|}
  % \caption{秦王政}\\
  \toprule
  \SimHei \normalsize 年数 & \SimHei \scriptsize 公元 & \SimHei 大事件 \tabularnewline
  % \midrule
  \endfirsthead
  \toprule
  \SimHei \normalsize 年数 & \SimHei \scriptsize 公元 & \SimHei 大事件 \tabularnewline
  \midrule
  \endhead
  \midrule
  元年 & -722 & \tabularnewline\hline
  二年 & -721 & \tabularnewline\hline
  三年 & -720 & \tabularnewline\hline
  四年 & -719 & \tabularnewline\hline
  五年 & -718 & \tabularnewline\hline
  六年 & -717 & \tabularnewline\hline
  七年 & -716 & \tabularnewline\hline
  八年 & -715 & \tabularnewline\hline
  九年 & -714 & \tabularnewline\hline
  十年 & -713 & \tabularnewline\hline
  十一年 & -712 & \tabularnewline
  \bottomrule
\end{longtable}

%%% Local Variables:
%%% mode: latex
%%% TeX-engine: xetex
%%% TeX-master: "../../Main"
%%% End:

%% -*- coding: utf-8 -*-
%% Time-stamp: <Chen Wang: 2018-07-12 22:55:13>

\subsection{桓公{\tiny(BC711-BC694)}}

% \centering
\begin{longtable}{|>{\centering\scriptsize}m{2em}|>{\centering\scriptsize}m{1.3em}|>{\centering}m{8.8em}|}
  % \caption{秦王政}\\
  \toprule
  \SimHei \normalsize 年数 & \SimHei \scriptsize 公元 & \SimHei 大事件 \tabularnewline
  % \midrule
  \endfirsthead
  \toprule
  \SimHei \normalsize 年数 & \SimHei \scriptsize 公元 & \SimHei 大事件 \tabularnewline
  \midrule
  \endhead
  \midrule
  元年 & -711 & \tabularnewline\hline
  二年 & -710 & \tabularnewline\hline
  三年 & -709 & \tabularnewline\hline
  四年 & -708 & \tabularnewline\hline
  五年 & -707 & \tabularnewline\hline
  六年 & -706 & \tabularnewline\hline
  七年 & -705 & \tabularnewline\hline
  八年 & -704 & \tabularnewline\hline
  九年 & -703 & \tabularnewline\hline
  十年 & -702 & \tabularnewline\hline
  十一年 & -701 & \tabularnewline\hline
  十二年 & -700 & \tabularnewline\hline
  十三年 & -699 & \tabularnewline\hline
  十四年 & -698 & \tabularnewline\hline
  十五年 & -697 & \tabularnewline\hline
  十六年 & -696 & \tabularnewline\hline
  十七年 & -695 & \tabularnewline\hline
  十八年 & -694 & \tabularnewline
  \bottomrule
\end{longtable}

%%% Local Variables:
%%% mode: latex
%%% TeX-engine: xetex
%%% TeX-master: "../../Main"
%%% End:

%% -*- coding: utf-8 -*-
%% Time-stamp: <Chen Wang: 2018-07-12 23:02:54>

\subsection{庄公{\tiny(BC693-BC662)}}

% \centering
\begin{longtable}{|>{\centering\scriptsize}m{2em}|>{\centering\scriptsize}m{1.3em}|>{\centering}m{8.8em}|}
  % \caption{秦王政}\\
  \toprule
  \SimHei \normalsize 年数 & \SimHei \scriptsize 公元 & \SimHei 大事件 \tabularnewline
  % \midrule
  \endfirsthead
  \toprule
  \SimHei \normalsize 年数 & \SimHei \scriptsize 公元 & \SimHei 大事件 \tabularnewline
  \midrule
  \endhead
  \midrule
  元年 & -693 & \tabularnewline\hline
  二年 & -692 & \tabularnewline\hline
  三年 & -691 & \tabularnewline\hline
  四年 & -690 & \tabularnewline\hline
  五年 & -689 & \tabularnewline\hline
  六年 & -688 & \tabularnewline\hline
  七年 & -687 & \tabularnewline\hline
  八年 & -686 & \tabularnewline\hline
  九年 & -685 & \tabularnewline\hline
  十年 & -684 & \tabularnewline\hline
  十一年 & -683 & \tabularnewline\hline
  十二年 & -682 & \tabularnewline\hline
  十三年 & -681 & \tabularnewline\hline
  十四年 & -680 & \tabularnewline\hline
  十五年 & -679 & \tabularnewline\hline
  十六年 & -678 & \tabularnewline\hline
  十七年 & -677 & \tabularnewline\hline
  十八年 & -676 & \tabularnewline\hline
  十九年 & -675 & \tabularnewline\hline
  二十年 & -674 & \tabularnewline\hline
  二一年 & -673 & \tabularnewline\hline
  二二年 & -672 & \tabularnewline\hline
  二三年 & -671 & \tabularnewline\hline
  二四年 & -670 & \tabularnewline\hline
  二五年 & -669 & \tabularnewline\hline
  二六年 & -668 & \tabularnewline\hline
  二七年 & -667 & \tabularnewline\hline
  二八年 & -666 & \tabularnewline\hline
  二九年 & -665 & \tabularnewline\hline
  三十年 & -664 & \tabularnewline\hline
  三一年 & -663 & \tabularnewline\hline
  三二年 & -662 & \tabularnewline
  \bottomrule
\end{longtable}

%%% Local Variables:
%%% mode: latex
%%% TeX-engine: xetex
%%% TeX-master: "../../Main"
%%% End:

%% -*- coding: utf-8 -*-
%% Time-stamp: <Chen Wang: 2018-07-12 23:06:30>

\subsection{闵公{\tiny(BC661-BC660)}}

% \centering
\begin{longtable}{|>{\centering\scriptsize}m{2em}|>{\centering\scriptsize}m{1.3em}|>{\centering}m{8.8em}|}
  % \caption{秦王政}\\
  \toprule
  \SimHei \normalsize 年数 & \SimHei \scriptsize 公元 & \SimHei 大事件 \tabularnewline
  % \midrule
  \endfirsthead
  \toprule
  \SimHei \normalsize 年数 & \SimHei \scriptsize 公元 & \SimHei 大事件 \tabularnewline
  \midrule
  \endhead
  \midrule
  元年 & -661 & \tabularnewline\hline
  二年 & -660 & \tabularnewline
  \bottomrule
\end{longtable}

%%% Local Variables:
%%% mode: latex
%%% TeX-engine: xetex
%%% TeX-master: "../../Main"
%%% End:

%% -*- coding: utf-8 -*-
%% Time-stamp: <Chen Wang: 2018-07-12 23:08:29>

\subsection{僖公{\tiny(BC659-BC627)}}

% \centering
\begin{longtable}{|>{\centering\scriptsize}m{2em}|>{\centering\scriptsize}m{1.3em}|>{\centering}m{8.8em}|}
  % \caption{秦王政}\\
  \toprule
  \SimHei \normalsize 年数 & \SimHei \scriptsize 公元 & \SimHei 大事件 \tabularnewline
  % \midrule
  \endfirsthead
  \toprule
  \SimHei \normalsize 年数 & \SimHei \scriptsize 公元 & \SimHei 大事件 \tabularnewline
  \midrule
  \endhead
  \midrule
  元年 & -659 & \tabularnewline\hline
  二年 & -658 & \tabularnewline\hline
  三年 & -657 & \tabularnewline\hline
  四年 & -656 & \tabularnewline\hline
  五年 & -655 & \tabularnewline\hline
  六年 & -654 & \tabularnewline\hline
  七年 & -653 & \tabularnewline\hline
  八年 & -652 & \tabularnewline\hline
  九年 & -651 & \tabularnewline\hline
  十年 & -650 & \tabularnewline\hline
  十一年 & -649 & \tabularnewline\hline
  十二年 & -648 & \tabularnewline\hline
  十三年 & -647 & \tabularnewline\hline
  十四年 & -646 & \tabularnewline\hline
  十五年 & -645 & \tabularnewline\hline
  十六年 & -644 & \tabularnewline\hline
  十七年 & -643 & \tabularnewline\hline
  十八年 & -642 & \tabularnewline\hline
  十九年 & -641 & \tabularnewline\hline
  二十年 & -640 & \tabularnewline\hline
  二一年 & -639 & \tabularnewline\hline
  二二年 & -638 & \tabularnewline\hline
  二三年 & -637 & \tabularnewline\hline
  二四年 & -636 & \tabularnewline\hline
  二五年 & -635 & \tabularnewline\hline
  二六年 & -634 & \tabularnewline\hline
  二七年 & -633 & \tabularnewline\hline
  二八年 & -632 & \tabularnewline\hline
  二九年 & -631 & \tabularnewline\hline
  三十年 & -630 & \tabularnewline\hline
  三一年 & -629 & \tabularnewline\hline
  三二年 & -628 & \tabularnewline\hline
  三三年 & -627 & \tabularnewline
  \bottomrule
\end{longtable}

%%% Local Variables:
%%% mode: latex
%%% TeX-engine: xetex
%%% TeX-master: "../../Main"
%%% End:

%% -*- coding: utf-8 -*-
%% Time-stamp: <Chen Wang: 2018-07-12 23:10:00>

\subsection{文公{\tiny(BC626-BC609)}}

% \centering
\begin{longtable}{|>{\centering\scriptsize}m{2em}|>{\centering\scriptsize}m{1.3em}|>{\centering}m{8.8em}|}
  % \caption{秦王政}\\
  \toprule
  \SimHei \normalsize 年数 & \SimHei \scriptsize 公元 & \SimHei 大事件 \tabularnewline
  % \midrule
  \endfirsthead
  \toprule
  \SimHei \normalsize 年数 & \SimHei \scriptsize 公元 & \SimHei 大事件 \tabularnewline
  \midrule
  \endhead
  \midrule
  元年 & -626 & \tabularnewline\hline
  二年 & -625 & \tabularnewline\hline
  三年 & -624 & \tabularnewline\hline
  四年 & -623 & \tabularnewline\hline
  五年 & -622 & \tabularnewline\hline
  六年 & -621 & \tabularnewline\hline
  七年 & -620 & \tabularnewline\hline
  八年 & -619 & \tabularnewline\hline
  九年 & -618 & \tabularnewline\hline
  十年 & -617 & \tabularnewline\hline
  十一年 & -616 & \tabularnewline\hline
  十二年 & -615 & \tabularnewline\hline
  十三年 & -614 & \tabularnewline\hline
  十四年 & -613 & \tabularnewline\hline
  十五年 & -612 & \tabularnewline\hline
  十六年 & -611 & \tabularnewline\hline
  十七年 & -610 & \tabularnewline\hline
  十八年 & -609 & \tabularnewline
  \bottomrule
\end{longtable}

%%% Local Variables:
%%% mode: latex
%%% TeX-engine: xetex
%%% TeX-master: "../../Main"
%%% End:

%% -*- coding: utf-8 -*-
%% Time-stamp: <Chen Wang: 2018-07-12 23:11:20>

\subsection{宣公{\tiny(BC608-BC591)}}

% \centering
\begin{longtable}{|>{\centering\scriptsize}m{2em}|>{\centering\scriptsize}m{1.3em}|>{\centering}m{8.8em}|}
  % \caption{秦王政}\\
  \toprule
  \SimHei \normalsize 年数 & \SimHei \scriptsize 公元 & \SimHei 大事件 \tabularnewline
  % \midrule
  \endfirsthead
  \toprule
  \SimHei \normalsize 年数 & \SimHei \scriptsize 公元 & \SimHei 大事件 \tabularnewline
  \midrule
  \endhead
  \midrule
  元年 & -608 & \tabularnewline\hline
  二年 & -607 & \tabularnewline\hline
  三年 & -606 & \tabularnewline\hline
  四年 & -605 & \tabularnewline\hline
  五年 & -604 & \tabularnewline\hline
  六年 & -603 & \tabularnewline\hline
  七年 & -602 & \tabularnewline\hline
  八年 & -601 & \tabularnewline\hline
  九年 & -600 & \tabularnewline\hline
  十年 & -599 & \tabularnewline\hline
  十一年 & -598 & \tabularnewline\hline
  十二年 & -597 & \tabularnewline\hline
  十三年 & -596 & \tabularnewline\hline
  十四年 & -595 & \tabularnewline\hline
  十五年 & -594 & \tabularnewline\hline
  十六年 & -593 & \tabularnewline\hline
  十七年 & -592 & \tabularnewline\hline
  十八年 & -591 & \tabularnewline
  \bottomrule
\end{longtable}

%%% Local Variables:
%%% mode: latex
%%% TeX-engine: xetex
%%% TeX-master: "../../Main"
%%% End:

%% -*- coding: utf-8 -*-
%% Time-stamp: <Chen Wang: 2018-07-12 23:12:19>

\subsection{成公{\tiny(BC590-BC573)}}

% \centering
\begin{longtable}{|>{\centering\scriptsize}m{2em}|>{\centering\scriptsize}m{1.3em}|>{\centering}m{8.8em}|}
  % \caption{秦王政}\\
  \toprule
  \SimHei \normalsize 年数 & \SimHei \scriptsize 公元 & \SimHei 大事件 \tabularnewline
  % \midrule
  \endfirsthead
  \toprule
  \SimHei \normalsize 年数 & \SimHei \scriptsize 公元 & \SimHei 大事件 \tabularnewline
  \midrule
  \endhead
  \midrule
  元年 & -590 & \tabularnewline\hline
  二年 & -589 & \tabularnewline\hline
  三年 & -588 & \tabularnewline\hline
  四年 & -587 & \tabularnewline\hline
  五年 & -586 & \tabularnewline\hline
  六年 & -585 & \tabularnewline\hline
  七年 & -584 & \tabularnewline\hline
  八年 & -583 & \tabularnewline\hline
  九年 & -582 & \tabularnewline\hline
  十年 & -581 & \tabularnewline\hline
  十一年 & -580 & \tabularnewline\hline
  十二年 & -579 & \tabularnewline\hline
  十三年 & -578 & \tabularnewline\hline
  十四年 & -577 & \tabularnewline\hline
  十五年 & -576 & \tabularnewline\hline
  十六年 & -575 & \tabularnewline\hline
  十七年 & -574 & \tabularnewline\hline
  十八年 & -573 & \tabularnewline
  \bottomrule
\end{longtable}

%%% Local Variables:
%%% mode: latex
%%% TeX-engine: xetex
%%% TeX-master: "../../Main"
%%% End:

%% -*- coding: utf-8 -*-
%% Time-stamp: <Chen Wang: 2018-07-12 23:13:57>

\subsection{襄公{\tiny(BC572-BC542)}}

% \centering
\begin{longtable}{|>{\centering\scriptsize}m{2em}|>{\centering\scriptsize}m{1.3em}|>{\centering}m{8.8em}|}
  % \caption{秦王政}\\
  \toprule
  \SimHei \normalsize 年数 & \SimHei \scriptsize 公元 & \SimHei 大事件 \tabularnewline
  % \midrule
  \endfirsthead
  \toprule
  \SimHei \normalsize 年数 & \SimHei \scriptsize 公元 & \SimHei 大事件 \tabularnewline
  \midrule
  \endhead
  \midrule
  元年 & -572 & \tabularnewline\hline
  二年 & -571 & \tabularnewline\hline
  三年 & -570 & \tabularnewline\hline
  四年 & -569 & \tabularnewline\hline
  五年 & -568 & \tabularnewline\hline
  六年 & -567 & \tabularnewline\hline
  七年 & -566 & \tabularnewline\hline
  八年 & -565 & \tabularnewline\hline
  九年 & -564 & \tabularnewline\hline
  十年 & -563 & \tabularnewline\hline
  十一年 & -562 & \tabularnewline\hline
  十二年 & -561 & \tabularnewline\hline
  十三年 & -560 & \tabularnewline\hline
  十四年 & -559 & \tabularnewline\hline
  十五年 & -558 & \tabularnewline\hline
  十六年 & -557 & \tabularnewline\hline
  十七年 & -556 & \tabularnewline\hline
  十八年 & -555 & \tabularnewline\hline
  十九年 & -554 & \tabularnewline\hline
  二十年 & -553 & \tabularnewline\hline
  二一年 & -552 & \tabularnewline\hline
  二二年 & -551 & \tabularnewline\hline
  二三年 & -550 & \tabularnewline\hline
  二四年 & -549 & \tabularnewline\hline
  二五年 & -548 & \tabularnewline\hline
  二六年 & -547 & \tabularnewline\hline
  二七年 & -546 & \tabularnewline\hline
  二八年 & -545 & \tabularnewline\hline
  二九年 & -544 & \tabularnewline\hline
  三十年 & -543 & \tabularnewline\hline
  三一年 & -542 & \tabularnewline
  \bottomrule
\end{longtable}

%%% Local Variables:
%%% mode: latex
%%% TeX-engine: xetex
%%% TeX-master: "../../Main"
%%% End:

%% -*- coding: utf-8 -*-
%% Time-stamp: <Chen Wang: 2018-07-12 23:14:41>

\subsection{昭公{\tiny(BC541-BC510)}}

% \centering
\begin{longtable}{|>{\centering\scriptsize}m{2em}|>{\centering\scriptsize}m{1.3em}|>{\centering}m{8.8em}|}
  % \caption{秦王政}\\
  \toprule
  \SimHei \normalsize 年数 & \SimHei \scriptsize 公元 & \SimHei 大事件 \tabularnewline
  % \midrule
  \endfirsthead
  \toprule
  \SimHei \normalsize 年数 & \SimHei \scriptsize 公元 & \SimHei 大事件 \tabularnewline
  \midrule
  \endhead
  \midrule
  元年 & -541 & \tabularnewline\hline
  二年 & -540 & \tabularnewline\hline
  三年 & -539 & \tabularnewline\hline
  四年 & -538 & \tabularnewline\hline
  五年 & -537 & \tabularnewline\hline
  六年 & -536 & \tabularnewline\hline
  七年 & -535 & \tabularnewline\hline
  八年 & -534 & \tabularnewline\hline
  九年 & -533 & \tabularnewline\hline
  十年 & -532 & \tabularnewline\hline
  十一年 & -531 & \tabularnewline\hline
  十二年 & -530 & \tabularnewline\hline
  十三年 & -529 & \tabularnewline\hline
  十四年 & -528 & \tabularnewline\hline
  十五年 & -527 & \tabularnewline\hline
  十六年 & -526 & \tabularnewline\hline
  十七年 & -525 & \tabularnewline\hline
  十八年 & -524 & \tabularnewline\hline
  十九年 & -523 & \tabularnewline\hline
  二十年 & -522 & \tabularnewline\hline
  二一年 & -521 & \tabularnewline\hline
  二二年 & -520 & \tabularnewline\hline
  二三年 & -519 & \tabularnewline\hline
  二四年 & -518 & \tabularnewline\hline
  二五年 & -517 & \tabularnewline\hline
  二六年 & -516 & \tabularnewline\hline
  二七年 & -515 & \tabularnewline\hline
  二八年 & -514 & \tabularnewline\hline
  二九年 & -513 & \tabularnewline\hline
  三十年 & -512 & \tabularnewline\hline
  三一年 & -511 & \tabularnewline\hline
  三二年 & -510 & \tabularnewline
  \bottomrule
\end{longtable}

%%% Local Variables:
%%% mode: latex
%%% TeX-engine: xetex
%%% TeX-master: "../../Main"
%%% End:

%% -*- coding: utf-8 -*-
%% Time-stamp: <Chen Wang: 2018-07-12 23:15:38>

\subsection{定公{\tiny(BC509-BC495)}}

% \centering
\begin{longtable}{|>{\centering\scriptsize}m{2em}|>{\centering\scriptsize}m{1.3em}|>{\centering}m{8.8em}|}
  % \caption{秦王政}\\
  \toprule
  \SimHei \normalsize 年数 & \SimHei \scriptsize 公元 & \SimHei 大事件 \tabularnewline
  % \midrule
  \endfirsthead
  \toprule
  \SimHei \normalsize 年数 & \SimHei \scriptsize 公元 & \SimHei 大事件 \tabularnewline
  \midrule
  \endhead
  \midrule
  元年 & -509 & \tabularnewline\hline
  二年 & -508 & \tabularnewline\hline
  三年 & -507 & \tabularnewline\hline
  四年 & -506 & \tabularnewline\hline
  五年 & -505 & \tabularnewline\hline
  六年 & -504 & \tabularnewline\hline
  七年 & -503 & \tabularnewline\hline
  八年 & -502 & \tabularnewline\hline
  九年 & -501 & \tabularnewline\hline
  十年 & -500 & \tabularnewline\hline
  十一年 & -499 & \tabularnewline\hline
  十二年 & -498 & \tabularnewline\hline
  十三年 & -497 & \tabularnewline\hline
  十四年 & -496 & \tabularnewline\hline
  十五年 & -495 & \tabularnewline
  \bottomrule
\end{longtable}

%%% Local Variables:
%%% mode: latex
%%% TeX-engine: xetex
%%% TeX-master: "../../Main"
%%% End:

%% -*- coding: utf-8 -*-
%% Time-stamp: <Chen Wang: 2018-07-12 23:19:56>

\subsection{哀公{\tiny(BC494-BC467)}}

% \centering
\begin{longtable}{|>{\centering\scriptsize}m{2em}|>{\centering\scriptsize}m{1.3em}|>{\centering}m{8.8em}|}
  % \caption{秦王政}\\
  \toprule
  \SimHei \normalsize 年数 & \SimHei \scriptsize 公元 & \SimHei 大事件 \tabularnewline
  % \midrule
  \endfirsthead
  \toprule
  \SimHei \normalsize 年数 & \SimHei \scriptsize 公元 & \SimHei 大事件 \tabularnewline
  \midrule
  \endhead
  \midrule
  元年 & -494 & \tabularnewline\hline
  二年 & -493 & \tabularnewline\hline
  三年 & -492 & \tabularnewline\hline
  四年 & -491 & \tabularnewline\hline
  五年 & -490 & \tabularnewline\hline
  六年 & -489 & \tabularnewline\hline
  七年 & -488 & \tabularnewline\hline
  八年 & -487 & \tabularnewline\hline
  九年 & -486 & \tabularnewline\hline
  十年 & -485 & \tabularnewline\hline
  十一年 & -484 & \tabularnewline\hline
  十二年 & -483 & \tabularnewline\hline
  十三年 & -482 & \tabularnewline\hline
  十四年 & -481 & \tabularnewline\hline
  十五年 & -480 & \tabularnewline\hline
  十六年 & -479 & \tabularnewline\hline
  十七年 & -478 & \tabularnewline\hline
  十八年 & -477 & \tabularnewline\hline
  十九年 & -476 & \tabularnewline\hline
  二十年 & -475 & \tabularnewline\hline
  二一年 & -474 & \tabularnewline\hline
  二二年 & -473 & \tabularnewline\hline
  二三年 & -472 & \tabularnewline\hline
  二四年 & -471 & \tabularnewline\hline
  二五年 & -470 & \tabularnewline\hline
  二六年 & -469 & \tabularnewline\hline
  二七年 & -468 & \tabularnewline\hline
  二八年 & -467 & \tabularnewline
  \bottomrule
\end{longtable}

%%% Local Variables:
%%% mode: latex
%%% TeX-engine: xetex
%%% TeX-master: "../../Main"
%%% End:

%% -*- coding: utf-8 -*-
%% Time-stamp: <Chen Wang: 2018-07-12 23:19:27>

\subsection{悼公{\tiny(BC466-BC429)}}

% \centering
\begin{longtable}{|>{\centering\scriptsize}m{2em}|>{\centering\scriptsize}m{1.3em}|>{\centering}m{8.8em}|}
  % \caption{秦王政}\\
  \toprule
  \SimHei \normalsize 年数 & \SimHei \scriptsize 公元 & \SimHei 大事件 \tabularnewline
  % \midrule
  \endfirsthead
  \toprule
  \SimHei \normalsize 年数 & \SimHei \scriptsize 公元 & \SimHei 大事件 \tabularnewline
  \midrule
  \endhead
  \midrule
  元年 & -466 & \tabularnewline\hline
  二年 & -465 & \tabularnewline\hline
  三年 & -464 & \tabularnewline\hline
  四年 & -463 & \tabularnewline\hline
  五年 & -462 & \tabularnewline\hline
  六年 & -461 & \tabularnewline\hline
  七年 & -460 & \tabularnewline\hline
  八年 & -459 & \tabularnewline\hline
  九年 & -458 & \tabularnewline\hline
  十年 & -457 & \tabularnewline\hline
  十一年 & -456 & \tabularnewline\hline
  十二年 & -455 & \tabularnewline\hline
  十三年 & -454 & \tabularnewline\hline
  十四年 & -453 & \tabularnewline\hline
  十五年 & -452 & \tabularnewline\hline
  十六年 & -451 & \tabularnewline\hline
  十七年 & -450 & \tabularnewline\hline
  十八年 & -449 & \tabularnewline\hline
  十九年 & -448 & \tabularnewline\hline
  二十年 & -447 & \tabularnewline\hline
  二一年 & -446 & \tabularnewline\hline
  二二年 & -445 & \tabularnewline\hline
  二三年 & -444 & \tabularnewline\hline
  二四年 & -443 & \tabularnewline\hline
  二五年 & -442 & \tabularnewline\hline
  二六年 & -441 & \tabularnewline\hline
  二七年 & -440 & \tabularnewline\hline
  二八年 & -439 & \tabularnewline\hline
  二九年 & -438 & \tabularnewline\hline
  三十年 & -437 & \tabularnewline\hline
  三一年 & -436 & \tabularnewline\hline
  三二年 & -435 & \tabularnewline\hline
  三三年 & -434 & \tabularnewline\hline
  三四年 & -433 & \tabularnewline\hline
  三五年 & -432 & \tabularnewline\hline
  三六年 & -431 & \tabularnewline\hline
  三七年 & -430 & \tabularnewline\hline
  三八年 & -429 & \tabularnewline
  \bottomrule
\end{longtable}

%%% Local Variables:
%%% mode: latex
%%% TeX-engine: xetex
%%% TeX-master: "../../Main"
%%% End:

%% -*- coding: utf-8 -*-
%% Time-stamp: <Chen Wang: 2018-07-12 23:20:46>

\subsection{元公{\tiny(BC428-BC408)}}

% \centering
\begin{longtable}{|>{\centering\scriptsize}m{2em}|>{\centering\scriptsize}m{1.3em}|>{\centering}m{8.8em}|}
  % \caption{秦王政}\\
  \toprule
  \SimHei \normalsize 年数 & \SimHei \scriptsize 公元 & \SimHei 大事件 \tabularnewline
  % \midrule
  \endfirsthead
  \toprule
  \SimHei \normalsize 年数 & \SimHei \scriptsize 公元 & \SimHei 大事件 \tabularnewline
  \midrule
  \endhead
  \midrule
  元年 & -428 & \tabularnewline\hline
  二年 & -427 & \tabularnewline\hline
  三年 & -426 & \tabularnewline\hline
  四年 & -425 & \tabularnewline\hline
  五年 & -424 & \tabularnewline\hline
  六年 & -423 & \tabularnewline\hline
  七年 & -422 & \tabularnewline\hline
  八年 & -421 & \tabularnewline\hline
  九年 & -420 & \tabularnewline\hline
  十年 & -419 & \tabularnewline\hline
  十一年 & -418 & \tabularnewline\hline
  十二年 & -417 & \tabularnewline\hline
  十三年 & -416 & \tabularnewline\hline
  十四年 & -415 & \tabularnewline\hline
  十五年 & -414 & \tabularnewline\hline
  十六年 & -413 & \tabularnewline\hline
  十七年 & -412 & \tabularnewline\hline
  十八年 & -411 & \tabularnewline\hline
  十九年 & -410 & \tabularnewline\hline
  二十年 & -409 & \tabularnewline\hline
  二一年 & -408 & \tabularnewline
  \bottomrule
\end{longtable}

%%% Local Variables:
%%% mode: latex
%%% TeX-engine: xetex
%%% TeX-master: "../../Main"
%%% End:

%% -*- coding: utf-8 -*-
%% Time-stamp: <Chen Wang: 2018-07-12 23:23:22>

\subsection{穆公{\tiny(BC407-BC376)}}

% \centering
\begin{longtable}{|>{\centering\scriptsize}m{2em}|>{\centering\scriptsize}m{1.3em}|>{\centering}m{8.8em}|}
  % \caption{秦王政}\\
  \toprule
  \SimHei \normalsize 年数 & \SimHei \scriptsize 公元 & \SimHei 大事件 \tabularnewline
  % \midrule
  \endfirsthead
  \toprule
  \SimHei \normalsize 年数 & \SimHei \scriptsize 公元 & \SimHei 大事件 \tabularnewline
  \midrule
  \endhead
  \midrule
  元年 & -407 & \tabularnewline\hline
  二年 & -406 & \tabularnewline\hline
  三年 & -405 & \tabularnewline\hline
  四年 & -404 & \tabularnewline\hline
  五年 & -403 & \tabularnewline
  \bottomrule
\end{longtable}

%%% Local Variables:
%%% mode: latex
%%% TeX-engine: xetex
%%% TeX-master: "../../Main"
%%% End:


%%% Local Variables:
%%% mode: latex
%%% TeX-engine: xetex
%%% TeX-master: "../../Main"
%%% End:


%%% Local Variables:
%%% mode: latex
%%% TeX-engine: xetex
%%% TeX-master: "../Main"
%%% End:
 % 春秋
%% -*- coding: utf-8 -*-
%% Time-stamp: <Chen Wang: 2018-07-14 16:56:44>

\chapter{战国{\tiny(BC402-BC221)}}

%% -*- coding: utf-8 -*-
%% Time-stamp: <Chen Wang: 2018-07-14 16:58:19>

\section{东周\tiny(BC770-BC256)}

%% -*- coding: utf-8 -*-
%% Time-stamp: <Chen Wang: 2018-07-14 15:17:20>

\subsection{威烈王\tiny(BC425-BC402)}

\textbf{周威烈王}(?-前402年),姬姓,名午,为周考王之子,中国东周第二十代国王。周考王十五年,崩,周威烈王即立。周威烈王二十三年(前403年)封晋国大夫韩虔、赵籍、魏斯为韩侯、赵侯、魏侯,这是历史上著名的“三家分晋”。三家分晋标志着春秋时代的结束,紧接着是战国时代的来临,本年也是司马光《资治通鉴》记载的起点,司马光还为三家分晋一事发表长篇的感言。二十四年(前402年),病死。葬今河南省洛阳市。其子骄继位。

% \centering
\begin{longtable}{|>{\centering\scriptsize}m{2em}|>{\centering\scriptsize}m{1.3em}|>{\centering}m{8.8em}|}
  % \caption{秦王政}\\
  \toprule
  \SimHei \normalsize 年数 & \SimHei \scriptsize 公元 & \SimHei 大事件 \tabularnewline
  % \midrule
  \endfirsthead
  \toprule
  \SimHei \normalsize 年数 & \SimHei \scriptsize 公元 & \SimHei 大事件 \tabularnewline
  \midrule
  \endhead
  \midrule
  二三年 & -403 & \begin{enumerate}
    \tiny
  \item 命晉大夫魏斯\footnote{魏文侯(?-前396年),安邑(今山西夏县)人。中国战国时魏国统治者。姬姓,魏氏,名斯。周贞定王二十四年(前445年)继魏桓子位,周威烈王二年(前424年)称侯改元,威烈王二十三年(前403年)与韩、赵两家一起被周威烈王册封为诸侯,是为三家分晋,周安王六年(前396年)卒。}、趙籍\footnote{赵烈侯(?-前400年),是中国战国时期赵国的君主,原名赵籍,赵献侯之子。在位时用公仲连、牛畜、荀欣、徐越等人,为政待以仁义,约以王道。}、韓虔\footnote{韩景侯(?-前400年),名虔,韩武子之子。}爲諸侯\footnote{晋国(首府新田【山西省侯马市】)长期以来,在魏、赵、韩三大家族控制之下,国君不过空拥虚名,只在形式上,看起来晋国仍是一个完整的独立封国。本年(前四〇三年),周王国(首都洛阳【河南省洛阳市白马寺东】)国王(三十八任威烈王)姬午,下令擢升三大家族族长,亦即晋国三位国务官(大夫):魏斯当魏国(首府安邑【山西省夏县】)国君、赵籍当赵国(首府晋阳【山西省太原市】)国君、韩虔当韩国(首府平阳【山西省临汾市】)国君。晋国被三国瓜分后,只剩下一小片国土。}。
  \item 魏文侯使乐羊\footnote{乐羊,中山国人,战国时魏国的大将。是乐毅先祖。}伐中山\footnote{中山国,姬姓,春秋战国时白狄的一支——鲜虞仿照东周各诸侯国于公元前507年建立的国家,位于今河北省中部太行山东麓一带,中山国当时位于赵国和燕国之间,都于顾,后迁都于灵寿(今中国河北省灵寿县),因城中有山得国名。},克之。
  \item 吴起\footnote{吴起(前440年-前381年),中国战国初期军事家、政治家、改革家,兵家代表人物。卫国左氏(今山东省定陶县,一说山东省曹县东北)人。}杀妻以求为鲁将,大破齐师。
  \item 燕愍公\footnote{燕国(首府蓟城【北京市】)国君(三十四任)。}薨,子僖公立。
  \end{enumerate} \tabularnewline\hline
  二四年 & -402 & \begin{enumerate}
    \tiny
  \item 周威烈王崩,子安王骄立。
  \item 盜殺楚聲王,國人立其子悼王\footnote{聲王,名當。悼王,名疑。«諡法»︰不生其國曰聲。«註»云︰生於外家。年中早夭曰悼。«註»云︰年不稱志。又云︰恐懼從處曰悼。}。
  \end{enumerate} \tabularnewline
  \bottomrule
\end{longtable}

%%% Local Variables:
%%% mode: latex
%%% TeX-engine: xetex
%%% TeX-master: "../../Main"
%%% End:

%% -*- coding: utf-8 -*-
%% Time-stamp: <Chen Wang: 2018-07-16 22:51:28>

\subsection{元安王{\tiny(BC401-BC376)}}

周安王姬骄(?—前376年),姬姓,名骄,华夏族,周威烈王之子,威烈王死后继位,在位26年,病死。葬处不明。在位时封齐国大夫田和为齐侯,是谓“田氏代齐”。

% \centering
\begin{longtable}{|>{\centering\scriptsize}m{2em}|>{\centering\scriptsize}m{1.3em}|>{\centering}m{8.8em}|}
  % \caption{秦王政}\\
  \toprule
  \SimHei \normalsize 年数 & \SimHei \scriptsize 公元 & \SimHei 大事件 \tabularnewline
  % \midrule
  \endfirsthead
  \toprule
  \SimHei \normalsize 年数 & \SimHei \scriptsize 公元 & \SimHei 大事件 \tabularnewline
  \midrule
  \endhead
  \midrule
  元年 & -401 & \begin{enumerate}
    \tiny
  \item 秦伐魏,至陽孤\footnote{秦国(首府雍县【陕西省凤翔县】)进攻魏国(首府安邑【山西省夏县】),大军进抵阳孤(山西省垣曲县东南)。}。
  \end{enumerate} \tabularnewline\hline
  二年 & -400 & \begin{enumerate}
    \tiny
  \item 魏、韓、趙伐楚,至桑丘\footnote{魏国(首府安邑【山西省夏县】)、韩国(首府平阳【山西省临汾市】)、赵国(首府晋阳【山西省太原市】),联合攻击楚王国(首都郢都【湖北省江陵县】),大军进抵桑丘(《史记》作乘丘【山东省兖州市西北】)。}。
  \item 鄭圍韓陽翟\footnote{郑国(首府新郑【河南省新郑县】)围攻韩国所属的阳翟(河南省禹州市)。}。
  \item 韓景侯薨,子烈侯取立。
  \item 趙烈侯薨,國人立其弟武侯。
  \item 秦簡公薨,子惠\footnote{«諡法»︰愛民好與曰惠。}公立。
  \end{enumerate} \tabularnewline\hline
  三年 & -399 & \begin{enumerate}
    \tiny
  \item 王子定奔晉。
  \item 虢山崩,壅河\footnote{虢山(河南省三门峡市西)发生崩塌,土石坠入黄河,河水壅塞。}。
  \end{enumerate} \tabularnewline\hline
  四年 & -398 & \begin{enumerate}
    \tiny
  \item 楚圍鄭。鄭人殺其相駟子陽\footnote{郑国十一任国君穆公姬兰的儿子姬腓,别名子驷。古人往往用祖父的名字最后一个字作自己这一支派的姓。这位驷子阳,姓驷,名子阳,也是郑国贵族。}。
  \end{enumerate} \tabularnewline\hline
  五年 & -397 & \begin{enumerate}
    \tiny
  \item 日有食之。
  \item 三月,盜殺韓相俠累\footnote{侠累跟濮阳(河南省濮阳市)人严仲子之间,有难解的怨毒,严仲子听说轵邑(河南省济源市东南)人聂政,勇猛过人,备了黄金二千四百两(百镒),送给聂政的母亲,作为祝寿礼物,请聂政代他报仇。聂政拒绝,说:“娘亲在堂,要我奉养,我不能轻言牺牲。”稍后,娘亲逝世,聂政才接受这项委托。当暗杀行动开始时,侠累正在宰相府主持会报,警卫森严。聂政像闪电一样,突击而入,在众人惊愕中,举刀直刺侠累的咽喉,侠累立即死亡。聂政自知难以逃生,咬紧牙关,用利刃自行毁容,脸皮全被割破,又自挖双眼,再自刺腹部自杀,肠出满地。韩国政府把尸首拖到市场,公开示众,要求市人辨识刺客身份。聂政的姐姐聂荌听到消息,赶到首府平阳(山西省临汾市),抚尸哀哭说:“他就是轵邑深井里(济通市东南十五千米)的聂政,只因为我这个姐姐尚在人间,恐怕连累我,才忍心重重的自我毁灭。弟弟啊,我怎么会贪生怕死,使你埋没英名?”就在尸旁,自杀殉难。}。
  \end{enumerate} \tabularnewline\hline
  六年 & -396 & \begin{enumerate}
    \tiny
  \item 鄭駟子陽之黨弑繻公\footnote{繻者,«諡法»所不載。},而立其弟乙,是爲康公\footnote{郑国(首府新郑【河南省新郑县】)故宰相(相)驷子阳的残余党羽,击杀国君(二十七任)繻公姬贻,拥立他的弟弟姬乙继位(二十八任),是为康公。}。
  \item 宋悼公薨,子休公田立\footnote{宋国(首府睢阳【河南省商丘县】)国君(三十一任悼公)宋购由逝世,子宋田继位(三十二任),是为休公。武王封微子啓於宋,唐宋州之睢陽縣是也。自微子二十七世至悼公,名購由。休,亦«諡法»所不載。}。
  \end{enumerate} \tabularnewline\hline
  七年 & -395 & \tiny \kaiti 无记载 \tabularnewline\hline
  八年 & -394 & \begin{enumerate}
    \tiny
  \item 齊\footnote{武王封太公於齊,唐青州之臨淄是也。«括地志»曰︰天齊水在臨淄東南十五里。«封禪書»曰︰齊之所以爲齊者,以天齊。是年,康公貸之十一年。自太公至康公二十九世。}伐魯\footnote{成王封伯禽於魯,唐兗州之曲阜是也。是年,穆公之十六年。自伯禽至穆公凡二十八世。},取最\footnote{山东省曲阜市东南}。
  \item 鄭負黍\footnote{負黍山在陽城縣西南二十七里,或云在西南三十五里。}叛,復歸韓\footnote{前四〇七年,郑国攻击韩国,占领负黍城。}。
  \end{enumerate} \tabularnewline\hline
  九年 & -393 & \begin{enumerate}
    \tiny
  \item 魏伐鄭。
  \item 晉烈公\footnote{周成王封弟叔虞於唐。«括地志»曰︰故唐城在幷州晉陽縣北二里,堯所築也。«都城記»曰︰唐叔虞之子燮父徙居晉水旁,今幷州理故唐城,卽燮父初徙之處;其城南半入州城中。«毛詩譜»曰︰燮父以堯墟南有晉水,改曰晉侯。自唐叔至烈公三十七世。烈公,名止。«諡法»︰慈惠愛親曰孝。}薨,子孝公傾立。
  \end{enumerate} \tabularnewline\hline
  十年 & -392 & \tiny \kaiti 无记载\tabularnewline\hline
  十一年 & -391 & \begin{enumerate}
    \tiny
  \item 秦伐韓宜陽,取六邑\footnote{班«志»,宜陽縣屬弘農郡。«史記正義»曰︰宜陽縣故城,在河南府福昌縣東十四里,故韓城是也。此邑卽«周禮»「四井爲邑」之邑。}。
  \item 齊田和\footnote{田常生襄子盤,盤生莊子白,白生太公和。此序齊田氏之世也。田常,卽«左傳»陳成子恆也。溫公避仁廟諱,改「恆」曰「常」。自陳公子完奔齊,五世至常得政。«諡法»︰勝敵志強曰莊。}遷齊康公於海上,使食一城,以奉其先祀。
  \end{enumerate} \tabularnewline\hline
  十二年 & -390 & \begin{enumerate}
    \tiny
  \item 秦、晉戰于武城\footnote{晋国【首府新田】自被瓜分后,连本身生存都有问题,已无力作任何战争。可能是和魏国【首府安邑·山西省夏县】,或韩国【首府平阳·山西省临汾市】会战。}。
  \item 齊伐魏,取襄陽。
  \item 魯敗齊師于平陸。
  \end{enumerate} \tabularnewline\hline
  十三年 & -389 & \begin{enumerate}
    \tiny
  \item 秦侵晉。
  \item 齊田和會\footnote{孔穎達曰︰諸侯未及期而相見曰遇。會者,謂及期之禮,旣及期,又至所期之地。}魏文侯、楚人、衞人于濁澤,求爲諸侯。魏文侯爲之請於王及諸侯,王許之。
  \end{enumerate} \tabularnewline\hline
  十四年 & -388 & \begin{enumerate}
    \tiny
  \item 齊田和逝世,子田剡继位。
  \end{enumerate} \tabularnewline\hline
  十五年 & -387 & \begin{enumerate}
    \tiny
  \item 秦伐蜀\footnote{«譜記»普[疑衍]云︰蜀之先,肇自人皇之際。黃帝子昌意娶蜀山氏女,生帝俈。旣立,封其支庶於蜀,歷虞、夏、商、周。周衰,先稱王者蠶叢。余據武王伐紂,庸、蜀諸國皆會于牧野。孔安國曰︰蜀,叟也,春秋之時不與中國通。班«志»,南鄭縣屬漢中郡,唐爲梁州治所。},取南鄭。
  \item 魏文侯薨,太子擊立,是爲武侯。魏置相,相田文\footnote{魏击任命田文担任宰相。吴起不高兴,对田文说:“我想跟你讨论一下你我对于国家的贡献,你以为如何?”田文说:“当然可以。”吴起说:“指挥武装部队,官兵们愿意牺牲性命,使敌国惊惧,不敢打我们的主意,你比我怎么样?”田文说:“我不如你。”吴起说:“使政府的功能充分发挥,使全国人民安居乐业、国库充实、社会富庶,你比我怎么样?”田文说:“我不如你。”吴起说:“防卫西河(潼关以北的黄河),秦国不敢向东侵略。而韩国(首府平阳【山西省临汾市】)与赵国(首府晋阳【山西省太原市】),不敢不对我们唯命是听,你比我怎么样?”田文说:“我不如你。”吴起说:“这三项重要大事,你都不如我,可是官位却比我高,那为什么?”田文说:“当君王年纪还小,有权势的重要官员互相猜忌,随时可能发动政变,民心恐慌。这个时候,宰相位置,应该属于你?还是属于我?”吴起沉默很久,抱歉说:“我承认,应该属于你。”}。
  \item 秦惠公薨,子出公\footnote{出,非諡也;以其失國出死,故曰出公。}立。
  \item 趙武侯薨,國人復立烈侯之太子章,是爲敬侯\footnote{«諡法»︰夙夜警戒曰敬。}。
  \item 韓烈侯薨,子文侯立。
  \end{enumerate} \tabularnewline\hline
  十六年 & -386 & \begin{enumerate}
    \tiny
  \item 趙公子朝作亂,奔魏;與魏襲邯鄲,不克\footnote{本年【前三八六年】,赵国首府自晋阳迁邯郸,赵朝当是利用迁府之际,发动政变。}。
  \end{enumerate} \tabularnewline\hline
  十七年 & -385 & \begin{enumerate}
    \tiny
  \item 秦庶長\footnote{後秦制爵,一級曰公士,二上造,三簪裊,四不更,五大夫,六官大夫,七公大夫,八公乘,九五大夫,十左庶長,十一右庶長,十二左更,十三中更,十四右更,十五少上造,十六大上造,十七駟車庶長,十八大庶長,十九關內侯,二十徹侯。師古曰︰庶長,言衆列之長。}改逆獻公\footnote{威烈王十一年秦靈公卒,子獻公師隰不得立,立靈公季父悼子,是爲簡公。出子,簡公之孫也。今庶長改迎獻公而殺出子。}于河西而立之;殺出子及其母,沈之淵旁。
  \item 齐伐魯。
  \item 韓伐鄭,取陽城;伐宋,執宋公。
  \end{enumerate} \tabularnewline\hline
  十八年 & -384 & \tiny \kaiti 无记载 \tabularnewline\hline
  十九年 & -383 & \begin{enumerate}
    \tiny
  \item 魏敗趙師于兔臺。
  \end{enumerate} \tabularnewline\hline
  二十年 & -382 & \begin{enumerate}
    \tiny
  \item 日有食之,旣\footnote{旣,盡也}。
  \end{enumerate} \tabularnewline\hline
  二一年 & -381 & \begin{enumerate}
    \tiny
  \item 楚悼王薨。貴戚大臣作亂,攻吳起;起走之王尸而伏之。擊起之徒因射刺起,並中王尸。旣葬,肅\footnote{«諡法»︰剛德克就曰肅;執心決斷曰肅。}王卽位,使令尹盡誅爲亂者;坐起夷宗者七十餘家。
  \end{enumerate} \tabularnewline\hline
  二二年 & -380 & \begin{enumerate}
    \tiny
  \item 齊伐燕,取桑丘。
  \item 魏、韓、趙伐齊,至桑丘。
  \end{enumerate} \tabularnewline\hline
  二三年 & -379 & \begin{enumerate}
    \tiny
  \item 趙襲衞\footnote{成王封康叔於衞,居河、淇之間,故殷墟也。至懿公爲狄所滅,東徙度河。文公徙居楚丘,遂國於濮陽。是年,愼公頹之三十五年。自康叔至愼公凡三十二世。},不克。
  \item 齊康公薨,無子,田氏遂幷齊而有之。姜氏至此滅矣。
  \end{enumerate} \tabularnewline\hline
  二四年 & -378 & \begin{enumerate}
    \tiny
  \item 狄\footnote{漢之中山、上黨、西河、上郡,自春秋以來,狄皆居之,此亦其種也。«水經»︰澮水出河東絳縣東澮山,西過絳縣南,又西南過虒祁宮南,又西南至王橋,入汾水。«括地志»︰澮山在絳州翼城縣東北。}敗魏師于澮。
  \item 魏、韓、趙伐齊,至靈丘。
  \item 晉孝公薨,子靖公\footnote{«諡法»︰柔衆安民曰靖;又,恭己鮮言曰靖。}俱酒立。
  \item 齐国(首府临淄)国君(二任)田剡逝世,子田午继位(三任),是为桓公。
  \end{enumerate} \tabularnewline\hline
  二五年 & -377 & \begin{enumerate}
    \tiny
  \item 蜀伐楚,取茲方(四川省奉节县)。
  \item 子思论卫\footnote{卫国(首府濮阳【河南省濮阳市】),孔伋(子思)向卫国国君(四十一任慎公)卫颓,推荐苟变,说:“他的才干可以指挥五百辆战车作战。”卫颓说:“我知道他的军事才能,但苟变曾经当过税务员,有次平白吃了民家两个鸡蛋,品德上有瑕疵。”孔伋说:“政府任用官吏,跟建筑师选择木材一样,取其所长,弃其所短。巨木高耸云际,几个人都合抱不住,却有几尺朽烂,优秀的建筑师不会不用它。现在,我们正处在大混战时代,应该积极物色英雄豪杰,却为了两个鸡蛋,丧失一员大将,这话可别让别国听见才好。”卫颓再三致谢说:“我接受你的指教。”卫颓做了一项错误的决定,全体官员却一致赞扬那决定非常正确。孔伋对公丘懿子说:“我看你们卫国,真是君不像君,臣不像臣。”(“君不君,臣不臣”,《论语》引齐国【首府临淄·山东省淄博市东临淄镇】国君【二十六任景公】姜杵臼的话。)公丘懿子说:“怎么会糟到这种程度?”孔伋说:“领袖人物经常的自以为是,大家就不敢贡献自己的意见。做对了而自以为是,还会排斥众人的智慧。何况做错了而仍自以为是,硬教大家赞扬,那简直是鼓励邪恶。不问事情的是非,而只一味喜欢听悦耳的声音,可以说绝顶糊涂。不管那是不是合理,而只努力露出忠贞嘴脸,满口顺调,那就是马屁精。君主昏庸、官员谄媚,而高高坐在人民头上,人民绝对不会认同。如果一直这样下去,国家必亡。”孔伋告诉卫颓说:“你的国家,恐怕将要没落了。”卫颓说:“什么原因?”孔伋说:“当然有原因,领袖说一句话,自以为是,官员们没有一个人敢指出他的错误;官员们说一句话,自以为是,民间没有一个人敢指出他的错误。领袖和官员,都自以为英明盖世,属下的小官小民也同声赞扬他们果然是真的英明盖世。马屁精就有福了,指出君王错误的人一定大祸临久。如此这般,有益于国家的善政,怎能产生?《诗经》说:‘都说自己是圣贤,谁分辨乌鸦的雌雄?’听起来好像就是指的你们。”}。
  \item 魯穆公薨,子共公奮立\footnote{«諡法»︰布德就義曰穆;中情見貌曰穆;尊賢敬讓曰共;旣過能改曰共;執事堅固曰共。}。
  \item 韓文侯薨,子哀侯立。
  \end{enumerate} \tabularnewline\hline
  二六年 & -376 & \begin{enumerate}
    \tiny
  \item 王崩,子烈王喜立。
  \item 魏、韓、趙共廢晉靖公爲家人而分其地。唐叔不祀矣。
  \end{enumerate} \tabularnewline
  \bottomrule
\end{longtable}

%%% Local Variables:
%%% mode: latex
%%% TeX-engine: xetex
%%% TeX-master: "../../Main"
%%% End:

%% -*- coding: utf-8 -*-
%% Time-stamp: <Chen Wang: 2018-07-10 17:30:09>

\subsection{烈王{\tiny(BC375-BC369)}}


% \centering
\begin{longtable}{|>{\centering\scriptsize}m{2em}|>{\centering\scriptsize}m{1.3em}|>{\centering}m{8.8em}|}
  % \caption{秦王政}\\
  \toprule
  \SimHei \normalsize 年数 & \SimHei \scriptsize 公元 & \SimHei 大事件 \tabularnewline
  % \midrule
  \endfirsthead
  \toprule
  \SimHei \normalsize 年数 & \SimHei \scriptsize 公元 & \SimHei 大事件 \tabularnewline
  \midrule
  \endhead
  \midrule
  元年 & -375 & \tabularnewline\hline
  二年 & -374 & \tabularnewline\hline
  三年 & -373 & \tabularnewline\hline
  四年 & -372 & \tabularnewline\hline
  五年 & -371 & \tabularnewline\hline
  六年 & -370 & \tabularnewline\hline
  七年 & -369 & \tabularnewline
  \bottomrule
\end{longtable}

%%% Local Variables:
%%% mode: latex
%%% TeX-engine: xetex
%%% TeX-master: "../../Main"
%%% End:

%% -*- coding: utf-8 -*-
%% Time-stamp: <Chen Wang: 2018-07-10 17:30:31>

\subsection{显王{\tiny(BC368-BC321)}}


% \centering
\begin{longtable}{|>{\centering\scriptsize}m{2em}|>{\centering\scriptsize}m{1.3em}|>{\centering}m{8.8em}|}
  % \caption{秦王政}\\
  \toprule
  \SimHei \normalsize 年数 & \SimHei \scriptsize 公元 & \SimHei 大事件 \tabularnewline
  % \midrule
  \endfirsthead
  \toprule
  \SimHei \normalsize 年数 & \SimHei \scriptsize 公元 & \SimHei 大事件 \tabularnewline
  \midrule
  \endhead
  \midrule
  元年 & -368 & \tabularnewline\hline
  二年 & -367 & \tabularnewline\hline
  三年 & -366 & \tabularnewline\hline
  四年 & -365 & \tabularnewline\hline
  五年 & -364 & \tabularnewline\hline
  六年 & -363 & \tabularnewline\hline
  七年 & -362 & \tabularnewline\hline
  八年 & -361 & \tabularnewline\hline
  九年 & -360 & \tabularnewline\hline
  十年 & -359 & \tabularnewline\hline
  十一年 & -358 & \tabularnewline\hline
  十二年 & -357 & \tabularnewline\hline
  十三年 & -356 & \tabularnewline\hline
  十四年 & -355 & \tabularnewline\hline
  十五年 & -354 & \tabularnewline\hline
  十六年 & -353 & \tabularnewline\hline
  十七年 & -352 & \tabularnewline\hline
  十八年 & -351 & \tabularnewline\hline
  十九年 & -350 & \tabularnewline\hline
  二十年 & -349 & \tabularnewline\hline
  二一年 & -348 & \tabularnewline\hline
  二二年 & -347 & \tabularnewline\hline
  二三年 & -346 & \tabularnewline\hline
  二四年 & -345 & \tabularnewline\hline
  二五年 & -344 & \tabularnewline\hline
  二六年 & -343 & \tabularnewline\hline
  二七年 & -342 & \tabularnewline\hline
  二八年 & -341 & \tabularnewline\hline
  二九年 & -340 & \tabularnewline\hline
  三十年 & -339 & \tabularnewline\hline
  三一年 & -338 & \tabularnewline\hline
  三二年 & -337 & \tabularnewline\hline
  三三年 & -336 & \tabularnewline\hline
  三四年 & -335 & \tabularnewline\hline
  三五年 & -334 & \tabularnewline\hline
  三六年 & -333 & \tabularnewline\hline
  三七年 & -332 & \tabularnewline\hline
  三八年 & -331 & \tabularnewline\hline
  三九年 & -330 & \tabularnewline\hline
  四十年 & -329 & \tabularnewline\hline
  四一年 & -328 & \tabularnewline\hline
  四二年 & -327 & \tabularnewline\hline
  四三年 & -326 & \tabularnewline\hline
  四四年 & -325 & \tabularnewline\hline
  四五年 & -324 & \tabularnewline\hline
  四六年 & -323 & \tabularnewline\hline
  四七年 & -322 & \tabularnewline\hline
  四八年 & -321 & \tabularnewline
  \bottomrule
\end{longtable}

%%% Local Variables:
%%% mode: latex
%%% TeX-engine: xetex
%%% TeX-master: "../../Main"
%%% End:

%% -*- coding: utf-8 -*-
%% Time-stamp: <Chen Wang: 2018-07-10 17:30:19>

\subsection{慎靓王{\tiny(BC320-BC315)}}


% \centering
\begin{longtable}{|>{\centering\scriptsize}m{2em}|>{\centering\scriptsize}m{1.3em}|>{\centering}m{8.8em}|}
  % \caption{秦王政}\\
  \toprule
  \SimHei \normalsize 年数 & \SimHei \scriptsize 公元 & \SimHei 大事件 \tabularnewline
  % \midrule
  \endfirsthead
  \toprule
  \SimHei \normalsize 年数 & \SimHei \scriptsize 公元 & \SimHei 大事件 \tabularnewline
  \midrule
  \endhead
  \midrule
  元年 & -320 & \tabularnewline\hline
  二年 & -319 & \tabularnewline\hline
  三年 & -318 & \tabularnewline\hline
  四年 & -317 & \tabularnewline\hline
  五年 & -316 & \tabularnewline\hline
  六年 & -315 & \tabularnewline
  \bottomrule
\end{longtable}

%%% Local Variables:
%%% mode: latex
%%% TeX-engine: xetex
%%% TeX-master: "../../Main"
%%% End:

%% -*- coding: utf-8 -*-
%% Time-stamp: <Chen Wang: 2018-07-10 17:30:15>

\subsection{赧王{\tiny(BC314-BC256)}}


% \centering
\begin{longtable}{|>{\centering\scriptsize}m{2em}|>{\centering\scriptsize}m{1.3em}|>{\centering}m{8.8em}|}
  % \caption{秦王政}\\
  \toprule
  \SimHei \normalsize 年数 & \SimHei \scriptsize 公元 & \SimHei 大事件 \tabularnewline
  % \midrule
  \endfirsthead
  \toprule
  \SimHei \normalsize 年数 & \SimHei \scriptsize 公元 & \SimHei 大事件 \tabularnewline
  \midrule
  \endhead
  \midrule
  元年 & -314 & \tabularnewline\hline
  二年 & -313 & \tabularnewline\hline
  三年 & -312 & \tabularnewline\hline
  四年 & -311 & \tabularnewline\hline
  五年 & -310 & \tabularnewline\hline
  六年 & -309 & \tabularnewline\hline
  七年 & -308 & \tabularnewline\hline
  八年 & -307 & \tabularnewline\hline
  九年 & -306 & \tabularnewline\hline
  十年 & -305 & \tabularnewline\hline
  十一年 & -304 & \tabularnewline\hline
  十二年 & -303 & \tabularnewline\hline
  十三年 & -302 & \tabularnewline\hline
  十四年 & -301 & \tabularnewline\hline
  十五年 & -300 & \tabularnewline\hline
  十六年 & -299 & \tabularnewline\hline
  十七年 & -298 & \tabularnewline\hline
  十八年 & -297 & \tabularnewline\hline
  十九年 & -296 & \tabularnewline\hline
  二十年 & -295 & \tabularnewline\hline
  二一年 & -294 & \tabularnewline\hline
  二二年 & -293 & \tabularnewline\hline
  二三年 & -292 & \tabularnewline\hline
  二四年 & -291 & \tabularnewline\hline
  二五年 & -290 & \tabularnewline\hline
  二六年 & -289 & \tabularnewline\hline
  二七年 & -288 & \tabularnewline\hline
  二八年 & -287 & \tabularnewline\hline
  二九年 & -286 & \tabularnewline\hline
  三十年 & -285 & \tabularnewline\hline
  三一年 & -284 & \tabularnewline\hline
  三二年 & -283 & \tabularnewline\hline
  三三年 & -282 & \tabularnewline\hline
  三四年 & -281 & \tabularnewline\hline
  三五年 & -280 & \tabularnewline\hline
  三六年 & -279 & \tabularnewline\hline
  三七年 & -278 & \tabularnewline\hline
  三八年 & -277 & \tabularnewline\hline
  三九年 & -276 & \tabularnewline\hline
  四十年 & -275 & \tabularnewline\hline
  四一年 & -274 & \tabularnewline\hline
  四二年 & -273 & \tabularnewline\hline
  四三年 & -272 & \tabularnewline\hline
  四四年 & -271 & \tabularnewline\hline
  四五年 & -270 & \tabularnewline\hline
  四六年 & -269 & \tabularnewline\hline
  四七年 & -268 & \tabularnewline\hline
  四八年 & -267 & \tabularnewline\hline
  四九年 & -266 & \tabularnewline\hline
  五十年 & -265 & \tabularnewline\hline
  五一年 & -264 & \tabularnewline\hline
  五二年 & -263 & \tabularnewline\hline
  五三年 & -262 & \tabularnewline\hline
  五四年 & -261 & \tabularnewline\hline
  五五年 & -260 & \tabularnewline\hline
  五六年 & -259 & \tabularnewline\hline
  五七年 & -258 & \tabularnewline\hline
  五八年 & -257 & \tabularnewline\hline
  五九年 & -256 & \tabularnewline
  \bottomrule
\end{longtable}

%%% Local Variables:
%%% mode: latex
%%% TeX-engine: xetex
%%% TeX-master: "../../Main"
%%% End:


%%% Local Variables:
%%% mode: latex
%%% TeX-engine: xetex
%%% TeX-master: "../../Main"
%%% End:

%% -*- coding: utf-8 -*-
%% Time-stamp: <Chen Wang: 2018-07-14 16:54:00>

\section{郑\tiny(BC806-BC375)}

%% -*- coding: utf-8 -*-
%% Time-stamp: <Chen Wang: 2018-07-16 22:28:46>

\subsection{繻公{\tiny(BC422-BC396)}}

% \centering
\begin{longtable}{|>{\centering\scriptsize}m{2em}|>{\centering\scriptsize}m{1.3em}|>{\centering}m{8.8em}|}
  % \caption{秦王政}\\
  \toprule
  \SimHei \normalsize 年数 & \SimHei \scriptsize 公元 & \SimHei 大事件 \tabularnewline
  % \midrule
  \endfirsthead
  \toprule
  \SimHei \normalsize 年数 & \SimHei \scriptsize 公元 & \SimHei 大事件 \tabularnewline
  \midrule
  \endhead
  \midrule
  元年 & -422 & \tabularnewline\hline
  二年 & -421 & \tabularnewline\hline
  三年 & -420 & \tabularnewline\hline
  四年 & -419 & \tabularnewline\hline
  五年 & -418 & \tabularnewline\hline
  六年 & -417 & \tabularnewline\hline
  七年 & -416 & \tabularnewline\hline
  八年 & -415 & \tabularnewline\hline
  九年 & -414 & \tabularnewline\hline
  十年 & -413 & \tabularnewline\hline
  十一年 & -412 & \tabularnewline\hline
  十二年 & -411 & \tabularnewline\hline
  十三年 & -410 & \tabularnewline\hline
  十四年 & -409 & \tabularnewline\hline
  十五年 & -408 & \tabularnewline\hline
  十六年 & -407 & \tabularnewline\hline
  十七年 & -406 & \tabularnewline\hline
  十八年 & -405 & \tabularnewline\hline
  十九年 & -404 & \tabularnewline\hline
  二十年 & -403 & \tabularnewline% \hline
  % 二一年 & -402 & \tabularnewline\hline
  % 二二年 & -401 & \tabularnewline\hline
  % 二三年 & -400 & \tabularnewline\hline
  % 二四年 & -399 & \tabularnewline\hline
  % 二五年 & -398 & \tabularnewline\hline
  % 二六年 & -397 & \tabularnewline\hline
  % 二七年 & -396 & \tabularnewline
  \bottomrule
\end{longtable}

%%% Local Variables:
%%% mode: latex
%%% TeX-engine: xetex
%%% TeX-master: "../../Main"
%%% End:


%%% Local Variables:
%%% mode: latex
%%% TeX-engine: xetex
%%% TeX-master: "../../Main"
%%% End:

%% -*- coding: utf-8 -*-
%% Time-stamp: <Chen Wang: 2018-07-14 16:55:09>

\section{宋\tiny(?-BC286)}

%% -*- coding: utf-8 -*-
%% Time-stamp: <Chen Wang: 2018-07-14 16:56:17>

\subsection{悼公{\tiny(BC403-BC396)}}

% \centering
\begin{longtable}{|>{\centering\scriptsize}m{2em}|>{\centering\scriptsize}m{1.3em}|>{\centering}m{8.8em}|}
  % \caption{秦王政}\\
  \toprule
  \SimHei \normalsize 年数 & \SimHei \scriptsize 公元 & \SimHei 大事件 \tabularnewline
  % \midrule
  \endfirsthead
  \toprule
  \SimHei \normalsize 年数 & \SimHei \scriptsize 公元 & \SimHei 大事件 \tabularnewline
  \midrule
  \endhead
  \midrule
  元年 & -403 & \tabularnewline\hline
  二年 & -402 & \tabularnewline\hline
  三年 & -401 & \tabularnewline\hline
  四年 & -400 & \tabularnewline\hline
  五年 & -399 & \tabularnewline\hline
  六年 & -398 & \tabularnewline\hline
  七年 & -397 & \tabularnewline\hline
  八年 & -396 & \tabularnewline
  \bottomrule
\end{longtable}

%%% Local Variables:
%%% mode: latex
%%% TeX-engine: xetex
%%% TeX-master: "../../Main"
%%% End:


%%% Local Variables:
%%% mode: latex
%%% TeX-engine: xetex
%%% TeX-master: "../../Main"
%%% End:

%% -*- coding: utf-8 -*-
%% Time-stamp: <Chen Wang: 2018-07-10 15:07:54>

\section{秦}

%% -*- coding: utf-8 -*-
%% Time-stamp: <Chen Wang: 2018-07-10 17:30:49>

\subsection{昭襄王{\tiny(BC306-BC251)}}


% \centering
\begin{longtable}{|>{\centering\scriptsize}m{2em}|>{\centering\scriptsize}m{1.3em}|>{\centering}m{8.8em}|}
  % \caption{秦王政}\\
  \toprule
  \SimHei \normalsize 年数 & \SimHei \scriptsize 公元 & \SimHei 大事件 \tabularnewline
  % \midrule
  \endfirsthead
  \toprule
  \SimHei \normalsize 年数 & \SimHei \scriptsize 公元 & \SimHei 大事件 \tabularnewline
  \midrule
  \endhead
  \midrule
  元年 & -306 & \tabularnewline\hline
  二年 & -305 & \tabularnewline\hline
  三年 & -304 & \tabularnewline\hline
  四年 & -303 & \tabularnewline\hline
  五年 & -302 & \tabularnewline\hline
  六年 & -301 & \tabularnewline\hline
  七年 & -300 & \tabularnewline\hline
  八年 & -299 & \tabularnewline\hline
  九年 & -298 & \tabularnewline\hline
  十年 & -297 & \tabularnewline\hline
  十一年 & -296 & \tabularnewline\hline
  十二年 & -295 & \tabularnewline\hline
  十三年 & -294 & \tabularnewline\hline
  十四年 & -293 & \tabularnewline\hline
  十五年 & -292 & \tabularnewline\hline
  十六年 & -291 & \tabularnewline\hline
  十七年 & -290 & \tabularnewline\hline
  十八年 & -289 & \tabularnewline\hline
  十九年 & -288 & \tabularnewline\hline
  二十年 & -287 & \tabularnewline\hline
  二一年 & -286 & \tabularnewline\hline
  二二年 & -285 & \tabularnewline\hline
  二三年 & -284 & \tabularnewline\hline
  二四年 & -283 & \tabularnewline\hline
  二五年 & -282 & \tabularnewline\hline
  二六年 & -281 & \tabularnewline\hline
  二七年 & -280 & \tabularnewline\hline
  二八年 & -279 & \tabularnewline\hline
  二九年 & -278 & \tabularnewline\hline
  三十年 & -277 & \tabularnewline\hline
  三一年 & -276 & \tabularnewline\hline
  三二年 & -275 & \tabularnewline\hline
  三三年 & -274 & \tabularnewline\hline
  三四年 & -273 & \tabularnewline\hline
  三五年 & -272 & \tabularnewline\hline
  三六年 & -271 & \tabularnewline\hline
  三七年 & -270 & \tabularnewline\hline
  三八年 & -269 & \tabularnewline\hline
  三九年 & -268 & \tabularnewline\hline
  四十年 & -267 & \tabularnewline\hline
  四一年 & -266 & \tabularnewline\hline
  四二年 & -265 & \tabularnewline\hline
  四三年 & -264 & \tabularnewline\hline
  四四年 & -263 & \tabularnewline\hline
  四五年 & -262 & \tabularnewline\hline
  四六年 & -261 & \tabularnewline\hline
  四七年 & -260 & \tabularnewline\hline
  四八年 & -259 & \tabularnewline\hline
  四九年 & -258 & \tabularnewline\hline
  五十年 & -257 & \tabularnewline\hline
  五一年 & -256 & \tabularnewline\hline
  五二年 & -255 & \tabularnewline\hline
  五三年 & -254 & \tabularnewline\hline
  五四年 & -253 & \tabularnewline\hline
  五五年 & -252 & \tabularnewline\hline
  五六年 & -251 & \tabularnewline
  \bottomrule
\end{longtable}

%%% Local Variables:
%%% mode: latex
%%% TeX-engine: xetex
%%% TeX-master: "../../Main"
%%% End:

%% -*- coding: utf-8 -*-
%% Time-stamp: <Chen Wang: 2018-07-10 17:30:44>

\subsection{孝文王{\tiny(BC250-BC250)}}


% \centering
\begin{longtable}{|>{\centering\scriptsize}m{2em}|>{\centering\scriptsize}m{1.3em}|>{\centering}m{8.8em}|}
  % \caption{秦王政}\\
  \toprule
  \SimHei \normalsize 年数 & \SimHei \scriptsize 公元 & \SimHei 大事件 \tabularnewline
  % \midrule
  \endfirsthead
  \toprule
  \SimHei \normalsize 年数 & \SimHei \scriptsize 公元 & \SimHei 大事件 \tabularnewline
  \midrule
  \endhead
  \midrule
  元年 & -250 & \tabularnewline
  \bottomrule
\end{longtable}

%%% Local Variables:
%%% mode: latex
%%% TeX-engine: xetex
%%% TeX-master: "../../Main"
%%% End:

%% -*- coding: utf-8 -*-
%% Time-stamp: <Chen Wang: 2018-07-10 17:30:54>

\subsection{庄襄王{\tiny(BC249-BC247)}}


% \centering
\begin{longtable}{|>{\centering\scriptsize}m{2em}|>{\centering\scriptsize}m{1.3em}|>{\centering}m{8.8em}|}
  % \caption{秦王政}\\
  \toprule
  \SimHei \normalsize 年数 & \SimHei \scriptsize 公元 & \SimHei 大事件 \tabularnewline
  % \midrule
  \endfirsthead
  \toprule
  \SimHei \normalsize 年数 & \SimHei \scriptsize 公元 & \SimHei 大事件 \tabularnewline
  \midrule
  \endhead
  \midrule
  元年 & -249 & \tabularnewline\hline
  二年 & -248 & \tabularnewline\hline
  三年 & -247 & \tabularnewline
  \bottomrule
\end{longtable}

%%% Local Variables:
%%% mode: latex
%%% TeX-engine: xetex
%%% TeX-master: "../../Main"
%%% End:

%% -*- coding: utf-8 -*-
%% Time-stamp: <Chen Wang: 2018-07-10 03:03:58>

\subsection{赢政{\tiny(BC246-BC221)}}


% \centering
\begin{longtable}{|>{\centering\scriptsize}m{2em}|>{\centering\scriptsize}m{1.3em}|>{\centering}m{9em}|}
  % \caption{秦王政}\\
  \toprule
  \SimHei \normalsize 年数 & \SimHei \scriptsize 公元 & \SimHei 大事件 \tabularnewline
  % \midrule
  \endfirsthead
  \toprule
  \SimHei \normalsize 年数 & \SimHei \scriptsize 公元 & \SimHei 大事件 \tabularnewline
  \midrule
  \endhead
  \midrule
  元年 & -246 & \begin{enumerate}
    \tiny
  \item 韩国水工郑国开始建造郑国渠,约十年后完工。
  \item 秦晋阳反,蒙骜击平之。
  \end{enumerate} \tabularnewline\hline
  二年 & -245 & \begin{enumerate}
    \tiny
  \item 秦麃公将卒攻卷,斩首三万。
  \item 赵以廉颇为假相国,伐魏,取繁阳。赵孝成王薨,子赵悼襄王偃立。
  \end{enumerate} \tabularnewline\hline
  三年 & -244 & \begin{enumerate}
    \tiny
  \item 秦蒙骜攻韩,取12城。
  \end{enumerate} \tabularnewline\hline
  四年 & -243 & \begin{enumerate}
    \tiny
  \item 春,秦蒙骜伐魏,取旸、有诡。三月,军罢。
  \item 秦质子归自赵;赵太子出归国。
  \item 七月,秦国蝗,疫。令百姓纳粟千石,拜爵一级。
  \item 魏安釐王薨,子魏景湣王增立。
  \item 赵悼襄王以李牧为将,伐燕,取武遂、方城。
  \item 逝世:魏安釐王、信陵君魏无忌。
  \end{enumerate} \tabularnewline\hline
  五年 & -242 & \begin{enumerate}
    \tiny
  \item 秦蒙骜伐魏,取酸枣、燕、虚、长平、雍丘、山阳等二十城;初置东郡。
  \item 燕王使剧辛将而伐赵。
  \end{enumerate} \tabularnewline\hline
  六年 & -241 & \begin{enumerate}
    \tiny
  \item 函谷关之战。
  \item 秦拔魏朝歌,及卫濮阳。
  \end{enumerate} \tabularnewline\hline
  七年 & -240 & \begin{enumerate}
    \tiny
  \item 秦置濮阳县,属东郡,并定其为东郡治所。
  \item 逝世:蒙骜、邹衍。
  \item 出生:陆贾。
  \item 天象:彗星光出东方,见北方,五月见西方。
  \end{enumerate} \tabularnewline\hline
  八年 & -239 & \begin{enumerate}
    \tiny
  \item 北扶余王国建立。
  \item 嫪毐封长信侯。
  \item 魏与赵邺。
  \item 文学:吕氏春秋编成。
  \item 逝世:长安君成蟜、韩桓惠王。
  \end{enumerate} \tabularnewline\hline
  九年 & -238 & \begin{enumerate}
    \tiny
  \item 嬴政亲政。
  \item 嫪毐叛乱,被秦王政夷灭三族。
  \item 秦伐魏,取垣、浦。
  \item 逝世:荀子、楚春申君黄歇、楚考烈王。
  \end{enumerate} \tabularnewline\hline
  十年 & -237 & \begin{enumerate}
    \tiny
  \item 齐王建拜会秦王政。
  \item 吕不韦免相。
  \item 秦王政下令驱除异邦客卿,李斯上书劝秦始皇收回逐客令。
  \end{enumerate} \tabularnewline\hline
  十一年 & -236 & \begin{enumerate}
    \tiny
  \item 郑国渠建成。
  \item 秦攻赵,赵攻燕\footnote{公元前236年,秦乘攻取赵的阏与、橑阳、邺、安阳等城,后又大举攻赵,遭到顽强抵抗。赵虽两次打败秦军,但兵力耗损殆尽。秦国西出太行山,突袭赵国邯郸拉开了统一战的的序幕。 赵国和燕国激战正酣,他想将秦国造成的领土损失在燕国身上补回来。这时秦国乘虚而入。赵国急忙命令大将李牧率军南下应敌。}。
  \end{enumerate} \tabularnewline\hline
  十二年 & -235 & \begin{enumerate}
    \tiny
  \item 秦攻楚国\footnote{秦继攻赵之后,即命辛梧率四郡兵,会同魏国,对楚国发起攻击。}。
  \item 吕不韦卒\footnote{因嫪毐集团叛乱事受牵连,被免除相邦职务,出居河南封地。不久,秦王政下令将其流放至蜀地(今四川),不韦忧惧交加,于是在三川郡(今河南洛阳)自鸩而亡。}。
  \end{enumerate} \tabularnewline\hline
  十三年 & -234 & \begin{enumerate}
    \tiny
  \item 秦攻赵\footnote{公元前234年,秦再度向赵南部进攻。桓龁避开正面渡河,改由漳河下游渡河迂回赵扈辄军的侧后,攻击邯郸东南的平阳。两军于平阳展开交战,赵军被击破,被斩10万人,赵将扈辄阵亡。赵王启用北部边疆名将李牧为统帅。李牧军曾歼灭匈奴入侵军10万之众,威震边疆,战斗力最强。李牧率军回赵,立即同秦桓龁军交战于宜安肥下地区,给秦军几乎全军覆灭的沉重打击,只有统帅桓龁带领少数护卫突围逃走。}。
  \item 韩非\footnote{韩非(约前281年-前233年),生活于战国末期时期的韩国(今属河南省新郑市)的思想家,为中国古代著名法家思想的代表人物,认为应该要“法”、“术”、“势”三者并重,是法家的集大成者。韩非出身韩国公族,与李斯均是荀子学生,后因其学识渊博,被秦始皇召唤入秦,正欲重用,却遭到妒忌的同窗李斯害死,在韩非死后,秦始皇在韩非的思想指引下,完成统一六国的帝业。韩非其学出于荀子,源于儒家,而成为法家,又推究老子思想,归本于道家。司马迁指出韩非喜好“刑名法术”且归本于道家的“黄老之学”,一套由“道”、“法”共同完善的政治统治理论。}作为韩国的使臣来到秦国,上书秦王,劝其先伐赵而缓伐韩。
  \end{enumerate} \tabularnewline\hline
  十四年 & -233 & \begin{enumerate}
    \tiny
  \item 韩非子卒。
  \item 燕抗秦\footnote{公元前233年,秦将樊於期叛逃至燕国后,太子丹的师傅鞠武害怕秦国以此借口攻燕,便策划送樊於期到头曼那里,利用熟悉秦国虚实的樊於期结连匈奴攻秦。可惜性急的太子丹等不得这种长远之计凑效,他决定派出荆轲刺杀自己的童年好友嬴政,为了能够解除嬴政的戒备,荆轲提出要携带两样礼物:樊於期的人头和燕国督亢地图(割地求和)。嬴政在逃过刺杀威胁后更以迅雷不及掩耳之势统一六国。}。
  \item 赵将李牧大败秦将桓齮\footnote{桓齮(yǐ)(?-前227年),战国末年秦国将军。杨宽的《战国史》认为桓齮就是樊於期。始皇十一年(前237年),桓齮与王翦和杨端和攻赵,取邺九城。秦始皇十四年,也就是赵王迁二年(前233年),桓齮从上党越太行山进攻赵的赤丽、宜安(石家庄东南),与赵将李牧战于肥下(宜安东北),为李牧所败,逃至燕国(《战国策》说是战败被杀,《资治通鉴》记载“秦师败绩,桓齮奔还”)后无相关记载。}于肥。
  \end{enumerate} \tabularnewline\hline
  十五年 & -232 & \begin{enumerate}
    \tiny
  \item 项羽出生。
  \item 太子丹回燕。
  \end{enumerate} \tabularnewline\hline
  十六年 & -231 & \begin{enumerate}
    \tiny
  \item 秦攻韩。
  \item 魏献丽邑。
  \item 赵国地震。
  \item 韩信出生。
  \end{enumerate} \tabularnewline\hline
  十七年 & -230 & \begin{enumerate}
    \tiny
  \item 韩国灭亡。
  \end{enumerate} \tabularnewline\hline
  十八年 & -229 & \begin{enumerate}
    \tiny
  \item 秦攻赵国。
  \item 李牧被杀。
  \end{enumerate} \tabularnewline\hline
  十九年 & -228 & \begin{enumerate}
    \tiny
  \item 秦破赵得和氏璧。
  \item 赵国灭亡。
  \end{enumerate} \tabularnewline\hline
  二十年 & -227 & \begin{enumerate}
    \tiny
  \item 荆轲刺秦王。
  \item 王翦、辛胜在易水西败燕、代联军。
  \end{enumerate} \tabularnewline\hline
  二一年 & -226 & \begin{enumerate}
    \tiny
  \item 秦军攻燕都。
  \item 秦攻蓟城。
  \end{enumerate} \tabularnewline\hline
  二二年 & -225 & \begin{enumerate}
    \tiny
  \item 魏国灭亡。
  \item 秦置砀郡,立浚仪(大梁)、启封两县。
  \end{enumerate} \tabularnewline\hline
  二三年 & -224 & \begin{enumerate}
    \tiny
  \item 秦楚之战。
  \item 秦置修武县。
  \end{enumerate} \tabularnewline\hline
  二四年 & -223 & \begin{enumerate}
    \tiny
  \item 楚将项燕自杀。
  \item 秦灭楚。
  \end{enumerate} \tabularnewline\hline
  二五年 & -222 & \begin{enumerate}
    \tiny
  \item 秦灭代。
  \item 秦灭燕。
  \end{enumerate} \tabularnewline
  \bottomrule
\end{longtable}

%%% Local Variables:
%%% mode: latex
%%% TeX-engine: xetex
%%% TeX-master: "../../Main"
%%% End:


%%% Local Variables:
%%% mode: latex
%%% TeX-engine: xetex
%%% TeX-master: "../../Main"
%%% End:

%% -*- coding: utf-8 -*-
%% Time-stamp: <Chen Wang: 2018-07-12 23:25:10>

\section{鲁}

%% -*- coding: utf-8 -*-
%% Time-stamp: <Chen Wang: 2018-07-12 23:23:22>

\subsection{穆公{\tiny(BC407-BC376)}}

% \centering
\begin{longtable}{|>{\centering\scriptsize}m{2em}|>{\centering\scriptsize}m{1.3em}|>{\centering}m{8.8em}|}
  % \caption{秦王政}\\
  \toprule
  \SimHei \normalsize 年数 & \SimHei \scriptsize 公元 & \SimHei 大事件 \tabularnewline
  % \midrule
  \endfirsthead
  \toprule
  \SimHei \normalsize 年数 & \SimHei \scriptsize 公元 & \SimHei 大事件 \tabularnewline
  \midrule
  \endhead
  \midrule
  元年 & -407 & \tabularnewline\hline
  二年 & -406 & \tabularnewline\hline
  三年 & -405 & \tabularnewline\hline
  四年 & -404 & \tabularnewline\hline
  五年 & -403 & \tabularnewline
  \bottomrule
\end{longtable}

%%% Local Variables:
%%% mode: latex
%%% TeX-engine: xetex
%%% TeX-master: "../../Main"
%%% End:


%%% Local Variables:
%%% mode: latex
%%% TeX-engine: xetex
%%% TeX-master: "../../Main"
%%% End:


%%% Local Variables:
%%% mode: latex
%%% TeX-engine: xetex
%%% TeX-master: "../Main"
%%% End:
 % 战国
% %% -*- coding: utf-8 -*-
%% Time-stamp: <Chen Wang: 2018-07-10 15:12:24>

\chapter{秦\tiny(BC221-BC207)}

%% -*- coding: utf-8 -*-
%% Time-stamp: <Chen Wang: 2018-07-10 17:29:45>

\section{始皇帝\tiny(BC221-BC210)}

\begin{longtable}{|>{\centering\scriptsize}m{2em}|>{\centering\scriptsize}m{1.3em}|>{\centering}m{8.8em}|}
  % \caption{秦王政}\
  \toprule
  \SimHei \normalsize 年数 & \SimHei \scriptsize 公元 & \SimHei 大事件 \tabularnewline
  % \midrule
  \endfirsthead
  \toprule
  \SimHei \normalsize 年数 & \SimHei \scriptsize 公元 & \SimHei 大事件 \tabularnewline
  \midrule
  \endhead
  \midrule
  二六年 & -221 & \begin{enumerate}
    \tiny
  \item 秦将王贲率军灭齐。
  \item 始皇统一中国。
  \item 秦攻百越\footnote{公元前221年,秦始皇统一后,令50万大军准备征服南方百越各部。秦军分5路南下,在越城岭遭到南方越人的顽强抵抗。}。
  \item 秦始凿灵渠\footnote{灵渠,建于秦始皇执政时期,是中国,也是世界上最早的运河之一。对中国岭南地区的开发起了重要作用。对今天的水利工程建设,仍然据有很好的参考价值}。
  \end{enumerate} \tabularnewline\hline
  二七年 & -220 & \begin{enumerate}
    \tiny
  \item 秦规划咸阳\footnote{公元前220年,秦始皇下令,将秦的东门由黄河延伸到上朐,并以咸阳和东门为中轴线规划新版图。}。
  \end{enumerate} \tabularnewline\hline
  二八年 & -219 & \begin{enumerate}
    \tiny
  \item 徐福\footnote{徐福,即徐巿”(在秦始皇本纪中称“徐巿”,在淮南衡山列传中称“徐福”)。(注意,是“巿”〔fú〕而不是“市”〔shì 〕),字君房,秦朝时齐地人,当时的著名方士。}出海。
  \item 始皇泰山封禅。
  \end{enumerate} \tabularnewline\hline
  二九年 & -218 & \begin{enumerate}
    \tiny
  \item 秦始皇第三次巡游,张良在博浪沙击始皇未中。
  \item 秦征岭南\footnote{尉佗真定人。公元前218年,奉秦始皇命令征岭南,略定南越后,任为南海龙川令。高后五年自立, 僭号“南越武帝”。 尉佗(?-前137年),真定(今石家庄市东古城)人。公元前218年,奉秦始皇命令征岭南,略定南越后,任为南海郡(治所在今广州市)龙川(今广档龙川县)令。秦二世时,赵佗受南海尉任嚣托,行南海尉事。秦亡后,出兵击并桂林郡( 治所在今广西桂平县西南古城)、象郡(治所在今广西崇左县),自立为南越王, 实行“和揖百越”的民族平等政策,采取一系列措施发展当地经济文化。}。
  \item 西瓯国反秦\footnote{公元前218年,西江中部的“西瓯国”起兵反秦,秦始皇派50万大军征讨。又派史禄在海阳山开凿灵渠,将湘江与漓江沟通,以保证军事上的运输。灵渠便成为中原汉人进入岭南的第一条主要通道。秦始皇灭了西瓯国,战争告一段落,秦“发诸尝捕亡人、赘婿、贾人略取陆梁地,为桂林、象郡、南海,以适遣戍。 ”(《史记.秦始皇本纪》)“五十万人守五岭。”(《集解》)这50万人,便是第一批汉族移民。秦始皇搞大迁徙,目的在于铲除六国的地方势力,把族人和故土分开,交叉汇编,徙到南蛮之地戍边,也就连根拔起,使之不能在秦的京城附近形成威胁,兹生复国复旧之梦。}。
  \end{enumerate} \tabularnewline\hline
  三十年 & -217 & \begin{enumerate}
    \tiny
  \item 始修建长城\footnote{秦灭六国之后,即开始北筑长城,每年征发民夫四十余万。全长7000多千米的长城,称作“九边重镇”,每镇设总兵官作为这一段长城的军事长官,受兵部的指挥,负责所辖军区内的防务或奉命支援相邻军区的防务。}。
  \end{enumerate} \tabularnewline\hline
  三一年 & -216 & \begin{enumerate}
    \tiny
  \item 秦改革屯田制\footnote{平民自报所占土地面积,自报耕地面积、土地产量及大小人丁。所报内容由乡出人审查核实,并统一评定产量,计算每户应纳税额,最后登记入册,上报到县,经批准后,即按登记数征收。此前著名的改革家商鞅还在秦国推行了包括土地制度在内的改革。提出了“算地”和“定分”的主张。“算地”就是对土地进行全面的调查核算,以作为制定土地政策的客观依据;“定分”就是用法律形式确认地主或平民对土地占有的“名分”,确认土地所有权。这些实际上都是土地登记的内容。}。
  \item 始皇微行咸阳,兰池遇盗,武士击杀之。大索二十日。
  \item 西汉七国之乱主谋,刘邦之侄,吴王刘濞出生。
  \end{enumerate} \tabularnewline\hline
  三二年 & -215 & \begin{enumerate}
    \tiny
  \item 始皇在今广西等地建立了桂林郡和象郡。
  \item 始皇东巡到达蓟城。
  \item 秦将蒙恬筑马邑城池,置马邑县。
  \end{enumerate} \tabularnewline\hline
  三三年 & -214 & \begin{enumerate}
    \tiny
  \item 灵渠建成。
  \item 秦设龙川县。
  \item 秦设南海郡。
  \item 秦占岭南,夺高阙、阳山、北假\footnote{公元前214年,秦始皇派遣50万军队分5路攻占岭南,任命任嚣为南海尉。派蒙恬渡过黄河去夺取高阙、阳山、北假一带地方,筑起堡垒以驱逐戎狄。迁移被贬谪的人,让他们充实新设置的县。}。
  \end{enumerate} \tabularnewline\hline
  三四年 & -213 & \begin{enumerate}
    \tiny
  \item 李斯任左丞相。
  \item 淳于越谏秦。
  \item 焚书事件。
  \item 秦颁行《挟书令》。
  \item 秦在五岭开山道筑三关,即横浦关、阳山关、湟鸡谷关。
  \item 秦始修筑驰道。
  \end{enumerate} \tabularnewline\hline
  三五年 & -212 & \begin{enumerate}
    \tiny
  \item 修建阿房宫。
  \item 扶苏被派往上郡(今天的陕西绥德)做大将蒙恬的监军。
  \item 焚书坑儒。
  \item 蒙恬率领大军修建了一条从咸阳到九原(今内蒙古包头市)的直道。
  \end{enumerate} \tabularnewline\hline
  三六年 & -211 & \begin{enumerate}
    \tiny
  \item 陨石事件\footnote{秦始皇三十六年,一颗流星坠落到了东郡。东郡是在秦始皇即位之初吕不韦主政时攻打下来的,当时此郡是齐、秦两国的交界地。现在已是大秦帝国的一个东方大郡。陨石落地还不可怕,可怕的是陨石上面刻的字“始皇帝死而地分”。这七个字非同小可!它代表了上天的旨意,预示着秦始皇将死,同时也预告了大秦帝国将亡。}。
  \item 汉惠帝刘盈出生。
  \item 秦置皮氏县。
  \end{enumerate} \tabularnewline\hline
  三七年 & -210 & \begin{enumerate}
    \tiny
  \item 始皇卒\footnote{秦始皇三十七年(公元前210年),秦始皇出巡至平原津(今德州平原县南六十里有张公故城,城东有水津)而病,秦始皇不愿意听到“死”,所以群臣莫敢言死事。8月28日行至沙丘(沙丘台在邢州平乡县东北二十里)病死。}。
  \item 扶苏被害。
  \item 胡亥\footnote{秦二世胡亥(前230年—前207年,在位时间前209年—前207年),也称二世皇帝。是秦始皇第二十六子,公子扶苏的弟弟。秦始皇出游南方病死途中时,在赵高与李斯的帮助下,杀害哥哥扶苏当上秦朝的二世皇帝。贾谊《过秦论》曰:“始皇既没,胡亥极愚,郦山未毕,复作阿房,以遂前策。云“凡所为贵有天下者,肆意极欲,大臣至欲罢先君所为”。诛斯、去疾,任用赵高。痛哉言乎!人头畜鸣。不威不伐恶,不笃不虚亡。距之不得留,残虐以促期,虽居形便之国,犹不得存。”}称帝,是为秦二世。
  \end{enumerate} \tabularnewline
  \bottomrule
\end{longtable}


%%% Local Variables:
%%% mode: latex
%%% TeX-engine: xetex
%%% TeX-master: "../Main"
%%% End:

%% -*- coding: utf-8 -*-
%% Time-stamp: <Chen Wang: 2018-07-10 17:29:32>

\section{秦二世\tiny(BC209-BC207)}

\begin{longtable}{|>{\centering\scriptsize}m{2em}|>{\centering\scriptsize}m{1.3em}|>{\centering}m{8.8em}|}
  % \caption{秦王政}\
  \toprule
  \SimHei \normalsize 年数 & \SimHei \scriptsize 公元 & \SimHei 大事件 \tabularnewline
  % \midrule
  \endfirsthead
  \toprule
  \SimHei \normalsize 年数 & \SimHei \scriptsize 公元 & \SimHei 大事件 \tabularnewline
  \midrule
  \endhead
  \midrule
  元年 & -209 & \begin{enumerate}
    \tiny
  \item 大泽乡起义。
  \item 刘邦起义。
  \item 项羽反秦。
  \item 冒顿即位。
  \end{enumerate} \tabularnewline\hline
  二年 & -208 & \begin{enumerate}
    \tiny
  \item 秦灭项梁。
  \item 孔鲋逝世。
  \item 陈胜卒。
  \item 李斯卒。
  \item 薛地会议。
  \item 统一越南。
  \end{enumerate} \tabularnewline\hline
  三年 & -207 & \begin{enumerate}
    \tiny
  \item 指鹿为马。
  \item 破釜沉舟。
  \item 胡亥被弑。
  \item 子婴即位,诛赵高,在位47天被废。
  \end{enumerate} \tabularnewline
  \bottomrule
\end{longtable}


%%% Local Variables:
%%% mode: latex
%%% TeX-engine: xetex
%%% TeX-master: "../Main"
%%% End:

%% -*- coding: utf-8 -*-
%% Time-stamp: <Chen Wang: 2018-07-10 17:29:53>

\section{子婴\tiny(BC206-BC206)}

\begin{longtable}{|>{\centering\scriptsize}m{2em}|>{\centering\scriptsize}m{1.3em}|>{\centering}m{8.8em}|}
  % \caption{秦王政}\
  \toprule
  \SimHei \normalsize 年数 & \SimHei \scriptsize 公元 & \SimHei 大事件 \tabularnewline
  % \midrule
  \endfirsthead
  \toprule
  \SimHei \normalsize 年数 & \SimHei \scriptsize 公元 & \SimHei 大事件 \tabularnewline
  \midrule
  \endhead
  \midrule
  元年 & -206 & \tabularnewline
  \bottomrule
\end{longtable}


%%% Local Variables:
%%% mode: latex
%%% TeX-engine: xetex
%%% TeX-master: "../Main"
%%% End:


%%% Local Variables:
%%% mode: latex
%%% TeX-engine: xetex
%%% TeX-master: "../Main"
%%% End:
 % 秦
% %% -*- coding: utf-8 -*-
%% Time-stamp: <Chen Wang: 2018-07-10 19:31:27>

\chapter{西汉\tiny(BC202-8)}

%% -*- coding: utf-8 -*-
%% Time-stamp: <Chen Wang: 2018-07-10 17:28:38>

\section{楚汉之争\tiny(BC206-BC203)}

\begin{longtable}{|>{\centering\scriptsize}m{2em}|>{\centering\scriptsize}m{1.3em}|>{\centering}m{8.8em}|}
  % \caption{秦王政}\
  \toprule
  \SimHei \normalsize 年数 & \SimHei \scriptsize 公元 & \SimHei 大事件 \tabularnewline
  % \midrule
  \endfirsthead
  \toprule
  \SimHei \normalsize 年数 & \SimHei \scriptsize 公元 & \SimHei 大事件 \tabularnewline
  \midrule
  \endhead
  \midrule
  高祖\\元年 & -206 & \begin{enumerate}
    \tiny
  \item 秦朝灭亡。
  \item 鸿门宴。
  \item 项羽建立西楚王朝,自称西楚霸王。
  \end{enumerate} \tabularnewline\hline
  二年 & -205 & \begin{enumerate}
    \tiny
  \item 彭城之战。
  \item 成皋之战。
  \item 韩信破代、赵。
  \item 韩信灭燕、齐。
  \end{enumerate} \tabularnewline\hline
  三年 & -204 & \begin{enumerate}
    \tiny
  \item 背水一战。
  \item 南越国建立。
  \item 成皋之战。
  \end{enumerate} \tabularnewline\hline
  四年 & -203 & \begin{enumerate}
    \tiny
  \item 英布封王。
  \item 张耳封王。
  \end{enumerate} \tabularnewline
  \bottomrule
\end{longtable}


%%% Local Variables:
%%% mode: latex
%%% TeX-engine: xetex
%%% TeX-master: "../Main"
%%% End:

%% -*- coding: utf-8 -*-
%% Time-stamp: <Chen Wang: 2018-07-10 17:28:48>

\section{汉高祖\tiny(BC206-BC195)}

\begin{longtable}{|>{\centering\scriptsize}m{2em}|>{\centering\scriptsize}m{1.3em}|>{\centering}m{8.8em}|}
  % \caption{秦王政}\
  \toprule
  \SimHei \normalsize 年数 & \SimHei \scriptsize 公元 & \SimHei 大事件 \tabularnewline
  % \midrule
  \endfirsthead
  \toprule
  \SimHei \normalsize 年数 & \SimHei \scriptsize 公元 & \SimHei 大事件 \tabularnewline
  \midrule
  \endhead
  \midrule
  五年 & -202 & \begin{enumerate}
    \tiny
  \item 十二月垓下之战,汉灭楚统一天下,汉王刘邦即皇帝位。
  \item 汉置长安县、无锡县。
  \item 七月,燕王臧荼起兵反汉。
  \item 十月,刘邦率军亲征灭燕,俘杀臧荼。刘邦立卢绾为燕王。
  \item 汉高祖册封无诸为闽越王,封国闽越,首都冶城位于今之福州。
  \end{enumerate} \tabularnewline\hline
  六年 & -201 & \tabularnewline\hline
  七年 & -200 & \tabularnewline\hline
  八年 & -199 & \tabularnewline\hline
  九年 & -198 & \tabularnewline\hline
  十年 & -197 & \tabularnewline\hline
  十一年 & -196 & \tabularnewline\hline
  十二年 & -195 & \tabularnewline
  \bottomrule
\end{longtable}


%%% Local Variables:
%%% mode: latex
%%% TeX-engine: xetex
%%% TeX-master: "../Main"
%%% End:

%% -*- coding: utf-8 -*-
%% Time-stamp: <Chen Wang: 2018-07-10 17:29:04>

\section{孝惠帝\tiny(BC195-BC188)}

\begin{longtable}{|>{\centering\scriptsize}m{2em}|>{\centering\scriptsize}m{1.3em}|>{\centering}m{8.8em}|}
  % \caption{秦王政}\
  \toprule
  \SimHei \normalsize 年数 & \SimHei \scriptsize 公元 & \SimHei 大事件 \tabularnewline
  % \midrule
  \endfirsthead
  \toprule
  \SimHei \normalsize 年数 & \SimHei \scriptsize 公元 & \SimHei 大事件 \tabularnewline
  \midrule
  \endhead
  \midrule
  元年 & -194 & \tabularnewline\hline
  二年 & -193 & \tabularnewline\hline
  三年 & -192 & \tabularnewline\hline
  四年 & -191 & \tabularnewline\hline
  五年 & -190 & \tabularnewline\hline
  六年 & -189 & \tabularnewline\hline
  七年 & -188 & \tabularnewline
  \bottomrule
\end{longtable}


%%% Local Variables:
%%% mode: latex
%%% TeX-engine: xetex
%%% TeX-master: "../Main"
%%% End:

%% -*- coding: utf-8 -*-
%% Time-stamp: <Chen Wang: 2018-07-10 17:28:59>

\section{前少帝\tiny(BC187-BC184)}

\begin{longtable}{|>{\centering\scriptsize}m{2em}|>{\centering\scriptsize}m{1.3em}|>{\centering}m{8.8em}|}
  % \caption{秦王政}\
  \toprule
  \SimHei \normalsize 年数 & \SimHei \scriptsize 公元 & \SimHei 大事件 \tabularnewline
  % \midrule
  \endfirsthead
  \toprule
  \SimHei \normalsize 年数 & \SimHei \scriptsize 公元 & \SimHei 大事件 \tabularnewline
  \midrule
  \endhead
  \midrule
  元年 & -187 & \tabularnewline\hline
  二年 & -186 & \tabularnewline\hline
  三年 & -185 & \tabularnewline\hline
  四年 & -184 & \tabularnewline
  \bottomrule
\end{longtable}


%%% Local Variables:
%%% mode: latex
%%% TeX-engine: xetex
%%% TeX-master: "../Main"
%%% End:

%% -*- coding: utf-8 -*-
%% Time-stamp: <Chen Wang: 2018-07-10 17:28:54>

\section{后少帝\tiny(BC183-BC180)}

\begin{longtable}{|>{\centering\scriptsize}m{2em}|>{\centering\scriptsize}m{1.3em}|>{\centering}m{8.8em}|}
  % \caption{秦王政}\
  \toprule
  \SimHei \normalsize 年数 & \SimHei \scriptsize 公元 & \SimHei 大事件 \tabularnewline
  % \midrule
  \endfirsthead
  \toprule
  \SimHei \normalsize 年数 & \SimHei \scriptsize 公元 & \SimHei 大事件 \tabularnewline
  \midrule
  \endhead
  \midrule
  元年 & -183 & \tabularnewline\hline
  二年 & -182 & \tabularnewline\hline
  三年 & -181 & \tabularnewline\hline
  四年 & -180 & \tabularnewline
  \bottomrule
\end{longtable}


%%% Local Variables:
%%% mode: latex
%%% TeX-engine: xetex
%%% TeX-master: "../Main"
%%% End:

%% -*- coding: utf-8 -*-
%% Time-stamp: <Chen Wang: 2018-07-10 17:28:15>

\section{孝文帝\tiny(BC179-BC157)}

\subsection{前元}

\begin{longtable}{|>{\centering\scriptsize}m{2em}|>{\centering\scriptsize}m{1.3em}|>{\centering}m{8.8em}|}
  % \caption{秦王政}\
  \toprule
  \SimHei \normalsize 年数 & \SimHei \scriptsize 公元 & \SimHei 大事件 \tabularnewline
  % \midrule
  \endfirsthead
  \toprule
  \SimHei \normalsize 年数 & \SimHei \scriptsize 公元 & \SimHei 大事件 \tabularnewline
  \midrule
  \endhead
  \midrule
  元年 & -179 & \tabularnewline\hline
  二年 & -178 & \tabularnewline\hline
  三年 & -177 & \tabularnewline\hline
  四年 & -176 & \tabularnewline\hline
  五年 & -175 & \tabularnewline\hline
  六年 & -174 & \tabularnewline\hline
  七年 & -173 & \tabularnewline\hline
  八年 & -172 & \tabularnewline\hline
  九年 & -171 & \tabularnewline\hline
  十年 & -170 & \tabularnewline\hline
  十一年 & -169 & \tabularnewline\hline
  十二年 & -168 & \tabularnewline\hline
  十三年 & -167 & \tabularnewline\hline
  十四年 & -166 & \tabularnewline\hline
  十五年 & -165 & \tabularnewline\hline
  十六年 & -164 & \tabularnewline
  \bottomrule
\end{longtable}


\subsection{后元}

\begin{longtable}{|>{\centering\scriptsize}m{2em}|>{\centering\scriptsize}m{1.3em}|>{\centering}m{8.8em}|}
  % \caption{秦王政}\
  \toprule
  \SimHei \normalsize 年数 & \SimHei \scriptsize 公元 & \SimHei 大事件 \tabularnewline
  % \midrule
  \endfirsthead
  \toprule
  \SimHei \normalsize 年数 & \SimHei \scriptsize 公元 & \SimHei 大事件 \tabularnewline
  \midrule
  \endhead
  \midrule
  元年 & -163 & \tabularnewline\hline
  二年 & -162 & \tabularnewline\hline
  三年 & -161 & \tabularnewline\hline
  四年 & -160 & \tabularnewline\hline
  五年 & -159 & \tabularnewline\hline
  六年 & -158 & \tabularnewline\hline
  七年 & -157 & \tabularnewline
  \bottomrule
\end{longtable}


%%% Local Variables:
%%% mode: latex
%%% TeX-engine: xetex
%%% TeX-master: "../Main"
%%% End:

%% -*- coding: utf-8 -*-
%% Time-stamp: <Chen Wang: 2018-07-10 17:44:44>

\section{孝景帝\tiny(BC156-BC141)}

\subsection{前元}

\begin{longtable}{|>{\centering\scriptsize}m{2em}|>{\centering\scriptsize}m{1.3em}|>{\centering}m{8.8em}|}
  % \caption{秦王政}\
  \toprule
  \SimHei \normalsize 年数 & \SimHei \scriptsize 公元 & \SimHei 大事件 \tabularnewline
  % \midrule
  \endfirsthead
  \toprule
  \SimHei \normalsize 年数 & \SimHei \scriptsize 公元 & \SimHei 大事件 \tabularnewline
  \midrule
  \endhead
  \midrule
  元年 & -156 & \tabularnewline\hline
  二年 & -155 & \tabularnewline\hline
  三年 & -154 & \tabularnewline\hline
  四年 & -153 & \tabularnewline\hline
  五年 & -152 & \tabularnewline\hline
  六年 & -151 & \tabularnewline\hline
  七年 & -150 & \tabularnewline
  \bottomrule
\end{longtable}


\subsection{中元}

\begin{longtable}{|>{\centering\scriptsize}m{2em}|>{\centering\scriptsize}m{1.3em}|>{\centering}m{8.8em}|}
  % \caption{秦王政}\
  \toprule
  \SimHei \normalsize 年数 & \SimHei \scriptsize 公元 & \SimHei 大事件 \tabularnewline
  % \midrule
  \endfirsthead
  \toprule
  \SimHei \normalsize 年数 & \SimHei \scriptsize 公元 & \SimHei 大事件 \tabularnewline
  \midrule
  \endhead
  \midrule
  元年 & -149 & \tabularnewline\hline
  二年 & -148 & \tabularnewline\hline
  三年 & -147 & \tabularnewline\hline
  四年 & -146 & \tabularnewline\hline
  五年 & -145 & \tabularnewline\hline
  六年 & -144 & \tabularnewline
  \bottomrule
\end{longtable}


\subsection{后元}

\begin{longtable}{|>{\centering\scriptsize}m{2em}|>{\centering\scriptsize}m{1.3em}|>{\centering}m{8.8em}|}
  % \caption{秦王政}\
  \toprule
  \SimHei \normalsize 年数 & \SimHei \scriptsize 公元 & \SimHei 大事件 \tabularnewline
  % \midrule
  \endfirsthead
  \toprule
  \SimHei \normalsize 年数 & \SimHei \scriptsize 公元 & \SimHei 大事件 \tabularnewline
  \midrule
  \endhead
  \midrule
  元年 & -143 & \tabularnewline\hline
  二年 & -142 & \tabularnewline\hline
  三年 & -141 & \tabularnewline
  \bottomrule
\end{longtable}


%%% Local Variables:
%%% mode: latex
%%% TeX-engine: xetex
%%% TeX-master: "../Main"
%%% End:

%% -*- coding: utf-8 -*-
%% Time-stamp: <Chen Wang: 2018-07-10 18:54:22>

\section{武帝\tiny(BC140-BC87)}

\subsection{建元}

\begin{longtable}{|>{\centering\scriptsize}m{2em}|>{\centering\scriptsize}m{1.3em}|>{\centering}m{8.8em}|}
  % \caption{秦王政}\
  \toprule
  \SimHei \normalsize 年数 & \SimHei \scriptsize 公元 & \SimHei 大事件 \tabularnewline
  % \midrule
  \endfirsthead
  \toprule
  \SimHei \normalsize 年数 & \SimHei \scriptsize 公元 & \SimHei 大事件 \tabularnewline
  \midrule
  \endhead
  \midrule
  元年 & -140 & \tabularnewline\hline
  二年 & -139 & \tabularnewline\hline
  三年 & -138 & \tabularnewline\hline
  四年 & -137 & \tabularnewline\hline
  五年 & -136 & \tabularnewline\hline
  六年 & -135 & \tabularnewline
  \bottomrule
\end{longtable}


\subsection{元光}

\begin{longtable}{|>{\centering\scriptsize}m{2em}|>{\centering\scriptsize}m{1.3em}|>{\centering}m{8.8em}|}
  % \caption{秦王政}\
  \toprule
  \SimHei \normalsize 年数 & \SimHei \scriptsize 公元 & \SimHei 大事件 \tabularnewline
  % \midrule
  \endfirsthead
  \toprule
  \SimHei \normalsize 年数 & \SimHei \scriptsize 公元 & \SimHei 大事件 \tabularnewline
  \midrule
  \endhead
  \midrule
  元年 & -134 & \tabularnewline\hline
  二年 & -133 & \tabularnewline\hline
  三年 & -132 & \tabularnewline\hline
  四年 & -131 & \tabularnewline\hline
  五年 & -130 & \tabularnewline\hline
  六年 & -129 & \tabularnewline
  \bottomrule
\end{longtable}


\subsection{元朔}

\begin{longtable}{|>{\centering\scriptsize}m{2em}|>{\centering\scriptsize}m{1.3em}|>{\centering}m{8.8em}|}
  % \caption{秦王政}\
  \toprule
  \SimHei \normalsize 年数 & \SimHei \scriptsize 公元 & \SimHei 大事件 \tabularnewline
  % \midrule
  \endfirsthead
  \toprule
  \SimHei \normalsize 年数 & \SimHei \scriptsize 公元 & \SimHei 大事件 \tabularnewline
  \midrule
  \endhead
  \midrule
  元年 & -128 & \tabularnewline\hline
  二年 & -127 & \tabularnewline\hline
  三年 & -126 & \tabularnewline\hline
  四年 & -125 & \tabularnewline\hline
  五年 & -124 & \tabularnewline\hline
  六年 & -123 & \tabularnewline
  \bottomrule
\end{longtable}

\subsection{元狩}

\begin{longtable}{|>{\centering\scriptsize}m{2em}|>{\centering\scriptsize}m{1.3em}|>{\centering}m{8.8em}|}
  % \caption{秦王政}\
  \toprule
  \SimHei \normalsize 年数 & \SimHei \scriptsize 公元 & \SimHei 大事件 \tabularnewline
  % \midrule
  \endfirsthead
  \toprule
  \SimHei \normalsize 年数 & \SimHei \scriptsize 公元 & \SimHei 大事件 \tabularnewline
  \midrule
  \endhead
  \midrule
  元年 & -122 & \tabularnewline\hline
  二年 & -121 & \tabularnewline\hline
  三年 & -120 & \tabularnewline\hline
  四年 & -119 & \tabularnewline\hline
  五年 & -118 & \tabularnewline\hline
  六年 & -117 & \tabularnewline  
  \bottomrule
\end{longtable}

\subsection{元鼎}

\begin{longtable}{|>{\centering\scriptsize}m{2em}|>{\centering\scriptsize}m{1.3em}|>{\centering}m{8.8em}|}
  % \caption{秦王政}\
  \toprule
  \SimHei \normalsize 年数 & \SimHei \scriptsize 公元 & \SimHei 大事件 \tabularnewline
  % \midrule
  \endfirsthead
  \toprule
  \SimHei \normalsize 年数 & \SimHei \scriptsize 公元 & \SimHei 大事件 \tabularnewline
  \midrule
  \endhead
  \midrule
  元年 & -116 & \tabularnewline\hline
  二年 & -115 & \tabularnewline\hline
  三年 & -114 & \tabularnewline\hline
  四年 & -113 & \tabularnewline\hline
  五年 & -112 & \tabularnewline\hline
  六年 & -111 & \tabularnewline  
  \bottomrule
\end{longtable}

\subsection{元封}

\begin{longtable}{|>{\centering\scriptsize}m{2em}|>{\centering\scriptsize}m{1.3em}|>{\centering}m{8.8em}|}
  % \caption{秦王政}\
  \toprule
  \SimHei \normalsize 年数 & \SimHei \scriptsize 公元 & \SimHei 大事件 \tabularnewline
  % \midrule
  \endfirsthead
  \toprule
  \SimHei \normalsize 年数 & \SimHei \scriptsize 公元 & \SimHei 大事件 \tabularnewline
  \midrule
  \endhead
  \midrule
  元年 & -110 & \tabularnewline\hline
  二年 & -109 & \tabularnewline\hline
  三年 & -108 & \tabularnewline\hline
  四年 & -107 & \tabularnewline\hline
  五年 & -106 & \tabularnewline\hline
  六年 & -105 & \tabularnewline
  \bottomrule
\end{longtable}

\subsection{太初}

\begin{longtable}{|>{\centering\scriptsize}m{2em}|>{\centering\scriptsize}m{1.3em}|>{\centering}m{8.8em}|}
  % \caption{秦王政}\
  \toprule
  \SimHei \normalsize 年数 & \SimHei \scriptsize 公元 & \SimHei 大事件 \tabularnewline
  % \midrule
  \endfirsthead
  \toprule
  \SimHei \normalsize 年数 & \SimHei \scriptsize 公元 & \SimHei 大事件 \tabularnewline
  \midrule
  \endhead
  \midrule
  元年 & -104 & \tabularnewline\hline
  二年 & -103 & \tabularnewline\hline
  三年 & -102 & \tabularnewline\hline
  四年 & -101 & \tabularnewline
  \bottomrule
\end{longtable}

\subsection{天汉}

\begin{longtable}{|>{\centering\scriptsize}m{2em}|>{\centering\scriptsize}m{1.3em}|>{\centering}m{8.8em}|}
  % \caption{秦王政}\
  \toprule
  \SimHei \normalsize 年数 & \SimHei \scriptsize 公元 & \SimHei 大事件 \tabularnewline
  % \midrule
  \endfirsthead
  \toprule
  \SimHei \normalsize 年数 & \SimHei \scriptsize 公元 & \SimHei 大事件 \tabularnewline
  \midrule
  \endhead
  \midrule
  元年 & -100 & \tabularnewline\hline
  二年 & -99 & \tabularnewline\hline
  三年 & -98 & \tabularnewline\hline
  四年 & -97 & \tabularnewline
  \bottomrule
\end{longtable}

\subsection{太始}

\begin{longtable}{|>{\centering\scriptsize}m{2em}|>{\centering\scriptsize}m{1.3em}|>{\centering}m{8.8em}|}
  % \caption{秦王政}\
  \toprule
  \SimHei \normalsize 年数 & \SimHei \scriptsize 公元 & \SimHei 大事件 \tabularnewline
  % \midrule
  \endfirsthead
  \toprule
  \SimHei \normalsize 年数 & \SimHei \scriptsize 公元 & \SimHei 大事件 \tabularnewline
  \midrule
  \endhead
  \midrule
  元年 & -96 & \tabularnewline\hline
  二年 & -95 & \tabularnewline\hline
  三年 & -94 & \tabularnewline\hline
  四年 & -93 & \tabularnewline
  \bottomrule
\end{longtable}

\subsection{征和}

\begin{longtable}{|>{\centering\scriptsize}m{2em}|>{\centering\scriptsize}m{1.3em}|>{\centering}m{8.8em}|}
  % \caption{秦王政}\
  \toprule
  \SimHei \normalsize 年数 & \SimHei \scriptsize 公元 & \SimHei 大事件 \tabularnewline
  % \midrule
  \endfirsthead
  \toprule
  \SimHei \normalsize 年数 & \SimHei \scriptsize 公元 & \SimHei 大事件 \tabularnewline
  \midrule
  \endhead
  \midrule
  元年 & -92 & \tabularnewline\hline
  二年 & -91 & \tabularnewline\hline
  三年 & -90 & \tabularnewline\hline
  四年 & -89 & \tabularnewline
  \bottomrule
\end{longtable}

\subsection{后元}

\begin{longtable}{|>{\centering\scriptsize}m{2em}|>{\centering\scriptsize}m{1.3em}|>{\centering}m{8.8em}|}
  % \caption{秦王政}\
  \toprule
  \SimHei \normalsize 年数 & \SimHei \scriptsize 公元 & \SimHei 大事件 \tabularnewline
  % \midrule
  \endfirsthead
  \toprule
  \SimHei \normalsize 年数 & \SimHei \scriptsize 公元 & \SimHei 大事件 \tabularnewline
  \midrule
  \endhead
  \midrule
  元年 & -88 & \tabularnewline\hline
  二年 & -87 & \tabularnewline
  \bottomrule
\end{longtable}


%%% Local Variables:
%%% mode: latex
%%% TeX-engine: xetex
%%% TeX-master: "../Main"
%%% End:

%% -*- coding: utf-8 -*-
%% Time-stamp: <Chen Wang: 2018-07-10 18:58:40>

\section{昭帝\tiny(BC87-BC74)}

\subsection{始元}

\begin{longtable}{|>{\centering\scriptsize}m{2em}|>{\centering\scriptsize}m{1.3em}|>{\centering}m{8.8em}|}
  % \caption{秦王政}\
  \toprule
  \SimHei \normalsize 年数 & \SimHei \scriptsize 公元 & \SimHei 大事件 \tabularnewline
  % \midrule
  \endfirsthead
  \toprule
  \SimHei \normalsize 年数 & \SimHei \scriptsize 公元 & \SimHei 大事件 \tabularnewline
  \midrule
  \endhead
  \midrule
  元年 & -86 & \tabularnewline\hline
  二年 & -85 & \tabularnewline\hline
  三年 & -84 & \tabularnewline\hline
  四年 & -83 & \tabularnewline\hline
  五年 & -82 & \tabularnewline\hline
  六年 & -81 & \tabularnewline\hline
  七年 & -80 & \tabularnewline
  \bottomrule
\end{longtable}


\subsection{元凤}

\begin{longtable}{|>{\centering\scriptsize}m{2em}|>{\centering\scriptsize}m{1.3em}|>{\centering}m{8.8em}|}
  % \caption{秦王政}\
  \toprule
  \SimHei \normalsize 年数 & \SimHei \scriptsize 公元 & \SimHei 大事件 \tabularnewline
  % \midrule
  \endfirsthead
  \toprule
  \SimHei \normalsize 年数 & \SimHei \scriptsize 公元 & \SimHei 大事件 \tabularnewline
  \midrule
  \endhead
  \midrule
  元年 & -80 & \tabularnewline\hline
  二年 & -79 & \tabularnewline\hline
  三年 & -78 & \tabularnewline\hline
  四年 & -77 & \tabularnewline\hline
  五年 & -76 & \tabularnewline\hline
  六年 & -75 & \tabularnewline
  \bottomrule
\end{longtable}


\subsection{元平}

\begin{longtable}{|>{\centering\scriptsize}m{2em}|>{\centering\scriptsize}m{1.3em}|>{\centering}m{8.8em}|}
  % \caption{秦王政}\
  \toprule
  \SimHei \normalsize 年数 & \SimHei \scriptsize 公元 & \SimHei 大事件 \tabularnewline
  % \midrule
  \endfirsthead
  \toprule
  \SimHei \normalsize 年数 & \SimHei \scriptsize 公元 & \SimHei 大事件 \tabularnewline
  \midrule
  \endhead
  \midrule
  元年 & -74 & \tabularnewline
  \bottomrule
\end{longtable}


%%% Local Variables:
%%% mode: latex
%%% TeX-engine: xetex
%%% TeX-master: "../Main"
%%% End:

%% -*- coding: utf-8 -*-
%% Time-stamp: <Chen Wang: 2018-07-10 19:05:08>

\section{宣帝\tiny(BC74-BC49)}

\subsection{本始}

\begin{longtable}{|>{\centering\scriptsize}m{2em}|>{\centering\scriptsize}m{1.3em}|>{\centering}m{8.8em}|}
  % \caption{秦王政}\
  \toprule
  \SimHei \normalsize 年数 & \SimHei \scriptsize 公元 & \SimHei 大事件 \tabularnewline
  % \midrule
  \endfirsthead
  \toprule
  \SimHei \normalsize 年数 & \SimHei \scriptsize 公元 & \SimHei 大事件 \tabularnewline
  \midrule
  \endhead
  \midrule
  元年 & -73 & \tabularnewline\hline
  二年 & -72 & \tabularnewline\hline
  三年 & -71 & \tabularnewline\hline
  四年 & -70 & \tabularnewline
  \bottomrule
\end{longtable}


\subsection{地节}

\begin{longtable}{|>{\centering\scriptsize}m{2em}|>{\centering\scriptsize}m{1.3em}|>{\centering}m{8.8em}|}
  % \caption{秦王政}\
  \toprule
  \SimHei \normalsize 年数 & \SimHei \scriptsize 公元 & \SimHei 大事件 \tabularnewline
  % \midrule
  \endfirsthead
  \toprule
  \SimHei \normalsize 年数 & \SimHei \scriptsize 公元 & \SimHei 大事件 \tabularnewline
  \midrule
  \endhead
  \midrule
  元年 & -69 & \tabularnewline\hline
  二年 & -68 & \tabularnewline\hline
  三年 & -67 & \tabularnewline\hline
  四年 & -66 & \tabularnewline
  \bottomrule
\end{longtable}


\subsection{元康}

\begin{longtable}{|>{\centering\scriptsize}m{2em}|>{\centering\scriptsize}m{1.3em}|>{\centering}m{8.8em}|}
  % \caption{秦王政}\
  \toprule
  \SimHei \normalsize 年数 & \SimHei \scriptsize 公元 & \SimHei 大事件 \tabularnewline
  % \midrule
  \endfirsthead
  \toprule
  \SimHei \normalsize 年数 & \SimHei \scriptsize 公元 & \SimHei 大事件 \tabularnewline
  \midrule
  \endhead
  \midrule
  元年 & -65 & \tabularnewline\hline
  二年 & -64 & \tabularnewline\hline
  三年 & -63 & \tabularnewline\hline
  四年 & -62 & \tabularnewline
  \bottomrule
\end{longtable}

\subsection{神爵}

\begin{longtable}{|>{\centering\scriptsize}m{2em}|>{\centering\scriptsize}m{1.3em}|>{\centering}m{8.8em}|}
  % \caption{秦王政}\
  \toprule
  \SimHei \normalsize 年数 & \SimHei \scriptsize 公元 & \SimHei 大事件 \tabularnewline
  % \midrule
  \endfirsthead
  \toprule
  \SimHei \normalsize 年数 & \SimHei \scriptsize 公元 & \SimHei 大事件 \tabularnewline
  \midrule
  \endhead
  \midrule
  元年 & -61 & \tabularnewline\hline
  二年 & -60 & \tabularnewline\hline
  三年 & -59 & \tabularnewline\hline
  四年 & -58 & \tabularnewline
  \bottomrule
\end{longtable}

\subsection{五凤}

\begin{longtable}{|>{\centering\scriptsize}m{2em}|>{\centering\scriptsize}m{1.3em}|>{\centering}m{8.8em}|}
  % \caption{秦王政}\
  \toprule
  \SimHei \normalsize 年数 & \SimHei \scriptsize 公元 & \SimHei 大事件 \tabularnewline
  % \midrule
  \endfirsthead
  \toprule
  \SimHei \normalsize 年数 & \SimHei \scriptsize 公元 & \SimHei 大事件 \tabularnewline
  \midrule
  \endhead
  \midrule
  元年 & -57 & \tabularnewline\hline
  二年 & -56 & \tabularnewline\hline
  三年 & -55 & \tabularnewline\hline
  四年 & -54 & \tabularnewline
  \bottomrule
\end{longtable}

\subsection{甘露}

\begin{longtable}{|>{\centering\scriptsize}m{2em}|>{\centering\scriptsize}m{1.3em}|>{\centering}m{8.8em}|}
  % \caption{秦王政}\
  \toprule
  \SimHei \normalsize 年数 & \SimHei \scriptsize 公元 & \SimHei 大事件 \tabularnewline
  % \midrule
  \endfirsthead
  \toprule
  \SimHei \normalsize 年数 & \SimHei \scriptsize 公元 & \SimHei 大事件 \tabularnewline
  \midrule
  \endhead
  \midrule
  元年 & -53 & \tabularnewline\hline
  二年 & -52 & \tabularnewline\hline
  三年 & -51 & \tabularnewline\hline
  四年 & -50 & \tabularnewline
  \bottomrule
\end{longtable}


\subsection{黄龙}

\begin{longtable}{|>{\centering\scriptsize}m{2em}|>{\centering\scriptsize}m{1.3em}|>{\centering}m{8.8em}|}
  % \caption{秦王政}\
  \toprule
  \SimHei \normalsize 年数 & \SimHei \scriptsize 公元 & \SimHei 大事件 \tabularnewline
  % \midrule
  \endfirsthead
  \toprule
  \SimHei \normalsize 年数 & \SimHei \scriptsize 公元 & \SimHei 大事件 \tabularnewline
  \midrule
  \endhead
  \midrule
  元年 & -49 & \tabularnewline
  \bottomrule
\end{longtable}


%%% Local Variables:
%%% mode: latex
%%% TeX-engine: xetex
%%% TeX-master: "../Main"
%%% End:

%% -*- coding: utf-8 -*-
%% Time-stamp: <Chen Wang: 2018-07-10 19:07:56>

\section{元帝\tiny(BC48-BC33)}

\subsection{初元}

\begin{longtable}{|>{\centering\scriptsize}m{2em}|>{\centering\scriptsize}m{1.3em}|>{\centering}m{8.8em}|}
  % \caption{秦王政}\
  \toprule
  \SimHei \normalsize 年数 & \SimHei \scriptsize 公元 & \SimHei 大事件 \tabularnewline
  % \midrule
  \endfirsthead
  \toprule
  \SimHei \normalsize 年数 & \SimHei \scriptsize 公元 & \SimHei 大事件 \tabularnewline
  \midrule
  \endhead
  \midrule
  元年 & -48 & \tabularnewline\hline
  二年 & -47 & \tabularnewline\hline
  三年 & -46 & \tabularnewline\hline
  四年 & -45 & \tabularnewline\hline
  五年 & -44 & \tabularnewline
  \bottomrule
\end{longtable}


\subsection{永光}

\begin{longtable}{|>{\centering\scriptsize}m{2em}|>{\centering\scriptsize}m{1.3em}|>{\centering}m{8.8em}|}
  % \caption{秦王政}\
  \toprule
  \SimHei \normalsize 年数 & \SimHei \scriptsize 公元 & \SimHei 大事件 \tabularnewline
  % \midrule
  \endfirsthead
  \toprule
  \SimHei \normalsize 年数 & \SimHei \scriptsize 公元 & \SimHei 大事件 \tabularnewline
  \midrule
  \endhead
  \midrule
  元年 & -43 & \tabularnewline\hline
  二年 & -42 & \tabularnewline\hline
  三年 & -41 & \tabularnewline\hline
  四年 & -40 & \tabularnewline\hline
  五年 & -39 & \tabularnewline
  \bottomrule
\end{longtable}


\subsection{建昭}

\begin{longtable}{|>{\centering\scriptsize}m{2em}|>{\centering\scriptsize}m{1.3em}|>{\centering}m{8.8em}|}
  % \caption{秦王政}\
  \toprule
  \SimHei \normalsize 年数 & \SimHei \scriptsize 公元 & \SimHei 大事件 \tabularnewline
  % \midrule
  \endfirsthead
  \toprule
  \SimHei \normalsize 年数 & \SimHei \scriptsize 公元 & \SimHei 大事件 \tabularnewline
  \midrule
  \endhead
  \midrule
  元年 & -38 & \tabularnewline\hline
  二年 & -37 & \tabularnewline\hline
  三年 & -36 & \tabularnewline\hline
  四年 & -35 & \tabularnewline\hline
  五年 & -34 & \tabularnewline
  \bottomrule
\end{longtable}

\subsection{竟宁}

\begin{longtable}{|>{\centering\scriptsize}m{2em}|>{\centering\scriptsize}m{1.3em}|>{\centering}m{8.8em}|}
  % \caption{秦王政}\
  \toprule
  \SimHei \normalsize 年数 & \SimHei \scriptsize 公元 & \SimHei 大事件 \tabularnewline
  % \midrule
  \endfirsthead
  \toprule
  \SimHei \normalsize 年数 & \SimHei \scriptsize 公元 & \SimHei 大事件 \tabularnewline
  \midrule
  \endhead
  \midrule
  元年 & -33 & \tabularnewline
  \bottomrule
\end{longtable}


%%% Local Variables:
%%% mode: latex
%%% TeX-engine: xetex
%%% TeX-master: "../Main"
%%% End:

%% -*- coding: utf-8 -*-
%% Time-stamp: <Chen Wang: 2018-07-10 19:14:14>

\section{成帝\tiny(BC33-BC7)}

\subsection{建始}

\begin{longtable}{|>{\centering\scriptsize}m{2em}|>{\centering\scriptsize}m{1.3em}|>{\centering}m{8.8em}|}
  % \caption{秦王政}\
  \toprule
  \SimHei \normalsize 年数 & \SimHei \scriptsize 公元 & \SimHei 大事件 \tabularnewline
  % \midrule
  \endfirsthead
  \toprule
  \SimHei \normalsize 年数 & \SimHei \scriptsize 公元 & \SimHei 大事件 \tabularnewline
  \midrule
  \endhead
  \midrule
  元年 & -32 & \tabularnewline\hline
  二年 & -31 & \tabularnewline\hline
  三年 & -30 & \tabularnewline\hline
  四年 & -29 & \tabularnewline
  \bottomrule
\end{longtable}


\subsection{河平}

\begin{longtable}{|>{\centering\scriptsize}m{2em}|>{\centering\scriptsize}m{1.3em}|>{\centering}m{8.8em}|}
  % \caption{秦王政}\
  \toprule
  \SimHei \normalsize 年数 & \SimHei \scriptsize 公元 & \SimHei 大事件 \tabularnewline
  % \midrule
  \endfirsthead
  \toprule
  \SimHei \normalsize 年数 & \SimHei \scriptsize 公元 & \SimHei 大事件 \tabularnewline
  \midrule
  \endhead
  \midrule
  元年 & -28 & \tabularnewline\hline
  二年 & -27 & \tabularnewline\hline
  三年 & -26 & \tabularnewline\hline
  四年 & -25 & \tabularnewline
  \bottomrule
\end{longtable}


\subsection{阳朔}

\begin{longtable}{|>{\centering\scriptsize}m{2em}|>{\centering\scriptsize}m{1.3em}|>{\centering}m{8.8em}|}
  % \caption{秦王政}\
  \toprule
  \SimHei \normalsize 年数 & \SimHei \scriptsize 公元 & \SimHei 大事件 \tabularnewline
  % \midrule
  \endfirsthead
  \toprule
  \SimHei \normalsize 年数 & \SimHei \scriptsize 公元 & \SimHei 大事件 \tabularnewline
  \midrule
  \endhead
  \midrule
  元年 & -24 & \tabularnewline\hline
  二年 & -23 & \tabularnewline\hline
  三年 & -22 & \tabularnewline\hline
  四年 & -21 & \tabularnewline
  \bottomrule
\end{longtable}


\subsection{鸿嘉}

\begin{longtable}{|>{\centering\scriptsize}m{2em}|>{\centering\scriptsize}m{1.3em}|>{\centering}m{8.8em}|}
  % \caption{秦王政}\
  \toprule
  \SimHei \normalsize 年数 & \SimHei \scriptsize 公元 & \SimHei 大事件 \tabularnewline
  % \midrule
  \endfirsthead
  \toprule
  \SimHei \normalsize 年数 & \SimHei \scriptsize 公元 & \SimHei 大事件 \tabularnewline
  \midrule
  \endhead
  \midrule
  元年 & -20 & \tabularnewline\hline
  二年 & -19 & \tabularnewline\hline
  三年 & -18 & \tabularnewline\hline
  四年 & -17 & \tabularnewline
  \bottomrule
\end{longtable}


\subsection{永始}

\begin{longtable}{|>{\centering\scriptsize}m{2em}|>{\centering\scriptsize}m{1.3em}|>{\centering}m{8.8em}|}
  % \caption{秦王政}\
  \toprule
  \SimHei \normalsize 年数 & \SimHei \scriptsize 公元 & \SimHei 大事件 \tabularnewline
  % \midrule
  \endfirsthead
  \toprule
  \SimHei \normalsize 年数 & \SimHei \scriptsize 公元 & \SimHei 大事件 \tabularnewline
  \midrule
  \endhead
  \midrule
  元年 & -16 & \tabularnewline\hline
  二年 & -15 & \tabularnewline\hline
  三年 & -14 & \tabularnewline\hline
  四年 & -13 & \tabularnewline
  \bottomrule
\end{longtable}


\subsection{元诞}

\begin{longtable}{|>{\centering\scriptsize}m{2em}|>{\centering\scriptsize}m{1.3em}|>{\centering}m{8.8em}|}
  % \caption{秦王政}\
  \toprule
  \SimHei \normalsize 年数 & \SimHei \scriptsize 公元 & \SimHei 大事件 \tabularnewline
  % \midrule
  \endfirsthead
  \toprule
  \SimHei \normalsize 年数 & \SimHei \scriptsize 公元 & \SimHei 大事件 \tabularnewline
  \midrule
  \endhead
  \midrule
  元年 & -12 & \tabularnewline\hline
  二年 & -11 & \tabularnewline\hline
  三年 & -10 & \tabularnewline\hline
  四年 & -9 & \tabularnewline
  \bottomrule
\end{longtable}

\subsection{绥和}

\begin{longtable}{|>{\centering\scriptsize}m{2em}|>{\centering\scriptsize}m{1.3em}|>{\centering}m{8.8em}|}
  % \caption{秦王政}\
  \toprule
  \SimHei \normalsize 年数 & \SimHei \scriptsize 公元 & \SimHei 大事件 \tabularnewline
  % \midrule
  \endfirsthead
  \toprule
  \SimHei \normalsize 年数 & \SimHei \scriptsize 公元 & \SimHei 大事件 \tabularnewline
  \midrule
  \endhead
  \midrule
  元年 & -8 & \tabularnewline\hline
  二年 & -7 & \tabularnewline
  \bottomrule
\end{longtable}


%%% Local Variables:
%%% mode: latex
%%% TeX-engine: xetex
%%% TeX-master: "../Main"
%%% End:

%% -*- coding: utf-8 -*-
%% Time-stamp: <Chen Wang: 2018-07-10 19:18:50>

\section{哀帝\tiny(BC7-BC1)}

\subsection{建平}

\begin{longtable}{|>{\centering\scriptsize}m{2em}|>{\centering\scriptsize}m{1.3em}|>{\centering}m{8.8em}|}
  % \caption{秦王政}\
  \toprule
  \SimHei \normalsize 年数 & \SimHei \scriptsize 公元 & \SimHei 大事件 \tabularnewline
  % \midrule
  \endfirsthead
  \toprule
  \SimHei \normalsize 年数 & \SimHei \scriptsize 公元 & \SimHei 大事件 \tabularnewline
  \midrule
  \endhead
  \midrule
  元年 & -6 & \tabularnewline\hline
  二年 & -5 & \tabularnewline\hline
  太初\\元将 & -5 & \tabularnewline\hline
  三年 & -4 & \tabularnewline\hline
  四年 & -3 & \tabularnewline
  \bottomrule
\end{longtable}


\subsection{元寿}

\begin{longtable}{|>{\centering\scriptsize}m{2em}|>{\centering\scriptsize}m{1.3em}|>{\centering}m{8.8em}|}
  % \caption{秦王政}\
  \toprule
  \SimHei \normalsize 年数 & \SimHei \scriptsize 公元 & \SimHei 大事件 \tabularnewline
  % \midrule
  \endfirsthead
  \toprule
  \SimHei \normalsize 年数 & \SimHei \scriptsize 公元 & \SimHei 大事件 \tabularnewline
  \midrule
  \endhead
  \midrule
  元年 & -2 & \tabularnewline\hline
  二年 & -1 & \tabularnewline
  \bottomrule
\end{longtable}


%%% Local Variables:
%%% mode: latex
%%% TeX-engine: xetex
%%% TeX-master: "../Main"
%%% End:

%% -*- coding: utf-8 -*-
%% Time-stamp: <Chen Wang: 2018-07-10 19:28:12>

\section{平帝\tiny(1-5)}

\subsection{元始}

\begin{longtable}{|>{\centering\scriptsize}m{2em}|>{\centering\scriptsize}m{1.3em}|>{\centering}m{8.8em}|}
  % \caption{秦王政}\
  \toprule
  \SimHei \normalsize 年数 & \SimHei \scriptsize 公元 & \SimHei 大事件 \tabularnewline
  % \midrule
  \endfirsthead
  \toprule
  \SimHei \normalsize 年数 & \SimHei \scriptsize 公元 & \SimHei 大事件 \tabularnewline
  \midrule
  \endhead
  \midrule
  元年 & 1 & \tabularnewline\hline
  二年 & 2 & \tabularnewline\hline
  三年 & 3 & \tabularnewline\hline
  四年 & 4 & \tabularnewline\hline
  五年 & 5 & \tabularnewline
  \bottomrule
\end{longtable}


%%% Local Variables:
%%% mode: latex
%%% TeX-engine: xetex
%%% TeX-master: "../Main"
%%% End:

%% -*- coding: utf-8 -*-
%% Time-stamp: <Chen Wang: 2018-07-10 19:27:36>

\section{刘婴\tiny(6-8)}

\subsection{居摄}

\begin{longtable}{|>{\centering\scriptsize}m{2em}|>{\centering\scriptsize}m{1.3em}|>{\centering}m{8.8em}|}
  % \caption{秦王政}\
  \toprule
  \SimHei \normalsize 年数 & \SimHei \scriptsize 公元 & \SimHei 大事件 \tabularnewline
  % \midrule
  \endfirsthead
  \toprule
  \SimHei \normalsize 年数 & \SimHei \scriptsize 公元 & \SimHei 大事件 \tabularnewline
  \midrule
  \endhead
  \midrule
  元年 & 6 & \tabularnewline\hline
  二年 & 7 & \tabularnewline\hline
  三年\\初始 & 8 & \tabularnewline
  \bottomrule
\end{longtable}


%%% Local Variables:
%%% mode: latex
%%% TeX-engine: xetex
%%% TeX-master: "../Main"
%%% End:


%%% Local Variables:
%%% mode: latex
%%% TeX-engine: xetex
%%% TeX-master: "../Main"
%%% End:

% %% -*- coding: utf-8 -*-
%% Time-stamp: <Chen Wang: 2018-07-10 20:26:41>

\chapter{东汉\tiny(25-220)}

%% -*- coding: utf-8 -*-
%% Time-stamp: <Chen Wang: 2018-07-10 19:49:27>

\section{小政权}

\subsection{汉复\tiny(23-34)}

\begin{longtable}{|>{\centering\scriptsize}m{2em}|>{\centering\scriptsize}m{1.3em}|>{\centering}m{8.8em}|}
  % \caption{秦王政}\
  \toprule
  \SimHei \normalsize 年数 & \SimHei \scriptsize 公元 & \SimHei 大事件 \tabularnewline
  % \midrule
  \endfirsthead
  \toprule
  \SimHei \normalsize 年数 & \SimHei \scriptsize 公元 & \SimHei 大事件 \tabularnewline
  \midrule
  \endhead
  \midrule
  元年 & 23 & \tabularnewline\hline
  二年 & 24 & \tabularnewline\hline
  三年 & 25 & \tabularnewline\hline
  四年 & 26 & \tabularnewline\hline
  五年 & 27 & \tabularnewline\hline
  六年 & 28 & \tabularnewline\hline
  七年 & 29 & \tabularnewline\hline
  八年 & 30 & \tabularnewline\hline
  九年 & 31 & \tabularnewline\hline
  十年 & 32 & \tabularnewline\hline
  十一年 & 33 & \tabularnewline\hline
  十二年 & 34 & \tabularnewline
  \bottomrule
\end{longtable}

\subsection{龙兴\tiny(25-36)}

\begin{longtable}{|>{\centering\scriptsize}m{2em}|>{\centering\scriptsize}m{1.3em}|>{\centering}m{8.8em}|}
  % \caption{秦王政}\
  \toprule
  \SimHei \normalsize 年数 & \SimHei \scriptsize 公元 & \SimHei 大事件 \tabularnewline
  % \midrule
  \endfirsthead
  \toprule
  \SimHei \normalsize 年数 & \SimHei \scriptsize 公元 & \SimHei 大事件 \tabularnewline
  \midrule
  \endhead
  \midrule
  元年 & 25 & \tabularnewline\hline
  二年 & 26 & \tabularnewline\hline
  三年 & 27 & \tabularnewline\hline
  四年 & 28 & \tabularnewline\hline
  五年 & 29 & \tabularnewline\hline
  六年 & 30 & \tabularnewline\hline
  七年 & 31 & \tabularnewline\hline
  八年 & 32 & \tabularnewline\hline
  九年 & 33 & \tabularnewline\hline
  十年 & 34 & \tabularnewline\hline
  十一年 & 35 & \tabularnewline\hline
  十二年 & 36 & \tabularnewline
  \bottomrule
\end{longtable}

\subsection{建世\tiny(25-27)}

\begin{longtable}{|>{\centering\scriptsize}m{2em}|>{\centering\scriptsize}m{1.3em}|>{\centering}m{8.8em}|}
  % \caption{秦王政}\
  \toprule
  \SimHei \normalsize 年数 & \SimHei \scriptsize 公元 & \SimHei 大事件 \tabularnewline
  % \midrule
  \endfirsthead
  \toprule
  \SimHei \normalsize 年数 & \SimHei \scriptsize 公元 & \SimHei 大事件 \tabularnewline
  \midrule
  \endhead
  \midrule
  元年 & 25 & \tabularnewline\hline
  二年 & 26 & \tabularnewline\hline
  三年 & 27 & \tabularnewline
  \bottomrule
\end{longtable}


%%% Local Variables:
%%% mode: latex
%%% TeX-engine: xetex
%%% TeX-master: "../Main"
%%% End:

%% -*- coding: utf-8 -*-
%% Time-stamp: <Chen Wang: 2018-07-10 19:52:28>

\section{光武帝\tiny(25-57)}

\subsection{建武}

\begin{longtable}{|>{\centering\scriptsize}m{2em}|>{\centering\scriptsize}m{1.3em}|>{\centering}m{8.8em}|}
  % \caption{秦王政}\
  \toprule
  \SimHei \normalsize 年数 & \SimHei \scriptsize 公元 & \SimHei 大事件 \tabularnewline
  % \midrule
  \endfirsthead
  \toprule
  \SimHei \normalsize 年数 & \SimHei \scriptsize 公元 & \SimHei 大事件 \tabularnewline
  \midrule
  \endhead
  \midrule
  元年 & 25 & \tabularnewline\hline
  二年 & 26 & \tabularnewline\hline
  三年 & 27 & \tabularnewline\hline
  四年 & 28 & \tabularnewline\hline
  五年 & 29 & \tabularnewline\hline
  六年 & 30 & \tabularnewline\hline
  七年 & 31 & \tabularnewline\hline
  八年 & 32 & \tabularnewline\hline
  九年 & 33 & \tabularnewline\hline
  十年 & 34 & \tabularnewline\hline
  十一年 & 35 & \tabularnewline\hline
  十二年 & 36 & \tabularnewline\hline
  十三年 & 37 & \tabularnewline\hline
  十四年 & 38 & \tabularnewline\hline
  十五年 & 39 & \tabularnewline\hline
  十六年 & 40 & \tabularnewline\hline
  十七年 & 41 & \tabularnewline\hline
  十八年 & 42 & \tabularnewline\hline
  十九年 & 43 & \tabularnewline\hline
  二十年 & 44 & \tabularnewline\hline
  二一年 & 45 & \tabularnewline\hline
  二二年 & 46 & \tabularnewline\hline
  二三年 & 47 & \tabularnewline\hline
  二四年 & 48 & \tabularnewline\hline
  二五年 & 49 & \tabularnewline\hline
  二六年 & 50 & \tabularnewline\hline
  二七年 & 51 & \tabularnewline\hline
  二八年 & 52 & \tabularnewline\hline
  二九年 & 53 & \tabularnewline\hline
  三十年 & 54 & \tabularnewline\hline
  三一年 & 55 & \tabularnewline\hline
  三二年 & 56 & \tabularnewline
  \bottomrule
\end{longtable}

\subsection{建武中元}

\begin{longtable}{|>{\centering\scriptsize}m{2em}|>{\centering\scriptsize}m{1.3em}|>{\centering}m{8.8em}|}
  % \caption{秦王政}\
  \toprule
  \SimHei \normalsize 年数 & \SimHei \scriptsize 公元 & \SimHei 大事件 \tabularnewline
  % \midrule
  \endfirsthead
  \toprule
  \SimHei \normalsize 年数 & \SimHei \scriptsize 公元 & \SimHei 大事件 \tabularnewline
  \midrule
  \endhead
  \midrule
  元年 & 56 & \tabularnewline\hline
  二年 & 57 & \tabularnewline
  \bottomrule
\end{longtable}


%%% Local Variables:
%%% mode: latex
%%% TeX-engine: xetex
%%% TeX-master: "../Main"
%%% End:

%% -*- coding: utf-8 -*-
%% Time-stamp: <Chen Wang: 2018-07-10 19:55:10>

\section{明帝\tiny(57-75)}

\subsection{永平}

\begin{longtable}{|>{\centering\scriptsize}m{2em}|>{\centering\scriptsize}m{1.3em}|>{\centering}m{8.8em}|}
  % \caption{秦王政}\
  \toprule
  \SimHei \normalsize 年数 & \SimHei \scriptsize 公元 & \SimHei 大事件 \tabularnewline
  % \midrule
  \endfirsthead
  \toprule
  \SimHei \normalsize 年数 & \SimHei \scriptsize 公元 & \SimHei 大事件 \tabularnewline
  \midrule
  \endhead
  \midrule
  元年 & 58 & \tabularnewline\hline
  二年 & 59 & \tabularnewline\hline
  三年 & 60 & \tabularnewline\hline
  四年 & 61 & \tabularnewline\hline
  五年 & 62 & \tabularnewline\hline
  六年 & 63 & \tabularnewline\hline
  七年 & 64 & \tabularnewline\hline
  八年 & 65 & \tabularnewline\hline
  九年 & 66 & \tabularnewline\hline
  十年 & 67 & \tabularnewline\hline
  十一年 & 68 & \tabularnewline\hline
  十二年 & 69 & \tabularnewline\hline
  十三年 & 70 & \tabularnewline\hline
  十四年 & 71 & \tabularnewline\hline
  十五年 & 72 & \tabularnewline\hline
  十六年 & 73 & \tabularnewline\hline
  十七年 & 74 & \tabularnewline\hline
  十八年 & 75 & \tabularnewline
  \bottomrule
\end{longtable}


%%% Local Variables:
%%% mode: latex
%%% TeX-engine: xetex
%%% TeX-master: "../Main"
%%% End:

%% -*- coding: utf-8 -*-
%% Time-stamp: <Chen Wang: 2018-07-10 20:00:59>

\section{章帝\tiny(75-88)}

\subsection{建初}

\begin{longtable}{|>{\centering\scriptsize}m{2em}|>{\centering\scriptsize}m{1.3em}|>{\centering}m{8.8em}|}
  % \caption{秦王政}\
  \toprule
  \SimHei \normalsize 年数 & \SimHei \scriptsize 公元 & \SimHei 大事件 \tabularnewline
  % \midrule
  \endfirsthead
  \toprule
  \SimHei \normalsize 年数 & \SimHei \scriptsize 公元 & \SimHei 大事件 \tabularnewline
  \midrule
  \endhead
  \midrule
  元年 & 76 & \tabularnewline\hline
  二年 & 77 & \tabularnewline\hline
  三年 & 78 & \tabularnewline\hline
  四年 & 79 & \tabularnewline\hline
  五年 & 80 & \tabularnewline\hline
  六年 & 81 & \tabularnewline\hline
  七年 & 82 & \tabularnewline\hline
  八年 & 83 & \tabularnewline\hline
  九年 & 84 & \tabularnewline
  \bottomrule
\end{longtable}

\subsection{元和}

\begin{longtable}{|>{\centering\scriptsize}m{2em}|>{\centering\scriptsize}m{1.3em}|>{\centering}m{8.8em}|}
  % \caption{秦王政}\
  \toprule
  \SimHei \normalsize 年数 & \SimHei \scriptsize 公元 & \SimHei 大事件 \tabularnewline
  % \midrule
  \endfirsthead
  \toprule
  \SimHei \normalsize 年数 & \SimHei \scriptsize 公元 & \SimHei 大事件 \tabularnewline
  \midrule
  \endhead
  \midrule
  元年 & 84 & \tabularnewline\hline
  二年 & 85 & \tabularnewline\hline
  三年 & 86 & \tabularnewline\hline
  四年 & 87 & \tabularnewline
  \bottomrule
\end{longtable}

\subsection{章和}

\begin{longtable}{|>{\centering\scriptsize}m{2em}|>{\centering\scriptsize}m{1.3em}|>{\centering}m{8.8em}|}
  % \caption{秦王政}\
  \toprule
  \SimHei \normalsize 年数 & \SimHei \scriptsize 公元 & \SimHei 大事件 \tabularnewline
  % \midrule
  \endfirsthead
  \toprule
  \SimHei \normalsize 年数 & \SimHei \scriptsize 公元 & \SimHei 大事件 \tabularnewline
  \midrule
  \endhead
  \midrule
  元年 & 87 & \tabularnewline\hline
  二年 & 88 & \tabularnewline
  \bottomrule
\end{longtable}


%%% Local Variables:
%%% mode: latex
%%% TeX-engine: xetex
%%% TeX-master: "../Main"
%%% End:

%% -*- coding: utf-8 -*-
%% Time-stamp: <Chen Wang: 2018-07-10 20:03:15>

\section{和帝\tiny(88-105)}

\subsection{永元}

\begin{longtable}{|>{\centering\scriptsize}m{2em}|>{\centering\scriptsize}m{1.3em}|>{\centering}m{8.8em}|}
  % \caption{秦王政}\
  \toprule
  \SimHei \normalsize 年数 & \SimHei \scriptsize 公元 & \SimHei 大事件 \tabularnewline
  % \midrule
  \endfirsthead
  \toprule
  \SimHei \normalsize 年数 & \SimHei \scriptsize 公元 & \SimHei 大事件 \tabularnewline
  \midrule
  \endhead
  \midrule
  元年 & 89 & \tabularnewline\hline
  二年 & 90 & \tabularnewline\hline
  三年 & 91 & \tabularnewline\hline
  四年 & 92 & \tabularnewline\hline
  五年 & 93 & \tabularnewline\hline
  六年 & 94 & \tabularnewline\hline
  七年 & 95 & \tabularnewline\hline
  八年 & 96 & \tabularnewline\hline
  九年 & 97 & \tabularnewline\hline
  十年 & 98 & \tabularnewline\hline
  十一年 & 99 & \tabularnewline\hline
  十二年 & 100 & \tabularnewline\hline
  十三年 & 101 & \tabularnewline\hline
  十四年 & 102 & \tabularnewline\hline
  十五年 & 103 & \tabularnewline\hline
  十六年 & 104 & \tabularnewline\hline
  十七年 & 105 & \tabularnewline
  \bottomrule
\end{longtable}

\subsection{元兴}

\begin{longtable}{|>{\centering\scriptsize}m{2em}|>{\centering\scriptsize}m{1.3em}|>{\centering}m{8.8em}|}
  % \caption{秦王政}\
  \toprule
  \SimHei \normalsize 年数 & \SimHei \scriptsize 公元 & \SimHei 大事件 \tabularnewline
  % \midrule
  \endfirsthead
  \toprule
  \SimHei \normalsize 年数 & \SimHei \scriptsize 公元 & \SimHei 大事件 \tabularnewline
  \midrule
  \endhead
  \midrule
  元年 & 105 & \tabularnewline
  \bottomrule
\end{longtable}


%%% Local Variables:
%%% mode: latex
%%% TeX-engine: xetex
%%% TeX-master: "../Main"
%%% End:

%% -*- coding: utf-8 -*-
%% Time-stamp: <Chen Wang: 2018-07-10 20:04:09>

\section{殇帝\tiny(106)}

\subsection{延平}

\begin{longtable}{|>{\centering\scriptsize}m{2em}|>{\centering\scriptsize}m{1.3em}|>{\centering}m{8.8em}|}
  % \caption{秦王政}\
  \toprule
  \SimHei \normalsize 年数 & \SimHei \scriptsize 公元 & \SimHei 大事件 \tabularnewline
  % \midrule
  \endfirsthead
  \toprule
  \SimHei \normalsize 年数 & \SimHei \scriptsize 公元 & \SimHei 大事件 \tabularnewline
  \midrule
  \endhead
  \midrule
  元年 & 106 & \tabularnewline
  \bottomrule
\end{longtable}


%%% Local Variables:
%%% mode: latex
%%% TeX-engine: xetex
%%% TeX-master: "../Main"
%%% End:

%% -*- coding: utf-8 -*-
%% Time-stamp: <Chen Wang: 2018-07-10 20:07:29>

\section{安帝\tiny(106-125)}

\subsection{永初}

\begin{longtable}{|>{\centering\scriptsize}m{2em}|>{\centering\scriptsize}m{1.3em}|>{\centering}m{8.8em}|}
  % \caption{秦王政}\
  \toprule
  \SimHei \normalsize 年数 & \SimHei \scriptsize 公元 & \SimHei 大事件 \tabularnewline
  % \midrule
  \endfirsthead
  \toprule
  \SimHei \normalsize 年数 & \SimHei \scriptsize 公元 & \SimHei 大事件 \tabularnewline
  \midrule
  \endhead
  \midrule
  元年 & 107 & \tabularnewline\hline
  二年 & 108 & \tabularnewline\hline
  三年 & 109 & \tabularnewline\hline
  四年 & 110 & \tabularnewline\hline
  五年 & 111 & \tabularnewline\hline
  六年 & 112 & \tabularnewline\hline
  七年 & 113 & \tabularnewline
  \bottomrule
\end{longtable}

\subsection{元初}

\begin{longtable}{|>{\centering\scriptsize}m{2em}|>{\centering\scriptsize}m{1.3em}|>{\centering}m{8.8em}|}
  % \caption{秦王政}\
  \toprule
  \SimHei \normalsize 年数 & \SimHei \scriptsize 公元 & \SimHei 大事件 \tabularnewline
  % \midrule
  \endfirsthead
  \toprule
  \SimHei \normalsize 年数 & \SimHei \scriptsize 公元 & \SimHei 大事件 \tabularnewline
  \midrule
  \endhead
  \midrule
  元年 & 114 & \tabularnewline\hline
  二年 & 115 & \tabularnewline\hline
  三年 & 116 & \tabularnewline\hline
  四年 & 117 & \tabularnewline\hline
  五年 & 118 & \tabularnewline\hline
  六年 & 119 & \tabularnewline\hline
  七年 & 120 & \tabularnewline
  \bottomrule
\end{longtable}

\subsection{永宁}

\begin{longtable}{|>{\centering\scriptsize}m{2em}|>{\centering\scriptsize}m{1.3em}|>{\centering}m{8.8em}|}
  % \caption{秦王政}\
  \toprule
  \SimHei \normalsize 年数 & \SimHei \scriptsize 公元 & \SimHei 大事件 \tabularnewline
  % \midrule
  \endfirsthead
  \toprule
  \SimHei \normalsize 年数 & \SimHei \scriptsize 公元 & \SimHei 大事件 \tabularnewline
  \midrule
  \endhead
  \midrule
  元年 & 120 & \tabularnewline\hline
  二年 & 121 & \tabularnewline
  \bottomrule
\end{longtable}

\subsection{建光}

\begin{longtable}{|>{\centering\scriptsize}m{2em}|>{\centering\scriptsize}m{1.3em}|>{\centering}m{8.8em}|}
  % \caption{秦王政}\
  \toprule
  \SimHei \normalsize 年数 & \SimHei \scriptsize 公元 & \SimHei 大事件 \tabularnewline
  % \midrule
  \endfirsthead
  \toprule
  \SimHei \normalsize 年数 & \SimHei \scriptsize 公元 & \SimHei 大事件 \tabularnewline
  \midrule
  \endhead
  \midrule
  元年 & 121 & \tabularnewline\hline
  二年 & 122 & \tabularnewline
  \bottomrule
\end{longtable}

\subsection{延光}

\begin{longtable}{|>{\centering\scriptsize}m{2em}|>{\centering\scriptsize}m{1.3em}|>{\centering}m{8.8em}|}
  % \caption{秦王政}\
  \toprule
  \SimHei \normalsize 年数 & \SimHei \scriptsize 公元 & \SimHei 大事件 \tabularnewline
  % \midrule
  \endfirsthead
  \toprule
  \SimHei \normalsize 年数 & \SimHei \scriptsize 公元 & \SimHei 大事件 \tabularnewline
  \midrule
  \endhead
  \midrule
  元年 & 122 & \tabularnewline\hline
  二年 & 123 & \tabularnewline\hline
  三年 & 124 & \tabularnewline\hline
  四年 & 125 & \tabularnewline
  \bottomrule
\end{longtable}


%%% Local Variables:
%%% mode: latex
%%% TeX-engine: xetex
%%% TeX-master: "../Main"
%%% End:

%% -*- coding: utf-8 -*-
%% Time-stamp: <Chen Wang: 2018-07-10 20:10:56>

\section{顺帝\tiny(125-144)}

\subsection{永建}

\begin{longtable}{|>{\centering\scriptsize}m{2em}|>{\centering\scriptsize}m{1.3em}|>{\centering}m{8.8em}|}
  % \caption{秦王政}\
  \toprule
  \SimHei \normalsize 年数 & \SimHei \scriptsize 公元 & \SimHei 大事件 \tabularnewline
  % \midrule
  \endfirsthead
  \toprule
  \SimHei \normalsize 年数 & \SimHei \scriptsize 公元 & \SimHei 大事件 \tabularnewline
  \midrule
  \endhead
  \midrule
  元年 & 126 & \tabularnewline\hline
  二年 & 127 & \tabularnewline\hline
  三年 & 128 & \tabularnewline\hline
  四年 & 129 & \tabularnewline\hline
  五年 & 130 & \tabularnewline\hline
  六年 & 131 & \tabularnewline\hline
  七年 & 132 & \tabularnewline
  \bottomrule
\end{longtable}

\subsection{阳嘉}

\begin{longtable}{|>{\centering\scriptsize}m{2em}|>{\centering\scriptsize}m{1.3em}|>{\centering}m{8.8em}|}
  % \caption{秦王政}\
  \toprule
  \SimHei \normalsize 年数 & \SimHei \scriptsize 公元 & \SimHei 大事件 \tabularnewline
  % \midrule
  \endfirsthead
  \toprule
  \SimHei \normalsize 年数 & \SimHei \scriptsize 公元 & \SimHei 大事件 \tabularnewline
  \midrule
  \endhead
  \midrule
  元年 & 132 & \tabularnewline\hline
  二年 & 133 & \tabularnewline\hline
  三年 & 134 & \tabularnewline\hline
  四年 & 135 & \tabularnewline
  \bottomrule
\end{longtable}

\subsection{永和}

\begin{longtable}{|>{\centering\scriptsize}m{2em}|>{\centering\scriptsize}m{1.3em}|>{\centering}m{8.8em}|}
  % \caption{秦王政}\
  \toprule
  \SimHei \normalsize 年数 & \SimHei \scriptsize 公元 & \SimHei 大事件 \tabularnewline
  % \midrule
  \endfirsthead
  \toprule
  \SimHei \normalsize 年数 & \SimHei \scriptsize 公元 & \SimHei 大事件 \tabularnewline
  \midrule
  \endhead
  \midrule
  元年 & 136 & \tabularnewline\hline
  二年 & 137 & \tabularnewline\hline
  三年 & 138 & \tabularnewline\hline
  四年 & 139 & \tabularnewline\hline
  五年 & 140 & \tabularnewline\hline
  六年 & 141 & \tabularnewline
  \bottomrule
\end{longtable}

\subsection{汉安}

\begin{longtable}{|>{\centering\scriptsize}m{2em}|>{\centering\scriptsize}m{1.3em}|>{\centering}m{8.8em}|}
  % \caption{秦王政}\
  \toprule
  \SimHei \normalsize 年数 & \SimHei \scriptsize 公元 & \SimHei 大事件 \tabularnewline
  % \midrule
  \endfirsthead
  \toprule
  \SimHei \normalsize 年数 & \SimHei \scriptsize 公元 & \SimHei 大事件 \tabularnewline
  \midrule
  \endhead
  \midrule
  元年 & 142 & \tabularnewline\hline
  二年 & 143 & \tabularnewline\hline
  三年 & 144 & \tabularnewline
  \bottomrule
\end{longtable}

\subsection{建康}

\begin{longtable}{|>{\centering\scriptsize}m{2em}|>{\centering\scriptsize}m{1.3em}|>{\centering}m{8.8em}|}
  % \caption{秦王政}\
  \toprule
  \SimHei \normalsize 年数 & \SimHei \scriptsize 公元 & \SimHei 大事件 \tabularnewline
  % \midrule
  \endfirsthead
  \toprule
  \SimHei \normalsize 年数 & \SimHei \scriptsize 公元 & \SimHei 大事件 \tabularnewline
  \midrule
  \endhead
  \midrule
  元年 & 144 & \tabularnewline
  \bottomrule
\end{longtable}


%%% Local Variables:
%%% mode: latex
%%% TeX-engine: xetex
%%% TeX-master: "../Main"
%%% End:

%% -*- coding: utf-8 -*-
%% Time-stamp: <Chen Wang: 2018-07-10 20:12:42>

\section{冲帝\tiny(144-145)}

\subsection{永嘉}

\begin{longtable}{|>{\centering\scriptsize}m{2em}|>{\centering\scriptsize}m{1.3em}|>{\centering}m{8.8em}|}
  % \caption{秦王政}\
  \toprule
  \SimHei \normalsize 年数 & \SimHei \scriptsize 公元 & \SimHei 大事件 \tabularnewline
  % \midrule
  \endfirsthead
  \toprule
  \SimHei \normalsize 年数 & \SimHei \scriptsize 公元 & \SimHei 大事件 \tabularnewline
  \midrule
  \endhead
  \midrule
  元年 & 145 & \tabularnewline
  \bottomrule
\end{longtable}

%%% Local Variables:
%%% mode: latex
%%% TeX-engine: xetex
%%% TeX-master: "../Main"
%%% End:

%% -*- coding: utf-8 -*-
%% Time-stamp: <Chen Wang: 2018-07-10 20:13:16>

\section{质帝\tiny(145-146)}

\subsection{本初}

\begin{longtable}{|>{\centering\scriptsize}m{2em}|>{\centering\scriptsize}m{1.3em}|>{\centering}m{8.8em}|}
  % \caption{秦王政}\
  \toprule
  \SimHei \normalsize 年数 & \SimHei \scriptsize 公元 & \SimHei 大事件 \tabularnewline
  % \midrule
  \endfirsthead
  \toprule
  \SimHei \normalsize 年数 & \SimHei \scriptsize 公元 & \SimHei 大事件 \tabularnewline
  \midrule
  \endhead
  \midrule
  元年 & 146 & \tabularnewline
  \bottomrule
\end{longtable}

%%% Local Variables:
%%% mode: latex
%%% TeX-engine: xetex
%%% TeX-master: "../Main"
%%% End:

%% -*- coding: utf-8 -*-
%% Time-stamp: <Chen Wang: 2018-07-10 20:18:43>

\section{桓帝\tiny(147-167)}

\subsection{建和}

\begin{longtable}{|>{\centering\scriptsize}m{2em}|>{\centering\scriptsize}m{1.3em}|>{\centering}m{8.8em}|}
  % \caption{秦王政}\
  \toprule
  \SimHei \normalsize 年数 & \SimHei \scriptsize 公元 & \SimHei 大事件 \tabularnewline
  % \midrule
  \endfirsthead
  \toprule
  \SimHei \normalsize 年数 & \SimHei \scriptsize 公元 & \SimHei 大事件 \tabularnewline
  \midrule
  \endhead
  \midrule
  元年 & 147 & \tabularnewline\hline
  二年 & 148 & \tabularnewline\hline
  三年 & 149 & \tabularnewline
  \bottomrule
\end{longtable}

\subsection{和平}

\begin{longtable}{|>{\centering\scriptsize}m{2em}|>{\centering\scriptsize}m{1.3em}|>{\centering}m{8.8em}|}
  % \caption{秦王政}\
  \toprule
  \SimHei \normalsize 年数 & \SimHei \scriptsize 公元 & \SimHei 大事件 \tabularnewline
  % \midrule
  \endfirsthead
  \toprule
  \SimHei \normalsize 年数 & \SimHei \scriptsize 公元 & \SimHei 大事件 \tabularnewline
  \midrule
  \endhead
  \midrule
  元年 & 150 & \tabularnewline
  \bottomrule
\end{longtable}

\subsection{元嘉}

\begin{longtable}{|>{\centering\scriptsize}m{2em}|>{\centering\scriptsize}m{1.3em}|>{\centering}m{8.8em}|}
  % \caption{秦王政}\
  \toprule
  \SimHei \normalsize 年数 & \SimHei \scriptsize 公元 & \SimHei 大事件 \tabularnewline
  % \midrule
  \endfirsthead
  \toprule
  \SimHei \normalsize 年数 & \SimHei \scriptsize 公元 & \SimHei 大事件 \tabularnewline
  \midrule
  \endhead
  \midrule
  元年 & 151 & \tabularnewline\hline
  二年 & 152 & \tabularnewline\hline
  三年 & 153 & \tabularnewline
  \bottomrule
\end{longtable}

\subsection{永兴}

\begin{longtable}{|>{\centering\scriptsize}m{2em}|>{\centering\scriptsize}m{1.3em}|>{\centering}m{8.8em}|}
  % \caption{秦王政}\
  \toprule
  \SimHei \normalsize 年数 & \SimHei \scriptsize 公元 & \SimHei 大事件 \tabularnewline
  % \midrule
  \endfirsthead
  \toprule
  \SimHei \normalsize 年数 & \SimHei \scriptsize 公元 & \SimHei 大事件 \tabularnewline
  \midrule
  \endhead
  \midrule
  元年 & 153 & \tabularnewline\hline
  二年 & 154 & \tabularnewline
  \bottomrule
\end{longtable}

\subsection{永寿}

\begin{longtable}{|>{\centering\scriptsize}m{2em}|>{\centering\scriptsize}m{1.3em}|>{\centering}m{8.8em}|}
  % \caption{秦王政}\
  \toprule
  \SimHei \normalsize 年数 & \SimHei \scriptsize 公元 & \SimHei 大事件 \tabularnewline
  % \midrule
  \endfirsthead
  \toprule
  \SimHei \normalsize 年数 & \SimHei \scriptsize 公元 & \SimHei 大事件 \tabularnewline
  \midrule
  \endhead
  \midrule
  元年 & 155 & \tabularnewline\hline
  二年 & 156 & \tabularnewline\hline
  三年 & 157 & \tabularnewline\hline
  四年 & 158 & \tabularnewline
  \bottomrule
\end{longtable}

\subsection{延熹}

\begin{longtable}{|>{\centering\scriptsize}m{2em}|>{\centering\scriptsize}m{1.3em}|>{\centering}m{8.8em}|}
  % \caption{秦王政}\
  \toprule
  \SimHei \normalsize 年数 & \SimHei \scriptsize 公元 & \SimHei 大事件 \tabularnewline
  % \midrule
  \endfirsthead
  \toprule
  \SimHei \normalsize 年数 & \SimHei \scriptsize 公元 & \SimHei 大事件 \tabularnewline
  \midrule
  \endhead
  \midrule
  元年 & 158 & \tabularnewline\hline
  二年 & 159 & \tabularnewline\hline
  三年 & 160 & \tabularnewline\hline
  四年 & 161 & \tabularnewline\hline
  五年 & 162 & \tabularnewline\hline
  六年 & 163 & \tabularnewline\hline
  七年 & 164 & \tabularnewline\hline
  八年 & 165 & \tabularnewline\hline
  九年 & 166 & \tabularnewline\hline
  十年 & 167 & \tabularnewline
  \bottomrule
\end{longtable}


\subsection{永康}

\begin{longtable}{|>{\centering\scriptsize}m{2em}|>{\centering\scriptsize}m{1.3em}|>{\centering}m{8.8em}|}
  % \caption{秦王政}\
  \toprule
  \SimHei \normalsize 年数 & \SimHei \scriptsize 公元 & \SimHei 大事件 \tabularnewline
  % \midrule
  \endfirsthead
  \toprule
  \SimHei \normalsize 年数 & \SimHei \scriptsize 公元 & \SimHei 大事件 \tabularnewline
  \midrule
  \endhead
  \midrule
  元年 & 167 & \tabularnewline
  \bottomrule
\end{longtable}


%%% Local Variables:
%%% mode: latex
%%% TeX-engine: xetex
%%% TeX-master: "../Main"
%%% End:

%% -*- coding: utf-8 -*-
%% Time-stamp: <Chen Wang: 2018-07-10 20:21:31>

\section{灵帝\tiny(168-189)}

\subsection{建宁}

\begin{longtable}{|>{\centering\scriptsize}m{2em}|>{\centering\scriptsize}m{1.3em}|>{\centering}m{8.8em}|}
  % \caption{秦王政}\
  \toprule
  \SimHei \normalsize 年数 & \SimHei \scriptsize 公元 & \SimHei 大事件 \tabularnewline
  % \midrule
  \endfirsthead
  \toprule
  \SimHei \normalsize 年数 & \SimHei \scriptsize 公元 & \SimHei 大事件 \tabularnewline
  \midrule
  \endhead
  \midrule
  元年 & 168 & \tabularnewline\hline
  二年 & 169 & \tabularnewline\hline
  三年 & 170 & \tabularnewline\hline
  四年 & 171 & \tabularnewline\hline
  五年 & 172 & \tabularnewline
  \bottomrule
\end{longtable}

\subsection{熹平}

\begin{longtable}{|>{\centering\scriptsize}m{2em}|>{\centering\scriptsize}m{1.3em}|>{\centering}m{8.8em}|}
  % \caption{秦王政}\
  \toprule
  \SimHei \normalsize 年数 & \SimHei \scriptsize 公元 & \SimHei 大事件 \tabularnewline
  % \midrule
  \endfirsthead
  \toprule
  \SimHei \normalsize 年数 & \SimHei \scriptsize 公元 & \SimHei 大事件 \tabularnewline
  \midrule
  \endhead
  \midrule
  元年 & 172 & \tabularnewline\hline
  二年 & 173 & \tabularnewline\hline
  三年 & 174 & \tabularnewline\hline
  四年 & 175 & \tabularnewline\hline
  五年 & 176 & \tabularnewline\hline
  六年 & 177 & \tabularnewline\hline
  七年 & 178 & \tabularnewline
  \bottomrule
\end{longtable}

\subsection{光和}

\begin{longtable}{|>{\centering\scriptsize}m{2em}|>{\centering\scriptsize}m{1.3em}|>{\centering}m{8.8em}|}
  % \caption{秦王政}\
  \toprule
  \SimHei \normalsize 年数 & \SimHei \scriptsize 公元 & \SimHei 大事件 \tabularnewline
  % \midrule
  \endfirsthead
  \toprule
  \SimHei \normalsize 年数 & \SimHei \scriptsize 公元 & \SimHei 大事件 \tabularnewline
  \midrule
  \endhead
  \midrule
  元年 & 178 & \tabularnewline\hline
  二年 & 179 & \tabularnewline\hline
  三年 & 180 & \tabularnewline\hline
  四年 & 181 & \tabularnewline\hline
  五年 & 182 & \tabularnewline\hline
  六年 & 183 & \tabularnewline\hline
  七年 & 184 & \tabularnewline
  \bottomrule
\end{longtable}

\subsection{中平}

\begin{longtable}{|>{\centering\scriptsize}m{2em}|>{\centering\scriptsize}m{1.3em}|>{\centering}m{8.8em}|}
  % \caption{秦王政}\
  \toprule
  \SimHei \normalsize 年数 & \SimHei \scriptsize 公元 & \SimHei 大事件 \tabularnewline
  % \midrule
  \endfirsthead
  \toprule
  \SimHei \normalsize 年数 & \SimHei \scriptsize 公元 & \SimHei 大事件 \tabularnewline
  \midrule
  \endhead
  \midrule
  元年 & 184 & \tabularnewline\hline
  二年 & 185 & \tabularnewline\hline
  三年 & 186 & \tabularnewline\hline
  四年 & 187 & \tabularnewline\hline
  五年 & 188 & \tabularnewline\hline
  六年 & 189 & \tabularnewline
  \bottomrule
\end{longtable}


%%% Local Variables:
%%% mode: latex
%%% TeX-engine: xetex
%%% TeX-master: "../Main"
%%% End:

%% -*- coding: utf-8 -*-
%% Time-stamp: <Chen Wang: 2018-07-10 20:22:57>

\section{刘辩\tiny(189)}

\subsection{光熹}

\begin{longtable}{|>{\centering\scriptsize}m{2em}|>{\centering\scriptsize}m{1.3em}|>{\centering}m{8.8em}|}
  % \caption{秦王政}\
  \toprule
  \SimHei \normalsize 年数 & \SimHei \scriptsize 公元 & \SimHei 大事件 \tabularnewline
  % \midrule
  \endfirsthead
  \toprule
  \SimHei \normalsize 年数 & \SimHei \scriptsize 公元 & \SimHei 大事件 \tabularnewline
  \midrule
  \endhead
  \midrule
  元年 & 189 & \tabularnewline
  \bottomrule
\end{longtable}

\subsection{昭宁}

\begin{longtable}{|>{\centering\scriptsize}m{2em}|>{\centering\scriptsize}m{1.3em}|>{\centering}m{8.8em}|}
  % \caption{秦王政}\
  \toprule
  \SimHei \normalsize 年数 & \SimHei \scriptsize 公元 & \SimHei 大事件 \tabularnewline
  % \midrule
  \endfirsthead
  \toprule
  \SimHei \normalsize 年数 & \SimHei \scriptsize 公元 & \SimHei 大事件 \tabularnewline
  \midrule
  \endhead
  \midrule
  元年 & 189 & \tabularnewline
  \bottomrule
\end{longtable}


%%% Local Variables:
%%% mode: latex
%%% TeX-engine: xetex
%%% TeX-master: "../Main"
%%% End:

%% -*- coding: utf-8 -*-
%% Time-stamp: <Chen Wang: 2018-07-10 20:26:01>

\section{献帝\tiny(189-220)}

\subsection{永汉}

\begin{longtable}{|>{\centering\scriptsize}m{2em}|>{\centering\scriptsize}m{1.3em}|>{\centering}m{8.8em}|}
  % \caption{秦王政}\
  \toprule
  \SimHei \normalsize 年数 & \SimHei \scriptsize 公元 & \SimHei 大事件 \tabularnewline
  % \midrule
  \endfirsthead
  \toprule
  \SimHei \normalsize 年数 & \SimHei \scriptsize 公元 & \SimHei 大事件 \tabularnewline
  \midrule
  \endhead
  \midrule
  元年 & 189 & \tabularnewline
  \bottomrule
\end{longtable}

\subsection{中平}

\begin{longtable}{|>{\centering\scriptsize}m{2em}|>{\centering\scriptsize}m{1.3em}|>{\centering}m{8.8em}|}
  % \caption{秦王政}\
  \toprule
  \SimHei \normalsize 年数 & \SimHei \scriptsize 公元 & \SimHei 大事件 \tabularnewline
  % \midrule
  \endfirsthead
  \toprule
  \SimHei \normalsize 年数 & \SimHei \scriptsize 公元 & \SimHei 大事件 \tabularnewline
  \midrule
  \endhead
  \midrule
  元年 & 189 & \tabularnewline
  \bottomrule
\end{longtable}

\subsection{初平}

\begin{longtable}{|>{\centering\scriptsize}m{2em}|>{\centering\scriptsize}m{1.3em}|>{\centering}m{8.8em}|}
  % \caption{秦王政}\
  \toprule
  \SimHei \normalsize 年数 & \SimHei \scriptsize 公元 & \SimHei 大事件 \tabularnewline
  % \midrule
  \endfirsthead
  \toprule
  \SimHei \normalsize 年数 & \SimHei \scriptsize 公元 & \SimHei 大事件 \tabularnewline
  \midrule
  \endhead
  \midrule
  元年 & 190 & \tabularnewline\hline
  二年 & 191 & \tabularnewline\hline
  三年 & 192 & \tabularnewline\hline
  四年 & 193 & \tabularnewline
  \bottomrule
\end{longtable}


\subsection{兴平}

\begin{longtable}{|>{\centering\scriptsize}m{2em}|>{\centering\scriptsize}m{1.3em}|>{\centering}m{8.8em}|}
  % \caption{秦王政}\
  \toprule
  \SimHei \normalsize 年数 & \SimHei \scriptsize 公元 & \SimHei 大事件 \tabularnewline
  % \midrule
  \endfirsthead
  \toprule
  \SimHei \normalsize 年数 & \SimHei \scriptsize 公元 & \SimHei 大事件 \tabularnewline
  \midrule
  \endhead
  \midrule
  元年 & 194 & \tabularnewline\hline
  二年 & 195 & \tabularnewline
  \bottomrule
\end{longtable}

\subsection{建安}

\begin{longtable}{|>{\centering\scriptsize}m{2em}|>{\centering\scriptsize}m{1.3em}|>{\centering}m{8.8em}|}
  % \caption{秦王政}\
  \toprule
  \SimHei \normalsize 年数 & \SimHei \scriptsize 公元 & \SimHei 大事件 \tabularnewline
  % \midrule
  \endfirsthead
  \toprule
  \SimHei \normalsize 年数 & \SimHei \scriptsize 公元 & \SimHei 大事件 \tabularnewline
  \midrule
  \endhead
  \midrule
  元年 & 196 & \tabularnewline\hline
  二年 & 197 & \tabularnewline\hline
  三年 & 198 & \tabularnewline\hline
  四年 & 199 & \tabularnewline\hline
  五年 & 200 & \tabularnewline\hline
  六年 & 201 & \tabularnewline\hline
  七年 & 202 & \tabularnewline\hline
  八年 & 203 & \tabularnewline\hline
  九年 & 204 & \tabularnewline\hline
  十年 & 205 & \tabularnewline\hline
  十一年 & 206 & \tabularnewline\hline
  十二年 & 207 & \tabularnewline\hline
  十三年 & 208 & \tabularnewline\hline
  十四年 & 209 & \tabularnewline\hline
  十五年 & 210 & \tabularnewline\hline
  十六年 & 211 & \tabularnewline\hline
  十七年 & 212 & \tabularnewline\hline
  十八年 & 213 & \tabularnewline\hline
  十九年 & 214 & \tabularnewline\hline
  二十年 & 215 & \tabularnewline\hline
  二一年 & 216 & \tabularnewline\hline
  二二年 & 217 & \tabularnewline\hline
  二三年 & 218 & \tabularnewline\hline
  二四年 & 219 & \tabularnewline\hline
  二五年 & 220 & \tabularnewline
  \bottomrule
\end{longtable}

\subsection{延康}

\begin{longtable}{|>{\centering\scriptsize}m{2em}|>{\centering\scriptsize}m{1.3em}|>{\centering}m{8.8em}|}
  % \caption{秦王政}\
  \toprule
  \SimHei \normalsize 年数 & \SimHei \scriptsize 公元 & \SimHei 大事件 \tabularnewline
  % \midrule
  \endfirsthead
  \toprule
  \SimHei \normalsize 年数 & \SimHei \scriptsize 公元 & \SimHei 大事件 \tabularnewline
  \midrule
  \endhead
  \midrule
  元年 & 220 & \tabularnewline
  \bottomrule
\end{longtable}


%%% Local Variables:
%%% mode: latex
%%% TeX-engine: xetex
%%% TeX-master: "../Main"
%%% End:


%%% Local Variables:
%%% mode: latex
%%% TeX-engine: xetex
%%% TeX-master: "../Main"
%%% End:

% %% -*- coding: utf-8 -*-
%% Time-stamp: <Chen Wang: 2018-07-10 22:02:02>

\chapter{三国\tiny(220-280)}


%% -*- coding: utf-8 -*-
%% Time-stamp: <Chen Wang: 2018-07-10 20:53:11>


\section{曹魏\tiny(220-265)}

%% -*- coding: utf-8 -*-
%% Time-stamp: <Chen Wang: 2018-07-10 20:33:47>

\subsection{文帝\tiny(220-226)}

\subsubsection{黄初}

\begin{longtable}{|>{\centering\scriptsize}m{2em}|>{\centering\scriptsize}m{1.3em}|>{\centering}m{8.8em}|}
  % \caption{秦王政}\
  \toprule
  \SimHei \normalsize 年数 & \SimHei \scriptsize 公元 & \SimHei 大事件 \tabularnewline
  % \midrule
  \endfirsthead
  \toprule
  \SimHei \normalsize 年数 & \SimHei \scriptsize 公元 & \SimHei 大事件 \tabularnewline
  \midrule
  \endhead
  \midrule
  元年 & 220 & \tabularnewline\hline
  二年 & 221 & \tabularnewline\hline
  三年 & 222 & \tabularnewline\hline
  四年 & 223 & \tabularnewline\hline
  五年 & 224 & \tabularnewline\hline
  六年 & 225 & \tabularnewline\hline
  七年 & 226 & \tabularnewline
  \bottomrule
\end{longtable}


%%% Local Variables:
%%% mode: latex
%%% TeX-engine: xetex
%%% TeX-master: "../../Main"
%%% End:

%% -*- coding: utf-8 -*-
%% Time-stamp: <Chen Wang: 2018-07-10 20:44:37>

\subsection{明帝\tiny(226-239)}

\subsubsection{太和}

\begin{longtable}{|>{\centering\scriptsize}m{2em}|>{\centering\scriptsize}m{1.3em}|>{\centering}m{8.8em}|}
  % \caption{秦王政}\
  \toprule
  \SimHei \normalsize 年数 & \SimHei \scriptsize 公元 & \SimHei 大事件 \tabularnewline
  % \midrule
  \endfirsthead
  \toprule
  \SimHei \normalsize 年数 & \SimHei \scriptsize 公元 & \SimHei 大事件 \tabularnewline
  \midrule
  \endhead
  \midrule
  元年 & 227 & \tabularnewline\hline
  二年 & 228 & \tabularnewline\hline
  三年 & 229 & \tabularnewline\hline
  四年 & 230 & \tabularnewline\hline
  五年 & 231 & \tabularnewline\hline
  六年 & 232 & \tabularnewline\hline
  七年 & 233 & \tabularnewline
  \bottomrule
\end{longtable}

\subsubsection{青龙}

\begin{longtable}{|>{\centering\scriptsize}m{2em}|>{\centering\scriptsize}m{1.3em}|>{\centering}m{8.8em}|}
  % \caption{秦王政}\
  \toprule
  \SimHei \normalsize 年数 & \SimHei \scriptsize 公元 & \SimHei 大事件 \tabularnewline
  % \midrule
  \endfirsthead
  \toprule
  \SimHei \normalsize 年数 & \SimHei \scriptsize 公元 & \SimHei 大事件 \tabularnewline
  \midrule
  \endhead
  \midrule
  元年 & 233 & \tabularnewline\hline
  二年 & 234 & \tabularnewline\hline
  三年 & 235 & \tabularnewline\hline
  四年 & 236 & \tabularnewline\hline
  五年 & 237 & \tabularnewline
  \bottomrule
\end{longtable}

\subsubsection{景初}

\begin{longtable}{|>{\centering\scriptsize}m{2em}|>{\centering\scriptsize}m{1.3em}|>{\centering}m{8.8em}|}
  % \caption{秦王政}\
  \toprule
  \SimHei \normalsize 年数 & \SimHei \scriptsize 公元 & \SimHei 大事件 \tabularnewline
  % \midrule
  \endfirsthead
  \toprule
  \SimHei \normalsize 年数 & \SimHei \scriptsize 公元 & \SimHei 大事件 \tabularnewline
  \midrule
  \endhead
  \midrule
  元年 & 237 & \tabularnewline\hline
  二年 & 238 & \tabularnewline\hline
  三年 & 239 & \tabularnewline
  \bottomrule
\end{longtable}


%%% Local Variables:
%%% mode: latex
%%% TeX-engine: xetex
%%% TeX-master: "../../Main"
%%% End:

%% -*- coding: utf-8 -*-
%% Time-stamp: <Chen Wang: 2018-07-10 20:48:48>

\subsection{曹芳\tiny(239-254)}

\subsubsection{正始}

\begin{longtable}{|>{\centering\scriptsize}m{2em}|>{\centering\scriptsize}m{1.3em}|>{\centering}m{8.8em}|}
  % \caption{秦王政}\
  \toprule
  \SimHei \normalsize 年数 & \SimHei \scriptsize 公元 & \SimHei 大事件 \tabularnewline
  % \midrule
  \endfirsthead
  \toprule
  \SimHei \normalsize 年数 & \SimHei \scriptsize 公元 & \SimHei 大事件 \tabularnewline
  \midrule
  \endhead
  \midrule
  元年 & 240 & \tabularnewline\hline
  二年 & 241 & \tabularnewline\hline
  三年 & 242 & \tabularnewline\hline
  四年 & 243 & \tabularnewline\hline
  五年 & 244 & \tabularnewline\hline
  六年 & 245 & \tabularnewline\hline
  七年 & 246 & \tabularnewline\hline
  八年 & 247 & \tabularnewline\hline
  九年 & 248 & \tabularnewline\hline
  十年 & 249 & \tabularnewline
  \bottomrule
\end{longtable}

\subsubsection{嘉平}

\begin{longtable}{|>{\centering\scriptsize}m{2em}|>{\centering\scriptsize}m{1.3em}|>{\centering}m{8.8em}|}
  % \caption{秦王政}\
  \toprule
  \SimHei \normalsize 年数 & \SimHei \scriptsize 公元 & \SimHei 大事件 \tabularnewline
  % \midrule
  \endfirsthead
  \toprule
  \SimHei \normalsize 年数 & \SimHei \scriptsize 公元 & \SimHei 大事件 \tabularnewline
  \midrule
  \endhead
  \midrule
  元年 & 249 & \tabularnewline\hline
  二年 & 250 & \tabularnewline\hline
  三年 & 251 & \tabularnewline\hline
  四年 & 252 & \tabularnewline\hline
  五年 & 253 & \tabularnewline\hline
  六年 & 254 & \tabularnewline
  \bottomrule
\end{longtable}


%%% Local Variables:
%%% mode: latex
%%% TeX-engine: xetex
%%% TeX-master: "../../Main"
%%% End:

%% -*- coding: utf-8 -*-
%% Time-stamp: <Chen Wang: 2018-07-10 20:50:14>

\subsection{曹髦\tiny(254-260)}

\subsubsection{正元}

\begin{longtable}{|>{\centering\scriptsize}m{2em}|>{\centering\scriptsize}m{1.3em}|>{\centering}m{8.8em}|}
  % \caption{秦王政}\
  \toprule
  \SimHei \normalsize 年数 & \SimHei \scriptsize 公元 & \SimHei 大事件 \tabularnewline
  % \midrule
  \endfirsthead
  \toprule
  \SimHei \normalsize 年数 & \SimHei \scriptsize 公元 & \SimHei 大事件 \tabularnewline
  \midrule
  \endhead
  \midrule
  元年 & 254 & \tabularnewline\hline
  二年 & 255 & \tabularnewline\hline
  三年 & 256 & \tabularnewline
  \bottomrule
\end{longtable}

\subsubsection{甘露}

\begin{longtable}{|>{\centering\scriptsize}m{2em}|>{\centering\scriptsize}m{1.3em}|>{\centering}m{8.8em}|}
  % \caption{秦王政}\
  \toprule
  \SimHei \normalsize 年数 & \SimHei \scriptsize 公元 & \SimHei 大事件 \tabularnewline
  % \midrule
  \endfirsthead
  \toprule
  \SimHei \normalsize 年数 & \SimHei \scriptsize 公元 & \SimHei 大事件 \tabularnewline
  \midrule
  \endhead
  \midrule
  元年 & 256 & \tabularnewline\hline
  二年 & 257 & \tabularnewline\hline
  三年 & 258 & \tabularnewline\hline
  四年 & 259 & \tabularnewline\hline
  五年 & 260 & \tabularnewline
  \bottomrule
\end{longtable}


%%% Local Variables:
%%% mode: latex
%%% TeX-engine: xetex
%%% TeX-master: "../../Main"
%%% End:

%% -*- coding: utf-8 -*-
%% Time-stamp: <Chen Wang: 2018-07-10 20:51:45>

\subsection{元帝\tiny(260-265)}

\subsubsection{景元}

\begin{longtable}{|>{\centering\scriptsize}m{2em}|>{\centering\scriptsize}m{1.3em}|>{\centering}m{8.8em}|}
  % \caption{秦王政}\
  \toprule
  \SimHei \normalsize 年数 & \SimHei \scriptsize 公元 & \SimHei 大事件 \tabularnewline
  % \midrule
  \endfirsthead
  \toprule
  \SimHei \normalsize 年数 & \SimHei \scriptsize 公元 & \SimHei 大事件 \tabularnewline
  \midrule
  \endhead
  \midrule
  元年 & 260 & \tabularnewline\hline
  二年 & 261 & \tabularnewline\hline
  三年 & 262 & \tabularnewline\hline
  四年 & 263 & \tabularnewline\hline
  五年 & 264 & \tabularnewline
  \bottomrule
\end{longtable}

\subsubsection{咸熙}

\begin{longtable}{|>{\centering\scriptsize}m{2em}|>{\centering\scriptsize}m{1.3em}|>{\centering}m{8.8em}|}
  % \caption{秦王政}\
  \toprule
  \SimHei \normalsize 年数 & \SimHei \scriptsize 公元 & \SimHei 大事件 \tabularnewline
  % \midrule
  \endfirsthead
  \toprule
  \SimHei \normalsize 年数 & \SimHei \scriptsize 公元 & \SimHei 大事件 \tabularnewline
  \midrule
  \endhead
  \midrule
  元年 & 264 & \tabularnewline\hline
  二年 & 265 & \tabularnewline
  \bottomrule
\end{longtable}


%%% Local Variables:
%%% mode: latex
%%% TeX-engine: xetex
%%% TeX-master: "../../Main"
%%% End:


%%% Local Variables:
%%% mode: latex
%%% TeX-engine: xetex
%%% TeX-master: "../../Main"
%%% End:

%% -*- coding: utf-8 -*-
%% Time-stamp: <Chen Wang: 2018-07-10 21:56:46>


\section{蜀汉\tiny(221-263)}

%% -*- coding: utf-8 -*-
%% Time-stamp: <Chen Wang: 2018-07-10 21:58:07>

\subsection{昭烈帝\tiny(221-223)}

\subsubsection{章武}

\begin{longtable}{|>{\centering\scriptsize}m{2em}|>{\centering\scriptsize}m{1.3em}|>{\centering}m{8.8em}|}
  % \caption{秦王政}\
  \toprule
  \SimHei \normalsize 年数 & \SimHei \scriptsize 公元 & \SimHei 大事件 \tabularnewline
  % \midrule
  \endfirsthead
  \toprule
  \SimHei \normalsize 年数 & \SimHei \scriptsize 公元 & \SimHei 大事件 \tabularnewline
  \midrule
  \endhead
  \midrule
  元年 & 221 & \tabularnewline\hline
  二年 & 222 & \tabularnewline\hline
  三年 & 223 & \tabularnewline
  \bottomrule
\end{longtable}


%%% Local Variables:
%%% mode: latex
%%% TeX-engine: xetex
%%% TeX-master: "../../Main"
%%% End:

%% -*- coding: utf-8 -*-
%% Time-stamp: <Chen Wang: 2018-07-10 22:01:18>

\subsection{后主\tiny(223-263)}

\subsubsection{建兴}

\begin{longtable}{|>{\centering\scriptsize}m{2em}|>{\centering\scriptsize}m{1.3em}|>{\centering}m{8.8em}|}
  % \caption{秦王政}\
  \toprule
  \SimHei \normalsize 年数 & \SimHei \scriptsize 公元 & \SimHei 大事件 \tabularnewline
  % \midrule
  \endfirsthead
  \toprule
  \SimHei \normalsize 年数 & \SimHei \scriptsize 公元 & \SimHei 大事件 \tabularnewline
  \midrule
  \endhead
  \midrule
  元年 & 223 & \tabularnewline\hline
  二年 & 224 & \tabularnewline\hline
  三年 & 225 & \tabularnewline\hline
  四年 & 226 & \tabularnewline\hline
  五年 & 227 & \tabularnewline\hline
  六年 & 228 & \tabularnewline\hline
  七年 & 229 & \tabularnewline\hline
  八年 & 230 & \tabularnewline\hline
  九年 & 231 & \tabularnewline\hline
  十年 & 232 & \tabularnewline\hline
  十一年 & 233 & \tabularnewline\hline
  十二年 & 234 & \tabularnewline\hline
  十三年 & 235 & \tabularnewline\hline
  十四年 & 236 & \tabularnewline\hline
  十五年 & 237 & \tabularnewline
  \bottomrule
\end{longtable}

\subsubsection{延熙}

\begin{longtable}{|>{\centering\scriptsize}m{2em}|>{\centering\scriptsize}m{1.3em}|>{\centering}m{8.8em}|}
  % \caption{秦王政}\
  \toprule
  \SimHei \normalsize 年数 & \SimHei \scriptsize 公元 & \SimHei 大事件 \tabularnewline
  % \midrule
  \endfirsthead
  \toprule
  \SimHei \normalsize 年数 & \SimHei \scriptsize 公元 & \SimHei 大事件 \tabularnewline
  \midrule
  \endhead
  \midrule
  元年 & 238 & \tabularnewline\hline
  二年 & 239 & \tabularnewline\hline
  三年 & 240 & \tabularnewline\hline
  四年 & 241 & \tabularnewline\hline
  五年 & 242 & \tabularnewline\hline
  六年 & 243 & \tabularnewline\hline
  七年 & 244 & \tabularnewline\hline
  八年 & 245 & \tabularnewline\hline
  九年 & 246 & \tabularnewline\hline
  十年 & 247 & \tabularnewline\hline
  十一年 & 248 & \tabularnewline\hline
  十二年 & 249 & \tabularnewline\hline
  十三年 & 250 & \tabularnewline\hline
  十四年 & 251 & \tabularnewline\hline
  十五年 & 252 & \tabularnewline\hline
  十六年 & 253 & \tabularnewline\hline
  十七年 & 254 & \tabularnewline\hline
  十八年 & 255 & \tabularnewline\hline
  十九年 & 256 & \tabularnewline\hline
  二十年 & 257 & \tabularnewline
  \bottomrule
\end{longtable}

\subsubsection{景耀}

\begin{longtable}{|>{\centering\scriptsize}m{2em}|>{\centering\scriptsize}m{1.3em}|>{\centering}m{8.8em}|}
  % \caption{秦王政}\
  \toprule
  \SimHei \normalsize 年数 & \SimHei \scriptsize 公元 & \SimHei 大事件 \tabularnewline
  % \midrule
  \endfirsthead
  \toprule
  \SimHei \normalsize 年数 & \SimHei \scriptsize 公元 & \SimHei 大事件 \tabularnewline
  \midrule
  \endhead
  \midrule
  元年 & 258 & \tabularnewline\hline
  二年 & 259 & \tabularnewline\hline
  三年 & 260 & \tabularnewline\hline
  四年 & 261 & \tabularnewline\hline
  五年 & 262 & \tabularnewline\hline
  六年 & 263 & \tabularnewline
  \bottomrule
\end{longtable}

\subsubsection{炎兴}

\begin{longtable}{|>{\centering\scriptsize}m{2em}|>{\centering\scriptsize}m{1.3em}|>{\centering}m{8.8em}|}
  % \caption{秦王政}\
  \toprule
  \SimHei \normalsize 年数 & \SimHei \scriptsize 公元 & \SimHei 大事件 \tabularnewline
  % \midrule
  \endfirsthead
  \toprule
  \SimHei \normalsize 年数 & \SimHei \scriptsize 公元 & \SimHei 大事件 \tabularnewline
  \midrule
  \endhead
  \midrule
  元年 & 263 & \tabularnewline
  \bottomrule
\end{longtable}


%%% Local Variables:
%%% mode: latex
%%% TeX-engine: xetex
%%% TeX-master: "../../Main"
%%% End:


%%% Local Variables:
%%% mode: latex
%%% TeX-engine: xetex
%%% TeX-master: "../../Main"
%%% End:

%% -*- coding: utf-8 -*-
%% Time-stamp: <Chen Wang: 2018-07-10 22:03:26>


\section{孙吴\tiny(229-280)}

%% -*- coding: utf-8 -*-
%% Time-stamp: <Chen Wang: 2018-07-10 22:07:57>

\subsection{大帝\tiny(229-252)}

\subsubsection{黄武}

\begin{longtable}{|>{\centering\scriptsize}m{2em}|>{\centering\scriptsize}m{1.3em}|>{\centering}m{8.8em}|}
  % \caption{秦王政}\
  \toprule
  \SimHei \normalsize 年数 & \SimHei \scriptsize 公元 & \SimHei 大事件 \tabularnewline
  % \midrule
  \endfirsthead
  \toprule
  \SimHei \normalsize 年数 & \SimHei \scriptsize 公元 & \SimHei 大事件 \tabularnewline
  \midrule
  \endhead
  \midrule
  元年 & 222 & \tabularnewline\hline
  二年 & 223 & \tabularnewline\hline
  三年 & 224 & \tabularnewline\hline
  四年 & 225 & \tabularnewline\hline
  五年 & 226 & \tabularnewline\hline
  六年 & 227 & \tabularnewline\hline
  七年 & 228 & \tabularnewline\hline
  八年 & 229 & \tabularnewline
  \bottomrule
\end{longtable}

\subsubsection{黄龙}

\begin{longtable}{|>{\centering\scriptsize}m{2em}|>{\centering\scriptsize}m{1.3em}|>{\centering}m{8.8em}|}
  % \caption{秦王政}\
  \toprule
  \SimHei \normalsize 年数 & \SimHei \scriptsize 公元 & \SimHei 大事件 \tabularnewline
  % \midrule
  \endfirsthead
  \toprule
  \SimHei \normalsize 年数 & \SimHei \scriptsize 公元 & \SimHei 大事件 \tabularnewline
  \midrule
  \endhead
  \midrule
  元年 & 229 & \tabularnewline\hline
  二年 & 230 & \tabularnewline\hline
  三年 & 231 & \tabularnewline
  \bottomrule
\end{longtable}

\subsubsection{嘉禾}

\begin{longtable}{|>{\centering\scriptsize}m{2em}|>{\centering\scriptsize}m{1.3em}|>{\centering}m{8.8em}|}
  % \caption{秦王政}\
  \toprule
  \SimHei \normalsize 年数 & \SimHei \scriptsize 公元 & \SimHei 大事件 \tabularnewline
  % \midrule
  \endfirsthead
  \toprule
  \SimHei \normalsize 年数 & \SimHei \scriptsize 公元 & \SimHei 大事件 \tabularnewline
  \midrule
  \endhead
  \midrule
  元年 & 232 & \tabularnewline\hline
  二年 & 233 & \tabularnewline\hline
  三年 & 234 & \tabularnewline\hline
  四年 & 235 & \tabularnewline\hline
  五年 & 236 & \tabularnewline\hline
  六年 & 237 & \tabularnewline\hline
  七年 & 238 & \tabularnewline
  \bottomrule
\end{longtable}

\subsubsection{赤乌}

\begin{longtable}{|>{\centering\scriptsize}m{2em}|>{\centering\scriptsize}m{1.3em}|>{\centering}m{8.8em}|}
  % \caption{秦王政}\
  \toprule
  \SimHei \normalsize 年数 & \SimHei \scriptsize 公元 & \SimHei 大事件 \tabularnewline
  % \midrule
  \endfirsthead
  \toprule
  \SimHei \normalsize 年数 & \SimHei \scriptsize 公元 & \SimHei 大事件 \tabularnewline
  \midrule
  \endhead
  \midrule
  元年 & 238 & \tabularnewline\hline
  二年 & 239 & \tabularnewline\hline
  三年 & 240 & \tabularnewline\hline
  四年 & 241 & \tabularnewline\hline
  五年 & 242 & \tabularnewline\hline
  六年 & 243 & \tabularnewline\hline
  七年 & 244 & \tabularnewline\hline
  八年 & 245 & \tabularnewline\hline
  九年 & 246 & \tabularnewline\hline
  十年 & 247 & \tabularnewline\hline
  十一年 & 248 & \tabularnewline\hline
  十二年 & 249 & \tabularnewline\hline
  十三年 & 250 & \tabularnewline\hline
  十四年 & 251 & \tabularnewline
  \bottomrule
\end{longtable}

\subsubsection{太元}

\begin{longtable}{|>{\centering\scriptsize}m{2em}|>{\centering\scriptsize}m{1.3em}|>{\centering}m{8.8em}|}
  % \caption{秦王政}\
  \toprule
  \SimHei \normalsize 年数 & \SimHei \scriptsize 公元 & \SimHei 大事件 \tabularnewline
  % \midrule
  \endfirsthead
  \toprule
  \SimHei \normalsize 年数 & \SimHei \scriptsize 公元 & \SimHei 大事件 \tabularnewline
  \midrule
  \endhead
  \midrule
  元年 & 251 & \tabularnewline\hline
  二年 & 252 & \tabularnewline
  \bottomrule
\end{longtable}

\subsubsection{神凤}

\begin{longtable}{|>{\centering\scriptsize}m{2em}|>{\centering\scriptsize}m{1.3em}|>{\centering}m{8.8em}|}
  % \caption{秦王政}\
  \toprule
  \SimHei \normalsize 年数 & \SimHei \scriptsize 公元 & \SimHei 大事件 \tabularnewline
  % \midrule
  \endfirsthead
  \toprule
  \SimHei \normalsize 年数 & \SimHei \scriptsize 公元 & \SimHei 大事件 \tabularnewline
  \midrule
  \endhead
  \midrule
  元年 & 252 & \tabularnewline
  \bottomrule
\end{longtable}


%%% Local Variables:
%%% mode: latex
%%% TeX-engine: xetex
%%% TeX-master: "../../Main"
%%% End:

%% -*- coding: utf-8 -*-
%% Time-stamp: <Chen Wang: 2018-07-10 22:09:30>

\subsection{孙亮\tiny(252-258)}

\subsubsection{建兴}

\begin{longtable}{|>{\centering\scriptsize}m{2em}|>{\centering\scriptsize}m{1.3em}|>{\centering}m{8.8em}|}
  % \caption{秦王政}\
  \toprule
  \SimHei \normalsize 年数 & \SimHei \scriptsize 公元 & \SimHei 大事件 \tabularnewline
  % \midrule
  \endfirsthead
  \toprule
  \SimHei \normalsize 年数 & \SimHei \scriptsize 公元 & \SimHei 大事件 \tabularnewline
  \midrule
  \endhead
  \midrule
  元年 & 252 & \tabularnewline\hline
  二年 & 253 & \tabularnewline
  \bottomrule
\end{longtable}

\subsubsection{五凤}

\begin{longtable}{|>{\centering\scriptsize}m{2em}|>{\centering\scriptsize}m{1.3em}|>{\centering}m{8.8em}|}
  % \caption{秦王政}\
  \toprule
  \SimHei \normalsize 年数 & \SimHei \scriptsize 公元 & \SimHei 大事件 \tabularnewline
  % \midrule
  \endfirsthead
  \toprule
  \SimHei \normalsize 年数 & \SimHei \scriptsize 公元 & \SimHei 大事件 \tabularnewline
  \midrule
  \endhead
  \midrule
  元年 & 254 & \tabularnewline\hline
  二年 & 255 & \tabularnewline\hline
  三年 & 256 & \tabularnewline
  \bottomrule
\end{longtable}

\subsubsection{太平}

\begin{longtable}{|>{\centering\scriptsize}m{2em}|>{\centering\scriptsize}m{1.3em}|>{\centering}m{8.8em}|}
  % \caption{秦王政}\
  \toprule
  \SimHei \normalsize 年数 & \SimHei \scriptsize 公元 & \SimHei 大事件 \tabularnewline
  % \midrule
  \endfirsthead
  \toprule
  \SimHei \normalsize 年数 & \SimHei \scriptsize 公元 & \SimHei 大事件 \tabularnewline
  \midrule
  \endhead
  \midrule
  元年 & 256 & \tabularnewline\hline
  二年 & 257 & \tabularnewline\hline
  三年 & 258 & \tabularnewline
  \bottomrule
\end{longtable}


%%% Local Variables:
%%% mode: latex
%%% TeX-engine: xetex
%%% TeX-master: "../../Main"
%%% End:

%% -*- coding: utf-8 -*-
%% Time-stamp: <Chen Wang: 2018-07-10 22:10:27>

\subsection{景帝\tiny(258-264)}

\subsubsection{永安}

\begin{longtable}{|>{\centering\scriptsize}m{2em}|>{\centering\scriptsize}m{1.3em}|>{\centering}m{8.8em}|}
  % \caption{秦王政}\
  \toprule
  \SimHei \normalsize 年数 & \SimHei \scriptsize 公元 & \SimHei 大事件 \tabularnewline
  % \midrule
  \endfirsthead
  \toprule
  \SimHei \normalsize 年数 & \SimHei \scriptsize 公元 & \SimHei 大事件 \tabularnewline
  \midrule
  \endhead
  \midrule
  元年 & 258 & \tabularnewline\hline
  二年 & 259 & \tabularnewline\hline
  三年 & 260 & \tabularnewline\hline
  四年 & 261 & \tabularnewline\hline
  五年 & 262 & \tabularnewline\hline
  六年 & 263 & \tabularnewline\hline
  七年 & 264 & \tabularnewline
  \bottomrule
\end{longtable}



%%% Local Variables:
%%% mode: latex
%%% TeX-engine: xetex
%%% TeX-master: "../../Main"
%%% End:

%% -*- coding: utf-8 -*-
%% Time-stamp: <Chen Wang: 2018-07-10 22:14:01>

\subsection{孙皓\tiny(264-280)}

\subsubsection{元兴}

\begin{longtable}{|>{\centering\scriptsize}m{2em}|>{\centering\scriptsize}m{1.3em}|>{\centering}m{8.8em}|}
  % \caption{秦王政}\
  \toprule
  \SimHei \normalsize 年数 & \SimHei \scriptsize 公元 & \SimHei 大事件 \tabularnewline
  % \midrule
  \endfirsthead
  \toprule
  \SimHei \normalsize 年数 & \SimHei \scriptsize 公元 & \SimHei 大事件 \tabularnewline
  \midrule
  \endhead
  \midrule
  元年 & 264 & \tabularnewline\hline
  二年 & 265 & \tabularnewline
  \bottomrule
\end{longtable}


\subsubsection{甘露}

\begin{longtable}{|>{\centering\scriptsize}m{2em}|>{\centering\scriptsize}m{1.3em}|>{\centering}m{8.8em}|}
  % \caption{秦王政}\
  \toprule
  \SimHei \normalsize 年数 & \SimHei \scriptsize 公元 & \SimHei 大事件 \tabularnewline
  % \midrule
  \endfirsthead
  \toprule
  \SimHei \normalsize 年数 & \SimHei \scriptsize 公元 & \SimHei 大事件 \tabularnewline
  \midrule
  \endhead
  \midrule
  元年 & 265 & \tabularnewline\hline
  二年 & 266 & \tabularnewline
  \bottomrule
\end{longtable}

\subsubsection{宝鼎}

\begin{longtable}{|>{\centering\scriptsize}m{2em}|>{\centering\scriptsize}m{1.3em}|>{\centering}m{8.8em}|}
  % \caption{秦王政}\
  \toprule
  \SimHei \normalsize 年数 & \SimHei \scriptsize 公元 & \SimHei 大事件 \tabularnewline
  % \midrule
  \endfirsthead
  \toprule
  \SimHei \normalsize 年数 & \SimHei \scriptsize 公元 & \SimHei 大事件 \tabularnewline
  \midrule
  \endhead
  \midrule
  元年 & 266 & \tabularnewline\hline
  二年 & 267 & \tabularnewline\hline
  三年 & 268 & \tabularnewline\hline
  四年 & 269 & \tabularnewline
  \bottomrule
\end{longtable}

\subsubsection{建衡}

\begin{longtable}{|>{\centering\scriptsize}m{2em}|>{\centering\scriptsize}m{1.3em}|>{\centering}m{8.8em}|}
  % \caption{秦王政}\
  \toprule
  \SimHei \normalsize 年数 & \SimHei \scriptsize 公元 & \SimHei 大事件 \tabularnewline
  % \midrule
  \endfirsthead
  \toprule
  \SimHei \normalsize 年数 & \SimHei \scriptsize 公元 & \SimHei 大事件 \tabularnewline
  \midrule
  \endhead
  \midrule
  元年 & 269 & \tabularnewline\hline
  二年 & 270 & \tabularnewline\hline
  三年 & 271 & \tabularnewline
  \bottomrule
\end{longtable}

\subsubsection{凤凰}

\begin{longtable}{|>{\centering\scriptsize}m{2em}|>{\centering\scriptsize}m{1.3em}|>{\centering}m{8.8em}|}
  % \caption{秦王政}\
  \toprule
  \SimHei \normalsize 年数 & \SimHei \scriptsize 公元 & \SimHei 大事件 \tabularnewline
  % \midrule
  \endfirsthead
  \toprule
  \SimHei \normalsize 年数 & \SimHei \scriptsize 公元 & \SimHei 大事件 \tabularnewline
  \midrule
  \endhead
  \midrule
  元年 & 272 & \tabularnewline\hline
  二年 & 273 & \tabularnewline\hline
  三年 & 274 & \tabularnewline
  \bottomrule
\end{longtable}

\subsubsection{天册}

\begin{longtable}{|>{\centering\scriptsize}m{2em}|>{\centering\scriptsize}m{1.3em}|>{\centering}m{8.8em}|}
  % \caption{秦王政}\
  \toprule
  \SimHei \normalsize 年数 & \SimHei \scriptsize 公元 & \SimHei 大事件 \tabularnewline
  % \midrule
  \endfirsthead
  \toprule
  \SimHei \normalsize 年数 & \SimHei \scriptsize 公元 & \SimHei 大事件 \tabularnewline
  \midrule
  \endhead
  \midrule
  元年 & 275 & \tabularnewline\hline
  二年 & 276 & \tabularnewline
  \bottomrule
\end{longtable}

\subsubsection{天玺}

\begin{longtable}{|>{\centering\scriptsize}m{2em}|>{\centering\scriptsize}m{1.3em}|>{\centering}m{8.8em}|}
  % \caption{秦王政}\
  \toprule
  \SimHei \normalsize 年数 & \SimHei \scriptsize 公元 & \SimHei 大事件 \tabularnewline
  % \midrule
  \endfirsthead
  \toprule
  \SimHei \normalsize 年数 & \SimHei \scriptsize 公元 & \SimHei 大事件 \tabularnewline
  \midrule
  \endhead
  \midrule
  元年 & 276 & \tabularnewline
  \bottomrule
\end{longtable}

\subsubsection{天纪}

\begin{longtable}{|>{\centering\scriptsize}m{2em}|>{\centering\scriptsize}m{1.3em}|>{\centering}m{8.8em}|}
  % \caption{秦王政}\
  \toprule
  \SimHei \normalsize 年数 & \SimHei \scriptsize 公元 & \SimHei 大事件 \tabularnewline
  % \midrule
  \endfirsthead
  \toprule
  \SimHei \normalsize 年数 & \SimHei \scriptsize 公元 & \SimHei 大事件 \tabularnewline
  \midrule
  \endhead
  \midrule
  元年 & 277 & \tabularnewline\hline
  二年 & 278 & \tabularnewline\hline
  三年 & 279 & \tabularnewline\hline
  四年 & 280 & \tabularnewline
  \bottomrule
\end{longtable}


%%% Local Variables:
%%% mode: latex
%%% TeX-engine: xetex
%%% TeX-master: "../../Main"
%%% End:


%%% Local Variables:
%%% mode: latex
%%% TeX-engine: xetex
%%% TeX-master: "../../Main"
%%% End:


%%% Local Variables:
%%% mode: latex
%%% TeX-engine: xetex
%%% TeX-master: "../Main"
%%% End:

% %% -*- coding: utf-8 -*-
%% Time-stamp: <Chen Wang: 2018-07-10 22:32:02>

\chapter{西晋\tiny(265-316)}

%% -*- coding: utf-8 -*-
%% Time-stamp: <Chen Wang: 2018-07-10 22:23:17>

\section{武帝\tiny(266-290)}

\subsection{泰始}

\begin{longtable}{|>{\centering\scriptsize}m{2em}|>{\centering\scriptsize}m{1.3em}|>{\centering}m{8.8em}|}
  % \caption{秦王政}\
  \toprule
  \SimHei \normalsize 年数 & \SimHei \scriptsize 公元 & \SimHei 大事件 \tabularnewline
  % \midrule
  \endfirsthead
  \toprule
  \SimHei \normalsize 年数 & \SimHei \scriptsize 公元 & \SimHei 大事件 \tabularnewline
  \midrule
  \endhead
  \midrule
  元年 & 265 & \tabularnewline\hline
  二年 & 266 & \tabularnewline\hline
  三年 & 267 & \tabularnewline\hline
  四年 & 268 & \tabularnewline\hline
  五年 & 269 & \tabularnewline\hline
  六年 & 270 & \tabularnewline\hline
  七年 & 271 & \tabularnewline\hline
  八年 & 272 & \tabularnewline\hline
  九年 & 273 & \tabularnewline\hline
  十年 & 274 & \tabularnewline
  \bottomrule
\end{longtable}

\subsection{咸宁}


\begin{longtable}{|>{\centering\scriptsize}m{2em}|>{\centering\scriptsize}m{1.3em}|>{\centering}m{8.8em}|}
  % \caption{秦王政}\
  \toprule
  \SimHei \normalsize 年数 & \SimHei \scriptsize 公元 & \SimHei 大事件 \tabularnewline
  % \midrule
  \endfirsthead
  \toprule
  \SimHei \normalsize 年数 & \SimHei \scriptsize 公元 & \SimHei 大事件 \tabularnewline
  \midrule
  \endhead
  \midrule
  元年 & 275 & \tabularnewline\hline
  二年 & 276 & \tabularnewline\hline
  三年 & 277 & \tabularnewline\hline
  四年 & 278 & \tabularnewline\hline
  五年 & 279 & \tabularnewline\hline
  六年 & 280 & \tabularnewline
  \bottomrule
\end{longtable}

\subsection{太康}

\begin{longtable}{|>{\centering\scriptsize}m{2em}|>{\centering\scriptsize}m{1.3em}|>{\centering}m{8.8em}|}
  % \caption{秦王政}\
  \toprule
  \SimHei \normalsize 年数 & \SimHei \scriptsize 公元 & \SimHei 大事件 \tabularnewline
  % \midrule
  \endfirsthead
  \toprule
  \SimHei \normalsize 年数 & \SimHei \scriptsize 公元 & \SimHei 大事件 \tabularnewline
  \midrule
  \endhead
  \midrule
  元年 & 280 & \tabularnewline\hline
  二年 & 281 & \tabularnewline\hline
  三年 & 282 & \tabularnewline\hline
  四年 & 283 & \tabularnewline\hline
  五年 & 284 & \tabularnewline\hline
  六年 & 285 & \tabularnewline\hline
  七年 & 286 & \tabularnewline\hline
  八年 & 287 & \tabularnewline\hline
  九年 & 288 & \tabularnewline\hline
  十年 & 289 & \tabularnewline
  \bottomrule
\end{longtable}

\subsection{太熙}

\begin{longtable}{|>{\centering\scriptsize}m{2em}|>{\centering\scriptsize}m{1.3em}|>{\centering}m{8.8em}|}
  % \caption{秦王政}\
  \toprule
  \SimHei \normalsize 年数 & \SimHei \scriptsize 公元 & \SimHei 大事件 \tabularnewline
  % \midrule
  \endfirsthead
  \toprule
  \SimHei \normalsize 年数 & \SimHei \scriptsize 公元 & \SimHei 大事件 \tabularnewline
  \midrule
  \endhead
  \midrule
  元年 & 290 & \tabularnewline
  \bottomrule
\end{longtable}


%%% Local Variables:
%%% mode: latex
%%% TeX-engine: xetex
%%% TeX-master: "../Main"
%%% End:

%% -*- coding: utf-8 -*-
%% Time-stamp: <Chen Wang: 2018-07-10 22:29:00>

\section{惠帝\tiny(290-306)}

\subsection{永熙}

\begin{longtable}{|>{\centering\scriptsize}m{2em}|>{\centering\scriptsize}m{1.3em}|>{\centering}m{8.8em}|}
  % \caption{秦王政}\
  \toprule
  \SimHei \normalsize 年数 & \SimHei \scriptsize 公元 & \SimHei 大事件 \tabularnewline
  % \midrule
  \endfirsthead
  \toprule
  \SimHei \normalsize 年数 & \SimHei \scriptsize 公元 & \SimHei 大事件 \tabularnewline
  \midrule
  \endhead
  \midrule
  元年 & 290 & \tabularnewline
  \bottomrule
\end{longtable}

\subsection{永平}

\begin{longtable}{|>{\centering\scriptsize}m{2em}|>{\centering\scriptsize}m{1.3em}|>{\centering}m{8.8em}|}
  % \caption{秦王政}\
  \toprule
  \SimHei \normalsize 年数 & \SimHei \scriptsize 公元 & \SimHei 大事件 \tabularnewline
  % \midrule
  \endfirsthead
  \toprule
  \SimHei \normalsize 年数 & \SimHei \scriptsize 公元 & \SimHei 大事件 \tabularnewline
  \midrule
  \endhead
  \midrule
  元年 & 291 & \tabularnewline
  \bottomrule
\end{longtable}

\subsection{元康}

\begin{longtable}{|>{\centering\scriptsize}m{2em}|>{\centering\scriptsize}m{1.3em}|>{\centering}m{8.8em}|}
  % \caption{秦王政}\
  \toprule
  \SimHei \normalsize 年数 & \SimHei \scriptsize 公元 & \SimHei 大事件 \tabularnewline
  % \midrule
  \endfirsthead
  \toprule
  \SimHei \normalsize 年数 & \SimHei \scriptsize 公元 & \SimHei 大事件 \tabularnewline
  \midrule
  \endhead
  \midrule
  元年 & 291 & \tabularnewline\hline
  二年 & 292 & \tabularnewline\hline
  三年 & 293 & \tabularnewline\hline
  四年 & 294 & \tabularnewline\hline
  五年 & 295 & \tabularnewline\hline
  六年 & 296 & \tabularnewline\hline
  七年 & 297 & \tabularnewline\hline
  八年 & 298 & \tabularnewline\hline
  九年 & 299 & \tabularnewline
  \bottomrule
\end{longtable}

\subsection{永康}

\begin{longtable}{|>{\centering\scriptsize}m{2em}|>{\centering\scriptsize}m{1.3em}|>{\centering}m{8.8em}|}
  % \caption{秦王政}\
  \toprule
  \SimHei \normalsize 年数 & \SimHei \scriptsize 公元 & \SimHei 大事件 \tabularnewline
  % \midrule
  \endfirsthead
  \toprule
  \SimHei \normalsize 年数 & \SimHei \scriptsize 公元 & \SimHei 大事件 \tabularnewline
  \midrule
  \endhead
  \midrule
  元年 & 300 & \tabularnewline\hline
  二年 & 301 & \tabularnewline
  \bottomrule
\end{longtable}

\subsection{永宁}

\begin{longtable}{|>{\centering\scriptsize}m{2em}|>{\centering\scriptsize}m{1.3em}|>{\centering}m{8.8em}|}
  % \caption{秦王政}\
  \toprule
  \SimHei \normalsize 年数 & \SimHei \scriptsize 公元 & \SimHei 大事件 \tabularnewline
  % \midrule
  \endfirsthead
  \toprule
  \SimHei \normalsize 年数 & \SimHei \scriptsize 公元 & \SimHei 大事件 \tabularnewline
  \midrule
  \endhead
  \midrule
  元年 & 301 & \tabularnewline\hline
  二年 & 302 & \tabularnewline
  \bottomrule
\end{longtable}

\subsection{太安}

\begin{longtable}{|>{\centering\scriptsize}m{2em}|>{\centering\scriptsize}m{1.3em}|>{\centering}m{8.8em}|}
  % \caption{秦王政}\
  \toprule
  \SimHei \normalsize 年数 & \SimHei \scriptsize 公元 & \SimHei 大事件 \tabularnewline
  % \midrule
  \endfirsthead
  \toprule
  \SimHei \normalsize 年数 & \SimHei \scriptsize 公元 & \SimHei 大事件 \tabularnewline
  \midrule
  \endhead
  \midrule
  元年 & 302 & \tabularnewline\hline
  二年 & 303 & \tabularnewline
  \bottomrule
\end{longtable}

\subsection{永安}

\begin{longtable}{|>{\centering\scriptsize}m{2em}|>{\centering\scriptsize}m{1.3em}|>{\centering}m{8.8em}|}
  % \caption{秦王政}\
  \toprule
  \SimHei \normalsize 年数 & \SimHei \scriptsize 公元 & \SimHei 大事件 \tabularnewline
  % \midrule
  \endfirsthead
  \toprule
  \SimHei \normalsize 年数 & \SimHei \scriptsize 公元 & \SimHei 大事件 \tabularnewline
  \midrule
  \endhead
  \midrule
  元年 & 304 & \tabularnewline
  \bottomrule
\end{longtable}

\subsection{建武}

\begin{longtable}{|>{\centering\scriptsize}m{2em}|>{\centering\scriptsize}m{1.3em}|>{\centering}m{8.8em}|}
  % \caption{秦王政}\
  \toprule
  \SimHei \normalsize 年数 & \SimHei \scriptsize 公元 & \SimHei 大事件 \tabularnewline
  % \midrule
  \endfirsthead
  \toprule
  \SimHei \normalsize 年数 & \SimHei \scriptsize 公元 & \SimHei 大事件 \tabularnewline
  \midrule
  \endhead
  \midrule
  元年 & 304 & \tabularnewline
  \bottomrule
\end{longtable}

\subsection{永兴}

\begin{longtable}{|>{\centering\scriptsize}m{2em}|>{\centering\scriptsize}m{1.3em}|>{\centering}m{8.8em}|}
  % \caption{秦王政}\
  \toprule
  \SimHei \normalsize 年数 & \SimHei \scriptsize 公元 & \SimHei 大事件 \tabularnewline
  % \midrule
  \endfirsthead
  \toprule
  \SimHei \normalsize 年数 & \SimHei \scriptsize 公元 & \SimHei 大事件 \tabularnewline
  \midrule
  \endhead
  \midrule
  元年 & 304 & \tabularnewline\hline
  二年 & 305 & \tabularnewline\hline
  三年 & 306 & \tabularnewline
  \bottomrule
\end{longtable}

\subsection{光熙}

\begin{longtable}{|>{\centering\scriptsize}m{2em}|>{\centering\scriptsize}m{1.3em}|>{\centering}m{8.8em}|}
  % \caption{秦王政}\
  \toprule
  \SimHei \normalsize 年数 & \SimHei \scriptsize 公元 & \SimHei 大事件 \tabularnewline
  % \midrule
  \endfirsthead
  \toprule
  \SimHei \normalsize 年数 & \SimHei \scriptsize 公元 & \SimHei 大事件 \tabularnewline
  \midrule
  \endhead
  \midrule
  元年 & 306 & \tabularnewline
  \bottomrule
\end{longtable}


%%% Local Variables:
%%% mode: latex
%%% TeX-engine: xetex
%%% TeX-master: "../Main"
%%% End:

%% -*- coding: utf-8 -*-
%% Time-stamp: <Chen Wang: 2018-07-10 22:30:05>

\section{怀帝\tiny(306-313)}

\subsection{永嘉}

\begin{longtable}{|>{\centering\scriptsize}m{2em}|>{\centering\scriptsize}m{1.3em}|>{\centering}m{8.8em}|}
  % \caption{秦王政}\
  \toprule
  \SimHei \normalsize 年数 & \SimHei \scriptsize 公元 & \SimHei 大事件 \tabularnewline
  % \midrule
  \endfirsthead
  \toprule
  \SimHei \normalsize 年数 & \SimHei \scriptsize 公元 & \SimHei 大事件 \tabularnewline
  \midrule
  \endhead
  \midrule
  元年 & 307 & \tabularnewline\hline
  二年 & 308 & \tabularnewline\hline
  三年 & 309 & \tabularnewline\hline
  四年 & 310 & \tabularnewline\hline
  五年 & 311 & \tabularnewline\hline
  六年 & 312 & \tabularnewline\hline
  七年 & 313 & \tabularnewline
  \bottomrule
\end{longtable}


%%% Local Variables:
%%% mode: latex
%%% TeX-engine: xetex
%%% TeX-master: "../Main"
%%% End:

%% -*- coding: utf-8 -*-
%% Time-stamp: <Chen Wang: 2018-07-10 22:31:10>

\section{愍帝\tiny(313-316)}

\subsection{建兴}

\begin{longtable}{|>{\centering\scriptsize}m{2em}|>{\centering\scriptsize}m{1.3em}|>{\centering}m{8.8em}|}
  % \caption{秦王政}\
  \toprule
  \SimHei \normalsize 年数 & \SimHei \scriptsize 公元 & \SimHei 大事件 \tabularnewline
  % \midrule
  \endfirsthead
  \toprule
  \SimHei \normalsize 年数 & \SimHei \scriptsize 公元 & \SimHei 大事件 \tabularnewline
  \midrule
  \endhead
  \midrule
  元年 & 313 & \tabularnewline\hline
  二年 & 314 & \tabularnewline\hline
  三年 & 315 & \tabularnewline\hline
  四年 & 316 & \tabularnewline\hline
  五年 & 317 & \tabularnewline
  \bottomrule
\end{longtable}


%%% Local Variables:
%%% mode: latex
%%% TeX-engine: xetex
%%% TeX-master: "../Main"
%%% End:


%%% Local Variables:
%%% mode: latex
%%% TeX-engine: xetex
%%% TeX-master: "../Main"
%%% End:

% %% -*- coding: utf-8 -*-
%% Time-stamp: <Chen Wang: 2018-07-10 22:56:32>

\chapter{东晋\tiny(317-420)}

%% -*- coding: utf-8 -*-
%% Time-stamp: <Chen Wang: 2018-07-10 22:35:57>

\section{元帝\tiny(318-322)}

\subsection{建武}

\begin{longtable}{|>{\centering\scriptsize}m{2em}|>{\centering\scriptsize}m{1.3em}|>{\centering}m{8.8em}|}
  % \caption{秦王政}\
  \toprule
  \SimHei \normalsize 年数 & \SimHei \scriptsize 公元 & \SimHei 大事件 \tabularnewline
  % \midrule
  \endfirsthead
  \toprule
  \SimHei \normalsize 年数 & \SimHei \scriptsize 公元 & \SimHei 大事件 \tabularnewline
  \midrule
  \endhead
  \midrule
  元年 & 317 & \tabularnewline\hline
  二年 & 318 & \tabularnewline
  \bottomrule
\end{longtable}

\subsection{大兴}

\begin{longtable}{|>{\centering\scriptsize}m{2em}|>{\centering\scriptsize}m{1.3em}|>{\centering}m{8.8em}|}
  % \caption{秦王政}\
  \toprule
  \SimHei \normalsize 年数 & \SimHei \scriptsize 公元 & \SimHei 大事件 \tabularnewline
  % \midrule
  \endfirsthead
  \toprule
  \SimHei \normalsize 年数 & \SimHei \scriptsize 公元 & \SimHei 大事件 \tabularnewline
  \midrule
  \endhead
  \midrule
  元年 & 318 & \tabularnewline\hline
  二年 & 319 & \tabularnewline\hline
  三年 & 320 & \tabularnewline\hline
  四年 & 321 & \tabularnewline
  \bottomrule
\end{longtable}

\subsection{永昌}

\begin{longtable}{|>{\centering\scriptsize}m{2em}|>{\centering\scriptsize}m{1.3em}|>{\centering}m{8.8em}|}
  % \caption{秦王政}\
  \toprule
  \SimHei \normalsize 年数 & \SimHei \scriptsize 公元 & \SimHei 大事件 \tabularnewline
  % \midrule
  \endfirsthead
  \toprule
  \SimHei \normalsize 年数 & \SimHei \scriptsize 公元 & \SimHei 大事件 \tabularnewline
  \midrule
  \endhead
  \midrule
  元年 & 322 & \tabularnewline\hline
  二年 & 323 & \tabularnewline
  \bottomrule
\end{longtable}


%%% Local Variables:
%%% mode: latex
%%% TeX-engine: xetex
%%% TeX-master: "../Main"
%%% End:

%% -*- coding: utf-8 -*-
%% Time-stamp: <Chen Wang: 2018-07-10 22:40:25>

\section{明帝\tiny(322-325)}

\subsection{太宁}

\begin{longtable}{|>{\centering\scriptsize}m{2em}|>{\centering\scriptsize}m{1.3em}|>{\centering}m{8.8em}|}
  % \caption{秦王政}\
  \toprule
  \SimHei \normalsize 年数 & \SimHei \scriptsize 公元 & \SimHei 大事件 \tabularnewline
  % \midrule
  \endfirsthead
  \toprule
  \SimHei \normalsize 年数 & \SimHei \scriptsize 公元 & \SimHei 大事件 \tabularnewline
  \midrule
  \endhead
  \midrule
  元年 & 323 & \tabularnewline\hline
  二年 & 324 & \tabularnewline\hline
  三年 & 325 & \tabularnewline\hline
  四年 & 326 & \tabularnewline
  \bottomrule
\end{longtable}


%%% Local Variables:
%%% mode: latex
%%% TeX-engine: xetex
%%% TeX-master: "../Main"
%%% End:

%% -*- coding: utf-8 -*-
%% Time-stamp: <Chen Wang: 2018-07-10 22:42:29>

\section{成帝\tiny(325-342)}

\subsection{咸和}

\begin{longtable}{|>{\centering\scriptsize}m{2em}|>{\centering\scriptsize}m{1.3em}|>{\centering}m{8.8em}|}
  % \caption{秦王政}\
  \toprule
  \SimHei \normalsize 年数 & \SimHei \scriptsize 公元 & \SimHei 大事件 \tabularnewline
  % \midrule
  \endfirsthead
  \toprule
  \SimHei \normalsize 年数 & \SimHei \scriptsize 公元 & \SimHei 大事件 \tabularnewline
  \midrule
  \endhead
  \midrule
  元年 & 326 & \tabularnewline\hline
  二年 & 327 & \tabularnewline\hline
  三年 & 328 & \tabularnewline\hline
  四年 & 329 & \tabularnewline\hline
  五年 & 330 & \tabularnewline\hline
  六年 & 331 & \tabularnewline\hline
  七年 & 332 & \tabularnewline\hline
  八年 & 333 & \tabularnewline\hline
  九年 & 334 & \tabularnewline
  \bottomrule
\end{longtable}

\subsection{咸康}

\begin{longtable}{|>{\centering\scriptsize}m{2em}|>{\centering\scriptsize}m{1.3em}|>{\centering}m{8.8em}|}
  % \caption{秦王政}\
  \toprule
  \SimHei \normalsize 年数 & \SimHei \scriptsize 公元 & \SimHei 大事件 \tabularnewline
  % \midrule
  \endfirsthead
  \toprule
  \SimHei \normalsize 年数 & \SimHei \scriptsize 公元 & \SimHei 大事件 \tabularnewline
  \midrule
  \endhead
  \midrule
  元年 & 335 & \tabularnewline\hline
  二年 & 336 & \tabularnewline\hline
  三年 & 337 & \tabularnewline\hline
  四年 & 338 & \tabularnewline\hline
  五年 & 339 & \tabularnewline\hline
  六年 & 340 & \tabularnewline\hline
  七年 & 341 & \tabularnewline\hline
  八年 & 342 & \tabularnewline
  \bottomrule
\end{longtable}


%%% Local Variables:
%%% mode: latex
%%% TeX-engine: xetex
%%% TeX-master: "../Main"
%%% End:

%% -*- coding: utf-8 -*-
%% Time-stamp: <Chen Wang: 2018-07-10 22:43:12>

\section{康帝\tiny(342-344)}

\subsection{建元}

\begin{longtable}{|>{\centering\scriptsize}m{2em}|>{\centering\scriptsize}m{1.3em}|>{\centering}m{8.8em}|}
  % \caption{秦王政}\
  \toprule
  \SimHei \normalsize 年数 & \SimHei \scriptsize 公元 & \SimHei 大事件 \tabularnewline
  % \midrule
  \endfirsthead
  \toprule
  \SimHei \normalsize 年数 & \SimHei \scriptsize 公元 & \SimHei 大事件 \tabularnewline
  \midrule
  \endhead
  \midrule
  元年 & 343 & \tabularnewline\hline
  二年 & 344 & \tabularnewline
  \bottomrule
\end{longtable}


%%% Local Variables:
%%% mode: latex
%%% TeX-engine: xetex
%%% TeX-master: "../Main"
%%% End:

%% -*- coding: utf-8 -*-
%% Time-stamp: <Chen Wang: 2018-07-10 22:44:53>

\section{穆帝\tiny(344-361)}

\subsection{永和}

\begin{longtable}{|>{\centering\scriptsize}m{2em}|>{\centering\scriptsize}m{1.3em}|>{\centering}m{8.8em}|}
  % \caption{秦王政}\
  \toprule
  \SimHei \normalsize 年数 & \SimHei \scriptsize 公元 & \SimHei 大事件 \tabularnewline
  % \midrule
  \endfirsthead
  \toprule
  \SimHei \normalsize 年数 & \SimHei \scriptsize 公元 & \SimHei 大事件 \tabularnewline
  \midrule
  \endhead
  \midrule
  元年 & 345 & \tabularnewline\hline
  二年 & 346 & \tabularnewline\hline
  三年 & 347 & \tabularnewline\hline
  四年 & 348 & \tabularnewline\hline
  五年 & 349 & \tabularnewline\hline
  六年 & 350 & \tabularnewline\hline
  七年 & 351 & \tabularnewline\hline
  八年 & 352 & \tabularnewline\hline
  九年 & 353 & \tabularnewline\hline
  十年 & 354 & \tabularnewline\hline
  十一年 & 355 & \tabularnewline\hline
  十二年 & 356 & \tabularnewline
  \bottomrule
\end{longtable}

\subsection{升平}

\begin{longtable}{|>{\centering\scriptsize}m{2em}|>{\centering\scriptsize}m{1.3em}|>{\centering}m{8.8em}|}
  % \caption{秦王政}\
  \toprule
  \SimHei \normalsize 年数 & \SimHei \scriptsize 公元 & \SimHei 大事件 \tabularnewline
  % \midrule
  \endfirsthead
  \toprule
  \SimHei \normalsize 年数 & \SimHei \scriptsize 公元 & \SimHei 大事件 \tabularnewline
  \midrule
  \endhead
  \midrule
  元年 & 357 & \tabularnewline\hline
  二年 & 358 & \tabularnewline\hline
  三年 & 359 & \tabularnewline\hline
  四年 & 360 & \tabularnewline\hline
  五年 & 361 & \tabularnewline
  \bottomrule
\end{longtable}


%%% Local Variables:
%%% mode: latex
%%% TeX-engine: xetex
%%% TeX-master: "../Main"
%%% End:

%% -*- coding: utf-8 -*-
%% Time-stamp: <Chen Wang: 2018-07-10 22:46:17>

\section{哀帝\tiny(361-365)}

\subsection{隆和}

\begin{longtable}{|>{\centering\scriptsize}m{2em}|>{\centering\scriptsize}m{1.3em}|>{\centering}m{8.8em}|}
  % \caption{秦王政}\
  \toprule
  \SimHei \normalsize 年数 & \SimHei \scriptsize 公元 & \SimHei 大事件 \tabularnewline
  % \midrule
  \endfirsthead
  \toprule
  \SimHei \normalsize 年数 & \SimHei \scriptsize 公元 & \SimHei 大事件 \tabularnewline
  \midrule
  \endhead
  \midrule
  元年 & 362 & \tabularnewline\hline
  二年 & 363 & \tabularnewline
  \bottomrule
\end{longtable}

\subsection{兴宁}

\begin{longtable}{|>{\centering\scriptsize}m{2em}|>{\centering\scriptsize}m{1.3em}|>{\centering}m{8.8em}|}
  % \caption{秦王政}\
  \toprule
  \SimHei \normalsize 年数 & \SimHei \scriptsize 公元 & \SimHei 大事件 \tabularnewline
  % \midrule
  \endfirsthead
  \toprule
  \SimHei \normalsize 年数 & \SimHei \scriptsize 公元 & \SimHei 大事件 \tabularnewline
  \midrule
  \endhead
  \midrule
  元年 & 363 & \tabularnewline\hline
  二年 & 364 & \tabularnewline\hline
  三年 & 365 & \tabularnewline
  \bottomrule
\end{longtable}


%%% Local Variables:
%%% mode: latex
%%% TeX-engine: xetex
%%% TeX-master: "../Main"
%%% End:

%% -*- coding: utf-8 -*-
%% Time-stamp: <Chen Wang: 2018-07-10 22:47:18>

\section{司马奕\tiny(365-371)}

\subsection{太和}

\begin{longtable}{|>{\centering\scriptsize}m{2em}|>{\centering\scriptsize}m{1.3em}|>{\centering}m{8.8em}|}
  % \caption{秦王政}\
  \toprule
  \SimHei \normalsize 年数 & \SimHei \scriptsize 公元 & \SimHei 大事件 \tabularnewline
  % \midrule
  \endfirsthead
  \toprule
  \SimHei \normalsize 年数 & \SimHei \scriptsize 公元 & \SimHei 大事件 \tabularnewline
  \midrule
  \endhead
  \midrule
  元年 & 366 & \tabularnewline\hline
  二年 & 367 & \tabularnewline\hline
  三年 & 368 & \tabularnewline\hline
  四年 & 369 & \tabularnewline\hline
  五年 & 370 & \tabularnewline\hline
  六年 & 371 & \tabularnewline
  \bottomrule
\end{longtable}



%%% Local Variables:
%%% mode: latex
%%% TeX-engine: xetex
%%% TeX-master: "../Main"
%%% End:

%% -*- coding: utf-8 -*-
%% Time-stamp: <Chen Wang: 2018-07-10 22:48:05>

\section{简文帝\tiny(371-372)}

\subsection{咸安}

\begin{longtable}{|>{\centering\scriptsize}m{2em}|>{\centering\scriptsize}m{1.3em}|>{\centering}m{8.8em}|}
  % \caption{秦王政}\
  \toprule
  \SimHei \normalsize 年数 & \SimHei \scriptsize 公元 & \SimHei 大事件 \tabularnewline
  % \midrule
  \endfirsthead
  \toprule
  \SimHei \normalsize 年数 & \SimHei \scriptsize 公元 & \SimHei 大事件 \tabularnewline
  \midrule
  \endhead
  \midrule
  元年 & 371 & \tabularnewline\hline
  二年 & 372 & \tabularnewline
  \bottomrule
\end{longtable}



%%% Local Variables:
%%% mode: latex
%%% TeX-engine: xetex
%%% TeX-master: "../Main"
%%% End:

%% -*- coding: utf-8 -*-
%% Time-stamp: <Chen Wang: 2018-07-10 22:49:43>

\section{孝武帝\tiny(372-396)}

\subsection{宁康}

\begin{longtable}{|>{\centering\scriptsize}m{2em}|>{\centering\scriptsize}m{1.3em}|>{\centering}m{8.8em}|}
  % \caption{秦王政}\
  \toprule
  \SimHei \normalsize 年数 & \SimHei \scriptsize 公元 & \SimHei 大事件 \tabularnewline
  % \midrule
  \endfirsthead
  \toprule
  \SimHei \normalsize 年数 & \SimHei \scriptsize 公元 & \SimHei 大事件 \tabularnewline
  \midrule
  \endhead
  \midrule
  元年 & 373 & \tabularnewline\hline
  二年 & 374 & \tabularnewline\hline
  三年 & 375 & \tabularnewline
  \bottomrule
\end{longtable}

\subsection{太元}

\begin{longtable}{|>{\centering\scriptsize}m{2em}|>{\centering\scriptsize}m{1.3em}|>{\centering}m{8.8em}|}
  % \caption{秦王政}\
  \toprule
  \SimHei \normalsize 年数 & \SimHei \scriptsize 公元 & \SimHei 大事件 \tabularnewline
  % \midrule
  \endfirsthead
  \toprule
  \SimHei \normalsize 年数 & \SimHei \scriptsize 公元 & \SimHei 大事件 \tabularnewline
  \midrule
  \endhead
  \midrule
  元年 & 376 & \tabularnewline\hline
  二年 & 377 & \tabularnewline\hline
  三年 & 378 & \tabularnewline\hline
  四年 & 379 & \tabularnewline\hline
  五年 & 380 & \tabularnewline\hline
  六年 & 381 & \tabularnewline\hline
  七年 & 382 & \tabularnewline\hline
  八年 & 383 & \tabularnewline\hline
  九年 & 384 & \tabularnewline\hline
  十年 & 385 & \tabularnewline\hline
  十一年 & 386 & \tabularnewline\hline
  十二年 & 387 & \tabularnewline\hline
  十三年 & 388 & \tabularnewline\hline
  十四年 & 389 & \tabularnewline\hline
  十五年 & 390 & \tabularnewline\hline
  十六年 & 391 & \tabularnewline\hline
  十七年 & 392 & \tabularnewline\hline
  十八年 & 393 & \tabularnewline\hline
  十九年 & 394 & \tabularnewline\hline
  二十年 & 395 & \tabularnewline\hline
  二一年 & 396 & \tabularnewline
  \bottomrule
\end{longtable}


%%% Local Variables:
%%% mode: latex
%%% TeX-engine: xetex
%%% TeX-master: "../Main"
%%% End:

%% -*- coding: utf-8 -*-
%% Time-stamp: <Chen Wang: 2018-07-10 22:52:49>

\section{安帝\tiny(397-418)}

\subsection{隆安}

\begin{longtable}{|>{\centering\scriptsize}m{2em}|>{\centering\scriptsize}m{1.3em}|>{\centering}m{8.8em}|}
  % \caption{秦王政}\
  \toprule
  \SimHei \normalsize 年数 & \SimHei \scriptsize 公元 & \SimHei 大事件 \tabularnewline
  % \midrule
  \endfirsthead
  \toprule
  \SimHei \normalsize 年数 & \SimHei \scriptsize 公元 & \SimHei 大事件 \tabularnewline
  \midrule
  \endhead
  \midrule
  元年 & 397 & \tabularnewline\hline
  二年 & 398 & \tabularnewline\hline
  三年 & 399 & \tabularnewline\hline
  四年 & 400 & \tabularnewline\hline
  五年 & 401 & \tabularnewline
  \bottomrule
\end{longtable}

\subsection{元兴}

\begin{longtable}{|>{\centering\scriptsize}m{2em}|>{\centering\scriptsize}m{1.3em}|>{\centering}m{8.8em}|}
  % \caption{秦王政}\
  \toprule
  \SimHei \normalsize 年数 & \SimHei \scriptsize 公元 & \SimHei 大事件 \tabularnewline
  % \midrule
  \endfirsthead
  \toprule
  \SimHei \normalsize 年数 & \SimHei \scriptsize 公元 & \SimHei 大事件 \tabularnewline
  \midrule
  \endhead
  \midrule
  元年 & 402 & \tabularnewline\hline
  二年 & 403 & \tabularnewline\hline
  三年 & 404 & \tabularnewline
  \bottomrule
\end{longtable}

\subsection{大亨}

\begin{longtable}{|>{\centering\scriptsize}m{2em}|>{\centering\scriptsize}m{1.3em}|>{\centering}m{8.8em}|}
  % \caption{秦王政}\
  \toprule
  \SimHei \normalsize 年数 & \SimHei \scriptsize 公元 & \SimHei 大事件 \tabularnewline
  % \midrule
  \endfirsthead
  \toprule
  \SimHei \normalsize 年数 & \SimHei \scriptsize 公元 & \SimHei 大事件 \tabularnewline
  \midrule
  \endhead
  \midrule
  元年 & 402 & \tabularnewline
  \bottomrule
\end{longtable}

\subsection{义熙}

\begin{longtable}{|>{\centering\scriptsize}m{2em}|>{\centering\scriptsize}m{1.3em}|>{\centering}m{8.8em}|}
  % \caption{秦王政}\
  \toprule
  \SimHei \normalsize 年数 & \SimHei \scriptsize 公元 & \SimHei 大事件 \tabularnewline
  % \midrule
  \endfirsthead
  \toprule
  \SimHei \normalsize 年数 & \SimHei \scriptsize 公元 & \SimHei 大事件 \tabularnewline
  \midrule
  \endhead
  \midrule
  元年 & 405 & \tabularnewline\hline
  二年 & 406 & \tabularnewline\hline
  三年 & 407 & \tabularnewline\hline
  四年 & 408 & \tabularnewline\hline
  五年 & 409 & \tabularnewline\hline
  六年 & 410 & \tabularnewline\hline
  七年 & 411 & \tabularnewline\hline
  八年 & 412 & \tabularnewline\hline
  九年 & 413 & \tabularnewline\hline
  十年 & 414 & \tabularnewline\hline
  十一年 & 415 & \tabularnewline\hline
  十二年 & 416 & \tabularnewline\hline
  十三年 & 417 & \tabularnewline\hline
  十四年 & 418 & \tabularnewline
  \bottomrule
\end{longtable}


%%% Local Variables:
%%% mode: latex
%%% TeX-engine: xetex
%%% TeX-master: "../Main"
%%% End:

%% -*- coding: utf-8 -*-
%% Time-stamp: <Chen Wang: 2018-07-10 22:53:41>

\section{恭帝\tiny(419-420)}

\subsection{元熙}

\begin{longtable}{|>{\centering\scriptsize}m{2em}|>{\centering\scriptsize}m{1.3em}|>{\centering}m{8.8em}|}
  % \caption{秦王政}\
  \toprule
  \SimHei \normalsize 年数 & \SimHei \scriptsize 公元 & \SimHei 大事件 \tabularnewline
  % \midrule
  \endfirsthead
  \toprule
  \SimHei \normalsize 年数 & \SimHei \scriptsize 公元 & \SimHei 大事件 \tabularnewline
  \midrule
  \endhead
  \midrule
  元年 & 419 & \tabularnewline\hline
  二年 & 429 & \tabularnewline
  \bottomrule
\end{longtable}


%%% Local Variables:
%%% mode: latex
%%% TeX-engine: xetex
%%% TeX-master: "../Main"
%%% End:

%% -*- coding: utf-8 -*-
%% Time-stamp: <Chen Wang: 2018-07-10 22:59:53>

\section{桓楚\tiny(403-405)}

\subsection{桓玄\tiny(403-404)}

\subsubsection{永始}


\begin{longtable}{|>{\centering\scriptsize}m{2em}|>{\centering\scriptsize}m{1.3em}|>{\centering}m{8.8em}|}
  % \caption{秦王政}\
  \toprule
  \SimHei \normalsize 年数 & \SimHei \scriptsize 公元 & \SimHei 大事件 \tabularnewline
  % \midrule
  \endfirsthead
  \toprule
  \SimHei \normalsize 年数 & \SimHei \scriptsize 公元 & \SimHei 大事件 \tabularnewline
  \midrule
  \endhead
  \midrule
  元年 & 403 & \tabularnewline\hline
  二年 & 404 & \tabularnewline
  \bottomrule
\end{longtable}

\subsection{桓谦\tiny(404-405)}

\subsubsection{天康}


\begin{longtable}{|>{\centering\scriptsize}m{2em}|>{\centering\scriptsize}m{1.3em}|>{\centering}m{8.8em}|}
  % \caption{秦王政}\
  \toprule
  \SimHei \normalsize 年数 & \SimHei \scriptsize 公元 & \SimHei 大事件 \tabularnewline
  % \midrule
  \endfirsthead
  \toprule
  \SimHei \normalsize 年数 & \SimHei \scriptsize 公元 & \SimHei 大事件 \tabularnewline
  \midrule
  \endhead
  \midrule
  元年 & 404 & \tabularnewline\hline
  二年 & 405 & \tabularnewline
  \bottomrule
\end{longtable}


%%% Local Variables:
%%% mode: latex
%%% TeX-engine: xetex
%%% TeX-master: "../Main"
%%% End:


%%% Local Variables:
%%% mode: latex
%%% TeX-engine: xetex
%%% TeX-master: "../Main"
%%% End:

% %% -*- coding: utf-8 -*-
%% Time-stamp: <Chen Wang: 2018-07-10 23:56:33>

\chapter{十六国\tiny(304-439)}


%% -*- coding: utf-8 -*-
%% Time-stamp: <Chen Wang: 2018-07-10 23:04:15>


\section{汉赵\tiny(304-329)}

%% -*- coding: utf-8 -*-
%% Time-stamp: <Chen Wang: 2018-07-10 23:06:56>

\subsection{光文帝\tiny(304-310)}

\subsubsection{元熙}

\begin{longtable}{|>{\centering\scriptsize}m{2em}|>{\centering\scriptsize}m{1.3em}|>{\centering}m{8.8em}|}
  % \caption{秦王政}\
  \toprule
  \SimHei \normalsize 年数 & \SimHei \scriptsize 公元 & \SimHei 大事件 \tabularnewline
  % \midrule
  \endfirsthead
  \toprule
  \SimHei \normalsize 年数 & \SimHei \scriptsize 公元 & \SimHei 大事件 \tabularnewline
  \midrule
  \endhead
  \midrule
  元年 & 304 & \tabularnewline\hline
  二年 & 305 & \tabularnewline\hline
  三年 & 306 & \tabularnewline\hline
  四年 & 307 & \tabularnewline\hline
  五年 & 308 & \tabularnewline
  \bottomrule
\end{longtable}

\subsubsection{永凤}

\begin{longtable}{|>{\centering\scriptsize}m{2em}|>{\centering\scriptsize}m{1.3em}|>{\centering}m{8.8em}|}
  % \caption{秦王政}\
  \toprule
  \SimHei \normalsize 年数 & \SimHei \scriptsize 公元 & \SimHei 大事件 \tabularnewline
  % \midrule
  \endfirsthead
  \toprule
  \SimHei \normalsize 年数 & \SimHei \scriptsize 公元 & \SimHei 大事件 \tabularnewline
  \midrule
  \endhead
  \midrule
  元年 & 308 & \tabularnewline\hline
  二年 & 309 & \tabularnewline
  \bottomrule
\end{longtable}

\subsubsection{河瑞}

\begin{longtable}{|>{\centering\scriptsize}m{2em}|>{\centering\scriptsize}m{1.3em}|>{\centering}m{8.8em}|}
  % \caption{秦王政}\
  \toprule
  \SimHei \normalsize 年数 & \SimHei \scriptsize 公元 & \SimHei 大事件 \tabularnewline
  % \midrule
  \endfirsthead
  \toprule
  \SimHei \normalsize 年数 & \SimHei \scriptsize 公元 & \SimHei 大事件 \tabularnewline
  \midrule
  \endhead
  \midrule
  元年 & 309 & \tabularnewline\hline
  二年 & 310 & \tabularnewline
  \bottomrule
\end{longtable}


%%% Local Variables:
%%% mode: latex
%%% TeX-engine: xetex
%%% TeX-master: "../../Main"
%%% End:


%%% Local Variables:
%%% mode: latex
%%% TeX-engine: xetex
%%% TeX-master: "../../Main"
%%% End:

%% -*- coding: utf-8 -*-
%% Time-stamp: <Chen Wang: 2018-07-10 23:29:36>


\section{成汉\tiny(306-347)}

%% -*- coding: utf-8 -*-
%% Time-stamp: <Chen Wang: 2018-07-10 23:23:37>

\subsection{李特\tiny(303)}

\subsubsection{建初}

\begin{longtable}{|>{\centering\scriptsize}m{2em}|>{\centering\scriptsize}m{1.3em}|>{\centering}m{8.8em}|}
  % \caption{秦王政}\
  \toprule
  \SimHei \normalsize 年数 & \SimHei \scriptsize 公元 & \SimHei 大事件 \tabularnewline
  % \midrule
  \endfirsthead
  \toprule
  \SimHei \normalsize 年数 & \SimHei \scriptsize 公元 & \SimHei 大事件 \tabularnewline
  \midrule
  \endhead
  \midrule
  元年 & 303 & \tabularnewline\hline
  二年 & 304 & \tabularnewline
  \bottomrule
\end{longtable}


%%% Local Variables:
%%% mode: latex
%%% TeX-engine: xetex
%%% TeX-master: "../../Main"
%%% End:

%% -*- coding: utf-8 -*-
%% Time-stamp: <Chen Wang: 2018-07-10 23:25:38>

\subsection{武帝\tiny(304-334)}

\subsubsection{建兴}

\begin{longtable}{|>{\centering\scriptsize}m{2em}|>{\centering\scriptsize}m{1.3em}|>{\centering}m{8.8em}|}
  % \caption{秦王政}\
  \toprule
  \SimHei \normalsize 年数 & \SimHei \scriptsize 公元 & \SimHei 大事件 \tabularnewline
  % \midrule
  \endfirsthead
  \toprule
  \SimHei \normalsize 年数 & \SimHei \scriptsize 公元 & \SimHei 大事件 \tabularnewline
  \midrule
  \endhead
  \midrule
  元年 & 304 & \tabularnewline\hline
  二年 & 305 & \tabularnewline\hline
  三年 & 306 & \tabularnewline
  \bottomrule
\end{longtable}

\subsubsection{晏平}

\begin{longtable}{|>{\centering\scriptsize}m{2em}|>{\centering\scriptsize}m{1.3em}|>{\centering}m{8.8em}|}
  % \caption{秦王政}\
  \toprule
  \SimHei \normalsize 年数 & \SimHei \scriptsize 公元 & \SimHei 大事件 \tabularnewline
  % \midrule
  \endfirsthead
  \toprule
  \SimHei \normalsize 年数 & \SimHei \scriptsize 公元 & \SimHei 大事件 \tabularnewline
  \midrule
  \endhead
  \midrule
  元年 & 306 & \tabularnewline\hline
  二年 & 307 & \tabularnewline\hline
  三年 & 308 & \tabularnewline\hline
  四年 & 309 & \tabularnewline\hline
  五年 & 310 & \tabularnewline
  \bottomrule
\end{longtable}

\subsubsection{玉衡}

\begin{longtable}{|>{\centering\scriptsize}m{2em}|>{\centering\scriptsize}m{1.3em}|>{\centering}m{8.8em}|}
  % \caption{秦王政}\
  \toprule
  \SimHei \normalsize 年数 & \SimHei \scriptsize 公元 & \SimHei 大事件 \tabularnewline
  % \midrule
  \endfirsthead
  \toprule
  \SimHei \normalsize 年数 & \SimHei \scriptsize 公元 & \SimHei 大事件 \tabularnewline
  \midrule
  \endhead
  \midrule
  元年 & 311 & \tabularnewline\hline
  二年 & 312 & \tabularnewline\hline
  三年 & 313 & \tabularnewline\hline
  四年 & 314 & \tabularnewline\hline
  五年 & 315 & \tabularnewline\hline
  六年 & 316 & \tabularnewline\hline
  七年 & 317 & \tabularnewline\hline
  八年 & 318 & \tabularnewline\hline
  九年 & 319 & \tabularnewline\hline
  十年 & 320 & \tabularnewline\hline
  十一年 & 321 & \tabularnewline\hline
  十二年 & 322 & \tabularnewline\hline
  十三年 & 323 & \tabularnewline\hline
  十四年 & 324 & \tabularnewline\hline
  十五年 & 325 & \tabularnewline\hline
  十六年 & 326 & \tabularnewline\hline
  十七年 & 327 & \tabularnewline\hline
  十八年 & 328 & \tabularnewline\hline
  十九年 & 329 & \tabularnewline\hline
  二十年 & 330 & \tabularnewline\hline
  二一年 & 331 & \tabularnewline\hline
  二二年 & 332 & \tabularnewline\hline
  二三年 & 333 & \tabularnewline\hline
  二四年 & 334 & \tabularnewline
  \bottomrule
\end{longtable}


%%% Local Variables:
%%% mode: latex
%%% TeX-engine: xetex
%%% TeX-master: "../../Main"
%%% End:

%% -*- coding: utf-8 -*-
%% Time-stamp: <Chen Wang: 2018-07-10 23:26:54>

\subsection{李期\tiny(334-338)}

\subsubsection{玉恒}

\begin{longtable}{|>{\centering\scriptsize}m{2em}|>{\centering\scriptsize}m{1.3em}|>{\centering}m{8.8em}|}
  % \caption{秦王政}\
  \toprule
  \SimHei \normalsize 年数 & \SimHei \scriptsize 公元 & \SimHei 大事件 \tabularnewline
  % \midrule
  \endfirsthead
  \toprule
  \SimHei \normalsize 年数 & \SimHei \scriptsize 公元 & \SimHei 大事件 \tabularnewline
  \midrule
  \endhead
  \midrule
  元年 & 335 & \tabularnewline\hline
  二年 & 336 & \tabularnewline\hline
  三年 & 337 & \tabularnewline\hline
  四年 & 338 & \tabularnewline
  \bottomrule
\end{longtable}


%%% Local Variables:
%%% mode: latex
%%% TeX-engine: xetex
%%% TeX-master: "../../Main"
%%% End:

%% -*- coding: utf-8 -*-
%% Time-stamp: <Chen Wang: 2018-07-10 23:27:51>

\subsection{昭文帝\tiny(338-343)}

\subsubsection{汉兴}

\begin{longtable}{|>{\centering\scriptsize}m{2em}|>{\centering\scriptsize}m{1.3em}|>{\centering}m{8.8em}|}
  % \caption{秦王政}\
  \toprule
  \SimHei \normalsize 年数 & \SimHei \scriptsize 公元 & \SimHei 大事件 \tabularnewline
  % \midrule
  \endfirsthead
  \toprule
  \SimHei \normalsize 年数 & \SimHei \scriptsize 公元 & \SimHei 大事件 \tabularnewline
  \midrule
  \endhead
  \midrule
  元年 & 338 & \tabularnewline\hline
  二年 & 339 & \tabularnewline\hline
  三年 & 340 & \tabularnewline\hline
  四年 & 341 & \tabularnewline\hline
  五年 & 342 & \tabularnewline\hline
  六年 & 343 & \tabularnewline
  \bottomrule
\end{longtable}


%%% Local Variables:
%%% mode: latex
%%% TeX-engine: xetex
%%% TeX-master: "../../Main"
%%% End:

%% -*- coding: utf-8 -*-
%% Time-stamp: <Chen Wang: 2018-07-10 23:29:02>

\subsection{李势\tiny(343-347)}

\subsubsection{太和}

\begin{longtable}{|>{\centering\scriptsize}m{2em}|>{\centering\scriptsize}m{1.3em}|>{\centering}m{8.8em}|}
  % \caption{秦王政}\
  \toprule
  \SimHei \normalsize 年数 & \SimHei \scriptsize 公元 & \SimHei 大事件 \tabularnewline
  % \midrule
  \endfirsthead
  \toprule
  \SimHei \normalsize 年数 & \SimHei \scriptsize 公元 & \SimHei 大事件 \tabularnewline
  \midrule
  \endhead
  \midrule
  元年 & 344 & \tabularnewline\hline
  二年 & 345 & \tabularnewline\hline
  三年 & 346 & \tabularnewline
  \bottomrule
\end{longtable}

\subsubsection{嘉宁}

\begin{longtable}{|>{\centering\scriptsize}m{2em}|>{\centering\scriptsize}m{1.3em}|>{\centering}m{8.8em}|}
  % \caption{秦王政}\
  \toprule
  \SimHei \normalsize 年数 & \SimHei \scriptsize 公元 & \SimHei 大事件 \tabularnewline
  % \midrule
  \endfirsthead
  \toprule
  \SimHei \normalsize 年数 & \SimHei \scriptsize 公元 & \SimHei 大事件 \tabularnewline
  \midrule
  \endhead
  \midrule
  元年 & 346 & \tabularnewline\hline
  二年 & 347 & \tabularnewline
  \bottomrule
\end{longtable}


%%% Local Variables:
%%% mode: latex
%%% TeX-engine: xetex
%%% TeX-master: "../../Main"
%%% End:


%%% Local Variables:
%%% mode: latex
%%% TeX-engine: xetex
%%% TeX-master: "../../Main"
%%% End:

%% -*- coding: utf-8 -*-
%% Time-stamp: <Chen Wang: 2018-07-10 23:52:24>


\section{前凉\tiny(301-376)}

%% -*- coding: utf-8 -*-
%% Time-stamp: <Chen Wang: 2018-07-10 23:51:39>

\subsection{威王\tiny(353-355)}

\subsubsection{和平}

\begin{longtable}{|>{\centering\scriptsize}m{2em}|>{\centering\scriptsize}m{1.3em}|>{\centering}m{8.8em}|}
  % \caption{秦王政}\
  \toprule
  \SimHei \normalsize 年数 & \SimHei \scriptsize 公元 & \SimHei 大事件 \tabularnewline
  % \midrule
  \endfirsthead
  \toprule
  \SimHei \normalsize 年数 & \SimHei \scriptsize 公元 & \SimHei 大事件 \tabularnewline
  \midrule
  \endhead
  \midrule
  元年 & 354 & \tabularnewline\hline
  二年 & 355 & \tabularnewline
  \bottomrule
\end{longtable}


%%% Local Variables:
%%% mode: latex
%%% TeX-engine: xetex
%%% TeX-master: "../../Main"
%%% End:

% %% -*- coding: utf-8 -*-
%% Time-stamp: <Chen Wang: 2018-07-10 23:19:15>

\subsection{刘曜\tiny(318-328)}

\subsubsection{光初}

\begin{longtable}{|>{\centering\scriptsize}m{2em}|>{\centering\scriptsize}m{1.3em}|>{\centering}m{8.8em}|}
  % \caption{秦王政}\
  \toprule
  \SimHei \normalsize 年数 & \SimHei \scriptsize 公元 & \SimHei 大事件 \tabularnewline
  % \midrule
  \endfirsthead
  \toprule
  \SimHei \normalsize 年数 & \SimHei \scriptsize 公元 & \SimHei 大事件 \tabularnewline
  \midrule
  \endhead
  \midrule
  元年 & 318 & \tabularnewline\hline
  二年 & 319 & \tabularnewline\hline
  三年 & 320 & \tabularnewline\hline
  四年 & 321 & \tabularnewline\hline
  五年 & 322 & \tabularnewline\hline
  六年 & 323 & \tabularnewline\hline
  七年 & 324 & \tabularnewline\hline
  八年 & 325 & \tabularnewline\hline
  九年 & 326 & \tabularnewline\hline
  十年 & 327 & \tabularnewline\hline
  十一年 & 328 & \tabularnewline\hline
  十二年 & 329 & \tabularnewline
  \bottomrule
\end{longtable}

%%% Local Variables:
%%% mode: latex
%%% TeX-engine: xetex
%%% TeX-master: "../../Main"
%%% End:

% %% -*- coding: utf-8 -*-
%% Time-stamp: <Chen Wang: 2018-07-10 23:17:58>

\subsection{隐帝\tiny(318)}

\subsubsection{汉昌}

\begin{longtable}{|>{\centering\scriptsize}m{2em}|>{\centering\scriptsize}m{1.3em}|>{\centering}m{8.8em}|}
  % \caption{秦王政}\
  \toprule
  \SimHei \normalsize 年数 & \SimHei \scriptsize 公元 & \SimHei 大事件 \tabularnewline
  % \midrule
  \endfirsthead
  \toprule
  \SimHei \normalsize 年数 & \SimHei \scriptsize 公元 & \SimHei 大事件 \tabularnewline
  \midrule
  \endhead
  \midrule
  元年 & 318 & \tabularnewline
  \bottomrule
\end{longtable}


%%% Local Variables:
%%% mode: latex
%%% TeX-engine: xetex
%%% TeX-master: "../../Main"
%%% End:



%%% Local Variables:
%%% mode: latex
%%% TeX-engine: xetex
%%% TeX-master: "../../Main"
%%% End:

%% -*- coding: utf-8 -*-
%% Time-stamp: <Chen Wang: 2018-07-11 00:02:35>


\section{后赵\tiny(319-351)}

%% -*- coding: utf-8 -*-
%% Time-stamp: <Chen Wang: 2018-07-10 23:56:07>

\subsection{明帝\tiny(319-333)}

\subsubsection{太和}

\begin{longtable}{|>{\centering\scriptsize}m{2em}|>{\centering\scriptsize}m{1.3em}|>{\centering}m{8.8em}|}
  % \caption{秦王政}\
  \toprule
  \SimHei \normalsize 年数 & \SimHei \scriptsize 公元 & \SimHei 大事件 \tabularnewline
  % \midrule
  \endfirsthead
  \toprule
  \SimHei \normalsize 年数 & \SimHei \scriptsize 公元 & \SimHei 大事件 \tabularnewline
  \midrule
  \endhead
  \midrule
  元年 & 328 & \tabularnewline\hline
  二年 & 329 & \tabularnewline\hline
  三年 & 330 & \tabularnewline
  \bottomrule
\end{longtable}

\subsubsection{建平}

\begin{longtable}{|>{\centering\scriptsize}m{2em}|>{\centering\scriptsize}m{1.3em}|>{\centering}m{8.8em}|}
  % \caption{秦王政}\
  \toprule
  \SimHei \normalsize 年数 & \SimHei \scriptsize 公元 & \SimHei 大事件 \tabularnewline
  % \midrule
  \endfirsthead
  \toprule
  \SimHei \normalsize 年数 & \SimHei \scriptsize 公元 & \SimHei 大事件 \tabularnewline
  \midrule
  \endhead
  \midrule
  元年 & 330 & \tabularnewline\hline
  二年 & 331 & \tabularnewline\hline
  三年 & 332 & \tabularnewline\hline
  四年 & 333 & \tabularnewline
  \bottomrule
\end{longtable}


%%% Local Variables:
%%% mode: latex
%%% TeX-engine: xetex
%%% TeX-master: "../../Main"
%%% End:

%% -*- coding: utf-8 -*-
%% Time-stamp: <Chen Wang: 2018-07-10 23:57:54>

\subsection{石弘\tiny(333-334)}

\subsubsection{延熙}

\begin{longtable}{|>{\centering\scriptsize}m{2em}|>{\centering\scriptsize}m{1.3em}|>{\centering}m{8.8em}|}
  % \caption{秦王政}\
  \toprule
  \SimHei \normalsize 年数 & \SimHei \scriptsize 公元 & \SimHei 大事件 \tabularnewline
  % \midrule
  \endfirsthead
  \toprule
  \SimHei \normalsize 年数 & \SimHei \scriptsize 公元 & \SimHei 大事件 \tabularnewline
  \midrule
  \endhead
  \midrule
  元年 & 334 & \tabularnewline
  \bottomrule
\end{longtable}


%%% Local Variables:
%%% mode: latex
%%% TeX-engine: xetex
%%% TeX-master: "../../Main"
%%% End:

%% -*- coding: utf-8 -*-
%% Time-stamp: <Chen Wang: 2018-07-10 23:59:32>

\subsection{武帝\tiny(334-349)}

\subsubsection{建武}

\begin{longtable}{|>{\centering\scriptsize}m{2em}|>{\centering\scriptsize}m{1.3em}|>{\centering}m{8.8em}|}
  % \caption{秦王政}\
  \toprule
  \SimHei \normalsize 年数 & \SimHei \scriptsize 公元 & \SimHei 大事件 \tabularnewline
  % \midrule
  \endfirsthead
  \toprule
  \SimHei \normalsize 年数 & \SimHei \scriptsize 公元 & \SimHei 大事件 \tabularnewline
  \midrule
  \endhead
  \midrule
  元年 & 335 & \tabularnewline\hline
  二年 & 336 & \tabularnewline\hline
  三年 & 337 & \tabularnewline\hline
  四年 & 338 & \tabularnewline\hline
  五年 & 339 & \tabularnewline\hline
  六年 & 340 & \tabularnewline\hline
  七年 & 341 & \tabularnewline\hline
  八年 & 342 & \tabularnewline\hline
  九年 & 343 & \tabularnewline\hline
  十年 & 344 & \tabularnewline\hline
  十一年 & 345 & \tabularnewline\hline
  十二年 & 346 & \tabularnewline\hline
  十三年 & 347 & \tabularnewline\hline
  十四年 & 348 & \tabularnewline
  \bottomrule
\end{longtable}

\subsubsection{太宁}

\begin{longtable}{|>{\centering\scriptsize}m{2em}|>{\centering\scriptsize}m{1.3em}|>{\centering}m{8.8em}|}
  % \caption{秦王政}\
  \toprule
  \SimHei \normalsize 年数 & \SimHei \scriptsize 公元 & \SimHei 大事件 \tabularnewline
  % \midrule
  \endfirsthead
  \toprule
  \SimHei \normalsize 年数 & \SimHei \scriptsize 公元 & \SimHei 大事件 \tabularnewline
  \midrule
  \endhead
  \midrule
  元年 & 349 & \tabularnewline
  \bottomrule
\end{longtable}


%%% Local Variables:
%%% mode: latex
%%% TeX-engine: xetex
%%% TeX-master: "../../Main"
%%% End:

%% -*- coding: utf-8 -*-
%% Time-stamp: <Chen Wang: 2018-07-11 00:00:47>

\subsection{石鉴\tiny(349-350)}

\subsubsection{青龙}

\begin{longtable}{|>{\centering\scriptsize}m{2em}|>{\centering\scriptsize}m{1.3em}|>{\centering}m{8.8em}|}
  % \caption{秦王政}\
  \toprule
  \SimHei \normalsize 年数 & \SimHei \scriptsize 公元 & \SimHei 大事件 \tabularnewline
  % \midrule
  \endfirsthead
  \toprule
  \SimHei \normalsize 年数 & \SimHei \scriptsize 公元 & \SimHei 大事件 \tabularnewline
  \midrule
  \endhead
  \midrule
  元年 & 350 & \tabularnewline
  \bottomrule
\end{longtable}


%%% Local Variables:
%%% mode: latex
%%% TeX-engine: xetex
%%% TeX-master: "../../Main"
%%% End:

%% -*- coding: utf-8 -*-
%% Time-stamp: <Chen Wang: 2018-07-11 00:02:01>

\subsection{石祗\tiny(350-351)}

\subsubsection{永宁}

\begin{longtable}{|>{\centering\scriptsize}m{2em}|>{\centering\scriptsize}m{1.3em}|>{\centering}m{8.8em}|}
  % \caption{秦王政}\
  \toprule
  \SimHei \normalsize 年数 & \SimHei \scriptsize 公元 & \SimHei 大事件 \tabularnewline
  % \midrule
  \endfirsthead
  \toprule
  \SimHei \normalsize 年数 & \SimHei \scriptsize 公元 & \SimHei 大事件 \tabularnewline
  \midrule
  \endhead
  \midrule
  元年 & 350 & \tabularnewline\hline
  一年 & 351 & \tabularnewline
  \bottomrule
\end{longtable}


%%% Local Variables:
%%% mode: latex
%%% TeX-engine: xetex
%%% TeX-master: "../../Main"
%%% End:



%%% Local Variables:
%%% mode: latex
%%% TeX-engine: xetex
%%% TeX-master: "../../Main"
%%% End:


%%% Local Variables:
%%% mode: latex
%%% TeX-engine: xetex
%%% TeX-master: "../Main"
%%% End:

% %% -*- coding: utf-8 -*-
%% Time-stamp: <Chen Wang: 2018-07-11 16:16:25>

\chapter{南北朝\tiny(420-589)}


%% -*- coding: utf-8 -*-
%% Time-stamp: <Chen Wang: 2018-07-11 16:15:50>


\section{刘宋\tiny(420-479)}

%% -*- coding: utf-8 -*-
%% Time-stamp: <Chen Wang: 2018-07-11 16:17:45>

\subsection{武帝\tiny(420-422)}

\subsubsection{永初}

\begin{longtable}{|>{\centering\scriptsize}m{2em}|>{\centering\scriptsize}m{1.3em}|>{\centering}m{8.8em}|}
  % \caption{秦王政}\
  \toprule
  \SimHei \normalsize 年数 & \SimHei \scriptsize 公元 & \SimHei 大事件 \tabularnewline
  % \midrule
  \endfirsthead
  \toprule
  \SimHei \normalsize 年数 & \SimHei \scriptsize 公元 & \SimHei 大事件 \tabularnewline
  \midrule
  \endhead
  \midrule
  元年 & 420 & \tabularnewline\hline
  二年 & 421 & \tabularnewline\hline
  三年 & 422 & \tabularnewline
  \bottomrule
\end{longtable}


%%% Local Variables:
%%% mode: latex
%%% TeX-engine: xetex
%%% TeX-master: "../../Main"
%%% End:



%%% Local Variables:
%%% mode: latex
%%% TeX-engine: xetex
%%% TeX-master: "../../Main"
%%% End:


%%% Local Variables:
%%% mode: latex
%%% TeX-engine: xetex
%%% TeX-master: "../Main"
%%% End:

% %% -*- coding: utf-8 -*-
%% Time-stamp: <Chen Wang: 2018-07-11 20:18:23>

\chapter{隋\tiny(581-619)}

%% -*- coding: utf-8 -*-
%% Time-stamp: <Chen Wang: 2018-07-11 20:20:31>

\section{文帝\tiny(581-604)}

\subsection{开皇}

\begin{longtable}{|>{\centering\scriptsize}m{2em}|>{\centering\scriptsize}m{1.3em}|>{\centering}m{8.8em}|}
  % \caption{秦王政}\
  \toprule
  \SimHei \normalsize 年数 & \SimHei \scriptsize 公元 & \SimHei 大事件 \tabularnewline
  % \midrule
  \endfirsthead
  \toprule
  \SimHei \normalsize 年数 & \SimHei \scriptsize 公元 & \SimHei 大事件 \tabularnewline
  \midrule
  \endhead
  \midrule
  元年 & 581 & \tabularnewline\hline
  二年 & 582 & \tabularnewline\hline
  三年 & 583 & \tabularnewline\hline
  四年 & 584 & \tabularnewline\hline
  五年 & 585 & \tabularnewline\hline
  六年 & 586 & \tabularnewline\hline
  七年 & 587 & \tabularnewline\hline
  八年 & 588 & \tabularnewline\hline
  九年 & 589 & \tabularnewline\hline
  十年 & 560 & \tabularnewline\hline
  十一年 & 561 & \tabularnewline\hline
  十二年 & 562 & \tabularnewline\hline
  十三年 & 563 & \tabularnewline\hline
  十四年 & 564 & \tabularnewline\hline
  十五年 & 565 & \tabularnewline\hline
  十六年 & 566 & \tabularnewline\hline
  十七年 & 567 & \tabularnewline\hline
  十八年 & 568 & \tabularnewline\hline
  十九年 & 569 & \tabularnewline\hline
  二十年 & 600 & \tabularnewline
  \bottomrule
\end{longtable}

\subsection{仁寿}

\begin{longtable}{|>{\centering\scriptsize}m{2em}|>{\centering\scriptsize}m{1.3em}|>{\centering}m{8.8em}|}
  % \caption{秦王政}\
  \toprule
  \SimHei \normalsize 年数 & \SimHei \scriptsize 公元 & \SimHei 大事件 \tabularnewline
  % \midrule
  \endfirsthead
  \toprule
  \SimHei \normalsize 年数 & \SimHei \scriptsize 公元 & \SimHei 大事件 \tabularnewline
  \midrule
  \endhead
  \midrule
  元年 & 601 & \tabularnewline\hline
  二年 & 602 & \tabularnewline\hline
  三年 & 603 & \tabularnewline\hline
  四年 & 604 & \tabularnewline
  \bottomrule
\end{longtable}


%%% Local Variables:
%%% mode: latex
%%% TeX-engine: xetex
%%% TeX-master: "../Main"
%%% End:

%% -*- coding: utf-8 -*-
%% Time-stamp: <Chen Wang: 2018-07-11 20:21:30>

\section{炀帝\tiny(604-618)}

\subsection{大业}

\begin{longtable}{|>{\centering\scriptsize}m{2em}|>{\centering\scriptsize}m{1.3em}|>{\centering}m{8.8em}|}
  % \caption{秦王政}\
  \toprule
  \SimHei \normalsize 年数 & \SimHei \scriptsize 公元 & \SimHei 大事件 \tabularnewline
  % \midrule
  \endfirsthead
  \toprule
  \SimHei \normalsize 年数 & \SimHei \scriptsize 公元 & \SimHei 大事件 \tabularnewline
  \midrule
  \endhead
  \midrule
  元年 & 605 & \tabularnewline\hline
  二年 & 606 & \tabularnewline\hline
  三年 & 607 & \tabularnewline\hline
  四年 & 608 & \tabularnewline\hline
  五年 & 609 & \tabularnewline\hline
  六年 & 610 & \tabularnewline\hline
  七年 & 611 & \tabularnewline\hline
  八年 & 612 & \tabularnewline\hline
  九年 & 613 & \tabularnewline\hline
  十年 & 614 & \tabularnewline\hline
  十一年 & 615 & \tabularnewline\hline
  十二年 & 616 & \tabularnewline\hline
  十三年 & 617 & \tabularnewline\hline
  十四年 & 618 & \tabularnewline
  \bottomrule
\end{longtable}


%%% Local Variables:
%%% mode: latex
%%% TeX-engine: xetex
%%% TeX-master: "../Main"
%%% End:

%% -*- coding: utf-8 -*-
%% Time-stamp: <Chen Wang: 2018-07-11 20:22:17>

\section{恭帝\tiny(617-618)}

\subsection{义宁}

\begin{longtable}{|>{\centering\scriptsize}m{2em}|>{\centering\scriptsize}m{1.3em}|>{\centering}m{8.8em}|}
  % \caption{秦王政}\
  \toprule
  \SimHei \normalsize 年数 & \SimHei \scriptsize 公元 & \SimHei 大事件 \tabularnewline
  % \midrule
  \endfirsthead
  \toprule
  \SimHei \normalsize 年数 & \SimHei \scriptsize 公元 & \SimHei 大事件 \tabularnewline
  \midrule
  \endhead
  \midrule
  元年 & 617 & \tabularnewline\hline
  二年 & 618 & \tabularnewline
  \bottomrule
\end{longtable}


%%% Local Variables:
%%% mode: latex
%%% TeX-engine: xetex
%%% TeX-master: "../Main"
%%% End:

%% -*- coding: utf-8 -*-
%% Time-stamp: <Chen Wang: 2018-07-11 20:22:58>

\section{杨侗\tiny(618-619)}

\subsection{皇泰}

\begin{longtable}{|>{\centering\scriptsize}m{2em}|>{\centering\scriptsize}m{1.3em}|>{\centering}m{8.8em}|}
  % \caption{秦王政}\
  \toprule
  \SimHei \normalsize 年数 & \SimHei \scriptsize 公元 & \SimHei 大事件 \tabularnewline
  % \midrule
  \endfirsthead
  \toprule
  \SimHei \normalsize 年数 & \SimHei \scriptsize 公元 & \SimHei 大事件 \tabularnewline
  \midrule
  \endhead
  \midrule
  元年 & 618 & \tabularnewline\hline
  二年 & 619 & \tabularnewline
  \bottomrule
\end{longtable}


%%% Local Variables:
%%% mode: latex
%%% TeX-engine: xetex
%%% TeX-master: "../Main"
%%% End:



%%% Local Variables:
%%% mode: latex
%%% TeX-engine: xetex
%%% TeX-master: "../Main"
%%% End:

% %% -*- coding: utf-8 -*-
%% Time-stamp: <Chen Wang: 2018-07-11 22:19:31>

\chapter{唐\tiny(618-907)}

%% -*- coding: utf-8 -*-
%% Time-stamp: <Chen Wang: 2018-07-11 20:36:04>

\section{高祖\tiny(618-626)}

\subsection{武德}

\begin{longtable}{|>{\centering\scriptsize}m{2em}|>{\centering\scriptsize}m{1.3em}|>{\centering}m{8.8em}|}
  % \caption{秦王政}\
  \toprule
  \SimHei \normalsize 年数 & \SimHei \scriptsize 公元 & \SimHei 大事件 \tabularnewline
  % \midrule
  \endfirsthead
  \toprule
  \SimHei \normalsize 年数 & \SimHei \scriptsize 公元 & \SimHei 大事件 \tabularnewline
  \midrule
  \endhead
  \midrule
  元年 & 618 & \tabularnewline\hline
  二年 & 619 & \tabularnewline\hline
  三年 & 620 & \tabularnewline\hline
  四年 & 621 & \tabularnewline\hline
  五年 & 622 & \tabularnewline\hline
  六年 & 623 & \tabularnewline\hline
  七年 & 624 & \tabularnewline\hline
  八年 & 625 & \tabularnewline\hline
  九年 & 626 & \tabularnewline
  \bottomrule
\end{longtable}


%%% Local Variables:
%%% mode: latex
%%% TeX-engine: xetex
%%% TeX-master: "../Main"
%%% End:

%% -*- coding: utf-8 -*-
%% Time-stamp: <Chen Wang: 2018-07-11 21:24:45>

\section{太宗\tiny(626-649)}

\subsection{贞观}

\begin{longtable}{|>{\centering\scriptsize}m{2em}|>{\centering\scriptsize}m{1.3em}|>{\centering}m{8.8em}|}
  % \caption{秦王政}\
  \toprule
  \SimHei \normalsize 年数 & \SimHei \scriptsize 公元 & \SimHei 大事件 \tabularnewline
  % \midrule
  \endfirsthead
  \toprule
  \SimHei \normalsize 年数 & \SimHei \scriptsize 公元 & \SimHei 大事件 \tabularnewline
  \midrule
  \endhead
  \midrule
  元年 & 627 & \tabularnewline\hline
  二年 & 628 & \tabularnewline\hline
  三年 & 629 & \tabularnewline\hline
  四年 & 630 & \tabularnewline\hline
  五年 & 631 & \tabularnewline\hline
  六年 & 632 & \tabularnewline\hline
  七年 & 633 & \tabularnewline\hline
  八年 & 634 & \tabularnewline\hline
  九年 & 635 & \tabularnewline\hline
  十年 & 636 & \tabularnewline\hline
  十一年 & 637 & \tabularnewline\hline
  十二年 & 638 & \tabularnewline\hline
  十三年 & 639 & \tabularnewline\hline
  十四年 & 640 & \tabularnewline\hline
  十五年 & 641 & \tabularnewline\hline
  十六年 & 642 & \tabularnewline\hline
  十七年 & 643 & \tabularnewline\hline
  十八年 & 644 & \tabularnewline\hline
  十九年 & 645 & \tabularnewline\hline
  二十年 & 646 & \tabularnewline\hline
  二一年 & 647 & \tabularnewline\hline
  二二年 & 648 & \tabularnewline\hline
  二三年 & 649 & \tabularnewline
  \bottomrule
\end{longtable}


%%% Local Variables:
%%% mode: latex
%%% TeX-engine: xetex
%%% TeX-master: "../Main"
%%% End:

%% -*- coding: utf-8 -*-
%% Time-stamp: <Chen Wang: 2018-07-11 21:30:52>

\section{高宗\tiny(649-683)}

\subsection{永徽}

\begin{longtable}{|>{\centering\scriptsize}m{2em}|>{\centering\scriptsize}m{1.3em}|>{\centering}m{8.8em}|}
  % \caption{秦王政}\
  \toprule
  \SimHei \normalsize 年数 & \SimHei \scriptsize 公元 & \SimHei 大事件 \tabularnewline
  % \midrule
  \endfirsthead
  \toprule
  \SimHei \normalsize 年数 & \SimHei \scriptsize 公元 & \SimHei 大事件 \tabularnewline
  \midrule
  \endhead
  \midrule
  元年 & 650 & \tabularnewline\hline
  二年 & 651 & \tabularnewline\hline
  三年 & 652 & \tabularnewline\hline
  四年 & 653 & \tabularnewline\hline
  五年 & 654 & \tabularnewline\hline
  六年 & 655 & \tabularnewline
  \bottomrule
\end{longtable}

\subsection{显庆}

\begin{longtable}{|>{\centering\scriptsize}m{2em}|>{\centering\scriptsize}m{1.3em}|>{\centering}m{8.8em}|}
  % \caption{秦王政}\
  \toprule
  \SimHei \normalsize 年数 & \SimHei \scriptsize 公元 & \SimHei 大事件 \tabularnewline
  % \midrule
  \endfirsthead
  \toprule
  \SimHei \normalsize 年数 & \SimHei \scriptsize 公元 & \SimHei 大事件 \tabularnewline
  \midrule
  \endhead
  \midrule
  元年 & 656 & \tabularnewline\hline
  二年 & 657 & \tabularnewline\hline
  三年 & 658 & \tabularnewline\hline
  四年 & 659 & \tabularnewline\hline
  五年 & 660 & \tabularnewline\hline
  六年 & 661 & \tabularnewline
  \bottomrule
\end{longtable}

\subsection{龙朔}

\begin{longtable}{|>{\centering\scriptsize}m{2em}|>{\centering\scriptsize}m{1.3em}|>{\centering}m{8.8em}|}
  % \caption{秦王政}\
  \toprule
  \SimHei \normalsize 年数 & \SimHei \scriptsize 公元 & \SimHei 大事件 \tabularnewline
  % \midrule
  \endfirsthead
  \toprule
  \SimHei \normalsize 年数 & \SimHei \scriptsize 公元 & \SimHei 大事件 \tabularnewline
  \midrule
  \endhead
  \midrule
  元年 & 661 & \tabularnewline\hline
  二年 & 662 & \tabularnewline\hline
  三年 & 663 & \tabularnewline
  \bottomrule
\end{longtable}

\subsection{麟德}

\begin{longtable}{|>{\centering\scriptsize}m{2em}|>{\centering\scriptsize}m{1.3em}|>{\centering}m{8.8em}|}
  % \caption{秦王政}\
  \toprule
  \SimHei \normalsize 年数 & \SimHei \scriptsize 公元 & \SimHei 大事件 \tabularnewline
  % \midrule
  \endfirsthead
  \toprule
  \SimHei \normalsize 年数 & \SimHei \scriptsize 公元 & \SimHei 大事件 \tabularnewline
  \midrule
  \endhead
  \midrule
  元年 & 664 & \tabularnewline\hline
  二年 & 665 & \tabularnewline
  \bottomrule
\end{longtable}

\subsection{乾封}

\begin{longtable}{|>{\centering\scriptsize}m{2em}|>{\centering\scriptsize}m{1.3em}|>{\centering}m{8.8em}|}
  % \caption{秦王政}\
  \toprule
  \SimHei \normalsize 年数 & \SimHei \scriptsize 公元 & \SimHei 大事件 \tabularnewline
  % \midrule
  \endfirsthead
  \toprule
  \SimHei \normalsize 年数 & \SimHei \scriptsize 公元 & \SimHei 大事件 \tabularnewline
  \midrule
  \endhead
  \midrule
  元年 & 666 & \tabularnewline\hline
  二年 & 667 & \tabularnewline\hline
  三年 & 668 & \tabularnewline
  \bottomrule
\end{longtable}

\subsection{总章}

\begin{longtable}{|>{\centering\scriptsize}m{2em}|>{\centering\scriptsize}m{1.3em}|>{\centering}m{8.8em}|}
  % \caption{秦王政}\
  \toprule
  \SimHei \normalsize 年数 & \SimHei \scriptsize 公元 & \SimHei 大事件 \tabularnewline
  % \midrule
  \endfirsthead
  \toprule
  \SimHei \normalsize 年数 & \SimHei \scriptsize 公元 & \SimHei 大事件 \tabularnewline
  \midrule
  \endhead
  \midrule
  元年 & 668 & \tabularnewline\hline
  二年 & 669 & \tabularnewline\hline
  三年 & 670 & \tabularnewline
  \bottomrule
\end{longtable}

\subsection{咸亨}

\begin{longtable}{|>{\centering\scriptsize}m{2em}|>{\centering\scriptsize}m{1.3em}|>{\centering}m{8.8em}|}
  % \caption{秦王政}\
  \toprule
  \SimHei \normalsize 年数 & \SimHei \scriptsize 公元 & \SimHei 大事件 \tabularnewline
  % \midrule
  \endfirsthead
  \toprule
  \SimHei \normalsize 年数 & \SimHei \scriptsize 公元 & \SimHei 大事件 \tabularnewline
  \midrule
  \endhead
  \midrule
  元年 & 670 & \tabularnewline\hline
  二年 & 671 & \tabularnewline\hline
  三年 & 672 & \tabularnewline\hline
  四年 & 673 & \tabularnewline\hline
  五年 & 674 & \tabularnewline
  \bottomrule
\end{longtable}

\subsection{上元}

\begin{longtable}{|>{\centering\scriptsize}m{2em}|>{\centering\scriptsize}m{1.3em}|>{\centering}m{8.8em}|}
  % \caption{秦王政}\
  \toprule
  \SimHei \normalsize 年数 & \SimHei \scriptsize 公元 & \SimHei 大事件 \tabularnewline
  % \midrule
  \endfirsthead
  \toprule
  \SimHei \normalsize 年数 & \SimHei \scriptsize 公元 & \SimHei 大事件 \tabularnewline
  \midrule
  \endhead
  \midrule
  元年 & 674 & \tabularnewline\hline
  二年 & 675 & \tabularnewline\hline
  三年 & 676 & \tabularnewline
  \bottomrule
\end{longtable}

\subsection{仪凤}

\begin{longtable}{|>{\centering\scriptsize}m{2em}|>{\centering\scriptsize}m{1.3em}|>{\centering}m{8.8em}|}
  % \caption{秦王政}\
  \toprule
  \SimHei \normalsize 年数 & \SimHei \scriptsize 公元 & \SimHei 大事件 \tabularnewline
  % \midrule
  \endfirsthead
  \toprule
  \SimHei \normalsize 年数 & \SimHei \scriptsize 公元 & \SimHei 大事件 \tabularnewline
  \midrule
  \endhead
  \midrule
  元年 & 676 & \tabularnewline\hline
  二年 & 677 & \tabularnewline\hline
  三年 & 678 & \tabularnewline\hline
  四年 & 679 & \tabularnewline
  \bottomrule
\end{longtable}

\subsection{调露}

\begin{longtable}{|>{\centering\scriptsize}m{2em}|>{\centering\scriptsize}m{1.3em}|>{\centering}m{8.8em}|}
  % \caption{秦王政}\
  \toprule
  \SimHei \normalsize 年数 & \SimHei \scriptsize 公元 & \SimHei 大事件 \tabularnewline
  % \midrule
  \endfirsthead
  \toprule
  \SimHei \normalsize 年数 & \SimHei \scriptsize 公元 & \SimHei 大事件 \tabularnewline
  \midrule
  \endhead
  \midrule
  元年 & 679 & \tabularnewline\hline
  二年 & 680 & \tabularnewline
  \bottomrule
\end{longtable}

\subsection{永隆}

\begin{longtable}{|>{\centering\scriptsize}m{2em}|>{\centering\scriptsize}m{1.3em}|>{\centering}m{8.8em}|}
  % \caption{秦王政}\
  \toprule
  \SimHei \normalsize 年数 & \SimHei \scriptsize 公元 & \SimHei 大事件 \tabularnewline
  % \midrule
  \endfirsthead
  \toprule
  \SimHei \normalsize 年数 & \SimHei \scriptsize 公元 & \SimHei 大事件 \tabularnewline
  \midrule
  \endhead
  \midrule
  元年 & 680 & \tabularnewline\hline
  二年 & 681 & \tabularnewline
  \bottomrule
\end{longtable}

\subsection{开耀}

\begin{longtable}{|>{\centering\scriptsize}m{2em}|>{\centering\scriptsize}m{1.3em}|>{\centering}m{8.8em}|}
  % \caption{秦王政}\
  \toprule
  \SimHei \normalsize 年数 & \SimHei \scriptsize 公元 & \SimHei 大事件 \tabularnewline
  % \midrule
  \endfirsthead
  \toprule
  \SimHei \normalsize 年数 & \SimHei \scriptsize 公元 & \SimHei 大事件 \tabularnewline
  \midrule
  \endhead
  \midrule
  元年 & 681 & \tabularnewline\hline
  二年 & 682 & \tabularnewline
  \bottomrule
\end{longtable}

\subsection{永淳}

\begin{longtable}{|>{\centering\scriptsize}m{2em}|>{\centering\scriptsize}m{1.3em}|>{\centering}m{8.8em}|}
  % \caption{秦王政}\
  \toprule
  \SimHei \normalsize 年数 & \SimHei \scriptsize 公元 & \SimHei 大事件 \tabularnewline
  % \midrule
  \endfirsthead
  \toprule
  \SimHei \normalsize 年数 & \SimHei \scriptsize 公元 & \SimHei 大事件 \tabularnewline
  \midrule
  \endhead
  \midrule
  元年 & 682 & \tabularnewline\hline
  二年 & 683 & \tabularnewline
  \bottomrule
\end{longtable}

\subsection{弘道}

\begin{longtable}{|>{\centering\scriptsize}m{2em}|>{\centering\scriptsize}m{1.3em}|>{\centering}m{8.8em}|}
  % \caption{秦王政}\
  \toprule
  \SimHei \normalsize 年数 & \SimHei \scriptsize 公元 & \SimHei 大事件 \tabularnewline
  % \midrule
  \endfirsthead
  \toprule
  \SimHei \normalsize 年数 & \SimHei \scriptsize 公元 & \SimHei 大事件 \tabularnewline
  \midrule
  \endhead
  \midrule
  元年 & 683 & \tabularnewline
  \bottomrule
\end{longtable}


%%% Local Variables:
%%% mode: latex
%%% TeX-engine: xetex
%%% TeX-master: "../Main"
%%% End:

%% -*- coding: utf-8 -*-
%% Time-stamp: <Chen Wang: 2018-07-11 21:31:51>

\section{中宗\tiny(683-684)}

\subsection{嗣圣}

\begin{longtable}{|>{\centering\scriptsize}m{2em}|>{\centering\scriptsize}m{1.3em}|>{\centering}m{8.8em}|}
  % \caption{秦王政}\
  \toprule
  \SimHei \normalsize 年数 & \SimHei \scriptsize 公元 & \SimHei 大事件 \tabularnewline
  % \midrule
  \endfirsthead
  \toprule
  \SimHei \normalsize 年数 & \SimHei \scriptsize 公元 & \SimHei 大事件 \tabularnewline
  \midrule
  \endhead
  \midrule
  元年 & 684 & \tabularnewline
  \bottomrule
\end{longtable}


%%% Local Variables:
%%% mode: latex
%%% TeX-engine: xetex
%%% TeX-master: "../Main"
%%% End:

%% -*- coding: utf-8 -*-
%% Time-stamp: <Chen Wang: 2018-07-11 21:33:46>

\section{睿宗\tiny(684-690)}

\subsection{文明}

\begin{longtable}{|>{\centering\scriptsize}m{2em}|>{\centering\scriptsize}m{1.3em}|>{\centering}m{8.8em}|}
  % \caption{秦王政}\
  \toprule
  \SimHei \normalsize 年数 & \SimHei \scriptsize 公元 & \SimHei 大事件 \tabularnewline
  % \midrule
  \endfirsthead
  \toprule
  \SimHei \normalsize 年数 & \SimHei \scriptsize 公元 & \SimHei 大事件 \tabularnewline
  \midrule
  \endhead
  \midrule
  元年 & 684 & \tabularnewline
  \bottomrule
\end{longtable}

\subsection{光宅}

\begin{longtable}{|>{\centering\scriptsize}m{2em}|>{\centering\scriptsize}m{1.3em}|>{\centering}m{8.8em}|}
  % \caption{秦王政}\
  \toprule
  \SimHei \normalsize 年数 & \SimHei \scriptsize 公元 & \SimHei 大事件 \tabularnewline
  % \midrule
  \endfirsthead
  \toprule
  \SimHei \normalsize 年数 & \SimHei \scriptsize 公元 & \SimHei 大事件 \tabularnewline
  \midrule
  \endhead
  \midrule
  元年 & 684 & \tabularnewline
  \bottomrule
\end{longtable}

\subsection{垂拱}

\begin{longtable}{|>{\centering\scriptsize}m{2em}|>{\centering\scriptsize}m{1.3em}|>{\centering}m{8.8em}|}
  % \caption{秦王政}\
  \toprule
  \SimHei \normalsize 年数 & \SimHei \scriptsize 公元 & \SimHei 大事件 \tabularnewline
  % \midrule
  \endfirsthead
  \toprule
  \SimHei \normalsize 年数 & \SimHei \scriptsize 公元 & \SimHei 大事件 \tabularnewline
  \midrule
  \endhead
  \midrule
  元年 & 685 & \tabularnewline\hline
  二年 & 686 & \tabularnewline\hline
  三年 & 687 & \tabularnewline\hline
  四年 & 688 & \tabularnewline
  \bottomrule
\end{longtable}

\subsection{永昌}

\begin{longtable}{|>{\centering\scriptsize}m{2em}|>{\centering\scriptsize}m{1.3em}|>{\centering}m{8.8em}|}
  % \caption{秦王政}\
  \toprule
  \SimHei \normalsize 年数 & \SimHei \scriptsize 公元 & \SimHei 大事件 \tabularnewline
  % \midrule
  \endfirsthead
  \toprule
  \SimHei \normalsize 年数 & \SimHei \scriptsize 公元 & \SimHei 大事件 \tabularnewline
  \midrule
  \endhead
  \midrule
  元年 & 689 & \tabularnewline
  \bottomrule
\end{longtable}

\subsection{载初}

\begin{longtable}{|>{\centering\scriptsize}m{2em}|>{\centering\scriptsize}m{1.3em}|>{\centering}m{8.8em}|}
  % \caption{秦王政}\
  \toprule
  \SimHei \normalsize 年数 & \SimHei \scriptsize 公元 & \SimHei 大事件 \tabularnewline
  % \midrule
  \endfirsthead
  \toprule
  \SimHei \normalsize 年数 & \SimHei \scriptsize 公元 & \SimHei 大事件 \tabularnewline
  \midrule
  \endhead
  \midrule
  元年 & 689 & \tabularnewline\hline
  二年 & 690 & \tabularnewline
  \bottomrule
\end{longtable}



%%% Local Variables:
%%% mode: latex
%%% TeX-engine: xetex
%%% TeX-master: "../Main"
%%% End:

%% -*- coding: utf-8 -*-
%% Time-stamp: <Chen Wang: 2018-07-11 21:39:38>

\section{武曌\tiny(683-705)}

\subsection{天授}

\begin{longtable}{|>{\centering\scriptsize}m{2em}|>{\centering\scriptsize}m{1.3em}|>{\centering}m{8.8em}|}
  % \caption{秦王政}\
  \toprule
  \SimHei \normalsize 年数 & \SimHei \scriptsize 公元 & \SimHei 大事件 \tabularnewline
  % \midrule
  \endfirsthead
  \toprule
  \SimHei \normalsize 年数 & \SimHei \scriptsize 公元 & \SimHei 大事件 \tabularnewline
  \midrule
  \endhead
  \midrule
  元年 & 690 & \tabularnewline\hline
  二年 & 691 & \tabularnewline\hline
  三年 & 692 & \tabularnewline
  \bottomrule
\end{longtable}

\subsection{如意}

\begin{longtable}{|>{\centering\scriptsize}m{2em}|>{\centering\scriptsize}m{1.3em}|>{\centering}m{8.8em}|}
  % \caption{秦王政}\
  \toprule
  \SimHei \normalsize 年数 & \SimHei \scriptsize 公元 & \SimHei 大事件 \tabularnewline
  % \midrule
  \endfirsthead
  \toprule
  \SimHei \normalsize 年数 & \SimHei \scriptsize 公元 & \SimHei 大事件 \tabularnewline
  \midrule
  \endhead
  \midrule
  元年 & 692 & \tabularnewline
  \bottomrule
\end{longtable}

\subsection{长寿}

\begin{longtable}{|>{\centering\scriptsize}m{2em}|>{\centering\scriptsize}m{1.3em}|>{\centering}m{8.8em}|}
  % \caption{秦王政}\
  \toprule
  \SimHei \normalsize 年数 & \SimHei \scriptsize 公元 & \SimHei 大事件 \tabularnewline
  % \midrule
  \endfirsthead
  \toprule
  \SimHei \normalsize 年数 & \SimHei \scriptsize 公元 & \SimHei 大事件 \tabularnewline
  \midrule
  \endhead
  \midrule
  元年 & 692 & \tabularnewline\hline
  二年 & 693 & \tabularnewline\hline
  三年 & 694 & \tabularnewline
  \bottomrule
\end{longtable}

\subsection{延载}

\begin{longtable}{|>{\centering\scriptsize}m{2em}|>{\centering\scriptsize}m{1.3em}|>{\centering}m{8.8em}|}
  % \caption{秦王政}\
  \toprule
  \SimHei \normalsize 年数 & \SimHei \scriptsize 公元 & \SimHei 大事件 \tabularnewline
  % \midrule
  \endfirsthead
  \toprule
  \SimHei \normalsize 年数 & \SimHei \scriptsize 公元 & \SimHei 大事件 \tabularnewline
  \midrule
  \endhead
  \midrule
  元年 & 694 & \tabularnewline
  \bottomrule
\end{longtable}

\subsection{证圣}

\begin{longtable}{|>{\centering\scriptsize}m{2em}|>{\centering\scriptsize}m{1.3em}|>{\centering}m{8.8em}|}
  % \caption{秦王政}\
  \toprule
  \SimHei \normalsize 年数 & \SimHei \scriptsize 公元 & \SimHei 大事件 \tabularnewline
  % \midrule
  \endfirsthead
  \toprule
  \SimHei \normalsize 年数 & \SimHei \scriptsize 公元 & \SimHei 大事件 \tabularnewline
  \midrule
  \endhead
  \midrule
  元年 & 695 & \tabularnewline
  \bottomrule
\end{longtable}

\subsection{天册万岁}

\begin{longtable}{|>{\centering\scriptsize}m{2em}|>{\centering\scriptsize}m{1.3em}|>{\centering}m{8.8em}|}
  % \caption{秦王政}\
  \toprule
  \SimHei \normalsize 年数 & \SimHei \scriptsize 公元 & \SimHei 大事件 \tabularnewline
  % \midrule
  \endfirsthead
  \toprule
  \SimHei \normalsize 年数 & \SimHei \scriptsize 公元 & \SimHei 大事件 \tabularnewline
  \midrule
  \endhead
  \midrule
  元年 & 695 & \tabularnewline
  \bottomrule
\end{longtable}

\subsection{万岁登封}

\begin{longtable}{|>{\centering\scriptsize}m{2em}|>{\centering\scriptsize}m{1.3em}|>{\centering}m{8.8em}|}
  % \caption{秦王政}\
  \toprule
  \SimHei \normalsize 年数 & \SimHei \scriptsize 公元 & \SimHei 大事件 \tabularnewline
  % \midrule
  \endfirsthead
  \toprule
  \SimHei \normalsize 年数 & \SimHei \scriptsize 公元 & \SimHei 大事件 \tabularnewline
  \midrule
  \endhead
  \midrule
  元年 & 695 & \tabularnewline\hline
  二年 & 696 & \tabularnewline
  \bottomrule
\end{longtable}

\subsection{万岁通天}

\begin{longtable}{|>{\centering\scriptsize}m{2em}|>{\centering\scriptsize}m{1.3em}|>{\centering}m{8.8em}|}
  % \caption{秦王政}\
  \toprule
  \SimHei \normalsize 年数 & \SimHei \scriptsize 公元 & \SimHei 大事件 \tabularnewline
  % \midrule
  \endfirsthead
  \toprule
  \SimHei \normalsize 年数 & \SimHei \scriptsize 公元 & \SimHei 大事件 \tabularnewline
  \midrule
  \endhead
  \midrule
  元年 & 696 & \tabularnewline\hline
  二年 & 697 & \tabularnewline
  \bottomrule
\end{longtable}

\subsection{神功}

\begin{longtable}{|>{\centering\scriptsize}m{2em}|>{\centering\scriptsize}m{1.3em}|>{\centering}m{8.8em}|}
  % \caption{秦王政}\
  \toprule
  \SimHei \normalsize 年数 & \SimHei \scriptsize 公元 & \SimHei 大事件 \tabularnewline
  % \midrule
  \endfirsthead
  \toprule
  \SimHei \normalsize 年数 & \SimHei \scriptsize 公元 & \SimHei 大事件 \tabularnewline
  \midrule
  \endhead
  \midrule
  元年 & 697 & \tabularnewline
  \bottomrule
\end{longtable}

\subsection{圣历}

\begin{longtable}{|>{\centering\scriptsize}m{2em}|>{\centering\scriptsize}m{1.3em}|>{\centering}m{8.8em}|}
  % \caption{秦王政}\
  \toprule
  \SimHei \normalsize 年数 & \SimHei \scriptsize 公元 & \SimHei 大事件 \tabularnewline
  % \midrule
  \endfirsthead
  \toprule
  \SimHei \normalsize 年数 & \SimHei \scriptsize 公元 & \SimHei 大事件 \tabularnewline
  \midrule
  \endhead
  \midrule
  元年 & 698 & \tabularnewline\hline
  二年 & 699 & \tabularnewline\hline
  三年 & 700 & \tabularnewline
  \bottomrule
\end{longtable}

\subsection{久视}

\begin{longtable}{|>{\centering\scriptsize}m{2em}|>{\centering\scriptsize}m{1.3em}|>{\centering}m{8.8em}|}
  % \caption{秦王政}\
  \toprule
  \SimHei \normalsize 年数 & \SimHei \scriptsize 公元 & \SimHei 大事件 \tabularnewline
  % \midrule
  \endfirsthead
  \toprule
  \SimHei \normalsize 年数 & \SimHei \scriptsize 公元 & \SimHei 大事件 \tabularnewline
  \midrule
  \endhead
  \midrule
  元年 & 700 & \tabularnewline\hline
  二年 & 701 & \tabularnewline
  \bottomrule
\end{longtable}

\subsection{大足}

\begin{longtable}{|>{\centering\scriptsize}m{2em}|>{\centering\scriptsize}m{1.3em}|>{\centering}m{8.8em}|}
  % \caption{秦王政}\
  \toprule
  \SimHei \normalsize 年数 & \SimHei \scriptsize 公元 & \SimHei 大事件 \tabularnewline
  % \midrule
  \endfirsthead
  \toprule
  \SimHei \normalsize 年数 & \SimHei \scriptsize 公元 & \SimHei 大事件 \tabularnewline
  \midrule
  \endhead
  \midrule
  元年 & 701 & \tabularnewline
  \bottomrule
\end{longtable}

\subsection{长安}

\begin{longtable}{|>{\centering\scriptsize}m{2em}|>{\centering\scriptsize}m{1.3em}|>{\centering}m{8.8em}|}
  % \caption{秦王政}\
  \toprule
  \SimHei \normalsize 年数 & \SimHei \scriptsize 公元 & \SimHei 大事件 \tabularnewline
  % \midrule
  \endfirsthead
  \toprule
  \SimHei \normalsize 年数 & \SimHei \scriptsize 公元 & \SimHei 大事件 \tabularnewline
  \midrule
  \endhead
  \midrule
  元年 & 701 & \tabularnewline\hline
  二年 & 702 & \tabularnewline\hline
  三年 & 703 & \tabularnewline\hline
  四年 & 704 & \tabularnewline
  \bottomrule
\end{longtable}

\subsection{神龙}

\begin{longtable}{|>{\centering\scriptsize}m{2em}|>{\centering\scriptsize}m{1.3em}|>{\centering}m{8.8em}|}
  % \caption{秦王政}\
  \toprule
  \SimHei \normalsize 年数 & \SimHei \scriptsize 公元 & \SimHei 大事件 \tabularnewline
  % \midrule
  \endfirsthead
  \toprule
  \SimHei \normalsize 年数 & \SimHei \scriptsize 公元 & \SimHei 大事件 \tabularnewline
  \midrule
  \endhead
  \midrule
  元年 & 705 & \tabularnewline\hline
  二年 & 706 & \tabularnewline\hline
  三年 & 707 & \tabularnewline
  \bottomrule
\end{longtable}


%%% Local Variables:
%%% mode: latex
%%% TeX-engine: xetex
%%% TeX-master: "../Main"
%%% End:

%% -*- coding: utf-8 -*-
%% Time-stamp: <Chen Wang: 2018-07-11 21:40:56>

\section{中宗复辟\tiny(705-710)}

\subsection{景龙}

\begin{longtable}{|>{\centering\scriptsize}m{2em}|>{\centering\scriptsize}m{1.3em}|>{\centering}m{8.8em}|}
  % \caption{秦王政}\
  \toprule
  \SimHei \normalsize 年数 & \SimHei \scriptsize 公元 & \SimHei 大事件 \tabularnewline
  % \midrule
  \endfirsthead
  \toprule
  \SimHei \normalsize 年数 & \SimHei \scriptsize 公元 & \SimHei 大事件 \tabularnewline
  \midrule
  \endhead
  \midrule
  元年 & 707 & \tabularnewline\hline
  二年 & 708 & \tabularnewline\hline
  三年 & 709 & \tabularnewline\hline
  四年 & 710 & \tabularnewline
  \bottomrule
\end{longtable}


%%% Local Variables:
%%% mode: latex
%%% TeX-engine: xetex
%%% TeX-master: "../Main"
%%% End:

%% -*- coding: utf-8 -*-
%% Time-stamp: <Chen Wang: 2018-07-11 21:49:36>

\section{睿宗复辟\tiny(710-712)}

\subsection{景云}

\begin{longtable}{|>{\centering\scriptsize}m{2em}|>{\centering\scriptsize}m{1.3em}|>{\centering}m{8.8em}|}
  % \caption{秦王政}\
  \toprule
  \SimHei \normalsize 年数 & \SimHei \scriptsize 公元 & \SimHei 大事件 \tabularnewline
  % \midrule
  \endfirsthead
  \toprule
  \SimHei \normalsize 年数 & \SimHei \scriptsize 公元 & \SimHei 大事件 \tabularnewline
  \midrule
  \endhead
  \midrule
  元年 & 710 & \tabularnewline\hline
  二年 & 711 & \tabularnewline\hline
  三年 & 712 & \tabularnewline
  \bottomrule
\end{longtable}

\subsection{太极}

\begin{longtable}{|>{\centering\scriptsize}m{2em}|>{\centering\scriptsize}m{1.3em}|>{\centering}m{8.8em}|}
  % \caption{秦王政}\
  \toprule
  \SimHei \normalsize 年数 & \SimHei \scriptsize 公元 & \SimHei 大事件 \tabularnewline
  % \midrule
  \endfirsthead
  \toprule
  \SimHei \normalsize 年数 & \SimHei \scriptsize 公元 & \SimHei 大事件 \tabularnewline
  \midrule
  \endhead
  \midrule
  元年 & 712 & \tabularnewline
  \bottomrule
\end{longtable}

\subsection{延和}

\begin{longtable}{|>{\centering\scriptsize}m{2em}|>{\centering\scriptsize}m{1.3em}|>{\centering}m{8.8em}|}
  % \caption{秦王政}\
  \toprule
  \SimHei \normalsize 年数 & \SimHei \scriptsize 公元 & \SimHei 大事件 \tabularnewline
  % \midrule
  \endfirsthead
  \toprule
  \SimHei \normalsize 年数 & \SimHei \scriptsize 公元 & \SimHei 大事件 \tabularnewline
  \midrule
  \endhead
  \midrule
  元年 & 712 & \tabularnewline
  \bottomrule
\end{longtable}


%%% Local Variables:
%%% mode: latex
%%% TeX-engine: xetex
%%% TeX-master: "../Main"
%%% End:

%% -*- coding: utf-8 -*-
%% Time-stamp: <Chen Wang: 2018-07-11 21:51:42>

\section{玄宗\tiny(712-756)}

\subsection{先天}

\begin{longtable}{|>{\centering\scriptsize}m{2em}|>{\centering\scriptsize}m{1.3em}|>{\centering}m{8.8em}|}
  % \caption{秦王政}\
  \toprule
  \SimHei \normalsize 年数 & \SimHei \scriptsize 公元 & \SimHei 大事件 \tabularnewline
  % \midrule
  \endfirsthead
  \toprule
  \SimHei \normalsize 年数 & \SimHei \scriptsize 公元 & \SimHei 大事件 \tabularnewline
  \midrule
  \endhead
  \midrule
  元年 & 712 & \tabularnewline\hline
  二年 & 713 & \tabularnewline
  \bottomrule
\end{longtable}

\subsection{开元}

\begin{longtable}{|>{\centering\scriptsize}m{2em}|>{\centering\scriptsize}m{1.3em}|>{\centering}m{8.8em}|}
  % \caption{秦王政}\
  \toprule
  \SimHei \normalsize 年数 & \SimHei \scriptsize 公元 & \SimHei 大事件 \tabularnewline
  % \midrule
  \endfirsthead
  \toprule
  \SimHei \normalsize 年数 & \SimHei \scriptsize 公元 & \SimHei 大事件 \tabularnewline
  \midrule
  \endhead
  \midrule
  元年 & 713 & \tabularnewline\hline
  二年 & 714 & \tabularnewline\hline
  三年 & 715 & \tabularnewline\hline
  四年 & 716 & \tabularnewline\hline
  五年 & 717 & \tabularnewline\hline
  六年 & 718 & \tabularnewline\hline
  七年 & 719 & \tabularnewline\hline
  八年 & 720 & \tabularnewline\hline
  九年 & 721 & \tabularnewline\hline
  十年 & 722 & \tabularnewline\hline
  十一年 & 723 & \tabularnewline\hline
  十二年 & 724 & \tabularnewline\hline
  十三年 & 725 & \tabularnewline\hline
  十四年 & 726 & \tabularnewline\hline
  十五年 & 727 & \tabularnewline\hline
  十六年 & 728 & \tabularnewline\hline
  十七年 & 729 & \tabularnewline\hline
  十八年 & 730 & \tabularnewline\hline
  十九年 & 731 & \tabularnewline\hline
  二十年 & 732 & \tabularnewline\hline
  二一年 & 733 & \tabularnewline\hline
  二二年 & 734 & \tabularnewline\hline
  二三年 & 735 & \tabularnewline\hline
  二四年 & 736 & \tabularnewline\hline
  二五年 & 737 & \tabularnewline\hline
  二六年 & 738 & \tabularnewline\hline
  二七年 & 739 & \tabularnewline\hline
  二八年 & 740 & \tabularnewline\hline
  二九年 & 741 & \tabularnewline
  \bottomrule
\end{longtable}

\subsection{天宝}

\begin{longtable}{|>{\centering\scriptsize}m{2em}|>{\centering\scriptsize}m{1.3em}|>{\centering}m{8.8em}|}
  % \caption{秦王政}\
  \toprule
  \SimHei \normalsize 年数 & \SimHei \scriptsize 公元 & \SimHei 大事件 \tabularnewline
  % \midrule
  \endfirsthead
  \toprule
  \SimHei \normalsize 年数 & \SimHei \scriptsize 公元 & \SimHei 大事件 \tabularnewline
  \midrule
  \endhead
  \midrule
  元年 & 742 & \tabularnewline\hline
  二年 & 743 & \tabularnewline\hline
  三年 & 744 & \tabularnewline\hline
  四年 & 745 & \tabularnewline\hline
  五年 & 746 & \tabularnewline\hline
  六年 & 747 & \tabularnewline\hline
  七年 & 748 & \tabularnewline\hline
  八年 & 749 & \tabularnewline\hline
  九年 & 750 & \tabularnewline\hline
  十年 & 751 & \tabularnewline\hline
  十一年 & 752 & \tabularnewline\hline
  十二年 & 753 & \tabularnewline\hline
  十三年 & 754 & \tabularnewline\hline
  十四年 & 755 & \tabularnewline\hline
  十五年 & 756 & \tabularnewline
  \bottomrule
\end{longtable}


%%% Local Variables:
%%% mode: latex
%%% TeX-engine: xetex
%%% TeX-master: "../Main"
%%% End:

%% -*- coding: utf-8 -*-
%% Time-stamp: <Chen Wang: 2018-07-11 21:53:35>

\section{肃宗\tiny(756-762)}

\subsection{至德}

\begin{longtable}{|>{\centering\scriptsize}m{2em}|>{\centering\scriptsize}m{1.3em}|>{\centering}m{8.8em}|}
  % \caption{秦王政}\
  \toprule
  \SimHei \normalsize 年数 & \SimHei \scriptsize 公元 & \SimHei 大事件 \tabularnewline
  % \midrule
  \endfirsthead
  \toprule
  \SimHei \normalsize 年数 & \SimHei \scriptsize 公元 & \SimHei 大事件 \tabularnewline
  \midrule
  \endhead
  \midrule
  元年 & 756 & \tabularnewline\hline
  二年 & 757 & \tabularnewline\hline
  三年 & 758 & \tabularnewline
  \bottomrule
\end{longtable}

\subsection{乾元}

\begin{longtable}{|>{\centering\scriptsize}m{2em}|>{\centering\scriptsize}m{1.3em}|>{\centering}m{8.8em}|}
  % \caption{秦王政}\
  \toprule
  \SimHei \normalsize 年数 & \SimHei \scriptsize 公元 & \SimHei 大事件 \tabularnewline
  % \midrule
  \endfirsthead
  \toprule
  \SimHei \normalsize 年数 & \SimHei \scriptsize 公元 & \SimHei 大事件 \tabularnewline
  \midrule
  \endhead
  \midrule
  元年 & 758 & \tabularnewline\hline
  二年 & 759 & \tabularnewline\hline
  三年 & 760 & \tabularnewline
  \bottomrule
\end{longtable}

\subsection{上元}

\begin{longtable}{|>{\centering\scriptsize}m{2em}|>{\centering\scriptsize}m{1.3em}|>{\centering}m{8.8em}|}
  % \caption{秦王政}\
  \toprule
  \SimHei \normalsize 年数 & \SimHei \scriptsize 公元 & \SimHei 大事件 \tabularnewline
  % \midrule
  \endfirsthead
  \toprule
  \SimHei \normalsize 年数 & \SimHei \scriptsize 公元 & \SimHei 大事件 \tabularnewline
  \midrule
  \endhead
  \midrule
  元年 & 760 & \tabularnewline\hline
  二年 & 761 & \tabularnewline
  \bottomrule
\end{longtable}

\subsection{宝应}

\begin{longtable}{|>{\centering\scriptsize}m{2em}|>{\centering\scriptsize}m{1.3em}|>{\centering}m{8.8em}|}
  % \caption{秦王政}\
  \toprule
  \SimHei \normalsize 年数 & \SimHei \scriptsize 公元 & \SimHei 大事件 \tabularnewline
  % \midrule
  \endfirsthead
  \toprule
  \SimHei \normalsize 年数 & \SimHei \scriptsize 公元 & \SimHei 大事件 \tabularnewline
  \midrule
  \endhead
  \midrule
  元年 & 762 & \tabularnewline\hline
  二年 & 763 & \tabularnewline
  \bottomrule
\end{longtable}


%%% Local Variables:
%%% mode: latex
%%% TeX-engine: xetex
%%% TeX-master: "../Main"
%%% End:

%% -*- coding: utf-8 -*-
%% Time-stamp: <Chen Wang: 2018-07-11 21:55:17>

\section{代宗\tiny(762-779)}

\subsection{广德}

\begin{longtable}{|>{\centering\scriptsize}m{2em}|>{\centering\scriptsize}m{1.3em}|>{\centering}m{8.8em}|}
  % \caption{秦王政}\
  \toprule
  \SimHei \normalsize 年数 & \SimHei \scriptsize 公元 & \SimHei 大事件 \tabularnewline
  % \midrule
  \endfirsthead
  \toprule
  \SimHei \normalsize 年数 & \SimHei \scriptsize 公元 & \SimHei 大事件 \tabularnewline
  \midrule
  \endhead
  \midrule
  元年 & 763 & \tabularnewline\hline
  二年 & 764 & \tabularnewline
  \bottomrule
\end{longtable}

\subsection{永泰}

\begin{longtable}{|>{\centering\scriptsize}m{2em}|>{\centering\scriptsize}m{1.3em}|>{\centering}m{8.8em}|}
  % \caption{秦王政}\
  \toprule
  \SimHei \normalsize 年数 & \SimHei \scriptsize 公元 & \SimHei 大事件 \tabularnewline
  % \midrule
  \endfirsthead
  \toprule
  \SimHei \normalsize 年数 & \SimHei \scriptsize 公元 & \SimHei 大事件 \tabularnewline
  \midrule
  \endhead
  \midrule
  元年 & 765 & \tabularnewline\hline
  二年 & 766 & \tabularnewline
  \bottomrule
\end{longtable}

\subsection{大历}

\begin{longtable}{|>{\centering\scriptsize}m{2em}|>{\centering\scriptsize}m{1.3em}|>{\centering}m{8.8em}|}
  % \caption{秦王政}\
  \toprule
  \SimHei \normalsize 年数 & \SimHei \scriptsize 公元 & \SimHei 大事件 \tabularnewline
  % \midrule
  \endfirsthead
  \toprule
  \SimHei \normalsize 年数 & \SimHei \scriptsize 公元 & \SimHei 大事件 \tabularnewline
  \midrule
  \endhead
  \midrule
  元年 & 766 & \tabularnewline\hline
  二年 & 767 & \tabularnewline\hline
  三年 & 768 & \tabularnewline\hline
  四年 & 769 & \tabularnewline\hline
  五年 & 770 & \tabularnewline\hline
  六年 & 771 & \tabularnewline\hline
  七年 & 772 & \tabularnewline\hline
  八年 & 773 & \tabularnewline\hline
  九年 & 774 & \tabularnewline\hline
  十年 & 775 & \tabularnewline\hline
  十一年 & 776 & \tabularnewline\hline
  十二年 & 777 & \tabularnewline\hline
  十三年 & 778 & \tabularnewline\hline
  十四年 & 779 & \tabularnewline
  \bottomrule
\end{longtable}


%%% Local Variables:
%%% mode: latex
%%% TeX-engine: xetex
%%% TeX-master: "../Main"
%%% End:

%% -*- coding: utf-8 -*-
%% Time-stamp: <Chen Wang: 2018-07-11 21:57:17>

\section{德宗\tiny(779-805)}

\subsection{建中}

\begin{longtable}{|>{\centering\scriptsize}m{2em}|>{\centering\scriptsize}m{1.3em}|>{\centering}m{8.8em}|}
  % \caption{秦王政}\
  \toprule
  \SimHei \normalsize 年数 & \SimHei \scriptsize 公元 & \SimHei 大事件 \tabularnewline
  % \midrule
  \endfirsthead
  \toprule
  \SimHei \normalsize 年数 & \SimHei \scriptsize 公元 & \SimHei 大事件 \tabularnewline
  \midrule
  \endhead
  \midrule
  元年 & 780 & \tabularnewline\hline
  二年 & 781 & \tabularnewline\hline
  三年 & 782 & \tabularnewline\hline
  四年 & 783 & \tabularnewline
  \bottomrule
\end{longtable}

\subsection{兴元}

\begin{longtable}{|>{\centering\scriptsize}m{2em}|>{\centering\scriptsize}m{1.3em}|>{\centering}m{8.8em}|}
  % \caption{秦王政}\
  \toprule
  \SimHei \normalsize 年数 & \SimHei \scriptsize 公元 & \SimHei 大事件 \tabularnewline
  % \midrule
  \endfirsthead
  \toprule
  \SimHei \normalsize 年数 & \SimHei \scriptsize 公元 & \SimHei 大事件 \tabularnewline
  \midrule
  \endhead
  \midrule
  元年 & 784 & \tabularnewline
  \bottomrule
\end{longtable}

\subsection{贞元}

\begin{longtable}{|>{\centering\scriptsize}m{2em}|>{\centering\scriptsize}m{1.3em}|>{\centering}m{8.8em}|}
  % \caption{秦王政}\
  \toprule
  \SimHei \normalsize 年数 & \SimHei \scriptsize 公元 & \SimHei 大事件 \tabularnewline
  % \midrule
  \endfirsthead
  \toprule
  \SimHei \normalsize 年数 & \SimHei \scriptsize 公元 & \SimHei 大事件 \tabularnewline
  \midrule
  \endhead
  \midrule
  元年 & 785 & \tabularnewline\hline
  二年 & 786 & \tabularnewline\hline
  三年 & 787 & \tabularnewline\hline
  四年 & 788 & \tabularnewline\hline
  五年 & 789 & \tabularnewline\hline
  六年 & 790 & \tabularnewline\hline
  七年 & 791 & \tabularnewline\hline
  八年 & 792 & \tabularnewline\hline
  九年 & 793 & \tabularnewline\hline
  十年 & 794 & \tabularnewline\hline
  十一年 & 795 & \tabularnewline\hline
  十二年 & 796 & \tabularnewline\hline
  十三年 & 797 & \tabularnewline\hline
  十四年 & 798 & \tabularnewline\hline
  十五年 & 799 & \tabularnewline\hline
  十六年 & 800 & \tabularnewline\hline
  十七年 & 801 & \tabularnewline\hline
  十八年 & 802 & \tabularnewline\hline
  十九年 & 803 & \tabularnewline\hline
  二十年 & 804 & \tabularnewline\hline
  二一年 & 805 & \tabularnewline
  \bottomrule
\end{longtable}


%%% Local Variables:
%%% mode: latex
%%% TeX-engine: xetex
%%% TeX-master: "../Main"
%%% End:

%% -*- coding: utf-8 -*-
%% Time-stamp: <Chen Wang: 2018-07-11 21:58:08>

\section{顺宗\tiny(805)}

\subsection{永贞}

\begin{longtable}{|>{\centering\scriptsize}m{2em}|>{\centering\scriptsize}m{1.3em}|>{\centering}m{8.8em}|}
  % \caption{秦王政}\
  \toprule
  \SimHei \normalsize 年数 & \SimHei \scriptsize 公元 & \SimHei 大事件 \tabularnewline
  % \midrule
  \endfirsthead
  \toprule
  \SimHei \normalsize 年数 & \SimHei \scriptsize 公元 & \SimHei 大事件 \tabularnewline
  \midrule
  \endhead
  \midrule
  元年 & 805 & \tabularnewline
  \bottomrule
\end{longtable}


%%% Local Variables:
%%% mode: latex
%%% TeX-engine: xetex
%%% TeX-master: "../Main"
%%% End:

%% -*- coding: utf-8 -*-
%% Time-stamp: <Chen Wang: 2018-07-11 21:59:10>

\section{宪宗\tiny(805-820)}

\subsection{元和}

\begin{longtable}{|>{\centering\scriptsize}m{2em}|>{\centering\scriptsize}m{1.3em}|>{\centering}m{8.8em}|}
  % \caption{秦王政}\
  \toprule
  \SimHei \normalsize 年数 & \SimHei \scriptsize 公元 & \SimHei 大事件 \tabularnewline
  % \midrule
  \endfirsthead
  \toprule
  \SimHei \normalsize 年数 & \SimHei \scriptsize 公元 & \SimHei 大事件 \tabularnewline
  \midrule
  \endhead
  \midrule
  元年 & 806 & \tabularnewline\hline
  二年 & 807 & \tabularnewline\hline
  三年 & 808 & \tabularnewline\hline
  四年 & 809 & \tabularnewline\hline
  五年 & 810 & \tabularnewline\hline
  六年 & 811 & \tabularnewline\hline
  七年 & 812 & \tabularnewline\hline
  八年 & 813 & \tabularnewline\hline
  九年 & 814 & \tabularnewline\hline
  十年 & 815 & \tabularnewline\hline
  十一年 & 816 & \tabularnewline\hline
  十二年 & 817 & \tabularnewline\hline
  十三年 & 818 & \tabularnewline\hline
  十四年 & 819 & \tabularnewline\hline
  十五年 & 820 & \tabularnewline
  \bottomrule
\end{longtable}


%%% Local Variables:
%%% mode: latex
%%% TeX-engine: xetex
%%% TeX-master: "../Main"
%%% End:

%% -*- coding: utf-8 -*-
%% Time-stamp: <Chen Wang: 2018-07-11 22:00:29>

\section{穆宗\tiny(820-824)}

\subsection{永新}

\begin{longtable}{|>{\centering\scriptsize}m{2em}|>{\centering\scriptsize}m{1.3em}|>{\centering}m{8.8em}|}
  % \caption{秦王政}\
  \toprule
  \SimHei \normalsize 年数 & \SimHei \scriptsize 公元 & \SimHei 大事件 \tabularnewline
  % \midrule
  \endfirsthead
  \toprule
  \SimHei \normalsize 年数 & \SimHei \scriptsize 公元 & \SimHei 大事件 \tabularnewline
  \midrule
  \endhead
  \midrule
  元年 & 820 & \tabularnewline
  \bottomrule
\end{longtable}

\subsection{长庆}

\begin{longtable}{|>{\centering\scriptsize}m{2em}|>{\centering\scriptsize}m{1.3em}|>{\centering}m{8.8em}|}
  % \caption{秦王政}\
  \toprule
  \SimHei \normalsize 年数 & \SimHei \scriptsize 公元 & \SimHei 大事件 \tabularnewline
  % \midrule
  \endfirsthead
  \toprule
  \SimHei \normalsize 年数 & \SimHei \scriptsize 公元 & \SimHei 大事件 \tabularnewline
  \midrule
  \endhead
  \midrule
  元年 & 821 & \tabularnewline\hline
  二年 & 822 & \tabularnewline\hline
  三年 & 823 & \tabularnewline\hline
  四年 & 824 & \tabularnewline
  \bottomrule
\end{longtable}


%%% Local Variables:
%%% mode: latex
%%% TeX-engine: xetex
%%% TeX-master: "../Main"
%%% End:

%% -*- coding: utf-8 -*-
%% Time-stamp: <Chen Wang: 2018-07-11 22:01:12>

\section{敬宗\tiny(824-826)}

\subsection{宝历}

\begin{longtable}{|>{\centering\scriptsize}m{2em}|>{\centering\scriptsize}m{1.3em}|>{\centering}m{8.8em}|}
  % \caption{秦王政}\
  \toprule
  \SimHei \normalsize 年数 & \SimHei \scriptsize 公元 & \SimHei 大事件 \tabularnewline
  % \midrule
  \endfirsthead
  \toprule
  \SimHei \normalsize 年数 & \SimHei \scriptsize 公元 & \SimHei 大事件 \tabularnewline
  \midrule
  \endhead
  \midrule
  元年 & 825 & \tabularnewline\hline
  二年 & 826 & \tabularnewline\hline
  三年 & 827 & \tabularnewline
  \bottomrule
\end{longtable}


%%% Local Variables:
%%% mode: latex
%%% TeX-engine: xetex
%%% TeX-master: "../Main"
%%% End:

%% -*- coding: utf-8 -*-
%% Time-stamp: <Chen Wang: 2018-07-11 22:03:11>

\section{文宗\tiny(826-840)}

\subsection{大和}

\begin{longtable}{|>{\centering\scriptsize}m{2em}|>{\centering\scriptsize}m{1.3em}|>{\centering}m{8.8em}|}
  % \caption{秦王政}\
  \toprule
  \SimHei \normalsize 年数 & \SimHei \scriptsize 公元 & \SimHei 大事件 \tabularnewline
  % \midrule
  \endfirsthead
  \toprule
  \SimHei \normalsize 年数 & \SimHei \scriptsize 公元 & \SimHei 大事件 \tabularnewline
  \midrule
  \endhead
  \midrule
  元年 & 827 & \tabularnewline\hline
  二年 & 828 & \tabularnewline\hline
  三年 & 829 & \tabularnewline\hline
  四年 & 830 & \tabularnewline\hline
  五年 & 831 & \tabularnewline\hline
  六年 & 832 & \tabularnewline\hline
  七年 & 833 & \tabularnewline\hline
  八年 & 834 & \tabularnewline\hline
  九年 & 835 & \tabularnewline
  \bottomrule
\end{longtable}

\subsection{开成}

\begin{longtable}{|>{\centering\scriptsize}m{2em}|>{\centering\scriptsize}m{1.3em}|>{\centering}m{8.8em}|}
  % \caption{秦王政}\
  \toprule
  \SimHei \normalsize 年数 & \SimHei \scriptsize 公元 & \SimHei 大事件 \tabularnewline
  % \midrule
  \endfirsthead
  \toprule
  \SimHei \normalsize 年数 & \SimHei \scriptsize 公元 & \SimHei 大事件 \tabularnewline
  \midrule
  \endhead
  \midrule
  元年 & 836 & \tabularnewline\hline
  二年 & 837 & \tabularnewline\hline
  三年 & 838 & \tabularnewline\hline
  四年 & 839 & \tabularnewline\hline
  五年 & 840 & \tabularnewline
  \bottomrule
\end{longtable}


%%% Local Variables:
%%% mode: latex
%%% TeX-engine: xetex
%%% TeX-master: "../Main"
%%% End:

%% -*- coding: utf-8 -*-
%% Time-stamp: <Chen Wang: 2018-07-11 22:04:49>

\section{武宗\tiny(840-846)}

\subsection{会昌}

\begin{longtable}{|>{\centering\scriptsize}m{2em}|>{\centering\scriptsize}m{1.3em}|>{\centering}m{8.8em}|}
  % \caption{秦王政}\
  \toprule
  \SimHei \normalsize 年数 & \SimHei \scriptsize 公元 & \SimHei 大事件 \tabularnewline
  % \midrule
  \endfirsthead
  \toprule
  \SimHei \normalsize 年数 & \SimHei \scriptsize 公元 & \SimHei 大事件 \tabularnewline
  \midrule
  \endhead
  \midrule
  元年 & 841 & \tabularnewline\hline
  二年 & 842 & \tabularnewline\hline
  三年 & 843 & \tabularnewline\hline
  四年 & 844 & \tabularnewline\hline
  五年 & 845 & \tabularnewline\hline
  六年 & 846 & \tabularnewline
  \bottomrule
\end{longtable}


%%% Local Variables:
%%% mode: latex
%%% TeX-engine: xetex
%%% TeX-master: "../Main"
%%% End:

%% -*- coding: utf-8 -*-
%% Time-stamp: <Chen Wang: 2018-07-11 22:06:16>

\section{宣宗\tiny(846-859)}

\subsection{大中}

\begin{longtable}{|>{\centering\scriptsize}m{2em}|>{\centering\scriptsize}m{1.3em}|>{\centering}m{8.8em}|}
  % \caption{秦王政}\
  \toprule
  \SimHei \normalsize 年数 & \SimHei \scriptsize 公元 & \SimHei 大事件 \tabularnewline
  % \midrule
  \endfirsthead
  \toprule
  \SimHei \normalsize 年数 & \SimHei \scriptsize 公元 & \SimHei 大事件 \tabularnewline
  \midrule
  \endhead
  \midrule
  元年 & 847 & \tabularnewline\hline
  二年 & 848 & \tabularnewline\hline
  三年 & 849 & \tabularnewline\hline
  四年 & 850 & \tabularnewline\hline
  五年 & 851 & \tabularnewline\hline
  六年 & 852 & \tabularnewline\hline
  七年 & 853 & \tabularnewline\hline
  八年 & 854 & \tabularnewline\hline
  九年 & 855 & \tabularnewline\hline
  十年 & 856 & \tabularnewline\hline
  十一年 & 857 & \tabularnewline\hline
  十二年 & 858 & \tabularnewline\hline
  十三年 & 859 & \tabularnewline\hline
  十四年 & 860 & \tabularnewline
  \bottomrule
\end{longtable}


%%% Local Variables:
%%% mode: latex
%%% TeX-engine: xetex
%%% TeX-master: "../Main"
%%% End:

%% -*- coding: utf-8 -*-
%% Time-stamp: <Chen Wang: 2018-07-11 22:07:59>

\section{懿宗\tiny(859-873)}

\subsection{咸通}

\begin{longtable}{|>{\centering\scriptsize}m{2em}|>{\centering\scriptsize}m{1.3em}|>{\centering}m{8.8em}|}
  % \caption{秦王政}\
  \toprule
  \SimHei \normalsize 年数 & \SimHei \scriptsize 公元 & \SimHei 大事件 \tabularnewline
  % \midrule
  \endfirsthead
  \toprule
  \SimHei \normalsize 年数 & \SimHei \scriptsize 公元 & \SimHei 大事件 \tabularnewline
  \midrule
  \endhead
  \midrule
  元年 & 860 & \tabularnewline\hline
  二年 & 861 & \tabularnewline\hline
  三年 & 862 & \tabularnewline\hline
  四年 & 863 & \tabularnewline\hline
  五年 & 864 & \tabularnewline\hline
  六年 & 865 & \tabularnewline\hline
  七年 & 866 & \tabularnewline\hline
  八年 & 867 & \tabularnewline\hline
  九年 & 868 & \tabularnewline\hline
  十年 & 869 & \tabularnewline\hline
  十一年 & 870 & \tabularnewline\hline
  十二年 & 871 & \tabularnewline\hline
  十三年 & 872 & \tabularnewline\hline
  十四年 & 873 & \tabularnewline\hline
  十五年 & 874 & \tabularnewline
  \bottomrule
\end{longtable}


%%% Local Variables:
%%% mode: latex
%%% TeX-engine: xetex
%%% TeX-master: "../Main"
%%% End:

%% -*- coding: utf-8 -*-
%% Time-stamp: <Chen Wang: 2018-07-11 22:10:19>

\section{僖宗\tiny(873-888)}

\subsection{乾符}

\begin{longtable}{|>{\centering\scriptsize}m{2em}|>{\centering\scriptsize}m{1.3em}|>{\centering}m{8.8em}|}
  % \caption{秦王政}\
  \toprule
  \SimHei \normalsize 年数 & \SimHei \scriptsize 公元 & \SimHei 大事件 \tabularnewline
  % \midrule
  \endfirsthead
  \toprule
  \SimHei \normalsize 年数 & \SimHei \scriptsize 公元 & \SimHei 大事件 \tabularnewline
  \midrule
  \endhead
  \midrule
  元年 & 874 & \tabularnewline\hline
  二年 & 875 & \tabularnewline\hline
  三年 & 876 & \tabularnewline\hline
  四年 & 877 & \tabularnewline\hline
  五年 & 878 & \tabularnewline\hline
  六年 & 879 & \tabularnewline
  \bottomrule
\end{longtable}

\subsection{广明}

\begin{longtable}{|>{\centering\scriptsize}m{2em}|>{\centering\scriptsize}m{1.3em}|>{\centering}m{8.8em}|}
  % \caption{秦王政}\
  \toprule
  \SimHei \normalsize 年数 & \SimHei \scriptsize 公元 & \SimHei 大事件 \tabularnewline
  % \midrule
  \endfirsthead
  \toprule
  \SimHei \normalsize 年数 & \SimHei \scriptsize 公元 & \SimHei 大事件 \tabularnewline
  \midrule
  \endhead
  \midrule
  元年 & 880 & \tabularnewline\hline
  二年 & 881 & \tabularnewline
  \bottomrule
\end{longtable}

\subsection{中和}

\begin{longtable}{|>{\centering\scriptsize}m{2em}|>{\centering\scriptsize}m{1.3em}|>{\centering}m{8.8em}|}
  % \caption{秦王政}\
  \toprule
  \SimHei \normalsize 年数 & \SimHei \scriptsize 公元 & \SimHei 大事件 \tabularnewline
  % \midrule
  \endfirsthead
  \toprule
  \SimHei \normalsize 年数 & \SimHei \scriptsize 公元 & \SimHei 大事件 \tabularnewline
  \midrule
  \endhead
  \midrule
  元年 & 881 & \tabularnewline\hline
  二年 & 882 & \tabularnewline\hline
  三年 & 883 & \tabularnewline\hline
  四年 & 884 & \tabularnewline\hline
  五年 & 885 & \tabularnewline
  \bottomrule
\end{longtable}

\subsection{光启}

\begin{longtable}{|>{\centering\scriptsize}m{2em}|>{\centering\scriptsize}m{1.3em}|>{\centering}m{8.8em}|}
  % \caption{秦王政}\
  \toprule
  \SimHei \normalsize 年数 & \SimHei \scriptsize 公元 & \SimHei 大事件 \tabularnewline
  % \midrule
  \endfirsthead
  \toprule
  \SimHei \normalsize 年数 & \SimHei \scriptsize 公元 & \SimHei 大事件 \tabularnewline
  \midrule
  \endhead
  \midrule
  元年 & 885 & \tabularnewline\hline
  二年 & 886 & \tabularnewline\hline
  三年 & 887 & \tabularnewline\hline
  四年 & 888 & \tabularnewline
  \bottomrule
\end{longtable}

\subsection{文德}

\begin{longtable}{|>{\centering\scriptsize}m{2em}|>{\centering\scriptsize}m{1.3em}|>{\centering}m{8.8em}|}
  % \caption{秦王政}\
  \toprule
  \SimHei \normalsize 年数 & \SimHei \scriptsize 公元 & \SimHei 大事件 \tabularnewline
  % \midrule
  \endfirsthead
  \toprule
  \SimHei \normalsize 年数 & \SimHei \scriptsize 公元 & \SimHei 大事件 \tabularnewline
  \midrule
  \endhead
  \midrule
  元年 & 888 & \tabularnewline
  \bottomrule
\end{longtable}


%%% Local Variables:
%%% mode: latex
%%% TeX-engine: xetex
%%% TeX-master: "../Main"
%%% End:

%% -*- coding: utf-8 -*-
%% Time-stamp: <Chen Wang: 2018-07-11 22:18:39>

\section{昭宗\tiny(888-904)}

\subsection{龙纪}

\begin{longtable}{|>{\centering\scriptsize}m{2em}|>{\centering\scriptsize}m{1.3em}|>{\centering}m{8.8em}|}
  % \caption{秦王政}\
  \toprule
  \SimHei \normalsize 年数 & \SimHei \scriptsize 公元 & \SimHei 大事件 \tabularnewline
  % \midrule
  \endfirsthead
  \toprule
  \SimHei \normalsize 年数 & \SimHei \scriptsize 公元 & \SimHei 大事件 \tabularnewline
  \midrule
  \endhead
  \midrule
  元年 & 889 & \tabularnewline
  \bottomrule
\end{longtable}

\subsection{大顺}

\begin{longtable}{|>{\centering\scriptsize}m{2em}|>{\centering\scriptsize}m{1.3em}|>{\centering}m{8.8em}|}
  % \caption{秦王政}\
  \toprule
  \SimHei \normalsize 年数 & \SimHei \scriptsize 公元 & \SimHei 大事件 \tabularnewline
  % \midrule
  \endfirsthead
  \toprule
  \SimHei \normalsize 年数 & \SimHei \scriptsize 公元 & \SimHei 大事件 \tabularnewline
  \midrule
  \endhead
  \midrule
  元年 & 890 & \tabularnewline\hline
  二年 & 891 & \tabularnewline
  \bottomrule
\end{longtable}

\subsection{景福}

\begin{longtable}{|>{\centering\scriptsize}m{2em}|>{\centering\scriptsize}m{1.3em}|>{\centering}m{8.8em}|}
  % \caption{秦王政}\
  \toprule
  \SimHei \normalsize 年数 & \SimHei \scriptsize 公元 & \SimHei 大事件 \tabularnewline
  % \midrule
  \endfirsthead
  \toprule
  \SimHei \normalsize 年数 & \SimHei \scriptsize 公元 & \SimHei 大事件 \tabularnewline
  \midrule
  \endhead
  \midrule
  元年 & 892 & \tabularnewline\hline
  二年 & 893 & \tabularnewline
  \bottomrule
\end{longtable}

\subsection{乾宁}

\begin{longtable}{|>{\centering\scriptsize}m{2em}|>{\centering\scriptsize}m{1.3em}|>{\centering}m{8.8em}|}
  % \caption{秦王政}\
  \toprule
  \SimHei \normalsize 年数 & \SimHei \scriptsize 公元 & \SimHei 大事件 \tabularnewline
  % \midrule
  \endfirsthead
  \toprule
  \SimHei \normalsize 年数 & \SimHei \scriptsize 公元 & \SimHei 大事件 \tabularnewline
  \midrule
  \endhead
  \midrule
  元年 & 894 & \tabularnewline\hline
  二年 & 895 & \tabularnewline\hline
  三年 & 896 & \tabularnewline\hline
  四年 & 897 & \tabularnewline\hline
  五年 & 898 & \tabularnewline
  \bottomrule
\end{longtable}

\subsection{光化}

\begin{longtable}{|>{\centering\scriptsize}m{2em}|>{\centering\scriptsize}m{1.3em}|>{\centering}m{8.8em}|}
  % \caption{秦王政}\
  \toprule
  \SimHei \normalsize 年数 & \SimHei \scriptsize 公元 & \SimHei 大事件 \tabularnewline
  % \midrule
  \endfirsthead
  \toprule
  \SimHei \normalsize 年数 & \SimHei \scriptsize 公元 & \SimHei 大事件 \tabularnewline
  \midrule
  \endhead
  \midrule
  元年 & 898 & \tabularnewline\hline
  二年 & 899 & \tabularnewline\hline
  三年 & 900 & \tabularnewline\hline
  四年 & 901 & \tabularnewline
  \bottomrule
\end{longtable}

\subsection{天复}

\begin{longtable}{|>{\centering\scriptsize}m{2em}|>{\centering\scriptsize}m{1.3em}|>{\centering}m{8.8em}|}
  % \caption{秦王政}\
  \toprule
  \SimHei \normalsize 年数 & \SimHei \scriptsize 公元 & \SimHei 大事件 \tabularnewline
  % \midrule
  \endfirsthead
  \toprule
  \SimHei \normalsize 年数 & \SimHei \scriptsize 公元 & \SimHei 大事件 \tabularnewline
  \midrule
  \endhead
  \midrule
  元年 & 901 & \tabularnewline\hline
  二年 & 902 & \tabularnewline\hline
  三年 & 903 & \tabularnewline\hline
  四年 & 904 & \tabularnewline
  \bottomrule
\end{longtable}


%%% Local Variables:
%%% mode: latex
%%% TeX-engine: xetex
%%% TeX-master: "../Main"
%%% End:

%% -*- coding: utf-8 -*-
%% Time-stamp: <Chen Wang: 2018-07-11 22:19:18>

\section{景宗\tiny(904-907)}

\subsection{天佑}

\begin{longtable}{|>{\centering\scriptsize}m{2em}|>{\centering\scriptsize}m{1.3em}|>{\centering}m{8.8em}|}
  % \caption{秦王政}\
  \toprule
  \SimHei \normalsize 年数 & \SimHei \scriptsize 公元 & \SimHei 大事件 \tabularnewline
  % \midrule
  \endfirsthead
  \toprule
  \SimHei \normalsize 年数 & \SimHei \scriptsize 公元 & \SimHei 大事件 \tabularnewline
  \midrule
  \endhead
  \midrule
  元年 & 904 & \tabularnewline\hline
  二年 & 905 & \tabularnewline\hline
  三年 & 906 & \tabularnewline\hline
  四年 & 907 & \tabularnewline
  \bottomrule
\end{longtable}


%%% Local Variables:
%%% mode: latex
%%% TeX-engine: xetex
%%% TeX-master: "../Main"
%%% End:


%%% Local Variables:
%%% mode: latex
%%% TeX-engine: xetex
%%% TeX-master: "../Main"
%%% End:

% %% -*- coding: utf-8 -*-
%% Time-stamp: <Chen Wang: 2018-07-11 22:27:45>

\chapter{五代\tiny(907-960)}


%% -*- coding: utf-8 -*-
%% Time-stamp: <Chen Wang: 2018-07-11 22:24:23>


\section{后梁\tiny(907-923)}

%% -*- coding: utf-8 -*-
%% Time-stamp: <Chen Wang: 2018-07-11 22:25:54>

\subsection{太祖\tiny(907-912)}

\subsubsection{开平}

\begin{longtable}{|>{\centering\scriptsize}m{2em}|>{\centering\scriptsize}m{1.3em}|>{\centering}m{8.8em}|}
  % \caption{秦王政}\
  \toprule
  \SimHei \normalsize 年数 & \SimHei \scriptsize 公元 & \SimHei 大事件 \tabularnewline
  % \midrule
  \endfirsthead
  \toprule
  \SimHei \normalsize 年数 & \SimHei \scriptsize 公元 & \SimHei 大事件 \tabularnewline
  \midrule
  \endhead
  \midrule
  元年 & 907 & \tabularnewline\hline
  二年 & 908 & \tabularnewline\hline
  三年 & 909 & \tabularnewline\hline
  四年 & 910 & \tabularnewline\hline
  五年 & 911 & \tabularnewline
  \bottomrule
\end{longtable}

\subsubsection{乾化}

\begin{longtable}{|>{\centering\scriptsize}m{2em}|>{\centering\scriptsize}m{1.3em}|>{\centering}m{8.8em}|}
  % \caption{秦王政}\
  \toprule
  \SimHei \normalsize 年数 & \SimHei \scriptsize 公元 & \SimHei 大事件 \tabularnewline
  % \midrule
  \endfirsthead
  \toprule
  \SimHei \normalsize 年数 & \SimHei \scriptsize 公元 & \SimHei 大事件 \tabularnewline
  \midrule
  \endhead
  \midrule
  元年 & 911 & \tabularnewline\hline
  二年 & 912 & \tabularnewline\hline
  三年 & 913 & \tabularnewline
  \bottomrule
\end{longtable}


%%% Local Variables:
%%% mode: latex
%%% TeX-engine: xetex
%%% TeX-master: "../../Main"
%%% End:


%%% Local Variables:
%%% mode: latex
%%% TeX-engine: xetex
%%% TeX-master: "../../Main"
%%% End:



%%% Local Variables:
%%% mode: latex
%%% TeX-engine: xetex
%%% TeX-master: "../Main"
%%% End:

% %% -*- coding: utf-8 -*-
%% Time-stamp: <Chen Wang: 2018-07-11 23:30:30>

\chapter{十国\tiny(907-979)}


%% -*- coding: utf-8 -*-
%% Time-stamp: <Chen Wang: 2018-07-11 23:22:18>


\section{吴\tiny(902-937)}

%% -*- coding: utf-8 -*-
%% Time-stamp: <Chen Wang: 2018-07-11 23:21:14>

\subsection{孝武王\tiny(902-905)}

\subsubsection{天复}

\begin{longtable}{|>{\centering\scriptsize}m{2em}|>{\centering\scriptsize}m{1.3em}|>{\centering}m{8.8em}|}
  % \caption{秦王政}\
  \toprule
  \SimHei \normalsize 年数 & \SimHei \scriptsize 公元 & \SimHei 大事件 \tabularnewline
  % \midrule
  \endfirsthead
  \toprule
  \SimHei \normalsize 年数 & \SimHei \scriptsize 公元 & \SimHei 大事件 \tabularnewline
  \midrule
  \endhead
  \midrule
  元年 & 902 & \tabularnewline\hline
  二年 & 903 & \tabularnewline\hline
  三年 & 904 & \tabularnewline
  \bottomrule
\end{longtable}

\subsubsection{天祐}

\begin{longtable}{|>{\centering\scriptsize}m{2em}|>{\centering\scriptsize}m{1.3em}|>{\centering}m{8.8em}|}
  % \caption{秦王政}\
  \toprule
  \SimHei \normalsize 年数 & \SimHei \scriptsize 公元 & \SimHei 大事件 \tabularnewline
  % \midrule
  \endfirsthead
  \toprule
  \SimHei \normalsize 年数 & \SimHei \scriptsize 公元 & \SimHei 大事件 \tabularnewline
  \midrule
  \endhead
  \midrule
  元年 & 904 & \tabularnewline\hline
  二年 & 905 & \tabularnewline
  \bottomrule
\end{longtable}



%%% Local Variables:
%%% mode: latex
%%% TeX-engine: xetex
%%% TeX-master: "../../Main"
%%% End:

%% -*- coding: utf-8 -*-
%% Time-stamp: <Chen Wang: 2018-07-11 23:23:00>

\subsection{杨渥\tiny(905-908)}

\subsubsection{天祐}

\begin{longtable}{|>{\centering\scriptsize}m{2em}|>{\centering\scriptsize}m{1.3em}|>{\centering}m{8.8em}|}
  % \caption{秦王政}\
  \toprule
  \SimHei \normalsize 年数 & \SimHei \scriptsize 公元 & \SimHei 大事件 \tabularnewline
  % \midrule
  \endfirsthead
  \toprule
  \SimHei \normalsize 年数 & \SimHei \scriptsize 公元 & \SimHei 大事件 \tabularnewline
  \midrule
  \endhead
  \midrule
  元年 & 905 & \tabularnewline\hline
  二年 & 906 & \tabularnewline\hline
  三年 & 907 & \tabularnewline\hline
  四年 & 908 & \tabularnewline
  \bottomrule
\end{longtable}


%%% Local Variables:
%%% mode: latex
%%% TeX-engine: xetex
%%% TeX-master: "../../Main"
%%% End:

%% -*- coding: utf-8 -*-
%% Time-stamp: <Chen Wang: 2018-07-11 23:24:26>

\subsection{宣王\tiny(908-920)}

\subsubsection{天祐}

\begin{longtable}{|>{\centering\scriptsize}m{2em}|>{\centering\scriptsize}m{1.3em}|>{\centering}m{8.8em}|}
  % \caption{秦王政}\
  \toprule
  \SimHei \normalsize 年数 & \SimHei \scriptsize 公元 & \SimHei 大事件 \tabularnewline
  % \midrule
  \endfirsthead
  \toprule
  \SimHei \normalsize 年数 & \SimHei \scriptsize 公元 & \SimHei 大事件 \tabularnewline
  \midrule
  \endhead
  \midrule
  元年 & 908 & \tabularnewline\hline
  二年 & 909 & \tabularnewline\hline
  三年 & 910 & \tabularnewline\hline
  四年 & 911 & \tabularnewline\hline
  五年 & 912 & \tabularnewline\hline
  六年 & 913 & \tabularnewline\hline
  七年 & 914 & \tabularnewline\hline
  八年 & 915 & \tabularnewline\hline
  九年 & 916 & \tabularnewline\hline
  十年 & 917 & \tabularnewline\hline
  十一年 & 918 & \tabularnewline\hline
  十二年 & 919 & \tabularnewline
  \bottomrule
\end{longtable}

\subsubsection{武义}

\begin{longtable}{|>{\centering\scriptsize}m{2em}|>{\centering\scriptsize}m{1.3em}|>{\centering}m{8.8em}|}
  % \caption{秦王政}\
  \toprule
  \SimHei \normalsize 年数 & \SimHei \scriptsize 公元 & \SimHei 大事件 \tabularnewline
  % \midrule
  \endfirsthead
  \toprule
  \SimHei \normalsize 年数 & \SimHei \scriptsize 公元 & \SimHei 大事件 \tabularnewline
  \midrule
  \endhead
  \midrule
  元年 & 919 & \tabularnewline\hline
  二年 & 920 & \tabularnewline\hline
  三年 & 921 & \tabularnewline
  \bottomrule
\end{longtable}


%%% Local Variables:
%%% mode: latex
%%% TeX-engine: xetex
%%% TeX-master: "../../Main"
%%% End:

%% -*- coding: utf-8 -*-
%% Time-stamp: <Chen Wang: 2018-07-11 23:26:38>

\subsection{睿帝\tiny(920-937)}

\subsubsection{顺义}

\begin{longtable}{|>{\centering\scriptsize}m{2em}|>{\centering\scriptsize}m{1.3em}|>{\centering}m{8.8em}|}
  % \caption{秦王政}\
  \toprule
  \SimHei \normalsize 年数 & \SimHei \scriptsize 公元 & \SimHei 大事件 \tabularnewline
  % \midrule
  \endfirsthead
  \toprule
  \SimHei \normalsize 年数 & \SimHei \scriptsize 公元 & \SimHei 大事件 \tabularnewline
  \midrule
  \endhead
  \midrule
  元年 & 921 & \tabularnewline\hline
  二年 & 922 & \tabularnewline\hline
  三年 & 923 & \tabularnewline\hline
  四年 & 924 & \tabularnewline\hline
  五年 & 925 & \tabularnewline\hline
  六年 & 926 & \tabularnewline\hline
  七年 & 927 & \tabularnewline
  \bottomrule
\end{longtable}

\subsubsection{乾贞}

\begin{longtable}{|>{\centering\scriptsize}m{2em}|>{\centering\scriptsize}m{1.3em}|>{\centering}m{8.8em}|}
  % \caption{秦王政}\
  \toprule
  \SimHei \normalsize 年数 & \SimHei \scriptsize 公元 & \SimHei 大事件 \tabularnewline
  % \midrule
  \endfirsthead
  \toprule
  \SimHei \normalsize 年数 & \SimHei \scriptsize 公元 & \SimHei 大事件 \tabularnewline
  \midrule
  \endhead
  \midrule
  元年 & 927 & \tabularnewline\hline
  二年 & 928 & \tabularnewline\hline
  三年 & 929 & \tabularnewline
  \bottomrule
\end{longtable}

\subsubsection{大和}

\begin{longtable}{|>{\centering\scriptsize}m{2em}|>{\centering\scriptsize}m{1.3em}|>{\centering}m{8.8em}|}
  % \caption{秦王政}\
  \toprule
  \SimHei \normalsize 年数 & \SimHei \scriptsize 公元 & \SimHei 大事件 \tabularnewline
  % \midrule
  \endfirsthead
  \toprule
  \SimHei \normalsize 年数 & \SimHei \scriptsize 公元 & \SimHei 大事件 \tabularnewline
  \midrule
  \endhead
  \midrule
  元年 & 929 & \tabularnewline\hline
  二年 & 930 & \tabularnewline\hline
  三年 & 931 & \tabularnewline\hline
  四年 & 932 & \tabularnewline\hline
  五年 & 933 & \tabularnewline\hline
  六年 & 934 & \tabularnewline\hline
  七年 & 935 & \tabularnewline
  \bottomrule
\end{longtable}

\subsubsection{天祚}

\begin{longtable}{|>{\centering\scriptsize}m{2em}|>{\centering\scriptsize}m{1.3em}|>{\centering}m{8.8em}|}
  % \caption{秦王政}\
  \toprule
  \SimHei \normalsize 年数 & \SimHei \scriptsize 公元 & \SimHei 大事件 \tabularnewline
  % \midrule
  \endfirsthead
  \toprule
  \SimHei \normalsize 年数 & \SimHei \scriptsize 公元 & \SimHei 大事件 \tabularnewline
  \midrule
  \endhead
  \midrule
  元年 & 935 & \tabularnewline\hline
  二年 & 936 & \tabularnewline\hline
  三年 & 937 & \tabularnewline
  \bottomrule
\end{longtable}


%%% Local Variables:
%%% mode: latex
%%% TeX-engine: xetex
%%% TeX-master: "../../Main"
%%% End:



%%% Local Variables:
%%% mode: latex
%%% TeX-engine: xetex
%%% TeX-master: "../../Main"
%%% End:

%% -*- coding: utf-8 -*-
%% Time-stamp: <Chen Wang: 2018-07-11 23:39:50>


\section{南唐\tiny(937-975)}

%% -*- coding: utf-8 -*-
%% Time-stamp: <Chen Wang: 2018-07-11 23:40:37>

\subsection{烈祖\tiny(937-943)}

\subsubsection{昇元}

\begin{longtable}{|>{\centering\scriptsize}m{2em}|>{\centering\scriptsize}m{1.3em}|>{\centering}m{8.8em}|}
  % \caption{秦王政}\
  \toprule
  \SimHei \normalsize 年数 & \SimHei \scriptsize 公元 & \SimHei 大事件 \tabularnewline
  % \midrule
  \endfirsthead
  \toprule
  \SimHei \normalsize 年数 & \SimHei \scriptsize 公元 & \SimHei 大事件 \tabularnewline
  \midrule
  \endhead
  \midrule
  元年 & 937 & \tabularnewline\hline
  二年 & 938 & \tabularnewline\hline
  三年 & 939 & \tabularnewline\hline
  四年 & 940 & \tabularnewline\hline
  五年 & 941 & \tabularnewline\hline
  六年 & 942 & \tabularnewline\hline
  七年 & 943 & \tabularnewline
  \bottomrule
\end{longtable}


%%% Local Variables:
%%% mode: latex
%%% TeX-engine: xetex
%%% TeX-master: "../../Main"
%%% End:

%% -*- coding: utf-8 -*-
%% Time-stamp: <Chen Wang: 2018-07-11 23:42:24>

\subsection{元宗\tiny(943-961)}

\subsubsection{保大}

\begin{longtable}{|>{\centering\scriptsize}m{2em}|>{\centering\scriptsize}m{1.3em}|>{\centering}m{8.8em}|}
  % \caption{秦王政}\
  \toprule
  \SimHei \normalsize 年数 & \SimHei \scriptsize 公元 & \SimHei 大事件 \tabularnewline
  % \midrule
  \endfirsthead
  \toprule
  \SimHei \normalsize 年数 & \SimHei \scriptsize 公元 & \SimHei 大事件 \tabularnewline
  \midrule
  \endhead
  \midrule
  元年 & 943 & \tabularnewline\hline
  二年 & 944 & \tabularnewline\hline
  三年 & 945 & \tabularnewline\hline
  四年 & 946 & \tabularnewline\hline
  五年 & 947 & \tabularnewline\hline
  六年 & 948 & \tabularnewline\hline
  七年 & 949 & \tabularnewline\hline
  八年 & 950 & \tabularnewline\hline
  九年 & 951 & \tabularnewline\hline
  十年 & 952 & \tabularnewline\hline
  十一年 & 953 & \tabularnewline\hline
  十二年 & 954 & \tabularnewline\hline
  十三年 & 955 & \tabularnewline\hline
  十四年 & 956 & \tabularnewline\hline
  十五年 & 957 & \tabularnewline
  \bottomrule
\end{longtable}

\subsubsection{中兴}

\begin{longtable}{|>{\centering\scriptsize}m{2em}|>{\centering\scriptsize}m{1.3em}|>{\centering}m{8.8em}|}
  % \caption{秦王政}\
  \toprule
  \SimHei \normalsize 年数 & \SimHei \scriptsize 公元 & \SimHei 大事件 \tabularnewline
  % \midrule
  \endfirsthead
  \toprule
  \SimHei \normalsize 年数 & \SimHei \scriptsize 公元 & \SimHei 大事件 \tabularnewline
  \midrule
  \endhead
  \midrule
  元年 & 958 & \tabularnewline
  \bottomrule
\end{longtable}

\subsubsection{交泰}

\begin{longtable}{|>{\centering\scriptsize}m{2em}|>{\centering\scriptsize}m{1.3em}|>{\centering}m{8.8em}|}
  % \caption{秦王政}\
  \toprule
  \SimHei \normalsize 年数 & \SimHei \scriptsize 公元 & \SimHei 大事件 \tabularnewline
  % \midrule
  \endfirsthead
  \toprule
  \SimHei \normalsize 年数 & \SimHei \scriptsize 公元 & \SimHei 大事件 \tabularnewline
  \midrule
  \endhead
  \midrule
  元年 & 958 & \tabularnewline
  \bottomrule
\end{longtable}

\subsubsection{显德}

\begin{longtable}{|>{\centering\scriptsize}m{2em}|>{\centering\scriptsize}m{1.3em}|>{\centering}m{8.8em}|}
  % \caption{秦王政}\
  \toprule
  \SimHei \normalsize 年数 & \SimHei \scriptsize 公元 & \SimHei 大事件 \tabularnewline
  % \midrule
  \endfirsthead
  \toprule
  \SimHei \normalsize 年数 & \SimHei \scriptsize 公元 & \SimHei 大事件 \tabularnewline
  \midrule
  \endhead
  \midrule
  元年 & 958 & \tabularnewline\hline
  二年 & 959 & \tabularnewline\hline
  三年 & 960 & \tabularnewline\hline
  四年 & 961 & \tabularnewline
  \bottomrule
\end{longtable}


%%% Local Variables:
%%% mode: latex
%%% TeX-engine: xetex
%%% TeX-master: "../../Main"
%%% End:

%% -*- coding: utf-8 -*-
%% Time-stamp: <Chen Wang: 2018-07-11 23:44:09>

\subsection{后主\tiny(961-975)}

\subsubsection{显德}

\begin{longtable}{|>{\centering\scriptsize}m{2em}|>{\centering\scriptsize}m{1.3em}|>{\centering}m{8.8em}|}
  % \caption{秦王政}\
  \toprule
  \SimHei \normalsize 年数 & \SimHei \scriptsize 公元 & \SimHei 大事件 \tabularnewline
  % \midrule
  \endfirsthead
  \toprule
  \SimHei \normalsize 年数 & \SimHei \scriptsize 公元 & \SimHei 大事件 \tabularnewline
  \midrule
  \endhead
  \midrule
  元年 & 961 & \tabularnewline\hline
  二年 & 962 & \tabularnewline
  \bottomrule
\end{longtable}

\subsubsection{建隆}

\begin{longtable}{|>{\centering\scriptsize}m{2em}|>{\centering\scriptsize}m{1.3em}|>{\centering}m{8.8em}|}
  % \caption{秦王政}\
  \toprule
  \SimHei \normalsize 年数 & \SimHei \scriptsize 公元 & \SimHei 大事件 \tabularnewline
  % \midrule
  \endfirsthead
  \toprule
  \SimHei \normalsize 年数 & \SimHei \scriptsize 公元 & \SimHei 大事件 \tabularnewline
  \midrule
  \endhead
  \midrule
  元年 & 963 & \tabularnewline
  \bottomrule
\end{longtable}

\subsubsection{乾德}

\begin{longtable}{|>{\centering\scriptsize}m{2em}|>{\centering\scriptsize}m{1.3em}|>{\centering}m{8.8em}|}
  % \caption{秦王政}\
  \toprule
  \SimHei \normalsize 年数 & \SimHei \scriptsize 公元 & \SimHei 大事件 \tabularnewline
  % \midrule
  \endfirsthead
  \toprule
  \SimHei \normalsize 年数 & \SimHei \scriptsize 公元 & \SimHei 大事件 \tabularnewline
  \midrule
  \endhead
  \midrule
  元年 & 963 & \tabularnewline\hline
  二年 & 964 & \tabularnewline\hline
  三年 & 965 & \tabularnewline\hline
  四年 & 966 & \tabularnewline\hline
  五年 & 967 & \tabularnewline\hline
  六年 & 968 & \tabularnewline
  \bottomrule
\end{longtable}

\subsubsection{开宝}

\begin{longtable}{|>{\centering\scriptsize}m{2em}|>{\centering\scriptsize}m{1.3em}|>{\centering}m{8.8em}|}
  % \caption{秦王政}\
  \toprule
  \SimHei \normalsize 年数 & \SimHei \scriptsize 公元 & \SimHei 大事件 \tabularnewline
  % \midrule
  \endfirsthead
  \toprule
  \SimHei \normalsize 年数 & \SimHei \scriptsize 公元 & \SimHei 大事件 \tabularnewline
  \midrule
  \endhead
  \midrule
  元年 & 968 & \tabularnewline\hline
  二年 & 969 & \tabularnewline\hline
  三年 & 970 & \tabularnewline\hline
  四年 & 971 & \tabularnewline\hline
  五年 & 972 & \tabularnewline\hline
  六年 & 973 & \tabularnewline\hline
  七年 & 974 & \tabularnewline\hline
  八年 & 975 & \tabularnewline
  \bottomrule
\end{longtable}


%%% Local Variables:
%%% mode: latex
%%% TeX-engine: xetex
%%% TeX-master: "../../Main"
%%% End:



%%% Local Variables:
%%% mode: latex
%%% TeX-engine: xetex
%%% TeX-master: "../../Main"
%%% End:

%% -*- coding: utf-8 -*-
%% Time-stamp: <Chen Wang: 2018-07-11 23:46:16>


\section{吴越\tiny(907-978)}

%% -*- coding: utf-8 -*-
%% Time-stamp: <Chen Wang: 2018-07-11 23:59:02>

\subsection{太祖\tiny(907-932)}

\subsubsection{天祐}

\begin{longtable}{|>{\centering\scriptsize}m{2em}|>{\centering\scriptsize}m{1.3em}|>{\centering}m{8.8em}|}
  % \caption{秦王政}\
  \toprule
  \SimHei \normalsize 年数 & \SimHei \scriptsize 公元 & \SimHei 大事件 \tabularnewline
  % \midrule
  \endfirsthead
  \toprule
  \SimHei \normalsize 年数 & \SimHei \scriptsize 公元 & \SimHei 大事件 \tabularnewline
  \midrule
  \endhead
  \midrule
  元年 & 907 & \tabularnewline
  \bottomrule
\end{longtable}

\subsubsection{天宝}

\begin{longtable}{|>{\centering\scriptsize}m{2em}|>{\centering\scriptsize}m{1.3em}|>{\centering}m{8.8em}|}
  % \caption{秦王政}\
  \toprule
  \SimHei \normalsize 年数 & \SimHei \scriptsize 公元 & \SimHei 大事件 \tabularnewline
  % \midrule
  \endfirsthead
  \toprule
  \SimHei \normalsize 年数 & \SimHei \scriptsize 公元 & \SimHei 大事件 \tabularnewline
  \midrule
  \endhead
  \midrule
  元年 & 908 & \tabularnewline\hline
  二年 & 909 & \tabularnewline\hline
  三年 & 910 & \tabularnewline\hline
  四年 & 911 & \tabularnewline\hline
  五年 & 912 & \tabularnewline
  \bottomrule
\end{longtable}

\subsubsection{凤历}

\begin{longtable}{|>{\centering\scriptsize}m{2em}|>{\centering\scriptsize}m{1.3em}|>{\centering}m{8.8em}|}
  % \caption{秦王政}\
  \toprule
  \SimHei \normalsize 年数 & \SimHei \scriptsize 公元 & \SimHei 大事件 \tabularnewline
  % \midrule
  \endfirsthead
  \toprule
  \SimHei \normalsize 年数 & \SimHei \scriptsize 公元 & \SimHei 大事件 \tabularnewline
  \midrule
  \endhead
  \midrule
  元年 & 913 & \tabularnewline
  \bottomrule
\end{longtable}

\subsubsection{乾化}

\begin{longtable}{|>{\centering\scriptsize}m{2em}|>{\centering\scriptsize}m{1.3em}|>{\centering}m{8.8em}|}
  % \caption{秦王政}\
  \toprule
  \SimHei \normalsize 年数 & \SimHei \scriptsize 公元 & \SimHei 大事件 \tabularnewline
  % \midrule
  \endfirsthead
  \toprule
  \SimHei \normalsize 年数 & \SimHei \scriptsize 公元 & \SimHei 大事件 \tabularnewline
  \midrule
  \endhead
  \midrule
  元年 & 913 & \tabularnewline\hline
  二年 & 914 & \tabularnewline\hline
  三年 & 915 & \tabularnewline
  \bottomrule
\end{longtable}

\subsubsection{贞明}

\begin{longtable}{|>{\centering\scriptsize}m{2em}|>{\centering\scriptsize}m{1.3em}|>{\centering}m{8.8em}|}
  % \caption{秦王政}\
  \toprule
  \SimHei \normalsize 年数 & \SimHei \scriptsize 公元 & \SimHei 大事件 \tabularnewline
  % \midrule
  \endfirsthead
  \toprule
  \SimHei \normalsize 年数 & \SimHei \scriptsize 公元 & \SimHei 大事件 \tabularnewline
  \midrule
  \endhead
  \midrule
  元年 & 915 & \tabularnewline\hline
  二年 & 916 & \tabularnewline\hline
  三年 & 917 & \tabularnewline\hline
  四年 & 918 & \tabularnewline\hline
  五年 & 919 & \tabularnewline\hline
  六年 & 920 & \tabularnewline\hline
  七年 & 921 & \tabularnewline
  \bottomrule
\end{longtable}

\subsubsection{龙德}

\begin{longtable}{|>{\centering\scriptsize}m{2em}|>{\centering\scriptsize}m{1.3em}|>{\centering}m{8.8em}|}
  % \caption{秦王政}\
  \toprule
  \SimHei \normalsize 年数 & \SimHei \scriptsize 公元 & \SimHei 大事件 \tabularnewline
  % \midrule
  \endfirsthead
  \toprule
  \SimHei \normalsize 年数 & \SimHei \scriptsize 公元 & \SimHei 大事件 \tabularnewline
  \midrule
  \endhead
  \midrule
  元年 & 921 & \tabularnewline\hline
  二年 & 922 & \tabularnewline\hline
  三年 & 923 & \tabularnewline
  \bottomrule
\end{longtable}

\subsubsection{宝大}

\begin{longtable}{|>{\centering\scriptsize}m{2em}|>{\centering\scriptsize}m{1.3em}|>{\centering}m{8.8em}|}
  % \caption{秦王政}\
  \toprule
  \SimHei \normalsize 年数 & \SimHei \scriptsize 公元 & \SimHei 大事件 \tabularnewline
  % \midrule
  \endfirsthead
  \toprule
  \SimHei \normalsize 年数 & \SimHei \scriptsize 公元 & \SimHei 大事件 \tabularnewline
  \midrule
  \endhead
  \midrule
  元年 & 924 & \tabularnewline\hline
  二年 & 925 & \tabularnewline
  \bottomrule
\end{longtable}

\subsubsection{宝正}

\begin{longtable}{|>{\centering\scriptsize}m{2em}|>{\centering\scriptsize}m{1.3em}|>{\centering}m{8.8em}|}
  % \caption{秦王政}\
  \toprule
  \SimHei \normalsize 年数 & \SimHei \scriptsize 公元 & \SimHei 大事件 \tabularnewline
  % \midrule
  \endfirsthead
  \toprule
  \SimHei \normalsize 年数 & \SimHei \scriptsize 公元 & \SimHei 大事件 \tabularnewline
  \midrule
  \endhead
  \midrule
  元年 & 926 & \tabularnewline\hline
  二年 & 927 & \tabularnewline\hline
  三年 & 928 & \tabularnewline\hline
  四年 & 929 & \tabularnewline\hline
  五年 & 930 & \tabularnewline\hline
  六年 & 931 & \tabularnewline
  \bottomrule
\end{longtable}




%%% Local Variables:
%%% mode: latex
%%% TeX-engine: xetex
%%% TeX-master: "../../Main"
%%% End:

%% -*- coding: utf-8 -*-
%% Time-stamp: <Chen Wang: 2018-07-12 00:00:51>

\subsection{世宗\tiny(932-941)}

\subsubsection{长兴}

\begin{longtable}{|>{\centering\scriptsize}m{2em}|>{\centering\scriptsize}m{1.3em}|>{\centering}m{8.8em}|}
  % \caption{秦王政}\
  \toprule
  \SimHei \normalsize 年数 & \SimHei \scriptsize 公元 & \SimHei 大事件 \tabularnewline
  % \midrule
  \endfirsthead
  \toprule
  \SimHei \normalsize 年数 & \SimHei \scriptsize 公元 & \SimHei 大事件 \tabularnewline
  \midrule
  \endhead
  \midrule
  元年 & 932 & \tabularnewline\hline
  二年 & 933 & \tabularnewline
  \bottomrule
\end{longtable}

\subsubsection{应顺}

\begin{longtable}{|>{\centering\scriptsize}m{2em}|>{\centering\scriptsize}m{1.3em}|>{\centering}m{8.8em}|}
  % \caption{秦王政}\
  \toprule
  \SimHei \normalsize 年数 & \SimHei \scriptsize 公元 & \SimHei 大事件 \tabularnewline
  % \midrule
  \endfirsthead
  \toprule
  \SimHei \normalsize 年数 & \SimHei \scriptsize 公元 & \SimHei 大事件 \tabularnewline
  \midrule
  \endhead
  \midrule
  元年 & 934 & \tabularnewline
  \bottomrule
\end{longtable}

\subsubsection{清泰}

\begin{longtable}{|>{\centering\scriptsize}m{2em}|>{\centering\scriptsize}m{1.3em}|>{\centering}m{8.8em}|}
  % \caption{秦王政}\
  \toprule
  \SimHei \normalsize 年数 & \SimHei \scriptsize 公元 & \SimHei 大事件 \tabularnewline
  % \midrule
  \endfirsthead
  \toprule
  \SimHei \normalsize 年数 & \SimHei \scriptsize 公元 & \SimHei 大事件 \tabularnewline
  \midrule
  \endhead
  \midrule
  元年 & 934 & \tabularnewline\hline
  二年 & 935 & \tabularnewline\hline
  三年 & 936 & \tabularnewline
  \bottomrule
\end{longtable}

\subsubsection{天福}

\begin{longtable}{|>{\centering\scriptsize}m{2em}|>{\centering\scriptsize}m{1.3em}|>{\centering}m{8.8em}|}
  % \caption{秦王政}\
  \toprule
  \SimHei \normalsize 年数 & \SimHei \scriptsize 公元 & \SimHei 大事件 \tabularnewline
  % \midrule
  \endfirsthead
  \toprule
  \SimHei \normalsize 年数 & \SimHei \scriptsize 公元 & \SimHei 大事件 \tabularnewline
  \midrule
  \endhead
  \midrule
  元年 & 936 & \tabularnewline\hline
  二年 & 937 & \tabularnewline\hline
  三年 & 938 & \tabularnewline\hline
  四年 & 939 & \tabularnewline\hline
  五年 & 940 & \tabularnewline\hline
  六年 & 941 & \tabularnewline
  \bottomrule
\end{longtable}



%%% Local Variables:
%%% mode: latex
%%% TeX-engine: xetex
%%% TeX-master: "../../Main"
%%% End:

%% -*- coding: utf-8 -*-
%% Time-stamp: <Chen Wang: 2018-07-12 00:02:09>

\subsection{成宗\tiny(941-947)}

\subsubsection{天福}

\begin{longtable}{|>{\centering\scriptsize}m{2em}|>{\centering\scriptsize}m{1.3em}|>{\centering}m{8.8em}|}
  % \caption{秦王政}\
  \toprule
  \SimHei \normalsize 年数 & \SimHei \scriptsize 公元 & \SimHei 大事件 \tabularnewline
  % \midrule
  \endfirsthead
  \toprule
  \SimHei \normalsize 年数 & \SimHei \scriptsize 公元 & \SimHei 大事件 \tabularnewline
  \midrule
  \endhead
  \midrule
  元年 & 941 & \tabularnewline\hline
  二年 & 942 & \tabularnewline\hline
  三年 & 943 & \tabularnewline\hline
  四年 & 944 & \tabularnewline
  \bottomrule
\end{longtable}

\subsubsection{开运}

\begin{longtable}{|>{\centering\scriptsize}m{2em}|>{\centering\scriptsize}m{1.3em}|>{\centering}m{8.8em}|}
  % \caption{秦王政}\
  \toprule
  \SimHei \normalsize 年数 & \SimHei \scriptsize 公元 & \SimHei 大事件 \tabularnewline
  % \midrule
  \endfirsthead
  \toprule
  \SimHei \normalsize 年数 & \SimHei \scriptsize 公元 & \SimHei 大事件 \tabularnewline
  \midrule
  \endhead
  \midrule
  元年 & 944 & \tabularnewline\hline
  二年 & 945 & \tabularnewline\hline
  三年 & 946 & \tabularnewline
  \bottomrule
\end{longtable}



%%% Local Variables:
%%% mode: latex
%%% TeX-engine: xetex
%%% TeX-master: "../../Main"
%%% End:

%% -*- coding: utf-8 -*-
%% Time-stamp: <Chen Wang: 2018-07-12 00:02:53>

\subsection{忠逊王\tiny(947)}

\subsubsection{天福}

\begin{longtable}{|>{\centering\scriptsize}m{2em}|>{\centering\scriptsize}m{1.3em}|>{\centering}m{8.8em}|}
  % \caption{秦王政}\
  \toprule
  \SimHei \normalsize 年数 & \SimHei \scriptsize 公元 & \SimHei 大事件 \tabularnewline
  % \midrule
  \endfirsthead
  \toprule
  \SimHei \normalsize 年数 & \SimHei \scriptsize 公元 & \SimHei 大事件 \tabularnewline
  \midrule
  \endhead
  \midrule
  元年 & 947 & \tabularnewline
  \bottomrule
\end{longtable}


%%% Local Variables:
%%% mode: latex
%%% TeX-engine: xetex
%%% TeX-master: "../../Main"
%%% End:

%% -*- coding: utf-8 -*-
%% Time-stamp: <Chen Wang: 2018-07-12 00:05:56>

\subsection{钱弘俶\tiny(949-978)}

\subsubsection{乾佑}

\begin{longtable}{|>{\centering\scriptsize}m{2em}|>{\centering\scriptsize}m{1.3em}|>{\centering}m{8.8em}|}
  % \caption{秦王政}\
  \toprule
  \SimHei \normalsize 年数 & \SimHei \scriptsize 公元 & \SimHei 大事件 \tabularnewline
  % \midrule
  \endfirsthead
  \toprule
  \SimHei \normalsize 年数 & \SimHei \scriptsize 公元 & \SimHei 大事件 \tabularnewline
  \midrule
  \endhead
  \midrule
  元年 & 948 & \tabularnewline\hline
  二年 & 949 & \tabularnewline\hline
  三年 & 950 & \tabularnewline
  \bottomrule
\end{longtable}

\subsubsection{广顺}

\begin{longtable}{|>{\centering\scriptsize}m{2em}|>{\centering\scriptsize}m{1.3em}|>{\centering}m{8.8em}|}
  % \caption{秦王政}\
  \toprule
  \SimHei \normalsize 年数 & \SimHei \scriptsize 公元 & \SimHei 大事件 \tabularnewline
  % \midrule
  \endfirsthead
  \toprule
  \SimHei \normalsize 年数 & \SimHei \scriptsize 公元 & \SimHei 大事件 \tabularnewline
  \midrule
  \endhead
  \midrule
  元年 & 951 & \tabularnewline\hline
  二年 & 952 & \tabularnewline\hline
  三年 & 953 & \tabularnewline
  \bottomrule
\end{longtable}

\subsubsection{显德}

\begin{longtable}{|>{\centering\scriptsize}m{2em}|>{\centering\scriptsize}m{1.3em}|>{\centering}m{8.8em}|}
  % \caption{秦王政}\
  \toprule
  \SimHei \normalsize 年数 & \SimHei \scriptsize 公元 & \SimHei 大事件 \tabularnewline
  % \midrule
  \endfirsthead
  \toprule
  \SimHei \normalsize 年数 & \SimHei \scriptsize 公元 & \SimHei 大事件 \tabularnewline
  \midrule
  \endhead
  \midrule
  元年 & 954 & \tabularnewline\hline
  二年 & 955 & \tabularnewline\hline
  三年 & 956 & \tabularnewline\hline
  四年 & 957 & \tabularnewline\hline
  五年 & 958 & \tabularnewline\hline
  六年 & 959 & \tabularnewline\hline
  七年 & 960 & \tabularnewline
  \bottomrule
\end{longtable}

\subsubsection{建隆}

\begin{longtable}{|>{\centering\scriptsize}m{2em}|>{\centering\scriptsize}m{1.3em}|>{\centering}m{8.8em}|}
  % \caption{秦王政}\
  \toprule
  \SimHei \normalsize 年数 & \SimHei \scriptsize 公元 & \SimHei 大事件 \tabularnewline
  % \midrule
  \endfirsthead
  \toprule
  \SimHei \normalsize 年数 & \SimHei \scriptsize 公元 & \SimHei 大事件 \tabularnewline
  \midrule
  \endhead
  \midrule
  元年 & 960 & \tabularnewline\hline
  二年 & 961 & \tabularnewline\hline
  三年 & 962 & \tabularnewline\hline
  四年 & 963 & \tabularnewline
  \bottomrule
\end{longtable}

\subsubsection{乾德}

\begin{longtable}{|>{\centering\scriptsize}m{2em}|>{\centering\scriptsize}m{1.3em}|>{\centering}m{8.8em}|}
  % \caption{秦王政}\
  \toprule
  \SimHei \normalsize 年数 & \SimHei \scriptsize 公元 & \SimHei 大事件 \tabularnewline
  % \midrule
  \endfirsthead
  \toprule
  \SimHei \normalsize 年数 & \SimHei \scriptsize 公元 & \SimHei 大事件 \tabularnewline
  \midrule
  \endhead
  \midrule
  元年 & 963 & \tabularnewline\hline
  二年 & 964 & \tabularnewline\hline
  三年 & 965 & \tabularnewline\hline
  四年 & 966 & \tabularnewline\hline
  五年 & 967 & \tabularnewline\hline
  六年 & 968 & \tabularnewline
  \bottomrule
\end{longtable}

\subsubsection{开宝}

\begin{longtable}{|>{\centering\scriptsize}m{2em}|>{\centering\scriptsize}m{1.3em}|>{\centering}m{8.8em}|}
  % \caption{秦王政}\
  \toprule
  \SimHei \normalsize 年数 & \SimHei \scriptsize 公元 & \SimHei 大事件 \tabularnewline
  % \midrule
  \endfirsthead
  \toprule
  \SimHei \normalsize 年数 & \SimHei \scriptsize 公元 & \SimHei 大事件 \tabularnewline
  \midrule
  \endhead
  \midrule
  元年 & 968 & \tabularnewline\hline
  二年 & 969 & \tabularnewline\hline
  三年 & 970 & \tabularnewline\hline
  四年 & 971 & \tabularnewline\hline
  五年 & 972 & \tabularnewline\hline
  六年 & 973 & \tabularnewline\hline
  七年 & 974 & \tabularnewline\hline
  八年 & 975 & \tabularnewline\hline
  九年 & 976 & \tabularnewline
  \bottomrule
\end{longtable}

\subsubsection{太平兴国}

\begin{longtable}{|>{\centering\scriptsize}m{2em}|>{\centering\scriptsize}m{1.3em}|>{\centering}m{8.8em}|}
  % \caption{秦王政}\
  \toprule
  \SimHei \normalsize 年数 & \SimHei \scriptsize 公元 & \SimHei 大事件 \tabularnewline
  % \midrule
  \endfirsthead
  \toprule
  \SimHei \normalsize 年数 & \SimHei \scriptsize 公元 & \SimHei 大事件 \tabularnewline
  \midrule
  \endhead
  \midrule
  元年 & 976 & \tabularnewline\hline
  二年 & 977 & \tabularnewline\hline
  三年 & 978 & \tabularnewline
  \bottomrule
\end{longtable}


%%% Local Variables:
%%% mode: latex
%%% TeX-engine: xetex
%%% TeX-master: "../../Main"
%%% End:



%%% Local Variables:
%%% mode: latex
%%% TeX-engine: xetex
%%% TeX-master: "../../Main"
%%% End:

%% -*- coding: utf-8 -*-
%% Time-stamp: <Chen Wang: 2018-07-12 00:13:06>


\section{楚\tiny(907-951)}

%% -*- coding: utf-8 -*-
%% Time-stamp: <Chen Wang: 2018-07-12 00:14:23>

\subsection{武穆王\tiny(907-930)}

\subsubsection{天成}

\begin{longtable}{|>{\centering\scriptsize}m{2em}|>{\centering\scriptsize}m{1.3em}|>{\centering}m{8.8em}|}
  % \caption{秦王政}\
  \toprule
  \SimHei \normalsize 年数 & \SimHei \scriptsize 公元 & \SimHei 大事件 \tabularnewline
  % \midrule
  \endfirsthead
  \toprule
  \SimHei \normalsize 年数 & \SimHei \scriptsize 公元 & \SimHei 大事件 \tabularnewline
  \midrule
  \endhead
  \midrule
  元年 & 927 & \tabularnewline\hline
  二年 & 928 & \tabularnewline\hline
  三年 & 929 & \tabularnewline\hline
  四年 & 930 & \tabularnewline
  \bottomrule
\end{longtable}


%%% Local Variables:
%%% mode: latex
%%% TeX-engine: xetex
%%% TeX-master: "../../Main"
%%% End:

%% -*- coding: utf-8 -*-
%% Time-stamp: <Chen Wang: 2018-07-12 00:15:06>

\subsection{马希声\tiny(930-932)}

\subsubsection{长兴}

\begin{longtable}{|>{\centering\scriptsize}m{2em}|>{\centering\scriptsize}m{1.3em}|>{\centering}m{8.8em}|}
  % \caption{秦王政}\
  \toprule
  \SimHei \normalsize 年数 & \SimHei \scriptsize 公元 & \SimHei 大事件 \tabularnewline
  % \midrule
  \endfirsthead
  \toprule
  \SimHei \normalsize 年数 & \SimHei \scriptsize 公元 & \SimHei 大事件 \tabularnewline
  \midrule
  \endhead
  \midrule
  元年 & 930 & \tabularnewline\hline
  二年 & 931 & \tabularnewline\hline
  三年 & 932 & \tabularnewline
  \bottomrule
\end{longtable}


%%% Local Variables:
%%% mode: latex
%%% TeX-engine: xetex
%%% TeX-master: "../../Main"
%%% End:

%% -*- coding: utf-8 -*-
%% Time-stamp: <Chen Wang: 2018-07-12 00:17:12>

\subsection{文昭王\tiny(932-947)}

\subsubsection{长兴}

\begin{longtable}{|>{\centering\scriptsize}m{2em}|>{\centering\scriptsize}m{1.3em}|>{\centering}m{8.8em}|}
  % \caption{秦王政}\
  \toprule
  \SimHei \normalsize 年数 & \SimHei \scriptsize 公元 & \SimHei 大事件 \tabularnewline
  % \midrule
  \endfirsthead
  \toprule
  \SimHei \normalsize 年数 & \SimHei \scriptsize 公元 & \SimHei 大事件 \tabularnewline
  \midrule
  \endhead
  \midrule
  元年 & 932 & \tabularnewline\hline
  二年 & 933 & \tabularnewline
  \bottomrule
\end{longtable}

\subsubsection{应顺}

\begin{longtable}{|>{\centering\scriptsize}m{2em}|>{\centering\scriptsize}m{1.3em}|>{\centering}m{8.8em}|}
  % \caption{秦王政}\
  \toprule
  \SimHei \normalsize 年数 & \SimHei \scriptsize 公元 & \SimHei 大事件 \tabularnewline
  % \midrule
  \endfirsthead
  \toprule
  \SimHei \normalsize 年数 & \SimHei \scriptsize 公元 & \SimHei 大事件 \tabularnewline
  \midrule
  \endhead
  \midrule
  元年 & 934 & \tabularnewline
  \bottomrule
\end{longtable}

\subsubsection{清泰}

\begin{longtable}{|>{\centering\scriptsize}m{2em}|>{\centering\scriptsize}m{1.3em}|>{\centering}m{8.8em}|}
  % \caption{秦王政}\
  \toprule
  \SimHei \normalsize 年数 & \SimHei \scriptsize 公元 & \SimHei 大事件 \tabularnewline
  % \midrule
  \endfirsthead
  \toprule
  \SimHei \normalsize 年数 & \SimHei \scriptsize 公元 & \SimHei 大事件 \tabularnewline
  \midrule
  \endhead
  \midrule
  元年 & 934 & \tabularnewline\hline
  二年 & 935 & \tabularnewline\hline
  三年 & 936 & \tabularnewline
  \bottomrule
\end{longtable}

\subsubsection{天福}

\begin{longtable}{|>{\centering\scriptsize}m{2em}|>{\centering\scriptsize}m{1.3em}|>{\centering}m{8.8em}|}
  % \caption{秦王政}\
  \toprule
  \SimHei \normalsize 年数 & \SimHei \scriptsize 公元 & \SimHei 大事件 \tabularnewline
  % \midrule
  \endfirsthead
  \toprule
  \SimHei \normalsize 年数 & \SimHei \scriptsize 公元 & \SimHei 大事件 \tabularnewline
  \midrule
  \endhead
  \midrule
  元年 & 936 & \tabularnewline\hline
  二年 & 937 & \tabularnewline\hline
  三年 & 938 & \tabularnewline\hline
  四年 & 939 & \tabularnewline\hline
  五年 & 940 & \tabularnewline\hline
  六年 & 941 & \tabularnewline\hline
  七年 & 942 & \tabularnewline\hline
  八年 & 943 & \tabularnewline\hline
  九年 & 944 & \tabularnewline
  \bottomrule
\end{longtable}

\subsubsection{开运}

\begin{longtable}{|>{\centering\scriptsize}m{2em}|>{\centering\scriptsize}m{1.3em}|>{\centering}m{8.8em}|}
  % \caption{秦王政}\
  \toprule
  \SimHei \normalsize 年数 & \SimHei \scriptsize 公元 & \SimHei 大事件 \tabularnewline
  % \midrule
  \endfirsthead
  \toprule
  \SimHei \normalsize 年数 & \SimHei \scriptsize 公元 & \SimHei 大事件 \tabularnewline
  \midrule
  \endhead
  \midrule
  元年 & 944 & \tabularnewline\hline
  二年 & 945 & \tabularnewline\hline
  三年 & 946 & \tabularnewline
  \bottomrule
\end{longtable}


%%% Local Variables:
%%% mode: latex
%%% TeX-engine: xetex
%%% TeX-master: "../../Main"
%%% End:

%% -*- coding: utf-8 -*-
%% Time-stamp: <Chen Wang: 2018-07-12 00:18:19>

\subsection{马希广\tiny(947-950)}

\subsubsection{天福}

\begin{longtable}{|>{\centering\scriptsize}m{2em}|>{\centering\scriptsize}m{1.3em}|>{\centering}m{8.8em}|}
  % \caption{秦王政}\
  \toprule
  \SimHei \normalsize 年数 & \SimHei \scriptsize 公元 & \SimHei 大事件 \tabularnewline
  % \midrule
  \endfirsthead
  \toprule
  \SimHei \normalsize 年数 & \SimHei \scriptsize 公元 & \SimHei 大事件 \tabularnewline
  \midrule
  \endhead
  \midrule
  元年 & 947 & \tabularnewline
  \bottomrule
\end{longtable}

\subsubsection{乾佑}

\begin{longtable}{|>{\centering\scriptsize}m{2em}|>{\centering\scriptsize}m{1.3em}|>{\centering}m{8.8em}|}
  % \caption{秦王政}\
  \toprule
  \SimHei \normalsize 年数 & \SimHei \scriptsize 公元 & \SimHei 大事件 \tabularnewline
  % \midrule
  \endfirsthead
  \toprule
  \SimHei \normalsize 年数 & \SimHei \scriptsize 公元 & \SimHei 大事件 \tabularnewline
  \midrule
  \endhead
  \midrule
  元年 & 948 & \tabularnewline\hline
  二年 & 949 & \tabularnewline\hline
  三年 & 950 & \tabularnewline
  \bottomrule
\end{longtable}


%%% Local Variables:
%%% mode: latex
%%% TeX-engine: xetex
%%% TeX-master: "../../Main"
%%% End:

%% -*- coding: utf-8 -*-
%% Time-stamp: <Chen Wang: 2018-07-12 00:19:14>

\subsection{恭孝王\tiny(950-951)}

\subsubsection{保大}

\begin{longtable}{|>{\centering\scriptsize}m{2em}|>{\centering\scriptsize}m{1.3em}|>{\centering}m{8.8em}|}
  % \caption{秦王政}\
  \toprule
  \SimHei \normalsize 年数 & \SimHei \scriptsize 公元 & \SimHei 大事件 \tabularnewline
  % \midrule
  \endfirsthead
  \toprule
  \SimHei \normalsize 年数 & \SimHei \scriptsize 公元 & \SimHei 大事件 \tabularnewline
  \midrule
  \endhead
  \midrule
  元年 & 950 & \tabularnewline\hline
  二年 & 951 & \tabularnewline
  \bottomrule
\end{longtable}


%%% Local Variables:
%%% mode: latex
%%% TeX-engine: xetex
%%% TeX-master: "../../Main"
%%% End:



%%% Local Variables:
%%% mode: latex
%%% TeX-engine: xetex
%%% TeX-master: "../../Main"
%%% End:

%% -*- coding: utf-8 -*-
%% Time-stamp: <Chen Wang: 2018-07-12 00:29:01>


\section{闽\tiny(909-945)}

%% -*- coding: utf-8 -*-
%% Time-stamp: <Chen Wang: 2018-07-12 00:23:26>

\subsection{太祖\tiny(909-925)}

\subsubsection{开平}

\begin{longtable}{|>{\centering\scriptsize}m{2em}|>{\centering\scriptsize}m{1.3em}|>{\centering}m{8.8em}|}
  % \caption{秦王政}\
  \toprule
  \SimHei \normalsize 年数 & \SimHei \scriptsize 公元 & \SimHei 大事件 \tabularnewline
  % \midrule
  \endfirsthead
  \toprule
  \SimHei \normalsize 年数 & \SimHei \scriptsize 公元 & \SimHei 大事件 \tabularnewline
  \midrule
  \endhead
  \midrule
  元年 & 909 & \tabularnewline\hline
  二年 & 910 & \tabularnewline\hline
  三年 & 911 & \tabularnewline
  \bottomrule
\end{longtable}

\subsubsection{乾化}

\begin{longtable}{|>{\centering\scriptsize}m{2em}|>{\centering\scriptsize}m{1.3em}|>{\centering}m{8.8em}|}
  % \caption{秦王政}\
  \toprule
  \SimHei \normalsize 年数 & \SimHei \scriptsize 公元 & \SimHei 大事件 \tabularnewline
  % \midrule
  \endfirsthead
  \toprule
  \SimHei \normalsize 年数 & \SimHei \scriptsize 公元 & \SimHei 大事件 \tabularnewline
  \midrule
  \endhead
  \midrule
  元年 & 911 & \tabularnewline\hline
  二年 & 912 & \tabularnewline\hline
  三年 & 913 & \tabularnewline\hline
  四年 & 914 & \tabularnewline\hline
  五年 & 915 & \tabularnewline
  \bottomrule
\end{longtable}

\subsubsection{贞明}

\begin{longtable}{|>{\centering\scriptsize}m{2em}|>{\centering\scriptsize}m{1.3em}|>{\centering}m{8.8em}|}
  % \caption{秦王政}\
  \toprule
  \SimHei \normalsize 年数 & \SimHei \scriptsize 公元 & \SimHei 大事件 \tabularnewline
  % \midrule
  \endfirsthead
  \toprule
  \SimHei \normalsize 年数 & \SimHei \scriptsize 公元 & \SimHei 大事件 \tabularnewline
  \midrule
  \endhead
  \midrule
  元年 & 915 & \tabularnewline\hline
  二年 & 916 & \tabularnewline\hline
  三年 & 917 & \tabularnewline\hline
  四年 & 918 & \tabularnewline\hline
  五年 & 919 & \tabularnewline\hline
  六年 & 920 & \tabularnewline\hline
  七年 & 921 & \tabularnewline
  \bottomrule
\end{longtable}

\subsubsection{龙德}

\begin{longtable}{|>{\centering\scriptsize}m{2em}|>{\centering\scriptsize}m{1.3em}|>{\centering}m{8.8em}|}
  % \caption{秦王政}\
  \toprule
  \SimHei \normalsize 年数 & \SimHei \scriptsize 公元 & \SimHei 大事件 \tabularnewline
  % \midrule
  \endfirsthead
  \toprule
  \SimHei \normalsize 年数 & \SimHei \scriptsize 公元 & \SimHei 大事件 \tabularnewline
  \midrule
  \endhead
  \midrule
  元年 & 921 & \tabularnewline\hline
  二年 & 922 & \tabularnewline\hline
  三年 & 923 & \tabularnewline
  \bottomrule
\end{longtable}


\subsubsection{同光}

\begin{longtable}{|>{\centering\scriptsize}m{2em}|>{\centering\scriptsize}m{1.3em}|>{\centering}m{8.8em}|}
  % \caption{秦王政}\
  \toprule
  \SimHei \normalsize 年数 & \SimHei \scriptsize 公元 & \SimHei 大事件 \tabularnewline
  % \midrule
  \endfirsthead
  \toprule
  \SimHei \normalsize 年数 & \SimHei \scriptsize 公元 & \SimHei 大事件 \tabularnewline
  \midrule
  \endhead
  \midrule
  元年 & 923 & \tabularnewline\hline
  二年 & 924 & \tabularnewline\hline
  三年 & 925 & \tabularnewline
  \bottomrule
\end{longtable}



%%% Local Variables:
%%% mode: latex
%%% TeX-engine: xetex
%%% TeX-master: "../../Main"
%%% End:

%% -*- coding: utf-8 -*-
%% Time-stamp: <Chen Wang: 2018-07-12 00:24:13>

\subsection{嗣王\tiny(926)}

\subsubsection{天成}

\begin{longtable}{|>{\centering\scriptsize}m{2em}|>{\centering\scriptsize}m{1.3em}|>{\centering}m{8.8em}|}
  % \caption{秦王政}\
  \toprule
  \SimHei \normalsize 年数 & \SimHei \scriptsize 公元 & \SimHei 大事件 \tabularnewline
  % \midrule
  \endfirsthead
  \toprule
  \SimHei \normalsize 年数 & \SimHei \scriptsize 公元 & \SimHei 大事件 \tabularnewline
  \midrule
  \endhead
  \midrule
  元年 & 926 & \tabularnewline
  \bottomrule
\end{longtable}



%%% Local Variables:
%%% mode: latex
%%% TeX-engine: xetex
%%% TeX-master: "../../Main"
%%% End:

%% -*- coding: utf-8 -*-
%% Time-stamp: <Chen Wang: 2018-07-12 00:25:57>

\subsection{惠宗\tiny(926-935)}

\subsubsection{天成}

\begin{longtable}{|>{\centering\scriptsize}m{2em}|>{\centering\scriptsize}m{1.3em}|>{\centering}m{8.8em}|}
  % \caption{秦王政}\
  \toprule
  \SimHei \normalsize 年数 & \SimHei \scriptsize 公元 & \SimHei 大事件 \tabularnewline
  % \midrule
  \endfirsthead
  \toprule
  \SimHei \normalsize 年数 & \SimHei \scriptsize 公元 & \SimHei 大事件 \tabularnewline
  \midrule
  \endhead
  \midrule
  元年 & 926 & \tabularnewline\hline
  二年 & 927 & \tabularnewline\hline
  三年 & 928 & \tabularnewline\hline
  四年 & 929 & \tabularnewline\hline
  五年 & 930 & \tabularnewline
  \bottomrule
\end{longtable}

\subsubsection{长兴}

\begin{longtable}{|>{\centering\scriptsize}m{2em}|>{\centering\scriptsize}m{1.3em}|>{\centering}m{8.8em}|}
  % \caption{秦王政}\
  \toprule
  \SimHei \normalsize 年数 & \SimHei \scriptsize 公元 & \SimHei 大事件 \tabularnewline
  % \midrule
  \endfirsthead
  \toprule
  \SimHei \normalsize 年数 & \SimHei \scriptsize 公元 & \SimHei 大事件 \tabularnewline
  \midrule
  \endhead
  \midrule
  元年 & 930 & \tabularnewline\hline
  二年 & 931 & \tabularnewline\hline
  三年 & 932 & \tabularnewline
  \bottomrule
\end{longtable}

\subsubsection{龙启}

\begin{longtable}{|>{\centering\scriptsize}m{2em}|>{\centering\scriptsize}m{1.3em}|>{\centering}m{8.8em}|}
  % \caption{秦王政}\
  \toprule
  \SimHei \normalsize 年数 & \SimHei \scriptsize 公元 & \SimHei 大事件 \tabularnewline
  % \midrule
  \endfirsthead
  \toprule
  \SimHei \normalsize 年数 & \SimHei \scriptsize 公元 & \SimHei 大事件 \tabularnewline
  \midrule
  \endhead
  \midrule
  元年 & 933 & \tabularnewline\hline
  二年 & 934 & \tabularnewline
  \bottomrule
\end{longtable}

\subsubsection{永和}

\begin{longtable}{|>{\centering\scriptsize}m{2em}|>{\centering\scriptsize}m{1.3em}|>{\centering}m{8.8em}|}
  % \caption{秦王政}\
  \toprule
  \SimHei \normalsize 年数 & \SimHei \scriptsize 公元 & \SimHei 大事件 \tabularnewline
  % \midrule
  \endfirsthead
  \toprule
  \SimHei \normalsize 年数 & \SimHei \scriptsize 公元 & \SimHei 大事件 \tabularnewline
  \midrule
  \endhead
  \midrule
  元年 & 935 & \tabularnewline\hline
  二年 & 936 & \tabularnewline
  \bottomrule
\end{longtable}


%%% Local Variables:
%%% mode: latex
%%% TeX-engine: xetex
%%% TeX-master: "../../Main"
%%% End:

%% -*- coding: utf-8 -*-
%% Time-stamp: <Chen Wang: 2018-07-12 00:27:57>

\subsection{康宗\tiny(935-939)}

\subsubsection{通文}

\begin{longtable}{|>{\centering\scriptsize}m{2em}|>{\centering\scriptsize}m{1.3em}|>{\centering}m{8.8em}|}
  % \caption{秦王政}\
  \toprule
  \SimHei \normalsize 年数 & \SimHei \scriptsize 公元 & \SimHei 大事件 \tabularnewline
  % \midrule
  \endfirsthead
  \toprule
  \SimHei \normalsize 年数 & \SimHei \scriptsize 公元 & \SimHei 大事件 \tabularnewline
  \midrule
  \endhead
  \midrule
  元年 & 936 & \tabularnewline\hline
  二年 & 937 & \tabularnewline\hline
  三年 & 938 & \tabularnewline\hline
  四年 & 939 & \tabularnewline
  \bottomrule
\end{longtable}


%%% Local Variables:
%%% mode: latex
%%% TeX-engine: xetex
%%% TeX-master: "../../Main"
%%% End:

%% -*- coding: utf-8 -*-
%% Time-stamp: <Chen Wang: 2018-07-12 00:27:04>

\subsection{景宗\tiny(939-944)}

\subsubsection{永隆}

\begin{longtable}{|>{\centering\scriptsize}m{2em}|>{\centering\scriptsize}m{1.3em}|>{\centering}m{8.8em}|}
  % \caption{秦王政}\
  \toprule
  \SimHei \normalsize 年数 & \SimHei \scriptsize 公元 & \SimHei 大事件 \tabularnewline
  % \midrule
  \endfirsthead
  \toprule
  \SimHei \normalsize 年数 & \SimHei \scriptsize 公元 & \SimHei 大事件 \tabularnewline
  \midrule
  \endhead
  \midrule
  元年 & 939 & \tabularnewline\hline
  二年 & 940 & \tabularnewline\hline
  三年 & 941 & \tabularnewline\hline
  四年 & 942 & \tabularnewline\hline
  五年 & 943 & \tabularnewline\hline
  六年 & 944 & \tabularnewline
  \bottomrule
\end{longtable}


%%% Local Variables:
%%% mode: latex
%%% TeX-engine: xetex
%%% TeX-master: "../../Main"
%%% End:

%% -*- coding: utf-8 -*-
%% Time-stamp: <Chen Wang: 2018-07-12 00:28:40>

\subsection{王延政\tiny(943-945)}

\subsubsection{天德}

\begin{longtable}{|>{\centering\scriptsize}m{2em}|>{\centering\scriptsize}m{1.3em}|>{\centering}m{8.8em}|}
  % \caption{秦王政}\
  \toprule
  \SimHei \normalsize 年数 & \SimHei \scriptsize 公元 & \SimHei 大事件 \tabularnewline
  % \midrule
  \endfirsthead
  \toprule
  \SimHei \normalsize 年数 & \SimHei \scriptsize 公元 & \SimHei 大事件 \tabularnewline
  \midrule
  \endhead
  \midrule
  元年 & 943 & \tabularnewline\hline
  二年 & 944 & \tabularnewline\hline
  三年 & 945 & \tabularnewline
  \bottomrule
\end{longtable}


%%% Local Variables:
%%% mode: latex
%%% TeX-engine: xetex
%%% TeX-master: "../../Main"
%%% End:



%%% Local Variables:
%%% mode: latex
%%% TeX-engine: xetex
%%% TeX-master: "../../Main"
%%% End:

%% -*- coding: utf-8 -*-
%% Time-stamp: <Chen Wang: 2018-07-12 01:08:24>


\section{南汉\tiny(917-971)}

%% -*- coding: utf-8 -*-
%% Time-stamp: <Chen Wang: 2018-07-12 00:33:05>

\subsection{高祖\tiny(917-942)}

\subsubsection{乾亨}

\begin{longtable}{|>{\centering\scriptsize}m{2em}|>{\centering\scriptsize}m{1.3em}|>{\centering}m{8.8em}|}
  % \caption{秦王政}\
  \toprule
  \SimHei \normalsize 年数 & \SimHei \scriptsize 公元 & \SimHei 大事件 \tabularnewline
  % \midrule
  \endfirsthead
  \toprule
  \SimHei \normalsize 年数 & \SimHei \scriptsize 公元 & \SimHei 大事件 \tabularnewline
  \midrule
  \endhead
  \midrule
  元年 & 917 & \tabularnewline\hline
  二年 & 918 & \tabularnewline\hline
  三年 & 919 & \tabularnewline\hline
  四年 & 920 & \tabularnewline\hline
  五年 & 921 & \tabularnewline\hline
  六年 & 922 & \tabularnewline\hline
  七年 & 923 & \tabularnewline\hline
  八年 & 924 & \tabularnewline\hline
  九年 & 925 & \tabularnewline
  \bottomrule
\end{longtable}

\subsubsection{白龙}

\begin{longtable}{|>{\centering\scriptsize}m{2em}|>{\centering\scriptsize}m{1.3em}|>{\centering}m{8.8em}|}
  % \caption{秦王政}\
  \toprule
  \SimHei \normalsize 年数 & \SimHei \scriptsize 公元 & \SimHei 大事件 \tabularnewline
  % \midrule
  \endfirsthead
  \toprule
  \SimHei \normalsize 年数 & \SimHei \scriptsize 公元 & \SimHei 大事件 \tabularnewline
  \midrule
  \endhead
  \midrule
  元年 & 925 & \tabularnewline\hline
  二年 & 926 & \tabularnewline\hline
  三年 & 927 & \tabularnewline\hline
  四年 & 928 & \tabularnewline
  \bottomrule
\end{longtable}

\subsubsection{大有}

\begin{longtable}{|>{\centering\scriptsize}m{2em}|>{\centering\scriptsize}m{1.3em}|>{\centering}m{8.8em}|}
  % \caption{秦王政}\
  \toprule
  \SimHei \normalsize 年数 & \SimHei \scriptsize 公元 & \SimHei 大事件 \tabularnewline
  % \midrule
  \endfirsthead
  \toprule
  \SimHei \normalsize 年数 & \SimHei \scriptsize 公元 & \SimHei 大事件 \tabularnewline
  \midrule
  \endhead
  \midrule
  元年 & 928 & \tabularnewline\hline
  二年 & 929 & \tabularnewline\hline
  三年 & 930 & \tabularnewline\hline
  四年 & 931 & \tabularnewline\hline
  五年 & 932 & \tabularnewline\hline
  六年 & 933 & \tabularnewline\hline
  七年 & 934 & \tabularnewline\hline
  八年 & 935 & \tabularnewline\hline
  九年 & 936 & \tabularnewline\hline
  十年 & 937 & \tabularnewline\hline
  十一年 & 938 & \tabularnewline\hline
  十二年 & 939 & \tabularnewline\hline
  十三年 & 940 & \tabularnewline\hline
  十四年 & 941 & \tabularnewline\hline
  十五年 & 942 & \tabularnewline
  \bottomrule
\end{longtable}


%%% Local Variables:
%%% mode: latex
%%% TeX-engine: xetex
%%% TeX-master: "../../Main"
%%% End:

%% -*- coding: utf-8 -*-
%% Time-stamp: <Chen Wang: 2018-07-12 00:33:45>

\subsection{殇帝\tiny(942-943)}

\subsubsection{光天}

\begin{longtable}{|>{\centering\scriptsize}m{2em}|>{\centering\scriptsize}m{1.3em}|>{\centering}m{8.8em}|}
  % \caption{秦王政}\
  \toprule
  \SimHei \normalsize 年数 & \SimHei \scriptsize 公元 & \SimHei 大事件 \tabularnewline
  % \midrule
  \endfirsthead
  \toprule
  \SimHei \normalsize 年数 & \SimHei \scriptsize 公元 & \SimHei 大事件 \tabularnewline
  \midrule
  \endhead
  \midrule
  元年 & 942 & \tabularnewline\hline
  二年 & 943 & \tabularnewline
  \bottomrule
\end{longtable}



%%% Local Variables:
%%% mode: latex
%%% TeX-engine: xetex
%%% TeX-master: "../../Main"
%%% End:

%% -*- coding: utf-8 -*-
%% Time-stamp: <Chen Wang: 2018-07-12 00:35:08>

\subsection{中宗\tiny(943-958)}

\subsubsection{应乾}

\begin{longtable}{|>{\centering\scriptsize}m{2em}|>{\centering\scriptsize}m{1.3em}|>{\centering}m{8.8em}|}
  % \caption{秦王政}\
  \toprule
  \SimHei \normalsize 年数 & \SimHei \scriptsize 公元 & \SimHei 大事件 \tabularnewline
  % \midrule
  \endfirsthead
  \toprule
  \SimHei \normalsize 年数 & \SimHei \scriptsize 公元 & \SimHei 大事件 \tabularnewline
  \midrule
  \endhead
  \midrule
  元年 & 943 & \tabularnewline
  \bottomrule
\end{longtable}

\subsubsection{乾和}

\begin{longtable}{|>{\centering\scriptsize}m{2em}|>{\centering\scriptsize}m{1.3em}|>{\centering}m{8.8em}|}
  % \caption{秦王政}\
  \toprule
  \SimHei \normalsize 年数 & \SimHei \scriptsize 公元 & \SimHei 大事件 \tabularnewline
  % \midrule
  \endfirsthead
  \toprule
  \SimHei \normalsize 年数 & \SimHei \scriptsize 公元 & \SimHei 大事件 \tabularnewline
  \midrule
  \endhead
  \midrule
  元年 & 943 & \tabularnewline\hline
  二年 & 944 & \tabularnewline\hline
  三年 & 945 & \tabularnewline\hline
  四年 & 946 & \tabularnewline\hline
  五年 & 947 & \tabularnewline\hline
  六年 & 948 & \tabularnewline\hline
  七年 & 949 & \tabularnewline\hline
  八年 & 950 & \tabularnewline\hline
  九年 & 951 & \tabularnewline\hline
  十年 & 952 & \tabularnewline\hline
  十一年 & 953 & \tabularnewline\hline
  十二年 & 954 & \tabularnewline\hline
  十三年 & 955 & \tabularnewline\hline
  十四年 & 956 & \tabularnewline\hline
  十五年 & 957 & \tabularnewline\hline
  十六年 & 958 & \tabularnewline
  \bottomrule
\end{longtable}



%%% Local Variables:
%%% mode: latex
%%% TeX-engine: xetex
%%% TeX-master: "../../Main"
%%% End:

%% -*- coding: utf-8 -*-
%% Time-stamp: <Chen Wang: 2018-07-12 00:36:14>

\subsection{刘鋹\tiny(958-971)}

\subsubsection{大宝}

\begin{longtable}{|>{\centering\scriptsize}m{2em}|>{\centering\scriptsize}m{1.3em}|>{\centering}m{8.8em}|}
  % \caption{秦王政}\
  \toprule
  \SimHei \normalsize 年数 & \SimHei \scriptsize 公元 & \SimHei 大事件 \tabularnewline
  % \midrule
  \endfirsthead
  \toprule
  \SimHei \normalsize 年数 & \SimHei \scriptsize 公元 & \SimHei 大事件 \tabularnewline
  \midrule
  \endhead
  \midrule
  元年 & 958 & \tabularnewline\hline
  二年 & 959 & \tabularnewline\hline
  三年 & 960 & \tabularnewline\hline
  四年 & 961 & \tabularnewline\hline
  五年 & 962 & \tabularnewline\hline
  六年 & 963 & \tabularnewline\hline
  七年 & 964 & \tabularnewline\hline
  八年 & 965 & \tabularnewline\hline
  九年 & 966 & \tabularnewline\hline
  十年 & 967 & \tabularnewline\hline
  十一年 & 968 & \tabularnewline\hline
  十二年 & 969 & \tabularnewline\hline
  十三年 & 970 & \tabularnewline\hline
  十四年 & 971 & \tabularnewline
  \bottomrule
\end{longtable}



%%% Local Variables:
%%% mode: latex
%%% TeX-engine: xetex
%%% TeX-master: "../../Main"
%%% End:



%%% Local Variables:
%%% mode: latex
%%% TeX-engine: xetex
%%% TeX-master: "../../Main"
%%% End:

%% -*- coding: utf-8 -*-
%% Time-stamp: <Chen Wang: 2018-07-12 00:37:46>


\section{前蜀\tiny(903-925)}

%% -*- coding: utf-8 -*-
%% Time-stamp: <Chen Wang: 2018-07-12 00:39:27>

\subsection{高祖\tiny(907-918)}

\subsubsection{天复}

\begin{longtable}{|>{\centering\scriptsize}m{2em}|>{\centering\scriptsize}m{1.3em}|>{\centering}m{8.8em}|}
  % \caption{秦王政}\
  \toprule
  \SimHei \normalsize 年数 & \SimHei \scriptsize 公元 & \SimHei 大事件 \tabularnewline
  % \midrule
  \endfirsthead
  \toprule
  \SimHei \normalsize 年数 & \SimHei \scriptsize 公元 & \SimHei 大事件 \tabularnewline
  \midrule
  \endhead
  \midrule
  元年 & 907 & \tabularnewline
  \bottomrule
\end{longtable}

\subsubsection{武成}

\begin{longtable}{|>{\centering\scriptsize}m{2em}|>{\centering\scriptsize}m{1.3em}|>{\centering}m{8.8em}|}
  % \caption{秦王政}\
  \toprule
  \SimHei \normalsize 年数 & \SimHei \scriptsize 公元 & \SimHei 大事件 \tabularnewline
  % \midrule
  \endfirsthead
  \toprule
  \SimHei \normalsize 年数 & \SimHei \scriptsize 公元 & \SimHei 大事件 \tabularnewline
  \midrule
  \endhead
  \midrule
  元年 & 908 & \tabularnewline\hline
  二年 & 909 & \tabularnewline\hline
  三年 & 910 & \tabularnewline
  \bottomrule
\end{longtable}

\subsubsection{通正}

\begin{longtable}{|>{\centering\scriptsize}m{2em}|>{\centering\scriptsize}m{1.3em}|>{\centering}m{8.8em}|}
  % \caption{秦王政}\
  \toprule
  \SimHei \normalsize 年数 & \SimHei \scriptsize 公元 & \SimHei 大事件 \tabularnewline
  % \midrule
  \endfirsthead
  \toprule
  \SimHei \normalsize 年数 & \SimHei \scriptsize 公元 & \SimHei 大事件 \tabularnewline
  \midrule
  \endhead
  \midrule
  元年 & 916 & \tabularnewline
  \bottomrule
\end{longtable}


\subsubsection{天汉}

\begin{longtable}{|>{\centering\scriptsize}m{2em}|>{\centering\scriptsize}m{1.3em}|>{\centering}m{8.8em}|}
  % \caption{秦王政}\
  \toprule
  \SimHei \normalsize 年数 & \SimHei \scriptsize 公元 & \SimHei 大事件 \tabularnewline
  % \midrule
  \endfirsthead
  \toprule
  \SimHei \normalsize 年数 & \SimHei \scriptsize 公元 & \SimHei 大事件 \tabularnewline
  \midrule
  \endhead
  \midrule
  元年 & 917 & \tabularnewline
  \bottomrule
\end{longtable}


\subsubsection{光天}

\begin{longtable}{|>{\centering\scriptsize}m{2em}|>{\centering\scriptsize}m{1.3em}|>{\centering}m{8.8em}|}
  % \caption{秦王政}\
  \toprule
  \SimHei \normalsize 年数 & \SimHei \scriptsize 公元 & \SimHei 大事件 \tabularnewline
  % \midrule
  \endfirsthead
  \toprule
  \SimHei \normalsize 年数 & \SimHei \scriptsize 公元 & \SimHei 大事件 \tabularnewline
  \midrule
  \endhead
  \midrule
  元年 & 918 & \tabularnewline
  \bottomrule
\end{longtable}


%%% Local Variables:
%%% mode: latex
%%% TeX-engine: xetex
%%% TeX-master: "../../Main"
%%% End:

%% -*- coding: utf-8 -*-
%% Time-stamp: <Chen Wang: 2018-07-12 00:40:38>

\subsection{王衍\tiny(918-925)}

\subsubsection{乾德}

\begin{longtable}{|>{\centering\scriptsize}m{2em}|>{\centering\scriptsize}m{1.3em}|>{\centering}m{8.8em}|}
  % \caption{秦王政}\
  \toprule
  \SimHei \normalsize 年数 & \SimHei \scriptsize 公元 & \SimHei 大事件 \tabularnewline
  % \midrule
  \endfirsthead
  \toprule
  \SimHei \normalsize 年数 & \SimHei \scriptsize 公元 & \SimHei 大事件 \tabularnewline
  \midrule
  \endhead
  \midrule
  元年 & 919 & \tabularnewline\hline
  二年 & 920 & \tabularnewline\hline
  三年 & 921 & \tabularnewline\hline
  四年 & 922 & \tabularnewline\hline
  五年 & 923 & \tabularnewline\hline
  六年 & 924 & \tabularnewline
  \bottomrule
\end{longtable}

\subsubsection{咸康}

\begin{longtable}{|>{\centering\scriptsize}m{2em}|>{\centering\scriptsize}m{1.3em}|>{\centering}m{8.8em}|}
  % \caption{秦王政}\
  \toprule
  \SimHei \normalsize 年数 & \SimHei \scriptsize 公元 & \SimHei 大事件 \tabularnewline
  % \midrule
  \endfirsthead
  \toprule
  \SimHei \normalsize 年数 & \SimHei \scriptsize 公元 & \SimHei 大事件 \tabularnewline
  \midrule
  \endhead
  \midrule
  元年 & 925 & \tabularnewline
  \bottomrule
\end{longtable}


%%% Local Variables:
%%% mode: latex
%%% TeX-engine: xetex
%%% TeX-master: "../../Main"
%%% End:



%%% Local Variables:
%%% mode: latex
%%% TeX-engine: xetex
%%% TeX-master: "../../Main"
%%% End:

%% -*- coding: utf-8 -*-
%% Time-stamp: <Chen Wang: 2018-07-12 00:41:41>


\section{后蜀\tiny(934-965)}

%% -*- coding: utf-8 -*-
%% Time-stamp: <Chen Wang: 2018-07-12 00:42:57>

\subsection{高祖\tiny(934-937)}

\subsubsection{明德}

\begin{longtable}{|>{\centering\scriptsize}m{2em}|>{\centering\scriptsize}m{1.3em}|>{\centering}m{8.8em}|}
  % \caption{秦王政}\
  \toprule
  \SimHei \normalsize 年数 & \SimHei \scriptsize 公元 & \SimHei 大事件 \tabularnewline
  % \midrule
  \endfirsthead
  \toprule
  \SimHei \normalsize 年数 & \SimHei \scriptsize 公元 & \SimHei 大事件 \tabularnewline
  \midrule
  \endhead
  \midrule
  元年 & 934 & \tabularnewline\hline
  二年 & 935 & \tabularnewline\hline
  三年 & 936 & \tabularnewline\hline
  四年 & 937 & \tabularnewline
  \bottomrule
\end{longtable}



%%% Local Variables:
%%% mode: latex
%%% TeX-engine: xetex
%%% TeX-master: "../../Main"
%%% End:

%% -*- coding: utf-8 -*-
%% Time-stamp: <Chen Wang: 2018-07-12 00:44:18>

\subsection{孟昶\tiny(938-965)}

\subsubsection{广政}

\begin{longtable}{|>{\centering\scriptsize}m{2em}|>{\centering\scriptsize}m{1.3em}|>{\centering}m{8.8em}|}
  % \caption{秦王政}\
  \toprule
  \SimHei \normalsize 年数 & \SimHei \scriptsize 公元 & \SimHei 大事件 \tabularnewline
  % \midrule
  \endfirsthead
  \toprule
  \SimHei \normalsize 年数 & \SimHei \scriptsize 公元 & \SimHei 大事件 \tabularnewline
  \midrule
  \endhead
  \midrule
  元年 & 938 & \tabularnewline\hline
  二年 & 939 & \tabularnewline\hline
  三年 & 940 & \tabularnewline\hline
  四年 & 941 & \tabularnewline\hline
  五年 & 942 & \tabularnewline\hline
  六年 & 943 & \tabularnewline\hline
  七年 & 944 & \tabularnewline\hline
  八年 & 945 & \tabularnewline\hline
  九年 & 946 & \tabularnewline\hline
  十年 & 947 & \tabularnewline\hline
  十一年 & 948 & \tabularnewline\hline
  十二年 & 949 & \tabularnewline\hline
  十三年 & 950 & \tabularnewline\hline
  十四年 & 951 & \tabularnewline\hline
  十五年 & 952 & \tabularnewline\hline
  十六年 & 953 & \tabularnewline\hline
  十七年 & 954 & \tabularnewline\hline
  十八年 & 955 & \tabularnewline\hline
  十九年 & 956 & \tabularnewline\hline
  二十年 & 957 & \tabularnewline\hline
  二一年 & 958 & \tabularnewline\hline
  二二年 & 959 & \tabularnewline\hline
  二三年 & 960 & \tabularnewline\hline
  二四年 & 961 & \tabularnewline\hline
  二五年 & 962 & \tabularnewline\hline
  二六年 & 963 & \tabularnewline\hline
  二七年 & 964 & \tabularnewline\hline
  二八年 & 965 & \tabularnewline
  \bottomrule
\end{longtable}



%%% Local Variables:
%%% mode: latex
%%% TeX-engine: xetex
%%% TeX-master: "../../Main"
%%% End:




%%% Local Variables:
%%% mode: latex
%%% TeX-engine: xetex
%%% TeX-master: "../../Main"
%%% End:

%% -*- coding: utf-8 -*-
%% Time-stamp: <Chen Wang: 2018-07-12 01:08:49>


\section{荆南\tiny(924-963)}

%% -*- coding: utf-8 -*-
%% Time-stamp: <Chen Wang: 2018-07-12 00:56:13>

\subsection{武信王\tiny(924-929)}

\subsubsection{同光}

\begin{longtable}{|>{\centering\scriptsize}m{2em}|>{\centering\scriptsize}m{1.3em}|>{\centering}m{8.8em}|}
  % \caption{秦王政}\
  \toprule
  \SimHei \normalsize 年数 & \SimHei \scriptsize 公元 & \SimHei 大事件 \tabularnewline
  % \midrule
  \endfirsthead
  \toprule
  \SimHei \normalsize 年数 & \SimHei \scriptsize 公元 & \SimHei 大事件 \tabularnewline
  \midrule
  \endhead
  \midrule
  元年 & 924 & \tabularnewline\hline
  二年 & 925 & \tabularnewline\hline
  三年 & 926 & \tabularnewline
  \bottomrule
\end{longtable}

\subsubsection{天成}

\begin{longtable}{|>{\centering\scriptsize}m{2em}|>{\centering\scriptsize}m{1.3em}|>{\centering}m{8.8em}|}
  % \caption{秦王政}\
  \toprule
  \SimHei \normalsize 年数 & \SimHei \scriptsize 公元 & \SimHei 大事件 \tabularnewline
  % \midrule
  \endfirsthead
  \toprule
  \SimHei \normalsize 年数 & \SimHei \scriptsize 公元 & \SimHei 大事件 \tabularnewline
  \midrule
  \endhead
  \midrule
  元年 & 926 & \tabularnewline\hline
  二年 & 927 & \tabularnewline\hline
  三年 & 928 & \tabularnewline
  \bottomrule
\end{longtable}

\subsubsection{乾贞}

\begin{longtable}{|>{\centering\scriptsize}m{2em}|>{\centering\scriptsize}m{1.3em}|>{\centering}m{8.8em}|}
  % \caption{秦王政}\
  \toprule
  \SimHei \normalsize 年数 & \SimHei \scriptsize 公元 & \SimHei 大事件 \tabularnewline
  % \midrule
  \endfirsthead
  \toprule
  \SimHei \normalsize 年数 & \SimHei \scriptsize 公元 & \SimHei 大事件 \tabularnewline
  \midrule
  \endhead
  \midrule
  元年 & 928 & \tabularnewline
  \bottomrule
\end{longtable}



%%% Local Variables:
%%% mode: latex
%%% TeX-engine: xetex
%%% TeX-master: "../../Main"
%%% End:

%% -*- coding: utf-8 -*-
%% Time-stamp: <Chen Wang: 2018-07-12 00:53:28>

\subsection{文献王\tiny(928-948)}

\subsubsection{乾贞}

\begin{longtable}{|>{\centering\scriptsize}m{2em}|>{\centering\scriptsize}m{1.3em}|>{\centering}m{8.8em}|}
  % \caption{秦王政}\
  \toprule
  \SimHei \normalsize 年数 & \SimHei \scriptsize 公元 & \SimHei 大事件 \tabularnewline
  % \midrule
  \endfirsthead
  \toprule
  \SimHei \normalsize 年数 & \SimHei \scriptsize 公元 & \SimHei 大事件 \tabularnewline
  \midrule
  \endhead
  \midrule
  元年 & 929 & \tabularnewline
  \bottomrule
\end{longtable}

\subsubsection{天成}

\begin{longtable}{|>{\centering\scriptsize}m{2em}|>{\centering\scriptsize}m{1.3em}|>{\centering}m{8.8em}|}
  % \caption{秦王政}\
  \toprule
  \SimHei \normalsize 年数 & \SimHei \scriptsize 公元 & \SimHei 大事件 \tabularnewline
  % \midrule
  \endfirsthead
  \toprule
  \SimHei \normalsize 年数 & \SimHei \scriptsize 公元 & \SimHei 大事件 \tabularnewline
  \midrule
  \endhead
  \midrule
  元年 & 929 & \tabularnewline\hline
  二年 & 930 & \tabularnewline
  \bottomrule
\end{longtable}

\subsubsection{长兴}

\begin{longtable}{|>{\centering\scriptsize}m{2em}|>{\centering\scriptsize}m{1.3em}|>{\centering}m{8.8em}|}
  % \caption{秦王政}\
  \toprule
  \SimHei \normalsize 年数 & \SimHei \scriptsize 公元 & \SimHei 大事件 \tabularnewline
  % \midrule
  \endfirsthead
  \toprule
  \SimHei \normalsize 年数 & \SimHei \scriptsize 公元 & \SimHei 大事件 \tabularnewline
  \midrule
  \endhead
  \midrule
  元年 & 930 & \tabularnewline\hline
  二年 & 931 & \tabularnewline\hline
  三年 & 932 & \tabularnewline\hline
  四年 & 933 & \tabularnewline
  \bottomrule
\end{longtable}

\subsubsection{应顺}

\begin{longtable}{|>{\centering\scriptsize}m{2em}|>{\centering\scriptsize}m{1.3em}|>{\centering}m{8.8em}|}
  % \caption{秦王政}\
  \toprule
  \SimHei \normalsize 年数 & \SimHei \scriptsize 公元 & \SimHei 大事件 \tabularnewline
  % \midrule
  \endfirsthead
  \toprule
  \SimHei \normalsize 年数 & \SimHei \scriptsize 公元 & \SimHei 大事件 \tabularnewline
  \midrule
  \endhead
  \midrule
  元年 & 934 & \tabularnewline
  \bottomrule
\end{longtable}

\subsubsection{清泰}

\begin{longtable}{|>{\centering\scriptsize}m{2em}|>{\centering\scriptsize}m{1.3em}|>{\centering}m{8.8em}|}
  % \caption{秦王政}\
  \toprule
  \SimHei \normalsize 年数 & \SimHei \scriptsize 公元 & \SimHei 大事件 \tabularnewline
  % \midrule
  \endfirsthead
  \toprule
  \SimHei \normalsize 年数 & \SimHei \scriptsize 公元 & \SimHei 大事件 \tabularnewline
  \midrule
  \endhead
  \midrule
  元年 & 934 & \tabularnewline\hline
  二年 & 935 & \tabularnewline\hline
  三年 & 936 & \tabularnewline
  \bottomrule
\end{longtable}

\subsubsection{天福}

\begin{longtable}{|>{\centering\scriptsize}m{2em}|>{\centering\scriptsize}m{1.3em}|>{\centering}m{8.8em}|}
  % \caption{秦王政}\
  \toprule
  \SimHei \normalsize 年数 & \SimHei \scriptsize 公元 & \SimHei 大事件 \tabularnewline
  % \midrule
  \endfirsthead
  \toprule
  \SimHei \normalsize 年数 & \SimHei \scriptsize 公元 & \SimHei 大事件 \tabularnewline
  \midrule
  \endhead
  \midrule
  元年 & 936 & \tabularnewline\hline
  二年 & 937 & \tabularnewline\hline
  三年 & 938 & \tabularnewline\hline
  四年 & 939 & \tabularnewline\hline
  五年 & 940 & \tabularnewline\hline
  六年 & 941 & \tabularnewline\hline
  七年 & 942 & \tabularnewline\hline
  八年 & 943 & \tabularnewline\hline
  九年 & 944 & \tabularnewline
  \bottomrule
\end{longtable}

\subsubsection{开运}

\begin{longtable}{|>{\centering\scriptsize}m{2em}|>{\centering\scriptsize}m{1.3em}|>{\centering}m{8.8em}|}
  % \caption{秦王政}\
  \toprule
  \SimHei \normalsize 年数 & \SimHei \scriptsize 公元 & \SimHei 大事件 \tabularnewline
  % \midrule
  \endfirsthead
  \toprule
  \SimHei \normalsize 年数 & \SimHei \scriptsize 公元 & \SimHei 大事件 \tabularnewline
  \midrule
  \endhead
  \midrule
  元年 & 944 & \tabularnewline\hline
  二年 & 945 & \tabularnewline\hline
  三年 & 946 & \tabularnewline
  \bottomrule
\end{longtable}

\subsubsection{天复}

\begin{longtable}{|>{\centering\scriptsize}m{2em}|>{\centering\scriptsize}m{1.3em}|>{\centering}m{8.8em}|}
  % \caption{秦王政}\
  \toprule
  \SimHei \normalsize 年数 & \SimHei \scriptsize 公元 & \SimHei 大事件 \tabularnewline
  % \midrule
  \endfirsthead
  \toprule
  \SimHei \normalsize 年数 & \SimHei \scriptsize 公元 & \SimHei 大事件 \tabularnewline
  \midrule
  \endhead
  \midrule
  元年 & 947 & \tabularnewline
  \bottomrule
\end{longtable}

\subsubsection{乾佑}

\begin{longtable}{|>{\centering\scriptsize}m{2em}|>{\centering\scriptsize}m{1.3em}|>{\centering}m{8.8em}|}
  % \caption{秦王政}\
  \toprule
  \SimHei \normalsize 年数 & \SimHei \scriptsize 公元 & \SimHei 大事件 \tabularnewline
  % \midrule
  \endfirsthead
  \toprule
  \SimHei \normalsize 年数 & \SimHei \scriptsize 公元 & \SimHei 大事件 \tabularnewline
  \midrule
  \endhead
  \midrule
  元年 & 948 & \tabularnewline
  \bottomrule
\end{longtable}



%%% Local Variables:
%%% mode: latex
%%% TeX-engine: xetex
%%% TeX-master: "../../Main"
%%% End:

%% -*- coding: utf-8 -*-
%% Time-stamp: <Chen Wang: 2018-07-12 00:57:59>

\subsection{贞懿王\tiny(948-960)}

\subsubsection{乾佑}

\begin{longtable}{|>{\centering\scriptsize}m{2em}|>{\centering\scriptsize}m{1.3em}|>{\centering}m{8.8em}|}
  % \caption{秦王政}\
  \toprule
  \SimHei \normalsize 年数 & \SimHei \scriptsize 公元 & \SimHei 大事件 \tabularnewline
  % \midrule
  \endfirsthead
  \toprule
  \SimHei \normalsize 年数 & \SimHei \scriptsize 公元 & \SimHei 大事件 \tabularnewline
  \midrule
  \endhead
  \midrule
  元年 & 948 & \tabularnewline\hline
  二年 & 949 & \tabularnewline\hline
  三年 & 950 & \tabularnewline
  \bottomrule
\end{longtable}

\subsubsection{广顺}

\begin{longtable}{|>{\centering\scriptsize}m{2em}|>{\centering\scriptsize}m{1.3em}|>{\centering}m{8.8em}|}
  % \caption{秦王政}\
  \toprule
  \SimHei \normalsize 年数 & \SimHei \scriptsize 公元 & \SimHei 大事件 \tabularnewline
  % \midrule
  \endfirsthead
  \toprule
  \SimHei \normalsize 年数 & \SimHei \scriptsize 公元 & \SimHei 大事件 \tabularnewline
  \midrule
  \endhead
  \midrule
  元年 & 951 & \tabularnewline\hline
  二年 & 952 & \tabularnewline\hline
  三年 & 953 & \tabularnewline
  \bottomrule
\end{longtable}

\subsubsection{显德}

\begin{longtable}{|>{\centering\scriptsize}m{2em}|>{\centering\scriptsize}m{1.3em}|>{\centering}m{8.8em}|}
  % \caption{秦王政}\
  \toprule
  \SimHei \normalsize 年数 & \SimHei \scriptsize 公元 & \SimHei 大事件 \tabularnewline
  % \midrule
  \endfirsthead
  \toprule
  \SimHei \normalsize 年数 & \SimHei \scriptsize 公元 & \SimHei 大事件 \tabularnewline
  \midrule
  \endhead
  \midrule
  元年 & 954 & \tabularnewline\hline
  二年 & 955 & \tabularnewline\hline
  三年 & 956 & \tabularnewline\hline
  四年 & 957 & \tabularnewline\hline
  五年 & 958 & \tabularnewline\hline
  六年 & 959 & \tabularnewline\hline
  七年 & 960 & \tabularnewline
  \bottomrule
\end{longtable}



%%% Local Variables:
%%% mode: latex
%%% TeX-engine: xetex
%%% TeX-master: "../../Main"
%%% End:

%% -*- coding: utf-8 -*-
%% Time-stamp: <Chen Wang: 2018-07-12 00:59:01>

\subsection{高保勗\tiny(960-962)}

\subsubsection{建隆}

\begin{longtable}{|>{\centering\scriptsize}m{2em}|>{\centering\scriptsize}m{1.3em}|>{\centering}m{8.8em}|}
  % \caption{秦王政}\
  \toprule
  \SimHei \normalsize 年数 & \SimHei \scriptsize 公元 & \SimHei 大事件 \tabularnewline
  % \midrule
  \endfirsthead
  \toprule
  \SimHei \normalsize 年数 & \SimHei \scriptsize 公元 & \SimHei 大事件 \tabularnewline
  \midrule
  \endhead
  \midrule
  元年 & 960 & \tabularnewline\hline
  二年 & 961 & \tabularnewline\hline
  三年 & 962 & \tabularnewline
  \bottomrule
\end{longtable}




%%% Local Variables:
%%% mode: latex
%%% TeX-engine: xetex
%%% TeX-master: "../../Main"
%%% End:

%% -*- coding: utf-8 -*-
%% Time-stamp: <Chen Wang: 2018-07-12 01:00:16>

\subsection{高继冲\tiny(962-963)}

\subsubsection{建隆}

\begin{longtable}{|>{\centering\scriptsize}m{2em}|>{\centering\scriptsize}m{1.3em}|>{\centering}m{8.8em}|}
  % \caption{秦王政}\
  \toprule
  \SimHei \normalsize 年数 & \SimHei \scriptsize 公元 & \SimHei 大事件 \tabularnewline
  % \midrule
  \endfirsthead
  \toprule
  \SimHei \normalsize 年数 & \SimHei \scriptsize 公元 & \SimHei 大事件 \tabularnewline
  \midrule
  \endhead
  \midrule
  元年 & 962 & \tabularnewline\hline
  二年 & 963 & \tabularnewline
  \bottomrule
\end{longtable}




%%% Local Variables:
%%% mode: latex
%%% TeX-engine: xetex
%%% TeX-master: "../../Main"
%%% End:



%%% Local Variables:
%%% mode: latex
%%% TeX-engine: xetex
%%% TeX-master: "../../Main"
%%% End:

%% -*- coding: utf-8 -*-
%% Time-stamp: <Chen Wang: 2018-07-12 01:02:27>


\section{北汉\tiny(951-979)}

%% -*- coding: utf-8 -*-
%% Time-stamp: <Chen Wang: 2018-07-12 01:03:10>

\subsection{世祖\tiny(951-954)}

\subsubsection{乾佑}

\begin{longtable}{|>{\centering\scriptsize}m{2em}|>{\centering\scriptsize}m{1.3em}|>{\centering}m{8.8em}|}
  % \caption{秦王政}\
  \toprule
  \SimHei \normalsize 年数 & \SimHei \scriptsize 公元 & \SimHei 大事件 \tabularnewline
  % \midrule
  \endfirsthead
  \toprule
  \SimHei \normalsize 年数 & \SimHei \scriptsize 公元 & \SimHei 大事件 \tabularnewline
  \midrule
  \endhead
  \midrule
  元年 & 951 & \tabularnewline\hline
  二年 & 952 & \tabularnewline\hline
  三年 & 953 & \tabularnewline\hline
  四年 & 954 & \tabularnewline
  \bottomrule
\end{longtable}


%%% Local Variables:
%%% mode: latex
%%% TeX-engine: xetex
%%% TeX-master: "../../Main"
%%% End:

%% -*- coding: utf-8 -*-
%% Time-stamp: <Chen Wang: 2018-07-12 01:05:48>

\subsection{睿宗\tiny(954-968)}

\subsubsection{乾佑}

\begin{longtable}{|>{\centering\scriptsize}m{2em}|>{\centering\scriptsize}m{1.3em}|>{\centering}m{8.8em}|}
  % \caption{秦王政}\
  \toprule
  \SimHei \normalsize 年数 & \SimHei \scriptsize 公元 & \SimHei 大事件 \tabularnewline
  % \midrule
  \endfirsthead
  \toprule
  \SimHei \normalsize 年数 & \SimHei \scriptsize 公元 & \SimHei 大事件 \tabularnewline
  \midrule
  \endhead
  \midrule
  元年 & 954 & \tabularnewline\hline
  二年 & 955 & \tabularnewline\hline
  三年 & 956 & \tabularnewline
  \bottomrule
\end{longtable}

\subsubsection{天会}

\begin{longtable}{|>{\centering\scriptsize}m{2em}|>{\centering\scriptsize}m{1.3em}|>{\centering}m{8.8em}|}
  % \caption{秦王政}\
  \toprule
  \SimHei \normalsize 年数 & \SimHei \scriptsize 公元 & \SimHei 大事件 \tabularnewline
  % \midrule
  \endfirsthead
  \toprule
  \SimHei \normalsize 年数 & \SimHei \scriptsize 公元 & \SimHei 大事件 \tabularnewline
  \midrule
  \endhead
  \midrule
  元年 & 957 & \tabularnewline\hline
  二年 & 958 & \tabularnewline\hline
  三年 & 959 & \tabularnewline\hline
  四年 & 960 & \tabularnewline\hline
  五年 & 961 & \tabularnewline\hline
  六年 & 962 & \tabularnewline\hline
  七年 & 963 & \tabularnewline\hline
  八年 & 964 & \tabularnewline\hline
  九年 & 965 & \tabularnewline\hline
  十年 & 966 & \tabularnewline\hline
  十一年 & 967 & \tabularnewline\hline
  十二年 & 968 & \tabularnewline\hline
  \bottomrule
\end{longtable}


%%% Local Variables:
%%% mode: latex
%%% TeX-engine: xetex
%%% TeX-master: "../../Main"
%%% End:

%% -*- coding: utf-8 -*-
%% Time-stamp: <Chen Wang: 2018-07-12 01:05:38>

\subsection{刘继恩\tiny(968)}

\subsubsection{天会}

\begin{longtable}{|>{\centering\scriptsize}m{2em}|>{\centering\scriptsize}m{1.3em}|>{\centering}m{8.8em}|}
  % \caption{秦王政}\
  \toprule
  \SimHei \normalsize 年数 & \SimHei \scriptsize 公元 & \SimHei 大事件 \tabularnewline
  % \midrule
  \endfirsthead
  \toprule
  \SimHei \normalsize 年数 & \SimHei \scriptsize 公元 & \SimHei 大事件 \tabularnewline
  \midrule
  \endhead
  \midrule
  元年 & 968 & \tabularnewline
  \bottomrule
\end{longtable}


%%% Local Variables:
%%% mode: latex
%%% TeX-engine: xetex
%%% TeX-master: "../../Main"
%%% End:

%% -*- coding: utf-8 -*-
%% Time-stamp: <Chen Wang: 2018-07-12 01:07:14>

\subsection{英武帝\tiny(968-979)}

\subsubsection{天会}

\begin{longtable}{|>{\centering\scriptsize}m{2em}|>{\centering\scriptsize}m{1.3em}|>{\centering}m{8.8em}|}
  % \caption{秦王政}\
  \toprule
  \SimHei \normalsize 年数 & \SimHei \scriptsize 公元 & \SimHei 大事件 \tabularnewline
  % \midrule
  \endfirsthead
  \toprule
  \SimHei \normalsize 年数 & \SimHei \scriptsize 公元 & \SimHei 大事件 \tabularnewline
  \midrule
  \endhead
  \midrule
  元年 & 968 & \tabularnewline\hline
  二年 & 969 & \tabularnewline\hline
  三年 & 970 & \tabularnewline\hline
  四年 & 971 & \tabularnewline\hline
  五年 & 972 & \tabularnewline\hline
  六年 & 973 & \tabularnewline
  \bottomrule
\end{longtable}

\subsubsection{广运}

\begin{longtable}{|>{\centering\scriptsize}m{2em}|>{\centering\scriptsize}m{1.3em}|>{\centering}m{8.8em}|}
  % \caption{秦王政}\
  \toprule
  \SimHei \normalsize 年数 & \SimHei \scriptsize 公元 & \SimHei 大事件 \tabularnewline
  % \midrule
  \endfirsthead
  \toprule
  \SimHei \normalsize 年数 & \SimHei \scriptsize 公元 & \SimHei 大事件 \tabularnewline
  \midrule
  \endhead
  \midrule
  元年 & 974 & \tabularnewline\hline
  二年 & 975 & \tabularnewline\hline
  三年 & 976 & \tabularnewline\hline
  四年 & 977 & \tabularnewline\hline
  五年 & 978 & \tabularnewline\hline
  六年 & 979 & \tabularnewline
  \bottomrule
\end{longtable}


%%% Local Variables:
%%% mode: latex
%%% TeX-engine: xetex
%%% TeX-master: "../../Main"
%%% End:



%%% Local Variables:
%%% mode: latex
%%% TeX-engine: xetex
%%% TeX-master: "../../Main"
%%% End:



%%% Local Variables:
%%% mode: latex
%%% TeX-engine: xetex
%%% TeX-master: "../Main"
%%% End:

% %% -*- coding: utf-8 -*-
%% Time-stamp: <Chen Wang: 2018-07-12 01:11:42>

\chapter{北宋\tiny(960-1127)}

%% -*- coding: utf-8 -*-
%% Time-stamp: <Chen Wang: 2018-07-12 01:13:53>

\section{太祖\tiny(960-976)}

\subsection{建隆}


\begin{longtable}{|>{\centering\scriptsize}m{2em}|>{\centering\scriptsize}m{1.3em}|>{\centering}m{8.8em}|}
  % \caption{秦王政}\
  \toprule
  \SimHei \normalsize 年数 & \SimHei \scriptsize 公元 & \SimHei 大事件 \tabularnewline
  % \midrule
  \endfirsthead
  \toprule
  \SimHei \normalsize 年数 & \SimHei \scriptsize 公元 & \SimHei 大事件 \tabularnewline
  \midrule
  \endhead
  \midrule
  元年 & 960 & \tabularnewline\hline
  二年 & 961 & \tabularnewline\hline
  三年 & 962 & \tabularnewline\hline
  四年 & 963 & \tabularnewline
  \bottomrule
\end{longtable}

\subsection{乾德}

\begin{longtable}{|>{\centering\scriptsize}m{2em}|>{\centering\scriptsize}m{1.3em}|>{\centering}m{8.8em}|}
  % \caption{秦王政}\
  \toprule
  \SimHei \normalsize 年数 & \SimHei \scriptsize 公元 & \SimHei 大事件 \tabularnewline
  % \midrule
  \endfirsthead
  \toprule
  \SimHei \normalsize 年数 & \SimHei \scriptsize 公元 & \SimHei 大事件 \tabularnewline
  \midrule
  \endhead
  \midrule
  元年 & 963 & \tabularnewline\hline
  二年 & 964 & \tabularnewline\hline
  三年 & 965 & \tabularnewline\hline
  四年 & 966 & \tabularnewline\hline
  五年 & 967 & \tabularnewline\hline
  六年 & 968 & \tabularnewline
  \bottomrule
\end{longtable}

\subsection{开宝}

\begin{longtable}{|>{\centering\scriptsize}m{2em}|>{\centering\scriptsize}m{1.3em}|>{\centering}m{8.8em}|}
  % \caption{秦王政}\
  \toprule
  \SimHei \normalsize 年数 & \SimHei \scriptsize 公元 & \SimHei 大事件 \tabularnewline
  % \midrule
  \endfirsthead
  \toprule
  \SimHei \normalsize 年数 & \SimHei \scriptsize 公元 & \SimHei 大事件 \tabularnewline
  \midrule
  \endhead
  \midrule
  元年 & 968 & \tabularnewline\hline
  二年 & 969 & \tabularnewline\hline
  三年 & 970 & \tabularnewline\hline
  四年 & 971 & \tabularnewline\hline
  五年 & 972 & \tabularnewline\hline
  六年 & 973 & \tabularnewline\hline
  七年 & 974 & \tabularnewline\hline
  八年 & 975 & \tabularnewline\hline
  九年 & 976 & \tabularnewline
  \bottomrule
\end{longtable}


%%% Local Variables:
%%% mode: latex
%%% TeX-engine: xetex
%%% TeX-master: "../Main"
%%% End:


%%% Local Variables:
%%% mode: latex
%%% TeX-engine: xetex
%%% TeX-master: "../Main"
%%% End:

% %% -*- coding: utf-8 -*-
%% Time-stamp: <Chen Wang: 2018-07-12 13:14:56>

\chapter{南宋\tiny(1127-1279)}

%% -*- coding: utf-8 -*-
%% Time-stamp: <Chen Wang: 2018-07-12 13:18:27>

\section{高宗\tiny(1127-1162)}

\subsection{建炎}


\begin{longtable}{|>{\centering\scriptsize}m{2em}|>{\centering\scriptsize}m{1.3em}|>{\centering}m{8.8em}|}
  % \caption{秦王政}\
  \toprule
  \SimHei \normalsize 年数 & \SimHei \scriptsize 公元 & \SimHei 大事件 \tabularnewline
  % \midrule
  \endfirsthead
  \toprule
  \SimHei \normalsize 年数 & \SimHei \scriptsize 公元 & \SimHei 大事件 \tabularnewline
  \midrule
  \endhead
  \midrule
  元年 & 1127 & \tabularnewline\hline
  二年 & 1128 & \tabularnewline\hline
  三年 & 1129 & \tabularnewline\hline
  四年 & 1130 & \tabularnewline
  \bottomrule
\end{longtable}

\subsection{绍兴}

\begin{longtable}{|>{\centering\scriptsize}m{2em}|>{\centering\scriptsize}m{1.3em}|>{\centering}m{8.8em}|}
  % \caption{秦王政}\
  \toprule
  \SimHei \normalsize 年数 & \SimHei \scriptsize 公元 & \SimHei 大事件 \tabularnewline
  % \midrule
  \endfirsthead
  \toprule
  \SimHei \normalsize 年数 & \SimHei \scriptsize 公元 & \SimHei 大事件 \tabularnewline
  \midrule
  \endhead
  \midrule
  元年 & 1131 & \tabularnewline\hline
  二年 & 1132 & \tabularnewline\hline
  三年 & 1133 & \tabularnewline\hline
  四年 & 1134 & \tabularnewline\hline
  五年 & 1135 & \tabularnewline\hline
  六年 & 1136 & \tabularnewline\hline
  七年 & 1137 & \tabularnewline\hline
  八年 & 1138 & \tabularnewline\hline
  九年 & 1139 & \tabularnewline\hline
  十年 & 1140 & \tabularnewline\hline
  十一年 & 1141 & \tabularnewline\hline
  十二年 & 1142 & \tabularnewline\hline
  十三年 & 1143 & \tabularnewline\hline
  十四年 & 1144 & \tabularnewline\hline
  十五年 & 1145 & \tabularnewline\hline
  十六年 & 1146 & \tabularnewline\hline
  十七年 & 1147 & \tabularnewline\hline
  十八年 & 1148 & \tabularnewline\hline
  十九年 & 1149 & \tabularnewline\hline
  二十年 & 1150 & \tabularnewline\hline
  二一年 & 1151 & \tabularnewline\hline
  二二年 & 1152 & \tabularnewline\hline
  二三年 & 1153 & \tabularnewline\hline
  二四年 & 1154 & \tabularnewline\hline
  二五年 & 1155 & \tabularnewline\hline
  二六年 & 1156 & \tabularnewline\hline
  二七年 & 1157 & \tabularnewline\hline
  二八年 & 1158 & \tabularnewline\hline
  二九年 & 1159 & \tabularnewline\hline
  三十年 & 1160 & \tabularnewline\hline
  三一年 & 1161 & \tabularnewline\hline
  三二年 & 1162 & \tabularnewline
  \bottomrule
\end{longtable}



%%% Local Variables:
%%% mode: latex
%%% TeX-engine: xetex
%%% TeX-master: "../Main"
%%% End:

%% -*- coding: utf-8 -*-
%% Time-stamp: <Chen Wang: 2018-07-12 13:30:05>

\section{孝宗\tiny(1162-1189)}

\subsection{隆兴}


\begin{longtable}{|>{\centering\scriptsize}m{2em}|>{\centering\scriptsize}m{1.3em}|>{\centering}m{8.8em}|}
  % \caption{秦王政}\
  \toprule
  \SimHei \normalsize 年数 & \SimHei \scriptsize 公元 & \SimHei 大事件 \tabularnewline
  % \midrule
  \endfirsthead
  \toprule
  \SimHei \normalsize 年数 & \SimHei \scriptsize 公元 & \SimHei 大事件 \tabularnewline
  \midrule
  \endhead
  \midrule
  元年 & 1163 & \tabularnewline\hline
  二年 & 1164 & \tabularnewline
  \bottomrule
\end{longtable}

\subsection{乾道}

\begin{longtable}{|>{\centering\scriptsize}m{2em}|>{\centering\scriptsize}m{1.3em}|>{\centering}m{8.8em}|}
  % \caption{秦王政}\
  \toprule
  \SimHei \normalsize 年数 & \SimHei \scriptsize 公元 & \SimHei 大事件 \tabularnewline
  % \midrule
  \endfirsthead
  \toprule
  \SimHei \normalsize 年数 & \SimHei \scriptsize 公元 & \SimHei 大事件 \tabularnewline
  \midrule
  \endhead
  \midrule
  元年 & 1165 & \tabularnewline\hline
  二年 & 1166 & \tabularnewline\hline
  三年 & 1167 & \tabularnewline\hline
  四年 & 1168 & \tabularnewline\hline
  五年 & 1169 & \tabularnewline\hline
  六年 & 1170 & \tabularnewline\hline
  七年 & 1171 & \tabularnewline\hline
  八年 & 1172 & \tabularnewline\hline
  九年 & 1173 & \tabularnewline
  \bottomrule
\end{longtable}

\subsection{淳熙}

\begin{longtable}{|>{\centering\scriptsize}m{2em}|>{\centering\scriptsize}m{1.3em}|>{\centering}m{8.8em}|}
  % \caption{秦王政}\
  \toprule
  \SimHei \normalsize 年数 & \SimHei \scriptsize 公元 & \SimHei 大事件 \tabularnewline
  % \midrule
  \endfirsthead
  \toprule
  \SimHei \normalsize 年数 & \SimHei \scriptsize 公元 & \SimHei 大事件 \tabularnewline
  \midrule
  \endhead
  \midrule
  元年 & 1174 & \tabularnewline\hline
  二年 & 1175 & \tabularnewline\hline
  三年 & 1176 & \tabularnewline\hline
  四年 & 1177 & \tabularnewline\hline
  五年 & 1178 & \tabularnewline\hline
  六年 & 1179 & \tabularnewline\hline
  七年 & 1180 & \tabularnewline\hline
  八年 & 1181 & \tabularnewline\hline
  九年 & 1182 & \tabularnewline\hline
  十年 & 1183 & \tabularnewline\hline
  十一年 & 1184 & \tabularnewline\hline
  十二年 & 1185 & \tabularnewline\hline
  十三年 & 1186 & \tabularnewline\hline
  十四年 & 1187 & \tabularnewline\hline
  十五年 & 1188 & \tabularnewline\hline
  十六年 & 1189 & \tabularnewline
  \bottomrule
\end{longtable}



%%% Local Variables:
%%% mode: latex
%%% TeX-engine: xetex
%%% TeX-master: "../Main"
%%% End:

%% -*- coding: utf-8 -*-
%% Time-stamp: <Chen Wang: 2018-07-12 13:31:09>

\section{光宗\tiny(1189-1194)}

\subsection{绍熙}


\begin{longtable}{|>{\centering\scriptsize}m{2em}|>{\centering\scriptsize}m{1.3em}|>{\centering}m{8.8em}|}
  % \caption{秦王政}\
  \toprule
  \SimHei \normalsize 年数 & \SimHei \scriptsize 公元 & \SimHei 大事件 \tabularnewline
  % \midrule
  \endfirsthead
  \toprule
  \SimHei \normalsize 年数 & \SimHei \scriptsize 公元 & \SimHei 大事件 \tabularnewline
  \midrule
  \endhead
  \midrule
  元年 & 1190 & \tabularnewline\hline
  二年 & 1191 & \tabularnewline\hline
  三年 & 1192 & \tabularnewline\hline
  四年 & 1193 & \tabularnewline\hline
  五年 & 1194 & \tabularnewline
  \bottomrule
\end{longtable}



%%% Local Variables:
%%% mode: latex
%%% TeX-engine: xetex
%%% TeX-master: "../Main"
%%% End:

%% -*- coding: utf-8 -*-
%% Time-stamp: <Chen Wang: 2018-07-12 13:33:18>

\section{宁宗\tiny(1194-1224)}

\subsection{庆元}


\begin{longtable}{|>{\centering\scriptsize}m{2em}|>{\centering\scriptsize}m{1.3em}|>{\centering}m{8.8em}|}
  % \caption{秦王政}\
  \toprule
  \SimHei \normalsize 年数 & \SimHei \scriptsize 公元 & \SimHei 大事件 \tabularnewline
  % \midrule
  \endfirsthead
  \toprule
  \SimHei \normalsize 年数 & \SimHei \scriptsize 公元 & \SimHei 大事件 \tabularnewline
  \midrule
  \endhead
  \midrule
  元年 & 1195 & \tabularnewline\hline
  二年 & 1196 & \tabularnewline\hline
  三年 & 1197 & \tabularnewline\hline
  四年 & 1198 & \tabularnewline\hline
  五年 & 1199 & \tabularnewline\hline
  六年 & 1200 & \tabularnewline
  \bottomrule
\end{longtable}

\subsection{嘉泰}

\begin{longtable}{|>{\centering\scriptsize}m{2em}|>{\centering\scriptsize}m{1.3em}|>{\centering}m{8.8em}|}
  % \caption{秦王政}\
  \toprule
  \SimHei \normalsize 年数 & \SimHei \scriptsize 公元 & \SimHei 大事件 \tabularnewline
  % \midrule
  \endfirsthead
  \toprule
  \SimHei \normalsize 年数 & \SimHei \scriptsize 公元 & \SimHei 大事件 \tabularnewline
  \midrule
  \endhead
  \midrule
  元年 & 1201 & \tabularnewline\hline
  二年 & 1202 & \tabularnewline\hline
  三年 & 1203 & \tabularnewline\hline
  四年 & 1204 & \tabularnewline
  \bottomrule
\end{longtable}

\subsection{开禧}

\begin{longtable}{|>{\centering\scriptsize}m{2em}|>{\centering\scriptsize}m{1.3em}|>{\centering}m{8.8em}|}
  % \caption{秦王政}\
  \toprule
  \SimHei \normalsize 年数 & \SimHei \scriptsize 公元 & \SimHei 大事件 \tabularnewline
  % \midrule
  \endfirsthead
  \toprule
  \SimHei \normalsize 年数 & \SimHei \scriptsize 公元 & \SimHei 大事件 \tabularnewline
  \midrule
  \endhead
  \midrule
  元年 & 1205 & \tabularnewline\hline
  二年 & 1206 & \tabularnewline\hline
  三年 & 1207 & \tabularnewline
  \bottomrule
\end{longtable}

\subsection{嘉定}

\begin{longtable}{|>{\centering\scriptsize}m{2em}|>{\centering\scriptsize}m{1.3em}|>{\centering}m{8.8em}|}
  % \caption{秦王政}\
  \toprule
  \SimHei \normalsize 年数 & \SimHei \scriptsize 公元 & \SimHei 大事件 \tabularnewline
  % \midrule
  \endfirsthead
  \toprule
  \SimHei \normalsize 年数 & \SimHei \scriptsize 公元 & \SimHei 大事件 \tabularnewline
  \midrule
  \endhead
  \midrule
  元年 & 1208 & \tabularnewline\hline
  二年 & 1209 & \tabularnewline\hline
  三年 & 1210 & \tabularnewline\hline
  四年 & 1211 & \tabularnewline\hline
  五年 & 1212 & \tabularnewline\hline
  六年 & 1213 & \tabularnewline\hline
  七年 & 1214 & \tabularnewline\hline
  八年 & 1215 & \tabularnewline\hline
  九年 & 1216 & \tabularnewline\hline
  十年 & 1217 & \tabularnewline\hline
  十一年 & 1218 & \tabularnewline\hline
  十二年 & 1219 & \tabularnewline\hline
  十三年 & 1220 & \tabularnewline\hline
  十四年 & 1221 & \tabularnewline\hline
  十五年 & 1222 & \tabularnewline\hline
  十六年 & 1223 & \tabularnewline\hline
  十七年 & 1224 & \tabularnewline
  \bottomrule
\end{longtable}



%%% Local Variables:
%%% mode: latex
%%% TeX-engine: xetex
%%% TeX-master: "../Main"
%%% End:

%% -*- coding: utf-8 -*-
%% Time-stamp: <Chen Wang: 2018-07-12 13:37:54>

\section{理宗\tiny(1224-1264)}

\subsection{宝庆}


\begin{longtable}{|>{\centering\scriptsize}m{2em}|>{\centering\scriptsize}m{1.3em}|>{\centering}m{8.8em}|}
  % \caption{秦王政}\
  \toprule
  \SimHei \normalsize 年数 & \SimHei \scriptsize 公元 & \SimHei 大事件 \tabularnewline
  % \midrule
  \endfirsthead
  \toprule
  \SimHei \normalsize 年数 & \SimHei \scriptsize 公元 & \SimHei 大事件 \tabularnewline
  \midrule
  \endhead
  \midrule
  元年 & 1225 & \tabularnewline\hline
  二年 & 1226 & \tabularnewline\hline
  三年 & 1227 & \tabularnewline
  \bottomrule
\end{longtable}

\subsection{绍定}

\begin{longtable}{|>{\centering\scriptsize}m{2em}|>{\centering\scriptsize}m{1.3em}|>{\centering}m{8.8em}|}
  % \caption{秦王政}\
  \toprule
  \SimHei \normalsize 年数 & \SimHei \scriptsize 公元 & \SimHei 大事件 \tabularnewline
  % \midrule
  \endfirsthead
  \toprule
  \SimHei \normalsize 年数 & \SimHei \scriptsize 公元 & \SimHei 大事件 \tabularnewline
  \midrule
  \endhead
  \midrule
  元年 & 1228 & \tabularnewline\hline
  二年 & 1229 & \tabularnewline\hline
  三年 & 1230 & \tabularnewline\hline
  四年 & 1231 & \tabularnewline\hline
  五年 & 1232 & \tabularnewline\hline
  六年 & 1233 & \tabularnewline
  \bottomrule
\end{longtable}

\subsection{端平}

\begin{longtable}{|>{\centering\scriptsize}m{2em}|>{\centering\scriptsize}m{1.3em}|>{\centering}m{8.8em}|}
  % \caption{秦王政}\
  \toprule
  \SimHei \normalsize 年数 & \SimHei \scriptsize 公元 & \SimHei 大事件 \tabularnewline
  % \midrule
  \endfirsthead
  \toprule
  \SimHei \normalsize 年数 & \SimHei \scriptsize 公元 & \SimHei 大事件 \tabularnewline
  \midrule
  \endhead
  \midrule
  元年 & 1234 & \tabularnewline\hline
  二年 & 1235 & \tabularnewline\hline
  三年 & 1236 & \tabularnewline
  \bottomrule
\end{longtable}

\subsection{嘉熙}

\begin{longtable}{|>{\centering\scriptsize}m{2em}|>{\centering\scriptsize}m{1.3em}|>{\centering}m{8.8em}|}
  % \caption{秦王政}\
  \toprule
  \SimHei \normalsize 年数 & \SimHei \scriptsize 公元 & \SimHei 大事件 \tabularnewline
  % \midrule
  \endfirsthead
  \toprule
  \SimHei \normalsize 年数 & \SimHei \scriptsize 公元 & \SimHei 大事件 \tabularnewline
  \midrule
  \endhead
  \midrule
  元年 & 1237 & \tabularnewline\hline
  二年 & 1238 & \tabularnewline\hline
  三年 & 1239 & \tabularnewline\hline
  四年 & 1240 & \tabularnewline
  \bottomrule
\end{longtable}

\subsection{淳祐}

\begin{longtable}{|>{\centering\scriptsize}m{2em}|>{\centering\scriptsize}m{1.3em}|>{\centering}m{8.8em}|}
  % \caption{秦王政}\
  \toprule
  \SimHei \normalsize 年数 & \SimHei \scriptsize 公元 & \SimHei 大事件 \tabularnewline
  % \midrule
  \endfirsthead
  \toprule
  \SimHei \normalsize 年数 & \SimHei \scriptsize 公元 & \SimHei 大事件 \tabularnewline
  \midrule
  \endhead
  \midrule
  元年 & 241 & \tabularnewline\hline
  二年 & 242 & \tabularnewline\hline
  三年 & 243 & \tabularnewline\hline
  四年 & 244 & \tabularnewline\hline
  五年 & 245 & \tabularnewline\hline
  六年 & 246 & \tabularnewline\hline
  七年 & 247 & \tabularnewline\hline
  八年 & 248 & \tabularnewline\hline
  九年 & 249 & \tabularnewline\hline
  十年 & 250 & \tabularnewline\hline
  十一年 & 251 & \tabularnewline\hline
  十二年 & 252 & \tabularnewline
  \bottomrule
\end{longtable}

\subsection{宝祐}

\begin{longtable}{|>{\centering\scriptsize}m{2em}|>{\centering\scriptsize}m{1.3em}|>{\centering}m{8.8em}|}
  % \caption{秦王政}\
  \toprule
  \SimHei \normalsize 年数 & \SimHei \scriptsize 公元 & \SimHei 大事件 \tabularnewline
  % \midrule
  \endfirsthead
  \toprule
  \SimHei \normalsize 年数 & \SimHei \scriptsize 公元 & \SimHei 大事件 \tabularnewline
  \midrule
  \endhead
  \midrule
  元年 & 1253 & \tabularnewline\hline
  二年 & 1254 & \tabularnewline\hline
  三年 & 1255 & \tabularnewline\hline
  四年 & 1256 & \tabularnewline\hline
  五年 & 1257 & \tabularnewline\hline
  六年 & 1258 & \tabularnewline
  \bottomrule
\end{longtable}

\subsection{开庆}

\begin{longtable}{|>{\centering\scriptsize}m{2em}|>{\centering\scriptsize}m{1.3em}|>{\centering}m{8.8em}|}
  % \caption{秦王政}\
  \toprule
  \SimHei \normalsize 年数 & \SimHei \scriptsize 公元 & \SimHei 大事件 \tabularnewline
  % \midrule
  \endfirsthead
  \toprule
  \SimHei \normalsize 年数 & \SimHei \scriptsize 公元 & \SimHei 大事件 \tabularnewline
  \midrule
  \endhead
  \midrule
  元年 & 1259 & \tabularnewline
  \bottomrule
\end{longtable}

\subsection{景定}

\begin{longtable}{|>{\centering\scriptsize}m{2em}|>{\centering\scriptsize}m{1.3em}|>{\centering}m{8.8em}|}
  % \caption{秦王政}\
  \toprule
  \SimHei \normalsize 年数 & \SimHei \scriptsize 公元 & \SimHei 大事件 \tabularnewline
  % \midrule
  \endfirsthead
  \toprule
  \SimHei \normalsize 年数 & \SimHei \scriptsize 公元 & \SimHei 大事件 \tabularnewline
  \midrule
  \endhead
  \midrule
  元年 & 1260 & \tabularnewline\hline
  二年 & 1261 & \tabularnewline\hline
  三年 & 1262 & \tabularnewline\hline
  四年 & 1263 & \tabularnewline\hline
  五年 & 1264 & \tabularnewline
  \bottomrule
\end{longtable}



%%% Local Variables:
%%% mode: latex
%%% TeX-engine: xetex
%%% TeX-master: "../Main"
%%% End:

%% -*- coding: utf-8 -*-
%% Time-stamp: <Chen Wang: 2018-07-12 13:38:36>

\section{度宗\tiny(1264-1274)}

\subsection{咸淳}


\begin{longtable}{|>{\centering\scriptsize}m{2em}|>{\centering\scriptsize}m{1.3em}|>{\centering}m{8.8em}|}
  % \caption{秦王政}\
  \toprule
  \SimHei \normalsize 年数 & \SimHei \scriptsize 公元 & \SimHei 大事件 \tabularnewline
  % \midrule
  \endfirsthead
  \toprule
  \SimHei \normalsize 年数 & \SimHei \scriptsize 公元 & \SimHei 大事件 \tabularnewline
  \midrule
  \endhead
  \midrule
  元年 & 1265 & \tabularnewline\hline
  二年 & 1266 & \tabularnewline\hline
  三年 & 1267 & \tabularnewline\hline
  四年 & 1268 & \tabularnewline\hline
  五年 & 1269 & \tabularnewline\hline
  六年 & 1270 & \tabularnewline\hline
  七年 & 1271 & \tabularnewline\hline
  八年 & 1272 & \tabularnewline\hline
  九年 & 1273 & \tabularnewline\hline
  十年 & 1274 & \tabularnewline
  \bottomrule
\end{longtable}



%%% Local Variables:
%%% mode: latex
%%% TeX-engine: xetex
%%% TeX-master: "../Main"
%%% End:

%% -*- coding: utf-8 -*-
%% Time-stamp: <Chen Wang: 2018-07-12 13:39:26>

\section{恭帝\tiny(1274-1276)}

\subsection{德祐}


\begin{longtable}{|>{\centering\scriptsize}m{2em}|>{\centering\scriptsize}m{1.3em}|>{\centering}m{8.8em}|}
  % \caption{秦王政}\
  \toprule
  \SimHei \normalsize 年数 & \SimHei \scriptsize 公元 & \SimHei 大事件 \tabularnewline
  % \midrule
  \endfirsthead
  \toprule
  \SimHei \normalsize 年数 & \SimHei \scriptsize 公元 & \SimHei 大事件 \tabularnewline
  \midrule
  \endhead
  \midrule
  元年 & 1275 & \tabularnewline\hline
  二年 & 1276 & \tabularnewline
  \bottomrule
\end{longtable}



%%% Local Variables:
%%% mode: latex
%%% TeX-engine: xetex
%%% TeX-master: "../Main"
%%% End:

%% -*- coding: utf-8 -*-
%% Time-stamp: <Chen Wang: 2018-07-12 13:40:08>

\section{端宗\tiny(1276-1278)}

\subsection{景炎}


\begin{longtable}{|>{\centering\scriptsize}m{2em}|>{\centering\scriptsize}m{1.3em}|>{\centering}m{8.8em}|}
  % \caption{秦王政}\
  \toprule
  \SimHei \normalsize 年数 & \SimHei \scriptsize 公元 & \SimHei 大事件 \tabularnewline
  % \midrule
  \endfirsthead
  \toprule
  \SimHei \normalsize 年数 & \SimHei \scriptsize 公元 & \SimHei 大事件 \tabularnewline
  \midrule
  \endhead
  \midrule
  元年 & 1276 & \tabularnewline\hline
  二年 & 1277 & \tabularnewline\hline
  三年 & 1278 & \tabularnewline
  \bottomrule
\end{longtable}



%%% Local Variables:
%%% mode: latex
%%% TeX-engine: xetex
%%% TeX-master: "../Main"
%%% End:

%% -*- coding: utf-8 -*-
%% Time-stamp: <Chen Wang: 2018-07-12 13:40:48>

\section{赵昺\tiny(1278-1279)}

\subsection{祥兴}


\begin{longtable}{|>{\centering\scriptsize}m{2em}|>{\centering\scriptsize}m{1.3em}|>{\centering}m{8.8em}|}
  % \caption{秦王政}\
  \toprule
  \SimHei \normalsize 年数 & \SimHei \scriptsize 公元 & \SimHei 大事件 \tabularnewline
  % \midrule
  \endfirsthead
  \toprule
  \SimHei \normalsize 年数 & \SimHei \scriptsize 公元 & \SimHei 大事件 \tabularnewline
  \midrule
  \endhead
  \midrule
  元年 & 1278 & \tabularnewline\hline
  二年 & 1279 & \tabularnewline
  \bottomrule
\end{longtable}



%%% Local Variables:
%%% mode: latex
%%% TeX-engine: xetex
%%% TeX-master: "../Main"
%%% End:


%%% Local Variables:
%%% mode: latex
%%% TeX-engine: xetex
%%% TeX-master: "../Main"
%%% End:

% %% -*- coding: utf-8 -*-
%% Time-stamp: <Chen Wang: 2018-07-12 13:48:12>

\chapter{辽\tiny(916-1218)}

%% -*- coding: utf-8 -*-
%% Time-stamp: <Chen Wang: 2018-07-12 13:50:31>

\section{太祖\tiny(916-926)}

\subsection{神册}


\begin{longtable}{|>{\centering\scriptsize}m{2em}|>{\centering\scriptsize}m{1.3em}|>{\centering}m{8.8em}|}
  % \caption{秦王政}\
  \toprule
  \SimHei \normalsize 年数 & \SimHei \scriptsize 公元 & \SimHei 大事件 \tabularnewline
  % \midrule
  \endfirsthead
  \toprule
  \SimHei \normalsize 年数 & \SimHei \scriptsize 公元 & \SimHei 大事件 \tabularnewline
  \midrule
  \endhead
  \midrule
  元年 & 916 & \tabularnewline\hline
  二年 & 917 & \tabularnewline\hline
  三年 & 918 & \tabularnewline\hline
  四年 & 919 & \tabularnewline\hline
  五年 & 920 & \tabularnewline\hline
  六年 & 921 & \tabularnewline\hline
  七年 & 922 & \tabularnewline
  \bottomrule
\end{longtable}

\subsection{天赞}

\begin{longtable}{|>{\centering\scriptsize}m{2em}|>{\centering\scriptsize}m{1.3em}|>{\centering}m{8.8em}|}
  % \caption{秦王政}\
  \toprule
  \SimHei \normalsize 年数 & \SimHei \scriptsize 公元 & \SimHei 大事件 \tabularnewline
  % \midrule
  \endfirsthead
  \toprule
  \SimHei \normalsize 年数 & \SimHei \scriptsize 公元 & \SimHei 大事件 \tabularnewline
  \midrule
  \endhead
  \midrule
  元年 & 922 & \tabularnewline\hline
  二年 & 923 & \tabularnewline\hline
  三年 & 924 & \tabularnewline\hline
  四年 & 925 & \tabularnewline\hline
  五年 & 926 & \tabularnewline
  \bottomrule
\end{longtable}

\subsection{天显}

\begin{longtable}{|>{\centering\scriptsize}m{2em}|>{\centering\scriptsize}m{1.3em}|>{\centering}m{8.8em}|}
  % \caption{秦王政}\
  \toprule
  \SimHei \normalsize 年数 & \SimHei \scriptsize 公元 & \SimHei 大事件 \tabularnewline
  % \midrule
  \endfirsthead
  \toprule
  \SimHei \normalsize 年数 & \SimHei \scriptsize 公元 & \SimHei 大事件 \tabularnewline
  \midrule
  \endhead
  \midrule
  元年 & 926 & \tabularnewline\hline
  二年 & 927 & \tabularnewline\hline
  三年 & 928 & \tabularnewline\hline
  四年 & 929 & \tabularnewline\hline
  五年 & 930 & \tabularnewline\hline
  六年 & 931 & \tabularnewline\hline
  七年 & 932 & \tabularnewline\hline
  八年 & 933 & \tabularnewline\hline
  九年 & 934 & \tabularnewline\hline
  十年 & 935 & \tabularnewline\hline
  十一年 & 936 & \tabularnewline\hline
  十二年 & 937 & \tabularnewline\hline
  十三年 & 938 & \tabularnewline
  \bottomrule
\end{longtable}


%%% Local Variables:
%%% mode: latex
%%% TeX-engine: xetex
%%% TeX-master: "../Main"
%%% End:



%%% Local Variables:
%%% mode: latex
%%% TeX-engine: xetex
%%% TeX-master: "../Main"
%%% End:

% %% -*- coding: utf-8 -*-
%% Time-stamp: <Chen Wang: 2018-07-12 19:35:40>

\chapter{西夏\tiny(1038-1227)}

%% -*- coding: utf-8 -*-
%% Time-stamp: <Chen Wang: 2018-07-12 19:02:34>

\section{景宗\tiny(1032-1048)}

\subsection{显道}

\begin{longtable}{|>{\centering\scriptsize}m{2em}|>{\centering\scriptsize}m{1.3em}|>{\centering}m{8.8em}|}
  % \caption{秦王政}\
  \toprule
  \SimHei \normalsize 年数 & \SimHei \scriptsize 公元 & \SimHei 大事件 \tabularnewline
  % \midrule
  \endfirsthead
  \toprule
  \SimHei \normalsize 年数 & \SimHei \scriptsize 公元 & \SimHei 大事件 \tabularnewline
  \midrule
  \endhead
  \midrule
  元年 & 1032 & \tabularnewline\hline
  二年 & 1033 & \tabularnewline\hline
  三年 & 1034 & \tabularnewline
  \bottomrule
\end{longtable}

\subsection{开运}

\begin{longtable}{|>{\centering\scriptsize}m{2em}|>{\centering\scriptsize}m{1.3em}|>{\centering}m{8.8em}|}
  % \caption{秦王政}\
  \toprule
  \SimHei \normalsize 年数 & \SimHei \scriptsize 公元 & \SimHei 大事件 \tabularnewline
  % \midrule
  \endfirsthead
  \toprule
  \SimHei \normalsize 年数 & \SimHei \scriptsize 公元 & \SimHei 大事件 \tabularnewline
  \midrule
  \endhead
  \midrule
  元年 & 1034 & \tabularnewline
  \bottomrule
\end{longtable}

\subsection{广运}

\begin{longtable}{|>{\centering\scriptsize}m{2em}|>{\centering\scriptsize}m{1.3em}|>{\centering}m{8.8em}|}
  % \caption{秦王政}\
  \toprule
  \SimHei \normalsize 年数 & \SimHei \scriptsize 公元 & \SimHei 大事件 \tabularnewline
  % \midrule
  \endfirsthead
  \toprule
  \SimHei \normalsize 年数 & \SimHei \scriptsize 公元 & \SimHei 大事件 \tabularnewline
  \midrule
  \endhead
  \midrule
  元年 & 1034 & \tabularnewline\hline
  二年 & 1035 & \tabularnewline\hline
  三年 & 1036 & \tabularnewline
  \bottomrule
\end{longtable}

\subsection{大庆}

\begin{longtable}{|>{\centering\scriptsize}m{2em}|>{\centering\scriptsize}m{1.3em}|>{\centering}m{8.8em}|}
  % \caption{秦王政}\
  \toprule
  \SimHei \normalsize 年数 & \SimHei \scriptsize 公元 & \SimHei 大事件 \tabularnewline
  % \midrule
  \endfirsthead
  \toprule
  \SimHei \normalsize 年数 & \SimHei \scriptsize 公元 & \SimHei 大事件 \tabularnewline
  \midrule
  \endhead
  \midrule
  元年 & 1036 & \tabularnewline\hline
  二年 & 1037 & \tabularnewline\hline
  三年 & 1038 & \tabularnewline
  \bottomrule
\end{longtable}

\subsection{天授}

\begin{longtable}{|>{\centering\scriptsize}m{2em}|>{\centering\scriptsize}m{1.3em}|>{\centering}m{8.8em}|}
  % \caption{秦王政}\
  \toprule
  \SimHei \normalsize 年数 & \SimHei \scriptsize 公元 & \SimHei 大事件 \tabularnewline
  % \midrule
  \endfirsthead
  \toprule
  \SimHei \normalsize 年数 & \SimHei \scriptsize 公元 & \SimHei 大事件 \tabularnewline
  \midrule
  \endhead
  \midrule
  元年 & 1038 & \tabularnewline\hline
  二年 & 1039 & \tabularnewline\hline
  三年 & 1040 & \tabularnewline\hline
  四年 & 1041 & \tabularnewline\hline
  五年 & 1042 & \tabularnewline\hline
  六年 & 1043 & \tabularnewline\hline
  七年 & 1044 & \tabularnewline\hline
  八年 & 1045 & \tabularnewline\hline
  九年 & 1046 & \tabularnewline\hline
  十年 & 1047 & \tabularnewline\hline
  十一年 & 1048 & \tabularnewline
  \bottomrule
\end{longtable}


%%% Local Variables:
%%% mode: latex
%%% TeX-engine: xetex
%%% TeX-master: "../Main"
%%% End:

%% -*- coding: utf-8 -*-
%% Time-stamp: <Chen Wang: 2018-07-12 19:21:19>

\section{毅宗\tiny(1048-1067)}

\subsection{延嗣宁国}

\begin{longtable}{|>{\centering\scriptsize}m{2em}|>{\centering\scriptsize}m{1.3em}|>{\centering}m{8.8em}|}
  % \caption{秦王政}\
  \toprule
  \SimHei \normalsize 年数 & \SimHei \scriptsize 公元 & \SimHei 大事件 \tabularnewline
  % \midrule
  \endfirsthead
  \toprule
  \SimHei \normalsize 年数 & \SimHei \scriptsize 公元 & \SimHei 大事件 \tabularnewline
  \midrule
  \endhead
  \midrule
  元年 & 1048 & \tabularnewline
  \bottomrule
\end{longtable}

\subsection{天祐垂圣}

\begin{longtable}{|>{\centering\scriptsize}m{2em}|>{\centering\scriptsize}m{1.3em}|>{\centering}m{8.8em}|}
  % \caption{秦王政}\
  \toprule
  \SimHei \normalsize 年数 & \SimHei \scriptsize 公元 & \SimHei 大事件 \tabularnewline
  % \midrule
  \endfirsthead
  \toprule
  \SimHei \normalsize 年数 & \SimHei \scriptsize 公元 & \SimHei 大事件 \tabularnewline
  \midrule
  \endhead
  \midrule
  元年 & 1050 & \tabularnewline\hline
  二年 & 1051 & \tabularnewline\hline
  三年 & 1052 & \tabularnewline
  \bottomrule
\end{longtable}

\subsection{福圣承道}

\begin{longtable}{|>{\centering\scriptsize}m{2em}|>{\centering\scriptsize}m{1.3em}|>{\centering}m{8.8em}|}
  % \caption{秦王政}\
  \toprule
  \SimHei \normalsize 年数 & \SimHei \scriptsize 公元 & \SimHei 大事件 \tabularnewline
  % \midrule
  \endfirsthead
  \toprule
  \SimHei \normalsize 年数 & \SimHei \scriptsize 公元 & \SimHei 大事件 \tabularnewline
  \midrule
  \endhead
  \midrule
  元年 & 1053 & \tabularnewline\hline
  二年 & 1054 & \tabularnewline\hline
  三年 & 1055 & \tabularnewline\hline
  四年 & 1056 & \tabularnewline
  \bottomrule
\end{longtable}

\subsection{奲都}

\begin{longtable}{|>{\centering\scriptsize}m{2em}|>{\centering\scriptsize}m{1.3em}|>{\centering}m{8.8em}|}
  % \caption{秦王政}\
  \toprule
  \SimHei \normalsize 年数 & \SimHei \scriptsize 公元 & \SimHei 大事件 \tabularnewline
  % \midrule
  \endfirsthead
  \toprule
  \SimHei \normalsize 年数 & \SimHei \scriptsize 公元 & \SimHei 大事件 \tabularnewline
  \midrule
  \endhead
  \midrule
  元年 & 1057 & \tabularnewline\hline
  二年 & 1058 & \tabularnewline\hline
  三年 & 1059 & \tabularnewline\hline
  四年 & 1060 & \tabularnewline\hline
  五年 & 1061 & \tabularnewline\hline
  六年 & 1062 & \tabularnewline
  \bottomrule
\end{longtable}

\subsection{拱化}

\begin{longtable}{|>{\centering\scriptsize}m{2em}|>{\centering\scriptsize}m{1.3em}|>{\centering}m{8.8em}|}
  % \caption{秦王政}\
  \toprule
  \SimHei \normalsize 年数 & \SimHei \scriptsize 公元 & \SimHei 大事件 \tabularnewline
  % \midrule
  \endfirsthead
  \toprule
  \SimHei \normalsize 年数 & \SimHei \scriptsize 公元 & \SimHei 大事件 \tabularnewline
  \midrule
  \endhead
  \midrule
  元年 & 1063 & \tabularnewline\hline
  二年 & 1064 & \tabularnewline\hline
  三年 & 1065 & \tabularnewline\hline
  四年 & 1066 & \tabularnewline\hline
  五年 & 1067 & \tabularnewline
  \bottomrule
\end{longtable}


%%% Local Variables:
%%% mode: latex
%%% TeX-engine: xetex
%%% TeX-master: "../Main"
%%% End:

%% -*- coding: utf-8 -*-
%% Time-stamp: <Chen Wang: 2018-07-12 19:23:19>

\section{惠宗\tiny(1067-1086)}

\subsection{乾道}

\begin{longtable}{|>{\centering\scriptsize}m{2em}|>{\centering\scriptsize}m{1.3em}|>{\centering}m{8.8em}|}
  % \caption{秦王政}\
  \toprule
  \SimHei \normalsize 年数 & \SimHei \scriptsize 公元 & \SimHei 大事件 \tabularnewline
  % \midrule
  \endfirsthead
  \toprule
  \SimHei \normalsize 年数 & \SimHei \scriptsize 公元 & \SimHei 大事件 \tabularnewline
  \midrule
  \endhead
  \midrule
  元年 & 1067 & \tabularnewline\hline
  二年 & 1068 & \tabularnewline
  \bottomrule
\end{longtable}

\subsection{天赐国庆}

\begin{longtable}{|>{\centering\scriptsize}m{2em}|>{\centering\scriptsize}m{1.3em}|>{\centering}m{8.8em}|}
  % \caption{秦王政}\
  \toprule
  \SimHei \normalsize 年数 & \SimHei \scriptsize 公元 & \SimHei 大事件 \tabularnewline
  % \midrule
  \endfirsthead
  \toprule
  \SimHei \normalsize 年数 & \SimHei \scriptsize 公元 & \SimHei 大事件 \tabularnewline
  \midrule
  \endhead
  \midrule
  元年 & 1069 & \tabularnewline\hline
  二年 & 1070 & \tabularnewline\hline
  三年 & 1071 & \tabularnewline\hline
  四年 & 1072 & \tabularnewline\hline
  五年 & 1073 & \tabularnewline\hline
  六年 & 1074 & \tabularnewline
  \bottomrule
\end{longtable}

\subsection{大安}

\begin{longtable}{|>{\centering\scriptsize}m{2em}|>{\centering\scriptsize}m{1.3em}|>{\centering}m{8.8em}|}
  % \caption{秦王政}\
  \toprule
  \SimHei \normalsize 年数 & \SimHei \scriptsize 公元 & \SimHei 大事件 \tabularnewline
  % \midrule
  \endfirsthead
  \toprule
  \SimHei \normalsize 年数 & \SimHei \scriptsize 公元 & \SimHei 大事件 \tabularnewline
  \midrule
  \endhead
  \midrule
  元年 & 1075 & \tabularnewline\hline
  二年 & 1076 & \tabularnewline\hline
  三年 & 1077 & \tabularnewline\hline
  四年 & 1078 & \tabularnewline\hline
  五年 & 1079 & \tabularnewline\hline
  六年 & 1080 & \tabularnewline\hline
  七年 & 1081 & \tabularnewline\hline
  八年 & 1082 & \tabularnewline\hline
  九年 & 1083 & \tabularnewline\hline
  十年 & 1084 & \tabularnewline\hline
  十一年 & 1085 & \tabularnewline
  \bottomrule
\end{longtable}

\subsection{天安礼定}

\begin{longtable}{|>{\centering\scriptsize}m{2em}|>{\centering\scriptsize}m{1.3em}|>{\centering}m{8.8em}|}
  % \caption{秦王政}\
  \toprule
  \SimHei \normalsize 年数 & \SimHei \scriptsize 公元 & \SimHei 大事件 \tabularnewline
  % \midrule
  \endfirsthead
  \toprule
  \SimHei \normalsize 年数 & \SimHei \scriptsize 公元 & \SimHei 大事件 \tabularnewline
  \midrule
  \endhead
  \midrule
  元年 & 1086 & \tabularnewline
  \bottomrule
\end{longtable}



%%% Local Variables:
%%% mode: latex
%%% TeX-engine: xetex
%%% TeX-master: "../Main"
%%% End:

%% -*- coding: utf-8 -*-
%% Time-stamp: <Chen Wang: 2018-07-12 19:27:29>

\section{崇宗\tiny(1086-1139)}

\subsection{天仪治平}

\begin{longtable}{|>{\centering\scriptsize}m{2em}|>{\centering\scriptsize}m{1.3em}|>{\centering}m{8.8em}|}
  % \caption{秦王政}\
  \toprule
  \SimHei \normalsize 年数 & \SimHei \scriptsize 公元 & \SimHei 大事件 \tabularnewline
  % \midrule
  \endfirsthead
  \toprule
  \SimHei \normalsize 年数 & \SimHei \scriptsize 公元 & \SimHei 大事件 \tabularnewline
  \midrule
  \endhead
  \midrule
  元年 & 1086 & \tabularnewline\hline
  二年 & 1087 & \tabularnewline\hline
  三年 & 1088 & \tabularnewline\hline
  四年 & 1089 & \tabularnewline
  \bottomrule
\end{longtable}

\subsection{天祐民安}

\begin{longtable}{|>{\centering\scriptsize}m{2em}|>{\centering\scriptsize}m{1.3em}|>{\centering}m{8.8em}|}
  % \caption{秦王政}\
  \toprule
  \SimHei \normalsize 年数 & \SimHei \scriptsize 公元 & \SimHei 大事件 \tabularnewline
  % \midrule
  \endfirsthead
  \toprule
  \SimHei \normalsize 年数 & \SimHei \scriptsize 公元 & \SimHei 大事件 \tabularnewline
  \midrule
  \endhead
  \midrule
  元年 & 1090 & \tabularnewline\hline
  二年 & 1091 & \tabularnewline\hline
  三年 & 1092 & \tabularnewline\hline
  四年 & 1093 & \tabularnewline\hline
  五年 & 1094 & \tabularnewline\hline
  六年 & 1095 & \tabularnewline\hline
  七年 & 1096 & \tabularnewline\hline
  八年 & 1097 & \tabularnewline
  \bottomrule
\end{longtable}

\subsection{永安}

\begin{longtable}{|>{\centering\scriptsize}m{2em}|>{\centering\scriptsize}m{1.3em}|>{\centering}m{8.8em}|}
  % \caption{秦王政}\
  \toprule
  \SimHei \normalsize 年数 & \SimHei \scriptsize 公元 & \SimHei 大事件 \tabularnewline
  % \midrule
  \endfirsthead
  \toprule
  \SimHei \normalsize 年数 & \SimHei \scriptsize 公元 & \SimHei 大事件 \tabularnewline
  \midrule
  \endhead
  \midrule
  元年 & 1098 & \tabularnewline\hline
  二年 & 1099 & \tabularnewline\hline
  三年 & 1100 & \tabularnewline
  \bottomrule
\end{longtable}

\subsection{贞观}

\begin{longtable}{|>{\centering\scriptsize}m{2em}|>{\centering\scriptsize}m{1.3em}|>{\centering}m{8.8em}|}
  % \caption{秦王政}\
  \toprule
  \SimHei \normalsize 年数 & \SimHei \scriptsize 公元 & \SimHei 大事件 \tabularnewline
  % \midrule
  \endfirsthead
  \toprule
  \SimHei \normalsize 年数 & \SimHei \scriptsize 公元 & \SimHei 大事件 \tabularnewline
  \midrule
  \endhead
  \midrule
  元年 & 1101 & \tabularnewline\hline
  二年 & 1102 & \tabularnewline\hline
  三年 & 1103 & \tabularnewline\hline
  四年 & 1104 & \tabularnewline\hline
  五年 & 1105 & \tabularnewline\hline
  六年 & 1106 & \tabularnewline\hline
  七年 & 1107 & \tabularnewline\hline
  八年 & 1108 & \tabularnewline\hline
  九年 & 1109 & \tabularnewline\hline
  十年 & 1110 & \tabularnewline\hline
  十一年 & 1111 & \tabularnewline\hline
  十二年 & 1112 & \tabularnewline\hline
  十三年 & 1113 & \tabularnewline
  \bottomrule
\end{longtable}

\subsection{雍宁}

\begin{longtable}{|>{\centering\scriptsize}m{2em}|>{\centering\scriptsize}m{1.3em}|>{\centering}m{8.8em}|}
  % \caption{秦王政}\
  \toprule
  \SimHei \normalsize 年数 & \SimHei \scriptsize 公元 & \SimHei 大事件 \tabularnewline
  % \midrule
  \endfirsthead
  \toprule
  \SimHei \normalsize 年数 & \SimHei \scriptsize 公元 & \SimHei 大事件 \tabularnewline
  \midrule
  \endhead
  \midrule
  元年 & 1114 & \tabularnewline\hline
  二年 & 1115 & \tabularnewline\hline
  三年 & 1116 & \tabularnewline\hline
  四年 & 1117 & \tabularnewline\hline
  五年 & 1118 & \tabularnewline
  \bottomrule
\end{longtable}

\subsection{元德}

\begin{longtable}{|>{\centering\scriptsize}m{2em}|>{\centering\scriptsize}m{1.3em}|>{\centering}m{8.8em}|}
  % \caption{秦王政}\
  \toprule
  \SimHei \normalsize 年数 & \SimHei \scriptsize 公元 & \SimHei 大事件 \tabularnewline
  % \midrule
  \endfirsthead
  \toprule
  \SimHei \normalsize 年数 & \SimHei \scriptsize 公元 & \SimHei 大事件 \tabularnewline
  \midrule
  \endhead
  \midrule
  元年 & 1119 & \tabularnewline\hline
  二年 & 1120 & \tabularnewline\hline
  三年 & 1121 & \tabularnewline\hline
  四年 & 1122 & \tabularnewline\hline
  五年 & 1123 & \tabularnewline\hline
  六年 & 1124 & \tabularnewline\hline
  七年 & 1125 & \tabularnewline\hline
  八年 & 1126 & \tabularnewline\hline
  九年 & 1127 & \tabularnewline
  \bottomrule
\end{longtable}

\subsection{正德}

\begin{longtable}{|>{\centering\scriptsize}m{2em}|>{\centering\scriptsize}m{1.3em}|>{\centering}m{8.8em}|}
  % \caption{秦王政}\
  \toprule
  \SimHei \normalsize 年数 & \SimHei \scriptsize 公元 & \SimHei 大事件 \tabularnewline
  % \midrule
  \endfirsthead
  \toprule
  \SimHei \normalsize 年数 & \SimHei \scriptsize 公元 & \SimHei 大事件 \tabularnewline
  \midrule
  \endhead
  \midrule
  元年 & 1127 & \tabularnewline\hline
  二年 & 1128 & \tabularnewline\hline
  三年 & 1129 & \tabularnewline\hline
  四年 & 1130 & \tabularnewline\hline
  五年 & 1131 & \tabularnewline\hline
  六年 & 1132 & \tabularnewline\hline
  七年 & 1133 & \tabularnewline\hline
  八年 & 1134 & \tabularnewline
  \bottomrule
\end{longtable}

\subsection{大德}

\begin{longtable}{|>{\centering\scriptsize}m{2em}|>{\centering\scriptsize}m{1.3em}|>{\centering}m{8.8em}|}
  % \caption{秦王政}\
  \toprule
  \SimHei \normalsize 年数 & \SimHei \scriptsize 公元 & \SimHei 大事件 \tabularnewline
  % \midrule
  \endfirsthead
  \toprule
  \SimHei \normalsize 年数 & \SimHei \scriptsize 公元 & \SimHei 大事件 \tabularnewline
  \midrule
  \endhead
  \midrule
  元年 & 1135 & \tabularnewline\hline
  二年 & 1136 & \tabularnewline\hline
  三年 & 1137 & \tabularnewline\hline
  四年 & 1138 & \tabularnewline\hline
  五年 & 1139 & \tabularnewline
  \bottomrule
\end{longtable}


%%% Local Variables:
%%% mode: latex
%%% TeX-engine: xetex
%%% TeX-master: "../Main"
%%% End:

%% -*- coding: utf-8 -*-
%% Time-stamp: <Chen Wang: 2018-07-12 19:29:32>

\section{仁宗\tiny(1139-1193)}

\subsection{大庆}

\begin{longtable}{|>{\centering\scriptsize}m{2em}|>{\centering\scriptsize}m{1.3em}|>{\centering}m{8.8em}|}
  % \caption{秦王政}\
  \toprule
  \SimHei \normalsize 年数 & \SimHei \scriptsize 公元 & \SimHei 大事件 \tabularnewline
  % \midrule
  \endfirsthead
  \toprule
  \SimHei \normalsize 年数 & \SimHei \scriptsize 公元 & \SimHei 大事件 \tabularnewline
  \midrule
  \endhead
  \midrule
  元年 & 1140 & \tabularnewline\hline
  二年 & 1141 & \tabularnewline\hline
  三年 & 1142 & \tabularnewline\hline
  四年 & 1143 & \tabularnewline
  \bottomrule
\end{longtable}

\subsection{人庆}

\begin{longtable}{|>{\centering\scriptsize}m{2em}|>{\centering\scriptsize}m{1.3em}|>{\centering}m{8.8em}|}
  % \caption{秦王政}\
  \toprule
  \SimHei \normalsize 年数 & \SimHei \scriptsize 公元 & \SimHei 大事件 \tabularnewline
  % \midrule
  \endfirsthead
  \toprule
  \SimHei \normalsize 年数 & \SimHei \scriptsize 公元 & \SimHei 大事件 \tabularnewline
  \midrule
  \endhead
  \midrule
  元年 & 1144 & \tabularnewline\hline
  二年 & 1145 & \tabularnewline\hline
  三年 & 1146 & \tabularnewline\hline
  四年 & 1147 & \tabularnewline\hline
  五年 & 1148 & \tabularnewline
  \bottomrule
\end{longtable}

\subsection{天盛}

\begin{longtable}{|>{\centering\scriptsize}m{2em}|>{\centering\scriptsize}m{1.3em}|>{\centering}m{8.8em}|}
  % \caption{秦王政}\
  \toprule
  \SimHei \normalsize 年数 & \SimHei \scriptsize 公元 & \SimHei 大事件 \tabularnewline
  % \midrule
  \endfirsthead
  \toprule
  \SimHei \normalsize 年数 & \SimHei \scriptsize 公元 & \SimHei 大事件 \tabularnewline
  \midrule
  \endhead
  \midrule
  元年 & 1149 & \tabularnewline\hline
  二年 & 1150 & \tabularnewline\hline
  三年 & 1151 & \tabularnewline\hline
  四年 & 1152 & \tabularnewline\hline
  五年 & 1153 & \tabularnewline\hline
  六年 & 1154 & \tabularnewline\hline
  七年 & 1155 & \tabularnewline\hline
  八年 & 1156 & \tabularnewline\hline
  九年 & 1157 & \tabularnewline\hline
  十年 & 1158 & \tabularnewline\hline
  十一年 & 1159 & \tabularnewline\hline
  十二年 & 1160 & \tabularnewline\hline
  十三年 & 1161 & \tabularnewline\hline
  十四年 & 1162 & \tabularnewline\hline
  十五年 & 1163 & \tabularnewline\hline
  十六年 & 1164 & \tabularnewline\hline
  十七年 & 1165 & \tabularnewline\hline
  十八年 & 1166 & \tabularnewline\hline
  十九年 & 1167 & \tabularnewline\hline
  二十年 & 1168 & \tabularnewline\hline
  二一年 & 1169 & \tabularnewline
  \bottomrule
\end{longtable}

\subsection{乾佑}

\begin{longtable}{|>{\centering\scriptsize}m{2em}|>{\centering\scriptsize}m{1.3em}|>{\centering}m{8.8em}|}
  % \caption{秦王政}\
  \toprule
  \SimHei \normalsize 年数 & \SimHei \scriptsize 公元 & \SimHei 大事件 \tabularnewline
  % \midrule
  \endfirsthead
  \toprule
  \SimHei \normalsize 年数 & \SimHei \scriptsize 公元 & \SimHei 大事件 \tabularnewline
  \midrule
  \endhead
  \midrule
  元年 & 1170 & \tabularnewline\hline
  二年 & 1171 & \tabularnewline\hline
  三年 & 1172 & \tabularnewline\hline
  四年 & 1173 & \tabularnewline\hline
  五年 & 1174 & \tabularnewline\hline
  六年 & 1175 & \tabularnewline\hline
  七年 & 1176 & \tabularnewline\hline
  八年 & 1177 & \tabularnewline\hline
  九年 & 1178 & \tabularnewline\hline
  十年 & 1179 & \tabularnewline\hline
  十一年 & 1180 & \tabularnewline\hline
  十二年 & 1181 & \tabularnewline\hline
  十三年 & 1182 & \tabularnewline\hline
  十四年 & 1183 & \tabularnewline\hline
  十五年 & 1184 & \tabularnewline\hline
  十六年 & 1185 & \tabularnewline\hline
  十七年 & 1186 & \tabularnewline\hline
  十八年 & 1187 & \tabularnewline\hline
  十九年 & 1188 & \tabularnewline\hline
  二十年 & 1189 & \tabularnewline\hline
  二一年 & 1190 & \tabularnewline\hline
  二二年 & 1191 & \tabularnewline\hline
  二三年 & 1192 & \tabularnewline\hline
  二四年 & 1193 & \tabularnewline
  \bottomrule
\end{longtable}


%%% Local Variables:
%%% mode: latex
%%% TeX-engine: xetex
%%% TeX-master: "../Main"
%%% End:

%% -*- coding: utf-8 -*-
%% Time-stamp: <Chen Wang: 2018-07-12 19:30:42>

\section{桓宗\tiny(1193-1206)}

\subsection{天庆}

\begin{longtable}{|>{\centering\scriptsize}m{2em}|>{\centering\scriptsize}m{1.3em}|>{\centering}m{8.8em}|}
  % \caption{秦王政}\
  \toprule
  \SimHei \normalsize 年数 & \SimHei \scriptsize 公元 & \SimHei 大事件 \tabularnewline
  % \midrule
  \endfirsthead
  \toprule
  \SimHei \normalsize 年数 & \SimHei \scriptsize 公元 & \SimHei 大事件 \tabularnewline
  \midrule
  \endhead
  \midrule
  元年 & 1194 & \tabularnewline\hline
  二年 & 1195 & \tabularnewline\hline
  三年 & 1196 & \tabularnewline\hline
  四年 & 1197 & \tabularnewline\hline
  五年 & 1198 & \tabularnewline\hline
  六年 & 1199 & \tabularnewline\hline
  七年 & 1200 & \tabularnewline\hline
  八年 & 1201 & \tabularnewline\hline
  九年 & 1202 & \tabularnewline\hline
  十年 & 1203 & \tabularnewline\hline
  十一年 & 1204 & \tabularnewline\hline
  十二年 & 1205 & \tabularnewline\hline
  十三年 & 1206 & \tabularnewline
  \bottomrule
\end{longtable}


%%% Local Variables:
%%% mode: latex
%%% TeX-engine: xetex
%%% TeX-master: "../Main"
%%% End:

%% -*- coding: utf-8 -*-
%% Time-stamp: <Chen Wang: 2018-07-12 19:31:53>

\section{襄宗\tiny(1206-1211)}

\subsection{应天}

\begin{longtable}{|>{\centering\scriptsize}m{2em}|>{\centering\scriptsize}m{1.3em}|>{\centering}m{8.8em}|}
  % \caption{秦王政}\
  \toprule
  \SimHei \normalsize 年数 & \SimHei \scriptsize 公元 & \SimHei 大事件 \tabularnewline
  % \midrule
  \endfirsthead
  \toprule
  \SimHei \normalsize 年数 & \SimHei \scriptsize 公元 & \SimHei 大事件 \tabularnewline
  \midrule
  \endhead
  \midrule
  元年 & 1206 & \tabularnewline\hline
  二年 & 1207 & \tabularnewline\hline
  三年 & 1208 & \tabularnewline\hline
  四年 & 1209 & \tabularnewline
  \bottomrule
\end{longtable}

\subsection{皇建}

\begin{longtable}{|>{\centering\scriptsize}m{2em}|>{\centering\scriptsize}m{1.3em}|>{\centering}m{8.8em}|}
  % \caption{秦王政}\
  \toprule
  \SimHei \normalsize 年数 & \SimHei \scriptsize 公元 & \SimHei 大事件 \tabularnewline
  % \midrule
  \endfirsthead
  \toprule
  \SimHei \normalsize 年数 & \SimHei \scriptsize 公元 & \SimHei 大事件 \tabularnewline
  \midrule
  \endhead
  \midrule
  元年 & 1210 & \tabularnewline\hline
  二年 & 1211 & \tabularnewline
  \bottomrule
\end{longtable}


%%% Local Variables:
%%% mode: latex
%%% TeX-engine: xetex
%%% TeX-master: "../Main"
%%% End:

%% -*- coding: utf-8 -*-
%% Time-stamp: <Chen Wang: 2018-07-12 19:32:42>

\section{神宗\tiny(1211-1223)}

\subsection{光定}

\begin{longtable}{|>{\centering\scriptsize}m{2em}|>{\centering\scriptsize}m{1.3em}|>{\centering}m{8.8em}|}
  % \caption{秦王政}\
  \toprule
  \SimHei \normalsize 年数 & \SimHei \scriptsize 公元 & \SimHei 大事件 \tabularnewline
  % \midrule
  \endfirsthead
  \toprule
  \SimHei \normalsize 年数 & \SimHei \scriptsize 公元 & \SimHei 大事件 \tabularnewline
  \midrule
  \endhead
  \midrule
  元年 & 1211 & \tabularnewline\hline
  二年 & 1212 & \tabularnewline\hline
  三年 & 1213 & \tabularnewline\hline
  四年 & 1214 & \tabularnewline\hline
  五年 & 1215 & \tabularnewline\hline
  六年 & 1216 & \tabularnewline\hline
  七年 & 1217 & \tabularnewline\hline
  八年 & 1218 & \tabularnewline\hline
  九年 & 1219 & \tabularnewline\hline
  十年 & 1220 & \tabularnewline\hline
  十一年 & 1221 & \tabularnewline\hline
  十二年 & 1222 & \tabularnewline\hline
  十三年 & 1223 & \tabularnewline
  \bottomrule
\end{longtable}


%%% Local Variables:
%%% mode: latex
%%% TeX-engine: xetex
%%% TeX-master: "../Main"
%%% End:

%% -*- coding: utf-8 -*-
%% Time-stamp: <Chen Wang: 2018-07-12 19:33:23>

\section{献宗\tiny(1223-1226)}

\subsection{乾定}

\begin{longtable}{|>{\centering\scriptsize}m{2em}|>{\centering\scriptsize}m{1.3em}|>{\centering}m{8.8em}|}
  % \caption{秦王政}\
  \toprule
  \SimHei \normalsize 年数 & \SimHei \scriptsize 公元 & \SimHei 大事件 \tabularnewline
  % \midrule
  \endfirsthead
  \toprule
  \SimHei \normalsize 年数 & \SimHei \scriptsize 公元 & \SimHei 大事件 \tabularnewline
  \midrule
  \endhead
  \midrule
  元年 & 1223 & \tabularnewline\hline
  二年 & 1224 & \tabularnewline\hline
  三年 & 1225 & \tabularnewline\hline
  四年 & 1226 & \tabularnewline
  \bottomrule
\end{longtable}


%%% Local Variables:
%%% mode: latex
%%% TeX-engine: xetex
%%% TeX-master: "../Main"
%%% End:

%% -*- coding: utf-8 -*-
%% Time-stamp: <Chen Wang: 2018-07-12 19:34:32>

\section{李睍\tiny(1226-1227)}

\subsection{宝义}

\begin{longtable}{|>{\centering\scriptsize}m{2em}|>{\centering\scriptsize}m{1.3em}|>{\centering}m{8.8em}|}
  % \caption{秦王政}\
  \toprule
  \SimHei \normalsize 年数 & \SimHei \scriptsize 公元 & \SimHei 大事件 \tabularnewline
  % \midrule
  \endfirsthead
  \toprule
  \SimHei \normalsize 年数 & \SimHei \scriptsize 公元 & \SimHei 大事件 \tabularnewline
  \midrule
  \endhead
  \midrule
  元年 & 1226 & \tabularnewline\hline
  二年 & 1227 & \tabularnewline
  \bottomrule
\end{longtable}


%%% Local Variables:
%%% mode: latex
%%% TeX-engine: xetex
%%% TeX-master: "../Main"
%%% End:



%%% Local Variables:
%%% mode: latex
%%% TeX-engine: xetex
%%% TeX-master: "../Main"
%%% End:

% %% -*- coding: utf-8 -*-
%% Time-stamp: <Chen Wang: 2018-07-12 19:51:31>

\chapter{金\tiny(1115-1234)}

%% -*- coding: utf-8 -*-
%% Time-stamp: <Chen Wang: 2018-07-12 19:39:08>

\section{太祖\tiny(1115-1123)}

\subsection{收国}


\begin{longtable}{|>{\centering\scriptsize}m{2em}|>{\centering\scriptsize}m{1.3em}|>{\centering}m{8.8em}|}
  % \caption{秦王政}\
  \toprule
  \SimHei \normalsize 年数 & \SimHei \scriptsize 公元 & \SimHei 大事件 \tabularnewline
  % \midrule
  \endfirsthead
  \toprule
  \SimHei \normalsize 年数 & \SimHei \scriptsize 公元 & \SimHei 大事件 \tabularnewline
  \midrule
  \endhead
  \midrule
  元年 & 1115 & \tabularnewline\hline
  二年 & 1116 & \tabularnewline
  \bottomrule
\end{longtable}

\subsection{天辅}

\begin{longtable}{|>{\centering\scriptsize}m{2em}|>{\centering\scriptsize}m{1.3em}|>{\centering}m{8.8em}|}
  % \caption{秦王政}\
  \toprule
  \SimHei \normalsize 年数 & \SimHei \scriptsize 公元 & \SimHei 大事件 \tabularnewline
  % \midrule
  \endfirsthead
  \toprule
  \SimHei \normalsize 年数 & \SimHei \scriptsize 公元 & \SimHei 大事件 \tabularnewline
  \midrule
  \endhead
  \midrule
  元年 & 1117 & \tabularnewline\hline
  二年 & 1118 & \tabularnewline\hline
  三年 & 1119 & \tabularnewline\hline
  四年 & 1120 & \tabularnewline\hline
  五年 & 1121 & \tabularnewline\hline
  六年 & 1122 & \tabularnewline\hline
  七年 & 1123 & \tabularnewline
  \bottomrule
\end{longtable}


%%% Local Variables:
%%% mode: latex
%%% TeX-engine: xetex
%%% TeX-master: "../Main"
%%% End:

%% -*- coding: utf-8 -*-
%% Time-stamp: <Chen Wang: 2018-07-12 19:40:17>

\section{太宗\tiny(1123-1135)}

\subsection{天会}


\begin{longtable}{|>{\centering\scriptsize}m{2em}|>{\centering\scriptsize}m{1.3em}|>{\centering}m{8.8em}|}
  % \caption{秦王政}\
  \toprule
  \SimHei \normalsize 年数 & \SimHei \scriptsize 公元 & \SimHei 大事件 \tabularnewline
  % \midrule
  \endfirsthead
  \toprule
  \SimHei \normalsize 年数 & \SimHei \scriptsize 公元 & \SimHei 大事件 \tabularnewline
  \midrule
  \endhead
  \midrule
  元年 & 1123 & \tabularnewline\hline
  二年 & 1124 & \tabularnewline\hline
  三年 & 1125 & \tabularnewline\hline
  四年 & 1126 & \tabularnewline\hline
  五年 & 1127 & \tabularnewline\hline
  六年 & 1128 & \tabularnewline\hline
  七年 & 1129 & \tabularnewline\hline
  八年 & 1130 & \tabularnewline\hline
  九年 & 1131 & \tabularnewline\hline
  十年 & 1132 & \tabularnewline\hline
  十一年 & 1133 & \tabularnewline\hline
  十二年 & 1134 & \tabularnewline\hline
  十三年 & 1135 & \tabularnewline\hline
  十四年 & 1136 & \tabularnewline\hline
  十五年 & 1137 & \tabularnewline
  \bottomrule
\end{longtable}


%%% Local Variables:
%%% mode: latex
%%% TeX-engine: xetex
%%% TeX-master: "../Main"
%%% End:

%% -*- coding: utf-8 -*-
%% Time-stamp: <Chen Wang: 2018-07-12 19:41:17>

\section{熙宗\tiny(1135-1149)}

\subsection{天眷}


\begin{longtable}{|>{\centering\scriptsize}m{2em}|>{\centering\scriptsize}m{1.3em}|>{\centering}m{8.8em}|}
  % \caption{秦王政}\
  \toprule
  \SimHei \normalsize 年数 & \SimHei \scriptsize 公元 & \SimHei 大事件 \tabularnewline
  % \midrule
  \endfirsthead
  \toprule
  \SimHei \normalsize 年数 & \SimHei \scriptsize 公元 & \SimHei 大事件 \tabularnewline
  \midrule
  \endhead
  \midrule
  元年 & 1138 & \tabularnewline\hline
  二年 & 1139 & \tabularnewline\hline
  三年 & 1140 & \tabularnewline
  \bottomrule
\end{longtable}

\subsection{皇统}

\begin{longtable}{|>{\centering\scriptsize}m{2em}|>{\centering\scriptsize}m{1.3em}|>{\centering}m{8.8em}|}
  % \caption{秦王政}\
  \toprule
  \SimHei \normalsize 年数 & \SimHei \scriptsize 公元 & \SimHei 大事件 \tabularnewline
  % \midrule
  \endfirsthead
  \toprule
  \SimHei \normalsize 年数 & \SimHei \scriptsize 公元 & \SimHei 大事件 \tabularnewline
  \midrule
  \endhead
  \midrule
  元年 & 1141 & \tabularnewline\hline
  二年 & 1142 & \tabularnewline\hline
  三年 & 1143 & \tabularnewline\hline
  四年 & 1144 & \tabularnewline\hline
  五年 & 1145 & \tabularnewline\hline
  六年 & 1146 & \tabularnewline\hline
  七年 & 1147 & \tabularnewline\hline
  八年 & 1148 & \tabularnewline\hline
  九年 & 1149 & \tabularnewline
  \bottomrule
\end{longtable}


%%% Local Variables:
%%% mode: latex
%%% TeX-engine: xetex
%%% TeX-master: "../Main"
%%% End:

%% -*- coding: utf-8 -*-
%% Time-stamp: <Chen Wang: 2018-07-12 19:42:42>

\section{完颜亮\tiny(1150-1161)}

\subsection{天德}


\begin{longtable}{|>{\centering\scriptsize}m{2em}|>{\centering\scriptsize}m{1.3em}|>{\centering}m{8.8em}|}
  % \caption{秦王政}\
  \toprule
  \SimHei \normalsize 年数 & \SimHei \scriptsize 公元 & \SimHei 大事件 \tabularnewline
  % \midrule
  \endfirsthead
  \toprule
  \SimHei \normalsize 年数 & \SimHei \scriptsize 公元 & \SimHei 大事件 \tabularnewline
  \midrule
  \endhead
  \midrule
  元年 & 1149 & \tabularnewline\hline
  二年 & 1150 & \tabularnewline\hline
  三年 & 1151 & \tabularnewline\hline
  四年 & 1152 & \tabularnewline\hline
  五年 & 1153 & \tabularnewline
  \bottomrule
\end{longtable}

\subsection{贞元}

\begin{longtable}{|>{\centering\scriptsize}m{2em}|>{\centering\scriptsize}m{1.3em}|>{\centering}m{8.8em}|}
  % \caption{秦王政}\
  \toprule
  \SimHei \normalsize 年数 & \SimHei \scriptsize 公元 & \SimHei 大事件 \tabularnewline
  % \midrule
  \endfirsthead
  \toprule
  \SimHei \normalsize 年数 & \SimHei \scriptsize 公元 & \SimHei 大事件 \tabularnewline
  \midrule
  \endhead
  \midrule
  元年 & 1153 & \tabularnewline\hline
  二年 & 1154 & \tabularnewline\hline
  三年 & 1155 & \tabularnewline\hline
  四年 & 1156 & \tabularnewline
  \bottomrule
\end{longtable}

\subsection{正隆}

\begin{longtable}{|>{\centering\scriptsize}m{2em}|>{\centering\scriptsize}m{1.3em}|>{\centering}m{8.8em}|}
  % \caption{秦王政}\
  \toprule
  \SimHei \normalsize 年数 & \SimHei \scriptsize 公元 & \SimHei 大事件 \tabularnewline
  % \midrule
  \endfirsthead
  \toprule
  \SimHei \normalsize 年数 & \SimHei \scriptsize 公元 & \SimHei 大事件 \tabularnewline
  \midrule
  \endhead
  \midrule
  元年 & 1156 & \tabularnewline\hline
  二年 & 1157 & \tabularnewline\hline
  三年 & 1158 & \tabularnewline\hline
  四年 & 1159 & \tabularnewline\hline
  五年 & 1160 & \tabularnewline\hline
  六年 & 1161 & \tabularnewline
  \bottomrule
\end{longtable}


%%% Local Variables:
%%% mode: latex
%%% TeX-engine: xetex
%%% TeX-master: "../Main"
%%% End:

%% -*- coding: utf-8 -*-
%% Time-stamp: <Chen Wang: 2018-07-12 19:43:27>

\section{世宗\tiny(1161-1189)}

\subsection{大定}


\begin{longtable}{|>{\centering\scriptsize}m{2em}|>{\centering\scriptsize}m{1.3em}|>{\centering}m{8.8em}|}
  % \caption{秦王政}\
  \toprule
  \SimHei \normalsize 年数 & \SimHei \scriptsize 公元 & \SimHei 大事件 \tabularnewline
  % \midrule
  \endfirsthead
  \toprule
  \SimHei \normalsize 年数 & \SimHei \scriptsize 公元 & \SimHei 大事件 \tabularnewline
  \midrule
  \endhead
  \midrule
  元年 & 1161 & \tabularnewline\hline
  二年 & 1162 & \tabularnewline\hline
  三年 & 1163 & \tabularnewline\hline
  四年 & 1164 & \tabularnewline\hline
  五年 & 1165 & \tabularnewline\hline
  六年 & 1166 & \tabularnewline\hline
  七年 & 1167 & \tabularnewline\hline
  八年 & 1168 & \tabularnewline\hline
  九年 & 1169 & \tabularnewline\hline
  十年 & 1170 & \tabularnewline\hline
  十一年 & 1171 & \tabularnewline\hline
  十二年 & 1172 & \tabularnewline\hline
  十三年 & 1173 & \tabularnewline\hline
  十四年 & 1174 & \tabularnewline\hline
  十五年 & 1175 & \tabularnewline\hline
  十六年 & 1176 & \tabularnewline\hline
  十七年 & 1177 & \tabularnewline\hline
  十八年 & 1178 & \tabularnewline\hline
  十九年 & 1179 & \tabularnewline\hline
  二十年 & 1180 & \tabularnewline\hline
  二一年 & 1181 & \tabularnewline\hline
  二二年 & 1182 & \tabularnewline\hline
  二三年 & 1183 & \tabularnewline\hline
  二四年 & 1184 & \tabularnewline\hline
  二五年 & 1185 & \tabularnewline\hline
  二六年 & 1186 & \tabularnewline\hline
  二七年 & 1187 & \tabularnewline\hline
  二八年 & 1188 & \tabularnewline\hline
  二九年 & 1189 & \tabularnewline
  \bottomrule
\end{longtable}


%%% Local Variables:
%%% mode: latex
%%% TeX-engine: xetex
%%% TeX-master: "../Main"
%%% End:

%% -*- coding: utf-8 -*-
%% Time-stamp: <Chen Wang: 2018-07-12 19:45:04>

\section{章宗\tiny(1189-1208)}

\subsection{明昌}


\begin{longtable}{|>{\centering\scriptsize}m{2em}|>{\centering\scriptsize}m{1.3em}|>{\centering}m{8.8em}|}
  % \caption{秦王政}\
  \toprule
  \SimHei \normalsize 年数 & \SimHei \scriptsize 公元 & \SimHei 大事件 \tabularnewline
  % \midrule
  \endfirsthead
  \toprule
  \SimHei \normalsize 年数 & \SimHei \scriptsize 公元 & \SimHei 大事件 \tabularnewline
  \midrule
  \endhead
  \midrule
  元年 & 1190 & \tabularnewline\hline
  二年 & 1191 & \tabularnewline\hline
  三年 & 1192 & \tabularnewline\hline
  四年 & 1193 & \tabularnewline\hline
  五年 & 1194 & \tabularnewline\hline
  六年 & 1195 & \tabularnewline\hline
  七年 & 1196 & \tabularnewline
  \bottomrule
\end{longtable}

\subsection{承安}

\begin{longtable}{|>{\centering\scriptsize}m{2em}|>{\centering\scriptsize}m{1.3em}|>{\centering}m{8.8em}|}
  % \caption{秦王政}\
  \toprule
  \SimHei \normalsize 年数 & \SimHei \scriptsize 公元 & \SimHei 大事件 \tabularnewline
  % \midrule
  \endfirsthead
  \toprule
  \SimHei \normalsize 年数 & \SimHei \scriptsize 公元 & \SimHei 大事件 \tabularnewline
  \midrule
  \endhead
  \midrule
  元年 & 1196 & \tabularnewline\hline
  二年 & 1197 & \tabularnewline\hline
  三年 & 1198 & \tabularnewline\hline
  四年 & 1199 & \tabularnewline\hline
  五年 & 1200 & \tabularnewline
  \bottomrule
\end{longtable}

\subsection{泰和}

\begin{longtable}{|>{\centering\scriptsize}m{2em}|>{\centering\scriptsize}m{1.3em}|>{\centering}m{8.8em}|}
  % \caption{秦王政}\
  \toprule
  \SimHei \normalsize 年数 & \SimHei \scriptsize 公元 & \SimHei 大事件 \tabularnewline
  % \midrule
  \endfirsthead
  \toprule
  \SimHei \normalsize 年数 & \SimHei \scriptsize 公元 & \SimHei 大事件 \tabularnewline
  \midrule
  \endhead
  \midrule
  元年 & 1201 & \tabularnewline\hline
  二年 & 1202 & \tabularnewline\hline
  三年 & 1203 & \tabularnewline\hline
  四年 & 1204 & \tabularnewline\hline
  五年 & 1205 & \tabularnewline\hline
  六年 & 1206 & \tabularnewline\hline
  七年 & 1207 & \tabularnewline\hline
  八年 & 1208 & \tabularnewline
  \bottomrule
\end{longtable}


%%% Local Variables:
%%% mode: latex
%%% TeX-engine: xetex
%%% TeX-master: "../Main"
%%% End:

%% -*- coding: utf-8 -*-
%% Time-stamp: <Chen Wang: 2018-07-12 19:46:22>

\section{完颜永济\tiny(1208-1213)}

\subsection{大安}


\begin{longtable}{|>{\centering\scriptsize}m{2em}|>{\centering\scriptsize}m{1.3em}|>{\centering}m{8.8em}|}
  % \caption{秦王政}\
  \toprule
  \SimHei \normalsize 年数 & \SimHei \scriptsize 公元 & \SimHei 大事件 \tabularnewline
  % \midrule
  \endfirsthead
  \toprule
  \SimHei \normalsize 年数 & \SimHei \scriptsize 公元 & \SimHei 大事件 \tabularnewline
  \midrule
  \endhead
  \midrule
  元年 & 1209 & \tabularnewline\hline
  二年 & 1210 & \tabularnewline\hline
  三年 & 1211 & \tabularnewline
  \bottomrule
\end{longtable}

\subsection{崇庆}

\begin{longtable}{|>{\centering\scriptsize}m{2em}|>{\centering\scriptsize}m{1.3em}|>{\centering}m{8.8em}|}
  % \caption{秦王政}\
  \toprule
  \SimHei \normalsize 年数 & \SimHei \scriptsize 公元 & \SimHei 大事件 \tabularnewline
  % \midrule
  \endfirsthead
  \toprule
  \SimHei \normalsize 年数 & \SimHei \scriptsize 公元 & \SimHei 大事件 \tabularnewline
  \midrule
  \endhead
  \midrule
  元年 & 1212 & \tabularnewline\hline
  二年 & 1213 & \tabularnewline
  \bottomrule
\end{longtable}

\subsection{至宁}

\begin{longtable}{|>{\centering\scriptsize}m{2em}|>{\centering\scriptsize}m{1.3em}|>{\centering}m{8.8em}|}
  % \caption{秦王政}\
  \toprule
  \SimHei \normalsize 年数 & \SimHei \scriptsize 公元 & \SimHei 大事件 \tabularnewline
  % \midrule
  \endfirsthead
  \toprule
  \SimHei \normalsize 年数 & \SimHei \scriptsize 公元 & \SimHei 大事件 \tabularnewline
  \midrule
  \endhead
  \midrule
  元年 & 1213 & \tabularnewline
  \bottomrule
\end{longtable}


%%% Local Variables:
%%% mode: latex
%%% TeX-engine: xetex
%%% TeX-master: "../Main"
%%% End:

%% -*- coding: utf-8 -*-
%% Time-stamp: <Chen Wang: 2018-07-12 19:48:07>

\section{宣宗\tiny(1213-1224)}

\subsection{贞祐}


\begin{longtable}{|>{\centering\scriptsize}m{2em}|>{\centering\scriptsize}m{1.3em}|>{\centering}m{8.8em}|}
  % \caption{秦王政}\
  \toprule
  \SimHei \normalsize 年数 & \SimHei \scriptsize 公元 & \SimHei 大事件 \tabularnewline
  % \midrule
  \endfirsthead
  \toprule
  \SimHei \normalsize 年数 & \SimHei \scriptsize 公元 & \SimHei 大事件 \tabularnewline
  \midrule
  \endhead
  \midrule
  元年 & 1213 & \tabularnewline\hline
  二年 & 1214 & \tabularnewline\hline
  三年 & 1215 & \tabularnewline\hline
  四年 & 1216 & \tabularnewline\hline
  五年 & 1217 & \tabularnewline
  \bottomrule
\end{longtable}

\subsection{兴定}

\begin{longtable}{|>{\centering\scriptsize}m{2em}|>{\centering\scriptsize}m{1.3em}|>{\centering}m{8.8em}|}
  % \caption{秦王政}\
  \toprule
  \SimHei \normalsize 年数 & \SimHei \scriptsize 公元 & \SimHei 大事件 \tabularnewline
  % \midrule
  \endfirsthead
  \toprule
  \SimHei \normalsize 年数 & \SimHei \scriptsize 公元 & \SimHei 大事件 \tabularnewline
  \midrule
  \endhead
  \midrule
  元年 & 1217 & \tabularnewline\hline
  二年 & 1218 & \tabularnewline\hline
  三年 & 1219 & \tabularnewline\hline
  四年 & 1220 & \tabularnewline\hline
  五年 & 1221 & \tabularnewline\hline
  六年 & 1222 & \tabularnewline
  \bottomrule
\end{longtable}

\subsection{元光}

\begin{longtable}{|>{\centering\scriptsize}m{2em}|>{\centering\scriptsize}m{1.3em}|>{\centering}m{8.8em}|}
  % \caption{秦王政}\
  \toprule
  \SimHei \normalsize 年数 & \SimHei \scriptsize 公元 & \SimHei 大事件 \tabularnewline
  % \midrule
  \endfirsthead
  \toprule
  \SimHei \normalsize 年数 & \SimHei \scriptsize 公元 & \SimHei 大事件 \tabularnewline
  \midrule
  \endhead
  \midrule
  元年 & 1222 & \tabularnewline\hline
  二年 & 1223 & \tabularnewline
  \bottomrule
\end{longtable}


%%% Local Variables:
%%% mode: latex
%%% TeX-engine: xetex
%%% TeX-master: "../Main"
%%% End:

%% -*- coding: utf-8 -*-
%% Time-stamp: <Chen Wang: 2018-07-12 19:49:23>

\section{哀宗\tiny(1224-1234)}

\subsection{正大}


\begin{longtable}{|>{\centering\scriptsize}m{2em}|>{\centering\scriptsize}m{1.3em}|>{\centering}m{8.8em}|}
  % \caption{秦王政}\
  \toprule
  \SimHei \normalsize 年数 & \SimHei \scriptsize 公元 & \SimHei 大事件 \tabularnewline
  % \midrule
  \endfirsthead
  \toprule
  \SimHei \normalsize 年数 & \SimHei \scriptsize 公元 & \SimHei 大事件 \tabularnewline
  \midrule
  \endhead
  \midrule
  元年 & 1224 & \tabularnewline\hline
  二年 & 1225 & \tabularnewline\hline
  三年 & 1226 & \tabularnewline\hline
  四年 & 1227 & \tabularnewline\hline
  五年 & 1228 & \tabularnewline\hline
  六年 & 1229 & \tabularnewline\hline
  七年 & 1230 & \tabularnewline\hline
  八年 & 1231 & \tabularnewline
  \bottomrule
\end{longtable}

\subsection{开兴}

\begin{longtable}{|>{\centering\scriptsize}m{2em}|>{\centering\scriptsize}m{1.3em}|>{\centering}m{8.8em}|}
  % \caption{秦王政}\
  \toprule
  \SimHei \normalsize 年数 & \SimHei \scriptsize 公元 & \SimHei 大事件 \tabularnewline
  % \midrule
  \endfirsthead
  \toprule
  \SimHei \normalsize 年数 & \SimHei \scriptsize 公元 & \SimHei 大事件 \tabularnewline
  \midrule
  \endhead
  \midrule
  元年 & 1232 & \tabularnewline
  \bottomrule
\end{longtable}

\subsection{天兴}

\begin{longtable}{|>{\centering\scriptsize}m{2em}|>{\centering\scriptsize}m{1.3em}|>{\centering}m{8.8em}|}
  % \caption{秦王政}\
  \toprule
  \SimHei \normalsize 年数 & \SimHei \scriptsize 公元 & \SimHei 大事件 \tabularnewline
  % \midrule
  \endfirsthead
  \toprule
  \SimHei \normalsize 年数 & \SimHei \scriptsize 公元 & \SimHei 大事件 \tabularnewline
  \midrule
  \endhead
  \midrule
  元年 & 1232 & \tabularnewline\hline
  二年 & 1233 & \tabularnewline\hline
  三年 & 1234 & \tabularnewline
  \bottomrule
\end{longtable}


%%% Local Variables:
%%% mode: latex
%%% TeX-engine: xetex
%%% TeX-master: "../Main"
%%% End:



%%% Local Variables:
%%% mode: latex
%%% TeX-engine: xetex
%%% TeX-master: "../Main"
%%% End:

% %% -*- coding: utf-8 -*-
%% Time-stamp: <Chen Wang: 2018-07-12 20:08:52>

\chapter{元\tiny(1271-1368)}

%% -*- coding: utf-8 -*-
%% Time-stamp: <Chen Wang: 2018-07-12 19:54:51>

\section{世祖\tiny(1260-1294)}

\subsection{中统}

\begin{longtable}{|>{\centering\scriptsize}m{2em}|>{\centering\scriptsize}m{1.3em}|>{\centering}m{8.8em}|}
  % \caption{秦王政}\
  \toprule
  \SimHei \normalsize 年数 & \SimHei \scriptsize 公元 & \SimHei 大事件 \tabularnewline
  % \midrule
  \endfirsthead
  \toprule
  \SimHei \normalsize 年数 & \SimHei \scriptsize 公元 & \SimHei 大事件 \tabularnewline
  \midrule
  \endhead
  \midrule
  元年 & 1260 & \tabularnewline\hline
  二年 & 1261 & \tabularnewline\hline
  三年 & 1262 & \tabularnewline\hline
  四年 & 1263 & \tabularnewline\hline
  五年 & 1264 & \tabularnewline
  \bottomrule
\end{longtable}

\subsection{至元}

\begin{longtable}{|>{\centering\scriptsize}m{2em}|>{\centering\scriptsize}m{1.3em}|>{\centering}m{8.8em}|}
  % \caption{秦王政}\
  \toprule
  \SimHei \normalsize 年数 & \SimHei \scriptsize 公元 & \SimHei 大事件 \tabularnewline
  % \midrule
  \endfirsthead
  \toprule
  \SimHei \normalsize 年数 & \SimHei \scriptsize 公元 & \SimHei 大事件 \tabularnewline
  \midrule
  \endhead
  \midrule
  元年 & 1264 & \tabularnewline\hline
  二年 & 1265 & \tabularnewline\hline
  三年 & 1266 & \tabularnewline\hline
  四年 & 1267 & \tabularnewline\hline
  五年 & 1268 & \tabularnewline\hline
  六年 & 1269 & \tabularnewline\hline
  七年 & 1270 & \tabularnewline\hline
  八年 & 1271 & \tabularnewline\hline
  九年 & 1272 & \tabularnewline\hline
  十年 & 1273 & \tabularnewline\hline
  十一年 & 1274 & \tabularnewline\hline
  十二年 & 1275 & \tabularnewline\hline
  十三年 & 1276 & \tabularnewline\hline
  十四年 & 1277 & \tabularnewline\hline
  十五年 & 1278 & \tabularnewline\hline
  十六年 & 1279 & \tabularnewline\hline
  十七年 & 1280 & \tabularnewline\hline
  十八年 & 1281 & \tabularnewline\hline
  十九年 & 1282 & \tabularnewline\hline
  二十年 & 1283 & \tabularnewline\hline
  二一年 & 1284 & \tabularnewline\hline
  二二年 & 1285 & \tabularnewline\hline
  二三年 & 1286 & \tabularnewline\hline
  二四年 & 1287 & \tabularnewline\hline
  二五年 & 1288 & \tabularnewline\hline
  二六年 & 1289 & \tabularnewline\hline
  二七年 & 1290 & \tabularnewline\hline
  二八年 & 1291 & \tabularnewline\hline
  二九年 & 1292 & \tabularnewline\hline
  三十年 & 1293 & \tabularnewline\hline
  三一年 & 1294 & \tabularnewline
  \bottomrule
\end{longtable}


%%% Local Variables:
%%% mode: latex
%%% TeX-engine: xetex
%%% TeX-master: "../Main"
%%% End:

%% -*- coding: utf-8 -*-
%% Time-stamp: <Chen Wang: 2018-07-12 19:55:50>

\section{成宗\tiny(1294-1307)}

\subsection{元贞}

\begin{longtable}{|>{\centering\scriptsize}m{2em}|>{\centering\scriptsize}m{1.3em}|>{\centering}m{8.8em}|}
  % \caption{秦王政}\
  \toprule
  \SimHei \normalsize 年数 & \SimHei \scriptsize 公元 & \SimHei 大事件 \tabularnewline
  % \midrule
  \endfirsthead
  \toprule
  \SimHei \normalsize 年数 & \SimHei \scriptsize 公元 & \SimHei 大事件 \tabularnewline
  \midrule
  \endhead
  \midrule
  元年 & 1295 & \tabularnewline\hline
  二年 & 1296 & \tabularnewline\hline
  三年 & 1297 & \tabularnewline
  \bottomrule
\end{longtable}

\subsection{大德}

\begin{longtable}{|>{\centering\scriptsize}m{2em}|>{\centering\scriptsize}m{1.3em}|>{\centering}m{8.8em}|}
  % \caption{秦王政}\
  \toprule
  \SimHei \normalsize 年数 & \SimHei \scriptsize 公元 & \SimHei 大事件 \tabularnewline
  % \midrule
  \endfirsthead
  \toprule
  \SimHei \normalsize 年数 & \SimHei \scriptsize 公元 & \SimHei 大事件 \tabularnewline
  \midrule
  \endhead
  \midrule
  元年 & 1297 & \tabularnewline\hline
  二年 & 1298 & \tabularnewline\hline
  三年 & 1299 & \tabularnewline\hline
  四年 & 1300 & \tabularnewline\hline
  五年 & 1301 & \tabularnewline\hline
  六年 & 1302 & \tabularnewline\hline
  七年 & 1303 & \tabularnewline\hline
  八年 & 1304 & \tabularnewline\hline
  九年 & 1305 & \tabularnewline\hline
  十年 & 1306 & \tabularnewline\hline
  十一年 & 1307 & \tabularnewline
  \bottomrule
\end{longtable}


%%% Local Variables:
%%% mode: latex
%%% TeX-engine: xetex
%%% TeX-master: "../Main"
%%% End:

%% -*- coding: utf-8 -*-
%% Time-stamp: <Chen Wang: 2018-07-12 19:56:41>

\section{武宗\tiny(1307-1311)}

\subsection{至大}

\begin{longtable}{|>{\centering\scriptsize}m{2em}|>{\centering\scriptsize}m{1.3em}|>{\centering}m{8.8em}|}
  % \caption{秦王政}\
  \toprule
  \SimHei \normalsize 年数 & \SimHei \scriptsize 公元 & \SimHei 大事件 \tabularnewline
  % \midrule
  \endfirsthead
  \toprule
  \SimHei \normalsize 年数 & \SimHei \scriptsize 公元 & \SimHei 大事件 \tabularnewline
  \midrule
  \endhead
  \midrule
  元年 & 1308 & \tabularnewline\hline
  二年 & 1309 & \tabularnewline\hline
  三年 & 1310 & \tabularnewline\hline
  四年 & 1311 & \tabularnewline
  \bottomrule
\end{longtable}


%%% Local Variables:
%%% mode: latex
%%% TeX-engine: xetex
%%% TeX-master: "../Main"
%%% End:

%% -*- coding: utf-8 -*-
%% Time-stamp: <Chen Wang: 2018-07-12 19:57:47>

\section{仁宗\tiny(1311-1320)}

\subsection{皇庆}

\begin{longtable}{|>{\centering\scriptsize}m{2em}|>{\centering\scriptsize}m{1.3em}|>{\centering}m{8.8em}|}
  % \caption{秦王政}\
  \toprule
  \SimHei \normalsize 年数 & \SimHei \scriptsize 公元 & \SimHei 大事件 \tabularnewline
  % \midrule
  \endfirsthead
  \toprule
  \SimHei \normalsize 年数 & \SimHei \scriptsize 公元 & \SimHei 大事件 \tabularnewline
  \midrule
  \endhead
  \midrule
  元年 & 1312 & \tabularnewline\hline
  二年 & 1313 & \tabularnewline
  \bottomrule
\end{longtable}

\subsection{延祐}

\begin{longtable}{|>{\centering\scriptsize}m{2em}|>{\centering\scriptsize}m{1.3em}|>{\centering}m{8.8em}|}
  % \caption{秦王政}\
  \toprule
  \SimHei \normalsize 年数 & \SimHei \scriptsize 公元 & \SimHei 大事件 \tabularnewline
  % \midrule
  \endfirsthead
  \toprule
  \SimHei \normalsize 年数 & \SimHei \scriptsize 公元 & \SimHei 大事件 \tabularnewline
  \midrule
  \endhead
  \midrule
  元年 & 1314 & \tabularnewline\hline
  二年 & 1315 & \tabularnewline\hline
  三年 & 1316 & \tabularnewline\hline
  四年 & 1317 & \tabularnewline\hline
  五年 & 1318 & \tabularnewline\hline
  六年 & 1319 & \tabularnewline\hline
  七年 & 1320 & \tabularnewline
  \bottomrule
\end{longtable}


%%% Local Variables:
%%% mode: latex
%%% TeX-engine: xetex
%%% TeX-master: "../Main"
%%% End:

%% -*- coding: utf-8 -*-
%% Time-stamp: <Chen Wang: 2018-07-12 19:58:41>

\section{英宗\tiny(1320-1323)}

\subsection{志治}

\begin{longtable}{|>{\centering\scriptsize}m{2em}|>{\centering\scriptsize}m{1.3em}|>{\centering}m{8.8em}|}
  % \caption{秦王政}\
  \toprule
  \SimHei \normalsize 年数 & \SimHei \scriptsize 公元 & \SimHei 大事件 \tabularnewline
  % \midrule
  \endfirsthead
  \toprule
  \SimHei \normalsize 年数 & \SimHei \scriptsize 公元 & \SimHei 大事件 \tabularnewline
  \midrule
  \endhead
  \midrule
  元年 & 1321 & \tabularnewline\hline
  二年 & 1322 & \tabularnewline\hline
  三年 & 1323 & \tabularnewline
  \bottomrule
\end{longtable}


%%% Local Variables:
%%% mode: latex
%%% TeX-engine: xetex
%%% TeX-master: "../Main"
%%% End:

%% -*- coding: utf-8 -*-
%% Time-stamp: <Chen Wang: 2018-07-12 19:59:50>

\section{泰定帝\tiny(1323-1328)}

\subsection{泰定}

\begin{longtable}{|>{\centering\scriptsize}m{2em}|>{\centering\scriptsize}m{1.3em}|>{\centering}m{8.8em}|}
  % \caption{秦王政}\
  \toprule
  \SimHei \normalsize 年数 & \SimHei \scriptsize 公元 & \SimHei 大事件 \tabularnewline
  % \midrule
  \endfirsthead
  \toprule
  \SimHei \normalsize 年数 & \SimHei \scriptsize 公元 & \SimHei 大事件 \tabularnewline
  \midrule
  \endhead
  \midrule
  元年 & 1324 & \tabularnewline\hline
  二年 & 1325 & \tabularnewline\hline
  三年 & 1326 & \tabularnewline\hline
  四年 & 1327 & \tabularnewline\hline
  五年 & 1328 & \tabularnewline
  \bottomrule
\end{longtable}

\subsection{致和}

\begin{longtable}{|>{\centering\scriptsize}m{2em}|>{\centering\scriptsize}m{1.3em}|>{\centering}m{8.8em}|}
  % \caption{秦王政}\
  \toprule
  \SimHei \normalsize 年数 & \SimHei \scriptsize 公元 & \SimHei 大事件 \tabularnewline
  % \midrule
  \endfirsthead
  \toprule
  \SimHei \normalsize 年数 & \SimHei \scriptsize 公元 & \SimHei 大事件 \tabularnewline
  \midrule
  \endhead
  \midrule
  元年 & 1328 & \tabularnewline
  \bottomrule
\end{longtable}


%%% Local Variables:
%%% mode: latex
%%% TeX-engine: xetex
%%% TeX-master: "../Main"
%%% End:

%% -*- coding: utf-8 -*-
%% Time-stamp: <Chen Wang: 2018-07-12 20:00:28>

\section{天顺帝\tiny(1328)}

\subsection{天顺}

\begin{longtable}{|>{\centering\scriptsize}m{2em}|>{\centering\scriptsize}m{1.3em}|>{\centering}m{8.8em}|}
  % \caption{秦王政}\
  \toprule
  \SimHei \normalsize 年数 & \SimHei \scriptsize 公元 & \SimHei 大事件 \tabularnewline
  % \midrule
  \endfirsthead
  \toprule
  \SimHei \normalsize 年数 & \SimHei \scriptsize 公元 & \SimHei 大事件 \tabularnewline
  \midrule
  \endhead
  \midrule
  元年 & 1328 & \tabularnewline
  \bottomrule
\end{longtable}


%%% Local Variables:
%%% mode: latex
%%% TeX-engine: xetex
%%% TeX-master: "../Main"
%%% End:

%% -*- coding: utf-8 -*-
%% Time-stamp: <Chen Wang: 2018-07-12 20:03:01>

\section{文宗\tiny(1328-1332)}

\subsection{天历}

\begin{longtable}{|>{\centering\scriptsize}m{2em}|>{\centering\scriptsize}m{1.3em}|>{\centering}m{8.8em}|}
  % \caption{秦王政}\
  \toprule
  \SimHei \normalsize 年数 & \SimHei \scriptsize 公元 & \SimHei 大事件 \tabularnewline
  % \midrule
  \endfirsthead
  \toprule
  \SimHei \normalsize 年数 & \SimHei \scriptsize 公元 & \SimHei 大事件 \tabularnewline
  \midrule
  \endhead
  \midrule
  元年 & 1328 & \tabularnewline\hline
  二年 & 1329 & \tabularnewline\hline
  三年 & 1330 & \tabularnewline
  \bottomrule
\end{longtable}

\subsection{志顺}

\begin{longtable}{|>{\centering\scriptsize}m{2em}|>{\centering\scriptsize}m{1.3em}|>{\centering}m{8.8em}|}
  % \caption{秦王政}\
  \toprule
  \SimHei \normalsize 年数 & \SimHei \scriptsize 公元 & \SimHei 大事件 \tabularnewline
  % \midrule
  \endfirsthead
  \toprule
  \SimHei \normalsize 年数 & \SimHei \scriptsize 公元 & \SimHei 大事件 \tabularnewline
  \midrule
  \endhead
  \midrule
  元年 & 1330 & \tabularnewline\hline
  二年 & 1331 & \tabularnewline\hline
  三年 & 1332 & \tabularnewline\hline
  四年 & 1333 & \tabularnewline
  \bottomrule
\end{longtable}


%%% Local Variables:
%%% mode: latex
%%% TeX-engine: xetex
%%% TeX-master: "../Main"
%%% End:

\input{Yuan/HuiZOng}
%% -*- coding: utf-8 -*-
%% Time-stamp: <Chen Wang: 2018-07-12 20:07:53>

\section{北元\tiny(1368-1388)}

\subsection{昭宗\tiny(1370-1368)}

\subsubsection{宣光}

\begin{longtable}{|>{\centering\scriptsize}m{2em}|>{\centering\scriptsize}m{1.3em}|>{\centering}m{8.8em}|}
  % \caption{秦王政}\
  \toprule
  \SimHei \normalsize 年数 & \SimHei \scriptsize 公元 & \SimHei 大事件 \tabularnewline
  % \midrule
  \endfirsthead
  \toprule
  \SimHei \normalsize 年数 & \SimHei \scriptsize 公元 & \SimHei 大事件 \tabularnewline
  \midrule
  \endhead
  \midrule
  元年 & 1371 & \tabularnewline\hline
  二年 & 1372 & \tabularnewline\hline
  三年 & 1373 & \tabularnewline\hline
  四年 & 1374 & \tabularnewline\hline
  五年 & 1375 & \tabularnewline\hline
  六年 & 1376 & \tabularnewline\hline
  七年 & 1377 & \tabularnewline\hline
  八年 & 1378 & \tabularnewline\hline
  九年 & 1379 & \tabularnewline
  \bottomrule
\end{longtable}

\subsection{后主\tiny(1378-1388)}

\subsubsection{天元}


\begin{longtable}{|>{\centering\scriptsize}m{2em}|>{\centering\scriptsize}m{1.3em}|>{\centering}m{8.8em}|}
  % \caption{秦王政}\
  \toprule
  \SimHei \normalsize 年数 & \SimHei \scriptsize 公元 & \SimHei 大事件 \tabularnewline
  % \midrule
  \endfirsthead
  \toprule
  \SimHei \normalsize 年数 & \SimHei \scriptsize 公元 & \SimHei 大事件 \tabularnewline
  \midrule
  \endhead
  \midrule
  元年 & 1379 & \tabularnewline\hline
  二年 & 1380 & \tabularnewline\hline
  三年 & 1381 & \tabularnewline\hline
  四年 & 1382 & \tabularnewline\hline
  五年 & 1383 & \tabularnewline\hline
  六年 & 1384 & \tabularnewline\hline
  七年 & 1385 & \tabularnewline\hline
  八年 & 1386 & \tabularnewline\hline
  九年 & 1387 & \tabularnewline\hline
  十年 & 1388 & \tabularnewline
  \bottomrule
\end{longtable}


%%% Local Variables:
%%% mode: latex
%%% TeX-engine: xetex
%%% TeX-master: "../Main"
%%% End:



%%% Local Variables:
%%% mode: latex
%%% TeX-engine: xetex
%%% TeX-master: "../Main"
%%% End:

% %% -*- coding: utf-8 -*-
%% Time-stamp: <Chen Wang: 2018-07-12 20:14:38>

\chapter{明\tiny(1368-1644)}

%% -*- coding: utf-8 -*-
%% Time-stamp: <Chen Wang: 2018-07-12 20:12:51>

\section{太祖\tiny(1368-1398)}

\subsection{洪武}

\begin{longtable}{|>{\centering\scriptsize}m{2em}|>{\centering\scriptsize}m{1.3em}|>{\centering}m{8.8em}|}
  % \caption{秦王政}\
  \toprule
  \SimHei \normalsize 年数 & \SimHei \scriptsize 公元 & \SimHei 大事件 \tabularnewline
  % \midrule
  \endfirsthead
  \toprule
  \SimHei \normalsize 年数 & \SimHei \scriptsize 公元 & \SimHei 大事件 \tabularnewline
  \midrule
  \endhead
  \midrule
  元年 & 1368 & \tabularnewline\hline
  二年 & 1369 & \tabularnewline\hline
  三年 & 1370 & \tabularnewline\hline
  四年 & 1371 & \tabularnewline\hline
  五年 & 1372 & \tabularnewline\hline
  六年 & 1373 & \tabularnewline\hline
  七年 & 1374 & \tabularnewline\hline
  八年 & 1375 & \tabularnewline\hline
  九年 & 1376 & \tabularnewline\hline
  十年 & 1377 & \tabularnewline\hline
  十一年 & 1378 & \tabularnewline\hline
  十二年 & 1379 & \tabularnewline\hline
  十三年 & 1380 & \tabularnewline\hline
  十四年 & 1381 & \tabularnewline\hline
  十五年 & 1382 & \tabularnewline\hline
  十六年 & 1383 & \tabularnewline\hline
  十七年 & 1384 & \tabularnewline\hline
  十八年 & 1385 & \tabularnewline\hline
  十九年 & 1386 & \tabularnewline\hline
  二十年 & 1387 & \tabularnewline\hline
  二一年 & 1388 & \tabularnewline\hline
  二二年 & 1389 & \tabularnewline\hline
  二三年 & 1390 & \tabularnewline\hline
  二四年 & 1391 & \tabularnewline\hline
  二五年 & 1392 & \tabularnewline\hline
  二六年 & 1393 & \tabularnewline\hline
  二七年 & 1394 & \tabularnewline\hline
  二八年 & 1395 & \tabularnewline\hline
  二九年 & 1396 & \tabularnewline\hline
  三十年 & 1397 & \tabularnewline\hline
  三一年 & 1398 & \tabularnewline
  \bottomrule
\end{longtable}


%%% Local Variables:
%%% mode: latex
%%% TeX-engine: xetex
%%% TeX-master: "../Main"
%%% End:

%% -*- coding: utf-8 -*-
%% Time-stamp: <Chen Wang: 2018-07-12 20:13:29>

\section{惠宗\tiny(1398-1402)}

\subsection{建文}

\begin{longtable}{|>{\centering\scriptsize}m{2em}|>{\centering\scriptsize}m{1.3em}|>{\centering}m{8.8em}|}
  % \caption{秦王政}\
  \toprule
  \SimHei \normalsize 年数 & \SimHei \scriptsize 公元 & \SimHei 大事件 \tabularnewline
  % \midrule
  \endfirsthead
  \toprule
  \SimHei \normalsize 年数 & \SimHei \scriptsize 公元 & \SimHei 大事件 \tabularnewline
  \midrule
  \endhead
  \midrule
  元年 & 1399 & \tabularnewline\hline
  二年 & 1400 & \tabularnewline\hline
  三年 & 1401 & \tabularnewline\hline
  四年 & 1402 & \tabularnewline
  \bottomrule
\end{longtable}


%%% Local Variables:
%%% mode: latex
%%% TeX-engine: xetex
%%% TeX-master: "../Main"
%%% End:

%% -*- coding: utf-8 -*-
%% Time-stamp: <Chen Wang: 2018-07-12 20:15:45>

\section{成祖\tiny(1402-1424)}

\subsection{洪武}

\begin{longtable}{|>{\centering\scriptsize}m{2em}|>{\centering\scriptsize}m{1.3em}|>{\centering}m{8.8em}|}
  % \caption{秦王政}\
  \toprule
  \SimHei \normalsize 年数 & \SimHei \scriptsize 公元 & \SimHei 大事件 \tabularnewline
  % \midrule
  \endfirsthead
  \toprule
  \SimHei \normalsize 年数 & \SimHei \scriptsize 公元 & \SimHei 大事件 \tabularnewline
  \midrule
  \endhead
  \midrule
  三五年 & 1402 & \tabularnewline
  \bottomrule
\end{longtable}

\subsection{永乐}

\begin{longtable}{|>{\centering\scriptsize}m{2em}|>{\centering\scriptsize}m{1.3em}|>{\centering}m{8.8em}|}
  % \caption{秦王政}\
  \toprule
  \SimHei \normalsize 年数 & \SimHei \scriptsize 公元 & \SimHei 大事件 \tabularnewline
  % \midrule
  \endfirsthead
  \toprule
  \SimHei \normalsize 年数 & \SimHei \scriptsize 公元 & \SimHei 大事件 \tabularnewline
  \midrule
  \endhead
  \midrule
  元年 & 1403 & \tabularnewline\hline
  二年 & 1404 & \tabularnewline\hline
  三年 & 1405 & \tabularnewline\hline
  四年 & 1406 & \tabularnewline\hline
  五年 & 1407 & \tabularnewline\hline
  六年 & 1408 & \tabularnewline\hline
  七年 & 1409 & \tabularnewline\hline
  八年 & 1410 & \tabularnewline\hline
  九年 & 1411 & \tabularnewline\hline
  十年 & 1412 & \tabularnewline\hline
  十一年 & 1413 & \tabularnewline\hline
  十二年 & 1414 & \tabularnewline\hline
  十三年 & 1415 & \tabularnewline\hline
  十四年 & 1416 & \tabularnewline\hline
  十五年 & 1417 & \tabularnewline\hline
  十六年 & 1418 & \tabularnewline\hline
  十七年 & 1419 & \tabularnewline\hline
  十八年 & 1420 & \tabularnewline\hline
  十九年 & 1421 & \tabularnewline\hline
  二十年 & 1422 & \tabularnewline\hline
  二一年 & 1423 & \tabularnewline\hline
  二二年 & 1424 & \tabularnewline
  \bottomrule
\end{longtable}


%%% Local Variables:
%%% mode: latex
%%% TeX-engine: xetex
%%% TeX-master: "../Main"
%%% End:



%%% Local Variables:
%%% mode: latex
%%% TeX-engine: xetex
%%% TeX-master: "../Main"
%%% End:

% %% -*- coding: utf-8 -*-
%% Time-stamp: <Chen Wang: 2018-07-12 22:23:37>

\chapter{清\tiny(1636-1912)}

%% -*- coding: utf-8 -*-
%% Time-stamp: <Chen Wang: 2018-07-12 22:07:49>

\section{后金\tiny(1616-1636)}

\subsection{努尔哈赤\tiny(1616-1626)}

\subsubsection{天命}

\begin{longtable}{|>{\centering\scriptsize}m{2em}|>{\centering\scriptsize}m{1.3em}|>{\centering}m{8.8em}|}
  % \caption{秦王政}\
  \toprule
  \SimHei \normalsize 年数 & \SimHei \scriptsize 公元 & \SimHei 大事件 \tabularnewline
  % \midrule
  \endfirsthead
  \toprule
  \SimHei \normalsize 年数 & \SimHei \scriptsize 公元 & \SimHei 大事件 \tabularnewline
  \midrule
  \endhead
  \midrule
  元年 & 1616 & \tabularnewline\hline
  二年 & 1617 & \tabularnewline\hline
  三年 & 1618 & \tabularnewline\hline
  四年 & 1619 & \tabularnewline\hline
  五年 & 1620 & \tabularnewline\hline
  六年 & 1621 & \tabularnewline\hline
  七年 & 1622 & \tabularnewline\hline
  八年 & 1623 & \tabularnewline\hline
  九年 & 1624 & \tabularnewline\hline
  十年 & 1625 & \tabularnewline\hline
  十一年 & 1626 & \tabularnewline
  \bottomrule
\end{longtable}

\subsection{皇太极\tiny(1626-1636)}

\subsubsection{天聪}

\begin{longtable}{|>{\centering\scriptsize}m{2em}|>{\centering\scriptsize}m{1.3em}|>{\centering}m{8.8em}|}
  % \caption{秦王政}\
  \toprule
  \SimHei \normalsize 年数 & \SimHei \scriptsize 公元 & \SimHei 大事件 \tabularnewline
  % \midrule
  \endfirsthead
  \toprule
  \SimHei \normalsize 年数 & \SimHei \scriptsize 公元 & \SimHei 大事件 \tabularnewline
  \midrule
  \endhead
  \midrule
  元年 & 1627 & \tabularnewline\hline
  二年 & 1628 & \tabularnewline\hline
  三年 & 1629 & \tabularnewline\hline
  四年 & 1630 & \tabularnewline\hline
  五年 & 1631 & \tabularnewline\hline
  六年 & 1632 & \tabularnewline\hline
  七年 & 1633 & \tabularnewline\hline
  八年 & 1634 & \tabularnewline\hline
  九年 & 1635 & \tabularnewline\hline
  十年 & 1636 & \tabularnewline
  \bottomrule
\end{longtable}


%%% Local Variables:
%%% mode: latex
%%% TeX-engine: xetex
%%% TeX-master: "../Main"
%%% End:

%% -*- coding: utf-8 -*-
%% Time-stamp: <Chen Wang: 2018-07-12 22:09:47>

\section{太宗\tiny(1626-1643)}

\subsection{崇德}

\begin{longtable}{|>{\centering\scriptsize}m{2em}|>{\centering\scriptsize}m{1.3em}|>{\centering}m{8.8em}|}
  % \caption{秦王政}\
  \toprule
  \SimHei \normalsize 年数 & \SimHei \scriptsize 公元 & \SimHei 大事件 \tabularnewline
  % \midrule
  \endfirsthead
  \toprule
  \SimHei \normalsize 年数 & \SimHei \scriptsize 公元 & \SimHei 大事件 \tabularnewline
  \midrule
  \endhead
  \midrule
  元年 & 1636 & \tabularnewline\hline
  二年 & 1637 & \tabularnewline\hline
  三年 & 1638 & \tabularnewline\hline
  四年 & 1639 & \tabularnewline\hline
  五年 & 1640 & \tabularnewline\hline
  六年 & 1641 & \tabularnewline\hline
  七年 & 1642 & \tabularnewline\hline
  八年 & 1643 & \tabularnewline
  \bottomrule
\end{longtable}


%%% Local Variables:
%%% mode: latex
%%% TeX-engine: xetex
%%% TeX-master: "../Main"
%%% End:

%% -*- coding: utf-8 -*-
%% Time-stamp: <Chen Wang: 2018-07-12 22:14:50>

\section{世祖\tiny(1643-1661)}

\subsection{顺治}

\begin{longtable}{|>{\centering\scriptsize}m{2em}|>{\centering\scriptsize}m{1.3em}|>{\centering}m{8.8em}|}
  % \caption{秦王政}\
  \toprule
  \SimHei \normalsize 年数 & \SimHei \scriptsize 公元 & \SimHei 大事件 \tabularnewline
  % \midrule
  \endfirsthead
  \toprule
  \SimHei \normalsize 年数 & \SimHei \scriptsize 公元 & \SimHei 大事件 \tabularnewline
  \midrule
  \endhead
  \midrule
  元年 & 1644 & \tabularnewline\hline
  二年 & 1645 & \tabularnewline\hline
  三年 & 1646 & \tabularnewline\hline
  四年 & 1647 & \tabularnewline\hline
  五年 & 1648 & \tabularnewline\hline
  六年 & 1649 & \tabularnewline\hline
  七年 & 1650 & \tabularnewline\hline
  八年 & 1651 & \tabularnewline\hline
  九年 & 1652 & \tabularnewline\hline
  十年 & 1653 & \tabularnewline\hline
  十一年 & 1654 & \tabularnewline\hline
  十二年 & 1655 & \tabularnewline\hline
  十三年 & 1656 & \tabularnewline\hline
  十四年 & 1657 & \tabularnewline\hline
  十五年 & 1658 & \tabularnewline\hline
  十六年 & 1659 & \tabularnewline\hline
  十七年 & 1660 & \tabularnewline\hline
  十八年 & 1661 & \tabularnewline
  \bottomrule
\end{longtable}


%%% Local Variables:
%%% mode: latex
%%% TeX-engine: xetex
%%% TeX-master: "../Main"
%%% End:

%% -*- coding: utf-8 -*-
%% Time-stamp: <Chen Wang: 2018-07-12 22:16:11>

\section{圣祖\tiny(1661-1722)}

\subsection{康熙}

\begin{longtable}{|>{\centering\scriptsize}m{2em}|>{\centering\scriptsize}m{1.3em}|>{\centering}m{8.8em}|}
  % \caption{秦王政}\
  \toprule
  \SimHei \normalsize 年数 & \SimHei \scriptsize 公元 & \SimHei 大事件 \tabularnewline
  % \midrule
  \endfirsthead
  \toprule
  \SimHei \normalsize 年数 & \SimHei \scriptsize 公元 & \SimHei 大事件 \tabularnewline
  \midrule
  \endhead
  \midrule
  元年 & 1662 & \tabularnewline\hline
  二年 & 1663 & \tabularnewline\hline
  三年 & 1664 & \tabularnewline\hline
  四年 & 1665 & \tabularnewline\hline
  五年 & 1666 & \tabularnewline\hline
  六年 & 1667 & \tabularnewline\hline
  七年 & 1668 & \tabularnewline\hline
  八年 & 1669 & \tabularnewline\hline
  九年 & 1670 & \tabularnewline\hline
  十年 & 1671 & \tabularnewline\hline
  十一年 & 1672 & \tabularnewline\hline
  十二年 & 1673 & \tabularnewline\hline
  十三年 & 1674 & \tabularnewline\hline
  十四年 & 1675 & \tabularnewline\hline
  十五年 & 1676 & \tabularnewline\hline
  十六年 & 1677 & \tabularnewline\hline
  十七年 & 1678 & \tabularnewline\hline
  十八年 & 1679 & \tabularnewline\hline
  十九年 & 1680 & \tabularnewline\hline
  二十年 & 1681 & \tabularnewline\hline
  二一年 & 1682 & \tabularnewline\hline
  二二年 & 1683 & \tabularnewline\hline
  二三年 & 1684 & \tabularnewline\hline
  二四年 & 1685 & \tabularnewline\hline
  二五年 & 1686 & \tabularnewline\hline
  二六年 & 1687 & \tabularnewline\hline
  二七年 & 1688 & \tabularnewline\hline
  二八年 & 1689 & \tabularnewline\hline
  二九年 & 1690 & \tabularnewline\hline
  三十年 & 1691 & \tabularnewline\hline
  三一年 & 1692 & \tabularnewline\hline
  三二年 & 1693 & \tabularnewline\hline
  三三年 & 1694 & \tabularnewline\hline
  三四年 & 1695 & \tabularnewline\hline
  三五年 & 1696 & \tabularnewline\hline
  三六年 & 1697 & \tabularnewline\hline
  三七年 & 1698 & \tabularnewline\hline
  三八年 & 1699 & \tabularnewline\hline
  三九年 & 1700 & \tabularnewline\hline
  四十年 & 1701 & \tabularnewline\hline
  四一年 & 1702 & \tabularnewline\hline
  四二年 & 1703 & \tabularnewline\hline
  四三年 & 1704 & \tabularnewline\hline
  四四年 & 1705 & \tabularnewline\hline
  四五年 & 1706 & \tabularnewline\hline
  四六年 & 1707 & \tabularnewline\hline
  四七年 & 1708 & \tabularnewline\hline
  四八年 & 1709 & \tabularnewline\hline
  四九年 & 1710 & \tabularnewline\hline
  五十年 & 1711 & \tabularnewline\hline
  五一年 & 1712 & \tabularnewline\hline
  五二年 & 1713 & \tabularnewline\hline
  五三年 & 1714 & \tabularnewline\hline
  五四年 & 1715 & \tabularnewline\hline
  五五年 & 1716 & \tabularnewline\hline
  五六年 & 1717 & \tabularnewline\hline
  五七年 & 1718 & \tabularnewline\hline
  五八年 & 1719 & \tabularnewline\hline
  五九年 & 1720 & \tabularnewline\hline
  六十年 & 1721 & \tabularnewline\hline
  六一年 & 1722 & \tabularnewline
  \bottomrule
\end{longtable}


%%% Local Variables:
%%% mode: latex
%%% TeX-engine: xetex
%%% TeX-master: "../Main"
%%% End:

%% -*- coding: utf-8 -*-
%% Time-stamp: <Chen Wang: 2018-07-12 22:16:54>

\section{世宗\tiny(1722-1735)}

\subsection{雍正}

\begin{longtable}{|>{\centering\scriptsize}m{2em}|>{\centering\scriptsize}m{1.3em}|>{\centering}m{8.8em}|}
  % \caption{秦王政}\
  \toprule
  \SimHei \normalsize 年数 & \SimHei \scriptsize 公元 & \SimHei 大事件 \tabularnewline
  % \midrule
  \endfirsthead
  \toprule
  \SimHei \normalsize 年数 & \SimHei \scriptsize 公元 & \SimHei 大事件 \tabularnewline
  \midrule
  \endhead
  \midrule
  元年 & 1723 & \tabularnewline\hline
  二年 & 1724 & \tabularnewline\hline
  三年 & 1725 & \tabularnewline\hline
  四年 & 1726 & \tabularnewline\hline
  五年 & 1727 & \tabularnewline\hline
  六年 & 1728 & \tabularnewline\hline
  七年 & 1729 & \tabularnewline\hline
  八年 & 1730 & \tabularnewline\hline
  九年 & 1731 & \tabularnewline\hline
  十年 & 1732 & \tabularnewline\hline
  十一年 & 1733 & \tabularnewline\hline
  十二年 & 1734 & \tabularnewline\hline
  十三年 & 1735 & \tabularnewline
  \bottomrule
\end{longtable}


%%% Local Variables:
%%% mode: latex
%%% TeX-engine: xetex
%%% TeX-master: "../Main"
%%% End:

%% -*- coding: utf-8 -*-
%% Time-stamp: <Chen Wang: 2018-07-12 22:17:39>

\section{高宗\tiny(1736-1795)}

\subsection{乾隆}

\begin{longtable}{|>{\centering\scriptsize}m{2em}|>{\centering\scriptsize}m{1.3em}|>{\centering}m{8.8em}|}
  % \caption{秦王政}\
  \toprule
  \SimHei \normalsize 年数 & \SimHei \scriptsize 公元 & \SimHei 大事件 \tabularnewline
  % \midrule
  \endfirsthead
  \toprule
  \SimHei \normalsize 年数 & \SimHei \scriptsize 公元 & \SimHei 大事件 \tabularnewline
  \midrule
  \endhead
  \midrule
  元年 & 1736 & \tabularnewline\hline
  二年 & 1737 & \tabularnewline\hline
  三年 & 1738 & \tabularnewline\hline
  四年 & 1739 & \tabularnewline\hline
  五年 & 1740 & \tabularnewline\hline
  六年 & 1741 & \tabularnewline\hline
  七年 & 1742 & \tabularnewline\hline
  八年 & 1743 & \tabularnewline\hline
  九年 & 1744 & \tabularnewline\hline
  十年 & 1745 & \tabularnewline\hline
  十一年 & 1746 & \tabularnewline\hline
  十二年 & 1747 & \tabularnewline\hline
  十三年 & 1748 & \tabularnewline\hline
  十四年 & 1749 & \tabularnewline\hline
  十五年 & 1750 & \tabularnewline\hline
  十六年 & 1751 & \tabularnewline\hline
  十七年 & 1752 & \tabularnewline\hline
  十八年 & 1753 & \tabularnewline\hline
  十九年 & 1754 & \tabularnewline\hline
  二十年 & 1755 & \tabularnewline\hline
  二一年 & 1756 & \tabularnewline\hline
  二二年 & 1757 & \tabularnewline\hline
  二三年 & 1758 & \tabularnewline\hline
  二四年 & 1759 & \tabularnewline\hline
  二五年 & 1760 & \tabularnewline\hline
  二六年 & 1761 & \tabularnewline\hline
  二七年 & 1762 & \tabularnewline\hline
  二八年 & 1763 & \tabularnewline\hline
  二九年 & 1764 & \tabularnewline\hline
  三十年 & 1765 & \tabularnewline\hline
  三一年 & 1766 & \tabularnewline\hline
  三二年 & 1767 & \tabularnewline\hline
  三三年 & 1768 & \tabularnewline\hline
  三四年 & 1769 & \tabularnewline\hline
  三五年 & 1770 & \tabularnewline\hline
  三六年 & 1771 & \tabularnewline\hline
  三七年 & 1772 & \tabularnewline\hline
  三八年 & 1773 & \tabularnewline\hline
  三九年 & 1774 & \tabularnewline\hline
  四十年 & 1775 & \tabularnewline\hline
  四一年 & 1776 & \tabularnewline\hline
  四二年 & 1777 & \tabularnewline\hline
  四三年 & 1778 & \tabularnewline\hline
  四四年 & 1779 & \tabularnewline\hline
  四五年 & 1780 & \tabularnewline\hline
  四六年 & 1781 & \tabularnewline\hline
  四七年 & 1782 & \tabularnewline\hline
  四八年 & 1783 & \tabularnewline\hline
  四九年 & 1784 & \tabularnewline\hline
  五十年 & 1785 & \tabularnewline\hline
  五一年 & 1786 & \tabularnewline\hline
  五二年 & 1787 & \tabularnewline\hline
  五三年 & 1788 & \tabularnewline\hline
  五四年 & 1789 & \tabularnewline\hline
  五五年 & 1790 & \tabularnewline\hline
  五六年 & 1791 & \tabularnewline\hline
  五七年 & 1792 & \tabularnewline\hline
  五八年 & 1793 & \tabularnewline\hline
  五九年 & 1794 & \tabularnewline\hline
  六十年 & 1795 & \tabularnewline
  \bottomrule
\end{longtable}


%%% Local Variables:
%%% mode: latex
%%% TeX-engine: xetex
%%% TeX-master: "../Main"
%%% End:

%% -*- coding: utf-8 -*-
%% Time-stamp: <Chen Wang: 2018-07-12 22:18:34>

\section{仁宗\tiny(1795-1820)}

\subsection{嘉庆}

\begin{longtable}{|>{\centering\scriptsize}m{2em}|>{\centering\scriptsize}m{1.3em}|>{\centering}m{8.8em}|}
  % \caption{秦王政}\
  \toprule
  \SimHei \normalsize 年数 & \SimHei \scriptsize 公元 & \SimHei 大事件 \tabularnewline
  % \midrule
  \endfirsthead
  \toprule
  \SimHei \normalsize 年数 & \SimHei \scriptsize 公元 & \SimHei 大事件 \tabularnewline
  \midrule
  \endhead
  \midrule
  元年 & 1796 & \tabularnewline\hline
  二年 & 1797 & \tabularnewline\hline
  三年 & 1798 & \tabularnewline\hline
  四年 & 1799 & \tabularnewline\hline
  五年 & 1800 & \tabularnewline\hline
  六年 & 1801 & \tabularnewline\hline
  七年 & 1802 & \tabularnewline\hline
  八年 & 1803 & \tabularnewline\hline
  九年 & 1804 & \tabularnewline\hline
  十年 & 1805 & \tabularnewline\hline
  十一年 & 1806 & \tabularnewline\hline
  十二年 & 1807 & \tabularnewline\hline
  十三年 & 1808 & \tabularnewline\hline
  十四年 & 1809 & \tabularnewline\hline
  十五年 & 1810 & \tabularnewline\hline
  十六年 & 1811 & \tabularnewline\hline
  十七年 & 1812 & \tabularnewline\hline
  十八年 & 1813 & \tabularnewline\hline
  十九年 & 1814 & \tabularnewline\hline
  二十年 & 1815 & \tabularnewline\hline
  二一年 & 1816 & \tabularnewline\hline
  二二年 & 1817 & \tabularnewline\hline
  二三年 & 1818 & \tabularnewline\hline
  二四年 & 1819 & \tabularnewline\hline
  二五年 & 1820 & \tabularnewline
  \bottomrule
\end{longtable}


%%% Local Variables:
%%% mode: latex
%%% TeX-engine: xetex
%%% TeX-master: "../Main"
%%% End:

%% -*- coding: utf-8 -*-
%% Time-stamp: <Chen Wang: 2018-07-12 22:19:20>

\section{宣宗\tiny(1821-1850)}

\subsection{道光}

\begin{longtable}{|>{\centering\scriptsize}m{2em}|>{\centering\scriptsize}m{1.3em}|>{\centering}m{8.8em}|}
  % \caption{秦王政}\
  \toprule
  \SimHei \normalsize 年数 & \SimHei \scriptsize 公元 & \SimHei 大事件 \tabularnewline
  % \midrule
  \endfirsthead
  \toprule
  \SimHei \normalsize 年数 & \SimHei \scriptsize 公元 & \SimHei 大事件 \tabularnewline
  \midrule
  \endhead
  \midrule
  元年 & 1821 & \tabularnewline\hline
  二年 & 1822 & \tabularnewline\hline
  三年 & 1823 & \tabularnewline\hline
  四年 & 1824 & \tabularnewline\hline
  五年 & 1825 & \tabularnewline\hline
  六年 & 1826 & \tabularnewline\hline
  七年 & 1827 & \tabularnewline\hline
  八年 & 1828 & \tabularnewline\hline
  九年 & 1829 & \tabularnewline\hline
  十年 & 1830 & \tabularnewline\hline
  十一年 & 1831 & \tabularnewline\hline
  十二年 & 1832 & \tabularnewline\hline
  十三年 & 1833 & \tabularnewline\hline
  十四年 & 1834 & \tabularnewline\hline
  十五年 & 1835 & \tabularnewline\hline
  十六年 & 1836 & \tabularnewline\hline
  十七年 & 1837 & \tabularnewline\hline
  十八年 & 1838 & \tabularnewline\hline
  十九年 & 1839 & \tabularnewline\hline
  二十年 & 1840 & \tabularnewline\hline
  二一年 & 1841 & \tabularnewline\hline
  二二年 & 1842 & \tabularnewline\hline
  二三年 & 1843 & \tabularnewline\hline
  二四年 & 1844 & \tabularnewline\hline
  二五年 & 1845 & \tabularnewline\hline
  二六年 & 1846 & \tabularnewline\hline
  二七年 & 1847 & \tabularnewline\hline
  二八年 & 1848 & \tabularnewline\hline
  二九年 & 1849 & \tabularnewline\hline
  三十年 & 1850 & \tabularnewline
  \bottomrule
\end{longtable}


%%% Local Variables:
%%% mode: latex
%%% TeX-engine: xetex
%%% TeX-master: "../Main"
%%% End:

%% -*- coding: utf-8 -*-
%% Time-stamp: <Chen Wang: 2018-07-12 22:20:19>

\section{文宗\tiny(1850-1861)}

\subsection{咸丰}

\begin{longtable}{|>{\centering\scriptsize}m{2em}|>{\centering\scriptsize}m{1.3em}|>{\centering}m{8.8em}|}
  % \caption{秦王政}\
  \toprule
  \SimHei \normalsize 年数 & \SimHei \scriptsize 公元 & \SimHei 大事件 \tabularnewline
  % \midrule
  \endfirsthead
  \toprule
  \SimHei \normalsize 年数 & \SimHei \scriptsize 公元 & \SimHei 大事件 \tabularnewline
  \midrule
  \endhead
  \midrule
  元年 & 1851 & \tabularnewline\hline
  二年 & 1852 & \tabularnewline\hline
  三年 & 1853 & \tabularnewline\hline
  四年 & 1854 & \tabularnewline\hline
  五年 & 1855 & \tabularnewline\hline
  六年 & 1856 & \tabularnewline\hline
  七年 & 1857 & \tabularnewline\hline
  八年 & 1858 & \tabularnewline\hline
  九年 & 1859 & \tabularnewline\hline
  十年 & 1860 & \tabularnewline\hline
  十一年 & 1861 & \tabularnewline
  \bottomrule
\end{longtable}


%%% Local Variables:
%%% mode: latex
%%% TeX-engine: xetex
%%% TeX-master: "../Main"
%%% End:

%% -*- coding: utf-8 -*-
%% Time-stamp: <Chen Wang: 2018-07-12 22:21:05>

\section{穆宗\tiny(1861-1875)}

\subsection{同治}

\begin{longtable}{|>{\centering\scriptsize}m{2em}|>{\centering\scriptsize}m{1.3em}|>{\centering}m{8.8em}|}
  % \caption{秦王政}\
  \toprule
  \SimHei \normalsize 年数 & \SimHei \scriptsize 公元 & \SimHei 大事件 \tabularnewline
  % \midrule
  \endfirsthead
  \toprule
  \SimHei \normalsize 年数 & \SimHei \scriptsize 公元 & \SimHei 大事件 \tabularnewline
  \midrule
  \endhead
  \midrule
  元年 & 1862 & \tabularnewline\hline
  二年 & 1863 & \tabularnewline\hline
  三年 & 1864 & \tabularnewline\hline
  四年 & 1865 & \tabularnewline\hline
  五年 & 1866 & \tabularnewline\hline
  六年 & 1867 & \tabularnewline\hline
  七年 & 1868 & \tabularnewline\hline
  八年 & 1869 & \tabularnewline\hline
  九年 & 1870 & \tabularnewline\hline
  十年 & 1871 & \tabularnewline\hline
  十一年 & 1872 & \tabularnewline\hline
  十二年 & 1873 & \tabularnewline\hline
  十三年 & 1874 & \tabularnewline
  \bottomrule
\end{longtable}


%%% Local Variables:
%%% mode: latex
%%% TeX-engine: xetex
%%% TeX-master: "../Main"
%%% End:

%% -*- coding: utf-8 -*-
%% Time-stamp: <Chen Wang: 2018-07-12 22:21:55>

\section{德宗\tiny(1875-1908)}

\subsection{光绪}

\begin{longtable}{|>{\centering\scriptsize}m{2em}|>{\centering\scriptsize}m{1.3em}|>{\centering}m{8.8em}|}
  % \caption{秦王政}\
  \toprule
  \SimHei \normalsize 年数 & \SimHei \scriptsize 公元 & \SimHei 大事件 \tabularnewline
  % \midrule
  \endfirsthead
  \toprule
  \SimHei \normalsize 年数 & \SimHei \scriptsize 公元 & \SimHei 大事件 \tabularnewline
  \midrule
  \endhead
  \midrule
  元年 & 1875 & \tabularnewline\hline
  二年 & 1876 & \tabularnewline\hline
  三年 & 1877 & \tabularnewline\hline
  四年 & 1878 & \tabularnewline\hline
  五年 & 1879 & \tabularnewline\hline
  六年 & 1880 & \tabularnewline\hline
  七年 & 1881 & \tabularnewline\hline
  八年 & 1882 & \tabularnewline\hline
  九年 & 1883 & \tabularnewline\hline
  十年 & 1884 & \tabularnewline\hline
  十一年 & 1885 & \tabularnewline\hline
  十二年 & 1886 & \tabularnewline\hline
  十三年 & 1887 & \tabularnewline\hline
  十四年 & 1888 & \tabularnewline\hline
  十五年 & 1889 & \tabularnewline\hline
  十六年 & 1890 & \tabularnewline\hline
  十七年 & 1891 & \tabularnewline\hline
  十八年 & 1892 & \tabularnewline\hline
  十九年 & 1893 & \tabularnewline\hline
  二十年 & 1894 & \tabularnewline\hline
  二一年 & 1895 & \tabularnewline\hline
  二二年 & 1896 & \tabularnewline\hline
  二三年 & 1897 & \tabularnewline\hline
  二四年 & 1898 & \tabularnewline\hline
  二五年 & 1899 & \tabularnewline\hline
  二六年 & 1900 & \tabularnewline\hline
  二七年 & 1901 & \tabularnewline\hline
  二八年 & 1902 & \tabularnewline\hline
  二九年 & 1903 & \tabularnewline\hline
  三十年 & 1904 & \tabularnewline\hline
  三一年 & 1905 & \tabularnewline\hline
  三二年 & 1906 & \tabularnewline\hline
  三三年 & 1907 & \tabularnewline\hline
  三四年 & 1908 & \tabularnewline
  \bottomrule
\end{longtable}


%%% Local Variables:
%%% mode: latex
%%% TeX-engine: xetex
%%% TeX-master: "../Main"
%%% End:

%% -*- coding: utf-8 -*-
%% Time-stamp: <Chen Wang: 2018-07-12 22:22:42>

\section{溥仪\tiny(1909-1912)}

\subsection{宣统}

\begin{longtable}{|>{\centering\scriptsize}m{2em}|>{\centering\scriptsize}m{1.3em}|>{\centering}m{8.8em}|}
  % \caption{秦王政}\
  \toprule
  \SimHei \normalsize 年数 & \SimHei \scriptsize 公元 & \SimHei 大事件 \tabularnewline
  % \midrule
  \endfirsthead
  \toprule
  \SimHei \normalsize 年数 & \SimHei \scriptsize 公元 & \SimHei 大事件 \tabularnewline
  \midrule
  \endhead
  \midrule
  元年 & 1909 & \tabularnewline\hline
  二年 & 1910 & \tabularnewline\hline
  三年 & 1911 & \tabularnewline\hline
  四年 & 1912 & \tabularnewline
  \bottomrule
\end{longtable}


%%% Local Variables:
%%% mode: latex
%%% TeX-engine: xetex
%%% TeX-master: "../Main"
%%% End:



%%% Local Variables:
%%% mode: latex
%%% TeX-engine: xetex
%%% TeX-master: "../Main"
%%% End:


\end{document}

%%% Local Variables:
%%% mode: latex
%%% TeX-engine: xetex
%%% TeX-master: t
%%% End:
