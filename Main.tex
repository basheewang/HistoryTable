%% -*- coding: utf-8 -*-
%% Time-stamp: <Chen Wang: 2021-11-01 18:08:19>

\documentclass[zihao=-4,AutoFakeBold=true]{ctexbook}
\ctexset{
  chapter = {
    name = {第,卷},
    format = \centering\Large\bfseries\heiti,
    beforeskip = 10pt,
    afterskip = 20pt,
    titleformat = \chaptertitleformat
  },
  section = {
    name = {第,章},
    number = \chinese{section},
    format = \newpage\large\heiti,
    afterskip = 10pt,
    beforeskip = 10pt,
  },
  subsection = {
    name = {第,节},
    number = \chinese{subsection},
    format = \color{red}\heiti,
    afterskip = 0pt,
    beforeskip = 0pt,
  },
  subsubsection = {
    % name = {第,},
    % numbering = true,
    number = \chinese{subsubsection},
    format = \color{blue}\heiti,
    afterskip = 0pt,
    beforeskip = 0pt,
  }
}
\setcounter{secnumdepth}{5}
% \setcounter{tocdepth}{3}

\usepackage{varwidth}
\newcommand{\chaptertitleformat}[1]{%%
  \begin{varwidth}[t]{.7\linewidth}#1\end{varwidth}}

% Some extra packages
\usepackage{listings}
\lstset{
  basicstyle=\ttfamily,
  escapeinside={||},
  mathescape=true
}

\usepackage{fancyvrb}
\newsavebox{\FVerbBox}
\newenvironment{FVerbatim}
{\VerbatimEnvironment
  \begin{center}
\begin{BVerbatim}[commandchars=\\\{\}]}
 {\end{BVerbatim}
   \end{center}}

\usepackage{color}
\usepackage[perpage,hang,flushmargin]{footmisc}

\usepackage{xpinyin}

% \usepackage{enotez}

\usepackage[hidelinks,bookmarksdepth=4]{hyperref}
\hypersetup{
  colorlinks=true,
  linkcolor=blue,
  filecolor=magenta,      
  urlcolor=cyan,
}
% \hypersetup{
%   colorlinks,
%   citecolor=black,
%   filecolor=black,
%   linkcolor=black,
%   urlcolor=black
% }

\usepackage{fancyhdr}
\fancyhf{} % clear all header and footers
% \renewcommand{\headrulewidth}{0pt} % remove the header rule
% \rfoot{\thepage}
% \pagestyle{fancy}
% \fancyhf{}% clearsall
% \fancyhead[RE,LO]{\normalsize foo}
% \rfoot{\thepage}
% \renewcommand{\footrulewidth}{-30pt}

% % For Meizu Pro5
% \usepackage[
%     % showframe,
%     % includefoot,
%     paperwidth=2.75in,
%     paperheight=4.9in,
%     left=0.1in,
%     right=0.1in,
%     top=0.1in,
%     bottom=0.18in,
%     footskip=10pt
% ]
% {geometry}

% For Meizu Hornor V10
\usepackage[
    % showframe,
    % includefoot,
    paperwidth=2.73in, % 2.68in,
    paperheight=5.36in,
    left=0.1in,
    right=0.1in,
    top=0.1in,
    bottom=0.18in,
    footskip=10pt
]
{geometry}

% For Kindle 6"
% \usepackage[
% % showframe,
% paperwidth=3.6in,
% paperheight=4.8in,
% left=0.1in,
% right=0.1in,
% top=0.1in,
% bottom=0.18in,
% % footskip=10pt
% ]
% {geometry}

\setCJKmainfont{FZLanTingSong}

\newCJKfontfamily[fzsong]\fzsong{FZLanTingSong}
\newCJKfontfamily[kaiti]\kaiti{KaiTi}
\newCJKfontfamily[hkxm]\hkxm{FZBeiWeiKaiShu-S19_GB18030}
\newCJKfontfamily[song]\pml{PMingLiU}
\newCJKfontfamily[song]\hnm{HanaMin}
\newCJKfontfamily[fzk]\fzk{FZKaiS-Extended(SIP)}
\newCJKfontfamily[HeiTi]\SimHei{SimHei}

\usepackage{booktabs}
\usepackage{array}

\usepackage{tocloft}
\renewcommand{\cftsubsecfont}{\small}

\usepackage{enumitem}
% \setlength{\parindent}{0pt}
\setlist[description]{
  leftmargin=1.4em,
  labelindent=0pt,
  labelsep=3pt,
  nosep,
  itemsep=0pt,
  parsep=0pt,
  topsep=0pt,
  partopsep=0pt,
  before=\vspace{-1.6em},
  after=\vspace{-1.8em},
  font=\SimHei,
}
\setlist[enumerate]{
  nosep,
  itemsep=0pt,
  parsep=0pt,
  topsep=3pt,
  partopsep=0pt,
  leftmargin=0.4em,
  labelsep=2pt,
  after=\vspace{-3.2em},
  before=\vspace{-1.6em},  
  }
\setlist[itemize]{
    nosep,
    itemsep=0pt,
    parsep=0pt,
    topsep=3pt,
    partopsep=0pt,
    leftmargin=2em,
    % labelsep=2pt,
    % after=\vspace{-3.2em},
    % before=\vspace{-1.6em},  
  }

\usepackage{longtable}
\usepackage{graphicx}

%% 改变页数字体大小
\renewcommand*{\thepage}{\scriptsize\arabic{page}}

%% 改变字体大小
\renewcommand{\footnotesize}{\fontsize{5pt}{6pt}\selectfont}
\newcommand{\threept}{\fontsize{3pt}{3pt}\selectfont}
\newcommand{\namefont}{\fontsize{8pt}{10pt}\selectfont}
\newcommand{\xzfont}{\fontsize{5pt}{6pt}\selectfont}

%% 使每页的脚注的数字为黑色圆圈数字
\usepackage{pifont}
\makeatletter
\newcommand*{\bcircnum}[1]{%
  \expandafter\@bcircnum\csname c@#1\endcsname
}
\newcommand*{\@bcircnum}[1]{%
  \ifnum#1<1 %
  \@ctrerr
  \else
  \ifnum#1>20 %
  \@ctrerr
  \else
  \ding{\the\numexpr 181+(#1)\relax}%
  \fi
  \fi
}
\makeatother

\renewcommand*{\thefootnote}{\bcircnum{footnote}}

% \makeatletter
% \newcommand*{\circnum}[1]{%
%   \expandafter\@circnum\csname c@#1\endcsname
% }
% \newcommand*{\@circnum}[1]{%
%   \ifnum#1<1 %
%   \@ctrerr
%   \else
%   \ifnum#1>10 %
%   \@ctrerr
%   \else
%   \ding{\the\numexpr 171+(#1)\relax}%
%   \fi
%   \fi
% }
% \makeatother

% \renewcommand*{\theenumi}{\circnum{enumi}}

\setlength{\footskip}{0pt} 
\setlength{\footnotesep}{0pt}

% \usepackage{footnote}
% \makesavenoteenv{tabular}
% \makesavenoteenv{table}

%% title & author
\title{\huge 历\quad{}代\quad{}年\quad{}表}
\author{}
\date{}

\begin{document}
\clearpage\maketitle
\clearpage\tableofcontents
% \thispagestyle{empty}

\thispagestyle{empty}

\setcounter{page}{1}

\addtocontents{toc}{\protect\thispagestyle{empty}}

% %% -*- coding: utf-8 -*-
%% Time-stamp: <Chen Wang: 2018-07-21 20:35:02>

\chapter{前言}

本书包括历代君王年号及大事件。下面为一些较为有用的链接:

\begin{itemize}
  \small \kaiti
  \item 十六国时期的其他割据势力参考此\href{https://zh.wikipedia.org/wiki/%E4%BA%94%E8%83%A1%E5%8D%81%E5%85%AD%E5%9B%BD#%E5%85%B6%E4%BB%96}{链接}。
  \item 中国年号、纪元、帝王查询\href{http://www.chinese-artists.net/year/}{网站}。
  \item 中国年号列表\href{https://zh.wikipedia.org/wiki/%E4%B8%AD%E5%9B%BD%E5%B9%B4%E5%8F%B7%E5%88%97%E8%A1%A8}{网站}
    \item 中国历代文人并称的\href{https://zh.wikipedia.org/wiki/%E4%B8%AD%E5%9C%8B%E6%AD%B7%E4%BB%A3%E6%96%87%E4%BA%BA%E4%B8%A6%E7%A8%B1}{维基页面}
\end{itemize}

%%% Local Variables:
%%% mode: latex
%%% TeX-engine: xetex
%%% TeX-master: "../Main"
%%% End: % 前言

%% -*- coding: utf-8 -*-
%% Time-stamp: <Chen Wang: 2021-11-02 15:41:26>

\chapter{春秋{\tiny(BC770-BC403)}}

\section{简介}

春秋时期(公元前770年-公元前476年/公元前403年),简称春秋, 是东周的前半段时期。

春秋时代周天子的势力减弱,群雄纷争,齐桓公、宋襄公、晋文公、秦穆公、楚庄王相继称霸,史称“春秋五霸”。當時齊桓公提出「尊周室,攘夷狄,禁篡弑,抑兼併」(尊王攘夷)的思想,因此周天子於表面上仍獲尊重。

春秋时期因孔子修订的《春秋》而得名。这部书记载了从鲁隐公元年(前722年)到鲁哀公十四年(前481年)的历史,共二百四十二年。后史学家为了方便起见,一般从周平王元年(前770年)东周立国,平王東遷到洛邑起,到周敬王四十三年(前477年)或四十四年(前476年)为止(也有学者认为应到《左傳》記載之終(前468年)、三家灭智(前453年)或三家分晋(前403年)),称为“春秋时期”。春秋时期之后是战国时期。

据史书记载,春秋二百四十二年间,有三十六名君主被臣下或敌国杀之,五十二个諸侯國被灭。。有大小战事四百八十多次,诸侯的朝聘和盟会四百五十餘次。鲁国朝王三次,聘周四次。

前771年,因周幽王宠信褒姒,废太子宜臼。宜臼逃至申国,他外公申侯联合曾侯、許文公及犬戎(外族)推翻周幽王,都城宗周被毁坏,後周平王上任,前770年周平王被迫将国都从镐京迁至成周(雒邑)。因雒邑在镐京之东,此后的周朝史称东周。

周室衰微:申侯引犬戎攻入京師,害死女婿周幽王,以恢复外孙周平王的太子地位,擁立平王,使平王有弒父之嫌,因而使周天子在諸侯間的威望下降,其次各诸侯国势力逐渐强大,互相攻伐,故平王東遷後,周室漸漸衰落。周王室放弃了原本的关中地区(宗周),只有洛邑周边(成周)一小塊王畿,而失去对其他诸侯国的控制。

卿士鄭莊公:另外,由於卿士鄭莊公連打勝仗,勢力越來越大,逐漸不把周王室放在眼裡。周平王看到鄭國太驕橫了,不願把處理朝政的大權都交給鄭莊公,想將一半的權力交給另一個卿士虢公忌父,鄭莊公知道後很不滿。

周平王不敢得罪鄭莊公,公元前720年,就將王子狐作為人質讓他住到鄭國去;而鄭國公子忽也作為人質住到都城雒邑,史稱“周鄭交質”。这兩件事使周天子的地位大為降低。

周桓王十三年(前707年),由於鄭莊公不尊王室的問題與鄭國爭執,周桓王率陈、蔡、卫等国军队讨伐郑国,郑莊公领兵抗拒,两军战于繻葛(今河南长葛北),史称“繻葛之战”,郑庄公打败了王师,一箭射中周王的肩膀。说明周王的地位已严重下降,只是还保存着天下共主的虚名罢了,诸侯争霸的时代正式到来。

齐桓公称霸:前685年,齐国君主齐桓公继位,以管仲为相,实施变法,废除井田制度,按土地的肥瘠,确定赋税,设盐、铁官和铸钱,增加财政收入,寓兵于农,将基层行政组织和军事组织合为一体,增加了兵源和作战能力,迅速成为华夏各国中最富强的国家。然后就打起了“尊王攘夷”的口号,多次大会诸侯,帮助或干涉其他国家,抗击夷狄。

周惠王二十一年(前656年),齐桓公带领八个诸侯国的联军,陈兵“蛮夷”楚国边境,质询楚国为何不向周王室朝贡,迫使楚国签订召陵之盟,又於公元前651年舉行葵丘会盟,成为春秋五霸之首。自此,齊桓公建立了會盟霸主的制度。

宋楚之争:齐桓公死后,五公子夺位,齐国内乱不止。傳说,齐桓公的五个儿子互相战争,箭矢射到了齐桓公的尸体上,都没有人顾及。南方的楚国兴起,自称为王,消灭了其北方的几个小国之后将矛头指向中原。宋襄公试图效法齐桓公,以抵抗楚国进攻为名,再次大会诸侯以成为霸主,但宋国实力与威望都不足。宋襄公十五年(前638年),宋楚两军交战于泓水。楚军渡河时宋大司马子鱼建议宋襄公“半渡击之”,宋襄公称趁敌渡河时攻击是为不仁不义拒绝建议;楚军渡河后子鱼建议趁楚军列阵混乱之时攻击,宋襄公再次以不仁不义为由拒绝。楚军列阵完毕后发起攻击,宋军大败,宋襄公大腿中箭,次年因伤重而死。楚成王称雄一时。

晋文公践土之盟:在北方的晉國,與周室同宗。晉獻公时期晋国向四面扩张,领土和国力大增。但献公寵信愛姬,废嫡立幼,致使國政大亂。前636年,晉獻公之子重耳在秦穆公派出的军队护送下继承晋国君位,是为晋文公。他改革政治,发展经济,整军经武,取信于民,安定王室,友好秦国(秦晋之好),在诸侯中威信很高。周襄王二十年(前633年),楚军包围宋国都城商丘。次年初,晋文公率兵救宋,在城濮之战大败楚军,然后会盟于践土,成为中原霸主。

秦穆公独霸西戎:晋文公死后,秦晋联盟被瓦解,秦穆公谋求向东方发展,被晋所阻。秦晋殽之战(前627年),秦全军覆没,大将孟明视被俘虏,隔年在彭衙之战再败,虽然以后有王官之戰的胜利,但终没法挑战晋在中原的地位,惟有转而向西发展,秦穆公任用由余,吞并了一些戎狄部族,寬地千里,独霸西戎。

楚庄王问鼎中原:楚国在城濮战后,向东发展,灭了许多小国,势力南到今云南,北达黄河。楚庄王改革内政,平息暴乱,啟用賢臣孫叔敖兴修水利,改革軍制,国力更为强大,在攻克陆浑戎后,竟陈兵周郊,向周定王的使者询问象徵國家政權的傳國寶器 - 九鼎的大小轻重,意在灭周自立,此即“问鼎”一词的来源。周定王十年(前597年),楚与晋会战于邲(邲之战),大胜晉国。前594年,楚围宋,宋告急于晋,晋不能救,宋遂与楚言和,尊楚。这时中原各国除晋、齐、鲁之外,尽尊楚庄王为霸主。

晋楚大战与弭兵会盟:晋楚两大国之间连续不断的战争给人民带来巨大的灾难,也引起中小国家的厌倦,加以晋楚两大国势均力敌,谁都无法击垮对方。于是由宋国发起,于周简王七年(前579年)举行第一次“弭兵”会盟,是为华元弭兵。但是不久之后,会盟破裂。晉楚兩國再度爆發兩次大規模戰役(前576年的鄢陵之戰、前557年的湛阪之戰),雖皆以晉國獲勝收場,但楚國在中原地區仍與晉國保持勢均力敵的態勢,很多中原小国都备受到影响,疲惫不堪。周灵王二十六年(前546年),出于地缘政治的影响,宋国再次出面斡旋,邀请晋楚和各诸侯国举行第二次“弭兵”会盟,此后战争大大减少。史稱『向戌弭兵』。

吴越雄霸东南:当中原诸侯争霸接近尾声时,地处江浙的吴、越开始发展。吴王阖闾重用孙武、伍子胥等人。周敬王十四年(前506年),吴王以伍子胥为大将,统兵伐楚。吴军攻进楚都郢,伍子胥为父兄报仇,掘楚平王墓,鞭尸三百。周敬王二十四年(前496年)吴军挥师南进伐越。越王勾践率兵迎战,越大夫灵姑浮一戈击中阖闾,阖闾因伤逝世。周敬王二十六年(前494年),吴王夫差为父报仇,兴兵败越。勾践求和,贿赂吴臣伯嚭并送给吴王珍宝和美女西施,自己亲自为夫差牵马。吴王拒绝了伍子胥联齐灭越的建议,接受越国求和,转兵向北进击,大败齐军,成为小霸。勾践卧薪尝胆,十年生聚,十年教训,终于在周元王三年(前473年)消灭吴国,夫差羞愤自杀。勾践北上与齐晋会盟于徐,成为最后一个霸主。

三家分晋:在晉文公回晉即位的時候,有不少隨從隨他回國,結果這些人漸漸在晉國成為世襲貴族,而晉國的國政亦落入這些貴族(士大夫)的手上。前455年,晉國貴族只餘下智、趙、韓、魏四家。智氏出兵攻赵氏,并胁迫魏韩两氏出兵。战事持续两年后,赵氏游说魏韩两家倒戈,灭智氏,瓜分智地并把持晋国国政,史稱三家分晋。到晋幽公仅余绛、曲沃两地。前403年周威烈王册立韩赵魏三家为侯国,即为资治通鉴中春秋战国的分界点。

春秋五霸的崛起:春秋时期,周王室衰微,实际上和一个中等诸侯国地位相近。各国之间互相攻伐,战争持续不断,小国被吞并。各国内部,卿大夫势力强大,动乱时有发生,弑君现象屡见不鲜。《春秋》和《左传》中记载的弑君事件达43次之多,主要集中在春秋前期,这也反映了西周东周交替时权力的急剧变化。「春秋戰國之時,已漸由封建而變為郡縣。」「周初千八百國,至春秋之初,僅存百二十四國。春秋諸國,吞併小弱,大抵以其國地為縣。因滅國而特置縣,因置縣而特命官,封建之制遂漸變為郡縣之制。」

經濟:春秋时期,铁制农具开始普及,春秋时期除使用块炼铁外,还掌握了冶炼生铁的先进技术。铁器的使用使大规模开垦荒地成为可能,促进了私田的发展,同时也为手工业提供了锐利的工具,牛耕渐趋普遍起来,牛耕技术的发展,只有与铁器的使用相配合,方可发挥出它的功能。在青铜冶铸方面发明了错金、错银、嵌红铜等新工艺。侯马大批铸造陶范的出土,显示出这一时期青铜冶铸业和采矿业的规模很大、水平很高。春秋中期以后,各诸侯国已经大量使用货币。金属货币的流通,促进了手工业、商业的发展。

文化:春秋战国是中国文化发展的时期。周天子及其诸侯政治权威的动摇与衰落,造成学在官府的局面被打破,如儒家的孔子创办私学,首开私学风气。孔子提倡有教无类的办学思想,促进教育事业的发展,以及为人们提高自己的社会地位提供了途径。而随之而出现的学术下移、典籍文化走向民间等社会方方面面的变化,又引起了人们思想观念的某种改变,这些变化正是春秋时期思想文化转型得以实现的历史条件,后这为战国的百家争鸣奠定了基础。

这一时期,由于政治的不稳定,禮樂崩壞,学术受政治影响小,学术思想得以获得发展,开始产生了不同的学派。如道家的老子等。老子著有《道德经》,道德经阐述了中国古代朴素的宇宙观,世界观,人生观,对后世中国文化影响深远。《论语》是孔子弟子将孔子的主要言行记录下来整理的。其后,儒家开始发韧,在学术上逐步占据主导地位。尤其至汉武帝“罢黜百家,独尊儒术”后,更是占据中国文化的统治地位,长达两千余年。

藝術:春秋时代的艺术,主要是青铜器上面的雕刻。著名的三足羊首鼎就是春秋时代的青铜艺术品。1923年,在新郑市出土了大量春秋时代的青铜鼎、爵,和西周时期的青铜器相比之下工艺已经大大发展。青铜器上的纹饰也很讲究。

春秋時代的木雕藝術以南方的楚國最為聞名。春秋時代的青銅祭器數量極多,且大小各異,西元前六至五世紀發展出來的精緻裝飾為其特色。相較之下,春秋時代這類的青銅器較通常不加裝飾的戰國時代青銅器為重要。考古挖掘出的春秋時代印章為目前所知最早的,然而有文獻證據顯示印章出現的時間更早。此外,中國的金器製作亦在春秋時代開始普及。

科技:铁器和牛耕在春秋时期得到推广,推动了历史的发展。在天文学、物理学、医学方面。

中国传统农业在春秋时期才开始形成。春秋时期的人们发明了以前没有的铁犁铧、铁锄、连枷、石磨等新农具。

春秋时期青铜器铸造也是这一时代的特征,以曾国和楚国、徐国的青铜器为代表。

建築:春秋时期诸侯国渐强盛,兴建大量城市和宫殿。其时多为以阶梯形夯土台为基的台榭式建筑。以夯土台为中心,附建木质结构房屋,形成多层次宫殿。

而楚國在楚靈王時期建的章華台更是春秋時的建築代表。

山东省临淄县郎家庄春秋时代墓葬出土的漆器残片,中画圆形,四面画四座建筑,柱顶上有栱,承托脊檩。窗仍为井字格,但另加小格。这种四室相背的建筑可能和台榭建筑有关。

歷史同一時期:前753年:古罗马进入王政时代(前753年-前509年) 前550年:波斯帝國的阿契美尼德王朝成立(前550年-前330年) 前509年:羅馬共和國時期開始。

%% -*- coding: utf-8 -*-
%% Time-stamp: <Chen Wang: 2021-11-02 15:39:05>

\section{东周}

\subsection{簡介}

东周(前770年至前256年),是自周平王東遷以后對周朝的称呼,相对于之前国都在镐京的時期即西周。东周也是「春秋時代」的开始。

東周京都於前770年自镐京(今陝西省西安市),东迁至雒邑(今河南省洛阳市)。传25王,前后515年。 这一时期是中国的社会制度剧烈转变的时期,以铁器的广泛使用为标志。

周幽王死後,諸侯擁立原先被廢的太子宜臼為王,是为周平王。他即位第二年,見鎬京被戰火破壞,又受到犬戎侵扰,便遷都雒邑,史稱「東周」,以別於在這以前的西周。東周的前半期,諸侯爭相稱霸,持續了二百多年,稱為「春秋時代」;東周的後半期,周天子地位漸失,亦持續了二百多年,稱為「戰國時代」。

周平王東遷以後,管轄範圍大減,形同一個小國,加上被指有弑父之嫌,在諸侯中的威望已经大不如前。面對諸侯之間互相攻伐和兼併,邊境的外族又乘機入侵,周天子不能擔負共主的責任,經常要向一些強大的諸侯求助。在這情況下,強大的諸侯便自居霸主。中原諸侯對四夷侵擾則以「尊王攘夷」口號團結自衛,战国时代徐州相王、五国相王后各大诸侯纷纷僭越称王(吴、越、楚三国春秋時代已称王),周王权威進一步受損。

周襄王十七年(前635年),发生“子带之乱”,襄王不能平,求救于晋文公,文公诛叔带,遂为伯而得河内地。周襄王二十年(前632年),襄王为晋文公所迫,于河阳踐土會盟。

周定王元年(前606年),楚庄王伐陆浑之戎,欲观九鼎。定王使王孙满应设以辞,楚人遂去。

周赧王时,東周国势益弱,同时内部争斗不休,以至分为东周国和西周国,赧王迁都西周國。周赧王八年(前307年),秦借道两周之间攻韩,周人两边都不敢得罪,左右为难。东西两周位于诸强国之间,不能同心协力,反而彼此攻杀。至赧王六十年(前256年),西周国为秦所灭,赧王死,七年后,东周国亦为秦所灭。

春秋时代的得名,是因孔子修订《春秋》而得名。一般而言,春秋时代从周平王五十年(前722年)起,直到周敬王四十三年(前477年)或四十四年(前476年)为止,也有学者认为春秋時代应到三家灭智(前453年)或三家分晋(前403年),原因是即使到三家分晉,除秦、楚、齊等國外,還有其他大小王國。

战国时代由三家分晉,春秋時代結束,直到秦統一中國(公元前221年)這段時間,一般稱為戰國時代。

值得注意的是,東周王朝在戰國後期(前256年)即已被秦所滅,所以戰國時代在時間上並不全然包含在東周裡面。

这一时期是中国歷史的社会制度转变的时期,这一转变是以铁器的广泛使用为标志的,東周基本上進入了鐵器時代。

在东周时期,铁器被广泛使用。农业有了长足的发展,从整体上来看人口不断成长,原来各诸侯国之间的无人地带,已不存在。各国因争夺土地或者水利资源,冲突时起。铜钱开始流行,甚至在楚国还出现了金币—郢钱,出现一定的商品经济和商人阶层。教育向平民普及。贵族与平民间的界限也被冲破。社会产生了一种革命性的变化,周王朝建立的宗法封建制度,已经不能适应这一变化。

東周與西周的地理位置差異反應在藝術表現上,尤其是東周晚期的藝術作品,展現了多元的風貌與高水準的技術。或許受到孔子反對以人殉葬的影響,以低溫燒製的陪葬塑像(明器,又稱「冥器」、「盟器」)數量增加。

東周時期亦出現低溫綠色鉛釉器皿、質地鬆的打磨黑色器皿、高溫釉器皿等。有些陶器仿效最新流行的漆器,色彩鮮明,有些則仿效青銅器。另有模製與裝飾的陶瓦、陶磚。西周時期較少見的玉雕再次成為重要的陪葬品與個人飾物。青銅的應用不限於宗教禮儀用途,變得較為世俗,常用作結婚贈禮之居家裝飾。青銅鐘及青銅鏡逐漸流行,動物和怪獸圖騰則由色彩繽紛而樣式化的裝飾圖案所取代。東周墓葬出土有最早繪於絲絹上的畫作。此外,亦發現了漢代及唐代陪葬陶器的前身。

%% -*- coding: utf-8 -*-
%% Time-stamp: <Chen Wang: 2021-11-01 18:13:59>

\subsection{平王宜臼{\tiny(BC772-BC720)}}

\subsubsection{生平}

周平王(前781年-前720年),姓姬,名宜臼,東周第一位天子。周幽王之子,母申后為申侯之女,後母褒姒。

姬宜臼本是周幽王太子,後因周幽王寵愛美女褒姒,褒姒生下一子伯服,得寵更甚。褒姒卻與權臣勾結,陰謀奪嫡,而幽王竟為了討美人歡心,便廢掉王后申后與太子宜臼母子,改封褒姒與其子伯服為王后和太子。宜臼逃奔申國。

前771年,周幽王被杀。申侯、缯侯及许文公,在申国擁立宜臼为周天子,即「周平王」。而虢公翰则在携,拥立周幽王之弟姬余臣为周天子,即「携王」(携惠王),形成「二王并立」的局面。

前770年,由於鎬京在戰後已殘破不堪,宜臼為避犬戎,在晉國和鄭國的支持下迁都雒邑,史稱「平王东迁」。周朝至此以後,為史家稱作「東周」。

前750年,携王被晋文侯所弑。前720年,周王宜臼去世,諡號平王。

當時傳聞平王和母親申后因被其父周幽王廢掉後一直懷恨在心,故意聯合諸侯國繒和外族犬戎入宮殺害幽王和褒姒,然後即位為天子,故此周平王有弑父之嫌,亦令周朝的國勢一落千丈。除晉鄭二國之外,其餘諸侯認為宜臼有弒父之嫌疑,故都不再聽從天子,平王唯“晉、鄭是依”,勉強支運國勢,而造成春秋時代的開始。

\subsubsection{年表}

% \centering
\begin{longtable}{|>{\centering\scriptsize}m{2em}|>{\centering\scriptsize}m{1.3em}|>{\centering}m{8.8em}|}
  % \caption{秦王政}\\
  \toprule
  \SimHei \normalsize 年数 & \SimHei \scriptsize 公元 & \SimHei 大事件 \tabularnewline
  % \midrule
  \endfirsthead
  \toprule
  \SimHei \normalsize 年数 & \SimHei \scriptsize 公元 & \SimHei 大事件 \tabularnewline
  \midrule
  \endhead
  \midrule
  % 元年 & -770 & \tabularnewline\hline
  % 二年 & -769 & \tabularnewline\hline
  % 三年 & -768 & \tabularnewline\hline
  % 四年 & -767 & \tabularnewline\hline
  % 五年 & -766 & \tabularnewline\hline
  % 六年 & -765 & \tabularnewline\hline
  % 七年 & -764 & \tabularnewline\hline
  % 八年 & -763 & \tabularnewline\hline
  % 九年 & -762 & \tabularnewline\hline
  % 十年 & -761 & \tabularnewline\hline
  % 十一年 & -760 & \tabularnewline\hline
  % 十二年 & -759 & \tabularnewline\hline
  % 十三年 & -758 & \tabularnewline\hline
  % 十四年 & -757 & \tabularnewline\hline
  % 十五年 & -756 & \tabularnewline\hline
  % 十六年 & -755 & \tabularnewline\hline
  % 十七年 & -754 & \tabularnewline\hline
  % 十八年 & -753 & \tabularnewline\hline
  % 十九年 & -752 & \tabularnewline\hline
  % 二十年 & -751 & \tabularnewline\hline
  % 二一年 & -750 & \tabularnewline\hline
  % 二二年 & -749 & \tabularnewline\hline
  % 二三年 & -748 & \tabularnewline\hline
  % 二四年 & -747 & \tabularnewline\hline
  % 二五年 & -746 & \tabularnewline\hline
  % 二六年 & -745 & \tabularnewline\hline
  % 二七年 & -744 & \tabularnewline\hline
  % 二八年 & -743 & \tabularnewline\hline
  % 二九年 & -742 & \tabularnewline\hline
  % 三十年 & -741 & \tabularnewline\hline
  % 三一年 & -740 & \tabularnewline\hline
  % 三二年 & -739 & \tabularnewline\hline
  % 三三年 & -738 & \tabularnewline\hline
  % 三四年 & -737 & \tabularnewline\hline
  % 三五年 & -736 & \tabularnewline\hline
  % 三六年 & -735 & \tabularnewline\hline
  % 三七年 & -734 & \tabularnewline\hline
  % 三八年 & -733 & \tabularnewline\hline
  % 三九年 & -732 & \tabularnewline\hline
  % 四十年 & -731 & \tabularnewline\hline
  % 四一年 & -730 & \tabularnewline\hline
  % 四二年 & -729 & \tabularnewline\hline
  % 四三年 & -728 & \tabularnewline\hline
  % 四四年 & -727 & \tabularnewline\hline
  % 四五年 & -726 & \tabularnewline\hline
  % 四六年 & -725 & \tabularnewline\hline
  % 四七年 & -724 & \tabularnewline\hline
  % 四八年 & -723 & \tabularnewline\hline
  四九年 & -722 & \tabularnewline\hline
  五十年 & -721 & \tabularnewline\hline
  五一年 & -720 & \tabularnewline
  \bottomrule
\end{longtable}

%%% Local Variables:
%%% mode: latex
%%% TeX-engine: xetex
%%% TeX-master: "../../Main"
%%% End:

%% -*- coding: utf-8 -*-
%% Time-stamp: <Chen Wang: 2021-11-01 18:25:56>

\subsection{桓王林\tiny{(BC719-BC697)}}

\subsubsection{生平}

周桓王(?-前697年),姓姬,名林,東周第二代君主,諡桓。他是周平王之孫,因平王駕崩時,太子洩父早死,其作为洩父之子得以繼承天子之位。

桓王(前720年)即位後,計劃削弱卿權以加強王權。加上周朝領地與鄭國領地鄰接,鄭國多次越界取禾,故桓王罷免了鄭莊公卿士之職。莊公不滿,便不再朝見天子。周王室始與鄭國交惡。

桓王二年(前718年)春天,晉國曲沃封君曲沃莊伯賄賂周桓王,聯合鄭國、邢國攻打晉國都城翼城,晉鄂侯戰敗,逃奔隨邑。同年夏天,晉鄂侯去世,曲沃莊伯於是再度攻打晉國。由於當時曲沃莊伯背叛周桓王,同年秋天,周桓王便反過來支持晉國,並派遣虢公 率領軍隊討伐曲沃莊伯,曲沃莊伯兵敗,只得逃回曲沃防守。周桓王立晉鄂侯之子晉哀侯為君。

桓王十二年(前708年)周桓王聯合秦國出兵包圍芮國,俘虜芮國國君芮伯萬。

桓王十三年(前707年),桓王率蔡、衞、陳聯軍攻鄭,大敗於繻葛。桓王本人更於此役中為鄭將祝聃射箭中傷。此後桓王雖然仍能影響虢國,但已無力阻止周王室轉衰之势,也無力阻止諸侯間的互相攻伐。

桓王十八年(前703年),虢仲向周桓王進讒言誣陷大夫詹父。周桓王認為詹父有理,詹父於是帶領周天子的軍隊進攻虢國。同年夏天,虢公逃亡到虞國

桓王二十三年(前697年),桓王三月因病駕崩,儿子王子佗继位,是為周莊王。

\subsubsection{年表}

% \centering
\begin{longtable}{|>{\centering\scriptsize}m{2em}|>{\centering\scriptsize}m{1.3em}|>{\centering}m{8.8em}|}
  % \caption{秦王政}\\
  \toprule
  \SimHei \normalsize 年数 & \SimHei \scriptsize 公元 & \SimHei 大事件 \tabularnewline
  % \midrule
  \endfirsthead
  \toprule
  \SimHei \normalsize 年数 & \SimHei \scriptsize 公元 & \SimHei 大事件 \tabularnewline
  \midrule
  \endhead
  \midrule
  元年 & -719 & \tabularnewline\hline
  二年 & -718 & \tabularnewline\hline
  三年 & -717 & \tabularnewline\hline
  四年 & -716 & \tabularnewline\hline
  五年 & -715 & \tabularnewline\hline
  六年 & -714 & \tabularnewline\hline
  七年 & -713 & \tabularnewline\hline
  八年 & -712 & \tabularnewline\hline
  九年 & -711 & \tabularnewline\hline
  十年 & -710 & \tabularnewline\hline
  十一年 & -709 & \tabularnewline\hline
  十二年 & -708 & \tabularnewline\hline
  十三年 & -707 & \tabularnewline\hline
  十四年 & -706 & \tabularnewline\hline
  十五年 & -705 & \tabularnewline\hline
  十六年 & -704 & \tabularnewline\hline
  十七年 & -703 & \tabularnewline\hline
  十八年 & -702 & \tabularnewline\hline
  十九年 & -701 & \tabularnewline\hline
  二十年 & -700 & \tabularnewline\hline
  二一年 & -699 & \tabularnewline\hline
  二二年 & -698 & \tabularnewline\hline
  二三年 & -697 & \tabularnewline
  \bottomrule
\end{longtable}

%%% Local Variables:
%%% mode: latex
%%% TeX-engine: xetex
%%% TeX-master: "../../Main"
%%% End:

%% -*- coding: utf-8 -*-
%% Time-stamp: <Chen Wang: 2021-11-02 14:37:57>

\subsection{莊王佗\tiny{(BC696-BC682)}}

\subsubsection{生平}

周莊王(?-前682年),姓姬,名佗,東周第三代國王,諡莊。他是周桓王之兒。莊王三年(前694年),周公黑肩欲殺莊王,而立莊王弟王子克,事泄,黑肩被莊王與辛伯所殺,克奔南燕(河南延津)。

在位期间执政为虢公林父、周公黑肩。

\subsubsection{年表}

% \centering
\begin{longtable}{|>{\centering\scriptsize}m{2em}|>{\centering\scriptsize}m{1.3em}|>{\centering}m{8.8em}|}
  % \caption{秦王政}\\
  \toprule
  \SimHei \normalsize 年数 & \SimHei \scriptsize 公元 & \SimHei 大事件 \tabularnewline
  % \midrule
  \endfirsthead
  \toprule
  \SimHei \normalsize 年数 & \SimHei \scriptsize 公元 & \SimHei 大事件 \tabularnewline
  \midrule
  \endhead
  \midrule
  元年 & -696 & \tabularnewline\hline
  二年 & -695 & \tabularnewline\hline
  三年 & -694 & \tabularnewline\hline
  四年 & -693 & \tabularnewline\hline
  五年 & -692 & \tabularnewline\hline
  六年 & -691 & \tabularnewline\hline
  七年 & -690 & \tabularnewline\hline
  八年 & -689 & \tabularnewline\hline
  九年 & -688 & \tabularnewline\hline
  十年 & -687 & \tabularnewline\hline
  十一年 & -686 & \tabularnewline\hline
  十二年 & -685 & \tabularnewline\hline
  十三年 & -684 & \tabularnewline\hline
  十四年 & -683 & \tabularnewline\hline
  十五年 & -682 & \tabularnewline  
  \bottomrule
\end{longtable}

%%% Local Variables:
%%% mode: latex
%%% TeX-engine: xetex
%%% TeX-master: "../../Main"
%%% End:

%% -*- coding: utf-8 -*-
%% Time-stamp: <Chen Wang: 2021-11-02 14:42:43>

\subsection{僖王胡齐\tiny{(BC681-BC677)}}

\subsubsection{生平}

周僖王(?-前677年),又作周釐王,姓姬,名胡齐,東周第四代君主,在位5年,號僖。

僖王爲周莊王長子,莊王雖偏爱姚姬生的少子王子頹,但未能廢長立幼。前681年,僖王即位。

前679年,曲沃克晉後,曲沃武公把晉國的寶器獻給僖王,僖王承認曲沃武公為晉君,列為諸侯。前678年,遭晉軍攻打並殺害夷邑大夫詭諸,執政大臣周公忌父逃奔虢國。

前677年,周僖王崩。

在位期间执政为虢公醜、周公忌父。

\subsubsection{年表}

% \centering
\begin{longtable}{|>{\centering\scriptsize}m{2em}|>{\centering\scriptsize}m{1.3em}|>{\centering}m{8.8em}|}
  % \caption{秦王政}\\
  \toprule
  \SimHei \normalsize 年数 & \SimHei \scriptsize 公元 & \SimHei 大事件 \tabularnewline
  % \midrule
  \endfirsthead
  \toprule
  \SimHei \normalsize 年数 & \SimHei \scriptsize 公元 & \SimHei 大事件 \tabularnewline
  \midrule
  \endhead
  \midrule
  元年 & -681 & \tabularnewline\hline
  二年 & -680 & \tabularnewline\hline
  三年 & -679 & \tabularnewline\hline
  四年 & -678 & \tabularnewline\hline
  五年 & -677 & \tabularnewline
  \bottomrule
\end{longtable}

%%% Local Variables:
%%% mode: latex
%%% TeX-engine: xetex
%%% TeX-master: "../../Main"
%%% End:

%% -*- coding: utf-8 -*-
%% Time-stamp: <Chen Wang: 2021-11-02 14:59:42>

\subsection{惠王閬\tiny{(BC676-BC652)}}

\subsubsection{生平}

周惠王(?-前652年),姓姬,名閬,又名聞,東周第五代君主,諡惠。他是周僖王之子。

周惠王在前677年繼位後,佔用蒍國的園圃飼養野獸,蒍國的人民不滿,惠王二年,有五大夫作亂,立王子頹為周天子,惠王奔溫(今河南溫縣南),鄭厲公在櫟地(今禹州市)收容惠王,並在惠王四年與虢國攻入周朝,協助平定“子頹之亂”,惠王復辟,鄭國因功獲賜予虎牢(今河南滎陽汜水鎮)以東的地方,虢國也獲賜土地。

周惠王晚年宠爱幼子王子带,欲立為嗣,约郑联楚、晋以成此事,但此时齐桓公稱霸天下,与诸侯会盟力挺太子,周惠王未能如愿。周惠王駕崩后,太子周襄王即位。

《史記·周本紀》稱惠王在位25年,《左傳》稱周惠王在魯僖公七年(前653年)冬天駕崩。

在位期间执政为虢公醜、周公忌父、宰孔。

\subsubsection{年表}

% \centering
\begin{longtable}{|>{\centering\scriptsize}m{2em}|>{\centering\scriptsize}m{1.3em}|>{\centering}m{8.8em}|}
  % \caption{秦王政}\\
  \toprule
  \SimHei \normalsize 年数 & \SimHei \scriptsize 公元 & \SimHei 大事件 \tabularnewline
  % \midrule
  \endfirsthead
  \toprule
  \SimHei \normalsize 年数 & \SimHei \scriptsize 公元 & \SimHei 大事件 \tabularnewline
  \midrule
  \endhead
  \midrule
  元年 & -676 & \tabularnewline\hline
  二年 & -675 & \tabularnewline\hline
  三年 & -674 & \tabularnewline\hline
  四年 & -673 & \tabularnewline\hline
  五年 & -672 & \tabularnewline\hline
  六年 & -671 & \tabularnewline\hline
  七年 & -670 & \tabularnewline\hline
  八年 & -669 & \tabularnewline\hline
  九年 & -668 & \tabularnewline\hline
  十年 & -667 & \tabularnewline\hline
  十一年 & -666 & \tabularnewline\hline
  十二年 & -665 & \tabularnewline\hline
  十三年 & -664 & \tabularnewline\hline
  十四年 & -663 & \tabularnewline\hline
  十五年 & -662 & \tabularnewline\hline
  十六年 & -661 & \tabularnewline\hline
  十七年 & -660 & \tabularnewline\hline
  十八年 & -659 & \tabularnewline\hline
  十九年 & -658 & \tabularnewline\hline
  二十年 & -657 & \tabularnewline\hline
  二一年 & -656 & \tabularnewline\hline
  二二年 & -655 & \tabularnewline\hline
  二三年 & -654 & \tabularnewline\hline
  二四年 & -653 & \tabularnewline\hline
  二五年 & -652 & \tabularnewline  
  \bottomrule
\end{longtable}

%%% Local Variables:
%%% mode: latex
%%% TeX-engine: xetex
%%% TeX-master: "../../Main"
%%% End:

%% -*- coding: utf-8 -*-
%% Time-stamp: <Chen Wang: 2021-11-02 15:06:10>

\subsection{襄王鄭\tiny{(BC651-BC619)}}

\subsubsection{生平}

周襄王(?-前619年),姬姓,名鄭,東周第六代君主,諡襄。襄王是周惠王之子, 《史記·周本紀》稱襄王在位32年,《左傳》稱襄王崩於魯文公八年(前619年)秋。

惠王死後,襄王懼怕異母弟王子帶爭奪王位繼承權,秘不發喪,並派人向齊國求援,襄王直到大局已定才公佈父王死訊。

周襄王時,因鄭伯不聽王命,曾經要求狄人攻打鄭國。在狄人擊敗鄭國後,周襄王娶隗姓狄人為妻,又將隗后退回。此舉引發狄人不滿,前636年,王子帶聯合狄人,攻打周襄王,周襄王逃到鄭國。

當時晉文公勢力強大,在前635年出兵助襄王,殺王子帶,迎接周襄王返回洛陽復位。前632年,晉文公召襄王,襄王親自到踐土(今河南原陽西南)會見他。

周襄王在位期间宋襄公、晋文公、秦穆公相继称霸。在位期间执政为宰孔、周公忌父、王子虎、周公閱、王叔桓公。

\subsubsection{年表}

% \centering
\begin{longtable}{|>{\centering\scriptsize}m{2em}|>{\centering\scriptsize}m{1.3em}|>{\centering}m{8.8em}|}
  % \caption{秦王政}\\
  \toprule
  \SimHei \normalsize 年数 & \SimHei \scriptsize 公元 & \SimHei 大事件 \tabularnewline
  % \midrule
  \endfirsthead
  \toprule
  \SimHei \normalsize 年数 & \SimHei \scriptsize 公元 & \SimHei 大事件 \tabularnewline
  \midrule
  \endhead
  \midrule
  元年 & -651 & \tabularnewline\hline
  二年 & -650 & \tabularnewline\hline
  三年 & -649 & \tabularnewline\hline
  四年 & -648 & \tabularnewline\hline
  五年 & -647 & \tabularnewline\hline
  六年 & -646 & \tabularnewline\hline
  七年 & -645 & \tabularnewline\hline
  八年 & -644 & \tabularnewline\hline
  九年 & -643 & \tabularnewline\hline
  十年 & -642 & \tabularnewline\hline
  十一年 & -641 & \tabularnewline\hline
  十二年 & -640 & \tabularnewline\hline
  十三年 & -639 & \tabularnewline\hline
  十四年 & -638 & \tabularnewline\hline
  十五年 & -637 & \tabularnewline\hline
  十六年 & -636 & \tabularnewline\hline
  十七年 & -635 & \tabularnewline\hline
  十八年 & -634 & \tabularnewline\hline
  十九年 & -633 & \tabularnewline\hline
  二十年 & -632 & \tabularnewline\hline
  二一年 & -631 & \tabularnewline\hline
  二二年 & -630 & \tabularnewline\hline
  二三年 & -629 & \tabularnewline\hline
  二四年 & -628 & \tabularnewline\hline
  二五年 & -627 & \tabularnewline\hline
  二六年 & -626 & \tabularnewline\hline
  二七年 & -625 & \tabularnewline\hline
  二八年 & -624 & \tabularnewline\hline
  二九年 & -623 & \tabularnewline\hline
  三十年 & -622 & \tabularnewline\hline
  三一年 & -621 & \tabularnewline\hline
  三二年 & -620 & \tabularnewline\hline
  三三年 & -619 & \tabularnewline  
  \bottomrule
\end{longtable}

%%% Local Variables:
%%% mode: latex
%%% TeX-engine: xetex
%%% TeX-master: "../../Main"
%%% End:

%% -*- coding: utf-8 -*-
%% Time-stamp: <Chen Wang: 2021-11-02 15:08:18>

\subsection{頃王壬臣\tiny{(BC618-BC613)}}

\subsubsection{生平}

周頃王(?-前613年),姓姬,名壬臣,為周襄王之子。周頃王在前618年繼位為東周第七代君主,當時王畿已縮小,王室財政一貧如洗,無法安葬襄王,頃王只得派毛伯衛向魯國討錢。後來魯文公派使者送錢到都城,才安葬了襄王。

頃王在前613年春天去世,在位6年,由兒子周匡王繼位。

在位期间执政为周公閱、王叔桓公、王孫蘇。

\subsubsection{年表}

% \centering
\begin{longtable}{|>{\centering\scriptsize}m{2em}|>{\centering\scriptsize}m{1.3em}|>{\centering}m{8.8em}|}
  % \caption{秦王政}\\
  \toprule
  \SimHei \normalsize 年数 & \SimHei \scriptsize 公元 & \SimHei 大事件 \tabularnewline
  % \midrule
  \endfirsthead
  \toprule
  \SimHei \normalsize 年数 & \SimHei \scriptsize 公元 & \SimHei 大事件 \tabularnewline
  \midrule
  \endhead
  \midrule
  元年 & -618 & \tabularnewline\hline
  二年 & -617 & \tabularnewline\hline
  三年 & -616 & \tabularnewline\hline
  四年 & -615 & \tabularnewline\hline
  五年 & -614 & \tabularnewline\hline
  六年 & -613 & \tabularnewline  
  \bottomrule
\end{longtable}

%%% Local Variables:
%%% mode: latex
%%% TeX-engine: xetex
%%% TeX-master: "../../Main"
%%% End:

%% -*- coding: utf-8 -*-
%% Time-stamp: <Chen Wang: 2021-11-02 15:10:26>

\subsection{匡王班\tiny{(BC612-BC607)}}

\subsubsection{生平}

周匡王(?-前607年),姓姬,名班,中国东周第8代君主,前612年至前607年在位,共6年。匡王是周頃王之子。前607年十月,周王班崩,諡“匡”,其弟王子瑜繼位,即周定王。

在位期间执政为周公閱、王孫蘇、召桓公、毛伯衛。

\subsubsection{年表}

% \centering
\begin{longtable}{|>{\centering\scriptsize}m{2em}|>{\centering\scriptsize}m{1.3em}|>{\centering}m{8.8em}|}
  % \caption{秦王政}\\
  \toprule
  \SimHei \normalsize 年数 & \SimHei \scriptsize 公元 & \SimHei 大事件 \tabularnewline
  % \midrule
  \endfirsthead
  \toprule
  \SimHei \normalsize 年数 & \SimHei \scriptsize 公元 & \SimHei 大事件 \tabularnewline
  \midrule
  \endhead
  \midrule
  元年 & -612 & \tabularnewline\hline
  二年 & -611 & \tabularnewline\hline
  三年 & -610 & \tabularnewline\hline
  四年 & -609 & \tabularnewline\hline
  五年 & -608 & \tabularnewline\hline
  六年 & -607 & \tabularnewline
  \bottomrule
\end{longtable}

%%% Local Variables:
%%% mode: latex
%%% TeX-engine: xetex
%%% TeX-master: "../../Main"
%%% End:

%% -*- coding: utf-8 -*-
%% Time-stamp: <Chen Wang: 2021-11-02 15:14:57>

\subsection{定王瑜\tiny{(BC606-BC586)}}

\subsubsection{生平}

周定王(?-前586年),姓姬,名瑜,中国東周第9代天子,前606年—前586年在位,周定王是匡王之弟。定王在位21年而卒,子夷立,為簡王。

在位期间执政为王孫蘇、召桓公、劉康公、毛伯衛、單襄公。

楚莊王为稱霸天下,不斷北侵並打敗了晋国、齐国、宋国、郑国、陈国、蔡国等國,在定王元年征伐陆浑之戎,進軍到周京雒邑的南郊,向周王耀武示威。定王不敢責問楚莊王,便派王孫滿去慰勞楚軍,楚莊王詢問周朝鎮國之寶的九鼎大小輕重,欲逼周取天下。後王孫滿以婉辭說服了楚莊王,使楚不敢輕舉妄動去取代周朝,便撤兵回國。

\subsubsection{年表}

% \centering
\begin{longtable}{|>{\centering\scriptsize}m{2em}|>{\centering\scriptsize}m{1.3em}|>{\centering}m{8.8em}|}
  % \caption{秦王政}\\
  \toprule
  \SimHei \normalsize 年数 & \SimHei \scriptsize 公元 & \SimHei 大事件 \tabularnewline
  % \midrule
  \endfirsthead
  \toprule
  \SimHei \normalsize 年数 & \SimHei \scriptsize 公元 & \SimHei 大事件 \tabularnewline
  \midrule
  \endhead
  \midrule
  元年 & -606 & \tabularnewline\hline
  二年 & -605 & \tabularnewline\hline
  三年 & -604 & \tabularnewline\hline
  四年 & -603 & \tabularnewline\hline
  五年 & -602 & \tabularnewline\hline
  六年 & -601 & \tabularnewline\hline
  七年 & -600 & \tabularnewline\hline
  八年 & -599 & \tabularnewline\hline
  九年 & -598 & \tabularnewline\hline
  十年 & -597 & \tabularnewline\hline
  十一年 & -596 & \tabularnewline\hline
  十二年 & -595 & \tabularnewline\hline
  十三年 & -594 & \tabularnewline\hline
  十四年 & -593 & \tabularnewline\hline
  十五年 & -592 & \tabularnewline\hline
  十六年 & -591 & \tabularnewline\hline
  十七年 & -590 & \tabularnewline\hline
  十八年 & -589 & \tabularnewline\hline
  十九年 & -588 & \tabularnewline\hline
  二十年 & -587 & \tabularnewline\hline
  二一年 & -586 & \tabularnewline  
  \bottomrule
\end{longtable}

%%% Local Variables:
%%% mode: latex
%%% TeX-engine: xetex
%%% TeX-master: "../../Main"
%%% End:

%% -*- coding: utf-8 -*-
%% Time-stamp: <Chen Wang: 2021-11-02 15:13:42>

\subsection{簡王夷\tiny{(BC585-BC572)}}

\subsubsection{生平}

周簡王(?-前572年),姓姬,名夷,為周定王之子。在位14年,此期間晋、楚、秦,宋、郑等国相互攻伐不止,吴国兴起,攻入楚国,幾乎亡楚。前572年九月,周王夷病死,諡号为简王。

在位期间执政为單襄公、劉康公、周公楚、尹武公。

子周灵王、儋季。

\subsubsection{年表}

% \centering
\begin{longtable}{|>{\centering\scriptsize}m{2em}|>{\centering\scriptsize}m{1.3em}|>{\centering}m{8.8em}|}
  % \caption{秦王政}\\
  \toprule
  \SimHei \normalsize 年数 & \SimHei \scriptsize 公元 & \SimHei 大事件 \tabularnewline
  % \midrule
  \endfirsthead
  \toprule
  \SimHei \normalsize 年数 & \SimHei \scriptsize 公元 & \SimHei 大事件 \tabularnewline
  \midrule
  \endhead
  \midrule
  元年 & -585 & \tabularnewline\hline
  二年 & -584 & \tabularnewline\hline
  三年 & -583 & \tabularnewline\hline
  四年 & -582 & \tabularnewline\hline
  五年 & -581 & \tabularnewline\hline
  六年 & -580 & \tabularnewline\hline
  七年 & -579 & \tabularnewline\hline
  八年 & -578 & \tabularnewline\hline
  九年 & -577 & \tabularnewline\hline
  十年 & -576 & \tabularnewline\hline
  十一年 & -575 & \tabularnewline\hline
  十二年 & -574 & \tabularnewline\hline
  十三年 & -573 & \tabularnewline\hline
  十四年 & -572 & \tabularnewline  
  \bottomrule
\end{longtable}

%%% Local Variables:
%%% mode: latex
%%% TeX-engine: xetex
%%% TeX-master: "../../Main"
%%% End:

%% -*- coding: utf-8 -*-
%% Time-stamp: <Chen Wang: 2021-11-02 15:17:28>

\subsection{靈王泄心\tiny{(BC571-BC545)}}

\subsubsection{生平}

周靈王(?-前545年),姓姬,名泄心,是周简王之子,東周第11代國王,在位27年。《幼学琼林》中说他出生时便有胡须。

《列仙傳》中記載:周灵王的长子太子晋天性聪明,善吹笙,立他为太子,不幸早逝。公元前545年十一月的某天夜裡,周灵王梦见太子骑着白鹤来迎接他。传位于次子王子贵,癸巳日,病死。孔子在周靈王二十一年出生于魯。

在位期間執政為王叔陳生、伯輿、單靖公。

\subsubsection{年表}

% \centering
\begin{longtable}{|>{\centering\scriptsize}m{2em}|>{\centering\scriptsize}m{1.3em}|>{\centering}m{8.8em}|}
  % \caption{秦王政}\\
  \toprule
  \SimHei \normalsize 年数 & \SimHei \scriptsize 公元 & \SimHei 大事件 \tabularnewline
  % \midrule
  \endfirsthead
  \toprule
  \SimHei \normalsize 年数 & \SimHei \scriptsize 公元 & \SimHei 大事件 \tabularnewline
  \midrule
  \endhead
  \midrule
  元年 & -571 & \tabularnewline\hline
  二年 & -570 & \tabularnewline\hline
  三年 & -569 & \tabularnewline\hline
  四年 & -568 & \tabularnewline\hline
  五年 & -567 & \tabularnewline\hline
  六年 & -566 & \tabularnewline\hline
  七年 & -565 & \tabularnewline\hline
  八年 & -564 & \tabularnewline\hline
  九年 & -563 & \tabularnewline\hline
  十年 & -562 & \tabularnewline\hline
  十一年 & -561 & \tabularnewline\hline
  十二年 & -560 & \tabularnewline\hline
  十三年 & -559 & \tabularnewline\hline
  十四年 & -558 & \tabularnewline\hline
  十五年 & -557 & \tabularnewline\hline
  十六年 & -556 & \tabularnewline\hline
  十七年 & -555 & \tabularnewline\hline
  十八年 & -554 & \tabularnewline\hline
  十九年 & -553 & \tabularnewline\hline
  二十年 & -552 & \tabularnewline\hline
  二一年 & -551 & \tabularnewline\hline
  二二年 & -550 & \tabularnewline\hline
  二三年 & -549 & \tabularnewline\hline
  二四年 & -548 & \tabularnewline\hline
  二五年 & -547 & \tabularnewline\hline
  二六年 & -546 & \tabularnewline\hline
  二七年 & -545 & \tabularnewline  
  \bottomrule
\end{longtable}

%%% Local Variables:
%%% mode: latex
%%% TeX-engine: xetex
%%% TeX-master: "../../Main"
%%% End:

%% -*- coding: utf-8 -*-
%% Time-stamp: <Chen Wang: 2021-11-02 15:21:09>

\subsection{景王貴\tiny{(BC544-BC520)}}

\subsubsection{生平}

周景王(?-前520年),姓姬,名貴,中國東周君主,諡號景,為周靈王之子。周景王在位时,财政困難,連器皿都要向各国索讨。

有一次,景王宴请晋国大臣知文子荀跞,指着鲁国送来的酒壶说:“各国都有器物送给天子,为何晋国没有?”荀跞答不出来,請副使籍谈答覆,籍谈说当初晋国受封时,未赐以礼器,現在晋国忙于对付戎狄,當然無法送出礼物。周景王列数王室赐给晋的土地器物,讽刺其“数典而忘其祖”,这是“数典忘祖”的典故。此時周天子的地位已经一落千丈。

周景王太子寿早死,立王子猛为太子,卻宠爱庶长子王子朝,還一度打算刺殺支持王子猛的單穆公、劉文公。公元前520年四月,周景王病重,嘱咐宾孟要擁立王子朝。王子朝未及立为嗣君,景王却突然病死,由王子猛即位[2]。

周景王在位期間執政為單靖公、劉定公、成簡公、單獻公、單成公、劉獻公、單穆公。

\subsubsection{年表}

% \centering
\begin{longtable}{|>{\centering\scriptsize}m{2em}|>{\centering\scriptsize}m{1.3em}|>{\centering}m{8.8em}|}
  % \caption{秦王政}\\
  \toprule
  \SimHei \normalsize 年数 & \SimHei \scriptsize 公元 & \SimHei 大事件 \tabularnewline
  % \midrule
  \endfirsthead
  \toprule
  \SimHei \normalsize 年数 & \SimHei \scriptsize 公元 & \SimHei 大事件 \tabularnewline
  \midrule
  \endhead
  \midrule
  元年 & -544 & \tabularnewline\hline
  二年 & -543 & \tabularnewline\hline
  三年 & -542 & \tabularnewline\hline
  四年 & -541 & \tabularnewline\hline
  五年 & -540 & \tabularnewline\hline
  六年 & -539 & \tabularnewline\hline
  七年 & -538 & \tabularnewline\hline
  八年 & -537 & \tabularnewline\hline
  九年 & -536 & \tabularnewline\hline
  十年 & -535 & \tabularnewline\hline
  十一年 & -534 & \tabularnewline\hline
  十二年 & -533 & \tabularnewline\hline
  十三年 & -532 & \tabularnewline\hline
  十四年 & -531 & \tabularnewline\hline
  十五年 & -530 & \tabularnewline\hline
  十六年 & -529 & \tabularnewline\hline
  十七年 & -528 & \tabularnewline\hline
  十八年 & -527 & \tabularnewline\hline
  十九年 & -526 & \tabularnewline\hline
  二十年 & -525 & \tabularnewline\hline
  二一年 & -524 & \tabularnewline\hline
  二二年 & -523 & \tabularnewline\hline
  二三年 & -522 & \tabularnewline\hline
  二四年 & -521 & \tabularnewline\hline
  二五年 & -520 & \tabularnewline  
  \bottomrule
\end{longtable}

%%% Local Variables:
%%% mode: latex
%%% TeX-engine: xetex
%%% TeX-master: "../../Main"
%%% End:

%% -*- coding: utf-8 -*-
%% Time-stamp: <Chen Wang: 2021-11-02 15:22:09>

\subsection{悼王猛\tiny{(BC520-BC520)}}

\subsubsection{生平}

周悼王(?-前520年),姓姬,名猛,中國東周君主,諡號悼王,未即位时称王子猛,即位后称王猛。他是周景王的儿子,景王病重時,囑咐大夫賓孟立王子朝。景王死,國人立王子猛為王,是為悼王。悼王後來被王子朝殺死。

周悼王为周景王之子。《左传》记载周景王十八年(前527年),其太子寿和王后去世。其后围绕着周景王的太子人选,朝中形成了对立的两派。一派为支持王子猛的大夫单子单穆公、刘子刘文公。另一派支持王子朝,以王子朝之傅宾起(又称为“宾孟”)为代表。周景王宠爱王子朝,并杀王子猛之傅下门子(见《国语·周语下第三》,取徐元诰《国语集解》之解释)。

周景王二十五年夏四月,景王命令公卿随同自己前往北山田猎。景王计划趁此杀死单子、刘子,然后册立王子朝为太子。但计划尚未实行,周景王突然驾崩,按《左传》的说法是死于心脏病。王子猛即位为王,单子杀宾起,王猛之位初定。即位之时王猛的身份,据《史记·周本纪》是“长子”,按《国语集解》之说是景王朝的太子,对此《左传》无明确记载。《左传·昭公二十六年》记载的王子朝告诸侯文,强调了王后无子的情况下,应该立长子为太子的礼制,而批评单子、刘子为私利而立少子的行为。由此可见,王子朝和王子猛都不是景王的王后之子。而王子朝比王子猛年长。

\subsubsection{年表}

% \centering
\begin{longtable}{|>{\centering\scriptsize}m{2em}|>{\centering\scriptsize}m{1.3em}|>{\centering}m{8.8em}|}
  % \caption{秦王政}\\
  \toprule
  \SimHei \normalsize 年数 & \SimHei \scriptsize 公元 & \SimHei 大事件 \tabularnewline
  % \midrule
  \endfirsthead
  \toprule
  \SimHei \normalsize 年数 & \SimHei \scriptsize 公元 & \SimHei 大事件 \tabularnewline
  \midrule
  \endhead
  \midrule
  元年 & -520 & \tabularnewline  
  \bottomrule
\end{longtable}

%%% Local Variables:
%%% mode: latex
%%% TeX-engine: xetex
%%% TeX-master: "../../Main"
%%% End:

%% -*- coding: utf-8 -*-
%% Time-stamp: <Chen Wang: 2021-11-02 15:24:59>

\subsection{敬王匄\tiny{(BC519-BC476)}}

\subsubsection{生平}

周敬王(前6世纪?-前477年),姬姓,名匄(音同「丐」),中國東周君主,諡號敬王。他是周景王的兒子,周悼王同母弟。

周景王的庶長子王子朝在悼王病死後,自立為王。晉國派兵攻打王子朝,立王子匄為王。此後敬王與王子朝不時仍有衝突。前516年王子朝逃到楚國。前505年春,楚国被吴国击败,险些亡国,周敬王趁机派人在楚地殺死王子朝。儋翩帶領王子朝支持者在次年起兵舉事,敬王出逃,在前503年得晉國幫助下回都。

東周自周平王開始以雒邑(洛邑,又稱成周)為都城,平王東遷後,又稱雒邑為王城。敬王時,因王子朝在雒邑勢大,乃遷都至雒邑之東,稱新都為成周,稱舊都為王城。

周敬王二十六年(前494年),吳王夫差為父報仇,起兵擊敗越國,越王勾踐求和,並獻上美女西施給夫差。《左傳·哀公十九年》記載,冬,周敬王去世,葬于三王陵(今河南省洛阳市西南10里处)。

在位期間執政為單穆公、劉文公、單武公、劉桓公、萇弘、單平公。

\subsubsection{年表}

% \centering
\begin{longtable}{|>{\centering\scriptsize}m{2em}|>{\centering\scriptsize}m{1.3em}|>{\centering}m{8.8em}|}
  % \caption{秦王政}\\
  \toprule
  \SimHei \normalsize 年数 & \SimHei \scriptsize 公元 & \SimHei 大事件 \tabularnewline
  % \midrule
  \endfirsthead
  \toprule
  \SimHei \normalsize 年数 & \SimHei \scriptsize 公元 & \SimHei 大事件 \tabularnewline
  \midrule
  \endhead
  \midrule
  元年 & -519 & \tabularnewline\hline
  二年 & -518 & \tabularnewline\hline
  三年 & -517 & \tabularnewline\hline
  四年 & -516 & \tabularnewline\hline
  五年 & -515 & \tabularnewline\hline
  六年 & -514 & \tabularnewline\hline
  七年 & -513 & \tabularnewline\hline
  八年 & -512 & \tabularnewline\hline
  九年 & -511 & \tabularnewline\hline
  十年 & -510 & \tabularnewline\hline
  十一年 & -509 & \tabularnewline\hline
  十二年 & -508 & \tabularnewline\hline
  十三年 & -507 & \tabularnewline\hline
  十四年 & -506 & \tabularnewline\hline
  十五年 & -505 & \tabularnewline\hline
  十六年 & -504 & \tabularnewline\hline
  十七年 & -503 & \tabularnewline\hline
  十八年 & -502 & \tabularnewline\hline
  十九年 & -501 & \tabularnewline\hline
  二十年 & -500 & \tabularnewline\hline
  二一年 & -499 & \tabularnewline\hline
  二二年 & -498 & \tabularnewline\hline
  二三年 & -497 & \tabularnewline\hline
  二四年 & -496 & \tabularnewline\hline
  二五年 & -495 & \tabularnewline\hline
  二六年 & -494 & \tabularnewline\hline
  二七年 & -493 & \tabularnewline\hline
  二八年 & -492 & \tabularnewline\hline
  二九年 & -491 & \tabularnewline\hline
  三十年 & -490 & \tabularnewline\hline
  三一年 & -489 & \tabularnewline\hline
  三二年 & -488 & \tabularnewline\hline
  三三年 & -487 & \tabularnewline\hline
  三四年 & -486 & \tabularnewline\hline
  三五年 & -485 & \tabularnewline\hline
  三六年 & -484 & \tabularnewline\hline
  三七年 & -483 & \tabularnewline\hline
  三八年 & -482 & \tabularnewline\hline
  三九年 & -481 & \tabularnewline\hline
  四十年 & -480 & \tabularnewline\hline
  四一年 & -479 & \tabularnewline\hline
  四二年 & -478 & \tabularnewline\hline
  四三年 & -477 & \tabularnewline\hline
  四四年 & -476 & \tabularnewline  
  \bottomrule
\end{longtable}

%%% Local Variables:
%%% mode: latex
%%% TeX-engine: xetex
%%% TeX-master: "../../Main"
%%% End:

%% -*- coding: utf-8 -*-
%% Time-stamp: <Chen Wang: 2021-11-02 15:26:32>

\subsection{元王仁\tiny{(BC475-BC469)}}

\subsubsection{生平}

周元王(?-前469年),姓姬,名仁,中國東周君主,在位8年,為周敬王之子。

周元王四年(前473年),越王句踐滅吳。句踐隨後北上遷都琅琊,與齊國、晉國等諸侯會盟於徐州(今山東滕縣南),「越兵橫行於江、淮東,諸侯畢賀,號稱霸王」,周元王正式承認句踐為霸主。

\subsubsection{年表}

% \centering
\begin{longtable}{|>{\centering\scriptsize}m{2em}|>{\centering\scriptsize}m{1.3em}|>{\centering}m{8.8em}|}
  % \caption{秦王政}\\
  \toprule
  \SimHei \normalsize 年数 & \SimHei \scriptsize 公元 & \SimHei 大事件 \tabularnewline
  % \midrule
  \endfirsthead
  \toprule
  \SimHei \normalsize 年数 & \SimHei \scriptsize 公元 & \SimHei 大事件 \tabularnewline
  \midrule
  \endhead
  \midrule
  元年 & -475 & \tabularnewline\hline
  二年 & -474 & \tabularnewline\hline
  三年 & -473 & \tabularnewline\hline
  四年 & -472 & \tabularnewline\hline
  五年 & -471 & \tabularnewline\hline
  六年 & -470 & \tabularnewline\hline
  七年 & -469 & \tabularnewline  
  \bottomrule
\end{longtable}

%%% Local Variables:
%%% mode: latex
%%% TeX-engine: xetex
%%% TeX-master: "../../Main"
%%% End:

%% -*- coding: utf-8 -*-
%% Time-stamp: <Chen Wang: 2021-11-02 15:29:48>

\subsection{貞定王介\tiny{(BC468-BC441)}}

\subsubsection{生平}

周貞定王(?-前441年),姓姬,名介,東周君主,周元王子,在位28年,諡號貞定王。

周貞定王十六年(前453年),晋国大夫韩康子、赵襄子、魏桓子共同攻灭了晉國最大勢力智伯瑤,是為三家滅智。

清朝学者黄式三在其《周季编略》中认为周王介的谥号貞定王的说法是一个错误。他指出,《史记·周本纪》中周王介被称为定王,与周定王瑜同谥,黄式三认为此处史记是沿袭了《国语》的错误记载。黄式三认为皇甫谧在《帝王世纪》中,按照《世本》和《史记》等称周王介为贞王或定王的记载,臆造了周貞定王的称谓,司马贞《史记索隐》就对皇甫谧的做法提出批评。而后世学者多从皇甫谧,黄式三认为应该根据《国语》韦昭注和司马贞《史记索隐》的说法,而称周王介为周贞王。

\subsubsection{年表}

% \centering
\begin{longtable}{|>{\centering\scriptsize}m{2em}|>{\centering\scriptsize}m{1.3em}|>{\centering}m{8.8em}|}
  % \caption{秦王政}\\
  \toprule
  \SimHei \normalsize 年数 & \SimHei \scriptsize 公元 & \SimHei 大事件 \tabularnewline
  % \midrule
  \endfirsthead
  \toprule
  \SimHei \normalsize 年数 & \SimHei \scriptsize 公元 & \SimHei 大事件 \tabularnewline
  \midrule
  \endhead
  \midrule
  元年 & -468 & \tabularnewline\hline
  二年 & -467 & \tabularnewline\hline
  三年 & -466 & \tabularnewline\hline
  四年 & -465 & \tabularnewline\hline
  五年 & -464 & \tabularnewline\hline
  六年 & -463 & \tabularnewline\hline
  七年 & -462 & \tabularnewline\hline
  八年 & -461 & \tabularnewline\hline
  九年 & -460 & \tabularnewline\hline
  十年 & -459 & \tabularnewline\hline
  十一年 & -458 & \tabularnewline\hline
  十二年 & -457 & \tabularnewline\hline
  十三年 & -456 & \tabularnewline\hline
  十四年 & -455 & \tabularnewline\hline
  十五年 & -454 & \tabularnewline\hline
  十六年 & -453 & \tabularnewline\hline
  十七年 & -452 & \tabularnewline\hline
  十八年 & -451 & \tabularnewline\hline
  十九年 & -450 & \tabularnewline\hline
  二十年 & -449 & \tabularnewline\hline
  二一年 & -448 & \tabularnewline\hline
  二二年 & -447 & \tabularnewline\hline
  二三年 & -446 & \tabularnewline\hline
  二四年 & -445 & \tabularnewline\hline
  二五年 & -444 & \tabularnewline\hline
  二六年 & -443 & \tabularnewline\hline
  二七年 & -442 & \tabularnewline\hline
  二八年 & -441 & \tabularnewline  
  \bottomrule
\end{longtable}

%%% Local Variables:
%%% mode: latex
%%% TeX-engine: xetex
%%% TeX-master: "../../Main"
%%% End:

%% -*- coding: utf-8 -*-
%% Time-stamp: <Chen Wang: 2021-11-02 15:31:46>

\subsection{哀王去疾\tiny{(BC441-BC441)}}

\subsubsection{生平}

周哀王(?-前441年),姓姬,名去疾,東周君主,為周貞定王長子。前441年即位,《史記·周本紀》稱哀王在位僅三個月,為弟叔襲殺害,諡號為哀王。

《搜神记·卷六》記载周哀王八年郑国一妇女共生育四十個子女,其中只有二十個長大成人。哀王九年,晉國有頭豬生了個人。

\subsubsection{年表}

% \centering
\begin{longtable}{|>{\centering\scriptsize}m{2em}|>{\centering\scriptsize}m{1.3em}|>{\centering}m{8.8em}|}
  % \caption{秦王政}\\
  \toprule
  \SimHei \normalsize 年数 & \SimHei \scriptsize 公元 & \SimHei 大事件 \tabularnewline
  % \midrule
  \endfirsthead
  \toprule
  \SimHei \normalsize 年数 & \SimHei \scriptsize 公元 & \SimHei 大事件 \tabularnewline
  \midrule
  \endhead
  \midrule
  元年 & -441 & \tabularnewline  
  \bottomrule
\end{longtable}

%%% Local Variables:
%%% mode: latex
%%% TeX-engine: xetex
%%% TeX-master: "../../Main"
%%% End:

%% -*- coding: utf-8 -*-
%% Time-stamp: <Chen Wang: 2021-11-02 15:32:57>

\subsection{思王叔襲\tiny{(BC441-BC441)}}

\subsubsection{生平}

周思王(?-前441年),姓姬,名叔襲,東周君主,為周貞定王之子,周哀王之弟。

前441年,叔襲殺害周哀王即位,為周思王;在位僅五個月,八月又被弟王子嵬所殺。

\subsubsection{年表}

% \centering
\begin{longtable}{|>{\centering\scriptsize}m{2em}|>{\centering\scriptsize}m{1.3em}|>{\centering}m{8.8em}|}
  % \caption{秦王政}\\
  \toprule
  \SimHei \normalsize 年数 & \SimHei \scriptsize 公元 & \SimHei 大事件 \tabularnewline
  % \midrule
  \endfirsthead
  \toprule
  \SimHei \normalsize 年数 & \SimHei \scriptsize 公元 & \SimHei 大事件 \tabularnewline
  \midrule
  \endhead
  \midrule
  元年 & -441 & \tabularnewline
  \bottomrule
\end{longtable}

%%% Local Variables:
%%% mode: latex
%%% TeX-engine: xetex
%%% TeX-master: "../../Main"
%%% End:

%% -*- coding: utf-8 -*-
%% Time-stamp: <Chen Wang: 2021-11-02 15:34:25>

\subsection{考王嵬\tiny{(BC440-BC426)}}

\subsubsection{生平}

周考王(?-前426年),又稱周考哲王,姓姬,名嵬,為中國東周第十九代國王,在位15年,為周貞定王之子、周哀王與周思王之弟。

前441年,姬嵬殺害周思王自立,是為周考王,以前440年為考王元年。前440年,周考王封其弟姬揭於王畿(位於今河南),是為西周桓公。西周桓公死後,子威公立。惠公繼承威公之位,在周顯王二年(前367年)又封少子姬班於鞏(今河南省鞏义市西南),史稱“東周”。此時周朝王畿內再分出「西周」、「東周」两小國,王畿便更為縮小。

周考王時期處於春秋時期與戰國時期之間,或戰國初期。

前426年周考王死去,其子姬午繼位,是為周威烈王。

\subsubsection{年表}

% \centering
\begin{longtable}{|>{\centering\scriptsize}m{2em}|>{\centering\scriptsize}m{1.3em}|>{\centering}m{8.8em}|}
  % \caption{秦王政}\\
  \toprule
  \SimHei \normalsize 年数 & \SimHei \scriptsize 公元 & \SimHei 大事件 \tabularnewline
  % \midrule
  \endfirsthead
  \toprule
  \SimHei \normalsize 年数 & \SimHei \scriptsize 公元 & \SimHei 大事件 \tabularnewline
  \midrule
  \endhead
  \midrule
  元年 & -440 & \tabularnewline\hline
  二年 & -439 & \tabularnewline\hline
  三年 & -438 & \tabularnewline\hline
  四年 & -437 & \tabularnewline\hline
  五年 & -436 & \tabularnewline\hline
  六年 & -435 & \tabularnewline\hline
  七年 & -434 & \tabularnewline\hline
  八年 & -433 & \tabularnewline\hline
  九年 & -432 & \tabularnewline\hline
  十年 & -431 & \tabularnewline\hline
  十一年 & -430 & \tabularnewline\hline
  十二年 & -429 & \tabularnewline\hline
  十三年 & -428 & \tabularnewline\hline
  十四年 & -427 & \tabularnewline\hline
  十五年 & -426 & \tabularnewline
  \bottomrule
\end{longtable}

%%% Local Variables:
%%% mode: latex
%%% TeX-engine: xetex
%%% TeX-master: "../../Main"
%%% End:


%%% Local Variables:
%%% mode: latex
%%% TeX-engine: xetex
%%% TeX-master: "../../Main"
%%% End:
 %東周
%% -*- coding: utf-8 -*-
%% Time-stamp: <Chen Wang: 2019-12-26 22:28:51>

\section{鲁国}

\subsection{简介}

鲁国,是周朝的一個姬姓諸侯國,為周成王的四叔周公旦及其子伯禽的封国。鲁国先後傳二十五世,三十六位君主,歷時八百餘年。首都在曲阜,疆域在泰山以南,略有今山东省西南部,國力鼎盛時期勢力遍及河南、江蘇及安徽三省。另外,魯国亦是孔子的出生地。

立國:西周初年周公輔佐天子周成王,周公东征打败了伙同武庚叛乱的殷商旧属国,之後周公长子伯禽代替周公前往受封的奄国故土建立鲁国。

伯禽到达封国之后,把曲阜作为自己封国的都城,然后依照周国的制度、习俗来进行治理。因为要去除当地的旧习俗,伯禽前前后后用了三年时间才完成了初步的稳定,然后返回成周报告政绩。而鲁的邻国齐却只用了五个月就返回成周报告结果了,这是因为齐国采取了和鲁国完全相反的政策。齐国的封君简化了周的制度,并依照当地风俗来治理封国,于是很快地稳定下来了。在管叔、蔡叔联合武庚作乱时,东方的淮夷、徐戎等也兴兵作乱,前来攻打鲁国。伯禽率领鲁国的军队前往抵抗,奋战两年最终在周、齐的帮助下平定了鲁国。伯禽在位四十余年,坚持使用周礼治理鲁国,又加上成王赋予了鲁国“郊祭文王”、“奏天子礼乐”的资格,鲁国因此在立国之初就奠定了丰厚的周文化基础。而在后来礼坏乐崩的时代,鲁国则成为了典型周礼的保存者和实施者,世人称“周礼尽在鲁矣”。

周室强藩:周王朝历来有厚同姓、薄异姓的国策,而周成王赋予鲁国“郊祭文王”、“奏天子礼乐”的资格,不仅仅是对周公旦功劳的一种追念,更是希望作为宗邦的鲁国能够“大启尔宇,为周室辅”。这是鲁国在政治上的优势。伐灭管蔡之乱,平定徐戎之叛,鲁国得到“殷民六族”。而本来是王族的殷商之民,拥有较高的文化水平,同时也善于发展经济;而鲁国地处东方海滨,盐铁等重要资源丰富。鲁国历经鲁公伯禽、考公酋(系本作“就”,邹诞本作“遒”)、炀公熙(一作怡,考公弟)、幽公宰(系本名圉)、魏公晞(幽公弟)、厉公擢(系本作“翟”)、献公具(厉公弟)、真公濞(本亦多作“慎公”),一直都是周室强藩,震慑并管理东方,充分发挥了宗邦的作用。此时的鲁国“奄有龟蒙,遂荒大东。至于海邦,淮夷来同”,其国力之强,使得国人和夷狄之民“莫我敢承”、“莫不率从”。这种情形一直延续到春秋,彼时曹、滕、薛、纪、杞、彀、邓、邾、牟、葛诸侯仍旧时常朝觐鲁国。

废长立幼:鲁真公薨,其弟敖立,是为鲁武公。武公有长子括、少子戏。武公九年,武公带着两个儿子,西去朝拜周宣王。宣王很喜欢戏,有意立戏为鲁国的太子。長子括為魯武公的嫡子(與正室夫人所生)、少子戲為魯武公的庶子(與側室夫人所生),依照當時的宗法,只有在正室夫人無子或所生之子死亡時才能立庶子為太子,宣王的做法嚴重犯了宗法的大忌。宣王的卿大夫樊仲山父说:「这个废长立幼,不合规矩。若您執意違背规矩的话,日后鲁国一定会违背您的旨意。」周宣王不顾重臣意见,下命令立戏为鲁国太子。鲁武公郁郁不乐,回到鲁国后就去世了。太子戏繼位,是为鲁懿公。懿公之後被其兄括的儿子伯御殺掉。伯御安安稳稳地做了十一年鲁国国君,最后被周宣王发兵伐魯,把伯御给诛杀了,再立魯懿公的弟弟称,是为鲁孝公。伐魯令周朝天子的威信受損,以後周诸侯国弑其君的事情时有发生。

隐公居摄:鲁孝公薨,子弗湟立,是为鲁惠公。鲁惠公的元配没有生子就死了,妾室声子倒是生了个儿子,名叫做息(一作息姑)。后来,惠公听说宋国有个女子生来手掌就有“鲁夫人”的纹状,于是就把她娶回鲁国,是为仲子。仲子为惠公生了个儿子,名叫做允(一作轨)。惠公没有立太子就死掉了。年长的公子息颇得鲁人的拥戴,于是他摄行君位。但是又担心其他人不服,于是立公子允为惠公太子,说是等他长大后就把政权返给他。历史上把公子息称作“隐公”,谥法:“不尸其位曰隐。”。

隐公时期,卿大夫羽父位高权重,逐渐掌握实权。羽父野心大,渴望与国君平起平坐,何况隐公甚至还不是名义上的国君。羽父要隐公立他为太宰,太宰就是周天子的王室正卿,就地位而言,跟诸侯平起平坐。隐公不答应。羽父于是到太子允前,谗言隐公想要霸占权位。太子允于是授命羽父派人弑杀了隐公。太子允即位,是为鲁桓公。魯隐公及桓公時(前722年-前662年),魯國多次戰勝齊、宋等国,且不断侵襲杞國、莒國等小国。鲁桓公初期,羽父还挺有权势,但是到了后来就不见经传,或许是桓公疏远了他也未可知。

三桓時期:春秋中期之後,魯國政權轉入貴族大臣手中。魯莊公的三个弟弟季友、叔牙及慶父的子孫長期掌握魯國實權,称為季孫氏、叔孫氏、孟孫氏三家,由於三家都是魯桓公之後,被称為「三桓」,魯國從此「政在大夫」。鲁桓公有庶长子庆父、太子同、公子牙、公子友。庆父、叔牙、季友的后代分别是孟氏、叔孙氏及季氏,合称三桓。三桓为孟氏、叔孙氏及季氏,而非孟孙氏、叔孙氏及季孙氏。以往有众多学者认为孟孙、叔孙、季孙皆为氏称,实误。“孙”为尊称,对于孟氏和季氏,“孟孙某”、“季孙某”仅限于宗主的称谓,宗族一般成员只能称“孟某”、“季某”。所以,“孟孙”、“季孙”并不是氏称。考之《左传》,只有“孟氏”、“季氏”的字样,而无“孟孙氏”、“季孙氏”的字样。叔孙氏的情况比较特殊,起先为叔氏,后来公子牙(字子叔)之后立叔氏,原来的叔氏改称叔孙氏。

桓公薨,太子同立,是为鲁庄公。庄公夫人哀姜,哀姜娣叔姜为庄公生子开。庄公晚年,筑高台,看到大夫党氏的女儿孟任,很是欢喜,就跟着她走。最后,庄公许诺说立孟任为夫人,如果她给自己生了儿子,就立为太子。孟任生子般(一作“斑”)。庄公想立般为太子,又担心其他臣子有意见。到了庄公三十二年,庄公病笃,又想到立太子的事情,就询问自己的兄弟叔牙、季友。叔牙说庆父有才能,季友则说就算死也要立公子般。庄公让季友派人赐鸩酒给叔牙。叔牙饮鸩而死,立其后为叔氏,后改称叔孙氏。

鲁庄公立般为太子,而季友辅佐。叔牙死后不久,庄公薨。于是季友立太子般为国君,为庄公治丧,因此尚未正式即位。而庆父发难,派人弑杀了在党氏居住的子般。季友惊慌之间,逃往陈国。庆父与庄公夫人哀姜一向都有私通,因此发难之后,他立哀姜陪嫁的叔姜之子,公子开为国君,是为鲁闵公(一作湣公)。庆父立闵公之后,跟哀姜私通,後來想把闵公也杀了,自己当国君。齐国仲孙湫就预言“不去庆父,鲁难未已”([子说庆父不死,鲁难未已。比喻不清除制造内乱的罪魁祸首,国家就得不到安宁。亦指了结或停止危害的关键事物。)。鲁闵公二年,庆父派大夫卜齮袭杀闵公于武闱。季友听闻,由陈國走到邾國,接庄公妾成风之子申,请鲁人以其为国君。庆父忧惧,出逃到莒國。于是,季友送公子申入鲁,并重金贿赂莒人,抓庆父回国。庆父请求让他出逃,季友不肯。于是庆父自杀。立其后为孟氏。关于孟氏,《春秋》又作仲氏。因为当初庆父虽为长兄,但为了表示君臣之别,于是自称仲,史称共仲。实际上,当时的人都以其年长而叫他的后代为孟氏。

季友立公子申,是为鲁僖公(史记作“釐公”)。僖公元年,季友帅师败“莒师于郦,获莒拏”,“公赐季友汶阳之田及费”,季友为鲁国相。季友相僖公,执政多年,把鲁国治理得井井有条。鲁人作《诗·鲁颂》称赞。僖公十六年,季友卒,谥成,史称“成季”,其后立为季氏。

公卿争权:鲁僖公、文公、宣公、成公、襄公、昭公、定公、哀公及悼公九位鲁侯在位期间,作为卿家的三桓与公室争权夺利,尤其是以季氏的执政与公室的反击最为激烈。鲁穆公时期实行改革,任命博士公仪休为鲁相,才遂渐从三桓手中收回政权。成季死后,庄公的公子遂(即襄仲)及其儿子公孙归父相继掌权,是为东门氏执政时期,而孟氏一度被东门氏赶出鲁国。然而,成季的孙子季孙行父(即季文子)利用三桓的势力,魯宣公十五年(前594年)實行「初税畝」,开初税亩,使得私田兴起,而“隐民”剧增,获得鲁国平民阶层的人心。公子遂杀嫡立庶,以公子俀为国君,是为鲁宣公。

宣公发现三桓日益强盛,同时有民不知君、只知三桓的说法甚嚣尘上,于是他欲去三桓,以张大公室。他与执政的公孙归父商量,是不是起兵灭了三桓,但是国人明显倾心于三桓,使用国内兵马或许不妥。于是,公孙归父前往晋国借兵。可惜公孙归父还没成功搬来晋国军队,宣公就死了,而季文子趁机发难,备述襄仲当政时的弊端,斥责他“南通于楚,既不能固,又不能坚事齐、晋”,使鲁国没有强援。鲁国司寇表示愿意随季文子除乱。公孙归父听到这样的消息,连忙逃到齐国躲起来。季文子开始执政。从此开启了季氏祖孙几代人的执政专权之路。

季文子、季武子、季平子辅佐鲁文公、宣公、成公、襄公、昭公及定公六位鲁侯,位列三卿之首,独专国政。魯成公元年(前590年)行「作丘甲」。季武子时期,通过一系列的政策从不同角度削弱公室的勢力。襄公十一年, 增设三军。季武子、叔孙穆叔、孟献子分三军,一卿主一军之征赋,由是三桓强于公室。当年,周武王封周公旦于鲁,按周礼“天子六军,诸侯大国三军”,鲁有三军。自文公以来,鲁国弱而从霸主之令,若军多则贡多,遂自减中军,只剩上下二军,属于公室,“有事,三卿更帅以征伐”不得专其民。季武子欲专其民,遂增设中军,三桓分三军之民。襄公十二年,三桓“十二分其国民,三家得七,公得五,国民不尽属公,公室已是卑矣”。

昭公外逃:昭公五年,季武子罢中军。四分公室,季孙称左师,孟氏称右师,叔孙氏则自以叔孙为军名,“三家自取其税,减已税以贡于公,国民不复属于公,公室弥益卑矣”。公室奋起反击,昭公二十五年,在郈昭伯、公若等人的劝说下,鲁昭公发兵伐季氏。而孟氏、叔孙氏认为唇亡齿寒,三桓是一荣俱荣、一损俱损,于是发兵救援。结果昭公外逃,而季平子专权,摄行君位将近十年。魯昭公被三家驅逐,客死異鄉。其后不久,三桓属下的家臣陽虎等人控制国政,一度形成“陪臣執国命”的局面。魯定公時(前509年~前495年),陽虎失敗出奔 ,三桓重新掌權 ,後魯哀公(前494年~前468年在位)圖謀恢復君權,同三家大臣衝突加劇,終致流亡越国。

隳三都:季平子的僭越行为,导致其家臣奋起模仿,其中影响最大的莫过于阳虎。定公五年,季平子、叔孙成子相继去世,阳虎发难,囚禁季桓子,逐仲梁怀,随后执掌鲁国权位长达三年。虽然阳虎被三桓赶出了鲁国,但是三桓的影响日渐削弱、公卿之别君臣之礼日渐败坏也成了趋势。这个时候,在位的鲁定公决心削弱三桓,而这个时候三桓内部并不稳定,因为季氏的专权,导致其他两家的不满。定公十年,齐鲁会盟,作为司仪的孔子不仅言谈之间退发难的莱夷之人,更以口舌之利,使得齐国归还汶阳之田。于是,定公以此为契机,重用孔子, 而孔子为了恢复公卿之别、君臣之分,决定以隳三都的方式,逐步消解三桓的强盛势力。季桓子出于防止家臣犯上的考虑,同意隳三都,并派仲由等臣子率兵毁掉自己的费城。然而三桓之中,孟氏反对,他坚持不毁掉自己的成城,结果定公发兵讨伐,却无法攻下。而定公在季氏的唆使下观齐女乐,败坏礼数,更寒了孔子的心。结果,三桓把公室的坚定拥护者孔子赶出了鲁国。

费国独立:魯哀公即位,哀公十二年(前483年)行「用田賦」。哀公十七年(西元前478年),孔子的弟子於曲阜孔子故里建孔庙。根据《史记》的记载,当时孔子的弟子将其“故所居堂”立庙祭祀,庙屋三间,内藏衣、冠、琴、车、书等孔子遗物。哀公想要伐灭三桓,结果反被三桓逐赶,死于有山氏。哀公死后,三桓立公子宁,是为鲁悼公。悼公时期,三桓胜,鲁如小侯,卑于三桓之家。魯元公時(前436年~前416年),三桓逐漸失勢,直到鲁穆公时期(前415年-前383年),鲁国实行改革,任命博士公仪休为鲁相,遂渐从三桓手中收回政权,国政开始奉法循理,摆脱了三桓专政的问题,重新确立了公室的权威。而三桓之一的季氏则据其封邑费、卞,独立成为了费国。

戰国時期

楚灭鲁国:戰国時期魯国國力已衰弱,仍多次與齊国作戰。前323年,鲁景公卒,鲁平公即位,此时正是韩、魏、赵、燕、中山五国相王之年。鲁顷公二年(前278年),秦国破楚国首都郢,楚顷王东迁至陈国。顷公十九年(前261年),楚伐鲁取徐州。顷公二十四年(前256年),鲁国为楚考烈王所灭,迁顷公于下邑,封鲁君于莒。后七年(前249年)鲁顷公死于柯(今山东东阿),鲁国绝祀。

秦朝末年,楚後懷王曾封項羽為魯公。項羽死後,楚地人民都投降漢高祖劉邦,只有魯國不歸順,劉邦本來要以重兵屠殺魯國,後認為魯國長老嚴守禮義,為主死節,所以把項羽的首級拿出來給魯國的長老看,並答應禮葬項羽。之後,劉邦以「魯公」的公爵禮儀,在穀城埋葬項羽,並親自為其哭喪,魯國長老才投降。

汉平帝时期,封鲁顷公八世孙公子宽为褒鲁侯,奉周公祀,公子宽死后谥为“节”,其子公孙相如袭爵。王莽新朝时期,又封公孙相如后裔姬就为褒鲁子。


%% -*- coding: utf-8 -*-
%% Time-stamp: <Chen Wang: 2019-12-26 22:49:48>

\subsection{隐公{\tiny(BC722-BC712)}}

\subsubsection{生平}

魯隱公(?-前712年),姬姓,名息姑,魯国第十四代国君,前722年-前712年在位。魯惠公之子,生母是声子。傳世的魯國史書《春秋》及其三傳的記事都是從魯隱公開始的。

公元前722年,魯惠公死,嫡妻所生的公子軌當時還年幼,所以國人共立息姑攝政,行君事。在位十一年,隱公始終牢記自己是攝政行君事,一心等待公子軌長大,把國君的位置禪讓給他。因此,當公子翬提出請求殺死公子軌時,隱公断然拒絶;結果公子翬怕消息走漏,反而先跟公子軌合作,把隱公刺殺而死,公子軌即位,是為魯桓公。

隱公行君事期間,重視政治、外交,魯國國力較強。隱公除了到棠地觀人捕魚,被認為不合于禮法之外,處理政事、軍事都還比較謹慎公正,與鄰國修好,所以周圍小國如滕國、薛國等國都到魯國朝拜。與鄭國、齊國等強國也結好。

在位期間的卿為公子翬、无骇、公子益师、公子彄、挟、公子豫。

\subsubsection{年表}

% \centering
\begin{longtable}{|>{\centering\scriptsize}m{2em}|>{\centering\scriptsize}m{1.3em}|>{\centering}m{8.8em}|}
  % \caption{秦王政}\\
  \toprule
  \SimHei \normalsize 年数 & \SimHei \scriptsize 公元 & \SimHei 大事件 \tabularnewline
  % \midrule
  \endfirsthead
  \toprule
  \SimHei \normalsize 年数 & \SimHei \scriptsize 公元 & \SimHei 大事件 \tabularnewline
  \midrule
  \endhead
  \midrule
  元年 & -722 & \begin{enumerate}
    \tiny
  \item 三月,公及\xpinyin*{邾}儀父盟于蔑\footnote{公及邾儀父盟于蔑,邾子克也,未王命,故不書爵,曰儀父,貴之也,公攝位,而欲求好於邾,故為蔑之盟。}。
  \item 夏,五月,鄭伯克段于鄢\footnote{初,鄭武公娶于申,曰武姜。生莊公及共叔段。莊公寤生,驚姜氏,故名曰寤生,遂惡之。愛共叔段,欲立之。亟請於武公,公弗許。及莊公即位,為之請制。公曰:「制,巖邑也,虢叔死焉,佗邑唯命。」請京,使居之,謂之「京城大叔」。祭仲曰:「都城過百雉,國之害也,先王之制:大都不過參國之一;中,五之一;小,九之一。今京不度,非制也,君將不堪。」公曰:「姜氏欲之,焉辟害?」對曰:「姜氏何厭之有!不如早為之所,無使滋蔓。蔓,難圖也。蔓草猶不可除,況君之寵弟乎!」公曰:「多行不義,必自斃,子姑待之。」既而大叔命西鄙、北鄙貳於己。公子呂曰:「國不堪貳,君將若之何?欲與大叔,臣請事之;若弗與,則請除之,無生民心。」公曰:「無庸,將自及。」大叔又收貳以為己邑,至于廩延。子封曰:「可矣,厚將得眾。」公曰:「不義不暱,厚將崩。」大叔完聚,繕甲兵,具卒乘,將襲鄭。夫人將啟之。公聞其期,曰:「可矣。」命子封帥車二百乘以伐京。京叛大叔段,段入于鄢,公伐諸鄢。五月辛丑,大叔出奔共。}。
  \item 八月,紀人伐夷。
  \item 秋,七月,天王使宰咺來歸惠公仲子之賵\footnote{天子七月而葬,同軌畢至,諸侯五月,同盟至,大夫三月,同位至,士踰月,外姻至,贈死不及尸,弔生不及哀,豫凶事,非禮也。}。
  \item 九月,及宋人盟于宿。
  \end{enumerate} \tabularnewline\hline
  二年 & -721 & \tabularnewline\hline
  三年 & -720 & \tabularnewline\hline
  四年 & -719 & \tabularnewline\hline
  五年 & -718 & \tabularnewline\hline
  六年 & -717 & \tabularnewline\hline
  七年 & -716 & \tabularnewline\hline
  八年 & -715 & \tabularnewline\hline
  九年 & -714 & \tabularnewline\hline
  十年 & -713 & \tabularnewline\hline
  十一年 & -712 & \tabularnewline
  \bottomrule
\end{longtable}



%%% Local Variables:
%%% mode: latex
%%% TeX-engine: xetex
%%% TeX-master: "../../Main"
%%% End:

% %% -*- coding: utf-8 -*-
%% Time-stamp: <Chen Wang: 2018-07-12 22:55:13>

\subsection{桓公{\tiny(BC711-BC694)}}

% \centering
\begin{longtable}{|>{\centering\scriptsize}m{2em}|>{\centering\scriptsize}m{1.3em}|>{\centering}m{8.8em}|}
  % \caption{秦王政}\\
  \toprule
  \SimHei \normalsize 年数 & \SimHei \scriptsize 公元 & \SimHei 大事件 \tabularnewline
  % \midrule
  \endfirsthead
  \toprule
  \SimHei \normalsize 年数 & \SimHei \scriptsize 公元 & \SimHei 大事件 \tabularnewline
  \midrule
  \endhead
  \midrule
  元年 & -711 & \tabularnewline\hline
  二年 & -710 & \tabularnewline\hline
  三年 & -709 & \tabularnewline\hline
  四年 & -708 & \tabularnewline\hline
  五年 & -707 & \tabularnewline\hline
  六年 & -706 & \tabularnewline\hline
  七年 & -705 & \tabularnewline\hline
  八年 & -704 & \tabularnewline\hline
  九年 & -703 & \tabularnewline\hline
  十年 & -702 & \tabularnewline\hline
  十一年 & -701 & \tabularnewline\hline
  十二年 & -700 & \tabularnewline\hline
  十三年 & -699 & \tabularnewline\hline
  十四年 & -698 & \tabularnewline\hline
  十五年 & -697 & \tabularnewline\hline
  十六年 & -696 & \tabularnewline\hline
  十七年 & -695 & \tabularnewline\hline
  十八年 & -694 & \tabularnewline
  \bottomrule
\end{longtable}

%%% Local Variables:
%%% mode: latex
%%% TeX-engine: xetex
%%% TeX-master: "../../Main"
%%% End:

% %% -*- coding: utf-8 -*-
%% Time-stamp: <Chen Wang: 2018-07-12 23:02:54>

\subsection{庄公{\tiny(BC693-BC662)}}

% \centering
\begin{longtable}{|>{\centering\scriptsize}m{2em}|>{\centering\scriptsize}m{1.3em}|>{\centering}m{8.8em}|}
  % \caption{秦王政}\\
  \toprule
  \SimHei \normalsize 年数 & \SimHei \scriptsize 公元 & \SimHei 大事件 \tabularnewline
  % \midrule
  \endfirsthead
  \toprule
  \SimHei \normalsize 年数 & \SimHei \scriptsize 公元 & \SimHei 大事件 \tabularnewline
  \midrule
  \endhead
  \midrule
  元年 & -693 & \tabularnewline\hline
  二年 & -692 & \tabularnewline\hline
  三年 & -691 & \tabularnewline\hline
  四年 & -690 & \tabularnewline\hline
  五年 & -689 & \tabularnewline\hline
  六年 & -688 & \tabularnewline\hline
  七年 & -687 & \tabularnewline\hline
  八年 & -686 & \tabularnewline\hline
  九年 & -685 & \tabularnewline\hline
  十年 & -684 & \tabularnewline\hline
  十一年 & -683 & \tabularnewline\hline
  十二年 & -682 & \tabularnewline\hline
  十三年 & -681 & \tabularnewline\hline
  十四年 & -680 & \tabularnewline\hline
  十五年 & -679 & \tabularnewline\hline
  十六年 & -678 & \tabularnewline\hline
  十七年 & -677 & \tabularnewline\hline
  十八年 & -676 & \tabularnewline\hline
  十九年 & -675 & \tabularnewline\hline
  二十年 & -674 & \tabularnewline\hline
  二一年 & -673 & \tabularnewline\hline
  二二年 & -672 & \tabularnewline\hline
  二三年 & -671 & \tabularnewline\hline
  二四年 & -670 & \tabularnewline\hline
  二五年 & -669 & \tabularnewline\hline
  二六年 & -668 & \tabularnewline\hline
  二七年 & -667 & \tabularnewline\hline
  二八年 & -666 & \tabularnewline\hline
  二九年 & -665 & \tabularnewline\hline
  三十年 & -664 & \tabularnewline\hline
  三一年 & -663 & \tabularnewline\hline
  三二年 & -662 & \tabularnewline
  \bottomrule
\end{longtable}

%%% Local Variables:
%%% mode: latex
%%% TeX-engine: xetex
%%% TeX-master: "../../Main"
%%% End:

% %% -*- coding: utf-8 -*-
%% Time-stamp: <Chen Wang: 2021-11-01 17:59:15>

\subsection{闵公啟{\tiny(BC661-BC660)}}

\subsubsection{子斑生平}

子般(?-前662年),姬姓,名般,一作斑,,称子表示此时是其父鲁庄公死的当年,魯莊公之子。魯莊公的夫人哀姜是齊國人,無子。莊公臨死前欲立庶子般為嗣君,莊公弟叔牙建議立庄公庶长兄公子慶父,另一位弟弟季友則支持立子般,季友於是借莊公之命赐死叔牙。

三十二年八月,莊公病逝,季友立子般為君,十月慶父殺子般,立莊公另一庶子啟為魯君,是為魯閔公。季友逃亡陳國。

\subsubsection{閔公啟生平}

魯閔公(前669年?-前660年),即姬啟,一名啟方,為春秋諸侯國魯國君主之一,是魯國第十七任君主。他為魯莊公、叔姜的兒子。近人考証謂於周惠王八年(前669年)出生,至周惠王十七年(前660年)去世,年約十歲。(楊伯峻《春秋左傳注》,頁254)

莊公死前,弟弟叔牙建議立莊公庶長兄慶父,另一位弟弟季友則支持立子般,季友於是借莊公之命賜死叔牙,莊公病逝,季友立子般為君,十月慶父殺子般,立莊公另一庶子啟為魯君,即魯閔公,魯閔公亦是齊桓公的外甥,對齊桓公很尊敬,因此齊魯無大事,直到兩年後公子慶父以毒餅殺死魯閔公,齊桓公才派兵迎立魯閔公之弟魯釐公。

在位期間的卿為公子慶父、季友。

\subsubsection{年表}

% \centering
\begin{longtable}{|>{\centering\scriptsize}m{2em}|>{\centering\scriptsize}m{1.3em}|>{\centering}m{8.8em}|}
  % \caption{秦王政}\\
  \toprule
  \SimHei \normalsize 年数 & \SimHei \scriptsize 公元 & \SimHei 大事件 \tabularnewline
  % \midrule
  \endfirsthead
  \toprule
  \SimHei \normalsize 年数 & \SimHei \scriptsize 公元 & \SimHei 大事件 \tabularnewline
  \midrule
  \endhead
  \midrule
  元年 & -661 & \tabularnewline\hline
  二年 & -660 & \tabularnewline
  \bottomrule
\end{longtable}

%%% Local Variables:
%%% mode: latex
%%% TeX-engine: xetex
%%% TeX-master: "../../Main"
%%% End:

% %% -*- coding: utf-8 -*-
%% Time-stamp: <Chen Wang: 2018-07-12 23:08:29>

\subsection{僖公{\tiny(BC659-BC627)}}

% \centering
\begin{longtable}{|>{\centering\scriptsize}m{2em}|>{\centering\scriptsize}m{1.3em}|>{\centering}m{8.8em}|}
  % \caption{秦王政}\\
  \toprule
  \SimHei \normalsize 年数 & \SimHei \scriptsize 公元 & \SimHei 大事件 \tabularnewline
  % \midrule
  \endfirsthead
  \toprule
  \SimHei \normalsize 年数 & \SimHei \scriptsize 公元 & \SimHei 大事件 \tabularnewline
  \midrule
  \endhead
  \midrule
  元年 & -659 & \tabularnewline\hline
  二年 & -658 & \tabularnewline\hline
  三年 & -657 & \tabularnewline\hline
  四年 & -656 & \tabularnewline\hline
  五年 & -655 & \tabularnewline\hline
  六年 & -654 & \tabularnewline\hline
  七年 & -653 & \tabularnewline\hline
  八年 & -652 & \tabularnewline\hline
  九年 & -651 & \tabularnewline\hline
  十年 & -650 & \tabularnewline\hline
  十一年 & -649 & \tabularnewline\hline
  十二年 & -648 & \tabularnewline\hline
  十三年 & -647 & \tabularnewline\hline
  十四年 & -646 & \tabularnewline\hline
  十五年 & -645 & \tabularnewline\hline
  十六年 & -644 & \tabularnewline\hline
  十七年 & -643 & \tabularnewline\hline
  十八年 & -642 & \tabularnewline\hline
  十九年 & -641 & \tabularnewline\hline
  二十年 & -640 & \tabularnewline\hline
  二一年 & -639 & \tabularnewline\hline
  二二年 & -638 & \tabularnewline\hline
  二三年 & -637 & \tabularnewline\hline
  二四年 & -636 & \tabularnewline\hline
  二五年 & -635 & \tabularnewline\hline
  二六年 & -634 & \tabularnewline\hline
  二七年 & -633 & \tabularnewline\hline
  二八年 & -632 & \tabularnewline\hline
  二九年 & -631 & \tabularnewline\hline
  三十年 & -630 & \tabularnewline\hline
  三一年 & -629 & \tabularnewline\hline
  三二年 & -628 & \tabularnewline\hline
  三三年 & -627 & \tabularnewline
  \bottomrule
\end{longtable}

%%% Local Variables:
%%% mode: latex
%%% TeX-engine: xetex
%%% TeX-master: "../../Main"
%%% End:

% %% -*- coding: utf-8 -*-
%% Time-stamp: <Chen Wang: 2018-07-12 23:10:00>

\subsection{文公{\tiny(BC626-BC609)}}

% \centering
\begin{longtable}{|>{\centering\scriptsize}m{2em}|>{\centering\scriptsize}m{1.3em}|>{\centering}m{8.8em}|}
  % \caption{秦王政}\\
  \toprule
  \SimHei \normalsize 年数 & \SimHei \scriptsize 公元 & \SimHei 大事件 \tabularnewline
  % \midrule
  \endfirsthead
  \toprule
  \SimHei \normalsize 年数 & \SimHei \scriptsize 公元 & \SimHei 大事件 \tabularnewline
  \midrule
  \endhead
  \midrule
  元年 & -626 & \tabularnewline\hline
  二年 & -625 & \tabularnewline\hline
  三年 & -624 & \tabularnewline\hline
  四年 & -623 & \tabularnewline\hline
  五年 & -622 & \tabularnewline\hline
  六年 & -621 & \tabularnewline\hline
  七年 & -620 & \tabularnewline\hline
  八年 & -619 & \tabularnewline\hline
  九年 & -618 & \tabularnewline\hline
  十年 & -617 & \tabularnewline\hline
  十一年 & -616 & \tabularnewline\hline
  十二年 & -615 & \tabularnewline\hline
  十三年 & -614 & \tabularnewline\hline
  十四年 & -613 & \tabularnewline\hline
  十五年 & -612 & \tabularnewline\hline
  十六年 & -611 & \tabularnewline\hline
  十七年 & -610 & \tabularnewline\hline
  十八年 & -609 & \tabularnewline
  \bottomrule
\end{longtable}

%%% Local Variables:
%%% mode: latex
%%% TeX-engine: xetex
%%% TeX-master: "../../Main"
%%% End:

% %% -*- coding: utf-8 -*-
%% Time-stamp: <Chen Wang: 2018-07-12 23:11:20>

\subsection{宣公{\tiny(BC608-BC591)}}

% \centering
\begin{longtable}{|>{\centering\scriptsize}m{2em}|>{\centering\scriptsize}m{1.3em}|>{\centering}m{8.8em}|}
  % \caption{秦王政}\\
  \toprule
  \SimHei \normalsize 年数 & \SimHei \scriptsize 公元 & \SimHei 大事件 \tabularnewline
  % \midrule
  \endfirsthead
  \toprule
  \SimHei \normalsize 年数 & \SimHei \scriptsize 公元 & \SimHei 大事件 \tabularnewline
  \midrule
  \endhead
  \midrule
  元年 & -608 & \tabularnewline\hline
  二年 & -607 & \tabularnewline\hline
  三年 & -606 & \tabularnewline\hline
  四年 & -605 & \tabularnewline\hline
  五年 & -604 & \tabularnewline\hline
  六年 & -603 & \tabularnewline\hline
  七年 & -602 & \tabularnewline\hline
  八年 & -601 & \tabularnewline\hline
  九年 & -600 & \tabularnewline\hline
  十年 & -599 & \tabularnewline\hline
  十一年 & -598 & \tabularnewline\hline
  十二年 & -597 & \tabularnewline\hline
  十三年 & -596 & \tabularnewline\hline
  十四年 & -595 & \tabularnewline\hline
  十五年 & -594 & \tabularnewline\hline
  十六年 & -593 & \tabularnewline\hline
  十七年 & -592 & \tabularnewline\hline
  十八年 & -591 & \tabularnewline
  \bottomrule
\end{longtable}

%%% Local Variables:
%%% mode: latex
%%% TeX-engine: xetex
%%% TeX-master: "../../Main"
%%% End:

% %% -*- coding: utf-8 -*-
%% Time-stamp: <Chen Wang: 2021-11-01 17:57:14>

\subsection{成公黑肱{\tiny(BC590-BC573)}}

\subsubsection{生平}

魯成公(?-前573年),姬姓,名黑肱,為東周春秋時期諸侯國魯國的一位君主,是魯國第二十一任君主,承襲父親魯宣公擔任該國君主,在位18年。

在位期間執政為季孫行父、仲孫蔑、叔孫僑如。

魯成公元年(前590年),季孫行父在魯國實行一種「作丘(地區單位)甲」的地區編制。

魯成公二年(前589年)春天,齊國要攻打魯、衛兩國。魯、衛兩國大夫請求晉國出兵,晉國以郤克為主將率兵救討伐齊國,以救援魯、衛二國。同一年,齊頃公親率齊軍南下攻打魯國龍邑(今山東泰安東南),寵臣盧蒲就癸被殺,頃公怒而攻至巢丘(今山東泰安境內)。季孫行父率魯軍幫晉、衛、曹等國,去攻打齊國的鞍(今山東濟南市)。齊頃公在鞍之戰大敗,齊頃公被晉軍追逼,「差點被俘,幸得大夫逢丑父相救,二人互換衣服,佯命齊頃公到山腳華泉取水,得以逃走。同年十一月,魯國的魯成公同蔡景侯、許靈公、秦國右大夫說、宋國華元、陳國公孫寧、衛國孫良夫、鄭國子良、齊國大夫、曹、邾、薛、鄫等多國代表參與由楚國公子嬰齊在蜀(今山東省泰安市東南)所主辦的會盟。

魯成公七年(前584年),吳國攻打鄰近魯國的郯國,郯國被納入吳國的領土的事,因此季孫行父向魯國國君發出「中國不振旅,蠻夷(指吳國)來伐」的警告。

魯成公十六年(前575年),魯成公的夫人定姒產下魯成公的兒子姬午。

魯成公十八年(前573年),魯成公薨,姬午即位(即後來的魯襄公)。

\subsubsection{年表}

% \centering
\begin{longtable}{|>{\centering\scriptsize}m{2em}|>{\centering\scriptsize}m{1.3em}|>{\centering}m{8.8em}|}
  % \caption{秦王政}\\
  \toprule
  \SimHei \normalsize 年数 & \SimHei \scriptsize 公元 & \SimHei 大事件 \tabularnewline
  % \midrule
  \endfirsthead
  \toprule
  \SimHei \normalsize 年数 & \SimHei \scriptsize 公元 & \SimHei 大事件 \tabularnewline
  \midrule
  \endhead
  \midrule
  元年 & -590 & \tabularnewline\hline
  二年 & -589 & \tabularnewline\hline
  三年 & -588 & \tabularnewline\hline
  四年 & -587 & \tabularnewline\hline
  五年 & -586 & \tabularnewline\hline
  六年 & -585 & \tabularnewline\hline
  七年 & -584 & \tabularnewline\hline
  八年 & -583 & \tabularnewline\hline
  九年 & -582 & \tabularnewline\hline
  十年 & -581 & \tabularnewline\hline
  十一年 & -580 & \tabularnewline\hline
  十二年 & -579 & \tabularnewline\hline
  十三年 & -578 & \tabularnewline\hline
  十四年 & -577 & \tabularnewline\hline
  十五年 & -576 & \tabularnewline\hline
  十六年 & -575 & \tabularnewline\hline
  十七年 & -574 & \tabularnewline\hline
  十八年 & -573 & \tabularnewline
  \bottomrule
\end{longtable}

%%% Local Variables:
%%% mode: latex
%%% TeX-engine: xetex
%%% TeX-master: "../../Main"
%%% End:

% %% -*- coding: utf-8 -*-
%% Time-stamp: <Chen Wang: 2021-11-01 17:58:32>

\subsection{襄公午{\tiny(BC572-BC542)}}

\subsubsection{生平}

魯襄公(前575年-前542年),姬姓,名午,春秋時代魯國的第二十二代君主,魯成公之子,於魯成公十六年(公元前575年)誕生。

魯成公十八年(前573年),魯成公去世,由四歲的太子午即君主之位,是為魯襄公。執政為正卿司徒季孫行父,保持魯國的相對穩定。

魯襄公五年(前568年),魯襄公九歲,執政正卿為季孫行父、仲孫蔑。季孫行父去世,行父以薄葬來進行下葬儀式,這時魯襄公很感動地稱讚的說行父是個廉吏,於是襄公給行父的諡號為「文」。

\subsubsection{年表}

% \centering
\begin{longtable}{|>{\centering\scriptsize}m{2em}|>{\centering\scriptsize}m{1.3em}|>{\centering}m{8.8em}|}
  % \caption{秦王政}\\
  \toprule
  \SimHei \normalsize 年数 & \SimHei \scriptsize 公元 & \SimHei 大事件 \tabularnewline
  % \midrule
  \endfirsthead
  \toprule
  \SimHei \normalsize 年数 & \SimHei \scriptsize 公元 & \SimHei 大事件 \tabularnewline
  \midrule
  \endhead
  \midrule
  元年 & -572 & \tabularnewline\hline
  二年 & -571 & \tabularnewline\hline
  三年 & -570 & \tabularnewline\hline
  四年 & -569 & \tabularnewline\hline
  五年 & -568 & \tabularnewline\hline
  六年 & -567 & \tabularnewline\hline
  七年 & -566 & \tabularnewline\hline
  八年 & -565 & \tabularnewline\hline
  九年 & -564 & \tabularnewline\hline
  十年 & -563 & \tabularnewline\hline
  十一年 & -562 & \tabularnewline\hline
  十二年 & -561 & \tabularnewline\hline
  十三年 & -560 & \tabularnewline\hline
  十四年 & -559 & \tabularnewline\hline
  十五年 & -558 & \tabularnewline\hline
  十六年 & -557 & \tabularnewline\hline
  十七年 & -556 & \tabularnewline\hline
  十八年 & -555 & \tabularnewline\hline
  十九年 & -554 & \tabularnewline\hline
  二十年 & -553 & \tabularnewline\hline
  二一年 & -552 & \tabularnewline\hline
  二二年 & -551 & \tabularnewline\hline
  二三年 & -550 & \tabularnewline\hline
  二四年 & -549 & \tabularnewline\hline
  二五年 & -548 & \tabularnewline\hline
  二六年 & -547 & \tabularnewline\hline
  二七年 & -546 & \tabularnewline\hline
  二八年 & -545 & \tabularnewline\hline
  二九年 & -544 & \tabularnewline\hline
  三十年 & -543 & \tabularnewline\hline
  三一年 & -542 & \tabularnewline
  \bottomrule
\end{longtable}

%%% Local Variables:
%%% mode: latex
%%% TeX-engine: xetex
%%% TeX-master: "../../Main"
%%% End:

% %% -*- coding: utf-8 -*-
%% Time-stamp: <Chen Wang: 2021-11-01 18:02:17>

\subsection{昭公稠{\tiny(BC541-BC510)}}

\subsubsection{子野生平}

子野(?-前542年),姬姓,名野,子表示此时是其父鲁襄公死的当年。子野是魯襄公的庶子,魯昭公和鲁定公之兄,母为敬归。

前542年六月二十八日,魯襄公去世,鲁国人拥立太子子野即位,住在季氏那里。九月十一日,子野由于哀痛过度而死。

魯國人便擁立子野生母敬归的妹妹齊歸生的儿子公子裯為国君,是為魯昭公。

\subsubsection{昭公稠生平}

魯昭公(?-前510年),姬姓,名稠,魯国之二十四代君主。前542年即位,前517年,魯昭公伐季孙氏,但大败,魯昭公逃到齐国,前510年,昭公死。在其任內,他嘗試與季平子政治角力,演變成「鬥雞之變」,使昭公逃到齊國。

在位期間執政為季孫宿、叔孫婼、仲孫貜。

魯昭公二十三年(前519年),叔孫昭子將魯政讓位給季孫意如。

鬥雞之變後,在位期間執政為仲孫何忌、叔孫不敢。

\subsubsection{年表}

% \centering
\begin{longtable}{|>{\centering\scriptsize}m{2em}|>{\centering\scriptsize}m{1.3em}|>{\centering}m{8.8em}|}
  % \caption{秦王政}\\
  \toprule
  \SimHei \normalsize 年数 & \SimHei \scriptsize 公元 & \SimHei 大事件 \tabularnewline
  % \midrule
  \endfirsthead
  \toprule
  \SimHei \normalsize 年数 & \SimHei \scriptsize 公元 & \SimHei 大事件 \tabularnewline
  \midrule
  \endhead
  \midrule
  元年 & -541 & \tabularnewline\hline
  二年 & -540 & \tabularnewline\hline
  三年 & -539 & \tabularnewline\hline
  四年 & -538 & \tabularnewline\hline
  五年 & -537 & \tabularnewline\hline
  六年 & -536 & \tabularnewline\hline
  七年 & -535 & \tabularnewline\hline
  八年 & -534 & \tabularnewline\hline
  九年 & -533 & \tabularnewline\hline
  十年 & -532 & \tabularnewline\hline
  十一年 & -531 & \tabularnewline\hline
  十二年 & -530 & \tabularnewline\hline
  十三年 & -529 & \tabularnewline\hline
  十四年 & -528 & \tabularnewline\hline
  十五年 & -527 & \tabularnewline\hline
  十六年 & -526 & \tabularnewline\hline
  十七年 & -525 & \tabularnewline\hline
  十八年 & -524 & \tabularnewline\hline
  十九年 & -523 & \tabularnewline\hline
  二十年 & -522 & \tabularnewline\hline
  二一年 & -521 & \tabularnewline\hline
  二二年 & -520 & \tabularnewline\hline
  二三年 & -519 & \tabularnewline\hline
  二四年 & -518 & \tabularnewline\hline
  二五年 & -517 & \tabularnewline\hline
  二六年 & -516 & \tabularnewline\hline
  二七年 & -515 & \tabularnewline\hline
  二八年 & -514 & \tabularnewline\hline
  二九年 & -513 & \tabularnewline\hline
  三十年 & -512 & \tabularnewline\hline
  三一年 & -511 & \tabularnewline\hline
  三二年 & -510 & \tabularnewline
  \bottomrule
\end{longtable}

%%% Local Variables:
%%% mode: latex
%%% TeX-engine: xetex
%%% TeX-master: "../../Main"
%%% End:

% %% -*- coding: utf-8 -*-
%% Time-stamp: <Chen Wang: 2021-11-01 18:02:49>

\subsection{定公宋{\tiny(BC509-BC495)}}

\subsubsection{生平}

魯定公(前556年-前495年),姬姓,名宋,為中國春秋時期諸侯國魯國君主之一,是魯國第二十五任君主。他為魯昭公的庶弟,承襲魯昭公擔任該國君主,在位15年。

在位期間執政為季孫意如、叔孫不敢、仲孫何忌、季孫斯、叔孫州仇,其中公元前505年~前503年,執政主官為季孫氏家宰陽虎。前501年~前497年,任命孔子為大司寇。期間行攝相事。

\subsubsection{年表}

% \centering
\begin{longtable}{|>{\centering\scriptsize}m{2em}|>{\centering\scriptsize}m{1.3em}|>{\centering}m{8.8em}|}
  % \caption{秦王政}\\
  \toprule
  \SimHei \normalsize 年数 & \SimHei \scriptsize 公元 & \SimHei 大事件 \tabularnewline
  % \midrule
  \endfirsthead
  \toprule
  \SimHei \normalsize 年数 & \SimHei \scriptsize 公元 & \SimHei 大事件 \tabularnewline
  \midrule
  \endhead
  \midrule
  元年 & -509 & \tabularnewline\hline
  二年 & -508 & \tabularnewline\hline
  三年 & -507 & \tabularnewline\hline
  四年 & -506 & \tabularnewline\hline
  五年 & -505 & \tabularnewline\hline
  六年 & -504 & \tabularnewline\hline
  七年 & -503 & \tabularnewline\hline
  八年 & -502 & \tabularnewline\hline
  九年 & -501 & \tabularnewline\hline
  十年 & -500 & \tabularnewline\hline
  十一年 & -499 & \tabularnewline\hline
  十二年 & -498 & \tabularnewline\hline
  十三年 & -497 & \tabularnewline\hline
  十四年 & -496 & \tabularnewline\hline
  十五年 & -495 & \tabularnewline
  \bottomrule
\end{longtable}

%%% Local Variables:
%%% mode: latex
%%% TeX-engine: xetex
%%% TeX-master: "../../Main"
%%% End:

% %% -*- coding: utf-8 -*-
%% Time-stamp: <Chen Wang: 2021-11-01 18:03:17>

\subsection{哀公將{\tiny(BC494-BC467)}}

\subsubsection{生平}

鲁哀公(约前508年-前468年),姬姓,名將,為春秋諸侯國魯國君主之一,是魯國第二十六任君主。魯定公之子,承襲魯定公擔任該國君主,在位27年。

鲁哀公在位时,鲁国大权被卿大夫家族把持,史称三桓,即所谓“政在大夫”。鲁哀公曾经试图恢复君主权力,同三家大夫冲突加剧,终致流亡越国。魯哀公27年,鲁哀公通过邾国逃到越国。

在位期間執政的士大夫為季孫斯、叔孫州仇、仲孫何忌、季孫肥、叔孫舒、仲孫彘。

\subsubsection{年表}

% \centering
\begin{longtable}{|>{\centering\scriptsize}m{2em}|>{\centering\scriptsize}m{1.3em}|>{\centering}m{8.8em}|}
  % \caption{秦王政}\\
  \toprule
  \SimHei \normalsize 年数 & \SimHei \scriptsize 公元 & \SimHei 大事件 \tabularnewline
  % \midrule
  \endfirsthead
  \toprule
  \SimHei \normalsize 年数 & \SimHei \scriptsize 公元 & \SimHei 大事件 \tabularnewline
  \midrule
  \endhead
  \midrule
  元年 & -494 & \tabularnewline\hline
  二年 & -493 & \tabularnewline\hline
  三年 & -492 & \tabularnewline\hline
  四年 & -491 & \tabularnewline\hline
  五年 & -490 & \tabularnewline\hline
  六年 & -489 & \tabularnewline\hline
  七年 & -488 & \tabularnewline\hline
  八年 & -487 & \tabularnewline\hline
  九年 & -486 & \tabularnewline\hline
  十年 & -485 & \tabularnewline\hline
  十一年 & -484 & \tabularnewline\hline
  十二年 & -483 & \tabularnewline\hline
  十三年 & -482 & \tabularnewline\hline
  十四年 & -481 & \tabularnewline\hline
  十五年 & -480 & \tabularnewline\hline
  十六年 & -479 & \tabularnewline\hline
  十七年 & -478 & \tabularnewline\hline
  十八年 & -477 & \tabularnewline\hline
  十九年 & -476 & \tabularnewline\hline
  二十年 & -475 & \tabularnewline\hline
  二一年 & -474 & \tabularnewline\hline
  二二年 & -473 & \tabularnewline\hline
  二三年 & -472 & \tabularnewline\hline
  二四年 & -471 & \tabularnewline\hline
  二五年 & -470 & \tabularnewline\hline
  二六年 & -469 & \tabularnewline\hline
  二七年 & -468 & \tabularnewline\hline
  二八年 & -467 & \tabularnewline
  \bottomrule
\end{longtable}

%%% Local Variables:
%%% mode: latex
%%% TeX-engine: xetex
%%% TeX-master: "../../Main"
%%% End:

% %% -*- coding: utf-8 -*-
%% Time-stamp: <Chen Wang: 2018-07-12 23:19:27>

\subsection{悼公{\tiny(BC466-BC429)}}

% \centering
\begin{longtable}{|>{\centering\scriptsize}m{2em}|>{\centering\scriptsize}m{1.3em}|>{\centering}m{8.8em}|}
  % \caption{秦王政}\\
  \toprule
  \SimHei \normalsize 年数 & \SimHei \scriptsize 公元 & \SimHei 大事件 \tabularnewline
  % \midrule
  \endfirsthead
  \toprule
  \SimHei \normalsize 年数 & \SimHei \scriptsize 公元 & \SimHei 大事件 \tabularnewline
  \midrule
  \endhead
  \midrule
  元年 & -466 & \tabularnewline\hline
  二年 & -465 & \tabularnewline\hline
  三年 & -464 & \tabularnewline\hline
  四年 & -463 & \tabularnewline\hline
  五年 & -462 & \tabularnewline\hline
  六年 & -461 & \tabularnewline\hline
  七年 & -460 & \tabularnewline\hline
  八年 & -459 & \tabularnewline\hline
  九年 & -458 & \tabularnewline\hline
  十年 & -457 & \tabularnewline\hline
  十一年 & -456 & \tabularnewline\hline
  十二年 & -455 & \tabularnewline\hline
  十三年 & -454 & \tabularnewline\hline
  十四年 & -453 & \tabularnewline\hline
  十五年 & -452 & \tabularnewline\hline
  十六年 & -451 & \tabularnewline\hline
  十七年 & -450 & \tabularnewline\hline
  十八年 & -449 & \tabularnewline\hline
  十九年 & -448 & \tabularnewline\hline
  二十年 & -447 & \tabularnewline\hline
  二一年 & -446 & \tabularnewline\hline
  二二年 & -445 & \tabularnewline\hline
  二三年 & -444 & \tabularnewline\hline
  二四年 & -443 & \tabularnewline\hline
  二五年 & -442 & \tabularnewline\hline
  二六年 & -441 & \tabularnewline\hline
  二七年 & -440 & \tabularnewline\hline
  二八年 & -439 & \tabularnewline\hline
  二九年 & -438 & \tabularnewline\hline
  三十年 & -437 & \tabularnewline\hline
  三一年 & -436 & \tabularnewline\hline
  三二年 & -435 & \tabularnewline\hline
  三三年 & -434 & \tabularnewline\hline
  三四年 & -433 & \tabularnewline\hline
  三五年 & -432 & \tabularnewline\hline
  三六年 & -431 & \tabularnewline\hline
  三七年 & -430 & \tabularnewline\hline
  三八年 & -429 & \tabularnewline
  \bottomrule
\end{longtable}

%%% Local Variables:
%%% mode: latex
%%% TeX-engine: xetex
%%% TeX-master: "../../Main"
%%% End:

% %% -*- coding: utf-8 -*-
%% Time-stamp: <Chen Wang: 2018-07-12 23:20:46>

\subsection{元公{\tiny(BC428-BC408)}}

% \centering
\begin{longtable}{|>{\centering\scriptsize}m{2em}|>{\centering\scriptsize}m{1.3em}|>{\centering}m{8.8em}|}
  % \caption{秦王政}\\
  \toprule
  \SimHei \normalsize 年数 & \SimHei \scriptsize 公元 & \SimHei 大事件 \tabularnewline
  % \midrule
  \endfirsthead
  \toprule
  \SimHei \normalsize 年数 & \SimHei \scriptsize 公元 & \SimHei 大事件 \tabularnewline
  \midrule
  \endhead
  \midrule
  元年 & -428 & \tabularnewline\hline
  二年 & -427 & \tabularnewline\hline
  三年 & -426 & \tabularnewline\hline
  四年 & -425 & \tabularnewline\hline
  五年 & -424 & \tabularnewline\hline
  六年 & -423 & \tabularnewline\hline
  七年 & -422 & \tabularnewline\hline
  八年 & -421 & \tabularnewline\hline
  九年 & -420 & \tabularnewline\hline
  十年 & -419 & \tabularnewline\hline
  十一年 & -418 & \tabularnewline\hline
  十二年 & -417 & \tabularnewline\hline
  十三年 & -416 & \tabularnewline\hline
  十四年 & -415 & \tabularnewline\hline
  十五年 & -414 & \tabularnewline\hline
  十六年 & -413 & \tabularnewline\hline
  十七年 & -412 & \tabularnewline\hline
  十八年 & -411 & \tabularnewline\hline
  十九年 & -410 & \tabularnewline\hline
  二十年 & -409 & \tabularnewline\hline
  二一年 & -408 & \tabularnewline
  \bottomrule
\end{longtable}

%%% Local Variables:
%%% mode: latex
%%% TeX-engine: xetex
%%% TeX-master: "../../Main"
%%% End:

% %% -*- coding: utf-8 -*-
%% Time-stamp: <Chen Wang: 2018-07-12 23:23:22>

\subsection{穆公{\tiny(BC407-BC376)}}

% \centering
\begin{longtable}{|>{\centering\scriptsize}m{2em}|>{\centering\scriptsize}m{1.3em}|>{\centering}m{8.8em}|}
  % \caption{秦王政}\\
  \toprule
  \SimHei \normalsize 年数 & \SimHei \scriptsize 公元 & \SimHei 大事件 \tabularnewline
  % \midrule
  \endfirsthead
  \toprule
  \SimHei \normalsize 年数 & \SimHei \scriptsize 公元 & \SimHei 大事件 \tabularnewline
  \midrule
  \endhead
  \midrule
  元年 & -407 & \tabularnewline\hline
  二年 & -406 & \tabularnewline\hline
  三年 & -405 & \tabularnewline\hline
  四年 & -404 & \tabularnewline\hline
  五年 & -403 & \tabularnewline
  \bottomrule
\end{longtable}

%%% Local Variables:
%%% mode: latex
%%% TeX-engine: xetex
%%% TeX-master: "../../Main"
%%% End:



%%% Local Variables:
%%% mode: latex
%%% TeX-engine: xetex
%%% TeX-master: "../../Main"
%%% End:
             %魯國
% %% -*- coding: utf-8 -*-
%% Time-stamp: <Chen Wang: 2019-12-26 22:14:20>

\section{郑}

%% -*- coding: utf-8 -*-
%% Time-stamp: <Chen Wang: 2018-07-16 21:59:24>

\subsection{庄公{\tiny(BC743-BC701)}}

% \centering
\begin{longtable}{|>{\centering\scriptsize}m{2em}|>{\centering\scriptsize}m{1.3em}|>{\centering}m{8.8em}|}
  % \caption{秦王政}\\
  \toprule
  \SimHei \normalsize 年数 & \SimHei \scriptsize 公元 & \SimHei 大事件 \tabularnewline
  % \midrule
  \endfirsthead
  \toprule
  \SimHei \normalsize 年数 & \SimHei \scriptsize 公元 & \SimHei 大事件 \tabularnewline
  \midrule
  \endhead
  \midrule
  % 元年 & -743 & \tabularnewline\hline
  % 二年 & -742 & \tabularnewline\hline
  % 三年 & -741 & \tabularnewline\hline
  % 四年 & -740 & \tabularnewline\hline
  % 五年 & -739 & \tabularnewline\hline
  % 六年 & -738 & \tabularnewline\hline
  % 七年 & -737 & \tabularnewline\hline
  % 八年 & -736 & \tabularnewline\hline
  % 九年 & -735 & \tabularnewline\hline
  % 十年 & -734 & \tabularnewline\hline
  % 十一年 & -733 & \tabularnewline\hline
  % 十二年 & -732 & \tabularnewline\hline
  % 十三年 & -731 & \tabularnewline\hline
  % 十四年 & -730 & \tabularnewline\hline
  % 十五年 & -729 & \tabularnewline\hline
  % 十六年 & -728 & \tabularnewline\hline
  % 十七年 & -727 & \tabularnewline\hline
  % 十八年 & -726 & \tabularnewline\hline
  % 十九年 & -725 & \tabularnewline\hline
  % 二十年 & -724 & \tabularnewline\hline
  % 二一年 & -723 & \tabularnewline\hline
  二二年 & -722 & \tabularnewline\hline
  二三年 & -721 & \tabularnewline\hline
  二四年 & -720 & \tabularnewline\hline
  二五年 & -719 & \tabularnewline\hline
  二六年 & -718 & \tabularnewline\hline
  二七年 & -717 & \tabularnewline\hline
  二八年 & -716 & \tabularnewline\hline
  二九年 & -715 & \tabularnewline\hline
  三十年 & -714 & \tabularnewline\hline
  三一年 & -713 & \tabularnewline\hline
  三二年 & -712 & \tabularnewline\hline
  三三年 & -711 & \tabularnewline\hline
  三四年 & -710 & \tabularnewline\hline
  三五年 & -709 & \tabularnewline\hline
  三六年 & -708 & \tabularnewline\hline
  三七年 & -707 & \tabularnewline\hline
  三八年 & -706 & \tabularnewline\hline
  三九年 & -705 & \tabularnewline\hline
  四十年 & -704 & \tabularnewline\hline
  四一年 & -703 & \tabularnewline\hline
  四二年 & -702 & \tabularnewline\hline
  四三年 & -701 & \tabularnewline
  \bottomrule
\end{longtable}

%%% Local Variables:
%%% mode: latex
%%% TeX-engine: xetex
%%% TeX-master: "../../Main"
%%% End:

%% -*- coding: utf-8 -*-
%% Time-stamp: <Chen Wang: 2018-07-16 22:01:47>

\subsection{昭公{\tiny(BC701)}}

% \centering
\begin{longtable}{|>{\centering\scriptsize}m{2em}|>{\centering\scriptsize}m{1.3em}|>{\centering}m{8.8em}|}
  % \caption{秦王政}\\
  \toprule
  \SimHei \normalsize 年数 & \SimHei \scriptsize 公元 & \SimHei 大事件 \tabularnewline
  % \midrule
  \endfirsthead
  \toprule
  \SimHei \normalsize 年数 & \SimHei \scriptsize 公元 & \SimHei 大事件 \tabularnewline
  \midrule
  \endhead
  \midrule
  元年 & -701 & \tabularnewline
  \bottomrule
\end{longtable}

%%% Local Variables:
%%% mode: latex
%%% TeX-engine: xetex
%%% TeX-master: "../../Main"
%%% End:

%% -*- coding: utf-8 -*-
%% Time-stamp: <Chen Wang: 2018-07-16 22:03:12>

\subsection{厉公{\tiny(BC700-BC697)}}

% \centering
\begin{longtable}{|>{\centering\scriptsize}m{2em}|>{\centering\scriptsize}m{1.3em}|>{\centering}m{8.8em}|}
  % \caption{秦王政}\\
  \toprule
  \SimHei \normalsize 年数 & \SimHei \scriptsize 公元 & \SimHei 大事件 \tabularnewline
  % \midrule
  \endfirsthead
  \toprule
  \SimHei \normalsize 年数 & \SimHei \scriptsize 公元 & \SimHei 大事件 \tabularnewline
  \midrule
  \endhead
  \midrule
  元年 & -700 & \tabularnewline\hline
  二年 & -699 & \tabularnewline\hline
  三年 & -698 & \tabularnewline\hline
  四年 & -697 & \tabularnewline
  \bottomrule
\end{longtable}

%%% Local Variables:
%%% mode: latex
%%% TeX-engine: xetex
%%% TeX-master: "../../Main"
%%% End:

%% -*- coding: utf-8 -*-
%% Time-stamp: <Chen Wang: 2018-07-16 22:09:37>

\subsection{昭公复辟{\tiny(BC696-BC695)}}

% \centering
\begin{longtable}{|>{\centering\scriptsize}m{2em}|>{\centering\scriptsize}m{1.3em}|>{\centering}m{8.8em}|}
  % \caption{秦王政}\\
  \toprule
  \SimHei \normalsize 年数 & \SimHei \scriptsize 公元 & \SimHei 大事件 \tabularnewline
  % \midrule
  \endfirsthead
  \toprule
  \SimHei \normalsize 年数 & \SimHei \scriptsize 公元 & \SimHei 大事件 \tabularnewline
  \midrule
  \endhead
  \midrule
  元年 & -696 & \tabularnewline\hline
  二年 & -695 & \tabularnewline
  \bottomrule
\end{longtable}

%%% Local Variables:
%%% mode: latex
%%% TeX-engine: xetex
%%% TeX-master: "../../Main"
%%% End:

%% -*- coding: utf-8 -*-
%% Time-stamp: <Chen Wang: 2018-07-16 22:11:25>

\subsection{子亹{\tiny(BC694)}}

% \centering
\begin{longtable}{|>{\centering\scriptsize}m{2em}|>{\centering\scriptsize}m{1.3em}|>{\centering}m{8.8em}|}
  % \caption{秦王政}\\
  \toprule
  \SimHei \normalsize 年数 & \SimHei \scriptsize 公元 & \SimHei 大事件 \tabularnewline
  % \midrule
  \endfirsthead
  \toprule
  \SimHei \normalsize 年数 & \SimHei \scriptsize 公元 & \SimHei 大事件 \tabularnewline
  \midrule
  \endhead
  \midrule
  元年 & -694 & \tabularnewline
  \bottomrule
\end{longtable}

%%% Local Variables:
%%% mode: latex
%%% TeX-engine: xetex
%%% TeX-master: "../../Main"
%%% End:

%% -*- coding: utf-8 -*-
%% Time-stamp: <Chen Wang: 2018-07-16 22:13:17>

\subsection{子婴{\tiny(BC693-BC680)}}

% \centering
\begin{longtable}{|>{\centering\scriptsize}m{2em}|>{\centering\scriptsize}m{1.3em}|>{\centering}m{8.8em}|}
  % \caption{秦王政}\\
  \toprule
  \SimHei \normalsize 年数 & \SimHei \scriptsize 公元 & \SimHei 大事件 \tabularnewline
  % \midrule
  \endfirsthead
  \toprule
  \SimHei \normalsize 年数 & \SimHei \scriptsize 公元 & \SimHei 大事件 \tabularnewline
  \midrule
  \endhead
  \midrule
  元年 & -693 & \tabularnewline\hline
  二年 & -692 & \tabularnewline\hline
  三年 & -691 & \tabularnewline\hline
  四年 & -690 & \tabularnewline\hline
  五年 & -689 & \tabularnewline\hline
  六年 & -688 & \tabularnewline\hline
  七年 & -687 & \tabularnewline\hline
  八年 & -686 & \tabularnewline\hline
  九年 & -685 & \tabularnewline\hline
  十年 & -684 & \tabularnewline\hline
  十一年 & -683 & \tabularnewline\hline
  十二年 & -682 & \tabularnewline\hline
  十三年 & -681 & \tabularnewline\hline
  十四年 & -680 & \tabularnewline
  \bottomrule
\end{longtable}

%%% Local Variables:
%%% mode: latex
%%% TeX-engine: xetex
%%% TeX-master: "../../Main"
%%% End:

%% -*- coding: utf-8 -*-
%% Time-stamp: <Chen Wang: 2018-07-16 22:33:25>

\subsection{厉公复辟{\tiny(BC679-BC673)}}

% \centering
\begin{longtable}{|>{\centering\scriptsize}m{2em}|>{\centering\scriptsize}m{1.3em}|>{\centering}m{8.8em}|}
  % \caption{秦王政}\\
  \toprule
  \SimHei \normalsize 年数 & \SimHei \scriptsize 公元 & \SimHei 大事件 \tabularnewline
  % \midrule
  \endfirsthead
  \toprule
  \SimHei \normalsize 年数 & \SimHei \scriptsize 公元 & \SimHei 大事件 \tabularnewline
  \midrule
  \endhead
  \midrule
  元年 & -679 & \tabularnewline\hline
  二年 & -678 & \tabularnewline\hline
  三年 & -677 & \tabularnewline\hline
  四年 & -676 & \tabularnewline\hline
  五年 & -675 & \tabularnewline\hline
  六年 & -674 & \tabularnewline\hline
  七年 & -673 & \tabularnewline
  \bottomrule
\end{longtable}

%%% Local Variables:
%%% mode: latex
%%% TeX-engine: xetex
%%% TeX-master: "../../Main"
%%% End:

%% -*- coding: utf-8 -*-
%% Time-stamp: <Chen Wang: 2018-07-16 22:15:22>

\subsection{文公{\tiny(BC672-BC628)}}

% \centering
\begin{longtable}{|>{\centering\scriptsize}m{2em}|>{\centering\scriptsize}m{1.3em}|>{\centering}m{8.8em}|}
  % \caption{秦王政}\\
  \toprule
  \SimHei \normalsize 年数 & \SimHei \scriptsize 公元 & \SimHei 大事件 \tabularnewline
  % \midrule
  \endfirsthead
  \toprule
  \SimHei \normalsize 年数 & \SimHei \scriptsize 公元 & \SimHei 大事件 \tabularnewline
  \midrule
  \endhead
  \midrule
  元年 & -672 & \tabularnewline\hline
  二年 & -671 & \tabularnewline\hline
  三年 & -670 & \tabularnewline\hline
  四年 & -669 & \tabularnewline\hline
  五年 & -668 & \tabularnewline\hline
  六年 & -667 & \tabularnewline\hline
  七年 & -666 & \tabularnewline\hline
  八年 & -665 & \tabularnewline\hline
  九年 & -664 & \tabularnewline\hline
  十年 & -663 & \tabularnewline\hline
  十一年 & -662 & \tabularnewline\hline
  十二年 & -661 & \tabularnewline\hline
  十三年 & -660 & \tabularnewline\hline
  十四年 & -659 & \tabularnewline\hline
  十五年 & -658 & \tabularnewline\hline
  十六年 & -657 & \tabularnewline\hline
  十七年 & -656 & \tabularnewline\hline
  十八年 & -655 & \tabularnewline\hline
  十九年 & -654 & \tabularnewline\hline
  二十年 & -653 & \tabularnewline\hline
  二一年 & -652 & \tabularnewline\hline
  二二年 & -651 & \tabularnewline\hline
  二三年 & -650 & \tabularnewline\hline
  二四年 & -649 & \tabularnewline\hline
  二五年 & -648 & \tabularnewline\hline
  二六年 & -647 & \tabularnewline\hline
  二七年 & -646 & \tabularnewline\hline
  二八年 & -645 & \tabularnewline\hline
  二九年 & -644 & \tabularnewline\hline
  三十年 & -643 & \tabularnewline\hline
  三一年 & -642 & \tabularnewline\hline
  三二年 & -641 & \tabularnewline\hline
  三三年 & -640 & \tabularnewline\hline
  三四年 & -639 & \tabularnewline\hline
  三五年 & -638 & \tabularnewline\hline
  三六年 & -637 & \tabularnewline\hline
  三七年 & -636 & \tabularnewline\hline
  三八年 & -635 & \tabularnewline\hline
  三九年 & -634 & \tabularnewline\hline
  四十年 & -633 & \tabularnewline\hline
  四一年 & -632 & \tabularnewline\hline
  四二年 & -631 & \tabularnewline\hline
  四三年 & -630 & \tabularnewline\hline
  四四年 & -629 & \tabularnewline\hline
  四五年 & -628 & \tabularnewline
  \bottomrule
\end{longtable}

%%% Local Variables:
%%% mode: latex
%%% TeX-engine: xetex
%%% TeX-master: "../../Main"
%%% End:

%% -*- coding: utf-8 -*-
%% Time-stamp: <Chen Wang: 2018-07-16 22:15:48>

\subsection{穆公{\tiny(BC627-BC606)}}

% \centering
\begin{longtable}{|>{\centering\scriptsize}m{2em}|>{\centering\scriptsize}m{1.3em}|>{\centering}m{8.8em}|}
  % \caption{秦王政}\\
  \toprule
  \SimHei \normalsize 年数 & \SimHei \scriptsize 公元 & \SimHei 大事件 \tabularnewline
  % \midrule
  \endfirsthead
  \toprule
  \SimHei \normalsize 年数 & \SimHei \scriptsize 公元 & \SimHei 大事件 \tabularnewline
  \midrule
  \endhead
  \midrule
  元年 & -627 & \tabularnewline\hline
  二年 & -626 & \tabularnewline\hline
  三年 & -625 & \tabularnewline\hline
  四年 & -624 & \tabularnewline\hline
  五年 & -623 & \tabularnewline\hline
  六年 & -622 & \tabularnewline\hline
  七年 & -621 & \tabularnewline\hline
  八年 & -620 & \tabularnewline\hline
  九年 & -619 & \tabularnewline\hline
  十年 & -618 & \tabularnewline\hline
  十一年 & -617 & \tabularnewline\hline
  十二年 & -616 & \tabularnewline\hline
  十三年 & -615 & \tabularnewline\hline
  十四年 & -614 & \tabularnewline\hline
  十五年 & -613 & \tabularnewline\hline
  十六年 & -612 & \tabularnewline\hline
  十七年 & -611 & \tabularnewline\hline
  十八年 & -610 & \tabularnewline\hline
  十九年 & -609 & \tabularnewline\hline
  二十年 & -608 & \tabularnewline\hline
  二一年 & -607 & \tabularnewline\hline
  二二年 & -606 & \tabularnewline
  \bottomrule
\end{longtable}

%%% Local Variables:
%%% mode: latex
%%% TeX-engine: xetex
%%% TeX-master: "../../Main"
%%% End:

%% -*- coding: utf-8 -*-
%% Time-stamp: <Chen Wang: 2018-07-16 22:16:22>

\subsection{灵公{\tiny(BC605)}}

% \centering
\begin{longtable}{|>{\centering\scriptsize}m{2em}|>{\centering\scriptsize}m{1.3em}|>{\centering}m{8.8em}|}
  % \caption{秦王政}\\
  \toprule
  \SimHei \normalsize 年数 & \SimHei \scriptsize 公元 & \SimHei 大事件 \tabularnewline
  % \midrule
  \endfirsthead
  \toprule
  \SimHei \normalsize 年数 & \SimHei \scriptsize 公元 & \SimHei 大事件 \tabularnewline
  \midrule
  \endhead
  \midrule
  元年 & -605 & \tabularnewline
  \bottomrule
\end{longtable}

%%% Local Variables:
%%% mode: latex
%%% TeX-engine: xetex
%%% TeX-master: "../../Main"
%%% End:

%% -*- coding: utf-8 -*-
%% Time-stamp: <Chen Wang: 2018-07-16 22:17:12>

\subsection{襄公{\tiny(BC604-BC587)}}

% \centering
\begin{longtable}{|>{\centering\scriptsize}m{2em}|>{\centering\scriptsize}m{1.3em}|>{\centering}m{8.8em}|}
  % \caption{秦王政}\\
  \toprule
  \SimHei \normalsize 年数 & \SimHei \scriptsize 公元 & \SimHei 大事件 \tabularnewline
  % \midrule
  \endfirsthead
  \toprule
  \SimHei \normalsize 年数 & \SimHei \scriptsize 公元 & \SimHei 大事件 \tabularnewline
  \midrule
  \endhead
  \midrule
  元年 & -604 & \tabularnewline\hline
  二年 & -603 & \tabularnewline\hline
  三年 & -602 & \tabularnewline\hline
  四年 & -601 & \tabularnewline\hline
  五年 & -600 & \tabularnewline\hline
  六年 & -599 & \tabularnewline\hline
  七年 & -598 & \tabularnewline\hline
  八年 & -597 & \tabularnewline\hline
  九年 & -596 & \tabularnewline\hline
  十年 & -595 & \tabularnewline\hline
  十一年 & -594 & \tabularnewline\hline
  十二年 & -593 & \tabularnewline\hline
  十三年 & -592 & \tabularnewline\hline
  十四年 & -591 & \tabularnewline\hline
  十五年 & -590 & \tabularnewline\hline
  十六年 & -589 & \tabularnewline\hline
  十七年 & -588 & \tabularnewline\hline
  十八年 & -587 & \tabularnewline
  \bottomrule
\end{longtable}

%%% Local Variables:
%%% mode: latex
%%% TeX-engine: xetex
%%% TeX-master: "../../Main"
%%% End:

%% -*- coding: utf-8 -*-
%% Time-stamp: <Chen Wang: 2018-07-16 22:18:44>

\subsection{悼公{\tiny(BC586-BC585)}}

% \centering
\begin{longtable}{|>{\centering\scriptsize}m{2em}|>{\centering\scriptsize}m{1.3em}|>{\centering}m{8.8em}|}
  % \caption{秦王政}\\
  \toprule
  \SimHei \normalsize 年数 & \SimHei \scriptsize 公元 & \SimHei 大事件 \tabularnewline
  % \midrule
  \endfirsthead
  \toprule
  \SimHei \normalsize 年数 & \SimHei \scriptsize 公元 & \SimHei 大事件 \tabularnewline
  \midrule
  \endhead
  \midrule
  元年 & -586 & \tabularnewline\hline
  二年 & -585 & \tabularnewline
  \bottomrule
\end{longtable}

%%% Local Variables:
%%% mode: latex
%%% TeX-engine: xetex
%%% TeX-master: "../../Main"
%%% End:

%% -*- coding: utf-8 -*-
%% Time-stamp: <Chen Wang: 2018-07-16 22:18:36>

\subsection{成公{\tiny(BC584-BC571)}}

% \centering
\begin{longtable}{|>{\centering\scriptsize}m{2em}|>{\centering\scriptsize}m{1.3em}|>{\centering}m{8.8em}|}
  % \caption{秦王政}\\
  \toprule
  \SimHei \normalsize 年数 & \SimHei \scriptsize 公元 & \SimHei 大事件 \tabularnewline
  % \midrule
  \endfirsthead
  \toprule
  \SimHei \normalsize 年数 & \SimHei \scriptsize 公元 & \SimHei 大事件 \tabularnewline
  \midrule
  \endhead
  \midrule
  元年 & -584 & \tabularnewline\hline
  二年 & -583 & \tabularnewline\hline
  三年 & -582 & \tabularnewline\hline
  四年 & -581 & \tabularnewline\hline
  五年 & -580 & \tabularnewline\hline
  六年 & -579 & \tabularnewline\hline
  七年 & -578 & \tabularnewline\hline
  八年 & -577 & \tabularnewline\hline
  九年 & -576 & \tabularnewline\hline
  十年 & -575 & \tabularnewline\hline
  十一年 & -574 & \tabularnewline\hline
  十二年 & -573 & \tabularnewline\hline
  十三年 & -572 & \tabularnewline\hline
  十四年 & -571 & \tabularnewline
  \bottomrule
\end{longtable}

%%% Local Variables:
%%% mode: latex
%%% TeX-engine: xetex
%%% TeX-master: "../../Main"
%%% End:

%% -*- coding: utf-8 -*-
%% Time-stamp: <Chen Wang: 2018-07-16 22:19:42>

\subsection{僖公{\tiny(BC570-BC566)}}

% \centering
\begin{longtable}{|>{\centering\scriptsize}m{2em}|>{\centering\scriptsize}m{1.3em}|>{\centering}m{8.8em}|}
  % \caption{秦王政}\\
  \toprule
  \SimHei \normalsize 年数 & \SimHei \scriptsize 公元 & \SimHei 大事件 \tabularnewline
  % \midrule
  \endfirsthead
  \toprule
  \SimHei \normalsize 年数 & \SimHei \scriptsize 公元 & \SimHei 大事件 \tabularnewline
  \midrule
  \endhead
  \midrule
  元年 & -570 & \tabularnewline\hline
  二年 & -569 & \tabularnewline\hline
  三年 & -568 & \tabularnewline\hline
  四年 & -567 & \tabularnewline\hline
  五年 & -566 & \tabularnewline
  \bottomrule
\end{longtable}

%%% Local Variables:
%%% mode: latex
%%% TeX-engine: xetex
%%% TeX-master: "../../Main"
%%% End:

%% -*- coding: utf-8 -*-
%% Time-stamp: <Chen Wang: 2018-07-16 22:20:30>

\subsection{简公{\tiny(BC565-BC530)}}

% \centering
\begin{longtable}{|>{\centering\scriptsize}m{2em}|>{\centering\scriptsize}m{1.3em}|>{\centering}m{8.8em}|}
  % \caption{秦王政}\\
  \toprule
  \SimHei \normalsize 年数 & \SimHei \scriptsize 公元 & \SimHei 大事件 \tabularnewline
  % \midrule
  \endfirsthead
  \toprule
  \SimHei \normalsize 年数 & \SimHei \scriptsize 公元 & \SimHei 大事件 \tabularnewline
  \midrule
  \endhead
  \midrule
  元年 & -565 & \tabularnewline\hline
  二年 & -564 & \tabularnewline\hline
  三年 & -563 & \tabularnewline\hline
  四年 & -562 & \tabularnewline\hline
  五年 & -561 & \tabularnewline\hline
  六年 & -560 & \tabularnewline\hline
  七年 & -559 & \tabularnewline\hline
  八年 & -558 & \tabularnewline\hline
  九年 & -557 & \tabularnewline\hline
  十年 & -556 & \tabularnewline\hline
  十一年 & -555 & \tabularnewline\hline
  十二年 & -554 & \tabularnewline\hline
  十三年 & -553 & \tabularnewline\hline
  十四年 & -552 & \tabularnewline\hline
  十五年 & -551 & \tabularnewline\hline
  十六年 & -550 & \tabularnewline\hline
  十七年 & -549 & \tabularnewline\hline
  十八年 & -548 & \tabularnewline\hline
  十九年 & -547 & \tabularnewline\hline
  二十年 & -546 & \tabularnewline\hline
  二一年 & -545 & \tabularnewline\hline
  二二年 & -544 & \tabularnewline\hline
  二三年 & -543 & \tabularnewline\hline
  二四年 & -542 & \tabularnewline\hline
  二五年 & -541 & \tabularnewline\hline
  二六年 & -540 & \tabularnewline\hline
  二七年 & -539 & \tabularnewline\hline
  二八年 & -538 & \tabularnewline\hline
  二九年 & -537 & \tabularnewline\hline
  三十年 & -536 & \tabularnewline\hline
  三一年 & -535 & \tabularnewline\hline
  三二年 & -534 & \tabularnewline\hline
  三三年 & -533 & \tabularnewline\hline
  三四年 & -532 & \tabularnewline\hline
  三五年 & -531 & \tabularnewline\hline
  三六年 & -530 & \tabularnewline
  \bottomrule
\end{longtable}

%%% Local Variables:
%%% mode: latex
%%% TeX-engine: xetex
%%% TeX-master: "../../Main"
%%% End:

%% -*- coding: utf-8 -*-
%% Time-stamp: <Chen Wang: 2018-07-16 22:21:18>

\subsection{定公{\tiny(BC529-BC514)}}

% \centering
\begin{longtable}{|>{\centering\scriptsize}m{2em}|>{\centering\scriptsize}m{1.3em}|>{\centering}m{8.8em}|}
  % \caption{秦王政}\\
  \toprule
  \SimHei \normalsize 年数 & \SimHei \scriptsize 公元 & \SimHei 大事件 \tabularnewline
  % \midrule
  \endfirsthead
  \toprule
  \SimHei \normalsize 年数 & \SimHei \scriptsize 公元 & \SimHei 大事件 \tabularnewline
  \midrule
  \endhead
  \midrule
  元年 & -529 & \tabularnewline\hline
  二年 & -528 & \tabularnewline\hline
  三年 & -527 & \tabularnewline\hline
  四年 & -526 & \tabularnewline\hline
  五年 & -525 & \tabularnewline\hline
  六年 & -524 & \tabularnewline\hline
  七年 & -523 & \tabularnewline\hline
  八年 & -522 & \tabularnewline\hline
  九年 & -521 & \tabularnewline\hline
  十年 & -520 & \tabularnewline\hline
  十一年 & -519 & \tabularnewline\hline
  十二年 & -518 & \tabularnewline\hline
  十三年 & -517 & \tabularnewline\hline
  十四年 & -516 & \tabularnewline\hline
  十五年 & -515 & \tabularnewline\hline
  十六年 & -514 & \tabularnewline
  \bottomrule
\end{longtable}

%%% Local Variables:
%%% mode: latex
%%% TeX-engine: xetex
%%% TeX-master: "../../Main"
%%% End:

%% -*- coding: utf-8 -*-
%% Time-stamp: <Chen Wang: 2018-07-16 22:24:08>

\subsection{献公{\tiny(BC513-BC501)}}

% \centering
\begin{longtable}{|>{\centering\scriptsize}m{2em}|>{\centering\scriptsize}m{1.3em}|>{\centering}m{8.8em}|}
  % \caption{秦王政}\\
  \toprule
  \SimHei \normalsize 年数 & \SimHei \scriptsize 公元 & \SimHei 大事件 \tabularnewline
  % \midrule
  \endfirsthead
  \toprule
  \SimHei \normalsize 年数 & \SimHei \scriptsize 公元 & \SimHei 大事件 \tabularnewline
  \midrule
  \endhead
  \midrule
  元年 & -513 & \tabularnewline\hline
  二年 & -512 & \tabularnewline\hline
  三年 & -511 & \tabularnewline\hline
  四年 & -510 & \tabularnewline\hline
  五年 & -509 & \tabularnewline\hline
  六年 & -508 & \tabularnewline\hline
  七年 & -507 & \tabularnewline\hline
  八年 & -506 & \tabularnewline\hline
  九年 & -505 & \tabularnewline\hline
  十年 & -504 & \tabularnewline\hline
  十一年 & -503 & \tabularnewline\hline
  十二年 & -502 & \tabularnewline\hline
  十三年 & -501 & \tabularnewline
  \bottomrule
\end{longtable}

%%% Local Variables:
%%% mode: latex
%%% TeX-engine: xetex
%%% TeX-master: "../../Main"
%%% End:

%% -*- coding: utf-8 -*-
%% Time-stamp: <Chen Wang: 2018-07-16 22:25:12>

\subsection{声公{\tiny(BC500-BC463)}}

% \centering
\begin{longtable}{|>{\centering\scriptsize}m{2em}|>{\centering\scriptsize}m{1.3em}|>{\centering}m{8.8em}|}
  % \caption{秦王政}\\
  \toprule
  \SimHei \normalsize 年数 & \SimHei \scriptsize 公元 & \SimHei 大事件 \tabularnewline
  % \midrule
  \endfirsthead
  \toprule
  \SimHei \normalsize 年数 & \SimHei \scriptsize 公元 & \SimHei 大事件 \tabularnewline
  \midrule
  \endhead
  \midrule
  元年 & -500 & \tabularnewline\hline
  二年 & -499 & \tabularnewline\hline
  三年 & -498 & \tabularnewline\hline
  四年 & -497 & \tabularnewline\hline
  五年 & -496 & \tabularnewline\hline
  六年 & -495 & \tabularnewline\hline
  七年 & -494 & \tabularnewline\hline
  八年 & -493 & \tabularnewline\hline
  九年 & -492 & \tabularnewline\hline
  十年 & -491 & \tabularnewline\hline
  十一年 & -490 & \tabularnewline\hline
  十二年 & -489 & \tabularnewline\hline
  十三年 & -488 & \tabularnewline\hline
  十四年 & -487 & \tabularnewline\hline
  十五年 & -486 & \tabularnewline\hline
  十六年 & -485 & \tabularnewline\hline
  十七年 & -484 & \tabularnewline\hline
  十八年 & -483 & \tabularnewline\hline
  十九年 & -482 & \tabularnewline\hline
  二十年 & -481 & \tabularnewline\hline
  二一年 & -480 & \tabularnewline\hline
  二二年 & -479 & \tabularnewline\hline
  二三年 & -478 & \tabularnewline\hline
  二四年 & -477 & \tabularnewline\hline
  二五年 & -476 & \tabularnewline\hline
  二六年 & -475 & \tabularnewline\hline
  二七年 & -474 & \tabularnewline\hline
  二八年 & -473 & \tabularnewline\hline
  二九年 & -472 & \tabularnewline\hline
  三十年 & -471 & \tabularnewline\hline
  三一年 & -470 & \tabularnewline\hline
  三二年 & -469 & \tabularnewline\hline
  三三年 & -468 & \tabularnewline\hline
  三四年 & -467 & \tabularnewline\hline
  三五年 & -466 & \tabularnewline\hline
  三六年 & -465 & \tabularnewline\hline
  三七年 & -464 & \tabularnewline\hline
  三八年 & -463 & \tabularnewline
  \bottomrule
\end{longtable}

%%% Local Variables:
%%% mode: latex
%%% TeX-engine: xetex
%%% TeX-master: "../../Main"
%%% End:

%% -*- coding: utf-8 -*-
%% Time-stamp: <Chen Wang: 2018-07-16 22:25:59>

\subsection{哀公{\tiny(BC462-BC455)}}

% \centering
\begin{longtable}{|>{\centering\scriptsize}m{2em}|>{\centering\scriptsize}m{1.3em}|>{\centering}m{8.8em}|}
  % \caption{秦王政}\\
  \toprule
  \SimHei \normalsize 年数 & \SimHei \scriptsize 公元 & \SimHei 大事件 \tabularnewline
  % \midrule
  \endfirsthead
  \toprule
  \SimHei \normalsize 年数 & \SimHei \scriptsize 公元 & \SimHei 大事件 \tabularnewline
  \midrule
  \endhead
  \midrule
  元年 & -462 & \tabularnewline\hline
  二年 & -461 & \tabularnewline\hline
  三年 & -460 & \tabularnewline\hline
  四年 & -459 & \tabularnewline\hline
  五年 & -458 & \tabularnewline\hline
  六年 & -457 & \tabularnewline\hline
  七年 & -456 & \tabularnewline\hline
  八年 & -455 & \tabularnewline
  \bottomrule
\end{longtable}

%%% Local Variables:
%%% mode: latex
%%% TeX-engine: xetex
%%% TeX-master: "../../Main"
%%% End:

%% -*- coding: utf-8 -*-
%% Time-stamp: <Chen Wang: 2018-07-16 22:26:47>

\subsection{共公{\tiny(BC454-BC424)}}

% \centering
\begin{longtable}{|>{\centering\scriptsize}m{2em}|>{\centering\scriptsize}m{1.3em}|>{\centering}m{8.8em}|}
  % \caption{秦王政}\\
  \toprule
  \SimHei \normalsize 年数 & \SimHei \scriptsize 公元 & \SimHei 大事件 \tabularnewline
  % \midrule
  \endfirsthead
  \toprule
  \SimHei \normalsize 年数 & \SimHei \scriptsize 公元 & \SimHei 大事件 \tabularnewline
  \midrule
  \endhead
  \midrule
  元年 & -454 & \tabularnewline\hline
  二年 & -453 & \tabularnewline\hline
  三年 & -452 & \tabularnewline\hline
  四年 & -451 & \tabularnewline\hline
  五年 & -450 & \tabularnewline\hline
  六年 & -449 & \tabularnewline\hline
  七年 & -448 & \tabularnewline\hline
  八年 & -447 & \tabularnewline\hline
  九年 & -446 & \tabularnewline\hline
  十年 & -445 & \tabularnewline\hline
  十一年 & -444 & \tabularnewline\hline
  十二年 & -443 & \tabularnewline\hline
  十三年 & -442 & \tabularnewline\hline
  十四年 & -441 & \tabularnewline\hline
  十五年 & -440 & \tabularnewline\hline
  十六年 & -439 & \tabularnewline\hline
  十七年 & -438 & \tabularnewline\hline
  十八年 & -437 & \tabularnewline\hline
  十九年 & -436 & \tabularnewline\hline
  二十年 & -435 & \tabularnewline\hline
  二一年 & -434 & \tabularnewline\hline
  二二年 & -433 & \tabularnewline\hline
  二三年 & -432 & \tabularnewline\hline
  二四年 & -431 & \tabularnewline\hline
  二五年 & -430 & \tabularnewline\hline
  二六年 & -429 & \tabularnewline\hline
  二七年 & -428 & \tabularnewline\hline
  二八年 & -427 & \tabularnewline\hline
  二九年 & -426 & \tabularnewline\hline
  三十年 & -425 & \tabularnewline\hline
  三一年 & -424 & \tabularnewline
  \bottomrule
\end{longtable}

%%% Local Variables:
%%% mode: latex
%%% TeX-engine: xetex
%%% TeX-master: "../../Main"
%%% End:

%% -*- coding: utf-8 -*-
%% Time-stamp: <Chen Wang: 2018-07-16 22:27:22>

\subsection{幽公{\tiny(BC423)}}

% \centering
\begin{longtable}{|>{\centering\scriptsize}m{2em}|>{\centering\scriptsize}m{1.3em}|>{\centering}m{8.8em}|}
  % \caption{秦王政}\\
  \toprule
  \SimHei \normalsize 年数 & \SimHei \scriptsize 公元 & \SimHei 大事件 \tabularnewline
  % \midrule
  \endfirsthead
  \toprule
  \SimHei \normalsize 年数 & \SimHei \scriptsize 公元 & \SimHei 大事件 \tabularnewline
  \midrule
  \endhead
  \midrule
  元年 & -423 & \tabularnewline
  \bottomrule
\end{longtable}

%%% Local Variables:
%%% mode: latex
%%% TeX-engine: xetex
%%% TeX-master: "../../Main"
%%% End:

%% -*- coding: utf-8 -*-
%% Time-stamp: <Chen Wang: 2018-07-16 22:28:46>

\subsection{繻公{\tiny(BC422-BC396)}}

% \centering
\begin{longtable}{|>{\centering\scriptsize}m{2em}|>{\centering\scriptsize}m{1.3em}|>{\centering}m{8.8em}|}
  % \caption{秦王政}\\
  \toprule
  \SimHei \normalsize 年数 & \SimHei \scriptsize 公元 & \SimHei 大事件 \tabularnewline
  % \midrule
  \endfirsthead
  \toprule
  \SimHei \normalsize 年数 & \SimHei \scriptsize 公元 & \SimHei 大事件 \tabularnewline
  \midrule
  \endhead
  \midrule
  元年 & -422 & \tabularnewline\hline
  二年 & -421 & \tabularnewline\hline
  三年 & -420 & \tabularnewline\hline
  四年 & -419 & \tabularnewline\hline
  五年 & -418 & \tabularnewline\hline
  六年 & -417 & \tabularnewline\hline
  七年 & -416 & \tabularnewline\hline
  八年 & -415 & \tabularnewline\hline
  九年 & -414 & \tabularnewline\hline
  十年 & -413 & \tabularnewline\hline
  十一年 & -412 & \tabularnewline\hline
  十二年 & -411 & \tabularnewline\hline
  十三年 & -410 & \tabularnewline\hline
  十四年 & -409 & \tabularnewline\hline
  十五年 & -408 & \tabularnewline\hline
  十六年 & -407 & \tabularnewline\hline
  十七年 & -406 & \tabularnewline\hline
  十八年 & -405 & \tabularnewline\hline
  十九年 & -404 & \tabularnewline\hline
  二十年 & -403 & \tabularnewline% \hline
  % 二一年 & -402 & \tabularnewline\hline
  % 二二年 & -401 & \tabularnewline\hline
  % 二三年 & -400 & \tabularnewline\hline
  % 二四年 & -399 & \tabularnewline\hline
  % 二五年 & -398 & \tabularnewline\hline
  % 二六年 & -397 & \tabularnewline\hline
  % 二七年 & -396 & \tabularnewline
  \bottomrule
\end{longtable}

%%% Local Variables:
%%% mode: latex
%%% TeX-engine: xetex
%%% TeX-master: "../../Main"
%%% End:



%%% Local Variables:
%%% mode: latex
%%% TeX-engine: xetex
%%% TeX-master: "../../Main"
%%% End:



%%% Local Variables:
%%% mode: latex
%%% TeX-engine: xetex
%%% TeX-master: "../Main"
%%% End:
 % 春秋
% %% -*- coding: utf-8 -*-
%% Time-stamp: <Chen Wang: 2021-11-02 15:51:41>

\chapter{战国{\tiny(BC402-BC221)}}

%% -*- coding: utf-8 -*-
%% Time-stamp: <Chen Wang: 2019-10-15 11:01:25>

\section{东周\tiny(BC770-BC256)}

%% -*- coding: utf-8 -*-
%% Time-stamp: <Chen Wang: 2018-07-14 15:17:20>

\subsection{威烈王\tiny(BC425-BC402)}

\textbf{周威烈王}(?-前402年),姬姓,名午,为周考王之子,中国东周第二十代国王。周考王十五年,崩,周威烈王即立。周威烈王二十三年(前403年)封晋国大夫韩虔、赵籍、魏斯为韩侯、赵侯、魏侯,这是历史上著名的“三家分晋”。三家分晋标志着春秋时代的结束,紧接着是战国时代的来临,本年也是司马光《资治通鉴》记载的起点,司马光还为三家分晋一事发表长篇的感言。二十四年(前402年),病死。葬今河南省洛阳市。其子骄继位。

% \centering
\begin{longtable}{|>{\centering\scriptsize}m{2em}|>{\centering\scriptsize}m{1.3em}|>{\centering}m{8.8em}|}
  % \caption{秦王政}\\
  \toprule
  \SimHei \normalsize 年数 & \SimHei \scriptsize 公元 & \SimHei 大事件 \tabularnewline
  % \midrule
  \endfirsthead
  \toprule
  \SimHei \normalsize 年数 & \SimHei \scriptsize 公元 & \SimHei 大事件 \tabularnewline
  \midrule
  \endhead
  \midrule
  二三年 & -403 & \begin{enumerate}
    \tiny
  \item 命晉大夫魏斯\footnote{魏文侯(?-前396年),安邑(今山西夏县)人。中国战国时魏国统治者。姬姓,魏氏,名斯。周贞定王二十四年(前445年)继魏桓子位,周威烈王二年(前424年)称侯改元,威烈王二十三年(前403年)与韩、赵两家一起被周威烈王册封为诸侯,是为三家分晋,周安王六年(前396年)卒。}、趙籍\footnote{赵烈侯(?-前400年),是中国战国时期赵国的君主,原名赵籍,赵献侯之子。在位时用公仲连、牛畜、荀欣、徐越等人,为政待以仁义,约以王道。}、韓虔\footnote{韩景侯(?-前400年),名虔,韩武子之子。}爲諸侯\footnote{晋国(首府新田【山西省侯马市】)长期以来,在魏、赵、韩三大家族控制之下,国君不过空拥虚名,只在形式上,看起来晋国仍是一个完整的独立封国。本年(前四〇三年),周王国(首都洛阳【河南省洛阳市白马寺东】)国王(三十八任威烈王)姬午,下令擢升三大家族族长,亦即晋国三位国务官(大夫):魏斯当魏国(首府安邑【山西省夏县】)国君、赵籍当赵国(首府晋阳【山西省太原市】)国君、韩虔当韩国(首府平阳【山西省临汾市】)国君。晋国被三国瓜分后,只剩下一小片国土。}。
  \item 魏文侯使乐羊\footnote{乐羊,中山国人,战国时魏国的大将。是乐毅先祖。}伐中山\footnote{中山国,姬姓,春秋战国时白狄的一支——鲜虞仿照东周各诸侯国于公元前507年建立的国家,位于今河北省中部太行山东麓一带,中山国当时位于赵国和燕国之间,都于顾,后迁都于灵寿(今中国河北省灵寿县),因城中有山得国名。},克之。
  \item 吴起\footnote{吴起(前440年-前381年),中国战国初期军事家、政治家、改革家,兵家代表人物。卫国左氏(今山东省定陶县,一说山东省曹县东北)人。}杀妻以求为鲁将,大破齐师。
  \item 燕愍公\footnote{燕国(首府蓟城【北京市】)国君(三十四任)。}薨,子僖公立。
  \end{enumerate} \tabularnewline\hline
  二四年 & -402 & \begin{enumerate}
    \tiny
  \item 周威烈王崩,子安王骄立。
  \item 盜殺楚聲王,國人立其子悼王\footnote{聲王,名當。悼王,名疑。«諡法»︰不生其國曰聲。«註»云︰生於外家。年中早夭曰悼。«註»云︰年不稱志。又云︰恐懼從處曰悼。}。
  \end{enumerate} \tabularnewline
  \bottomrule
\end{longtable}

%%% Local Variables:
%%% mode: latex
%%% TeX-engine: xetex
%%% TeX-master: "../../Main"
%%% End:

%% -*- coding: utf-8 -*-
%% Time-stamp: <Chen Wang: 2019-10-15 11:27:51>

\subsection{元安王{\tiny(BC401-BC376)}}

\textbf{周安王}姬骄(?—前376年),姬姓,名骄,华夏族,周威烈王之子,威烈王死后继位,在位26年,病死。葬处不明。在位时封齐国大夫田和为齐侯,是谓“田氏代齐”。

% \centering
\begin{longtable}{|>{\centering\scriptsize}m{2em}|>{\centering\scriptsize}m{1.3em}|>{\centering}m{8.8em}|}
  % \caption{秦王政}\\
  \toprule
  \SimHei \normalsize 年数 & \SimHei \scriptsize 公元 & \SimHei 大事件 \tabularnewline
  % \midrule
  \endfirsthead
  \toprule
  \SimHei \normalsize 年数 & \SimHei \scriptsize 公元 & \SimHei 大事件 \tabularnewline
  \midrule
  \endhead
  \midrule
  元年 & -401 & \begin{enumerate}
    \tiny
  \item 秦伐魏,至陽孤\footnote{秦国(首府雍县【陕西省凤翔县】)进攻魏国(首府安邑【山西省夏县】),大军进抵阳孤(山西省垣曲县东南)。}。
  \end{enumerate} \tabularnewline\hline
  二年 & -400 & \begin{enumerate}
    \tiny
  \item 魏、韓、趙伐楚,至桑丘\footnote{魏国(首府安邑【山西省夏县】)、韩国(首府平阳【山西省临汾市】)、赵国(首府晋阳【山西省太原市】),联合攻击楚王国(首都郢都【湖北省江陵县】),大军进抵桑丘(《史记》作乘丘【山东省兖州市西北】)。}。
  \item 鄭圍韓陽翟\footnote{郑国(首府新郑【河南省新郑县】)围攻韩国所属的阳翟(河南省禹州市)。}。
  \item 韓景侯薨,子烈侯取立。
  \item 趙烈侯薨,國人立其弟武侯。
  \item 秦簡公薨,子惠(«諡法»︰愛民好與曰惠。)公立。
  \end{enumerate} \tabularnewline\hline
  三年 & -399 & \begin{enumerate}
    \tiny
  \item 王子定奔晉。
  \item 虢山崩,壅河\footnote{虢山(河南省三门峡市西)发生崩塌,土石坠入黄河,河水壅塞。}。
  \end{enumerate} \tabularnewline\hline
  四年 & -398 & \begin{enumerate}
    \tiny
  \item 楚圍鄭。鄭人殺其相駟子陽\footnote{郑国十一任国君穆公姬兰的儿子姬腓,别名子驷。古人往往用祖父的名字最后一个字作自己这一支派的姓。这位驷子阳,姓驷,名子阳,也是郑国贵族。}。
  \end{enumerate} \tabularnewline\hline
  五年 & -397 & \begin{enumerate}
    \tiny
  \item 日有食之。
  \item 三月,盜殺韓相俠累\footnote{侠累跟濮阳(河南省濮阳市)人严仲子之间,有难解的怨毒,严仲子听说轵邑(河南省济源市东南)人聂政,勇猛过人,备了黄金二千四百两(百镒),送给聂政的母亲,作为祝寿礼物,请聂政代他报仇。聂政拒绝,说:“娘亲在堂,要我奉养,我不能轻言牺牲。”稍后,娘亲逝世,聂政才接受这项委托。当暗杀行动开始时,侠累正在宰相府主持会报,警卫森严。聂政像闪电一样,突击而入,在众人惊愕中,举刀直刺侠累的咽喉,侠累立即死亡。聂政自知难以逃生,咬紧牙关,用利刃自行毁容,脸皮全被割破,又自挖双眼,再自刺腹部自杀,肠出满地。韩国政府把尸首拖到市场,公开示众,要求市人辨识刺客身份。聂政的姐姐聂荌听到消息,赶到首府平阳(山西省临汾市),抚尸哀哭说:“他就是轵邑深井里(济通市东南十五千米)的聂政,只因为我这个姐姐尚在人间,恐怕连累我,才忍心重重的自我毁灭。弟弟啊,我怎么会贪生怕死,使你埋没英名?”就在尸旁,自杀殉难。}。
  \end{enumerate} \tabularnewline\hline
  六年 & -396 & \begin{enumerate}
    \tiny
  \item 鄭駟子陽之黨弑繻公(繻者,«諡法»所不載。),而立其弟乙,是爲康公\footnote{郑国(首府新郑【河南省新郑县】)故宰相(相)驷子阳的残余党羽,击杀国君(二十七任)繻公姬贻,拥立他的弟弟姬乙继位(二十八任),是为康公。}。
  \item 宋悼公薨,子休公田立\footnote{宋国(首府睢阳【河南省商丘县】)国君(三十一任悼公)宋购由逝世,子宋田继位(三十二任),是为休公。武王封微子啓於宋,唐宋州之睢陽縣是也。自微子二十七世至悼公,名購由。休,亦«諡法»所不載。}。
  \end{enumerate} \tabularnewline\hline
  七年 & -395 & \tiny \kaiti 无记载 \tabularnewline\hline
  八年 & -394 & \begin{enumerate}
    \tiny
  \item 齊\footnote{武王封太公於齊,唐青州之臨淄是也。«括地志»曰︰天齊水在臨淄東南十五里。«封禪書»曰︰齊之所以爲齊者,以天齊。是年,康公貸之十一年。自太公至康公二十九世。}伐魯\footnote{成王封伯禽於魯,唐兗州之曲阜是也。是年,穆公之十六年。自伯禽至穆公凡二十八世。},取最\footnote{山东省曲阜市东南}。
  \item 鄭負黍\footnote{負黍山在陽城縣西南二十七里,或云在西南三十五里。}叛,復歸韓\footnote{前四〇七年,郑国攻击韩国,占领负黍城。}。
  \end{enumerate} \tabularnewline\hline
  九年 & -393 & \begin{enumerate}
    \tiny
  \item 魏伐鄭。
  \item 晉烈公\footnote{周成王封弟叔虞於唐。«括地志»曰︰故唐城在幷州晉陽縣北二里,堯所築也。«都城記»曰︰唐叔虞之子燮父徙居晉水旁,今幷州理故唐城,卽燮父初徙之處;其城南半入州城中。«毛詩譜»曰︰燮父以堯墟南有晉水,改曰晉侯。自唐叔至烈公三十七世。烈公,名止。«諡法»︰慈惠愛親曰孝。}薨,子孝公傾立。
  \end{enumerate} \tabularnewline\hline
  十年 & -392 & \tiny \kaiti 无记载\tabularnewline\hline
  十一年 & -391 & \begin{enumerate}
    \tiny
  \item 秦伐韓宜陽,取六邑\footnote{班«志»,宜陽縣屬弘農郡。«史記正義»曰︰宜陽縣故城,在河南府福昌縣東十四里,故韓城是也。此邑卽«周禮»「四井爲邑」之邑。}。
  \item 齊田和\footnote{田常生襄子盤,盤生莊子白,白生太公和。此序齊田氏之世也。田常,卽«左傳»陳成子恆也。溫公避仁廟諱,改「恆」曰「常」。自陳公子完奔齊,五世至常得政。«諡法»︰勝敵志強曰莊。}遷齊康公於海上,使食一城,以奉其先祀。
  \end{enumerate} \tabularnewline\hline
  十二年 & -390 & \begin{enumerate}
    \tiny
  \item 秦、晉戰于武城\footnote{晋国【首府新田】自被瓜分后,连本身生存都有问题,已无力作任何战争。可能是和魏国【首府安邑·山西省夏县】,或韩国【首府平阳·山西省临汾市】会战。}。
  \item 齊伐魏,取襄陽。
  \item 魯敗齊師于平陸。
  \end{enumerate} \tabularnewline\hline
  十三年 & -389 & \begin{enumerate}
    \tiny
  \item 秦侵晉。
  \item 齊田和會\footnote{孔穎達曰︰諸侯未及期而相見曰遇。會者,謂及期之禮,旣及期,又至所期之地。}魏文侯、楚人、衞人于濁澤,求爲諸侯。魏文侯爲之請於王及諸侯,王許之。
  \end{enumerate} \tabularnewline\hline
  十四年 & -388 & \begin{enumerate}
    \tiny
  \item 齊田和逝世,子田剡继位。
  \end{enumerate} \tabularnewline\hline
  十五年 & -387 & \begin{enumerate}
    \tiny
  \item 秦伐蜀\footnote{«譜記»普[疑衍]云︰蜀之先,肇自人皇之際。黃帝子昌意娶蜀山氏女,生帝俈。旣立,封其支庶於蜀,歷虞、夏、商、周。周衰,先稱王者蠶叢。余據武王伐紂,庸、蜀諸國皆會于牧野。孔安國曰︰蜀,叟也,春秋之時不與中國通。班«志»,南鄭縣屬漢中郡,唐爲梁州治所。},取南鄭。
  \item 魏文侯薨,太子擊立,是爲武侯。魏置相,相田文\footnote{魏击任命田文担任宰相。吴起不高兴,对田文说:“我想跟你讨论一下你我对于国家的贡献,你以为如何?”田文说:“当然可以。”吴起说:“指挥武装部队,官兵们愿意牺牲性命,使敌国惊惧,不敢打我们的主意,你比我怎么样?”田文说:“我不如你。”吴起说:“使政府的功能充分发挥,使全国人民安居乐业、国库充实、社会富庶,你比我怎么样?”田文说:“我不如你。”吴起说:“防卫西河(潼关以北的黄河),秦国不敢向东侵略。而韩国(首府平阳【山西省临汾市】)与赵国(首府晋阳【山西省太原市】),不敢不对我们唯命是听,你比我怎么样?”田文说:“我不如你。”吴起说:“这三项重要大事,你都不如我,可是官位却比我高,那为什么?”田文说:“当君王年纪还小,有权势的重要官员互相猜忌,随时可能发动政变,民心恐慌。这个时候,宰相位置,应该属于你?还是属于我?”吴起沉默很久,抱歉说:“我承认,应该属于你。”}。
  \item 秦惠公薨,子出公\footnote{出,非諡也;以其失國出死,故曰出公。}立。
  \item 趙武侯薨,國人復立烈侯之太子章,是爲敬侯(«諡法»︰夙夜警戒曰敬。)。
  \item 韓烈侯薨,子文侯立。
  \end{enumerate} \tabularnewline\hline
  十六年 & -386 & \begin{enumerate}
    \tiny
  \item 趙公子朝作亂,奔魏;與魏襲邯鄲,不克\footnote{本年【前三八六年】,赵国首府自晋阳迁邯郸,赵朝当是利用迁府之际,发动政变。}。
  \end{enumerate} \tabularnewline\hline
  十七年 & -385 & \begin{enumerate}
    \tiny
  \item 秦庶長\footnote{後秦制爵,一級曰公士,二上造,三簪裊,四不更,五大夫,六官大夫,七公大夫,八公乘,九五大夫,十左庶長,十一右庶長,十二左更,十三中更,十四右更,十五少上造,十六大上造,十七駟車庶長,十八大庶長,十九關內侯,二十徹侯。師古曰︰庶長,言衆列之長。}改逆獻公\footnote{威烈王十一年秦靈公卒,子獻公師隰不得立,立靈公季父悼子,是爲簡公。出子,簡公之孫也。今庶長改迎獻公而殺出子。}于河西而立之;殺出子及其母,沈之淵旁。
  \item 齐伐魯。
  \item 韓伐鄭,取陽城;伐宋,執宋公。
  \end{enumerate} \tabularnewline\hline
  十八年 & -384 & \tiny \kaiti 无记载 \tabularnewline\hline
  十九年 & -383 & \begin{enumerate}
    \tiny
  \item 魏敗趙師于兔臺。
  \end{enumerate} \tabularnewline\hline
  二十年 & -382 & \begin{enumerate}
    \tiny
  \item 日有食之,旣\footnote{旣,盡也}。
  \end{enumerate} \tabularnewline\hline
  二一年 & -381 & \begin{enumerate}
    \tiny
  \item 楚悼王薨。貴戚大臣作亂,攻吳起;起走之王尸而伏之。擊起之徒因射刺起,並中王尸。旣葬,肅\footnote{«諡法»︰剛德克就曰肅;執心決斷曰肅。}王卽位,使令尹盡誅爲亂者;坐起夷宗者七十餘家。
  \end{enumerate} \tabularnewline\hline
  二二年 & -380 & \begin{enumerate}
    \tiny
  \item 齊伐燕,取桑丘。
  \item 魏、韓、趙伐齊,至桑丘。
  \end{enumerate} \tabularnewline\hline
  二三年 & -379 & \begin{enumerate}
    \tiny
  \item 趙襲衞\footnote{成王封康叔於衞,居河、淇之間,故殷墟也。至懿公爲狄所滅,東徙度河。文公徙居楚丘,遂國於濮陽。是年,愼公頹之三十五年。自康叔至愼公凡三十二世。},不克。
  \item 齊康公薨,無子,田氏遂幷齊而有之。姜氏至此滅矣。
  \end{enumerate} \tabularnewline\hline
  二四年 & -378 & \begin{enumerate}
    \tiny
  \item 狄\footnote{漢之中山、上黨、西河、上郡,自春秋以來,狄皆居之,此亦其種也。«水經»︰澮水出河東絳縣東澮山,西過絳縣南,又西南過虒祁宮南,又西南至王橋,入汾水。«括地志»︰澮山在絳州翼城縣東北。}敗魏師于澮。
  \item 魏、韓、趙伐齊,至靈丘。
  \item 晉孝公薨,子靖公\footnote{«諡法»︰柔衆安民曰靖;又,恭己鮮言曰靖。}俱酒立。
  \item 齐国(首府临淄)国君(二任)田剡逝世,子田午继位(三任),是为桓公。
  \end{enumerate} \tabularnewline\hline
  二五年 & -377 & \begin{enumerate}
    \tiny
  \item 蜀伐楚,取茲方(四川省奉节县)。
  \item 子思论卫\footnote{卫国(首府濮阳【河南省濮阳市】),孔伋(子思)向卫国国君(四十一任慎公)卫颓,推荐苟变,说:“他的才干可以指挥五百辆战车作战。”卫颓说:“我知道他的军事才能,但苟变曾经当过税务员,有次平白吃了民家两个鸡蛋,品德上有瑕疵。”孔伋说:“政府任用官吏,跟建筑师选择木材一样,取其所长,弃其所短。巨木高耸云际,几个人都合抱不住,却有几尺朽烂,优秀的建筑师不会不用它。现在,我们正处在大混战时代,应该积极物色英雄豪杰,却为了两个鸡蛋,丧失一员大将,这话可别让别国听见才好。”卫颓再三致谢说:“我接受你的指教。”卫颓做了一项错误的决定,全体官员却一致赞扬那决定非常正确。孔伋对公丘懿子说:“我看你们卫国,真是君不像君,臣不像臣。”(“君不君,臣不臣”,《论语》引齐国【首府临淄·山东省淄博市东临淄镇】国君【二十六任景公】姜杵臼的话。)公丘懿子说:“怎么会糟到这种程度?”孔伋说:“领袖人物经常的自以为是,大家就不敢贡献自己的意见。做对了而自以为是,还会排斥众人的智慧。何况做错了而仍自以为是,硬教大家赞扬,那简直是鼓励邪恶。不问事情的是非,而只一味喜欢听悦耳的声音,可以说绝顶糊涂。不管那是不是合理,而只努力露出忠贞嘴脸,满口顺调,那就是马屁精。君主昏庸、官员谄媚,而高高坐在人民头上,人民绝对不会认同。如果一直这样下去,国家必亡。”孔伋告诉卫颓说:“你的国家,恐怕将要没落了。”卫颓说:“什么原因?”孔伋说:“当然有原因,领袖说一句话,自以为是,官员们没有一个人敢指出他的错误;官员们说一句话,自以为是,民间没有一个人敢指出他的错误。领袖和官员,都自以为英明盖世,属下的小官小民也同声赞扬他们果然是真的英明盖世。马屁精就有福了,指出君王错误的人一定大祸临久。如此这般,有益于国家的善政,怎能产生?《诗经》说:‘都说自己是圣贤,谁分辨乌鸦的雌雄?’听起来好像就是指的你们。”}。
  \item 魯穆公薨,子共公奮立\footnote{«諡法»︰布德就義曰穆;中情見貌曰穆;尊賢敬讓曰共;旣過能改曰共;執事堅固曰共。}。
  \item 韓文侯薨,子哀侯立。
  \end{enumerate} \tabularnewline\hline
  二六年 & -376 & \begin{enumerate}
    \tiny
  \item 王崩,子烈王喜立。
  \item 魏、韓、趙共廢晉靖公爲家人而分其地。唐叔不祀矣。
  \end{enumerate} \tabularnewline
  \bottomrule
\end{longtable}

%%% Local Variables:
%%% mode: latex
%%% TeX-engine: xetex
%%% TeX-master: "../../Main"
%%% End:

%% -*- coding: utf-8 -*-
%% Time-stamp: <Chen Wang: 2018-07-16 22:54:57>

\subsection{烈王{\tiny(BC375-BC369)}}

周烈王(?-前369年),又称周夷烈王,姓姬,名喜,中国东周君主,在位7年。他是周安王之子。周烈王在位期间,秦献公迁都栎阳(今陕西省临潼市),开启秦国强盛的序幕。周烈王五年(庚戌,前371年),秦献公发兵攻占韩国六座城市。烈王六年(前370年)齐威王朝见周天子,威王贤名更盛。

% \centering
\begin{longtable}{|>{\centering\scriptsize}m{2em}|>{\centering\scriptsize}m{1.3em}|>{\centering}m{8.8em}|}
  % \caption{秦王政}\\
  \toprule
  \SimHei \normalsize 年数 & \SimHei \scriptsize 公元 & \SimHei 大事件 \tabularnewline
  % \midrule
  \endfirsthead
  \toprule
  \SimHei \normalsize 年数 & \SimHei \scriptsize 公元 & \SimHei 大事件 \tabularnewline
  \midrule
  \endhead
  \midrule
  元年 & -375 & \begin{enumerate}
    \tiny
  \item 日有食之。
  \item 韓滅鄭,因徙都之\footnote{韓本都平陽,其地屬漢之河東郡;中間徙都陽翟。鄭都新鄭,其地屬漢之河南郡。鄭桓公始封於鄭,其地屬漢之京兆;後滅虢、鄶而國於溱、洧之間,故曰新鄭,«左傳»鄭莊公所謂「吾先君新邑於此」是也。今韓旣滅鄭,自陽翟徙都之。韓旣都鄭,故時人亦謂韓王爲鄭王,考之«戰國策»、«韓非子»可見。}。
  \item 趙敬侯薨,子成侯種立。
  \end{enumerate} \tabularnewline\hline
  二年 & -374 & \tiny \kaiti 无记载 \tabularnewline\hline
  三年 & -373 & \begin{enumerate}
    \tiny
  \item 燕敗齊師於林狐。
  \item 魯伐齊,入陽關。
  \item 魏伐齊,至博陵。
  \item 燕僖公薨,子桓公立。
  \item 宋休公薨,子辟公立。
  \item 衞愼公\footnote{«諡法»︰敏以敬曰愼。«戴記»︰思慮深遠曰愼。}薨,子聲公訓立。
  \end{enumerate} \tabularnewline\hline
  四年 & -372 & \begin{enumerate}
    \tiny
  \item 趙伐衞,取都鄙\footnote{«周禮»︰太宰以八則治都鄙。«註»云︰都之所居曰鄙。都鄙,卿大夫之采邑。蓋周之制,四縣爲都,方四十里,一千六百井,積一萬四千四百夫;五酇爲鄙,鄙五百家也。此時衞國褊小,若都鄙七十三,以成周之制率之,其地廣矣,盡衞之提封,未必能及此數也。更俟博考。}七十三。
  \item 魏敗趙師于北藺。
  \end{enumerate} \tabularnewline\hline
  五年 & -371 & \begin{enumerate}
    \tiny
  \item 魏伐楚,取魯陽。
  \item 韓嚴遂弑哀侯,國人立其子懿侯\footnote{哀侯任命韩廆当宰相,但对严遂却更亲信。韩廆跟严遂之间,结仇至深,已不可解,互相想置对方于死地。严遂雇请杀手行刺韩廆。韩廆急奔哀侯身旁,哀侯为了保护他,把他抱住。然而杀手并不停止,仍刺杀韩廆;刀锋所及,哀侯也中刃而亡。(《战国策》认为聂政杀侠累和严遂杀哀侯是一件事,《史记》认为是两件事,《资治通鉴》根据《史记》。然而,二十六年间,韩国政府发生两次重大凶案,一次杀宰相,一次除了杀宰相外,还顺手杀了国君,太过突出。所以司马光对此并不敢十分肯定,在给刘道原信中,也曾表示他的怀疑。)}。
  \item 魏武侯薨,不立太子,子罃與公中緩爭立,國內亂。
  \end{enumerate} \tabularnewline\hline
  六年 & -370 & \begin{enumerate}
    \tiny
  \item 齊威王來朝。是時周室微弱,諸侯莫朝,而齊獨朝之,天下以此益賢威王。
  \item 趙伐齊,至鄄。
  \item 魏敗趙師于懷。
  \item 齊威王奖卽墨大夫,惩阿大夫,羣臣聳懼,莫敢飾詐,務盡其情,齊國大治,強於天下\footnote{齐国国君(四任)田因齐把即墨(山东省平度市东南)城主(大夫)召到首府临淄(山东省淄博市东临淄镇),对他说:“自从命你前去即墨,我每天都接到控告你的报告。然而我派人去即墨秘密调查,发现你开荒辟田,农作物遍野,人民生活富庶,官员清廉,齐国东部,得到平安。你之所以口碑不好,我了解,是你没有巴结我左右那些当权派而已。”于是,增加他一万户人家的封邑,作为奖励。又把阿邑(山东省东阿县)城主(大夫)召到首府临淄,对他说:“自从命你前去阿邑,我几乎每天都听到对你的赞扬。可是,我派人去阿邑秘密调查,发现完全不是那么回事,那里田野荒芜,农民贫困。前些时,赵国攻击鄄城(山东省鄄城县),你不率军救援。卫国占领薛陵(山东省阳谷县东北,薛陵跟阿邑之间航空距离不到十千米),你却假装不知道。我了解,我所听到的那些捧你场的话,都是你拿钱买来的。”于是下令把阿邑城主以及平常赞扬阿邑城主的一批官员,全都用大锅烹杀。全国大为震动,官员悚然戒惧,不敢再弄玄虚,大家改变态度,认真做事。齐国大治,成为强国。}。
  \item 楚肅王薨,無子,立其弟良夫,是爲宣王。
  \item 宋辟公薨,子剔成立。
  \end{enumerate} \tabularnewline\hline \newpage
  七年 & -369 & \begin{enumerate}
    \tiny
  \item 日有食之。
  \item 王崩,弟扁(音篇)立,是爲顯王。
  \item 魏大夫王錯出奔韓,韩懿侯乃與趙成侯合兵伐魏\footnote{魏国(首府安邑【山西省夏县】)内乱(参考前三七一年),已历时三年,国务官(大夫)王错,投奔韩国(首府新郑【河南省新郑县】)。韩国国务官(大夫)公孙颀,向国君(五任懿侯)韩若山建议说:“魏国已经腐烂,亡在眉睫,我们应该把它吞并。”韩若山遂跟赵国(首府邯郸【河北省邯郸市】)国君(四任成侯)赵种结盟,联合攻击魏国,在浊泽(山西省永济县西,与安邑航空距离五十千米)会战,魏军大败,韩、赵联军遂包围魏国首府安邑(山西省夏县)。赵种主张:“杀掉魏罃,立公中缓当魏国国君,割一部分士地给我们,我们就退兵。”韩若山说:“杀掉魏罃,我们落得一个残暴的名声。割让土地,又落得一个贪心的名声。不如把魏国一分为二,分成两个国家,使他们二人都当国君。魏国一分为二之后,就跟宋国、卫国一样,成了一个小国,我们就可永远摆脱魏国的压力。”赵种不同意,韩若山大不高兴,在夜晚撤军而去。赵种人单势孤,也只好撤军而去。魏罃遂乘机袭杀他的对头公中缓,继任国君(三任)。}。
  \end{enumerate} \tabularnewline
  \bottomrule
\end{longtable}

%%% Local Variables:
%%% mode: latex
%%% TeX-engine: xetex
%%% TeX-master: "../../Main"
%%% End:

%% -*- coding: utf-8 -*-
%% Time-stamp: <Chen Wang: 2018-07-17 16:47:57>

\subsection{显王{\tiny(BC368-BC321)}}

周显王(?-前321年),又称周显圣王或周显声王,姓姬,名扁,中国东周君主,在位48年,为周烈王之弟。

周显王五年(前364年)发生河西之战,秦献公亲率主力进至河东,秦将章蟜在石门山(今山西省运城市西南)大败魏军,斩首6万。由于赵国出兵救援魏国,秦才退兵。此战为秦国对魏国的首次重大胜利,诸侯震动,周显王亦祝贺“献公称伯”,并颁赏他绣着黼黻图案的服饰。

周显王十三年(前356年),秦国商鞅变法。周显王十六年(前353年)发生桂陵之战,周显王二十八年(前341年)发生马陵之战。

% \centering
\begin{longtable}{|>{\centering\scriptsize}m{2em}|>{\centering\scriptsize}m{1.3em}|>{\centering}m{8.8em}|}
  % \caption{秦王政}\\
  \toprule
  \SimHei \normalsize 年数 & \SimHei \scriptsize 公元 & \SimHei 大事件 \tabularnewline
  % \midrule
  \endfirsthead
  \toprule
  \SimHei \normalsize 年数 & \SimHei \scriptsize 公元 & \SimHei 大事件 \tabularnewline
  \midrule
  \endhead
  \midrule
  元年 & -368 & \begin{enumerate}
    \tiny
  \item 齊伐魏,取觀津。
  \item 趙侵齊,取長城。
  \end{enumerate} \tabularnewline\hline
  二年 & -367 & \tiny \kaiti 无记载 \tabularnewline\hline
  三年 & -366 & \begin{enumerate}
    \tiny
  \item 魏、韓會于宅陽。
  \item 秦敗魏師、韓師于洛陽。
  \end{enumerate} \tabularnewline\hline
  四年 & -365 & \begin{enumerate}
    \tiny
  \item 魏伐宋。
  \end{enumerate} \tabularnewline\hline
  五年 & -364 & \begin{enumerate}
    \tiny
  \item 秦獻公敗三晉之師于石門,斬首六萬。王賜以黼黻\footnote{黼者,刺繡爲斧形;黻者,刺繡爲兩「己」相背。孔穎達曰:白與黑謂之黼,黑與青謂之黻。}之服。
  \end{enumerate} \tabularnewline\hline
  六年 & -363 & \tiny \kaiti 无记载 \tabularnewline\hline
  七年 & -362 & \begin{enumerate}
    \tiny
  \item 魏敗韓師、趙師于澮。
  \item 秦、魏戰于少梁,魏師敗績;獲魏公孫痤。
  \item 衞聲公薨,子成侯速立。
  \item 燕桓公薨,子文公立。
  \item 秦獻公薨,子孝公立\footnote{这个时候,黄河和华山(陕西省华阴市南)以东,有六个强国(齐国【首府临淄·山东省淄博市东临淄镇】、韩国【首府新郑】、赵国【首府邯郸】、魏国【首府安邑】、燕国【首府蓟城】、楚王国【首都郢都·湖北省江陵县】)。淮河、泗水之间的小封国还有十余国。楚王国、魏国都跟秦国接壤。魏国为了防御秦国,从郑县(陕西省华县)沿着洛河(纵贯陕西省中部,在陕西省大荔县东南注入渭河),直到上郡(陕西省延安市),修筑长城。楚王国自汉中(陕西省汉中市),经巴城(重庆市),南到黔中(湖南省沅陵县),分别拥有广大领土,都把秦国看作落后地区的蛮族部落,中国境内(中原)各种国际会议,一向拒绝秦国参加。这种歧视使嬴渠梁深感羞辱,决心整顿内政,提高文化水准,追求强大。}。
  \end{enumerate} \tabularnewline\hline
  八年 & -361 & \begin{enumerate}
    \tiny
  \item 衞公孫鞅西入秦\footnote{秦国(首府栎阳【陕西省临潼县】)国君(二十五任孝公)嬴渠梁颁布招贤令:“从前我们的国君穆公(九任嬴任好),在岐山(陕西省岐山县东北)、雍县(陕西省凤翔县),励精图治。东方与晋国以黄河为界,协助他们,削平内乱。西方称霸夷狄,地广千里,天子封为盟主,封国国君们都来祝贺,开辟后世万年基业。不幸出现一连串不肖的国君,如厉公(十七任嬴刺)、躁公(十八任,名不详)、简公(二十一任嬴悼子)、出公(二十三任,名不详),国家动乱,无力顾及外事。于是,晋国占领我祖先的河西领土(陕西省合阳县、大荔县一带,魏长城至黄河之间),使我们丢丑。我父亲献公(二十四任嬴师隰)即位,把首府迁到栎阳(陕西省临潼县),准备东征,收复失地,复兴当年声势。可惜壮志未遂,即与世长辞,每一思及,万分痛心。现在我们公开征聘贤才,无论是本国人民,或外国宾客,只要有谋略可以使秦国强大,我愿任命他当高官,分封采邑。”卫国(首府濮阳【河南省濮阳市】)贵族公孙鞅,听到消息,西行投奔。公孙鞅既到秦国,通过宠臣景监的推荐,晋见嬴渠梁,提出富国强兵的具体方案,嬴渠梁喜出望外,要求公孙鞅负责执行。}。
  \end{enumerate} \tabularnewline\hline
  九年 & -360 & \tiny \kaiti 无记载 \tabularnewline\hline
  十年 & -359 & \begin{enumerate}
    \tiny
  \item 衞鞅于秦變法。
  \end{enumerate} \tabularnewline\hline
  十一年 & -358 & \tabularnewline\hline
  十二年 & -357 & \tabularnewline\hline
  十三年 & -356 & \tabularnewline\hline
  十四年 & -355 & \tabularnewline\hline
  十五年 & -354 & \tabularnewline\hline
  十六年 & -353 & \tabularnewline\hline
  十七年 & -352 & \tabularnewline\hline
  十八年 & -351 & \tabularnewline\hline
  十九年 & -350 & \tabularnewline\hline
  二十年 & -349 & \tabularnewline\hline
  二一年 & -348 & \tabularnewline\hline
  二二年 & -347 & \tabularnewline\hline
  二三年 & -346 & \tabularnewline\hline
  二四年 & -345 & \tabularnewline\hline
  二五年 & -344 & \tabularnewline\hline
  二六年 & -343 & \tabularnewline\hline
  二七年 & -342 & \tabularnewline\hline
  二八年 & -341 & \tabularnewline\hline
  二九年 & -340 & \tabularnewline\hline
  三十年 & -339 & \tabularnewline\hline
  三一年 & -338 & \tabularnewline\hline
  三二年 & -337 & \tabularnewline\hline
  三三年 & -336 & \tabularnewline\hline
  三四年 & -335 & \tabularnewline\hline
  三五年 & -334 & \tabularnewline\hline
  三六年 & -333 & \tabularnewline\hline
  三七年 & -332 & \tabularnewline\hline
  三八年 & -331 & \tabularnewline\hline
  三九年 & -330 & \tabularnewline\hline
  四十年 & -329 & \tabularnewline\hline
  四一年 & -328 & \tabularnewline\hline
  四二年 & -327 & \tabularnewline\hline
  四三年 & -326 & \tabularnewline\hline
  四四年 & -325 & \tabularnewline\hline
  四五年 & -324 & \tabularnewline\hline
  四六年 & -323 & \tabularnewline\hline
  四七年 & -322 & \tabularnewline\hline
  四八年 & -321 & \tabularnewline
  \bottomrule
\end{longtable}

%%% Local Variables:
%%% mode: latex
%%% TeX-engine: xetex
%%% TeX-master: "../../Main"
%%% End:

%% -*- coding: utf-8 -*-
%% Time-stamp: <Chen Wang: 2018-07-10 17:30:19>

\subsection{慎靓王{\tiny(BC320-BC315)}}


% \centering
\begin{longtable}{|>{\centering\scriptsize}m{2em}|>{\centering\scriptsize}m{1.3em}|>{\centering}m{8.8em}|}
  % \caption{秦王政}\\
  \toprule
  \SimHei \normalsize 年数 & \SimHei \scriptsize 公元 & \SimHei 大事件 \tabularnewline
  % \midrule
  \endfirsthead
  \toprule
  \SimHei \normalsize 年数 & \SimHei \scriptsize 公元 & \SimHei 大事件 \tabularnewline
  \midrule
  \endhead
  \midrule
  元年 & -320 & \tabularnewline\hline
  二年 & -319 & \tabularnewline\hline
  三年 & -318 & \tabularnewline\hline
  四年 & -317 & \tabularnewline\hline
  五年 & -316 & \tabularnewline\hline
  六年 & -315 & \tabularnewline
  \bottomrule
\end{longtable}

%%% Local Variables:
%%% mode: latex
%%% TeX-engine: xetex
%%% TeX-master: "../../Main"
%%% End:

%% -*- coding: utf-8 -*-
%% Time-stamp: <Chen Wang: 2018-07-10 17:30:15>

\subsection{赧王{\tiny(BC314-BC256)}}


% \centering
\begin{longtable}{|>{\centering\scriptsize}m{2em}|>{\centering\scriptsize}m{1.3em}|>{\centering}m{8.8em}|}
  % \caption{秦王政}\\
  \toprule
  \SimHei \normalsize 年数 & \SimHei \scriptsize 公元 & \SimHei 大事件 \tabularnewline
  % \midrule
  \endfirsthead
  \toprule
  \SimHei \normalsize 年数 & \SimHei \scriptsize 公元 & \SimHei 大事件 \tabularnewline
  \midrule
  \endhead
  \midrule
  元年 & -314 & \tabularnewline\hline
  二年 & -313 & \tabularnewline\hline
  三年 & -312 & \tabularnewline\hline
  四年 & -311 & \tabularnewline\hline
  五年 & -310 & \tabularnewline\hline
  六年 & -309 & \tabularnewline\hline
  七年 & -308 & \tabularnewline\hline
  八年 & -307 & \tabularnewline\hline
  九年 & -306 & \tabularnewline\hline
  十年 & -305 & \tabularnewline\hline
  十一年 & -304 & \tabularnewline\hline
  十二年 & -303 & \tabularnewline\hline
  十三年 & -302 & \tabularnewline\hline
  十四年 & -301 & \tabularnewline\hline
  十五年 & -300 & \tabularnewline\hline
  十六年 & -299 & \tabularnewline\hline
  十七年 & -298 & \tabularnewline\hline
  十八年 & -297 & \tabularnewline\hline
  十九年 & -296 & \tabularnewline\hline
  二十年 & -295 & \tabularnewline\hline
  二一年 & -294 & \tabularnewline\hline
  二二年 & -293 & \tabularnewline\hline
  二三年 & -292 & \tabularnewline\hline
  二四年 & -291 & \tabularnewline\hline
  二五年 & -290 & \tabularnewline\hline
  二六年 & -289 & \tabularnewline\hline
  二七年 & -288 & \tabularnewline\hline
  二八年 & -287 & \tabularnewline\hline
  二九年 & -286 & \tabularnewline\hline
  三十年 & -285 & \tabularnewline\hline
  三一年 & -284 & \tabularnewline\hline
  三二年 & -283 & \tabularnewline\hline
  三三年 & -282 & \tabularnewline\hline
  三四年 & -281 & \tabularnewline\hline
  三五年 & -280 & \tabularnewline\hline
  三六年 & -279 & \tabularnewline\hline
  三七年 & -278 & \tabularnewline\hline
  三八年 & -277 & \tabularnewline\hline
  三九年 & -276 & \tabularnewline\hline
  四十年 & -275 & \tabularnewline\hline
  四一年 & -274 & \tabularnewline\hline
  四二年 & -273 & \tabularnewline\hline
  四三年 & -272 & \tabularnewline\hline
  四四年 & -271 & \tabularnewline\hline
  四五年 & -270 & \tabularnewline\hline
  四六年 & -269 & \tabularnewline\hline
  四七年 & -268 & \tabularnewline\hline
  四八年 & -267 & \tabularnewline\hline
  四九年 & -266 & \tabularnewline\hline
  五十年 & -265 & \tabularnewline\hline
  五一年 & -264 & \tabularnewline\hline
  五二年 & -263 & \tabularnewline\hline
  五三年 & -262 & \tabularnewline\hline
  五四年 & -261 & \tabularnewline\hline
  五五年 & -260 & \tabularnewline\hline
  五六年 & -259 & \tabularnewline\hline
  五七年 & -258 & \tabularnewline\hline
  五八年 & -257 & \tabularnewline\hline
  五九年 & -256 & \tabularnewline
  \bottomrule
\end{longtable}

%%% Local Variables:
%%% mode: latex
%%% TeX-engine: xetex
%%% TeX-master: "../../Main"
%%% End:


%%% Local Variables:
%%% mode: latex
%%% TeX-engine: xetex
%%% TeX-master: "../../Main"
%%% End:

%% -*- coding: utf-8 -*-
%% Time-stamp: <Chen Wang: 2019-10-15 11:02:04>

\section{郑\tiny(BC806-BC375)}

%% -*- coding: utf-8 -*-
%% Time-stamp: <Chen Wang: 2018-07-16 22:29:38>

\subsection{繻公{\tiny(BC422-BC396)}}

% \centering
\begin{longtable}{|>{\centering\scriptsize}m{2em}|>{\centering\scriptsize}m{1.3em}|>{\centering}m{8.8em}|}
  % \caption{秦王政}\\
  \toprule
  \SimHei \normalsize 年数 & \SimHei \scriptsize 公元 & \SimHei 大事件 \tabularnewline
  % \midrule
  \endfirsthead
  \toprule
  \SimHei \normalsize 年数 & \SimHei \scriptsize 公元 & \SimHei 大事件 \tabularnewline
  \midrule
  \endhead
  \midrule
  % 元年 & -422 & \tabularnewline\hline
  % 二年 & -421 & \tabularnewline\hline
  % 三年 & -420 & \tabularnewline\hline
  % 四年 & -419 & \tabularnewline\hline
  % 五年 & -418 & \tabularnewline\hline
  % 六年 & -417 & \tabularnewline\hline
  % 七年 & -416 & \tabularnewline\hline
  % 八年 & -415 & \tabularnewline\hline
  % 九年 & -414 & \tabularnewline\hline
  % 十年 & -413 & \tabularnewline\hline
  % 十一年 & -412 & \tabularnewline\hline
  % 十二年 & -411 & \tabularnewline\hline
  % 十三年 & -410 & \tabularnewline\hline
  % 十四年 & -409 & \tabularnewline\hline
  % 十五年 & -408 & \tabularnewline\hline
  % 十六年 & -407 & \tabularnewline\hline
  % 十七年 & -406 & \tabularnewline\hline
  % 十八年 & -405 & \tabularnewline\hline
  % 十九年 & -404 & \tabularnewline\hline
  % 二十年 & -403 & \tabularnewline% \hline
  二一年 & -402 & \tabularnewline\hline
  二二年 & -401 & \tabularnewline\hline
  二三年 & -400 & \tabularnewline\hline
  二四年 & -399 & \tabularnewline\hline
  二五年 & -398 & \tabularnewline\hline
  二六年 & -397 & \tabularnewline\hline
  二七年 & -396 & \tabularnewline
  \bottomrule
\end{longtable}

%%% Local Variables:
%%% mode: latex
%%% TeX-engine: xetex
%%% TeX-master: "../../Main"
%%% End:

%% -*- coding: utf-8 -*-
%% Time-stamp: <Chen Wang: 2018-07-14 16:51:17>

\subsection{康公{\tiny(BC395-BC375)}}

% \centering
\begin{longtable}{|>{\centering\scriptsize}m{2em}|>{\centering\scriptsize}m{1.3em}|>{\centering}m{8.8em}|}
  % \caption{秦王政}\\
  \toprule
  \SimHei \normalsize 年数 & \SimHei \scriptsize 公元 & \SimHei 大事件 \tabularnewline
  % \midrule
  \endfirsthead
  \toprule
  \SimHei \normalsize 年数 & \SimHei \scriptsize 公元 & \SimHei 大事件 \tabularnewline
  \midrule
  \endhead
  \midrule
  元年 & -395 & \tabularnewline\hline
  二年 & -394 & \tabularnewline\hline
  三年 & -393 & \tabularnewline\hline
  四年 & -392 & \tabularnewline\hline
  五年 & -391 & \tabularnewline\hline
  六年 & -390 & \tabularnewline\hline
  七年 & -389 & \tabularnewline\hline
  八年 & -388 & \tabularnewline\hline
  九年 & -387 & \tabularnewline\hline
  十年 & -386 & \tabularnewline\hline
  十一年 & -385 & \tabularnewline\hline
  十二年 & -384 & \tabularnewline\hline
  十三年 & -383 & \tabularnewline\hline
  十四年 & -382 & \tabularnewline\hline
  十五年 & -381 & \tabularnewline\hline
  十六年 & -380 & \tabularnewline\hline
  十七年 & -379 & \tabularnewline\hline
  十八年 & -378 & \tabularnewline\hline
  十九年 & -377 & \tabularnewline\hline
  二十年 & -376 & \tabularnewline\hline
  二一年 & -375 & \tabularnewline
  \bottomrule
\end{longtable}

%%% Local Variables:
%%% mode: latex
%%% TeX-engine: xetex
%%% TeX-master: "../../Main"
%%% End:


%%% Local Variables:
%%% mode: latex
%%% TeX-engine: xetex
%%% TeX-master: "../../Main"
%%% End:

%% -*- coding: utf-8 -*-
%% Time-stamp: <Chen Wang: 2019-10-15 11:01:57>

\section{宋\tiny(?-BC286)}

%% -*- coding: utf-8 -*-
%% Time-stamp: <Chen Wang: 2018-07-14 16:56:17>

\subsection{悼公{\tiny(BC403-BC396)}}

% \centering
\begin{longtable}{|>{\centering\scriptsize}m{2em}|>{\centering\scriptsize}m{1.3em}|>{\centering}m{8.8em}|}
  % \caption{秦王政}\\
  \toprule
  \SimHei \normalsize 年数 & \SimHei \scriptsize 公元 & \SimHei 大事件 \tabularnewline
  % \midrule
  \endfirsthead
  \toprule
  \SimHei \normalsize 年数 & \SimHei \scriptsize 公元 & \SimHei 大事件 \tabularnewline
  \midrule
  \endhead
  \midrule
  元年 & -403 & \tabularnewline\hline
  二年 & -402 & \tabularnewline\hline
  三年 & -401 & \tabularnewline\hline
  四年 & -400 & \tabularnewline\hline
  五年 & -399 & \tabularnewline\hline
  六年 & -398 & \tabularnewline\hline
  七年 & -397 & \tabularnewline\hline
  八年 & -396 & \tabularnewline
  \bottomrule
\end{longtable}

%%% Local Variables:
%%% mode: latex
%%% TeX-engine: xetex
%%% TeX-master: "../../Main"
%%% End:


%%% Local Variables:
%%% mode: latex
%%% TeX-engine: xetex
%%% TeX-master: "../../Main"
%%% End:

%% -*- coding: utf-8 -*-
%% Time-stamp: <Chen Wang: 2019-10-15 11:01:50>

\section{秦}

%% -*- coding: utf-8 -*-
%% Time-stamp: <Chen Wang: 2018-07-10 17:30:49>

\subsection{昭襄王{\tiny(BC306-BC251)}}


% \centering
\begin{longtable}{|>{\centering\scriptsize}m{2em}|>{\centering\scriptsize}m{1.3em}|>{\centering}m{8.8em}|}
  % \caption{秦王政}\\
  \toprule
  \SimHei \normalsize 年数 & \SimHei \scriptsize 公元 & \SimHei 大事件 \tabularnewline
  % \midrule
  \endfirsthead
  \toprule
  \SimHei \normalsize 年数 & \SimHei \scriptsize 公元 & \SimHei 大事件 \tabularnewline
  \midrule
  \endhead
  \midrule
  元年 & -306 & \tabularnewline\hline
  二年 & -305 & \tabularnewline\hline
  三年 & -304 & \tabularnewline\hline
  四年 & -303 & \tabularnewline\hline
  五年 & -302 & \tabularnewline\hline
  六年 & -301 & \tabularnewline\hline
  七年 & -300 & \tabularnewline\hline
  八年 & -299 & \tabularnewline\hline
  九年 & -298 & \tabularnewline\hline
  十年 & -297 & \tabularnewline\hline
  十一年 & -296 & \tabularnewline\hline
  十二年 & -295 & \tabularnewline\hline
  十三年 & -294 & \tabularnewline\hline
  十四年 & -293 & \tabularnewline\hline
  十五年 & -292 & \tabularnewline\hline
  十六年 & -291 & \tabularnewline\hline
  十七年 & -290 & \tabularnewline\hline
  十八年 & -289 & \tabularnewline\hline
  十九年 & -288 & \tabularnewline\hline
  二十年 & -287 & \tabularnewline\hline
  二一年 & -286 & \tabularnewline\hline
  二二年 & -285 & \tabularnewline\hline
  二三年 & -284 & \tabularnewline\hline
  二四年 & -283 & \tabularnewline\hline
  二五年 & -282 & \tabularnewline\hline
  二六年 & -281 & \tabularnewline\hline
  二七年 & -280 & \tabularnewline\hline
  二八年 & -279 & \tabularnewline\hline
  二九年 & -278 & \tabularnewline\hline
  三十年 & -277 & \tabularnewline\hline
  三一年 & -276 & \tabularnewline\hline
  三二年 & -275 & \tabularnewline\hline
  三三年 & -274 & \tabularnewline\hline
  三四年 & -273 & \tabularnewline\hline
  三五年 & -272 & \tabularnewline\hline
  三六年 & -271 & \tabularnewline\hline
  三七年 & -270 & \tabularnewline\hline
  三八年 & -269 & \tabularnewline\hline
  三九年 & -268 & \tabularnewline\hline
  四十年 & -267 & \tabularnewline\hline
  四一年 & -266 & \tabularnewline\hline
  四二年 & -265 & \tabularnewline\hline
  四三年 & -264 & \tabularnewline\hline
  四四年 & -263 & \tabularnewline\hline
  四五年 & -262 & \tabularnewline\hline
  四六年 & -261 & \tabularnewline\hline
  四七年 & -260 & \tabularnewline\hline
  四八年 & -259 & \tabularnewline\hline
  四九年 & -258 & \tabularnewline\hline
  五十年 & -257 & \tabularnewline\hline
  五一年 & -256 & \tabularnewline\hline
  五二年 & -255 & \tabularnewline\hline
  五三年 & -254 & \tabularnewline\hline
  五四年 & -253 & \tabularnewline\hline
  五五年 & -252 & \tabularnewline\hline
  五六年 & -251 & \tabularnewline
  \bottomrule
\end{longtable}

%%% Local Variables:
%%% mode: latex
%%% TeX-engine: xetex
%%% TeX-master: "../../Main"
%%% End:

%% -*- coding: utf-8 -*-
%% Time-stamp: <Chen Wang: 2018-07-10 17:30:44>

\subsection{孝文王{\tiny(BC250-BC250)}}


% \centering
\begin{longtable}{|>{\centering\scriptsize}m{2em}|>{\centering\scriptsize}m{1.3em}|>{\centering}m{8.8em}|}
  % \caption{秦王政}\\
  \toprule
  \SimHei \normalsize 年数 & \SimHei \scriptsize 公元 & \SimHei 大事件 \tabularnewline
  % \midrule
  \endfirsthead
  \toprule
  \SimHei \normalsize 年数 & \SimHei \scriptsize 公元 & \SimHei 大事件 \tabularnewline
  \midrule
  \endhead
  \midrule
  元年 & -250 & \tabularnewline
  \bottomrule
\end{longtable}

%%% Local Variables:
%%% mode: latex
%%% TeX-engine: xetex
%%% TeX-master: "../../Main"
%%% End:

%% -*- coding: utf-8 -*-
%% Time-stamp: <Chen Wang: 2018-07-10 17:30:54>

\subsection{庄襄王{\tiny(BC249-BC247)}}


% \centering
\begin{longtable}{|>{\centering\scriptsize}m{2em}|>{\centering\scriptsize}m{1.3em}|>{\centering}m{8.8em}|}
  % \caption{秦王政}\\
  \toprule
  \SimHei \normalsize 年数 & \SimHei \scriptsize 公元 & \SimHei 大事件 \tabularnewline
  % \midrule
  \endfirsthead
  \toprule
  \SimHei \normalsize 年数 & \SimHei \scriptsize 公元 & \SimHei 大事件 \tabularnewline
  \midrule
  \endhead
  \midrule
  元年 & -249 & \tabularnewline\hline
  二年 & -248 & \tabularnewline\hline
  三年 & -247 & \tabularnewline
  \bottomrule
\end{longtable}

%%% Local Variables:
%%% mode: latex
%%% TeX-engine: xetex
%%% TeX-master: "../../Main"
%%% End:

%% -*- coding: utf-8 -*-
%% Time-stamp: <Chen Wang: 2018-07-10 17:31:00>

\subsection{赢政{\tiny(BC246-BC221)}}


% \centering
\begin{longtable}{|>{\centering\scriptsize}m{2em}|>{\centering\scriptsize}m{1.3em}|>{\centering}m{8.8em}|}
  % \caption{秦王政}\\
  \toprule
  \SimHei \normalsize 年数 & \SimHei \scriptsize 公元 & \SimHei 大事件 \tabularnewline
  % \midrule
  \endfirsthead
  \toprule
  \SimHei \normalsize 年数 & \SimHei \scriptsize 公元 & \SimHei 大事件 \tabularnewline
  \midrule
  \endhead
  \midrule
  元年 & -246 & \begin{enumerate}
    \tiny
  \item 韩国水工郑国开始建造郑国渠,约十年后完工。
  \item 秦晋阳反,蒙骜击平之。
  \end{enumerate} \tabularnewline\hline
  二年 & -245 & \begin{enumerate}
    \tiny
  \item 秦麃公将卒攻卷,斩首三万。
  \item 赵以廉颇为假相国,伐魏,取繁阳。赵孝成王薨,子赵悼襄王偃立。
  \end{enumerate} \tabularnewline\hline
  三年 & -244 & \begin{enumerate}
    \tiny
  \item 秦蒙骜攻韩,取12城。
  \end{enumerate} \tabularnewline\hline
  四年 & -243 & \begin{enumerate}
    \tiny
  \item 春,秦蒙骜伐魏,取旸、有诡。三月,军罢。
  \item 秦质子归自赵;赵太子出归国。
  \item 七月,秦国蝗,疫。令百姓纳粟千石,拜爵一级。
  \item 魏安釐王薨,子魏景湣王增立。
  \item 赵悼襄王以李牧为将,伐燕,取武遂、方城。
  \item 逝世:魏安釐王、信陵君魏无忌。
  \end{enumerate} \tabularnewline\hline
  五年 & -242 & \begin{enumerate}
    \tiny
  \item 秦蒙骜伐魏,取酸枣、燕、虚、长平、雍丘、山阳等二十城;初置东郡。
  \item 燕王使剧辛将而伐赵。
  \end{enumerate} \tabularnewline\hline
  六年 & -241 & \begin{enumerate}
    \tiny
  \item 函谷关之战。
  \item 秦拔魏朝歌,及卫濮阳。
  \end{enumerate} \tabularnewline\hline
  七年 & -240 & \begin{enumerate}
    \tiny
  \item 秦置濮阳县,属东郡,并定其为东郡治所。
  \item 逝世:蒙骜、邹衍。
  \item 出生:陆贾。
  \item 天象:彗星光出东方,见北方,五月见西方。
  \end{enumerate} \tabularnewline\hline
  八年 & -239 & \begin{enumerate}
    \tiny
  \item 北扶余王国建立。
  \item 嫪毐封长信侯。
  \item 魏与赵邺。
  \item 文学:吕氏春秋编成。
  \item 逝世:长安君成蟜、韩桓惠王。
  \end{enumerate} \tabularnewline\hline
  九年 & -238 & \begin{enumerate}
    \tiny
  \item 嬴政亲政。
  \item 嫪毐叛乱,被秦王政夷灭三族。
  \item 秦伐魏,取垣、浦。
  \item 逝世:荀子、楚春申君黄歇、楚考烈王。
  \end{enumerate} \tabularnewline\hline
  十年 & -237 & \begin{enumerate}
    \tiny
  \item 齐王建拜会秦王政。
  \item 吕不韦免相。
  \item 秦王政下令驱除异邦客卿,李斯上书劝秦始皇收回逐客令。
  \end{enumerate} \tabularnewline\hline
  十一年 & -236 & \begin{enumerate}
    \tiny
  \item 郑国渠建成。
  \item 秦攻赵,赵攻燕\footnote{公元前236年,秦乘攻取赵的阏与、橑阳、邺、安阳等城,后又大举攻赵,遭到顽强抵抗。赵虽两次打败秦军,但兵力耗损殆尽。秦国西出太行山,突袭赵国邯郸拉开了统一战的的序幕。 赵国和燕国激战正酣,他想将秦国造成的领土损失在燕国身上补回来。这时秦国乘虚而入。赵国急忙命令大将李牧率军南下应敌。}。
  \end{enumerate} \tabularnewline\hline
  十二年 & -235 & \begin{enumerate}
    \tiny
  \item 秦攻楚国\footnote{秦继攻赵之后,即命辛梧率四郡兵,会同魏国,对楚国发起攻击。}。
  \item 吕不韦卒\footnote{因嫪毐集团叛乱事受牵连,被免除相邦职务,出居河南封地。不久,秦王政下令将其流放至蜀地(今四川),不韦忧惧交加,于是在三川郡(今河南洛阳)自鸩而亡。}。
  \end{enumerate} \tabularnewline\hline
  十三年 & -234 & \begin{enumerate}
    \tiny
  \item 秦攻赵\footnote{公元前234年,秦再度向赵南部进攻。桓龁避开正面渡河,改由漳河下游渡河迂回赵扈辄军的侧后,攻击邯郸东南的平阳。两军于平阳展开交战,赵军被击破,被斩10万人,赵将扈辄阵亡。赵王启用北部边疆名将李牧为统帅。李牧军曾歼灭匈奴入侵军10万之众,威震边疆,战斗力最强。李牧率军回赵,立即同秦桓龁军交战于宜安肥下地区,给秦军几乎全军覆灭的沉重打击,只有统帅桓龁带领少数护卫突围逃走。}。
  \item 韩非\footnote{韩非(约前281年-前233年),生活于战国末期时期的韩国(今属河南省新郑市)的思想家,为中国古代著名法家思想的代表人物,认为应该要“法”、“术”、“势”三者并重,是法家的集大成者。韩非出身韩国公族,与李斯均是荀子学生,后因其学识渊博,被秦始皇召唤入秦,正欲重用,却遭到妒忌的同窗李斯害死,在韩非死后,秦始皇在韩非的思想指引下,完成统一六国的帝业。韩非其学出于荀子,源于儒家,而成为法家,又推究老子思想,归本于道家。司马迁指出韩非喜好“刑名法术”且归本于道家的“黄老之学”,一套由“道”、“法”共同完善的政治统治理论。}作为韩国的使臣来到秦国,上书秦王,劝其先伐赵而缓伐韩。
  \end{enumerate} \tabularnewline\hline
  十四年 & -233 & \begin{enumerate}
    \tiny
  \item 韩非子卒。
  \item 燕抗秦\footnote{公元前233年,秦将樊於期叛逃至燕国后,太子丹的师傅鞠武害怕秦国以此借口攻燕,便策划送樊於期到头曼那里,利用熟悉秦国虚实的樊於期结连匈奴攻秦。可惜性急的太子丹等不得这种长远之计凑效,他决定派出荆轲刺杀自己的童年好友嬴政,为了能够解除嬴政的戒备,荆轲提出要携带两样礼物:樊於期的人头和燕国督亢地图(割地求和)。嬴政在逃过刺杀威胁后更以迅雷不及掩耳之势统一六国。}。
  \item 赵将李牧大败秦将桓齮\footnote{桓齮(yǐ)(?-前227年),战国末年秦国将军。杨宽的《战国史》认为桓齮就是樊於期。始皇十一年(前237年),桓齮与王翦和杨端和攻赵,取邺九城。秦始皇十四年,也就是赵王迁二年(前233年),桓齮从上党越太行山进攻赵的赤丽、宜安(石家庄东南),与赵将李牧战于肥下(宜安东北),为李牧所败,逃至燕国(《战国策》说是战败被杀,《资治通鉴》记载“秦师败绩,桓齮奔还”)后无相关记载。}于肥。
  \end{enumerate} \tabularnewline\hline
  十五年 & -232 & \begin{enumerate}
    \tiny
  \item 项羽出生。
  \item 太子丹回燕。
  \end{enumerate} \tabularnewline\hline
  十六年 & -231 & \begin{enumerate}
    \tiny
  \item 秦攻韩。
  \item 魏献丽邑。
  \item 赵国地震。
  \item 韩信出生。
  \end{enumerate} \tabularnewline\hline
  十七年 & -230 & \begin{enumerate}
    \tiny
  \item 韩国灭亡。
  \end{enumerate} \tabularnewline\hline
  十八年 & -229 & \begin{enumerate}
    \tiny
  \item 秦攻赵国。
  \item 李牧被杀。
  \end{enumerate} \tabularnewline\hline
  十九年 & -228 & \begin{enumerate}
    \tiny
  \item 秦破赵得和氏璧。
  \item 赵国灭亡。
  \end{enumerate} \tabularnewline\hline
  二十年 & -227 & \begin{enumerate}
    \tiny
  \item 荆轲刺秦王。
  \item 王翦、辛胜在易水西败燕、代联军。
  \end{enumerate} \tabularnewline\hline
  二一年 & -226 & \begin{enumerate}
    \tiny
  \item 秦军攻燕都。
  \item 秦攻蓟城。
  \end{enumerate} \tabularnewline\hline
  二二年 & -225 & \begin{enumerate}
    \tiny
  \item 魏国灭亡。
  \item 秦置砀郡,立浚仪(大梁)、启封两县。
  \end{enumerate} \tabularnewline\hline
  二三年 & -224 & \begin{enumerate}
    \tiny
  \item 秦楚之战。
  \item 秦置修武县。
  \end{enumerate} \tabularnewline\hline
  二四年 & -223 & \begin{enumerate}
    \tiny
  \item 楚将项燕自杀。
  \item 秦灭楚。
  \end{enumerate} \tabularnewline\hline
  二五年 & -222 & \begin{enumerate}
    \tiny
  \item 秦灭代。
  \item 秦灭燕。
  \end{enumerate} \tabularnewline
  \bottomrule
\end{longtable}

%%% Local Variables:
%%% mode: latex
%%% TeX-engine: xetex
%%% TeX-master: "../../Main"
%%% End:


%%% Local Variables:
%%% mode: latex
%%% TeX-engine: xetex
%%% TeX-master: "../../Main"
%%% End:

%% -*- coding: utf-8 -*-
%% Time-stamp: <Chen Wang: 2019-10-15 11:01:40>

\section{鲁}

%% -*- coding: utf-8 -*-
%% Time-stamp: <Chen Wang: 2018-07-12 23:25:47>

\subsection{穆公{\tiny(BC407-BC376)}}

% \centering
\begin{longtable}{|>{\centering\scriptsize}m{2em}|>{\centering\scriptsize}m{1.3em}|>{\centering}m{8.8em}|}
  % \caption{秦王政}\\
  \toprule
  \SimHei \normalsize 年数 & \SimHei \scriptsize 公元 & \SimHei 大事件 \tabularnewline
  % \midrule
  \endfirsthead
  \toprule
  \SimHei \normalsize 年数 & \SimHei \scriptsize 公元 & \SimHei 大事件 \tabularnewline
  \midrule
  \endhead
  \midrule
  六年 & -402 & \tabularnewline\hline
  七年 & -401 & \tabularnewline\hline
  八年 & -400 & \tabularnewline\hline
  九年 & -399 & \tabularnewline\hline
  十年 & -398 & \tabularnewline\hline
  十一年 & -397 & \tabularnewline\hline
  十二年 & -396 & \tabularnewline\hline
  十三年 & -395 & \tabularnewline\hline
  十四年 & -394 & \tabularnewline\hline
  十五年 & -393 & \tabularnewline\hline
  十六年 & -392 & \tabularnewline\hline
  十七年 & -391 & \tabularnewline\hline
  十八年 & -390 & \tabularnewline\hline
  十九年 & -389 & \tabularnewline\hline
  二十年 & -388 & \tabularnewline\hline
  二一年 & -387 & \tabularnewline\hline
  二二年 & -386 & \tabularnewline\hline
  二三年 & -385 & \tabularnewline\hline
  二四年 & -384 & \tabularnewline\hline
  二五年 & -383 & \tabularnewline\hline
  二六年 & -382 & \tabularnewline\hline
  二七年 & -381 & \tabularnewline\hline
  二八年 & -380 & \tabularnewline\hline
  二九年 & -379 & \tabularnewline\hline
  三十年 & -378 & \tabularnewline\hline
  三一年 & -377 & \tabularnewline\hline
  三二年 & -376 & \tabularnewline
  \bottomrule
\end{longtable}

%%% Local Variables:
%%% mode: latex
%%% TeX-engine: xetex
%%% TeX-master: "../../Main"
%%% End:


%%% Local Variables:
%%% mode: latex
%%% TeX-engine: xetex
%%% TeX-master: "../../Main"
%%% End:


%%% Local Variables:
%%% mode: latex
%%% TeX-engine: xetex
%%% TeX-master: "../Main"
%%% End:
 % 战国
% %% -*- coding: utf-8 -*-
%% Time-stamp: <Chen Wang: 2019-10-15 10:46:54>

\chapter{秦\tiny(BC221-BC207)}

%% -*- coding: utf-8 -*-
%% Time-stamp: <Chen Wang: 2018-07-10 17:29:45>

\section{始皇帝\tiny(BC221-BC210)}

\begin{longtable}{|>{\centering\scriptsize}m{2em}|>{\centering\scriptsize}m{1.3em}|>{\centering}m{8.8em}|}
  % \caption{秦王政}\
  \toprule
  \SimHei \normalsize 年数 & \SimHei \scriptsize 公元 & \SimHei 大事件 \tabularnewline
  % \midrule
  \endfirsthead
  \toprule
  \SimHei \normalsize 年数 & \SimHei \scriptsize 公元 & \SimHei 大事件 \tabularnewline
  \midrule
  \endhead
  \midrule
  二六年 & -221 & \begin{enumerate}
    \tiny
  \item 秦将王贲率军灭齐。
  \item 始皇统一中国。
  \item 秦攻百越\footnote{公元前221年,秦始皇统一后,令50万大军准备征服南方百越各部。秦军分5路南下,在越城岭遭到南方越人的顽强抵抗。}。
  \item 秦始凿灵渠\footnote{灵渠,建于秦始皇执政时期,是中国,也是世界上最早的运河之一。对中国岭南地区的开发起了重要作用。对今天的水利工程建设,仍然据有很好的参考价值}。
  \end{enumerate} \tabularnewline\hline
  二七年 & -220 & \begin{enumerate}
    \tiny
  \item 秦规划咸阳\footnote{公元前220年,秦始皇下令,将秦的东门由黄河延伸到上朐,并以咸阳和东门为中轴线规划新版图。}。
  \end{enumerate} \tabularnewline\hline
  二八年 & -219 & \begin{enumerate}
    \tiny
  \item 徐福\footnote{徐福,即徐巿”(在秦始皇本纪中称“徐巿”,在淮南衡山列传中称“徐福”)。(注意,是“巿”〔fú〕而不是“市”〔shì 〕),字君房,秦朝时齐地人,当时的著名方士。}出海。
  \item 始皇泰山封禅。
  \end{enumerate} \tabularnewline\hline
  二九年 & -218 & \begin{enumerate}
    \tiny
  \item 秦始皇第三次巡游,张良在博浪沙击始皇未中。
  \item 秦征岭南\footnote{尉佗真定人。公元前218年,奉秦始皇命令征岭南,略定南越后,任为南海龙川令。高后五年自立, 僭号“南越武帝”。 尉佗(?-前137年),真定(今石家庄市东古城)人。公元前218年,奉秦始皇命令征岭南,略定南越后,任为南海郡(治所在今广州市)龙川(今广档龙川县)令。秦二世时,赵佗受南海尉任嚣托,行南海尉事。秦亡后,出兵击并桂林郡( 治所在今广西桂平县西南古城)、象郡(治所在今广西崇左县),自立为南越王, 实行“和揖百越”的民族平等政策,采取一系列措施发展当地经济文化。}。
  \item 西瓯国反秦\footnote{公元前218年,西江中部的“西瓯国”起兵反秦,秦始皇派50万大军征讨。又派史禄在海阳山开凿灵渠,将湘江与漓江沟通,以保证军事上的运输。灵渠便成为中原汉人进入岭南的第一条主要通道。秦始皇灭了西瓯国,战争告一段落,秦“发诸尝捕亡人、赘婿、贾人略取陆梁地,为桂林、象郡、南海,以适遣戍。 ”(《史记.秦始皇本纪》)“五十万人守五岭。”(《集解》)这50万人,便是第一批汉族移民。秦始皇搞大迁徙,目的在于铲除六国的地方势力,把族人和故土分开,交叉汇编,徙到南蛮之地戍边,也就连根拔起,使之不能在秦的京城附近形成威胁,兹生复国复旧之梦。}。
  \end{enumerate} \tabularnewline\hline
  三十年 & -217 & \begin{enumerate}
    \tiny
  \item 始修建长城\footnote{秦灭六国之后,即开始北筑长城,每年征发民夫四十余万。全长7000多千米的长城,称作“九边重镇”,每镇设总兵官作为这一段长城的军事长官,受兵部的指挥,负责所辖军区内的防务或奉命支援相邻军区的防务。}。
  \end{enumerate} \tabularnewline\hline
  三一年 & -216 & \begin{enumerate}
    \tiny
  \item 秦改革屯田制\footnote{平民自报所占土地面积,自报耕地面积、土地产量及大小人丁。所报内容由乡出人审查核实,并统一评定产量,计算每户应纳税额,最后登记入册,上报到县,经批准后,即按登记数征收。此前著名的改革家商鞅还在秦国推行了包括土地制度在内的改革。提出了“算地”和“定分”的主张。“算地”就是对土地进行全面的调查核算,以作为制定土地政策的客观依据;“定分”就是用法律形式确认地主或平民对土地占有的“名分”,确认土地所有权。这些实际上都是土地登记的内容。}。
  \item 始皇微行咸阳,兰池遇盗,武士击杀之。大索二十日。
  \item 西汉七国之乱主谋,刘邦之侄,吴王刘濞出生。
  \end{enumerate} \tabularnewline\hline
  三二年 & -215 & \begin{enumerate}
    \tiny
  \item 始皇在今广西等地建立了桂林郡和象郡。
  \item 始皇东巡到达蓟城。
  \item 秦将蒙恬筑马邑城池,置马邑县。
  \end{enumerate} \tabularnewline\hline
  三三年 & -214 & \begin{enumerate}
    \tiny
  \item 灵渠建成。
  \item 秦设龙川县。
  \item 秦设南海郡。
  \item 秦占岭南,夺高阙、阳山、北假\footnote{公元前214年,秦始皇派遣50万军队分5路攻占岭南,任命任嚣为南海尉。派蒙恬渡过黄河去夺取高阙、阳山、北假一带地方,筑起堡垒以驱逐戎狄。迁移被贬谪的人,让他们充实新设置的县。}。
  \end{enumerate} \tabularnewline\hline
  三四年 & -213 & \begin{enumerate}
    \tiny
  \item 李斯任左丞相。
  \item 淳于越谏秦。
  \item 焚书事件。
  \item 秦颁行《挟书令》。
  \item 秦在五岭开山道筑三关,即横浦关、阳山关、湟鸡谷关。
  \item 秦始修筑驰道。
  \end{enumerate} \tabularnewline\hline
  三五年 & -212 & \begin{enumerate}
    \tiny
  \item 修建阿房宫。
  \item 扶苏被派往上郡(今天的陕西绥德)做大将蒙恬的监军。
  \item 焚书坑儒。
  \item 蒙恬率领大军修建了一条从咸阳到九原(今内蒙古包头市)的直道。
  \end{enumerate} \tabularnewline\hline
  三六年 & -211 & \begin{enumerate}
    \tiny
  \item 陨石事件\footnote{秦始皇三十六年,一颗流星坠落到了东郡。东郡是在秦始皇即位之初吕不韦主政时攻打下来的,当时此郡是齐、秦两国的交界地。现在已是大秦帝国的一个东方大郡。陨石落地还不可怕,可怕的是陨石上面刻的字“始皇帝死而地分”。这七个字非同小可!它代表了上天的旨意,预示着秦始皇将死,同时也预告了大秦帝国将亡。}。
  \item 汉惠帝刘盈出生。
  \item 秦置皮氏县。
  \end{enumerate} \tabularnewline\hline
  三七年 & -210 & \begin{enumerate}
    \tiny
  \item 始皇卒\footnote{秦始皇三十七年(公元前210年),秦始皇出巡至平原津(今德州平原县南六十里有张公故城,城东有水津)而病,秦始皇不愿意听到“死”,所以群臣莫敢言死事。8月28日行至沙丘(沙丘台在邢州平乡县东北二十里)病死。}。
  \item 扶苏被害。
  \item 胡亥\footnote{秦二世胡亥(前230年—前207年,在位时间前209年—前207年),也称二世皇帝。是秦始皇第二十六子,公子扶苏的弟弟。秦始皇出游南方病死途中时,在赵高与李斯的帮助下,杀害哥哥扶苏当上秦朝的二世皇帝。贾谊《过秦论》曰:“始皇既没,胡亥极愚,郦山未毕,复作阿房,以遂前策。云“凡所为贵有天下者,肆意极欲,大臣至欲罢先君所为”。诛斯、去疾,任用赵高。痛哉言乎!人头畜鸣。不威不伐恶,不笃不虚亡。距之不得留,残虐以促期,虽居形便之国,犹不得存。”}称帝,是为秦二世。
  \end{enumerate} \tabularnewline
  \bottomrule
\end{longtable}


%%% Local Variables:
%%% mode: latex
%%% TeX-engine: xetex
%%% TeX-master: "../Main"
%%% End:

%% -*- coding: utf-8 -*-
%% Time-stamp: <Chen Wang: 2018-07-10 17:29:32>

\section{秦二世\tiny(BC209-BC207)}

\begin{longtable}{|>{\centering\scriptsize}m{2em}|>{\centering\scriptsize}m{1.3em}|>{\centering}m{8.8em}|}
  % \caption{秦王政}\
  \toprule
  \SimHei \normalsize 年数 & \SimHei \scriptsize 公元 & \SimHei 大事件 \tabularnewline
  % \midrule
  \endfirsthead
  \toprule
  \SimHei \normalsize 年数 & \SimHei \scriptsize 公元 & \SimHei 大事件 \tabularnewline
  \midrule
  \endhead
  \midrule
  元年 & -209 & \begin{enumerate}
    \tiny
  \item 大泽乡起义。
  \item 刘邦起义。
  \item 项羽反秦。
  \item 冒顿即位。
  \end{enumerate} \tabularnewline\hline
  二年 & -208 & \begin{enumerate}
    \tiny
  \item 秦灭项梁。
  \item 孔鲋逝世。
  \item 陈胜卒。
  \item 李斯卒。
  \item 薛地会议。
  \item 统一越南。
  \end{enumerate} \tabularnewline\hline
  三年 & -207 & \begin{enumerate}
    \tiny
  \item 指鹿为马。
  \item 破釜沉舟。
  \item 胡亥被弑。
  \item 子婴即位,诛赵高,在位47天被废。
  \end{enumerate} \tabularnewline
  \bottomrule
\end{longtable}


%%% Local Variables:
%%% mode: latex
%%% TeX-engine: xetex
%%% TeX-master: "../Main"
%%% End:

%% -*- coding: utf-8 -*-
%% Time-stamp: <Chen Wang: 2018-07-10 17:29:53>

\section{子婴\tiny(BC206-BC206)}

\begin{longtable}{|>{\centering\scriptsize}m{2em}|>{\centering\scriptsize}m{1.3em}|>{\centering}m{8.8em}|}
  % \caption{秦王政}\
  \toprule
  \SimHei \normalsize 年数 & \SimHei \scriptsize 公元 & \SimHei 大事件 \tabularnewline
  % \midrule
  \endfirsthead
  \toprule
  \SimHei \normalsize 年数 & \SimHei \scriptsize 公元 & \SimHei 大事件 \tabularnewline
  \midrule
  \endhead
  \midrule
  元年 & -206 & \tabularnewline
  \bottomrule
\end{longtable}


%%% Local Variables:
%%% mode: latex
%%% TeX-engine: xetex
%%% TeX-master: "../Main"
%%% End:


%%% Local Variables:
%%% mode: latex
%%% TeX-engine: xetex
%%% TeX-master: "../Main"
%%% End:
 % 秦
% %% -*- coding: utf-8 -*-
%% Time-stamp: <Chen Wang: 2019-10-15 11:06:58>

\chapter{西汉\tiny(BC202-8)}

%% -*- coding: utf-8 -*-
%% Time-stamp: <Chen Wang: 2018-07-10 17:28:38>

\section{楚汉之争\tiny(BC206-BC203)}

\begin{longtable}{|>{\centering\scriptsize}m{2em}|>{\centering\scriptsize}m{1.3em}|>{\centering}m{8.8em}|}
  % \caption{秦王政}\
  \toprule
  \SimHei \normalsize 年数 & \SimHei \scriptsize 公元 & \SimHei 大事件 \tabularnewline
  % \midrule
  \endfirsthead
  \toprule
  \SimHei \normalsize 年数 & \SimHei \scriptsize 公元 & \SimHei 大事件 \tabularnewline
  \midrule
  \endhead
  \midrule
  高祖\\元年 & -206 & \begin{enumerate}
    \tiny
  \item 秦朝灭亡。
  \item 鸿门宴。
  \item 项羽建立西楚王朝,自称西楚霸王。
  \end{enumerate} \tabularnewline\hline
  二年 & -205 & \begin{enumerate}
    \tiny
  \item 彭城之战。
  \item 成皋之战。
  \item 韩信破代、赵。
  \item 韩信灭燕、齐。
  \end{enumerate} \tabularnewline\hline
  三年 & -204 & \begin{enumerate}
    \tiny
  \item 背水一战。
  \item 南越国建立。
  \item 成皋之战。
  \end{enumerate} \tabularnewline\hline
  四年 & -203 & \begin{enumerate}
    \tiny
  \item 英布封王。
  \item 张耳封王。
  \end{enumerate} \tabularnewline
  \bottomrule
\end{longtable}


%%% Local Variables:
%%% mode: latex
%%% TeX-engine: xetex
%%% TeX-master: "../Main"
%%% End:

%% -*- coding: utf-8 -*-
%% Time-stamp: <Chen Wang: 2018-07-10 17:28:48>

\section{汉高祖\tiny(BC206-BC195)}

\begin{longtable}{|>{\centering\scriptsize}m{2em}|>{\centering\scriptsize}m{1.3em}|>{\centering}m{8.8em}|}
  % \caption{秦王政}\
  \toprule
  \SimHei \normalsize 年数 & \SimHei \scriptsize 公元 & \SimHei 大事件 \tabularnewline
  % \midrule
  \endfirsthead
  \toprule
  \SimHei \normalsize 年数 & \SimHei \scriptsize 公元 & \SimHei 大事件 \tabularnewline
  \midrule
  \endhead
  \midrule
  五年 & -202 & \begin{enumerate}
    \tiny
  \item 十二月垓下之战,汉灭楚统一天下,汉王刘邦即皇帝位。
  \item 汉置长安县、无锡县。
  \item 七月,燕王臧荼起兵反汉。
  \item 十月,刘邦率军亲征灭燕,俘杀臧荼。刘邦立卢绾为燕王。
  \item 汉高祖册封无诸为闽越王,封国闽越,首都冶城位于今之福州。
  \end{enumerate} \tabularnewline\hline
  六年 & -201 & \tabularnewline\hline
  七年 & -200 & \tabularnewline\hline
  八年 & -199 & \tabularnewline\hline
  九年 & -198 & \tabularnewline\hline
  十年 & -197 & \tabularnewline\hline
  十一年 & -196 & \tabularnewline\hline
  十二年 & -195 & \tabularnewline
  \bottomrule
\end{longtable}


%%% Local Variables:
%%% mode: latex
%%% TeX-engine: xetex
%%% TeX-master: "../Main"
%%% End:

%% -*- coding: utf-8 -*-
%% Time-stamp: <Chen Wang: 2018-07-10 17:29:04>

\section{孝惠帝\tiny(BC195-BC188)}

\begin{longtable}{|>{\centering\scriptsize}m{2em}|>{\centering\scriptsize}m{1.3em}|>{\centering}m{8.8em}|}
  % \caption{秦王政}\
  \toprule
  \SimHei \normalsize 年数 & \SimHei \scriptsize 公元 & \SimHei 大事件 \tabularnewline
  % \midrule
  \endfirsthead
  \toprule
  \SimHei \normalsize 年数 & \SimHei \scriptsize 公元 & \SimHei 大事件 \tabularnewline
  \midrule
  \endhead
  \midrule
  元年 & -194 & \tabularnewline\hline
  二年 & -193 & \tabularnewline\hline
  三年 & -192 & \tabularnewline\hline
  四年 & -191 & \tabularnewline\hline
  五年 & -190 & \tabularnewline\hline
  六年 & -189 & \tabularnewline\hline
  七年 & -188 & \tabularnewline
  \bottomrule
\end{longtable}


%%% Local Variables:
%%% mode: latex
%%% TeX-engine: xetex
%%% TeX-master: "../Main"
%%% End:

%% -*- coding: utf-8 -*-
%% Time-stamp: <Chen Wang: 2018-07-10 17:28:59>

\section{前少帝\tiny(BC187-BC184)}

\begin{longtable}{|>{\centering\scriptsize}m{2em}|>{\centering\scriptsize}m{1.3em}|>{\centering}m{8.8em}|}
  % \caption{秦王政}\
  \toprule
  \SimHei \normalsize 年数 & \SimHei \scriptsize 公元 & \SimHei 大事件 \tabularnewline
  % \midrule
  \endfirsthead
  \toprule
  \SimHei \normalsize 年数 & \SimHei \scriptsize 公元 & \SimHei 大事件 \tabularnewline
  \midrule
  \endhead
  \midrule
  元年 & -187 & \tabularnewline\hline
  二年 & -186 & \tabularnewline\hline
  三年 & -185 & \tabularnewline\hline
  四年 & -184 & \tabularnewline
  \bottomrule
\end{longtable}


%%% Local Variables:
%%% mode: latex
%%% TeX-engine: xetex
%%% TeX-master: "../Main"
%%% End:

%% -*- coding: utf-8 -*-
%% Time-stamp: <Chen Wang: 2018-07-10 17:28:54>

\section{后少帝\tiny(BC183-BC180)}

\begin{longtable}{|>{\centering\scriptsize}m{2em}|>{\centering\scriptsize}m{1.3em}|>{\centering}m{8.8em}|}
  % \caption{秦王政}\
  \toprule
  \SimHei \normalsize 年数 & \SimHei \scriptsize 公元 & \SimHei 大事件 \tabularnewline
  % \midrule
  \endfirsthead
  \toprule
  \SimHei \normalsize 年数 & \SimHei \scriptsize 公元 & \SimHei 大事件 \tabularnewline
  \midrule
  \endhead
  \midrule
  元年 & -183 & \tabularnewline\hline
  二年 & -182 & \tabularnewline\hline
  三年 & -181 & \tabularnewline\hline
  四年 & -180 & \tabularnewline
  \bottomrule
\end{longtable}


%%% Local Variables:
%%% mode: latex
%%% TeX-engine: xetex
%%% TeX-master: "../Main"
%%% End:

%% -*- coding: utf-8 -*-
%% Time-stamp: <Chen Wang: 2018-07-10 17:28:15>

\section{孝文帝\tiny(BC179-BC157)}

\subsection{前元}

\begin{longtable}{|>{\centering\scriptsize}m{2em}|>{\centering\scriptsize}m{1.3em}|>{\centering}m{8.8em}|}
  % \caption{秦王政}\
  \toprule
  \SimHei \normalsize 年数 & \SimHei \scriptsize 公元 & \SimHei 大事件 \tabularnewline
  % \midrule
  \endfirsthead
  \toprule
  \SimHei \normalsize 年数 & \SimHei \scriptsize 公元 & \SimHei 大事件 \tabularnewline
  \midrule
  \endhead
  \midrule
  元年 & -179 & \tabularnewline\hline
  二年 & -178 & \tabularnewline\hline
  三年 & -177 & \tabularnewline\hline
  四年 & -176 & \tabularnewline\hline
  五年 & -175 & \tabularnewline\hline
  六年 & -174 & \tabularnewline\hline
  七年 & -173 & \tabularnewline\hline
  八年 & -172 & \tabularnewline\hline
  九年 & -171 & \tabularnewline\hline
  十年 & -170 & \tabularnewline\hline
  十一年 & -169 & \tabularnewline\hline
  十二年 & -168 & \tabularnewline\hline
  十三年 & -167 & \tabularnewline\hline
  十四年 & -166 & \tabularnewline\hline
  十五年 & -165 & \tabularnewline\hline
  十六年 & -164 & \tabularnewline
  \bottomrule
\end{longtable}


\subsection{后元}

\begin{longtable}{|>{\centering\scriptsize}m{2em}|>{\centering\scriptsize}m{1.3em}|>{\centering}m{8.8em}|}
  % \caption{秦王政}\
  \toprule
  \SimHei \normalsize 年数 & \SimHei \scriptsize 公元 & \SimHei 大事件 \tabularnewline
  % \midrule
  \endfirsthead
  \toprule
  \SimHei \normalsize 年数 & \SimHei \scriptsize 公元 & \SimHei 大事件 \tabularnewline
  \midrule
  \endhead
  \midrule
  元年 & -163 & \tabularnewline\hline
  二年 & -162 & \tabularnewline\hline
  三年 & -161 & \tabularnewline\hline
  四年 & -160 & \tabularnewline\hline
  五年 & -159 & \tabularnewline\hline
  六年 & -158 & \tabularnewline\hline
  七年 & -157 & \tabularnewline
  \bottomrule
\end{longtable}


%%% Local Variables:
%%% mode: latex
%%% TeX-engine: xetex
%%% TeX-master: "../Main"
%%% End:

%% -*- coding: utf-8 -*-
%% Time-stamp: <Chen Wang: 2018-07-10 17:44:44>

\section{孝景帝\tiny(BC156-BC141)}

\subsection{前元}

\begin{longtable}{|>{\centering\scriptsize}m{2em}|>{\centering\scriptsize}m{1.3em}|>{\centering}m{8.8em}|}
  % \caption{秦王政}\
  \toprule
  \SimHei \normalsize 年数 & \SimHei \scriptsize 公元 & \SimHei 大事件 \tabularnewline
  % \midrule
  \endfirsthead
  \toprule
  \SimHei \normalsize 年数 & \SimHei \scriptsize 公元 & \SimHei 大事件 \tabularnewline
  \midrule
  \endhead
  \midrule
  元年 & -156 & \tabularnewline\hline
  二年 & -155 & \tabularnewline\hline
  三年 & -154 & \tabularnewline\hline
  四年 & -153 & \tabularnewline\hline
  五年 & -152 & \tabularnewline\hline
  六年 & -151 & \tabularnewline\hline
  七年 & -150 & \tabularnewline
  \bottomrule
\end{longtable}


\subsection{中元}

\begin{longtable}{|>{\centering\scriptsize}m{2em}|>{\centering\scriptsize}m{1.3em}|>{\centering}m{8.8em}|}
  % \caption{秦王政}\
  \toprule
  \SimHei \normalsize 年数 & \SimHei \scriptsize 公元 & \SimHei 大事件 \tabularnewline
  % \midrule
  \endfirsthead
  \toprule
  \SimHei \normalsize 年数 & \SimHei \scriptsize 公元 & \SimHei 大事件 \tabularnewline
  \midrule
  \endhead
  \midrule
  元年 & -149 & \tabularnewline\hline
  二年 & -148 & \tabularnewline\hline
  三年 & -147 & \tabularnewline\hline
  四年 & -146 & \tabularnewline\hline
  五年 & -145 & \tabularnewline\hline
  六年 & -144 & \tabularnewline
  \bottomrule
\end{longtable}


\subsection{后元}

\begin{longtable}{|>{\centering\scriptsize}m{2em}|>{\centering\scriptsize}m{1.3em}|>{\centering}m{8.8em}|}
  % \caption{秦王政}\
  \toprule
  \SimHei \normalsize 年数 & \SimHei \scriptsize 公元 & \SimHei 大事件 \tabularnewline
  % \midrule
  \endfirsthead
  \toprule
  \SimHei \normalsize 年数 & \SimHei \scriptsize 公元 & \SimHei 大事件 \tabularnewline
  \midrule
  \endhead
  \midrule
  元年 & -143 & \tabularnewline\hline
  二年 & -142 & \tabularnewline\hline
  三年 & -141 & \tabularnewline
  \bottomrule
\end{longtable}


%%% Local Variables:
%%% mode: latex
%%% TeX-engine: xetex
%%% TeX-master: "../Main"
%%% End:

%% -*- coding: utf-8 -*-
%% Time-stamp: <Chen Wang: 2018-07-10 18:54:22>

\section{武帝\tiny(BC140-BC87)}

\subsection{建元}

\begin{longtable}{|>{\centering\scriptsize}m{2em}|>{\centering\scriptsize}m{1.3em}|>{\centering}m{8.8em}|}
  % \caption{秦王政}\
  \toprule
  \SimHei \normalsize 年数 & \SimHei \scriptsize 公元 & \SimHei 大事件 \tabularnewline
  % \midrule
  \endfirsthead
  \toprule
  \SimHei \normalsize 年数 & \SimHei \scriptsize 公元 & \SimHei 大事件 \tabularnewline
  \midrule
  \endhead
  \midrule
  元年 & -140 & \tabularnewline\hline
  二年 & -139 & \tabularnewline\hline
  三年 & -138 & \tabularnewline\hline
  四年 & -137 & \tabularnewline\hline
  五年 & -136 & \tabularnewline\hline
  六年 & -135 & \tabularnewline
  \bottomrule
\end{longtable}


\subsection{元光}

\begin{longtable}{|>{\centering\scriptsize}m{2em}|>{\centering\scriptsize}m{1.3em}|>{\centering}m{8.8em}|}
  % \caption{秦王政}\
  \toprule
  \SimHei \normalsize 年数 & \SimHei \scriptsize 公元 & \SimHei 大事件 \tabularnewline
  % \midrule
  \endfirsthead
  \toprule
  \SimHei \normalsize 年数 & \SimHei \scriptsize 公元 & \SimHei 大事件 \tabularnewline
  \midrule
  \endhead
  \midrule
  元年 & -134 & \tabularnewline\hline
  二年 & -133 & \tabularnewline\hline
  三年 & -132 & \tabularnewline\hline
  四年 & -131 & \tabularnewline\hline
  五年 & -130 & \tabularnewline\hline
  六年 & -129 & \tabularnewline
  \bottomrule
\end{longtable}


\subsection{元朔}

\begin{longtable}{|>{\centering\scriptsize}m{2em}|>{\centering\scriptsize}m{1.3em}|>{\centering}m{8.8em}|}
  % \caption{秦王政}\
  \toprule
  \SimHei \normalsize 年数 & \SimHei \scriptsize 公元 & \SimHei 大事件 \tabularnewline
  % \midrule
  \endfirsthead
  \toprule
  \SimHei \normalsize 年数 & \SimHei \scriptsize 公元 & \SimHei 大事件 \tabularnewline
  \midrule
  \endhead
  \midrule
  元年 & -128 & \tabularnewline\hline
  二年 & -127 & \tabularnewline\hline
  三年 & -126 & \tabularnewline\hline
  四年 & -125 & \tabularnewline\hline
  五年 & -124 & \tabularnewline\hline
  六年 & -123 & \tabularnewline
  \bottomrule
\end{longtable}

\subsection{元狩}

\begin{longtable}{|>{\centering\scriptsize}m{2em}|>{\centering\scriptsize}m{1.3em}|>{\centering}m{8.8em}|}
  % \caption{秦王政}\
  \toprule
  \SimHei \normalsize 年数 & \SimHei \scriptsize 公元 & \SimHei 大事件 \tabularnewline
  % \midrule
  \endfirsthead
  \toprule
  \SimHei \normalsize 年数 & \SimHei \scriptsize 公元 & \SimHei 大事件 \tabularnewline
  \midrule
  \endhead
  \midrule
  元年 & -122 & \tabularnewline\hline
  二年 & -121 & \tabularnewline\hline
  三年 & -120 & \tabularnewline\hline
  四年 & -119 & \tabularnewline\hline
  五年 & -118 & \tabularnewline\hline
  六年 & -117 & \tabularnewline  
  \bottomrule
\end{longtable}

\subsection{元鼎}

\begin{longtable}{|>{\centering\scriptsize}m{2em}|>{\centering\scriptsize}m{1.3em}|>{\centering}m{8.8em}|}
  % \caption{秦王政}\
  \toprule
  \SimHei \normalsize 年数 & \SimHei \scriptsize 公元 & \SimHei 大事件 \tabularnewline
  % \midrule
  \endfirsthead
  \toprule
  \SimHei \normalsize 年数 & \SimHei \scriptsize 公元 & \SimHei 大事件 \tabularnewline
  \midrule
  \endhead
  \midrule
  元年 & -116 & \tabularnewline\hline
  二年 & -115 & \tabularnewline\hline
  三年 & -114 & \tabularnewline\hline
  四年 & -113 & \tabularnewline\hline
  五年 & -112 & \tabularnewline\hline
  六年 & -111 & \tabularnewline  
  \bottomrule
\end{longtable}

\subsection{元封}

\begin{longtable}{|>{\centering\scriptsize}m{2em}|>{\centering\scriptsize}m{1.3em}|>{\centering}m{8.8em}|}
  % \caption{秦王政}\
  \toprule
  \SimHei \normalsize 年数 & \SimHei \scriptsize 公元 & \SimHei 大事件 \tabularnewline
  % \midrule
  \endfirsthead
  \toprule
  \SimHei \normalsize 年数 & \SimHei \scriptsize 公元 & \SimHei 大事件 \tabularnewline
  \midrule
  \endhead
  \midrule
  元年 & -110 & \tabularnewline\hline
  二年 & -109 & \tabularnewline\hline
  三年 & -108 & \tabularnewline\hline
  四年 & -107 & \tabularnewline\hline
  五年 & -106 & \tabularnewline\hline
  六年 & -105 & \tabularnewline
  \bottomrule
\end{longtable}

\subsection{太初}

\begin{longtable}{|>{\centering\scriptsize}m{2em}|>{\centering\scriptsize}m{1.3em}|>{\centering}m{8.8em}|}
  % \caption{秦王政}\
  \toprule
  \SimHei \normalsize 年数 & \SimHei \scriptsize 公元 & \SimHei 大事件 \tabularnewline
  % \midrule
  \endfirsthead
  \toprule
  \SimHei \normalsize 年数 & \SimHei \scriptsize 公元 & \SimHei 大事件 \tabularnewline
  \midrule
  \endhead
  \midrule
  元年 & -104 & \tabularnewline\hline
  二年 & -103 & \tabularnewline\hline
  三年 & -102 & \tabularnewline\hline
  四年 & -101 & \tabularnewline
  \bottomrule
\end{longtable}

\subsection{天汉}

\begin{longtable}{|>{\centering\scriptsize}m{2em}|>{\centering\scriptsize}m{1.3em}|>{\centering}m{8.8em}|}
  % \caption{秦王政}\
  \toprule
  \SimHei \normalsize 年数 & \SimHei \scriptsize 公元 & \SimHei 大事件 \tabularnewline
  % \midrule
  \endfirsthead
  \toprule
  \SimHei \normalsize 年数 & \SimHei \scriptsize 公元 & \SimHei 大事件 \tabularnewline
  \midrule
  \endhead
  \midrule
  元年 & -100 & \tabularnewline\hline
  二年 & -99 & \tabularnewline\hline
  三年 & -98 & \tabularnewline\hline
  四年 & -97 & \tabularnewline
  \bottomrule
\end{longtable}

\subsection{太始}

\begin{longtable}{|>{\centering\scriptsize}m{2em}|>{\centering\scriptsize}m{1.3em}|>{\centering}m{8.8em}|}
  % \caption{秦王政}\
  \toprule
  \SimHei \normalsize 年数 & \SimHei \scriptsize 公元 & \SimHei 大事件 \tabularnewline
  % \midrule
  \endfirsthead
  \toprule
  \SimHei \normalsize 年数 & \SimHei \scriptsize 公元 & \SimHei 大事件 \tabularnewline
  \midrule
  \endhead
  \midrule
  元年 & -96 & \tabularnewline\hline
  二年 & -95 & \tabularnewline\hline
  三年 & -94 & \tabularnewline\hline
  四年 & -93 & \tabularnewline
  \bottomrule
\end{longtable}

\subsection{征和}

\begin{longtable}{|>{\centering\scriptsize}m{2em}|>{\centering\scriptsize}m{1.3em}|>{\centering}m{8.8em}|}
  % \caption{秦王政}\
  \toprule
  \SimHei \normalsize 年数 & \SimHei \scriptsize 公元 & \SimHei 大事件 \tabularnewline
  % \midrule
  \endfirsthead
  \toprule
  \SimHei \normalsize 年数 & \SimHei \scriptsize 公元 & \SimHei 大事件 \tabularnewline
  \midrule
  \endhead
  \midrule
  元年 & -92 & \tabularnewline\hline
  二年 & -91 & \tabularnewline\hline
  三年 & -90 & \tabularnewline\hline
  四年 & -89 & \tabularnewline
  \bottomrule
\end{longtable}

\subsection{后元}

\begin{longtable}{|>{\centering\scriptsize}m{2em}|>{\centering\scriptsize}m{1.3em}|>{\centering}m{8.8em}|}
  % \caption{秦王政}\
  \toprule
  \SimHei \normalsize 年数 & \SimHei \scriptsize 公元 & \SimHei 大事件 \tabularnewline
  % \midrule
  \endfirsthead
  \toprule
  \SimHei \normalsize 年数 & \SimHei \scriptsize 公元 & \SimHei 大事件 \tabularnewline
  \midrule
  \endhead
  \midrule
  元年 & -88 & \tabularnewline\hline
  二年 & -87 & \tabularnewline
  \bottomrule
\end{longtable}


%%% Local Variables:
%%% mode: latex
%%% TeX-engine: xetex
%%% TeX-master: "../Main"
%%% End:

%% -*- coding: utf-8 -*-
%% Time-stamp: <Chen Wang: 2018-07-10 18:58:40>

\section{昭帝\tiny(BC87-BC74)}

\subsection{始元}

\begin{longtable}{|>{\centering\scriptsize}m{2em}|>{\centering\scriptsize}m{1.3em}|>{\centering}m{8.8em}|}
  % \caption{秦王政}\
  \toprule
  \SimHei \normalsize 年数 & \SimHei \scriptsize 公元 & \SimHei 大事件 \tabularnewline
  % \midrule
  \endfirsthead
  \toprule
  \SimHei \normalsize 年数 & \SimHei \scriptsize 公元 & \SimHei 大事件 \tabularnewline
  \midrule
  \endhead
  \midrule
  元年 & -86 & \tabularnewline\hline
  二年 & -85 & \tabularnewline\hline
  三年 & -84 & \tabularnewline\hline
  四年 & -83 & \tabularnewline\hline
  五年 & -82 & \tabularnewline\hline
  六年 & -81 & \tabularnewline\hline
  七年 & -80 & \tabularnewline
  \bottomrule
\end{longtable}


\subsection{元凤}

\begin{longtable}{|>{\centering\scriptsize}m{2em}|>{\centering\scriptsize}m{1.3em}|>{\centering}m{8.8em}|}
  % \caption{秦王政}\
  \toprule
  \SimHei \normalsize 年数 & \SimHei \scriptsize 公元 & \SimHei 大事件 \tabularnewline
  % \midrule
  \endfirsthead
  \toprule
  \SimHei \normalsize 年数 & \SimHei \scriptsize 公元 & \SimHei 大事件 \tabularnewline
  \midrule
  \endhead
  \midrule
  元年 & -80 & \tabularnewline\hline
  二年 & -79 & \tabularnewline\hline
  三年 & -78 & \tabularnewline\hline
  四年 & -77 & \tabularnewline\hline
  五年 & -76 & \tabularnewline\hline
  六年 & -75 & \tabularnewline
  \bottomrule
\end{longtable}


\subsection{元平}

\begin{longtable}{|>{\centering\scriptsize}m{2em}|>{\centering\scriptsize}m{1.3em}|>{\centering}m{8.8em}|}
  % \caption{秦王政}\
  \toprule
  \SimHei \normalsize 年数 & \SimHei \scriptsize 公元 & \SimHei 大事件 \tabularnewline
  % \midrule
  \endfirsthead
  \toprule
  \SimHei \normalsize 年数 & \SimHei \scriptsize 公元 & \SimHei 大事件 \tabularnewline
  \midrule
  \endhead
  \midrule
  元年 & -74 & \tabularnewline
  \bottomrule
\end{longtable}


%%% Local Variables:
%%% mode: latex
%%% TeX-engine: xetex
%%% TeX-master: "../Main"
%%% End:

%% -*- coding: utf-8 -*-
%% Time-stamp: <Chen Wang: 2018-07-10 19:05:08>

\section{宣帝\tiny(BC74-BC49)}

\subsection{本始}

\begin{longtable}{|>{\centering\scriptsize}m{2em}|>{\centering\scriptsize}m{1.3em}|>{\centering}m{8.8em}|}
  % \caption{秦王政}\
  \toprule
  \SimHei \normalsize 年数 & \SimHei \scriptsize 公元 & \SimHei 大事件 \tabularnewline
  % \midrule
  \endfirsthead
  \toprule
  \SimHei \normalsize 年数 & \SimHei \scriptsize 公元 & \SimHei 大事件 \tabularnewline
  \midrule
  \endhead
  \midrule
  元年 & -73 & \tabularnewline\hline
  二年 & -72 & \tabularnewline\hline
  三年 & -71 & \tabularnewline\hline
  四年 & -70 & \tabularnewline
  \bottomrule
\end{longtable}


\subsection{地节}

\begin{longtable}{|>{\centering\scriptsize}m{2em}|>{\centering\scriptsize}m{1.3em}|>{\centering}m{8.8em}|}
  % \caption{秦王政}\
  \toprule
  \SimHei \normalsize 年数 & \SimHei \scriptsize 公元 & \SimHei 大事件 \tabularnewline
  % \midrule
  \endfirsthead
  \toprule
  \SimHei \normalsize 年数 & \SimHei \scriptsize 公元 & \SimHei 大事件 \tabularnewline
  \midrule
  \endhead
  \midrule
  元年 & -69 & \tabularnewline\hline
  二年 & -68 & \tabularnewline\hline
  三年 & -67 & \tabularnewline\hline
  四年 & -66 & \tabularnewline
  \bottomrule
\end{longtable}


\subsection{元康}

\begin{longtable}{|>{\centering\scriptsize}m{2em}|>{\centering\scriptsize}m{1.3em}|>{\centering}m{8.8em}|}
  % \caption{秦王政}\
  \toprule
  \SimHei \normalsize 年数 & \SimHei \scriptsize 公元 & \SimHei 大事件 \tabularnewline
  % \midrule
  \endfirsthead
  \toprule
  \SimHei \normalsize 年数 & \SimHei \scriptsize 公元 & \SimHei 大事件 \tabularnewline
  \midrule
  \endhead
  \midrule
  元年 & -65 & \tabularnewline\hline
  二年 & -64 & \tabularnewline\hline
  三年 & -63 & \tabularnewline\hline
  四年 & -62 & \tabularnewline
  \bottomrule
\end{longtable}

\subsection{神爵}

\begin{longtable}{|>{\centering\scriptsize}m{2em}|>{\centering\scriptsize}m{1.3em}|>{\centering}m{8.8em}|}
  % \caption{秦王政}\
  \toprule
  \SimHei \normalsize 年数 & \SimHei \scriptsize 公元 & \SimHei 大事件 \tabularnewline
  % \midrule
  \endfirsthead
  \toprule
  \SimHei \normalsize 年数 & \SimHei \scriptsize 公元 & \SimHei 大事件 \tabularnewline
  \midrule
  \endhead
  \midrule
  元年 & -61 & \tabularnewline\hline
  二年 & -60 & \tabularnewline\hline
  三年 & -59 & \tabularnewline\hline
  四年 & -58 & \tabularnewline
  \bottomrule
\end{longtable}

\subsection{五凤}

\begin{longtable}{|>{\centering\scriptsize}m{2em}|>{\centering\scriptsize}m{1.3em}|>{\centering}m{8.8em}|}
  % \caption{秦王政}\
  \toprule
  \SimHei \normalsize 年数 & \SimHei \scriptsize 公元 & \SimHei 大事件 \tabularnewline
  % \midrule
  \endfirsthead
  \toprule
  \SimHei \normalsize 年数 & \SimHei \scriptsize 公元 & \SimHei 大事件 \tabularnewline
  \midrule
  \endhead
  \midrule
  元年 & -57 & \tabularnewline\hline
  二年 & -56 & \tabularnewline\hline
  三年 & -55 & \tabularnewline\hline
  四年 & -54 & \tabularnewline
  \bottomrule
\end{longtable}

\subsection{甘露}

\begin{longtable}{|>{\centering\scriptsize}m{2em}|>{\centering\scriptsize}m{1.3em}|>{\centering}m{8.8em}|}
  % \caption{秦王政}\
  \toprule
  \SimHei \normalsize 年数 & \SimHei \scriptsize 公元 & \SimHei 大事件 \tabularnewline
  % \midrule
  \endfirsthead
  \toprule
  \SimHei \normalsize 年数 & \SimHei \scriptsize 公元 & \SimHei 大事件 \tabularnewline
  \midrule
  \endhead
  \midrule
  元年 & -53 & \tabularnewline\hline
  二年 & -52 & \tabularnewline\hline
  三年 & -51 & \tabularnewline\hline
  四年 & -50 & \tabularnewline
  \bottomrule
\end{longtable}


\subsection{黄龙}

\begin{longtable}{|>{\centering\scriptsize}m{2em}|>{\centering\scriptsize}m{1.3em}|>{\centering}m{8.8em}|}
  % \caption{秦王政}\
  \toprule
  \SimHei \normalsize 年数 & \SimHei \scriptsize 公元 & \SimHei 大事件 \tabularnewline
  % \midrule
  \endfirsthead
  \toprule
  \SimHei \normalsize 年数 & \SimHei \scriptsize 公元 & \SimHei 大事件 \tabularnewline
  \midrule
  \endhead
  \midrule
  元年 & -49 & \tabularnewline
  \bottomrule
\end{longtable}


%%% Local Variables:
%%% mode: latex
%%% TeX-engine: xetex
%%% TeX-master: "../Main"
%%% End:

%% -*- coding: utf-8 -*-
%% Time-stamp: <Chen Wang: 2018-07-10 19:07:56>

\section{元帝\tiny(BC48-BC33)}

\subsection{初元}

\begin{longtable}{|>{\centering\scriptsize}m{2em}|>{\centering\scriptsize}m{1.3em}|>{\centering}m{8.8em}|}
  % \caption{秦王政}\
  \toprule
  \SimHei \normalsize 年数 & \SimHei \scriptsize 公元 & \SimHei 大事件 \tabularnewline
  % \midrule
  \endfirsthead
  \toprule
  \SimHei \normalsize 年数 & \SimHei \scriptsize 公元 & \SimHei 大事件 \tabularnewline
  \midrule
  \endhead
  \midrule
  元年 & -48 & \tabularnewline\hline
  二年 & -47 & \tabularnewline\hline
  三年 & -46 & \tabularnewline\hline
  四年 & -45 & \tabularnewline\hline
  五年 & -44 & \tabularnewline
  \bottomrule
\end{longtable}


\subsection{永光}

\begin{longtable}{|>{\centering\scriptsize}m{2em}|>{\centering\scriptsize}m{1.3em}|>{\centering}m{8.8em}|}
  % \caption{秦王政}\
  \toprule
  \SimHei \normalsize 年数 & \SimHei \scriptsize 公元 & \SimHei 大事件 \tabularnewline
  % \midrule
  \endfirsthead
  \toprule
  \SimHei \normalsize 年数 & \SimHei \scriptsize 公元 & \SimHei 大事件 \tabularnewline
  \midrule
  \endhead
  \midrule
  元年 & -43 & \tabularnewline\hline
  二年 & -42 & \tabularnewline\hline
  三年 & -41 & \tabularnewline\hline
  四年 & -40 & \tabularnewline\hline
  五年 & -39 & \tabularnewline
  \bottomrule
\end{longtable}


\subsection{建昭}

\begin{longtable}{|>{\centering\scriptsize}m{2em}|>{\centering\scriptsize}m{1.3em}|>{\centering}m{8.8em}|}
  % \caption{秦王政}\
  \toprule
  \SimHei \normalsize 年数 & \SimHei \scriptsize 公元 & \SimHei 大事件 \tabularnewline
  % \midrule
  \endfirsthead
  \toprule
  \SimHei \normalsize 年数 & \SimHei \scriptsize 公元 & \SimHei 大事件 \tabularnewline
  \midrule
  \endhead
  \midrule
  元年 & -38 & \tabularnewline\hline
  二年 & -37 & \tabularnewline\hline
  三年 & -36 & \tabularnewline\hline
  四年 & -35 & \tabularnewline\hline
  五年 & -34 & \tabularnewline
  \bottomrule
\end{longtable}

\subsection{竟宁}

\begin{longtable}{|>{\centering\scriptsize}m{2em}|>{\centering\scriptsize}m{1.3em}|>{\centering}m{8.8em}|}
  % \caption{秦王政}\
  \toprule
  \SimHei \normalsize 年数 & \SimHei \scriptsize 公元 & \SimHei 大事件 \tabularnewline
  % \midrule
  \endfirsthead
  \toprule
  \SimHei \normalsize 年数 & \SimHei \scriptsize 公元 & \SimHei 大事件 \tabularnewline
  \midrule
  \endhead
  \midrule
  元年 & -33 & \tabularnewline
  \bottomrule
\end{longtable}


%%% Local Variables:
%%% mode: latex
%%% TeX-engine: xetex
%%% TeX-master: "../Main"
%%% End:

%% -*- coding: utf-8 -*-
%% Time-stamp: <Chen Wang: 2018-07-10 19:14:14>

\section{成帝\tiny(BC33-BC7)}

\subsection{建始}

\begin{longtable}{|>{\centering\scriptsize}m{2em}|>{\centering\scriptsize}m{1.3em}|>{\centering}m{8.8em}|}
  % \caption{秦王政}\
  \toprule
  \SimHei \normalsize 年数 & \SimHei \scriptsize 公元 & \SimHei 大事件 \tabularnewline
  % \midrule
  \endfirsthead
  \toprule
  \SimHei \normalsize 年数 & \SimHei \scriptsize 公元 & \SimHei 大事件 \tabularnewline
  \midrule
  \endhead
  \midrule
  元年 & -32 & \tabularnewline\hline
  二年 & -31 & \tabularnewline\hline
  三年 & -30 & \tabularnewline\hline
  四年 & -29 & \tabularnewline
  \bottomrule
\end{longtable}


\subsection{河平}

\begin{longtable}{|>{\centering\scriptsize}m{2em}|>{\centering\scriptsize}m{1.3em}|>{\centering}m{8.8em}|}
  % \caption{秦王政}\
  \toprule
  \SimHei \normalsize 年数 & \SimHei \scriptsize 公元 & \SimHei 大事件 \tabularnewline
  % \midrule
  \endfirsthead
  \toprule
  \SimHei \normalsize 年数 & \SimHei \scriptsize 公元 & \SimHei 大事件 \tabularnewline
  \midrule
  \endhead
  \midrule
  元年 & -28 & \tabularnewline\hline
  二年 & -27 & \tabularnewline\hline
  三年 & -26 & \tabularnewline\hline
  四年 & -25 & \tabularnewline
  \bottomrule
\end{longtable}


\subsection{阳朔}

\begin{longtable}{|>{\centering\scriptsize}m{2em}|>{\centering\scriptsize}m{1.3em}|>{\centering}m{8.8em}|}
  % \caption{秦王政}\
  \toprule
  \SimHei \normalsize 年数 & \SimHei \scriptsize 公元 & \SimHei 大事件 \tabularnewline
  % \midrule
  \endfirsthead
  \toprule
  \SimHei \normalsize 年数 & \SimHei \scriptsize 公元 & \SimHei 大事件 \tabularnewline
  \midrule
  \endhead
  \midrule
  元年 & -24 & \tabularnewline\hline
  二年 & -23 & \tabularnewline\hline
  三年 & -22 & \tabularnewline\hline
  四年 & -21 & \tabularnewline
  \bottomrule
\end{longtable}


\subsection{鸿嘉}

\begin{longtable}{|>{\centering\scriptsize}m{2em}|>{\centering\scriptsize}m{1.3em}|>{\centering}m{8.8em}|}
  % \caption{秦王政}\
  \toprule
  \SimHei \normalsize 年数 & \SimHei \scriptsize 公元 & \SimHei 大事件 \tabularnewline
  % \midrule
  \endfirsthead
  \toprule
  \SimHei \normalsize 年数 & \SimHei \scriptsize 公元 & \SimHei 大事件 \tabularnewline
  \midrule
  \endhead
  \midrule
  元年 & -20 & \tabularnewline\hline
  二年 & -19 & \tabularnewline\hline
  三年 & -18 & \tabularnewline\hline
  四年 & -17 & \tabularnewline
  \bottomrule
\end{longtable}


\subsection{永始}

\begin{longtable}{|>{\centering\scriptsize}m{2em}|>{\centering\scriptsize}m{1.3em}|>{\centering}m{8.8em}|}
  % \caption{秦王政}\
  \toprule
  \SimHei \normalsize 年数 & \SimHei \scriptsize 公元 & \SimHei 大事件 \tabularnewline
  % \midrule
  \endfirsthead
  \toprule
  \SimHei \normalsize 年数 & \SimHei \scriptsize 公元 & \SimHei 大事件 \tabularnewline
  \midrule
  \endhead
  \midrule
  元年 & -16 & \tabularnewline\hline
  二年 & -15 & \tabularnewline\hline
  三年 & -14 & \tabularnewline\hline
  四年 & -13 & \tabularnewline
  \bottomrule
\end{longtable}


\subsection{元诞}

\begin{longtable}{|>{\centering\scriptsize}m{2em}|>{\centering\scriptsize}m{1.3em}|>{\centering}m{8.8em}|}
  % \caption{秦王政}\
  \toprule
  \SimHei \normalsize 年数 & \SimHei \scriptsize 公元 & \SimHei 大事件 \tabularnewline
  % \midrule
  \endfirsthead
  \toprule
  \SimHei \normalsize 年数 & \SimHei \scriptsize 公元 & \SimHei 大事件 \tabularnewline
  \midrule
  \endhead
  \midrule
  元年 & -12 & \tabularnewline\hline
  二年 & -11 & \tabularnewline\hline
  三年 & -10 & \tabularnewline\hline
  四年 & -9 & \tabularnewline
  \bottomrule
\end{longtable}

\subsection{绥和}

\begin{longtable}{|>{\centering\scriptsize}m{2em}|>{\centering\scriptsize}m{1.3em}|>{\centering}m{8.8em}|}
  % \caption{秦王政}\
  \toprule
  \SimHei \normalsize 年数 & \SimHei \scriptsize 公元 & \SimHei 大事件 \tabularnewline
  % \midrule
  \endfirsthead
  \toprule
  \SimHei \normalsize 年数 & \SimHei \scriptsize 公元 & \SimHei 大事件 \tabularnewline
  \midrule
  \endhead
  \midrule
  元年 & -8 & \tabularnewline\hline
  二年 & -7 & \tabularnewline
  \bottomrule
\end{longtable}


%%% Local Variables:
%%% mode: latex
%%% TeX-engine: xetex
%%% TeX-master: "../Main"
%%% End:

%% -*- coding: utf-8 -*-
%% Time-stamp: <Chen Wang: 2018-07-10 19:18:50>

\section{哀帝\tiny(BC7-BC1)}

\subsection{建平}

\begin{longtable}{|>{\centering\scriptsize}m{2em}|>{\centering\scriptsize}m{1.3em}|>{\centering}m{8.8em}|}
  % \caption{秦王政}\
  \toprule
  \SimHei \normalsize 年数 & \SimHei \scriptsize 公元 & \SimHei 大事件 \tabularnewline
  % \midrule
  \endfirsthead
  \toprule
  \SimHei \normalsize 年数 & \SimHei \scriptsize 公元 & \SimHei 大事件 \tabularnewline
  \midrule
  \endhead
  \midrule
  元年 & -6 & \tabularnewline\hline
  二年 & -5 & \tabularnewline\hline
  太初\\元将 & -5 & \tabularnewline\hline
  三年 & -4 & \tabularnewline\hline
  四年 & -3 & \tabularnewline
  \bottomrule
\end{longtable}


\subsection{元寿}

\begin{longtable}{|>{\centering\scriptsize}m{2em}|>{\centering\scriptsize}m{1.3em}|>{\centering}m{8.8em}|}
  % \caption{秦王政}\
  \toprule
  \SimHei \normalsize 年数 & \SimHei \scriptsize 公元 & \SimHei 大事件 \tabularnewline
  % \midrule
  \endfirsthead
  \toprule
  \SimHei \normalsize 年数 & \SimHei \scriptsize 公元 & \SimHei 大事件 \tabularnewline
  \midrule
  \endhead
  \midrule
  元年 & -2 & \tabularnewline\hline
  二年 & -1 & \tabularnewline
  \bottomrule
\end{longtable}


%%% Local Variables:
%%% mode: latex
%%% TeX-engine: xetex
%%% TeX-master: "../Main"
%%% End:

%% -*- coding: utf-8 -*-
%% Time-stamp: <Chen Wang: 2018-07-10 19:28:12>

\section{平帝\tiny(1-5)}

\subsection{元始}

\begin{longtable}{|>{\centering\scriptsize}m{2em}|>{\centering\scriptsize}m{1.3em}|>{\centering}m{8.8em}|}
  % \caption{秦王政}\
  \toprule
  \SimHei \normalsize 年数 & \SimHei \scriptsize 公元 & \SimHei 大事件 \tabularnewline
  % \midrule
  \endfirsthead
  \toprule
  \SimHei \normalsize 年数 & \SimHei \scriptsize 公元 & \SimHei 大事件 \tabularnewline
  \midrule
  \endhead
  \midrule
  元年 & 1 & \tabularnewline\hline
  二年 & 2 & \tabularnewline\hline
  三年 & 3 & \tabularnewline\hline
  四年 & 4 & \tabularnewline\hline
  五年 & 5 & \tabularnewline
  \bottomrule
\end{longtable}


%%% Local Variables:
%%% mode: latex
%%% TeX-engine: xetex
%%% TeX-master: "../Main"
%%% End:

%% -*- coding: utf-8 -*-
%% Time-stamp: <Chen Wang: 2018-07-10 19:27:36>

\section{刘婴\tiny(6-8)}

\subsection{居摄}

\begin{longtable}{|>{\centering\scriptsize}m{2em}|>{\centering\scriptsize}m{1.3em}|>{\centering}m{8.8em}|}
  % \caption{秦王政}\
  \toprule
  \SimHei \normalsize 年数 & \SimHei \scriptsize 公元 & \SimHei 大事件 \tabularnewline
  % \midrule
  \endfirsthead
  \toprule
  \SimHei \normalsize 年数 & \SimHei \scriptsize 公元 & \SimHei 大事件 \tabularnewline
  \midrule
  \endhead
  \midrule
  元年 & 6 & \tabularnewline\hline
  二年 & 7 & \tabularnewline\hline
  三年\\初始 & 8 & \tabularnewline
  \bottomrule
\end{longtable}


%%% Local Variables:
%%% mode: latex
%%% TeX-engine: xetex
%%% TeX-master: "../Main"
%%% End:

%% -*- coding: utf-8 -*-
%% Time-stamp: <Chen Wang: 2018-07-10 19:37:48>

\section{新莽\tiny(9-23)}

\subsection{始建国}

\begin{longtable}{|>{\centering\scriptsize}m{2em}|>{\centering\scriptsize}m{1.3em}|>{\centering}m{8.8em}|}
  % \caption{秦王政}\
  \toprule
  \SimHei \normalsize 年数 & \SimHei \scriptsize 公元 & \SimHei 大事件 \tabularnewline
  % \midrule
  \endfirsthead
  \toprule
  \SimHei \normalsize 年数 & \SimHei \scriptsize 公元 & \SimHei 大事件 \tabularnewline
  \midrule
  \endhead
  \midrule
  元年 & 9 & \tabularnewline\hline
  二年 & 10 & \tabularnewline\hline
  三年 & 11 & \tabularnewline\hline
  四年 & 12 & \tabularnewline\hline
  五年 & 13 & \tabularnewline
  \bottomrule
\end{longtable}

\subsection{天凤}

\begin{longtable}{|>{\centering\scriptsize}m{2em}|>{\centering\scriptsize}m{1.3em}|>{\centering}m{8.8em}|}
  % \caption{秦王政}\
  \toprule
  \SimHei \normalsize 年数 & \SimHei \scriptsize 公元 & \SimHei 大事件 \tabularnewline
  % \midrule
  \endfirsthead
  \toprule
  \SimHei \normalsize 年数 & \SimHei \scriptsize 公元 & \SimHei 大事件 \tabularnewline
  \midrule
  \endhead
  \midrule
  元年 & 14 & \tabularnewline\hline
  二年 & 15 & \tabularnewline\hline
  三年 & 16 & \tabularnewline\hline
  四年 & 17 & \tabularnewline\hline
  五年 & 18 & \tabularnewline\hline
  六年 & 19 & \tabularnewline
  \bottomrule
\end{longtable}

\subsection{地皇}

\begin{longtable}{|>{\centering\scriptsize}m{2em}|>{\centering\scriptsize}m{1.3em}|>{\centering}m{8.8em}|}
  % \caption{秦王政}\
  \toprule
  \SimHei \normalsize 年数 & \SimHei \scriptsize 公元 & \SimHei 大事件 \tabularnewline
  % \midrule
  \endfirsthead
  \toprule
  \SimHei \normalsize 年数 & \SimHei \scriptsize 公元 & \SimHei 大事件 \tabularnewline
  \midrule
  \endhead
  \midrule
  元年 & 20 & \tabularnewline\hline
  二年 & 21 & \tabularnewline\hline
  三年 & 22 & \tabularnewline\hline
  四年 & 23 & \tabularnewline
  \bottomrule
\end{longtable}


%%% Local Variables:
%%% mode: latex
%%% TeX-engine: xetex
%%% TeX-master: "../Main"
%%% End:

%% -*- coding: utf-8 -*-
%% Time-stamp: <Chen Wang: 2018-07-10 19:42:25>

\section{玄汉\tiny(23-25)}

\subsection{更始}

\begin{longtable}{|>{\centering\scriptsize}m{2em}|>{\centering\scriptsize}m{1.3em}|>{\centering}m{8.8em}|}
  % \caption{秦王政}\
  \toprule
  \SimHei \normalsize 年数 & \SimHei \scriptsize 公元 & \SimHei 大事件 \tabularnewline
  % \midrule
  \endfirsthead
  \toprule
  \SimHei \normalsize 年数 & \SimHei \scriptsize 公元 & \SimHei 大事件 \tabularnewline
  \midrule
  \endhead
  \midrule
  元年 & 23 & \tabularnewline\hline
  二年 & 24 & \tabularnewline\hline
  三年 & 25 & \tabularnewline
  \bottomrule
\end{longtable}


%%% Local Variables:
%%% mode: latex
%%% TeX-engine: xetex
%%% TeX-master: "../Main"
%%% End:


%%% Local Variables:
%%% mode: latex
%%% TeX-engine: xetex
%%% TeX-master: "../Main"
%%% End:
 % 西汉
% %% -*- coding: utf-8 -*-
%% Time-stamp: <Chen Wang: 2019-10-22 11:36:38>

\chapter{东汉\tiny(25-220)}

%% -*- coding: utf-8 -*-
%% Time-stamp: <Chen Wang: 2018-07-10 19:49:27>

\section{小政权}

\subsection{汉复\tiny(23-34)}

\begin{longtable}{|>{\centering\scriptsize}m{2em}|>{\centering\scriptsize}m{1.3em}|>{\centering}m{8.8em}|}
  % \caption{秦王政}\
  \toprule
  \SimHei \normalsize 年数 & \SimHei \scriptsize 公元 & \SimHei 大事件 \tabularnewline
  % \midrule
  \endfirsthead
  \toprule
  \SimHei \normalsize 年数 & \SimHei \scriptsize 公元 & \SimHei 大事件 \tabularnewline
  \midrule
  \endhead
  \midrule
  元年 & 23 & \tabularnewline\hline
  二年 & 24 & \tabularnewline\hline
  三年 & 25 & \tabularnewline\hline
  四年 & 26 & \tabularnewline\hline
  五年 & 27 & \tabularnewline\hline
  六年 & 28 & \tabularnewline\hline
  七年 & 29 & \tabularnewline\hline
  八年 & 30 & \tabularnewline\hline
  九年 & 31 & \tabularnewline\hline
  十年 & 32 & \tabularnewline\hline
  十一年 & 33 & \tabularnewline\hline
  十二年 & 34 & \tabularnewline
  \bottomrule
\end{longtable}

\subsection{龙兴\tiny(25-36)}

\begin{longtable}{|>{\centering\scriptsize}m{2em}|>{\centering\scriptsize}m{1.3em}|>{\centering}m{8.8em}|}
  % \caption{秦王政}\
  \toprule
  \SimHei \normalsize 年数 & \SimHei \scriptsize 公元 & \SimHei 大事件 \tabularnewline
  % \midrule
  \endfirsthead
  \toprule
  \SimHei \normalsize 年数 & \SimHei \scriptsize 公元 & \SimHei 大事件 \tabularnewline
  \midrule
  \endhead
  \midrule
  元年 & 25 & \tabularnewline\hline
  二年 & 26 & \tabularnewline\hline
  三年 & 27 & \tabularnewline\hline
  四年 & 28 & \tabularnewline\hline
  五年 & 29 & \tabularnewline\hline
  六年 & 30 & \tabularnewline\hline
  七年 & 31 & \tabularnewline\hline
  八年 & 32 & \tabularnewline\hline
  九年 & 33 & \tabularnewline\hline
  十年 & 34 & \tabularnewline\hline
  十一年 & 35 & \tabularnewline\hline
  十二年 & 36 & \tabularnewline
  \bottomrule
\end{longtable}

\subsection{建世\tiny(25-27)}

\begin{longtable}{|>{\centering\scriptsize}m{2em}|>{\centering\scriptsize}m{1.3em}|>{\centering}m{8.8em}|}
  % \caption{秦王政}\
  \toprule
  \SimHei \normalsize 年数 & \SimHei \scriptsize 公元 & \SimHei 大事件 \tabularnewline
  % \midrule
  \endfirsthead
  \toprule
  \SimHei \normalsize 年数 & \SimHei \scriptsize 公元 & \SimHei 大事件 \tabularnewline
  \midrule
  \endhead
  \midrule
  元年 & 25 & \tabularnewline\hline
  二年 & 26 & \tabularnewline\hline
  三年 & 27 & \tabularnewline
  \bottomrule
\end{longtable}


%%% Local Variables:
%%% mode: latex
%%% TeX-engine: xetex
%%% TeX-master: "../Main"
%%% End:

%% -*- coding: utf-8 -*-
%% Time-stamp: <Chen Wang: 2021-11-01 11:24:58>

\section{光武帝劉秀\tiny(25-57)}

\subsection{生平}

漢光武帝劉秀(前5年1月15日-57年3月29日),字文叔,小名呼 南陽郡蔡陽縣人[a](今湖北省襄阳枣阳市),祖籍徐州沛縣,東漢第一位皇帝,25年8月5日-57年3月29日在位。廟號世祖,諡號光武皇帝。

劉秀為漢高帝九世孫,漢景帝七世孫,长沙定王刘发之后,出身於南陽郡的地方豪族。新朝末年國家動蕩,各地寇盜蜂起。地皇三年(22年),劉秀與其兄長劉縯在宛(今河南省南陽市)起兵。25年,在鄗縣(今河北省石家莊市高邑縣)登基稱帝,改元建武,國號為「漢」,史稱東漢。此後,劉秀逐步掃平各方勢力,最終統一中國。劉秀在位三十二年,社會逐漸從新朝末年的動蕩中恢復,故稱「光武中興」。建武中元二年(57年),劉秀逝世於雒陽。

劉秀的軍事才能很高。稱帝之後遣眾將攻伐四方,往往能從前方上報的排兵布陣形勢中發現問題,有時因前方不能及時得到糾正,便為敵人所敗。此外,劉秀待人誠懇簡約,寬厚有信,竇融、馬援等均由此歸心。對外政策方面,引南匈奴內遷入塞,分置諸部於北地、朔方、五原、雲中、定襄、雁門、代郡、西河緣邊八郡,詔單于徙居西河美稷。但此舉也成為東漢朝廷和民眾沈重的經濟負擔,在東漢與北匈奴的戰爭中南匈奴僅起到出兵助攻的作用,談不上替東漢守衛北邊。到了東漢中期由於羌患,使得南匈奴在北邊不斷發起暴亂,對東漢北邊邊防乃至北方內地的安全構成了嚴重威脅,從而成為東漢北邊的一大邊患。

漢哀帝建平元年十二月甲子(前5年1月15日)夜於陳留郡濟陽縣出生。刘秀出生的时候,有赤光照耀整個房間,當年稻禾(嘉禾)一茎九穗,因此得名秀。

刘秀是汉高帝刘邦九世孙,西汉景帝子长沙定王刘发之子舂陵節侯劉買的玄孙,與更始帝有同一位高祖父劉買。其父为南顿令劉欽,母樊娴都。世代居住在南阳郡蔡陽(今湖北省枣阳市西南),屬地方豪族。刘秀九岁时,父亲逝世,便由叔父劉良抚养。由于刘秀勤于农事,而兄劉縯好侠养士,经常取笑刘秀,将他比做刘邦的兄弟刘喜。新朝天凤年间(14年—19年),刘秀至长安,学习《尚书》,略通大义。成年後劉秀身高七尺三寸(身高175厘米以上)。

刘秀在新野县时,听闻陰麗華的美貌,心悦之。後至長安,見執金吾車騎甚盛,因歎曰:“仕宦當作執金吾,娶妻當得陰麗華。”

時值新莽天鳳五年(17年),天下大亂,赤眉軍與綠林軍各自起兵反王莽。地皇三年(22年),劉秀避吏於新野,因賣穀而至宛(今河南省南陽市),經李通勸說在宛起兵。地皇四年(23年)二月,劉縯、劉秀兄弟與綠林兵共同擁護劉玄稱帝,國號仍為漢,改元更始,史稱更始帝。同年劉秀率綠林軍1萬以少勝多於昆陽滅王莽軍42萬,殺其主帥王尋,史稱昆陽之戰。

此後劉縯、劉秀兄弟威望大盛,遭到劉玄的猜忌。劉秀有所察覺,但劉縯不以為意,終被劉玄藉故殺死,同被殺死的還有同宗劉稷。此時劉秀也處於危險之中,只得向劉玄謝罪,並不敢為哥哥服喪,飲食言笑如常。劉玄心有所慚,故而拜劉秀為破虜大將軍、武信侯。

後更始帝劉玄攻占长安,新莽灭亡。时河北王郎起兵,于是更始帝派劉秀巡視黃河以北,劉秀始得脫離險境。劉秀遂在河北積蓄力量,日益壯大,被更始帝封為「蕭王」。劉秀率吳漢、鄧禹等手下大將,繼續在北方大破銅馬等割據勢力,被關西人號為「銅馬帝」。由於劉秀與更始帝心生二意,自此劉秀手下便不斷勸進。

更始三年(25年)六月,赤眉军立刘盆子为傀儡皇帝。同月二十二己未日(25年8月5日),劉秀于鄗城即皇帝位,改元建武,国号仍为汉,史称东汉。九月,赤眉军击败刘玄,漢更始帝投降,同年十二月被杀。

劉秀因为汉朝是火德的缘故迁都洛阳,改洛阳为「雒阳」。刘秀先后荡平赤眉、张步、隗嚣等割据势力,然割据一方的卢芳,刘秀屡次遣吴汉、杜茂往击,均不克。建武十二年(36年),卢芳进攻云中郡,留守九原的部将随育胁迫卢芳降伏刘秀。卢芳放弃军队,逃往匈奴。同年十一月十九己卯日(36年12月25日),吴汉攻克成都,割据四川的公孙述成家政权灭亡,东汉统一中国。

公元前108年,漢武帝滅朝鮮。 建武中元二年(公元57年),第一次有明文記載「倭人國家」與中國往來。九州北部(博多灣沿岸)的倭奴國接受漢王朝的策封,光武帝封其為「倭奴國王」,並授予金印。

1784年,在日本北九州地區博多灣志賀島,出土一枚刻有「漢委奴國王」五個字的金印。這一枚金印也為中日兩國最早交往的證明。

刘秀勤于政事,“每旦视朝,日仄乃罢,数引公卿郎将议论经理,夜分乃寐”。在位期间,多次发布释放奴婢和禁止残害奴婢的诏书。为减少贫民卖身为奴婢,经常发救济粮,减少租徭役,兴修水利,发展农业生产。裁并郡县,精简官员。结果,裁并四百余县,官员十置其一。历史上称其统治时期为光武中兴。其间国势昌隆,号称“建武盛世”。 刘秀统一中国后,厌武事,不言军旅,建武二十七年(51年),朗陵侯臧宫、扬虚侯马武上书:请乘匈奴分裂、北匈奴衰弱之际发兵击之,立“万世刻石之功”。光武却下诏:“今国无善政,灾变不息,人不自保,而复欲远事边外乎!……不如息民。”刘秀不同于明太祖朱元璋得天下后诛杀大批功臣只留下汤和徐達的无情,刘秀分封三百六十多位功臣为列侯,给予他们尊崇的地位,只解其兵权,刘秀诛杀功臣一说源于戏剧,令刘秀蒙受「不白之冤」。其实,在统一中国之前,他就开始削弱国防建设,废郡国兵制,罢郡国都尉。削弱地方兵权的同时,导致后来无力抵御外患,而豪强地主的部曲家兵则迅速发展,像东汉末年的董卓就是一例。刘秀以后不设丞相,而是“虽置三公”但“事归台阁”;一方面削弱三公权力,使三公成为虚位,另一方面又扩大尚书台的职权,成为皇帝发号施令的执行机构,所有权力集中于皇帝一身。”《后汉书·申屠刚传》说:“时内外群官,多帝自选举,加以法理严察,职事过苦,尚书近臣,至乃捶扑牵曳于前,群臣莫敢正言。”“自是大臣难居相任”。建武二十八年(52年)他借故搜捕王侯宾客,“坐死者数千人”,严禁结党营私。

建武中元二年二月初五日戊戌(57年3月29日),崩于雒阳南宫前殿,享壽六十二岁,在位三十二年。三月丁卯(4月27日),安葬于漢原陵(今河南孟津县铁谢村附近),庙号世祖,谥光武皇帝。刘秀駕崩后,其子汉明帝刘庄将统一战争中功劳最大的二十八人的影像画在云台阁,称云台二十八将。

《资治通鉴》称刘秀是个宽厚简易的人。在统一过程中,刘玄的一些手下曾参与谋害他的哥哥,他能够不计前嫌地招降并厚待;分封功臣时,不顾他人劝说,将最大的封地劃到了四县之广;战争尚未结束,就将原来十分之一的税率减到三十分之一;马援为隗嚣所使,分别访问公孙述和刘秀,独为刘秀的人格魅力折服;耿弇、窦融曾专制一方,以兵多权大心不自安,而刘秀对他们未有半点疑虑。凡此种种,都成为他成功的决定性因素。甚至在统一之后,他废郭皇后及太子劉彊,立阴皇后及次子刘阳(后改名莊),犹能令郭皇后到其子中山王的封国安享餘年,两子之间不生嫌隙,也没有受到臣下及后人的议论。

范曄:「雖身濟大業,競競如不及,故能明慎政體,總欖權綱,量時度力,舉無過事,退功臣而進文吏,戢弓矢而散馬牛,雖道未方古,斯亦止戈之武焉。」

诸葛亮:“光武神略计较,生于天心,故帷幄无他所思,六奇无他所出,于是以谋合议同,共成王业而已。”

明朝官修皇帝实录《明太祖实录》记载,明太祖朱元璋在洪武七年八月初一日(1374年9月7日),亲自前往南京历代帝王庙祭祀三皇、五帝、夏禹王、商汤王、周武王、汉高祖、汉光武帝、隋文帝、唐太宗、宋太祖、元世祖一共十七位帝王,其中对汉光武帝刘秀的祝文是:“惟汉光武皇帝延揽英雄,励精图治,载兴炎运,四海咸安。有君天下之德而安万世之功者也。元璋以菲德荷天佑人助,君临天下,继承中国帝王正统,伏念列圣去世已远,神灵在天,万古长存,崇报之礼,多未举行,故于祭祀有阙。是用肇新庙宇于京师,列序圣像及历代开基帝王,每岁祀以春、秋仲月,永为常典。今礼奠之初,谨奉牲醴、庶品致祭,伏惟神鉴。尚享!”

王夫之《读通鉴论》:“光武之得天下,较高帝而尤难矣。光武之神武不可测也!三代而下,取天下者,唯光武独焉!”“自三代而下,唯光武允冠百王矣”。

光武帝承袭西汉后期法律宽松的弊病,又过于剝奪三公的職權,明章之治以后皇帝幼小,陷入了长期的戚宦之爭之黑暗和混乱。

刘秀迷信图谶,與不信谶的大臣發生衝突,且时而感情用事,处事不公,韩歆因直谏被逼死,劉秀又包庇湖阳公主险些杀死董宣。

刘秀縱容部下吴汉军对邓奉的家乡进行劫掠,导致邓奉反叛,后来邓奉兵败投降被杀。平狄将军庞萌与盖延共击董宪,而诏书却只下达给盖延、不给庞萌,庞萌以为盖延说自己坏话,起疑,反叛,后来庞萌兵败被杀。

\subsection{建武}

\begin{longtable}{|>{\centering\scriptsize}m{2em}|>{\centering\scriptsize}m{1.3em}|>{\centering}m{8.8em}|}
  % \caption{秦王政}\
  \toprule
  \SimHei \normalsize 年数 & \SimHei \scriptsize 公元 & \SimHei 大事件 \tabularnewline
  % \midrule
  \endfirsthead
  \toprule
  \SimHei \normalsize 年数 & \SimHei \scriptsize 公元 & \SimHei 大事件 \tabularnewline
  \midrule
  \endhead
  \midrule
  元年 & 25 & \tabularnewline\hline
  二年 & 26 & \tabularnewline\hline
  三年 & 27 & \tabularnewline\hline
  四年 & 28 & \tabularnewline\hline
  五年 & 29 & \tabularnewline\hline
  六年 & 30 & \tabularnewline\hline
  七年 & 31 & \tabularnewline\hline
  八年 & 32 & \tabularnewline\hline
  九年 & 33 & \tabularnewline\hline
  十年 & 34 & \tabularnewline\hline
  十一年 & 35 & \tabularnewline\hline
  十二年 & 36 & \tabularnewline\hline
  十三年 & 37 & \tabularnewline\hline
  十四年 & 38 & \tabularnewline\hline
  十五年 & 39 & \tabularnewline\hline
  十六年 & 40 & \tabularnewline\hline
  十七年 & 41 & \tabularnewline\hline
  十八年 & 42 & \tabularnewline\hline
  十九年 & 43 & \tabularnewline\hline
  二十年 & 44 & \tabularnewline\hline
  二一年 & 45 & \tabularnewline\hline
  二二年 & 46 & \tabularnewline\hline
  二三年 & 47 & \tabularnewline\hline
  二四年 & 48 & \tabularnewline\hline
  二五年 & 49 & \tabularnewline\hline
  二六年 & 50 & \tabularnewline\hline
  二七年 & 51 & \tabularnewline\hline
  二八年 & 52 & \tabularnewline\hline
  二九年 & 53 & \tabularnewline\hline
  三十年 & 54 & \tabularnewline\hline
  三一年 & 55 & \tabularnewline\hline
  三二年 & 56 & \tabularnewline
  \bottomrule
\end{longtable}

\subsection{建武中元}

\begin{longtable}{|>{\centering\scriptsize}m{2em}|>{\centering\scriptsize}m{1.3em}|>{\centering}m{8.8em}|}
  % \caption{秦王政}\
  \toprule
  \SimHei \normalsize 年数 & \SimHei \scriptsize 公元 & \SimHei 大事件 \tabularnewline
  % \midrule
  \endfirsthead
  \toprule
  \SimHei \normalsize 年数 & \SimHei \scriptsize 公元 & \SimHei 大事件 \tabularnewline
  \midrule
  \endhead
  \midrule
  元年 & 56 & \tabularnewline\hline
  二年 & 57 & \tabularnewline
  \bottomrule
\end{longtable}


%%% Local Variables:
%%% mode: latex
%%% TeX-engine: xetex
%%% TeX-master: "../Main"
%%% End:

%% -*- coding: utf-8 -*-
%% Time-stamp: <Chen Wang: 2021-11-01 11:25:11>

\section{明帝劉莊\tiny(57-75)}

\subsection{生平}

漢明帝劉莊(28年6月15日-75年9月5日),原名刘阳,字子丽,东汉第二位皇帝,在位十八年。其正式諡號為「孝明皇帝」,後世省略「孝」字稱「漢明帝」,庙号显宗。汉光武帝刘秀的第四子,母亲为光烈皇后阴丽华。

汉明帝生于建武四年五月甲申(28年6月15日)。他从小就聪明好学,十岁时能够通读《春秋》。

建武十五年(39年)封东海公,十七年(41年)进爵为东海王,十九年(43年)被立为皇太子。建武中元二年初五戊戌(57年3月29日),三十岁的刘庄即皇帝位。

明帝即位后,一切遵奉汉光武帝的制度。明帝热心提倡儒学,注重刑名文法,为政苛察,总揽权柄,权不借下。他严令后妃之家不得封侯与政,对贵戚功臣也多方防范。同时,基本上消除了因为王莽虐政而引起的周边蛮夷侵扰的威胁,使汉跟周边蛮夷的友好关系得到了恢复和发展。

明帝允北匈奴互市之请,但并未消弥北匈奴的寇掠,反而动摇了早已归附的南匈奴。只得改变光武时期息兵养民的策略,重新对匈奴开战。永平十六年(73年),命祭肜、窦固、耿秉、來苗征伐北匈奴,汉军进抵天山,击呼衍王,斩首千余级,追至蒲类海(今新疆巴里坤湖),取伊吾卢地。永平十七年(74年),命窦固、耿秉、劉張征白山虜於蒲類海,复置西域都护府,用来管辖西域地区。其后,窦固又以班超出使西域,由是西域诸国皆遣子入侍。自新朝地皇四年(23年)以来,西域与中原断绝关系50年后又恢复了正常交往。班超以三十六人征服鄯善、于寘诸国、耿恭守疏勒城力拒匈奴等故事都发生在这一时期。

此外,随着对外交往的正常发展,佛教已在西汉末年传入西域,永平十年(67年),明帝梦见金人,其名曰佛,于是派使者赴天竺求得其书及沙门,并于雒阳建立中国第一座佛教庙宇白马寺。

明帝之世,吏治非常清明,境内安定。加以多次下诏招抚流民,以郡国公田赐贫人、贷种食,并兴修水利。因此,史书记载当时民安其业,户口滋殖。据《后汉书》记载:光武帝建武中元二年(57年),人口为2100万,至汉明帝永平十八年(75年),在不到20年的时间里增加至3412万。明帝以及随后的章帝在位时期,史称“明章之治”。

永平十八年八月初六壬子(75年9月5日),汉明帝逝世于雒阳东宫前殿,终年四十八岁。八月壬戌(9月15日),葬于显节陵(今河南洛阳市东南)。庙号显宗,谥号孝明皇帝。

\subsection{永平}

\begin{longtable}{|>{\centering\scriptsize}m{2em}|>{\centering\scriptsize}m{1.3em}|>{\centering}m{8.8em}|}
  % \caption{秦王政}\
  \toprule
  \SimHei \normalsize 年数 & \SimHei \scriptsize 公元 & \SimHei 大事件 \tabularnewline
  % \midrule
  \endfirsthead
  \toprule
  \SimHei \normalsize 年数 & \SimHei \scriptsize 公元 & \SimHei 大事件 \tabularnewline
  \midrule
  \endhead
  \midrule
  元年 & 58 & \tabularnewline\hline
  二年 & 59 & \tabularnewline\hline
  三年 & 60 & \tabularnewline\hline
  四年 & 61 & \tabularnewline\hline
  五年 & 62 & \tabularnewline\hline
  六年 & 63 & \tabularnewline\hline
  七年 & 64 & \tabularnewline\hline
  八年 & 65 & \tabularnewline\hline
  九年 & 66 & \tabularnewline\hline
  十年 & 67 & \tabularnewline\hline
  十一年 & 68 & \tabularnewline\hline
  十二年 & 69 & \tabularnewline\hline
  十三年 & 70 & \tabularnewline\hline
  十四年 & 71 & \tabularnewline\hline
  十五年 & 72 & \tabularnewline\hline
  十六年 & 73 & \tabularnewline\hline
  十七年 & 74 & \tabularnewline\hline
  十八年 & 75 & \tabularnewline
  \bottomrule
\end{longtable}


%%% Local Variables:
%%% mode: latex
%%% TeX-engine: xetex
%%% TeX-master: "../Main"
%%% End:

%% -*- coding: utf-8 -*-
%% Time-stamp: <Chen Wang: 2018-07-10 20:00:59>

\section{章帝\tiny(75-88)}

\subsection{建初}

\begin{longtable}{|>{\centering\scriptsize}m{2em}|>{\centering\scriptsize}m{1.3em}|>{\centering}m{8.8em}|}
  % \caption{秦王政}\
  \toprule
  \SimHei \normalsize 年数 & \SimHei \scriptsize 公元 & \SimHei 大事件 \tabularnewline
  % \midrule
  \endfirsthead
  \toprule
  \SimHei \normalsize 年数 & \SimHei \scriptsize 公元 & \SimHei 大事件 \tabularnewline
  \midrule
  \endhead
  \midrule
  元年 & 76 & \tabularnewline\hline
  二年 & 77 & \tabularnewline\hline
  三年 & 78 & \tabularnewline\hline
  四年 & 79 & \tabularnewline\hline
  五年 & 80 & \tabularnewline\hline
  六年 & 81 & \tabularnewline\hline
  七年 & 82 & \tabularnewline\hline
  八年 & 83 & \tabularnewline\hline
  九年 & 84 & \tabularnewline
  \bottomrule
\end{longtable}

\subsection{元和}

\begin{longtable}{|>{\centering\scriptsize}m{2em}|>{\centering\scriptsize}m{1.3em}|>{\centering}m{8.8em}|}
  % \caption{秦王政}\
  \toprule
  \SimHei \normalsize 年数 & \SimHei \scriptsize 公元 & \SimHei 大事件 \tabularnewline
  % \midrule
  \endfirsthead
  \toprule
  \SimHei \normalsize 年数 & \SimHei \scriptsize 公元 & \SimHei 大事件 \tabularnewline
  \midrule
  \endhead
  \midrule
  元年 & 84 & \tabularnewline\hline
  二年 & 85 & \tabularnewline\hline
  三年 & 86 & \tabularnewline\hline
  四年 & 87 & \tabularnewline
  \bottomrule
\end{longtable}

\subsection{章和}

\begin{longtable}{|>{\centering\scriptsize}m{2em}|>{\centering\scriptsize}m{1.3em}|>{\centering}m{8.8em}|}
  % \caption{秦王政}\
  \toprule
  \SimHei \normalsize 年数 & \SimHei \scriptsize 公元 & \SimHei 大事件 \tabularnewline
  % \midrule
  \endfirsthead
  \toprule
  \SimHei \normalsize 年数 & \SimHei \scriptsize 公元 & \SimHei 大事件 \tabularnewline
  \midrule
  \endhead
  \midrule
  元年 & 87 & \tabularnewline\hline
  二年 & 88 & \tabularnewline
  \bottomrule
\end{longtable}


%%% Local Variables:
%%% mode: latex
%%% TeX-engine: xetex
%%% TeX-master: "../Main"
%%% End:

%% -*- coding: utf-8 -*-
%% Time-stamp: <Chen Wang: 2021-11-01 11:28:16>

\section{和帝刘肇\tiny(88-105)}

\subsection{生平}

汉和帝刘肇(79年-106年2月13日),东汉第四位皇帝(88年4月9日—106年2月13日在位),在位17年,得年僅27岁,其正式諡號為「孝和皇帝」,後世省略「孝」字稱「漢和帝」,他是章帝第四子,母贵人梁氏,死後庙号穆宗,葬于慎陵。

建初四年(79年),梁贵人生刘肇。皇后窦氏将刘肇养为己子。建初七年(82年),汉章帝废太子刘庆,立刘肇为皇太子。

章和二年二月三十壬辰(88年4月9日),汉章帝逝世,刘肇即位,是为汉和帝。当时他只有十岁,由养母窦太后执政,窦太后排斥异己,让哥哥窦宪掌权,窦家人一犯法,窦太后就再三庇护,窦氏的专横跋扈,引起汉和帝的不满。永元四年壬辰年六月二十三日(92年8月14日),汉和帝联合宦官鄭眾将窦氏一网打尽,但也导致“于是中官始盛焉”。

在一举扫平了外戚窦氏集团的势力之后,汉和帝开始亲理政事,他每天早起临朝,深夜批阅奏章,从不荒怠政事,故有「劳谦有终」之称,但却因而积劳成疾,加上和帝本身已体弱多病,所以年仅二十七岁便英年早逝,从他亲政后的政绩,不失为一代贤君英主。和帝当政时期,曾多次下诏赈济灾民、减免赋税、安置流民、勿违农时,并多次下诏纳贤,在法制上也主张宽刑,并在西域复置西域都护。汉和帝十分体恤民众疾苦,多次诏令理冤狱,恤鳏寡,矜孤弱,薄赋敛,告诫上下官吏认真思考造成天灾人祸的自身原因。汉和帝亲政后使东汉国力达到极盛,时人称为「永元之隆」。

汉和帝在位时期,在科技、文化、军事、外交上也有不少建树,蔡伦改进了造纸术,班固修成《汉书》,窦宪击破北匈奴促使其西迁,班超平定西域,并派遣甘英出使大秦。元興元年乙巳年十二月廿二日辛未(106年2月13日),汉和帝病逝于京都洛阳的章德前殿,时年二十七岁。4月27日,葬於漢慎陵。

《后汉书》:“自中兴以后,逮于永元,虽颇有弛张,而俱存不扰,是以齐民岁增,辟土世广。偏师出塞,则漠北地空;都护西指,则通译四万。岂其道远三代,术长前世?将服叛去来,自有数也?”

《东观汉记》:“孝和皇帝,章帝中子也,上自歧嶷,至於总角,孝顺聪明,宽和仁孝,帝由是深珍之,以为宜承天位,年四岁,立为太子,初治尚书,遂兼览书传,好古乐道,无所不照,上以五经义异,书传意殊,亲幸东观,览书林,阅篇藉,朝无宠族,惠泽沾濡,外忧庶绩,内勤经艺,自左右近臣,皆诵诗书,德教在宽,仁恕并洽,是以黎元宁康,万国协和,符瑞八十馀品,帝让而不宣,故靡得而纪。”

《帝王世纪》:“孝和之嗣世,正身履道,以奉大业,宾礼耆艾,动式旧典,宫无嫔嫱郑卫之燕,囿无般乐游畋之豫,躬履至德,虚静自损,是以屡获丰年,远近承风。”

後汉苏顺和帝诔曰:“天王徂登,率土奄伤,如何昊穹,夺我圣皇,恩德累代,乃作铭章,其辞曰:恭惟大行,配天建德,陶元二化,风流万国,立我蒸民,宜此仪则,厥初生民,三五作刚,载藉之盛,著於虞唐,恭惟大行,爰同其光,自昔何为,钦明允塞,恭惟大行,天覆地载,无为而治,冠斯往代,往代崎岖,诸夏擅命,爰兹发号,民乐其政,奄有万国,民臣咸祑,大孝备矣,閟宫有侐,由昔姜嫄,祖妣之室,本枝百世,神契惟一,弥留不豫,道扬末命,劳谦有终,实惟其性,衣不制新,犀玉远屏,履和而行,威棱上古,洪泽滂流,茂化沾溥,不玦少留,民斯何怙,歔欷成云,泣涕成雨,昊天不吊,丧我慈父。”

後汉崔瑗和帝诔曰:“玄景寝曜,云物见徵,冯相考妖,遂当帝躬,三载四海,遏密八音,如丧考妣,擗踊号吟,大遂既启,乃徂玄宫,永背神器,升遐皇穹,长夜冥冥,曷云其穷。”

洪迈《容斋随笔‧卷三》:“汉昭帝年十四,能察霍光之忠,知燕王上书之诈,诛桑弘羊、上官桀,后世称其明。然和帝时,窦宪兄弟专权,太后临朝,共图杀害。帝阴知其谋,而与内外臣僚莫由亲接,独知中常侍郑众不事豪党,遂与定议诛宪,时亦年十四,其刚决不下昭帝,但范史发明不出,故后世无称焉。”

《续汉书》:“论曰:孝和年十四,能折外戚骄横之权,即昭帝毙上官之类矣。朝政遂一,民安职业,勤恤本务,苑囿希幸,远夷稽服,西域开泰,郡国言符瑞八十余品,咸惧虚妄,抑而不宣云尔。”

李贤注引《序例》曰:“凡瑞应,自和帝以上,政事多美,近於有实,故书见於某处。自安帝以下,王道衰缺,容或虚饰,故书某处上言也。”

李尤《辟雍赋》曰:“卓矣煌煌,永元之隆。含弘该要,周建大中。蓄纯和之优渥兮,化盛溢而兹丰。”

《通典》:“明章之后,天下无事,务在养民。至於孝和,人户滋殖。”

叶适《习学记言序目》:“东汉至孝和八十年间,上无败政,天下乂安。”

\subsection{永元}

\begin{longtable}{|>{\centering\scriptsize}m{2em}|>{\centering\scriptsize}m{1.3em}|>{\centering}m{8.8em}|}
  % \caption{秦王政}\
  \toprule
  \SimHei \normalsize 年数 & \SimHei \scriptsize 公元 & \SimHei 大事件 \tabularnewline
  % \midrule
  \endfirsthead
  \toprule
  \SimHei \normalsize 年数 & \SimHei \scriptsize 公元 & \SimHei 大事件 \tabularnewline
  \midrule
  \endhead
  \midrule
  元年 & 89 & \tabularnewline\hline
  二年 & 90 & \tabularnewline\hline
  三年 & 91 & \tabularnewline\hline
  四年 & 92 & \tabularnewline\hline
  五年 & 93 & \tabularnewline\hline
  六年 & 94 & \tabularnewline\hline
  七年 & 95 & \tabularnewline\hline
  八年 & 96 & \tabularnewline\hline
  九年 & 97 & \tabularnewline\hline
  十年 & 98 & \tabularnewline\hline
  十一年 & 99 & \tabularnewline\hline
  十二年 & 100 & \tabularnewline\hline
  十三年 & 101 & \tabularnewline\hline
  十四年 & 102 & \tabularnewline\hline
  十五年 & 103 & \tabularnewline\hline
  十六年 & 104 & \tabularnewline\hline
  十七年 & 105 & \tabularnewline
  \bottomrule
\end{longtable}

\subsection{元兴}

\begin{longtable}{|>{\centering\scriptsize}m{2em}|>{\centering\scriptsize}m{1.3em}|>{\centering}m{8.8em}|}
  % \caption{秦王政}\
  \toprule
  \SimHei \normalsize 年数 & \SimHei \scriptsize 公元 & \SimHei 大事件 \tabularnewline
  % \midrule
  \endfirsthead
  \toprule
  \SimHei \normalsize 年数 & \SimHei \scriptsize 公元 & \SimHei 大事件 \tabularnewline
  \midrule
  \endhead
  \midrule
  元年 & 105 & \tabularnewline
  \bottomrule
\end{longtable}


%%% Local Variables:
%%% mode: latex
%%% TeX-engine: xetex
%%% TeX-master: "../Main"
%%% End:

%% -*- coding: utf-8 -*-
%% Time-stamp: <Chen Wang: 2019-12-17 17:23:12>

\section{殇帝\tiny(106)}

\subsection{生平}

汉殇帝刘隆(105年10月或11月-106年9月21日),汉和帝幼子,养于民间,东汉第五位皇帝(106年在位),其正式諡號為「孝殤皇帝」,後世省略「孝」字稱「漢殤帝」。汉殇帝是即位年龄最小、寿命最短的中国皇帝。

和帝在世的时候,生了许多皇子,大都夭折。和帝以为宦官、外戚在谋害他的儿子,便将剩余的皇子留在民间扶养。元興元年乙巳年十二月廿二日辛未(106年2月13日),汉和帝死,邓皇后因长子刘胜有絕症,将刘隆迎回皇宫做皇帝,將刘胜封为平原王。刘隆登基时候才出生100余天,改元“延平”。朝政由外戚邓騭掌权。

仍在襁褓之中的汉殇帝,于延平元年八月辛亥(西元106年9月21日)得了场大病后驾崩,在位只有短短八個月。

刘隆年幼,邓太后以女主临政,期间政事多委以宦官。自汉明帝至延平年间,宦官的人数逐渐增多,中常侍达到十名,小黄门有二十名。

\subsection{延平}

\begin{longtable}{|>{\centering\scriptsize}m{2em}|>{\centering\scriptsize}m{1.3em}|>{\centering}m{8.8em}|}
  % \caption{秦王政}\
  \toprule
  \SimHei \normalsize 年数 & \SimHei \scriptsize 公元 & \SimHei 大事件 \tabularnewline
  % \midrule
  \endfirsthead
  \toprule
  \SimHei \normalsize 年数 & \SimHei \scriptsize 公元 & \SimHei 大事件 \tabularnewline
  \midrule
  \endhead
  \midrule
  元年 & 106 & \tabularnewline
  \bottomrule
\end{longtable}


%%% Local Variables:
%%% mode: latex
%%% TeX-engine: xetex
%%% TeX-master: "../Main"
%%% End:

%% -*- coding: utf-8 -*-
%% Time-stamp: <Chen Wang: 2021-11-01 11:28:50>

\section{安帝刘祜\tiny(106-125)}

\subsection{生平}

汉安帝刘祜(94年-125年4月30日),东汉第六位皇帝(106年9月21日-125年4月30日在位),在位19年,其正式諡號為「孝安皇帝」,後世省略「孝」字稱「漢安帝」。

他是汉章帝的孙子、当年被废太子清河王刘庆的儿子,母左小娥。

延平元年八月辛亥(106年9月21日),是汉殇帝崩,他被外戚邓氏拥立为帝,承嗣汉和帝刘肇,改元永初。

汉安帝即位后,仍由邓太后聽政。外戚邓氏吸取窦氏灭亡的教训,联合宦官,袒护族人。永宁元年(120年),立李氏之子刘保为皇太子。

永宁二年(121年),邓太后去世,安帝才亲政。當初漢安帝號稱聰明,鄧太后才立他為帝,後來對漢安帝不滿意,漢安帝乳母王聖得知內情,又看到鄧太后遲遲不歸政,擔心鄧太后會廢除漢安帝。鄧太后去世后,向漢安帝告發鄧太后兄弟邓悝等曾經想立平原王劉勝。安帝大怒,下令灭了邓氏一族。

安帝虽灭邓氏,但未制止外戚干政的局面。再加上安帝不理朝政,沉湎于酒色,昏庸不堪,且在掖庭挑選了一位美女,封為貴人,非常寵愛,未滿一年,便立即封她為皇后,而這個皇后即為閻姬,导致当时东汉朝政腐败,社会黑暗,奸佞当道,社会矛盾日益尖锐,边患也十分严重。全国多地震,水旱蝗灾频繁不断,外有西羌等入侵边境,内有杜琦等领导的长达十多年民變,社会危机日益加深。东汉王朝衰落。

延光三年(124年),安帝乳母王圣与樊豐、江京共同构陷太子,太子刘保被废为济阴王。

延光四年三月庚申(125年4月23日),漢安帝在外巡遊途中在宛出現身體不適。三月丁卯(125年4月30日),汉安帝在葉死在乘舆上,享年32岁。當時閻姬、閻顯等人隨同漢安帝出遊,而前太子在洛陽,閻姬等害怕大臣立前太子為帝,於是詐稱安帝重病,秘不發喪,一路星夜兼程。庚午(5月3日),巡遊車隊回到皇宮。辛未(5月4日)晚上安帝駕崩消息才被公佈。四月己酉(6月11日),安帝葬于恭陵。谥号孝安皇帝,庙号恭宗(後於漢獻帝初平元年因其無功德故除去廟號)。后阎皇后迎立刘寿子刘懿为帝。刘懿死后,漢安帝独子刘保(漢順帝)才在宦官的拥戴下登基。

\subsection{永初}

\begin{longtable}{|>{\centering\scriptsize}m{2em}|>{\centering\scriptsize}m{1.3em}|>{\centering}m{8.8em}|}
  % \caption{秦王政}\
  \toprule
  \SimHei \normalsize 年数 & \SimHei \scriptsize 公元 & \SimHei 大事件 \tabularnewline
  % \midrule
  \endfirsthead
  \toprule
  \SimHei \normalsize 年数 & \SimHei \scriptsize 公元 & \SimHei 大事件 \tabularnewline
  \midrule
  \endhead
  \midrule
  元年 & 107 & \tabularnewline\hline
  二年 & 108 & \tabularnewline\hline
  三年 & 109 & \tabularnewline\hline
  四年 & 110 & \tabularnewline\hline
  五年 & 111 & \tabularnewline\hline
  六年 & 112 & \tabularnewline\hline
  七年 & 113 & \tabularnewline
  \bottomrule
\end{longtable}

\subsection{元初}

\begin{longtable}{|>{\centering\scriptsize}m{2em}|>{\centering\scriptsize}m{1.3em}|>{\centering}m{8.8em}|}
  % \caption{秦王政}\
  \toprule
  \SimHei \normalsize 年数 & \SimHei \scriptsize 公元 & \SimHei 大事件 \tabularnewline
  % \midrule
  \endfirsthead
  \toprule
  \SimHei \normalsize 年数 & \SimHei \scriptsize 公元 & \SimHei 大事件 \tabularnewline
  \midrule
  \endhead
  \midrule
  元年 & 114 & \tabularnewline\hline
  二年 & 115 & \tabularnewline\hline
  三年 & 116 & \tabularnewline\hline
  四年 & 117 & \tabularnewline\hline
  五年 & 118 & \tabularnewline\hline
  六年 & 119 & \tabularnewline\hline
  七年 & 120 & \tabularnewline
  \bottomrule
\end{longtable}

\subsection{永宁}

\begin{longtable}{|>{\centering\scriptsize}m{2em}|>{\centering\scriptsize}m{1.3em}|>{\centering}m{8.8em}|}
  % \caption{秦王政}\
  \toprule
  \SimHei \normalsize 年数 & \SimHei \scriptsize 公元 & \SimHei 大事件 \tabularnewline
  % \midrule
  \endfirsthead
  \toprule
  \SimHei \normalsize 年数 & \SimHei \scriptsize 公元 & \SimHei 大事件 \tabularnewline
  \midrule
  \endhead
  \midrule
  元年 & 120 & \tabularnewline\hline
  二年 & 121 & \tabularnewline
  \bottomrule
\end{longtable}

\subsection{建光}

\begin{longtable}{|>{\centering\scriptsize}m{2em}|>{\centering\scriptsize}m{1.3em}|>{\centering}m{8.8em}|}
  % \caption{秦王政}\
  \toprule
  \SimHei \normalsize 年数 & \SimHei \scriptsize 公元 & \SimHei 大事件 \tabularnewline
  % \midrule
  \endfirsthead
  \toprule
  \SimHei \normalsize 年数 & \SimHei \scriptsize 公元 & \SimHei 大事件 \tabularnewline
  \midrule
  \endhead
  \midrule
  元年 & 121 & \tabularnewline\hline
  二年 & 122 & \tabularnewline
  \bottomrule
\end{longtable}

\subsection{延光}

\begin{longtable}{|>{\centering\scriptsize}m{2em}|>{\centering\scriptsize}m{1.3em}|>{\centering}m{8.8em}|}
  % \caption{秦王政}\
  \toprule
  \SimHei \normalsize 年数 & \SimHei \scriptsize 公元 & \SimHei 大事件 \tabularnewline
  % \midrule
  \endfirsthead
  \toprule
  \SimHei \normalsize 年数 & \SimHei \scriptsize 公元 & \SimHei 大事件 \tabularnewline
  \midrule
  \endhead
  \midrule
  元年 & 122 & \tabularnewline\hline
  二年 & 123 & \tabularnewline\hline
  三年 & 124 & \tabularnewline\hline
  四年 & 125 & \tabularnewline
  \bottomrule
\end{longtable}


%%% Local Variables:
%%% mode: latex
%%% TeX-engine: xetex
%%% TeX-master: "../Main"
%%% End:

%% -*- coding: utf-8 -*-
%% Time-stamp: <Chen Wang: 2021-11-01 11:29:39>

\section{顺帝刘保\tiny(125-144)}

\subsection{前少帝生平}

刘懿(?-125年12月10日),一名犊,东汉第七位皇帝(125年5月18日-12月10日在位)。济北惠王刘寿的兒子,即位前為北鄉侯,东汉前少帝,汉朝官方没有把他算作汉朝皇帝之一。

汉安帝病危期间,征济北、河间王子年十四以下、七岁以上前往洛阳。汉安帝去世后,阎皇后为了把持国政,在阎显支持下,迎立北乡侯刘懿为帝,承嗣汉安帝(虽说两者是堂兄弟关系) 。少帝在位时,阎显兄弟把持朝政,作威作福。但少帝即位數月后就因病去世,之后宦官孙程等人合谋诛杀阎显兄弟和江京,并迎立济阴王刘保为帝,是为汉顺帝。

刘懿去世后以诸侯王规格下葬。永和元年(136年),災異頻繁,漢順帝感到恐懼,認為是北鄉侯當過皇帝卻以諸侯王規格下葬導致的報應。漢順帝打算追謚北鄉侯,納入漢朝皇帝體系。周舉不讚成,認為北鄉侯是奸臣閻顯等所立,並非正統,且在位一年不到就去世,年號未改,加上北鄉侯沒有其他功德,用諸侯王規格下葬已經很好了,不值得給他加上謚號和追認為皇帝。漢順帝聽從。

\subsection{顺帝生平}

汉顺帝刘保(115年-144年9月20日),东汉第八位皇帝(125年12月16日—144年9月20日在位),其正式諡號為「孝順皇帝」,後世省略「孝」字稱「漢順帝」。汉安帝和宫人李氏之子。

刘保出生后,生母李氏就被皇后阎姬毒杀。

劉保從小學習孝經章句,很得鄧太后欣賞,認為他可以繼承大統。永宁元年(120年),身为汉安帝独子的刘保被立为皇太子。

延光三年(124年),刘保生病,来到汉安帝乳母王圣家居住。当时王圣宅邸刚完成不久,刘保乳母王男、厨监邴吉认为不祥,反对太子刘保前去居住,于是与王圣等人爆发激烈争吵。王圣等人大怒,于是联合大长秋江京、中常侍樊丰等诬陷太子劉保的乳母王男、厨监邴吉。两人被杀,太子数为叹息。王圣等人惧有后祸,遂与樊丰、江京、汉安帝皇后阎姬共同构陷太子劉保。汉安帝召集大臣议论,太常桓焉、太仆来历、廷尉张皓等反对,汉安帝派人威胁反对废太子的大臣,最后只有来历坚决阻止汉安帝废太子。汉安帝大怒,下令罢免来历的官位,并立即废太子劉保为济阴王。来历不服,纠集11位官员和百姓上书喊冤,汉安帝不为所动。

汉安帝死后,阎皇后无子,便找个幼儿刘懿为皇帝,自己垂帘听政,掌握朝政大权。漢安帝喪葬期間,阎皇后等不讓劉保上殿靠近棺材,劉保悲傷吐血,餐粥不食。刘懿做了7个月的皇帝就死了,阎显等認為先前不立劉保,現在如果立他為帝,劉保會怨恨我們。於是稟告閻太后,繼續讓諸侯王子來京師挑選繼承人。宦官王康、孙程等19人看不下去,便发动宫廷政变,赶走阎太后,将时年11岁的刘保拥立为帝,改元“永建”,那19位拥立刘保的宦官也全部封侯。同時閻太后黨羽也被罷黜。阎太后被幽禁离宫,但顺帝拒绝了陈禅等人以无母子之情为由废太后的提议,仍尊奉阎太后直至其去世。

汉顺帝雖本为太子,但他的皇位是靠宦官得来的,所以将大权交给宦官。順帝本人則溫和但是軟弱,無法阻止宦官与外戚专政的局面。

后来宦官与外戚梁氏勾結,开始了长达20多年的梁冀专权。宦官、外戚互相勾结,弄权专横,東漢政治更加腐败,阶级矛盾日益尖锐,百姓怨声载道。

建康元年(144年)9月20日,汉顺帝死,享年30岁,在位19年。10月26日,葬於漢憲陵。汉顺帝安葬当年,憲陵就被盗贼盗掘。

顺帝死后谥号孝顺皇帝,庙号敬宗,後於漢獻帝初平元年因其無功德故除去廟號。

\subsection{永建}

\begin{longtable}{|>{\centering\scriptsize}m{2em}|>{\centering\scriptsize}m{1.3em}|>{\centering}m{8.8em}|}
  % \caption{秦王政}\
  \toprule
  \SimHei \normalsize 年数 & \SimHei \scriptsize 公元 & \SimHei 大事件 \tabularnewline
  % \midrule
  \endfirsthead
  \toprule
  \SimHei \normalsize 年数 & \SimHei \scriptsize 公元 & \SimHei 大事件 \tabularnewline
  \midrule
  \endhead
  \midrule
  元年 & 126 & \tabularnewline\hline
  二年 & 127 & \tabularnewline\hline
  三年 & 128 & \tabularnewline\hline
  四年 & 129 & \tabularnewline\hline
  五年 & 130 & \tabularnewline\hline
  六年 & 131 & \tabularnewline\hline
  七年 & 132 & \tabularnewline
  \bottomrule
\end{longtable}

\subsection{阳嘉}

\begin{longtable}{|>{\centering\scriptsize}m{2em}|>{\centering\scriptsize}m{1.3em}|>{\centering}m{8.8em}|}
  % \caption{秦王政}\
  \toprule
  \SimHei \normalsize 年数 & \SimHei \scriptsize 公元 & \SimHei 大事件 \tabularnewline
  % \midrule
  \endfirsthead
  \toprule
  \SimHei \normalsize 年数 & \SimHei \scriptsize 公元 & \SimHei 大事件 \tabularnewline
  \midrule
  \endhead
  \midrule
  元年 & 132 & \tabularnewline\hline
  二年 & 133 & \tabularnewline\hline
  三年 & 134 & \tabularnewline\hline
  四年 & 135 & \tabularnewline
  \bottomrule
\end{longtable}

\subsection{永和}

\begin{longtable}{|>{\centering\scriptsize}m{2em}|>{\centering\scriptsize}m{1.3em}|>{\centering}m{8.8em}|}
  % \caption{秦王政}\
  \toprule
  \SimHei \normalsize 年数 & \SimHei \scriptsize 公元 & \SimHei 大事件 \tabularnewline
  % \midrule
  \endfirsthead
  \toprule
  \SimHei \normalsize 年数 & \SimHei \scriptsize 公元 & \SimHei 大事件 \tabularnewline
  \midrule
  \endhead
  \midrule
  元年 & 136 & \tabularnewline\hline
  二年 & 137 & \tabularnewline\hline
  三年 & 138 & \tabularnewline\hline
  四年 & 139 & \tabularnewline\hline
  五年 & 140 & \tabularnewline\hline
  六年 & 141 & \tabularnewline
  \bottomrule
\end{longtable}

\subsection{汉安}

\begin{longtable}{|>{\centering\scriptsize}m{2em}|>{\centering\scriptsize}m{1.3em}|>{\centering}m{8.8em}|}
  % \caption{秦王政}\
  \toprule
  \SimHei \normalsize 年数 & \SimHei \scriptsize 公元 & \SimHei 大事件 \tabularnewline
  % \midrule
  \endfirsthead
  \toprule
  \SimHei \normalsize 年数 & \SimHei \scriptsize 公元 & \SimHei 大事件 \tabularnewline
  \midrule
  \endhead
  \midrule
  元年 & 142 & \tabularnewline\hline
  二年 & 143 & \tabularnewline\hline
  三年 & 144 & \tabularnewline
  \bottomrule
\end{longtable}

\subsection{建康}

\begin{longtable}{|>{\centering\scriptsize}m{2em}|>{\centering\scriptsize}m{1.3em}|>{\centering}m{8.8em}|}
  % \caption{秦王政}\
  \toprule
  \SimHei \normalsize 年数 & \SimHei \scriptsize 公元 & \SimHei 大事件 \tabularnewline
  % \midrule
  \endfirsthead
  \toprule
  \SimHei \normalsize 年数 & \SimHei \scriptsize 公元 & \SimHei 大事件 \tabularnewline
  \midrule
  \endhead
  \midrule
  元年 & 144 & \tabularnewline
  \bottomrule
\end{longtable}


%%% Local Variables:
%%% mode: latex
%%% TeX-engine: xetex
%%% TeX-master: "../Main"
%%% End:

%% -*- coding: utf-8 -*-
%% Time-stamp: <Chen Wang: 2018-07-10 20:12:42>

\section{冲帝\tiny(144-145)}

\subsection{永嘉}

\begin{longtable}{|>{\centering\scriptsize}m{2em}|>{\centering\scriptsize}m{1.3em}|>{\centering}m{8.8em}|}
  % \caption{秦王政}\
  \toprule
  \SimHei \normalsize 年数 & \SimHei \scriptsize 公元 & \SimHei 大事件 \tabularnewline
  % \midrule
  \endfirsthead
  \toprule
  \SimHei \normalsize 年数 & \SimHei \scriptsize 公元 & \SimHei 大事件 \tabularnewline
  \midrule
  \endhead
  \midrule
  元年 & 145 & \tabularnewline
  \bottomrule
\end{longtable}

%%% Local Variables:
%%% mode: latex
%%% TeX-engine: xetex
%%% TeX-master: "../Main"
%%% End:

%% -*- coding: utf-8 -*-
%% Time-stamp: <Chen Wang: 2019-12-17 21:09:32>

\section{质帝\tiny(145-146)}

\subsection{生平}

汉质帝刘缵(138年-146年7月26日),一名续,东汉第十位皇帝。145年3月6日即位,在位时间1年余,其正式諡號為「孝質皇帝」,後世省略「孝」字稱「漢質帝」。

前任皇帝汉冲帝駕崩时只有3岁,當時尊爲梁太后(漢順帝皇后)之弟梁冀拥立汉章帝玄孙刘缵为帝,承汉顺帝嗣,改元本初,是为汉质帝。

當時梁冀一家专权,朝政腐败,吏治不修。梁冀當時權勢極盛,威勢橫行朝廷和宮外;大臣們害怕梁冀的威勢,不敢抗命。质帝虽年幼,但他聪明伶俐,不堪梁冀的专横跋扈。质帝曾在朝見大臣時當面對梁冀说:「此跋扈将军也!」。

梁冀听罢,大为反感,便命手下在质帝的饼裏下毒弒君,146年7月26日,9岁的质帝食用毒餅後死亡。8月26日,葬於漢靜陵。

质帝崩后,继任的汉桓帝终于诛灭了梁氏。

\subsection{本初}

\begin{longtable}{|>{\centering\scriptsize}m{2em}|>{\centering\scriptsize}m{1.3em}|>{\centering}m{8.8em}|}
  % \caption{秦王政}\
  \toprule
  \SimHei \normalsize 年数 & \SimHei \scriptsize 公元 & \SimHei 大事件 \tabularnewline
  % \midrule
  \endfirsthead
  \toprule
  \SimHei \normalsize 年数 & \SimHei \scriptsize 公元 & \SimHei 大事件 \tabularnewline
  \midrule
  \endhead
  \midrule
  元年 & 146 & \tabularnewline
  \bottomrule
\end{longtable}

%%% Local Variables:
%%% mode: latex
%%% TeX-engine: xetex
%%% TeX-master: "../Main"
%%% End:

%% -*- coding: utf-8 -*-
%% Time-stamp: <Chen Wang: 2021-11-01 11:30:35>

\section{桓帝刘志\tiny(147-167)}

\subsection{生平}

汉桓帝刘志(132年-168年1月25日),东汉第十一位皇帝(146年8月1日-168年1月25日在位),其正式諡號為「孝桓皇帝」,後世省略「孝」字稱「漢桓帝」,他是汉章帝曾孙,河間孝王劉開之孫,蠡吾侯劉翼之子,在位21年。

146年,外戚梁冀毒死九岁的汉质帝,立十五岁的刘志即位,承汉顺帝嗣。

刘志从小就对梁氏不满,他即位后,就想方设法的诛灭梁氏。延熹二年(159年),桓帝联合宦官单超等5人一舉殲灭了梁氏,5人同日被封侯,称之为“五侯”。不過,五侯比外戚更加腐敗,他们对百姓们勒索抢劫,民不聊生,四处怨声载道,東汉政治更加衰頹,国势益弱。汉桓帝统治后期,一批太学士看到朝政败壞,便要求朝廷整肅宦官、改革政治。宦官气極败坏,在延熹九年(166年)与德揚天下的司隸校尉李膺发生大规模冲突。桓帝大怒,下令逮捕替李膺請願的太学生200余人,后来在太傅陈蕃、将军窦武的反对下才释放太学生,但是禁锢终身,不许再做官,史称“党锢之祸”,東漢朝政更加黑暗腐敗。汉桓帝在位期间沉迷女色,荒淫无度,后宫人数竟达五六千人。

汉桓帝於168年1月25日去世,死后谥号孝桓皇帝,庙号为威宗,168年3月9日葬於宣陵。後於漢獻帝初平元年因其無功德故除去廟號。

诸葛亮:“亲贤臣,远小人,此先汉所以兴隆也;亲小人,远贤臣,此后汉所以倾颓也。先帝在时,每与臣论此事,未尝不叹息痛恨于桓、灵也。”

范晔:“前史称桓帝好音乐,善琴笙。饰芳林而考濯龙之宫,设华盖以祠浮图、老子,斯将所谓“听于神”乎!及诛梁冀,奋威怒,天下犹企其休息。而五邪嗣虐,流衍四方。自非忠贤力争,屡折奸锋,虽愿依斟流彘,亦不可得已。”

虞世南:“桓帝赫然奋怒,诛灭梁冀,有刚断之节焉。然阉人擅命,党锢事起,非乎乱阶,始於桓帝。”

周昙:“能嫌跋扈斩梁王,宁便荣枯信段张。襄楷忠言谁佞惑,忍教奸祸起萧墙。”

\subsection{建和}

\begin{longtable}{|>{\centering\scriptsize}m{2em}|>{\centering\scriptsize}m{1.3em}|>{\centering}m{8.8em}|}
  % \caption{秦王政}\
  \toprule
  \SimHei \normalsize 年数 & \SimHei \scriptsize 公元 & \SimHei 大事件 \tabularnewline
  % \midrule
  \endfirsthead
  \toprule
  \SimHei \normalsize 年数 & \SimHei \scriptsize 公元 & \SimHei 大事件 \tabularnewline
  \midrule
  \endhead
  \midrule
  元年 & 147 & \tabularnewline\hline
  二年 & 148 & \tabularnewline\hline
  三年 & 149 & \tabularnewline
  \bottomrule
\end{longtable}

\subsection{和平}

\begin{longtable}{|>{\centering\scriptsize}m{2em}|>{\centering\scriptsize}m{1.3em}|>{\centering}m{8.8em}|}
  % \caption{秦王政}\
  \toprule
  \SimHei \normalsize 年数 & \SimHei \scriptsize 公元 & \SimHei 大事件 \tabularnewline
  % \midrule
  \endfirsthead
  \toprule
  \SimHei \normalsize 年数 & \SimHei \scriptsize 公元 & \SimHei 大事件 \tabularnewline
  \midrule
  \endhead
  \midrule
  元年 & 150 & \tabularnewline
  \bottomrule
\end{longtable}

\subsection{元嘉}

\begin{longtable}{|>{\centering\scriptsize}m{2em}|>{\centering\scriptsize}m{1.3em}|>{\centering}m{8.8em}|}
  % \caption{秦王政}\
  \toprule
  \SimHei \normalsize 年数 & \SimHei \scriptsize 公元 & \SimHei 大事件 \tabularnewline
  % \midrule
  \endfirsthead
  \toprule
  \SimHei \normalsize 年数 & \SimHei \scriptsize 公元 & \SimHei 大事件 \tabularnewline
  \midrule
  \endhead
  \midrule
  元年 & 151 & \tabularnewline\hline
  二年 & 152 & \tabularnewline\hline
  三年 & 153 & \tabularnewline
  \bottomrule
\end{longtable}

\subsection{永兴}

\begin{longtable}{|>{\centering\scriptsize}m{2em}|>{\centering\scriptsize}m{1.3em}|>{\centering}m{8.8em}|}
  % \caption{秦王政}\
  \toprule
  \SimHei \normalsize 年数 & \SimHei \scriptsize 公元 & \SimHei 大事件 \tabularnewline
  % \midrule
  \endfirsthead
  \toprule
  \SimHei \normalsize 年数 & \SimHei \scriptsize 公元 & \SimHei 大事件 \tabularnewline
  \midrule
  \endhead
  \midrule
  元年 & 153 & \tabularnewline\hline
  二年 & 154 & \tabularnewline
  \bottomrule
\end{longtable}

\subsection{永寿}

\begin{longtable}{|>{\centering\scriptsize}m{2em}|>{\centering\scriptsize}m{1.3em}|>{\centering}m{8.8em}|}
  % \caption{秦王政}\
  \toprule
  \SimHei \normalsize 年数 & \SimHei \scriptsize 公元 & \SimHei 大事件 \tabularnewline
  % \midrule
  \endfirsthead
  \toprule
  \SimHei \normalsize 年数 & \SimHei \scriptsize 公元 & \SimHei 大事件 \tabularnewline
  \midrule
  \endhead
  \midrule
  元年 & 155 & \tabularnewline\hline
  二年 & 156 & \tabularnewline\hline
  三年 & 157 & \tabularnewline\hline
  四年 & 158 & \tabularnewline
  \bottomrule
\end{longtable}

\subsection{延熹}

\begin{longtable}{|>{\centering\scriptsize}m{2em}|>{\centering\scriptsize}m{1.3em}|>{\centering}m{8.8em}|}
  % \caption{秦王政}\
  \toprule
  \SimHei \normalsize 年数 & \SimHei \scriptsize 公元 & \SimHei 大事件 \tabularnewline
  % \midrule
  \endfirsthead
  \toprule
  \SimHei \normalsize 年数 & \SimHei \scriptsize 公元 & \SimHei 大事件 \tabularnewline
  \midrule
  \endhead
  \midrule
  元年 & 158 & \tabularnewline\hline
  二年 & 159 & \tabularnewline\hline
  三年 & 160 & \tabularnewline\hline
  四年 & 161 & \tabularnewline\hline
  五年 & 162 & \tabularnewline\hline
  六年 & 163 & \tabularnewline\hline
  七年 & 164 & \tabularnewline\hline
  八年 & 165 & \tabularnewline\hline
  九年 & 166 & \tabularnewline\hline
  十年 & 167 & \tabularnewline
  \bottomrule
\end{longtable}


\subsection{永康}

\begin{longtable}{|>{\centering\scriptsize}m{2em}|>{\centering\scriptsize}m{1.3em}|>{\centering}m{8.8em}|}
  % \caption{秦王政}\
  \toprule
  \SimHei \normalsize 年数 & \SimHei \scriptsize 公元 & \SimHei 大事件 \tabularnewline
  % \midrule
  \endfirsthead
  \toprule
  \SimHei \normalsize 年数 & \SimHei \scriptsize 公元 & \SimHei 大事件 \tabularnewline
  \midrule
  \endhead
  \midrule
  元年 & 167 & \tabularnewline
  \bottomrule
\end{longtable}


%%% Local Variables:
%%% mode: latex
%%% TeX-engine: xetex
%%% TeX-master: "../Main"
%%% End:

%% -*- coding: utf-8 -*-
%% Time-stamp: <Chen Wang: 2021-11-01 11:30:41>

\section{灵帝刘宏\tiny(168-189)}

\subsection{生平}

汉灵帝刘宏(157年-189年5月13日),东汉第十二位皇帝(168年2月17日—189年5月13日在位),在位22年,葬于汉文陵,其正式諡號為「孝靈皇帝」,後世省略「孝」字稱「漢灵帝」。灵帝是东汉最后一个握有实权的皇帝。自從靈帝崩後,外戚何太后、何進掌權,漢帝自此淪爲傀儡。再後董卓、李傕、曹操相繼把持朝政,東漢大權完全落入董卓、李傕、曹操手中。

刘宏本封解渎亭侯,为承袭其父刘苌的爵位。母董夫人。他是漢章帝的玄孫,漢桓帝的堂侄。

永康元年(168年1月25日)桓帝崩,刘儵以光禄大夫身份与中常侍曹节带领中黄门、虎贲、羽林军一千多人,前往河间迎接刘宏。建宁元年正月二十日(168年2月16日),刘宏来到夏门亭,窦武亲自持节用青盖车把他迎入殿内。第二天,登基称帝,改元为“建宁”。由桓帝的皇后竇妙立為皇帝,承嗣汉桓帝,是为汉灵帝。 

汉灵帝即位后,东汉政治已经病入膏肓,天下水灾、旱灾、蝗灾、瘟疫等灾祸频繁,四处怨声载道,百姓民不聊生,国势进一步衰落。再加上宦官与外戚争权夺利,最后宦官曹节、王甫等推翻外戚窦氏並軟禁竇太后,夺得了大权,又杀死正义的太学生李膺、范滂等100余人,流放、关押800多人,多惨死于狱中,造成第二次党锢之祸。灵帝一方面保留窦太后的尊号,一方面将生母董氏迎入宫中尊为太后。将军张奂、郎中谢弼、黄门令董萌都为窦太后求情,灵帝感念窦太后拥立之恩,也一度被打动,率群臣为其上寿及增加供奉,但始终没有解除其幽禁,谢弼、董萌反而被宦官报复而死。窦太后忧死后,曹节、王甫因深恨窦氏,提出追废她及改以冯贵人配享桓帝,在廷尉陈球、太尉李咸的据理力争及灵帝本人坚持下,未果。

熹平四年(175年),议郎蔡邕认为儒家经典流传过程中出现许多错误,于是联合中常侍李巡、五官中郎将堂谿典、光禄大夫杨赐、谏议大夫马日磾、议郎张驯、韩说、太史令单飏等人共同上书要求校勘儒家经典。于是汉灵帝设立熹平石经,将校勘后的儒家经典分别刻在四十六块石碑之上,并安置在太学门外,作为经典标准,供人学习。

熹平六年(177年),鑒於鮮卑多次侵擾漢朝邊境。夏育建議討伐鮮卑,在朝廷多次商討后,派夏育、田晏、臧旻三路大軍討伐鮮卑。結果大敗而歸,夏育、田晏、臧旻被廢為庶人。

光和元年(178年),靈帝建立鴻都門學,最初號稱以研究儒术经义为名,后招集众多文士从事辞赋及书法等文艺创作活动。因鴻都門學专重文艺而轻儒家經典,引起不少大臣反對。

光和四年(181年),靈帝在皇宫之中扩建西园,修建集市供自己享乐。靈帝和宮女模仿民间市集里的商人、窃贼、地痞,并驾着白驴在西园中来回穿梭。汉灵帝同時长期沉迷于女色,灵帝特别喜欢一些冰肌玉洁的少女,还为此修建水池园林,是为裸游馆,和一群美女嬉戏于其中,并命令宫女只能穿开档裤,原因竟是为了方便自己临幸宫女。

昏庸荒淫的灵帝除了沉湎酒色以外,还一味宠信宦官,尊张让等人为“十常侍”,并说“张常侍乃我父、赵常侍乃我母”,宦官杖着皇帝的宠幸,胡作非为,对百姓勒索钱财,大肆搜刮民脂民膏,可谓腐败到极点。靈帝還多次賣官,先后有段颎、张温、崔烈、樊陵、曹嵩等人花钱买到三公之位。

在朝政腐败和天灾的双重压迫之下,叛乱有了广大的市场,巨鹿(今河北省平乡县)人张角煽动百姓,聚众造反。光和七年(184年)张角兄弟三人以“苍天已死、黄天当立、岁在甲子、天下大吉”为口号举事,史称“黃巾之亂”,这次暴乱所向披靡,给病入膏肓的东汉王朝以沉重打击。同時涼州爆發北宮伯玉之亂,國家一片衰敗。但靈帝不思悔改,繼續大幅增修宮殿,為此靈帝不惜增加民眾賦稅。

中平五年(188年),張純、張舉等人勾結烏桓叛亂,而冀州刺史王芬看見局勢混亂,圖謀廢除靈帝,但最終失敗。

鑒於漢室朝綱廢弛民變頻繁,靈帝以宦官蹇硕為統帥組建西園軍,自號無上將軍,令西園軍一度權勢高於大將軍何進。

公元189年5月13日,汉灵帝去世,终年32岁。7月17日,葬於漢文陵。漢靈帝死後引發漢朝最後一次戚宦相爭之宮變。

漢靈帝荒淫昏庸,曾于西園起裸游館千間,灵帝特别喜欢娇嫩纯洁的幼女,選十四歲以上十八歲以下的宮女于池中裸游,又曾于西園弄狗與人獸交。其人貪財,公開賣官鬻爵,致使朝政更加黑暗。但早年又有辞赋、書法和音樂爱好。

范晔《后汉书·孝灵帝纪》:“《秦本纪》说赵高谲二世,指鹿为马,而赵忠、张让亦绐灵帝不得登高临观,故知亡敝者同其致矣。然则灵帝之为灵也优哉!”、“灵帝负乘,委体宦孽。征亡备兆,《小雅》尽缺。麋鹿霜露,遂栖宫卫。”

董卓:“天下之主,宜得贤明,每念灵帝,令人愤毒!”《后汉书·卷七十四上·袁绍刘表列传第六十四上》

盖勋:“吾仍见上,上甚聪明,但拥蔽于左右耳。”《后汉书·虞傅盖臧列传第四十八》

张超《靈帝河閒舊廬碑》:赫赫在上.陶唐是承.繼德二祖.四宗是憑.上納鑒乎羲農.中結軌乎夏商.元首既明.股肱惟良.乃因舊宇.福德所基.修飾經構.農隙得時.樹中天之雙闕.崇冠山之華堂.通樓閑道.丹階紫房.金窗鬱律.玉璧內璫.青蒲充庖.朱草栖箱.川魚踊躍.雲鳥舞翔.煌煌大漢.含德乾綱.體效日月.驗化陰陽.格于上下.震暢八荒.三光宣曜.四靈效祥.天其嘉享.豐年穰穰.騶虞奏樂.鹿鳴薦觴.二祝致告.福祿來將.永保萬國.南山無量.(《艺文类聚 卷六十四》)

汉灵帝与其前任皇帝汉桓帝的统治时期是东汉最黑暗的时期,诸葛亮的《出师表》中就有蜀汉开国皇帝刘备每次“叹息痛恨于桓灵”的陈述:“亲贤臣,远小人,此先汉所以兴隆也;亲小人,远贤臣,此后汉所以倾颓也。先帝在时,每与臣论此事,未尝不叹息痛恨于桓、灵也。”

薛莹:“汉氏中兴,至于延平而世业损矣。冲质短祚,孝桓无嗣,母后称制,奸臣执政。孝灵以支庶而登至尊,由蕃侯而绍皇统,不恤宗绪,不祗天命;上亏三光之明,下伤亿兆之望。于时爵服横流,官以贿成。自公侯卿士降于皂隶,迁官袭级无不以货,刑戮无辜,摧扑忠良;佞谀在侧,直言不闻。是以贤智退而穷处,忠良摈于下位;遂至奸雄蜂起,当防隳坏,夷狄并侵,盗贼糜沸。小者带城邑,大者连州郡。编户骚动,人人思乱。当此之时,已无天子矣。会灵帝即世,盗贼相寻,其後宫室。焚灭,郊社无主,危自上起,覃及华夏。使京室为墟,海内萧条,岂不痛哉!”(《全晋文·卷八十一》)

王嘉《拾遗记》:“安、灵二帝,同为败德。夫悦目快心,罕不沦乎情欲,自非远鉴兴亡,孰能移隔下俗。佣才缘心,缅乎嗜欲,塞谏任邪,没情于淫靡。至如列代亡主,莫不凭威猛以丧家国,肆奢丽以覆宗祀。询考先坟,往往而载,佥求历古,所记非一。贩爵鬻官,乖分职之本;露宿郊居,违省方之义。”

虞世南:“灵帝承疲民之后,易为善政,黎庶倾耳。咸冀中兴,而帝袭彼覆车,毒逾前辈,倾覆宗社,职帝之由。天年厌世,为幸多矣。”(《唐文拾遗·卷十三》)

杜牧:“桓、灵四十年间杀千百比干,毒流其社稷,可以血食乎?可以坛?单父天拜郊乎?”(《樊川文集》)

周昙:“榜悬金价鬻官荣,千万为公五百卿。公瑾孔明穷退者,安知高卧遇雄英。”(《全唐诗·卷七百二十九》)

胡三省:“观灵帝以尚但之言不敢复升台榭,诚恐百姓虚散也,谓无爱民之心可乎!使其以信尚但者信诸君子之言,则汉之为汉,未可知也。”(《资治通鉴·卷第五十八·汉纪五十》)

蔡东藩《后汉演义》:“汉季之中常侍,谁不曰可杀?惟庸主如桓灵,方信而用之。”「国家赏罚有明经,宵小谗言怎可听?功罪不分昏愦甚,从知灵帝本无灵!」“若平乐观中之讲武,设坛张盖,夸示威风,灵帝自以为耀武,而盖勋乃以黩武为对,犹非知本之谈。黩武二字,惟汉武足以当之,灵帝岂足语此?彼之所信任者,妇寺而已,如皇甫嵩、朱儁诸才,皆不知重用;甚至一病不起,犹视赛硕为忠贞,托孤寄命,《范史》谓灵帝负扆,委体宦孽,征亡备兆,小雅尽缺,其亦所谓月旦之定评也乎?”


\subsection{建宁}

\begin{longtable}{|>{\centering\scriptsize}m{2em}|>{\centering\scriptsize}m{1.3em}|>{\centering}m{8.8em}|}
  % \caption{秦王政}\
  \toprule
  \SimHei \normalsize 年数 & \SimHei \scriptsize 公元 & \SimHei 大事件 \tabularnewline
  % \midrule
  \endfirsthead
  \toprule
  \SimHei \normalsize 年数 & \SimHei \scriptsize 公元 & \SimHei 大事件 \tabularnewline
  \midrule
  \endhead
  \midrule
  元年 & 168 & \tabularnewline\hline
  二年 & 169 & \tabularnewline\hline
  三年 & 170 & \tabularnewline\hline
  四年 & 171 & \tabularnewline\hline
  五年 & 172 & \tabularnewline
  \bottomrule
\end{longtable}

\subsection{熹平}

\begin{longtable}{|>{\centering\scriptsize}m{2em}|>{\centering\scriptsize}m{1.3em}|>{\centering}m{8.8em}|}
  % \caption{秦王政}\
  \toprule
  \SimHei \normalsize 年数 & \SimHei \scriptsize 公元 & \SimHei 大事件 \tabularnewline
  % \midrule
  \endfirsthead
  \toprule
  \SimHei \normalsize 年数 & \SimHei \scriptsize 公元 & \SimHei 大事件 \tabularnewline
  \midrule
  \endhead
  \midrule
  元年 & 172 & \tabularnewline\hline
  二年 & 173 & \tabularnewline\hline
  三年 & 174 & \tabularnewline\hline
  四年 & 175 & \tabularnewline\hline
  五年 & 176 & \tabularnewline\hline
  六年 & 177 & \tabularnewline\hline
  七年 & 178 & \tabularnewline
  \bottomrule
\end{longtable}

\subsection{光和}

\begin{longtable}{|>{\centering\scriptsize}m{2em}|>{\centering\scriptsize}m{1.3em}|>{\centering}m{8.8em}|}
  % \caption{秦王政}\
  \toprule
  \SimHei \normalsize 年数 & \SimHei \scriptsize 公元 & \SimHei 大事件 \tabularnewline
  % \midrule
  \endfirsthead
  \toprule
  \SimHei \normalsize 年数 & \SimHei \scriptsize 公元 & \SimHei 大事件 \tabularnewline
  \midrule
  \endhead
  \midrule
  元年 & 178 & \tabularnewline\hline
  二年 & 179 & \tabularnewline\hline
  三年 & 180 & \tabularnewline\hline
  四年 & 181 & \tabularnewline\hline
  五年 & 182 & \tabularnewline\hline
  六年 & 183 & \tabularnewline\hline
  七年 & 184 & \tabularnewline
  \bottomrule
\end{longtable}

\subsection{中平}

\begin{longtable}{|>{\centering\scriptsize}m{2em}|>{\centering\scriptsize}m{1.3em}|>{\centering}m{8.8em}|}
  % \caption{秦王政}\
  \toprule
  \SimHei \normalsize 年数 & \SimHei \scriptsize 公元 & \SimHei 大事件 \tabularnewline
  % \midrule
  \endfirsthead
  \toprule
  \SimHei \normalsize 年数 & \SimHei \scriptsize 公元 & \SimHei 大事件 \tabularnewline
  \midrule
  \endhead
  \midrule
  元年 & 184 & \tabularnewline\hline
  二年 & 185 & \tabularnewline\hline
  三年 & 186 & \tabularnewline\hline
  四年 & 187 & \tabularnewline\hline
  五年 & 188 & \tabularnewline\hline
  六年 & 189 & \tabularnewline
  \bottomrule
\end{longtable}


%%% Local Variables:
%%% mode: latex
%%% TeX-engine: xetex
%%% TeX-master: "../Main"
%%% End:

%% -*- coding: utf-8 -*-
%% Time-stamp: <Chen Wang: 2018-07-10 20:22:57>

\section{刘辩\tiny(189)}

\subsection{光熹}

\begin{longtable}{|>{\centering\scriptsize}m{2em}|>{\centering\scriptsize}m{1.3em}|>{\centering}m{8.8em}|}
  % \caption{秦王政}\
  \toprule
  \SimHei \normalsize 年数 & \SimHei \scriptsize 公元 & \SimHei 大事件 \tabularnewline
  % \midrule
  \endfirsthead
  \toprule
  \SimHei \normalsize 年数 & \SimHei \scriptsize 公元 & \SimHei 大事件 \tabularnewline
  \midrule
  \endhead
  \midrule
  元年 & 189 & \tabularnewline
  \bottomrule
\end{longtable}

\subsection{昭宁}

\begin{longtable}{|>{\centering\scriptsize}m{2em}|>{\centering\scriptsize}m{1.3em}|>{\centering}m{8.8em}|}
  % \caption{秦王政}\
  \toprule
  \SimHei \normalsize 年数 & \SimHei \scriptsize 公元 & \SimHei 大事件 \tabularnewline
  % \midrule
  \endfirsthead
  \toprule
  \SimHei \normalsize 年数 & \SimHei \scriptsize 公元 & \SimHei 大事件 \tabularnewline
  \midrule
  \endhead
  \midrule
  元年 & 189 & \tabularnewline
  \bottomrule
\end{longtable}


%%% Local Variables:
%%% mode: latex
%%% TeX-engine: xetex
%%% TeX-master: "../Main"
%%% End:

%% -*- coding: utf-8 -*-
%% Time-stamp: <Chen Wang: 2018-07-10 20:26:01>

\section{献帝\tiny(189-220)}

\subsection{永汉}

\begin{longtable}{|>{\centering\scriptsize}m{2em}|>{\centering\scriptsize}m{1.3em}|>{\centering}m{8.8em}|}
  % \caption{秦王政}\
  \toprule
  \SimHei \normalsize 年数 & \SimHei \scriptsize 公元 & \SimHei 大事件 \tabularnewline
  % \midrule
  \endfirsthead
  \toprule
  \SimHei \normalsize 年数 & \SimHei \scriptsize 公元 & \SimHei 大事件 \tabularnewline
  \midrule
  \endhead
  \midrule
  元年 & 189 & \tabularnewline
  \bottomrule
\end{longtable}

\subsection{中平}

\begin{longtable}{|>{\centering\scriptsize}m{2em}|>{\centering\scriptsize}m{1.3em}|>{\centering}m{8.8em}|}
  % \caption{秦王政}\
  \toprule
  \SimHei \normalsize 年数 & \SimHei \scriptsize 公元 & \SimHei 大事件 \tabularnewline
  % \midrule
  \endfirsthead
  \toprule
  \SimHei \normalsize 年数 & \SimHei \scriptsize 公元 & \SimHei 大事件 \tabularnewline
  \midrule
  \endhead
  \midrule
  元年 & 189 & \tabularnewline
  \bottomrule
\end{longtable}

\subsection{初平}

\begin{longtable}{|>{\centering\scriptsize}m{2em}|>{\centering\scriptsize}m{1.3em}|>{\centering}m{8.8em}|}
  % \caption{秦王政}\
  \toprule
  \SimHei \normalsize 年数 & \SimHei \scriptsize 公元 & \SimHei 大事件 \tabularnewline
  % \midrule
  \endfirsthead
  \toprule
  \SimHei \normalsize 年数 & \SimHei \scriptsize 公元 & \SimHei 大事件 \tabularnewline
  \midrule
  \endhead
  \midrule
  元年 & 190 & \tabularnewline\hline
  二年 & 191 & \tabularnewline\hline
  三年 & 192 & \tabularnewline\hline
  四年 & 193 & \tabularnewline
  \bottomrule
\end{longtable}


\subsection{兴平}

\begin{longtable}{|>{\centering\scriptsize}m{2em}|>{\centering\scriptsize}m{1.3em}|>{\centering}m{8.8em}|}
  % \caption{秦王政}\
  \toprule
  \SimHei \normalsize 年数 & \SimHei \scriptsize 公元 & \SimHei 大事件 \tabularnewline
  % \midrule
  \endfirsthead
  \toprule
  \SimHei \normalsize 年数 & \SimHei \scriptsize 公元 & \SimHei 大事件 \tabularnewline
  \midrule
  \endhead
  \midrule
  元年 & 194 & \tabularnewline\hline
  二年 & 195 & \tabularnewline
  \bottomrule
\end{longtable}

\subsection{建安}

\begin{longtable}{|>{\centering\scriptsize}m{2em}|>{\centering\scriptsize}m{1.3em}|>{\centering}m{8.8em}|}
  % \caption{秦王政}\
  \toprule
  \SimHei \normalsize 年数 & \SimHei \scriptsize 公元 & \SimHei 大事件 \tabularnewline
  % \midrule
  \endfirsthead
  \toprule
  \SimHei \normalsize 年数 & \SimHei \scriptsize 公元 & \SimHei 大事件 \tabularnewline
  \midrule
  \endhead
  \midrule
  元年 & 196 & \tabularnewline\hline
  二年 & 197 & \tabularnewline\hline
  三年 & 198 & \tabularnewline\hline
  四年 & 199 & \tabularnewline\hline
  五年 & 200 & \tabularnewline\hline
  六年 & 201 & \tabularnewline\hline
  七年 & 202 & \tabularnewline\hline
  八年 & 203 & \tabularnewline\hline
  九年 & 204 & \tabularnewline\hline
  十年 & 205 & \tabularnewline\hline
  十一年 & 206 & \tabularnewline\hline
  十二年 & 207 & \tabularnewline\hline
  十三年 & 208 & \tabularnewline\hline
  十四年 & 209 & \tabularnewline\hline
  十五年 & 210 & \tabularnewline\hline
  十六年 & 211 & \tabularnewline\hline
  十七年 & 212 & \tabularnewline\hline
  十八年 & 213 & \tabularnewline\hline
  十九年 & 214 & \tabularnewline\hline
  二十年 & 215 & \tabularnewline\hline
  二一年 & 216 & \tabularnewline\hline
  二二年 & 217 & \tabularnewline\hline
  二三年 & 218 & \tabularnewline\hline
  二四年 & 219 & \tabularnewline\hline
  二五年 & 220 & \tabularnewline
  \bottomrule
\end{longtable}

\subsection{延康}

\begin{longtable}{|>{\centering\scriptsize}m{2em}|>{\centering\scriptsize}m{1.3em}|>{\centering}m{8.8em}|}
  % \caption{秦王政}\
  \toprule
  \SimHei \normalsize 年数 & \SimHei \scriptsize 公元 & \SimHei 大事件 \tabularnewline
  % \midrule
  \endfirsthead
  \toprule
  \SimHei \normalsize 年数 & \SimHei \scriptsize 公元 & \SimHei 大事件 \tabularnewline
  \midrule
  \endhead
  \midrule
  元年 & 220 & \tabularnewline
  \bottomrule
\end{longtable}


%%% Local Variables:
%%% mode: latex
%%% TeX-engine: xetex
%%% TeX-master: "../Main"
%%% End:


%%% Local Variables:
%%% mode: latex
%%% TeX-engine: xetex
%%% TeX-master: "../Main"
%%% End:
 % 东汉
% %% -*- coding: utf-8 -*-
%% Time-stamp: <Chen Wang: 2019-10-15 11:08:01>

\chapter{三国\tiny(220-280)}


%% -*- coding: utf-8 -*-
%% Time-stamp: <Chen Wang: 2019-10-15 11:08:14>


\section{曹魏\tiny(220-265)}

%% -*- coding: utf-8 -*-
%% Time-stamp: <Chen Wang: 2018-07-10 20:33:47>

\subsection{文帝\tiny(220-226)}

\subsubsection{黄初}

\begin{longtable}{|>{\centering\scriptsize}m{2em}|>{\centering\scriptsize}m{1.3em}|>{\centering}m{8.8em}|}
  % \caption{秦王政}\
  \toprule
  \SimHei \normalsize 年数 & \SimHei \scriptsize 公元 & \SimHei 大事件 \tabularnewline
  % \midrule
  \endfirsthead
  \toprule
  \SimHei \normalsize 年数 & \SimHei \scriptsize 公元 & \SimHei 大事件 \tabularnewline
  \midrule
  \endhead
  \midrule
  元年 & 220 & \tabularnewline\hline
  二年 & 221 & \tabularnewline\hline
  三年 & 222 & \tabularnewline\hline
  四年 & 223 & \tabularnewline\hline
  五年 & 224 & \tabularnewline\hline
  六年 & 225 & \tabularnewline\hline
  七年 & 226 & \tabularnewline
  \bottomrule
\end{longtable}


%%% Local Variables:
%%% mode: latex
%%% TeX-engine: xetex
%%% TeX-master: "../../Main"
%%% End:

%% -*- coding: utf-8 -*-
%% Time-stamp: <Chen Wang: 2018-07-10 20:44:37>

\subsection{明帝\tiny(226-239)}

\subsubsection{太和}

\begin{longtable}{|>{\centering\scriptsize}m{2em}|>{\centering\scriptsize}m{1.3em}|>{\centering}m{8.8em}|}
  % \caption{秦王政}\
  \toprule
  \SimHei \normalsize 年数 & \SimHei \scriptsize 公元 & \SimHei 大事件 \tabularnewline
  % \midrule
  \endfirsthead
  \toprule
  \SimHei \normalsize 年数 & \SimHei \scriptsize 公元 & \SimHei 大事件 \tabularnewline
  \midrule
  \endhead
  \midrule
  元年 & 227 & \tabularnewline\hline
  二年 & 228 & \tabularnewline\hline
  三年 & 229 & \tabularnewline\hline
  四年 & 230 & \tabularnewline\hline
  五年 & 231 & \tabularnewline\hline
  六年 & 232 & \tabularnewline\hline
  七年 & 233 & \tabularnewline
  \bottomrule
\end{longtable}

\subsubsection{青龙}

\begin{longtable}{|>{\centering\scriptsize}m{2em}|>{\centering\scriptsize}m{1.3em}|>{\centering}m{8.8em}|}
  % \caption{秦王政}\
  \toprule
  \SimHei \normalsize 年数 & \SimHei \scriptsize 公元 & \SimHei 大事件 \tabularnewline
  % \midrule
  \endfirsthead
  \toprule
  \SimHei \normalsize 年数 & \SimHei \scriptsize 公元 & \SimHei 大事件 \tabularnewline
  \midrule
  \endhead
  \midrule
  元年 & 233 & \tabularnewline\hline
  二年 & 234 & \tabularnewline\hline
  三年 & 235 & \tabularnewline\hline
  四年 & 236 & \tabularnewline\hline
  五年 & 237 & \tabularnewline
  \bottomrule
\end{longtable}

\subsubsection{景初}

\begin{longtable}{|>{\centering\scriptsize}m{2em}|>{\centering\scriptsize}m{1.3em}|>{\centering}m{8.8em}|}
  % \caption{秦王政}\
  \toprule
  \SimHei \normalsize 年数 & \SimHei \scriptsize 公元 & \SimHei 大事件 \tabularnewline
  % \midrule
  \endfirsthead
  \toprule
  \SimHei \normalsize 年数 & \SimHei \scriptsize 公元 & \SimHei 大事件 \tabularnewline
  \midrule
  \endhead
  \midrule
  元年 & 237 & \tabularnewline\hline
  二年 & 238 & \tabularnewline\hline
  三年 & 239 & \tabularnewline
  \bottomrule
\end{longtable}


%%% Local Variables:
%%% mode: latex
%%% TeX-engine: xetex
%%% TeX-master: "../../Main"
%%% End:

%% -*- coding: utf-8 -*-
%% Time-stamp: <Chen Wang: 2018-07-10 20:48:48>

\subsection{曹芳\tiny(239-254)}

\subsubsection{正始}

\begin{longtable}{|>{\centering\scriptsize}m{2em}|>{\centering\scriptsize}m{1.3em}|>{\centering}m{8.8em}|}
  % \caption{秦王政}\
  \toprule
  \SimHei \normalsize 年数 & \SimHei \scriptsize 公元 & \SimHei 大事件 \tabularnewline
  % \midrule
  \endfirsthead
  \toprule
  \SimHei \normalsize 年数 & \SimHei \scriptsize 公元 & \SimHei 大事件 \tabularnewline
  \midrule
  \endhead
  \midrule
  元年 & 240 & \tabularnewline\hline
  二年 & 241 & \tabularnewline\hline
  三年 & 242 & \tabularnewline\hline
  四年 & 243 & \tabularnewline\hline
  五年 & 244 & \tabularnewline\hline
  六年 & 245 & \tabularnewline\hline
  七年 & 246 & \tabularnewline\hline
  八年 & 247 & \tabularnewline\hline
  九年 & 248 & \tabularnewline\hline
  十年 & 249 & \tabularnewline
  \bottomrule
\end{longtable}

\subsubsection{嘉平}

\begin{longtable}{|>{\centering\scriptsize}m{2em}|>{\centering\scriptsize}m{1.3em}|>{\centering}m{8.8em}|}
  % \caption{秦王政}\
  \toprule
  \SimHei \normalsize 年数 & \SimHei \scriptsize 公元 & \SimHei 大事件 \tabularnewline
  % \midrule
  \endfirsthead
  \toprule
  \SimHei \normalsize 年数 & \SimHei \scriptsize 公元 & \SimHei 大事件 \tabularnewline
  \midrule
  \endhead
  \midrule
  元年 & 249 & \tabularnewline\hline
  二年 & 250 & \tabularnewline\hline
  三年 & 251 & \tabularnewline\hline
  四年 & 252 & \tabularnewline\hline
  五年 & 253 & \tabularnewline\hline
  六年 & 254 & \tabularnewline
  \bottomrule
\end{longtable}


%%% Local Variables:
%%% mode: latex
%%% TeX-engine: xetex
%%% TeX-master: "../../Main"
%%% End:

%% -*- coding: utf-8 -*-
%% Time-stamp: <Chen Wang: 2018-07-10 20:50:14>

\subsection{曹髦\tiny(254-260)}

\subsubsection{正元}

\begin{longtable}{|>{\centering\scriptsize}m{2em}|>{\centering\scriptsize}m{1.3em}|>{\centering}m{8.8em}|}
  % \caption{秦王政}\
  \toprule
  \SimHei \normalsize 年数 & \SimHei \scriptsize 公元 & \SimHei 大事件 \tabularnewline
  % \midrule
  \endfirsthead
  \toprule
  \SimHei \normalsize 年数 & \SimHei \scriptsize 公元 & \SimHei 大事件 \tabularnewline
  \midrule
  \endhead
  \midrule
  元年 & 254 & \tabularnewline\hline
  二年 & 255 & \tabularnewline\hline
  三年 & 256 & \tabularnewline
  \bottomrule
\end{longtable}

\subsubsection{甘露}

\begin{longtable}{|>{\centering\scriptsize}m{2em}|>{\centering\scriptsize}m{1.3em}|>{\centering}m{8.8em}|}
  % \caption{秦王政}\
  \toprule
  \SimHei \normalsize 年数 & \SimHei \scriptsize 公元 & \SimHei 大事件 \tabularnewline
  % \midrule
  \endfirsthead
  \toprule
  \SimHei \normalsize 年数 & \SimHei \scriptsize 公元 & \SimHei 大事件 \tabularnewline
  \midrule
  \endhead
  \midrule
  元年 & 256 & \tabularnewline\hline
  二年 & 257 & \tabularnewline\hline
  三年 & 258 & \tabularnewline\hline
  四年 & 259 & \tabularnewline\hline
  五年 & 260 & \tabularnewline
  \bottomrule
\end{longtable}


%%% Local Variables:
%%% mode: latex
%%% TeX-engine: xetex
%%% TeX-master: "../../Main"
%%% End:

%% -*- coding: utf-8 -*-
%% Time-stamp: <Chen Wang: 2018-07-10 20:51:45>

\subsection{元帝\tiny(260-265)}

\subsubsection{景元}

\begin{longtable}{|>{\centering\scriptsize}m{2em}|>{\centering\scriptsize}m{1.3em}|>{\centering}m{8.8em}|}
  % \caption{秦王政}\
  \toprule
  \SimHei \normalsize 年数 & \SimHei \scriptsize 公元 & \SimHei 大事件 \tabularnewline
  % \midrule
  \endfirsthead
  \toprule
  \SimHei \normalsize 年数 & \SimHei \scriptsize 公元 & \SimHei 大事件 \tabularnewline
  \midrule
  \endhead
  \midrule
  元年 & 260 & \tabularnewline\hline
  二年 & 261 & \tabularnewline\hline
  三年 & 262 & \tabularnewline\hline
  四年 & 263 & \tabularnewline\hline
  五年 & 264 & \tabularnewline
  \bottomrule
\end{longtable}

\subsubsection{咸熙}

\begin{longtable}{|>{\centering\scriptsize}m{2em}|>{\centering\scriptsize}m{1.3em}|>{\centering}m{8.8em}|}
  % \caption{秦王政}\
  \toprule
  \SimHei \normalsize 年数 & \SimHei \scriptsize 公元 & \SimHei 大事件 \tabularnewline
  % \midrule
  \endfirsthead
  \toprule
  \SimHei \normalsize 年数 & \SimHei \scriptsize 公元 & \SimHei 大事件 \tabularnewline
  \midrule
  \endhead
  \midrule
  元年 & 264 & \tabularnewline\hline
  二年 & 265 & \tabularnewline
  \bottomrule
\end{longtable}


%%% Local Variables:
%%% mode: latex
%%% TeX-engine: xetex
%%% TeX-master: "../../Main"
%%% End:


%%% Local Variables:
%%% mode: latex
%%% TeX-engine: xetex
%%% TeX-master: "../../Main"
%%% End:

%% -*- coding: utf-8 -*-
%% Time-stamp: <Chen Wang: 2019-12-17 22:35:45>


\section{蜀汉\tiny(221-263)}

\subsection{简介}

漢(221年-263年,又稱蜀漢)為中国历史上三國時期西南方的一個政權。於221年由昭烈帝稱帝開始,至263年曹魏攻入蜀地,後主投降為終,共經過43年,二帝統治。漢昭烈帝劉備、漢丞相武鄉侯諸葛亮的統治下,政治清明,為大漢復興、北伐奠定基礎。

刘备以延续汉代劉氏皇室政权,称国号为“漢”,有時自稱「季漢」。不過,魏晉政權皆不承認漢政權承繼漢室、國號為「漢」,而因其主要領土古稱蜀地,而稱之為「蜀」,「蜀」遂成為其俗稱。由於漢為曹魏所滅亡,晉又取代魏國,所以《三國志》作者陳壽(漢出身,漢滅亡後仕西晉)為保持政治正確,以「蜀」稱呼其國號,而不使用正式國號「漢」;同為曾出仕漢的李密在《陳情表》中稱呼蜀漢為「偽朝」;資治通鑑稱其為「漢」。後來歷史為将其區別于西漢和東漢,稱劉備政權為「蜀漢」、「季漢」。

東漢末年,群雄割據,漢景帝後裔劉備先佔據荊州,再從攻取劉璋治的益州,又自曹操手中奪取漢中,於219年自封漢中王。220年,孫權進攻荊州,殺死守將關羽,使刘备元气大伤。同年曹丕逼漢獻帝禪讓,篡代東漢,221年劉備於成都稱帝,設立高廟,合祭漢朝皇帝,以汉室宗亲的身份承继漢祚,國號仍为「漢」,稱「東漢」為「中漢」,而刘备称帝后这段时期稱為「季漢」。

同年,劉備以為關羽報仇的名义,發兵討伐孫權,意图夺回荆州,但卻不幸大將張飞在戰爭前被部下張達、范彊殺害,後來更于222年夏被陸遜在夷陵之戰中打敗,最終撤退到白帝城。劉備於223年四月駕崩,諡號為昭烈帝。太子劉禪繼位,由託孤大臣諸葛亮、李严扶助朝政。诸葛亮立即与东吴修好,恢复了联吴抗曹的政策,双方从此再无互相争战。

225年,諸葛亮平定南中多郡的叛乱,并利用降服了南中少数民族部落,削弱李严的势力,解決蜀漢的後方問題。蜀汉此后的三十多年历史中,内外几乎只有对曹魏作战一个焦点,小有出现魏吴两国政变、叛乱等情况。

228年,諸葛亮率領大軍出漢中,開始第一次北伐曹魏,却在街亭战役中失败,并不得不依法处斩对此负有重大责任的參軍馬謖。之后诸葛亮继续北伐,但多次因补给线太长、粮草不济被迫撤军,致使北伐始终无法获得重大成效,其中在建威之戰後進佔原屬曹魏的武都、阴平两个郡。234年,諸葛亮於第五次北伐中病故於五丈原。

諸葛亮死后劉禪开始自摄国政,蜀汉由蔣琬、費禕、董允等接手执掌朝政。大将军蔣琬数次派出姜維等率军对曹魏进攻。246年蔣琬死后費禕掌权,不主张过多军事进攻。同年董允去世,刘禅开始寵信宦官黃皓和寵臣陳祗,令朝政開始變壞。費禕253年遇刺身亡。

大将姜維在247年至262年不斷的北伐,甚至一度年年大規模征戰,严重消耗蜀汉国力,人民也困苦不堪。

姜維讨厌宦官黄皓擅权,曾上书刘禅要求处死黃皓,黄皓得知后也预谋废姜维立阎宇,同时朝中大臣诸葛瞻、董厥等也對姜維多次伐魏但收效甚微感到反感,上书刘禅要求召还姜维为益州刺史,夺其兵权。姜維惟有避居陇西沓中屯田,內外產生嚴重分歧,汉中门户大开。而當時曹魏實質控制者司馬昭決定伐蜀,

263年八月司馬昭派征西將軍鄧艾、中護軍諸葛緒和鎮西將軍鍾會率三路南下,開始魏滅蜀之戰。漢中被破,鍾會軍雖被从沓中赶回来的姜維擋於劍閣,但鄧艾率軍偷襲涪城(今绵阳市),蜀漢江油守將馬邈見魏軍突然出現,投降魏軍。鄧艾繼續進攻,擊敗迎戰的衛將軍諸葛瞻。十一月,劉禪接受譙周意見,帶領文武百官出降,蜀漢正式滅亡,但姜維詐降鍾會,打算利用鍾會野心造成魏军内耗再杀死钟会夺取军权复国,但因事敗,死於亂軍之中。

蜀漢主要佔領益州,共分為二十二郡,擁有一百三十一個縣國,劉備於建安十三年(208年)赤壁之戰後始得荊州的南郡部分及长沙、武陵等荊南4郡。建安十九年(214年)入蜀取得益州,明年與孫權議定平分荊州,219年劉備於漢中之戰打敗曹操得汉中郡。至建安二十四年(219年)時,計此時共轄有2州21郡1屬國。同年孫吳擊取荊州而喪失南郡、武陵郡、零陵郡3郡,诸葛亮北伐曹魏取得武都郡、陰平郡2郡,分割設置數郡。蜀漢滅亡以前,計轄有1州22郡。

蜀漢政律《蜀科》由諸葛亮、法正、劉巴、李嚴、伊籍所編列。後來,劉備逝世,繼位的劉禪幼小,政策多由諸葛亮所主持。在朝內制定八務、七戒、六恐、五懼,訓誡大臣;而朝外風氣清廉,法家思想治蜀地,人心不亂,使蜀中政事、民事都能成功進行。

當時實行安撫百姓,展示法度規範,約制官職,嚴格遵從權制,廣開誠心,公平行事。做到盡忠益的人雖有錯,必定賞賜;犯法怠慢的人雖是親屬之人,但都懲罰。如能順從懲罰,雖重罪仍會得釋;巧言令色的人,雖輕罪仍會受重罰。而且刑政雖嚴峻,但都無人怨恨。

诸葛亮的治国取得了很大的成果,即使是连年对曹魏作战,在诸葛亮在世时,蜀汉的经济仍然得到了较大的发展。袁準评价为“亮之治蜀,田畴辟,仓廪实,器械利,蓄积饶,朝会不华,路无醉人。”正因为诸葛亮清廉而公正,并能使得百姓安居乐业,生活富足稳定,百姓对诸葛亮极为爱戴,陈寿称之为“至今梁、益之民,咨述亮者,言犹在耳,虽甘棠之咏召公,郑人之歌子产,无以远譬也。”

諸葛亮死後,蔣琬、費禕、董允等都繼續諸葛亮的政策;不過後來劉禪寵信寵臣陳祗、宦官黃皓,並開始相信鬼神之說,令朝政漸下;儘管如此,至蜀漢滅亡為止,國家政風仍算清廉,官吏約四萬人。

蜀漢政治中,在劉備、諸葛亮的經營下,人才多能發揮己用,最初起兵的麾下、荊襄舊部、蜀中降將,在劉備、諸葛亮等人的協調下得以事才任用,荊州舊部與劉璋下屬降將;有能者多能位居要職。 如法正成為僅次於諸葛亮的謀士,蜀中舊將的李嚴;劉備稱帝後任尚書令;在劉備逝世後出鎮江州,後更遷為驃騎將軍,其後由於延誤軍機而招致流放,然其子李豐仍受諸葛亮任用,並未因此有所偏廢。又如吳懿,劉備定蜀後任討逆將軍;稱帝後遷為關中都督;建興八年任左將軍,諸葛亮過世後,陞至車騎將軍,並擔任漢中的防務總指揮,其族弟吳班,官位常僅次於吳懿,後主世,任驃騎將軍。黃權於劉備領益州牧時任治中從事。又如董和、李恢等亦受劉備、諸葛亮重用。諸葛亮治事期間,不僅提拔荊州舊部的蒋琬、魏延、费祎、楊儀等,亦大加重用蜀中人才;文如董和、董允父子,武如王平、張嶷、張翼、馬忠等如是,縱如降將的姜维,亦受到諸葛亮的重用,其後更成為蜀漢後期北伐的重要將領。

其人才可謂兼容並蓄,不論荊襄舊部、起兵之初的麾下、蜀中舊將、魏國降將等,皆能依照其能力擔任相應的職務。蜀漢後期亦有不少人才,武將如霍峻之子霍弋、句扶、柳隱、羅憲等,皆為一時之選。

蜀漢將士,在劉備在位最盛時期(未失荊州時)約有十六萬至二十萬之間。至蜀亡國,仍有將士十萬二千人。

蜀漢的對外戰爭多向曹魏發動,前期最著名是諸葛亮北伐與後期最著名姜維北伐。

然而姜維的北伐卻間接導致蜀漢壓力加劇,國力日漸下滑,人心厭戰,以致蜀國重要據點缺乏士兵固守。

蜀漢內部政局較東吳及曹魏平和,除夷陵之戰及諸葛亮南征之外,鮮有內亂或向東吳方面發起攻擊。

214年,劉備入蜀後,巴蜀地區財政混亂,劉巴提出鑄直百錢,平衡物價,解決問題。當中五銖錢與直百錢並用,錢面有鑄字「直百五銖」、背有「好右有為」,為犍為郡所鑄,從中知道蜀鑄錢不只在一地。

而蜀漢的收入有田租,但暫未有實例;鹽鐵對蜀漢有不錯的利潤;而南中金、銀、丹、漆、耕牛、戰馬、蜀錦等貢品,令蜀漢軍費有所供給,國家富裕,為諸葛亮北伐提供物資;另有其他收入沒有記錄。另一方面,支出包官俸、軍糧、賞賜等,至蜀漢亡時,官府仍有金、銀各二千斤。

劉備入蜀後,藉著戰國時代李冰所開的都江堰所提供的充足水利,不斷增加耕地的灌溉面積;後至諸葛亮北伐時,為了防護都江堰而任命的督堰官人數已有一千二百人之多,足見其灌溉農業規模之大。諸葛亮在保山市法寶山下設有諸葛堰。在漢中增築山河堰,成都重築九里堤。蜀漢由於地形上山地較多,除了較主要的灌溉農地以外,幾乎沒有興修水利、屯田的必要,遠不及魏、吳,雖然有督農的官職,但只設在與魏國接壤的前線漢中郡。蜀中雖不缺糧,但受限於地理限制,在與魏國作戰時補給線往往較魏國長;諸葛亮第五次北伐時,曾在魏地屯田,只為解決運糧問題。而至姜維時,他亦在沓中種麥,但主要作用是避開黃皓。至蜀漢亡時,官府仍有四十多萬斛米糧。

三國中,蜀漢在絲織業最為興盛,以蜀郡為盛產地,稱為「蜀錦」。蜀漢設立專門的錦官製造蜀錦,甚至諸葛亮北伐的經濟來源都嚴重依賴蜀錦,此外蜀錦亦多用以外交禮物及賞賜。漢亡之時,官府藏有錦、綺、彩、絹,各二十萬匹,在當時是十分驚人的數目。井鹽也是蜀地特產,為蜀漢其中一項重要工業。另有開發天然氣。三國時期的商業不太發達,一是由於生產量減少,人民多以物易物;二是由於金屬貨幣不流通;三則是由於割據的局面,商人不能遠行。當中,魏、漢因對立而沒有發生貿易;與東吳則貿易頗多;蜀漢因地處西南,所以鮮有對外域進行貿易。

蜀漢全盛時期擁有三十多萬戶(未失荊州時),人口約一百萬,為三國中最少。 汉昭烈帝章武元年(221年),在籍户口分别为二十万户与九十万人,经诸葛亮治蜀,至蜀亡时(263年)共有1082000人 ,其中户数二十八万, 民数九十四万, 带甲将士十万二千, 官吏四万, 當中蜀郡擁有戶口最多。

因地處著名產茶的西南地區,所以飲茶的風氣甚盛,足以代酒,因此飲酒之風不及魏、吳。民間亦有拜祭鬼神,諸葛亮死後,劉禪初期不與立廟,百姓仍在路上祭祀。朝廷中,蜀漢後期,劉禪聽信黃皓、陳祗巫鬼之說,最後間接令蜀漢被曹魏入侵。

%% -*- coding: utf-8 -*-
%% Time-stamp: <Chen Wang: 2021-11-01 11:36:07>

\subsection{昭烈帝劉備\tiny(221-223)}

\subsubsection{生平}

漢昭烈帝劉備(161年7月16日-223年6月10日),字玄德,涿郡涿縣(今河北省涿州市)人,祖籍徐州沛縣(今徐州市沛縣),亦稱漢先主,三國時代蜀漢第一位皇帝,諡號昭烈皇帝,三國志、華陽國志等稱為先主 ,繼其帝位的劉禪則被稱為「後主」,資治通鑑稱劉備父子為漢主。

劉備雖為漢景帝後代,但世系久遠,實由布衣起步而終得一方天下。

劉備是汉景帝第九子中山靖王劉勝之子刘贞的後代,而裴松之三国志注所引《典略》记载,刘备为“临邑侯枝属”。祖父名雄,父親名弘,世代皆仕於州郡,祖父劉雄曾被推舉為孝廉,官至東郡範令。世居酈亭樓桑里。

劉弘在劉備少時已逝,劉備便與母親販賣草鞋、織草蓆為業。家裡房舍的東南角的圍籬上有種植桑樹,高五丈餘,從遠處觀看像是一臺當時小車的車頂,路過的人皆訝異此樹的非凡,或說此家必當出貴人。劉備小時候與家族中年齡相近的小孩在樹下遊戲時,曾說:「吾必當乘此羽葆蓋車。」他的叔父劉子敬說:「汝勿妄語,滅吾門也!」劉備15歲時,刘备母亲要他外出求學,與同宗刘德然、遼西公孫瓚同入大儒盧植門下求學。

劉德然之父劉元起常資助劉備,所給錢物與自己兒子劉德然等同。劉元起妻罵:「各自一家,何能常爾邪!」元起答:「吾宗中有此兒,非常人也。」公孫瓚與劉備結為好友,公孫瓚較年長,劉備以兄事之。

劉備不甚樂讀書,喜歡評馬論犬、音樂、華美的衣服。身長七尺五寸(約173公分,漢時一尺約為23.1公分),垂手下膝,有一對招風大耳,不需攬鏡自照,眼可自見其耳。少說話,善於待人,喜怒不形於色。好交結豪俠義士,年輕人爭相趨附他。中山大商人張世平、蘇雙等多給與金錢資助,劉備由是得用以糾合组织部下。由於個性與行事風格酷似先祖劉邦,而被評為有高祖之風。

184年(23歲),黃巾之亂爆發,各州郡皆有人民組織義軍討伐。劉備率領耿雍、關羽、張飛、牵招及一干下屬跟隨鄒靖討伐黃巾軍,立下戰功,被任為安喜尉。後來,漢室有令:如因軍功而成為長吏的人,都要被選精汰穢,督郵到安喜要遣散劉備,劉備知道消息後,到督郵入住的驛站休息房舍求見,督郵聲稱有病不肯相見,劉備因此感到不悅,便徑直闖入房舍,將督郵綑綁,杖打二百下,然後棄官逃亡。後來,大將軍何進派都尉毌丘毅到丹楊募兵,劉備也在途中加入,到下邳時與盜賊力戰立功,任為下密縣丞,不久又辭官。

191年(30歲),刘备時任高唐令,但被盗贼击败而投奔公孫瓚,公孫瓚隨即上表,保奏劉備為別部司馬,任為平原令、平原相。劉備平原外禦賊寇,在內則屯糧分發給百姓,士以下的人,都可與他同席而坐,同簋而食,不會有所揀擇。據說郡民劉平不服從劉備的治理,唆使刺客前去暗殺。劉備毫不知情,還對刺客十分禮遇,刺客深受感動,不忍心殺害劉備,便坦露實情離去。劉備治理平原郡深得人心、相當成功。

黄巾餘黨管亥率眾軍攻打北海郡,北海相孔融被大軍所圍,情勢危急,便派太史慈突圍向劉備求救。太史慈對劉備說:「慈,東萊之鄙人也,孔北海親非骨肉,比非鄉黨,特以名志相好,有分災共患之義。今管亥暴亂,北海被圍,孤窮無援,危在旦夕。以君有仁義之名,能救人之急。故北海區區,延頸恃仰,使慈冒白刃,突重圍,從万死之中自托于君,惟君所以存之。(我太史慈只是東萊一個無名之人。北海相孔融和我並不是有著骨肉相連的親族,也稱不上是志同道合的同鄉朋友,只是他認為我有前途而看重我,所以我有為他分擔災禍、共赴患難之義理。現在管亥起兵擾境,包圍北海城,城內居民徬徨無助,危在旦夕。孔融大人聽說劉備大人有仁義之名,能救人之危難急迫。因此盼望著能得到您的幫助,命令我突破管亥兵眾的包圍,冒著萬死無生的可能,來向劉備大人求助,惟有借重您的力量能使北海城脫危。)」劉備驚訝地答道:「孔北海知世間有劉備邪!(北海相孔融居然知道世間有我劉備啊!)」便立即派三千精兵隨太史慈去北海救援。黄巾军聞知援軍至,都四散而逃,孔融逐得以解圍。後袁紹攻公孫瓚,劉備與田楷東屯齊。

193年(32岁),曹操征討徐州,徐州牧陶謙敗退,曹操在徐州大屠殺。陶謙遺使告急於田楷,田楷與劉備俱前往相救。當時劉備自有士兵千餘人及幽州烏桓攙雜胡族騎兵,又略得饑民數千人。既到,與陶謙將領曹豹屯在郯東,被曹操擊敗。後曹操因後方生事而撤退,陶謙以丹楊兵四千人給劉備,劉備遂離開田楷,依附陶謙。陶謙表劉備為豫州刺史,屯兵於小沛。

194年(33岁),陶謙病重,對別駕從事麋竺說:「非劉備不能安此州也。」陶謙死後,麋竺便率徐州人民迎劉備入主徐州,劉備未敢當。下邳陳登對劉備說:「今漢室陵遲,海內傾覆,立功立事,在於今日。彼州殷富,戶口百萬,欲屈使君撫臨州事。(現今漢室漸趨衰敗,海內傾覆,立功名、立事業,就在於今日。本州殷實富足,戶口百萬,希望屈就使君親臨撫牧本州事務。)」劉備說:「袁公路近在壽春,此君四世五公,海內所歸,君可以州與之。(袁公路就近在壽春,此人為四世代有五人為三公,海內民心所歸,你可以徐州給與他。)」陳登答:「公路驕豪,非治亂之主。今欲為使君合步騎十萬,上可以匡主濟民,成五霸之業,下可以割地守境,書功於竹帛。若使君不見聽許,登亦未敢聽使君也。(袁術驕縱橫豪,不是治理亂局之主。現在希望您使君合共步兵騎兵十萬,對上可以匡扶主上、救濟人民,成就像春秋五霸之功業;對下可以割地自守、保境安民,寫下功業於竹帛上。若不見聽使君答許,在下亦未敢聽從使君。)」北海相孔融對劉備說:「袁公路豈憂國忘家者邪?冢中枯骨,何足介意。今日之事,百姓與能,天與不取,悔不可追。(袁公路豈是因憂慮國事而忘卻家庭之人?墓中之枯骨,不足以在意。今日之事情,是百姓讓與賢能,天意讓與你而不取,後悔不可追。)」劉備遂領徐州牧。

195年(34岁),吕布被曹操打敗來投靠,劉備善待禮遇他。吕布見劉備,極為尊敬,說:「我與卿同邊地人也。布見關東起兵,欲誅董卓。布殺卓東出,關東諸將無安布者,皆欲殺布爾。(我與你同為邊地出身的人(呂布出身五原郡,劉備出身涿郡,皆屬漢朝疆界北方邊境之地)。我見關東諸侯起兵,想要誅殺董卓。後來我殺董卓向東走,關東諸將卻沒有一個安置我,更加要殺死我啊。)」請劉備於帳中坐,並令妻子行禮,酌酒飲宴,又稱呼劉備為其弟。劉備見吕布胡言亂語,表面上雖不當一回事而心裏卻對其有所戒備。最後劉備仍讓吕布屯於小沛。

建安元年(196年,35岁),袁术來攻徐州,劉備於盱眙、淮陰抵抗袁軍。曹操上表朝廷,劉備成為鎮東將軍,封為宜城亭侯。劉備與袁术相持經一個月,大戰互有勝負。吕布乘下邳之虛,趁機偷襲。下邳守將曹豹倒戈,迎接呂布,趕走張飛,佔據下邳。吕布擄獲劉備妻子,劉備轉戰海西。東漢建安二年(197年)夏天,楊奉、韓暹等賊軍在徐、揚二州之間作惡,劉備與其決戰,盡為劉備所斬首。後來劉備向吕布求和,吕布歸還其妻子。劉備遺派關羽守下邳。

劉備還軍小沛,恢復集合兵馬得萬餘人。吕布嫌惡於此,自行出兵攻打劉備,劉備兵敗走投歸順曹操。曹操厚待禮遇劉備,以其為豫州牧。劉備與曹操一同返回許都後,被任命為左將軍。劉備來投奔,曹操謀士程昱就曾警告「觀劉備有雄才而甚得眾心,終不為人下」,勸曹操趁早解決後患,但曹操認為當時是收英雄之時,不可失天下之心。

(198年37岁)春天,吕布派人攜金到河內買馬,但被劉備兵所掠取。吕布於是派高順、張遼等攻劉備,雖然曹操曾派夏侯惇前往解救,但仍敗陣,劉備妻子又被吕布所擄。十月,曹操親自東征吕布,劉備在梁國界中與曹操相遇,便合兵成功消滅吕布。劉備復得妻子,跟從曹操還師許都。曹操表劉備為左將軍,禮之愈重,出則同車,坐則同席。

漢獻帝因曹操挾天子以令諸侯,發出衣带詔令其岳父董承誅殺曹操,劉備尚未加入。一日,曹操宴請劉備,對劉備說:「今天下英雄,唯使君與操耳。本初之徒,不足數也。(當今天下英雄唯獨是你與我,袁紹這類人稱不上)」劉備聽心中一震,筷子從手中掉落。此時剛好打雷,劉備便對曹操說:「『聖人迅雷風烈必變』,良有以也。一震之威,乃可至於此也!(『即使是圣人遇见打雷也会改变表情』,確有原因。一聲雷鳴,乃可以令我變成如此!)」《華陽國志》記載當時碰巧雷聲大作,劉備便把自己的失態歸咎於雷鳴,而此事後,劉備便加入董承。不久,在南方失利的袁术想北投袁紹,劉備便向曹操借兵出擊袁术,趁机摆脱曹操的控制。曹操便派他督朱靈、路招攻擊袁术,但軍未到,袁术已病死。

199年(38歲),劉備遣朱靈、路招佔據下邳。200年(39歲),反曹事迹敗露,董承被殺。劉備便殺死徐州刺史車冑,留關羽守下邳,自己回守小沛,另一方面派遣孫乾與袁紹連合,打出對抗曹操的名目。曹操曾派劉岱、王忠領軍攻打劉備,但不克。同時,東海昌霸反叛,郡縣多投靠劉備,劉備軍再次聚起數萬人,並連同多個地方勢力一起反曹。曹操決定親自東征劉備,雖然曹軍中將領多認為袁紹才是大敵,但曹操卻覺得劉備是英傑,必要先行討伐,郭嘉亦贊同曹操。

最後劉備大敗,小沛被佔,曹操虜獲劉備妻子及生擒關羽、夏侯博。劉備逃至青州,青州刺史袁譚親自迎接,並報知其父袁紹,袁紹出鄴城200里迎接。刘备泄露曹操曾经对自己说的密言予袁绍,袁绍才知道曹操原来有针对自己的阴谋。刘备待了一個多月後,以前的部下又重新聚會。不久,曹操與袁绍於官渡交戰,汝南郡黃巾餘軍劉辟等响應袁绍叛曹,袁绍便派劉備率軍與劉辟會合。曹操派曹仁攻打汝南,劉備惟有再次還軍袁绍。當時劉備想離開袁绍,便說服袁绍應南連劉表,袁绍再次派劉備到汝南與龔都會合。曹操另派蔡陽攻擊劉備,為劉備所殺。曹操於官渡之戰大敗袁绍。

建安六年(201年40歲),曹操又出兵南擊劉備,劉備便乘機放棄汝南,入荊州投靠劉表。劉備並派麋竺、孫乾與劉表會面。劉備到達荊州,受到劉表熱情接待。劉表接納劉備後,便為他增加兵馬,屯兵於新野,守衛荊州北大門。建安七年(202年42歲),曹操與袁尚、袁譚大戰於黎陽,許昌空虛,奉劉表命令北伐曹操。曹將夏侯惇、于禁、李典等人率軍南下,劉備奉命北上迎敵。在新野北博望,劉備設好伏兵以後,便燒毁營屯假裝懼敵退卻。夏侯惇讓李典留守,自己和于禁追擊,追到博望,劉備伏兵將夏侯惇殺得大敗,曹軍損失慘重,向北退走。劉備在荊州聲望日高,引起劉表疑心劉備,處處戒備。

建安十二年(207年46歲),曹操基本統一黃河流域之後,開始北上征伐北方烏丸,刘备力勸刘表乘機袭取许都,刘表没有採纳劉備建議。

劉備在荊州幾年,知道水鏡先生就是司馬徽,便前去請教世事。司馬徽知道劉備來意,便對他說:「儒生俗士,豈識時務?識時務者為俊傑。此間自有卧龍、鳳雛。(一個儒生見識淺俗之士,豈會認識時勢事務?認識時勢事務者,是那些英俊豪傑。從此地中,有臥龍(诸葛亮)、鳳雛(龐統)。)」亮又受徐庶推薦,劉備希望徐庶引亮來見,但徐庶卻建議:「此人可就見,不可屈致也。將軍宜枉駕顧之。(此人只能前去拜謁,不可委屈他前來。將軍宜枉屈尊駕以顧望。)」

207年(46岁),劉備三顧茅廬,問計於諸葛亮:「漢室傾頹,奸臣竊命,主上蒙塵。孤不度德量力,欲信大義於天下,而智術淺短,遂用猖獗,至於今日。然志猶未已,君謂計將安出?(漢室衰敗,奸臣掌權,使天子(漢獻帝)蒙受苦難。我不自量德行與能力,欲伸張大義於天下,然而智術淺薄,時至今日,一無所成。然則志向仍未減,先生可以出謀畫策嗎?)」諸葛亮遂向他陳述三分天下之計,分析此時曹操挾天子而令諸侯,此誠不可與爭鋒;孫權據有江東,可以爲援而不可圖;又詳述荊州用武之國、戰略要地,而其主劉表不能守,此恐怕是上天賜予劉備;益州是漢高祖成就帝業之地,其主劉璋闇弱;更建議劉備等待時局有變,由荊州、益州進攻中原。這篇論說後世稱為《隆中對》,是此後數十年劉備和蜀漢基本國策。。諸葛亮剛從隆中出來,受到劉備重視,只是由於劉備與自己情好日密,就引得「關羽、張飛等不悅」,最後還是劉備出來說:「孤之有孔明,猶魚之有水也。願諸君勿復言。(我有孔明,猶如魚得到水。但願諸君勿再說。)」;關羽、張飛才作罷。劉備在荊州擴軍,諸葛亮籌措軍需,何宇度《益部談資》記載:「先主寓荊州。從南陽大姓晁氏貸錢千𦻼,以為軍需。諸葛孔明作保,券至宋猶存。」

208年(47歲),曹操南下,時劉備屯於樊城。八月劉表病卒,次子劉琮繼任荊州牧,遣使曹操舉州投降。起初劉備不知劉琮決定投降,得知時曹軍尚在宛縣,尚未到達新野,劉備連忙棄城南撤。。在南渡漢水至襄陽時,諸葛亮曾勸劉備攻劉琮奪襄陽,但劉備不忍心進攻劉表之子,沒有攻打襄陽,只是在城下駐馬高呼劉琮出來相見,只來到劉表墓前祭奠,涕泣拜辭而去。劉備一行南下,荊州官吏百姓加入,走到當陽時,人數達10餘萬,輜重數千輛,一日只能走10幾里。惟有另派關羽乘數百艘船,直到江陵。有人向劉備進言:「宜速行保江陵,今雖擁大眾,披甲者少,若曹公兵至,何以拒之?(適宜速行而保江陵,現今雖然擁有很多隨行者,但士兵很少,若曹操軍追至,如何抵抗?)」劉備答道:「夫濟大事必以人為本,今人歸吾,吾何忍棄去!(做大事必以人為本,現今人眾歸附於我,我又如何忍心離棄而去!)」

當時江陵貯有劉表的大量糧儲、器械等軍實,曹操深怕劉備先佔領江陵,就拋棄輜重,以輕軍急行到襄陽。曹操聽聞劉備軍已離開襄陽,與曹純等領五千精騎急追,一日一夜疾行三百餘里。曹軍五千輕騎奔至當陽長坂坡追上劉備一行,劉備棄妻子,與諸葛亮、張飛、趙雲等數十騎走,10餘萬眾土崩瓦解,曹軍大舉擒獲劉備人眾輜重,張飛率20騎拒後,與曹兵邊打邊退。孫權之前派出魯肅來打探消息,在當陽長坂迎堵劉備。長坂會面後,魯肅隨劉備向東南斜趨漢津,在此適逢與關羽水軍會合,渡過沔水後向江夏進發。江夏太守劉琦聞劉備軍到來,率軍前去迎接,將劉備迎到夏口。此後,魯肅返回江東覆命,劉備進至樊口,同時派諸葛亮隨魯肅出使孫權,與孫權結盟。

孫權正式任命周瑜為左都督,程普為右都督,魯肅為贊軍校尉,率三萬水軍,與諸葛亮一起溯江西上,與樊口劉備軍會合。建安十三年冬,曹操親率20餘萬大軍從江陵順江東下,討伐孙权。黃蓋便向周瑜建議說:「今寇眾我寡,難與持久。然觀操軍船艦首尾相接,可燒而走也。」十二月,孫劉聯軍在赤壁至烏林一線以火攻大破曹軍,更追至南郡,曹操敗北。曹操一到江陵,便部署征南將軍曹仁、橫野將軍徐晃守江陵,折衝將軍樂進守襄陽,曹操撤回北方。

赤壁之戰後,劉備撤出江陵戰鬥,全力占據荊州江南四郡,先上表漢帝奏請劉琦為荊州刺史,兩萬大軍南下,武陵太守金旋獻城、長沙太守韓玄迎降、桂陽太守趙範讓位、零陵太守劉度稽顙。廬江人雷緒也率部曲數萬人投效。建安十三年(208年48歲)十二月,荊州江南四郡盡為劉備所占領。劉琦死,群下推劉備為荊州牧,劉備即遣諸葛亮為軍師中郎將,督令零陵、桂陽、長沙三郡,收其租賦,以供軍實,又以關羽為襄陽太守、蕩寇將軍駐江北,張飛為宜都太守、征虜將軍在南郡,趙雲為偏將軍領桂陽太守。廖立為長沙太守,郝普為零陵太守,向朗督秭歸、夷道、巫縣、夷陵四縣軍民事。劉備治於公安。而孫權為與劉備建立更鞏固的關係,在周瑜死後便依魯肅之策將南郡、江陵借給劉備,再分部份長沙郡給他,以及確認劉備佔有武陵和桂陽兩郡,遂提出將其妹嫁予劉備,史稱孫夫人。劉備到京口見孫權,關係表現親密、寬度。時劉備擁有荊州大部份屬地,又收取荊襄名士龐統和馬良,整日操練人馬,伺機南征北伐。

以後,孫權曾派使希望與劉備一起取益州,劉備本想答應,因東吳不可能越荆州而有蜀,蜀地就可據為己有。但荊州主簿殷觀卻反對:「若為吳先驅,進未能克蜀,退為吳所乘,即事去矣。今但可然贊其伐蜀,而自說新據諸郡,未可興動,吳必不敢越我而獨取蜀。如此進退之計,可以收吳、蜀之利。(若我們為吳開路,前進未必能攻克蜀地,後退可能為吳乘虛而入,那時即大勢而去。現今但可以贊同他伐蜀,而自己推卻說剛佔據荊南諸郡,未能興兵妄動,吳必定不敢越過我境而單獨取蜀。依照此進退得宜之計謀,便可以收吳、蜀兩地之利。)」劉備依從其計,孫權果然終輟計劃。殷觀遂升遷為別駕從事。

建安十六年(211年51歲)三月,曹操下令鍾繇率軍西征漢中張魯,讓夏侯淵出河東與鍾繇相會。益州牧劉璋遙聞曹操將遺鍾繇等向漢中討張魯,內心懷有恐懼。別駕從事蜀郡張松說服劉璋稱:「曹操兵強,無敵於天下,若因張魯之資源用以攻取益州土地,誰能抵禦?」劉璋說:「我固然擔憂,而未有計。」張松說:「劉備,使君之宗室,而且是曹操之深仇,善於用兵,若使之討伐張魯,張魯必可攻破。張魯攻破,則益州強大,曹操雖來,也無能為力。」在張松出言下,益州牧劉璋採納請劉備入蜀之意見,並派軍議校尉法正為使,孟達為副,各領兵2,000人,前往荊州邀請劉備入蜀助攻張魯。劉備親自統帥進軍益州,龐統任軍師中郎將,將領黃忠、魏延、卓膺等輔助劉備。劉備與龐統一同進入益州。諸葛亮、關羽、張飛、趙雲、劉封、孟達、馬良等留在荊州。然而劉備要知道蜀中的闊狹,兵器、府庫、人馬多少及多個要害之地的遠近,便向二人請教,張松、法正都一一詳述,更畫出地圖指示山川所在,所以劉備知道益州內情。

到達涪城,劉璋親自出迎,相見甚歡。張松、法正及龐統都提議劉備可乘機殺了劉璋,當時龐統主張趁此機會,擒住劉璋。劉備以初來到蜀地,人心尚未信服,不宜輕舉妄動為由,未採納龐統建言。劉璋推薦劉備行大司馬,領司隸校尉,劉備又推薦劉璋行鎮西大將軍,領益州牧。劉璋配給劉備士兵,及督白水軍,令他攻擊張魯。劉備當時總計有三萬多人,車甲、器械、資貨甚多。但劉備卻到葭萌時,未出兵,而是樹立恩德,收買民心。

建安十七年(212年51歲)冬十月,曹操出兵攻打孫權,孫權向劉備告急,劉備對劉璋說欲還救荊州有急。劉備請求劉璋撥出兵士萬人與軍事物資。但劉璋只允諾給予四千兵馬,其餘物資僅提供一半。劉備受此激怒,忿忿說道:「我為了益州征討強敵,軍隊勤瘁,無暇休息;現今劉璋積存起財富而不用於賞功,卻希望士大夫能為他出力死戰,這又怎可能!」當時張松不知劉備用意,寫信質問:「眼看就要大事底定,為何拋下一切離去?」結果被其兄張肅據此告密,張松遭到處死,導致劉備與劉璋關係惡化。十二月,劉備與劉璋決裂。劉備依龐統提出的計謀,召白水关守将杨怀、高沛到來並將其斬殺。另外又派黃忠、卓膺率軍攻劉璋,一路佔領至涪城。劉璋連忙派出劉璝、冷苞、張任、鄧賢、中郎将吴懿等與對抗劉備,皆破败,退保绵竹,吴懿至刘备军前投降,拜为讨逆将军。刘璋后遣护军李嚴、参军费观督绵竹军拒刘备,两人陣前倒戈亦率众投降,同拜裨將軍,劉備軍勢強,分軍平定各郡縣。但劉備軍卻被雒城守將劉循阻擋攻勢。從建安十八年建安十九年,劉備圍攻雒城將近一年,龐統被流矢射中,重創身亡。張飛、趙雲、劉封等隨諸葛亮率軍入蜀,關羽留下鎮守荊州,馬良、麋芳、士仁、廖化協助關羽鎮守荆州。建安十九年(214年54歲)夏,諸葛亮入蜀援軍溯江而上。諸葛亮分兵進攻成都:張飛從墊江北上直取巴西郡治閬中,從北面攻成都;趙雲從長江西攻取江陽北上犍為郡治武陽,從南面攻成都;諸葛亮親自沿涪江取德陽,直取成都。

214年夏天(53岁),雒城終被攻破。李恢受劉備派遣到漢中與馬超交好,馬超正想離開張魯,劉備暗暗派出人馬與馬超兵眾會合,馬超率領大隊人馬開到成都城北屯駐。關羽聽說馬超歸降備,便寫信給諸葛亮,問馬超才能可與誰相比,諸葛亮回信說:「馬超文武兼備,氣概雄烈,過於常人,可稱得上一世之豪傑,是黥布、彭越一流之人物,可以與張飛相提並論,但是趕不上美髯公你超逸絕群。」劉備乘勢率漢軍進圍成都數十日。劉備派簡雍進入成都勸說劉璋投降,劉璋與簡雍「同輿而載,出城歸命」;劉璋向劉備繳械投降,益州易主,歸屬劉備。由於蜀中繁盛、安樂,劉備便設宴大慰勞士卒,又取蜀城中的金銀,分賜將士,還其谷帛。劉備皆處之顯任,盡其器能,有志之士,無不競勸,益州之民,是以大和。有議論勸劉備將成都城中房舍及城外園地桑田分賜給諸將,但趙雲反駁說:「從前漢朝大將霍去病曾說匈奴未滅,無用家為,何況現在國賊不只像匈奴只有一個,還不到可以安定下來的時候,必須等到天下的亂賊都平定之後,才可讓眾人返回家鄉去種植桑梓,回歸故土去耕作田地,這樣才是正道。益州的人民是第一次遭遇到戰爭,應該將田宅房產歸還給百姓,先讓他們安居樂業,然後才能叫他們服兵役,納錢糧,也才能得到益州的民心。」劉備便聽從趙雲的建議,有志之士便都紛紛來投。

在建安二十年(215年54歲)三月,曹操征伐漢中,七月破南鄭,十一月最終降服張魯,搶在劉備之前占有漢中。孫權以劉備已得益州,派人討還荊州,劉備答道:「須得涼州,當以荊州相與。」孫權忿恨,乃派遣呂蒙施襲,爭奪長沙、零陵、桂陽三郡。劉備率兵五萬到公安,下令關羽進軍益陽,與孫軍對峙。時曹操勢破張魯,威脅蜀地。劉備遂派使者向孫權議和,孫權派諸葛瑾答覆劉備,雙方和好。為盡快解決荊州問題,回兵保衛益州,劉備以湘水為界,將江夏、長沙、桂陽三郡劃給孫吳。南郡、零陵、武陵以西屬劉備所有,劉回軍江州。八月,孫在東線進攻合肥,曹將張遼、李典據城抵抗,擊退孫權。十一月,張魯逃遁至巴西,偏將軍黃權對劉備說:「若失漢中,則三巴不振,此為割蜀之股臂也。」又遣黃權率兵迎向張魯,但張魯已降曹操。曹操派夏侯淵、張郃屯兵漢中,數次武力侵犯巴郡邊界。劉備令張飛進兵宕渠,與張郃等於瓦口爭戰,大敗張郃等。張郃收兵還退南鄭。翌年二月,曹操留夏侯淵、張郃鎮守漢中,自己回鄴城。

建安二十三年(218年57歲),劉備採法正勸諫率軍進攻漢中。諸葛亮鎮守成都,劉備親率大軍征漢中,法正隨從參謀軍機,趙雲、黃忠、魏延、張飛、馬超、吳蘭等從征,曹操、劉備爭奪漢中之戰開始。但漢軍先頭部隊卻被曹軍打敗。劉備一路直攻漢中,進兵至陽平關與夏侯淵、張郃等曹軍對峙,為保證道路暢通,劉備派大將陳式率10餘營兵士駐紮在馬嗚閣道,曹將夏侯淵派大將徐晃襲擊陳兵,陳式軍被打敗,士兵紛紛跳入山谷,傷亡慘重。曹操下令賜徐晃節杖,並說:「此閣道,漢中之險要喉也。劉備欲斷絕外內,以取漢中。將軍一舉,克奪賊計,善之善者也。」劉備「急書發益州兵」,諸葛亮與從事楊洪商議對策,楊洪說:「漢中則益州咽喉,存亡之機會,若無漢中則無蜀矣,此家門之禍也。方今之事,男子當戰,女子當運,發兵何疑!」;諸葛亮非常看重楊洪見識,當即發兵支援漢中前線。從建安二十二年(217年)劉備出兵起,雙方在漢中僵持一年多,建安二十四年(219年)春劉備聽從法正計策,從陽平南渡沔水,依定軍山恃險安營。夏侯淵帶少數兵力爭奪定軍山營地,法正對劉備說:「可擊矣!」;劉備便命黃忠乘高鼓噪攻之,居高臨下,衝入敵陣,殺死夏侯淵。黃忠斬殺夏侯淵及曹操所置的益州刺史趙顒等。建安二十四年三月(219年58歲),曹操自長安率兵經褒斜谷趕往漢中,劉備說:「曹公雖來,無能為也,我必有漢川矣。」劉備在險處死守,不與曹軍交戰。諸葛亮親坐益州,將人力、物力及時補充到劉備軍中。夏五月,曹操引兵撤出漢中,漢中歸劉備所有。而另一方面,又遣劉封、孟達、李嚴等進攻上庸,上庸守將申耽等見曹操率軍返回中原,逐開城投降。秋七月,馬超、龐羲、射援、諸葛亮、關羽、張飛、黃忠、法正、李嚴等120人聯名上表劉備為漢中王。劉備於沔陽設置祭壇場地,陳兵列眾,群臣陪位,宣讀奏訖,自立漢中王。後還治成都。提拔魏延為都督漢中太守,坐鎮漢中。劉備於是建起館舍,修築亭障,從成都至白水關,四百餘區。關羽率軍從江陵北上,發動襄樊戰役。于禁七軍火速增援曹仁,關羽與于禁交鋒,時至八月,大雨滂沱,山洪暴發,漢水驟漲,水淹七軍,于禁束手就擒,部下幾乎全部投降,副將龐德被活捉不降,最後被關羽所殺。孫權將呂蒙白衣渡江。十月,呂蒙任征荊州大督,率兵西上,公安士仁、江陵麋芳開城投降。關羽回軍江陵途中,陸遜任右護軍、鎮西將軍屯駐夷陵,呂蒙任南郡太守駐江陵。關羽至當陽西保麥城,敗走麥城後,士兵繼續逃散,關羽身邊只剩十餘騎。十二月關羽被孫權大將潘璋部馬忠捕殺,孫權將其首級送至洛陽曹操處。孫劉聯盟正式決裂。

建安二十五年(220年59歲)正月,曹操逝世,劉備也曾派遣韓冉奉書弔唁,「並致賻贈之禮」,但最後卻失敗。三月改元延康,十月曹丕代漢稱帝。十二月,當時有謠言指漢獻帝劉協已被加害,劉備便穿喪服發喪,諡劉協為孝愍皇帝(但實際上劉協仍在世)。同年,法正、黃忠去世。

221年,群臣勸劉備登基為帝,劉備不答應,諸葛亮用耿純遊說劉秀登基故事勸劉備(光武帝劉秀登基時,同為漢室的更始帝劉玄仍在世,此後綠林軍攻破長安殺劉玄,此後劉秀建東漢),劉備才決定接受群臣擁立,四月初六在成都武擔山之南接受皇帝璽綬,改元章武。諸葛亮、許靖、黃權等人上書勸劉備即帝位,國號仍為「漢」,史稱蜀漢。四月丙午日(5月15日),大赦天下,改元章武。以諸葛亮為丞相,許靖為司徒。設置百官,建立宗廟,祭祀漢高祖以下。五月,立皇后吳氏,劉禪為皇太子。六月,以劉永為魯王,劉理為梁王。

魏文帝曹丕召集眾臣討論,侍中劉曄認為蜀漢一定要出兵攻打孫吳,理由是:「蜀雖狹弱,而備之謀欲以威武自強,勢必用眾以示其有餘。且關羽與備,義為君臣,恩猶父子;羽死不能為興軍報敵,於終始之分不足。」七月,劉備不采纳赵云等人劝告,率軍沿江而下,討伐東吳。張飛被部下暗殺。孫權先派人給蜀漢送信求和,又令諸葛亮哥哥諸葛謹致箋勸劉備息兵罷戰,劉備一概拒絕。孫權把國都從建業遷到武昌,以便指揮戰爭。起初,漢軍氣勢如虹,不過吳將陸遜採以逸待勞兵法而戰之,於章武二年(222年)大敗漢軍。陸遜大敗劉備,「殺其兵八萬餘人,備僅以身免」。劉備退至秭歸,趙雲率兵到達白帝城,巴西太守閻芝派馬忠率5千人馬隨後到達。劉備退到永安縣。孫權聽聞劉備住白帝,甚為懼怕,遣使請和。章武二年十二月,孫權派太中大夫鄭泉到白帝城見劉備,正式表示向蜀漢請和。劉備也遣太中大夫宗瑋使吳,表示贊同蜀漢、東吳兩國和好。

當劉封失掉漢中東面三郡逃回成都後,諸葛亮勸劉備除掉劉封。漢嘉郡太守黃元聽說劉備在永安病重,於章武二年十二月舉兵反叛。同年,太傅許靖、尚書令劉巴、驃騎將軍馬超先後病逝。南中越夷高定曾向新道進攻,被李嚴打退。

章武三年(223年62歲)二月,諸葛亮接到劉備詔書,帶著劉永、劉理從成都來到永安。三月,黃元又乘諸葛亮到永安見劉備之機,率軍進攻臨邛縣,火燒臨邛城。益州治中從事楊洪立即把黃元之動向報告給劉禪,劉禪派將軍陳曶、鄭綽進討黃元。陳曶、鄭綽兩人在南安峽口生擒黃元,將其押回成都正法。四月下旬,劉備對諸葛亮說:「君才十倍曹丕,必能安國,終定大事。若嗣子可輔,輔之;如其不才,君可自取。(你的才能是曹丕的十倍,必定能夠安定國家,終可成就大事。如果嗣子(劉禪)可以輔助,便輔助他;如果他沒有才幹,你可以自取其位。)」諸葛亮涕泣說:「臣敢竭股肱之力,效忠貞之節,繼之以死!(臣必定竭盡自己所有力量,報效忠貞之氣節,繼續至死為止!)」劉備又要劉禪和其他兒子「與丞相從事,事之如父。」。劉備臨終前託孤於丞相諸葛亮,尚書令李嚴為副。臨終時,與劉永說:「吾亡之後,汝兄弟父事丞相,令卿與丞相共事而已。(我死後,你們兄弟要對父親般奉事丞相(諸葛亮),你們與丞相只是共事而已。)」。四月廿四(6月10日) 劉備崩逝於永安宮,享壽六十二歲。孫權派立信都尉馮熙出使蜀漢,弔唁劉備。諸葛亮上言讚揚劉備。五月癸巳日(6月21日),遺體自永安運返成都发丧,諡為昭烈皇帝。八月,入葬惠陵。

亦有郭沫若等学者认为由于条件所限,刘备就地下葬于今奉节县,并未归葬成都。

《三国志·先主传》中并没有记载刘备庙号。李慈铭怀疑刘备庙号烈祖是由刘渊所追尊。章学诚根据《三国志·先主传》中诸葛亮宣读的遗诏,指出刘备庙号是太宗。卢弼认为章学诚的说法不足据,如果刘备庙号太宗,《三国志》本传没有不记载的道理。郭善兵则认为刘备庙号缺失不能归咎于史书记载疏漏,而是受到郑玄礼学“一祖二宗与四亲庙”七庙学说影响所致。

刘备喜怒不形于色,常以谦虚恭敬待人,深知「得人心者得天下」的道理,重視以寬仁厚德待人,與那些殘民以逞、暴虐嗜殺的軍閥判然有別,因此而爭取到了人心。刘备不怎么喜欢读书,喜欢評馬論犬、音乐和華美的衣物。

小时候,家中有棵大桑树,遙望見如同车盖,刘备與宗中小兒於樹下玩耍時說過:「吾必當乘此羽葆蓋車。(我必定會乘坐此羽飾華蓋之車。)」叔父聽到後,不禁當下斥責他:「汝勿妄語,滅吾門也。(你不要胡說,會招來滅門之禍)」

劉備在部下聲譽受損或是特殊的理由發生背叛的可能時往往站出來捍衛部下聲譽和保護部下家眷,徐庶母被抓,庶淚崩辭別劉備、糜芳背叛,劉備對愧疚的糜竺說兄弟罪不相及、夷陵之敗黃權不得已降魏,劉備依然善待其家人「孤負黃權,權不負孤也。」

劉備由於沒有鬍鬚,因此曾被张裕取笑。有一次,劉備與劉璋於涪縣會面時,張裕時為璋從事,在一旁陪坐。由於裕的鬍鬚濃密被備嘲笑說道:“過去吾在涿縣時遇到好多姓毛的人,四方許多毛,涿縣縣令聲稱說:「許多毛(毛)繞涿(歜)而居。」,但裕立刻反唇相譏:“過去某人當上黨郡潞縣長,後來升任為涿縣縣令,其辭任歸家時,有人寫信給其,要是寫了潞縣就丟了涿縣,而寫上涿縣又失去潞縣,就寫道「潞(露)涿(歜)君」。因此裕就用此方法反譏備。讓劉備對張裕一直沒什麼好感,劉備攻漢中之前,張裕說會出師不利,但劉備仍照著既定計畫出兵。結果劉備拿下漢中,不過兩名大將吳蘭、雷銅等也在此戰中身亡,以致於劉備記恨張裕,某天,張裕私下對人說:「庚子年間(220年)會改朝換代;主公入主蜀地的九年後,也會再次失去蜀地,劉氏運氣將會消盡。」謠言亂傳,最終入劉備之耳,劉備不滿張裕散布滅亡謠言,以張裕的話語沒有應驗,把他關入獄中。諸葛亮請求劉備寬恕他,劉備只說:「芳蘭生門,不得不鉏。」於是殺了張裕,棄屍於街頭。

劉備死前告誡其子劉禪的遺詔,其言辭懇切,令人莫不動容。文中,劉備勸劉禪最重要的一句話,便是「勿以惡小而為之,勿以善小而不為。惟賢惟德,能服於人。」古人教子,常以德為根基,因為唯有賢德之人,才能服人。

刘备一生争战,乍看之下胜少败多。劉備攻打益州時,趙戩曾言:“刘备拙于用兵,每战必败。”認為劉備不會用兵,沒本事拿下益州,傅幹卻說:「劉備得人心,又有諸葛亮、關羽、張飛等人傑輔助,怎會不濟呢?」結果劉備果真取攻佔益州。在荊州依附劉表時,曾建議劉表北伐曹操,劉表不接受。劉備住荊州數年,一次與劉表飲酒時起至廁所,見大腿贅肉生,慨然流涕。還坐,劉表奇怪問起劉備,劉備說:「我戎馬半生,常常身不離鞍,大腿贅肉皆消。今天不復騎馬,大腿贅肉生。日月若馳,老年快將至矣,而功業不能建立,是以為之悲嘆。」

刘备与诸葛亮的君臣际遇,通常被史家视为君臣之典范。三顾茅庐后刘备称得到诸葛亮是“鱼之有水”。诸葛亮在刘备尚在时,就已经为丞相录尚书事假节,张飞被暗杀后又领司隶校尉,集政治实际权力于一身,这在古代是很罕见的。刘备去世时举国托孤诸葛亮,被陈寿称为“君臣之至公,古今之盛轨”。

陳壽评曰:“先主之弘毅宽厚,知人待士,盖有高祖之风,英雄之器焉。及其举国托孤于诸葛亮,而心神无贰,诚君臣之至公,古今之盛轨也。机权干略,不逮魏武,是以基宇亦狭。然折而不挠,终不为下者,抑揆彼之量必不容己,非唯竞利,且以避害云尔。”(《三國志·蜀書·先主傳第二》)、「劉備天下稱雄,一世所憚」(《三國志·吳書·陸遜傳第十三》)。尽管刘备并非西晋认为的正统政权,陈寿在三国志内仍然坚持使用同帝王本纪接近的用词,例如在刘备本传称刘备先主,称讳且不直呼其名,去世用和崩相等的殂字。这与三国时代另一位君主孙权的处理手法是不同的。这可以体现陈寿对刘备的尊重,抑或是故国情怀。

刘元起:“吾宗中有此儿,非常人也。”(《三國志·蜀書·先主傳第二》)

陈登:“雄姿杰出,有王霸之略,吾敬刘玄德。”(《三國志·魏書·桓二陳徐衛盧傳第二十二》)

袁绍:“刘玄德弘雅有信义,今徐州乐戴之,诚副所望也。”(《三國志·蜀書·先主傳第二》)

程昱:“观刘备有雄才而甚得众心,终不为人下,不如早图之。” 、“劉備有英名,關羽、張飛皆萬人敵也”(《三國志·魏書·程郭董劉蔣劉傳第十四》)

郭嘉:「备有雄才而甚得众心。张飞、关羽者,皆万人之敌也,为之死用。(郭)嘉觀之,(劉)備終不為人下,其謀未可測也。古人有言:『一日縱敵,數世之患。』宜早為之所。」(《三國志·魏書·程郭董劉蔣劉傳第十四》)

曹操:“方今收英雄时也,杀一人而失天下之心,不可。”、“夫刘备,人杰也,今不击,必为后患,将生忧寡人。”、“刘备,吾俦也。但得计少晚。”(《三國志·魏書·武帝紀第一》)“今天下英雄,唯使君与操耳。本初之徒,不足数也。”(《三國志·蜀書·先主傳第二》)

曹丕:「備不曉兵,豈有七百里營可以拒敵者乎!『苞原隰險阻而為軍者為敵所禽』,此兵忌也。孫權上事今至矣。」(《三國志·魏書·文帝紀第二》)

刘晔:「明公(曹操)以步卒五千,將誅董卓,北破袁紹,南征劉表,九州百郡,十並其八,威震天下,勢慴海外。今舉漢中,蜀人望風,破膽失守,推此而前,蜀可傳檄而定。刘备,人傑也,有度而迟,得蜀日淺,蜀人未恃也。今破漢中,蜀人震恐,其勢自傾。以公之神明,因其傾而壓之,無不克也。若小緩之,諸葛亮明於治而為相,關羽、張飛勇冠三軍而為將,蜀民既定,據險守要,則不可犯矣。今不取,必為後憂。」、「蜀雖狹弱,而備之謀欲以威武自強,勢必用眾以示其有餘。且關羽與備,義為君臣,恩猶父子。羽死不能為興軍報敵,於終始之分不足。」(《三國志·魏書·程郭董劉蔣劉傳第十四》)

贾诩:「吳、蜀雖蕞爾小國,依阻山水,有雄才,諸葛亮善治國,孫權識虛實,陸議見兵勢,據險守要,汎舟江湖,皆難卒謀也。用兵之道,先勝後戰,量敵論將,故舉無遺策。臣竊料群臣,無備、權對,雖以天威臨之,未見萬全之勢也。昔舜舞干戚而有苗服,臣以為當今宜先文後武。」(《三國志·魏書·荀彧荀攸賈詡傳第十》)

孫盛:“刘备雄才,處必亡之地,告急於吳,而獲奔助,無緣復顧望江渚而懷後計。”(《三國志·蜀書·先主傳第二》)

诸葛亮:“刘公雄才盖世,据有荆土,莫不归德,天人去就,已可知矣。”(《三國志·蜀書·董劉馬陳董呂傳第九》)“刘豫州王室之胄,英才盖世,众士仰慕,若水之归海,若事之不济,此乃天也,安能復为之下乎!”(《三國志·蜀書·諸葛亮傳第五》)

关羽:“吾受劉將軍厚恩,誓以共死,不可背之。”(《三國志·蜀書·關張馬黃趙傳第六》)

趙戩:“刘备其不济乎?拙于用兵,每战则败,奔亡不暇,何以图人?”(《三國志·蜀書·先主傳第二》)

傅幹:“刘备宽仁有度,能得人死力。諸葛亮達治知變,正而有謀,而為之相;張飛、關羽勇而有義,皆萬人之敵,而為之將;此三人者,皆人傑也。以備之略,三傑佐之,何為不濟也?”(《三國志·蜀書·先主傳第二》)

孙权:“非刘豫州莫可以当曹操者。”(《三國志·蜀書·諸葛亮傳第五》)「猾虜乃敢挾詐!」(《三國志·吳書·周瑜魯肅吕蒙傳第九》)

周瑜:“刘备以枭雄之姿,而有關羽、張飛熊虎之將,必非久屈為人用者。”(《三國志·吳書·周瑜魯肅吕蒙傳第九》)

陸遜:「備干天常,不守窟穴,而敢自送……尋備前後行軍,多敗少成,推此论之,不足为戚。」、「備是猾虜,更嘗事多」、「劉備天下知名,曹操所憚,今在境界,此强对也。」、「斯三虏者(曹操、劉備、關羽)当世雄杰,皆摧其锋。」(《三國志·吳書·陸遜傳第十三》)

张松:“刘豫州,使君之宗室而曹公之深雠也,善用兵,若使之讨鲁,鲁必破。鲁破,则益州强,曹公虽来,无能为也。”「劉豫州,使君之肺腑,可與交通。」「今州中諸將龐羲、李異等皆恃功驕豪,欲有外意,不得豫州(劉備),則敵攻其外,民攻其內,必敗之道也。」

刘巴:“备,雄人也,入必为害,不可内也。”

彭羕:“仆昔有事於诸侯,以为曹操暴虐,孙权无道,振威闇弱,其惟主公有霸王之器,可与兴业致治,故乃翻然有轻举之志。”(《三国志·卷四十·蜀书十·刘彭廖李刘魏杨传第十》)

锺会:“益州先主以命世英才,兴兵朔野,困踬冀、徐之郊,制命绍、布之手,太祖拯而济之,与隆大好。”

杨戏的《季汉辅臣赞》中赞昭烈皇帝:“皇帝遗植,爰滋八方,别自中山,灵精是锺,顺期挺生,杰起龙骧。始于燕、代,伯豫君荆,吴、越凭赖,望风请盟,挟巴跨蜀,庸汉以并。乾坤复秩,宗祀惟宁,蹑基履迹,播德芳声。华夏思美,西伯其音,开庆来世,历载攸兴。”

诸葛亮上言於刘禅曰:“伏惟大行皇帝迈仁树德,覆焘无疆,昊天不吊,寝疾弥留,今月二十四日奄忽升遐,臣妾号咷,若丧考妣。乃顾遗诏,事惟大宗,动容损益;百寮发哀,满三日除服,到葬期復如礼;其郡国太守、相、都尉、县令长,三日便除服。臣亮亲受敕戒,震畏神灵,不敢有违。臣请宣下奉行。”(《三國志·蜀書·先主傳第二》)

裴潜:“使居中國,能亂人,不能為治。若乘邊守險,足為一方之主。”(《世說新語·識鑒第七》)(《三國志·魏書·和常楊杜趙裴傳第二十三》)

吕布:“是儿最叵信者。”(《三国志·卷七·魏书七·吕布臧洪传》)

吕布诸将:“备数反覆难养,宜早图之。”(《三国志·卷32》注引王沈《魏书》)

习凿齿曰:“先主虽颠沛险难而信义愈明,势偪事危而言不失道。追景升之顾,则情感三军;恋赴义之士,则甘与同败。观其所以结物情者,岂徒投醪抚寒含蓼问疾而已哉!其终济大业,不亦宜乎!”(《三國志·蜀書·先主傳第二》)

常璩曰:「先主名微人鮮,而能龍興鳳舉,伯豫君徐,假翼荊楚,翻飛梁、益之地,克胤漢祚,而吳、魏與之鼎峙。非英才名世,孰克如之!」(《華陽國志·劉先主志》)

裴松之:「漢武用虛罔之言,滅李陵之家,劉主拒憲司所執,宥黃權之室,二主得失縣(懸)邈遠矣。《詩》云『樂只君子,保艾爾後』,其劉主之謂也。」(裴松之注《三國志·蜀書·黃李呂馬王張傳第十三》)

张辅:“刘备威而有恩,勇而有义,宽宏而大略”(《藝文類聚卷二十二》)

朱敬则:“蜀先主抱英济之器,无角逐之材。远窜荆蛮,畏曹公之神武;奄有庸蜀,乘刘璋之政衰。国小人夷,风颓俗陋。”(《全唐文》)

杜甫:“蜀主窥吴幸三峡,崩年亦在永安宫。翠华想像空山里,玉殿虚无野寺中。古庙杉松巢水鹤,岁时伏腊走村翁。武侯祠堂常邻近,一体君臣祭祀同。”

劉禹錫:“天地英雄氣,千秋尚凜然。勢分三足鼎,業复五銖錢。得相能開國,生兒不像賢。淒涼蜀故妓,來舞魏宮前。”

王勃:“以先主之宽仁得众,张飞、关羽万人之敌,诸葛孔明管、乐之俦,左提右挈,以取天下,庶几有济矣。然而丧师失律,败不旋踵。奔波谦、瓒之间,羁旅袁、曹之手,岂拙于用武,将遇非常敌乎?”

司馬光:「昭烈之漢,雖云中山靖王之後,而族屬疏遠,不能紀其世數名位,亦猶宋高祖稱楚元王後,南唐烈祖稱吳王恪後,是非難辨,故不敢以光武及晉元帝為比,使得紹漢氏之遺統也。」(《資治通鑑·第六十九卷·魏紀一》)

苏洵:“项籍有取天下之才,而无取天下之虑;曹操有取天下之虑,而无取天下之量;玄德有取天下之量,而无取天下之才。”

苏辙:“世之言者曰:孙不如曹,而刘不如孙。刘备唯智短而勇不足,故有所不若于二人者,而不知因其所不足以求胜,则亦已惑矣。盖刘备之才,近似于高祖,而不知所以用之之术。昔高祖之所以自用其才者,其道有三焉耳:先据势胜之地,以示天下之形;广收信、越出奇之将,以自辅其所不逮;有果锐刚猛之气而不用,以深折项籍猖狂之势。此三事者,三国之君,其才皆无有能行之者。独一刘备近之而未至,其中犹有翘然自喜之心,欲为椎鲁而不能纯,欲为果锐而不能达,二者交战于中,而未有所定。是故所为而不成,所欲而不遂。弃天下而入巴蜀,则非地也;用诸葛孔明治国之才,而当纷纭征伐之冲,则非将也;不忍忿忿之心,犯其所短,而自将以攻人,则是其气不足尚也。嗟夫!方其奔走于二袁之间,困于吕布而狼狈于荆州,百败而其志不折,不可谓无高祖之风矣,而终不知所以自用之方。”

謝采伯:「孫權運籌於內,劉備、諸葛亮、周瑜、關侯等,合謀並智,方拒得曹操,敗之於赤壁,亦未為竒政縁。」(《密齋筆記·卷二》)

何去非:“方其豪杰并起,而备已与之周旋于中原矣。始得徐州而吕布夺之,中得豫州而曹公夺之,晚得荆州而孙权夺之。备将兴复刘氏之大业,其志未尝一日而忘中州也。然卒无以暂寓其足,委而西入者,有曹操、孙权之兵轧之也。”

萧常:“昭烈父子以帝室支属,介在一隅,而正位号,尚数十年,由先汉至是,垂祀五百,三代以还,葢未之有。人主之结人心,其效廼尔,有大物者,庸可忽诸。”(《萧氏续后汉书》)

郝经:“汉得天统,莽簒而在光武,操窃而在昭烈。魏吴虽僣,犹夫吴楚也。昭烈天资仁厚,宇量(阙)毅,岿然一世之雄。以兴复汉室为己任,崎岖百折,偾而益坚。颠沛之际,信义逾明。故能终系景命,信大义于天下。任贤使能,洒落诚尽,使诸葛亮以死自效。复见三代君臣,高、光为不亡矣。国贼未讨,境土未复,而偾军崩殂,哀哉!”(《郝氏续后汉书》)

陶宗仪:“备又非人望之所归。周瑜以枭雄目之,刘巴以谁人视之,司马懿以诈力鄙之,孙权以猾虏呼之。”(《南村辍耕录》卷二十五)

杨璟:“昔据蜀最盛者,莫如汉昭烈。且以诸葛武侯佐之,综核官守,训练士卒,财用不足,皆取之南诏。然犹朝不谋夕,仅能自保。”

孙承恩:“贤矣昭烈,宽厚弘毅。崎岖立国,仗信履义。推诚任贤,肝胆孚契。顾命数词,可训后世。”(《文简集·卷三十八》)

王夫之:“刘先主以汉室之裔,保蜀土,奉宗祧,任贤图治,民用乂安,尚矣。”(《宋论·卷一·太祖》)

毛泽东:“刘备的优点主要于是善于用人,善于团结各方人士。而缺点则表现在两个方面:一是好感情用事;二是不能区分主次矛盾。”

趙翼:「關、張、趙雲自少結契,終身奉以周旋,即羈旅奔逃,寄人籬下,無寸土可以立業,而數人者患難相隨,別無貳志,此固數人者之忠義,而備亦必有深結其隱微而不可解者矣。」(《廿二史劄記·卷七》)

歷史學家、《三國史話》作者呂思勉認為,如其通觀前後,則劉備急於併吞劉璋,實在是失敗之遠因。倘使劉備老實一些,替劉璋北攻張魯,這是可以攻下;張魯既下,而馬超、韓遂等還未全敗,彼此聯合,以擾中原,曹操倒難於對付;劉備心計太工,不肯北攻張魯,而要反噬劉璋,以至替曹操騰出平定關中和涼州之時間,而且仍給以削平張魯之機。然而本可聯合涼州諸將共擾關中,卻變做獨當大敵。伐吳之役,劉備因為是能做一番事業,意志必較堅定,理智必較細密,斷不會輕易動於感情;況且感情必是動於當時,時間稍久,感情就漸漸衰退,理智就漸漸清醒。然其禍根,亦因急於要取益州,以致對於荊州不能兼顧之故;所以心計過工,有時也會成為失敗原因,真個閲歷多之人,倒覺得凡事還是少用機謀,依著正義而行好。

《劉備傳》作者張作耀認為,劉備人生道路危機四伏、滿途坎坷。這是一個戰鬥歷程:起步、挫折、爬起、再挫,發展至立足一方。撇開劉備政治動機不談,折而不撓、敗不氣餒、終不為下,為憧憬之目標而奮鬥不懈,始終如一。劉備與關羽、張飛一經結義,終身不易。與下士同席而坐,無所簡擇;善待部下,士卒感恩,願為驅使。

\subsubsection{章武}

\begin{longtable}{|>{\centering\scriptsize}m{2em}|>{\centering\scriptsize}m{1.3em}|>{\centering}m{8.8em}|}
  % \caption{秦王政}\
  \toprule
  \SimHei \normalsize 年数 & \SimHei \scriptsize 公元 & \SimHei 大事件 \tabularnewline
  % \midrule
  \endfirsthead
  \toprule
  \SimHei \normalsize 年数 & \SimHei \scriptsize 公元 & \SimHei 大事件 \tabularnewline
  \midrule
  \endhead
  \midrule
  元年 & 221 & \tabularnewline\hline
  二年 & 222 & \tabularnewline\hline
  三年 & 223 & \tabularnewline
  \bottomrule
\end{longtable}


%%% Local Variables:
%%% mode: latex
%%% TeX-engine: xetex
%%% TeX-master: "../../Main"
%%% End:

%% -*- coding: utf-8 -*-
%% Time-stamp: <Chen Wang: 2021-11-01 11:36:18>

\subsection{后主劉禪\tiny(223-263)}

\subsubsection{生平}

劉禪(207年-271年),字公嗣,又字升之。蜀漢昭烈帝劉備之子,蜀漢最後一位皇帝,史學家称蜀漢後主,223年—263年在位,歷時四十一年,是三國在位最長之皇帝。

据《三国志》记载,劉禪由刘备的妾室甘夫人所生,是刘备三位庶子中最为年长的。

212年(建安十七年),刘备入蜀,孙权派人接回孫夫人,孫夫人想将五歲的刘禅一并带走,诸葛亮派遣赵云夺回。

刘禅继位初期确实听从父亲的遗命,放权于丞相诸葛亮处理军政大事,“政事无巨细,咸决于亮”。

延熙元年(公元238年),詔命蔣琬應嚴整治軍,率各軍屯紮漢中,等東吳行動,兩國構成東西犄角之勢,伺機伐魏。

劉禪始“乃自摄国事”,由蔣琬、費禕、董允等人主政,修养生息,积蓄力量后从长计议再北伐的政策。劉禪對於寵臣陳祗與宦官黃皓也頗為寵信,姜維畏懼黃皓,只得擁兵屯墾汉中的沓中(今甘肃甘南藏族自治州迭部)。

景耀六年(公元263年),姜維上表後主:「聽聞鐘會治兵關中,欲規畫進一步拓取土地之意,宜一併派遺張翼、廖化督率各軍,分別護陽安關口、陰平橋頭,以防患於未然」,黃皓徵求鬼巫信息,謂敵人終究不會自來,而劉禪也信了鬼巫,滿朝文武竟沒有一人知曉。

最后邓艾偷渡阴平大军压境,刘禅與群臣商議如何抵禦,決定派諸葛瞻領兵迎戰,但諸葛瞻戰敗。最後,刘禅接受谯周的建议,在农历十一月向曹魏投降。劉禪派太僕蔣顯至劍閣,傳令姜維等部投降,蜀軍悲憤不已,紛紛拔刀砍石。邓艾承制拜刘禅为骠骑将军。

蜀漢亡后,刘禅移居魏国都城洛阳,封為安乐县公(常璩则作北巫县安乐乡公)。某日司马昭设宴款待刘禅,囑咐演奏蜀乐曲,并以歌舞助兴时,蜀漢旧臣们想起亡国之痛,个个掩面或低頭流泪。獨刘禅怡然自若,不為悲傷。司马昭见到,便问刘禅:“安樂公是否思念蜀?”刘禅答道:“此間樂,不思蜀也。”他的旧臣郤正闻此言,趁上廁所時对他说:“陛下,下次如司马昭若再问同一件事,您就先注視著宮殿的上方,接著閉上眼睛一陣子,最後張開雙眼,很認真地說:‘先人坟墓,远在蜀地,我没有一天不想念啊!’这样,司马昭就能让陛下回蜀了。”刘禅听后,牢记在心。酒至半酣,司马昭又问同样的问题,刘禅赶忙把郤正教他的学了一遍。司马昭听了,即回以:“咦,这话怎么像是郤正说的?”刘禅大感惊奇道:“你怎麼知道呀!”司马昭及左右大臣哈哈大笑。司马昭见刘禅如此老实忠懇,从此再也不怀疑他,刘禅就这样在洛阳度过餘生,也是乐不思蜀一詞的典故。

西晉晉武帝泰始七年(271年),刘禅去世,諡刘禅為思公。

刘禅太子刘璿在钟会之乱中丧生,按次序应该立次子刘瑶为继承人,但刘禅偏爱六子刘恂,立刘恂为继承人,旧臣文立劝谏,不听,于是刘恂袭为安乐公。

西晉末年,刘渊起事,國號為漢,即汉赵政权,追諡刘禅為孝怀皇帝,但其子孙皆已被灭族,而刘渊是匈奴血统,与刘禅并无直接血缘关系。

刘备在遗诏中说:「射君(射援)到,说丞相叹卿(即刘禅)智量,甚大增修,过于所望,审能如此,吾复何忧!勉之,勉之!」

諸葛亮在與杜微書中評價後主說:「朝廷年方十八,天資仁敏,愛德下士。」

蜀郡太守王崇論後主曰:「昔世祖内資神武之大才、外拔四屯之奇將、猶勤而獲濟。然乃登天衢、車不輟駕、坐不安席。非淵明弘鑒、則中興之業何容易哉。後主庸常之君、雖有一亮之經緯、内無胥附之謀、外無爪牙之將、焉可包括天下也。」“邓艾以疲兵二万溢出江油。姜维举十万之师,案道南归,艾易成禽。禽艾已讫,复还拒会,则蜀之存亡未可量也。乃回道之巴,远至五城。使艾轻进,径及成都。兵分家灭,己自招之。然以钟会之知略,称为子房;姜维陷之莫至,克揵筹斥相应优劣。惜哉!”(華陽國志)

司马昭:「人之无情,乃可至於是乎!虽使诸葛亮在,不能辅之久全,而况姜维邪?」

陳壽於《三国志》:“后主任贤相则为循理之君,惑阉竖则为昬闇之后,传曰‘素丝无常,唯所染之’,信矣哉!礼,国君继体,逾年改元,而章武之三年,则革称建兴,考之古义,体理为违。又国不置史,注记无官,是以行事多遗,灾异靡书。诸葛亮虽达于为政,凡此之类,犹有未周焉。然经载十二而年名不易,军旅屡兴而赦不妄下,不亦卓乎!自亮没后,兹制渐亏,优劣著矣!”、認為劉禪是「素絲無常,唯所染之」,早年得諸葛亮輔助,所以「任賢相則為循理之君」;但後來寵信黃皓,敗壞政事,卻是「惑閹豎則為昏闇之后」。但與暴虐好殺的孫皓相比,劉禪要更為善於處理政務且與大臣們保持著良好的互動。

薛珝:“主暗而不知其过,臣下容身以求免罪,入其朝不闻正言,经其野民有菜色。”

晉朝張華問李密:「安樂公(劉禪)何如?」密曰:「可次齊桓。」華問其故,對曰:「齊桓得管仲而霸,用豎刁而蟲流。安樂公得諸葛亮而抗魏,任黃皓而喪國,是知成敗一也。」(晉書‧李密傳)

裴松之为《三国志·三少帝纪》作注,在评论郭修刺杀费祎事时,称刘禅为“凡下之主”。

孫盛:“刘禅暗弱,无猜险之性。”“禅虽庸主,实无桀、纣之酷,战虽屡北,未有土崩之乱,纵不能君臣固守,背城借一,自可退次东鄙以思后图。”,認為劉禪是「庸主」。

李特:“刘禅有如此江山而降于人,岂非庸才?”(華陽國志)

常璩:“主非中兴之器。”(華陽國志)

张璠:“刘禅懦弱,心无害戾。”

朱敬则:“若乃投井求生,横奔畏死,面缚请罪,膝行待刑,是其谋也。马上唱无愁之歌,侍宴索达摩之曲,刘禅不思陇蜀,叔宝绝无心肝,对贾充以不忠之词,和晋帝以邻国之咏,是其才也。纵黄皓,嬖岑昏,宠高壤,狎江总,是其任也。剥面凿眼,孙皓之刑;弃亲即雠,高纬之志。其馀细故,不可殚论。听吾子之悬衡,任夫人之明镜。”(《全唐文》)

陈世崇:“孔明之子瞻、孙尚战死,张飞之孙遵,赵云次子广亦战死,北平王谌哭于昭烈庙,先杀妻子乃自杀,魏以蜀宫人赐将士,李昭仪不辱自杀。禅不特愧于将士,亦且愧于妇人矣。”

俞德邻:“禅以暗弱之资,而又惑于阉竖,使无此谶,其能与魏争乎?”

郑玉:“孔明盖社稷之臣也,今刘禅昏愚暗弱,纵使伊尹阿衡、周公辅相,亦必危亡而后已,虽百孔明,如之何哉?”“孔明既死,刘禅卒就擒缚。及其入魏,屈辱百端,略无愧耻。岂惟刘氏之宗社不嗣,遂使高祖、光武含羞地下,抱憾无穷。”

王夫之:「後主失德而亡,非失險也,恃險也,恃則未有不失者也。君恃之而棄德,將恃之而棄謀,士卒恃之而棄勇。伏弩飛石,恃以卻敵;危石叢薄,恃以全身;無致死之心,一失其恃,則匍伏奔竄之恐後;扼以於蹊徑,而淩峭壁以下攻,則首尾不相顧而潰。故謂後主信巫言而失陰平之守以亡國,非也。陰平守,而亙數百里之山厓谿谷,皆可度越,陰平一旅,亦贅疣而已。李特過劍閣而歎劉禪之不能守,艸竅之智,乘晉亂以茍延爾。譙縱、王建、孟知祥、明玉珍蹶然而起,熸然而滅,恃險愈甚,其亡愈速矣。」《讀通鑒論·卷十》

罗贯中:“祈哀请命拜征尘,盖为当时宠乱臣。五十四州王霸业,等闲抛弃属他人。”“魏兵数万入川来,后主偷生失自裁。黄皓终存欺国意,姜维空负济时才。全忠义士心何烈,守节王孙志可哀。昭烈经营良不易,一朝功业顿成灰。”

潘时彤:“可惜三分鼎,空怜六尺孤。大权归宦竖,强敌问神巫。斫石军心愤,回天将胆粗。山头曾学射,一矢报仇无。”

《三國志》盧弼集解引周壽昌說:「五丈原头大星夜陨,至千载下犹有余恫。廖公渊、李正方俱为武侯贬退,侯死皆痛泣而卒。李邈何人敢为此疏,直是全无心肝。使非后主之明断,则谗慝生心,乘间构衅,恐唐魏元成仆碑之祸,明张太岳籍没之惨,不待死肉寒而君心早变矣。见疏生怒,立正刑诛,君子谓后主之贤,于是乎不可及。」「(樂不思蜀一事)恐傳聞失實,不則養晦以自全耳。」

清朝方苞《望溪先生文集》中有〈蜀漢後主論〉一文,論曰:「亡國之君若劉後主者,其為世詬歷也久矣,而有合乎聖人之道一焉,則任賢勿貳是也。其奉先主之遗命也,一以国事推之孔明而己不与,世犹曰以师保受寄托,威望信于国人,故不敢贰也。然孔明既殁,而奉其遗言以任蒋琬、董允者,一如受命于先主。及琬与允殁,然后以军事属姜维,而维亦孔明所识任也。夫孔明之殁,其年乃五十有四耳。使天假之年而得乘司马氏君臣之瑕衅,虽北定中原可也。即琬与允不相继以殁,亦长保蜀汉可也。然则蜀之亡,会汉祚之当终耳,岂后主有必亡之道哉!嗚呼!使置後主他行而獨舉其任孔明以衡君德,則太甲、成王當之有愧色矣!」

蔡东藩:“成都虽危,尚堪背城借一,后主宁从谯周,不从北地王谌,面缚出降,坐丧蜀土,是咎在后主。”

魏国史书《魏略》中记载,刘禅在刘备于徐州被曹操攻打时与家人走失,因而被人口贩子拐卖,到了汉中,被一个叫做刘括的人收养。后来刘备入蜀之后,一名簡姓將軍(疑為簡雍)到汉中出使,刘禅找到他并讲解儿时故事,記得父親字玄德,证明自己的确是刘备儿子。张鲁于是下令把刘禅还给刘备,刘备才把他立为继承人。间接来讲,若这个记载为真,赵云在当阳救刘禅以及拦江截阿斗都是蜀汉编造的故事。然而裴松之根据《三国志》的说法对这个记载提出质疑,指出年齡上並不符合,后世也多采信裴松之。


\subsubsection{建兴}

\begin{longtable}{|>{\centering\scriptsize}m{2em}|>{\centering\scriptsize}m{1.3em}|>{\centering}m{8.8em}|}
  % \caption{秦王政}\
  \toprule
  \SimHei \normalsize 年数 & \SimHei \scriptsize 公元 & \SimHei 大事件 \tabularnewline
  % \midrule
  \endfirsthead
  \toprule
  \SimHei \normalsize 年数 & \SimHei \scriptsize 公元 & \SimHei 大事件 \tabularnewline
  \midrule
  \endhead
  \midrule
  元年 & 223 & \tabularnewline\hline
  二年 & 224 & \tabularnewline\hline
  三年 & 225 & \tabularnewline\hline
  四年 & 226 & \tabularnewline\hline
  五年 & 227 & \tabularnewline\hline
  六年 & 228 & \tabularnewline\hline
  七年 & 229 & \tabularnewline\hline
  八年 & 230 & \tabularnewline\hline
  九年 & 231 & \tabularnewline\hline
  十年 & 232 & \tabularnewline\hline
  十一年 & 233 & \tabularnewline\hline
  十二年 & 234 & \tabularnewline\hline
  十三年 & 235 & \tabularnewline\hline
  十四年 & 236 & \tabularnewline\hline
  十五年 & 237 & \tabularnewline
  \bottomrule
\end{longtable}

\subsubsection{延熙}

\begin{longtable}{|>{\centering\scriptsize}m{2em}|>{\centering\scriptsize}m{1.3em}|>{\centering}m{8.8em}|}
  % \caption{秦王政}\
  \toprule
  \SimHei \normalsize 年数 & \SimHei \scriptsize 公元 & \SimHei 大事件 \tabularnewline
  % \midrule
  \endfirsthead
  \toprule
  \SimHei \normalsize 年数 & \SimHei \scriptsize 公元 & \SimHei 大事件 \tabularnewline
  \midrule
  \endhead
  \midrule
  元年 & 238 & \tabularnewline\hline
  二年 & 239 & \tabularnewline\hline
  三年 & 240 & \tabularnewline\hline
  四年 & 241 & \tabularnewline\hline
  五年 & 242 & \tabularnewline\hline
  六年 & 243 & \tabularnewline\hline
  七年 & 244 & \tabularnewline\hline
  八年 & 245 & \tabularnewline\hline
  九年 & 246 & \tabularnewline\hline
  十年 & 247 & \tabularnewline\hline
  十一年 & 248 & \tabularnewline\hline
  十二年 & 249 & \tabularnewline\hline
  十三年 & 250 & \tabularnewline\hline
  十四年 & 251 & \tabularnewline\hline
  十五年 & 252 & \tabularnewline\hline
  十六年 & 253 & \tabularnewline\hline
  十七年 & 254 & \tabularnewline\hline
  十八年 & 255 & \tabularnewline\hline
  十九年 & 256 & \tabularnewline\hline
  二十年 & 257 & \tabularnewline
  \bottomrule
\end{longtable}

\subsubsection{景耀}

\begin{longtable}{|>{\centering\scriptsize}m{2em}|>{\centering\scriptsize}m{1.3em}|>{\centering}m{8.8em}|}
  % \caption{秦王政}\
  \toprule
  \SimHei \normalsize 年数 & \SimHei \scriptsize 公元 & \SimHei 大事件 \tabularnewline
  % \midrule
  \endfirsthead
  \toprule
  \SimHei \normalsize 年数 & \SimHei \scriptsize 公元 & \SimHei 大事件 \tabularnewline
  \midrule
  \endhead
  \midrule
  元年 & 258 & \tabularnewline\hline
  二年 & 259 & \tabularnewline\hline
  三年 & 260 & \tabularnewline\hline
  四年 & 261 & \tabularnewline\hline
  五年 & 262 & \tabularnewline\hline
  六年 & 263 & \tabularnewline
  \bottomrule
\end{longtable}

\subsubsection{炎兴}

\begin{longtable}{|>{\centering\scriptsize}m{2em}|>{\centering\scriptsize}m{1.3em}|>{\centering}m{8.8em}|}
  % \caption{秦王政}\
  \toprule
  \SimHei \normalsize 年数 & \SimHei \scriptsize 公元 & \SimHei 大事件 \tabularnewline
  % \midrule
  \endfirsthead
  \toprule
  \SimHei \normalsize 年数 & \SimHei \scriptsize 公元 & \SimHei 大事件 \tabularnewline
  \midrule
  \endhead
  \midrule
  元年 & 263 & \tabularnewline
  \bottomrule
\end{longtable}


%%% Local Variables:
%%% mode: latex
%%% TeX-engine: xetex
%%% TeX-master: "../../Main"
%%% End:


%%% Local Variables:
%%% mode: latex
%%% TeX-engine: xetex
%%% TeX-master: "../../Main"
%%% End:

%% -*- coding: utf-8 -*-
%% Time-stamp: <Chen Wang: 2019-12-18 10:23:25>


\section{孙吴\tiny(229-280)}

%% -*- coding: utf-8 -*-
%% Time-stamp: <Chen Wang: 2021-11-01 11:36:31>

\subsection{大皇帝孫權\tiny(229-252)}

\subsubsection{生平}

孫權(182年12月22日-252年5月21日),字仲謀,吴郡富春(今浙江省杭州市富阳区)人,東漢末三国時期吳[註 1]的著名政治家、戰略家,同時也是吳的締造者及建国皇帝。而在孫權稱帝之前,吳的群臣等對其稱呼為將軍或至尊。在位23年,享年69歲,諡號為大皇帝,廟號太祖。

富春孙氏是江东不顯赫的豪族,世代仕於吳。生父孫堅據傳是春秋时期军事家孫武後人,孤微發跡;孙权亦因此可能是孙武的第22代孙。

孫權生母为吳郡豪族出身的吴夫人,當初懷孕的時候,夢見月亮進去懷裡,之後生下了孫策。及後在懷孫權的時候,又夢見太陽進去懷裡。之後告訴孫堅說:“妾昔日懷著孫策的時候,夢見月亮入懷裡;如今又夢見太陽入懷裡,為什麼會這樣呢?”孫堅回答:「太陽和月亮,是陰陽的能量精氣,是極其貴象的征兆。我們的子孫大概會興家赤旺吧!」。漢光和五年五月十八日(182年7月5日),孫堅擔任下邳县丞的時候嫡次子孫權出生,其面相方頰大口,銳目有神,孫堅覺得驚奇,認為有貴氣的象相。

漢光和七年(184年),朱儁奏请孙坚担任佐军司马,孙坚随朱儁南征北战,将妻吴氏和孙权等诸子都留在九江郡寿春县。

漢中平六年(189年),汉灵帝逝世,长沙(治所在今湖南省长沙市)太守的孙坚起兵从长沙经荆州响应讨伐董卓的关东联军。当时孙权的长兄孙策已在寿春淮南一带颇有名气。其中有庐江人周瑜前来拜会,在周瑜的建议下,孙策于是携母弟搬到庐江郡舒县(今安徽省庐江县西南)。

漢初平二年(191年),孙坚奉袁术之命讨伐荆州刘表,结果中刘表手下的黄祖的埋伏身亡。孙权和家人迁居广陵郡江都。孫策託付张紘照看母弟。自孫堅死後,孫權經常跟隨兄長。孙权性格寬宏有氣度,不但仁厚而且能夠根據不同情況作出多方面判斷。他以厚恩养士而出名,其名气渐渐不输给父兄。孙策也对这个弟弟感到很惊奇,自认为不如他。每当宴请宾客时,孙策常常回头看着孙权说:“这些人,以后都会是你的将领。”

漢初平四年(193年)因孙策决定跟随袁术,就派吕范将孙权等人护送到住在曲阿的舅舅吴景那里居住。翌年,孙策击破了陆康为袁术取得了庐江郡。当时,还是扬州刺史的刘繇担心自己也会被袁氏吞并,与袁术和孙策产生嫌隙,于是将孙权堂兄孙贲和吴景驱逐出曲阿,只有孙权及其母弟弟们还留在那里,于是朱治特意将其从曲阿接到自己家里奉养卫护。孙权和母亲后来又迁至历阳县和阜陵县居住。

漢興平二年(195年)孙策渡江击败刘繇后,孙权和家人跟随着陈宝回到了曲阿居住。孙权到江東以后,与朱然胡综一起读书,结下了深厚的友谊。

漢建安元年(196年),孙权15岁的时候,由朱治举为孝廉,任阳羡县(今江苏宜兴)长,代行奉义校尉。曹操也任命严象将其举为茂才,当时已有属下周泰和潘璋。

孙策平定江东的丹阳、会稽和吴三郡后开始给汉廷进贡。建安二年(197年),汉廷派刘琬前往江东授予孙策会稽太守的职务,刘琬对人说:“我看孙家的兄弟虽然每个都才华横溢,智慧通达,都是荣华福贵不长久。只有次男孝廉,相貌高大挺拔,有大贵之表,且会是最為长寿的,你们等着瞧吧。”袁术与孙策决裂后,拉拢丹阳等六县及山贼头目祖郎,鼓动山越和自己一起共同对付孙策。当时孙策率兵前往讨山贼,仅孙权等数百人留在宣城,山贼数千人蜂拥而至,年轻的孙权在周泰的保护之下得以幸免。

漢建安四年(199年)末至次年初,孙权随同孙策征庐江太守刘勋于皖城。刘勋败逃后,又进军沙羡讨伐黄祖,与仇敌黄祖在沙羡一带展开大战,黄祖几乎全军覆没,韩唏战死,黄祖只身逃走,士卒溺死者达万人,豫章太守华歆又举城投降。平定了庐江豫章二郡。孫策與曹操交好,表面臣服於漢朝廷之下,曹操並加封孫策為吳侯,並以礼征辟孙权和孙权的弟弟孙翊到漢朝廷擔當漢臣職務,但二人均沒有前往。

漢建安五年(200年)春,孙策遭到刺殺,不選擇與自己性格極其相似,眾人看重為最適合繼任者的三弟孫翊,而是選擇性格與自己大不相同的二弟孫權。孫權對兄長的去世痛哭不已沒能親自視察政事。經過長史张昭勸說,乃除去喪服,由張昭扶上馬外出巡察軍營,于是眾人之心都歸附於孫權。

曹操見孫策已死本打算伐吳,侍御史張紘勸諫曹操不該乘人之危。曹操聽從其言,通過东汉朝廷冊封孫權为討虏将军,兼领会稽太守,以吴县为治所。

孫權剛上任,只占有会稽、吴郡、丹杨、豫章、庐陵、庐江六郡,除孫權本人為會稽太守外,其他五人都是孙策生前所任命的部將,分別是:丹陽太守吳景、豫章太守孫賁、廬陵太守孫輔、吳郡太守朱治、廬江太守李術。然而孫策死後,孫輔認為孫權沒有能力保衛江東,於是与曹操暗通打算出賣孫權,被孫權察覺後給予制裁。堂兄孙暠(孫靜長男)欲攻打会稽郡夺取政權,被虞翻阻止。這其中李术尤为不服從孫權,與梅乾、雷緒、陳蘭數萬人在集結淮水一帶騷擾破壞,孫權寫信要求李術扣留這些叛逃者。李術公開表示有德見歸,無德見叛,不應復還為由拒絕。於是孫權用計策寫信給曹操,說嚴象被殺是李術所為,所以不應該理會李術。孫權隨後與孫河、徐琨一起親征叛徒李術于皖城。皖城被孫權包圍,李術向曹操求救,但是曹操沒有到來,一切發展正如孫權所設計的一樣。城內糧盡只能用泥丸代替食糧充飢,隨即破皖城,李術被梟首,孫權迁徙城裡人及李術部将三万余人到江東,留下一座空城。

孫策平定江東的時候,曾對當地士族進行打壓、屠戮,導致孫家在本土得不到支持。孙权以张昭为師傅,並任用父兄留下的部將,以部曲私兵世襲制作為條件懷柔本土豪族,大量起用豪族子弟,穩定江東孫家政權。陆逊、徐盛、留贊、诸葛瑾、步骘、顾雍、顧徽、是仪、吕岱、朱桓、骆統等贤才良将都在这一时期加入孙权麾下。周瑜斷言他以後能成就帝王大業,并将好友鲁肃推荐给孫權认识。鲁肃则向孙权說出漢室不能復興,曹操不能一時間消滅。應該鼎足江東靜觀其變,在北方多戰亂的時候乘勢应消滅刘表,佔據长江以南建立帝業的方案。

未開發山地潛藏的山越也大規模發動叛亂,而江東許多的本地豪族士族與山越族群都有緊密聯繫,因此,在孫氏每次出征對外的時候,都給予江東內部很大的侵擾,也一直牽制著吳國數十年的對外作戰,漢建安八年(203年),豫章鄱陽縣等地山越再起,孫權即刻命征虜中郎將呂範平定鄱陽、蕩寇中郎將程普討伐樂安,派賀齊討平東冶地,建昌都尉太史慈分頭進討山越,又派別部司馬黃蓋、韓當、呂蒙等人扼守山越經常出沒的郡縣,恢復了原設縣邑,穩定了秩序。漢建安十一年(206年)又率領孫瑜,周瑜,淩統,成功討平山越麻、保二屯。

漢建安十二年(207年),自黄祖一处来降的甘寧說:「今漢已經日漸衰微,曹操為滿足自己的心,終於成了篡漢的盜賊。南荊之地,山陵地勢有利,江川流通,國的西邊的確是這樣的形勢。我已看透劉表,考慮的不夠長遠,兒子也是無能的人,不是能夠承傳基業之才。主公應當盡早規劃,不能落入曹操手上。進圖之計,先取黃祖為佳。黃祖如今年老,老邁衰退嚴重,錢財糧谷都已經缺乏,左右矇騙他,事出於錢財私利,侵要吏士的錢財,吏士心裏都憤怒。舟船戰具,廢棄也不修理,耕農懶惰,軍隊沒有法紀。如果主公現在去攻打,必定能大敗。一旦打敗黃祖軍,擊鼓行軍至西,西據楚關,大局趨勢擴張,這樣就可以逐漸進取巴蜀。」孫權贊同並採納。張昭當時就在席上坐,難言道:「吳國如今危懼,如果行軍攻打,必然招致恐慌。」甘寧回答道:「國家將蕭何的重任交給君,君留置守護卻擔心憂亂,那為什麼還要仰慕古人?」孫權對舉起酒杯附於甘寧說:「興霸,今年行軍討伐,就如這杯酒,決意託付給卿你。卿盡量提出方略,如能夠破黃祖,則是卿的功勞,不要因為張長史(張昭)之言而放棄。」出兵虏其人民而还。

漢建安十三年(208年),孙权發動江夏之戰再讨黄祖,以周瑜為都督。呂蒙隨軍出征。黃祖見孫權兵來,急派水軍都督陳就率兵反擊,呂蒙統率前鋒部隊,身先戰陣,親自斬殺陳就。擄獲其船隻、士兵。返回到孫權大軍,並引領自軍兼程趕路,水路兩路齊進。凌統先攻下城池,黃祖隻身逃竄,被孫權軍中的騎兵馮則所斬殺。此戰,孫權大獲全勝,但是劉表長子劉琦及時前來禦敵江夏北部,孫權只能有效佔領江夏郡南部區域,後将治所自吴移居至京口。

漢建安十三年(208年)秋,曹操對孫權發出以八十萬軍力會獵江東的書信,孫權打算與曹操決一死戰,但張昭等群臣勸孫權歸降,礙於豪族群臣的壓力下孫權沒有表達自己的意見,聽後離開席間換衣服,唯獨魯肅離座找孫權說要對抗曹操,孫權很高興魯肅與自己的想法一致,對張昭等人所說的感到非常失望,魯肅勸孫權召回進兵鄱陽的周瑜,並邀請劉備加盟的提議。孫權答應,隨即派魯肅到荊州打探情況。當時荆州牧刘表病死,劉表次子劉琮及其母蔡氏其舅蔡瑁因仇視刘备而投降曹操。鲁肃到荊州之前劉備被曹操打敗,荊州已經落入曹操之手,劉備南渡长江,魯肅與他相遇詢問去向,劉備打算到蒼梧投靠朋友吳巨,魯肅則說明孫權的意向和實力,邀請劉備加盟孫權共同對抗曹操,而不是投靠力弱的人。刘备很高興孫權的邀請,聽後見事態緊急隨即派诸葛亮去求見孙权。孫權故意刁難諸葛亮,藉此通過他對曹操軍的分析,去說服江東豪族及投降派,孫權聽後大悅。之後群臣商議,眾人勸孫權投降,但周瑜向孫權分析曹操與孫權兩軍的優劣勝敗,指出:「其一,曹軍背後仍有後顧之憂,西涼有馬騰、韓遂等軍閥,戰端一開,必偷襲曹軍背後。」、「其二,北方人慣習陸戰而不擅水戰,竟敢捨馬鞍而就船槳,此乃捨長就短。」、「其三,寒冬將至,曹軍兵缺衣食,馬無藁草,兵卒士氣低落。」、「其四,曹軍遠途跋涉,奔襲千里,水土不服,多生病患。」既而進步分析了曹軍的實際力量,指出來自中原的曹軍不過十五六萬,而且所得劉表新降的七八萬人,人心並不向曹。」此時只有周瑜、鲁肃坚持抗击曹操的主张,意见与孙权相合。隨即以決斷之勢拔劍砍掉桌子一角,說:「敢再有言降曹者,如同此案!」,藉此將投降派氣焰壓倒,並將一早已經準備好的三萬軍隊交給周瑜指揮。周瑜、程普分别被任命为左、右都督,魯肅為贊軍校尉輔助周瑜。孫權派周瑜抵禦曹操大軍,在赤壁与曹军相遇,周瑜大败曹操军队。周瑜等又追击到烏林破曹軍,曹操只好撤回北方,乘勝進攻荊州南郡。甘宁在夷陵城,被曹仁的部队所包围,周瑜采纳吕蒙的计策,留下凌统抗拒曹仁,用其中一半兵力驰救甘宁,南郡相持一年間,孙权為了減低周瑜們的前線壓力,親率剩餘的小量軍力军围合肥,相持合肥一个多月,聽從張紘的建議撤退。而劉備以張飛和一千人換二千人為條件向周瑜借兵,然後在孫曹交戰間乘機攻取了长沙、桂阳、武陵、零陵荊南四郡,並上表劉琦為荊州牧,領有了荊南四郡。

漢建安十四年(209年),周瑜攻破南郡。孙权以周瑜为南郡太守。刘备上表奏封孙权代理车骑将军,兼任徐州牧。孙权又招揽了滕耽、吾粲等人。當時劉琦去世,失去了荊州四郡領有權,劉備隨即向周瑜借地,周瑜分南岸給刘备,後劉備將油江口改名為公安。劉備嫌地少無法容納人馬,親自到京口見孙权借荊州數郡(南郡、长沙、桂阳、武陵、零陵)並督領荊州,周瑜、呂範提議軟禁劉備,孙权聽從魯肅所說而不採納,並借出荆州数郡於刘备,孫權暫表劉備為荊州牧,孫權以孫夫人聯姻來鞏固孙刘联盟的關係,也奠定了三国鼎立的基础。周瑜和甘寧勸孫權入蜀,孫權邀請劉備共同取益州,劉備以劉璋是同祖宗為由拒絕,並說如果孫權打劉璋自己一定阻止,如果我打劉璋的話,我必定會披髮入山林歸隱,不做攻取同宗的事。孫權不聽,並派孫瑜進攻益州,劉備阻止並不給孫瑜前進的去路,並說不能這樣做,孫權只有下令退還。

漢建安十五年(210年),孙权任命鲁肃为太守,驻守陆口,又遣步骘为交州刺史,挥师南征。吴军压境,交州各郡守无不俯首,士燮率领家族奉承节度。唯有刘表所置苍梧太守吴巨阳奉阴违,最后被步骘發現有異心,隨即斬殺。孙权自此得到交州九郡領有權,并加封臣服于自己的士燮为左将军。南海郡、郁林郡、苍梧郡則是孫權管治,交阯郡、日南郡、珠崖郡、儋耳郡、九真郡、合浦郡則是士夑獨立管治。

漢建安十六年(211年),孙权将治所迁至秣陵。次年,孙权修筑石头城,改秣陵为建业。聽聞曹操率四十萬大軍進攻,孫權打算興建水塢,部將大家都認為直接上下船就能著陸登船,建造水塢沒用,只有呂蒙認為這個塢可以給步兵快速登船進退不失的便利,於是孫權同意呂蒙看法,派呂蒙建造濡须坞作為進出濡須到巢湖的水軍防衛要塞,也是濡須之戰的重要補給據點,與日後曹魏建造的合肥新城是互相對應的防衛設施。

漢建安十八年(213年)正月,曹操親率四十萬大軍攻孫權於濡須口,孫權向劉備發出救援。當時劉備作為客將在劉璋之下,劉備以救孫權為由向劉璋借兵去救荊州關羽,劉璋對劉備猜疑只給他一半軍需和4000兵馬。劉備憤怒劉璋給物資和兵少,隨後密謀反戈偷襲了益州劉璋,劉備也沒有理會孫權求援,任由他們自生自滅。孫權知道劉備攻劉璋而不來救援,出爾反爾,大罵劉備是狡猾的傢伙竟然敢使詐。濡須戰場最後只有孫權軍獨力以七萬大軍抵擋曹操號稱的四十萬大軍,起初戰況不好。孫權出戰沒有得到收穫,濡須口的江西營被曹軍打破並俘虜都督公孫陽,而董襲趕往救援的途中遇溺去世,孫權軍霎時間頓挫。下半場戰鬥,曹操作油船在夜中親率打算襲擊洲上孫權軍。孫權親自率軍乘機反擊,驅使水軍突襲包圍了曹操,曹操軍落水溺死有數千人,俘虜敵兵人數也有三千餘人。孫權乘勝追擊,並對曹操進行多次挑釁,但是曹操受到孫權的打擊下而不敢出擊接戰。孫權見曹操堅守不出,親自督一艘船從濡須塢出擊進入曹操大軍陣地觀陣,《吳錄》記載曹操軍眾人打算射擊孫權的船,但曹操知道孫權來觀陣下令不要妄動,孫權在曹操大營饒了一圈,曹操看到孫權的膽量還有船上士兵器械嚴整,讚歎:「生子當如孫仲謀,劉景升兒子像豬狗」。隨後,孫權下令吹號回營。而《魏略》記載则是說孫權進了曹操軍陣地,曹操命部下拉箭亂射,孫權船身一則被箭矢射滿將要翻船,孫權隨即下令調轉船身擋箭,船身也因此得到平衡,孫權從容地回營。戰鬥已經一個月有餘,曹操仍然無法打敗孫權也無法攻克嚴防的濡須塢,孫權便寫信給曹操說春天水增,你快點走吧,並在另一封信寫上,你如果不死,我不安樂。孫權給曹操一個撤退的下台階,曹操收到信後,對左右說孫權不會騙我,隨即下令撤軍。

漢建安十九年(214年)五月,孙权亲征庐江治所皖城。闰月,在一天的时间内就攻破皖城,俘获庐江太守朱光及参军董和,男女数万人,将北线扩展至合肥一带。劉備得到益州,於是孙权派诸葛瑾向刘备讨还荆州各郡。刘备拒絕,並說得到涼州後再把荊州所有郡歸還(當時益州與涼州之間還有一個漢中,漢中當時是張魯的領地),孫權經過濡須和益州一事後知道劉備的推託假話,隨即派遣魯肅到益陽與關羽對峙,再派吕蒙指挥孫皎、潘璋、呂岱、鲜于丹、徐忠、孙规等领兵二万,攻取长沙、零陵、桂阳三郡。孙权住在陆口,为各路军队的指挥、调度。吕蒙军队一到,长沙、桂阳二郡全部归服,同時通過心理戰把零陵把太守誘至開城投降。最後魯肅和關羽在益陽交鋒對峙,以及談判磋商。這個時候,曹操準備攻取漢中,刘备害怕丢失益州,便派使者求和。两国因此停戰,于是以湘水為界,劉備被逼把长沙、桂阳兩個郡以東還給孫權。江夏郡是孫權攻破黃祖後一直領有並沒有借出,而長沙郡則是生前孫堅所管治的。曹操再伐吳,但是曹操大軍被甘寧100人奇襲而全軍撤退。

215年,孫權北征合肥。孙权作战勇敢,進軍時與數名部將作為先頭部隊率先到達戰場扎寨立營,因為大軍還沒有集結只有數名部將的軍隊,所以被張遼有機可乘突襲成功。而撤军时孫權亦親自與四名部將及1000人在後方穩定士氣,張遼見此率七千人偷襲,當時全軍撤出,兵力只有一千人不如張遼七千人,呂蒙、蔣欽、凌统、甘宁等在逍遥津以北被张辽所袭击,凌统等拼死保护孙权,孙权弓馬嫻熟迎擊張遼,最後骑着骏马飛躍津桥成功撤出。張遼在戰後對孫權的弓射騎術感歎,當魏軍知道這個弓騎勇將是孫權而悔恨沒有捉到他。

漢建安二十二年(216年)冬,曹操再次兴师伐吴,丹阳四郡(今安徽定量城)民帅尤突、费栈受曹操授權联合山越,聚集數萬人起兵反叛。孙权即命賀齊和陆逊进兵征讨。賀齊和陆逊大破尤突及费栈等眾,降服丹阳、吴郡、故鄣等三郡山越,得精兵数万人。曹操屯军至居巢(今安徽巢县东北)準備進軍。孫曹交戰,關羽聯絡長沙郡縣長吳碭、袁龍再次發起叛亂,孫權派鎮守陸口的魯肅前去幫助呂岱,最終平定了叛亂。217年,曹操進攻濡須口,孙权在濡須塢前方築城,但被曹操軍先鋒逼退。之後濡須战线胶着,連日暴雨水面上漲,孫權驅使水軍前進曹操軍非常惶恐,曹操下令撤退。孫權便以吕蒙为都督,與蔣欽共同擔任此戰的總指揮。呂蒙据守之前建成的城坞,并设置万张强弓硬弩,以拒曹操。结果曹军所有先鋒尚未安然立屯,便被吕蒙攻破了,曹操軍敗走退回到居巢,最後攻不下孫權而下令撤軍,曹操自己也引軍撤退。在217年濡須口之戰孫權擊退曹操之後,要著手處理揚州內部的山越問題、自己國家的利益、孫劉關係,所以對漢朝偽降主动与曹操修好,避免日後受到曹操、劉備、內部山越的三方面侵擾。曹操當時被孫權擊敗而引軍撤退,另一方劉備也進軍漢中,因此接受請降。魯肅非常後悔借出荊州給劉備,同時也怒斥劉備、關羽沒有信用,他死後呂蒙接替他在前線總指揮職務,並向孫權提出要警戒關羽,不依靠劉備獨立對抗曹操的建議,孫權經過多年獨立對抗曹操見識過劉備等人的反復態度,於是採納呂蒙提議,與關羽表面交好。

漢建安二十四年(219年),孫權有進攻合肥態勢,魏軍全部州郡的軍隊進入戒備狀態。孫權得知劉備獲得漢中後,再次派諸葛瑾向劉備索還荊州的訴求,但劉備拒絕。孫權打算與关羽以聯姻修好孫劉關係,但關羽以虎女怎能嫁犬子為由拒絕,並怒罵使者。關羽發兵圍攻襄阳曹仁,孫權打算派兵救援,關羽嫌孫權增援太慢大罵:「狢子(對東吳人的貶稱),等我滅了樊城之後回去就把你滅了。」。孫權知道關羽傲慢輕視自己,便寫信道歉。曹操派人聯絡孫權,以荊州為條件希望孫權相助,但孫權沒有馬上答應。後来關羽在樊城之戰俘虜于禁數萬降兵,把所有人送到南郡關押,但是還假借食糧不足為藉口,對吳國湘水邊境侵略而且搶奪軍需糧食。此前,孫權跟呂蒙分析局勢時,孫權想打徐州,但呂蒙認為應該打關羽的荊州,分析認為曹軍多為騎兵擅於陸戰,徐州雖然拿得下來,但也守不住。不如著手準備拿下荊州,完全控制整條長江,對外進可攻退可守,對內下游的吳國也會十分安全。此時孫權對關羽的種種作為已經難以忍受,命吕蒙至陆口實施之前商量好的計劃,並向漢朝廷申請討伐關羽,獻帝同意。十月,孙权西征关羽,以吕蒙陆逊为先锋,孫皎為殿後,孫權則潛軍一同北上。呂蒙以白衣渡江之策計,在夜半時分計破連綿不斷的烽火台屏障,然後再占据南郡,關羽被徐晃等人打敗從樊城回來,知道南郡丟失,隨後撤退到麥城駐守。孫權沒有打算殺關羽的意圖,於是派使者對關羽勸降,關羽假裝答應,在城上立旗後逃跑。孫權知道後派潘璋和朱然截擊,呂蒙當時留在南郡指揮大局,陆逊则另率军攻取宜都郡房陵等。呂蒙通過善政安撫荊州民心,把蜀漢軍家人的情況告訴給關羽軍,頓時間關羽部下失去戰意四散,關羽軍數萬人有的向孫軍投降,有的被孫軍的將軍吸納,最後在臨沮馬忠擒獲了關羽、關平等人。孫權想用關羽制衡曹劉打算再次招降關羽,左右文臣此時對主子孫權說狼子不可養,曹操當年收留關羽,如今換來遷都的惡果。孫權聽後把關羽斬首,並把首級送去給曹操,孫權則以諸侯的禮遇安葬關羽的身軀在當陽。自此荊州南北為曹、孫兩家佔有,于是孙权免除荆州百姓的所有租税。曹操向漢獻帝上表任命孙权为骠骑将军,假节兼任荆州牧,封南昌侯,同時也征召了张承、劉基等人。

漢建安二十五年(220年)年初,魏王曹操及吳大督呂蒙等名將相繼病故。11月,繼承王位的曹丕逼劉協禪讓,正式建號,是為魏文帝。孫權並沒有向曹丕投降,而是命都尉趙咨出使魏国承認曹丕的禪讓帝位並以諸侯身份向他稱臣,再將于禁等敗將送回北方,令新上任需要彰顯功名的魏帝曹丕的虛榮自負的心迅速膨脹,同時解除對孫權的戒心。孙权又派遣趙咨、陈化、冯熙、沈珩为使节,曹丕也派侍中辛毗、尚书桓阶前来东吴与之立誓结盟,曹丕冊封孙权为諸侯藩王吴王,以大将军使持节的身份监督交州,兼任荆州牧,孫權立长子孙登为王太子。當時,群臣勸孫權不應該受封吳王應該自稱九州伯、上將軍,孫權則說當年劉邦也是受封了項羽的漢王,最後還是成就了偉業。曹丕處事浮華,在他守喪期間向孫權索求雀頭香、大貝、明珠、象牙、犀角、玳瑁、孔雀、翡翠、鬥鴨、長鳴雞,吳群臣聽後說這些是珍稀貴重物勸孫權不要給,孫權則認為這只是瓦片石頭罷了,並不介意。期間在外交上一直由孫權主導,曹丕過於天真相信孫權而拒絕劉曄順江而下伐吳的建議,還几次拒绝了大臣们的伐蜀的提案。

劉備宣稱獻帝被害,於建安二十六年(221年)4月也登基稱帝。同年7月,借以關羽報仇的名義討伐孫權欲吞併江東領土發動猇亭之战,孫權自公安迁都鄂縣,改名武昌進入備戰,以六县设置武昌郡。孫權讓諸葛瑾寫信給劉備勸說他不要開戰希望和睦相處,並陳說利害分清楚敵人主次,不要上了曹魏的當,如果真要開戰他們也不會手軟,劉備不聽。孫權派周泰準備向白帝城作攻防姿態,任命陆逊为大都督,率领朱然、韓當、駱統、潘璋、孫桓等领兵前往抵抗。黃初三年(222年)六月,陆逊彻底击败蜀军。蜀军被斩杀和放下武器投降者有几万人。刘备被孫桓追至差點被擒獲,最後仅保得自身不死。當時,徐盛、潘璋、宋謙等人認為只要繼續追擊劉備,必能把他殺掉,但陸遜、朱然、駱統等認為不要追擊,他們察覺到曹丕有進攻江東的態勢。而孫權根據自己的判斷,採納陸遜等人的看法,下令不要對逃往白帝城方向的劉備展開追擊。

夷陵之戰期間,孫權一直在自己領地橫江屯兵提防曹魏,果然如陸遜等所料,曹丕派曹休襲擊孫權的領地曆陽及蕪湖,曹丕打算控制孫權,要求孫權將孫登送到魏國都城做人質,孫權知道其用意所以以藉口多次推辭,最後曹丕發覺孫權誠心不款,於是有意發兵攻打江東。夷陵之戰一結束,吴魏之间就開始有交戰態勢,曹丕派出三十萬大軍,命令曹休、张辽、臧霸出兵洞口,曹仁出兵濡须口,曹真、夏侯尚、张郃、徐晃率军围攻南郡。當時要處理揚州境內的山越問題,孙权故意示弱,謙卑上書誘騙曹丕,只是為了拖延曹丕進軍期限爭取平定山越內亂的時間,曹丕則相信他送兒子而沒有進軍,而另一方,孫權則派遣吕範、朱然、朱桓率領其他部將暗中部署。黃初三年(222年)十月,曹丕見孫權沒有送兒子來,正式開戰。於是孙权改年号为黄武,同曹魏斷絕來往。孫權命呂範率領徐盛、孫韶、全琮、賀齊等人在水路抵御曹休等,孫盛、诸葛瑾、潘璋、杨粲前往南郡增援朱然,朱桓接替周泰以濡须督的身份在濡須塢抵擋曹仁。三方面戰鬥中,曹仁被朱桓多個戰術配搭被打得大敗,曹休、張遼、臧霸則是強弩之末被徐盛、全琮、賀齊等人反擊而敗退,曹真、夏侯尚、徐晃、張郃則團團圍攻江陵,但久攻不下朱然。最終在次年三月春魏軍全部撤走,江南國境皆得安寧。另一邊,孫權收到在白帝城休養的劉備的求和信後,於十二月派遣太中大夫鄭泉出使蜀汉,蜀、吴两国自此重結盟好。

黃武二年(223年),孙权在江夏修筑山城。改用乾象历。夏四月,孙权的大臣们进劝他称帝,孙权不答应。此前,魏國令吳領地的戲口太守晉宗造反殺同僚王直,騷擾江南國境,孫權因為三方面大戰分身不下而不能馬上消滅他。六月孫權派賀齊、胡綜等人率軍平定,最終擒獲晉宗。劉備死後,諸葛亮派鄧芝向東吳再次確立聯盟關係,孫權知道諸葛亮用意,也非常器重鄧芝,所以答應修好。孫權便斷絕同曹魏來往,派辅义中郎将张温回訪蜀漢。

黃武五年(226年),孙权下令各州郡守,对百姓实行宽容安息政策。这时陆逊因驻守的地方缺粮,上表孙权,命令诸将广开农田。七月,孙权听说魏文帝曹丕去世,兴兵征讨江夏郡,围攻石阳城,卻久攻不下。江夏郡高城被孫奐攻陷。孙权任命全琮为东安郡太守,讨伐山越的反叛。孙权分交州另置广州,不久又复合为交州。

黃武七年(228年)五月,孙权命鄱阳太守周鲂以斷髮詐降,假装叛离东吴,引诱魏将曹休。爆发石亭之战,秋八月,孙权前往皖口,派征西将军陆逊率领朱桓、全琮在石亭大败曹休。

黃龍元年(229年)夏四月十三日丙申(5月23日),統治江東三十年的孙权在南郊正式登基为帝,改年号为黄龙。四月,孫權大赦改年,在南郊拜天,即皇帝位,諸葛亮派衛尉陳震去東吳祝賀孫權登皇帝位,3個月後孫權把國都從武昌遷回建業。追谥父亲孙坚为武烈皇帝,母亲吴氏为武烈皇后,長兄孙策为长沙桓王。立吴王太子孙登为皇太子。将军官吏都晋爵加赏。六月,蜀国派人前来庆贺孙权登基。孙权還禮,承認東西二帝共存,並与蜀漢使節商议平分天下。其中,豫、青、徐、幽四州属吴;兖、冀、并、凉四州歸蜀。司州的土地,以函谷关为界分属两国,雙方制定盟书,共同声讨曹叡。秋九月,孙权將都城從武昌遷到建業(今江蘇省南京市),就住在原来的府第中,不再另建新宫殿,征召上大将军陆逊辅佐太子孙登,掌管武昌事宜。

孙权即位後,曾多次派人出海。黃龍二年(230年),他派衛溫、諸葛直等航行到達夷洲;242年,他又派聶友等航行到珠崖儋耳(指現今的海南島)。

嘉禾元年(232年),孙权派遣将军周贺等航海到辽东。十二月,辽东太守公孙渊向孙权称藩。

嘉禾二年(233年),孙权派太常张弥、贺达等万人,带上金银财宝奇货异物,加上九锡,经海路送给公孙渊。举朝大臣全都规劝孙权,认为公孙渊其人不可信,对他的恩宠礼遇不要太过分。孙权一意孤行,没有接受规劝。后来公孙渊果然将张弥等杀死,以其首級並東吳賜予的金印送往曹魏邀功。孙权聞之,大感慚恨,企图亲自征讨公孙渊,尚书仆射薛综等极力谏阻,最终中止了这个计划。

嘉禾三年(234年)二月,諸葛亮再次與兵北伐。诸葛亮集中在漢中十萬大軍全部出動,木牛流馬,運糧不停,同時相約東吳東西並舉。五月,東吳出兵,七月退兵。孙权下诏放宽徭役,夏五月,孙权派遣陆逊、诸葛瑾等驻军江夏、沔口,派孙韶、张承等进军广陵、淮阳,孙权自己亲率大军进围合肥新城,爆发合肥新城之战后退兵回返,孙韶也停止进军广陵等地。秋八月,孙权任命诸葛恪为丹杨太守,讨伐山越部族。次年,孙权派吕岱领兵讨伐贼寇李桓等。

嘉禾六年(237年),孙权让群臣讨论奔丧立科、丞相顾雍奏请违法奔丧应处以死罪。此后吴县县令孟宗违法奔母丧归家,事后在武昌将自己拘禁起来听候处罚。陆逊向孙权说明孟宗的平时作为,并借机为孟宗求情,孙权于是给孟宗减刑一等,并申明下不为例,于是违法奔丧的事绝迹。

赤烏元年(238年),改年号为赤乌。当时,孙权利用吕壹打擊豪族,吕壹本性苛刻残忍,执法严酷。太子孙登屡次进谏,孙权都不采纳,大臣们于是都不敢进言。后来吕壹奸邪的罪行败露被处死,孙权自我批评,认错误,派中书郎袁礼代自己向曾經規勸但未被採納的大臣們致歉。

赤烏二年(239年),公孙渊不滿曹魏對其待遇不高,便又復叛魏國,自立为燕王,结果受到魏国司马懿攻击。公孙渊派遣使者向吴国求助。当时吴人都对公孙渊的反复无常历历在目,劝说孫權斩杀使者。唯有羊衜说:“陛下,斩首公孙渊的使者固然能讓您出口恶气,可这样做是出了匹夫的怒气,而放弃了霸王之计。臣以为,朝廷不如借此机会,出奇兵前往以观动静。如果魏国进攻公孙渊失败,那么我军远赴辽东解救,是恩结于远夷,义盖于万里;如果魏军和公孙渊相持不下,公孙渊首尾不能相顾,那我军正好进攻辽东,这样也足以让上天惩罚公孙逆贼,一雪往日之耻。”羊衜此言深得孙權赞许,于是派遣使者羊衜、郑胄、将军孙怡以海军前往辽东,击败魏国守将张持、高虑,並俘虜當地居民南還。十月,孙权派遣将军吕岱、唐咨前往剿滅少数民族的叛乱,将他们全部屠戮。

赤乌四年(241年),太子孙登去世。孙权後立三子孙和为太子。孙权听从百官封建诸子的意见,又立孙和之弟孙霸为鲁王。但孙霸始终不服孙和。遂召集手下宾客及结交诸大臣,常与围绕在孙和一侧的太子党分庭抗礼。赤乌十三年(250年),孙权决定废黜太子孙和并赐死鲁王孙霸,同时改立七子孙亮为皇太子。第二年册立孙亮之母潘氏为皇后。

太元元年(251年),冬十一月,孙权祭祀南郊回来后,就因風疾(相當於今稱中風)生病卧床。十二月,遣驿使传书召大将军诸葛恪回京,拜为太子太傅,孙权下诏省徭役、减征赋,将将国家大事交给诸葛恪管理,并修改诸多不便法令。

太元二年(252年),孙权立原太子孙和为南阳王。五子孙奋为齐王。六子孙休为琅琊王。二月,大赦,改年号为神凤。神凤元年四月廿六日(公元252年5月21日),孫權於太初宮内殿中驾崩,享壽六十九歲。滕胤与太子太傅诸葛恪、少傅孙弘、荡魏将军吕据、侍中孙峻等人一同受遗诏辅佐太子。孙权稱帝后在位23年。葬於建業蔣陵,謚大皇帝,廟號太祖。

孫權擔任家督弱冠繼承江南政權以来,對外招納人才培養部下,以懷柔策略籠絡不服從孫家的江南豪族,以白手興家統合內部對抗外壓,鞏固孫家政權在江南的地位,另一方面通過平定揚越叛亂進行強兵吸收老幼弱者補戶的政策,同時給予落後山越民提供漢文化的學習。諸侯時期的孫權不屬於任何一方,也沒有所謂的興漢滅漢的政治口號,故此可以根據時勢局勢發展,判斷哪一方有利用價值,並進行聯合的自由外交戰略獲取自己的利益。作為開創基業的帝王,孫權以出色的政治智慧及戰略判斷,深諳縱橫捭闔,最終締造一方霸業。赤壁之戰後加強控制江東,并将江東六郡扩展到揚、荆、交三州,積極開發南方的荒蕪之地,穩健控制中國東南。

孫權在數年間將國土政權安定,以適才適所為第一原則,而不以輩分、資歷、交情、名氣為優先,深知人無完人,故此不追究缺點而用其優點的用人風格,處事也是嚴罰主義者,就算親族或功臣的家族犯罪,也會給予嚴刑處分。例如選擇寒門出身的周泰為平虜將軍,與孫權為同窗的朱然则身居其下,在夷陵之戰任命资历尚浅的陸遜为大都督,許多人因是孫策舊將或者公室貴戚,一度有所不滿,但最终都心服口服,步骘虽然出身豪族,但是避難到江東而家道中落,最终竟做到丞相一职。孙权亦能主动培养部下,同時對待功臣的態度是忘其短而貴其長。孙权以顧雍為丞相而非眾人所推薦的張昭,就是因为丞相位置處理的事情多且繁重,而張昭性情剛烈固執,不遵從他的意見則會埋怨歸咎到底,到時反而對公事沒有益處。

孫權崇尚節儉,並效法大禹以卑宮為美,原本住的建業宮其實只是孙权早期的將軍府而已,一直住到赤烏十年建材腐朽,還詔令將武昌宮拆了,把木材運來建業修繕,但其實當時武昌宮也有二十八年的歷史不堪使用,這麼做的目的是節省木料避免妨礙農桑工作,由此也可知孫權對農業的重視。陆凯向孙皓劝谏时也称孙权时代“后宫列女,及诸织络,数不满百,米有畜积,货财有余”。

孫權執法嚴格,即使面對至親也是法律優先從不循私。孫輔因通敵而被流放、庶弟孫朗因違反軍令燒毀自軍軍用而被呂範送回,於是改姓丁並禁錮終身、愛子孫霸更因圖危太子,而被賜死。另一方面,孙瑜孙桓孙韶等孙氏宗室或委以重任,女则嫁于国家重臣,即使是谋反者的后代也能不计前嫌。可见孙权实际上对于同族亲戚相当重视。陈寿因此赞曰“况此诸孙,或赞兴初基,或镇据边陲,克堪厥任,不忝其荣者乎”。

对内廣納諫言,任用父兄旧部稳定局面,平定叛徒和山越,攻滅殺父仇人黃祖,吸納北方難民。在一片降曹之声时果斷与曹操一战並與刘备结盟,任用周瑜打败曹操穩定江南地盤,後來因為劉備一連串背離同盟關係的所作所為及荊州等问题而與蜀汉決裂,連本帶利奪回荊州並在夷陵之战重創劉備,最終確立了三分天下的局面。黄龙元年(229年),孫權于武昌称帝,建国为吴,孙吴建立。称帝以后他分部諸將,鎮撫山越,增設縣邑,編制戶籍,設置農官,推行軍屯與民屯;收容南遷移民,興修水利,增廣農田;親自下田採用牛耕,大幅度改良農業生產技術,大興佛教,奠定了六朝的經濟與文化基礎。

在晚年大批豪族過分插手孫權的家事,而且分黨立派,造成政局动荡不安,孫權對此非常不滿。之後孫權得知自己繼承人意向的消息外洩之後大為憤怒,並將相關人員等捉拿問罪。。後來孫權遭到豪族暗殺及背叛,所以在二宮之戰爆發後,通過部下彈劾而削弱豪族權力來鞏固政權,之後難能可貴的是孫權同時也具有認錯的勇氣,從陸遜之子陸抗的任職態度以及陸機所著《辯亡論》看來,孫陸二家情誼仍然十分深厚,陸氏對孫權亦持肯定態度,不因孫權老年冷酷而有所怨言。神凤元年(252年)夏四月,孫權在内殿驾崩,终年七十一岁。

由於孫權大力開拓海上事業並且開拓江南,因此在中國史上有非常重要的地位,然而他死後的待遇與他的功績完全不成正比,詩人曾極在其作品《吳大帝陵》中提到“四十帝中功第一,壞陵無主使人愁”,劉克莊也在《吳大帝廟》中嘆息“今人渾忘卻,江左是誰開”。

東吳的部曲私兵和世襲制度是有利有弊,執政者需要確立王朝威信的時候,有利於鞏固自己的君主地位;當執政者權威衰微的時候,威權容易被擁有私兵的部下奪取。因此孙权晚年後世評價兩極,一方認為他晚年昏庸而做出一連串錯誤導致王朝衰退;而一方則認為孫氏在江東的權力較弱,所以孫權晚年處事冷酷無情,通過制約擁有政治影響力及私兵權部曲世襲制的豪族,把權力集中在孫氏家族手上,避免如同魏國一樣被士族奪權蠶食政權的情況發生,從而確保孫氏政權現況及未來的威權地位,只是後來的權力者不能維繫這個政權發展而導致逐漸衰落。

北方戰亂,孫權也吸納南渡的北方民眾,其北方的手工技術也在江南得到發揚及應用。另外由於孫權積極擴張海上事業,並曾發兵遼東,因此江南造船業大大興盛。

首都建業原名秣陵,最初是一小縣,因孫權定都建業並開鑿運河而成為一流都市,被稱為六朝古都,現名南京。

《吳曆》曰,黃武四年,扶南諸外國來獻琉璃。這是中國最早與南海諸國交流的記載。孫權主動派出朱應與康泰出訪南海各國,先後到過林邑(今越南中南部)、扶南(今柬埔寨)、西南大海州(今南洋群島)、大秦(羅馬)、天竺(今印度),並記下各國物產以利貿易奠定了南海貿易的基礎,回國後,二人分別撰寫《扶南異物志》及《外國傳》(又稱《吳時外國傳》),之後繼續派出使節進行南國宣化,同扶南、林邑、堂明(今柬埔寨)建立關係。,這是史無前例的事情,雖然南海諸國之前已與中國有接觸,但是由官方政府主動派出官員積極尋求國際貿易,孫權卻是創舉,貿易的範圍甚至到達了羅馬,並在建業接見了羅馬商人秦論。

《江表传》中记载孙权方頤大口,眼神很有光彩。漢朝遣使者劉琬為孫策加錫命之時看見孫權,形容孫權的相貌高大挺拔。刘备和张辽都看到孙权坐着时显得很高,认为他躯干较长,中国民国学者黎东方分析,只有不需要站著伺候人,而是坐著讓人伺候的貴人才會是所謂軀幹長而雙腿短的外形,古代这被視為大貴之相,劉備被形容為手長過膝也是基於同樣的道理。

《三國演義》里孫權则被记载“碧眼紫髯,堂堂一表人才”。

在閻立本《十三帝王圖》之中,孫權為站姿,此為開國之君之意,身著的冕服應有天子十二章,在圖中有被畫出來的有「日、月、藻、火、黼」五章 ,其中日月為明,明火三章表示的是孫權振興經濟,教化百姓,讓其光明之面普照天下之意,藻則代表孫權稱帝隨時代順應天意而起,黼則表示孫權「能斷割」,這與三國志中孫權好俠養士仁而多斷的人格特質以及遇曹操來攻能拒絕臣服決心抗曹、遇劉備來攻則稱臣曹丕以保全江東等正確的重大決斷相呼應,圖中孫權手持麈尾扇,表現了他的帝王風度,為十三帝王圖中唯一持扇者,“麈”是領隊大鹿尾,魏晉以來,尚清談,手執麈尾有“領袖群倫”含意,藝文類聚亦記載司馬懿見諸葛亮乘素輿、葛巾毛扇指揮三軍,嘆諸葛亮為名士,諸葛亮在《三國演義》中也常持羽扇指揮軍隊,扇子有善戰之意,因此蘇軾在《念奴嬌》形容周瑜時亦說周瑜「羽扇綸巾」談笑間強虜灰飛煙滅,孫權在位期間,赤壁與夷陵之戰均以少勝多,甚至取荊州而兵不血刃,足見他用人正確調度有方的善戰特質,因此辛棄疾會說「天下英雄誰敵手?曹劉,生子當如孫仲謀」。

孙权性格旷达开朗,仁爱明断,喜欢供养贤才,因此很早就与父兄齐名。由于非常重视集体的力量,能毫无保留地信任臣下,甚至部下死後代為教養其孤兒贍養其妻儿及其父母。也會調解部屬糾紛,亦下诏勿杀叛逃将领的妻子子女。孙权与臣下的亲密关系也体现在称呼其表字上,甚至是对于初见的潘濬,曾與陸遜當眾對舞,又将自身所穿衣物皆赐之。对于他国贤才,孙权也毫不掩饰地表达喜爱,如诸葛亮费祎邓芝宗预等。孫盛因而稱許孫權盡心關愛部下,令其甘心為自己拼命,是東吳能夠立於江東的原因。

孙权天性活潑奔放,能言善辩,常常肆无忌惮地恶作剧、戏弄人,经常开些无关紧要的玩笑,即使是面对蜀汉来使也不例外。其本人亦参与配合部下的戏谑。

孫權的忍辱負重性格在向曹操與曹丕稱臣時一覽無遺。因此臥薪嘗膽一詞出自蘇軾的《擬孫權答曹操書》,也因為忍辱負重,所以孫權面對荊州問題時選擇與蜀國結盟,而不與劉備爭鬥,避免曹操坐收漁翁之利。三國之中,也是東吳最晚稱帝。陳壽亦提過「孫權屈身忍辱,任才尚計,有句踐之奇英,人之傑矣」,趙咨答曹丕時亦說「屈身於陛下,是其略也」。因孫權處世手段極其柔軟,曹丕也曾以嫵媚形容孫權,所以有詩歌詠孫權時說「孝廉嫵媚還能霸」。

孙权善于判断国内外人物局势。如认定魏延和杨仪会在诸葛亮死后内讧。也预见到曹魏亡国的先兆。

孙权擅长骑術和弓術,在合肥面對張遼的突襲能平安躍馬過橋,他的弓術也给张辽留下了很深的印象。

孙权有六口宝剑,分别是白虹、紫电、辟邪、流星、青冥、百里。

關於孫權嗜好,其中射猎(射虎、射雉)與好酒和开宴会派对尤其出名。每当猛兽近前,孙权总是以亲手击打为乐趣。宴会中常常对部下进行劝酒,孙权喜好冒险,如顶着大风天坐船出航,乘轻船去见曹操军队, 密令甘宁夜袭曹营等等。

孙权也喜爱读书,据其本人所言,其所涉猎内容涵盖《诗经》、《尚书》、《礼记》、《左传》、《国语》及三史(《史记》、《汉书》和《东观汉记》),惟不曾研读《周易》,孙权在书法上亦有成就,被认为擅长行书和草书。

在宗教方面,孙权早年信仰道术,与诸多方术之士交往甚密。主要人物为吴范、刘惇、赵达、姚光、介象等人。而被后世尊为道教天师的葛玄也与孙权有过交往。孙权也对当时的新宗教佛教非常开明,赤乌年间为高僧康僧会建立建初寺。

孫權因在家事上随心所欲,表现的不在乎上下尊卑而招致陳壽的批評,称其可比拟春秋时代的齐桓公,对外“有识士之明”,对内却“嫡庶不分,闺庭错乱”,最终在繼承人問題上埋下祸根,導致很长一段时间内国家都动荡不安。裴松之则意見相反,認為孫權廢掉無罪的太子,雖然是開啟禍亂的前兆,但最多只是東吳滅亡的次因而非主因,畢竟東吳滅亡已是孫權死後二十八年的事情,而且滅亡主因仍是暴君孫皓,即使孫權當時傳位於孫和,最後也是孫皓登基,國之滅亡的根本問題其實是出在為政者昏虐,並非只有孫權廢黜一事就能造成,如孫亮能保住國祚,或者孫休不早死,都不至於讓東吳滅亡。陸遜的孫子陸機更著有《辯亡論上》《辯亡論下》詳細說明東吳亡國非因蜀國滅亡,而是孫權死後的當政者用人不當。

陳壽於《三國志》认为孫策為開國奠基人但子孙未被封为王爵,孫權於義儉矣。后人据此穿凿附会,认定孙权对孙策有所怠慢,从尚书仆射存和胡综的上书可知孙权以谦虚为美德,不愿效仿汉代旧制过分尊崇皇族,就连孙权自己的皇子也不例外,如被孙权宠爱的次子孫慮也止在侯爵。另一爱子孫和在十九岁封为太子前也从未获得任何爵位。群臣请立孙权余下四子为王时也被孙权拒绝。

孫盛从国家大局方面对陈寿的看法也表示了不同意见,认为當時天下局勢尚未統一,宜正名定本貴賤疏邈,不宜給與孫策之子更高的權力與爵位製造內亂機會,此為穩定局勢之必要行為,況天倫篤愛,孫權既已將孫策宗廟立於建業,應不會刻意吝於給予地位,這明顯是為了穩定國家局勢的必要處置方式。

从实际史料出发,孙权反倒有相当多不忘旧情的举动,如孙盛所言为孙策建庙于建业并派太子祭祀。在赤乌年间再次为孙策进行厚葬,因吕范往日对其兄的帮助而对之大加溢美,以致严峻私下认为夸大其词了,直到后来才信服。

孫策臨終傳權時:「舉江東之眾,決機於兩陳(陣)之間,與天下爭衡,卿(孫權)不如我。舉賢任能,各盡其心,以保江東,我不如卿。」(《三國志·吳書·孫破虜討逆傳第一》)

曹操於濡須之戰:「生子當如孫仲謀,劉景升(劉表)兒子若豚犬耳!」(《三國志·吳書·吳主傳第二》裴松之註引《吳歷》)於孫權稱臣時「此兒欲踞吾著爐炭上邪!」(《晉書·宣帝紀第一》)

刘备:「孙车骑长上短下,其难为下,吾不可以再见之。」

关羽:「鰂子敢爾,如使樊城拔,吾不能滅汝邪!」(《三國志·蜀書·關張馬黃趙傳第六》)

周瑜:「將軍以神武雄才,兼仗父兄之烈,割據江東,地方數千里,兵精足用,英雄樂業,尚當橫行天下,為漢家除殘去穢。」(《三國志·吳書·周瑜魯肅吕蒙傳第九》)「今主人亲贤贵士,纳奇录异。」

魯肅:「将军神武命世。」「孫討虜聰明仁惠,敬賢禮士,江表英豪,咸歸附之」(《三國志·蜀書·先主傳第二》裴松之註引《江表傳》)

張紘:「自古帝王受天命的君主,雖有皇靈在上輔佐,文德傳播天下,也要靠武功顯著。要開墾種植,任賢使能,務崇寬惠,順天命去誅討,這樣不勞師眾定天下。」

陸遜:「陛下(孫權)以神武之姿,涎膺期運,破操(曹操)烏林,敗備(劉備)西陵,禽羽(關羽)荊州,斯三虜者當世雄傑,皆摧其鋒。」(《三國志·吳書·陸遜傳第十三》)

諸葛亮:「海內大亂,將軍(孫權)起兵據有江東,劉豫州亦收眾漢南,與曹操并爭天下。今操芟夷大難,略已平矣,遂破荊州,威震四海。英雄無所用武,故豫州遁逃至此。將軍量力而處之:若能以吳、越之眾與中國抗衡,不如早與之絕﹔若不能當,何不案兵束甲,北面而事之!今將軍外託服從之名,而內懷猶豫之計,事急而不斷,禍至無日矣!」(《三國志·蜀書·諸葛亮傳第五》)「權有僭逆之心久矣」(《三國志·蜀書·諸葛亮傳第五》裴松之註引《漢晉春秋》)「孫將軍可謂人主,然觀其度,能賢亮而不能盡亮,吾是以不留。」(《三國志·蜀書·諸葛亮傳第五》裴松之註引《袁子》)「孙权据有江东,已历三世,国险而民附,贤能为之用。」「议者咸以权利在鼎足,不能并力,且志望以满,无上岸之情,推此,皆似是而非也。何者?其智力不侔,故限江自保;权之不能越江,犹魏贼之不能渡汉,非力有馀而利不取也。」

司馬懿:「權之稱臣,天人之意也。」(《晉書·宣帝紀第一》)

张辽:「向有紫髯将军,长上短下,便马善射。」

程昱:「权有谋。」(《三国志·魏书 ·程郭董刘蒋刘传第十四》)

陈琳:「夫天道助顺,人道助信,事上之谓义,亲亲之谓仁。盛孝章,君也,而权诛之,孙辅,兄也,而权杀之。贼义残仁,莫斯为甚。乃神灵之逋罪,下民所同雠。辜雠之人,谓之凶贼。」(《檄吴将校部曲文》)

彭羕:「仆昔有事於诸侯,以为曹操暴虐,孙权无道,振威闇弱,其惟主公有霸王之器,可与兴业致治,故乃翻然有轻举之志。」(《三国志·卷四十·蜀书十·刘彭廖李刘魏杨传第十》)

趙咨:「聰明仁智,雄略之主也」、「納魯肅於凡品,是其聰也;拔呂蒙於行陳,是其明也;獲於禁而不害,是其仁也;取荊州而兵不血刃,是其智也;據三州虎視於天下,是其雄也;屈身於陛下(曹丕),是其略也。」(《三國志·吳書·吳主傳第二》)

贾诩:「孙权识虚实,陆议见兵势。据险守要,泛舟江湖,皆难卒谋也。用兵之道,先胜后战,量敌论将,故举无遗策。臣窃料群臣,无备、权对,雖以天威臨之,未見萬全之勢也。」(《三国志·魏书·荀彧荀攸贾诩传第十》)

邓芝:「大王命世之英。」

刘晔:「權無故求降,必內有急。權前襲殺關羽,取荊州四郡,備怒,必大興師伐之。外有強寇,眾心不安,又恐中國承其釁而伐之,故委地求降,一以卻中國之兵,二則假中國之援,以強其眾而疑敵人。權善用兵,見策知變,其計必出於此。」、「權雖有雄才,故漢驃騎將軍南昌侯耳,官輕勢卑。士民有畏中國心,不可強迫與成所謀也。不得已受其降,可進其將軍號,封十萬戶侯,不可即以為王也。夫王位,去天子一階耳,其禮秩服御相亂也。彼直為侯,江南士民未有君臣之義也。我信其偽降,就封殖之,崇其位號,定其君臣,是為虎傅翼也。權既受王位,卻蜀兵之後,外盡禮事中國,使其國內皆聞之,內為無禮以怒陛下。」(《三國志·魏書·程郭董劉蔣劉傳第十四》)

冯熙:「吴王体量聪明,善于任使。赋政施役,每事必咨。教养宾旅,亲贤爱士。赏不择怨仇,而罚必加有罪。臣下皆感恩怀德,惟忠与义。带甲百万,谷帛如山。稻田沃野,民无饥岁。所谓金城汤池,强富之国也。」

刘基:「大王以能容贤蓄众,故海内望风。」

钟繇:「顾念孙权,了更妩媚。」(《三國志·魏書·鍾繇華歆王朗傳第十三》)

刘琬:「吾观孙氏兄弟虽各才秀明达,然皆禄祚不终,惟中弟孝廉,形貌奇伟,骨体不恒,有大贵之表,年又最寿,尔试识之。」

陳壽:「孫權屈身忍辱,任才尚計,有勾踐之奇,英人之傑矣。故能自擅江表,成鼎峙之業。然性多嫌忌,果於殺戮,暨臻末年,彌以滋甚。至於讒說殄行,胤嗣廢斃,豈所謂賜厥孫謀以燕冀於者哉?其後葉陵遲,遂致覆國,未必不由此也。」(《三國志·吳書·吳主傳第二》)「割據江東,策之基兆也,而權尊祟未至,子止侯爵,於義儉矣。」(《三國志·吳書·孫破虜討逆傳第一》)

陆凯:「自昔先帝时,后宫列女,及诸织络,数不满百,米有畜积,货财有余。先帝崩后,幼、景在位,更改奢侈,不蹈先迹。」(《三国志·吴书·潘濬陆凯传第十六》)

孙楚:「吴之先主,起自荆州,遭时扰攘,播潜江表,刘备震惧,逃迹巴岷,遂依丘陵积石之固,三江五湖,浩汗无涯,假气游魂,迄于四纪,二邦合从,东西唱和,卒相扇动,拒捍中国。」

陸機:「吳桓王基之以武,太祖(孫權)成之以德,聰明睿達,懿度深遠矣。其求賢如不及,恤民如稚子,接士盡盛德之容,親仁罄丹府之愛。拔呂蒙於戎行,識潘濬於系虜。推誠信士,不恤人之我欺;量能授器,不患權之我逼。執鞭鞠躬,以重陸公之威;悉委武衛,以濟周瑜之師。卑宮菲食,以豐功臣之賞;披懷虛己,以納謨士之算。故魯肅一面而自讬,士燮蒙險而效命。高張公之德而省游田之娛,賢諸葛之言而割情欲之歡,感陸公之規而除刑政之煩,奇劉基之議而作三爵之誓,屏氣跼蹐以伺子明之疾,分滋損甘以育凌統之孤,登壇慷慨歸魯肅之功,削投惡言信子瑜之節。是以忠臣競盡其謀,志士鹹得肆力,洪規遠略,固不厭夫區區者也。故百官苟合,庶務未遑。」(《辯亡論》下)「用集我大皇帝,以奇踪袭於逸轨,叡心发乎令图,从政咨於故实,播宪稽乎遗风,而加之以笃固,申之以节俭,畴咨俊茂,好谋善断,东帛旅於丘园,旌命交于涂巷。故豪彦寻声而响臻,志士希光而影骛,异人辐辏,猛士如林。於是张昭为师傅,周瑜、陆公(陆逊)、鲁肃、吕蒙之畴入为腹心,出作股肱;甘宁、凌统、程普、贺齐、朱桓、朱然之徒奋其威,韩当、潘璋、黄盖、蒋钦、周泰之属宣其力;风雅则诸葛瑾、张承、步骘以声名光国,政事则顾雍、潘濬、吕范、吕岱以器任干职,奇伟则虞翻、陆绩、张温、张惇以讽议举正,奉使则赵咨、沈珩以敏达延誉,术数则吴范、赵达以禨祥协德,董袭、陈武杀身以卫主,骆统、刘基强谏以补过,谋无遗算,举不失策。故遂割据山川,跨制荆、吴,而与天下争衡矣。」(《辯亡論》上)

华谭:「赖先主承运,雄谋天挺,尚内倚慈母仁明之教,外杖子布廷争之忠,又有诸葛、顾、步、张、朱、陆、全之族,故能鞭笞百越,称制南州。」。「吴武烈父子皆以英杰之才,继承大业。今以陈敏凶狡,七弟顽冗,欲蹑桓王之高踪,蹈大皇之绝轨,远度诸贤,犹当未许也。」

裴松之:「孙权横废无罪之子,为兆乱。」「权愎谏违众,信渊意了,非有攻伐之规,重复之虑。宣达锡命,乃用万人,是何不爱其民,昏虐之甚乎?此役也,非惟闇塞,实为无道。」

孙盛:「盛闻国将兴,听於民;国将亡,听於神。权年老志衰,谗臣在侧,废适立庶,以妾为妻,可谓多凉德矣。而伪设符命,求福妖邪,将亡之兆,不亦显乎!」「观孙权之养士也,倾心竭思,以求其死力,泣周泰之夷,殉陈武之妾,请吕蒙之命,育凌统之孤,卑曲苦志,如此之勤也。是故虽令德无闻,仁泽(内)著,而能屈强荆吴,僭拟年岁者,抑有由也。然霸王之道,期於大者远者,是以先王建德义之基,恢信顺之宇,制经略之纲,明贵贱之序,易简而其亲可久,体全而其功可大,岂委璅近务,邀利於当年哉?语曰“虽小道,必有可观者焉,致远恐泥”,其是之谓乎!」

虞溥:「性度弘朗,仁而多断,好侠养士,始有知名,侔于父兄矣。」(《三國志·吳書·吳主傳第二》裴松之註引《江表傳》)

《荆州先德传》:“权好嘲戏以观人。”

王勃:「孙仲谋承父兄之余事,委瑜肃之良图,泣周泰之痍,请吕蒙之命,惜休穆之才不加其罪,贤子布之谏而造其门。用能南开交趾,驱五岭之卒;东届海隅,兼百越之众。地方五千里,带甲数十万。」

朱敬则:「孙仲谋藉父兄之资,负江海之固,未敢争盟上国,竞鹿中原,自守未馀,何足言也。」(《全唐文》)

徐夤:「一主参差六十年,父兄犹庆授孙权。不迎曹操真长策,终谢张昭见硕贤。建业龙盘虽可贵,武昌鱼味亦何偏。秦嬴谩作东游计,紫气黄旗岂偶然。」

司马光:「文帝承父兄之烈,师友忠贤,以成前志,赤壁之役,决策定虑,以摧大敌,非明而有勇能如是乎?奄有荆扬,薄于南海,传祚累世,宜矣。」(《历代名贤确论·卷五十七》)

苏轼:「亲射虎,看孙郎。」(《江城子·密州出猎》)「孙权勇而有谋,此不可以声势恐喝取也。」

苏辙:「吴大帝方其属任贤将,抗衡中原,曹公惮之。及其老也,贤臣死亡略尽,喜诸葛恪之劲悍,越众而付以后事。闼其用兵劳民之后,继起大役,兵折于外,既归而不能自克,将复肆志于僚友。恪既以丧其躯,而孙氏因之三世绝统,吴、越之民陷于炮烙之地,国随以亡。彼以进取之资用进取之臣,以徼一时之功可耳,至于托六尺之孤,寄千里之命,而亦属之斯人,其势必至是哉。」(《栾城后集·孙仲谋》)「今夫曹操、孙权、刘备,此三人者,皆知以其才相取,而未知以不才取人也。世之言者曰:孙不如曹,而刘不如孙。」

謝采伯:「孫權運籌於內,劉備、諸葛亮、周瑜、關侯等,合謀並智,方拒得曹操,敗之於赤壁,亦未為竒政縁。」

何去非:「权之勇决进取,无以逮其父兄,然审机察变,持保江东,于权有焉。」(《何博士备论》)

辛弃疾:「千古江山,英雄无觅,孙仲谋处。」(《永遇乐·京口北固亭怀古》)「何处望神州,满眼风光北固楼,千古兴亡多少事,悠悠。不尽长江滚滚流。 年少万兜鍪,坐断东南战未休,天下英雄谁敌手,曹刘。生子当如孙仲谋。」

吕祖谦:「孙权起于江东,拓境荆楚,北图襄阳,西图巴、蜀而不得。北敌曹操、西敌刘备,二人皆天下英雄。所用将帅,亦一时之杰。权左右胜之而后能定其国。及权国既定,曹公已死,丕、叡继世,中原有可图之衅。权之名将死丧且尽,权亦老矣。」(《吴论》)

晁补之:「吴人轻而无谋,自古记之矣。孙坚、孙策皆无王霸器。虽赖周瑜、鲁肃辈辅权嗣立,亦权稍持重,故卒建吴国也。」(《鸡肋集》)

萧常:「权承父兄之资,勇而有谋,愤曹操窃国,尝有讨贼之志;乌林之捷,亦一时之隽功。其后关羽围襄阳,降于禁,威振北方,操大惧,欲徙都以避之。权于是时,诚能与羽协力、东西并举,则操可图而汉室可兴。今乃不然,反袭杀羽以媚曹氏,不能少降意于帝室之胄,而甘心臣贼,昭烈之不能混一区夏,由此故也。他日虽有犄角之功,亦无及矣。吁,惜哉!」(《萧氏续后汉书》)

叶适:「权有地数千里,立国数十年,以力战为强,以独任为能。残民以逞,终无毫髪爱利之意,身死而其后不复振,操术使之然也。」(《习学记言·读吴志》)

元好问:「孙郎矫矫人中龙,顾盼叱咤生云风。」

郝经:「東漢之衰,孫權承父兄之烈,尊禮英賢,撫納豪右,誅黄祖,走曹操,襲關侯,遂奄有荆颺,今年出濡須,明年戰合肥,嶷然勢常北向,而以守爲攻,稱臣於魏,結援於漢,始忍勾踐之辱,終爲熊通之譖,保據江淮,奄征南海,卒與漢魏鼎峙而立,先起而後亡,非惟智勇足抗衡,亦國勢便利然也。」(《續後漢書》)

胡三省:「當方面者,當如呂岱;委人以方面者,當如孫權。」(《資治通鑒注》)

朱元璋:「君臣之间,以敬为主。敬者,礼之本也。故礼立而上下之分定,分定而名正,名正而天下治矣。孙权盖不知此,轻与臣下戏狎,狎其臣而亵其父,失君臣之礼。」(《明太祖宝训》)

罗贯中在《三國演義》有詩贊曰:「紫髯碧眼號英雄,能使臣僚肯盡忠,二十四年興大業,龍磐虎踞在江東。」

孙承恩:「仲谋强明,委任才智。听言能断,业乃鼎峙。倍义负汉,屈身事曹。传世四君,霸图亦消。」(《文简集·卷三十八》)

王夫之:「于是而知先主之知人而能任,不及仲谋远矣。」「于子瑜也、陆逊也、顾雍也、张昭也,委任之不如先主之于公,而信之也笃,岂不贤哉?」(《宋论·卷一·太祖》)

王懋竑:「至权时,张昭、张紘虽见尊礼而不复任用,昭且几不免,而翻竟以窜死,惟顾雍、潘濬辈从容讽议,得安有位。陆逊有大功,而以数直谏愤恚而卒。周瑜、鲁肃幸已早死,不与陆逊同祸,而亦恩不及嗣。有所爱重者,惟吕蒙、凌统、甘宁、周泰辈,以视策万万不逮矣。其保有江东者,以有吕蒙辈为之用,得其死力,而其不能廓大基业,窥中原者,亦以此。」(《三国志集解》)

赵翼:「至孙氏兄弟之用人,亦自有不可及者。」「以人主而自悔其过,开诚告语如此,其谁不感泣?使操当此,早挟一‘宁我负人,勿人负我’之见,而老羞成怒矣!此孙氏兄弟之用人,所谓以意气相感也。」

王鸣盛:「孙权称臣事魏已久,及黄武元年春大破蜀,刘备奔走,势愈强盛,则魏欲与盟而不受,九月魏兵来征,又卑辞上书求自改悔,乞寄命交州乃随,又改年临江拒守,彼此互有杀伤,不分胜负。十二月又通聘于蜀,乃既和于蜀,又不绝于魏,且业已改元而仍称吴王。五年令曰北虏缩窜,方外无事,乃益务农亩,称帝之举,直隐忍以至魏明帝太和三年,而后发,反覆倾危,惟利是视,用柔胜刚,阴谋狡猾,史评以勾践相比,非虚语也。」(《三国志集解》)

何焯:「老悖昏惑,吴亡不待皓而决。」

李慈铭:「三国时,魏既屡兴大狱,吴孙皓之残刑以逞,所诛名臣,如贺邵、王蕃、楼玄等尤多。少帝之诛诸葛恪、滕胤,皆逆臣专制,又当别论。惟大帝号称贤主,而太子和被废之际,群臣以直谏受诛者,如吾粲、朱据、张休、屈晃、张纯等十数人,被流者顾谭、顾承、姚信等又数人,而陈正、陈象至加族诛,吁,何其酷哉!自是宫闱之衅,未有至此者也。」(《越缦堂读书笔记》)

蔡东藩:「黄祖本无才智,而孙坚死于祖手;孙策又不能亲复父仇,命为之,势为之也。坚阻于命,策限于势;至权承父兄之业,用瑜蒙诸将,一出再出,方举黄祖而枭夷之,春秋之义大复仇,如孙仲谋者,其固不愧为令子乎?曹操谓生子至如孙仲谋,若刘景升诸儿,与豚犬等,原非虚言。」「孙权承父兄遗业,任才尚计,史谓其有勾践遗风,乃内宠相寻,晚年益愦,废长立幼,乱本已成。」(《後漢演義》)

盧弼:「竊謂有勾踐之志則可,無勾踐之志則終爲奴虜而已,南宋其已事也。仲謀操縱其間,以江東而抗衡大國承祚,方之勾踐其信然矣。」(《三國志集解》)

柏杨:「孙权是中国历史上最可爱最有人情味的皇帝之一。」

李宗吾:「他和刘备同盟,并且是郎舅之亲,忽然夺取荆州,把关羽杀了,心之黑,仿佛曹操,无奈黑不到底,跟著向蜀请和,其黑的程度,就要比曹操稍逊一点;他与曹操比肩称雄,抗不相下,忽然在曹丞相驾下称臣,脸皮之厚,仿佛刘备,无奈厚不到底,跟著与魏绝交,其厚的程度也比刘备稍逊一点。他虽是黑不如操,厚不如备,却是二者兼备,也不能不算是一个英雄。」

毛泽东:「孙权是个很能干的人。」「当今惜无孙仲谋。」(《毛泽东读古书实录》)

\subsubsection{黄武}

\begin{longtable}{|>{\centering\scriptsize}m{2em}|>{\centering\scriptsize}m{1.3em}|>{\centering}m{8.8em}|}
  % \caption{秦王政}\
  \toprule
  \SimHei \normalsize 年数 & \SimHei \scriptsize 公元 & \SimHei 大事件 \tabularnewline
  % \midrule
  \endfirsthead
  \toprule
  \SimHei \normalsize 年数 & \SimHei \scriptsize 公元 & \SimHei 大事件 \tabularnewline
  \midrule
  \endhead
  \midrule
  元年 & 222 & \tabularnewline\hline
  二年 & 223 & \tabularnewline\hline
  三年 & 224 & \tabularnewline\hline
  四年 & 225 & \tabularnewline\hline
  五年 & 226 & \tabularnewline\hline
  六年 & 227 & \tabularnewline\hline
  七年 & 228 & \tabularnewline\hline
  八年 & 229 & \tabularnewline
  \bottomrule
\end{longtable}

\subsubsection{黄龙}

\begin{longtable}{|>{\centering\scriptsize}m{2em}|>{\centering\scriptsize}m{1.3em}|>{\centering}m{8.8em}|}
  % \caption{秦王政}\
  \toprule
  \SimHei \normalsize 年数 & \SimHei \scriptsize 公元 & \SimHei 大事件 \tabularnewline
  % \midrule
  \endfirsthead
  \toprule
  \SimHei \normalsize 年数 & \SimHei \scriptsize 公元 & \SimHei 大事件 \tabularnewline
  \midrule
  \endhead
  \midrule
  元年 & 229 & \tabularnewline\hline
  二年 & 230 & \tabularnewline\hline
  三年 & 231 & \tabularnewline
  \bottomrule
\end{longtable}

\subsubsection{嘉禾}

\begin{longtable}{|>{\centering\scriptsize}m{2em}|>{\centering\scriptsize}m{1.3em}|>{\centering}m{8.8em}|}
  % \caption{秦王政}\
  \toprule
  \SimHei \normalsize 年数 & \SimHei \scriptsize 公元 & \SimHei 大事件 \tabularnewline
  % \midrule
  \endfirsthead
  \toprule
  \SimHei \normalsize 年数 & \SimHei \scriptsize 公元 & \SimHei 大事件 \tabularnewline
  \midrule
  \endhead
  \midrule
  元年 & 232 & \tabularnewline\hline
  二年 & 233 & \tabularnewline\hline
  三年 & 234 & \tabularnewline\hline
  四年 & 235 & \tabularnewline\hline
  五年 & 236 & \tabularnewline\hline
  六年 & 237 & \tabularnewline\hline
  七年 & 238 & \tabularnewline
  \bottomrule
\end{longtable}

\subsubsection{赤乌}

\begin{longtable}{|>{\centering\scriptsize}m{2em}|>{\centering\scriptsize}m{1.3em}|>{\centering}m{8.8em}|}
  % \caption{秦王政}\
  \toprule
  \SimHei \normalsize 年数 & \SimHei \scriptsize 公元 & \SimHei 大事件 \tabularnewline
  % \midrule
  \endfirsthead
  \toprule
  \SimHei \normalsize 年数 & \SimHei \scriptsize 公元 & \SimHei 大事件 \tabularnewline
  \midrule
  \endhead
  \midrule
  元年 & 238 & \tabularnewline\hline
  二年 & 239 & \tabularnewline\hline
  三年 & 240 & \tabularnewline\hline
  四年 & 241 & \tabularnewline\hline
  五年 & 242 & \tabularnewline\hline
  六年 & 243 & \tabularnewline\hline
  七年 & 244 & \tabularnewline\hline
  八年 & 245 & \tabularnewline\hline
  九年 & 246 & \tabularnewline\hline
  十年 & 247 & \tabularnewline\hline
  十一年 & 248 & \tabularnewline\hline
  十二年 & 249 & \tabularnewline\hline
  十三年 & 250 & \tabularnewline\hline
  十四年 & 251 & \tabularnewline
  \bottomrule
\end{longtable}

\subsubsection{太元}

\begin{longtable}{|>{\centering\scriptsize}m{2em}|>{\centering\scriptsize}m{1.3em}|>{\centering}m{8.8em}|}
  % \caption{秦王政}\
  \toprule
  \SimHei \normalsize 年数 & \SimHei \scriptsize 公元 & \SimHei 大事件 \tabularnewline
  % \midrule
  \endfirsthead
  \toprule
  \SimHei \normalsize 年数 & \SimHei \scriptsize 公元 & \SimHei 大事件 \tabularnewline
  \midrule
  \endhead
  \midrule
  元年 & 251 & \tabularnewline\hline
  二年 & 252 & \tabularnewline
  \bottomrule
\end{longtable}

\subsubsection{神凤}

\begin{longtable}{|>{\centering\scriptsize}m{2em}|>{\centering\scriptsize}m{1.3em}|>{\centering}m{8.8em}|}
  % \caption{秦王政}\
  \toprule
  \SimHei \normalsize 年数 & \SimHei \scriptsize 公元 & \SimHei 大事件 \tabularnewline
  % \midrule
  \endfirsthead
  \toprule
  \SimHei \normalsize 年数 & \SimHei \scriptsize 公元 & \SimHei 大事件 \tabularnewline
  \midrule
  \endhead
  \midrule
  元年 & 252 & \tabularnewline
  \bottomrule
\end{longtable}


%%% Local Variables:
%%% mode: latex
%%% TeX-engine: xetex
%%% TeX-master: "../../Main"
%%% End:

%% -*- coding: utf-8 -*-
%% Time-stamp: <Chen Wang: 2021-11-01 11:36:41>

\subsection{会稽王孫亮\tiny(252-258)}

\subsubsection{生平}

孫亮(243年-260年),字子明,是中國三國時代吳國的第二代君主,在位六年(252年-258年),後世史書多稱之為吳廢帝、會稽王。

孫亮生于赤乌六年(243年),是吳大帝孫權的幼子,因此特别受到疼爱。孙亮出生的时候,他的長兄孫登、二兄孫慮早已去世。当时的皇太子为三兄孫和,后来孙和被陷害廢去太子之位。於是赤乌十三年(250年)孫權便立孫亮為皇太子,不久又立其母潘淑为皇后。

潘皇后於神凤元年(252年)被宫女所杀,同年孫權也去世,孫亮繼位,時為四月廿八日丁酉(5月23日)。

孫亮登基时年方十岁,却聪明伶俐,受到大臣的尊敬。孫亮曾欲喫酸梅,讓黃門到庫裏去取蜂蜜,蜜中有鼠屎;就召來守庫官詢問,守庫官叩頭謝罪。少帝說:“黃門從你那兒要過蜂蜜嗎?”守庫官說:“曾要過,我沒敢給他。”黃門不服。少帝讓人破開鼠屎,屎中是乾燥的,於是他大笑著對左右說:“如果鼠屎事先就在蜜中,那麽裏外都應是濕的,現在外面濕而裏面乾燥,這必定是黃門放進去的。”詰問黃門,他果然服了罪。左右之人都很震驚恐懼。

孫亮即位之初,諸葛恪、滕胤、孙峻、吕据受顾命之托輔政孙亮(孙弘本来也是顾命大臣之一,因夺权失败而被诸葛恪先行杀害),又有旧臣吕岱、丁奉等人。曹魏乘孙权驾崩之际,于建兴元年(252年)11月发动东兴之战,结果却被太傅諸葛恪為統帥的吴军大败而归。第二年諸葛恪依仗顾命之托,不顾众臣劝阻欲乘胜出兵北伐魏國,但最後因瘟疫而失敗。

铩羽而归后的诸葛恪显得愈发刚愎自用,最终招致万民所怨、众口所嫌。建兴二年(253年),孫峻利用这个机会说服孙亮,于是在宴会上發動政變,殺死諸葛恪。孫峻因功出任丞相。

孫峻为人骄矜险害,动辄使用重刑,因此招致不少人的不满,但最後反对他的人均事敗被迫自殺或處死。他与滕胤、吕据两位顾命大臣虽然关系谈不上友好,但还能够一起融洽的共事,朝廷高层因此平静了一段时间。

255年(五凤二年),孫峻帶兵與魏國於淮河一帶交戰獲勝,魏將文欽投降。次年,孫峻派遣呂據等將領進攻魏國,但孫峻在戰爭期間病逝,由從弟孫綝接掌權力。呂岱亦於是年去世。因孙綝本不是大帝所指定的顾命大臣,呂據、文欽对孙綝完全继承孫峻权力一事非常不满,要求封滕胤為丞相。孫綝沒有理會他們的訴求,改封滕胤为大司马。于是滕胤和呂據发动政變,却反遭孙綝所殺,自此五位顾命大臣已经全部亡去。另一位將領王惇密謀殺死孫綝,亦事敗被殺。

257年(太平二年),孫亮親政,他对孙綝轻视自己的态度感到非常厌恶,於是推行多項措施(如訓練少年軍)以準備推翻他。同年,魏國的諸葛誕在壽春發動叛亂,把兒子諸葛靚送到吳國做人質。孫綝派兵協助諸葛誕但最終失敗。孙綝将失败的缘由归于大都督朱异并在镬里杀害了他,其他一些參戰的將領也因為怕被孫綝殺死而投降了魏國。孙綝返回建业後,得知孙亮对他有所戒备,内心也很恐惧,于是称病不上朝并命自己兄弟把守宫门以求自保。

258年(太平三年),孫亮因孙綝不听自己指挥进军并擅杀朱异等事对孙綝不满到了极点,于是与全尚,全公主,刘承等人密谋除掉孙綝。但消息被孙綝的从姐(全尚之妻)或从外甥女(全皇后)泄露给孙綝。孙綝获悉密报后,于6月26日率先包围皇宫,以孙亮患有精神病为由强迫众臣同意将孫亮廢為會稽王,改立孫休為帝。和孫亮一起策劃政變的大臣都被孫綝殺死。群臣也因为畏惧孙綝的声势不敢多言。

260年,孫亮的封地會稽傳出謠言,說孫亮將返回建業復辟;而孫亮的侍從亦聲稱孫亮在祭祀時口出惡言。

經審判後,孫亮再被貶為侯官侯(侯官,今福建省閩侯縣)和流放,途中死去。據《三國志》記載,孫亮可能是自殺,也可能是被孫休派人毒死的。孫亮死時只有18歲。

吴国灭亡后,吴国的少府卿丹阳人戴显上表朝廷,于是迎回孙亮遗体安葬赖乡(今江苏省南京市溧水区)

\subsubsection{建兴}

\begin{longtable}{|>{\centering\scriptsize}m{2em}|>{\centering\scriptsize}m{1.3em}|>{\centering}m{8.8em}|}
  % \caption{秦王政}\
  \toprule
  \SimHei \normalsize 年数 & \SimHei \scriptsize 公元 & \SimHei 大事件 \tabularnewline
  % \midrule
  \endfirsthead
  \toprule
  \SimHei \normalsize 年数 & \SimHei \scriptsize 公元 & \SimHei 大事件 \tabularnewline
  \midrule
  \endhead
  \midrule
  元年 & 252 & \tabularnewline\hline
  二年 & 253 & \tabularnewline
  \bottomrule
\end{longtable}

\subsubsection{五凤}

\begin{longtable}{|>{\centering\scriptsize}m{2em}|>{\centering\scriptsize}m{1.3em}|>{\centering}m{8.8em}|}
  % \caption{秦王政}\
  \toprule
  \SimHei \normalsize 年数 & \SimHei \scriptsize 公元 & \SimHei 大事件 \tabularnewline
  % \midrule
  \endfirsthead
  \toprule
  \SimHei \normalsize 年数 & \SimHei \scriptsize 公元 & \SimHei 大事件 \tabularnewline
  \midrule
  \endhead
  \midrule
  元年 & 254 & \tabularnewline\hline
  二年 & 255 & \tabularnewline\hline
  三年 & 256 & \tabularnewline
  \bottomrule
\end{longtable}

\subsubsection{太平}

\begin{longtable}{|>{\centering\scriptsize}m{2em}|>{\centering\scriptsize}m{1.3em}|>{\centering}m{8.8em}|}
  % \caption{秦王政}\
  \toprule
  \SimHei \normalsize 年数 & \SimHei \scriptsize 公元 & \SimHei 大事件 \tabularnewline
  % \midrule
  \endfirsthead
  \toprule
  \SimHei \normalsize 年数 & \SimHei \scriptsize 公元 & \SimHei 大事件 \tabularnewline
  \midrule
  \endhead
  \midrule
  元年 & 256 & \tabularnewline\hline
  二年 & 257 & \tabularnewline\hline
  三年 & 258 & \tabularnewline
  \bottomrule
\end{longtable}


%%% Local Variables:
%%% mode: latex
%%% TeX-engine: xetex
%%% TeX-master: "../../Main"
%%% End:

%% -*- coding: utf-8 -*-
%% Time-stamp: <Chen Wang: 2021-11-01 11:36:46>

\subsection{景帝孫休\tiny(258-264)}

\subsubsection{生平}

吴景帝孫休(235年-264年9月3日),字子烈,為孫權第六子,在父親孫權、弟孫亮後繼任為吳國第三任皇帝,在位六年。

孙休生于嘉禾四年(235年),母王夫人,13岁时,跟随谢慈和盛冲就学。

太元二年(252年)受封為琅琊王,居於虎林,當時十八歲。同年四月,大帝因患風疾病死於建業,孫休的嫡弟孫亮繼位,由太傅諸葛恪秉政,諸葛恪不欲諸王在濱江兵馬之地,遂徙孫休至丹楊郡。其後又因丹楊太守李衡數次以事侵擾孫休,孫休上書乞求徙往其他郡,孫亮遂下詔徙孫休至會稽郡。

孙休的岳母和姐姐孙鲁育被权臣孙峻冤杀,孙休害怕,将妻子朱氏送回建业,执手泣别。朱氏到建业,被孙峻遣回。

太平三年九月廿六日(258年11月9日),宗室孫綝發動政變,罷黜孫亮為會稽王,立孫休為帝,孫休三次辭讓而受,改元永安,封孫綝為丞相,孫綝五兄弟皆封侯掌禁军,權傾朝野,時為十月十八日己卯(11月30日)。

孙休先假意麻痹孙綝,将举报孙綝谋反之人,交給孙綝处置,后又加孙綝弟孙恩为侍中分其权,年末设宴请孙綝,孙綝称病不赴,孙休十多次派人去请,孙綝终于赴宴,席间想借故早退,被丁奉等擒住,孙休历数孫綝罪状斩之,灭其三族。孙綝的权臣地位继承自其堂兄权臣孙峻,孙休又将孙峻棺材削薄后重新下葬,并将孙峻、孙綝开除宗籍,称之为“故峻”“故綝”,并赦免被孙峻、孙綝所害之人。

孫休在位期間,以衛將軍濮陽興為丞相,廷尉丁密、光祿勳孟宗為左右御史大夫。布典宮省,興關軍國。

孫休崇尚文化。永安元年創設國學,置學官,立五經博士,選送吏中及將吏子弟好學者就學,为南京太学之滥觞,韋昭為首任博士祭酒。

武功方面,无甚建树,曾图先统一南方。永安七年(264年)二月,趁蜀中無主,西征巴蜀,以鎮軍将军陸抗、撫軍将军步協、征西將軍留平、建平太守盛曼,率大軍圍蜀巴東守將羅憲。魏使將軍胡烈率步騎二萬侵擾西陵,以救羅憲,陸抗等遂引軍退回吳國,最终丝毫未能夺取蜀汉故地。

同年七月,孫休病重,不能語,尚能書;同月廿四日(9月2日)大赦天下,但次日(9月3日)以三十歲英年早逝。丞相濮陽興、左將軍張布遊說朱皇后,因蜀國初亡,而交阯攜叛,國內震懼,希望立長君,所以意欲孙休亡兄孫和之子孫皓嗣位。

孫皓繼位後,於元興元年(264年)十一月,誅殺濮陽興及張布。又於甘露元年(265年)七月,逼殺孫休之妻景皇后朱氏,只於苑中小屋治喪,又送孫休四子於吳小城,再復追殺年長的孫{\fzk 𩅦}及孫{\fzk 𩃙}。

\subsubsection{永安}

\begin{longtable}{|>{\centering\scriptsize}m{2em}|>{\centering\scriptsize}m{1.3em}|>{\centering}m{8.8em}|}
  % \caption{秦王政}\
  \toprule
  \SimHei \normalsize 年数 & \SimHei \scriptsize 公元 & \SimHei 大事件 \tabularnewline
  % \midrule
  \endfirsthead
  \toprule
  \SimHei \normalsize 年数 & \SimHei \scriptsize 公元 & \SimHei 大事件 \tabularnewline
  \midrule
  \endhead
  \midrule
  元年 & 258 & \tabularnewline\hline
  二年 & 259 & \tabularnewline\hline
  三年 & 260 & \tabularnewline\hline
  四年 & 261 & \tabularnewline\hline
  五年 & 262 & \tabularnewline\hline
  六年 & 263 & \tabularnewline\hline
  七年 & 264 & \tabularnewline
  \bottomrule
\end{longtable}



%%% Local Variables:
%%% mode: latex
%%% TeX-engine: xetex
%%% TeX-master: "../../Main"
%%% End:

%% -*- coding: utf-8 -*-
%% Time-stamp: <Chen Wang: 2019-12-18 13:05:26>

\subsection{末帝\tiny(264-280)}

\subsubsection{生平}

吴末帝孙皓(243年-284年),字元宗,幼名彭祖,又字皓宗,《三国志》原名为孫晧。為廢太子孫和之子,吳大帝孫權之孫,在位十七年(264年—280年),是三国時期孫吴的第四位,同時也是最後一位皇帝。

吳景帝孙休逝世時,太子非常年幼。因當時吳國處於內憂外患之中,大臣們便合議改立較年長的孫皓即位。孫皓即位後,初期雖然英明施政並多行善舉,在西陵之戰一度挽回吳國的厄運,但中後期實行暴政並過度役使民力,加深了亡國危機。最終,吳國於280年被西晉征服,三國時代也因此終結。

孫皓並無廟號與謚號,後世史書中多將孫皓稱為吳後主、吳末帝,也有用他即位前的封號烏程侯,或是歸晉後的封號歸命侯來指代他。

孫皓出生於赤烏六年(243年),是吳主孫權三子孫和的長子。孫皓的嫡母張妃是張承的女兒,生母何姬是孫和的一名庶妃。在他出生的同一年,孫和被立為太子,直到赤烏十三年(250年)因陷入“二宮之爭”被孫權廢黜,流放到故鄣(今浙江省安吉縣)。太元二年(252年)正月,孫權又將孫和封為南陽王,孫和帶家眷移居封地長沙(郡治今湖南省長沙縣)。不久後,孫權逝世,由十歲的幼子孫亮即位。

孫亮即位後,張妃的舅舅諸葛恪秉持朝政。建興二年(253年),宗室孫峻誅殺諸葛恪後,借故民間有傳言稱諸葛恪想迎孫和即位,而剝奪了孫和的王位,並將孫和流放到新都(治今浙江省淳安縣),隨後賜死,張妃也一同死去。此時的孫皓年僅十二歲,還有三個異母弟弟,何姬為了將孫皓等人撫養長大而保全了性命。太平三年(258年),孫亮被繼承孫峻權力的孫綝廢黜,孫權的六子孫休被立為帝。

孫休即位後,將孫皓封為烏程侯,命孫皓前往封地烏程(今浙江省湖州市)。他的異母弟孫德、孫謙也分別被封為錢塘侯、永安侯。在當烏程侯期間,孫皓與烏程令萬彧相識,彼此交好。永安七年(264年,魏咸熙元年)七月,孫休托孤於丞相濮陽興後逝世。在孫休死後,濮陽興並未遵從他的意愿立太子孫𩅦為帝。當時吳國的盟國蜀漢已經滅亡,交趾一帶又發生了叛亂,大臣們考慮著擁立一位較年長的君主。已升任為左典軍的萬彧便向濮陽興和另一位權臣張布推薦孫皓,稱孫皓英明果斷,有長沙桓王孫策的風範,并且行事遵守法度。濮陽興與張布被萬彧說服,便一起勸朱太后將孫皓迎立為帝。這一年,孫皓23歲。

統治前期(264-268年):孫皓即位後,採取了一系列的舉措來鞏固自己的地位。一方面,他大行封賞,將迎立有功的丞相濮陽興,加封侍中,兼領青州牧,左將軍張布升為驃騎將軍,加封侍中,又把吳國宿將施績、丁奉升為左、右大司馬,以拉攏臣子。另一方面,他發放糧食,救濟窮人,從皇宮放出大量侍女讓她們可以婚配,並放歸宮中圈養的一些野獸,以一系列惠民政策來爭取民心。當時人們都把他稱為明主。

但一段時間後,治國有成、志得意滿的孫皓便顯露出魯莽暴躁、驕傲自滿、迷信以及好酒色的一面。此外,他還將景帝孫休的妻子朱太后貶為景皇后,追謚自己的父親孫和為文皇帝,將自己的生母何姬奉為太后,妻子滕氏立為皇后,將孫休的太子及其它三個兒子封為王爵,以加強自己繼位的合法性。當初擁立他的濮陽興、張布對孫皓的轉變感到震驚和失望,結果被萬彧秘密向孫皓揭發,孫皓將兩人處斬,並夷三族。此時,距離兩人迎立孫皓才過了四個月。之後,孫皓扶植外戚,將滕皇后的父親滕牧和何太后的弟子何洪、何蔣、何植都封為侯。甘露元年(265年,魏咸熙二年,晉泰始元年)七月,孫皓迫使前太后朱氏自殺,又軟禁了孫休的四個兒子,並殺死了其中較年長的兩人。九月,孫皓聽信術士之言(“荊州有王氣,當破揚州”),又為了防禦司馬氏軍事包夾的迫切需要,決定遷都武昌(今湖北省鄂州市)。這一年十二月,繼承司馬昭權力的司馬炎迫使曹魏禪讓,正式建立了晉王朝。

寶鼎元年(266年,晉泰始二年),出使晉國的使臣丁忠回到武昌。孫皓召集群臣宴會,因常侍王蕃酒醉失態而大怒,雖然有滕牧、留平等重臣出面為王蕃求情,但孫皓依然下令將王蕃斬首。孫皓的這一舉動令大臣們感到震驚遺憾,賀邵、陸抗後來在270年代上疏勸諫時,都有舉此事為例來指責孫皓,重臣陸凱(同年任左丞相)更是在上疏中將王蕃比為吳國的關龍逢,隱含有將孫皓比為暴君夏桀之意。當時,晉國因為才吞併蜀地不久,有意與吳國暫時維持和平。但使臣丁忠發現晉國戰備有其漏洞,勸說孫皓攻取弋阳郡(郡治今河南省潢川縣西),遭到時任鎮西大將軍的陸凱堅決反對,孫皓表面上贊同了陸凱的意見,並未出兵,但最後還是與晉國絕交了。八月,孫皓分置左、右丞相,左丞相由陸凱擔任,右丞相則安排自己的親信萬彧擔任。

起初,孫皓遷都武昌後,因土地貧乏,而孫皓施政不當處漸多,所需的供給大多要從長江下游運上來,使江東百姓頗有不滿,兒童再度傳唱孫權定都武昌時的歌謠:「寧飲建業水,不食武昌魚,寧還建業死,不止武昌居」。十月,永安山民施旦聚眾數千人起義,劫持孫皓的異母弟永安侯孫謙後向吳故都建業(今江蘇省南京市)進發,一路上不斷有人加入,到建業城外時已有數萬人之多,但還是被吳將丁固、諸葛靚擊潰,孫謙被救回。孫皓當初聽說施旦謀反的消息後,不僅不擔憂,反倒覺得這是應驗了之前術士說的“荊州有王氣,當破揚州”一事,肯定了自己遷都的決斷,命令數百人到建業城大喊“天子使荊州兵來破揚州賊”,來壓制之前的晦氣。丁固請示皓如何處理孫謙,孫皓下令將孫謙母子一起毒殺,後來他還殺了亡父孙和的嫡子即自己的另一个异母弟孫俊。十二月時,孫皓將都城遷回了建業。

寶鼎二年(267年,晉泰始三年)夏六月,孫皓下令新建更大的宮殿──昭明宮。為了昭明宮的修建,呂秩二千石以下的官吏都被派往山中督伐木料,昭明宮的修建歷時半年,工程耗資巨大,而且耽誤了農時。當時陸凱、華覈等人上疏勸止,孫皓拒絕聽從。

統治中期(268-272年):寶鼎三年(268年,晉泰始四年),孫皓開始向晉國發起攻擊。這一年,他親率大軍屯駐東關(今安徽省含山縣西南),令左大司馬施績攻江夏(今湖北省雲夢縣南),右丞相萬彧攻襄陽(今湖北省襄陽市),右大司馬丁奉、右將軍諸葛靚進攻合肥(今安徽省合肥市西),交州刺史劉俊、前部督脩則、將軍顧容等率攻擊投降晉國的交阯(郡治今越南北寧市)叛軍,但都沒有取得成功。北伐大軍被司馬望大軍所拒,兩路主力施績、丁奉分別為晉將胡烈、司馬駿所敗,而南征交阯軍隊更是被晉將楊稷大敗,劉俊、脩則戰死,顧容率殘軍退守合浦(郡治今廣西省合浦縣東北)。

建衡元年(269年,晉泰始五年),孫皓派監軍虞汜、威南將軍薛珝、蒼梧太守陶璜從荊州出發,監軍李勖、督軍徐存從建安海路出發,令兩軍在合浦會合共同剿滅交阯叛軍。此外,還派遣右大司馬丁奉再次北征,攻打谷陽(今安徽省靈壁縣)。但到了建衡二年(270年,晉泰始六年),丁奉部在渦口(今安徽省懷遠縣)一帶被晉將牽弘擊退,李勖部以道路不通為由,殺死向導馮斐後率軍無功而返。孫皓為此大怒,丁奉的向導被處死,李勖更是在被何定揭發後,同徐存被全家誅殺。不久後,何定率領五千人馬到夏口(今湖北省武漢市)打獵,吳宗室前將軍、夏口督孫秀害怕是孫皓令何定來抓自己,提前帶領家送眷數百人投奔晉國。晉武帝拜孫秀為驃騎將軍,儀同三司,封會稽公,禮遇備至。

建衡三年(271年,晉泰始七年)正月,孫皓聽信刁玄增改的讖文(“黃旗紫蓋,見於東南,終有天下者,荊、揚之君!”),認為自己是天命所歸,不顧眾人反對,用車載著自己的母親、妻子、孩子以及後宮上千人,親率大軍從牛渚(今安徽省當塗縣)西進伐晉。晉軍派司馬望率軍駐屯在壽春(今安徽省壽縣)作為防備。結果孫皓的軍隊途中被大雪所阻,士兵忍受天寒地凍的同時還要負責拉孫皓的車隊,都難以忍受這樣的勞苦,軍中漸漸出現倒戈的傳言,因此孫皓只好下令還師。孫皓還師前,右丞相萬彧與右大司馬丁奉、左將軍留平曾私下商議先自行回去,後被孫皓得知,雖然心懷不滿,但介於三人都是老臣並沒有馬上處置。當年,丁奉病逝。翌年,孫皓試圖用毒酒毒死萬彧和留平,二人卻都倖免未死,但不久後,萬彧自殺,留平愁悶而死。

在這兩年間裡,孫吳在軍事上接連取得了重大勝利,使得孫皓的自傲心大幅膨脹。先是在建衡三年(271年,晉泰始七年),南征的薛珝、虞汜、陶璜攻破交阯,擒殺晉軍守將,並收復了九真(郡治今越南清化市)、日南(郡治今越南洞海市南)兩郡,後又平定了扶嚴夷,使持續多年的交阯之亂暫告停歇。接著於鳳凰元年(272年,晉泰始八年)秋八月,陸抗成功討伐了因擔心被孫皓加害而叛投西晉的西陵督步闡,不僅成功收復了戰略要地西陵(今湖北省宜昌市),將步闡等人夷三族,並且擊退了由名將羊祜率領的五萬大軍,圍殲晉將楊肇的三萬援軍。西陵大捷之後,孫皓因為兩年內成功收復失土及大敗西晉,越發自志得意滿,更加相信自己是有上天相助,還召術士尚廣為他占卜看是否能取得天下,占卜的結果顯示他將在庚子年“青蓋入洛陽”。孫皓非常高興,從此專門謀劃統一大業,頻繁派遣軍隊襲擊晉國邊境,但都勞而無功。陸抗上疏反對孫皓的窮兵黷武,希望孫皓看清晉強吳弱的事實,建議“蹔息進取小規,以畜士民之力,觀釁伺隙”,又上疏指出西陵、建平戰略地位的重要,請求加強兩地的兵力。建平太守吾彥也憑借從長江上游漂下的大量木屑,斷定晉國將從巴蜀由水路大舉伐吳,上書孫皓請求加強防備。但孫皓不僅沒有重視這些意見,反而在鳳凰三年(274年,晉泰始十年)陸抗病逝後,將他的兵馬一分為五,交給陸抗的五個兒子分別統領。

統治後期(272-279年):軍事上取得耀人成果的同時,吳國內部卻越發不穩定。

自建衡元年(269年,晉泰始五年)左丞相陸凱病逝後,左大司馬施績、右大司馬丁奉、司空孟仁、右丞相萬彧、左將軍留平、太尉范慎、司徒丁固、大司馬陸抗等重臣在六年時間裡先後逝世,吳國有名望的舊臣死亡殆盡。當孫皓忌憚、尊重的重臣都不復存在之後,他的施政也更加殘暴,對於其他忠臣的勸諫也就不再接納容忍。大約從272年開始,他每次召集群臣宴會,都要故意讓每個人都喝得大醉,讓人在邊上專門檢舉他們的過失。甚至剝人臉皮,挖人眼珠。272年後孫皓對勸諫忠臣的容忍度也大幅下降,不惜痛下殺手以杜絕煩人的諫言:大司農樓玄因為多次直諫忤逆孫皓,被流放廣州,服毒而死;中書令賀邵也因直諫而使孫皓痛恨,當賀邵因中風不能說話,被孫皓懷疑是裝病,拷打致死;侍中韋昭因多次堅持己見,被以不聽從詔命為由處死;東觀令華覈多次上書勸諫,結果為了一些小事被免官遣返;豫章太守張俊因為給孫奮的母親掃墓,而被孫皓處以車裂極刑,並夷三族;會稽太守車浚因為開倉賑濟飢民,被懷疑收買人心而處斬;湘東太守張詠因為征稅不足,被孫皓派人斬殺;尚書熊睦對孫皓稍加勸阻,就被孫皓派人用刀環生生打死;甚至連他曾經寵信的何定、陳聲、張俶也被他處決,其中張俶受車裂之刑,陳聲更是被鋸斷頭顱而死。

相比於272年後孫皓殘暴的高壓政策,晉國都督荊州諸軍事的羊祜則對吳國展開懷柔政策。天璽元年(276年,晉泰始十二年),繼孫秀、步闡之後,吳國又一位重要將領——吳國宗室武衛將軍、京下督孫楷叛投晉國,在此前後,平虜將軍孟泰、偏將軍王嗣、威北將軍嚴聰、揚威將軍嚴整、偏將軍朱買、邵凱、夏祥、昭武將軍劉翻、厲武將軍祖始也都紛紛向晉軍投降。但孫皓絲毫沒有感受到危機的來臨。在吳國在接下来的几年裡,各地奉承他的人爭相獻上有吉祥象徵的事物,讓迷信的孫皓始終堅信自己將一統天下。

天紀三年(279年,晉泰始五年),郭馬攻殺廣州督虞授,在廣州發起叛亂。孫皓派遣滕脩、陶濬、陶璜率軍剿滅郭馬叛軍。冬十一月,晉武帝司馬炎依羊祜生前擬制的計劃,令鎮軍將軍司馬伷、安東將軍王渾、建威將軍王戎、平南將軍胡奮、鎮南大將軍杜預、龍驤將軍王濬、巴東監軍唐彬等分六路大舉伐吳。天紀四年(280年,晉泰始六年)正月,杜預、王渾兩軍分別向江陵(今湖北省荊州市)、橫江(今安徽省和縣)進軍,接連進克吳軍要塞。王渾部率先攻克尋陽(今湖北省黃梅縣西南)、賴鄉等城,屯兵橫江,距建業僅百里之遙。二月,在王濬、唐彬部和杜預部、胡奮部、王戎部的攻擊下,荊州的軍事重鎮丹陽(今湖北省秭歸縣)、西陵、荊門(今湖北省宜昌市東南)、夷道(今湖北省宜都市)、樂鄉(今湖北省松滋縣東)、江陵、江安(今湖北省公安縣)、夏口、武昌等先後失守,吳軍僅戰死或投降的都督、監軍就有十四人,牙門將、郡守一級的將領更是有一百二十多人,荊南各郡望風而降。

三月,由丞相張悌率領的吳軍精銳在版橋(今安徽省和縣)被王渾部擊潰,張悌、孫震、沈瑩全部戰死。孫皓自知滅亡在即,在給舅舅何植的信中自責道:“天匪亡吳,孤所招也。瞑目黃壤,當複何顏見四帝乎!”不久後,何植也向王渾軍投降。這時,王濬率水軍從武昌順流而下,直取建業。吳主孫皓派遣張象率水軍一萬餘人前往抵擋,但王濬大軍一到,張象便立即投降。孫皓又另遣陶濬率軍兩萬迎敵,結果士兵全部連夜逃竄。孫皓周圍數百人又請求他殺死寵臣岑昬,他不得以而被迫答應。後來,孫皓聽從光祿勛薛瑩和中書令胡沖的計策,分別遣送使節向王濬、司馬伷、王渾請降,試圖分化晉軍,未能奏效。三月壬寅日(280年5月1日),王濬率大軍進入石頭城,孫皓率太子孫瑾、魯王孫虔等二十一人出降,全家被遣送至洛陽。吳國至此滅亡。晉武帝下詔封孫皓為歸命侯。孫皓決定投降後,為了讓晉軍順利接收各地,廣發勸降書信給臣僚,信中寫道:“孤以不德,忝继先轨。处位历年,政教凶悖,遂令百姓久困涂炭,至使一朝归命有道,社稷倾覆,宗庙无主,惭愧山积,没有余罪。自惟空薄,过偷尊号,才琐质秽,任重王公,故《周易》有折鼎之诫,诗人有彼其之讥。自居宫室。仍抱笃疾,计有不足,思虑失中,多所荒替。边侧小人,因生酷虐,虐毒横流,忠顺被害。闇昧不觉,寻其壅蔽,孤负诸君,事已难图,覆水不可收也。今大晋平治四海,劳心务于擢贤,诚是英俊展节之秋也。管仲极雠,桓公用之,良、平去楚,入为汉臣,舍乱就理,非不忠也。莫以移朝改朔,用损厥志。嘉勖休尚,爱敬动静。夫复何言,投笔而已!”

五月,孫皓到達洛陽後,得到了晉武帝較為優厚的待遇。後來晉武帝大會群臣時,召孫皓進見,孫皓上前叩首請罪,晉武帝對孫皓說:“朕設此座以待卿久矣。”孫皓回應道:“臣於南方,亦設此座以待陛下。”賈充故意刁難他說:“聞君在南方鑿人目,剝人臉皮,此何等刑也?”孫皓回答說:“人臣有弒其君及姦回不忠者,則加此刑耳。”賈充曾指使手下殺害魏帝曹髦,聽了孫皓的話後羞愧不已,而孫皓本人則面不改色。晉武帝曾与王济下棋,孙皓在旁边,晉武帝对孙皓说:“何以好剥人面皮?”孙皓回应道:“见无礼于君者则剥之。”王济当时把脚伸到了棋盘下,因而孙皓讥讽王济。晋武帝有一次问孙皓:“闻南人好作《尔汝歌》,颇能为不?”孙皓正饮酒,于是举羽觞吟诵:“昔与汝为邻,今为汝做臣;上汝一杯酒,令汝寿万春!”孙皓直呼晋武帝为汝,晋武帝感到后悔,自讨没趣。太康四年十二月(284年初),孫皓在洛陽逝世,享年四十二歲,葬於河南縣界。

孫皓是中國歷史上有名的暴君,生性多疑且殘暴,設立諸多酷刑。他曾殺死或流放多名重要宗室,如殺害再从兄弟孫奉,流放从兄弟孙基、孙壹,誅殺五叔孫奮及其五子,殺死異母弟孫謙、孫俊等。對大臣,他也常常施以重刑,僅丞相一級的官員為例:除張悌在亡國之際戰死外,濮陽興被流放處死,夷三族,萬彧被譴自殺,全家遭流放;陸凱死後數年,全家被處以流放。。此外,孙皓非常迷信,常憑借運曆、望氣、卜筮、讖緯之類的原因來決定如遷都、用兵、皇后廢立等重大事件,並因此一直堅信自己將統一天下。

孫皓富有才氣,能吟詩,並有一定書法造詣。他曾在宴會上應晉武帝之邀,當場作《爾汝歌》一首:“昔與汝為鄰,今與汝為臣。上汝一杯酒,令汝壽萬春。”

同絕大多數亡國之君一樣,對孫皓的評價以負面評價為主。從孫皓受到的各種評價來看,他是一個典型的暴君形象,吳國重臣陸凱、陸抗在給孫皓的上疏中,曾多次暗示他堪比夏桀、商紂。曾擔任吳國光祿勛的薛瑩稱在孫皓當政時“昵近小人,刑罰妄加,大臣大將無所親信,人人憂恐,各不自安”。而孫皓被晉軍俘虜後,也批判自己“虐毒横流,忠顺被害。闇昧不觉,寻其壅蔽”,要求未降的吳軍儘快投降。此外,有些吳國人士曾對孫皓做出正面評價,如吳將吾彥在歸降西晉之後,在晉武帝面前力讚孫皓的英明:「吳主英俊,宰輔賢明」;而孫皓早期的好友萬彧也稱讚孫皓好學不倦、英明果斷,有孫策的風範。

在敵國眼中,272年後孫皓因殘暴所導致的吳國內部不安,已經促成了進攻吳國的最佳時機,晉臣張華曾對晉武帝說:“吳主荒淫驕虐,誅殺賢能,當今討之,可不勞而定”。晉將羊祜更是在上疏中稱:“孫皓暴虐已甚,於今可不戰而克。若皓不幸而沒,吳人更立令主,雖有百萬之眾,長江未可窺也,將為後患矣!”認為如果孫皓死掉,伐吳的難度將會大大增加。晉代史臣陳壽、孫盛在批評孫皓的暴虐以外,還指責晉武帝對孫皓的處置太過寬厚,認為像孫皓這樣的“肆行殘暴,忠諫者誅,讒諛者進,虐用其民,窮淫極侈”的禍國殃民之君,“梟首素旗,猶不足以謝冤魂”,“宜腰首分離,以謝百姓”。

同時,關於孫皓也有一些其它方面的評價,如晉臣秦秀在為滅吳的王濬請功時,曾言及孫皓對晉國用兵給晉國帶來的心理威脅,稱“以孫皓之虛名,足以驚動諸夏,每一小出,雖聖心知其垂亡,然中國輒懷惶怖。”後世的李世民則從另一個角度出發,將孫皓前期“權施恩惠之風”與王莽稱帝前“偽行仁義之道”相提並論,認為兩人的失敗就在於“有始無終”,以此得出二人都迅速覆亡的結論。唐代的朱敬則在批評孫皓的同時,還對他的權謀和才華予以一定程度的肯定。孫皓的書跡流傳到唐代,庾肩吾在他所著《書品》中则把孙皓评为「中中」,与曹操杜预并列,称“魏帝(曹操)筆墨雄贍、呉主(孫皓)體裁綿密”,書法家韋續则把他的行隸,評為「下中」品。

\subsubsection{元兴}

\begin{longtable}{|>{\centering\scriptsize}m{2em}|>{\centering\scriptsize}m{1.3em}|>{\centering}m{8.8em}|}
  % \caption{秦王政}\
  \toprule
  \SimHei \normalsize 年数 & \SimHei \scriptsize 公元 & \SimHei 大事件 \tabularnewline
  % \midrule
  \endfirsthead
  \toprule
  \SimHei \normalsize 年数 & \SimHei \scriptsize 公元 & \SimHei 大事件 \tabularnewline
  \midrule
  \endhead
  \midrule
  元年 & 264 & \tabularnewline\hline
  二年 & 265 & \tabularnewline
  \bottomrule
\end{longtable}


\subsubsection{甘露}

\begin{longtable}{|>{\centering\scriptsize}m{2em}|>{\centering\scriptsize}m{1.3em}|>{\centering}m{8.8em}|}
  % \caption{秦王政}\
  \toprule
  \SimHei \normalsize 年数 & \SimHei \scriptsize 公元 & \SimHei 大事件 \tabularnewline
  % \midrule
  \endfirsthead
  \toprule
  \SimHei \normalsize 年数 & \SimHei \scriptsize 公元 & \SimHei 大事件 \tabularnewline
  \midrule
  \endhead
  \midrule
  元年 & 265 & \tabularnewline\hline
  二年 & 266 & \tabularnewline
  \bottomrule
\end{longtable}

\subsubsection{宝鼎}

\begin{longtable}{|>{\centering\scriptsize}m{2em}|>{\centering\scriptsize}m{1.3em}|>{\centering}m{8.8em}|}
  % \caption{秦王政}\
  \toprule
  \SimHei \normalsize 年数 & \SimHei \scriptsize 公元 & \SimHei 大事件 \tabularnewline
  % \midrule
  \endfirsthead
  \toprule
  \SimHei \normalsize 年数 & \SimHei \scriptsize 公元 & \SimHei 大事件 \tabularnewline
  \midrule
  \endhead
  \midrule
  元年 & 266 & \tabularnewline\hline
  二年 & 267 & \tabularnewline\hline
  三年 & 268 & \tabularnewline\hline
  四年 & 269 & \tabularnewline
  \bottomrule
\end{longtable}

\subsubsection{建衡}

\begin{longtable}{|>{\centering\scriptsize}m{2em}|>{\centering\scriptsize}m{1.3em}|>{\centering}m{8.8em}|}
  % \caption{秦王政}\
  \toprule
  \SimHei \normalsize 年数 & \SimHei \scriptsize 公元 & \SimHei 大事件 \tabularnewline
  % \midrule
  \endfirsthead
  \toprule
  \SimHei \normalsize 年数 & \SimHei \scriptsize 公元 & \SimHei 大事件 \tabularnewline
  \midrule
  \endhead
  \midrule
  元年 & 269 & \tabularnewline\hline
  二年 & 270 & \tabularnewline\hline
  三年 & 271 & \tabularnewline
  \bottomrule
\end{longtable}

\subsubsection{凤凰}

\begin{longtable}{|>{\centering\scriptsize}m{2em}|>{\centering\scriptsize}m{1.3em}|>{\centering}m{8.8em}|}
  % \caption{秦王政}\
  \toprule
  \SimHei \normalsize 年数 & \SimHei \scriptsize 公元 & \SimHei 大事件 \tabularnewline
  % \midrule
  \endfirsthead
  \toprule
  \SimHei \normalsize 年数 & \SimHei \scriptsize 公元 & \SimHei 大事件 \tabularnewline
  \midrule
  \endhead
  \midrule
  元年 & 272 & \tabularnewline\hline
  二年 & 273 & \tabularnewline\hline
  三年 & 274 & \tabularnewline
  \bottomrule
\end{longtable}

\subsubsection{天册}

\begin{longtable}{|>{\centering\scriptsize}m{2em}|>{\centering\scriptsize}m{1.3em}|>{\centering}m{8.8em}|}
  % \caption{秦王政}\
  \toprule
  \SimHei \normalsize 年数 & \SimHei \scriptsize 公元 & \SimHei 大事件 \tabularnewline
  % \midrule
  \endfirsthead
  \toprule
  \SimHei \normalsize 年数 & \SimHei \scriptsize 公元 & \SimHei 大事件 \tabularnewline
  \midrule
  \endhead
  \midrule
  元年 & 275 & \tabularnewline\hline
  二年 & 276 & \tabularnewline
  \bottomrule
\end{longtable}

\subsubsection{天玺}

\begin{longtable}{|>{\centering\scriptsize}m{2em}|>{\centering\scriptsize}m{1.3em}|>{\centering}m{8.8em}|}
  % \caption{秦王政}\
  \toprule
  \SimHei \normalsize 年数 & \SimHei \scriptsize 公元 & \SimHei 大事件 \tabularnewline
  % \midrule
  \endfirsthead
  \toprule
  \SimHei \normalsize 年数 & \SimHei \scriptsize 公元 & \SimHei 大事件 \tabularnewline
  \midrule
  \endhead
  \midrule
  元年 & 276 & \tabularnewline
  \bottomrule
\end{longtable}

\subsubsection{天纪}

\begin{longtable}{|>{\centering\scriptsize}m{2em}|>{\centering\scriptsize}m{1.3em}|>{\centering}m{8.8em}|}
  % \caption{秦王政}\
  \toprule
  \SimHei \normalsize 年数 & \SimHei \scriptsize 公元 & \SimHei 大事件 \tabularnewline
  % \midrule
  \endfirsthead
  \toprule
  \SimHei \normalsize 年数 & \SimHei \scriptsize 公元 & \SimHei 大事件 \tabularnewline
  \midrule
  \endhead
  \midrule
  元年 & 277 & \tabularnewline\hline
  二年 & 278 & \tabularnewline\hline
  三年 & 279 & \tabularnewline\hline
  四年 & 280 & \tabularnewline
  \bottomrule
\end{longtable}


%%% Local Variables:
%%% mode: latex
%%% TeX-engine: xetex
%%% TeX-master: "../../Main"
%%% End:


%%% Local Variables:
%%% mode: latex
%%% TeX-engine: xetex
%%% TeX-master: "../../Main"
%%% End:


%%% Local Variables:
%%% mode: latex
%%% TeX-engine: xetex
%%% TeX-master: "../Main"
%%% End:
 % 三国
% %% -*- coding: utf-8 -*-
%% Time-stamp: <Chen Wang: 2019-10-15 11:10:10>

\chapter{西晋\tiny(265-316)}

%% -*- coding: utf-8 -*-
%% Time-stamp: <Chen Wang: 2018-07-10 22:23:17>

\section{武帝\tiny(266-290)}

\subsection{泰始}

\begin{longtable}{|>{\centering\scriptsize}m{2em}|>{\centering\scriptsize}m{1.3em}|>{\centering}m{8.8em}|}
  % \caption{秦王政}\
  \toprule
  \SimHei \normalsize 年数 & \SimHei \scriptsize 公元 & \SimHei 大事件 \tabularnewline
  % \midrule
  \endfirsthead
  \toprule
  \SimHei \normalsize 年数 & \SimHei \scriptsize 公元 & \SimHei 大事件 \tabularnewline
  \midrule
  \endhead
  \midrule
  元年 & 265 & \tabularnewline\hline
  二年 & 266 & \tabularnewline\hline
  三年 & 267 & \tabularnewline\hline
  四年 & 268 & \tabularnewline\hline
  五年 & 269 & \tabularnewline\hline
  六年 & 270 & \tabularnewline\hline
  七年 & 271 & \tabularnewline\hline
  八年 & 272 & \tabularnewline\hline
  九年 & 273 & \tabularnewline\hline
  十年 & 274 & \tabularnewline
  \bottomrule
\end{longtable}

\subsection{咸宁}


\begin{longtable}{|>{\centering\scriptsize}m{2em}|>{\centering\scriptsize}m{1.3em}|>{\centering}m{8.8em}|}
  % \caption{秦王政}\
  \toprule
  \SimHei \normalsize 年数 & \SimHei \scriptsize 公元 & \SimHei 大事件 \tabularnewline
  % \midrule
  \endfirsthead
  \toprule
  \SimHei \normalsize 年数 & \SimHei \scriptsize 公元 & \SimHei 大事件 \tabularnewline
  \midrule
  \endhead
  \midrule
  元年 & 275 & \tabularnewline\hline
  二年 & 276 & \tabularnewline\hline
  三年 & 277 & \tabularnewline\hline
  四年 & 278 & \tabularnewline\hline
  五年 & 279 & \tabularnewline\hline
  六年 & 280 & \tabularnewline
  \bottomrule
\end{longtable}

\subsection{太康}

\begin{longtable}{|>{\centering\scriptsize}m{2em}|>{\centering\scriptsize}m{1.3em}|>{\centering}m{8.8em}|}
  % \caption{秦王政}\
  \toprule
  \SimHei \normalsize 年数 & \SimHei \scriptsize 公元 & \SimHei 大事件 \tabularnewline
  % \midrule
  \endfirsthead
  \toprule
  \SimHei \normalsize 年数 & \SimHei \scriptsize 公元 & \SimHei 大事件 \tabularnewline
  \midrule
  \endhead
  \midrule
  元年 & 280 & \tabularnewline\hline
  二年 & 281 & \tabularnewline\hline
  三年 & 282 & \tabularnewline\hline
  四年 & 283 & \tabularnewline\hline
  五年 & 284 & \tabularnewline\hline
  六年 & 285 & \tabularnewline\hline
  七年 & 286 & \tabularnewline\hline
  八年 & 287 & \tabularnewline\hline
  九年 & 288 & \tabularnewline\hline
  十年 & 289 & \tabularnewline
  \bottomrule
\end{longtable}

\subsection{太熙}

\begin{longtable}{|>{\centering\scriptsize}m{2em}|>{\centering\scriptsize}m{1.3em}|>{\centering}m{8.8em}|}
  % \caption{秦王政}\
  \toprule
  \SimHei \normalsize 年数 & \SimHei \scriptsize 公元 & \SimHei 大事件 \tabularnewline
  % \midrule
  \endfirsthead
  \toprule
  \SimHei \normalsize 年数 & \SimHei \scriptsize 公元 & \SimHei 大事件 \tabularnewline
  \midrule
  \endhead
  \midrule
  元年 & 290 & \tabularnewline
  \bottomrule
\end{longtable}


%%% Local Variables:
%%% mode: latex
%%% TeX-engine: xetex
%%% TeX-master: "../Main"
%%% End:

%% -*- coding: utf-8 -*-
%% Time-stamp: <Chen Wang: 2018-07-10 22:29:00>

\section{惠帝\tiny(290-306)}

\subsection{永熙}

\begin{longtable}{|>{\centering\scriptsize}m{2em}|>{\centering\scriptsize}m{1.3em}|>{\centering}m{8.8em}|}
  % \caption{秦王政}\
  \toprule
  \SimHei \normalsize 年数 & \SimHei \scriptsize 公元 & \SimHei 大事件 \tabularnewline
  % \midrule
  \endfirsthead
  \toprule
  \SimHei \normalsize 年数 & \SimHei \scriptsize 公元 & \SimHei 大事件 \tabularnewline
  \midrule
  \endhead
  \midrule
  元年 & 290 & \tabularnewline
  \bottomrule
\end{longtable}

\subsection{永平}

\begin{longtable}{|>{\centering\scriptsize}m{2em}|>{\centering\scriptsize}m{1.3em}|>{\centering}m{8.8em}|}
  % \caption{秦王政}\
  \toprule
  \SimHei \normalsize 年数 & \SimHei \scriptsize 公元 & \SimHei 大事件 \tabularnewline
  % \midrule
  \endfirsthead
  \toprule
  \SimHei \normalsize 年数 & \SimHei \scriptsize 公元 & \SimHei 大事件 \tabularnewline
  \midrule
  \endhead
  \midrule
  元年 & 291 & \tabularnewline
  \bottomrule
\end{longtable}

\subsection{元康}

\begin{longtable}{|>{\centering\scriptsize}m{2em}|>{\centering\scriptsize}m{1.3em}|>{\centering}m{8.8em}|}
  % \caption{秦王政}\
  \toprule
  \SimHei \normalsize 年数 & \SimHei \scriptsize 公元 & \SimHei 大事件 \tabularnewline
  % \midrule
  \endfirsthead
  \toprule
  \SimHei \normalsize 年数 & \SimHei \scriptsize 公元 & \SimHei 大事件 \tabularnewline
  \midrule
  \endhead
  \midrule
  元年 & 291 & \tabularnewline\hline
  二年 & 292 & \tabularnewline\hline
  三年 & 293 & \tabularnewline\hline
  四年 & 294 & \tabularnewline\hline
  五年 & 295 & \tabularnewline\hline
  六年 & 296 & \tabularnewline\hline
  七年 & 297 & \tabularnewline\hline
  八年 & 298 & \tabularnewline\hline
  九年 & 299 & \tabularnewline
  \bottomrule
\end{longtable}

\subsection{永康}

\begin{longtable}{|>{\centering\scriptsize}m{2em}|>{\centering\scriptsize}m{1.3em}|>{\centering}m{8.8em}|}
  % \caption{秦王政}\
  \toprule
  \SimHei \normalsize 年数 & \SimHei \scriptsize 公元 & \SimHei 大事件 \tabularnewline
  % \midrule
  \endfirsthead
  \toprule
  \SimHei \normalsize 年数 & \SimHei \scriptsize 公元 & \SimHei 大事件 \tabularnewline
  \midrule
  \endhead
  \midrule
  元年 & 300 & \tabularnewline\hline
  二年 & 301 & \tabularnewline
  \bottomrule
\end{longtable}

\subsection{永宁}

\begin{longtable}{|>{\centering\scriptsize}m{2em}|>{\centering\scriptsize}m{1.3em}|>{\centering}m{8.8em}|}
  % \caption{秦王政}\
  \toprule
  \SimHei \normalsize 年数 & \SimHei \scriptsize 公元 & \SimHei 大事件 \tabularnewline
  % \midrule
  \endfirsthead
  \toprule
  \SimHei \normalsize 年数 & \SimHei \scriptsize 公元 & \SimHei 大事件 \tabularnewline
  \midrule
  \endhead
  \midrule
  元年 & 301 & \tabularnewline\hline
  二年 & 302 & \tabularnewline
  \bottomrule
\end{longtable}

\subsection{太安}

\begin{longtable}{|>{\centering\scriptsize}m{2em}|>{\centering\scriptsize}m{1.3em}|>{\centering}m{8.8em}|}
  % \caption{秦王政}\
  \toprule
  \SimHei \normalsize 年数 & \SimHei \scriptsize 公元 & \SimHei 大事件 \tabularnewline
  % \midrule
  \endfirsthead
  \toprule
  \SimHei \normalsize 年数 & \SimHei \scriptsize 公元 & \SimHei 大事件 \tabularnewline
  \midrule
  \endhead
  \midrule
  元年 & 302 & \tabularnewline\hline
  二年 & 303 & \tabularnewline
  \bottomrule
\end{longtable}

\subsection{永安}

\begin{longtable}{|>{\centering\scriptsize}m{2em}|>{\centering\scriptsize}m{1.3em}|>{\centering}m{8.8em}|}
  % \caption{秦王政}\
  \toprule
  \SimHei \normalsize 年数 & \SimHei \scriptsize 公元 & \SimHei 大事件 \tabularnewline
  % \midrule
  \endfirsthead
  \toprule
  \SimHei \normalsize 年数 & \SimHei \scriptsize 公元 & \SimHei 大事件 \tabularnewline
  \midrule
  \endhead
  \midrule
  元年 & 304 & \tabularnewline
  \bottomrule
\end{longtable}

\subsection{建武}

\begin{longtable}{|>{\centering\scriptsize}m{2em}|>{\centering\scriptsize}m{1.3em}|>{\centering}m{8.8em}|}
  % \caption{秦王政}\
  \toprule
  \SimHei \normalsize 年数 & \SimHei \scriptsize 公元 & \SimHei 大事件 \tabularnewline
  % \midrule
  \endfirsthead
  \toprule
  \SimHei \normalsize 年数 & \SimHei \scriptsize 公元 & \SimHei 大事件 \tabularnewline
  \midrule
  \endhead
  \midrule
  元年 & 304 & \tabularnewline
  \bottomrule
\end{longtable}

\subsection{永兴}

\begin{longtable}{|>{\centering\scriptsize}m{2em}|>{\centering\scriptsize}m{1.3em}|>{\centering}m{8.8em}|}
  % \caption{秦王政}\
  \toprule
  \SimHei \normalsize 年数 & \SimHei \scriptsize 公元 & \SimHei 大事件 \tabularnewline
  % \midrule
  \endfirsthead
  \toprule
  \SimHei \normalsize 年数 & \SimHei \scriptsize 公元 & \SimHei 大事件 \tabularnewline
  \midrule
  \endhead
  \midrule
  元年 & 304 & \tabularnewline\hline
  二年 & 305 & \tabularnewline\hline
  三年 & 306 & \tabularnewline
  \bottomrule
\end{longtable}

\subsection{光熙}

\begin{longtable}{|>{\centering\scriptsize}m{2em}|>{\centering\scriptsize}m{1.3em}|>{\centering}m{8.8em}|}
  % \caption{秦王政}\
  \toprule
  \SimHei \normalsize 年数 & \SimHei \scriptsize 公元 & \SimHei 大事件 \tabularnewline
  % \midrule
  \endfirsthead
  \toprule
  \SimHei \normalsize 年数 & \SimHei \scriptsize 公元 & \SimHei 大事件 \tabularnewline
  \midrule
  \endhead
  \midrule
  元年 & 306 & \tabularnewline
  \bottomrule
\end{longtable}


%%% Local Variables:
%%% mode: latex
%%% TeX-engine: xetex
%%% TeX-master: "../Main"
%%% End:

%% -*- coding: utf-8 -*-
%% Time-stamp: <Chen Wang: 2018-07-10 22:30:05>

\section{怀帝\tiny(306-313)}

\subsection{永嘉}

\begin{longtable}{|>{\centering\scriptsize}m{2em}|>{\centering\scriptsize}m{1.3em}|>{\centering}m{8.8em}|}
  % \caption{秦王政}\
  \toprule
  \SimHei \normalsize 年数 & \SimHei \scriptsize 公元 & \SimHei 大事件 \tabularnewline
  % \midrule
  \endfirsthead
  \toprule
  \SimHei \normalsize 年数 & \SimHei \scriptsize 公元 & \SimHei 大事件 \tabularnewline
  \midrule
  \endhead
  \midrule
  元年 & 307 & \tabularnewline\hline
  二年 & 308 & \tabularnewline\hline
  三年 & 309 & \tabularnewline\hline
  四年 & 310 & \tabularnewline\hline
  五年 & 311 & \tabularnewline\hline
  六年 & 312 & \tabularnewline\hline
  七年 & 313 & \tabularnewline
  \bottomrule
\end{longtable}


%%% Local Variables:
%%% mode: latex
%%% TeX-engine: xetex
%%% TeX-master: "../Main"
%%% End:

%% -*- coding: utf-8 -*-
%% Time-stamp: <Chen Wang: 2018-07-10 22:31:10>

\section{愍帝\tiny(313-316)}

\subsection{建兴}

\begin{longtable}{|>{\centering\scriptsize}m{2em}|>{\centering\scriptsize}m{1.3em}|>{\centering}m{8.8em}|}
  % \caption{秦王政}\
  \toprule
  \SimHei \normalsize 年数 & \SimHei \scriptsize 公元 & \SimHei 大事件 \tabularnewline
  % \midrule
  \endfirsthead
  \toprule
  \SimHei \normalsize 年数 & \SimHei \scriptsize 公元 & \SimHei 大事件 \tabularnewline
  \midrule
  \endhead
  \midrule
  元年 & 313 & \tabularnewline\hline
  二年 & 314 & \tabularnewline\hline
  三年 & 315 & \tabularnewline\hline
  四年 & 316 & \tabularnewline\hline
  五年 & 317 & \tabularnewline
  \bottomrule
\end{longtable}


%%% Local Variables:
%%% mode: latex
%%% TeX-engine: xetex
%%% TeX-master: "../Main"
%%% End:


%%% Local Variables:
%%% mode: latex
%%% TeX-engine: xetex
%%% TeX-master: "../Main"
%%% End:
 % 西晋
% %% -*- coding: utf-8 -*-
%% Time-stamp: <Chen Wang: 2019-12-18 13:34:52>

\chapter{东晋\tiny(317-420)}

\section{简介}

东晋(317年4月6日-420年7月10日),中國朝代,乃西晉司馬氏政權的延續。因内迁的北方游牧民族造反,晉懷帝與晉愍帝先後被俘殺,琅琊王司馬睿在群臣擁戴下在建康(今南京)即位,即晉元帝,史稱東晉。東晉與先前三国时期的东吴以及其後的宋、齊、梁、陳,合稱為六朝。此外,東晉又仿蜀汉称東漢為中汉,称西晋为中朝;又东晋统治地区大部分在江东,古称江左,因此以江左代指东晋。因江東被晉人視為東方,因而司馬睿之後的晉朝被劉宋稱作東晉。同时北方有多个游牧民族建立政权並连年征战,史称五胡十六国时期。

東晉雖然是司馬氏政權的延續,但司馬氏在政治上威望不高,朝廷由世族大家把持,最先的一個乃出身琅琊王氏的王導,其後又有陳郡謝氏的謝安、謝玄等等。而世家大族中的代表者有南下的王、謝、袁、蕭等僑姓,和本身居於江南的朱、張、顧、陸的吳姓。最初東晉有賴權臣王導主持大局,一方面拉攏江南士族,一方面又安排予從中原南下的士族,並以司馬氏作為共同擁戴的對象,司馬氏實際上成為傀儡。世家大族本身並不真正忠於司馬氏,尤其是他們本身都擁有大量田地,以至擁有自家部隊(即所謂「部曲」),有足夠實力抗衡司馬氏政權。最初有王導主持大局,東晉政權得以穩定,故時人稱「王與馬,共天下」。但晉元帝以降則內亂頻生,如有早期王敦之亂、蘇峻之亂,後期又有孫盧之亂等。

东晋也曾多次试图北伐,但由于内部不团结,除了最后篡晋的刘裕取得一定成果外,其余都无建树。祖逖本有希望恢复旧土,但他被晉元帝及世家大族挾制,郁郁而终。桓温的北伐则被慕容垂击败。

376年,前秦苻坚滅掉代國,統一了北方,南北分立之势从此而成。其後苻坚開始率兵南侵。383年,苻坚率約八十七萬兵馬大幅南侵,东晋宰相谢安力主抗击,派谢石谢玄率军,在淝水之战大获全胜,苻坚逃回北方。之後苻坚力量衰弱,因此原本統一的北方再次分裂為多國。后有桓玄叛乱,废安帝,自立为天子,後为大将刘裕所平,拥恭帝,然大权落于刘裕。

420年,刘裕篡位建立劉宋,開啟南北朝時代,東晉亡。

317年,皇族司馬睿在建康城(今江苏省南京市)稱晋王(318年称帝),是為晉元帝,史稱東晉。東晉本身並沒有強大的實力,主要是憑著長江天險,偏安江南;及依靠丞相王導號召南遷避難的中原士族,並聯合南方大族,取得他們的支持。不過,南北大族之間時常發生衝突,內亂頻生,導致東晉政權並不穩定。

自西晉末年劉淵建汉赵以來,南匈奴、羯、白奴、丁零、铁弗、卢水胡、拓跋鲜卑、宇文鲜卑、段氏鲜卑、慕容鲜卑、秃发鲜卑、乞伏鲜卑、九大石胡、大月氐、小月氐和巴氐、姜、夫余、乌桓、高句丽,在中國北方的黃河流域一帶先後建立六十二个割據政權,連同漢族所建立的政權,較重要的有十六個國家,歷史上稱為「五胡十六國」。

從北方南遷的人民時常懷念家鄉,因此一些有志之士多次進行北伐,希望能夠收復北方的國土。祖逖是東晉率先北伐的將領,他曾經率軍收復黃河以南地區,但由於東晉內部出現糾紛,朝廷又擔心他北伐成功後威望太高,結果沒有給予支持,以致功敗垂成,祖逖於321年憂憤而死,曾收復的土地又被胡人重新佔領。

繼祖逖之後,又有桓溫於354年、356年及369年三次北伐,曾一度收復洛陽,他屢次請求朝廷把都城遷回洛陽,但遭到大族的反對,東晉君臣又怕他權勢太大,難以控制,因而無法實現。其後劉裕北伐亦曾收復洛陽和長安。

氐族所建立的前秦,在苻堅時,任用漢人王猛為相,大修政教,富國強兵。前秦強大起來,統一了五胡所據之華北大部分地區。383年,苻堅率軍南下,聲勢浩大,企圖一舉消滅東晉,於是發生了歷史上著名的「淝水之戰」。淝水之戰後,前秦瓦解,北方大亂,再次陷入長期分裂的狀態,胡人無暇南侵。东晋以弱勝強,局勢暫時穩定下來。

東晉的宗室和士族,經常爭權奪利,人民生活相當困苦,以致盜賊四起。淝水之戰後,南方人民暫獲安定,但政治混亂和貪污腐敗的情況,並沒有改善。東晉大臣桓溫死後,其子桓玄逼晋安帝禅位给他,改国号为楚,史称“桓楚”;劉裕起兵聲討,殺死桓玄,恢復東晉的統治。但劉裕自己有奪位的野心,終於在420年,廢晉恭帝自立,改國號為宋,史称“刘宋”。東晉至此正式滅亡。

東晉偏安江南,士族掌權,國君權力旁落,同時各士族之間常為了爭權而北伐,並無單一世族能將司馬氏取而代之,這是政治上特點。

但在另一方面,東晉在文學上卻有一定成就,各類詩文歌賦都大盛於西晉。著名的文學家,有谢灵运、陶渊明、王羲之等人,也流行了駢文。而繪畫、書法也有頗傑出的成就,如東晉人顧愷之的畫作,王羲之的書法,都有很高藝術價值。

著名的中国四大民间传说之一的梁山伯與祝英台的故事背景也發生在東晉時代。

東晉雖非中國史上強盛的時期,卻為文學、藝術極興盛的時代。首都建康成為文化中心,吸引許多東南亞、印度的佛教僧侶及商人前來。338年所鑄造模仿罽賓的佛教模型,為今日所知最早的鎏金銅佛像。中國史上最具影響力的書法家王羲之活躍於此時期。東晉的陶器形式較西晉時期創新。南京富貴山曾挖掘出此時期的墓穴,根據史料記載,此處為東晉皇室墓葬的地點。

東晉也是中國清談盛行的時代。

%% -*- coding: utf-8 -*-
%% Time-stamp: <Chen Wang: 2019-12-18 13:27:04>

\section{元帝\tiny(318-322)}

\subsection{生平}

晉元帝司馬睿(276年5月27日-323年1月3日),字景文,東晉時期第一位皇帝。司馬懿的曾孫、琅邪武王司馬伷之孫、琅邪恭王司馬覲之子,母為琅邪王妃夏侯光姬。《魏書》說司馬睿是牛金和夏侯光姬的私生子。

司馬睿於290年袭封琅邪王,曾經參與討伐成都王司馬穎的戰役;但是由於作戰失利,司馬睿便離開洛陽,回到封國;晉懷帝即位後,司馬睿被封為鎮東大將軍、都督揚州諸軍事,後來在王導的建議之下前往建康,並且極力結交江東大族。311年晉懷帝被俘遇害後,晉愍帝即位,晉愍帝封司馬睿為丞相、大都督中外諸軍事。晉愍帝被俘後,司馬睿在晉朝貴族與江東大族的支持下於317年三月辛卯(公历4月6日)称晋王,318年三月丙辰(公历4月26日)即帝位,為晉元帝。

即位之初曾嘗試北伐,其中祖逖本有希望恢复旧土,但他被晉元帝及世家大族挾制,郁郁而终。

晉元帝實際上為一個被扶持者,本身並無實際權力,大權掌握在王導與王敦之手。晉元帝聽從刁協與劉隗的言論並有意削弱琅邪王氏權力,導致王敦於322年反叛,攻入建康,並且殺害重臣戴淵、周顗等人。但是王敦無力消滅東晉,最後採取與晉元帝和睦的策略。晉元帝便在王敦之亂中因憂鬱過度而過世。

唐代房玄齡於《晉書》的「史臣曰」評論說:「晉氏不虞,自中流外,五胡扛鼎,七廟隳尊,滔天方駕,則民懷其舊德者矣。昔光武以數郡加名,元皇(案:晉元帝)以一州臨極,豈武、宣餘化,猶暢於琅邪,文、景垂仁,傳芳於南頓?所謂後乎天時,先諸人事者也。馳章獻號,高蓋成陰,星斗呈祥,金陵表慶。陶士行擁三州之旅,郢外以安;王茂弘爲分陝之計,江東可立。或高旌未拂,而遐心斯偃,回首朝陽,仰希乾棟,帝猶六讓不居,七辭而不免也。布帳綀帷,詳刑簡化,抑揚前軌,光啓中興。古首私家不蓄甲兵,大臣不爲威福,王之常制,以訓股肱。中宗失馭強臣,自亡齊斧,兩京胡羯,風埃相望。雖復《六月》之駕無聞,而《鴻雁》之歌方遠,享國無幾,哀哉!」

唐代某貴族「公子」與虞世南的對話:「公子曰:『中宗值天下崩離,創立江左,俱為中興之主,比於前代,功德云何?』先生曰:『元帝自居藩邸,少有令聞,及建策南渡,興亡繼絕,委任宏茂,撫綏新舊,故能嗣晉配天,良有以也。然仁恕為懷,剛毅情少,是以王敦縱暴,幾危社稷,蹙國舒禍,其周平之匹乎?』」

\subsection{建武}

\begin{longtable}{|>{\centering\scriptsize}m{2em}|>{\centering\scriptsize}m{1.3em}|>{\centering}m{8.8em}|}
  % \caption{秦王政}\
  \toprule
  \SimHei \normalsize 年数 & \SimHei \scriptsize 公元 & \SimHei 大事件 \tabularnewline
  % \midrule
  \endfirsthead
  \toprule
  \SimHei \normalsize 年数 & \SimHei \scriptsize 公元 & \SimHei 大事件 \tabularnewline
  \midrule
  \endhead
  \midrule
  元年 & 317 & \tabularnewline\hline
  二年 & 318 & \tabularnewline
  \bottomrule
\end{longtable}

\subsection{大兴}

\begin{longtable}{|>{\centering\scriptsize}m{2em}|>{\centering\scriptsize}m{1.3em}|>{\centering}m{8.8em}|}
  % \caption{秦王政}\
  \toprule
  \SimHei \normalsize 年数 & \SimHei \scriptsize 公元 & \SimHei 大事件 \tabularnewline
  % \midrule
  \endfirsthead
  \toprule
  \SimHei \normalsize 年数 & \SimHei \scriptsize 公元 & \SimHei 大事件 \tabularnewline
  \midrule
  \endhead
  \midrule
  元年 & 318 & \tabularnewline\hline
  二年 & 319 & \tabularnewline\hline
  三年 & 320 & \tabularnewline\hline
  四年 & 321 & \tabularnewline
  \bottomrule
\end{longtable}

\subsection{永昌}

\begin{longtable}{|>{\centering\scriptsize}m{2em}|>{\centering\scriptsize}m{1.3em}|>{\centering}m{8.8em}|}
  % \caption{秦王政}\
  \toprule
  \SimHei \normalsize 年数 & \SimHei \scriptsize 公元 & \SimHei 大事件 \tabularnewline
  % \midrule
  \endfirsthead
  \toprule
  \SimHei \normalsize 年数 & \SimHei \scriptsize 公元 & \SimHei 大事件 \tabularnewline
  \midrule
  \endhead
  \midrule
  元年 & 322 & \tabularnewline\hline
  二年 & 323 & \tabularnewline
  \bottomrule
\end{longtable}


%%% Local Variables:
%%% mode: latex
%%% TeX-engine: xetex
%%% TeX-master: "../Main"
%%% End:

%% -*- coding: utf-8 -*-
%% Time-stamp: <Chen Wang: 2019-12-18 13:28:48>

\section{明帝\tiny(322-325)}

\subsection{生平}

晉明帝司馬紹(299年-325年),字道畿,東晉的第二代皇帝,晉元帝司馬睿長子。母親是豫章郡君荀氏。在位不足三年,但在位期間平定了王敦之亂。

司馬紹自小聰慧,故此特別受父親司馬睿所寵愛。後於永嘉元年(307年)隨父親一同移鎮建業(後改建康,今江蘇南京市)。建興元年(313年),司馬睿升任左丞相,拜司馬紹為東中郎將,鎮守廣陵。316年,晉愍帝所在的長安被前趙攻陷,晉愍帝出降,西晋灭亡。有鉴于此,317年,司馬睿稱晉王,建元建武,並立司馬紹為晉王太子。318年,司馬睿即位称帝,改元太興,司馬紹被立為皇太子。

永昌元年(322年)發生王敦之亂,大將軍王敦領兵進攻建康並佔領石頭城,晉元帝派王導等人進攻石頭城但都被王敦擊敗,司馬紹於是打算率領將士與王敦決一死戰,即將出發時因遭太子中庶子溫嶠極力勸阻而沒有實行。隨後王敦自任丞相並掌握朝政,見司馬紹勇而有謀,而且朝野中亦有很高名望,於是打算誣陷他不孝而將他廢掉,但因溫嶠等大臣支持司馬紹,王敦終也不能廢掉司馬紹。

晉元帝因王敦之亂而憂憤成疾,於當年閏十一月己丑日(323年1月3日)病逝,司馬紹在次日繼位,为晉明帝,並由司空王導輔政。

王敦雖於永昌元年(322年)就回到武昌遙控朝廷,但因為圖謀篡位,於太寧元年(323年)暗示要朝廷徵召自己入朝,晉明帝於是以手詔徵召王敦。同年,王允之乘酒宴而知道王敦的圖謀,於是回京告訴其父王舒,王舒於是與王導一同報告晉明帝,得以早作防備。

次年,晉明帝既心知王敦意圖,於是騎馬微服去視察王敦於于湖的營地,但遭到軍人發現,並派五名騎兵追捕。晉明帝逃走時,用水浸濕所騎馬匹的粪便来使其降温,又拿出七寶鞭交給路旁賣食物的婆婆,並要她出示給追來的騎兵。晉明帝走後不久,追兵就來到,並詢問婆婆,婆婆於是取出七寶鞭,並稱那人已經走得很遠。騎兵們顧著傳玩七寶鞭而在那裏停留了很久,而且見馬糞已冷,以為追不及了,於是都沒有再追,晉明帝因此成功逃脫。

及後,晉明帝積極準備京師建康的防護,最終於當年成功擊敗王敦派來進攻的軍隊,平定了王敦之亂。王敦之亂後,晉明帝下令不再問罪於王敦一眾官屬,又分別以應詹為江州刺史、劉遐為徐州刺史、陶侃為荊州刺史、王舒為湘州刺史,重整各州形勢,消除王敦以王氏宗族各領諸州以凌弱帝室的失衡情形。

太寧三年閏八月戊子(325年10月18日),司馬紹病逝於東堂,年僅二十七歲。葬於武平陵,廟號肅祖。

司馬紹年少聰明,小時候便曾經與父親就「太陽與長安孰近」的問題作出不同答案的爭辯。長大後聰明有機斷,精於事理,於是能讓國家從王敦之亂的亂局回復平定。

司馬紹性至孝,有文武才略,敬重賢人,素好文辭,於是當時如王導、庾亮、溫嶠、桓彝、阮放等名臣都親待他。而因他習武藝,善於安撫將士,於是任太子時東宮聚集很多人,亦得遠近各人歸心。

王敦曾称呼晋明帝为:「黄鬚鲜卑奴」,這是因為其母建安郡君荀氏是燕代人,混雜了當地鮮卑人血統,故明帝可能也長得有一些像外族,鬚為黃色。

司馬紹任太子時,想修建池苑樓臺,但元帝不許。司馬紹於是命手下的武士在一晚之間修好太子西池。

司馬紹有寵妃宋褘,褘國色天香,善吹笛,乃石崇妾綠珠之女弟子,不久司馬紹病篤,群臣进谏,请出宋袆,最後宋褘被送給吏部尚书阮孚。

司馬紹在位時,曾問晉室得天下的事。王導於是告訴他司馬懿當日發動高平陵之變誅除曹爽,樹立蔣濟等與自己同心的大臣;又說道曹髦被司馬昭親信賈充所命的成濟弒殺一事。司馬紹聽後,將面龐伏在牀上,說:「若真的像你所說,晉室國祚又怎能夠長遠!」

唐代房玄齡於《晉書》的「史臣曰」評論說:「維揚作宇,憑帶洪流,楚江恆戰,方城對敵,不得不推誠將相,以總戎麾。樓船萬計,兵倍王室,處其利而無心者,周公其人也。威權外假,嫌隙內興,彼有順流之師,此無強籓之援。商逢九亂,堯止八音,明皇(案:晉明帝)負圖,屬在茲日。運龍韜於掌握,起天旆於江靡,燎其餘燼,有若秋原。去縗絰而踐戎場,斬鯨鯢而拜園闕。鎮削威權,州分江漢,覆車不踐,貽厥孫謀。其後七十餘年,終罹敬道之害。或曰:『興亡在運,非止上流。』豈創制不殊,而弘之者異也。」

唐代某貴族「公子」與虞世南的對話:「公子曰:『東晉自元帝已下,何為賢主?』先生曰:『晉自遷都江左,強臣擅命,(天子)垂拱南面,政非己出。王敦以磐石之宗,居上流之地,負才矜地,志懷沖問鼎,非明帝之雄斷,王導之忠誠,則晉祚其移於他族矣。若使降年永久,佐任群賢,因洛、澗之遺黎,乘劉、石之衰運,興復中原,不難圖也。』」


\subsection{太宁}

\begin{longtable}{|>{\centering\scriptsize}m{2em}|>{\centering\scriptsize}m{1.3em}|>{\centering}m{8.8em}|}
  % \caption{秦王政}\
  \toprule
  \SimHei \normalsize 年数 & \SimHei \scriptsize 公元 & \SimHei 大事件 \tabularnewline
  % \midrule
  \endfirsthead
  \toprule
  \SimHei \normalsize 年数 & \SimHei \scriptsize 公元 & \SimHei 大事件 \tabularnewline
  \midrule
  \endhead
  \midrule
  元年 & 323 & \tabularnewline\hline
  二年 & 324 & \tabularnewline\hline
  三年 & 325 & \tabularnewline\hline
  四年 & 326 & \tabularnewline
  \bottomrule
\end{longtable}


%%% Local Variables:
%%% mode: latex
%%% TeX-engine: xetex
%%% TeX-master: "../Main"
%%% End:

%% -*- coding: utf-8 -*-
%% Time-stamp: <Chen Wang: 2021-11-01 11:46:28>

\section{成帝司馬衍\tiny(325-342)}

\subsection{生平}

晉成帝司馬衍(321年12月或322年1月-342年7月26日),字世根,東晉的第三代皇帝,晉明帝之長子。晉成帝年幼即位,即位不久即遇上蘇峻之亂,成帝亦一度遭蘇峻叛軍劫持。成帝一朝軍政主要由外戚穎川庾氏把持,在庾亮的主導下還曾謀北伐,但因後趙強盛而遭到失敗。

太寧三年三月戊辰(325年4月1日),晉明帝立司馬衍為皇太子。同年閏八月戊子(325年10月18日),晉明帝去世,翌日五歲的晉成帝即位為帝。由於年幼,由母親皇太后庾文君臨朝稱制,由七位顧命大臣輔政,中書令庾亮以国舅身份主政。。

咸和二年(327年)年末,歷陽內史蘇峻與豫州刺史祖約叛亂,並在翌年率兵攻至建康,庾亮試圖抵抗但失敗,被逼出逃,晉成帝就與王導等眾官為蘇峻所挾持,宮中就遭到蘇峻軍搶掠和焚燒,太官也僅餘下數石米供成帝食用。咸和三年五月乙未,蘇峻強逼晉成帝遷居至石頭城一個倉庫中,成帝哭著登車出發,宮中人們亦都傷心痛哭。咸和四年(329年),以陶侃為首的軍隊平定蘇峻之亂,迎回成帝,因為宮殿遭戰火破壞,故修繕建平園作為宮室,至咸和七年(332年)新建的建康宮落城後才遷去新宮。

蘇峻之亂後,朝內就由王導專制,成帝對王導亦相當敬重,甚至屢幸王導宅第;庾亮則領豫州刺史出鎮蕪湖,主掌軍事,隨著陶侃去世,庾亮更兼荊江豫三州,轉鎮武昌,並著眼對後趙的北伐。咸康五年(339年),庾亮作出北伐部署,上奏移鎮襄陽石城,並且增兵長江、漢水流域以及淮泗壽陽地區要地,為一舉北伐作好準備,當時庾亮更派兵進攻巴郡,攻至江陽,俘獲後趙將領李閎及黃植。晉成帝下給群臣議論,上疏得王導支持,但郗鑒以資源不足為由反對。不過未等到允許,庾亮的行動就遭後趙以軍事行動作回應,派軍大舉南侵,庾亮所定的重鎮邾城更加被攻陷,庾亮北伐遂流產。

咸康二年(336年)晉成帝頒布壬辰詔書,禁止士族、官吏將私佔山川大澤;咸康七年(341年),又以土斷方式將自江北遷來的世族編入戶籍。

咸康八年(342年)7月23日,晉成帝患病,中書監庾冰為了留住穎川庾氏家族與皇帝的血緣親近,於是以國家外有強敵,宜立年長君主為由勸服成帝以弟弟琅琊王司馬岳為儲君。7月26日,晉成帝駕崩,年僅22歲,廟號顯宗。8月18日,葬於興平陵。

晉成帝年紀小小就很聰敏,有成年人的量度。蘇峻之亂前,庾亮以謀反罪誅殺了南頓王司馬宗,成帝一直不知,至亂事平定後才問及失蹤的司馬宗,庾亮答稱他因謀反而被誅,豈料成帝卻哭著說:「舅言人作賊,便殺之。人言舅作賊,復若何?」嚇得庾亮恐懼失色。至後來,庾懌送毒酒意圖毒殺江州刺史王允之,被揭發後成帝就怒道:「大舅已亂天下,小舅復欲爾邪?」庾懌被逼自殺。不過成帝年輕時被舅舅家族颖川庾氏勢力所限制,並不親政。至後來長大,卻留心事務,而且生活儉約,曾因射堂需耗用四十金而放棄建造。

在石頭城時右衞將軍劉超仍為成帝講授《孝經》及《論語》,但因劉超與鍾雅帶成帝逃出去的圖謀泄漏,二人遭蘇峻派任讓收捕殺害,期間晉成帝抱住任讓哭求:「還我侍中、右衞!」但任讓不聽小皇帝的命令,將二人殺了。蘇峻之亂被平定後,任讓原本因與陶侃有舊情而得免死,但成帝記恨他,任讓還是被誅殺。

\subsection{咸和}

\begin{longtable}{|>{\centering\scriptsize}m{2em}|>{\centering\scriptsize}m{1.3em}|>{\centering}m{8.8em}|}
  % \caption{秦王政}\
  \toprule
  \SimHei \normalsize 年数 & \SimHei \scriptsize 公元 & \SimHei 大事件 \tabularnewline
  % \midrule
  \endfirsthead
  \toprule
  \SimHei \normalsize 年数 & \SimHei \scriptsize 公元 & \SimHei 大事件 \tabularnewline
  \midrule
  \endhead
  \midrule
  元年 & 326 & \tabularnewline\hline
  二年 & 327 & \tabularnewline\hline
  三年 & 328 & \tabularnewline\hline
  四年 & 329 & \tabularnewline\hline
  五年 & 330 & \tabularnewline\hline
  六年 & 331 & \tabularnewline\hline
  七年 & 332 & \tabularnewline\hline
  八年 & 333 & \tabularnewline\hline
  九年 & 334 & \tabularnewline
  \bottomrule
\end{longtable}

\subsection{咸康}

\begin{longtable}{|>{\centering\scriptsize}m{2em}|>{\centering\scriptsize}m{1.3em}|>{\centering}m{8.8em}|}
  % \caption{秦王政}\
  \toprule
  \SimHei \normalsize 年数 & \SimHei \scriptsize 公元 & \SimHei 大事件 \tabularnewline
  % \midrule
  \endfirsthead
  \toprule
  \SimHei \normalsize 年数 & \SimHei \scriptsize 公元 & \SimHei 大事件 \tabularnewline
  \midrule
  \endhead
  \midrule
  元年 & 335 & \tabularnewline\hline
  二年 & 336 & \tabularnewline\hline
  三年 & 337 & \tabularnewline\hline
  四年 & 338 & \tabularnewline\hline
  五年 & 339 & \tabularnewline\hline
  六年 & 340 & \tabularnewline\hline
  七年 & 341 & \tabularnewline\hline
  八年 & 342 & \tabularnewline
  \bottomrule
\end{longtable}


%%% Local Variables:
%%% mode: latex
%%% TeX-engine: xetex
%%% TeX-master: "../Main"
%%% End:

%% -*- coding: utf-8 -*-
%% Time-stamp: <Chen Wang: 2019-12-18 13:33:01>

\section{康帝\tiny(342-344)}

\subsection{生平}

晉康帝司馬岳(322年-344年11月17日),字世同,東晉的第四代皇帝。晉康帝是晉明帝之子,母庾文君,是晉成帝的同母弟。

晉康帝於326年被封為吳王,後封琅琊王,342年晉成帝死後,由於權臣庾冰與庾翼力主之故,晉康帝才得以用兄終弟及的方式繼承帝位,但不久便於建元二年344年患病駕崩,時年二十三歲,葬於崇平陵,其子司馬聃繼位。

\subsection{建元}

\begin{longtable}{|>{\centering\scriptsize}m{2em}|>{\centering\scriptsize}m{1.3em}|>{\centering}m{8.8em}|}
  % \caption{秦王政}\
  \toprule
  \SimHei \normalsize 年数 & \SimHei \scriptsize 公元 & \SimHei 大事件 \tabularnewline
  % \midrule
  \endfirsthead
  \toprule
  \SimHei \normalsize 年数 & \SimHei \scriptsize 公元 & \SimHei 大事件 \tabularnewline
  \midrule
  \endhead
  \midrule
  元年 & 343 & \tabularnewline\hline
  二年 & 344 & \tabularnewline
  \bottomrule
\end{longtable}


%%% Local Variables:
%%% mode: latex
%%% TeX-engine: xetex
%%% TeX-master: "../Main"
%%% End:

%% -*- coding: utf-8 -*-
%% Time-stamp: <Chen Wang: 2021-11-01 11:46:39>

\section{穆帝司馬聃\tiny(344-361)}

\subsection{生平}

晉穆帝司馬聃(343年-361年7月10日),字彭子,東晉第五代皇帝,廟號孝宗。晉穆帝是晉康帝之子,母褚蒜子。

建元二年(344年),由於晉康帝僅22歲便駕崩,晉穆帝即位,時年兩歲;由於年幼而由褚太后掌政,並由何充輔政。何充過世後改由蔡謨與司馬昱輔政。晉穆帝在位期間東晉雖然北伐失敗,但是由於桓溫消滅了在四川立國的成漢,並且於永和十二年(356年)三月奪回洛陽,雖然不久就因為糧運不繼而撤退,東晉的版圖仍然有所擴大。昇平元年 ( 357年 ) 晉穆帝行冠禮後褚太后歸政,晉穆帝開始親政。

361年7月10日,晉穆帝過世,得年十八歲,9月9日,葬永平陵。由晉成帝長子琅邪王司馬丕繼位。


\subsection{永和}

\begin{longtable}{|>{\centering\scriptsize}m{2em}|>{\centering\scriptsize}m{1.3em}|>{\centering}m{8.8em}|}
  % \caption{秦王政}\
  \toprule
  \SimHei \normalsize 年数 & \SimHei \scriptsize 公元 & \SimHei 大事件 \tabularnewline
  % \midrule
  \endfirsthead
  \toprule
  \SimHei \normalsize 年数 & \SimHei \scriptsize 公元 & \SimHei 大事件 \tabularnewline
  \midrule
  \endhead
  \midrule
  元年 & 345 & \tabularnewline\hline
  二年 & 346 & \tabularnewline\hline
  三年 & 347 & \tabularnewline\hline
  四年 & 348 & \tabularnewline\hline
  五年 & 349 & \tabularnewline\hline
  六年 & 350 & \tabularnewline\hline
  七年 & 351 & \tabularnewline\hline
  八年 & 352 & \tabularnewline\hline
  九年 & 353 & \tabularnewline\hline
  十年 & 354 & \tabularnewline\hline
  十一年 & 355 & \tabularnewline\hline
  十二年 & 356 & \tabularnewline
  \bottomrule
\end{longtable}

\subsection{升平}

\begin{longtable}{|>{\centering\scriptsize}m{2em}|>{\centering\scriptsize}m{1.3em}|>{\centering}m{8.8em}|}
  % \caption{秦王政}\
  \toprule
  \SimHei \normalsize 年数 & \SimHei \scriptsize 公元 & \SimHei 大事件 \tabularnewline
  % \midrule
  \endfirsthead
  \toprule
  \SimHei \normalsize 年数 & \SimHei \scriptsize 公元 & \SimHei 大事件 \tabularnewline
  \midrule
  \endhead
  \midrule
  元年 & 357 & \tabularnewline\hline
  二年 & 358 & \tabularnewline\hline
  三年 & 359 & \tabularnewline\hline
  四年 & 360 & \tabularnewline\hline
  五年 & 361 & \tabularnewline
  \bottomrule
\end{longtable}


%%% Local Variables:
%%% mode: latex
%%% TeX-engine: xetex
%%% TeX-master: "../Main"
%%% End:

%% -*- coding: utf-8 -*-
%% Time-stamp: <Chen Wang: 2021-11-01 11:46:53>

\section{哀帝司馬丕\tiny(361-365)}

\subsection{生平}

晉哀帝司馬丕(341年-365年3月30日),字千齡,為東晉的第六代皇帝,晉成帝之子,晉穆帝之堂兄。

342年封为琅琊王,345年拜散骑常侍。356年加中军将军,359年十二月除骠骑将军。晉哀帝本應繼晉成帝之位即位,但是由於權臣庾冰的意見而無法即位;司馬丕於361年在晉穆帝死後即位,改元隆和,但是大將桓溫當國,晉哀帝形同傀儡。

晉哀帝好重佛法,又好黄老道,即位不久就迷上了長生術,按照道士傳授的長生法,斷榖、服丹藥,結果服藥後藥性大發而不能聽政,遂由褚太后再次臨朝。興甯三年(365年),晉哀帝因藥物中毒死於太極殿,年僅二十五歲,葬於安平陵。

\subsection{隆和}

\begin{longtable}{|>{\centering\scriptsize}m{2em}|>{\centering\scriptsize}m{1.3em}|>{\centering}m{8.8em}|}
  % \caption{秦王政}\
  \toprule
  \SimHei \normalsize 年数 & \SimHei \scriptsize 公元 & \SimHei 大事件 \tabularnewline
  % \midrule
  \endfirsthead
  \toprule
  \SimHei \normalsize 年数 & \SimHei \scriptsize 公元 & \SimHei 大事件 \tabularnewline
  \midrule
  \endhead
  \midrule
  元年 & 362 & \tabularnewline\hline
  二年 & 363 & \tabularnewline
  \bottomrule
\end{longtable}

\subsection{兴宁}

\begin{longtable}{|>{\centering\scriptsize}m{2em}|>{\centering\scriptsize}m{1.3em}|>{\centering}m{8.8em}|}
  % \caption{秦王政}\
  \toprule
  \SimHei \normalsize 年数 & \SimHei \scriptsize 公元 & \SimHei 大事件 \tabularnewline
  % \midrule
  \endfirsthead
  \toprule
  \SimHei \normalsize 年数 & \SimHei \scriptsize 公元 & \SimHei 大事件 \tabularnewline
  \midrule
  \endhead
  \midrule
  元年 & 363 & \tabularnewline\hline
  二年 & 364 & \tabularnewline\hline
  三年 & 365 & \tabularnewline
  \bottomrule
\end{longtable}


%%% Local Variables:
%%% mode: latex
%%% TeX-engine: xetex
%%% TeX-master: "../Main"
%%% End:

%% -*- coding: utf-8 -*-
%% Time-stamp: <Chen Wang: 2019-12-18 13:35:41>

\section{废帝\tiny(365-371)}

\subsection{生平}

司馬奕(342年-386年),字延齡,東晉的第七代皇帝,晉成帝之子、晉哀帝之弟。晉哀帝死後於365年即帝位,史稱「廢帝」。

342年六月封为东海王,352年拜散骑常侍镇军将军。360年升车骑将军,361年改封琅琊王。362年七月為侍中骠骑大将军开府仪同三司。司馬奕即位之時,桓溫掌握朝政,桓的幕府參軍郗超建議桓溫效仿伊尹、霍光,廢除天子以立威信,但司馬奕本身並無過失可言,桓溫便指司馬奕陽痿不能人道,指田、孟二妃所生三皇子为司马奕的男宠相龙、计好及朱灵宝所生,於太和六年(371年)廢司馬奕為東海王,之後再貶為海西縣公,遷居吳縣西柴里,并将田、孟二妃及三皇子处死。

司馬奕遭廢位后心灰意冷,又怕再遭禍端,便苟且偷生。之後司馬奕更是沉迷於酒色,成日过着荒淫的生活,甚至生了孩子也不养,桓溫及之後继位的晋孝武帝也因此对他不再防范。

司马奕於386年過世,享年四十五歲,他亦是東晉較為長壽的皇帝。

\subsection{太和}

\begin{longtable}{|>{\centering\scriptsize}m{2em}|>{\centering\scriptsize}m{1.3em}|>{\centering}m{8.8em}|}
  % \caption{秦王政}\
  \toprule
  \SimHei \normalsize 年数 & \SimHei \scriptsize 公元 & \SimHei 大事件 \tabularnewline
  % \midrule
  \endfirsthead
  \toprule
  \SimHei \normalsize 年数 & \SimHei \scriptsize 公元 & \SimHei 大事件 \tabularnewline
  \midrule
  \endhead
  \midrule
  元年 & 366 & \tabularnewline\hline
  二年 & 367 & \tabularnewline\hline
  三年 & 368 & \tabularnewline\hline
  四年 & 369 & \tabularnewline\hline
  五年 & 370 & \tabularnewline\hline
  六年 & 371 & \tabularnewline
  \bottomrule
\end{longtable}



%%% Local Variables:
%%% mode: latex
%%% TeX-engine: xetex
%%% TeX-master: "../Main"
%%% End:

%% -*- coding: utf-8 -*-
%% Time-stamp: <Chen Wang: 2019-12-18 13:37:03>

\section{简文帝\tiny(371-372)}

\subsection{生平}

晉簡文帝司馬\xpinyin*{昱}(320年-372年9月12日),字道萬。東晉第八代皇帝。东晋开国皇帝晋元帝少子,母郑阿春。自永和元年(345年)開始一直以會稽王輔政,掌握朝廷的實權,但其時權臣桓溫的勢力亦一直增強。52歲時於太和六年十一月己酉(372年1月6日)被桓溫擁立为帝,改年号为咸安。次年七月己未(372年9月12日)病逝。在位期間只有250日,期間桓温擅权。

永昌元年(322年),晉元帝下詔封司馬昱為琅邪王,作為自己入繼大宗後父親爵位的繼嗣。咸和二年(327年)因其母喪,請求服重而改封會稽王,官拜散騎常侍。咸和九年(334年)轉任右將軍,加侍中。咸康六年(340年)進撫軍將軍,領祕書監。建元元年(343年)加領太常。

永和元年(345年),因著上一年晉康帝去世,年幼的晉穆帝登位,崇德太后褚蒜子抱晉穆帝臨朝。當時輔政的驃騎將軍何充希望由太后父親褚裒入朝輔政但對方辭讓,當時司馬昱聲望高,故升司馬昱為撫軍大將軍,錄尚書六條事,與何充輔政。同年,荊州刺史庾翼去世,死前請求以其子庾爰之接代其位,何充則屬意徐州刺史桓溫取代庾氏掌握荊州。司馬昱倚重的名士劉惔熟悉桓溫,指桓溫雖然有才幹但也極有野心,不能讓他居於荊州這個控制長江上游的「形勝之地」,建議由司馬昱親自外鎮荊州,或由劉惔自己任荊州刺史。司馬昱沒有聽從劉惔的建言,桓溫任荊州刺史,獲得了日後奪權掌政的資本。

永和二年(346年),何充去世,左光祿大夫蔡謨加領司徒兼錄尚書六條事,與司馬昱一同輔政,司馬昱總理萬機,其實是東晉朝廷的決策者。何充兼領的揚州刺史此時出缺,褚裒舉薦了名士殷浩,殷浩辭讓並寫信給司馬昱說明理由,但司馬昱勸他出仕,四個月後殷浩終於出仕。次年,桓溫平滅成漢,建立大功,威望和勢力都大為提升,同時也引來朝內對其的忌憚。司馬昱決定以殷浩抗衡桓溫,後來後趙國內大亂,授殷浩以北伐的重任。然而殷浩北伐失敗,桓溫在永和十年借朝野對殷浩北伐失敗的不滿廢掉殷浩,司馬昱亦無力抗衡桓溫高漲的力量,令得桓溫在朝中獨大。

永和八年(352年),詔升司馬昱為司徒,司馬昱辭讓。興寧三年(365年),琅邪王司馬奕即位為晉廢帝,琅邪國無嗣,晉廢帝封司馬昱爲琅邪王,改以司馬昱子司馬昌明為會稽王。司馬昱辭讓,所以仍以會稽王號封琅邪王。次年,詔進司馬昱為丞相、錄尚書事、入朝不趨,贊拜不名,劍履上殿、賜羽葆,鼓吹及班劍六十人,司馬昱又辭讓。

太和四年(369年),桓溫第三次北伐大敗於前燕和前秦聯軍,豫州刺史袁真不堪被桓溫所誣要負上北伐失敗的責任而叛變。司馬昱在涂中與桓溫會面,商討隨後的行動,以桓溫子桓熙為豫州刺史。

太和六年十一月己酉日(372年1月6日),大司馬桓溫廢黜晉廢帝為東海王,率百官到會稽王府奉迎司馬昱,司馬昱即日即位為帝,是為晉簡文帝,改元咸安。桓溫及後就寫了講辭,打算向司馬昱陳述自己廢立的本意,但司馬昱每接見他都不停流淚,如此令桓溫恐懼,居然不能說一句話。

司馬昱的哥哥武陵王司馬晞有軍事才幹,被桓溫所忌。廢立不久,桓溫就誣陷司馬晞謀反將其免官,及後更逼令新蔡王司馬晃自誣與司馬晞及庾倩等人謀反,以求翦滅陳郡殷氏和穎川庾氏在朝中的勢力。隨後桓溫指示御史中丞司馬恬奏請司馬昱依律法處死司馬晞,司馬昱不肯,下令再作詳細議論。桓溫再次上奏求誅司馬晞,言詞十分嚴厲急切,司馬昱於是手詔給桓溫,寫道:「若果晉室國祚長久,那麼你就應該依從早前的詔命從事;如晉室大勢已去,那你就讓我退位讓賢吧。」桓溫看後流汗色變,不敢再逼,只上奏廢掉司馬晞和他三名兒子,並流放其家屬。

桓溫既行廢立,亦誅滅了與司馬皇室親密的殷氏和庾氏,威勢達至高峯。不過,桓溫在當時仍受制於以王坦之為首的太原王氏及謝安為首的陳郡謝氏世族力量,有篡位心而不能得逞。而司馬昱雖天子,其實如同傀儡皇帝,未敢多言,更怕又被桓溫所廢。當時司馬昱見熒惑入太微垣,因晉廢帝被廢時亦有同樣天象,故此十分不安,甚至對桓溫親信也是自己昔日僚屬的郗超問桓溫會否再行廢立之事。郗超斷言桓溫不會這樣作,司馬昱仍十分感慨,並詠庾闡之詩:「志士痛朝危,忠臣哀主辱。」司馬昱憂憤而得病,在咸安二年七月甲寅日(372年9月7日)因病急召桓溫入朝輔政,桓溫數度辭讓,司馬昱於是在己未日(9月12日)立兒子司馬昌明為太子。同日在東堂去世,享年五十三歲。臨終前,司馬昱寫了遺詔,要桓溫依周公先例居攝,更寫:「少子可輔者輔之,如不可,君自取之。」面對桓溫的野心,此舉幾近讓國。王坦之在司馬昱面前親手撕毀遺詔。司馬昱說:「晉室天下,只是因好運而意外獲得,你又對這個決定有甚麼不滿呢!」王坦之卻說:「晉室天下,是晉宣帝和晉元帝建立的,怎由得陛下你獨斷獨行!」司馬昱於是命王坦之改寫遺詔,寫道:「家國事一稟大司馬,如諸葛武侯、王丞相故事。」桓溫其實亦希望司馬昱臨終禪讓帝位給自己,又或者讓他像周公般居攝行事,王坦之改寫後的遺詔令桓溫大失所望。

司馬昱葬高平陵,廟號為太宗,諡為簡文皇帝。

简文帝外表清秀俊朗,擅长玄學清谈。尽管有文人雅士的风度,恬靜豁達,但政治手腕可说非常平庸,無濟世大略。谢安曾尖刻地评论:“比晉惠帝(以「何不食肉糜」著名的白痴皇帝)惟有清谈差胜耳!”謝靈運亦以他比作周赧王及漢獻帝等亡國之君。

司馬昱崇尚清談,長期坐著的胡床上即使積了灰塵也不清理。一次司馬昱發現有老鼠走過胡床的痕跡,覺得是好事。參軍見到有老鼠在白天走了出來,以手板將老鼠殺掉,司馬昱很不高興。當時門下的部屬就檢舉殺鼠的人,以圖取悅司馬昱,司馬昱卻說:「老鼠被殺,到現在還不能忘記;而現在又因老鼠而影響到他人,豈不是更不應該嗎?」可見其在醉心玄學之餘亦聰明有仁心。

司馬昱輔政時,一些政事拖了整年才得批准,桓溫覺得太慢,常常勸告司馬昱。但司馬昱卻說:「一日萬機,怎能快呀。」

王濛昔日請求當東陽太守,司馬昱不答應。及至王濛病重臨終,司馬昱就悲哀地說:「我將有負於仲祖(王濛字)呀!」下令命其為東陽太守。王濛說:「人們說會稽王痴心,真是痴心呀。」

司馬昱看見稻田,不知是甚麼,於是問左右是甚麼草,左右於是答那是稻。司馬昱事後三日沒有出外,說:「哪有依賴其結果而不知其根本。」

《晉書》載一次桓溫與司馬晞及司馬昱同車遊板桥,桓溫特意命人吹響號角,令馬匹受驚狂奔,藉此看兩人的反應。司馬晞當時大驚而想下車,而司馬昱就處之泰然。《世說新語》亦有類似記載。不過有認為司馬晞既然受桓溫所忌,不應有如此反應,這是對日後成為皇帝的司馬昱的溢美之作。

\subsection{咸安}

\begin{longtable}{|>{\centering\scriptsize}m{2em}|>{\centering\scriptsize}m{1.3em}|>{\centering}m{8.8em}|}
  % \caption{秦王政}\
  \toprule
  \SimHei \normalsize 年数 & \SimHei \scriptsize 公元 & \SimHei 大事件 \tabularnewline
  % \midrule
  \endfirsthead
  \toprule
  \SimHei \normalsize 年数 & \SimHei \scriptsize 公元 & \SimHei 大事件 \tabularnewline
  \midrule
  \endhead
  \midrule
  元年 & 371 & \tabularnewline\hline
  二年 & 372 & \tabularnewline
  \bottomrule
\end{longtable}



%%% Local Variables:
%%% mode: latex
%%% TeX-engine: xetex
%%% TeX-master: "../Main"
%%% End:

%% -*- coding: utf-8 -*-
%% Time-stamp: <Chen Wang: 2021-11-01 11:47:57>

\section{孝武帝司马曜\tiny(372-396)}

\subsection{生平}

晋孝武帝司马曜(362年-396年11月6日),字昌明,东晋的第九个皇帝,在位时间是372年至396年。他是晋简文帝的第三个儿子,晋安帝和晋恭帝的父亲,母李陵容。

晋孝武帝四岁时被封为会稽王,372年9月12日被立为太子,同日晋简文帝逝,继位時年僅十一岁。次年年号为宁康,由太后摄政。

14岁时(376年)开始亲政,改年号为太元。当年他改革税收,放弃以田地多少来收税的方法,改为王公以下每人收米三斛,在役的人不交税。此外他在位期间大力加强皇帝的权力和地位,史載他“威權己出”,扭轉了東晉自晉明帝死後皇权旁落的局面。

383年前秦进攻东晋,试图消灭长年偏安的东晋,结果在淝水之战中,晋军大胜。

384年後,晉孝武帝趁著前秦崩解的契機北伐,陸續收復了黃河以南的所有領土(包含河南洛陽及山東半島),甚至劉牢之一度佔領河北鄴城。這使得390年代的東晉版圖,達到了自東晉開始以來的最大值。但是連年征戰,遽增的兵役賦稅使人民痛苦難當,既疲又怨。

晉孝武帝即位初期由於稅賦改革與謝安當國,被稱為東晉後期的復興;但是謝安死後司馬道子當國,以及晋孝武帝北伐成功后开始嗜酒,“醒日既少”,連帶導致“刑網峻急,風俗奢宕”的不良政風。

396年11月6日,晉孝武帝由於对他当时宠信的张贵人开玩笑说:“你已经快要三十歲了,按年龄应该要被废弃了”,導致当晚张贵人一怒之下在清暑殿杀了他,享年34歲。11月30日,葬于今江苏南京的隆平陵。

孝武帝自幼年聰穎,他十歲時父親簡文帝崩逝,但他到了下午仍不去父親遺體旁哭喪,侍從勸告他應按照禮節哭喪,他卻回答說:「哀痛時就是哭喪的最好時機,哪裡需要被常規禮節束縛呢?」宰相謝安對他的清談義理頗為讚嘆,認為他所掌握的精微義理,不下於其父簡文帝。孝武帝親政後將治國大權收歸己手,很有君主的才幹器量。但他年長後沉溺於酒色之中,將政務細節交給位居宰相的弟弟司馬道子,常與道子一同飲酒酣歌。他晚年更通宵飲酒而睡到大白天,因此少有白日清醒的時刻。周遭缺乏剛正的大臣規勸,因此沒法改正嗜酒缺失。

唐代房玄齡於《晉書》評論說:「太宗晏駕,寧康(按:以年號代稱晉孝武帝)纂業,天誘其衷,姦臣自隕,于時西踰劍岫而跨靈山,北振長河而臨清、洛;荊、吳戰旅,嘯吒成雲;名賢間出,舊德斯在:謝安可以鎮雅俗,彪之足以正紀綱,桓沖之夙夜王家,謝玄之善料軍事。于時上天乃眷,強氐自泯。五尺童子,振袂臨江,思所以挂旆天山,封泥函谷;而條綱弗垂,威恩罕樹,道子荒乎朝政,國寶彙以小人,拜授之榮,初非天旨,鬻刑之貨,自走權門,毒賦年滋,愁民歲廣。是以聞人、許榮馳書詣闕,烈宗知其抗直,而惡聞逆耳,肆一醉於崇朝,飛千觴於長夜。雖復『昌明』表夢,安聽神言?而金行穨弛,抑亦人事,語曰『大國之政未陵夷,小邦之亂已傾覆』也。屬苻堅百六之秋,棄肥水之眾,帝號為 『武』,不亦優哉!」

唐代某貴族「公子」與虞世南的對話:「公子曰:『(東晉)中興之政,咸歸大臣,唯孝武為君,威福自己,外摧疆寇,人安吏肅。比于明帝,功業何如?』先生(虞世南)曰:『孝武克夷外難,乃謝安之力也,非人主之功。至于委任會稽,棟梁已撓,殷、王作鎮,亂階斯起,昌明之讖,乃驗于茲。加以末年沉晏,卒致傾覆,比蹤前哲(按:前哲指晉明帝),其何遠乎?』」

\subsection{宁康}

\begin{longtable}{|>{\centering\scriptsize}m{2em}|>{\centering\scriptsize}m{1.3em}|>{\centering}m{8.8em}|}
  % \caption{秦王政}\
  \toprule
  \SimHei \normalsize 年数 & \SimHei \scriptsize 公元 & \SimHei 大事件 \tabularnewline
  % \midrule
  \endfirsthead
  \toprule
  \SimHei \normalsize 年数 & \SimHei \scriptsize 公元 & \SimHei 大事件 \tabularnewline
  \midrule
  \endhead
  \midrule
  元年 & 373 & \tabularnewline\hline
  二年 & 374 & \tabularnewline\hline
  三年 & 375 & \tabularnewline
  \bottomrule
\end{longtable}

\subsection{太元}

\begin{longtable}{|>{\centering\scriptsize}m{2em}|>{\centering\scriptsize}m{1.3em}|>{\centering}m{8.8em}|}
  % \caption{秦王政}\
  \toprule
  \SimHei \normalsize 年数 & \SimHei \scriptsize 公元 & \SimHei 大事件 \tabularnewline
  % \midrule
  \endfirsthead
  \toprule
  \SimHei \normalsize 年数 & \SimHei \scriptsize 公元 & \SimHei 大事件 \tabularnewline
  \midrule
  \endhead
  \midrule
  元年 & 376 & \tabularnewline\hline
  二年 & 377 & \tabularnewline\hline
  三年 & 378 & \tabularnewline\hline
  四年 & 379 & \tabularnewline\hline
  五年 & 380 & \tabularnewline\hline
  六年 & 381 & \tabularnewline\hline
  七年 & 382 & \tabularnewline\hline
  八年 & 383 & \tabularnewline\hline
  九年 & 384 & \tabularnewline\hline
  十年 & 385 & \tabularnewline\hline
  十一年 & 386 & \tabularnewline\hline
  十二年 & 387 & \tabularnewline\hline
  十三年 & 388 & \tabularnewline\hline
  十四年 & 389 & \tabularnewline\hline
  十五年 & 390 & \tabularnewline\hline
  十六年 & 391 & \tabularnewline\hline
  十七年 & 392 & \tabularnewline\hline
  十八年 & 393 & \tabularnewline\hline
  十九年 & 394 & \tabularnewline\hline
  二十年 & 395 & \tabularnewline\hline
  二一年 & 396 & \tabularnewline
  \bottomrule
\end{longtable}


%%% Local Variables:
%%% mode: latex
%%% TeX-engine: xetex
%%% TeX-master: "../Main"
%%% End:

%% -*- coding: utf-8 -*-
%% Time-stamp: <Chen Wang: 2019-12-18 13:39:00>

\section{安帝\tiny(397-418)}

\subsection{生平}

晋安帝司马德宗(382年-419年1月28日),字德宗,东晋的第十位皇帝。晋孝武帝司马曜的长子,母亲是陈归女。晉安帝由於痴愚而無能力掌握國政,在位廿二年間朝權都旁落在臣下之中,國內內亂頻仍,期間甚至發生了桓玄篡位的事件。最後東晉國祚及國力在北府將領劉裕的主持下獲得恢復,但亦為劉裕奠下篡位的基礎,安帝自己亦因劉裕欲篡而遇害。

司马德宗於太元十二年八月辛巳(387年9月16日)被立为皇太子。太元二十一年九月庚申(396年11月6日)孝武帝被張貴人所弒,次日安帝正式继位。

安帝本愚,從小到大连话都不太会说,就连冬夏的区别都认不出来,因此朝廷的权力实际上完全由当朝大臣掌握,沒有一道詔旨,一個行動是出自安帝自己的意願。安帝初期朝廷政策主要由以太傅攝政旳会稽王司马道子主持,王恭之亂後則由會稽世子司馬元顯掌握。元興元年(402年),司馬元顯讓安帝改元「元興」,預備出兵討伐桓玄,但其年為桓玄所敗,朝政亦從此轉歸桓玄掌握。桓玄先廢元興年號,改元「大亨」,後專制朝廷,準備篡位。翌年十二月壬辰(404年1月1日),桓玄篡位,建立桓楚政權,廢安帝為平固王,並在次日被送至尋陽。

桓玄篡位後僅兩月,北府將領劉裕等人於京口、廣陵成功舉兵,並合軍進攻建康,桓玄軍隊不敵並撤到尋陽,隨即又挾持晉安帝退至江陵。桓玄在江陵試圖重整旗鼓,於五月癸酉(404年6月10日)逼令安帝隨軍之下於崢嶸洲與劉毅等軍決戰。不過,桓玄再敗,只有率敗軍及安帝退還江陵,隨後在往蜀地的路上被殺。留在江陵的安帝在荊州別駕王康產及南郡太守王騰之的支援下於江陵復位。然而,不久桓振率領桓楚殘部進襲江陷,安帝再度被俘。義熙元年(405年)正月,晉軍收復江陵,安帝再度復位,改元「義熙」,隨後獲迎回建康。時朝廷就由劉裕為首,至義熙四年(408年)接替去世的王謐領揚州刺史、錄尚書事時完全掌握朝權。

安帝即位後,先有王恭兩度舉兵,荊州各地亦陷入割據狀態,朝廷影響力量在司馬道子父子主政時一度僅及三吳。內亂不休的晉廷因而無力守衞北方領土,至隆安三年(399年),後秦佔領洛陽及河南地區,南燕則佔領山東半島,建都廣固城。東晉再次喪失了淮水以北的大部分領土。接著發生的孫恩盧循之亂以及桓玄篡位事件,亦讓東晉未能有暇收復失地,蜀地甚至在討伐桓玄期間由譙縱建立的譙蜀控制。劉裕主政之下,東晉於義熙五年(409年)出兵進攻南燕,並翌年滅南燕,收復齊地;雖然盧循乘劉裕北伐的機會襲擊建康,但劉裕適時回軍,成功防禦京師,並於稍後擊潰盧循主力,盧循及其勢力於義熙七年(411年)完全肅清。劉裕又在義熙九年(413年)派將領朱齡石收復蜀地。另一方面,他又於義熙八年(412年)先後消滅了當日與他一同起兵討伐桓玄的劉毅及諸葛長民,又於義熙十一年(415年)消滅了宗室司馬休之,除去了政敵。義熙十二年(416年),劉裕奉琅邪王司馬德文的名義北伐後秦,於翌年成功滅掉後秦,不但收服了河南及洛陽失地,更重奪關中。雖然關中於義熙十四年(418年)劉裕班師後為夏國所佔領,但劉裕地位已經穩固,受封十郡宋公,亦圖篡位,因為「昌明之後有二帝」的預言,故晉安帝在劉裕授意下於十二月戊寅日(419年1月28日)被中書侍郎王韶之殺害,其弟司馬德文被擁立,以應預言。

\subsection{隆安}

\begin{longtable}{|>{\centering\scriptsize}m{2em}|>{\centering\scriptsize}m{1.3em}|>{\centering}m{8.8em}|}
  % \caption{秦王政}\
  \toprule
  \SimHei \normalsize 年数 & \SimHei \scriptsize 公元 & \SimHei 大事件 \tabularnewline
  % \midrule
  \endfirsthead
  \toprule
  \SimHei \normalsize 年数 & \SimHei \scriptsize 公元 & \SimHei 大事件 \tabularnewline
  \midrule
  \endhead
  \midrule
  元年 & 397 & \tabularnewline\hline
  二年 & 398 & \tabularnewline\hline
  三年 & 399 & \tabularnewline\hline
  四年 & 400 & \tabularnewline\hline
  五年 & 401 & \tabularnewline
  \bottomrule
\end{longtable}

\subsection{元兴}

\begin{longtable}{|>{\centering\scriptsize}m{2em}|>{\centering\scriptsize}m{1.3em}|>{\centering}m{8.8em}|}
  % \caption{秦王政}\
  \toprule
  \SimHei \normalsize 年数 & \SimHei \scriptsize 公元 & \SimHei 大事件 \tabularnewline
  % \midrule
  \endfirsthead
  \toprule
  \SimHei \normalsize 年数 & \SimHei \scriptsize 公元 & \SimHei 大事件 \tabularnewline
  \midrule
  \endhead
  \midrule
  元年 & 402 & \tabularnewline\hline
  二年 & 403 & \tabularnewline\hline
  三年 & 404 & \tabularnewline
  \bottomrule
\end{longtable}

\subsection{大亨}

\begin{longtable}{|>{\centering\scriptsize}m{2em}|>{\centering\scriptsize}m{1.3em}|>{\centering}m{8.8em}|}
  % \caption{秦王政}\
  \toprule
  \SimHei \normalsize 年数 & \SimHei \scriptsize 公元 & \SimHei 大事件 \tabularnewline
  % \midrule
  \endfirsthead
  \toprule
  \SimHei \normalsize 年数 & \SimHei \scriptsize 公元 & \SimHei 大事件 \tabularnewline
  \midrule
  \endhead
  \midrule
  元年 & 402 & \tabularnewline
  \bottomrule
\end{longtable}

\subsection{义熙}

\begin{longtable}{|>{\centering\scriptsize}m{2em}|>{\centering\scriptsize}m{1.3em}|>{\centering}m{8.8em}|}
  % \caption{秦王政}\
  \toprule
  \SimHei \normalsize 年数 & \SimHei \scriptsize 公元 & \SimHei 大事件 \tabularnewline
  % \midrule
  \endfirsthead
  \toprule
  \SimHei \normalsize 年数 & \SimHei \scriptsize 公元 & \SimHei 大事件 \tabularnewline
  \midrule
  \endhead
  \midrule
  元年 & 405 & \tabularnewline\hline
  二年 & 406 & \tabularnewline\hline
  三年 & 407 & \tabularnewline\hline
  四年 & 408 & \tabularnewline\hline
  五年 & 409 & \tabularnewline\hline
  六年 & 410 & \tabularnewline\hline
  七年 & 411 & \tabularnewline\hline
  八年 & 412 & \tabularnewline\hline
  九年 & 413 & \tabularnewline\hline
  十年 & 414 & \tabularnewline\hline
  十一年 & 415 & \tabularnewline\hline
  十二年 & 416 & \tabularnewline\hline
  十三年 & 417 & \tabularnewline\hline
  十四年 & 418 & \tabularnewline
  \bottomrule
\end{longtable}


%%% Local Variables:
%%% mode: latex
%%% TeX-engine: xetex
%%% TeX-master: "../Main"
%%% End:

%% -*- coding: utf-8 -*-
%% Time-stamp: <Chen Wang: 2019-12-18 13:40:15>

\section{恭帝\tiny(419-420)}

\subsection{生平}

晉恭帝司馬德文(386年-421年11月2日),字德文,河內溫縣(今河南溫縣)人。東晉的末代皇帝。為晉孝武帝之子,晉安帝之胞弟,母親是淑媛陳歸女。初封琅邪王,後在桓玄篡位後長期侍奉晉安帝左右。晉安帝死後被劉裕以遺詔立為皇帝,但其時劉裕已經完全掌握東晉朝政,司馬德文僅為傀儡而已。劉裕篡晉後為零陵王,次年遇害。

太元十七年十一月庚寅日(392年12月27日)受封為琅邪王,後又拜中軍將軍、散騎常侍。隆安二年(398年)轉衞將軍、開府儀同三司。隆安三年(399年)遷侍中,領司徒、錄尚書六條事。元興元年(402年),桓玄擊敗司馬道子父子,掌握朝政,改以司馬德文為太宰。

元興二年十二月壬辰日(404年1月1日),桓玄篡位稱帝,貶晉安帝為平固王,司馬德文亦因而降封「石陽縣公」。不久桓玄遷安帝至尋陽(今江西九江市),司馬德文亦跟隨。元興三年(404年),劉裕起兵討伐桓玄,桓玄兵敗逃到尋陽,得郭昶之給予器具及士兵後再逼晉安帝與其同至江陵(今湖北江陵);及至桓玄敗死於逃往益州途中,荊州別駕王康產及南郡太守王騰之迎晉安帝至南郡府舍時,司馬德文亦緊隨。然而,桓振等桓楚餘眾趁劉毅等軍未及趕至江陵,乘虛來襲,最終江陵城陷,王康產及王騰之遇害,桓振亦騎馬揮戈直入,問桓玄子桓昇下落,並在得知其死訊後大怒,指責他們屠殺桓氏。當時司馬德文辯護道:「這又豈會是我們兄弟的意思!」在桓謙苦勸下,桓振才沒有加害安帝。隨後桓振繼續控制江陵,並以司馬德文為徐州刺史,繼續對抗由劉毅所統領的討伐軍隊。

義熙元年(405年),劉毅等軍攻下江陵,司馬德文亦與安帝一同在何無忌護送下返回建康。回建康後,司馬德文遷大司馬,並於義熙四年(408年)加領司徒。

義熙十二年(416年),劉裕預備北伐後秦,時劉裕圖以晉室名聲安撫北方人民,故想奉司馬德文之名北伐,司馬德文因而上書出兵,以修謁晉室山陵,最終劉裕就與司馬德文一同率兵出發。義熙十三年(417年),劉裕成功滅亡後秦,同年年末班師東歸,司馬德文亦跟隨,至次年(418年)夏季,劉裕到達彭城(今江蘇徐州市),司馬德文先回建康。不久,劉裕受九錫,封宋王。

劉裕當時指派了中書侍郎王韶之圖謀殺害晉安帝,立司馬德文為帝,以應「昌明之後尚有二帝」的預言。不過因司馬德文無論飲食還是睡覺都和晉安帝在一起,王韶之無法下手。可是司馬德文卻於當年年末患病,離開了安帝,王韶之趁機會下手,將安帝殺死。劉裕則假稱遺詔,以司馬德文繼位。

元熙二年六月壬戌(420年7月5日),劉裕入朝,傅亮暗示司馬德文禪讓帝位給劉裕,並將禪讓詔書的草稿上呈,要他抄寫。司馬德文欣然接受,執筆抄寫,並說:「桓玄篡位那時,晉室經已失去天下了,又因劉公延長了國祚,至今已將近二十年了;今日作這種事,是心甘情願的。」兩日後,司馬德文退居琅邪王府,百官向晉帝告別,東晉至此滅亡。又三日後,劉裕正式登位,並奉司馬德文為零陵王,讓他遷至秣陵縣(今江蘇江寧縣)的舊縣治作為其府第,正朔、車駕、衣服等都依晉朝規格,正如昔日晉篡魏的先例,並命劉遵派兵守衞。

及後劉裕就有殺害司馬德文的意圖,最初就命前琅邪國郎中令張偉拿毒酒去殺司馬德文,但張偉就嘆道:「要毒殺主君去讓自己活下去,不如死了!」竟在路上喝下毒酒自盡。司馬德文自己也十分害怕會遭毒手,於是起居飲食都由王妃褚靈媛打點,食物也在自己面前烹煮,令加害者無從下手。不過,褚靈媛兄褚秀之及褚淡之都忠於劉裕,一直以來司馬德文生下的男嬰都被二人借故害死。至永初二年(421年)九月,劉裕即命褚淡之及褚叔度去見褚靈媛,乘機支開她到另一個房間。及後士兵就翻過牆進入府內,逼司馬德文服食毒藥。但司馬德文不肯,更說:「佛教所稱,自殺的人都不能輪迴再生為人。」士兵於是用被褥將其悶死,享年三十六歲。司馬德文以晉禮下葬於沖平陵,諡恭皇帝。

史載,晉安帝司馬德宗從小到大都不會說話,甚至連冬夏的氣侯轉變也不能分辨。而司馬德文一直侍奉左右,打理他的生活起居,以恭敬謹慎而聞名,亦得當時人們稱許。

司馬德文信奉佛教,曾下令打造了一個高一丈六寸的黃金佛像,並親身到瓦官寺迎其上位。其死前所言亦証其篤信佛教。

據說司馬德文年幼時頗為殘忍急躁,在琅邪國時更曾命擅長射箭的人射擊馬匹作為娛樂。當時有人說:「馬是國姓,而你自己就去殺牠,這是很不祥的事呀!」司馬德文明白此言,亦甚為後悔。

\subsection{元熙}

\begin{longtable}{|>{\centering\scriptsize}m{2em}|>{\centering\scriptsize}m{1.3em}|>{\centering}m{8.8em}|}
  % \caption{秦王政}\
  \toprule
  \SimHei \normalsize 年数 & \SimHei \scriptsize 公元 & \SimHei 大事件 \tabularnewline
  % \midrule
  \endfirsthead
  \toprule
  \SimHei \normalsize 年数 & \SimHei \scriptsize 公元 & \SimHei 大事件 \tabularnewline
  \midrule
  \endhead
  \midrule
  元年 & 419 & \tabularnewline\hline
  二年 & 429 & \tabularnewline
  \bottomrule
\end{longtable}


%%% Local Variables:
%%% mode: latex
%%% TeX-engine: xetex
%%% TeX-master: "../Main"
%%% End:

%% -*- coding: utf-8 -*-
%% Time-stamp: <Chen Wang: 2019-12-18 13:45:57>

\section{桓楚\tiny(403-405)}

\subsection{简介}

桓楚,中國東晉時期由將領桓玄所建立的一個短期政權,存續期間為403年至404年。

東晉晉安帝元興二年(403年),控制東晉中央政府的楚王桓玄篡奪政權。十一月二十一日(陽曆為403年12月20日),安帝獻上玉璽,禪位於桓玄。十二月三日(陽曆為404年1月1日),桓玄正式稱帝,國號楚,改元永始。為與其餘國號為楚的政權區分,故史家稱桓玄建立者為桓楚。

名義上,桓楚於建立後直接繼承東晉的領土,但實際上其勢力範圍僅及江陵(今湖北江陵)以東的長江中下游一帶。404年,以劉裕為首的數名將領,起兵勤王,楚軍不敵,桓玄退出建康(今江蘇南京),並挾持安帝西逃至江陵。同年稍後,桓玄敗死,其堂弟桓謙將國璽奉還安帝,桓楚亡。

桓楚亡後,桓氏家族仍不斷在長江中游一帶興兵與東晉政府軍對抗,直到數年後才被消滅。

\subsection{桓温生平}

桓溫(312年-373年),字元子,譙國龍亢(今安徽懷遠縣龍亢鎮)人。東晉重要將領及權臣、軍事家,譙國桓氏代表人物。官至大司馬、錄尚書事。宣城內史桓彝長子,因領兵消滅成漢而聲名大盛,又曾三次領導北伐,掌握朝政並曾操縱廢立,更有意奪取帝位,但終因最後一次北伐大敗而令聲望受損,受制於朝中王氏和謝氏勢力而未能如願。死前欲得九錫亦因謝安等人借故拖延,直至去世時也未能實現。因桓溫獲賜諡號宣武,故《世說新語》稱其為「桓宣武」。其子桓玄後來一度篡奪東晉帝位而建立桓楚,追尊桓溫為「楚宣武帝」。

桓溫出生後還未夠一歲,就被溫嶠稱許,父親桓彝於是以「溫」作為桓溫的名字。年少與殷浩齊名。咸和三年(328年),桓彝在蘇峻之亂中被蘇峻將領韓晃所殺,當時桓彝所駐涇縣的縣令江播亦有協助。桓溫當時極度痛心,且一直想著為父報仇。桓溫十八歲時,江播已死,江播的三名兒子則在守喪,但他們仍有防備桓溫,將刀刃藏在杖中。桓溫則以弔唁為名,得以進入三人守喪的廬屋內,立殺江彪,及後追殺其餘兩人。

後桓溫娶南康長公主,拜任駙馬都尉,並承襲父爵萬寧男。咸康元年(335年)任琅琊太守。後升輔國將軍。建元元年(343年),桓溫配合征西將軍庾翼的北伐行動,假節任前鋒小督,進據臨淮。三個月後,桓溫升為都督青徐兗三州諸軍事、徐州刺史。永和元年(345年),庾翼病死,臨終前上表求以兒子庾爰之接掌荊州,作為自己繼任者。但輔政的何充則推薦桓溫,桓溫於是於當年獲升任安西將軍,持節都督荊司雍益梁寧六州諸軍事、領護南蠻校尉、荊州刺史,代替庾氏鎮守荊州。

永和二年(346年),桓溫趁成漢內部不穩,汉主李勢荒淫無道令國家衰弱,決心征伐。當年十一月就上表朝廷,並立刻率領益州刺史周撫、南郡太守譙王司馬無忌和建武將軍袁喬等進攻成漢。當時朝廷內部多數都認為蜀地險要偏遠,而且桓溫兵少而深入蜀境,都為他擔憂。次年三月,桓溫進兵至彭模,並聽從袁喬全軍進擊,只帶三日糧食直攻成都的計謀,只留參軍孫盛和周楚以弱兵在彭模守輜重,桓溫則親自率兵直攻成都。

及後李勢所派抵抗桓溫的將領李福嘗試襲擊彭模,但在孫盛等人奮戰之下被擊退。而桓溫進兵時遇到守將李權,三戰三勝,並一直逼近成都,李勢於是在笮橋率所有兵力抵抗桓溫。桓溫前鋒初時陷於不利,參軍龔護戰死,箭矢更射到桓溫所騎馬匹以前,兵眾十分恐懼而要撤退。但當時戰鼓鼓手卻錯誤擂鼓命兵眾進攻,袁喬於是拔劍領兵與成漢軍激戰,終大敗對方,桓溫於是進攻至成都城下並燒了城門。成漢人見此,再無鬥志,李勢亦乘夜棄城逃至葭萌。不久,李勢決定投降,桓溫受降並遷李勢及成漢宗室到建康。

桓溫平蜀後留駐成都一個月,在當地舉任賢能,表彰美善。又以成漢舊臣譙獻之、常璩等人作為自己參佐,成功安撫當地人民。桓溫即將返回荊州時,隗文、鄧定等人在蜀地叛亂,桓溫與袁喬、周撫等各自領兵討伐,都將對方擊破。後桓溫領兵還鎮江陵(今湖北江陵縣)。

永和四年(348年),朝廷以桓溫平蜀的功勳,升桓溫為征西大將軍、開府儀同三司,封臨賀郡公。

桓溫雖然滅掉成漢,聲名大振,但亦因此令朝廷忌憚他功高不能控制,輔政的會稽王司馬昱於是擢升揚州刺史殷浩處理朝政,以抗衡桓溫日漸增長的勢力。永和五年(349年),後趙君主石虎死,北方因石虎諸子爭位而再度混亂。桓溫見此,即進據安陸,並上疏請求北伐,但久久都沒有回音。至永和六年(350年),朝廷以殷浩為中軍將軍、都督五州諸軍事,委以北伐重任,以此抗衡桓溫。桓溫亦知朝廷以殷浩抗衡自己,感到很不忿,但桓溫亦知殷浩為人,並不憂心。當時桓溫除本官所都督六州外亦加都督交、廣二州共八州,此八州士兵和資源調配都不由朝廷掌握。故此當時桓溫屢次上表請求北伐不果後,再次上表請求北伐並立刻自行率四、五萬兵沿長江東進武昌,便令當時人心驚駭,殷浩亦曾打算辭官迴避,而司馬昱亦要寫信勸止桓溫,終令桓溫退兵回荊州。朝廷及後讓桓溫進位太尉,但桓溫辭讓不拜。

隨後兩年,殷浩都有率兵進行北伐,但沒有成果,反倒屢次戰敗,軍需物資更被略奪殆盡,令朝野怨恨。永和十年(354年),桓溫趁機上奏列舉殷浩罪行,逼使朝廷廢殷浩為庶人。桓溫開始掌權。

桓溫分別於354年、356年及369年發動北伐北方十六國的戰役。但除了第二次北伐成功收復洛陽,其餘兩次皆被擊退,成效不大。

永和十年(354年)二月,桓溫奏免殷浩後不久便發動第一次北伐,親率步騎四萬餘人進攻武關,水軍直指南鄉(今河南淅川县滔河乡),命司馬勳從子午道(秦嶺棧道,通向漢中)進攻以關中地區為根據地的前秦。桓溫後率軍在藍田(今陝西藍田縣)擊破氐族苻健軍隊數萬人,進駐長安東面的霸上,逼使前秦君主苻健以數千人退守長安小城。當地民眾很多都以牛和酒款待桓溫軍,而老人亦感觸得哭泣著說:「沒想過今天還能看到官軍!」然而,桓溫未有聽從順陽太守薛珍所言追逼長安,反待敵自潰。六月,苻雄率所有軍力在白鹿原擊敗桓溫。九月,因桓溫本想收割作軍糧的麥子被秦軍搶先收割,並堅壁清野,令晉軍糧秣不繼,被迫徙關中三千多戶一同撤返江陵。撤軍時更遭前秦軍攻擊,死亡失蹤者數以萬計。

桓溫在北伐期間,王猛曾經前來拜見,並大談當世之事,並署任王猛為軍諮祭酒。桓溫撤退時曾請王猛一同南行,並任命他為高官督護,但王猛沒有跟隨。

永和十二年(356年)三月,桓溫打算發動第二次北伐,請求移都洛陽,修復園陵。雖然桓溫上奏十多次都不被允許,但朝廷卻拜桓溫為征討大都督,督司、冀二州諸軍事,主征討之事。七月,桓溫從江陵起兵發動第二次北伐。八月,桓溫進軍洛陽以南的伊水,當時羌人姚襄正在圍攻洛陽,見桓溫攻來,於是撤去洛陽的圍城軍隊去抵禦桓溫。桓溫終在伊水大破姚襄,姚襄逃走。及後,據有洛陽的周成獻城向桓溫投降,桓溫於是成功收復故都洛陽。桓溫及後拜謁各皇陵及修復其中已被毀壞者。桓溫留穎川太守毛穆之、河南太守戴施等守護洛陽,自己則領三千多家歸降的人民南遷至長江、漢水一帶,返回荊州。升平四年(360年),桓溫進封南郡公。

隆和元年(362年),前燕將領呂護進攻洛陽,桓溫派庾希及鄧遐助陳祐守城。桓溫亦上奏請求晉室遷都洛陽,又建議南遷的士族返鄉,朝廷畏懼桓溫,不敢有異議;但士族們卻已安於南方,根本不願北返。在此憂慮之時,揚州刺史王述認為桓溫只是以遷都之名威壓朝廷,並非真心想還都洛陽,只要表示順從便可,毋須實行。詔書下達後,晉室始終沒有還都洛陽。

興寧元年(363年),桓溫獲加授侍中、大司馬、都督中外諸軍事、錄尚書事、假黃鉞。次年,桓溫率水軍移守合肥,朝廷改以桓溫為揚州牧、錄尚書事,並兩度徵桓溫入朝。桓溫在第二次徵召時才入朝,行至赭圻時停止並留駐當地。當時前燕又再進攻洛陽,守將陳祐留兵出奔。司馬昱知道後,於是於興寧三年(365年)與桓溫商議征討之事,並讓桓溫移鎮姑孰。但同年因晉哀帝死,征伐之事就暫停。同年,前燕攻陷洛陽。

太和四年(369年),桓溫為了樹立更高的威望,發動第三次北伐,並請與徐、兗二州刺史郗愔、江州刺史桓沖及豫州刺史袁真等一同討伐前燕。而桓溫其實一直都希望控制郗愔在京口(今江蘇鎮江)所統領的精兵,郗愔子郗超時為桓溫參軍,便修改了父親寫給桓溫的書信,變成以老病辭任二州刺史職位,並勸桓溫接掌自己所領軍隊。桓溫看信後十分高興,桓溫亦因而得以自領徐、兗二州刺史。及後,桓溫正式起兵,率五萬人從姑孰出發北伐。

桓溫前進至金鄉,因大旱引水讓水軍舟船得以進入黃河。當時郗超認為如此難以運輸補給,建議直攻前燕都城鄴城,或者停駐黃河、濟水一帶管理漕運,積聚足夠的物資待次年夏天才進攻。但桓溫都沒有聽從。桓溫派軍先後攻敗湖陸守軍、在黃墟迎擊的慕容厲和林渚的傅顏,前燕於是向前秦求救,桓溫亦前進至枋頭。桓溫及後沒有再進逼前燕,反希望以持久戰坐取全勝。九月,因袁真無法開通石門以通水路運輸,而前燕亦斷了桓溫糧道,桓溫見戰事不利而糧食又已盡,更聽闻前秦援兵将至,於是燒船、弃輜重鎧甲,自陆道撤退。途中遭前燕騎兵追擊,損失三萬餘人;更被前秦軍在譙郡擊敗,於是這次北伐以大败告終。

桓溫北伐後,命人修築廣陵城池並移鎮當地。又因北伐失敗而感到十分羞恥,並將罪責推給未能開通石門水道的袁真。袁真不甘心被桓溫誣以罪責,而上奏桓溫罪狀又不果,於是以壽春(今安徽寿县)叛歸前燕。同年袁真逝世,太和六年(371年),桓溫率军擊敗前秦援軍,並攻陷寿春,俘斩袁真子袁瑾。

桓溫雖然自從363年獲錄尚書事開始就干預朝政,而且自負有才能,早就有異志,所以才發起北伐希望先建功勳,然後領受九錫並進圖篡位。但因第三次北伐遭前燕及前秦擊敗,聲名和實力都減弱,圖謀不成。壽春被桓溫攻下後,參軍郗超知道桓溫的心意,於是建議廢立之計而加強桓溫聲威。桓溫亦早有此謀,於是在當年便廢晉廢帝司馬奕為東海王,改立司馬昱為帝,即晉簡文帝,自己以大司馬專權。

桓溫隨後就因厭惡殷氏和庾氏強盛,又忌憚時任太宰的武陵王司馬晞的軍事才幹,於是先上奏彈劾司馬晞「聚納輕剽,苞藏亡命」,並誣司馬晞將成叛亂禍根,成功將司馬晞及其子司馬綜免官。及後又派弟弟桓祕逼迫新蔡王司馬晃誣稱自己與司馬晞、司馬綜、著作郎殷涓、太宰長史庾倩、太宰掾曹秀、散騎常侍庾柔等人謀反。桓溫下令將他們收付廷尉,晉簡文帝只有哭泣。後在桓溫意願下,廷尉上奏要賜死司馬晞,簡文帝不願,下詔要再作議論。桓溫於是上書請誅司馬晞,言辭十分嚴厲急切。簡文帝見此,只得寫書給桓溫:「若果晉室國祚長久,那麼你就應該依從早前的詔命從事;如晉室大勢已去,那你就讓我退位讓賢吧。」桓溫見後,流汗色變,而司馬晞亦只被廢為庶人,未被誅殺。但庾柔、殷涓等人都被族誅。桓溫此後,威勢極盛,連謝安見他亦對他遙拜,更以君臣稱作二人關係,足見當時桓溫權勢已經比皇室更高,如同君主。

次年,簡文帝死,死前遺詔由桓溫輔政,如諸葛亮、王導的先例。當時群臣都因桓溫權勢而不敢以皇太子司馬曜為帝,反等待桓溫的決定。尚書僕射王彪之則以太子即位之正當性釋除群臣疑慮,迎司馬曜繼位為晉孝武帝。桓溫原本寄望簡文帝會將帝位禪讓給自己,或讓自己倣效周公為君主主理朝政。如今兩者皆否,大失所望,因而十分怨憤,更懷疑這是王坦之、謝安做的。不久,朝廷下詔桓溫入朝輔政,並加前部羽葆鼓吹,武賁六十人,桓溫辭讓。寧康元年(373年),桓溫入朝拜山陵,朝廷詔謝安及王坦之到新亭迎接桓溫,百官拜於道側。三月,桓溫患病,停建康十四日後退還姑孰。當時桓溫表示想受九錫,多番催促,而王彪之及謝安見桓溫病重,則借修改袁宏所寫的錫文暗中拖延。七月己亥日(8月18日),桓溫逝世,享年六十二歲,至此錫文仍未完成。朝廷追贈丞相,諡號為宣武。喪禮依司馬孚、霍光的儀式,葬姑孰青山。

桓溫少時有豪邁風氣。姿貌甚偉,面有七星。

桓溫伐蜀時經過長江三峽,部隊中有人捉了一隻小猿,母猿則在岸邊哀號,一直跟了桓溫的船隊行了百多里。及更跳了上船,但隨即就死了。及後有人剖開其腹,見其腸臟都斷成很多小段。桓溫知道後大怒,貶黜了捉了小猿的人。又一次,桓溫與眾人一同吃飯,一名參軍用筷子夾蒸薤菜時,薤菜秥在一起夾不起,而其他同桌的人又不幫助,看見參軍夾著不放的模樣更笑起來。桓溫見此,說:「一同吃飯仍不相助,何況遇到危難時呢?」於是將他們免官。都見到桓溫雖然是極具野心的將領,但亦能在小事上顯出顧及人情的性格。

桓溫性格儉樸,每次宴會只吃七個奠柈茶果。

桓溫每逢大事無靜氣。既要行伊霍之事,又慮太后意異(褚蒜子時為崇德太后),等待時「悚動流汗,見於顏色」(《晉書·后妃傳》)。太后禮佛畢,從容作答,桓溫又大喜過望。

桓溫因晉成帝姊南康公主而大貴,及掌軍權,盡廢明帝後人。簡文帝說,找郎婿找得王敦、桓溫輩,稍得志便要廢立人。

桓溫待南康公主寡仁義。行軍司馬謝奕為人狂放,醉後窮追不捨,桓溫只得到處逃竄。南康公主見之,訴曰:“非是狂司馬,安得見郎君。”

桓溫與弟桓沖志趣不合,情誼頗深。溫死,沖為國家故,立其幼子玄。

桓溫年少家貧,與人玩摴蒱曾大敗,要找袁耽求取勝之法。

有人曾問桓溫有關謝安及王坦之的優劣,桓溫正想說,但就後悔說:「卿喜傳人語,不能復語卿。」

桓溫第三次北伐時,行軍至金城,看到自己任琅邪太守時所種的柳樹已經十分粗大,慨嘆:「木猶如此,人何以堪。」於是扶著枝幹,拿著枝條,流下眼淚。

桓溫曾躺臥著說:「作此寂寂,將為文、景所笑。」然後起坐,又說:「既不能流芳後世,亦不足復遺臭萬歲邪?」。又曾經在經過王敦墓,說:「可人,可人!」

桓溫自以雄姿風氣是司馬懿、劉琨之流,若有人將他比作王敦就會很不高興。第一次北伐後,在北方獲得了一個老婢,是昔日劉琨的女伎。老婢見桓溫後就掩面哭泣,桓溫追問,老婢則答:「你很像劉司空大人(劉琨)。」桓溫聽後十分高興,便去整理衣冠,及後又召來老婢來問。老婢則說:「脣很像,但可惜太薄;鬚很像,但可惜是赤色;體形很像,但可惜太矮;聲音很像,但可惜不雄壯。」桓溫聽後脫下冠帶去睡,不高興了數日。

桓溫一次乘下雪打獵,先見王濛、劉惔等人。劉惔見他一身戎裝,問:「老賊欲持此何作?」桓溫說:「我若不為此,卿輩那得坐談?」

桓温曾讀皇甫謐的《高士傳》,讀到於陵仲子時就擲去書本,說:「誰能這樣苛刻對待自己!」

《晉書》評:「桓溫挺雄豪之逸氣,韞文武之奇才,見賞通人,夙標令譽。時既豺狼孔熾,疆場多虞,受寄扞城,用恢威略,乃踰越險阻,戡定岷峨,獨克之功,有可稱矣。及觀兵洛汭,修復五陵,引斾秦郊,威懷三輔,雖未能梟除凶逆,亦足以宣暢王靈。既而總戎馬之權,居形勝之地,自謂英猷不世,勳績冠時。挾震主之威,蓄無君之志,企景文而概息,想處仲而思齊,睥睨漢廷,窺覦周鼎。復欲立奇功於趙魏,允歸望於天人;然後步驟前王,憲章虞夏。逮乎石門路阻,襄邑兵摧,懟謀略之乖違,恥師徒之撓敗,遷怒於朝廷,委罪於偏裨,廢主以立威,殺人以逞欲,曾弗知寶命不可以求得,神器不可以力征。豈不悖哉!豈不悖哉!斯實斧鉞之所宜加,人神之所同棄。然猶存極光寵,沒享哀榮,是知朝政之無章,主威之不立也。」

庾翼:「桓溫少有雄略,願陛下勿以常人遇之,常壻畜之,宜委以方召之任,託其弘濟艱難之勳。」

何充:「桓溫英略過人,有文武識度。」

孫綽:「高爽邁出。」

劉惔:「鬚如反猬毛,眼如紫石稜,自是孫仲謀、司馬宣王一流人。」

\subsection{桓玄\tiny(403-404)}

\subsubsection{生平}

桓玄(369年-404年6月19日),字敬道,一名靈寶,譙國龍亢(今安徽懷遠)人,譙國桓氏代表人物,東晉名將桓溫之子,東晉末期桓楚政權建立者。曾消滅殷仲堪和楊佺期佔據荊江廣大土地,後更消滅了掌握朝政的司馬道子父子,掌握朝權。次年桓玄就篡位建立桓楚,但三個月後劉裕就舉義兵反抗桓玄,桓玄不敵而逃奔江陵重整軍力,但後再遭西討的義軍擊敗。試圖入蜀途中遇上護送毛璠靈柩的費恬等人,遭益州督護馮遷殺害。因曾襲父親「南郡公」之爵,故世稱「桓南郡」。

桓玄自幼為桓溫所喜愛。寧康元年(373年),桓溫去世,遺命其弟桓沖統率其軍隊,並接替他任揚州刺史,並以時年五歲的桓玄承襲其封爵南郡公。兩年後,桓玄的服喪期滿,桓沖亦離任揚州刺史,揚州文武官員與桓沖告別,桓沖摸著桓玄的頭說:「這是你家的舊官屬呀。」桓玄聽後就掩面哭泣,眾人都對這反應感到詫異。

桓玄長大後,相貌奇偉,神態爽朗,博通藝術,亦善寫文章。他對自己的才能和門第頗為自負,總認為自己是英雄豪傑,然而由於其父桓溫晚年有篡位的跡象,所以朝廷一直對他深懷戒心而不敢任用。直至太元十六年(391年),二十三歲的桓玄才被任命為太子洗馬。幾年後出京任義興(今江蘇宜興)太守,但還是頗覺不得志,曾感歎:「父為九州伯,兒為五湖長!」於是就棄官回到其封國南郡。

桓玄住在南郡的治所,也就是荊州的治所江陵,優游無事,荊州刺史殷仲堪本來對他十分敬憚,而桓玄因著父叔長年治理荊州的威望而專橫荊州,士民畏懼他更過於殷仲堪,殷仲堪因而與其深交。桓玄也打算借助其軍力,故此取悅他。

隆安元年(397年),尚書僕射王國寶、建威將軍王緒倚仗當權的會稽王司馬道子,因畏懼青兗二州刺史王恭,圖謀削弱各方鎮,桓玄知道王恭面對王國寶亂政有憂國之言,故此勸說殷仲堪起兵討伐王國寶,並派人勸說王恭,推王恭為盟主。當時,殷仲堪個人擔憂沒有孝武帝的支持,自己被群眾認為能力未達一州方伯的情況下會被王國寶等人利用,終令他被調離荊州。桓玄亦利用這個擔憂勸說殷仲堪,但殷仲堪始終遲疑。不過,當時王恭原來已決定主動起兵,並聯結殷仲堪,殷仲堪此時得報,於是答應了響應王恭。不久朝廷畏懼,故殺王國寶、王緒以息事寧人,王恭亦罷兵。然而,始終殷仲堪與桓玄始終沒有進行實質的軍事行動。

王恭舉兵以後,司馬道子憂慮王恭和殷仲堪的威脅,於是引司馬尚之和司馬休之為心腹。隆安二年(398年),因著桓玄請求朝廷讓他任廣州刺史,而司馬道子亦忌憚他,不想他繼續長據荊州,於是下詔以他督交廣二州、建威將軍、平越中郎將、廣州刺史、假節。桓玄受命但沒有到廣州上任。同時司馬道子聽從司馬尚之多樹外藩的建議,不料卻因削奪了豫州刺史庾楷都督地區而令其勸王恭再度舉兵,王恭遂於當年聯結桓玄、殷仲堪等舉兵討伐司馬尚之兄弟,桓、殷亦奉其為盟主。殷仲堪認為王恭這次肯定成功,於是積極參戰,更分五千兵給桓玄,緊隨擔任前鋒的南郡相楊佺期順江南下。楊、桓二人到湓口時,亦為討伐對象的江州刺史王愉逃奔臨川,但被桓玄派兵追獲。及後雖然庾楷大敗給司馬尚之,前來投奔桓玄,但桓玄也於白石大敗朝廷軍隊。及後雖然王恭敗死,但桓玄和楊佺期進至石頭,令司馬元顯回防京師,並命丹楊尹王愷守石頭城。不過,因為剛剛背叛王恭的劉牢之率北府軍入援京師,桓玄和楊佺期因畏懼而撤回蔡州,與朝廷軍對峙。

當時司馬道子打算利誘桓玄和楊佺期,令二人倒伐攻擊殷仲堪,於是以桓玄為江州刺史,楊佺期為雍州刺史,而殷仲堪就被貶廣州刺史。此舉卻令殷仲堪大怒,並命桓玄和楊佺期率兵進攻建康。不過桓玄卻對任命十分高興,打算接受,卻猶豫不決。當時殷仲堪從堂弟殷遹口中又聽聞楊佺期也決定受命,於是開始撤軍。隨著殷仲堪撤退,楊佺期部將劉系亦先行撤退,桓玄等大懼,又狼狽西退,直至尋陽(今江西九江市)追上殷仲堪。殷仲堪既失荊州刺史,倚仗桓玄為援;而桓玄本身亦要借助殷仲堪的兵力,故此據勢相結,殷仲堪與楊佺期因著其家世聲望,共推桓玄為盟主,皆不受朝命。朝廷見此大加恐懼,唯有下詔安撫,並讓殷仲堪復任荊州刺史,請求和解。眾人於是受命返回駐地。

早在桓玄在江陵橫行時,殷仲堪親黨就已勸殷仲堪殺死桓玄,但沒得聽從。桓玄自被推為盟主後,就更加矜侉倨傲,而楊佺期就被桓玄以寒門相待,然而出身弘農楊氏的楊佺期卻自以其族是華夏貴冑,一直都認為江東其他士族根本比不上他家,於是對桓玄十分不滿,更打算襲殺桓玄,可是因殷仲堪顧忌桓玄死後無法控制楊佺期兄弟才阻止。當時桓玄亦知楊佺期想殺死自己,於是有了消滅楊佺期的意圖,更屯駐夏口,並以始安太守卞範之為謀主。

隆安三年(399年)請求擴大其轄區,而司馬元顯亦想以此離間桓玄與殷、楊二人的關係,故此加桓玄都督荊州長沙郡、衡陽郡、湘東郡及零陵郡四郡諸軍事,並改以桓玄兄桓偉代楊佺期兄楊廣為南蠻校尉。此舉觸怒了楊佺期兄弟,楊佺期更以支援後秦圍攻的洛陽為名起兵,但皆被殷仲堪阻止。當年荊州有大水,殷仲堪開倉賑濟災民,桓玄就乘此機會起兵,亦以救援洛陽為名。當時桓玄寫信給殷仲堪,稱他要消滅楊佺期,並命殷仲堪收殺楊廣,否則會進攻江陵。桓玄並襲取殷仲堪在巴陵的積糧,又向路經夏口的梁州刺史郭銓假稱收到朝廷下令命郭銓為自己前鋒以討楊佺期,故此授江夏兵予他,命他督諸軍前進。

當時桓玄密報桓偉作為內應,但桓偉遑恐,更向殷仲堪自首,於是被對方擄為人質,並命其寫信給桓玄,在信中苦勸桓玄罷兵,不過桓玄不為所動,自度桓偉必因殷仲堪優柔寡斷,常慮兒子的性格而無危險。殷仲堪亦派了殷遹率七千水軍至西江口,桓玄派郭銓和苻宏擊敗他;及後殷仲堪又派楊廣及殷道護進攻,桓玄再在楊口擊敗他們,直逼至離江陵二十里的零口,震動江陵。後楊佺期自襄陽來攻,桓玄一度退後避其鋒銳,但終大敗楊佺期,及後由部將馮該並追獲及殺掉他。殷仲堪出奔,又被馮該追獲,及後被桓玄逼令自殺。

桓玄年末消滅了楊佺期和殷仲堪,於是在次年(400年)向朝廷求領荊江二州刺史。朝廷下詔以桓玄都督荊司雍秦梁益寧七州諸軍事、後將軍、荊州刺史、假節;另以桓偉為江州刺史。但桓玄堅持要由自己領江州刺史,朝廷唯有讓桓玄加都督江州及揚州豫州共八郡諸軍事,領江州刺史;桓玄又以桓偉為雍州刺史,朝廷礙於當時孫恩叛亂惡化,不能違抗。桓玄於是趁機在荊州任用腹心,訓練兵馬,並屢次請求討伐孫恩,但都被朝廷阻止。

隆安五年(401年),孫恩循海道進攻京口,逼近建康,桓玄聲稱勤王起兵,實質想乘亂而入,司馬元顯於是在孫恩北走遠離京師後下詔命桓玄解嚴。不過,桓玄當時完全控制了其轄區,不但作出調桓偉為江州、鎮守夏口,又以司馬刁暢督八郡、鎮守襄陽,桓振、皇甫敷、馮該等駐湓口等軍事調動,更建立了武寧郡和綏安郡分別安置遷徙的蠻族以及招集的流民。朝廷曾下詔徵廣州刺史刁逵和豫章太守郭昶之,亦被桓玄所留。

元興元年(402年),司馬元顯下詔討伐桓玄,在京的堂兄桓石生密報桓玄。桓玄既封鎖長江漕運,令東土饑乏,又因孫恩之亂未平,故認為司馬元顯無力討伐,於是一直在荊州等待時機,蓄勢待發。然而收到桓石生的通報後,桓玄甚懼,打算堅守江陵。不過卞範之卻勸桓玄出兵東下,以桓玄的威名和軍力,令其土崩瓦解;反不應主動示弱於人。桓玄於是留桓偉守江陵,親自率兵東下。桓玄初仍憂抗拒朝命,手下士兵都不會為他所用,然而過了潯陽仍未見朝廷軍隊,於是十分高興,士氣亦上升,移檄上奏司馬元顯之罪。桓玄到姑孰時,派馮該等擊敗並俘獲豫州刺史司馬尚之,並奪取了歷陽(今安徽和縣)。當時司馬元顯因畏懼,登船而未敢出兵,而劉牢之因擔憂擊敗桓玄後會不容於司馬元顯,竟與其手下北府軍向桓玄投降。桓玄逼近建康,司馬元顯試圖守城但潰敗。桓玄入京後,稱詔解嚴,並以自己總掌國事,受命侍中、都督中外諸軍事、丞相、錄尚書事、揚州牧,領徐州刺史,加假黃鉞、羽葆鼓吹、班劍二十人。

桓玄又列會稽王司馬道子及司馬元顯的罪惡,流放司馬道子到安成郡,數月後桓玄更派人殺死司馬道子;又殺司馬元顯、庾楷、司馬尚之和司馬道子的太傅府中屬吏。桓玄又圖除去劉牢之,先命他為會稽太守,令其遠離京口。劉牢之意圖反叛但得不到北府軍將領支持,於是北逃廣陵投靠廣陵相高雅之,於途中自殺。司馬休之、高雅之和劉牢之子劉敬宣於是北逃南燕。

桓玄在三月攻入建康時就廢除了元興年號,恢復隆安年號,不久又改元大亨。及後,桓玄自讓丞相及荊江徐三州刺史,以桓偉出任荊州刺史、桓脩為徐、兗二州刺史、桓石生為江州刺史、卞範之為丹楊尹、桓謙為尚書左僕射,分派桓氏宗族和親信出任內外職位。自置為太尉、平西將軍、都督中外諸軍事、揚州牧、領豫州刺史。另外又加袞冕之服,綠綟綬,增班劍至六十人,劍履上殿,入朝不趨,讚奏不名的禮遇。

四月,桓玄出鎮姑孰,辭錄尚書事,但朝中大事仍要諮詢他,小事則由朝中桓謙和卞範之決定。自晉安帝繼位以來,東晉國內戰禍連年,人民都厭戰不已,而桓玄上台後就罷黜奸佞之徒,擢用俊賢之士,令建康城中都一片歡欣景象,希望能過安定日子。不過很快,桓玄凌侮朝廷,豪奢縱欲,政令無常,故令人民失望。當時三吳大飢荒,很多人死亡,即使是富有的也不過著金玉財寶活活餓死家中,桓玄雖曾下令賑災,但米糧不多,給予不足,縱然會稽內史王愉召還出外尋食的飢民回去領糧,也就有很多人在道旁餓死。

另一方面,桓玄亦先後殺害吳興太守高素、竺謙之、高平相竺朗之、劉襲、彭城內史劉季武、冠軍將軍孫無終等北府軍舊將,以圖消滅劉牢之領下北府軍勢力。另亦要朝廷追論平司馬元顯和殷仲堪、楊佺期的功勳,分別加封豫章公及桂陽公,並轉讓給兒子桓昇及侄兒桓濬。又下詔全國避其父桓溫名諱,同名同姓者皆要改名,又贈其生母氏為豫章公太夫人。

元興二年(403年),桓玄遷大將軍,又上請率軍北伐後秦,但隨後就暗示朝廷下詔不准。桓玄本身就無意北伐,就裝作出尊重詔命的姿態停止。同年,桓偉去世,桓玄因公簡約禮儀,脫下喪服後又作樂。而桓偉一直是桓玄親仗的人,桓偉死後桓玄孤危,桓玄不臣之心已露,同時全國對其有怨氣,於是打算加快篡位工作。而桓玄親信殷仲文及卞範之當時亦勸桓玄早日篡位,連朝廷加授桓玄九錫的詔命和冊命都暗中寫好。桓玄於是進升桓謙、王謐和桓脩等人,讓朝廷命自己為相國,更劃南郡、南平郡、天門郡、零陵郡、營陽郡、桂陽郡、衡陽郡、義陽郡和建平郡共十郡封自己為楚王,加九錫,並能置楚國國內官屬。及後桓玄自解平西將軍和豫州刺史,將官屬併入相國府。

當時桓玄的行動令原為殷仲堪黨眾的庾仄起兵七千人反抗,趁著接替桓偉的荊州刺史桓石康未到就襲取襄陽,震動江陵,不過不久就被桓石康等所平定。桓玄及後又假意上表歸藩,卻又自己代朝廷作詔挽留自己,然後再請歸藩,又要晉安帝下手詔挽留,只因桓玄喜歡炫耀這些詔文,故此常常做這些自篇自導的上表和下詔事件。另桓玄亦命人報告祥瑞出現,又想像歷代般有高士出現,不惜命皇甫謐六世孫皇甫希之假扮高士,最終竟被時人稱作「充隱」。而桓玄對政令執行亦無堅定意志,常改變主意,令政命不一,改變起來亂七八糟。

元興二年(403年)十一月,桓玄加自己的冠冕至皇帝規格的十二旒,又加車馬儀仗及樂器,以楚王妃為王后,楚國世子為太子。十一月丁丑日(12月17日),由卞範之寫好禪讓詔書並命臨川王司馬寶逼晉安帝抄寫。庚辰日(12月20日),由兼太保、司徒王謐奉璽綬,將晉安帝的帝位禪讓給桓玄,隨後遷晉安帝至永安宮,又遷太廟的晉朝諸帝神主至琅邪國。及後百官到姑孰勸進,桓玄又假意辭讓,官員又堅持勸請,桓玄於是築壇告天,於十二月壬辰日(404年1月1日)正式登位為帝,並改元「永始」,改封晉安帝為平固王,不久遷於尋陽。

桓玄即帝位後,好行小惠以籠絡人心,例如他親自審訊囚犯時,不管罪刑輕重,多予釋放;攔御駕喊冤者,通常也可以得到救濟;然而為政繁瑣苛刻,又喜歡炫耀自己,官員有將詔書中「春蒐」字誤繕為「春菟」,經辦人員即全被降級或免職。

桓玄篡位以後,驕奢荒侈,遊獵無度,夜以繼日地遊樂。即使是兄長桓偉下葬的日子,桓玄白天哭喪到晚上就去遊玩了,有時甚至一日之間多次出遊。又因桓玄性格急躁,呼召時都要快速,當值官員都在省前繫馬備用,令宮禁內煩雜,已經不像朝廷了;另桓玄又興修宮殿、建造可容納三十人的大乘輿。百姓更因而疲憊困苦,民心思變。北府舊將劉裕、何無忌與劉毅等人於是乘時舉義兵討伐桓玄。元興三年二月乙卯日(404年3月24日),劉裕等人正式舉兵,計劃在京口(今江蘇鎮江)、廣陵(今江蘇揚州市)、歷陽和建康四地一同舉兵。其中劉裕派了周安穆向建康的劉毅兄劉邁報告,通知他作內應,然而劉邁惶恐,後更以為圖謀被揭向桓玄報告,桓玄初封劉邁為重安侯,但後又以劉邁沒有及時收捕周安穆,於是殺害劉邁和其他劉裕於建康的內應。原於歷陽舉兵的諸葛長民亦被刁逵所捕,但劉裕等終也成功奪取了京口和廣陵,鎮守兩地的桓脩和桓弘皆被殺。

劉裕率義軍進軍至竹里,桓玄加桓謙為征討都督。桓謙請求桓玄派兵攻劉裕,但桓玄畏於劉裕兵銳,打算屯兵覆舟山等待劉裕,認為對方自京口到建康後見到大軍必然驚愕,且桓玄軍堅守不出,對方求戰不得,會自動散走。不過桓謙堅持,桓玄就派了頓丘太守吳甫之及右衞將軍皇甫敷迎擊。不過二人皆在與劉裕作戰中戰死,桓玄大懼,就召見一眾會道術的人作法試圖對抗劉裕。後桓玄又命桓謙、何澹之屯東陵,卞範之屯覆舟山西,共以二萬兵抵抗劉裕。不過劉裕進至覆舟山東時故設疑兵,令敵方以為劉裕兵力眾多,桓玄得報後更派庾賾之率兵增援諸軍。然而,因為劉裕的兵眾大多是北府軍出身,故桓謙軍隊都畏懼劉裕,未有戰意,而劉裕則領兵死戰,並乘風施以火攻,終擊潰桓謙等。

在桓玄派桓謙等抵抗劉裕時,其實已經萌生離去的念頭,並命殷仲文準備船隻。桓謙等敗後,桓玄就於三月己未日(3月28日)與一眾親信西走。桓玄當天沒有進食,隨行人員就進糙米飯給桓玄,但桓玄吞不下,年幼的桓昇抱著桓玄撫慰他,更令桓玄忍不住心中悲傷。

桓玄一直到尋陽,得江州刺史郭昶之供給其物資及軍隊。後挾持晉安帝至江陵,在江陵署置百官,並且大修水軍,不足一個月就已有兵二萬,樓船和兵器都顯得很強盛的樣子。不過桓玄西奔後就怕法令不能認真執行,就輕易處以死刑,故令人心離異。

及後何無忌擊敗桓玄所派何澹之等軍,攻陷湓口,進佔尋陽,然後與劉毅等一直西進。桓玄亦自江陵率軍迎擊,兩軍於五月癸酉日(6月10日)在崢嶸洲相遇,當時桓玄軍雖然有兵力優勢,但因桓玄經常在船側泛舟,預演敗走時的動作,於是士眾毫無鬥志,在劉毅的進攻下潰敗,焚毁輜重乘夜逃走,郭銓遂向劉毅投降。桓玄於是挾晉安帝繼續西走,留晉穆帝皇后何法倪及安帝皇后王神愛於巴陵。殷仲文當時以收集散卒為名移駐別船,並趁機叛變,迎二后回建康。

桓玄於五月己卯日(6月16日)再到江陵,馮該勸桓玄再戰,但桓玄不肯,更想投奔梁州刺史桓希。不過當時人心已離,桓玄的命令都沒有人執行了。次日,江陵城中大亂,桓玄與心腹數百人出發,到城門時隨行有人從暗處走出要斬殺桓玄,但不中,於是彼此廝殺,桓玄勉強登船,身邊人員因亂分散,只有卞範之跟隨在側。桓玄正打算到梁州治所漢中(今陝西漢中市)時,但屯騎校尉毛脩之誘使桓玄入蜀,桓玄聽從。而當時正值寧州刺史毛璠去世,益州刺史毛璩派了侄孫毛祐之及參軍費恬等領數百人送毛璠喪至江陵,並於五月壬午日(6月19日)在枚回洲與桓玄相遇,二人於是進攻桓玄,箭矢如雨,桓玄寵信的丁仙期、萬蓋等為桓玄擋箭而死,益州都護馮遷跳上桓玄坐船,抽刀向前,桓玄拔下頭上玉飾遞給馮遷,說:「你是什麼人,竟敢殺天子?」馮遷說:「我這只是在殺天子之叛賊而已!」桓玄遂被殺,享年三十六歲。桓玄死後,堂弟桓謙在沮中為桓玄舉哀,上諡為武悼皇帝。桓玄頭顱則被傳至建康,掛在大桁上,百姓看見後都十分欣喜。

桓玄擅寫文章,可從其事跡中看到。王恭死後,桓玄曾登江陵城南樓,說:「我現在想為王孝伯作悼詞。」吟嘯良久後就下筆,很快就寫好了。桓玄消滅殷仲堪、楊佺期後,荊州刺史府、江州刺史府、後將軍府、七州都督府、南郡公府皆來賀,五個版牘一同進入,桓玄見版至使即答,皆美而成章,並不揉雜。

桓玄小時,與一眾堂兄弟鬥鵝,但桓玄的鵝總是不及堂兄弟強,十分不忿。於是有一晚到鵝欄殺死了堂兄弟們的鵝。天亮後家人都驚駭不已,以為發生了怪事,向桓沖報告。但桓沖心知是桓玄作的,一問,果然如此。

桓玄喜好裝飾和書畫,在擊敗司馬元顯後,桓玄遷鎮姑孰,就大築城內官府,建築物和假山水池等都十分壯麗。另又曾以輕舟載著他的書畫、服飾和玩物,有人因而勸諫他,桓玄竟說這些東西應該隨身,而且稱當時兵凶戰危,若發生問題就可以很快運走。眾人聽後都笑他。《晉書》又載他性格貪鄙,極愛奇珍異寶,珠玉等寶物更時不離手。別人有好書畫或佳園田宅,桓玄都想得到手,逼不到就在賭桌上奪得。桓玄又曾派下屬四出遷移果樹美竹收歸己有,令數千里內好的果樹和竹子都被一掃而空。

桓玄尊崇其父桓溫,故在篡位稱帝後就追尊桓溫為「宣武皇帝」,太廟都只供奉他,卻沒有追尊祖父桓彝或以上的祖宗。故及至桓玄遭受劉裕義軍來勢洶洶的進攻時,曹靖之稱其令晉室神主流離飄泊以及追尊不及祖父觸怒神明,令桓玄很是恐懼忿怨。

桓玄因劉裕討伐而西走江陵時,就於道上作《起居注》,內容都是他抵抗劉裕義軍的事,自稱自己指揮各軍,算無遺策,只因諸將違反其節度才兵敗,是非戰之罪。由於桓玄專心寫《起居注》,所以都沒閑暇時間和群下商議對策。寫成後桓玄就將《起居注》宣示遠近。

據說桓玄出生時,有光照亮房間,占卜者都感到奇異,故得桓玄小名靈寶。

桓玄早年頗善騎馬,曾在荊州刺史殷仲堪的江陵公廳前駕馬使矟,耀武揚威,卻被殷的部下劉邁(北府兵將劉毅之兄)貶低為:「馬矟的才能很夠,清談的義理卻不足」,桓玄因此痛恨劉邁,派人刺殺他,幸虧劉邁在殷仲堪的主意下,早一步回到京師,才躲過殺身之禍。

桓玄稱帝之後,入宮,因為身材發福肥大,當他坐上御牀後,不堪重擔的御牀就被壓爛陷地,眾人見此皆失色,殷仲文奉承說:「將由聖德深厚,地不能載。」令桓玄十分高興。又因為桓玄喜歡到宮外出遊,但肥大的體型對他上馬下馬諸多不便,他因此設計了能夠四面轉動的迴轉車,自己坐在上面可以方便地轉向移動。

據說,元興年間衡陽有母雞變成雄雞,八十日後雞冠卻萎縮了。後來桓玄建立楚國,衡陽郡亦在十郡以內,而自桓玄即位至敗走建康,也大約是八十日。當時亦有童謠:「長干巷,巷長干,今年殺郎君,後年斬諸桓。」郎君即司馬元顯,司馬元顯於元興元年(402年)被殺,桓氏則於元興三年(404年)因桓玄敗死而遭誅殺。

唐代房玄齡於晉書的「史臣曰」評論說:「桓玄纂凶,父之餘基。挾姦回之本性,含怒於失職;苞藏其豕心,抗表以稱冤。登高以發憤,觀釁而動,竊圖非望。始則假寵於仲堪,俄而戮殷以逞欲,遂得據全楚之地,驅勁勇之兵,因晉政之陵遲,乘會稽之酗醟,縱其狙詐之計,扇其陵暴之心,敢率犬羊,稱兵內侮。天長喪亂,凶力實繁,踰年之間,奄傾晉祚,自謂法堯禪舜,改物君臨,鼎業方隆,卜年惟永。俄而義旗電發,忠勇雷奔,半辰而都邑廓清,踰月而凶渠即戮,更延墜曆,復振頹綱。是知神器不可以闇干,天祿不可以妄處者也。夫帝王者,功高宇內,道濟含靈,龍宮鳳曆表其祥,彤雲玄石呈其瑞,然後光臨大寶,克享鴻名,允徯后之心,副樂推之望。若桓玄之么麼,豈足數哉!適所以干紀亂常,傾宗絕嗣,肇金行之禍難,成宋氏之驅除者乎!」

唐代某貴族「公子」與士族虞世南的對話:「公子曰:『桓玄聰明夙智,有奇才遠略,亦一代之異人,而遂至滅亡,運祚不終,何也?』先生(虞世南)曰:『夫人君之量,必器度宏遠,虛己應物,覆載同於天地,信誓合於寒暄,然後萬姓樂推而不厭也。彼桓玄者,蓋有浮狡之小智,而無含宏之大德,值晉室衰亂,威不迨下,故能肆其爪牙,一時篡奪,安德治民無聞焉。以僥幸之才,逢神武之運,至於夷滅,固其宜也。』」

梁代史家裴子野評論:「桓敬道有文武奇才,志雪餘恥,校〔狡〕動離亂之中,掩天下而不血刃,既而嘯命六合,規模凌取,未及逾年,坐盜社稷。自以名高漢祖,事捷魏、晉,思專其侈,而莫己知。王謐以民望鎮領〔袖〕,王綏、謝混以後進〔相〕光輝,群從兄弟,方州連郡,民駭其速而服其強,無異望矣。(宋)高祖是時,殊〔朱〕方之一匹夫也,無千百之眾,糾合同盟,雷擊三州,曾未及旬,蕩清京邑,號令群后,長驅江、漢,推亡楚於已拔,拯衰晉於已顛,自羲、軒以來,用兵之速,未始有也。自非雄略蓋世,天命至止,焉能若此者乎!於是,民知攸暨而王跡興。」

\subsubsection{永始}


\begin{longtable}{|>{\centering\scriptsize}m{2em}|>{\centering\scriptsize}m{1.3em}|>{\centering}m{8.8em}|}
  % \caption{秦王政}\
  \toprule
  \SimHei \normalsize 年数 & \SimHei \scriptsize 公元 & \SimHei 大事件 \tabularnewline
  % \midrule
  \endfirsthead
  \toprule
  \SimHei \normalsize 年数 & \SimHei \scriptsize 公元 & \SimHei 大事件 \tabularnewline
  \midrule
  \endhead
  \midrule
  元年 & 403 & \tabularnewline\hline
  二年 & 404 & \tabularnewline
  \bottomrule
\end{longtable}

\subsection{桓谦\tiny(404-405)}

\subsubsection{生平}

桓謙(4世紀-410年),字敬祖,譙國龍亢(今安徽懷遠)人。東晉末期人物,車騎將軍桓沖次子。在晉官至西中郎將、荊州刺史;桓楚時官至侍中、衞將軍。桓玄死後,桓謙仍然抵抗東晉,並於失敗後出奔後秦。後又因支持西蜀王譙縱對抗東晉而入蜀,終在西蜀的軍事行動下而再度與東晉作戰,被劉道規擊敗,被殺。

他初以父親的功勞封宜陽縣開國侯,歷次升遷官拜輔國將軍、吳國內史。隆安三年(399年),孫恩率眾進攻下會稽,並殺太守王凝之,三吳諸郡都有人起兵響應孫恩,桓謙聞亂出奔無錫(今江蘇無錫)。後桓謙獲徵召入朝擔任尚書,不久又先後轉任驃騎大將軍司馬元顯的諮議參軍及司馬。

元興元年(402年),司馬元顯要討伐荊州刺史桓玄,司馬元顯心腹張法順認為桓謙是桓玄在朝中的的耳目,應該除去,又建議命令劉牢之去下手,以測試其忠心。但司馬元顯不聽從,反而想借助桓謙父桓沖在荊州的威望去安撫荊州人,於是調桓謙為都督荊益寧梁四州諸軍事、西中郎將、荊州刺史。

同年,桓玄消滅司馬道子、司馬元顯勢力,掌握朝政,就以桓謙為尚書左僕射,領吏部,加中軍將軍,甚得桓玄倚仗。後改封寧都侯,升任尚書令,加散騎常侍,不久再遷任侍中、衞將軍、開府、錄尚書事。元興二年(403年),桓玄篡位稱帝,桓謙加領揚州刺史,封新安郡王。

元興三年(404年),劉裕起兵討伐桓玄,並進攻建康,桓玄於是命桓謙與何澹之出屯東陵(今南京九華山東北),與屯於覆舟山西的卞範之一同抵禦劉裕。但因桓謙等軍主要也是北府軍出身,面對北府軍將領出身的劉裕並沒鬥志,於是桓謙等大敗。及後隨桓玄西奔江陵(今湖北荊州市荊州區)。

同年五月,桓玄敗死,江陵亦被晉軍收復,桓謙藏匿在沮中。不久桓振襲取江陵,桓謙亦召集部眾響應,至閏五月己丑日(6月26日)重奪江陵,並俘虏仍在江陵的晉安帝。當時桓振打算殺害晉安帝,在桓謙竭力勸止下,終保存了晉安帝的性命,又與江陵群臣奉還玉璽給晉安帝。桓謙於是復任侍中、衞將軍,加江、豫二州刺史。桓振奪江陵後縱情酒色,肆意誅殺,當時桓謙勸桓振率兵出戰,自己留守江陵,但因桓振向來輕視桓謙而沒有聽從。

義熙元年(405年),晉軍反攻江陵,桓振留桓謙及馮該守江陵,親自率兵進攻南陽太守魯宗之,但當時劉毅已於江陵城外二十里的豫章口擊敗馮該,桓謙於是棄城出逃,劉毅於是成功收復江陵,桓振見此亦自潰。桓謙與桓怡、桓蔚、何澹之及溫楷等人於是投奔後秦。

義熙三年(407年),西蜀君主譙縱向後秦稱藩,後更上表以討伐劉裕為名向後秦借兵,又求後秦派桓謙入蜀協助。當時後秦天王姚興就特別問桓謙意見,桓謙也同意入蜀,然而姚興卻說:「小水池容不下大魚,若果譙縱他憑自己力量可成事,也就不必請你去協助他了。你最好還是自求多福吧。」桓謙到成都後虛心招引蜀地士人,終惹來譙縱懷疑,安置他於龍格(今四川雙流),並命人監視他。

義熙六年(410年),當時東晉正在鎮壓盧循的叛亂,譙縱於是趁機向後秦請兵進伐東晉,桓謙於是獲譙縱任命為荊州刺史,與譙道福共率二萬進攻東晉荊州。桓謙在道上招集當地支持桓氏的民眾,又招得了二萬人,並屯駐於枝江(今湖北枝江西南),一度威脅江陵,江陵人民甚至向桓謙報告城內狀況。東晉荊州刺史劉道規決定水陸並進,進攻桓謙;桓謙亦以水軍配以步騎兵與劉道規決戰,但桓謙最終戰敗,想要投靠前來助攻的後秦前將軍苟林,但被追擊的劉道規所殺。

\subsubsection{天康}


\begin{longtable}{|>{\centering\scriptsize}m{2em}|>{\centering\scriptsize}m{1.3em}|>{\centering}m{8.8em}|}
  % \caption{秦王政}\
  \toprule
  \SimHei \normalsize 年数 & \SimHei \scriptsize 公元 & \SimHei 大事件 \tabularnewline
  % \midrule
  \endfirsthead
  \toprule
  \SimHei \normalsize 年数 & \SimHei \scriptsize 公元 & \SimHei 大事件 \tabularnewline
  \midrule
  \endhead
  \midrule
  元年 & 404 & \tabularnewline\hline
  二年 & 405 & \tabularnewline
  \bottomrule
\end{longtable}


%%% Local Variables:
%%% mode: latex
%%% TeX-engine: xetex
%%% TeX-master: "../Main"
%%% End:


%%% Local Variables:
%%% mode: latex
%%% TeX-engine: xetex
%%% TeX-master: "../Main"
%%% End:
 % 东晋
% %% -*- coding: utf-8 -*-
%% Time-stamp: <Chen Wang: 2019-12-18 13:59:00>

\chapter{十六国\tiny(304-439)}

\section{简介}

五胡十六国(304年-439年),是中国历史上的一段时期。該時期自304年劉淵及李雄分別建立漢趙及成汉起至439年北魏拓跋燾(太武帝)灭北凉為止。範圍大致上涵蓋華北、蜀地、遼東,最遠可達漠北、江淮及西域。在入主中原眾多民族中,以匈奴、羯、鮮卑、羌及氐為主,統稱五胡。他們在這個範圍內相继建立許多國家,而北魏史學家崔鴻以其中十六個國家撰写《十六国春秋》(五凉、四燕、三秦、二赵,成漢、胡夏为十六國),於是後世史學家稱這時期为「五胡十六国」。

在西晉時期,五胡居於西晉北方、西方的邊陲地區,對晉王朝呈現半包圍局面。由於晉廷的腐敗和漢官的貪污殘暴,五胡在八王之亂後紛紛舉兵,史稱五胡乱华。在西晉滅亡後,華北地區战火纷飞,掠奪與屠殺不斷。经济受到嚴重摧毀,影响中華的民族、文化、政治、军事等发展走向。永嘉之亂帶給人民巨大痛苦,大多逃難到涼州、遼東以及江南地區,使這些地區的經濟文化漸漸繁榮。在諸國混戰期間,前秦宣昭帝苻堅一度統一華北,但在南征東晉時,於淝水之戰慘敗。其後各族於關東及空虛的關中叛變,加上東晉北伐,前秦全面崩潰,北方再度混亂。北魏立國後,經過道武帝拓跋珪、明元帝拓跋嗣及太武帝拓跋燾的經營,最後於439年統一華北,進入南北朝時期。

北方各族的内徙促成民族大融合,在中国作为多民族国家的发展过程中具有重要意义。同时,各國的君主为了增强实力,也在各自的根據地上实行一些发展生产的政策,使得各地区在華北动荡的背景下,形成局部稳定的局面。該時期的民族大融合持續到隋朝時期才大致上完成。

五胡十六國時期是西晉滅亡到北魏統一華北期間,涵蓋華北、華中北部和四川等地區的北方諸國的概稱,相對於南方的東晉時期。「五胡」即匈奴、羯、鮮卑、氐、羌五个民族,代表統治北方諸國的民族。實際上,北方諸國的統治者還包含漢人(前涼、西涼等等)、丁零人(翟魏)、盧水胡(北涼)與匈奴人鐵弗(胡夏)等民族。而地方人民也遺留不少漢人,與統治民族形成合作關係。「十六國」則是源自北魏末年的史官崔鴻私下撰寫的《十六國春秋》而得名。他自北方諸國中選出國祚較長、影響力大、較具代表性的十六國(“五凉、四燕、三秦、二趙,成漢、胡夏为十六”),共有:成汉、前赵、后赵、前涼、前燕、前秦、后燕、后秦、西秦、後涼、南涼、西涼、北涼、南燕、北燕及胡夏等國;實際上,北方諸國還包括冉魏、翟魏、西燕等等國家。總之,五胡十六國只是北方諸國的概稱,並不是北方諸國只有五個民族統治,以及只存在十六個國家。

關於「五胡十六國」稱呼的出處。在文獻上,五胡之名最早出自苻堅之口,但沒有明確定義五胡是哪五胡。而定義五胡內容的來源,學界有所爭議。歷史學家王樹民、孫仲匯、雷家驥等人認為五胡即五部胡人,源自劉淵領導的五部匈奴,但在這個時期的史書中,五胡常被當成所有胡人的泛稱,未特定指某個種族,在談到匈奴時,通常直接稱其匈奴,因此這個說法未得到學界一致認同。而陳寅恪認為五胡之名起自於五德終始說,是圖緯符命思想下的產物,周一良也支持這個說法。

川本芳昭認為,在《十六國春秋》成書之後,中國傳統史家依此思路整理史料,才開始將五胡的具體內涵確定下來,日本學者礪波護在《隋唐帝國與古代朝鮮》一書中認為五胡十六國這個概念是在唐朝初期編定正史時才形成的,南宋洪邁在《容齋隨筆》〈五胡亂華〉條中列舉七個人︰劉聰、劉曜、石勒、石虎、慕容皝、苻堅、慕容垂,這七個人分屬四個民族:匈奴、鮮卑、羯、氐,因此五胡的內涵在南宋時可能仍未完全確定,王應麟將五胡解釋為「劉淵匈奴,石勒羯,慕容鮮卑,苻洪氐,姚萇羌。」元胡三省註《資治通鑑》時,將五胡定義為「匈奴、羯、鮮卑、氐、羌」,這個定義可能來自劉曜。在胡三省之後,五胡即「匈奴、羯、鮮卑、氐、羌」這個定義開始被廣泛接受。陳寅恪曾認為五胡與十六國是兩個不同概念,不可混合

秦末漢初,漠北的匈奴成為一個強大帝國,並多次南下劫掠,在被漢朝打敗後,一部分受到漢王朝控制。公元46年之後,東漢朝廷常以招引的方式,將邊疆的草原各族內遷,以便監控或是增加兵源和勞動力。朝廷有意識的削弱邊疆民族的勢力,降低其地位,以方便監控。

到了西晉時,中國漢地北部、東部和西部,尤其是并州和關中一帶,大量胡族與漢族雜住。史書記載「西北諸郡皆爲戎居」,關中百萬餘口「戎狄居半」,對晉帝國呈現半包圍形勢。除了辽河流域的鲜卑和青海、甘肃的氐、羌外,大都由其原住地遷來。這些胡族逐漸成為漢人管轄下的編戶,由於他們需要納稅,且時時受漢官欺壓或歧視,因此心生不滿,時有舉兵之事。270年晉武帝時,河西鮮卑禿髮樹機能與匈奴劉猛率眾內侵,直至九年後始平。294年晉惠帝時,匈奴郝散叛,不久平定。兩年後其弟劉度元以齊萬年為首,聯合西北馬蘭羌、盧水胡舉兵,晉將周處陣亡,此事至299年方平。而後郭欽與江統相繼建議強制遷離胡族,江統更著有《徙戎論》,但晉室不予採納。由於胡漢摩擦的狀況沒有改善,當朝廷元氣大傷後,周邊胡族便趁機舉兵。

八王之亂的爆發,使晉廷失去在地方的影響力,胡族陸續叛變。晉惠帝時,益州內亂,巴氐勢力擴大。之後益州刺史羅尚擊殺巴氐領袖李特。304年,李特子李雄繼立後擊敗羅尚,攻入成都,自稱“成都王”(此时晋廷所封成都王为驻邺城的司馬穎),又於306年稱帝,國號「大成」,338年改国号为“汉”,史稱成漢。匈奴劉淵統領五部匈奴,成都王司馬穎結其為外援。304年司馬穎遭王浚圍攻,遣劉淵回并州發兵支援。劉淵回并州后乘機宣布獨立,稱漢王,自稱繼承漢朝正統。308年劉淵稱帝並遷都至平陽,國號「漢」,后稱「趙」。304年成漢與汉趙的建立,開啟了「五胡十六國」時期。

八王之亂結束後,劉淵為了擴充版圖,遣子劉聰掠奪洛陽,大將石勒及王彌掠奪關東各州。310年劉淵去世,劉聰殺新帝劉和自立為帝。同年,石勒經宛城、襄陽,掠奪江漢一帶,隔年北返。當時關東發生蝗災,洛陽缺糧,司馬越棄晉懷帝於洛陽,率朝中重臣及諸將東行。而後懷帝動員諸將討伐,司馬越病逝,王衍率軍歸葬封國(在東海)。石勒趁王衍東行至苦縣(今河南鹿邑縣)時率軍襲擊,晉軍精銳受屠盡亡,重臣降後被殺。劉聰、王彌及石勒趁洛陽空虛之際合兵攻破,殺害官員百姓三萬餘人,擄走晉懷帝,史稱「永嘉之亂」。313年晉懷帝被殺,晉愍帝於長安繼立帝位,劉聰派劉曜持續攻打。316年晉愍帝投降,最後受辱被殺,至此西晉灭亡。北方諸國紛紛成立。313年張軌控制涼州,封西平公,史稱前涼。315年拓跋猗盧建立代國。334年慕容鮮卑據遼東立國。

劉聰滅西晉後安逸豪奢,疏忽政事,當時曹嶷、石勒等將領分別佔據山東及關東。實際範圍只有山西和劉曜鎮守的關中。318年劉粲繼立,但遭靳準殺害奪權。劉曜與石勒得知後共同平亂,期間劉曜稱帝,改國號為「趙」,史称前赵。石勒得知後也於襄國稱趙王,史稱後趙,雙方決裂。劉曜平定上郡羌、仇池氐等關隴羌氐,威服前涼,雄踞關中。石勒則派石虎擊敗晉將段匹磾奪幽州,擊敗曹嶷奪青州。石勒雄踞關東後,於328年西征劉曜,329年攻滅前趙。330年石勒稱帝,國號亦為「趙」。前涼方面,由於戰亂較少,難民紛紛前往安居,保存了晉代典章制度,久之形成「河西文化」。

石勒為一時雄才,他得漢人張賓相助,安撫世族,重建經濟。當時胡漢關係欠佳,石勒採胡漢分治,於皇帝外另設大單于。稱胡人為國人,漢人為趙人。但這樣未能緩和雙方關係,仍然有衝突發生。石虎於石勒去世後殺石弘自立為天王。他奢侈極淫,任意濫殺,又聽信讒言,奴役非國人的漢人及「六夷」,後趙國勢漸衰。因帝位等因素,石虎與其子石邃(太子)、石宣、石韜發生骨肉相殘,宗室關係降至冰點。349年石虎稱帝後,舊太子黨人梁犢於關中叛變,石虎遣羌將姚弋仲及氐將苻洪平定,羌氐二族坐大。石虎去世後,諸子爭位,殘殺甚烈,後為養子漢人冉閔(石閔)奪得,於350年改國號為魏,是為冉魏。冉闵重用漢人,並鼓勵誅殺羯人,造成對胡人的大屠殺。之後石祗於襄國稱王,號召鮮卑、氐、羌等族抵抗冉閔。冉閔欲聯合東晉驅除胡族,但晉廷因為他稱帝而不理,反而支持向東晉稱臣的鮮卑慕容儁。352年慕容儁攻破邺都,杀冉闵,冉魏灭亡。另外,346年東晉將領桓溫攻擊成漢(成漢於338年為李壽篡位,改國號為「漢」),次年攻入蜀地,成漢亡。

慕容鲜卑於晉室南渡後佔據遼東。337年慕容皝稱燕王,他擊潰來犯的石虎,攻滅遼西段氏鮮卑,繼而重創高句麗,其勢壯盛。慕容儁繼位後,乘後趙内讧之際發兵南侵。352年攻滅冉魏,冉閔兵敗被殺,慕容儁稱帝,建國前燕。先前前燕向東晉稱臣,等冉魏滅後,慕容儁對東晉使者言道:「汝還白汝天子,我承人之乏,為中國所推,已為帝矣」。此時前燕據有關東,關中則為前秦據之。之後慕容儁又派慕容垂、慕容虔與平熙等北伐大破丁零(敕勒)。356年桓溫北伐前燕,攻陷洛陽以及司、兗、青、豫四州,之後桓溫返國,前燕復奪回四州。358年慕容儁下令全國州郡整頓戶口,準備組織150萬大軍以滅東晉,但於隔年閱軍時逝世。慕容暐繼立後,以名將慕容恪輔政,期間慕容恪將東晉收復的洛陽攻下。但慕容暐窮奢極慾,國庫逐漸掏空。慕容恪去世後由慕容評執政,他貪墨昏庸,國政更亂。369年東晉桓溫率軍北伐,進駐枋頭(今河南浚縣附近)。慕容垂率軍嚴防,最後追擊晉軍,晉軍大潰。戰後慕容垂聲名日盛,但遭慕容評排擠而投奔前秦。

氐將苻洪在石虎去世後投降東晉,在後趙内讧時意圖奪下關中,但遭人毒死。350年其子苻健成功奪下關中,建國前秦,與東晉斷絕。之後東晉履次派褚裒、殷浩、桓溫等率軍伐之,苻健皆成功抵禦,國勢漸固。之後苻健之子苻生繼立,他淫殺無度,苻健之侄苻堅殺而代之。苻堅崇尚儒學,獎勵文教。他得王猛輔政,得以集權中央,經濟提升,國勢大盛,史稱「關隴清晏,百姓豐樂」。前秦強盛後,苻堅有意一統天下。當時前燕混亂,369年慕容垂投奔前秦。苻堅趁勢派王猛、慕容垂率軍於隔年成功滅燕,取得關東地區。隨後於373年滅前仇池,376年滅代國(拓拔鮮卑)及前涼,前秦統一北方。

在統一北方前,苻堅也開始入侵東晉,於373年攻下東晉梁益二州。五年後派苻丕攻下襄陽,俘虜朱序;派彭超圍攻彭城,但被謝玄擊敗。383年派吕光西定西域,這是自東漢之後再度佔據西域。前秦統一北方後,四周諸國遣使通好,此時只剩東晉,苻堅有意伐之。鮮卑慕容垂與羌將姚萇皆盡力支持苻堅,但王猛與苻融等氐族大臣則強烈反對。這時因為苻堅將諸胡遷入關中以便控制,又將氐族勢力置於國內要衝,以鞏固勢力,此法卻使京師空虛。而且他為人寬弘,亡國君臣皆授官位,但任其率領舊部,造成隱憂。

王猛去世前告誡苻堅應該先整合好國內異族再南征,但苻堅仍一意孤行。383年5月桓沖率10萬兵攻襄陽,苻堅派苻睿、慕容垂等人防禦。苻堅認為時機已到,於8月率舉國之師南征東晉,兵分三路,聲勢浩大。他親率步兵60萬抵達項城,派苻融為先鋒率27萬兵攻打壽陽,梁成等人屯洛澗以控淮河。東晉謝安則命謝石、謝玄等人率8萬北府兵北上救援。10月秦軍前鋒攻陷壽陽後,苻堅趕往指揮,並派朱序向謝石諸將勸降,但朱序盡洩秦軍虛實。11月晉將謝玄派劉牢之率五千精兵攻破洛澗並率軍西行,與秦軍對峙淝水。12月謝玄向苻堅建議後退決戰。諸秦將認為阻敵淝水畔比較安全,但苻堅認為半渡而擊可主動對決。當秦軍後移時,晉軍渡水突擊,朱序於後軍大喊秦軍已敗。此時秦軍大亂,謝玄等人乘勝追擊,秦軍全面崩潰,苻融戰死,苻堅中箭,孤身北返,後由慕容垂護送,史稱淝水之戰。

由於前秦的主力在前方,京師兵力不足,關中的鮮卑、羌、羯等族在得知前秦大敗後紛紛獨立。隔年東晉發動北伐,攻下山東河南一帶。至此前秦崩潰,北方再度回到諸雄混戰的局面。淝水之戰隔年(384年),各胡族紛紛獨立。鮮卑慕容垂於河北復國,史稱後燕;前燕皇族慕容泓與慕容沖於山西建國西燕;前秦羌將姚萇自立,建國後秦。第二年(385年)西燕軍攻陷長安,苻堅最後被姚萇所殺。由於前秦鄴城被後燕攻下,苻丕於晉陽繼立。苻堅被殺後,鎮守前秦勇士川(今甘肅榆中)的鮮卑將乞伏國仁自立,建國西秦。仇池氐楊定也宣佈復國,並稱藩於東晉,史稱後仇池。

第三年(386年)拓跋鮮卑拓跋珪於代地復國,國號「魏」,臣服於後燕,史稱北魏。西定西域的前秦氐將呂光返國並佔據涼州,在得知苻堅被殺後於姑臧(今甘肅武威)建國後涼。西燕的人民(鮮卑族)欲東歸故鄉而發生內亂,最後由慕容永率眾東征佔據并州(今山西省範圍),建都長子。而前秦苻丕欲西行關中但被西燕帝慕容永所阻,南下東桓被東晉守將馮該殺死。前秦苻登於南安繼立,據有隴西。三年內,北方八國並立,關隴地區有前秦、後秦、西秦、後涼、後仇池,關東地區則有後燕、西燕及北魏,維持了九年。

關中方面,後秦帝姚興於394年連同西秦帝乞伏乾歸滅前秦。六年後後秦攻滅西秦,乞伏乾歸投降,受姚興重用。而匈奴鐵弗部族長劉衛辰因攻北魏戰敗而亡,其子劉勃勃(後改姓赫連)投奔後秦。在後涼投降後秦後,關中暫時為後秦盤據。407年赫連勃勃叛秦,於統萬建國胡夏,並屢次攻擊後秦。後秦國勢大衰,乞伏乾歸趁機光復西秦。其子乞伏熾磐繼立後攻滅南涼,據有隴西。416年12月後秦幼主姚泓初立,東晉劉裕發動第二次北伐,率王鎮惡等將伐後秦。晉軍連克許昌、洛陽。隔年攻破長安,後秦亡。之後劉裕因故返國,留守將領發生內鬨。夏帝赫連勃勃趁機率軍攻下長安,據有關中。另外,於405年建國譙蜀的譙縱,早在劉裕第一次北伐後就派朱齡石攻陷成都,譙蜀亡。

河西方面,後涼分裂出南涼及北涼,由於四周強敵漸漸威脅,最後向後秦投降。397年禿髮烏孤脫離後涼,建國南涼,最後南涼敗於北涼和夏,為西秦所滅。同年匈奴別部盧水胡沮渠蒙遜擁漢人段業於張掖獨立,401年沮渠蒙遜殺段業取代,史稱北涼。405年敦煌太守李暠(漢族)叛北涼,建國西涼,後亡於北涼。此時關隴地區有胡夏、西秦、北涼及後仇池四國。

關東方面,西燕在并州(今山西省範圍)建國後,於394年被後燕帝慕容垂所滅。由於北魏帝拓跋珪派兵幫助西燕,所以隔年慕容垂派太子慕容寶北伐北魏。慕容寶於參合陂之戰慘敗給拓跋珪後,請求其父慕容垂為他雪恥。於是慕容垂於隔年親率大軍伐魏,攻陷平城,拓跋珪則率眾北遁以迴避之。但慕容垂於返途中去世,之後後燕逐漸衰弱。396年拓跋珪攻下并州,隔年慕容寶企圖反擊并州,最後被拓跋珪擊敗。而後拓跋珪大舉入侵,圍陷後燕首都中山,並遷都到平城。慕容寶則撤至根本之地龍城,後燕分裂為兩地。此時慕容德不願撤往北方,南下滑台,建國南燕,之後遷都至廣固。後燕在慕容熙稱帝後,君主昏庸,百姓勞苦,國家衰敗。409年馮跋舉兵殺慕容熙,擁高雲為帝,建都龍城,之後馮跋繼立,史稱北燕。而南燕在慕容超繼任後屢次攻伐東晉,最後於隔年被東晉的劉裕討伐而亡。此時關東僅北魏、北燕兩國。

北魏拓跋嗣繼立後,時常攻掠劉宋(劉裕篡東晉後所建之國)的河南地。423年北魏拓跋燾繼立,他勵精圖治,國力大盛。拓跋燾在解除北方柔然的威脅後,開始統一華北。北魏對各民族的文化與制度採取包容態度,這減少北魏進軍的阻礙,但也使北方民戶複雜化。三年後拓跋燾大舉伐夏,攻下關中,胡夏遷至平涼。430年西秦為北涼所逼,意圖投降北魏,但隔年為夏帝赫連定所滅。赫連定意圖再滅北涼以維持胡夏,但卻被吐谷渾君主慕容慕璝襲擊而俘虜,最後斬於北魏,胡夏亡。436年拓跋燾率軍遠征北燕,馮弘逃至高句麗,北燕亡,馮弘最後被殺。439年北魏大軍圍攻姑臧,沮渠牧犍出降,北涼亡。至此,北魏統一華北,進入「南北朝時期」。然而,還有後仇池未滅,直至443年方亡於北魏。

西晉末年,全國共有21州。十六國時期,北方諸國的範圍大約是華北地區及四川地區。疆域的變更可分成五期,分別是:前趙、後趙、成漢及東晉時期;前燕、前秦及東晉時期;前秦東晉對峙時期;諸國混戰與東晉時期,此時北方以後燕及後秦最盛;北魏、胡夏、北涼及東晉時期。在諸國分立的時期,只有局部地區短暫的統一,例如前趙、後趙、前燕、後燕先後統一中原。只有前秦一度統一華北、華中北部與四川等地,為五胡十六國單一國家的最大範圍。

北方諸國的行政區劃大多繼承西晉,為州、郡、縣三級制。雖然各國佔地不大,但往往分置許多州,以致州境縮小。並且將自己沒有統治的州郡也常常在境內設置,例如前趙將幽州設在北地郡,後秦將冀州設在蒲坂,南燕將徐州設在莒縣等。由於一些國家採胡漢分治的制度,所以設置各種族專屬的行政區。例如前趙劉聰置左、右司隸及內史,用來統治漢人。單于左、右輔及都尉,則用來統治胡人。為求虛名,以表示境域廣大,常將境外鄰境的州增設於本國內。例如後燕設置雍州於長子(原屬并州),成漢設置荊州於巴郡(原屬梁州),南燕置并州於陰平(今江蘇沭陽北)。所以往往多個國家同時擁有同名異地的州。北魏統一華北後即整合政區。由於州境縮小,郡失去意義而逐漸廢除。

此外,北方諸國會成立僑州郡縣以安置流民,通常會依據流民原籍來定新州郡名。如前燕慕容廆立國於遼東時,他將投奔來的冀州人設冀陽郡、豫州人設成周郡、青州人設營丘郡、并州人設唐國郡。河西在西晉末已有為流民設置的郡縣,在張軌為涼州刺史時,就為秦、雍流民設置武興郡。405年,西涼李暠即為南人置會稽郡、中州人置廣夏郡。這些郡縣略同於東晉南朝的僑州郡縣,只是使用大略的地名而非流民原籍。

十六國也如同西晉一樣,設有行台制,但性質較為不同。行台是魏晉時期的機構,在戰爭發生時,中央機構尚書台派出機動的行政單位(可能只是一部分尚書台官員或只有使者之類的),代表朝廷隨軍都督。但是十六國的行台是尚書台派出單位,設置於戰略位置的地方最高軍政機構。例如後趙石勒建都於襄國,設行台於洛陽。後燕慕容垂建都中山,設行台於薊,以慕容盛為尚書左僕射錄行尚書事。北魏初期設有鄴、中山行台,皆為軍事重鎮。到南北朝時期,行台逐漸成為最高一級地方行政機構而凌駕於州郡之上。由於行台掌握地方軍政大權,減弱中央對地方的控制力,外重內輕的狀況更嚴重。

十六國時期的政治比較混亂,皇權不穩固,諸侯想要獨立,常常是強者奪位,弱者被殺。例如前趙靳準殺害皇帝劉粲奪權、後趙石虎殺皇帝石弘自立為帝等等。或是各地豪強舉兵叛變,例如前秦在淝水之戰敗給東晉後,諸族分裂獨立,前秦帝苻堅、苻丕也先後被各地叛軍殺害;後涼分裂出南涼及北涼,而北涼又分裂出西涼;胡夏叛變,屢次襲擊後秦,使得西秦得以脫離後秦復國;北魏叛後燕,襲擊攻入後燕首都,使得後燕分裂成北燕與南燕等。皇帝為了鞏固政權,往往採取卑劣和殘暴的手段來消除妨礙皇權的不安因素,甚至骨肉相殘,手足相鬥。例如石虎與其子石邃(太子)、石宣、石韜發生骨肉相殘。慕容垂被慕容評與前燕帝慕容暐排擠,最後投奔敵國前秦。直到北魏一統華北,皇權不穩的問題才告一段落。

十六國時期政治的一個特色是胡汉分治,將漢人與胡人以不同的制度作統治。以漢國(即前趙)為例,劉聰同時居皇帝(漢人的君主)和單于(胡人的首領),漢人以戶為單位設官統治,而胡人以落(指以帳篷營生的單位)為單位,設不同系統的官員來統治。另一個統治特色是,以種族、部族為中心的政軍結構。許多國家延續原本遊牧社會中,以部族和血緣為中心的體制,國家僅是各部族之間的聯盟,因此各部族領袖在軍政上有較高的權力,皇帝的君權較不能如其他朝代那樣直接透過官僚機構達成,也容易造成因宗室、部族領袖之間發生內訌而造成內戰。。前秦的苻堅和王猛即希望針對加以改革但尚未完全成功,後來北魏的拓跋珪將部落解散,設立新的公家統治機構,才逐漸減弱這種統治特色。

許多五胡的君主如劉淵、苻堅等等皆深染中國文化,所以皆採用其文化如提倡儒術、禁止烝妻報嫂等等。九品中正制也繼續使用,用來拔選世族人才,使為己用。當時世族之所以和胡族君主合作,主要為了苟全性命,許多世族輕視胡族君主文化低落。甚至有些世族,告誡子孫不可將出仕胡族的經過寫在墓碑上。石勒曾典定士族九法、慕容寶定士族舊籍貫、苻堅復魏晉士籍,皆用來承認世族權利石勒每破一州,必集中世族於「君子城」或「君子營」,下令不可欺辱之。華北動亂時,眾多人民逃往遼東,慕容皝設僑郡收留,並辨別世族清濁,後來這些世族成為前燕的基石。直到慕容氏諸燕後燕、西燕及南燕仍然繼續執行。前秦君主苻堅受謀士王猛影響,十分熱愛漢文化。他在攻滅前燕後,即聽王猛建議,重用關東世族。後來在王猛與眾士大夫經營之下,前秦國力提昇。苻堅也接受「大一統」的思想,發兵南征,但大敗。鮮卑北魏拓跋自開國之初即重用清河崔氏,大約亦採用九品官人法,至拓跋燾時期已出現了「中正官」的記載。這些都助長北方世族的發展。

東北以高句麗和慕容鮮卑最強。高句麗原受慕容鮮卑多次打擊,342年前燕慕容皝攻陷其都城丸都。在諸王勵精圖治下,於廣開土王高談德即位後,入侵新羅、百濟及夫餘等國,並與後燕的戰爭中奪得遼河流域及遼東半島。後燕帝慕容熙兩次出兵反擊,力圖奪回遼東地區,均未達到目的。436年北燕被北魏攻滅後,燕帝馮弘投奔高句麗。然而馮弘在高句麗號令如在本國,引起長壽王高璉嫌惡,最後殺之。位於幽州北方,宇文鮮卑的別支庫莫奚與契丹也開始崛起。414年庫莫奚虞出庫真率部落與北燕在營丘互市,隨後與契丹歸附北燕。北魏滅北燕後,契丹與庫莫奚也先後歸附北魏。

蒙古高原則為拓跋鮮卑(即後來的北魏)的勢力範圍,於南北朝時期為鮮卑別支柔然佔領。柔然始祖木骨閭是鮮卑拓跋部奴隸。鮮卑拓跋一部份南遷中原後,留下的部份進居陰山一帶。402年首領社侖自號「豆伐可汗」,建庭於鹿渾(今蒙古國哈爾和林西北),合併附近部落建立柔然汗國。柔然稱霸漠南漠北,在土拉河一帶打敗敕勒,也多次與拓跋部建國的北魏對戰。敕勒最早生活在勒拿河至貝加爾湖附近,又被稱為丁零、鐵勒與高車。五胡亂華後,在中原的丁零人曾建立翟魏國。487年漠北的阿伏至羅擺脫柔然統治,率10萬多人西遷,在車師地區建立高車國。著名的《敕勒歌》,是北齊時敕勒人的鮮卑語牧歌,後被翻譯成漢語。

西域方面有鄯善、龜兹、于闐、車師及疏勒等國,屬於涼州各國的勢力範圍。前涼、後涼、西涼及北涼都先後擁有部份西域地區。前涼在西域設置高昌郡和闐地縣,歸沙州刺史屬下的高昌太守管轄。還設置西域長史營、戊己校尉營及玉門大護軍營等管理西域日常事務。其他涼州國家一直延續此制度。382年前秦苻堅應車師前部王彌闐等人要求,派呂光遠征西域大宛諸國,並於西域設置都護。呂光後來建立後涼,派其子呂覆為西域大都護。420年西域歸北涼時,鄯善王比龍入朝北涼,西域各國也紛紛向北涼稱臣納貢。

吐谷渾原為鮮卑慕容部的一支,283年鮮卑單于慕容涉歸的庶長子慕容吐谷渾,因與慕容廆雙方不和,率所部西遷。313年時至隴西枹罕立國,統治今青海省、甘肅省南部、四川省西北等地的氐、羌民族。碎奚繼位時,於371年隨仇池氐王向前秦稱臣,被封為安遠將軍。後來,繼位的視連、視羆都臣服於西秦,被封為沙州牧、白蘭王。慕璝繼位時,率軍襲擊胡夏末帝赫連定,使胡夏滅亡。

十六國時期的北方諸國多實行異族分治制度,或稱為胡漢分治制度,在一國之中,實行兩種不同的軍政體制。對漢族人民,仍按漢族的傳統方式進行統治。對少數民族,則按各自的部落傳統進行統治。這使得軍事統帥被分為單于台與都督中外諸軍事並立,後來隨形勢發展漸漸合併。在軍隊形式上大致同西晉兵制,具有中軍、外軍組織及都督、將領等職務。中軍直屬中央,編為軍、營,主要保衛京師;外軍為中央直轄的各州都督所統率的軍隊。各國兵權,大多掌握在宗室手裡,任都督中外諸軍事除了前秦王猛(非宗室)外,有前趙劉宣、劉曜等人,後趙石弘、石斌,前秦苻雄、苻法等人。這本來是加強朝廷的措施。但往往變成皇位之爭而與太子自相殘殺,最後導致亡國。

各國軍隊以騎兵為主,步兵其次。各國本民族的部落兵多為騎兵。隨著攻城戰的出現以及讓漢人編列為軍隊,步兵數量也逐漸增加。如前秦南征東晉之際,即以步兵六十萬,騎兵為二十七萬,不過各國並非都改任步兵為主力。在兵役制度方面,則是實行本族全民皆兵的部落兵制,並兼有魏晉世兵制的特點。只要是凡識於戰鬥的本族人民,皆作為軍隊基本兵力。基本上中軍為終身制,其家屬通常隨營聚居,稱營戶,負責供應軍糧。鎮守各地的外軍,其隨營聚居的家屬則稱鎮戶。營戶與鎮戶都是其兵力來源。其他人民方面皆實行徵兵制,徵發各郡、縣的各族人民補充軍隊。其中漢族兵的來源,還包括來自投降的塢堡和招募的農民,一般都是終身為兵。

十六國時期各國騎兵均已強化。當時馬蹬已經十分普遍,其最大功能是可以解放雙手,騎兵開始可以靠雙腳控制平衡在馬上衝、刺、劈、擊,這大大提升騎兵的戰鬥力。馬鎧也成為騎兵較普遍的裝備,來保護戰馬免受遠射兵器攻擊。

五胡十六国前期,西晉人民为了躲避战乱,大量人口南迁,其规模之大、持续时间之长可谓史无前例。成汉的益州(四川)、前凉的河西走廊、前燕的辽河流域吸引了大量難民,成為立國的基石。河西姑臧還成為丝绸之路上的经贸外来重镇。至於留在中原地區的人民則庇護在塢堡或是部落貴族。塢堡大多由世族豪強建立,主要作為軍事防衛。世族豪強所擁有蔭戶不承擔國家賦役,僅對塢主負有義務。為了保證國庫收入和勞役來源,各族君主往往進行戶口檢查,將蔭戶復歸於編戶。

當時在中原活躍的北方民族有鮮卑、烏桓、高句麗、丁零、羯、南匈奴、匈奴別支铁弗及盧水胡、以及西部的羌、氐、巴等人。卡尔·魏特夫認為這些民族所建立國家屬於滲透王朝。這些遷入的民族與滯留北方的漢人產生「文化採借」,雙方逐漸進行文化交流與民族的融合,其中北方諸國的典章制度與禮儀法律幾乎交由漢人制定。杉山正明認為,這些游牧民族原本就存在於中原,在農耕為主的漢族居地之間活動,並逐漸定居,改成以農業生活。認為中原是漢族固有土地,這些民族是外來滲透,是中國傳統上以漢族為中心的史觀所造成。

在交流中,因為思想衝突、種族糾紛及政治鬥爭等因素,時常發生破壞、屠殺等衝突。在前秦之前,由于相互攻杀,导致游牧民族和汉族的人口大幅减少,以冉魏为例,冉闵曾经下令漢人屠杀胡羯,為殺胡令,造成二十余万胡人死亡,羯人滅族;石赵被灭时,各路百姓和游牧民族各自返回原住地,来往途中相互攻杀,加上粮食短缺,最终能够返回家乡的只有出发时的十分之三不到。

在五胡十六國中期,北方各族與漢民族彼此間展開民族與文化的融合,社会环境趋于稳定,人口开始逐渐回升。早東漢至魏晉時期,北方各族陸續內遷至中原,與漢族一同居住,但是時常受到漢官欺壓或受漢人歧視。當時北方各族即受漢文化影響。如匈奴、氐族改用漢姓並學漢語及經書。中原也流行胡族文化,對於北方民族的生活用具、服裝及音樂均感興趣,並普遍食用牛羊酪漿。當北方諸國一一滅亡之後,由於草原故鄉被柔然等新興民族佔據,而且已經適應中原文化與生活。所以這些民族絕大部分沒有退返草原,而是留在中原與漢族合為一體。民族的融合直到北周、隋朝方完成。在東晉南朝方面,中原漢人在衣冠南渡後,也和當地漢人、山越等百越诸族、及南方其他各民族發生衝突及融合。在隋朝統一中國後,南北漢人的界線逐漸模糊,融為一體。

當時黃河南北與關中地區是遭受戰禍最劇,經濟破壞最為嚴重。當時人民不是依附塢堡,成為塢主的部曲。就是遷移至各國首都附近,提供生產或兵役用。各國也會互相掠奪人民、財富以充實國力或是補給軍隊。由於人民頻繁的遷移,使得在初期難有經濟發展。

有些國家在穩定之後,開始發展經濟。例如後趙石勒在崛起過程中,大廝殺掠。但在立國後開始發展經濟,勸課農桑,頒布的稅收卻比西晉還輕,經濟逐漸復甦。但在石虎統治之後,勞役漢人,揮霍無度,經濟下滑。另外,有些國家早在開創時期就已經打下基礎,做好內政,吸引不少流民投靠。早在成漢成立之前,已有大批流民投靠巴氐李氏。李雄建立成漢後,在他統治之下「事少役稀,百姓富實」,成為最安定的地區。前燕慕容皝在統治遼東時即仿照曹魏,開放荒地讓流民種植。前涼統治的河西地區,由於相對中原較少戰亂,大量流民投奔。農業、畜牧業都有所發展。絲路也能保持暢通,使得首都姑臧成為商旅往來的樞紐,漸漸發展出「河西文化」。

前秦苻堅崇尚儒學,獎勵文教。他任漢人王猛輔政,王猛發展經濟,關中的農業、手工業和商業獲得恢復和發展。使得前秦國勢大盛,史稱「關隴清晏,百姓豐樂」,打下統一華北的基礎。前秦崩潰之後,後秦姚興注重刑罰,懲治貪污,關中經濟稍微恢復。之後西涼李暠在玉門關、陽關開墾荒地,史籍記載「年穀頻登,百姓樂業」。北燕馮跋減輕賦役,南涼禿髮烏孤注重農業,皆重視根據地的經濟發展。

邊疆各族在華北地區立國後,互相混戰。在這些國家中,以前秦(氐族)和後秦(羌族)的文化最為興盛,其次則是鮮卑慕容氏建立的前燕及後燕。此外,漢族張軌、李暠所建立的前涼和西涼,更是當時的文化中心,史稱「河西文化」。各国的统治者为了维护政权的稳定也发展教育。前赵刘曜设置太学、小学,选拔人才。前燕慕容皝设置官学,并著教材《太上章》和《典诫》。后秦、南凉设置律学,召集地方散吏入学。這促使北方各族接受漢文化,對於民族融合具有積極意義。

當時流傳下來的詩及賦不多,可能因為藝術價值不高,所以流傳不廣。至於章奏符檄,《周書‧王褒庾信傳論》認為有可觀之作,文風上接近西晉末年的風格。民歌方面,著名的大抵保存於《樂府詩集》的《梁鼓角橫吹曲》。其中有出於氐族的《企喻歌》、出於羌族的《瑯琊王歌辭》、出於鮮卑族的《慕容垂歌辭》。《晉書》的「載記」還保存一些當時的諺語,如流傳於前秦的「長鞘馬鞭擊左股,太歲南行當復虜」、「河水清復清,苻詔死新城」等。

該時期的作品以前涼和前秦的文人居多。前涼張駿著有樂府詩《薤露》、《東門行》兩首,收錄於《樂府詩集》。前涼大臣謝艾的奏疏曾被《文心雕龍》提到,他的文集可在《隋書‧經籍誌》看到。西涼李暠所著的《述志賦》載於《晉書》本傳,這篇賦表現出他建功立業的志趣和對西涼局勢的憂慮,內容頗有文采。前秦趙整著有兩首五言四句詩,用比興的手法諷諫苻堅。他還有一首琴歌《阿得脂》是雜言體,有些字句難解,大約雜用氐語。苻堅的侄子苻朗為散文家,作有《苻子》,其中有不少片斷頗具文學意味。女詩人蘇蕙的迴文詩《璇璣圖》雖然有文字遊戲的意味,但仍表現出遣詞用語的功力,成為流傳不絕的佳話。另外,後秦宗敞為王尚申辯的奏章,被呂超認為可與曹魏的陳琳、徐幹,以及西晉的潘岳、陸機相比。後秦胡義周(作者存疑。)為赫連勃勃作《統萬城銘》,獲《周書‧王褒庾信傳論》讚揚為典雅莊重。

佛教早就在東漢時期傳入中國,當時由於儒教興盛,所以沒有廣泛發展。等到十六國時期,北方動盪不安,以致人人厭苦、家家思亂。時而感到人生無常,精神缺乏寄託。此時五胡君主希望利用佛教教理的戒惡修善、六道輪迴來安撫各族百姓,並藉由屬於外來宗教的佛教來支持其政權。最後佛教得以在北方流行,並與南方佛教互相交流。至於道教,雖然在西晉就有五斗米道(天師道)的出現,但在十六國時期衰弱下來。一直到十六國末期北魏的寇謙之改革道教,才有能力與佛教抗衡。

當時從西域進入中土的僧侶,為數眾多,或譯經論,或弘教理。在佛圖澄、道安及鳩摩羅什的推廣下,為佛教奠定發揚的基石。五胡君主中,石勒、石虎、苻堅與姚興等極力支持佛教發展。苻坚的從兄之子苻朗著有佛学論書《苻子》。佛圖澄為西域僧人,他精通經文並擅長幻術。後趙的石勒、石虎奉他為「大和尚」,讓他參與軍政機要。並支持佛教發展,甚至下令不論華夷貴賤,都可以出家,開啟漢人出家之端。一时人民多营寺庙,争先出家。和佛图澄同时在后赵的,还有敦煌人单道开,襄阳羊叔子寺竺法慧和中山帛法桥等。道安為佛圖澄的弟子,在晚年備受前秦苻堅的崇敬。他致力整理和翻譯佛經,將長安經營成北方佛教的譯經中心。他於襄陽編定《綜理眾經目錄》,還為僧團制定法規,為寺院制度奠定基礎。中國出家僧人改姓「釋」,即是從道安開始。道安的弟子後來分佈各地,成為傳教的主要力量。

鳩摩羅什為西域龜茲人。382年,前秦苻堅聽從道安之建議,命大將呂光西征龜茲、迎接鳩摩羅什到長安。但後來前秦大亂,呂光隨即割據涼州,鳩摩羅什留居涼州共十七年。直到401年,後秦姚興得以迎至長安。鳩摩羅什備受姚興尊敬,待以國師之禮,入长安西明阁和逍遥园从事翻译。405年姚兴以罗什的弟子僧略为“僧正”,僧迁为“悦众”,法钦、慧斌为“僧录”,令管理僧尼的事务。鳩摩羅什主持下譯出《般若經》和大乘中觀學派的論書《中論》、《十二門論》、《百論》及《大智度論》、《法華經》等三十五部兩百多卷經典。這些皆成為後來佛學教派和宗派所依據的主要法典。其时四方的义学沙门群集长安,次第增加到三千人。

当时北方凿窟造像之风兴起,366年后秦沙门乐僔在敦煌东南鸣沙山麓,开凿石窟,镌造佛像,这就是著名的莫高窟。麥積山石窟始建於十六國的後秦,大約384年前後,當時佛教在中國開始興盛。麥積山石窟同是中國唯一保存北朝造像體系最完整的石窟,也是唯一能比較全面反映北魏至明清時期中國泥塑藝術演變歷史的石窟。后期北魏太武帝、北周武帝进行大规模的灭佛活动,对佛教的发展造成严重破坏。

民族的大融合帶來藝術文化的交流與整合,由於多元民族文化的淵源,不僅增補了固有文化停滯的不足,更可以強化文化新生發展的生機。由於佛教的興盛,帶動石窟雕像的發展。這個時期最突出的建築類型是佛寺、佛塔和石窟。

佛教的興盛帶來高層佛塔的建築以及印度、中亞一帶的雕刻、繪畫藝術。使當時的石窟、佛像、壁畫等有了巨大發展,將漢代比較樸直的風格,變得更為成熟、圓淳。位居中國四大石窟的敦煌莫高窟。和麦积山石窟,都是在十六國時期建造。

麥積山石窟始建於後秦時期(約384年前後),素有「東方雕塑陳列館」美譽。敦煌莫高窟則建於前秦時期,是世界上現存規模最大、內容最豐富的佛教藝術地,以精美的壁畫和塑像聞名於世。由於當時敦煌與西域各國交流頻繁,使得早期的莫高窟包含河西文化及西域藝術的風格。其中屬於十六國時期的275窟,繪有本生、佛傳等故事畫。這些繪畫以圈圈暈染的方式凸顯出人體特徵,並以細線勾勒,畫風豪放生動,是當時壁畫的典型風格。

書法方面,著名的作品有前涼的《李柏文書》。、前秦的《譬喻經》、西涼的《十誦比丘戒本經》和《妙法蓮華經》等。其中《李柏文書》與東晉王羲之的《姨母帖》皆保存行、楷書變遷過程,對書寫考究與風格變化有很高的參考價值。其他作品則介於書、楷之間。至於碑刻方面,著名作品有前秦的《廣武將軍碑》及《鄭太尉祠碑》、北涼的《沮渠安周造像碑》等。其字體大多在隸、楷之間,風格墣茂古拙。《沮渠安周造像碑》為沮渠安周在高昌所立,原石在新疆吐魯番高昌故城出土。《廣武將軍碑》則於前秦建元四年(368)刻。筆劃渾樸,結構拙厚,天趣渾成。書法家于右任曾作《廣武將軍歌》以推崇之。由於前秦碑文稀少,所以此碑與《鄧太尉祠碑》皆備受珍惜。

经过八王之乱和永嘉之乱後,中原殘破不堪,人民四處逃難,形成流民潮。諸國君主亦掠奪人口,以充實國力,深深破壞北方的社會結構。殘留在北方的世族,在面對險惡的環境下,有些聚集鄉民和自家的附屬人口,建立塢堡以便自守。而流民也紛紛投靠,形成人數眾多的部曲。有些則與諸國君主合作,以保本族安全。五胡君主在建國後,為了能夠統治中原地區,也需要熟悉典章制度的士大夫(世族)的協助。由於處境艱困,北方世族對同族常存抱恤的溫情,家族組織趨向大家庭制,有遠來相投的親戚,莫不極力相助。在團結力量及參與政事後,北方世族並沒有因戰亂而衰落,反而經過長期相處,使胡人融入漢人文化中。

塢堡是一個自給自足的社會組織,投奔的流民可以受塢堡保護。人民必須服從塢主命令,平時接受軍事訓練及農業生產,戰時成為保衛塢堡的戰士。人民的生產所得也須課稅給塢主。塢主除負責生產與作戰外,也要提倡教育及制定法律。由於塢堡眾多又難攻破,往往會左右戰局,使得五胡君主十分忌諱。例如祖逖北伐時,由於與當地塢堡合作,最後成功收復黃河以南領土,與石勒隔河相持。胡人君主為了解決塢堡問題,往往會與其妥協以籠絡之。到北魏宗主督護出現,塢堡的時代漸漸過去。

%% -*- coding: utf-8 -*-
%% Time-stamp: <Chen Wang: 2019-12-18 14:02:10>


\section{汉赵\tiny(304-329)}

\subsection{简介}

漢趙(304年-329年),又称前趙,是匈奴人劉渊所建的君主制割据政权,都平阳郡(今山西临汾西北),這是十六国時期建立的第一個政權。

304年,劉淵起兵,称漢王。308年称帝,国号“汉”。310年劉聰即位,311年和316年兩次攻破西晋都城洛陽、長安。318年劉曜即位,殺死靳準,次年改国号為「趙」。329年被後趙所滅,立國凡26年。其统治地区包含并州刺史部、雍州刺史部、秦州刺史部、豫州刺史部、司隶校尉部、冀州刺史部部分地区。

劉淵以自己祖先與漢朝宗室劉氏約為兄弟而自稱漢王,并自称继承汉朝,故以“汉”为国号,史稱「前汉」;以多为匈奴人,又称「胡漢」或「匈奴汉」;又统治地区位于中原北方,故称「北汉」,但此稱呼因易于與五代十国时期的北汉混淆而很少使用。劉曜以其发迹之地为战国时赵国之地,改国号为赵,为别于石勒的后赵,而史称「前趙」,或合稱之為「漢趙」。

劉淵為南匈奴單于的後裔,其父劉豹為匈奴左部帥,在五部中勢力最強。劉豹卒后,代父為左部帥。西晉有意削弱他與部落的關係,後二遷為離石將兵都尉,劉淵則利用此職位的權限,暗中擴展勢力。楊駿輔政時,為了拉攏劉淵,命他為建威將軍、五部大都督,封漢光乡侯,給予統率匈奴五部軍事的大權。到元康末年,成都王司馬穎為了擴大自己的勢力,極力拉攏劉淵,表其為「行寧朔將軍,監五部軍事」,加強劉淵在匈奴五部中的地位,並命劉淵居鄴城,以便控制。

到晉惠帝太安中(302年─303年),因河間王司馬顒、成都王司馬穎、齊王司馬冏、長沙王司馬乂等諸王相互殘殺,益州刺史部流民起義爆發,各地局勢不穩,在并州刺史部的匈奴五部右賢王劉宣等人也醞釀著反晋兴匈奴。右賢王劉宣與各部貴族商議共推劉淵為大單于,并派呼延攸告诉在邺城的刘渊,劉淵让呼延攸先回去告诉劉宣等召集各部,聲言聚集五部協助司馬穎,實際是為反晉作準備。

晋惠帝永興元年(304年)三月,司馬穎等攻占洛陽,司馬越挾持晉惠帝攻鄴,司马颖打敗司馬越,並虜獲晉惠帝。八月,司马越势力王浚、司馬騰攻鄴城,刘渊请求带领匈奴五部帮助司马颖抵御,司馬穎同意,并拜劉淵為北單于,派遣回并州刺史部的平阳郡調發匈奴五部為援。劉淵返回并州離石,眾人共推劉淵為大單于,并聚集五萬之眾。刘渊得知王浚军队已攻破邺城,司马颖南逃洛阳。刘渊还想遵守先前承诺帮助司马颖,劉宣等劝说刘渊起兵反晋。十月,劉淵從離石遷于左國城,稱漢王,改年號為元熙,置百官,大赦境內,並以復漢為名義,正式建立政權。

漢元熙元年(304年)十二月,晉并州刺史司馬騰遣兵攻漢,雙方大戰于大陵(今山西省文水北),劉淵大勝,並遣劉曜等攻取上黨、太原、西河各郡縣。當時在青、徐二州的王彌,魏郡的汲桑、石勒,上郡四部鮮卑陸逐延,氐族酋長單徵等人均擁立劉淵為共主。劉淵命王彌、石勒等人攻取河北各郡縣,並一度攻入西晉的重鎮許昌,其兵鋒進抵至西晉的首都洛陽城下。308年十月,劉淵正式稱帝,改年號為永鳳。309年,劉淵遣將攻占黎陽(今河南省浚縣東北),擊敗晉將王湛於延津(今河南省延津縣北),沉殺男女三萬人,又派遣四子劉聰進攻包圍洛陽。

310年,劉淵病重,命劉聰輔佐太子劉和。劉淵病死,劉和繼位,不久劉聰殺死劉和自立為帝。

劉聰繼位後,派遣族弟劉曜、大將王彌等率領四萬大軍攻取洛陽周邊的郡縣,以孤立斷絕洛陽。311年,石勒在苦縣(今河南鹿邑)消滅西晉主力部隊十多萬人。同年夏季,劉曜、王彌攻破洛陽,虜走晉懷帝,殺害官員百姓三萬餘人,史稱永嘉之亂。晉懷帝於次年被殺後,晉愍帝於長安即位。316年,劉聰派遣劉曜攻破長安,俘晉愍帝,西晉滅亡。隨著西晉的滅亡,中原廣大的地區,皆成為漢政權的統治範圍。

雖然劉聰名義上是中原的共主,但隨著领域的擴大,地方的割據迅速形成,漢國統治的地區實際上只有一小部分。

318年,劉聰病死,太子劉粲繼位。匈奴貴族靳準殺死劉粲奪權,在平陽的劉氏男女不分老少全部被殺,靳準自立為漢天王。鎮守長安的劉聰族弟劉曜得知平陽有變,自立為皇帝,派遣軍隊至平陽,族滅靳氏。與此同時,石勒亦以討伐靳準為名,率軍至漢都平陽,于是,平陽、洛陽以東的地區,皆落入石勒勢力之中。漢國於是遷都到長安。

319年,劉曜改國號「漢」為「趙」,史稱「前趙」或「漢趙」。同年,石勒在襄國自稱趙王,從前趙中分離出來,史稱「後趙」,双方決裂。後數年,關中地區連年叛亂及大疫,百姓死者眾多,劉曜撲滅了關中各地氐、羌人的反抗,於是遷徙上郡氐、羌二十萬人及隴西大姓楊、姜等一萬多戶到關中以充實人口。

前趙政權初步鞏固後,即向外擴張,平定隴右一帶的陳安,並向西進擊前涼。雙方在黃河沿岸僵持,張茂稱藩,並獻貢。前趙全盛时,擁兵二十八万五千餘人,據有司隶州、雍州、并州、豫州、秦州各一部,時關隴氐、羌,莫不降附。

324年,前赵军队開始向東挺进,意图夺取石勒所占的河南。325年,劉曜命劉岳率兵一萬五千人圍攻後趙石生於洛陽金墉城,石勒命從子石虎率軍救援,與劉岳在洛水西岸交戰,劉岳兵敗,退守石梁戌,石虎包圍石梁戌。劉曜率軍救援,屯兵于金谷(今河南省洛陽市西北),夜中前赵軍中哗变,士卒潰散,劉曜退歸長安。不久,石虎攻下石梁戌,生擒劉岳等人。

328年,石勒命石虎率大軍四萬從軹關(今河南省濟源市西北十五里)西攻蒲坂(今山西省永濟市蒲州鎮),劉曜親自率領水陸大軍從潼關渡河救援,石虎引兵撤退,劉曜追及並大破,石虎逃奔朝歌。劉曜取得這次大勝之後,從大陽關(今山西省平陸縣茅津渡)南渡,在洛陽金墉城圍攻石生。後趙的滎陽郡太守尹矩、野王郡太守張進等人相繼投降。这次战败震動了後趙。石勒認為洛陽一失守,劉曜必定會進攻河北,於是集結步兵六萬,騎兵二萬七千,從鞏縣渡洛水,進抵洛陽城下。

劉曜得知石勒親率大軍增援,撤走包圍金墉城的軍隊,在洛陽之西列陣十多萬軍隊,南北距離十多里。石勒率軍進入洛陽。到了決戰當天,由石虎率步兵三萬,從洛陽北方向西移動,攻擊劉曜的中軍;石堪、石聰各率騎兵八千,從洛陽西方向北移動,攻擊劉曜的前鋒。雙方大戰於洛陽西面的宣陽門外,交戰之後,石勒親自帶領主力,從西北大門出城,夾擊前趙軍,前趙軍大潰。劉曜飲酒過量,在昏醉中退走,為石堪所擒,這一仗前趙軍被斬首五萬人,主力部隊損失殆盡。

劉曜戰敗被擒,不久被殺。石勒軍乘勝西進,劉曜子劉熙、劉胤等人放棄長安,逃奔上邽(今甘肅省天水市)。329年九月,後趙出兵攻占上邽,殺趙太子劉熙及諸王公侯、將相卿校以下三千餘人,又在洛陽坑殺其王公及五郡屠各五千多人,並遷徙其百官、關東流民、秦雍大族九千多人到襄國,前趙滅亡。

在劉淵、劉聰時期,其範圍控有冀州刺史部、兖州刺史部、青州刺史部、徐州刺史部、豫州刺史部、并州刺史部、雍州刺史部、司隶校尉部、秦州刺史部一帶,然而實際控制範圍不大,劉聰時期,只局限在并州的一角(其餘部分在劉琨手中)和由劉曜坐鎮的關中一部分地區。黄河以北地区一帶由石勒所有,王彌的部將曹嶷控有青州、兗州、徐州一帶,慕容鲜卑更是趁机向南统治到幽州的一帶。

劉曜時期,史稱「東不踰太行,南不越嵩、洛,西不踰隴坻,北不出汾、晉」(引顧祖禹《讀史方輿紀要》),疆域範圍包括雍州、司隶州的渭水流域以及并州、豫州、秦州黃河以東一帶。

基本上,前趙的政治制度承襲漢魏以來的制度而又雜以舊俗。漢國的官制,自304年劉淵稱漢王建立割据的君主制政權後,即採取漢朝的官制,設丞相、御史大夫、太尉及六卿等中樞之官。軍事之官有大司馬、太尉、大將軍等高級將軍以及雜號將軍。而地方之官則沿習魏晉以來的州郡制,採用漢胡分治的政策來進行統治。大單于的權力極大,僅次於皇帝。到劉聰嘉平四年(314年),達到了較為完善的階段。而劉曜的前趙,繼承漢國之制度,小有改革。劉曜繼承君主制的前赵政權胡、汉分治的政策。以子劉胤為大司馬、大單于,置單于台于渭城(今陝西咸陽),自左、右賢王以下皆用少數族豪酋充當。另方面又大体沿用魏晉九品官人法(見九品中正制),設立學校,肯定士族特權,笼络漢人的世家大族、士族,以巩固其統治。

劉淵時,設單于台,最高長官為大單于,統率六夷部落,單于台的設置,是沿匈奴舊制而來。劉聰時,在統治區內設置左、右司隸,各領戶20多萬,每1萬戶設置一名內史,內史共有43人。在大單于下設置單于左、右輔,各主六夷十萬落,萬落叟置一名都尉。

前趙的社會經濟主要是農業,其次是畜牧業,其生產方式,沿襲漢魏以來的生產方式。

在前趙社會中,從事農業、手工業、牧業生產的還有奴隸。奴隸的來源主要是戰俘,其次是犯罪的官吏。國內還有大量從事遊牧及畜牧業的「六夷」部落,因歸降及征服的部落日益增多,故設單于台進行管理。

漢在劉聰時(310—318年),杂夷戶口大約有六十三萬戶,人口大約有三四百萬人以上;汉户未详。

在劉曜全盛時期,有兵力二十八萬五千人,在他出兵時,史稱「臨河列陣,百餘里中,鍾鼓之聲沸河動地,自古軍旅之盛未有斯比」(《晉書》.劉曜載記》)。

%% -*- coding: utf-8 -*-
%% Time-stamp: <Chen Wang: 2019-12-18 14:03:32>

\subsection{光文帝\tiny(304-310)}

\subsubsection{生平}

漢趙光文帝劉淵(249年至254年間-310年8月19日),字元海,新興匈奴人(今山西忻州市北),出身匈奴屠各部。為五胡十六國時代中,汉赵的開國君王。西晉末年八王之亂時諸王互相攻伐,南匈奴族人擁立其為大單于。304年,劉淵乘朝廷內亂而在并州自立,稱漢王,国号为漢(后改为趙,史称前漢、前趙或漢趙),5年後稱帝,改元永鳳。310年,劉淵在位六年病死,諡光文皇帝。

劉淵出身屠各族(南匈奴),是西漢冒頓單于的後代挛鞮家族的人,該家族因西漢劉邦以來,長期與漢朝王室通婚,同時兼具漢朝王室與匈奴貴族的血脈,故漢名多採取漢朝王族的劉姓為姓氏。

東漢獻帝年間,曹操統一華北地區後,重整匈奴五部,劉淵父親劉豹原是匈奴王族的左賢王,在此一時期被曹操任命為「左部元帥」;而劉淵的母親呼延氏,亦是《史記》紀載下的三大匈奴貴族姓氏之一,足見劉淵身份之高貴。

劉淵童稚時已十分聰明,七歲時母親呼延氏逝世,劉淵傷心得捶胸頓足地號叫,旁人都被其哀傷所感染,宗族部落的人都因其表現而對他十分欣賞。連當時曹魏司空王昶聽聞其行為後都讚賞他,又派人弔唁和送禮物。劉淵亦十分好學,拜崔游為師,學習《毛詩》、《京氏易》和《馬氏尚書》,劉淵尤其喜歡《春秋左氏傳》及《孫吳兵法》,《史記》、《漢書》等歷史典籍亦一一看過。同時,劉淵自以書傳中都因隨何、陸賈無武跡;周勃、灌嬰沒文才而都遭後人看不起,認為文武兼備才能獲世人欣賞,因而習武。劉淵臂力過人,善於射擊,可謂文武雙全。崔懿之、公師彧、王渾等都與他結交。

咸熙年間,劉淵到洛陽作任子,受到當時曹魏權臣司馬昭厚待。司馬炎篡魏建立西晉後,王渾向晉武帝司馬炎推薦劉淵,武帝接見劉淵後亦對他十分欣賞,更打算任命他參與平滅東吳的事,但因孔恂和楊珧以「非我族類,其心必異」為由,擔心一旦向劉淵委以重任並平滅東吳,他會在當地叛晉自立。武帝聽後才將擱置這打算。及後禿髮樹機能先後擊敗秦州刺史胡烈及涼州刺史楊欣,李熹建議任用劉淵討伐,但孔恂仍指劉淵可能會作亂涼州,武帝因而又否決了建議。當時在洛陽流浪的王彌正要回故鄉東萊,與劉淵餞別時,劉淵泣訴被人屢進讒言中傷,恐怕將會在洛陽遇害而不能再見到他。劉淵於是縱酒長嘯,同坐的都因他流淚。齊王司馬攸見劉淵後,更建議武帝殺劉淵,以免日後回匈奴五部所在的并州後會禍亂當地,但王渾反對。武帝同意王渾所言,最終沒有殺劉淵。

正巧任匈奴左部帥的父親劉豹於當時逝世,劉淵於是回到并州接替父親左部帥之位。太康末年劉淵官拜北部都尉。劉淵在當地申明刑法,禁止奸邪惡行,而且誠心與人交往,於是匈奴五部中的俊才都投歸劉淵,連幽州和冀州的名儒和寒門秀士都前來與他結交。永熙元年(290年),晉惠帝司馬衷繼位,由外戚楊駿輔政。楊駿為了拉攏遠人,樹立私恩,便任命劉淵為建威將軍、五部大都督,封漢光鄉侯。但至元康末年劉淵便因部下族人叛變出塞而免官。不久成都王司馬穎出鎮鄴城(今河北臨漳縣西南),為拉攏劉淵而表他行寧朔將軍、監五部軍事,並召他至鄴城。

當時八王之亂戰火再起,趙王司馬倫、齊王司馬冏及長沙王司馬乂先後以軍事力量上台掌權,司馬倫更曾篡位稱帝,天下大亂,盜賊蜂起。劉淵叔祖父劉宣見此,決心乘著西晉朝政混亂振興匈奴,於是秘密與族人推舉劉淵為大單于,又派遣呼延攸到鄴城通知劉淵。劉淵向司馬穎請歸不果,於是派呼延攸先回并州,命劉宣召集五部匈奴和在宜陽的一眾胡人,名為支持司馬穎,實質上卻圖謀叛變。

永安元年(304年)司馬穎擊敗司馬乂,成為皇太弟,任命劉淵為屯騎校尉。不久東海王司馬越和陳昣等與惠帝征討司馬穎,司馬穎又任命劉淵為輔國將軍、督北城守事。及至惠帝兵敗蕩陰(今河南湯陰縣)被俘至鄴城,司馬穎再任命劉淵為冠軍將軍,封"盧奴伯"。但在蕩陰之戰後不久,東嬴公司馬騰和安北將軍王淩等就起兵討伐司馬穎,劉淵趁機向司馬穎建議讓他回匈奴五部領部眾支援司馬穎,共同抵抗司馬騰和王淩的討伐部隊。司馬穎同意並拜劉淵為北單于、參丞相軍事。

劉淵回左國城(今山西吕梁市离石区)後,劉宣便為劉淵上大單于稱號,二十日之間就聚眾五萬,定都離石。及後劉淵被司馬騰盟友拓跋猗㐌和拓跋猗盧擊敗,同時司馬穎亦因受不住王淩大軍的進逼而棄守鄴城,帶惠帝逃回洛陽。劉淵在劉宣的反對下,最終決定不援救司馬穎,遷至左國城(今山西吕梁市离石区東北),又吸引數萬人歸附。

永興元年(304年)十一月,劉淵以自己祖先與漢朝宗室劉氏約為兄弟而自稱“漢王”,建國號漢,改元元熙,並追尊蜀漢後主劉禪為孝懷皇帝,又設漢高祖劉邦、漢世祖劉秀、漢昭烈帝劉備、漢文帝劉恆、漢武帝劉徹、漢宣帝劉詢、漢明帝劉莊和漢章帝劉炟等八位西漢、東漢和蜀漢皇帝的牌位;前三者為三祖,後五者為五宗,以漢室繼承者自居。同時自置百官,正式建立一個脫離西晉朝廷的獨立政權。

劉淵稱王後,身為并州刺史的司馬騰便派將軍聶玄討伐,但遭劉淵於大陵(今山西文水縣)擊敗。司馬騰知道聶玄兵敗後十分恐懼,率并州二萬多戶人南下山東地區。劉淵亦派劉曜先後攻陷太原、泫氏、屯留、長子、中都等地方,擴闊領土。次年(305年),劉淵所派將領劉欽再度擊敗司馬騰所派的討伐軍。同年并州爆發大饑荒,離石亦受影響,劉淵於是遷都黎亭。永嘉元年(307年),劉淵已攻陷并州大部份郡縣,並派兵進攻新任并州刺史劉琨。但劉琨擊敗漢軍,成功保著治所晉陽(今山西太原市)。戰後劉琨努力經營并州,更離間收降劉淵部下雜虜,漢軍向并州北部擴張的計劃因而受阻。劉淵於是聽從侍中劉殷和王育派兵進攻其他州郡,南侵進據長安(今陝西西安市未央區)和洛陽(今河南洛陽市)的建議;同時,汲桑、石勒、王彌、鮮卑陸逐延和氐酋大單于單徵數個在其他地方的軍事力量都相繼歸降劉淵,劉淵亦一一任官封爵,令漢國力量更為壯大;亦因這些加入者起事和影響的地方在冀州、徐州、青州等地,西晉受漢國侵襲的地區大大增加。永嘉二年(308年),劉淵攻破司州河東郡的蒲阪和平陽郡的平陽城(今山西臨汾市),更遷都蒲子(今山西交口縣),令兩郡屬下各縣抵抗劉淵的營壘都全部投降。同時亦派劉聰、石勒等南攻太行山、趙、魏地區。

十月甲戌日(308年11月2日),劉淵稱帝,改元永鳳。永嘉三年(309年),太史令宣于脩之認為都城蒲子所處崎嶇難以久安,建議遷都平陽。劉淵聽從並立刻遷都至平陽,改元河瑞。劉淵及後派劉聰、王彌等進攻壺關,先破劉琨所派援軍,後於長平擊敗晉東海王司馬越所派的援軍,成功攻陷壺關。劉淵於是先後於當年八月和十月派劉聰等領兵進攻洛陽,但都被晉軍擊敗,劉淵唯有撤軍。

次年劉淵病重,命太宰劉歡樂、太傅劉洋等宗室重臣入宮接受遺詔輔政。七月己卯日(8月19日),劉淵逝世,由太子劉和繼位。九月辛未日(10月20日)下葬永光陵,諡光文皇帝,廟號高祖,後改太祖。

劉淵對部眾的暴行顯得不能容忍,如一次派遣喬晞進攻西河郡,喬晞先殺不肯投降的介休縣令賈渾,後殺哭罵他的賈渾妻宗氏。劉淵知道後大怒,將喬晞追回並降秩四等,又為賈渾收葬。又將領劉景一次進攻黎陽,在延津擊敗晉將王堪後在黃河將三萬多人溺死,劉淵知道後大怒,更說:「劉景還有何顏面見朕!天道又怎能接受這種事!朕想消滅的只是司馬氏,平民有何罪!」於是貶劉景的官位。

根據《晉書》所載,劉淵膂力過人,姿儀魁偉奇特,身長超過兩米,鬍鬚長三尺有餘,其中雜有少量赤色毛髮。

\subsubsection{元熙}

\begin{longtable}{|>{\centering\scriptsize}m{2em}|>{\centering\scriptsize}m{1.3em}|>{\centering}m{8.8em}|}
  % \caption{秦王政}\
  \toprule
  \SimHei \normalsize 年数 & \SimHei \scriptsize 公元 & \SimHei 大事件 \tabularnewline
  % \midrule
  \endfirsthead
  \toprule
  \SimHei \normalsize 年数 & \SimHei \scriptsize 公元 & \SimHei 大事件 \tabularnewline
  \midrule
  \endhead
  \midrule
  元年 & 304 & \tabularnewline\hline
  二年 & 305 & \tabularnewline\hline
  三年 & 306 & \tabularnewline\hline
  四年 & 307 & \tabularnewline\hline
  五年 & 308 & \tabularnewline
  \bottomrule
\end{longtable}

\subsubsection{永凤}

\begin{longtable}{|>{\centering\scriptsize}m{2em}|>{\centering\scriptsize}m{1.3em}|>{\centering}m{8.8em}|}
  % \caption{秦王政}\
  \toprule
  \SimHei \normalsize 年数 & \SimHei \scriptsize 公元 & \SimHei 大事件 \tabularnewline
  % \midrule
  \endfirsthead
  \toprule
  \SimHei \normalsize 年数 & \SimHei \scriptsize 公元 & \SimHei 大事件 \tabularnewline
  \midrule
  \endhead
  \midrule
  元年 & 308 & \tabularnewline\hline
  二年 & 309 & \tabularnewline
  \bottomrule
\end{longtable}

\subsubsection{河瑞}

\begin{longtable}{|>{\centering\scriptsize}m{2em}|>{\centering\scriptsize}m{1.3em}|>{\centering}m{8.8em}|}
  % \caption{秦王政}\
  \toprule
  \SimHei \normalsize 年数 & \SimHei \scriptsize 公元 & \SimHei 大事件 \tabularnewline
  % \midrule
  \endfirsthead
  \toprule
  \SimHei \normalsize 年数 & \SimHei \scriptsize 公元 & \SimHei 大事件 \tabularnewline
  \midrule
  \endhead
  \midrule
  元年 & 309 & \tabularnewline\hline
  二年 & 310 & \tabularnewline
  \bottomrule
\end{longtable}


%%% Local Variables:
%%% mode: latex
%%% TeX-engine: xetex
%%% TeX-master: "../../Main"
%%% End:

%% -*- coding: utf-8 -*-
%% Time-stamp: <Chen Wang: 2021-11-01 11:51:18>

\subsection{昭武帝刘和\tiny(310-318)}

\subsubsection{戾太子生平}

刘和(?-310年),字玄泰,新兴(今山西忻州市)匈奴人。十六國時汉赵國君,光文帝劉淵長子,呼延皇后所生。劉淵死後以太子身份繼位,但即位後即試圖剷除劉聰等勢力,反被劉聰所殺。

劉和身长八尺,雄毅美姿仪,好學,從小开始学习《毛诗》、《左氏春秋》、《郑氏易》。但性格多作猜忌,對屬下無恩德。

永鳳元年(晉永嘉二年,308年),劉淵稱帝,任命劉和為大將軍。兩個月後遷大司馬,封梁王。河瑞二年(晉永嘉四年,310年)被刘渊立为太子。同年刘渊病死,刘和即位,由太宰劉歡樂、太傅劉洋等人輔政。

即位后,卫尉刘锐和劉和舅父宗正呼延攸怨恨自己不被任命為輔政大臣;侍中刘乘則厭惡握有重兵的楚王劉聰,於是共同合謀,向劉和進讒,稱諸王擁兵於都城平陽內外,其中劉聰更加擁兵十萬,嚴重影響劉和的皇權,要劉和有所行動。劉和聽信,及後召其領軍安昌王劉盛和安邑王劉欽將意圖告知。劉盛聽後勸諫劉和不要懷疑兄弟們,但遭呼延攸和劉銳命左右殺死;劉欽見此畏懼,只好對劉和唯命是從。

翌日,劉銳率馬景領兵攻劉聰,呼延攸率永安王劉安國攻齊王劉裕,劉乘則率劉欽攻魯王劉隆,劉和又派尚書田密、武衞將軍劉璿攻北海王劉乂。但田密和劉璿命人攻破城門,並歸降劉聰;而劉銳知劉聰早作準備於是聯合呼延攸等攻擊並殺害劉隆及劉裕,又因害怕劉安國和劉欽有異心而將二人殺害。此時劉聰率軍攻克西明門入宮,劉銳等在劉聰軍前鋒緊追下逃到南宮。劉和則在光極殿西室被殺,劉銳等皆被收捕並被斬首示眾。

劉聰及後自立为帝,改元“光兴”,即昭武皇帝。

\subsubsection{武帝生平}

漢昭武帝刘聪(?-318年8月31日),字玄明,新兴(今山西忻州市)匈奴人。十六国时汉赵国君。汉光文帝劉淵第四子,母张夫人。劉聰學習漢人典籍,深受漢化。執政時期先後派兵攻破洛陽和長安,俘虜並殺害晉懷帝及晉愍帝,覆滅西晉政權並拓展大片疆土。政治上创建了一套胡、汉分治的政治体制。但同時大行殺戮,又寵信宦官和靳準等人,甚至在在位晚期疏於朝政,只顧情色享樂。其執政末期甚至出現「三后並立」的情況。

劉聰年幼時就已經很聰明和好學,令到博士朱紀都覺得十分驚奇。劉聰非但通曉經史和百家之學,更熟讀《孫吳兵法》,而且善寫文章,又習書法,擅長草書和隸書;另外,劉聰亦學習武藝,擅長射箭,能張開三百斤的弓,勇猛矯捷,冠絕一時。可謂文武皆能。

劉聰二十歲後到洛陽遊歷,得到大量名士結交。後擔任新興太守郭頤的主簿。及後遷任右部都尉,因安撫接納得宜而得到匈奴五部豪族的歸心。河間王司馬顒表劉聰為赤沙中郎將,但當時劉淵在鄴城任官,因害怕駐守鄴城的成都王司馬穎加害父親,於是投奔司馬穎,任右積弩將軍,參前鋒戰事。

永安元年(304年),司馬穎任命劉淵為北單于,劉聰於是被立為右賢王,並與父親應命回到匈奴五部為司馬穎帶來匈奴援軍。但劉淵回到五部後就稱大單于,劉聰亦改拜鹿蠡王。劉淵聚眾自立,同年即稱漢王,建立漢國。後來任命劉聰為撫軍將軍。

元熙五年(晉永嘉二年,308年),劉聰被派遣南據太行山。同年年末淵稱帝,劉聰升任車騎大將軍。不久封楚王。次年與王彌和石勒等進攻壺關,擊敗司馬越派去抵抗的施融和曹超,攻破屯留和長子,令上黨太守龐淳獻壺關投降。數月後又領兵攻洛陽,擊敗平北將軍曹武,長驅直進至宜陽。但劉聰因連番勝利而輕敵,被詐降的弘農太守垣延率兵乘夜偷襲劉聰,最終劉聰大敗而還。兩月後劉聰再與王彌、劉曜、呼延翼等進攻洛陽。晉室以為漢國剛遭大敗,短時間不會再南侵,於是疏於防備,知道劉聰等來攻十分畏懼,劉聰更一度進兵至洛陽附近的洛水。當時晉將北宮純率兵夜襲漢國軍壁壘,斬殺將領呼延顥;及後呼延翼更被部下所殺,所率部隊因喪失主帥而潰退,劉淵於是下令撤兵。劉聰則上表稱晉朝軍隊又少又弱,不能因呼延翼等人之死而放棄進攻,堅持要留下來。劉淵允許。而面對漢軍,防守洛陽的司馬越唯有嬰城固守。但及後司馬越乘劉聰到嵩山祭祀的機會派兵進攻留守的漢軍,斬殺呼延朗。安陽王劉厲見此,害怕劉聰怪罪自己而跳進洛水自殺。王彌此時以洛陽守備仍堅固和糧食不繼勸劉聰撤軍,但劉聰因為是自己請求留下,不敢自行撤軍。劉淵及後聽從宣于脩之之言,命劉聰領軍撤退,劉聰見此才撤軍。

劉淵回到平陽後,任命劉聰為大司徒。河瑞二年(晉永嘉四年,310年),劉淵患病,任命劉聰為大司馬、大單于,與太宰劉歡樂和太傅劉洋共錄尚書事,並在都城平陽西置單于臺。不久劉淵逝世,由太子刘和即位。

劉和即位後,受宗正呼延攸、衞尉劉銳及素來厭惡劉聰的侍中劉乘進言唆擺,決意要消除諸王勢力,尤其當時擁兵十萬的劉聰。劉和不久就採取行動,但因劉聰有備而戰,最終劉聰率軍從西明門攻進皇宮,並於光極殿西室殺害劉和,又收捕逃到南宮的呼延攸等人,並將他們斬首示眾。

劉和死後,群臣請劉聰繼位,劉聰以其弟北海王劉乂是單皇后之子而讓位給他,但劉乂仍堅持由劉聰繼位。劉聰最終答應,並說要在劉乂長大後將皇位讓給他,登位後即立劉乂為皇太弟。

刘聪为了稳固地位,又杀死嫡兄刘恭。

劉聰即位後三個月,即派劉曜、王彌和其子河內王劉粲領兵進攻洛陽,因與石勒於大陽會師並在澠池擊敗晉將裴邈,因此直入洛川,擄掠梁、陳、汝南、潁川之間大片土地,並攻陷百多個壁壘。次年,又派前軍大將軍呼延晏領二萬七千人進攻洛陽,行軍至河南時就已十二度擊敗抵抗的晉軍,殺三萬多人。後劉曜、王彌和石勒都奉命與呼延晏會合。呼延晏在劉曜等人未到時就先行進攻洛陽城,攻陷平昌門並大肆搶掠,更於洛水焚毀晉懷帝打算出逃用的船隻。劉曜等人到達後就一起攻進洛陽城,並攻進皇宮縱兵搶掠,盡收皇宮中的宮人和珍寶,又大殺官員和宗室。另外更俘擄晉怀帝和羊皇后,將他們移送到平陽。

永嘉之亂後,劉聰又因晉牙門趙染叛晉歸降而命劉曜和劉粲攻打關中,最終攻陷長安並殺晉南陽王司馬模,並讓劉曜據守長安。但不久就被晉馮翊太守索綝、安定太守賈疋和雍州刺史麴特等反擊,劉曜等兵敗,劉曜更被圍困於長安。終於嘉平二年(永嘉六年,312年)被逼退出長安,撤回平陽。

嘉平二年(312年)年初,劉聰曾派靳沖和卜翊圍困晉并州治所晉陽,但因拓跋猗盧率兵營救而失敗。不久,令狐泥因其父令狐盛被晉并州刺史劉琨殺害而投奔漢國,並說出晉陽虛實。劉聰十分高興,便派劉粲和劉曜攻晉陽,由令狐泥作嚮導。劉琨知道漢國來攻後就到中山郡和常山郡招兵,並向拓跋猗盧求救;同時由張喬和郝詵領兵擋住漢軍。但張、郝皆敗死,劉粲於是乘劉琨未及救援而攻陷並佔領晉陽。但不久拓跋猗盧則親率大軍與劉琨反攻晉陽,劉曜兵敗,唯有棄守晉陽,撤走時遭拓跋猗盧追及,在藍谷交戰但大敗。晉陽得而復失。

晉懷帝被擄至平陽後,就被劉聰任命為特進、左光祿大夫、平阿公。後來改封會稽郡公。劉聰曾與懷帝回憶昔日與王濟造訪他的往事,亦談到西晉八王之亂,宗室相殘之事。劉聰談得十分高興,更賜小劉貴人給懷帝。但於嘉平三年(313年)正月,劉聰在與群臣的宴會中命懷帝以青衣行酒,晉朝舊臣庾珉和王儁見此忍不住心中悲憤而號哭,令劉聰十分厭惡。當時又有人流傳庾珉等會作劉琨的內應以助他攻取平陽,於是殺害懷帝和庾珉等十多名晉朝舊臣。

晉懷帝被殺的消息於四月傳至長安後,在長安的皇太子司馬鄴便即位為晉愍帝。劉聰則派趙染與劉曜和司隸校尉喬智明等進攻長安,多次擊敗抵抗的麴允。趙染後更乘夜攻進長安外城縱火搶掠,至天亮才因麴鑒救援長安而撤出長安,但麴鑒追擊時又遭劉曜擊敗。後因劉曜輕敵而被麴允偷襲,喬智明被殺,劉曜唯有撤兵回平陽。

次年,再派劉曜與趙染出兵長安,索綝領兵抵抗,但趙染初戰於新豐城西因輕敵而敗北。不久二人與將軍殷凱再攻長安,在馮翊擊敗麴允,但當晚又被麴允夜襲殷凱軍營,殷凱戰死。隨後劉曜到懷縣轉攻晉河內太守郭默,但郭默固守不降。在新鄭的李矩此時還到劉琨所派的鮮卑騎兵,說服帶領他們的張肇進攻劉曜。漢國士兵看見鮮卑騎兵就不戰而走,劉聰見進攻不成,打算先消滅劉琨,故命令劉曜撤軍。

建元元年(晉建興三年,315年),劉曜在襄垣擊敗劉琨所派軍隊,並打算進攻陽曲。但此時劉聰又認為要先攻取長安,於是命劉曜撤軍回蒲阪。

劉聰命劉曜撤回蒲阪後數月即派劉曜進攻北地,劉曜先攻破馮翊,後攻上郡,麴允雖然領兵在靈武抵抗但因兵少而不敢進攻。建元二年(316年),劉曜攻取北地,後即進逼長安。雖然有多批援兵救援長安,但都因畏懼漢國軍隊而不敢進擊。而司馬保所派將領胡崧雖然在靈臺擊敗劉曜,但卻因不願見擊退劉曜後麴允和索綝勢力變得強大,竟然勒兵退還槐里。劉曜因而得以攻佔長安外城並圍困愍帝所在小城。在爆發飢荒的小城內死守兩個月後,愍帝決定出降,被送至平陽。西晉正式滅亡。次年出獵時命愍帝穿戎服執戟作前導,被認出後有老人哭泣。劉粲勸劉聰殺愍帝但劉聰想再作觀望。及後又命愍帝行酒、洗爵和執蓋等僕役工作,令晉朝舊臣流淚哭泣,辛賓更抱著愍帝大哭。劉聰終也殺害愍帝。

劉聰自嘉平三年(晉建興二年,314年)十一月立劉粲為相國、大單于,總管各事務後,就將國事委託給他。自己則開始貪圖享樂,次年更設上皇后、左皇后和右皇后以封妃嬪,造成「三后並立」。後來更立中皇后。在委託政務給劉粲的同時,劉聰亦寵信中常侍王沈、宣懷、俞容等人,劉聰因於後宮享樂而長時間不去朝會,群臣有事都會向王沈等人報告而不是上表送呈劉聰。而王沈亦大多不報告劉聰,只以自己喜惡去議決事項。王沈等人又貶抑朝中賢良,任命奸佞小人任官。劉聰又聽信王沈等人的讒言,於建元二年(316年)二月殺特進綦毋達、太中大夫公師彧、尚書王琰等七名宦官厭惡的官員,侍中卜幹哭著勸諫但就遭劉聰免為庶人。

太宰劉易、御史大夫陳元達、金紫光祿大夫劉延和劉聰子大將軍劉敷都曾上表勸諫劉聰不要寵信宦官。但劉聰完全相信王沈等,都不聽從。劉粲與王沈等人勾結,因此向劉聰大讚王沈等人,劉聰聽後即將王沈等人封列侯。劉易見此又上表進諫,終令劉聰發怒,更親手毀壞劉易的諫書,劉易於是怨憤而死;陳元達見劉易之死,亦對劉聰失望,憤而自殺。朝廷在王沈和劉粲等人把持之下綱紀全無,而且貪污盛行,臣下只會奉承上級;對後宮妃嬪宮人的賞賜豐盛,反而在外軍隊卻資源不足。劉敷見此就曾多次勸諫,劉聰卻責罵劉敷常常在他面前哭諫,令劉敷憂憤得病,不久逝世。

因為劉聰的完全信任,王沈和劉粲等人又與靳準聯手誣稱皇太弟劉乂叛變,不但廢掉並殺害劉乂,更趁機誅除一些自己討厭的官員,又坑殺平陽城中一萬五千多名士兵。劉粲在劉乂死後被立為皇太子。

麟嘉三年(318年),刘聪患病,以太宰劉景、大司馬劉驥、太師劉顗、太傅朱紀和太保呼延晏並錄尚書事,又命范隆為守尚書令、儀同三司,靳準為大司空,二人皆決尚書奏事,以作輔政。七月癸亥日(8月31日)逝世,在位九年。諡為昭武皇帝,庙号烈宗。

據說劉聰出生時形體非常,左耳有一白毛,長逾二尺,有光澤。

劉聰雖然因眾意而登位,但仍認為自己是不依長幼次序而被擁立,於是忌憚兄長劉恭,並乘他睡覺時將他刺殺。

劉聰因單太后的絕美姿貌而與她亂倫,單太后子皇太弟劉乂曾多次規勸母親,單太后因而慚愧憤恨而死。雖然及後知道劉乂曾作規勸間接令單太后逝世,但因懷念單太后而沒有廢去其皇太弟身份。

劉聰曾濫殺大臣,如左都水使者王攄就曾因魚蟹供應不足而被劉聰殺害;將作大匠靳陵就因未能如期建成「溫明」、「徽光」二殿而被殺。王彰曾勸諫劉聰不要游獵過度,要劉聰念及劉淵建國艱難,應專心朝政。但劉聰聽後大怒,又要殺王彰,只因太后張夫人絕食以及劉乂和劉粲死諫才赦免王彰。後來設立中皇后時,尚書令王鑒和中書監崔懿之等又諫止劉聰濫封皇后,亦被劉聰所殺。

\subsubsection{光兴}

\begin{longtable}{|>{\centering\scriptsize}m{2em}|>{\centering\scriptsize}m{1.3em}|>{\centering}m{8.8em}|}
  % \caption{秦王政}\
  \toprule
  \SimHei \normalsize 年数 & \SimHei \scriptsize 公元 & \SimHei 大事件 \tabularnewline
  % \midrule
  \endfirsthead
  \toprule
  \SimHei \normalsize 年数 & \SimHei \scriptsize 公元 & \SimHei 大事件 \tabularnewline
  \midrule
  \endhead
  \midrule
  元年 & 310 & \tabularnewline\hline
  二年 & 311 & \tabularnewline
  \bottomrule
\end{longtable}

\subsubsection{嘉平}

\begin{longtable}{|>{\centering\scriptsize}m{2em}|>{\centering\scriptsize}m{1.3em}|>{\centering}m{8.8em}|}
  % \caption{秦王政}\
  \toprule
  \SimHei \normalsize 年数 & \SimHei \scriptsize 公元 & \SimHei 大事件 \tabularnewline
  % \midrule
  \endfirsthead
  \toprule
  \SimHei \normalsize 年数 & \SimHei \scriptsize 公元 & \SimHei 大事件 \tabularnewline
  \midrule
  \endhead
  \midrule
  元年 & 311 & \tabularnewline\hline
  二年 & 312 & \tabularnewline\hline
  三年 & 313 & \tabularnewline\hline
  四年 & 314 & \tabularnewline\hline
  五年 & 315 & \tabularnewline
  \bottomrule
\end{longtable}

\subsubsection{建元}

\begin{longtable}{|>{\centering\scriptsize}m{2em}|>{\centering\scriptsize}m{1.3em}|>{\centering}m{8.8em}|}
  % \caption{秦王政}\
  \toprule
  \SimHei \normalsize 年数 & \SimHei \scriptsize 公元 & \SimHei 大事件 \tabularnewline
  % \midrule
  \endfirsthead
  \toprule
  \SimHei \normalsize 年数 & \SimHei \scriptsize 公元 & \SimHei 大事件 \tabularnewline
  \midrule
  \endhead
  \midrule
  元年 & 315 & \tabularnewline\hline
  二年 & 316 & \tabularnewline
  \bottomrule
\end{longtable}

\subsubsection{麟嘉}

\begin{longtable}{|>{\centering\scriptsize}m{2em}|>{\centering\scriptsize}m{1.3em}|>{\centering}m{8.8em}|}
  % \caption{秦王政}\
  \toprule
  \SimHei \normalsize 年数 & \SimHei \scriptsize 公元 & \SimHei 大事件 \tabularnewline
  % \midrule
  \endfirsthead
  \toprule
  \SimHei \normalsize 年数 & \SimHei \scriptsize 公元 & \SimHei 大事件 \tabularnewline
  \midrule
  \endhead
  \midrule
  元年 & 316 & \tabularnewline\hline
  二年 & 317 & \tabularnewline\hline
  三年 & 318 & \tabularnewline
  \bottomrule
\end{longtable}


%%% Local Variables:
%%% mode: latex
%%% TeX-engine: xetex
%%% TeX-master: "../../Main"
%%% End:

%% -*- coding: utf-8 -*-
%% Time-stamp: <Chen Wang: 2021-11-01 11:51:31>

\subsection{隐帝刘粲\tiny(318)}

\subsubsection{生平}

汉隐帝刘粲(?-318年),字士光,新興(今山西忻州市)匈奴人,是十六国时汉赵国君。汉昭武帝刘聪子。劉粲即位後便沉醉於酒色,更與其父的四位皇后亂倫,又大殺輔政大臣,將軍國大事全交給靳準。最終令靳準成功在平陽叛亂,劉粲亦在其中被殺。

劉粲才兼文武,年輕時即為當時俊傑。光興元年(晉永嘉四年,310年)劉聰即位為帝後,封劉粲為河內王,任命為撫軍大將軍,都督中外諸軍事。

永嘉之亂後,因晉牙門趙染叛晉歸降,劉聰命趙染等進攻鎮守長安的南陽王司馬模,劉粲與劉曜則領大軍作趙染後繼。同年攻陷長安,司馬模投降並被劉粲所殺。劉粲於是與劉曜等留守關中地區。但不久,司馬模從事中郎索綝等人圖謀復興晉室,聯合一些不肯投降的郡守起兵進攻長安,並於新豐擊敗劉粲,劉粲被逼撤還首都平陽。

次年,劉聰命劉粲與劉曜領兵進攻晉并州刺史劉琨所在的并州,並成功攻陷治所晉陽。不久劉琨與拓跋猗盧領大軍反攻晉陽,於汾河以東擊敗劉曜。劉曜回晉陽後,與劉粲等擄晉陽城中平民撤退,但被拓跋猗盧追及並於藍谷大戰,最終漢軍大敗,屍橫遍野,但劉粲等人成功撤退。

嘉平四年(建興二年,314年),劉聰升劉粲為丞相、領大將軍、錄尚書事,並進封晉王。年末再升相國、大單于,總管百事。劉聰於是將朝事都交給劉粲等人,漸漸不理朝政。劉粲亦專橫放肆,親近中護軍靳準和中常侍王沈等人而疏遠朝中如陳元禮等忠良官員。性格刻薄無恩,又不聽勸諫。而且又喜好營造宮室,將相國府建得像皇宮一般華麗,國民都開始厭惡他。

及後,因宦官郭猗和靳準都與皇太弟劉乂有積怨,於是建議劉粲誣陷劉乂謀反,以讓劉粲奪去儲君的地位。劉粲聽從,於是命卜抽領兵到東宮監視劉乂。麟嘉二年(建武元年,317年),劉粲命黨羽王平向劉乂說有詔稱京師平陽將有事變,要劉乂要穿護甲在衣內以作防備。劉乂信以為真,更命東宮臣下都穿護甲衣在衣服內。消息被劉粲知道後就派人報告王沈和靳準,靳準於是向劉聰稱劉乂將謀反作亂。劉聰初時不信,但王沈等都說:「臣等早就聽聞了,但怕說出來陛下不相信而已。」劉聰於是派劉粲領兵包圍東宮。同時劉粲又派王沈和靳準收捕十多個氐族和羌族酋長並對他們審問,更加將他們吊起在高處,並用燒熱的鐵灼他們的眼,逼他們誣陷自己與劉乂串通作亂。劉聰於是認定劉乂謀反而靳準等盡忠於他,於是廢劉乂為北部王。劉粲及後就派靳準殺死劉乂。事後劉粲被立為皇太子。

麒嘉三年(318年),劉聰病逝,死前遺命太宰劉景、大司馬劉驥、太師劉顗、太傅朱紀、太保呼延晏、守尚書令范隆和大司空靳準輔政。劉粲隨後繼位。靳準心有異志,於是先打算剷除朝中劉氏勢力,於是向劉粲誣稱一眾王公大臣想行廢立之事,謀圖誅殺皇太后靳月華及自己,改以劉粲弟劉驥掌權,勸劉粲盡早行動。但劉粲不接納。靳準為了令劉粲聽從自己,於是恐嚇靳月華和皇后靳氏,稱一旦劉粲被廢,靳氏一族就會遭到誅殺。二人於是趁劉粲寵幸之機勸說劉粲,終令劉粲聽從,並殺害劉景、劉顗、劉驥、齊王劉勱和大將軍劉逞等人,朱紀和范隆則被逼出奔長安投靠劉曜。八月,劉粲於上林苑閱兵,謀圖進攻擁兵在外的石勒,又以靳準為大將軍,錄尚書事。而劉粲又繼續貪圖酒色歡樂,將軍政大權都交給靳準。而靳準亦扶植宗族勢力,命堂弟靳明為車騎將軍,靳康為衞將軍。

後來,靳準即將作亂,於是招攬年長有德而且有聲望的金紫光祿大夫王延。但王廷不肯與他一同叛亂,並立刻趕去向劉粲報告,但途中遇到靳康並被對方抓去。靳準及後便領兵入宮,在光極前殿命士兵去將劉粲抓來,盡數其罪後將他殺害。諡劉粲為隱皇帝。


\subsubsection{汉昌}

\begin{longtable}{|>{\centering\scriptsize}m{2em}|>{\centering\scriptsize}m{1.3em}|>{\centering}m{8.8em}|}
  % \caption{秦王政}\
  \toprule
  \SimHei \normalsize 年数 & \SimHei \scriptsize 公元 & \SimHei 大事件 \tabularnewline
  % \midrule
  \endfirsthead
  \toprule
  \SimHei \normalsize 年数 & \SimHei \scriptsize 公元 & \SimHei 大事件 \tabularnewline
  \midrule
  \endhead
  \midrule
  元年 & 318 & \tabularnewline
  \bottomrule
\end{longtable}


%%% Local Variables:
%%% mode: latex
%%% TeX-engine: xetex
%%% TeX-master: "../../Main"
%%% End:

%% -*- coding: utf-8 -*-
%% Time-stamp: <Chen Wang: 2019-12-18 15:53:14>

\subsection{刘曜\tiny(318-328)}

\subsubsection{生平}

刘曜(?-329年),字永明,新興(今山西忻州市)匈奴人。是十六国时汉赵(又称前趙)国君。漢趙光文帝劉淵族子。劉曜由漢趙建國開始就經已為國征戰,參與覆滅西晉的戰爭,並於西晉亡後駐鎮長安(今陝西西安市)。後於靳準之亂中登上帝位,後遷都長安。但登位後不久,將領石勒就自立後趙,國家分裂。劉曜在其在位期間多番出兵平定和招降西戎和西方的割據勢力如仇池和前涼等。在國內亦提倡漢學,設立學校。及後與後趙交戰,一度大敗後趙軍並圍攻洛陽(今河南洛陽市),但終被石勒擊敗並被俘。劉曜及後被殺,死後不久前趙亦被後趙所滅。

劉曜年幼喪父,於是由劉淵撫養。年幼聰慧,有非凡氣度。八歲時隨劉淵到西山狩獵,其間因天雨而在一棵樹下避雨,突然一下雷電令該樹震動,旁邊的人都嚇得跌倒,但劉曜卻神色自若,因而得到劉淵欣賞。劉曜喜歡看書,但志在廣泛涉獵而非精讀文句,尤其喜愛兵書,大致都熟讀。劉曜亦擅长写作和書法,習草書和隸書。另一方面劉曜亦雄健威武,箭术娴熟,能一箭射穿寸余厚的铁板,號稱神射。劉曜亦时常自比乐毅、蕭何和曹參,当时人們都不認同,唯刘聪知道其才能。

二十歲時到洛陽遊歷,但期間就被定罪而要被誅殺,於是逃亡到朝鮮,後來遇到朝廷大赦才敢回來。劉曜亦覺得自己外表異於常人,怕不被世人所接納,於是在管涔山隱居。

晉永興元年(304年),劉淵自稱漢王,國號漢,改元元熙任命劉曜為建武將軍。劉曜當年就被派往進攻并州郡縣以開拓疆土。漢永鳳元年(晉永嘉二年,308年),劉淵稱帝,拜劉曜為龍驤大將軍。後封為始安王。漢河瑞元年(晉永嘉三年,309年),劉曜與劉聰等進攻洛陽,但被晉軍乘虛擊敗。河瑞二年(310年),劉淵患病,命劉曜為征討大都督、領單于左輔。不久劉淵逝世,太子劉和繼位。劉和後又被劉聰所殺,劉聰及後登位為帝。

劉聰登位後,不久就命劉曜與河內王劉粲等進攻洛陽,並擊敗晉將裴邈,在梁、陳、汝南、潁川之間大肆搶掠。次年,劉聰命呼延晏領兵攻洛陽,劉曜奉命領兵與其會合,並於六月壬辰日(7月8日)抵達西明門。五日後,劉曜等便攻入洛陽大肆搶掠和殺害大臣,並擄晉懷帝等人,將他們送到平陽(今山西臨汾市)。史稱「永嘉之亂」。當時王彌認為洛陽城池和宮室都完好,建議劉曜向劉聰建議遷都洛陽,但劉曜認為天下未定而洛陽四面受敵,並不可守,於是焚毀洛陽宮殿。

永嘉之亂後,鎮守長安的南陽王司馬模命牙門趙染領兵在蒲阪(今山西省永濟市)守備,但趙染因請求馮翊太守一職被拒絕而投降漢國,劉聰於是在八月命趙染攻取長安,又命劉曜和劉粲領兵跟隨。司馬模兵敗投降,並於九月被劉粲所殺,劉曜則獲任命為車騎大將軍、雍州牧,並改封中山王,鎮守長安。

劉曜取得長安後,司馬模的從事中郎索綝投靠安定太守賈疋,並與賈疋等人圖謀復興晉室,於是推舉賈疋為平西將軍,率五萬兵攻向長安。當時拒降漢國的晉雍州刺史麴特等人亦領兵與賈疋會合。劉曜於是領兵在黃丘與賈疋大戰,但被擊敗。梁州刺史彭蕩仲和駐守新豐(今陝西西安市臨潼區)的劉粲都先後被賈疋等人所擊敗,彭蕩仲死而劉粲北歸平陽,賈疋等人於是聲勢大振,關西胡人和漢人都響應。劉曜只得據守長安。嘉平二年(晉永嘉六年,312年),劉曜因賈疋圍困長安經已數月,且連續戰敗,於是掠長安八萬多名平民棄守長安,逃奔平陽。劉曜及後因長安失守而被貶為龍驤大將軍,行大司馬。

同年,晉并州刺史劉琨部下令狐泥叛歸漢國,劉聰於是命令狐泥作嚮導,以劉粲和劉曜領兵進攻并州治所晉陽(今山西太原市)。二人最終乘虛攻陷晉陽,奪取劉琨的根據地。因此功績,劉曜復任車騎大將軍。但兩個月後,劉琨即與拓跋猗盧聯手反攻晉陽,劉曜在汾河以東與拓跋六脩交戰,但兵敗墜馬並受重傷,因討虜將軍傅虎協助才得以逃回晉陽。劉曜及後掠晉陽城中人民逃歸平陽,但遭拓跋猗盧追及,在藍谷交戰但慘敗。但仍成功回到平陽。

嘉平三年(晉建興元年,313年),劉曜與司隸校尉喬智明等進攻長安,但遭麴允擊敗。次年又與趙染和殷凱進攻長安,但殷凱被麴允擊殺。劉曜於是轉攻河內太守郭默但不能攻破,後更被鮮卑騎兵所嚇退。建元元年(晉建興三年,315年),劉曜一度轉戰并州,雖曾獲勝,但不久又再回到蒲坂準備再次進攻長安。及後劉曜即被派往進攻北地,先攻馮翊而再攻上郡,前去抵抗的麴允不敢進擊。

建元二年(316年),劉曜圍困並攻陷北地,並逼近長安。九月,劉曜雖被司馬保將領胡崧所敗,但胡崧並沒有進一步攻擊,反而退守槐里,而其他援軍亦因懼怕漢國軍而不敢進逼,劉曜於是成功攻陷長安外城,逼得麴允和索綝只好據守城內小城。終於在十一月,因為城內被圍困三個月而食糧嚴重困乏,晉愍帝被逼向劉曜投降。劉曜受降並於隨後遷晉愍帝和眾官員到平陽,西晉正式滅亡。劉曜因此功而獲任命以假黃鉞、大都督、督陝西諸軍事、太宰。並被改封為秦王,再度鎮守長安。

麒嘉三年(晉太興元年,318年),劉聰患病,徵召劉曜為丞相,錄尚書事,與石勒一同受遺詔輔政。但劉曜和石勒都辭讓。劉聰於是任命劉曜為丞相、領雍州牧。同年劉聰死,太子劉粲登位。八月升劉曜為相國、都督中外諸軍事,仍舊鎮守長安。但當月大將軍靳準就叛變,殺害劉粲和大殺劉氏,並自稱漢天王,向東晉稱藩。劉曜知道靳準作亂,於是進兵平陽。

十月,劉曜進佔赤壁(今山西河津縣西北赤石川),太保呼延晏等人從平陽前來歸附,並興早前因靳準誅殺王公而逃至長安的太傅朱紀等共推劉曜為帝。劉曜稱帝後,派征北將軍劉雅和鎮北將軍劉策進屯汾陰(今山西萬榮),與石勒有掎角之勢,共同討伐靳準。

靳準先前已敗於石勒,見劉曜和石勒現在共同討伐自己,於是在十一月派侍中卜泰向石勒請和,但石勒將卜泰囚禁被送交劉曜。劉曜於是向卜泰說:「先帝劉粲在位時確實亂了倫常,司空靳準你執行伊尹和霍光廢立之權,令我得以登位,實在是很大的功勳。若你早早迎接我入平陽,我就要將朝政大事都全部委託給你了,何止免死?你就為我人入城傳話吧。」於是將卜泰送返平陽。靳準聽到卜泰的傳話後,因為自知當日奪權時殺了劉曜母親胡氏和劉曜兄長,於是猶豫不決。十二月,靳康聯結喬泰和王騰等人殺死靳準,共推尚書令靳明為主,又命卜泰帶六顆傳國璽向劉曜投降。此舉令石勒十分憤怒,領兵進攻靳明,靳明大敗而只得退入平陽,嬰城固守。隨後石勒與石虎一同進攻平陽,靳明於是向劉曜求救,劉曜於是派劉雅和劉策迎接,靳明於是帶著一萬五千名平陽人民逃出平陽。劉曜及後卻大殺靳氏,一如靳準殺劉氏一樣。在其欲纳靳康女为妾时,靳女说及家族被灭,号泣请死,刘曜出于哀怜才放过了靳康的一个儿子。

石勒在靳明逃離後亦攻入平陽,留兵戍守後東歸,並於光初元年(晉太興二年,319年)年初命左長史王脩獻捷報給劉曜。劉曜於是派司徒郭汜授予他趙王和太宰、領大將軍的職位,並加如同曹操輔東漢時的特殊禮待。但留仕劉曜的王脩舍人曹平樂卻向劉曜稱王脩此行其實是要來探聽劉曜虛實,以讓石勒趁機襲擊劉曜。劉曜眼見其軍隊疲憊不堪,於是聽信曹平樂之言,追還郭汜並處斬王脩。石勒及後從逃亡回來的王脩副手劉茂口中得知王脩被殺,因此大怒,開始與劉曜交惡。

劉曜回到長安後,即遷都長安,並設立宗廟、社稷壇和祭天地的南北郊。又改國號為「趙」,史稱「前趙」。同年,石勒自稱趙王,正式建立「後趙」。漢國就此一分為二。

及後,黃石屠各人路松多在新平郡和扶風郡起兵,依附南陽王司馬保。司馬保又讓雍州刺史楊曼及扶風太守王連據守陳倉(今陝西寶雞市東),路松多據守草壁。劉曜派劉雅等人進攻但不能攻下。光初二年(320年),劉曜親自率軍進攻陳倉,擊殺王連並逼楊曼投奔氐族。接著接連攻下草壁和安定,令司馬保恐懼而遷守桑城(今甘肃临洮县东)。不久司馬保被部下張春所殺,已向劉曜投降的司馬保部將陳安則請求進攻張春等,劉曜於是任命陳安為大將軍,進攻張春。陳安最終令張春逃至枹罕(今甘肅臨夏),並殺死張春同黨楊次,消滅司馬保殘餘勢力。

不久,前趙將領解虎和長水校尉尹車與巴氐酋長句徐和庫彭等聯結,意圖謀反。但事敗露,劉曜於是誅殺解虎和尹車,並囚禁句徐和庫彭等五十多人,打算誅殺。光祿大夫游子遠極力勸阻,但劉曜都不聽,游子遠一直叩頭至流血,更惹怒劉曜而將他囚禁;劉曜後盡殺句徐等人,更在長安市內將曝屍十日,然後丟進河中。此舉終令巴氐悉數反叛,自稱大秦,並得其他少數民族共三十多萬人響應,於是關中大亂,城門都日夜緊閉。在獄中的游子遠再度上書勸諫劉曜,劉曜看後大怒,命人要立刻殺害游子遠,幸得劉雅等人勸止劉曜,游子遠才得被赦免。劉曜下令內外戒嚴,打算親自討伐叛亂首領句渠知。但此時游子遠向劉曜進言獻策,認為出兵強行鎮壓會耗費太多時間和資源,建議劉曜大赦叛民,讓他們重回正常生活,讓他們自動歸降。又請給兵五千人讓他討伐可能不肯歸降的句渠知。劉曜聽從。隨著大赦令下達,游子遠所到之都有大批人歸降,游子遠又於陰密平定不肯投降的句氏宗族黨眾。及後游子遠更進兵隴右,擊敗自號秦王的虛除權渠,並令他歸降。由於虛除權渠一部是西戎中力量最強的,故此其他西戎部族都相繼歸降前趙。

光初五年(322年),劉曜親征仇池,仇池首領楊難敵率兵迎擊但被擊敗,被逼退保仇池城。此時仇池轄下的氐羌部落大多都向前趙投降。及後劉曜轉攻楊韜,楊韜因畏懼而與隴西太守梁勛等人投降。劉曜於是再攻仇池,但此時劉曜患病,而且軍中有疫症,被逼退兵。劉曜因怕楊難敵乘機追擊,於是派光國中郎將王獷游說楊難敵,最終令楊難敵投降。劉曜於是臣服仇池,並領兵撤回長安。

此時,秦州刺史陳安請求朝見劉曜,但劉曜以患病為由推辭,陳安於是大怒,以為劉曜已死,於是決心反叛。劉曜此時病情卻愈來愈嚴重,改乘馬輿先回長安,而命呼延寔在後守護輜重。但陳安卻領騎兵邀截,俘獲呼延寔並奪取輜重,後更將呼延寔殺害。陳安又派其弟陳集等領騎兵三萬追劉曜車駕,劉曜則派呼延瑜擊殺陳集並盡俘部眾。陳安見此感到恐懼,退還上邽(今甘肃天水市),但隨後又佔領汧城,並得到隴上少數民族的歸附,於是自稱涼王。次年,陳安圍攻前趙征西將軍劉貢,但被歸附前趙的休屠王石武與劉貢的聯軍擊敗,只得收拾兵眾退保隴城(今秦安縣東北)。不久劉曜親自率軍圍困隴城,並派別軍進攻陳安根據地上邽和平襄。陳安於是出城,試圖領上邽和平襄的軍隊解圍,當知道上邽被圍而平襄被攻破後,改為南逃陝中,最終被前趙將領呼延清追及並殺害。上邽和隴城都先後投降,原本歸附陳安的隴上部落都歸降前趙。

平定陳安後,劉曜於當年即進攻前涼,親自率兵臨西河並命二十八萬兵眾沿黃河立營,延綿百多里,軍中鐘鼓之聲震動河水和大地,嚇得前涼沿河的軍旅都望風奔退。劉曜又聲言讓軍隊分百道一同渡河進攻前涼都城姑臧,令前涼震動。前涼君主張茂於是向前趙稱藩。劉曜亦達成目的,領兵退還。

光初七年(324年),後趙司州刺史石生在新安擊斬前趙河南太守尹平,並掠五千多戶東歸。自此前趙和後趙在河東、弘農之間就常有戰事。光初八年(325年),後趙將領石佗攻前趙北羌王盆句徐,大掠而歸。劉曜因而大怒,派中山王劉岳追擊,自己更移屯富平作為聲援,終大敗後趙軍並斬殺石佗。不久後趙西夷中郎將王騰以并州投降前趙。

五月,晉司州刺史李矩等因多次被後趙石生所攻,投靠前趙。劉曜於是派劉岳和呼延謨領兵與李矩等人共同進攻石生。但劉岳圍困石生於金鏞城時,被救援石生的石虎擊敗,退保石梁,更反被石虎所圍;呼延謨亦被石虎所殺。劉曜於是親自率兵救援劉岳,但及後卻因軍中夜驚而被逼退回長安。劉岳因無援而且物資缺乏,終被石虎所俘並送往後趙都城襄國(今河北邢台)。王騰亦為石虎擊敗並殺害,郭默和李矩亦被逼南歸東晉,李矩長史崔宣則向後趙投降。此戰令後趙盡得司州。

光初十一年(328年),石虎領四萬人進攻河東,獲五十多縣反叛響應,於是進攻蒲阪。因楊難敵先於光初八年(325年)反攻前趙於光初六年(323年)所佔領的仇池;又成功抵抗前趙於光初十年(327年)的攻擊。另一方面前涼於光初十年知道前趙光初八年被後趙擊敗後,即恢復其晉朝的官爵,並侵略前趙。劉曜於是派河間王劉述領氐族和羌族兵眾守備秦州以防仇池和前涼從後偷襲,自己則親率全國精銳救援蒲阪。石虎恐懼退軍,劉曜追擊並在高候大敗石虎,斬殺石曕。後劉曜又進攻石生所駐的金鏞城,以千金堨之水灌城,又派兵攻汲郡和河內,令後趙滎陽太守尹矩和野王太守張進等投降。這次大敗震動後趙人心。而劉曜此時卻不安撫士眾,只與寵臣飲酒博戲。

三個月後,石勒親率大軍救援石生,並命石堪等人在滎陽與石勒會師。劉曜在得悉石勒已渡黃河,才建議增加滎陽守戍和封鎖黃馬關以阻後趙軍。不久洛水斥候與石勒前鋒交戰,劉曜從俘獲的羯人口中得知石勒來攻的軍隊強盛才感懼怕,於是解金鏞之圍,在洛水以西佈陣。石勒則領兵進入洛陽城。

後前趙前鋒在西陽門與後趙軍大戰,劉曜親自出戰,但未出戰就已飲酒數斗;出戰後再飲酒一斗多。後趙將石堪乘其酒醉大敗趙軍,劉曜在昏醉中退走,期間墮馬重傷,被石堪俘獲。

劉曜被俘後被送往襄國,途中石勒派李永醫治劉曜。到襄國後,石勒让他住在永豐小城,給予侍姬,更命令劉岳等人去探望劉曜。石勒後來命劉曜寫信勸留守長安的太子劉熙儘快投降,但劉曜卻在信中命令劉熙和大臣們匡正和維護國家,不要因為自己而放棄。石勒看見後感到厭惡,後來劉曜還是被石勒所殺。

刘曜在霸陵西南建寿陵,侍中乔豫、和苞上疏进谏,刘曜对规谏还听得进去。但是刘曜身死国灭,他的实际墓葬地不详。

劉曜高九尺三寸(2.2米以上),垂手過膝,目有赤光,眉色發白,鬚髯雖長卻相當稀疏。劉曜自少就酗酒,及至後來就更加嚴重。在其在洛陽兵敗被俘一戰中臨陣昏醉,可謂其戰敗的其中一個原因。劉曜亦好殺,如靳準之亂中報復性盡誅靳氏和誅殺句徐等人等都可見。對大臣亦時見殺戮,差點殺了游子遠;又以毒酒殺害進言勸諫的大臣郝述和支當。

劉淵:「此吾家千里駒也,從兄為不亡矣。」

劉聰:「永明,世祖、魏武之流,何數公足道哉!」

晉書評:「曜則天資虓勇,運偶時艱,用兵則王翦之倫,好殺亦董公之亞。而承基醜類,或有可稱。」

张茂:“曜可方吕布、关羽,而云孟德不及,岂不过哉。”(《十六国春秋》)

\subsubsection{光初}

\begin{longtable}{|>{\centering\scriptsize}m{2em}|>{\centering\scriptsize}m{1.3em}|>{\centering}m{8.8em}|}
  % \caption{秦王政}\
  \toprule
  \SimHei \normalsize 年数 & \SimHei \scriptsize 公元 & \SimHei 大事件 \tabularnewline
  % \midrule
  \endfirsthead
  \toprule
  \SimHei \normalsize 年数 & \SimHei \scriptsize 公元 & \SimHei 大事件 \tabularnewline
  \midrule
  \endhead
  \midrule
  元年 & 318 & \tabularnewline\hline
  二年 & 319 & \tabularnewline\hline
  三年 & 320 & \tabularnewline\hline
  四年 & 321 & \tabularnewline\hline
  五年 & 322 & \tabularnewline\hline
  六年 & 323 & \tabularnewline\hline
  七年 & 324 & \tabularnewline\hline
  八年 & 325 & \tabularnewline\hline
  九年 & 326 & \tabularnewline\hline
  十年 & 327 & \tabularnewline\hline
  十一年 & 328 & \tabularnewline\hline
  十二年 & 329 & \tabularnewline
  \bottomrule
\end{longtable}

%%% Local Variables:
%%% mode: latex
%%% TeX-engine: xetex
%%% TeX-master: "../../Main"
%%% End:


%%% Local Variables:
%%% mode: latex
%%% TeX-engine: xetex
%%% TeX-master: "../../Main"
%%% End:

%% -*- coding: utf-8 -*-
%% Time-stamp: <Chen Wang: 2019-12-18 16:28:29>


\section{成汉\tiny(306-347)}

\subsection{简介}

成汉(304年-347年)也称成、后蜀,是中国历史上五胡十六国时期之割据政權之一。

301年益州的蜀郡的巴氐族领袖李特在蜀郡地领导西北難民反抗西晋的統治,304年其子李雄称成都王,306年李雄称帝,建国号“成”,建都蜀郡的治所成都。338年李寿改国号为“汉”。其领土疆域为益州全部。347年为东晋桓温攻破成都。

在成漢建國之前,李雄之父李特就在益州發展勢力。297年,李特率領關中流民團南下漢中。302年,李特招集流民團起兵,自稱為使持節、大都督、鎮北大將軍,第二年定年號建初元年(因此有人认为303年也可作为建国年)。率軍攻打成都,益州刺史羅尚拒守,李特敗亡,其弟李流繼續統領流民作戰,然不久後病死。李特之子李雄繼位,並於304年攻下成都,開始稱王,國號「大成」,年號建興。:159

334年,李雄病死,其兄之子李班繼位,不久之後李雄之子李期即殺李班自立。338年,李驤之子李壽又殺了李期自立為帝,將國號改為「漢」。大修宮殿,生活奢侈荒淫,人民受到嚴酷的徭役壓迫。李壽死後,其子李勢繼位,大肆殺伐,國勢更加衰弱。347年,東晉桓溫率兵入蜀,李勢投降,成漢滅亡,立國共44年,两年后残余力量也被东晋消灭。:160

其国号先为“成”,史书也有称“大成”,或以为“大”是尊称。国号“成”来自于成都这个地名,也有说是袭用公孙述的旧称(成家)。后李特弟李骧之子汉王李寿发动兵变夺取政权,改国号为“汉”,史书上又合称为“成汉”,以区别于其他称为“汉”的政权。又其统治地区主要为益州的蜀地,故又被一些史书称为“蜀”(例如《十六国春秋·蜀录》)。《晋书》又称之为“后蜀”,以别于三国时期刘备的前蜀,唐代以后,已基本不使用“后蜀”來指称,而专用作五代十国时期后蜀政权的专称。


%% -*- coding: utf-8 -*-
%% Time-stamp: <Chen Wang: 2019-12-18 15:58:30>

\subsection{李特\tiny(303)}

\subsubsection{生平}

李特(3世紀?-303年),字玄休,西晉末年巴氐人(一說為賨人),其父為李慕;十六國時期成漢國建立者李雄之父,是成漢政權的奠基者。後來李特之子李雄稱王時,追諡李特為成都景王,等到稱帝時,再追諡為景皇帝,廟號為始祖。

李特祖籍為巴西郡宕渠縣(今中國四川省渠縣),其先祖後於曹魏時被遷至略陽(今中國甘肅省秦安縣)。李特身長八尺,在兄弟間排行第二,並與其兄弟都精於騎射,以武略聞名,鄉里紛紛歸附李氏兄弟。西晉元康八年(298年),因齊萬年叛亂使得關中混亂,加上多年饑荒,李特兄弟於是與關中人民一同入蜀。原本朝廷不容許他們進入蜀地,僅讓他們留駐漢中等地,並派侍御史李苾前往慰勞並監察,不容許他們經劍閣入蜀。但因李苾受賄並上奏朝廷,故此李特和一眾中流民都得以在益州和梁州一帶居住。

永康元年(300年),益州刺史趙廞被朝廷徵召為大長秋,原職由成都內史耿滕接任。趙廞身為皇后賈南風姻親,但當年趙王司馬倫就廢黜賈南風並執掌朝政,趙廞因此害怕會因為自己與賈南風的關係而受逼害;而且趙廞亦見晉室宗室相殘,暗有割據巴蜀之意,於是決心叛晉,不旦開倉賑擠流民以收買人心,亦因李特兄弟和其黨眾都強壯勇猛,趙廞於是厚待他們並作為自己爪牙。李特等人亦恃仗趙廞的勢力,聚眾為盜,蜀人視為大患。及後趙廞擊殺耿滕,自稱大都督、大將軍、益州牧。當時李特三弟李庠率親族、黨眾及四千騎兵歸附趙廞,但趙廞因李庠通曉兵法,軍容齊整而感到不快,最終於次年(301年)殺害李庠。

趙廞雖然歸還李庠屍體給李特,並任用李特兄弟為督護以作安撫,但李特兄弟都怨恨趙廞,引兵北歸緜竹。李特後秘密地招收到七千多名兵眾,夜襲並大破趙廞所派北防晉兵的軍隊,並進攻成都(今四川省成都市)。趙廞猝不及防,逃亡被殺,李特則攻陷成都,縱兵大掠,殺趙廞屬官及任命的官員,並派牙門王角及李基向西晉朝廷陳述趙廞罪狀。

在趙廞叛變之時,朝廷另派梁州刺史羅尚入蜀任益州刺史。李特知道羅尚入蜀的消息後十分畏懼,特意派其弟李驤帶著寶物迎接,令羅尚十分高興。李流及後在緜竹為羅尚勞軍,但廣漢太守辛冉和羅尚牙門將王敦卻勸羅尚殺李流。羅尚雖未接納,但李流已經十分畏懼。

及後,朝廷命秦、雍二州召還入蜀的流民。但李特在後來才入蜀的兄長李輔口中得知中國已亂,因此不欲回到關中,於是派閻式請求羅尚,又賄賂羅尚及監督流民回州的御史馮該等,成功讓他們延遲到秋天才起行。同時,朝廷以平定趙廞之功封賞李特,拜李特為宣威將軍,封長樂鄉侯。同時下詔命州府列出當地與李特平定趙廞的流人以作封賞,但辛冉卻沒有如實上報,意圖將平定趙廞作為自己的功勳,於是招來眾人的怨恨。

七月,羅尚再催逼流民起程,然而流民都不願歸去,而且未收割穀物,未有旅費,於是深感憂慮。李特於是再派閻式請求再延遲至冬季才起行,但羅尚聽從辛冉和李苾之言,不再答允。辛冉當時又打算殺害流民首領以獲得他們的物資,於是以當日趙廞敗死時流民大掠成都為由,要在關口搜奪經過的流民的物資財寶。閻式看到這些情形,於是回到李特所駐的緜竹,並勸李特防備可能進襲的辛冉。而當時李特亦因多次為流民發聲,於是獲流民歸心和歸附。而常時李特又將辛冉懸紅捕殺李特兄弟的文告全部收下並改為求取當地李氏、任氏、閻氏等豪族和氐、叟侯王首級,於是令流民大懼,短時間內就有超過二萬人在李特麾下。李特於是特定將部眾分為兩營,分別由自己和李流統率。

不久,辛冉就派广漢都尉曾元、牙門張顯等領兵三萬進攻李特,羅尚亦派督護田佐助戰,而李特因早有準備,下令戒嚴等待曾元等到來。曾元等人到後,李特仍安然躺臥著,沒有任何動作,但當約半數軍隊進入營壘時,李特就命伏兵突擊曾元,大敗敵軍,並殺死曾元、張顯和田佐,並送首給羅尚。李特至此反叛。

當地流民於是共推李特為主,並上書請行鎮北大將軍,承制封拜。隨後便領兵進攻辛冉所在的廣漢,辛冉不敵而退奔德陽。李特在攻佔廣漢後便進攻成都。因羅尚貪婪殘暴,對比李特與蜀人約法三章,並且施捨人民,賑濟借貸,禮賢下士,拔擢人才,軍紀及施政肅然,人民都支持李特。李特屢次擊敗羅尚,羅尚唯有死守成都,並向梁州及南夷都尉李毅求救。

永寧二年(302年),平西將軍、河間王司馬顒派衙博及張微討伐李特,李毅亦派兵支援羅尚,羅尚亦派張龜進攻李特。但李特自領兵擊潰張龜,並命李蕩和李雄攻衙博,不但擊退對方,並收降了巴西郡和葭萌。同年,李特自稱為大將軍、益州牧,都督梁、益二州諸軍事。及後李特就進攻張微,但張微居高據險防守,並趁李特營壘空虛時派兵進攻李特。當時李特處於劣勢,幸李蕩援軍趕到並拼死一戰擊潰張微,才令李特脫險;及後更進攻並斬殺張微。當時羅尚繼續進攻城外李特等軍,但多次交戰皆戰敗,更令李特軍獲得大量兵器和盔甲。及後又多次擊敗梁州刺史許雄所派的軍隊。

太安二年(303年),李特擊潰羅尚駐紥在郫水的水軍,並再進攻成都,蜀郡太守徐儉於是以成都少城投降。但李特進城後僅取用馬匹作軍隊使用,並沒有進行搶掠,並且改元建初。當時蜀人聚居成各個塢自守,都款待李特,李特亦派人安撫並讓流民到各個塢內取食以節省軍糧開支。當時李流和上官惇都勸李特小心各塢都不是誠心支持自己,提防他們反叛,但李特決意安民,不去提防他們。

及後荊州刺史宗岱和建平太守孫阜率水軍救援據守成都太城的羅尚,李特派李蕩等與任臧合兵抵禦。其時宗岱等軍軍勢強盛,令各塢生有二心;同時羅尚又派益州從事任叡又詐降李特,暗中聯結塢主與羅尚一同舉兵,並假稱成都太城內糧食將盡。二月,羅尚率兵乘虛襲擊李特,各塢都響應,於是李特大敗,收兵駐守新繁。後李特見羅尚退兵,於是追擊,最終被羅尚出大軍反擊,李特及李輔和李遠都戰死,屍體被焚毀並送首級到首都洛陽。其弟李流接管其部眾。

\subsubsection{建初}

\begin{longtable}{|>{\centering\scriptsize}m{2em}|>{\centering\scriptsize}m{1.3em}|>{\centering}m{8.8em}|}
  % \caption{秦王政}\
  \toprule
  \SimHei \normalsize 年数 & \SimHei \scriptsize 公元 & \SimHei 大事件 \tabularnewline
  % \midrule
  \endfirsthead
  \toprule
  \SimHei \normalsize 年数 & \SimHei \scriptsize 公元 & \SimHei 大事件 \tabularnewline
  \midrule
  \endhead
  \midrule
  元年 & 303 & \tabularnewline\hline
  二年 & 304 & \tabularnewline
  \bottomrule
\end{longtable}


%%% Local Variables:
%%% mode: latex
%%% TeX-engine: xetex
%%% TeX-master: "../../Main"
%%% End:

%% -*- coding: utf-8 -*-
%% Time-stamp: <Chen Wang: 2019-12-18 16:24:45>

\subsection{武帝\tiny(304-334)}

\subsubsection{生平}

成武帝李雄(274年-334年),字仲儁,氐人,十六国时期成漢開國皇帝(304年至334年在位)。李特第三子,母羅氏。304年李雄自稱成都王,建年號建興。306年正式稱帝,國號大成,史稱成漢。

李雄身高八尺三寸,容貌俊美。少年時以剛烈氣概聞名,常常在鄉里間周旋,有見識的人士都很器重他。有個叫劉化的人,是道家術士,常對人說:“關、隴一帶的士人都將往南去,李家兒子中只有仲俊有非凡的儀表,終歸會成為人主的。 ”

李特在蜀地率流民起義,承皇帝旨意,任命李雄為前將軍。西晉太安二年(303年),李特被益州刺史羅尚擊殺。繼任者李流旋亦病故,李雄自稱大都督、大將軍、益州牧,住在郫城。羅尚派部將攻打李雄,李雄將其擊跑。叔父李驤攻打犍為,切斷羅尚運糧路錢,羅尚的軍隊非常缺糧,攻打得又很急,於是留下牙門羅特固守,羅尚棄城在夜晚逃走。羅特打開城門迎李雄進城,接著攻克成都。在當時李雄的軍隊非常飢餓,於是就率部眾到郪地去就食,挖掘野芋頭來吃。蜀人流亡逃散,往東下到江陽,往南進入七郡。李雄因為西山的范長生居住在山崖洞穴裡,求道養志,想要迎他來立為君而自己做他的臣子。范長生執意推辭。李雄於是盡量避讓,不敢稱制,無論大小事情,都由李國、李離兄弟決斷。李國等人事奉李雄更加恭謹。

永興元年(304年),將領們執意請李雄即尊位,於是李雄自稱成都王,赦免境內罪犯,建年號建興,廢除晉朝法律,約法七章。任命叔父李驤為太傅,兄長李始為太保,折衝將軍李離為太尉,建威將軍李雲為司徒,翊軍將軍李璜為司空,材官李國為太宰,其餘的人委任各自不同。追尊他的曾祖父李虎(即李武)為巴郡桓公,祖父李慕為隴西襄王,父親李特為成都景王,母親羅氏為王太后。范長生從西山乘坐素車來到成都,李雄在門前迎接,執版讓坐,拜為丞相,尊稱為範賢。建興三年(306年),范長生勸李雄稱帝,李雄於是即皇帝位,赦免境內罪犯,改年號為晏平,國號大成,史称成漢。追尊父親李特為景皇帝,廟號始祖,母親羅氏為太后。加授范長生為天地太師,封為西山侯,允許他的部下不參與軍事征伐,租稅全部歸入​​他的家裡。李雄當時建國初始,本來沒有法紀禮儀,將軍們仗著恩情,各自爭奪班次位置。他的尚書令閻式上疏說:“凡是治理國家製定法紀,總是以遵循舊制度為好。漢、晉舊例,只有太尉、大司馬執掌兵權,太傅、太保是父兄一樣的官,講論道義的職位,司徒、司空掌管五教九土的事情。秦代設置丞相,統掌各類政務。漢武末期,破例讓大將軍統掌政務。如今國家的基業剛剛建立,百事還沒有周全,諸公大將們的班列位次有不同,隨之競相請求設置官職,和典章舊制不相符合,應該建立制度來作為楷模法式。”李雄聽從了他的建議。

晏平二年(307年),李雄派李離、李國、李雲等率領二萬徒眾攻入漢中,梁州刺史張殷逃奔到長安。李國等人攻陷南鄭,將漢中人全部遷到蜀地。晏平四年(309年),當時李離駐鎮梓潼,他的部將訇琦、羅羕、張金苟等殺了李離和閻式,以梓潼歸降晉益州刺史羅尚。羅尚派他的部將向奮屯兵在安漢的宜福來威逼李雄,李雄率兵攻打向奮,但是不能克敵。晏平五年(310年),鎮守巴西的李國也被他帳下的文碩殺死,並以巴西投降羅尚。面對如此情況,李雄於是率眾退回,但派他的部將張寶以殺了人逃亡的名義進入了梓潼,並取得訇琦等人的信任。不久,張寶趁訇琦等人出迎羅尚使者的機會關了城門,成功重奪梓潼。正逢羅尚去世,巴郡混亂,李驤攻打涪城。玉衡元年(311年)正月,李驤攻陷涪城,擒獲梓潼太守譙登,接著乘勝進軍討伐文碩,將文碩殺死。李雄很高興,赦免境內罪犯,改年號為玉衡。

玉衡四年(314年),成漢南得漢嘉、涪陵二城,遠方的人相繼歸附,李雄於是下了有關寬大的命令,對投降依附的人都寬免他們的徭役賦稅。虛心而愛惜人才,授職任用都符合接受者的才能,益州於是安定下來。玉衡五年(315年),李雄立其妻任氏為皇后。當時氐王楊難敵兄弟被前趙劉曜打敗,逃奔葭萌,派兒子來成漢作人質。隴西賊人的統帥陳安又依附了李雄。

王衡九年(319年),李雄派李驤征伐越巂郡,於次年逼降越巂太守李釗。李驤進兵從小會攻打寧州刺史王遜,王遜讓他的部將姚岳率全部兵眾迎戰。李驤的軍隊失利,又遇上連日大雨,李驤領軍隊撤回,爭著渡過瀘水,士卒死了很多。李釗到了成都,李雄對待他非常優厚,朝廷的儀式,喪期的禮節,都由李釗決定。

楊難敵、楊堅頭兄弟因敗予前趙而逃奔葭萌時,李雄的侄兒安北將軍李稚優厚地撫慰他們,沒有送其到成都,反待前趙退兵時放他們兄弟回武都,楊難敵於是仗著天險幹了很多不守法紀的事,李稚請求討伐他。李雄不聽群臣諫言,派李稚的長兄中領軍李琀和將軍樂次、費他、李乾等從白水橋進攻下辯,征東將軍李壽督統李琀的弟弟李玝攻打陰平。楊難敵派軍隊抵禦他們,李壽不能推進,可是李琀、李稚長驅直入到達武街。楊難敵派兵切斷他們的後路,四面圍攻,俘虜李琀、李稚,死了數千人。李琀和李稚都是李雄的兄長李蕩的兒子。李雄深深痛悼他們,幾天不吃飯,說起來就流淚,深深地責備自己。

玉衡十四年(324年),李雄打算立兄李蕩之子李班為太子。李雄有十多個兒子,群臣都想立李雄親生的。李雄說:“當初起兵,好比常人舉手保護腦袋一樣,本來不希求帝王的基業。適逢天下喪亂,西晉皇室流離,群情舉兵起義,志在拯救塗炭的生靈,而各位於是推舉我,處在王公的地位之上。這一份基業的建立,功勞本來是先帝的。我兄長是嫡親血統,大柞應歸他繼承,恢弘懿美明智聰睿,就像是上天賦予了他這一使命,大事垂成,死於戰場。李班姿質性情仁厚孝順,好學素有所成,必定會成為大器。”李驤和司徒王達諫阻說:“先王樹立太子的原因,是用來防止篡位奪權的萌芽產生,不能不慎重。吳子捨棄他的兒子而立他的弟弟,所以會有專諸行刺的大禍;宋宣公不立與夷而立宋穆公,終於導致宋督的事變。說到像兒子的話,哪裡比得上真兒子呢?懇請陛下深思。”李雄不聽從,終於立了李班。李驤退下後流著淚說:“禍亂從此開始了!”

前涼文王張駿派遣使者給李雄一封信,勸他去掉皇帝尊號,向晉朝稱藩做屬臣。李雄回信說: “我以前被士大夫們推舉,卻原本無心做帝王,進一步說想成為晉室有大功的臣子,退一步說想和你一樣同為守禦邊藩的將領,掃除亂氛塵埃,以使皇帝的天下安康太平。可是晉室衰微頹敗,恩德聲譽都沒有,我引領東望,有些年月了。正好收到你的來信,在暗室獨處時體會你的真情,感慨無限。知道你想要按照古時候楚漢的舊事,尊奉楚義帝,《春秋》的大義,在這方面沒有人比得上你。”張駿很重視他的話,不斷派使者來往。巴郡曾告急,說有東面來的軍隊。李雄說:“我曾憂慮石勒飛揚跋扈,侵犯威逼琅邪,為這點耿耿於懷。沒想到竟然能夠舉兵,使人感到欣然。”李雄平時清談,有很多類似這樣的話。

李雄因為中原地區喪亡禍亂,就頻繁派遣使者朝貢,和晉穆帝分割天下。張駿統領秦梁二州,在這之前,派傅穎向成漢借道,以便向京師報送表章,李雄不答應。張駿又派治中從事張淳向成漢自稱藩屬,以此來借道。李雄很高興,對張淳說:“貴主英名蓋世,地形險要兵馬強盛,為什麼不自己在一方稱帝?”張淳說:“寡君因為先祖世代是忠良,沒能夠為天下雪恥,解眾人於倒懸,因而日頭偏西還想不起吃飯,枕戈待旦。想憑藉琅邪來中興江東,所以遠隔萬里仍然翼戴朝廷,打算成就齊桓公、晉文公一樣的事業,說什麼自取天下呢!”李雄表情慚愧,說:“我的先祖先父也是晉朝臣民,從前和六郡人避難到此,被同盟的人推舉,才有今天。琅邪如果能在中原使大晉中興,我也會率眾人助他一臂之力。”張淳回去後,向京師報送了表章,天子讚揚了他們。

當時李驤去世,李雄任命李驤的兒子李壽為大將軍、西夷校尉。玉衡二十年(330年)十月,李壽督率征南將軍費黑、征東將軍任巳攻陷巴東,太守楊謙退守建平。李壽另派費黑侵擾建平,東晉巴東監軍毌丘奧退守宜都。

玉衡二十一年(331年)七月,李壽進攻陰平、武都,氐王楊難敵投降。。玉衡二十三年(333年),李雄再派李壽進攻朱提,任命費黑、仰攀為先鋒,又派鎮南將軍任回征伐木落,分散寧州的援兵。寧州刺史尹奉投降,於是佔有南中地區。李雄在這種情況下赦免境內罪犯,派李班討伐平定寧州的夷人,任命李班為撫軍。

玉衡二十四年(334年),李雄頭上生毒瘡。六月二十五日,李雄去世,時年六十一歲,在位三十一年。諡號武皇帝,廟號太宗。葬於安都陵。

李雄的母親羅氏,夢見兩道彩虹從門口升向天空,其中一道虹中間斷開,而後生下李蕩。後來羅氏因為去打水,忽然間像是睡著了,又夢見大蛇繞在她的身上,於是有了身孕,十四個月之後才生下李雄。羅氏常常說:「我的兩個兒子如果有先死的,活著的必定有大富貴。」最終李盪死在李雄前面。

李雄的母親羅氏去世時,李雄相信巫師的話,有很多忌諱,以至於想不入葬。他的司空趙肅諫阻他,李雄才聽從了。李雄想行三年守喪之禮,群臣執意諫阻,李雄不聽。李驤對司空上官惇說:“如今正有急難還沒有消解,我想堅持諫阻,不讓主上最終守居喪之禮,你認為怎麼樣?”上官惇說:“三年的喪制,從天子直到庶人,所以孔子說:'不一定是高宗,古時候的人都是這樣。'但是漢魏以後,天下多難,宗廟是最重要的,不能長時間無人管理,所以不行衰絰一類的禮,盡哀就罷了。”李驤說:“任回將要到來,這個人在處事方面很有決斷,而且主上常常很難不聽他的話,等他到了,就和他一起去請求。”任回抵達後,李驤和任回一同去見李雄。李驤脫去冠流著淚,一再請求因公除去喪服。李雄大哭不答應。任回跪著上前說:“如今王業剛剛開始建立,各種事情都在草創階段,一天沒有主上,天下人心惶惶。從前周武王披著素甲檢閱軍隊,晉襄公繫著墨絰出征,難道是他們希望做的嗎?是為了天下人而委屈自己的原故呀!希望陛下割捨親情順從權宜的方法,以使國運永遠興隆。”於是強行扶李雄起來,脫去喪服親理政事。


\subsubsection{建兴}

\begin{longtable}{|>{\centering\scriptsize}m{2em}|>{\centering\scriptsize}m{1.3em}|>{\centering}m{8.8em}|}
  % \caption{秦王政}\
  \toprule
  \SimHei \normalsize 年数 & \SimHei \scriptsize 公元 & \SimHei 大事件 \tabularnewline
  % \midrule
  \endfirsthead
  \toprule
  \SimHei \normalsize 年数 & \SimHei \scriptsize 公元 & \SimHei 大事件 \tabularnewline
  \midrule
  \endhead
  \midrule
  元年 & 304 & \tabularnewline\hline
  二年 & 305 & \tabularnewline\hline
  三年 & 306 & \tabularnewline
  \bottomrule
\end{longtable}

\subsubsection{晏平}

\begin{longtable}{|>{\centering\scriptsize}m{2em}|>{\centering\scriptsize}m{1.3em}|>{\centering}m{8.8em}|}
  % \caption{秦王政}\
  \toprule
  \SimHei \normalsize 年数 & \SimHei \scriptsize 公元 & \SimHei 大事件 \tabularnewline
  % \midrule
  \endfirsthead
  \toprule
  \SimHei \normalsize 年数 & \SimHei \scriptsize 公元 & \SimHei 大事件 \tabularnewline
  \midrule
  \endhead
  \midrule
  元年 & 306 & \tabularnewline\hline
  二年 & 307 & \tabularnewline\hline
  三年 & 308 & \tabularnewline\hline
  四年 & 309 & \tabularnewline\hline
  五年 & 310 & \tabularnewline
  \bottomrule
\end{longtable}

\subsubsection{玉衡}

\begin{longtable}{|>{\centering\scriptsize}m{2em}|>{\centering\scriptsize}m{1.3em}|>{\centering}m{8.8em}|}
  % \caption{秦王政}\
  \toprule
  \SimHei \normalsize 年数 & \SimHei \scriptsize 公元 & \SimHei 大事件 \tabularnewline
  % \midrule
  \endfirsthead
  \toprule
  \SimHei \normalsize 年数 & \SimHei \scriptsize 公元 & \SimHei 大事件 \tabularnewline
  \midrule
  \endhead
  \midrule
  元年 & 311 & \tabularnewline\hline
  二年 & 312 & \tabularnewline\hline
  三年 & 313 & \tabularnewline\hline
  四年 & 314 & \tabularnewline\hline
  五年 & 315 & \tabularnewline\hline
  六年 & 316 & \tabularnewline\hline
  七年 & 317 & \tabularnewline\hline
  八年 & 318 & \tabularnewline\hline
  九年 & 319 & \tabularnewline\hline
  十年 & 320 & \tabularnewline\hline
  十一年 & 321 & \tabularnewline\hline
  十二年 & 322 & \tabularnewline\hline
  十三年 & 323 & \tabularnewline\hline
  十四年 & 324 & \tabularnewline\hline
  十五年 & 325 & \tabularnewline\hline
  十六年 & 326 & \tabularnewline\hline
  十七年 & 327 & \tabularnewline\hline
  十八年 & 328 & \tabularnewline\hline
  十九年 & 329 & \tabularnewline\hline
  二十年 & 330 & \tabularnewline\hline
  二一年 & 331 & \tabularnewline\hline
  二二年 & 332 & \tabularnewline\hline
  二三年 & 333 & \tabularnewline\hline
  二四年 & 334 & \tabularnewline
  \bottomrule
\end{longtable}


%%% Local Variables:
%%% mode: latex
%%% TeX-engine: xetex
%%% TeX-master: "../../Main"
%%% End:

%% -*- coding: utf-8 -*-
%% Time-stamp: <Chen Wang: 2019-12-18 16:30:33>

\subsection{幽公\tiny(334-338)}

\subsubsection{哀帝生平}

成哀帝李班(288年-334年),字世文。十六国时期成汉政权的皇帝。为李雄之兄李荡之子。

李班初任平南將軍。李班的叔父李雄雖然有十個兒子,但都不成氣候,所以李雄捨棄自己的兒子而立李班為太子。

李班為人謙虛能廣泛採納意見,尊敬愛護儒士賢人,從何點、李釗以下,李班皆以老師的禮節對待他們,又接納名士王嘏和隴西人董融、天水人文夔等作為賓客朋友。常常對董融等人說:“看到周景王的太子晉、曹魏的太子曹丕、東吳的太子孫登,文章審察辨識的能力,超然出群,自己總是感到慚愧。怎麼古代的賢人那樣高明,而後人就是望塵莫及呀!”李班為人性情博愛,行為符合軌範法度。當時李氏的子弟都崇尚奢侈靡費,可是李班常常自省自勉。每當朝廷上有重大問題要討論,叔父李雄總是讓他參與。李班認為:“古時候開墾的田地平均分配,不論貧富可以一樣獲得土地,如今顯貴人物佔有大面積的荒田,貧苦人想耕種卻沒有土地,佔地多的人將自己多餘的土地出售給他們,這哪裡是王者使天下均等的大義呀!”李雄採納了他的意見。

玉衡二十四年(334年),李雄臥病不起,李班日夜侍奉在身邊。李雄年輕時頻頻作戰,受了很多傷,到這時病重,疤痕全部化膿潰爛,李雄的兒子李越等人都因厭惡而遠遠躲開。李班替他吸吮膿汁。毫無為難的表情,往往在嘗藥時流淚,不脫衣冠地服侍,他的孝心誠意大多如此。

同年六月二十五日,李雄去世,李班即位。任命堂叔建寧王李壽為錄尚書事,來輔佐朝政。李班在宮中依禮服喪,政事都委託給李壽和司徒何點、尚書令王瑰等人。當時李越鎮守江陽,因為李班不是父親李雄的兒子,心中很是不滿。同年九月,李越回到成都奔喪,和他的弟弟安東將軍李期密謀除掉李班。李班的兄弟李玝勸李班遣送李越回江陽,任命李期為梁州刺史,鎮守葭萌。李班認為李雄還未下葬,不忍心讓他們走,推誠待人而心地仁厚,沒有一點嫌隙。當時有兩道白氣出現在天空中,太史令韓豹奏道:“宮中有秘密陰謀的殺氣,要對親戚加以戒備。”李班沒有明白。十月,李班因為夜晚去哭靈,李越在殯宮殺了李班,時年四十七歲,李班共在位一年,於是群臣立李雄的兒子李期繼位。

\subsubsection{幽公生平}

李期(314年-338年),字世運,是十六国時期成汉政权的皇帝。为李雄第四子。

李期聰慧好學,二十歲時就能作文章,輕財物而好施捨,虛心招納人才。初任建威將軍,其父李雄讓兒子們和宗室的子弟們各自憑恩德信義聚集徒眾,多的不到數百人,可是李期單單招到了上千人。他推薦的人,李雄多半任用,所以長史、列署有不少出自他的門下。

玉衡二十四年(334年),李雄死,太子李班繼位。李雄之子李越回成都奔喪時與李期殺掉李班。

因李期多才多兿、并由皇后任氏(李雄正妻)養大而被擁立,即位改元玉恆。李期即位後,首先誅殺李班的弟弟李都。派堂叔李壽到涪城討伐李都之弟李玝,李玝棄城投降東晉。李期封李壽為漢王,任命他為梁州刺史、東羌校尉、中護軍、錄尚書事;封兄長李越為建寧王,任命為相國、大將軍、錄尚書事。立妻子閻氏為皇后。任命衛將軍尹奉為右丞相、驃騎將軍、尚書令,王瑰為司徒。李期自認為圖謀大事已經成功,不重視各位舊臣,在外則信任尚書令景騫、尚書姚華、田褒。田褒沒有別的才能,李雄在位時期,曾勸其立李期為太子,所以李期非常寵幸厚待他。對內則相信宦官許涪等人。國家的刑獄政事,很少讓卿相過問,獎賞和刑罰,都由這幾個人決定,於是國家的法紀紊亂。竟然誣陷尚書僕射、武陵公李載謀反,致使李載被下獄而死。在此之前,東晉建威將軍司馬勛屯兵漢中,李期派李壽攻陷漢中,於是設置守官,設防於南鄭。

李雄的兒子李霸、李保都無病而死,都說是李期毒死了他們,於是大臣們心懷恐懼,人人不能心安。李期誅殺夷滅了很多人家,抄沒他們的婦女和財物來充實自己的後庭,宮內宮外人心惶惶,路上相見也只敢用目光打招呼,勸諫的人都定了罪,人人只想苟且免禍。李期又毒死李壽的養弟安北將軍李攸,和李越、景騫、田褒、姚華商議襲擊李壽等人,打算燒毀市橋而發兵。李期又多次派中常侍許涪到李壽那裡去,察看他的動靜。

李攸死後,李壽非常害怕,又疑心許涪往來頻繁的情況。於玉恒四年(338年),率領一萬步兵、騎兵,從涪城出發前往成都,聲稱景騫、田褒擾亂朝政,所以發動晉陽兵士,以清除李期身邊的惡人。李壽到達成都,李期、李越沒料到他會來,一向不加防備,李壽於是佔領成都,駐兵到宮門前。李期派侍中慰勞李壽,李壽上奏章說李越、景騫,田褒、姚華、許涪、征西將軍李遐、將軍李西等人都心懷奸詐擾亂朝政,圖謀傾覆社稷,大逆不道,罪該誅殺。李期順從了李壽的意見,於是殺死李越、景騫等人。李壽假託太后任氏的名義下令,將李期廢為“邛都縣公”,幽禁在別宮內。李期嘆息說天下的君主竟然成了一個小小的縣公,真是生不如死。同年(338年),李期自缢而死,時年25歲,諡號幽公。

\subsubsection{玉恒}

\begin{longtable}{|>{\centering\scriptsize}m{2em}|>{\centering\scriptsize}m{1.3em}|>{\centering}m{8.8em}|}
  % \caption{秦王政}\
  \toprule
  \SimHei \normalsize 年数 & \SimHei \scriptsize 公元 & \SimHei 大事件 \tabularnewline
  % \midrule
  \endfirsthead
  \toprule
  \SimHei \normalsize 年数 & \SimHei \scriptsize 公元 & \SimHei 大事件 \tabularnewline
  \midrule
  \endhead
  \midrule
  元年 & 335 & \tabularnewline\hline
  二年 & 336 & \tabularnewline\hline
  三年 & 337 & \tabularnewline\hline
  四年 & 338 & \tabularnewline
  \bottomrule
\end{longtable}


%%% Local Variables:
%%% mode: latex
%%% TeX-engine: xetex
%%% TeX-master: "../../Main"
%%% End:

%% -*- coding: utf-8 -*-
%% Time-stamp: <Chen Wang: 2019-12-18 16:31:47>

\subsection{昭文帝\tiny(338-343)}

\subsubsection{生平}

汉昭文帝李寿(300年-343年),字武考,十六国时期成汉政权的皇帝。为李特之弟李骧少子。

338年即位后改国号为“汉”。343年病死。

李壽天生聰敏好學,少尚禮容,在李氏諸子中相當突出,受到李雄欣賞,認為他足以擔當大任,乃授以前將軍,統領巴蜀軍事,彼遷征東將軍,當時年僅十九歲。在任期間以處士譙秀為謀主,對其言聽計從,令他在巴蜀威德日隆。

李驤卒,李壽再先後升遷為大將軍、大都督、侍中、並封扶風公、錄尚書事。在出征寧州時,圍攻百餘日,最終悉數平定諸郡,李雄大悅,再加封建寧王。

李雄卒,受遺命輔政,李期繼位,改封漢王,兼任梁州刺史,獲賜封梁州五郡。

自此,李壽威名遠播,卻同時深為李越、景騫等所忌憚,令李壽深為擔憂。在暫代李玝屯田涪水期間,每次朝覲日期到來,往趁以邊景賊寇橫行,不可放鬆戒備離開而推卻。同時李壽又因為李期、李越兄弟等十餘人年紀漸長,又擁有精兵,擔心不能自全,便數次欲聘得龔壯為其效命。龔壯雖然不答應,不過仍多次與李壽見面。時值岷山崩塌,江水因此枯竭,李壽認為此乃上天預示災劫,因而非常厭惡,便問龔壯自安之法。龔壯的父親及叔父,被李特殺害,為了假借李壽之手報仇,便向李壽提出起兵自立以自保的建議,最終獲得李壽採納,之後便暗中與長史略陽羅恆、巴西解思明共同謀奪首都成都,並得數千人加入。李壽軍起兵突襲成都,將其攻克,縱兵擄掠,甚至姦污李雄女兒及李氏諸婦,並將之殘殺。羅恆、解思明、李奕、王利等人乃勸李壽自稱鎮西將軍、益州牧、成都王,並向晉朝稱藩。

成玉恆四年(338年),大臣任調、司馬蔡興、侍中李艷以及張烈等勸李壽自立。李壽命巫師卜卦,得出「可當數年天子」的預示,任調大喜,進言「一日尚且滿足,何況數年!」(一日尚為足,而況數年乎!)李壽以「有道是「早上聽到警世的道理,就算當晚要死亦無悔無憾」(朝聞道,夕死可矣),任調的進言,實在是上乘之策!」乃自稱為帝,舉國大赦,並改元漢興,以董皎為相國、羅恆、馬當為股肱,李奕、任調、李閎為親信,解思明為謀主。李壽本想向龔壯,授安車束帛以命為大師,然而龔壯拒絕,僅接收縞巾素帶,以師友之位自居。同時拔擢幽滯,授以顯位。並追尊李驤為獻帝、母昝氏為太后、妻阎氏為皇后、世子李勢為太子。

李壽稱帝後,有人狀告廣漢太守李乾與大臣串通,密謀廢帝。李壽命兒子李廣與大臣齊集殿前,將李乾徙為漢嘉太守。一次遇上狂風暴雨,震動大履門柱,李壽為此深自悔責,下命郡臣要盡忠進言,切切拘泥忌諱。

後來後趙石虎向李壽提議結盟出兵晉朝,事成後兩人並分天下,李壽大悅,先是大修船艦,嚴兵繕甲,又令吏卒準備充足糧草。繼而以尚書令馬當為六軍都督,準備以七萬人兵力,乘舟溯江而上。當船隊經過成都時,鼓聲震天,李壽登城檢閱,群臣趁機以國小眾寡,吳越、會稽路遠,不易成功為由出言阻止,尤其解思明更是切誎懇至,於是李壽便讓群臣力陳利害。龔壯誎曰﹕「陞下與胡人互通,是否會比與晉朝更好呢? 胡人素來是豺狼一樣的國家。晉朝被滅,才不得不北面事之。如果與他們爭奪天下,結果只會令強弱更加懸殊,昔日虞國、虢國(成語「假途滅虢」的典故)的教訓在前,希望陛下可以深思熟慮。」群臣都同意龔壯的進言,更叩頭泣誎,終使李壽放棄,士眾大喜,更連聲萬歲。

之後李壽派遣鎮東將軍李奕征討牂柯,太守謝恕據城堅守多日未能攻克,適逢李奕糧盡,因而撤兵。同時以太子李勢為大將軍、錄尚書事。

李壽繼承李雄,同樣為政寬儉,在篡位之初,亦未表現其欲望。某次李閎、王嘏從鄴城歸來,盛讚石季龍的宮殿華麗,鄴中戶口殷實。卻同時聽聞石季龍濫用刑法,王侯表現不遜,亦以殺罰懲戒,反而能夠控制各地邦域,令李壽相當羨慕,決心效法他。下臣每有小過,動輒處死以立威。又以都城空虛、鄉效戶口未至充實、工匠器械仍未滿盈為由,遷徙鄰郡戶有三名男丁以上的家戶到成都,又建造尚方御府,派遣各州郡能工巧匠以充實之,並廣修宮室、引水入城,極盡奢華,又擴充太學、建立宴殿等,令百姓疲於奔命,悲呼嗟嘆怨聲載道,以致人心思亂者,竟有十之八九。左僕射蔡興進誎阻止,李壽以其散播謗言為由,將他處死。右僕射李嶷因為經常直言忤逆意旨,李壽對他素有積怨,便假以他罪將他收監然後處死。

後來李壽患有重病,經常夢見李期、蔡興索命。解思明等復議再次尊奉皇室,李壽不從。李演亦由越雟上書,勸他歸正返本,放棄稱帝,復稱為王,李壽大怒殺之,以警告龔壯、解思明等。龔壯於是作詩七篇,假借應璩之口諷刺李壽,李壽便回應道﹕「有道是「反省詩詞便可知其意思」,如果這篇是今人所作,就是賢哲之話語; 假若是古人所作,便只是死去鬼魂的平常辭令!」

漢興六年(343年),最終在憂患之中病死,享年四十四歲。李壽在位五年,谥昭文皇帝,廟號中宗,葬安昌陵。

李壽為帝之初,好學愛士,即使庶民小兒也對他稱道不已。每次閱到良將賢相建功立業的事蹟時,沒有一次不會反覆誦讀,故能征伐四克,開闢千里疆土。未稱帝之前,相對李雄一心求上,李壽亦能盡誠於下,因此被稱為賢相。到即位之後,改立宗廟,以父李驤為漢始祖廟、李特、李雄為大成廟,又下旨强調與李期、李越並非同族,大凡期、越時定制,都有所改動。公卿之下,悉數任用自己的幕僚輔佐。李雄時的舊臣以及六郡士人,全部罷黜。而李壽相當仰慕漢武帝、魏明帝的所為,同時恥於聞說父兄之事,禁止上書者妄言前任教化功績,而只能提及李壽在位時的當世事功。

\subsubsection{汉兴}

\begin{longtable}{|>{\centering\scriptsize}m{2em}|>{\centering\scriptsize}m{1.3em}|>{\centering}m{8.8em}|}
  % \caption{秦王政}\
  \toprule
  \SimHei \normalsize 年数 & \SimHei \scriptsize 公元 & \SimHei 大事件 \tabularnewline
  % \midrule
  \endfirsthead
  \toprule
  \SimHei \normalsize 年数 & \SimHei \scriptsize 公元 & \SimHei 大事件 \tabularnewline
  \midrule
  \endhead
  \midrule
  元年 & 338 & \tabularnewline\hline
  二年 & 339 & \tabularnewline\hline
  三年 & 340 & \tabularnewline\hline
  四年 & 341 & \tabularnewline\hline
  五年 & 342 & \tabularnewline\hline
  六年 & 343 & \tabularnewline
  \bottomrule
\end{longtable}


%%% Local Variables:
%%% mode: latex
%%% TeX-engine: xetex
%%% TeX-master: "../../Main"
%%% End:

%% -*- coding: utf-8 -*-
%% Time-stamp: <Chen Wang: 2019-12-18 16:34:50>

\subsection{李势\tiny(343-347)}

\subsubsection{生平}

李势(4世紀-361年),字子仁,十六国成汉末主。李寿长子,母李氏。降晉後,封歸義侯,卒於建康,後世稱「後主」。無子。

初,李壽妻閻氏無子,李驤殺李鳳,為李壽納李鳳之女為妻,生李勢。李期愛李勢姿貌,拜他為翊軍將軍、漢王世子。李勢身長七尺九寸,腰帶十四圍,善於俯仰,時人異之。成漢漢興六年(343年),李壽死,李勢嗣偽位,赦其境內,改元曰太和。尊母閻氏為太后,妻李氏為皇后。

太史令韓皓奏熒惑守心,以過廟禮廢,勢命群臣議之。其相國董皎、侍中王嘏等以為景武昌業,獻文承基,至親不遠,無宜疏絕。勢更令祭特、雄,同號曰漢王。

李勢弟大將軍、漢王李廣以李勢無子,求為太弟,李勢不許。馬當、解思明以李勢兄弟不多,若有所廢,則益孤危,固勸李勢准許。李勢疑當等與李廣有叛謀,遣其太保李奕襲廣於涪城,命董皎收馬當、解思明二人斬殺,夷其三族。貶李廣為臨邛侯,李廣自殺。解思明有計謀,強作諫諍,馬當甚得人心。自此之後,無複紀綱及諫諍者。

李奕自晉壽舉兵反之,蜀人多有從奕者,眾至數萬。勢登城距戰。奕單騎突門,門者射而殺之,眾乃潰散。勢既誅奕,大赦境內,改年嘉寧。

初,蜀土無獠,至此,始從山而出,北至犍為,梓潼,布在山谷,十余萬落,不可禁制,大為百姓之患。勢既驕吝,而性愛財色,常殺人而取其妻,荒淫不恤國事。夷獠叛亂,軍守離缺,境宇日蹙。加之荒儉,性多忌害,誅殘大臣,刑獄濫加,人懷危懼。斥外父祖臣佐,親任左右小人,群小因行威福。又常居內,少見公卿。史官屢陳災譴,乃加董皎太師,以名位優之,實欲與分災眚。

晉永和二年(346年)末,晉大司馬桓溫率水軍伐勢。桓溫次青衣,李勢大發軍距守,又遣李福與昝堅等數千人從山陽趣合水距溫。謂溫從步道而上,諸將皆欲設伏於江南以待王師,昝堅不從,率諸軍從江北鴛鴦碕渡向犍為,而桓溫從山陽出江南,昝堅到犍為,方知與溫異道,乃回從沙頭津北渡。及昝堅至,溫已造成都之十裏陌,昝堅之兵眾自潰。桓溫至城下,縱火燒其大城諸門。李勢的兵眾惶懼,無複固志,其中書監王嘏、散騎常侍常璩等勸李勢投降。

李勢以問侍中馮孚,馮孚言:「昔吳漢征蜀,盡誅公孫氏。今晉下書,不赦諸李,雖降,恐無全理。」勢乃夜出東門,與昝堅走至晉壽(今四川广元),然後送降文于溫曰:「偽嘉寧二年三月十七日,略陽李勢叩頭死罪。伏惟大將軍節下,先人播流,恃險因釁,竊自汶、蜀。勢以暗弱,複統未緒,偷安荏苒,未能改圖。猥煩硃軒,踐冒險阻。將士狂愚,干犯天威。仰慚俯愧,精魂飛散,甘受斧鑕,以釁軍鼓。伏惟大晉,天網恢弘,澤及四海,恩過陽日。逼迫倉卒,自投草野。即日到白水城,謹遣私署散騎常侍王幼奉箋以聞,並敕州郡投戈釋杖。窮池之魚,待命漏刻。」勢尋輿櫬面縛軍門,溫解其縛,焚其櫬,遷勢及弟福、從兄權親族十余人于建康,封勢歸義侯。升平五年(361年),卒於建康。在位五年而敗。

《妒记》記載晉時宣武候桓溫平蜀後,娶了李勢的妹妹為妾。桓溫妻南康公主知道後大為妒忌,乃拔刃往李氏居所,準備砍人。結果看見李氏正在窗前梳頭,「姿貌端丽,徐徐结发,敛手向主,神色闲正,辞甚凄惋。」見美而生憐生愛,於是擲刀前抱之曰:「阿子,我見汝亦憐,何況老奴。」意思是說連我看了都會心動了,更何況是那老傢伙。成語「我見猶憐」因此而來。

\subsubsection{太和}

\begin{longtable}{|>{\centering\scriptsize}m{2em}|>{\centering\scriptsize}m{1.3em}|>{\centering}m{8.8em}|}
  % \caption{秦王政}\
  \toprule
  \SimHei \normalsize 年数 & \SimHei \scriptsize 公元 & \SimHei 大事件 \tabularnewline
  % \midrule
  \endfirsthead
  \toprule
  \SimHei \normalsize 年数 & \SimHei \scriptsize 公元 & \SimHei 大事件 \tabularnewline
  \midrule
  \endhead
  \midrule
  元年 & 344 & \tabularnewline\hline
  二年 & 345 & \tabularnewline\hline
  三年 & 346 & \tabularnewline
  \bottomrule
\end{longtable}

\subsubsection{嘉宁}

\begin{longtable}{|>{\centering\scriptsize}m{2em}|>{\centering\scriptsize}m{1.3em}|>{\centering}m{8.8em}|}
  % \caption{秦王政}\
  \toprule
  \SimHei \normalsize 年数 & \SimHei \scriptsize 公元 & \SimHei 大事件 \tabularnewline
  % \midrule
  \endfirsthead
  \toprule
  \SimHei \normalsize 年数 & \SimHei \scriptsize 公元 & \SimHei 大事件 \tabularnewline
  \midrule
  \endhead
  \midrule
  元年 & 346 & \tabularnewline\hline
  二年 & 347 & \tabularnewline
  \bottomrule
\end{longtable}

\subsubsection{范賁生平}

范賁(3世紀?-349年),中國十六國初期東晉境內蜀地(今中國四川省)的民變領袖之一,是成漢丞相范長生之子。曾任成漢的侍中一職,318年范長生去世後,接任丞相。

范長生博學多聞,年近百歲才去世,而被蜀地之人敬若神明。347年,成漢被東晉所滅,成漢將領因此推范賁為帝,根據史書記載,范賁「以妖異惑眾」,因此蜀地很多人歸附。

349年,東晉益州刺史周撫、龍驤將軍朱燾攻擊范賁,范賁被殺,遂平定益州。


%%% Local Variables:
%%% mode: latex
%%% TeX-engine: xetex
%%% TeX-master: "../../Main"
%%% End:


%%% Local Variables:
%%% mode: latex
%%% TeX-engine: xetex
%%% TeX-master: "../../Main"
%%% End:

%% -*- coding: utf-8 -*-
%% Time-stamp: <Chen Wang: 2019-12-18 17:03:15>


\section{前凉\tiny(301-376)}

\subsection{简介}

前凉(320年-376年)是十六国政权之一。都姑臧(今甘肃武威)。 301年,凉州大姓汉人张轨被晋朝封为凉州刺史,313年封西平公,課農桑、立學校,多所建樹。又鑄五銖錢,全境通行。314年张轨病死,其子张\xpinyin*{寔}袭位。西晋灭亡后,仍然据守凉州,使用司马邺(晉愍帝)的建興年號,成为割据政权。

320年,张茂改元永元,前凉遂彻底成为独立政权。

345年,张寔子张骏称凉王,都姑臧,以所在地凉州为国号“凉”,史称“前凉”,以别于其他以“凉”为国号的政权。張駿、張重華父子統治時期,前涼極盛。353年張重華病死,宗室內亂不止,國勢大衰。

前涼極盛之時,统治范围包括甘肃、宁夏西部以及新疆大部。史載“南逾河、湟,東至秦、隴,西包蔥嶺,北暨居延”。張天錫時已失去甘肅南部。

376年,前秦天王苻堅以十三萬步騎大舉進攻,張天錫投降,前涼滅亡。

\subsection{武王生平}

張軌(255年-314年),字士彥,安定郡烏氏縣(今甘肅平涼市西北)人。西漢常山王張耳的十七世孫。晉朝時任涼州牧,是前涼政權奠定者,張寔、張茂皆為其子。314年去世,晉諡曰武公。至其曾孫張祚時,被追諡為武王,廟號太祖。

張軌其家世孝廉,以儒學著稱。張軌年少時已聰明好學,甚有名望,曾隱居於宜陽郡的女几山上。西晉建立後入朝任官,因與中書監張華議論經籍意義和政事而深得對方的器重。張軌歷任太子舍人、尚書郎、太子洗馬、太子中庶子、散騎常侍,征西將軍司馬。

晉惠帝元康元年(291年),「八王之亂」開始,天下大亂,張軌於是想佔據河西之地(今甘肅西部、新疆東部一帶),於是就要求調任涼州。在朝中官員的支持之下,張軌於永寧元年(301年)被任命為護羌校尉、涼州刺史。張軌到任後,使立刻領兵擊敗當時在涼州叛亂的鮮卑族,又消滅橫行當地的盜賊,斬首萬多人,從此威震西土,亦安定了涼州。張軌任用有才幹的涼州大姓如宋配、陰充、氾瑗和陰澹為股肱謀主,共同治理涼州。他又勸農桑,立學校,又設與州別駕同等的崇文祭酒、春秋行鄉射之禮,在涼州大行教化。

永興二年(305年),鮮卑若羅拔能侵襲涼州,張軌派司馬宋配討伐,最終斬殺若羅拔能,並俘據十多萬人,因而聲名大振。晉惠帝亦因此加張軌安西將軍,封安樂鄉侯,邑千戶。同時又大修涼州治所姑臧(今甘肅武威市)。此時,東羌校尉韓稚殺害秦州刺史張輔,張軌少府司馬楊胤主張討伐韓稚,亦勸張軌效法齊桓公主持地方,對韓稚擅殺刺史的行為予以嚴懲。張軌於是命中督護領二萬兵討伐,並先寫信給韓稚勸降。韓稚拉到書信後就向張軌投降。張軌報告南陽王司馬模後,司馬模十分高興,並將皇帝賜的劍送給張軌,並將隴西地區交給張軌管理。

張軌始終對西晉表示忠誠,以維繫民心。如太安三年(304年)河間王司馬顒和成都王司馬穎到洛陽討伐掌權的司馬乂,張軌亦曾派三千兵支援朝廷。永嘉二年(308年),劉淵部將王彌進攻洛陽,張軌派北宮純、張纂、馬魴和陰濬等領兵入衛洛陽,北宮純及後派百多名勇士突擊王彌軍,協助朝廷擊退王彌。不久北宮純在河東擊敗劉淵兒子劉聰,晉懷帝於是詔封張軌為西平郡公,但張辭讓。西晉自八王之亂起,天下大亂,各州都不再向西晉朝廷賦貢,亦惟有張軌貢獻不絕。

永嘉二年(308年),張軌因患風搐而不能說話,命兒子張茂代管涼州。張越是涼州大族,聽說有預言說張氏會興盛涼州,自以為自己就是預言中的張氏,於是不惜放下梁州刺史的職務告病回涼州,更與兄長酒泉太守張鎮等人合謀要除去張軌。張越兄弟意圖以秦州刺史賈龕取代張軌,於是派密使到洛陽請尚書侍郎曹祛任西平太守,作為援助。張軌別駕麴晁亦意圖借機弄權,派使者到長安告訴司馬模,說張軌已病得不再能繼續行使刺史職權,要求以賈龕代替張軌。賈龕原打算應命,但被兄長勸止。

張鎮和曹祛知道賈龕拒絕應命後,再上表請求新派刺史,但未上呈就已率先以軍司杜耽代領州事,讓杜耽支持並表張越為新任刺史。張軌見此,打算退避,想要回到曾經隱居的宜陽,但長史王融和參軍孟暢接到張鎮等人以杜耽代理涼州的檄命後並不服氣,決意支持張軌,於是領兵戒嚴,又命剛從洛陽回來的張軌長子張寔為中督護,領兵討伐張鎮。同時又派張鎮甥子令狐亞遊說張鎮。最終張鎮聽從,哭著說受了誤導,將事情都推給功曹魯連,更將魯連殺死向張寔請罪。張寔及後攻打曹祛,曹祛逃走。在王融舉兵同時,武威太守張琠亦派兒子張坦到洛陽上表支持張軌;而治中楊澹亦到長安向司馬模控訴張軌被誣,令司馬模上表停止選調新任刺史。張坦到洛陽後,晉懷帝慰勞張軌,又下令誅殺曹祛。張軌知道後十分高興,又命張寔領兵三萬討伐曹祛,最終將曹祛擊敗並殺死。

張軌及後命治中張閬送五千義兵和大量物資到洛陽。永嘉五年(311年)光祿大夫傅祗和太常摯虞及後寫信給張軌說洛陽物資缺乏,張軌又立刻派參軍杜勵進獻五百匹馬和氈布三萬匹。晉懷帝於是進拜張軌為鎮西將軍、都督隴右諸軍事,封霸城侯,並進車騎將軍、開府儀同三司。但使者還未到,王彌就再次進逼洛陽,張軌派將軍張斐、北宮純和郭敷等率五千名精銳騎兵保衛洛陽,但洛陽最終都被漢國大將劉曜攻克。

永嘉之亂後,洛陽和長安兩大重鎮都先後被漢國軍隊攻陷,中原和關中地區的很多百姓流入涼州避難,張軌在姑臧西北置武興郡;又分西平郡(今青海西寧市)界置晉興郡以收容流民。同時,張軌亦繼續支持西晉,晉懷帝被擄到平陽後,張軌曾打算傾一州之力進攻平陽。不久秦王司馬鄴入關,張軌又派兵支持。次年司馬鄴被擁立為皇太子,張軌獲拜驃騎大將軍、儀同三司,張軌辭讓。同時張軌又協助消滅在附近地區叛亂的勢力,如秦州刺史裴苞、西平郡的麴恪、鞠儒等。司馬鄴及後再度任命,但張軌亦再次辭讓。

永嘉七年(313年),晉懷帝被殺,司馬鄴繼位為晉愍帝,並升張軌為司空,張軌再辭讓。同時又聽從索輔的建議,復鑄五銖錢,恢復境內的錢幣流通,大大便利了當地人的生活,不必再以布匹作貨幣。同時,劉曜進逼長安,張軌又派參軍麴陶領三千兵入衛長安。

建興二年(314年),晉愍帝任命張軌為侍中、太尉、涼州牧,封西平公,但張軌仍然辭讓。五月己丑日,張軌病死,享年六十歲,諡曰武公。張軌的親信部下及後擁立張軌長子張寔繼任了涼州牧之職。

张轨墓在今凉州区境内,史称“建陵”,前凉国主的陵墓位置,学术界有三种推测:陵墓上方筑台,可能在今灵钧台、雷台等古台下;或依照汉制,重臣死后多陪葬君主墓旁,根据已出土的“梁舒墓”的方位,可能在今武威城西北太平滩一带;或因山为陵,可能在武威城南祁连山山坡地带。

\subsection{昭王生平}

张寔(271年-320年),字安遜,安定烏氏人。十六国时期前凉政权的君主。为张轨长子。張寔任內保持與晉廷關係,也支持晉元帝即位為帝,但仍一直在割據狀態,即使西晉亡後仍然用晉愍帝「建興」年號。

張寔高學識,觀察入微,而且敬重並愛惜有才德的人,獲舉為秀才,授命為郎中。永嘉初年,張寔辭讓驍騎將軍,並請求回當時由父親任刺史的涼州。朝廷准許張寔所求,遂改授張寔議郎。不過,張寔回到涼州治所姑臧(今甘肅武威市)時正遇上涼州大姓張鎮、張越兄弟與曹祛等人圖謀逼走父親,奪取涼州控制權的行動。張軌長史王融及參軍孟暢支持張軌,決意作出反擊,於是讓張寔為中督護,率兵討伐張鎮。張鎮恐懼並委罪予功曹魯連,將其處死後便向張寔請罪。張寔隨後攻伐曹祛,曹祛逃走。時前往洛陽為張軌陳情的張坦帶著晉懷帝慰問張軌和誅殺曹祛的詔命回來,張軌於是命張寔率尹員、宋配等領三萬步騎兵攻曹祛。曹祛派麴晃到黃阪抵禦,張寔就用計騙了麴晃,令自己得以進至浩亹(今青海海東市樂都區東),並戰於破羌(今青海海東市樂都區西)。曹祛等終為張軌所殺。戰後張寔獲封建武亭侯。

不久,張寔遷西中郎將,封福祿縣侯。晉愍帝即位後,又以西中郎將領護羌校尉。建興二年(314年),張軌去世,長史張璽等人表張寔代行張軌官位。晉愍帝及後下詔授予張寔持節都督涼州諸軍事、西中郎將、涼州刺史、領護羌校尉,封西平公。張寔接掌涼州後鼓勵諫言,當面進諫的賞布帛;書面進諫的賞竹器;在坊間論政的賞羊和米。另又聽從賊曹佐隗瑾所言設立諫官,處理大小事務時都與部屬們討論,廣納眾言,從而鼓勵吏民進言。

建興四年(316年),前趙將領劉曜率軍逼近長安(今陝西西安),張寔派王該率兵救援,晉愍帝於是加授張寔都督陝西諸軍事。同年晉愍帝被圍困被逼投降,降前下詔進張寔為大都督、涼州牧、侍中、司空,承制行事。張寔受詔後以愍帝被俘為由辭讓。及後晉愍帝遇害的消息傳至涼州,南陽王司馬保卻圖謀稱帝,張寔則支持在江東的司馬睿,並在建興六年(318年)派牙門蔡忠上表勸進。同年,司馬睿即位為帝,即晉元帝。

建興七年(319年),司馬保自稱晉王,置百官並改年號,又以張寔為征西大將軍、儀同三司,增食邑三千戶。但不久司馬保就因部將陳安叛變而陷險境,張寔先後派兵協助。次年(320年),司馬保因劉曜逼近而遷至桑城(今甘中肅臨洮县附近),並意圖到涼州避難,但張寔考慮到司馬保宗室的身份,若果到涼州肯定會對當地人心有所影響,於是派將領陰監派兵迎接司馬保,聲稱是保衞,其實是想阻止他前來。不久,司馬保被其部將張春所殺,餘眾離散並有萬多人逃至涼州,張寔至此自恃涼州險遠,頗為驕傲放縱。

當時天梯山上有一個叫劉弘的人,因為法術而有上千信眾,連張寔身邊的人都有其信眾。當時帳下閻涉及牙門趙卬都是劉弘同鄉,而劉弘向閻涉說:「上天賜我神璽,要我治理涼州。」二人深信不疑,於是秘密聯結張寔身邊十多人,意圖行刺張寔,奉劉弘為主。張寔已經從張茂口中得知這圖謀,於是派了牙門將史初收捕劉弘,但閻涉等人不知道,依計眾人懷刀而入,在外寢殺死張寔。劉弘見史初來,還說:「使君已經死了,還殺我做甚麼!」史初憤怒,割了他的舌頭然後囚禁,及後張茂更施車裂之刑,閻涉及其黨羽數百人亦被誅殺。享年五十歲。私諡為昭公,晉元帝則赐諡號元公。張祚稱帝時以張寔為昭王。

張寔死後,因兒子張駿年幼,由弟弟張茂繼位。墓葬称“宁陵”,亦在今凉州区境内。


\subsection{成王生平}

张茂(277年-324年),字成遜,安定烏氏人。中國十六国时期前凉政权的君主,为昭王张寔之同母弟。張茂任內前涼遭前趙出兵威壓,被逼向前趙稱藩,並接受其官爵。

永嘉二年(308年)張軌患病不能說話,時兄長張寔仍在朝中,故張茂就代父管理涼州事務。建興元年(312年),南陽王司馬保曾經請張茂作自己的從事中郎,後又推薦他任散騎侍郎、中壘將軍,但張茂都不應命。次年,張茂被徵召為侍中,但張茂以父親年老為由推辭。不久,改拜平西將軍、秦州刺史。

建興八年(320年),張寔被部下所殺,因其子駿年幼,张茂就代攝其位,殺劉弘數百名同黨。時涼州府推舉張茂為大都督、太尉、涼州牧,但張茂不肯受,只以使持節平西將軍、涼州牧職位,又以張寔子張駿為撫軍將軍、武威太守、西平公。。

建興十年(322年),張茂派韓璞率兵佔領隴西南安郡境,並在當地置秦州。建興十一年(323年),前趙劉曜派部將劉咸攻冀城,呼延晏攻桑壁,時臨洮人翟楷及石琮等又驅逐其地方官員響應劉曜,劉曜本人更发兵二十八万五千人沿黃河列陣百多里,張茂設在黃河沿岸守戍的軍隊望風奔逃,劉曜更聲言面率大軍渡河,直攻姑臧,遂震動河西。張茂聽從參軍馬岌所言出屯姑臧城東的石頭,在聽參軍陳珍分析劉曜其實不會盡力攻涼後,便派陳珍出兵救援在冀城的韓璞。劉曜亦自知其強大的兵力有三分之二是因為人們怯於其聲威而來,主力軍隊已經相當疲累,難以渡河進攻,於是一直按兵不動,想用聲威脅服張茂。張茂最終派使者向前趙稱藩,進獻大量物品,劉則授張茂使持節、假黃鉞、侍中、都督涼南北秦梁益巴漢隴右西域雜夷匈奴諸軍事、太師、領大司馬、涼州牧、領西域大都護、護氐羌校尉。封涼王。

建興十二年(324年)張茂病死,享年四十八歲。前涼私下為張茂上諡號成公,刘曜則遣使赠太宰,諡成烈王。後來张祚稱帝,追尊張茂為成王,庙号太宗。张茂临终时交代张骏“谨守人臣之节,无或失坠”。因張茂無子,張駿就被推舉繼位。

張茂為人謙虛恭敬,且又好學,不因為利祿而動心,又有志向及節操,有決定大事的能力。當時有一個人叫賈摹,不但出身涼州大姓,他也是張寔妻子的弟弟,勢力很大。曾經有一首童謠這樣說:「手莫頭,圖涼州。」張茂以此誘殺賈摹,終令涼州豪門大族不敢橫行,更有助前涼張氏對涼州的管治。


\subsection{文王生平}

张骏(307年-346年),字公庭,安定烏氏人。中國十六国时期前凉政权的君主,在位二十二年。为前凉明王张寔之子,前凉成王张茂之侄。張駿任內前涼國力提升,也乘前趙滅亡而盡得河南(甘肃地区黄河以南)隴西之地,又進攻西域。張駿亦先後接受後趙及東晉的官位,在位晚期亦建設起天子規格器物、儀式及官職架構。

建興四年(316年),張駿受封霸城侯。建興八年(320年),張寔去世,涼州州府推舉其叔張茂繼位,張茂於是以張駿為撫軍將軍、武威太守,襲爵西平公。建興十二年(324年),張駿在張茂死後繼位,並暗示時滯留在姑臧的晉愍帝使者史淑以晉廷名義授予自己使持節大都督、大將軍、涼州牧、領護羌校尉,封西平公。時前趙皇帝劉曜也授張駿大將軍、涼州牧,封涼王。

張駿繼位時,守枹罕的涼州將領辛晏據城反對張駿,不服其統治。張駿打算討伐但為從事劉慶勸止,而翌年辛晏也向張駿投降,收服了河南之地。建興十五年(327年),張駿聽聞前趙軍隊敗於後趙,於是除去前趙所授的官爵,用回晉朝的官爵,並派兵進攻前趙秦州。可時前涼軍敗於前趙南陽王劉胤所率的軍隊,劉胤更乘勝渡過黃河,攻陷令居並殺二萬多人,又進佔振武,震動河西,張駿派皇甫該前往防禦。金城太守張閬及枹罕護軍辛晏都向前趙投降,河南之地復失。建興十七年(329年),前趙亡於後趙,張駿於次年就乘機重奪河南地,進軍至狄道,置武街、石門、侯和、漒川及甘松五屯護軍。不久,後趙派了鴻臚孟毅授予張駿征西大將軍、涼州牧,但張駿恥於為後趙之臣,不接受並留下孟毅。但不久就因畏懼後趙強大而向其稱臣,送還孟毅。東晉朝廷也進張駿鎮西大將軍,仍授涼州刺史、領護羌校尉並封西平公,詔命於建興二十一年(333年)到達前涼,張駿接受任命,派王豐等人陳謝並上疏稱臣,但仍然用建興年號,不用東晉年號。次年東晉又進張駿為大將軍,此後晉涼每年都有使者往來。

張駿又曾派將領楊宣率兵進攻龜茲及鄯善,終令西域諸國都歸附前涼,焉耆前部王及于寘王都派人進貢。也曾上疏晉廷請求配合司空郗鑒及征西將軍庾亮進行北伐。

西域長史李柏敗於不肯服從張駿的戊己校尉趙貞,有人認為是李柏自設計謀導致失敗,請張駿誅殺他,然而張駿終免李柏一死,更得眾人歡心。張駿也改易原本犯下死罪者的親屬不得留在朝中的律令,只是限制他們不能參與宿衞,於是令前涼刑法清明,國家富強,群僚更於建興二十年(332年)勸張駿稱「涼王」,自領秦涼二州牧,置公卿百官。雖然張駿嚴詞拒絕,但其實前涼境內都用涼王去稱呼張駿。後張駿更努力改變自己,勤於庶政,統掌涼州文武事務,治績不錯,得四方稱頌,叫他做「積賢君」。而涼州自晉末以來連年都有戰事,至張駿在位時漸見平穩安定。建興二十七年(339年),張駿又設辟雍、明堂以行禮教。

張駿攻西域後,在涼州西界劃出設沙州,又將涼州東界劃出設河州,時屬官們都稱臣。張駿亦在姑臧附近增築新城,又修建用金玉和五色畫裝飾的謙光殿,極盡珍貴精巧,其四面都各建一殿,四季各居一殿。他又自稱大都督、大將軍、假涼王,督攝涼沙河三州,設六佾之舞,設天子的豹尾車,所設祭酒、郎中、大夫、舍人、謁者等官職官號都模仿晉朝體系,只是稍稍改了名字。

建興三十四年(346年),張駿去世,享年四十歲,前涼私諡為文公,晉廷則賜諡號忠成,贈大司馬,歸葬大陵。其子張祚稱帝時,追尊為文王,廟號世祖。

后凉年间,有名叫安据的即序胡人盗张骏墓,见张骏貌如生前,并盗得真珠簏、琉璃榼、白玉樽、赤玉箫、紫玉笛、珊瑚鞭、马脑钟、水陆奇珍不可胜数。后凉皇帝吕纂诛安据党徒五十余家,遣使吊祭张骏,并缮修其墓。

張駿年輕就已顯得奇特雄偉,十歲時就能寫文章,為人卓越不羈,但曾經縱情淫慾,常夜遊城邑里巷。《魏書·張寔傳》記載張駿為人貪婪,為求進圖秦隴而給予人民穀物布帛,一年後收一倍稅收,不夠的都要用田地屋宅抵償。然《晉書·張軌傳》則載是譚詳建議將倉庫的穀贈予百姓,然後在秋季收三倍稅收,為陰據所諫而放棄實行。《魏書》又寫其因畏懼大姓陰氏勢力大而逼陰澹弟陰鑒自殺,大失人心。此事《晉書》亦無載。


\subsection{桓王生平}

张重华(327年-353年),字泰臨,是十六国时期前凉政权的君主。为前凉文王张骏次子,353年病死。張重華統治時期,前涼國勢達於極盛,多次擊敗後趙石虎的進攻,後更乘後趙末年國亂而進取秦州。在位七年病死,年僅二十七歲。

建興二十年(332年),群僚請張駿立世子,張駿最初不肯,但在中堅將軍宋輯的勸說下,張駿還是立了張重華為世子。建興三十三年(345年),張駿從涼州分劃出沙州及河州,以武威、武興、西平、張掖、酒泉、建康、西郡、湟河、晉興、須武及安故十一郡仍為涼州,由張重華任五官中郎將、涼州刺史。

建興三十四年(346年),張駿去世,涼州官屬推張重華為使持節大都督、太尉、護羌校尉、涼州牧,襲爵西平公,假涼王。張重華即位後減輕賦斂,免除關稅,減省園囿,以撫恤貧窮者。同年後趙派麻秋、王擢等侵涼,金城太守張沖降趙,涼州震動。張重華任用謝艾抵抗,終大破趙軍,殺五千人。翌年,後趙再派石寧領二萬兵作為麻秋後援,前涼將領宋秦更加率二萬戶人投降後趙。張重華再度起用謝艾,命其率三萬步騎進軍臨河,又破趙軍,斬殺趙將杜勳、汲魚,一萬三千人被俘或陣亡。不久,石寧聯合麻秋等率十二萬兵進屯河南,再度進攻,張重華想要親自出擊,但為謝艾及索遐所勸止,遂派二人率兵二萬抵抗。時後趙將孫伏都、劉渾率步騎二萬增援麻秋,眾人渡過黃河並屯於長最。謝艾等進軍至神鳥,擊敗王擢前鋒,令其退回黃河以南,接著就進攻長最,再敗趙軍,麻秋等退還金城。石虎聞麻秋戰敗,也嘆息道:「我以偏師就平定了九個州,現在用九個州的力量卻在枹罕寸步難行,真是對方有能人,還不可以謀取呀。」不過,麻秋隨後擊敗了張瑁,枹罕護軍李逵降趙,於是河南地區羌、氐族人都附趙。

建興三十五年(347年),東晉派侍御史俞歸到涼州,授予張重華假節、侍中、大都督、督隴右關中諸軍事、護羌校尉、大將軍、涼州刺史,封西平公。俞歸到涼時,時張重華想稱涼王,故未受詔,更命親信沈猛向俞歸表示,但為俞歸拒絕,並言:「今天你的主公剛剛繼位就要稱王,若果率領河右部眾平定東方的胡、羯,修復晉朝帝陵及宗廟,迎天子還都洛陽,還有甚麼可以嘉獎呀?」張重華於是不圖稱王。

建興四十年(352年),因後趙國亂,苻健乘時於關中建立前秦,時任後趙西中郎將的王擢向東晉請降,獲授征西將軍、秦州刺史,但同年就被前秦將領苻雄擊破,於是出奔涼州,向前涼歸降。張重華厚待他,任命他為征虜將軍、秦州刺史。張重華更派了將軍張弘及宋修率一萬五千兵與王擢會合,讓他進攻前秦。次年(353年)兩軍交戰,王擢大敗逃奔姑臧,張弘及宋修都戰死。張重華素服為陣亡將士舉哀,也安慰其家人,更加再命王擢進攻前秦秦州,最終取勝,奪取秦州。張重華因而上疏東晉請求伐秦,東晉則進張重華涼州牧。

建興四十一年(353年),張重華因病去世,享年二十七,葬顯陵。私諡為昭公,後改桓公,東晉則賜諡號敬烈公。重華病重時曾下手令徵召謝艾為衞將軍、監中外諸軍事以輔政,但最終為重華兄張祚等人壓下,終由張祚輔政,不久更廢掉張重華的世子張曜靈,自己登位。張祚稱帝,追諡張重華為桓王,上廟號世宗。

張重華寬厚平和,深沉穩重,又寡言。不過在擊退後趙連番進攻後表現怠惰,疏於政事且很少親身接見賓客,司直索遐曾經進言勸諫,張重華雖然大感高興,但沒有改變。他又喜和身邊小人玩樂,更多次向左右近臣賞賜金錢。


\subsection{哀公生平}

張曜靈(344年-355年),字元舒,是十六国時期前凉政權的君主,前凉桓王張重華子。張曜靈即位不久就被伯父張祚奪位,及後更被殺害。

建興四十一年(353年),張重華患病,遂立張曜靈為世子。同年張重華去世,實歲仅九岁的張曜靈继位,稱大司馬、護羌校尉、涼州刺史、西平公。張重華原本想以謝艾輔政,但遭其兄張祚與寵臣趙長、尉緝等勾結而壓下張重華的命令,於是張曜靈繼位後,張祚就矯令擔任輔政工作。不久,趙長等以張曜靈太過年輕,建議立年長君主,其祖母马氏與張祚私通,遂废曜靈爲涼寧侯,由張祚繼位。

和平二年八月(355年),张瓘等大臣试图废黜张祚、迎張曜靈复位,未成,張曜靈被張祚派遣杨秋胡暗杀,匿尸沙坑。同年张祚被杀,私谥為哀公。

%% -*- coding: utf-8 -*-
%% Time-stamp: <Chen Wang: 2019-12-18 17:06:58>

\subsection{威王\tiny(353-355)}

\subsubsection{生平}

涼威王张祚(327年前-355年),字太伯,安定烏氏人。十六国时期前凉皇帝,前凉文王张骏庶长子,前凉桓王张重华异母兄。張祚與張重華寵臣勾結,又與太后通姦,得以在張曜靈繼位不久即廢其自立,更曾經稱帝。然而在稱帝翌年就被政變推翻及被殺。

張祚受封長寧侯,他博學且強壯勇武,又有政治才能,可是為人狡詐善於奉承,與張重華寵臣趙長、尉緝等人勾結並結為異姓兄弟。建興四十一年(353年)張重華病重時,曾下手令召酒泉太守謝艾入朝輔政,但為趙長等壓下。同年張重華死,由其年幼的长子張曜靈繼位,趙長等就假稱張重華遺令,以張祚為持節、都督中外諸軍事、撫軍將軍身份輔政。時趙長等以張曜靈年幼,稱國家需要年長君主,张祚因与张重华之母馬太后通奸,遂煽动马太后废黜了張曜靈,立张祚為主。張祚於是自稱大都督、大將軍、涼州牧、涼公。張祚位後即淫亂張重華的妻妾及其未嫁女兒。

和平元年(354年),张祚称帝,改元「和平」,設宗廟、八佾舞,並置百官,尚書馬岌因切諫被免官,郎中丁琪進諫更被殺,又殺謝艾。張祚又曾進攻驪靬,但大敗而還。同年東晉桓溫北伐,也有配合北伐的秦州刺史王擢派人報告張祚稱桓溫善於用兵,軍勢難測。張祚聞訊恐懼,但還擔憂王擢會倒過來進攻自己,於是派人暗殺他,但因被王擢發現而不成。張祚在暗殺失敗後更加恐懼,於是出兵聲稱要東征,實則是想西退至敦煌自保,只是遇上桓溫退兵才取消行動。不過,張祚仍繼續打擊王擢,派了牛霸率三千兵打敗王擢,逼使王擢投降前秦。

张祚治国不道,曾置五都尉去專抓別人過失,又限定四品以下官員不得送贈衣布,庶人不能畜養奴婢及乘坐車馬。张祚为人荒淫暴虐,国人无不侧目,都作諷刺其淫亂的詩。和平二年(355年),張祚因不欲河州刺史張瓘強大,於是命令他去討伐叛胡,其實已派易揣及張玲率三千兵襲擊張瓘。王鸞識術數,向張祚說:「這支軍隊出去,肯定不會回來,涼國會陷於危險。」更上陳張祚三不道。張祚聞言大怒,認定王鸞所說是妖言,將他處斬。王鸞臨死前就說:「我死後,軍隊在外面戰敗,大王在內死亡,肯定會發生的!」張祚更誅殺王鸞一族人。不過,張瓘就殺了張祚派去代其守枹罕的索孚,易揣等渡過黃河途中就被張瓘攻擊,張瓘更出兵跟隨單騎逃還的易揣,兵向姑臧。張瓘軍前來的消息震動姑臧人心,時宋混、宋澄兄弟因其兄宋修與張祚有前嫌,就出城聚眾響應張瓘,並反攻姑臧。時張瓘傳檄州郡,要復立張曜靈,故張祚就派楊秋胡殺害張曜靈;另又收捕並處死張瓘的兩個弟弟張琚及張嵩。二人知要被捕時卻在市招募數百,大叫張瓘大軍已經到達姑臧城東,恐嚇敢動手的人要被誅三族。收捕的人果被嚇退,然後二人西城門迎宋混等入城。趙長等人懼怕因擁立張祚獲罪,於是請馬太后出殿,改立張玄靚為主,不過易揣等人卻引兵入殿,收殺趙長等人。宋混等入城後,張祚按劍命令部眾死戰,但因為他失眾心,將士根本毫無鬥志,張祚於是為宋混等殺死,頭被斬下宣示內外,更遭曝屍在大道左邊,城內人民都大呼萬歲。

事後張祚以庶人的禮儀下葬,直至其弟張天錫即位時,才改葬到愍陵,追諡為威王。

\subsubsection{和平}

\begin{longtable}{|>{\centering\scriptsize}m{2em}|>{\centering\scriptsize}m{1.3em}|>{\centering}m{8.8em}|}
  % \caption{秦王政}\
  \toprule
  \SimHei \normalsize 年数 & \SimHei \scriptsize 公元 & \SimHei 大事件 \tabularnewline
  % \midrule
  \endfirsthead
  \toprule
  \SimHei \normalsize 年数 & \SimHei \scriptsize 公元 & \SimHei 大事件 \tabularnewline
  \midrule
  \endhead
  \midrule
  元年 & 354 & \tabularnewline\hline
  二年 & 355 & \tabularnewline
  \bottomrule
\end{longtable}


%%% Local Variables:
%%% mode: latex
%%% TeX-engine: xetex
%%% TeX-master: "../../Main"
%%% End:

%% -*- coding: utf-8 -*-
%% Time-stamp: <Chen Wang: 2019-12-18 17:31:16>

\subsection{冲王\tiny(355-363)}

\subsubsection{生平}

涼沖王張玄靚(350年-363年),字元安,十六國時期前涼國君主,為張重華之子,張曜靈之弟。張玄靚年幼繼位,前涼國政先後在張瓘、宋混、宋澄、張邕及張天錫手中掌握,期間政變頻仍,張玄靚最終亦因張天錫政變而被殺。

張玄靚於和平元年(354年)獲張祚封為涼武侯。和平二年(355年),張祚被殺,張玄靚被宋混、張琚推為大將軍、涼州牧、護羌校尉、西平公,恢復年號為建興四十三年。不久,河州刺史張瓘返都城姑臧(今甘肅武威),張玄靚再被推為涼王,政事決於張瓘。次年(356年)前秦派使者閻負、梁殊前來,要勸說前涼臣服於前秦,張瓘恐懼,於是勸導張玄靚向前秦稱藩,而前秦亦以張玄靚所稱的官爵授命。

張玄靚繼位後,前涼國內先後有李儼、衞綝和馬基等人反叛,張瓘擊敗了衞綝並討平馬基。其時張瓘、張琚兄弟賞罰都依從自己愛惡,無視綱紀,又不聽諫言,故並不得人心。可是他們自以勢力強大,且有功勳,所以有篡位的意圖,然而就忌憚忠心剛直的宋混。建興四十七年(359年),張瓘徵集了數萬兵並會聚於姑臧,想要消滅宋混兄弟,宋混及宋澄知道後就率領壯士楊和等四十多騎到南城,並向各個兵營宣稱張瓘謀反,太后下令誅除他,很快就召集到二千多人。隨後宋混率眾與張瓘決戰,張瓘兵敗,其部眾都背棄張瓘,向宋混投降,張瓘兄弟於是自殺。事後宋混代替張瓘掌政,張玄靚為宋混所建議去涼王稱號,改稱涼州牧。建興四十九年(361年),宋混去世,張玄靚順從宋混遺言而讓宋澄掌政,不過右司馬張邕不滿宋澄專政,同年即起兵攻滅宋澄,並誅殺宋氏一族。張玄靚隨後又改讓張邕與叔父張天錫共同掌政。可是,張邕自恃功勳大而行事驕縱,濫用刑法,更與馬太后私通,樹立黨羽,很不得人心,張天錫就是再次發動政變,張邕兵敗自殺,其黨眾皆被張天錫誅殺。張玄靚遂以張天錫一人掌政。十二月,張天錫讓張玄靚改奉當時東晉的升平年號,稱升平五年。晉廷則授張玄靚大都督隴右諸軍事、護羌校尉、涼州刺史,西平公。

升平七年(363年),馬太后去世,張玄靚以其母郭夫人為太妃,而郭夫人因不滿張天錫專政而與張欽圖謀發動政變,可是圖謀外泄,張欽等人都被張天錫殺害。張天錫隨後便發動政變,派兵入宮殺死張玄靚,向外宣稱張玄靚暴斃,享年十四歲。

張玄靚被下葬平陵。張天錫私諡為沖公。東晉孝武帝司馬曜賜諡號敬悼。

\subsubsection{建兴}

\begin{longtable}{|>{\centering\scriptsize}m{2em}|>{\centering\scriptsize}m{1.3em}|>{\centering}m{8.8em}|}
  % \caption{秦王政}\
  \toprule
  \SimHei \normalsize 年数 & \SimHei \scriptsize 公元 & \SimHei 大事件 \tabularnewline
  % \midrule
  \endfirsthead
  \toprule
  \SimHei \normalsize 年数 & \SimHei \scriptsize 公元 & \SimHei 大事件 \tabularnewline
  \midrule
  \endhead
  \midrule
  四三年 & 355 & \tabularnewline\hline
  四四年 & 356 & \tabularnewline\hline
  四五年 & 357 & \tabularnewline\hline
  四六年 & 358 & \tabularnewline\hline
  四七年 & 359 & \tabularnewline\hline
  四八年 & 360 & \tabularnewline\hline
  四九年 & 361 & \tabularnewline
  \bottomrule
\end{longtable}

\subsubsection{升平}

\begin{longtable}{|>{\centering\scriptsize}m{2em}|>{\centering\scriptsize}m{1.3em}|>{\centering}m{8.8em}|}
  % \caption{秦王政}\
  \toprule
  \SimHei \normalsize 年数 & \SimHei \scriptsize 公元 & \SimHei 大事件 \tabularnewline
  % \midrule
  \endfirsthead
  \toprule
  \SimHei \normalsize 年数 & \SimHei \scriptsize 公元 & \SimHei 大事件 \tabularnewline
  \midrule
  \endhead
  \midrule
  五年 & 361 & \tabularnewline\hline
  六年 & 362 & \tabularnewline\hline
  七年 & 363 & \tabularnewline
  \bottomrule
\end{longtable}


%%% Local Variables:
%%% mode: latex
%%% TeX-engine: xetex
%%% TeX-master: "../../Main"
%%% End:

%% -*- coding: utf-8 -*-
%% Time-stamp: <Chen Wang: 2019-12-18 17:32:19>

\subsection{悼公\tiny(363-376)}

\subsubsection{生平}

張天錫(346年-406年),字純嘏,本字公純嘏,因被人嘲笑是三字而自行改字,小名獨活,安定烏氏人。中國十六國時期前涼政權的最後一位君主。張天錫為前涼文王張駿少子,前涼桓王張重華之弟。張天錫在位時前秦國力強盛,雖曾主動斷絕與前秦關係,但最終仍逼於軍事力量而再度稱藩。及後張天錫反抗前秦徵召入朝的命令並射殺使者,前秦大軍遂攻伐前涼,張天錫不敵投降,前涼國於是滅亡。淝水之戰後張天錫南歸東晉,並在東晉終老。

和平元年(354年),張祚封張天錫為長寧王。建興四十九年(361年),張邕殺死當政的宋澄,當時的前涼君主張玄靚就以張天錫為中領軍,與張邕共輔朝政。不過,張邕因樹立黨羽專權,經常濫用刑法殺人,很不得人心,張天錫親信劉肅則與其共謀除去他。十一月,張天錫與張邕一同入朝,劉肅就與趙白駒跟著張天錫行動,二人先後襲擊張邕但都失敗,於是與張天錫一同走入宮中。逃走的張邕率三百軍人進攻宮門,張天錫登門樓指責張邕凶惡悖逆,聲言自己是在冒死保衞國家社稷,並只會針對張邕而已。張邕兵眾聞言都逃散,張邕自殺,張天錫又誅殺了張邕黨羽,專掌朝政。

升平七年(363年),郭太妃以張天錫專政,與張欽密謀誅殺張天錫,事洩,欽等皆死;右將軍劉肅於是勸張天錫自立,天錫遂會劉肅夜襲皇宮,殺張玄靚。張天錫自稱使持節、大都督、大將軍、護羌校尉、涼州牧、西平公,並派使者出使建康請命,東晉於是在366年授張天錫為大將軍、大都督、督隴右關中諸軍事、護羌校尉、涼州刺史,封西平公。前秦亦派大鴻臚授張天錫大將軍、涼州牧、西平公。

張天錫登位後多次在園池設宴,又沉迷於歌舞和女色,荒廢政事。張天錫更將兩個親信劉肅及梁景收為養子,讓二人參與朝政,令人們有怨言和恐懼,索商及天錫堂弟張憲曾經勸諫他但不獲授納。張天錫於升平十年(366年)與前秦斷交,並在進攻李儼時與前秦發生軍事衝突,並俘獲了陰據和他率領的五千兵。升平十五年(371年),前秦攻滅仇池,送還陰據及其士兵回國,並派梁殊及閻負隨行,順道送達前秦丞相王猛的書信,暗示要張天錫別和前秦作對。張天錫看後十分恐懼,於是派使者向前秦謝罪,向前秦稱藩,前秦天王苻堅任命其為使持節、散騎常侍、都督河右諸軍事、驃騎大將軍、開府儀同三司,涼州刺史、西域都護、西平公。然而因張天錫因為懼怕前秦吞併,於同年在姑臧設壇,遙與晉三公盟誓,又派從事中郎韓博出使東晉,並寫信給東晉大司馬桓溫,約定大舉出兵北伐,會師上邽。

升平二十年(376年),苻堅徵召張天錫入朝任武衞將軍,同時派了苟萇、毛盛、梁熙及姚萇等率十三萬步騎至西河郡,預備一旦張天錫拒絕應命就進攻前涼。張天錫接到梁殊、閻負送來的詔命後問及眾僚意見,除席仂建議送貨款和質子,徐圖後計外,大部份人都認為涼州有精兵及天險,可以取勝。張天錫於是決定反抗,派馬建率兵二萬抵抗秦軍,並命人射殺兩名前秦使節。面對秦軍進攻,馬建懼而退守清塞,張天錫又派掌據率兵三萬與馬建屯於洪池,自率五萬屯金昌城。可是,苟萇隨後進攻掌據時馬建就投降前秦,掌據戰死,張天錫又派趙充哲為前鋒,率五萬兵與苟萇等作戰,但又在赤岸大敗,張天錫出城意圖再戰,但因金昌城中反叛而被逼逃回姑臧並請降。苟萇等到姑臧後受降,並送張天錫到長安,其他郡縣都降秦,前涼滅亡。苻堅在長安為張天錫建了府邸,任命他為侍中、北部尚書,封歸義侯。

晉太元八年(383年),晉軍於淝水之戰擊潰來攻的前秦軍,當時張天錫任征南大將軍苻融的司馬隨軍,趁機南奔東晉,東晉朝廷下詔以張天錫為散騎常侍左員外,復封為西平郡公。後轉拜金紫光祿大夫。後曾加授廬江太守,桓玄掌政時為了招撫四方而任命張天錫為護羌校尉、涼州刺史。義熙二年(406年),張天錫去世,享年六十一歲,追贈為鎮西將軍,諡號悼公。

張天錫因文才而聲名遠著,回歸晉廷後亦甚得晉孝武帝知遇,可是朝中官員卻以其曾經亡國被俘而中傷他。會稽王司馬道子曾經問及涼州有甚出產,張天錫立即就答道:「桑葚甘甜、鴟鴞會變聲音、乳酪養生、人沒有嫉妒之心。」不過,後來張天錫表現得昏亂喪志,雖然有公爵爵位也得不到別人禮遇。至晉安帝隆安年間,當政的會稽王世子司馬元顯更常常請他來戲弄他。擔任廬江太守亦因為其家貧而獲授。

\subsubsection{升平}

\begin{longtable}{|>{\centering\scriptsize}m{2em}|>{\centering\scriptsize}m{1.3em}|>{\centering}m{8.8em}|}
  % \caption{秦王政}\
  \toprule
  \SimHei \normalsize 年数 & \SimHei \scriptsize 公元 & \SimHei 大事件 \tabularnewline
  % \midrule
  \endfirsthead
  \toprule
  \SimHei \normalsize 年数 & \SimHei \scriptsize 公元 & \SimHei 大事件 \tabularnewline
  \midrule
  \endhead
  \midrule
  七年 & 363 & \tabularnewline\hline
  八年 & 364 & \tabularnewline\hline
  九年 & 365 & \tabularnewline\hline
  十年 & 366 & \tabularnewline\hline
  十一年 & 367 & \tabularnewline\hline
  十二年 & 368 & \tabularnewline\hline
  十三年 & 369 & \tabularnewline\hline
  十四年 & 370 & \tabularnewline\hline
  十五年 & 370 & \tabularnewline\hline
  十六年 & 372 & \tabularnewline\hline
  十七年 & 373 & \tabularnewline\hline
  十八年 & 374 & \tabularnewline\hline
  十九年 & 375 & \tabularnewline\hline
  二十年 & 376 & \tabularnewline
  \bottomrule
\end{longtable}


%%% Local Variables:
%%% mode: latex
%%% TeX-engine: xetex
%%% TeX-master: "../../Main"
%%% End:



%%% Local Variables:
%%% mode: latex
%%% TeX-engine: xetex
%%% TeX-master: "../../Main"
%%% End:

%% -*- coding: utf-8 -*-
%% Time-stamp: <Chen Wang: 2019-12-18 17:35:29>


\section{后赵\tiny(319-351)}

\subsection{简介}

后赵(319年-351年)是十六国时期羯族首领石勒建立的政权。

因石勒统治地区为战国时赵国故地,因此刘曜封其为赵王,立国即以此为号。为别于先建国的前赵,故史称“后赵”,又以其王室姓石,又称“石赵”。

在晋怀帝末年反晋浪潮中,石勒投附在并州刺史部的南匈奴贵族刘渊为部将,屡立战功,势力强盛。308年10月,刘渊正式称帝,建国号“汉”,(刘曜后改为赵),建都平阳(今山西临汾)年号为永凤。318年,国丈靳准杀死隐帝刘粲夺权,自立为汉天王。镇守长安的刘粲叔父刘曜得知平阳有变,自立为皇帝,派遣军队至平阳,族灭靳氏,迁都到长安。与此同时,石勒亦参与讨伐靳准,后来试图挑起城中变乱促其投降的计划失败,导致靳明掌权并倒向刘曜,石勒大怒,攻破平阳城。319年,刘曜在长安改国号“汉”为“赵”,史称前赵。同年,石勒在襄国(今河北邢台)自称大单于、赵王,与前赵决裂,史称后赵。329年石勒灭前赵,次年称帝。

石勒开拓疆土,灭前赵,占有除辽东、河西以外的北方地区。后赵前期仍采取胡汉分治政策,但注意笼络汉族士族,减轻租赋,发展农业生产,推行儒家教育,社会呈现丰裕景象。统治地区包括冀州、并州、豫州、兖州、青州、司州、雍州、秦州、徐州、凉州、荆州部分地区、幽州部分地区。

后赵建平四年(333年)石勒卒。次年其从子石虎篡位,335年迁都邺城(今河北临漳境内)。石虎非常残暴,征役无时,大兴土木,荒淫无度,社会矛盾十分尖锐。太宁元年(349年)后赵爆发梁犊领导的雍凉戍卒舉兵,一度攻克长安,有众40余万。同年石虎卒,其子为争帝位互相残杀。石虎养孙冉闵大杀石氏子孙及羯胡,次年(350年)自立为帝,改国号魏,史称冉魏。石虎子新兴王石祗在襄国称帝,与冉魏对抗。后石祗为得前燕相助,降称赵王。351年,石祗被手下刘显所杀,后赵亡。次年,其他幸存的石氏子孙投降东晋,也被杀及诛滅。

%% -*- coding: utf-8 -*-
%% Time-stamp: <Chen Wang: 2021-11-01 11:54:04>

\subsection{明帝石勒\tiny(319-333)}

\subsubsection{生平}

趙明帝石勒(274年-333年8月17日),字世龍,原名㔨,小字匐勒,上黨武鄉(今山西榆社)羯族人,是五胡十六國時代後趙的開國君主。

石勒初期因公師藩而起兵,後投靠漢趙君主劉淵,之後卻與漢國決裂,由漢國分裂出去。石勒在他的謀臣,漢人張賓輔助之下以襄國(今河北邢台)為根據地,並陸續消滅了王浚、邵續、段匹磾等西晉於北方的勢力,繼而又消滅曹嶷,進侵東晉以及消滅劉曜領導的前趙,又北征代國,率領後趙成為當時北方最強盛的國家。石勒又實行多項措施,推動文教和經濟發展。另外他厚待來自西域的佛教僧侶佛圖澄,對當時佛教的傳播有一定貢獻。

石勒出身羯胡,為南匈奴羌渠人。其祖先為匈奴分支部落的贵族。石勒原沒有漢文姓名,其姓與名皆是由牧人汲桑所起。

羯人的起源不詳,可能起源自小月氏,而歷史學家陳寅恪認為可能起源於中亞康居。

石勒壯健有膽量和魄力,雄健威武,更喜愛騎射。父親周曷朱為部落小帥,因性格粗暴凶惡而不被一眾胡人心服,常命石勒代他領導部眾,卻得眾人信賴。當時相士和父老都稱石勒相貌奇特,氣度非常,前途無可限量,勸邑中人厚待他。但大部份人對這說法都嗤之以鼻,唯獨郭敬和甯驅相信,更加借資源給他,石勒亦感恩,盡心為他耕作。

太安二年(303年),并州發生大饑荒,石勒與一眾胡人逃散,於是去依靠甯驅。當時北澤都尉劉監打算將他賣掉,幸得甯驅協助才沒有成事。之後石勒暗中改投都尉李川,路上遇見郭敬,於是向他哭訴飢寒之苦。郭敬聽後傷心流涕,送他衣服和食物。當時石勒向郭敬建議誘一眾胡人到冀州吃糧,借故賣掉他們換取金錢,既可解諸胡饑困,亦能獲利。而同時建威將軍閻粹說服并州刺史司馬騰遷諸胡到太行山以東地區販賣,以獲得軍事資本,於是司馬騰就派人到冀州捕捉一眾胡人,連石勒都被抓著。當時負責捕捉胡人的張隆多次毆打石勒,而且路上常有人飢餓或病倒,石勒全靠郭敬親族郭陽和郭時的资助才成功到冀州。到冀州後石勒被賣給師懽為奴,師懽却因其儀表堂堂,氣质出眾,讓他做了自己的佃客。

當時師懽家在牧苑側,石勒於是與牧帥汲桑往來,更以自己有相馬的能力而自薦給汲桑。後結集王陽、夔安、支雄、冀保、吳豫、劉膺、桃豹、逯明、郭敖、劉徵、張曀僕、呼延莫、郭黑略、張越、孔豚、趙鹿、支屈六十八個壯士一同號稱為「十八騎」,並與他們搶掠園林,以財寶巴結汲桑。

永興二年(305年),成都王司馬穎被河間王司馬顒廢去官位和皇太弟身份,因司馬穎曾鎮鄴城,很多河北人都可憐司馬穎的遭遇。司馬穎舊將公師藩於是自稱將軍,以司馬穎之名在趙、魏之間舉兵,聚眾數萬,汲桑與石勒亦率數百騎師附公師藩。此時,汲桑才命石勒以石為姓,以勒為名。公師藩則拜石勒為前隊督,並與他進攻守鄴城的平昌公司馬模,卻被苟晞、丁紹和司馬模部將馮嵩擊敗。次年,公師藩在白馬縣打算南渡黃河,被苟晞擊殺。

公師藩死後,石勒與汲桑逃回茌平牧苑,石勒被汲桑命為伏夜牙門,率領牧人劫掠郡縣的囚犯,又招納潛居山間的亡命之徙。汲桑於是在永嘉元年(307年)自稱大將軍,聲稱要為上一年被殺的司馬穎報仇。汲桑以石勒為前驅,屢次取勝,於是署石勒為討虜將軍、忠明亭侯。石勒即隨汲桑進攻鄴城,擔任前鋒都督,大破馮嵩,並且長驅直進,於五月攻陷鄴城。汲桑在鄴城殺司馬騰和萬多個兵民,焚毀鄴城宮室和搶掠城中婦女珍寶後才離開。

石勒及後又跟汲桑進攻幽州刺史石尟。石勒在樂陵擊殺石尟後又擊敗率五萬兵營救石尟的乞活軍將領田禋,並與苟晞相持於平原、陽平之間數月,期間發生三十多場戰事,互有勝負,迫使太傅司馬越率兵在官渡為苟晞聲援。石勒和汲桑於九月大敗給苟晞,於是收拾餘眾,打算投奔劉淵建立的漢國,但又於赤橋敗於冀州刺史丁紹,石勒於是逃到樂平。後汲桑更在樂陵被晉兵所殺。

石勒投漢國後,於十月就成功讓據守上黨的㔨督和馮莫突歸降漢國,劉淵於是封石勒為輔漢將軍、平晉王。後又因據守樂平的烏桓人張伏利度不肯加盟漢國,石勒於是假稱得罪劉淵而投奔張伏利度,並與他結為兄弟,與其胡人部眾一同搶掠郡縣,所向無敵,於是眾人畏服。石勒在眾人心附自己後乘宴會抓著張伏利度,讓部眾推舉自己為主。石勒後釋放張伏利度而率領其部眾歸附漢國。劉淵於是加石勒為督山東征討諸軍事,並讓這些胡人部眾跟隨他。

劉淵派兵向外擴張,於永嘉二年(308年),派石勒領兵東侵。石勒於九月攻陷鄴城,征北將軍和郁逃走。十月劉淵稱帝,授予使持節,平東大將軍。不久石勒又率三萬進攻魏郡、汲郡和頓丘,五十多個由當地人集結的壁壘望風歸附,於是獲假壘主將軍、都尉印綬。後更殺魏郡太守王粹和冀州西部都尉馮沖,並擊敗殺害乞活軍將領赦亭和田禋。劉淵於是授予石勒安東大將軍、開府。石勒於永嘉三年(309年)進攻鉅鹿和常山,部眾增加至十多萬人,更有文士加入,以他們成立「君子營」,石勒以漢人張賓為謀主,刁膺、張敬為股肱。因軍事力量強大,在石勒派張斯游說之下,并州的胡羯大多亦跟從石勒。

劉淵之後派兵進攻壺關,石勒後被任命為前鋒都督,擊破劉琨派來救援壺關的軍隊,助漢國攻陷壺關。九月,晉司空王浚派祁弘與段務勿塵在飛龍山進攻石勒,石勒大敗,退屯黎陽,但仍能分派諸將攻打未及叛變的部眾,收降三十多個壁壘,並置守宰安撫。十一月,石勒進攻信都,殺害冀州刺史王斌。當時,王浚命裴整和王堪領兵討伐石勒,石勒於是立刻回軍抵禦。石勒到黎陽後,裴憲拋棄軍隊逃到淮南,王堪則退守倉垣。劉淵於是授命石勒為鎮東大將軍,封汲郡公,石勒辭讓封爵。

永嘉四年(310年),石勒南渡黃河,攻陷白馬後與王彌一同進攻徐、豫、兗三州。不久更攻下鄄城和倉垣,並北渡黃河進攻冀州諸軍,投降他的平民多達九萬多人。及後又協助劉聰等人進攻河內,並進攻冠軍將軍梁巨,晉懷帝派兵援救。梁巨因兵敗請降,石勒不許,最終坑殺萬多名降卒並殺死梁巨,援兵亦退還。此戰戰果使得河北各個自守的堡壘都震驚,紛紛送人質到石勒處。同年劉淵逝世,劉聰殺兄劉和繼位,任命石勒為征東大將軍、并州刺史、汲郡公。石勒這次辭讓征東大將軍。隨後便會合劉粲、劉曜、王彌大軍進攻洛陽,直入洛川。石勒又進攻倉垣,但被守將王讚擊敗。

石勒後來改攻南陽,早前在荊州叛變的雍州流民王如、侯脫和嚴嶷等都感到恐懼,於是派了一萬兵屯守襄城以作抵抗。但石勒到後擊敗守軍並將部眾全數俘虜,進駐宛城以北。當時侯脫據有宛城而王如守穰縣,王如怕石勒進攻,於是以珍寶賄賂石勒,與他結為兄弟;同時又因王如與侯脫不睦,於是勸石勒進攻侯脫。嚴嶷知道石勒攻宛後領兵救援,但石勒十二日便攻陷宛城,嚴嶷趕不及而直接向石勒投降。石勒誅殺侯脫,囚禁嚴嶷,呑併了二人部眾,軍力愈為強盛。

石勒於是進一步南侵,進攻襄陽並循漢水攻陷三十多個處於江西的壁壘。石勒留刁膺守襄陽後就率三萬精銳騎兵還攻王如,但因怕王如強盛,於是改攻襄城。王如知道後就命弟弟王璃率兵,假稱犒軍而襲擊石勒,但遭石勒擊滅。石勒至此有雄據長江、漢水一帶的意願,張賓雖然反對並勸他北歸但都不聽。

永嘉五年(311年),駐鎮建業的琅琊王司馬睿見石勒南侵荊州,於是派王導率兵討伐。而石勒軍糧不繼,更加因疫症損失大半士兵。石勒於是接納張賓建議,焚毀輜重,收好糧食和卷起盔甲,輕兵渡過沔水並進攻江夏,然後北歸,先攻陷新蔡,殺新蔡王司馬確,後再攻陷許昌。

永嘉五年(311年)三月,率領行臺和二十多萬晉兵討伐石勒的司馬越死在項縣,大軍於是在王衍及襄陽王司馬範帶領下護送司馬越靈柩回東海國。四月,石勒率輕騎追擊晉軍,終在苦縣寧平城追上大軍,並殺敗王衍所派的將軍錢端。晉兵在錢端敗死後潰敗,被石勒包圍並射殺,士兵在混亂中互相踐踏,全軍覆沒。石勒誅殺包括王衍以內隨行的官員和西晉宗室。不久石勒在洧倉追上司馬越世子司馬毗由洛陽東歸的部眾,又將司馬毗及宗室王等人殺害。

隨後,劉聰派呼延晏率大軍進攻洛陽,石勒領三萬騎兵到洛陽與大軍會合,攻陷洛陽,俘虜晉懷帝。戰後石勒將戰功歸於王彌和劉曜,於出屯許昌。七月,石勒領兵攻晉大將軍苟晞所駐蒙城,生擒苟晞並任用為左司馬。劉聰於是以石勒為幽州牧。

苟晞被擒後,王彌寫了一封言辭卑屈的書信祝賀石勒,同時又知道王彌忌憚自己,打算引自己到青州然後殺害。石勒於是聽從張賓的建議:乘王彌當時兵力減弱而消滅他。不久石勒就聽從張賓的建議,率兵救援與乞活軍相持不下的王彌以換取王彌的信任,隨後就借宴會的機會襲殺王彌,吞併了他的部眾,並假稱王彌謀反。劉聰知道石勒殺王彌後大怒,但又因怕他生了異心而不敢處罰,反而加授鎮東大將軍、督并、幽二州諸軍事、領并州刺史。

後來晉并州刺史劉琨將早年與石勒失散的石勒生母以及侄兒石虎送返,並授予侍中、車騎大將軍、領護匈奴中郎將、襄城郡公給石勒以作招降。但石勒拒絕,僅厚待劉琨使者和送名馬及珍寶給劉琨以作謝禮。

永嘉六年(312年),石勒在葛陂建屋宇,推廣耕作,營造船隻,打算攻略建業。但當年正遇上連綿三個月的大雨,司馬睿知道石勒的行動後更招集江南的兵眾會聚壽春以作抵禦。石勒軍中缺糧和有疫症,大量士兵死亡,而且多次收到來自司馬睿的討伐文告,似乎即將攻來,於是召集眾人討論。最後石勒接納張賓的建議,放棄留駐南方而北據鄴城三臺,經營河北,並以該處作根據地發展勢力。

石勒於是先將輜重北歸,又派石虎領兵攻壽春以防晉軍追擊輜重,最終晉兵雖然擊敗石虎,但仍因怕石勒有伏兵而只駐守壽春。然而石勒北歸時經過地方都堅壁清野,石勒試圖掠取物資都一無所獲,於是軍中有大飢荒,士兵相食。到東燕郡時因引誘當地建壁壘自守的向冰並成功在棘津擊敗向冰的軍隊,從而獲得軍需品,重振軍力,得以長驅直進,向鄴城進發。

守鄴城的晉北中郎將劉演知道石勒將來攻擊就加緊守城,然而其部將臨深和牟穆率部眾向石勒投降。石勒諸將當時打算強攻鄴城,但是張賓認為劉演仍能倚仗鄴城三臺而負隅頑抗,強攻未必能輕易奪取,反而暫時放棄攻取能讓劉演自己潰敗。於是建議石勒先消滅大司馬、領幽州刺史王浚和并州刺史劉琨這兩個大敵,並提出邯鄲和襄國兩處作為取鄴城前的臨時根據地。石勒聽從,率軍進據襄國。

石勒駐鎮襄國後,就上表劉聰陳述駐鎮當地的意圖,又分遣諸將進攻冀州各郡縣的壘壁,使他們大多都歸附,並運糧給石勒。劉聰收到上表後署石勒為使持節、散騎常侍、都督冀幽并營四州雜夷、征討諸軍事、冀州牧,進封上黨郡公,開府、幽州牧、東夷校尉如故。

石勒後進攻王浚將領游綸、張豺所駐的苑鄉,遭王浚派兵聯同段部鮮卑的段疾陸眷、段末柸和段匹磾所率部眾共五萬多人前來討伐。石勒屢次敗於段疾陸眷,更發現對方打算攻城,在張賓及孔萇的建言下,石勒在北城城內設立二十多道突門,並在門內藏伏兵;期間不出戰以示弱,待對方鬆懈來攻時,突門中的伏兵出擊,出其不意。石勒最終因而成功生擒段部鮮卑中最勇悍的段末柸,逼得段疾陸眷退兵。石勒之後派使者向段疾陵眷求和,並與其結為兄弟。隨著段疾陸眷退兵,王浚軍不能獨留,石勒於是解除了危機。同時,石勒厚待並送還段末柸的行動令他歸心於石勒,削弱了一直支持著王浚的鮮卑力量。游綸、張豺在戰後也向石勒請藩。

建興元年(313年),石勒派石虎攻陷鄴城,當地流人都向石勒歸降。石勒後又派孔萇攻定陵,殺兗州刺史田徽,王浚所任的青州刺史薄盛歸降石勒,山東地區各個郡縣相繼被石勒奪取,劉聰於是升石勒為侍中、征東大將軍。一直支持王浚的烏桓也背叛王浚,暗中歸附石勒,使得王浚勢力更弱。

永嘉之亂後,王浚就假立太子,設立行臺,自置百官,更打算自立為帝,驕奢淫虐。石勒打算消滅王浚,吞併其勢力,張賓建議石勒假意投降王浚。石勒於是卑屈的向王浚請降歸附,在王浚使者來時特意讓弱兵示人,並且故作卑下,接受王浚的書信時朝北向使者下拜和朝夕下拜王浚送來的塵尾,更假稱見塵尾如見王浚;又派人向王浚聲稱想親至幽州支持王浚稱帝。王浚於是完全相信石勒的忠誠。然而,石勒一直派去作為使者的王子春卻為石勒刺探了王浚的虛實,讓石勒做好充足準備。

建興二年(314年),石勒正式進兵攻打王浚,乘夜行軍至柏人縣,接受張賓的建議,利用王浚和劉琨的積怨,寫信並送人質向劉琨請和,聲稱要為他消滅王浚。因此劉琨最終都沒有救援王浚,樂見王浚被石勒所滅。石勒一直進軍至幽州治所薊縣,先以送王浚禮物為由驅趕數千頭牛羊入城,阻塞道路,之後更縱容士兵入城搶掠,並捕捉王浚,數落王浚不忠於晉室,殘害忠良的罪行。石勒命將領王洛生押解王浚到襄國處斬,又盡殺王浚手下精兵萬人,擢用裴憲和荀綽為官屬。石勒留薊兩日後就焚毀王浚宮殿,留劉翰守城而返。

石勒回到襄國後將王浚首級送給劉聰,劉聰於是任命石勒為大都督、督陝東諸軍事、驃騎大將軍、東單于,並增封二郡。劉聰更與建興三年(315年)賜石勒弓矢,加崇為陝東伯,專掌征伐,他所拜授的刺史、將軍、守宰、列侯每年將名字及官職上呈就可,又以石勒長子石興為上黨國世子。

建興四年(316年),石勒率兵在玷城圍困晉樂平太守韓據,韓據向劉琨求援。劉琨因不久以前代國內亂而獲得拓跋猗廬舊部箕澹及衞雄率代國晉人和烏桓人加入而大大強化了軍力,於是打算借此討伐石勒,因此不顧箕澹和衞雄的勸阻,動用所有軍力,派箕澹率二萬作前鋒,自己則進屯廣牧為箕澹聲援。石勒以箕澹部眾遠道而來而筋疲力竭,而且烏合之眾,號令不齊,不難應付,決意迎擊。石勒於是在山中設下伏兵,自己率兵與箕澹作戰,然後向北退兵引箕澹深入,與伏兵夾擊箕澹而大敗對方,箕澹北逃到代郡而韓據則棄城奔劉琨。此戰震動并州,守著治所陽曲的劉琨長史李弘竟以并州投降石勒,使得劉琨進退失據,唯有投奔幽州刺史段匹磾。

太興元年(318年),劉聰患病,徵石勒為大將軍、錄尚書事,受遺詔輔政,但石勒不受。劉聰於是又命石勒為大將軍、持節鉞,都督等如故,並增封十郡,又不受。不久劉聰死,太子劉粲繼位後不久便被靳準所殺,自稱漢天王。石勒於是命張敬率五千兵作前鋒,自己親率五萬兵討伐靳準。石勒進據襄陵北原,羌羯四萬多個部落向石勒投降,靳準數度挑戰都不能攻破石勒的防禦。十月劉曜北上討伐靳準,並於赤壁(今山西河津縣西北赤石川)即位為帝,任命石勒為大司馬、大將軍,加九錫,增封十郡,進爵為趙公。

隨後石勒進攻首都平陽,各族共十多萬部落都向石勒投降。十一月,靳準派卜泰向石勒請和,石勒將使者囚禁後送交劉曜,以示城內並無歸附劉曜之意。但劉曜卻由卜泰為他傳話,勸靳準迎接他到平陽。靳準考慮未決,於十二月被靳康等人所殺,推靳明為主,向劉曜請降。石勒見靳氏不向自己歸降,大怒,率軍進攻靳明,靳明出戰但被擊敗,於是閉門自守。不久石虎與石勒會合,共攻平陽,靳明向劉曜求救,劉曜派兵迎靳明出城。石勒則進平陽城,焚毀平陽宮室,遷城內渾儀、樂器到襄國,留兵戍守後返回襄國。

太興二年(319年)二月,石勒派左長史王脩獻捷報給劉曜,劉曜於是授予石勒太宰、領大將軍,進爵趙王,並加一系列特殊禮待,如同昔日曹操輔東漢的先例。劉曜讓王脩返回襄國後,石勒舍人曹平樂卻對劉曜說王脩前來的的目的是要探劉曜的虛實,王脩返回報告後,石勒就會進襲劉曜。當時劉曜實力的確大為損耗,聽到曹平樂的話後十分害怕王脩會向石勒報告他的虛實,於是追還王脩並殺害王脩,原本授予石勒的官位、封爵及禮遇亦擱置。王脩副手劉茂卻成功逃脫,到石勒於三月回到襄國時就報告王脩之死,石勒大怒:「我事奉劉氏,盡心做得比起人臣的本份更有餘了。他們的基業都是我打下來的,今日得志了竟想來謀算我。趙王、趙帝,我自己也能給自己,哪用得著由他們賜予!」自此與前趙結了仇怨。

當年十一月,石勒稱大將軍、大單于、領冀州牧、趙王,於襄國即趙王位,正式建立後趙,稱趙王元年。

雖然石勒於建興二年(314年)殺害王浚,取得薊縣,但不久石勒所命駐守薊縣的劉翰背叛石勒而歸附段匹磾,段匹磾於是進據薊縣。然而,因段匹磾多次與段末柸相攻,又於太興元年(318年)殺死劉琨,使得大批胡人和漢人投奔邵續、段末柸或石勒,導致實力大減。段匹磾於次年因石勒將領孔萇進攻幽州,不能自立,因而投奔晉冀州刺史邵續還據有的厭次。至太興三年(320年)段末柸再擊敗段匹磾,段匹磾與邵續聯手追擊段末柸並擊敗他,隨後就與弟弟段文鴦北攻段末柸弟弟駐守的薊城。此時,石勒知道邵續勢孤,於是派石虎進攻厭次,最終生擒出城迎擊的邵續,但厭次城尚由邵續子邵緝等人據守。段匹磾此時回軍,尚離厭次城八十里時就聽聞邵續被擒的消息,于是部眾潰散,石虎也前來襲擊,只因段文鴦奮戰才得以進入厭次城。

太興四年(321年),石勒又派石虎和孔萇進攻厭次,段文鴦力戰被擒,段匹磾無力抵抗,試圖南奔東晉又不行,亦被石虎所捕。至此,晉朝於河北的各個藩鎮皆被攻陷。

建興元年(313年),司馬睿以祖逖為奮威將軍、豫州刺史,祖逖由此開始收復中原的行動,並進據譙城。太興二年(319年)豫州一塢主陳川與祖逖相爭但不敵,於是向石勒投降,祖逖因此討伐陳川,石勒則派石虎率兵救援,將祖逖擊敗,祖逖敗退至淮南。但祖逖於下一年就發動反擊,擊敗守著陳川故城的將領桃豹,並多次邀擊當地的後趙軍隊,當地留戍的後趙兵鎮深為困擾,很多都歸附祖逖。

因為祖逖擅於安撫,不但黃河以南地區的人民歸附祖逖,連石勒根據地河北的塢主也向祖逖報告後趙的情況,以至於石勒不敢以軍事力量強攻豫州,因而決定與祖逖修好,又允許兩地通商。當時祖逖牙門童建殺新蔡內史周密歸降石勒,石勒卻殺死童建並將首級送交祖逖。而祖逖也不接納背叛後趙而歸降的人,因此兩國邊境安定,兗、豫二州人民得以休息,但不少人其實都有雙重身份,同時歸屬東晉與後趙。

實際上,祖逖一直未忘北伐,他將通商獲得的利錢用來準備軍需物資,而且又修繕虎牢城,瞭望四方,並建立壁壘,作為守護豫州土地的堡壘。但壁壘未建成祖逖就死去。永昌元年(322年),石勒因祖逖已死而再度南侵,接替祖逖的祖約不能抵抗,南退至壽春,石勒於是留兵駐屯豫州,豫州再次混亂,再次進入後趙的勢力範圍。同時石勒派兵侵擾徐、兗二州,東晉駐守當地的部隊都只有南退,很多當地塢主都向石勒歸降。

太寧元年(323年),石勒派石虎攻滅一直割據青州的曹嶷,盡有青州。

太宁二年(324年),後趙司州刺史石生進攻前趙河南太守尹平並殺害他,而且掠奪了新安縣五千多戶人。自此開始兩國之間的戰事,作為兩國邊界的河東和弘農兩郡之間淪為戰場。次年西夷中郎將王騰殺并州刺史崔琨並以并州歸降前趙,屢敗於石生的晉司州刺史李矩、穎川太守郭默等也遣使依附前趙,於是前趙大舉進攻後趙。但前趙所派的劉岳被石虎擊敗,遭生擒和坑殺九千餘人,王騰也被石虎攻滅,李矩等被擊敗而南奔東晉,大量部眾歸降後趙。戰後後趙盡有司、豫、徐、兗四州之地。

太和元年(328年),石虎攻蒲阪,前趙帝劉曜親率全國精兵救援蒲阪,大敗石虎,於是乘勢進攻石生鎮守的洛陽,以水灌城,同時又派諸將攻打汲郡和河內郡,後趙舉國震驚。石勒見此,不顧程遐的勸阻執意親自救援洛陽,於是命桃豹、石聰、石堪等到滎陽會合,自己領兵直攻洛陽金鏞城。及至十二月,石勒與後趙諸軍於成皋集合,發現劉曜竟不設守軍,於是輕兵潛行。劉曜直至石勒渡過黃河後才開始準備防禦,從前線捕獲的羯人口中知道石勒親率大軍前來進攻後更為害怕,於是解圍而於洛西列陣。石勒在開始進攻之時曾說:「劉曜設大軍於成皋關防禦,是他的上策;列兵於洛水阻截則次之;坐守洛陽,就會讓我生擒了。」見劉曜列陣於洛西,石勒十分高興,認為必勝無疑,隨後就與石虎及石堪、石聰分三道夾擊劉曜,最終大敗前趙,更生擒劉曜,押送到襄國。

次年,留守長安的前趙太子劉熙知道劉曜被擒後大驚,於是放棄長安而西奔上邽,各征鎮都棄守防地跟隨,導致關中大亂,前趙將領以長安城歸降後趙,石勒又派石虎進攻關中的前趙殘餘力量。終於當年八月,前趙劉胤率大軍反攻長安時被石虎擊敗,前趙一眾王公大臣都被石虎所捕,同年石勒亦殺劉曜,前趙亡。石勒又於咸和二年(327年)派石虎擊敗代王拓跋紇那,逼得對方徙居大寧迴避其軍事威脅。至此後趙除前涼、段部鮮卑的遼西國及慕容鮮卑的遼東國三個政權外幾乎佔領整個中國北方。

太和三年(330年)二月,石勒稱大趙天王,行皇帝事,並設立百官,分封一眾宗室。至九月,石勒正式稱帝。

石勒稱帝後,於次年四月到鄴城,打算營建鄴城新宮,如張賓昔日所言,以其作為新的都城。當時廷尉續咸大力反對,石勒堅決不納;後中山郡有洪水災害,有百多萬根大木頭隨水沖到堂陽,石勒視此為上天協助自己營建鄴都,於是正式施行,自己親自視察工程。

石勒在稱帝時立了兒子石弘為皇太子,石弘愛好文章,對儒士親敬,並沒有石勒的強悍。然而當時任太尉、尚書令石虎因為戰功顯赫,掌有重兵和實權,徐光和程遐都認為一旦石勒去世,石弘不能駕馭石虎;同時又因石虎怨恨二人,二人擔心一旦石虎奪權會誅滅二人及其宗族,於是多次向石勒進言,要求強化太子權力,讓太子親近朝政,並削弱石虎權力。石勒最終命太子省批核上書奏事,並由中常侍嚴震協助判斷,只有征伐殺人的大事才送交石勒裁決。於是嚴震權力高漲,石虎則失勢,心有不滿。但石勒始終沒有聽從二人除去石虎的建議。

建平三年(332年),石勒到鄴城,到石虎的府第中,石勒知道石虎的不滿,於是允諾皇宮建成後會為他建設新府第,以此作安撫。但其實石虎自太和三年(330年)石勒稱天王時將大單于位封給石宏就十分不滿;對於咸和元年(326年)石勒讓石弘駐鎮鄴城和修建鄴城三臺時逼遷其家室的事也懷恨在心。

石勒於建平四年(333年)患病,石虎入侍並詔不許親戚大臣見石勒,因此無人知道石勒的病況。後又矯詔召命石勒用以防備石虎而出為外藩的秦王石宏及彭城王石堪到襄國,將他們留在襄國,即使石勒知道後立刻命二人回到駐地,石虎仍然不讓他們回去,更騙石勒說二人已在歸途上。七月戊辰日(8月17日),石勒逝世,享年六十歲。廟號高祖,諡號明皇帝,葬於高平陵。

石勒重視教育,在段部鮮卑和烏桓都相繼歸附支持自己,王浚勢弱,領下司州、冀州等地安定,人民開始繳納租稅時,在當地設立太學,以明經善書的官吏作文學掾,選了部下子弟三百人接受教育。後來,石勒又在襄國增置宣文、宣教、崇儒、崇訓等十多間小學,選了部下和豪族子弟入學。石勒更曾親臨學校,考核學生對經典意義的理解,成績好的就獲獎賞。

石勒稱趙王後,命支雄和王陽為門臣祭酒,專掌胡人訴訟,命張離、劉謨等人為門生主書,專掌胡人出入,且禁制胡人欺侮衣冠華族,以胡人為國人。另又遷徙三百家士族到襄國,置崇仁里讓他們聚居,又置公族大夫統領,實行胡漢分治。

石勒亦重視修史工作,命任播、崔濬為史學祭酒,又命記室佐明楷、程陰、徐機撰寫《上黨國記》,中太夫傅彪、賈蒲、江軌撰寫《大將軍起居注》,參軍石泰、石同、石謙、孔隆撰寫《大單于志》。稱帝後又擢升五個太學生為佐著作郎,記錄時事。

石勒實行考試機制,初建五品,由張賓領選舉事。後又定九品,命左右執法郎典定士族,並且副任選舉職能。又令僚佐及州郡每年都舉秀才、至孝、廉清、賢良、直言、武勇之士各一人。後來更以王波為記室參軍,典定九流,始立秀、孝試經的制度。又於稱帝後命各郡國設立學官,每郡都置博士祭酒二人,學生一百五十人,經三次考試後才畢業入仕。

石勒見百姓久經戰亂,社會秩序剛剛恢復,資源不足,於是下令禁止釀酒,祭祀時都只用發酵一晚的甜酒。數年以後就再沒人釀酒了。

石勒又命人重訂度量衡。

石勒在北方推度耕作,以右常侍霍皓為勸課大夫,與典農使者朱表及典農都尉陸充等巡核各州郡,核實戶籍,鼓勵農桑。讓收獲最多的人爵五大夫。

石勒感恩,並會作出報答。例如郭敬在早年曾經對他有恩,接濟過他。後來石勒在上白攻滅乞活軍將領李惲時重遇郭敬,竟立刻下馬抓著他的手,說:「今日相遇,是天意呀!」於是賜他衣服車馬,署他為上將軍,更將原本打算坑殺的李惲餘眾賜給他作為部眾。劉琨曾送還石勒母親以圖招降石勒,雖然石勒拒絕,但仍以厚禮作回報;後來劉琨及石勒雖然互相敵對,但在石勒攻打北中郎將劉演時擒獲其弟劉啓,而劉演和劉啓都是劉琨的侄兒,石勒此時仍然感謝劉琨讓他母子重聚的恩德,不但沒有殺死劉啓,還賜他田宅,命儒官教授他經典。

石勒下令禁止說「胡」字,更是所有忌諱字中懲罰最重者。但一次有胡人喝醉了,騎馬突入止車門,違反門禁,於是石勒在憤怒之下召責宮門小執法馮翥。馮翥見石勒十分恐懼,只顧申辯而忘了忌諱,說:「剛才有個醉了的胡人,騎馬進了門,我已經大聲喝止並攔住他,但都不能和他對話。」石勒聽後,沒有憤怒,反而笑說:「胡人正就是難以與之對話的了。」並寬恕了他的罪。

石勒雖不識字,但喜好文史,即使行在軍旅仍常聽漢儒講讀中國歷史,隨時發表自己的見解。一次聽到酈食其勸劉邦得天下後分封六國諸王,大喊糟糕,懷疑劉邦怎能平定天下。後來知道張良勸阻,才連忙說「賴有此耳。」可見他天資之高,英明賢達。

石勒曾在夜間微服出行,到營衞時曾以錢財賄賂守門者讓他出去,但永昌門門候王假卻不受金錢,更打算收捕他,只因隨從及時來到才未被捕。下一日石勒就召王假為振中都尉,賜爵關內侯。

石勒曾問大臣徐光他能比作昔日哪位君主,徐光說石勒神謀武略,比漢朝開國君主劉邦更高,而劉邦以後再沒有人能和石勒比較。石勒笑言徐光說得太誇張,自我評價道:「我若果與劉邦同時,就當作他的臣下,與韓信、彭越皆為其將;若果與漢光武帝劉秀同時,就會與他爭奪中原,不知鹿死誰手。大丈夫行事,應該磊磊落落,如日月皎潔,絕不可以像曹操、司馬懿那樣欺負孤兒寡婦,用奸計奪取天下。」足見石勒尊崇劉邦、劉秀白手興家而貶抑曹操和司馬懿的奪權行為。

\subsubsection{太和}

\begin{longtable}{|>{\centering\scriptsize}m{2em}|>{\centering\scriptsize}m{1.3em}|>{\centering}m{8.8em}|}
  % \caption{秦王政}\
  \toprule
  \SimHei \normalsize 年数 & \SimHei \scriptsize 公元 & \SimHei 大事件 \tabularnewline
  % \midrule
  \endfirsthead
  \toprule
  \SimHei \normalsize 年数 & \SimHei \scriptsize 公元 & \SimHei 大事件 \tabularnewline
  \midrule
  \endhead
  \midrule
  元年 & 328 & \tabularnewline\hline
  二年 & 329 & \tabularnewline\hline
  三年 & 330 & \tabularnewline
  \bottomrule
\end{longtable}

\subsubsection{建平}

\begin{longtable}{|>{\centering\scriptsize}m{2em}|>{\centering\scriptsize}m{1.3em}|>{\centering}m{8.8em}|}
  % \caption{秦王政}\
  \toprule
  \SimHei \normalsize 年数 & \SimHei \scriptsize 公元 & \SimHei 大事件 \tabularnewline
  % \midrule
  \endfirsthead
  \toprule
  \SimHei \normalsize 年数 & \SimHei \scriptsize 公元 & \SimHei 大事件 \tabularnewline
  \midrule
  \endhead
  \midrule
  元年 & 330 & \tabularnewline\hline
  二年 & 331 & \tabularnewline\hline
  三年 & 332 & \tabularnewline\hline
  四年 & 333 & \tabularnewline
  \bottomrule
\end{longtable}


%%% Local Variables:
%%% mode: latex
%%% TeX-engine: xetex
%%% TeX-master: "../../Main"
%%% End:

%% -*- coding: utf-8 -*-
%% Time-stamp: <Chen Wang: 2021-11-01 11:54:33>

\subsection{海阳王石弘\tiny(333-334)}

\subsubsection{生平}

石弘(314年-335年),字大雅,是中國五胡十六國時代後趙的君王。上黨武鄉(今山西榆社)人,后赵明帝石勒二子,母程氏。

史載石弘「幼有孝行,以恭謹自守」,受经于杜嘏,诵律于续咸。石勒觉得他不似将门之子,派刘征、任播授以兵书,王阳教之击刺。石勒病重时,中山王石虎与石弘、中常侍严震在宫中侍候,石虎矫诏断绝内外消息。建平四年(333年)九月,石勒一死,石弘繼位,立嫡母劉氏為皇太后。石虎下達第一個“詔令”,將石弘舅父右光祿大夫程遐、中書令徐光論罪誅斬,拜石虎為丞相、魏王、大單于,加九錫,以魏郡等十三郡為邑。石弘恐懼丞相石虎,欲讓位於石虎。石虎拒絕:“君薨而世子立,臣安敢亂之!”遂即位,拜石虎为丞相。

刘太后与石勒养子彭城王石堪谋除石虎,擁皇弟南陽王石恢為盟主。石堪單騎出逃,直奔兗州。到達廩丘時,因事機不密,逮送至襄國,被活活烤死。劉太后被石虎发现参与其中,遭废黜弒害,石虎改尊石弘生母程氏为皇太后。河東王石生在關中起兵,石朗在洛陽起兵,聲言滅石虎。石虎擒下石朗,他先砍掉石朗的雙腳,再斬首。長安一戰,石虎大敗,“枕尸三百余里”,此時石生同盟的鮮卑人竟然反叛,石虎重振軍勢,石生被部下斬首,獻給石虎。延熙元年(334年)十月石弘持玺绶向石虎表明願意禅位。石虎说:“天下人自当有议,何为自论此也!”意思是只能自己逼石弘退位,而不能接受石弘禅位。石弘哭着回宫对程太后说:“先帝真要灭种了!”不久石虎称石弘居丧不孝,废为海阳王,与程太后及弟秦王石宏、石恢一同幽禁崇训宫,不久皆殺之。

\subsubsection{延熙}

\begin{longtable}{|>{\centering\scriptsize}m{2em}|>{\centering\scriptsize}m{1.3em}|>{\centering}m{8.8em}|}
  % \caption{秦王政}\
  \toprule
  \SimHei \normalsize 年数 & \SimHei \scriptsize 公元 & \SimHei 大事件 \tabularnewline
  % \midrule
  \endfirsthead
  \toprule
  \SimHei \normalsize 年数 & \SimHei \scriptsize 公元 & \SimHei 大事件 \tabularnewline
  \midrule
  \endhead
  \midrule
  元年 & 334 & \tabularnewline
  \bottomrule
\end{longtable}


%%% Local Variables:
%%% mode: latex
%%% TeX-engine: xetex
%%% TeX-master: "../../Main"
%%% End:

%% -*- coding: utf-8 -*-
%% Time-stamp: <Chen Wang: 2019-12-18 17:43:31>

\subsection{武帝\tiny(334-349)}

\subsubsection{孝帝生平}

石寇覓(3世紀-?),是後趙武帝石虎的父親。他早逝,因此石虎被石勒的父親石周曷朱收养,所以又有人稱石虎是石勒的弟弟。

石虎稱帝後,追封他為皇帝,諡號孝皇帝,廟號太宗。

\subsubsection{武帝生平}

趙武帝石虎(295年-349年5月26日),字季龍,上黨武鄉(今山西榆社)人。中國五胡十六國時代中,後趙的第三位皇帝。廟號太祖,諡號武帝。石虎是後趙開國君主石勒的侄兒。石虎生性殘忍,發家前,不僅用殘酷的手段先後殺死兩位妻子,即使在軍隊中如果遇到與他一樣強健的戰士,他會以打獵戲鬥為由,借機將對手殺死,以解心頭之快;戰鬥中,對俘獲的俘虜,不分好壞,不分男女一律坑殺,很少有俘虜生還。

333年,石勒駕崩,其皇位由兒子石弘繼承。因石虎掌握兵權勢大,石勒妻刘太后與養子彭城王石堪擁立石勒子南陽王石恢欲舉兵反對石虎,不幸事洩,劉太后被殺,石堪被捕活活烤死,石恢被召回,咸康元年(334年)十月石弘持璽綬向石虎表明願意禪位,石虎拒绝。十一月,石虎称居摄赵天王,石弘被廢為海陽王,同年石虎殺海陽王石弘、弘母程氏、石弘弟秦王石宏、南陽王石恢。至335年,其首都由襄國(今中國河北邢台)遷至鄴(今河北邯郸市臨漳县城西南20公里邺城遗址),並特地派人到洛陽將九龍、翁仲、銅駝、飛廉轉運到鄴裝點宮殿。337年4月11日(二月辛巳),石虎称大赵天王,349年2月4日(正月初一辛未朔)正式即皇帝位。同年5月26日(四月己巳),患病而死,随后,他的儿子争夺皇位,后赵很快灭亡。石虎在位期間,表現了其殘暴好色的一面,如史書載石虎曾經下達過一條命令:全國二十歲以下、十三歲以上的女子,不論是否嫁人,都要做好準備隨時成為他後宮佳麗中的一員,「百姓妻有美色,豪勢因而脅之,率多自殺」,因此被評為五胡十六國中的暴君。

生性殘暴的石虎,少年時喜歡用彈弓打人為樂。十八歲時,由於其武藝超凡且勇猛過人,因此受到石勒的寵信,被封為征虜將軍。石勒其後又為石虎納聘將軍郭榮的妹妹為妻,但石虎心儀的是當時的雜技名角鄭櫻桃。於是便把郭氏殺死,而後迎娶鄭氏。之後,石虎又娶了崔氏,但崔氏最後因鄭氏的挑撥而死於石虎手中。

在軍中,凡是比石虎有才藝或有武藝的,石虎就會設法把他們殺死,死於他手上的人不可計數。石虎是好殺的人,每次攻下一座城後,不論男女都一律殺死。一次,石虎攻下青州後又下令屠城。此次血腥屠城,僅餘七百多人保全性命。

太和三年(330年)二月,石勒称大赵天王,行皇帝事;以妃刘氏为王后,世子石弘为皇太子,程遐为右仆射、领吏部尚书。中山王石虎怒,秘密对长子齐王石邃说:“我亲冒矢石随主上征战二十余年,是成大赵之业者,应该做大单于,主上却授予‘黄吻婢儿’,想起来就令人气塞,不能寝食!待主上晏驾之后,我不会给他留种。”

石勒臨終前,石虎威迫太子石弘把曾勸石勒除掉自己的大臣程遐和徐光逮捕入獄并杀死。又命兒子石邃率兵入宿衛,文武百官害怕不已,太子石弘也嚇得連忙對石虎說道自己不是治天下的人材,石虎才是真正的天子。但石虎明白石勒屍骨未寒,就這樣強登上皇帝只會眾叛親離,並受後世人的唾罵。因此寧願有點耐性,演齣曹操的「挾天子以令諸侯」的戲,由這位太子登位。

石弘坐上寶座後,成為了傀儡皇帝。石弘登基後便被石虎所逼,将程遐、徐光论罪诛斩,封石虎為丞相、魏王、大單于,再封土地,封邦建土。而他的三名兒子都被封為擁有軍權的職位,至於他的親人和親信都放排在有大權的職位上,而之前石勒的文武百官就放置在毫無權力的閑職上。這時後趙已真正的形成「挾天子以令諸侯」的局面。刘太后与石勒养子石堪合谋起兵拥戴石弘的弟弟石恢为盟主,石堪兵败被杀,石恢被征召回京,刘太后被石虎废黜杀害。石弘生母程氏被尊为太后,也没有实权。延熙元年(334年)十月石弘持玺绶向石虎表明愿意禅位。石虎说:“天下人自当有议,何为自论此也!”意思是只能自己逼石弘退位,而不能接受石弘禅位。石弘哭着回宫对程太后说:“先帝真要灭种了!”不久石虎称石弘居丧不孝,废为海阳王,自称天王,並把石弘、程太后和石弘的弟弟石宏、石恢都幽禁于崇训宫,旋即殺死他們。

石虎稱天王後,石邃為太子(之前为魏太子),並開始他極為奢侈的統治。石虎不顧人民負擔到處征殺,使人民的兵役和力役負擔相當重大,他又下令凡是有免兵役特權的家族,五丁取二,四丁取其二,而沒有特權的家族則所有丁壯都需服役。為了攻打東晉,在全國征調士兵的物品:每五人出車一乘、牛兩頭、米穀五十斛、絹十份,不交者格殺勿論。無數的百姓為了安全,不得不把自己的子女賣掉。

後趙建武二年(336年),石虎為了裝飾鄴城,令牙門將張彌把洛陽的鐘虞、九龍、翁仲、銅駝、飛廉等相生物運到去鄴城。在運送途中,一隻鐘虞沒入了黃河,於是張彌便下令三百多名人潛到水中,把鐘虞繫上繩,再利用百多頭牛和許多架轆轤把鐘虞拉上來,之後就地造了可裝萬斛的大船,把這些相生運過黃河。其後又製造了特大的車子以把相生運送到鄴城,這次的行動單是運送就足足用了人民千千萬萬的勞力和血汗了。

在鄴城以西三里,有石虎所建的桑梓苑,苑內臨漳水修建了很多座豪華的宮殿,下令从民间强行掠夺十三岁至二十岁的女子三万余人。仅在345年一年间,各郡县官吏为搜罗美女上交差事,公然抢掠貌美的有夫之妇九千余人,不忍受夺妻之辱而反抗的男人均遭残杀,被夺女子为避免受辱也大多自杀,一大批家庭夫妻离散,家破人亡。但石虎征集女人倒不完全是好色,石虎内置女官十有八等,教宫人星占及马步射。置女太史于灵台,仰观灾祥,以考外太史之虚实(《晋书·石季龙载记》)。石虎还鉴于东汉太监专权的危害,不信任太监,因此宫中没有太监,相关职务只能由女人充当。苑內養有奇珍異獸,石虎經常在此遊玩設宴。從襄國至鄴城的二百里內,每隔四十里使建一行宮,每宮都有一位夫人,數十位的侍婢居住,由黃門官守門。

而在浴室上,更是別出心裁:在皇后浴室中,門窗都是由木刻成的鏤孔圖案,石虎就是在這兒和皇后梳洗。而每年的4月8日,在這裏精工製造的九龍吐水浴太子之像。在太武殿前,溝的中間有多層以紗等的「過濾器」。

「鳳詔」也是石虎的發明之一,石虎處理政事時會和皇后一起坐在高高在上的樓觀上,並用五色紙上寫下詔書,把詔書放在一只由木雕刻成、外施漆畫、金腿的「鳳凰」口中。金鳳凰繫在轆轤牽引的繩上。當下詔時,待人把轆轤搖動,「鳳凰」就像從天空飛下來般,大臣們都要跪下接詔。

每隔不久,石虎便會大會群臣,每次都頭戴通天冠、身佩玉璽、循周禮的規定禮樂一番,然後觀賞雜技表演,群臣大會幾乎都有美酒佳釀給自己和群臣所飲用。殿上掛著了大鐵燈一百二十支。在燈下有數千戴金銀佩飾的宮女和石虎觀看表演。在殿外,三十部鼓吹同時演奏,鼓樂震天,場面極為震撼。

石虎好射獵,但因體胖而無法騎馬,因而改為用獵輦。而他的獵輦裝有豪華的華蓋羽葆,由二十人推行,座下有轉軸裝置,可以根據獵物的所在地轉動。在出獵時,石虎會戴上由金鏤織成的合歡帽、穿上合歡褲,手拿著弓箭。而石虎為了方便行獵,於是把黃河以北的大片良田為獵區,派御史監督,规定除自己外有敢在獵區獵獸者处死。而这“犯兽”的刑法,又被各官员用来欺压百姓,若百姓家有美女或好的牛马等家畜,官员要求不给,就诬陷其“犯獸”,因此被判死刑者甚多。

石虎像他伯父石勒一样崇拜大和尚佛圖澄,石勒因信佛圖澄之言而減少了很多殺虐。有次石虎向佛圖澄問甚麼是佛法,佛圖澄只說了四字:「佛法不殺」.石虎沒有聽取佛圖澄的勸告,後來倒是聽了一個叫吳進的假和尚說胡人的氣數已衰,而晉人的氣數開始恢復,一定要苦役晉人才能壓著他們的氣數。結果石虎下令強徵鄴城附近各郡的男女百姓十六萬多人、車十萬乘在鄴城東修華林苑,並圍苑建數十里的長牆。

在中國歷史上還記載著石虎父子的相互殘殺。

事緣石虎兒子石邃不滿父親寵愛其餘的兩個兒子石宣和石韜,漸漸地,這種不滿轉化為仇恨,對父亲石虎恨之入骨,恨不得弒父奪位。石虎得知後,把石邃的手下李顏捉來審問,李顏嚇得不知如何是好,便一五一十地都事情告訴石虎:石邃密謀殺石宣和弒石虎奪位。石虎得知後把李顏及其家人三十多人斬首處死,再把石邃幽禁於東宮。石邃被幽禁後仍然目中無人,石虎一怒之下,下令把石邃和他的妻子、家人殺死,再塞進同一口棺材內,同一時間又把石邃的黨羽二百多人殺死。

石邃死後,石宣為皇太子,石宣之母杜昭儀為天王皇后,鄭櫻桃廢為東海太妃。同时又让石韬掌握军政大权,打算让石宣和石韬之间达成一定的平衡。结果却引发新一轮内讧。

到了其後,石宣因不滿其父石虎較寵愛石韜而要除掉石韜。不久之後,兩兄弟經常發生衝突,石宣於是把石韜砍掉手足、雙眼刺爛、破肚慘死。石宣並計劃在石韜的喪禮上弒父,以奪皇位。

石虎得知愛兒石韜死了,昏迷了好一段時間,他本想出席兒子的喪禮,幸而大臣提醒,沒有出席喪禮。後來,石虎得到知情人的報告,得知皇太子石宣殺了石韜。憤怒到極點的石虎在设计控制石宣后,下令用鐵環穿透石宣下巴鎖著,又將他的飯菜倒入大木槽,使石宣進食時像豬、狗般。石虎逼石宣用舌頭舐著殺石韜的劍上的血,石宣發出了震動宮殿的哀聲。石虎下令在鄴城城北埋起柴堆,上面設置了木竿、竿上安裝了轆轤。並讓石韜生前最寵的宦官,郝稚和劉霸二人拽著石宣的舌头和頭髮,沿著梯子拉上柴堆,之後用轆轤把他絞起來,再用一模一樣的方法向石宣施刑。當石宣已奄奄一息時在柴堆四處點火,石宣被燒成了灰燼。這還未能平熄石虎的怒火,再下令把灰燼分散到名門道中,任人、馬、馬車的輾踏,又將石宣的妻、子九人殺死,又把石宣的衛士、宦官等數百人車裂,將屍體投進漳河。石宣的一个年幼的儿子抱着石虎的大腿求饶,石虎心生怜悯想赦免但大臣们却将其夺走处死,石虎的腰带都被孙子扯断。东宫卫兵十余万被流放边疆,途中举行暴动,石虎急忙调集重兵镇压了下去。但后赵统治基础动摇了。

连杀两位太子后,太尉张举认为燕公石斌、彭城公石遵都有武艺文德,建议从二人中选择储君。但戎昭将军张豺先前曾将刘曜的女儿献给石虎,生有齐公石世,于是他说服石虎立石世,这样刘氏成为太后,他可以辅政。石虎说:“太子二十多岁就想弑父,石世才十岁,等他二十岁了,我已经老了。”于是与张举、李农定议,敕令公卿上书请立石世为太子,于是立石世为太子,其母刘氏为皇后。

石虎病重时,以石遵为大将军,镇关右,石斌为丞相、录尚书事,张豺为镇卫大将军、领军将军、吏部尚书,同受遗诏辅政。刘皇后怕石斌辅政不利于石世,就与张豺合谋,派使者诈称石虎病愈,石斌性好酒猎,于是又恣意而为。刘皇后便矫命称石斌无忠孝之心,免其官,以王归第,派张豺弟张雄率龙腾五百人看守。石遵从幽州来朝,被打发走,石虎知道后说“恨不见之”。一次石虎驾临西阁,龙腾将军、中郎二百余人列拜于前,说宜令燕王石斌入宿卫,典兵马,也有请求以石斌为皇太子。石虎不知石斌已被罢官囚禁,命召石斌来,左右说石斌饮酒得病不能入。石虎又命以辇迎之,要将其玺绶交给他,最后也没人前去。不久石虎昏眩入内。张豺让张雄等矫石虎命杀石斌,刘皇后又矫命以张豺为太保、都督中外诸军、录尚书事,加千兵百骑,一依霍光辅汉故事。

石虎死后,石世继位,不久就被推翻,石虎诸子石遵、石鉴、石祗相继登基,又相继被杀。石虎死后三年,后赵就灭亡了。

\subsubsection{建武}

\begin{longtable}{|>{\centering\scriptsize}m{2em}|>{\centering\scriptsize}m{1.3em}|>{\centering}m{8.8em}|}
  % \caption{秦王政}\
  \toprule
  \SimHei \normalsize 年数 & \SimHei \scriptsize 公元 & \SimHei 大事件 \tabularnewline
  % \midrule
  \endfirsthead
  \toprule
  \SimHei \normalsize 年数 & \SimHei \scriptsize 公元 & \SimHei 大事件 \tabularnewline
  \midrule
  \endhead
  \midrule
  元年 & 335 & \tabularnewline\hline
  二年 & 336 & \tabularnewline\hline
  三年 & 337 & \tabularnewline\hline
  四年 & 338 & \tabularnewline\hline
  五年 & 339 & \tabularnewline\hline
  六年 & 340 & \tabularnewline\hline
  七年 & 341 & \tabularnewline\hline
  八年 & 342 & \tabularnewline\hline
  九年 & 343 & \tabularnewline\hline
  十年 & 344 & \tabularnewline\hline
  十一年 & 345 & \tabularnewline\hline
  十二年 & 346 & \tabularnewline\hline
  十三年 & 347 & \tabularnewline\hline
  十四年 & 348 & \tabularnewline
  \bottomrule
\end{longtable}

\subsubsection{太宁}

\begin{longtable}{|>{\centering\scriptsize}m{2em}|>{\centering\scriptsize}m{1.3em}|>{\centering}m{8.8em}|}
  % \caption{秦王政}\
  \toprule
  \SimHei \normalsize 年数 & \SimHei \scriptsize 公元 & \SimHei 大事件 \tabularnewline
  % \midrule
  \endfirsthead
  \toprule
  \SimHei \normalsize 年数 & \SimHei \scriptsize 公元 & \SimHei 大事件 \tabularnewline
  \midrule
  \endhead
  \midrule
  元年 & 349 & \tabularnewline
  \bottomrule
\end{longtable}


%%% Local Variables:
%%% mode: latex
%%% TeX-engine: xetex
%%% TeX-master: "../../Main"
%%% End:

%% -*- coding: utf-8 -*-
%% Time-stamp: <Chen Wang: 2021-11-01 11:55:43>

\subsection{义阳王石鑒\tiny(349-350)}

\subsubsection{少帝石世生平}

石世(339年-349年),字元安,十六國時期後趙國君主,後世稱「少帝」,為石虎之子。母為前趙帝劉曜幼女安定公主,後趙太和二年(329年),前趙被後趙所滅,石虎將年僅12歲的安定公主強占為妾,十年後安定公主生了石世。石虎在天王位時,石世被封為齊公,安定公主封為昭儀。

後趙建武十三年(348年),石虎廢殺了太子石宣之後,受石世之母昭儀劉氏及她的死黨將軍張豺的教唆鼓動,將劉氏立為皇后,年僅10歲的石世立為太子。次年(349年),石虎正式稱帝,並改元太寧。不久,石虎去世,石世遂即帝位,然而大權皆握在劉太后及張豺之手。

彭城王石遵得知石虎去世後,立即率軍攻回都城鄴城(今河北臨漳縣),殺張豺。數日後,石遵自即帝位,石世被改封為譙王,劉太后被廢為太妃,石世在位僅33日。不久,石世與劉太妃皆被殺。

\subsubsection{彭城王石遵生平}

石遵(?-349年),字大祗,十六國時期後趙皇帝,為石虎第九子,石世之兄,母為鄭櫻桃。後趙建平三年(333年),後趙帝石勒去世,石虎掌控大權,石遵當時被封為齊王。建武三年(337年),石虎改稱天王後,被降封為彭城公。太寧元年(349年),石虎稱帝後,再被進封為彭城王。

太尉张举曾建议石虎立石遵或燕公石斌為太子,然而因昭儀劉氏及戎昭将军張豺從中作梗,石虎遂立劉氏之子石世為太子。太寧元年(349年),石虎病重,石遵被任命為大將軍,鎮守關右。石遵从幽州来朝,被打发走,石虎知道后说“恨不见之”。

不久,石虎去世,石世即位,大權握於劉太后及張豺之手。石遵與姚弋仲、蒲洪、石閔等人商量後決定反擊,遂以石閔為前鋒,攻打都城鄴(今河北臨漳縣),不久,鄴城陷,劉太后不得已只好任命石遵為丞相、领大司马、大都督中外诸军、录尚书事,加黄钺、九锡,增封十郡。數日後,石遵假刘太后令廢石世、立石遵为帝,假装再三辞让后在群臣劝进下自登帝位于太武前殿。封石世为谯王,邑万户,待以不臣之礼,废刘太后为太妃,不久皆杀之。石遵兄沛王石冲讨伐石遵,石遵派将军王擢骑马以书信说和不成,派石闵、司空李农击败石冲于平棘,在元氏俘获石冲并赐死。

石遵可能没有儿子,當初在謀反前,曾答應事成後以石閔為太子,可是等到石遵登帝位後,太子卻是石遵之姪石衍,因此石閔頗為不滿,有反叛之意。經過旁人提醒,石遵遂召其兄石鑒、弟石苞與母親鄭櫻桃等人商議,不料會後卻被石鑒出賣,將此事告知石閔。不久,石閔即率軍入宮,派将军苏彦、周成率领披甲士兵三千人去南台的如意观抓石遵。石遵正在和女人弹棋,问周成:“造反的是谁?”周成说:“义阳王石鉴当立。”石遵说:“我尚且如此,石鉴又能支撑多长时间!”被殺,在位僅183日。

\subsubsection{义阳王石鑒生平}

石鑒(?-350年),字大郎,一作大朗,十六國時期後趙國君主,為石虎第三子,石遵、石世之兄。後趙建平三年(333年),後趙帝石勒去世,石虎掌控大權,石鑒當時被封為代王。建武三年(337年),石虎改稱天王後,被降封為義陽公。

建武五年(339年)九月,东晋征西将军庾亮镇武昌,让豫州刺史毛宝、西阳太守樊峻以一万精兵戍守邾城。石虎厌恶晋军如此动向,以夔安为大都督,率石鉴、养孙石闵、李农、张贺度、李菟五将军及兵五万人攻打荆、扬北境,以二万骑攻邾城。张贺度攻陷邾城,杀死六千人,又败毛宝于邾西,杀死万余人。赵军进犯江夏、义阳,毛宝、樊峻及东晋义阳太守郑进皆死。夔安等进围石城,被竟陵太守李阳所破才退兵。

太寧元年(349年),石虎稱帝後,再被進封為義陽王。

石鑒在鎮守關中的時候,賦役繁重,文武官員只要頭髮長得比較長,就會被拔下來做帽帶,有剩下的會給宮女,曾因為這種荒唐的行徑,被石虎召回都城鄴城(今河北臨漳縣)。

太寧元年(349年),石遵廢皇帝石世,自登帝位,石鑒被命為侍中、太傅。石遵因石閔有叛變之意,召两位兄弟石鑒、乐平王石苞與太后鄭櫻桃等人商議,不料會後石鑒出賣其他人,將此事告知石閔。不久,石閔即率軍入宮,殺石遵,石鑒因此被擁立為帝。石遵被杀时说:“我尚且如此,石鉴能长久吗?”

然而石鑒登位後,處處受制於大將軍石閔,於是派石苞和将军李松、张才暗殺之,然而卻事敗,他装作自己不知情,杀死石苞三人;后又鼓励将军孙伏都攻打石闵,不果,又对石闵说孙伏都谋反,命石闵讨灭。石閔知道石鑒有殺己之意,遂頒殺胡令,被殺的人共有20餘萬;并软禁石鉴于御龙观,派尚书王简、少府王郁率数千人看守,用绳子把食物吊给他。

次年(350年),完全控制國政的石閔將後趙國號改為魏(衛),石閔也将包括自己在内的后赵皇族改姓为李,並改年號為青龍。不久,石鑒為求擺脫控制,遂趁李閔外出作戰,秘密派宦官告知在外的將軍抚军将军张沈等,命他们趁虛攻都城鄴城,但宦官告知李閔此事,李閔因而回軍,石鑒遂被誅殺,在位僅103日。

\subsubsection{青龙}

\begin{longtable}{|>{\centering\scriptsize}m{2em}|>{\centering\scriptsize}m{1.3em}|>{\centering}m{8.8em}|}
  % \caption{秦王政}\
  \toprule
  \SimHei \normalsize 年数 & \SimHei \scriptsize 公元 & \SimHei 大事件 \tabularnewline
  % \midrule
  \endfirsthead
  \toprule
  \SimHei \normalsize 年数 & \SimHei \scriptsize 公元 & \SimHei 大事件 \tabularnewline
  \midrule
  \endhead
  \midrule
  元年 & 350 & \tabularnewline
  \bottomrule
\end{longtable}


%%% Local Variables:
%%% mode: latex
%%% TeX-engine: xetex
%%% TeX-master: "../../Main"
%%% End:

%% -*- coding: utf-8 -*-
%% Time-stamp: <Chen Wang: 2019-12-18 17:48:34>

\subsection{新兴王\tiny(350-351)}

\subsubsection{生平}

石祗(?-351年),中國五胡十六國時代中,後趙的皇帝。为石虎子。

石祗早年经历不详。初封新兴王。大将军武德王石闵掌权后,开始杀戮羯族,羯族人纷纷出逃投奔石祗。350年石祗听说其兄皇帝石鉴被冉闵(即石闵,恢复本姓)杀死,于是在襄国(今河北省邢台市)自立为帝,并起兵讨伐冉闵。351年二月自去帝号,称赵王,以求获得前燕支持助讨冉闵。四月,战败被部将刘显杀死,后赵灭亡。

\subsubsection{永宁}

\begin{longtable}{|>{\centering\scriptsize}m{2em}|>{\centering\scriptsize}m{1.3em}|>{\centering}m{8.8em}|}
  % \caption{秦王政}\
  \toprule
  \SimHei \normalsize 年数 & \SimHei \scriptsize 公元 & \SimHei 大事件 \tabularnewline
  % \midrule
  \endfirsthead
  \toprule
  \SimHei \normalsize 年数 & \SimHei \scriptsize 公元 & \SimHei 大事件 \tabularnewline
  \midrule
  \endhead
  \midrule
  元年 & 350 & \tabularnewline\hline
  一年 & 351 & \tabularnewline
  \bottomrule
\end{longtable}


%%% Local Variables:
%%% mode: latex
%%% TeX-engine: xetex
%%% TeX-master: "../../Main"
%%% End:



%%% Local Variables:
%%% mode: latex
%%% TeX-engine: xetex
%%% TeX-master: "../../Main"
%%% End:

%% -*- coding: utf-8 -*-
%% Time-stamp: <Chen Wang: 2019-12-19 09:47:53>


\section{前燕\tiny(337-370)}

\subsection{简介}

前燕(337年 - 370年)是十六國时代由鮮卑人首領慕容皝所建立的政權,至慕容儁正式稱帝建國,其國號為「燕」。其全盛时的统治地区包括冀州、兖州、青州、并州、豫州、徐州、幽州等部分。

以其所在地为战国时燕国旧地,故国号为“燕”。《十六国春秋》始用“前燕”之名,為區別同期的慕容氏諸燕,歷史學家遂袭用之。又以其王室姓慕容,又称为“慕容燕”,而其他慕容氏諸燕都不用这个称呼,「慕容燕」成为前燕的专称。

西晉時,慕容廆為鮮卑族慕容氏的首領,曾效忠西晉,與鮮卑族以外的民族作戰。後來其兒子慕容皝於337年自稱為燕王。342年擊敗了後趙的二十萬大軍,解除了來自中原的壓力,建都龍城(今遼寧省朝陽市)。東破夫餘及高句麗,攻滅鮮卑宇文部,成為遼西唯一的武裝勢力,為慕容儁攻占中原奠定了坚实的基礎。

352年,皝子慕容儁滅冉魏稱帝,遷都薊,并隨後的幾年平定了北方的局勢,於357年遷都鄴。其地「南至汝颍,東盡青齊,西抵崤黽,北守雲中」,與關中的前秦平分黃河流域。358年,慕容儁下令全國州郡檢查戶口,每戶僅留一丁,此外全部徵發當兵,擬拼集150萬大軍以滅東晉、前秦以統一天下。

360年正月,慕容儁在鄴檢閱軍隊,但隨即逝世。其子慕容暐即位,改元「建熙」,此後宮廷裡發生了一次內訌。

慕容儁死時命弟慕容恪輔政,後慕容恪阻止了宮廷的內訌。從360年慕容恪輔政到367年病死其間,為前燕政治較為穩定的時期。自建熙二年、東晉升平五年(361年),以前燕河內太守呂護倒戈反覆为导火索,前燕与东晋在中原展开了连绵的战事。

前燕建熙四年、東晉興寧元年(公元363年)前燕全面的攻势开始发动,四月,慕容忠攻滎陽(今河南滎陽東北),東晉滎陽太守逃到魯陽(今河南魯山)。建熙五年、東晉興寧二年(公元364年)二月,前燕李洪開始略地河南,四月,前燕攻許昌、汝南、陳郡,徙上述三地萬餘戶於幽州,遣鎮南將軍慕容塵屯許昌。七月,太宰慕容恪親自領兵攻打洛陽,東晉洛陽守軍战败逃离。建熙六年、東晉興寧三年(公元365年)三月,前燕攻克洛陽。

这一系列的战役后,前燕从东晋手中获得了中原的控制权。但东晋仍然不放弃收复失地的计划。东晋太和四年(公元369年)东晋大将桓温北伐,前燕一度陷入危机。不久,东晋军中绝粮,桓温被迫撤退,途中遭前燕军队伏击,损失三万余人,大败而归。

太宰慕容恪死後,太輔慕容评掌握大權。慕容评為人貪婪,並得太后所信任,得以掌握大權。

東晉乘慕容恪死時,由大司馬桓溫領兵北上,被慕容皝的兒子、少帝慕容暐之叔吳王慕容垂击败,慕容垂却被掌权者慕容评所猜忌。慕容垂被逼无奈,出走前秦,被苻坚收留。苻坚早就想消灭前燕,一直忌惮慕容垂,如今最大劲敌已经投誠,苻坚遂开始讨伐前燕的计划。前燕军团起初并未处于下风,但由于當權的慕容评为人贪鄙,致使军心离散,结果前燕15万主力部队被王猛所率领的前秦军歼灭。苻坚趁势率10万军队包围前燕的首都邺城。“散骑侍郎徐蔚等率扶余、高句丽及上党质子五百余人,夜开城门以纳坚军。”公元370年十一月,慕容暐逃出邺城,试图返回辽东的根据地龙城,中途被前秦军抓获,前燕灭亡。

\subsection{武宣帝生平}

慕容\xpinyin*{廆}(269年-333年6月4日),字弈洛瓌,昌黎棘城(今遼寧義縣)人。晉朝時鮮卑人,慕容部首領慕容涉歸之子,前燕建立者慕容皝之父,吐谷渾第一代首領慕容吐谷渾是其庶兄。

慕容廆年紀輕輕就已經長得魁梧高大,才能出眾且有雄大抱負。張華出鎮北方時慕容廆曾去拜訪他,雖然當時慕容廆仍是兒童,但張華卻十分欣賞他,更與他結交。

西晉武帝太康四年(283年)慕容涉歸死,其弟慕容删篡奪政權,更意圖殺害慕容廆,慕容廆於是投奔躲藏於遼東郡人徐郁家。至太康六年(285年),慕容删被其部下所殺,其部眾於是迎慕容廆繼位。

由于宇文部鲜卑和慕容涉歸有仇,慕容廆繼位後就请求晋朝政府允许其出兵討伐,但遭到拒绝。慕容廆于是反叛晋朝,出兵劫掠遼西郡,令當地傷亡和財物損失都十分嚴重。不久雖然為晋軍所敗,但仍常常侵掠昌黎郡,又進攻東邊的扶餘國,逼死其王依虑,更毀滅了扶餘國都。晋东夷校尉派部將贾沈援助扶餘,助扶餘王子依羅復國,當時慕容廆派兵截擊但被擊敗,扶餘亦成功復國。

慕容廆後自以先世世代臣服於中原王朝,而且力量懸殊,不能與當時是統一王朝的晉朝爭鋒,又稱不能因與晉朝不和而令當地百姓受戰禍之苦,遂於太康十年(289年)重新归顺晋朝。晉廷受降並封慕容廆為鮮卑都督。慕容廆當時就去東夷校尉府拜見東夷校尉何龕,以士大夫禮,穿著巾衣前去;但到後見何龕嚴兵以待,慕容廆於是改穿戎服,又稱主人不以禮待客,客亦不以禮相待。何龕聽聞慕容廆這樣說,慚愧之餘亦敬重他。

慕容廆又因當時勢弱和聲威日上而受宇文部鮮卑與段部鮮卑不斷侵擾,採取忍讓政策,以卑下的言辭和大量金錢去討好對方,段部鮮卑酋長段階於是將女兒下嫁慕容廆。慕容廆認為遼東郡過於僻遠,遂向西遷徙至徒河縣(今遼寧省錦州市)境的青山(今遼寧省義縣東)。元康四年(西元294年),慕容廆遷居大棘城(今遼寧省義縣西)。慕容廆又於勢力範圍內推廣農桑,並且施行與晉朝一樣的法制。至永寧二年(302年)兗、豫、徐、冀四州發生水災,鄰近冀州的幽州亦受影響,慕容廆則開倉賑災,助幽州人民渡過困境。

太安元年(西元302年),宇文部鮮卑酋長宇文莫圭命其弟宇文屈雲率軍進攻慕容廆,慕容廆避其主力反擊重創其別部將領宇文素怒延;宇文素怒延因羞憤而動員十萬人包圍慕容廆所在的大棘城,當時城內部眾都十分恐懼,沒有抵抗的意志,然而慕容廆稱這是在其計劃之中,勉勵部眾作戰,並親自領軍出擊,再度重創宇文素怒延兵團,追擊一百華里並俘虜及斬殺近萬人。原在宇文部下的遼東郡人孟暉率眾數千家歸降慕容廆,慕容廆任命孟暉當建威將軍。

永嘉元年(307年),慕容廆自稱鮮卑大單于。永嘉三年(309年)遼東郡太守龐本因私怨而殺害東夷校尉李臻。當地的附塞鮮卑素喜連和木丸津以為李臻報仇為名起兵,但卻沒有因新任東夷校尉封釋設計殺死龐本而罷兵,竟乘機攻略遼東郡中諸縣。當地晉兵更屢次兵敗。亂事持續了兩年,封釋已經無力再戰,請和但不果。而遼東百姓期間大多因戰火而投靠慕容廆。永嘉五年(311年),慕容廆面對這個情況,接納兒子慕容翰的建議,起兵討伐素喜連等人,將兩人殺害並吞併其部眾,將他們所掠的三千多家人及早前歸附自己的遼東郡人送還本郡,保全了遼東郡。

永嘉之亂後,大司馬王浚承制假立太子,並以慕容廆為散騎常侍、冠軍將軍、前鋒大都督、大單于,但慕容廆以不是王命所授而拒絕。及至建武元年(317年),時為晉王的晉元帝司馬睿承制拜慕容廆為假節、散騎常侍、都督遼左雜夷流人諸軍事、龍驤將軍、大單于、昌黎公。但慕容廆辭讓。當時魯昌勸說慕容廆支持司馬睿為帝,並以司馬睿晉朝正統之名討伐其他擁兵的鮮卑部落。慕容廆接納並命人循海路到建康勸進。至次年司馬睿即位為帝,再次要授予上一年慕容廆拒絕的官位,慕容廆這次就接受昌黎公以外的職位。

當時慕容廆政事修明,愛護人才,在北方紛亂的環境下,士大夫和民眾多歸附之,好像永嘉五年(311年)東夷校尉封釋病死前就托付孫兒封奕給慕容廆,其子封悛和封抽前來奔喪後因道路不通而不能返回,亦願留在當地,被慕容廆任命為長史和參軍。建興元年(313年),据有乐浪、带方二郡的张统因不堪长期孤军与高句丽作战而率千余家投靠慕容廆,慕容廆为其在侨置乐浪郡。為著管理大批的流人,他為冀州人設冀陽郡、豫州人設成周郡、青州人設營丘郡、并州人設唐國郡。同時又任用大批漢人賢才去處理政事和作自己的參謀。同時又推行儒學,除了讓世子慕容皝受學以外,自己在有餘暇時也會去聽講,故此令他統領的地方到處都有頌讚之聲,守禮謙讓之風亦流行。

但慕容廆如此受流徙當地的漢人支持,受到出身清河崔氏的平州刺史、东夷校尉崔毖的妒忌。崔毖曾數度遣使招請慕容廆前去但都不果,於是打算以武力拘禁他。崔毖於太興二年(319年)成功游说宇文部鮮卑、段部鲜卑和高句丽联合攻伐慕容廆,並許約戰後瓜分其領地。

當時面對三國聯軍來攻,慕容廆拒絕諸將出擊,認為他們新聚而銳不可擋,反而應該固守去令他們漸漸互相猜忌,待人心離異後才一舉擊破。及後三國聯軍包圍棘城,慕容廆閉門自守,卻特意送牛酒去宇文部那裏勞軍,以離間計挑起其餘兩國對宇文部的懷疑,最終令兩國各自率軍離去。但當時宇文部大人宇文悉獨官自以兵強,仍然留下進攻棘城。面對當時宇文部數十萬兵力,連營四十里的軍勢,慕容廆打算召留守徒河的兒子慕容翰入援,然而慕容翰卻認為棘城守軍足以守城,派使者向父親表示自己應該作為奇兵伺機突襲,配合城中守軍出擊就能夠大破對手;若自己也進去守城,那宇文部就能專心攻城,而且更示以部眾勢弱,將會削弱士氣。慕容廆在韓壽的進言下接受慕容翰的建言,不再召他回防。而此時宇文悉獨官亦聽聞慕容翰沒有入援棘城,擔憂不久成為後方大患,於是分兵先行消滅慕容翰。但慕容翰則設計打敗來攻的軍隊,更乘勝進攻宇文部大軍,慕容廆在接到慕容翰的消息後亦從城內出兵,成功大敗宇文部。

戰後,三國都遣使請和,崔毖亦因畏懼而派侄兒崔燾前來假意祝賀,以消對方對自己的怨恨。慕容廆卻借由崔燾傳話,要崔毖投降或出走,崔毖最終出奔高句麗,慕容廆就併吞其部眾。同時,主簿宋該亦勸慕容廆向東晉獻捷報,慕容廆於是命其作表,由長史裴嶷出使,同時將大敗宇文部時獲得的三顆印璽送呈建康。明年,裴嶷到建康時盛讚慕容廆,晉元帝於是拜慕容廆為安北將軍、平州刺史。太興四年(321年)再升慕容廆為都督幽、平二州及東夷諸軍事、車騎將軍、平州牧,封遼東郡公,賜丹書鐵券,允許他承制選置平州官員。

太寧元年(323年),後趙王石勒派使者來與慕容廆結好,但慕容廆卻收捕使者並押送到建康。石勒知道後大怒,於太寧三年(325年)命宇文乞得歸進攻慕容廆,卻被慕容廆所派去抵抗的軍隊擊敗,慕容仁等更乘勝攻破宇文部國都並掠奪其大量物資和人馬。

後來,慕容廆與太尉陶侃通信,稱讚王導和庾亮,並稱陶侃是「海內之望中唯足為楚漢輕重者」,表示願意為復興晉朝作出努力,只是礙於自己孤軍進攻難有成果,期待東晉大舉北伐時響應。同時還附著封抽、韓矯等建議封慕容廆為燕王、行大將軍事的上疏。陶侃將封抽的上疏報告朝廷,讓朝議定奪。咸和八年五月甲寅日(333年6月4日),慕容廆去世,享年六十五歲,當時朝議仍未有定論,知道慕容廆去世後就停止了。東晉遣使贈慕容廆大將軍、開府儀同三司,諡號為襄。咸康三年(337年)慕容皝自稱燕王時追諡為武宣王。至永和八年(352年)慕容廆孫慕容儁稱帝時,追諡為武宣皇帝。

\subsection{文明帝生平}

燕文明帝慕容\xpinyin*{皝}(297年-348年10月25日),字元真,小字万年,昌黎棘城(今遼寧義縣)鲜卑族人。中國五胡十六國時代前燕的開國君主,不過當時仍名義上臣屬於東晉,直至其子慕容儁正式稱帝後,才追尊廟號太祖,諡號為文明皇帝。其父為慕容部落的首領、遼東公慕容廆,其母段夫人。其庶長兄為建威将军慕容翰。

慕容皝勇武剛毅且多有謀略,崇尚經學,熟悉天文。建武初年拜冠軍將軍、左賢王、封望平侯。太兴四年(321年)十二月,慕容廆封遼東郡公,立身為嫡子的慕容皝为世子。其曾率眾出征,累有戰功,如於永昌元年(322年)率眾入侵段末柸的都城令支(今河北遷安縣西)。太寧末年,慕容皝拜平北將軍,封朝鮮公。咸和八年五月甲寅(333年6月4日),慕容廆去世。六月,慕容皝嗣辽东郡公,以平北将军行平州刺史,督摄部内,统治辽东。

同年,宇文乞得歸被宇文逸豆歸逼逐而在外去世,慕容皝出兵討伐,令宇文逸豆歸畏懼請和,慕容皝於是修築了榆陰和安晉二城後回軍。慕容皝弟征虜將軍慕容仁和廣武將軍慕容昭很得慕容廆寵愛,惹來慕容皝不滿,而二人在慕容皝登位後怕慕容皝不能接納自己,於是在慕容仁在平郭(今遼寧熊岳城)舉兵西行至棘城以攻慕容皝,並以慕容昭為內應。不過,慕容仁尚在途中,其計劃就被揭發,慕容昭被慕容皝賜死,慕容仁唯有回軍據守平郭。慕容皝於是派兵讨伐,却大败于汶城以北。及後孫機更以遼東郡向慕容仁投降,令其盡得遼東之地,而且獲得段部鮮卑首領段遼和鮮卑諸部的支持,遙遙相援。

咸和九年(334年),慕容皝接連派兵攻殺鮮卑木堤和烏丸悉羅侯。段遼亦攻徒何,不能攻破後更派段蘭和在上一年因懼慕容皝猜忌而出奔段部的慕容翰進攻柳城(今遼寧朝陽市),守將石琮死守,終於退軍。同年,派往東晉報喪的隊伍回遼東,慕容皝獲東晉授予鎮軍大將軍、平州刺史、大單于、遼東公,持節,並因以往慕容廆之事,都督幽、平二州及東夷諸軍事並承制置百官,但皆被慕容仁所留,慕容皝一直至次年隊伍被放回棘城才獲受命。慕容皝亦於同年率军讨辽东,成功奪取襄平(今遼寧遼陽市),居就、新昌兩縣亦歸降,慕容皝置和陽、武次和西樂三縣就撤軍,又將遼東大姓分徒於棘城。

咸康二年(336年)正月,慕容皝堅持趁海面結冰而從海路進攻慕容仁,於是在壬午日(2月17日)自昌黎東出發,經結冰海面走三百多里,至歷林口就放下輜重輕兵直取平郭。慕容皝軍行至平郭七里以外時,慕容仁斥候才向慕容仁報告,令慕容仁狼狽到城西北迎戰。當時慕容軍率部向慕容皝投降,震動慕容仁軍心,慕容皝於是趁機進攻,大敗對方,慕容仁亦被擒和被賜死。

平定慕容仁後,至六月,段遼又派兵進攻慕容皝,分別攻擊武興以及柳城,當時宇文逸豆歸亦攻進安晉以作聲援,慕容皝別將擊破攻武興之軍,而自己率兵增援柳城,逼走屯於城西的段蘭後轉攻安晉,並派封奕大敗逃走的宇文逸豆歸部眾。及後慕容皝預料二部會再來,於是命封奕在馬兜山設伏,成攻大敗下月來攻的段遼。其後又命世子慕容儁和封奕分別進攻段部和宇文部,皆大勝。慕容又下令在乙連東築好城並置戍,又建曲水城作好城之援,以威逼乙連。當時乙連大饑,段遼命人輸送糧食,但就被戍守好城的蘭勃所敗。後段遼部將段屈雲進攻興國,又被慕容皝將慕容遵擊敗並盡俘其部眾。

咸康三年十月丁卯(337年11月23日),慕容皝聽從封奕的勸告,自称燕王,建前燕,追慕容廆为武宣王,夫人段氏为武宣后,立世子慕容儁为王太子。當年又因段部鮮卑多番入侵,於是派宋回向後趙稱藩,以其弟慕容汗為人質,請求後趙與其聯兵進攻段部鮮卑。後趙天王石虎大悅,答允並送還慕容汗,約定明年進攻。随后在咸康四年(338年),石虎率眾進攻段部鮮卑,慕容皝則出兵進掠令支以北諸城,並大敗追擊的段蘭,大掠而還。而因石虎一直進攻,四十多座城被石虎所得,段遼於是棄守令支而逃至密雲山。石虎入令支後,不滿慕容皝自掠人民牲畜後回軍,不與其會師,於是下令進攻慕容皝。後趙軍一直進攻,至五月戊子日(6月12日)攻至棘城時,慕容皝打算逃亡,但被慕輿根勸阻,當時玄菟太守劉佩更率敢死隊數百騎出城衝擊後趙軍,所向披靡,令城中士氣大增;封奕亦勸慕容皝堅守,終令慕容皝安心不降。兩軍相持十多日後,後趙軍引兵退還,慕容皝派慕容恪率二千騎進攻後趙軍,驚擾敵軍而令其棄甲潰散,殺三萬餘人。及後慕容皝分兵收復原本叛歸後趙的各個郡縣,並擴境至凡城,置戍而還。十二月,段遼降後趙,不久又悔而轉投慕容皝,而後趙已派麻秋支援段遼,慕容皝於是命慕容恪設伏於密雲山,大敗麻秋,並帶著段遼和其部眾撤還。

咸康五年(339年),慕容皝守將擊退來攻凡城的後趙軍隊,又因自稱燕王未受東晉朝命,於是命長史劉翔向建康獻捷,兼求假燕王璽綬,又請大舉出兵平定中原。不過當時朝廷議論未肯容讓慕容皝稱王。此時慕容皝得知庾亮去世,其弟庾冰及庾翼分掌朝廷中樞及荊州要地,於是上書要晉成帝以史為鑑,親近賢達,不要親信外戚。又寫信給庾冰,指責他掌握朝權,卻未能為國雪恥,只「安枕逍遙,雅談卒歲」。庾冰知道慕容皝的上表和書信後十分恐懼,自以道遠而不能控制他,於是奏請順應慕容皝的請求。咸康七年(341年),慕容皝獲東晉任命為使持節、大將軍、都督河北諸軍事、幽州牧、大單于,封燕王。

在受封燕王的同一年,慕容皝下令在柳城以北,龍山以西修建龍城,並改柳城為龍城。至次年(342年)正式遷入龍城。遷都後,慕容皝聽從早前歸國的慕容翰建議,先襲破高句麗,後才再攻取宇文鮮卑,以解後顧之憂,專心圖取中原土地。慕容皝並自率精兵四萬從險狹的南道進攻高句麗,以慕容翰及慕容垂為前鋒,以王寓領偏師五千走平廣開闊的北道引誘敵軍,終出其不意,成功攻陷高句麗都城丸都(今吉林集安),高句麗王高釗出逃,慕容皝招引不出,且因王寓敗沒而沒有追擊,於是挖出高釗父高乙弗利的屍體,連同丸都城中府庫收藏的珍寶、高釗母親和妻子及擄掠的五萬多人一同西還,更毀丸都。高句麗因而於翌年(343年)向慕容皝稱臣,慕容皝於是送還其父親屍體,留其母為人質。

高句麗稱臣於慕容皝後,慕容皝又擊敗了宇文逸豆歸派來進攻的國相莫淺渾。建元二年(344年),慕容皝親自率二萬騎兵討伐宇文鮮卑,又派慕容翰為前鋒,慕容軍、慕容恪、慕容垂及慕輿根兵分三路一同進攻。慕容翰與宇文逸豆歸大將涉奕于大戰,涉奕于戰死,宇文部軍心瓦解,被慕容皝所敗,都城紫蒙川陷落,宇文逸豆歸敗死漠北,宇文鮮卑至此被慕容皝所併。

此战后慕容皝终究不能对慕容翰放心,将其赐死。

永和元年(345年),慕容皝又派慕容恪攻高句麗,攻克南蘇並置戍而還。永和二年(346年)又命慕容儁與慕容軍、慕容恪及慕輿根率一萬七千兵東襲夫餘,成功俘虜夫餘王餘玄等五萬多人回國。

而早在咸康六年(340年),後趙已大舉徵兵,大行屯田,並收集戰馬,準備進攻慕容皝。當時慕容皝認為薊城因樂安得重兵駐守而防禦空虛,突襲薊城,守城的石光驚懼而不出擊,慕容皝攻陷高陽並焚毀積聚的軍糧,更掠奪了三萬餘戶。此舉打亂了後趙進攻計劃,而慕容皝平定高句麗和宇文部等主要對手後,前燕就能更集中對抗後趙,終令前燕得以專心在永和六年(350年)乘後趙內亂出兵中原。

永和四年九月丙申日(348年10月25日),慕容皝去世,时年五十二,諡為文明王。

永和元年(345年),慕容皝自以古時諸侯即位皆稱元年,故此不再用晉朝年號,追咸和八年(333年)登位起計,改稱十二年。

慕容皝汉化较深,崇尚儒学,设东庠(学校),以大臣子弟为官学生,号高门生。亲临讲授,每月考试优劣。

慕容皝鼓勵農耕,例如就曾在朝陽門東設籍田,置官主理。後又親自巡行各郡縣,鼓勵和督察農業活動。更加罷園林供沒有土地的農民耕種,更贈送一頭牧牛給沒有牛的農民。

慕容皝曾樹立納諫之木,以示他願意接受正直諫言。

史載慕容皝身長七尺八寸(約191厘米)。

慕容皝好文學典籍,故他勸於到東庠講授,學生多達千餘人。慕容皝更親作《太上章》以取代《急就篇》作學生識字的書籍,又寫了《典誡》共十五篇,皆用來教授學生。

《晉書》載慕容皝一次在國境西邊畋獵,將渡河時見一個騎白馬,穿紅衣的老人,舉手指揮著慕容皝,說該處不是狩獵場,要慕容皝離開。不過慕容皝沒有將事件說出來,更渡河狩獵,接連幾日大有收獲。慕容皝及後見到一隻白兔,於是策馬追射,但馬匹卻跌倒,慕容皝亦墮馬受傷,這時才說出他看見老人一事。慕容皝回龍城後將後事託付給世子慕容儁,後就死去了。王隱《晉書》亦有相近記載,不過是老人說話後就不見了,而後追獵白兔時墮馬撼石,當場死亡。


%% -*- coding: utf-8 -*-
%% Time-stamp: <Chen Wang: 2021-11-01 11:56:21>

\subsection{景昭帝慕容儁\tiny(348-359)}

\subsubsection{生平}

燕景昭帝慕容\xpinyin*{儁}(319年-360年2月23日),字宣英,鮮卑名賀賴跋,昌黎棘城(今遼寧義縣)鲜卑人,五胡十六國時代前燕的君主。前燕文明帝慕容皝次子。慕容儁即位時仍名義上為東晉的燕王,然而於永和八年(352年)正式稱帝獨立。慕容儁在位期間消滅了冉魏,入據原本由後趙所佔領的中原地區,勢力大增,並移都鄴城,終與南方的東晉和關中的前秦政權三足鼎立。

慕容儁博覽群書,有文武才幹,曾領兵攻略段部鮮卑並大勝而還。咸康七年(341年),東晉封慕容皝為燕王,亦以慕容儁為燕王世子,假節、安北將軍、東夷校尉、左賢王。

永和四年九月丙申日(348年10月25日),慕容皝去世。十一月甲辰日(349年1月1日),太子慕容儁繼襲燕王爵位。派使臣到建康向東晉報告了喪事。他還任命弟弟慕容友為左賢王,任命左長史陽鶩為郎中令。次年(349年)稱元年,仍不用東晉年號。同年後趙皇帝石虎去世,諸子爭位令國內大亂,慕容儁圖謀奪取中原土地,於是以慕容垂為前鋒都督、建鋒將軍,另外任命慕容恪為輔國將軍、慕容評為輔弼將軍和陽騖為輔義將軍,人稱三輔。挑選了二十多萬精兵等待時機。而同年東晉朝廷亦任命慕容儁為使持節、侍中、大都督、都督河北諸軍事、幽冀并平四州牧、大將軍、大單于、燕王,並依慕容廆和慕容皝的先例能承制封拜官員,在東晉授命下正式繼承了對遼東的管治。

永和六年(350年),後趙大將軍冉閔在鄴城稱帝,慕容儁亦乘機兵分三路南攻,自己親自率中軍出兵盧龍,攻下了薊城,並遷都至薊。因慕容儁聽從慕容垂不要坑殺薊城士卒的勸言,故得中原士民歸附。其他幽州郡縣多亦奪取,慕容儁於是設置幽州諸郡縣的官員。後慕容儁意圖進攻後趙幽州刺史王午和征東將軍鄧恆所守的魯口,不過被其將鹿勃早夜襲,雖然最終成功擊退對方,不過軍隊鋒銳已因這次突襲而受挫,只得暫緩戰事,返回薊城。不久代郡人趙榼率三百餘家叛歸後趙,慕容儁於是遷廣寧、上谷二郡人到徐無,代郡人到凡城,以防其再次叛歸後趙。不過,慕容儁亦南攻冀州,攻下了章武、河間二郡。

另一方面,守襄國的後趙皇帝石祗自永和六年起就被冉閔所圍攻。圍困百多日後,石祗被逼於永和七年(351年)向前燕求援,並許以傳國璽作交換。慕容儁欲得傳國璽,於是相信了後趙並派了悅綰救援襄國。同年冉閔被擊敗,襄國之圍解除,但悅綰沒有獲得傳國璽,慕容儁於是殺掉當日前來求援的後趙太尉張舉。慕容儁又派兵奪取中山和趙郡,又進攻魯口,擊敗王午派來迎擊的軍隊。

永和八年(352年),前燕王慕容儁派廣威將軍慕容軍、殿中將軍慕輿根、右司馬皇甫真等人率二萬人步、騎兵協助慕容評攻打冉魏鄴城。 永和八年(352年),冉閔攻陷襄國,將殺後趙皇帝石祗的將領劉顯勢力消滅。同年四月甲子日(5月5日),慕容儁命慕容恪等攻伐冉魏,最終擊敗冉閔並將其俘虜。己卯日(5月20日),冉閔被押送到薊城,慕容儁指責冉閔:「你只是配當奴僕的低下才幹,憑甚麼去稱帝?」冉閔卻說:「天下大亂,你這些夷狄禽獸都能稱帝,那我這種中土英雄,怎能不稱帝呀!」慕容儁聽後大怒,鞭打他三百下並送到龍城處死。同時,先前叛燕的段勤既受慕容垂進攻據地繹幕,看見慕容恪進據常山後就因畏懼而請降。

甲申日(5月25日),慕容儁命慕容評等進攻鄴城,冉魏太子冉智與將領蔣幹閉城門自守,得晉將戴施率百餘人入鄴助守,並以傳國璽向東晉請糧。不過,慕容評終於八月庚午日(9月8日)攻下鄴城,俘冉智等人至中山。冉魏亡後,當時擁兵據守州郡的後趙官員都派使者向前燕請降。

攻下鄴城後,慕容儁假稱冉閔皇后董氏獻傳國璽予他,賜董氏號「奉璽君」。十一月丁卯日(353年1月3日),慕容儁置百官,次日即位為皇帝,改年號為「元璽」,追尊慕容廆和慕容皝為皇帝並上廟號。當時東晉使者到了前燕,慕容儁就對他說:「你回去告訴你的天子,中原無主,我被士民推舉為主,已經做了皇帝了!」

前燕南侵幽州時據守魯口的王午在永和八年(352年)自稱安國王,同年被殺,由呂護承襲稱號並繼續據守魯口。永和九年(353年),衛將軍慕容恪、撫軍將軍慕容軍、左將軍慕容彪等人屢次薦舉給事黃門侍郎慕容霸,說他有顯赫於世之才,應總攬重任。前燕皇帝慕容儁任命慕容霸為使持節、安東將軍、北冀州刺史、鎮守常山。永和九年(353年),慕容儁派慕容恪進兵討伐,終令呂護於永和十年(354年)歸降。後慕容儁又命慕容恪鎮守洛水,以慕容強為前鋒都督,進據黃河以南地方。永和十年(354年),慕容儁封弟弟慕容恪為太原王,慕容評為上庸王,封左將軍慕容彭為武昌王,封撫軍將軍慕容軍為襄陽王,封安東將軍慕容霸為吳王,左賢王慕容友為范陽王,散騎常侍慕容厲為下邳王,散騎常侍慕容宜為廬江王,寧北將軍慕容度為樂浪王。慕容桓為宜都王,慕容遵為臨賀王,慕容徽為河間王,慕容龍為歷陽王,慕容納為北海王,慕容秀為蘭陵王,慕容岳為安豐王,慕容德為梁公,慕容默為始安公,慕容僂為南康公。兒子慕容臧為樂安王,慕容亮為勃海王,慕容溫為帶方王,慕容涉為漁陽王,慕容暐為中山王。

永和十一年(355年),東晉蘭陵太守孫黑、濟北太守高柱、建興太守高甕及前秦河內太守王會、黎陽太守韓高都以所在郡投降前燕。而先前屯據蕕城,歸降前秦的前車騎將軍劉寧亦率二千戶人到薊城歸降請罪,慕容儁亦任命劉寧為後將軍。高句麗王高釗亦向前燕進貢。同年,據守廣固並向東晉稱藩的段龕寫信非議慕容儁稱帝之事,觸怒了慕容儁並令他派了慕容恪進討段龕,終於在次年攻陷廣固,俘虜了段龕。升平元年(357年),慕容儁又命慕容垂等率八萬兵到塞北進攻丁零敕勒,大敗對方並俘殺十多萬人,奪去十三萬匹馬和億萬頭牛羊。及後匈奴單于賀賴頭率部歸降前燕。

升平元年十一月癸酉日(357年12月14日),慕容儁遷都鄴城。升平二年(358年),東晉泰山太守諸葛攸進攻東郡,被慕容恪等擊敗,慕容恪更乘機掠奪河南土地。不久東晉北中郎將荀羨攻陷山茌,處死太守賈堅,亦被前燕軍隊擊敗並收復失地。升平三年(359年)諸葛攸再攻前燕,在東阿被慕容評等人擊敗。同年十月,東晉西中郎將謝萬與北中郎將郗曇北伐,但因郗曇因病退兵以及謝萬統率失誤而令軍隊驚潰敗退,前燕得以乘機奪取許昌、穎川、譙及沛諸郡各城。

另一方面,前秦平州刺史劉特率眾向前燕投降。慕容儁又於升平二年(358年)派了司徒慕容評等人進攻盤據并州自立的將領張平、李歷等,令張平的部下諸葛驤、蘇象等率當地一百三十八個壁壘歸降前燕。及後張平等先後出奔,前燕於是收降了其部眾。

此時前燕正与东晋、前秦形成三足鼎立之势,并且在当时是国力最强的。

升平二年(358年),慕容儁因於擴張領土的戰爭中屢次獲勝,於是更圖謀消滅東晉以及前秦。為此下令州郡核實男丁數目,每戶只留下一個男丁,其餘都被徴為士兵,務求令全國步兵達至一百五十萬人。慕容儁更命士兵於明年就要集合,並攻取洛陽。在劉貴的諫止下,慕容儁才與官員議論,最終改為「三五占兵」,並將集合期限寬貸至一年後,定於下一年冬季於鄴城集合。

不過慕容儁於升平三年(359年)就患病,他向弟弟慕容恪表示他擔心自己一病不起,而前秦和東晉尚未滅亡,憂心皇太子慕容暐未有足夠能力治理國家,於是打算仿效宋宣公,以慕容恪繼位。不過慕容恪堅決拒絕,更矢言會輔助慕容暐。升平四年(360年)正月,慕容儁於鄴城閱兵後不久就於當月甲午日(2月23日)病死,臨終遺命大司馬太原王慕容恪、司徒上庸王慕容評、司空陽騖、領軍將軍慕輿根為輔政大臣,虚龄四十二,諡為景昭皇帝,廟號烈祖。

慕容儁于建熙元年三月葬于龙城(今辽宁省朝阳市)的龙陵(具体方位不详)。

慕容儁長子,獻懷太子慕容曄於永和十二年(356年)去世,慕容儁對此十分傷心。一次慕容儁在蒲池與群臣飲宴,因為談及東周時周靈王的太子晉,竟流下淚來,更表示自己在慕容曄死後「鬚髮中白」,更明白為何曹操和孫權昔日要分別為兒子曹沖和孫登早逝而痛惜不已。

慕容儁喜好文學典籍,即位以來都講論不斷,處理政務以外都是和侍臣交流典籍的義理,更有四十多篇著述。慕容儁亦於顯賢里設小學教育冑子。

慕容儁曾夢見石虎咬他的手臂,令慕容儁十分厭惡,於是下令挖開石虎的墓穴,罵道:「死胡竟然敢夢中嚇天子!」於是命御史中尉陽約數其殘酷之罪,鞭屍後丟到漳水去。《資治通鑑》更謂慕容儁在石虎墓找不到石虎屍首,於是懸賞百金求屍;後因鄴城女子李菟報告,在東明觀找到石虎屍首,發現他竟僵硬不腐;石虎屍首被投進漳水後,更靠在柱邊不流走。

\subsubsection{元玺}

\begin{longtable}{|>{\centering\scriptsize}m{2em}|>{\centering\scriptsize}m{1.3em}|>{\centering}m{8.8em}|}
  % \caption{秦王政}\
  \toprule
  \SimHei \normalsize 年数 & \SimHei \scriptsize 公元 & \SimHei 大事件 \tabularnewline
  % \midrule
  \endfirsthead
  \toprule
  \SimHei \normalsize 年数 & \SimHei \scriptsize 公元 & \SimHei 大事件 \tabularnewline
  \midrule
  \endhead
  \midrule
  元年 & 352 & \tabularnewline\hline
  二年 & 353 & \tabularnewline\hline
  三年 & 354 & \tabularnewline\hline
  四年 & 355 & \tabularnewline\hline
  五年 & 356 & \tabularnewline\hline
  六年 & 357 & \tabularnewline
  \bottomrule
\end{longtable}

\subsubsection{光寿}

\begin{longtable}{|>{\centering\scriptsize}m{2em}|>{\centering\scriptsize}m{1.3em}|>{\centering}m{8.8em}|}
  % \caption{秦王政}\
  \toprule
  \SimHei \normalsize 年数 & \SimHei \scriptsize 公元 & \SimHei 大事件 \tabularnewline
  % \midrule
  \endfirsthead
  \toprule
  \SimHei \normalsize 年数 & \SimHei \scriptsize 公元 & \SimHei 大事件 \tabularnewline
  \midrule
  \endhead
  \midrule
  元年 & 357 & \tabularnewline\hline
  二年 & 358 & \tabularnewline\hline
  三年 & 359 & \tabularnewline
  \bottomrule
\end{longtable}


%%% Local Variables:
%%% mode: latex
%%% TeX-engine: xetex
%%% TeX-master: "../../Main"
%%% End:

%% -*- coding: utf-8 -*-
%% Time-stamp: <Chen Wang: 2019-12-19 09:51:40>

\subsection{幽帝\tiny(360-370)}

\subsubsection{生平}

燕幽帝慕容\xpinyin*{暐}(350年-384年),字景茂,昌黎棘城(今遼寧義縣)鮮卑人。五胡十六國時代前燕的最後一位君主,前燕景昭帝慕容儁第三子。前期在慕容恪攝政之下仍能保持國家穩定,但後期在慕容評主政之下就漸漸衰落,最終被前秦所滅。慕容暐在前燕亡後成為前秦的臣下,獲封為新興侯。前秦於淝水之戰後崩潰,慕容垂、慕容泓先後舉兵建立「後燕」和「西燕」,慕容暐亦在西燕進攻前秦都城長安(今陝西西安)時意圖殺死苻堅並令城內混亂,響應外軍,但失敗被殺。

慕容暐最初獲封中山王。永和十二年(356年),皇太子慕容曄去世,慕容儁於是在次年立八歲的慕容暐為皇太子。升平四年(360年),慕容儁去世,臨終時遺命大司馬慕容恪、司空陽騖、司徒慕容評及領軍將軍慕輿根輔政。當時群臣打算立作為慕容儁弟弟的慕容恪繼位,但被慕容恪拒絕,而支持作為儲君的慕容暐即位。

慕容暐即位後便以慕容恪為太宰,讓他專攝朝政,而慕容評、陽騖和慕輿根則分別獲授太傅、太保及太師,參輔朝政。不過,當時慕輿根就自恃自己屢有戰功,顯得高傲自大,心中不服慕容恪。當時慕輿根打算作亂,初以可足渾太后干政煽動慕容恪謀反失敗,於是改向可足渾太后及慕容暐中傷慕容恪,想要他們誅殺慕容恪及慕容評。不過此時慕容暐卻信任慕容恪,勸止打算聽從的可足渾太后。及後慕容恪及慕容評密奏慕輿根罪狀,慕容暐於是命侍中皇甫真、右衞將軍傅顏等收捕慕輿根,並將其家人黨羽一併誅殺。

此時前燕國內正因慕容儁之死而混亂,原本徵集在鄴城的大軍都常常私下逃散,但在慕容恪的輔助下,最終都成功穩定了國家。在慕容恪攝政之下,先擊敗據守野王叛變的寧南將軍呂護,後更進侵當時為東晉所控的洛陽,終於興寧三年(365年)攻下洛陽。後又攻取了東晉的兗州諸郡。

不過,慕容恪於太和二年(367年)去世,死前想以吳王慕容垂代替自己為大司馬,但最終慕容評改以慕容暐弟慕容沖接替慕容恪。慕容恪死後,陽騖在同年亦死,唯一仍在世的輔政大臣慕容評就以太傅主政。當時僕射悅綰上奏盡罷軍封蔭戶,以釋放人口以充實國家地方,防止人口隱匿。慕容暐同意之下,最終在悅綰的規劃下釋放了二十多萬戶人,政令亦令朝野震驚,慕容評更是十分不滿,派人暗殺了悅綰。

太和四年(369年),東晉桓溫發動北伐戰爭,主動進攻前燕,慕容暐所派的慕容厲、傅顏及慕容臧皆不能抵抗桓溫進攻,於是令慕容暐及慕容評十分恐懼,向前秦求援以外還打算逃回和龍(今遼寧錦州)。這時慕容垂自請進攻,最終成功扭轉局勢,更在逼桓溫撤軍時大敗晉兵。然而慕容評在後十分忌憚剛取得大功的慕容垂,二人更因將領孫蓋軍功問題發生爭論。因可足渾太后亦討厭慕容垂,於是就與慕容評謀殺慕容垂,慕容垂只得與家人逃奔前秦。

不久,出使前秦的黃門侍郎梁琛歸國,報告前秦國內揚兵講武,而且運糧至陝東,更逢慕容垂出奔前秦,表示擔憂前秦和和前燕開戰,建議朝廷早作防備。然而慕容評不認為前秦會破壞和前燕的和平,慕容暐於是和慕容評都沒重視梁琛的話。及後皇甫真又上言表示擔憂前秦對前燕有所圖謀,建議增強洛陽、并州和壺關(今山西長治東南)各城的軍力。慕容暐於是召慕容評討論,但因慕容評認為前秦「國小力弱」,要倚靠前燕為援,前秦天王苻堅也不會因慕容垂而攻燕,勸慕容暐不要自亂陣腳。慕容暐於是亦沒有聽從皇甫真的話。

當日前燕向前秦求援時,允諾割讓虎牢(今河南滎陽西北汜水鎮)以西的土地給前秦,但戰後反悔。苻堅於是以此派王猛等進攻前燕,進攻洛陽。慕容暐於是派了慕容臧救援洛陽,然而卻在滎陽大敗給前秦軍,無法有效營救洛陽,洛陽守將慕容筑唯有向前秦投降,洛陽陷落。慕容臧只得築新樂城而退。面對當時的軍事形勢,而且太后干政、慕容評貪污,尚書左丞申紹上疏要改革,提出令將士用命,對士兵「習兵教戰」、「從戎之外,足營私業」等。又要君臣「罷浮華,禁絕奢,峻明婚姻喪葬之條」以及增加重地守備軍隊等措施,但慕容暐都沒聽從。

洛陽陷落的同年(370年),前秦再攻前燕,王猛攻壺關而楊安攻晉陽(今山西太原)。慕容暐命慕容評等率中外精兵三十多萬抵禦。不過,慕容評竟禁止士兵取水和柴,而自據水源和山,向士兵販賣柴水以斂財,導致軍心全無。最終被前秦軍夜燒輜重,火光連鄴城都看得見。慕容暐見狀十分恐懼,下令慕容評將金錢財帛都分給士兵,命他們作戰,慕容評因恐懼而向前秦請戰。最終前秦軍大敗前燕軍,俘殺超過十五萬人,慕容評單騎奔鄴城。

王猛在戰後追擊至鄴,苻堅亦派大軍後繼。面對前秦軍兵臨城下,慕容暐只得與慕容評等人逃奔龍城(今遼寧朝陽),但隨行衞士一出城就散走,只餘十多名仍然隨行。當時前奏將領郭慶亦在後追擊慕容暐,途中道路艱險難行而且時有盜賊,保衞慕容暐的左衞將軍孟高、殿中將軍艾朗皆戰死,慕容暐更因失去馬匹而只得徒步逃亡,最終在高陽被郭慶所俘。慕容暐隨後被押見苻堅,苻堅質問慕容暐為何不降而逃走,慕容暐答:「狐狸快要死時,也會將頭朝向自己出生的山丘,我都是想死在先人墳墓那裏而已。」苻堅憐憫慕容暐而將他釋放,命他回去率文武百官出降。另外逃奔遼東的前燕殘餘勢力不久亦被消滅,前燕正式滅亡。

同年十二月,慕容暐與慕容皇族及鮮卑族四萬戶一同被苻堅遷往長安安置,並受封為新興侯,署為尚書。

太元八年(383年),前秦大舉南侵東晉,即淝水之戰,慕容暐亦以平南將軍、別部都督隨軍。前秦於淝水之戰大敗後,時駐鄖城的慕容暐隨前秦軍北撤,並護送苻堅的張夫人。至滎陽時,叔父慕容德勸慕容暐乘前秦軍力大損而復國,但慕容暐又不聽從。慕容暐終與苻堅回到長安,但當時前秦對全國的控制已不如前。太元九年(384年),慕容垂在河北叛變建立後燕,不久,慕容暐之弟慕容泓也在關中叛變建立西燕。當時慕容泓向苻堅要求送還慕容暐以換取燕秦兩國和平,但為苻堅所拒絕。苻堅亦因此召慕容暐來斥責,終在慕容暐叩頭陳謝之下原諒他,並命他寫信招撫慕容垂、慕容泓和慕容沖。不過慕容暐就暗中派密使向慕容泓說:「我是鐵籠裡的人,肯定無法回去了;而且,我也是帝國的罪人,無必要顧慮我了。你就建立大業,以吳王慕容垂為相國,中山王慕容沖為太宰、領大司馬,你可以做大將軍、領司徒,承制封拜,收到我去世的消息後,你就自己稱帝吧。」及後西燕與前秦在長安多有戰事,慕容暐與慕容肅共謀聯同長安城中數千鮮卑人作亂,以應進攻長安的西燕軍,於是借兒子新婚為由設計在其家殺害苻堅。然而苻堅因大雨而沒有去,事情洩露,苻堅召慕容暐和慕容肅並殺害二人,更誅連城中的鮮卑人。慕容暐死時三十五歲。

西燕、后燕均没有追谥慕容暐。慕容德建立南燕時,諡慕容暐為幽皇帝。

慕容暐很在意他人對其的批評,如李績曾經向慕容儁表示慕容暐的缺點是「雅好遊田,娛心絲竹」,慕容儁亦因而要慕容暐好好記著李績的話,好作改善。但慕容暐登位後,慕容恪雖然屢請以李績為尚書右僕射,但慕容暐都不同意,更說:「萬機之事都交由叔父處理,但李績一人,我想自己決定。」最終李績憂死。

\subsubsection{建熙}

\begin{longtable}{|>{\centering\scriptsize}m{2em}|>{\centering\scriptsize}m{1.3em}|>{\centering}m{8.8em}|}
  % \caption{秦王政}\
  \toprule
  \SimHei \normalsize 年数 & \SimHei \scriptsize 公元 & \SimHei 大事件 \tabularnewline
  % \midrule
  \endfirsthead
  \toprule
  \SimHei \normalsize 年数 & \SimHei \scriptsize 公元 & \SimHei 大事件 \tabularnewline
  \midrule
  \endhead
  \midrule
  元年 & 360 & \tabularnewline\hline
  二年 & 361 & \tabularnewline\hline
  三年 & 362 & \tabularnewline\hline
  四年 & 363 & \tabularnewline\hline
  五年 & 364 & \tabularnewline\hline
  六年 & 365 & \tabularnewline\hline
  七年 & 366 & \tabularnewline\hline
  八年 & 367 & \tabularnewline\hline
  九年 & 368 & \tabularnewline\hline
  十年 & 369 & \tabularnewline\hline
  十一年 & 370 & \tabularnewline
  \bottomrule
\end{longtable}


%%% Local Variables:
%%% mode: latex
%%% TeX-engine: xetex
%%% TeX-master: "../../Main"
%%% End:


%%% Local Variables:
%%% mode: latex
%%% TeX-engine: xetex
%%% TeX-master: "../../Main"
%%% End:

%% -*- coding: utf-8 -*-
%% Time-stamp: <Chen Wang: 2019-12-19 10:03:55>


\section{前秦\tiny(351-394)}

\subsection{简介}

前秦(350年—394年)是十六国之一。350年氐族人苻洪占据关中,称三秦王。352年苻健称帝,定都长安,国号“秦”。370年起,先後灭前燕、前凉及代国,统一北方。394年被西秦和後秦所灭。當時朝鮮半島由高句麗、百濟、新羅割據,接受前秦册封。北方外族有柔然、庫莫奚、契丹及高車。西有吐谷渾及白蘭。

因其所据为战国时秦国故地,故以此立国号。前秦之称最早见于《十六国春秋》,后为别于其他以“秦”为国号政权,而袭用之。又以其王室姓苻,故又称为苻秦。

西晉末年,西晉政權顛覆之際,略陽氐族推出貴族苻洪為首領。前趙劉曜在長安稱帝,以苻洪為氐王。後石勒滅前趙,苻洪降於石勒。333年,石虎徙關中豪傑及羌戎至關東,以苻洪為流民都督,居於枋頭。苻洪自稱大都督、大將軍、大單于、三秦王,不久為石虎舊將麻秋所毒死,其第三子苻健代統其眾。

苻健自枋頭而西,關中氐人紛起響應,苻健乘机進占关中,據有关陇。351年建都長安,苻健自稱大秦天王、大單于。352年,改稱皇帝,国号秦,史稱前秦。

起初苻健知道中原「民心思晉」,在枋頭時,打著晉征西大將軍、都督關中諸軍事、雍州刺史來作號召;抵達關中之後,遣使向東晉稱臣,以緩和關中地區的矛盾,直到他稱帝後,才和東晉斷絕關係。354年,東晉大將桓溫親率大軍四萬攻秦,因苻健採清野政策,晉軍在給養問題遇到困難,只好撤退。355年苻健死,子苻生繼位,因淫殺過度,357年,苻健弟苻雄之子苻堅殺死苻生自立。

苻堅在登位以前,就聽見王猛的名聲,並約見王猛,談得十分投契。即帝位,任王猛以政。王猛採取政治改革,加強中央集權,抑制貴族勢力發展來強化中央力量,並興修關中水利,前秦國力逐漸增強。370年,前秦滅前燕,擒慕容暐;371年,滅仇池氐楊氏;373年,攻取東晉梁、益二州,西南夷邛、筰、夜郎皆歸附於秦;376年,滅前涼張氏;同年,乘鮮卑拓跋氏衰亂之際,進兵滅代;382年,命呂光駐西域。中原地區盡為前秦版土之下,史稱「東極滄海,西併龜茲,南包襄陽,北盡沙漠」。東北、西域各國都遣使和前秦建立關係,只有東南一隅的東晉與他對峙。

378年,前秦征南大將軍苻丕等率領步騎兵七萬人,攻擊東晉所屬的襄陽,東晉梁州刺史朱序死守近一年,城池陷落被俘。

379年,前秦右將軍毛當、強弩將軍王顯,率二萬人自襄陽出發,跟後將軍俱難、兗州刺史彭超會師,攻擊東晉淮河以南各城池,攻陷盱眙,包圍三阿。東晉兗州刺史謝玄出兵救援,四次擊敗秦軍,俱難、彭超向北逃走,僅保住一命。

建元十八年(382年)苻堅之大将吕光率兵七万伐龟兹,龟兹王白纯不降,吕光进军讨平龟兹。

383年,前秦天王苻堅親率騎兵二十七萬、步兵六十萬南下,對東晉發動總攻擊。弟弟陽平公苻融擔任前鋒,十月,攻陷壽陽。苻堅派東晉降將朱序向東晉征討大都督謝石勸降,但朱序反而將秦軍狀況密告謝石,建議晉軍乘秦軍未全部集結時發動攻擊。十一月,東晉前鋒都督謝玄的部將劉牢之率兵五千突襲洛澗,秦軍大敗,死一萬五千人。

晉軍乘勝西進,秦軍在淝水西岸佈陣對峙。謝玄派人要求秦軍略向後撤,讓晉軍渡水決戰。苻堅企圖乘晉軍半渡淝水時予以截擊,同意後退,但秦軍軍心不穩,一退陣腳大亂,不能停止。晉軍乘勢渡水猛烈攻擊,混亂中苻融墮馬被晉軍所殺,朱序又在陣後大呼「秦兵敗矣!」於是秦軍崩潰,四散逃亡,前鋒三十萬人中,死亡的佔十分之七八。

淝水之戰後,原先歸附前秦的其他民族,紛紛乘機獨立,黃河以北又再陷入分裂的狀態。

383年,前燕降將、鮮卑族的冠軍將軍慕容垂,奉命攻擊在新安起兵的丁零部落首領翟斌,途中屠殺副將苻飛龍及一千人的氐人部隊。384年,慕容垂自稱「燕王」,廢除前秦年號,建立後燕,並進攻駐守鄴城的前秦長樂公苻丕。

前秦北地長史慕容泓(前燕帝慕容暐的弟弟),聽到叔父慕容垂攻鄴的消息,投奔關東集結數千鮮卑人,自稱大將軍、濟北王,建立西燕。苻堅派兒子鉅鹿公苻叡當統帥,羌人將領姚萇任參謀,出兵討伐,在華澤大敗,苻叡被斬殺。苻堅大怒,姚萇畏罪逃到渭北,被族人推為盟主。姚萇遂自稱大將軍、大單于、萬年秦王,建立後秦。

385年,後秦包圍新平郡,苻堅投奔五將山,被後秦將領吳忠俘擄,送回新平郡單獨囚禁。八月,姚萇派人向苻堅索取傳國玉璽,又遊說苻堅禪讓帝位,苻堅大怒拒絕,痛罵姚萇,只求一死,又先殺女兒苻寶、苻錦。八月二十六日,姚萇派人闖入囚禁苻堅的佛寺,縊殺苻堅,時年四十八歲。姚萇為掩飾自己的弒逆惡名,追尊苻堅為「壯烈天王」。

394年,七月,前秦帝苻登在馬毛山以南跟後秦帝姚興交戰,被生擒後斬首,太子苻崇投奔湟中繼承帝位。十月,苻崇被西秦首領乞伏乾歸驅逐,投奔隴西王楊定,兩人於攻擊西秦時被西秦涼州刺史乞伏軻彈斬殺,前秦到此滅亡。

\subsection{惠武帝生平}

苻洪(285年-350年),字廣世,略陽臨渭氐人,是前秦政權奠基者。苻洪原名蒲洪,後以讖文有「艸付應王」,遂改苻姓。氐族部落小帥蒲怀归之子,亦是前秦開國君主苻健之父。苻洪先後歸附前趙和後趙兩個政權,後在後趙內大亂時試圖謀取中原。最終雖然遭毒殺,但他所累積的力量令其子苻健在關中成功建立前秦。

蒲洪原是氐族酋長,因為驍勇而多謀略而得氐人畏服,永嘉四年(310年)時曾獲前趙(當時國號為「漢」)皇帝劉聰任命為平遠將軍,但蒲洪不受,反自稱護氐權尉、秦州刺史、「略陽公」。後因在永嘉之亂時大散錢財以向英傑之士訪尋轉危為安的方法,於是被宗人蒲光及蒲突推舉為盟主。太興二年(319年),前趙皇帝劉曜遷都長安,蒲光等逼蒲洪附前趙,於是獲授為率義侯。咸和三年(328年),劉曜在與石勒的決戰中被俘,蒲洪於是西保隴山自守。次年,前趙退保上邽的殘餘力量遭後趙將領消滅,蒲洪於是向石虎歸降。石虎於是以蒲洪為冠軍將軍,監六夷軍事,委以西方之事。

咸和八年(333年),後趙鎮守關中的河東王石生聯同鎮洛陽的石朗反抗丞相石虎,蒲洪於是自稱雍州刺史,歸附前涼。同年石生及石朗敗死,石虎命麻秋討伐蒲洪,蒲洪於是率二萬戶向石虎歸降。石虎則以蒲洪為光烈將軍、護氐校尉。蒲洪到長安後游說石虎遷關中豪傑及氐、羌人到東方,充實京師襄國。石虎於是遷秦、雍二州人民及氐、羌族人到關東,並以蒲洪為龍驤將軍、流民都督,命其居於枋頭。

蒲洪後多有征戰,累有戰功,於咸康四年(338年)被石虎拜為使持節、都督六夷諸軍事、冠軍大將軍,西平郡公。當時石虎養子石閔就因蒲洪實力強大,諸子有才且接近京畿,勸石虎誅除蒲洪,但石虎不單沒聽從,對蒲洪的待遇反更為優厚。永和五年(349年)蒲洪遷車騎大將軍、開府儀同三司、都督雍、秦二州諸軍事、雍州刺史,進封為略陽郡公。

同年,石虎去世,石遵繼位,石閔再提出對付蒲洪的建議,石遵於是削去蒲洪都督一職。此舉觸怒了蒲洪,於是憤而向東晉投降。同時,因為當時後趙因諸子爭位內亂,原先被遷到關東的秦雍流民都西歸故土,經過枋頭時就以蒲洪為主,於是令蒲洪的部眾增至十多萬人,當時在首都鄴城的蒲洪子蒲健亦出奔枋頭。新登位的後趙皇帝石鑒為怕蒲洪以其力量威脅中央,於是以蒲洪都督關中諸軍事、征西大將軍、雍州牧、領秦州刺史,讓他率眾返回關中。不過,當時蒲洪其實已經有心稱帝得天下。次年,東晉朝廷以蒲洪歸降,以其為氐王、使持節、征北大將軍、都督河北諸軍事、冀州刺史、廣川郡公。

當時,有人勸蒲洪稱王,蒲洪於是以「艸付應王」的讖文而改姓「苻」,自稱大將軍、大單于、三秦王。早前,後趙將領麻秋自長安率眾返鄴,途中被苻洪派兵俘獲,以其為軍師將軍。麻秋及後勸苻洪先放棄中原,先取關中作為基地,然後才再圖中原。苻洪十分同意,但不久麻秋就在宴會中以毒酒毒殺苻洪,意圖併吞其部眾;苻健於是斬殺麻秋。中毒的苻洪在死前囑咐苻健在其死後要速速入關,享年六十六歲。

苻健於永和七年(351年)即天王位,建立前秦,追諡苻洪為惠武皇帝,廟號太祖。

%% -*- coding: utf-8 -*-
%% Time-stamp: <Chen Wang: 2021-11-01 11:56:49>

\subsection{景明帝苻健\tiny(351-355)}

\subsubsection{生平}

秦景明帝苻健(317年-355年7月10日),字建業,略阳临渭(今甘肃秦安)人,氐族,苻洪第三子,十六国前秦開國皇帝。苻健繼父親苻洪統領部眾並成功入關,定都長安(今陝西西安),建立前秦。後屢次作戰征服其他反抗前秦的關內勢力,更擊敗北伐的晉軍。

苻健弓馬嫻熟,驍勇果敢,好施予亦善於事奉人,故此深得後趙皇帝石虎父子寵愛,當時石虎心中仍提防苻氏,暗殺了苻健的兩個兄長,但就沒有加害苻健。永和六年(350年),因應苻洪歸降東晉,苻健獲授假節、右將軍、監河北征討前鋒諸軍事、襄國縣公。同年苻洪軍師將軍麻秋毒殺苻洪,意圖併吞苻洪部眾,苻健於是收殺麻秋。苻洪臨死時向苻健說:「我之所以一直未入關中,就是以為能夠奪得中原;今天卻不幸被麻秋那小子加害。中原不是你們兄弟能夠爭奪到的,我死了後,你就快快入關中呀!」苻健於是接領父親的部眾,去掉父親自稱的大都督、大將軍、「三秦王」的稱號,稱東晉所授的官爵,並派叔父苻安到東晉報喪,請示朝命。

同年,後趙新興王石祗在襄國(今河北邢台)即位為帝,又以苻健為都督河南諸軍事、鎮南大將軍、開府儀同三司、兗州牧、「略陽郡公」。不過,苻健當時並沒有助石祗對付冉閔,反將目標對準關中,只為麻痺當時據有關中的杜洪才接受後趙的任命。苻健又在駐地枋頭(今河南浚縣西)興治宮室,教人種麥,顯得根本沒有心思佔領關中。但及後苻健就自稱晉征西大將軍、都督關中諸軍事、雍州刺史,率眾西進,並在盟津渡過黃河。渡河前,苻健命苻雄和苻菁分別領兵從潼關(今陝西渭南市潼關縣北)和軹關(今河南濟源東北)進攻,自己則跟隨苻雄渡河,並在渡河後燒掉浮橋,意在死戰。杜洪部將張先在潼關抵抗苻健軍,但被擊敗。及後苻健派苻雄兵行渭北,附近的氐、羌酋長都斬杜洪使而向苻健投降,苻菁、魚遵經過的城邑亦都投降,更在渭北生擒張先,令三輔地區大致都落在苻健之手。杜洪見局勢如此,唯有退守長安,但苻健隨即進攻長安,杜洪被逼棄長安而逃奔司竹(今陝西司竹鄉),苻健於是進據長安。苻健見長安人心思晉,於是向東晉獻捷報,並與東晉征西大將軍桓溫修好。於是令秦雍二州的少數民族和漢人都向苻健歸附,苻健亦攻滅佔領上邽(今甘肅天水市),不肯歸降的後趙涼州刺史石寧。

永和七年(351年),左長史賈玄碩請苻健依劉備稱「漢中王」事,表苻健為都督關中諸軍事、大將軍、大單于、「秦王」。但苻健則假裝憤怒的說:「我豈有能力當秦王呀!而且出使東晉的使者還未回來,你們又怎知我的官爵呀。」然而,不久就又暗示賈玄碩等為他上尊號,最終在再三推讓後,于正月丙辰日(351年3月4日)即天王、大單于位,大封宗室及諸子為公爵,建國號大秦,年號「皇始」,正式建立前秦政權。次年正月辛卯日(352年2月2日),苻健稱帝,進諸公爵為王爵,並授大單于位予太子苻萇。

皇始元年(351年),被苻氏驅逐的杜洪引東晉梁州刺史司馬勳伐前秦,苻健於是率兵在五丈原擊退他。司馬勳敗歸漢中(今陝西漢中)後,杜洪被其部將張琚所殺,不久苻健領二萬兵攻滅張琚,更派兵擄掠關東,助後趙豫州刺史張遇擊敗東晉將領謝尚,及後擄張遇及其部眾回長安,並對張遇授官。後張遇謀反事敗,引發雍州孔特等人舉兵反抗前秦。最終苻健亦派兵成功平定。

皇始四年(354年),桓溫北伐,自率主力軍自武關(今陝西丹鳳縣東)直取長安,另命司馬勳在進攻隴西。前秦初戰不利,被桓溫進攻至長安東南防近的灞上,逼得苻健要盡出三萬精兵出城抵禦桓溫。然而因桓溫並不急於進攻,而且苻健先晉兵一步收取熟麥,故此最終逼得桓溫退兵,苻健更乘勢追擊晉軍,大敗對方。

苻健勤於政事,多次召見公卿談論治國之道,而且一改後趙時苛刻奢侈之風,改以薄賦節儉,更專崇儒學,禮待長者,故此得到人們稱許。

皇始四年(354年),皇太子苻萇在追擊桓溫時受傷,同年傷重而死。次年(晉永和十一年,355年),苻健因讖文中有「三羊五眼」字句,遂以淮南王苻生當太子。至當年六月,苻健患病,苻生在苻健宮室侍疾,而當時任太尉的平昌公苻菁則以為苻健已死,直接領兵入宮,打算殺死苻生自立。但到東掖門時,苻健知道宮中發生事變,自登端門,陳兵自衞。當時苻菁部眾見苻健未死,於是驚懼潰散,苻健於是拿下苻菁,將他殺死。不久,苻健以太師魚遵、丞相雷弱兒、太傅毛貴、司空王墮、尚書令梁楞、尚書左僕射梁安、尚書右僕射段純及吏部尚書辛牢等為輔政大臣。但又告訴太子:「六夷酋帥及掌權的大臣,若果不遵從你的命令,那就立即除去他們。」。六月乙酉日(7月10日),苻健病逝,享年三十九歲。苻健死後諡為明皇帝,廟號稱世宗,後改諡為景明皇帝,廟稱高祖。

\subsubsection{皇始}

\begin{longtable}{|>{\centering\scriptsize}m{2em}|>{\centering\scriptsize}m{1.3em}|>{\centering}m{8.8em}|}
  % \caption{秦王政}\
  \toprule
  \SimHei \normalsize 年数 & \SimHei \scriptsize 公元 & \SimHei 大事件 \tabularnewline
  % \midrule
  \endfirsthead
  \toprule
  \SimHei \normalsize 年数 & \SimHei \scriptsize 公元 & \SimHei 大事件 \tabularnewline
  \midrule
  \endhead
  \midrule
  元年 & 351 & \tabularnewline\hline
  二年 & 352 & \tabularnewline\hline
  三年 & 353 & \tabularnewline\hline
  四年 & 354 & \tabularnewline\hline
  五年 & 355 & \tabularnewline
  \bottomrule
\end{longtable}


%%% Local Variables:
%%% mode: latex
%%% TeX-engine: xetex
%%% TeX-master: "../../Main"
%%% End:

%% -*- coding: utf-8 -*-
%% Time-stamp: <Chen Wang: 2019-12-19 10:09:16>

\subsection{苻生\tiny(355-357)}

\subsubsection{生平}

秦越厉王苻生(335年-357年),字長生,略陽臨渭(今甘肅秦安)氐族人。十六國時期前秦景明帝苻健的第三子。史載苻生「荒耽淫虐,殺戮無道,常彎弓露刃以見朝臣,錘鉗鋸鑿備置左右」在位兩年期間殺害了多位大臣,以及做了多項殘忍變態的事。最終苻生被苻堅發動政變推翻,降封越王,不久被殺。不過後世亦有人認為苻生的暴政其實是史家誣捏渲染的結果。

苻生天生就只有一隻眼,年幼而無賴,爺爺苻洪因而十分討厭他。一次苻洪特地戲弄他,問侍者:「我聽說瞎子都只有一行眼淚,這是真的嗎?」侍者回答:「是呀。」在場的苻生聽後大怒,取出佩刀自殘,流出一行血,說:「這也是一行眼淚呀。」苻洪見狀大驚,鞭打他。苻生說:「我耐得下兵器,受不住鞭打!」。苻洪骂他:「你再是這樣,我就要把你送去当奴隶!」苻生竟答:「那可不就像石勒那樣嗎?」當時苻洪正歸屬後趙,而當時後趙皇帝石虎心中其實十分忌憚苻氏的勢力,故苻洪聽後震惊,光着脚就跑来遮住他的嘴巴。苻洪隨後勸苻健把苻生殺掉,但苻健要動手時就被其弟苻雄制止,說:「男孩子長大後就會改過的了,為何要這樣做呢!」

苻生长大后,力大無比,能徒手格击猛兽,奔跑速度飛快,而擊、刺、騎射的能力亦勇冠一時。皇始元年(351年),苻健稱天王,建立前秦,苻生獲封為淮南公,次年苻健稱帝,苻生進封淮南王。皇始四年(354年),桓温北伐前秦,苻生與太子苻萇、丞相苻雄等出兵迎擊,他就曾經十多次单马突擊晉軍,令晉軍傷亡甚大。

太子苻萇在追擊撤退的桓溫軍隊時受了傷,不久死去,苻健以讖言「三羊五眼」應符,於皇始五年(晉永和十一年,355年)立苻生為太子。同年,苻健患病,太尉苻菁乘時想殺苻生奪位但失敗被殺。隨後苻健以太師魚遵、丞相雷弱兒、太傅毛貴、司空王墮、尚書令梁楞、尚書左僕射梁安、尚書右僕射段純及吏部尚書辛牢八人為顧命大臣,輔助苻生。然而,苻健慮及苻生凶暴嗜酒,擔心他不能保全家業,被大臣有機可乘,於是對苻生說:「六夷酋帥及掌權的大臣,若果不遵從你的命令,那就立即除去他們。」

同年六月乙酉日(355年7月10日),苻健死,次日苻生即位為帝,改元壽光。不過,苻生本身酗酒,在登位後就常常酒醉,群臣上朝都很少見到苻生,連群臣的上奏都因苻生長醉而被擱在一邊。即使上朝,苻生每當發怒都只會殺人,即位後就多次出現殺戮大臣以至殘害生命的凶殘事件,苻健設的八名輔政大臣全都被苻生所殺。最終造成「宗室、勳舊、親戚、忠良殺害殆盡,王公在位者悉以告歸,人情危駭,道路以目」的狀況。

苻生即位後,立刻就改了年號,當時群臣上奏:「先帝死後未逾年而改元,不合禮法呀。」苻生卻大怒,要找出最初提出這上奏的大臣,最終找出了顧命大臣之一的段純,就將他殺害。後來,中書監胡文及中書令王魚向苻生報告天象:「最近有客星(彗星)在大角,熒惑(火星)入東井。大角,是皇帝之坐;東井,表示秦地;按占卜,不出三年,國內將有大喪,大臣會被殺戮,希望陛下自脩德行以避禍。」苻生卻說:「皇后和朕位置相應,可以應了國喪之劫。毛太傅、梁車騎、梁僕射受遺詔輔政,就應了大臣被戮的劫。」於是就殺了梁皇后、毛貴、梁楞及梁安四人;太師魚遵亦於壽光三年(357年)因民謠「東海大魚化為龍,男皆為王女為公」而被殺。又一次苻生與大臣飲宴,更在奏樂時唱起歌來,命尚書令辛牢勸酒。然而,就因為大臣們都沒有全都醉倒,於是就拿起弓將辛牢射殺。更有一次在咸陽故城設宴,將遲到的大臣殺害。

另外,苻生亦寵信趙韶、董榮等人,丞相雷弱兒以他們亂政,經常公開在朝堂批評他們,他們於是在苻生面前中傷雷弱兒。苻生於是誅殺雷弱兒及其家人,最終因為雷弱兒南安羌族酋長的身分,各羌族部落都有離心。司空王墮亦痛恨董榮等人,不肯親附,在董榮的唆使下,苻生又殺王墮以應日蝕之變。

亦因苻生天生殘疾,「不足、不具、少、無、缺、傷、殘、毀、偏、隻」等字都是要避諱的,絕不能說。但就有不少大臣和侍從因此而死。其中太醫令程延在研安胎藥時向苻生解釋人參,就說了「雖小小不具,自可堪用」而被苻生下令鑿出雙眼,然後斬首。

苻生賞罰沒有準則,大臣不論稱頌他還是批評他稍有不當,都可能被殺,但寵臣的姦佞之言卻都接納。而苻生的姬妾只要表現得稍不合其意,都會被殺,並棄屍渭水。苻生又愛虐待動物,活活的剝下牛、羊、驢、馬的皮毛,或者用熱水燙雞、豬、鵝,將三、五十隻這樣的動物一起放在殿上觀賞。苻生甚至還將死囚的臉皮活活剝掉,命其在群臣面前跳舞。苻生更曾命宮女與男子裸體在其面前性交,甚至曾在路上看見一對同行的兄妹,就命他們亂倫,兄妹最終因不肯聽從而被殺。受斬腳、刳胎、拉脅、鋸頸等其他酷刑的人亦數以千計。

苻生聽到有對自己的怨言,更下詔書稱自己並沒有不善,自己所作的根本不算濫刑暴虐。據說當時還有食人野獸橫行,平民為了避開猛獸自保就聚居而且荒廢農業。苻生則認為野獸吃飽了人就會走,不會長久的,且認為天降災劫其實正是對應平民一直犯罪,協助天子以刑罰教導平民而已,只要不犯罪就不必怨天尤人。

苻生曾經命三輔居民興建渭橋,金紫光祿大夫程肱以妨礙農業為由勸諫,反觸怒苻生,被殺。又一次長安突然颳起大風,極之影響人們活動,苻生舅舅左光祿大夫強平於是借天變而勸諫苻生愛護禮待公卿,致敬宗社,去如秋霜的威嚴而立三春般的恩澤等。但苻生則認為強平是妖言,不顧臣下以至太后的懇求,堅持殺死強平。

壽光三年(晉升平元年,357年),姚襄進圖關中,更派人招納因雷弱兒被誅而產生離心的關內羌胡。苻生於是派了衞大將軍苻黃眉等率兵抵抗,最終大敗敵軍,更殺姚襄,令姚襄弟姚萇率眾歸降。苻黃眉立了大功,但凱旋後卻沒有獲得苻生褒賞,反而被多次當眾侮辱。苻黃眉因而憤怒,圖謀殺死苻生,但風聲洩露,反被殺,更株連不少王公親戚。

而當時御史中丞梁平老等人都勸有時譽的苻生堂弟、東海王苻堅殺苻生以救國,苻堅同意但不敢發難。但六月有一晚,苻生對侍婢表示翌日就要殺苻法、苻堅兩兄弟,侍婢於是立刻告訴二人,於是二人與強汪、梁平老和呂婆樓等都率兵衝入宮,宮中宿衞將士知道苻堅奪位都向其投降。苻生當時仍然在酒醉中,知有人攻來,就大驚,問侍從:「那是甚麼人?」侍從答:「是賊!」苻生就說:「為甚麼不下拜!」苻堅兵眾聽後大笑,苻生更說:「還不快快下拜,不拜的我就斬了他!」苻堅於是廢苻生為越王,自己繼承帝位,並降稱天王。不久,苻生被苻堅杀害,享年二十三歲,諡為厲王。儿子苻馗被封为越侯。

苻生无后。苻坚后来平定苻生弟苻廋等人叛乱,赐苻廋死,赦免苻廋诸子,并安排苻廋的儿子过继苻生为后。

苻洪:「此兒狂悖勃,宜早除之,不然,長大必破人家。」

薛讚、權翼:「主上猜忍暴虐,中外離心。」

《晉書》史臣曰:「長生慘虐,稟自率由。覩辰象之災,謂法星之夜飲;忍生靈之命,疑猛獸之朝飢。但肆毒於刑殘,曾無心於戒懼。招亂速禍,不亦宜乎!」

《晉書》贊曰:「長生昏虐,敗不旋踵。」

但有些記述表示所謂苻生暴虐也可能是史臣渲染的結果。楊衒之《洛陽伽藍記》卷二記載隱士趙逸之言,云:「國滅之後,觀其史書,皆非實錄,莫不推過於人,引善自向」,如「苻生雖好勇嗜酒,亦仁而不殺。觀其治典,未為凶暴,及詳其史,天下之惡皆歸焉。苻堅自是賢主,然賊君取位,妄書君惡,凡諸史官,皆是類也。」劉知幾《史通》曲筆篇云:「昔秦人不死,驗苻生之厚誣」,即是據此。

呂思勉亦懷疑苻生的一系列殘忍殺人、文詞避諱、以刀刃錘斧威懾群臣等都是史官誣陷、醜化苻生的結果。誅殺梁安、雷弱兒等人亦因他們其實是有通晉的嫌疑,是不得已,卻招來謗毀之聲。稱「他如怠荒、淫穢,自更易誣。《金史·海陵本紀》述其不德之亂,連章累牘,而篇末著論,即明言其不足信,正同一律。」

\subsubsection{寿光}

\begin{longtable}{|>{\centering\scriptsize}m{2em}|>{\centering\scriptsize}m{1.3em}|>{\centering}m{8.8em}|}
  % \caption{秦王政}\
  \toprule
  \SimHei \normalsize 年数 & \SimHei \scriptsize 公元 & \SimHei 大事件 \tabularnewline
  % \midrule
  \endfirsthead
  \toprule
  \SimHei \normalsize 年数 & \SimHei \scriptsize 公元 & \SimHei 大事件 \tabularnewline
  \midrule
  \endhead
  \midrule
  元年 & 355 & \tabularnewline\hline
  二年 & 356 & \tabularnewline\hline
  三年 & 357 & \tabularnewline
  \bottomrule
\end{longtable}


%%% Local Variables:
%%% mode: latex
%%% TeX-engine: xetex
%%% TeX-master: "../../Main"
%%% End:

%% -*- coding: utf-8 -*-
%% Time-stamp: <Chen Wang: 2019-12-19 10:13:46>

\subsection{宣昭帝\tiny(357-385)}

\subsubsection{文桓帝生平}

苻雄(4世纪?-354年7月26日),字元才,略陽臨渭(今甘肅秦安)氐族人。十六国時前秦的开国元勋、宗室。苻洪之幼子,景明帝苻健之弟,屡建军功,官至丞相,曾參與抵抗東晉將領桓溫發動的北伐戰爭。

苻雄年輕就已熟讀兵書、富有謀略,亦擅長射馬射箭;此外亦有為政治術,慷慨施予地位不高但有才德的人。因著父親苻洪在後趙滅前趙後歸附後趙並接受其官職,苻雄亦仕於後趙,更因戰功而獲後趙君主石虎授予龍驤將軍。

永和五年(349年),石虎去世,諸子爭位令國內漸亂,而苻洪亦因遭當時後趙皇帝石遵削職而叛投東晉。次年,後趙將領麻秋東歸鄴城,苻雄受父命領兵迎擊,成功俘獲麻秋,並以其為軍師將軍。但不久麻秋就藉宴會而以毒酒毒殺苻洪,意圖盡收苻氏部眾。苻健殺麻秋後接掌苻洪部眾,並順從父親遺言,進據關中。

當時關中為自稱晉臣的杜洪所控制,苻健因應人心思晉,於是稱東晉早前加予的官爵,並以苻雄為輔國將軍。不久苻健正式出兵關中,苻雄在大軍渡過黃河後就受命率五千兵取道潼關進攻長安,作為苻健的前驅。苻雄在潼關以北擊敗杜洪派去抵抗的張先,及後苻雄北巡渭北,所過的城邑都向其歸降。苻健進據長安後向東晉獻捷,更得秦、雍二州的胡族及漢人歸附,苻雄亦攻陷後趙涼州刺史石寧據守的上邽,斬殺石寧,鞏固苻氏在關中的統治。

皇始元年(351年),苻健稱天王、大單于,正式建立前秦,並封苻雄為東海公,以其為都督中外諸軍事、丞相、領車騎大將軍、雍州牧。次年,以苻雄為首的百官上請苻健稱帝,苻雄於是進封東海王。

同年,東晉安西將軍謝尚因不能安撫歸附的後趙豫州刺史張遇而令其叛變,謝尚於是與姚襄攻伐張遇,而苻雄就奉命與苻菁出兵略地關東,並救援張遇。最終苻雄在潁水的誡橋擊敗晉軍,逼其退還淮南,並略陳郡、穎川、許昌及洛陽附近共五萬多戶及張遇回軍關中。

同年,苻雄又在隴西擊敗後趙將領王擢,令其逃奔前涼。皇始三年(353年)二月,王擢會同前涼軍隊伐秦,苻雄又率兵擊敗王擢等。但隨著秦州刺史苻願及將軍苻飛分別敗於王擢及楊初,苻雄等於是回屯隴東,而不久張遇更在聯結關中豪族反叛,雖然張還事敗被殺,但孔特、、劉珍、夏侯顯、喬秉、胡陽赤及呼延毒就各自據城起兵叛秦。苻雄於是回軍長安並與苻法、苻飛等分兵平定孔特等人。

皇始四年(354年),就在苻雄攻下胡陽赤據守的司竹,張遇謀反引發的數個主要叛亂勢力僅剩下呼延毒及喬秉時,東晉征西大將軍桓溫發動北伐,自荊州進攻前秦,另由司馬勳率偏師由梁州北上關中。苻健於是派苻雄、太子苻萇等人共率五萬軍抵抗,但大軍在藍田縣被桓溫率領的主力擊敗,苻雄亦在白鹿原敗於桓沖,於是與苻萇等退守長安城南,與雷弱兒所率的三萬精兵共同抵抗。不過桓溫當時只駐屯長安東南的灞上,未有進逼長安,苻雄此時就率領七千騎兵突襲時正經子午谷入關中的司馬勳軍,令其敗退至女媧堡。苻雄及後返回長安,於白鹿原擊敗桓溫,而桓溫亦因乏糧而在六月被逼退兵,呼延毒亦跟隨桓溫南走。苻雄接著討伐佔領了陳倉的王擢及司馬勳,令兩人分別敗退回漢中及略陽,成功解除了桓溫這次北伐帶來的危機。

六月丙申日(7月26日),苻雄在進攻喬秉據守的雍城時去世,苻健聞訊悲傷得嘔血,說:「上天不讓我平定四海麼!為何這麼快就奪去我的元才呀?」追贈魏王,賜諡號敬武,葬禮依西晉安平獻王司馬孚的先例。

其子苻堅後來即位稱天王,追尊苻雄為文桓皇帝。

有載苻雄「醜形貌,頭大而足短」,並沒有雄偉的形象,在後趙任龍驤將軍時更被人稱為「大頭龍驤」。
苻雄作為前秦開國元勳,亦是君主的親弟弟,位至丞相,位高權重,但為人謙虛恭順,遵奉法度,加上在政事和軍事都有才能,故深受苻健倚重,更稱:「元才,是我的姬旦。」

\subsubsection{宣昭帝生平}

秦宣昭帝苻坚(338年-385年10月16日),字永固,一名文玉,略阳临渭(今甘肃秦安)人,氐族,苻雄之子,苻洪之孫,苻健之侄,是十六国时期前秦的君主,称大秦天王。

初封東海王,後發動政變推翻堂兄苻生而即位,在位期間重用漢人王猛,亦推行一系列政策與民休息,加強生產,終令國家強盛,接著以軍事力量消滅北方多個獨立政權,成功統一北方,並攻佔了東晉領有的蜀地,與東晉南北對峙。苻堅於383年發動戰爭意圖消滅東晉,史稱淝水之戰,但最終敗給東晉謝安、謝玄領導的北府兵,國家亦陷入混亂,各民族紛紛叛變獨立,苻堅最終亦遭羌人姚萇殺害,谥号宣昭,庙号世祖。

苻堅的母親曾夢見受到神靈寵幸而懷孕,足足懷了十二個月才生下苻堅,苻堅生於正月初二日,年僅七歲就已顯現其聰敏善良的特質,且舉止都循規蹈矩。又因妥貼地侍奉祖父苻洪,不需詢問就能猜到祖父的想要取甚麼,並及時為祖父拿來,故此深得苻洪疼愛。八歲時,苻堅就請求苻洪請老師到家教他學習,苻洪見他熱心求學十分高興,欣然同意。

皇始四年(354年),苻雄去世,苻堅承襲父爵東海王。另苻堅亦獲授龍驤將軍,苻健更以苻洪曾經在後趙獲授此號勉勵苻堅, 苻堅當時亦「揮劍捶馬」,被苻健的話所感動和激勵,士卒見此,亦心服苻堅。苻堅當時亦博學多才藝,更有經略大志,廣交豪傑,結交了呂婆樓、強汪、梁平老及王猛等人,都成為其左右手。

壽光三年(357年),姚襄謀圖關中,並聯結前秦境內的羌人,苻堅與苻黃眉、鄧羌等人率兵抵抗,終在鄧羌成功誘使姚襄出擊而由苻黃眉率主力將姚襄擊敗,並擒殺姚襄,逼令姚襄弟姚萇率其部眾歸降前秦。然而,當時前秦皇帝苻生賞罰失當,兇殘好殺,苻黃眉因立大功後未受褒賞,反受侮辱而謀反。雖然最終苻黃眉謀反失敗,但苻堅當時很有聲譽,姚襄舊將薛讚和權翼亦欣賞苻堅的才能,並勸苻堅學湯武伐昏君奪帝位 ;梁平老等人亦勸苻堅謀反。同年,苻堅與其兄苻法得知苻生有意加害,於是先發制人,入宮罷黜苻生,不久更殺死苻生。苻堅將帝位讓給苻法,但苻法自以庶出不敢受。苻坚在群臣的勸進下即位,並降號天王,稱大秦天王。即位後苻堅先誅除苻生寵信的董榮等人,隨後擢用李威、呂婆樓、王猛、權翼、薛讚等人。又追復被苻生所誅殺的八個顧命大臣的官位,隨才選用其子孫為官。

苻堅即位後,亦修整一些名實不符的官職,恢復已絕的宗祀,上禮神祇,鼓勵農業,設立學校,扶持鰥寡孤獨和年老無依者。另亦褒揚稱頌一些有特殊才行、孝友忠義、有德業的人。後苻堅下令各地方官員都上舉孝悌、廉直、文學、政事四項才德的人才,若真的是人才就得賞賜,否則就被降罪;另苻堅亦不優待宗室,即使是宗室中人,若無才幹都會被棄用,於是當時國內官員都十分稱職。而通過開墾耕地,令前秦倉庫充實,人民溫飽而令盜賊也少了。

苻堅亦下令與民休息,在即位次年(358年)討平於并州叛變的張平後就下令偃甲息兵,直至365年出兵平定劉衞辰及曹轂的叛亂前都沒有大型的軍事行動。苻堅又應當時秋旱而下令減省膳食和暫停奏樂,將金玉錦繡等貴重物品散發給軍士,並命後宮省儉服飾。苻堅更開發山澤,且得出的資源不限於官府,連平民也可用。

至甘露六年(364年),苻堅下令各公國自置中尉、大農及其他官屬,然而眾人卻以當時富商趙掇、丁妃等人車服盛如王侯,紛紛延攬這些富商為二卿。苻堅於是下令延攬富商為卿者降爵位到侯爵,並下令沒爵位或官職的人都不能在都城百里以內乘車馬;工商、奴隸及婦人亦不得穿戴金銀錦繡,違者處死。

另一方面,苻堅重用漢人王猛,機要之事王猛幾乎無不知道,這令一眾氐族豪族及元勳十分不滿。其中特進樊世自恃是氐族豪族,且有大功勳,當眾直斥王猛竊取為前秦立下赫赫功勳的功臣之成果。苻堅知道後,決意殺樊世以威懾所有氐族豪族。樊世死後,各氐人都爭相批評王猛,苻堅更為王猛而謾罵和鞭撻大臣,終令氐人都畏懼王猛,壓制了氐族豪強對王猛新政的反抗力量。而王猛於359年捕殺酗酒橫行,掠貨擄人的強太后弟強德,苻堅想下令赦免亦趕不及,後來不但沒有問罪王猛,更讓王猛在數十日內處罰了二十多個權豪貴戚,其嚴正執法亦為苻堅所允許,亦為苻堅所認同。

甘露六年(364年),汝南公苻騰謀反被殺,當時王猛以苻生諸弟尚有五人,建議苻堅除去五人,否則終會為患,然而苻堅不聽。至次年,因著劉衞辰及曹轂的叛亂,苻堅親自率軍出征平定,並北巡朔方以撫諸胡。時為征北將軍的苻幼趁機領兵進攻當時由太子苻宏、王猛及李威留守的首都長安,只因李威領兵擊斬苻幼而平定亂事。

苻幼起事時其實還暗中聯結了征東大將軍、并州牧、晉公苻柳以及征西大將軍、秦州刺史、趙公苻雙,但苻堅以二人分別為伯父苻健愛子及同母弟弟而不問罪,亦不將此事公布。然而,二人卻與時為鎮東將軍、洛州刺史的魏公苻廋及安西將軍、雍州刺史的燕公苻武共謀作亂。苻堅得知,於是召眾人到長安,但四人就在建元三年(367年)十月各據州治起兵反叛,苻堅試圖勸其罷兵,答應一切如故,不作追究,並以齧棃 為信物,但四人都沒有任何動搖。次年正月,苻堅正式派軍鎮壓叛亂,派楊成世、毛嵩、鄧羌、王猛、張蚝等人分途出兵,分別進攻四地。但當時楊成世及毛嵩都分別敗於苻雙和苻武的叛軍,逼使苻堅再將王鑒、呂光等人率兵再攻。最終王鑒、呂光及王猛等先後擊敗並斬殺四公,才令亂事成功於當年平定。而在苻堅進攻苻廋時,苻廋主動獻州治陝城(今河南陝縣)歸降前燕,並請兵接應。此舉震動前秦,更逼使苻堅派大軍至華陰(今陝西華陰)防備,只因前燕太傅慕容評拒絕迎降,才避免了更大的危機。

建元五年(369年),前燕吳王慕容垂在擊退東晉桓溫的北伐軍後因受到慕容評排擠,於是出奔降秦。苻堅早於兩年前知道慕容恪去世的消息時就已經有吞併前燕的計劃,還特地派了使者出使前燕以探虛實,然而苻堅因為慕容垂的威名而不敢出兵。現在慕容垂自來,苻堅十分高興,並親自出郊迎接,對其極為禮待,更以其為冠軍將軍,不顧王猛要他提防慕容垂的諫言。

同年十二月,苻堅以前燕違背當日請兵的諾言,不割讓虎牢(今河南滎陽汜水縣西北)以西土地予前秦為藉口出兵前燕,以王猛、梁成和鄧羌率軍,進攻洛陽(今河南洛陽市),並於次年年初攻下。六月,苻堅再命王猛等出兵前燕,自己更親自送行。王猛終在潞川擊潰率領三十多萬 大軍的前燕太傅慕容評,並乘勝直取前燕首都鄴城(今河北臨漳縣西南),苻堅更在王猛圍攻鄴城時親自率軍前往鄴城助戰。拿下鄴城後,正出奔遼東的前燕皇帝慕容暐被前秦追兵生擒,前燕在遼東的殘餘反抗力量亦遭消滅,前秦正式吞併前燕。

另一方面,369年,東晉將領袁真在桓溫北伐失敗後因被桓溫委以戰爭失利的罪責,憤而據壽春(今安徽壽縣)叛變,聯結前燕及前秦。 袁真不久去世,但其子袁瑾仍然堅守壽春,並在前燕亡後繼續向苻堅求救。苻堅於是於371年派王鑒及張蚝救援,但圍城的桓溫派將領桓伊等擊敗王鑒等,逼其退屯慎城(今安徽穎上縣),不久壽春被晉軍攻陷。

在前秦吞併前燕,收撫前燕領土的同一年,名義上臣服於前秦的仇池公楊世死,其子楊纂襲位後只受東晉朝命,斷絕與前秦的臣屬關係,苻堅遂在次年(371年)派兵進攻仇池。當時楊纂叔父楊統正與楊纂兵戎相見,東晉梁州刺史楊亮知道前秦進攻後亦派了郭寶等領兵協助楊纂,然而最終楊纂軍大敗,在仇池兵臨城下及楊統率眾降秦之下,楊纂只得出降。此後前秦命參與進攻仇池的將領楊安鎮守仇池。前仇池至此滅亡。當時苻堅有意在河西樹立威信,以德懷民,於是盡釋早前俘獲的前涼將領陰據及其所統五千兵士,前涼君主張天錫在佑道前仇池被前秦攻滅後甚為畏懼,至此就被逼向前秦稱藩。吐谷渾君主碎奚亦因前仇池滅亡而遺使向前秦進貢,苻堅亦授予其官職爵位。另外,苻堅又出兵攻伐隴西鮮卑首領乞伏司繁,盡降其眾,苻堅留乞伏司繁在長安,只由其堂叔乞伏吐雷統眾。

建元九年(373年),東晉梁州刺史楊亮派其子楊廣進攻仇池。但楊廣敗於仇池守將楊安,原先駐守沮水防備前秦的各軍戍更因而棄守潰逃,逼使楊亮退守磬險。而楊安亦趁機進攻東晉,進攻漢川。不久,苻堅更命益州刺史王統領攻漢川,毛當等攻劍門(今四川劍閣東北),大舉進攻東晉梁、益二州。楊亮在青谷率巴獠抵抗但失敗,只得退保西城(今陝西安康西北),結果漢中(今陝西漢中)、劍閣(今四川劍閣)、梓潼(今四川梓潼縣)等地先後失陷。東晉益州刺史周仲孫在緜竹(今四川綿竹縣)要抵抗來侵的朱肜部時,另一邊的毛當已經快攻到益州治所成都(今四川成都),周仲孫唯有逃到南中,於是前秦攻下了益、梁二州。

次年,益州發生叛亂,蜀人張育、楊光起兵反抗前秦,並向東晉稱藩,而巴獠酋帥張重、尹萬等亦參與,苻堅於是命鄧羌入蜀鎮壓;同一時間,東晉益州刺史竺瑤及威遠將軍桓石虔則受命入蜀,進攻墊江(今重慶墊江縣)。當時張育等人圍攻成都,但期間他們內訌爭權,終被鄧羌等人擊敗,叛亂被平定。竺瑤和桓石虔雖於墊江擊敗寧州刺史姚萇,但不能擴大戰事,只得退還巴東,前秦始終固守了蜀地。

建元十二年(376年),苻堅以張天錫「雖稱藩受位,然臣道未純」為由出兵十三萬進攻前涼。當時苻堅亦派閻負和梁殊出使前涼,徵召張天錫到長安,然而張天錫不願投降,決意與前秦決一死戰,下令斬殺二人,並派馬建抵抗前秦。隨著前秦軍西渡黃河,攻下纏縮城(今甘肅永登縣南),張天錫更派掌據到洪池(今甘肅天祝縣西北烏鞘嶺)協同馬建作戰,自己更親自率兵到金昌助戰。然而,前秦軍進攻二人時,馬建竟向前秦投降而掌據戰死,張天錫驚懼而退還都城姑臧(今甘肅武威)。前秦軍接著直攻姑臧,張天錫被逼出降,前涼至此滅亡。

隨著先後攻滅前燕、前仇池和前涼三個割據政權,北方唯一的割據政權就是拓跋氏建立的代國。在滅前涼的同一年,苻堅以應劉衞辰求救為由,命幽州刺史苻洛率兵十萬,另派鄧羌等率兵二十萬,一起北征代國。當時代王拓跋什翼犍先後命白部、獨狐部及南部大人劉庫仁抵禦,但都失敗,而什翼犍因病而不能率兵,被逼北走陰山,但高車部族此時卻叛變,什翼犍只得回到漠南,並看準前秦軍稍退,於是返回雲中郡盛樂(今內蒙古和林格爾北)的都城。此時,拓跋斤挑撥什翼犍子拓跋寔君,令其起兵殺死父親及其他弟弟;前秦軍聞訊亦立刻出兵雲中,代國於是崩潰,為前秦所滅。

苻堅隨後殺死拓跋斤及拓跋寔君,拓跋窟咄被強遷至長安,而什翼犍諸子亦被殺,什翼犍孫拓跋珪尚幼,再無於當地有效控制代國統下諸部的人。苻堅因而聽從燕鳳的話,分別以劉庫仁及劉衞辰分統代國諸部,借兩人之間的矛盾互相制衡。至此,前秦成功統一北方,只剩下據有江南地區的東晉。

建元十四年(378年),苻堅派苻丕等人進攻襄陽(今湖北襄陽市),另分一路由慕容垂、姚萇率領的軍隊經武當,配合苻丕進攻襄陽。數月後,兗州刺史彭超請求進攻彭城(今江蘇徐州市),並上言請派重將出兵淮南,與進攻襄陽的苻丕配合,形成東西並進之勢,最終消滅東晉。苻堅同意並派了俱難、毛盛等人進攻淮陰(今江蘇淮陰)、盱眙(今江蘇盱眙縣東北),由彭超都督東討諸軍事。

進攻襄陽的軍隊因著守將朱序堅守以及苟萇意圖孤立襄陽而逼其自降的戰略,一直與晉軍相持至年末。此事令苻丕等遭到彈劾,苻堅亦下令要求苻丕最遲在明年春季就要取勝。苻丕於是轉而急攻,終於在次年正月攻下襄陽。另一方面,晉兗州刺史謝玄於建元十五年(379年)奉命救援彭城,最終雖然護送城內的晉軍和沛郡太守戴逯離開,但彭城仍被前秦攻下,及後秦軍亦先後攻下盱眙和淮陰,並在三阿(今江蘇寶應)圍困晉幽州刺史田洛,威脅東晉江北重鎮廣陵(今江蘇揚州市)。此時,晉軍發動反擊,成功擊敗圍攻三阿的俱難、彭超等,逼他們退屯盱眙;次月二人再失盱眙,退保淮陰,但晉軍水軍當時乘潮北上,焚毀秦軍建在淮河上的橋,並擊敗俱難等,逼其退還淮北。而面對謝玄等的追擊,二人終在君川(今江蘇盱眙縣北)大敗給晉軍。面對東線的大敗,苻堅大怒並收捕彭超,嚇得彭超自殺,又將俱難貶為庶民。

就在建元十四年(378年)東西二線南攻東晉之時,鎮守洛陽的北海公苻重謀反,不過很快就因苻重長史呂光忠於苻堅而被平定,苻重獲赦而返回府第。至建元十六年(380年),苻堅卻再度命苻重為鎮北大將軍,駐鎮薊(今北京)。同年,苻堅亦命行唐公苻洛為征南大將軍,鎮守成都,並命其由襄陽循漢水西上上任。但其實苻洛在立下滅亡代國的大功後因為沒有獲苻堅封為將相重臣,反倒仍以其作為邊境州牧深感不滿,更懷疑命他到襄陽其實是苻堅殺他的陰謀,於是決定叛變。當時雖然只有苻重支持苻洛,但苻洛仍自和龍(今遼寧錦州)率兵七萬直指長安。關中人民恐懼戰亂,人心騷動,盜賊興起,苻堅試圖勸降,於是以永封幽州請苻洛罷兵。然而苻洛拒絕,並聲言要「還王咸陽,以承高祖之業」,更反說若苻堅在潼關候駕,他會以他為上公,還爵東海。苻堅於是大怒,出兵討伐,並在中山與苻重及苻洛的十萬聯軍會戰,終生擒苻洛並斬殺苻重,平定亂事。

事後,苻堅認為關東地區地廣人多,於是決定從原居於三原(今陝西三原縣)、九嵕(今陝西乾縣東北)、武都(今甘肅成縣西)、汧(今陝西鳳翔縣南)、雍(今陝西鳳翔縣南)的氐族人中分出十五萬戶,由各宗室統領分布於各方鎮,如古時諸侯一般。不過,被遷移居方鎮的人們因為要與家人分別,都哀傷號哭,路人看見都感到傷心。

王猛於建元十一年(375年)去世,臨死時說:「晉室現在雖然立於偏遠的江南地區,但承繼正統。現在國家最寶貴的就是親近仁德之人以及與鄰國友好。臣死以後,希望不要對東晉有所圖謀。鮮卑、羌虜都是我們的仇敵,終會成為禍患,應該將他們除去,以利社稷。」 希望苻堅先解決國內鮮卑和羌族等其他少數民族對前秦政權的暗藏問題。不過,苻堅在統一北方後仍未聽從王猛之言,著力解決國內民族問題。

苻堅從車師前部王彌窴及鄯善國王休密馱等處聽說西域有高僧鳩摩羅什,苻堅視為國寶,請求西域派羅什入秦遭到拒絕。建元十八年(382年),苻堅派呂光領七萬大軍征伐西域不服前秦要求的,並於次年正月出發。呂光征伐西域龜茲等國大獲全勝,西域諸國歸附前秦。鳩摩羅什也被呂光攜帶身邊。中國境內只剩東晉一地不是前秦版圖,苻堅急於統一中國,開始謀劃出兵東晉。

建元十八年(382年),苻堅大會群臣,自以能得九十七萬兵力,提出親征東晉,統一全國的計劃。當時秘書監朱肜表示支持,尚書左僕射權翼及太子左衞率石越卻都以東晉君臣和睦,且當時為重臣的謝安及桓沖都是人才,皆予以反對。而當時群臣亦各有意見,未有共識。苻堅見此,就說:「像在道旁建房子去問意見,就因聽太多不同的議論而一事無成,我心中自有決斷。」群臣退下後,苻堅留下其弟苻融繼續和他討論,然而苻融亦以天象不利、晉室上下和睦以及兵疲將倦三點為由反對。苻堅因而大怒,苻融後哭著勸諫,並重提王猛死前的話也未能說動苻堅。後名僧釋道安、太子苻宏、幼子中山公苻詵以至寵妃張夫人皆反對伐晉,苻融等人亦屢次上書表示反對,苻堅仍然不肯放棄出兵東晉的計劃,可見苻堅當時其實下了決心。相反,慕容垂向苻堅表示支持出兵東晉,苻堅聽後十分高興,於是向慕容垂說:「與我平定天下的人,就只有你一個呀。」更賜其五百匹布帛。

建元十九年(383年)五月,東晉荊州刺史桓沖出兵襄陽、沔北及蜀地。桓沖於七月退軍後,苻堅便下令大舉出兵東晉,每十丁就遣一人為兵;二十歲以下的良家子但凡有武藝、驍勇、富有、有雄材都拜為羽林郎,最終召得三萬多人。八月,苻堅命苻融率張蚝、梁成和慕容垂等以二十五萬步騎兵作為前鋒,自己則隨後自長安發兵,率領六十餘萬戎卒及二十七萬騎兵的主力,大軍旗鼓相望,前後千里。十月,苻融攻陷壽陽(今安徽壽縣),並以梁成率五萬兵駐守洛澗,阻止率領晉軍主力的謝石和謝玄等人的進攻。當時正進攻晉將胡彬的苻融捕獲胡彬的所派去聯絡謝石的使者,得知胡彬糧盡乏援的困境,於是派使者向正率大軍在項城的苻堅聯絡:「晉軍兵少易擒,但就怕他們會逃走,應該快快進攻他們。」苻堅於是留下大軍,秘密自率八千輕騎直抵壽陽 。然而,晉將劉牢之及後率軍進攻洛澗,擊殺梁成,前秦軍隊潰敗,謝石等於是率領大軍水陸並進,與前秦軍隔淝水對峙。苻堅和苻融此時從壽陽城觀察晉軍,見其軍容整齊,連八公山上的草木都以為是晉軍,於是說:「這也是勁敵,怎能說他們弱呀!」由此悵然失意並有懼色。苻堅及後答允晉軍要他們稍為後撤,讓晉軍渡過淝水作戰的要求,並認為能待晉軍半渡、陷於河中之時出擊,便能將其一舉擊潰。但當前秦大軍開始後退時,先前於襄陽被擒投誠的降將朱序大叫「秦兵敗矣」,秦軍頓時軍心大亂而潰散。苻融親自騎馬入陣中試圖重整亂軍,但反而墮馬被踩死,晉軍於是追擊潰敗的前秦軍,令前秦軍傷亡慘重,連苻堅本人亦中流矢受傷,單騎逃到淮北。

苻堅敗退到淮北時十分飢餓,有平民送他飯菜,苻堅於是給予賞賜,然而該平民卻拒絕,更稱苻堅自取厄困,自己身為其子民即為其子,不圖回報。苻堅因而大感慚愧。及後苻堅與慕容垂的三萬軍隊會合,隨後一直沿途收集逃散的敗兵,到洛陽時聚集了十餘萬人,百官、儀物和軍容都大致齊備了。後苻堅返回長安,哭悼苻融並告罪宗廟後下令大赦,下令鍛煉兵器並監督農務,撫順孤老及陣亡士兵的家屬,試圖重建國家秩序。

隴西鮮卑的乞伏步頹在苻堅出兵東晉時乘機反叛,苻堅派乞伏步頹的侄兒、原降於前秦的乞伏司繁子乞伏國仁出兵討伐,但二人卻相結。淝水戰敗後,乞伏國仁於是裹脅隴西鮮卑諸部叛變,後建立起西秦。而苻堅在洛陽時,不顧權翼的反對,答允讓慕容垂到河北地區安撫民眾及拜謁慕容氏宗廟陵墓。然而慕容垂後來則乘被當時駐鎮鄴城的長樂公苻丕派往鎮壓丁零人翟斌叛亂的機會,聯結丁零人叛秦,並於建元二十年(384年)反與丁零人圍攻鄴城,建立後燕。在圍攻鄴城的同年,慕容泓知道慕容垂的行動亦在關東收集部眾自立,甚為強盛;慕容沖亦在平陽叛變,後投奔慕容泓,慕容泓於是建立西燕,並率眾進攻長安。

為征討大舉叛變的慕容鮮卑,苻堅徵召鉅鹿公苻叡,令其與竇衝及姚萇同討慕容泓,但最終苻叡兵敗戰死,姚萇遣使謝罪卻因苻堅殺其使者而逃到渭北牧馬場,乘機煽動羌族豪帥共五萬餘家歸附,自稱秦王,建立後秦。苻堅自率二萬步騎討伐後秦軍,屢敗後秦軍,更逼得後秦軍中缺水,更有人渴死,但此時天降大雨,後秦軍隊再起,隨後更反敗前秦軍隊。苻堅見慕容沖等已逼近長安,於是回軍長安並組織抵抗,但所派的苻琳、姜宇都兵敗,慕容沖成功佔領阿房城(今陝西西安市西),長安遭圍困。建元二十一年(385年),苻堅在長安宴請群臣,但當時長安已鬧饑荒,發生人食人的事,諸將回家後都吐出宴中吃下的肉來餵饑餓的妻兒。隨後前秦與西燕軍互相攻伐,互有勝負,但在衞將軍楊定被西燕所俘後,苻堅大懼,竟相信他曾經下令禁止的讖諱之言,留太子苻宏留守長安,自己率數百騎及張夫人、苻詵和苻寶、苻錦兩名女兒一同出奔五將山。然而苻堅到五將山後,後秦將領吳忠就來圍攻。苻堅雖見身邊的前秦軍都潰散,但亦神色自若,坐著安然等待吳忠。吳忠及後將苻堅送至新平幽禁。

姚萇及後向苻堅索要傳國玉璽,苻堅張目喝道:「小小羌胡竟敢逼迫天子,五胡的曆數次序,沒有你這個羌人的名字。玉璽已送到晉朝那裏,你得不到的了!」姚萇於是又派人提出苻堅禪讓給他,苻堅亦說:「禪代,是聖賢的事,姚萇是叛賊,有甚麼資格做這事!」苻堅自以平生都待姚萇不薄,甚至在淝水之戰前將「龍驤將軍」這個祖父曾受以及自己殺苻生奪位時有的將軍號珍而重之地封予姚萇,現在姚萇反叛並逼迫他,於是屢次責罵姚萇以求死,並為免姚萇凌辱兩名女兒,於是先殺苻寶和苻錦。八月辛丑日(10月16日),姚萇命人將苻堅絞死於新平佛寺(今彬縣南靜光寺)內,享年四十八歲。張夫人及苻詵亦跟著自殺。

姚萇為掩飾他殺死苻堅的事,故意諡苻堅為壯烈天王。而苻堅去世同年,苻丕得知其死訊,便即位為帝,諡苻堅為宣昭皇帝,上廟號世祖。征西域後回到涼州的呂光得知苻堅去世,亦諡其為文昭皇帝。

苻坚死後就地埋葬,當地人稱“長角塚”。許多人民尊其為苻王爺奉祀之,謂能避免疫病、兵亂。根據《晉書》記載,姚萇被苻堅冤魂作祟,終至發狂,武士欲去救援,竟然打傷其陰部,大出血而死。萇死前還一直跪地叩首,請求苻堅原諒他。

苻堅除了一系列減省奢侈品、鼓勵農業、停止征戰外,更建立學校,重視文教,尤其留心儒學。苻堅曾下令廣收學官,重視經學,郡國弟子員只要通曉一經或以上就獲授官,亦表彰有才德和努力營田之人,令人們都望得朝廷勸勵,崇尚清廉正直,物資亦豐盛。苻堅更每月親臨太學考拔學生,消滅前燕後更在長安祭祀孔子。而王猛亦助苻堅整順風俗,令全國學校漸興。在苻堅治下的關隴地區豐盛安定,地區回復秩序,工商業興盛,一片繁華景象。及至後來王猛去世後,苻堅仍然尊崇儒學,不但命太子、公侯和官員之子以及中外四禁 、二衞、四軍長上 的將士都要受學,連帶後宮亦設有典學,教宮內宦官及宮婢經學。另亦嚴厲禁止老莊以及圖讖學說。後來西域大宛獻馬,苻堅效法西漢漢文帝送還進貢的千里馬,更加命群臣作《止馬詩》送到西域,以示沒有取千里馬的欲望。最終共有四百多人獻詩。

苻堅亦重視生產,遇上天旱不但曾下令節儉及開山澤資源與民共享,亦督導百姓耕種,自己更親身躬耕藉田,讓苟皇后親身養蠶,以示對農業的重視。後又徵集王侯以下及豪門富戶的家僮奴僕共三萬人開通涇水上流,引水灌溉解決關中水旱問題。

苻堅對於前秦這個多民族組成的國家其實沒有作出民族融合的措施。如隴西鮮卑首領乞伏司繁投降後,只遷乞伏司繁到長安,仍留其部眾在隴西地區;前燕鮮卑族人除了慕容氏皇族及部分關東豪族被遷至關中地區外,尚有大部分留在前燕故地,另亦遷原居中山的丁零族人到新安(今河南新安縣);消滅代國後,苻堅雖然由北方匈奴人代領代國遺眾,但仍居北方。在苻洛叛亂被平定後,苻堅則為更好的管理關東以至各地民族,於是從原集中於關中的氐族人分出十五萬戶,各由宗親率領出鎮,如古分封諸侯般管治地方 。然而此舉卻分散了氐族的民族力量,影響對各地的軍事影響力,而移居關中的各少數民族更成前秦的心腹大患 。

史載苻堅「臂垂過膝,目有紫光」。

苻堅與苻法兄弟友好,然而在苻堅即位之初其母苟氏以苻法年長、賢能以及得人心而殺害苻法,苻堅無奈下只有哭著與他訣別,傷心得吐血。後來其子苻陽因憤恨父親無罪遭戮,而謀反,苻堅亦不誅殺。

苻堅寛貸容人,如後趙舊將張平在秦、燕之間搖擺,維持半獨立狀態 ,357年更以并州叛秦,但仍然加以寛貸,署為右將軍。後苻重在洛陽叛變,苻堅也赦而不誅,後更再派他出鎮,終招來苻重聯同苻洛再叛;而苻洛敗後苻堅仍不殺,只流放他到西海郡。另苻堅亦善待亡國貴族,如前涼張氏、前燕慕容氏等都沒有進行屠殺,甚至頗見親待。

苻堅執政前期大推善政,崇尚節儉,然而在王猛死後,苻堅卻因聽後趙前將作功曹熊邈講述後趙宮室器具的規模,下令以其為將作長史,大修舟艦、兵器,並以金銀裝飾,講求精巧,一改之前節儉之風。慕容農亦因而說:「自從王猛死後,秦的法制日漸頹靡,今日又著重奢侈,大禍將來了。」

苻堅初年虛心接納臣下的諫言,如即位初期曾經登龍門,向群臣展現他甚為滿足於關中的穩固。而權翼、薛讚當時則以夏、商、周、秦四個朝代由興盛的基礎而到最終遭他人所滅,表達出修備德行的重要,穩固的地勢並不足以固國。苻堅聽後大喜,隨後就施行一系列新政與民休息。後苻堅在鄴附近狩獵十多日,樂而忘返,亦聽從伶人王洛的勸言,不再出獵。但後來苻堅卻在出兵東晉等事上聽不下諫言,只想聽到支持自己的論調。

苻堅生母因為年輕守寡,於是寵幸將軍李威,當時史官亦記載此事。但苻堅後來看起居注和史官所著的著作發現載有這種事,於是發怒燒書並大檢史官,要加罪於史官,因著作郎趙淵、車敬等已死才了事。

《晉書》史臣曰:「永固雅量瓌姿,變夷從夏,叶魚龍之遙詠,挺莫苻之休徵,克翦姦回,纂承偽曆,遵明王之德教,闡先聖之儒風,撫育黎元,憂勤庶政。……乃平燕定蜀,擒代吞涼,跨三分之二,居九州之七,遐荒慕義,幽險宅心,因止馬而獻歌,託棲以成頌,因以功侔曩烈,豈直化洽當年!雖五胡之盛,莫之比也。既而足己夸世,複諫違謀,輕敵怒鄰,窮兵黷武。懟三正之未叶,恥五運之猶乖,傾率土之師,起滔天之寇,負其犬羊之力,肆其吞噬之能。自謂戰必勝,攻必取,便欲鳴鷥禹穴,駐蹕疑山,疏爵以侯楚材,築館以須歸命。曾鬥知人道助順,神理害盈,雖矜涿野之強,終致昆陽之敗。道使文渠候隙,狡寇伺間,步搖啟其禍先,燒當乘其亂極,宗社遷於他族,身首罄於賊臣,賊戒將來,取笑天下,豈不哀哉!豈不謬哉!」

《晉書》贊曰:「永固禎祥,肇自龍驤。垂旒負扆,竊帝圖王。患生縱敵,難起矜強。」

苻洪:「此兒姿貌瓖偉,質性過人,非常相也。」

徐統:「此兒有霸王之相。」又曰:「苻郎骨相不恒,後當大貴,但僕不見。」

薛禮、權翼:「非常人也!」

苻廋:「苻堅、王猛,皆人傑也。」

司馬光:「夫有功不賞,有罪不誅,雖堯、舜不能為治,況他人乎!秦王堅每得反者輒宥之,使其臣狃於為逆,行險徼幸,力屈被擒,猶不憂死,亂何自而息哉!《書》曰:『威克厥愛,允濟;愛克厥威,允罔功。』《詩》云:『毋縱詭隨,以謹罔極;式遏寇虐,無俾作慝。』今堅違之,能無亡乎!」又言:「論者皆以為秦王堅之亡,由不殺慕容垂、姚萇故也。臣獨以為不然。許劭謂魏武帝治世之能臣,亂世之姦雄。使堅治國無失其道,則垂、萇皆秦之能臣也,烏能為亂哉!堅之所以亡,由驟勝而驕故也。魏文侯問李克,吳之所以亡,對曰:『數戰數勝。』文侯曰:『數戰數勝,國之福也,何故亡?』對曰:『數戰則民疲,數勝則主驕,以驕主御疲民,未有不亡者也。』秦王堅似之矣。」

歷史學家陳登原認為苻堅有四大善事——文學優良,內政修明,大度容人,武功赫赫。后人对待亡国贵族往往以苻坚之仁为戒,选择屠杀殆尽。

呂思勉:「苻堅在諸胡中,尚為稍知治體者,然究非大器。嘗縣珠簾於正殿,以朝群臣。宮宇、車乘、器物、服御、乘以珠璣、琅玕、奇寶、珍怪飾之。雖以尚書裴元略之諫,命去珠簾,且以元略為諫議大夫,然此特好名之為,其諸事不免淫侈,則可想見矣。」後又以苻堅以慕容沖及前燕清河公主姐弟皆有美色而皆寵幸,直斥其「荒淫」。又指其命呂光征西域是「蓋一欲誇耀武功,一亦貪其珍寶也。」又曰:「堅知晉終為秦患,命將出師之不足以晉,而未知躬自入犯之更招大禍,仍是失之於疏;而其疏,亦仍是失之於驕耳。」

著名作家柏楊於柏楊版資治通鑑第25冊的序言中寫到:在大分裂時代中,苻堅大帝以超時代的睿智之姿,出現舞台,為苦難的北中國人民,帶來一個太平盛世。

據說苻堅生母苟氏曾在漳水遊玩,並在西門豹寺祈子,在當晚夢與神交,於是懷有苻堅。十二個月後苻堅才出生,當時天上有神光照耀門庭,苻堅背上亦有紅色字,寫著「草苻臣又土王咸陽」。後苻洪以此及「艸付應王」的讖言改姓苻氏。

姚苌曾把苻坚的屍體挖出来鞭尸,脫掉衣服用荆棘裹起来,再以土坑埋掉。苻坚的冤魂作祟非常顯著,姚萇後來諸事不順,屢屢敗陣,認為是苻堅顯靈,於是也在軍中樹立苻堅像祈求道:「新平之禍,不是臣姚萇的錯啊,臣的兄長姚襄從陝州北渡,順著道路要往西邊去,像狐狸死時把頭朝向原本洞穴一樣,只是想要見一見鄉里啊。陛下與苻眉攔阻於路上攻擊他,害他不能成功就死了,姚襄遺命臣一定要報仇。苻登是陛下的遠親亦想復仇,臣為自己的兄長報仇,又怎麼說是辜負了義理呢?當年陛下封我為龍驤將軍,跟我說:『朕從龍驤將軍當上了皇帝,卿也好好努力罷!』這明明白白的詔諭非常顯然,好像還在耳邊一樣。陛下已經過世成為神明了,怎麼會透過苻登而謀害臣,忘卻當年說的話呢!現在為陛下立神像,請陛下的靈魂進入這裏,不要計較臣的過失了,聽臣至誠的禱告。」 不過姚萇戰況仍未有改善,反而睡不安穩,並招來苻登批評「古今以來,豈有人殺了主公卻反而為主公立神像請求賜福?他期望會有好處嗎?」姚萇終毀了苻堅神像。據說姚萇死前曾夢見過苻堅率天官、鬼兵去襲擊他,期間他被救援自己的士兵誤傷陰部至大量出血。醒後就發現陰部腫脹,醫者刺腫處則如夢中一樣大量出血,一石有餘。,如此嚇得姚萇發狂胡言,又求苻堅原諒,姚萇不久傷重身亡,臨終前跪伏床頭,叩首不已。

據《湧幢小品》言:傳聞死於新平寺之苻堅託夢該寺寺主摩訶,望該寺改為祭祀苻堅及侍衛十餘人的廟宇。住持不從,該寺所在縣鎮,果然死疫相繼,後不得已,即尊其靈示,改廟後,果真無疾。

道教信徒衍其義,逢瘟疫競建祠避禍,稱為苻王爺、苻家神,並於每年正月初二以太牢奉之,稱為祭苻家神。祭苻家神為台灣道教現有祭典之一,祭典日為每年農曆正月初二。

\subsubsection{永光}

\begin{longtable}{|>{\centering\scriptsize}m{2em}|>{\centering\scriptsize}m{1.3em}|>{\centering}m{8.8em}|}
  % \caption{秦王政}\
  \toprule
  \SimHei \normalsize 年数 & \SimHei \scriptsize 公元 & \SimHei 大事件 \tabularnewline
  % \midrule
  \endfirsthead
  \toprule
  \SimHei \normalsize 年数 & \SimHei \scriptsize 公元 & \SimHei 大事件 \tabularnewline
  \midrule
  \endhead
  \midrule
  元年 & 357 & \tabularnewline\hline
  二年 & 358 & \tabularnewline\hline
  三年 & 359 & \tabularnewline
  \bottomrule
\end{longtable}

\subsubsection{甘露}

\begin{longtable}{|>{\centering\scriptsize}m{2em}|>{\centering\scriptsize}m{1.3em}|>{\centering}m{8.8em}|}
  % \caption{秦王政}\
  \toprule
  \SimHei \normalsize 年数 & \SimHei \scriptsize 公元 & \SimHei 大事件 \tabularnewline
  % \midrule
  \endfirsthead
  \toprule
  \SimHei \normalsize 年数 & \SimHei \scriptsize 公元 & \SimHei 大事件 \tabularnewline
  \midrule
  \endhead
  \midrule
  元年 & 359 & \tabularnewline\hline
  二年 & 360 & \tabularnewline\hline
  三年 & 361 & \tabularnewline\hline
  四年 & 362 & \tabularnewline\hline
  五年 & 363 & \tabularnewline\hline
  六年 & 364 & \tabularnewline
  \bottomrule
\end{longtable}

\subsubsection{建元}

\begin{longtable}{|>{\centering\scriptsize}m{2em}|>{\centering\scriptsize}m{1.3em}|>{\centering}m{8.8em}|}
  % \caption{秦王政}\
  \toprule
  \SimHei \normalsize 年数 & \SimHei \scriptsize 公元 & \SimHei 大事件 \tabularnewline
  % \midrule
  \endfirsthead
  \toprule
  \SimHei \normalsize 年数 & \SimHei \scriptsize 公元 & \SimHei 大事件 \tabularnewline
  \midrule
  \endhead
  \midrule
  元年 & 365 & \tabularnewline\hline
  二年 & 366 & \tabularnewline\hline
  三年 & 367 & \tabularnewline\hline
  四年 & 368 & \tabularnewline\hline
  五年 & 369 & \tabularnewline\hline
  六年 & 370 & \tabularnewline\hline
  七年 & 371 & \tabularnewline\hline
  八年 & 372 & \tabularnewline\hline
  九年 & 373 & \tabularnewline\hline
  十年 & 374 & \tabularnewline\hline
  十一年 & 375 & \tabularnewline\hline
  十二年 & 376 & \tabularnewline\hline
  十三年 & 377 & \tabularnewline\hline
  十四年 & 378 & \tabularnewline\hline
  十五年 & 379 & \tabularnewline\hline
  十六年 & 380 & \tabularnewline\hline
  十七年 & 381 & \tabularnewline\hline
  十八年 & 382 & \tabularnewline\hline
  十九年 & 383 & \tabularnewline\hline
  二十年 & 384 & \tabularnewline\hline
  二一年 & 385 & \tabularnewline
  \bottomrule
\end{longtable}


%%% Local Variables:
%%% mode: latex
%%% TeX-engine: xetex
%%% TeX-master: "../../Main"
%%% End:

%% -*- coding: utf-8 -*-
%% Time-stamp: <Chen Wang: 2019-12-19 10:15:33>

\subsection{哀平帝\tiny(385-386)}

\subsubsection{生平}

秦哀平帝苻丕(4世紀-386年),字永叔(或作永敘),略陽臨渭(今甘肅秦安)氐族人。前秦皇帝,宣昭帝苻堅的庶長子,淝水之戰後與後燕君主慕容垂一度相持於鄴城。並在苻堅死後繼承帝位,繼續與後秦、西燕及後燕勢力對抗。最終在進攻洛陽時遭晉軍所殺,死後獲諡為哀平皇帝。

苻丕少時聰慧好學,博通經史。苻堅曾經與他談將略,嘉許了他並命鄧羌教他兵法。苻丕的文武才幹不及叔父苻融,不過他當將領時善於籠絡士卒之心。永興元年(357年)苻堅稱天王時受封為長樂公。

建元四年(368年),苻堅在攻滅叛亂的雍州刺史苻武後,以苻丕為雍州刺史。建元六年(370年)因取消雍州而離任,但次年苻丕就因雍州復置而任使持節、征東大將軍、雍州刺史。後遷征南大將軍,都督征討諸軍事,守尚書令。建元十四年(378年)二月,奉命與苟萇等進攻東晉襄陽。當時前秦軍很快就攻下了襄陽外城,守將朱序只得固守內城,苻丕於是打算急攻內城。然而最終卻聽從了苟萇長期圍困,待其自降的策略,雖然及得慕容垂攻陷南陽郡後與苻丕會合,秦軍仍只一直圍困襄陽;而當時的荊州刺史桓沖以及在次年受命領兵救援襄陽的劉波皆因畏懼秦軍而未敢前進,都沒有起到實質作用。不過,朱序仍一直堅持到年末,前秦御史中丞李柔因而彈劾苻丕等人圍攻襄陽近一年仍未能攻陷,耗費日深而無收效。苻堅亦下詔苻丕要以攻取襄陽贖罪,並命人賜劍苻丕,明言若果不能在下一年春天攻下襄陽就要苻丕以劍自殺。苻丕得詔後惶恐,並下令各軍加緊進攻,終於在次年二月攻下襄陽。

建元十六年(380年),苻堅為加強管理關東領土,於是決定分十五萬戶關中氐族人並分配給宗親重臣,在他們帶領下分駐各重鎮,如同古代諸侯。苻丕則為都督關東諸軍事、征東大將軍、冀州牧,派遣他到鄴鎮守。

苻堅在建元十九年(383年)的淝水之戰大敗給晉軍後返率敗軍回長安,並在洛陽答允讓冠軍將軍慕容垂出撫河北地區。慕容垂到鄴城西南的安陽時修書苻丕,而苻丕知慕容垂北來就已思疑他圖謀作亂,但仍親身迎接,又聽從侍郎姜讓的諫言,放棄襲殺慕容垂的計劃。不久,在新安的丁零人翟斌起兵叛變,苻堅命慕容垂討伐。苻丕當時自覺慕容垂長在鄴城令自己終日都提防他,於是想趁此機會將慕容垂調離鄴城,更希望翟斌和慕容垂打得兩敗俱傷,令自己能從容控制他們,於是給了慕容垂兩千弱兵以及差劣的兵器,並以苻飛龍領一千氐族騎兵作為其副手,作提防監視之用。不久慕容垂請求拜謁前燕在鄴城宗廟遭苻丕拒絕,微服而入亦被亭吏阻止,令其殺掉亭吏,燒亭而去;慕容垂出發後又因知道苻丕想用苻飛龍除掉自己,所以就借機殺了苻飛龍,並開始招集兵士,更密召留鄴的慕容農、慕容楷等出城起兵響應自己。

建元二十年(384年)春,苻丕大宴賓客卻請不來慕容農等,調查三天才知他們已在列人起兵了,而慕容垂及後亦自稱燕王,率兵進攻鄴城。苻丕派了重將石越討伐慕容農等但石越卻兵敗被殺,石越之死更令當地人心騷動。隨著前燕舊臣想相繼響應慕容垂並到鄴城會同慕容垂進攻,慕容垂更寫信給苻丕及苻堅,向其陳述利害,想苻堅放棄鄴城,送苻丕回長安,但遭二人憤怒地拒絕,並回信嚴厲指責慕容垂叛秦。二月,慕容垂就開始進攻鄴城,直至八月仍未能攻下鄴城,但城內糧草已盡,要以松木餵飼戰馬。苻丕向張蚝及并州刺史王騰請兵不得,亦不想向東晉求援;此時謝玄率兵北伐,苻丕派兵抵抗但失敗,終令苻丕屈服,寫信給謝玄說:「我想向你求糧,以西赴國難,當我與援軍相接時就會交鄴城給你。若果不能西進而長安失陷,請你領兵助我保護鄴城。」不過姜讓、最早請苻丕南附東晉的司馬楊膺以及擔任使者的焦逵皆認為苻丕至此仍不肯放下身段,認定事必無成,反而自己修改苻丕的信,改成願意在晉軍來後向東晉歸降,更決定若苻丕屆時不肯就想辦法逼他就範。而當時慕容垂亦派兵圍困鄴城,只留西走長安的路,仍願苻丕自願棄城;而謝玄亦答應出兵救鄴,不但派劉牢之等領二萬兵作援,亦運二千斛米以解城中糧荒。

就在次年(385年)劉牢之北行至枋頭時,楊膺等人改寫苻丕書信並想逼苻丕就範的事被揭發,苻丕於是殺害他們,而因焦逵亦向謝玄等提及此事,令劉牢之聞訊後盤桓不進。此時慕容垂亦因鄴城久久未下而想先取冀州,於是調了慕容農到鄴城。及後因應劉牢之進攻黎陽,苻丕趁慕容垂出兵,留慕容農守鄴圍的機會試圖突圍但失敗,慕容垂亦在擊退劉牢之後回軍鄴城。四月,劉牢之在鄴擊敗慕容垂,終解了鄴城之圍,令慕容垂北走。雖然劉牢之追擊燕軍失敗還須苻丕救援,但苻丕最終都能夠率眾到枋頭獲得晉軍糧食,解決部眾缺糧問題。然而苻丕並非真心與晉合作,亦不曾想放棄鄴城,於是在重返鄴城時就與晉將檀玄發生了戰鬥,終由苻丕取勝並重奪鄴城。

燕、秦兩軍至此時已經相持一整年了,弄得幽、冀地區發生饑荒,人食人且城池都蕭條;而且當時長安亦受到西燕軍隊的攻擊,苻丕於是在當地收兵並要西赴長安。幽州刺史王永因為抵抗不了燕軍進攻而率兵退至壺關,並派使者招請苻丕,苻丕於是率鄴城中六萬多人西赴潞川,並獲張蚝和王騰迎至晉陽。苻丕到了晉陽才知苻堅已經被姚萇所殺,於是發喪並於晉陽南即位為帝,改年號為太安。

在王永等人的協助下,關中及隴右的前秦遺眾都相繼起兵響應苻丕,以對抗慕容氏及姚氏的勢力。太安二年(386年),苻丕留戍晉陽及壺關,自率四萬兵進屯平陽。西燕君主慕容永見此擔憂抵抗不了秦軍,於是請求苻丕讓他取道東歸河北會合慕容垂。但苻丕拒絕並命左丞相王永、俱石子等進攻慕容永。西燕軍於是在襄陵與王永所率的秦軍發生戰鬥,王永及俱石子皆兵敗被殺,苻丕以兵敗,更怕他一直猜忌的苻纂趁他新敗而對其不利,於是率數千南奔東垣,更圖進攻當時受東晉控制的洛陽。晉將馮該就從陝城出兵邀擊苻丕,最終苻丕被殺,除了苻纂等率數萬兵出走杏城外,苻丕統下的官員皆為西燕所得。苻丕死後,族子苻登繼位,諡苻丕為哀平皇帝。

\subsubsection{太安}

\begin{longtable}{|>{\centering\scriptsize}m{2em}|>{\centering\scriptsize}m{1.3em}|>{\centering}m{8.8em}|}
  % \caption{秦王政}\
  \toprule
  \SimHei \normalsize 年数 & \SimHei \scriptsize 公元 & \SimHei 大事件 \tabularnewline
  % \midrule
  \endfirsthead
  \toprule
  \SimHei \normalsize 年数 & \SimHei \scriptsize 公元 & \SimHei 大事件 \tabularnewline
  \midrule
  \endhead
  \midrule
  元年 & 385 & \tabularnewline\hline
  二年 & 386 & \tabularnewline
  \bottomrule
\end{longtable}


%%% Local Variables:
%%% mode: latex
%%% TeX-engine: xetex
%%% TeX-master: "../../Main"
%%% End:

%% -*- coding: utf-8 -*-
%% Time-stamp: <Chen Wang: 2019-12-19 10:16:57>

\subsection{高帝\tiny(386-394)}

\subsubsection{生平}

秦高帝苻登(343年-394年),字文高,略陽臨渭(今甘肅秦安)氐族人,十六国前秦皇帝,苻堅族孫,建節將軍苻敞之子。在苻堅遭後秦君主姚萇殺害後,苻登曾領率氐族殘餘力量於關隴地區對抗後秦,後更被擁立為前秦皇帝。苻登起初屢次獲勝,但終敗給後秦,更遭俘殺。死後獲上廟號太宗,諡高皇帝。

苻登年輕時就勇猛威武,有雄壯的氣慨,但為人粗豪好險而不修小節,並不受苻堅重視。苻登長大成人後卻一改舊習,謹慎厚道,亦看典籍。苻堅曾以其為殿上將軍、羽林監、揚武將軍、長安令,後來因過失被降為狄道長。

淝水之戰後,關中地區大亂,苻登逃到河州牧毛興駐守的枹罕(今甘肅臨夏市)。苻登兄苻同成是毛興的長史,於是請毛興以苻登任其司馬。苻登當時表現得器量不凡,喜歡設奇謀,而他對事物的分析連毛興也十分佩服,然而卻因受毛興所憚而沒有獲重用。

前秦太安二年(386年),時與後秦姚碩德對抗的毛興亦同時與同屬前秦的益州牧王廣及秦州牧王統作戰,頻繁的戰事令厭戰的氐人將其殺害。毛興臨死時就表示苻登能夠消滅姚碩德[1]。同年七月,苻登獲眾人推舉,取代被指年老的氐豪衞平統領原毛興部眾,自稱使持節、都督隴右諸軍事、撫軍大將軍、雍河二州牧、「略陽公」,並即率兵五萬攻佔南安,又派使者向當時前秦皇帝苻丕請求任命。苻丕亦應其自稱授官,並以其為征西大將軍、開府儀同三司、南安王。苻登佔據南安後獲當地胡、漢共三萬多戶歸附,聲勢漸盛,於是進攻姚碩德,並在胡奴阜大敗前往救援的姚萇,更令其身受重傷。

十月,苻丕進攻洛陽時被東晉將領所殺,當時苻丕子苻懿及苻昶都被帶到南安,苻登於是打算立苻懿為帝。但部眾都力勸苻登立長君,並指出非苻登一人不可。苻登於是即位為帝,改元「太初」,立了苻懿為太弟。

當時前秦宗室苻纂為另一軍事力量,他支持苻登令前秦聲勢大盛,並曾與楊定於涇陽(今陝西涇陽縣)大敗姚碩德,更圖謀攻取後秦都城長安。不過苻纂不久卻因不肯自立為帝而遭其弟苻師奴殺害,苻師奴亦遭姚萇擊敗,部眾遭後秦吸納,進攻長安行動亦告吹。

太初三年(388年)二月,苻登與姚萇各據朝那(今寧夏彭陽縣)及武都相持不下,互有勝負。當時關西豪傑見後秦久久不能消滅前秦勢力,很多都轉歸前秦。姚萇終於十月退還根據地安定,苻登亦到新平取軍糧以解軍中饑饉的狀況,並自率萬餘人兵圍姚萇軍營,四面以哭聲震動其軍心;不過姚萇亦命軍人以哭聲回應,苻登見不成功就退兵。

太初四年(389年),苻登在大界留下輜重,自率萬多名輕騎兵進攻安定,先後擊敗安定羌密造保及後秦將吳忠等,並於八月進逼安定。但姚萇卻奇兵夜襲大界,殺害留守的毛氏並擒獲數十名名將,擄掠五萬多人。苻登見此唯有退守根據地胡空堡(今陝西省彬縣西南)。太初六年(391年),苻登因苟曜作為內應而進攻後秦,並擊敗姚萇,殺後秦將吳忠,但姚萇立刻重整軍勢再戰,苻登這次大敗,退兵至郿縣(今陝西眉縣)。同年苻登先後進攻新平及安定,但都遭姚萇擊敗。而當時氐族人強金槌叛歸後秦,兩年前勇略過人的羌人雷惡地亦因遭苻登所忌憚而出奔後秦,次年驃騎將軍沒弈干亦叛降後秦,這些事件都削弱了苻登的力量。

太初七年(392年),苻登以姚萇患病而出兵安定,但在城外九十多里就遭姚萇所派的軍隊攻擊,被逼退還;而姚萇更特意在夜裏命軍隊旁出跟隨苻登軍,苻登聽聞姚萇軍營空無一人,驚懼得說:「他究竟是甚麼人,離開時我不知道,來到時我亦不察覺,人說他快死了,突然又來了。朕和這個羌人活在同一年代,根本是不幸。」

太初九年(394年),苻登知姚萇已死,於是十分高興,並盡率大軍進攻後秦。至夏季,苻登進攻廢橋以得水源,但為後秦將尹緯所阻,部分士兵更渴死。苻登因而急攻尹緯,而尹緯卻大敗苻登,兵眾潰散,苻登單騎逃返原由其弟苻廣留守的雍城(今陝西鳳翔縣南)卻發現苻廣已棄城,另一根據地胡空堡亦遭留守的太子苻崇所棄,苻登無處容身,只有逃到平涼(今甘肅平涼市),收集部眾據守馬髦山。苻登及後向乞伏乾歸求救,得其命乞伏益州領兵救援,卻就在七月苻登率兵迎接乞伏益州時就遇上後秦軍,苻登被生擒並處決,享年五十二歲。

其子苻崇在湟中稱帝,追諡苻登為高皇帝,上廟號太宗。

苻登與後秦軍作戰多年,其為史傳所載的生平事跡多為他在軍旅中的事跡。

苻登取代衞平後,銳意出兵,但當年天旱,兵眾都吃不飽,苻登於是都將戰爭中殺死的敵軍都叫做「熟食」,更向軍人說:「你們早上作戰,黃昏就能吃飽肉了,還怕飢餓麼!」士兵於是就吃屍肉為生,吃飽後都有氣力戰鬥,逼得姚萇急召姚碩德:「你再不來,我們就要被苻登吃光了。」

苻登曾在軍中設苻堅神主,每次作戰或有所決定都會向其稟告。而苻登即皇帝位後要出兵後秦,亦向苻堅稟告,發言後欷歔流涕,更感染了將士們,令他們都在鎧甲和矛上都刻上「死休」二字,以作至死方休之志。立神主一事甚至令時屢敗的姚萇認為這真是苻堅神助,也一度在軍中設了苻堅像。

\subsubsection{太初}

\begin{longtable}{|>{\centering\scriptsize}m{2em}|>{\centering\scriptsize}m{1.3em}|>{\centering}m{8.8em}|}
  % \caption{秦王政}\
  \toprule
  \SimHei \normalsize 年数 & \SimHei \scriptsize 公元 & \SimHei 大事件 \tabularnewline
  % \midrule
  \endfirsthead
  \toprule
  \SimHei \normalsize 年数 & \SimHei \scriptsize 公元 & \SimHei 大事件 \tabularnewline
  \midrule
  \endhead
  \midrule
  元年 & 386 & \tabularnewline\hline
  二年 & 387 & \tabularnewline\hline
  三年 & 388 & \tabularnewline\hline
  四年 & 389 & \tabularnewline\hline
  五年 & 390 & \tabularnewline\hline
  六年 & 391 & \tabularnewline\hline
  七年 & 392 & \tabularnewline\hline
  八年 & 393 & \tabularnewline\hline
  九年 & 394 & \tabularnewline
  \bottomrule
\end{longtable}


%%% Local Variables:
%%% mode: latex
%%% TeX-engine: xetex
%%% TeX-master: "../../Main"
%%% End:

%% -*- coding: utf-8 -*-
%% Time-stamp: <Chen Wang: 2019-12-19 10:17:55>

\subsection{苻崇\tiny(394)}

\subsubsection{生平}

苻崇(4世紀-394年),十六国前秦末主。

苻崇是高帝苻登之子,太初三年(388年)八月立為太子。

太初九年(394年)七月,苻登兵敗,被後秦姚興殺死,崇逃到湟中即帝位,改元延初。十月,苻崇被隴西鮮卑的梁王乞伏乾歸(後來的西秦王)驅逐,逃到隴西王楊定那裡。楊定率領二萬人與苻崇共攻乾歸,先勝後大敗,定及崇俱被殺,乾歸盡有隴西之地。

前秦太子苻宣投靠仇池楊盛,不再設置郡縣,前秦亡。

\subsubsection{延初}

\begin{longtable}{|>{\centering\scriptsize}m{2em}|>{\centering\scriptsize}m{1.3em}|>{\centering}m{8.8em}|}
  % \caption{秦王政}\
  \toprule
  \SimHei \normalsize 年数 & \SimHei \scriptsize 公元 & \SimHei 大事件 \tabularnewline
  % \midrule
  \endfirsthead
  \toprule
  \SimHei \normalsize 年数 & \SimHei \scriptsize 公元 & \SimHei 大事件 \tabularnewline
  \midrule
  \endhead
  \midrule
  元年 & 394 & \tabularnewline
  \bottomrule
\end{longtable}


%%% Local Variables:
%%% mode: latex
%%% TeX-engine: xetex
%%% TeX-master: "../../Main"
%%% End:


%%% Local Variables:
%%% mode: latex
%%% TeX-engine: xetex
%%% TeX-master: "../../Main"
%%% End:
  
%% -*- coding: utf-8 -*-
%% Time-stamp: <Chen Wang: 2019-12-19 10:22:14>


\section{后秦\tiny(384-417)}

\subsection{简介}

后秦(384年-417年,或稱姚秦)是十六国时期羌人贵族姚苌建立的政权。

前秦苻坚淝水兵败后,关中空虚,原降于前秦的羌人贵族姚苌在渭北叛秦,晋太元九年(384年)自称“万年秦王”,都北地(今陕西耀县东南)。次年(385年)擒杀苻坚。太元十一年(386年)姚苌称帝于长安(今陕西西安西北),国号秦,史称后秦。

其国号以所统治地区为战国时秦国故地为名。《十六国春秋》始称“后秦”,以别于前秦和西秦,后世袭用之。又以王室姓姚而别称姚秦。

统治地区包括今陕西、甘肃东部和河南部分地区。

后秦建初七年(393年)姚苌卒,子姚兴继位,攻杀前秦苻登,扫除前秦残部;又乘后燕灭西燕,尽占原西燕河东之地;弘始元年(399年)乘东晋内乱,陷洛阳,淮汉以北诸城多请降,国势遂与后燕相当。伐後涼,得鳩摩羅什。是年,法顯從長安出發西行求經。

弘始十八年(416年)姚兴卒,子姚泓继位。國內曾歸降的多族勢力趁機反叛,乘丧发兵。东晋劉裕派檀道濟等北伐,陷洛阳。后秦宗室皇弟為奪位反叛,被姚泓消滅。永和二年(417年)东晋圍攻长安,姚泓舉家投降,竟被劉裕滅族,后秦亡。后秦共存在32年(384-417)。

\subsection{景元帝生平}

姚弋仲(280年-352年),南安郡赤亭(今甘肅省隴西縣西)羌人。西晉末期至五胡十六國前期人物,南安羌族酋長,先後降於前趙、後趙及東晉。姚弋仲亦是後秦開國君主姚萇之父。

姚弋仲為燒當羌後代,漢光武帝建武中元年間其先祖滇虞因侵擾東漢而受東漢朝廷討伐,被逼逃亡出塞。至遷那時內附,至此獲居於南安郡赤亭縣。姚弋仲是遷那的五世孫,其父是曹魏鎮西將軍、西羌都督柯回。

姚弋仲年少聰明而勇猛,英明果斷,雄武剛毅,不治產業而以收容救濟為務,故很受眾人敬服。永嘉六年(312年),時值永嘉之亂次年,姚弋仲舉眾東遷榆眉,胡漢人民扶老攜幼跟隨者有數萬,姚弋仲並於此時自稱護西羌校尉、雍州刺史、「扶風公」。太寧元年(323年),前趙帝劉曜消滅盤據隴西的陳安後,關隴地區的氐、羌部落都向前趙請降,劉曜就以姚弋仲為平西將軍,封平襄公。

劉曜於咸和三年(328年)敗於後趙天王石勒後,留守長安的太子劉熙於次年棄守長安,出奔上邽(今甘肅天水市),導致關中大亂,後趙乘時進取關中。不久石虎更領兵攻下上邽,消滅前趙殘餘勢力,姚弋仲亦於是歸降後趙,並獲石虎推薦行安西將軍、六夷左都督。姚弋仲當時向石虎建議遷移隴上豪族,以削弱其實力並充實京畿地區,得石虎聽從。

至咸和八年(333年),後趙帝石勒去世,石虎以丞相掌握朝權,因著姚弋仲前言及氐酋蒲洪的勸言,於是遷關中豪族及氐、羌共十萬戶到首都襄國(今河北邢台)所在的關東地區,並命姚弋仲為奮武將軍、西羌大都督,封襄平縣公,讓他的部眾遷居於清河郡的灄頭(今河北棗強縣東北)。後又遷持節、十郡六夷大都督、冠軍大將軍。

永和五年(349年),高力督梁犢與其部眾兵變,聲勢浩大,並擊敗石虎派往討伐的李農。石虎當時大為恐懼,並召姚弋仲與燕王石斌討伐梁犢。姚弋仲率其部眾八千餘人輕騎至首都鄴城(今河北省臨漳縣)。當時石虎已重病,不能馬上接見,只先賞賜姚弋仲酒食。姚弋仲怒而不食,說:「召我擊賊,豈來覓食邪!我不知上存亡,若一見,雖死無恨。」石虎接見後加授姚弋仲使持節、侍中、征西大將軍,賜鎧馬。隨後姚弋仲就不辭而出,策馬南奔,大破叛軍,斬梁犢。因功加劍履上殿,入朝不趨,進封西平郡公。

同年,石虎去世,太子石世繼位,而征梁犢歸來的姚弋仲、蒲洪等人亦於此時回軍,並與彭城王石遵相遇於李城(今河南溫縣),並共同勸說石遵起兵奪位。石遵隨後起兵,不久就殺石世繼位,並讓冉閔掌有兵權。然而不久冉閔就廢殺石遵,立石鑒為帝,掌握朝政。新興王石祗於是與姚弋仲及蒲洪連兵,移檄討伐冉閔。次年,冉閔殺石鑒並誅殺石氏宗室,姚弋仲就率眾討伐冉閔,移兵至混橋。不久石祗於襄國即位為後趙帝,以姚弋仲為右丞相,封親趙王,並殊有禮待。永和七年(351年),冉閔圍攻襄國,姚弋仲命其子姚襄率兵救援石祗,並配合後趙太尉張舉的行動,遣使向前燕求援。最終在汝陰王石琨、姚襄、前燕三軍以及襄國守軍夾擊之下,圍城的冉閔兵敗,敗退鄴城。雖然姚襄取勝,但因為沒有應姚弋仲在出發前所要求的擒得冉閔,遭姚弋仲以杖打一百責罰。而同年石祗亦被殺,後趙滅亡,姚弋仲於是遣使向東晉投降,獲授使持節、六夷大都督、都督江淮諸軍事、車騎大將軍、儀同三司、大單于,封高陵郡公。

次年(352年),姚弋仲在患病時向諸子說:「石氏厚待我,我本來想盡力幫助他們。而今天石氏已經滅了,中原無主;我死了以後,你們要盡快歸降晉室,並固守臣節,不要做不義的事呀!」及後去世,享年七十三歲。其五子姚襄續統其眾。

姚襄後為苻生所敗,弋仲的靈柩為其所獲,苻生以王禮葬弋仲於天水冀縣。後來,姚弋仲第二十四子姚萇稱後秦帝時,追諡姚弋仲為景元皇帝,廟號始祖,其墓稱為「高陵」,置园邑五百家。现为天水市域重点文物古迹。

《晉書》載姚弋仲個性「清儉鯁直,不修威儀,屢獻讜言,無所回避」,連殘暴的石虎也敬重三分,334年,石虎廢皇帝石弘自立,弋仲稱病不來朝賀,經石虎不斷召見才至,弋仲正色向石虎說:「奈何把臂受託而反奪之乎!」石虎也因為弋仲正直而不責怪他。後石虎一名寵姬的弟弟任武城左尉,擾亂姚弋仲的部眾,姚弋仲就捕捉並數責他,更命人殺了他,雖然最終因對方叩頭至流血作請求以及左右的諫言而不殺他,但也見姚弋仲為事剛直,毫不顧忌對方背景。後討梁犢前得石虎召見,又責備患病的石虎:「兒死,愁邪,何為而病?兒幼時不擇善人教之,使至於為逆;既為逆而誅之,又何愁焉!且汝久病,所立兒幼,汝若不愈,天下必亂,當先憂此,勿憂賊也!犢等窮困思歸,相聚為盜,所過殘暴,何所能至!老羌為汝一舉了之」除了看見他梗直而言,直指他教子無道而導致石宣殺害太子石韜的事件發生,亦見其不論尊卑皆直稱「汝」的行為,連作為皇帝的石虎也不例外。

姚弋仲曾有一個叫馬何羅的部曲曾在張豺主政時叛歸對方。後因石世被廢,張豺亦遭誅殺,馬何羅於是回到姚弋仲那裏。當時眾人都建議姚弋仲殺了他,但姚弋仲就以「招才納奇」為由寬恕他,不但不作加害,反以其為參軍。

姚弋仲在後趙末年一直顯得忠於石氏,不過《資治通鑑》亦有見載於後趙混亂,冉閔篡權時姚弋仲與蒲洪爭奪關中的行動。

\subsection{魏武王生平}

姚襄(约331-357年),字景國,南安赤亭(今甘肅省隴西縣西)羌族酋長,五胡十六國時期諸侯、軍閥,是姚弋仲的第五子,也是後秦開國君主姚萇之兄。

姚襄父親姚弋仲是南安羌酋長,在後趙滅前趙後歸降後趙並接受其官爵。而姚襄雄健威武,多才多藝,觀察入微且善於安撫人心,故獲得部眾愛戴和敬重,眾人並因此請求姚弋仲立姚襄為繼承人。姚弋仲起初以姚襄不是長子,並不允許,然請求的百姓很多,姚弋仲才開始給姚襄帶兵。

永和六年(350年),冉閔殺後趙皇帝石鑒,建立冉魏。隨後後趙新興王石祗就於襄國即位為後趙帝,並以姚襄為使持節、驃騎將軍、護烏丸校尉、豫州刺史、新昌公。永和七年(351年),姚襄奉父命領二萬八千騎兵營救正遭冉閔圍攻的石祗,姚弋仲並於出發前作訓誡:「冉閔背棄仁義,屠滅石氏。我受了人家的優厚待遇,就應為其復仇,我卻因年老患病而不能親身去做;你才能比冉閔高出十倍,若果不能擒殺他回來,就不要再來見我了!」姚襄雖然聯同前燕、石琨及襄國守軍大敗冉閔,暫時解了襄國的危機,卻因無法擒得冉閔,遭姚弋仲以一百杖作處罰。

同年,後趙為冉魏所滅,姚襄隨其父向東晉投降,獲晉廷任命為為持節、平北將軍、都督并州諸軍事、并州刺史、平鄉縣公。次年(352年)姚弋仲去世,死前命諸子在其死後歸降晉室,作晉的忠臣。姚襄接手統率父親部眾,不發布父親去世的消息,並攻下陽平(今山東莘城)、元城(今河北大名縣)及發干(今山東聊城市東昌府區),駐於碻磝津(今山东省茌平县西南古益河上);但不久卻敗給前秦,南走至滎陽(今河南滎陽市)才發喪,後在滎陽與洛陽之間的麻田與前秦軍作戰時,座騎中箭死亡,因姚萇贈馬及援軍趕到才免於被擒。此時姚襄才以五個弟弟為人質,歸降東晉。東晉以姚襄駐屯譙城(今安徽亳州),而姚襄隨後單人匹馬渡過淮河,於壽春(今安徽壽縣)面見豫州刺史謝尚,當時謝尚對其名氣亦有所聽聞,於是撤去衞士,以代表高雅的幅巾接見他。二人一見如故,又因姚襄博學及善於談論,很得江東人士敬重。

同年,姚襄與謝尚一同進攻據守許昌(今河南省許昌市)的後趙豫州牧張遇,但遭前秦丞相苻雄等擊敗,謝尚因大敗而退守淮南,得姚襄棄輜重而護送至芍陂,故此將善後工作都委託給姚襄。謝尚因戰敗而受貶降,及後更被調回京師建康(今江蘇南京市),而當時駐屯歷陽(今安徽和縣)的姚襄亦以當時佔據北方的前秦及前燕皆強盛,無意北伐,反在淮河兩岸大興屯田,訓練將士,不過當時主政的殷浩就要北伐,更忌憚姚襄兵力強盛,不但囚禁姚襄送去當質子的弟弟,更屢次派刺客行刺姚襄,但刺客不能下手,反告訴姚襄實情。後殷浩再暗中派魏憬率兵襲殺他,但魏憬反被姚襄所殺,於是令殷浩更加厭惡姚襄,遷姚襄到梁國的蠡臺,表授他為梁國內史。

永和九年(353年),殷浩北伐,以姚襄為前軀,而當時姚襄已決心叛離東晉,於是算好殷浩快來到時就假意命部眾乘夜逃遁,其實暗中設伏伏擊殷浩。殷浩聽聞姚襄部眾逃遁,追至山桑(今安徽蒙城縣北)就被姚襄伏兵擊敗,被逼退守譙城,而姚襄就俘殺萬多人,盡收其輜重,南據淮南郡一帶。不久姚襄北屯盱眙(今江蘇盱眙縣),在當地收納流民,令部眾增至七萬人,並分置地方官員,鼓勵農事生產,又遣使到建康狀告殷浩,並作道歉。永和十年(354年),姚襄向前燕歸降。

永和十一年(355年),姚襄因應部眾要求北歸,於是自稱大將軍、大單于,北攻晉冠軍將軍高季,卻為高季所敗。姚襄撫恤散敗的兵眾,重新集結力量,及後乘高季去世而據有許昌。次年,姚襄進攻當時據有洛陽的周成,但用了一個多月都不能攻下,當時長史王亮就勸他放棄進攻,免得被他人有機可乘,危及自己。不過姚襄沒有聽從。然而,桓溫不久就發動北伐,征伐姚襄,姚襄被逼放棄圍城而抵抗桓溫,並在伊水以北的樹林中設下精兵,並聲稱自願歸降,請桓溫稍為退兵。然而桓溫沒有答應,並親身督戰,組以兵陣進攻沿河岸抵抗的姚襄軍,姚襄兵敗而北逃至北芒山。姚襄隨後西逃,桓溫因追不到而放棄。姚襄逃到平陽(今山西臨汾市)時得時為前秦并州刺史的舊部尹赤叛秦歸附,於是據守襄陵(今山西襄汾縣);同據并州的張平因而攻打姚襄,姚襄雖不敵,但與張平結為兄弟,換取兩者和平。

升平元年(357年),姚襄謀取關中,先移鎮北屈(今山西吉縣),後進屯杏城(今陝西黃陵縣西南),命姚蘭進攻敷城(今陝西富縣)、姚益及王欽盧聯結關中一帶的羌胡外族,共收得胡漢共五萬多戶。不過姚襄就與時據關中的前秦軍發生衝突,前秦帝苻生遣苻黃眉、苻堅、鄧羌進攻,姚襄堅守不戰。但鄧羌就在其軍壘門外列陣,激得姚襄不聽僧人智通的勸言,親自率眾出戰,最終被鄧羌詐敗誘至三原(今陝西三原縣)。姚襄遭受到鄧羌及苻黃眉的合擊,所乘駿馬「黧眉騧」倒地,姚襄為前秦軍所擒斬,享年二十七歲。其弟姚萇率餘眾投降。苻生後以公爵之禮葬姚襄。

後來,姚萇稱後秦帝時,追諡姚襄為魏武王。

姚襄深得人心,如在他大敗給桓溫而北逃時,就有五千多人於當晚拋下妻兒去追隨他。前後幾次敗仗,百姓只要知道他在哪裡就奔赴投靠。當時謠傳姚襄重傷而死,被桓溫俘虜的百姓沒有不痛哭流涕的。姚襄部屬楊亮後來歸降桓溫,桓溫向楊亮詢問姚襄的為人,楊亮的評價是:「神明器宇,孫策之儔,而雄武過之。」

楊亮:「神明器宇,孫策之儔,而雄武過之。」

姚萇:「吾不如亡兄有四:身長八尺五寸,臂垂過膝,人望而畏之,一也;當十萬之眾,與天下爭衡,望麾而進,前無橫陣,二也;溫古知今,講論道藝,駕馭英雄,收羅儁異,三也;統率大眾,履險若夷,上下咸允,人盡死力,四也。所以得建立功業,策任群賢者,正望算略中一片耳。」

呂思勉:「其(姚襄)才略或在苻健之上。然寄居晉地,四面追敵,不如健之入關,有施展之地矣。」

2009年12月27日,河南有关方面宣布在安阳发现曹操墓。这一发现,引发许多质疑。 西安市委党校历史教授胡觉照接受记者采访时称,安阳“曹操墓”实则五胡十六国时期军阀姚襄墓穴。


%% -*- coding: utf-8 -*-
%% Time-stamp: <Chen Wang: 2021-11-01 11:58:29>

\subsection{武昭帝姚苌\tiny(384-394)}

\subsubsection{生平}

秦武昭帝姚\xpinyin*{苌}(329年-393年),字景茂。南安赤亭(今甘肅省隴西縣西)羌族人。十六国时期后秦政权的开国君主。後趙末年南安羌酋長姚弋仲第二十四子,姚襄之弟。姚萇在姚襄死後率其部眾入秦,成為前秦的將領。淝水之戰後姚萇在關中羌人的推舉下自稱萬年秦王,建立後秦,並與苻堅領導下的前秦作戰。姚萇後來殺害了苻堅,並乘西燕東退而進駐長安,不久稱帝。前秦宗室苻登在關中氐族殘餘力量支持下繼續與姚萇作戰,姚萇一度處於不利形勢,但終大敗苻登,漸處優勢,但在消滅前秦勢力前去世,直至兒子姚興即位後才完全消滅前秦勢力。

姚萇年少時已聰慧明智,多有權略,豁達率性,並沒有專注於德行和學業之上,而其眾位兄長都認為他很特別。後來姚萇跟隨姚襄四處出兵,經常參與重要的決策。永和八年(352年),姚襄在麻田敗於前秦軍,其坐騎更中箭死亡,姚萇冒險將自己的坐騎送給姚襄助其出逃。最後姚萇因援軍趕至才得倖免。

升平元年(357年),姚襄謀取關中失敗,在三原(今陝西三原縣)與前秦將領苻黃眉、鄧羌等的交戰中戰死。姚萇當時就率姚襄餘眾盡降前秦。同年前秦宗室苻堅發動政變推翻皇帝苻生,自任天王,並以姚萇為揚武將軍。

太和二年(367年),姚萇隨同王猛參與討伐以略陽郡叛變的羌人斂岐,並因姚弋仲昔日統領斂岐的部落,大量部眾知道姚萇到來都向前秦歸降,令得前秦順利取下略陽。太和六年(371年)三月,与苻雅、杨安、王统、徐成及朱彤等讨伐據有仇池的氐王杨纂,双方決戰於峡谷,杨纂大败,损失三成兵力,終被逼投降。

宁康元年(373年)十一月,前秦攻下東晉領下的益、梁二州,姚萇出任宁州刺史,屯兵於垫江(今重慶市墊江縣)。後遷任步兵校尉,封益都侯。太元元年(376年)五月,与武卫将军苟苌、左将军毛盛、中书令梁熙等進逼黃河,並於八月對前涼發動攻擊,攻滅前涼。

太元八年(383年),東晉荊州刺史桓沖北伐,其中涪城(今四川綿陽市)受到晉將楊亮攻擊,姚萇遂與張蚝出兵救援,逼楊亮退兵。同年苻堅大舉攻晉,意圖滅掉東晉,統一全國,史稱淝水之戰。當時苻堅就以姚萇為龍驤將軍,督益、梁二州諸軍事,讓其從蜀地率軍進攻東晉西方,更說:「朕昔日就是以龍驤將軍建立大業,這個將軍號從來都沒有改授他人,今天特別對你授予此號,山南之事都交給你了。」

苻堅於淝水之戰中大敗,姚萇返回長安。而前秦在戰敗後國力大衰,其中北地長史慕容泓於戰後第二年在關東起兵叛亂,回屯華陰(今陝西華陰市),響應於河北地區叛變的叔父慕容垂。苻堅於是命雍州牧苻叡出兵討伐,而姚萇則任其司馬。當時慕容泓因畏懼而率眾東逃關東,苻叡因輕敵而決心追去邀擊,不聽姚萇的諫言,最終遭慕容泓擊敗,苻叡亦戰死。姚萇在敗後派長史趙都及參軍姜協向苻堅謝罪,但二人卻被憤怒的苻堅殺死,驚懼的姚萇於是逃到渭北的牧馬場。在當地,尹緯、尹詳及龐演等人聯結羌族豪強共五萬多戶向姚萇歸降,並推姚萇為盟主。姚萇於是在太元九年(384年)自稱大將軍、大單于、萬年秦王,改元「白雀」,建立後秦政權。

姚萇接著進屯北地,華陰、北地、新平及安定各郡共有十多萬名羌胡外族歸附。不久苻堅親自率軍討伐姚萇,姚萇屢敗更遭前秦軍斷絕水源。然而就在後秦軍中有人渴死及在恐懼當中時就遇上天雨,營中水深三尺,解決了水荒,亦令後秦軍心復振。不久姚萇出兵反擊,擊敗前秦將楊璧並俘獲楊璧、徐成及毛盛等數十人,皆禮待而送還。而隨著西燕軍隊逼近長安,苻堅率兵回防長安。雖然姚萇在早前向西燕送質請和,但當時姚萇群臣卻建議姚萇加入戰鬥以奪取長安,建立根本之地。不過姚萇自度慕容氏獲勝並後不會長留關中,必會東歸河北,故此打算北屯九嵕(今陝西乾縣東北)以北一帶地區(嶺北)以積聚實力和資源,待前秦亡國而西燕東歸後自取長安。姚萇隨後親自率軍進攻新平郡城(今陝西省邠縣),卻遭守將苟輔頑強抵抗,有萬多人陣亡。苟輔又詐降誘騙姚萇入城,雖然姚萇入城前就察覺而沒進城,但仍受到苟輔伏兵攻擊,萬多人戰死之餘亦險些被擒。

因為新平久久不下,姚萇於是在白雀二年(385年)正月留兵繼續攻城,自己另外出兵安定郡,擒下前秦安西将军苻珍,亦令嶺北諸城降,唯新平未下。至四月,新平物資匱乏,亦無外援,苟輔接受後秦軍的勸降,率城內五千人出降。姚苌下令將所有人坑殺,奪取了新平。五月,苻堅離開長安,出屯五將山,至七月時後秦將吴忠捕獲苻坚,送至新平。同年八月,姚萇因向苻堅索取傳國玉璽不遂,更遭其出言侮辱,於是縊殺苻堅於新平佛寺(今彬縣南靜光寺)。姚萇為了掩飾他殺死苻堅的行為,諡苻堅為「壯烈天王」。

十月,已據有長安的西燕王慕容沖派高蓋攻伐姚萇,遭後秦軍擊敗並投降。白雀三年(386年),西燕國內政變頻生,並開始棄守長安東歸。時盧水胡郝奴乘虛入據長安並稱帝,更命其弟郝多進攻於馬嵬(今陝西興平市馬嵬鎮)自守的王驎。姚萇此時從安定東攻,逼走王驎並擒得郝多,並進攻長安,令郝奴懼而請降。取長安後姚萇就於同月即位為帝,改年號「建初」,建國號大秦。不久又擊敗了前秦秦州刺史王統,奪取秦州。

但同一年,前秦宗室苻登就在關中氐族殘餘勢力的推舉下與後秦對抗,不久在前秦帝苻丕遇害後更稱帝繼位。起初苻登力量甚盛,在涇陽(今陝西涇陽縣)大敗姚碩德,要姚萇親自出兵救援;更謀攻長安。不過當時前秦重將苻纂為苻師奴所殺,將領蘭櫝遂與苻師奴反目。蘭櫝因受西燕皇帝慕容永攻擊而向後秦求援,姚萇以苻登遲疑慎重而少決斷,不敢出兵深入而冒著遭乘虛後襲的危險,決意親自率軍救援。最終先破苻師奴並盡收其眾,後敗慕容永並生擒蘭櫝。

另姚方成亦擊敗徐嵩,徐嵩雖然被俘仍大罵姚萇不僅背叛對其有恩的苻堅,更將他殺害,不惜恩情就連狗和馬都不如。姚方成殺死徐嵩後,姚萇又掘出苻堅的屍首不斷鞭撻,更脫光屍身的衣服,裹以荊棘並以土坑埋掉,以釋心中憤怨。建初三年(388年),自春季開始夏末,姚、苻兩軍就分別據朝那(今寧夏朝那縣)及武都(今甘肅武都縣)相持並交戰,互有勝負而不能擊倒對方,於是都解兵歸還。但關西豪傑都以後秦久久未能站穩關中,反多次敗給苻登,大多都投向前秦,唯齊難、徐洛生、劉郭單等人仍然忠於後秦,提供軍糧並跟隨姚萇征戰。

建初四年(389年),姚萇屢次敗於苻登,命姚崇襲擊苻登於大界的輜重又不得,而苻登就已威脅安定。面對如此局面,姚萇堅拒與苻登正面決戰,力圖以計取勝,於是乘夜率兵三萬再攻大界,終攻克大界並殺毛皇后等人及生擒數十名前秦名將。姚萇隨後亦不貪勝,堅拒乘勝進擊苻登,苻登於是收餘眾退守胡空堡,但已元氣大傷。

在大敗苻登輜重後的四個月後,姚萇設計讓其將任盆詐降以誘殺苻登,雖然最終因雷惡地識破而事敗,但苻登卻忌憚雷惡地,逼其降於姚萇。次年(390年)魏揭飛攻後秦,雷惡地叛迎魏揭飛,雖然當時苻登正在長安附近的新豐(今陝西西安市臨潼區),但姚萇以雷惡地「智略非常」,於是親自出兵攻伐魏揭飛。魏揭飛見姚萇兵少就讓全軍進擊,姚萇特意示弱不戰,卻派了姚崇從敵軍後方攻擊令其混亂,接著就出兵直擊,大敗對方並陣斬魏揭飛,又再降雷惡地並不減昔日待遇。雷惡地兩度歸於姚萇,終對其心服。另外姚萇亦不怕前秦兗州刺史強金槌詐降,只帶著數百騎兵隨其訪問強金槌的軍營,以坦誠獲得了身為氐族人的強金槌的信任,令其不應其他氐族勢力的計謀而加害姚萇。

至建初六年(391年)十二月,苻登進攻安定,姚萇在安定城東擊敗他。次年三月,前秦將沒弈干亦向後秦歸降,但姚萇不久就患病。苻登得知姚萇患病就乘機進攻安定,至八月姚萇病情轉好就親自率兵抵抗,更乘苻登出營迎擊而命姚熙隆進襲前秦軍營,令苻登懼而退兵。姚萇又讓軍隊旁出跟隨苻登,苻登得知後秦營壘空空如也,失去其影蹤後更為驚懼,只得敗還雍城(今陝西鳳翔縣南)。

建初八年(393年)十月,姚萇病重而回長安。至同年十二月,姚萇召太尉姚旻、僕射尹緯及姚晃、將軍姚大目和尚書狄伯支受遺詔輔政,輔助太子姚興。及後姚萇去世,享年六十四歲。姚興先秘不發喪,至次年才發布死訊,上諡號為武昭皇帝,廟號太祖。

姚萇簡單率直,即使當了君主,屬下有過錯可能還會直加責罵。權翼曾勸他不要這樣對待屬下,但姚萇自以這是自己本性,更稱自己聽正直之言,能知己過。

姚萇甚得苻堅重用,尤以其為龍驤將軍,並以自己從龍驤將軍登位至前秦君主一事作勉勵。但姚萇終殺害苻堅,此行為成了前秦將領反對及討伐他的理由,而姚萇亦曾挖屍洩忿。不過在屢敗於苻登後,卻認為是苻堅亡魂的助力,於是也在軍中樹立苻堅神像祈求道:「新平之禍,不是臣姚萇的錯啊,臣的兄長姚襄從陝州北渡,順著道路要往西邊去,像狐狸死時把頭朝向原本洞穴一樣,只是想要見一見鄉里啊。陛下與苻眉攔阻於路上攻擊他,害他不能成功就死了,姚襄遺命臣一定要報仇。苻登是陛下的遠親亦想復仇,臣為自己的兄長報仇,又怎麼說是辜負了義理呢?當年陛下封我為龍驤將軍,跟我說:『朕從龍驤將軍當上了皇帝,卿也好好努力罷!』這明明白白的詔諭非常顯然,好像還在耳邊一樣。陛下已經過世成為神明了,怎麼會透過苻登而謀害臣,忘卻當年說的話呢!現在為陛下立神像,請陛下的靈魂進入這裏,聽臣至誠的禱告。」 不過戰況仍未有改善,反時有夜驚,並招來苻登批評,終毀了苻堅像。據說姚萇死前曾夢見過苻堅率天官、鬼兵去襲擊他(《晉書》「將天官使者、鬼兵數百突入營中」),期間他被救援自己的士兵誤傷陰部至大量出血。醒後就發現陰部腫脹,醫者刺腫處則如夢中一樣大量出血(《晉書》「誤中萇陰,出血石餘」),如此嚇得姚萇發狂胡言,又求苻堅原諒,姚萇不久傷重身亡,臨終前跪伏床頭,叩首不已。

即使姚萇在位期間皆與前秦等勢力戰鬥,但仍設立太學,禮遇先賢後代;又曾命各鎮都要設置學官,由他們評核人才優劣再隨其才能擢用,皆可見其重視文教和吸納文人的行為。而他在安定亦修治德政,大行教化,省卻不必要的支出,亦表彰平民戶中有善行的人。

姚萇長期征戰,雖為君主亦不貪圖逸樂,於與前秦相持不下,部分豪族轉為支持前秦時更寫書自責,並賣掉後宮珍寶去支持軍事,而自己與妻子都力行簡約,對為國戰死的將士皆有所褒揚和追贈。


\subsubsection{白雀}

\begin{longtable}{|>{\centering\scriptsize}m{2em}|>{\centering\scriptsize}m{1.3em}|>{\centering}m{8.8em}|}
  % \caption{秦王政}\
  \toprule
  \SimHei \normalsize 年数 & \SimHei \scriptsize 公元 & \SimHei 大事件 \tabularnewline
  % \midrule
  \endfirsthead
  \toprule
  \SimHei \normalsize 年数 & \SimHei \scriptsize 公元 & \SimHei 大事件 \tabularnewline
  \midrule
  \endhead
  \midrule
  元年 & 384 & \tabularnewline\hline
  二年 & 385 & \tabularnewline\hline
  三年 & 386 & \tabularnewline
  \bottomrule
\end{longtable}

\subsubsection{建初}

\begin{longtable}{|>{\centering\scriptsize}m{2em}|>{\centering\scriptsize}m{1.3em}|>{\centering}m{8.8em}|}
  % \caption{秦王政}\
  \toprule
  \SimHei \normalsize 年数 & \SimHei \scriptsize 公元 & \SimHei 大事件 \tabularnewline
  % \midrule
  \endfirsthead
  \toprule
  \SimHei \normalsize 年数 & \SimHei \scriptsize 公元 & \SimHei 大事件 \tabularnewline
  \midrule
  \endhead
  \midrule
  元年 & 386 & \tabularnewline\hline
  二年 & 387 & \tabularnewline\hline
  三年 & 388 & \tabularnewline\hline
  四年 & 389 & \tabularnewline\hline
  五年 & 390 & \tabularnewline\hline
  六年 & 391 & \tabularnewline\hline
  七年 & 392 & \tabularnewline\hline
  八年 & 393 & \tabularnewline\hline
  九年 & 394 & \tabularnewline
  \bottomrule
\end{longtable}

%%% Local Variables:
%%% mode: latex
%%% TeX-engine: xetex
%%% TeX-master: "../../Main"
%%% End:

%% -*- coding: utf-8 -*-
%% Time-stamp: <Chen Wang: 2021-11-01 11:58:34>

\subsection{文桓帝姚兴\tiny(394-416)}

\subsubsection{生平}

秦文桓帝姚兴(366年-416年),字子略,南安赤亭(今甘肅省隴西縣西)羌族人。十六国时期后秦皇帝,后秦武昭帝姚苌长子。姚興即位之初就俘殺了父親在位時面對的最強對手苻登,基本覆滅了前秦。後又出兵後涼,令其投降之餘亦令盤據秦涼一帶的政權如北涼、南涼及西秦等政權臣服,還率兵進攻東晉,一舉攻取洛陽等地,使統治疆域迅速擴大。不過隨後與北魏在柴壁之戰中卻大敗,面對新興的夏國亦不能有效對付,反屢遭侵擾;且國內出現兒子姚弼與太子姚泓爭位的事件,令後秦國勢漸弱。弘始十八年(416年),姚興病逝,諡文桓皇帝,庙号高祖,下葬偶陵。

姚興在前秦時任太子舍人。白雀元年(384年),姚萇在渭北馬牧稱萬年秦王,建後秦,姚興時在長安,冒險出走與父親會合。建初元年(386年),姚萇在奪得長安(今陝西西安)後稱帝,就立了姚興為皇太子。其時姚萇屢次在外與前秦對抗,姚興就經常留鎮長安以統後事。其時又與太子中舍人梁喜及太子洗馬范勖講論經籍,不以兵戎廢業,當時的人亦受他們影響。

建初八年(393年)十二月,姚萇去世,死前命太尉姚旻、僕射尹緯、姚晃、將軍姚大目及尚書狄伯支為輔政大臣,並向姚興說:「若有人謗毀這幾位大臣,小心不要聽信。你以仁管教子女,以禮對待大臣,以信處事,以恩治民,這四項你能做到,我就不憂心了。」姚萇死後,姚興秘不發喪,分命姚緒、姚碩德及姚崇駐安定、陰密及長安,自己就自稱大將軍,領兵進攻前秦。

次年春,前秦皇帝苻登聽聞姚萇已死即十分高興,又輕視姚興,隨即率眾東進。至夏季,苻登要進攻廢橋,尹緯則受命支援守馬嵬堡的姚詳,尹緯於是據守廢橋等待前秦軍。前秦軍因無法取得水源而缺水,兩三成士兵更因而渴死,於是急攻尹緯希望能奪取水源。姚興當時認為苻登已是窮寇,於是派狄伯支命令尹緯要持重拒戰,不要輕易與前秦軍決戰。不過尹緯認為姚萇新死,人心恐懼不安,應當用盡力量消滅敵人,安定眾心。尹緯於是與苻登決戰,終大敗前秦軍,苻登因兵眾潰散而逃走,逃到馬毛山。戰後,姚興才正式發喪,並在槐里(今陝西興平東南)即位為帝,改元「皇初」。七月,姚興進攻苻登並在馬毛山南作戰,擒殺苻登,並解散其部眾。不久繼位的前秦皇帝苻崇因被乞伏乾歸逼逐而聯結楊定進攻乞伏乾歸,卻遭對方所殺,前秦正式滅亡。

皇初七年(397年),姚興率兵進攻東晉控制的湖城,弘農太守陶仲山及華山太守董邁都投降。姚興於是進至陝城(今河南陝縣),並攻下上洛(今陝西商洛市)。另又分遣姚崇進攻洛陽(今河南洛陽),因晉河南太守夏侯宗之守金鏞城而未能攻克,於是改攻柏谷,強遷兩萬多戶流民西歸。及至皇初九年(399年),姚興命姚崇及楊佛嵩再攻洛陽,守將辛恭靖堅守一百多日後失守,後秦奪得洛陽。取洛陽後,淮河、漢水以北各城大多都向後秦請降,並送人質。

弘始二年(400年),姚碩德進攻西秦,西秦王乞伏乾歸率眾抵抗,兩軍對峙期間姚碩德軍中柴草缺乏,姚興就暗中領兵支援。乞伏乾歸知道姚興派軍前來,於是命慕兀率二萬中軍屯柏楊(今甘肅清水縣西南),羅敦率外軍屯侯辰谷,自己領數千輕騎等候秦軍。不過其夜遇上大風和大霧,乞伏乾歸與慕兀的中軍失去聯絡,被逼與外軍會合。天亮後,乞伏乾歸就與後秦軍作戰,終大敗並逃返苑川(今甘肅榆中縣北),後秦軍受降共三萬六千多人,姚興則進軍枹罕(今甘肅臨夏市)。乞伏乾歸初降禿髮利鹿孤,但因怕不為對方所容,最終決定歸降後秦。

弘始三年(401年),姚興命姚碩德進攻後涼,並兵圍後涼首都姑臧(今甘肅武威)。後涼王呂隆被逼請降。而在後秦攻涼時,西涼李暠、南涼禿髮利鹿孤及北涼沮渠蒙遜都遣使向後秦請降。直至弘始五年(403年),後涼被南涼和北涼所逼,最終請後秦派軍迎來歸附,姚興因而派了齊難等人到姑臧,駐兵當地並送呂氏宗族內徙長安,吞併後涼。另外在攻打後涼姑臧時,連帶的將名僧鳩摩羅什請回長安。爾後為鳩摩羅什講解《法華經》,建造「長安大寺」。鳩摩羅什於長安圓寂,其生前將大乘佛教的主要經典(如《中論》、《法華經》、《維摩詰經》等)譯為漢文。

北魏君主拓跋珪曾經送一千匹馬到後秦請婚,姚興原先答應,但知拓跋珪已立了后,於是拒絕並留下使者賀狄干。弘始四年(402年),北魏將領拓跋遵進攻高平(今甘肅固原),沒弈干拋棄部眾,帶著數千騎兵及赫連勃勃逃到秦州。北魏軍追擊至瓦亭仍未追上,於是盡遷高平的物資回國;及後北魏平陽太守貮塵又進攻河東。北魏的一系列軍事行動震動長安,關中各城日間也緊閉城門,姚興於是在城西閱兵,並做好戰爭準備。同年,姚興派姚平及狄伯支等率四萬步騎兵進攻北魏,姚興則親率大軍在後。北魏帝拓跋珪則命拓跋順及長孫肥統六萬騎兵為先鋒,自己也率大軍在後以作抵抗。姚平用了六十多天攻陷了北魏屯積糧食的乾壁,又派二百精騎偵察魏軍,卻為長孫肥襲擊,所有人都被生擒。姚平因而後撤,又遭拓跋珪追擊,並在柴壁(今山西襄汾縣西南)被追上;姚平當時據柴壁城固守,北魏軍則圍困城池。姚興於是自領四萬七千兵營救姚平,並打算佔領天渡以運糧支援姚平。不過北魏加強了包圍圈,又在汾水建浮橋,在汾水西岸築圍堵截姚興援軍,務求引姚興取道汾東,經長達三百多里而缺乏小路通行的蒙坑進攻。而姚興到蒲阪後因怕魏軍強盛,很久才正式進攻。及後姚興在蒙坑以南與拓跋珪所率三萬步騎兵作戰,後秦軍共千多人被殺,姚興被逼退走四十多里,而姚平亦未能突圍。接著拓跋珪分兵各據險要,不讓後秦軍接近柴壁。姚興駐屯汾西,在汾水上游放木材打算沖毀北魏浮橋,但木材都被魏軍截取。至十月,姚平軍需用盡,在夜間試圖向西南方突圍,姚興列兵汾西,點起烽火和擂鼓響應,不過姚興欲救姚平盡力突陣,姚平反望姚興攻圍接應,兩軍雖然能夠以叫喊相通,但始終都沒能壓逼圍城魏軍。姚平最終無法成功突圍,於是率眾投水自殺,然而拓跋珪卻都派人潛下水捕捉,赴水諸將與城中狄伯支、唐小方等人及兩萬多兵眾都被俘。姚興只能見城中軍隊束手就擒而無力支援,全軍都哀傷痛哭,哭聲震動山谷。接著姚興數度派遣使者求和,但都被拒,魏軍更乘勝進攻蒲阪。防禦蒲阪的姚緒固守不戰,又正因柔然要進攻北魏,拓跋珪才撤兵。

弘始九年(407年),北魏歸還柴壁之戰中被俘的唐小方等人,姚興又以良馬千匹贖回狄伯支,與北魏通和。赫連勃勃因後秦與北魏連和而大怒,竟搶奪了柔然送給後秦的八千匹馬,並襲殺沒弈干叛變,稱大夏天王,建夏國。赫連勃勃隨後又攻破鮮卑薛干等三部,並進攻後秦三城以北諸戍,後秦將楊丕、姚石生等都被殺,接著又侵掠嶺北,令嶺北各城城門白天也要緊閉。姚興此時感嘆:「我不聽黃兒(姚興弟姚邕小字)的話,才弄成今天這樣子。」

隨後禿髮傉檀大敗於赫連勃勃,名將折損達六七成,接著成七兒及梁裒、邊憲等又先後謀反,姚興見其並受外憂外患夾擊,不顧尚書郎韋宗的勸阻和吏部尚書尹昭命北涼及西涼進攻禿髮傉檀的建議,堅持分兵兩道進攻夏和禿髮傉檀。姚興於弘始十年(408年)派了齊難領二萬騎兵攻夏,又派姚弼、斂成及乞伏乾歸攻禿髮傉檀,更寫信給禿髮傉檀聲稱姚弼等其實只是配合齊難進攻夏國的行動,禿髮傉檀不作防備。不過姚弼等到姑臧後反被禿髮傉檀的奇兵擊敗,後又特地釋放牛羊引誘後秦軍掠奪,大敗秦軍。作為後繼的姚顯知姚弼兵敗,加快趕到姑臧,並命孟欽等五名擅長射擊的人於涼風門挑戰,卻遭南涼材官將軍宋益擊殺。姚顯見此委罪於斂成,派使者向禿髮傉檀謝罪,撫慰河西後就撤還。而禿髮傉檀亦派使者徐宿向後秦謝罪。不過在當年又再稱涼王。

而赫連勃勃知齊難來攻,於是退守河曲。齊難見赫連勃勃仍在很遠,於是先行縱兵野略;赫連勃勃因而潛軍來襲,俘殺七千多人,齊難逃走但在木城遭赫連勃勃生擒,其餘將士亦被俘。戰後嶺北共計有數萬人歸附赫連勃勃。弘始十一年(409年),姚興再派姚沖及狄伯支率四萬騎再攻夏,但姚沖竟圖謀反,並殺了不肯支持的狄伯支,姚興終賜死姚沖。同年,姚興親自率軍攻夏,至貮城後就派姚詳、斂曼嵬及彭白狼分督租運。其時諸軍未集合,而赫連勃勃乘虛來襲,姚興恐懼之下打算逃到姚詳那裏,但被右僕射韋華勸止。姚興派姚文宗等迎戰,雖將領姚榆生被擒,但在姚文宗力戰之下也成功擊退赫連勃勃。姚興唯有留五千禁軍助姚詳守貮城,自己撤還長安。

赫連勃勃攻破了敕奇堡、黃石固及我羅城。次年又派胡金纂攻平涼,雖然姚興親自率軍擊殺胡金纂,但赫連勃勃侄赫連羅提又攻下定陽,殺四千多人並俘姚廣都。當時秦將曹熾、曹雲及王肆佛等被逼領數千戶內徙,姚興就讓他們住在湟山及陳倉。接著赫連勃勃又進攻隴,攻略陽太守姚壽都守的清水城,姚壽都棄城奔上邽,而赫連勃勃就遷了城中一萬六千戶人到大城。姚興試圖從安定追擊赫連勃勃,但追不上。及後赫連勃勃仍屢屢侵擾後秦,但姚興都無法消滅夏國。

姚興子廣平公姚弼得父親寵愛,任雍州刺史,出鎮安定時天水人姜紀接近姚弼,並勸他巴結姚興左右以望還朝,姚弼於是巴結常山公姚顯。至弘始十三年(411年),姚興就召了姚弼回長安,讓他為尚書令、侍中、大將軍。姚弼於是擔當將相要職,更心引見人才,收結朝士,形成了一股比太子姚泓更大的勢力,更有圖取其太子之位。後來姚弼因為厭惡姚泓親信姚文宗,就誣陷他有所怨言,並讓侍御史廉桃生作證。姚興信以為真,一怒之下就賜死姚文宗。朝中大臣於是都不敢再說姚弼不是了。

因著對姚弼的寵愛,姚興對姚弼幾乎言聽計從,於是機要職位都由姚弼親信出任。當時右僕射梁喜、侍中任謙及京兆尹尹昭就找機會向姚興表示姚弼有奪嫡的志向,指出姚興不當的寵愛他,令傾險無賴的人都在其身邊,又說民間都說姚興有廢立之意,三人同時表示反對易儲。姚興立即否認有易儲計劃,三人就是更勸姚興削減姚弼權力並除去其身邊黨羽,既保姚弼,亦保國家。姚興聽後就沉默不言。

弘始十六年(414年),姚興病重,太子姚泓屯兵東華門,並在諮議堂侍疾。當時姚弼卻意圖作亂,招集了數千人並藏匿在其府中。姚裕當時與任謙、梁喜等人都掌禁軍守衞皇宮,而姚裕就派使者將姚弼謀反的行狀告知各個外藩,於是駐蒲阪的姚懿、洛陽的姚洸及雍城的姚諶都將要領兵入長安討伐姚弼。此時姚興病情好轉,召見了群臣,征虜將軍劉羌向姚興泣告姚弼謀反之事,尹昭等都建議姚興即使不按法處死,也應削其權力,讓他散居藩國。姚興仍然欣賞才兼文武的姚弼,不忍殺他,於是免去其尚書令職位,以大將軍、廣平公身份還第。

及後姚懿、姚洸、姚宣及姚諶來朝,見面時姚宣哭請姚興按法處置姚弼,但姚興拒絕。撫軍東曹屬姜虬也上書指姚弼雖然被姑息,但其黨羽仍然活躍,姚弼變亂的心是不會變的,更請消除姚弼黨羽,以絕禍根。姚興就問梁喜:「天下的人全都以我兒子作為口實,要如何處理?」梁喜則說:「真的如姜虬所言,陛下應該早點有個決定。」姚興又沉默不言。

弘始十七年(415年),姚弼知姚宣在父親面前說自己不是,十分憤恨,於是就向姚興誣陷姚宣。姚興又相信,並召見當時到了長安的姚宣司馬權丕,責怪他沒有好好匡輔姚宣並要處死他。但權丕竟然捏造了姚宣的罪狀報告姚興。姚興於是大怒,收捕了姚宣並派姚弼率兵三萬出鎮秦州。尹昭知道後向姚興指讓姚弼統大軍在外,一旦姚興去世,就會是太子姚泓的大大威脅,試圖勸止姚興,但姚興不聽。

同年,姚興食五石散中毒,姚弼卻稱病不朝,又再次在府中招集兵眾。姚興知道後大怒,殺了姚弼黨羽殿中侍御史唐盛及孫元。姚泓卻在怪責自己,請姚興殺了他,或處之外藩。姚興於是召了姚讚、梁喜、尹昭及斂曼嵬,和他們討論後囚禁了姚弼並準備殺了他,又要將姚弼黨羽全部治罪。不過姚泓請命之下,都將他們寛恕。

弘始十八年(416年),姚興出行華陰,留姚泓監國。及後姚興病重回長安,姚弼黨羽尹沖等仍想發難,想趁姚泓出迎姚興而將其殺害,但姚泓只在黃龍門拜迎。其時尚書姚沙彌更意圖劫奪姚興到廣平公府,以姚興招引眾人支持,從而從姚泓手中奪去儲君之位。尹沖雖不從,但仍然考慮隨姚興乘輿入宮中作亂,只是未知姚興生死而不敢行動。姚興則命姚泓錄尚書事,並命姚紹及胡翼度掌禁軍,又命斂曼嵬收去姚弼府中的武器。

不久,姚興病情更趨嚴重,姚興妹南安長公主去探望他也得不到回應,姚興幼子姚耕兒就向哥哥姚愔報告姚興已死,叫他快點做決定。姚愔於是就帶其他的士兵攻端門,斂曼嵬領兵抵禦,而胡翼度就關上宮中四門。姚愔派壯士爬上門並進入宮內,並走到馬道。時在諮議堂侍疾的姚泓命斂曼嵬登武庫抵禦,而太子右衞率姚和都亦已率東宮士兵在馬道南駐屯。姚愔無法前進,只得燒毀端門。姚興此時竭力走到前殿,並下令賜死姚弼。禁軍見到姚興士氣大振,向姚愔軍發動進攻,姚和都也在後夾擊,最終姚愔軍潰敗,姚愔逃到驪山,呂隆則逃到雍城,尹沖及尹泓就南奔東晉。

姚興召姚紹、姚讚、梁喜、尹昭和斂曼嵬入寢宮,遺命他們為輔政大臣。姚興即逝世,享年五十一歲。諡文桓皇帝,庙号高祖,下葬偶陵。

姚興曾命各郡國每年都上報一個品行純潔的孝廉,又留心政事,廣納百言,包容各種意見。即使只是說了一句姚興認為有益的建言,都會得到特別禮待。如杜瑾、吉默和周寶就曾因向姚興陳述當時國中大事而獲授要職。姚興又重文教,當時有姜龕、淳于岐及郭高等有大德的老儒士在長安教學,各有數百門生,其中有不少門生更遠道而來。而姚興就在處理政務以外的時間請姜龕等到東堂和他談論學問和技藝。當時一叫胡辯的人在當時仍是東晉佔領的洛陽授學,很多關中人都去拜師,姚興更下令各關守長盡量方便這些求學的人出入。種種措施都令後秦儒學興盛。

皇初九年(399年),姚興以国内天灾频频,於是自降帝号,称秦王;另又下令郡國將因戰亂而賣身為奴婢的人變回良人,更將幾個貪財官員誅殺,整頓官員風氣。及後又在長安建立法律學校,讓各郡縣散吏入讀,學成者就送回郡縣以處理形獄事項,又下令郡縣無法裁決的都上交廷尉處理。姚興更經常到諮議堂聽訟和作判決,大大減少了冤獄。

弘始三年(401年)呂隆向後秦請降後,姚興就迎在後涼的僧人鳩摩羅什入長安,並奉其為國師,奉之如神。鳩摩羅什在長安組織了大規模的翻譯佛經事業,姚興亦信了佛,於是群下都跟著信奉佛教,又吸引了五千多個僧人遠道而來。姚興又在永貴里建了佛塔、在中宮建了波若臺,佛教興盛,各州郡都受到佛教影響,至「求佛者十室而九。」

姚興在位後期,國庫不足,曾增加關隘和渡口的稅,又向鹽、竹、山林和木材徵稅。群臣曾勸諫但姚興認為能夠常出入關隘及取利於山水資源的都是富人,現在增稅其實只是取富人多餘的而彌補國家不足,並無不妥。

姚興生性儉約,所乘車馬都沒有黃金或玉石裝飾,以身作則之下,群下都崇尚清正廉潔。不過姚興卻喜歡打獵,常傷及農作物。杜挻及相雲曾分別作《豐草詩》及《德獵賦》以作暗示,姚興雖然明白並以黃金及布帛作賞賜,但始終改變不了打獵的習慣。

每當大臣去世,姚興都不會只按慣例在東堂發哀,而會親身去臨喪。

姚興十分看重親族,更對兩名叔叔姚碩德及姚緒十分恭敬。姚興降號為王時,本為王爵的姚碩德及姚緒應當降為公爵,但姚興不允,在二人再三辭讓下才得允許。姚興又曾下令所有官員取名時不得犯二人名諱,所有車馬、衣服及器玩都先給二人,自己只用次一等的,見面時行家人之禮,朝中大事亦必定先諮詢二人。姚沖叛變不遂殺了顧命大臣之一的狄伯支,姚興仍然顧念他是最小的弟弟,雄武絕人,還想對他寬容一次,不過在斂成規勸下,姚興想到他殺了狄伯支,就下書賜死姚沖。

\subsubsection{皇初}

\begin{longtable}{|>{\centering\scriptsize}m{2em}|>{\centering\scriptsize}m{1.3em}|>{\centering}m{8.8em}|}
  % \caption{秦王政}\
  \toprule
  \SimHei \normalsize 年数 & \SimHei \scriptsize 公元 & \SimHei 大事件 \tabularnewline
  % \midrule
  \endfirsthead
  \toprule
  \SimHei \normalsize 年数 & \SimHei \scriptsize 公元 & \SimHei 大事件 \tabularnewline
  \midrule
  \endhead
  \midrule
  元年 & 394 & \tabularnewline\hline
  二年 & 395 & \tabularnewline\hline
  三年 & 396 & \tabularnewline\hline
  四年 & 397 & \tabularnewline\hline
  五年 & 398 & \tabularnewline\hline
  六年 & 399 & \tabularnewline
  \bottomrule
\end{longtable}

\subsubsection{弘始}

\begin{longtable}{|>{\centering\scriptsize}m{2em}|>{\centering\scriptsize}m{1.3em}|>{\centering}m{8.8em}|}
  % \caption{秦王政}\
  \toprule
  \SimHei \normalsize 年数 & \SimHei \scriptsize 公元 & \SimHei 大事件 \tabularnewline
  % \midrule
  \endfirsthead
  \toprule
  \SimHei \normalsize 年数 & \SimHei \scriptsize 公元 & \SimHei 大事件 \tabularnewline
  \midrule
  \endhead
  \midrule
  元年 & 399 & \tabularnewline\hline
  二年 & 400 & \tabularnewline\hline
  三年 & 401 & \tabularnewline\hline
  四年 & 402 & \tabularnewline\hline
  五年 & 403 & \tabularnewline\hline
  六年 & 404 & \tabularnewline\hline
  七年 & 405 & \tabularnewline\hline
  八年 & 406 & \tabularnewline\hline
  九年 & 407 & \tabularnewline\hline
  十年 & 408 & \tabularnewline\hline
  十一年 & 409 & \tabularnewline\hline
  十二年 & 410 & \tabularnewline\hline
  十三年 & 411 & \tabularnewline\hline
  十四年 & 412 & \tabularnewline\hline
  十五年 & 413 & \tabularnewline\hline
  十六年 & 414 & \tabularnewline\hline
  十七年 & 415 & \tabularnewline\hline
  十八年 & 416 & \tabularnewline
  \bottomrule
\end{longtable}

%%% Local Variables:
%%% mode: latex
%%% TeX-engine: xetex
%%% TeX-master: "../../Main"
%%% End:

%% -*- coding: utf-8 -*-
%% Time-stamp: <Chen Wang: 2019-12-19 15:09:10>

\subsection{姚泓\tiny(416-417)}

\subsubsection{生平}

姚泓(388年-417年),字元子,十六国时期后秦末主,后秦文桓帝姚兴长子。

后秦弘始十八年(416年)正月,姚兴卒,姚泓即位。兄弟相争,国中大乱。八月东晋刘裕起兵伐秦。后秦疲于应敌之际,国内又相继发生姚懿、姚恢的叛乱。

次年八月癸亥(417年9月20日),刘裕帐下大将王镇恶攻入长安平朔门。姚泓无计可出,准备出降,他十一岁的儿子姚佛念说,晋朝人“将逞其欲”,(即使投降)我们也一定不能保全自己,我愿自杀。姚泓怃然不知所对。佛念登上城墙自投而死。姚泓率一家老小至王镇恶大營投降,其堂叔姚赞也率宗室子弟一百余人投降。刘裕將後秦王室全部处死,其余宗族成员迁往江南。姚泓被押往建康斩首。后秦亡。

《晋书》载,泓孝友宽和,而无经世之用,又多疾病,兴将以为嗣而疑焉。久之,乃立为太子。兴每征伐巡游,常留总后事。博学善谈论,尤好诗咏。

\subsubsection{永和}

\begin{longtable}{|>{\centering\scriptsize}m{2em}|>{\centering\scriptsize}m{1.3em}|>{\centering}m{8.8em}|}
  % \caption{秦王政}\
  \toprule
  \SimHei \normalsize 年数 & \SimHei \scriptsize 公元 & \SimHei 大事件 \tabularnewline
  % \midrule
  \endfirsthead
  \toprule
  \SimHei \normalsize 年数 & \SimHei \scriptsize 公元 & \SimHei 大事件 \tabularnewline
  \midrule
  \endhead
  \midrule
  元年 & 416 & \tabularnewline\hline
  二年 & 417 & \tabularnewline
  \bottomrule
\end{longtable}


%%% Local Variables:
%%% mode: latex
%%% TeX-engine: xetex
%%% TeX-master: "../../Main"
%%% End:



%%% Local Variables:
%%% mode: latex
%%% TeX-engine: xetex
%%% TeX-master: "../../Main"
%%% End:

%% -*- coding: utf-8 -*-
%% Time-stamp: <Chen Wang: 2019-12-19 15:30:59>


\section{后燕\tiny(384-407)}

\subsection{简介}

後燕(384年-407年或409年)是中国五胡十六国時慕容氏諸燕之一,由鮮卑人前燕文明帝慕容皝第五子慕容垂所建立的政權。

後燕建國之初定都中山(今河北省定州市),後遷往龍城(今遼寧省朝陽市)。全盛時統治範圍「南至琅琊,東訖遼海,西屆河汾,北暨燕代」(《讀史方輿紀要》),即今河北、山東、山西和河南、遼寧的一部分。自384年慕容垂稱燕王到407年慕容熙被殺(或到409年慕容雲被杀),立國凡24年(一说26年)。

《十六国春秋》始称后燕,以别于慕容氏諸燕,后世袭用之。

重建燕國(383年─385年):前秦在383年淝水之戰大敗後,投降前秦的前燕貴族慕容垂在苻堅同意下回到鄴城。時丁零族翟斌在洛陽新安一帶起兵反秦,鎮守鄴城苻堅庶長子苻丕撥兵2千給慕容垂,派宗室苻飛龍領兵1千為慕容垂的副手,前去對付翟斌,但慕容垂於行軍中襲殺苻飛龍,與前秦正式決裂。

384年正月,慕容垂渡過黃河移至洛陽附近,與翟斌聯兵攻洛陽。後引兵東下,在滎陽自稱大將軍、大都督、燕王,建元燕元元年。後自石門渡黃河,向鄴前進,時有眾20餘萬。因苻丕堅守鄴城,慕容垂久攻不下,因战争,河北經濟受到很大的破壞。到了385年八月,苻丕撤出鄴城,退往晉陽,整個河北,皆落入慕容垂手中。386年正月,慕容垂稱帝,定都中山(今河北省定州),改元建興,史稱後燕。

攻滅西燕(386年─394年):386年十月,西燕慕容永進至長子(今山西省長子縣西),稱帝,改元中興,佔有今山西省一帶。由於後燕不容許作為宗室一方的西燕「僭舉位號,惑民視聽」,與後燕爭奪燕國領導權。在392年消滅翟魏後,出兵攻伐西燕。

393年冬,慕容垂徵發步騎兵7萬,命丹陽王慕容瓚出井陘關(今河北省井陘縣井陘山),攻晉陽,西燕守將慕容友領兵5萬防守潞川。明年春,慕容垂增調司、冀、青、兗四州兵,分兵三路出滏口(今河北省磁縣西北石鼓山)、壺關、沙亭,西燕分兵拒守。後慕容垂在鄴城西南屯兵月餘,慕容永懷疑後燕欲從太行山南口進兵,將大部兵力調往軹關。夏,慕容垂率大軍出滏口,由天井關向南直趨臺壁(今山西省黎城縣西南),慕容永倉卒集結5萬精兵,與後燕軍大戰於臺壁南,西燕軍中伏大敗,慕容永逃回長子。後燕攻下晉陽,進圍長子,八月間滅西燕。

慕容垂滅西燕後,趁東晉衰亂之際,略地青、兗,把疆域向南擴展到今山東的臨沂、棗莊一帶。

燕魏對峙(394年─396年):386年,拓跋珪建立北魏。起初後燕與北魏的關係本來是友好的,因後燕戰馬缺乏,屢求於魏,甚至發生扣留北魏使者以求名馬的事,兩國關係告結。而北魏採取聯西燕拒後燕的政策,對付後燕。394年西燕危急時,北魏派兵5萬為西燕聲援。次年五月,慕容垂命太子慕容寶、趙王慕容麟率兵8萬伐魏,遣范陽王慕容德率步騎1.8萬為後繼。北魏聽說燕軍北上,把部落、畜產及大軍轉移至黃河以南(今內蒙古伊克昭盟),避開燕軍。到十月,由於塞外嚴寒、士氣低落,後燕不得不撤退。這時北魏派拓跋遵領騎兵7萬,堵塞燕軍南歸之路。拓跋珪自領2萬,進擊後燕軍,後燕軍大敗,亂不成軍,四、五萬兵投降,北魏俘虜了後燕文武將吏數千人,繳獲了兵器、衣甲、糧食無數,拓跋珪將後燕降兵全部坑殺於參合陂,慕容寶等單騎逃回。史稱參合陂之役。

慕容寶等逃回中山後,屢請求再次伐魏,慕容德也勸說慕容垂趁自己尚健在時親征,以免遺留後患。慕容垂接受了他們的意見。396年三月,慕容垂率大軍再次伐魏,敗北魏陳留公拓跋虔,其後慕容垂病情加重,急忙退兵。四月,慕容垂病死。慕容垂此次的北伐,並沒有能夠挽回後燕軍事上的頹勢。此後,拓跋珪就挾其三、四十萬騎兵,長驅進入中原。

衰落滅亡(396年─409年):396年四月,慕容垂病死,子慕容寶繼承帝位。後燕在全國重要的戰略及政治中心有五處,即中山、龍城、鄴、晉陽、薊。八、九月間,北魏拓跋珪率領40餘萬大軍,攻取晉陽。十一月,攻下常山、信都,河北許多郡縣的官員,不是逃亡就是投降。這時慕容寶在中山有步兵12萬、騎兵3.7萬,悉數出抗拒魏軍,大敗而還。魏軍進軍包圍了中山,397年三月,慕容寶率軍突圍,退往中山。十月,魏軍攻下中山,後燕官吏兵投降兩萬餘人,後燕的疆域被切斷為南、北二部。

398年,慕容德在滑臺稱燕王,建立南燕。蘭汗殺死慕容寶,自稱大將軍、大單于、昌黎王。慕容盛殺蘭汗自立,後來討伐高麗及庫莫奚有功,然因治下太嚴,刑罰殘忍,在401年為大臣段璣所暗殺。鮮卑貴族立慕容垂少子慕容熙為帝,他採行了胡漢分治的政策來統治國家。這時的後燕疆域,僅有遼西一帶,疆域狹小,民戶不多,但他卻大興土木,營建宮苑殿閣,給人民帶來無窮的災難。407年,馮跋兄弟趁慕容熙送葬苻后時起事,推高雲(慕容雲)為燕王,殺死慕容熙。409年高雲的禁衛離班、桃仁殺死高雲,馮跋稱燕天王,後燕滅亡。

384年 建国。慕容垂包圍鄴。

385年 打敗高句麗,進入遼東。

394年 攻滅西燕。

395年 燕軍與北魏軍在参合陂大戦,燕軍大敗。

396年 慕容垂親率大軍進攻北魏,途中病故。太子慕容宝即位。

397年 魏拓跋珪攻擊燕都中山。燕王慕容宝逃往龍城。

398年 慕容宝被殺,慕容盛推翻弑君的蘭汗,奪取皇位。

401年 慕容盛被暗殺。太后丁氏擁立慕容熙。

407年 漢人將軍馮跋擁立高雲為燕天王,殺死慕容熙。有些史學家把這年認定為後燕滅亡,北燕建立之年。

409年 高雲被殺,馮跋繼位。有些史學家把這年認定為後燕滅亡,北燕建立之年。


%% -*- coding: utf-8 -*-
%% Time-stamp: <Chen Wang: 2019-12-19 15:20:37>

\subsection{成武帝\tiny(384-396)}

\subsubsection{生平}

燕成武帝慕容垂(326年-396年6月2日),字道明,原名霸,字道業,一說字叔仁,鮮卑名阿六敦,昌黎棘城(今遼寧義縣)鮮卑族人。十六國後燕開國君主。前燕文明帝慕容皝的第五子。在前燕時屢有戰功,更加曾擊退東晉桓溫的北伐軍。然而因為受到當政的慕容評排擠而被逼出走前秦,但很受前秦君主苻堅的寵信。淝水之戰後慕容垂乘時而起,復建燕國,建立後燕,後又滅了同為慕容氏所建的西燕。參合陂之戰戰敗後率軍再攻北魏,在期間發病病重,並在退軍時去世。

原名慕容霸的慕容垂甚得父親慕容皝寵愛,甚至比起身為世子的哥哥慕容儁更多,故此慕容儁忿忿不平。咸康八年(342年),慕容皝進攻高句麗,慕容霸與慕容翰作前鋒,終攻陷高句麗都城丸都(今吉林集安西)。建元二年(344年),慕容皝攻伐宇文逸豆歸,慕容翰為前鋒都督,慕容霸與慕容軍、慕容恪及慕輿根則受命兵分三道進攻。當時逸豆歸派遣涉奕于率領精兵抵禦,慕容翰決意以擊敗涉奕于以摧毀宇文部士氣,令宇文部自潰,於是主動進攻,涉奕于親自迎戰,慕容霸於是在側邀擊,與慕容翰擊敗涉奕于。宇文部士兵於戰後果然自潰,宇文逸豆歸出逃敗死漠北,成功消滅了宇文部。慕容霸則以此功封都鄉侯。永和元年(345年),後趙將領鄧恆領兵數萬駐屯樂安(今河北樂亭縣東北),意圖併吞前燕。慕容皝以慕容霸為平狄將軍,駐軍徒河(今遼寧錦州西北),鄧恆因為畏懼慕容霸而不敢進犯。

永和四年(348年),慕容皝去世,慕容儁繼位燕王,就以慕容霸曾經墮馬而撞斷了牙齒為由改其名為「慕容𡙇」,後更去「夬」而改名慕容垂。次年後趙皇帝石虎去世,國內因諸子爭位而大亂,慕容垂於是上書慕容儁建議出兵後趙。慕容儁初以慕容皝新死而不允,但慕容垂親往都城龍城(今遼寧朝陽市)勸說慕容儁,更自請為前驅領兵威逼鄧恆。在封奕等人的支持下,慕容儁以慕容垂為前鋒都督、建鋒將軍,選二十多萬精兵準備伐趙。

永和六年(350年)二月,慕容儁命慕容垂領二萬兵經循東路經徒河伐趙,另遣慕輿于出西道,自率中軍,兵分三路伐趙。慕容垂到三陘(今河北撫寧縣矛石山),鄧恆驚懼而燒倉庫出逃,退保薊城(今北京)。慕容垂到後盡收樂安、北平兩郡兵糧,與慕容儁會合共攻薊城。三月,燕軍攻下薊城,慕容垂勸止了慕容儁阬殺後趙士卒的決定。不久慕容儁又親率軍隊進攻鄧恆,至清梁(今河北清苑縣西南)時趙將鹿勃早率數千人夜襲燕軍,突入慕容垂幕下,慕容垂於是奮力反擊,手刃了十多人,遏制了鹿勃早的攻擊,及後慕輿根等人領兵擊敗鹿勃早,成功擊退了來襲。

元璽元年(352年),慕容儁稱帝,任黃門侍郎,又遷安東將軍、冀州刺史,鎮常山。至元璽三年(354年)封慕容垂為吳王,並移鎮信都(今河北冀縣)。後召為侍中、右禁將軍、錄留臺事,轉鎮龍城,但因慕容垂在當地很得人心,故被慕容儁召還。後又轉撫軍將軍,並於光壽元年(357年)與中軍將軍慕容虔等率軍大敗敕勒。

光壽二年(358年),中常侍涅皓知慕容儁不喜歡慕容垂,又因可足渾皇后不滿慕容垂妻段氏,於是誣稱段氏與吳國典書令高弼行巫蠱之術,意圖以此牽連慕容垂。段氏寧死不屈,雖然最終死在獄中,但都沒有將慕容垂牽連到事件中,後慕容垂遷鎮東將軍、平州刺史,外鎮遼東。

建熙元年(360年),慕容儁去世,太子慕容暐繼位,以慕容垂為河南大都督、征南將軍、兗州牧、荊州刺史,領護南蠻校尉,鎮梁國。建熙六年(365年),慕容垂與慕容恪共攻東晉控制的洛陽(今河南洛陽市),擊敗並俘虜晉將沈勁,攻下了洛陽,隨後遷都督荊揚洛徐兗豫雍益涼秦十州諸軍事、征南大將軍、荊州牧,鎮魯陽。

太宰慕容恪深知慕容垂的才能,故此在建熙八年(367年)病死前向樂安王慕容臧指出應以慕容垂擔任大司馬一職,又向慕容暐推薦慕容垂在其死後接替自己,將政事都交給慕容垂處理。慕容臧雖將慕容恪的話告訴主政的太傅慕容評,但慕容評沒有按慕容恪的意思做,以慕容沖為大司馬,又調慕容垂為侍中、車騎大將軍、儀同三司。

建熙十年(369年)四月,東晉大司馬桓溫北伐前燕,諸將都無法抵抗晉軍,讓晉軍於七月進駐枋頭(今河南浚縣)。當時慕容暐及慕容評皆大驚,想逃回故都龍城避難。慕容垂於是請求讓他出戰。慕容暐就任命他接替慕容臧擔任南討大都督,率慕容德等五萬兵出戰。慕容垂又請了黃門侍郎封孚、司徒左長史申胤及尚書郎悉羅騰從軍。桓溫當時以降人段思為響導,悉羅騰與晉軍接戰,生擒了段思;接著桓溫派李述進攻,又被悉羅騰所敗,李述更戰死,晉軍士氣於是下降。同時慕容德等又至石門阻止晉軍開通漕運,豫州刺史李邽又斷晉軍糧道,桓溫屢戰不利,糧食又不足,終於九月循陸路撤軍。當時諸將打算立刻追擊,但慕容垂以晉軍初退,必定嚴加戒備,以精銳軍隊斷後,於是打算遲點才追擊,待晉軍乘追兵未至而加速行軍,令兵士筋疲力盡時才進攻。慕容垂因而率領八千騎兵緩緩尾隨晉軍,發現桓溫果然在看不見追兵後加速。數日後慕容垂下令進攻,騎兵於是加速,於襄邑(今河南睢縣西)趕上晉軍,配合慕容德所率埋伏於襄邑的伏兵夾擊桓溫,於是大敗晉軍,殺三萬人。桓溫只有收拾殘軍南退。

枋頭之戰大勝後,慕容垂威名大振,卻令慕容評更加嫌忌他,慕容垂上請有戰功的將領獲得封賞都沒得批准,兩人就因此事在廷上互相爭論,更加深化了兩人的嫌隙。時為太后的可足渾皇后亦厭惡慕容垂,於是與慕容評密謀誅除他。慕容恪子慕容楷及慕容垂舅舅蘭建得悉陰謀,於是建議慕容垂先發制人,除去慕容臧及慕容評。然而慕容垂卻表示寧願出奔國外亦不想骨肉相殘。世子慕容令得知後建議慕容垂北奔龍城,並向慕容暐謝罪,盼望慕容暐感悟召還;即使不然,仍可以固守當地以求自保。慕容垂聽從,於同年十一月就上請到大陸澤狩獵,微服潛歸龍城。然而到邯鄲(今河北邯鄲)時,向來不得寵的兒子慕容麟卻逃還鄴城(今河北臨漳西)告發父親的意圖,於是跟隨慕容垂的人大多都逃走,慕容強亦奉命追捕慕容垂。至范陽(今河北涿縣)時慕容強追上慕容垂,但因慕容令親自斷後,慕容強也不敢進逼。日落後,慕容令表示原本的計劃已不再可行,又建議投奔前秦,慕容垂計窮,亦得接受,於是棄用馬匹以免留下蹤跡,悄悄回鄴城並躲於顯原陵。不久竟有數百個獵人從四方向他們所在聚集,慕容垂等人敵不過他們,卻又無處可逃,甚麼也做不了。就在此時,獵人的獵鷹卻同時飛起,獵人於是散去,慕容垂因而殺白馬祭天,與隨行者誓盟。慕容令在那時又建議讓他回鄴城襲殺慕容評,並以慕容垂的名望取而代之,入輔朝廷。但慕容垂以此危險而否決,於是與妻段氏、慕容令、慕容寶、慕容農、慕容楷及蘭建、高弼等西奔前秦。前秦天王苻堅得知慕容垂來奔,十分高興並親自迎接,以慕容垂為冠軍將軍,封賓徒侯。

慕容垂奔秦次年,前秦就滅了前燕,而慕容垂在前秦官至京兆尹,進封為泉州侯。建元十八年(382年),苻堅執意要攻伐東晉,苻融、石越、苻宏等人都反對,而慕容垂卻說:「弱者被強者所吞,小的被大的兼併,這是合乎自然的,並不難理解。以陛下神武,順應天期,聲威布於海外,百萬衞士,滿朝韓信、白起那樣的良將,晉這個於江南的小國獨獨違抗王命,怎可以再留她給子孫。《詩經》說:「谋夫孔多,是用不集」陛下自己決定就夠了,又何必詢問一眾朝臣!晉武帝平滅東吳,也不過只有張華、杜預幾個臣子支持而已,若果他順從朝臣主流意見,又怎能成就統一大業!」苻堅聽後大喜,更說:「和我一起平定天下的人,就只有你呀。」建元十九年(383年)五月,東晉荊州刺史桓沖北伐,親率主力進攻襄陽(今湖北襄陽市),慕容垂就與苻叡率兵救援。苻叡以慕容垂為前鋒進至沔水,慕容垂在夜間命士兵每人拿十個火把,將它們縛在樹枝上,讓桓沖以為援軍兵力很強,成功逼使他撤還。同年八月,苻堅正式出兵伐晉,並命苻融及慕容垂率二十五萬兵作為前鋒。苻融攻下了壽春(今安徽壽縣),而慕容垂就率別軍攻下了鄖城(今湖北鄖縣)。

十一月,苻堅於淝水大敗給晉軍,前線的前秦軍隊潰敗,就只有沒有參加淝水之戰的慕容垂一軍是完整的,故此苻堅就率殘軍投靠他。當時慕容寶等人就勸慕容垂殺了苻堅,但慕容垂不肯,更分兵給苻堅。苻堅到了洛陽後已經又招聚了十多萬人,一直到了澠池(今河南澠池縣西),慕容垂表示想去安撫河北,並想去拜謁宗廟。苻堅不顧權翼反對而准許慕容垂所請。

當時駐守鄴城的苻丕知道慕容垂要來,懷疑他意圖作亂,更想襲擊他,只是姜讓以慕容垂未有謀反舉動,勸苻丕先嚴兵守衞,注意其舉動,苻丕才安置慕容垂住在鄴城西部。慕容垂當時雖然不肯乘機殺死苻丕,但仍暗中聯結前燕舊臣,密謀復國。此時,丁零人翟斌起兵,苻堅命慕容垂討伐,苻丕一直怕慕容垂於鄴城作亂,正就打算借此機會送走他,更期望他與翟斌打得兩敗俱傷,好讓自己消滅兩股勢力。於是給了慕容垂二千弱兵及差劣的兵器鎧甲,更派了苻飛龍為副手,意圖以他解決慕容垂。

慕容垂留了慕容農、慕容楷及慕容紹於鄴,在行軍途中閔亮和李毗就從鄴來到,並告知苻丕與苻飛龍的圖謀。慕容垂於是以此激怒士眾,又以兵少為由留於河內郡募兵,十日間就令部眾增至八千人。及後正受翟斌攻擊的豫州刺史苻暉請慕容垂快點進兵,慕容垂向苻飛龍說要改在夜裏行軍,出其不意,然而其實就已與諸子計劃襲殺苻飛龍,終在晚上襲殺了苻飛龍及他手下的一千氐兵。第二日,慕容垂命田山回鄴告知留於鄴城的慕容農等起兵響應自己,三人於是與數十騎微服出走,在列人(今河北肥鄉縣東北)起兵。

燕元元年(384年),慕容垂圖攻洛陽,當時翟斌帳下有前燕宗室慕容鳳及前燕舊臣之子段延等,都勸翟斌奉慕容垂為盟主,慕容垂原本不知翟斌究竟是否真心歸附,並沒答允,但到洛陽後苻暉因知苻飛龍遇害而拒絕以營救苻暉為名的慕容垂進城,至此慕容垂才接受了翟斌。不久慕容垂以洛陽是四戰之地,於是改攻鄴城,至滎陽(今河南滎陽)時,群下請慕容垂稱帝。正月丙戌(384年2月9日),慕容垂則以晉元帝的先例,先稱大將軍、大都督,燕王,承制行事。接著率二十多萬大軍直攻鄴城。慕容垂至鄴後改元「燕元」。

慕容垂接著引兵攻鄴,苻丕派了姜讓去責備慕容垂,又勸他放棄叛變。然而慕容垂卻表示只想苻丕和平離開,獻出鄴城,並允諾與前秦世代友好;又恐嚇若果苻丕不從,將要以兵力強攻,怕苻丕到時即使想全身而退也不能。姜讓聽後指責慕容垂背叛王室,不顧昔日前秦收留自己的恩德,現在要做叛逆的鬼。慕容垂聽後沉默,但沒有聽從旁人所說將姜讓殺害,反表示尊敬,讓他回去。然而最終仍然陳述利害,勸苻丕棄城出走,激得苻堅及苻丕再寫書指責。游說不果後,燕軍開始進攻鄴城,並攻下其外城,苻丕退守中城。接著慕容垂又用二十多萬丁零及烏桓人用梯及地道戰術攻城,但都不成功,於是下令修築長圍作防守,築新興城放置輜重,作長期戰。不久又以漳水灌城,仍不能攻下,於是改為圍困鄴城,只留西邊缺口讓秦軍西走。

燕元二年(385年)四月,東晉將領劉牢之入援鄴城,慕容垂詐敗誘敵,於是撤圍退屯新城,不久再北撤,劉牢之於是追擊,苻丕聞訊亦率軍後繼,劉牢之一路追擊至五橋澤,因為軍隊忙於搶奪燕軍輜重而遭慕容垂擊敗。至八月,苻丕棄守鄴城,燕軍終成功佔領鄴城。十二月,慕容垂正式定都中山(今河北定州市)。燕元三年(386年)正月,慕容垂稱帝,二月改元「建興」,始置百官。八月,慕容垂率兵南征以擴疆土,並於次年正月襲河東地區,擊敗晉濟北太守溫詳。

慕容柔、慕容盛及慕容會於建興三年(387年)從西燕都城長子(今山西長子縣西)到達中山,投奔後燕,當時慕容垂就問當地情況,意圖攻取。不久,慕容永将治下慕容儁、慕容垂子孙不问男女全部杀死。建興八年(392年),慕容垂率軍擊潰了丁零人翟釗,吞併了其部眾。次年十一月,慕容垂就親率七萬兵西征西燕;次年二月慕容垂大發司、冀、青、兗四州兵,分置各兵準備進攻。至五月,燕軍經天井關進攻臺壁,先後擊敗大逸豆歸及小逸豆歸,圍困了臺壁。慕容永自太行回軍臺壁,慕容垂亦率軍到臺壁,兩軍於是交戰。事前慕容垂派了驍騎將軍慕容國在澗下設伏,於是假裝撤退引慕容永追擊,數里後慕容國伏兵出現斷慕容永後路,燕軍於是四面進攻,大敗慕容永。慕容永敗後逃回長子,慕容垂就於六月追至,並圍困城池。至八月,被圍的慕容永困急,先後向東晉及北魏求援,但在援軍到來前大逸豆歸部將伐勤就開城門迎燕軍,慕容垂於是俘虜慕容永並將其殺害,吞併了西燕。

建興二年(386年),拓跋珪復代國,不久改稱魏王,建立了北魏。同年因國內不穩而請後燕援軍,慕容垂派慕容麟救援,終助拓跋珪解決事件。事後雖然拓跋珪不接受後燕封爵,但燕魏兩國每年都有使臣往來。建興七年(391年),拓跋珪派弟弟拓跋觚出使後燕,但當時主事的慕容垂諸子為求良馬,竟扣留了拓跋觚,如此令拓跋珪中斷兩國交往。至建興十一年(395年)五月,慕容垂因北魏侵擾邊塞諸郡而命太子慕容寶等人率兵伐魏。當時魏軍率眾迴避,燕軍於七月到了五原(今內蒙古包頭西北),收降三萬多家及大量糧食,但未與魏軍決戰。而拓跋珪乘當時慕容垂患病,故意阻截燕軍通往中山的道通,捕捉後燕使者,令燕軍與其國內通訊斷絕,從而以慕容垂已死的假消息擾動燕軍軍心。兩軍自九月臨五原河相持至十月,慕容寶及慕容麟因為慕容麟部將慕輿嵩相信慕容垂死訊而圖謀作亂的事件而互相猜疑,終於燒船乘夜撤退。當時河面尚未結冰,慕容寶認為魏軍不能渡河追擊,於是不設斥候監視魏軍。至十一月,魏軍因暴風令河面結冰而追擊,在參合陂追上燕軍,並發動突襲大敗燕軍,大量文武官員及四五萬人的燕軍士兵都被俘,後北魏更阬殺全數燕軍士兵。

慕容寶敗逃回中山,並以參合陂之戰為恥,再請進攻北魏。當時司徒慕容德建言說慕容寶大敗後已被北魏輕視,想要慕容垂親自率兵征服他們,以免留為後患。慕容垂於是命幽州牧慕容隆及留守薊城行臺的慕容盛率手下精兵到中山,決定次年再度伐魏。

三月,慕容垂秘密出兵,跨越青嶺(今河北易縣西南五廻山),經天門(今河北淶源縣)鑿山開路,出魏軍不意直攻雲中郡。慕容垂率軍至獵嶺(今山西代縣夏屋山)時就命慕容隆及慕容農為前鋒,進襲平城(今山西大同市)。當時燕國軍隊都因參合陂之戰大敗而畏懼魏軍,就只有慕容隆這批來自龍城的士兵仍然奮勇進攻;而留守平城的魏將拓跋虔亦沒作防備,故此在閏三月慕容隆兵臨平城時才發現燕軍,率眾抵抗,最終敗死,部眾都被燕軍接收。拓跋虔戰死的消息令身處盛樂(今內蒙古和林格爾北)的拓跋珪感到恐懼,打算出走迴避,但各諸知拓跋虔死訊亦各懷二心,令拓跋珪不知何去何從。

慕容垂經過參合陂戰場時看見被阬殺的士兵骸骨堆積如山,就為他們置祭,士兵們見此皆傷心痛哭,這令慕容垂既慚愧又憤恨,終因而嘔血病發,要坐馬車前進,到平城西北三十里處停駐。當時慕容寶已領兵至雲中,聞訊亦退兵。有叛燕軍人就因而向北魏報告慕容垂已死的消息,拓跋珪想去追擊,但知平城陷落後就打消念頭。慕容垂在平城停留了十日後病情加重,於是修築燕昌城而南歸,至四月癸未日(6月2日)於沮陽(今河北懷來縣)去世,享年七十一歲。諡號為成武皇帝,廟號世祖。

崔浩:「垂藉父兄之資,修復舊業,國人歸之,若夜蟲之就火,少加倚仗,易以立功。」(《資治通鑑·卷一百一十八·晉紀四十》)

\subsubsection{燕元}

\begin{longtable}{|>{\centering\scriptsize}m{2em}|>{\centering\scriptsize}m{1.3em}|>{\centering}m{8.8em}|}
  % \caption{秦王政}\
  \toprule
  \SimHei \normalsize 年数 & \SimHei \scriptsize 公元 & \SimHei 大事件 \tabularnewline
  % \midrule
  \endfirsthead
  \toprule
  \SimHei \normalsize 年数 & \SimHei \scriptsize 公元 & \SimHei 大事件 \tabularnewline
  \midrule
  \endhead
  \midrule
  元年 & 384 & \tabularnewline\hline
  二年 & 385 & \tabularnewline\hline
  三年 & 386 & \tabularnewline
  \bottomrule
\end{longtable}

\subsubsection{建兴}

\begin{longtable}{|>{\centering\scriptsize}m{2em}|>{\centering\scriptsize}m{1.3em}|>{\centering}m{8.8em}|}
  % \caption{秦王政}\
  \toprule
  \SimHei \normalsize 年数 & \SimHei \scriptsize 公元 & \SimHei 大事件 \tabularnewline
  % \midrule
  \endfirsthead
  \toprule
  \SimHei \normalsize 年数 & \SimHei \scriptsize 公元 & \SimHei 大事件 \tabularnewline
  \midrule
  \endhead
  \midrule
  元年 & 386 & \tabularnewline\hline
  二年 & 387 & \tabularnewline\hline
  三年 & 388 & \tabularnewline\hline
  四年 & 389 & \tabularnewline\hline
  五年 & 390 & \tabularnewline\hline
  六年 & 391 & \tabularnewline\hline
  七年 & 392 & \tabularnewline\hline
  八年 & 393 & \tabularnewline\hline
  九年 & 394 & \tabularnewline\hline
  十年 & 395 & \tabularnewline\hline
  十一年 & 396 & \tabularnewline
  \bottomrule
\end{longtable}


%%% Local Variables:
%%% mode: latex
%%% TeX-engine: xetex
%%% TeX-master: "../../Main"
%%% End:

%% -*- coding: utf-8 -*-
%% Time-stamp: <Chen Wang: 2019-12-19 15:24:49>

\subsection{惠愍帝\tiny(396-398)}

\subsubsection{生平}

燕惠愍帝慕容寶(355年-398年5月27日),字道祐,昌黎郡棘城县(今辽宁省锦州市义县西北)人,後燕第二任君主,慕容垂的第四子,母親是先段后。慕容垂建後燕後,立慕容寶為太子,曾領燕軍攻伐北魏,但在參合陂之戰慘敗。慕容垂死後慕容寶繼位為帝,但就面對北魏南侵,最終慕容寶沒能保住後燕在中原的土地,率眾北走龍城(今遼寧朝陽市),但先後遇上兒子慕容會及大臣段速骨的叛亂。慕容寶出走後為蘭汗所誘而歸龍城,最終被其殺害。

369年,慕容寶隨父親慕容垂等人自前燕逃亡至前秦,在前秦曾任太子洗馬及萬年令。

《太平御覽》載慕容寶玩樗蒲時向神祈禱富貴,擲出機率只有1/32768的三次「盧」采,讓他決心復國。

383年,前秦天王苻堅南伐東晉,慕容寶任陵江將軍。同年苻堅於淝水之戰大敗,軍隊潰散,只有未參與戰事的慕容垂軍隊仍然完整,於是前往投奔。慕容寶於是向父親建議趁機殺掉苻堅,復興燕國,不過慕容垂不肯。慕容垂終於384年稱燕王,立慕容寶為太子,建後燕。

慕容寶隨後經常留守後燕首都中山(今河北定州市),並在慕容垂在外時留守。387年,時慕容垂南征翟遼,井陘人賈鮑招引北山丁零翟瑤等夜襲中山,並攻下外城。章武王慕容宙率奇兵出外,而慕容寶在內鳴鼓抗敵,兩人夾擊之下大敗賈鮑等人,盡俘其眾,賈鮑及翟瑤隻身逃走。

395年五月,因北魏侵略後燕附塞諸部,慕容垂派慕容寶與慕容農、慕容麟等率八萬進攻北魏。拓跋珪率眾西渡黃河作迴避,並在河南治軍。慕容寶率眾到黃河邊就建造船隻打算渡河進攻,不過就在九月要列兵渡河時就遇上大風,船隻都被吹到南岸去。拓跋珪又派人從後阻截慕容寶與後燕國內的通訊,更派抓來的後燕使者訛稱慕容垂已死,令得軍心不穩,慕容寶亦都相當恐懼。十月辛末(11月23日),慕容寶燒船乘夜逃走,當時黃河尚未結冰,慕容寶以為北魏軍隊不能即時渡河追擊,故此不設斥候監察。不過八日後黃河面就因大風而結了冰,拓跋珪率眾渡河,並派二萬騎兵追擊。燕軍至參合陂時有遇上大風,更有一大片黑色塵土從後而來。僧人支曇猛認為這些都預示魏軍將來,建議慕容寶派兵防禦,但慕容寶以為已經走得很遠,笑而不答。慕容麟更奉承地說:「以殿下神武及強盛的兵眾,足以橫行沙漠了,索虜怎敢遠來呀!曇猛亂說話動搖眾心,應該處死呀!」支曇猛堅持,慕容德亦勸慕容寶聽從,慕容寶於是就派了慕容麟率三萬兵在後防備。不過慕容麟根本沒有防備的心,只顧著打獵。最終魏軍於參合陂突襲燕軍,大量兵眾不是在驚慌下互相踐踏或在河中遇溺而死就是束手就擒。慕容寶等人就帶著數千騎兵一同逃返後燕。戰後魏軍更盡坑俘獲的燕軍。

慕容寶回國後以此敗為恥,屢請慕容垂再次攻魏,慕容垂於是於次年(396年)大舉伐魏,並攻下平城(今山西大同市)。不過慕容垂在經過參合陂時看到被坑殺的燕兵骸骨堆積如山,士兵的痛哭聲又遍布山谷,令慕容垂在愧疚及憤恨下患病,被逼終止北伐。時慕容寶等人正率軍至雲中,追擊迴避的拓跋珪,但知慕容垂患病亦只好撤還。

慕容垂在回軍途中去世,慕容寶待回到中山時才為父發喪,並即位為帝,改元為永康。慕容寶年少無大志,喜歡別人奉承。但當太子後則磨煉自己,崇尚儒學,變得善談論,能作文,又卑委地討好慕容垂身邊小臣,以求得美譽。當時朝野都稱許慕容寶,而慕容垂亦認為他能夠保住家業,相當敬重他。後慕容垂為其建承華觀,又於388年以他錄尚書事,授予處理政務的權力,自己只處理一些重要的事務;又以其領大單于職位。不過慕容垂皇后段氏就曾指出太子才能不足,建議慕容垂立遼西王慕容農或高陽王慕容隆。又指出慕容麟為人奸詐而不肯屈於人下,有輕視太子之心,建議慕容垂早日除去他。不過慕容垂並不接納。慕容寶及慕容麟聽聞段皇后有這番話更是十分痛恨。慕容寶即位後,便派了慕容麟去逼令段后自殺。段后憤怒地說:「你們兄弟連逼殺嫡母的事也做,怎能保護國家!我怎會怕死,就可惜國家快滅亡了。」隨後便自殺。段后死後,慕容寶更因痛恨段后,以其無母后之道而打算不為其行居喪之禮,不過計劃最終在中書令眭邃反對之下擱置。

慕容寶繼位不久,北魏就出兵進攻後燕,並進攻中山,但被慕容隆擊退。及後北魏大人沒根因被拓跋珪厭惡而投降後燕,並請還攻北魏。慕容寶不敢給他重兵,只分了數百騎兵給他。沒根接著夜襲魏營,拓跋珪發覺有變而狼狽逃走,但沒根礙於兵少,無法對魏軍造成大傷害。永康二年(397年),任北魏并州監軍的沒根侄兒醜提因沒根降燕而害怕被株連,於是率部眾回國預備作亂。拓跋珪聞訊就想北返,派使者向後燕求和,但其時慕容寶知北魏有內亂,故此不肯答允,並率步兵十二萬及騎兵三萬七千的大軍到柏肆預備截擊返兵的魏軍。不久,魏軍到了滹沱水南岸紥營,慕容寶就率兵在夜間渡河,並招募了勇士一萬多人夜襲魏營,而慕容寶就在營北列陣作支援。夜襲部隊乘風縱火並迅速發動進攻,魏軍大亂,拓跋珪亦在驚惶中棄營出逃,燕軍到來帳中只得其衣物。不過接著燕軍竟然自亂,互相攻擊。拓跋珪於營外看見這情況就鳴鼓收整部眾,終大敗夜襲軍,更轉攻慕容寶軍,慕容寶只得回到北岸。次日,魏軍已經重整並與燕軍對峙,相反燕軍就士氣盡失。慕容寶最終只得退還中山,北魏軍跟著追擊,屢敗燕軍。慕容寶因屢敗而恐懼,竟拋棄大軍,自率二萬騎兵速速退回中山,又命士兵拋棄戰袍武器,以加快速度,丟失了大量軍需品,而且其時正遇大風雪,大量士兵凍死道上。拓跋珪及後再派兵進圍中山,駐屯在芳林園。當時中山城中將士都想出戰擊退圍城魏軍,慕容隆亦向慕容寶建議乘城中將士的鬥志進攻。慕容寶原本同意,但慕容麟卻多次反對,令慕容寶反悔,慕容隆於是多次列兵備戰都被逼罷兵。後慕容寶又意圖求和,以交還拓跋觚及割常山以西土地為條件,但不久即反悔,氣得拓跋珪親自率軍圍攻中山。當時有數千將士都自願請戰,但慕容隆披甲上馬,正待命令與魏軍決戰時,慕容麟再次勸止慕容寶,令兵眾忿恨,慕容隆亦痛心哭泣。

早前,慕輿皓謀弒慕容寶而改立慕容麟,失敗出逃,但令慕容麟內心不安。就在慕容麟勸止慕容寶派兵出戰當晚,以兵劫逼左衞將軍北地王慕容精,要他率禁軍弒慕容寶。慕容精拒絕,慕容麟就殺害慕容精,出奔西山依附丁零餘眾。其時慕容寶知慕容會正領兵前來,怕慕容麟劫奪慕容會的軍隊,先一步據有龍城,於是召見慕容隆及慕容農,想放棄中山,退保龍城,最終就與太子慕容策、慕容農、慕容隆、慕容盛等人率萬餘騎出城與慕容會軍會合。慕容寶到薊城時身邊的近衞已經散盡,只餘慕容隆的數百騎守。慕容會率眾於薊南迎接後,慕容寶削減慕容會的軍隊而分給慕容農及慕容隆,不久使率眾北歸龍城。當時慕容會整兵與慕容隆及慕容農的騎兵擊敗前來追擊的魏將石河頭,而其時慕容會的兵眾都不想歸屬於慕容農及慕容隆,於是向慕容寶提議讓慕容會率兵解中山之圍,然後還都中山。不過慕容寶拒絕,而慕容寶身邊的人則勸慕容寶殺掉慕容會,慕容寶亦感到慕容會謀反之心,意圖除去他,只因慕容農及慕容隆反對而作罷。慕容會恐懼,就派了仇尼歸襲擊慕容隆及慕容農,殺了慕容隆並重創慕容農。慕容會自宣稱二人謀逆,已經被殺,慕容寶一心要殺慕容會,於是出言讓他安心,接著就暗中命慕輿騰斬殺慕容會,但失敗。慕容會回到其軍中,接著進攻慕容寶,慕容寶就率數百騎直奔龍城。慕容寶及後拒絕慕容會誅除左右,立其為皇太子的要求,於是引來慕容會進攻龍城。慕容寶更在西門特意責罵慕容會,令慕容會下令士兵向慕容寶鼓譟揚威,藉此激起城中士兵憤怒。慕容寶軍於是在黃昏大敗慕容會,接著又派了高雲率敢死隊夜襲慕容會,再敗慕容會,令其逃奔中山。

永康三年(398年),慕容德派李延北上告知拓跋珪北歸的消息,慕容寶於是決意南征。慕容寶率兵至乙連時,長上段速骨、宋赤眉等人因為兵眾恐懼出征作亂,先逼高陽王慕容崇為主,殺害慕容宙及段誼等人。慕容寶與慕容農及慕輿騰會合,試圖討伐段速骨,但因為士兵厭戰,兵眾都潰散,慕容寶等人唯有奔還龍城。其時蘭汗暗中與段速骨勾結,將龍城軍隊帶到龍城以東,大大削弱了龍城的防禦,而慕容盛則內徙附近的人民,選取了一萬多個男丁守城。段速骨攻城時,慕容農因受蘭汗所誘,竟然叛歸段速骨。原本龍城守軍戰鬥力尚足以抵禦段速骨,令段速骨軍死傷甚大,但段速骨讓守軍看見慕容農後就瓦解了軍心,最終令龍城失守,慕容寶等人出走。

慕容寶到薊城後,在慕容盛等人反對下沒有回龍城,轉而想南投慕容德,可是在知道慕容德已稱燕王後就不敢繼續前進。當時慕容盛等人在冀州成功招集了一些支持慕容寶的力量,但其時蘭汗又派人來迎慕容寶,慕容寶以為蘭汗是忠臣,又想到蘭汗是父親慕容垂的舅舅,於是都不再懷疑,決意回龍城。慕容寶快到龍城時,蘭汗就派了弟蘭加難去迎接,但同時又命兄蘭堤封閉城門,最終蘭加難引慕容寶到龍城外邸並將之殺害,享年四十四歲。蘭汗殺太子慕容策及王公大臣,自稱大都督、大將軍、大單于,昌黎王。不久慕容盛殺蘭汗,改慕容寶諡號為惠愍皇帝,上廟號烈宗。

\subsubsection{永康}

\begin{longtable}{|>{\centering\scriptsize}m{2em}|>{\centering\scriptsize}m{1.3em}|>{\centering}m{8.8em}|}
  % \caption{秦王政}\
  \toprule
  \SimHei \normalsize 年数 & \SimHei \scriptsize 公元 & \SimHei 大事件 \tabularnewline
  % \midrule
  \endfirsthead
  \toprule
  \SimHei \normalsize 年数 & \SimHei \scriptsize 公元 & \SimHei 大事件 \tabularnewline
  \midrule
  \endhead
  \midrule
  元年 & 396 & \tabularnewline\hline
  二年 & 397 & \tabularnewline\hline
  三年 & 398 & \tabularnewline
  \bottomrule
\end{longtable}


%%% Local Variables:
%%% mode: latex
%%% TeX-engine: xetex
%%% TeX-master: "../../Main"
%%% End:

%% -*- coding: utf-8 -*-
%% Time-stamp: <Chen Wang: 2021-11-01 11:59:56>

\subsection{昭武帝慕容盛\tiny(398-401)}

\subsubsection{开封公慕容详生平}

慕容詳(?-397年),昌黎郡棘城县人(今辽宁省锦州市义县)人,追尊燕文明帝慕容皝的曾孙,后燕宗室,封开封公,後一度稱燕帝。

北魏君主、魏王拓跋珪率領魏軍圍攻后燕首都中山,後燕永康二年(397年),後燕不敵北魏的進攻,皇帝慕容寶等撤出都城中山(今中國河北省定州市),出逃龍城(今辽宁省朝阳市)。依然使用永康年號至398年。城內大亂,慕容詳當時不及跟隨撤退,因此被推為盟主以抵禦北魏的攻擊。然而,慕容詳為鞏固自己的地位,不斷翦除城內其他勢力。同年稍後不久,魏軍退卻后,慕容詳即皇帝位,改元建始。

由於慕容詳嗜酒好殺,不恤士民。七月,中山城民遂迎趙王慕容麟入城,慕容麟入城後,慕容詳被逮捕後處死。慕容麟自立,改元延平。

\subsubsection{趙王慕容麟生平}

慕容麟(4世纪-398年),昌黎郡棘城县(今辽宁省锦州市义县)人,後燕成武帝慕容垂之子,婢妾所生。惠愍帝慕容寶庶弟。原為後燕的趙王,後來一度稱燕帝。

早年慕容垂於前燕時期,叛前燕奔前秦時,慕容麟曾逃回前燕告發(369年)。其嫡長兄慕容令被前燕放逐後,欲偷襲龍城(今中國遼寧省遼陽縣),亦是被慕容麟告發,事敗身死(370年)。雖然慕容麟屢次出賣父兄,但後來前秦統一華北,慕容垂回到前燕故地時,還是不忍心殺掉慕容麟,最後是殺了慕容麟的母親頂罪,而把他放逐在外,很少見面。

383年,慕容垂於前秦淝水之戰敗後,陰謀背叛,慕容麟從中貢獻不少計策,慕容垂大為讚賞,待慕容麟開始與其他兒子相同。384年,慕容垂建後燕,慕容麟被任命為撫軍大將軍。同年,率軍攻陷中山(今中國河北省定州市),聲威大振,遂留守中山。386年,慕容垂稱帝後,慕容麟被封為趙王。其後數年,帶領燕軍南征北討,立下不少戰功。396年,慕容垂去世,太子慕容寶繼位,慕容麟被任命為尚書左僕射。

395年的參合陂之战及397年的柏肆之战,後燕二度慘敗給北魏,國力大衰。397年,北魏君主魏王拓跋珪率領魏軍進圍後燕都城中山,慕容麟謀叛,遂以武力威脅北地王慕容精,命其率領禁軍謀殺慕容寶,慕容精拒絕,慕容麟於是殺慕容精,逃出中山,依附丁零遺眾。不久,慕容寶等率領部下撤出中山,出逃龍城,依然使用永康年號至398年。城內大亂,開封公慕容詳被推為盟主以抵禦北魏的攻擊,後來魏軍退卻后,慕容詳即皇帝位,改元建始。但由於慕容詳嗜酒好殺,不恤士民,中山城民遂迎慕容麟入城,慕容麟入城後,殺慕容詳,亦稱帝,改元延平。但隨後北魏再攻中山,又被北魏擊敗,慕容麟自去年號,南奔鄴城(今中國河南省臨漳縣)投靠范陽王慕容德,並不再稱帝。

398年,慕容麟向慕容德上尊號,慕容德於是稱燕王,建立南燕,但不久慕容麟又陰謀推翻慕容德,因此被慕容德所殺。

\subsubsection{兰汗生平}

蘭汗(?-398年8月15日),昌黎郡棘城县(今辽宁省锦州市义县)人,慕容垂堂舅。

昌黎王蘭汗與段速骨密謀叛亂,后又杀段速骨,派兰加难诱杀慕容寶,改元青龍。夺位当年即为慕容盛所杀。蘭穆、蘭堤、蘭加難、蘭和、蘭揚也都被杀。慕容盛继位。

\subsubsection{昭武帝慕容盛生平}

燕昭武帝慕容盛(373年-401年9月13日),字道運,十六国后燕国主,慕容寶之庶長子。

年少時沈實敏銳,富謀略。當前秦天王苻堅誅殺慕容氏時,与叔父慕容柔潛逃投靠慕容沖。385年,慕容冲在阿房宫即皇帝位。慕容盛对慕容柔说:“夫十人之长,亦须才过九人,然后得安。今中山王才不逮人,功未有成,而骄汰已甚,殆难济乎!”慕容冲果然很快被杀,慕容柔、慕容盛以及慕容盛的弟弟慕容会又投靠慕容永。慕容盛指出三人是慕容垂子孙,正被世系疏远的慕容永猜疑,不如投奔祖父慕容垂。387年,他们从长子县一起逃回了后燕。不久慕容永果然尽杀治下的慕容儁、慕容垂子孙。

其父慕容宝登基后,慕容盛反对立祖父慕容垂所爱的庶弟慕容会为储,而支持嫡出的三弟慕容策。慕容会后来谋反被诛。

段速骨叛变后,慕容寶想南奔投靠叔叔慕容德,被慕容盛劝阻。慕容德已自称燕王,不但无意迎接慕容宝还意图谋害,慕容宝又回到龙城,受兰汗诓骗而遭殺害,慕容策亦遇害。慕容盛因為是蘭汗的女婿,与妻子關係甚篤,非但得以不死,還被蘭汗封做侍中。慕容盛乘机离间蘭汗、兰堤和兰加难三兄弟,派遣太原王慕容奇(兰汗外孙)在建安聚眾討伐蘭汗,当兰汗派出其兄长太尉兰堤讨伐时,慕容盛又反間蘭汗称慕容奇实力不足,兰堤才是慕容奇的幕后主谋。于是,兰汗将兰堤之职务转予抚军将军仇尼慕。如此种种,导致兰堤和兰加难兄弟生惧而背叛兰汗。太子兰穆提醒兰汗,慕容盛是仇家,必与慕容奇勾结,兰汗因而召见慕容盛,但慕容盛在妻兰王妃告密下佯病不出,躲过一劫。蘭汗派遣兄子蘭全反擊慕容奇卻反被滅。兰穆出兵讨伐兰堤、兰加难前大宴将士,兰汗父子喝得酩酊大醉,慕容盛与李旱等人趁機杀死兰穆,又引兵將兰汗乱刀砍死。又遣李旱及张真袭杀兰汗子鲁公兰和于令支及陈公兰扬于白狼,並捕杀兰堤和兰加难。

為父復仇之後,慕容盛一度想斬草除根殺死妻子蘭氏,母后丁氏不忍,進行勸阻,因此只廢黜蘭氏,终身未曾立后。慕容盛於398年8月19日(七月廿一辛亥)只改元建平,仍以长乐王称制,诸王皆降为公。又命慕容奇停止用兵,慕容奇抗命且带兵来攻,被慕容盛打败並赐死。11月12日(十月十七丙子)即皇帝位,並誅殺幽州刺史慕容豪、尚書左僕射張通及昌黎尹張順等人。399年改年號為長樂。400年2月11日(正月初一壬子)慕容盛自贬号为庶人天王。慕容盛後來討伐高丽及庫莫奚有功,然治法太嚴,刑罰殘忍,401年9月13日(八月二十壬辰),左將軍慕容國與殿中將軍秦輿、段贊等人密謀暗殺慕容盛,卻東窗事發,眾人皆被誅,軍中大亂。最終平亂時,慕容盛本人卻身中暗器,傷重不治,享年29歲,在位僅三年,慕容熙繼其位。

父親慕容寶。妻子蘭汗的女兒蘭氏。兒子慕容定在慕容盛被殺害後的時候還是年紀幼小。

慕容盛于401年闰八月十九葬于兴平陵(具体方位不详),庙号中宗,谥号昭武皇帝。

\subsubsection{建平}

\begin{longtable}{|>{\centering\scriptsize}m{2em}|>{\centering\scriptsize}m{1.3em}|>{\centering}m{8.8em}|}
  % \caption{秦王政}\
  \toprule
  \SimHei \normalsize 年数 & \SimHei \scriptsize 公元 & \SimHei 大事件 \tabularnewline
  % \midrule
  \endfirsthead
  \toprule
  \SimHei \normalsize 年数 & \SimHei \scriptsize 公元 & \SimHei 大事件 \tabularnewline
  \midrule
  \endhead
  \midrule
  元年 & 396 & \tabularnewline
  \bottomrule
\end{longtable}

\subsubsection{长乐}

\begin{longtable}{|>{\centering\scriptsize}m{2em}|>{\centering\scriptsize}m{1.3em}|>{\centering}m{8.8em}|}
  % \caption{秦王政}\
  \toprule
  \SimHei \normalsize 年数 & \SimHei \scriptsize 公元 & \SimHei 大事件 \tabularnewline
  % \midrule
  \endfirsthead
  \toprule
  \SimHei \normalsize 年数 & \SimHei \scriptsize 公元 & \SimHei 大事件 \tabularnewline
  \midrule
  \endhead
  \midrule
  元年 & 399 & \tabularnewline\hline
  二年 & 400 & \tabularnewline\hline
  三年 & 401 & \tabularnewline
  \bottomrule
\end{longtable}


%%% Local Variables:
%%% mode: latex
%%% TeX-engine: xetex
%%% TeX-master: "../../Main"
%%% End:

%% -*- coding: utf-8 -*-
%% Time-stamp: <Chen Wang: 2021-11-01 12:00:09>

\subsection{昭文帝慕容熙\tiny(401-407)}

\subsubsection{生平}

燕昭文帝慕容熙(385年-407年9月14日),字道文,一字長生,十六國時期後燕國君主,鮮卑人,成武帝慕容垂的幼子,惠愍帝慕容寶之弟,母親是貴嬪段氏。原封河間王,蘭汗之亂時曾被封為遼東公,慕容盛即位後,封河間公。

後燕長樂三年(401年),慕容盛被變軍殺害。慕容盛有兒子慕容定,年紀幼小。群臣希望慕容盛之弟慕容元繼位。但慕容熙因与慕容盛之母丁太后有私情,備受她寵愛,八月癸巳日(9月14日),遂被密迎入宮即天王位,慕容元被賜死,不久慕容熙改元光始。次年(402年),慕容熙又害死了慕容定,且娶了苻秦中山尹苻謨的兩個女兒苻娀娥為貴人、苻訓英為貴嬪,苻訓英極其受寵。丁太后怨恨,遂謀廢慕容熙,事洩,丁太后被殺。

慕容熙立苻訓英為皇后,苻娀娥為貴人,非常寵愛苻氏姐妹,因此興築宮殿、遊玩打獵,導致軍民死亡的數以萬計。苻娀娥生病,有人自稱能醫,結果醫死了,慕容熙遂將醫生支解後焚燒,追封苻娀娥為愍皇后。慕容熙與苻訓英更是玩樂不知節制。元始五年(405年)攻高句麗遼東城,原本城將攻陷,慕容熙只為了要與苻后一同坐輦車進城,因而命軍暫緩登城,以致延誤戰機,不能攻下遼東。次年(406年),後燕攻契丹未果而回師,又為了苻后想要觀戰而臨時拋棄輜重轉而偷襲高句麗,致士卒馬匹,疲累寒冷,沿路死亡不可勝數。又如苻后夏天想要吃凍魚,冬天要吃生地黃,官員也因不能取得而被斬首。

建始元年(407年),苻后去世,慕容熙痛不欲生,喪禮上命檢查百官有無哭泣,規定未哭者給予處罰,群臣只好口含辣物以刺激流淚。又賜死高陽王慕容隆的王妃張氏以殉葬,右仆射韦璆等人都害怕自己去殉葬,每天都洗澡换衣等候命令。此外規定家家戶戶都要參與建造苻后陵墓的工程,更使得國家財政揮霍一空。臨葬,慕容熙竟打開棺材,与苻訓英的屍體親熱一番,才准下葬。

由於早先中衛將軍馮跋與其弟馮素弗曾因事獲罪於後燕帝慕容熙,因此慕容熙一直有殺馮跋兄弟之意。七月甲子日(407年9月14日),馮跋兄弟於是趁慕容熙送葬苻后時起事,推高雲(慕容雲)為燕王,慕容熙被生擒後斬首,和苻訓英合葬。後來被諡昭文皇帝。

\subsubsection{光始}

\begin{longtable}{|>{\centering\scriptsize}m{2em}|>{\centering\scriptsize}m{1.3em}|>{\centering}m{8.8em}|}
  % \caption{秦王政}\
  \toprule
  \SimHei \normalsize 年数 & \SimHei \scriptsize 公元 & \SimHei 大事件 \tabularnewline
  % \midrule
  \endfirsthead
  \toprule
  \SimHei \normalsize 年数 & \SimHei \scriptsize 公元 & \SimHei 大事件 \tabularnewline
  \midrule
  \endhead
  \midrule
  元年 & 401 & \tabularnewline\hline
  二年 & 402 & \tabularnewline\hline
  三年 & 403 & \tabularnewline\hline
  四年 & 404 & \tabularnewline\hline
  五年 & 405 & \tabularnewline\hline
  六年 & 406 & \tabularnewline
  \bottomrule
\end{longtable}

\subsubsection{建始}

\begin{longtable}{|>{\centering\scriptsize}m{2em}|>{\centering\scriptsize}m{1.3em}|>{\centering}m{8.8em}|}
  % \caption{秦王政}\
  \toprule
  \SimHei \normalsize 年数 & \SimHei \scriptsize 公元 & \SimHei 大事件 \tabularnewline
  % \midrule
  \endfirsthead
  \toprule
  \SimHei \normalsize 年数 & \SimHei \scriptsize 公元 & \SimHei 大事件 \tabularnewline
  \midrule
  \endhead
  \midrule
  元年 & 407 & \tabularnewline
  \bottomrule
\end{longtable}


%%% Local Variables:
%%% mode: latex
%%% TeX-engine: xetex
%%% TeX-master: "../../Main"
%%% End:

%% -*- coding: utf-8 -*-
%% Time-stamp: <Chen Wang: 2021-11-01 12:00:17>

\subsection{惠懿帝高雲\tiny(401-407)}

\subsubsection{生平}

燕惠懿帝高雲(4世纪-409年11月6日),曾改名慕容雲,字子雨,高句驪人。十六国時期後燕末代君主,一說為北燕开国国主,称号天王。

早期的高雲於後燕時沉默寡言,並沒有什麼名氣,只有中衛將軍馮跋看出他的氣度與他結交。

後燕永康二年(397年),高雲因率軍擊敗慕容寶之子慕容會的叛軍,被慕容寶收養,賜姓慕容氏,封夕陽公。

後燕建初元年(407年)馮跋反,殺皇帝慕容熙,在馮跋支持之下,慕容雲即天王位,改元曰正始,國號大燕,恢復原本的高姓。高雲自知無功而登大位,因此培養一批禁衛保護自己,但後來反被禁衛離班和桃仁所殺,高雲死後被諡惠懿皇帝。

由於對高雲是否屬後燕慕容氏一族成員的看法不同,因此有人認為高雲是後燕末任君主,也有人把他視為北燕立國君主。

\subsubsection{正始}

\begin{longtable}{|>{\centering\scriptsize}m{2em}|>{\centering\scriptsize}m{1.3em}|>{\centering}m{8.8em}|}
  % \caption{秦王政}\
  \toprule
  \SimHei \normalsize 年数 & \SimHei \scriptsize 公元 & \SimHei 大事件 \tabularnewline
  % \midrule
  \endfirsthead
  \toprule
  \SimHei \normalsize 年数 & \SimHei \scriptsize 公元 & \SimHei 大事件 \tabularnewline
  \midrule
  \endhead
  \midrule
  元年 & 407 & \tabularnewline\hline
  二年 & 409 & \tabularnewline
  \bottomrule
\end{longtable}


%%% Local Variables:
%%% mode: latex
%%% TeX-engine: xetex
%%% TeX-master: "../../Main"
%%% End:



%%% Local Variables:
%%% mode: latex
%%% TeX-engine: xetex
%%% TeX-master: "../../Main"
%%% End:

%% -*- coding: utf-8 -*-
%% Time-stamp: <Chen Wang: 2019-12-19 15:37:04>


\section{西秦\tiny(385-431)}

\subsection{简介}

西秦(385年-400年,409年-431年)是中国历史上十六国时期鲜卑人乞伏國仁建立的政权。其国号“秦”以地处战国时秦国故地为名。《十六国春秋》始用西秦之称,以别于前秦和后秦,后世袭用之。

公元385年,鲜卑酋长乞伏国仁在陇西称大单于,又被前秦封为苑川王,都勇士川(今甘肃榆中)。388年,其弟乞伏乾歸立,称大单于,河南王,迁都金城(今甘肃兰州西)。400年為後秦所滅。409年,二月,乞伏乾归自后秦返回苑川。七月,西秦复国,复都苑川。412年,乞伏熾磐又迁都枹罕(今甘肃临夏市东北)。

最盛时期,其统治范围包括甘肃西南部,青海部分地区。

431年被夏國所灭。

%% -*- coding: utf-8 -*-
%% Time-stamp: <Chen Wang: 2019-12-19 15:40:51>

\subsection{乞伏国仁\tiny(385-388)}

\subsubsection{生平}

乞伏國仁(?-388年),陇西鲜卑人。十六国时期西秦政權奠定者。在前秦官至前將軍,淝水之戰後乘機自立,但仍與前秦保持一定關係。雖然一般認為乞伏國仁是西秦建立者,惟其在位期間,只受前秦封為苑川王,尚未正式稱秦王。一直至394年,國仁繼承人乞伏乾歸才稱秦王。

其父乞伏司繁受前秦天王苻堅封為南單于,並駐鎮勇士川(今甘肅榆中)。秦建元十二年(376年),司繁死,乞伏國仁繼位。前秦建元十九年(383年)淝水之戰時,苻堅原命國仁为前将军,领先锋骑,後國仁叔父乞伏步頹叛于陇西,苻堅派國仁回師討伐,步頹反而迎接國仁。及前秦淝水之戰失利,國仁即趁機吞併其他部族,聚眾共十多萬。前秦太安元年(385年),苻堅為姚萇所殺後,國仁自称大都督、大将军、大单于、领秦、河二州牧,改元建义,建都勇士城(今甘肅榆中)。

就在乞伏國仁自立次年,南安郡豪族祕宜就率領五萬羌、胡人進攻乞伏國仁,並四面來攻。乞伏國仁決意先聲奪人,於是自率五千人突襲祕宜,並大敗對方。祕宜於是逃奔南安郡,同年便率三萬多戶人口歸降。建義三年(387年),前秦皇帝苻登以乞伏國仁為大都督、都督雜夷諸軍事、大將軍、大單于、苑川王。同年乞伏國仁率軍進攻密貴、裕苟及提倫三位鮮卑大人,又大敗來攻的高平鮮卑首領沒弈干及東胡金熙,密貴等三人於是大懼,率部歸降。建義四年,乞伏國仁又擊敗了鮮卑人越質叱黎。同年国仁死,谥宣烈王,庙号烈祖,弟乞伏乾歸繼位。

\subsubsection{建义}

\begin{longtable}{|>{\centering\scriptsize}m{2em}|>{\centering\scriptsize}m{1.3em}|>{\centering}m{8.8em}|}
  % \caption{秦王政}\
  \toprule
  \SimHei \normalsize 年数 & \SimHei \scriptsize 公元 & \SimHei 大事件 \tabularnewline
  % \midrule
  \endfirsthead
  \toprule
  \SimHei \normalsize 年数 & \SimHei \scriptsize 公元 & \SimHei 大事件 \tabularnewline
  \midrule
  \endhead
  \midrule
  元年 & 385 & \tabularnewline\hline
  二年 & 386 & \tabularnewline\hline
  三年 & 387 & \tabularnewline\hline
  四年 & 388 & \tabularnewline
  \bottomrule
\end{longtable}


%%% Local Variables:
%%% mode: latex
%%% TeX-engine: xetex
%%% TeX-master: "../../Main"
%%% End:

%% -*- coding: utf-8 -*-
%% Time-stamp: <Chen Wang: 2019-12-19 15:41:44>

\subsection{武元王\tiny(388-412)}

\subsubsection{生平}

秦武元王乞伏乾歸(?-412年),陇西鲜卑人。十六国时期西秦開國君王,苑川王乞伏國仁弟。乾歸在位初期曾受前秦官爵,並曾響應前秦號召領兵協助,但皇帝苻登敗死後就逼逐繼承的苻崇,後苻崇討伐乾歸時更敗死,令前秦亡國,並乘機併吞其隴西土地,後稱「秦王」,西秦故此得名。後乾歸敗給後秦,被逼投降南涼,最終向後秦歸降,暫時亡國。但因後秦王姚興將其放回原地,並將部眾還給他,令其有機會復興,最終趁後秦漸漸衰弱時復國,並進攻鄰近的南涼、後秦、吐谷渾及其他胡人部落。乞伏乾歸於412年被侄兒乞伏公府所殺,其太子乞伏熾磐討平後繼位。

建义元年(385年)乞伏國仁自称大都督、大将军、单于,领秦、河二州牧。任命乾歸為上將軍。建義四年(388年)國仁去世,群臣認為國仁子乞伏公府年幼,乃推乾归为大都督、大将军、大单于、河南王,改元太初,遷都金城(今甘肅蘭州)。太初二年(389年)受前秦帝苻登封為金城王。

乾歸於太初二年(389年)即討平了休官部落的阿敦及侯年二部,盡降其眾,於是威振西部,鮮卑的豆留螱奇、叱豆渾、南丘鹿結、休官部的曷呼奴及盧水尉地跋都率眾歸降,而乾歸亦各署官爵;枹罕羌彭奚念亦來歸附,乾歸以其為北河州刺史。次年(390年),吐谷渾亦遣使上貢,乾歸又以吐谷渾君主視連為白蘭王、沙州牧。

太初四年(391年),沒弈干遣使結好,並派兩個兒子為人質請兵一共進攻鮮卑大兜,乾歸答允並領兵進攻大兜的安陽城,大兜退守鳴蟬堡但還是被乾歸攻陷,乾歸於是收擄其部眾回國。戰後乾歸歸還了沒弈干的兩個兒子,但沒弈干不久又改結劉衞辰,乾歸於是率兵一萬攻伐沒弈干,並在他樓城射傷沒弈干的眼睛。

太初七年(394年),苻登知後秦皇帝姚萇去世,認為滅後秦時機已到,於是起兵進攻後秦,又拜乾歸為左丞相、河南王、領秦梁益涼沙五州牧,加賜九錫。可是苻登卻遭姚萇太子姚興擊敗,退屯馬毛山,並派了兒子苻宗為質子,向乾歸請兵,並進封乾歸為梁王。乾歸於是派了乞伏益州率兵一萬營救,但苻登要出迎乞伏益州時被姚興擊敗,更被俘殺。苻登太子苻崇於湟中繼位,但不久乾歸就驅逐苻崇,苻崇只好投奔氐族仇池部隴西王楊定。二人組成聯軍反攻乾歸,乾歸派兵抵抗,終擊敗聯軍,斬楊定及苻崇,前秦滅亡,西秦自此盡有隴西。不久,乾歸自稱秦王,又於次年(395年)遷都苑川西城(今甘肅靖遠)。

早於太初五年(392年),呂光就曾派呂方及呂寶進攻乾歸,乾歸初敗於鳴雀峽,退屯青岸。而呂方屯黃河北,呂寶則渡河追擊,乾歸於是派彭奚念斷絕呂寶歸路,率兵反擊,屢敗呂寶,終呂寶等一萬多人戰死。至太初八年(395年),呂光親自率十萬軍進攻乾歸,左輔將軍密貴周及莫者羖羝就勸乾歸向呂光稱藩,乾歸終聽從並以兒子乞伏敕勃作為人質,呂光亦率軍退還。可是不久乾歸就後悔了,殺了密貴周及莫者羖羝。

太初九年(396年),涼州牧乞伏軻彈因與秦州牧乞伏益州不睦,故出奔呂光,呂光於是以乾歸多次反覆而興兵討伐。其時眾臣都請乾歸出奔成紀迴避,但乾歸不願。呂光派呂延等人攻下了臨洮、武始、河關,又命呂纂進攻金城,乾歸率兵救援,但呂光派了王寶及徐炅率兵五千邊擊,令乾歸恐懼不敢前進,終令金城陷落。乾歸於是行反間計,傳出假消息稱乾歸部眾已潰散,乾歸已東逃到成紀。呂延信以為真,於是輕軍進攻,最終被乾歸擊敗,呂延更戰死。呂延敗後,呂光亦退兵。

太初三年(390年),視連去世,視羆繼位,拒絕接受乾歸的封號。乾歸知道後大怒,但因為忌憚吐谷渾強盛,於是暫時容忍,仍然交好。至太初十一年(398年)就派了乞伏益州、慕兀及翟瑥率二萬騎進攻吐谷渾,在度周川大敗視羆,逼其送兒子宕豈為質求和。

太初十三年(400年),乾歸復遷都苑川(今甘肅榆中縣北)。同年,後秦姚碩德來攻,乾歸率眾到隴西對抗。兩軍對峙期間,姚碩德軍柴草缺乏,後秦王姚興於是親自出軍。乾歸見已是國家存亡的危機,於是放手一搏,決定集中力量消滅姚興軍隊,殺死姚興,欲求消除危機之餘更吞併後秦。乾歸因而命慕兀率二萬兵為中軍,駐柏楊(今甘肅清水縣西南);羅敦率四萬兵為外軍,駐侯辰谷。而自己就率數千騎等候姚興軍。但一晚,乾歸遇上大風和大霧,與中軍失去聯絡,被逼與外軍會合。天亮後乾歸就與姚興軍交戰,大敗。乾歸敗歸苑川,接著又逃到金城,並命手下各豪帥留下來歸降後秦,自己西走允吾(今甘肅皋蘭縣西北),望一天復興國家時再見。西秦滅亡。乾歸到允吾後向禿髮利鹿孤投降,被禿髮傉檀迎到晉興,待以上賓之禮。

後秦退兵後,南羌梁戈等人招引乾歸,乾歸打算前赴,但事情卻洩漏給禿髮利鹿孤知道,禿髮吐雷因而出屯捫天嶺。乾歸恐為禿髮利鹿孤所殺,於是送妻子及乞伏熾磐等諸子到西平為人質,自己出奔枹罕(今甘肅臨夏市),向後秦投降。

乾歸到長安後,受封為持节、都督河南诸军事、镇远将军、河州刺史、归义侯,隔年(401年)更被派還西秦故都苑川鎮守,並歸還其部眾。至後秦弘始四年(402年),乞伏熾磐逃奔後秦,姚興也授他官位,不久更加乾歸散騎常侍、左賢王。乾歸於降後秦時期,曾經受命與齊難等後秦將領到姑臧(今甘肅武威)接受後涼王呂隆投降。乾歸又屢攻仇池,先後攻破仇池所領的皮氏堡和西陽堡。乾歸更於405年攻破吐谷渾,其中吐谷渾君主大孩更在敗走後不久去世,乾歸俘擄了一萬多人。

弘始九年(407年),姚興認為乾归的勢力逐漸強大,難以控制,於是趁其入朝的機會將其留在長安當主客尚書,讓其子乞伏熾磐代領其眾。弘始十一年(409年),乞伏熾磐攻伐彭奚念,攻陷其佔領的枹罕。其時乾歸正隨姚興在平涼,得到熾磐的通報後就逃回苑川。乾歸回去後不久到枹罕聚集三萬部眾,並帶他們遷居度堅山,留熾磐守枹罕,接著乾歸更稱秦王,改元「更始」,再次置官爵並讓手下恢復原來在西秦的職位,正式復國。

乾歸復國後,先派兵進攻薄地延,將其部落遷至苑川,後又派兵攻下後秦的金城郡,並置守戍,從而於更始二年(410年)遷都回苑川。略陽、南安、隴西等後秦轄郡都先後遭西秦軍攻下。當時後秦無力討伐,只得任命乾歸為使持節、散騎常侍、都督隴西北匈奴雜胡諸軍事征西大將軍、河州牧、大單于、河南王。乾歸當時正欲攻取河西地區,於是暫時接受。

乾歸又派兵攻伐南涼,擊敗了南涼太子禿髮虎台。另又率兵攻下後秦略陽太守姚龍的柏龍堡及南平太守王憬的水洛城。後又攻殺襲據枹罕的彭利髮,收復了枹罕。更始四年(412年),乾歸更率二萬騎攻破吐谷渾支統阿若干,令吐谷渾向其投降。

同年六月,乾歸為其侄乞伏公府所弒,十余个儿子一并遇害。乞伏熾磐消滅乞伏公府後繼位,諡乾歸為武元王,庙號高祖,葬於枹罕。

\subsubsection{太初}

\begin{longtable}{|>{\centering\scriptsize}m{2em}|>{\centering\scriptsize}m{1.3em}|>{\centering}m{8.8em}|}
  % \caption{秦王政}\
  \toprule
  \SimHei \normalsize 年数 & \SimHei \scriptsize 公元 & \SimHei 大事件 \tabularnewline
  % \midrule
  \endfirsthead
  \toprule
  \SimHei \normalsize 年数 & \SimHei \scriptsize 公元 & \SimHei 大事件 \tabularnewline
  \midrule
  \endhead
  \midrule
  元年 & 388 & \tabularnewline\hline
  二年 & 389 & \tabularnewline\hline
  三年 & 390 & \tabularnewline\hline
  四年 & 391 & \tabularnewline\hline
  五年 & 392 & \tabularnewline\hline
  六年 & 393 & \tabularnewline\hline
  七年 & 394 & \tabularnewline\hline
  八年 & 395 & \tabularnewline\hline
  九年 & 396 & \tabularnewline\hline
  十年 & 397 & \tabularnewline\hline
  十一年 & 398 & \tabularnewline\hline
  十二年 & 399 & \tabularnewline\hline
  十三年 & 400 & \tabularnewline
  \bottomrule
\end{longtable}

\subsubsection{更始}

\begin{longtable}{|>{\centering\scriptsize}m{2em}|>{\centering\scriptsize}m{1.3em}|>{\centering}m{8.8em}|}
  % \caption{秦王政}\
  \toprule
  \SimHei \normalsize 年数 & \SimHei \scriptsize 公元 & \SimHei 大事件 \tabularnewline
  % \midrule
  \endfirsthead
  \toprule
  \SimHei \normalsize 年数 & \SimHei \scriptsize 公元 & \SimHei 大事件 \tabularnewline
  \midrule
  \endhead
  \midrule
  元年 & 409 & \tabularnewline\hline
  二年 & 410 & \tabularnewline\hline
  三年 & 411 & \tabularnewline\hline
  四年 & 412 & \tabularnewline
  \bottomrule
\end{longtable}


%%% Local Variables:
%%% mode: latex
%%% TeX-engine: xetex
%%% TeX-master: "../../Main"
%%% End:

%% -*- coding: utf-8 -*-
%% Time-stamp: <Chen Wang: 2021-11-01 12:02:19>

\subsection{文昭王乞伏熾磐\tiny(412-428)}

\subsubsection{生平}

文昭王乞伏熾磐(?-428年),十六国时期西秦国君主,乞伏乾歸長子。

熾磐個性勇略過人,400年,西秦第一次亡國後,被送往南涼為人質。後秦弘始四年(402年)熾磐自南涼奔後秦與乾歸會合。熾磐於後秦期間,召集軍隊據地自立。弘始十一年(409年)乾歸逃回西秦舊地,再稱秦王,西秦復國,熾磐又被立為太子。西秦更始四年(412年)乾歸為侄乞伏公府所弒,熾磐擒殺公府,繼位,稱河南王,改元永康。永康三年(414年)滅南涼,復稱秦王,其後主要與北涼爭戰。建弘九年(428年)病死,諡文昭王,廟號太祖。其子乞伏暮末继位。

\subsubsection{永康}

\begin{longtable}{|>{\centering\scriptsize}m{2em}|>{\centering\scriptsize}m{1.3em}|>{\centering}m{8.8em}|}
  % \caption{秦王政}\
  \toprule
  \SimHei \normalsize 年数 & \SimHei \scriptsize 公元 & \SimHei 大事件 \tabularnewline
  % \midrule
  \endfirsthead
  \toprule
  \SimHei \normalsize 年数 & \SimHei \scriptsize 公元 & \SimHei 大事件 \tabularnewline
  \midrule
  \endhead
  \midrule
  元年 & 412 & \tabularnewline\hline
  二年 & 413 & \tabularnewline\hline
  三年 & 414 & \tabularnewline\hline
  四年 & 415 & \tabularnewline\hline
  五年 & 416 & \tabularnewline\hline
  六年 & 417 & \tabularnewline\hline
  七年 & 418 & \tabularnewline\hline
  八年 & 419 & \tabularnewline
  \bottomrule
\end{longtable}

\subsubsection{建弘}

\begin{longtable}{|>{\centering\scriptsize}m{2em}|>{\centering\scriptsize}m{1.3em}|>{\centering}m{8.8em}|}
  % \caption{秦王政}\
  \toprule
  \SimHei \normalsize 年数 & \SimHei \scriptsize 公元 & \SimHei 大事件 \tabularnewline
  % \midrule
  \endfirsthead
  \toprule
  \SimHei \normalsize 年数 & \SimHei \scriptsize 公元 & \SimHei 大事件 \tabularnewline
  \midrule
  \endhead
  \midrule
  元年 & 420 & \tabularnewline\hline
  二年 & 421 & \tabularnewline\hline
  三年 & 422 & \tabularnewline\hline
  四年 & 423 & \tabularnewline\hline
  五年 & 424 & \tabularnewline\hline
  六年 & 425 & \tabularnewline\hline
  七年 & 426 & \tabularnewline\hline
  八年 & 427 & \tabularnewline\hline
  九年 & 428 & \tabularnewline
  \bottomrule
\end{longtable}


%%% Local Variables:
%%% mode: latex
%%% TeX-engine: xetex
%%% TeX-master: "../../Main"
%%% End:

%% -*- coding: utf-8 -*-
%% Time-stamp: <Chen Wang: 2019-12-19 15:45:01>

\subsection{乞伏暮末\tiny(428-431)}

\subsubsection{生平}

乞伏暮末(-431年),一名慕末,十六国时期西秦国君主,乞伏熾磐二子。

西秦建弘九年(428年)熾磐去世,暮末繼秦王位,改元永弘。暮末在位期間濫刑好殺,於是人心思叛。永弘三年(430年)因受北涼所迫,暮末擬歸附北魏,未料為夏國所阻。永弘四年(431年)夏國攻西秦都城南安,暮末出降,西秦亡。不久,暮末為夏國皇帝赫連定所殺。

\subsubsection{永弘}

\begin{longtable}{|>{\centering\scriptsize}m{2em}|>{\centering\scriptsize}m{1.3em}|>{\centering}m{8.8em}|}
  % \caption{秦王政}\
  \toprule
  \SimHei \normalsize 年数 & \SimHei \scriptsize 公元 & \SimHei 大事件 \tabularnewline
  % \midrule
  \endfirsthead
  \toprule
  \SimHei \normalsize 年数 & \SimHei \scriptsize 公元 & \SimHei 大事件 \tabularnewline
  \midrule
  \endhead
  \midrule
  元年 & 428 & \tabularnewline\hline
  二年 & 429 & \tabularnewline\hline
  三年 & 430 & \tabularnewline\hline
  四年 & 431 & \tabularnewline
  \bottomrule
\end{longtable}


%%% Local Variables:
%%% mode: latex
%%% TeX-engine: xetex
%%% TeX-master: "../../Main"
%%% End:


%%% Local Variables:
%%% mode: latex
%%% TeX-engine: xetex
%%% TeX-master: "../../Main"
%%% End:

%% -*- coding: utf-8 -*-
%% Time-stamp: <Chen Wang: 2019-12-19 15:47:32>


\section{后凉\tiny(389-403)}

\subsection{简介}

後凉(386年-403年)是十六国时期氐人贵族吕光建立的政权。

其国号以地处凉州为名。《十六国春秋》始称“后凉”,以别于其他以“凉”为国号的政权,后世袭用之。

東晋太元八年(383年)前秦将军吕光受命率7万餘众讨平西域。苻坚淝水兵败後前秦瓦解,吕光据有姑臧(今甘肃武威)于太元十一年(386年)称大将军、凉州牧。太元十四年(389年)吕光称三河王,後改称天王,史称後凉。

统治范围包括甘肃西部和宁夏、青海、新疆一部分。

後凉以氐人军事力量为基础,势力孤弱,刑法峻重,社会局势不稳,叛者连城。

後凉龙飞四年(399年)吕光卒,子吕绍继位,庶长子吕纂又杀吕绍自立。後凉咸宁三年(401年)吕隆(吕光弟吕宝之子)又杀吕纂自立,国势益衰。连年战争,经济凋敝,太元十二年(403年),穀价昂贵,人相食。

神鼎三年(403年),吕隆因后秦、南凉、北凉交相攻逼,降于後秦,後凉亡。

%% -*- coding: utf-8 -*-
%% Time-stamp: <Chen Wang: 2019-12-19 15:49:27>

\subsection{懿武帝\tiny(386-399)}

\subsubsection{生平}

涼懿武帝呂光(338年-399年),字世明,略陽(今甘肅天水)氐人,前秦太尉呂婆樓之子。十六國時期後涼建立者。呂光初為前秦將領,屢立戰功,前秦天王苻堅就派了他出兵西域。呂光降服西域,但當時前秦因淝水之戰戰敗而國亂,回軍時為涼州刺史梁熙所阻,呂光消滅了梁熙而入主涼州,遂在當地建立政權。

呂光得王猛看重,並將他推薦給苻堅,苻堅於是以呂光為美陽令,任內呂光得當地人民愛戴信服。呂光後遷鷹揚將軍,以功封關內侯,並於永興二年(358年)隨苻堅等討伐張平。苻堅與張平於銅壁決戰,張平驍勇大力的養子張蚝單騎屢次進出前秦軍陣中,呂光於是去襲擊張蚝並成功擊傷他。張蚝受傷被擒,張平潰敗,呂光亦因而聲名大噪。

建元四年(368年),呂光與王鑒等因應楊成世討伐上邽叛變的苻雙失敗而率軍再行討伐,王鑒到後打算與苻雙前鋒苟興速戰速決,但呂光慮及對方因剛獲勝而士氣高漲,建議謹慎待敵,讓其糧盡退兵時就是進攻的時機。二十日後苟興退兵,王鑒追擊並擊敗苟興,隨後又大敗苻雙,終攻下上邽,斬殺苻雙。建元六年(370年),呂光隨軍攻滅前燕,獲封都亭侯。後苻重出鎮洛陽,呂光擔任其長史。苻重於建元十四年(378年)謀反,苻堅以呂光忠誠正直,不會與苻重連謀,於是下令呂光收捕苻重,呂光聽命並以檻車押送苻重回長安。後呂光遷太子右率,頗受敬重。次年呂光又以破虜將軍身份率兵擊敗進攻成都的李烏,遷步兵校尉。建元十六年(380年)呂光又奉命與左將軍竇衝共領四萬兵討伐叛亂的苻重,又將其生擒,戰後獲授驍騎將軍。

前秦十八年(382年),呂光受命征討西域,以使持節都督西討諸軍事身份率領姜飛等將領、七萬兵及五千鐵騎出發。呂光越過三百多里長的沙漠到達西域,降服焉耆等西域各國,又擊破唯一拒守的龜茲,威震西域。苻堅知呂光征服西域,即任命其為使持節、散騎常侍、都督玉門以西諸軍事、安西將軍、西域校尉,封順鄉侯,但因前秦於淝水之戰後國內大亂而道路不通,未能傳達。呂光本來想要留在龜茲,但是受到名僧鳩摩羅什勸阻,而且部眾們也想回到中原,遂回師。

太安元年(385年),呂光軍抵宜禾(今新疆安西南),高昌太守楊翰告訴涼州刺史梁熙,稱呂光還軍必定別有所圖,建議關閉天險要道,拒之於外,但梁熙沒有聽從。呂光最初知道楊翰的計劃時曾打算不再前進,但在杜進勸告下還是繼續,楊翰即在呂光到達高昌時向呂光請降。梁熙在呂光到遠玉門時傳檄指責呂光擅自班師,又派其子梁胤等率軍五萬往酒泉阻擊呂光。呂光也傳檄指責梁熙沒有為前秦赴國難的忠誠,還阻攔歸國軍隊,並派了姜飛等為前鋒進攻梁胤。姜飛等在安彌大破梁胤並生擒他,於是周邊外族都紛紛依附呂光,武威太守彭濟更將梁熙抓起來叛歸呂光。呂光殺死梁熙,入主姑臧,自領涼州刺史、護羌校尉。

386年,呂光收到苻堅死訊,改元太安,並自稱使持節、侍中、中外大都督、督隴右河西諸軍事、大將軍、涼州牧、酒泉公。呂光入主涼州時,因尉祐與彭濟共謀抓住梁熙的功勞而寵任他,但呂光卻在尉祐中傷下殺了姚皓、尹景等十多個名士,人心見離。當時國內米價也高漲至一斗五百,饑荒中更發生人吃人事件,死了很多人。呂光與群僚在飲宴中談及為政時用嚴峻刑法的問題,在參軍段業勸言下終下令自省並行寬簡之政。

呂光於太安二年(387年)殺了進逼姑臧的張大豫,但王穆尚據酒泉;西平太守康寧也叛變,阻兵據守,呂光試圖討伐但都不果。及後連呂光部將徐炅及張掖太守彭晃都謀叛,並聯結了王穆及康寧。呂光力排眾議親率三萬兵速攻彭晃,二十日後攻破張掖,殺了彭晃。不久,呂光乘王穆進攻其將索嘏的機會率二萬兵襲破酒泉,王穆率兵東返但部眾在途中就潰散,王穆隻身逃走但為騂馬令郭文所殺。

389年,呂光稱三河王,改元麟嘉。396年六月又改稱天王,國號大涼,改元龍飛。呂光曾先後多次進攻西秦,其中呂光弟呂延於龍飛二年(397年)的進攻中兵敗被殺。呂光聽信讒言,怪罪從軍的尚書沮渠羅仇及三河太守沮渠麴粥,並殺二人。二人歸葬時,因諸部聯姻而共計有萬多人參與葬禮,羅仇之侄沮渠蒙遜遂反,蒙遜堂兄沮渠男成舉兵響應,並推建康太守段業為主,建北涼與後涼對抗,呂光曾派呂纂討伐,但最終無法消滅北涼。

同年,善於天文術數的太常郭黁與僕射王詳認為呂光年老、太子闇弱而呂纂等凶悍,料定呂光死後必會有禍亂,並禍及自己,故圖謀攻奪姑臧東西苑城,推王乞基為主。不過王詳因事泄而被殺,郭黁遂據東苑叛變,當時民間還有很多人支持郭黁。呂光召呂纂回兵討伐郭黁,呂纂遂屢破郭黁,令其於龍飛三年(398年)出走西秦,平定亂事。

龍飛四年(399年),呂光病重,立太子呂紹為天王,自號太上皇帝(太上天王)。呂光又讓呂纂及呂弘分任太尉及司徒,告誡呂紹要倚重二人,放權讓他們處理軍政大事才能保國家安穩;另也對呂纂及呂弘說二人要與天王呂紹同心合力才能保全國家,否則禍亂必會來。呂光於不久去世,享年六十三歲,諡懿武皇帝,廟號太祖。

呂光年輕時已展現其軍事能力,十歲時與其他小童一起玩耍時就創制戰爭陣法,於是同年的人都推其為主,而呂光處事平允,更令眾小童佩服。呂光也不喜歡讀書,只好打獵。

呂光高八尺四寸,雙目重瞳,為人沈著堅毅,凝重且寛大有度量,喜怒不形於色,故王猛賞識他,稱:「此非常人。」

呂光出生於枋頭(今河南浚縣西南),當夜有神光,全家覺得奇怪,遂以光为名。

呂光左肘有一肉印,據說在一次戰爭中肉印隱約顯出「巨霸」兩字。

\subsubsection{太安}

\begin{longtable}{|>{\centering\scriptsize}m{2em}|>{\centering\scriptsize}m{1.3em}|>{\centering}m{8.8em}|}
  % \caption{秦王政}\
  \toprule
  \SimHei \normalsize 年数 & \SimHei \scriptsize 公元 & \SimHei 大事件 \tabularnewline
  % \midrule
  \endfirsthead
  \toprule
  \SimHei \normalsize 年数 & \SimHei \scriptsize 公元 & \SimHei 大事件 \tabularnewline
  \midrule
  \endhead
  \midrule
  元年 & 386 & \tabularnewline\hline
  二年 & 387 & \tabularnewline\hline
  三年 & 388 & \tabularnewline\hline
  四年 & 389 & \tabularnewline
  \bottomrule
\end{longtable}

\subsubsection{麟嘉}

\begin{longtable}{|>{\centering\scriptsize}m{2em}|>{\centering\scriptsize}m{1.3em}|>{\centering}m{8.8em}|}
  % \caption{秦王政}\
  \toprule
  \SimHei \normalsize 年数 & \SimHei \scriptsize 公元 & \SimHei 大事件 \tabularnewline
  % \midrule
  \endfirsthead
  \toprule
  \SimHei \normalsize 年数 & \SimHei \scriptsize 公元 & \SimHei 大事件 \tabularnewline
  \midrule
  \endhead
  \midrule
  元年 & 389 & \tabularnewline\hline
  二年 & 390 & \tabularnewline\hline
  三年 & 391 & \tabularnewline\hline
  四年 & 392 & \tabularnewline\hline
  五年 & 393 & \tabularnewline\hline
  六年 & 394 & \tabularnewline\hline
  七年 & 395 & \tabularnewline\hline
  八年 & 396 & \tabularnewline
  \bottomrule
\end{longtable}

\subsubsection{龙飞}

\begin{longtable}{|>{\centering\scriptsize}m{2em}|>{\centering\scriptsize}m{1.3em}|>{\centering}m{8.8em}|}
  % \caption{秦王政}\
  \toprule
  \SimHei \normalsize 年数 & \SimHei \scriptsize 公元 & \SimHei 大事件 \tabularnewline
  % \midrule
  \endfirsthead
  \toprule
  \SimHei \normalsize 年数 & \SimHei \scriptsize 公元 & \SimHei 大事件 \tabularnewline
  \midrule
  \endhead
  \midrule
  元年 & 396 & \tabularnewline\hline
  二年 & 397 & \tabularnewline\hline
  三年 & 398 & \tabularnewline\hline
  四年 & 399 & \tabularnewline
  \bottomrule
\end{longtable}

%%% Local Variables:
%%% mode: latex
%%% TeX-engine: xetex
%%% TeX-master: "../../Main"
%%% End:

%% -*- coding: utf-8 -*-
%% Time-stamp: <Chen Wang: 2021-11-01 14:51:50>

\subsection{灵帝呂紹\tiny(399-401)}

\subsubsection{隐王生平}

涼隱王呂紹(380年代-399年),字永業,略陽(今甘肅天水)氐人。十六國時期後涼國第二任君主,後涼懿武帝呂光嫡子。呂紹登位不久即被呂纂及呂弘兩位兄長發動政變所推翻,呂紹自殺。

呂光出征西域時,呂紹與石氏等人留在前秦。淝水之戰後,前秦因戰敗而國亂,長安亦構亂,呂紹等人於是出奔仇池,直至麟嘉元年(389年)才到後涼,甫稱三河王的呂光遂立吕紹為世子。龍飛元年(396年),呂光立其為太子。

吕绍唯一一次有记载的亲自指挥的军事行动在龍飛四年(399年),当时他与庶兄吕纂攻打北凉天王段业,段业求助于南凉天王秃发乌孤。秃发乌孤的弟弟秃发利鹿孤率援军赶到,吕绍和吕纂只得撤退。

同年年末,呂光病重,立呂紹為天王,以吕纂为太尉,吕弘為司徒,臨終前叮囑呂紹說:「如今三寇(乞伏乾歸、段業和禿髮烏孤)未平,我死之後,呂纂帶領軍隊,呂弘治理朝政,你自己無為而治,把重任交給兩個哥哥」。也对两位长子有所嘱咐:“永业并非治理乱世的人才,只不过因嫡长的规举才让其处元首之位。现在外有强寇,人心不定,你们兄弟和睦则会让国家流传万世;若果自己内斗,则祸乱立即就会来了。”还对吕纂说:“你本性粗豪勇武,很令我担心。开展基业本来就艰难,守成也不容易。好好辅助永业,不要听谗言呀。”不久呂光去世,呂紹秘不發喪,呂纂推門入殿哭喪,竭盡哀思才出來。呂紹害怕被殺害,想讓位給他,但呂纂以呂紹是嫡子身份推辭,呂紹固請也不獲呂纂答允,於是即位。呂光侄子呂超勸呂紹及早除去既有兵權,又有極高威名的呂纂,但呂紹雖也憂心呂纂,但仍以父親遺命及袁尚兄弟相爭之事一再拒絕對付呂纂,令呂超很失望。呂紹在湛露堂面見呂纂時,呂超持刀在側侍候,用眼神請求呂紹收捕呂纂,但呂紹都不肯。

在呂紹到後涼前,呂光曾經想立呂弘為世子,不過因為知道呂紹在仇池而打消念頭。可是呂弘一直記恨在心,不久即派尚書姜紀唆使呂纂和他一起叛變。呂纂順從,於是在一夜率軍攻入宮廷,呂紹試圖出兵抵抗,但兵眾都因為忌憚呂纂威名而潰散。呂紹見此便在紫閣自殺。呂纂即位後諡呂紹為隱王。

\subsubsection{灵帝生平}

涼靈帝呂纂(4世紀?-401年),字永緒,略陽(今甘肅天水)氐人。十六國時期後涼國君主,後涼開國君主呂光庶長子,母親是趙淑媛,隱王呂紹兄。呂纂在呂光死後不久即以政變逼死呂紹登位,但在位一年多就在呂超等人的變亂被殺。

呂纂年少時已熟練弓馬,雖然入了太學,但不愛讀書,只會交結公侯。淝水之戰後前秦國亂,呂纂逃到上邽(今甘肅天水市),至太安元年(386年)才到達後涼都城姑臧(今甘肅武威市),拜虎賁中郎將。麟嘉四年(392年),呂光派了呂纂進攻南羌彭奚念,但在盤夷大敗而還。呂光遂親率大軍再攻,讓呂纂及楊軌、沮渠羅仇進軍左南(今青海西寧市東),逼得彭奚念憑湟河自守,然呂光還是派兵渡過湟河,攻下枹罕(今甘肅臨夏市),令彭奚念敗走甘松(今甘肅叠部縣東南)。

龍飛元年(396年),呂光稱天王,以呂纂為太原公。次年,呂光攻西秦,派呂纂、楊軌及竇苟等率三萬兵攻金城(今甘肅蘭州市),攻陷了金城。同年,呂光殺沮渠羅仇及沮渠麴粥,令得羅仇侄沮渠蒙遜反叛。蒙遜堂兄沮渠男成也推了建康太守段業為主,呂纂奉命討伐段業,然而因為沮渠蒙遜率眾到臨洮為聲援段業,呂纂在合離大敗給段業。同時,太常郭黁在姑臧作亂,呂光立即召回呂纂,當時諸將顧慮段業會乘大軍撤退而從後跟隨,建議乘夜暗中撤走,不過呂纂看准段業無謀略,乘夜退走只會助長敵人,於是在退兵時前派了使者向段業說:「郭黁作亂,吾今還都。卿能決者,可出戰。」段業果然不敢追擊。郭黁派軍於白石邀擊呂纂,呂纂大敗,但不久因西安太守石元良率兵援救才得以擊敗郭黁,攻入姑臧。呂纂隨後在城西擊破郭黁將王斐,令郭黁勢力開始衰敗。不過郭黁卻推了楊軌為盟主,讓楊軌前赴姑臧支援自己。時呂弘為段業所逼,呂纂就前去迎接呂弘,楊軌認為這是機會,於是率兵邀擊,但卻為呂纂所敗,郭黁於是出奔西秦,楊軌隨後亦奔廉川,亂事終告平定。

龍飛四年(399年),吕纂与吕绍一同统兵攻打北凉天王段业,段业求救于南凉天王秃发乌孤,秃发乌孤之弟秃发利鹿孤率援军赶到,段业坚守不战,吕纂、吕绍于是退兵。

同年,呂光病重,立呂紹為天王,以呂纂為太尉,掌握軍權。呂光死前曾向呂纂及呂弘說:「永業並非治理亂世的人才,只不過因嫡長的規舉才讓其處元首之位。現在外有強寇,人心不定,你們兄弟和睦則會讓國家流傳萬世;若果自己內鬥,則禍亂立即就會來了。」另也特別對呂纂說:「你本性粗豪勇武,很令我擔心。開展基業本來就艱難,守成也不容易。好好輔助永業,不要聽讒言呀。」不久呂光去世,呂紹懼怕呂纂,曾經想要讓位給呂纂,然而呂纂以嫡庶之別拒絕;另呂光侄呂超又勸呂紹殺了呂纂,但呂紹不肯。可是不久吕纂就在呂弘的煽動下反叛,夜裏率壯士數百進攻廣夏門,守融明觀的齊從抽劍攻擊呂纂,擊中其額,但為呂纂部眾制服。呂紹所派部隊因懼怕呂纂而潰散,吕紹被逼自殺。呂纂遂即天王位,改年號咸寧。

咸寧二年(400年),呂弘舉兵反叛,但為呂纂將焦辨擊敗,出奔廣武(今甘肅永登縣),不久為呂方所捕,被殺。呂纂隨後縱兵大掠,以原屬呂弘的東苑中之婦女賞給軍士,呂弘的妻兒都被士兵侵辱。呂纂笑著對群臣說:「今日一戰怎樣呀?」侍中房晷卻答:「天要降禍給涼室,故藩王起兵釁。先帝駕崩不久,隱王幽逼而死,山陵才剛建好,大司馬就因驚懼疑惑而反叛肆逆,京邑成了兄弟交戰的戰場。雖然呂弘自取滅亡,亦是因為陛下沒有棠棣所說的兄弟之義。現在應該反思自省,以為向百姓謝過,卻反而縱容士兵大肆掠奪,侮辱士女。兵釁因呂弘而起,百姓有甚麼錯!而且呂弘的妻子是陛下的弟婦,女兒也是陛下的姪女,怎能讓她們成為無賴小人的婢妾。天地神明怎會忍心見到這樣!」呂纂聽後向房晷道歉,又接回呂弘的妻兒到東宮。

隨後,呂纂不顧中書令楊穎反對堅決攻伐南涼,卻為南涼將禿髮傉檀所敗。呂纂不久又不聽姜紀諫言而攻北涼,圍攻張掖(今甘肅張掖)並攻略建康郡地,然而禿髮傉檀果如姜紀所言進攻姑臧,呂纂亦被逼退兵。呂纂在位時沉溺於酒色,又常常出獵,諸大臣皆曾勸阻,然而呂纂皆不能聽從。

咸寧三年(401年)呂纂因番禾太守呂超擅攻鮮卑思盤一事召呂超及思盤入朝,呂超因恐懼而事先結交了殿中監杜尚。呂纂憤怒地斥責呂超,更聲言「要當斬卿,然後天下可定」,嚇得呂超叩頭稱不敢。不過呂纂及後就和呂超及眾大臣宴會,呂超兄呂隆於是頻頻向呂纂勸酒要灌醉他。呂纂飲至昏醉便乘坐步輓車與呂超等人在宮內遊走,在到琨華殿東閤時步輓車過不了去,呂纂親將竇川及駱騰於是放下配劍推車。呂超乘此機會拿起二人配劍襲擊呂纂,呂纂試圖下車抓住呂超但被對方刺穿胸部;呂超又殺了竇川和駱騰。呂纂后楊氏下令禁軍討伐呂超,但杜尚卻命禁軍放下武器。將軍魏益多遂斬下呂纂的頭,聲言:「呂纂違反先帝遺命,殺害太子、沉溺飲酒和田獵、親近小人、輕易殺害忠良、視百姓為草芥。番禾太守呂超以骨肉之親,恐懼國家傾覆,已經除去他了。上可以安寧宗廟,下可為太子報仇。但凡國人都應歡慶。」

呂隆不久繼位,諡呂纂為靈皇帝,葬白石陵。

即序胡安據曾盜張駿的墓,獲得大量珍寶,呂纂誅殺安據和其親黨五十多家人,派使者弔祭張駿,並復修其陵墓。

咸寧二年,有母豬生下小豬,一身三頭,又有飛龍夜裡從東廂的井中出現,名僧鳩摩羅什以為不祥,勸纂廣施仁德。一日羅什與呂纂玩博戲,呂纂吃多子,玩笑道:“砍胡奴頭!”羅什糾正說:“不斫胡奴頭,胡奴斫人頭。”預言了呂纂因小字「胡奴」的呂超而被殺的命運。

\subsubsection{咸宁}

\begin{longtable}{|>{\centering\scriptsize}m{2em}|>{\centering\scriptsize}m{1.3em}|>{\centering}m{8.8em}|}
  % \caption{秦王政}\
  \toprule
  \SimHei \normalsize 年数 & \SimHei \scriptsize 公元 & \SimHei 大事件 \tabularnewline
  % \midrule
  \endfirsthead
  \toprule
  \SimHei \normalsize 年数 & \SimHei \scriptsize 公元 & \SimHei 大事件 \tabularnewline
  \midrule
  \endhead
  \midrule
  元年 & 399 & \tabularnewline\hline
  二年 & 400 & \tabularnewline\hline
  三年 & 401 & \tabularnewline
  \bottomrule
\end{longtable}


%%% Local Variables:
%%% mode: latex
%%% TeX-engine: xetex
%%% TeX-master: "../../Main"
%%% End:

%% -*- coding: utf-8 -*-
%% Time-stamp: <Chen Wang: 2019-12-19 15:51:37>

\subsection{吕隆\tiny(401-403)}

\subsubsection{生平}

呂隆(4世紀?-416年),字永基,略陽(今甘肅天水)氐人。十六國時期後涼最後一位君主,後涼開國君主呂光之弟呂寶子。呂隆即位不久即遭後秦攻擊,被逼向後秦請降,其在位時間亦不斷遭南涼及北涼二國攻擊,國力大衰,最終呂隆向後秦請求迎其東遷,後涼遂為後秦所併。

呂隆長得俊美,擅長騎射。呂光時曾任北部護軍。咸寧三年(401年),呂隆弟呂超以兵變弒殺天王呂纂,隨後就擁立呂隆。呂隆面有難色,但呂超說:「現在就好像騎著龍飛在天上,豈可以中途下來!」呂隆於是登位,改元神鼎。

呂隆登位後多殺豪望以圖立威,反不得人心,令人人自危。魏安人焦朗遂招請後秦將領姚碩德攻涼,姚碩德聽從並率軍進攻,兵臨姑臧。呂隆派了呂超及呂邈抵抗但大敗而還,呂邈更戰死,呂隆只得嬰城固守。不過,後秦軍接著數月的圍困令城中原來自東面的人圖謀叛變,將軍魏益多更煽動人們殺呂隆及呂超,呂隆遂在事件被揭發後誅殺共三百多家人。當時後涼群臣勸呂隆和後秦請和,呂隆原本不肯,但在呂超勸諫下向後秦請降。姚碩德於是表呂隆為鎮西大將軍、涼州刺史、建康公。

神鼎二年(402年),北涼沮渠蒙遜率兵進攻姑臧,呂隆請得南涼將禿髮傉檀援救,但傉檀未到呂隆就擊敗蒙遜。蒙遜於是與呂隆結盟,並留下萬多斛穀。但其時姑臧穀價已經高達五千文一斗,發生人吃人事件,死了十多萬人。百姓因為姑臧整天關上城門而無法出城找食物,於是每日都有數百人請求出城當別人奴婢以求生,呂隆怕他們會動搖人心,遂將這些人都盡數殺害,屍體堆滿路上。然而,接著後涼仍不斷受到北涼及南涼攻擊,呂隆被逼於神鼎三年(403年)借後秦徵呂超入侍的機會命其帶著珍寶,請後秦派兵迎其離開。秦將齊難等於該年八月到達姑臧,呂隆率眾隨之東遷長安,呂隆獲後秦授散騎常侍,後涼至此滅亡。後秦弘始十八年(416年),受後秦皇帝姚興子廣平公姚弼謀反案牽連,被殺。

\subsubsection{神鼎}

\begin{longtable}{|>{\centering\scriptsize}m{2em}|>{\centering\scriptsize}m{1.3em}|>{\centering}m{8.8em}|}
  % \caption{秦王政}\
  \toprule
  \SimHei \normalsize 年数 & \SimHei \scriptsize 公元 & \SimHei 大事件 \tabularnewline
  % \midrule
  \endfirsthead
  \toprule
  \SimHei \normalsize 年数 & \SimHei \scriptsize 公元 & \SimHei 大事件 \tabularnewline
  \midrule
  \endhead
  \midrule
  元年 & 401 & \tabularnewline\hline
  二年 & 402 & \tabularnewline\hline
  三年 & 403 & \tabularnewline
  \bottomrule
\end{longtable}


%%% Local Variables:
%%% mode: latex
%%% TeX-engine: xetex
%%% TeX-master: "../../Main"
%%% End:


%%% Local Variables:
%%% mode: latex
%%% TeX-engine: xetex
%%% TeX-master: "../../Main"
%%% End:

%% -*- coding: utf-8 -*-
%% Time-stamp: <Chen Wang: 2019-12-19 15:57:26>


\section{南凉\tiny(397-414)}

\subsection{简介}

南凉(397年-414年)是十六国时期河西鲜卑贵族秃发乌孤建立的政权。

历史
河西鲜卑秃发氏是塞北拓跋氏鲜卑之一支,汉魏时徙至河西,聚族而居。至十六国时,秃发乌孤继位,务农桑,修邻好,境内安定。

東晋隆安元年(397年)乌孤据廉川堡(今青海西宁)称西平王,后改称武威王,徙都乐都(今属青海),建立南凉政权。南凉太初三年(399年)秃发乌孤卒,弟利鹿孤继位,徙都西平(今青海西宁),后改称河西王。

建和三年(402年)利鹿孤卒,弟傉檀继位,回徙乐都,改称凉王。一度降附后秦,镇姑臧(今甘肃武威),后势力既强,又与后秦决裂。连年用兵,先败于北凉沮渠蒙逊,后又败于夏国赫连勃勃,名臣勇将损失十之六七,只好又迁回乐都。

南凉嘉平七年(414年)傉檀率兵袭青海乙弗部,西秦乞伏炽磐乘虚袭取乐都。傉檀降西秦,南凉亡。南凉共存在18年(397-414)。统治地区包括甘肃西部和青海一部分。

其国号源于所处为凉州故名。又其所处为凉州南部,也为区别其他国号为“凉”的政权,故史称“南凉”。又以其王室姓拓跋,又称拓跋凉。

%% -*- coding: utf-8 -*-
%% Time-stamp: <Chen Wang: 2019-12-19 16:03:57>

\subsection{武威武王\tiny(397-399)}

\subsubsection{生平}

武威武王\xpinyin*{禿髮烏孤}(4世紀-399年),河西鮮卑人,十六国时期南涼政權建立者。禿髮鮮卑首領,稱武威王,其父禿髮思復鞬亦為禿髮鮮卑族首領。

禿髮思復鞬死後,禿髮烏孤接任禿髮鮮卑首領,他勇猛威武,且有大志,並圖謀奪取時由後涼控制的涼州。禿髮烏孤於是致力發展農業,與鄰邦修好,禮待賢士,以積聚力量。後涼麟嘉六年(394年),涼王吕光封為冠军大将军、河西鲜卑大都统、广武县侯,其部屬石真若留認為當時禿髮烏孤的根基未穩,尚未是後涼的對手,建議禿髮烏孤暫時接受,等待機會。禿髮烏孤因而接受。

次年(395年)破乙弗、折掘二部,並自建廉川堡(今青海民和縣西北)作都城,此後再受後涼封為广武郡公。後涼龍飛元年(396年)呂光稱天王,又遣使署征南大将军、益州牧、左贤王,禿髮乌孤指呂光諸子貪淫,甥子暴虐,以不違百姓之心及不受不義爵位為由拒絕不受。及於次年正式叛後涼自立,自称大都督、大将军、大单于、西平王,改年号太初,並攻克後涼控制的金城(今甘肅蘭州市西北),後更於街亭(今甘肅秦安縣东北)擊敗前來討伐的後涼軍。次年(398年),後秦樂都、湟河、澆河三郡、嶺南羌胡數萬落及後涼將領楊軌、王乞基皆向禿髮烏孤歸降。禿髮烏孤於同年改稱武威王。太初三年(399年),烏孤遷都樂都(今青海樂都)。當時禿髮烏孤選任官員包括了胡人豪族、當地有德望之士、文武才俊、中原遷來的有才之士以及秦雍世族子弟,皆以其才授官。同年禿髮烏孤因酒後坠马伤及肋骨,傷重而死,死前向臣下表示應當立年長新君,故由其弟禿髮利鹿孤繼位。諡號為武王,廟號烈祖。

\subsubsection{太初}

\begin{longtable}{|>{\centering\scriptsize}m{2em}|>{\centering\scriptsize}m{1.3em}|>{\centering}m{8.8em}|}
  % \caption{秦王政}\
  \toprule
  \SimHei \normalsize 年数 & \SimHei \scriptsize 公元 & \SimHei 大事件 \tabularnewline
  % \midrule
  \endfirsthead
  \toprule
  \SimHei \normalsize 年数 & \SimHei \scriptsize 公元 & \SimHei 大事件 \tabularnewline
  \midrule
  \endhead
  \midrule
  元年 & 397 & \tabularnewline\hline
  二年 & 398 & \tabularnewline\hline
  三年 & 399 & \tabularnewline
  \bottomrule
\end{longtable}


%%% Local Variables:
%%% mode: latex
%%% TeX-engine: xetex
%%% TeX-master: "../../Main"
%%% End:

%% -*- coding: utf-8 -*-
%% Time-stamp: <Chen Wang: 2021-11-01 14:52:54>

\subsection{河西康王禿髮利鹿孤\tiny(399-402)}

\subsubsection{生平}

河西康王禿髮利鹿孤(4世紀?-402年),河西鮮卑人,十六國時期南涼國君主(河西王)。禿髮烏孤之弟。

太初三年(399年),禿髮烏孤遷都樂都(今青海樂都),並署利鹿孤為驃騎大將軍、西平公,駐鎮安夷(今青海平安區)。同年,後涼呂紹及呂纂進攻北涼,禿髮烏孤應北涼王段業求援,命利鹿孤及楊軌率軍救援。呂紹等人最終撤退,利鹿孤就以涼州牧改鎮西平(今青海湟中)。同年禿髮烏孤去世,利鹿孤繼位,就將都城遷至西平。

建和二年(401年)以祥瑞為由打算稱帝,但在安國將軍鍮勿崙的勸喻下改稱河西王。同年率軍攻伐後涼,大敗涼軍,俘獲楊桓及強遷其二千戶人口。建和三年(402年),利鹿孤又派兵攻破魏安,俘獲佔據當地的焦朗。同年利鹿孤去世,諡康王,葬於西平東南。因著利鹿孤父禿髮思復鞬向來疼愛並重視弟禿髮傉檀,而利鹿孤在位期間很多軍國大事都是由禿髮傉檀處理,故就以禿髮傉檀繼位。

\subsubsection{建和}

\begin{longtable}{|>{\centering\scriptsize}m{2em}|>{\centering\scriptsize}m{1.3em}|>{\centering}m{8.8em}|}
  % \caption{秦王政}\
  \toprule
  \SimHei \normalsize 年数 & \SimHei \scriptsize 公元 & \SimHei 大事件 \tabularnewline
  % \midrule
  \endfirsthead
  \toprule
  \SimHei \normalsize 年数 & \SimHei \scriptsize 公元 & \SimHei 大事件 \tabularnewline
  \midrule
  \endhead
  \midrule
  元年 & 400 & \tabularnewline\hline
  二年 & 401 & \tabularnewline\hline
  三年 & 402 & \tabularnewline
  \bottomrule
\end{longtable}


%%% Local Variables:
%%% mode: latex
%%% TeX-engine: xetex
%%% TeX-master: "../../Main"
%%% End:

%% -*- coding: utf-8 -*-
%% Time-stamp: <Chen Wang: 2021-11-01 14:53:04>

\subsection{景王禿髮傉檀\tiny(402-414)}

\subsubsection{生平}

涼景王禿髮傉檀(365年-415年),河西鮮卑人。十六國時期南涼國君主,他也是第一位正式稱「涼王」的君主。禿髮鮮卑首領禿髮思復鞬之子,南涼君主禿髮烏孤、利鹿孤之弟。

禿髮傉檀機警有才略。太初三年(399年),自稱武威王的禿髮烏孤移都樂都(今青海樂都),以傉檀為車騎大將軍、廣武公,鎮守西平(治今青海西寧),不久又改讓禿髮利鹿孤鎮守西平,召還傉檀錄府國事。同年去世,傳位予利鹿孤。

建和元年(400年),後涼王呂纂進攻南涼,利鹿孤以傉檀抵抗,傉檀在三堆(大通河以南,今甘肅永登縣境)擊敗後涼軍隊,殺二千多人。不久,呂纂又攻北涼王段業,傉檀聞訊就率一萬騎兵進襲後涼都城姑臧(今甘肅武威市)。當時呂纂弟呂緯據北城防禦,傉檀就置酒於姑臧南門朱明門,嗚鐘鼓,大宴將士並在東門青陽門展示兵力,終掠奪八千戶回去。呂纂知姑臧受襲,亦得退兵撤還。

建和二年(401年),利鹿孤稱河西王,以傉檀為都督中外諸軍事、涼州牧、錄尚書事。同年,後涼呂超攻擊據有魏安的焦朗。焦朗派了侄兒焦嵩為質向南涼求援,利鹿孤就是派傉檀率軍救援。但傉檀到後,呂超已撤退,焦朗卻閉門拒守。傉檀因而大怒,打算攻城,但為鎮北將軍俱延所諫止,於是改與焦朗連和,接著又到姑臧展示兵力,並在姑臧西的胡阬駐防。傉檀知道呂超必定會來攻,於是準備好火把。呂超隨後果然派了王集領二千精兵進攻傉檀,傉檀待王集闖進傉檀營壘中時命營壘內外將士都舉起燃著的火把,令營中十分光亮,接著就命軍隊進攻王集軍,絡終斬殺王集及殺三百多人。後涼王呂隆聞訊恐懼,於是假意與傉檀通和,並請他到苑內結盟。傉檀於是派了俱延去參加結盟,但遭呂超伏兵襲擊。傉檀因而大怒,進攻後涼昌松太守孟禕所駐的顯美(今甘肅永昌縣東南),呂隆雖派苟安國及石可救援,但二人都因表懼傉檀兵強而撤還。傉檀攻下顯美後生擒孟禕,初怪摃他不早早投降,但孟緯辯解說他只是盡了為後涼呂氏守衞疆土的職責,令傉檀改以禮待。接著傉檀遷二千多戶回國,想以孟禕為左司馬,又因孟禕表示想為國盡忠到最後,不欲失守城池反獲對方授予顯職而將其送還後涼。建和三年(402年),北涼沮渠蒙遜進攻後涼,後掠因而向利鹿孤求援,利鹿孤就派傉檀領兵一萬救援。傉檀到昌松時知沮渠蒙遜已退兵,就是遷涼澤、段冢五百多戶人回國。不久又受命進攻魏安的焦朗,逼令其出降。

傉檀父親禿髮思復鞬在傉檀年輕時就已喜愛他,更向其諸子說:「傉檀明識榦藝,非汝等輩也。」因此其兄禿髮烏孤以立長君為由命弟禿髮利鹿孤繼位,利鹿孤就在建和三年(402年)病逝前遺命傉檀繼位,兩兄皆傳弟不傳子,最終將君主之位傳傉檀。而其實利鹿孤在位時,軍國大事都交了給傉檀處理。傉檀繼位後,自稱涼王,改元弘昌,並把都城遷回樂都,並在次年正月大肆修築樂都城。後秦王姚興遣使拜傉檀為車騎將軍、廣武公。

傉檀繼位當年十月就率軍進攻後涼,至次年(403年),呂隆因不堪沮渠蒙遜及傉檀的接連進攻,認為再難固守姑臧,決定投歸後秦,向後秦請兵迎接。後秦王姚興於是派了齊難等領兵迎接,並吞併後涼領地,設置守宰。傉檀則攝昌松及魏安二戍作迴避。於傉檀進攻後涼時,其弟禿髮文真曾魏安攻擊後秦派往為後涼協防姑臧的王松怱軍,並俘擄王松怱。傉檀得知後大怒,送王松怱回長安並懇切地向後秦道歉。及至弘昌三年(404年)二月,傉檀更因畏懼後秦強大,自去年號,罷去尚書各官,並派參軍關尚出使後秦。姚興當時曾經以傉檀擅興戰事及大築城池而向關尚表示傉檀無為臣之道;關尚則答禿髮傉檀有羌人及沮渠蒙遜等強敵在附近,這些舉動都是為了守著後秦的門戶,希望姚興不要疑忌。姚興也對這答覆甚為滿意。後傉檀派禿髮文支大破南羌、西虜,接著就上表求姚興讓他領涼州,但被拒絕。後獲加官散騎常侍及增食邑二千戶,更於後秦弘始八年(406年)率兵進攻沮渠蒙遜。沮渠蒙遜當時嬰城固守,傉檀則割了其莊稼,攻至赤泉退兵。接著,傉檀又向後秦進獻三千匹馬及三萬頭羊。姚興至此認為傉檀是忠心的,於是以傉檀為使持節、都督河右諸軍事、車騎大將軍、領護匈奴中郎將、涼州刺史,鎮守姑臧,並召還涼州刺史王尚。傉檀終於得到涼州治權,但其時涼州人申屠英等派了主簿胡威力勸姚興不要召還王尚,放棄河西土地,終令姚興後悔,命車普阻止王尚離開,又派使者告知傉檀。傉檀率其三萬兵到姑臧南的五澗時遇上車普並得知情況,於是立即逼走王尚,還是得以成功入主涼州。原涼州別駕宗敞送王尚回去,傉檀一直都很欣賞他,而臨行前宗敞進薦了多位文武人材,亦得傉檀接納。同年八月,傉檀命禿髮文支留守姑臧,自回都城樂鄉,至十一月正式遷都至姑臧。而傉檀當時雖然是受後秦任命的官員,但車駕、服飾及禮儀都是國王格式。

及後,傉檀進襲西平、湟河各個羌人部落,並遷他們到武興、番禾、武威及昌松四郡。後又於弘始九年(407年)徵集士兵五萬多人,在方亭閱兵後就進攻沮渠蒙遜。沮渠蒙遜率兵迎擊,兩軍在均石(今甘肅張掖市東)交戰,傉檀戰敗。接著傉檀率二萬騎兵運四萬石穀到西郡,但蒙遜就進攻西郡治所日勒(今甘肅山丹縣東南),西郡太守楊統投降。

同年,夏國君主赫連勃勃因向傉檀求結姻親不遂,自率二萬兵進攻傉檀,進軍至支陽(今甘肅會寧縣)時已殺傷一萬多人,並掠二萬七千多人及數十萬頭牲畜回去。傉檀當時親自率兵追擊,焦朗認為赫連勃勃不可輕視,建議經溫圍水北渡黃河,奪萬斛堆(今寧夏中衞縣與甘肅靖遠縣交界),並阻水結營,扼其咽喉;不過將領賀連卻以為赫連勃勃只是烏合之眾,根本不需迴避其軍,應該快點追擊。傉檀聽從賀連所言但在陽武(今甘肅靖遠縣)遭赫連勃勃擊敗,更被追擊了八十多里,死傷數以萬計,損失了南涼六至七成的名臣勇將。傉檀自己就帶著數個騎兵逃至枝陽以南的南山,差點還被追兵抓住。此戰大敗後,傉檀恐懼外離侵逼,於是逼遷方圓三百里以內所有平民到姑臧城內,此舉令人民既驚且怨。故此屠各成七兒就乘著百姓混亂而起兵叛變,一夜之間部眾增至數千人。其時殿中都尉張猛勸說眾人,請其懸崖勒馬,竟成功令眾人散去,成七兒逃亡時間被殺。另一方面,軍諮祭酒梁裒及輔國司馬邊憲等共七人亦謀反,被傉檀誅殺。

弘始十年(408年),姚興見傉檀剛剛大敗給赫連勃勃,又接連發生內亂,想乘機內憂外患的時機消滅他,於是就派了姚弼、斂成及乞伏乾歸領兵三萬進攻傉檀。其時姚興也派了齊難進攻赫連勃勃,姚興因而寫信給傉檀,聲稱姚弼等軍只是用來截擊可能西逃的赫連勃勃。傉檀信以為真,沒有對姚弼軍設防。姚弼於是一直率大軍進攻,俘殺了昌松太守蘇霸並進攻至姑臧,屯兵西苑,傉檀只能嬰城固守。當時涼州人王鍾、宋鍾及王娥等人偷偷去為後秦做內應,但東窗事發,傉檀原本只想殺主事的幾個人,但終也接納伊力延侯的建議,將涉及事件的共五千人全部殺害,並將他們的妻女賞給將士。傉檀又下令郡縣都將牛羊放出城外,引誘了斂成出兵搶掠,傉檀將俱延及敬歸於是趁機進攻,大敗秦軍,殺了七千多人。姚弼此時只得堅守營壘,傉檀主動進攻,但未能攻下。七月,領二萬騎兵作為後援的姚顯還在高平(今甘肅固原),知姚弼進攻失敗,於是加速趕到姑臧。姚顯派了孟欽等五個神射手在涼風門挑戰,但箭還未射就被傉檀的材官將軍宋益擊殺。姚顯見無法取勝,唯有將罪責推給斂成,派使者向傉檀道歉,並在安撫河西人民引兵退還。傉檀亦派使者徐宿到後秦謝罪。可是,同年十一月,傉檀就再度稱涼王,並設年號「嘉平」,又設百官。

及後,傉檀與沮渠蒙遜互相攻伐,至嘉平三年(410年),傉檀又自率五萬騎進攻沮渠蒙遜,但在窮泉大敗,只得隻身騎馬逃歸姑臧;蒙遜更乘勝進攻姑臧。當時姑臧人仍想起兩年前傉檀大殺王鍾等五千人的事,都十分恐懼,於是漢、胡共一萬多戶人都向蒙遜投降。傉檀恐懼之下派了敬歸及敬佗父子作為人質,向蒙遜請和。蒙遜走時雖然敬歸逃回姑臧,但仍強遷八千多戶人。另一方面,右衞將軍折掘奇鎮據石驢山(今青海西寧北川西北)叛變。傉檀害怕沮渠蒙遜進逼,又怕南部領地被折掘奇鎮佔領,於是遷都回樂都,讓成公緖留守姑臧。可是傉檀甫出城,侯諶等人就閉門作亂,推了焦朗為主,向沮渠蒙遜投降。及後沮渠蒙遜於411年攻克姑臧。

沮渠蒙遜乘著取姑臧威勢,於是進攻傉檀,傉檀派將段苟及雲連出兵番禾襲其後方,遷了三千多戶到西平,但蒙遜依然進圍樂都。傉檀堅守三十日仍未失守,蒙遜就是派使者誘傉檀以寵愛的兒子作人質換取自己退兵,但遭傉檀拒絕。蒙遜憤怒之下決定建屋並進行耕作,預備持久圍困樂都。群臣於是請傉檀考慮蒙遜的條件,最終傉檀被逼以兒子禿髮安周為人質,蒙遜亦退兵。不久,傉檀不聽孟愷諫言進攻沮渠蒙遜,五路俱進,掠番禾、苕藋兩地共五千多戶人回國。當時將軍屈右顧慮蒙遜輕兵來襲,建議傉檀加快行軍,早早回到險要能守之地。不過傉檀聽伊力延所言,認為沮渠蒙遜的步兵趕不上傉檀的騎兵,且快速行軍會丟損戰利品,並非良策。可是一夜就遇上迷霧和風雨,沮渠蒙遜大軍趕到,又打得傉檀大敗。蒙遜再次圍攻樂都,傉檀唯有再以兒子禿髮染干為人質求和。

嘉平六年(413年),傉檀再攻蒙遜,在若厚塢兵敗,蒙遜於是又再圍攻樂都,攻了二十日未能攻克就退兵。但時為鎮南將軍、湟河太守的兒子禿髮文支卻向蒙遜投降。不久蒙遜再攻,傉檀只得以太尉俱延為質請和。

嘉平七年(414年),乙弗部落叛變,傉檀堅持進攻乙弗,當時孟愷以當時南涼國內連年糧食失收,而且南有乞伏熾磐,北有沮渠蒙遜這些大敵,都令百姓不安,認為這次遠征即使克捷,但也是後患無窮,建議與乞伏熾磐結盟,請其資給糧食以解厄困,並積聚實力,待合適時機才出兵。但傉檀並不聽信。於是傉檀親領七千騎大破乙弗部,奪得牛馬羊共四十多萬頭。不過,臨行前傉檀曾囑咐留守的太子禿髮虎台要小心的乞伏熾磐果然來攻,虎台試圖據守城池但遭熾磐四面攻擊,十日就已告失陷。

傉檀得知樂都陷落後,對部眾說希望借著從乙弗部奪取的物資攻取契汗部,並贖回眾人被乞伏熾磐俘擄的妻兒,否則投降乞伏熾磐就只成奴僕。接著傉檀就率眾西進,但很多部眾知樂都陷落都逃走了,連傉檀派去追回逃兵的段苟也逃了,於是傉檀部眾幾乎全部潰散。傉檀至此,唯有向乞伏熾磐投降。傉檀到西平時,乞伏熾磐遣使出城迎接,並以上賓之禮接待,又拜其為驃騎大將軍,封左南公,南涼亡。

一年多後,乞伏熾磐毒死傉檀,當時身邊的人都給傉檀找解藥,但傉檀卻說:「我的病哪該醫治呀!」於是中毒去世,享年五十一歲。其死後獲諡為景王。

\subsubsection{弘昌}

\begin{longtable}{|>{\centering\scriptsize}m{2em}|>{\centering\scriptsize}m{1.3em}|>{\centering}m{8.8em}|}
  % \caption{秦王政}\
  \toprule
  \SimHei \normalsize 年数 & \SimHei \scriptsize 公元 & \SimHei 大事件 \tabularnewline
  % \midrule
  \endfirsthead
  \toprule
  \SimHei \normalsize 年数 & \SimHei \scriptsize 公元 & \SimHei 大事件 \tabularnewline
  \midrule
  \endhead
  \midrule
  元年 & 402 & \tabularnewline\hline
  二年 & 403 & \tabularnewline\hline
  三年 & 404 & \tabularnewline
  \bottomrule
\end{longtable}

\subsubsection{嘉平}

\begin{longtable}{|>{\centering\scriptsize}m{2em}|>{\centering\scriptsize}m{1.3em}|>{\centering}m{8.8em}|}
  % \caption{秦王政}\
  \toprule
  \SimHei \normalsize 年数 & \SimHei \scriptsize 公元 & \SimHei 大事件 \tabularnewline
  % \midrule
  \endfirsthead
  \toprule
  \SimHei \normalsize 年数 & \SimHei \scriptsize 公元 & \SimHei 大事件 \tabularnewline
  \midrule
  \endhead
  \midrule
  元年 & 408 & \tabularnewline\hline
  二年 & 409 & \tabularnewline\hline
  三年 & 410 & \tabularnewline\hline
  四年 & 411 & \tabularnewline\hline
  五年 & 412 & \tabularnewline\hline
  六年 & 413 & \tabularnewline\hline
  七年 & 414 & \tabularnewline
  \bottomrule
\end{longtable}


%%% Local Variables:
%%% mode: latex
%%% TeX-engine: xetex
%%% TeX-master: "../../Main"
%%% End:



%%% Local Variables:
%%% mode: latex
%%% TeX-engine: xetex
%%% TeX-master: "../../Main"
%%% End:

%% -*- coding: utf-8 -*-
%% Time-stamp: <Chen Wang: 2019-12-19 16:23:37>


\section{南燕\tiny(398-405)}

\subsection{简介}

南燕(398年-410年)是中國南北朝時五胡十六国中,由鮮卑慕容部的慕容德所建立的國家,是慕容氏諸燕之一。国号燕。

慕容德原是後燕宗室范陽王。397年,當後燕君主慕容宝於參合陂之戰為北魏所敗之後,后燕被截成南北两部分。次年慕容德於滑台(今河南滑县)自稱燕王,拒绝接纳逃难的慕容宝,甚至险些将其杀害。公元400年迁广固(今山东青州西北)稱帝。南燕的国土,东到大海,南达泗上,西至巨野泽,北临黄河,共有十五个郡、八十二个县,约三十三万户,基本上就是原西晋的青州。统治范围包括今山东、河南、江苏各一部分。慕容德将南燕国土一分为五:青州,治所设在东莱(今山东莱州);幽州,治所设在发干(今山东沂水县西北);徐州,治所设在东莞(今山东莒县);兖州,治所设在梁父(今山东泰安南);并州,治所设在阴平(今江苏沭阳)。所以南燕官方在提到本国疆域时,常自称“五州之地”。

后主慕容超在位時,被東晋的劉裕擊敗,經歷兩代後滅國。

“南燕”之别称,始于当时人张诠所写《南燕书》(已佚),因相对于北燕位于南方故名。

%% -*- coding: utf-8 -*-
%% Time-stamp: <Chen Wang: 2019-12-19 16:24:33>

\subsection{献武帝\tiny(398-405)}

\subsubsection{生平}

燕献武帝慕容德(336年-405年11月17日),後改名慕容備德,字玄明,十六國時期南燕皇帝,鮮卑人,是前燕文明帝慕容皝之幼子,前燕景昭帝慕容儁、後燕成武帝慕容垂皆為其兄。《晉書》載其「年未弱冠,身長八尺二寸,姿貌雄偉」。又「博觀群書,性清慎,多才藝」。

前燕時期慕容儁在位時,慕容德被封為梁公。後來慕容儁之子慕容暐繼帝位,再被改封為范陽王。369年,曾與慕容垂一同大敗東晉桓溫的進攻。370年,前燕為前秦所滅後,一度被前秦帝苻堅任命為張掖(今中國甘肅省張掖市)太守,數年後被免職。

後來苻堅欲南征東晉,慕容德被任命為奮威將軍隨軍,留下金刀拜別母親公孫氏及胞兄原北海王慕容納而去。383年,前秦於淝水之战敗北,苻坚宠妃张夫人走失投靠慕容暐,慕容暐送她回京,慕容德阻止并劝他趁乱复国,未果。

后来慕容垂趁機起兵建後燕,慕容德嚮應之,被命為車騎大將軍,重新受封為范陽王,然其諸子及慕容纳皆因留在後方而被前秦所殺。

396年,慕容垂臨終,遺命太子慕容寶將鄴城(今中國河南省臨漳縣)委由慕容德鎮守。慕容寶繼位後,即以慕容德為使持節、都督冀、兗、青、徐、荊、豫六州諸軍事、特進、車騎大將軍、冀州牧,領南蠻校尉,鎮守鄴城。

397年,北魏攻擊後燕,後燕兵敗如山倒,皇帝慕容寶向北方故地逃亡,後燕國土被截為南北二部,位在南方的慕容德被慕容寶任命為丞相,領冀州牧。不久,慕容垂另一子趙王慕容麟來逃至鄴城,以鄴城難守,建議慕容德南遷滑台(今中國河南省滑縣)。398年正月,慕容德又受慕容麟建議先稱燕王,稱燕王元年,史稱此一政權為南燕。次年(399年),再遷廣固(今中國山東省青州市),以為都城。

400年,慕容德正式稱帝,改元建平,並在此時把自己名字改為慕容備德,以便臣民避諱。慕容德正式登基時年已65歲,在中國歷史上僅次於唐朝的武則天,武則天登基時已經67歲了。

慕容备德不知道母亲公孙夫人和胞兄慕容纳都已经不在人世,曾于建平二年(401年)十月派平原人杜弘去长安寻访。杜弘说:“臣至长安,若不能得知太后动止,当西往张掖,以死效力。”并为自己年逾六十的父亲杜雄乞求本县县令之职。慕容备德不顾中书令张华反对,认为杜弘“为君迎母,为父求禄,忠孝备矣,何罪之有!”以杜雄为平原令。杜弘到张掖为盗贼所杀。四年(403年),慕容备德旧部赵融从长安前来,告知公孙夫人和慕容纳的死讯,慕容备德放声痛哭以至于吐血,因而卧病不起,从此健康恶化。

慕容备德有女兒無兒子,為繼承人心焦,慕容納之子慕容超持當年慕容德拜別母親的金刀來歸,慕容备德遂以慕容超袭封北海王,后立為太子。

建平六年九月戊午(405年11月17日),慕容备德去世,慕容超繼位。去世當晚從四方城門抬出十餘口棺木,秘密埋葬在山谷之中,因此實際上他並未葬於其陵寢「東陽陵」,後人遂不知其安葬之處。慕容德後來被諡為獻武皇帝,廟號世宗。

\subsubsection{燕平}

\begin{longtable}{|>{\centering\scriptsize}m{2em}|>{\centering\scriptsize}m{1.3em}|>{\centering}m{8.8em}|}
  % \caption{秦王政}\
  \toprule
  \SimHei \normalsize 年数 & \SimHei \scriptsize 公元 & \SimHei 大事件 \tabularnewline
  % \midrule
  \endfirsthead
  \toprule
  \SimHei \normalsize 年数 & \SimHei \scriptsize 公元 & \SimHei 大事件 \tabularnewline
  \midrule
  \endhead
  \midrule
  元年 & 398 & \tabularnewline\hline
  二年 & 399 & \tabularnewline
  \bottomrule
\end{longtable}

\subsubsection{建平}

\begin{longtable}{|>{\centering\scriptsize}m{2em}|>{\centering\scriptsize}m{1.3em}|>{\centering}m{8.8em}|}
  % \caption{秦王政}\
  \toprule
  \SimHei \normalsize 年数 & \SimHei \scriptsize 公元 & \SimHei 大事件 \tabularnewline
  % \midrule
  \endfirsthead
  \toprule
  \SimHei \normalsize 年数 & \SimHei \scriptsize 公元 & \SimHei 大事件 \tabularnewline
  \midrule
  \endhead
  \midrule
  元年 & 400 & \tabularnewline\hline
  二年 & 401 & \tabularnewline\hline
  三年 & 402 & \tabularnewline\hline
  四年 & 403 & \tabularnewline\hline
  五年 & 404 & \tabularnewline\hline
  六年 & 405 & \tabularnewline
  \bottomrule
\end{longtable}


%%% Local Variables:
%%% mode: latex
%%% TeX-engine: xetex
%%% TeX-master: "../../Main"
%%% End:

%% -*- coding: utf-8 -*-
%% Time-stamp: <Chen Wang: 2019-12-19 16:25:10>

\subsection{慕容超\tiny(405-410)}

\subsubsection{生平}

燕末主慕容超(385年-410年),字祖明,十六國南燕末代皇帝,鮮卑人。

前秦建元六年(370年),前燕為前秦所滅後,慕容超之父慕容納一度仕於前秦,後來遷居於張掖(今中國甘肅省張掖市)。慕容納之弟慕容德受前秦帝苻堅之命隨軍南征東晉,留下金刀拜別母親公孫氏而去。

前秦建元十九年(383年),前秦於肥水之戰敗北,慕容納、德之兄慕容垂趁機起兵建後燕,前秦遂殺慕容納本人及慕容德諸子。公孫氏因年老而免死,慕容納之妻段氏正好懷孕,暫不執行死刑,羈押於獄中。有個叫呼延平的獄卒,是慕容德以前的下屬,慕容德曾免其死罪,對其有恩。因此,他幫助公孫氏及段氏逃至羌地,慕容超即於該處誕生。

慕容超十歲時,祖母公孫氏去世,臨終前把金刀給慕容超,說:「如果天下太平,你能夠向東回到故土,可以將這把刀還給你叔叔(慕容德)。」呼延平後來又讓慕容超母子逃亡到呂光在位時的後涼。後來的後涼王呂隆向後秦姚興投降,慕容超母子又被遷往長安(今中國陝西省西安市)。呼延平去世後,慕容超之母段氏讓慕容超娶呼延平之女。

慕容超認為幾位伯叔父先後在東方稱帝,恐怕被後秦知道身分,所以就裝成神智失常之人,並以行乞維生。後秦人都看不起他,遂對他不起疑,所以行動自由不受限制。當時已登上南燕帝位的慕容德聽說這件事,立即派使者迎接他,慕容超不告別母親、妻子即東行。後來到達南燕,呈獻金刀給慕容德,並告以其祖母也就是慕容德之母臨終的遺言,慕容德聽了之後哀傷不已。將慕容超封為北海王(即慕容纳在前燕的王爵),任命為侍中、驃騎大將軍、司隸校尉,開王府置僚佐。

史載「慕容超身高八尺,腰帶九圍,姿器魁傑」,和慕容德頗為相似,而且「精彩秀髮,容止可觀」,《晉書》和《十六國春秋》皆載此時他才被取名為慕容超。慕容德由於年輕時生的兒子已經在前秦被殺害,晚年只有女兒沒有兒子,所以動了讓慕容超繼承之心。而慕容超亦深知慕容德的意思,因此「入則盡歡承奉,出則傾身下士」,於是輿論一致稱讚,不久即被立為太子。

南燕建平六年九月戊午(405年11月17日),慕容德去世,九月己未(11月18日),慕容超即皇帝位,改元太上。慕容超登位後,寵信舊部公孫五樓,聽信其言,大殺功臣,時稱“欲得侯,事五樓”。又喜好遊獵,使得人民苦不堪言。他的嬸母、皇太后段季妃等密謀廢掉他立慕容鐘,事發,慕容超殺了相關諸臣,廢黜了段季妃。

太上三年(407年),因母段氏、妻呼延氏尚留在後秦,遂向後秦稱藩,後秦就將其母、妻送還。慕容超追尊其父慕容納為穆皇帝,立其母為皇太后,妻為皇后。

南燕向後秦稱藩後,慕容超即計畫南下攻擊淮北,使得東晉不堪其擾。太上五年(409年),東晉將領劉裕率軍進攻南燕反擊。次年二月丁亥日(410年3月25日),南燕都城廣固(今中國山東省青州市)陷落,慕容超被俘,被送往東晉都城建康(今中國江蘇省南京市)斬首。死後無諡號及廟號,有史家稱他為南燕末主。

慕容超同時也是除了系出同源的吐谷渾外,五胡十六國時期源自鲜卑慕容部的最後一位帝王。

\subsubsection{太上}

\begin{longtable}{|>{\centering\scriptsize}m{2em}|>{\centering\scriptsize}m{1.3em}|>{\centering}m{8.8em}|}
  % \caption{秦王政}\
  \toprule
  \SimHei \normalsize 年数 & \SimHei \scriptsize 公元 & \SimHei 大事件 \tabularnewline
  % \midrule
  \endfirsthead
  \toprule
  \SimHei \normalsize 年数 & \SimHei \scriptsize 公元 & \SimHei 大事件 \tabularnewline
  \midrule
  \endhead
  \midrule
  元年 & 405 & \tabularnewline\hline
  二年 & 406 & \tabularnewline\hline
  三年 & 407 & \tabularnewline\hline
  四年 & 408 & \tabularnewline\hline
  五年 & 409 & \tabularnewline\hline
  六年 & 410 & \tabularnewline
  \bottomrule
\end{longtable}


%%% Local Variables:
%%% mode: latex
%%% TeX-engine: xetex
%%% TeX-master: "../../Main"
%%% End:



%%% Local Variables:
%%% mode: latex
%%% TeX-engine: xetex
%%% TeX-master: "../../Main"
%%% End:

%% -*- coding: utf-8 -*-
%% Time-stamp: <Chen Wang: 2019-12-19 16:26:47>


\section{西凉\tiny(400-417)}

\subsection{简介}

西涼(400年—421年)是十六國之一。

400年李暠在敦煌郡称“凉公”。405年遷都酒泉郡,逼近北涼。疆域在今中國甘肅西部及新疆部分。417年,李暠卒,子李歆嗣位。420年,李歆與北涼交戰被殺,其弟敦煌太守李恂在敦煌嗣位。但次年,北涼軍圍敦煌,李恂戰敗,乞降不成後自殺。西涼因此亡於北涼。二十余年之后,李恂的侄子李宝趁北魏攻灭北凉之际,一度恢复先人的基业,同年向北魏投诚,该政权被称为后西凉。

因其统治地区古为凉州,故国号为“凉”,又位于凉州西部,故名“西凉”。

%% -*- coding: utf-8 -*-
%% Time-stamp: <Chen Wang: 2019-12-19 16:27:48>

\subsection{武昭王\tiny(400-417)}

\subsubsection{生平}

涼武昭王李\xpinyin*{暠}(351年-417年),字玄盛,小字長生,陇西郡狄道县(今甘肃省定西市临洮县)人,是李昶的遺腹子,十六國時期西涼的建立者。自稱是西漢將領李廣之十六世孫。李暠的后代形成了陇西李氏的镇远将军房、平凉房、武阳房、姑臧房、敦煌房、仆射房和绛郡房,唐朝皇室和诗人李白亦稱李暠為其先祖。天宝二年(743),唐玄宗追尊李暠为兴圣皇帝。

李暠年少好學,性格寬和,讀遍經史,尤能理解文章的義理。李暠長大後也學習武藝,讀孫吳兵法。北涼神璽元年(397年),後涼建康太守段業自立,次年孟敏降北涼獲授沙州刺史,以李暠為效穀縣令。不久孟敏去世,敦煌护军郭谦及沙洲治中索仙認為李暠在任縣令期間治績頗可取,故推舉李暠为敦煌太守及甯朔將軍,李暠遂向段業請命,獲授安西將軍、敦煌太守,領護西胡校尉。後來北涼右衞將軍索嗣向段業中傷李暠,讓其改以自己擔當敦煌太守。索嗣率五百騎到敦煌外二十里時才通告李暠去迎接自己,李暠聞訊驚訝疑惑,一度想順從出迎,但在張邈及宋繇勸阻下改為派兵抵抗,索嗣兵敗退還張掖。李暠昔日與索嗣十分友好,但知道他在段業面前中傷他並奪去其官位後就相當痛恨他,遂上陳索嗣罪狀。在沮渠男成的勸說下,段業就將索嗣殺了,並派使者向李暠陳謝,又分劃出涼興郡,進李暠為持節,都督涼興以西諸軍事、鎮西將軍,領西夷校尉。

北涼天璽二年(400年),晉昌太守唐瑤移檄六郡,推李暠為大都督、大將軍、護羌校尉、領秦涼二州牧、涼公,改元庚子,以敦煌為都城,建立西涼。李暠又派兵東伐涼興,又西攻玉門西諸城,令疆域廣及西域。次年,北涼將沮渠蒙遜殺段業,自立為北涼君主,李暠又派唐瑤攻酒泉,擒北涼酒泉太守沮渠益生。建初元年(405年),李暠改元並遣使奉表於晉,又遷都酒泉,與北涼長期爭戰。

李暠立國以後鼓勵農事,為對抗北涼積聚軍資,亦令百姓安居樂業。他亦喜好讀書,因此在位時注重文化教育,境內文風頗盛。

建初十三年(417年),李暠過世,享年六十七歲,臨終遺命宋繇輔助諸子。西涼諡武昭王,廟號太祖,次子李歆繼位。

\subsubsection{庚子}

\begin{longtable}{|>{\centering\scriptsize}m{2em}|>{\centering\scriptsize}m{1.3em}|>{\centering}m{8.8em}|}
  % \caption{秦王政}\
  \toprule
  \SimHei \normalsize 年数 & \SimHei \scriptsize 公元 & \SimHei 大事件 \tabularnewline
  % \midrule
  \endfirsthead
  \toprule
  \SimHei \normalsize 年数 & \SimHei \scriptsize 公元 & \SimHei 大事件 \tabularnewline
  \midrule
  \endhead
  \midrule
  元年 & 400 & \tabularnewline\hline
  二年 & 401 & \tabularnewline\hline
  三年 & 402 & \tabularnewline\hline
  四年 & 403 & \tabularnewline\hline
  五年 & 404 & \tabularnewline
  \bottomrule
\end{longtable}

\subsubsection{建初}

\begin{longtable}{|>{\centering\scriptsize}m{2em}|>{\centering\scriptsize}m{1.3em}|>{\centering}m{8.8em}|}
  % \caption{秦王政}\
  \toprule
  \SimHei \normalsize 年数 & \SimHei \scriptsize 公元 & \SimHei 大事件 \tabularnewline
  % \midrule
  \endfirsthead
  \toprule
  \SimHei \normalsize 年数 & \SimHei \scriptsize 公元 & \SimHei 大事件 \tabularnewline
  \midrule
  \endhead
  \midrule
  元年 & 405 & \tabularnewline\hline
  二年 & 406 & \tabularnewline\hline
  三年 & 407 & \tabularnewline\hline
  四年 & 408 & \tabularnewline\hline
  五年 & 409 & \tabularnewline\hline
  六年 & 410 & \tabularnewline\hline
  七年 & 411 & \tabularnewline\hline
  八年 & 412 & \tabularnewline\hline
  九年 & 413 & \tabularnewline\hline
  十年 & 414 & \tabularnewline\hline
  十一年 & 415 & \tabularnewline\hline
  十二年 & 416 & \tabularnewline\hline
  十三年 & 417 & \tabularnewline
  \bottomrule
\end{longtable}


%%% Local Variables:
%%% mode: latex
%%% TeX-engine: xetex
%%% TeX-master: "../../Main"
%%% End:

%% -*- coding: utf-8 -*-
%% Time-stamp: <Chen Wang: 2019-12-19 16:28:26>

\subsection{李歆\tiny(417-420)}

\subsubsection{生平}

李\xpinyin*{歆}(?-420年),字士業,小字桐椎,陇西狄道(今甘肃临洮县)人,十六國西涼公,為李暠世子。其第三子李重耳是李唐王朝皇室的直系祖先。

西涼建初十三年(417年),李暠過世,李歆被部下擁護為大都督、大將軍、涼公、涼州牧,改元嘉興。李歆在位時,繼承其父稱臣於東晉的政策,因此東晉封其為酒泉公。

李歆用刑頗嚴,又喜歡建築宮殿,臣屬多有勸諫,然而李歆並不能接納。嘉興四年(420年),北涼佯攻西秦以誘西涼,李歆因此出兵攻擊,戰敗被殺。

\subsubsection{建兴}

\begin{longtable}{|>{\centering\scriptsize}m{2em}|>{\centering\scriptsize}m{1.3em}|>{\centering}m{8.8em}|}
  % \caption{秦王政}\
  \toprule
  \SimHei \normalsize 年数 & \SimHei \scriptsize 公元 & \SimHei 大事件 \tabularnewline
  % \midrule
  \endfirsthead
  \toprule
  \SimHei \normalsize 年数 & \SimHei \scriptsize 公元 & \SimHei 大事件 \tabularnewline
  \midrule
  \endhead
  \midrule
  元年 & 417 & \tabularnewline\hline
  二年 & 418 & \tabularnewline\hline
  三年 & 419 & \tabularnewline\hline
  四年 & 420 & \tabularnewline
  \bottomrule
\end{longtable}


%%% Local Variables:
%%% mode: latex
%%% TeX-engine: xetex
%%% TeX-master: "../../Main"
%%% End:

%% -*- coding: utf-8 -*-
%% Time-stamp: <Chen Wang: 2019-12-19 16:28:45>

\subsection{李恂\tiny(420-421)}

\subsubsection{生平}

李恂(?-421年),字士如,陇西狄道(今甘肃临洮县)人,十六國時期西涼的君主,凉武昭王李暠第五子,李歆之弟,李歆在位時任敦煌太守。

西涼嘉興四年(420年),北涼敗西涼軍殺李歆,隨即攻佔西涼都城酒泉,李恂及其他諸弟逃往北山。數月後,因北涼王沮渠蒙遜所派敦煌太守索元緒凶險好殺,大失人心,而李恂在敦煌施政名聲卓著,敦煌人民遂密迎李恂,李恂率數十騎入敦煌,索元緒逃走,李恂被推為冠軍將軍、涼州刺史,改元永建。不久,沮渠蒙遜派軍討伐。隔年(421年),北涼軍引水灌敦煌,李恂乞降不成,部下投降,李恂遂自殺,西涼亦亡。

\subsubsection{永建}

\begin{longtable}{|>{\centering\scriptsize}m{2em}|>{\centering\scriptsize}m{1.3em}|>{\centering}m{8.8em}|}
  % \caption{秦王政}\
  \toprule
  \SimHei \normalsize 年数 & \SimHei \scriptsize 公元 & \SimHei 大事件 \tabularnewline
  % \midrule
  \endfirsthead
  \toprule
  \SimHei \normalsize 年数 & \SimHei \scriptsize 公元 & \SimHei 大事件 \tabularnewline
  \midrule
  \endhead
  \midrule
  元年 & 420 & \tabularnewline\hline
  二年 & 421 & \tabularnewline
  \bottomrule
\end{longtable}


%%% Local Variables:
%%% mode: latex
%%% TeX-engine: xetex
%%% TeX-master: "../../Main"
%%% End:


%%% Local Variables:
%%% mode: latex
%%% TeX-engine: xetex
%%% TeX-master: "../../Main"
%%% End:

%% -*- coding: utf-8 -*-
%% Time-stamp: <Chen Wang: 2019-12-19 16:30:29>


\section{夏\tiny(407-431)}

\subsection{简介}

夏(407年-431年)又称为大夏或北夏,因为建立者赫连勃勃是匈奴铁弗部人,故又称胡夏或赫连夏,是407年到431年存在于关中与河套地区的一个国家,国都在大部分时间都位于统万城(今陕西省靖边县红墩界乡白城子村无定河北岸)。夏是匈奴铁弗部首领赫连勃勃在407年自称大单于后所建。418年,赫连勃勃在攻陷长安后称帝。427年,北魏太武帝出兵攻陷统万城并于428年俘虏夏国第二代君主赫连昌。赫连昌之弟赫连定随后被拥立为君主并在430年灭亡西秦。但431年赫连定被吐谷浑俘虏,之后在432年被送往北魏处死。夏国共存在25年,历经三代君主。

夏国是五胡十六国中最晚建立的政权,其都城统万城遗址是至今唯一保存基本完好的早期北方王国都城遗址,也是匈奴人历史上留下的唯一都城遗址。

夏国最初占有大城(今内蒙古杭锦旗东南),之后于413年又修建统万城作为首都。至417年后秦灭亡前,夏国占据河套至陇东与陕西北部,

夏国的地方行政主要分为州、城(县)两级。有史可考的有幽、朔、秦、北秦、雍、并、梁、豫、荆九州。文献记载最多的只有幽州。

%% -*- coding: utf-8 -*-
%% Time-stamp: <Chen Wang: 2019-12-19 16:31:30>

\subsection{武烈帝\tiny(407-425)}

\subsubsection{生平}

夏武烈帝赫连勃勃(381年-425年),字屈孑,匈奴铁弗部人,原名劉勃勃,中國十六国时期夏國建立者。勃勃是南匈奴單于的後裔,其父劉衞辰死於北魏進攻後,勃勃依靠後秦高平公沒弈干,又得後秦君主姚興賞識。及後就以後秦與魏通好而叛秦,殺害沒奕干並自立,建北夏國,屢度進攻後秦。隨後更乘東晉滅後秦後班師的機會佔領關中。

赫連勃勃曾祖父劉虎領導鐵弗部,並曾與代國發生戰鬥,為代國所敗。祖父劉務桓重整部眾,重新壯大鐵弗部,並受後趙封為平北將軍、左賢王。父刘卫辰繼位後搖擺於前秦及代國之間,前秦皇帝苻坚滅代國後更任命劉衞辰为西单于,屯駐代來城(今內蒙古伊克昭盟東勝區西),督摄河西诸部族。前秦瓦解後,劉衞辰一度據有朔方一帶,但在391年受到北魏的攻擊,代來城被攻陷,劉衛辰被殺。年幼的劉勃勃逃奔薛干部,薛干酋長把劉勃勃送給後秦高平公沒弈干,沒弈干就把女兒嫁給劉勃勃。勃勃受到後秦姚興的寵遇,任為安北將軍、五原公,鎮朔方。此後一直从属后秦。

後秦弘始九年(407年),赫連勃勃因怨恨後秦與北魏通訊,決意背叛後秦。於是先扣押起柔然可汗送給後秦的八千匹馬,然後假裝在高平川(今寧夏南清水河)狩獵,襲殺沒弈干,併吞其部眾。勃勃自以是匈奴夏后氏後裔,建國號「大夏」,自立为天王,大单于,国号夏,改年號龙升。

勃勃自立後不久就出兵薛干部等三部,收降數萬人後轉攻後秦三城(今陝西綏德縣)以北諸戍。當時諸將都反對出兵後秦,建議勃勃先固守高平,穩固根本,然後才圖長安。但勃勃認為夏國初建,實力仍弱,關中仍因後秦強大而未能攻取,若果自守一城,必會引來後秦各鎮的聯手攻伐,終兵敗亡國,故此特意不長居一處,以游擊戰術,出其不意,讓對方疲於奔命,以取嶺兵、河東之地,再待後秦君主姚興死後才攻取長安。最終勃勃的進襲令到嶺兵各城日間也要緊閉城門。勃勃稱天王後曾向南涼王禿髮傉檀請婚但遭拒,於是怒而率兵進攻南涼,殺傷一萬多人並掠奪二萬七千人及數十萬頭牲畜。後更在陽武(今甘肅靖遠縣境)大敗來攻的禿髮傉檀,殺傷甚眾,很多南涼的名將都戰死。

大破南涼後,勃勃與後秦屢有戰事。勃勃先於青石原(今甘肅涇川縣境)擊敗秦將張佛生,次年(408年)後秦將齊難來攻,勃勃先退守河曲,待齊難縱兵掠奪時就進軍,並追擊至木城(今陝西榆林市榆陽區),生擒齊難,俘獲大量兵眾及戰馬。戰後,歸降夏國的嶺北胡漢數以萬計,勃勃更置地方官員安撫他們。龍升三年(409年),勃勃又率兵攻秦,掠奪平涼雜胡共七千多戶配給後援軍隊,進據依力川。同年姚興親征勃勃,但勃勃乘秦軍未集,率軍進攻姚興所駐的貮城,秦軍兵敗,姚興只得退回長安。接著勃勃又攻下了敕奇堡、黃石固及我羅城。龍升四年(410年),勃勃派兵攻平涼,為姚興所敗,但進攻定陽(今陝西宜川縣西北)的一軍卻取勝,接著勃勃親自率軍進攻隴右,攻破白崖堡,並進逼清水城,令後秦略陽太守姚壽都棄城逃走。龍升五年(411年),勃勃攻安定,在青石以北平原擊破楊佛嵩,又攻下東鄉。

鳳翔元年(413年),勃勃下令修築夏國都城统万城(今陕西靖边北白城子),並任用了殘忍的叱干阿利為將作大匠。工人以蒸土築城,而巡工发现墙面能用铁锥子刺入一寸,便把修築那處的人处死,尸体也被筑入墙内,因此,统万城的城墙坚硬如铁。其時又用銅鑄造了大鼓、飛廉、翁仲、銅駝、龍獸等裝飾物,並用黃金裝飾,排在宮殿前,但製作這些東西又殺了數千個工匠。同時,又以其祖輩跟隨母系姓劉不合禮,於是改姓赫連,表示「徽赫實與天連」;又將非皇族的其他鐵弗部眾改姓「鐵伐」,以示「剛銳如鐵,皆堪伐人。」

此後,勃勃仍然繼續侵襲後秦,先於鳳翔三年(415年)攻下杏城(今陝西黃陵縣西南);次年又乘後秦與仇池楊盛爭戰的時機先後攻下上邽(仱甘肅天水市)及陰密(今陝西靈臺縣西),更令駐守安定的姚恢棄城出走,當地人胡儼等於是獻城夏國。勃勃隨後又進攻雍城(今陝西鳳翔縣南)及郿城(今陝西郿縣),但在郿城遭姚紹抵抗而未能攻下,於是退回安定。胡儼等人此時卻殺勃勃所命留守安定的羊苟兒,轉而降秦。勃勃唯有退返杏城。不過,其時勃勃知東晉劉裕要進攻後秦,他估計劉裕必定能攻滅後秦,但肯定很快班師,留下子弟及將領守關中。勃勃認定這就是他奪取關中的好時機,並且十分輕易,故此不必耗費兵力與後秦作戰。故此秣馬厲兵,休養士卒。不久勃勃再引兵佔據安定,後秦在嶺北的各戍及郡縣都向夏國投降,嶺北全境盡入夏國。另一方面,勃勃又先後與北燕及北涼結盟。

凤翔五年(417年),東晉大将刘裕滅後秦,同年年末班師,留兒子劉義真及王鎮惡、沈田子、傅弘之等諸將守關中。勃勃聞訊十分高興,就派兒子赫連璝督前鋒攻長安、赫連昌出兵堵塞潼關,又派王買德阻斷青泥,然後自率大軍在後。次年,赫連璝行軍至渭陽時已經有很多人在路邊請降,其時晉軍沈田子作戰失利,更因與王鎮惡不和而殺了他,沈田子隨後亦被劉義真長史王脩處死。劉義真於是召集外軍入城並閉門拒守,關中各郡縣於是都降夏。勃勃隨後進據咸陽,令長安城無法獲得物資補給。劉裕見此唯有派朱齡石接替劉義真,並命劉義真東歸。當時劉義真部眾大肆掠奪物資才離開,令關中人民驅逐朱齡石,迎勃勃入主長安。勃勃入長安後大宴將士,不久就在灞上(今陝西蓝田县)称帝,改元昌武。及後群臣都勸勃勃遷都長安,但勃勃慮及全國中心南遷長安後,北魏會易於攻擊距邊界才百里的統萬,認為定都統萬才能阻遏北魏侵襲北境。於是在次年(419年)於長安置南臺,留太子赫連璝留守。不久回師,因統萬宮殿完工而刻石於城南,歌功頌德。

真兴六年(424年),勃勃想要廢黜太子赫連璝,改立幼子赫連倫。赫連璝知道後率兵七萬自長安攻伐赫連倫,終在高平一戰中擊敗並殺死對方。赫連倫兄赫連昌則率軍襲擊赫連璝,將其殺死,勃勃於是立赫連昌為太子。勃勃於真興七年(425年)死于帝位,諡号武烈皇帝,廟號世祖。

勃勃身材魁武,高八尺五寸,且聰慧有儀態,有辯才的機悟。

勃勃頗有權謀智術,當劉裕滅後秦、占關中之時,勃勃一度震攝於劉裕的兵鋒威勢,答應劉裕「約為兄弟」的和平要求,但勃勃為了在氣勢上勝過劉裕,在接見劉裕的使者之前,先讓文才優異的部下皇甫徽寫好給劉裕的回書,再將文字背的爛熟,然後才接見使者,命令部下將自己當場背出的回書寫下並交付給使者。結果使者就以為勃勃真的即席創作出文采斐然的回書,將此事回報給劉裕,果然讓老粗一名的劉裕敬佩勃勃的文思敏捷,邊讀回書邊感嘆說:「吾不如也!」

勃勃凶暴好殺,在陽武大敗南涼軍及關中大敗東晉軍時曾將屍體或人頭堆積起來,建起「髑髏臺」,當作景觀觀賞。也常常在城上,身旁準備好弓箭刀劍,一旦對人有所不滿就會動手殺人。而群臣若敢直接與其對視就會被弄瞎,敢笑就割下其嘴唇,敢進諫就先割下其舌頭再斬殺。這令當時人們十分不安。

勃勃十分自大,建的統萬城四個城門,東門叫招魏門,南門叫朝宋門,西門叫服涼門,北門叫平朔門。

後秦君臣姚興、姚邕兩方的意見:「姚邕說:『勃勃不可近也。』姚興說:『勃勃有濟世之才,吾方與之平天下,柰何逆忌之?』姚邕說:『勃勃奉上慢,御眾殘,貪猾不仁,輕為去就;寵之踰分,恐終為邊患。』」後來勃勃反叛後秦並成為大患,姚興因此感嘆說:「吾不用黃兒之言,以至於此!」(按:姚邕小字黃兒)

南涼大臣焦朗評論:「勃勃天姿雄健,御軍嚴整,未可輕也。」

\subsubsection{龙昇}

\begin{longtable}{|>{\centering\scriptsize}m{2em}|>{\centering\scriptsize}m{1.3em}|>{\centering}m{8.8em}|}
  % \caption{秦王政}\
  \toprule
  \SimHei \normalsize 年数 & \SimHei \scriptsize 公元 & \SimHei 大事件 \tabularnewline
  % \midrule
  \endfirsthead
  \toprule
  \SimHei \normalsize 年数 & \SimHei \scriptsize 公元 & \SimHei 大事件 \tabularnewline
  \midrule
  \endhead
  \midrule
  元年 & 407 & \tabularnewline\hline
  二年 & 408 & \tabularnewline\hline
  三年 & 409 & \tabularnewline\hline
  四年 & 410 & \tabularnewline\hline
  五年 & 411 & \tabularnewline\hline
  六年 & 412 & \tabularnewline\hline
  七年 & 413 & \tabularnewline
  \bottomrule
\end{longtable}

\subsubsection{凤翔}

\begin{longtable}{|>{\centering\scriptsize}m{2em}|>{\centering\scriptsize}m{1.3em}|>{\centering}m{8.8em}|}
  % \caption{秦王政}\
  \toprule
  \SimHei \normalsize 年数 & \SimHei \scriptsize 公元 & \SimHei 大事件 \tabularnewline
  % \midrule
  \endfirsthead
  \toprule
  \SimHei \normalsize 年数 & \SimHei \scriptsize 公元 & \SimHei 大事件 \tabularnewline
  \midrule
  \endhead
  \midrule
  元年 & 413 & \tabularnewline\hline
  二年 & 414 & \tabularnewline\hline
  三年 & 415 & \tabularnewline\hline
  四年 & 416 & \tabularnewline\hline
  五年 & 417 & \tabularnewline\hline
  六年 & 418 & \tabularnewline
  \bottomrule
\end{longtable}

\subsubsection{昌武}

\begin{longtable}{|>{\centering\scriptsize}m{2em}|>{\centering\scriptsize}m{1.3em}|>{\centering}m{8.8em}|}
  % \caption{秦王政}\
  \toprule
  \SimHei \normalsize 年数 & \SimHei \scriptsize 公元 & \SimHei 大事件 \tabularnewline
  % \midrule
  \endfirsthead
  \toprule
  \SimHei \normalsize 年数 & \SimHei \scriptsize 公元 & \SimHei 大事件 \tabularnewline
  \midrule
  \endhead
  \midrule
  元年 & 418 & \tabularnewline\hline
  二年 & 419 & \tabularnewline
  \bottomrule
\end{longtable}

\subsubsection{真兴}

\begin{longtable}{|>{\centering\scriptsize}m{2em}|>{\centering\scriptsize}m{1.3em}|>{\centering}m{8.8em}|}
  % \caption{秦王政}\
  \toprule
  \SimHei \normalsize 年数 & \SimHei \scriptsize 公元 & \SimHei 大事件 \tabularnewline
  % \midrule
  \endfirsthead
  \toprule
  \SimHei \normalsize 年数 & \SimHei \scriptsize 公元 & \SimHei 大事件 \tabularnewline
  \midrule
  \endhead
  \midrule
  元年 & 419 & \tabularnewline\hline
  二年 & 420 & \tabularnewline\hline
  三年 & 421 & \tabularnewline\hline
  四年 & 422 & \tabularnewline\hline
  五年 & 423 & \tabularnewline\hline
  六年 & 424 & \tabularnewline\hline
  七年 & 425 & \tabularnewline
  \bottomrule
\end{longtable}


%%% Local Variables:
%%% mode: latex
%%% TeX-engine: xetex
%%% TeX-master: "../../Main"
%%% End:

%% -*- coding: utf-8 -*-
%% Time-stamp: <Chen Wang: 2021-11-01 14:59:16>

\subsection{赫连昌\tiny(425-428)}

\subsubsection{生平}

赫連昌(?-434年),一名折,字還國,十六國時期夏國皇帝,匈奴鐵弗部人,赫連勃勃三子,赫連勃勃在位時被封太原公。

夏真興六年(424年),赫連勃勃欲廢太子赫連璝,改立酒泉公赫連倫,赫連璝發兵攻殺赫連倫,後赫連昌再襲殺赫連璝平亂,赫連勃勃遂以赫連昌為太子。真興七年(425年)赫連勃勃去世,赫連昌繼位,改元承光。

承光二年(426年),北魏大舉攻夏,克長安。次年(427年),占領夏國都城統萬(今內蒙古烏審旗南白城子),赫連昌逃往上邽(今甘肅天水)。承光四年(428年),北魏攻上邽,會戰中赫連昌因馬失前蹄墜地而被生擒。

赫連昌被俘後,北魏太武帝拓跋燾十分禮遇他,不僅使其住在西宮,更把皇妹嫁給他,並封會稽公。拓跋燾亦常命赫連昌隨待在側,打獵時二人有時亦單獨並騎,赫連昌素有勇名,因此拓跋燾可說對赫連昌十分信任。北魏神䴥三年(430年)三月又被封為秦王。

北魏延和三年(434年)閏三月,赫連昌叛魏西逃,途中被抓獲格殺。

\subsubsection{承光}

\begin{longtable}{|>{\centering\scriptsize}m{2em}|>{\centering\scriptsize}m{1.3em}|>{\centering}m{8.8em}|}
  % \caption{秦王政}\
  \toprule
  \SimHei \normalsize 年数 & \SimHei \scriptsize 公元 & \SimHei 大事件 \tabularnewline
  % \midrule
  \endfirsthead
  \toprule
  \SimHei \normalsize 年数 & \SimHei \scriptsize 公元 & \SimHei 大事件 \tabularnewline
  \midrule
  \endhead
  \midrule
  元年 & 425 & \tabularnewline\hline
  二年 & 426 & \tabularnewline\hline
  三年 & 427 & \tabularnewline\hline
  四年 & 428 & \tabularnewline
  \bottomrule
\end{longtable}


%%% Local Variables:
%%% mode: latex
%%% TeX-engine: xetex
%%% TeX-master: "../../Main"
%%% End:

%% -*- coding: utf-8 -*-
%% Time-stamp: <Chen Wang: 2019-12-19 16:32:20>

\subsection{郝连定\tiny(428-437)}

\subsubsection{生平}

赫連定(?-432年),小字直獖,十六國時期夏國君主,匈奴鐵弗部人,赫連勃勃五子,赫連昌之弟,赫連勃勃在位時被封平原公,鎮守長安。

夏國皇帝赫連昌承光二年(426年),北魏大舉攻夏,赫連定與北魏軍對峙於長安一帶。次年(427年),夏國都城統萬(今內蒙古烏審旗南白城子)陷落,赫連定逃往上邽(今甘肅天水)與赫連昌會合,被進封平原王。承光四年(428年),北魏攻上邽,赫連昌被擒,赫連定逃奔平涼(今甘肅平涼),即皇帝位,改年號勝光。

赫連定繼位時夏國已侷促一隅,情勢窘迫,不復當年,因此欲與正在北伐的南朝宋結盟,北魏得到消息後決定一舉滅夏國。勝光四年(431年),一路敗退的赫連定無路可退,遂向西攻滅為北涼所逼情勢更加窘迫的西秦。數月後欲再攻北涼,於半渡黃河時,被吐谷渾首領慕容慕璝派軍襲擊,赫連定被俘。次年(432年),赫連定被吐谷渾送往北魏,北魏將其處死。

\subsubsection{胜光}

\begin{longtable}{|>{\centering\scriptsize}m{2em}|>{\centering\scriptsize}m{1.3em}|>{\centering}m{8.8em}|}
  % \caption{秦王政}\
  \toprule
  \SimHei \normalsize 年数 & \SimHei \scriptsize 公元 & \SimHei 大事件 \tabularnewline
  % \midrule
  \endfirsthead
  \toprule
  \SimHei \normalsize 年数 & \SimHei \scriptsize 公元 & \SimHei 大事件 \tabularnewline
  \midrule
  \endhead
  \midrule
  元年 & 428 & \tabularnewline\hline
  二年 & 429 & \tabularnewline\hline
  三年 & 430 & \tabularnewline\hline
  四年 & 431 & \tabularnewline
  \bottomrule
\end{longtable}


%%% Local Variables:
%%% mode: latex
%%% TeX-engine: xetex
%%% TeX-master: "../../Main"
%%% End:



%%% Local Variables:
%%% mode: latex
%%% TeX-engine: xetex
%%% TeX-master: "../../Main"
%%% End:

%% -*- coding: utf-8 -*-
%% Time-stamp: <Chen Wang: 2019-12-19 16:34:34>


\section{北燕\tiny(407-436)}

\subsection{简介}

北燕(407年或409年-436年)是十六國时期汉人馮跋建立的政权。407年,馮跋灭后燕,拥立高云(慕容云)为天王,建都龙城(今遼寧省朝陽市),仍旧沿用后燕国号。409年,高云被部下離班、桃仁所杀,馮跋平定政变後即天王位于昌黎(今辽宁省义县)。據有今遼寧省西南部和河北省东北部。436年被北魏所灭。

因其都龙城,又名黄龙,故南朝宋称其为黄龙国。也有史书因其地处东北地区南部,又称其为东燕,但较为罕见。

%% -*- coding: utf-8 -*-
%% Time-stamp: <Chen Wang: 2021-11-01 14:59:55>

\subsection{惠懿帝高雲\tiny(407-409)}

\subsubsection{生平}

燕惠懿帝高雲(4世纪-409年11月6日),曾改名慕容雲,字子雨,高句驪人。十六国時期後燕末代君主,一說為北燕开国国主,称号天王。

早期的高雲於後燕時沉默寡言,並沒有什麼名氣,只有中衛將軍馮跋看出他的氣度與他結交。

後燕永康二年(397年),高雲因率軍擊敗慕容寶之子慕容會的叛軍,被慕容寶收養,賜姓慕容氏,封夕陽公。

後燕建初元年(407年)馮跋反,殺皇帝慕容熙,在馮跋支持之下,慕容雲即天王位,改元曰正始,國號大燕,恢復原本的高姓。高雲自知無功而登大位,因此培養一批禁衛保護自己,但後來反被禁衛離班和桃仁所殺,高雲死後被諡惠懿皇帝。

由於對高雲是否屬後燕慕容氏一族成員的看法不同,因此有人認為高雲是後燕末任君主,也有人把他視為北燕立國君主。

\subsubsection{正始}

\begin{longtable}{|>{\centering\scriptsize}m{2em}|>{\centering\scriptsize}m{1.3em}|>{\centering}m{8.8em}|}
  % \caption{秦王政}\
  \toprule
  \SimHei \normalsize 年数 & \SimHei \scriptsize 公元 & \SimHei 大事件 \tabularnewline
  % \midrule
  \endfirsthead
  \toprule
  \SimHei \normalsize 年数 & \SimHei \scriptsize 公元 & \SimHei 大事件 \tabularnewline
  \midrule
  \endhead
  \midrule
  元年 & 407 & \tabularnewline\hline
  二年 & 408 & \tabularnewline\hline
  三年 & 409 & \tabularnewline
  \bottomrule
\end{longtable}


%%% Local Variables:
%%% mode: latex
%%% TeX-engine: xetex
%%% TeX-master: "../../Main"
%%% End:

%% -*- coding: utf-8 -*-
%% Time-stamp: <Chen Wang: 2021-11-01 15:00:08>

\subsection{文成帝冯跋\tiny(409-430)}

\subsubsection{生平}

北燕文成帝冯\xpinyin*{跋}(4世紀?-430年),十六國時期北燕君主,字文起,小名乞直伐,是胡化的漢族人,长乐信都(今河北省衡水市冀州区)人。

冯跋是馮和之孫,其父馮安曾任西燕將軍。西燕亡,馮跋東遷後燕,於後燕帝慕容寶在位時被任命為中卫将军。

馮跋與其弟馮素弗先前曾因事獲罪於後燕帝慕容熙,因此慕容熙有殺馮跋兄弟之意,馮跋兄弟遂逃匿深山。馮跋兄弟商量說:「熙今昏虐,兼忌吾兄弟,既還首無路,不可坐受誅滅。當及時而起,立公侯之業。事若不成,死其晚乎!」於是與從兄萬泥等二十二人合謀。後燕建始元年(407年)馮跋兄弟乘車,由婦人禦,潛入都城和龙(今辽宁朝阳),匿於北部司馬孫護家。趁慕容熙送葬苻后之際起事,推高雲(慕容雲)為燕王,改元正始,不久擒殺慕容熙。高雲登位後以馮跋為侍中、征北大將軍、開府儀同三司,封武邑公,政事皆決於馮跋兄弟。

正始三年(409年),高雲為寵臣離班、桃仁所殺,亂事平定後,眾人推馮跋為主,馮跋遂即天王位,改元太平。馮跋勤於政事,獎勵農桑,輕薄徭役,因此人民喜悅,雖外有強大的北魏相逼,也維持20餘年的安定。

北燕太平二十二年(430年),馮跋病重,命太子馮翼攝理國家大事,未料宋夫人有為其子馮受居圖謀王位之意,馮跋之弟馮弘於是帶兵進宮平變,倉促間馮跋於驚懼中去世。後被諡文成皇帝,廟號太祖。冯弘篡位,将包括冯翼在内的冯跋之子一百余人一并杀死。

\subsubsection{太平}

\begin{longtable}{|>{\centering\scriptsize}m{2em}|>{\centering\scriptsize}m{1.3em}|>{\centering}m{8.8em}|}
  % \caption{秦王政}\
  \toprule
  \SimHei \normalsize 年数 & \SimHei \scriptsize 公元 & \SimHei 大事件 \tabularnewline
  % \midrule
  \endfirsthead
  \toprule
  \SimHei \normalsize 年数 & \SimHei \scriptsize 公元 & \SimHei 大事件 \tabularnewline
  \midrule
  \endhead
  \midrule
  元年 & 409 & \tabularnewline\hline
  二年 & 410 & \tabularnewline\hline
  三年 & 411 & \tabularnewline\hline
  四年 & 412 & \tabularnewline\hline
  五年 & 413 & \tabularnewline\hline
  六年 & 414 & \tabularnewline\hline
  七年 & 415 & \tabularnewline\hline
  八年 & 416 & \tabularnewline\hline
  九年 & 417 & \tabularnewline\hline
  十年 & 418 & \tabularnewline\hline
  十一年 & 419 & \tabularnewline\hline
  十二年 & 420 & \tabularnewline\hline
  十三年 & 421 & \tabularnewline\hline
  十四年 & 422 & \tabularnewline\hline
  十五年 & 423 & \tabularnewline\hline
  十六年 & 424 & \tabularnewline\hline
  十七年 & 425 & \tabularnewline\hline
  十八年 & 426 & \tabularnewline\hline
  十九年 & 427 & \tabularnewline\hline
  二十年 & 428 & \tabularnewline\hline
  二一年 & 429 & \tabularnewline\hline
  二二年 & 430 & \tabularnewline
  \bottomrule
\end{longtable}


%%% Local Variables:
%%% mode: latex
%%% TeX-engine: xetex
%%% TeX-master: "../../Main"
%%% End:

%% -*- coding: utf-8 -*-
%% Time-stamp: <Chen Wang: 2021-11-01 15:00:15>

\subsection{昭成帝馮弘\tiny(430-436)}

\subsubsection{生平}

北燕昭成帝馮弘(?-438年),十六國時期北燕國君主,字文通,長樂信都(今河北省衡水市冀州区)人,北燕太祖馮跋之弟。

馮跋在位時,馮弘被封中山公司徒錄尚書事,輔政。

馮跋病重時,宋夫人有為其子馮受居圖謀王位之意,馮弘於是帶兵入宮平變,倉促間馮跋於驚懼中去世,馮弘遂即天王位,並下詔書說:「天降凶禍,大行崩背,太子不侍疾,群公不奔喪,疑有逆謀,社稷將危。吾備介弟之親,遂攝大位以寧國家;百官叩門入者,進陛二等。」盡殺包括太子馮翼在內的馮跋諸子百人。

翌年(431年),改元太興。將自己元配夫人王氏及其所生之子、太子馮崇廢掉。於第二年四月,冊立後燕皇族之女慕容氏為天后,藉以抬高其身價。於是,長樂公馮崇,以及馮崇之同母弟、廣平公馮朗,樂陵公馮邈也懼繼母迫害,禍及自身,於是舉郡向北魏投誠。第三年,春正月,馮弘冊立「後妻慕容氏子馮王仁為世子」。

北燕國小民弱,馮弘在位時,因北魏屢次攻伐,數次向北魏朝貢請和,但仍持續受到攻擊,因此亦曾遣使向南朝宋稱藩納貢。

太興六年(436年),北魏再攻北燕,馮弘於五月乙卯日(6月4日),馮弘帶領子女、後宮、宗族,及龍城之百姓,隨高句麗援軍從都城龍城(今遼寧朝陽)撤退,臨行焚其宮室、城邑,大火一旬不滅,北燕亡。

馮弘在高句麗號令如在本國,引起高句麗長壽王高璉嫌惡,長壽王將其侍衛撤走,又將其太子馮王仁押回興京,扣作人質。復有歸刘宋之意,於是又派使者帶著三百人出使建康,請求宋文帝允許其全家移居建康;宋文帝答應,並派遣將軍王白駒,率兵七千,北上迎接。當時,高句麗也向刘宋稱臣。高句麗王不欲馮弘南下成仇,好言規勸,而馮弘不聽。遂於438年殺馮弘及其妻子,並為其上諡號曰昭成皇帝,一作昭文皇帝。

\subsubsection{太兴}

\begin{longtable}{|>{\centering\scriptsize}m{2em}|>{\centering\scriptsize}m{1.3em}|>{\centering}m{8.8em}|}
  % \caption{秦王政}\
  \toprule
  \SimHei \normalsize 年数 & \SimHei \scriptsize 公元 & \SimHei 大事件 \tabularnewline
  % \midrule
  \endfirsthead
  \toprule
  \SimHei \normalsize 年数 & \SimHei \scriptsize 公元 & \SimHei 大事件 \tabularnewline
  \midrule
  \endhead
  \midrule
  元年 & 431 & \tabularnewline\hline
  二年 & 432 & \tabularnewline\hline
  三年 & 433 & \tabularnewline\hline
  四年 & 434 & \tabularnewline\hline
  五年 & 435 & \tabularnewline\hline
  六年 & 436 & \tabularnewline
  \bottomrule
\end{longtable}


%%% Local Variables:
%%% mode: latex
%%% TeX-engine: xetex
%%% TeX-master: "../../Main"
%%% End:



%%% Local Variables:
%%% mode: latex
%%% TeX-engine: xetex
%%% TeX-master: "../../Main"
%%% End:

%% -*- coding: utf-8 -*-
%% Time-stamp: <Chen Wang: 2019-12-19 16:38:03>


\section{北凉\tiny(397-439)}

\subsection{简介}

北凉(397年或401年-439年)是十六国之一。由匈奴支系盧水胡族的首領沮渠蒙逊所建立;另有一種看法認為建立者為段業,此說是以蒙遜堂兄沮渠男成擁立段業稱涼州牧,並改元神璽為立國之始(397年)。

401年蒙遜誣男成謀反,段業斬男成,蒙遜以此為藉口攻滅段業,仍稱涼州牧,改元永安,因此亦有人以此為北涼立國之時。

北涼首都为张掖,蒙遜自称张掖公。412年迁都姑臧(今甘肃武威),称河西王。最强盛的时候控制今甘肃西部、宁夏、新疆、青海的一部分,是河西一帶最強大的勢力。420年灭西凉。433年蒙逊去世,其子沮渠牧犍继位。439年北魏大军围攻姑臧,沮渠牧犍出降,北涼亡,北魏統一華北。

後牧犍弟沮渠無諱西行至高昌,建立高昌北涼,一般認為已脫離五胡十六國時代之範圍,460年高昌北涼為柔然所攻滅,無諱弟沮渠安周被殺,高昌北涼亦亡。

%% -*- coding: utf-8 -*-
%% Time-stamp: <Chen Wang: 2019-12-19 16:39:00>

\subsection{段业\tiny(397-401)}

\subsubsection{生平}

段業(?-401年),京兆郡(治今陝西西安)漢人。十六国时期北涼国開國君主,但其本身只是為盧水胡沮渠蒙遜及沮渠男成所推,他也很忌憚沮渠蒙遜,蒙遜亦十分不安,最終沮渠蒙遜發動兵變推翻並殺害段業。

段業博覽史傳,有文辭才學,原是前秦將領吕光部將杜進僚屬,從征西域。後呂光建後涼,出任建康太守。後涼龍飛二年(397年),沮渠蒙遜叛後涼,其堂兄沮渠男成亦叛,並進攻段業所守的建康。男成派使者勸段業支持自己,段業最初不肯,但在男成圍困二十日後還是答應了,遂被推為大都督、龍驤大將軍、涼州牧、建康公,改年號神璽。

神璽元年(397年)以沮渠男成為輔國將軍、沮渠男成的堂弟沮渠蒙遜為張掖太守,委以軍國重任。

神璽二年(398年),段業在沮渠蒙遜的支持下,力排眾議命蒙遜進攻西郡,終擒太守呂純,隨後晉昌太守王德及敦煌太守孟敏都向段業投降。不久段業又攻呂弘鎮守的張掖,呂弘率兵棄城東歸,段業不聽沮渠蒙遜歸師勿遏、窮寇勿追的諫言,執意追擊,終大敗而還。神璽三年(399年),稱涼王,改元天璽。同年後涼太子呂紹及呂纂來攻,段業請得禿髮烏孤派楊軌等協助,就打算進攻結陣迎戰的後涼軍。蒙遜卻認為楊軌伺機圖謀北涼,而且後涼軍兵處死地,肯定會奮戰求生,故段業不要出戰,免陷入危機。段業同意,最終按兵不戰,後涼軍也退兵。

段業本來只是一個有德望的儒者,因緣際會被推上王位,其實本人並沒有權謀,無法約束下屬,只信任卜卦、巫術。而一直以來,段業對於沮渠蒙遜的勇略就頗為忌憚,最初就讓沮渠蒙遜由尚書左丞外調到臨池郡任太守,想疏遠他。段業又親近信任門下侍郎馬權,以其代替蒙遜張掖太守之位,但蒙遜怨恨馬權常輕侮自己,於是向段業中傷馬權,段業遂殺馬權。另段業因索嗣認為李暠不能留在敦煌,任由其發展其勢力的建議而派索嗣接替李暠任敦煌太守,然李暠擊敗了索嗣,並上請段業誅殺索嗣,段業在沮渠男成勸告下就將索嗣殺了。此時,蒙遜有除掉段業之意,遂和男成表示既馬權、索嗣二人已死,應當殺死段業,改奉男成為主,但為男成拒絕。蒙遜因段業忌憚自己而愈見不安,遂自請任西安太守,段業亦怕蒙遜很快會反叛,答應了其請求。

天璽三年(401年),沮渠蒙遜誣沮渠男成謀反,段業收捕沮渠男成並命其自殺,男成死前對段業說:「蒙遜早就和臣說過他要叛亂了,只是臣以兄弟緣故才不說出來。蒙遜以臣還在,怕部眾不聽從他,於是約臣與其祭山,反派人誣告臣。臣若果死了,蒙遜肯定很快就起兵了。請假稱臣死了,宣告臣的罪行,蒙遜肯定會起兵叛亂,而臣立即就會討伐他,必會成功。」可是,段業沒有聽信。男成死後,沮渠蒙遜以此為藉口激怒將士,並率領他們攻擊段業,連羌胡都起兵響應。段業見此就讓田昂及梁中庸率兵攻蒙遜,當時將領王豐孫警告稱西平田氏世代都有反叛者,而田昂「貌恭而心狠,志大而情險」,並不可信;但段業自以只能倚仗他對抗蒙遜,還是沒有聽從。最終田昂果然臨陣降於蒙遜,梁中庸亦被逼投降。接著田昂侄田承愛在蒙遜兵臨張掖時讓蒙遜入城,段業左右潰散,段業請求蒙遜饒他一命,讓其東歸與妻兒見面,但蒙遜還是殺了他。沮渠蒙遜隨後獲推為張掖公,繼立為北涼君主。

\subsubsection{神玺}

\begin{longtable}{|>{\centering\scriptsize}m{2em}|>{\centering\scriptsize}m{1.3em}|>{\centering}m{8.8em}|}
  % \caption{秦王政}\
  \toprule
  \SimHei \normalsize 年数 & \SimHei \scriptsize 公元 & \SimHei 大事件 \tabularnewline
  % \midrule
  \endfirsthead
  \toprule
  \SimHei \normalsize 年数 & \SimHei \scriptsize 公元 & \SimHei 大事件 \tabularnewline
  \midrule
  \endhead
  \midrule
  元年 & 397 & \tabularnewline\hline
  二年 & 398 & \tabularnewline\hline
  三年 & 399 & \tabularnewline
  \bottomrule
\end{longtable}


\subsubsection{天玺}

\begin{longtable}{|>{\centering\scriptsize}m{2em}|>{\centering\scriptsize}m{1.3em}|>{\centering}m{8.8em}|}
  % \caption{秦王政}\
  \toprule
  \SimHei \normalsize 年数 & \SimHei \scriptsize 公元 & \SimHei 大事件 \tabularnewline
  % \midrule
  \endfirsthead
  \toprule
  \SimHei \normalsize 年数 & \SimHei \scriptsize 公元 & \SimHei 大事件 \tabularnewline
  \midrule
  \endhead
  \midrule
  元年 & 399 & \tabularnewline\hline
  二年 & 400 & \tabularnewline\hline
  三年 & 401 & \tabularnewline
  \bottomrule
\end{longtable}


%%% Local Variables:
%%% mode: latex
%%% TeX-engine: xetex
%%% TeX-master: "../../Main"
%%% End:

%% -*- coding: utf-8 -*-
%% Time-stamp: <Chen Wang: 2021-11-01 15:00:53>

\subsection{武宣王沮渠蒙遜\tiny(401-433)}

\subsubsection{生平}

沮渠蒙遜(368年-433年),臨松匈奴人,十六国时期北涼第二任君主。原係匈奴支系卢水胡族首領,曾反叛後涼並推段業建北涼,後攻殺段業自己登位。沮渠蒙遜有勇略,在位期間,北涼於強敵環伺之際擴張成為河西一帶最強大的勢力。

沮渠原是匈奴官名,分為左沮渠與右沮渠。沮渠蒙遜出身匈奴貴族,為盧水胡領袖。

沮渠蒙遜博覽史籍,知曉天文,才智出眾又有謀略,為人圓滑又靈活變通,故前秦將領如梁熙及呂光都對其才能既感驚異,亦生畏懼。沮渠蒙遜知道後亦常飲酒出遊,故作低調。前秦亡後,蒙遜一族依附呂光建立的後涼。397年,蒙遜伯父后凉尚书沮渠罗仇和三河太守沮渠麹粥随从后凉进攻西秦的乞伏乾歸,吕光弟吕延轻敌,兵败被杀,后凉军被迫撤退。呂光以败军之罪杀罗仇、麹粥二人,蒙遜在宗族聚集參加二人喪禮的機會舉眾叛涼,斬後涼中田護軍馬邃及臨松令井祥與眾盟誓,十日之間就招合了萬多人,屯兵金山。同年,蒙遜堂兄沮渠男成擁立段業稱涼州牧,建北涼,蒙遜附之,獲授鎮西將軍、張掖太守。

398年,蒙遜深知西郡戰略價值高,遂大力支持段業進攻該郡的決定,並受命進攻。然而蒙遜攻郡城十餘日不下,改為引水灌城,終逮獲太守呂純而返,晉昌郡王德及敦煌郡孟敏戰後皆向北涼歸降。蒙遜以功封為臨池侯。同年後涼張掖守將呂弘率眾棄城東歸,蒙遜以「歸師勿遏,窮寇弗追」為理反對段業追擊,但段業不聽,終為呂弘所敗,段業更因蒙遜才得安全撤退,因而嘆道:「我沒有聽從子房的話,才會有此結果!」後蒙遜又反對段業以將領臧莫孩擔任新建西安城的太守,稱臧莫孩「勇而無謀,知進忘退」,必會失敗。段業又不聽,不久臧莫孩就被後涼呂纂擊敗。天璽元年(399年),段業稱涼王,以蒙遜為尚書左丞。不久,后凉太子呂紹及呂纂來攻,段業請得南涼禿髮烏孤派楊軌等救援,就打算迎擊,蒙遜就說:「楊軌恃著騎兵戰力強,有伺機圖謀我們的意圖。而呂紹和呂纂在死地,肯定會與我們決戰以求生。拒絕對戰將有如泰山般安穩,出戰則像疊起的蛋一樣危險。」段業同意,遂按兵拒絕接戰,後涼軍沒有辦法,亦退兵。

雖然蒙遜屢次建言協助段業,但卻害怕對方容不下自己,所以每每特意不顯露自己的智謀。段業也畏懼蒙遜的能力,故此調蒙遜為臨池太守,改以門下侍郎馬權為張掖太守。馬權得段業信任和重用,其人亦有過人軍事謀略,卻輕視並常欺侮蒙遜,令蒙遜對他又恨又怕,於是向段業進言中傷馬權,卻令段業將馬權殺死。蒙遜隨後向沮渠男成建議除去段業,改奉男成為主,但被男成拒絕。蒙遜心中不安,自求外任西安太守,也得段業批准。

不過蒙遜天玺三年(401年)四月约男成一同去祭告兰门山(甘肃省山丹县西南)時,暗中派人告诉段业说男成准备发动变乱,段業斬男成,男成死前对段业说:“蒙逊早就和臣说过他要叛乱了,只是臣以兄弟缘故才不说出来。蒙逊以臣还在,怕部众不听从他,于是约臣与其祭山,反派人诬告臣。臣若果死了,蒙逊肯定很快就起兵了。请假称臣死了,宣告臣的罪行,蒙逊肯定会起兵叛乱,而臣立即就会讨伐他,必会成功。”段业不听。蒙遜以此為藉口出兵攻段,並进屯侯坞,段业急派右将军田昂、武威将军梁中庸反击蒙逊,田昂、梁中庸至侯坞反降蒙逊,五月,蒙逊大军抵张掖(今甘肃张掖西北),田昂侄子田承爱开城门内应,蒙逊入城,殺段业,遂稱大都督、大将军、凉州牧、张掖公,改年號永安。

後秦亦在永安二年(402年)任命沮渠蒙遜為鎮西將軍、沙州刺史、西海侯。蒙遜登位後提拔人才,得文武官員支持。

蒙遜曾經送子沮渠奚念到南涼做人質,想與其結好,然而南涼主禿髮利鹿孤嫌奚念年幼,要求改以蒙遜弟沮渠挐為質。蒙遜寫信表示不願,竟惹怒利鹿孤並遭進攻,蒙遜唯有答應利鹿孤的要求。永安七年(407年),禿髮傉檀率兵五萬進攻蒙遜,蒙遜於均石擊敗傉檀,並進攻南涼西郡太守楊統。永安十年(410年),蒙遜因之前南涼枯木及胡康攻掠臨松而攻南涼,至顯美強遷數千戶人退兵。傉檀率兵追擊,並在窮泉追上蒙遜,蒙遜大敗傉檀,更乘勝攻至姑臧,萬多戶姑臧人民向蒙遜歸降。蒙遜隨後接受傉檀求和,遷八千多戶人離開。傉檀不久就遷都至樂都,焦朗等人乘勢據姑臧自立,蒙遜遂率三萬兵進攻,奪取了姑臧。412年,蒙遜遷都姑臧,稱河西王,改元玄始。

西涼在沮渠蒙遜殺段業登位前一年自立,蒙遜曾於永安十一年(411年)輕兵襲擊西涼,西涼君主李暠閉門拒戰,蒙遜撤兵時更被西涼世子李歆擊敗。至玄始六年(417年)李歆即位,蒙遜命張掖太守沮渠廣宗詐降西涼,李歆中計出兵迎接但及後卻發現蒙遜所領的三萬伏兵而撤走,蒙遜追擊卻在鮮支澗一戰中大敗予李歆。蒙遜一度想重結敗兵再戰,但為沮渠成都勸止,在增築建康城後班師。玄始九年(420年),李歆乘蒙遜攻西秦浩亹的機會進攻,蒙遜聞訊時正自浩亹回師至川巖,於是發布浩亹已下,即將進攻黃谷的假消息,讓李歆以為蒙遜仍在外,實質正暗中回援。李歆果然繼續進攻,兩軍遂於懷城決戰,李歆兵敗但不肯撤退,堅持再戰,於是在蓼泉再敗並被殺。蒙遜因而乘勢攻陷西涼都城酒泉,滅亡了西涼。次年,蒙遜率軍進攻李恂領導之西涼殘餘勢力所據的敦煌,成功攻陷,徹底滅亡西涼勢力。

朱齡石滅蜀後曾與蒙遜有使者往來,蒙遜亦上表表示其臣服於東晉,晉廷亦授予涼州刺史。玄始十年(421年),把持東晉軍政的劉裕代晉建南朝宋後,於十月任命沮渠蒙遜為使持節、散騎常侍、都督涼州諸軍事、鎮軍大將軍、開府儀同三司、涼州刺史、張掖公。玄始十二年(423年)二月,蒙遜遣使南朝宋,宋廷進蒙遜侍持節、開府、侍中、都督涼秦河沙四州諸軍事、驃騎大將軍、領護匈奴中郎將、西夷校尉、涼州牧,河西王。玄始十五年(426年)五月又獲改授車騎大將軍。承玄四年(431年),蒙遜又曾命人出使北魏,更派兒子沮渠安周入魏,北魏遂命其為假節,侍中,都督涼州西域羌戎諸軍事,太傅,行征西大將軍,涼州牧,涼王。

義和三年(433年),蒙逊去世,享年六十六,諡武宣王,庙號太祖。因他生前所立继承人沮渠菩提年幼,贵族拥立其年长之子沮渠牧犍继位。

沮渠蒙遜有軍事才能,故屢次向段業提供意見助其解兵厄,亦讓其國能立於河西諸國間。登位後,蒙遜伯父中田護軍沮渠親信及臨松太守沮渠孔篤驕橫奢侈,侵害人民,蒙遜說:「禍亂我國家的就是兩位伯父呀,還怎治理百姓呀!」於是命二人自殺。不過他用計陷害堂兄男成,接著攻殺他推舉的段業,令《晉書》評價他「見利忘義,苞禍滅親。」蒙遜知劉裕滅後秦的消息後十分憤怒,門下校郎劉詳其時有事報告,蒙遜卻回應:「你知道劉裕入關,竟敢這樣得意!」就將劉詳殺了,亦見其嚴酷殘暴一面。

據《晉書》所載,蒙遜頗信天象,並寫其多次憑天象指引而勝利。亦有載蒙遜曾祭祀西王母寺,並命中書侍郎張穆為寺內的《玄石神圖》作賦,銘於寺前;蒙遜又曾派世子沮渠興國到南朝宋借《周易》等書,又曾向南朝宋司徒王弘求《搜神記》。沮渠蒙逊曾在母车太后病重时引咎于己,同时大赦死罪以下,车太后仍然去世。当旱灾时,他也有同样举动,次日就下大雨了。

蒙遜亦信佛,其時有一名自西域東來的僧人曇無讖在涼州譯經,又「以男女交接之術教授婦人」,時蒙遜諸女及子媳都信奉他。曇無讖亦通術數和咒術,屢次準確說出其他國家的事,沮渠蒙逊遂奉昙无谶为国师,每以国事谘之。後北魏聽聞曇無讖的事跡,要求蒙遜將曇無讖送到北魏,蒙遜不肯,及後還將他殺了。

\subsubsection{永安}

\begin{longtable}{|>{\centering\scriptsize}m{2em}|>{\centering\scriptsize}m{1.3em}|>{\centering}m{8.8em}|}
  % \caption{秦王政}\
  \toprule
  \SimHei \normalsize 年数 & \SimHei \scriptsize 公元 & \SimHei 大事件 \tabularnewline
  % \midrule
  \endfirsthead
  \toprule
  \SimHei \normalsize 年数 & \SimHei \scriptsize 公元 & \SimHei 大事件 \tabularnewline
  \midrule
  \endhead
  \midrule
  元年 & 401 & \tabularnewline\hline
  二年 & 402 & \tabularnewline\hline
  三年 & 403 & \tabularnewline\hline
  四年 & 404 & \tabularnewline\hline
  五年 & 405 & \tabularnewline\hline
  六年 & 406 & \tabularnewline\hline
  七年 & 407 & \tabularnewline\hline
  八年 & 408 & \tabularnewline\hline
  九年 & 409 & \tabularnewline\hline
  十年 & 410 & \tabularnewline\hline
  十一年 & 411 & \tabularnewline\hline
  十二年 & 412 & \tabularnewline
  \bottomrule
\end{longtable}


\subsubsection{玄始}

\begin{longtable}{|>{\centering\scriptsize}m{2em}|>{\centering\scriptsize}m{1.3em}|>{\centering}m{8.8em}|}
  % \caption{秦王政}\
  \toprule
  \SimHei \normalsize 年数 & \SimHei \scriptsize 公元 & \SimHei 大事件 \tabularnewline
  % \midrule
  \endfirsthead
  \toprule
  \SimHei \normalsize 年数 & \SimHei \scriptsize 公元 & \SimHei 大事件 \tabularnewline
  \midrule
  \endhead
  \midrule
  元年 & 412 & \tabularnewline\hline
  二年 & 413 & \tabularnewline\hline
  三年 & 414 & \tabularnewline\hline
  四年 & 415 & \tabularnewline\hline
  五年 & 416 & \tabularnewline\hline
  六年 & 417 & \tabularnewline\hline
  七年 & 418 & \tabularnewline\hline
  八年 & 419 & \tabularnewline\hline
  九年 & 420 & \tabularnewline\hline
  十年 & 421 & \tabularnewline\hline
  十一年 & 422 & \tabularnewline\hline
  十二年 & 423 & \tabularnewline\hline
  十三年 & 424 & \tabularnewline\hline
  十四年 & 425 & \tabularnewline\hline
  十五年 & 426 & \tabularnewline\hline
  十六年 & 427 & \tabularnewline\hline
  十七年 & 428 & \tabularnewline
  \bottomrule
\end{longtable}

\subsubsection{承玄}

\begin{longtable}{|>{\centering\scriptsize}m{2em}|>{\centering\scriptsize}m{1.3em}|>{\centering}m{8.8em}|}
  % \caption{秦王政}\
  \toprule
  \SimHei \normalsize 年数 & \SimHei \scriptsize 公元 & \SimHei 大事件 \tabularnewline
  % \midrule
  \endfirsthead
  \toprule
  \SimHei \normalsize 年数 & \SimHei \scriptsize 公元 & \SimHei 大事件 \tabularnewline
  \midrule
  \endhead
  \midrule
  元年 & 428 & \tabularnewline\hline
  二年 & 429 & \tabularnewline\hline
  三年 & 430 & \tabularnewline\hline
  四年 & 431 & \tabularnewline
  \bottomrule
\end{longtable}

\subsubsection{义和}

\begin{longtable}{|>{\centering\scriptsize}m{2em}|>{\centering\scriptsize}m{1.3em}|>{\centering}m{8.8em}|}
  % \caption{秦王政}\
  \toprule
  \SimHei \normalsize 年数 & \SimHei \scriptsize 公元 & \SimHei 大事件 \tabularnewline
  % \midrule
  \endfirsthead
  \toprule
  \SimHei \normalsize 年数 & \SimHei \scriptsize 公元 & \SimHei 大事件 \tabularnewline
  \midrule
  \endhead
  \midrule
  元年 & 431 & \tabularnewline\hline
  二年 & 432 & \tabularnewline\hline
  三年 & 433 & \tabularnewline
  \bottomrule
\end{longtable}


%%% Local Variables:
%%% mode: latex
%%% TeX-engine: xetex
%%% TeX-master: "../../Main"
%%% End:

%% -*- coding: utf-8 -*-
%% Time-stamp: <Chen Wang: 2019-12-19 16:41:42>

\subsection{哀王\tiny(433-439)}

\subsubsection{生平}

沮渠牧犍(?-447年),一名茂虔,匈奴支系盧水胡族人,沮渠蒙遜之子。十六國時期北涼國末代君主。沮渠牧犍原非蒙遜指定的繼承人,因國內眾臣推舉而登位,任內保持了父親一貫與北魏及南朝宋的關係,然而北魏既滅北燕,魏太武帝亦因毒殺武威公主圖謀和西域使者之言對牧犍不滿,遂出兵攻涼。牧犍初堅守姑臧城不降,但終在北魏軍圍攻下城陷,被逼投降,北涼亡。牧犍弟沮渠無諱帶領北涼殘餘勢力西走,後立起高昌北涼以承涼祚。

沮渠牧犍生于368年,歷任酒泉太守及敦煌太守,北涼義和三年(433年)沮渠蒙遜去世,因继承人沮渠菩提尚幼,眾臣在蒙遜病重時就推較年長的沮渠牧犍為世子,加中外都督、大將軍、錄尚書事。牧犍在蒙遜死後即襲河西王位,改元永和。牧犍隨後向北魏請求任命,獲授都督涼河沙三州西域諸羌戎諸軍事、車騎將軍、開府儀同三司、涼州刺史、河西王。又以父親遺願為由,將妹妹興平公主嫁給魏太武帝拓跋焘。另一方面,牧犍亦向南朝宋上表告知繼位一事,又獲授持節、散騎常侍、都督涼秦河沙四州諸軍事、征西大將軍、領護匈奴中郎將、西夷校尉、涼州刺史、河西王。

永和五年(437年),拓跋焘將其妹武威公主嫁予牧犍,牧犍與嫂子李氏偷情,李氏既得寵,竟與牧犍之姊共毒殺武威公主,幸得解藥不死。拓跋燾要求押解李氏至北魏,牧犍不肯,把李氏安置到酒泉。另北魏西域使者從北涼官員口中得知柔然可汗宣稱他們擊敗了北魏及牧犍聞言大喜並向國內宣傳的事,並向拓跋燾報告,拓跋燾特意派尚書賀多羅去探聽北涼國內的情況,賀多羅回來時亦稱牧犍表面上臣服於魏,實質上並不服從。拓跋焘遂於永和七年(439年)下詔列牧犍十二項罪狀並大舉進攻北涼,詔中亦勸導牧犍自動請降。牧犍聞訊大驚,聽從左丞姚定國計謀不出城迎降,反向柔然求援並命弟弟沮渠董來率兵在城南抗擊魏軍,可是董來軍卻望風潰敗。魏軍兵臨姑臧,牧犍當時聽聞柔然會進攻北魏,於是期望魏軍會因而東還,故此決意固守不降。不過,當時牧犍侄沮渠祖出降並將牧犍的想法告知拓跋燾,拓跋壽遂分兵圍困姑臧,又派源賀招撫北涼諸部,以專心攻城。姑臧最終失守,沮渠牧犍率領文武百官五千人面缚请降,北涼亡,北魏遂統一北方。

拓跋燾將沮渠牧犍及其宗族官民共三萬戶遷至魏都平城,仍以妹婿身份對待他,仍任征西大將軍及河西王爵。北魏太平真君八年(447年),牧犍親族及守護國庫者告發牧犍在姑臧城陷前將國庫中的金銀財寶都拿走,其餘則任由平民搶奪,最終魏人在牧犍家中果然搜得那些財寶;牧犍父子又被指曾毒死数以百计的无辜者,同时在他家找到毒药,姐妹又習曇無讖之術,行為放蕩無愧色;還有指牧犍與北涼舊臣聯絡,意圖謀反,太武帝派遣太常卿崔浩至牧犍家中,将其賜死,諡哀王,其他宗族除沮渠祖外亦被處死。

自西晉末年大亂,不少中原文士都去河西一帶避亂,前涼張氏主政時亦禮遇他們,故涼州文士傳承,號稱「多士」。牧犍亦喜好文學,任用不少文士。任內又曾獻書南朝,亦向南朝求晉、趙《起居注》等書。

\subsubsection{承和}

\begin{longtable}{|>{\centering\scriptsize}m{2em}|>{\centering\scriptsize}m{1.3em}|>{\centering}m{8.8em}|}
  % \caption{秦王政}\
  \toprule
  \SimHei \normalsize 年数 & \SimHei \scriptsize 公元 & \SimHei 大事件 \tabularnewline
  % \midrule
  \endfirsthead
  \toprule
  \SimHei \normalsize 年数 & \SimHei \scriptsize 公元 & \SimHei 大事件 \tabularnewline
  \midrule
  \endhead
  \midrule
  元年 & 433 & \tabularnewline\hline
  二年 & 434 & \tabularnewline\hline
  三年 & 435 & \tabularnewline\hline
  四年 & 436 & \tabularnewline\hline
  五年 & 437 & \tabularnewline\hline
  六年 & 438 & \tabularnewline\hline
  七年 & 439 & \tabularnewline
  \bottomrule
\end{longtable}


%%% Local Variables:
%%% mode: latex
%%% TeX-engine: xetex
%%% TeX-master: "../../Main"
%%% End:



%%% Local Variables:
%%% mode: latex
%%% TeX-engine: xetex
%%% TeX-master: "../../Main"
%%% End:


%%% Local Variables:
%%% mode: latex
%%% TeX-engine: xetex
%%% TeX-master: "../Main"
%%% End:
 % 十六国
% %% -*- coding: utf-8 -*-
%% Time-stamp: <Chen Wang: 2019-10-15 11:13:37>

\chapter{南北朝\tiny(420-589)}


%% -*- coding: utf-8 -*-
%% Time-stamp: <Chen Wang: 2019-10-15 11:14:12>


\section{刘宋\tiny(420-479)}

%% -*- coding: utf-8 -*-
%% Time-stamp: <Chen Wang: 2018-07-11 16:17:45>

\subsection{武帝\tiny(420-422)}

\subsubsection{永初}

\begin{longtable}{|>{\centering\scriptsize}m{2em}|>{\centering\scriptsize}m{1.3em}|>{\centering}m{8.8em}|}
  % \caption{秦王政}\
  \toprule
  \SimHei \normalsize 年数 & \SimHei \scriptsize 公元 & \SimHei 大事件 \tabularnewline
  % \midrule
  \endfirsthead
  \toprule
  \SimHei \normalsize 年数 & \SimHei \scriptsize 公元 & \SimHei 大事件 \tabularnewline
  \midrule
  \endhead
  \midrule
  元年 & 420 & \tabularnewline\hline
  二年 & 421 & \tabularnewline\hline
  三年 & 422 & \tabularnewline
  \bottomrule
\end{longtable}


%%% Local Variables:
%%% mode: latex
%%% TeX-engine: xetex
%%% TeX-master: "../../Main"
%%% End:

%% -*- coding: utf-8 -*-
%% Time-stamp: <Chen Wang: 2018-07-11 18:26:23>

\subsection{刘义符\tiny(422-424)}

\subsubsection{景平}

\begin{longtable}{|>{\centering\scriptsize}m{2em}|>{\centering\scriptsize}m{1.3em}|>{\centering}m{8.8em}|}
  % \caption{秦王政}\
  \toprule
  \SimHei \normalsize 年数 & \SimHei \scriptsize 公元 & \SimHei 大事件 \tabularnewline
  % \midrule
  \endfirsthead
  \toprule
  \SimHei \normalsize 年数 & \SimHei \scriptsize 公元 & \SimHei 大事件 \tabularnewline
  \midrule
  \endhead
  \midrule
  元年 & 423 & \tabularnewline\hline
  二年 & 424 & \tabularnewline
  \bottomrule
\end{longtable}


%%% Local Variables:
%%% mode: latex
%%% TeX-engine: xetex
%%% TeX-master: "../../Main"
%%% End:

%% -*- coding: utf-8 -*-
%% Time-stamp: <Chen Wang: 2018-07-11 18:28:16>

\subsection{文帝\tiny(424-453)}

\subsubsection{元嘉}

\begin{longtable}{|>{\centering\scriptsize}m{2em}|>{\centering\scriptsize}m{1.3em}|>{\centering}m{8.8em}|}
  % \caption{秦王政}\
  \toprule
  \SimHei \normalsize 年数 & \SimHei \scriptsize 公元 & \SimHei 大事件 \tabularnewline
  % \midrule
  \endfirsthead
  \toprule
  \SimHei \normalsize 年数 & \SimHei \scriptsize 公元 & \SimHei 大事件 \tabularnewline
  \midrule
  \endhead
  \midrule
  元年 & 424 & \tabularnewline\hline
  二年 & 425 & \tabularnewline\hline
  三年 & 426 & \tabularnewline\hline
  四年 & 427 & \tabularnewline\hline
  五年 & 428 & \tabularnewline\hline
  六年 & 429 & \tabularnewline\hline
  七年 & 430 & \tabularnewline\hline
  八年 & 431 & \tabularnewline\hline
  九年 & 432 & \tabularnewline\hline
  十年 & 433 & \tabularnewline\hline
  十一年 & 434 & \tabularnewline\hline
  十二年 & 435 & \tabularnewline\hline
  十三年 & 436 & \tabularnewline\hline
  十四年 & 437 & \tabularnewline\hline
  十五年 & 438 & \tabularnewline\hline
  十六年 & 439 & \tabularnewline\hline
  十七年 & 440 & \tabularnewline\hline
  十八年 & 441 & \tabularnewline\hline
  十九年 & 442 & \tabularnewline\hline
  二十年 & 443 & \tabularnewline\hline
  二一年 & 444 & \tabularnewline\hline
  二二年 & 445 & \tabularnewline\hline
  二三年 & 446 & \tabularnewline\hline
  二四年 & 447 & \tabularnewline\hline
  二五年 & 448 & \tabularnewline\hline
  二六年 & 449 & \tabularnewline\hline
  二七年 & 450 & \tabularnewline\hline
  二八年 & 451 & \tabularnewline\hline
  二九年 & 452 & \tabularnewline\hline
  三十年 & 453 & \tabularnewline\hline
  \bottomrule
\end{longtable}


%%% Local Variables:
%%% mode: latex
%%% TeX-engine: xetex
%%% TeX-master: "../../Main"
%%% End:

%% -*- coding: utf-8 -*-
%% Time-stamp: <Chen Wang: 2018-07-11 19:35:07>

\subsection{孝武帝\tiny(532-534)}

\subsubsection{太昌}

\begin{longtable}{|>{\centering\scriptsize}m{2em}|>{\centering\scriptsize}m{1.3em}|>{\centering}m{8.8em}|}
  % \caption{秦王政}\
  \toprule
  \SimHei \normalsize 年数 & \SimHei \scriptsize 公元 & \SimHei 大事件 \tabularnewline
  % \midrule
  \endfirsthead
  \toprule
  \SimHei \normalsize 年数 & \SimHei \scriptsize 公元 & \SimHei 大事件 \tabularnewline
  \midrule
  \endhead
  \midrule
  元年 & 532 & \tabularnewline
  \bottomrule
\end{longtable}

\subsubsection{永兴}

\begin{longtable}{|>{\centering\scriptsize}m{2em}|>{\centering\scriptsize}m{1.3em}|>{\centering}m{8.8em}|}
  % \caption{秦王政}\
  \toprule
  \SimHei \normalsize 年数 & \SimHei \scriptsize 公元 & \SimHei 大事件 \tabularnewline
  % \midrule
  \endfirsthead
  \toprule
  \SimHei \normalsize 年数 & \SimHei \scriptsize 公元 & \SimHei 大事件 \tabularnewline
  \midrule
  \endhead
  \midrule
  元年 & 532 & \tabularnewline
  \bottomrule
\end{longtable}

\subsubsection{永熙}

\begin{longtable}{|>{\centering\scriptsize}m{2em}|>{\centering\scriptsize}m{1.3em}|>{\centering}m{8.8em}|}
  % \caption{秦王政}\
  \toprule
  \SimHei \normalsize 年数 & \SimHei \scriptsize 公元 & \SimHei 大事件 \tabularnewline
  % \midrule
  \endfirsthead
  \toprule
  \SimHei \normalsize 年数 & \SimHei \scriptsize 公元 & \SimHei 大事件 \tabularnewline
  \midrule
  \endhead
  \midrule
  元年 & 532 & \tabularnewline\hline
  二年 & 533 & \tabularnewline\hline
  三年 & 534 & \tabularnewline
  \bottomrule
\end{longtable}


%%% Local Variables:
%%% mode: latex
%%% TeX-engine: xetex
%%% TeX-master: "../../Main"
%%% End:

%% -*- coding: utf-8 -*-
%% Time-stamp: <Chen Wang: 2018-07-11 18:30:51>

\subsection{刘子业\tiny(464-465)}

\subsubsection{永光}

\begin{longtable}{|>{\centering\scriptsize}m{2em}|>{\centering\scriptsize}m{1.3em}|>{\centering}m{8.8em}|}
  % \caption{秦王政}\
  \toprule
  \SimHei \normalsize 年数 & \SimHei \scriptsize 公元 & \SimHei 大事件 \tabularnewline
  % \midrule
  \endfirsthead
  \toprule
  \SimHei \normalsize 年数 & \SimHei \scriptsize 公元 & \SimHei 大事件 \tabularnewline
  \midrule
  \endhead
  \midrule
  元年 & 465 & \tabularnewline
  \bottomrule
\end{longtable}

\subsubsection{景和}

\begin{longtable}{|>{\centering\scriptsize}m{2em}|>{\centering\scriptsize}m{1.3em}|>{\centering}m{8.8em}|}
  % \caption{秦王政}\
  \toprule
  \SimHei \normalsize 年数 & \SimHei \scriptsize 公元 & \SimHei 大事件 \tabularnewline
  % \midrule
  \endfirsthead
  \toprule
  \SimHei \normalsize 年数 & \SimHei \scriptsize 公元 & \SimHei 大事件 \tabularnewline
  \midrule
  \endhead
  \midrule
  元年 & 465 & \tabularnewline
  \bottomrule
\end{longtable}


%%% Local Variables:
%%% mode: latex
%%% TeX-engine: xetex
%%% TeX-master: "../../Main"
%%% End:

%% -*- coding: utf-8 -*-
%% Time-stamp: <Chen Wang: 2018-07-11 18:31:56>

\subsection{明帝\tiny(465-472)}

\subsubsection{泰始}

\begin{longtable}{|>{\centering\scriptsize}m{2em}|>{\centering\scriptsize}m{1.3em}|>{\centering}m{8.8em}|}
  % \caption{秦王政}\
  \toprule
  \SimHei \normalsize 年数 & \SimHei \scriptsize 公元 & \SimHei 大事件 \tabularnewline
  % \midrule
  \endfirsthead
  \toprule
  \SimHei \normalsize 年数 & \SimHei \scriptsize 公元 & \SimHei 大事件 \tabularnewline
  \midrule
  \endhead
  \midrule
  元年 & 465 & \tabularnewline\hline
  二年 & 466 & \tabularnewline\hline
  三年 & 467 & \tabularnewline\hline
  四年 & 468 & \tabularnewline\hline
  五年 & 469 & \tabularnewline\hline
  六年 & 470 & \tabularnewline\hline
  七年 & 471 & \tabularnewline
  \bottomrule
\end{longtable}

\subsubsection{泰豫}

\begin{longtable}{|>{\centering\scriptsize}m{2em}|>{\centering\scriptsize}m{1.3em}|>{\centering}m{8.8em}|}
  % \caption{秦王政}\
  \toprule
  \SimHei \normalsize 年数 & \SimHei \scriptsize 公元 & \SimHei 大事件 \tabularnewline
  % \midrule
  \endfirsthead
  \toprule
  \SimHei \normalsize 年数 & \SimHei \scriptsize 公元 & \SimHei 大事件 \tabularnewline
  \midrule
  \endhead
  \midrule
  元年 & 472 & \tabularnewline
  \bottomrule
\end{longtable}


%%% Local Variables:
%%% mode: latex
%%% TeX-engine: xetex
%%% TeX-master: "../../Main"
%%% End:

%% -*- coding: utf-8 -*-
%% Time-stamp: <Chen Wang: 2018-07-11 18:32:53>

\subsection{刘昱\tiny(472-477)}

\subsubsection{元徽}

\begin{longtable}{|>{\centering\scriptsize}m{2em}|>{\centering\scriptsize}m{1.3em}|>{\centering}m{8.8em}|}
  % \caption{秦王政}\
  \toprule
  \SimHei \normalsize 年数 & \SimHei \scriptsize 公元 & \SimHei 大事件 \tabularnewline
  % \midrule
  \endfirsthead
  \toprule
  \SimHei \normalsize 年数 & \SimHei \scriptsize 公元 & \SimHei 大事件 \tabularnewline
  \midrule
  \endhead
  \midrule
  元年 & 473 & \tabularnewline\hline
  二年 & 474 & \tabularnewline\hline
  三年 & 475 & \tabularnewline\hline
  四年 & 476 & \tabularnewline\hline
  五年 & 477 & \tabularnewline
  \bottomrule
\end{longtable}


%%% Local Variables:
%%% mode: latex
%%% TeX-engine: xetex
%%% TeX-master: "../../Main"
%%% End:

%% -*- coding: utf-8 -*-
%% Time-stamp: <Chen Wang: 2018-07-11 18:33:31>

\subsection{顺帝\tiny(477-479)}

\subsubsection{昇明}

\begin{longtable}{|>{\centering\scriptsize}m{2em}|>{\centering\scriptsize}m{1.3em}|>{\centering}m{8.8em}|}
  % \caption{秦王政}\
  \toprule
  \SimHei \normalsize 年数 & \SimHei \scriptsize 公元 & \SimHei 大事件 \tabularnewline
  % \midrule
  \endfirsthead
  \toprule
  \SimHei \normalsize 年数 & \SimHei \scriptsize 公元 & \SimHei 大事件 \tabularnewline
  \midrule
  \endhead
  \midrule
  元年 & 477 & \tabularnewline\hline
  二年 & 478 & \tabularnewline\hline
  三年 & 479 & \tabularnewline
  \bottomrule
\end{longtable}


%%% Local Variables:
%%% mode: latex
%%% TeX-engine: xetex
%%% TeX-master: "../../Main"
%%% End:



%%% Local Variables:
%%% mode: latex
%%% TeX-engine: xetex
%%% TeX-master: "../../Main"
%%% End:

%% -*- coding: utf-8 -*-
%% Time-stamp: <Chen Wang: 2019-10-15 11:14:34>


\section{南齐\tiny(479-502)}

%% -*- coding: utf-8 -*-
%% Time-stamp: <Chen Wang: 2018-07-11 18:37:14>

\subsection{高帝\tiny(479-482)}

\subsubsection{建元}

\begin{longtable}{|>{\centering\scriptsize}m{2em}|>{\centering\scriptsize}m{1.3em}|>{\centering}m{8.8em}|}
  % \caption{秦王政}\
  \toprule
  \SimHei \normalsize 年数 & \SimHei \scriptsize 公元 & \SimHei 大事件 \tabularnewline
  % \midrule
  \endfirsthead
  \toprule
  \SimHei \normalsize 年数 & \SimHei \scriptsize 公元 & \SimHei 大事件 \tabularnewline
  \midrule
  \endhead
  \midrule
  元年 & 479 & \tabularnewline\hline
  二年 & 480 & \tabularnewline\hline
  三年 & 481 & \tabularnewline\hline
  四年 & 482 & \tabularnewline
  \bottomrule
\end{longtable}


%%% Local Variables:
%%% mode: latex
%%% TeX-engine: xetex
%%% TeX-master: "../../Main"
%%% End:

%% -*- coding: utf-8 -*-
%% Time-stamp: <Chen Wang: 2018-07-11 18:38:10>

\subsection{武帝\tiny(482-493)}

\subsubsection{永明}

\begin{longtable}{|>{\centering\scriptsize}m{2em}|>{\centering\scriptsize}m{1.3em}|>{\centering}m{8.8em}|}
  % \caption{秦王政}\
  \toprule
  \SimHei \normalsize 年数 & \SimHei \scriptsize 公元 & \SimHei 大事件 \tabularnewline
  % \midrule
  \endfirsthead
  \toprule
  \SimHei \normalsize 年数 & \SimHei \scriptsize 公元 & \SimHei 大事件 \tabularnewline
  \midrule
  \endhead
  \midrule
  元年 & 483 & \tabularnewline\hline
  二年 & 484 & \tabularnewline\hline
  三年 & 485 & \tabularnewline\hline
  四年 & 486 & \tabularnewline\hline
  五年 & 487 & \tabularnewline\hline
  六年 & 488 & \tabularnewline\hline
  七年 & 489 & \tabularnewline\hline
  八年 & 490 & \tabularnewline\hline
  九年 & 491 & \tabularnewline\hline
  十年 & 492 & \tabularnewline\hline
  十一年 & 493 & \tabularnewline
  \bottomrule
\end{longtable}


%%% Local Variables:
%%% mode: latex
%%% TeX-engine: xetex
%%% TeX-master: "../../Main"
%%% End:

%% -*- coding: utf-8 -*-
%% Time-stamp: <Chen Wang: 2018-07-11 18:39:01>

\subsection{萧昭业\tiny(493-494)}

\subsubsection{隆昌}

\begin{longtable}{|>{\centering\scriptsize}m{2em}|>{\centering\scriptsize}m{1.3em}|>{\centering}m{8.8em}|}
  % \caption{秦王政}\
  \toprule
  \SimHei \normalsize 年数 & \SimHei \scriptsize 公元 & \SimHei 大事件 \tabularnewline
  % \midrule
  \endfirsthead
  \toprule
  \SimHei \normalsize 年数 & \SimHei \scriptsize 公元 & \SimHei 大事件 \tabularnewline
  \midrule
  \endhead
  \midrule
  元年 & 494 & \tabularnewline
  \bottomrule
\end{longtable}


%%% Local Variables:
%%% mode: latex
%%% TeX-engine: xetex
%%% TeX-master: "../../Main"
%%% End:

%% -*- coding: utf-8 -*-
%% Time-stamp: <Chen Wang: 2018-07-11 18:39:32>

\subsection{萧昭文\tiny(494)}

\subsubsection{延兴}

\begin{longtable}{|>{\centering\scriptsize}m{2em}|>{\centering\scriptsize}m{1.3em}|>{\centering}m{8.8em}|}
  % \caption{秦王政}\
  \toprule
  \SimHei \normalsize 年数 & \SimHei \scriptsize 公元 & \SimHei 大事件 \tabularnewline
  % \midrule
  \endfirsthead
  \toprule
  \SimHei \normalsize 年数 & \SimHei \scriptsize 公元 & \SimHei 大事件 \tabularnewline
  \midrule
  \endhead
  \midrule
  元年 & 494 & \tabularnewline
  \bottomrule
\end{longtable}


%%% Local Variables:
%%% mode: latex
%%% TeX-engine: xetex
%%% TeX-master: "../../Main"
%%% End:

%% -*- coding: utf-8 -*-
%% Time-stamp: <Chen Wang: 2018-07-11 18:40:28>

\subsection{明帝\tiny(494-498)}

\subsubsection{建武}

\begin{longtable}{|>{\centering\scriptsize}m{2em}|>{\centering\scriptsize}m{1.3em}|>{\centering}m{8.8em}|}
  % \caption{秦王政}\
  \toprule
  \SimHei \normalsize 年数 & \SimHei \scriptsize 公元 & \SimHei 大事件 \tabularnewline
  % \midrule
  \endfirsthead
  \toprule
  \SimHei \normalsize 年数 & \SimHei \scriptsize 公元 & \SimHei 大事件 \tabularnewline
  \midrule
  \endhead
  \midrule
  元年 & 494 & \tabularnewline\hline
  二年 & 495 & \tabularnewline\hline
  三年 & 496 & \tabularnewline\hline
  四年 & 497 & \tabularnewline\hline
  五年 & 498 & \tabularnewline
  \bottomrule
\end{longtable}

\subsubsection{永泰}

\begin{longtable}{|>{\centering\scriptsize}m{2em}|>{\centering\scriptsize}m{1.3em}|>{\centering}m{8.8em}|}
  % \caption{秦王政}\
  \toprule
  \SimHei \normalsize 年数 & \SimHei \scriptsize 公元 & \SimHei 大事件 \tabularnewline
  % \midrule
  \endfirsthead
  \toprule
  \SimHei \normalsize 年数 & \SimHei \scriptsize 公元 & \SimHei 大事件 \tabularnewline
  \midrule
  \endhead
  \midrule
  元年 & 498 & \tabularnewline
  \bottomrule
\end{longtable}


%%% Local Variables:
%%% mode: latex
%%% TeX-engine: xetex
%%% TeX-master: "../../Main"
%%% End:

%% -*- coding: utf-8 -*-
%% Time-stamp: <Chen Wang: 2018-07-11 18:41:12>

\subsection{萧宝卷\tiny(498-501)}

\subsubsection{永元}

\begin{longtable}{|>{\centering\scriptsize}m{2em}|>{\centering\scriptsize}m{1.3em}|>{\centering}m{8.8em}|}
  % \caption{秦王政}\
  \toprule
  \SimHei \normalsize 年数 & \SimHei \scriptsize 公元 & \SimHei 大事件 \tabularnewline
  % \midrule
  \endfirsthead
  \toprule
  \SimHei \normalsize 年数 & \SimHei \scriptsize 公元 & \SimHei 大事件 \tabularnewline
  \midrule
  \endhead
  \midrule
  元年 & 499 & \tabularnewline\hline
  二年 & 500 & \tabularnewline\hline
  三年 & 501 & \tabularnewline
  \bottomrule
\end{longtable}


%%% Local Variables:
%%% mode: latex
%%% TeX-engine: xetex
%%% TeX-master: "../../Main"
%%% End:

%% -*- coding: utf-8 -*-
%% Time-stamp: <Chen Wang: 2018-07-11 18:41:55>

\subsection{和帝\tiny(501-502)}

\subsubsection{中兴}

\begin{longtable}{|>{\centering\scriptsize}m{2em}|>{\centering\scriptsize}m{1.3em}|>{\centering}m{8.8em}|}
  % \caption{秦王政}\
  \toprule
  \SimHei \normalsize 年数 & \SimHei \scriptsize 公元 & \SimHei 大事件 \tabularnewline
  % \midrule
  \endfirsthead
  \toprule
  \SimHei \normalsize 年数 & \SimHei \scriptsize 公元 & \SimHei 大事件 \tabularnewline
  \midrule
  \endhead
  \midrule
  元年 & 501 & \tabularnewline\hline
  二年 & 502 & \tabularnewline
  \bottomrule
\end{longtable}


%%% Local Variables:
%%% mode: latex
%%% TeX-engine: xetex
%%% TeX-master: "../../Main"
%%% End:



%%% Local Variables:
%%% mode: latex
%%% TeX-engine: xetex
%%% TeX-master: "../../Main"
%%% End:

%% -*- coding: utf-8 -*-
%% Time-stamp: <Chen Wang: 2019-10-15 11:14:27>


\section{南梁\tiny(502-557)}

%% -*- coding: utf-8 -*-
%% Time-stamp: <Chen Wang: 2018-07-11 18:54:07>

\subsection{武帝\tiny(502-549)}

\subsubsection{天监}

\begin{longtable}{|>{\centering\scriptsize}m{2em}|>{\centering\scriptsize}m{1.3em}|>{\centering}m{8.8em}|}
  % \caption{秦王政}\
  \toprule
  \SimHei \normalsize 年数 & \SimHei \scriptsize 公元 & \SimHei 大事件 \tabularnewline
  % \midrule
  \endfirsthead
  \toprule
  \SimHei \normalsize 年数 & \SimHei \scriptsize 公元 & \SimHei 大事件 \tabularnewline
  \midrule
  \endhead
  \midrule
  元年 & 502 & \tabularnewline\hline
  二年 & 503 & \tabularnewline\hline
  三年 & 504 & \tabularnewline\hline
  四年 & 505 & \tabularnewline\hline
  五年 & 506 & \tabularnewline\hline
  六年 & 507 & \tabularnewline\hline
  七年 & 508 & \tabularnewline\hline
  八年 & 509 & \tabularnewline\hline
  九年 & 510 & \tabularnewline\hline
  十年 & 511 & \tabularnewline\hline
  十一年 & 512 & \tabularnewline\hline
  十二年 & 513 & \tabularnewline\hline
  十三年 & 514 & \tabularnewline\hline
  十四年 & 515 & \tabularnewline\hline
  十五年 & 516 & \tabularnewline\hline
  十六年 & 517 & \tabularnewline\hline
  十七年 & 518 & \tabularnewline\hline
  十八年 & 519 & \tabularnewline
  \bottomrule
\end{longtable}

\subsubsection{普通}

\begin{longtable}{|>{\centering\scriptsize}m{2em}|>{\centering\scriptsize}m{1.3em}|>{\centering}m{8.8em}|}
  % \caption{秦王政}\
  \toprule
  \SimHei \normalsize 年数 & \SimHei \scriptsize 公元 & \SimHei 大事件 \tabularnewline
  % \midrule
  \endfirsthead
  \toprule
  \SimHei \normalsize 年数 & \SimHei \scriptsize 公元 & \SimHei 大事件 \tabularnewline
  \midrule
  \endhead
  \midrule
  元年 & 520 & \tabularnewline\hline
  二年 & 521 & \tabularnewline\hline
  三年 & 522 & \tabularnewline\hline
  四年 & 523 & \tabularnewline\hline
  五年 & 524 & \tabularnewline\hline
  六年 & 525 & \tabularnewline\hline
  七年 & 526 & \tabularnewline\hline
  八年 & 527 & \tabularnewline
  \bottomrule
\end{longtable}

\subsubsection{大通}

\begin{longtable}{|>{\centering\scriptsize}m{2em}|>{\centering\scriptsize}m{1.3em}|>{\centering}m{8.8em}|}
  % \caption{秦王政}\
  \toprule
  \SimHei \normalsize 年数 & \SimHei \scriptsize 公元 & \SimHei 大事件 \tabularnewline
  % \midrule
  \endfirsthead
  \toprule
  \SimHei \normalsize 年数 & \SimHei \scriptsize 公元 & \SimHei 大事件 \tabularnewline
  \midrule
  \endhead
  \midrule
  元年 & 527 & \tabularnewline\hline
  二年 & 528 & \tabularnewline\hline
  三年 & 529 & \tabularnewline
  \bottomrule
\end{longtable}

\subsubsection{中大通}

\begin{longtable}{|>{\centering\scriptsize}m{2em}|>{\centering\scriptsize}m{1.3em}|>{\centering}m{8.8em}|}
  % \caption{秦王政}\
  \toprule
  \SimHei \normalsize 年数 & \SimHei \scriptsize 公元 & \SimHei 大事件 \tabularnewline
  % \midrule
  \endfirsthead
  \toprule
  \SimHei \normalsize 年数 & \SimHei \scriptsize 公元 & \SimHei 大事件 \tabularnewline
  \midrule
  \endhead
  \midrule
  元年 & 529 & \tabularnewline\hline
  二年 & 530 & \tabularnewline\hline
  三年 & 531 & \tabularnewline\hline
  四年 & 532 & \tabularnewline\hline
  五年 & 533 & \tabularnewline\hline
  六年 & 534 & \tabularnewline
  \bottomrule
\end{longtable}

\subsubsection{大同}

\begin{longtable}{|>{\centering\scriptsize}m{2em}|>{\centering\scriptsize}m{1.3em}|>{\centering}m{8.8em}|}
  % \caption{秦王政}\
  \toprule
  \SimHei \normalsize 年数 & \SimHei \scriptsize 公元 & \SimHei 大事件 \tabularnewline
  % \midrule
  \endfirsthead
  \toprule
  \SimHei \normalsize 年数 & \SimHei \scriptsize 公元 & \SimHei 大事件 \tabularnewline
  \midrule
  \endhead
  \midrule
  元年 & 535 & \tabularnewline\hline
  二年 & 536 & \tabularnewline\hline
  三年 & 537 & \tabularnewline\hline
  四年 & 538 & \tabularnewline\hline
  五年 & 539 & \tabularnewline\hline
  六年 & 540 & \tabularnewline\hline
  七年 & 541 & \tabularnewline\hline
  八年 & 542 & \tabularnewline\hline
  九年 & 543 & \tabularnewline\hline
  十年 & 544 & \tabularnewline\hline
  十一年 & 545 & \tabularnewline\hline
  十二年 & 546 & \tabularnewline
  \bottomrule
\end{longtable}

\subsubsection{中大同}

\begin{longtable}{|>{\centering\scriptsize}m{2em}|>{\centering\scriptsize}m{1.3em}|>{\centering}m{8.8em}|}
  % \caption{秦王政}\
  \toprule
  \SimHei \normalsize 年数 & \SimHei \scriptsize 公元 & \SimHei 大事件 \tabularnewline
  % \midrule
  \endfirsthead
  \toprule
  \SimHei \normalsize 年数 & \SimHei \scriptsize 公元 & \SimHei 大事件 \tabularnewline
  \midrule
  \endhead
  \midrule
  元年 & 546 & \tabularnewline\hline
  二年 & 547 & \tabularnewline
  \bottomrule
\end{longtable}

\subsubsection{太清}

\begin{longtable}{|>{\centering\scriptsize}m{2em}|>{\centering\scriptsize}m{1.3em}|>{\centering}m{8.8em}|}
  % \caption{秦王政}\
  \toprule
  \SimHei \normalsize 年数 & \SimHei \scriptsize 公元 & \SimHei 大事件 \tabularnewline
  % \midrule
  \endfirsthead
  \toprule
  \SimHei \normalsize 年数 & \SimHei \scriptsize 公元 & \SimHei 大事件 \tabularnewline
  \midrule
  \endhead
  \midrule
  元年 & 547 & \tabularnewline\hline
  二年 & 548 & \tabularnewline\hline
  三年 & 549 & \tabularnewline
  \bottomrule
\end{longtable}


%%% Local Variables:
%%% mode: latex
%%% TeX-engine: xetex
%%% TeX-master: "../../Main"
%%% End:

%% -*- coding: utf-8 -*-
%% Time-stamp: <Chen Wang: 2018-07-11 18:54:50>

\subsection{简文帝\tiny(549-551)}

\subsubsection{大宝}

\begin{longtable}{|>{\centering\scriptsize}m{2em}|>{\centering\scriptsize}m{1.3em}|>{\centering}m{8.8em}|}
  % \caption{秦王政}\
  \toprule
  \SimHei \normalsize 年数 & \SimHei \scriptsize 公元 & \SimHei 大事件 \tabularnewline
  % \midrule
  \endfirsthead
  \toprule
  \SimHei \normalsize 年数 & \SimHei \scriptsize 公元 & \SimHei 大事件 \tabularnewline
  \midrule
  \endhead
  \midrule
  元年 & 550 & \tabularnewline\hline
  二年 & 551 & \tabularnewline
  \bottomrule
\end{longtable}


%%% Local Variables:
%%% mode: latex
%%% TeX-engine: xetex
%%% TeX-master: "../../Main"
%%% End:

%% -*- coding: utf-8 -*-
%% Time-stamp: <Chen Wang: 2018-07-11 18:55:21>

\subsection{萧栋\tiny(551)}

\subsubsection{天正}

\begin{longtable}{|>{\centering\scriptsize}m{2em}|>{\centering\scriptsize}m{1.3em}|>{\centering}m{8.8em}|}
  % \caption{秦王政}\
  \toprule
  \SimHei \normalsize 年数 & \SimHei \scriptsize 公元 & \SimHei 大事件 \tabularnewline
  % \midrule
  \endfirsthead
  \toprule
  \SimHei \normalsize 年数 & \SimHei \scriptsize 公元 & \SimHei 大事件 \tabularnewline
  \midrule
  \endhead
  \midrule
  元年 & 551 & \tabularnewline
  \bottomrule
\end{longtable}


%%% Local Variables:
%%% mode: latex
%%% TeX-engine: xetex
%%% TeX-master: "../../Main"
%%% End:

%% -*- coding: utf-8 -*-
%% Time-stamp: <Chen Wang: 2018-07-11 18:56:09>

\subsection{元帝\tiny(552-554)}

\subsubsection{承圣}

\begin{longtable}{|>{\centering\scriptsize}m{2em}|>{\centering\scriptsize}m{1.3em}|>{\centering}m{8.8em}|}
  % \caption{秦王政}\
  \toprule
  \SimHei \normalsize 年数 & \SimHei \scriptsize 公元 & \SimHei 大事件 \tabularnewline
  % \midrule
  \endfirsthead
  \toprule
  \SimHei \normalsize 年数 & \SimHei \scriptsize 公元 & \SimHei 大事件 \tabularnewline
  \midrule
  \endhead
  \midrule
  元年 & 552 & \tabularnewline\hline
  二年 & 553 & \tabularnewline\hline
  三年 & 554 & \tabularnewline\hline
  四年 & 555 & \tabularnewline
  \bottomrule
\end{longtable}


%%% Local Variables:
%%% mode: latex
%%% TeX-engine: xetex
%%% TeX-master: "../../Main"
%%% End:

%% -*- coding: utf-8 -*-
%% Time-stamp: <Chen Wang: 2018-07-11 18:56:43>

\subsection{闵帝\tiny(555)}

\subsubsection{天成}

\begin{longtable}{|>{\centering\scriptsize}m{2em}|>{\centering\scriptsize}m{1.3em}|>{\centering}m{8.8em}|}
  % \caption{秦王政}\
  \toprule
  \SimHei \normalsize 年数 & \SimHei \scriptsize 公元 & \SimHei 大事件 \tabularnewline
  % \midrule
  \endfirsthead
  \toprule
  \SimHei \normalsize 年数 & \SimHei \scriptsize 公元 & \SimHei 大事件 \tabularnewline
  \midrule
  \endhead
  \midrule
  元年 & 555 & \tabularnewline
  \bottomrule
\end{longtable}


%%% Local Variables:
%%% mode: latex
%%% TeX-engine: xetex
%%% TeX-master: "../../Main"
%%% End:

%% -*- coding: utf-8 -*-
%% Time-stamp: <Chen Wang: 2018-07-11 18:57:33>

\subsection{敬帝\tiny(555-557)}

\subsubsection{绍泰}

\begin{longtable}{|>{\centering\scriptsize}m{2em}|>{\centering\scriptsize}m{1.3em}|>{\centering}m{8.8em}|}
  % \caption{秦王政}\
  \toprule
  \SimHei \normalsize 年数 & \SimHei \scriptsize 公元 & \SimHei 大事件 \tabularnewline
  % \midrule
  \endfirsthead
  \toprule
  \SimHei \normalsize 年数 & \SimHei \scriptsize 公元 & \SimHei 大事件 \tabularnewline
  \midrule
  \endhead
  \midrule
  元年 & 555 & \tabularnewline\hline
  二年 & 556 & \tabularnewline
  \bottomrule
\end{longtable}

\subsubsection{太平}

\begin{longtable}{|>{\centering\scriptsize}m{2em}|>{\centering\scriptsize}m{1.3em}|>{\centering}m{8.8em}|}
  % \caption{秦王政}\
  \toprule
  \SimHei \normalsize 年数 & \SimHei \scriptsize 公元 & \SimHei 大事件 \tabularnewline
  % \midrule
  \endfirsthead
  \toprule
  \SimHei \normalsize 年数 & \SimHei \scriptsize 公元 & \SimHei 大事件 \tabularnewline
  \midrule
  \endhead
  \midrule
  元年 & 556 & \tabularnewline\hline
  二年 & 557 & \tabularnewline
  \bottomrule
\end{longtable}


%%% Local Variables:
%%% mode: latex
%%% TeX-engine: xetex
%%% TeX-master: "../../Main"
%%% End:



%%% Local Variables:
%%% mode: latex
%%% TeX-engine: xetex
%%% TeX-master: "../../Main"
%%% End:

%% -*- coding: utf-8 -*-
%% Time-stamp: <Chen Wang: 2019-10-15 11:14:19>


\section{南陈\tiny(557-589)}

%% -*- coding: utf-8 -*-
%% Time-stamp: <Chen Wang: 2018-07-11 19:00:22>

\subsection{武帝\tiny(557-559)}

\subsubsection{永定}

\begin{longtable}{|>{\centering\scriptsize}m{2em}|>{\centering\scriptsize}m{1.3em}|>{\centering}m{8.8em}|}
  % \caption{秦王政}\
  \toprule
  \SimHei \normalsize 年数 & \SimHei \scriptsize 公元 & \SimHei 大事件 \tabularnewline
  % \midrule
  \endfirsthead
  \toprule
  \SimHei \normalsize 年数 & \SimHei \scriptsize 公元 & \SimHei 大事件 \tabularnewline
  \midrule
  \endhead
  \midrule
  元年 & 557 & \tabularnewline\hline
  二年 & 558 & \tabularnewline\hline
  三年 & 559 & \tabularnewline
  \bottomrule
\end{longtable}


%%% Local Variables:
%%% mode: latex
%%% TeX-engine: xetex
%%% TeX-master: "../../Main"
%%% End:

%% -*- coding: utf-8 -*-
%% Time-stamp: <Chen Wang: 2018-07-11 19:01:35>

\subsection{文帝\tiny(559-566)}

\subsubsection{天嘉}

\begin{longtable}{|>{\centering\scriptsize}m{2em}|>{\centering\scriptsize}m{1.3em}|>{\centering}m{8.8em}|}
  % \caption{秦王政}\
  \toprule
  \SimHei \normalsize 年数 & \SimHei \scriptsize 公元 & \SimHei 大事件 \tabularnewline
  % \midrule
  \endfirsthead
  \toprule
  \SimHei \normalsize 年数 & \SimHei \scriptsize 公元 & \SimHei 大事件 \tabularnewline
  \midrule
  \endhead
  \midrule
  元年 & 560 & \tabularnewline\hline
  二年 & 561 & \tabularnewline\hline
  三年 & 562 & \tabularnewline\hline
  四年 & 563 & \tabularnewline\hline
  五年 & 564 & \tabularnewline\hline
  六年 & 565 & \tabularnewline\hline
  七年 & 566 & \tabularnewline
  \bottomrule
\end{longtable}

\subsubsection{天康}

\begin{longtable}{|>{\centering\scriptsize}m{2em}|>{\centering\scriptsize}m{1.3em}|>{\centering}m{8.8em}|}
  % \caption{秦王政}\
  \toprule
  \SimHei \normalsize 年数 & \SimHei \scriptsize 公元 & \SimHei 大事件 \tabularnewline
  % \midrule
  \endfirsthead
  \toprule
  \SimHei \normalsize 年数 & \SimHei \scriptsize 公元 & \SimHei 大事件 \tabularnewline
  \midrule
  \endhead
  \midrule
  元年 & 566 & \tabularnewline
  \bottomrule
\end{longtable}


%%% Local Variables:
%%% mode: latex
%%% TeX-engine: xetex
%%% TeX-master: "../../Main"
%%% End:

%% -*- coding: utf-8 -*-
%% Time-stamp: <Chen Wang: 2018-07-11 19:02:25>

\subsection{陈伯宗\tiny(566-568)}

\subsubsection{光大}

\begin{longtable}{|>{\centering\scriptsize}m{2em}|>{\centering\scriptsize}m{1.3em}|>{\centering}m{8.8em}|}
  % \caption{秦王政}\
  \toprule
  \SimHei \normalsize 年数 & \SimHei \scriptsize 公元 & \SimHei 大事件 \tabularnewline
  % \midrule
  \endfirsthead
  \toprule
  \SimHei \normalsize 年数 & \SimHei \scriptsize 公元 & \SimHei 大事件 \tabularnewline
  \midrule
  \endhead
  \midrule
  元年 & 567 & \tabularnewline\hline
  二年 & 568 & \tabularnewline
  \bottomrule
\end{longtable}



%%% Local Variables:
%%% mode: latex
%%% TeX-engine: xetex
%%% TeX-master: "../../Main"
%%% End:

%% -*- coding: utf-8 -*-
%% Time-stamp: <Chen Wang: 2018-07-11 19:03:30>

\subsection{宣帝\tiny(568-582)}

\subsubsection{太建}

\begin{longtable}{|>{\centering\scriptsize}m{2em}|>{\centering\scriptsize}m{1.3em}|>{\centering}m{8.8em}|}
  % \caption{秦王政}\
  \toprule
  \SimHei \normalsize 年数 & \SimHei \scriptsize 公元 & \SimHei 大事件 \tabularnewline
  % \midrule
  \endfirsthead
  \toprule
  \SimHei \normalsize 年数 & \SimHei \scriptsize 公元 & \SimHei 大事件 \tabularnewline
  \midrule
  \endhead
  \midrule
  元年 & 569 & \tabularnewline\hline
  二年 & 570 & \tabularnewline\hline
  三年 & 571 & \tabularnewline\hline
  四年 & 572 & \tabularnewline\hline
  五年 & 573 & \tabularnewline\hline
  六年 & 574 & \tabularnewline\hline
  七年 & 575 & \tabularnewline\hline
  八年 & 576 & \tabularnewline\hline
  九年 & 577 & \tabularnewline\hline
  十年 & 578 & \tabularnewline\hline
  十一年 & 579 & \tabularnewline\hline
  十二年 & 580 & \tabularnewline\hline
  十三年 & 581 & \tabularnewline\hline
  十四年 & 582 & \tabularnewline
  \bottomrule
\end{longtable}



%%% Local Variables:
%%% mode: latex
%%% TeX-engine: xetex
%%% TeX-master: "../../Main"
%%% End:

%% -*- coding: utf-8 -*-
%% Time-stamp: <Chen Wang: 2018-07-11 19:04:29>

\subsection{陈叔宝\tiny(582-589)}

\subsubsection{至德}

\begin{longtable}{|>{\centering\scriptsize}m{2em}|>{\centering\scriptsize}m{1.3em}|>{\centering}m{8.8em}|}
  % \caption{秦王政}\
  \toprule
  \SimHei \normalsize 年数 & \SimHei \scriptsize 公元 & \SimHei 大事件 \tabularnewline
  % \midrule
  \endfirsthead
  \toprule
  \SimHei \normalsize 年数 & \SimHei \scriptsize 公元 & \SimHei 大事件 \tabularnewline
  \midrule
  \endhead
  \midrule
  元年 & 583 & \tabularnewline\hline
  二年 & 584 & \tabularnewline\hline
  三年 & 585 & \tabularnewline\hline
  四年 & 586 & \tabularnewline
  \bottomrule
\end{longtable}

\subsubsection{祯明}

\begin{longtable}{|>{\centering\scriptsize}m{2em}|>{\centering\scriptsize}m{1.3em}|>{\centering}m{8.8em}|}
  % \caption{秦王政}\
  \toprule
  \SimHei \normalsize 年数 & \SimHei \scriptsize 公元 & \SimHei 大事件 \tabularnewline
  % \midrule
  \endfirsthead
  \toprule
  \SimHei \normalsize 年数 & \SimHei \scriptsize 公元 & \SimHei 大事件 \tabularnewline
  \midrule
  \endhead
  \midrule
  元年 & 587 & \tabularnewline\hline
  二年 & 588 & \tabularnewline\hline
  三年 & 589 & \tabularnewline
  \bottomrule
\end{longtable}



%%% Local Variables:
%%% mode: latex
%%% TeX-engine: xetex
%%% TeX-master: "../../Main"
%%% End:



%%% Local Variables:
%%% mode: latex
%%% TeX-engine: xetex
%%% TeX-master: "../../Main"
%%% End:

%% -*- coding: utf-8 -*-
%% Time-stamp: <Chen Wang: 2019-10-15 11:13:52>


\section{北魏\tiny(386-534)}

%% -*- coding: utf-8 -*-
%% Time-stamp: <Chen Wang: 2018-07-11 19:15:31>

\subsection{道武帝\tiny(386-409)}

\subsubsection{登国}

\begin{longtable}{|>{\centering\scriptsize}m{2em}|>{\centering\scriptsize}m{1.3em}|>{\centering}m{8.8em}|}
  % \caption{秦王政}\
  \toprule
  \SimHei \normalsize 年数 & \SimHei \scriptsize 公元 & \SimHei 大事件 \tabularnewline
  % \midrule
  \endfirsthead
  \toprule
  \SimHei \normalsize 年数 & \SimHei \scriptsize 公元 & \SimHei 大事件 \tabularnewline
  \midrule
  \endhead
  \midrule
  元年 & 386 & \tabularnewline\hline
  二年 & 387 & \tabularnewline\hline
  三年 & 388 & \tabularnewline\hline
  四年 & 389 & \tabularnewline\hline
  五年 & 390 & \tabularnewline\hline
  六年 & 391 & \tabularnewline\hline
  七年 & 392 & \tabularnewline\hline
  八年 & 393 & \tabularnewline\hline
  九年 & 394 & \tabularnewline\hline
  十年 & 395 & \tabularnewline\hline
  十一年 & 396 & \tabularnewline
  \bottomrule
\end{longtable}

\subsubsection{皇始}

\begin{longtable}{|>{\centering\scriptsize}m{2em}|>{\centering\scriptsize}m{1.3em}|>{\centering}m{8.8em}|}
  % \caption{秦王政}\
  \toprule
  \SimHei \normalsize 年数 & \SimHei \scriptsize 公元 & \SimHei 大事件 \tabularnewline
  % \midrule
  \endfirsthead
  \toprule
  \SimHei \normalsize 年数 & \SimHei \scriptsize 公元 & \SimHei 大事件 \tabularnewline
  \midrule
  \endhead
  \midrule
  元年 & 396 & \tabularnewline\hline
  二年 & 397 & \tabularnewline\hline
  三年 & 398 & \tabularnewline
  \bottomrule
\end{longtable}

\subsubsection{天兴}

\begin{longtable}{|>{\centering\scriptsize}m{2em}|>{\centering\scriptsize}m{1.3em}|>{\centering}m{8.8em}|}
  % \caption{秦王政}\
  \toprule
  \SimHei \normalsize 年数 & \SimHei \scriptsize 公元 & \SimHei 大事件 \tabularnewline
  % \midrule
  \endfirsthead
  \toprule
  \SimHei \normalsize 年数 & \SimHei \scriptsize 公元 & \SimHei 大事件 \tabularnewline
  \midrule
  \endhead
  \midrule
  元年 & 398 & \tabularnewline\hline
  二年 & 399 & \tabularnewline\hline
  三年 & 400 & \tabularnewline\hline
  四年 & 401 & \tabularnewline\hline
  五年 & 402 & \tabularnewline\hline
  六年 & 403 & \tabularnewline\hline
  七年 & 404 & \tabularnewline
  \bottomrule
\end{longtable}

\subsubsection{天赐}

\begin{longtable}{|>{\centering\scriptsize}m{2em}|>{\centering\scriptsize}m{1.3em}|>{\centering}m{8.8em}|}
  % \caption{秦王政}\
  \toprule
  \SimHei \normalsize 年数 & \SimHei \scriptsize 公元 & \SimHei 大事件 \tabularnewline
  % \midrule
  \endfirsthead
  \toprule
  \SimHei \normalsize 年数 & \SimHei \scriptsize 公元 & \SimHei 大事件 \tabularnewline
  \midrule
  \endhead
  \midrule
  元年 & 404 & \tabularnewline\hline
  二年 & 405 & \tabularnewline\hline
  三年 & 406 & \tabularnewline\hline
  四年 & 407 & \tabularnewline\hline
  五年 & 408 & \tabularnewline\hline
  六年 & 409 & \tabularnewline
  \bottomrule
\end{longtable}


%%% Local Variables:
%%% mode: latex
%%% TeX-engine: xetex
%%% TeX-master: "../../Main"
%%% End:

%% -*- coding: utf-8 -*-
%% Time-stamp: <Chen Wang: 2018-07-11 19:17:27>

\subsection{明元帝\tiny(409-423)}

\subsubsection{永兴}

\begin{longtable}{|>{\centering\scriptsize}m{2em}|>{\centering\scriptsize}m{1.3em}|>{\centering}m{8.8em}|}
  % \caption{秦王政}\
  \toprule
  \SimHei \normalsize 年数 & \SimHei \scriptsize 公元 & \SimHei 大事件 \tabularnewline
  % \midrule
  \endfirsthead
  \toprule
  \SimHei \normalsize 年数 & \SimHei \scriptsize 公元 & \SimHei 大事件 \tabularnewline
  \midrule
  \endhead
  \midrule
  元年 & 409 & \tabularnewline\hline
  二年 & 410 & \tabularnewline\hline
  三年 & 411 & \tabularnewline\hline
  四年 & 412 & \tabularnewline\hline
  五年 & 413 & \tabularnewline
  \bottomrule
\end{longtable}

\subsubsection{神瑞}

\begin{longtable}{|>{\centering\scriptsize}m{2em}|>{\centering\scriptsize}m{1.3em}|>{\centering}m{8.8em}|}
  % \caption{秦王政}\
  \toprule
  \SimHei \normalsize 年数 & \SimHei \scriptsize 公元 & \SimHei 大事件 \tabularnewline
  % \midrule
  \endfirsthead
  \toprule
  \SimHei \normalsize 年数 & \SimHei \scriptsize 公元 & \SimHei 大事件 \tabularnewline
  \midrule
  \endhead
  \midrule
  元年 & 414 & \tabularnewline\hline
  二年 & 415 & \tabularnewline\hline
  三年 & 416 & \tabularnewline
  \bottomrule
\end{longtable}

\subsubsection{泰常}

\begin{longtable}{|>{\centering\scriptsize}m{2em}|>{\centering\scriptsize}m{1.3em}|>{\centering}m{8.8em}|}
  % \caption{秦王政}\
  \toprule
  \SimHei \normalsize 年数 & \SimHei \scriptsize 公元 & \SimHei 大事件 \tabularnewline
  % \midrule
  \endfirsthead
  \toprule
  \SimHei \normalsize 年数 & \SimHei \scriptsize 公元 & \SimHei 大事件 \tabularnewline
  \midrule
  \endhead
  \midrule
  元年 & 416 & \tabularnewline\hline
  二年 & 417 & \tabularnewline\hline
  三年 & 418 & \tabularnewline\hline
  四年 & 419 & \tabularnewline\hline
  五年 & 420 & \tabularnewline\hline
  六年 & 421 & \tabularnewline\hline
  七年 & 422 & \tabularnewline\hline
  八年 & 423 & \tabularnewline
  \bottomrule
\end{longtable}


%%% Local Variables:
%%% mode: latex
%%% TeX-engine: xetex
%%% TeX-master: "../../Main"
%%% End:

%% -*- coding: utf-8 -*-
%% Time-stamp: <Chen Wang: 2018-07-11 19:20:47>

\subsection{太武帝\tiny(423-452)}

\subsubsection{始光}

\begin{longtable}{|>{\centering\scriptsize}m{2em}|>{\centering\scriptsize}m{1.3em}|>{\centering}m{8.8em}|}
  % \caption{秦王政}\
  \toprule
  \SimHei \normalsize 年数 & \SimHei \scriptsize 公元 & \SimHei 大事件 \tabularnewline
  % \midrule
  \endfirsthead
  \toprule
  \SimHei \normalsize 年数 & \SimHei \scriptsize 公元 & \SimHei 大事件 \tabularnewline
  \midrule
  \endhead
  \midrule
  元年 & 424 & \tabularnewline\hline
  二年 & 425 & \tabularnewline\hline
  三年 & 426 & \tabularnewline\hline
  四年 & 427 & \tabularnewline\hline
  五年 & 428 & \tabularnewline
  \bottomrule
\end{longtable}

\subsubsection{神䴥}

\begin{longtable}{|>{\centering\scriptsize}m{2em}|>{\centering\scriptsize}m{1.3em}|>{\centering}m{8.8em}|}
  % \caption{秦王政}\
  \toprule
  \SimHei \normalsize 年数 & \SimHei \scriptsize 公元 & \SimHei 大事件 \tabularnewline
  % \midrule
  \endfirsthead
  \toprule
  \SimHei \normalsize 年数 & \SimHei \scriptsize 公元 & \SimHei 大事件 \tabularnewline
  \midrule
  \endhead
  \midrule
  元年 & 428 & \tabularnewline\hline
  二年 & 429 & \tabularnewline\hline
  三年 & 430 & \tabularnewline\hline
  四年 & 431 & \tabularnewline
  \bottomrule
\end{longtable}

\subsubsection{延和}

\begin{longtable}{|>{\centering\scriptsize}m{2em}|>{\centering\scriptsize}m{1.3em}|>{\centering}m{8.8em}|}
  % \caption{秦王政}\
  \toprule
  \SimHei \normalsize 年数 & \SimHei \scriptsize 公元 & \SimHei 大事件 \tabularnewline
  % \midrule
  \endfirsthead
  \toprule
  \SimHei \normalsize 年数 & \SimHei \scriptsize 公元 & \SimHei 大事件 \tabularnewline
  \midrule
  \endhead
  \midrule
  元年 & 432 & \tabularnewline\hline
  二年 & 433 & \tabularnewline\hline
  三年 & 434 & \tabularnewline\hline
  四年 & 435 & \tabularnewline
  \bottomrule
\end{longtable}

\subsubsection{太延}

\begin{longtable}{|>{\centering\scriptsize}m{2em}|>{\centering\scriptsize}m{1.3em}|>{\centering}m{8.8em}|}
  % \caption{秦王政}\
  \toprule
  \SimHei \normalsize 年数 & \SimHei \scriptsize 公元 & \SimHei 大事件 \tabularnewline
  % \midrule
  \endfirsthead
  \toprule
  \SimHei \normalsize 年数 & \SimHei \scriptsize 公元 & \SimHei 大事件 \tabularnewline
  \midrule
  \endhead
  \midrule
  元年 & 435 & \tabularnewline\hline
  二年 & 436 & \tabularnewline\hline
  三年 & 437 & \tabularnewline\hline
  四年 & 438 & \tabularnewline\hline
  五年 & 439 & \tabularnewline\hline
  六年 & 440 & \tabularnewline
  \bottomrule
\end{longtable}

\subsubsection{太平真君}

\begin{longtable}{|>{\centering\scriptsize}m{2em}|>{\centering\scriptsize}m{1.3em}|>{\centering}m{8.8em}|}
  % \caption{秦王政}\
  \toprule
  \SimHei \normalsize 年数 & \SimHei \scriptsize 公元 & \SimHei 大事件 \tabularnewline
  % \midrule
  \endfirsthead
  \toprule
  \SimHei \normalsize 年数 & \SimHei \scriptsize 公元 & \SimHei 大事件 \tabularnewline
  \midrule
  \endhead
  \midrule
  元年 & 440 & \tabularnewline\hline
  二年 & 441 & \tabularnewline\hline
  三年 & 442 & \tabularnewline\hline
  四年 & 443 & \tabularnewline\hline
  五年 & 444 & \tabularnewline\hline
  六年 & 445 & \tabularnewline\hline
  七年 & 446 & \tabularnewline\hline
  八年 & 447 & \tabularnewline\hline
  九年 & 448 & \tabularnewline\hline
  十年 & 449 & \tabularnewline\hline
  十一年 & 450 & \tabularnewline\hline
  十二年 & 451 & \tabularnewline
  \bottomrule
\end{longtable}

\subsubsection{正平}

\begin{longtable}{|>{\centering\scriptsize}m{2em}|>{\centering\scriptsize}m{1.3em}|>{\centering}m{8.8em}|}
  % \caption{秦王政}\
  \toprule
  \SimHei \normalsize 年数 & \SimHei \scriptsize 公元 & \SimHei 大事件 \tabularnewline
  % \midrule
  \endfirsthead
  \toprule
  \SimHei \normalsize 年数 & \SimHei \scriptsize 公元 & \SimHei 大事件 \tabularnewline
  \midrule
  \endhead
  \midrule
  元年 & 451 & \tabularnewline\hline
  二年 & 452 & \tabularnewline
  \bottomrule
\end{longtable}


%%% Local Variables:
%%% mode: latex
%%% TeX-engine: xetex
%%% TeX-master: "../../Main"
%%% End:

%% -*- coding: utf-8 -*-
%% Time-stamp: <Chen Wang: 2018-07-11 19:21:30>

\subsection{拓跋余\tiny(452)}

\subsubsection{承平}

\begin{longtable}{|>{\centering\scriptsize}m{2em}|>{\centering\scriptsize}m{1.3em}|>{\centering}m{8.8em}|}
  % \caption{秦王政}\
  \toprule
  \SimHei \normalsize 年数 & \SimHei \scriptsize 公元 & \SimHei 大事件 \tabularnewline
  % \midrule
  \endfirsthead
  \toprule
  \SimHei \normalsize 年数 & \SimHei \scriptsize 公元 & \SimHei 大事件 \tabularnewline
  \midrule
  \endhead
  \midrule
  元年 & 452 & \tabularnewline\hline
  \bottomrule
\end{longtable}


%%% Local Variables:
%%% mode: latex
%%% TeX-engine: xetex
%%% TeX-master: "../../Main"
%%% End:

%% -*- coding: utf-8 -*-
%% Time-stamp: <Chen Wang: 2018-07-11 19:23:19>

\subsection{文成帝\tiny(452-465)}

\subsubsection{兴安}

\begin{longtable}{|>{\centering\scriptsize}m{2em}|>{\centering\scriptsize}m{1.3em}|>{\centering}m{8.8em}|}
  % \caption{秦王政}\
  \toprule
  \SimHei \normalsize 年数 & \SimHei \scriptsize 公元 & \SimHei 大事件 \tabularnewline
  % \midrule
  \endfirsthead
  \toprule
  \SimHei \normalsize 年数 & \SimHei \scriptsize 公元 & \SimHei 大事件 \tabularnewline
  \midrule
  \endhead
  \midrule
  元年 & 452 & \tabularnewline\hline
  二年 & 453 & \tabularnewline\hline
  三年 & 454 & \tabularnewline
  \bottomrule
\end{longtable}

\subsubsection{兴光}

\begin{longtable}{|>{\centering\scriptsize}m{2em}|>{\centering\scriptsize}m{1.3em}|>{\centering}m{8.8em}|}
  % \caption{秦王政}\
  \toprule
  \SimHei \normalsize 年数 & \SimHei \scriptsize 公元 & \SimHei 大事件 \tabularnewline
  % \midrule
  \endfirsthead
  \toprule
  \SimHei \normalsize 年数 & \SimHei \scriptsize 公元 & \SimHei 大事件 \tabularnewline
  \midrule
  \endhead
  \midrule
  元年 & 454 & \tabularnewline\hline
  二年 & 455 & \tabularnewline
  \bottomrule
\end{longtable}

\subsubsection{太安}

\begin{longtable}{|>{\centering\scriptsize}m{2em}|>{\centering\scriptsize}m{1.3em}|>{\centering}m{8.8em}|}
  % \caption{秦王政}\
  \toprule
  \SimHei \normalsize 年数 & \SimHei \scriptsize 公元 & \SimHei 大事件 \tabularnewline
  % \midrule
  \endfirsthead
  \toprule
  \SimHei \normalsize 年数 & \SimHei \scriptsize 公元 & \SimHei 大事件 \tabularnewline
  \midrule
  \endhead
  \midrule
  元年 & 455 & \tabularnewline\hline
  二年 & 456 & \tabularnewline\hline
  三年 & 457 & \tabularnewline\hline
  四年 & 458 & \tabularnewline\hline
  五年 & 459 & \tabularnewline
  \bottomrule
\end{longtable}

\subsubsection{和平}

\begin{longtable}{|>{\centering\scriptsize}m{2em}|>{\centering\scriptsize}m{1.3em}|>{\centering}m{8.8em}|}
  % \caption{秦王政}\
  \toprule
  \SimHei \normalsize 年数 & \SimHei \scriptsize 公元 & \SimHei 大事件 \tabularnewline
  % \midrule
  \endfirsthead
  \toprule
  \SimHei \normalsize 年数 & \SimHei \scriptsize 公元 & \SimHei 大事件 \tabularnewline
  \midrule
  \endhead
  \midrule
  元年 & 460 & \tabularnewline\hline
  二年 & 461 & \tabularnewline\hline
  三年 & 462 & \tabularnewline\hline
  四年 & 463 & \tabularnewline\hline
  五年 & 464 & \tabularnewline\hline
  六年 & 465 & \tabularnewline
  \bottomrule
\end{longtable}


%%% Local Variables:
%%% mode: latex
%%% TeX-engine: xetex
%%% TeX-master: "../../Main"
%%% End:

%% -*- coding: utf-8 -*-
%% Time-stamp: <Chen Wang: 2018-07-11 19:24:53>

\subsection{献文帝\tiny(465-471)}

\subsubsection{天安}

\begin{longtable}{|>{\centering\scriptsize}m{2em}|>{\centering\scriptsize}m{1.3em}|>{\centering}m{8.8em}|}
  % \caption{秦王政}\
  \toprule
  \SimHei \normalsize 年数 & \SimHei \scriptsize 公元 & \SimHei 大事件 \tabularnewline
  % \midrule
  \endfirsthead
  \toprule
  \SimHei \normalsize 年数 & \SimHei \scriptsize 公元 & \SimHei 大事件 \tabularnewline
  \midrule
  \endhead
  \midrule
  元年 & 466 & \tabularnewline\hline
  二年 & 467 & \tabularnewline
  \bottomrule
\end{longtable}

\subsubsection{皇兴}

\begin{longtable}{|>{\centering\scriptsize}m{2em}|>{\centering\scriptsize}m{1.3em}|>{\centering}m{8.8em}|}
  % \caption{秦王政}\
  \toprule
  \SimHei \normalsize 年数 & \SimHei \scriptsize 公元 & \SimHei 大事件 \tabularnewline
  % \midrule
  \endfirsthead
  \toprule
  \SimHei \normalsize 年数 & \SimHei \scriptsize 公元 & \SimHei 大事件 \tabularnewline
  \midrule
  \endhead
  \midrule
  元年 & 467 & \tabularnewline\hline
  二年 & 468 & \tabularnewline\hline
  三年 & 469 & \tabularnewline\hline
  四年 & 470 & \tabularnewline\hline
  五年 & 471 & \tabularnewline
  \bottomrule
\end{longtable}


%%% Local Variables:
%%% mode: latex
%%% TeX-engine: xetex
%%% TeX-master: "../../Main"
%%% End:

%% -*- coding: utf-8 -*-
%% Time-stamp: <Chen Wang: 2018-07-11 19:26:49>

\subsection{孝文帝\tiny(471-499)}

\subsubsection{延兴}

\begin{longtable}{|>{\centering\scriptsize}m{2em}|>{\centering\scriptsize}m{1.3em}|>{\centering}m{8.8em}|}
  % \caption{秦王政}\
  \toprule
  \SimHei \normalsize 年数 & \SimHei \scriptsize 公元 & \SimHei 大事件 \tabularnewline
  % \midrule
  \endfirsthead
  \toprule
  \SimHei \normalsize 年数 & \SimHei \scriptsize 公元 & \SimHei 大事件 \tabularnewline
  \midrule
  \endhead
  \midrule
  元年 & 471 & \tabularnewline\hline
  二年 & 472 & \tabularnewline\hline
  三年 & 473 & \tabularnewline\hline
  四年 & 474 & \tabularnewline\hline
  五年 & 475 & \tabularnewline\hline
  六年 & 476 & \tabularnewline
  \bottomrule
\end{longtable}

\subsubsection{承明}

\begin{longtable}{|>{\centering\scriptsize}m{2em}|>{\centering\scriptsize}m{1.3em}|>{\centering}m{8.8em}|}
  % \caption{秦王政}\
  \toprule
  \SimHei \normalsize 年数 & \SimHei \scriptsize 公元 & \SimHei 大事件 \tabularnewline
  % \midrule
  \endfirsthead
  \toprule
  \SimHei \normalsize 年数 & \SimHei \scriptsize 公元 & \SimHei 大事件 \tabularnewline
  \midrule
  \endhead
  \midrule
  元年 & 476 & \tabularnewline
  \bottomrule
\end{longtable}

\subsubsection{太和}

\begin{longtable}{|>{\centering\scriptsize}m{2em}|>{\centering\scriptsize}m{1.3em}|>{\centering}m{8.8em}|}
  % \caption{秦王政}\
  \toprule
  \SimHei \normalsize 年数 & \SimHei \scriptsize 公元 & \SimHei 大事件 \tabularnewline
  % \midrule
  \endfirsthead
  \toprule
  \SimHei \normalsize 年数 & \SimHei \scriptsize 公元 & \SimHei 大事件 \tabularnewline
  \midrule
  \endhead
  \midrule
  元年 & 477 & \tabularnewline\hline
  二年 & 478 & \tabularnewline\hline
  三年 & 479 & \tabularnewline\hline
  四年 & 480 & \tabularnewline\hline
  五年 & 481 & \tabularnewline\hline
  六年 & 482 & \tabularnewline\hline
  七年 & 483 & \tabularnewline\hline
  八年 & 484 & \tabularnewline\hline
  九年 & 485 & \tabularnewline\hline
  十年 & 486 & \tabularnewline\hline
  十一年 & 487 & \tabularnewline\hline
  十二年 & 488 & \tabularnewline\hline
  十三年 & 489 & \tabularnewline\hline
  十四年 & 490 & \tabularnewline\hline
  十五年 & 491 & \tabularnewline\hline
  十六年 & 492 & \tabularnewline\hline
  十七年 & 493 & \tabularnewline\hline
  十八年 & 494 & \tabularnewline\hline
  十九年 & 495 & \tabularnewline\hline
  二十年 & 496 & \tabularnewline\hline
  二一年 & 497 & \tabularnewline\hline
  二二年 & 498 & \tabularnewline\hline
  二三年 & 499 & \tabularnewline
  \bottomrule
\end{longtable}


%%% Local Variables:
%%% mode: latex
%%% TeX-engine: xetex
%%% TeX-master: "../../Main"
%%% End:

%% -*- coding: utf-8 -*-
%% Time-stamp: <Chen Wang: 2018-07-11 19:28:59>

\subsection{宣武帝\tiny(499-515)}

\subsubsection{景明}

\begin{longtable}{|>{\centering\scriptsize}m{2em}|>{\centering\scriptsize}m{1.3em}|>{\centering}m{8.8em}|}
  % \caption{秦王政}\
  \toprule
  \SimHei \normalsize 年数 & \SimHei \scriptsize 公元 & \SimHei 大事件 \tabularnewline
  % \midrule
  \endfirsthead
  \toprule
  \SimHei \normalsize 年数 & \SimHei \scriptsize 公元 & \SimHei 大事件 \tabularnewline
  \midrule
  \endhead
  \midrule
  元年 & 500 & \tabularnewline\hline
  二年 & 501 & \tabularnewline\hline
  三年 & 502 & \tabularnewline\hline
  四年 & 503 & \tabularnewline\hline
  五年 & 504 & \tabularnewline
  \bottomrule
\end{longtable}

\subsubsection{正始}

\begin{longtable}{|>{\centering\scriptsize}m{2em}|>{\centering\scriptsize}m{1.3em}|>{\centering}m{8.8em}|}
  % \caption{秦王政}\
  \toprule
  \SimHei \normalsize 年数 & \SimHei \scriptsize 公元 & \SimHei 大事件 \tabularnewline
  % \midrule
  \endfirsthead
  \toprule
  \SimHei \normalsize 年数 & \SimHei \scriptsize 公元 & \SimHei 大事件 \tabularnewline
  \midrule
  \endhead
  \midrule
  元年 & 504 & \tabularnewline\hline
  二年 & 505 & \tabularnewline\hline
  三年 & 506 & \tabularnewline\hline
  四年 & 507 & \tabularnewline\hline
  五年 & 508 & \tabularnewline
  \bottomrule
\end{longtable}

\subsubsection{永平}

\begin{longtable}{|>{\centering\scriptsize}m{2em}|>{\centering\scriptsize}m{1.3em}|>{\centering}m{8.8em}|}
  % \caption{秦王政}\
  \toprule
  \SimHei \normalsize 年数 & \SimHei \scriptsize 公元 & \SimHei 大事件 \tabularnewline
  % \midrule
  \endfirsthead
  \toprule
  \SimHei \normalsize 年数 & \SimHei \scriptsize 公元 & \SimHei 大事件 \tabularnewline
  \midrule
  \endhead
  \midrule
  元年 & 508 & \tabularnewline\hline
  二年 & 509 & \tabularnewline\hline
  三年 & 510 & \tabularnewline\hline
  四年 & 511 & \tabularnewline\hline
  五年 & 512 & \tabularnewline
  \bottomrule
\end{longtable}

\subsubsection{延昌}

\begin{longtable}{|>{\centering\scriptsize}m{2em}|>{\centering\scriptsize}m{1.3em}|>{\centering}m{8.8em}|}
  % \caption{秦王政}\
  \toprule
  \SimHei \normalsize 年数 & \SimHei \scriptsize 公元 & \SimHei 大事件 \tabularnewline
  % \midrule
  \endfirsthead
  \toprule
  \SimHei \normalsize 年数 & \SimHei \scriptsize 公元 & \SimHei 大事件 \tabularnewline
  \midrule
  \endhead
  \midrule
  元年 & 512 & \tabularnewline\hline
  二年 & 513 & \tabularnewline\hline
  三年 & 514 & \tabularnewline\hline
  四年 & 515 & \tabularnewline
  \bottomrule
\end{longtable}


%%% Local Variables:
%%% mode: latex
%%% TeX-engine: xetex
%%% TeX-master: "../../Main"
%%% End:

%% -*- coding: utf-8 -*-
%% Time-stamp: <Chen Wang: 2018-07-11 19:31:09>

\subsection{孝明帝\tiny(515-528)}

\subsubsection{熙平}

\begin{longtable}{|>{\centering\scriptsize}m{2em}|>{\centering\scriptsize}m{1.3em}|>{\centering}m{8.8em}|}
  % \caption{秦王政}\
  \toprule
  \SimHei \normalsize 年数 & \SimHei \scriptsize 公元 & \SimHei 大事件 \tabularnewline
  % \midrule
  \endfirsthead
  \toprule
  \SimHei \normalsize 年数 & \SimHei \scriptsize 公元 & \SimHei 大事件 \tabularnewline
  \midrule
  \endhead
  \midrule
  元年 & 516 & \tabularnewline\hline
  二年 & 517 & \tabularnewline\hline
  三年 & 518 & \tabularnewline
  \bottomrule
\end{longtable}

\subsubsection{神龟}

\begin{longtable}{|>{\centering\scriptsize}m{2em}|>{\centering\scriptsize}m{1.3em}|>{\centering}m{8.8em}|}
  % \caption{秦王政}\
  \toprule
  \SimHei \normalsize 年数 & \SimHei \scriptsize 公元 & \SimHei 大事件 \tabularnewline
  % \midrule
  \endfirsthead
  \toprule
  \SimHei \normalsize 年数 & \SimHei \scriptsize 公元 & \SimHei 大事件 \tabularnewline
  \midrule
  \endhead
  \midrule
  元年 & 518 & \tabularnewline\hline
  二年 & 519 & \tabularnewline\hline
  三年 & 520 & \tabularnewline
  \bottomrule
\end{longtable}

\subsubsection{正光}

\begin{longtable}{|>{\centering\scriptsize}m{2em}|>{\centering\scriptsize}m{1.3em}|>{\centering}m{8.8em}|}
  % \caption{秦王政}\
  \toprule
  \SimHei \normalsize 年数 & \SimHei \scriptsize 公元 & \SimHei 大事件 \tabularnewline
  % \midrule
  \endfirsthead
  \toprule
  \SimHei \normalsize 年数 & \SimHei \scriptsize 公元 & \SimHei 大事件 \tabularnewline
  \midrule
  \endhead
  \midrule
  元年 & 520 & \tabularnewline\hline
  二年 & 521 & \tabularnewline\hline
  三年 & 522 & \tabularnewline\hline
  四年 & 523 & \tabularnewline\hline
  五年 & 524 & \tabularnewline\hline
  六年 & 525 & \tabularnewline
  \bottomrule
\end{longtable}

\subsubsection{孝昌}

\begin{longtable}{|>{\centering\scriptsize}m{2em}|>{\centering\scriptsize}m{1.3em}|>{\centering}m{8.8em}|}
  % \caption{秦王政}\
  \toprule
  \SimHei \normalsize 年数 & \SimHei \scriptsize 公元 & \SimHei 大事件 \tabularnewline
  % \midrule
  \endfirsthead
  \toprule
  \SimHei \normalsize 年数 & \SimHei \scriptsize 公元 & \SimHei 大事件 \tabularnewline
  \midrule
  \endhead
  \midrule
  元年 & 525 & \tabularnewline\hline
  二年 & 526 & \tabularnewline\hline
  三年 & 527 & \tabularnewline\hline
  四年 & 528 & \tabularnewline
  \bottomrule
\end{longtable}

\subsubsection{武泰}

\begin{longtable}{|>{\centering\scriptsize}m{2em}|>{\centering\scriptsize}m{1.3em}|>{\centering}m{8.8em}|}
  % \caption{秦王政}\
  \toprule
  \SimHei \normalsize 年数 & \SimHei \scriptsize 公元 & \SimHei 大事件 \tabularnewline
  % \midrule
  \endfirsthead
  \toprule
  \SimHei \normalsize 年数 & \SimHei \scriptsize 公元 & \SimHei 大事件 \tabularnewline
  \midrule
  \endhead
  \midrule
  元年 & 528 & \tabularnewline
  \bottomrule
\end{longtable}


%%% Local Variables:
%%% mode: latex
%%% TeX-engine: xetex
%%% TeX-master: "../../Main"
%%% End:

%% -*- coding: utf-8 -*-
%% Time-stamp: <Chen Wang: 2018-07-11 19:32:06>

\subsection{孝庄帝\tiny(528-530)}

\subsubsection{建义}

\begin{longtable}{|>{\centering\scriptsize}m{2em}|>{\centering\scriptsize}m{1.3em}|>{\centering}m{8.8em}|}
  % \caption{秦王政}\
  \toprule
  \SimHei \normalsize 年数 & \SimHei \scriptsize 公元 & \SimHei 大事件 \tabularnewline
  % \midrule
  \endfirsthead
  \toprule
  \SimHei \normalsize 年数 & \SimHei \scriptsize 公元 & \SimHei 大事件 \tabularnewline
  \midrule
  \endhead
  \midrule
  元年 & 528 & \tabularnewline
  \bottomrule
\end{longtable}

\subsubsection{永安}

\begin{longtable}{|>{\centering\scriptsize}m{2em}|>{\centering\scriptsize}m{1.3em}|>{\centering}m{8.8em}|}
  % \caption{秦王政}\
  \toprule
  \SimHei \normalsize 年数 & \SimHei \scriptsize 公元 & \SimHei 大事件 \tabularnewline
  % \midrule
  \endfirsthead
  \toprule
  \SimHei \normalsize 年数 & \SimHei \scriptsize 公元 & \SimHei 大事件 \tabularnewline
  \midrule
  \endhead
  \midrule
  元年 & 528 & \tabularnewline\hline
  二年 & 529 & \tabularnewline\hline
  三年 & 530 & \tabularnewline
  \bottomrule
\end{longtable}


%%% Local Variables:
%%% mode: latex
%%% TeX-engine: xetex
%%% TeX-master: "../../Main"
%%% End:

%% -*- coding: utf-8 -*-
%% Time-stamp: <Chen Wang: 2018-07-11 19:32:43>

\subsection{元晔\tiny(530-531)}

\subsubsection{建明}

\begin{longtable}{|>{\centering\scriptsize}m{2em}|>{\centering\scriptsize}m{1.3em}|>{\centering}m{8.8em}|}
  % \caption{秦王政}\
  \toprule
  \SimHei \normalsize 年数 & \SimHei \scriptsize 公元 & \SimHei 大事件 \tabularnewline
  % \midrule
  \endfirsthead
  \toprule
  \SimHei \normalsize 年数 & \SimHei \scriptsize 公元 & \SimHei 大事件 \tabularnewline
  \midrule
  \endhead
  \midrule
  元年 & 530 & \tabularnewline\hline
  二年 & 531 & \tabularnewline
  \bottomrule
\end{longtable}


%%% Local Variables:
%%% mode: latex
%%% TeX-engine: xetex
%%% TeX-master: "../../Main"
%%% End:

%% -*- coding: utf-8 -*-
%% Time-stamp: <Chen Wang: 2018-07-11 19:33:16>

\subsection{节闵帝\tiny(531)}

\subsubsection{普泰}

\begin{longtable}{|>{\centering\scriptsize}m{2em}|>{\centering\scriptsize}m{1.3em}|>{\centering}m{8.8em}|}
  % \caption{秦王政}\
  \toprule
  \SimHei \normalsize 年数 & \SimHei \scriptsize 公元 & \SimHei 大事件 \tabularnewline
  % \midrule
  \endfirsthead
  \toprule
  \SimHei \normalsize 年数 & \SimHei \scriptsize 公元 & \SimHei 大事件 \tabularnewline
  \midrule
  \endhead
  \midrule
  元年 & 531 & \tabularnewline
  \bottomrule
\end{longtable}


%%% Local Variables:
%%% mode: latex
%%% TeX-engine: xetex
%%% TeX-master: "../../Main"
%%% End:

%% -*- coding: utf-8 -*-
%% Time-stamp: <Chen Wang: 2018-07-11 19:33:53>

\subsection{元朗\tiny(531-532)}

\subsubsection{中兴}

\begin{longtable}{|>{\centering\scriptsize}m{2em}|>{\centering\scriptsize}m{1.3em}|>{\centering}m{8.8em}|}
  % \caption{秦王政}\
  \toprule
  \SimHei \normalsize 年数 & \SimHei \scriptsize 公元 & \SimHei 大事件 \tabularnewline
  % \midrule
  \endfirsthead
  \toprule
  \SimHei \normalsize 年数 & \SimHei \scriptsize 公元 & \SimHei 大事件 \tabularnewline
  \midrule
  \endhead
  \midrule
  元年 & 531 & \tabularnewline\hline
  二年 & 532 & \tabularnewline
  \bottomrule
\end{longtable}


%%% Local Variables:
%%% mode: latex
%%% TeX-engine: xetex
%%% TeX-master: "../../Main"
%%% End:

%% -*- coding: utf-8 -*-
%% Time-stamp: <Chen Wang: 2018-07-11 19:35:07>

\subsection{孝武帝\tiny(532-534)}

\subsubsection{太昌}

\begin{longtable}{|>{\centering\scriptsize}m{2em}|>{\centering\scriptsize}m{1.3em}|>{\centering}m{8.8em}|}
  % \caption{秦王政}\
  \toprule
  \SimHei \normalsize 年数 & \SimHei \scriptsize 公元 & \SimHei 大事件 \tabularnewline
  % \midrule
  \endfirsthead
  \toprule
  \SimHei \normalsize 年数 & \SimHei \scriptsize 公元 & \SimHei 大事件 \tabularnewline
  \midrule
  \endhead
  \midrule
  元年 & 532 & \tabularnewline
  \bottomrule
\end{longtable}

\subsubsection{永兴}

\begin{longtable}{|>{\centering\scriptsize}m{2em}|>{\centering\scriptsize}m{1.3em}|>{\centering}m{8.8em}|}
  % \caption{秦王政}\
  \toprule
  \SimHei \normalsize 年数 & \SimHei \scriptsize 公元 & \SimHei 大事件 \tabularnewline
  % \midrule
  \endfirsthead
  \toprule
  \SimHei \normalsize 年数 & \SimHei \scriptsize 公元 & \SimHei 大事件 \tabularnewline
  \midrule
  \endhead
  \midrule
  元年 & 532 & \tabularnewline
  \bottomrule
\end{longtable}

\subsubsection{永熙}

\begin{longtable}{|>{\centering\scriptsize}m{2em}|>{\centering\scriptsize}m{1.3em}|>{\centering}m{8.8em}|}
  % \caption{秦王政}\
  \toprule
  \SimHei \normalsize 年数 & \SimHei \scriptsize 公元 & \SimHei 大事件 \tabularnewline
  % \midrule
  \endfirsthead
  \toprule
  \SimHei \normalsize 年数 & \SimHei \scriptsize 公元 & \SimHei 大事件 \tabularnewline
  \midrule
  \endhead
  \midrule
  元年 & 532 & \tabularnewline\hline
  二年 & 533 & \tabularnewline\hline
  三年 & 534 & \tabularnewline
  \bottomrule
\end{longtable}


%%% Local Variables:
%%% mode: latex
%%% TeX-engine: xetex
%%% TeX-master: "../../Main"
%%% End:



%%% Local Variables:
%%% mode: latex
%%% TeX-engine: xetex
%%% TeX-master: "../../Main"
%%% End:

%% -*- coding: utf-8 -*-
%% Time-stamp: <Chen Wang: 2019-10-15 11:14:05>


\section{东魏\tiny(534-550)}

%% -*- coding: utf-8 -*-
%% Time-stamp: <Chen Wang: 2018-07-11 19:39:50>

\subsection{孝静帝\tiny(534-550)}

\subsubsection{天平}

\begin{longtable}{|>{\centering\scriptsize}m{2em}|>{\centering\scriptsize}m{1.3em}|>{\centering}m{8.8em}|}
  % \caption{秦王政}\
  \toprule
  \SimHei \normalsize 年数 & \SimHei \scriptsize 公元 & \SimHei 大事件 \tabularnewline
  % \midrule
  \endfirsthead
  \toprule
  \SimHei \normalsize 年数 & \SimHei \scriptsize 公元 & \SimHei 大事件 \tabularnewline
  \midrule
  \endhead
  \midrule
  元年 & 534 & \tabularnewline\hline
  二年 & 535 & \tabularnewline\hline
  三年 & 536 & \tabularnewline\hline
  四年 & 537 & \tabularnewline
  \bottomrule
\end{longtable}

\subsubsection{元象}

\begin{longtable}{|>{\centering\scriptsize}m{2em}|>{\centering\scriptsize}m{1.3em}|>{\centering}m{8.8em}|}
  % \caption{秦王政}\
  \toprule
  \SimHei \normalsize 年数 & \SimHei \scriptsize 公元 & \SimHei 大事件 \tabularnewline
  % \midrule
  \endfirsthead
  \toprule
  \SimHei \normalsize 年数 & \SimHei \scriptsize 公元 & \SimHei 大事件 \tabularnewline
  \midrule
  \endhead
  \midrule
  元年 & 538 & \tabularnewline\hline
  二年 & 539 & \tabularnewline
  \bottomrule
\end{longtable}

\subsubsection{兴和}

\begin{longtable}{|>{\centering\scriptsize}m{2em}|>{\centering\scriptsize}m{1.3em}|>{\centering}m{8.8em}|}
  % \caption{秦王政}\
  \toprule
  \SimHei \normalsize 年数 & \SimHei \scriptsize 公元 & \SimHei 大事件 \tabularnewline
  % \midrule
  \endfirsthead
  \toprule
  \SimHei \normalsize 年数 & \SimHei \scriptsize 公元 & \SimHei 大事件 \tabularnewline
  \midrule
  \endhead
  \midrule
  元年 & 539 & \tabularnewline\hline
  二年 & 540 & \tabularnewline\hline
  三年 & 541 & \tabularnewline\hline
  四年 & 542 & \tabularnewline
  \bottomrule
\end{longtable}

\subsubsection{武定}

\begin{longtable}{|>{\centering\scriptsize}m{2em}|>{\centering\scriptsize}m{1.3em}|>{\centering}m{8.8em}|}
  % \caption{秦王政}\
  \toprule
  \SimHei \normalsize 年数 & \SimHei \scriptsize 公元 & \SimHei 大事件 \tabularnewline
  % \midrule
  \endfirsthead
  \toprule
  \SimHei \normalsize 年数 & \SimHei \scriptsize 公元 & \SimHei 大事件 \tabularnewline
  \midrule
  \endhead
  \midrule
  元年 & 543 & \tabularnewline\hline
  二年 & 544 & \tabularnewline\hline
  三年 & 545 & \tabularnewline\hline
  四年 & 546 & \tabularnewline\hline
  五年 & 547 & \tabularnewline\hline
  六年 & 548 & \tabularnewline\hline
  七年 & 549 & \tabularnewline\hline
  八年 & 550 & \tabularnewline
  \bottomrule
\end{longtable}


%%% Local Variables:
%%% mode: latex
%%% TeX-engine: xetex
%%% TeX-master: "../../Main"
%%% End:



%%% Local Variables:
%%% mode: latex
%%% TeX-engine: xetex
%%% TeX-master: "../../Main"
%%% End:

%% -*- coding: utf-8 -*-
%% Time-stamp: <Chen Wang: 2019-10-15 11:14:44>


\section{西魏\tiny(535-557)}

%% -*- coding: utf-8 -*-
%% Time-stamp: <Chen Wang: 2018-07-11 19:54:14>

\subsection{文帝\tiny(535-551)}

\subsubsection{大统}

\begin{longtable}{|>{\centering\scriptsize}m{2em}|>{\centering\scriptsize}m{1.3em}|>{\centering}m{8.8em}|}
  % \caption{秦王政}\
  \toprule
  \SimHei \normalsize 年数 & \SimHei \scriptsize 公元 & \SimHei 大事件 \tabularnewline
  % \midrule
  \endfirsthead
  \toprule
  \SimHei \normalsize 年数 & \SimHei \scriptsize 公元 & \SimHei 大事件 \tabularnewline
  \midrule
  \endhead
  \midrule
  元年 & 535 & \tabularnewline\hline
  二年 & 536 & \tabularnewline\hline
  三年 & 537 & \tabularnewline\hline
  四年 & 538 & \tabularnewline\hline
  五年 & 539 & \tabularnewline\hline
  六年 & 540 & \tabularnewline\hline
  七年 & 541 & \tabularnewline\hline
  八年 & 512 & \tabularnewline\hline
  九年 & 513 & \tabularnewline\hline
  十年 & 544 & \tabularnewline\hline
  十一年 & 545 & \tabularnewline\hline
  十二年 & 546 & \tabularnewline\hline
  十三年 & 547 & \tabularnewline\hline
  十四年 & 548 & \tabularnewline\hline
  十五年 & 549 & \tabularnewline\hline
  十六年 & 550 & \tabularnewline\hline
  十七年 & 551 & \tabularnewline
  \bottomrule
\end{longtable}


%%% Local Variables:
%%% mode: latex
%%% TeX-engine: xetex
%%% TeX-master: "../../Main"
%%% End:

%% -*- coding: utf-8 -*-
%% Time-stamp: <Chen Wang: 2018-07-11 19:55:15>

\subsection{废帝\tiny(551-554)}

\begin{longtable}{|>{\centering\scriptsize}m{2em}|>{\centering\scriptsize}m{1.3em}|>{\centering}m{8.8em}|}
  % \caption{秦王政}\
  \toprule
  \SimHei \normalsize 年数 & \SimHei \scriptsize 公元 & \SimHei 大事件 \tabularnewline
  % \midrule
  \endfirsthead
  \toprule
  \SimHei \normalsize 年数 & \SimHei \scriptsize 公元 & \SimHei 大事件 \tabularnewline
  \midrule
  \endhead
  \midrule
  元年 & 551 & \tabularnewline\hline
  二年 & 552 & \tabularnewline\hline
  三年 & 553 & \tabularnewline\hline
  四年 & 554 & \tabularnewline
  \bottomrule
\end{longtable}


%%% Local Variables:
%%% mode: latex
%%% TeX-engine: xetex
%%% TeX-master: "../../Main"
%%% End:

%% -*- coding: utf-8 -*-
%% Time-stamp: <Chen Wang: 2018-07-11 19:56:01>

\subsection{恭帝\tiny(554-557)}

\begin{longtable}{|>{\centering\scriptsize}m{2em}|>{\centering\scriptsize}m{1.3em}|>{\centering}m{8.8em}|}
  % \caption{秦王政}\
  \toprule
  \SimHei \normalsize 年数 & \SimHei \scriptsize 公元 & \SimHei 大事件 \tabularnewline
  % \midrule
  \endfirsthead
  \toprule
  \SimHei \normalsize 年数 & \SimHei \scriptsize 公元 & \SimHei 大事件 \tabularnewline
  \midrule
  \endhead
  \midrule
  元年 & 554 & \tabularnewline\hline
  二年 & 555 & \tabularnewline\hline
  三年 & 556 & \tabularnewline\hline
  四年 & 557 & \tabularnewline
  \bottomrule
\end{longtable}


%%% Local Variables:
%%% mode: latex
%%% TeX-engine: xetex
%%% TeX-master: "../../Main"
%%% End:



%%% Local Variables:
%%% mode: latex
%%% TeX-engine: xetex
%%% TeX-master: "../../Main"
%%% End:

%% -*- coding: utf-8 -*-
%% Time-stamp: <Chen Wang: 2019-10-15 11:13:44>


\section{北齐\tiny(550-577)}

%% -*- coding: utf-8 -*-
%% Time-stamp: <Chen Wang: 2018-07-11 19:58:48>

\subsection{文宣帝\tiny(550-559)}

\subsubsection{天保}

\begin{longtable}{|>{\centering\scriptsize}m{2em}|>{\centering\scriptsize}m{1.3em}|>{\centering}m{8.8em}|}
  % \caption{秦王政}\
  \toprule
  \SimHei \normalsize 年数 & \SimHei \scriptsize 公元 & \SimHei 大事件 \tabularnewline
  % \midrule
  \endfirsthead
  \toprule
  \SimHei \normalsize 年数 & \SimHei \scriptsize 公元 & \SimHei 大事件 \tabularnewline
  \midrule
  \endhead
  \midrule
  元年 & 550 & \tabularnewline\hline
  二年 & 551 & \tabularnewline\hline
  三年 & 552 & \tabularnewline\hline
  四年 & 553 & \tabularnewline\hline
  五年 & 554 & \tabularnewline\hline
  六年 & 555 & \tabularnewline\hline
  七年 & 556 & \tabularnewline\hline
  八年 & 557 & \tabularnewline\hline
  九年 & 558 & \tabularnewline\hline
  十年 & 559 & \tabularnewline
  \bottomrule
\end{longtable}


%%% Local Variables:
%%% mode: latex
%%% TeX-engine: xetex
%%% TeX-master: "../../Main"
%%% End:

%% -*- coding: utf-8 -*-
%% Time-stamp: <Chen Wang: 2018-07-11 19:59:24>

\subsection{高殷\tiny(559-560)}

\subsubsection{乾明}

\begin{longtable}{|>{\centering\scriptsize}m{2em}|>{\centering\scriptsize}m{1.3em}|>{\centering}m{8.8em}|}
  % \caption{秦王政}\
  \toprule
  \SimHei \normalsize 年数 & \SimHei \scriptsize 公元 & \SimHei 大事件 \tabularnewline
  % \midrule
  \endfirsthead
  \toprule
  \SimHei \normalsize 年数 & \SimHei \scriptsize 公元 & \SimHei 大事件 \tabularnewline
  \midrule
  \endhead
  \midrule
  元年 & 560 & \tabularnewline
  \bottomrule
\end{longtable}


%%% Local Variables:
%%% mode: latex
%%% TeX-engine: xetex
%%% TeX-master: "../../Main"
%%% End:

%% -*- coding: utf-8 -*-
%% Time-stamp: <Chen Wang: 2018-07-11 20:00:21>

\subsection{孝昭帝\tiny(560-561)}

\subsubsection{皇建}

\begin{longtable}{|>{\centering\scriptsize}m{2em}|>{\centering\scriptsize}m{1.3em}|>{\centering}m{8.8em}|}
  % \caption{秦王政}\
  \toprule
  \SimHei \normalsize 年数 & \SimHei \scriptsize 公元 & \SimHei 大事件 \tabularnewline
  % \midrule
  \endfirsthead
  \toprule
  \SimHei \normalsize 年数 & \SimHei \scriptsize 公元 & \SimHei 大事件 \tabularnewline
  \midrule
  \endhead
  \midrule
  元年 & 560 & \tabularnewline\hline
  二年 & 561 & \tabularnewline
  \bottomrule
\end{longtable}


%%% Local Variables:
%%% mode: latex
%%% TeX-engine: xetex
%%% TeX-master: "../../Main"
%%% End:

%% -*- coding: utf-8 -*-
%% Time-stamp: <Chen Wang: 2018-07-11 20:01:13>

\subsection{武成帝\tiny(561-565)}

\subsubsection{太宁}

\begin{longtable}{|>{\centering\scriptsize}m{2em}|>{\centering\scriptsize}m{1.3em}|>{\centering}m{8.8em}|}
  % \caption{秦王政}\
  \toprule
  \SimHei \normalsize 年数 & \SimHei \scriptsize 公元 & \SimHei 大事件 \tabularnewline
  % \midrule
  \endfirsthead
  \toprule
  \SimHei \normalsize 年数 & \SimHei \scriptsize 公元 & \SimHei 大事件 \tabularnewline
  \midrule
  \endhead
  \midrule
  元年 & 561 & \tabularnewline\hline
  二年 & 562 & \tabularnewline
  \bottomrule
\end{longtable}

\subsubsection{河清}

\begin{longtable}{|>{\centering\scriptsize}m{2em}|>{\centering\scriptsize}m{1.3em}|>{\centering}m{8.8em}|}
  % \caption{秦王政}\
  \toprule
  \SimHei \normalsize 年数 & \SimHei \scriptsize 公元 & \SimHei 大事件 \tabularnewline
  % \midrule
  \endfirsthead
  \toprule
  \SimHei \normalsize 年数 & \SimHei \scriptsize 公元 & \SimHei 大事件 \tabularnewline
  \midrule
  \endhead
  \midrule
  元年 & 562 & \tabularnewline\hline
  二年 & 563 & \tabularnewline\hline
  三年 & 564 & \tabularnewline\hline
  四年 & 565 & \tabularnewline
  \bottomrule
\end{longtable}


%%% Local Variables:
%%% mode: latex
%%% TeX-engine: xetex
%%% TeX-master: "../../Main"
%%% End:

%% -*- coding: utf-8 -*-
%% Time-stamp: <Chen Wang: 2018-07-11 20:03:52>

\subsection{高纬\tiny(565-576)}

\subsubsection{天统}

\begin{longtable}{|>{\centering\scriptsize}m{2em}|>{\centering\scriptsize}m{1.3em}|>{\centering}m{8.8em}|}
  % \caption{秦王政}\
  \toprule
  \SimHei \normalsize 年数 & \SimHei \scriptsize 公元 & \SimHei 大事件 \tabularnewline
  % \midrule
  \endfirsthead
  \toprule
  \SimHei \normalsize 年数 & \SimHei \scriptsize 公元 & \SimHei 大事件 \tabularnewline
  \midrule
  \endhead
  \midrule
  元年 & 565 & \tabularnewline\hline
  二年 & 566 & \tabularnewline\hline
  三年 & 567 & \tabularnewline\hline
  四年 & 568 & \tabularnewline\hline
  五年 & 569 & \tabularnewline
  \bottomrule
\end{longtable}

\subsubsection{武平}

\begin{longtable}{|>{\centering\scriptsize}m{2em}|>{\centering\scriptsize}m{1.3em}|>{\centering}m{8.8em}|}
  % \caption{秦王政}\
  \toprule
  \SimHei \normalsize 年数 & \SimHei \scriptsize 公元 & \SimHei 大事件 \tabularnewline
  % \midrule
  \endfirsthead
  \toprule
  \SimHei \normalsize 年数 & \SimHei \scriptsize 公元 & \SimHei 大事件 \tabularnewline
  \midrule
  \endhead
  \midrule
  元年 & 570 & \tabularnewline\hline
  二年 & 571 & \tabularnewline\hline
  三年 & 572 & \tabularnewline\hline
  四年 & 573 & \tabularnewline\hline
  五年 & 574 & \tabularnewline\hline
  六年 & 575 & \tabularnewline\hline
  七年 & 576 & \tabularnewline
  \bottomrule
\end{longtable}

\subsubsection{隆化}

\begin{longtable}{|>{\centering\scriptsize}m{2em}|>{\centering\scriptsize}m{1.3em}|>{\centering}m{8.8em}|}
  % \caption{秦王政}\
  \toprule
  \SimHei \normalsize 年数 & \SimHei \scriptsize 公元 & \SimHei 大事件 \tabularnewline
  % \midrule
  \endfirsthead
  \toprule
  \SimHei \normalsize 年数 & \SimHei \scriptsize 公元 & \SimHei 大事件 \tabularnewline
  \midrule
  \endhead
  \midrule
  元年 & 576 & \tabularnewline
  \bottomrule
\end{longtable}


%%% Local Variables:
%%% mode: latex
%%% TeX-engine: xetex
%%% TeX-master: "../../Main"
%%% End:

%% -*- coding: utf-8 -*-
%% Time-stamp: <Chen Wang: 2018-07-11 20:04:24>

\subsection{高延宗\tiny(576)}

\subsubsection{德昌}

\begin{longtable}{|>{\centering\scriptsize}m{2em}|>{\centering\scriptsize}m{1.3em}|>{\centering}m{8.8em}|}
  % \caption{秦王政}\
  \toprule
  \SimHei \normalsize 年数 & \SimHei \scriptsize 公元 & \SimHei 大事件 \tabularnewline
  % \midrule
  \endfirsthead
  \toprule
  \SimHei \normalsize 年数 & \SimHei \scriptsize 公元 & \SimHei 大事件 \tabularnewline
  \midrule
  \endhead
  \midrule
  元年 & 576 & \tabularnewline
  \bottomrule
\end{longtable}


%%% Local Variables:
%%% mode: latex
%%% TeX-engine: xetex
%%% TeX-master: "../../Main"
%%% End:

%% -*- coding: utf-8 -*-
%% Time-stamp: <Chen Wang: 2018-07-11 20:04:51>

\subsection{高桓\tiny(577)}

\subsubsection{承光}

\begin{longtable}{|>{\centering\scriptsize}m{2em}|>{\centering\scriptsize}m{1.3em}|>{\centering}m{8.8em}|}
  % \caption{秦王政}\
  \toprule
  \SimHei \normalsize 年数 & \SimHei \scriptsize 公元 & \SimHei 大事件 \tabularnewline
  % \midrule
  \endfirsthead
  \toprule
  \SimHei \normalsize 年数 & \SimHei \scriptsize 公元 & \SimHei 大事件 \tabularnewline
  \midrule
  \endhead
  \midrule
  元年 & 577 & \tabularnewline
  \bottomrule
\end{longtable}


%%% Local Variables:
%%% mode: latex
%%% TeX-engine: xetex
%%% TeX-master: "../../Main"
%%% End:



%%% Local Variables:
%%% mode: latex
%%% TeX-engine: xetex
%%% TeX-master: "../../Main"
%%% End:

%% -*- coding: utf-8 -*-
%% Time-stamp: <Chen Wang: 2019-10-15 11:13:59>


\section{北周\tiny(557-581)}

%% -*- coding: utf-8 -*-
%% Time-stamp: <Chen Wang: 2018-07-11 20:06:52>

\subsection{明帝\tiny(557-560)}

\subsubsection{武成}

\begin{longtable}{|>{\centering\scriptsize}m{2em}|>{\centering\scriptsize}m{1.3em}|>{\centering}m{8.8em}|}
  % \caption{秦王政}\
  \toprule
  \SimHei \normalsize 年数 & \SimHei \scriptsize 公元 & \SimHei 大事件 \tabularnewline
  % \midrule
  \endfirsthead
  \toprule
  \SimHei \normalsize 年数 & \SimHei \scriptsize 公元 & \SimHei 大事件 \tabularnewline
  \midrule
  \endhead
  \midrule
  元年 & 559 & \tabularnewline\hline
  二年 & 560 & \tabularnewline
  \bottomrule
\end{longtable}


%%% Local Variables:
%%% mode: latex
%%% TeX-engine: xetex
%%% TeX-master: "../../Main"
%%% End:

%% -*- coding: utf-8 -*-
%% Time-stamp: <Chen Wang: 2018-07-11 20:08:36>

\subsection{武帝\tiny(560-578)}

\subsubsection{保定}

\begin{longtable}{|>{\centering\scriptsize}m{2em}|>{\centering\scriptsize}m{1.3em}|>{\centering}m{8.8em}|}
  % \caption{秦王政}\
  \toprule
  \SimHei \normalsize 年数 & \SimHei \scriptsize 公元 & \SimHei 大事件 \tabularnewline
  % \midrule
  \endfirsthead
  \toprule
  \SimHei \normalsize 年数 & \SimHei \scriptsize 公元 & \SimHei 大事件 \tabularnewline
  \midrule
  \endhead
  \midrule
  元年 & 561 & \tabularnewline\hline
  二年 & 562 & \tabularnewline\hline
  三年 & 563 & \tabularnewline\hline
  四年 & 564 & \tabularnewline\hline
  五年 & 565 & \tabularnewline
  \bottomrule
\end{longtable}

\subsubsection{天和}

\begin{longtable}{|>{\centering\scriptsize}m{2em}|>{\centering\scriptsize}m{1.3em}|>{\centering}m{8.8em}|}
  % \caption{秦王政}\
  \toprule
  \SimHei \normalsize 年数 & \SimHei \scriptsize 公元 & \SimHei 大事件 \tabularnewline
  % \midrule
  \endfirsthead
  \toprule
  \SimHei \normalsize 年数 & \SimHei \scriptsize 公元 & \SimHei 大事件 \tabularnewline
  \midrule
  \endhead
  \midrule
  元年 & 566 & \tabularnewline\hline
  二年 & 567 & \tabularnewline\hline
  三年 & 568 & \tabularnewline\hline
  四年 & 569 & \tabularnewline\hline
  五年 & 570 & \tabularnewline\hline
  六年 & 571 & \tabularnewline\hline
  七年 & 572 & \tabularnewline
  \bottomrule
\end{longtable}

\subsubsection{建德}

\begin{longtable}{|>{\centering\scriptsize}m{2em}|>{\centering\scriptsize}m{1.3em}|>{\centering}m{8.8em}|}
  % \caption{秦王政}\
  \toprule
  \SimHei \normalsize 年数 & \SimHei \scriptsize 公元 & \SimHei 大事件 \tabularnewline
  % \midrule
  \endfirsthead
  \toprule
  \SimHei \normalsize 年数 & \SimHei \scriptsize 公元 & \SimHei 大事件 \tabularnewline
  \midrule
  \endhead
  \midrule
  元年 & 572 & \tabularnewline\hline
  二年 & 573 & \tabularnewline\hline
  三年 & 574 & \tabularnewline\hline
  四年 & 575 & \tabularnewline\hline
  五年 & 576 & \tabularnewline\hline
  六年 & 578 & \tabularnewline
  \bottomrule
\end{longtable}

\subsubsection{宣政}

\begin{longtable}{|>{\centering\scriptsize}m{2em}|>{\centering\scriptsize}m{1.3em}|>{\centering}m{8.8em}|}
  % \caption{秦王政}\
  \toprule
  \SimHei \normalsize 年数 & \SimHei \scriptsize 公元 & \SimHei 大事件 \tabularnewline
  % \midrule
  \endfirsthead
  \toprule
  \SimHei \normalsize 年数 & \SimHei \scriptsize 公元 & \SimHei 大事件 \tabularnewline
  \midrule
  \endhead
  \midrule
  元年 & 578 & \tabularnewline
  \bottomrule
\end{longtable}


%%% Local Variables:
%%% mode: latex
%%% TeX-engine: xetex
%%% TeX-master: "../../Main"
%%% End:

%% -*- coding: utf-8 -*-
%% Time-stamp: <Chen Wang: 2018-07-11 20:09:08>

\subsection{宣帝\tiny(578-579)}

\subsubsection{大成}

\begin{longtable}{|>{\centering\scriptsize}m{2em}|>{\centering\scriptsize}m{1.3em}|>{\centering}m{8.8em}|}
  % \caption{秦王政}\
  \toprule
  \SimHei \normalsize 年数 & \SimHei \scriptsize 公元 & \SimHei 大事件 \tabularnewline
  % \midrule
  \endfirsthead
  \toprule
  \SimHei \normalsize 年数 & \SimHei \scriptsize 公元 & \SimHei 大事件 \tabularnewline
  \midrule
  \endhead
  \midrule
  元年 & 579 & \tabularnewline
  \bottomrule
\end{longtable}


%%% Local Variables:
%%% mode: latex
%%% TeX-engine: xetex
%%% TeX-master: "../../Main"
%%% End:

%% -*- coding: utf-8 -*-
%% Time-stamp: <Chen Wang: 2018-07-11 20:10:14>

\subsection{静帝\tiny(579-581)}

\subsubsection{大象}

\begin{longtable}{|>{\centering\scriptsize}m{2em}|>{\centering\scriptsize}m{1.3em}|>{\centering}m{8.8em}|}
  % \caption{秦王政}\
  \toprule
  \SimHei \normalsize 年数 & \SimHei \scriptsize 公元 & \SimHei 大事件 \tabularnewline
  % \midrule
  \endfirsthead
  \toprule
  \SimHei \normalsize 年数 & \SimHei \scriptsize 公元 & \SimHei 大事件 \tabularnewline
  \midrule
  \endhead
  \midrule
  元年 & 579 & \tabularnewline\hline
  二年 & 580 & \tabularnewline
  \bottomrule
\end{longtable}

\subsubsection{大定}

\begin{longtable}{|>{\centering\scriptsize}m{2em}|>{\centering\scriptsize}m{1.3em}|>{\centering}m{8.8em}|}
  % \caption{秦王政}\
  \toprule
  \SimHei \normalsize 年数 & \SimHei \scriptsize 公元 & \SimHei 大事件 \tabularnewline
  % \midrule
  \endfirsthead
  \toprule
  \SimHei \normalsize 年数 & \SimHei \scriptsize 公元 & \SimHei 大事件 \tabularnewline
  \midrule
  \endhead
  \midrule
  元年 & 581 & \tabularnewline
  \bottomrule
\end{longtable}


%%% Local Variables:
%%% mode: latex
%%% TeX-engine: xetex
%%% TeX-master: "../../Main"
%%% End:



%%% Local Variables:
%%% mode: latex
%%% TeX-engine: xetex
%%% TeX-master: "../../Main"
%%% End:


%%% Local Variables:
%%% mode: latex
%%% TeX-engine: xetex
%%% TeX-master: "../Main"
%%% End:
 % 南北朝
% %% -*- coding: utf-8 -*-
%% Time-stamp: <Chen Wang: 2019-10-15 11:15:05>

\chapter{隋\tiny(581-619)}

%% -*- coding: utf-8 -*-
%% Time-stamp: <Chen Wang: 2018-07-11 20:20:31>

\section{文帝\tiny(581-604)}

\subsection{开皇}

\begin{longtable}{|>{\centering\scriptsize}m{2em}|>{\centering\scriptsize}m{1.3em}|>{\centering}m{8.8em}|}
  % \caption{秦王政}\
  \toprule
  \SimHei \normalsize 年数 & \SimHei \scriptsize 公元 & \SimHei 大事件 \tabularnewline
  % \midrule
  \endfirsthead
  \toprule
  \SimHei \normalsize 年数 & \SimHei \scriptsize 公元 & \SimHei 大事件 \tabularnewline
  \midrule
  \endhead
  \midrule
  元年 & 581 & \tabularnewline\hline
  二年 & 582 & \tabularnewline\hline
  三年 & 583 & \tabularnewline\hline
  四年 & 584 & \tabularnewline\hline
  五年 & 585 & \tabularnewline\hline
  六年 & 586 & \tabularnewline\hline
  七年 & 587 & \tabularnewline\hline
  八年 & 588 & \tabularnewline\hline
  九年 & 589 & \tabularnewline\hline
  十年 & 560 & \tabularnewline\hline
  十一年 & 561 & \tabularnewline\hline
  十二年 & 562 & \tabularnewline\hline
  十三年 & 563 & \tabularnewline\hline
  十四年 & 564 & \tabularnewline\hline
  十五年 & 565 & \tabularnewline\hline
  十六年 & 566 & \tabularnewline\hline
  十七年 & 567 & \tabularnewline\hline
  十八年 & 568 & \tabularnewline\hline
  十九年 & 569 & \tabularnewline\hline
  二十年 & 600 & \tabularnewline
  \bottomrule
\end{longtable}

\subsection{仁寿}

\begin{longtable}{|>{\centering\scriptsize}m{2em}|>{\centering\scriptsize}m{1.3em}|>{\centering}m{8.8em}|}
  % \caption{秦王政}\
  \toprule
  \SimHei \normalsize 年数 & \SimHei \scriptsize 公元 & \SimHei 大事件 \tabularnewline
  % \midrule
  \endfirsthead
  \toprule
  \SimHei \normalsize 年数 & \SimHei \scriptsize 公元 & \SimHei 大事件 \tabularnewline
  \midrule
  \endhead
  \midrule
  元年 & 601 & \tabularnewline\hline
  二年 & 602 & \tabularnewline\hline
  三年 & 603 & \tabularnewline\hline
  四年 & 604 & \tabularnewline
  \bottomrule
\end{longtable}


%%% Local Variables:
%%% mode: latex
%%% TeX-engine: xetex
%%% TeX-master: "../Main"
%%% End:

%% -*- coding: utf-8 -*-
%% Time-stamp: <Chen Wang: 2018-07-11 20:21:30>

\section{炀帝\tiny(604-618)}

\subsection{大业}

\begin{longtable}{|>{\centering\scriptsize}m{2em}|>{\centering\scriptsize}m{1.3em}|>{\centering}m{8.8em}|}
  % \caption{秦王政}\
  \toprule
  \SimHei \normalsize 年数 & \SimHei \scriptsize 公元 & \SimHei 大事件 \tabularnewline
  % \midrule
  \endfirsthead
  \toprule
  \SimHei \normalsize 年数 & \SimHei \scriptsize 公元 & \SimHei 大事件 \tabularnewline
  \midrule
  \endhead
  \midrule
  元年 & 605 & \tabularnewline\hline
  二年 & 606 & \tabularnewline\hline
  三年 & 607 & \tabularnewline\hline
  四年 & 608 & \tabularnewline\hline
  五年 & 609 & \tabularnewline\hline
  六年 & 610 & \tabularnewline\hline
  七年 & 611 & \tabularnewline\hline
  八年 & 612 & \tabularnewline\hline
  九年 & 613 & \tabularnewline\hline
  十年 & 614 & \tabularnewline\hline
  十一年 & 615 & \tabularnewline\hline
  十二年 & 616 & \tabularnewline\hline
  十三年 & 617 & \tabularnewline\hline
  十四年 & 618 & \tabularnewline
  \bottomrule
\end{longtable}


%%% Local Variables:
%%% mode: latex
%%% TeX-engine: xetex
%%% TeX-master: "../Main"
%%% End:

%% -*- coding: utf-8 -*-
%% Time-stamp: <Chen Wang: 2018-07-11 20:22:17>

\section{恭帝\tiny(617-618)}

\subsection{义宁}

\begin{longtable}{|>{\centering\scriptsize}m{2em}|>{\centering\scriptsize}m{1.3em}|>{\centering}m{8.8em}|}
  % \caption{秦王政}\
  \toprule
  \SimHei \normalsize 年数 & \SimHei \scriptsize 公元 & \SimHei 大事件 \tabularnewline
  % \midrule
  \endfirsthead
  \toprule
  \SimHei \normalsize 年数 & \SimHei \scriptsize 公元 & \SimHei 大事件 \tabularnewline
  \midrule
  \endhead
  \midrule
  元年 & 617 & \tabularnewline\hline
  二年 & 618 & \tabularnewline
  \bottomrule
\end{longtable}


%%% Local Variables:
%%% mode: latex
%%% TeX-engine: xetex
%%% TeX-master: "../Main"
%%% End:

%% -*- coding: utf-8 -*-
%% Time-stamp: <Chen Wang: 2018-07-11 20:22:58>

\section{杨侗\tiny(618-619)}

\subsection{皇泰}

\begin{longtable}{|>{\centering\scriptsize}m{2em}|>{\centering\scriptsize}m{1.3em}|>{\centering}m{8.8em}|}
  % \caption{秦王政}\
  \toprule
  \SimHei \normalsize 年数 & \SimHei \scriptsize 公元 & \SimHei 大事件 \tabularnewline
  % \midrule
  \endfirsthead
  \toprule
  \SimHei \normalsize 年数 & \SimHei \scriptsize 公元 & \SimHei 大事件 \tabularnewline
  \midrule
  \endhead
  \midrule
  元年 & 618 & \tabularnewline\hline
  二年 & 619 & \tabularnewline
  \bottomrule
\end{longtable}


%%% Local Variables:
%%% mode: latex
%%% TeX-engine: xetex
%%% TeX-master: "../Main"
%%% End:



%%% Local Variables:
%%% mode: latex
%%% TeX-engine: xetex
%%% TeX-master: "../Main"
%%% End:
 % 隋
% %% -*- coding: utf-8 -*-
%% Time-stamp: <Chen Wang: 2019-10-15 11:17:01>

\chapter{唐\tiny(618-907)}

%% -*- coding: utf-8 -*-
%% Time-stamp: <Chen Wang: 2018-07-11 20:36:04>

\section{高祖\tiny(618-626)}

\subsection{武德}

\begin{longtable}{|>{\centering\scriptsize}m{2em}|>{\centering\scriptsize}m{1.3em}|>{\centering}m{8.8em}|}
  % \caption{秦王政}\
  \toprule
  \SimHei \normalsize 年数 & \SimHei \scriptsize 公元 & \SimHei 大事件 \tabularnewline
  % \midrule
  \endfirsthead
  \toprule
  \SimHei \normalsize 年数 & \SimHei \scriptsize 公元 & \SimHei 大事件 \tabularnewline
  \midrule
  \endhead
  \midrule
  元年 & 618 & \tabularnewline\hline
  二年 & 619 & \tabularnewline\hline
  三年 & 620 & \tabularnewline\hline
  四年 & 621 & \tabularnewline\hline
  五年 & 622 & \tabularnewline\hline
  六年 & 623 & \tabularnewline\hline
  七年 & 624 & \tabularnewline\hline
  八年 & 625 & \tabularnewline\hline
  九年 & 626 & \tabularnewline
  \bottomrule
\end{longtable}


%%% Local Variables:
%%% mode: latex
%%% TeX-engine: xetex
%%% TeX-master: "../Main"
%%% End:

%% -*- coding: utf-8 -*-
%% Time-stamp: <Chen Wang: 2018-07-11 21:24:45>

\section{太宗\tiny(626-649)}

\subsection{贞观}

\begin{longtable}{|>{\centering\scriptsize}m{2em}|>{\centering\scriptsize}m{1.3em}|>{\centering}m{8.8em}|}
  % \caption{秦王政}\
  \toprule
  \SimHei \normalsize 年数 & \SimHei \scriptsize 公元 & \SimHei 大事件 \tabularnewline
  % \midrule
  \endfirsthead
  \toprule
  \SimHei \normalsize 年数 & \SimHei \scriptsize 公元 & \SimHei 大事件 \tabularnewline
  \midrule
  \endhead
  \midrule
  元年 & 627 & \tabularnewline\hline
  二年 & 628 & \tabularnewline\hline
  三年 & 629 & \tabularnewline\hline
  四年 & 630 & \tabularnewline\hline
  五年 & 631 & \tabularnewline\hline
  六年 & 632 & \tabularnewline\hline
  七年 & 633 & \tabularnewline\hline
  八年 & 634 & \tabularnewline\hline
  九年 & 635 & \tabularnewline\hline
  十年 & 636 & \tabularnewline\hline
  十一年 & 637 & \tabularnewline\hline
  十二年 & 638 & \tabularnewline\hline
  十三年 & 639 & \tabularnewline\hline
  十四年 & 640 & \tabularnewline\hline
  十五年 & 641 & \tabularnewline\hline
  十六年 & 642 & \tabularnewline\hline
  十七年 & 643 & \tabularnewline\hline
  十八年 & 644 & \tabularnewline\hline
  十九年 & 645 & \tabularnewline\hline
  二十年 & 646 & \tabularnewline\hline
  二一年 & 647 & \tabularnewline\hline
  二二年 & 648 & \tabularnewline\hline
  二三年 & 649 & \tabularnewline
  \bottomrule
\end{longtable}


%%% Local Variables:
%%% mode: latex
%%% TeX-engine: xetex
%%% TeX-master: "../Main"
%%% End:

%% -*- coding: utf-8 -*-
%% Time-stamp: <Chen Wang: 2018-07-11 21:30:52>

\section{高宗\tiny(649-683)}

\subsection{永徽}

\begin{longtable}{|>{\centering\scriptsize}m{2em}|>{\centering\scriptsize}m{1.3em}|>{\centering}m{8.8em}|}
  % \caption{秦王政}\
  \toprule
  \SimHei \normalsize 年数 & \SimHei \scriptsize 公元 & \SimHei 大事件 \tabularnewline
  % \midrule
  \endfirsthead
  \toprule
  \SimHei \normalsize 年数 & \SimHei \scriptsize 公元 & \SimHei 大事件 \tabularnewline
  \midrule
  \endhead
  \midrule
  元年 & 650 & \tabularnewline\hline
  二年 & 651 & \tabularnewline\hline
  三年 & 652 & \tabularnewline\hline
  四年 & 653 & \tabularnewline\hline
  五年 & 654 & \tabularnewline\hline
  六年 & 655 & \tabularnewline
  \bottomrule
\end{longtable}

\subsection{显庆}

\begin{longtable}{|>{\centering\scriptsize}m{2em}|>{\centering\scriptsize}m{1.3em}|>{\centering}m{8.8em}|}
  % \caption{秦王政}\
  \toprule
  \SimHei \normalsize 年数 & \SimHei \scriptsize 公元 & \SimHei 大事件 \tabularnewline
  % \midrule
  \endfirsthead
  \toprule
  \SimHei \normalsize 年数 & \SimHei \scriptsize 公元 & \SimHei 大事件 \tabularnewline
  \midrule
  \endhead
  \midrule
  元年 & 656 & \tabularnewline\hline
  二年 & 657 & \tabularnewline\hline
  三年 & 658 & \tabularnewline\hline
  四年 & 659 & \tabularnewline\hline
  五年 & 660 & \tabularnewline\hline
  六年 & 661 & \tabularnewline
  \bottomrule
\end{longtable}

\subsection{龙朔}

\begin{longtable}{|>{\centering\scriptsize}m{2em}|>{\centering\scriptsize}m{1.3em}|>{\centering}m{8.8em}|}
  % \caption{秦王政}\
  \toprule
  \SimHei \normalsize 年数 & \SimHei \scriptsize 公元 & \SimHei 大事件 \tabularnewline
  % \midrule
  \endfirsthead
  \toprule
  \SimHei \normalsize 年数 & \SimHei \scriptsize 公元 & \SimHei 大事件 \tabularnewline
  \midrule
  \endhead
  \midrule
  元年 & 661 & \tabularnewline\hline
  二年 & 662 & \tabularnewline\hline
  三年 & 663 & \tabularnewline
  \bottomrule
\end{longtable}

\subsection{麟德}

\begin{longtable}{|>{\centering\scriptsize}m{2em}|>{\centering\scriptsize}m{1.3em}|>{\centering}m{8.8em}|}
  % \caption{秦王政}\
  \toprule
  \SimHei \normalsize 年数 & \SimHei \scriptsize 公元 & \SimHei 大事件 \tabularnewline
  % \midrule
  \endfirsthead
  \toprule
  \SimHei \normalsize 年数 & \SimHei \scriptsize 公元 & \SimHei 大事件 \tabularnewline
  \midrule
  \endhead
  \midrule
  元年 & 664 & \tabularnewline\hline
  二年 & 665 & \tabularnewline
  \bottomrule
\end{longtable}

\subsection{乾封}

\begin{longtable}{|>{\centering\scriptsize}m{2em}|>{\centering\scriptsize}m{1.3em}|>{\centering}m{8.8em}|}
  % \caption{秦王政}\
  \toprule
  \SimHei \normalsize 年数 & \SimHei \scriptsize 公元 & \SimHei 大事件 \tabularnewline
  % \midrule
  \endfirsthead
  \toprule
  \SimHei \normalsize 年数 & \SimHei \scriptsize 公元 & \SimHei 大事件 \tabularnewline
  \midrule
  \endhead
  \midrule
  元年 & 666 & \tabularnewline\hline
  二年 & 667 & \tabularnewline\hline
  三年 & 668 & \tabularnewline
  \bottomrule
\end{longtable}

\subsection{总章}

\begin{longtable}{|>{\centering\scriptsize}m{2em}|>{\centering\scriptsize}m{1.3em}|>{\centering}m{8.8em}|}
  % \caption{秦王政}\
  \toprule
  \SimHei \normalsize 年数 & \SimHei \scriptsize 公元 & \SimHei 大事件 \tabularnewline
  % \midrule
  \endfirsthead
  \toprule
  \SimHei \normalsize 年数 & \SimHei \scriptsize 公元 & \SimHei 大事件 \tabularnewline
  \midrule
  \endhead
  \midrule
  元年 & 668 & \tabularnewline\hline
  二年 & 669 & \tabularnewline\hline
  三年 & 670 & \tabularnewline
  \bottomrule
\end{longtable}

\subsection{咸亨}

\begin{longtable}{|>{\centering\scriptsize}m{2em}|>{\centering\scriptsize}m{1.3em}|>{\centering}m{8.8em}|}
  % \caption{秦王政}\
  \toprule
  \SimHei \normalsize 年数 & \SimHei \scriptsize 公元 & \SimHei 大事件 \tabularnewline
  % \midrule
  \endfirsthead
  \toprule
  \SimHei \normalsize 年数 & \SimHei \scriptsize 公元 & \SimHei 大事件 \tabularnewline
  \midrule
  \endhead
  \midrule
  元年 & 670 & \tabularnewline\hline
  二年 & 671 & \tabularnewline\hline
  三年 & 672 & \tabularnewline\hline
  四年 & 673 & \tabularnewline\hline
  五年 & 674 & \tabularnewline
  \bottomrule
\end{longtable}

\subsection{上元}

\begin{longtable}{|>{\centering\scriptsize}m{2em}|>{\centering\scriptsize}m{1.3em}|>{\centering}m{8.8em}|}
  % \caption{秦王政}\
  \toprule
  \SimHei \normalsize 年数 & \SimHei \scriptsize 公元 & \SimHei 大事件 \tabularnewline
  % \midrule
  \endfirsthead
  \toprule
  \SimHei \normalsize 年数 & \SimHei \scriptsize 公元 & \SimHei 大事件 \tabularnewline
  \midrule
  \endhead
  \midrule
  元年 & 674 & \tabularnewline\hline
  二年 & 675 & \tabularnewline\hline
  三年 & 676 & \tabularnewline
  \bottomrule
\end{longtable}

\subsection{仪凤}

\begin{longtable}{|>{\centering\scriptsize}m{2em}|>{\centering\scriptsize}m{1.3em}|>{\centering}m{8.8em}|}
  % \caption{秦王政}\
  \toprule
  \SimHei \normalsize 年数 & \SimHei \scriptsize 公元 & \SimHei 大事件 \tabularnewline
  % \midrule
  \endfirsthead
  \toprule
  \SimHei \normalsize 年数 & \SimHei \scriptsize 公元 & \SimHei 大事件 \tabularnewline
  \midrule
  \endhead
  \midrule
  元年 & 676 & \tabularnewline\hline
  二年 & 677 & \tabularnewline\hline
  三年 & 678 & \tabularnewline\hline
  四年 & 679 & \tabularnewline
  \bottomrule
\end{longtable}

\subsection{调露}

\begin{longtable}{|>{\centering\scriptsize}m{2em}|>{\centering\scriptsize}m{1.3em}|>{\centering}m{8.8em}|}
  % \caption{秦王政}\
  \toprule
  \SimHei \normalsize 年数 & \SimHei \scriptsize 公元 & \SimHei 大事件 \tabularnewline
  % \midrule
  \endfirsthead
  \toprule
  \SimHei \normalsize 年数 & \SimHei \scriptsize 公元 & \SimHei 大事件 \tabularnewline
  \midrule
  \endhead
  \midrule
  元年 & 679 & \tabularnewline\hline
  二年 & 680 & \tabularnewline
  \bottomrule
\end{longtable}

\subsection{永隆}

\begin{longtable}{|>{\centering\scriptsize}m{2em}|>{\centering\scriptsize}m{1.3em}|>{\centering}m{8.8em}|}
  % \caption{秦王政}\
  \toprule
  \SimHei \normalsize 年数 & \SimHei \scriptsize 公元 & \SimHei 大事件 \tabularnewline
  % \midrule
  \endfirsthead
  \toprule
  \SimHei \normalsize 年数 & \SimHei \scriptsize 公元 & \SimHei 大事件 \tabularnewline
  \midrule
  \endhead
  \midrule
  元年 & 680 & \tabularnewline\hline
  二年 & 681 & \tabularnewline
  \bottomrule
\end{longtable}

\subsection{开耀}

\begin{longtable}{|>{\centering\scriptsize}m{2em}|>{\centering\scriptsize}m{1.3em}|>{\centering}m{8.8em}|}
  % \caption{秦王政}\
  \toprule
  \SimHei \normalsize 年数 & \SimHei \scriptsize 公元 & \SimHei 大事件 \tabularnewline
  % \midrule
  \endfirsthead
  \toprule
  \SimHei \normalsize 年数 & \SimHei \scriptsize 公元 & \SimHei 大事件 \tabularnewline
  \midrule
  \endhead
  \midrule
  元年 & 681 & \tabularnewline\hline
  二年 & 682 & \tabularnewline
  \bottomrule
\end{longtable}

\subsection{永淳}

\begin{longtable}{|>{\centering\scriptsize}m{2em}|>{\centering\scriptsize}m{1.3em}|>{\centering}m{8.8em}|}
  % \caption{秦王政}\
  \toprule
  \SimHei \normalsize 年数 & \SimHei \scriptsize 公元 & \SimHei 大事件 \tabularnewline
  % \midrule
  \endfirsthead
  \toprule
  \SimHei \normalsize 年数 & \SimHei \scriptsize 公元 & \SimHei 大事件 \tabularnewline
  \midrule
  \endhead
  \midrule
  元年 & 682 & \tabularnewline\hline
  二年 & 683 & \tabularnewline
  \bottomrule
\end{longtable}

\subsection{弘道}

\begin{longtable}{|>{\centering\scriptsize}m{2em}|>{\centering\scriptsize}m{1.3em}|>{\centering}m{8.8em}|}
  % \caption{秦王政}\
  \toprule
  \SimHei \normalsize 年数 & \SimHei \scriptsize 公元 & \SimHei 大事件 \tabularnewline
  % \midrule
  \endfirsthead
  \toprule
  \SimHei \normalsize 年数 & \SimHei \scriptsize 公元 & \SimHei 大事件 \tabularnewline
  \midrule
  \endhead
  \midrule
  元年 & 683 & \tabularnewline
  \bottomrule
\end{longtable}


%%% Local Variables:
%%% mode: latex
%%% TeX-engine: xetex
%%% TeX-master: "../Main"
%%% End:

%% -*- coding: utf-8 -*-
%% Time-stamp: <Chen Wang: 2018-07-11 21:31:51>

\section{中宗\tiny(683-684)}

\subsection{嗣圣}

\begin{longtable}{|>{\centering\scriptsize}m{2em}|>{\centering\scriptsize}m{1.3em}|>{\centering}m{8.8em}|}
  % \caption{秦王政}\
  \toprule
  \SimHei \normalsize 年数 & \SimHei \scriptsize 公元 & \SimHei 大事件 \tabularnewline
  % \midrule
  \endfirsthead
  \toprule
  \SimHei \normalsize 年数 & \SimHei \scriptsize 公元 & \SimHei 大事件 \tabularnewline
  \midrule
  \endhead
  \midrule
  元年 & 684 & \tabularnewline
  \bottomrule
\end{longtable}


%%% Local Variables:
%%% mode: latex
%%% TeX-engine: xetex
%%% TeX-master: "../Main"
%%% End:

%% -*- coding: utf-8 -*-
%% Time-stamp: <Chen Wang: 2018-07-11 21:33:46>

\section{睿宗\tiny(684-690)}

\subsection{文明}

\begin{longtable}{|>{\centering\scriptsize}m{2em}|>{\centering\scriptsize}m{1.3em}|>{\centering}m{8.8em}|}
  % \caption{秦王政}\
  \toprule
  \SimHei \normalsize 年数 & \SimHei \scriptsize 公元 & \SimHei 大事件 \tabularnewline
  % \midrule
  \endfirsthead
  \toprule
  \SimHei \normalsize 年数 & \SimHei \scriptsize 公元 & \SimHei 大事件 \tabularnewline
  \midrule
  \endhead
  \midrule
  元年 & 684 & \tabularnewline
  \bottomrule
\end{longtable}

\subsection{光宅}

\begin{longtable}{|>{\centering\scriptsize}m{2em}|>{\centering\scriptsize}m{1.3em}|>{\centering}m{8.8em}|}
  % \caption{秦王政}\
  \toprule
  \SimHei \normalsize 年数 & \SimHei \scriptsize 公元 & \SimHei 大事件 \tabularnewline
  % \midrule
  \endfirsthead
  \toprule
  \SimHei \normalsize 年数 & \SimHei \scriptsize 公元 & \SimHei 大事件 \tabularnewline
  \midrule
  \endhead
  \midrule
  元年 & 684 & \tabularnewline
  \bottomrule
\end{longtable}

\subsection{垂拱}

\begin{longtable}{|>{\centering\scriptsize}m{2em}|>{\centering\scriptsize}m{1.3em}|>{\centering}m{8.8em}|}
  % \caption{秦王政}\
  \toprule
  \SimHei \normalsize 年数 & \SimHei \scriptsize 公元 & \SimHei 大事件 \tabularnewline
  % \midrule
  \endfirsthead
  \toprule
  \SimHei \normalsize 年数 & \SimHei \scriptsize 公元 & \SimHei 大事件 \tabularnewline
  \midrule
  \endhead
  \midrule
  元年 & 685 & \tabularnewline\hline
  二年 & 686 & \tabularnewline\hline
  三年 & 687 & \tabularnewline\hline
  四年 & 688 & \tabularnewline
  \bottomrule
\end{longtable}

\subsection{永昌}

\begin{longtable}{|>{\centering\scriptsize}m{2em}|>{\centering\scriptsize}m{1.3em}|>{\centering}m{8.8em}|}
  % \caption{秦王政}\
  \toprule
  \SimHei \normalsize 年数 & \SimHei \scriptsize 公元 & \SimHei 大事件 \tabularnewline
  % \midrule
  \endfirsthead
  \toprule
  \SimHei \normalsize 年数 & \SimHei \scriptsize 公元 & \SimHei 大事件 \tabularnewline
  \midrule
  \endhead
  \midrule
  元年 & 689 & \tabularnewline
  \bottomrule
\end{longtable}

\subsection{载初}

\begin{longtable}{|>{\centering\scriptsize}m{2em}|>{\centering\scriptsize}m{1.3em}|>{\centering}m{8.8em}|}
  % \caption{秦王政}\
  \toprule
  \SimHei \normalsize 年数 & \SimHei \scriptsize 公元 & \SimHei 大事件 \tabularnewline
  % \midrule
  \endfirsthead
  \toprule
  \SimHei \normalsize 年数 & \SimHei \scriptsize 公元 & \SimHei 大事件 \tabularnewline
  \midrule
  \endhead
  \midrule
  元年 & 689 & \tabularnewline\hline
  二年 & 690 & \tabularnewline
  \bottomrule
\end{longtable}



%%% Local Variables:
%%% mode: latex
%%% TeX-engine: xetex
%%% TeX-master: "../Main"
%%% End:

%% -*- coding: utf-8 -*-
%% Time-stamp: <Chen Wang: 2018-07-11 21:39:38>

\section{武曌\tiny(683-705)}

\subsection{天授}

\begin{longtable}{|>{\centering\scriptsize}m{2em}|>{\centering\scriptsize}m{1.3em}|>{\centering}m{8.8em}|}
  % \caption{秦王政}\
  \toprule
  \SimHei \normalsize 年数 & \SimHei \scriptsize 公元 & \SimHei 大事件 \tabularnewline
  % \midrule
  \endfirsthead
  \toprule
  \SimHei \normalsize 年数 & \SimHei \scriptsize 公元 & \SimHei 大事件 \tabularnewline
  \midrule
  \endhead
  \midrule
  元年 & 690 & \tabularnewline\hline
  二年 & 691 & \tabularnewline\hline
  三年 & 692 & \tabularnewline
  \bottomrule
\end{longtable}

\subsection{如意}

\begin{longtable}{|>{\centering\scriptsize}m{2em}|>{\centering\scriptsize}m{1.3em}|>{\centering}m{8.8em}|}
  % \caption{秦王政}\
  \toprule
  \SimHei \normalsize 年数 & \SimHei \scriptsize 公元 & \SimHei 大事件 \tabularnewline
  % \midrule
  \endfirsthead
  \toprule
  \SimHei \normalsize 年数 & \SimHei \scriptsize 公元 & \SimHei 大事件 \tabularnewline
  \midrule
  \endhead
  \midrule
  元年 & 692 & \tabularnewline
  \bottomrule
\end{longtable}

\subsection{长寿}

\begin{longtable}{|>{\centering\scriptsize}m{2em}|>{\centering\scriptsize}m{1.3em}|>{\centering}m{8.8em}|}
  % \caption{秦王政}\
  \toprule
  \SimHei \normalsize 年数 & \SimHei \scriptsize 公元 & \SimHei 大事件 \tabularnewline
  % \midrule
  \endfirsthead
  \toprule
  \SimHei \normalsize 年数 & \SimHei \scriptsize 公元 & \SimHei 大事件 \tabularnewline
  \midrule
  \endhead
  \midrule
  元年 & 692 & \tabularnewline\hline
  二年 & 693 & \tabularnewline\hline
  三年 & 694 & \tabularnewline
  \bottomrule
\end{longtable}

\subsection{延载}

\begin{longtable}{|>{\centering\scriptsize}m{2em}|>{\centering\scriptsize}m{1.3em}|>{\centering}m{8.8em}|}
  % \caption{秦王政}\
  \toprule
  \SimHei \normalsize 年数 & \SimHei \scriptsize 公元 & \SimHei 大事件 \tabularnewline
  % \midrule
  \endfirsthead
  \toprule
  \SimHei \normalsize 年数 & \SimHei \scriptsize 公元 & \SimHei 大事件 \tabularnewline
  \midrule
  \endhead
  \midrule
  元年 & 694 & \tabularnewline
  \bottomrule
\end{longtable}

\subsection{证圣}

\begin{longtable}{|>{\centering\scriptsize}m{2em}|>{\centering\scriptsize}m{1.3em}|>{\centering}m{8.8em}|}
  % \caption{秦王政}\
  \toprule
  \SimHei \normalsize 年数 & \SimHei \scriptsize 公元 & \SimHei 大事件 \tabularnewline
  % \midrule
  \endfirsthead
  \toprule
  \SimHei \normalsize 年数 & \SimHei \scriptsize 公元 & \SimHei 大事件 \tabularnewline
  \midrule
  \endhead
  \midrule
  元年 & 695 & \tabularnewline
  \bottomrule
\end{longtable}

\subsection{天册万岁}

\begin{longtable}{|>{\centering\scriptsize}m{2em}|>{\centering\scriptsize}m{1.3em}|>{\centering}m{8.8em}|}
  % \caption{秦王政}\
  \toprule
  \SimHei \normalsize 年数 & \SimHei \scriptsize 公元 & \SimHei 大事件 \tabularnewline
  % \midrule
  \endfirsthead
  \toprule
  \SimHei \normalsize 年数 & \SimHei \scriptsize 公元 & \SimHei 大事件 \tabularnewline
  \midrule
  \endhead
  \midrule
  元年 & 695 & \tabularnewline
  \bottomrule
\end{longtable}

\subsection{万岁登封}

\begin{longtable}{|>{\centering\scriptsize}m{2em}|>{\centering\scriptsize}m{1.3em}|>{\centering}m{8.8em}|}
  % \caption{秦王政}\
  \toprule
  \SimHei \normalsize 年数 & \SimHei \scriptsize 公元 & \SimHei 大事件 \tabularnewline
  % \midrule
  \endfirsthead
  \toprule
  \SimHei \normalsize 年数 & \SimHei \scriptsize 公元 & \SimHei 大事件 \tabularnewline
  \midrule
  \endhead
  \midrule
  元年 & 695 & \tabularnewline\hline
  二年 & 696 & \tabularnewline
  \bottomrule
\end{longtable}

\subsection{万岁通天}

\begin{longtable}{|>{\centering\scriptsize}m{2em}|>{\centering\scriptsize}m{1.3em}|>{\centering}m{8.8em}|}
  % \caption{秦王政}\
  \toprule
  \SimHei \normalsize 年数 & \SimHei \scriptsize 公元 & \SimHei 大事件 \tabularnewline
  % \midrule
  \endfirsthead
  \toprule
  \SimHei \normalsize 年数 & \SimHei \scriptsize 公元 & \SimHei 大事件 \tabularnewline
  \midrule
  \endhead
  \midrule
  元年 & 696 & \tabularnewline\hline
  二年 & 697 & \tabularnewline
  \bottomrule
\end{longtable}

\subsection{神功}

\begin{longtable}{|>{\centering\scriptsize}m{2em}|>{\centering\scriptsize}m{1.3em}|>{\centering}m{8.8em}|}
  % \caption{秦王政}\
  \toprule
  \SimHei \normalsize 年数 & \SimHei \scriptsize 公元 & \SimHei 大事件 \tabularnewline
  % \midrule
  \endfirsthead
  \toprule
  \SimHei \normalsize 年数 & \SimHei \scriptsize 公元 & \SimHei 大事件 \tabularnewline
  \midrule
  \endhead
  \midrule
  元年 & 697 & \tabularnewline
  \bottomrule
\end{longtable}

\subsection{圣历}

\begin{longtable}{|>{\centering\scriptsize}m{2em}|>{\centering\scriptsize}m{1.3em}|>{\centering}m{8.8em}|}
  % \caption{秦王政}\
  \toprule
  \SimHei \normalsize 年数 & \SimHei \scriptsize 公元 & \SimHei 大事件 \tabularnewline
  % \midrule
  \endfirsthead
  \toprule
  \SimHei \normalsize 年数 & \SimHei \scriptsize 公元 & \SimHei 大事件 \tabularnewline
  \midrule
  \endhead
  \midrule
  元年 & 698 & \tabularnewline\hline
  二年 & 699 & \tabularnewline\hline
  三年 & 700 & \tabularnewline
  \bottomrule
\end{longtable}

\subsection{久视}

\begin{longtable}{|>{\centering\scriptsize}m{2em}|>{\centering\scriptsize}m{1.3em}|>{\centering}m{8.8em}|}
  % \caption{秦王政}\
  \toprule
  \SimHei \normalsize 年数 & \SimHei \scriptsize 公元 & \SimHei 大事件 \tabularnewline
  % \midrule
  \endfirsthead
  \toprule
  \SimHei \normalsize 年数 & \SimHei \scriptsize 公元 & \SimHei 大事件 \tabularnewline
  \midrule
  \endhead
  \midrule
  元年 & 700 & \tabularnewline\hline
  二年 & 701 & \tabularnewline
  \bottomrule
\end{longtable}

\subsection{大足}

\begin{longtable}{|>{\centering\scriptsize}m{2em}|>{\centering\scriptsize}m{1.3em}|>{\centering}m{8.8em}|}
  % \caption{秦王政}\
  \toprule
  \SimHei \normalsize 年数 & \SimHei \scriptsize 公元 & \SimHei 大事件 \tabularnewline
  % \midrule
  \endfirsthead
  \toprule
  \SimHei \normalsize 年数 & \SimHei \scriptsize 公元 & \SimHei 大事件 \tabularnewline
  \midrule
  \endhead
  \midrule
  元年 & 701 & \tabularnewline
  \bottomrule
\end{longtable}

\subsection{长安}

\begin{longtable}{|>{\centering\scriptsize}m{2em}|>{\centering\scriptsize}m{1.3em}|>{\centering}m{8.8em}|}
  % \caption{秦王政}\
  \toprule
  \SimHei \normalsize 年数 & \SimHei \scriptsize 公元 & \SimHei 大事件 \tabularnewline
  % \midrule
  \endfirsthead
  \toprule
  \SimHei \normalsize 年数 & \SimHei \scriptsize 公元 & \SimHei 大事件 \tabularnewline
  \midrule
  \endhead
  \midrule
  元年 & 701 & \tabularnewline\hline
  二年 & 702 & \tabularnewline\hline
  三年 & 703 & \tabularnewline\hline
  四年 & 704 & \tabularnewline
  \bottomrule
\end{longtable}

\subsection{神龙}

\begin{longtable}{|>{\centering\scriptsize}m{2em}|>{\centering\scriptsize}m{1.3em}|>{\centering}m{8.8em}|}
  % \caption{秦王政}\
  \toprule
  \SimHei \normalsize 年数 & \SimHei \scriptsize 公元 & \SimHei 大事件 \tabularnewline
  % \midrule
  \endfirsthead
  \toprule
  \SimHei \normalsize 年数 & \SimHei \scriptsize 公元 & \SimHei 大事件 \tabularnewline
  \midrule
  \endhead
  \midrule
  元年 & 705 & \tabularnewline\hline
  二年 & 706 & \tabularnewline\hline
  三年 & 707 & \tabularnewline
  \bottomrule
\end{longtable}


%%% Local Variables:
%%% mode: latex
%%% TeX-engine: xetex
%%% TeX-master: "../Main"
%%% End:

%% -*- coding: utf-8 -*-
%% Time-stamp: <Chen Wang: 2018-07-11 21:40:56>

\section{中宗复辟\tiny(705-710)}

\subsection{景龙}

\begin{longtable}{|>{\centering\scriptsize}m{2em}|>{\centering\scriptsize}m{1.3em}|>{\centering}m{8.8em}|}
  % \caption{秦王政}\
  \toprule
  \SimHei \normalsize 年数 & \SimHei \scriptsize 公元 & \SimHei 大事件 \tabularnewline
  % \midrule
  \endfirsthead
  \toprule
  \SimHei \normalsize 年数 & \SimHei \scriptsize 公元 & \SimHei 大事件 \tabularnewline
  \midrule
  \endhead
  \midrule
  元年 & 707 & \tabularnewline\hline
  二年 & 708 & \tabularnewline\hline
  三年 & 709 & \tabularnewline\hline
  四年 & 710 & \tabularnewline
  \bottomrule
\end{longtable}


%%% Local Variables:
%%% mode: latex
%%% TeX-engine: xetex
%%% TeX-master: "../Main"
%%% End:

%% -*- coding: utf-8 -*-
%% Time-stamp: <Chen Wang: 2018-07-11 21:49:36>

\section{睿宗复辟\tiny(710-712)}

\subsection{景云}

\begin{longtable}{|>{\centering\scriptsize}m{2em}|>{\centering\scriptsize}m{1.3em}|>{\centering}m{8.8em}|}
  % \caption{秦王政}\
  \toprule
  \SimHei \normalsize 年数 & \SimHei \scriptsize 公元 & \SimHei 大事件 \tabularnewline
  % \midrule
  \endfirsthead
  \toprule
  \SimHei \normalsize 年数 & \SimHei \scriptsize 公元 & \SimHei 大事件 \tabularnewline
  \midrule
  \endhead
  \midrule
  元年 & 710 & \tabularnewline\hline
  二年 & 711 & \tabularnewline\hline
  三年 & 712 & \tabularnewline
  \bottomrule
\end{longtable}

\subsection{太极}

\begin{longtable}{|>{\centering\scriptsize}m{2em}|>{\centering\scriptsize}m{1.3em}|>{\centering}m{8.8em}|}
  % \caption{秦王政}\
  \toprule
  \SimHei \normalsize 年数 & \SimHei \scriptsize 公元 & \SimHei 大事件 \tabularnewline
  % \midrule
  \endfirsthead
  \toprule
  \SimHei \normalsize 年数 & \SimHei \scriptsize 公元 & \SimHei 大事件 \tabularnewline
  \midrule
  \endhead
  \midrule
  元年 & 712 & \tabularnewline
  \bottomrule
\end{longtable}

\subsection{延和}

\begin{longtable}{|>{\centering\scriptsize}m{2em}|>{\centering\scriptsize}m{1.3em}|>{\centering}m{8.8em}|}
  % \caption{秦王政}\
  \toprule
  \SimHei \normalsize 年数 & \SimHei \scriptsize 公元 & \SimHei 大事件 \tabularnewline
  % \midrule
  \endfirsthead
  \toprule
  \SimHei \normalsize 年数 & \SimHei \scriptsize 公元 & \SimHei 大事件 \tabularnewline
  \midrule
  \endhead
  \midrule
  元年 & 712 & \tabularnewline
  \bottomrule
\end{longtable}


%%% Local Variables:
%%% mode: latex
%%% TeX-engine: xetex
%%% TeX-master: "../Main"
%%% End:

%% -*- coding: utf-8 -*-
%% Time-stamp: <Chen Wang: 2018-07-11 21:51:42>

\section{玄宗\tiny(712-756)}

\subsection{先天}

\begin{longtable}{|>{\centering\scriptsize}m{2em}|>{\centering\scriptsize}m{1.3em}|>{\centering}m{8.8em}|}
  % \caption{秦王政}\
  \toprule
  \SimHei \normalsize 年数 & \SimHei \scriptsize 公元 & \SimHei 大事件 \tabularnewline
  % \midrule
  \endfirsthead
  \toprule
  \SimHei \normalsize 年数 & \SimHei \scriptsize 公元 & \SimHei 大事件 \tabularnewline
  \midrule
  \endhead
  \midrule
  元年 & 712 & \tabularnewline\hline
  二年 & 713 & \tabularnewline
  \bottomrule
\end{longtable}

\subsection{开元}

\begin{longtable}{|>{\centering\scriptsize}m{2em}|>{\centering\scriptsize}m{1.3em}|>{\centering}m{8.8em}|}
  % \caption{秦王政}\
  \toprule
  \SimHei \normalsize 年数 & \SimHei \scriptsize 公元 & \SimHei 大事件 \tabularnewline
  % \midrule
  \endfirsthead
  \toprule
  \SimHei \normalsize 年数 & \SimHei \scriptsize 公元 & \SimHei 大事件 \tabularnewline
  \midrule
  \endhead
  \midrule
  元年 & 713 & \tabularnewline\hline
  二年 & 714 & \tabularnewline\hline
  三年 & 715 & \tabularnewline\hline
  四年 & 716 & \tabularnewline\hline
  五年 & 717 & \tabularnewline\hline
  六年 & 718 & \tabularnewline\hline
  七年 & 719 & \tabularnewline\hline
  八年 & 720 & \tabularnewline\hline
  九年 & 721 & \tabularnewline\hline
  十年 & 722 & \tabularnewline\hline
  十一年 & 723 & \tabularnewline\hline
  十二年 & 724 & \tabularnewline\hline
  十三年 & 725 & \tabularnewline\hline
  十四年 & 726 & \tabularnewline\hline
  十五年 & 727 & \tabularnewline\hline
  十六年 & 728 & \tabularnewline\hline
  十七年 & 729 & \tabularnewline\hline
  十八年 & 730 & \tabularnewline\hline
  十九年 & 731 & \tabularnewline\hline
  二十年 & 732 & \tabularnewline\hline
  二一年 & 733 & \tabularnewline\hline
  二二年 & 734 & \tabularnewline\hline
  二三年 & 735 & \tabularnewline\hline
  二四年 & 736 & \tabularnewline\hline
  二五年 & 737 & \tabularnewline\hline
  二六年 & 738 & \tabularnewline\hline
  二七年 & 739 & \tabularnewline\hline
  二八年 & 740 & \tabularnewline\hline
  二九年 & 741 & \tabularnewline
  \bottomrule
\end{longtable}

\subsection{天宝}

\begin{longtable}{|>{\centering\scriptsize}m{2em}|>{\centering\scriptsize}m{1.3em}|>{\centering}m{8.8em}|}
  % \caption{秦王政}\
  \toprule
  \SimHei \normalsize 年数 & \SimHei \scriptsize 公元 & \SimHei 大事件 \tabularnewline
  % \midrule
  \endfirsthead
  \toprule
  \SimHei \normalsize 年数 & \SimHei \scriptsize 公元 & \SimHei 大事件 \tabularnewline
  \midrule
  \endhead
  \midrule
  元年 & 742 & \tabularnewline\hline
  二年 & 743 & \tabularnewline\hline
  三年 & 744 & \tabularnewline\hline
  四年 & 745 & \tabularnewline\hline
  五年 & 746 & \tabularnewline\hline
  六年 & 747 & \tabularnewline\hline
  七年 & 748 & \tabularnewline\hline
  八年 & 749 & \tabularnewline\hline
  九年 & 750 & \tabularnewline\hline
  十年 & 751 & \tabularnewline\hline
  十一年 & 752 & \tabularnewline\hline
  十二年 & 753 & \tabularnewline\hline
  十三年 & 754 & \tabularnewline\hline
  十四年 & 755 & \tabularnewline\hline
  十五年 & 756 & \tabularnewline
  \bottomrule
\end{longtable}


%%% Local Variables:
%%% mode: latex
%%% TeX-engine: xetex
%%% TeX-master: "../Main"
%%% End:

%% -*- coding: utf-8 -*-
%% Time-stamp: <Chen Wang: 2018-07-11 21:53:35>

\section{肃宗\tiny(756-762)}

\subsection{至德}

\begin{longtable}{|>{\centering\scriptsize}m{2em}|>{\centering\scriptsize}m{1.3em}|>{\centering}m{8.8em}|}
  % \caption{秦王政}\
  \toprule
  \SimHei \normalsize 年数 & \SimHei \scriptsize 公元 & \SimHei 大事件 \tabularnewline
  % \midrule
  \endfirsthead
  \toprule
  \SimHei \normalsize 年数 & \SimHei \scriptsize 公元 & \SimHei 大事件 \tabularnewline
  \midrule
  \endhead
  \midrule
  元年 & 756 & \tabularnewline\hline
  二年 & 757 & \tabularnewline\hline
  三年 & 758 & \tabularnewline
  \bottomrule
\end{longtable}

\subsection{乾元}

\begin{longtable}{|>{\centering\scriptsize}m{2em}|>{\centering\scriptsize}m{1.3em}|>{\centering}m{8.8em}|}
  % \caption{秦王政}\
  \toprule
  \SimHei \normalsize 年数 & \SimHei \scriptsize 公元 & \SimHei 大事件 \tabularnewline
  % \midrule
  \endfirsthead
  \toprule
  \SimHei \normalsize 年数 & \SimHei \scriptsize 公元 & \SimHei 大事件 \tabularnewline
  \midrule
  \endhead
  \midrule
  元年 & 758 & \tabularnewline\hline
  二年 & 759 & \tabularnewline\hline
  三年 & 760 & \tabularnewline
  \bottomrule
\end{longtable}

\subsection{上元}

\begin{longtable}{|>{\centering\scriptsize}m{2em}|>{\centering\scriptsize}m{1.3em}|>{\centering}m{8.8em}|}
  % \caption{秦王政}\
  \toprule
  \SimHei \normalsize 年数 & \SimHei \scriptsize 公元 & \SimHei 大事件 \tabularnewline
  % \midrule
  \endfirsthead
  \toprule
  \SimHei \normalsize 年数 & \SimHei \scriptsize 公元 & \SimHei 大事件 \tabularnewline
  \midrule
  \endhead
  \midrule
  元年 & 760 & \tabularnewline\hline
  二年 & 761 & \tabularnewline
  \bottomrule
\end{longtable}

\subsection{宝应}

\begin{longtable}{|>{\centering\scriptsize}m{2em}|>{\centering\scriptsize}m{1.3em}|>{\centering}m{8.8em}|}
  % \caption{秦王政}\
  \toprule
  \SimHei \normalsize 年数 & \SimHei \scriptsize 公元 & \SimHei 大事件 \tabularnewline
  % \midrule
  \endfirsthead
  \toprule
  \SimHei \normalsize 年数 & \SimHei \scriptsize 公元 & \SimHei 大事件 \tabularnewline
  \midrule
  \endhead
  \midrule
  元年 & 762 & \tabularnewline\hline
  二年 & 763 & \tabularnewline
  \bottomrule
\end{longtable}


%%% Local Variables:
%%% mode: latex
%%% TeX-engine: xetex
%%% TeX-master: "../Main"
%%% End:

%% -*- coding: utf-8 -*-
%% Time-stamp: <Chen Wang: 2018-07-11 21:55:17>

\section{代宗\tiny(762-779)}

\subsection{广德}

\begin{longtable}{|>{\centering\scriptsize}m{2em}|>{\centering\scriptsize}m{1.3em}|>{\centering}m{8.8em}|}
  % \caption{秦王政}\
  \toprule
  \SimHei \normalsize 年数 & \SimHei \scriptsize 公元 & \SimHei 大事件 \tabularnewline
  % \midrule
  \endfirsthead
  \toprule
  \SimHei \normalsize 年数 & \SimHei \scriptsize 公元 & \SimHei 大事件 \tabularnewline
  \midrule
  \endhead
  \midrule
  元年 & 763 & \tabularnewline\hline
  二年 & 764 & \tabularnewline
  \bottomrule
\end{longtable}

\subsection{永泰}

\begin{longtable}{|>{\centering\scriptsize}m{2em}|>{\centering\scriptsize}m{1.3em}|>{\centering}m{8.8em}|}
  % \caption{秦王政}\
  \toprule
  \SimHei \normalsize 年数 & \SimHei \scriptsize 公元 & \SimHei 大事件 \tabularnewline
  % \midrule
  \endfirsthead
  \toprule
  \SimHei \normalsize 年数 & \SimHei \scriptsize 公元 & \SimHei 大事件 \tabularnewline
  \midrule
  \endhead
  \midrule
  元年 & 765 & \tabularnewline\hline
  二年 & 766 & \tabularnewline
  \bottomrule
\end{longtable}

\subsection{大历}

\begin{longtable}{|>{\centering\scriptsize}m{2em}|>{\centering\scriptsize}m{1.3em}|>{\centering}m{8.8em}|}
  % \caption{秦王政}\
  \toprule
  \SimHei \normalsize 年数 & \SimHei \scriptsize 公元 & \SimHei 大事件 \tabularnewline
  % \midrule
  \endfirsthead
  \toprule
  \SimHei \normalsize 年数 & \SimHei \scriptsize 公元 & \SimHei 大事件 \tabularnewline
  \midrule
  \endhead
  \midrule
  元年 & 766 & \tabularnewline\hline
  二年 & 767 & \tabularnewline\hline
  三年 & 768 & \tabularnewline\hline
  四年 & 769 & \tabularnewline\hline
  五年 & 770 & \tabularnewline\hline
  六年 & 771 & \tabularnewline\hline
  七年 & 772 & \tabularnewline\hline
  八年 & 773 & \tabularnewline\hline
  九年 & 774 & \tabularnewline\hline
  十年 & 775 & \tabularnewline\hline
  十一年 & 776 & \tabularnewline\hline
  十二年 & 777 & \tabularnewline\hline
  十三年 & 778 & \tabularnewline\hline
  十四年 & 779 & \tabularnewline
  \bottomrule
\end{longtable}


%%% Local Variables:
%%% mode: latex
%%% TeX-engine: xetex
%%% TeX-master: "../Main"
%%% End:

%% -*- coding: utf-8 -*-
%% Time-stamp: <Chen Wang: 2018-07-11 21:57:17>

\section{德宗\tiny(779-805)}

\subsection{建中}

\begin{longtable}{|>{\centering\scriptsize}m{2em}|>{\centering\scriptsize}m{1.3em}|>{\centering}m{8.8em}|}
  % \caption{秦王政}\
  \toprule
  \SimHei \normalsize 年数 & \SimHei \scriptsize 公元 & \SimHei 大事件 \tabularnewline
  % \midrule
  \endfirsthead
  \toprule
  \SimHei \normalsize 年数 & \SimHei \scriptsize 公元 & \SimHei 大事件 \tabularnewline
  \midrule
  \endhead
  \midrule
  元年 & 780 & \tabularnewline\hline
  二年 & 781 & \tabularnewline\hline
  三年 & 782 & \tabularnewline\hline
  四年 & 783 & \tabularnewline
  \bottomrule
\end{longtable}

\subsection{兴元}

\begin{longtable}{|>{\centering\scriptsize}m{2em}|>{\centering\scriptsize}m{1.3em}|>{\centering}m{8.8em}|}
  % \caption{秦王政}\
  \toprule
  \SimHei \normalsize 年数 & \SimHei \scriptsize 公元 & \SimHei 大事件 \tabularnewline
  % \midrule
  \endfirsthead
  \toprule
  \SimHei \normalsize 年数 & \SimHei \scriptsize 公元 & \SimHei 大事件 \tabularnewline
  \midrule
  \endhead
  \midrule
  元年 & 784 & \tabularnewline
  \bottomrule
\end{longtable}

\subsection{贞元}

\begin{longtable}{|>{\centering\scriptsize}m{2em}|>{\centering\scriptsize}m{1.3em}|>{\centering}m{8.8em}|}
  % \caption{秦王政}\
  \toprule
  \SimHei \normalsize 年数 & \SimHei \scriptsize 公元 & \SimHei 大事件 \tabularnewline
  % \midrule
  \endfirsthead
  \toprule
  \SimHei \normalsize 年数 & \SimHei \scriptsize 公元 & \SimHei 大事件 \tabularnewline
  \midrule
  \endhead
  \midrule
  元年 & 785 & \tabularnewline\hline
  二年 & 786 & \tabularnewline\hline
  三年 & 787 & \tabularnewline\hline
  四年 & 788 & \tabularnewline\hline
  五年 & 789 & \tabularnewline\hline
  六年 & 790 & \tabularnewline\hline
  七年 & 791 & \tabularnewline\hline
  八年 & 792 & \tabularnewline\hline
  九年 & 793 & \tabularnewline\hline
  十年 & 794 & \tabularnewline\hline
  十一年 & 795 & \tabularnewline\hline
  十二年 & 796 & \tabularnewline\hline
  十三年 & 797 & \tabularnewline\hline
  十四年 & 798 & \tabularnewline\hline
  十五年 & 799 & \tabularnewline\hline
  十六年 & 800 & \tabularnewline\hline
  十七年 & 801 & \tabularnewline\hline
  十八年 & 802 & \tabularnewline\hline
  十九年 & 803 & \tabularnewline\hline
  二十年 & 804 & \tabularnewline\hline
  二一年 & 805 & \tabularnewline
  \bottomrule
\end{longtable}


%%% Local Variables:
%%% mode: latex
%%% TeX-engine: xetex
%%% TeX-master: "../Main"
%%% End:

%% -*- coding: utf-8 -*-
%% Time-stamp: <Chen Wang: 2018-07-11 21:58:08>

\section{顺宗\tiny(805)}

\subsection{永贞}

\begin{longtable}{|>{\centering\scriptsize}m{2em}|>{\centering\scriptsize}m{1.3em}|>{\centering}m{8.8em}|}
  % \caption{秦王政}\
  \toprule
  \SimHei \normalsize 年数 & \SimHei \scriptsize 公元 & \SimHei 大事件 \tabularnewline
  % \midrule
  \endfirsthead
  \toprule
  \SimHei \normalsize 年数 & \SimHei \scriptsize 公元 & \SimHei 大事件 \tabularnewline
  \midrule
  \endhead
  \midrule
  元年 & 805 & \tabularnewline
  \bottomrule
\end{longtable}


%%% Local Variables:
%%% mode: latex
%%% TeX-engine: xetex
%%% TeX-master: "../Main"
%%% End:

%% -*- coding: utf-8 -*-
%% Time-stamp: <Chen Wang: 2018-07-11 21:59:10>

\section{宪宗\tiny(805-820)}

\subsection{元和}

\begin{longtable}{|>{\centering\scriptsize}m{2em}|>{\centering\scriptsize}m{1.3em}|>{\centering}m{8.8em}|}
  % \caption{秦王政}\
  \toprule
  \SimHei \normalsize 年数 & \SimHei \scriptsize 公元 & \SimHei 大事件 \tabularnewline
  % \midrule
  \endfirsthead
  \toprule
  \SimHei \normalsize 年数 & \SimHei \scriptsize 公元 & \SimHei 大事件 \tabularnewline
  \midrule
  \endhead
  \midrule
  元年 & 806 & \tabularnewline\hline
  二年 & 807 & \tabularnewline\hline
  三年 & 808 & \tabularnewline\hline
  四年 & 809 & \tabularnewline\hline
  五年 & 810 & \tabularnewline\hline
  六年 & 811 & \tabularnewline\hline
  七年 & 812 & \tabularnewline\hline
  八年 & 813 & \tabularnewline\hline
  九年 & 814 & \tabularnewline\hline
  十年 & 815 & \tabularnewline\hline
  十一年 & 816 & \tabularnewline\hline
  十二年 & 817 & \tabularnewline\hline
  十三年 & 818 & \tabularnewline\hline
  十四年 & 819 & \tabularnewline\hline
  十五年 & 820 & \tabularnewline
  \bottomrule
\end{longtable}


%%% Local Variables:
%%% mode: latex
%%% TeX-engine: xetex
%%% TeX-master: "../Main"
%%% End:

%% -*- coding: utf-8 -*-
%% Time-stamp: <Chen Wang: 2018-07-11 22:00:29>

\section{穆宗\tiny(820-824)}

\subsection{永新}

\begin{longtable}{|>{\centering\scriptsize}m{2em}|>{\centering\scriptsize}m{1.3em}|>{\centering}m{8.8em}|}
  % \caption{秦王政}\
  \toprule
  \SimHei \normalsize 年数 & \SimHei \scriptsize 公元 & \SimHei 大事件 \tabularnewline
  % \midrule
  \endfirsthead
  \toprule
  \SimHei \normalsize 年数 & \SimHei \scriptsize 公元 & \SimHei 大事件 \tabularnewline
  \midrule
  \endhead
  \midrule
  元年 & 820 & \tabularnewline
  \bottomrule
\end{longtable}

\subsection{长庆}

\begin{longtable}{|>{\centering\scriptsize}m{2em}|>{\centering\scriptsize}m{1.3em}|>{\centering}m{8.8em}|}
  % \caption{秦王政}\
  \toprule
  \SimHei \normalsize 年数 & \SimHei \scriptsize 公元 & \SimHei 大事件 \tabularnewline
  % \midrule
  \endfirsthead
  \toprule
  \SimHei \normalsize 年数 & \SimHei \scriptsize 公元 & \SimHei 大事件 \tabularnewline
  \midrule
  \endhead
  \midrule
  元年 & 821 & \tabularnewline\hline
  二年 & 822 & \tabularnewline\hline
  三年 & 823 & \tabularnewline\hline
  四年 & 824 & \tabularnewline
  \bottomrule
\end{longtable}


%%% Local Variables:
%%% mode: latex
%%% TeX-engine: xetex
%%% TeX-master: "../Main"
%%% End:

%% -*- coding: utf-8 -*-
%% Time-stamp: <Chen Wang: 2018-07-11 22:01:12>

\section{敬宗\tiny(824-826)}

\subsection{宝历}

\begin{longtable}{|>{\centering\scriptsize}m{2em}|>{\centering\scriptsize}m{1.3em}|>{\centering}m{8.8em}|}
  % \caption{秦王政}\
  \toprule
  \SimHei \normalsize 年数 & \SimHei \scriptsize 公元 & \SimHei 大事件 \tabularnewline
  % \midrule
  \endfirsthead
  \toprule
  \SimHei \normalsize 年数 & \SimHei \scriptsize 公元 & \SimHei 大事件 \tabularnewline
  \midrule
  \endhead
  \midrule
  元年 & 825 & \tabularnewline\hline
  二年 & 826 & \tabularnewline\hline
  三年 & 827 & \tabularnewline
  \bottomrule
\end{longtable}


%%% Local Variables:
%%% mode: latex
%%% TeX-engine: xetex
%%% TeX-master: "../Main"
%%% End:

%% -*- coding: utf-8 -*-
%% Time-stamp: <Chen Wang: 2018-07-11 22:03:11>

\section{文宗\tiny(826-840)}

\subsection{大和}

\begin{longtable}{|>{\centering\scriptsize}m{2em}|>{\centering\scriptsize}m{1.3em}|>{\centering}m{8.8em}|}
  % \caption{秦王政}\
  \toprule
  \SimHei \normalsize 年数 & \SimHei \scriptsize 公元 & \SimHei 大事件 \tabularnewline
  % \midrule
  \endfirsthead
  \toprule
  \SimHei \normalsize 年数 & \SimHei \scriptsize 公元 & \SimHei 大事件 \tabularnewline
  \midrule
  \endhead
  \midrule
  元年 & 827 & \tabularnewline\hline
  二年 & 828 & \tabularnewline\hline
  三年 & 829 & \tabularnewline\hline
  四年 & 830 & \tabularnewline\hline
  五年 & 831 & \tabularnewline\hline
  六年 & 832 & \tabularnewline\hline
  七年 & 833 & \tabularnewline\hline
  八年 & 834 & \tabularnewline\hline
  九年 & 835 & \tabularnewline
  \bottomrule
\end{longtable}

\subsection{开成}

\begin{longtable}{|>{\centering\scriptsize}m{2em}|>{\centering\scriptsize}m{1.3em}|>{\centering}m{8.8em}|}
  % \caption{秦王政}\
  \toprule
  \SimHei \normalsize 年数 & \SimHei \scriptsize 公元 & \SimHei 大事件 \tabularnewline
  % \midrule
  \endfirsthead
  \toprule
  \SimHei \normalsize 年数 & \SimHei \scriptsize 公元 & \SimHei 大事件 \tabularnewline
  \midrule
  \endhead
  \midrule
  元年 & 836 & \tabularnewline\hline
  二年 & 837 & \tabularnewline\hline
  三年 & 838 & \tabularnewline\hline
  四年 & 839 & \tabularnewline\hline
  五年 & 840 & \tabularnewline
  \bottomrule
\end{longtable}


%%% Local Variables:
%%% mode: latex
%%% TeX-engine: xetex
%%% TeX-master: "../Main"
%%% End:

%% -*- coding: utf-8 -*-
%% Time-stamp: <Chen Wang: 2018-07-11 22:04:49>

\section{武宗\tiny(840-846)}

\subsection{会昌}

\begin{longtable}{|>{\centering\scriptsize}m{2em}|>{\centering\scriptsize}m{1.3em}|>{\centering}m{8.8em}|}
  % \caption{秦王政}\
  \toprule
  \SimHei \normalsize 年数 & \SimHei \scriptsize 公元 & \SimHei 大事件 \tabularnewline
  % \midrule
  \endfirsthead
  \toprule
  \SimHei \normalsize 年数 & \SimHei \scriptsize 公元 & \SimHei 大事件 \tabularnewline
  \midrule
  \endhead
  \midrule
  元年 & 841 & \tabularnewline\hline
  二年 & 842 & \tabularnewline\hline
  三年 & 843 & \tabularnewline\hline
  四年 & 844 & \tabularnewline\hline
  五年 & 845 & \tabularnewline\hline
  六年 & 846 & \tabularnewline
  \bottomrule
\end{longtable}


%%% Local Variables:
%%% mode: latex
%%% TeX-engine: xetex
%%% TeX-master: "../Main"
%%% End:

%% -*- coding: utf-8 -*-
%% Time-stamp: <Chen Wang: 2018-07-11 22:06:16>

\section{宣宗\tiny(846-859)}

\subsection{大中}

\begin{longtable}{|>{\centering\scriptsize}m{2em}|>{\centering\scriptsize}m{1.3em}|>{\centering}m{8.8em}|}
  % \caption{秦王政}\
  \toprule
  \SimHei \normalsize 年数 & \SimHei \scriptsize 公元 & \SimHei 大事件 \tabularnewline
  % \midrule
  \endfirsthead
  \toprule
  \SimHei \normalsize 年数 & \SimHei \scriptsize 公元 & \SimHei 大事件 \tabularnewline
  \midrule
  \endhead
  \midrule
  元年 & 847 & \tabularnewline\hline
  二年 & 848 & \tabularnewline\hline
  三年 & 849 & \tabularnewline\hline
  四年 & 850 & \tabularnewline\hline
  五年 & 851 & \tabularnewline\hline
  六年 & 852 & \tabularnewline\hline
  七年 & 853 & \tabularnewline\hline
  八年 & 854 & \tabularnewline\hline
  九年 & 855 & \tabularnewline\hline
  十年 & 856 & \tabularnewline\hline
  十一年 & 857 & \tabularnewline\hline
  十二年 & 858 & \tabularnewline\hline
  十三年 & 859 & \tabularnewline\hline
  十四年 & 860 & \tabularnewline
  \bottomrule
\end{longtable}


%%% Local Variables:
%%% mode: latex
%%% TeX-engine: xetex
%%% TeX-master: "../Main"
%%% End:

%% -*- coding: utf-8 -*-
%% Time-stamp: <Chen Wang: 2018-07-11 22:07:59>

\section{懿宗\tiny(859-873)}

\subsection{咸通}

\begin{longtable}{|>{\centering\scriptsize}m{2em}|>{\centering\scriptsize}m{1.3em}|>{\centering}m{8.8em}|}
  % \caption{秦王政}\
  \toprule
  \SimHei \normalsize 年数 & \SimHei \scriptsize 公元 & \SimHei 大事件 \tabularnewline
  % \midrule
  \endfirsthead
  \toprule
  \SimHei \normalsize 年数 & \SimHei \scriptsize 公元 & \SimHei 大事件 \tabularnewline
  \midrule
  \endhead
  \midrule
  元年 & 860 & \tabularnewline\hline
  二年 & 861 & \tabularnewline\hline
  三年 & 862 & \tabularnewline\hline
  四年 & 863 & \tabularnewline\hline
  五年 & 864 & \tabularnewline\hline
  六年 & 865 & \tabularnewline\hline
  七年 & 866 & \tabularnewline\hline
  八年 & 867 & \tabularnewline\hline
  九年 & 868 & \tabularnewline\hline
  十年 & 869 & \tabularnewline\hline
  十一年 & 870 & \tabularnewline\hline
  十二年 & 871 & \tabularnewline\hline
  十三年 & 872 & \tabularnewline\hline
  十四年 & 873 & \tabularnewline\hline
  十五年 & 874 & \tabularnewline
  \bottomrule
\end{longtable}


%%% Local Variables:
%%% mode: latex
%%% TeX-engine: xetex
%%% TeX-master: "../Main"
%%% End:

%% -*- coding: utf-8 -*-
%% Time-stamp: <Chen Wang: 2018-07-11 22:10:19>

\section{僖宗\tiny(873-888)}

\subsection{乾符}

\begin{longtable}{|>{\centering\scriptsize}m{2em}|>{\centering\scriptsize}m{1.3em}|>{\centering}m{8.8em}|}
  % \caption{秦王政}\
  \toprule
  \SimHei \normalsize 年数 & \SimHei \scriptsize 公元 & \SimHei 大事件 \tabularnewline
  % \midrule
  \endfirsthead
  \toprule
  \SimHei \normalsize 年数 & \SimHei \scriptsize 公元 & \SimHei 大事件 \tabularnewline
  \midrule
  \endhead
  \midrule
  元年 & 874 & \tabularnewline\hline
  二年 & 875 & \tabularnewline\hline
  三年 & 876 & \tabularnewline\hline
  四年 & 877 & \tabularnewline\hline
  五年 & 878 & \tabularnewline\hline
  六年 & 879 & \tabularnewline
  \bottomrule
\end{longtable}

\subsection{广明}

\begin{longtable}{|>{\centering\scriptsize}m{2em}|>{\centering\scriptsize}m{1.3em}|>{\centering}m{8.8em}|}
  % \caption{秦王政}\
  \toprule
  \SimHei \normalsize 年数 & \SimHei \scriptsize 公元 & \SimHei 大事件 \tabularnewline
  % \midrule
  \endfirsthead
  \toprule
  \SimHei \normalsize 年数 & \SimHei \scriptsize 公元 & \SimHei 大事件 \tabularnewline
  \midrule
  \endhead
  \midrule
  元年 & 880 & \tabularnewline\hline
  二年 & 881 & \tabularnewline
  \bottomrule
\end{longtable}

\subsection{中和}

\begin{longtable}{|>{\centering\scriptsize}m{2em}|>{\centering\scriptsize}m{1.3em}|>{\centering}m{8.8em}|}
  % \caption{秦王政}\
  \toprule
  \SimHei \normalsize 年数 & \SimHei \scriptsize 公元 & \SimHei 大事件 \tabularnewline
  % \midrule
  \endfirsthead
  \toprule
  \SimHei \normalsize 年数 & \SimHei \scriptsize 公元 & \SimHei 大事件 \tabularnewline
  \midrule
  \endhead
  \midrule
  元年 & 881 & \tabularnewline\hline
  二年 & 882 & \tabularnewline\hline
  三年 & 883 & \tabularnewline\hline
  四年 & 884 & \tabularnewline\hline
  五年 & 885 & \tabularnewline
  \bottomrule
\end{longtable}

\subsection{光启}

\begin{longtable}{|>{\centering\scriptsize}m{2em}|>{\centering\scriptsize}m{1.3em}|>{\centering}m{8.8em}|}
  % \caption{秦王政}\
  \toprule
  \SimHei \normalsize 年数 & \SimHei \scriptsize 公元 & \SimHei 大事件 \tabularnewline
  % \midrule
  \endfirsthead
  \toprule
  \SimHei \normalsize 年数 & \SimHei \scriptsize 公元 & \SimHei 大事件 \tabularnewline
  \midrule
  \endhead
  \midrule
  元年 & 885 & \tabularnewline\hline
  二年 & 886 & \tabularnewline\hline
  三年 & 887 & \tabularnewline\hline
  四年 & 888 & \tabularnewline
  \bottomrule
\end{longtable}

\subsection{文德}

\begin{longtable}{|>{\centering\scriptsize}m{2em}|>{\centering\scriptsize}m{1.3em}|>{\centering}m{8.8em}|}
  % \caption{秦王政}\
  \toprule
  \SimHei \normalsize 年数 & \SimHei \scriptsize 公元 & \SimHei 大事件 \tabularnewline
  % \midrule
  \endfirsthead
  \toprule
  \SimHei \normalsize 年数 & \SimHei \scriptsize 公元 & \SimHei 大事件 \tabularnewline
  \midrule
  \endhead
  \midrule
  元年 & 888 & \tabularnewline
  \bottomrule
\end{longtable}


%%% Local Variables:
%%% mode: latex
%%% TeX-engine: xetex
%%% TeX-master: "../Main"
%%% End:

%% -*- coding: utf-8 -*-
%% Time-stamp: <Chen Wang: 2018-07-11 22:18:39>

\section{昭宗\tiny(888-904)}

\subsection{龙纪}

\begin{longtable}{|>{\centering\scriptsize}m{2em}|>{\centering\scriptsize}m{1.3em}|>{\centering}m{8.8em}|}
  % \caption{秦王政}\
  \toprule
  \SimHei \normalsize 年数 & \SimHei \scriptsize 公元 & \SimHei 大事件 \tabularnewline
  % \midrule
  \endfirsthead
  \toprule
  \SimHei \normalsize 年数 & \SimHei \scriptsize 公元 & \SimHei 大事件 \tabularnewline
  \midrule
  \endhead
  \midrule
  元年 & 889 & \tabularnewline
  \bottomrule
\end{longtable}

\subsection{大顺}

\begin{longtable}{|>{\centering\scriptsize}m{2em}|>{\centering\scriptsize}m{1.3em}|>{\centering}m{8.8em}|}
  % \caption{秦王政}\
  \toprule
  \SimHei \normalsize 年数 & \SimHei \scriptsize 公元 & \SimHei 大事件 \tabularnewline
  % \midrule
  \endfirsthead
  \toprule
  \SimHei \normalsize 年数 & \SimHei \scriptsize 公元 & \SimHei 大事件 \tabularnewline
  \midrule
  \endhead
  \midrule
  元年 & 890 & \tabularnewline\hline
  二年 & 891 & \tabularnewline
  \bottomrule
\end{longtable}

\subsection{景福}

\begin{longtable}{|>{\centering\scriptsize}m{2em}|>{\centering\scriptsize}m{1.3em}|>{\centering}m{8.8em}|}
  % \caption{秦王政}\
  \toprule
  \SimHei \normalsize 年数 & \SimHei \scriptsize 公元 & \SimHei 大事件 \tabularnewline
  % \midrule
  \endfirsthead
  \toprule
  \SimHei \normalsize 年数 & \SimHei \scriptsize 公元 & \SimHei 大事件 \tabularnewline
  \midrule
  \endhead
  \midrule
  元年 & 892 & \tabularnewline\hline
  二年 & 893 & \tabularnewline
  \bottomrule
\end{longtable}

\subsection{乾宁}

\begin{longtable}{|>{\centering\scriptsize}m{2em}|>{\centering\scriptsize}m{1.3em}|>{\centering}m{8.8em}|}
  % \caption{秦王政}\
  \toprule
  \SimHei \normalsize 年数 & \SimHei \scriptsize 公元 & \SimHei 大事件 \tabularnewline
  % \midrule
  \endfirsthead
  \toprule
  \SimHei \normalsize 年数 & \SimHei \scriptsize 公元 & \SimHei 大事件 \tabularnewline
  \midrule
  \endhead
  \midrule
  元年 & 894 & \tabularnewline\hline
  二年 & 895 & \tabularnewline\hline
  三年 & 896 & \tabularnewline\hline
  四年 & 897 & \tabularnewline\hline
  五年 & 898 & \tabularnewline
  \bottomrule
\end{longtable}

\subsection{光化}

\begin{longtable}{|>{\centering\scriptsize}m{2em}|>{\centering\scriptsize}m{1.3em}|>{\centering}m{8.8em}|}
  % \caption{秦王政}\
  \toprule
  \SimHei \normalsize 年数 & \SimHei \scriptsize 公元 & \SimHei 大事件 \tabularnewline
  % \midrule
  \endfirsthead
  \toprule
  \SimHei \normalsize 年数 & \SimHei \scriptsize 公元 & \SimHei 大事件 \tabularnewline
  \midrule
  \endhead
  \midrule
  元年 & 898 & \tabularnewline\hline
  二年 & 899 & \tabularnewline\hline
  三年 & 900 & \tabularnewline\hline
  四年 & 901 & \tabularnewline
  \bottomrule
\end{longtable}

\subsection{天复}

\begin{longtable}{|>{\centering\scriptsize}m{2em}|>{\centering\scriptsize}m{1.3em}|>{\centering}m{8.8em}|}
  % \caption{秦王政}\
  \toprule
  \SimHei \normalsize 年数 & \SimHei \scriptsize 公元 & \SimHei 大事件 \tabularnewline
  % \midrule
  \endfirsthead
  \toprule
  \SimHei \normalsize 年数 & \SimHei \scriptsize 公元 & \SimHei 大事件 \tabularnewline
  \midrule
  \endhead
  \midrule
  元年 & 901 & \tabularnewline\hline
  二年 & 902 & \tabularnewline\hline
  三年 & 903 & \tabularnewline\hline
  四年 & 904 & \tabularnewline
  \bottomrule
\end{longtable}


%%% Local Variables:
%%% mode: latex
%%% TeX-engine: xetex
%%% TeX-master: "../Main"
%%% End:

%% -*- coding: utf-8 -*-
%% Time-stamp: <Chen Wang: 2018-07-11 22:19:18>

\section{景宗\tiny(904-907)}

\subsection{天佑}

\begin{longtable}{|>{\centering\scriptsize}m{2em}|>{\centering\scriptsize}m{1.3em}|>{\centering}m{8.8em}|}
  % \caption{秦王政}\
  \toprule
  \SimHei \normalsize 年数 & \SimHei \scriptsize 公元 & \SimHei 大事件 \tabularnewline
  % \midrule
  \endfirsthead
  \toprule
  \SimHei \normalsize 年数 & \SimHei \scriptsize 公元 & \SimHei 大事件 \tabularnewline
  \midrule
  \endhead
  \midrule
  元年 & 904 & \tabularnewline\hline
  二年 & 905 & \tabularnewline\hline
  三年 & 906 & \tabularnewline\hline
  四年 & 907 & \tabularnewline
  \bottomrule
\end{longtable}


%%% Local Variables:
%%% mode: latex
%%% TeX-engine: xetex
%%% TeX-master: "../Main"
%%% End:


%%% Local Variables:
%%% mode: latex
%%% TeX-engine: xetex
%%% TeX-master: "../Main"
%%% End:
 % 唐
% %% -*- coding: utf-8 -*-
%% Time-stamp: <Chen Wang: 2019-10-15 11:17:24>

\chapter{五代\tiny(907-960)}


%% -*- coding: utf-8 -*-
%% Time-stamp: <Chen Wang: 2019-10-15 11:17:44>


\section{后梁\tiny(907-923)}

%% -*- coding: utf-8 -*-
%% Time-stamp: <Chen Wang: 2018-07-11 22:25:54>

\subsection{太祖\tiny(907-912)}

\subsubsection{开平}

\begin{longtable}{|>{\centering\scriptsize}m{2em}|>{\centering\scriptsize}m{1.3em}|>{\centering}m{8.8em}|}
  % \caption{秦王政}\
  \toprule
  \SimHei \normalsize 年数 & \SimHei \scriptsize 公元 & \SimHei 大事件 \tabularnewline
  % \midrule
  \endfirsthead
  \toprule
  \SimHei \normalsize 年数 & \SimHei \scriptsize 公元 & \SimHei 大事件 \tabularnewline
  \midrule
  \endhead
  \midrule
  元年 & 907 & \tabularnewline\hline
  二年 & 908 & \tabularnewline\hline
  三年 & 909 & \tabularnewline\hline
  四年 & 910 & \tabularnewline\hline
  五年 & 911 & \tabularnewline
  \bottomrule
\end{longtable}

\subsubsection{乾化}

\begin{longtable}{|>{\centering\scriptsize}m{2em}|>{\centering\scriptsize}m{1.3em}|>{\centering}m{8.8em}|}
  % \caption{秦王政}\
  \toprule
  \SimHei \normalsize 年数 & \SimHei \scriptsize 公元 & \SimHei 大事件 \tabularnewline
  % \midrule
  \endfirsthead
  \toprule
  \SimHei \normalsize 年数 & \SimHei \scriptsize 公元 & \SimHei 大事件 \tabularnewline
  \midrule
  \endhead
  \midrule
  元年 & 911 & \tabularnewline\hline
  二年 & 912 & \tabularnewline\hline
  三年 & 913 & \tabularnewline
  \bottomrule
\end{longtable}


%%% Local Variables:
%%% mode: latex
%%% TeX-engine: xetex
%%% TeX-master: "../../Main"
%%% End:

%% -*- coding: utf-8 -*-
%% Time-stamp: <Chen Wang: 2018-07-11 22:33:05>

\subsection{朱友珪\tiny(912-913)}

\subsubsection{凤历}

\begin{longtable}{|>{\centering\scriptsize}m{2em}|>{\centering\scriptsize}m{1.3em}|>{\centering}m{8.8em}|}
  % \caption{秦王政}\
  \toprule
  \SimHei \normalsize 年数 & \SimHei \scriptsize 公元 & \SimHei 大事件 \tabularnewline
  % \midrule
  \endfirsthead
  \toprule
  \SimHei \normalsize 年数 & \SimHei \scriptsize 公元 & \SimHei 大事件 \tabularnewline
  \midrule
  \endhead
  \midrule
  元年 & 913 & \tabularnewline
  \bottomrule
\end{longtable}


%%% Local Variables:
%%% mode: latex
%%% TeX-engine: xetex
%%% TeX-master: "../../Main"
%%% End:

%% -*- coding: utf-8 -*-
%% Time-stamp: <Chen Wang: 2018-07-11 22:34:19>

\subsection{朱友贞\tiny(913-923)}

\subsubsection{乾化}

\begin{longtable}{|>{\centering\scriptsize}m{2em}|>{\centering\scriptsize}m{1.3em}|>{\centering}m{8.8em}|}
  % \caption{秦王政}\
  \toprule
  \SimHei \normalsize 年数 & \SimHei \scriptsize 公元 & \SimHei 大事件 \tabularnewline
  % \midrule
  \endfirsthead
  \toprule
  \SimHei \normalsize 年数 & \SimHei \scriptsize 公元 & \SimHei 大事件 \tabularnewline
  \midrule
  \endhead
  \midrule
  元年 & 913 & \tabularnewline\hline
  二年 & 914 & \tabularnewline\hline
  三年 & 915 & \tabularnewline
  \bottomrule
\end{longtable}

\subsubsection{贞明}

\begin{longtable}{|>{\centering\scriptsize}m{2em}|>{\centering\scriptsize}m{1.3em}|>{\centering}m{8.8em}|}
  % \caption{秦王政}\
  \toprule
  \SimHei \normalsize 年数 & \SimHei \scriptsize 公元 & \SimHei 大事件 \tabularnewline
  % \midrule
  \endfirsthead
  \toprule
  \SimHei \normalsize 年数 & \SimHei \scriptsize 公元 & \SimHei 大事件 \tabularnewline
  \midrule
  \endhead
  \midrule
  元年 & 915 & \tabularnewline\hline
  二年 & 916 & \tabularnewline\hline
  三年 & 917 & \tabularnewline\hline
  四年 & 918 & \tabularnewline\hline
  五年 & 919 & \tabularnewline\hline
  六年 & 920 & \tabularnewline\hline
  七年 & 921 & \tabularnewline
  \bottomrule
\end{longtable}

\subsubsection{龙德}

\begin{longtable}{|>{\centering\scriptsize}m{2em}|>{\centering\scriptsize}m{1.3em}|>{\centering}m{8.8em}|}
  % \caption{秦王政}\
  \toprule
  \SimHei \normalsize 年数 & \SimHei \scriptsize 公元 & \SimHei 大事件 \tabularnewline
  % \midrule
  \endfirsthead
  \toprule
  \SimHei \normalsize 年数 & \SimHei \scriptsize 公元 & \SimHei 大事件 \tabularnewline
  \midrule
  \endhead
  \midrule
  元年 & 921 & \tabularnewline\hline
  二年 & 922 & \tabularnewline\hline
  三年 & 923 & \tabularnewline
  \bottomrule
\end{longtable}


%%% Local Variables:
%%% mode: latex
%%% TeX-engine: xetex
%%% TeX-master: "../../Main"
%%% End:


%%% Local Variables:
%%% mode: latex
%%% TeX-engine: xetex
%%% TeX-master: "../../Main"
%%% End:

%% -*- coding: utf-8 -*-
%% Time-stamp: <Chen Wang: 2019-10-15 11:17:51>


\section{后唐\tiny(923-937)}

%% -*- coding: utf-8 -*-
%% Time-stamp: <Chen Wang: 2018-07-11 22:51:01>

\subsection{庄宗\tiny(923-926)}

\subsubsection{同光}

\begin{longtable}{|>{\centering\scriptsize}m{2em}|>{\centering\scriptsize}m{1.3em}|>{\centering}m{8.8em}|}
  % \caption{秦王政}\
  \toprule
  \SimHei \normalsize 年数 & \SimHei \scriptsize 公元 & \SimHei 大事件 \tabularnewline
  % \midrule
  \endfirsthead
  \toprule
  \SimHei \normalsize 年数 & \SimHei \scriptsize 公元 & \SimHei 大事件 \tabularnewline
  \midrule
  \endhead
  \midrule
  元年 & 923 & \tabularnewline\hline
  二年 & 924 & \tabularnewline\hline
  三年 & 925 & \tabularnewline\hline
  四年 & 926 & \tabularnewline
  \bottomrule
\end{longtable}


%%% Local Variables:
%%% mode: latex
%%% TeX-engine: xetex
%%% TeX-master: "../../Main"
%%% End:

%% -*- coding: utf-8 -*-
%% Time-stamp: <Chen Wang: 2018-07-11 22:52:07>

\subsection{明宗\tiny(926-933)}

\subsubsection{天成}

\begin{longtable}{|>{\centering\scriptsize}m{2em}|>{\centering\scriptsize}m{1.3em}|>{\centering}m{8.8em}|}
  % \caption{秦王政}\
  \toprule
  \SimHei \normalsize 年数 & \SimHei \scriptsize 公元 & \SimHei 大事件 \tabularnewline
  % \midrule
  \endfirsthead
  \toprule
  \SimHei \normalsize 年数 & \SimHei \scriptsize 公元 & \SimHei 大事件 \tabularnewline
  \midrule
  \endhead
  \midrule
  元年 & 926 & \tabularnewline\hline
  二年 & 927 & \tabularnewline\hline
  三年 & 928 & \tabularnewline\hline
  四年 & 929 & \tabularnewline\hline
  五年 & 930 & \tabularnewline
  \bottomrule
\end{longtable}

\subsubsection{长兴}

\begin{longtable}{|>{\centering\scriptsize}m{2em}|>{\centering\scriptsize}m{1.3em}|>{\centering}m{8.8em}|}
  % \caption{秦王政}\
  \toprule
  \SimHei \normalsize 年数 & \SimHei \scriptsize 公元 & \SimHei 大事件 \tabularnewline
  % \midrule
  \endfirsthead
  \toprule
  \SimHei \normalsize 年数 & \SimHei \scriptsize 公元 & \SimHei 大事件 \tabularnewline
  \midrule
  \endhead
  \midrule
  元年 & 930 & \tabularnewline\hline
  二年 & 931 & \tabularnewline\hline
  三年 & 932 & \tabularnewline\hline
  四年 & 933 & \tabularnewline
  \bottomrule
\end{longtable}


%%% Local Variables:
%%% mode: latex
%%% TeX-engine: xetex
%%% TeX-master: "../../Main"
%%% End:

%% -*- coding: utf-8 -*-
%% Time-stamp: <Chen Wang: 2018-07-11 22:52:45>

\subsection{闵帝\tiny(933-934)}

\subsubsection{应顺}

\begin{longtable}{|>{\centering\scriptsize}m{2em}|>{\centering\scriptsize}m{1.3em}|>{\centering}m{8.8em}|}
  % \caption{秦王政}\
  \toprule
  \SimHei \normalsize 年数 & \SimHei \scriptsize 公元 & \SimHei 大事件 \tabularnewline
  % \midrule
  \endfirsthead
  \toprule
  \SimHei \normalsize 年数 & \SimHei \scriptsize 公元 & \SimHei 大事件 \tabularnewline
  \midrule
  \endhead
  \midrule
  元年 & 934 & \tabularnewline
  \bottomrule
\end{longtable}


%%% Local Variables:
%%% mode: latex
%%% TeX-engine: xetex
%%% TeX-master: "../../Main"
%%% End:

%% -*- coding: utf-8 -*-
%% Time-stamp: <Chen Wang: 2018-07-11 22:53:21>

\subsection{李从珂\tiny(934-937)}

\subsubsection{清泰}

\begin{longtable}{|>{\centering\scriptsize}m{2em}|>{\centering\scriptsize}m{1.3em}|>{\centering}m{8.8em}|}
  % \caption{秦王政}\
  \toprule
  \SimHei \normalsize 年数 & \SimHei \scriptsize 公元 & \SimHei 大事件 \tabularnewline
  % \midrule
  \endfirsthead
  \toprule
  \SimHei \normalsize 年数 & \SimHei \scriptsize 公元 & \SimHei 大事件 \tabularnewline
  \midrule
  \endhead
  \midrule
  元年 & 934 & \tabularnewline\hline
  二年 & 935 & \tabularnewline\hline
  三年 & 936 & \tabularnewline
  \bottomrule
\end{longtable}


%%% Local Variables:
%%% mode: latex
%%% TeX-engine: xetex
%%% TeX-master: "../../Main"
%%% End:


%%% Local Variables:
%%% mode: latex
%%% TeX-engine: xetex
%%% TeX-master: "../../Main"
%%% End:

%% -*- coding: utf-8 -*-
%% Time-stamp: <Chen Wang: 2019-10-15 11:17:38>


\section{后晋\tiny(936-947)}

%% -*- coding: utf-8 -*-
%% Time-stamp: <Chen Wang: 2018-07-11 22:55:41>

\subsection{高祖\tiny(936-942)}

\subsubsection{天福}

\begin{longtable}{|>{\centering\scriptsize}m{2em}|>{\centering\scriptsize}m{1.3em}|>{\centering}m{8.8em}|}
  % \caption{秦王政}\
  \toprule
  \SimHei \normalsize 年数 & \SimHei \scriptsize 公元 & \SimHei 大事件 \tabularnewline
  % \midrule
  \endfirsthead
  \toprule
  \SimHei \normalsize 年数 & \SimHei \scriptsize 公元 & \SimHei 大事件 \tabularnewline
  \midrule
  \endhead
  \midrule
  元年 & 936 & \tabularnewline\hline
  二年 & 937 & \tabularnewline\hline
  三年 & 938 & \tabularnewline\hline
  四年 & 939 & \tabularnewline\hline
  五年 & 940 & \tabularnewline\hline
  六年 & 941 & \tabularnewline\hline
  七年 & 942 & \tabularnewline\hline
  八年 & 943 & \tabularnewline\hline
  九年 & 944 & \tabularnewline
  \bottomrule
\end{longtable}


%%% Local Variables:
%%% mode: latex
%%% TeX-engine: xetex
%%% TeX-master: "../../Main"
%%% End:

%% -*- coding: utf-8 -*-
%% Time-stamp: <Chen Wang: 2018-07-11 22:56:16>

\subsection{出帝\tiny(942-946)}

\subsubsection{开运}

\begin{longtable}{|>{\centering\scriptsize}m{2em}|>{\centering\scriptsize}m{1.3em}|>{\centering}m{8.8em}|}
  % \caption{秦王政}\
  \toprule
  \SimHei \normalsize 年数 & \SimHei \scriptsize 公元 & \SimHei 大事件 \tabularnewline
  % \midrule
  \endfirsthead
  \toprule
  \SimHei \normalsize 年数 & \SimHei \scriptsize 公元 & \SimHei 大事件 \tabularnewline
  \midrule
  \endhead
  \midrule
  元年 & 944 & \tabularnewline\hline
  二年 & 945 & \tabularnewline\hline
  三年 & 946 & \tabularnewline
  \bottomrule
\end{longtable}


%%% Local Variables:
%%% mode: latex
%%% TeX-engine: xetex
%%% TeX-master: "../../Main"
%%% End:



%%% Local Variables:
%%% mode: latex
%%% TeX-engine: xetex
%%% TeX-master: "../../Main"
%%% End:

%% -*- coding: utf-8 -*-
%% Time-stamp: <Chen Wang: 2019-10-15 11:17:31>


\section{后汉\tiny(947-951)}

%% -*- coding: utf-8 -*-
%% Time-stamp: <Chen Wang: 2018-07-11 22:59:02>

\subsection{高祖\tiny(947-948)}

\subsubsection{天福}

\begin{longtable}{|>{\centering\scriptsize}m{2em}|>{\centering\scriptsize}m{1.3em}|>{\centering}m{8.8em}|}
  % \caption{秦王政}\
  \toprule
  \SimHei \normalsize 年数 & \SimHei \scriptsize 公元 & \SimHei 大事件 \tabularnewline
  % \midrule
  \endfirsthead
  \toprule
  \SimHei \normalsize 年数 & \SimHei \scriptsize 公元 & \SimHei 大事件 \tabularnewline
  \midrule
  \endhead
  \midrule
  元年 & 947 & \tabularnewline
  \bottomrule
\end{longtable}

\subsubsection{乾祐}

\begin{longtable}{|>{\centering\scriptsize}m{2em}|>{\centering\scriptsize}m{1.3em}|>{\centering}m{8.8em}|}
  % \caption{秦王政}\
  \toprule
  \SimHei \normalsize 年数 & \SimHei \scriptsize 公元 & \SimHei 大事件 \tabularnewline
  % \midrule
  \endfirsthead
  \toprule
  \SimHei \normalsize 年数 & \SimHei \scriptsize 公元 & \SimHei 大事件 \tabularnewline
  \midrule
  \endhead
  \midrule
  元年 & 948 & \tabularnewline\hline
  \bottomrule
\end{longtable}


%%% Local Variables:
%%% mode: latex
%%% TeX-engine: xetex
%%% TeX-master: "../../Main"
%%% End:

%% -*- coding: utf-8 -*-
%% Time-stamp: <Chen Wang: 2018-07-11 22:59:44>

\subsection{隐帝\tiny(948-950)}

\subsubsection{乾祐}

\begin{longtable}{|>{\centering\scriptsize}m{2em}|>{\centering\scriptsize}m{1.3em}|>{\centering}m{8.8em}|}
  % \caption{秦王政}\
  \toprule
  \SimHei \normalsize 年数 & \SimHei \scriptsize 公元 & \SimHei 大事件 \tabularnewline
  % \midrule
  \endfirsthead
  \toprule
  \SimHei \normalsize 年数 & \SimHei \scriptsize 公元 & \SimHei 大事件 \tabularnewline
  \midrule
  \endhead
  \midrule
  元年 & 948 & \tabularnewline\hline
  二年 & 949 & \tabularnewline\hline
  三年 & 950 & \tabularnewline
  \bottomrule
\end{longtable}


%%% Local Variables:
%%% mode: latex
%%% TeX-engine: xetex
%%% TeX-master: "../../Main"
%%% End:


%%% Local Variables:
%%% mode: latex
%%% TeX-engine: xetex
%%% TeX-master: "../../Main"
%%% End:

%% -*- coding: utf-8 -*-
%% Time-stamp: <Chen Wang: 2019-10-15 11:17:57>


\section{后周\tiny(951-960)}

%% -*- coding: utf-8 -*-
%% Time-stamp: <Chen Wang: 2018-07-11 23:02:40>

\subsection{太祖\tiny(951-954)}

\subsubsection{广顺}

\begin{longtable}{|>{\centering\scriptsize}m{2em}|>{\centering\scriptsize}m{1.3em}|>{\centering}m{8.8em}|}
  % \caption{秦王政}\
  \toprule
  \SimHei \normalsize 年数 & \SimHei \scriptsize 公元 & \SimHei 大事件 \tabularnewline
  % \midrule
  \endfirsthead
  \toprule
  \SimHei \normalsize 年数 & \SimHei \scriptsize 公元 & \SimHei 大事件 \tabularnewline
  \midrule
  \endhead
  \midrule
  元年 & 951 & \tabularnewline\hline
  二年 & 952 & \tabularnewline\hline
  三年 & 953 & \tabularnewline\hline
  四年 & 954 & \tabularnewline
  \bottomrule
\end{longtable}

\subsubsection{显德}

\begin{longtable}{|>{\centering\scriptsize}m{2em}|>{\centering\scriptsize}m{1.3em}|>{\centering}m{8.8em}|}
  % \caption{秦王政}\
  \toprule
  \SimHei \normalsize 年数 & \SimHei \scriptsize 公元 & \SimHei 大事件 \tabularnewline
  % \midrule
  \endfirsthead
  \toprule
  \SimHei \normalsize 年数 & \SimHei \scriptsize 公元 & \SimHei 大事件 \tabularnewline
  \midrule
  \endhead
  \midrule
  元年 & 954 & \tabularnewline
  \bottomrule
\end{longtable}


%%% Local Variables:
%%% mode: latex
%%% TeX-engine: xetex
%%% TeX-master: "../../Main"
%%% End:

%% -*- coding: utf-8 -*-
%% Time-stamp: <Chen Wang: 2018-07-11 23:03:18>

\subsection{世宗\tiny(954-959)}

\subsubsection{显德}

\begin{longtable}{|>{\centering\scriptsize}m{2em}|>{\centering\scriptsize}m{1.3em}|>{\centering}m{8.8em}|}
  % \caption{秦王政}\
  \toprule
  \SimHei \normalsize 年数 & \SimHei \scriptsize 公元 & \SimHei 大事件 \tabularnewline
  % \midrule
  \endfirsthead
  \toprule
  \SimHei \normalsize 年数 & \SimHei \scriptsize 公元 & \SimHei 大事件 \tabularnewline
  \midrule
  \endhead
  \midrule
  元年 & 954 & \tabularnewline\hline
  二年 & 955 & \tabularnewline\hline
  三年 & 956 & \tabularnewline\hline
  四年 & 957 & \tabularnewline\hline
  五年 & 958 & \tabularnewline\hline
  六年 & 959 & \tabularnewline
  \bottomrule
\end{longtable}


%%% Local Variables:
%%% mode: latex
%%% TeX-engine: xetex
%%% TeX-master: "../../Main"
%%% End:

%% -*- coding: utf-8 -*-
%% Time-stamp: <Chen Wang: 2018-07-11 23:04:07>

\subsection{恭帝\tiny(959-960)}

\subsubsection{显德}

\begin{longtable}{|>{\centering\scriptsize}m{2em}|>{\centering\scriptsize}m{1.3em}|>{\centering}m{8.8em}|}
  % \caption{秦王政}\
  \toprule
  \SimHei \normalsize 年数 & \SimHei \scriptsize 公元 & \SimHei 大事件 \tabularnewline
  % \midrule
  \endfirsthead
  \toprule
  \SimHei \normalsize 年数 & \SimHei \scriptsize 公元 & \SimHei 大事件 \tabularnewline
  \midrule
  \endhead
  \midrule
  元年 & 959 & \tabularnewline\hline
  二年 & 960 & \tabularnewline
  \bottomrule
\end{longtable}


%%% Local Variables:
%%% mode: latex
%%% TeX-engine: xetex
%%% TeX-master: "../../Main"
%%% End:


%%% Local Variables:
%%% mode: latex
%%% TeX-engine: xetex
%%% TeX-master: "../../Main"
%%% End:



%%% Local Variables:
%%% mode: latex
%%% TeX-engine: xetex
%%% TeX-master: "../Main"
%%% End:
 % 五代
% %% -*- coding: utf-8 -*-
%% Time-stamp: <Chen Wang: 2019-12-24 17:11:06>

\chapter{十国\tiny(907-979)}


%% -*- coding: utf-8 -*-
%% Time-stamp: <Chen Wang: 2019-12-24 17:42:35>


\section{吴\tiny(902-937)}

\subsection{简介}

吴(902年-937年)是五代时十国之一,为杨行密所建,又称杨吴、南吴、弘農、淮南。

唐昭宗景福元年(892年)杨行密为唐淮南节度使,据扬州。天复二年(902年)封为吴王。建都广陵(即扬州),称江都府。杨行密注意招集拒絕朱溫統治的唐人,奖励农桑,使江淮一带社会经济有所恢复。

吴最强盛时,有今江苏、安徽、江西和湖北等省的一部分。唐哀帝天祐二年(905年)杨行密去世,其子杨渥继位,仅称弘农郡王,在唐朝灭亡后不承认后梁,仍用天祐年号,兼并镇南军,但不久遭牙将张颢、徐温夺权杀害,不久徐温杀张颢,以摄政身份掌握吴国大权。天祐七年(910年),弘农郡王杨隆演复称吴王,919年,称吴国王,建年号,以示脱离唐朝体系,以徐温为大丞相。徐温死后,大权落入其养子宰相和继任摄政徐知誥之手。顺义七年(927年),吴国王楊溥称帝,其实是为徐知诰篡位称帝做准备。吴天祚三年(937年),徐知诰杀死图谋反抗的历阳郡公杨濛,迫使杨溥禅让,建立南唐。吴国共历4主,36年,但大部分时间杨氏受徐氏控制。

楊吳的统治地区包括今江苏、安徽、江西、湖北等一部分。

%% -*- coding: utf-8 -*-
%% Time-stamp: <Chen Wang: 2019-12-24 17:47:12>

\subsection{武帝\tiny(902-905)}

\subsubsection{生平}

吳孝武王杨行密(852年-905年),字化源,原名行愍,庐州合肥(今安徽合肥长丰)人,唐朝末年著名政治家、军事家,五代十国時期吴国政權奠定者。唐乾宁二年(895年)进同中书门下平章事、弘农郡王。天复二年(902)进中书令、封吴王,天佑二年(905)病死,唐谥武忠王,吴国武义年间改谥孝武王,其子杨溥稱帝时,追尊其为武皇帝,庙号太祖。

杨行密原为庐州牙将,中和三年(883)拜庐州刺史,归淮南节度使高骈。886年,高骈賜名為杨行密。

中和五年(885)毕师铎反高骈,召宣歙观察使秦彦助战,高骈向行密求救,行密尚未赶到,毕师铎已俘高骈,行密一到,大败毕师铎,秦彥一气之下,杀死高骈,行密占领扬州,毕师铎投奔秦宗权部下孙儒,孙儒杀毕师铎并吞并其军队,发兵围攻扬州,欲一举消灭杨行密,将扬州收归己有。

行密采谋士袁袭建议,放弃扬州,先退守庐州(今安徽合肥),后攻克宣州(今安徽宣城)。龙纪元年(889)拜宣州观察使。行密據有宣州後,趁势向东、南、西三个方向发展,占领苏州、常州、润州(今江苏镇江)、滁州、和州(今安徽和县)等地,势力急剧扩大,领地包括了现在的安徽、江苏、浙江和江西、湖北等省部分地区。

景福元年(892)取楚州(今江苏淮安)、杨行密的发展,使占有扬州的孙儒受到三面包围,孙儒杀向宣州。行密击溃孙儒,并当众將之斬殺。复入扬州,进淮南节度使。

此后行密又出兵扩大地盘,将淮河以南和长江以东大片领土都纳入自己势力范围,为后来楊吳疆土基本上定型。乾宁二年(895)进同中书门下平章事、弘农郡王。

行密为扩大势力范围,乘蔡州四面行营都统朱温忙于对兖州、郓州(治今山东郓城)用兵之机,主动出击作战。攻取濠州(治今安徽凤阳)、寿州(治今安徽寿县),袭占涟水。又通过对依附于朱温的州县主动出击作战,拓展地盘,阻遏朱温插足淮南,为稳定和发展自己势力创造条件。

乾宁四年(897)宣武节度使朱温大举南侵,行密亲战,先集中自己的精锐主力攻击东边庞师古部,命朱瑾掘开淮河河堤,用水大淹庞师古部,同时遣朱瑾、张训领兵击败庞师古部于清口(今江苏淮安),宣武军损失惨重,大败而归,庞师古阵亡,葛从周逃回。朱温此后即无力南下,此后数十年间,南北遂成分裂之局。

吴越王钱鏐派兵攻打行密,兵進苏州。行密命周本禦敵,卻作战失利,失苏州。行密经过充分准备,派李神福进攻钱鏐,于杭州大败钱鏐军队并活捉其大将顾全武。经过长期混战,行密在江淮一带立足。

天复二年(902)进中书令、封吴王。天复三年(903年),行密遣李神福击破武昌节度使杜洪于君山(今湖南岳阳)。朱温败于行密后,向东进攻王师范,王师范求救于行密。行密于是年四月遣王茂章领兵出征,六月王茂章击破朱温军队,朱温再不敢对江淮用兵。八月至十二月,宁国节度使田頵与润州团练使安仁义起兵反行密,行密令李神福、臺濛、王茂章等将击溃田頵部于芜湖,广德、黄池(今安徽当涂)、宣州(治今安徽宣城)等地,田頵亡于宣州战场。令王茂章击破安仁义于润州并斩于广陵(今江苏扬州)。后行密诈瞎诛杀朱延寿叛逆势力。

天佑二年(905)十一月庚辰(二十六)日(12月24日)吴王杨行密病逝,其子杨渥继立。唐朝谥武忠王,吴国武义年间改谥孝武王,杨溥即帝位时追尊其为武皇帝,庙号太祖。

行密为政颇能选拔贤才,招集流散,轻徭薄赋,劝课农桑,使江淮一带社会经济在战争的间隙有较大恢复。

大唐衰变后,藩鎮割據,諸侯并起。行密在江淮地区举起割据大旗,强力遏止中原軍閥朱温南进步伐,成功避免全国更大范围动乱。经略淮南过程中,其政治方略、经济措施和军事思想,对五代十国及其后来社会产生深远影响。其奠基之吴国,初步实现由藩镇向王国转型,继而,南方割据势力与北方中原政权并存局面得以实现。

政治上,行密为後代的南唐奠定经济文化基础,开启唐宋之交政治整合和经济文化中心南渐先河,原因在于一仍吴旧的南唐是南方最为重要的割据政权,中国古代经济与文化中心的初步南移实际上是在以南唐为龙头、以吴越和马楚等政权为呼应的统治区域内实现的,这个时期是唐宋之交社会分野的标点,为后来社会的强劲发展提供了前瞻、新鲜的要素。杨行密经略江淮,实为十国第一人。


\subsubsection{天复}

\begin{longtable}{|>{\centering\scriptsize}m{2em}|>{\centering\scriptsize}m{1.3em}|>{\centering}m{8.8em}|}
  % \caption{秦王政}\
  \toprule
  \SimHei \normalsize 年数 & \SimHei \scriptsize 公元 & \SimHei 大事件 \tabularnewline
  % \midrule
  \endfirsthead
  \toprule
  \SimHei \normalsize 年数 & \SimHei \scriptsize 公元 & \SimHei 大事件 \tabularnewline
  \midrule
  \endhead
  \midrule
  元年 & 902 & \tabularnewline\hline
  二年 & 903 & \tabularnewline\hline
  三年 & 904 & \tabularnewline
  \bottomrule
\end{longtable}

\subsubsection{天祐}

\begin{longtable}{|>{\centering\scriptsize}m{2em}|>{\centering\scriptsize}m{1.3em}|>{\centering}m{8.8em}|}
  % \caption{秦王政}\
  \toprule
  \SimHei \normalsize 年数 & \SimHei \scriptsize 公元 & \SimHei 大事件 \tabularnewline
  % \midrule
  \endfirsthead
  \toprule
  \SimHei \normalsize 年数 & \SimHei \scriptsize 公元 & \SimHei 大事件 \tabularnewline
  \midrule
  \endhead
  \midrule
  元年 & 904 & \tabularnewline\hline
  二年 & 905 & \tabularnewline
  \bottomrule
\end{longtable}



%%% Local Variables:
%%% mode: latex
%%% TeX-engine: xetex
%%% TeX-master: "../../Main"
%%% End:

%% -*- coding: utf-8 -*-
%% Time-stamp: <Chen Wang: 2019-12-24 17:47:48>

\subsection{杨渥\tiny(905-908)}

\subsubsection{生平}

楊渥(886年-908年6月9日),字奉天(《九国志》作承天),五代十國時期南吳君主,但從未稱「吳王」,南吳太祖楊行密長子。

楊行密在位時,任牙內諸軍使,楊行密晚年病重後被任命為宣州觀察使。杨行密临终欲召回杨渥以传位,节度判官周隐认为杨渥不务正业且好酒,反对其继业,建议杨行密将领地托管给庐州刺史刘威,等杨行密诸子年长后刘威自然会还政。但在左牙(衙)指揮使張顥、右牙(衙)指揮使徐溫等劝说下,杨行密仍然决定传位杨渥。徐温和幕僚严可求得知周隐没有发出召回杨渥的牒文,夺取牒文将其发出,杨渥回到扬州军部。

唐哀帝天祐二年(905年),楊行密過世,楊渥嗣位,為宣諭使李儼承制授為淮南節度使、東南諸道行營都統、兼侍中、弘農郡王。

楊渥喜好遊玩作樂,居丧当中燃十围之烛以击毬,一烛费钱数万。又常单骑出游,左右莫知所之。天祐三年(906年)杨渥任命的西南行营都招讨使秦裴吞并镇南军,杨渥愈发骄傲,杀死周隐,致使将佐不自安。张颢、徐温屢勸,杨渥不聽,说:“你们认为我不才,为什么不杀了我,自己坐我的位子!”其親信又不斷欺壓元勳舊臣,將領們頗感不安。楊渥为修建毬场,将扬州牙城中的亲军悉数迁出,張顥、徐溫二人因此无所忌惮。他们让杨渥从宣州带来的指挥使朱思勍、范思从、陈璠帮助秦裴平定镇南军,又诬陷三将谋反,派别将陈祐前去秦裴帐中处死三人。杨渥因而想杀死张颢、徐温,但天祐四年(907年),張顥、徐溫抢先發動兵變,露刃入宫,以铁挝击杀楊渥亲信数十人。此后诸将与張顥、徐溫意见不同者,辄被杀,二人遂控制軍政。楊渥大權盡失。

天祐五年(908年)五月戊寅,張顥、徐溫遣亲信纪祥、陈晖、黎璠、孙殷等人入子城,弑楊渥于寝室。杨渥说:“你们要是杀了张颢、徐温,我让你们做刺史。”很多刺客都被说动,但纪祥仍将杨渥缢死。杨渥终年二十三岁,张颢、徐温对外声称暴卒。死後諡威王(弘農威王);楊隆演登南吳國王位時,改諡景王(南吳景王),廟號烈祖;楊溥登南吳帝位時,再改諡景皇帝(南吳景帝)。楊渥雖被認為是南吳君主之一,惟其在位時尚未稱吳王。

杨溥还封兄子南昌公杨珙为建安王。杨溥的三个哥哥中,二哥杨隆演仅有一子见于《十国春秋》,三哥杨濛尚在人世且一并被封为常山王。故杨珙很可能是杨渥之子。杨溥禅位后,杨珙降为公。

其弟楊溥即皇帝位時,追尊楊渥為烈祖景皇帝,陵墓号绍陵。绍陵地望不详,应在其父杨行密墓附近。


\subsubsection{天祐}

\begin{longtable}{|>{\centering\scriptsize}m{2em}|>{\centering\scriptsize}m{1.3em}|>{\centering}m{8.8em}|}
  % \caption{秦王政}\
  \toprule
  \SimHei \normalsize 年数 & \SimHei \scriptsize 公元 & \SimHei 大事件 \tabularnewline
  % \midrule
  \endfirsthead
  \toprule
  \SimHei \normalsize 年数 & \SimHei \scriptsize 公元 & \SimHei 大事件 \tabularnewline
  \midrule
  \endhead
  \midrule
  元年 & 905 & \tabularnewline\hline
  二年 & 906 & \tabularnewline\hline
  三年 & 907 & \tabularnewline\hline
  四年 & 908 & \tabularnewline
  \bottomrule
\end{longtable}


%%% Local Variables:
%%% mode: latex
%%% TeX-engine: xetex
%%% TeX-master: "../../Main"
%%% End:

%% -*- coding: utf-8 -*-
%% Time-stamp: <Chen Wang: 2021-11-01 15:39:25>

\subsection{宣帝楊隆演\tiny(908-920)}

\subsubsection{生平}

吳宣王楊隆演(897年-920年),字鴻源,原名楊瀛,又名楊渭,中国五代時期南吳君主,孝武王次子,楊渥之弟。

天祐五年(908年),弘農王楊渥為張顥、徐溫所殺。張顥欲自立,而徐温力主立楊隆演。徐溫尋殺張顥,因此專權。雖楊隆演不久為宣諭使李儼承制授為淮南節度使、東南諸道行營都統、同平章事、弘農郡王,然而大權仍掌握在徐溫之手。天祐七年(910年),再為岐王李茂貞承制加中書令,並繼承楊行密吳王之位。

楊隆演個性穩重恭順,对于徐溫父子專權不會顯露出不平之色,因此徐溫也很放心。但因大权旁落,杨隆演建立吳國後並不快樂。徐温长子徐知训骄横恣肆,常侮弄杨隆演。在看戏时一时兴起要杨隆演和他一起演戏,自己演参军,让杨隆演扮作他的僮奴,扎着小辫子,穿着破衣服拿着帽子跟在后面。徐知训又与杨隆演泛舟于河,杨隆演比他先登岸,他就用弹子打杨隆演,被杨隆演随卒挡下才未中。一次在禅智寺赏花喝酒时,徐知训借酒意谩骂杨隆演,其悖慢之状竟将杨隆演吓哭。徐知训还因追赶杨隆演不及,就打死杨隆演的亲吏。

徐知训的种种所为,其父徐温都不知道。副都统朱瑾设计杀死徐知训,提首入宫见杨隆演,杨隆演不但没有振作,反而连称与自己无关,朱瑾最终被徐温部下逼死,徐温养子徐知诰代徐知训执掌杨吴国政。於是杨隆演放縱自己飲酒,而很少吃東西,因此生病臥床。

天祐十六年(919年),徐温奉楊隆演即吳國國王位,改元武義,建宗庙社稷,置百官如天子之制。南吳自是斷絕與唐朝的法統关系。徐温受封为大丞相、都督中外诸军事、诸道都统、镇海宁国节度使、守太尉、兼中书令、东海郡王。其养子徐知诰为左仆射、参知政事、同平章事、领江州观察使、奉化军节度使。

南吳武義二年(920年)五月,楊隆演疾寝,临终时召大丞相徐温入宫,试探其意称“蜀先主谓武侯‘嗣子不才,君宜自取。’”徐温正色称“吾果有意取之,当在诛張顥之初,岂至今日!使杨氏无男,有女亦当立之。敢妄言者斩!”楊隆演去世,諡宣王,徐温因杨隆演三弟杨濛年长且与自己不和,迎其四弟丹阳公楊溥繼位。杨溥后称帝,改諡杨隆演为宣皇帝,廟號高祖。

其弟楊溥即皇帝位時,追尊楊隆演為高祖宣皇帝,陵墓号肃陵。肃陵地望不详,应在其父杨行密墓附近。


\subsubsection{天祐}

\begin{longtable}{|>{\centering\scriptsize}m{2em}|>{\centering\scriptsize}m{1.3em}|>{\centering}m{8.8em}|}
  % \caption{秦王政}\
  \toprule
  \SimHei \normalsize 年数 & \SimHei \scriptsize 公元 & \SimHei 大事件 \tabularnewline
  % \midrule
  \endfirsthead
  \toprule
  \SimHei \normalsize 年数 & \SimHei \scriptsize 公元 & \SimHei 大事件 \tabularnewline
  \midrule
  \endhead
  \midrule
  元年 & 908 & \tabularnewline\hline
  二年 & 909 & \tabularnewline\hline
  三年 & 910 & \tabularnewline\hline
  四年 & 911 & \tabularnewline\hline
  五年 & 912 & \tabularnewline\hline
  六年 & 913 & \tabularnewline\hline
  七年 & 914 & \tabularnewline\hline
  八年 & 915 & \tabularnewline\hline
  九年 & 916 & \tabularnewline\hline
  十年 & 917 & \tabularnewline\hline
  十一年 & 918 & \tabularnewline\hline
  十二年 & 919 & \tabularnewline
  \bottomrule
\end{longtable}

\subsubsection{武义}

\begin{longtable}{|>{\centering\scriptsize}m{2em}|>{\centering\scriptsize}m{1.3em}|>{\centering}m{8.8em}|}
  % \caption{秦王政}\
  \toprule
  \SimHei \normalsize 年数 & \SimHei \scriptsize 公元 & \SimHei 大事件 \tabularnewline
  % \midrule
  \endfirsthead
  \toprule
  \SimHei \normalsize 年数 & \SimHei \scriptsize 公元 & \SimHei 大事件 \tabularnewline
  \midrule
  \endhead
  \midrule
  元年 & 919 & \tabularnewline\hline
  二年 & 920 & \tabularnewline\hline
  三年 & 921 & \tabularnewline
  \bottomrule
\end{longtable}


%%% Local Variables:
%%% mode: latex
%%% TeX-engine: xetex
%%% TeX-master: "../../Main"
%%% End:

%% -*- coding: utf-8 -*-
%% Time-stamp: <Chen Wang: 2021-11-01 15:39:32>

\subsection{睿帝楊溥\tiny(920-937)}

\subsubsection{生平}

吳睿帝楊溥(900年-938年),五代時期南吳君主,楊行密四子,母王氏。楊渥、楊隆演之弟,南吳唯一正式稱帝的君主(先前僅稱王)。

杨隆演称吴国王时,封杨溥为丹阳郡公。吳武義二年(920年)楊隆演去世,因其三弟杨濛年长且不为权臣徐温所喜,楊溥為徐溫所迎繼吳國王位,明年(921年),改元順義。順義七年(927年),即皇帝位,改年號乾貞。乾貞三年(929年)改元大和。大和七年(935年),再改元天祚。

南吳於楊隆演及楊溥在位時,軍政大權皆操之在徐溫、徐知誥父子之中,之所以即國王位、帝位,只是為徐氏父子篡位稱帝之準備而已。

天祚元年,楊溥加中书令徐知誥为尚父、太师、大丞相、天下兵马大元帅,进封齐王,以昇州、润州、宣州、池州、歙州、常州、江州、饶州、信州、海州为齐国。徐知誥置百官,以金陵府为西都。

天祚三年(937年)正月,徐知誥建齐国,立宗庙、社稷,改金陵府为江宁府,子城称宫城,厅堂曰殿,册王妃为王后,世子为王太子,太妃为王太后。置左右丞相、百官如天子之制。当年十月乙酉,楊溥讓位予徐知誥,南吳亡。

楊溥被徐知誥上尊號為高尚思玄弘古讓皇帝,安置于江都宫殿居住,其宗庙、正朔、乘舆、服御、均从吴国旧制,宫殿名称则从道教仙经中取名。楊溥在宫中多穿羽衣,习辟谷之术。

南唐昇元二年(938年),徐知誥(改名李昪)改润州牙城为丹杨宫,迁楊溥于其中,以严兵守护之。当年十一月辛丑,有使者来丹杨宫,楊溥方颂佛经于楼上,使者趋前,楊溥以香炉掷之,俄而去世,终年三十八岁。李昪废朝二十七日,追諡楊溥為睿皇帝。

昇元六年,南唐听宋齐丘之谋尽迁杨吴宗室于泰州,号“永宁宫”,守卫甚严,不使与外人通婚,久而男女自为婚配。后周显德三年,周世宗征淮南,下诏安抚杨氏子孙。南唐元宗李璟遣园苑使尹廷范将杨氏宗族迁置京口。尹廷范杀楊溥二弟及男口六十余人,携妇女渡江。李璟怒曰“小人以不义之名累我”,下令腰斩尹廷范于市。后来宋齐丘也失势被逼自杀,临死感叹这是自己献计幽禁杨溥一族的报应。


\subsubsection{顺义}

\begin{longtable}{|>{\centering\scriptsize}m{2em}|>{\centering\scriptsize}m{1.3em}|>{\centering}m{8.8em}|}
  % \caption{秦王政}\
  \toprule
  \SimHei \normalsize 年数 & \SimHei \scriptsize 公元 & \SimHei 大事件 \tabularnewline
  % \midrule
  \endfirsthead
  \toprule
  \SimHei \normalsize 年数 & \SimHei \scriptsize 公元 & \SimHei 大事件 \tabularnewline
  \midrule
  \endhead
  \midrule
  元年 & 921 & \tabularnewline\hline
  二年 & 922 & \tabularnewline\hline
  三年 & 923 & \tabularnewline\hline
  四年 & 924 & \tabularnewline\hline
  五年 & 925 & \tabularnewline\hline
  六年 & 926 & \tabularnewline\hline
  七年 & 927 & \tabularnewline
  \bottomrule
\end{longtable}

\subsubsection{乾贞}

\begin{longtable}{|>{\centering\scriptsize}m{2em}|>{\centering\scriptsize}m{1.3em}|>{\centering}m{8.8em}|}
  % \caption{秦王政}\
  \toprule
  \SimHei \normalsize 年数 & \SimHei \scriptsize 公元 & \SimHei 大事件 \tabularnewline
  % \midrule
  \endfirsthead
  \toprule
  \SimHei \normalsize 年数 & \SimHei \scriptsize 公元 & \SimHei 大事件 \tabularnewline
  \midrule
  \endhead
  \midrule
  元年 & 927 & \tabularnewline\hline
  二年 & 928 & \tabularnewline\hline
  三年 & 929 & \tabularnewline
  \bottomrule
\end{longtable}

\subsubsection{大和}

\begin{longtable}{|>{\centering\scriptsize}m{2em}|>{\centering\scriptsize}m{1.3em}|>{\centering}m{8.8em}|}
  % \caption{秦王政}\
  \toprule
  \SimHei \normalsize 年数 & \SimHei \scriptsize 公元 & \SimHei 大事件 \tabularnewline
  % \midrule
  \endfirsthead
  \toprule
  \SimHei \normalsize 年数 & \SimHei \scriptsize 公元 & \SimHei 大事件 \tabularnewline
  \midrule
  \endhead
  \midrule
  元年 & 929 & \tabularnewline\hline
  二年 & 930 & \tabularnewline\hline
  三年 & 931 & \tabularnewline\hline
  四年 & 932 & \tabularnewline\hline
  五年 & 933 & \tabularnewline\hline
  六年 & 934 & \tabularnewline\hline
  七年 & 935 & \tabularnewline
  \bottomrule
\end{longtable}

\subsubsection{天祚}

\begin{longtable}{|>{\centering\scriptsize}m{2em}|>{\centering\scriptsize}m{1.3em}|>{\centering}m{8.8em}|}
  % \caption{秦王政}\
  \toprule
  \SimHei \normalsize 年数 & \SimHei \scriptsize 公元 & \SimHei 大事件 \tabularnewline
  % \midrule
  \endfirsthead
  \toprule
  \SimHei \normalsize 年数 & \SimHei \scriptsize 公元 & \SimHei 大事件 \tabularnewline
  \midrule
  \endhead
  \midrule
  元年 & 935 & \tabularnewline\hline
  二年 & 936 & \tabularnewline\hline
  三年 & 937 & \tabularnewline
  \bottomrule
\end{longtable}


%%% Local Variables:
%%% mode: latex
%%% TeX-engine: xetex
%%% TeX-master: "../../Main"
%%% End:



%%% Local Variables:
%%% mode: latex
%%% TeX-engine: xetex
%%% TeX-master: "../../Main"
%%% End:

%% -*- coding: utf-8 -*-
%% Time-stamp: <Chen Wang: 2019-12-24 17:53:23>


\section{南唐\tiny(937-975)}

\subsection{简介}

南唐(937年-975年)是五代十國的十國之一,定都金陵,歷時39年,有烈祖李昪、元宗李璟和後主李煜三位帝王。

南唐的成立可以追溯到吳國權臣徐溫的身上。徐溫原本是吳國(南吳)的開國功臣,後來他漸漸掌握了南吳的實權。他年老的時候,因亲子徐知训骄狂被杀、徐知询等年少能力不足,信任養子徐知誥,也漸給與他繼承人的地位。

徐溫去世後,徐知诰设计控制了徐知询,掌握了吴国的军政。937年,徐知诰代吴稱帝建國。根據《新五代史》、《舊五代史》、《南唐書》、《十國春秋》等史籍記載,徐知誥篡奪政權所建國號为齊,都金陵,号江宁府(今江苏南京),史家稱之為徐齊。939年,徐知誥回復自称的本姓李姓,並改名李昪,為了附會已滅亡的唐朝,把國號改為唐,以其位于南方,史称南唐。不過,《資治通鑑》卻記載937年徐知誥即帝位時,即以唐為國號,並不認為徐齊曾經存在過。后又称江南国。江淮地区的吴与后继的南唐国势强盛,他们采取联合北方契丹国制约中原的策略,屡次征讨周边国家壮大势力,成为中原王朝的一大威胁。

李昪称帝时期是南唐的盛世,經濟繁榮,文化昌盛。中主李璟時由於與周邊各國多次興兵,945年滅閩國、入侵楚國等使國力衰退,不但因此失去进取中原的良机,还因为策略不当等原因只占领了闽国的少部分地区,灭楚所得的土地更是都被武平军节度使刘言收复。

955年,周世宗發動后周攻南唐之战,南唐大敗,958年被迫將長江以北十四州割讓給後周,並且稱臣,奉后周年号,去帝號改稱唐国主,而周世宗称其为「江南國主」。宋朝建立后,南唐维持现状,用宋朝年号。961年,为应对宋朝巨大的军事压力迁都南昌府。后主李煜继位后,仍都江宁。

971年,宋朝改唐国主为江南国主,李煜去鸱吻,诸王降封为公。974年,後主李煜为保政权拒絕入朝,被宋讨伐,于是弃用宋年号,改以干支纪年。公元976年1月1日,宋灭南唐。李璟與李煜兩父子在中國文學上是有名的詞人。

南唐极盛时统治地区包括今江苏、安徽两省淮河以南、苏北东部、福建、江西、湖南、湖北东部。郑方坤称:“十国文物,首推南唐、西蜀。”马令《南唐书》卷十三《儒者传论》说:“五代之乱也,礼乐崩坏,文献俱亡,而儒衣书服,盛于南唐。岂斯文之未丧,而天将有所寓欤?不然,则圣王之大典,扫地尽矣。南唐累世好儒,而儒者之盛,见于载籍,灿然可观。如韩熙载之不羁,江文蔚之高才,徐锴之典赡,高越之华藻,潘佑之清逸,皆能擅价于一时,而徐铉、汤悦、张洎之徒,又足以争名于天下。其馀落落,不可胜数。故曰江左三十年间,文物有元和之风,岂虚言乎?”

%% -*- coding: utf-8 -*-
%% Time-stamp: <Chen Wang: 2019-12-24 17:54:33>

\subsection{烈祖\tiny(937-943)}

\subsubsection{生平}

唐烈祖李昪(889年1月7日-943年3月30日),字正倫,小字彭奴,五代十國時期南唐開國皇帝。海州人(今属江苏连云港),一说徐州人(今属江苏),另一说湖州安吉人(今属浙江),原稱「徐知誥」,是南吳大臣徐溫養子。

关于李昪的身世,历史上众说纷纭,莫衷一是,宋朝司马光《资治通鉴考异》就收录了四种不同的说法。

其中一种观点认为李昪是唐朝皇族后裔,持这种观点者以私修史著及杂史、稗史居多。据《十国春秋》总结,南唐灭亡后,南唐旧臣徐铉作《江南录》记录南唐历史,其中就提出李昪是唐宪宗第八子建王李恪的玄孙,释文莹《玉壶清话》采用了这种说法,李昪之孙李从浦墓志铭《宋故左龙武卫大将军李公墓志铭》也自称是建王李恪的后裔。陆游《南唐书》进一步提出了具体的世系是李恪生李超,李超生李荣,李荣生李昪,龙衮《江南野史》和马令《南唐书》世系谱与陆游书类似,但认为李超仅仅是李恪的后裔而非儿子,赵世延《南唐书序》,陈霆《唐余纪传》沿袭了这种观点;李昊《蜀后主实录》记载李昪是曾任岭南节度使的薛王李知柔之子,郑文宝《江表志》认为李昪是唐朝郑王的疏属支脉,陈彭年《江南别录》仅称之为唐之宗室,没有指明是谁的后代,《旧五代史》则记载李昪自称是唐玄宗之子永王李璘的后代。李昪自称唐朝皇室后裔,在五代十国时期就受到诟病,钱元瓘与沈韬文曾出言讽刺。

而宋代以来另有一种观点认为李昪祖先不过是平民,正史所持都是这种观点。《旧五代史》记载李昪仅仅是“自称”唐朝皇室后裔,《新五代史》同样是记载李昪“自称”建王李恪的玄孙,且称其出身微贱,而《资治通鉴》记载李昪打算以吴王李恪为祖先时曾有部下建议以郑王李元懿为祖先,李昪下令有关部门考察李恪和李元懿的后代,因为李恪的孙子李祎曾有军功,李祎的儿子李岘又做过宰相,于是李昪才以李恪为祖先,自称李岘下传五世到李昪的父亲李荣,这五世的名字大部分都是杜撰出来的,李昪又觉得唐朝经历了十九个皇帝历时三百年,怀疑自己的世系十代人太少,有关部门奏称一代人三十年,而李昪出生于唐僖宗文德年间,已经五十年了,李昪于是依从了他们。

另一种观点则见于钱俨《吴越备史》,其中称李昪之父本姓潘,因为被敌将李神福掳走而成为李神福的家奴,后徐温在李神福家见到李昪,对其十分惊异,遂请求收为养子。刘恕《十国纪年》认为李昪附会祖宗,不是唐朝宗室后裔,不过吴越与南唐是仇敌,《吴越备史》也非史实。李昪少年时就遭遇战乱成了孤儿,其祖先世系根本无法得知,李超、李志的名字都与徐温曾祖和祖父同名,完全是附会。

李昪,唐朝光启四年十二月二日(889年1月7日)生人[來源請求],小名李彭奴。父李荣性格谨厚,多游于佛寺。李彭奴六岁时其父於動亂中喪生,与伯父李球逃亡濠州。不久之后生母刘氏卒,遂托养于濠州开元寺。

乾宁二年,楊吳太祖杨行密攻濠州,得李彭奴,奇其相貌,欲收养为己子,而杨行密诸亲子以其身世微贱,不齿为兄弟。杨行密遂将李彭奴交给徐温为养子,遂改名为徐知誥。徐温妻子李氏以李彭奴与己为同姓,甚为爱护。

楊吳時期,徐知誥因功累升昇州防遏使、楼船使、昇州刺史、潤州團練使、检校司徒。徐知誥為政寬仁,又能節儉自處,獎勵農桑,因此府庫充實。當時,徐溫居昇州,並以長子徐知訓居南吳都城揚州控制南吳政權。天祐十五年(918年)徐知訓因驕傲荒淫為朱瑾所殺,徐知誥就近自潤州渡長江平變,自是徐溫乃以其为淮南节度行军副使、内外马步都军副使,代替徐知訓留揚州,日常政事皆由徐知誥處斷。

徐知誥在揚州,一反徐知訓的作為,恭敬事奉吳王楊隆演,並且謙卑對待士大夫。對待部屬寬大,生活儉僕,並以宋齊丘為謀士,改革稅制,因此國勢漸強,人心歸附。武义元年,徐知誥拜为左仆射、参知政事。顺义初年加封同平章事、领江州观察使、奉化军节度使。

南吳順義七年(927年)徐溫去世,徐知誥與徐溫親子徐知詢爭權,徐知誥趁徐知詢入朝的機會,將其扣留,自此完全掌握南吳政權。太和三年,徐知誥升为太尉、中书令、领镇海宁国诸军节度使,封东海郡王,出镇金陵。天祚元年(935年),加封尚父、太师、大丞相、天下兵马大元帅,进封齐王,以升州、润州、宣州、池州、歙州、常州、江州、饶州、信州、海州为齐国。徐知诰置百官,以金陵府为西都。

天祚三年(937年),徐知誥改名徐誥。同年,杨溥让位,南吴亡。徐誥即皇帝位,国号大齐,改年號昇元,以昇州金陵府(建康)为西都,扬州广陵府(江都)为东都。追尊徐温为太祖武皇帝。

昇元三年(939年)正月庚戌,江王徐知证、饶王徐知谔表奏,请徐誥恢復原姓,徐誥不许。正月癸亥,左丞相宋齐丘等人再次上表,乃允之。二月乙亥,徐誥自認是唐朝宗室,改国号为大唐,改徐温庙号为义祖。复李姓,初改自名为昂,犯唐文宗名讳;旋改名晃,又其与后梁太祖朱温同名,又改名为旦,犯唐睿宗庙讳。最终改名为昪。立天子七庙,以唐高祖、唐太宗、义祖徐温为不迁之祖。李昪由於家族譜系不詳,附會唐朝宗室,欲以唐朝吴王李恪为远祖,大臣奏以李恪被長孫無忌絞死,不如以郑王李元懿为祖。李昪命诸臣考二王苗裔,李恪之孙李祎有功,李祎之子李岘为宰相,遂以李恪为祖。创家谱,曰生父李荣,李荣之父李志,李志之父李超,李超之祖为李岘。其名字与官衔皆杜撰。当年三月,李昪下诏尊十世祖李恪为定宗孝静皇帝,曾祖李超为成宗孝平皇帝,祖李志为惠宗孝安皇帝,父李荣为庆宗孝德皇帝。但李昪孙李从镒墓志又认祖唐宪宗子建王李恪,未详孰是。

李昪登帝位後,改旧邸为崇德宫,正厅为光庆殿。又改东都文明殿为乾元殿、英武殿为明光殿、应乾殿为垂拱殿、朝阳殿为福昌殿、积庆宫为崇道宫;改西都崇英殿为延英殿、凝华前殿为昇元殿、后殿为雍和殿、兴祥殿为昭德殿、积庆殿为穆清殿。李昪勤於政事,並興利除弊,變更舊法。保境安民,與民休息。又與吳越和解,昇元五年吴越国大火,群臣请趁机攻打,而李昪称“奈何利人之灾!”遣使厚赠金帛慰问。吴越水灾,其民就食于南唐境内,李昪也遣官员赈济。

然而李昪崇尚道术,因服用丹藥中毒,個性變得暴躁易怒。昇元七年(943年),李昪服食方士史守冲所献“金丹”,背上生瘡,不久病情惡化,在昇元殿去世,终年五十六岁。临终前召子齐王李璟,嘱曰“德昌宫储戎器金帛七百余万,汝守成业,宜善交邻国,以保社稷。吾服金石,欲求延年,反以速死,汝宜视以为戒”;又啮齐王手指出血,称“他日北方必有事,勿忘吾言”。李璟繼位,上李昪谥号光文肅武孝神烈高皇帝,廟號烈祖,葬于永陵(后改陵号为钦陵)。

\subsubsection{昇元}

\begin{longtable}{|>{\centering\scriptsize}m{2em}|>{\centering\scriptsize}m{1.3em}|>{\centering}m{8.8em}|}
  % \caption{秦王政}\
  \toprule
  \SimHei \normalsize 年数 & \SimHei \scriptsize 公元 & \SimHei 大事件 \tabularnewline
  % \midrule
  \endfirsthead
  \toprule
  \SimHei \normalsize 年数 & \SimHei \scriptsize 公元 & \SimHei 大事件 \tabularnewline
  \midrule
  \endhead
  \midrule
  元年 & 937 & \tabularnewline\hline
  二年 & 938 & \tabularnewline\hline
  三年 & 939 & \tabularnewline\hline
  四年 & 940 & \tabularnewline\hline
  五年 & 941 & \tabularnewline\hline
  六年 & 942 & \tabularnewline\hline
  七年 & 943 & \tabularnewline
  \bottomrule
\end{longtable}


%%% Local Variables:
%%% mode: latex
%%% TeX-engine: xetex
%%% TeX-master: "../../Main"
%%% End:

%% -*- coding: utf-8 -*-
%% Time-stamp: <Chen Wang: 2021-11-01 15:39:47>

\subsection{元宗李璟\tiny(943-961)}

\subsubsection{生平}

唐元宗李璟(916年-961年),字伯玉,原稱徐景通,南唐建立後,復本姓李,改名璟。對後周稱臣後,又為避後周信祖諱,而改名景。南唐烈祖李昪的長子。五代十国时期南唐第二位君主,因此也被称为中主、嗣主。李璟的书法頗佳,词亦有名,與其子李煜並稱「南唐二主」。其詞“小楼吹彻玉笙寒”是流芳千古的名句。作品被收入《南唐二主词》中。

昇元七年(943年)李昪過世,李璟繼位,改元保大。

李璟即位后,改变父皇李昪保守的政策,开始大规模对外用兵,消滅因繼承人爭位而內亂的马楚及闽国,他在位的大部分时間,南唐疆土最大。

李璟一心想著建功立業,但沒有治世之才。過度好大喜功的他不守父皇遺命,罷黜先皇時期的元老重臣,反而起用五個專事諂媚和自己興趣相投的佞臣──馮延巳、馮延魯、魏岑、陳覺、查文徽,史稱「南唐五鬼」。這幾人阿諛奉承、結黨營私,致使南唐政治陷入一片黑暗。

李璟改變先皇保境安民的國策,不斷地侵犯周邊國家,陸續攻滅閩、楚。雖然增加七州的新土地,但隨之而來反叛鬥爭,更令南唐疲於應付。與後周的兩次戰爭,消耗南唐大量庫存軍費,南唐戰敗,也使國力顯露頹敗之勢。

李璟奢侈无度,导致政治腐败,百姓民不聊生,怨声载道。

957年后周派兵侵入南唐,占领了南唐淮南江北的大片土地,并长驱直入到长江一带,迫近金陵,李璟只好向后周世宗柴榮称臣,去帝號,自稱唐國主,年號由原本的交泰改為後周的顯德。

961年8月12日卒,时年46岁,庙号烈宗,谥号为明道崇德文宣孝皇帝。

\subsubsection{保大}

\begin{longtable}{|>{\centering\scriptsize}m{2em}|>{\centering\scriptsize}m{1.3em}|>{\centering}m{8.8em}|}
  % \caption{秦王政}\
  \toprule
  \SimHei \normalsize 年数 & \SimHei \scriptsize 公元 & \SimHei 大事件 \tabularnewline
  % \midrule
  \endfirsthead
  \toprule
  \SimHei \normalsize 年数 & \SimHei \scriptsize 公元 & \SimHei 大事件 \tabularnewline
  \midrule
  \endhead
  \midrule
  元年 & 943 & \tabularnewline\hline
  二年 & 944 & \tabularnewline\hline
  三年 & 945 & \tabularnewline\hline
  四年 & 946 & \tabularnewline\hline
  五年 & 947 & \tabularnewline\hline
  六年 & 948 & \tabularnewline\hline
  七年 & 949 & \tabularnewline\hline
  八年 & 950 & \tabularnewline\hline
  九年 & 951 & \tabularnewline\hline
  十年 & 952 & \tabularnewline\hline
  十一年 & 953 & \tabularnewline\hline
  十二年 & 954 & \tabularnewline\hline
  十三年 & 955 & \tabularnewline\hline
  十四年 & 956 & \tabularnewline\hline
  十五年 & 957 & \tabularnewline
  \bottomrule
\end{longtable}

\subsubsection{中兴}

\begin{longtable}{|>{\centering\scriptsize}m{2em}|>{\centering\scriptsize}m{1.3em}|>{\centering}m{8.8em}|}
  % \caption{秦王政}\
  \toprule
  \SimHei \normalsize 年数 & \SimHei \scriptsize 公元 & \SimHei 大事件 \tabularnewline
  % \midrule
  \endfirsthead
  \toprule
  \SimHei \normalsize 年数 & \SimHei \scriptsize 公元 & \SimHei 大事件 \tabularnewline
  \midrule
  \endhead
  \midrule
  元年 & 958 & \tabularnewline
  \bottomrule
\end{longtable}

\subsubsection{交泰}

\begin{longtable}{|>{\centering\scriptsize}m{2em}|>{\centering\scriptsize}m{1.3em}|>{\centering}m{8.8em}|}
  % \caption{秦王政}\
  \toprule
  \SimHei \normalsize 年数 & \SimHei \scriptsize 公元 & \SimHei 大事件 \tabularnewline
  % \midrule
  \endfirsthead
  \toprule
  \SimHei \normalsize 年数 & \SimHei \scriptsize 公元 & \SimHei 大事件 \tabularnewline
  \midrule
  \endhead
  \midrule
  元年 & 958 & \tabularnewline
  \bottomrule
\end{longtable}

\subsubsection{显德}

\begin{longtable}{|>{\centering\scriptsize}m{2em}|>{\centering\scriptsize}m{1.3em}|>{\centering}m{8.8em}|}
  % \caption{秦王政}\
  \toprule
  \SimHei \normalsize 年数 & \SimHei \scriptsize 公元 & \SimHei 大事件 \tabularnewline
  % \midrule
  \endfirsthead
  \toprule
  \SimHei \normalsize 年数 & \SimHei \scriptsize 公元 & \SimHei 大事件 \tabularnewline
  \midrule
  \endhead
  \midrule
  元年 & 958 & \tabularnewline\hline
  二年 & 959 & \tabularnewline\hline
  三年 & 960 & \tabularnewline\hline
  四年 & 961 & \tabularnewline
  \bottomrule
\end{longtable}


%%% Local Variables:
%%% mode: latex
%%% TeX-engine: xetex
%%% TeX-master: "../../Main"
%%% End:

%% -*- coding: utf-8 -*-
%% Time-stamp: <Chen Wang: 2021-11-01 15:39:52>

\subsection{后主李煜\tiny(961-975)}

\subsubsection{生平}

李煜(937年8月15日-978年8月13日),或稱李後主,為南唐的末代君主,徐州人。李煜原名從嘉,字重光,號鍾山隱士、鍾峰隠者、白蓮居士、蓮峰居士等。史書描述其政治上毫无建树,李煜在南唐灭亡后被北宋俘虏,但是却成为了中国历史上首屈一指的词人,獲誉为「詞聖」、「千古詞帝」,作品千古流传。

李煜“为人仁孝,善属文,工书画,而廣顙丰额骈齿,一目重瞳子”,是南唐元宗(南唐中主)李璟的第六子。由于李璟的第二子到第五子均早死,故李煜长兄李弘冀为皇太子时,其为事实上的第二子。李弘冀“为人猜忌严刻”,时为安定公的李煜因而惶恐,不敢参与政事,每天只醉心研究典籍,以读书为乐。

959年李弘冀在毒死李景遂后不久亦死。李璟欲立李煜为太子,钟谟说“从嘉德轻志懦,又酷信释氏,非人主才。从善果敢凝重,宜为嗣。”李璟怒,将钟谟贬为国子监司,流放到饶州。封李煜为吴王、尚书令、知政事,令其住在东宫,就近學習處理政事。

宋建隆二年(961年),李璟迁都南昌并立李煜为太子、監國,令其留在金陵。六月李璟死后,李煜在金陵登基即位。李煜“性骄侈,好声色,又喜浮图,为高谈,不恤政事。”笃信佛教,“酷好浮屠,崇塔庙,度僧尼不可胜算。罢朝,辄造佛屋,易服膜拜,颇废政事。”在宫内和国内大兴宗教,甚至在军国大事上都以佛事为凭,自己每日穿袈裟诵佛经。直到宋軍临城下,李煜还在净居寺听和尚念经。[來源請求]

971年宋军灭南汉后,李煜为了表示他不对抗宋,对宋称臣,将自己的称呼改为江南国主,去鸱吻,诸王降封为公。

973年,宋太祖令李煜至汴京,李煜托病不往。974年,宋太祖遂派曹彬领军攻南唐。李煜因此弃用北宋年号,改用干支纪年。

12月,曹彬攻克金陵,南唐灭亡。李煜在位十五年,後世称李后主或南唐后主。

975年,李煜被俘后,在汴京被封为违命侯,拜左千牛卫将军。

976年,宋太祖逝世,弟赵光义继位为宋太宗,改封隴國公。嘗與金陵舊宮人書寫:「此中日夕,以淚珠洗面」。宋人笔记上說趙光義多次逼迫小周后侍寢。李煜在痛苦鬱悶中,寫下《望江南》、《子夜歌》、《虞美人》等名曲。

978年,徐铉奉宋太宗之命探视李煜,李煜對徐鉉叹息:“当初我错杀潘佑、李平,悔之不已!”徐铉退而告之,宋太宗闻之大怒。史載三年七月初七(978年8月13日),农历七夕,当李煜在其42岁生日那天与后妃们聚会,李煜卒,年四十二。一說李煜因寫“故国不堪回首月明中”、“恰似一江春水向東流”之词,宋太宗再也不能容忍,用牵机毒杀之。牽機藥或說是中藥馬錢子,其主要成分番木鳖碱有劇毒,服後會破壞中樞神經系統,全身抽搐,腳往腹部縮,頭亦彎至腹部,狀極痛苦。李煜死后,葬洛陽北邙山,小周后悲痛欲絕,不久也隨之死去。

李煜“生于深宫之中,长于妇人之手”,雖無力治國,然“性宽恕,威令不素著”,好生戒杀,性格出了名的善良,故在他死后,江南人闻之,“皆巷哭为斋”。

李煜在艺术方面具有很高的成就。劉毓盤说李后主“于富贵时能作富贵语,愁苦时能作愁苦语,无一字不真。”

李煜词本有集,已失传。现存词四十四首。其中几首前期作品或为他人所作,可以确定者仅三十八首。李煜的词的风格可以以975年被俘而分为两个时期:


李煜亡國前的詞,透插富麗奢華的宮廷生活,言詞多溫軟綺麗,卿卿我我,呈現「花間詞」氣息。根据内容可大致分为两类:一类是描写富丽堂皇的宫廷生活和风花雪月的男女情事,如《菩萨蛮》:“花明月暗籠輕霧,今宵好向郎邊去。剗袜步香階,手提金缕鞋。画堂南畔见,一晌偎人颤。奴為出来難,教君恣意憐。”又如《一斛珠》:“曉妝初過,沈檀輕注些兒個,向人微露丁香顆,一曲清歌,暫引櫻桃破。羅袖裛殘殷色可,杯深旋被香醪涴。繡床斜憑嬌無那,爛嚼紅茸,笑向檀郎唾。”

李煜亡國後,晚年的詞寫家國之恨,拓展了詞的題材,感慨既深,詞益悲壯。李煜詞最大特色,是自然真率,醇厚率真,情感真摯。喜用白描手法,通俗生動,語言精鍊而明淨洗煉,接近口語,與「花間詞」縷金刻翠,堆砌華麗詞藻的作風迥然不同。李煜后期的词由于生活的巨变,以一首首泣尽以血的绝唱,使亡国之君成为千古词坛的“南面王”(清沈雄《古今词话》语),正是“国家不幸诗家幸,话到沧桑语始工”。这些后期词作,凄凉悲壮,意境深远,为词史上承前启后的大宗师。至于其语句的清丽,音韵的和谐,更是空前绝后。如《破阵子》:“四十年来家国,三千里地山河。凤阁龙楼连霄汉,玉树琼枝作烟萝。几曾识干戈?一旦归为臣虏,沈腰潘鬓消磨。最是仓皇辞庙日,教坊犹奏别离歌。揮泪对宫娥。”《虞美人》:“春花秋月何时了,往事知多少。小楼昨夜又东风,故国不堪回首月明中。雕栏玉砌应犹在,只是朱颜改。问君能有几多愁,恰似一江春水向東流。”《浪淘沙令》:“帘外雨潺潺,春意阑珊。罗衾不耐五更寒,梦里不知身是客,一晌贪欢。独自莫凭栏,无限江山,别时容易见时难。流水落花春去也,天上人间。”

他能书善画,对其书法:陶穀《清異錄》曾云:“后主善书,作颤笔樛曲之状,遒劲如寒松霜竹,谓之‘金错刀’。作大字不事笔,卷帛书之,皆能如意,世谓‘撮襟书’。”。对其的画,宋代郭若虛的《图书见闻志》曰:“江南后主李煜,才识清赡,书画兼精。尝观所画林石、飞鸟,远过常流,高出意外。”。

歐陽修在《新五代史》中描述李煜:“煜字重光,初名從嘉,景第六子也。煜為人仁孝,善屬文,工書畫,而豐額駢齒,一目重瞳子。”

《漁隱叢話前集·西清詩話》提到宋太祖征服南唐统一中国后感叹:“李煜若以作诗词工夫治国家,岂为吾所俘也!”

近代学者王国维认为:“温飞卿之词,句秀也;韦端己之词,骨秀也;李重光之词,神秀也。”“词至李后主而眼界始大,感慨遂深,遂变伶工之词而为士大夫之词。周介存置诸温、韦之下,可谓颠倒黑白矣。”。此最后一句乃是针对周济在《介存斋论词杂著》中所道:“毛嫱、西施,天下美妇人也,严妆佳,淡妆亦佳,粗服乱头不掩国色。飞卿,严妆也;端己,淡妆也;后主,则粗服乱头矣。”王氏认为此评乃扬温、韦,抑后主。而学术界亦有观点认为,周济的本意是指李煜在词句的工整对仗等修饰方面不如温庭筠、韦庄,然而在词作的生动和流畅度方面,则前者显然更为生机勃发,浑然天成,“粗服乱头不掩国色”。

李煜词摆脱了《花间集》的浮靡,他的词不假雕饰,语言明快,形象生动,性格鲜明,用情真挚,亡国后作更是题材广阔,含意深沉,超过晚唐五代的词,不但成为宋初婉约派词的开山,也为豪放派打下基础,後世尊稱他為「詞聖」。

后代念及李煜的诗词中以清朝袁枚引《南唐雜詠》最有名:“作個才人真絕代,可憐薄命作君王。”

《宋史·潘慎修傳》記載:南唐滅亡後,一些南唐舊臣開始批評李煜為人愚昧懦弱,添油加醋地成份越來越多。宋真宗問潘慎修李煜是不是真的如此,潘慎修回答:「如果李煜真的這麼愚昧懦弱的話,他怎麼能治國十餘年?」

另一位南唐舊臣徐鉉在《大宋左千牛衛上將軍追封吴王隴西公墓誌銘》中評價李煜:「以厭兵之俗當用武之世,孔明罕應變之略,不成近功;偃王,躬仁義之行,終于亡國,道有所在,復何媿歟?」

徐铉:“王以世嫡嗣服,以古道驭民,钦若彝伦,率循先志。奉蒸尝、恭色养,必以孝;事耇老、宾大臣,必以礼。居处服御必以节,言动施舍必以时。至于荷全济之恩,谨藩国之度,勤修九贡,府无虚月,祗奉百役,知无不为。十五年间,天眷弥渥。”“精究六经,旁综百氏。常以周孔之道不可暂离,经国化民,发号施令,造次于是,始终不渝。”“酷好文辞,多所述作。一游一豫,必以颂宣。载笑载言,不忘经义。洞晓音律,精别雅郑;穷先王制作之意,审风俗淳薄之原,为文论之,以续《乐记》。所著文集三十卷,杂说百篇,味其文、知其道矣。至于弧矢之善,笔札之工,天纵多能,必造精绝。”“本以恻隐之性,仍好竺干之教。草木不杀,禽鱼咸遂。赏人之善,常若不及;掩人之过,惟恐其闻。以至法不胜奸,威不克爱。以厌兵之俗当用武之世,孔明罕应变之略,不成近功;偃王躬仁义之行,终于亡国。道有所在,复何愧欤!”

郑文宝:“后主奉竺乾之教,多不茹晕,常买禽鱼为放生。”“后主天性纯孝,孜孜儒学,虚怀接下,宾对大臣,倾奉中国,惟恐不及。但以著述勤于政事,至于书画皆尽精妙。然颇耽竺乾之教,果于自信,所以奸邪得计。排斥忠谠,土地曰削,贡举不充。越人肆谋,遂为敌国。又求援于北虏行人设谋,兵遂不解矣。”(《江表志》)

陆游:“后主天资纯孝......专以爱民为急,蠲赋息役,以裕民力。尊事中原,不惮卑屈,境内赖以少安者十有五年。”“然酷好浮屠,崇塔庙,度僧尼不可胜算。罢朝辄造佛屋,易服膜拜,以故颇废政事。兵兴之际,降御札移易将帅,大臣无知者。虽仁爱足以感其遗民,而卒不能保社稷。”(《南唐书·卷三·后主本纪第三》)

龙衮:“后主自少俊迈,喜肄儒学,工诗,能属文,晓悟音律。姿仪风雅,举止儒措,宛若士人。”(《江南野史·卷三后主、宜春王》)

陈彭年:“(后主煜)幼而好古,为文有汉魏风。”(《江南别录》)

欧阳修:“煜性骄侈,好声色,又喜浮图,为高谈,不恤政事。”

王世贞:“花间犹伤促碎,至南唐李王父子而妙矣。”(《弇州山人词评》)

胡应麟:“后主目重瞳子,乐府为宋人一代开山。盖温韦虽藻丽,而气颇伤促,意不胜辞。至此君方为当行作家,清便宛转,词家王、孟。”(《诗薮·杂篇》)

纳兰性德:“花间之词,如古玉器,贵重而不适用;宋词适用而少质重,李后主兼有其美,更饶烟水迷离之致。”(《渌水亭杂识·卷四》)

王夫之:“(李璟父子)无殃兆民,绝彝伦淫虐之巨惹。”“生聚完,文教兴,犹然彼都人士之余风也。”(《读通鉴论》)

余怀:“李重光风流才子,误作人主,至有入宋牵机之恨。其所作之词,一字一珠,非他家所能及也。”(《玉琴斋词·序》)

沈谦:“男中李后主,女中李易安,极是当行本色。”(徐釚《词苑丛谈》引语)“后主疏于治国,在词中犹不失南面王。”(沈雄《古今词话·词话》卷上引语)

郭麐:“作个才子真绝代,可怜薄命作君王。”(清代袁枚《随园诗话补遗》引郭麐《南唐杂咏》)

周济:“李后主词如生马驹,不受控捉。”“毛嫱西施,天下美妇人也。严妆佳,淡妆亦佳,粗服乱头,不掩国色。飞卿,严妆也;端己,淡妆也;后主则粗服乱头矣。”(《介存斋论词杂著》)

周之琦:“予谓重光天籁也,恐非人力所及。”

陈廷焯:“后主词思路凄惋,词场本色,不及飞卿之厚,自胜牛松卿辈。”“余尝谓后主之视飞卿,合而离者也;端己之视飞卿,离而合者也。”“李后主、晏叔原,皆非词中正声,而其词无人不爱,以其情胜也。”(《白雨斋词话·卷一》)

王鹏运:“莲峰居士(李煜)词,超逸绝伦,虚灵在骨。芝兰空谷,未足比其芳华;笙鹤瑶天,讵能方兹清怨?后起之秀,格调气韵之间,或月日至,得十一于千首。若小晏、若徽庙,其殆庶几。断代南流,嗣音阒然,盖间气所钟,以谓词中之大成者,当之无愧色矣。”(《半塘老人遣稿》)

冯煦:“词至南唐,二主作于上,正中和于下,诣微造极,得未曾有。宋初诸家,靡不祖述二主。”(《宋六十一家词选·例言》)

王国维:“温飞卿之词,句秀也;韦端己之词,骨秀也;李重光之词,神秀也。”“词至李后主而眼界始大,感慨遂深,遂变伶工之词而为士大夫之词。”“词人者,不失其赤子之心者也。故生于深宫之中,长于妇人之手,是后主为人君所短处,亦即为词人所长处。”“主观之诗人,不必多阅世,阅世愈浅,则性情愈真,李后主是也。”“尼采谓一切文字,余爱以血书者,后主之词,真所谓以血书者也。宋道君皇帝《燕山亭》词,亦略似之。然道君不过自道身世之感,后主则俨有释迦、基督担荷人类罪恶之意,其大小固不同矣。”“唐五代之词,有句而无篇;南宋名家之词,有篇而无句。有篇有句,唯李后主之作及永叔、少游、美成、稼轩数人而已。”(《人间词话》)

毛泽东:“南唐李后主虽多才多艺,但不抓政治,终于亡国。”(毛泽东评价历史人物)

柏杨:“南唐皇帝李煜先生词学的造诣,空前绝后,用在填词上的精力,远超过用在治国上。”(《浊世人间》)

叶嘉莹:“李后主的词是他对生活的敏锐而真切的体验,无论是享乐的欢愉,还是悲哀的痛苦,他都全身心的投入其间。我们有的人活过一生,既没有好好的体会过快乐,也没有好好的体验过悲哀,因为他从来没有以全部的心灵感情投注入某一件事,这是人生的遗憾。”(《唐宋名家词赏析》)

\subsubsection{显德}

\begin{longtable}{|>{\centering\scriptsize}m{2em}|>{\centering\scriptsize}m{1.3em}|>{\centering}m{8.8em}|}
  % \caption{秦王政}\
  \toprule
  \SimHei \normalsize 年数 & \SimHei \scriptsize 公元 & \SimHei 大事件 \tabularnewline
  % \midrule
  \endfirsthead
  \toprule
  \SimHei \normalsize 年数 & \SimHei \scriptsize 公元 & \SimHei 大事件 \tabularnewline
  \midrule
  \endhead
  \midrule
  元年 & 961 & \tabularnewline\hline
  二年 & 962 & \tabularnewline
  \bottomrule
\end{longtable}

\subsubsection{建隆}

\begin{longtable}{|>{\centering\scriptsize}m{2em}|>{\centering\scriptsize}m{1.3em}|>{\centering}m{8.8em}|}
  % \caption{秦王政}\
  \toprule
  \SimHei \normalsize 年数 & \SimHei \scriptsize 公元 & \SimHei 大事件 \tabularnewline
  % \midrule
  \endfirsthead
  \toprule
  \SimHei \normalsize 年数 & \SimHei \scriptsize 公元 & \SimHei 大事件 \tabularnewline
  \midrule
  \endhead
  \midrule
  元年 & 963 & \tabularnewline
  \bottomrule
\end{longtable}

\subsubsection{乾德}

\begin{longtable}{|>{\centering\scriptsize}m{2em}|>{\centering\scriptsize}m{1.3em}|>{\centering}m{8.8em}|}
  % \caption{秦王政}\
  \toprule
  \SimHei \normalsize 年数 & \SimHei \scriptsize 公元 & \SimHei 大事件 \tabularnewline
  % \midrule
  \endfirsthead
  \toprule
  \SimHei \normalsize 年数 & \SimHei \scriptsize 公元 & \SimHei 大事件 \tabularnewline
  \midrule
  \endhead
  \midrule
  元年 & 963 & \tabularnewline\hline
  二年 & 964 & \tabularnewline\hline
  三年 & 965 & \tabularnewline\hline
  四年 & 966 & \tabularnewline\hline
  五年 & 967 & \tabularnewline\hline
  六年 & 968 & \tabularnewline
  \bottomrule
\end{longtable}

\subsubsection{开宝}

\begin{longtable}{|>{\centering\scriptsize}m{2em}|>{\centering\scriptsize}m{1.3em}|>{\centering}m{8.8em}|}
  % \caption{秦王政}\
  \toprule
  \SimHei \normalsize 年数 & \SimHei \scriptsize 公元 & \SimHei 大事件 \tabularnewline
  % \midrule
  \endfirsthead
  \toprule
  \SimHei \normalsize 年数 & \SimHei \scriptsize 公元 & \SimHei 大事件 \tabularnewline
  \midrule
  \endhead
  \midrule
  元年 & 968 & \tabularnewline\hline
  二年 & 969 & \tabularnewline\hline
  三年 & 970 & \tabularnewline\hline
  四年 & 971 & \tabularnewline\hline
  五年 & 972 & \tabularnewline\hline
  六年 & 973 & \tabularnewline\hline
  七年 & 974 & \tabularnewline\hline
  八年 & 975 & \tabularnewline
  \bottomrule
\end{longtable}


%%% Local Variables:
%%% mode: latex
%%% TeX-engine: xetex
%%% TeX-master: "../../Main"
%%% End:



%%% Local Variables:
%%% mode: latex
%%% TeX-engine: xetex
%%% TeX-master: "../../Main"
%%% End:

%% -*- coding: utf-8 -*-
%% Time-stamp: <Chen Wang: 2019-12-25 10:28:42>


\section{吴越\tiny(907-978)}

\subsection{简介}

吳越(907-978)是五代十國時期的十國之一,由錢鏐在公元907年所建。都城為錢塘(杭州)。強盛時擁有十三州疆域,約為現今浙江全省、江蘇東南部和福建東北部。吳越國共有五位君主,傳國七十一年,末主錢弘俶於公元978年獻土入宋。

吳越國前身與基礎可一直上溯至唐末大混亂時期杭州地方的鄉兵集團杭州八都。

錢鏐本為石鏡都的副將,助主將董昌取得杭州、擊敗浙東觀察使劉漢宏的侵略後,董昌將杭州刺史一職以及杭州八都集團的大部分讓給了錢鏐,是錢鏐獲得獨立地盤之始。以下表列出此政權之擴張事件及領域變動。

893年,錢鏐為唐鎮海節度使。 907年被後梁封為吳越王。

975年援北宋滅南唐,978年吳越末代國王錢俶為了避免戰亂主動献土并入北宋。

886年,浙西鎮海軍兵變,錢鏐以平亂為名出兵攻陷常州及潤州。

887年,消滅佔據蘇州的徐約。

891年,孫儒亂江南,與楊行密、錢鏐爭奪常、潤、蘇三州。潤、常最終為楊行密所據。

896年,為唐朝討伐稱帝的越州董昌,平之;過程中楊行密援董昌,攻陷蘇州。

896年,湖州刺史李師悅病逝,部下都將聯合趕走其子李繼徽、歸附錢鏐。

898年,克復蘇州。分蘇州嘉興為秀州。陸續與浙東諸州勢力交戰,或在該州土豪病逝後收服之。

945年,出兵援助閩國抵抗南唐,閩國部將李仁達以福州歸附。

吴越国采取保境安民的政策,经济繁荣,渔盐桑蚕之利甲于江南;文士荟萃,人才济济,文艺也著称于世。由于吴国阻隔陆路,因此吴越朝贡中原王朝多经登、莱海路,海上交通发达,与後百濟、新罗、日本的海上贸易和文化交流频繁。

吳越國的水利在十國中是最著名的。錢鏐設撩湖軍,開浚錢塘湖,得其遊覽、灌溉兩利,又引湖水為湧金池,與運河相通。此外,在唐末時期,錢塘江口地區因海潮襲擊,“自秦望山東南十八堡,數千萬畝田地悉成江面,民不堪命”。後梁開平四年/吳越天寶三年(910年),錢鏐動員大批勞力,修築“捍海石塘”。用木樁把裝滿石塊的巨大石籠固定在江邊,形成堅固的海堤,保護了江邊農田不再受潮水侵蝕。並且由於石塘具有蓄水作用,使得江邊農田得獲灌溉之利。由是“錢塘富庶盛於東南”。

錢鏐還在太湖地區設“撩水軍”四部、七八千人,專門負責浚湖、築堤、疏濬河浦,使得蘇州、嘉興、長洲等地得享灌溉之利。此外錢氏還修建武義縣的長安堰,受益農田上萬頃;東府的鑑湖,餘杭縣的上湖、下湖、北湖,諸暨的完浦,慈溪的慈濟湖,明州的南湖,鄞縣的廣德湖、東錢湖、它山堰,也都是重要的灌溉水源。吳越境內田塘眾多,土地膏腴,有“近澤知田美”之語。

錢鏐在位時,即鼓勵擴大墾田,下令“荒田任開,不起稅額”。由是“境內無棄田”,歲熟豐稔,民間五十錢可糴白米一石。兩浙又為著名桑麻產地,湖州顧渚山出產著名的“紫筍茶”,天福七年(942年)忠獻王錢弘佐一次就向後晉進貢二萬五千斤之多。

吳越國的手工業高度發達,官府生產的各色繡金錦緞綾絹不僅供王宮之需,還大量進貢中原王朝。吳越國的陶瓷業也相當興盛,主要的陶瓷器生產場地是越州餘姚上林湖的越州窯,此外還在處州龍泉、上虞窯前寺等地設立官窯。吳越生產的“秘色瓷”昔日為錢氏內用,大臣非有功不得賜,故名。其工藝細膩,胎骨均勻,底部光潔,為吳越進貢及海外貿易的主要物資之一。溫州出產的蠲紙,潔白堅滑,專門供應官府。

佛教是吴越国文化的重要组成部分,历代吴越国王均笃信佛教,吴越境内佛寺林立。忠懿王时期,境内佛塔达到8万4千座,有名的如今天杭州的保俶塔、雷峰塔。受吴越国的佛教氛围影响,杭州的雕版印刷业异常发达,仅忠懿王钱俶(钱弘俶)时期印刷的经书、佛像就达六十六万两千卷之多。宋代杭州印刷业居全国第一,即缘于此故。

吴越与南汉、新罗、后百济、高丽、日本、琉球等国通商于海上。吴越向海外出口瓷器、锦缎、绫绢,进口苏木、乳香、沉香、龙脑、玳瑁、珍珠、日本椤木、铜器、扇子等货物,甚至包括大食的猛火油。吴越与中原内地的贸易最初通过楚国和荆南,后吴国占领江西全境,吴越的北方贸易改走海路,在登州、莱州、青州等地登陆,经陆路至汴梁、洛阳。杭州与明州是最重要的两座港口,都城杭州“舟楫辐辏,望之不见首尾”。

吳越國在錢氏家族治理下,政治上比較安定,對外謹事中原王朝,奉正朔,歲時進貢;內無楊吳、馬楚、南漢的兄弟相殘之禍,保境安民,社會繁榮,經濟富裕。

司馬光《資治通鑑》載,吳越忠獻王錢弘佐年十四即位,問倉吏“今蓄積幾何?”答曰“十年”,錢弘佐曰“軍食足矣,可以寬吾民”,於是命境內免稅三年。明朝朱國楨則評價吳越政治稱“錢立國,置營田數千人於松江,闢土而耕,…民老死無他纏累,且完國歸朝,不殺一人,則其功德大矣!”

然而北宋歐陽修《五代史·吳越世家》評價吳越國則稱,自錢鏐時起,賦稅繁苛,小至雞、魚、雞卵、雞雛,也要納稅。貧民欠稅被捉到官府,按各稅欠數多少定笞數,往往積至笞數十以至百餘(一說五百餘),民尤不勝其苦。而宋代即有論者稱歐陽修此舉為挾私怨於褒貶之間。

吳越曾向後唐進貢萬壽節金器、連花金食器、盤龍鳳錦織成紅羅袍、金排方盤龍帶禦衣、通犀瑞象腰帶、紅地龍鳳錦被、錦綺、越綾、吳綾、異紋綾、金條紗、絹布、綿布、秘色瓷器、銀妝花櫚木櫥子、龍鳳紗紋櫥、紅藤龍鳳箱、佛頭螺子青、山螺子青、菩薩石蟹子、白龍腦、大茶、腦源茶等物。向後晉進貢銀、絹、絲、細甲、弓弩、箭、扇子、靴履、細酒、細紙、蘇木、乾薑、秘色瓷、乳香、啟聖節金大排方坐龍腰帶、禦衣等物。向後漢進貢禦衣、通犀帶、戲龍金帶、金器、銀器、兵仗、綾絹、茶、香、藥物、鞍履、海味。向後周進貢禦衣、銀、綾、絹、龍舟、天祿舟。

宋朝開國後,吳越國日益向其進貢無算,宋太祖曾說:「這些我遲早都要拿的,哪需要獻來呢?」吳越常向宋朝進貢赭黃犀角、龍鳳龜魚、仙人鰲山寶樹等物。進貢通犀寶帶七十餘條,玉帶二十條,紫金獅子帶一條,塗金銀香龍一座,及珊瑚樹十棵、高三尺五寸,皆稀世之寶。又向宋進貢金飾玳瑁器皿一千五百餘件,水晶瑪瑙玉器四千餘件,金銀器及秘色瓷十四萬餘件,金銀飾龍鳳舟二百餘艘,銀飾器物七十萬件,金九萬五千餘兩,銀一百一十萬餘兩,繡金盤龍鳳錦緞衣料數万匹,白龍腦二百餘斤。至於珍珠、玳瑁、琥珀、乳香、沉香、龍涎香、蘇木、貢茶、御酒、綾絹、海味等物無數。而宋朝舉朝文武及宮中內官亦多有饋贈。以十三州之力,供大國歡心,吳越國力以是漸貧。

吳越最大的敵國是其西北的楊吳政權,吳越與其攻伐多年,919年無錫​​之戰後應吳國齊王徐溫要求,兩國修和。但楊吳的繼承者南唐也是吳越發展的主要競爭對手。為保護吳越國,歷代國王尊奉後梁、後唐、後晉、後漢、後周、宋朝六個中原王朝,事大主義,抵抗兩淮,保境安民。在宋滅南唐的戰役中,吳越出兵援宋。此外吳越國亦曾和南唐共同攻打閩國,佔領福州。

935年,吳越與日本的國交始開。次年,藤原忠平向吳越王送抵國書,建立兩國良好關係。940年藤原仲平、947年藤原實賴、953年藤原師輔相繼向吳越王遞交國書。957年吳越王回送黃金。

%% -*- coding: utf-8 -*-
%% Time-stamp: <Chen Wang: 2021-11-01 15:40:07>

\subsection{太祖钱镠\tiny(907-932)}

\subsubsection{生平}

吴越太祖钱\xpinyin*{镠}(852年3月10日-932年5月6日),字具美(一作巨美),浙江杭州临安(今临安区)人。五代十国时期吴越國開國國王。

唐末跟从石镜镇将军董昌镇压農民反抗軍,任镇海节度使,乾宁年间击败董昌,占有两浙十三州,后梁开平初年被封为吴越王。在位期间,曾征用民工,修建钱塘江海塘,又在太湖流域,普造堰闸,以时蓄洪,不畏旱涝,并建立水网圩区的维修制度,有利于这一地区的农业经济。

由於吳越國小力弱,又同鄰近的吳、閩政權不和,投靠中原王朝,不斷遣使進貢以求庇護。先臣服後梁,又臣服後唐。后唐明宗時因惹怒樞密使安重誨,被削去官職,安重誨死後又恢復。長興三年(932年)病死,葬安國縣(现临安区)衣錦鄉茅山。庙号太祖,諡號武肃王。

唐朝大中六年二月十六日,钱镠生于临安县石镜乡大官山下的临水里钱坞垅。父亲钱宽,母亲水丘氏。一家以农耕打渔为生。传说钱镠出生时突现红光,且相貌奇丑,父亲本欲弃之,但因其祖母怜惜,最后得以保全性命,因此钱鏐小名“婆留”(“阿婆留其命”之义)。

钱鏐自幼不喜诗文,偏好习武,常与邻里诸小儿戏于里中大木之下,指挥群儿为队伍,号令颇有法(钱鏐即位后将此树封为“将军木”。钱鏐在16岁的时候就弃学贩盐。当时私贩盐料是官府严厉禁止的,但由于利润极高,因此钱鏐铤而走险,在杭州、越州(今绍兴)、宣州等地贩卖私盐和粮食。这段贩卖私盐的经历,练就了钱鏐体魄和胆略,也为他日后发展提供了充足的经济基础。

17岁开始,钱鏐苦练硬弓长矛,并读些《孙子兵法》,史书称其“善射与槊,稍通图纬诸书”。到21岁时,他在石镜镇充当“义兵”,并将小名“钱婆留”改为大名“钱鏐”(其为金字辈,并取“留”字音,故改“鏐”)。由于钱镠武艺高强,受到石镜镇指挥使董昌重用,经过平定王郢、朱直管、曹师雄、王知新等叛乱之后,逐渐提拔为偏将、副指挥使、兵马使、镇海军右副使等职。

879年(唐僖宗乾符六年)七月,黄巢起义军进犯临安。钱鏐以少敌多,巧妙运用伏击和虚张声势等战术,阻吓了黄巢军的进攻。880年,唐朝内乱四起,为保护地方安定,董昌、钱鏐联合各县民团,建立“八都军”(临安县“石镜都”、余杭县“清平都”、於潜县“於潜都”、盐官县“盐官都”、新城县“武安都”、唐山县“唐山都”、富阳县“富春都”和龙泉县“龙泉都”),次年,钱鏐授“都知兵马使”,并注意团结各都力量和下层头目,还将其弟钱銶、钱镒、钱铧、钱镖,以及儿子钱元璙、钱元瓘等人安插到部队中担任将领,从而将八都军逐渐培养成坚强的嫡系部队。

唐末、五代时期所称“两浙十四州”,包括现在浙江全境和江苏长江以南部分地区。七五八年,江南东道下属的浙江东道 和浙江西道 共有十四州,其中除去润州和常州,再加上福建的福州和临安县的安国衣锦军,共为一军十三州,号称“十四州”,便是钱鏐创立的吴越国的大致范围。

自讨伐王郢起,钱鏐身经百战,先后与刘汉宏、董昌等地方主要军阀作战,最终平定了两浙范围内的敌对势力,建立了巩固的地方割据政权。

882年7月起,占据浙东的义胜军节度使刘汉宏发兵西进,欲并吞浙西。董昌、钱鏐率八都军在钱塘江边御敌。由于出奇制胜,加上利用江上夜雾遮掩,钱鏐突袭敌营,获得首胜。之后,又在江干、富阳、诸暨、萧山西陵等地屡败刘军。最后,刘汉宏亲自督战,率十万大军与钱鏐在萧山西陵一带决战,结果被钱镠击溃,刘汉宏本人易装成屠户逃跑。这一次西陵大捷,是钱鏐取得的第一次重大战果,据说,从此钱鏐将西陵改名为西兴至今(现钱江三桥又名“西兴大桥”)。

此后,刘汉宏仍不断骚扰浙西,导致董昌和钱鏐决心彻底平定浙东之患。886年10起,钱鏐仅用了2个月左右的时间,就率军攻克越州,并将潜逃被捕的刘汉宏斩于会稽街市。此后,钱鏐为杭州刺史,董昌升任浙东观察使、检校太尉、陇西郡王等职。

董昌其人昏庸残暴,野心日增,随后就即位称帝,国号大越罗平,改元顺天。895年2月,唐朝封钱鏐为浙东招讨使,令其讨伐董昌。但钱鏐起初感念董昌提携之恩,犹豫不决,但董昌却联合淮南杨行密偷袭苏州、杭州,最终使得钱鏐下定决心,攻克越州。董昌在被押付杭州途中,心存惭愧,投江自杀。从此,钱鏐基本控制两浙,并于896年10月,被授为镇海、镇东军节度使,加检校太尉,兼中书令。

897年8月,鉴于钱鏐招讨董昌有功,唐昭宗特赐金书铁券于他,免其本人九死或子孙三死。这件钱镠铁券后经宋代陆游、明代刘基等人为其写跋,还呈宋太宗、宋仁宗、宋神宗、明太祖、明成祖和清高宗等七位帝王御览。900年,为了表彰钱王的功绩,唐王朝派人取钱鏐画像,悬于凌烟阁。

钱鏐在平定了两浙内部的敌对势力后,基本停止了大规模的征讨。但由于三面受敌,仍经历了多次边境保卫战,有时还将战斗延伸至江西的信州(今上饶)和虔州(今赣州)等地。其主要对手就是淮南军阀杨行密和内部的“徐许之乱”。

钱鏐和杨行密的关系时而友好,时而敌对,体现出五代十国乱世的特点。双方的冲突共持续了三十年,其间钱曾出兵援助杨擒斩孙儒、安仁义等叛逆,并正式通婚,但也因董昌之战等发生过激烈的战斗。最后通过两次衣锦军保卫战和一次浪山江水战,才结束了双方的敌对状态。从此两浙地区进入休养生息的安定建设阶段。

902年,钱鏐刚被封为越王不久,其部下的徐绾和许再思起兵叛变,使钱鏐大伤元气。最后钱鏐支付了二十万缗犒军钱,并派两个儿子作为人质,才使得叛军撤兵。这次内乱后,钱鏐吸取了教训,治国更为谨慎。

904年被封为吴王;907年,后梁封钱鏐为吴越王,吴越国自此创建。龙德三年(923年),钱镠被册封为吴越国王,吴越建立王国体制。他改府署为朝廷,设置丞相、侍郎等百官,一切礼制皆按照君主的规格。

结束了与周边敌对势力的战争后,钱鏐开始转向对内的大规模经济和文化建设。唐大顺元年(890年)钱鏐开始着手建设杭州城。先后建造了夹城、罗城和子城。杭州罗城筑于唐景福元年(892年)七月,筑城时发动余杭、盐官、新城、唐山、富阳、龙泉“八都兵”,及紫溪、保城、龙通、三泉、三镇,合计“十三都兵”二十余万人。城区范围广袤七十里,四至分别是:南到六和塔;东至侯潮门和艮山门一线;北达武林门;西临涌金门和清波门一带,设朝天门、龙山门、竹车门、南土门、北土门、盐桥门、西关门(涵水门)、北关门、宝德门共十门。天宝三年(910年)又扩杭州城,凤凰山柳浦隋唐所筑子城被改造为府城,南为通越门,北为只门,子城内大修台馆,有天册堂(即王位之所)、天宠堂(即位、理政之所)、思政堂、功臣堂(寝宫)、握发殿、咸宁院、义和院、碧波亭、虚白堂、八会亭、都会堂、蓬莱阁、直仪门(设厅)、青史楼、天长楼、玉华楼、瑞萼园等建筑。钱鏐筑杭州城,在客观上为杭州成为日后南宋的都城打下了基础,南宋临安宫城即原吴越王宫。

钱王还在城内开凿水井(据说杭州的百井坊巷原有99眼,就开凿于此时),建设钱塘江堤,为杭州的饮水淡化问题做出了很大贡献。此外,钱鏐及其继承者崇信佛教,前后修建了不少寺院佛塔,使杭州在当时就有“佛国”之称。其中著名的灵隐寺、净慈寺、昭庆寺等寺院,以及雷峰塔、六和塔、保俶塔、闸口白塔和临安功臣塔等都是在吴越国时期兴建或扩建的。

钱鏐在内政建设上的主要成就体现在修筑海塘和疏浚内湖上。910年起,钱鏐上书后梁朝廷,指出“目击平原沃野,尽成江水汪洋,虽值干戈扰攘之后,即兴筑塘修堤之举。”,并开始着手修筑钱塘江沿岸石塘。由于钱江潮汛,工程进展困难,后钱鏐以竹器填以巨石,才奠定了基础。当时修筑的石塘,从六和塔一直到艮山门,长33万8593丈。此外,钱鏐还重点抓了疏浚西湖、太湖和鉴湖等工作。当时他设置了7000名撩湖兵,专门从事西湖的开浚工作 后代的苏轼也是在参考了钱鏐治湖的经验上,才开始大规模疏浚西湖。

然而据欧阳修《五代史》吴越世家所称,吳越自錢鏐時起,賦稅繁苛,小至雞、魚、雞卵、雞雛,也要納稅。貧民欠稅被捉到官府,按各稅欠數多少定笞數,往往積至笞數十以至百餘(一說五百余),民尤不勝其苦。於杭州建造「地上天宮」,耗盡民財民力。

钱鏐做節度使時,有人獻詩,詩中有「一條江水檻前流」句,「前流」與「钱鏐」是諧音,钱鏐認為獻詩人諷刺自己,於是暗殺此人。羅隱聲名大,曾作詩譏笑钱鏐出身寒家,錢鏐卻欣然不怒。錢鏐留心收買名士,皮日休(當是黃巢失敗後,逃來依靠錢鏐)、羅隱、胡嶽等都得到優待,自己也學吟詠,與名士唱和。天宝三年十月钱鏐巡视故乡衣锦军,置酒宴请父老,赏八十岁以上者金樽,百岁以上者玉樽,又作《还乡歌》:“三节还乡兮挂锦衣,碧天朗朗兮爱日晖。功臣道上兮列旌旗,父老远来兮相追随。家山乡眷兮会时稀,今朝设宴兮觥散飞。斗牛无孛兮民无欺,吴越一王兮驷马归”。父老不解其意,钱鏐复用吴语为歌:“你辈见侬底欢喜,则是一般滋味子,长在我侬心底里”,举座叫笑振席。

由于钱鏐在其晚年坚持保境安民政策,不参与军阀混战,而且对内统治相对廉洁清明,使得这一时期杭州的发展超越了中原地区的许多大城市,成为东南地区的经济中心。

后唐长兴三年(932年)三月己酉,錢鏐薨于临安王府正寝,年八十一岁,在位四十一年。葬安国县衣锦乡茅山,建庙于东府。后唐赐谥号武肃,吴越国上庙号太祖。

錢鏐累事三朝,唐、后梁、后唐屡加封号,累赐启圣匡运同德功臣、定乱安国启圣昌运同德守道戴功臣、淮南镇海镇东等军节度使、淮南浙江东西等道管内观察处置、充淮南四面都统营田安抚、兼两浙盐铁制置发运等使、天下兵马都元帅、开府仪同三司、尚父、检校太师、尚书令、兼中书令、上柱国、吴越国王,赐剑履上殿、诏书不名,食邑一万五千户。

欧阳修《五代史》称吴越“有改元而无称帝之事”。吴越国从908年(后梁开平二年)至913年(后梁乾化三年),曾用天宝年号;924年(后唐同光二年)至931年(后唐长兴二年)用宝大、宝正年号,皆仅行于吴越国中。

后世一般对钱氏评价较高,认为他促进了地方经济发展,保障了民众安居乐业的局面。主要有:“时维五纪乱何如?史册闲观亦皱眉。是地却逢钱节度,民间无事看花嬉!”——北宋·赵抃

“钱立国,置营田数千人于松江,辟土而耕,…民老死无他缠累,且完国归朝,不杀一人,则其功德大矣!”—— 明·朱国桢

史书载钱鏐性俭朴,衣衾杂用细布,常膳用瓷器、漆器。除夕子夜与子孙宴于府城内,未鼓数曲而令罢宴,称“闻者以我为长夜之歌”。其寝居之殿名为“握发殿”,取周公“一沐三握发”典故。

欧阳修在《新五代史·吴越世家》中谴责钱氏严刑酷法。而宋代别史《丹铅录》称,欧阳修为推官时,昵一妓,比而为忠懿王之子钱惟演得去,欧阳修深衔之,后作《五代史》时乃诬以钱氏诸王“重敛虐民”之语,以公报私。钱世昭撰《钱氏私志》也稱歐陽修是挾怨報复。

目前在西湖南岸,建有钱王祠,供后人瞻仰钱王业绩。

\subsubsection{天祐}

\begin{longtable}{|>{\centering\scriptsize}m{2em}|>{\centering\scriptsize}m{1.3em}|>{\centering}m{8.8em}|}
  % \caption{秦王政}\
  \toprule
  \SimHei \normalsize 年数 & \SimHei \scriptsize 公元 & \SimHei 大事件 \tabularnewline
  % \midrule
  \endfirsthead
  \toprule
  \SimHei \normalsize 年数 & \SimHei \scriptsize 公元 & \SimHei 大事件 \tabularnewline
  \midrule
  \endhead
  \midrule
  元年 & 907 & \tabularnewline
  \bottomrule
\end{longtable}

\subsubsection{天宝}

\begin{longtable}{|>{\centering\scriptsize}m{2em}|>{\centering\scriptsize}m{1.3em}|>{\centering}m{8.8em}|}
  % \caption{秦王政}\
  \toprule
  \SimHei \normalsize 年数 & \SimHei \scriptsize 公元 & \SimHei 大事件 \tabularnewline
  % \midrule
  \endfirsthead
  \toprule
  \SimHei \normalsize 年数 & \SimHei \scriptsize 公元 & \SimHei 大事件 \tabularnewline
  \midrule
  \endhead
  \midrule
  元年 & 908 & \tabularnewline\hline
  二年 & 909 & \tabularnewline\hline
  三年 & 910 & \tabularnewline\hline
  四年 & 911 & \tabularnewline\hline
  五年 & 912 & \tabularnewline
  \bottomrule
\end{longtable}

\subsubsection{凤历}

\begin{longtable}{|>{\centering\scriptsize}m{2em}|>{\centering\scriptsize}m{1.3em}|>{\centering}m{8.8em}|}
  % \caption{秦王政}\
  \toprule
  \SimHei \normalsize 年数 & \SimHei \scriptsize 公元 & \SimHei 大事件 \tabularnewline
  % \midrule
  \endfirsthead
  \toprule
  \SimHei \normalsize 年数 & \SimHei \scriptsize 公元 & \SimHei 大事件 \tabularnewline
  \midrule
  \endhead
  \midrule
  元年 & 913 & \tabularnewline
  \bottomrule
\end{longtable}

\subsubsection{乾化}

\begin{longtable}{|>{\centering\scriptsize}m{2em}|>{\centering\scriptsize}m{1.3em}|>{\centering}m{8.8em}|}
  % \caption{秦王政}\
  \toprule
  \SimHei \normalsize 年数 & \SimHei \scriptsize 公元 & \SimHei 大事件 \tabularnewline
  % \midrule
  \endfirsthead
  \toprule
  \SimHei \normalsize 年数 & \SimHei \scriptsize 公元 & \SimHei 大事件 \tabularnewline
  \midrule
  \endhead
  \midrule
  元年 & 913 & \tabularnewline\hline
  二年 & 914 & \tabularnewline\hline
  三年 & 915 & \tabularnewline
  \bottomrule
\end{longtable}

\subsubsection{贞明}

\begin{longtable}{|>{\centering\scriptsize}m{2em}|>{\centering\scriptsize}m{1.3em}|>{\centering}m{8.8em}|}
  % \caption{秦王政}\
  \toprule
  \SimHei \normalsize 年数 & \SimHei \scriptsize 公元 & \SimHei 大事件 \tabularnewline
  % \midrule
  \endfirsthead
  \toprule
  \SimHei \normalsize 年数 & \SimHei \scriptsize 公元 & \SimHei 大事件 \tabularnewline
  \midrule
  \endhead
  \midrule
  元年 & 915 & \tabularnewline\hline
  二年 & 916 & \tabularnewline\hline
  三年 & 917 & \tabularnewline\hline
  四年 & 918 & \tabularnewline\hline
  五年 & 919 & \tabularnewline\hline
  六年 & 920 & \tabularnewline\hline
  七年 & 921 & \tabularnewline
  \bottomrule
\end{longtable}

\subsubsection{龙德}

\begin{longtable}{|>{\centering\scriptsize}m{2em}|>{\centering\scriptsize}m{1.3em}|>{\centering}m{8.8em}|}
  % \caption{秦王政}\
  \toprule
  \SimHei \normalsize 年数 & \SimHei \scriptsize 公元 & \SimHei 大事件 \tabularnewline
  % \midrule
  \endfirsthead
  \toprule
  \SimHei \normalsize 年数 & \SimHei \scriptsize 公元 & \SimHei 大事件 \tabularnewline
  \midrule
  \endhead
  \midrule
  元年 & 921 & \tabularnewline\hline
  二年 & 922 & \tabularnewline\hline
  三年 & 923 & \tabularnewline
  \bottomrule
\end{longtable}

\subsubsection{宝大}

\begin{longtable}{|>{\centering\scriptsize}m{2em}|>{\centering\scriptsize}m{1.3em}|>{\centering}m{8.8em}|}
  % \caption{秦王政}\
  \toprule
  \SimHei \normalsize 年数 & \SimHei \scriptsize 公元 & \SimHei 大事件 \tabularnewline
  % \midrule
  \endfirsthead
  \toprule
  \SimHei \normalsize 年数 & \SimHei \scriptsize 公元 & \SimHei 大事件 \tabularnewline
  \midrule
  \endhead
  \midrule
  元年 & 924 & \tabularnewline\hline
  二年 & 925 & \tabularnewline
  \bottomrule
\end{longtable}

\subsubsection{宝正}

\begin{longtable}{|>{\centering\scriptsize}m{2em}|>{\centering\scriptsize}m{1.3em}|>{\centering}m{8.8em}|}
  % \caption{秦王政}\
  \toprule
  \SimHei \normalsize 年数 & \SimHei \scriptsize 公元 & \SimHei 大事件 \tabularnewline
  % \midrule
  \endfirsthead
  \toprule
  \SimHei \normalsize 年数 & \SimHei \scriptsize 公元 & \SimHei 大事件 \tabularnewline
  \midrule
  \endhead
  \midrule
  元年 & 926 & \tabularnewline\hline
  二年 & 927 & \tabularnewline\hline
  三年 & 928 & \tabularnewline\hline
  四年 & 929 & \tabularnewline\hline
  五年 & 930 & \tabularnewline\hline
  六年 & 931 & \tabularnewline
  \bottomrule
\end{longtable}




%%% Local Variables:
%%% mode: latex
%%% TeX-engine: xetex
%%% TeX-master: "../../Main"
%%% End:

%% -*- coding: utf-8 -*-
%% Time-stamp: <Chen Wang: 2019-12-26 09:39:23>

\subsection{世宗\tiny(932-941)}

\subsubsection{生平}

吳越世宗錢元瓘(887年-941年),字明寶,原名錢傳瓘,五代時期吳越國君主,是吳越國建立者錢鏐之子。

錢傳瓘为吴越武肃王錢鏐第七子,唐朝光启三年十一月十二日生于杭州东院,母妃陈氏。

唐昭宗天復二年(902年),寧國節度使田頵攻擊時為鎮海節度使(地處今浙江杭州)的錢鏐,將回師時,要求錢鏐以一子為質,並將女兒嫁其子。錢鏐諸子皆不願去,只有錢傳瓘自願前往,錢鏐因之稱奇。後來田頵敗死,錢傳瓘得以復歸杭州。等到長大成人後,率吳越軍爭戰各地,頗有戰功。贞明五年三月,吴越国应同盟後梁的邀请,进攻杨吴。传瓘任诸军都指挥使,帅战舰五百艘进攻。于狼山江大败吴军,进击常州。吴国实权者徐温亲自拒战,傳瓘力不能及,爲其所败。

吳錢鏐欲立儲,諸兄钱传懿、钱传璙、钱传璟皆相讓。越寶正七年(932年)錢鏐去世,錢傳瓘繼立,改名錢元瓘,不稱王,使用後唐長興年號。後唐明宗李嗣源長興四年(933年),為後唐封吳王。明年(934年)改封錢傳瓘为吳越王。後晉天福二年(937年)四月,後晉高祖石敬瑭進封錢元瓘為兴邦保运崇德志道功臣、天下兵马副元帅、镇海镇东等军节度使、浙江东西等道管内观察处置使兼两浙盐铁制置使、开府仪同三司、检校太师、守中书令、杭州越州大都督府长史、上柱国、食邑一万五千户实封一千五百户、吳越國國王,赐天下兵马副元帅金印。錢傳瓘于四月甲午在杭州行即位礼,吳越再度開國。是年十一月后晋赐吴越国王金册。

天福五年(940年),闽国内乱,王延政在建州起兵,钱元瓘派兵四万支持王延政,然而吴越军到达建州后王延羲与王延政已经休兵,吴越将领仰仁銓不肯班师,王延政惧怕,倒戈攻击,吴越死伤惨重。随后在同年,初次设立秀州(辖嘉兴、海盐、华亭、崇德四县,包括现今嘉兴和上海城区)和新昌县。

天福六年(941年),吳越王宮丽春院失火,延及内城,宮室府庫幾乎完全燒燬,錢元瓘逃到何處,火即蔓延何處。受此驚嚇,錢元瓘因而發瘋,迁居于杭州城东北的瑶台院(原为錢元瓘为孝献世子钱弘僔营建的世子府)。八月辛亥,錢元瓘在瑶台院綵云堂去世,终年五十五岁,在位十年。后晋赠諡庄穆,后改文穆。吴越上廟號世宗 。葬于今浙江萧山龙山南。子錢弘佐繼位。

\subsubsection{长兴}

\begin{longtable}{|>{\centering\scriptsize}m{2em}|>{\centering\scriptsize}m{1.3em}|>{\centering}m{8.8em}|}
  % \caption{秦王政}\
  \toprule
  \SimHei \normalsize 年数 & \SimHei \scriptsize 公元 & \SimHei 大事件 \tabularnewline
  % \midrule
  \endfirsthead
  \toprule
  \SimHei \normalsize 年数 & \SimHei \scriptsize 公元 & \SimHei 大事件 \tabularnewline
  \midrule
  \endhead
  \midrule
  元年 & 932 & \tabularnewline\hline
  二年 & 933 & \tabularnewline
  \bottomrule
\end{longtable}

\subsubsection{应顺}

\begin{longtable}{|>{\centering\scriptsize}m{2em}|>{\centering\scriptsize}m{1.3em}|>{\centering}m{8.8em}|}
  % \caption{秦王政}\
  \toprule
  \SimHei \normalsize 年数 & \SimHei \scriptsize 公元 & \SimHei 大事件 \tabularnewline
  % \midrule
  \endfirsthead
  \toprule
  \SimHei \normalsize 年数 & \SimHei \scriptsize 公元 & \SimHei 大事件 \tabularnewline
  \midrule
  \endhead
  \midrule
  元年 & 934 & \tabularnewline
  \bottomrule
\end{longtable}

\subsubsection{清泰}

\begin{longtable}{|>{\centering\scriptsize}m{2em}|>{\centering\scriptsize}m{1.3em}|>{\centering}m{8.8em}|}
  % \caption{秦王政}\
  \toprule
  \SimHei \normalsize 年数 & \SimHei \scriptsize 公元 & \SimHei 大事件 \tabularnewline
  % \midrule
  \endfirsthead
  \toprule
  \SimHei \normalsize 年数 & \SimHei \scriptsize 公元 & \SimHei 大事件 \tabularnewline
  \midrule
  \endhead
  \midrule
  元年 & 934 & \tabularnewline\hline
  二年 & 935 & \tabularnewline\hline
  三年 & 936 & \tabularnewline
  \bottomrule
\end{longtable}

\subsubsection{天福}

\begin{longtable}{|>{\centering\scriptsize}m{2em}|>{\centering\scriptsize}m{1.3em}|>{\centering}m{8.8em}|}
  % \caption{秦王政}\
  \toprule
  \SimHei \normalsize 年数 & \SimHei \scriptsize 公元 & \SimHei 大事件 \tabularnewline
  % \midrule
  \endfirsthead
  \toprule
  \SimHei \normalsize 年数 & \SimHei \scriptsize 公元 & \SimHei 大事件 \tabularnewline
  \midrule
  \endhead
  \midrule
  元年 & 936 & \tabularnewline\hline
  二年 & 937 & \tabularnewline\hline
  三年 & 938 & \tabularnewline\hline
  四年 & 939 & \tabularnewline\hline
  五年 & 940 & \tabularnewline\hline
  六年 & 941 & \tabularnewline
  \bottomrule
\end{longtable}



%%% Local Variables:
%%% mode: latex
%%% TeX-engine: xetex
%%% TeX-master: "../../Main"
%%% End:

%% -*- coding: utf-8 -*-
%% Time-stamp: <Chen Wang: 2021-11-01 15:40:37>

\subsection{成宗錢弘佐\tiny(941-947)}

\subsubsection{生平}

吳越成宗錢弘佐(928年-947年6月22日),字元祐,五代時期吳越國君主。

錢弘佐為吴越文穆王錢元瓘第六子,存世第二子。宝正三年七月二十六日生于杭州吴越王宫功臣堂,母许氏。

錢元瓘初以五子钱弘僔为世子,并为其建世子府于杭州城东北。一日钱弘僔与钱弘佐博彩于王宫青史楼,世子称“君王方为我营府署,愿与若博之”。骰四掷,钱弘佐得六赤色,世子失色。钱弘佐从容称“五哥入府,弘佐当将符印之命”,世子变色,投骰盘于楼下而去。天福五年钱弘僔薨,追赠孝献世子(世子府改为瑶台院),而钱弘佐得封镇海、镇东节度副使、检校太傅。

後晉天福六年(941年),錢元瓘去世,錢弘佐于当年九月庚申即位于王宫倦居堂。十一月,後晉封以镇国大将军、右金吾卫上将军、员外置同正员、领镇海镇东等军节度使、检校太师、兼中书令、吳越國王,食邑一万户,实封一千户。天福七年赐保邦宣化忠正戴功臣,加食邑七千户。天福八年赐吴越国王玉册。

後晉開運二年(945年),闽国内乱,钱弘佐派軍與南唐瓜分閩國,佔領福州。

錢弘佐喜好讀書,性情溫順,很會做詩。即位後,因尚年幼,無力控制下屬的驕橫,又曾寵信諂媚之人,然而終能摘奸發伏,亦不失果斷。

後漢天福十二年(遼國會同十年,947年)六月乙卯,錢弘佐去世于王宫咸宁院西堂,终年二十岁,在位七年。后漢赠諡忠獻王。吴越上廟號成宗。因其子尚年幼,故由其弟錢弘倧繼位。

\subsubsection{天福}

\begin{longtable}{|>{\centering\scriptsize}m{2em}|>{\centering\scriptsize}m{1.3em}|>{\centering}m{8.8em}|}
  % \caption{秦王政}\
  \toprule
  \SimHei \normalsize 年数 & \SimHei \scriptsize 公元 & \SimHei 大事件 \tabularnewline
  % \midrule
  \endfirsthead
  \toprule
  \SimHei \normalsize 年数 & \SimHei \scriptsize 公元 & \SimHei 大事件 \tabularnewline
  \midrule
  \endhead
  \midrule
  元年 & 941 & \tabularnewline\hline
  二年 & 942 & \tabularnewline\hline
  三年 & 943 & \tabularnewline\hline
  四年 & 944 & \tabularnewline
  \bottomrule
\end{longtable}

\subsubsection{开运}

\begin{longtable}{|>{\centering\scriptsize}m{2em}|>{\centering\scriptsize}m{1.3em}|>{\centering}m{8.8em}|}
  % \caption{秦王政}\
  \toprule
  \SimHei \normalsize 年数 & \SimHei \scriptsize 公元 & \SimHei 大事件 \tabularnewline
  % \midrule
  \endfirsthead
  \toprule
  \SimHei \normalsize 年数 & \SimHei \scriptsize 公元 & \SimHei 大事件 \tabularnewline
  \midrule
  \endhead
  \midrule
  元年 & 944 & \tabularnewline\hline
  二年 & 945 & \tabularnewline\hline
  三年 & 946 & \tabularnewline
  \bottomrule
\end{longtable}



%%% Local Variables:
%%% mode: latex
%%% TeX-engine: xetex
%%% TeX-master: "../../Main"
%%% End:

%% -*- coding: utf-8 -*-
%% Time-stamp: <Chen Wang: 2019-12-26 09:40:32>

\subsection{忠逊王\tiny(947)}

\subsubsection{生平}

吳越忠遜王錢弘倧(928年-971年),字隆道,五代時期吳越國君主。

錢弘倧為文穆王錢元瓘第七子,忠献王錢弘佐之弟,孝献世子钱弘僔同母弟。诞生时,其父梦人献黄金一箧,故幼名万金。

後漢天福十二年(遼國會同十年,947年),錢弘佐去世,子尚年幼,因此在遗诏中命弟錢弘倧繼立。天福十二年六月丙寅,即王位于杭州吴越王宫天册堂。當時遼太宗耶律德光滅後晉,佔據中原,於是錢弘倧向其稱臣;不久遼軍退去,復對後漢稱臣,奉其正朔。

先前忠献王錢弘佐在位時,諸將驕橫,雖然擅權者旋遭誅殺,然而對下屬還是頗為寬大;而錢弘倧個性嚴厲堅定,等到繼位後,急欲改變這種情形,因此極力抑制將領。即位后不久在碧波亭检阅水師,內牙統軍使胡进思进谏说颁赏太厚,钱弘倧怒,掷笔于水中。胡进思因害怕被剷除,遂先發制人。

後漢天福十二年(947年)十二月三十日(陽曆為948年2月12日),錢弘倧在王宫中夜宴诸将。胡进思怀疑王将图己,于是率内牙亲兵戎服入宫,發動政變。錢弘倧被軟禁于义和院,胡進思假传钱弘倧命令,称钱弘倧中风,并迎錢弘倧之異母弟錢弘俶于私第,将其策立为王。

钱弘俶即位后,迁钱弘倧于太祖錢鏐故里衣锦軍,派匡武都头薛温保护,并嘱咐薛温:「自己没有杀兄的意思,一旦傳來类似的命令,必须拚死拒绝。」胡进思屡次请求钱弘俶杀钱弘倧,钱弘俶都拒绝,胡进思又假传王命要薛温杀钱弘倧,薛温也拒绝;胡进思自己派刺客方安等二人持兵器翻墙去杀钱弘倧,钱弘倧发现后闭门呼救,薛温率军赶来在庭院击杀方安二人。雖然胡進思不久後即病逝,但钱弘倧還是繼續被軟禁。

後周廣順元年(951年),錢弘俶把錢弘倧遷至東府越州(今浙江紹興),並為其興築宮室,以东府官物为供给。在西寝殿后的卧龙山为钱弘倧开辟花园,遍植花木。遇良辰美景,钱弘倧穿道士服,拥妓乐,旦暮登山赏景。每年元夜张灯于山谷,用油数千斤;七夕在山顶以绫罗结为彩楼,钱弘倧登山击鼓,声达于外,官吏报之,钱弘俶都不追究。以後每年逢年過節時的贈禮都非常豐厚。

北宋建立后,吴越国臣服北宋,为宋之先祖趙弘殷避讳,钱弘倧改名钱倧。宋太祖開寶年間,錢倧因病去世,享年四十四歲,以王禮葬之,赠諡忠遜王(一作諡讓王)。

\subsubsection{天福}

\begin{longtable}{|>{\centering\scriptsize}m{2em}|>{\centering\scriptsize}m{1.3em}|>{\centering}m{8.8em}|}
  % \caption{秦王政}\
  \toprule
  \SimHei \normalsize 年数 & \SimHei \scriptsize 公元 & \SimHei 大事件 \tabularnewline
  % \midrule
  \endfirsthead
  \toprule
  \SimHei \normalsize 年数 & \SimHei \scriptsize 公元 & \SimHei 大事件 \tabularnewline
  \midrule
  \endhead
  \midrule
  元年 & 947 & \tabularnewline
  \bottomrule
\end{longtable}


%%% Local Variables:
%%% mode: latex
%%% TeX-engine: xetex
%%% TeX-master: "../../Main"
%%% End:

%% -*- coding: utf-8 -*-
%% Time-stamp: <Chen Wang: 2021-11-01 15:40:55>

\subsection{钱弘俶钱俶\tiny(949-978)}

\subsubsection{生平}

钱俶(929年9月29日-988年10月7日)本名弘俶,因犯宋宣祖赵弘殷名讳,入宋後避諱,只称钱俶。字文德,小字虎子,五代十國時期吳越國文穆王錢元瓘第九子,吳越最后一位國王。

钱弘俶为吴越文穆王錢元瓘第九子,宝正四年八月二十四日(后唐天成四年,929年9月29日)生于杭州吴越王宫功臣堂,生母吴氏。累授内牙诸军指挥使、检校司空、检校太尉。开运四年出镇台州。忠逊王錢弘倧即位后,召钱弘俶返回杭州为同参相府事,居于南邸。

後漢天福十二年十二月三十(陽曆948年2月12日),吴越將領胡进思趁錢弘倧夜宴將吏时發動政變,錢弘倧被軟禁,錢弘俶被胡進思迎立為吳越王。乾祐元年正月乙卯,钱弘俶即位于杭州吴越王宫天宠堂。乾祐二年十月,后汉册封钱弘俶为匡圣广运同德保定功臣、东南面兵马都元帅、镇海镇东等军节度使、浙江东西等道管内观察处置使兼两浙盐铁制置发运营田等使、开府仪同三司、检校太师、兼中书令、杭州越州大都督、上柱国、吴越国王,食邑一万户、实封一千户。

钱弘俶嗣位三十餘年,期间恭事後漢、后周和北宋。后周广顺元年(951年)后周加封钱弘俶诸道兵马都元帅,加食邑一千户、实封三百户,翌年封天下兵马都元帅,加食邑二千户、实封五百户,改授推诚保德安邦致理忠正功臣。

后周显德三年(956年)正月,钱弘俶奉后周世宗诏,出兵攻南唐。显德五年四月,杭州城南失火,延烧内城,官府庐舍夷为平地,钱弘俶出居都城驿。后因大火即将烧及镇国仓,乃命官兵伐林木,火势方绝。尽管火灾惨重,吴越仍在当月向后周进贡绫、绢各二万匹,银子一万两;当年七月又向后周进贡银五千两、绢一万匹、龙舟一艘、天禄舟一艘,皆饰以银;十一月又进贡贺正钱一千贯、绢一千匹。周恭帝加封钱弘俶为崇仁昭德宣忠保庆扶天亮功臣。

宋建隆元年(960年)正月,后周殿前都点检赵匡胤發動陳橋兵變称帝,改国号为宋,遣使宣谕吴越。当年三月,钱弘俶改名为钱俶,以避赵匡胤之父赵弘殷名讳。建隆元年四月,赵匡胤封钱俶为天下兵马大元帅,加食邑一千户,实封五百户。乾德元年钱俶向宋朝进贡犀角、象牙各十株,香药十五万斤,珍珠、玳瑁器皿数百件,宋朝改赐钱俶为承家保国宣德守道忠贞恭顺忠臣,加食邑一千户、实封四百户。乾德二年加封食邑一千户、实封四百户。当年十一月宋朝伐后蜀,钱俶派孙承佑率军与宋会师。

开宝元年(968年),宋朝重新封钱俶为吴越国王,加食邑一千户、实封一百户,十二月再加封三千九百户,实封三百户。开宝四年加食邑二千户、实封六百户,改赐钱俶开吴镇海崇文耀武宣德守道功臣。开宝七年(974年)七月,宋太祖诏钱俶出兵协助宋朝攻打江南国(南唐),钱俶率军亲征,攻打常州城。翌年五月,宋朝赐钱俶守太师、尚书令,加食邑两千户、实封九百户。

吴越对宋谨遵事大之礼,世子钱惟濬曾四次入使宋朝,宋太祖亦屡赐钱俶御衣、剑佩、玉带、玉鞍、金银器、锦绣、金锁甲、御酒、马、羊、驼等物,及生辰礼物。

开宝九年(976年)正月,钱俶携王妃孙氏、世子钱惟濬自杭州出发,前往汴京觐见宋太祖,宋太祖命在礼贤宅为钱俶营建府第。钱俶贡通犀玉带、宝玉金器五千余件、上酒一千瓶、银十六万两、绢十一万匹、乳香五万斤,宋太祖赏赐钱俶黄金照匣、黄金钞锣、金二千两、银三万两、绢二万匹,赐以剑履上殿、诏书不名之礼。宋太祖宴钱俶于迎春苑,并许以“尽我一世”及“誓不杀钱王”之语。及钱俶辞行,宋太祖赏赐锦衣、玉带、玉鞍、玳瑁鞭,及金银锦彩二十余万、银装兵器八百余件,又赐王妃金器三百两、衣料二千匹、银二千两。又给钱俶黄袱一件,嘱曰“途中密视”。钱俶中途开袱检视,皆宋朝诸臣劝说扣留钱俶的奏章。当年五月、十一月,宋朝又两次加封钱俶食邑八千户、实封两千户。

开宝九年十月,宋太祖在宫中斧聲燭影而驾崩,其弟晋王赵光义即位,为宋太宗。太平兴国二年(977年,宋太宗即位,當年改元)三月封钱俶为尚书令、兼中书令、天下兵马大元帅。

太平兴国三年二月,钱俶自杭州出发,再次入宋朝觐。宋太宗赐宴于长春殿,命南唐后主违命侯李煜、南汉末帝恩赦侯劉鋹陪座。钱俶上吴越军甲器物名册,又乞辞天下兵马大元帅,宋太宗不许。五月乙酉,随同朝觐的吴越丞相崔仁冀劝钱俶上表纳土,否则祸患立至。钱俶遂当即上奏,献吴越国十三州、一军、八十六县、户五十五万六百八十、兵一十一万五千三十六于宋。

宋太宗升扬州为淮海国,虚封钱俶为淮海国王,食邑一万户、实封一千户,仍充天下兵马大元帅、守太师、尚书令、兼中书令,授宁淮镇海崇文耀武宣德守道功臣,赐剑履上殿。王世子钱惟濬为节度使兼侍中,其余各子亦授节度使、团练使、刺史等官。吴越幕僚宰相以下拜官者两千五百余人。

钱俶献土后居于东京礼贤宅,屡被宋太宗召入宫中赐宴、击球,并多次赏赐金银器、水晶、玛瑙、珊瑚、珍珠、龙涎香、贡茶、银、钱、绢等物,加食邑二万户、实封两千户。太平兴国四年,钱俶随宋太宗征北汉。雍熙元年改封汉南国王,加食邑两千户、实封两百户。

雍熙四年宋太宗改封钱俶为武胜军节度使、南阳国王,出居南阳,赐玉带、金唾壶;旋又加封为许王,加食邑一万户、实封两千户,赐安时镇国崇文耀武宣德守道功臣。端拱元年改封钱俶为邓王,加食邑一万户、实封三千户。

端拱元年八月二十四日(988年10月7日),钱俶六十大寿,宋太宗遣皇城使李惠、河州团练使王继恩至南阳,赐钱俶生辰礼物。钱俶与使者宴饮极欢。当天傍晚,钱俶在南阳住宅西轩命左右读《唐书》,又令子孙颂诗,忽然因风眩(脑卒中)发作,而于四漏时薨逝。或有怀疑其被毒杀者。

钱俶去世时的封号为安时镇国崇文耀武宣德守道中正功臣、武胜军节度使、开府仪同三司、守太师、尚书令兼中书令、使持节邓州诸军事、行邓州刺史、上柱国、邓王、食邑九万七千户、实封一万六千九百户、赐剑履上殿、诏书不名。宋太宗为其废朝七日,追封秦国王,赐谥号忠懿,葬于洛阳贤相里陶公原。宋真宗时,特诏追赠钱俶为尚父。有司请以礼贤宅为司天监,真宗不许。

钱俶好吟詠,自編其詩爲《政本集》,陶穀爲序,共十卷,今存一首“宫中作”。

\subsubsection{乾佑}

\begin{longtable}{|>{\centering\scriptsize}m{2em}|>{\centering\scriptsize}m{1.3em}|>{\centering}m{8.8em}|}
  % \caption{秦王政}\
  \toprule
  \SimHei \normalsize 年数 & \SimHei \scriptsize 公元 & \SimHei 大事件 \tabularnewline
  % \midrule
  \endfirsthead
  \toprule
  \SimHei \normalsize 年数 & \SimHei \scriptsize 公元 & \SimHei 大事件 \tabularnewline
  \midrule
  \endhead
  \midrule
  元年 & 948 & \tabularnewline\hline
  二年 & 949 & \tabularnewline\hline
  三年 & 950 & \tabularnewline
  \bottomrule
\end{longtable}

\subsubsection{广顺}

\begin{longtable}{|>{\centering\scriptsize}m{2em}|>{\centering\scriptsize}m{1.3em}|>{\centering}m{8.8em}|}
  % \caption{秦王政}\
  \toprule
  \SimHei \normalsize 年数 & \SimHei \scriptsize 公元 & \SimHei 大事件 \tabularnewline
  % \midrule
  \endfirsthead
  \toprule
  \SimHei \normalsize 年数 & \SimHei \scriptsize 公元 & \SimHei 大事件 \tabularnewline
  \midrule
  \endhead
  \midrule
  元年 & 951 & \tabularnewline\hline
  二年 & 952 & \tabularnewline\hline
  三年 & 953 & \tabularnewline
  \bottomrule
\end{longtable}

\subsubsection{显德}

\begin{longtable}{|>{\centering\scriptsize}m{2em}|>{\centering\scriptsize}m{1.3em}|>{\centering}m{8.8em}|}
  % \caption{秦王政}\
  \toprule
  \SimHei \normalsize 年数 & \SimHei \scriptsize 公元 & \SimHei 大事件 \tabularnewline
  % \midrule
  \endfirsthead
  \toprule
  \SimHei \normalsize 年数 & \SimHei \scriptsize 公元 & \SimHei 大事件 \tabularnewline
  \midrule
  \endhead
  \midrule
  元年 & 954 & \tabularnewline\hline
  二年 & 955 & \tabularnewline\hline
  三年 & 956 & \tabularnewline\hline
  四年 & 957 & \tabularnewline\hline
  五年 & 958 & \tabularnewline\hline
  六年 & 959 & \tabularnewline\hline
  七年 & 960 & \tabularnewline
  \bottomrule
\end{longtable}

\subsubsection{建隆}

\begin{longtable}{|>{\centering\scriptsize}m{2em}|>{\centering\scriptsize}m{1.3em}|>{\centering}m{8.8em}|}
  % \caption{秦王政}\
  \toprule
  \SimHei \normalsize 年数 & \SimHei \scriptsize 公元 & \SimHei 大事件 \tabularnewline
  % \midrule
  \endfirsthead
  \toprule
  \SimHei \normalsize 年数 & \SimHei \scriptsize 公元 & \SimHei 大事件 \tabularnewline
  \midrule
  \endhead
  \midrule
  元年 & 960 & \tabularnewline\hline
  二年 & 961 & \tabularnewline\hline
  三年 & 962 & \tabularnewline\hline
  四年 & 963 & \tabularnewline
  \bottomrule
\end{longtable}

\subsubsection{乾德}

\begin{longtable}{|>{\centering\scriptsize}m{2em}|>{\centering\scriptsize}m{1.3em}|>{\centering}m{8.8em}|}
  % \caption{秦王政}\
  \toprule
  \SimHei \normalsize 年数 & \SimHei \scriptsize 公元 & \SimHei 大事件 \tabularnewline
  % \midrule
  \endfirsthead
  \toprule
  \SimHei \normalsize 年数 & \SimHei \scriptsize 公元 & \SimHei 大事件 \tabularnewline
  \midrule
  \endhead
  \midrule
  元年 & 963 & \tabularnewline\hline
  二年 & 964 & \tabularnewline\hline
  三年 & 965 & \tabularnewline\hline
  四年 & 966 & \tabularnewline\hline
  五年 & 967 & \tabularnewline\hline
  六年 & 968 & \tabularnewline
  \bottomrule
\end{longtable}

\subsubsection{开宝}

\begin{longtable}{|>{\centering\scriptsize}m{2em}|>{\centering\scriptsize}m{1.3em}|>{\centering}m{8.8em}|}
  % \caption{秦王政}\
  \toprule
  \SimHei \normalsize 年数 & \SimHei \scriptsize 公元 & \SimHei 大事件 \tabularnewline
  % \midrule
  \endfirsthead
  \toprule
  \SimHei \normalsize 年数 & \SimHei \scriptsize 公元 & \SimHei 大事件 \tabularnewline
  \midrule
  \endhead
  \midrule
  元年 & 968 & \tabularnewline\hline
  二年 & 969 & \tabularnewline\hline
  三年 & 970 & \tabularnewline\hline
  四年 & 971 & \tabularnewline\hline
  五年 & 972 & \tabularnewline\hline
  六年 & 973 & \tabularnewline\hline
  七年 & 974 & \tabularnewline\hline
  八年 & 975 & \tabularnewline\hline
  九年 & 976 & \tabularnewline
  \bottomrule
\end{longtable}

\subsubsection{太平兴国}

\begin{longtable}{|>{\centering\scriptsize}m{2em}|>{\centering\scriptsize}m{1.3em}|>{\centering}m{8.8em}|}
  % \caption{秦王政}\
  \toprule
  \SimHei \normalsize 年数 & \SimHei \scriptsize 公元 & \SimHei 大事件 \tabularnewline
  % \midrule
  \endfirsthead
  \toprule
  \SimHei \normalsize 年数 & \SimHei \scriptsize 公元 & \SimHei 大事件 \tabularnewline
  \midrule
  \endhead
  \midrule
  元年 & 976 & \tabularnewline\hline
  二年 & 977 & \tabularnewline\hline
  三年 & 978 & \tabularnewline
  \bottomrule
\end{longtable}


%%% Local Variables:
%%% mode: latex
%%% TeX-engine: xetex
%%% TeX-master: "../../Main"
%%% End:



%%% Local Variables:
%%% mode: latex
%%% TeX-engine: xetex
%%% TeX-master: "../../Main"
%%% End:

%% -*- coding: utf-8 -*-
%% Time-stamp: <Chen Wang: 2019-12-26 09:44:15>


\section{楚\tiny(907-951)}

\subsection{简介}

楚(907年-951年)是五代十国时期的十国之一,湖南历史上唯一以湖南为中心建立的王朝。以其为马氏所建,史称马楚,又称南楚,都长沙。楚国创始人马殷,许州鄢陵(今河南省鄢陵)人。

楚全盛时,辖域包括潭、衡、永、道、郴、邵、岳、朗、澧、辰、溆、连、昭、宜、全、桂、梧、贺、蒙、富、严、柳、象、容共24州,下设武安、武平、静江等5个节镇,即今湖南省全境和广西壮族自治区大部、贵州省东部和广东省北部。楚自896年马殷受命节度使到951年楚国灭于南唐,共存世56年,在湖南历史上产生重要影响。通过战争,消灭了湖南境内割据势力,实现了湖南的统一。

马殷政权时期,政治上采取“上奉天子、下抚士民”、内靖乱军、外御强藩等政策,使百姓获得了一个相对安定的环境。经济上,采取兴修水利、奖励农桑、发展茶业、提倡纺织、通商中原等措施,使社会经济得到了较快的发展。

马殷本是唐朝末年军阀孙儒部将,孙儒败亡后,马殷助孙儒余部龙骧指挥使刘建锋夺取武安军。后刘建锋被杀,马殷被推举接手武安军。

896年,唐朝朝廷任马殷为武安军节度使,奠定了他在湖南立足的根基。

907年,后梁封马殷为楚王,都潭州,号长沙府。927年,后唐天成二年,正式册封马殷为楚国王,楚国正式成立。马殷仿效朝廷体制,改潭州为长沙府,作为国都,并在长沙城内修宫殿,置百官,建立了一个名符其实的独立王国,成为五代时期十个封建割据国家之一。

930年马殷死,马殷次子马希声继位。932年马希声死,马殷子馬希範继位。947年马希範死,马希广继位。950年马希萼攻打长沙,马希广兵败被杀。于是马希萼自立楚王。

951年11月,南唐乘马楚内乱,派大将边镐率军进入楚国,占领长沙,楚灭亡。南唐还未站稳脚跟,马殷旧将刘言又起兵击败了南唐军,继续据有湖南。947年到951年的这段争夺王位的战争,被称为众驹争槽。

952年,王进逵杀刘言控制湖南。955年,部将潘叔嗣又杀了王进逵;潭州军府事周行逢又进军朗州杀了潘叔嗣,湖南全境遂为周行逢所控制。962年,行逢死,子周保权继位,手下大将张文表起兵反叛。此时赵匡胤已发动陈桥兵变即帝位,建立宋朝。周保权一边平叛一边求救于宋,虽然在宋军到来前败杀张文表,但宋军也趁机挥军南下攻占潭州。963年,湖南完全并入宋的版图。

据史料记载,马殷“土宇既广,乃养士息民”,由于政治上采取上奉天子、下抚士民的保境息民政策,同时奉行奖励农桑、发展茶叶、倡导纺织、重视商业贸易。马楚利用湖南地处南方各政权中心的地理优势,大力发展与中原和周边的商业贸易,采取免收关税,鼓励进出口贸易,招徕各国商人。《十国春秋·楚武穆王世家》载:“是时王关市无征,四方商旅闻风辐。”

茶税为当时楚国主要税收来源,因此政府每年税收“凡百万计”。为促进茶叶的生产与贸易,马楚政权采取“令民自造茶”、“听民售茶北客”的宽松政策,让百姓自己制造茶叶“以通商旅”。同时,马楚政权全国各地设置商业货栈(回图务),组织商人收购茶叶(茶商号“八床主人”),销往中原地区的商人,换回战马和丝织品。

由于马楚政权重商政策,那时的潭州已成为南方最大的茶市,城市化水平有了较大的发展。当时手工业和矿冶业也比较发达,其时采取“命民输税者皆以帛代钱”后“民间机抒大盛”,长沙棉纺业也始于马楚时期,其时楚地已种棉,故有胡三省之“木棉,今南方多有焉。于春中作畦种之,至夏秋之交结实,至秋丰其实之外皮四裂,中踊出自如绵。土人取而纺之,织之以布,细密厚暖,宜以御冬。”矿冶业方面,楚时潭州境内丹砂矿的开采风行一时,据说州的东境山崩,“涌出丹砂,委积如丘陵”,主要用于作为涂料之用,楚王马希範丹砂涂壁,“凡用数十万斤”。

为了发展商业,马殷采纳大臣高郁的建议,铸造铅、铁钱币在境内流通。由于铅铁钱币笨重,携带不便,在南唐等国又被禁用,因此商旅出境外贸易,大都“无所用钱”,往往销货后又在楚就地购买大量产品销往外地,这样楚地境内生产的产品通过“易天下百货”而变得富饶。当时楚国的茶叶和粮食等为与周边的主要贸易产品。

楚国发行的钱币主要为小平钱的铅钱“开元通宝”和折十钱的铁钱“乾封泉宝”。另外还有铜钱“天策府宝”、“乾元重宝”等。铜钱的数量十分稀少,天策府宝为古泉五十名珍之一。


%% -*- coding: utf-8 -*-
%% Time-stamp: <Chen Wang: 2021-11-01 15:41:09>

\subsection{武穆王马殷\tiny(907-930)}

\subsubsection{生平}

楚武穆王马殷(852年-930年12月2日),字霸图,许州鄢陵(今河南鄢陵)人,五代十国时期南楚开国君王。

马殷早年家贫,以木匠为業,后投入秦宗权军中,属孙儒部下,随孙儒渡淮攻下广陵(今江苏扬州东北)。唐僖宗光启三年(887年),秦宗权派其弟秦宗衡为主将,孙儒为副将,将兵三万,南下渡过淮河,同杨行密争夺扬州。不久,孙儒杀秦宗衡,自立为帅,号“土团白条军”。

大顺二年(891年),马殷受命率军击败杨行密部将田頵,随刘建锋镇守常州,后被调往宣州(今属安徽)参与围攻杨行密。景福元年(892年),孙儒兵败,战死军中,刘建锋和马殷率残军7000人逃走。其后,马殷作为刘建锋的先锋,一路攻占洪州、潭州等城。乾寧三年(896年)四月,刘建锋为部将所杀,马殷被众将推举为主,唐朝任其为湖南留後、判湖南軍府事。光化元年(898年)又授为武安军节度使。天復三年(903年),杨行密派刘存攻打鄂州(今武昌)的杜洪,马殷派秦彦晖、许德勋以舟兵救之,但同时又派许德勋与武贞军节度使雷彦威部将欧阳思联手,趁荆南节度使成汭出救杜洪之机洗劫其军府江陵,导致成汭军心涣散,最终被杨行密部将李神福所败而自杀。杜洪败死,刘存等攻马殷,马殷便沿江布防,埋下伏兵,双方激战甚烈,刘存战死,马殷夺下岳州(今湖南岳阳)。

后梁开平元年(907年),朱温封马殷为楚王,都于潭州(今长沙),开平四年(910年)六月加封“天策上将军”。后唐灭后梁后,明宗天成二年(927年)六月又册封为楚國王,同年八月冊封使至,马殷乃建立楚国,立宮殿、置百官,以潭州为都城,改名长沙府,使用后唐年号。

马殷在位期间,采取“上奉天子,下奉士民”的策略即保境安民的政策,自其于897年占据湖南后,很少主动对外交战,之后与杨吴的几次战争也是对方先发动进攻的,对于北边的荆南,也只进行了相当有限的战争。马殷对内采取措施发展农业生产,减轻百姓的赋税,“不征商旅,由是四方商旅輻凑”。他下令百姓可以用帛代替钱交纳赋税,减少了官吏加重赋税的机会,並且促进了湖南的桑蚕业的发展。因而楚国的经济非常繁荣。

長興元年十一月十日(930年12月2日),马殷去世,時年79歲,遺命諸子兄弟相繼。后唐明宗罷朝三天,下诏赐马殷谥号武穆王。次子马希声继其位,遵从父亲的遗命,不再称楚国,而是降低规格,恢复了节度使的称号,将楚政权延续了二十一年。

\subsubsection{天成}

\begin{longtable}{|>{\centering\scriptsize}m{2em}|>{\centering\scriptsize}m{1.3em}|>{\centering}m{8.8em}|}
  % \caption{秦王政}\
  \toprule
  \SimHei \normalsize 年数 & \SimHei \scriptsize 公元 & \SimHei 大事件 \tabularnewline
  % \midrule
  \endfirsthead
  \toprule
  \SimHei \normalsize 年数 & \SimHei \scriptsize 公元 & \SimHei 大事件 \tabularnewline
  \midrule
  \endhead
  \midrule
  元年 & 927 & \tabularnewline\hline
  二年 & 928 & \tabularnewline\hline
  三年 & 929 & \tabularnewline\hline
  四年 & 930 & \tabularnewline
  \bottomrule
\end{longtable}


%%% Local Variables:
%%% mode: latex
%%% TeX-engine: xetex
%%% TeX-master: "../../Main"
%%% End:

%% -*- coding: utf-8 -*-
%% Time-stamp: <Chen Wang: 2021-11-01 15:41:19>

\subsection{衡阳王馬希聲\tiny(930-932)}

\subsubsection{生平}

馬希聲(898年-932年8月15日),字若訥,五代十國時期南楚國君主,是楚王馬殷的次子,馬殷在位時任武安節度副使,为其内定的接班人。妻杨氏。

929年,马希声听信后唐和荆南离间,排挤谋主高郁,高郁因而口出怨言,马希声恼怒,没有告知马殷,就以谋反为由诛杀高郁亲党。马殷虽然为高郁之死大恸,却没有处罚马希声。後唐明宗長興元年(930年),馬殷去世,馬希聲繼立,不稱王,只稱藩鎮。後唐則任命馬希聲武安、靜江節度使,兼中書令。馬希聲聽說朱全忠喜歡吃雞,很是羨慕,因此繼位後,每天都殺五十隻雞做菜;服喪期間也沒有哀傷的表情,馬殷要下葬時,還吃了好幾盤雞肉。

長興三年(932年)馬希聲去世,弟馬希範繼立。馬希聲在位時並未稱王,只在死後被追封為衡陽王。

\subsubsection{长兴}

\begin{longtable}{|>{\centering\scriptsize}m{2em}|>{\centering\scriptsize}m{1.3em}|>{\centering}m{8.8em}|}
  % \caption{秦王政}\
  \toprule
  \SimHei \normalsize 年数 & \SimHei \scriptsize 公元 & \SimHei 大事件 \tabularnewline
  % \midrule
  \endfirsthead
  \toprule
  \SimHei \normalsize 年数 & \SimHei \scriptsize 公元 & \SimHei 大事件 \tabularnewline
  \midrule
  \endhead
  \midrule
  元年 & 930 & \tabularnewline\hline
  二年 & 931 & \tabularnewline\hline
  三年 & 932 & \tabularnewline
  \bottomrule
\end{longtable}


%%% Local Variables:
%%% mode: latex
%%% TeX-engine: xetex
%%% TeX-master: "../../Main"
%%% End:

%% -*- coding: utf-8 -*-
%% Time-stamp: <Chen Wang: 2021-11-01 15:41:28>

\subsection{文昭王馬希範\tiny(932-947)}

\subsubsection{生平}

楚文昭王馬希範(899年-947年5月30日),字寶規,五代十國時期南楚國君主,是楚王馬殷的四子,馬希聲之弟,與馬希聲同年同月同日生。

後唐明宗長興三年(932年)馬希聲去世,因之前馬殷去世時遺命兄終弟及,因此群臣迎接時任鎮南節度使的馬希範繼位。後唐則任命馬希範為武安、武平節度使,兼中書令。後唐明宗清泰元年(934年),馬希範被封為楚王,之後又被封為天策上將軍。

馬希範好學,很會做詩,然而非常奢侈,其妻彭夫人“貌陋而治家有法”,希範“惮之”。彭夫人死后,马希範“始纵声色,为长夜之饮”。門戶檻杆都用金玉裝飾,塗抹牆壁的丹砂用量數十萬斤,常與子弟及部屬在內遊玩宴會。原本楚地多產金銀,而販賣茶葉的利潤更多,因此十分富庶,但是在無節制的揮霍下,只好向人民加稅,又賣官鬻爵,規定捐錢可贖罪刑,人民困苦不堪。

後漢高祖天福十二年(947年),馬希範去世,諡文昭王,弟馬希廣繼立。

\subsubsection{长兴}

\begin{longtable}{|>{\centering\scriptsize}m{2em}|>{\centering\scriptsize}m{1.3em}|>{\centering}m{8.8em}|}
  % \caption{秦王政}\
  \toprule
  \SimHei \normalsize 年数 & \SimHei \scriptsize 公元 & \SimHei 大事件 \tabularnewline
  % \midrule
  \endfirsthead
  \toprule
  \SimHei \normalsize 年数 & \SimHei \scriptsize 公元 & \SimHei 大事件 \tabularnewline
  \midrule
  \endhead
  \midrule
  元年 & 932 & \tabularnewline\hline
  二年 & 933 & \tabularnewline
  \bottomrule
\end{longtable}

\subsubsection{应顺}

\begin{longtable}{|>{\centering\scriptsize}m{2em}|>{\centering\scriptsize}m{1.3em}|>{\centering}m{8.8em}|}
  % \caption{秦王政}\
  \toprule
  \SimHei \normalsize 年数 & \SimHei \scriptsize 公元 & \SimHei 大事件 \tabularnewline
  % \midrule
  \endfirsthead
  \toprule
  \SimHei \normalsize 年数 & \SimHei \scriptsize 公元 & \SimHei 大事件 \tabularnewline
  \midrule
  \endhead
  \midrule
  元年 & 934 & \tabularnewline
  \bottomrule
\end{longtable}

\subsubsection{清泰}

\begin{longtable}{|>{\centering\scriptsize}m{2em}|>{\centering\scriptsize}m{1.3em}|>{\centering}m{8.8em}|}
  % \caption{秦王政}\
  \toprule
  \SimHei \normalsize 年数 & \SimHei \scriptsize 公元 & \SimHei 大事件 \tabularnewline
  % \midrule
  \endfirsthead
  \toprule
  \SimHei \normalsize 年数 & \SimHei \scriptsize 公元 & \SimHei 大事件 \tabularnewline
  \midrule
  \endhead
  \midrule
  元年 & 934 & \tabularnewline\hline
  二年 & 935 & \tabularnewline\hline
  三年 & 936 & \tabularnewline
  \bottomrule
\end{longtable}

\subsubsection{天福}

\begin{longtable}{|>{\centering\scriptsize}m{2em}|>{\centering\scriptsize}m{1.3em}|>{\centering}m{8.8em}|}
  % \caption{秦王政}\
  \toprule
  \SimHei \normalsize 年数 & \SimHei \scriptsize 公元 & \SimHei 大事件 \tabularnewline
  % \midrule
  \endfirsthead
  \toprule
  \SimHei \normalsize 年数 & \SimHei \scriptsize 公元 & \SimHei 大事件 \tabularnewline
  \midrule
  \endhead
  \midrule
  元年 & 936 & \tabularnewline\hline
  二年 & 937 & \tabularnewline\hline
  三年 & 938 & \tabularnewline\hline
  四年 & 939 & \tabularnewline\hline
  五年 & 940 & \tabularnewline\hline
  六年 & 941 & \tabularnewline\hline
  七年 & 942 & \tabularnewline\hline
  八年 & 943 & \tabularnewline\hline
  九年 & 944 & \tabularnewline
  \bottomrule
\end{longtable}

\subsubsection{开运}

\begin{longtable}{|>{\centering\scriptsize}m{2em}|>{\centering\scriptsize}m{1.3em}|>{\centering}m{8.8em}|}
  % \caption{秦王政}\
  \toprule
  \SimHei \normalsize 年数 & \SimHei \scriptsize 公元 & \SimHei 大事件 \tabularnewline
  % \midrule
  \endfirsthead
  \toprule
  \SimHei \normalsize 年数 & \SimHei \scriptsize 公元 & \SimHei 大事件 \tabularnewline
  \midrule
  \endhead
  \midrule
  元年 & 944 & \tabularnewline\hline
  二年 & 945 & \tabularnewline\hline
  三年 & 946 & \tabularnewline
  \bottomrule
\end{longtable}


%%% Local Variables:
%%% mode: latex
%%% TeX-engine: xetex
%%% TeX-master: "../../Main"
%%% End:

%% -*- coding: utf-8 -*-
%% Time-stamp: <Chen Wang: 2019-12-26 09:47:28>

\subsection{马希广\tiny(947-950)}

\subsubsection{生平}

馬希廣(?-951年1月25日),字德丕,五代十國時期南楚國君主,楚王馬殷第三十五子,馬希範一母同胞之弟,個性謹慎溫順,馬希範對他疼愛有加。

後漢高祖天福十二年(947年),馬希範去世,將領排除馬希範諸弟中年齡最長的馬希萼,而擁護馬希廣繼立,後漢封馬希廣天策上將軍、楚王,因而馬希廣、希萼之弟馬希崇就以馬希廣之繼位違反父親兄終弟及的遺命挑撥馬希萼。

後漢隱帝乾祐二年(949年),時任武貞(武平)節度使的馬希萼叛,率軍南下進攻南楚都城潭州(今湖南長沙),馬希萼戰敗,馬希廣以不願傷其兄為由,放棄追擊。乾祐三年(950年)馬希萼結合蠻族軍再度攻擊馬希廣,並向南唐稱臣,請求發兵攻潭州。馬希廣派軍討伐馬希萼,大敗。馬希萼遂與蠻族軍兵圍潭州,守將許可瓊投降,潭州陷落,馬希廣夫妇被擒。马希萼为了避免后患,处死马希广。马希广临死还在背诵佛经。马希广夫人被杖杀于闹市。马希广有子藏在慈堂,后不知所终。马军指挥使李彦温与战棹指挥使刘彦瑫奉马希广其余诸子去袁州奔南唐,马希广诸子后在南唐都城金陵去世。

\subsubsection{天福}

\begin{longtable}{|>{\centering\scriptsize}m{2em}|>{\centering\scriptsize}m{1.3em}|>{\centering}m{8.8em}|}
  % \caption{秦王政}\
  \toprule
  \SimHei \normalsize 年数 & \SimHei \scriptsize 公元 & \SimHei 大事件 \tabularnewline
  % \midrule
  \endfirsthead
  \toprule
  \SimHei \normalsize 年数 & \SimHei \scriptsize 公元 & \SimHei 大事件 \tabularnewline
  \midrule
  \endhead
  \midrule
  元年 & 947 & \tabularnewline
  \bottomrule
\end{longtable}

\subsubsection{乾佑}

\begin{longtable}{|>{\centering\scriptsize}m{2em}|>{\centering\scriptsize}m{1.3em}|>{\centering}m{8.8em}|}
  % \caption{秦王政}\
  \toprule
  \SimHei \normalsize 年数 & \SimHei \scriptsize 公元 & \SimHei 大事件 \tabularnewline
  % \midrule
  \endfirsthead
  \toprule
  \SimHei \normalsize 年数 & \SimHei \scriptsize 公元 & \SimHei 大事件 \tabularnewline
  \midrule
  \endhead
  \midrule
  元年 & 948 & \tabularnewline\hline
  二年 & 949 & \tabularnewline\hline
  三年 & 950 & \tabularnewline
  \bottomrule
\end{longtable}


%%% Local Variables:
%%% mode: latex
%%% TeX-engine: xetex
%%% TeX-master: "../../Main"
%%% End:

%% -*- coding: utf-8 -*-
%% Time-stamp: <Chen Wang: 2019-12-26 09:47:50>

\subsection{恭孝王\tiny(950-951)}

\subsubsection{生平}

楚恭孝王馬希萼(900年-953年),五代十國時期南楚國君主,楚武穆王馬殷之子,馬希聲、馬希範之弟,馬希廣之兄。馬希範在位時任武貞(武平)節度使,鎮守朗州(今湖南常德)。

後漢高祖天福十二年(947年),馬希範去世,將領排除馬希範諸弟中年齡最長的馬希萼,而擁護馬希廣繼立,因而馬希廣、希萼之弟馬希崇就以馬希廣之繼位違反父親兄終弟及的遺命挑撥馬希萼。

後漢隱帝乾祐二年(949年),馬希萼叛變,率軍南下進攻南楚都城潭州(今湖南長沙),馬希萼戰敗,馬希廣以不願傷其兄為由,放棄追擊。乾祐三年(950年)馬希萼結合蠻族軍再度攻擊馬希廣,並向南唐稱臣,請求發兵攻潭州。馬希廣派軍討伐馬希萼,大敗。馬希萼遂自稱順天王,並與蠻族軍兵圍潭州,守將許可瓊投降,佔領潭州,擒馬希廣。不久,將馬希廣賜死。

馬希萼當初因認為後漢偏袒馬希廣而轉向南唐稱臣,故一改馬殷以來臣服中原的態度,未待冊封即自稱天策上將軍、武安、武平、靜江、寧遠等軍節度使、楚王。登位後,志得意滿,殺戮報復,縱酒荒淫,將事務都交給馬希崇,然而馬希崇也只是交給下屬而已,因此政事混亂,又對士卒不加賞賜,遂軍心思變。

後周太祖廣順元年(951年),王逵、周行逢首先佔據朗州,擁護馬殷長子馬希振之子馬光惠當節度使。數月後,徐威等將領兵變,擁護馬希崇為武安留後,馬希萼被囚禁於衡山縣。馬希萼抵衡山後,復被廖偃、廖匡凝、彭師暠等擁護稱衡山王。不久,南楚為南唐所滅,馬希萼被南唐任命為江南西道觀察使,仍封楚王。其後在入朝的時候,被南唐元宗李璟留下,三年後在南唐都城金陵(今江蘇南京)去世寿五十四岁,諡恭孝王。

\subsubsection{保大}

\begin{longtable}{|>{\centering\scriptsize}m{2em}|>{\centering\scriptsize}m{1.3em}|>{\centering}m{8.8em}|}
  % \caption{秦王政}\
  \toprule
  \SimHei \normalsize 年数 & \SimHei \scriptsize 公元 & \SimHei 大事件 \tabularnewline
  % \midrule
  \endfirsthead
  \toprule
  \SimHei \normalsize 年数 & \SimHei \scriptsize 公元 & \SimHei 大事件 \tabularnewline
  \midrule
  \endhead
  \midrule
  元年 & 950 & \tabularnewline\hline
  二年 & 951 & \tabularnewline
  \bottomrule
\end{longtable}


%%% Local Variables:
%%% mode: latex
%%% TeX-engine: xetex
%%% TeX-master: "../../Main"
%%% End:



%%% Local Variables:
%%% mode: latex
%%% TeX-engine: xetex
%%% TeX-master: "../../Main"
%%% End:

%% -*- coding: utf-8 -*-
%% Time-stamp: <Chen Wang: 2019-12-26 09:50:52>


\section{闽\tiny(909-945)}

\subsection{简介}

闽(閩東語:Mìng;閩南語:Bân;909年-945年),五代十国的十國之一,由閩太祖王審知於909年時所建立。933年,閩惠宗王廷鈞稱帝,定國號大閩,是繼閩越國後福建第二次獨立於中原政權。943年,富沙王王延政在建州自立反抗景宗王延羲,並一度將國號改為大殷。閩國內亂造成最後被南唐所滅。閩國前後共歷經6位君王的統治,享國37年;而若從王潮攻佔福州當上福建節度使開始算起,王閩皇族統治福建共長達55年。

唐昭宗景福二年(893年)王潮、王審邽、王審知兄弟攻占福州,并逐渐据有福建全地。王潮被唐朝廷授职为福建观察使,不久升为威武军节度使。乾宁四年十二月(陽曆為898年1月)王潮卒,遺命以王審知繼位。

审知受封为琅琊王,后梁太祖开平三年(909年),王审知受后梁封为闽王,都长乐(今福建省福州市)。王审知称臣后梁,交好邻国,提倡节俭,减轻稅金、勞役,以保境息民为立国方针,他还建立学校,奖励通商。在他在位期间,闽地的经济、文化都得以迅速发展。

925年王审知死后,其子王延翰即位,926年被兄弟王延禀、王延钧杀害。

926年,王延钧继任闽王。后唐长兴二年(931年)王延钧殺王延稟。

后唐长兴四年(933年)王延钧称帝,建都长乐(今福建福州),国号闽,年号龙启。

935年,王延钧在政变中为其子王继鹏所弑。继鹏继位,改名王昶。

939年,王昶被连重遇所弑,重遇立王曦(原名王延羲)为闽王。王曦在位时猜忌其弟建州刺史王延政,二人结怨,导致940年王曦进攻建州,开始了闽国内乱。941年,王曦先后称大闽皇和大闽皇帝。943年,王延政于建州称帝,改国号为殷,年号天德。

944年,朱文进、连重遇杀王曦,朱文进自立为闽王,王延政出兵讨伐,945年朱文进、连重遇被部下所杀。王延政恢复国号为“闽”。同年,南唐进攻闽国,王延政战败,闽国灭亡。

946年,闽国旧将留从效驱逐了南唐在泉州、漳州的驻军,但仍向南唐稱臣,留及后继者占有泉、漳二州直至北宋建国之后。

西元881年(中和元年),王審潮、王審邽與王審知三兄弟加入王緒的麾下,王審潮擔任軍正。885年,王審潮推翻王緒,成為軍隊的領導人。886年(光啟二年),王審潮攻克泉州,畏時任福建觀察使的陳巖威名,向其投降,陈岩上表授予王審潮泉州刺史的官職。893年,陳巖死後,王審潮命令三弟王審知與堂弟王彥復率軍攻佔福州,並逐漸控制了整個福建,因而受唐朝政府封為福建觀察使,不久又晉升威武軍節度使。898年,王審潮去世,遺命跳過二弟王審邽與其四位兒子,將節度使之職傳給三弟王審知。王審知繼承節度使後,又受唐昭宗敕封為琅琊王,不久朱溫篡位建立後梁,唐朝滅亡。

西元909年,後梁太祖敕封王審知為「閩王」,閩國正式開國,建都長樂府(今福州市)。王審知鑒於開國初期應當休息養民,因而實行黄老治術的統治方針;政治上稱臣於中原的後梁政權,並與鄰近國家政權交好、聯姻,而經濟民生上則是提倡節儉,減輕稅金、勞役。另外他還建立學校,獎勵通商,使得閩國的經濟、文化在他在位時期得以迅速發展。西元925年,王審知薨,廟號太祖,葬宣陵,諡昭武孝皇帝。

西元925年,嫡長子王延翰在太祖薨後登基,隔年自稱「大閩國王」,但仍舊稱臣於後唐中原政權。不久,後唐莊宗遭弒,中原政權內部陷入混亂,王延翰推崇閩越王騶無諸的建國事蹟,並以繼承閩越國為由建國稱王,但國家的年號依然使用後唐的天成[註 1]。王延翰個性荒淫無道、殘忍凶暴,經常為了貪圖美色而收括民女;泉州刺史王延鈞與建州刺史王延禀遂以此為藉口謀反弒君,殺死了兄長王延翰。

西元945年,王延政投降南唐後,閩國滅亡。946年,割據福州的李仁達派其弟李弘通進攻泉州,閩南當地將領留從效趁此機會罷黜了身為閩國皇族的泉州刺史王繼勳,改由自己出任,次年更佔領漳州。不久将南唐军队赶出泉漳二州。949年,南唐中主李璟不得不任命割據泉州的留從效為清源軍節度使,最後晉封晉江王,為閩南真正的統治者。南唐歸降北宋後,留從效亦請求歸屬,宋朝朝廷同意之,但清源軍後來經過留紹鎡、張漢思與陳洪進等節度使的統馭,持續割據閩南泉漳直至978年奉表正式獻出泉、漳二州,前後共歷經33年。


%% -*- coding: utf-8 -*-
%% Time-stamp: <Chen Wang: 2019-12-26 09:52:35>

\subsection{太祖\tiny(909-925)}

\subsubsection{生平}

閩太祖王審知(閩東語:Uòng Sīng-dĭ;闽南语:Ông Sím-ti;862年-925年12月30日),表字信通,一字详卿,庙号太祖(閩東語:Mìng Tái-cū;閩南語:Bân Thài-chó͘),是五代十國時期閩國開國國王,909年至925年在位。

淮南道光州固始(今河南省固始縣)人,為王恁第三子,也是王潮與王審邽之弟。出身貧苦,後在唐末民變期間,與兩位兄長一起加入王緒的軍隊,隨之轉戰福建。其兄王潮被唐昭宗任命為福建觀察使後,他也獲封觀察副使,後福建觀察使升格為威武軍節度使,898年,繼承兄位。後梁篡唐後,後梁太祖於909年冊封王審知為閩王,正式建立閩國。

王審知出身貧苦,故能節儉自處,統治福建期間省刑惜費,輕徭薄賦,與民休息,儘量避免戰爭,並與中原王朝保持朝貢關係。另一方面,他注重教育,吸納中原逃離戰亂的人才,又積極發展海外貿易,使福建的經濟和文化得到很大發展。

同光三年十二月辛未(925年12月30日),王審知在福州逝世,其長子王延翰繼位。後唐得知後賜諡忠懿。王延翰稱大閩國王之後,諡他為昭武王。王延鈞稱大閩皇帝後,再改諡號昭武孝皇帝,廟號太祖,陵號為宣陵。

因王審知三兄弟對福建發展貢獻很大,福建人尊稱王審知為「開閩尊王」、「開閩聖王」或「忠惠尊王」;尊其長兄王潮為「威武尊王」「广武尊王」、次兄王審邽為「泉安尊王」,視為鄉土神明供奉,合稱開閩三王。946年(開運三年),南唐佔領福州之後,為紀念王審知的德政,將其府邸改建為閩王祠,對他進行祭祀,即今日福州市鼓樓區的忠懿閩王祠。

據《十国春秋》記載:王審知是秦朝名将王翦,东晋王导的后代,為瑯琊王氏士族。其五代祖名王曄,為固始令。因「民愛其仁」,被當地百姓挽留,最終定居於固始。父親王恁,為當地一個富有的農民。王恁生王潮(又名王審潮)、王審邽、王審知三個兄弟。

唐僖宗在位期間,蜀地盜賊起兵叛亂。壽州(今安徽壽縣)的屠夫王緒與妹夫劉行全也起兵響應,佔據江淮一帶,自稱將軍,不久攻取光州。當時王審知的長兄王潮是固始縣長史,王潮、王審邽和王審知三兄弟以才氣知名,邑人號曰「三龍」。王緒將他們擄走後,委任王潮為軍正,主管糧草之事,得到部下的擁護。三兄弟皆受重用。根據《十國春秋》的說法,王審知「身長七尺六寸,紫色方口隆準,常乘白馬,軍中號白馬三郎,所居處恒有紫氣幕其上」,時人認為這是大貴之相。

汝南節度使秦宗權封王緒為光州刺史,要求王緒率兵一起討伐黃巢,王緒按兵不動。於是在中和五年(885年),秦宗權率兵攻打王緒,王緒率部眾逃往福建,攻汀州、漳州。

王緒軍中缺糧,下令不准部將攜帶老人孺子,違者斬首。當時王審知三兄弟奉母親董氏一起行軍,王緒命其棄母,三兄弟請求與母同死,王緒只能赦免。當時有方士對王緒預言,軍中會出現推翻他的人,因此王緒對諸將多疑猜忌,身材魁梧、才能雄傑者多被他找藉口斬殺。王潮日夜憂患,遂說服一位前鋒將軍發動兵變,將王緒擒獲捆綁之,王緒不堪其辱,自殺身亡。王潮聲稱要推戴這位將領為主,遭到推辭,但這位將軍取了一把寶劍插在地上,權作神位,讓諸將輪流跪拜,向上天祈禱,希望神明能夠指點迷津,相傳當輪到王審知跪拜的時候,寶劍從地上躍起。眾人便認為是神諭,推戴王審知為主帥。王審知將主帥之位讓給了兄長王潮,自己擔任副帥。

光啓元年八月,泉州的張延魯等人聲稱泉州刺史廖彥若貪婪殘暴,聽說王潮治軍有法,要求前來討伐。王潮便率軍圍泉州。翌年八月,攻陷泉州,殺廖彥若,據有其地。王潮歸附新上任的福建觀察使陳巖,陳巖表奏王潮為泉州刺史。王潮三兄弟治理泉州期間,「招懷離散,均賦繕兵,吏民悅之」。

大順二年(891年),陳巖病危,作書予王潮,希望他來福州授以軍政。王潮未至,陳巖即病逝。陳巖的妻舅福州護軍使范暉,自稱留後。范暉「驕侈,失眾心」,陳巖的舊將多與王潮友善,聲稱范暉可取。王潮便派從弟王彥復為都統、三弟王審知為都監,攻打福州,「彌年不下」。范暉向威勝節度使董昌求援。董昌派溫、台、婺州之兵五千人救援。王審知等人要求班師,被王潮拒絕;又請求王潮親自前來督戰,王潮回覆稱:「兵盡,益兵;將盡,益將;將盡,則吾至矣。」於是王審知等人並立攻城,最終在景福二年(893年)攻克福州。范暉棄城逃跑,被部將殺死。汀州刺史鐘全慕舉州來降,福建各地勢力紛紛歸附。乾寧年間,唐昭宗任命王潮為福建觀察使,王審知為副觀察使。

根據《十國春秋》的記載,王潮擔元帥的時候,曾請占卜師給自己的兩個弟弟算命,得到的結論是「一人勝一人」。當時王審知就在王潮身邊,渾身大汗而退。王潮在任期間執法嚴明,即便是王審知「有過」,王潮也「輙加捶楚,不以為嫌」。王審知也毫無怨色。王潮臨終之前,認為自己的兒子都沒有王審知有才能,便捨弃了自己的兒子,任命王審知為「權知軍府事」。王潮病逝後,王審知推戴次兄王審邽為泉州刺史,但王審邽認為王審知有功,於是推辭不受。王審知便自稱福建留後,上表於唐朝朝廷。光化元年(898年)春三月,被唐朝冊封為威武軍節度留後、檢校太保、刑部尚書。冬十月,又授金紫光祿大夫、尚書省右僕射、威武軍節度使。三年春二月,加同中書門下平章事;不久又改授光祿大夫、檢校司空、特進、檢校司徒。天復二年(902年),授賜武庫戟十二枝,立於私邸大門之前。天祐元年(904年)夏四月,唐朝派遣右拾遺翁承贊前往福州,加王審知為檢校太保,封琅琊郡王,食邑四千戶,實封一百戶。朱晃建立後梁以後,於開平三年(909年)封為閩王,加中書令,升福州為大都督府,正式建立閩國。後唐建立後,後唐莊宗加王審知為檢校太師守中書令。

王審知在位期間謹事四鄰,儘量地避免戰爭。開平三年(909年)時,楊吳遣使張知遠來聘,因其舉止倨慢而被王審知斬首。因此閩國與楊吳關係不佳,但在位期間兩國並未發生軍事衝突。王審知於貞明二年(916年)將女兒嫁給吳越國國王錢鏐之子錢傳珦(錢元珦)。翌年,王審知命次子王延鈞娶南漢君主劉龑(劉巖)之女。貞明四年(918年)夏六月,吳鎮南軍節度使、虔州行營招討使劉信率兵攻打虔州,百勝軍防禦使譚全播向王審知與楚王馬殷求救。王審知出兵鄠都救援,但在秋八月得知南楚戰敗後,便率軍班師。同光二年(924年)夏四月,劉巖領兵犯境,屯兵於汀、漳之境。王審知前去攻打,敗績。

經過王審知的努力,在戰亂的五代十國時期,福建相對來說比較安定,逃難的中原人相繼遷入福建。史載王審知「為人儉約,好禮下士」,「王雖據有一方,府舍卑陋,未常葺;居,恒常躡麻屢;寬刑薄賦,公私富實,境內以安」。正因為如此,招攬了不少中原名士前來投奔,其中包括唐朝學士韓偓、王淡(王溥之子)、楊沂(楊涉從弟)、徐寅(進士)等人。他也注重教育,「建學四門,以教閩士之秀者」。王審知積極發展海外貿易,招攬海外商賈,佛齊等國相繼前來朝貢。另一方面,他奉中原王朝後梁的正朔,並向後梁朝貢。當時楊吳的楊行密控制了江淮一帶,陸路朝貢路線被阻斷,王審知每年都遣使自登、萊入貢於後梁。後唐攻滅後梁後,王審知又繼續向後唐朝貢。

此外,王審知也著手擴建福州城。902年,王審知築福州外羅城四十里。905年,又築南北夾城,稱為「南北月城」,與大城合起來共計方圓二十六里四千八百丈。他也是一位佛教的虔誠供養者,在位期間曾向開元寺進獻菩薩之像,並舉行道場。

同光三年(925年)夏五月,王審知病危,命由長子威武節度副使王延翰為「權知軍府事」。冬十二月辛未薨,在位二十九年,年六十有四。葬于福州城北凤池山。長興三年,改葬莲花山,即今日晋安区新店镇斗顶村斗顶山。後唐朝廷賜諡忠懿,又賜神道碑,命張文蔚撰文。翌年,王延翰諡王審知为昭武王。王延钧即位后,追尊廟號太祖,谥号昭武孝皇帝,陵号宣陵。

王審知有八子,可是諸子积相猜忌,治兵相取。在诸多内乱纠纷中,许多王審知后人都被杀害。朱文进篡夺王氏政权时,王審知后人50余人尽被杀戮,僅存王延政一脈。

《舊五代史》:「審知起自隴畝,以至富貴。每以節儉自處,選任良吏,省刑惜費,輕徭薄斂,與民休息。三十年間,一境晏然。」

《十國春秋》:「太祖昆弟英姿傑出,號稱三龍。據有閩疆,賓賢禮士,衣冠懷之。抑亦可謂開國之雄歟?廼卒之,臣服中原,息兵養民,大指與吳越畧同,豈非度量有過人者遠哉!」

王審知有功於福建,故受福建人民崇敬。福州建有閩王祠,於市區立有閩王塑像。再如馬祖北竿鄉坂里村王家大宅內即有供奉閩王王審知,是凝聚坂里王家的宗族中心。

除以宗祠形式紀念外,更有列為神明以神廟作為信仰供奉。由於王審知喜乘白馬,並排行第三,故稱「白馬三郎」,死後被立廟奉祀,號「白馬尊王」。不過另有閩越王郢的第三子,也被福州人尊為「白馬三郎」。馬祖地區有數座白馬尊王廟,為當地民間信仰神祇之一,但其中目前僅莒光鄉東莒島福正村白馬尊王廟證實供奉為閩王王審知,為唯一供奉閩王王審知的白馬尊王廟,源自長樂沙洋鐃鈸境白馬忠懿王宮。

福建人尊稱王審知為「開閩尊王」、「開閩聖王」或「忠惠尊王」;尊審知長兄王潮為「威武尊王」、次兄王審邽為「泉安尊王」,視為鄉土神明供奉,合稱開閩三王。

\subsubsection{开平}

\begin{longtable}{|>{\centering\scriptsize}m{2em}|>{\centering\scriptsize}m{1.3em}|>{\centering}m{8.8em}|}
  % \caption{秦王政}\
  \toprule
  \SimHei \normalsize 年数 & \SimHei \scriptsize 公元 & \SimHei 大事件 \tabularnewline
  % \midrule
  \endfirsthead
  \toprule
  \SimHei \normalsize 年数 & \SimHei \scriptsize 公元 & \SimHei 大事件 \tabularnewline
  \midrule
  \endhead
  \midrule
  元年 & 909 & \tabularnewline\hline
  二年 & 910 & \tabularnewline\hline
  三年 & 911 & \tabularnewline
  \bottomrule
\end{longtable}

\subsubsection{乾化}

\begin{longtable}{|>{\centering\scriptsize}m{2em}|>{\centering\scriptsize}m{1.3em}|>{\centering}m{8.8em}|}
  % \caption{秦王政}\
  \toprule
  \SimHei \normalsize 年数 & \SimHei \scriptsize 公元 & \SimHei 大事件 \tabularnewline
  % \midrule
  \endfirsthead
  \toprule
  \SimHei \normalsize 年数 & \SimHei \scriptsize 公元 & \SimHei 大事件 \tabularnewline
  \midrule
  \endhead
  \midrule
  元年 & 911 & \tabularnewline\hline
  二年 & 912 & \tabularnewline\hline
  三年 & 913 & \tabularnewline\hline
  四年 & 914 & \tabularnewline\hline
  五年 & 915 & \tabularnewline
  \bottomrule
\end{longtable}

\subsubsection{贞明}

\begin{longtable}{|>{\centering\scriptsize}m{2em}|>{\centering\scriptsize}m{1.3em}|>{\centering}m{8.8em}|}
  % \caption{秦王政}\
  \toprule
  \SimHei \normalsize 年数 & \SimHei \scriptsize 公元 & \SimHei 大事件 \tabularnewline
  % \midrule
  \endfirsthead
  \toprule
  \SimHei \normalsize 年数 & \SimHei \scriptsize 公元 & \SimHei 大事件 \tabularnewline
  \midrule
  \endhead
  \midrule
  元年 & 915 & \tabularnewline\hline
  二年 & 916 & \tabularnewline\hline
  三年 & 917 & \tabularnewline\hline
  四年 & 918 & \tabularnewline\hline
  五年 & 919 & \tabularnewline\hline
  六年 & 920 & \tabularnewline\hline
  七年 & 921 & \tabularnewline
  \bottomrule
\end{longtable}

\subsubsection{龙德}

\begin{longtable}{|>{\centering\scriptsize}m{2em}|>{\centering\scriptsize}m{1.3em}|>{\centering}m{8.8em}|}
  % \caption{秦王政}\
  \toprule
  \SimHei \normalsize 年数 & \SimHei \scriptsize 公元 & \SimHei 大事件 \tabularnewline
  % \midrule
  \endfirsthead
  \toprule
  \SimHei \normalsize 年数 & \SimHei \scriptsize 公元 & \SimHei 大事件 \tabularnewline
  \midrule
  \endhead
  \midrule
  元年 & 921 & \tabularnewline\hline
  二年 & 922 & \tabularnewline\hline
  三年 & 923 & \tabularnewline
  \bottomrule
\end{longtable}


\subsubsection{同光}

\begin{longtable}{|>{\centering\scriptsize}m{2em}|>{\centering\scriptsize}m{1.3em}|>{\centering}m{8.8em}|}
  % \caption{秦王政}\
  \toprule
  \SimHei \normalsize 年数 & \SimHei \scriptsize 公元 & \SimHei 大事件 \tabularnewline
  % \midrule
  \endfirsthead
  \toprule
  \SimHei \normalsize 年数 & \SimHei \scriptsize 公元 & \SimHei 大事件 \tabularnewline
  \midrule
  \endhead
  \midrule
  元年 & 923 & \tabularnewline\hline
  二年 & 924 & \tabularnewline\hline
  三年 & 925 & \tabularnewline
  \bottomrule
\end{longtable}



%%% Local Variables:
%%% mode: latex
%%% TeX-engine: xetex
%%% TeX-master: "../../Main"
%%% End:

%% -*- coding: utf-8 -*-
%% Time-stamp: <Chen Wang: 2021-11-01 15:43:10>

\subsection{嗣王王延翰\tiny(926)}

\subsubsection{生平}

閩嗣王延翰(闽东语平話字:Uòng Iòng-hâng;?-927年),字子逸,五代時期閩國君主,王審知之長子。妻博陵郡夫人崔氏。

王審知在位時任威武軍節度副使。同光三年(925年),王審知去世,王延翰奉遺命繼立,權知軍府事,自稱威武留後,向後唐朝貢。汀州人陳本起兵,聚集三萬人圍攻汀州。王延翰派右軍都監柳邕等人,以二萬人前往討伐,將其斬殺。翌年春二月,後唐莊宗得知王延翰繼任之後,任命他為威武軍節度使。不久莊宗被弑,明宗繼位,改元天成,於夏五月,加其為同平章事。王延翰得知中原大亂,便有割據福建稱王之心。十月,王延翰取出《史記》,將其中的「東越列傳」的無諸一節翻出來給諸將吏看,並說:「閩,自古王國也。吾今不王,何待之有!」將吏們紛紛勸他割據自立,於是王延翰自稱大閩國王,立宮殿,置百官,威儀、文物皆擬天子制,羣下稱之曰殿下。又追諡王審知為昭武王,但仍奉後唐的正朔。

根據《十國春秋》的記載,王延翰「為人長大,美皙如玉,而好讀書、通經史」。但他個性驕傲荒淫,殘忍兇暴。《新五代史》中也記載了王延翰的荒淫,他在王審知喪服未除的時候便開始飲酒作樂。他於福州的西湖「築室十餘里,號曰水晶宮;每攜後庭游宴,從子城複道以出」。王延翰命人四處尋找民女,投入後宮作為自己的妾。其妻崔氏「陋而淫」,「延翰不能制」。這些被選入宮中的民女命運悲慘,被崔氏關押起來,「繫以大械,刻木為人手以擊頰,又以鐵錐刺之,一歲中死者八十四人」。後來崔氏大病一場,見者認為是被其害死者作祟,不久崔氏便死了。

王延翰看不起自己的兄弟,繼位才一個月,便將弟弟王延鈞貶為泉州刺史。建州刺史王延稟是王審知的養子,與王延翰一向不睦。當二人得知王延翰四處尋找民女之後,都上書勸阻,王延翰更加大怒。十二月,王延鈞與王延稟便聯手反叛,進軍福州。王延稟自建州順流而下,先至福州。福州指揮使陳陶率軍抵抗,兵敗自殺。天成元年十二月初八(陽曆927年1月14日)夜,王延稟率壯士百餘人,從西門架梯登城而入,攻取武庫,奪取兵器,直趨寢門。王延翰聞變,匿於別室,第二天早晨被王延稟擒獲。王延稟歷數其暴虐之罪,並聲稱他和崔氏一起害死了王審知,告諭百姓,斬於紫宸門之外。

王延鈞將王延翰葬於城北太平山,建太平地藏院,派丁守墓,稱為「王墓」。其地位於今福州市晉安區新店鎮,今日被福州人稱為「黃墓」。閩東語「黃」與「王」同音,因此被訛作「黃墓」。

\subsubsection{天成}

\begin{longtable}{|>{\centering\scriptsize}m{2em}|>{\centering\scriptsize}m{1.3em}|>{\centering}m{8.8em}|}
  % \caption{秦王政}\
  \toprule
  \SimHei \normalsize 年数 & \SimHei \scriptsize 公元 & \SimHei 大事件 \tabularnewline
  % \midrule
  \endfirsthead
  \toprule
  \SimHei \normalsize 年数 & \SimHei \scriptsize 公元 & \SimHei 大事件 \tabularnewline
  \midrule
  \endhead
  \midrule
  元年 & 926 & \tabularnewline
  \bottomrule
\end{longtable}



%%% Local Variables:
%%% mode: latex
%%% TeX-engine: xetex
%%% TeX-master: "../../Main"
%%% End:

%% -*- coding: utf-8 -*-
%% Time-stamp: <Chen Wang: 2021-11-01 15:43:42>

\subsection{惠宗王延鈞\tiny(926-935)}

\subsubsection{生平}

閩惠宗王延鈞(?-935年),繼位後更名王鏻(又作王璘),五代時期閩國第一代稱皇帝的君主,926年至935年在位。

王延鈞是太祖王審知的次子,也是嗣王王延翰之弟。王延翰繼位後,任王延鈞為泉州刺史,並在福州四處尋找民女,納入宮中為妾,王延鈞與建州刺史王延稟為此上書勸諫,王延翰大怒。王延稟與王延翰一向不睦,便與王延鈞聯軍攻打福州。王延稟自建州順流而下,於天成元年(926年)十二月初八(陽曆927年1月14日)先攻破福州,殺王延翰。不久王延鈞亦至,被王延稟推戴为武威留后。夏五月,後唐明宗任命王延鈞為威武軍節度使守中書令,封琅琊王。十一月,王延鈞遣使,向後唐進貢犀牛、香藥、海味。天成三年七月,後唐又遣吏部郎中裴羽、右散騎常侍陸崇,進封王延鈞為閩王。十二月,奉國節度使知建州王延稟向後唐上表稱疾,後唐任命其子王繼雄為建州刺史。

長興二年(931年),王延稟誤信王延鈞患重病,命次子王繼昇為建州留後,與王繼雄一起率水軍攻打福州。王延稟攻西門、王繼雄攻東門。王延鈞派樓船指揮使王仁達拒戰於南臺江。王仁達把士兵藏在船中,自己假裝出降。王繼雄登船撫慰,被王仁達殺死,梟首示眾於西門。王延稟正在放火攻城,見其子首級,不禁慟哭,軍心大亂。王仁達趁機發起攻擊,將其擒獲。王延鈞將王延稟斬於市,恢復其原名周彥琛,又派使者前去建州招撫其餘黨。王延稟之子王繼昇、王繼倫得知後出奔吳越。王延鈞便派弟弟都教練使王延政去鎮守建州。

當時福建僧侶眾多。王延鈞在位期間,下令丈量土地,分為三等,上等賜給僧侶,中等授予土著百姓,下等給流寓之人耕種。閩國的科舉之法模仿唐朝,但兩稅卻被加重。他喜好神仙之術,寵信左道巫者徐彥、朴盛韜、陳守元等人,還建造寶皇宮給陳守元居住,稱其為宮主。當時福州有王霸壇、煉丹井。壇旁皂莢木枯萎已久,一日突然長出枝葉來。井中又有白龜浮出。掘地,得石銘,有「王霸裔孫」之文。王延鈞便認為這在自己身上應驗了,便在壇旁建造宮殿,極盡奢華。

長興二年十二月,陳守元假借寶皇之命,建議王延鈞「避位受道,當為天子六十年。」於是王延鈞遜位給長子威武軍節度副使王繼鵬,成為道士,取道號玄錫。翌年春三月復位,要求後唐仿吴越钱镠、南楚马殷之例,封自己為尚書令。後唐不答,王延鈞遂斷絕雙方關係。

後唐長興四年(933年),黄龙现于真封宅,王延鈞下令改宅为龙跃宫,又建东华宫。当年正月,王延鈞在寶皇宮受册稱帝,國號大閩,改元龍啟,改名王鏻,立五庙,追谥王审知为太祖,封高盖山为西岳。王延鈞自知國土狹小,土地偏僻,因此謹慎與四鄰相處,境內還算安定。

王延鈞的元配是南漢清遠公主劉德秀,十分美麗,但早逝。繼室金氏賢而無寵。王延鈞相當寵愛淑妃陳金鳳,筑水晶宫于福州西湖旁,在湖中造彩舫数十,每舫置宫女二十余人,自乘大龙舟与陳金鳳同游。后又筑长春宫,与陳金鳳居于其中,每晚欢宴,燃金龙烛数百,使宫女擎金玉、玛瑙、琥珀、水晶之杯盘进馔,酒酣时裸体追逐嬉笑为乐。陳金鳳本是王審知的婢女,有才藝又十分淫蕩。因王延鈞晚年得风疾,陳金鳳遂與王延鈞的男宠歸守明、李可殷私通,閩人都很痛恨他們。

閩國永和元年(935年),陳金鳳被立為后。同年,王延鈞病重,王延鈞之子王繼鵬與皇城使李倣欲聯手提前了結陳后之勢力。李倣派兵進宮,殺死皇后陳金鳳及其黨羽。王延鈞躲到為他特製的九龍帳下,變軍刺了幾下後才出去。王延鈞重傷未死卻痛不欲生,宮女不忍見其受苦,遂殺死王延鈞。王延鈞死後,為王繼鵬諡為齊肅明孝皇帝,廟號惠宗,唯《新五代史》則作廟號太宗,諡號惠皇帝。

\subsubsection{天成}

\begin{longtable}{|>{\centering\scriptsize}m{2em}|>{\centering\scriptsize}m{1.3em}|>{\centering}m{8.8em}|}
  % \caption{秦王政}\
  \toprule
  \SimHei \normalsize 年数 & \SimHei \scriptsize 公元 & \SimHei 大事件 \tabularnewline
  % \midrule
  \endfirsthead
  \toprule
  \SimHei \normalsize 年数 & \SimHei \scriptsize 公元 & \SimHei 大事件 \tabularnewline
  \midrule
  \endhead
  \midrule
  元年 & 926 & \tabularnewline\hline
  二年 & 927 & \tabularnewline\hline
  三年 & 928 & \tabularnewline\hline
  四年 & 929 & \tabularnewline\hline
  五年 & 930 & \tabularnewline
  \bottomrule
\end{longtable}

\subsubsection{长兴}

\begin{longtable}{|>{\centering\scriptsize}m{2em}|>{\centering\scriptsize}m{1.3em}|>{\centering}m{8.8em}|}
  % \caption{秦王政}\
  \toprule
  \SimHei \normalsize 年数 & \SimHei \scriptsize 公元 & \SimHei 大事件 \tabularnewline
  % \midrule
  \endfirsthead
  \toprule
  \SimHei \normalsize 年数 & \SimHei \scriptsize 公元 & \SimHei 大事件 \tabularnewline
  \midrule
  \endhead
  \midrule
  元年 & 930 & \tabularnewline\hline
  二年 & 931 & \tabularnewline\hline
  三年 & 932 & \tabularnewline
  \bottomrule
\end{longtable}

\subsubsection{龙启}

\begin{longtable}{|>{\centering\scriptsize}m{2em}|>{\centering\scriptsize}m{1.3em}|>{\centering}m{8.8em}|}
  % \caption{秦王政}\
  \toprule
  \SimHei \normalsize 年数 & \SimHei \scriptsize 公元 & \SimHei 大事件 \tabularnewline
  % \midrule
  \endfirsthead
  \toprule
  \SimHei \normalsize 年数 & \SimHei \scriptsize 公元 & \SimHei 大事件 \tabularnewline
  \midrule
  \endhead
  \midrule
  元年 & 933 & \tabularnewline\hline
  二年 & 934 & \tabularnewline
  \bottomrule
\end{longtable}

\subsubsection{永和}

\begin{longtable}{|>{\centering\scriptsize}m{2em}|>{\centering\scriptsize}m{1.3em}|>{\centering}m{8.8em}|}
  % \caption{秦王政}\
  \toprule
  \SimHei \normalsize 年数 & \SimHei \scriptsize 公元 & \SimHei 大事件 \tabularnewline
  % \midrule
  \endfirsthead
  \toprule
  \SimHei \normalsize 年数 & \SimHei \scriptsize 公元 & \SimHei 大事件 \tabularnewline
  \midrule
  \endhead
  \midrule
  元年 & 935 & \tabularnewline\hline
  二年 & 936 & \tabularnewline
  \bottomrule
\end{longtable}


%%% Local Variables:
%%% mode: latex
%%% TeX-engine: xetex
%%% TeX-master: "../../Main"
%%% End:

%% -*- coding: utf-8 -*-
%% Time-stamp: <Chen Wang: 2021-11-01 15:43:59>

\subsection{康宗王繼鵬\tiny(935-939)}

\subsubsection{生平}

闽康宗王繼鵬(閩東語:Uòng Gié-bèng;閩南語:Ông Kè-phêng;?-939年),後改名王昶,五代時期閩國君主,王延鈞的嫡出次子,母親是南漢清遠公主劉德秀。

王繼鵬原封福王。寵妾李春鷰本為王延鈞的宮女,王繼鵬與之私通,因此向繼母陳金鳳求助,說服王延鈞將其賜給王繼鵬。

閩國永和元年(935年),王繼鵬與李倣政變,殺王延鈞、陳皇后和弟王繼韜,繼位稱帝,改名王昶,封李春鷰為賢妃。次年(936年),改元通文,再封李春鷰為皇后。

王繼鵬亦如其父,十分寵信道士陳守元,連政事亦與之商量,興建紫微宮,以水晶装饰,工程浩大,更勝于寶皇宮,又因工程繁多而費用不足,因此賣官鬻爵,橫徵暴斂。

王繼鵬個性猜忌因此屢殺宗室,其叔王延羲為避禍,遂裝瘋賣傻,被王繼鵬軟禁自宅。當時原王審知的親軍「拱宸都」、「控鶴都」因賞賜不如王繼鵬自己的親軍「宸衛都」而迭有怨言。閩國通文四年(939年),北宫失火,宫殿焚烧殆尽。拱宸、控鶴軍使朱文進、連重遇因被王繼鵬懷疑對皇宮縱火,恐懼之餘遂先發難。乱兵焚长春宮,随后迎王延羲于长春宫瓦砾中登基,並攻擊王繼鵬。王繼鵬携皇后逃往宸卫都,天明后「拱宸都」、「控鶴都」进攻「宸衛都」,后者兵败,王繼鵬自福州北门出逃,在梧桐岭為追兵所獲,與皇后李春鷰及諸子一同被其堂兄王继业所殺。王延羲隨後把王繼鵬被殺之責推到「宸衛都」身上,並追諡王繼鵬為聖神英睿文明廣武應道大弘孝皇帝,廟號康宗。

\subsubsection{通文}

\begin{longtable}{|>{\centering\scriptsize}m{2em}|>{\centering\scriptsize}m{1.3em}|>{\centering}m{8.8em}|}
  % \caption{秦王政}\
  \toprule
  \SimHei \normalsize 年数 & \SimHei \scriptsize 公元 & \SimHei 大事件 \tabularnewline
  % \midrule
  \endfirsthead
  \toprule
  \SimHei \normalsize 年数 & \SimHei \scriptsize 公元 & \SimHei 大事件 \tabularnewline
  \midrule
  \endhead
  \midrule
  元年 & 936 & \tabularnewline\hline
  二年 & 937 & \tabularnewline\hline
  三年 & 938 & \tabularnewline\hline
  四年 & 939 & \tabularnewline
  \bottomrule
\end{longtable}


%%% Local Variables:
%%% mode: latex
%%% TeX-engine: xetex
%%% TeX-master: "../../Main"
%%% End:

%% -*- coding: utf-8 -*-
%% Time-stamp: <Chen Wang: 2021-11-01 15:45:37>

\subsection{景宗王延羲\tiny(939-944)}

\subsubsection{生平}

闽景宗王延羲(?-944年),继位後改名王曦,五代時期閩國君主,王審知之子,王延翰、王延鈞之弟,王繼鵬之叔。

王延羲於王繼鵬在位時任左僕射、同平章事,因王繼鵬猜忌宗室,遂裝瘋賣傻,因此被軟禁自宅。閩國通文四年(939年)拱宸、控鶴軍使朱文進、連重遇反,迎王延羲進宮,並殺王繼鵬,王延羲遂自稱威武節度使、閩國王,更名王曦,改元永隆,稱臣於後晉,但在國內官制就如同皇帝一樣。

然而王延羲繼位後,驕傲奢侈,荒淫無度,猜忌宗族,比王繼鵬有過之而無不及,其弟建州刺史王延政多有規勸,王延羲反而回信怒罵,又差人探聽王延政的隱私,二人因此結怨。永隆二年(940年)王延羲攻建州,開啟了閩國內戰,二人於數年爭戰中互有勝負。永隆三年(941年)七月,王延羲自稱大闽皇、威武军節度使。十月,再晋尊为大闽皇帝,加尊号睿明文廣武聖光德隆道大孝皇帝。王延政随后称帝,国号殷,闽国分裂。

由於王延羲個性一向暴虐,而朱文進、連重遇自從殺了王繼鵬後,就一直擔心為人所害,二人因此認為王延羲有加害之意,永隆六年(944年),連朱二人先下手為強,王延羲被刺殺。王延羲死後被諡為睿文廣武明聖元德隆道大孝皇帝,廟號景宗。

\subsubsection{永隆}

\begin{longtable}{|>{\centering\scriptsize}m{2em}|>{\centering\scriptsize}m{1.3em}|>{\centering}m{8.8em}|}
  % \caption{秦王政}\
  \toprule
  \SimHei \normalsize 年数 & \SimHei \scriptsize 公元 & \SimHei 大事件 \tabularnewline
  % \midrule
  \endfirsthead
  \toprule
  \SimHei \normalsize 年数 & \SimHei \scriptsize 公元 & \SimHei 大事件 \tabularnewline
  \midrule
  \endhead
  \midrule
  元年 & 939 & \tabularnewline\hline
  二年 & 940 & \tabularnewline\hline
  三年 & 941 & \tabularnewline\hline
  四年 & 942 & \tabularnewline\hline
  五年 & 943 & \tabularnewline\hline
  六年 & 944 & \tabularnewline
  \bottomrule
\end{longtable}


%%% Local Variables:
%%% mode: latex
%%% TeX-engine: xetex
%%% TeX-master: "../../Main"
%%% End:

%% -*- coding: utf-8 -*-
%% Time-stamp: <Chen Wang: 2019-12-26 09:57:33>

\subsection{王延政\tiny(943-945)}

\subsubsection{朱文进生平}

朱文進(閩東語:Ciŏ Ùng-Céng;閩南語:Tsu Bûn-Tsìn;?-945年),永泰(今福建永泰)人,五代時期閩國君主。閩帝王繼鵬在位時任拱宸軍使。

「拱宸都」與「控鶴都」原來都是閩太祖王審知的親軍,閩康宗王繼鵬即位後建立自己的親軍名喚「宸衛都」,而待之比拱宸、控鶴二都更厚,二都迭有怨言。朱文進並與控鶴軍使連重遇曾被王繼鵬三番四次的侮辱,二人因此十分不滿。

閩國通文四年(939年),北宮失火,連重遇奉派率軍清理火場殘餘的灰燼,工作勞苦,士卒怨懟。而連重遇又被王繼鵬懷疑參與縱火,因此率軍叛變,迎立王繼鵬之叔王延羲為帝,並殺害王繼鵬。朱文進在這次政變後,被任命為拱宸都指揮使。

朱文進與連重遇自從殺了王繼鵬後,就一直擔心為人所害,而王延羲個性一向暴虐,二人因此認為王延羲有加害之意,閩國永隆六年(944年),朱、連二人先發制人,刺殺王延羲,朱文進並被連重遇推舉,自稱閩主,殺害境內王姓皇族成員五十餘人,並放宮女出宮,停止興建中的工程,企圖與王延羲的暴政完全相反以拉攏人心。

不久,朱文進取消帝號自稱威武留後,向後晉稱臣,而後晉任命朱文進為威武節度使。後晉開運元年(944年)十二月十五日(陽曆為945年1月1日),朱文進正式被後晉出帝石重貴冊封為閩國王。

但在此時,朱、連二人的軍隊不斷被由將領留從效、陳洪進以及殷帝王延政所率領的討伐軍擊敗,情勢日漸窘迫,部下因此離心。後晉開運元年(944年)閏十二月二十九日(陽曆為945年2月14日)朱文進及連重遇被部屬林仁翰誅殺。

\subsubsection{王延政生平}

閩天德帝王延政(閩東語:Uòng Iòng-céng;?-951年),五代時期閩國最後一位君主,也是殷国唯一君主,王審知的第十三子,人稱十三郎。王延翰、王延鈞、王延羲之弟,王延羲在位時任建州刺史。

由於王延羲繼位後,驕傲奢侈,荒淫無度,猜忌宗族,王延政因此多有規勸,然而王延羲反而回信怒罵,又差人探聽王延政的隱私,二人因此結怨。閩國永隆二年(940年)王延羲攻建州,開啟了閩國內戰,二人於數年爭戰中互有勝負。

永隆三年(941年)二人短暫休兵,王延政被王延羲封為富沙王,惟不久又重啟戰端。閩國永隆五年(943年),王延政自行於建州稱帝,國號殷,改元天德。王延政與王繼鵬、王延羲一樣橫徵暴斂,因此人民生活困苦。

閩國永隆六年(944年),朱文進、連重遇殺王延羲,朱文進自稱閩主。王延政遂出兵討伐,而朱文進、連重遇尋為部下所殺。天德三年(945年)諸臣請王延政還都福州並恢復閩國國號,惟當時南唐大軍已趁閩國內亂時壓境,只好任命姪兒王繼昌出鎮福州,並派黃仁諷協助王繼昌。但黃仁諷受李仁達慫恿,殺死王繼昌,延政聞之,族誅黃仁諷一家,並派張漢真領水軍討伐福州。不久,建州城陷,王延政投降,閩國亡。

王延政後來被送往南唐都城金陵,南唐帝李璟封他為羽林大將軍;南唐保大五年(947年)改封為鄱陽王;保大九年(951年),再改封為光山王,不久過世,被追贈為福王,諡號恭懿。

\subsubsection{天德}

\begin{longtable}{|>{\centering\scriptsize}m{2em}|>{\centering\scriptsize}m{1.3em}|>{\centering}m{8.8em}|}
  % \caption{秦王政}\
  \toprule
  \SimHei \normalsize 年数 & \SimHei \scriptsize 公元 & \SimHei 大事件 \tabularnewline
  % \midrule
  \endfirsthead
  \toprule
  \SimHei \normalsize 年数 & \SimHei \scriptsize 公元 & \SimHei 大事件 \tabularnewline
  \midrule
  \endhead
  \midrule
  元年 & 943 & \tabularnewline\hline
  二年 & 944 & \tabularnewline\hline
  三年 & 945 & \tabularnewline
  \bottomrule
\end{longtable}


%%% Local Variables:
%%% mode: latex
%%% TeX-engine: xetex
%%% TeX-master: "../../Main"
%%% End:



%%% Local Variables:
%%% mode: latex
%%% TeX-engine: xetex
%%% TeX-master: "../../Main"
%%% End:

%% -*- coding: utf-8 -*-
%% Time-stamp: <Chen Wang: 2019-12-26 10:03:02>


\section{南汉\tiny(917-971)}

\subsection{简介}

南汉(917年-971年)是五代十国时期的地方政权之一,位于现广东、广西、海南三省及越南北部(后失)。971年为北宋所灭。

唐朝末年,刘谦在岭南的封州(今广东封开)任刺史,拥兵过万,战舰百餘。刘谦死后,其长子刘隐继承父职。唐天祐二年(905年),刘隐任清海军(岭南东道)节度使。907年,刘隐受后梁封为彭郡王,909年改封为南平王,次年又改封为南海王。刘隐死后,其弟劉龑(又名刘岩)袭封南海王。劉龑凭借父兄在岭南的基业,于后梁贞明三年(917年)在番禺(今广州,号兴王府)称帝,国号「大越」。次年,劉龑以汉朝刘氏后裔的身份改国号为「漢」,史称南汉,以别于北汉。

刘龑迷信,他非常喜欢《周易》,年号的改变,以及名字的变动,原因都是算卦所致。南汉、南唐曾经是友好邻邦,潘佑《为李后主与南汉后主书》称两国“情若弟兄,义同交契”。南漢末年政治黑暗,“作燒煮剝剔、刀山劍樹之刑,或令罪人鬥虎抵象。又賦斂煩重,人不聊生。”,邕州“民入城者,人输一钱”,“置媚川都,定其課,令入海五百尺采珠。所居宮殿以珠、玳瑁飾之。陳延壽作諸淫巧,日費數萬金。”

大有十年(937年),南汉的交州发生兵变,属将矯公羨杀死了主管官员,割据一方,另一属将吳權领兵攻打矯公羨,矯公羨便求救于劉龑,刘龑封儿子刘弘操为交王,然后领兵进攻吴权,结果被吴权打败,刘弘操阵亡。吴权从此占有了交州,吴权建立的王朝即越南吳朝,越南由此正式从中国独立出去。

乾和十四年,周世宗遣使通好,南汉不把中原王朝放在眼里,“馆接者遗茉莉,文其名曰小南强”。大宝十三年,宋太祖授意南唐后主遣使游说南汉归宋,后主派陈省躬出使南汉,游说南汉向宋称臣,刘鋹不从。不久,宋以潭州防御使潘美为贺州道行营兵马都总管,朗州团练使尹崇珂为副,发兵攻南汉。十四年二月,南汉亡。

南汉是一个商业气息浓郁的国家,“岭北行商至国都,必召示之夸其富”,“每见北人,盛夸岭海之强”。马端临谓:“宋兴,而吴、蜀、江南、荆湖、南粤,皆号富强”。钱多为主,钱少为奴视为一般的准则。

劉鋹时,日与波斯女等大宫中游宴,“无名之费,日有千万”,其官制则有特殊的规定,科举被录取者,若要做官必须先净身,也就是閹割。在劉鋹看来,百官们有家有室,有妻儿老小,肯定不能对皇上尽忠。


%% -*- coding: utf-8 -*-
%% Time-stamp: <Chen Wang: 2019-12-26 10:01:09>

\subsection{高祖\tiny(917-942)}

\subsubsection{生平}

漢高祖劉龑yǎn(889年-942年),原稱劉巖,又名刘陟。五代十国时期南汉开國皇帝,蔡州上蔡(今河南省上蔡县)人,郡望彭城(今江苏省徐州市),父亲是南汉追赠圣武皇帝刘知谦,母亲是妾段氏。

祖父劉安仁為蔡州上蔡(今河南上蔡)人,遷居福建,以經商為生,又因為生意需要,遷居嶺南,生刘知谦。唐朝末年,知谦從軍,受到當時的南海軍節度使韋宙賞識,與韋的姪女韦氏結婚,生下劉隱、劉臺兩個兒子。黃巢之亂出征有功,882年任封州刺史。

刘知谦后来又私纳小妾段氏,生下劉巖。正妻韦氏大怒,杀了段氏,但未忍伤害还是婴儿的劉巖,抱回家中和自己的两个儿子一起抚育。劉巖长大之後,聪慧又擅長武艺,且精通占卜算命之术,但又天性苛酷,每视杀人则喜,人皆以为蛟蜃化身。

知谦死後,長子劉隱擁兵自重,克肇慶、番禺(今屬广州),割据岭南地区,逐渐坐大,但劉隱向後梁稱臣納貢,後梁封為南海王,乾化元年(911年),劉隱还没来得及称帝,就因病在本郡去世,諡襄王。

劉隱有子,但劉巖篡奪其位,自命為留後,又襲位交趾节度使,其后又袭封南海王称号。劉隱死后六年(917年),劉巖在番禺称帝,建国号为「大越」, 改元为「乾亨」,定都番禺(今日廣州市)。次年,刘巖自称是汉朝皇室的后裔,为了表示自己建国是恢复昔日的汉家天下,于是又改国号为「大汉」,史称南汉。

後唐同光四年(926年),劉巖取《周易》中「飞龍在天」之意,改名为「劉龑」。據說他原先为自己改名为「劉龔」,後才自创一「龑」字。

除了劉龑及祖父劉安仁为上蔡(今河南省上蔡县)人的说法外,其侄女刘华的墓志称家族出于东晋时南渡的彭城刘氏,“而家于五羊,今为封州贺水人也”。藤田豐八认为他是大食商人后裔。藤田氏此说遭到另一位日本学者桑原骘藏的质疑。桑原认为,关于宋元祐间广州蕃坊娶宗女的刘姓人为阿拉伯人的说法,不能用以证明南汉刘氏是阿拉伯人的后裔,相反,宋元祐间广州蕃客的刘姓,可能是南汉所赐,“余谓唐代每以国姓赐外国人,此刘姓回民,或南汉刘氏赐与广州蕃客者,因为得姓之源欤?”

\subsubsection{乾亨}

\begin{longtable}{|>{\centering\scriptsize}m{2em}|>{\centering\scriptsize}m{1.3em}|>{\centering}m{8.8em}|}
  % \caption{秦王政}\
  \toprule
  \SimHei \normalsize 年数 & \SimHei \scriptsize 公元 & \SimHei 大事件 \tabularnewline
  % \midrule
  \endfirsthead
  \toprule
  \SimHei \normalsize 年数 & \SimHei \scriptsize 公元 & \SimHei 大事件 \tabularnewline
  \midrule
  \endhead
  \midrule
  元年 & 917 & \tabularnewline\hline
  二年 & 918 & \tabularnewline\hline
  三年 & 919 & \tabularnewline\hline
  四年 & 920 & \tabularnewline\hline
  五年 & 921 & \tabularnewline\hline
  六年 & 922 & \tabularnewline\hline
  七年 & 923 & \tabularnewline\hline
  八年 & 924 & \tabularnewline\hline
  九年 & 925 & \tabularnewline
  \bottomrule
\end{longtable}

\subsubsection{白龙}

\begin{longtable}{|>{\centering\scriptsize}m{2em}|>{\centering\scriptsize}m{1.3em}|>{\centering}m{8.8em}|}
  % \caption{秦王政}\
  \toprule
  \SimHei \normalsize 年数 & \SimHei \scriptsize 公元 & \SimHei 大事件 \tabularnewline
  % \midrule
  \endfirsthead
  \toprule
  \SimHei \normalsize 年数 & \SimHei \scriptsize 公元 & \SimHei 大事件 \tabularnewline
  \midrule
  \endhead
  \midrule
  元年 & 925 & \tabularnewline\hline
  二年 & 926 & \tabularnewline\hline
  三年 & 927 & \tabularnewline\hline
  四年 & 928 & \tabularnewline
  \bottomrule
\end{longtable}

\subsubsection{大有}

\begin{longtable}{|>{\centering\scriptsize}m{2em}|>{\centering\scriptsize}m{1.3em}|>{\centering}m{8.8em}|}
  % \caption{秦王政}\
  \toprule
  \SimHei \normalsize 年数 & \SimHei \scriptsize 公元 & \SimHei 大事件 \tabularnewline
  % \midrule
  \endfirsthead
  \toprule
  \SimHei \normalsize 年数 & \SimHei \scriptsize 公元 & \SimHei 大事件 \tabularnewline
  \midrule
  \endhead
  \midrule
  元年 & 928 & \tabularnewline\hline
  二年 & 929 & \tabularnewline\hline
  三年 & 930 & \tabularnewline\hline
  四年 & 931 & \tabularnewline\hline
  五年 & 932 & \tabularnewline\hline
  六年 & 933 & \tabularnewline\hline
  七年 & 934 & \tabularnewline\hline
  八年 & 935 & \tabularnewline\hline
  九年 & 936 & \tabularnewline\hline
  十年 & 937 & \tabularnewline\hline
  十一年 & 938 & \tabularnewline\hline
  十二年 & 939 & \tabularnewline\hline
  十三年 & 940 & \tabularnewline\hline
  十四年 & 941 & \tabularnewline\hline
  十五年 & 942 & \tabularnewline
  \bottomrule
\end{longtable}


%%% Local Variables:
%%% mode: latex
%%% TeX-engine: xetex
%%% TeX-master: "../../Main"
%%% End:

%% -*- coding: utf-8 -*-
%% Time-stamp: <Chen Wang: 2019-12-26 10:01:32>

\subsection{殇帝\tiny(942-943)}

\subsubsection{生平}

漢殤帝劉玢(920年-943年),原名劉弘度,五代時期南漢君主,是南漢建立者劉龑之第三子,原封賓王,後改封秦王。

南漢大有十五年(942年),南漢高祖劉龑病重,原以劉弘度及晉王劉弘熙驕傲放縱,因此欲立幼子越王劉弘昌,為臣下所力諫,乃打消此意。不久,劉龑病逝,劉弘度遂繼位,改名劉玢,並改年號光天。

同年,循州(今廣東龍川)變民共推縣吏張遇賢為主,稱中天八國王,改元永樂,一時間南漢的東方州縣多被張遇賢所攻陷。

劉玢驕傲奢侈,荒淫無度。劉龑還在停靈的時條,就肆無忌憚地奏樂飲酒;夜晚則穿著黑色的喪服外出與娼妓混在一起;又命男女赤身裸體供其觀賞;政事廢弛,因此民變愈演愈烈。而左右有忤逆其意者動輒被處死,以致於除了劉弘昌及內常侍吳懷恩外,沒有人敢勸諫,而受勸諫後又不聽。劉玢亦猜忌諸弟及官員,所以叫宦官把守宮門,進宮時要將衣服脫掉檢查,才可以入內,劉弘熙遂生政變之意。

因為知道劉玢喜歡角力,所以南漢光天二年(943年),劉弘熙命陳道庠找來力士數人並告知劉玢,劉玢聽到後大為高興,即與諸王於長春宮聚會飲宴,並觀賞角力。當晚宴會結束,劉玢大醉,劉弘熙於是命力士抓住劉玢,摧擊其前胸斃命。劉玢死後,被諡殤帝。

\subsubsection{光天}

\begin{longtable}{|>{\centering\scriptsize}m{2em}|>{\centering\scriptsize}m{1.3em}|>{\centering}m{8.8em}|}
  % \caption{秦王政}\
  \toprule
  \SimHei \normalsize 年数 & \SimHei \scriptsize 公元 & \SimHei 大事件 \tabularnewline
  % \midrule
  \endfirsthead
  \toprule
  \SimHei \normalsize 年数 & \SimHei \scriptsize 公元 & \SimHei 大事件 \tabularnewline
  \midrule
  \endhead
  \midrule
  元年 & 942 & \tabularnewline\hline
  二年 & 943 & \tabularnewline
  \bottomrule
\end{longtable}



%%% Local Variables:
%%% mode: latex
%%% TeX-engine: xetex
%%% TeX-master: "../../Main"
%%% End:

%% -*- coding: utf-8 -*-
%% Time-stamp: <Chen Wang: 2021-11-01 15:47:55>

\subsection{中宗劉晟\tiny(943-958)}

\subsubsection{生平}

漢中宗劉\xpinyin*{晟}(920年-958年),原名劉弘熙,五代十國時期南漢君主,是南漢建立者劉龑之子,劉玢之弟。

南漢帝劉玢即位後驕傲奢侈,不理政事,荒淫無道,並且猜忌諸弟,原封晉王的劉弘熙因此有政變之意。南漢光天二年(943年),劉弘熙找來力士數人表演角力,與劉玢飲宴觀賞,當晚宴會結束,劉玢大醉,劉弘熙即命力士抓住劉玢,摧擊其前胸,斃命。翌晨,越王劉弘昌率諸弟至寢殿,迎劉弘熙即皇帝位,劉弘熙繼位後,改名劉晟,改年號应乾,同年稍後又改元乾和。

然而劉晟掌握神器後,亦如其兄劉玢一樣猜忌諸弟,因此不久後就逐漸殺光其弟,並將他們的兒子殺死,女兒收入後宮。又興建離宮一千餘間,以珠寶裝飾,並設有許多殘酷的刑具,號「生地獄」。命宮女為女侍中,參與政事,由於宗室元勳幾乎剷除殆盡,當權者就是宦官、女官這些人而已。

刘晟生性荒淫暴虐,得志之后,专门用威势刑法统治下民,多诛灭旧臣以及自己的兄弟、侄子,将侄女收入后宫。数年之间,刘家被他差不多杀尽。又修造「活地狱」,大凡开水锅、铁烙床之类,无不齐备。人们犯有小的过失,就备受其刑罚之苦。到南楚的马家兄弟互动干戈时,刘晟趁此机会,派兵进攻桂林管区内各郡以及彬州、连州、梧州、贺州,都被攻克,从此全部拥有南越之地。

乾和六年(948年)劉晟派兵攻南楚,不久南楚內亂,乾和九年(951年)趁南楚為南唐所滅之際,占有南楚嶺南之地。

乾和十三年(955年),刘晟又杀其弟刘弘政,於是,刘龑的诸子被诛杀殆尽。

显德三年(956年),后周世宗柴刘晟忧虑万分。刘晟曾说过知晓占星术,同年六月在甘泉宫观天,牛女星间有月食,刘晟去對照占星之书,立即把書丢到地下,叹道:“自古以来,有谁能不死吗!”从此彻夜放纵饮酒。

乾和十六年(958年),在城北选定墓址,修建陵墓,刘晟亲自视察。同年秋去世,终年三十九岁,諡文武光聖明孝皇帝,廟號中宗,子劉鋹繼位。

\subsubsection{应乾}

\begin{longtable}{|>{\centering\scriptsize}m{2em}|>{\centering\scriptsize}m{1.3em}|>{\centering}m{8.8em}|}
  % \caption{秦王政}\
  \toprule
  \SimHei \normalsize 年数 & \SimHei \scriptsize 公元 & \SimHei 大事件 \tabularnewline
  % \midrule
  \endfirsthead
  \toprule
  \SimHei \normalsize 年数 & \SimHei \scriptsize 公元 & \SimHei 大事件 \tabularnewline
  \midrule
  \endhead
  \midrule
  元年 & 943 & \tabularnewline
  \bottomrule
\end{longtable}

\subsubsection{乾和}

\begin{longtable}{|>{\centering\scriptsize}m{2em}|>{\centering\scriptsize}m{1.3em}|>{\centering}m{8.8em}|}
  % \caption{秦王政}\
  \toprule
  \SimHei \normalsize 年数 & \SimHei \scriptsize 公元 & \SimHei 大事件 \tabularnewline
  % \midrule
  \endfirsthead
  \toprule
  \SimHei \normalsize 年数 & \SimHei \scriptsize 公元 & \SimHei 大事件 \tabularnewline
  \midrule
  \endhead
  \midrule
  元年 & 943 & \tabularnewline\hline
  二年 & 944 & \tabularnewline\hline
  三年 & 945 & \tabularnewline\hline
  四年 & 946 & \tabularnewline\hline
  五年 & 947 & \tabularnewline\hline
  六年 & 948 & \tabularnewline\hline
  七年 & 949 & \tabularnewline\hline
  八年 & 950 & \tabularnewline\hline
  九年 & 951 & \tabularnewline\hline
  十年 & 952 & \tabularnewline\hline
  十一年 & 953 & \tabularnewline\hline
  十二年 & 954 & \tabularnewline\hline
  十三年 & 955 & \tabularnewline\hline
  十四年 & 956 & \tabularnewline\hline
  十五年 & 957 & \tabularnewline\hline
  十六年 & 958 & \tabularnewline
  \bottomrule
\end{longtable}



%%% Local Variables:
%%% mode: latex
%%% TeX-engine: xetex
%%% TeX-master: "../../Main"
%%% End:

%% -*- coding: utf-8 -*-
%% Time-stamp: <Chen Wang: 2019-12-26 10:02:48>

\subsection{后主\tiny(958-971)}

\subsubsection{生平}

劉\xpinyin*{鋹}(942年-980年)原名劉繼興,五代十國時期南漢末代君主,是南漢中宗劉晟之長子,原封衛王。

南漢乾和十六年(958年)劉晟去世,劉繼興繼位,改名劉鋹,改元大寶。

劉鋹不會治國,政事皆委諸宦官龔澄樞及女侍中盧瓊仙等人,女官亦任命為參政官員,其餘官員只是聊備一格而已。

劉鋹又認為群臣都有家室,會為了顧及子孫不肯盡忠,因此只信任宦官,臣屬必須自宮才會被進用,以致於一度宦官高達二萬人之多。

又相當寵愛一名波斯女子,與之淫戲於後宮,叫她「媚豬」,而自稱「蕭閒大夫」,不理政事。後來將政事又交予女巫樊胡,連龔澄樞及盧瓊仙都依附她,政事紊亂。

南漢大寶十三年(970年),宋朝派潭州防禦使潘美攻南漢。南漢舊將先前多因讒言而被殺,宗室亦遭翦除殆盡,掌兵權的只有宦官而已,城牆、護城河,都裝飾為宮殿、水塘;樓船戰艦、武器盔甲,全部腐朽。

大寶十四年(宋開寶四年,971年),宋軍節節進逼。劉鋹纵火焚毁宫殿、府库,挑選十幾艘船,滿載金銀財寶及嬪妃,準備逃亡入海;還沒出發,宦官與衛兵就盜取船舶逃走,劉鋹只好投降,南漢亡。

劉鋹歸順宋朝後,押送至东京開封府,囚于玉津园,后以帛系颈,与南汉官属一同献俘于太庙、太社。宋太祖遣吕余庆问责焚烧府库之事。劉鋹將責任完全推給龔澄樞,其曰“臣年十六僭伪号,澄枢等皆先臣旧人,每事,臣不得自由,在国时,臣是臣下,澄枢是国主。”宋太祖趙匡胤就將龔澄樞斬首,而赦免劉鋹的罪,並任命其為金紫光禄大夫、检校太保、右千牛衛大將軍、员外置同正员,封恩赦侯。

荒淫無度的劉鋹投降後,為宋太祖、宋太宗厚待,也出現不少趣事,與南唐後主李煜的國愁家恨形成強烈對比。劉鋹本人體態豐滿,眉清目秀。有巧思,亦能言善辯,曾用珠子將馬鞍串成戲龍的形狀獻予宋太祖。宋太祖因此感嘆說:「劉鋹如果能將這項技藝用在治國上,怎麼會滅亡!」劉鋹稱帝在位时,多置鴆酒,毒死臣下。降宋後,一日宋太祖乘肩舆,与随从数十人幸讲武池。从官未至,而劉鋹先至。宋太祖賜以酒,劉鋹以為要毒殺自己,大哭曰:“臣承祖父基业,违拒朝廷,劳王师讨致,罪固当诛。陛下既待臣不死,愿为大梁布衣,观太平之盛。臣未敢饮此酒。”宋太祖笑而取酒自飲,劉鋹大感慚愧。

開寶八年(975年),宋滅南唐後,將劉鋹改命左監門衛上將軍,封彭城郡公。宋太宗即帝位,再改封其為衛國公。

太平興國四年(979年),宋太宗將伐北漢劉繼元,在長春殿宴請潘美等將領。當時劉鋹與已降宋的前吳越王錢俶、前平海军節度使陳洪進都參加,劉鋹因此說:「朝廷威靈遠播,四方竊位僭主的君王,今日都在座,不久又要平定太原,劉繼元又將到達,臣率先來朝,願揮舞大棒,替陛下吶喊助威,望成為各國降王的領袖。」宋太宗因此大笑。

太平興國五年(980年),劉鋹去世,獲贈授太師,追封為南越王。由於劉鋹是南漢最後一位君主,復無諡號、廟號,史家所以習稱其為南漢後主。

\subsubsection{大宝}

\begin{longtable}{|>{\centering\scriptsize}m{2em}|>{\centering\scriptsize}m{1.3em}|>{\centering}m{8.8em}|}
  % \caption{秦王政}\
  \toprule
  \SimHei \normalsize 年数 & \SimHei \scriptsize 公元 & \SimHei 大事件 \tabularnewline
  % \midrule
  \endfirsthead
  \toprule
  \SimHei \normalsize 年数 & \SimHei \scriptsize 公元 & \SimHei 大事件 \tabularnewline
  \midrule
  \endhead
  \midrule
  元年 & 958 & \tabularnewline\hline
  二年 & 959 & \tabularnewline\hline
  三年 & 960 & \tabularnewline\hline
  四年 & 961 & \tabularnewline\hline
  五年 & 962 & \tabularnewline\hline
  六年 & 963 & \tabularnewline\hline
  七年 & 964 & \tabularnewline\hline
  八年 & 965 & \tabularnewline\hline
  九年 & 966 & \tabularnewline\hline
  十年 & 967 & \tabularnewline\hline
  十一年 & 968 & \tabularnewline\hline
  十二年 & 969 & \tabularnewline\hline
  十三年 & 970 & \tabularnewline\hline
  十四年 & 971 & \tabularnewline
  \bottomrule
\end{longtable}



%%% Local Variables:
%%% mode: latex
%%% TeX-engine: xetex
%%% TeX-master: "../../Main"
%%% End:



%%% Local Variables:
%%% mode: latex
%%% TeX-engine: xetex
%%% TeX-master: "../../Main"
%%% End:

%% -*- coding: utf-8 -*-
%% Time-stamp: <Chen Wang: 2019-12-26 10:04:32>


\section{前蜀\tiny(903-925)}

\subsection{简介}

前蜀(907年—925年),是中国五代十国时期由王建建立的政权,十国之一。前蜀疆域辽阔,东控荆襄,南通南诏,西达维州(今四川理县),北过秦州(今甘肃天水),占领了今天四川、湖北、陕西以及甘肃大部,重庆、贵州全部以及云南部分地区,方圆数千里。

前蜀是唐朝的“蜀王”、西川节度使王建在成都建立的,早在891年,王建就开始统辖全川。903年,唐昭宗封王建为蜀王。唐哀帝天祐四年(907年),王建不服后梁统治,建国号“蜀”,史称“前蜀”,定都成都。917年,行刘备在成都称汉故事,改国号为汉,次年又恢复国号蜀。

王氏父子共统治两川35年。前蜀初年,王建励精图治,开拓疆土,兴修水利,注重农桑,实行“与民休息”的政策。在没有战争的情况下,在拥有沃地千里、丰饶五谷的成都平原的情况下,前蜀的经济、文化、军事大大的发展,成为了当时的一个强国。可是王建死后,继承人王衍奢侈无度,残暴昏庸,后唐趁机伐蜀,蜀军溃败,成都沦陷,前蜀灭亡。

今日成都前蜀永陵(王建墓)規模頗可觀。

前蜀建國後,典章制度主要由宰相韋莊制定。韋莊熟習唐代制度,因此前蜀的政制充滿唐代遺風。前蜀宰相是同平章事,大多由中書侍郎門下侍郎兼領。唐代以中書令為宰相,地位崇高,前蜀的中書令則多以宗室兼任,並非專員。

前蜀樞密使亦是機要官職,自韋莊之後掌握朝廷大權。唐代內樞密使以宦官擔任,前蜀起初則以士人擔任,後來擔心將領不受控制,也起用宦官擔任樞密使,朝政漸壞。此外,大學士亦參與朝政。

宮廷中,內飛龍使一職掌握禁軍,往往干預朝政。地方上掌握兵權的,有節度使、團練使、觀察使等,屬下有判官和掌書記。地方官制中,州有刺史,下有參軍;縣有縣令,下有主簿,大體上和唐代相同。


%% -*- coding: utf-8 -*-
%% Time-stamp: <Chen Wang: 2019-12-26 10:05:05>

\subsection{高祖\tiny(907-918)}

\subsubsection{生平}

前蜀高祖王建(847年2月26日-918年7月11日),字光圖,五代十國时期前蜀開國皇帝(907年—918年在位),许州舞阳(今河南舞阳)人。

少年时为无赖,以屠牛驴和贩私盐为业,乡里称为“贼王八”,黄巢起事时期投效唐朝军队,隶属忠武军。长安沦陷时他奋不顾身地护驾,號為“隨駕五都”,为忠武八都的都将之一,被唐僖宗封为西川节度使、壁州刺史,十軍觀軍容使田令孜也收他為養子。僖宗還長安後,升為御林軍宿衛將領。光啟二年(886年),僖宗又逃往興元(今陝西漢中),任命王建為“清道使”,以后他向四方发展势力。

大顺二年(891年)以精兵二千奔往成都,為陳敬瑄所阻,王建攻破鹿頭關,取漢州,攻彭州,大敗陳敬瑄五萬兵,不久攻占成都,陳敬瑄與田令孜開門出降,据西川,杀陈敬瑄、田令孜,接著又降黔南節度使王建肇,殺東川節度使顧彥暉、降武定節度使拓拔思敬。897年占有东川梓(今四川三台)、渝(今重庆)诸州,遂有有兩川兼三峽之地。902年取得山南西道控制權。天復三年(903年),唐昭宗又封他为蜀王,遂成为当时最大的割据势力。次年,朱温挟持唐昭宗迁都洛阳,改元天祐,王建不承认,继续使用天復年号。

唐哀帝天祐四年(907年)唐亡,后王建因不服后梁而自立为皇帝,国号“大蜀”,史称“前蜀”,定都成都,当年沿用唐朝天復年号,908年建年号“武成”。在位12年。在位时期,励精图治,注重农桑,兴修水利,扩张疆土,实行“与民休息”的政策,蜀中大治。死后谥号神武圣文孝德明惠皇帝,庙号高祖,葬于成都的永陵(今成都市西延线永陵路)。

\subsubsection{天复}

\begin{longtable}{|>{\centering\scriptsize}m{2em}|>{\centering\scriptsize}m{1.3em}|>{\centering}m{8.8em}|}
  % \caption{秦王政}\
  \toprule
  \SimHei \normalsize 年数 & \SimHei \scriptsize 公元 & \SimHei 大事件 \tabularnewline
  % \midrule
  \endfirsthead
  \toprule
  \SimHei \normalsize 年数 & \SimHei \scriptsize 公元 & \SimHei 大事件 \tabularnewline
  \midrule
  \endhead
  \midrule
  元年 & 907 & \tabularnewline
  \bottomrule
\end{longtable}

\subsubsection{武成}

\begin{longtable}{|>{\centering\scriptsize}m{2em}|>{\centering\scriptsize}m{1.3em}|>{\centering}m{8.8em}|}
  % \caption{秦王政}\
  \toprule
  \SimHei \normalsize 年数 & \SimHei \scriptsize 公元 & \SimHei 大事件 \tabularnewline
  % \midrule
  \endfirsthead
  \toprule
  \SimHei \normalsize 年数 & \SimHei \scriptsize 公元 & \SimHei 大事件 \tabularnewline
  \midrule
  \endhead
  \midrule
  元年 & 908 & \tabularnewline\hline
  二年 & 909 & \tabularnewline\hline
  三年 & 910 & \tabularnewline
  \bottomrule
\end{longtable}

\subsubsection{通正}

\begin{longtable}{|>{\centering\scriptsize}m{2em}|>{\centering\scriptsize}m{1.3em}|>{\centering}m{8.8em}|}
  % \caption{秦王政}\
  \toprule
  \SimHei \normalsize 年数 & \SimHei \scriptsize 公元 & \SimHei 大事件 \tabularnewline
  % \midrule
  \endfirsthead
  \toprule
  \SimHei \normalsize 年数 & \SimHei \scriptsize 公元 & \SimHei 大事件 \tabularnewline
  \midrule
  \endhead
  \midrule
  元年 & 916 & \tabularnewline
  \bottomrule
\end{longtable}


\subsubsection{天汉}

\begin{longtable}{|>{\centering\scriptsize}m{2em}|>{\centering\scriptsize}m{1.3em}|>{\centering}m{8.8em}|}
  % \caption{秦王政}\
  \toprule
  \SimHei \normalsize 年数 & \SimHei \scriptsize 公元 & \SimHei 大事件 \tabularnewline
  % \midrule
  \endfirsthead
  \toprule
  \SimHei \normalsize 年数 & \SimHei \scriptsize 公元 & \SimHei 大事件 \tabularnewline
  \midrule
  \endhead
  \midrule
  元年 & 917 & \tabularnewline
  \bottomrule
\end{longtable}


\subsubsection{光天}

\begin{longtable}{|>{\centering\scriptsize}m{2em}|>{\centering\scriptsize}m{1.3em}|>{\centering}m{8.8em}|}
  % \caption{秦王政}\
  \toprule
  \SimHei \normalsize 年数 & \SimHei \scriptsize 公元 & \SimHei 大事件 \tabularnewline
  % \midrule
  \endfirsthead
  \toprule
  \SimHei \normalsize 年数 & \SimHei \scriptsize 公元 & \SimHei 大事件 \tabularnewline
  \midrule
  \endhead
  \midrule
  元年 & 918 & \tabularnewline
  \bottomrule
\end{longtable}


%%% Local Variables:
%%% mode: latex
%%% TeX-engine: xetex
%%% TeX-master: "../../Main"
%%% End:

%% -*- coding: utf-8 -*-
%% Time-stamp: <Chen Wang: 2019-12-26 10:05:35>

\subsection{后主\tiny(918-925)}

\subsubsection{生平}

王衍(901年8月31日-926年5月18日),本名王宗衍,字化源,前蜀末代皇帝(第二代,918年—925年在位),在位7年,史稱「後主」。前蜀灭亡后,入后唐。在押送途中全族被殺。

王衍是蜀高祖王建第11子,也是幼子,母親是徐賢妃。当初,王建因长子卫王王宗仁有病而立次子王宗懿(又名王元膺)为皇太子,但王宗懿因与王建宠臣唐道袭冲突而发动兵变,杀死唐道袭,自己也被杀。王建本意从长得像自己的三子豳王王宗辂和诸子中最贤的八子信王王宗杰中择嗣,但在徐贤妃和宰相张格等人经营下,功臣王宗侃等误以为王建中意王宗衍,都请求立王宗衍为太子,而王建也因此误以为王宗衍得众心,虽然认为他幼懦,怀疑他是否堪任,仍然立他为太子。918年,王建死,他继承了皇位,是为蜀后主。

王衍是一个十分荒淫腐朽的皇帝,他迷恋美色,宦官王承休「多以邪僻姦穢之事媚其主,主愈寵之」,对北方小朝廷后梁、后唐的攻击不闻不问,前蜀国势一天不如一天。终于在咸康元年(925年),唐庄宗派遣大军进攻前蜀,蜀军溃败,王衍养兄齐王王宗弼劫持王衍,迫使王衍举国投降,前蜀灭亡。

次年即926年,庄宗正欲对付邺都变兵时,伶人景进向莊宗进言说王衍是个祸害,应当设法翦除,莊宗便杀害了王衍及其亲族。但枢密使张居翰擅自将庄宗诛杀“王衍一行”的诏书改成诛杀“王衍一家”,使得跟随王衍的臣仆千余人得活。天成三年,在前蜀旧臣王宗寿等乞求下,後唐明宗李嗣源追封王衍順正公,以诸侯礼葬于长安南三赵屯。

\subsubsection{乾德}

\begin{longtable}{|>{\centering\scriptsize}m{2em}|>{\centering\scriptsize}m{1.3em}|>{\centering}m{8.8em}|}
  % \caption{秦王政}\
  \toprule
  \SimHei \normalsize 年数 & \SimHei \scriptsize 公元 & \SimHei 大事件 \tabularnewline
  % \midrule
  \endfirsthead
  \toprule
  \SimHei \normalsize 年数 & \SimHei \scriptsize 公元 & \SimHei 大事件 \tabularnewline
  \midrule
  \endhead
  \midrule
  元年 & 919 & \tabularnewline\hline
  二年 & 920 & \tabularnewline\hline
  三年 & 921 & \tabularnewline\hline
  四年 & 922 & \tabularnewline\hline
  五年 & 923 & \tabularnewline\hline
  六年 & 924 & \tabularnewline
  \bottomrule
\end{longtable}

\subsubsection{咸康}

\begin{longtable}{|>{\centering\scriptsize}m{2em}|>{\centering\scriptsize}m{1.3em}|>{\centering}m{8.8em}|}
  % \caption{秦王政}\
  \toprule
  \SimHei \normalsize 年数 & \SimHei \scriptsize 公元 & \SimHei 大事件 \tabularnewline
  % \midrule
  \endfirsthead
  \toprule
  \SimHei \normalsize 年数 & \SimHei \scriptsize 公元 & \SimHei 大事件 \tabularnewline
  \midrule
  \endhead
  \midrule
  元年 & 925 & \tabularnewline
  \bottomrule
\end{longtable}


%%% Local Variables:
%%% mode: latex
%%% TeX-engine: xetex
%%% TeX-master: "../../Main"
%%% End:



%%% Local Variables:
%%% mode: latex
%%% TeX-engine: xetex
%%% TeX-master: "../../Main"
%%% End:

%% -*- coding: utf-8 -*-
%% Time-stamp: <Chen Wang: 2019-12-26 10:06:27>


\section{后蜀\tiny(934-965)}

\subsection{简介}

後蜀(934年-965年,又稱孟蜀)是中國歷史上的十國之一,其疆域较前蜀而言要小,其中后蜀的疆域东线和北线最为显著。东由襄阳退至重庆一带,北也由甘陕退到广元。后因中原变乱得到一些中原节度使的归附,鼎盛时期达到前蜀的疆域。

同光三年(925年)後唐滅前蜀,孟知祥(874年—934年)被任为西川節度使,次年入成都,平定叛将李绍琛,整頓吏治,成都始安。長興二年(931年),杀后唐派来的监军李严,与东川节度使董璋起兵叛乱,并击退唐明宗派来的军队。次年(932年),孟知祥与后唐朝廷和解,董璋不满,兴兵来攻,孟知祥反而败殺董璋,取得東川。

後唐閔帝應順元年(934年),孟知祥在成都建都稱帝,年號明德,國號蜀,史稱“後蜀”,以别于王建的前蜀。同年,后唐凤翔节度使李从珂起兵推翻闵帝,后唐武定军节度使孙汉韶、山南西道节度使张虔钊因曾讨伐李从珂而不安,以本镇归后蜀。但孟知祥在位7月即卒。

其子孟昶嗣位,沿用明德年號,明德五年(938年)後改元廣政,頗能勵精圖治,境內很少發生戰爭,是五代時期經濟文化較發達的地區,後蜀維持近三十年和平。辽朝灭后晋时,晋雄武军节度使何重建不附辽,以辖区归附后蜀,后蜀又迫降凤州防御使石奉頵,达到之前前蜀的疆域。但后蜀随即被新建立的后汉所败,本已降蜀的凤翔节度使侯益亦因而投靠后汉;后来后汉叛将河中李守贞、长安赵思绾、凤翔王景崇也求援于后蜀,孟昶不顾劝阻出兵,但此次出师仍无功,三将叛乱被后汉平定。

晚年,孟昶生活日漸荒淫。廣政十八年(955年),秦、階、成、鳳四州為後周攻佔。

北宋乾德二年(964年)十一月宋太祖發兵攻蜀,次年正月孟昶向宋朝投降,同年去世。後蜀亡,前後僅二帝。

欧阳修《新五代史》中对此有“尽力不过二袋(代)”的典故。

%% -*- coding: utf-8 -*-
%% Time-stamp: <Chen Wang: 2021-11-01 15:48:51>

\subsection{高祖孟知祥\tiny(934-937)}

\subsubsection{生平}

後蜀高祖孟知祥(874年6月9日-934年9月7日),字保胤,邢州龙冈(今河北邢台市西南)人。后唐太祖李克用婿。他是五代十国时期后蜀開國皇帝(934年在位),在位不到1年,享壽61岁。

后唐建立后,以孟知祥为太原尹充西京副留守。

后唐灭前蜀后,任孟知祥为西川节度使。孟知祥不久即谋求独立,先和东川节度使董璋合作,击退后唐来讨伐的军队。932年,又与董璋决裂,董璋来攻打孟知祥,反被消灭。933年,孟知祥被后唐封为蜀王。934年正月,他不受后唐闵帝官爵,在成都即皇帝位,建国号“大蜀”,史称“后蜀”,改元“明德”。

孟知祥只做了7个月皇帝就病重了,臨終時,由第三子(按孟知祥夫妇墓志铭,实为第五子,孟知祥前两个儿子疑因早夭故未序齿)孟昶监国。孟知祥死后,秘不发丧,王处回与赵季良立孟昶后才发丧。孟知祥谥号为文武圣德英明孝昭烈武皇帝,庙号高祖。其墓称和陵,位于四川省成都市北郊青龙乡石岭村的磨盘山南。陵墓以青石砌筑,建22级阶梯通向墓室。墓呈圆锥形,主室高8.16米,直径6.7米,是在南方少有的具五代后蜀的北方草原建筑风格的陵墓。现已发掘。

\subsubsection{明德}

\begin{longtable}{|>{\centering\scriptsize}m{2em}|>{\centering\scriptsize}m{1.3em}|>{\centering}m{8.8em}|}
  % \caption{秦王政}\
  \toprule
  \SimHei \normalsize 年数 & \SimHei \scriptsize 公元 & \SimHei 大事件 \tabularnewline
  % \midrule
  \endfirsthead
  \toprule
  \SimHei \normalsize 年数 & \SimHei \scriptsize 公元 & \SimHei 大事件 \tabularnewline
  \midrule
  \endhead
  \midrule
  元年 & 934 & \tabularnewline\hline
  二年 & 935 & \tabularnewline\hline
  三年 & 936 & \tabularnewline\hline
  四年 & 937 & \tabularnewline
  \bottomrule
\end{longtable}



%%% Local Variables:
%%% mode: latex
%%% TeX-engine: xetex
%%% TeX-master: "../../Main"
%%% End:

%% -*- coding: utf-8 -*-
%% Time-stamp: <Chen Wang: 2021-11-01 15:49:00>

\subsection{后主孟昶\tiny(938-965)}

\subsubsection{生平}

孟昶chǎng(919年12月9日-965年7月12日),初名仁赞,表字保元。後蜀高祖孟知祥第三子(据孟知祥夫妇墓誌铭则为第五子,疑两位兄长因早夭未序齿),母李贵妃。後蜀末代皇帝(第二代,934年~964年在位),在位31年,享年47岁,史書作「後主」。

孟昶一般被称为后主。即位初年,即铲除桀骜不驯的宿将李仁罕,慑服傲慢的老将李肇,励精图治,衣着朴素,兴修水利,注重农桑,实行“与民休息”政策,後蜀国势强盛,将北线疆土扩张到长安。

但是他在位後期,貪圖逸樂、沉湎酒色,不思国政,生活荒淫,奢侈无度,连夜壶都用珍宝制成,称为“七宝溺器”,朝政十分腐败。[來源請求]

後蜀广政二十八年(965年),宋师在大将王全斌的指挥下以两路伐後蜀,蜀军与宋军在剑门关外进行一场大战,蜀军全军覆灭,後蜀精兵被全歼,灭亡之势已不可免了。宋军包围成都府,孟昶投降,後蜀灭亡。

孟昶被俘後被封为检校太师兼中书令、秦国公,居住在汴京。北宋乾德三年(965年),孟昶入开封七日后郁郁而终(一说被赵光义毒死),追封楚王,諡恭孝。

孟昶的寵妃花蕊夫人在亡國之後寫下了悲憤的詩句:“君王城上豎降旗,妾在深宮哪得知,十四萬人齊解甲,更無一個是男兒。”

孟昶注重吏民關係,曾頒布四句箴言,令刻於巨石上,爾後被宋太宗拿來引註於各種官箴。這四句箴言即是:「爾俸爾祿,民膏民脂;下民易虐,上天難欺!」亦是流傳至今的「官箴」之一。

因孟昶喜好音樂,善於作曲,被南管界尊為祖師爺,稱為孟府郎君。

有人認為送子張仙一神,實為花蕊夫人入趙宋後,紀念孟昶的偽稱。《金台紀聞》記載:世所傳「張仙像」者,乃蜀王孟昶挾彈圖也。初,花蕊夫人入宋宮,念其故主,偶攜此圖,遂懸於壁,且祀之謹。太祖幸而見之,致詰焉。夫人詭答之曰:「此蜀中張仙神也。祀之能令人有子。」

另說後蜀亡後,花蕊夫人或感念孟昶的百姓,以「二郎神」、「孟府郎君」等名義加以供奉。

張太華,原為最受寵之妃,深得后主宠爱,与后主孟昶同游于青城山时被霹雷震死。(王文才、王炎《蜀檮杌校箋》認為此說不確,張太華為明人小說家言。)

徐慧妃(花蕊夫人),張太華去世後最受寵之妃(五代蜀主孟昶宠愛慧妃徐氏,徐国璋的女儿,被蜀主封为慧妃,慧妃常与后主(孟昶)登楼,以龙脑末涂白扇。扇坠地,为人所得。蜀人争效其制,名曰“雪香扇”。见清吴任臣《十国春秋·後蜀·慧妃徐氏传》。涂以香料的白色扇子。宋陶谷《清异录·雪香扇》:“孟昶夏月水调龙脑末,涂白扇上,用以挥风。一夜,与花蕊夫人登楼望月,悮堕其扇,为人所得。外有效者,名雪香扇。”)

李豔娘,因献舞而入宫为妃,封为昭容,并赐其家人钱财十万。李豔娘好梳高髻,宫人皆学她以邀宠幸,也唤作“朝天髻”。《十国宫词》露台灯耀舞衣妍,一搦纤腰十万钱。进御乞颁新位号,梳将高髻学朝天。

\subsubsection{广政}

\begin{longtable}{|>{\centering\scriptsize}m{2em}|>{\centering\scriptsize}m{1.3em}|>{\centering}m{8.8em}|}
  % \caption{秦王政}\
  \toprule
  \SimHei \normalsize 年数 & \SimHei \scriptsize 公元 & \SimHei 大事件 \tabularnewline
  % \midrule
  \endfirsthead
  \toprule
  \SimHei \normalsize 年数 & \SimHei \scriptsize 公元 & \SimHei 大事件 \tabularnewline
  \midrule
  \endhead
  \midrule
  元年 & 938 & \tabularnewline\hline
  二年 & 939 & \tabularnewline\hline
  三年 & 940 & \tabularnewline\hline
  四年 & 941 & \tabularnewline\hline
  五年 & 942 & \tabularnewline\hline
  六年 & 943 & \tabularnewline\hline
  七年 & 944 & \tabularnewline\hline
  八年 & 945 & \tabularnewline\hline
  九年 & 946 & \tabularnewline\hline
  十年 & 947 & \tabularnewline\hline
  十一年 & 948 & \tabularnewline\hline
  十二年 & 949 & \tabularnewline\hline
  十三年 & 950 & \tabularnewline\hline
  十四年 & 951 & \tabularnewline\hline
  十五年 & 952 & \tabularnewline\hline
  十六年 & 953 & \tabularnewline\hline
  十七年 & 954 & \tabularnewline\hline
  十八年 & 955 & \tabularnewline\hline
  十九年 & 956 & \tabularnewline\hline
  二十年 & 957 & \tabularnewline\hline
  二一年 & 958 & \tabularnewline\hline
  二二年 & 959 & \tabularnewline\hline
  二三年 & 960 & \tabularnewline\hline
  二四年 & 961 & \tabularnewline\hline
  二五年 & 962 & \tabularnewline\hline
  二六年 & 963 & \tabularnewline\hline
  二七年 & 964 & \tabularnewline\hline
  二八年 & 965 & \tabularnewline
  \bottomrule
\end{longtable}



%%% Local Variables:
%%% mode: latex
%%% TeX-engine: xetex
%%% TeX-master: "../../Main"
%%% End:




%%% Local Variables:
%%% mode: latex
%%% TeX-engine: xetex
%%% TeX-master: "../../Main"
%%% End:

%% -*- coding: utf-8 -*-
%% Time-stamp: <Chen Wang: 2019-12-26 10:10:54>


\section{荆南\tiny(924-963)}

\subsection{简介}

荆南(924年-963年),又称南平、北楚,是五代时十国之一,但值得注意的是,荊南雖曰「十國」之一,但自始自終並未真正稱帝,事實上僅能算是一割據政權。高季兴所建。统治范围包括今湖北的江陵、公安一带。

后梁开平元年(907年)高季兴任荆南节度使。当时荆南所辖的10州为邻道侵夺,只有江陵一城。高季兴到任后,招集流亡,民渐复业,又收用一些文武官作辅佐,暗中准备割据。后唐同光二年(924年)封为南平王,建都荆州(今湖北荆州市荆州区)[來源請求],史称「南平」。又以方镇名为「荆南」,后世以此称之。

荆南虽地狭兵弱,但却是南北的交通要冲。其时南汉、闽、楚皆向后梁称臣,而每年贡奉均假道于荆南;因此高季兴便邀留使者,劫其财物。至南汉、闽、楚各称帝后,高氏对南北称帝诸国,一概上表称臣,以获取赏赐和维持商贸往来,由是被诸国视为“高赖子”。据有今湖北江陵、公安一带,建都荆州(今湖北江陵)。[來源請求]或后唐灭前蜀以后,高季兴得到了归、峡二州。他本欲夺取夔、忠、万等州,终不敌后唐而作罢。929年,高季兴死后,其子高从诲继位,重新向后唐称臣,因此后唐明宗始追封高季兴为楚王,谥武信(楚武信王),故南平又被称为北楚。后经高保融、高保勗,直到第五主高继冲,于宋太祖建隆四年荊湖之戰(963年)戰敗,纳地归降。

由高季兴公元907年担任节度使至荆南963年亡国,前后历五十七年。

%% -*- coding: utf-8 -*-
%% Time-stamp: <Chen Wang: 2021-11-01 15:49:20>

\subsection{武信王高季兴\tiny(924-929)}

\subsubsection{生平}

楚武信王高季兴(858年-929年),原名高季昌,因避后唐庄宗李存勗祖父李国昌的名讳,改为季兴,字贻孙,陕州峡石(今河南三门峡东南)人,五代十國时期荆南建立者(楚王),以江陵一城周旋于中原、诸侯之间,长于纵横之术。自称东魏司徒高昂之后。

高季兴幼年为汴州商人李七郎(李七郎后为朱全忠义子,改名朱友让)家奴,后为朱友让收为义子,改姓为朱,由此得入军门,为朱全忠亲随牙将,因破凤翔救唐昭宗有功,被唐昭宗授“迎銮毅勇功臣”之号,迁宋州刺史。又随朱全忠扫荡青州,累功升颍州防御使,并復姓高氏。907年朱全忠称帝,建立后梁,授其为荆南节度使。但由于战乱,本来荆南所辖八州只有江陵一城为季昌所领。开平二年,与雷彦恭、马殷、杨行密鏖战,互有胜负。加同中书门下平章事。开平四年,大败马楚于油口。后梁末帝朱友贞乾化三年(913年),被册封为渤海王。是年九月,季昌私造战舰五百艘,招募亡命之徒,与杨吴、前蜀交好,后梁对其失去控制。乾化四年(914年)春正月,季昌試圖攻取前蜀的夔、万、忠、涪四州,結果大敗而還。贞明三年,季昌修筑堤坝以防长江水患,荆南人称高氏堤。龙德元年,季昌命都指挥使倪可福修江陵外城。

至后唐灭梁,高季兴向后唐称臣,携300骑入朝觐见,升中书令。在唐时建议李存勗攻取前蜀,李存勗认为十分有道理。但险些为后唐所留,季兴丢弃辎重和随从,连夜赶路才侥幸逃脱。回到江陵后,季兴认为李存勗耽于声色,定不能长久,于是修缮城池,招纳梁军旧部,以备万全之策。后唐庄宗同光二年(924年),后唐帝李存勗为笼络其心,册封其为南平王。随即与后唐一同攻蜀,不克而还。而蜀国仍被灭亡。消息传至荆南,季兴为之前给后唐出计而感到十分懊悔。高季兴在荆南,常截留各国供品,或也为讨得赐物向诸国称臣,反复无常,时称「高赖子」。

其后后唐灭前蜀,未几,又逢后唐的鄴都之變,李存勗被杀,天成元年,季兴趁机向李嗣源要回了夔、忠、万、归、峡五州。高季兴截获蜀地入朝贡物,又厚颜向后唐索地,妄图扩大地盘,后唐明宗李嗣源怒其无耻,罢其官爵,发湖南和蜀地两地兵马来征,荆南不敌,辖地日蹙,求兵于南吴。后因江南雨季,粮草不济,后唐罢兵,方才逃此一劫,此后归顺南吴,得封秦王。后唐明宗天成三年(928年),马楚兴师进攻江陵,季兴不敌。九月,在白田击败楚军。然而后唐又兴兵来攻。十二月十五日(阳历为929年1月28日)高季兴病死,时年七十一。葬于江陵城西的龙山乡。其子高从诲继位,重新向后唐称臣,因此后唐始追封高季兴为楚王,谥武信(楚武信王),故南平又被称为北楚。

\subsubsection{同光}

\begin{longtable}{|>{\centering\scriptsize}m{2em}|>{\centering\scriptsize}m{1.3em}|>{\centering}m{8.8em}|}
  % \caption{秦王政}\
  \toprule
  \SimHei \normalsize 年数 & \SimHei \scriptsize 公元 & \SimHei 大事件 \tabularnewline
  % \midrule
  \endfirsthead
  \toprule
  \SimHei \normalsize 年数 & \SimHei \scriptsize 公元 & \SimHei 大事件 \tabularnewline
  \midrule
  \endhead
  \midrule
  元年 & 924 & \tabularnewline\hline
  二年 & 925 & \tabularnewline\hline
  三年 & 926 & \tabularnewline
  \bottomrule
\end{longtable}

\subsubsection{天成}

\begin{longtable}{|>{\centering\scriptsize}m{2em}|>{\centering\scriptsize}m{1.3em}|>{\centering}m{8.8em}|}
  % \caption{秦王政}\
  \toprule
  \SimHei \normalsize 年数 & \SimHei \scriptsize 公元 & \SimHei 大事件 \tabularnewline
  % \midrule
  \endfirsthead
  \toprule
  \SimHei \normalsize 年数 & \SimHei \scriptsize 公元 & \SimHei 大事件 \tabularnewline
  \midrule
  \endhead
  \midrule
  元年 & 926 & \tabularnewline\hline
  二年 & 927 & \tabularnewline\hline
  三年 & 928 & \tabularnewline
  \bottomrule
\end{longtable}

\subsubsection{乾贞}

\begin{longtable}{|>{\centering\scriptsize}m{2em}|>{\centering\scriptsize}m{1.3em}|>{\centering}m{8.8em}|}
  % \caption{秦王政}\
  \toprule
  \SimHei \normalsize 年数 & \SimHei \scriptsize 公元 & \SimHei 大事件 \tabularnewline
  % \midrule
  \endfirsthead
  \toprule
  \SimHei \normalsize 年数 & \SimHei \scriptsize 公元 & \SimHei 大事件 \tabularnewline
  \midrule
  \endhead
  \midrule
  元年 & 928 & \tabularnewline
  \bottomrule
\end{longtable}



%%% Local Variables:
%%% mode: latex
%%% TeX-engine: xetex
%%% TeX-master: "../../Main"
%%% End:

%% -*- coding: utf-8 -*-
%% Time-stamp: <Chen Wang: 2021-11-01 15:49:28>

\subsection{文献王高從誨\tiny(928-948)}

\subsubsection{生平}

高從誨(891年-948年),字遵聖,五代時期荊南君主(南平王)。為高季興之長子。曾仕於後梁中央政府,高季興為荊南節度使時,告歸其父,被高季興任命為馬步軍都指揮使、行軍司馬。

後唐明宗李嗣源天成三年(928年)十二月十五日(929年1月28日),被後唐冊封為南平王的高季興去世,高從誨嗣位。由於高季興在位末期曾與後唐決裂,並向南吳稱臣,而唐強吳弱、唐近吳遠,因此高從誨嗣位後,回歸向後唐稱臣,為後唐任命為荊南節度使,兼侍中。

長興三年(932年),被封為渤海王。後唐閔帝李從厚應順元年(934年)被改封南平王。

高從誨與部屬聊天,談到鄰國楚王馬希範非常奢侈。高從誨說:「像馬王那樣,可以說是大丈夫了」。孫光憲對答說:「天子諸侯,禮有等差。馬希範那種乳臭兒驕奢僭越,圖一時之樂,不謀遠慮,沒多久就要滅亡了,又有什麼好羨慕的!」高從誨久而悟,說:「你說的是」。之後,高從誨對梁震說:「我自己反省平生花費,實在太多了」。於是捨棄各種玩樂,以讀書作為消遣,省刑薄賦,南平國於是十分安寧。

荊南(南平)地狹兵弱,但因位處交通要道,每年各地區向中原政權的進貢,只要經過荊南,高季興、高從誨父子就會款待使者,掠奪財物,等到對方加以款待或讚賞,就把財物歸還,而且還會覺得這種行為很丟臉。後來後唐、後晉、遼國、後漢先後據有中原,南漢、閩國、南吳、南唐、後蜀皆稱帝,高從誨為求賞賜向他們都稱臣,所以各國都叫他們為「高賴子」或是「高無賴」。

後漢隱帝劉承祐乾祐元年(948年),高從誨去世,贈尚書令,諡文獻王。其子高保融繼位。

\subsubsection{乾贞}

\begin{longtable}{|>{\centering\scriptsize}m{2em}|>{\centering\scriptsize}m{1.3em}|>{\centering}m{8.8em}|}
  % \caption{秦王政}\
  \toprule
  \SimHei \normalsize 年数 & \SimHei \scriptsize 公元 & \SimHei 大事件 \tabularnewline
  % \midrule
  \endfirsthead
  \toprule
  \SimHei \normalsize 年数 & \SimHei \scriptsize 公元 & \SimHei 大事件 \tabularnewline
  \midrule
  \endhead
  \midrule
  元年 & 929 & \tabularnewline
  \bottomrule
\end{longtable}

\subsubsection{天成}

\begin{longtable}{|>{\centering\scriptsize}m{2em}|>{\centering\scriptsize}m{1.3em}|>{\centering}m{8.8em}|}
  % \caption{秦王政}\
  \toprule
  \SimHei \normalsize 年数 & \SimHei \scriptsize 公元 & \SimHei 大事件 \tabularnewline
  % \midrule
  \endfirsthead
  \toprule
  \SimHei \normalsize 年数 & \SimHei \scriptsize 公元 & \SimHei 大事件 \tabularnewline
  \midrule
  \endhead
  \midrule
  元年 & 929 & \tabularnewline\hline
  二年 & 930 & \tabularnewline
  \bottomrule
\end{longtable}

\subsubsection{长兴}

\begin{longtable}{|>{\centering\scriptsize}m{2em}|>{\centering\scriptsize}m{1.3em}|>{\centering}m{8.8em}|}
  % \caption{秦王政}\
  \toprule
  \SimHei \normalsize 年数 & \SimHei \scriptsize 公元 & \SimHei 大事件 \tabularnewline
  % \midrule
  \endfirsthead
  \toprule
  \SimHei \normalsize 年数 & \SimHei \scriptsize 公元 & \SimHei 大事件 \tabularnewline
  \midrule
  \endhead
  \midrule
  元年 & 930 & \tabularnewline\hline
  二年 & 931 & \tabularnewline\hline
  三年 & 932 & \tabularnewline\hline
  四年 & 933 & \tabularnewline
  \bottomrule
\end{longtable}

\subsubsection{应顺}

\begin{longtable}{|>{\centering\scriptsize}m{2em}|>{\centering\scriptsize}m{1.3em}|>{\centering}m{8.8em}|}
  % \caption{秦王政}\
  \toprule
  \SimHei \normalsize 年数 & \SimHei \scriptsize 公元 & \SimHei 大事件 \tabularnewline
  % \midrule
  \endfirsthead
  \toprule
  \SimHei \normalsize 年数 & \SimHei \scriptsize 公元 & \SimHei 大事件 \tabularnewline
  \midrule
  \endhead
  \midrule
  元年 & 934 & \tabularnewline
  \bottomrule
\end{longtable}

\subsubsection{清泰}

\begin{longtable}{|>{\centering\scriptsize}m{2em}|>{\centering\scriptsize}m{1.3em}|>{\centering}m{8.8em}|}
  % \caption{秦王政}\
  \toprule
  \SimHei \normalsize 年数 & \SimHei \scriptsize 公元 & \SimHei 大事件 \tabularnewline
  % \midrule
  \endfirsthead
  \toprule
  \SimHei \normalsize 年数 & \SimHei \scriptsize 公元 & \SimHei 大事件 \tabularnewline
  \midrule
  \endhead
  \midrule
  元年 & 934 & \tabularnewline\hline
  二年 & 935 & \tabularnewline\hline
  三年 & 936 & \tabularnewline
  \bottomrule
\end{longtable}

\subsubsection{天福}

\begin{longtable}{|>{\centering\scriptsize}m{2em}|>{\centering\scriptsize}m{1.3em}|>{\centering}m{8.8em}|}
  % \caption{秦王政}\
  \toprule
  \SimHei \normalsize 年数 & \SimHei \scriptsize 公元 & \SimHei 大事件 \tabularnewline
  % \midrule
  \endfirsthead
  \toprule
  \SimHei \normalsize 年数 & \SimHei \scriptsize 公元 & \SimHei 大事件 \tabularnewline
  \midrule
  \endhead
  \midrule
  元年 & 936 & \tabularnewline\hline
  二年 & 937 & \tabularnewline\hline
  三年 & 938 & \tabularnewline\hline
  四年 & 939 & \tabularnewline\hline
  五年 & 940 & \tabularnewline\hline
  六年 & 941 & \tabularnewline\hline
  七年 & 942 & \tabularnewline\hline
  八年 & 943 & \tabularnewline\hline
  九年 & 944 & \tabularnewline
  \bottomrule
\end{longtable}

\subsubsection{开运}

\begin{longtable}{|>{\centering\scriptsize}m{2em}|>{\centering\scriptsize}m{1.3em}|>{\centering}m{8.8em}|}
  % \caption{秦王政}\
  \toprule
  \SimHei \normalsize 年数 & \SimHei \scriptsize 公元 & \SimHei 大事件 \tabularnewline
  % \midrule
  \endfirsthead
  \toprule
  \SimHei \normalsize 年数 & \SimHei \scriptsize 公元 & \SimHei 大事件 \tabularnewline
  \midrule
  \endhead
  \midrule
  元年 & 944 & \tabularnewline\hline
  二年 & 945 & \tabularnewline\hline
  三年 & 946 & \tabularnewline
  \bottomrule
\end{longtable}

\subsubsection{天复}

\begin{longtable}{|>{\centering\scriptsize}m{2em}|>{\centering\scriptsize}m{1.3em}|>{\centering}m{8.8em}|}
  % \caption{秦王政}\
  \toprule
  \SimHei \normalsize 年数 & \SimHei \scriptsize 公元 & \SimHei 大事件 \tabularnewline
  % \midrule
  \endfirsthead
  \toprule
  \SimHei \normalsize 年数 & \SimHei \scriptsize 公元 & \SimHei 大事件 \tabularnewline
  \midrule
  \endhead
  \midrule
  元年 & 947 & \tabularnewline
  \bottomrule
\end{longtable}

\subsubsection{乾佑}

\begin{longtable}{|>{\centering\scriptsize}m{2em}|>{\centering\scriptsize}m{1.3em}|>{\centering}m{8.8em}|}
  % \caption{秦王政}\
  \toprule
  \SimHei \normalsize 年数 & \SimHei \scriptsize 公元 & \SimHei 大事件 \tabularnewline
  % \midrule
  \endfirsthead
  \toprule
  \SimHei \normalsize 年数 & \SimHei \scriptsize 公元 & \SimHei 大事件 \tabularnewline
  \midrule
  \endhead
  \midrule
  元年 & 948 & \tabularnewline
  \bottomrule
\end{longtable}



%%% Local Variables:
%%% mode: latex
%%% TeX-engine: xetex
%%% TeX-master: "../../Main"
%%% End:

%% -*- coding: utf-8 -*-
%% Time-stamp: <Chen Wang: 2019-12-26 10:11:49>

\subsection{贞懿王\tiny(948-960)}

\subsubsection{生平}

南平貞懿王高保融(920年-960年),字德長,五代時期荊南君主(南平王)。為高從誨之第三子。

後漢隱帝劉承祐乾祐元年(948年),南平王高從誨去世,高保融繼位。不久,即被後漢任命為荊南節度使、同平章事、兼侍中。後周太祖郭威廣順元年(951年),被封為渤海郡王。顯德元年(954年),再被進封為南平王。

高保融個性遲鈍緩慢,沒有什麼才能,無論事情大小,皆委由其弟高保勗決定。宋太祖趙匡胤建隆元年(960年),宋朝建立後,高保融愈發感到恐懼,因此一年之間三次進貢。同年,因病去世,贈太尉,諡貞懿王。其子高繼沖年紀尚小,因此遺命高保勗繼位。

\subsubsection{乾佑}

\begin{longtable}{|>{\centering\scriptsize}m{2em}|>{\centering\scriptsize}m{1.3em}|>{\centering}m{8.8em}|}
  % \caption{秦王政}\
  \toprule
  \SimHei \normalsize 年数 & \SimHei \scriptsize 公元 & \SimHei 大事件 \tabularnewline
  % \midrule
  \endfirsthead
  \toprule
  \SimHei \normalsize 年数 & \SimHei \scriptsize 公元 & \SimHei 大事件 \tabularnewline
  \midrule
  \endhead
  \midrule
  元年 & 948 & \tabularnewline\hline
  二年 & 949 & \tabularnewline\hline
  三年 & 950 & \tabularnewline
  \bottomrule
\end{longtable}

\subsubsection{广顺}

\begin{longtable}{|>{\centering\scriptsize}m{2em}|>{\centering\scriptsize}m{1.3em}|>{\centering}m{8.8em}|}
  % \caption{秦王政}\
  \toprule
  \SimHei \normalsize 年数 & \SimHei \scriptsize 公元 & \SimHei 大事件 \tabularnewline
  % \midrule
  \endfirsthead
  \toprule
  \SimHei \normalsize 年数 & \SimHei \scriptsize 公元 & \SimHei 大事件 \tabularnewline
  \midrule
  \endhead
  \midrule
  元年 & 951 & \tabularnewline\hline
  二年 & 952 & \tabularnewline\hline
  三年 & 953 & \tabularnewline
  \bottomrule
\end{longtable}

\subsubsection{显德}

\begin{longtable}{|>{\centering\scriptsize}m{2em}|>{\centering\scriptsize}m{1.3em}|>{\centering}m{8.8em}|}
  % \caption{秦王政}\
  \toprule
  \SimHei \normalsize 年数 & \SimHei \scriptsize 公元 & \SimHei 大事件 \tabularnewline
  % \midrule
  \endfirsthead
  \toprule
  \SimHei \normalsize 年数 & \SimHei \scriptsize 公元 & \SimHei 大事件 \tabularnewline
  \midrule
  \endhead
  \midrule
  元年 & 954 & \tabularnewline\hline
  二年 & 955 & \tabularnewline\hline
  三年 & 956 & \tabularnewline\hline
  四年 & 957 & \tabularnewline\hline
  五年 & 958 & \tabularnewline\hline
  六年 & 959 & \tabularnewline\hline
  七年 & 960 & \tabularnewline
  \bottomrule
\end{longtable}



%%% Local Variables:
%%% mode: latex
%%% TeX-engine: xetex
%%% TeX-master: "../../Main"
%%% End:

%% -*- coding: utf-8 -*-
%% Time-stamp: <Chen Wang: 2019-12-26 10:12:04>

\subsection{高保勗\tiny(960-962)}

\subsubsection{生平}

高保勗(924年-962年),字省躬,五代時期荊南君主(荊南節度使)。為高從誨之第十子,高保融之弟。高保勗幼時為高從誨所喜愛,高從誨因事盛怒,見到高保勗必釋然而笑,是故百姓稱之為「萬事休」。

宋太祖建隆元年(960年),高保融因病去世,其子高繼沖年紀尚小,因此遺命高保勗繼位,總判內外軍馬事。不久,即為宋任命為荊南節度使。

高保勗少時多病,體態瘦弱,但頗有治事之才。然而繼位後,放縱荒淫而沒有節制,白天召娼妓至官府,而挑選強壯的士兵,命其隨便調戲淫謔,然後自己再和姬妾一同觀賞做為娛樂。又喜歡營造亭台樓閣,花費人力物力無數,而不理國政,人民都很不滿。

建隆三年(962年),高保勗因病去世,被宋朝贈侍中。遺命其姪即高保融之子高繼沖嗣位。

高保勗去世後數月,南平即為宋所滅,有附會者即以其綽號「萬事休」為預兆。

\subsubsection{建隆}

\begin{longtable}{|>{\centering\scriptsize}m{2em}|>{\centering\scriptsize}m{1.3em}|>{\centering}m{8.8em}|}
  % \caption{秦王政}\
  \toprule
  \SimHei \normalsize 年数 & \SimHei \scriptsize 公元 & \SimHei 大事件 \tabularnewline
  % \midrule
  \endfirsthead
  \toprule
  \SimHei \normalsize 年数 & \SimHei \scriptsize 公元 & \SimHei 大事件 \tabularnewline
  \midrule
  \endhead
  \midrule
  元年 & 960 & \tabularnewline\hline
  二年 & 961 & \tabularnewline\hline
  三年 & 962 & \tabularnewline
  \bottomrule
\end{longtable}




%%% Local Variables:
%%% mode: latex
%%% TeX-engine: xetex
%%% TeX-master: "../../Main"
%%% End:

%% -*- coding: utf-8 -*-
%% Time-stamp: <Chen Wang: 2019-12-26 10:12:19>

\subsection{高继冲\tiny(962-963)}

\subsubsection{生平}

高繼沖(943年-973年),字成和(一作字贊平),五代十國荊南政權末期君主(荊南節度使)。為高保融之長子,高保勗之姪。

宋太祖趙匡胤建隆三年(962年),高保勗因病去世,遺命高繼沖權判內外軍馬事以繼其位。後來高繼沖亦被宋任命為荊南節度使。

同年,湖南的武平節度使周行逢亦去世,年僅11歲的周保權嗣位,而境內大將張文表叛變,周保權向宋朝求援。建隆四年(963年),宋軍討伐張文表,假道荊南,發生荊湖之戰,趁機控制南平都城江陵(今湖北江陵)城巷,高繼沖只得納地以歸,南平亡。

南平亡後,宋一度仍任命高繼沖為荊南節度使。不久,高繼沖舉族歸朝,被改命為武寧節度使(約在今江蘇、安徽一帶)。

開寶六年(973年)高繼沖於武寧節度使任內去世,贈侍中。高繼沖鎮守彭門(今江蘇徐州),政事委諸僚佐,然有德政,因此被百姓請求留葬當地,但不被宋太祖允許。

\subsubsection{建隆}

\begin{longtable}{|>{\centering\scriptsize}m{2em}|>{\centering\scriptsize}m{1.3em}|>{\centering}m{8.8em}|}
  % \caption{秦王政}\
  \toprule
  \SimHei \normalsize 年数 & \SimHei \scriptsize 公元 & \SimHei 大事件 \tabularnewline
  % \midrule
  \endfirsthead
  \toprule
  \SimHei \normalsize 年数 & \SimHei \scriptsize 公元 & \SimHei 大事件 \tabularnewline
  \midrule
  \endhead
  \midrule
  元年 & 962 & \tabularnewline\hline
  二年 & 963 & \tabularnewline
  \bottomrule
\end{longtable}




%%% Local Variables:
%%% mode: latex
%%% TeX-engine: xetex
%%% TeX-master: "../../Main"
%%% End:



%%% Local Variables:
%%% mode: latex
%%% TeX-engine: xetex
%%% TeX-master: "../../Main"
%%% End:

%% -*- coding: utf-8 -*-
%% Time-stamp: <Chen Wang: 2019-12-26 10:14:55>


\section{北汉\tiny(951-979)}

\subsection{简介}

北漢(951年-979年)是中国五代十国時在今山西省北部、陕西省、河北省局部的政權,為十国之一。

951年,后汉被郭威所篡,改国号周,史称后周。郭威并废杀原本将被立为汉帝的後漢高祖劉知遠的养子,也是高祖弟鎮守晉陽的河东节度使刘崇的嫡长子刘赟。刘崇原本以为儿子将被拥立为帝而按兵不动,得知儿子死讯后在太原继位,繼承後漢,但国家疆域和地位已发生巨大变化,史学家将其定位为新政权或残余政权,为别于后汉和南方的南汉,史称北汉。又以所统治的山西古称河东,而被称为“东汉”(如欧阳修《新五代史·東漢世家》)。

统治范围包括今山西北部、陕西、河北部分地区。

为与后周抗衡,曾向辽朝请封,援后晋故事,自称侄皇帝。

北漢国兵役繁重,与后周、北宋进行多次的战争,人口锐减到只有盛唐时的八分之一。

北漢最後在979年宋太宗年間被包围,杨业归宋后,太原城内军心动摇,最终投降,宋太宗在战事中损兵折将,气愤之下将太原城平毁再引汾、晋二水灌城,给屡遭战火的北方百姓又带来严重的损失。

%% -*- coding: utf-8 -*-
%% Time-stamp: <Chen Wang: 2021-11-01 15:50:53>

\subsection{世祖劉旻\tiny(951-954)}

\subsubsection{生平}

漢世祖劉旻(895年-954年),並州晉陽(今山西太原)人,沙陀族,原名刘彦崇、劉崇,五代十國時期北漢開國皇帝,為後漢高祖劉知遠之弟,父劉琠。

劉崇年輕時喜歡飲酒賭博,曾經於臉上刺青從軍。劉知遠於後晉任河東節度使時,他擔任都指揮使。劉知遠建後漢之後,任太原尹,後漢隱帝劉承祐在位時任河東節度使(位於太原),鎮守河東地區。

後漢乾祐三年(950年),樞密使郭威為劉承祐逼反,進軍後漢都城大梁(今河南開封),劉承祐逃亡中為下屬郭允明所殺,郭威遂控制朝政。劉崇此時原欲舉兵南下,但聽到郭威計畫迎立劉崇之長子武寧節度使劉贇為帝,遂打消此意。然而不久郭威被黃旗加身後自登帝位,建立後周,改元廣順,並殺劉贇,當時為951年。因此劉崇隨即亦在太原(今山西省太原市)登帝位,延續後漢,改名劉旻,仍維持乾祐年號,稱乾祐四年。後世把劉崇稱帝後的政權稱作北漢。

北漢地小民貧,又以興復後漢為業,遂向遼國乞援,與遼國約為父子之國,由劉旻稱遼帝為叔,而自稱姪皇帝;遼國則封劉旻為大漢神武皇帝。北漢因遼國的援助,而與後周進行了不少戰爭,但仍勝少敗多。北漢乾祐七年(954年),趁郭威去世之際,聯合遼國南攻後周,然為後周世宗柴榮率軍敗於高平,劉旻穿著農人的衣服隨百餘騎逃走,途中一度迷路,劉旻年老力衰,差點無法支撐回到太原。

經此一役,北漢元氣大傷,無力南下,而劉旻亦憂憤成疾,不久去世,廟號世祖,次子劉承鈞繼位。

\subsubsection{乾佑}

\begin{longtable}{|>{\centering\scriptsize}m{2em}|>{\centering\scriptsize}m{1.3em}|>{\centering}m{8.8em}|}
  % \caption{秦王政}\
  \toprule
  \SimHei \normalsize 年数 & \SimHei \scriptsize 公元 & \SimHei 大事件 \tabularnewline
  % \midrule
  \endfirsthead
  \toprule
  \SimHei \normalsize 年数 & \SimHei \scriptsize 公元 & \SimHei 大事件 \tabularnewline
  \midrule
  \endhead
  \midrule
  元年 & 951 & \tabularnewline\hline
  二年 & 952 & \tabularnewline\hline
  三年 & 953 & \tabularnewline\hline
  四年 & 954 & \tabularnewline
  \bottomrule
\end{longtable}


%%% Local Variables:
%%% mode: latex
%%% TeX-engine: xetex
%%% TeX-master: "../../Main"
%%% End:

%% -*- coding: utf-8 -*-
%% Time-stamp: <Chen Wang: 2019-12-26 10:14:15>

\subsection{睿宗\tiny(954-968)}

\subsubsection{生平}

汉睿宗劉鈞(926年-968年),原名劉承鈞,五代時期北漢在位最久的君主,為劉旻之次子。

劉承鈞個性孝順恭謹,喜歡讀書,擅長書法,北漢乾祐七年(954年),劉旻去世,劉承鈞為遼國冊封為帝之後繼位,不改年號,改名劉鈞。上表於遼帝時都自稱「男」,遼帝下詔時,都稱呼他「兒皇帝」。

劉鈞繼位後,勤政愛民,禮敬士大夫,任用郭無為為相,並減少南侵,因此境內還算安定。然而劉鈞並不像其父事奉遼國之恭敬,以致在位後期遼國援助漸少。

劉鈞於957年,改元天會。天會十二年(968年)忧郁死,諡孝和皇帝,廟號睿宗,劉鈞的外甥同時也是養子的劉繼恩繼位。

\subsubsection{乾佑}

\begin{longtable}{|>{\centering\scriptsize}m{2em}|>{\centering\scriptsize}m{1.3em}|>{\centering}m{8.8em}|}
  % \caption{秦王政}\
  \toprule
  \SimHei \normalsize 年数 & \SimHei \scriptsize 公元 & \SimHei 大事件 \tabularnewline
  % \midrule
  \endfirsthead
  \toprule
  \SimHei \normalsize 年数 & \SimHei \scriptsize 公元 & \SimHei 大事件 \tabularnewline
  \midrule
  \endhead
  \midrule
  元年 & 954 & \tabularnewline\hline
  二年 & 955 & \tabularnewline\hline
  三年 & 956 & \tabularnewline
  \bottomrule
\end{longtable}

\subsubsection{天会}

\begin{longtable}{|>{\centering\scriptsize}m{2em}|>{\centering\scriptsize}m{1.3em}|>{\centering}m{8.8em}|}
  % \caption{秦王政}\
  \toprule
  \SimHei \normalsize 年数 & \SimHei \scriptsize 公元 & \SimHei 大事件 \tabularnewline
  % \midrule
  \endfirsthead
  \toprule
  \SimHei \normalsize 年数 & \SimHei \scriptsize 公元 & \SimHei 大事件 \tabularnewline
  \midrule
  \endhead
  \midrule
  元年 & 957 & \tabularnewline\hline
  二年 & 958 & \tabularnewline\hline
  三年 & 959 & \tabularnewline\hline
  四年 & 960 & \tabularnewline\hline
  五年 & 961 & \tabularnewline\hline
  六年 & 962 & \tabularnewline\hline
  七年 & 963 & \tabularnewline\hline
  八年 & 964 & \tabularnewline\hline
  九年 & 965 & \tabularnewline\hline
  十年 & 966 & \tabularnewline\hline
  十一年 & 967 & \tabularnewline\hline
  十二年 & 968 & \tabularnewline\hline
  \bottomrule
\end{longtable}


%%% Local Variables:
%%% mode: latex
%%% TeX-engine: xetex
%%% TeX-master: "../../Main"
%%% End:

%% -*- coding: utf-8 -*-
%% Time-stamp: <Chen Wang: 2019-12-26 10:14:40>

\subsection{少主\tiny(968)}

\subsubsection{生平}

劉繼恩(10世纪?-968年),本姓薛,五代時期北漢君主,世祖刘旻外孙,睿宗刘钧外甥、养子,史稱「少主」。

其母刘氏為劉旻之女,因此劉鈞是他的舅父;其父薛釗,本來只是士兵,早年娶劉旻之女,劉旻之兄劉知遠發跡後,薛釗就很難見到其妻,因此很不快樂,有一天乘著酒意求見,竟拿刀刺傷其妻,薛釗後來因此自盡。當時劉繼恩年紀還小,劉旻因為劉鈞無子,遂命劉鈞收養劉繼恩。

劉繼恩在劉鈞在位時任太原尹,然而他本人資質平庸,劉鈞生前也曾向郭無為說過劉繼恩不是濟世之才。北漢天會十二年(968年)劉鈞病逝,劉繼恩繼位,仍以天會為年號,不久就將郭無為的權力架空。兩個月後,劉繼恩於酒宴後被供奉官侯霸榮刺殺[1],死後無諡號及廟號,史家習稱其為少主。

\subsubsection{天会}

\begin{longtable}{|>{\centering\scriptsize}m{2em}|>{\centering\scriptsize}m{1.3em}|>{\centering}m{8.8em}|}
  % \caption{秦王政}\
  \toprule
  \SimHei \normalsize 年数 & \SimHei \scriptsize 公元 & \SimHei 大事件 \tabularnewline
  % \midrule
  \endfirsthead
  \toprule
  \SimHei \normalsize 年数 & \SimHei \scriptsize 公元 & \SimHei 大事件 \tabularnewline
  \midrule
  \endhead
  \midrule
  元年 & 968 & \tabularnewline
  \bottomrule
\end{longtable}


%%% Local Variables:
%%% mode: latex
%%% TeX-engine: xetex
%%% TeX-master: "../../Main"
%%% End:

%% -*- coding: utf-8 -*-
%% Time-stamp: <Chen Wang: 2019-12-26 10:15:18>

\subsection{英武帝\tiny(968-979)}

\subsubsection{生平}

漢英武帝劉繼元(10世纪?-992年),本姓何,五代時期北漢君主,世祖刘旻外孙,睿宗刘钧外甥、养子。

其母刘氏為劉旻之女,因此劉鈞是他的舅父。母亲刘氏在劉繼恩之父薛釗自殺後改嫁何氏而生下劉繼元,所以劉繼恩是他的同母異父之兄,劉繼元的父母都去世以後,劉鈞收他為養子。

劉繼元在劉繼恩在位時任太原尹,北漢天會十二年(968年)劉繼恩為侯霸榮刺殺後,劉繼元被司空郭無為迎立為帝,繼位後即緩和與遼國間的緊張關係。劉繼元為人殘忍嗜殺,嫡母劉承鈞之妻郭皇后及劉旻之子皆被其所殺;亦動輒將忤逆他的臣屬滅族。

天會十三年(969年)宋太祖趙匡胤親征北漢,宋軍久攻不下而退兵,北漢收取宋軍所拋棄輜重,瀕臨枯竭的國力賴以恢復。

974年,改年號廣運。廣運六年(宋太平興國四年,979年),宋朝將南方各國併入版圖之後,再度決意北伐,由宋太宗趙光義親征,宋軍攻勢猛烈,遼國援軍亦被擊退,五月初六日劉繼元投降,北漢亡。投降後被任命為右衛上將軍,封彭城郡公。太平興國六年(981年),進封為彭城公。雍熙三年(986年),再被任命為保康軍節度使。淳化二年十二月十八日(陽曆為992年1月25日)去世,被贈中書令,追封為彭城郡王。

\subsubsection{天会}

\begin{longtable}{|>{\centering\scriptsize}m{2em}|>{\centering\scriptsize}m{1.3em}|>{\centering}m{8.8em}|}
  % \caption{秦王政}\
  \toprule
  \SimHei \normalsize 年数 & \SimHei \scriptsize 公元 & \SimHei 大事件 \tabularnewline
  % \midrule
  \endfirsthead
  \toprule
  \SimHei \normalsize 年数 & \SimHei \scriptsize 公元 & \SimHei 大事件 \tabularnewline
  \midrule
  \endhead
  \midrule
  元年 & 968 & \tabularnewline\hline
  二年 & 969 & \tabularnewline\hline
  三年 & 970 & \tabularnewline\hline
  四年 & 971 & \tabularnewline\hline
  五年 & 972 & \tabularnewline\hline
  六年 & 973 & \tabularnewline
  \bottomrule
\end{longtable}

\subsubsection{广运}

\begin{longtable}{|>{\centering\scriptsize}m{2em}|>{\centering\scriptsize}m{1.3em}|>{\centering}m{8.8em}|}
  % \caption{秦王政}\
  \toprule
  \SimHei \normalsize 年数 & \SimHei \scriptsize 公元 & \SimHei 大事件 \tabularnewline
  % \midrule
  \endfirsthead
  \toprule
  \SimHei \normalsize 年数 & \SimHei \scriptsize 公元 & \SimHei 大事件 \tabularnewline
  \midrule
  \endhead
  \midrule
  元年 & 974 & \tabularnewline\hline
  二年 & 975 & \tabularnewline\hline
  三年 & 976 & \tabularnewline\hline
  四年 & 977 & \tabularnewline\hline
  五年 & 978 & \tabularnewline\hline
  六年 & 979 & \tabularnewline
  \bottomrule
\end{longtable}


%%% Local Variables:
%%% mode: latex
%%% TeX-engine: xetex
%%% TeX-master: "../../Main"
%%% End:



%%% Local Variables:
%%% mode: latex
%%% TeX-engine: xetex
%%% TeX-master: "../../Main"
%%% End:



%%% Local Variables:
%%% mode: latex
%%% TeX-engine: xetex
%%% TeX-master: "../Main"
%%% End:
 % 十国
% %% -*- coding: utf-8 -*-
%% Time-stamp: <Chen Wang: 2019-10-15 11:20:46>

\chapter{北宋\tiny(960-1127)}

%% -*- coding: utf-8 -*-
%% Time-stamp: <Chen Wang: 2018-07-12 01:13:53>

\section{太祖\tiny(960-976)}

\subsection{建隆}


\begin{longtable}{|>{\centering\scriptsize}m{2em}|>{\centering\scriptsize}m{1.3em}|>{\centering}m{8.8em}|}
  % \caption{秦王政}\
  \toprule
  \SimHei \normalsize 年数 & \SimHei \scriptsize 公元 & \SimHei 大事件 \tabularnewline
  % \midrule
  \endfirsthead
  \toprule
  \SimHei \normalsize 年数 & \SimHei \scriptsize 公元 & \SimHei 大事件 \tabularnewline
  \midrule
  \endhead
  \midrule
  元年 & 960 & \tabularnewline\hline
  二年 & 961 & \tabularnewline\hline
  三年 & 962 & \tabularnewline\hline
  四年 & 963 & \tabularnewline
  \bottomrule
\end{longtable}

\subsection{乾德}

\begin{longtable}{|>{\centering\scriptsize}m{2em}|>{\centering\scriptsize}m{1.3em}|>{\centering}m{8.8em}|}
  % \caption{秦王政}\
  \toprule
  \SimHei \normalsize 年数 & \SimHei \scriptsize 公元 & \SimHei 大事件 \tabularnewline
  % \midrule
  \endfirsthead
  \toprule
  \SimHei \normalsize 年数 & \SimHei \scriptsize 公元 & \SimHei 大事件 \tabularnewline
  \midrule
  \endhead
  \midrule
  元年 & 963 & \tabularnewline\hline
  二年 & 964 & \tabularnewline\hline
  三年 & 965 & \tabularnewline\hline
  四年 & 966 & \tabularnewline\hline
  五年 & 967 & \tabularnewline\hline
  六年 & 968 & \tabularnewline
  \bottomrule
\end{longtable}

\subsection{开宝}

\begin{longtable}{|>{\centering\scriptsize}m{2em}|>{\centering\scriptsize}m{1.3em}|>{\centering}m{8.8em}|}
  % \caption{秦王政}\
  \toprule
  \SimHei \normalsize 年数 & \SimHei \scriptsize 公元 & \SimHei 大事件 \tabularnewline
  % \midrule
  \endfirsthead
  \toprule
  \SimHei \normalsize 年数 & \SimHei \scriptsize 公元 & \SimHei 大事件 \tabularnewline
  \midrule
  \endhead
  \midrule
  元年 & 968 & \tabularnewline\hline
  二年 & 969 & \tabularnewline\hline
  三年 & 970 & \tabularnewline\hline
  四年 & 971 & \tabularnewline\hline
  五年 & 972 & \tabularnewline\hline
  六年 & 973 & \tabularnewline\hline
  七年 & 974 & \tabularnewline\hline
  八年 & 975 & \tabularnewline\hline
  九年 & 976 & \tabularnewline
  \bottomrule
\end{longtable}


%%% Local Variables:
%%% mode: latex
%%% TeX-engine: xetex
%%% TeX-master: "../Main"
%%% End:

%% -*- coding: utf-8 -*-
%% Time-stamp: <Chen Wang: 2018-07-12 11:30:58>

\section{太宗\tiny(976-997)}

\subsection{太平兴国}


\begin{longtable}{|>{\centering\scriptsize}m{2em}|>{\centering\scriptsize}m{1.3em}|>{\centering}m{8.8em}|}
  % \caption{秦王政}\
  \toprule
  \SimHei \normalsize 年数 & \SimHei \scriptsize 公元 & \SimHei 大事件 \tabularnewline
  % \midrule
  \endfirsthead
  \toprule
  \SimHei \normalsize 年数 & \SimHei \scriptsize 公元 & \SimHei 大事件 \tabularnewline
  \midrule
  \endhead
  \midrule
  元年 & 976 & \tabularnewline\hline
  二年 & 977 & \tabularnewline\hline
  三年 & 978 & \tabularnewline\hline
  四年 & 979 & \tabularnewline\hline
  五年 & 980 & \tabularnewline\hline
  六年 & 981 & \tabularnewline\hline
  七年 & 982 & \tabularnewline\hline
  八年 & 983 & \tabularnewline\hline
  九年 & 984 & \tabularnewline
  \bottomrule
\end{longtable}

\subsection{雍熙}

\begin{longtable}{|>{\centering\scriptsize}m{2em}|>{\centering\scriptsize}m{1.3em}|>{\centering}m{8.8em}|}
  % \caption{秦王政}\
  \toprule
  \SimHei \normalsize 年数 & \SimHei \scriptsize 公元 & \SimHei 大事件 \tabularnewline
  % \midrule
  \endfirsthead
  \toprule
  \SimHei \normalsize 年数 & \SimHei \scriptsize 公元 & \SimHei 大事件 \tabularnewline
  \midrule
  \endhead
  \midrule
  元年 & 984 & \tabularnewline\hline
  二年 & 985 & \tabularnewline\hline
  三年 & 986 & \tabularnewline\hline
  四年 & 987 & \tabularnewline
  \bottomrule
\end{longtable}

\subsection{端拱}

\begin{longtable}{|>{\centering\scriptsize}m{2em}|>{\centering\scriptsize}m{1.3em}|>{\centering}m{8.8em}|}
  % \caption{秦王政}\
  \toprule
  \SimHei \normalsize 年数 & \SimHei \scriptsize 公元 & \SimHei 大事件 \tabularnewline
  % \midrule
  \endfirsthead
  \toprule
  \SimHei \normalsize 年数 & \SimHei \scriptsize 公元 & \SimHei 大事件 \tabularnewline
  \midrule
  \endhead
  \midrule
  元年 & 988 & \tabularnewline\hline
  二年 & 989 & \tabularnewline
  \bottomrule
\end{longtable}

\subsection{淳化}

\begin{longtable}{|>{\centering\scriptsize}m{2em}|>{\centering\scriptsize}m{1.3em}|>{\centering}m{8.8em}|}
  % \caption{秦王政}\
  \toprule
  \SimHei \normalsize 年数 & \SimHei \scriptsize 公元 & \SimHei 大事件 \tabularnewline
  % \midrule
  \endfirsthead
  \toprule
  \SimHei \normalsize 年数 & \SimHei \scriptsize 公元 & \SimHei 大事件 \tabularnewline
  \midrule
  \endhead
  \midrule
  元年 & 990 & \tabularnewline\hline
  二年 & 991 & \tabularnewline\hline
  三年 & 992 & \tabularnewline\hline
  四年 & 993 & \tabularnewline\hline
  五年 & 994 & \tabularnewline
  \bottomrule
\end{longtable}

\subsection{至道}

\begin{longtable}{|>{\centering\scriptsize}m{2em}|>{\centering\scriptsize}m{1.3em}|>{\centering}m{8.8em}|}
  % \caption{秦王政}\
  \toprule
  \SimHei \normalsize 年数 & \SimHei \scriptsize 公元 & \SimHei 大事件 \tabularnewline
  % \midrule
  \endfirsthead
  \toprule
  \SimHei \normalsize 年数 & \SimHei \scriptsize 公元 & \SimHei 大事件 \tabularnewline
  \midrule
  \endhead
  \midrule
  元年 & 995 & \tabularnewline\hline
  二年 & 996 & \tabularnewline\hline
  三年 & 997 & \tabularnewline
  \bottomrule
\end{longtable}


%%% Local Variables:
%%% mode: latex
%%% TeX-engine: xetex
%%% TeX-master: "../Main"
%%% End:

%% -*- coding: utf-8 -*-
%% Time-stamp: <Chen Wang: 2018-07-12 12:53:38>

\section{真宗\tiny(997-1022)}

\subsection{咸平}


\begin{longtable}{|>{\centering\scriptsize}m{2em}|>{\centering\scriptsize}m{1.3em}|>{\centering}m{8.8em}|}
  % \caption{秦王政}\
  \toprule
  \SimHei \normalsize 年数 & \SimHei \scriptsize 公元 & \SimHei 大事件 \tabularnewline
  % \midrule
  \endfirsthead
  \toprule
  \SimHei \normalsize 年数 & \SimHei \scriptsize 公元 & \SimHei 大事件 \tabularnewline
  \midrule
  \endhead
  \midrule
  元年 & 998 & \tabularnewline\hline
  二年 & 999 & \tabularnewline\hline
  三年 & 1000 & \tabularnewline\hline
  四年 & 1001 & \tabularnewline\hline
  五年 & 1002 & \tabularnewline\hline
  六年 & 1003 & \tabularnewline
  \bottomrule
\end{longtable}

\subsection{景德}

\begin{longtable}{|>{\centering\scriptsize}m{2em}|>{\centering\scriptsize}m{1.3em}|>{\centering}m{8.8em}|}
  % \caption{秦王政}\
  \toprule
  \SimHei \normalsize 年数 & \SimHei \scriptsize 公元 & \SimHei 大事件 \tabularnewline
  % \midrule
  \endfirsthead
  \toprule
  \SimHei \normalsize 年数 & \SimHei \scriptsize 公元 & \SimHei 大事件 \tabularnewline
  \midrule
  \endhead
  \midrule
  元年 & 1004 & \tabularnewline\hline
  二年 & 1005 & \tabularnewline\hline
  三年 & 1006 & \tabularnewline\hline
  四年 & 1007 & \tabularnewline
  \bottomrule
\end{longtable}

\subsection{大中祥符}

\begin{longtable}{|>{\centering\scriptsize}m{2em}|>{\centering\scriptsize}m{1.3em}|>{\centering}m{8.8em}|}
  % \caption{秦王政}\
  \toprule
  \SimHei \normalsize 年数 & \SimHei \scriptsize 公元 & \SimHei 大事件 \tabularnewline
  % \midrule
  \endfirsthead
  \toprule
  \SimHei \normalsize 年数 & \SimHei \scriptsize 公元 & \SimHei 大事件 \tabularnewline
  \midrule
  \endhead
  \midrule
  元年 & 1008 & \tabularnewline\hline
  二年 & 1009 & \tabularnewline\hline
  三年 & 1010 & \tabularnewline\hline
  四年 & 1011 & \tabularnewline\hline
  五年 & 1012 & \tabularnewline\hline
  六年 & 1013 & \tabularnewline\hline
  七年 & 1014 & \tabularnewline\hline
  八年 & 1015 & \tabularnewline\hline
  九年 & 1016 & \tabularnewline
  \bottomrule
\end{longtable}

\subsection{天禧}

\begin{longtable}{|>{\centering\scriptsize}m{2em}|>{\centering\scriptsize}m{1.3em}|>{\centering}m{8.8em}|}
  % \caption{秦王政}\
  \toprule
  \SimHei \normalsize 年数 & \SimHei \scriptsize 公元 & \SimHei 大事件 \tabularnewline
  % \midrule
  \endfirsthead
  \toprule
  \SimHei \normalsize 年数 & \SimHei \scriptsize 公元 & \SimHei 大事件 \tabularnewline
  \midrule
  \endhead
  \midrule
  元年 & 1017 & \tabularnewline\hline
  二年 & 1018 & \tabularnewline\hline
  三年 & 1019 & \tabularnewline\hline
  四年 & 1020 & \tabularnewline\hline
  五年 & 1021 & \tabularnewline
  \bottomrule
\end{longtable}

\subsection{乾兴}

\begin{longtable}{|>{\centering\scriptsize}m{2em}|>{\centering\scriptsize}m{1.3em}|>{\centering}m{8.8em}|}
  % \caption{秦王政}\
  \toprule
  \SimHei \normalsize 年数 & \SimHei \scriptsize 公元 & \SimHei 大事件 \tabularnewline
  % \midrule
  \endfirsthead
  \toprule
  \SimHei \normalsize 年数 & \SimHei \scriptsize 公元 & \SimHei 大事件 \tabularnewline
  \midrule
  \endhead
  \midrule
  元年 & 1022 & \tabularnewline
  \bottomrule
\end{longtable}


%%% Local Variables:
%%% mode: latex
%%% TeX-engine: xetex
%%% TeX-master: "../Main"
%%% End:

%% -*- coding: utf-8 -*-
%% Time-stamp: <Chen Wang: 2018-07-12 13:02:40>

\section{仁宗\tiny(1022-1063)}

\subsection{天圣}


\begin{longtable}{|>{\centering\scriptsize}m{2em}|>{\centering\scriptsize}m{1.3em}|>{\centering}m{8.8em}|}
  % \caption{秦王政}\
  \toprule
  \SimHei \normalsize 年数 & \SimHei \scriptsize 公元 & \SimHei 大事件 \tabularnewline
  % \midrule
  \endfirsthead
  \toprule
  \SimHei \normalsize 年数 & \SimHei \scriptsize 公元 & \SimHei 大事件 \tabularnewline
  \midrule
  \endhead
  \midrule
  元年 & 1023 & \tabularnewline\hline
  二年 & 1024 & \tabularnewline\hline
  三年 & 1025 & \tabularnewline\hline
  四年 & 1026 & \tabularnewline\hline
  五年 & 1027 & \tabularnewline\hline
  六年 & 1028 & \tabularnewline\hline
  七年 & 1029 & \tabularnewline\hline
  八年 & 1030 & \tabularnewline\hline
  九年 & 1031 & \tabularnewline\hline
  十年 & 1032 & \tabularnewline
  \bottomrule
\end{longtable}

\subsection{明道}

\begin{longtable}{|>{\centering\scriptsize}m{2em}|>{\centering\scriptsize}m{1.3em}|>{\centering}m{8.8em}|}
  % \caption{秦王政}\
  \toprule
  \SimHei \normalsize 年数 & \SimHei \scriptsize 公元 & \SimHei 大事件 \tabularnewline
  % \midrule
  \endfirsthead
  \toprule
  \SimHei \normalsize 年数 & \SimHei \scriptsize 公元 & \SimHei 大事件 \tabularnewline
  \midrule
  \endhead
  \midrule
  元年 & 1032 & \tabularnewline\hline
  二年 & 1033 & \tabularnewline
  \bottomrule
\end{longtable}

\subsection{景祐}

\begin{longtable}{|>{\centering\scriptsize}m{2em}|>{\centering\scriptsize}m{1.3em}|>{\centering}m{8.8em}|}
  % \caption{秦王政}\
  \toprule
  \SimHei \normalsize 年数 & \SimHei \scriptsize 公元 & \SimHei 大事件 \tabularnewline
  % \midrule
  \endfirsthead
  \toprule
  \SimHei \normalsize 年数 & \SimHei \scriptsize 公元 & \SimHei 大事件 \tabularnewline
  \midrule
  \endhead
  \midrule
  元年 & 1034 & \tabularnewline\hline
  二年 & 1035 & \tabularnewline\hline
  三年 & 1036 & \tabularnewline\hline
  四年 & 1037 & \tabularnewline\hline
  五年 & 1038 & \tabularnewline
  \bottomrule
\end{longtable}

\subsection{宝元}

\begin{longtable}{|>{\centering\scriptsize}m{2em}|>{\centering\scriptsize}m{1.3em}|>{\centering}m{8.8em}|}
  % \caption{秦王政}\
  \toprule
  \SimHei \normalsize 年数 & \SimHei \scriptsize 公元 & \SimHei 大事件 \tabularnewline
  % \midrule
  \endfirsthead
  \toprule
  \SimHei \normalsize 年数 & \SimHei \scriptsize 公元 & \SimHei 大事件 \tabularnewline
  \midrule
  \endhead
  \midrule
  元年 & 1038 & \tabularnewline\hline
  二年 & 1039 & \tabularnewline\hline
  三年 & 1040 & \tabularnewline
  \bottomrule
\end{longtable}

\subsection{康定}

\begin{longtable}{|>{\centering\scriptsize}m{2em}|>{\centering\scriptsize}m{1.3em}|>{\centering}m{8.8em}|}
  % \caption{秦王政}\
  \toprule
  \SimHei \normalsize 年数 & \SimHei \scriptsize 公元 & \SimHei 大事件 \tabularnewline
  % \midrule
  \endfirsthead
  \toprule
  \SimHei \normalsize 年数 & \SimHei \scriptsize 公元 & \SimHei 大事件 \tabularnewline
  \midrule
  \endhead
  \midrule
  元年 & 1040 & \tabularnewline\hline
  二年 & 1041 & \tabularnewline
  \bottomrule
\end{longtable}

\subsection{庆历}

\begin{longtable}{|>{\centering\scriptsize}m{2em}|>{\centering\scriptsize}m{1.3em}|>{\centering}m{8.8em}|}
  % \caption{秦王政}\
  \toprule
  \SimHei \normalsize 年数 & \SimHei \scriptsize 公元 & \SimHei 大事件 \tabularnewline
  % \midrule
  \endfirsthead
  \toprule
  \SimHei \normalsize 年数 & \SimHei \scriptsize 公元 & \SimHei 大事件 \tabularnewline
  \midrule
  \endhead
  \midrule
  元年 & 1041 & \tabularnewline\hline
  二年 & 1042 & \tabularnewline\hline
  三年 & 1043 & \tabularnewline\hline
  四年 & 1044 & \tabularnewline\hline
  五年 & 1045 & \tabularnewline\hline
  六年 & 1046 & \tabularnewline\hline
  七年 & 1047 & \tabularnewline\hline
  八年 & 1048 & \tabularnewline
  \bottomrule
\end{longtable}

\subsection{皇祐}

\begin{longtable}{|>{\centering\scriptsize}m{2em}|>{\centering\scriptsize}m{1.3em}|>{\centering}m{8.8em}|}
  % \caption{秦王政}\
  \toprule
  \SimHei \normalsize 年数 & \SimHei \scriptsize 公元 & \SimHei 大事件 \tabularnewline
  % \midrule
  \endfirsthead
  \toprule
  \SimHei \normalsize 年数 & \SimHei \scriptsize 公元 & \SimHei 大事件 \tabularnewline
  \midrule
  \endhead
  \midrule
  元年 & 1049 & \tabularnewline\hline
  二年 & 1050 & \tabularnewline\hline
  三年 & 1051 & \tabularnewline\hline
  四年 & 1052 & \tabularnewline\hline
  五年 & 1053 & \tabularnewline\hline
  六年 & 1054 & \tabularnewline
  \bottomrule
\end{longtable}

\subsection{至和}

\begin{longtable}{|>{\centering\scriptsize}m{2em}|>{\centering\scriptsize}m{1.3em}|>{\centering}m{8.8em}|}
  % \caption{秦王政}\
  \toprule
  \SimHei \normalsize 年数 & \SimHei \scriptsize 公元 & \SimHei 大事件 \tabularnewline
  % \midrule
  \endfirsthead
  \toprule
  \SimHei \normalsize 年数 & \SimHei \scriptsize 公元 & \SimHei 大事件 \tabularnewline
  \midrule
  \endhead
  \midrule
  元年 & 1054 & \tabularnewline\hline
  二年 & 1055 & \tabularnewline\hline
  三年 & 1056 & \tabularnewline
  \bottomrule
\end{longtable}

\subsection{嘉佑}

\begin{longtable}{|>{\centering\scriptsize}m{2em}|>{\centering\scriptsize}m{1.3em}|>{\centering}m{8.8em}|}
  % \caption{秦王政}\
  \toprule
  \SimHei \normalsize 年数 & \SimHei \scriptsize 公元 & \SimHei 大事件 \tabularnewline
  % \midrule
  \endfirsthead
  \toprule
  \SimHei \normalsize 年数 & \SimHei \scriptsize 公元 & \SimHei 大事件 \tabularnewline
  \midrule
  \endhead
  \midrule
  元年 & 1056 & \tabularnewline\hline
  二年 & 1057 & \tabularnewline\hline
  三年 & 1058 & \tabularnewline\hline
  四年 & 1059 & \tabularnewline\hline
  五年 & 1060 & \tabularnewline\hline
  六年 & 1061 & \tabularnewline\hline
  七年 & 1062 & \tabularnewline\hline
  八年 & 1063 & \tabularnewline
  \bottomrule
\end{longtable}


%%% Local Variables:
%%% mode: latex
%%% TeX-engine: xetex
%%% TeX-master: "../Main"
%%% End:

%% -*- coding: utf-8 -*-
%% Time-stamp: <Chen Wang: 2018-07-12 13:03:32>

\section{英宗\tiny(1063-1067)}

\subsection{治平}


\begin{longtable}{|>{\centering\scriptsize}m{2em}|>{\centering\scriptsize}m{1.3em}|>{\centering}m{8.8em}|}
  % \caption{秦王政}\
  \toprule
  \SimHei \normalsize 年数 & \SimHei \scriptsize 公元 & \SimHei 大事件 \tabularnewline
  % \midrule
  \endfirsthead
  \toprule
  \SimHei \normalsize 年数 & \SimHei \scriptsize 公元 & \SimHei 大事件 \tabularnewline
  \midrule
  \endhead
  \midrule
  元年 & 1064 & \tabularnewline\hline
  二年 & 1065 & \tabularnewline\hline
  三年 & 1066 & \tabularnewline\hline
  四年 & 1067 & \tabularnewline
  \bottomrule
\end{longtable}



%%% Local Variables:
%%% mode: latex
%%% TeX-engine: xetex
%%% TeX-master: "../Main"
%%% End:

%% -*- coding: utf-8 -*-
%% Time-stamp: <Chen Wang: 2018-07-12 13:04:43>

\section{神宗\tiny(1067-1085)}

\subsection{熙宁}


\begin{longtable}{|>{\centering\scriptsize}m{2em}|>{\centering\scriptsize}m{1.3em}|>{\centering}m{8.8em}|}
  % \caption{秦王政}\
  \toprule
  \SimHei \normalsize 年数 & \SimHei \scriptsize 公元 & \SimHei 大事件 \tabularnewline
  % \midrule
  \endfirsthead
  \toprule
  \SimHei \normalsize 年数 & \SimHei \scriptsize 公元 & \SimHei 大事件 \tabularnewline
  \midrule
  \endhead
  \midrule
  元年 & 1068 & \tabularnewline\hline
  二年 & 1069 & \tabularnewline\hline
  三年 & 1070 & \tabularnewline\hline
  四年 & 1071 & \tabularnewline\hline
  五年 & 1072 & \tabularnewline\hline
  六年 & 1073 & \tabularnewline\hline
  七年 & 1074 & \tabularnewline\hline
  八年 & 1075 & \tabularnewline\hline
  九年 & 1076 & \tabularnewline\hline
  十年 & 1077 & \tabularnewline
  \bottomrule
\end{longtable}

\subsection{元丰}

\begin{longtable}{|>{\centering\scriptsize}m{2em}|>{\centering\scriptsize}m{1.3em}|>{\centering}m{8.8em}|}
  % \caption{秦王政}\
  \toprule
  \SimHei \normalsize 年数 & \SimHei \scriptsize 公元 & \SimHei 大事件 \tabularnewline
  % \midrule
  \endfirsthead
  \toprule
  \SimHei \normalsize 年数 & \SimHei \scriptsize 公元 & \SimHei 大事件 \tabularnewline
  \midrule
  \endhead
  \midrule
  元年 & 1078 & \tabularnewline\hline
  二年 & 1079 & \tabularnewline\hline
  三年 & 1080 & \tabularnewline\hline
  四年 & 1081 & \tabularnewline\hline
  五年 & 1082 & \tabularnewline\hline
  六年 & 1083 & \tabularnewline\hline
  七年 & 1084 & \tabularnewline\hline
  八年 & 1085 & \tabularnewline
  \bottomrule
\end{longtable}



%%% Local Variables:
%%% mode: latex
%%% TeX-engine: xetex
%%% TeX-master: "../Main"
%%% End:

%% -*- coding: utf-8 -*-
%% Time-stamp: <Chen Wang: 2018-07-12 13:06:40>

\section{哲宗\tiny(1085-1100)}

\subsection{元祐}


\begin{longtable}{|>{\centering\scriptsize}m{2em}|>{\centering\scriptsize}m{1.3em}|>{\centering}m{8.8em}|}
  % \caption{秦王政}\
  \toprule
  \SimHei \normalsize 年数 & \SimHei \scriptsize 公元 & \SimHei 大事件 \tabularnewline
  % \midrule
  \endfirsthead
  \toprule
  \SimHei \normalsize 年数 & \SimHei \scriptsize 公元 & \SimHei 大事件 \tabularnewline
  \midrule
  \endhead
  \midrule
  元年 & 1086 & \tabularnewline\hline
  二年 & 1087 & \tabularnewline\hline
  三年 & 1088 & \tabularnewline\hline
  四年 & 1089 & \tabularnewline\hline
  五年 & 1090 & \tabularnewline\hline
  六年 & 1091 & \tabularnewline\hline
  七年 & 1092 & \tabularnewline\hline
  八年 & 1093 & \tabularnewline\hline
  九年 & 1094 & \tabularnewline
  \bottomrule
\end{longtable}

\subsection{绍圣}

\begin{longtable}{|>{\centering\scriptsize}m{2em}|>{\centering\scriptsize}m{1.3em}|>{\centering}m{8.8em}|}
  % \caption{秦王政}\
  \toprule
  \SimHei \normalsize 年数 & \SimHei \scriptsize 公元 & \SimHei 大事件 \tabularnewline
  % \midrule
  \endfirsthead
  \toprule
  \SimHei \normalsize 年数 & \SimHei \scriptsize 公元 & \SimHei 大事件 \tabularnewline
  \midrule
  \endhead
  \midrule
  元年 & 1094 & \tabularnewline\hline
  二年 & 1095 & \tabularnewline\hline
  三年 & 1096 & \tabularnewline\hline
  四年 & 1097 & \tabularnewline\hline
  五年 & 1098 & \tabularnewline
  \bottomrule
\end{longtable}

\subsection{元符}

\begin{longtable}{|>{\centering\scriptsize}m{2em}|>{\centering\scriptsize}m{1.3em}|>{\centering}m{8.8em}|}
  % \caption{秦王政}\
  \toprule
  \SimHei \normalsize 年数 & \SimHei \scriptsize 公元 & \SimHei 大事件 \tabularnewline
  % \midrule
  \endfirsthead
  \toprule
  \SimHei \normalsize 年数 & \SimHei \scriptsize 公元 & \SimHei 大事件 \tabularnewline
  \midrule
  \endhead
  \midrule
  元年 & 1098 & \tabularnewline\hline
  二年 & 1099 & \tabularnewline\hline
  三年 & 1100 & \tabularnewline
  \bottomrule
\end{longtable}



%%% Local Variables:
%%% mode: latex
%%% TeX-engine: xetex
%%% TeX-master: "../Main"
%%% End:

%% -*- coding: utf-8 -*-
%% Time-stamp: <Chen Wang: 2018-07-12 13:09:42>

\section{徽宗\tiny(1100-1125)}

\subsection{建中靖国}


\begin{longtable}{|>{\centering\scriptsize}m{2em}|>{\centering\scriptsize}m{1.3em}|>{\centering}m{8.8em}|}
  % \caption{秦王政}\
  \toprule
  \SimHei \normalsize 年数 & \SimHei \scriptsize 公元 & \SimHei 大事件 \tabularnewline
  % \midrule
  \endfirsthead
  \toprule
  \SimHei \normalsize 年数 & \SimHei \scriptsize 公元 & \SimHei 大事件 \tabularnewline
  \midrule
  \endhead
  \midrule
  元年 & 1101 & \tabularnewline
  \bottomrule
\end{longtable}

\subsection{崇宁}

\begin{longtable}{|>{\centering\scriptsize}m{2em}|>{\centering\scriptsize}m{1.3em}|>{\centering}m{8.8em}|}
  % \caption{秦王政}\
  \toprule
  \SimHei \normalsize 年数 & \SimHei \scriptsize 公元 & \SimHei 大事件 \tabularnewline
  % \midrule
  \endfirsthead
  \toprule
  \SimHei \normalsize 年数 & \SimHei \scriptsize 公元 & \SimHei 大事件 \tabularnewline
  \midrule
  \endhead
  \midrule
  元年 & 1102 & \tabularnewline\hline
  二年 & 1103 & \tabularnewline\hline
  三年 & 1104 & \tabularnewline\hline
  四年 & 1105 & \tabularnewline\hline
  五年 & 1106 & \tabularnewline
  \bottomrule
\end{longtable}

\subsection{大观}

\begin{longtable}{|>{\centering\scriptsize}m{2em}|>{\centering\scriptsize}m{1.3em}|>{\centering}m{8.8em}|}
  % \caption{秦王政}\
  \toprule
  \SimHei \normalsize 年数 & \SimHei \scriptsize 公元 & \SimHei 大事件 \tabularnewline
  % \midrule
  \endfirsthead
  \toprule
  \SimHei \normalsize 年数 & \SimHei \scriptsize 公元 & \SimHei 大事件 \tabularnewline
  \midrule
  \endhead
  \midrule
  元年 & 1107 & \tabularnewline\hline
  二年 & 1108 & \tabularnewline\hline
  三年 & 1109 & \tabularnewline\hline
  四年 & 1110 & \tabularnewline
  \bottomrule
\end{longtable}

\subsection{政和}

\begin{longtable}{|>{\centering\scriptsize}m{2em}|>{\centering\scriptsize}m{1.3em}|>{\centering}m{8.8em}|}
  % \caption{秦王政}\
  \toprule
  \SimHei \normalsize 年数 & \SimHei \scriptsize 公元 & \SimHei 大事件 \tabularnewline
  % \midrule
  \endfirsthead
  \toprule
  \SimHei \normalsize 年数 & \SimHei \scriptsize 公元 & \SimHei 大事件 \tabularnewline
  \midrule
  \endhead
  \midrule
  元年 & 1111 & \tabularnewline\hline
  二年 & 1112 & \tabularnewline\hline
  三年 & 1113 & \tabularnewline\hline
  四年 & 1114 & \tabularnewline\hline
  五年 & 1115 & \tabularnewline\hline
  六年 & 1116 & \tabularnewline\hline
  七年 & 1117 & \tabularnewline\hline
  八年 & 1118 & \tabularnewline
  \bottomrule
\end{longtable}

\subsection{重和}

\begin{longtable}{|>{\centering\scriptsize}m{2em}|>{\centering\scriptsize}m{1.3em}|>{\centering}m{8.8em}|}
  % \caption{秦王政}\
  \toprule
  \SimHei \normalsize 年数 & \SimHei \scriptsize 公元 & \SimHei 大事件 \tabularnewline
  % \midrule
  \endfirsthead
  \toprule
  \SimHei \normalsize 年数 & \SimHei \scriptsize 公元 & \SimHei 大事件 \tabularnewline
  \midrule
  \endhead
  \midrule
  元年 & 1118 & \tabularnewline\hline
  二年 & 1119 & \tabularnewline
  \bottomrule
\end{longtable}

\subsection{宣和}

\begin{longtable}{|>{\centering\scriptsize}m{2em}|>{\centering\scriptsize}m{1.3em}|>{\centering}m{8.8em}|}
  % \caption{秦王政}\
  \toprule
  \SimHei \normalsize 年数 & \SimHei \scriptsize 公元 & \SimHei 大事件 \tabularnewline
  % \midrule
  \endfirsthead
  \toprule
  \SimHei \normalsize 年数 & \SimHei \scriptsize 公元 & \SimHei 大事件 \tabularnewline
  \midrule
  \endhead
  \midrule
  元年 & 1119 & \tabularnewline\hline
  二年 & 1120 & \tabularnewline\hline
  三年 & 1121 & \tabularnewline\hline
  四年 & 1122 & \tabularnewline\hline
  五年 & 1123 & \tabularnewline\hline
  六年 & 1124 & \tabularnewline\hline
  七年 & 1125 & \tabularnewline
  \bottomrule
\end{longtable}



%%% Local Variables:
%%% mode: latex
%%% TeX-engine: xetex
%%% TeX-master: "../Main"
%%% End:

%% -*- coding: utf-8 -*-
%% Time-stamp: <Chen Wang: 2018-07-12 13:10:29>

\section{钦宗\tiny(1126-1127)}

\subsection{靖康}


\begin{longtable}{|>{\centering\scriptsize}m{2em}|>{\centering\scriptsize}m{1.3em}|>{\centering}m{8.8em}|}
  % \caption{秦王政}\
  \toprule
  \SimHei \normalsize 年数 & \SimHei \scriptsize 公元 & \SimHei 大事件 \tabularnewline
  % \midrule
  \endfirsthead
  \toprule
  \SimHei \normalsize 年数 & \SimHei \scriptsize 公元 & \SimHei 大事件 \tabularnewline
  \midrule
  \endhead
  \midrule
  元年 & 1126 & \tabularnewline\hline
  二年 & 1127 & \tabularnewline
  \bottomrule
\end{longtable}



%%% Local Variables:
%%% mode: latex
%%% TeX-engine: xetex
%%% TeX-master: "../Main"
%%% End:


%%% Local Variables:
%%% mode: latex
%%% TeX-engine: xetex
%%% TeX-master: "../Main"
%%% End:
 % 北宋
% %% -*- coding: utf-8 -*-
%% Time-stamp: <Chen Wang: 2019-10-15 11:21:03>

\chapter{南宋\tiny(1127-1279)}

%% -*- coding: utf-8 -*-
%% Time-stamp: <Chen Wang: 2018-07-12 13:18:27>

\section{高宗\tiny(1127-1162)}

\subsection{建炎}


\begin{longtable}{|>{\centering\scriptsize}m{2em}|>{\centering\scriptsize}m{1.3em}|>{\centering}m{8.8em}|}
  % \caption{秦王政}\
  \toprule
  \SimHei \normalsize 年数 & \SimHei \scriptsize 公元 & \SimHei 大事件 \tabularnewline
  % \midrule
  \endfirsthead
  \toprule
  \SimHei \normalsize 年数 & \SimHei \scriptsize 公元 & \SimHei 大事件 \tabularnewline
  \midrule
  \endhead
  \midrule
  元年 & 1127 & \tabularnewline\hline
  二年 & 1128 & \tabularnewline\hline
  三年 & 1129 & \tabularnewline\hline
  四年 & 1130 & \tabularnewline
  \bottomrule
\end{longtable}

\subsection{绍兴}

\begin{longtable}{|>{\centering\scriptsize}m{2em}|>{\centering\scriptsize}m{1.3em}|>{\centering}m{8.8em}|}
  % \caption{秦王政}\
  \toprule
  \SimHei \normalsize 年数 & \SimHei \scriptsize 公元 & \SimHei 大事件 \tabularnewline
  % \midrule
  \endfirsthead
  \toprule
  \SimHei \normalsize 年数 & \SimHei \scriptsize 公元 & \SimHei 大事件 \tabularnewline
  \midrule
  \endhead
  \midrule
  元年 & 1131 & \tabularnewline\hline
  二年 & 1132 & \tabularnewline\hline
  三年 & 1133 & \tabularnewline\hline
  四年 & 1134 & \tabularnewline\hline
  五年 & 1135 & \tabularnewline\hline
  六年 & 1136 & \tabularnewline\hline
  七年 & 1137 & \tabularnewline\hline
  八年 & 1138 & \tabularnewline\hline
  九年 & 1139 & \tabularnewline\hline
  十年 & 1140 & \tabularnewline\hline
  十一年 & 1141 & \tabularnewline\hline
  十二年 & 1142 & \tabularnewline\hline
  十三年 & 1143 & \tabularnewline\hline
  十四年 & 1144 & \tabularnewline\hline
  十五年 & 1145 & \tabularnewline\hline
  十六年 & 1146 & \tabularnewline\hline
  十七年 & 1147 & \tabularnewline\hline
  十八年 & 1148 & \tabularnewline\hline
  十九年 & 1149 & \tabularnewline\hline
  二十年 & 1150 & \tabularnewline\hline
  二一年 & 1151 & \tabularnewline\hline
  二二年 & 1152 & \tabularnewline\hline
  二三年 & 1153 & \tabularnewline\hline
  二四年 & 1154 & \tabularnewline\hline
  二五年 & 1155 & \tabularnewline\hline
  二六年 & 1156 & \tabularnewline\hline
  二七年 & 1157 & \tabularnewline\hline
  二八年 & 1158 & \tabularnewline\hline
  二九年 & 1159 & \tabularnewline\hline
  三十年 & 1160 & \tabularnewline\hline
  三一年 & 1161 & \tabularnewline\hline
  三二年 & 1162 & \tabularnewline
  \bottomrule
\end{longtable}



%%% Local Variables:
%%% mode: latex
%%% TeX-engine: xetex
%%% TeX-master: "../Main"
%%% End:

%% -*- coding: utf-8 -*-
%% Time-stamp: <Chen Wang: 2018-07-12 13:30:05>

\section{孝宗\tiny(1162-1189)}

\subsection{隆兴}


\begin{longtable}{|>{\centering\scriptsize}m{2em}|>{\centering\scriptsize}m{1.3em}|>{\centering}m{8.8em}|}
  % \caption{秦王政}\
  \toprule
  \SimHei \normalsize 年数 & \SimHei \scriptsize 公元 & \SimHei 大事件 \tabularnewline
  % \midrule
  \endfirsthead
  \toprule
  \SimHei \normalsize 年数 & \SimHei \scriptsize 公元 & \SimHei 大事件 \tabularnewline
  \midrule
  \endhead
  \midrule
  元年 & 1163 & \tabularnewline\hline
  二年 & 1164 & \tabularnewline
  \bottomrule
\end{longtable}

\subsection{乾道}

\begin{longtable}{|>{\centering\scriptsize}m{2em}|>{\centering\scriptsize}m{1.3em}|>{\centering}m{8.8em}|}
  % \caption{秦王政}\
  \toprule
  \SimHei \normalsize 年数 & \SimHei \scriptsize 公元 & \SimHei 大事件 \tabularnewline
  % \midrule
  \endfirsthead
  \toprule
  \SimHei \normalsize 年数 & \SimHei \scriptsize 公元 & \SimHei 大事件 \tabularnewline
  \midrule
  \endhead
  \midrule
  元年 & 1165 & \tabularnewline\hline
  二年 & 1166 & \tabularnewline\hline
  三年 & 1167 & \tabularnewline\hline
  四年 & 1168 & \tabularnewline\hline
  五年 & 1169 & \tabularnewline\hline
  六年 & 1170 & \tabularnewline\hline
  七年 & 1171 & \tabularnewline\hline
  八年 & 1172 & \tabularnewline\hline
  九年 & 1173 & \tabularnewline
  \bottomrule
\end{longtable}

\subsection{淳熙}

\begin{longtable}{|>{\centering\scriptsize}m{2em}|>{\centering\scriptsize}m{1.3em}|>{\centering}m{8.8em}|}
  % \caption{秦王政}\
  \toprule
  \SimHei \normalsize 年数 & \SimHei \scriptsize 公元 & \SimHei 大事件 \tabularnewline
  % \midrule
  \endfirsthead
  \toprule
  \SimHei \normalsize 年数 & \SimHei \scriptsize 公元 & \SimHei 大事件 \tabularnewline
  \midrule
  \endhead
  \midrule
  元年 & 1174 & \tabularnewline\hline
  二年 & 1175 & \tabularnewline\hline
  三年 & 1176 & \tabularnewline\hline
  四年 & 1177 & \tabularnewline\hline
  五年 & 1178 & \tabularnewline\hline
  六年 & 1179 & \tabularnewline\hline
  七年 & 1180 & \tabularnewline\hline
  八年 & 1181 & \tabularnewline\hline
  九年 & 1182 & \tabularnewline\hline
  十年 & 1183 & \tabularnewline\hline
  十一年 & 1184 & \tabularnewline\hline
  十二年 & 1185 & \tabularnewline\hline
  十三年 & 1186 & \tabularnewline\hline
  十四年 & 1187 & \tabularnewline\hline
  十五年 & 1188 & \tabularnewline\hline
  十六年 & 1189 & \tabularnewline
  \bottomrule
\end{longtable}



%%% Local Variables:
%%% mode: latex
%%% TeX-engine: xetex
%%% TeX-master: "../Main"
%%% End:

%% -*- coding: utf-8 -*-
%% Time-stamp: <Chen Wang: 2018-07-12 13:31:09>

\section{光宗\tiny(1189-1194)}

\subsection{绍熙}


\begin{longtable}{|>{\centering\scriptsize}m{2em}|>{\centering\scriptsize}m{1.3em}|>{\centering}m{8.8em}|}
  % \caption{秦王政}\
  \toprule
  \SimHei \normalsize 年数 & \SimHei \scriptsize 公元 & \SimHei 大事件 \tabularnewline
  % \midrule
  \endfirsthead
  \toprule
  \SimHei \normalsize 年数 & \SimHei \scriptsize 公元 & \SimHei 大事件 \tabularnewline
  \midrule
  \endhead
  \midrule
  元年 & 1190 & \tabularnewline\hline
  二年 & 1191 & \tabularnewline\hline
  三年 & 1192 & \tabularnewline\hline
  四年 & 1193 & \tabularnewline\hline
  五年 & 1194 & \tabularnewline
  \bottomrule
\end{longtable}



%%% Local Variables:
%%% mode: latex
%%% TeX-engine: xetex
%%% TeX-master: "../Main"
%%% End:

%% -*- coding: utf-8 -*-
%% Time-stamp: <Chen Wang: 2018-07-12 13:33:18>

\section{宁宗\tiny(1194-1224)}

\subsection{庆元}


\begin{longtable}{|>{\centering\scriptsize}m{2em}|>{\centering\scriptsize}m{1.3em}|>{\centering}m{8.8em}|}
  % \caption{秦王政}\
  \toprule
  \SimHei \normalsize 年数 & \SimHei \scriptsize 公元 & \SimHei 大事件 \tabularnewline
  % \midrule
  \endfirsthead
  \toprule
  \SimHei \normalsize 年数 & \SimHei \scriptsize 公元 & \SimHei 大事件 \tabularnewline
  \midrule
  \endhead
  \midrule
  元年 & 1195 & \tabularnewline\hline
  二年 & 1196 & \tabularnewline\hline
  三年 & 1197 & \tabularnewline\hline
  四年 & 1198 & \tabularnewline\hline
  五年 & 1199 & \tabularnewline\hline
  六年 & 1200 & \tabularnewline
  \bottomrule
\end{longtable}

\subsection{嘉泰}

\begin{longtable}{|>{\centering\scriptsize}m{2em}|>{\centering\scriptsize}m{1.3em}|>{\centering}m{8.8em}|}
  % \caption{秦王政}\
  \toprule
  \SimHei \normalsize 年数 & \SimHei \scriptsize 公元 & \SimHei 大事件 \tabularnewline
  % \midrule
  \endfirsthead
  \toprule
  \SimHei \normalsize 年数 & \SimHei \scriptsize 公元 & \SimHei 大事件 \tabularnewline
  \midrule
  \endhead
  \midrule
  元年 & 1201 & \tabularnewline\hline
  二年 & 1202 & \tabularnewline\hline
  三年 & 1203 & \tabularnewline\hline
  四年 & 1204 & \tabularnewline
  \bottomrule
\end{longtable}

\subsection{开禧}

\begin{longtable}{|>{\centering\scriptsize}m{2em}|>{\centering\scriptsize}m{1.3em}|>{\centering}m{8.8em}|}
  % \caption{秦王政}\
  \toprule
  \SimHei \normalsize 年数 & \SimHei \scriptsize 公元 & \SimHei 大事件 \tabularnewline
  % \midrule
  \endfirsthead
  \toprule
  \SimHei \normalsize 年数 & \SimHei \scriptsize 公元 & \SimHei 大事件 \tabularnewline
  \midrule
  \endhead
  \midrule
  元年 & 1205 & \tabularnewline\hline
  二年 & 1206 & \tabularnewline\hline
  三年 & 1207 & \tabularnewline
  \bottomrule
\end{longtable}

\subsection{嘉定}

\begin{longtable}{|>{\centering\scriptsize}m{2em}|>{\centering\scriptsize}m{1.3em}|>{\centering}m{8.8em}|}
  % \caption{秦王政}\
  \toprule
  \SimHei \normalsize 年数 & \SimHei \scriptsize 公元 & \SimHei 大事件 \tabularnewline
  % \midrule
  \endfirsthead
  \toprule
  \SimHei \normalsize 年数 & \SimHei \scriptsize 公元 & \SimHei 大事件 \tabularnewline
  \midrule
  \endhead
  \midrule
  元年 & 1208 & \tabularnewline\hline
  二年 & 1209 & \tabularnewline\hline
  三年 & 1210 & \tabularnewline\hline
  四年 & 1211 & \tabularnewline\hline
  五年 & 1212 & \tabularnewline\hline
  六年 & 1213 & \tabularnewline\hline
  七年 & 1214 & \tabularnewline\hline
  八年 & 1215 & \tabularnewline\hline
  九年 & 1216 & \tabularnewline\hline
  十年 & 1217 & \tabularnewline\hline
  十一年 & 1218 & \tabularnewline\hline
  十二年 & 1219 & \tabularnewline\hline
  十三年 & 1220 & \tabularnewline\hline
  十四年 & 1221 & \tabularnewline\hline
  十五年 & 1222 & \tabularnewline\hline
  十六年 & 1223 & \tabularnewline\hline
  十七年 & 1224 & \tabularnewline
  \bottomrule
\end{longtable}



%%% Local Variables:
%%% mode: latex
%%% TeX-engine: xetex
%%% TeX-master: "../Main"
%%% End:

%% -*- coding: utf-8 -*-
%% Time-stamp: <Chen Wang: 2018-07-12 13:37:54>

\section{理宗\tiny(1224-1264)}

\subsection{宝庆}


\begin{longtable}{|>{\centering\scriptsize}m{2em}|>{\centering\scriptsize}m{1.3em}|>{\centering}m{8.8em}|}
  % \caption{秦王政}\
  \toprule
  \SimHei \normalsize 年数 & \SimHei \scriptsize 公元 & \SimHei 大事件 \tabularnewline
  % \midrule
  \endfirsthead
  \toprule
  \SimHei \normalsize 年数 & \SimHei \scriptsize 公元 & \SimHei 大事件 \tabularnewline
  \midrule
  \endhead
  \midrule
  元年 & 1225 & \tabularnewline\hline
  二年 & 1226 & \tabularnewline\hline
  三年 & 1227 & \tabularnewline
  \bottomrule
\end{longtable}

\subsection{绍定}

\begin{longtable}{|>{\centering\scriptsize}m{2em}|>{\centering\scriptsize}m{1.3em}|>{\centering}m{8.8em}|}
  % \caption{秦王政}\
  \toprule
  \SimHei \normalsize 年数 & \SimHei \scriptsize 公元 & \SimHei 大事件 \tabularnewline
  % \midrule
  \endfirsthead
  \toprule
  \SimHei \normalsize 年数 & \SimHei \scriptsize 公元 & \SimHei 大事件 \tabularnewline
  \midrule
  \endhead
  \midrule
  元年 & 1228 & \tabularnewline\hline
  二年 & 1229 & \tabularnewline\hline
  三年 & 1230 & \tabularnewline\hline
  四年 & 1231 & \tabularnewline\hline
  五年 & 1232 & \tabularnewline\hline
  六年 & 1233 & \tabularnewline
  \bottomrule
\end{longtable}

\subsection{端平}

\begin{longtable}{|>{\centering\scriptsize}m{2em}|>{\centering\scriptsize}m{1.3em}|>{\centering}m{8.8em}|}
  % \caption{秦王政}\
  \toprule
  \SimHei \normalsize 年数 & \SimHei \scriptsize 公元 & \SimHei 大事件 \tabularnewline
  % \midrule
  \endfirsthead
  \toprule
  \SimHei \normalsize 年数 & \SimHei \scriptsize 公元 & \SimHei 大事件 \tabularnewline
  \midrule
  \endhead
  \midrule
  元年 & 1234 & \tabularnewline\hline
  二年 & 1235 & \tabularnewline\hline
  三年 & 1236 & \tabularnewline
  \bottomrule
\end{longtable}

\subsection{嘉熙}

\begin{longtable}{|>{\centering\scriptsize}m{2em}|>{\centering\scriptsize}m{1.3em}|>{\centering}m{8.8em}|}
  % \caption{秦王政}\
  \toprule
  \SimHei \normalsize 年数 & \SimHei \scriptsize 公元 & \SimHei 大事件 \tabularnewline
  % \midrule
  \endfirsthead
  \toprule
  \SimHei \normalsize 年数 & \SimHei \scriptsize 公元 & \SimHei 大事件 \tabularnewline
  \midrule
  \endhead
  \midrule
  元年 & 1237 & \tabularnewline\hline
  二年 & 1238 & \tabularnewline\hline
  三年 & 1239 & \tabularnewline\hline
  四年 & 1240 & \tabularnewline
  \bottomrule
\end{longtable}

\subsection{淳祐}

\begin{longtable}{|>{\centering\scriptsize}m{2em}|>{\centering\scriptsize}m{1.3em}|>{\centering}m{8.8em}|}
  % \caption{秦王政}\
  \toprule
  \SimHei \normalsize 年数 & \SimHei \scriptsize 公元 & \SimHei 大事件 \tabularnewline
  % \midrule
  \endfirsthead
  \toprule
  \SimHei \normalsize 年数 & \SimHei \scriptsize 公元 & \SimHei 大事件 \tabularnewline
  \midrule
  \endhead
  \midrule
  元年 & 241 & \tabularnewline\hline
  二年 & 242 & \tabularnewline\hline
  三年 & 243 & \tabularnewline\hline
  四年 & 244 & \tabularnewline\hline
  五年 & 245 & \tabularnewline\hline
  六年 & 246 & \tabularnewline\hline
  七年 & 247 & \tabularnewline\hline
  八年 & 248 & \tabularnewline\hline
  九年 & 249 & \tabularnewline\hline
  十年 & 250 & \tabularnewline\hline
  十一年 & 251 & \tabularnewline\hline
  十二年 & 252 & \tabularnewline
  \bottomrule
\end{longtable}

\subsection{宝祐}

\begin{longtable}{|>{\centering\scriptsize}m{2em}|>{\centering\scriptsize}m{1.3em}|>{\centering}m{8.8em}|}
  % \caption{秦王政}\
  \toprule
  \SimHei \normalsize 年数 & \SimHei \scriptsize 公元 & \SimHei 大事件 \tabularnewline
  % \midrule
  \endfirsthead
  \toprule
  \SimHei \normalsize 年数 & \SimHei \scriptsize 公元 & \SimHei 大事件 \tabularnewline
  \midrule
  \endhead
  \midrule
  元年 & 1253 & \tabularnewline\hline
  二年 & 1254 & \tabularnewline\hline
  三年 & 1255 & \tabularnewline\hline
  四年 & 1256 & \tabularnewline\hline
  五年 & 1257 & \tabularnewline\hline
  六年 & 1258 & \tabularnewline
  \bottomrule
\end{longtable}

\subsection{开庆}

\begin{longtable}{|>{\centering\scriptsize}m{2em}|>{\centering\scriptsize}m{1.3em}|>{\centering}m{8.8em}|}
  % \caption{秦王政}\
  \toprule
  \SimHei \normalsize 年数 & \SimHei \scriptsize 公元 & \SimHei 大事件 \tabularnewline
  % \midrule
  \endfirsthead
  \toprule
  \SimHei \normalsize 年数 & \SimHei \scriptsize 公元 & \SimHei 大事件 \tabularnewline
  \midrule
  \endhead
  \midrule
  元年 & 1259 & \tabularnewline
  \bottomrule
\end{longtable}

\subsection{景定}

\begin{longtable}{|>{\centering\scriptsize}m{2em}|>{\centering\scriptsize}m{1.3em}|>{\centering}m{8.8em}|}
  % \caption{秦王政}\
  \toprule
  \SimHei \normalsize 年数 & \SimHei \scriptsize 公元 & \SimHei 大事件 \tabularnewline
  % \midrule
  \endfirsthead
  \toprule
  \SimHei \normalsize 年数 & \SimHei \scriptsize 公元 & \SimHei 大事件 \tabularnewline
  \midrule
  \endhead
  \midrule
  元年 & 1260 & \tabularnewline\hline
  二年 & 1261 & \tabularnewline\hline
  三年 & 1262 & \tabularnewline\hline
  四年 & 1263 & \tabularnewline\hline
  五年 & 1264 & \tabularnewline
  \bottomrule
\end{longtable}



%%% Local Variables:
%%% mode: latex
%%% TeX-engine: xetex
%%% TeX-master: "../Main"
%%% End:

%% -*- coding: utf-8 -*-
%% Time-stamp: <Chen Wang: 2018-07-12 13:38:36>

\section{度宗\tiny(1264-1274)}

\subsection{咸淳}


\begin{longtable}{|>{\centering\scriptsize}m{2em}|>{\centering\scriptsize}m{1.3em}|>{\centering}m{8.8em}|}
  % \caption{秦王政}\
  \toprule
  \SimHei \normalsize 年数 & \SimHei \scriptsize 公元 & \SimHei 大事件 \tabularnewline
  % \midrule
  \endfirsthead
  \toprule
  \SimHei \normalsize 年数 & \SimHei \scriptsize 公元 & \SimHei 大事件 \tabularnewline
  \midrule
  \endhead
  \midrule
  元年 & 1265 & \tabularnewline\hline
  二年 & 1266 & \tabularnewline\hline
  三年 & 1267 & \tabularnewline\hline
  四年 & 1268 & \tabularnewline\hline
  五年 & 1269 & \tabularnewline\hline
  六年 & 1270 & \tabularnewline\hline
  七年 & 1271 & \tabularnewline\hline
  八年 & 1272 & \tabularnewline\hline
  九年 & 1273 & \tabularnewline\hline
  十年 & 1274 & \tabularnewline
  \bottomrule
\end{longtable}



%%% Local Variables:
%%% mode: latex
%%% TeX-engine: xetex
%%% TeX-master: "../Main"
%%% End:

%% -*- coding: utf-8 -*-
%% Time-stamp: <Chen Wang: 2018-07-12 13:39:26>

\section{恭帝\tiny(1274-1276)}

\subsection{德祐}


\begin{longtable}{|>{\centering\scriptsize}m{2em}|>{\centering\scriptsize}m{1.3em}|>{\centering}m{8.8em}|}
  % \caption{秦王政}\
  \toprule
  \SimHei \normalsize 年数 & \SimHei \scriptsize 公元 & \SimHei 大事件 \tabularnewline
  % \midrule
  \endfirsthead
  \toprule
  \SimHei \normalsize 年数 & \SimHei \scriptsize 公元 & \SimHei 大事件 \tabularnewline
  \midrule
  \endhead
  \midrule
  元年 & 1275 & \tabularnewline\hline
  二年 & 1276 & \tabularnewline
  \bottomrule
\end{longtable}



%%% Local Variables:
%%% mode: latex
%%% TeX-engine: xetex
%%% TeX-master: "../Main"
%%% End:

%% -*- coding: utf-8 -*-
%% Time-stamp: <Chen Wang: 2018-07-12 13:40:08>

\section{端宗\tiny(1276-1278)}

\subsection{景炎}


\begin{longtable}{|>{\centering\scriptsize}m{2em}|>{\centering\scriptsize}m{1.3em}|>{\centering}m{8.8em}|}
  % \caption{秦王政}\
  \toprule
  \SimHei \normalsize 年数 & \SimHei \scriptsize 公元 & \SimHei 大事件 \tabularnewline
  % \midrule
  \endfirsthead
  \toprule
  \SimHei \normalsize 年数 & \SimHei \scriptsize 公元 & \SimHei 大事件 \tabularnewline
  \midrule
  \endhead
  \midrule
  元年 & 1276 & \tabularnewline\hline
  二年 & 1277 & \tabularnewline\hline
  三年 & 1278 & \tabularnewline
  \bottomrule
\end{longtable}



%%% Local Variables:
%%% mode: latex
%%% TeX-engine: xetex
%%% TeX-master: "../Main"
%%% End:

%% -*- coding: utf-8 -*-
%% Time-stamp: <Chen Wang: 2018-07-12 13:40:48>

\section{赵昺\tiny(1278-1279)}

\subsection{祥兴}


\begin{longtable}{|>{\centering\scriptsize}m{2em}|>{\centering\scriptsize}m{1.3em}|>{\centering}m{8.8em}|}
  % \caption{秦王政}\
  \toprule
  \SimHei \normalsize 年数 & \SimHei \scriptsize 公元 & \SimHei 大事件 \tabularnewline
  % \midrule
  \endfirsthead
  \toprule
  \SimHei \normalsize 年数 & \SimHei \scriptsize 公元 & \SimHei 大事件 \tabularnewline
  \midrule
  \endhead
  \midrule
  元年 & 1278 & \tabularnewline\hline
  二年 & 1279 & \tabularnewline
  \bottomrule
\end{longtable}



%%% Local Variables:
%%% mode: latex
%%% TeX-engine: xetex
%%% TeX-master: "../Main"
%%% End:


%%% Local Variables:
%%% mode: latex
%%% TeX-engine: xetex
%%% TeX-master: "../Main"
%%% End:
 % 南宋
% %% -*- coding: utf-8 -*-
%% Time-stamp: <Chen Wang: 2019-10-15 15:56:04>

\chapter{辽\tiny(916-1218)}

%% -*- coding: utf-8 -*-
%% Time-stamp: <Chen Wang: 2019-10-15 16:08:29>

\section{太祖\tiny(916-926)}

辽太祖耶律阿保机(872年-926年9月6日),清輯本《旧五代史》改譯安巴堅,汉名耶律亿,是大契丹國的第一位皇帝(916年3月17日-926年9月6日在位),在位10年。

《辽史·后妃传》记载:“太祖慕汉高皇帝,故耶律氏兼称刘氏;以乙室、拔里比萧相国,遂为萧氏”。《辽史·国语解》记载:“耶律和萧两个姓,以汉字书者曰刘、萧,以契丹字书者曰移喇、石抹”。《金史·国语解》记载:“移喇曰刘,石抹曰萧”。

耶律阿保机的前辈是契丹迭剌部的酋长和军事首领(夷里堇),为耶律撒剌的的长子,母萧岩母斤。耶律是其氏族名。他本人于901年被立为军事首领(夷里堇兼任于越),后不久被选为酋长。他以武力征服契丹附近的地区,掠虏了许多汉人和其他人。907年2月27日他被选为部落联盟的首领,连任九年。任用汉人,采纳他们的建议,决定要将这种三年一次的选举制度改为世袭的制度。為了鞏固統治,史載遼太祖初元,韓廷徽助其正君臣,定名分。廢除三年一次的選汗制度造成諸弟之亂,後來叛亂被平定。

公元915年,耶律阿保机出征室韦得胜回国,但被迫交出汗位,但他在在滦河边建设了一座仿幽州式的汉城。耶律阿保机后伏杀了他的敌人,吞并了契丹的各个部落。916年3月17日,耶律阿保机登基称皇帝,立国号“契丹”,建立“大契丹国”(947年2月24日,辽太宗耶律德光改国号为“大辽”),建年号为神册。此外他还令人建立自己的契丹文。

耶律阿保机建国后继续进攻其周围的民族或政权,渤海国、室韦和奚分别被他消灭。926年9月6日去世于扶余城,终年55岁。

耶律阿保機將其母親、祖母、曾祖母、高祖母家族的姓氏拔里氏、乙室氏賜姓蕭氏。相傳是因為他本人羨慕蕭何輔助劉邦的典故。耶律阿保機的皇后名述律平,其子耶律德光即位後,亦將述律氏賜姓蕭氏。故蕭氏有遼朝后族之稱。阿保機汉名姓刘名亿,長子耶律突欲汉名劉倍。

\subsection{神册}


\begin{longtable}{|>{\centering\scriptsize}m{2em}|>{\centering\scriptsize}m{1.3em}|>{\centering}m{8.8em}|}
  % \caption{秦王政}\
  \toprule
  \SimHei \normalsize 年数 & \SimHei \scriptsize 公元 & \SimHei 大事件 \tabularnewline
  % \midrule
  \endfirsthead
  \toprule
  \SimHei \normalsize 年数 & \SimHei \scriptsize 公元 & \SimHei 大事件 \tabularnewline
  \midrule
  \endhead
  \midrule
  元年 & 916 & \tabularnewline\hline
  二年 & 917 & \tabularnewline\hline
  三年 & 918 & \tabularnewline\hline
  四年 & 919 & \tabularnewline\hline
  五年 & 920 & \tabularnewline\hline
  六年 & 921 & \tabularnewline\hline
  七年 & 922 & \tabularnewline
  \bottomrule
\end{longtable}

\subsection{天赞}

\begin{longtable}{|>{\centering\scriptsize}m{2em}|>{\centering\scriptsize}m{1.3em}|>{\centering}m{8.8em}|}
  % \caption{秦王政}\
  \toprule
  \SimHei \normalsize 年数 & \SimHei \scriptsize 公元 & \SimHei 大事件 \tabularnewline
  % \midrule
  \endfirsthead
  \toprule
  \SimHei \normalsize 年数 & \SimHei \scriptsize 公元 & \SimHei 大事件 \tabularnewline
  \midrule
  \endhead
  \midrule
  元年 & 922 & \tabularnewline\hline
  二年 & 923 & \tabularnewline\hline
  三年 & 924 & \tabularnewline\hline
  四年 & 925 & \tabularnewline\hline
  五年 & 926 & \tabularnewline
  \bottomrule
\end{longtable}

\subsection{天显}

\begin{longtable}{|>{\centering\scriptsize}m{2em}|>{\centering\scriptsize}m{1.3em}|>{\centering}m{8.8em}|}
  % \caption{秦王政}\
  \toprule
  \SimHei \normalsize 年数 & \SimHei \scriptsize 公元 & \SimHei 大事件 \tabularnewline
  % \midrule
  \endfirsthead
  \toprule
  \SimHei \normalsize 年数 & \SimHei \scriptsize 公元 & \SimHei 大事件 \tabularnewline
  \midrule
  \endhead
  \midrule
  元年 & 926 & \tabularnewline
  \bottomrule
\end{longtable}


%%% Local Variables:
%%% mode: latex
%%% TeX-engine: xetex
%%% TeX-master: "../Main"
%%% End:

%% -*- coding: utf-8 -*-
%% Time-stamp: <Chen Wang: 2019-12-26 10:57:29>

\section{太宗\tiny(927-947)}


\subsection{生平}

遼太宗耶律德光(902年11月25日-947年5月15日),大契丹國第二位皇帝(927年12月11日至947年5月15日在位),在位20年。字德谨,契丹名耶律尧骨,辽太祖耶律阿保机次子。947年2月24日,辽太宗耶律德光将国号由“大契丹国”改为“大辽”,成为遼朝首位皇帝。

辽太祖天赞元年(922年),被任命为天下兵马大元帅,随同太祖参加了一系列征服战争,尤其是在南征幽州、西征吐谷浑、回鹘期间,战功卓著。天显元年(926年),又随同太祖灭渤海国,作为前锋攻克渤海首都忽汗城。

天显元年七月二十七日(926年9月6日)辽太祖死后,述律后称制,耶律德光总揽朝政,927年12月11日,在述律后的支持下即位。天显六年(930年),割据原渤海国疆域的东丹王耶律倍南逃后唐,耶律德光统一了契丹。

天显十一年(936年),后唐河东节度使石敬瑭以称子、割让燕云十六州为条件,乞求耶律德光出兵助其反对后唐。耶律德光遂亲率5万骑兵,在晋阳城下击败后唐军,册立石敬塘为后晋皇帝。其后,更率军南下上党,助石敬塘灭后唐。

割取燕云十六州后,耶律德光采取“因俗而治”的统治方式,实行南北两面官制度,分治汉人和契丹。又改幽州为南京、云州为西京,将燕云十六州建设成为进一步南下的基地。

会同四年(944年),后晋出帝石重贵即位,拒不称臣。耶律德光于是率军南下。会同九年十二月十六日(947年1月10日),耶律德光率军攻入后晋首都东京汴梁(今河南开封),俘虏后晋出帝石重贵,灭后晋。

会同十年正月初一(947年1月25日),耶律德光以中原皇帝的仪仗进入东京汴梁,在崇元殿接受百官朝贺。大同元年二月初一(947年2月24日),耶律德光在东京皇宫下诏将国号“大契丹国”改为“大辽”,改会同十年为大同元年,升镇州为中京。

大同元年四月初一(947年4月24日),因遼人實施的「打草穀」物資掠奪政策導致中原反抗不断,无法巩固统治,耶律德光被迫离开东京汴梁,引军北返,在临城县(今河北省临城县)得熱疾。四月二十二日(947年5月15日),在栾城县殺胡林(今河北石家庄市欒城区西北)病逝,遼人将耶律德光的尸体破腹,取出内脏,装入几斗盐,带回北方,時人稱為「帝羓」,即「皇帝醃肉」之意。

\subsection{天显}

\begin{longtable}{|>{\centering\scriptsize}m{2em}|>{\centering\scriptsize}m{1.3em}|>{\centering}m{8.8em}|}
  % \caption{秦王政}\
  \toprule
  \SimHei \normalsize 年数 & \SimHei \scriptsize 公元 & \SimHei 大事件 \tabularnewline
  % \midrule
  \endfirsthead
  \toprule
  \SimHei \normalsize 年数 & \SimHei \scriptsize 公元 & \SimHei 大事件 \tabularnewline
  \midrule
  \endhead
  \midrule
  二年 & 927 & \tabularnewline\hline
  三年 & 928 & \tabularnewline\hline
  四年 & 929 & \tabularnewline\hline
  五年 & 930 & \tabularnewline\hline
  六年 & 931 & \tabularnewline\hline
  七年 & 932 & \tabularnewline\hline
  八年 & 933 & \tabularnewline\hline
  九年 & 934 & \tabularnewline\hline
  十年 & 935 & \tabularnewline\hline
  十一年 & 936 & \tabularnewline\hline
  十二年 & 937 & \tabularnewline\hline
  十三年 & 938 & \tabularnewline
  \bottomrule
\end{longtable}


\subsection{会同}


\begin{longtable}{|>{\centering\scriptsize}m{2em}|>{\centering\scriptsize}m{1.3em}|>{\centering}m{8.8em}|}
  % \caption{秦王政}\
  \toprule
  \SimHei \normalsize 年数 & \SimHei \scriptsize 公元 & \SimHei 大事件 \tabularnewline
  % \midrule
  \endfirsthead
  \toprule
  \SimHei \normalsize 年数 & \SimHei \scriptsize 公元 & \SimHei 大事件 \tabularnewline
  \midrule
  \endhead
  \midrule
  元年 & 938 & \tabularnewline\hline
  二年 & 939 & \tabularnewline\hline
  三年 & 940 & \tabularnewline\hline
  四年 & 941 & \tabularnewline\hline
  五年 & 942 & \tabularnewline\hline
  六年 & 943 & \tabularnewline\hline
  七年 & 944 & \tabularnewline\hline
  八年 & 945 & \tabularnewline\hline
  九年 & 946 & \tabularnewline\hline
  十年 & 947 & \tabularnewline
  \bottomrule
\end{longtable}

\subsection{大同}

\begin{longtable}{|>{\centering\scriptsize}m{2em}|>{\centering\scriptsize}m{1.3em}|>{\centering}m{8.8em}|}
  % \caption{秦王政}\
  \toprule
  \SimHei \normalsize 年数 & \SimHei \scriptsize 公元 & \SimHei 大事件 \tabularnewline
  % \midrule
  \endfirsthead
  \toprule
  \SimHei \normalsize 年数 & \SimHei \scriptsize 公元 & \SimHei 大事件 \tabularnewline
  \midrule
  \endhead
  \midrule
  元年 & 947 & \tabularnewline
  \bottomrule
\end{longtable}


%%% Local Variables:
%%% mode: latex
%%% TeX-engine: xetex
%%% TeX-master: "../Main"
%%% End:

%% -*- coding: utf-8 -*-
%% Time-stamp: <Chen Wang: 2019-10-15 16:18:25>

\section{世宗\tiny(947-951)}

遼世宗耶律阮(919年1月29日-951年10月7日),中国遼朝第三位皇帝(947年5月16日-951年10月7日在位),在位4年。契丹迭剌部霞濑益石烈乡耶律里(今中国内蒙古阿鲁科尔沁旗东)人,姓耶律,汉文名阮,契丹文名兀欲(又名隈欲、烏雲),他是大契丹国(後改称大辽国)皇太子、人皇王、東丹国王、遼義宗让国皇帝(追尊,未即位)耶律倍的長子、太祖耶律阿保機的长孙、太宗耶律德光之侄。

阿保机死后,世宗耶律阮之父人皇王耶律倍在权力斗争中失利,未能即位为帝,耶律阮遂失去继承皇位的权利。人皇王後来愤而投奔后唐,终于客死他乡。耶律阮则留在国内,后随叔父太宗耶律德光南征後晋。太宗在北归途中病逝后,耶律阮被随军将领拥立为帝,是为辽世宗。但世宗即位后发生多起夺权事变,统治活动被严重干扰,最终遇刺身亡,在位仅四年有余,其堂弟耶律璟继位,是为辽穆宗。

世宗虽在辽代诸帝中享国最短,却是一位有作为的皇帝。受其父耶律倍的影响,世宗在位期间推崇汉文化,推广中原制度,在世宗之孙圣宗时最后完成,促进了辽国社会的发展。

契丹神册三年,耶律阮出生,他的父亲是契丹开国皇帝耶律阿保机的长子耶律倍,母亲是耶律倍之妃萧氏(死后追谥柔贞皇后)。祖父阿保机死后,父亲耶律倍在权力斗争中失利,不得立为皇帝,耶律阮也就失去了继承皇位的权利。耶律倍后来愤而渡海投奔后唐,终于於932年末被後唐末帝李從珂殺害,客死他乡。耶律阮则留在契丹国内,其叔父太宗耶律德光爱之如己出。契丹会同九年、后晋开运三年(946年),太宗以后晋皇帝石重贵不肯称臣为由大举入侵中原,耶律阮随行军中。第二年(947年)契丹军入后晋国都东京开封府(今河南省开封市),晋帝石重贵投降,后晋灭亡。太宗改国号为“大辽”,改元大同,封耶律阮为永康王。

太宗滅後晉后在北归途中逝世,耶律阮發兵奪取南京析津府(今北京),並在随军将领拥戴下自立為皇帝,在上京(今內蒙古巴林左旗)的蕭太后述律平派其子耶律李胡在南京北部的泰德泉交戰,大敗。經過大臣耶律屋質的勸阻,太后才同意耶律阮當皇帝。世宗時任用賢臣耶律屋質,進行一系列改革,將太宗時的南面官和北面官合併,成立南北樞密院,廢南、北大王,後來南北樞密院合併,形成一個樞密院。這些改革使遼朝從部落聯盟形式進入中央集權,這些都是與遼世宗的改革分不開的。但是世宗好酒色,喜愛打獵。他晚年更是任用奸佞,大興封賞降殺,導致朝政不修,政治腐敗。遼天祿五年(951年)9月,世宗協助北漢攻後周,行軍至歸化(今內蒙古呼和浩特)的祥古山,由於其他部隊未到,所以駐紮在火神澱。其間喝酒、打人、打獵,眾將很是不滿。晚上,一直有篡位之心的耶律察割將遼世宗耶律阮殺死於夢鄉。耶律阮死時年僅34歲,在位4年。其諡號為孝和莊憲皇帝,廟號世宗。

\subsection{天禄}

\begin{longtable}{|>{\centering\scriptsize}m{2em}|>{\centering\scriptsize}m{1.3em}|>{\centering}m{8.8em}|}
  % \caption{秦王政}\
  \toprule
  \SimHei \normalsize 年数 & \SimHei \scriptsize 公元 & \SimHei 大事件 \tabularnewline
  % \midrule
  \endfirsthead
  \toprule
  \SimHei \normalsize 年数 & \SimHei \scriptsize 公元 & \SimHei 大事件 \tabularnewline
  \midrule
  \endhead
  \midrule
  元年 & 947 & \tabularnewline\hline
  二年 & 948 & \tabularnewline\hline
  三年 & 949 & \tabularnewline\hline
  四年 & 950 & \tabularnewline\hline
  五年 & 951 & \tabularnewline
  \bottomrule
\end{longtable}



%%% Local Variables:
%%% mode: latex
%%% TeX-engine: xetex
%%% TeX-master: "../Main"
%%% End:

%% -*- coding: utf-8 -*-
%% Time-stamp: <Chen Wang: 2021-11-01 16:01:26>

\section{穆宗耶律璟\tiny(951-969)}

\subsection{生平}

遼穆宗耶律璟(931年9月19日-969年3月12日),一说名耶律明,小字述律,遼朝第四位皇帝(951年10月11日-969年3月12日在位),在位18年,是為遼太宗之長子,其母為靖安皇后萧温。

於會同二年(939年)三月被封為壽安王。妻子萧氏。於天祿五年九月初八日(951年10月11日)火神淀之乱后,被立為帝,尊稱天順皇帝,改年号應曆。

遼穆宗雖討厭女色,而無所出,但卻經常酗酒,天亮才睡,中午方醒,因此長時期不理朝政,人稱之為「睡王」。另外,穆宗又好殺,經常親手殺人。同時,他又愛好打獵而「竟月不視朝」。

不過,遼穆宗也曾有才華之士可破格提拔,年老或是無能官員可增俸歸鄉,以免在其位而不謀其政的做法。

應曆十九年二月廿二日(969年3月12日),遼穆宗被侍人所弒,享年三十九歲,死後遼景宗繼位。

元朝官修正史《辽史》脱脱等的評價是:“穆宗在位十八年,知女巫妖妄见诛,谕臣下滥刑切谏,非不明也。而荒耽于酒,畋猎无厌。侦鹅失期,加炮烙铁梳之刑;获鸭甚欢,除鹰坊刺面之令。赏罚无章,朝政不视,而嗜杀不已。变起肘腋,宜哉!”

\subsection{应历}

\begin{longtable}{|>{\centering\scriptsize}m{2em}|>{\centering\scriptsize}m{1.3em}|>{\centering}m{8.8em}|}
  % \caption{秦王政}\
  \toprule
  \SimHei \normalsize 年数 & \SimHei \scriptsize 公元 & \SimHei 大事件 \tabularnewline
  % \midrule
  \endfirsthead
  \toprule
  \SimHei \normalsize 年数 & \SimHei \scriptsize 公元 & \SimHei 大事件 \tabularnewline
  \midrule
  \endhead
  \midrule
  元年 & 951 & \tabularnewline\hline
  二年 & 952 & \tabularnewline\hline
  三年 & 953 & \tabularnewline\hline
  四年 & 954 & \tabularnewline\hline
  五年 & 955 & \tabularnewline\hline
  六年 & 956 & \tabularnewline\hline
  七年 & 957 & \tabularnewline\hline
  八年 & 958 & \tabularnewline\hline
  九年 & 959 & \tabularnewline\hline
  十年 & 960 & \tabularnewline\hline
  十一年 & 961 & \tabularnewline\hline
  十二年 & 962 & \tabularnewline\hline
  十三年 & 963 & \tabularnewline\hline
  十四年 & 964 & \tabularnewline\hline
  十五年 & 965 & \tabularnewline\hline
  十六年 & 966 & \tabularnewline\hline
  十七年 & 967 & \tabularnewline\hline
  十八年 & 968 & \tabularnewline\hline
  十九年 & 969 & \tabularnewline
  \bottomrule
\end{longtable}



%%% Local Variables:
%%% mode: latex
%%% TeX-engine: xetex
%%% TeX-master: "../Main"
%%% End:

%% -*- coding: utf-8 -*-
%% Time-stamp: <Chen Wang: 2021-11-01 16:01:53>

\section{景宗耶律賢\tiny(969-982)}

\subsection{生平}

遼景宗耶律賢(948年9月1日-982年10月13日),字賢寧,遼朝第五位皇帝(969年3月13日-982年10月13日在位),在位13年,遼世宗的次子,其母為懷節皇后蕭氏。

在遼世宗在位時的政變中,耶律賢險而被殺,後來得人所救。951年,其父辽世宗被刺身亡,堂叔耶律璟即位,是为辽穆宗。969年3月12日,堂叔遼穆宗被弑,次日,耶律賢被推舉為帝,尊號天贊皇帝,改元為保寧。

耶律賢从小惊吓过度,体弱多病,皇后萧绰(953年-1009年,小字燕燕,原姓拔黎氏)则成了辽国政治军事的参与者。景宗在位時復回登聞鼓院,令百姓有申冤之地,又寬減刑法,對百姓加以安撫。

後來,景宗於乾亨四年九月廿四日(即982年10月13日)死於現今的山西省大同市,享年三十五歲,葬於乾陵,位于今辽宁省北镇市。

元朝官修正史《辽史》脱脱等的評價是:“辽兴六十馀年,神册、会同之间,日不暇给;天禄、应历之君,不令其终;保宁而来,人人望治。以景宗之资,任人不疑,信赏必罚,若可与有为也。而竭国之力以助河东,破军杀将,无救灭亡。虽一取偿于宋,得不偿失。知匡嗣之罪,数而不罚;善郭袭之谏,纳而不用;沙门昭敏以左道乱德,宠以侍中。不亦惑乎!”

\subsection{保宁}

\begin{longtable}{|>{\centering\scriptsize}m{2em}|>{\centering\scriptsize}m{1.3em}|>{\centering}m{8.8em}|}
  % \caption{秦王政}\
  \toprule
  \SimHei \normalsize 年数 & \SimHei \scriptsize 公元 & \SimHei 大事件 \tabularnewline
  % \midrule
  \endfirsthead
  \toprule
  \SimHei \normalsize 年数 & \SimHei \scriptsize 公元 & \SimHei 大事件 \tabularnewline
  \midrule
  \endhead
  \midrule
  元年 & 969 & \tabularnewline\hline
  二年 & 970 & \tabularnewline\hline
  三年 & 971 & \tabularnewline\hline
  四年 & 972 & \tabularnewline\hline
  五年 & 973 & \tabularnewline\hline
  六年 & 974 & \tabularnewline\hline
  七年 & 975 & \tabularnewline\hline
  八年 & 976 & \tabularnewline\hline
  九年 & 977 & \tabularnewline\hline
  十年 & 978 & \tabularnewline\hline
  十一年 & 979 & \tabularnewline
  \bottomrule
\end{longtable}

\subsection{乾亨}

\begin{longtable}{|>{\centering\scriptsize}m{2em}|>{\centering\scriptsize}m{1.3em}|>{\centering}m{8.8em}|}
  % \caption{秦王政}\
  \toprule
  \SimHei \normalsize 年数 & \SimHei \scriptsize 公元 & \SimHei 大事件 \tabularnewline
  % \midrule
  \endfirsthead
  \toprule
  \SimHei \normalsize 年数 & \SimHei \scriptsize 公元 & \SimHei 大事件 \tabularnewline
  \midrule
  \endhead
  \midrule
  元年 & 979 & \tabularnewline\hline
  二年 & 980 & \tabularnewline\hline
  三年 & 981 & \tabularnewline\hline
  四年 & 982 & \tabularnewline\hline
  五年 & 983 & \tabularnewline
  \bottomrule
\end{longtable}



%%% Local Variables:
%%% mode: latex
%%% TeX-engine: xetex
%%% TeX-master: "../Main"
%%% End:

%% -*- coding: utf-8 -*-
%% Time-stamp: <Chen Wang: 2021-11-01 16:02:03>

\section{圣宗耶律隆緒\tiny(982-1031)}

\subsection{生平}

遼聖宗耶律隆緒(972年1月16日-1031年6月25日),遼朝第六位皇帝(982年10月14日-1031年6月25日在位),契丹名文殊奴。是遼在位最長的皇帝,在位49年。遼景宗長子,母皇后萧绰。

辽圣宗即位前曾被封為梁王。乾亨四年(982年)九月壬子(10月13日),遼景宗去世,次日,耶律隆绪登基,即辽圣宗。

他即位時,年12歲,太后蕭綽執政。983年改元統和,并将国号“大辽”改为“大契丹”。统和四年(986年),立皇后萧氏。蕭太后執政期間,進行了改革,並且勵精圖治,注重農桑,興修水利,減少賦稅,整頓吏治,訓練軍隊,使百姓富裕,國勢強盛。統和二十二年(1004年)遼聖宗与宋真宗達成澶淵之盟。

統和二十七年(1009年)聖宗全面親政後,遼朝(契丹)已進入鼎盛,基本上延續蕭太后執政時的遼朝風貌,並且還反對嚴刑峻法,不給貪官可乘之機。在位其間四方征戰,進入遼朝疆域的頂峰。

晚年时,辽圣宗迷信佛教,窮途奢侈,遼國勢走向下坡路。遼聖宗死於太平十一年六月初三日(1031年6月25日),終年61歲,葬於庆云山。謚號為文武大孝宣肅景皇帝。

元朝官修正史《辽史》脱脱等的評價是:“圣宗幼冲嗣位,政出慈闱。及宋人二道来攻,亲御甲胄,一举而复燕、云,破信、彬,再举而躏河、朔,不亦伟欤!既而侈心一启,佳兵不祥,东有茶、陀之败,西有甘州之丧,此狃于常胜之过也。然其践阼四十九年,理冤滞,举才行,察贪残,抑奢僣,录死事之子孙,振诸部之贫乏,责迎合不忠之罪,却高丽女乐之归。辽之诸帝,在位长久,令名无穷,其唯圣宗乎!”

\subsection{统合}

\begin{longtable}{|>{\centering\scriptsize}m{2em}|>{\centering\scriptsize}m{1.3em}|>{\centering}m{8.8em}|}
  % \caption{秦王政}\
  \toprule
  \SimHei \normalsize 年数 & \SimHei \scriptsize 公元 & \SimHei 大事件 \tabularnewline
  % \midrule
  \endfirsthead
  \toprule
  \SimHei \normalsize 年数 & \SimHei \scriptsize 公元 & \SimHei 大事件 \tabularnewline
  \midrule
  \endhead
  \midrule
  元年 & 983 & \tabularnewline\hline
  二年 & 984 & \tabularnewline\hline
  三年 & 985 & \tabularnewline\hline
  四年 & 986 & \tabularnewline\hline
  五年 & 987 & \tabularnewline\hline
  六年 & 988 & \tabularnewline\hline
  七年 & 989 & \tabularnewline\hline
  八年 & 990 & \tabularnewline\hline
  九年 & 991 & \tabularnewline\hline
  十年 & 992 & \tabularnewline\hline
  十一年 & 993 & \tabularnewline\hline
  十二年 & 994 & \tabularnewline\hline
  十三年 & 995 & \tabularnewline\hline
  十四年 & 996 & \tabularnewline\hline
  十五年 & 997 & \tabularnewline\hline
  十六年 & 998 & \tabularnewline\hline
  十七年 & 999 & \tabularnewline\hline
  十八年 & 1000 & \tabularnewline\hline
  十九年 & 1001 & \tabularnewline\hline
  二十年 & 1002 & \tabularnewline\hline
  二一年 & 1003 & \tabularnewline\hline
  二二年 & 1004 & \tabularnewline\hline
  二三年 & 1005 & \tabularnewline\hline
  二四年 & 1006 & \tabularnewline\hline
  二五年 & 1007 & \tabularnewline\hline
  二六年 & 1008 & \tabularnewline\hline
  二七年 & 1009 & \tabularnewline\hline
  二八年 & 1010 & \tabularnewline\hline
  二九年 & 1011 & \tabularnewline\hline
  三十年 & 1012 & \tabularnewline
  \bottomrule
\end{longtable}

\subsection{开泰}

\begin{longtable}{|>{\centering\scriptsize}m{2em}|>{\centering\scriptsize}m{1.3em}|>{\centering}m{8.8em}|}
  % \caption{秦王政}\
  \toprule
  \SimHei \normalsize 年数 & \SimHei \scriptsize 公元 & \SimHei 大事件 \tabularnewline
  % \midrule
  \endfirsthead
  \toprule
  \SimHei \normalsize 年数 & \SimHei \scriptsize 公元 & \SimHei 大事件 \tabularnewline
  \midrule
  \endhead
  \midrule
  元年 & 1012 & \tabularnewline\hline
  二年 & 1013 & \tabularnewline\hline
  三年 & 1014 & \tabularnewline\hline
  四年 & 1015 & \tabularnewline\hline
  五年 & 1016 & \tabularnewline\hline
  六年 & 1017 & \tabularnewline\hline
  七年 & 1018 & \tabularnewline\hline
  八年 & 1019 & \tabularnewline\hline
  九年 & 1020 & \tabularnewline\hline
  十年 & 1021 & \tabularnewline
  \bottomrule
\end{longtable}

\subsection{太平}

\begin{longtable}{|>{\centering\scriptsize}m{2em}|>{\centering\scriptsize}m{1.3em}|>{\centering}m{8.8em}|}
  % \caption{秦王政}\
  \toprule
  \SimHei \normalsize 年数 & \SimHei \scriptsize 公元 & \SimHei 大事件 \tabularnewline
  % \midrule
  \endfirsthead
  \toprule
  \SimHei \normalsize 年数 & \SimHei \scriptsize 公元 & \SimHei 大事件 \tabularnewline
  \midrule
  \endhead
  \midrule
  元年 & 1021 & \tabularnewline\hline
  二年 & 1022 & \tabularnewline\hline
  三年 & 1023 & \tabularnewline\hline
  四年 & 1024 & \tabularnewline\hline
  五年 & 1025 & \tabularnewline\hline
  六年 & 1026 & \tabularnewline\hline
  七年 & 1027 & \tabularnewline\hline
  八年 & 1028 & \tabularnewline\hline
  九年 & 1029 & \tabularnewline\hline
  十年 & 1030 & \tabularnewline\hline
  十一年 & 1031 & \tabularnewline
  \bottomrule
\end{longtable}



%%% Local Variables:
%%% mode: latex
%%% TeX-engine: xetex
%%% TeX-master: "../Main"
%%% End:

%% -*- coding: utf-8 -*-
%% Time-stamp: <Chen Wang: 2021-11-01 16:02:08>

\section{兴宗耶律宗真\tiny(1031-1055)}

\subsection{生平}

遼興宗耶律宗真(1016年4月3日-1055年8月28日),契丹第七位皇帝(1031年6月25日-1055年8月28日在位),契丹名只骨。在位24年,享年40歲,謚孝章皇帝。他是遼聖宗的長子,母乃宮女蕭耨斤。

耶律宗真生于1016年,其后,由辽圣宗的皇后蕭菩薩哥抚养。《辽史》记耶律宗真为“圣宗长子”,实际上辽圣宗第六子耶律宗愿的生年是1008年或1009年,耶律宗真和同母弟耶律宗元应该是辽圣宗最年幼的两个儿子。虽然年幼,但与其他四位皇子相比,只有耶律宗真兄弟的生母蕭耨斤出身于契丹萧氏,其他皇子的生母出身于汉族或不详。

太平元年(1021年),耶律宗真被冊立為太子,太平十年(1030年)六月判北南院枢密院事。太平十一年(1031年6月25日)夏六月己卯,辽圣宗逝世,同时,耶律宗真繼承皇位,改元景福。興宗繼位後,其母順聖元妃蕭耨斤自立為皇太后攝政,並把聖宗的齊天皇后迫死。並重用了在聖宗時代被裁示永不錄用的貪官污吏以及其娘家的人。

景福二年十一月,興宗上太后尊號為法天應運仁德章聖皇太后(法天太后),而興宗被群臣上尊號為文武仁聖昭孝皇帝,改元重熙。

重熙三年(1034年),法天太后企圖廢掉興宗,改立次子宗元(遼史作重元),重元告訴其兄興宗,興宗發動政變,迫法天太后「躬守慶陵」。大殺太后親信。七月,興宗親政。

興宗在位時,遼國勢已日益衰落。而有興宗一朝,奸佞當權,政治腐敗,百姓困苦,軍隊衰弱。面對日益衰落的國勢,興宗連年征戰,多次征伐西夏;逼迫宋朝多交納歲幣,反而使遼內部百姓怨聲載道,民不聊生。興宗還迷信佛教,窮途奢極。興宗曾與其弟宗元賭博,一連輸了幾個城池。

他對自己的弟弟宗元非常感激,一次酒醉時答應百年之後傳位給宗元,其子耶律洪基(後來的遼道宗)也未曾封為皇太子,只封為天下兵馬大元帥而已。種下了道宗繼位後,宗元父子企圖謀奪帝位的惡果。

重熙二十四年八月初四日(1055年8月28日),興宗駕崩。

元朝官修正史《辽史》脱脱等的評價是:“兴宗即位,年十有六矣,不能先尊母后而尊其母,以致临朝专政,贼杀不辜,又不能以礼几谏,使齐天死于弑逆,有亏王者之孝,惜哉!若夫大行在殡,饮酒博鞠,叠见简书。及其谒遗像而哀恸,受宋吊而衰绖,所为若出二人。何为其然欤?至于感富弼之言而申南宋之好,许谅祚之盟而罢西夏之兵,边鄙不耸,政治内修,亲策进士,大修条制,下至士庶,得陈便宜,则求治之志切矣。于时左右大臣,曾不闻一贤之进,一事之谏,欲庶几古帝王之风,其可得乎?虽然,圣宗而下,可谓贤君矣。 ”

\subsection{景福}

\begin{longtable}{|>{\centering\scriptsize}m{2em}|>{\centering\scriptsize}m{1.3em}|>{\centering}m{8.8em}|}
  % \caption{秦王政}\
  \toprule
  \SimHei \normalsize 年数 & \SimHei \scriptsize 公元 & \SimHei 大事件 \tabularnewline
  % \midrule
  \endfirsthead
  \toprule
  \SimHei \normalsize 年数 & \SimHei \scriptsize 公元 & \SimHei 大事件 \tabularnewline
  \midrule
  \endhead
  \midrule
  元年 & 1031 & \tabularnewline\hline
  二年 & 1032 & \tabularnewline
  \bottomrule
\end{longtable}

\subsection{重熙}

\begin{longtable}{|>{\centering\scriptsize}m{2em}|>{\centering\scriptsize}m{1.3em}|>{\centering}m{8.8em}|}
  % \caption{秦王政}\
  \toprule
  \SimHei \normalsize 年数 & \SimHei \scriptsize 公元 & \SimHei 大事件 \tabularnewline
  % \midrule
  \endfirsthead
  \toprule
  \SimHei \normalsize 年数 & \SimHei \scriptsize 公元 & \SimHei 大事件 \tabularnewline
  \midrule
  \endhead
  \midrule
  元年 & 1032 & \tabularnewline\hline
  二年 & 1033 & \tabularnewline\hline
  三年 & 1034 & \tabularnewline\hline
  四年 & 1035 & \tabularnewline\hline
  五年 & 1036 & \tabularnewline\hline
  六年 & 1037 & \tabularnewline\hline
  七年 & 1038 & \tabularnewline\hline
  八年 & 1039 & \tabularnewline\hline
  九年 & 1040 & \tabularnewline\hline
  十年 & 1041 & \tabularnewline\hline
  十一年 & 1042 & \tabularnewline\hline
  十二年 & 1043 & \tabularnewline\hline
  十三年 & 1044 & \tabularnewline\hline
  十四年 & 1045 & \tabularnewline\hline
  十五年 & 1046 & \tabularnewline\hline
  十六年 & 1047 & \tabularnewline\hline
  十七年 & 1048 & \tabularnewline\hline
  十八年 & 1049 & \tabularnewline\hline
  十九年 & 1050 & \tabularnewline\hline
  二十年 & 1051 & \tabularnewline\hline
  二一年 & 1052 & \tabularnewline\hline
  二二年 & 1053 & \tabularnewline\hline
  二三年 & 1054 & \tabularnewline\hline
  二四年 & 1055 & \tabularnewline
  \bottomrule
\end{longtable}


%%% Local Variables:
%%% mode: latex
%%% TeX-engine: xetex
%%% TeX-master: "../Main"
%%% End:

%% -*- coding: utf-8 -*-
%% Time-stamp: <Chen Wang: 2019-12-26 10:58:15>

\section{道宗\tiny(1055-1101)}

\subsection{生平}

遼道宗耶律洪基(1032年9月14日-1101年2月12日),契丹及遼朝第八位皇帝(1055年8月28日-1101年2月12日在位),在位長達46年,僅次於遼聖宗。他是遼興宗的長子,契丹名查剌。

重熙二十四年八月初四(1055年8月28日),興宗駕崩,即位於柩前。改元清寧。

道宗繼位後,封皇叔宗元為皇太叔,清寧二年又加天下兵馬大元帥。四年又賜金券等,極盡榮寵。但宗元始終有謀奪帝位的意圖,在清寧九年(1063年)七月,宗元聽從兒子的勸說,發動叛亂,自立為帝,未幾被道宗所平,宗元自盡。史稱灤河之亂。

咸雍二年(1066年),辽道宗把国号“契丹”改为“大辽”。

他在位期間,遼政治腐敗,國勢逐漸衰落。道宗並沒有進行改革圖新,而且本人也腐朽奢侈,這時地主官僚急劇兼併土地,百姓痛苦不堪,怨聲載道。道宗還重用耶律乙辛等奸佞,自己不理朝政,導致他聽信乙辛的讒言,相信皇后蕭觀音與伶官趙惟一通姦而賜死皇后,史稱十香詞冤案。而同時乙辛為防太子耶律濬登基對自己不利(因為道宗只有皇太子這個兒子),故陷害皇太子謀反,殺害了皇太子。

後來,一位姓李的婦女向道宗進「挾穀歌」,道宗才把皇太子的兒女接進宮,大康五年(1079年)七月,耶律乙辛乘道宗遊獵的時候謀害皇孫,道宗接納大臣的勸諫,命皇孫一同秋獵,才化解乙辛的陰謀。

大康九年,道宗追封故太子為昭懷太子,以天子禮改葬。同年十月,耶律乙辛企圖帶私藏武器到宋朝避難,事發,被誅。

道宗篤信佛教,在位期間曾大修佛寺、佛塔。遼的腐朽統治引起了各族人民的不滿,其間被遼統治者壓迫的女真族開始興起,最終成為遼的掘墓人。

寿昌七年正月十三日(1101年2月12日),遼道宗去世,終年70歲。

元朝官修正史《辽史》脱脱等的評價是:“道宗初即位,求直言,访治道,劝农兴学,救灾恤患,粲然可观。及夫谤讪之令既行,告讦之赏日重。群邪并兴,谗巧竞进。贼及骨肉,皇基浸危。众正沦胥,诸部反侧,甲兵之用,无宁岁矣。一岁而饭僧三十六万,一日而祝发三千。徒勤小惠,蔑计大本,尚足与论治哉? ”

\subsection{清宁}

\begin{longtable}{|>{\centering\scriptsize}m{2em}|>{\centering\scriptsize}m{1.3em}|>{\centering}m{8.8em}|}
  % \caption{秦王政}\
  \toprule
  \SimHei \normalsize 年数 & \SimHei \scriptsize 公元 & \SimHei 大事件 \tabularnewline
  % \midrule
  \endfirsthead
  \toprule
  \SimHei \normalsize 年数 & \SimHei \scriptsize 公元 & \SimHei 大事件 \tabularnewline
  \midrule
  \endhead
  \midrule
  元年 & 1055 & \tabularnewline\hline
  二年 & 1056 & \tabularnewline\hline
  三年 & 1057 & \tabularnewline\hline
  四年 & 1058 & \tabularnewline\hline
  五年 & 1059 & \tabularnewline\hline
  六年 & 1060 & \tabularnewline\hline
  七年 & 1061 & \tabularnewline\hline
  八年 & 1062 & \tabularnewline\hline
  九年 & 1063 & \tabularnewline\hline
  十年 & 1064 & \tabularnewline
  \bottomrule
\end{longtable}

\subsection{咸雍}

\begin{longtable}{|>{\centering\scriptsize}m{2em}|>{\centering\scriptsize}m{1.3em}|>{\centering}m{8.8em}|}
  % \caption{秦王政}\
  \toprule
  \SimHei \normalsize 年数 & \SimHei \scriptsize 公元 & \SimHei 大事件 \tabularnewline
  % \midrule
  \endfirsthead
  \toprule
  \SimHei \normalsize 年数 & \SimHei \scriptsize 公元 & \SimHei 大事件 \tabularnewline
  \midrule
  \endhead
  \midrule
  元年 & 1065 & \tabularnewline\hline
  二年 & 1066 & \tabularnewline\hline
  三年 & 1067 & \tabularnewline\hline
  四年 & 1068 & \tabularnewline\hline
  五年 & 1069 & \tabularnewline\hline
  六年 & 1070 & \tabularnewline\hline
  七年 & 1071 & \tabularnewline\hline
  八年 & 1072 & \tabularnewline\hline
  九年 & 1073 & \tabularnewline\hline
  十年 & 1074 & \tabularnewline
  \bottomrule
\end{longtable}

\subsection{大康}

\begin{longtable}{|>{\centering\scriptsize}m{2em}|>{\centering\scriptsize}m{1.3em}|>{\centering}m{8.8em}|}
  % \caption{秦王政}\
  \toprule
  \SimHei \normalsize 年数 & \SimHei \scriptsize 公元 & \SimHei 大事件 \tabularnewline
  % \midrule
  \endfirsthead
  \toprule
  \SimHei \normalsize 年数 & \SimHei \scriptsize 公元 & \SimHei 大事件 \tabularnewline
  \midrule
  \endhead
  \midrule
  元年 & 1075 & \tabularnewline\hline
  二年 & 1076 & \tabularnewline\hline
  三年 & 1077 & \tabularnewline\hline
  四年 & 1078 & \tabularnewline\hline
  五年 & 1079 & \tabularnewline\hline
  六年 & 1080 & \tabularnewline\hline
  七年 & 1081 & \tabularnewline\hline
  八年 & 1082 & \tabularnewline\hline
  九年 & 1083 & \tabularnewline\hline
  十年 & 1084 & \tabularnewline
  \bottomrule
\end{longtable}

\subsection{大安}

\begin{longtable}{|>{\centering\scriptsize}m{2em}|>{\centering\scriptsize}m{1.3em}|>{\centering}m{8.8em}|}
  % \caption{秦王政}\
  \toprule
  \SimHei \normalsize 年数 & \SimHei \scriptsize 公元 & \SimHei 大事件 \tabularnewline
  % \midrule
  \endfirsthead
  \toprule
  \SimHei \normalsize 年数 & \SimHei \scriptsize 公元 & \SimHei 大事件 \tabularnewline
  \midrule
  \endhead
  \midrule
  元年 & 1085 & \tabularnewline\hline
  二年 & 1086 & \tabularnewline\hline
  三年 & 1087 & \tabularnewline\hline
  四年 & 1088 & \tabularnewline\hline
  五年 & 1089 & \tabularnewline\hline
  六年 & 1090 & \tabularnewline\hline
  七年 & 1091 & \tabularnewline\hline
  八年 & 1092 & \tabularnewline\hline
  九年 & 1093 & \tabularnewline\hline
  十年 & 1094 & \tabularnewline
  \bottomrule
\end{longtable}

\subsection{寿昌}

\begin{longtable}{|>{\centering\scriptsize}m{2em}|>{\centering\scriptsize}m{1.3em}|>{\centering}m{8.8em}|}
  % \caption{秦王政}\
  \toprule
  \SimHei \normalsize 年数 & \SimHei \scriptsize 公元 & \SimHei 大事件 \tabularnewline
  % \midrule
  \endfirsthead
  \toprule
  \SimHei \normalsize 年数 & \SimHei \scriptsize 公元 & \SimHei 大事件 \tabularnewline
  \midrule
  \endhead
  \midrule
  元年 & 1095 & \tabularnewline\hline
  二年 & 1096 & \tabularnewline\hline
  三年 & 1097 & \tabularnewline\hline
  四年 & 1098 & \tabularnewline\hline
  五年 & 1099 & \tabularnewline\hline
  六年 & 1100 & \tabularnewline\hline
  七年 & 1101 & \tabularnewline
  \bottomrule
\end{longtable}


%%% Local Variables:
%%% mode: latex
%%% TeX-engine: xetex
%%% TeX-master: "../Main"
%%% End:

%% -*- coding: utf-8 -*-
%% Time-stamp: <Chen Wang: 2019-10-15 16:31:37>

\section{天祚帝\tiny(1101-1125)}

遼天祚帝耶律延禧(1075年6月5日-1128年或1156年),字延宁,小名阿果,是遼國西遷前的最后一位皇帝,他的统治时间是从1101年2月12日至1125年3月26日,在位24年。

天祚帝是辽道宗的孙子,他的父亲是道宗的太子耶律濬,母亲是貞順皇后萧氏。六岁时他被封为梁王,九岁时封为燕国王。

寿昌七年正月十三日(1101年2月12日),道宗崩,临死前立耶律延禧为继承人,耶律延禧奉遗诏即皇帝位于柩前。延禧以「天祚皇帝」作為自己的尊號。二月壬辰改元乾統。

天祚帝继位后西夏崇宗因受到北宋攻击一再向辽求援,并求尚天祚帝女公主为妻,最后天祚帝于1105年将一个族女封为公主嫁给了夏崇宗,并派使者赴宋,劝宋对西夏罢兵。

1112年二月丁酉天祚帝赴春州,召集附近的女真族酋长来朝,宴席中醉酒后令女真酋长为他跳舞,只有完颜阿骨打不肯。天祚帝不以为意,但从此完颜阿骨打与遼國之间不和。从九月开始完颜阿骨打不再奉诏,并开始对其他不服从自己的女真部落用兵。1114年春,完颜阿骨打正式起兵反辽。一开始天祚帝不将阿骨打当作大威胁,但是1114年天祚帝所有派去镇压阿骨打的军队全部被战败。

1115年天祚帝終於开始觉察到女真的威胁勢力,下令亲征,但是辽军到处被女真打败,与此同时遼國国内也发生叛乱,耶律章奴在上京临潢府叛乱,虽然这场叛乱很快就被平定,但是这场叛乱分裂了遼國内部。此后位于原渤海国的东京辽阳府也发生叛乱自立。这场叛乱一直到1116年四月才被平定。但是在五月女真就借机占领了辽阳和瀋州。1117年女真攻春州,辽军不战自败。这年完颜阿骨打称帝,建立金朝。

1120年金攻克上京臨潢府,留守降。到1121年辽已经失去了其疆域之半。而遼國内部又发生了因为皇位继承问题而爆发的内乱,1122年天祚帝杀了自己的长子耶律敖卢斡,这使得更多的辽國军人感到不安而投靠金朝。四月,金攻克辽西京大同府。由于战场上消息不通,遼國内部又以为天祚帝在前线阵亡或被围,于是在臨潢立耶律淳为皇帝,进一步扩大了遼國内部的混乱。而遼國的大臣也各不自保,有的与北宋大臣童贯通气打算投降宋朝的,有的则想投降金朝。十一月居庸关失守,十二月辽南京被攻破。1123年正月上京叛金。

到1124年天祚帝已经失去了遼國的大部分土地而退出漠外,他的儿子和家属大多数被杀或被俘,虽然他还打算重新守護燕州和云州,但是实际上他已经没有多少希望了。保大五年二月二十日(1125年3月26日)天祚帝在应州为金人完颜娄室等所俘,八月被解送金上京,被降为海滨王。金太宗天會六年(1128年)病死。金皇統元年(1141年),改封豫王。皇統五年(1145年),葬於乾陵旁。

《大宋宣和遺事》則記載南宋紹興二十六年(金朝正隆元年,1156年)六月,金朝皇帝完顏亮命令56歲的宋欽宗趙桓和81歲的耶律延禧去比賽馬球,趙桓中途從馬上跌下來,被馬亂踐而死,耶律延禧則因善騎術,企圖縱馬衝出重圍逃命,結果被金人以亂箭射死。

元朝官修正史《辽史》脱脱等的評價是:“辽起朔野,兵甲之盛,鼓行皞外,席卷河朔,树晋植汉,何其壮欤?太祖、太宗乘百战之势,辑新造之邦,英谋睿略,可谓远矣。虽以世宗中才,穆宗残暴,连遘弑逆,而神器不摇。盖由祖宗威令犹足以震叠其国人也。圣宗以来,内修政治,外拓疆宇,既而申固邻好,四境乂安。维侍二百余年之基,有自来矣。降臻天祚,既丁末运,又觖人望,崇信奸回,自椓国本,群下离心。金兵一集,内难先作,废立之谋,叛亡之迹,相继蜂起。驯致土崩瓦解,不可复支,良可哀也!耶律与萧,世为甥舅,义同休戚,奉先挟私灭公,首祸构难,一至于斯。天祚穷蹙,始悟奉先误己,不几晚乎!淳、雅里所谓名不正,言不顺,事不成者也。大石苟延,彼善于此,亦几何哉?”

\subsection{乾统}

\begin{longtable}{|>{\centering\scriptsize}m{2em}|>{\centering\scriptsize}m{1.3em}|>{\centering}m{8.8em}|}
  % \caption{秦王政}\
  \toprule
  \SimHei \normalsize 年数 & \SimHei \scriptsize 公元 & \SimHei 大事件 \tabularnewline
  % \midrule
  \endfirsthead
  \toprule
  \SimHei \normalsize 年数 & \SimHei \scriptsize 公元 & \SimHei 大事件 \tabularnewline
  \midrule
  \endhead
  \midrule
  元年 & 1101 & \tabularnewline\hline
  二年 & 1102 & \tabularnewline\hline
  三年 & 1103 & \tabularnewline\hline
  四年 & 1104 & \tabularnewline\hline
  五年 & 1105 & \tabularnewline\hline
  六年 & 1106 & \tabularnewline\hline
  七年 & 1107 & \tabularnewline\hline
  八年 & 1108 & \tabularnewline\hline
  九年 & 1109 & \tabularnewline\hline
  十年 & 1110 & \tabularnewline
  \bottomrule
\end{longtable}

\subsection{天庆}

\begin{longtable}{|>{\centering\scriptsize}m{2em}|>{\centering\scriptsize}m{1.3em}|>{\centering}m{8.8em}|}
  % \caption{秦王政}\
  \toprule
  \SimHei \normalsize 年数 & \SimHei \scriptsize 公元 & \SimHei 大事件 \tabularnewline
  % \midrule
  \endfirsthead
  \toprule
  \SimHei \normalsize 年数 & \SimHei \scriptsize 公元 & \SimHei 大事件 \tabularnewline
  \midrule
  \endhead
  \midrule
  元年 & 1111 & \tabularnewline\hline
  二年 & 1112 & \tabularnewline\hline
  三年 & 1113 & \tabularnewline\hline
  四年 & 1114 & \tabularnewline\hline
  五年 & 1115 & \tabularnewline\hline
  六年 & 1116 & \tabularnewline\hline
  七年 & 1117 & \tabularnewline\hline
  八年 & 1118 & \tabularnewline\hline
  九年 & 1119 & \tabularnewline\hline
  十年 & 1120 & \tabularnewline
  \bottomrule
\end{longtable}

\subsection{保大}

\begin{longtable}{|>{\centering\scriptsize}m{2em}|>{\centering\scriptsize}m{1.3em}|>{\centering}m{8.8em}|}
  % \caption{秦王政}\
  \toprule
  \SimHei \normalsize 年数 & \SimHei \scriptsize 公元 & \SimHei 大事件 \tabularnewline
  % \midrule
  \endfirsthead
  \toprule
  \SimHei \normalsize 年数 & \SimHei \scriptsize 公元 & \SimHei 大事件 \tabularnewline
  \midrule
  \endhead
  \midrule
  元年 & 1121 & \tabularnewline\hline
  二年 & 1122 & \tabularnewline\hline
  三年 & 1123 & \tabularnewline\hline
  四年 & 1124 & \tabularnewline\hline
  五年 & 1125 & \tabularnewline
  \bottomrule
\end{longtable}


%%% Local Variables:
%%% mode: latex
%%% TeX-engine: xetex
%%% TeX-master: "../Main"
%%% End:

%% -*- coding: utf-8 -*-
%% Time-stamp: <Chen Wang: 2019-10-15 16:34:10>

\section{北辽\tiny(1122)}

北遼,於1122年3月立國,是時辽朝天祚帝被金兵所迫,流亡夹山,耶律淳在燕京被耶律大石等人擁立為君主,是為北遼的開始。1122年6月24日,耶律淳病死,德妃蕭普賢女以皇太后身份攝政,期间击退宋朝进攻(宣和北伐)。1123年2月2日,金朝攻佔燕京,蕭德妃和耶律大石投奔天祚帝,北遼滅亡,國祚不足一年。後來,萧德妃因為謀反而被殺,但耶律大石卻得到赦免。

\subsection{宣宗\tiny(1122)}

遼宣宗耶律淳(1063年-1122年),小字涅里,是北遼開國皇帝,為遼兴宗第四子宋魏國王耶律和鲁斡之子。淳一出生就由其祖母遼興宗的仁懿皇后撫養,長大成人之後,好文學。遼道宗太子耶律濬被殺害之後,遼道宗曾打算立侄子淳為嗣,後罷,封北平郡王,出為彰聖等軍節度使。

天祚帝即位。乾統元年(1101年)封耶律和鲁斡為天下兵馬大元帅,此意味著有皇位的繼承權,封淳為鄭王。乾統三年(1103年)封耶律和鲁斡為皇太叔,進封淳為越國王。乾統六年(1106年),拜為南府宰相,創議制訂兩府禮儀,進封為魏國王。乾統十年(1110年),耶律和鲁斡去世,淳襲南京留職,冬夏入朝,寵冠諸王。

天慶五年(1115年),耶律章奴謀反,打算迎立耶律淳為帝。耶律淳不從。次年(1116年)六月,耶律淳進封秦晉國王,拜都元帥,賜金券,免漢拜禮,不名。

保大二年(1122年)正月,金軍攻克遼中京,天祚帝被金兵所迫,流亡夾山。奚王回離保和林牙耶律大石援引唐肅宗靈武稱帝的例子,勸說耶律淳稱帝。三月,淳即皇帝位,百官上尊號為天錫皇帝,改年號建福元年,遥降天祚皇帝为湘阴王,封妻蕭普賢女為德妃,並遣使奉表于金國,乞为附庸。

六月,耶律淳事未完成就病死,終年六十歲。百官上諡号孝章皇帝,庙号宣宗,葬燕京西部的香山永安陵。

\subsubsection{建福}


\begin{longtable}{|>{\centering\scriptsize}m{2em}|>{\centering\scriptsize}m{1.3em}|>{\centering}m{8.8em}|}
  % \caption{秦王政}\
  \toprule
  \SimHei \normalsize 年数 & \SimHei \scriptsize 公元 & \SimHei 大事件 \tabularnewline
  % \midrule
  \endfirsthead
  \toprule
  \SimHei \normalsize 年数 & \SimHei \scriptsize 公元 & \SimHei 大事件 \tabularnewline
  \midrule
  \endhead
  \midrule
  元年 & 1122 & \tabularnewline
  \bottomrule
\end{longtable}

\subsection{萧普贤女\tiny(1122)}

蕭普賢女(?-1123年),為北遼宣宗耶律淳的德妃,宣宗遺詔立天祚帝耶律延禧第五子耶律定為皇帝,但他在天祚帝身邊,不在燕京,只能遙立。德妃被立為皇太后,稱制,改建福元年為德興元年。

此時大臣李處溫父子覺得前景不妙,打算向南私通宋的童貫,欲劫持德妃納土於宋。向北私通金人,作金的內應。後她發現他私通宋、金的罪行把他拘捕並賜死。

當年十一月,德妃五次上表給金朝,只要允許立耶律定為北遼皇帝,其他條件均答應,金人不許,她只好派兵把守居庸關,沒能守住,金兵直奔燕京。德妃帶著隨從的官員投靠天祚帝,天祚帝將她誅殺。

\subsubsection{德兴}

\begin{longtable}{|>{\centering\scriptsize}m{2em}|>{\centering\scriptsize}m{1.3em}|>{\centering}m{8.8em}|}
  % \caption{秦王政}\
  \toprule
  \SimHei \normalsize 年数 & \SimHei \scriptsize 公元 & \SimHei 大事件 \tabularnewline
  % \midrule
  \endfirsthead
  \toprule
  \SimHei \normalsize 年数 & \SimHei \scriptsize 公元 & \SimHei 大事件 \tabularnewline
  \midrule
  \endhead
  \midrule
  元年 & 1122 & \tabularnewline
  \bottomrule
\end{longtable}



%%% Local Variables:
%%% mode: latex
%%% TeX-engine: xetex
%%% TeX-master: "../Main"
%%% End:

%% -*- coding: utf-8 -*-
%% Time-stamp: <Chen Wang: 2021-11-01 16:12:23>

\section{西辽\tiny(1124-1218)}

\subsection{简介}

西辽(1124年-1218年),又称喀喇契丹,是契丹人耶律大石建立的国家。耶律大石原本效力于辽天祚帝,在辽朝即将灭亡之际出奔。1124年,耶律大石称王,到达可敦城(今蒙古国布尔干省青托罗盖古回鹘城)建立根据地。1132年,在叶密立(今新疆维吾尔自治区额敏县)称“菊儿汗”,西辽帝国正式建立。随后耶律大石向新疆、蒙古高原、中亚及西亚地区扩张,建都于虎思斡鲁朵(今吉尔吉斯斯坦托克玛克东南布拉纳)。在1141年的卡特万之战,击败塞尔柱帝国联军,成为中亚霸主,将威名远播至欧洲。高昌回鹘、西喀喇汗国、东喀喇汗国及花剌子模先后臣服于强盛期的西辽。耶律大石死后,历经萧塔不烟、耶律夷列、耶律普速完三代君主后,到耶律直鲁古时期,由于长期对外战争,使西辽的国力走向衰落,最终被屈出律篡国。蒙古帝国崛起后,1218年,西辽被蒙古帝国灭亡。

\subsection{德宗耶律大石\tiny(1124-1143)}

\subsubsection{生平}

遼德宗耶律大石(1094年-1143年),又称大石林牙或林牙大石。字重德,契丹人。西辽開國皇帝,庙号德宗,在位20年。

耶律大石早年效力于辽天祚帝,辽天祚帝出奔后,耶律大石参与拥立耶律淳和萧德妃,在北宋、金朝两面夹击的情况下,积极维持风雨飘摇的北辽,两次率军以少胜多击败北宋的进攻。北辽灭亡后,耶律大石投奔天祚帝,在辽朝即将灭亡之际出奔。1124年,耶律大石称遼王建號延慶,到达可敦城(今蒙古国布尔干省青托罗盖古回鹘城)建立根据地。1132年,在叶密立(今新疆维吾尔自治区额敏县)称“菊儿汗”,西辽帝国正式建立。随后耶律大石向新疆、蒙古高原、中亚及西亚地区扩张,建都于虎思斡鲁朵(今吉尔吉斯斯坦托克玛克东南布拉纳)。在1141年的卡特万之战,击败塞尔柱帝国联军,成为中亚霸主,将威名远播至欧洲。高昌回鹘、西喀喇汗国、东喀喇汗国及花剌子模先后臣服于强盛期的西辽。1143年,耶律大石去世。

耶律大石在军事、政治和外交上都有成就,欧洲得知其西征的事迹,流传着祭司王约翰的传说。耶律大石的名字也成为西辽帝国的代称,在耶律大石去世后多年,很多国家仍用“大石”称呼西辽的后代君主。

耶律大石是辽朝开国君主耶律阿保机的八世孙,精通契丹语和汉语,擅长弓马骑射。1115年,耶律大石中进士入翰林,初为翰林应奉,不久累迁翰林承旨。根据辽朝的科举制度,殿试头名才有入翰林应奉的资格。因契丹语称翰林为林牙,耶律大石又被称为大石林牙或林牙大石。后历任泰州、祥州刺史,辽兴军节度使。

历经200多年统治的辽朝国力逐渐走向衰弱,取而代之的是女真族建立的金朝。在金军势如破竹的攻击下,辽朝节节败退。1122年,金军攻克辽中京大定府和泽州,辽天祚帝如惊弓之鸟,从居庸关至鸳鸯泺(今河北省张北县安固里淖)到白水泺(今内蒙古自治区乌兰察布市察右前旗黄旗海),再到女古底仓,一路仓皇逃跑至夹山(今内蒙古自治区武川县附近)。数日后,宰相李处温与南京(即燕京,今北京市西南)都统萧干、耶律大石等拥立秦晋国王耶律淳为帝,建立北辽。耶律大石被视为肱骨之臣,官至太师。

1120年,一心想收复燕云十六州的北宋与金朝缔结了海上之盟,约定南北夹击辽朝。1122年5月,宋徽宗得知金朝大举进攻的消息后,任命童贯为宣抚使,蔡攸为副使,率军15万巡边,伺机收复燕云十六州。耶律淳委派耶律大石为西南路都统,牛栏监军萧遏鲁为副将,率领奚、契丹骑兵2000,驻扎于涿州新城县(今河北省高碑店市)防备。

宋军裨将杨可世听闻燕地百姓早有归宋之心,如果宋军到达,燕人必定箪食壶浆迎接,便率轻骑数千奇袭燕京,但7月1日在兰沟甸遭到耶律大石军的掩杀,大败而归。耶律淳得知消息后,又增兵3万。耶律大石率军渡过白沟河,4日与宋军东路统制种师道隔河对峙。战前,杨可世派赵明持黄榜旗前往耶律大石的营帐劝降,耶律大石毁旗怒骂:“无多言,有死而已。”话语未完,辽军矢石如雨。耶律大石指挥骑兵从西部浅滩处渡河,分左右两翼包抄宋军,宋军大败,杨可世中铁蒺藜负伤。次日,驻扎于范村(今河北省涿州市西南)的宋军西路统制辛兴宗的部队也遭到四军大王萧干的围攻。

7月8日,种师道下令撤兵,耶律大石得知消息后,率轻骑追击至古城,双方交战,宋军大乱,种师道几乎不能脱逃。宋军一路逃奔至雄州,辽军一路跟随,童贯禁止宋军入城,契丹人斥责北宋背弃澶渊之盟,挑起战争。正逢此日北风大雨冰雹交加,宋军一败再败,阵亡者不计其数,种师道也因燕京之战的失利遭到童贯的弹劾,责令致仕。

7月11日,耶律大石在涿州召见北宋使者马扩,责问他辽朝与北宋通好百年,现今北宋为何率军前来抢夺辽朝的领土。马扩以“宋不取怕金来取”作答辩。耶律大石斥责马扩,说西夏屡次派使者唆使辽朝进攻北宋,但辽朝不肯见利忘义,将表章封存后交给北宋,如今北宋只听信了女真人的一句话,便于辽朝兵戈相见。耶律大石又质问马扩既为使者,为何与叛将刘宗吉有联系,并让他转告童贯,如果两国想和好仍可交好,如果不愿和好便可提兵来战,不要在天热时打仗使士兵受苦。

1122年7月29日,耶律淳病死,其妻萧德妃临朝称制。宰相李处温南通童贯,想纳土降宋,北联络金朝作为内应,事发后被处死。李处温死后,北辽的军政事务由太师耶律大石和四军大王萧干掌控。

北宋得知耶律淳去世的消息后,在太宰王黼的倡议下,再次兴兵攻打北辽。8月29日,宋徽宗下诏集结各道兵20万,以刘延庆为都统制,于10月在三关(草桥关、益津关、瓦桥关)汇合。10月25日,北辽都管押常胜军、涿州留守郭药师叛降北宋。11月19日,刘延庆、何灌、郭药师等率军从雄州出发,进入新城县;刘光世、杨可世从安肃州(今河北省徐水县安肃镇)出发,进入易州,两军于涿州汇合,共50万。耶律大石和萧干统帅的北辽军不足2万人,在泸沟河部署。宋辽两军隔河对峙,双方曾战于料石冈,但未分胜负。11月24日,郭药师率军6000奇袭燕京,入外城。契丹守军拼力死战,而宋军毫无军纪,饮酒后到处奸淫掳掠。萧德妃秘遣使者召耶律大石、萧干军,昼夜疾行,自南暗门入城,宋军大败,仅百余骑得以逃脱。29日,泸沟河北面四处火起,宋军以为辽军将至,烧营落荒而逃。逃兵自相践踏,坠落山涧者不计其数,丢弃的军需物资绵延数百里。

北辽刚刚击退南方的宋军,北方的金军又再次逼近。萧德妃曾五次上表金朝,请求立秦王耶律定为帝,称臣求和,金太祖不许。萧德妃只好派精兵防守居庸关,但金兵到来时,居庸关城墙倒塌,士兵多被压死,其余守军不战而溃。萧德妃闻讯后连夜逃离燕京,声称御敌,实为出奔。萧德妃、耶律大石、萧干等经古北口(今北京市密云县古北口镇),向东逃至松亭关(今河北省宽城满族自治县西南),但因去往何处,发生争执。萧干主张去奚王府立国,而耶律大石则主张投奔天祚帝。驸马都尉萧勃迭反对耶律大石的意见,被耶律大石下令斩首。耶律大石又传令军中,有异议者斩。于是北辽军兵分两路,萧干率领奚、渤海军前往奚王府,耶律大石挟持萧德妃去夹山投奔天祚帝。萧干到达奚王府后,自立为帝,国号大奚,半年后败亡。耶律大石与萧德妃率军7000,于1123年3月至夹山。天祚帝因耶律淳被立之事杀萧德妃及外甥耶律常哥。天祚帝又质问耶律大石为何擅立耶律淳,耶律大石指出天祚帝以辽朝全国国力不能抵御金朝的进攻,弃国而逃,致使生灵涂炭。耶律淳为辽太祖子孙,立其为帝保社稷远胜于投降金朝。在耶律大石的辩解下,天祚帝下令赦免其余众人。

耶律大石在辽天祚帝帐下任都统一职,1123年,率辽军进攻奉圣州,驻军于龙门山东二十五里处。金朝都统完颜斡鲁派完颜照立、完颜娄室、马和尚等率军攻打,耶律大石战败被完颜娄室俘虏,所部投降。完颜宗望用绳子绑着耶律大石,强迫他作为向导,率军袭击了天祚帝位于青冢泺(今内蒙古自治区呼和浩特市南)的大营,俘获了天祚帝之子秦王耶律定、许王耶律寧和嫔妃、公主、从臣多人,获取辎重车万余辆,只有梁王耶律雅里和天祚帝长女趁乱逃出。耶律大石因作为向导有功,免其罪并特受金太祖降诏奖谕。金太祖还十分欣赏耶律大石的仪表俊美,为人聪辩,特赐予其妻子。同年9月,耶律大石跟随金朝西征,带领家眷自金营逃出,率领一支部队投奔天祚帝。关于耶律大石在金营中的生活,《契丹国志》记载耶律大石投降金朝后与粘罕不和,粘罕想杀掉耶律大石,耶律大石带着五个儿子夜间逃脱,但把妻子留在金营中。粘罕将耶律大石的妻子赐给部落中地位最低贱的人,但他的妻子坚贞不屈,最后被粘罕射杀,但此段资料真实性待考。

1124年,在得到耶律大石的部队和阴山室韦首领毛割石的援助后,辽天祚帝认为反攻的时机已经来临,决定亲自出兵收复燕州、云州地区。耶律大石认为金军气盛,应当养精蓄锐,不能贸然出击,天祚帝不听,坚持出兵。耶律大石知道天祚帝无法完成复兴辽朝的大业,又害怕得到天祚帝的猜忌,于是杀掉萧乙薛和坡里括后自立为王,率领铁骑200出奔。耶律大石走后,辽天祚帝虽然取得一些战役的胜利,但不久便被金朝所败。1125年,辽天祚帝在投奔西夏的途中被俘,辽朝灭亡。

耶律大石率军从夹山出发,北行三日渡过黑水(爱毕哈河),途中遇到白鞑靼人首领床古儿,床古儿给予耶律大石四百匹马,二十头骆驼,若干只羊的援助。耶律大石一路向西北,于1124年到达可敦城,召集威武、崇德、会蕃、新、大林、紫河、驼等七个军州的长官和大黄室韦、敌剌、王纪剌、茶赤剌、也喜、鼻古德、尼剌、达剌乖、达密里、密儿纪、合主、乌古里、阻蔔、普速完、唐古、忽母思、奚的、纠而毕十八个部族的首领举行大会。在大会上,耶律大石慷慨激昂地指出先祖创建辽朝的艰难以及由于金朝对于辽朝侵略,造成天祚帝流亡在外、生灵涂炭,号召各军州和部族驱逐仇敌,复兴大辽。由于可敦城是辽朝的西北边防重镇,边防军队不得随意征调,军队在战乱中得以保存,并且此地还拥有可骑乘的战马数十万匹。耶律大石安置官吏,整顿兵马,磨砺武器,得到精兵万余人。

耶律大石在可敦城建立根据地后,积攒实力,不断派使者联络白鞑靼人、西夏以及北宋,从外交上孤立金朝。1125年夏,西夏联络耶律大石攻取金朝的山西诸郡。同年末,耶律大石派使者联络北宋,提议合力攻打金朝。1127年,白鞑靼人与耶律大石通好,拒绝将马匹卖给金朝。金太宗派使者问罪,双方关系紧张。1129年,耶律大石率军攻取了金朝的北方二营。次年,金太宗派耶律余睹、石家奴、拔离速征讨耶律大石,但由于诸部落不同意出兵,大军行进至兀纳水后收兵。

经过休整,耶律大石的军事实力得到壮大。1130年3月,耶律大石以青牛、白马祭告天地、列祖,准备西征。耶律大石先派使者送信给高昌回鹘首领毕勒哥,阐明两国先代的友好并要求借道去大食。毕勒哥得到书信后,迎接耶律大石至宫邸大宴三日,临行前毕勒哥亲自护送耶律大石出境,赠送耶律大石马匹六百、骆驼数百、羊三千只作为礼物,并约定交出人质,作为耶律大石的附庸国。

耶律大石率军离开高昌回鹘,进入吉尔吉斯境内,遭到了当地的抵抗,但双方未发生大规模的战争。耶律大石率军继续西进,到达叶密立。大军所到之处望风披靡,获取骆驼、牛、马、羊等辎重无数。1131年春,金朝统帅粘罕及耶律余睹率领云中、燕、云州汉军、金军1万人攻打耶律大石的根据地可敦城,但遭到失败。耶律大石到达叶密立后,虽然与高昌回鹘发生过摩擦,但基本得到了当地突厥部族的支持,户数达到4万。1132年,耶律大石在新建成的叶密立正式称“菊儿汗”,群臣又尊汉号为“天祐皇帝”,建元延庆,追尊祖父为元皇帝,祖母为宣义皇后,册封元妃萧氏为昭德皇后,西辽帝国正式建立。

西辽帝国建立后,耶律大石开始酝酿向周边地区扩张。1132年,耶律大石亲率大军向南进发,高昌回鹘再次臣服于西辽。随后耶律大石率军越过天山,沿塔里木盆地北向西推进,与东喀喇汗国发生冲突。西辽军被东喀喇汗国阿尔斯兰汗阿赫马德·伊本·哈桑的军队击败,大将阿勒·阿瓦尔被俘,损失惨重。耶律大石撤军后向七河地区进发,收编了当地的契丹人和突厥人,共16000帐,使西辽军队的人数增加了一倍。耶律大石率军驻扎于西辽与东喀喇汗国边境地区,等待时机准备反攻。

1132年,阿赫马德·伊本·哈桑去世,其子伊卜拉欣二世继任。伊卜拉欣二世软弱无能,原本臣属于东喀喇汗国的葛逻禄和康里人趁机袭击他的部属和牲畜,进行劫掠。伊卜拉欣二世不能控制住国内的局势,于是派使者请求耶律大石进入八剌沙衮(今吉尔吉斯斯坦托克馬克東)接管他的国家,使他“摆脱这尘世的烦恼”。耶律大石接到请求后,率军进入东喀喇汗国首都八剌沙衮,“登上那不费分文的宝座”。耶律大石将伊卜拉欣二世降为伊列克·突厥蛮(意为突厥王),保留了他对喀什噶尔(今新疆维吾尔自治区喀什市)、和田地区的控制,东喀喇汗国成为西辽的附庸。由于八剌沙衮附近是可耕可牧的肥沃地区,耶律大石决定建都于此,将八剌沙衮改名为虎思斡耳朵(意为强而有力的宫帐),并改元康国。耶律大石随后又派军队战胜了吉尔吉斯人,征服了别失八里(今新疆维吾尔自治区吉木萨尔县境内),康里人不久也臣服于西辽。

1134年4月,耶律大石任命六院司大王萧斡里剌为兵马都元帅,敌剌部前同知枢密院事萧查剌阿不为副元帅,茶赤剌部秃鲁耶律燕山为都部署,护卫耶律铁哥为都监,率军7万征讨金朝。在战前的誓师大会上,耶律大石用白马青牛祭天,指出先祖创业艰难,是由于后代君主耽于享乐致使社稷倾覆。中亚并非久居之地,应当荣归故里,复兴大辽。他又劝谕萧斡里剌要与士卒同甘共苦,赏罚分明。作战时要选择水草丰富处扎营,谨慎用兵。但由于西辽与金朝两国相隔遥远,西辽军队行进万里一无所获,兵马损失惨重,不得不撤军回国。另据《三朝北盟会编》记载,1135年,耶律大石再次率军攻打金朝,金熙宗派粘罕迎战。金军进入沙漠后与西辽军征战三昼夜不分胜败,但金军粮草断绝,人马冻死很多,加上本为契丹人的副将临阵倒戈,致使粘罕大败而归。但此段史料的真实性待考。

自1137年起,耶律大石开始了第二次扩张。1137年,耶律大石率军向察赤(今乌兹别克斯坦塔什干)、费尔干纳盆地及泽拉夫尚河流域进兵。同年5至6月,在忽毡(今塔吉克斯坦苦盏)遭到了西喀喇汗国可汗马赫穆德·伊本·穆海默德的抵抗。西喀喇汗国战败,马赫穆德败逃回撒马尔罕。这次战败使马黑木二世的臣民感到震惊、惊恐和沮丧,但耶律大石并没有继续进兵。1141年,西喀喇汗国与葛逻禄人爆发冲突,马赫穆德向宗主国塞尔柱帝国求援。塞尔柱苏丹桑贾尔动员伊斯兰诸国参战,集中了呼罗珊、锡斯坦、伽色尼、马赞德兰、古尔等国的军队近10万人,单单阅兵就耗费了半年时间。同年7月,桑贾尔率军渡过阿姆河,进入河中地区,葛逻禄人急忙派使者向耶律大石求救。

耶律大石写信给桑贾尔替葛逻禄人说情,但桑贾尔十分傲慢的回信命令耶律大石加入伊斯兰教,并称自己的军队能用箭截断敌人的须发。当耶律大石听完桑贾尔的使者读完书信后,下令拔下他的一撮胡须,然后给他一根针让他当场示范,使者不能做到。耶律大石说既然针不能截断胡须,那那个人又怎么能用箭折断须发呢?于是下令进兵,双方在撒马尔罕以北的卡特万草原对峙,西辽的军队中有契丹人、突厥人、汉人和蒙古人。耶律大石观察了战场的地形后,让军队背靠达尔加姆峡谷安营。两军于1141年9月9日展开会战,战前耶律大石指出桑贾尔的联军人多少谋,如果全力进攻,他们就会首尾不顾。耶律大石派六院司大王萧斡里剌、招讨副使耶律松山等率兵2500攻打联军右翼,枢密副使萧剌阿不、招讨使耶律术薛等率兵2500攻打其左翼,耶律大石亲率部队攻打中军;桑贾尔的联军右翼是埃米尔库马吉,左翼是锡斯坦埃米尔胡马希,他自己亲率中军,有战斗经验的老兵负责殿后。

在战场上,锡斯坦贵族作战英勇,但西辽军队中的葛逻禄人发挥了重要的作用,迫使桑贾尔的联军败逃。桑贾尔和马赫穆德逃奔至泰尔梅兹,桑贾尔的妻子、左、右翼统帅和伊斯兰法学家胡萨姆·奥玛尔·伊本·阿布杜·阿齐兹·伊本·马扎·布哈里均被俘虏。桑贾尔的联军损失惨重,仅达尔加姆峡谷就装下1万名死者。《辽史》记载塞尔柱帝国联军的阵亡者横尸数十里。卡特万之战后,塞尔柱帝国的势力退出河中地区,西辽成为中亚霸主。耶律大石随后率军进入撒马尔罕,立马赫穆德之弟伊卜拉欣·伊本·穆海默德为桃花石汗,继续让其统治西喀喇汗国。 他还下令处死布哈拉的伊斯兰教教长胡沙穆丁·倭玛尔,任命阿尔普·的斤统治该地。随后派大将额儿布思(一说即萧斡里剌)出兵花剌子模,在该地烧杀抢掠,迫使花剌子模沙阿阿拉丁·阿比兹向西辽臣服并且每年缴纳价值3万金第纳尔的货物和牲畜。耶律大石在撒马尔罕驻扎90天后,至起儿漫(今乌兹别克斯坦卡尼梅赫镇)巡行后班师返回虎思斡耳朵。

1143年,耶律大石去世,在位20年,庙号德宗。因耶律大石之子耶律夷列年幼,遗诏命皇后萧塔不烟临朝称制,改元咸清,称感天皇后。

耶律大石的西征事迹被传到欧洲,正逢第二次十字军东征,于是在欧洲流传着东方世界有一位神秘的祭司王约翰,是基督教的捍卫者。俄语、阿拉伯语、拉丁语和古英语中中国的发音类似于“契丹”,都是受耶律大石西征的影响。而耶律大石的名字也成了西辽帝国的代称,在耶律大石死后,金、西夏、南宋等国家对西辽的后代君主皆称为“大石”。

耶律大石凭借卓越的军事、政治、外交才能,在伊斯兰世界建立了幅员辽阔的西辽帝国,将辽朝的国祚延续了近百年,他为东西方文化、经济方面的交流作出了积极的贡献。东西方史学家对于耶律大石的成就多有赞誉:穆斯林史学家朱兹贾尼评价耶律大石:是一位公正的君主,因为公正和才干而受到崇敬;耶律楚材评价耶律大石:颇尚文教,西域人至今思之。拉施特称赞耶律大石:是一个有智慧而又有才干的人。他有条不紊地从这些地区上把队伍召集到身边,占领了整个突厥斯坦地区,(从而)获得了古儿汗,即伟大的君主的称号。清代史学家谭宗浚评价耶律大石:德宗起自词臣,兼通藩俗,削平各部,殄定诸藩,意其典章制度必可多采。

\subsubsection{延庆}


\begin{longtable}{|>{\centering\scriptsize}m{2em}|>{\centering\scriptsize}m{1.3em}|>{\centering}m{8.8em}|}
  % \caption{秦王政}\
  \toprule
  \SimHei \normalsize 年数 & \SimHei \scriptsize 公元 & \SimHei 大事件 \tabularnewline
  % \midrule
  \endfirsthead
  \toprule
  \SimHei \normalsize 年数 & \SimHei \scriptsize 公元 & \SimHei 大事件 \tabularnewline
  \midrule
  \endhead
  \midrule
  元年 & 1124 & \tabularnewline\hline
  二年 & 1125 & \tabularnewline\hline
  三年 & 1126 & \tabularnewline\hline
  四年 & 1127 & \tabularnewline\hline
  五年 & 1128 & \tabularnewline\hline
  六年 & 1129 & \tabularnewline\hline
  七年 & 1130 & \tabularnewline\hline
  八年 & 1131 & \tabularnewline\hline
  九年 & 1132 & \tabularnewline\hline
  十年 & 1133 & \tabularnewline
  \bottomrule
\end{longtable}

\subsubsection{康国}

\begin{longtable}{|>{\centering\scriptsize}m{2em}|>{\centering\scriptsize}m{1.3em}|>{\centering}m{8.8em}|}
  % \caption{秦王政}\
  \toprule
  \SimHei \normalsize 年数 & \SimHei \scriptsize 公元 & \SimHei 大事件 \tabularnewline
  % \midrule
  \endfirsthead
  \toprule
  \SimHei \normalsize 年数 & \SimHei \scriptsize 公元 & \SimHei 大事件 \tabularnewline
  \midrule
  \endhead
  \midrule
  元年 & 1134 & \tabularnewline\hline
  二年 & 1135 & \tabularnewline\hline
  三年 & 1136 & \tabularnewline\hline
  四年 & 1137 & \tabularnewline\hline
  五年 & 1138 & \tabularnewline\hline
  六年 & 1139 & \tabularnewline\hline
  七年 & 1140 & \tabularnewline\hline
  八年 & 1141 & \tabularnewline\hline
  九年 & 1142 & \tabularnewline\hline
  十年 & 1143 & \tabularnewline
  \bottomrule
\end{longtable}


\subsection{萧塔不烟\tiny(1143-1150)}

\subsubsection{生平}

萧塔不烟,生卒年不详,西辽开国君主遼德宗的皇后,德宗死後稱制,執政7年。

1143年,耶律大石去世后,其子耶律夷列年幼,遗诏命皇后萧塔不烟临朝称制,改元咸清,称感天皇后。

1144年,金熙宗得知耶律大石去世的消息後,派使者粘割韩奴前往劝降西辽。粘割韩奴進入西遼國境後,正好遇上外出打獵的萧塔不烟。見到萧塔不烟後,粘割韩奴不但沒有下马跪拜,反而讓她下马接诏。萧塔不烟於是命人将粘割韩奴拉下马,讓他跪下。粘割韩奴痛骂不止,斥責其為反賊,侮辱上国使臣。萧塔不烟发怒,派人将其杀死。

執政七年後,萧塔不烟退位。耶律夷列親政,改年號為紹興。

一说萧塔不烟与耶律大石在叶密立(今新疆维吾尔自治区额敏县)称菊儿汗时册封的昭德皇后萧氏为同一人;也有观点认为二者并非同一人。

\subsubsection{咸清}

\begin{longtable}{|>{\centering\scriptsize}m{2em}|>{\centering\scriptsize}m{1.3em}|>{\centering}m{8.8em}|}
  % \caption{秦王政}\
  \toprule
  \SimHei \normalsize 年数 & \SimHei \scriptsize 公元 & \SimHei 大事件 \tabularnewline
  % \midrule
  \endfirsthead
  \toprule
  \SimHei \normalsize 年数 & \SimHei \scriptsize 公元 & \SimHei 大事件 \tabularnewline
  \midrule
  \endhead
  \midrule
  元年 & 1144 & \tabularnewline\hline
  二年 & 1145 & \tabularnewline\hline
  三年 & 1146 & \tabularnewline\hline
  四年 & 1147 & \tabularnewline\hline
  五年 & 1148 & \tabularnewline\hline
  六年 & 1149 & \tabularnewline\hline
  七年 & 1150 & \tabularnewline
  \bottomrule
\end{longtable}

\subsection{仁宗耶律夷列\tiny(1150-1163)}

\subsubsection{生平}

辽仁宗耶律夷列(?-1163年),耶律大石和萧塔不烟之子,耶律普速完之兄,西遼第二任君主,在位13年。

耶律大石去世后,其子耶律夷列年幼,遗诏命皇后萧塔不烟临朝称制,改元咸清,称感天皇后。萧塔不烟在位7年后,还政于子耶律夷列。

1150年,耶律夷列即位,改元绍兴。耶律夷列在位期间普查首都虎思斡耳朵内畿18岁以上成年男子的人口,共84500户[a]。1156年,西喀喇汗国大汗伊卜拉欣三世与葛逻禄军队长官艾亚尔伯克发生冲突,双方在饥饿草原发生战争,伊卜拉欣三世战败被暴尸荒野,其子阿里·本·哈桑继任,称恰格雷汗。恰格雷汗随后对葛逻禄人展开报复,杀死其首领比古汗。葛逻禄的拉钦伯克和比古汗之子向花剌子模求助,而恰格雷汗则向西辽求援。耶律夷列派东喀喇汗国土库曼王伊卜拉欣·本·阿赫马德率军1万前去救援,双方隔粟特河对峙。经撒马尔罕的宗教人士调节,双方签订合约,恰格雷汗恢复了葛逻禄首领的军事职务,双方撤军(後來西遼把葛邏祿人安置在阿力麻里單獨管理)。

耶律夷列在位13年,于1163年去世,庙号仁宗。由于其子年幼,遗诏命其妹耶律普速完临朝称制,改元崇福,称承天太后。

\subsubsection{绍兴}


\begin{longtable}{|>{\centering\scriptsize}m{2em}|>{\centering\scriptsize}m{1.3em}|>{\centering}m{8.8em}|}
  % \caption{秦王政}\
  \toprule
  \SimHei \normalsize 年数 & \SimHei \scriptsize 公元 & \SimHei 大事件 \tabularnewline
  % \midrule
  \endfirsthead
  \toprule
  \SimHei \normalsize 年数 & \SimHei \scriptsize 公元 & \SimHei 大事件 \tabularnewline
  \midrule
  \endhead
  \midrule
  元年 & 1151 & \tabularnewline\hline
  二年 & 1152 & \tabularnewline\hline
  三年 & 1153 & \tabularnewline\hline
  四年 & 1154 & \tabularnewline\hline
  五年 & 1155 & \tabularnewline\hline
  六年 & 1156 & \tabularnewline\hline
  七年 & 1157 & \tabularnewline\hline
  八年 & 1158 & \tabularnewline\hline
  九年 & 1159 & \tabularnewline\hline
  十年 & 1160 & \tabularnewline\hline
  十一年 & 1161 & \tabularnewline\hline
  十二年 & 1162 & \tabularnewline\hline
  十三年 & 1163 & \tabularnewline
  \bottomrule
\end{longtable}

\subsection{承天太后耶律普速完\tiny(1163-1177)}

\subsubsection{生平}

耶律普速完(?-1177年)是遼仁宗耶律夷列的妹妹,為西遼第四任統治者。仁宗在1163年死後,其子尚年幼,遺詔由其妹耶律普速完權理國事,臨朝稱制,並改元崇福,號承天太后。

她的丈夫是蕭朵魯不。她與丈夫之弟樸古只沙里私通,把丈夫改為東平王,後來又殺了他。崇福十四年(1177年),蕭朵魯不之父斡里剌以兵圍其宮,射殺普速完及樸古只沙里。仁宗子耶律直魯古即位,改元天禧,是為西遼末主。

中國歷朝的臨朝稱制,皆為皇太后、皇后在君主因故無法理朝時的一種權宜之計(也有例外,如武則天曾與唐高宗並稱二聖並臨朝聽政),但耶律普速完是中國歷史上唯一一個以先朝公主與當朝君主姑母的身分臨朝稱制者,可為前無古人、後無來者。耶律普速完稱制時,又更改年號,在實質意義上已經得到等同君王的待遇和地位。

\subsubsection{崇福}

\begin{longtable}{|>{\centering\scriptsize}m{2em}|>{\centering\scriptsize}m{1.3em}|>{\centering}m{8.8em}|}
  % \caption{秦王政}\
  \toprule
  \SimHei \normalsize 年数 & \SimHei \scriptsize 公元 & \SimHei 大事件 \tabularnewline
  % \midrule
  \endfirsthead
  \toprule
  \SimHei \normalsize 年数 & \SimHei \scriptsize 公元 & \SimHei 大事件 \tabularnewline
  \midrule
  \endhead
  \midrule
  元年 & 1164 & \tabularnewline\hline
  二年 & 1165 & \tabularnewline\hline
  三年 & 1166 & \tabularnewline\hline
  四年 & 1167 & \tabularnewline\hline
  五年 & 1168 & \tabularnewline\hline
  六年 & 1169 & \tabularnewline\hline
  七年 & 1170 & \tabularnewline\hline
  八年 & 1171 & \tabularnewline\hline
  九年 & 1172 & \tabularnewline\hline
  十年 & 1173 & \tabularnewline\hline
  十一年 & 1174 & \tabularnewline\hline
  十二年 & 1175 & \tabularnewline\hline
  十三年 & 1176 & \tabularnewline\hline
  十四年 & 1177 & \tabularnewline
  \bottomrule
\end{longtable}

\subsection{天禧帝耶律直鲁古\tiny(1177-1211)}

\subsubsection{生平}

耶律直魯古(12世纪-1213年),是西遼皇帝耶律夷列的次子。姑姑耶律普速完在崇福十四年(1177年)被殺,耶律直魯古即位,改元天禧,史称天禧帝。

乃蠻王子屈出律于1208年流亡至西辽,天禧帝耶律直鲁古不仅信任他还将女儿嫁给他。天禧三十四年(1211年),屈出律以伏兵八千擒直魯古,強迫天禧帝直鲁古讓位,尊他為太上皇,皇后為皇太后。1213年,天禧帝直魯古去世。1218年,蒙古攻西遼,殺屈出律,西遼亡。

遼史(卷三十 本紀第三十):“仁宗次子直魯古即位,改元天禧,在位三十四年,天禧帝。時秋出獵,乃蠻王屈出律以伏兵八千擒之,而據其位。遂襲遼衣冠,尊直魯古為太上皇,皇后為皇太后,朝夕問起居,以侍終焉。直魯古死,遼絕。”

\subsubsection{天禧}

\begin{longtable}{|>{\centering\scriptsize}m{2em}|>{\centering\scriptsize}m{1.3em}|>{\centering}m{8.8em}|}
  % \caption{秦王政}\
  \toprule
  \SimHei \normalsize 年数 & \SimHei \scriptsize 公元 & \SimHei 大事件 \tabularnewline
  % \midrule
  \endfirsthead
  \toprule
  \SimHei \normalsize 年数 & \SimHei \scriptsize 公元 & \SimHei 大事件 \tabularnewline
  \midrule
  \endhead
  \midrule
  元年 & 1178 & \tabularnewline\hline
  二年 & 1179 & \tabularnewline\hline
  三年 & 1180 & \tabularnewline\hline
  四年 & 1181 & \tabularnewline\hline
  五年 & 1182 & \tabularnewline\hline
  六年 & 1183 & \tabularnewline\hline
  七年 & 1184 & \tabularnewline\hline
  八年 & 1185 & \tabularnewline\hline
  九年 & 1186 & \tabularnewline\hline
  十年 & 1187 & \tabularnewline\hline
  十一年 & 1188 & \tabularnewline\hline
  十二年 & 1189 & \tabularnewline\hline
  十三年 & 1190 & \tabularnewline\hline
  十四年 & 1191 & \tabularnewline\hline
  十五年 & 1192 & \tabularnewline\hline
  十六年 & 1193 & \tabularnewline\hline
  十七年 & 1194 & \tabularnewline\hline
  十八年 & 1195 & \tabularnewline\hline
  十九年 & 1196 & \tabularnewline\hline
  二十年 & 1197 & \tabularnewline\hline
  二一年 & 1198 & \tabularnewline\hline
  二二年 & 1199 & \tabularnewline\hline
  二三年 & 1200 & \tabularnewline\hline
  二四年 & 1201 & \tabularnewline\hline
  二五年 & 1202 & \tabularnewline\hline
  二六年 & 1203 & \tabularnewline\hline
  二七年 & 1204 & \tabularnewline\hline
  二八年 & 1205 & \tabularnewline\hline
  二九年 & 1206 & \tabularnewline\hline
  三十年 & 1207 & \tabularnewline\hline
  三一年 & 1208 & \tabularnewline\hline
  三二年 & 1209 & \tabularnewline\hline
  三三年 & 1210 & \tabularnewline\hline
  三四年 & 1211 & \tabularnewline
  \bottomrule
\end{longtable}

\subsection{末帝屈出律\tiny{1211-1218}}

\subsubsection{生平}

屈出律是乃蠻太陽汗之子,1204年,成吉思汗攻滅乃蠻部,太陽汗戰死。屈出律投奔其叔父不亦鲁黑汗。不亦鲁黑汗死後,屈出律又聯合蔑儿乞首领脱黑脱阿對抗成吉思汗。1208年,屈出律和脱黑脱阿在也儿的石河(今额尔齐斯河)上游被成吉思汗击败,脱黑脱阿戰死。屈出律逃奔至别失八里(今新疆维吾尔自治区吉木萨尔县境内),又抵达苦叉(今新疆维吾尔自治区库车县),他的部队缺乏給養又没有粮食,一路上纷纷散去。屈出律隨後投奔西遼,侍奉於菊儿汗耶律直鲁古。《史集》记载屈出律觐见耶律直鲁古时化妆成一名马夫,这身装扮触怒了耶律直鲁古的大臣,但得到了耶律直鲁古正妻菊儿别速的女儿渾忽公主的賞識,三天後渾忽公主便嫁给了屈出律。

随着西遼国力的衰落,附庸国高昌回鹘、花剌子模和西喀喇汗国纷纷背叛西遼,屈出律便向耶律直鲁古建议自己返回叶密立(今新疆维吾尔自治区额敏县)、海押立(今哈萨克斯坦塔尔迪库尔干)、别失八里地区召集乃蠻舊部,帮助耶律直鲁古鎮壓叛乱。耶律直鲁古封屈出律为可汗,并赠送他很多禮物。屈出律收集失散的乃蠻人,组成军隊,劫掠七河地区。同时派使者聯絡花剌子模沙阿阿拉乌丁·摩诃末,雙方约定誰先奪取西遼就占有它的土地。屈出律先擊败了西遼的军隊,劫掠了耶律直鲁古位于乌兹根的府库,隨後又進攻西遼首都虎思斡耳朵(今吉尔吉斯斯坦托克玛克东南布拉纳),但在真兀赤被耶律直鲁古擊败。屈出律返回葉密立,圖謀再次進攻。

1210年,西遼國内發生西遼軍隊燒殺劫掠首都虎思斡耳朵的事件,由于宰相马合木·太处理不當,致使軍隊纷纷離散。屈出律得知此消息后,于次年秋率軍8千“像雲中的閃電一樣”襲擊了正在外出打獵的耶律直鲁古,奪取了皇位。屈出律惺惺作態,尊耶律直鲁古为太上皇,皇后为皇太后,早晚问候他们的衣食起居。1213年,耶律直鲁古在憤恨中死去。

东喀喇汗国土库曼王穆罕默德三世起兵反抗西辽的统治,遭到耶律直鲁古的镇压,穆罕默德三世被俘。屈出律篡位后,释放了穆罕默德三世,将其送回喀什噶尔,但他不受当地贵族的欢迎,入城时被刺死于城门洞中。由于喀什噶尔不肯归附屈出律,屈出律每逢秋收时节派兵烧毁他们的庄稼。三、四年后,当地百姓因为饥荒不得已而归顺。在占领喀什噶尔后,屈出律下令在每家每户派驻一名士兵,这些士兵毫无军纪,到处烧杀抢掠。屈出律随后又派兵征服了和田。阿力麻里(今新疆维吾尔自治区霍城县一带)汗不札兒不肯服从屈出律,屈出律多次派军队征讨无果,最终趁不札兒出獵時將其擒殺。

屈出律篡位后虽然没有更改西辽的国号和政治制度,但他一改前任西辽君主的宗教自由政策,转而实行宗教迫害。屈出律原本信奉景教,后在浑忽公主的劝说下改信佛教。他用强制手段强迫西辽当地的穆斯林和基督徒改信佛教,穿戴契丹人的服装,这引起了当地人民的强烈不满。屈出律征服和田后,下令召集当地的伊斯兰教阿訇讨论教义,教长阿訇阿剌丁·摩诃末极力维护伊斯兰教,屈出律命人将其严刑拷打,强迫他改教,摩诃末不从,被屈出律下令钉死于清真寺的大门上。

1218年,成吉思汗派哲别、曷思麦里率2萬蒙古軍攻打屈出律,屈出律闻讯带领随从从喀什噶尔逃跑。哲别进入喀什噶尔后宣布宗教自由,城中居民開始對屈出律展开报复,大肆屠杀屈出律的军队。屈出律逃至巴达克山(今阿富汗巴达赫尚省),在瓦罕河谷东部的达拉兹峡谷迷路。由于当地山路崎岖难行,哲别向当地猎户许诺以屈出律随身携带的财物为条件,抓捕屈出律。屈出律被俘获后交予哲别,哲别将屈出律斩首后,命曷思麦里拿着他的首级传示于喀什噶尔、押儿牵(今新疆维吾尔自治区莎车县)、斡端(今新疆维吾尔自治区和田市)等城,城中将领皆率部投降,西辽彻底被蒙古帝国所征服。

\subsubsection{天禧}

\begin{longtable}{|>{\centering\scriptsize}m{2em}|>{\centering\scriptsize}m{1.3em}|>{\centering}m{8.8em}|}
  % \caption{秦王政}\
  \toprule
  \SimHei \normalsize 年数 & \SimHei \scriptsize 公元 & \SimHei 大事件 \tabularnewline
  % \midrule
  \endfirsthead
  \toprule
  \SimHei \normalsize 年数 & \SimHei \scriptsize 公元 & \SimHei 大事件 \tabularnewline
  \midrule
  \endhead
  \midrule
  三四年 & 1211 & \tabularnewline\hline
  三五年 & 1212 & \tabularnewline\hline
  三六年 & 1213 & \tabularnewline\hline
  三七年 & 1214 & \tabularnewline\hline
  三八年 & 1215 & \tabularnewline\hline
  三九年 & 1216 & \tabularnewline\hline
  四十年 & 1217 & \tabularnewline\hline
  四一年 & 1218 & \tabularnewline
  \bottomrule
\end{longtable}

%%% Local Variables:
%%% mode: latex
%%% TeX-engine: xetex
%%% TeX-master: "../Main"
%%% End:


%%% Local Variables:
%%% mode: latex
%%% TeX-engine: xetex
%%% TeX-master: "../Main"
%%% End:
 % 辽
% %% -*- coding: utf-8 -*-
%% Time-stamp: <Chen Wang: 2019-12-26 11:12:29>

\chapter{西夏\tiny(1038-1227)}

\subsection{简介}

西夏(1038年-1227年),国号大夏、邦泥定国或白高大夏國等,是中國歷史上由党項族建立的一個朝代。主要以党項族為主體,包括漢族、回鶻族與吐蕃族等民族在內的國家。因位於中原地區的西北方,國土佔據黃河中上游,史稱西夏。

党項族原居四川松潘高原,唐朝時遷居陕北。因平亂有功被唐帝封為夏州節度使,先後臣服於唐朝、五代諸朝與宋朝。夏州政權被北宋併吞後,由於李繼遷不願投降而再次立國,並且取得遼帝的冊封。李繼遷採取連遼抵宋的方式,陸續占領蘭州與河西走廊地區。1038年11月10日李元昊稱帝建國,即夏景宗,西夏正式立國。西夏在宋夏戰爭與遼夏戰爭中,戰況大致上取得優勢,形成三國鼎立的局面。夏景宗去世後,大權掌握在皇帝的太后與母黨手中,史稱母黨專政時期。西夏因為皇黨與母黨的對峙而內亂,北宋趁機多次伐夏。西夏抵禦成功並擊潰宋軍,但是橫山的喪失讓防線出現破洞。金朝崛起並滅遼、北宋,西夏改臣服金朝,獲得不少土地,兩國建立金夏同盟,大致上維持著和平關係。夏仁宗期間發生天災與任得敬分國事件,但經過改革後,到天盛年間出現盛世。然而漠北的蒙古帝國崛起,六次入侵西夏後拆散金夏同盟,讓西夏與金朝自相殘殺。西夏內部也多次發生弒君、內亂之事,經濟也因戰爭而趨於崩潰。最後於1227年8月28日亡於蒙古。

西夏屬於番漢聯合政治,以党項族為主導,漢族與其他族群為輔。制度由番漢兩元政治逐漸變成一元化的漢法制度。西夏的皇權備受貴族、母黨與權臣等勢力的挑戰而動盪不安。由於處於列強環視的河西走廊與河套地區,對外採取依附強者,攻擊弱者、以戰求和的外交策略。軍事手段十分靈活,配合沙漠地形,採取有利則進,不利則退,誘敵設伏、斷敵糧道的戰術;並且有铁鹞子、步跋子與潑喜等特殊兵種輔助。經濟方面以畜牧業與商業為主力,對外貿易易受中原王朝的影響,壟斷河西走廊與北宋的歲幣為西夏經濟帶來很大的幫助。

西夏是一個佛教王國,興建大量的佛塔與佛寺,以承天寺塔最有名。然而也是崇尚儒學漢法的帝國,立國前積極漢化;雖然夏景宗為了維護本身文化而提倡党項、吐蕃與回鶻文化,並且創立西夏文、立番官、建番俗等措施;但自夏毅宗到夏仁宗後,西夏已經由番漢同行轉為普遍漢化。文學方面以詩歌和諺語為主。在藝術方面於敦煌莫高窟、安西榆林窟有豐富的佛教壁畫,具有「綠壁畫」的特色。此外在雕塑、音樂與舞蹈等方面都有獨特之處。

西夏由党項族所建立。党項族是羌族的一支,帶有鮮卑的血統。唐朝時居住在四川松潘高原一帶,是唐朝的羈糜州之一。當時分有八部,以拓跋氏最為強盛。吐蕃於唐朝安史之亂後占領河西一帶,並且壓迫党項。唐高宗時期,党項首領拓跋赤辞在唐朝幫助下遷移到陝北一帶,奠定党項興起的根據地。881年占據宥州的平夏部拓跋思恭因平黄巢之亂有功,被唐僖宗封为夏州節度使,賜號定難軍。協助收復長安後又封夏国公,賜姓李,領有夏銀等地,夏州政權(正式稱呼是夏州節度使或定難軍)形成一個割據陝北的籓鎮。五代十國時,夏州政權避免介入中原各勢力間的內鬥,向五代與北漢稱臣以鞏固在陝北的勢力。然而後唐時,唐明宗意圖將延州節度使安從進與夏州節度使李彝超對調以併吞夏州政權。李彝超極力反對,成功擊退安從進率領的後唐軍,在北宋初年時累積雄厚的實力。

960年宋太祖赵匡胤建立宋朝後,夏州政權首領李彝殷向北宋稱臣,並且多次協助北宋對抗北漢。當北宋陸續平定南方各國後,宋太宗開始將目光放在北方,有意削除夏州政權。此時李氏家族反對李繼捧擔任夏州節度使。982年宋太宗招李繼捧與其族人遷居開封,命親宋的李克文繼任之,夏州政權被北宋併吞。李繼捧族弟李繼遷不願投降宋朝,率族人逃往地斤澤(今陝西橫山縣東北)抗宋。984年宋將尹憲、曹光實擊破夏軍。隔年李繼遷由弱轉強,攻破宋軍後陸續收復銀、夏等等夏州領地。990年被遼朝遼聖宗冊封為夏國王,即被追尊的夏太祖。宋廷採取以夷制夷方式,派李繼捧回任夏州,招撫李繼遷任銀州,並對此二人賜姓趙。不久李繼遷又叛,於996年擊退宋將李繼隆率領的五路大軍。在鞏固夏州領地後,一直努力西擴河西,最後於1002年第三次攻打靈州(今寧夏靈武西南)時成功攻下,改名西平府。宋朝至此無力圍堵,於隔年承認李繼遷領有夏州領地。李繼遷陸續占領涼州(今甘肅武威縣)等河西重鎮,擊退與宋朝聯手的河西涼州吐蕃六谷部。兩年後李繼遷被六谷部首領潘羅支襲擊而亡,其子李德明繼位,即被追尊的夏太宗。

李德明繼位後,因為國土快速膨脹,為了穩固國力以抵禦四方敵國,有意與宋和談;而北宋對外戰事也由擴張轉為和平,與遼朝簽署澶淵之盟後也希望西北也能穩定下來。最後雙方於1006年雙方簽署景德和議。李德明為了維護自身獨立,東和宋朝,北附遼朝,並讓太子李元昊迎娶遼朝的興平公主。對內方面,李德明定都興州(今宁夏银川东南),採取保境安民、注重生產的策略。並且請求北宋於保安軍(今陝西丹縣)設置榷场,聽許兩國貿易。此外積極西征河西,1028年派太子李元昊攻下甘州(今甘肅張掖縣),甘州回鶻首領夜落隔通順自殺,降服吐蕃六谷部首領折逋游龙钵。而後又奪肅州,降服瓜州歸義軍的曹賢順。至此夏州政權國力大盛,為日後李元昊稱帝立國建立穩固的基礎。1032年李德明去世,其子李元昊继位。

李元昊繼位後完成河西走廊的佔領,並且積極準備脫離宋朝獨立建國。他首先棄李姓,自稱嵬名氏,以北魏王室後裔自居。採楊守素的建議,以避父諱為由,改宋朝年號明道为显道,以建立自己的年號。隨後建宮殿,下禿髮令,恢復故俗,都興慶府,設文武二班,立軍名,用兵制,創造西夏文,改定禮樂等。

1038年11月10日(大庆三年十月十一日)李元昊稱帝,即夏景宗,改年号为天授礼法延祚,定都興州並改稱為興慶府,国号“大夏”,亦称西夏,至此西夏正式立國。夏景宗霸氣縱橫,脫離對北宋、遼朝的臣屬關係。為了要獨霸西方,他四處擴土,先後和宋遼開戰,為西夏武力顛峰的時候。

夏景宗於隔年採取聯遼抗宋的戰略不斷入侵宋邊境,並且要求宋朝承認西夏獨立。當時北宋在橫山山脈一帶建立不少堡壘,不過東方重鎮延州防禦薄弱,守將范雍無能。1040年夏景宗發動三川口之戰,率10萬大軍包圍延州,於三川口襲擊宋將劉平、石元孫的援軍,最後夏軍因大雪而解圍撤退。宋廷面對西夏大舉入侵,派夏竦為正使、韓琦與范仲淹為副使經略西夏。當時宋軍兵多於西夏,但是不擅野戰、補給也不易,主攻的韓琦與主守的范仲淹對此爭執不斷。1041年夏景宗發動大軍包圍宋朝西線的渭川、懷遠一帶,韓琦不聽派范仲淹建議派大將任福率大軍救援懷遠,夏景宗引誘至埋伏地好水川口襲擊,此即好水川之戰。此後宋廷轉為防禦,改派陳執中、夏竦經略,並且建立四路防線。1042年西夏謀臣張元建議避開宋防線,繞道奇襲京兆府(長安)。同年夏景宗於防線薄弱的涇原路發動定川寨之戰,於定川寨包圍殲滅宋軍,目標是長安,但另一路遇到原州景泰的阻擊而罷。宋夏戰爭一直到1044年才平息。雙方簽訂慶曆和議,宋朝承認西夏的割據地位,給予若干財物茶葉,封夏景帝為夏國主。西夏對宋稱臣,但對內依舊稱帝,實際上仍然是獨立的國家。

西夏擊敗宋朝後,自稱西朝,稱遼朝為北朝。遼朝遼興宗不滿西夏壯大,意圖再度壓服之。1043年在慶曆增幣后,遼興宗為報答宋朝,以國內西南部的党項叛附西夏為由,於隔年冬率大軍伐夏。西夏求和不成,採取堅壁清野方式擊潰遼軍。戰後四年夏景宗去世之際,遼軍又於1049年來犯。夏軍極力抵抗,最後雙方和談而止。

夏景宗建國西夏並推行中央集權,雖然鞏固了帝權,但同時與貴族的矛盾進一步加深。他獨裁專制、日益驕淫並且貪好女色。後宮之亂引來貴族衛慕氏的叛變(1034年)。又中种世衡的反間計而錯殺野利旺榮與野利遇乞,並且迷戀迎娶野利遇乞的妻子沒藏氏,生李諒祚。太子李寧林格與夏景宗因廢母(野利皇后)奪妻(沒移皇后)之仇,被沒藏氏的弟弟沒藏訛龐教唆刺殺夏景宗。夏景宗死後沒藏訛龐殺太子,立兩歲的李諒祚繼位,即夏毅宗。

夏毅宗與夏惠宗時期時,夏廷對內進一步鞏固統治,對外常與宋遼兩國處於戰爭與議和的狀態。夏毅宗繼位時年幼,由其母没藏太后與沒藏訛龐专政。當時辽朝遼興宗再度攻打西夏,最後西夏向辽朝称臣。没藏太后荒淫好色,多次勾結外人,其中李守貴與吃多己多次爭寵。最後李守貴殺太后與吃多己,事後也被沒藏訛龐所殺。沒藏訛龐又將其女許配夏毅宗以控制小皇帝。1059年夏毅宗參與政事,沒藏訛龐密謀刺殺夏毅宗,後被夏帝誅殺全家。親政後,夏毅宗娶協助他剷除沒藏訛龐的梁氏,任用梁乙埋與景詢等人。對內整治軍隊使地方軍政分立,文武官員互相牽制,提倡漢文化與技術,廢行蕃禮,改用漢儀,並於1063年改姓為李。對外方面,與宋重新劃定邊界,恢復榷場,貿易正常化。對吐蕃多次戰事,占領河湟與青海一带,於1063年招撫西域城(今甘肅定西縣)吐蕃首領禹藏花麻。夏毅宗的改革對以後各朝產生了深遠影響,然而於1066年與北宋作戰時受箭傷,兩年後去世,由其子7歲的李秉常即位,即夏惠宗。


1081年宋神宗命李憲率領五路宋軍伐夏,西夏梁太后採取誘敵深入、斷其糧道,最後只讓北宋奪得蘭州。

1082年年宋軍採取碉堡戰術,派徐禧興建永樂城,以步步逼近興慶府。西夏梁太后緊急率領30萬大軍突襲攻陷此城,北宋至此暫停伐夏,史稱永樂城之戰

由於夏惠宗年幼,由其母梁太后掌握大權,形成了以梁太后與梁乙埋為首的母黨專權。母黨大力發展其勢力,提倡番禮,重用都羅尾與罔萌訛,排擠夏景宗的弟弟嵬名浪遇等反對派。1080年,夏惠宗最後在皇族嵬名氏的協助下得以親政。夏惠宗崇尚漢法,下令以漢禮藩儀,遭到梁太后為主的保守派極力反對。對此,夏惠宗想用大臣李清策的建議,將河南地區歸還宋朝,以利用宋朝削弱外戚勢力。不料機密洩漏,梁太后殺李清策,幽禁夏惠宗。梁太后此舉引來皇黨、仁多族的叛亂,連吐蕃禹藏花麻都向宋朝請求派兵攻打梁太后。此時宋朝正值宋神宗王安石變法而國力增強,並在1071年由王韶於熙河之戰占領熙河路,對西夏右廂地區造成威脅。1081年宋神宗聽從种諤建議,趁西夏內亂之際,以李憲為總指揮發動五路伐夏,目標興慶府。梁太后採取堅壁清野策略,襲擊糧道以粉碎五路宋軍,宋軍最後只奪下蘭州。隔年年宋軍採取碉堡戰術,派徐禧興建永樂城,步步壓縮西夏在橫山的軍事空間。梁太后趁永樂城新建之初,率30萬大軍包圍攻陷,宋軍慘敗,史稱永樂城之戰。西夏雖然多次擊潰宋軍,但與宋朝貿易中斷使經濟衰退,戰事頻繁又大耗國力,人民不滿。梁太后與梁乙埋最後於1083年讓夏惠宗復位,以平和矛盾,然而夏惠宗依舊沒有掌握實權。梁乙埋去世後,政權轉由其子梁乙逋掌握。1086年夏惠宗在憂憤之下去世,由3岁儿子李乾顺即位,即夏崇宗。

此時西夏政權又落入小梁太后及梁乙逋手中。宋朝宋哲宗時期,知渭州章楶建議對西夏採取經濟制裁與碉堡作戰,其後為了實踐這套戰術,他在公元1096年於西邊的渭川修建平夏城與灵平砦,並且多次擊退夏軍。隔年宋軍攻入東邊的洪州、鹽州。1098年小梁太后偕同夏崇宗猛攻平夏城而敗,大將嵬名阿埋與妹勒都逋均被擒,史稱平夏城之戰。宋軍隨後興建西安州與天都寨,打通涇原路與熙河路,秦州變成內地。宋朝控制橫山地區後,西夏處境日益艱困。1099年在遼朝遼道宗的周旋下,宋夏再度和談,西夏向宋朝請罪而終。西夏在母黨專權的十年裡,梁乞逋依仗「梁氏一門二后」的威勢,連連發動与北宋和辽朝的战争,使西夏蒙受嚴重損失。他經常在朝廷上向眾大臣誇耀自己的功勞,認為西夏連年出征,主要就是讓宋朝屈服,只有這樣才能使西夏獲得和平。環慶之戰時,梁乙逋被小梁太后制止出征而懷恨在心。他意圖叛變,但是事機敗露。小梁太后命嵬名阿吳、仁多宗保與撒辰率兵逮捕處死。小梁太后親自專權後,為了加強對宋朝戰事,多次向遼朝請求援軍被拒。遼廷對小梁太后極度反感,認為遼夏戰爭是由她引起,而小梁太后因多次被拒也惡言相向。1099年夏崇宗親政在即,但「梁氏專恣,不許主國事」。遼朝遼道宗派使至西夏,用毒藥毒死小梁太后。至此長期的太后專政終止,西夏皇帝得以親政。

1099年夏崇宗亲政后采取依附辽朝,與北宋修和的策略,逐年減少战争。對內推廣漢文化,注重經濟,使得社会经济得到恢复和发展。宋朝宋徽宗時期,宋廷執行「紹盛開邊」政策。1114年童貫經略西夏,率領六路宋軍(包含永興、秦鳳兩路)伐夏。雖然西夏多次擊敗劉法、劉仲武與种師道等宋將,但宋軍也攻陷不少堡壘。最後西夏緊急向遼朝請求周旋,到1119年宋軍才率軍而退,夏崇宗再度向宋朝表示臣服。此時西夏國勢不如以往,而北宋也瀕臨崩潰邊緣。

1115年金朝興起,三國鼎立的局面被破壞,遼朝、北宋先後被滅,西夏經濟被金朝掌控。1123年退往漠北的辽朝遼天祚帝有意逃往西夏,金朝將領完顏宗望勸誘夏崇宗捕捉辽帝,許以下寨以北、陰山以南的遼地,並以進攻西夏為脅。夏崇宗答应条件,轉而連金滅遼,从此西夏归服金朝。1125年遼朝亡後,金朝約西夏夾攻北宋,並且給予宋地為誘餌。西夏占領天德軍、雲內等地後,1126年又被金朝強占,並且被強索河東八館之地。金朝為了補償西夏,同意占領陝西後將橫山地區歸還,但又違約。這些都讓金夏關係處於不信任的狀態。然而西夏與南宋隔絕,又讓西夏只能依賴金朝的經濟。這些都使得西夏對金朝維持鬆弛的和平,最多只有小規模的戰事。1141年金朝同意設置榷場,一度開放鐵禁。但是金世宗時不願以紡織品換取西夏的奢侈品,採取貿易緊縮的方式,到十年後才恢復正常貿易。夏崇宗於1139年去世後由其子李仁孝繼位,即夏仁宗。

夏仁宗時期是西夏的文化思想的發展到達高峰,對金朝大致上處於和平狀態。但是他重文輕武、務虛廢實的方式,使西夏軍力逐步走向衰落。宋朝降將任得敬才智均佳,陸續平定1140年夏將蕭合達叛變與隔年的山讹首領慕洧、慕濬投奔金朝之亂,備受夏仁宗重用。1143年發生大饑荒和地震,民不聊生,哆訛等人於威州、靜州與定州發動民變,夏仁宗又派任得敬平定。任得敬因被重用而野心膨脹,受晉王李察哥推薦入京。1156年李察哥去世後掌握政權,擴大私有勢力。1160年被封為楚王,出入等同皇帝。任得敬有意篡位,他以靈州為都城,1170年又迫夏仁宗給予靈州、西夏西南等領地。然而屢次不受金廷支持,任得敬與其弟任得聰等人陰謀叛亂。夏仁宗在金朝的支持下成功撲滅任黨,這個掌握政權二十年的權臣終於被拔除。夏仁宗於1143年的民變後,為了促進經濟穩定而推行改革。他改良地租和賦稅制度;發展教育,實行科舉;推崇儒術,以科舉取仕,這些措施對抑制世家大族有一定的作用;改革禮樂和法律。到天盛年間出現了盛世。1193年夏仁宗去世,子李純佑立,即夏桓宗。

夏桓宗基本上依循夏仁宗的國策,對內安國養民,推行漢法儒學,對外與金朝和好。但是,此時西夏過於安逸,軍力大大衰減。不久,北方的蒙古帝國興起,打破金宋與西夏三國鼎立的地位。夏仁宗的弟弟越王李仁友在挫敗任得敬之亂有功,去世後其子李安全上表請求表彰其父功勳與承襲王位。然而夏桓宗不但不同意,還降其為鎮夷郡王。李安全不滿,遂萌篡奪帝位之心。1206年與夏桓宗之母羅太后聯合廢夏桓宗,自立為帝,即夏襄宗。不久,夏桓宗去世。

西夏與漠北鄰國克烈部的關係十分友好,但是蒙古部的鐵木真崛起後開始威脅克烈部,成為西夏晚期的外患。1203年克烈部被鐵木真攻滅,其領袖王汗之子桑昆逃往西夏。兩年後,鐵木真率軍攻打西夏,掠奪西夏邊界城市而去。夏桓宗為擊退外患,改興慶府名為中興府,取夏國中興之意,實際上西夏反而壟罩在蒙古的威脅之下。1206年鐵木真建立蒙古帝國,即成吉思汗,後被尊稱元太祖。成吉思汗為了要攻滅敵國金朝,勢必要切斷金夏聯盟,所以西夏成為他的目標之一。隔年夏襄宗奪位不久,成吉思汗率大軍攻破西夏要塞斡羅孩城(今内蒙古乌拉特中后旗西境),因各路夏軍奮力抵抗而擊退之。1209年蒙古降服高昌回鶻,河西地區也暴露在蒙古威脅之下。蒙古第三次征夏即自河西入侵,出黑水城,圍攻斡羅孩關口。夏襄宗派其子李承禎率軍抵抗失敗,夏將高逸被俘而死。蒙軍又攻陷西壁讹答守備的斡羅孩城,直逼中興府的最後防線克夷門。夏將嵬名令公率軍伏擊蒙軍,最後仍被蒙軍擊潰。中興府被蒙軍圍困,夏襄宗派使向金朝金帝完顏永濟求救,但是金帝拒絕,還以鄰國遭攻打為樂而坐視不救。最後夏襄宗納女請和,貢獻大量物資,並且附蒙伐金。

夏襄宗附蒙伐金後,對金朝進行長達十餘年的戰爭,使雙方損失很大。國內方面,西夏百姓十分贫困,經濟生產受到破壞,军队衰弱,政治腐败。夏襄宗本身也沉湎于酒色之中,整日不理朝政。1211年齊王李遵頊發動宮庭政變,廢夏襄宗自立為帝,即夏神宗,史書稱為状元皇帝。夏神宗不顧國內大臣反對,仍然堅持附蒙抗金,金宣宗也多次反擊之。此時西夏國內社會經濟凋蔽,民變不斷。1216年因為西夏不肯幫助成吉思汗西征,次年成吉思汗率軍第四次进攻西夏。夏神宗以太子李德任守中興府,自己逃至西京靈州。最後李德任派使向蒙古和談才終戰。1223年由于夏神宗不愿做亡国之君,便让位给次子李德旺(原太子李德任被廢),即夏獻宗。此時夏廷已經認清蒙古將會滅亡西夏,夏獻宗決定採取聯金抗蒙的策略,趁成吉思汗西征時派使聯合漠北諸部落抗蒙,以便鞏固西夏北疆。當時總管漢地的蒙將孛魯(木華黎之子)察覺西夏的意圖,於1224年率軍從東面攻入西夏,攻陷銀州,夏將塔海被俘。隔年成吉思汗得勝返國,同時率軍攻打沙洲,但夏將籍辣思義極力防守。最後夏獻宗同意蒙軍條件請和,蒙軍撤退。

1226年成吉思汗以夏獻宗沒有履約及接納蒙古仇人為由,越大漠向西夏出征,此即蒙古滅西夏之戰。成吉思汗與速不台率大軍逼降黑水城(今內蒙額濟納旗),而後成吉思汗屯軍渾垂山(今甘肅酒泉北)避暑,並令速不台率別部迂迴攻入撒里畏吾兒(即黃頭回紇)與西蕃邊部(吐蕃諸部)等部。成吉思汗派忽都鐵穆兒、昔里铃部、察罕等將先後攻下肅州、甘州與沙州(1227年淪陷),大軍前進至涼州並降伏守將斡扎篑,至此河西走廊全數淪陷。夏獻宗憂患而死,由其侄南平王李睍繼位,即夏末帝。同年八月,成吉思汗率軍穿越沙漠,攻占應理(今寧夏中衛)進軍黃河九渡渡過黃河,主力逼近廢太子李德任守衛的西平府靈州。夏末帝派嵬名令公率軍與李德任會合,蒙夏雙方於凍結的黃河決戰,此役西夏軍死傷慘重,最後城陷被殺,但蒙軍也受損不少,成吉思汗駐守鹽州川(今陝西定邊花馬池)休整軍隊。1227年,成吉思汗此時將目標訂為攻佔西夏退路,並迂迴侵入金朝關中。他命蒙軍(應為蒙將阿術魯率領)包圍中興府(今寧夏銀川),並且率軍西南渡河攻下西夏積石州(今青海循化),與早已入侵西夏西南的速不台會合,陸續拿下西夏西寧(今青海西寧)、臨洮府(今甘肅臨洮)、金朝德順(今甘肅隆德)等西夏與金朝領地。同年6月,成吉思汗駐夏六盘山,又南下攻取金朝秦州(今甘肅天水),逼近關中京兆府(今陝西西安)。1227年夏末帝在中興府被圍半年後投降蒙古,西夏灭亡。成吉思汗此时已病死,但密不发丧,以免西夏反悔。而後诸将遵照成吉思汗遗命将夏末帝杀死,並且殺盡西夏宗室。而中興府百姓因蒙将察罕的勸諫而沒有被屠城。

1038年西夏立國時,疆域範圍在今宁夏,甘肃西北部、青海东北部、内蒙古西部以及陕西北部地区。东尽黄河,西至玉门,南接萧关(今宁夏同心南),北控大漠,佔地兩萬餘里。西夏東北與遼朝西京道相鄰,東面與東南面與宋朝為鄰。金朝滅遼宋後,西夏的東北、東與南都與金朝相鄰。西夏南部和西部是吐蕃諸部、黃頭回鶻與西州回鶻相鄰。國內三分之二以上是沙漠地形,水源以黃河與山上雪水形成的地下水為主。首都興慶府所在的銀川平原,西有賀蘭山作屏障,東有黃河灌溉,有「天下黃河富寧夏之稱」。

西夏是党项族建立的王朝,党项族原本定居四川松潘高原一帶。唐高宗時期受吐蕃壓迫,最後在唐廷協助下遷移到河套陝北一帶,分為平夏部與東山部,至此建立西夏的龍興之地。881年因平夏部拓跋思恭平黃巢之亂有功,被封為夏州節度使,至此正式領有銀州(今陝西米脂縣)、夏州(今陝西橫山縣)、綏州(今陝西綏德縣)、宥州(今陝西靜邊縣)與靜州(今陝西米脂縣西)等五州之地。宋朝時,宋太宗併吞夏州節度使之地。然而李繼遷不願意投降,率部四處攻擊,最後收復五州之地。攻下靈州後,將勢力擴展到黃河河套地區與河西走廊。夏景宗繼位後持續鞏固河西走廊,並且開國稱帝,疆域擴大到二十個州。而後夏景宗與宋朝征戰於橫山地區,並有意占領關中。夏景宗之後,西夏與北宋展開拉鋸戰,雙方互相占領對方的堡壘城寨,並且擴大到河煌青海地區。夏崇宗後期喪失橫山地區,一度造成危機。金朝滅遼朝與北宋後,西夏陸續收復失地,並且占領黃河前套地區。然而其勢力被金朝所侷限,領土擴張不大。到夏仁宗時期大約有22個州,這是西夏版圖最後穩固的狀態。

西夏行政區劃大體上是州(府)縣兩級,一些重點州則設府。另分左右廂十二監軍司,作為軍管區。西夏在建國前只領有五州之地。占領河套地區與河西走廊後,在夏崇宗、夏仁宗時期達到22州:河南9州、河西9州,熙秦河外4州。文獻記載比較明確的有32州。州所轄縣不多,有的就是堡壘和城鎮,其規模比不上宋朝的州。其目的只是壯大聲勢,安置親信以嚴密控制而已。升州為府的有河套的興州(興慶府、中興府)、靈州(西平府)與河西走廊的涼州(西涼府)、甘州(宣化府)等。夏州、靈州與興州相繼是西夏立國前的都城,地位十分重要。涼州控管河西走廊與河套地區的樞紐,地理位置重要。甘州所設的宣化府,負責處理吐蕃、回鶻的事務。左右廂與十二監軍司主要是夏景宗為了方便對軍隊的管理與調遣、佈防而設置的。每一監軍司都仿宋制立有軍名,規定駐地。

西夏政治是蕃漢聯合政治,党項族為主要統治民族,並且聯合漢族、吐蕃族、回鶻族共同統治。皇族注意與党項貴族的關係,以通婚與權力分享攏絡,而母黨「貴寵用事」。這些都使皇族與母黨、党項貴族之間時常發生衝突。西夏在前期即有像遼朝那樣的蕃漢官制,但是到中後期全面採用宋朝制度後,蕃官逐漸式微。

西夏的国家体制和统治方式深受儒家政治文化影响。官制自1038年夏景宗立國後確立,大體上學自宋朝制度。官分文武兩班,中書司、樞密司與三司(鹽鐵部、度支部與戶部)分別管理行政、軍事與財政。御史台管監察、開封府管理首都地區的事務,其他還有翊衛司、官計司、受納司、農田司、群牧司、飛龍苑、磨勘司、文思院、番學與漢學等機構。隔年,夏景宗仿照宋朝制度設立總理庶務的尚書令,改宋朝二十四司為十六司,分理功、倉、戶、兵、法、士六曹,使西夏官制和機構已頗具規模。到夏毅宗時又增設各部尚書、侍郎、南北宣徽使及中書、學士等官。一來職官和機構愈分愈細,二來官制改革由擴充政治軍事的官職轉向擴充社會經濟文化方面的官職。

蕃官是專由党項族擔任的官職,有一說此為爵位制度。蕃官主要是為了保持党項貴族在政權中的主導地位,非党項族不能擔任,有寧令(大王)、謨寧令(天大王)、丁盧、丁弩、素齋、祖儒、呂則、樞銘等等官稱。夏景宗增設番官後,還學習遼朝與吐蕃的一些制度,如南北面官制。西夏的蕃官制度很雜亂,夏毅宗時又增設不少官職,至今仍不太清楚其官職功能,有一說蕃官只是西夏文表示的漢官官名而已。西夏文諺語也提到「衙門官員曾幾何,要數弭藥為最多」,表明党項族當官為數不少。隨着西夏皇帝越來越崇尚漢法,改蕃禮、用漢儀,蕃官系統逐漸式微。夏崇宗以後,蕃官就在也沒出現在相關文獻中。

關於法律方面,因為西夏舊律有不明疑礙處,夏仁宗在“尚文重法”的主張下頒布《天盛改舊新定律令》,又稱《天盛律令》、《開盛律令》。主要由北王兼中書令嵬名地暴與中書、樞密院宰輔要員及中興府、殿前司、閤門司等重要官員參與編寫。該法典參考了唐朝、宋朝的法典,並且結合本國的國情、民情和軍情,使得更加切合實際。在某些方面(如畜牧業、軍制、民俗,等等)更具有本民族的特點。

西夏地理處於四戰之地,陸續要應付後唐、回鶻、吐蕃、宋朝、遼朝、金朝、西遼與蒙古的威脅與戰爭,所以外交是夏廷十分重視的環節。外交策略主要是聯合或依附強者,並且攻擊弱者、以戰求和。這些策略使自己得以不斷延續、發展。然而依附國過於強大,最後難逃滅亡之命運。

西夏早在夏州政權時期(定難軍)就奉唐朝、五代諸國與北宋為宗主國,以維持自身勢力。後來北宋併吞夏州政權,李繼遷舉兵再起。此時他採取事奉遼朝、連遼抗宋的策略,多次擊退宋軍,並且擴張勢力。並且於990年被遼朝遼聖宗冊封為夏國王。到李德明時,為了鞏固新領地,對北宋和談,於1006年簽署景德和議。然而李德明依舊維持與遼朝的關係。除了應付遼、宋的戰事外,為了稱霸河西、先後攻滅甘州回鶻、沙洲歸義軍,對抗吐蕃六谷部、唃廝囉國等,與西州回鶻為鄰。

夏景宗時正式稱帝建國,自稱邦泥定國,稱男不稱臣,並且多次入侵宋朝邊疆。宋仁宗不滿西夏獨立,派兵攻打之,至此宋夏戰爭爆發。夏景宗在三大戰役(三川口之戰、好水川之戰與定川寨之戰)戰勝宋朝後,雙方於1044年簽訂慶曆和約。宋朝給予「夏國主」名號,西夏皇帝對宋朝稱臣,但實際上西夏皇帝在國內仍以君王自稱。宋朝給與金錢、茶葉等大量物資。西夏雖然擊敗北宋,但惹來遼朝不滿,雙方發生三次戰爭(賀蘭山之戰),最後以西夏稱臣作收。而後北宋的宋神宗為了擊敗西夏,趁西夏內亂之際發動五路伐夏與永樂城之戰,最後都以西夏戰勝作收。然而西夏國力漸衰,橫山地區又被北宋占領,此後有賴遼朝周旋方能穩定宋、遼、西夏三國鼎立的關係。

金朝崛起後滅遼朝與北宋,西夏為了自保,放棄遼夏同盟,臣服於金朝。金朝包圍西夏的東方與南方,掌握西夏的經濟力,所以夏廷對金朝不敢輕舉妄動,最多只有小規模的戰事。蒙古帝國崛起後,多次入侵西夏,破壞金夏同盟。夏襄宗與夏神宗改採取聯蒙攻金的策略,多次與金朝發生戰爭,然而此為錯誤的方針。到夏獻宗時才改連金抗蒙,但不久就在蒙夏戰爭中於1227年亡國。金史稱西夏「立國二百餘年,抗衡遼、金、宋三國,偭鄉無常,視三國之勢強弱以為異同焉。」。

西夏對於回鶻、吐蕃等少數民族採取懷柔與招撫的方式,似乎比宋朝還要好。例如西使城(今甘肃定西西南)吐蕃首领禹藏花麻不愿降宋朝,又受到宋军王韶的攻掠。夏毅宗立即派兵支援,將宗女嫁給他。禹藏花麻遂把西使城及兰州献给西夏。

軍事制度是以党項部落兵制為基礎,加入宋朝制度而改良。西夏是以戰立國的國家,軍隊是賴以維生的基礎。所以實行全民皆兵的制度,平時生產,戰時作戰,軍事與社會經濟合為一。除了給予軍官和正軍很少的軍事裝備之外,其餘作戰一律自帶糧食。最小單位是「抄」,每抄由三人組成,主力一人,輔主一人,負擔一人。樞密院是西夏最高的軍事統御機構,下設諸司。指揮系統分別是統軍、行主、佐將、首領、佐首領、小首領。全國軍隊分成中央軍與地方軍等兩個系統。軍事佈防以賀蘭山、興慶府與靈州為三角防線,成為西夏作戰的核心。當大敵逼近時,邊防軍迅速回京助守;邊將吃緊時,主力軍立即機動支援,軍隊調動十分靈活。作戰時有利則進,不利則退。由於地形以沙漠、山岳為主,所以夏軍擅長誘敵設伏、斷敵糧道、集中兵力作運動戰,所以時常能以少擊多。

中央軍分為擒生軍、衛戍軍、侍衛軍、潑喜、铁鹞子、撞令郎與步跋子等。擒生軍人數約十萬,主要任務是在作战中掳掠生口作奴隶,相當遼軍的「打草穀騎」,不擔負決戰任務。衛戍軍人數約兩萬五千,部屬在興慶府周圍地區,裝備精良,所配置的副兵多達七萬,是夏軍的主力部隊。侍衛軍又號「御園內六班直」,人數約五千,是由豪族子弟中選拔善於騎射者組成的一支衛戍部隊,負責保衛皇帝安全,分三番宿衛。潑喜約兩百人,是西夏的砲兵,掌握火蔟黎的技術,又能拋射石彈。由於由騎兵施放,十分靈活。鐵鷂子約三百人,後擴充至萬人,是西夏的重甲騎兵,具有機動靈活的特點,在平地作戰具有威力,時常隨皇帝出入作戰。撞令郎,由俘獲的健壯漢族士兵擔任,作為戰事的砲灰,減少西夏軍人的損失。步跋子,西夏的步兵,擅長山區作戰,由橫山等山間部落的丁壯組成。時常與鐵鷂子聯合突擊敵軍。

地方軍部分,夏景宗將全國軍區分為左廂、右廂與十二監軍司,共有左廂神勇軍司駐銀川彌陀洞(今陝西榆林市東)、祥祐軍司駐石州、嘉寧軍司駐宥州、靜塞軍司駐韋州、西壽保泰軍司駐柔狼山北(今甘肅平川)、卓囉和南軍司駐蘭州黃河北岸喀羅川(今甘肅永登)、右廂朝順軍司駐賀蘭山克夷門(今寧夏石嘴山區)、甘州甘肅軍司駐甘州刪丹縣故地、瓜州西平軍司駐瓜州、黑水鎮燕軍司駐居延海黑水城(內蒙古額濟納旗)、白馬強鎮軍司駐婁博貝(內蒙古阿拉善左旗吉蘭泰鎮)、黑山威福軍司駐河套(內蒙古五原縣)。全盛時期「諸軍兵總計五十餘萬」,軍兵種主要是騎兵和步兵兩種。每一監軍司都仿宋制立有軍名,設有都統軍、副統軍和監軍司各一員,由皇帝任命貴族擔任。下設指揮使、教練使及左右侍禁官等數十員,由党項人和漢人分任。 

由於沒有專為記錄西夏人口的史書,使得西夏人口的統計十分模糊,現今史學界也沒有一個統一的數據。然而西夏採取全民皆兵的制度,可以由兵力數量去類推人口量。目前認為人口數的下限不低於三十萬戶,上限不超過兩百萬。據《宋史》記載,西夏具有五十萬大軍,中間相差十幾萬。今日一種論點為西夏軍具有三十七萬,並在《東都事略·西夏傳》找到「曩宵有兵十五萬八千五百人」的記載,認為此差距是指党項族的西夏軍。另一種論點是將西夏地方軍五十餘萬人加上中央軍約十九萬人,總共約七十萬左右。由於西夏採取全民皆兵的制度,人口數推算是二百萬至三百萬左右。然而還有一派論點認為此數據過於誇張,他們根據《續資治通鑒長編》與《隆平集》的記載,西夏兵力約在十五萬至十八萬,總戶數約三十萬左右。

關於西夏人口的變化,各家說法不同。根據《中国人口史》赵文林與谢淑君的版本推算,西夏人口峰值在1038年,243万人,后因西夏和辽朝、北宋陸續發生战争而不断减少,战争停止后又缓慢回升。1069年,西夏建国后的人口峰值,230万人,后又因为战争不断减少。最後在1131年至1210年年間,人口一直维持在120万左右。而《中国人口发展史》葛剑雄的版本推算,西夏人口的峰值在夏景宗超过300万。1127年后,西夏人口一直未超过300万。根據《中国人口史》(第三卷)辽宋金元时期,吴松弟的版本推算,西夏人口的峰值在1100年夏崇宗時期,大约300万人。西夏人口密度低於北宋各路與唐朝各道的人口密度,然而比唐朝的隴右道高。這是因為西夏領土大多是由沙漠組成,適居範圍不大,在加上西夏採取全民皆兵的制度,因連年戰事不斷,人口消耗不少。

西夏是一個多民族的朝代,其主要民族有党項族、漢族、回鶻族與吐蕃族等。西夏人大都身材修長高大,充分表現出党項羌人粗獷、剽悍、豪爽的民族性格。社會等級分明,以日常服飾禮儀區分。皇帝、文官與武官的服裝均有規定標準,平民百姓只准穿青綠色衣服,貴賤等級分明。冠飾也是區分等級的依據之一,皇帝氈冠、皇后龍鳳冠、命婦花杈冠,文官幞頭、武官還有各種頭冠樣式。

西夏的經濟是以畜牧業為基礎,主要以牛、羊、馬和駱駝為大宗。農產品主要有大麥、稻、蓽豆和青稞等物。藥材和一部分手工製品也特別有名。西夏在冶煉、採鹽製鹽、磚瓦、陶瓷、紡織、造紙、印刷、釀造、金銀木器製作等手工業生產也都具有一定的規模和水平。慶曆和議後,宋廷設置榷場,恢復雙方貿易往來,西夏的手工業生產和商業貿易迅速發展。夏崇宗與夏仁宗時期,西夏經濟大大的發展,四方的物品會集到首都興慶,進入西夏經濟最鼎盛的時期。

党項族是游牧民族,其農業較畜牧業晚發展,農牧並重是西夏社會經濟的特色。李繼遷時期陸續領有河套與河西走廊地區如靈州(今寧夏吳忠市)、興慶(今寧夏銀川)、涼州(今甘肅武威)和瓜州(今甘肅安西)等地後,由於這些地區豐饒五穀,「地饒五穀,尤宜稻麥」。其中興靈地區與橫山地區為西夏糧食的主要產地,其產量还可以用来救济灾民,而橫山地區的糧食時常提供給伐宋夏軍使用。西夏主要的農產品有大麥、稻、蓽豆和青稞等物,當發生戰亂或天災時只能以大麦、荜豆、青麻子等物充饥,並且等待自靈夏所運來的糧食。藥材中比較有名的有大黃、枸杞與甘草,皆是商人極力採購的重點商品之一。其他還有麝臍、羱羚角、柴胡、蓯蓉、紅花和蜜蠟等。党項族向漢族學習比較先進的耕種技術,已普遍使用鐵製農具和牛耕。西夏領地以沙漠居多,水源得來不易,所以十分重視水利設施。西夏古渠主要分布在兴州和灵州,其中以兴州的汉源渠和唐徕渠最有名。夏景宗時興修從今青銅峽至平羅的灌渠,世稱「昊王渠」或「李王渠」。在甘州、凉州一带,则利用祁连山雪水,疏浚河渠,引水灌田。在這些水源中,又以甘州境内的黑水最为著名。横山地區則以无定河、白马川等等為水源。夏仁宗時期修訂的法典《天盛改舊新定律令》中,鼓励人民开垦荒地,並規定水利灌溉事宜。

西夏的畜牧業十分發達,夏廷還設立群牧司以專屬管理。牧區分布在橫山以北和河西走廊地區,重要的牧區有夏州(今陝西靖邊北白城子)、綏州(今綏德)、銀州(今米脂西北)、鹽州(今寧夏鹽池北)與宥州(今陝西定邊東)諸州,還有鄂爾多斯高原、阿拉善和額濟納草原及河西走廊草原,都是興盛的牧區。畜類主要以牛、羊、馬和駱駝為大宗,其他還有驢、騾、豬等。馬匹可做軍事與生產用途,並且是對外的重點商品與貢品,以「党項馬」最有名。駱駝主要產於阿拉善和額濟納地區,是高原和沙漠地區的重要運輸工具。在西夏辭書《文海》中對牲畜的研究十分細緻,有關餵養、疾病、生產與品種的區分都有詳細的說明,表現出西夏人對畜牧的經驗十分豐富。除畜牧业外,狩猎业也十分興盛,主要有兔鹘、沙狐皮、犬、马等。其規模不小,例如對遼朝的貢品中,即有沙狐皮一千张。狩猎业在西夏中後期時仍然興盛,受西夏大臣所重視,西夏軍隊也時常以狩獵為軍事訓練或演習。

西夏手工业分官营民营两种,主要以官營為主。其生產目的主要是供西夏貴族使用,其次則是生產外銷。手工業門比較齊全,夏仁宗修訂的法典《天盛改舊新定律令·司序行文門》中即分類詳細。手工業以紡織、冶煉、金銀、木器製作、採鹽、釀造、陶瓷、建築、磚瓦等為主,兵器製造也較為發達。

西夏的青鹽是宋夏邊界人民最喜歡的商品,也是西夏重要的財源之一。主要產地有鹽州(今寧夏鹽池北)的烏池、白池、瓦池與細項池,河西走廊和西安州(今宁夏海原西)的鹽州與鹽山,靈州(今寧夏吳忠市)的溫泉池等等老井。所出產的青鹽味甘價賤,比宋朝的河東解鹽更具歡迎,另外西安州的碱隈川還產白鹽、紅鹽,只是質量不如青鹽。西夏青白盐除了供西夏人民食用外,主要用于同宋朝、辽朝、金朝进行官方贸易,其中运往宋关中地区最多,并以此换回大批粮食。宋廷為此禁止西夏進口青鹽,宋人只能透過走私進口,谋取暴利。

西夏的氈毯是外銷的名貴商品,其中以白駱駝毛製成的白氈於《马可波罗遊記》記載有「為世界最良之氈」的美稱。西夏矿产比较丰富,所以其兵器製造業,如神臂弓、旋風炮以及勁弩不能射入的冷鍛鎧甲均為世人稱道,特別值得一提的是「夏國劍」,鋒利無比,貴重一時,當時就為宋人所珍視。

西夏印刷業頗為發達,西夏人為了吸收漢族文化,並且維護自己的文化,用夏漢兩種文字雕印書籍。為了發展印刷業,夏廷還設置刻字司以專司出版,另外私人和學校也可能刻印書籍。刻書種類繁多,有佛經、漢學經典、文學詩書、音韻、卜筮咒文、醫學技術等等書籍,其中以佛經數量最多。如1189年夏仁宗就在大度民寺作大法會,散發蕃漢《觀彌勒上升兜率天經》十萬卷,漢《金剛普賢行誦經》、《觀音經》等五萬卷。

西夏本來沒有瓷器,主要靠掠奪宋人來獲得。慶曆和議後,西夏自漢族學得制瓷技術。夏毅宗時期開始發展制瓷業,主要以興慶為生產中心。从考古出土的陶瓷看,西夏烧制的瓷器大多以白瓷碗、白瓷盘等等為主。其瓷器技术上比不上宋瓷,但樸實凝重,形成獨具一格的西夏瓷器。

由於西夏領有絲路商業要道河西走廊,再加上國內只盛產畜牧,對於糧食、茶葉與部分手工品的需求量大,所以對外貿易是西夏經濟的命脈之一,主要分為朝貢貿易、榷場貿易與竊市(私市)。西夏國內的城市商業十分繁榮,興慶、涼州、甘州、黑水城都十分興盛。商品以糧食、布、絹帛、牲畜、肉類為大宗。西夏可以藉由掌控河西走廊以管理西域與中原的貿易往來,與北宋、遼朝、金朝、西州回鶻及吐蕃諸部有頻繁的商業貿易。由於西夏過度壟斷河西走廊,使得部分西域商人改走柴達木盆地,經鄯州(今青海西寧)沿湟水而到達宋朝的秦州(今陝西天水),史稱吐谷渾路。

西夏對中原或北亞的宗主國採取朝貢貿易,時常以駱駝或牛羊等價換取糧食、茶葉或重要物資。西夏在宋夏戰爭獲勝,於慶曆和議中,每年自宋朝獲得銀5萬兩,絹13萬匹,茶2萬斤,每年還在可以各種節日中獲得銀22000兩,絹23000匹,茶1萬斤。西夏自李繼遷叛宋附遼開始向遼朝貢,至遼天祚帝亡國,總計向遼朝貢24次。西夏也會以宋朝的茶葉與歲幣換取回鶻、吐蕃的羊隻,再轉賣給宋、遼、金等國,從中牟取暴利。由於朝貢貿易時常因為戰事而中斷,並不是很穩定。

比較大宗且穩定的貿易為榷場貿易,西夏與北宋、遼朝和金朝的邊境地帶設有共同使用的榷場進行和市,例如與宋朝制定的保安軍(今陝西志丹)、鎮戎軍(今寧夏固原)、麟州、延州等地的榷場;與遼朝在遼西京西北的天德府、雲內和雲中西北的銀瓮口、過腰帶與上石楞坡等地的榷場等。在榷市中,有固定的貿易場地和牙人評定貨色等級,由雙方官府派遣的監督、稽查人員共同管理市場,徵收稅務。貿易內容以牲畜、毛织品、药材為大宗。而“官市”以外的商品种类不受此限。金滅北宋後,由於南宋與西夏隔絕,西夏對外貿易掌握在金朝手中,經濟上不能不依賴於金朝。1141年金朝同意開放保安軍、蘭州、綏德、環州與東勝州的榷場。1172年金朝金世宗以保安軍、蘭州、綏德不產布為由關閉這些榷場,認為以紡織品換取西夏的奢侈品不划算。這使得雙方關係緊張,在夏仁宗末期戰事不斷,十年後才恢復正常貿易。最後比較大量且分散的是「竊市」(私市),也就是非正式市場與走私貿易,例如青鹽貿易即採取走私方式換取宋朝的糧食。

由於西夏商業的興盛,作為流通的貨幣也十分重要:一類是本國鑄造的西夏貨幣;另一類是從宋、金進口的貨幣。早在夏景宗時期即鑄造貨幣,各代皇帝除了夏獻宗、夏末帝之外都有鑄錢實例,夏仁宗還於1158年設立通濟監鑄錢。西夏錢幣的鑄造大都比較精美,而且書法俊逸、流暢。目前面文為西夏文的錢幣已經發現有五種,分別是「福聖寶錢」、「大安寶錢」、「貞觀寶錢」、「乾祐寶錢」以及「天慶寶錢」。

西夏文化深受漢族河隴文化及吐蕃、回鶻文化的影響。並且積極吸收漢族文化與典章制度。發展儒學,宏揚佛學,形成具儒家典章制度的佛教王國。西夏起初是游牧部落,佛教在1世纪东传凉州刺史部以后,於该区逐渐兴盛起来,在西夏建国后开始创造自己独有的佛教艺术文化。内蒙古鄂托克旗的百眼窑石窟寺,是西夏佛教壁画艺术的宝库。在额济纳旗黑水城中发现的西夏文佛经、释迦佛塔、彩塑观音像等,是荒漠的重大发现。另外西夏也大力發展敦煌莫高窟。1036年西夏攻滅归义军後,占領瓜州、沙州,領有莫高窟。从夏景宗到夏仁宗,西夏皇帝多次下令修改莫高窟,使其更加增添了几分光辉。当时莫高窟涂绿油漆,接受中原文化與畏兀儿、吐鲁番风格。此外,表现西夏文化的还有西夏文,又称蕃书。西夏设立蕃學和漢學,使西夏民族意识增强,百姓“通蕃汉字”,文化也增加了许多。值得一提,李元昊曾經頒布禿髮令,命令全國男人三天內必須禿髮,違者格殺勿論。西夏还设立蕃学和太学。史家戴锡章《西夏记》曾言:“夫西夏声明文物,诚不能与宋相匹,然观其制国书、厘官制、定新律、兴汉学、立养贤务、置博士弟子员。尊孔子为文宣帝,彬彬乎质有其文,固未尝不可与辽金比烈!”。

西夏儒學的發展是一種處在儒家影響下的官僚體制與政治文化,制度深受儒家文化影響,從李繼遷伊始至西夏末年,歷代帝王莫不學習與模仿漢制。例如李繼遷時「潛設中官,盡異羌夷之體,曲延儒士,漸行中國之風。」,李德明時 「大輦方輿,鹵薄儀衛,一如中國制。」。西夏党項世代皇親宗室,崇拜孔子,欽慕漢族文化。除了崇儒尚文,還編寫了一些融合和宣揚儒家學說的書籍,如《聖立義海》、《三才雜字》、《德行記》、《新集慈孝傳》、《新集錦合道理》、《德事要文》等。其儒學經過夏景宗、夏毅宗、夏惠宗與夏崇宗的提倡,到夏仁宗之時出現盛況。

夏景宗在建立官制的同時設立了蕃學和漢學,作為文化培養的搖籃。以博學多才的野利仁榮主持蕃學以重視蕃學,並於各州蕃學裡設置教授,進行教學。西夏大致設立了五種學校:蕃學、國學、小學、宮學、太學。西夏建立學校的目的主要是為了培養人才的需要,尊孔子為文宣帝。西夏在中後期還發展科舉制度,夏崇宗後期開始設童子科實行科舉考試,1147年夏仁宗策舉人,立唱名法,復設童子科。西夏後期基本以科舉取士選拔官吏,不論蕃漢及宗室貴族由科舉而進入仕途成為必然的途徑。

西夏崇尚漢文化,但漢文創作的文學作品傳世不多,大多以詩歌和諺語為主。詩歌有宮廷詩、宗教勸善詩、啟蒙詩、紀事詩與史詩等幾類。西夏詩歌有韻律,一般為對稱結構,通常是五言或七言體,也有多言體,每一詩句的音節數目不同。比較有名的有頌揚西夏文創製者野利仁榮的《大頌詩》。史詩性的作品《夏聖根讚歌》,內容多為民間傳說,遣詞造句帶有濃重的民謠色彩。其中開首三句:「黑頭石城漠水邊,赤面父冢白河上,高彌藥國在彼方」,被西夏學學者用來研究党項歷史源起。另外還有讚美重建太學的《新修太學歌》,具有宮廷詩的風格。夏崇宗重視文學,本人曾作《靈芝歌》與大臣王仁忠酬唱,傳為佳話。

西夏諺語對偶工整,結構嚴謹,字數多少不一,內容廣泛地反映了西夏社會的各種面向、並涉及百姓生產、風俗與宗教等內容。著名的西夏諺語集《新集錦合辭》,是由西夏人梁德養於1176年初編、1187年由王仁持補編,共有364条谚语。其內容有「谚语不熟不要说话」的记载,「千千诸人」、「万万民庶」都离不开谚语,凸顯出諺語對西夏人民的重要性。

西夏皇帝十分重視本國國史的編撰工作。斡道沖於李德明時期就掌管撰修西夏國史之職,其後代亦同。夏仁宗時設置翰林學士院,命王僉、焦景顏參照宋朝編修實錄的辦法纂修國史,負責修《李氏實錄》。1225年南院宣徽使羅世昌罷官後,撰寫《夏國世次》,可惜已失。

西夏立國前夕,夏景宗為了建議屬於本國的語言,派野利仁榮仿照漢字結構創建西夏文,於1036年頒行,又稱「國書」或「蕃書」,與周圍王朝往來表奏、文書,都使用西夏文。文字構成多採用類似漢字六書構造,但筆畫比漢字繁多。西夏文學家骨勒茂才認為西夏文和漢文的關係是「論末則殊,考本則同」。西夏文創製後,廣泛運用在歷史、法律、文學、醫學著作,鐫刻碑文,鑄造錢幣、符牌等也都使用西夏文。夏廷又設立蕃學,由野利仁榮主持,選派貴族官僚子弟翻譯漢文典籍與佛教經典等。為了翻譯漢夏文字,又有骨勒茂才於1190年所撰寫的《番汉合时掌中珠》,序言有西夏文和漢文兩種,內容相同。謂「不學番言,則豈和番人之眾;不會漢語,則豈入漢人之數。」表明本書目的在於便於西夏人和漢人互相學習對方語言,是現今研究西夏歷史的重要鑰匙。

西夏人民大致上以佛教為主要信仰,在建國之前則是以自然崇拜為主。 党項族在唐朝四川松潘地區時,就以「天」為崇拜對象。當党項族遷移到陝北之後,由自然崇拜發展到對鬼神的信仰。在建國之後,仍然崇尚多神信仰,有山神、水神、龙神、树神、土地诸神等自然神。例如夏景宗曾「自诣西凉府祠神」。夏仁宗曾在甘州黑水河边立黑水桥碑,祭告诸神,祈求保护桥梁,平息水患。除了崇拜鬼神,党項族還崇尚巫術,並且備受重視。党項族稱巫為「廝」,巫師被稱為「廝乩」,是溝通人和鬼神間的橋樑,主要負責驅鬼與占卜。在战争前實行占卜以问吉凶,於戰爭中经常施行「杀鬼招魂」的巫术。

佛教是西夏的國教,建國前後六次向宋求贖佛經,宋朝賜以《大藏經》。夏景宗在立國後,便開始用西夏文翻譯佛經。五十多年內譯出大小乘佛經820部,3579卷,滿足人民對佛教的需求。除此之外,夏景宗等歷代夏帝與太后也興建許多佛教寺廟高台寺,概括地分為興慶府—賀蘭山中心、甘州—涼州中心、敦煌—安西中心以及黑水城中心。例如有名的承天寺是應夏毅宗母后沒藏太后要求而興建,1093年更重修涼州感通塔及寺廟,隔年立「重修護國寺感通塔碑」。夏崇宗時期更在甘州建築臥佛寺。西夏朝廷大力提倡佛教,提高僧人地位,使僧人不用納稅與負擔雜役;犯罪也可減免罪刑;寺院環境也受朝廷保護。西夏後期受藏傳佛教影響的趨勢日益增大,1159年吐蕃迦瑪迦舉系教派初祖都松欽巴建立粗布寺,夏仁宗派使入藏迎奉。都松欽巴派大弟子格西藏瑣布帶經文到西夏興慶府,被夏仁宗尊為上師,並參與翻譯經文。西夏比元朝還要早設立帝師,提高藏傳佛教的地位。除帝師外,還有國師以及其它有高級職稱的僧人,在推動西夏佛教發展方面起著核心和中堅的作用。

除了佛教以外,西夏也包容其他宗教。西夏有流傳道教,例如夏景宗之子寧明就是學習道家的辟穀術而死。《文海》解釋「仙」字為「山中求道者」,「山中求長壽者」。在西夏晚期,在沙州和甘州一帶還有流傳景教和伊斯蘭教。例如《馬可波羅遊記》中記載敦煌(唐古忒省)與甘州有部分景教和伊斯蘭教徒。

西夏的藝術文化十分多元且豐富,在繪畫、書法、雕刻、舞蹈與音樂都有成就。繪畫方面,以佛教繪畫流傳至今,主要呈現在石窟與寺廟壁畫等,現今以敦煌莫高窟、安西榆林窟等最為豐富。早期學習北宋風格,後來受回鶻佛教與吐蕃藏傳佛教的壁畫藝術的影響,最後形成獨特的藝術風格。在線條採用鐵線與蘭葉描為主,輔以折蘆、蓴菜條;敷彩大量使用石綠打底,使畫面呈獨具風格的冷色調的「綠壁畫」。繪畫內容分別有佛教故事與說法、供養菩薩與人像以及洞窟裝飾圖案等,以《文殊變圖》、《普賢變圖》、《水月觀音圖》與《千手千眼观音经变圖》最為有名。此外,也可在《千手千眼觀世音像》內的《農耕圖》、《踏碓圖》、《釀酒圖》與《鍛鐵圖》中觀察到西夏社會生產和生活內容。木刻版畫方面,大多來自西夏文和漢文佛經中。在黑水城出土的大量佛畫中,有《文殊圖》、《普賢圖》、《勝三世明王曼荼羅圖》等等。內容濃抹重彩,色調深沉。而版畫《賣肉圖》和《魔鬼現世圖》,描繪生動,反映出西夏繪畫的深度。

書法在楷書多見於寫經與碑文,篆書見於碑額與官印。夏仁宗時期的翰林學士劉志直,工於書法,他用黃羊尾毫製作之筆,為時人所效法。雕塑方面十分發達,有鑄銅、石雕、磚雕、木雕、竹雕、泥塑與陶瓷等。其特點比例均衡,刀法細膩,十分寫實。泥塑以佛寺塑像為代表,多運用寫實與藝術誇張手法,刻劃現實生活的人物形象。例如夏崇宗時期修建的甘州大佛寺釋迦牟尼涅槃像、敦煌莫高窟第491窟西夏供養天女彩塑等等。其他陶瓷藝術品也是刻工精細而生動。

西夏在党項時期的樂器以琵琶、橫吹,擊缶為主,其中橫吹即竹笛。後來接受中原音樂的文化,李德明時採用宋制製樂而逐漸繁多。夏景宗建國後,革除唐宋縟節之音,「革樂之五音為一音」。1148年,夏仁宗令樂官李元儒更定音律,賜名《鼎新律》。西夏音樂十分豐富,且設有蕃漢樂人院,夏惠宗時曾招誘漢界娼婦、樂人加入樂院,戲曲如《劉知遠諸宮調》等也已經傳入西夏。西夏時期的舞蹈在碑刻和石窟壁畫中留有生動的形象,富含唐宋舞蹈與蒙古舞蹈的風格。如《涼州護國寺感應塔碑》碑額兩側的線刻舞伎,舞姿對稱,裸身赤足,執巾佩瓔,於豪放中又顯出嫵媚。榆林窟第3窟西夏壁畫中的《樂舞圖》,左右相對吸腿舞狀,姿態雄健。

西夏本身的科技比較薄弱,主要以吸收宋朝或金朝的技術為主,然而在武器鍛鍊方面有獨到之處。在天文氣象方面,主要是學習宋朝的天文與曆法。西夏人設置司天監以觀察天文,並列有分析、解釋天文的「太史」「司天」和「占者」以分析天文。在骨勒茂才的《番漢合時掌中珠·天相》中有對天文星象的詳細記載。例如將天空分為青龍(東)、白虎(西)、朱雀(南)、玄武(北)等方位,每個方位設有7個星宿。在氣象方面也有詳細的分析,例如風有和風、清風、金風、朔風、黑風、旋風;雨有膏雨、穀雨、時雨、絲雨;雲有煙雲、鶴雲、拳雲、羅雲、同雲,等等。曆法方面,西夏至1004年材自北宋獲得《儀天曆》,這是西夏第一本曆書。立國後,設“大恒曆院”的機構掌管曆法的編制和頒行。西夏曆書採用番漢合璧曆書與宋朝頒賜曆書兩類,其詳細情形仍需研究。

在醫學方面,在党項時期,醫學知識十分匱乏,百姓迷信鬼神,大多向神明求醫。在立國後,積極吸收宋朝的醫學與藥學,並且出版《治療惡瘡要論》等醫學著作。並且設有“醫人院”,在政府機構中屬“中等司”。西夏人對病理的認知大多分成血脈不通、傳染、「四大不和」(地、水、火、風)等觀點,其中四大不和是緣自藏傳佛教的說法。由於西夏本身醫學不如中原的朝代,所以一些疑難病症無法醫治,只好求助於宋朝或金朝。例如夏仁宗時,權臣任得敬患病,久治不愈。所以夏仁宗派使者向金朝請求醫療支援。夏桓宗時,其母患病,也派使至金朝求醫。這些都表示西夏醫學不如中原的朝代。

西夏武器製作十分精實,其中以夏國劍最有名,在宋朝被譽為「天下第一」。北宋文學家蘇軾曾請晁補之為其作歌,內有「試人一縷立褫魄,戲客三招森動容」。而西夏鎧甲被譽稱為堅滑光瑩,非勁弩可入,專給鐵鷂子使用。其他有名的攻城武器有名叫「對壘」的戰車、可以越壕溝而進;裝在駱駝鞍上的「旋風炮」,可以發射大石彈;以及最厲害的「神臂弓」,可以射240步至300步,「能洞重扎」。


%% -*- coding: utf-8 -*-
%% Time-stamp: <Chen Wang: 2019-12-26 11:08:11>

\section{景宗\tiny(1032-1048)}

\subsection{生平}

夏景宗李元昊(1003年6月7日-1048年1月19日),又名趙元昊,小字嵬埋,出身党項拓跋氏,即皇帝位後,放棄唐朝賜姓李與宋朝赐姓趙,改姓嵬名氏,更名曩霄(或作曩甯、曩宁),是西夏開國皇帝(1038年11月10日-1048年1月19日在位),為李繼遷孫,李德明長子,生母衛慕氏。生于1003年农历五月五日。

李元昊少年時身型魁梧,而且勤奮好學,手不釋卷,尤好法律和兵書。通漢、蕃語言,精繪畫,多才多藝。其父在位時,他率軍不斷對外出戰,擴大勢力,野心勃勃。1032年以太子身份繼位,仍称藩於宋朝。後來為表獨立,廢唐宋分別賜李姓、趙姓,改姓嵬名,改名曩霄,自称“兀卒”(党项语天子之意),以元魏王室后裔自居,並以嚴酷手段徹底翦除守舊派。大庆三年十月十一日(1038年11月10日)自立为帝,自称世祖始文本武兴法建礼仁孝皇帝,改年号为天授礼法延祚,脱离北宋,国号“大夏”,亦称西夏,定都興慶府。

建國後命大臣野利仁榮創西夏文,大力發展西夏的文化。推動教育,創蕃學,大啟西夏文教之風。開鑿「李王渠」,以便西夏國民耕種。他重用張元等漢人。他三次分別於三川口(今陝西延安西北)、好水川(今寧夏隆德東)及定川砦(今寧夏固原西北)的戰役中大敗北宋,並於遼夏第一次賀蘭山之戰,大勝遼國,奠定西夏與遼、宋兩國并列的地位。本來有意奪取關中之地,攻占長安,但因宋軍頑強抵抗,夏軍戰敗,直搗關中之美夢就此破滅。由於戰事繁多,西夏經濟破損,遂於1044年與北宋簽訂慶曆和議,向宋稱臣,被封為夏國王。為西夏建樹良多,堪稱一代英豪。

李元昊一朝文治武功達於鼎盛,但其人亦有不足之處。在位16年(1032年继承王位起計),猜忌功臣,稍有不滿即罷或殺,反而導致日後母黨專權;另外,晚年沉湎酒色,好大喜功,导致西夏内部日益腐朽,众叛亲离。據說他下令民伕每日建一座陵墓,足足建了三百六十座,作為他的疑塚,其後竟把那批民伕統統殺掉。廢皇后野利氏、太子寧令哥,改立與太子訂親的沒移氏為新皇后,招致殺身之禍,延祚十一年正月初二(1048年1月19日),其子寧令哥趁元昊酒醉時,割其鼻子,元昊最後因失血過多而死,享年46岁,庙号景宗,諡号武烈皇帝,葬泰陵。寧令哥後來因弒父之罪被處死。

\subsection{显道}

\begin{longtable}{|>{\centering\scriptsize}m{2em}|>{\centering\scriptsize}m{1.3em}|>{\centering}m{8.8em}|}
  % \caption{秦王政}\
  \toprule
  \SimHei \normalsize 年数 & \SimHei \scriptsize 公元 & \SimHei 大事件 \tabularnewline
  % \midrule
  \endfirsthead
  \toprule
  \SimHei \normalsize 年数 & \SimHei \scriptsize 公元 & \SimHei 大事件 \tabularnewline
  \midrule
  \endhead
  \midrule
  元年 & 1032 & \tabularnewline\hline
  二年 & 1033 & \tabularnewline\hline
  三年 & 1034 & \tabularnewline
  \bottomrule
\end{longtable}

\subsection{开运}

\begin{longtable}{|>{\centering\scriptsize}m{2em}|>{\centering\scriptsize}m{1.3em}|>{\centering}m{8.8em}|}
  % \caption{秦王政}\
  \toprule
  \SimHei \normalsize 年数 & \SimHei \scriptsize 公元 & \SimHei 大事件 \tabularnewline
  % \midrule
  \endfirsthead
  \toprule
  \SimHei \normalsize 年数 & \SimHei \scriptsize 公元 & \SimHei 大事件 \tabularnewline
  \midrule
  \endhead
  \midrule
  元年 & 1034 & \tabularnewline
  \bottomrule
\end{longtable}

\subsection{广运}

\begin{longtable}{|>{\centering\scriptsize}m{2em}|>{\centering\scriptsize}m{1.3em}|>{\centering}m{8.8em}|}
  % \caption{秦王政}\
  \toprule
  \SimHei \normalsize 年数 & \SimHei \scriptsize 公元 & \SimHei 大事件 \tabularnewline
  % \midrule
  \endfirsthead
  \toprule
  \SimHei \normalsize 年数 & \SimHei \scriptsize 公元 & \SimHei 大事件 \tabularnewline
  \midrule
  \endhead
  \midrule
  元年 & 1034 & \tabularnewline\hline
  二年 & 1035 & \tabularnewline\hline
  三年 & 1036 & \tabularnewline
  \bottomrule
\end{longtable}

\subsection{大庆}

\begin{longtable}{|>{\centering\scriptsize}m{2em}|>{\centering\scriptsize}m{1.3em}|>{\centering}m{8.8em}|}
  % \caption{秦王政}\
  \toprule
  \SimHei \normalsize 年数 & \SimHei \scriptsize 公元 & \SimHei 大事件 \tabularnewline
  % \midrule
  \endfirsthead
  \toprule
  \SimHei \normalsize 年数 & \SimHei \scriptsize 公元 & \SimHei 大事件 \tabularnewline
  \midrule
  \endhead
  \midrule
  元年 & 1036 & \tabularnewline\hline
  二年 & 1037 & \tabularnewline\hline
  三年 & 1038 & \tabularnewline
  \bottomrule
\end{longtable}

\subsection{天授}

\begin{longtable}{|>{\centering\scriptsize}m{2em}|>{\centering\scriptsize}m{1.3em}|>{\centering}m{8.8em}|}
  % \caption{秦王政}\
  \toprule
  \SimHei \normalsize 年数 & \SimHei \scriptsize 公元 & \SimHei 大事件 \tabularnewline
  % \midrule
  \endfirsthead
  \toprule
  \SimHei \normalsize 年数 & \SimHei \scriptsize 公元 & \SimHei 大事件 \tabularnewline
  \midrule
  \endhead
  \midrule
  元年 & 1038 & \tabularnewline\hline
  二年 & 1039 & \tabularnewline\hline
  三年 & 1040 & \tabularnewline\hline
  四年 & 1041 & \tabularnewline\hline
  五年 & 1042 & \tabularnewline\hline
  六年 & 1043 & \tabularnewline\hline
  七年 & 1044 & \tabularnewline\hline
  八年 & 1045 & \tabularnewline\hline
  九年 & 1046 & \tabularnewline\hline
  十年 & 1047 & \tabularnewline\hline
  十一年 & 1048 & \tabularnewline
  \bottomrule
\end{longtable}


%%% Local Variables:
%%% mode: latex
%%% TeX-engine: xetex
%%% TeX-master: "../Main"
%%% End:

%% -*- coding: utf-8 -*-
%% Time-stamp: <Chen Wang: 2021-11-01 16:13:57>

\section{毅宗李諒祚\tiny(1048-1067)}

\subsection{生平}

夏毅宗李諒祚(1047年3月5日-1068年1月),名諒祚,本名寧令兩岔,是西夏第二位皇帝(1048年—1068年1月在位)。夏景宗之子,生母沒藏氏,党項族人。生于1047年农历二月六日。

嘉祐八年丙辰(1063年),谅祚派使者前往宋朝,上表要求改回李姓,宋仁宗下诏谴责他,命令他遵守旧约。然而此后西夏皇室仍多用嵬名为姓氏。

1048年1月19日,景宗被殺,毅宗以一歲幼齡繼位,其母沒藏太后及其家族專權。即位次年(1049年),遼國乘景宗新喪之機,與西夏爆發第二次賀蘭山之戰,西夏大敗,損失慘重,向遼稱臣。福聖承道四年(1056年),太后被殺,舅舅沒藏訛龐執政。十二歲開始預政。奲都五年(1061年),訛龐父子密謀害他,遂殺訛龐及皇后(訛龐之女),立梁氏為皇后,親掌國政。廢行蕃禮,改用漢儀;並增設各部尚書、侍郎等多種官職,以完善中央行政體制。調整州軍,以加強對地方統治。這些措施使皇帝對軍政權力的控制得到加強。他連年對宋用兵,攻掠臨近州縣。先後收降吐蕃首領瞎氈的兒子木征和青唐吐蕃部。後期注意修好與遼、宋關係,減少戰役。拱化四年(1066年),在与北宋作战时受箭伤,拱化五年十二月(1068年1月)去世,享年僅21岁,諡号昭英皇帝。

\subsection{延嗣宁国}

\begin{longtable}{|>{\centering\scriptsize}m{2em}|>{\centering\scriptsize}m{1.3em}|>{\centering}m{8.8em}|}
  % \caption{秦王政}\
  \toprule
  \SimHei \normalsize 年数 & \SimHei \scriptsize 公元 & \SimHei 大事件 \tabularnewline
  % \midrule
  \endfirsthead
  \toprule
  \SimHei \normalsize 年数 & \SimHei \scriptsize 公元 & \SimHei 大事件 \tabularnewline
  \midrule
  \endhead
  \midrule
  元年 & 1048 & \tabularnewline
  \bottomrule
\end{longtable}

\subsection{天祐垂圣}

\begin{longtable}{|>{\centering\scriptsize}m{2em}|>{\centering\scriptsize}m{1.3em}|>{\centering}m{8.8em}|}
  % \caption{秦王政}\
  \toprule
  \SimHei \normalsize 年数 & \SimHei \scriptsize 公元 & \SimHei 大事件 \tabularnewline
  % \midrule
  \endfirsthead
  \toprule
  \SimHei \normalsize 年数 & \SimHei \scriptsize 公元 & \SimHei 大事件 \tabularnewline
  \midrule
  \endhead
  \midrule
  元年 & 1050 & \tabularnewline\hline
  二年 & 1051 & \tabularnewline\hline
  三年 & 1052 & \tabularnewline
  \bottomrule
\end{longtable}

\subsection{福圣承道}

\begin{longtable}{|>{\centering\scriptsize}m{2em}|>{\centering\scriptsize}m{1.3em}|>{\centering}m{8.8em}|}
  % \caption{秦王政}\
  \toprule
  \SimHei \normalsize 年数 & \SimHei \scriptsize 公元 & \SimHei 大事件 \tabularnewline
  % \midrule
  \endfirsthead
  \toprule
  \SimHei \normalsize 年数 & \SimHei \scriptsize 公元 & \SimHei 大事件 \tabularnewline
  \midrule
  \endhead
  \midrule
  元年 & 1053 & \tabularnewline\hline
  二年 & 1054 & \tabularnewline\hline
  三年 & 1055 & \tabularnewline\hline
  四年 & 1056 & \tabularnewline
  \bottomrule
\end{longtable}

\subsection{奲都}

\begin{longtable}{|>{\centering\scriptsize}m{2em}|>{\centering\scriptsize}m{1.3em}|>{\centering}m{8.8em}|}
  % \caption{秦王政}\
  \toprule
  \SimHei \normalsize 年数 & \SimHei \scriptsize 公元 & \SimHei 大事件 \tabularnewline
  % \midrule
  \endfirsthead
  \toprule
  \SimHei \normalsize 年数 & \SimHei \scriptsize 公元 & \SimHei 大事件 \tabularnewline
  \midrule
  \endhead
  \midrule
  元年 & 1057 & \tabularnewline\hline
  二年 & 1058 & \tabularnewline\hline
  三年 & 1059 & \tabularnewline\hline
  四年 & 1060 & \tabularnewline\hline
  五年 & 1061 & \tabularnewline\hline
  六年 & 1062 & \tabularnewline
  \bottomrule
\end{longtable}

\subsection{拱化}

\begin{longtable}{|>{\centering\scriptsize}m{2em}|>{\centering\scriptsize}m{1.3em}|>{\centering}m{8.8em}|}
  % \caption{秦王政}\
  \toprule
  \SimHei \normalsize 年数 & \SimHei \scriptsize 公元 & \SimHei 大事件 \tabularnewline
  % \midrule
  \endfirsthead
  \toprule
  \SimHei \normalsize 年数 & \SimHei \scriptsize 公元 & \SimHei 大事件 \tabularnewline
  \midrule
  \endhead
  \midrule
  元年 & 1063 & \tabularnewline\hline
  二年 & 1064 & \tabularnewline\hline
  三年 & 1065 & \tabularnewline\hline
  四年 & 1066 & \tabularnewline\hline
  五年 & 1067 & \tabularnewline
  \bottomrule
\end{longtable}


%%% Local Variables:
%%% mode: latex
%%% TeX-engine: xetex
%%% TeX-master: "../Main"
%%% End:

%% -*- coding: utf-8 -*-
%% Time-stamp: <Chen Wang: 2021-11-01 16:14:03>

\section{惠宗李秉常\tiny(1067-1086)}

\subsection{生平}

夏惠宗李秉常(1061年-1086年8月21日),西夏第三位皇帝(1068年1月-1086年8月21日在位)。父親夏毅宗,梁皇后所生。

拱化五年农历十二月(1068年1月),毅宗突然病死,英年早逝,年二十一,惠宗以七歲稚齡繼位,生母梁太后及其家族專權,執政期間沒有任何治國良策,西夏國勢積弱,北宋乘機入侵。十六歲時本能親政,但梁氏勢力很大,不能輕易翦滅,因此他仍然不能親政。後來因長期不能親政,憂憤而死,1086年农历七月十日去世,享年僅二十六岁,諡号康靖皇帝。

\subsection{乾道}

\begin{longtable}{|>{\centering\scriptsize}m{2em}|>{\centering\scriptsize}m{1.3em}|>{\centering}m{8.8em}|}
  % \caption{秦王政}\
  \toprule
  \SimHei \normalsize 年数 & \SimHei \scriptsize 公元 & \SimHei 大事件 \tabularnewline
  % \midrule
  \endfirsthead
  \toprule
  \SimHei \normalsize 年数 & \SimHei \scriptsize 公元 & \SimHei 大事件 \tabularnewline
  \midrule
  \endhead
  \midrule
  元年 & 1067 & \tabularnewline\hline
  二年 & 1068 & \tabularnewline
  \bottomrule
\end{longtable}

\subsection{天赐国庆}

\begin{longtable}{|>{\centering\scriptsize}m{2em}|>{\centering\scriptsize}m{1.3em}|>{\centering}m{8.8em}|}
  % \caption{秦王政}\
  \toprule
  \SimHei \normalsize 年数 & \SimHei \scriptsize 公元 & \SimHei 大事件 \tabularnewline
  % \midrule
  \endfirsthead
  \toprule
  \SimHei \normalsize 年数 & \SimHei \scriptsize 公元 & \SimHei 大事件 \tabularnewline
  \midrule
  \endhead
  \midrule
  元年 & 1069 & \tabularnewline\hline
  二年 & 1070 & \tabularnewline\hline
  三年 & 1071 & \tabularnewline\hline
  四年 & 1072 & \tabularnewline\hline
  五年 & 1073 & \tabularnewline\hline
  六年 & 1074 & \tabularnewline
  \bottomrule
\end{longtable}

\subsection{大安}

\begin{longtable}{|>{\centering\scriptsize}m{2em}|>{\centering\scriptsize}m{1.3em}|>{\centering}m{8.8em}|}
  % \caption{秦王政}\
  \toprule
  \SimHei \normalsize 年数 & \SimHei \scriptsize 公元 & \SimHei 大事件 \tabularnewline
  % \midrule
  \endfirsthead
  \toprule
  \SimHei \normalsize 年数 & \SimHei \scriptsize 公元 & \SimHei 大事件 \tabularnewline
  \midrule
  \endhead
  \midrule
  元年 & 1075 & \tabularnewline\hline
  二年 & 1076 & \tabularnewline\hline
  三年 & 1077 & \tabularnewline\hline
  四年 & 1078 & \tabularnewline\hline
  五年 & 1079 & \tabularnewline\hline
  六年 & 1080 & \tabularnewline\hline
  七年 & 1081 & \tabularnewline\hline
  八年 & 1082 & \tabularnewline\hline
  九年 & 1083 & \tabularnewline\hline
  十年 & 1084 & \tabularnewline\hline
  十一年 & 1085 & \tabularnewline
  \bottomrule
\end{longtable}

\subsection{天安礼定}

\begin{longtable}{|>{\centering\scriptsize}m{2em}|>{\centering\scriptsize}m{1.3em}|>{\centering}m{8.8em}|}
  % \caption{秦王政}\
  \toprule
  \SimHei \normalsize 年数 & \SimHei \scriptsize 公元 & \SimHei 大事件 \tabularnewline
  % \midrule
  \endfirsthead
  \toprule
  \SimHei \normalsize 年数 & \SimHei \scriptsize 公元 & \SimHei 大事件 \tabularnewline
  \midrule
  \endhead
  \midrule
  元年 & 1086 & \tabularnewline
  \bottomrule
\end{longtable}



%%% Local Variables:
%%% mode: latex
%%% TeX-engine: xetex
%%% TeX-master: "../Main"
%%% End:

%% -*- coding: utf-8 -*-
%% Time-stamp: <Chen Wang: 2021-11-01 16:14:22>

\section{崇宗李乾順\tiny(1086-1139)}

\subsection{生平}

夏崇宗李乾順(1083年-1139年7月1日),西夏第四位皇帝(1086年8月-1139年7月1日在位)。父惠宗李秉常,母梁皇后。

幼时祖母梁太后对他寵愛有加。1086年8月21日,父亲李秉常去世。他即位时仅3岁,梁氏专政。梁氏统治期间,西夏政治腐败,军队衰弱,北宋趁机来攻,夏军屡战屡败,自幼雄才大略的李乾顺看到了这一点,于1099年即他16岁时灭梁氏而亲政。他亲政后整顿吏治,减少赋税,注重农桑,兴修水利。

在李乾顺的励精图治下,西夏国势强盛,政治清明,社会经济得到很好的发展。另外,李乾顺的外交政策也非常巧妙。贞观初年,李乾顺多次请求辽朝下嫁公主。贞观五年(1105年)三月壬申,娶辽宗室女耶律南仙,立耶律南仙为皇后。贞观八年(1108年)六月,耶律南仙为他生下嫡长子。当时,辽朝、北宋都日益衰落,李乾顺先联辽侵宋,夺大片土地;又在辽天祚帝向西夏求救时断然拒绝,联合金朝灭辽、北宋,趁机取河西千余里之地。大德五年农历六月四日(1139年7月1日)去世。李乾顺庙号为崇宗,谥号为圣文皇帝。

\subsection{天仪治平}

\begin{longtable}{|>{\centering\scriptsize}m{2em}|>{\centering\scriptsize}m{1.3em}|>{\centering}m{8.8em}|}
  % \caption{秦王政}\
  \toprule
  \SimHei \normalsize 年数 & \SimHei \scriptsize 公元 & \SimHei 大事件 \tabularnewline
  % \midrule
  \endfirsthead
  \toprule
  \SimHei \normalsize 年数 & \SimHei \scriptsize 公元 & \SimHei 大事件 \tabularnewline
  \midrule
  \endhead
  \midrule
  元年 & 1086 & \tabularnewline\hline
  二年 & 1087 & \tabularnewline\hline
  三年 & 1088 & \tabularnewline\hline
  四年 & 1089 & \tabularnewline
  \bottomrule
\end{longtable}

\subsection{天祐民安}

\begin{longtable}{|>{\centering\scriptsize}m{2em}|>{\centering\scriptsize}m{1.3em}|>{\centering}m{8.8em}|}
  % \caption{秦王政}\
  \toprule
  \SimHei \normalsize 年数 & \SimHei \scriptsize 公元 & \SimHei 大事件 \tabularnewline
  % \midrule
  \endfirsthead
  \toprule
  \SimHei \normalsize 年数 & \SimHei \scriptsize 公元 & \SimHei 大事件 \tabularnewline
  \midrule
  \endhead
  \midrule
  元年 & 1090 & \tabularnewline\hline
  二年 & 1091 & \tabularnewline\hline
  三年 & 1092 & \tabularnewline\hline
  四年 & 1093 & \tabularnewline\hline
  五年 & 1094 & \tabularnewline\hline
  六年 & 1095 & \tabularnewline\hline
  七年 & 1096 & \tabularnewline\hline
  八年 & 1097 & \tabularnewline
  \bottomrule
\end{longtable}

\subsection{永安}

\begin{longtable}{|>{\centering\scriptsize}m{2em}|>{\centering\scriptsize}m{1.3em}|>{\centering}m{8.8em}|}
  % \caption{秦王政}\
  \toprule
  \SimHei \normalsize 年数 & \SimHei \scriptsize 公元 & \SimHei 大事件 \tabularnewline
  % \midrule
  \endfirsthead
  \toprule
  \SimHei \normalsize 年数 & \SimHei \scriptsize 公元 & \SimHei 大事件 \tabularnewline
  \midrule
  \endhead
  \midrule
  元年 & 1098 & \tabularnewline\hline
  二年 & 1099 & \tabularnewline\hline
  三年 & 1100 & \tabularnewline
  \bottomrule
\end{longtable}

\subsection{贞观}

\begin{longtable}{|>{\centering\scriptsize}m{2em}|>{\centering\scriptsize}m{1.3em}|>{\centering}m{8.8em}|}
  % \caption{秦王政}\
  \toprule
  \SimHei \normalsize 年数 & \SimHei \scriptsize 公元 & \SimHei 大事件 \tabularnewline
  % \midrule
  \endfirsthead
  \toprule
  \SimHei \normalsize 年数 & \SimHei \scriptsize 公元 & \SimHei 大事件 \tabularnewline
  \midrule
  \endhead
  \midrule
  元年 & 1101 & \tabularnewline\hline
  二年 & 1102 & \tabularnewline\hline
  三年 & 1103 & \tabularnewline\hline
  四年 & 1104 & \tabularnewline\hline
  五年 & 1105 & \tabularnewline\hline
  六年 & 1106 & \tabularnewline\hline
  七年 & 1107 & \tabularnewline\hline
  八年 & 1108 & \tabularnewline\hline
  九年 & 1109 & \tabularnewline\hline
  十年 & 1110 & \tabularnewline\hline
  十一年 & 1111 & \tabularnewline\hline
  十二年 & 1112 & \tabularnewline\hline
  十三年 & 1113 & \tabularnewline
  \bottomrule
\end{longtable}

\subsection{雍宁}

\begin{longtable}{|>{\centering\scriptsize}m{2em}|>{\centering\scriptsize}m{1.3em}|>{\centering}m{8.8em}|}
  % \caption{秦王政}\
  \toprule
  \SimHei \normalsize 年数 & \SimHei \scriptsize 公元 & \SimHei 大事件 \tabularnewline
  % \midrule
  \endfirsthead
  \toprule
  \SimHei \normalsize 年数 & \SimHei \scriptsize 公元 & \SimHei 大事件 \tabularnewline
  \midrule
  \endhead
  \midrule
  元年 & 1114 & \tabularnewline\hline
  二年 & 1115 & \tabularnewline\hline
  三年 & 1116 & \tabularnewline\hline
  四年 & 1117 & \tabularnewline\hline
  五年 & 1118 & \tabularnewline
  \bottomrule
\end{longtable}

\subsection{元德}

\begin{longtable}{|>{\centering\scriptsize}m{2em}|>{\centering\scriptsize}m{1.3em}|>{\centering}m{8.8em}|}
  % \caption{秦王政}\
  \toprule
  \SimHei \normalsize 年数 & \SimHei \scriptsize 公元 & \SimHei 大事件 \tabularnewline
  % \midrule
  \endfirsthead
  \toprule
  \SimHei \normalsize 年数 & \SimHei \scriptsize 公元 & \SimHei 大事件 \tabularnewline
  \midrule
  \endhead
  \midrule
  元年 & 1119 & \tabularnewline\hline
  二年 & 1120 & \tabularnewline\hline
  三年 & 1121 & \tabularnewline\hline
  四年 & 1122 & \tabularnewline\hline
  五年 & 1123 & \tabularnewline\hline
  六年 & 1124 & \tabularnewline\hline
  七年 & 1125 & \tabularnewline\hline
  八年 & 1126 & \tabularnewline\hline
  九年 & 1127 & \tabularnewline
  \bottomrule
\end{longtable}

\subsection{正德}

\begin{longtable}{|>{\centering\scriptsize}m{2em}|>{\centering\scriptsize}m{1.3em}|>{\centering}m{8.8em}|}
  % \caption{秦王政}\
  \toprule
  \SimHei \normalsize 年数 & \SimHei \scriptsize 公元 & \SimHei 大事件 \tabularnewline
  % \midrule
  \endfirsthead
  \toprule
  \SimHei \normalsize 年数 & \SimHei \scriptsize 公元 & \SimHei 大事件 \tabularnewline
  \midrule
  \endhead
  \midrule
  元年 & 1127 & \tabularnewline\hline
  二年 & 1128 & \tabularnewline\hline
  三年 & 1129 & \tabularnewline\hline
  四年 & 1130 & \tabularnewline\hline
  五年 & 1131 & \tabularnewline\hline
  六年 & 1132 & \tabularnewline\hline
  七年 & 1133 & \tabularnewline\hline
  八年 & 1134 & \tabularnewline
  \bottomrule
\end{longtable}

\subsection{大德}

\begin{longtable}{|>{\centering\scriptsize}m{2em}|>{\centering\scriptsize}m{1.3em}|>{\centering}m{8.8em}|}
  % \caption{秦王政}\
  \toprule
  \SimHei \normalsize 年数 & \SimHei \scriptsize 公元 & \SimHei 大事件 \tabularnewline
  % \midrule
  \endfirsthead
  \toprule
  \SimHei \normalsize 年数 & \SimHei \scriptsize 公元 & \SimHei 大事件 \tabularnewline
  \midrule
  \endhead
  \midrule
  元年 & 1135 & \tabularnewline\hline
  二年 & 1136 & \tabularnewline\hline
  三年 & 1137 & \tabularnewline\hline
  四年 & 1138 & \tabularnewline\hline
  五年 & 1139 & \tabularnewline
  \bottomrule
\end{longtable}


%%% Local Variables:
%%% mode: latex
%%% TeX-engine: xetex
%%% TeX-master: "../Main"
%%% End:

%% -*- coding: utf-8 -*-
%% Time-stamp: <Chen Wang: 2019-12-26 11:10:34>

\section{仁宗\tiny(1139-1193)}

\subsection{生平}

夏仁宗李仁孝(1124年-1193年10月16日),夏崇宗次子,母為漢人,不知名。

由于其兄李仁爱先于崇宗而死,故被立为太子。1139年7月1日夏崇宗李乾顺去世,李仁孝即位,時年十六歲。在位期間結好金國,以穩定外部環境;重用文化程度較高的党項和漢族大臣主持國政;設立各級學校,以推廣教育;實行科舉,以選拔人才;尊崇儒學,大修孔廟及尊奉孔子為文宣帝;建立翰林學士院,編纂歷朝實錄;重視禮樂,修樂書《新律》;天盛年間,頒行法典《天盛年改新定律令》;尊尚佛教,供奉藏傳佛教僧人為國師,並刻印佛經多種。

乾祐元年(1170年),得金之助,處死權相任得敬,粉碎其分國陰謀。可能因為任得敬的專權跋扈,令仁孝對武官不太信任,政策多數重文輕武,導致軍備開始廢弛,戰鬥力減弱,晚夏戰爭屢戰屢敗,國家於仁宗末年開始走下坡。但總結來說,他統治期間為西夏的盛世,也是金國、南宋的盛世,三國之間戰爭甚少,因此仁孝能專心料理國家內政。各汗國羡慕西夏之強盛,紛紛朝貢。文化臻於鼎盛,為党項文化寫下光輝燦爛的一頁。

乾祐二十四年九月二十日(1193年10月16日)崩,年七十,諡聖德皇帝,廟號仁宗。

\subsection{大庆}

\begin{longtable}{|>{\centering\scriptsize}m{2em}|>{\centering\scriptsize}m{1.3em}|>{\centering}m{8.8em}|}
  % \caption{秦王政}\
  \toprule
  \SimHei \normalsize 年数 & \SimHei \scriptsize 公元 & \SimHei 大事件 \tabularnewline
  % \midrule
  \endfirsthead
  \toprule
  \SimHei \normalsize 年数 & \SimHei \scriptsize 公元 & \SimHei 大事件 \tabularnewline
  \midrule
  \endhead
  \midrule
  元年 & 1140 & \tabularnewline\hline
  二年 & 1141 & \tabularnewline\hline
  三年 & 1142 & \tabularnewline\hline
  四年 & 1143 & \tabularnewline
  \bottomrule
\end{longtable}

\subsection{人庆}

\begin{longtable}{|>{\centering\scriptsize}m{2em}|>{\centering\scriptsize}m{1.3em}|>{\centering}m{8.8em}|}
  % \caption{秦王政}\
  \toprule
  \SimHei \normalsize 年数 & \SimHei \scriptsize 公元 & \SimHei 大事件 \tabularnewline
  % \midrule
  \endfirsthead
  \toprule
  \SimHei \normalsize 年数 & \SimHei \scriptsize 公元 & \SimHei 大事件 \tabularnewline
  \midrule
  \endhead
  \midrule
  元年 & 1144 & \tabularnewline\hline
  二年 & 1145 & \tabularnewline\hline
  三年 & 1146 & \tabularnewline\hline
  四年 & 1147 & \tabularnewline\hline
  五年 & 1148 & \tabularnewline
  \bottomrule
\end{longtable}

\subsection{天盛}

\begin{longtable}{|>{\centering\scriptsize}m{2em}|>{\centering\scriptsize}m{1.3em}|>{\centering}m{8.8em}|}
  % \caption{秦王政}\
  \toprule
  \SimHei \normalsize 年数 & \SimHei \scriptsize 公元 & \SimHei 大事件 \tabularnewline
  % \midrule
  \endfirsthead
  \toprule
  \SimHei \normalsize 年数 & \SimHei \scriptsize 公元 & \SimHei 大事件 \tabularnewline
  \midrule
  \endhead
  \midrule
  元年 & 1149 & \tabularnewline\hline
  二年 & 1150 & \tabularnewline\hline
  三年 & 1151 & \tabularnewline\hline
  四年 & 1152 & \tabularnewline\hline
  五年 & 1153 & \tabularnewline\hline
  六年 & 1154 & \tabularnewline\hline
  七年 & 1155 & \tabularnewline\hline
  八年 & 1156 & \tabularnewline\hline
  九年 & 1157 & \tabularnewline\hline
  十年 & 1158 & \tabularnewline\hline
  十一年 & 1159 & \tabularnewline\hline
  十二年 & 1160 & \tabularnewline\hline
  十三年 & 1161 & \tabularnewline\hline
  十四年 & 1162 & \tabularnewline\hline
  十五年 & 1163 & \tabularnewline\hline
  十六年 & 1164 & \tabularnewline\hline
  十七年 & 1165 & \tabularnewline\hline
  十八年 & 1166 & \tabularnewline\hline
  十九年 & 1167 & \tabularnewline\hline
  二十年 & 1168 & \tabularnewline\hline
  二一年 & 1169 & \tabularnewline
  \bottomrule
\end{longtable}

\subsection{乾佑}

\begin{longtable}{|>{\centering\scriptsize}m{2em}|>{\centering\scriptsize}m{1.3em}|>{\centering}m{8.8em}|}
  % \caption{秦王政}\
  \toprule
  \SimHei \normalsize 年数 & \SimHei \scriptsize 公元 & \SimHei 大事件 \tabularnewline
  % \midrule
  \endfirsthead
  \toprule
  \SimHei \normalsize 年数 & \SimHei \scriptsize 公元 & \SimHei 大事件 \tabularnewline
  \midrule
  \endhead
  \midrule
  元年 & 1170 & \tabularnewline\hline
  二年 & 1171 & \tabularnewline\hline
  三年 & 1172 & \tabularnewline\hline
  四年 & 1173 & \tabularnewline\hline
  五年 & 1174 & \tabularnewline\hline
  六年 & 1175 & \tabularnewline\hline
  七年 & 1176 & \tabularnewline\hline
  八年 & 1177 & \tabularnewline\hline
  九年 & 1178 & \tabularnewline\hline
  十年 & 1179 & \tabularnewline\hline
  十一年 & 1180 & \tabularnewline\hline
  十二年 & 1181 & \tabularnewline\hline
  十三年 & 1182 & \tabularnewline\hline
  十四年 & 1183 & \tabularnewline\hline
  十五年 & 1184 & \tabularnewline\hline
  十六年 & 1185 & \tabularnewline\hline
  十七年 & 1186 & \tabularnewline\hline
  十八年 & 1187 & \tabularnewline\hline
  十九年 & 1188 & \tabularnewline\hline
  二十年 & 1189 & \tabularnewline\hline
  二一年 & 1190 & \tabularnewline\hline
  二二年 & 1191 & \tabularnewline\hline
  二三年 & 1192 & \tabularnewline\hline
  二四年 & 1193 & \tabularnewline
  \bottomrule
\end{longtable}


%%% Local Variables:
%%% mode: latex
%%% TeX-engine: xetex
%%% TeX-master: "../Main"
%%% End:

%% -*- coding: utf-8 -*-
%% Time-stamp: <Chen Wang: 2019-12-26 11:10:54>

\section{桓宗\tiny(1193-1206)}

\subsection{生平}

夏桓宗李純佑(1177年-1206年3月),夏仁宗子。性溫和厚實。1193年10月16日,夏仁宗李仁孝去世,李純佑即位,時年十七,改元天慶。桓宗基本上還能奉行仁宗时期的政治方針和外交政策,對内安國養民,對外附金和宋。但隨着國家的安定承平日久,統治階層開始貪圖安逸,日益腐朽墮落,从此西夏政治日益腐败,国势衰落。從盛到衰已成為西夏社會不可逆轉的趨勢。同时,桓宗統治時期,正是蒙古崛起並日漸強大的时期,來自蒙古的嚴重威脅加速了西夏由盛而衰的歷史進程。1205年,改興慶府名為中興府,取夏國中興之意,但同年蒙古第一次進攻西夏,自此迄無寧日了。1206年3月1日,李安全發動政變,桓宗被廢。不久暴卒,年僅三十。諡昭簡皇帝。

\subsection{天庆}

\begin{longtable}{|>{\centering\scriptsize}m{2em}|>{\centering\scriptsize}m{1.3em}|>{\centering}m{8.8em}|}
  % \caption{秦王政}\
  \toprule
  \SimHei \normalsize 年数 & \SimHei \scriptsize 公元 & \SimHei 大事件 \tabularnewline
  % \midrule
  \endfirsthead
  \toprule
  \SimHei \normalsize 年数 & \SimHei \scriptsize 公元 & \SimHei 大事件 \tabularnewline
  \midrule
  \endhead
  \midrule
  元年 & 1194 & \tabularnewline\hline
  二年 & 1195 & \tabularnewline\hline
  三年 & 1196 & \tabularnewline\hline
  四年 & 1197 & \tabularnewline\hline
  五年 & 1198 & \tabularnewline\hline
  六年 & 1199 & \tabularnewline\hline
  七年 & 1200 & \tabularnewline\hline
  八年 & 1201 & \tabularnewline\hline
  九年 & 1202 & \tabularnewline\hline
  十年 & 1203 & \tabularnewline\hline
  十一年 & 1204 & \tabularnewline\hline
  十二年 & 1205 & \tabularnewline\hline
  十三年 & 1206 & \tabularnewline
  \bottomrule
\end{longtable}


%%% Local Variables:
%%% mode: latex
%%% TeX-engine: xetex
%%% TeX-master: "../Main"
%%% End:

%% -*- coding: utf-8 -*-
%% Time-stamp: <Chen Wang: 2019-12-26 11:11:10>

\section{襄宗\tiny(1206-1211)}

\subsection{生平}

夏襄宗李安全(1170年-1211年9月13日),夏崇宗孫,其父乃夏仁宗弟越王李仁友,1196年,仁友逝,安全上書要求襲越王爵位,桓宗不許,安全被降封為鎮夷郡王,他極為不滿,於是萌生了篡奪皇位之心。1206年3月1日與桓宗母羅氏合謀,廢桓宗自立,改元應天。在位時昏庸無能,破壞金國與西夏長期的友好關係,發兵侵金,為後來一場場令夏金耗盡精兵的戰役掀起序幕,改附不斷強大起來的蒙古帝国,但這一切都沒有為西夏帶來利益和跟蒙古之友好,蒙古也以西夏作為侵略目標,西夏不斷積弱。1211年8月12日,宗室齊王李遵頊發動政變,被廢,並於一個月後不明不白地死去,去世于1211年9月13日,終年四十二。諡敬慕皇帝,廟號襄宗。

\subsection{应天}

\begin{longtable}{|>{\centering\scriptsize}m{2em}|>{\centering\scriptsize}m{1.3em}|>{\centering}m{8.8em}|}
  % \caption{秦王政}\
  \toprule
  \SimHei \normalsize 年数 & \SimHei \scriptsize 公元 & \SimHei 大事件 \tabularnewline
  % \midrule
  \endfirsthead
  \toprule
  \SimHei \normalsize 年数 & \SimHei \scriptsize 公元 & \SimHei 大事件 \tabularnewline
  \midrule
  \endhead
  \midrule
  元年 & 1206 & \tabularnewline\hline
  二年 & 1207 & \tabularnewline\hline
  三年 & 1208 & \tabularnewline\hline
  四年 & 1209 & \tabularnewline
  \bottomrule
\end{longtable}

\subsection{皇建}

\begin{longtable}{|>{\centering\scriptsize}m{2em}|>{\centering\scriptsize}m{1.3em}|>{\centering}m{8.8em}|}
  % \caption{秦王政}\
  \toprule
  \SimHei \normalsize 年数 & \SimHei \scriptsize 公元 & \SimHei 大事件 \tabularnewline
  % \midrule
  \endfirsthead
  \toprule
  \SimHei \normalsize 年数 & \SimHei \scriptsize 公元 & \SimHei 大事件 \tabularnewline
  \midrule
  \endhead
  \midrule
  元年 & 1210 & \tabularnewline\hline
  二年 & 1211 & \tabularnewline
  \bottomrule
\end{longtable}


%%% Local Variables:
%%% mode: latex
%%% TeX-engine: xetex
%%% TeX-master: "../Main"
%%% End:

%% -*- coding: utf-8 -*-
%% Time-stamp: <Chen Wang: 2019-12-26 11:11:27>

\section{神宗\tiny(1211-1223)}

\subsection{生平}

夏神宗李遵頊(1163年-1226年),西夏宗室齐国忠武王李彥宗之子。史書記載:「端重明粹,少力學,長博通群書,工隸篆」。1203年,廷试进士第一名,1211年8月12日廢襄宗自立,改元光定,成为中国历史上唯一的状元皇帝。任内全盤承襲襄宗自取滅亡的政策,繼續破壞金國與西夏關係,發兵侵金;金宣宗也不遑多讓,決定痛擊西夏。更不幸的是,夏軍軍力早已廢弛,因此不斷戰敗,反而沒有令遵頊知難而退,更激起他的野心和戰慾,繼續發動戰爭,令人民家破人亡,民怨四起,經濟嚴重破壞,國力直線下降。在位期間根本沒有想過與金國議和,雖不斷有忠良之士直言進諫,但一一被他痛罵,包括其子李德任。1223年,傳位於子李德旺,為西夏唯一的太上皇。1226年病卒,年64,諡英文皇帝,廟號神宗。

\subsection{光定}

\begin{longtable}{|>{\centering\scriptsize}m{2em}|>{\centering\scriptsize}m{1.3em}|>{\centering}m{8.8em}|}
  % \caption{秦王政}\
  \toprule
  \SimHei \normalsize 年数 & \SimHei \scriptsize 公元 & \SimHei 大事件 \tabularnewline
  % \midrule
  \endfirsthead
  \toprule
  \SimHei \normalsize 年数 & \SimHei \scriptsize 公元 & \SimHei 大事件 \tabularnewline
  \midrule
  \endhead
  \midrule
  元年 & 1211 & \tabularnewline\hline
  二年 & 1212 & \tabularnewline\hline
  三年 & 1213 & \tabularnewline\hline
  四年 & 1214 & \tabularnewline\hline
  五年 & 1215 & \tabularnewline\hline
  六年 & 1216 & \tabularnewline\hline
  七年 & 1217 & \tabularnewline\hline
  八年 & 1218 & \tabularnewline\hline
  九年 & 1219 & \tabularnewline\hline
  十年 & 1220 & \tabularnewline\hline
  十一年 & 1221 & \tabularnewline\hline
  十二年 & 1222 & \tabularnewline\hline
  十三年 & 1223 & \tabularnewline
  \bottomrule
\end{longtable}


%%% Local Variables:
%%% mode: latex
%%% TeX-engine: xetex
%%% TeX-master: "../Main"
%%% End:

%% -*- coding: utf-8 -*-
%% Time-stamp: <Chen Wang: 2021-11-01 16:14:48>

\section{献宗李德旺\tiny(1223-1226)}

\subsection{生平}

夏獻宗李德旺(1181年-1226年),夏神宗之次子。力挽面臨滅亡的西夏,但西夏經過襄宗、神宗兩朝的亡國政策,人民早已生活於水深火熱之困境中,經濟疲弊,他根本沒有回天之力。他一改前朝政策,決心與金國修好,於1225年正式和好;然金都也被蒙古包圍,金國已是泥菩薩過江,自身難保。又改國策附蒙為抗蒙,但西夏精兵早於夏金戰役中消耗殆盡,无力抵抗蒙古军,終於1226年驚憂而死,年46岁,廟號獻宗。


\subsection{乾定}

\begin{longtable}{|>{\centering\scriptsize}m{2em}|>{\centering\scriptsize}m{1.3em}|>{\centering}m{8.8em}|}
  % \caption{秦王政}\
  \toprule
  \SimHei \normalsize 年数 & \SimHei \scriptsize 公元 & \SimHei 大事件 \tabularnewline
  % \midrule
  \endfirsthead
  \toprule
  \SimHei \normalsize 年数 & \SimHei \scriptsize 公元 & \SimHei 大事件 \tabularnewline
  \midrule
  \endhead
  \midrule
  元年 & 1223 & \tabularnewline\hline
  二年 & 1224 & \tabularnewline\hline
  三年 & 1225 & \tabularnewline\hline
  四年 & 1226 & \tabularnewline
  \bottomrule
\end{longtable}


%%% Local Variables:
%%% mode: latex
%%% TeX-engine: xetex
%%% TeX-master: "../Main"
%%% End:

%% -*- coding: utf-8 -*-
%% Time-stamp: <Chen Wang: 2019-12-26 11:12:13>

\section{末帝\tiny(1226-1227)}

\subsection{生平}
李睍(?-1227年8月28日),夏献宗侄,父清平郡王,夏獻宗病危時被推舉為帝,史稱「末帝」。

李睍在位時已為西夏滅亡的前夕,曾拒降蒙古。右丞相高良惠及各將士積極抵抗蒙古,但無奈天意造化弄人,中興府發生大地震,以致瘟疫肆虐,糧水短缺,軍民死傷過半,西夏注定滅亡了。最後於1227年农历六月投降蒙古,1227年8月25日,成吉思汗病故,蒙軍恐夏主有變,李睍开城投降后,按照成吉思汗遗嘱,在成吉思汗去世三天后,也就是寶義二年七月十五日(1227年8月28日),李睍被杀。西夏滅亡後,蒙古軍商议屠城,最終在出身党項族的蒙古軍將领察罕的极力勸諫下使中興府內的百姓避免了被屠殺的命运,察罕随后入城安撫城内軍民,使西夏遗民得以保全。


\subsection{宝义}

\begin{longtable}{|>{\centering\scriptsize}m{2em}|>{\centering\scriptsize}m{1.3em}|>{\centering}m{8.8em}|}
  % \caption{秦王政}\
  \toprule
  \SimHei \normalsize 年数 & \SimHei \scriptsize 公元 & \SimHei 大事件 \tabularnewline
  % \midrule
  \endfirsthead
  \toprule
  \SimHei \normalsize 年数 & \SimHei \scriptsize 公元 & \SimHei 大事件 \tabularnewline
  \midrule
  \endhead
  \midrule
  元年 & 1226 & \tabularnewline\hline
  二年 & 1227 & \tabularnewline
  \bottomrule
\end{longtable}


%%% Local Variables:
%%% mode: latex
%%% TeX-engine: xetex
%%% TeX-master: "../Main"
%%% End:



%%% Local Variables:
%%% mode: latex
%%% TeX-engine: xetex
%%% TeX-master: "../Main"
%%% End:
 % 西夏
% %% -*- coding: utf-8 -*-
%% Time-stamp: <Chen Wang: 2019-10-18 14:55:09>

\chapter{金\tiny(1115-1234)}

%% -*- coding: utf-8 -*-
%% Time-stamp: <Chen Wang: 2019-10-18 15:09:20>

\section{太祖\tiny(1115-1123)}

金太祖完顏阿骨打(1068年8月1日-1123年9月19日),漢名完顏旻,金朝開國皇帝(1115年1月28日—1123年9月19日在位)。按出虎水(今黑龍江省哈爾濱東南阿什河)女真族完顏部酋長烏骨迺之孫,劾里鉢之次子,完顏部首領。善騎射,力大過人。在位9年,終年56歲。

祖父是生女真完顔部的族長烏古廼(景祖)、父劾里鉢是烏古廼的次子。阿骨打是劾里鉢的次子)。生母是女真挐懶部首長的女兒翼簡皇后。

遼國天慶三年(1113年)十月,其兄烏雅束死,繼位女真各部落聯盟長,稱都勃極烈。天慶四年,率2500人起兵叛遼,破寧江州(今吉林省扶餘市東南)。蕭嗣先率7000精兵集結於出河店,阿骨打率兵3700乘夜奔襲,渡混同江(今松花江),大敗遼軍。天慶五年農歷正月初一(1115年1月28日),阿骨打在會寧(今黑龍江省哈爾濱市阿城區南白城)稱帝,建立大金,年號收國,改名完顏旻。天慶五年九月,攻佔黃龍府(今吉林省農安縣)城。

天輔三年(1119年),遼天祚帝冊封完顏旻為東懷國皇帝,但冊文不稱完顏旻為兄長、國號不稱大金,故他不接受冊封,繼續攻打遼國。

天輔四年(1120年),與宋朝訂攻遼計劃,攻陷遼上京臨潢府(今內蒙古自治區巴林左旗南)。天輔六年(1122年),取遼中京(今內蒙古自治區寧城縣西);是年年底,攻陷燕京(今北京市)。天輔七年(1123年)八月,返金上京(今黑龍江省哈爾濱市阿城區附近)途中病逝。他死後,在天會三年六月上諡號大聖皇帝,同年十二月改為大聖武元皇帝,廟號是太祖。皇統五年十月,增諡為應乾興運昭德定功仁明莊孝大聖武元皇帝。

2003年9月5日,北京市政府文物局發表:1980年代在北京市西南郊外的九龍山的金朝陵墓,證實是完顏阿骨打的石棺、遺骨及裝飾物。

阿骨打痛恨遼,但對宋相當和善,在建國之初就有意與宋聯合,和後來諸代金朝帝王對宋朝充滿敵對大不相同。《靖康稗史箋證》中記錄其二子完顏宗望曾說過:「太祖止我伐宋,言猶在耳」。 當宋以「海上之盟」求燕京(今北京西南)及西京(今山西大同)地,金國大臣左企弓(張覺叛金時被殺)曾勸阿骨打不要歸還「燕雲十六州」,但阿骨打還是如約歸還了「燕雲十六州」中的燕京、涿州、易州、檀州、順州、景州、薊州。其中景州雖在長城之內,但並不屬於石敬瑭割給遼的燕雲十六州之一。易州是遼統和七年(989年)夺自宋,也不算作十六州之一。莫、瀛兩州早已收復,為北宋河間府所治。這樣一來,山西、河北太行山(後明在此建內長城)以內的燕、涿、檀、順、薊、莫、瀛七州都已經歸還宋,而太行山以外的儒、媯、武、新、蔚、應、寰、朔、雲九州當時遼金尚在爭奪,金太祖也無法歸還。

和阿骨打生前相處時間較長的幾個年長兒子,如長子完顏宗幹、二子完顏宗望、四子完顏宗弼都很崇尚漢文化,這對以後金國的漢化影響很大。這也從另一個側面反映了阿骨打的喜好。

元朝官修正史《金史》脱脱等的評價是:“太祖英谟睿略,豁达大度,知人善任,人乐为用。世祖阴有取辽之志,是以兄弟相授,传及康宗,遂及太祖。临终以太祖属穆宗,其素志盖如是也。初定东京,即除去辽法,减省租税,用本国制度。辽主播越,宋纳岁币,以幽、蓟、武、朔等州与宋,而置南京于平州。宋人终不能守燕、代,卒之辽主见获,宋主被执。虽功成于天会间,而规摹运为宾自此始。金有天下百十有九年,太祖数年之间算无遗策,兵无留行,底定大业,传之子孙。嗚呼,雄哉!”

\subsection{收国}


\begin{longtable}{|>{\centering\scriptsize}m{2em}|>{\centering\scriptsize}m{1.3em}|>{\centering}m{8.8em}|}
  % \caption{秦王政}\
  \toprule
  \SimHei \normalsize 年数 & \SimHei \scriptsize 公元 & \SimHei 大事件 \tabularnewline
  % \midrule
  \endfirsthead
  \toprule
  \SimHei \normalsize 年数 & \SimHei \scriptsize 公元 & \SimHei 大事件 \tabularnewline
  \midrule
  \endhead
  \midrule
  元年 & 1115 & \tabularnewline\hline
  二年 & 1116 & \tabularnewline
  \bottomrule
\end{longtable}

\subsection{天辅}

\begin{longtable}{|>{\centering\scriptsize}m{2em}|>{\centering\scriptsize}m{1.3em}|>{\centering}m{8.8em}|}
  % \caption{秦王政}\
  \toprule
  \SimHei \normalsize 年数 & \SimHei \scriptsize 公元 & \SimHei 大事件 \tabularnewline
  % \midrule
  \endfirsthead
  \toprule
  \SimHei \normalsize 年数 & \SimHei \scriptsize 公元 & \SimHei 大事件 \tabularnewline
  \midrule
  \endhead
  \midrule
  元年 & 1117 & \tabularnewline\hline
  二年 & 1118 & \tabularnewline\hline
  三年 & 1119 & \tabularnewline\hline
  四年 & 1120 & \tabularnewline\hline
  五年 & 1121 & \tabularnewline\hline
  六年 & 1122 & \tabularnewline\hline
  七年 & 1123 & \tabularnewline
  \bottomrule
\end{longtable}


%%% Local Variables:
%%% mode: latex
%%% TeX-engine: xetex
%%% TeX-master: "../Main"
%%% End:

%% -*- coding: utf-8 -*-
%% Time-stamp: <Chen Wang: 2019-10-18 15:19:07>

\section{太宗\tiny(1123-1135)}

金太宗完顏晟(1075年11月25日-1135年2月9日),金朝第二位皇帝(1123年9月27日—1135年2月9日在位)。女真名吳乞買,金太祖之弟,身材魁梧,力大無比,能親手搏熊刺虎。在位12年,终年61岁。先后滅遼朝及北宋。

完颜吴乞买出生于1075年11月25日。天會三年二月二十日(1125年3月26日),辽天祚帝在应州被金朝将领完颜娄室等所俘,八月被解送金上京,被降为海滨王,辽朝灭亡。

天會三年(1125年)十月,发动宋金战争,令諳班勃極烈完颜斜也為都元帥,統領金軍,兵分東、西兩路,逼進北宋首都汴京,由於李綱頑強抵抗,金兵一時不能得逞,雙方訂「城下之盟」。天會四年(1126年)八月,經過半年的休整,金太宗再次命宗望、宗翰兩路軍大舉南侵,汴京再度被包圍,破郭京「六甲法」,汴京城陷。天會五年二月初六(1127年3月20日),金太宗下詔廢徽、欽二帝,貶為庶人,俘虏二帝北上,并携带掠夺来的大量财宝和皇室大臣宫女等15000人,北宋滅亡。天會六年(1128年)八月二十四日,吳乞買封宋徽宗為昏德公,宋欽宗為重昏侯,移遷五國城(今黑龍江省依蘭縣城北舊古城)。

他在位时期创建了各种典章制度,奠定金代经国规模,晚年改变兄终弟及的旧制,立太祖孙完颜亶(金熙宗)为继承人。

天會十三年正月二十五日(1135年2月9日),太宗病死於明德宮,終年六十一歲。遺體葬和陵。其后代全被海陵王完颜亮所杀,海陵王遷都後,改葬於大房山,稱金恭陵。

他死後,於天會十三年三月七日上諡號文烈皇帝,廟號太宗。皇統五年閏十一月增諡体元应运世德昭功哲惠仁圣文烈皇帝。

吴乞买與宋太祖的畫像神似,民間相傳宋太宗當年殺太祖奪位,甚至還說吳乞買是宋太祖投胎來報仇,滅了宋太宗一家,宋高宗為了統治的正統性,寧可把帝位傳回宋太祖一脈,於是以太祖後代趙眘為養子,禪以帝位。

元朝官修正史《金史》脱脱等的評價是:“天辅草创,未遑礼乐之事。太宗以斜也、宗干知国政,以宗翰、宗望总戎事。既灭辽举宋,即议礼制度,治历明时,缵以武功,述以文事,经国规摹,至是始定。在位十三年,宫室苑籞无所增益。末听大臣计,传位熙宗,使太祖世嗣不失正绪,可谓行其所甚难矣!”

\subsection{天会}


\begin{longtable}{|>{\centering\scriptsize}m{2em}|>{\centering\scriptsize}m{1.3em}|>{\centering}m{8.8em}|}
  % \caption{秦王政}\
  \toprule
  \SimHei \normalsize 年数 & \SimHei \scriptsize 公元 & \SimHei 大事件 \tabularnewline
  % \midrule
  \endfirsthead
  \toprule
  \SimHei \normalsize 年数 & \SimHei \scriptsize 公元 & \SimHei 大事件 \tabularnewline
  \midrule
  \endhead
  \midrule
  元年 & 1123 & \tabularnewline\hline
  二年 & 1124 & \tabularnewline\hline
  三年 & 1125 & \tabularnewline\hline
  四年 & 1126 & \tabularnewline\hline
  五年 & 1127 & \tabularnewline\hline
  六年 & 1128 & \tabularnewline\hline
  七年 & 1129 & \tabularnewline\hline
  八年 & 1130 & \tabularnewline\hline
  九年 & 1131 & \tabularnewline\hline
  十年 & 1132 & \tabularnewline\hline
  十一年 & 1133 & \tabularnewline\hline
  十二年 & 1134 & \tabularnewline\hline
  十三年 & 1135 & \tabularnewline\hline
  十四年 & 1136 & \tabularnewline\hline
  十五年 & 1137 & \tabularnewline
  \bottomrule
\end{longtable}


%%% Local Variables:
%%% mode: latex
%%% TeX-engine: xetex
%%% TeX-master: "../Main"
%%% End:

%% -*- coding: utf-8 -*-
%% Time-stamp: <Chen Wang: 2019-10-18 15:19:57>

\section{熙宗\tiny(1135-1149)}

金熙宗完顏亶(1119年8月14日-1150年1月9日),金朝第三位皇帝(1135年2月10日—1150年1月9日在位)。女真名合剌,漢名亶,是金太祖完顏阿骨打之嫡長孫,父為太祖嫡長子完顏宗峻、母為蒲察氏。生于天輔三年七月七日(1119年8月14日),卒于皇統九年十二月九日(1150年1月9日)。在位15年,终年31岁。

天輔三年(己亥年)出生,本名合剌,母親是蒲察氏,父親是完顏阿骨打的嫡長子。

天會八年,諳班勃極烈完顏杲薨逝,金太宗意久未決。天會十年,左副元帥完顏宗翰、右副元帥完顏宗輔、左監軍完顏希尹等大臣進入朝廷與完顏宗幹討論國事,稱:「諳班勃極烈虛位已久,今不早定,恐授非其人。合剌,先帝嫡孫,當立。」相與請於太宗者再三,乃從之。

天會十三年正月己巳,金太宗駕崩。

1135年2月10日,即皇帝位。不久,對外公佈、並下令公私部門皆禁止飲酒與相關娛樂,並向偽齊、高麗、夏等國派遣使節稱金朝皇帝已經即位;並詔令劉齊今後稱自己為臣,不能稱子。

天會十五年(1137年)十一月丙午,为鞏固政权,金熙宗下詔廢除伪齐,降封劉豫為蜀王,並与南宋議和。十二月戊辰,劉豫上表感謝封爵。不久,發佈詔令改明年爲天眷元年,並大赦,命韓昉、耶律紹文等人編修國史。之後命令蜀王劉豫遷徙至臨潢府。

天眷二年(1139年)正月,金、宋議和成立,南宋代替偽齊政權成為金的屬國,宋對金稱臣,金朝歸還河南、陝西。但是主戰派很快占了上風。天眷三年(1140年)五月,金熙宗詔令兀朮收復河南、陝西等地。

皇統元年(1141年),完顏宗弼再次帶兵南侵,被岳飛、韓世忠等擊退,但宋高宗急於求和,再次達成紹興和議,金朝至此控制淮河以北。

皇统五年(1145年),取消辽东汉人、渤海猛安谋克世袭的制度,逐渐将兵权转移到女真人手中,分猛安谋克为上中下三等,宗室为上等,其余次之。

熙宗廢除了太祖、太宗傳下來的勃極烈制度。完顏阿骨打庶長子完顏宗幹(同時也是熙宗的養父)崇尚漢化,在開國之初太宗任命宗幹輔助朝政制定各種制度,為女真漢化及鞏固金在华北統治打下基礎。

熙宗自幼接受漢化教育,加上養父的影響,登基後開始了漢制改革、重用漢人。太祖四子完顏宗弼(又名金兀朮)是推動漢制的重臣,熙宗授以軍政大權。天會十四年(1136年),宗磐、宗幹和宗翰三人共同總管政府機構,「並領三省事」。金朝官制此時基本漢化,建立了以尚書省為中心的三省制,以三師(太師、太傅、太保)以及三公(太尉、司徒、司空)領三省事。

勃極烈制度廢除前,女真的傳統一般是同代相傳,比如景祖烏古迺將權力傳給世祖劾里缽,然後是劾里缽的四弟肅宗頗剌淑和五弟穆宗盈歌(長子劾者和三子劾孫因為柔善而被景袓跳過),這一輪過後才是最有勢力家族的下一代,世祖劾里缽之子康宗烏雅束、太祖阿骨打、太宗吳乞買和遼王斜也。斜也一死,太宗把皇儲諳班勃極烈的位置空閒了兩年,在大家的催促下才選了一個太祖阿骨打家族的嫡長孫作皇儲。

等到熙宗繼位後,漢化的結果就是廢除了諳班勃極烈這種舊的皇儲制度,皇帝立自己的兒子作太子。這引起了本來能在太宗朝成為太子的太宗長子完顏宗磐的不滿。为免出现宋朝太祖太宗朝纷爭局面,熙宗因此對太宗子孫比較忍讓。後來宗磐還是發動了叛亂,但被平息。

宋金议和以後,宗翰、宗幹、宗弼等太祖太宗朝的老功臣相繼秉政,熙宗臨朝一般不说话。等到皇統(1148年)十月,宗弼去世,熙宗才有機會親政。但悼平皇后裴滿氏又很潑辣,干預政事,無所忌憚。加上熙宗的兩個年幼兒子,太子濟安、魏王道濟相繼在皇統三、四年去世,帝位失嗣。熙宗便徹底崩潰,開始嗜酒如命,不理朝政,濫殺無辜,更杀死了蒙古族的俺巴孩汗,朝野人心惶惶。皇統九年(1149)十月,熙宗弟族完顏元、完顏阿愣等人因受海陵王完顏亮誣告而被熙宗全數殺害,熙宗因此被孤立,也給完顏亮日後的篡位埋下了禍根。

皇統九年十二月初九丁巳日(儒略曆1150年1月9日),被右丞相海陵王完顏亮所殺,終年31歲。

天德二年(1150年)二月庚戌,被海陵王降為東昏王,葬於皇后裴滿氏墓中。貞元三年(1155年),改葬於大房山蓼香甸諸王墓群。海陵王死後,金世宗於大定元年(1161年)十一月恢復完顏亶帝號,追諡武靈皇帝,廟號閔宗,墓稱思陵。大定十九年(1179年)四月,升祔於太廟,增諡弘基纘武莊靖孝成皇帝。大定二十七年(1187年)二月,改廟號熙宗。大定二十八年(1188年),以思陵狹小,改葬於峨眉谷,仍號思陵。

元朝官修正史《金史》脱脱等的評價是:“熙宗之时,四方无事,敬礼宗室大臣,委以国政,其继体守文之治,有足观者。末年酗酒妄杀,人怀危惧。所谓前有谗而不见,后有贼而不知。驯致其祸,非一朝一夕故也。”

\subsection{天眷}


\begin{longtable}{|>{\centering\scriptsize}m{2em}|>{\centering\scriptsize}m{1.3em}|>{\centering}m{8.8em}|}
  % \caption{秦王政}\
  \toprule
  \SimHei \normalsize 年数 & \SimHei \scriptsize 公元 & \SimHei 大事件 \tabularnewline
  % \midrule
  \endfirsthead
  \toprule
  \SimHei \normalsize 年数 & \SimHei \scriptsize 公元 & \SimHei 大事件 \tabularnewline
  \midrule
  \endhead
  \midrule
  元年 & 1138 & \tabularnewline\hline
  二年 & 1139 & \tabularnewline\hline
  三年 & 1140 & \tabularnewline
  \bottomrule
\end{longtable}

\subsection{皇统}

\begin{longtable}{|>{\centering\scriptsize}m{2em}|>{\centering\scriptsize}m{1.3em}|>{\centering}m{8.8em}|}
  % \caption{秦王政}\
  \toprule
  \SimHei \normalsize 年数 & \SimHei \scriptsize 公元 & \SimHei 大事件 \tabularnewline
  % \midrule
  \endfirsthead
  \toprule
  \SimHei \normalsize 年数 & \SimHei \scriptsize 公元 & \SimHei 大事件 \tabularnewline
  \midrule
  \endhead
  \midrule
  元年 & 1141 & \tabularnewline\hline
  二年 & 1142 & \tabularnewline\hline
  三年 & 1143 & \tabularnewline\hline
  四年 & 1144 & \tabularnewline\hline
  五年 & 1145 & \tabularnewline\hline
  六年 & 1146 & \tabularnewline\hline
  七年 & 1147 & \tabularnewline\hline
  八年 & 1148 & \tabularnewline\hline
  九年 & 1149 & \tabularnewline
  \bottomrule
\end{longtable}


%%% Local Variables:
%%% mode: latex
%%% TeX-engine: xetex
%%% TeX-master: "../Main"
%%% End:

%% -*- coding: utf-8 -*-
%% Time-stamp: <Chen Wang: 2019-10-18 15:21:40>

\section{完颜亮\tiny(1150-1161)}

完顏亮(1122年2月24日-1161年12月15日),字元功,女真名迪古乃,金朝第四代皇帝(1150年1月9日-1161年12月15日),金太祖阿骨打之孙,太祖庶长子遼王完顏宗幹第二子,母大氏。

完顏亮弒金熙宗而篡位,任內遷都燕京(今北京),把金朝的政治中心遷至華北,逐步汉化,使北京自此逐漸成為中國的政治中心。因伐南宋的采石大战失利,被部下所殺。完颜亮在位12年,终年40岁。完颜亮死後,继位的金世宗将他追貶為庶人,史称海陵煬王、海陵庶人、金废帝。

完颜亮生于1122年2月24日(天辅六年正月十六丙子日天眷三年)。(1140年)十八歲時以宗室子為奉國上將軍,赴梁王完顏宗弼(兀朮)幕府任使,管理萬人,遷驃騎上將軍。皇統四年(1144年),加龍虎衛上將軍,為金國中京(位於今北京市一帶)留守,遷光祿大夫。

皇統七年(1147年)五月,召入當時的金國首都上京(今黑龍江省哈尔滨市阿城区內)為同判大宗正事,加特進。十一月,拜尚書省左丞,把持了權柄,安插自己的心腹擔任要職,其中蕭裕 成為兵部侍郎。十一月某日和熙宗談話時,談到金太祖創業艱難,完顏亮痛哭流涕,熙宗認為他很忠心。後來升職加快。第二年(1148年)六月,拜平章事。十一月,拜右丞相。1149年正月,兼都元帥。三月,拜太保、領三省事,更加八面玲瓏,和有權勢家族來往密切,結其歡心。

1149年熙宗對完顏亮突然膨脹的勢力不滿。正月,熙宗派寢殿小底大興國以宋名臣司馬光畫像及其它珍玩賜完顏亮生日禮物,悼平皇后裴滿氏也附賜禮物,結果引起熙宗不悅,罰小底大興國一百杖,追回其賜物,完顏亮知道後由此不安。四月,學士張鈞起草詔書時擅自改動,被查出處死。熙宗問是誰指使的,左丞相完顏宗賢回答說是太保完顏亮。熙宗不悅,遂貶完顏亮到汴京(今河南開封),領行台尚書省事。完顏亮路過中京時,和那裡的兵部侍郎蕭裕密謀定約而去。走到良鄉,又被熙宗召還。完顏亮不知熙宗的意圖,非常恐懼。回到上京,又恢復為平章政事。但完顏亮反意已決。

《金史》說完顏亮“為人僄急,多猜忌,殘忍任數。”當熙宗以太祖的嫡孫身份嗣位時,完顏亮認為自己是太祖長子完顏宗幹的兒子,也是太祖的孫子,所以對皇位“遂懷覬覦。”早在皇統七年(1147年),熙宗就開始胡亂發脾氣殺人,比如賜宴時因為一些小事濫殺無辜,引起朝臣的不滿。皇統八年(1148年)七月,以駙馬尚書左丞唐括辯奉職不謹,杖之。皇統九年(1149年)八月,杖平章政事完顏秉德。對熙宗不滿的人即有廢立的想法,唐括辯、秉德先和大理卿烏帶(完顏言)謀劃廢掉熙宗,而烏帶就此引入完顏亮。完顏亮與唐括辯密謀廢立,問到若廢熙宗,可以立誰繼位?唐括辯與秉德初意並不在完顏亮。唐括辯說胙王完顏常勝(完顏元)似乎可以。完顏亮再問其次是誰,唐括辯說鄧王完顏奭之子完顏阿楞可以。完顏亮反駁說阿楞不行。唐括辯反問:“公豈有意邪?”完顏亮說:“果不得已,舍我其誰!”不久完顏亮和唐括辯等旦夕密謀,引起了護衛將軍完顏特思的懷疑。特思告訴了悼平皇后裴滿氏,因此熙宗得知。熙宗發怒召唐括辯並杖之。完顏亮因此非常忌諱完顏元、完顏阿楞,並且極其討厭完顏特思。

正好當時河南有士兵孫進冒稱皇弟按察大王,而熙宗之弟只有完顏元和完顏查剌。熙宗懷疑是完顏元,派完顏特思調查,卻甚麼也沒有。完顏亮乘機誣陷,對熙宗說:“孫進反有端,不稱他人,乃稱皇弟大王。陛下弟止有常勝、查刺。特思鞫不以實,故出之矣。”熙宗以為然,派唐括辯、蕭肄拷問完顏特思,完顏特思被逼招認,完顏元於是獲罪。十月,殺完顏元,一併連完顏查刺、完顏特思、完顏阿楞以及阿楞弟完顏撻楞 一起殺掉。這樣一來,熙宗殺光了自己的親兄弟,更加孤立。

到了皇統九年(1149年)十二月,要廢熙宗的人已經結黨行事。從前因送禮一事被杖責一百的大興國,因為和完顏亮的心腹尚書省令史李老僧是親戚,於是和完顏亮結黨,當時正在伺候熙宗在寢殿內的起居生活,總是有意無意地乘夜從主事者那裡帶皇宮鑰匙回家,大家習以為常。護衛十人長僕散忽土要報答完顏亮之父完顏宗幹的舊恩,徒單阿里出虎是完顏亮的姻親。十二月初九丁巳日(儒略曆1150年1月9日),此二人值班之夜,大興國用皇宮鑰匙打開所有宮門,和完顏亮、秉德、唐括辯、烏帶、徒單貞、李老僧 至寢殿。熙宗本來常置佩刀於床上,這天夜裡大興國先取之放到床下,等到事發,熙宗求佩刀不得,遂遇弒。眾人拜完顏亮為皇帝。改皇統九年為天德元年。並假稱熙宗想要商議立皇后事宜,召眾大臣入宮,殺曹國王完顏宗敏、左丞相完顏宗賢。

贞元元年三月二十六日(1153年4月21日),完颜亮正式迁都,改燕京为中都,定名为中都大兴府,同時定北宋故都開封府為金南京,使金朝逐步汉化。

海陵王在位期間不但擴大皇帝權威,甚至於濫用權力,誅殺大臣;而且海陵王的宮廷生活相當荒淫,史載「營南京(燕京)宮殿,運一木之費至二千萬,率一車之力至五百人。宮殿之飾,遍傅黃金而後間以五彩,金屑飛空如落雪。一殿之費以億萬計,成而復毀,務極華麗。」(《金史》)。據說他讀罷柳永的《望海潮》一詞:「東南形勝,三吳都會,錢塘自古繁華……有三秋桂子,十里荷花」,「遂起投鞭渡江、立馬吳山之志」,即興題詩稱:“万里车书一混同,江南岂有别疆封? 提兵百万西湖侧,立马吴山第一峰。”(《鶴林玉露》卷一) 完顏亮曾大顏不慚地說:「吾有三志,國家大事,皆我所出,一也;帥師伐遠,執其君長問罪於前,二也;得天下絕色而妻之,三也。」而被他收入深宮而「妻之」的「天下絕色」,竟有他的堂姐妹、叔母 、舅母、外甥女、侄女以及弟媳、小姨子等等。完顏亮上台後,為了壓制皇族宗室的反抗,曾大加誅戮,諸叔及其子弟幾乎屠殺殆盡,他們的妻子、女兒,或被納為嬪妃,或被強納宮中,「命諸從姊妹皆分屬諸妃,出入禁中,與為淫亂。」昭妃阿懶,就是完顏亮的親嬸嬸,完顏亮殺死叔叔曹國王宗敏,便把阿懶納入宮中,封為昭妃。

紹興三十一年(正隆六年,1161年)出兵伐宋,進迫長江。但是東京留守曹國公完顏雍杀副留守高存福,自立為帝,是为金世宗。采石大战中了南宋江淮參軍虞允文的埋伏,退兵瓜洲渡,命令所有士兵即刻南征,軍心大亂,为部下完颜元宜所弑,享年40歲。死後追貶為「海陵王」,又追貶為「海陵庶人」,被以庶人之禮安葬。

金世宗大定二年(1162年)四月,降封為海陵郡王,諡号为煬,所以又称海陵煬王,葬於大房山鹿門谷諸王的墓地中。大定二十一年(1181年)正月,由于为海陵王所弒的金熙宗于大定十九年供入太廟,完顏亮又再被降為海陵庶人,改葬于山陵西南四十里。今北京市房山区有海陵王陵。

南宋洪邁出使金世宗後歸國,向宋高宗報告完顏亮被諡為「煬」的事。宋高宗表示,當時人們都把完顏亮比作苻堅,唯獨他認為完顏亮與隋煬帝類似。宋高宗因此認為,隋煬帝和完顏亮死在同一地方,又被加上同一諡號,乃是天意。元朝官修正史《金史》脱脱等的評價是:“海陵智足以拒諫,言足以飾非。欲爲君則弑其君,欲伐國則弑其母,欲奪人之妻則使之殺其夫。三綱絕矣,何暇他論。至于屠滅宗族,剪刈忠良,婦姑姊妹盡入嬪御。方以三十二總管之兵圖一天下,卒之戾氣感召,身由惡終,使天下後世稱無道主以海陵爲首。可不戒哉!可不戒哉!”。其中,「智足以拒諫,言足以飾非」一句,是司馬遷在《史記》中對商紂王的評語原文。《醒世恆言》中有《金海陵縱慾亡身》一篇(改編自更早的話本),將海陵王描寫為淫蕩的昏君,使得海陵王的負面形象深入人心。

\subsection{天德}


\begin{longtable}{|>{\centering\scriptsize}m{2em}|>{\centering\scriptsize}m{1.3em}|>{\centering}m{8.8em}|}
  % \caption{秦王政}\
  \toprule
  \SimHei \normalsize 年数 & \SimHei \scriptsize 公元 & \SimHei 大事件 \tabularnewline
  % \midrule
  \endfirsthead
  \toprule
  \SimHei \normalsize 年数 & \SimHei \scriptsize 公元 & \SimHei 大事件 \tabularnewline
  \midrule
  \endhead
  \midrule
  元年 & 1149 & \tabularnewline\hline
  二年 & 1150 & \tabularnewline\hline
  三年 & 1151 & \tabularnewline\hline
  四年 & 1152 & \tabularnewline\hline
  五年 & 1153 & \tabularnewline
  \bottomrule
\end{longtable}

\subsection{贞元}

\begin{longtable}{|>{\centering\scriptsize}m{2em}|>{\centering\scriptsize}m{1.3em}|>{\centering}m{8.8em}|}
  % \caption{秦王政}\
  \toprule
  \SimHei \normalsize 年数 & \SimHei \scriptsize 公元 & \SimHei 大事件 \tabularnewline
  % \midrule
  \endfirsthead
  \toprule
  \SimHei \normalsize 年数 & \SimHei \scriptsize 公元 & \SimHei 大事件 \tabularnewline
  \midrule
  \endhead
  \midrule
  元年 & 1153 & \tabularnewline\hline
  二年 & 1154 & \tabularnewline\hline
  三年 & 1155 & \tabularnewline\hline
  四年 & 1156 & \tabularnewline
  \bottomrule
\end{longtable}

\subsection{正隆}

\begin{longtable}{|>{\centering\scriptsize}m{2em}|>{\centering\scriptsize}m{1.3em}|>{\centering}m{8.8em}|}
  % \caption{秦王政}\
  \toprule
  \SimHei \normalsize 年数 & \SimHei \scriptsize 公元 & \SimHei 大事件 \tabularnewline
  % \midrule
  \endfirsthead
  \toprule
  \SimHei \normalsize 年数 & \SimHei \scriptsize 公元 & \SimHei 大事件 \tabularnewline
  \midrule
  \endhead
  \midrule
  元年 & 1156 & \tabularnewline\hline
  二年 & 1157 & \tabularnewline\hline
  三年 & 1158 & \tabularnewline\hline
  四年 & 1159 & \tabularnewline\hline
  五年 & 1160 & \tabularnewline\hline
  六年 & 1161 & \tabularnewline
  \bottomrule
\end{longtable}


%%% Local Variables:
%%% mode: latex
%%% TeX-engine: xetex
%%% TeX-master: "../Main"
%%% End:

%% -*- coding: utf-8 -*-
%% Time-stamp: <Chen Wang: 2021-11-01 16:56:46>

\section{世宗完顏雍\tiny(1161-1189)}

\subsection{生平}

金世宗完顏雍(天輔七年三月初一甲寅日,儒略曆1123年3月29日—大定二十九年正月初二癸巳日,儒略曆1189年1月20日)),原名完顏褎(xiù、ㄒㄧㄡˋ),金朝第五位皇帝(1161年10月27日—1189年1月20日在位)。女真名乌禄,金太祖完颜阿骨打孙,海陵王完颜亮征宋时为辽东留守,后被拥立为帝,在位28年,终年67岁,葬于兴陵(今北京市房山区)。

1161年十月初八日,完颜亮率领大军渡过淮水,进兵南宋庐州。东京辽阳府发生了政变。曹国公完颜雍时任东京留守,完颜秉德以谋立葛王完颜雍之罪被杀后,完颜雍从海路献珍宝以表明他的忠诚。完颜亮命渤海人高存福为副留守,监视完颜雍的行动。契丹撒八等起义,完颜雍出兵阻击括里。完颜亮命婆速府路总管完颜谋衍(完颜娄室之子)领兵五千助战。完颜亮自辽东征调大批女真兵南下侵宋,女真兵多不愿南下。行至山东时,南征万户、曷苏馆女真猛安完颜福寿等领一万多人,中途叛变,逃回辽阳。完颜福寿与完颜谋衍等在辽阳发动政变,杀高存福,拥立完颜雍作皇帝,即金世宗。十月初八日,金世宗下诏废黜完颜亮,改元大定。完颜谋衍为右副元帅,福寿为右监军。十一月,在东京的政权,逐渐巩固。中都留守阿琐等起而响应金世宗。金世宗决定迁赴中都。十一月二十七日拂晓,完颜元宜率领将士袭击完颜亮营帐,完颜亮被乱箭射死。

金世宗即位后,首先对南宋的进攻保持守势,着手平息契丹起义,待平息契丹起义后,开始对南宋采取强硬态度,击退了南宋的隆兴北伐,并在形势占优时,在与宋孝宗和谈时做出让步,最终签署了《隆兴和议》,开启了双方四十余年的和平局面。

金世宗在内政管理上,励精图治,革除了完顏亮统治时期的很多弊政。更值得称道的是,金世宗十分朴素,不穿丝织龙袍,使金朝国库充盈,农民也过上富裕的日子,天下小康,实现了“大定盛世”的繁荣鼎盛局面,金世宗也被称为“小尧舜”。

金世宗统治时期,如移剌窩幹等各族人民纷纷起义,他为了维持统治,利用科举、学校等制度,争取汉人支持,又加强猛安谋克权力,扩大女真族占有的土地。同时多次发布有关保留女真人旧习、语言的诏令,甚或要求所有皇子必须有女真语名、所有女真官员必须通晓女真語,卫士不准讲汉语。

他死後谥号是光天兴运文德武功圣明仁孝皇帝,庙号是世宗。

元朝官修正史《金史》脱脱等的評價是:“世宗之立,虽由劝进,然天命人心之所归,虽古圣贤之君,亦不能辞也。盖自太祖以来,海内用兵,宁岁无几。重以海陵无道,赋役繁兴,盗贼满野,兵甲并起,万姓盼盼,国内骚然,老无留养之丁,幼无顾复之爱,颠危愁困,待尽朝夕。世宗久典外郡,明祸乱之故,知吏治之得失。即位五载,而南北讲好,与民休息。于是躬节俭,崇孝弟,信赏罚,重农桑,慎守令之选,严廉察之责,却任得敬分国之请,拒赵位宠郡县之献,孳孳为治,夜以继日,可谓得为君之道矣!当此之时,群臣守职,上下相安,家给人足,仓廪有余,刑部岁断死罪,或十七人,或二十人,号称“小尧舜”,此其效验也。然举贤之急,求言之切,不绝于训辞,而群臣偷安苟禄,不能将顺其美,以底大顺,惜哉!”

\subsection{大定}


\begin{longtable}{|>{\centering\scriptsize}m{2em}|>{\centering\scriptsize}m{1.3em}|>{\centering}m{8.8em}|}
  % \caption{秦王政}\
  \toprule
  \SimHei \normalsize 年数 & \SimHei \scriptsize 公元 & \SimHei 大事件 \tabularnewline
  % \midrule
  \endfirsthead
  \toprule
  \SimHei \normalsize 年数 & \SimHei \scriptsize 公元 & \SimHei 大事件 \tabularnewline
  \midrule
  \endhead
  \midrule
  元年 & 1161 & \tabularnewline\hline
  二年 & 1162 & \tabularnewline\hline
  三年 & 1163 & \tabularnewline\hline
  四年 & 1164 & \tabularnewline\hline
  五年 & 1165 & \tabularnewline\hline
  六年 & 1166 & \tabularnewline\hline
  七年 & 1167 & \tabularnewline\hline
  八年 & 1168 & \tabularnewline\hline
  九年 & 1169 & \tabularnewline\hline
  十年 & 1170 & \tabularnewline\hline
  十一年 & 1171 & \tabularnewline\hline
  十二年 & 1172 & \tabularnewline\hline
  十三年 & 1173 & \tabularnewline\hline
  十四年 & 1174 & \tabularnewline\hline
  十五年 & 1175 & \tabularnewline\hline
  十六年 & 1176 & \tabularnewline\hline
  十七年 & 1177 & \tabularnewline\hline
  十八年 & 1178 & \tabularnewline\hline
  十九年 & 1179 & \tabularnewline\hline
  二十年 & 1180 & \tabularnewline\hline
  二一年 & 1181 & \tabularnewline\hline
  二二年 & 1182 & \tabularnewline\hline
  二三年 & 1183 & \tabularnewline\hline
  二四年 & 1184 & \tabularnewline\hline
  二五年 & 1185 & \tabularnewline\hline
  二六年 & 1186 & \tabularnewline\hline
  二七年 & 1187 & \tabularnewline\hline
  二八年 & 1188 & \tabularnewline\hline
  二九年 & 1189 & \tabularnewline
  \bottomrule
\end{longtable}


%%% Local Variables:
%%% mode: latex
%%% TeX-engine: xetex
%%% TeX-master: "../Main"
%%% End:

%% -*- coding: utf-8 -*-
%% Time-stamp: <Chen Wang: 2019-10-18 15:23:34>

\section{章宗\tiny(1189-1208)}

金章宗完顏璟(1168年8月31日(农历七月二十七)-1208年12月29日),女真名麻達葛,金朝第6位皇帝(1189年1月20日—1208年12月29日在位),在位19年,享年41岁。章宗為金世宗完颜雍之嫡孙,其在位期間修訂國內律法,政治清明,世稱明昌之治。章宗統治下的金朝文化發展達至頂峰,但同時軍事能力卻也日益低下,蒙古帝國也於同時崛起。

南宋主戰派權臣韓侂胄於章宗年間北伐,但遭到金軍擊敗,簽定「嘉定和議」。1208年駕崩,叔衛紹王完顏永济繼位。

金世宗在大定初年立章宗之父完顏允恭為太子,允恭在大定二十五年(1185年)逝世後,世宗在次年立章宗為皇太孫。大定二十九年正月初二,世宗去世,章宗隨即繼位。

當時金朝立國七十五年,「禮樂刑政因遼、宋舊制,雜亂無貫,章宗即位,乃更定修正,為一代法。」章宗時期的政治尚算清明,後世稱為明昌之治。

章宗時代,国内的文化發展達至最高峰。他不單對國內文化發展加以獎勵,而他本身亦能寫得一手好字,與北宋徽宗的「瘦金體」形似。但與此同時,軍事能力卻日益低下,使屬國紛紛離異、並招引鄰國侵略。章宗整日与文人饮酒作诗,不思朝政。金朝日益腐朽衰败,漠北已失去控制。此外,黄河氾濫等各種天災相繼出現,使国力開始衰退。在位后期蒙古帝国崛起,成为了日后金覆灭的隐患。

1196年,原來從屬金朝的塔塔兒部叛離,改為歸順蒙古。南宋權臣韓侂冑見金朝開始走下坡,以為有機可乘,在1206年大舉出兵攻金,結果宋軍大敗,東線金兵渡過淮河,佔領淮南多個州縣;中線金兵攻襄陽;西線宋將吳曦以四川附金,不久事敗被殺。宋寧宗殺韓侂冑向金求和,1208年「嘉定和議」成,宋尊金為伯,增加每年歲幣至銀三十萬兩、絹三十萬匹及向金朝納「犒軍錢」三百萬兩,金朝始歸還南宋失地,維持紹興和議時的局面。

1208年12月29日,金章宗駕崩。他的六個兒子都在三歲前夭折。由於他沒有後嗣,所以由叔父衛紹王完顏永济繼位。金章宗駕崩、完顏永濟繼位後,成吉思汗知道完顏永濟是個無能之輩,所以在次年立即揮軍南下開始侵略金朝。

他死後諡號是憲天光運仁文義武神聖英孝皇帝,廟號是章宗。葬于道陵。

元朝官修正史《金史》脱脱等的評價是:“章宗在位二十年,承世宗治平日久,宇内小康,乃正礼乐,修刑法,定官制,典章文物粲然成一代治规。又数问群臣汉宣综核名实、唐代考课之法,盖欲跨辽、宋而比迹于汉、唐,亦可谓有志于治者矣!然婢宠擅朝,冢嗣未立,疏忌宗室而传授非人。向之所谓维持巩固于久远者,徒为文具,而不得为后世子孙一日之用,金源氏从此衰矣!昔扬雄氏有云:‘秦之有司负秦之法度,秦之法度负圣人之法度。’盖有以夫。”

\subsection{明昌}


\begin{longtable}{|>{\centering\scriptsize}m{2em}|>{\centering\scriptsize}m{1.3em}|>{\centering}m{8.8em}|}
  % \caption{秦王政}\
  \toprule
  \SimHei \normalsize 年数 & \SimHei \scriptsize 公元 & \SimHei 大事件 \tabularnewline
  % \midrule
  \endfirsthead
  \toprule
  \SimHei \normalsize 年数 & \SimHei \scriptsize 公元 & \SimHei 大事件 \tabularnewline
  \midrule
  \endhead
  \midrule
  元年 & 1190 & \tabularnewline\hline
  二年 & 1191 & \tabularnewline\hline
  三年 & 1192 & \tabularnewline\hline
  四年 & 1193 & \tabularnewline\hline
  五年 & 1194 & \tabularnewline\hline
  六年 & 1195 & \tabularnewline\hline
  七年 & 1196 & \tabularnewline
  \bottomrule
\end{longtable}

\subsection{承安}

\begin{longtable}{|>{\centering\scriptsize}m{2em}|>{\centering\scriptsize}m{1.3em}|>{\centering}m{8.8em}|}
  % \caption{秦王政}\
  \toprule
  \SimHei \normalsize 年数 & \SimHei \scriptsize 公元 & \SimHei 大事件 \tabularnewline
  % \midrule
  \endfirsthead
  \toprule
  \SimHei \normalsize 年数 & \SimHei \scriptsize 公元 & \SimHei 大事件 \tabularnewline
  \midrule
  \endhead
  \midrule
  元年 & 1196 & \tabularnewline\hline
  二年 & 1197 & \tabularnewline\hline
  三年 & 1198 & \tabularnewline\hline
  四年 & 1199 & \tabularnewline\hline
  五年 & 1200 & \tabularnewline
  \bottomrule
\end{longtable}

\subsection{泰和}

\begin{longtable}{|>{\centering\scriptsize}m{2em}|>{\centering\scriptsize}m{1.3em}|>{\centering}m{8.8em}|}
  % \caption{秦王政}\
  \toprule
  \SimHei \normalsize 年数 & \SimHei \scriptsize 公元 & \SimHei 大事件 \tabularnewline
  % \midrule
  \endfirsthead
  \toprule
  \SimHei \normalsize 年数 & \SimHei \scriptsize 公元 & \SimHei 大事件 \tabularnewline
  \midrule
  \endhead
  \midrule
  元年 & 1201 & \tabularnewline\hline
  二年 & 1202 & \tabularnewline\hline
  三年 & 1203 & \tabularnewline\hline
  四年 & 1204 & \tabularnewline\hline
  五年 & 1205 & \tabularnewline\hline
  六年 & 1206 & \tabularnewline\hline
  七年 & 1207 & \tabularnewline\hline
  八年 & 1208 & \tabularnewline
  \bottomrule
\end{longtable}


%%% Local Variables:
%%% mode: latex
%%% TeX-engine: xetex
%%% TeX-master: "../Main"
%%% End:

%% -*- coding: utf-8 -*-
%% Time-stamp: <Chen Wang: 2021-11-01 16:57:16>

\section{紹王完颜永济\tiny(1208-1213)}

\subsection{生平}

完颜允济(?-1213年9月11日),小字興勝,金章宗時避章宗父完顏允恭諱改為完颜永济。他是金朝第七位皇帝(1208年12月29日—1213年9月11日在位),被篡位後降封衛王,卒諡「紹王」,在位5年。

完颜允济是完顏允恭之弟,金章宗之叔,金世宗完颜雍第七子,母元妃李氏。他在金世宗大定十一年(1171年)被封薛王,同年改封禭王,先後改封潞王、韓王及衛王。章宗在泰和八年(1208年)农历十一月二十日病死,無嗣,衛王完颜允济被迎立为帝。

蒙古帝國的成吉思汗有意進攻金國,首先出兵進攻臣屬金朝的西夏,西夏向金求援,卫绍王坐視不救。西夏向蒙古屈服後,成吉思汗自大安三年(1211年)起大舉攻金,屢敗金兵。是年九月,蒙古軍逼近中都,因城防堅固兼有重兵防守,於是退兵。次年成吉思汗再次親征金國,一度包圍金西京大同府。同年契丹人耶律留哥在今吉林省境起兵反金,數月之間發展至十餘萬人。耶律留哥依附蒙古,又在迪吉腦兒(今辽宁昌图附近)擊敗六十萬金兵,金國的處境更加不妙。

衛绍王为人优柔寡断,没有安邦治国之才,只是俭约守成而已。他不善于用人,忠奸不分,最终导致杀身之祸。至寧元年(1213年)八月,蒙古軍再次逼近中都,右副元帥胡沙虎(紇石烈執中)起兵叛亂,弑卫绍王。九月,迎立完顏珣為帝,即金宣宗。胡沙虎請廢允济為庶人,詔百官三百餘人議於朝堂。太子少傅奧屯忠孝、侍讀學士蒲察思忠支持胡沙虎,但戶部尚書武都、拾遺田庭芳等三十人請降允济為王侯。胡沙虎固執前議,金宣宗不得已,乃降封允济為東海郡侯。十月,元帥右監軍朮虎高琪殺胡沙虎。

貞祐四年(1216年),金宣宗詔追復允濟為衛王,諡曰紹,後世稱他為衛绍王。

元朝官修正史《金史》脱脱等的評價是:“卫绍王政乱于内,兵败于外,其灭亡已有征矣。身弑国蹙,记注亡失,南迁后不复纪载。皇朝中统三年,翰林学士承旨王鹗有志论著,求大安、崇庆事不可得,采摭当时诏令,故金部令史窦祥年八十九,耳目聪明,能记忆旧事,从之得二十余条。司天提点张正之写灾异十六条,张承旨家手本载旧事五条,金礼部尚书杨云翼日录四十条,陈老日录三十条,藏在史馆。条件虽多,重复者三之二。惟所载李妃、完颜匡定策,独吉千家奴兵败,纥石烈执中作难,及日食、星变、地震、氛昆,不相背盭。今校其重出,删其繁杂。《章宗实录》详其前事,《宣宗实录》详其后事。又于金掌奏目女官大明居士王氏所纪,得资明夫人援玺一事,附著于篇,亦可以存其梗概云尔。”明朝官修《元史》,成吉思汗对完颜永济的評價是:“我谓中原皇帝是天上人做,此等庸懦亦为之耶?”

\subsection{大安}


\begin{longtable}{|>{\centering\scriptsize}m{2em}|>{\centering\scriptsize}m{1.3em}|>{\centering}m{8.8em}|}
  % \caption{秦王政}\
  \toprule
  \SimHei \normalsize 年数 & \SimHei \scriptsize 公元 & \SimHei 大事件 \tabularnewline
  % \midrule
  \endfirsthead
  \toprule
  \SimHei \normalsize 年数 & \SimHei \scriptsize 公元 & \SimHei 大事件 \tabularnewline
  \midrule
  \endhead
  \midrule
  元年 & 1209 & \tabularnewline\hline
  二年 & 1210 & \tabularnewline\hline
  三年 & 1211 & \tabularnewline
  \bottomrule
\end{longtable}

\subsection{崇庆}

\begin{longtable}{|>{\centering\scriptsize}m{2em}|>{\centering\scriptsize}m{1.3em}|>{\centering}m{8.8em}|}
  % \caption{秦王政}\
  \toprule
  \SimHei \normalsize 年数 & \SimHei \scriptsize 公元 & \SimHei 大事件 \tabularnewline
  % \midrule
  \endfirsthead
  \toprule
  \SimHei \normalsize 年数 & \SimHei \scriptsize 公元 & \SimHei 大事件 \tabularnewline
  \midrule
  \endhead
  \midrule
  元年 & 1212 & \tabularnewline\hline
  二年 & 1213 & \tabularnewline
  \bottomrule
\end{longtable}

\subsection{至宁}

\begin{longtable}{|>{\centering\scriptsize}m{2em}|>{\centering\scriptsize}m{1.3em}|>{\centering}m{8.8em}|}
  % \caption{秦王政}\
  \toprule
  \SimHei \normalsize 年数 & \SimHei \scriptsize 公元 & \SimHei 大事件 \tabularnewline
  % \midrule
  \endfirsthead
  \toprule
  \SimHei \normalsize 年数 & \SimHei \scriptsize 公元 & \SimHei 大事件 \tabularnewline
  \midrule
  \endhead
  \midrule
  元年 & 1213 & \tabularnewline
  \bottomrule
\end{longtable}


%%% Local Variables:
%%% mode: latex
%%% TeX-engine: xetex
%%% TeX-master: "../Main"
%%% End:

%% -*- coding: utf-8 -*-
%% Time-stamp: <Chen Wang: 2019-12-26 11:17:45>

\section{宣宗\tiny(1213-1224)}

\subsection{生平}

金宣宗完颜珣(1163年4月18日-1224年1月14日),女真名吾睹補。金世宗完颜雍长孙,卫绍王侄,父完顏允恭,母昭華劉氏。他是金朝第八位皇帝(1213年9月22日—1224年1月14日在位),在位11年,终年61岁。

金世宗大定十八年(1178年),封溫國公,二十六年賜名珣,二十九年封豐王。承安元年,封翼王。泰和五年,改賜名從嘉,其後又改封邢王及升王。

至寧元年(1213年)八月,胡沙虎杀死卫绍王,迎立從嘉为帝,由於從嘉在河北鎮守,於是暫時以從嘉長子完顏守忠監國。九月即位,是為宣宗,以胡沙虎為太師、尚書令兼都元帥,封澤王,同月改元貞祐。閏九月,宣宗復舊名珣。十月,朮虎高琪殺胡沙虎,宣宗赦免高琪,封他為左副元帥。是年秋,蒙古軍分三路攻金,幾乎攻破所有河北郡縣,金國只有中都、真定、大名等十一城未曾失守。

貞祐二年三月,蒙金和議成,五月十八日(1214年6月27日),金宣宗逃離中都,七月金宣宗南逃到達汴京,此舉觸怒蒙古,戰爭再起。貞祐三年五月初二(1215年5月31日),中都失守,十月,蒲鮮萬奴在遼東自立。興定三年十二月(1220年初),宣宗誅高琪。

宣宗对外措施十分不当,直接导致金朝灭亡。他先向蒙古大汗成吉思汗屈辱求和,又与西夏断交,還不顧丞相徒单镒和諸多大臣等的反對,将都城由中都南迁至汴京,并且发动侵宋战争。金国三面受敌,加上內部不和,叛亂頻生,國家危在旦夕。

宣宗在元光二年十二月二十二日(1224年1月14日)去世,死後諡號是繼天興統述道勤仁英武聖孝皇帝,廟號是宣宗,葬于德陵(在今河南開封)。

元朝官修正史《金史》脱脱等的評價是:“宣宗当金源末运,虽乏拨乱反正之材,而有励精图治之志。迹其勤政忧民,中兴之业盖可期也,然而卒无成功者何哉?良由性本猜忌,崇信翙御,奖用吏胥,苛刻成风,举措失当故也。执中元恶,此岂可相者乎,顾乃怀其援立之私,自除廉陛之分,悖礼甚矣。高琪之诛执中,虽云除恶,律以《春秋》之法,岂逃赵鞅晋阳之责?既不能罪而遂相之,失之又失者也。迁汴之后,北顾大元之朝日益隆盛,智识之士孰不先知?方且狃于余威,牵制群议,南开宋衅,西启夏侮,兵力既分,功不补患。曾未数年,昔也日辟国百里,今也日蹙国里,其能济乎?再迁遂至失国,岂不重可叹哉!”

\subsection{贞祐}


\begin{longtable}{|>{\centering\scriptsize}m{2em}|>{\centering\scriptsize}m{1.3em}|>{\centering}m{8.8em}|}
  % \caption{秦王政}\
  \toprule
  \SimHei \normalsize 年数 & \SimHei \scriptsize 公元 & \SimHei 大事件 \tabularnewline
  % \midrule
  \endfirsthead
  \toprule
  \SimHei \normalsize 年数 & \SimHei \scriptsize 公元 & \SimHei 大事件 \tabularnewline
  \midrule
  \endhead
  \midrule
  元年 & 1213 & \tabularnewline\hline
  二年 & 1214 & \tabularnewline\hline
  三年 & 1215 & \tabularnewline\hline
  四年 & 1216 & \tabularnewline\hline
  五年 & 1217 & \tabularnewline
  \bottomrule
\end{longtable}

\subsection{兴定}

\begin{longtable}{|>{\centering\scriptsize}m{2em}|>{\centering\scriptsize}m{1.3em}|>{\centering}m{8.8em}|}
  % \caption{秦王政}\
  \toprule
  \SimHei \normalsize 年数 & \SimHei \scriptsize 公元 & \SimHei 大事件 \tabularnewline
  % \midrule
  \endfirsthead
  \toprule
  \SimHei \normalsize 年数 & \SimHei \scriptsize 公元 & \SimHei 大事件 \tabularnewline
  \midrule
  \endhead
  \midrule
  元年 & 1217 & \tabularnewline\hline
  二年 & 1218 & \tabularnewline\hline
  三年 & 1219 & \tabularnewline\hline
  四年 & 1220 & \tabularnewline\hline
  五年 & 1221 & \tabularnewline\hline
  六年 & 1222 & \tabularnewline
  \bottomrule
\end{longtable}

\subsection{元光}

\begin{longtable}{|>{\centering\scriptsize}m{2em}|>{\centering\scriptsize}m{1.3em}|>{\centering}m{8.8em}|}
  % \caption{秦王政}\
  \toprule
  \SimHei \normalsize 年数 & \SimHei \scriptsize 公元 & \SimHei 大事件 \tabularnewline
  % \midrule
  \endfirsthead
  \toprule
  \SimHei \normalsize 年数 & \SimHei \scriptsize 公元 & \SimHei 大事件 \tabularnewline
  \midrule
  \endhead
  \midrule
  元年 & 1222 & \tabularnewline\hline
  二年 & 1223 & \tabularnewline
  \bottomrule
\end{longtable}


%%% Local Variables:
%%% mode: latex
%%% TeX-engine: xetex
%%% TeX-master: "../Main"
%%% End:

%% -*- coding: utf-8 -*-
%% Time-stamp: <Chen Wang: 2019-12-26 11:17:50>

\section{哀宗\tiny(1224-1234)}

\subsection{生平}

金哀宗完颜守緒(1198年9月25日-1234年2月9日),金朝第九位皇帝(1224年1月15日—1234年2月9日在位),女真名寧甲速,金朝亡國之君。哀宗在位10年,国破后自缢而死,终年37岁。

哀宗生於金承安三年八月二十三日(1198年9月25日),初名守禮,是金宣宗第三子,母明惠皇后王氏。宣宗登位後,封守禮為遂王。皇太子守忠及皇太孫鏗早逝,貞祐四年(1216年)正月,立守禮為皇太子,同年賜名守緒。元光二年十二月(1224年1月),宣宗去世,守緒繼位,是為哀宗。正大元年(1224年)六月立妃徒單氏為皇后。

哀宗本是一位比较有作为的皇帝,即位後,鼓励农业生产,停止侵宋战争,与西夏修好,进行内部改革,铲除奸佞,重用抗蒙名将,收复了不少土地,使金朝呈现出一片全新的景象。可是此時的蒙古勢不可擋,正大四年(1227年)滅西夏後即全力伐金。

在天興元年(1232年)的三峰山之战中,金军主力被蒙軍消滅,金国滅亡之勢已不可免。蒙軍進圍汴京,守軍奮力抵抗,當年汴京大疫,凡五十日,從各城門運出的死者有九十餘萬人,貧不能葬者尚未包括在內。哀宗在十二月逃離汴京,北渡黃河,後奔歸德(今河南商丘),最後來到蔡州(今河南汝南),然蒙古大將史天澤一路緊追不捨,在蒲城殲滅了完顏白撒的八萬精兵。天興二年(1233年)八月,蒙古召宋兵攻破唐州(今河南唐河),哀宗欲與宋連和,派使者向宋人說:「蒙古滅國四十,以及西夏,夏亡及我,我亡必及宋。唇亡齒寒,自然之理。」宋人不許。天興三年正月己酉(儒略曆1234年2月9日),蒙宋聯軍攻破蔡州,哀宗不願做亡國之君,便把皇位传给统帅完颜承麟,自己在蔡州幽蘭軒上吊自盡。末帝完顏承麟聞知哀宗死訊,“率群臣入哭,諡曰哀宗”,“哭奠未畢,城潰。”末帝同日死于亂軍中,金亡。

金哀宗之遗骸則被宋将孟珙与蒙将塔察儿所分屍。据蒙古伊兒汗国宰相拉施特主编的《史集》载,塔察儿仅获得金哀宗的一只手。当时金哀宗的尸首被贴身的近侍烧掉,并埋于汝水之上,所以拉施特的说法值得商榷。宋朝視金哀宗為金國亡國之君,把其大部分遗骸被宋軍帶回首都臨安告太廟。宋理宗最后按洪咨夔的建議處理了金哀宗遗骸,葬于大理寺狱库。

元朝官修正史《金史》脱脱等的評價是:“金之初兴,天下莫强焉。太祖、太宗威制中国,大概欲效辽初故事,立楚立齐,委而去之,宋人不竞,遂失故物。熙宗、海陵济以虐政,中原觖望,金事几去。天厌南北之兵,挺生世宗,以仁易暴,休息斯民。是故金祚百有余年,由大定之政有以固结人心,乃克尔也。章宗志存润色,而秕政日多,诛求无艺,民力浸竭,明昌、承安盛极衰始。至于卫绍,纪纲大坏,亡征已见。宣宗南度,弃厥本根,外狃余威,连兵宋、夏,内致困惫,自速土崩。哀宗之世无足为者。皇元功德日盛,天人属心,日出爝息,理势必然。区区生聚,图存于亡,力尽乃毙,可哀也矣。虽然,在《礼》“国君死社稷”,哀宗无愧焉。”

\subsection{正大}


\begin{longtable}{|>{\centering\scriptsize}m{2em}|>{\centering\scriptsize}m{1.3em}|>{\centering}m{8.8em}|}
  % \caption{秦王政}\
  \toprule
  \SimHei \normalsize 年数 & \SimHei \scriptsize 公元 & \SimHei 大事件 \tabularnewline
  % \midrule
  \endfirsthead
  \toprule
  \SimHei \normalsize 年数 & \SimHei \scriptsize 公元 & \SimHei 大事件 \tabularnewline
  \midrule
  \endhead
  \midrule
  元年 & 1224 & \tabularnewline\hline
  二年 & 1225 & \tabularnewline\hline
  三年 & 1226 & \tabularnewline\hline
  四年 & 1227 & \tabularnewline\hline
  五年 & 1228 & \tabularnewline\hline
  六年 & 1229 & \tabularnewline\hline
  七年 & 1230 & \tabularnewline\hline
  八年 & 1231 & \tabularnewline
  \bottomrule
\end{longtable}

\subsection{开兴}

\begin{longtable}{|>{\centering\scriptsize}m{2em}|>{\centering\scriptsize}m{1.3em}|>{\centering}m{8.8em}|}
  % \caption{秦王政}\
  \toprule
  \SimHei \normalsize 年数 & \SimHei \scriptsize 公元 & \SimHei 大事件 \tabularnewline
  % \midrule
  \endfirsthead
  \toprule
  \SimHei \normalsize 年数 & \SimHei \scriptsize 公元 & \SimHei 大事件 \tabularnewline
  \midrule
  \endhead
  \midrule
  元年 & 1232 & \tabularnewline
  \bottomrule
\end{longtable}

\subsection{天兴}

\begin{longtable}{|>{\centering\scriptsize}m{2em}|>{\centering\scriptsize}m{1.3em}|>{\centering}m{8.8em}|}
  % \caption{秦王政}\
  \toprule
  \SimHei \normalsize 年数 & \SimHei \scriptsize 公元 & \SimHei 大事件 \tabularnewline
  % \midrule
  \endfirsthead
  \toprule
  \SimHei \normalsize 年数 & \SimHei \scriptsize 公元 & \SimHei 大事件 \tabularnewline
  \midrule
  \endhead
  \midrule
  元年 & 1232 & \tabularnewline\hline
  二年 & 1233 & \tabularnewline\hline
  三年 & 1234 & \tabularnewline
  \bottomrule
\end{longtable}


%%% Local Variables:
%%% mode: latex
%%% TeX-engine: xetex
%%% TeX-master: "../Main"
%%% End:



%%% Local Variables:
%%% mode: latex
%%% TeX-engine: xetex
%%% TeX-master: "../Main"
%%% End:
 % 金
% %% -*- coding: utf-8 -*-
%% Time-stamp: <Chen Wang: 2019-12-26 14:36:18>

\chapter{元\tiny(1271-1368)}

\section{简介}

元朝(1271年-1368年),漢語国号全稱为大元,蒙古語國號全稱大元也克蒙古兀鲁思(意为大元大蒙古國),是中國歷史上由蒙古人所建立的大一統王朝。1260年,忽必烈稱帝,自立為第五代大蒙古國大汗,後於1271年取儒士劉秉忠建議,定漢文國號為「大元」,改蒙古語国号「大蒙古国」為「大元大蒙古国」,定都於漢地大都(今北京市),建立元朝。1279年元軍徹底攻灭南宋殘餘勢力,一統中國並結束南宋與金朝南北政權对峙之局面。雖然傳統以南宋為正統王朝,由於金朝認為已繼承宋朝正統,有一說認為元朝繼承金朝正統,並選取根據五行相生順序生自金朝「土」德的「金」德為王朝德運,同時選取與金德對應的白色為王朝正色。

元朝的基础為乞颜部族的首领铁木真于1206年统一漠北诸部族后建立的大蒙古國,铁木真被称为“成吉思汗”。當時蒙古诸部受金朝统辖,然而由於金朝與西夏均走向衰落,成吉思汗先後攻打西夏與金朝,並於西元1227年8月攻滅西夏、1234年3月攻滅金朝,取得中国華北地区和黄土高原地区。同一时间,大蒙古国在西方不断扩张,先後發動三次西征,形成稱霸歐亞大陸的国家,被欧洲称为蒙古帝國(Mongol Empire)。

1259年,第四代蒙古大汗蒙哥(拖雷長子)於征伐南宋的戰爭中去世後,領有漢地、主張漢化、陪同主持对南宋战争的忽必烈(拖雷第四子)與受漠北蒙古貴族擁護的阿里不哥(拖雷第七子)為了爭奪汗位而發生战争,最後忽必烈於1264年獲勝,而蒙古帝国也宣告徹底地分裂。自元太宗窩闊臺去世以來,蒙古四大汗國先後自立,而忽必烈对于“蒙古大汗”称号的继承也没有得到蒙古诸部的一致承认。

1260年三月忽必烈召集擁護自己的部分蒙古宗王,在開平府召開忽里勒台大會,舉行例行的大汗選舉儀式,宣佈即蒙古大汗位,是為薛禪汗,漢文廟號定為世祖。忽必烈建號「中統」,意即「中原正統」。1271年,忽必烈取《周易》“乾元”之语,公佈《建國號詔》,建立汉语國號為大元,宣佈新王朝為繼承歷代中原王朝的中華正統王朝,史称元朝,忽必烈即元朝的开国皇帝,庙号元世祖。1279年元朝攻滅南宋,統治全中國地區,结束自窩闊台攻宋以来40多年的蒙宋戰爭。元世祖到元武宗期間元朝國力鼎盛時期,軍事上平定西北,但在侵略日本、东南亚诸国卻屢次失利,其中在元日戰爭战败。元中期皇位之争愈演愈烈、政治动荡不安,诸帝施政亦不甚如意。元惠宗晚期,由於怠于政事、滥发纸币导致通货膨胀、為了治理氾濫的黄河又加重徭役,最後导致1351年爆發紅巾軍起事。1368年朱元璋建立明朝後,派徐達北伐攻陷大都,元朝結束。元廷退居漠北,史称北元。北元後主天元十年(1388年)去大元国号(一说1402年元臣鬼力赤篡位建國鞑靼),北元亡。

元朝建立后,承袭了蒙古帝國在中国北方、蒙古高原以及西伯利亚的領土,蒙古帝国西征而来的土地却不在元朝统治范围之内。元朝领土經過多次擴展後,於1310年元武宗時期達到全盛,西到吐鲁番,西南包括西藏、云南及缅甸北部,东到日本海,北至都播南部與北海、鄂畢河東部,被譽稱「东尽辽左西极流沙,北逾阴山南越海表,汉唐极盛之时不及也」。元朝至元成宗时,经过一系列战争和协商,获得欽察汗國、察合台汗國、窩闊台汗國與伊兒汗國等四大汗國承认为宗主國,并且元朝皇帝为名义上的“蒙古大汗”继任者;其藩屬國涵蓋高麗與東南亞各國。

元朝在經濟方面仍以農業為主,整體生產力向前發展,尤其是邊陲地區的經濟發展最為顯著,在生产技术、垦田面积、粮食产量、水利兴修以及棉花广泛种植等方面仍然取得一定進步。蒙古人是游牧民族,草原时期以畜牧为主,经济单一,无所谓土地制度。蒙古軍在攻打華北時,出於與舊主金朝的恩怨採取報複性政策,殘酷的屠杀和劫掠带来很大的破坏。攻滅金朝后,在耶律楚材勸諫下,窩闊台汗同意復甦农业,鼓勵漢人墾殖以期長治久安。元世祖即位之后,实行些鼓励生产、安抚流民的措施。到元朝時,由於经济作物棉花不断推广種植,與棉纺织品在江南一带都比较興盛。经济作物商品性生产的发展,就使当时基本上自给自足的农村经济,在某些方面渗入商品货币经济关系。但是,由於元帝集中控制大量的手工业工匠,经营日用工艺品的生产,官营手工业特别发达,对民间手工业则有限制。

元朝對中國傳統文化的影響大過對社會經濟的影響。不同於中国历史上其他征服王朝為了提升本身文化而積極吸收中華文化,元朝皇室对于宗教兴趣浓厚,极力推崇伊斯蘭教與藏傳佛教乃至景教,对中華文化则采取与西亞文化并重的模式进行发展。在政治上,政府大量使用来自西亚的色目人,降低契丹人、汉人儒者的地位,压制南人。雖然元朝前期沒有系統性舉辦科舉。,但对儒家文化有着应有的尊重,並且將儒家推廣至邊遠地區,元朝创建了24400所各级官学,使全国平均每2600人即拥有一所学校的政绩,创造了「書院之设,莫盛于元」的历史记录 。由於士大夫文化式微,意味宋朝顯貴的傳統社會秩序已經崩潰。這使得在士大夫文化底下,屬於中下層的庶民文化反而有機會迅速的抬頭並普及。這個現象在政治方面是重用胥吏,在藝術與文學方面則是發展以庶民為對象的戲劇與藝能,其中以元曲最為興盛。

元朝的汉文国号「大元」出自《易经·乾卦》“大哉乾元,万物资始,乃统天”。1271年12月18日,忽必烈汗公布《建国号诏》,宣佈新王朝為繼承歷代中原王朝的中華正統王朝 ,國號為大元。元朝是中国历史上第一个把“大”字加于正式国号之中的大一统王朝,除此前仅统治了华北地区的辽朝和金朝等外,之前各朝的“大”字均为尊称。

元朝歷史通常可以分為兩個到三個階段:

1206年元太祖铁木真統一蒙古,立國位於漠北的蒙古草原,定國號為「大蒙古國」;到1271年元世祖忽必烈定都元大都,將國號改為大元之际,共六十五年,稱為大蒙古國時期,又稱蒙古帝國;

元世祖忽必烈定都元大都,1271年將國號改為大元後,直到1368年元惠宗出亡為止,共九十七年,才是嚴格意義上的元朝歷史;

元惠宗出亡後依舊以大元為國號,至1402年鬼力赤殺順天帝坤帖木兒去國號為止(一說1388年天元帝脱古思帖木儿被也速迭尔杀害後去國號),稱為北元時期。

去國號後稱蒙古,明廷稱韃靼。

辽朝时期,蒙古草原上的诸部归于辽朝统辖。金灭辽后,草原各部歸屬不一,汪古部等成為金朝的臣屬,而乞顏部的合不勒汗乘金军大举南下而无暇北顾之机,建立了早期的蒙古国家,即蒙兀国,此后一直侵袭金朝的边境。合不勒汗死后,俺巴孩汗成为新的大汗。由于塔塔儿人的出卖,俺巴孩汗被金朝皇帝金熙宗钉在木驴上致死,此事件埋下了蒙古对金朝复仇的种子。在金章宗死后,13世紀初,金朝在衛紹王完颜永济的統治下走向衰落,蒙古乞颜部铁木真开始了统一蒙古草原的征程。先后在克烈部首领王罕以及他的安达扎答兰部首领札木合的军事援助下,打败了蔑兒乞人,夺回了被蔑儿乞人夺取的众多部众(以及其妻孛儿帖),力量逐渐壮大。1189年,在经过激烈的争夺之后,铁木真被乞颜贵族推举为部落的可汗。然而,铁木真部族的逐渐强大,危及了援助他的札木合在蒙古草原上的地位,于是札木合联合泰赤乌等部,合兵三万余人,向铁木真发起进攻。面对来势汹汹的札木合,铁木真将自己的部众3万人组成十三翼。在战斗中铁木真暂时战败,为保存实力退至斡难河的哲列捏山峡,扼险而守。史称“十三翼之战”。札木合虽然取得战役的胜利,但札木合的暴虐受到了其所属部落首领的不满,而铁木真对部众进行笼络,故部众归心于铁木真。于是畏答儿、赤老温、术赤台、晃豁坛等族人纷纷来附。此后,铁木真力量进一步壮大。1196年,从属于金朝的蒙古部族塔塔儿部叛金,完颜永济派丞相完颜襄率军征讨。铁木真联合克烈部,以“为父亲报仇”的名义,在斡里匝河击溃了塔塔儿部,使塔塔儿一蹶不振。战后,金朝授铁木真糺军统领之职,使他可以用金朝属官名义号令蒙古部众。1200年,铁木真与王汗会于萨里川(今蒙古国克鲁伦河上游之西),大败泰赤乌与蔑儿乞的联军,首领塔里忽台等被杀。1201年,铁木真又在呼伦贝尔海剌尔河支流帖尼河之野,击败以札木合为首的塔塔儿、弘吉剌、合答斤等十一部联军,史称“帖尼河之战”。宋嘉泰二年,铁木真与王汗联军又在阔亦田击败了札木合同乃蛮、泰赤乌、塔塔儿、蔑儿乞等联军,取得了阔亦田之战的胜利。接着招降了呼伦贝尔一带的弘吉剌惕等部。至此,蒙古高原都被铁木真控制了。最后平定蒙古高原,统一蒙古各部,1206年春,蒙古贵族在斡难河(今鄂嫩河)源头召开库里尔台大会,蒙古部鐵木真得到成吉思汗稱號,建國大蒙古国(即蒙古帝國),後被尊稱元太祖。

金朝與蒙古為世仇,成吉思汗有意伐金復仇,然而西南的西夏與金朝聯盟,為了避免被西夏牽制,先後三次率軍(1205年、1207年與1209年—1210年)进攻之,迫使西夏夏襄宗稱臣。1210年成吉思汗與金斷交,隔年發動蒙金戰爭,於野狐嶺戰役大破四十萬金軍,隨後攻入華北地區並四處屠杀。1214年蒙軍包圍金朝首都中都(今北京市),金宣宗被迫求和称臣,並在蒙古退兵後遷都北宋故都汴京。隔年5月31日蒙軍南下攻佔金中都,並且獲得名相耶律楚材,這對於巩固華北地區有很大的幫助。1217年,成吉思汗為了西征花剌子模,命木华黎統領漢地,封为“太师国王”,命他持續进攻金朝。木华黎為了鞏度漢地,收降地方自衛勢力如真定史天澤、滿城張柔、東平嚴實與濟南張宏,史稱漢族四大世侯,後來他們也扶佐忽必烈建立元朝。木华黎除了對金朝的戰爭讓金朝疆域萎縮剩河南與關中地區之外,並於1231年派兵進攻高麗,使高麗退到江華島以南(即今日南韓)。

西域方面,為了建立通往西方的道路,早在1209年—1210年就讓新疆东部的畏兀儿與伊犁河谷的哈剌魯先後歸順。當金朝遷都並將要滅亡之際,中亞新興大國花剌子模在沙阿摩诃末时期崛起,該國訛答剌地方大臣海儿汗亦纳勒术前后两次屠杀蒙古商队並侮辱蒙古使臣,成吉思汗遂决心發動第一次西征。1218年蒙將哲别殺死占領西遼並稱遼帝的屈出律,攻占塔里木地區,史稱蒙古攻西遼之戰。隔年六月,成吉思汗親率蒙古主力軍十万西征花剌子模。由於沙阿摩诃末抵擋不了蒙軍攻勢,畏懼而逃,在屠杀掉花剌子模的40个城镇之后,花剌子模也於1221年亡國。成吉思汗命速不臺和哲别追殺摩诃末,摩诃末最後死於裡海。其子札蘭丁於八魯灣之戰英勇抗敵,最後南逃印度,並於1224年復國於大不里士(今伊朗西北部)。1230年,札蘭丁被蒙古将军绰儿马罕攻滅。速不臺和哲别最後于1222年从撒马尔罕出发经过今伊朗高原北部,进攻杀掠高加索三国(亚美尼亚王国、格鲁吉亚、阿塞拜疆)之后,并越太和岭(今高加索山脈),抵達欽察(位於俄南),期間攻占不少國家。於1223年的迦勒迦河之战(今乌克兰日丹诺夫市北)更是擊潰基辅罗斯諸國與钦察忽炭汗的联军,并向西进军到今乌克兰西部的德涅斯特河,折转围攻基辅后东返,并于1223年9月攻击伏尔加河中上游的河谷伏尔加保加利亚,最後渡过伏尔加河東返中亞。成吉思汗將新拓展的疆土分封給長子朮赤、次子察合台和三子窝阔台,四子拖雷領有蒙古本土,三子窩闊台成為大汗繼承人。1225年蒙古回師後,因西夏不配合西征,成吉思汗又率归師滅西夏。1227年,成吉思汗病逝,由幼子拖雷监国。

拖雷监国两年后於1229年舉辦库里尔台大会,窝阔台被推举为蒙古大汗,後尊稱元太宗。1231年窝阔台汗率軍南征金朝,並命四弟拖雷自漢中借宋道沿漢水攻打汴京,隔年拖雷在河南三峰山之战擊潰金军。1234年蒙宋联軍聯合攻破蔡州,金哀宗自杀,金朝亡。南宋雖然發起端平入洛以收復河南地,但是華北地區最後全由蒙古占領。1235年,窝阔台汗定都哈拉和林(今乌兰巴托西南)後,藉此率軍南征南宋以報復之,掠奪兩淮地區後北返。蒙古為了防止華北的漢人世侯叛變,派探馬赤軍(振戍軍)進駐漢地;進行兩次人口調查,將半數漢人分封給蒙古功臣。由於需要人才治理国家,窝阔台汗接受耶律楚材的建议,於1238年命术忽德和刘中舉辦科舉,史稱戊戌选试。这次考试录取东平杨奂等名士,為統治華北帶來不少人才,但后来以“当世或以为非便,事复中止”。

西線方面,1235年窝阔台汗命术赤长子拔都、貴由與蒙哥、速不台等第二代蒙古王子發起蒙古第二次西征,史稱拔都西征,總指揮為拔都與速不台。1236年至1242年間攻占欽察草原、基輔羅斯等各公國并进犯匈牙利、摩尔达维亚、波蘭、立陶宛大公国、摩拉维亚原南斯拉夫地区、保加利亚第二帝国、拉什卡等中東歐各國。1241年11月窩闊台汗去世,由皇后乃马真脫列哥那監國,1246年3月的库里尔台大會由其子贵由即位,後追尊稱元定宗。1247年吐蕃诸部归附大蒙古,史稱涼州會盟。1248年8月貴由汗在遠征拔都的途中去世,皇后斡兀立海迷失立孫子失烈門並監國。然而在1251年7月的大會,因為拔都與兀良哈台大力支持拖雷系的蒙哥,使得窩闊台系的失烈失去汗位。蒙哥繼承汗位,後尊稱元憲宗。

1252年蒙哥即位後推行中央集權化,在漢地、中亞與伊朗等直轄地設置行中书省,分遣拖雷系諸王分守各地,以其弟忽必烈總領漠南漢地大總督以管理漢地。忽必烈統治漢地期間任用了大批漢族幕僚和儒士,鞏固了華北地區,並且與兀良合台迂迴南滅大理,擴大南宋防線缺口。1258年高麗崔氏政權跨台,高麗成為藩屬國。同年蒙哥汗宣布兵分三路南征南宋,蒙哥汗率軍攻打四川合州(今重慶)、忽必烈攻打湖北鄂州(今武昌)、兀良合台由雲南晏当(今云南丽江北部)直攻经过安南,进攻宋广南西路而直攻荆湖南路,并兵临潭州(今長沙),三軍意圖在華中會合,再大舉下長江圍攻臨安。隔年蒙哥汗在合州的釣魚城之戰戰死,忽必烈等人停止南征,北返奪位。西線方面,蒙哥汗派其弟旭烈兀西征西亞,史稱蒙古第三次西征,1256年旭烈兀攻滅伊斯蘭教的暗殺組織木剌夷。1258年西征軍攻佔阿拔斯王朝最後領地美索不达米亚的巴格達。1260年佔領大馬士革和阿勒頗。然而當旭烈兀得知蒙哥於南征南宋時去世的消息後,立即率大軍回師爭位。留下的蒙軍也在今以色列加利利的阿音札魯特戰役敗於埃及馬木留克王朝,第三次西征結束。

蒙哥汗去世後,身在戰事的忽必烈立即與南宋和談,返回華北與留守蒙古本土的七弟阿里不哥爭奪汗位。1260年5月5日忽必烈在部分宗王和蒙漢大臣的擁立下於开平(後稱上都,今内蒙古多伦县北石别苏木)自立为蒙古皇帝(又稱蒙古大汗),年號中統。忽必烈登基后不久,阿里不哥在蒙古帝國首都哈拉和林召開库里尔台大會,被阿速台等宗王和大臣選立蒙古大汗,並獲得欽察、察合台與窩闊台汗國的支持。爭奪汗位戰爭最後於1264年8月21日由阿里不哥兵敗投降,忽必烈穩固其位。

忽必烈汗為了成為中國皇帝而推行漢法,主要內容有改元建號,1267年忽必烈汗迁都中都(今北京市),並命劉秉忠兴建中都城。1272年改中都为大都(突厥语称汗八里,帝都之意),將上都作为陪都。1271年12月18日,忽必烈汗公布《建国号诏》,採納漢人儒士劉秉忠的建議,取《易经》中“乾元”之意,宣佈新王朝為繼承歷代中原王朝的中華正統王朝 ,将國號由大蒙古国改为大元,建立元朝,即元世祖;1260年设立中书省,1263年设立樞密院,1268年设立御史台等等國家機構;設置大司農司並且提倡農業;尊孔崇儒並大力發展儒學等推行漢法的政策。然而為了保留原蒙古制度,最後形成蒙漢兩元政治。元世祖雖然於爭奪汗位戰爭獲得蒙古大汗的汗位,並且最後成為中國皇帝,但由於汗位取得不合法與崇尚漢法,使得蒙古宗室不承認忽必烈的汗位,四大汗國有三國不奉忽必烈的命令,蒙古帝國完全解體。最後引發窩闊台系的海都出兵爭奪汗位,造成漠北地區動盪不安,史稱海都之亂。

早在元世祖在與阿里不哥作戰與整頓國內之際,因為無暇對付南宋,於是派郝經對南宋提出議和。當時南宋大權由謊稱擊退蒙古軍的賈似道掌握,然而賈似道由於畏懼謊言被擊破幽禁了郝經。南宋並於1262年拉攏山東漢人世侯李璮,發起李璮叛亂。元軍平定叛亂後,元世祖斷然廢止漢人世侯,以蒙古人直接管理地方事務,並且準備南征南宋。1268年元世祖發起元滅宋之戰,首先派劉整與阿朮率軍攻打襄陽,史稱襄樊之戰。1274年元軍攻下襄陽,宋將呂文煥投降,隨後中书丞相史天泽和枢密院使伯顏率軍順漢水南下長江,目標建康。1275年降將呂文煥率元水陸聯軍於芜湖擊潰贾似道的南宋水軍,史稱丁家洲之戰。隔年元军攻陷临安(今浙江杭州),谢太后與宋恭帝投降元軍。然而陸秀夫等擁立7歲的宋端宗在福州即位,文天祥、張世傑與陳宜中等大臣持續在江西、福建與廣東等地抗元。元軍陸續攻下華南各地,1278年南宋朝廷退至廣東崖山。隔年3月,張弘範在崖山海戰攻滅南宋海軍,陸秀夫帶着8岁的小皇帝宋幼主趙昺投海而死,南宋亡。元朝统一中国地區,结束自唐朝安史之乱以来520多年的分裂局面。

在此前后,元朝曾要求周边一些国家或地区(包括日本、安南、占城、缅甸、爪哇)臣服,加入元朝的朝贡关系,但遭到拒绝,元世祖於是出兵攻打这些国家,其中以入侵日本的元日战争最为著名,因為范文虎指挥不当與颱風來襲而失敗。由於元朝廷需要賞賜大量財寶予宗室貴族,加上開支繁重,財政日漸緊張,朝臣為了財政問題發生爭執,分裂成以许衡等漢人與漢化蒙古人为首的儒臣派與以阿合馬、盧世榮與桑哥等色目人與漢人为首的理財派。儒臣派認為元廷應該節省經費、減免稅收。理財派認為南人藏有大量財物,應沒收以解決朝廷的財政問題。由於元世祖信任阿合馬,設立尚書省解決財政問題。而儒臣則以受漢化更深的太子真金為核心與阿合馬抗衡。最後阿合馬被刺殺,太子真金也因為得病而死。然而元世祖不信任儒臣派,依舊任用理財派官員來解決財政問題,導致財政惡化。

1294年元世祖駕崩後,雖然太子真金早死,但是元世祖曾賜真金的三子鐵穆爾「皇太子寶」並且讓他鎮守和林。隨後鐵穆爾在库里尔台大会中獲得重臣伯颜與玉昔帖木儿等支持,打敗真金的長子甘麻剌與次子答剌麻八剌等繼位,即元成宗。元成宗主要恪守元世祖時期的成宪,任用其侄海山(答剌麻八剌之子)鎮守和林以平定西北海都之亂,並且下令停止征討日本與安南。在內政方面专力整顿国内政治,減免江南部分賦稅。然而,由於元成宗過度賞賜,入不敷出,使國庫資財匱乏。1307年正月,元成宗駕崩,由於太子德寿早逝,左丞相阿忽台擁護皇后卜魯罕與信奉伊斯蘭教的安西王阿難答監國,並有意讓阿難答稱帝。海山之弟愛育黎拔力八達與右丞相哈剌哈孫發動大都政變。他們斬殺阿忽台,控制大都局勢,擁護率軍南下的海山稱帝,即元武宗。皇后與阿難答被元武宗斬殺,其回回部下退入西域吐魯番地區。

元武宗因愛育黎拔力八達有功,冊封他為皇太弟(即未來的元仁宗),相約武宗系與仁宗系交替稱帝,即武仁之約。元武宗時期,加封孔子为“大成至圣文宣王”,并給予孔子的家族與弟子一些稱號。為了解決元成宗時期的财政危机,元武宗設置常平倉以平抑物價,下令印製至大銀鈔,然而反而使銀鈔嚴重贬值。此外他將中書省宣敕與用人權劃歸給尚書省。1311年元武宗因沉耽淫樂、酗酒過度而逝,由皇太弟愛育黎拔力八達繼位,是為元仁宗,這次是元朝首次和平繼承帝位。

西北地區方面,早在元世祖時期,由於他的大汗之位不受四大汗國的承認,使得當時窩闊臺汗海都有意奪回蒙古汗位。海都统辖叶密立(今新疆额敏东南)一带且與欽察汗國友好。元世祖為了避免在南征南宋時被海都背刺,遂扶持八剌獲得察合台汗位以牽制海都。然而在1268年,海都、八剌和欽察汗忙哥帖木兒以元世祖過度漢化、違背祖宗成法為由,在塔拉斯河招開庫里爾台大會結盟反元。他們以海都為盟主,共同瓜分中亞行省,聯合對抗元朝與伊兒汗國,史稱海都之亂。元世祖派伯顏北上平亂,海都與新任察合台汗篤哇採用游擊戰的方式迴避決戰。1287年海都聯軍夥同鎮守遼東的東道諸王乃顏與哈丹襲擊和林(今蒙古國哈尔和林),元世祖親率大軍擊敗之,派伯颜、玉昔帖木儿與李庭平定東北乃顏,主持西北军事。1289年海都再犯和林,最後其勢力被驱出阿尔泰山以西。而哈丹於遼東高麗一帶游擊,至1292年敗亡。

元成宗即位后,任命其侄海山(後繼位為元武宗)总领漠北诸军。1301年海都聯軍被海山和晉王甘麻剌擊潰,史稱鐵堅古山之役。海都於戰後去世,其子察八儿繼位,窩闊台汗國被篤哇掌控。1303年由於篤哇被欽察汗脫脫蒙哥擊潰,就與察八兒共同派使者向元廷請和,脫脫蒙哥也向元廷請和,而伊兒汗本來就支持元廷,至此四大汗國皆承認元朝的宗主地位,雙方廣設驛路,解除封禁。不久之後,窩闊臺汗國被察合台汗篤哇與元朝元武宗先後攻滅而亡,察八兒投降元朝。

元仁宗力图改变元武宗时造成的财政枯竭、政制混乱的局面,他推行「以儒治國」政策,並且減裁冗員、加強中央集權以整頓朝政。他曾令王约将《大学衍义》譯為蒙文,赐臣下说“治天下,此一书足矣。”并将《貞觀政要》和《資治通鑒》等書摘譯為蒙文,令蒙古人與色目人誦習。1312年元仁宗将其儒师王约特拜集贤大学士并将王约“兴科举”的建议“著为令甲”,至此恢复科举制度。本次科舉以程朱理学為考试的内容,史稱延祐復科,最後录取护都答儿、张起岩等56人为进士。他还倚重汉人文臣,处死蒙丞相脱虎脱等,排除朝中异己。財政方面,仁宗取消武宗的经济措施,並且於1314年在江浙、江西、河南等地查清地方田產,史稱延祐經理。任用床兀儿统军,擊敗察合台汗王也先不花以平定西北地區。然而元仁宗未能制止太后答己干预朝政,也無力制裁備受太后重用的重臣鐵木迭兒贪赃枉法。在繼承問題方面,元仁宗以王约輔助皇太子碩德八剌,並且聽從鐵木迭兒的建議,廢除武仁之約。他将元武宗長子周王和世琜外放鎮守云南、次子圖帖睦爾放逐海南島。同年冬天,元武宗舊臣皆感憤怒而擁護和世琜叛變,最後敗走漠北,依附察合台汗國。1320年元仁宗駕崩後,皇太子碩德八剌即位為元英宗。

元英宗繼續實行元仁宗的以儒治国、加强中央集权和官僚体制的政策,並于1323年下令编成并颁布元朝正式法典——《大元通制》,共2539条,他还下令拔除權臣鐵木迭兒在朝廷的势力。然而支持鐵木迭兒的蒙古與色目保守派厭惡英宗的新政,有意發動政變。1323年鐵木迭兒的義子鐵失趁英宗去上都避暑之際,在上都以南15公里的南坡地刺殺英宗及宰相拜住等人,史稱南坡之變,仁宗系自此未能再奪得皇位。晉王甘麻剌的長子,鎮守和林的也孫鐵木兒率兵南下,殺掉行刺元英宗的叛臣並稱帝,即元泰定帝。

泰定帝召回被放逐到海南島的武宗系圖帖睦爾為懷王。泰定帝於1328年七月崩於上都,丞相倒剌沙擁立七歲的阿速吉八為帝,是為元天順帝。而鎮守大都的燕帖木兒與伯顏擁立周王和世琜於漠北、懷王圖帖睦爾於江陵,同年圖帖睦爾先至大都繼位,是為元文宗。燕帖木兒率軍攻入上都,天順帝不知所終。隔年和世琜於漠北和林稱帝,即元明宗。元文宗放棄帝位,派燕帖木兒迎元明宗繼位,並且被立為皇太子。然而燕帖木兒毒死元明宗,元文宗復位,改元天曆,史稱天曆之變。

元文宗時期大兴文治,1329年設立了奎章閣學士院,掌進講經史之書,考察歷代治亂。又令所有勛貴大臣的子孫都要到奎章閣學習。於奎章閣下設藝文監,專門負責將儒家典籍譯成蒙古文字,以及校勘。同年下令編纂《元經世大典》,兩年後修成,為元朝一部重要的記述典章制度的巨著。然而丞相燕帖木儿自恃有功,玩弄朝廷,导致朝政更加腐败。1333年元文宗去世后,为洗刷毒死元明宗的罪行,遗诏立年仅七岁的明宗次子懿璘质班为帝,是为元宁宗。但元宁宗仅在位不到两个月即去世,不久后燕帖木儿也去世。元明宗的长子妥懽贴睦尔被文宗皇后卜答失里从静江(广西桂林)召回并立为帝,是为元惠宗,又称元顺帝。元朝在十三年內,換了八個皇帝。

元惠宗(元順帝)在位之初,1335年燕帖木兒的兒子唐其勢陰謀推翻,另立元文宗義子答剌海。幸右丞相伯顏粉碎叛亂,但屬於保守派的他掌握朝政,權力盛大。他禁止漢人參政並取消科舉,這些都与元惠宗發生衝突。1340年元惠宗在伯颜之侄脱脱的帮助下,终于废黜伯颜。脫脫為相與元惠宗親政前期時,元廷推行一系列改革措施如颁行《至正条格》法規,使得革新政治,社會矛盾緩和,史稱至正新政。1343年元惠宗下令修撰《辽史》、《金史》、《宋史》三史,由右丞相脱脱(后改由阿鲁图)主持,兩年後修成。然而元惠宗後期怠於政事,以至於在1350年發生天災人禍後引來民變。

元朝后期,特别是1340年代中后期至1350年代期间,乾旱、瘟疫與水災時常發生,且自宋朝奸臣杜充挖開黃河大堤以致奪淮入海後,黄河地区水患尤其严重,若以歷代中國王朝的次數作比較,秦漢平均8.8年一次,兩宋為3.5年,元代為 1.6 年,明、清兩代均為2.8年。与此同时,元廷財政體系崩潰,通貨膨漲嚴重,不断收取各种赋税,使百姓的生活更加艰苦,使得白蓮教逐漸流行,並成為對抗元廷的勢力。早在1325年就发生過河南赵丑厮、郭菩萨领导的武裝起事。1338年江西袁州(今江西宜春)彭和尚、周子旺等白莲教徒起义失败,彭和尚逃至淮西。1350年元廷下令變更钞法,鑄造“至正通寶”錢,並大量發行新“中統元寶交钞”,導致物價迅速上漲。隔年元惠宗派賈魯治黃河,欲归故道,動用民伕十五万,士兵二万。而官吏乘機敲詐勒索,造成不滿。白莲教首领韓山童、刘福通等人決定在5月率教眾起事,但事洩,韓山童被捕殺。劉福通再立韓山童之子韓林兒殺出重圍,指韩山童为宋徽宗八世孙,打出“复宋”旗号,以紅巾为标志。其後郭子興於安徽濠州起事,芝麻李等人占領徐州,此為東系紅巾軍。西系紅巾軍方面,彭瑩玉、鄒普勝與徐壽輝在湖北蘄州起事,國號天完。紅巾軍勢力遍佈河南江北、江南、兩湖與四川等地,還有非紅巾軍的张士诚等部的起事,民变揭开元朝灭亡的序幕。於元末民變期間,士人多不屑參加叛軍,叛軍也很少利用士人。

元廷派兵镇压各地紅巾軍,丞相脫脫親自率軍南下攻陷徐州芝麻李軍,一度壓制民變軍。然而脫脫在1354年南攻高邮张士诚軍之際,被元廷大臣彈劾而功虧一簣。徐寿辉部最後分裂成兩湖的陳友諒與四川的明玉珍。兩淮郭子興的部下朱元璋於1356年以南京為根據地開始擴充地盤;1363年與據有兩湖的陳友諒作戰,最後於鄱陽湖之戰獲得勝利;1365年占領兩湖後於同年冬東進攻打據有江蘇沿海的張士誠;1367年平定張士誠後,繼續南下壓制浙江的方國珍,至此江南無一人反抗朱元璋。另外,福建於1357年至1366年間發生色目軍亂,史稱亦思巴奚兵亂。與此同時,元朝在察罕帖木兒和李思齊等率領元軍反擊北方紅巾軍,1363年北方紅巾軍最後在安豐之役中敗給降元後的張士誠,劉福通戰死,韓林兒南下投奔朱元璋,隨後被殺。朱元璋統一江南後於1367年下令北伐,他派徐达、常遇春率明軍分別攻打山東與河南,並且封鎖潼關以防止關中元軍進援中原。明軍于1368年八月攻陷元大都,元惠宗北逃,史書稱此為元朝結束之年。然而元廷仍在上都,往後史書稱之為北元。而明廷认为元惠宗顺天明命,谥号为元顺帝。

元明之際有士人奉元朝為正朔,對元朝皆有故國之情,對於張士誠則有深厚的同情,而對於农民朱元璋則多表厭惡,當時江南士人,不論是否參加張吳政權,或參加朱明政權,乃至獨立人士,都相當懷念元朝。元明之際,由於元代的漢化色彩,漢人文士的華夷之辨觀念極為淡薄,而他們又不滿朱明所為,因而呈現強烈的遺民情結。朱明統治者憑藉紅巾武裝取得政權,在當時正統士大夫看來是“取天下非其道”,難逃僭偽之名,而且元末紅巾運動還帶有濃重宗教色彩,正統士人不僅視其為“賊”、“寇”,而且視之為“妖”。正如紅巾軍於汝陽起事,時人鄭元祐作詩稱“近者汝陽妖賊起,揮刀殺人丹汝水”,1359年,朱元璋部攻杭州,時人陳基記稱“妖寇犯杭”,洪武元年,明軍克大都,戴良作詩感慨“王氣幽州歇,妖氛國步屯”。

明初,不願仕官和不願效忠新朝廷的地主文人為了逃避徵辟而採取自殺、自殘、逃往漠北、 隱居深山等方法,誓不出仕(中國古代銓選,有「身言書判」四方面標準,身體有殘疾者不能任官)。為應對元遺民對明政權的鄙夷與漠視,朱元璋設立深受後人詬病的新刑罰,宣告「士大夫不為君用」律,大規模徵辟前朝遺老、搜羅岩穴隱士,並且殺害許多不願效忠明朝以及為新朝當官的學者:「率土之濱,莫非王臣。寰中士大夫不為君用,是自外其教者,誅其身而沒其家,不為之過」,導致「才能之士,數年來倖存者百無一二,今所任率迂儒俗吏」。

而居於中原的蒙古人則大量留于中原,在明代做官或參軍,史稱「達官」和「達軍」。

1368年元廷退回蒙古草原,元惠宗退至上都,隔年又至应昌。他继续使用“大元”国号,史稱北元。當時北方除了元惠宗據有漠南漠北,關中還有元將王保保駐守甘肅定西,此外元廷還領有东北地区與雲南地區。明太祖為了占領北方,採取兵分二路,各個擊破的方式,此即第一次北伐。元惠宗戰敗后于1370年在应昌去世,元昭宗即位后北逃至漠北和林。明将冯胜夺取了甘肃地区。然而元將王保保仍然在漠北多次与明将徐达等人作戰。明太祖曾多次寫信招降,但王保保從不理会,被朱元璋稱為「當世奇男子」。1378年四月,元昭宗去世,继位的元天元帝继续和明朝对抗,屢次侵犯明境。

至於北元領有的东北地区與雲南地區方面:1371年,元朝辽阳行省平章刘益降明,明朝占領辽宁南部。然而其餘东北地区仍由元朝太尉纳哈出控制,纳哈出屯兵二十万于金山(今辽宁省昌图金山堡以北辽河南岸一带),自持畜牧丰盛,与明军对峙了十几年,多次拒绝明朝的招抚。1387年冯胜、傅友德、蓝玉等人發動第五次北伐,目标是攻占纳哈出的金山。经过多次战争,1387年10月,纳哈出投降蓝玉,明朝占領东北地区。鎮守雲南的元朝梁王把匝剌瓦尔密,在元廷退回草原后仍然繼續忠效之。1371年明太祖派湯和等人領兵平定據有四川的明玉珍,並且勸降梁王未果。1381年12月,明军攻入雲南,1382年梁王逃离昆明並自杀,隨後明军攻克大理,明軍平定雲南地區。

明太祖為了徹底掃除北元勢力,於1388年5月命蓝玉率领明军十五万發動第六次北伐。明军横跨戈壁至捕鱼儿海(今中蒙边境之貝爾湖)擊潰元军,俘虜八萬餘人,元天元帝和他的长子天保奴逃走,但是幼子地保奴被明军擒住,至此北元國勢大衰。1388年元天元帝被阿里不哥后裔也速迭尔杀害(此後去年號,一說去國號),1402年鬼力赤殺元帝坤帖木兒後去國號,明人稱為鞑靼,北元亡。

元朝的前身為蒙古帝國,1206年元太祖成吉思汗成立時領有大漠南北與林木中地區(今貝加爾湖一帶)。經由成吉思汗等蒙古諸汗的經營,以及三次西征之後,蒙古帝國東達日本海與高麗、北達貝加爾湖、南與南宋對峙、西達東歐、黑海與伊拉克地區。成吉思汗時期分疆裂土給東道諸王與西道諸王,東道諸王是成吉思汗的弟弟,大多分封於塞北東部與東北地區,從屬性很高。西道諸王是成吉思汗的兒子,獨立性很好,其中分封長子朮赤於鹹海、裏海以北的欽察草原,後由拔都成立欽察汗國;封次子察合台於錫爾河以北的西遼舊地,史稱察合台汗國;三子窩闊台分封於乃蠻舊地,後由海都建立窩闊台汗國;蒙古本部由幼子拖雷獲得,後由蒙古大汗直轄。至於又稱漢地的華北地區、阿姆河與錫爾河之間的河中地區、伊朗地區與吐蕃由蒙古大汗直轄。1252年拖雷系的蒙哥即位後,命其弟旭烈兀西征西亞,最後建立伊兒汗國,與其他西道諸王合稱四大汗國。命忽必烈經營漢地、最後南滅大理。然而蒙哥於攻宋之役去世,隨後忽必烈與阿里不哥爭位使四大汗國紛紛不受蒙古大汗管制,蒙古帝國至此分裂。

元世祖忽必烈鑒於四大汗國不服於他,於是將西亞地區大汗直轄地割讓給旭烈兀(後來建立伊兒汗國),河中地區大汗直轄地割讓給察合台汗阿魯忽,以換取他們的支持。1279年元世祖在建立元朝後南滅南宋,一統中國地區,當時的疆域是:北到西伯利亚南部,越过贝加尔湖,南到南海,西南包括今西藏、云南,西北至今新疆东部,东北至外兴安岭、鄂霍次克海、日本海,包括库页岛,总面积超过1300萬平方千米。自灭亡南宋後雖然多次對日本、緬甸與爪哇等國有所衝突,然而疆域大体趋于稳定。1309年元武宗時期,元朝和察合台汗国先後攻滅窝阔台汗国,元朝取得窝阔台汗国東部部分领土,領土達1400萬平方公里(如果北方領土延伸至北冰洋,則為2200萬平方公里)。元朝的藩屬國有高麗、緬甸、安南、占城、爪哇及钦察汗国、察合台汗国、與伊儿汗国等國。北有漠北諸部、南有南洋諸國、西有四大汗國。其中有兩個直屬的藩屬國,即高麗王朝與緬甸蒲甘王朝,分別建立征東行省與緬中行省。

西北方面,1268年窩闊台汗國的海都意圖奪回汗位而聯合欽察汗國與察合台汗國反元,史稱海都之亂。直到1304年元成宗時期,元廷與這三大汗国达成和议,並與伊兒汗國一同承認元朝的宗主地位,成為元朝的藩属国,而元朝设立的行政机构(如行中书省和宣政院)也未包括这些领土。而且元成宗并赐伊儿汗国君主刻有“真命皇帝和顺万夷之宝”等汉文印玺,實質上也承認其獨立性。到1309年元武宗時期,元朝和察合台汗国先後攻滅窝阔台汗国,於元文宗年间编纂《经世大典》时,将钦察汗国、察合台汗国與伊儿汗国作为元朝的藩属国。

元朝行政區劃大致上承襲金朝與宋朝制度,然而有兩個不同之處:元朝時的路統轄的面積減少,一路僅轄二州;元廷在路上設有行省等中书省外派單位,最後行省取代路成為一級行政區,形成行省制,这是中国历史上首次正式在全国实行行省制度。元朝行政區劃由高至低依序分為行省、路、府、州與縣,另有等同行省的宣政院辖地、歸中書省直轄的「腹裏」以及等同州的土司。

腹裏是由中書省直轄的路府,宣政院(初名总制院)辖地主管吐蕃地區。行政首長以蒙古人為主、漢人為副。每省設置丞相一員,其下有平章、左右丞相即參知政事官,名稱大略與中書省相同。元代在行省以下各行政區均設置達魯花赤作(斷事官)為地方首長,並以漢人或當地土人為副,以利蒙古人控制地方區域。每路以達魯花赤為主、總管為副各一員。而府州縣均以達魯花赤為主、尹為副。州、縣均分上中下三等,中下州改州尹為知州。土司分有宣慰使、宣撫使與安撫使,於湖廣行省境內設置十五個安撫司,又於湖廣、四川行省分至四個軍。邊區的安撫司和軍,約當內地的下州,也置達魯花赤為主,其副為地方人士。縣以下基層行政區劃設有城關的坊里制與農村的村社制。坊里制於城內分若干片,名曰隅(如東西隅、西南隅之類)。隅下設坊,置坊官、坊司。坊下設里或社,置里正、社長;有的設巷而不設里,置巷長。村社制又稱村疃制度,於縣下設鄉,置鄉長,有的改設里正。鄉之下設都,置主首。都之下設村社,社設社長。

行中书省全称为“某某等处行中书省”,简称“某某行中书省”或“某某行省”,源自金朝的行尚书省。這是基於新征服之地的文化差異太大,所以中央政府就專門設置外派單位來管轄之。由於战争等需求,行省除了負責行政之外也負責軍事,最後逐渐形成一级行政区。早在蒙古時期就設有燕京(華北漢地)、别失八里(西遼等今新疆地區)、阿母河(中亞河中地區)等三断事官或行尚書省。元朝初年的行省管辖范围很大,改变也比较频繁,主要由中书省宰执带相衔临时到某一地区负责行政或征伐事务。1260年,元世祖於國內設置十路宣撫司,次年罷之。隔年改設十路宣慰司,漸成定制,並且設置陝西四川行省。往後直到滅宋為止,大多採行宣慰司與行省並行的制度。行省大多依據西夏、大理疆域與南宋新失之地設置,稱為「中書省臣出行省事」,滅南宋将全国分为中書省直轄的腹裏、宣政院辖地與十多個行中书省,並設置專司征討外國的行省。1321年元英宗時期共設置十一個行省(包含在藩屬國高麗設置的征東行省)。至元朝末年,行省增至十五個。

腹裏:由中书省直辖首都大都附近的中心之地,約今河北、山东、山西及内蒙古部分地区。

宣政院轄地:宣政院除了管理全國佛教事務外,尚管辖吐蕃地区軍政事務,約今青海、西藏。

行中書省:元世祖至元成宗時期設有十個,陕西、辽阳、甘肃、河南江北、四川、云南、湖广、江浙、江西、岭北行中书省。

另外甘肃行省之西的哈密力(今哈密地区)、北庭都元帅府(别失八里)與火州之地不属任何行省管轄。

征討行省分布:

征宋行省:如中统和至元前期的陕西四川行省、河东行省、北京行省、山东行省、西夏中兴行省、南京河南府等路行省、云南行省、平宋战争前后的荆湖行省、江淮行省等。滅宋後定型為一般的行中書省。

征外行省:於高麗設置征東行省(又称征日本行省)、於缅甸(蒲甘王朝)設置缅中行省(又称征缅行省)、於安南(陳朝)設置交趾行省(又称安南行省)、於占城設置占城行省(蒙越戰爭失敗後撤銷)。這些都是臨時性的建置,事畢即罷。只有征東行省,到元朝中期之後,穩定成高麗王的頭銜。行省丞相分别由該國國王或遠征軍主將擔任,自辟官屬,且財賦不入都省,视作藩属国,故與其他行省性質不同。

平亂行省:元末民變時,元廷爲便於鎮壓民變軍,先後於腹裏地區的濟寧(今山東巨野)、彰德(今河南安陽)、冀寧(今山西太原)、保定、真定(今河北正定)、大同等地置中書分省。又分別設立淮南江北行省(至正十二年設於扬州)、福建行省(至正十六年設於福州,後分省泉州、建寧)、山東行省(至正十七年)、廣西行省、膠東行省(至正二十三年)和福建江西行省(至正二十六年)。

另外元末民變的群雄也設置行省以便於統治,如天完之江南行省、汴梁行省、隴蜀行省、江西行省,韓宋之江南行省、益都行省,以及朱元璋所置江西行省、湖廣行省、江淮行省、江浙行省等。

元代行省之下的政区划分十分复杂且时常变化,简单时只存在行省、府州、县三级,复杂时则会出现行省、道(宣慰司)、路(总管府)、府州、县五级的情况。这跟元代“投下封邑”制度息息相关,具体政区分级可能有:

道(宣慰司):元代的道的直接来源即宋金的道路制度。中统三年李璮之乱爆发后,元廷为监察境内汉族世侯,开始仿照宋制设立临时且辖区不定的宣慰司,此时宣慰司多数兼行省相副衔。随着中国的统一,过于庞大的行省已经无法有效处理省内事务,且也有外重内轻之嫌,故至元十五年以后,对宣慰司进行大量的改革,裁撤了宣慰使相副衔并改任行省下属,使之成为辖区固定的行省分支机构及分管区域,其辖区划分也大致与宋金的道路级政区重合。同时由于行省首府附近的地域不设宣慰司,因此产生了直属省部的路州以及分属诸道的路州,但性质上这些都属于“直隶路州”。

直隶路州与封邑型政区:元代直隶于省部或宣慰司道的路州中存在大量的投下封邑型政区,这也是造成元代行政区划层级严重混乱的主要原因。基本上,直隶省部或宣慰司道的路州政区除少数冲要繁盛之地外,都是分封予汉族世侯和蒙古宗室的投下封邑。根据其规模户口的大小,可以分为总管府路、府、州三类,其关系则可参考吴澄所云“皇元因前代郡县之制损益之。郡之大者曰路。其次曰府若州……府若州,如古次国、小国。路设总管府,如古大国之为连率”。

总管府路:总管府路的设置与宣慰司道相似,也是源于宋金的道路制度,但目的性质不尽相同。蒙古初入主中原,以四大世侯为首的汉族地方军阀向蒙廷效忠,蒙廷则依仿金代制度,授予“某路都元帅”“某路都总管”的头衔,确认其在地方的高度世袭自治权,从而建立在汉地的政权机构,是为总管府路之滥觞,此时诸路规模建制与金代诸路相仿,四大世侯为首的有力总管其辖区更大。李璮之乱爆发后,元世祖为削弱地方割据势力,不但开始设置流官监察的宣慰司道,同时也对这类具有封邑性质的总管府路进行拆分,使一路仅辖三至四府州,但并没有改变总管府路封邑的政区性质,而是把它们转封给蒙古宗室,转封过程遵从“画境之制”,尽量使一王之封自成一路。灭宋后,置路以封诸侯的制度也在旧宋属地推行,这次的划分则更加零散,甚至到了“一州自成一路”的状况。

直隶府:除了总管府路的属府属州,一些府因为地处冲要或者以一府为封邑(主要在北方)而直隶于省部或宣慰司。少数人口众多地域广大的直隶散府(如南阳府、汝宁府、归德府等)经过后世的属区调整后更辖属州。直隶府与总管府路相比数量非常稀少,并非投下封邑的主要形态。

直隶州:与直隶府相似,极少数一些地处冲要或以一州为封邑的州(主要在北方)也直隶于省部或宣慰司。比较特殊的状况是,假如一些宗王的封地只有一县(比如蒙古开国功臣畏答儿之孙忽都虎郡王的封邑阳山县)的话,该县一般会升格为直隶州(升为桂阳州)。直隶州的数目比直隶府稍多,但仍远不及总管府路。

封邑型政区与其他直隶路州的最大区别在于达鲁花赤的设置,封邑型政区的达鲁花赤最早不由中央简任,而是由封君选任,作为封君在其封邑的代理人,行使最高决策权,保证封君在封邑的利益,而为了强化中央集权,一般上实际负责路州行政的总管、知府等为朝廷选任。

统县型政区:统县型政区即直接统领县级政区的中层政区,同样分为路(实质上为总管府路之首府即总府,总府所辖县在史料中多记述为直辖于路)、府、州三类,这些政区或作为投下封邑的一部分隶属于总管府路或部分直隶府(称为属府、属州),或作为独立的封邑直隶于省部或宣慰司道。其中属府的数量非常少,主要的统县型政区依然是属州。

元朝與蒙古帝國的皇位繼承異於中國歷代王朝,採取庫力台大會推舉的制度,由王室貴族公推大家的領袖。而元朝皇帝也是兼任蒙古帝國的可汗,由於元世祖的汗位沒有經過庫力台大會的認可,使得四大汗國紛紛不奉正朔,直到元成宗方恢復宗主關係。元世祖建立元朝後,有意立真金為太子,定傳子之局,可惜真金早死而使繼承問題又浮現。元朝而後常因皇太子早死或兄弟爭位而動盪不安,中期又有武仁之約的協定,武宗系與仁宗系交替繼承皇位,然而又因元仁宗廢除協定而再度混亂。元朝的繼承問題直到元惠宗方穩定,但也進入元朝末期。元朝政治制度與金朝一樣承襲宋朝制度,採取文武分權的制度,以中書省總理政務,樞密院掌管兵權。然而元朝的中書省已成為中央最高行政機關,元朝不設置門下省,尚書省時設時不設,僅元世祖時期與元武宗時期有設置,所以門下省與尚書省的權力皆交給中書省。中書省統領六部,主持全国政务,形成明清內閣制的先驅。其組織架構繼承南宋體制,宰相的稱呼共有中書令、司統率百官與總理政務等,常以皇太子兼任。下分左右丞相,中書令缺則總領中書事務。平章政事又居次,凡軍國重事,無不參決。副相方面有左右丞、參政等。六部共有吏部、户部、礼部、兵部、刑部與工部,內有尚書、侍郎。尚书省主要负责财政事务,不过时置时废。枢密院执掌军事,御史台负责督察,與宋朝制度大致相同,然而在地方設有行中書省、行樞密院與行御史台。此外又有掌管學校的集賢院、掌管御膳的宣徽院、掌管驛傳的通政院,其他還有太常禮儀院、太史院、太醫院與將作院,略前代的九寺諸監。最後新成立的是宣政院(初名总制院),负责佛教及吐蕃(今西藏)地区军政事务,這是前代所沒有的。

元朝在推行漢人的典章制度與維護蒙古舊法之間,時常發生衝突,並且分裂成守舊派與崇漢派。早在元太祖成吉思汗攻佔漢地後,有賴耶律楚材與木華黎推行漢法以維護其典章制度。當時近臣別迭建議將汉人驱赶並把中原变成大牧场以收取財富,遭到耶律楚材的反對,他認為可用徵稅的方式獲得財富,因此保留了漢地的典章制度。他積極改變蒙古軍以往「凡攻城邑,敵以矢石相加者,即為拒命,既克,必殺之。」的作風,努力興科祟儒、整頓吏治,實為漢法推行之祖。木華黎為了便於管理漢地,也於漢族四大世侯合作,逐漸鞏固了對河北、山西等地的治理。

後來管理漢地的元世祖忽必烈也積極推動漢法,任用了大批漢族幕僚和儒士等创设典章制度,如劉秉忠、許衡和姚樞等,並提出了「行漢法」的主張。積極推動了學習漢文的熱潮。如元世祖就非常熟悉漢文典籍、禮儀制度,並能用漢文創作詩歌,並且還以法律的形式規定,太子必須學習漢文。接受儒士元好問和張德輝提議的「儒教大宗師」稱號。忽必烈最後在大都建元稱帝,創建中國式的元朝,建立了一套以傳統中國中央集權作藍本的政治體制,例如设立了三省六部和司农司等一系列专司机构,使用中原的统治机构来统治人民,任劉秉忠等人的规划建立首都大都。然而,元世祖在李璮叛亂後,對漢人的信任下降。而四大汗國以及守舊派蒙古王室都不滿元世祖行漢法的舉動,或叛變或疏遠之。元世祖晚年也漸與儒臣疏遠,任用阿合馬、盧世榮與桑哥等色目人與漢人為首的理財派,漢法最後未成為一套完整的體系。後來的元仁宗、元英宗、元文宗與元惠宗等人更是可以純熟地運用漢文進行創作。一些入居中原的蒙古貴族,羨慕漢文化,還請了儒生當家庭教師教育子女。為了學習方便還翻譯了許多漢文典籍,諸如《通鑒節要》、《論語》、《孟子》、《大學》、《中庸》、《周禮》、《春秋》、《孝經》等。但崇漢派與守舊派時常發生衝突與政變,例如南坡之變等。

在人才選用方面,元朝雖然许多制度都沿袭了宋朝,但關於科舉,元朝前期並沒有常態化的定期舉辦科舉,因此高級官僚的錄用端看與元廷關係遠近而決定,主要採取世襲、恩蔭與推舉制的方式。此外尚有循胥吏(小公務員)昇進為官僚的方式,這與宋朝制度大異。宋朝官與吏的界限分明,胥吏大多以胥吏為終,然而元朝因為缺乏科舉取才,就以推舉或考試胥吏的方式晉升為官,這打破官吏屛障,使官吏成為上下的關係。科舉選材方面,窩闊台汗聽從耶律楚材建議,召集名儒講經於東宮,率大臣子弟聽講。又置“編修所”於燕京,“經籍所”於平陽,倡導學習漢族古代文化,又在1234年設“經書國子學”,以馮誌常為總教習,命侍臣子弟 18人入學,學習漢文化。並且於1238年以術忽德和劉中舉辦戊戌選試,此次科举取士录取4030人,並且建立儒户以保護士大夫。但最後仍廢除科舉,改採推舉制度,往後於1252年與1276年兩次共入選3890儒户。元世祖忽必烈即位後,正式設立了國子學,以河南許衡為集賢大學士兼國子祭酒,親擇蒙古子弟使教之,遍學儒家經典文史,培養統治人才。1289年元世祖下诏登記江南人口户籍,次年正式施行推舉制度,此次登記成为后来户计的依据。直到1313年,提倡漢化運動的元仁宗下诏恢复科举,元仁宗恢复科举,由程钜夫、李孟、许师敬拟定元朝科举制度。1314年八月在全国的17处考场,举行乡试,1315年二月和三月相继在大都举行会试和殿试(廷试),因为是在延祐年间举行的,史称“延祐復科”,本次科舉以程朱理學為考試的內容。榜分左右兩榜,官位相同,第一名從六品,第二名以下及第二甲,皆正七品,进士三甲以下都能授正八品官员,如1238年戊戌选试的状元杨奂,1315年的乙卯科左榜状元张起岩。元朝前後共舉行過16次,選舉蒙古、色目、漢人、南人進士約 1100余人。蒙古、色目人應舉者遠遠少於漢人、南人。然而為了保障蒙古人與色目人的名額,實行難度不同的「分榜取士」,並且給蒙古人與色目人保留了超過其應舉比例的名額,這也讓蒙古與色目子弟失去了學習漢族文化的積極性和進取精神。《元统元年进士录》的记载称四等人名额相等,各25人,但读书人总数确实南人、汉人要远多于蒙古、色目,因此也有破例,如延佑首科的录取名额给左榜的要远多于右榜。雖然是聊勝於無的科舉,但在形式上已經恢復,且持續坚持下去。原來放弃科举的士子重新獲得了入仕機會,因此漢族士大夫莫不對元廷忠心耿耿。在元朝滅亡之際,捨身殉國的就有很多是科舉出身者,可見科舉復辦對懷柔漢族士大夫有一定效果。

元朝時與各國外交往來頻繁,各地派遣的使節、傳教士、商旅等絡繹不絕,其中威尼斯商人尼可羅兄弟及其子馬可波羅成為得到元朝皇帝寵信,在元朝擔任外交專使的外國人。元廷曾要求周边一些国家或地区(包括日本、安南、占城、缅甸、爪哇)臣服,接受与元朝的朝贡关系,但遭到拒绝,故派遣军队进攻攻打这些国家或地区,其中以元日戰爭最为著名,也最惨烈。

東北方面有高麗王朝與日本鎌倉幕府。高麗王朝領有朝鮮半島,之後被崔氏政權統治,高麗王變成傀儡。高麗先後臣服於遼朝與金朝,蒙古興起後與高麗共同伐金,並約為兄弟之國。1225年蒙古要求高麗向其朝貢,蒙古使節抵達義州邊境時,被高麗所害,當時蒙古忙於西征,無暇征討。1231年窩闊臺汗派撒禮塔率兵入侵高麗,崔氏政權領袖崔瑀抵禦失敗,高麗首都松都(今開城)被攻陷,史稱高麗蒙古戰爭。蒙軍設置多位達魯花赤以監督高麗政事。隔年崔瑀殺死達魯花赤,擁護高麗王高麗高宗從松都遷往江華島,並且長期抗蒙,另外三別抄軍抵抗蒙軍至1273年。然而高麗朝廷分裂成反戰的文派,與抗蒙的崔氏政權。貴由、蒙哥時又四次討伐掠奪高麗地,1258年崔氏政權被顛覆後,高麗高宗遣子稱臣,正式成為蒙古的藩屬國。1283年元世祖為了討伐日本,於高麗國設置征東行省,高麗王為行省的左丞相,內政受蒙古人控制。高麗君主從忠烈王開始娶蒙古公主為妻,高麗君主繼承人按照約定,必須在元大都以蒙古人的方式長大成人後,方可回高麗。高麗成為元朝的藩屬國後,元世祖六次遣使者要求日本朝貢,均告失敗,於是發起元日戰爭。1274年元军發動第一次侵日戰爭,,日本史書稱為“文永之役”,元廷派三萬二千餘人東征日本,最後因為颱風侵襲而傷亡慘重。1281年七月,忽必烈又發動第二次侵日戰爭,日本史書稱為“弘安之役”,由范文虎、李庭率江南軍十餘萬人,到達次能、志賀二島,因日軍積極抵抗,且元軍又遇到颱風,最後再度慘敗。通常认为台风(日本人称之为“神风”)與元軍不擅水戰是造成失败的最大原因(另一方面高麗和南宋工匠故意製作式樣錯誤的戰船)。而後元世祖又準備第三次東征,因大臣勸阻,再加上出兵安南的緣故而罷。而後元世祖多次遣使均遭日本拒絕,通使关系一直未能建立,但是元朝與日本的经济和文化交流仍然十分繁盛,来元日本人以商人與禅僧最多。元廷令沿海官司通日本国人市舶,主要港口是庆元(今寧波)。

南洋諸國有安南(陳朝)、占城與爪哇(滿者伯夷)等國。安南國據有今越南北部,於五代北宋時期獨立於中華。蒙古大汗蒙哥於1257年派兀良哈台南攻安南,蒙越戰爭爆發。越南陳太宗被蒙軍擊敗,上表稱臣,蒙哥封為安南國王,而越南陳聖宗繼位後不願向元朝稱臣。當時在安南南方還有占城國,1282年占城國王因陀羅跋摩六世遣使朝貢,元世祖因此設置荊湖占城行中書省,以阿里海牙為該行省的平章政事。由於占城王扣留元使,元世祖藉此發兵分水陸攻打占城與安南。他以唆都率水軍由廣州渡海攻打占城。隔年蒙古水軍攻下占城國王據守的木城,占城國王因陀羅跋摩六世求和,但於蒙古退軍後殺使者。1284年元世祖再派鎮南王脫歡、阿里海牙與唆都率陸軍借道安南南征占城,被時任太上皇的陳聖宗反抗而爆發戰爭。元軍大舉入侵,占領安南國都。但陳聖宗、陳興道率領的陳軍積極抵抗,並且瘟疫四竄。最後元軍於1285年撤退,途中遭安南軍襲擊,損失過半。而後1288年又南征失敗,隨後安南請和。這場戰爭至元成宗才廢止,安南與占城相繼入貢元廷。當時南洋群島諸國,也多貢於元朝。有名的有馬蘭丹(今馬六甲)、蘇木都拉(今蘇門答臘)等。1292年元世祖命亦黑迷失、史弼與高兴率福建水軍南征爪哇滿者伯夷王國,並降其鄰國葛郎(爪哇島以東),但中計受突擊,戰敗而還,以後爪哇仍然派使朝貢。此外元世祖亦派使者招降琉求國,然使者僅至澎湖而返。

西南地區有大理國、吐蕃、緬甸(蒲甘王朝)、八百媳婦國(蘭納泰王國)與暹邏。大理源自唐朝的南詔,937年由段思平滅南詔建國,占有現今雲南地區,後由高昇泰等高氏政權掌控。1252年蒙哥汗命忽必烈與兀良合台自四川迂迴南滅大理,原大理國王段氏被任為大理世襲總管。吐蕃自晚唐就走向衰退,但其境內藏傳佛教(又被汉人贬稱为喇嘛教)日漸興盛,喇嘛的勢力超過贊普(吐蕃王)的地位。1247年窩闊台汗次子闊端召請喇嘛班智達來涼州,史稱涼州會盟,此後吐蕃喇嘛與蒙古大汗形成了布施關係(詳見元朝治藏歷史)。忽必烈南征大理時,分兵伐吐蕃,喇嘛班智達與贊普同時投降,吐蕃亡。元世祖封班智達的繼承人八思巴為「帝師」,兼任總制院(後改為宣政院)院使,取得了統治烏思藏地區的權力,使西藏統治者由贊普轉為喇嘛。緬甸為唐朝的驃國,宋朝以後稱緬,國內部落稱甸,所以又稱緬甸。元朝初期緬甸為蒲甘王朝,其王朝西併阿剌干(今孟加拉灣一帶),南併勃固(今仰光以北),並進占暹羅。元世祖派使招降不從,緬甸反派軍入侵雲南,元緬戰爭爆發,而後元兵又多次進攻緬甸。1283年元世祖派軍入侵緬甸,兩年後緬甸王請和。1287年緬甸內亂,元軍乘機進攻緬甸,蒲甘城破,緬甸成為元朝的藩屬,緬甸王那羅梯訶波帝失去王位,元廷建緬中行省,而後以蒲甘國王任行省左丞相,成為元朝傀儡。1368年撣族於緬甸東部阿瓦建立阿瓦王國,首領為阿散哥。孟族建都於馬達班,1369年遷都勃固,建立勃固王朝,二王國南北交戰。撣族阿散哥挾持緬甸王,使元成宗派元軍討伐,最後迫使阿散哥派使朝貢。蘭納泰王國(元人稱八百媳婦國)位於撣族東邊的金三角,曾聯合阿散哥抵抗元軍,元廷多次討伐未果,直到元泰定帝時才內附。暹羅地區原有素可泰王朝(元人稱暹國)、大城王國(元人稱羅斛)以及其他小國。暹國曾擴張其勢力於馬來半島,元成宗後遣使進貢八次。羅斛自元世祖末年就開始進貢,並於元末時期併吞暹國等小國,統一為暹羅國。

蒙古帝國的三次西征的同時,正值羅馬教皇提倡十字軍東征西亞的伊斯蘭國家以收復耶路撒冷。由於羅馬教皇急需外援以抗衡伊斯蘭教徒,而歐洲基督教國家剛剛經歷蒙古第二次西征,再加上東西交通十分便利,紛紛派使者東行了解這個東方大國。1245年羅馬教皇曾派柏朗嘉賓經欽察汗國到和林謁見貴由汗,返國著成《柏朗嘉賓蒙古行紀》。1253年法國國王路易九世派魯布魯克以傳教為名到和林進見蒙哥汗,返國著有《魯布魯克東行紀》。1316年義大利人鄂多立克經海路至元大都,參加了元泰定帝的宮廷慶典,回國著成《鄂多立克東遊錄》,範圍遠達西藏,對元大都及宮廷的描寫較細。最著名的是義大利探險家馬可波羅,他隨經商的父親、叔父於1275年到元朝進見元世祖,直至1291年才離去。他擔任元廷官吏,歷游元朝各地,其著寫的《馬可波羅遊記》對元朝進行多角度反映,吸引歐洲人東行中國。另外元朝與非洲地區諸國也有來往,汪大淵在1330年和1337年二度飄洋過海親身經歷的南洋和西洋二百多個地方的地理、風土、物產,最後著成《島夷誌略》,影響明代初期的鄭和下西洋。

元朝军队按照親疏關係分成蒙古军、探马赤军、汉军與新附军等四個等級。蒙古軍與探马赤军主要是骑兵。汉军、新附军大多为步军,也配有部分骑兵。水军编有水军万户府、水军千户所等。炮军由炮手和制炮工匠组成,编有炮手万户府、炮手千户所,设有炮手总管等。一部分侍卫亲军中,还专置弩军千户所,管领禁卫军中的弓箭手。

蒙古军是元朝軍隊的骨幹,主要由蒙古族組成。蒙古軍早在成吉思汗統一蒙古時即創立,平时分布在草原上驻牧,战时临时招集。採用兵民合一的萬户制,按十进制编组成十户、百户、千户。只要是十五歲至七十以內的人皆服兵役,其童子稍微年長者也組成「漸丁軍」。元朝時期在汉地和江南军户中签发丁男应役。探馬赤軍又名簽軍,随着战争的发展,统治者需要一支蒙古军队长期留守被征服地区,于是从蒙古各部中“签发”了部分士兵,组成专门用于镇戍的探马赤军。自1217年木華黎討伐金朝時建立,由弘吉剌、兀魯兀、忙兀、札剌亦兒及亦乞烈思五部組成,西征花剌子模後回族、維吾爾族與突厥族等族成為探馬赤軍的一部分。探馬赤軍精於火砲與西方的回回砲,攻城力強。「下馬則屯聚牧養,上馬則備戰」。

汉军是蒙古帝國占領漢地後發民為兵,主要由金朝女真與契丹降军、早期降蒙的南宋軍、漢地的地方漢族武裝勢力與签发漢地百姓等所組成。窩闊台汗於1229年收編金朝女真與契丹降軍,在漢地民戶中大規模簽發士兵,補充漢軍兵員,將蒙古軍的編製和官稱用於漢軍系統強。各漢軍萬戶統軍人數不等,「大者五、六萬,小者不下二、三萬」。漢軍有「舊軍」與「新軍」的區別。舊軍主要指敵國降軍和地方武裝勢力,新軍指從漢地百姓簽發的新兵。元世祖忽必烈即位後,蒙元帝國的統治重心由漠北草原移到了中原漢地。元世祖對軍隊體制進行改革,逐步建成中央宿衛軍和地方鎮戍軍兩大系統,確定了元軍的編製和隸屬關係,在元朝對外戰爭中,漢軍發揮了重要的作用。新附军主要是元朝南征南宋期間收邊的降軍,又被稱為新附漢軍、南軍等。新附軍內名號繁雜,而是元廷因士兵所具不同特點而起的名稱,如券軍、手號軍與鹽軍等等。估計當時新附軍的數量在二十萬人上下,元帝將新附軍分編到元軍的侍衛軍和鎮戌軍中;或以蒙古、漢人、南人建立新的軍府,管領新附軍人。每當有戰事發生,首先調發各軍中的新附軍出征,其餘則從事屯田和工役造作。經過多年的戰爭消耗和自然減員,新附軍數量日益減少,最後式微。

元朝的防衛分宿衛和鎮戍兩大系統。宿衛軍由怯薛和侍衛親軍構成,其中怯薛軍保留自成吉思汗創立的四怯薛番直宿衛,常額在萬人以上,元朝功臣博爾忽、博爾朮、木華黎、赤老溫或其後人充任怯薛長。在戰爭中,怯薛則是全軍的中堅力量,被稱之為「也客豁勒」(大中軍);侍卫亲军则是忽必烈在华北汉人世侯的建议下所置,在初期蒙制怯薛未形成战斗力之时负责宿卫之职以及与阿里不哥争夺权力。其后,侍衛親軍用於保衛大都,衛設都指揮史或率史,隸屬於樞密院。鎮戍軍由蒙古軍和探馬赤軍守衛靠近京畿的要地,華北、陝西、四川的蒙古軍、探馬赤軍由各地的都萬戶府(都元帥府)統領,隸屬於樞密院。南方以蒙古軍、漢軍、新附軍共同駐守,防禦重點是江淮地區,隸屬於各行省。鎮戍諸軍,有警時由行樞密院統領,平時日常事務歸行省,但調遣更防等重要軍務則歸屬樞密院決定。

元朝水軍原是為了元滅宋之戰而準備,1270年命劉整建造大量水軍。襄樊之戰時元朝水軍與陸軍協同包圍襄陽,攻下後降將呂文煥又率元水軍與河岸陸軍協同於丁家洲之戰擊潰南宋水軍精銳,至此領有全部長江水域。而後張弘範又率元朝水軍(平底船)渡海南下追擊南宋海軍,最後於崖山海戰包圍殲滅之,元朝水軍在滅宋之戰有重要的功能。元朝融合了南宋和阿拉伯航海技术,使海軍技術更加成熟,然而在對外戰事中,元日戰爭與元爪戰爭均以失敗結束,而且對日戰爭兩次均被颱風所毀,只有對占城的戰役獲勝而已。

早在蒙古時期,北方人口就不斷的南逃,總人數約占北方人口的十分之一,這種現象到惠宗时都還持續發生,元廷屢禁而不能止。在大蒙古國征服金朝期間在戰地进行了大规模屠杀和掠夺。随后的瘟疫与饥荒导致東亞地區大量人口消失,其中又以金朝的華北和南宋的川陕四路十分严重。这是导致“湖广填四川”移民运动发生的重大原因[需要更好来源]。

1234年3月9日金灭亡后,華北地區約有110万户與600万人,只有1208年的金朝人口5353万的13\%。蒙古宋战争期間,南宋境内因战争总计消灭了大约1500万人口,主要集中在川陕四路地区。1279年元军完全剿灭四川的抗元勢力後,在1280年的户口调查仅为9万余户與50万余人,只有1231年蒙古入侵川陕四路地区前的4\%。大蒙古國時期有過兩次戶口統計,先有1235年窝阔台汗推行的乙未籍户,獲得華北地區如燕京(今北京)、顺天(今河北保定)等三十六路的人口資料,後有1252年蒙哥汗完成的壬子籍户,顯示華北人口略有增加。1271年元世祖建国号为大元。雖然在元成宗到元惠宗至正初年期間政治動盪不安,尽管每年也成百上千次人民起义,但社會上基本處於安定狀態,經濟大體上也是呈現增长的狀態,這些都促使人口增长,大約在惠宗至正十年(1351年)達到高峰。元惠宗至正年間(1341年-1370年)全國發生多次大規模的災荒饑饉疾病和瘟疫,最終促使紅巾軍起義爆發。红巾军起义之后又造成人口大量減少。明太祖建國後論到:「前代革命之际,肆行屠戮,违天虐民,朕实不忍。」

元代戶口統計並不是准确,无法涵盖的人口包括逃戶、因土地兼併而蔭蔽的隱戶、流民以及私属人口等。朝廷不納入戶口統計的人口包括:嶺北等处行中书省、雲南等处行中书省、西南土司地區和宣政院轄地的居民;蒙古諸王、貴族、軍將的大量私屬人口(驅口、投下戶,怯憐口、打捕鷹房人戶);獨立於州縣以外的諸色戶計(軍戶、站戶、匠戶、民屯戶、釋、道、儒戶、游食者)等。現在歷史學者只能根據史書的原始數據與他們掌握的歷史資料的來推斷,所以差異甚大,僅作參考。人口逃亡的现象很严重,如1241年,忽都虎等元籍诸路民户1,004,656户,逃户即达280,746户,占全部人户的28%。另外,隨著民族關係日益密切,往來與雜居也相當普遍。從蒙金战争时期就陸續有大批漢人被迁往蒙古草原以及天山南北、遼陽等处行中书省與雲南等处行中书省各地;蒙古與色目官員、軍戶、商人等也大量移居中原內地;雲南地區居住的蒙古人約十萬人左右;大都、上都等政治城市及杭州、泉州、鎮江等商業城市都居住許多蒙古人、畏兀儿(維吾爾祖先)、穆斯林、黨項人、女真人與契丹人等,促成民族之間經濟文化的交流。

「四等人制」:有說法認為由於蒙古人與漢人的人數比例極不平均,元廷為了保護蒙古人地位,主張蒙古至上主義,推行蒙古人、色目人(包括西域各族和西夏人)、汉人(原金朝统治下的人民)、南人(南宋统治下的漢人)等四個階級的制度,但該制度并不见于官方文告及档案。有學者認為,元廷給蒙古人與色目人極大的權利,並讓汉人與南人負擔較大的賦稅與勞役,民族压迫和阶级压迫十分沉重。尽管学术界迄今并没有发现元代有把臣民明确划分为四等的专门法令,但元廷对于各民族的不平等态度却反映在一些政策和规定中,例如汉人打死怯薛需要偿命,而怯薛打死汉人只需「断罚出征,并全征烧埋银」(原文為怯薛歹蒙古人,怯薛歹為元代一特權階級)。此外汉人做官也往往只能做副貳(雖然實際上存在很多例外情況,終元一代朝廷仍任用不少漢人為官,如史天澤、贺惟一等)。

「九儒十丐」:有說法認為「九儒十丐」是元朝的定制,顯示出在蒙古統治下儒士在社會的下等地位。此「九儒十丐」的說法來自南宋遺民謝枋得,其〈送方伯載歸三山序〉云:「滑稽之雄,以儒為戲者曰:『我大元制典,人有十等:一官、二吏;先之者,貴之也,貴之者,謂有益於國也。七匠、八娼、九儒、十丐;後之者,賤之也,賤之者,謂無益於國也。』嗟乎卑哉!介乎娼之下,丐之上者,今之儒也。」及同樣是南宋遺民的鄭思肖〈大義略序〉曰:「韃法,一官、二吏、三僧、四道、五醫、六工、七獵、八民、九儒、十丐。」但因其政治立場,並不能完全盡信,或作為元朝儒士社會地位低下的佐證。中外史學界已有學者對元代儒士的地位問題進行過深入的研究,否定了元代儒人地位低落的說法。

元代经济呈现多元格局,经济活跃发达,大致上以农业为主,有学者认为其整體生產力雖然不如宋朝,但在生产技术、垦田面积、粮食产量、水利兴修以及棉花广泛种植等方面都取得了较大发展。蒙古人原来是游牧民族,草原时期以畜牧为主,经济单一,无所谓土地制度。蒙金战争时期,大臣耶律楚材建议保留汉人的农业生产,以提供財政上的收入来源,这个建议受到铁木真的采纳。窝阔台之后,为了巩固对汉地统治,实行了一些鼓励生产、安抚流亡的措施,农业生产逐漸恢复。特别是经济作物棉花的种植不断推广,棉花及棉纺织品在江南一带种植和运销都在南宋基础上有所增加。经济作物商品性生产的发展,就使当时基本上自给自足的农村经济,在某些方面渗入了商品货币经济关系。但是,由於元帝集中控制了大量的手工业工匠,经营日用工艺品的生产,官营手工业特别发达,对民间手工业则有一定的限制。

由于蒙古对商品交换依赖較大,同时受儒家轻商思想较少,故元朝比較提倡商業,使得商品经济十分繁荣,使其成为当时世界上相當富庶的国家。而元朝的首都大都,也成為當時闻名世界的商业中心。为了适应商品交换,元朝建立起世界上最早的完全的纸币流通制度,是中国历史上第一个完全以纸币作为流通货币的朝代,然而因濫發紙幣也造成通貨膨脹。商品交流也促进了元代交通业的发展,改善了陆路、漕运,内河与海路交通。

農業方面,宋真宗时推行的占城稻在元朝時已經推廣到全國各地。农业生产继续发展,1329年,南粮北运多达三百五十多万石,说明粮食生产的丰富。这一阶段,经济作物也有较大发展,茶叶、棉花與甘蔗是重要的经济作物。江南地区早在南宋時已盛產棉花,北方陕甘一带又从西域传来了新的棉种。1289年元廷设置了浙东、江东、江西、湖广、福建等省木棉提举司,年征木棉布十万匹。1296年复定江南夏税折征木棉等物,反映出棉花种植的普遍及棉纺织业的发达。元朝水利設施以華中、華南地區比較發達。元初曾设立了都水监和河渠司,专掌水利,逐步修复了前代的水利工程。陕西三白渠工程到元朝后期仍可溉田七万余顷。所修复的浙江海塘,对保护农业生产也起了较大作用。元朝農業技術繼承宋朝,南方人民曾采用了圩田、柜田、架田、涂田、沙田、梯田等扩大耕地的种植方法,對於生产工具又有改进。关于元朝的农具,在王祯的《农书》中有不少詳細的敘述。

元世祖為了清查土地徵收賦稅曾實行過土地所有者自報田地的經理法,由於未能確實執行,1314年元仁宗又派大臣往江浙、江西、河南三地實施經理法,但實施結果仍然弊端極多,人民紛起反抗,以至仁宗不得不下詔免三省自實田租二年,最後不了了之。

元朝土地仍可分为官田和私田两种。官田主要来自宋、金的官田,两朝皇亲国戚、权贵、豪右的土地,掠夺的民田,以及经过长期战乱所形成的无主荒地。元廷把所掌握的官田一部分作为屯田,一部分赏赐王公贵族和寺院僧侣,余下的则由政府直接招民耕种,收取地租。其屯田的數量極大,遍及全國,其中以河北、河南兩省最多。其中民屯是役使汉人屯垦收租,军屯则分给各军户,强迫相当于奴隶的“驱丁”耕种。私田是蒙古贵族和汉族地主的占地以及少量自耕农所有的田地。元朝以大量土地赏赐寺院,例如1316年元仁宗曾赐给上都开元寺江浙田二百顷、华严寺百顷。元朝也有一定数量的自耕农,然而地位很低下,生活十分困苦。

元朝的畜牧政策以开辟牧场,扩大牲畜的牧养繁殖為主,尤其是孳息马群。畜牧业发展趋势不稳定,由元世祖时的盛况渐渐趋向衰退,到了元惠宗時,畜牧业的衰败更为严重,其原因最大的是自然灾害。元朝完善了养马的管道,设立太仆寺、尚乘寺、群牧都转运司和买马制度等制度。元朝在全国设立了14个官马道,所有水草丰美的地方都用来牧放马群,自上都、大都以及玉你伯牙、折连怯呆儿,周回万里,无非牧地。元朝牧场广阔,西抵流沙,北际沙漠,东及辽海,凡属地气高寒,水甘草美,无非牧养之地。当时,大漠南北和西南地区的优良牧场,庐帐而居,随水草畜牧。江南和遼東諸處亦散滿了牧場,早已打破了國馬牧於北方,往年無飼於南者的界線。內地各郡縣亦有牧場。除作為官田者以外,這些牧場的部分地段往往由奪取民田而得。

牧場分為官牧場與私人牧場。官牧場是12世紀形成的大畜群所有制的高度發展形態,也是大汗和各級蒙古貴族的財產。大汗和貴族們通過戰爭掠奪,對所屬牧民徵收貢賦,收買和沒收所謂無主牲畜等方式進行大規模的畜牧業生產。元朝諸王分地都有王府的私有牧場,安西王忙哥剌,佔領大量田地進行牧馬,又擴占旁近世業民田30萬頃為牧場。雲南王忽哥赤的王府畜馬繁多,悉縱之郊,敗民禾稼,而牧人又在農家宿食,室無寧居。1331年以河間路清池、南皮縣牧地賜斡羅思駐冬。元世祖時,東平布衣趙天麟上《太平金鏡策》,云:今王公大人之家,或占民田近於千頃,不耕不稼,謂之草場,專放孳畜。可見,當時蒙古貴族的私人牧場所佔面積之大。

嶺北行省作為元朝皇室的祖宗根本之地,为了维护诸王、贵族的利益和保持国族的强盛,元帝对这个地区给予了特别的关注。畜牧业是岭北行省的主要经济生产部门,遇有自然灾害发生,元朝就从中原调拨大量粮食、布帛进行赈济,或赐银、钞,或购买羊马分给灾民;其灾民,也常由元廷发给资粮,遣送回居本部。元帝对诸王、公主、后妃、勋臣给予巨额赏赐,其目的在于巩固贵族、官僚集团之间的团结,以维持自己的皇权统治。皇帝对蒙古本土的巨额赏赐,无形中是对这一地区畜牧业生产的投资。

元朝手工業生產也有些進步,絲織業的發展以南方為主,長江下游的絹,在產量上居於首位,超過了黃河流域。元朝的加金絲織物稱為「納石矢」金錦,當時的織金錦包括兩大類:一類是用片金法織成的,用這種方法織成的金錦,金光奪目。另一類是用圓金法織成的,牢固耐用,但其金光色彩比較暗淡。棉纺织业到宋末元初起了变化,棉花由西北和东南两路迅速传入长江中下游平原和關中平原。加上元朝在五个省区设置了木棉提举司,“责民岁输木绵(棉)十万匹”,可见长江流域的棉布产量已相当可观。但当时由于工具简陋,技术低下,成品尚比较粗糙。1295年前后,婦女黄道婆把海南岛黎族的纺织技术带到松江府的乌泥泾,提升了纺织技術,被尊称为黄娘娘。

元朝的瓷器在宋代的基础上又有进步,著名的青花瓷就是元代的新产品。青花瓷器,造型优美,色彩清新,有很高的艺术价值。造船業十分發達,还有起碇用的轮车,并已经使用罗盘针导航。元朝的印刷技术,又比宋朝更有进步。活字印刷术不断改进,陆续发明了锡活字和木活字,并用来排印蒙文和汉文书籍。自1276年以来,已使用小块铜版铸印小型的蒙文和汉文印刷品,如纸币“至元通行宝钞”。套色版印刷术应用于刻书,如中兴路刊印的无闻和尚注《金刚经》。1298年王禎用木活字来印他所纂修的《大德旌德县志》,不到一月百部齐成,其效率很高。他又发明了转轮排字架,使用简单的机械,提高排字的效率。最後他總結成《造活字印书法》。

元朝行会组织还有应付官府需索、维护同业利益的作用,其組織的内部还更日趋周密。在元朝,“和雇”及“和买”,名义上是给价的,实际上却给价很少,常成为非法需索。虽然各行会多由豪商把持,对中小户进行剥削,但是由于官府科索繁重,同业需要共同来应付官府的需求,同时官府也要利用行会来控制手工业的各个行业。

元朝透過专卖政策控制盐、酒、茶、农具、竹木等一切日用必需品的贸易,影响国内商业的发展。可是元朝幅员广阔,交通发达,所以往往鼓勵对外贸易政策,因而终元之世对外贸易颇为繁盛。元朝的对外贸易主要采取官营政策,并禁止汉人往海外经商。但实际上私商入海贸易的仍然很多,政府始终无法禁绝。元代海外贸易输出入商品,大体上与宋代相同。但奴隶贸易却有相当规模,贩运进口的有“黑厮”和“高丽奴”。

在生产发展的基础上,物资交流频繁,从而促进了商业城市的发展。元朝時临安仍改名杭州,其繁荣并不因南宋覆灭而衰退多少。由于北方人纷纷南迁,城厢内外人口更加稠密,商业繁荣。杭州是江浙行省的省会,地位重要,水陆交通便利,驿站最多,不但是南方国内商业中心,也是对外贸易的重要港口之一。江浙行中书省居各行中书省征收的商税和酒醋课的第一位,城内中外商民荟萃,住有不少埃及人和突厥人,还有古印度等国富商所建的大厦。泉州在宋元時期是東方第一大港,貨物的運輸量十分巨大,泉州的稅收僅次於前朝首都杭州。然而在元朝末年色目軍爆發亦思巴奚兵亂,導致外僑大量撤離,對外貿易中斷而衰。大都(今北京)是元朝的首都,在原来中都城的东北方建立新城,规模宏大,是全国政治、军事中心,也是陆路对外贸易和国内商业中心。达官贵人、富商大贾多在此聚居,人口稠密,城厢内外街道纵横,商肆栉比鳞次,工商业很繁荣,是世界闻名的大城市。州县以上的城市,商业比较发达的还有:

长江下游和苏浙闽等地区的建康(南京)、平江(苏州)、扬州、镇江、吴江、吴兴、绍兴、衢州、福州等城市;

长江中游地区的荆南、沙市、汉阳、襄阳、黄池、太平州、江州、隆兴等城市;

长江上游川蜀地区的成都、叙州、遂宁等城市;

沿海对外贸易城市的广州、泉州、明州、秀州、温州和江阴等等。

元朝為了加強對經濟的統制,以使用紙幣為主,鑄造錢幣比其他朝代為少。1260年元世祖發行了以絲为本位的寶鈔與以白銀或金為本位的中統鈔(中統鈔没有设定流通期限),鈔幣持有者可以按照法令比價兌換銀或金,虽然其后曾一度废除,但持续使用到元朝末期,成为元朝货币的核心的纸币。全國各路都設有兌換的機關——「平準庫」。兌換基金充足,准許兌現,兑换的时候征收两到三分的手续费(工墨鈔)。1276年由於元廷大肆搜括,增發紙幣,並將各路準備金銀運往大都,引起物價上漲,紙鈔貶值。1280年,紙幣貶值成為原來的十分之一。1287年物價已經「相去幾十餘倍」了。為了穩定物價,元廷發行「至元寶鈔」和中統鈔並行。1350年元惠宗又發行「至正寶鈔」,發行不久,貶值嚴重,物價暴漲。事實上,民間的日常交易、借貸、商品標價等多有用銀的。這時使用的白銀,主要是銀錠和元寶。

元代的賦稅依舊包括田賦、開採礦產的歲課、鹽稅等。但由於元代商業發達,商稅亦成為了政府的重要收入之一

關於元朝的田賦,《元史·食貨志一》說:「元之取民,大率以唐為法。其取於內郡者曰丁稅,曰地稅,此仿唐之租庸調也;取於江南者曰秋稅,曰夏稅,此仿唐之兩稅也。」這段話雖然並不確切,但至少說明了南北田賦制度的差異。中原田賦的徵收大概始於耶律楚材輔政以後。在這之前蒙古帝國根本沒有賦稅之制。元朝行於江南的田賦制度基本上沿用了宋代的兩稅制。

元朝人民還有一項很沉重的財政負擔,即科差,是徭役向賦稅轉化的一種形式。

元朝統治中原,對中原傳統文化的影響大過對社會經濟的影響。像遼朝、金朝與西夏等征服王朝,他們為了提升本國文化,積極的吸收中華文化,進而逐漸漢化,然而蒙元對漢文化卻不甚積極。他們主要是為了維護本身文化,同時採用西亞文化與漢文化,並且提倡蒙古至上主義,來防止被漢化。例如他們提倡藏傳佛教高過於中原的佛教與道教,在政治上大量使用色目人,儒者的地位下降以及長時間沒有舉辦科舉。由於士大夫文化式微,意味宋朝的傳統社會秩序已經崩潰。這使得在士大夫文化低下,屬於中下層的的庶民文化迅速的抬頭。這個現象在政治方面是重用胥吏,在藝術與文學方面則是發展以庶民為對象的戲劇與藝能,其中以元曲最為興盛。

元朝的思想上也是兼收並用的,他們對各種思想幾乎一視同仁,都加以承認與提倡,「三教九流,莫不崇奉」。然而元廷在一定程度上尊重儒學,特別是於宋朝形成的理學,更是尊為官學,使得理學得以北傳。元仁宗初年恢復科舉,史稱延祐復科,在「明經」、「經疑」和「經義」的考試都規定用南宋儒者朱熹等人的注釋,影響後來明朝的科舉考試皆採用朱熹注釋。理學在元朝還有一些變化,南宋時期即有調和程朱理學的朱熹與心學的陸九齡等兩家學派的思想,元代理學家大多捨棄兩派其短而綜匯所長,最後「合會朱陸」成為元代理學的重要特點。當時有名的理學家有黃震、許衡與劉因與調和朱陸學的吳澄、鄭玉與趙偕。朱學的後繼者為了配合元帝的需求,更注重在程朱理學的倫理道德學說,其道德蒙昧主義的特徵日趨明顯。從而把注意力由學問思變的道問學轉向對道德實踐的尊德性的重視,這也促成朱、陸思想的合流。元代理學的發展,也為明朝朱學與陽明心學的崛起提供某些思想的開端。

江南統一後,元朝崇尚儒學的政策有新發展,漢蒙官員上書建議興舉和重視學校,於是元政府在推廣有關儒學教育政策的同時,亦更加注意優待和勉勵儒學。從元世祖到元世宗時期,元朝的重視、勉勵學改的政策已經完備。元成宗以後,這些政策基本上得到歷代皇帝的實行。例如為了維護儒學的正常運行,元世祖於至元二十五年下聖旨:「(江淮等處)仍禁約使臣人等勿得於廟學安下,非禮騷擾」,此後元政府兩次重申這一禁令,對元朝儒學教育的正常運作起到了保護作用。另外,元朝亦實行宋朝以來的學田政策,允許學校支配學田收入。元朝政府還將儒學推廣至邊遠地區,在雲南、兩廣、海南、西部地區如原西夏政權控制的範圍和原宋朝和吐蕃的邊境地區、北部和東北地區(岭北行省和遼陽行省)建立、推廣和發展儒學。元朝的統一對儒學教育向中國邊遠地區的擴散作出了推動作用,並且取得了明顯的成績。

由於元朝由蒙古人所統治,漢族士大夫基於異族統治的考量,在蒙元初期大多分成合作派與抵抗派。合作一派是華北儒者如耶律楚材、楊奐、郝經與許衡等人。他們主張與蒙古統治者和平共存,認為華、夷並非固定不變,如果夷而進於「中國」,則「中國」之。如果蒙古統治者有德行,也可以完全入主中原。他們提倡安定社會,保護百姓,將中華的典章制度帶進蒙元,以教感化蒙古人。另一派是江南南宋遺民的儒者如謝訪、鄭思肖、王應麟、胡三省、鄧牧、馬端臨等人。他們緬懷南宋故國,為了消極抵抗元廷,採取隱遁鄉里,終生不願意出仕的方式。並且以著述書籍為業,將思想化為書中主旨。到元朝後期,由於元仁宗實行延祐復科,恢復科舉,及第者都感謝天子的恩寵,紛紛願意為元廷解憂。元朝後期國勢大墬,政治腐敗、財政困難,使得當時士大夫如趙天麟、鄭介夫、張養皓與劉基等人紛紛提出各種政治主張,或從弊端中總結經驗教訓。他們大多提倡勤政愛民、廉潔公正、任用賢才等措施。元末民變的爆發使得南方有不少士大夫出於衛身、保鄉、勤王之目的,紛紛組織義兵護國,有些士大夫甚至捨身殉國。在明朝建立後,部分元朝遺老紛紛歸隱不出。

元朝文學以元曲与小说為主,對於史學研究也十分興盛。相對的元朝的詩詞成就较少,内容比較贫乏,但文以虞集為長,詩以劉因為著。明朝王世贞说“元无文”,但叙事文學如戲曲、小说第一次有主導地位。元朝使華北誕生元曲,江南則出現以浙江為中心的文人階層,孕育出《三国演义》和《水浒传》等長篇小說,自由奔放的文人如杨维桢、倪瓚等人,在城市發放出市民文化的花朵。

元曲分成散曲與雜劇,散曲具有詩獨立生命,雜劇則具有戲劇的獨立生命。當時城市經濟興盛,元廷不重視中国文學與科舉,當時社會提倡歌舞戲曲作為大眾的娛樂品,這些都使宋、金以來的戲曲昇華為元曲。散曲是元代的新體詩,也是元代一種新的韻文形式,以抒情為主,主要給舞台上清唱的流行歌曲,可以單獨唱也可以融入歌劇內,與唐宋詩詞關係密切。;雜劇是元代的歌劇,產生於金末元初,發展和興盛於元代至元大德年間。根據《太和正音譜》中所記,大約有五百三十五本,創作十分巨大而輝煌。元朝后期,雜劇創作中心逐步南移,加強與溫州發揚的南戲的交流,到元末成為傳奇,明清時發展出崑劇和粵劇。当时散曲四大名家有关汉卿、马致远、张可久与乔吉,有名的《南呂‧一枝花》(《不伏老》)反映作者樂觀和頑強精神;《恁闌人》(《江夜》)追求文字技巧,脫離散曲特有風格;描寫景物的《水仙子》(《重觀瀑布》)雅俗兼備,以出奇制勝;其中描寫自然景物的曲子《天淨沙》(《秋思》)刻劃出一幅秋郊夕照圖,情景交融,色彩鮮明,被稱為「秋思之祖」。雜劇五大名家除了關漢卿與馬致遠之外,還有白樸、王實甫與鄭光祖,有名的作品有《竇娥冤》、《拜月亭》、《漢宮秋》、《梧桐雨》、《西廂記》與《倩女離魂》,主要表現社會與生活情況、歌頌歷史人物與事件,強調人物的情感。元曲的興盛,最後成为与漢賦、唐诗、宋词并称的中国优秀文学遗产。

元朝长篇小说源自戲曲說白的平話,這些話本最後寫成書的即是小說,以《三国演义》和《水浒传》最有名,與明朝的《西遊記》、清朝的《红楼梦 》合稱中國古典四大文学名著。《三国演义》的作者是羅貫中,敘述三國時期曹操、劉備與諸葛亮等人物,小說通篇精巧敘述謀略,雖與史實多有出入,仍譽之「中國謀略全書」;《水浒传》一般認為是施耐庵所著,而羅貫中負責整理。其內容講述梁山泊以宋江為首的綠林好漢,由被迫落草,發展壯大,直至受到朝廷招安。現存宋元平話共約八種,包括《大唐三藏取經詩話》。

元代的歷史研究也十分興盛。胡三省潛心研究歷史巨著《資治通鑑》,1286年《資治通鑒音注》全部成編,公認是對《資治通鑑》的注釋最佳者。馬端臨在歷史文獻的收集和整理方面有很深的造詣,著有《文獻通考》,記載上古至宋寧宗嘉定末年曆代典章制度的政書,十通之一。蘇天爵、歐陽玄、虞集與趙世延一同編寫的《经世大典》。脫脫主編,由歐陽玄等人編寫《遼史》、《宋史》與《金史》。元朝還有記述大蒙古國立國至窩闊台汗時期的《蒙古秘史》。

元朝的文字與語言方面,一般是通用蒙古語與漢語,然而一些说法认为入聲字最早被認為在元朝官話消失。文字通用漢文與蒙古的八思巴字。八思巴文是元世祖時由國師八思巴根據當時的吐蕃文字而制定的一種文字,用以取代標音不夠準確的粟特语蒙古文字。然而此時橫跨歐亞的蒙古帝國已經析為元朝和四大汗国:蒙古欽察汗國、察合台汗国、窝阔台汗国、伊儿汗国,因此八思巴文一直只有元朝採用,並主要用作為漢字標音符號。元朝滅亡後,仍然推行於北元,到了16世纪末期,蒙古高原的蒙古人受其他蒙古民族同化,轉而重新採用蒙古文字。

元朝與四大汗国(欽察汗國、察合台汗国、窝阔台汗国、伊儿汗国)橫跨歐亞大陸,幅員遼闊,其疆土內的种族也十分繁多,這些都使得元朝的宗教呈現多元化,各類佛教(含漢傳佛教與藏傳佛教)、道教、白蓮教等都取得了较大的发展;東西方的商旅、教士亦来往频繁,自西方傳來的伊斯兰教、基督教(含景教和天主教)與猶太教的影響力也逐漸增加。由于元朝对境内各种宗教基本採取自由放任的態度,對信仰宗教的問題採取兼容並包的政策,甚且優容禮遇之,這種环境自然有利於宗教的傳播與發展。元朝僧人有免税免役特权,致使一些不法之徒投机为僧,甚至干预诉讼,横行乡里,成为元代的一个社会问题。不过,元世祖曾在禮節上歧視伊斯蘭教,例如不尊重其宰羊方法,伊斯蘭教徒被逼吃死肉,此法令亦適用於基督教徒。元朝对宗教管制较为宽松,使得民间如白莲教、明教等藉此建立秘密组织,进行抗元起事。

各類佛教中以藏传佛教最為興盛. 藏传佛教約唐中期自吐蕃傳入唐朝,專以祈禱禁咒為事。漢傳佛教在唐武宗时遭受打擊,宋朝時只剩禪宗慢慢恢復,然敵不過道教與理學。藏传佛教中,薩迦派(花教)自窩闊台汗至元世祖期間逐漸获得蒙元朝廷的尊重。忽必烈早在攻擊吐蕃時即於薩迦派的喇嘛扮底達講和,而後扮底達的繼承人八思巴被元世祖奉為國師(後升為帝師),賜玉印,任中原法王,命統天下佛教,並兼任總制院(後改名為宣政院)使來管理吐蕃(今西藏)事務,這是以宗教領袖統治西藏地區之始。八思巴還為元朝建立八思巴文。藏传佛教在元朝皇帝的推崇下,在社會與政治上均有極高的地位。諸位元朝皇帝均受藏传佛教的戒律,藏传佛教也逐漸推廣到蒙古各部。然而皇室用於佛事之錢要占國家財政支出一半(皇帝即位前要灌顶),寺院也擁有龐大的產業,部分喇嘛也驕縱不法,危害社會。例如元世祖時,江南佛教總統喇嘛楊璉真珈喜好掘墓,曾挖掘宋朝諸陵與諸大臣墳墓百餘所;包庇平民不輸租賦者,達兩萬三千戶,其餘如奪人產業,姦污婦女等類之事,更為常見。

道教自宋朝即十分興盛,金朝與南宋時期即有全真教、太一教與大道教三派。全真教由王喆創立,主張修孝僅存一之德,然後學道。成吉思汗於西征時邀请全真教道士丘处机西行中亞,十分禮遇他,並且他掌管天下道教。丘處機後來與其弟子李志常寫成《長春真人西遊記》一書,具有重要的史料價值。大道教主張苦節危行,不妄取於人,不苟奢於自,從創教教主劉德仁五傳至酈希誠,被蒙哥冊封為太玄真人,掌管教務。太一教以傳授太一三元法籙之術為主,從創教教主蕭抱珍五傳至李居壽時,元世祖興建太一宮,並讓他居之,獲得太一掌教宗師印。然而元朝以藏传佛教為國教,元世祖曾命燒去一些「捏合不實」的道經如《老子化胡經》等,然而仍然冊封各派宗師以安撫之。

元朝的基督教(即天主教)稱為也里可溫教,唐朝時基督教的分支景教(聶斯脫里派)因唐武宗的禁止而式微,到元朝時基督教再度傳入中國。當蒙古人數度西征時,歐洲频繁发动数次十字軍東征,征伐西亞的伊斯蘭教徒,因此歐洲人有意和蒙古結盟,共抗伊斯蘭教徒。貴由汗時,羅馬教皇曾派使者到和林見貴由汗;元世祖時教廷又派方濟各會教主由海道抵大都,元世祖同意其傳教,而景教教徒分布在揚州、杭州、鎮江與泉州等地,最後分布到華北、西北與西南。然而基督教時常與佛道兩教衝突,方聶兩派也自相牽制。元朝晚期,教皇有意派主教來華整頓教務,然而主事者漠不關心,元朝滅亡後東西交通斷絕,基督教再度式微。猶太教稱為術忽或主吾,犹太人大多定居開封、杭州、大都與和林等城市。由於猶太商人擅長理財,元廷視為財政來源之一。

元朝的伊斯兰教(又稱回教)稱為木速蠻教,也是於唐武宗後式微於中國,而後流行於西域中亞各國如畏吾兒、花剌子模等國。成吉思汗西征時降服許多西域回教國家,使得伊斯兰教徒仕於蒙古朝廷甚多。由於色目人(即西域各族)擅長理財,元世祖統一中國後更任用色目人,給予極大的權力。這些都使得伊斯蘭教盛行於中國西部、雲南地區等,部分色目商人也有定居於沿海廣州、泉州、杭州與揚州等地區,漸漸形成大分散、小集中的特色,幾乎覆蓋全國。1357年至1366年間更在福建發生色目軍亂,史稱亦思巴奚兵亂。當時蒙古王公大臣也有信奉伊斯蘭教,其中安西王阿難答更是虔誠的伊斯蘭教徒。他於元成宗駕崩後擔任監國,並且很有機會繼承為皇帝。如果他擔任皇帝,可能使元朝國教改為伊斯蘭教。

元朝經濟發達,城市文化興起,又因為交通發達,東西文化交流,使得元朝藝術呈現多元化。繪畫方面,文人畫成為主流,著重個人及書法表現,風格與元代強調裝飾的宮廷繪畫迥然不同。元初趙孟頫、高克恭等人提倡復古,回歸唐朝和北宋的風格,並且將書法入畫,創造出重氣韻、輕格律,注重主觀抒情的元畫風格。元朝中晚期以黃公望、王蒙、倪瓚、吳鎮等元代四大家 為主,其中又以黄公望为冠。他們寄託清高人格的理念於繪畫上,以隱逸山水與梅、蘭、竹、菊、松、石等為象徵。黃公望創始「淺降山水」,先以水墨鉤勒皴染為基礎,加上以赭石為主色的淡彩山水畫。由於元人以較乾的筆法在紙上作畫,這不同於宋人繪於絹上。山水畫除了皴法以外,增多擦的效果,猶如中國書法一樣。為了使畫面的上方可以題上詩句,所以故意留出一角,題上自己作的詩句,使詩、書、畫三者合成一體,影響明清國畫至今。元代的花鳥,以錢選最為有名,他學習宋人趙昌的畫風,具有宋人厚重典雅的趣味。其他如趙孟、趙雍、陳琳與劉貫道等均以兼善花鳥出名。

元朝書法的核心人物是趙孟頫,他的書法深受東晉書法家王羲之的影響,所創立的楷書趙體與唐楷之歐體、顏體與柳體並稱四體,成為後代規摹的主要書體,表現為“溫潤閒雅”“秀研飄逸“的風格面貌。審美觀趨向飄逸的超然之態獲得一種精神解脫有一定聯繫。鲜于枢与赵孟頫齐名,但影响略小,尤其擅长行、草书。与他们同时代的书法家邓文原则擅长章草,是研习这种古书体不多见的名家之一。康里巙巙稍晚于赵孟頫,也以草书名世,是少数民族书法家的代表人物。  

元朝的工艺美术十分发达,在传统的工艺美术上吸收了藏族等其他民族文化,对元代工艺美术带来了新的发展。官办手工业人材荟萃,技艺精湛,生产出了大量高级手工艺品和消费品,最明显的如陶瓷工艺、织绣工艺等。元朝瓷器及漆器等實用藝術常有創新。元朝是景德镇真正驰名的时期,最著名的瓷器即为青花瓷和釉里红。受到中東文化影響,瓷器有豐富的藍白色裝飾,中東商人也會訂製大量的龍泉青瓷。元朝也完成許多佛教雕刻,其中,密宗多手佛像顯示蒙古人對尼泊爾、西藏地區藏传佛教藝術的愛好。銀器工藝家朱碧山知名的銀器的雕造技術也是在此時發展。此外元代也製作生產雕漆工藝品。

由於元朝朝廷與社會提倡思想多元,經濟發達提供可靠的物質保證,交通發達與中外交往空前活躍又為吸收世界各地科技創造條件,使得科学技术有很高的成就,主要表現在天文歷法、數學、農牧業、醫藥學與地理學等方面。中國古代的發明印刷術及火藥等出現了印刷活字盤與火銃等技術,西傳西方後促進歐洲國家的進步。波斯、阿拉伯素稱發達的天文、醫學等成就,也在元朝被大量傳至中國。由於東西貿易的興旺,西域的玉石、紡織品、食品及珍禽異獸等也源源不斷輸入中國。中外的科技交流,促進了各自的科技進步,元朝正好為這種交流提供了比以前歷代都優越的條件。

元朝在天文歷法方面十分發達,元世祖邀請阿拉伯的天文學家來華,吸收了阿拉伯天文學的技術,並且先後在上都、大都、登封等處興建天文臺與回回司天臺,設立了遠達極北南海的27處天文觀測站,在測定黃道和恒星觀測方面取得了遠超前代的突出成就。元朝有名的天文學家有郭守敬、王恂、耶律楚材、紮馬魯丁等人。耶律楚材曾編訂有《西征庚午元歷》,1267年扎马鲁丁撰进《万年历》,郭守敬等人修改曆法,以近世截元法主持編訂了《授時歷》,《授時歷》於1280年頒行,延用了400多年,是人類歷法史上的一大進步。扎马鲁丁與後來的郭守敬研制出了簡儀、仰儀、圭表、景符、闚幾、正方案、候極儀、立運儀、證理儀、定時儀、日月食儀等十幾種天文儀器,當時在天文台里工作的还有阿拉伯天文学家可马剌丁、苫思丁等人。回回司天台一直存在到元末明初,仍由回回司天监黑的儿、阿都剌、司天监丞迭里月实等修定历数。元朝數學湧現出了一批傑出數學家及其著作。如李冶及其《測圓海鏡》、《益古演段》;朱世傑及其《算學啟蒙》、《四元玉鑒》;李冶提出的天元術(即立方程的方法)及朱世傑提出的四元術(即多元高次聯立方程的解法),是具有世界性影響的新成就。算盤在元代也初具規模。

元代的農業技術主要可見於《農桑輯要》、《王禎農書》與《農桑衣食撮要》等三部書。《農桑輯要》由元廷主持編纂,全書分七卷十篇,對元及其以前的作物栽培、牲畜飼養做了總結,並保存了大量古農書資料,對推廣農牧業技術,指導農牧業生產有重要作用。《農書》為著名農學家王禎所著,,全書分“農桑通訣”、“百谷譜”、“農器圖譜”三大部分,总结了古代的农业生产经验,又介绍了当时的新技术,是继北魏贾思勰的《齐民要术》之后又一部重要的农业科学著作。王禎認為要不違農時、適時播種、因地制宜、及時施肥、興修水利才是取得農業豐收的保證,其中關於棉桑種植具有現實意義。《農桑衣食撮要》為魯明善所著,此書重在實用,按月記載農事活動,特別還涉及到遊牧生產,可補《農桑輯要》及其它古農書之不足。

醫藥學方面,史稱金元四大家中有兩位生活在蒙元時期。李杲師承張元素,強調補脾胃,創立了“補土派”,著有《脾胃論》、《傷寒會要》等。朱震亨拜羅知悌為師,發展劉完素火熱學說,主張以補陰為主,多用滋陰降火之劑,後人稱其為“滋陰派”,著有《格致余論》、《局方發揮》、《傷寒辨疑》等書。外科骨傷科方面成就更為突出,危亦林在麻醉與骨折復位手術上有創新。薩德彌實的《瑞竹堂經驗方》很註意北方的寒冷氣候及蒙古族遊牧生活實際,有不少治療骨傷及風寒濕痹的方劑,有的時至今日仍為醫家所使用。元廷太醫忽思慧的《飲膳正要》反映了當時國內各少數民族及中外人民的飲食文化交流。

地理學方面《元一統誌》的編纂、河源的探索、《輿地圖》的問世及大批遊記類著作的出版是其主要成就。《元一統誌》由政府主持,紮馬魯丁、虞應龍具體負責。該書對全國各路府州縣的建置沿革、城郭鄉鎮、山川裏至、土產風俗、古跡人物均有詳細描述,具有較高史料價值。1280年元世祖命女真人都實探求黃河河源,認為星宿海(火敦腦兒)即河源,比較接近實際。潘昂霄還據此撰成《河源誌》。道士朱思本考察了今華北、華東、中南等廣大地區地理形勢,參閱《元一統誌》等地理學著作,以“計裏劃方”法,繪制成《輿地圖》,成為元朝地理學及中國地圖史上劃時代的人物。遊記類地理學著作有耶律楚材《西遊錄》,李志常整理的《長春真人西遊記》,周達觀《真臘風土記》,汪大淵《島夷誌略》等,對元朝國內外的地理地貌、風土人情、貿易來往等頗多描繪。

元代水陸交通的發達,使中外交往範圍空前擴大。當時,東西方使臣、商旅的往來非常方便。元人形容說:“適千裏者如在戶庭,之萬裏者如出鄰家。”同時代的歐洲商人也說,從裏海沿岸城市到中國各地,沿途十分安全。這對發展中外各國之間,國內各民族之間的科技文化交流是十分有利的。元朝與中亞、西亞地區的蒙古势力保持著來往關係,东西方海运及陆路交通十分畅通,使得西方与元朝中国的交往更加频繁,技术交流更加迅速。其中主要分陸路和水路兩部分。

陸路有發達的驛道,主要遞送朝廷、郡縣的文書。早在成吉思汗時代,就在西域地區新添了許多驛站。著名的長春真人丘處機在興都庫什山覲見元太祖成吉思汗時,即曾經過這些驛站。元世祖忽必烈統一中原後,在遼闊的國土上,建立了嚴密的驛傳制度(蒙古語“站赤”),使郵驛通信十分有效地發揮效能。元朝的驛路分為三種:一稱帖裏幹道,蒙古語意為車道;二稱木憐道,蒙語意為馬道;三為納憐道,蒙語意為小道。從地區講,帖裏幹和木憐道,多用於嶺北至上都、大都間的郵驛;納憐道僅用於西北軍務,大部分驛站在今甘肅省境內,所以亦稱“甘肅納憐驛”。驛道國內可達吐蕃、大理、天山南北路、蒙古草原,國外遠及波斯、敘利亞、俄羅斯及歐洲其它地區。

水路主要指河運和海運。河運方面元代鑿通了南起鎮江、北達大都的大運河。其中從鎮江至杭州的江南運河段,從淮安經揚州入長江的揚州運河段,大體是隋代運河舊道。元世祖以郭守時擔任都水監,負責修治元大都至通州的運河(其后被忽必烈命名通惠河),再加上修建濟州河、會通河等其它幾項重大工程,這使得連接大都至杭州的京杭大運河全線貫通。海運方面,当時元朝的船只已经航行于印度洋各地,包括锡兰(今斯里兰卡)、印度、波斯湾和阿拉伯半岛,甚至达到非洲的索马里亞。威尼斯人馬可·波羅在忽必烈時期隨從他的父親和叔叔來到中國,在其口述並由魯斯蒂謙記錄的《馬可·波羅遊記》中描繪出元朝中國的繁華景象。

元代社會因思想多元化、商業經濟發達與交通便利,使得元帝國的強盛,是東亞地區的富裕大國,在歐洲人馬可波羅的遊記中,可以看出當時的盛況。隨著理學影響的下降,長期以來壓在人們心頭的封建禮教的磐石隨之鬆動,下層人民和青年男女,蔑視禮教違反封建倫理的舉動越來越多,以至王惲對宣揚禮教的做法,發出了「終無分寸之效者,徒具虛名而已」的慨嘆。孔齊言道:「浙間婦女,雖有夫在,亦如無夫,有子亦如無子,非理處事,習以成風。」。在此說明元朝社會的價值觀念在變化,說明元代文學作品出現眾多違背封建禮教的人物,有著廣泛的社會基礎。

由於元帝對科舉的輕忽,使得大批文化人失去了優越的社會地位和政治上的前途,從而也就擺脫了對政權的依附。他們作為社會的普通成員而存在,通過向社會出賣自己的智力創造謀取生活資料,因而既加強了個人的獨立意識,也加強了同一般民眾尤其是市民階層的聯繫,他們的人生觀念、審美情趣,由此發生了與以往所謂「士人」明顯不同的變化。至於蒙古族的生活方式,原本純粹是游牧民族,逐水草而居。早期社會中的婚姻以外婚、仇家禁婚、無倫理上限制為主。他們有傳統的婚禮習俗,但在統一中國後,由於蒙漢通婚,以及漢化的影響,部分也採漢禮。


%% -*- coding: utf-8 -*-
%% Time-stamp: <Chen Wang: 2021-11-01 17:05:06>

\section{大蒙古国\tiny(1206-1260)}

\subsection{太祖成吉思汗生平}

成吉思汗(1162年5月31日-1227年8月25日),即元太祖,又稱成吉思皇帝、成吉思可汗。民國以前的漢文蒙古史料中除史集及新元史本紀外都以成吉思可汗及成吉思皇帝稱呼,成吉思汗稱呼為民國自西方翻譯而來。(《元朝秘史》記載為成吉思皇帝,《蒙古秘史》漢文版是現代翻譯。)為蒙古人,蒙古帝国奠基者、政治家、军事统帅,皇帝(大蒙古国可汗)。名铁木真,满清官译为特穆津。也有其他译法忒没真,意為“鐵匠”或“鐵一般堅強的人”、“鐵人”。奇渥温·孛儿只斤氏,尼倫蒙古乞顏部人。1206年春天—1227年8月25日在位,在位22年。1206年他登基时,诸王和群臣为他上蒙语尊号成吉思合罕。

至元二年(1265年)十月,元世祖忽必烈追尊成吉思汗廟號为太祖,至元三年(1266年)十月,太庙建成,制尊谥庙号,元世祖追尊成吉思汗諡號为聖武皇帝。至大二年十二月六日(1310年1月7日),元武宗海山加上尊谥法天啟運,庙号太祖。从此之后,成吉思汗的諡號变为法天啟運聖武皇帝。 

在他众子中,最为著名的四位分别是朮赤、察合台、窩闊台和拖雷。成吉思汗分封了朮赤和察合台为国主,欽定窩闊台为继承人。1227年成吉思汗去世后,拖雷自动退出继承人的選拔,担任监国两年后,1229年,拖雷和宗王们一起拥戴自己的三哥窝阔台登基。於1232年九月,在消灭金朝军队精锐主力后,拖雷去世,1234年2月9日,蒙古帝國灭金朝,為將來忽必烈揮軍南下攻打南宋打下基礎。

成吉思汗因其作戰的殘酷性而聞名,並被許多人視為種族滅絕的統治者。 然而,他也將絲綢之路置於一個有凝聚力的政治環境之下。 這使得東北亞,西亞和基督教歐洲之間的交流和貿易相對容易,擴大了這三個地區的文化視野。

金世宗大定二年(1162年),成吉思汗生于漠北草原。成吉思汗父親為其乞顏部酋長也速该。其名字「铁木真」之由來,乃是因為在他出生時,其父也速该正好俘虜到一位屬於塔塔儿部族,名為铁木真兀格的勇士。按當時蒙古人信仰,在抓到敵對部落勇士時,如正好有嬰兒出生,該勇士的勇氣會轉移到該嬰兒身上。成吉思汗「铁木真」之名遂因此而來。传说成吉思汗出生时,手中正拿著一血块,寓意天降將掌生殺大權。成吉思汗九歲时喪父,約於1170年。

在帶铁木真去弘吉剌部娶親後回来的路上,途經塔塔兒部,也速該遭到塔塔儿部殺害(怀疑被毒死),之後乞顏部族的泰赤乌氏首領塔里忽台因不滿也速該生前的所作所為,在也速該死後對鐵木真一家進行報復,命令部眾們遷至他地,孤立铁木真一家,但铁木真一家靠著毅力艱苦的活了下去。

就在铁木真漸漸出落成一個魁梧英俊的少年時,有三次劫難卻意外地降臨到他的頭上。

第一次是:脫離他們家族的泰赤乌氏擔心铁木真長大后報仇,於是就對铁木真家進行了突襲,並且計劃將被捕的铁木真處死。铁木真靠著父親的舊部鎖兒罕失剌以及其子沈白、赤老溫,其女合答安的協助脫逃,才因此逃过了一劫。身為長子的他,要攜母和弟妹們走到不兒罕山區,逃避泰赤乌氏追捕長達數年,自此形成他剛毅忍辱性格。

第二次是:在一個風雪交加的夜晚,一幫盜賊把他家僅有的幾匹馬搶走。在與盜賊的搏斗中,铁木真被盜賊射中喉嚨。危難之際,一個名叫博爾朮的青年拔刀相助,趕跑了盜賊,奪回了馬匹,铁木真得以幸免于難。

第三次是:成年後,铁木真與孛兒帖結婚時,三姓蔑兒乞部的首领脫黑脫阿,為報其弟赤列都的未婚妻訶額侖當年被铁木真的父親也速該所搶之仇,突袭了铁木真的營帳。在混戰中,铁木真逃進了不兒罕山(今肯特山),他的妻子和異母卻變成了脫黑脫阿的俘虜。

然而,三次劫難並未擊垮铁木真,反倒增強了他的復仇心理。他發誓要奪回家裡失去的一切。铁木真深知,要想立足,必須擁有實力。於是,他把妻子嫁妝中最珍貴的“黑貂皮”獻給了當時草原上實力最雄厚的克烈部落統領王汗。利用王汗的勢力,铁木真不僅收攏了他家離散的部族,還在王汗及幼時“安答”(義兄弟)札木合的幫助下,擊敗了三姓蔑兒乞部首领脫黑脫阿、忽都父子,救出了妻子孛兒帖和異母。

自此铁木真和札木合两人一起在部落共同生活。

由于铁木真提拔一些非贵族的人为将领,引發札木合不满,最终雙方决裂。1182年,铁木真被推举成为蒙古乞颜部的可汗。

統一蒙古各部:

1190年,在铁木真的領導下,乞顏迅速發展壯大,引起札达兰部首領札木合的不滿。札木合以其弟弟绐察兒被铁木真部下所殺為藉口,糾集了13個部落三萬余人,向铁木真發起進攻。铁木真也動員了部眾十三翼(即13個部落)迎擊,即著名的十三翼之戰。铁木真雖兵敗退至斡難河畔哲列捏狹地,但萬萬沒想到獲勝的札木合卻失去了人心。戰後,因為札木合把俘虜全部處死,將俘虜分七十大鍋煮殺,史稱「七十鍋慘案」。這種慘不忍睹的場面,連其部下也“多苦其主非法”,甚至擔心起自己的命運來。相反的,寬厚仁容的铁木真贏得了人心,那些擔心自己命運的札木合的部下紛紛倒向铁木真。此戰铁木真敗而得眾,使其軍力得以迅速恢復和壯大。铁木真的部眾一下子增加了許多。1196年,塔塔儿部首领蔑兀真笑里徒反抗金朝,金朝丞相完颜襄约克烈部王汗和铁木真联合出兵进攻塔塔儿,塔塔儿部大败,蔑兀真笑里徒被杀。铁木真遂被金朝封為“札兀惕忽里”,即部落官。

主兒乞部偷襲鐵木真的後方營地,被鐵木真剿滅。1201年,泰赤乌部、塔塔儿部、蔑兒乞部等11部推舉札达兰部的札木合為“古兒汗”,联兵攻打铁木真。铁木真联合王汗,於阔亦田之戰击败札木合等十二部联軍。聯軍潰散後,鐵木真追擊並剿滅了泰赤烏部。1202年,杀死塔塔儿部首领札鄰不合並屠殺残余的塔塔儿人,憶起少年時,父親也速该遭塔塔儿所害,临命终時的遗言,遂將凡是身高超過車輪高的塔塔儿士兵、男子通通都殺光,手法殘忍震驚蒙古諸部族。

1203年,王汗將铁木真收為义子,導致桑昆跟铁木真仇恨,札木合鼓動桑昆联合王汗夹击铁木真。合蘭真沙陀之戰爆发,这是铁木真经历的最为惨烈的一仗,只剩下19人隨他敗走班朱尼河,北上贝尔湖途中陸續追隨而來的部眾也只有2千6百人。同年秋天突袭王汗驻地,三天后完全消灭克烈部。王汗逃到鄂尔浑河畔之后被乃蛮人杀死。而其子桑昆則逃到庫車,被當地人杀死。

1204年,铁木真征伐蒙古草原西边的太阳汗,於納忽崖之戰击败乃蛮大軍,太阳汗当场被杀。秋,於合剌答勒忽札兀兒擊敗蔑兒乞部首領脫黑脫阿。1205年,鐵木真於額爾齊斯河擊敗蔑兒乞和乃蠻殘部聯軍,蔑兒乞首領脫黑脫阿陣亡,其子逃往康里、欽察,乃蛮部王子屈出律則逃亡西辽。1206年,札木合被叛变的将领送到铁木真之手,札木合请死,铁木真便殺了他。爾後,铁木真統一蒙古各部。

称成吉思汗:

1206年春天,蒙古贵族们在斡难河(今鄂嫩河)源头召开大会,諸王和群臣為鐵木真上尊號“成吉思汗”,正式登基成为大蒙古国皇帝 (蒙古帝国大汗),这是蒙古帝國的開始。成吉思汗遂颁布了《成吉思汗法典》,是世界上第一套應用範圍最廣泛的成文法典,建立了一套以贵族民主為基礎的蒙古贵族共和政體制度。

威脅西夏:

蒙古分别在1205年、1207年及1209年三次入侵西夏,逼使西夏臣服。1210年,西夏向蒙古称臣,並保证派军队支持蒙古以后的军事行动,此外,西夏皇帝夏襄宗献女求和,把察合公主嫁给了成吉思汗。

征服森林部落:1207年,成吉思汗命長子朮赤征森林部落。

降葛邏祿:1210年,成吉思汗命忽必來征葛邏祿,首領阿兒思蘭汗率部降。

消滅金朝未果:1210年,成吉思汗与金朝断绝了朝贡关系(约从1195年开始)。

1211年二月,成吉思汗亲率大军入侵金朝,在1211年的野狐嶺會戰击败四十萬金軍,并在次年和第三年陆续攻破金朝河北、河东北路和山东各州县,1214年三月,金宣宗遣使向蒙古求和,送上大量黄金、丝绸、马匹,并将金卫绍王的女儿岐国公主送给成吉思汗为妻,还有童男女五百陪嫁。成吉思汗从中都撤兵。

在金朝的东北地区,1212年,契丹人耶律留哥在辽东起兵反抗金朝,并宣布归附蒙古,耶律留哥和蒙古联军打败前来征讨的六十万金朝军队,1213年,耶律留哥自称辽王,1215年春,耶律留哥攻克金朝东京(今辽宁省辽阳),并占领金朝东北大部分地区。1215年十一月耶律留哥秘密与其子耶律薛阇带着厚礼前往漠北草原朝觐成吉思汗,成吉思汗极为高兴,赐给耶律留哥金虎符,仍旧封他为辽王。

为了远离蒙古的威胁,1214年6月27日,金宣宗离开中都,遷都汴京,得知金朝皇帝离开,成吉思汗下令入侵中都,蒙古軍在1215年5月31日占领中都,金朝在黃河以北之地陸續失守。

占领中都后,成吉思汗返回蒙古草原,1217年,成吉思汗任命大将木华黎为“太师国王”,让他负责继续入侵金朝,经过木华黎和他的儿子孛鲁十年的战争,到1227年成吉思汗去世前夕,蒙古军队基本占领金朝黄河以北的所有领土,金朝的领土仅局限于河南、陕西等地(当时的黄河取道江苏北部的淮河入海)。

1217年,成吉思汗派大将速不台追击脫黑脫阿諸子忽都、合剌、赤剌溫,次年於楚河地區剿滅蔑儿乞殘部。

正當金朝危在旦夕時,中亞的花剌子模王国惹怒蒙古,成吉思汗性急,转而報仇,暂时无暇顾及继续入侵金朝。

滅西辽及花剌子模:

早在1211年春天,畏兀儿亦都護巴而朮·阿而忒·的斤便归附蒙古。至1218年春季,成吉思汗派遣的蒙古使团到达花剌子模王国,强迫摩诃末苏丹签订与蒙古的条约。条约签订后,花剌子模城市讹答剌长官杀死路过此城的一支来自蒙古的由500人穆斯林组成的商队,夺取货物,仅有一人幸免于难逃回蒙古,成吉思汗派三个使臣前往花剌子模向摩诃末交涉,结果为首者被杀,另外二人被辱,成吉思汗更加愤怒,决定入侵花剌子模。

1218年,成吉思汗派大将哲别灭西辽,杀死西辽末代皇帝屈出律,平定西域。西征花剌子模进兵路上的障碍被扫除了。

1219年六月,成吉思汗親率蒙古主力(大约十万人)向西侵略,并在中途收编了5万突厥军,1220年底,一直被蒙古军队追击的花剌子模算端摩诃末病死在宽田吉思海(今里海)中的一个名為額別思寬島(或譯為阿必思昆島,已陸沉)的小岛上,并在临死前传位札兰丁。蒙古军先后取得河中地区和呼罗珊等地,1221年,蒙古军队消滅花剌子模王国,1221年十一月,成吉思汗率军追击札兰丁一直追到申河(今印度河)岸边,札兰丁大败,仅仅率少数人渡河逃走。

当初,成吉思汗命令速不台和哲别率领二万骑兵追击向西逃亡的摩诃末,摩诃末逃入里海后,他们率領蒙古軍继续向西进发,征服了太和岭(今高加索山)一带的很多国家,然后继续向西进入欽察草原擴張。1223年,者别與速不台於迦勒迦河之战(今乌克兰日丹诺夫市北)中击溃基辅罗斯诸国王公与钦察忽炭汗的联军,然后又攻入黑海北岸的克里木半岛。

1223年底,哲别與速不台率军东返,经过也的里河(今伏尔加河的突厥名,又译亦的勒),攻入此河中游的不里阿耳,遭遇顽强抵抗后,沿河南下,经由里海,咸海之北,与成吉思汗会师东归。在东返途中,哲别病逝。

攻西夏·去世:

成吉思汗回師後幹,再次入侵西夏。1227年8月25日(农历七月十二己丑日),在蒙古軍圍困西夏首都時,成吉思汗病逝於今宁夏南部六盘山(一说灵州),享壽六十五歲。其死因至今眾說紛紜,《元史》记载:“(元太祖二十二年)秋七月壬午,不豫。己丑,崩于萨里川啥老徒之行宫。”

成吉思汗去世前向儿子们交代了灭金的計劃:“假道宋境,包抄汴京。”后来窝阔台和拖雷灭金朝,采用的就是成吉思汗的这个战略。

此前西夏末代皇帝李睍已经答应投降,成吉思汗去世后,蒙古军密不发丧,李睍开城投降后,前去参见成吉思汗,诸将托言成吉思汗有疾,不让他参见。在成吉思汗去世三天后,1227年8月28日,诸将遵照成吉思汗遗命将西夏末帝杀死,西夏灭亡。蒙古军将领察罕努力使西夏首都中興府(今宁夏銀川)避免了屠城的命运,入城安抚城内军民,城内的军民得以保全。

《元朝祕史》记载成吉思坠马跌伤。而罗马天主教教廷使节约翰·普兰诺·加宾尼在《被我们称为鞑靼的蒙古人的历史》稱成吉思汗可能是被雷电击中身亡。

据《蒙古秘史》记载,成吉思汗的遗体被葬在不兒罕山接近斡難河源頭的地方,这是他生前指定的墓地。《元史》则记载他和历代元朝皇帝都葬于起辇谷。起辇谷的具体位置不详。在今日蒙古国肯特省的不儿罕山间有一片被称为“大禁忌”的土地,为达尔扈特人世代守护,相传是成吉思汗的墓地所在。在内蒙古自治区西部的鄂尔多斯高原上,有一座蒙古包式建筑宫殿,為成吉思汗的衣冠冢,经过多次迁移後直到1954年才由湟中县的塔尔寺迁回故地伊金霍洛旗,北距包头市185公里。每年的农历三月廿一、五月十五、八月十二和十月初三,为一年四次的大祭。

有傳言認為成吉思汗可能是遭三子窩闊台毒杀,原因是当时大汗打算传位给窝阔台,但突然改变注意,欲传位给四子拖雷,窝阔台为保汗位,所以毒杀其父。《成吉思汗与今日世界之形成》关于成吉思汗之死的論述与诸多的死亡故事相反,認為成吉思汗在游牧帐篷中去世,与他在游牧帐篷中的出生情形相似,这说明他在保存其本民族传统生活方式方面非常成功;然而,他保持其自身生活方式的过程中,却改变了人类社会。他在故土安葬,没有一座陵墓,没有一座寺庙,甚至没有一块用来标示其长眠之地的小墓碑。按照蒙古人的信仰,遗体应该在静穆中离去,并不需要纪念碑,因为灵魂已经不在那里了;灵魂继续活在精神之旗中。但他的精神之旗在1937年从蒙古中部的黑尚赫山下月亮河畔的寺庙里消失了。虔诚的喇嘛们护卫几个世纪的圣物,在由当时斯大林的追随者霍尔洛·乔巴山开展的遏制蒙古文化与宗教的运动中,永远的消失了。

尊谥庙号:

至元二年十月十四日(1265年11月23日),元世祖忽必烈追尊成吉思汗廟號为太祖。

至元三年十月十八日(1266年11月16日),太庙建成,制尊谥庙号,元世祖追尊成吉思汗諡號为聖武皇帝。

至大二年十二月六日(1310年1月7日),元武宗海山加上尊谥法天啟運,庙号太祖。从此之后,成吉思汗的諡號变为法天啟運聖武皇帝。《太祖皇帝加上尊谥册文》,内容如下:

维至大二年、岁次己酉、某月、某日,孝曾孙嗣皇帝臣某,谨再拜稽首言:

{\fzk 伏以恢皇纲,廓帝纮,建万世无疆之业;铺宏休,扬伟绩,遵累朝已定之规。式当继统之元,盍有称天之诔。孝弗忘于率履,制庸谨于加崇。钦惟太祖圣武皇帝陛下,渊量圣姿,睿谋雄断,沛仁恩而济屯厄,振羁策以驭豪英。惟解衣推食于初年,见君国子民之大略。玄符颛握,诸部悉平;黄钺载麾,百城随下。裔土兼收于夏孽,余波克殄于金源。荡荡乎无能名迹,远追于汤武;灏灏尔其为训道,允协于唐虞。根深峻岳而维者四焉,囊括殊封而统之一也。

肆予小子,承此丕基。两袛见于太宫,恒僾临于端扆。祚垂鸿兮锡裕,尚期昭报之申;牒镂玉以增辉,敢缓弥文之举。谨遣某官某,奉玉册玉宝,加上尊谥曰法天启运圣武皇帝,庙号太祖。

伏惟威灵昭假,景贶潜臻,阐绎吾元,与天并久。}

称号来源:“成吉思汗”是铁木真於1206年获得的称号。“成吉思”的含义不明确,一种说法由“成”派生而来。另一种说法是来自海洋一词,代表他像海洋一样伟大。

现存的13世纪和14世纪期的众多史料以及考古文物和摩崖石刻证明,1206年成吉思汗建立大蒙古国后,可能已经拥有皇帝和大汗的双重身份。生活在草原地区的蒙古等民族用蒙古语称呼铁木真为“大汗”、“成吉思汗”;生活在西北地区的突厥和其他民族用突厥语或其他语言称铁木真为“汗”或者“可汗”;生活在漠南汉地和东北地区的契丹人、女真人、党项人等民族,在13世纪前期的时候,历经辽朝、金朝、西夏等汉化政权,大部分已经汉化,通用汉语汉字,多称铁木真为“皇帝”;而生活在漠南汉地和东北地区的汉族人则直接使用“成吉思皇帝”一词。大量历史记载资料证明,1215年成吉思汗在攻取包括金中都在内的整个幽云十六州之后,其在长城以南汉地的统治保留了一些辽、金等朝的旧俗,并且在这些区域的官方文件,直接应用了“皇帝”的尊号来指代历任大蒙古国大汗。例如:

1219年农历五月,铁木真派刘仲禄邀请长春真人丘处机前往蒙古草原的诏书中,自称为“朕”,将自己建国登基称为“践祚”。

1220年农历二月丘处机抵达燕京后,得知铁木真在中亚进行西征花剌子模的战争,觉得自己年事已高,西行太远,希望约铁木真在燕京相见,于是在三月写了一份陈情表,在陈情表中,丘处机对铁木真的称呼是“皇帝”。同年收到丘处机的陈情表后,铁木真第二次派曷剌邀请丘处机前往中亚草原的诏书中,以“成吉思皇帝”和“朕”自称。

1221年南宋使者赵珙出使大蒙古国,回来后著有《蒙鞑备录》,书中对铁木真的称呼是“成吉思皇帝”。《蒙鞑备录》中提到,铁木真在位时期,朝廷使用的金牌,带两虎相向,曰虎头金牌,上书汉字:“天赐成吉思皇帝圣旨,当便宜行事”;其次为素金牌,书:“天赐成吉思皇帝圣旨疾”。1998年,一块“圣旨金牌”发现于河北廊坊,正面刻双钩汉字:“天赐成吉思皇帝圣旨疾。”和《蒙鞑备录》所记载的素金牌上汉文完全相同;背面牌心刻双钩契丹文,其汉语意思为:“速、走马,或快马”。这块圣旨牌的发现,说明铁木真在世时,其官方中文称谓作“成吉思皇帝”。

1227年全真教道士李志常写成的《长春真人西游记》,记录了丘处机从1219年受邀西行直至1227年去世的事迹,书中对铁木真的称呼是“成吉思皇帝”,将他下的命令称为“聖旨”;书中也提到了铁木真的侍臣刘仲禄前来邀请丘处机时携带了虎头金牌,金牌上面的文字是:“如朕亲行、便宜行事”,似乎在铁木真时期,凡是针对汉地的蒙古官方文件,均把成吉思汗翻译为“成吉思皇帝”。

1232年南宋使者彭大雅随奉使到大蒙古国,使者徐霆1235年—1236年随奉使到大蒙古国,二人返回南宋后,彭大雅撰写,并由徐霆作疏,合著《黑鞑事略》,书中对铁木真的称呼是“成吉思皇帝”。

2010年,刻有多位蒙古皇帝圣旨的全真教炼神庵摩崖石刻于山东徂徕山被发现,石刻一共四方,全部以汉语白话文写就,记述了大蒙古国皇室成员历代颁发给全真教掌教的官方文牒,其中有成吉思皇帝、合罕皇帝(窝阔台)、贵由皇帝,孛罗真皇后(窝阔台之妻)、唆鲁古唐妃,以及昔列门太子、和皙太子(均为窝阔台之子)等字样,其中记叙的“甲辰年十月初八日”表明该条圣旨是乃马真后称制的1244年颁发,落款“庚戌年十二月”则表明该石刻刻于海迷失后称制的1250年。圣旨石刻以汉语写就,包含不同时期、不同蒙古大汗的圣旨记录,为大蒙古国时期在汉地以中文“皇帝”作为蒙古大汗官方尊号的有力文物证据。

至元三年(1266年)忽必烈给日本的国书中,国书开头自称“大蒙古国皇帝”,在后面的内容中,自称为“朕”,此时距离他1271年正式立国号“大元”,还有五年时间。

然而大蒙古国时期的“皇帝”,和后来元朝的“皇帝”称号有本质的不同;前者是对“蒙古大汗”的汉式翻译,而后者则是按照中原文明的传统开立的新王朝君主,其“皇帝”称号上承秦汉隋唐宋等朝代。在1259年蒙哥汗去世后,忽必烈认为自己是大蒙古国汗位的正式继承者,自立为大汗,称“大蒙古国皇帝”,并于1263年将大蒙古国的历代大汗一并列入了自己新落成的太庙中;由于最终忽必烈没能获得蒙古各部贵族认可为新一任大汗,其于1271年按照中原文明的传统,建国号“大元”,因而元朝以后官方正史一直依照庙号将成吉思汗称作“太祖”。此时的大元皇帝,与之前大蒙古国时期被称作“皇帝”的蒙古大汗有本质区别——蒙古四大汗国的独立、大蒙古国的分裂,标志着忽必烈没能正式继承“大蒙古国”大汗之位;元朝,则是其新开创的王朝。元成宗时期,經過與蒙古四大汗國協商,元朝皇帝作为整个蒙古帝国共主的身份獲得四大汗國承認,作为中国历史上最高统治者称号的“皇帝”称号和作为“大蒙古国”最高统治者称号的“大汗”称号,同时集合在了後代的元朝皇帝的身上,如同中世纪歐洲由某王國國王或某公国大公出任神聖羅馬帝國皇帝。

整个元朝时期乃至后世王朝,官修历史一直保持了元朝的传统,将大蒙古国时期与元朝时期的统治一并而论,不作区分,统一将君主称为“皇帝”。《元史》中的<太祖本纪>記載鐵木真於1206年建大蒙古国时,称其“即皇帝位于斡难河之源,诸王群臣共尊其為成吉思皇帝”。元惠宗至正五年(1345年)十一月修成的法律《至正条格》中,称铁木真为“成吉思皇帝”,将他下的命令称为“聖旨”。明初官修《元史》,书中出现过“成吉思皇帝”一词多次,从未出现过“成吉思汗”一词。1252年成书的《元朝秘史》(《蒙古秘史》),蒙文音译作“成吉思合罕”,旁注释为“太祖皇帝”。直到近代中国,《新元史》中出现了“成吉思合罕”、“成吉思可汗”等词语,原因是《新元史》完成于民初(1920年),而《史集》、《世界征服者史》等西方的史书在清朝末年传入中国,《新元史》作者柯劭忞也深受其影响。

然而对于中国以外的地区,则仍将“大蒙古国”的君主称谓记作“大汗”。关于“成吉思汗”的记载见于拉施特《史集》、志费尼《世界征服者史》等中亚史籍,这两位作者均为蒙古帝国时期伊儿汗国(位于西亚)史学家,与元朝《元史》等史书基本处于同一时代,其书可为依据。四大汗国治下以的西亚国家以及欧洲公国仅知“成吉思汗”,同一时期的中国仅知“成吉思皇帝”,可见“成吉思皇帝”一词是针对古代汉字文化圈地区特设的翻译用词;由于西亚及欧洲文字皆为表音文字,其记载最能说明,大蒙古国君主的官方称谓仍为“大汗”,而非“皇帝”。

也速該,鐵木真父親,從蔑兒乞部手中奪走訶額侖,1170年被塔塔儿部首領札鄰不合毒害。也速該死後,族人離散,令鐵木真一家被逼過著流離生活。1266年元世祖忽必烈追尊也速该为皇帝,为也速该上庙号烈祖,諡號神元皇帝。

訶額侖,鐵木真母親,1206年尊为皇太后,1266年元世祖忽必烈上谥号宣懿皇后。

金末元初长春真人丘处机,拒绝金朝皇帝和南宋皇帝的邀请,答应前往草原和铁木真相见,抵达燕京后,得知铁木真已在中亚西征花剌子模,觉得自己年事已高,西行太远,希望约铁木真在燕京相见,于是在1220年三月写了一份陈情表,在陈情表中,对铁木真的评价是:“前者南京及宋国屡召不从,今者龙庭一呼即至,何也?伏闻皇帝天赐勇智,今古绝伦,道协威灵,华夷率服。是故便欲投山窜海,不忍相违;且当冒雪冲霜,图其一见。”(南京指的是当时的金朝首都开封,1214年,金朝从中都迁都到南京开封府)

南宋使者赵珙,1221年出使大蒙古国,在燕京(原为金中都,1215年被蒙古军队攻取,1217年木华黎改名燕京,今北京市)见到主持进攻金朝的太师国王木华黎,回来后著有《蒙鞑备录》,书中的評價是:“今成吉思皇帝者,……。其人英勇果决,有度量,能容众,敬天地,重信义。”

蒙古帝国伊儿汗国史学家志费尼《世界征服者史》的評價是:“倘若那善于运筹帷幄、料敌如神的亚历山大活在成吉思汗时代,他会在使计用策方面当成吉思汗的学生,而且,在攻略城池的种种妙策中,他会发现,最好莫如盲目地跟成吉思汗走。”

明朝官修正史《元史》宋濂等的評價是:“帝深沉有大略,用兵如神,故能灭国四十,遂平西夏。其奇勋伟迹甚众,惜乎当时史官不备,或多失于纪载云。”

明朝官修皇帝实录《明太祖实录》记载,洪武二十二年(1389年)五月,明太祖朱元璋给北元阿札失里大王的信中,对成吉思汗、元太宗窝阔台、元定宗贵由、元宪宗蒙哥、元世祖忽必烈这五位在一统天下中均作出重要贡献的帝王的综合评价如下:“覆载之间,生民之众,天必择君以主之,天之道福善祸淫,始古至今,无有僣差。人君能上奉天道,勤政不贰,则福祚无期,若怠政殃民,天必改择焉。昔者,二百年前,华夷异统,势分南北,奈何宋君失政,金主不仁,天择元君起于草野,戡定朔方,抚有中夏,混一南北,逮其后嗣不君,于是天更元运,以付于朕。”

明朝官修皇帝实录《明太祖实录》记载,洪武二十二年(1389年)十二月,明太祖朱元璋给哈密国兀纳失里大王的信中,对成吉思汗和元世祖忽必烈的评价如下:“昔中国大宋皇帝主天下三百一十余年,后其子孙不能敬天爱民,故天生元朝太祖皇帝,起于漠北,凡达达、回回、诸番君长尽平定之,太祖之孙以仁德著称,为世祖皇帝,混一天下,九夷八蛮、海外番国归于一统,百年之间,其恩德孰不思慕,号令孰不畏惧,是时四方无虞,民康物阜。”

清朝史学家邵远平《元史类编》的評價是:“册曰:天造鸿图,艰难开创;浑河启源,角端呈像;芟夏蹙金,电扫莫抗;栉沭廿年,驱指四将;止杀一言,皇猷弥广。”

清朝史学家毕沅《续资治通鉴》的評價是:“太祖深沉有大略,用兵如神,故能灭国四十,遂平西夏。”

清朝史学家魏源《元史新编》的評價是:“帝深沉有大略,用兵如神,故能灭国四十,遂平夏克金,有中原三分之二。使舍其攻西域之力,以从事汴京,则不俟太宗而大业定矣。然兵行西海、北海万里之外,昆仑、月竁重译不至之区,皆马足之所躏,如出入户闼焉。天地解而雷雨作,鹍鹏运而溟海立,固鸿荒未辟之乾坤矣。”

清朝史学家曾廉《元书》的評價是:“论曰:太祖崛起三河之源,奄有汉代匈奴故地,而兼西域城郭诸国,朔方之雄盛未有及之者也。遗谋灭金,竟如其策,金亡而宋亦下矣,此非其略有大过人者乎?又明于求才,近则辽金,远则西域,仇敌之裔,俘囚之虏,皆收为爪牙腹心,厥功烂焉,何其宏也,立贤无方,太祖有之矣。羽翼盛,斯其负风也大,子孙蒙业,遂一宇宙,不亦宜乎。”

民国史学家屠寄《蒙兀儿史记》的評價是:“论曰:旧史称成吉思汗深沉有大度,用兵如神,故能灭国四十,遂平西夏,信然。独惜军锋所至,屠刿生民如鹿豕,何其暴也。及至五星聚见东南,末命谆谆,始戒杀掠,岂所谓人之将死,其言善欤!蒙兀一代,并漠北四君数之,卜世十四,卜年蕲百六十,唐宋以降,享国历数,为由蹙于是者。于戏,可以观天道矣!”

民国官修正史《新元史》柯劭忞的評價是:“天下之势,由分而合,虽阻山限海、异类殊俗,终门于统一。太祖龙兴朔漠,践夏戡金,荡平西域,师行万里,犹出入户闼之内,三代而后未尝有也。天将大九州而一中外,使太祖抉其藩、躏其途,以穷其兵力之所及,虽谓华夷之大同,肇于博尔济锦氏,可也。” 

民国史学家张振佩《成吉思汗评传》(1943年版)绪言部分的評價是:“成吉思汗之功业扩大人类之世界观——促进中西文化之交流——创造民族新文化。”

1939年,处于抗战时期的中国共产党对成吉思汗做出了高度评价。6月21日,成吉思汗灵柩西迁途中到达延安时,中共中央和各界人士二万余人夹道迎灵,并在延安十里铺搭设灵堂,举行了盛大的祭祀活动。在此次祭祀仪式上,中共中央将成吉思汗正式尊称为“世界巨人”、“世界英杰”,并首次提出“继承成吉思汗精神坚持抗战到底”的口号。延安十里铺灵堂两侧悬挂一幅对联,灵堂正上方有一横联,内容如下:

横联:世界巨人
上联:蒙漢兩大民族更親密地團結起來

下联:繼承成吉思汗精神堅持抗戰到底

灵堂前面搭建一座牌楼,悬挂“恭迎成吉思汗靈柩”匾额。代表们将灵柩迎入灵堂后,举行祭典。中共中央、毛澤東、周恩來、朱德等敬献了花圈。由陕甘宁边区政府秘书长曹力如代表党政军民学各界恭读祭文:维中华民国二十八年六月二十一日,中国共产党中央委员会代表谢觉哉、国民革命军第八路军代表滕代远、陕甘宁边区政府代表高自立,率延安党政军民学各界,谨以清酌庶馐之奠,致祭于圣武皇帝成吉思汗之灵曰:

日寇逞兵,为祸中国,不分蒙汉,如出一辙。
嚣然反共,实则残良,汉蒙各族,皆眼中钉。
乃有奸人,蠢然附敌,汉有汉奸,蒙有蒙贼。
驱除败类,整我阵容,抗战到底,大义是宏。
顽固分子,准投降派,摩擦愈凶,敌愈称快。
巩固团结,唯一方针,有破坏者,群起而攻。
元朝太祖,世界英杰,今日郊迎,河山聚色。
而今而后,五族一家,真正团结,唯敌是挝。
平等自由,共同目的,道路虽艰,在乎努力。
艰苦奋斗,共产党人,煌煌纲领,救国救民。
祖武克绳,当仁不让,太旱盼霓,国人之望。
清凉岳岳,延水汤汤,此物此志,寄在酒浆。
尚飨!

1940年3月31日,中国共产党在延安成立了“蒙古文化促进会”,4月,在延安建立了“成吉思汗纪念堂”和“蒙古文化陈列馆”,敬立成吉思汗半身塑像,并由毛澤東题写了“成吉思汗紀念堂”七个大字。在这里每年农历三月二十一日,也就是成吉思汗春季查干苏鲁克大祭之日,延安各界举行盛大的祭奠仪式,以蒙汉两种语言诵读成吉思汗祭文。1942年5月5日,蒙古文化促进会还编辑出版了《延安各界纪念成吉思汗专刊》。毛澤東和朱德分别为专刊題詞,内容如下:毛澤東題詞:團結抗戰;朱德題詞:中華民族英雄。

毛泽东在1964年3月24日,在一次听取汇报时的插话中对成吉思汗、汉高祖刘邦、明太祖朱元璋的治国能力评价如下:“可不要看不起老粗。”“知识分子是比较最没有知识的,历史上当皇帝的,有许多是知识分子,是没有出息的:隋炀帝,就是一个会做文章、诗词的人;陈后主、李后主,都是能诗善赋的人;宋徽宗,既能写诗又能绘画。一些老粗能办大事:成吉思汗,是不识字的老粗;刘邦,也不认识几个字,是老粗;朱元璋也不识字,是个放牛的。”(毛泽东举例只是为了强调“一些老粗能办大事”,并不是说成吉思汗和刘邦真的不识字,也不是说刘邦只认识几个字。事实上,成吉思汗,刘邦,朱元璋三人原本可能僅能粗通文字,但當他們身为帝王時,他们的文化水平已經达到批阅奏折和签署命令的程度,甚至能為唱和文章。刘邦和朱元璋的文化水平不必细谈,相关史书记载很多,至于成吉思汗,元初名臣耶律楚材在《玄风庆会录》一书中提到成吉思汗是可以亲自阅览文件的。)

1941年十一月三日国民政府正式宣布对日本及德国、意大利宣战前夕,蒋介石赶赴甘肃省榆中县兴隆山,对成吉思汗灵寝举行了大祭。蒙藏委员会委员长吴中信代表国民政府恭读祭文:維中華民國三十年十一月三日國防最高委員會委員長蔣中正,特派蒙藏委員會委員長吳中信,以馬羊帛酒香花之儀,致祭於成吉思汗之靈而昭告以文曰:

繄我中華,五族為家,自昔漢唐盛世,文德所被,蓋已統乎西域極於流沙,洎夫大汗崛起,武功熠耀,馬嘶弓振,風撥雲拏,縱橫帶甲,馳驟歐亞,奄有萬邦,混一書車,其天縱神武之所肇造,雖曆稽往古九有之英傑而莫之能加,比者蝦夷小醜,虺毒包藏,興戎問鼎,豕突倡狂,致我先哲之靈寢乍寧處而不遑,中正忝領全民,撻伐斯張,一心一德,慷慨騰驤,前僕後興,誓殄強梁,請聽億萬鐵馬金戈之凱奏,終將相複於伊金霍洛之故鄉,緬威靈之赫赫兮天蒼蒼,撫大漠之蕩蕩兮風泱泱,修精誠以感通兮興隆在望,萬馬胙而陳體漿兮神其來嘗。尚饗。

1957年三月十二日,蒋介石在在主持陸軍指揮參謀學校正×期開學典禮講——《軍事哲學對於一般將領的重要性》中,评价成吉思汗:“我在此還要舉出我們中國歷史中兩位最有名的勇將來作一對照,以供我們今日軍人的抉擇。這兩位勇將中的第一位,就是漢楚時代的項羽。第二位就是縱橫歐亞的成吉思汗。這二位英勇無比的名將,其平生戰績乃是眾所周知,無待詳述,可是其結果則完全不同。茲據其二人所製的歌詞的氣概與精神,就可想見膽力的強弱與事業的成敗了。當成吉思汗西征時的歌詞是:「上天與下地,俯伏嘯以齊,何物蠢小醜,而敢當馬蹄」。而項羽最後失敗時的歌詞則是:「力拔山兮氣蓋世,時不濟兮騅不逝,騅不逝兮可奈何,虞兮虞兮奈若何?」後來還有許多人評判項羽這首歌詞是悲歌慷慨,不失為英雄氣概;我以為項羽的歌詞充滿了「恐懼」「憤怒」「疑惑」的氣氛,毫無英勇鎮定與自信的心理,更沒有如克勞塞維茨所說:「在絕望中之奮鬥」的軍人精神。所以到了最後他只有在烏江自刎了事。我以為這種卑怯自殺,而不能抱定榮譽戰死的軍人,只可說是一個最無志氣的懦夫,那能配稱為勇將!故無論他過去有如何勇敢的史蹟,我們不僅不屑敬仰他,而且應在棄絕不齒之列。至於成吉思汗的這首歌詞,我認為是充滿了他自信、勇敢與鎮定的心理,誠不失為一首英勇壯烈的歌詞,正與項羽的歌詞語意完全相反,所以他成功亦自不同。因為他既有這樣一個戰勝一切的信心,自然不會再有恐懼憤怒與疑惑的心理了。所以成吉思汗,實為我們中國軍人所應該效法與崇敬的第一等模範英雄。”
中華民國總統馬英九在2009年4月16日(农历三月二十一日)“二00九年中枢致祭成陵大典”中,特派蒙藏委员会委员长高思博主祭成吉思汗。祭坛上陈放有成吉思汗的画像,摆放有鲜花、水果和糕点,点燃供烛。仪式遵循古礼。台北市国乐团演奏乐曲《万寿无疆》。身穿长袍马褂的高思博,依序向成吉思汗像献香、献花、献爵(献酒)、献帛(献哈达)。司仪宣读祭文:“马英九特派蒙藏委员会委员长高思博敬以香花清酌之仪致祭于成吉思汗之灵曰:‘维汗休烈,雄才大略。天挺英明,龙兴溯漠。……礼仪孔修,有芘其芳。神之格思,德音不忘。’”

馬英九在2010年5月4日(农历三月二十一日)蒙藏委員會上午舉辦的“99年中樞祭成吉思汗大祭”典禮中,指派蒙藏委員會委員長高思博以香花清酌儀式祭拜成吉思汗。典禮安排向成吉思汗像獻花、獻香、獻爵(獻酒)、獻帛(獻哈达),並宣讀“中華民國總統祭文”,相關司祭者皆穿著蒙古傳統服飾,儀式遵循古禮,場面莊嚴隆重,馬英九在祭文中,肯定成吉思汗“雄才大略,天挺英明,拓土開疆,威震萬國。”

馬克思在《馬克思印度史編年稿》中谈到成吉思汗时曾说:“成吉思汗戎馬倥傯,征戰終生,統一了蒙古,為中國統一而戰,祖孫三代鏖戰六七十年,其後征服民族多至720部。”

瑞典學者多桑在其《蒙古史》中對成吉思汗的一生總結分析,多桑認為為成吉思汗之成功乃由於其具有極強的貪慾以及非常之野心。多桑稱他“狂傲”地妄想征服世界,死前還囑咐其子孫完成他的事業。

英国学者莱穆在《全人类帝王成吉思汗》一书中说:“成吉思汗是比欧洲历史舞台上所有的优秀人物更大规模的征服者。他不是通常尺度能够衡量的人物。他所统率的军队的足迹不能以里数来计量,实际上只能以经纬度来衡量。”

印度总理尼赫鲁在《怎样对待世界历史》一书中说:“蒙古人在战场上取得如此伟大的胜利,这并不靠兵马之众多,而靠的是严谨的纪律、制度和可行的组织。也可以说,那些辉煌的成就来自于成吉思汗的指挥艺术。”

「卡內基全球生態研究部」:「歷史上『最環保的侵略者』。因為殺人無數,讓大片耕地恢復成為森林,讓大氣中的碳大幅減量達7億噸!」

美国西維吉尼亞大學的研究人员指出成吉思汗的成功恰逢当时1000年来最温和、最潮湿的天气,之前的1180-1190年间,蒙古曾经历严重干旱,之后的温和湿润气候有助于青草的繁茂生长,为以骑兵为主的蒙古大军的战马提供了丰富的饲料。

1999年12月的美国A+E电视网评选出过去千年影响最深远的100大人物,成吉思汗被列为第22位(在亚洲人中仅次于第17位的甘地)。

\subsection{睿宗拖雷生平}

拖雷(1191年-1232年)又译图垒,是元太祖成吉思汗的幼子,排行第四。拖雷和正妻唆鲁禾帖尼生有四子:蒙哥、忽必烈、旭烈兀、阿里不哥。据《元史》,拖雷一共有子十一人。1227年8月25日至1229年9月13日担任大蒙古国(蒙古帝国)监国,历时二年。

1213年,成吉思汗分兵伐金,拖雷从其父率领中路军,攻克宣德府,再攻德兴府。拖雷与驸马赤驹先登,拔其城。即而挥师南下,拨涿州、易州,残破河北、山东诸郡县。1219年,从成吉思汗西征,攻陷布哈拉、撒馬爾罕。1221年,分领一军进入呼罗珊境,陷马鲁、尼沙不儿,渡搠搠阑河,降也里。遂与成吉思汗合兵攻塔里寒寨。

按照蒙古习俗,幼子继承父业,而年长诸子则分析外出、自谋生计。故成吉思汗生前分封诸子,拖雷留置父母身边,继承父亲所有在斡难和怯绿连的斡耳朵、牧地及军队。成吉思汗留下的军队共有12.9万人。其中10.1万的精銳俱由拖雷继承。

1221年拖雷屠杀木鹿(今梅尔夫)城中居民,超过100萬人。除去四百个工匠之外,其余人口被屠杀殆尽,城墙被毁。从此木鹿结束了繁荣的历史。

1227年8月25日成吉思汗病逝後,由拖雷監国,称也可那颜。直至两年后在选举大汗的忽里臺時,拖雷和察合台等宗王们在1229年9月13日一起推举元太宗窩闊臺即大汗位。

1231年,与窩闊臺分道伐金,拖雷总右军自陝西鳳翔渡渭水,过宝鸡,入大散关。11月,蒙古军假道南宋境,沿汉水而下,经兴元(今陕西汉中)、洋州(今陕西洋县)在均州(今湖北均县西北)、光化(今湖北光化北)一带,渡汉水,迂迴北上入金境。1232年初与金军在均州(今河南禹县)遭遇。拖雷乘雪夜天寒(有一康里人作法)大败金将完颜合达、移剌蒲阿、完颜陳和尚於三峰山,尽歼金军精锐。此役毕,拖雷与自白坡渡河南下的窩闊臺军会合。

1232年农历九月,拖雷在北返蒙古草原途中逝世。據蒙古秘史,窩闊臺在一場重要戰爭中得了重疾,為了治愈窩闊臺,拖雷決定犧牲自己。巫師認為,窩闊臺所得的惡疾的病源是由中土的水和土之靈而來。水土之靈不滿蒙古人把中土臣民趕出中土,和不滿蒙古人令中土滿目瘡痍。若以中土的土地,動物和人作祭品,只會令窩闊臺的病情更加惡化,但若是他們願意犧牲家庭成員,窩闊臺便能好過來,於是拖雷主動飮了被詛咒的飲料後死亡。另一說法,是拖雷可能因酗酒過量而死。

1251年7月1日,元宪宗蒙哥即位,大蒙古国(蒙古帝国)皇位从窝阔台家族转入拖雷家族,元宪宗追尊父亲拖雷为皇帝,为拖雷追上尊谥庙号,庙号睿宗,谥号英武皇帝。

至元三年十月十八日(1266年11月16日),太庙建成,制尊谥庙号,元世祖忽必烈将父亲拖雷的谥号由英武皇帝改谥为景襄皇帝。 

至大二年十二月六日(1310年1月7日),元武宗海山为拖雷加上尊谥仁圣,从此之后,拖雷的谥号变为仁圣景襄皇帝。《睿宗皇帝加上尊谥册文》,内容如下:“伏以诣泰坛而请命,有称天以诔之文;荐清庙而致严,盖若昔相承之典。刚辰爰卜,遗美载扬。钦惟睿宗景襄皇帝孝友温恭,聪明浚哲。属我家肈造于朔土,佐圣祖遄征于四方。逮天讨之奉行,致皇威之远畅。金源假两河之息,天水渝通好之盟,遂移秦陇之师,爰有褒斜之举。既平南郑,顺流而东,再涉襄江,自上而下,乃眷三峰之捷,实开万世之基。唇既亡而齿亦寒,虢可伐而虞不腊。适英文之违豫,图中夏之底宁。毋作神羞,请以身代。爰俟金縢之启,已知宝祚之归。迪我后人,绍兹明命。徽称显号,虽已拟诸形容;玉检金泥,尚未遑于润色。奉玉册玉宝,加上尊谥曰仁圣景襄皇帝,庙号睿宗。伏惟端临扆座,诞受鸿名。亿万斯年,永锡繁祉。”

据《元史》,拖雷有子十一人:长子蒙哥,次子忽睹都、四子忽必烈、六子旭烈兀、七子阿里不哥、八子拨绰(不者克)、九子末哥、十子岁哥都、十一子雪别台,第三子和第五子失其名。

拖雷和正妻唆鲁禾帖尼所生的四子皆有所成,元宪宗蒙哥和元世祖忽必烈相继做过大元(大蒙古国)的帝王,蒙元皇帝由拖雷一系繼承。旭烈兀在西亚开创了伊儿汗国,1259年蒙哥去世后,阿里不哥在1260年在蒙古本土的庫力臺大會被部分王公推举即位,并和忽必烈争位达四年之久。

拖雷可考的女儿有二:赵国公主独木干,下嫁汪古部聂古得;鲁国公主也速不花,下嫁弘吉剌部斡陈。

民国官修正史《新元史》柯劭忞的评价是:“周公金縢之事,三代以后能继之者,惟拖雷一人。太宗愈,而拖雷竟卒,或为事之适然,然孝弟之至,可以感动鬼神无疑也。世俗浅薄者,乃疑其诬妄,过矣!”

\subsection{太宗窝阔台生平}

窝阔台汗(1186年11月7日-1241年12月11日),又作斡歌歹、和歌台、倭闊岱等,孛儿只斤氏,成吉思汗第三子,大蒙古国大汗。他是蒙古帝国第二位大汗,1229年9月13日—1241年12月11日在位,在位12年零3个月。他登基时接受大汗的称号,和诸汗相区别。

至元三年(1266年)十月,太庙成,元廷追尊庙号太宗,谥英文皇帝。

1229年9月13日(农历八月二十四日),窝阔台在库里尔台大会中被察合台、拖雷、铁木哥斡赤斤等宗王和大臣推举为大蒙古国大汗,管理整個蒙古帝国,有史料载诸宗王和百官为窝阔台上尊号曰木亦坚合罕(合罕为大汗的别译)。

他继承父親的遺志擴張領土,主要是繼續西征和南下中原。他在位期間成功完全征服中亞和華北。内政方面,以契丹人耶律楚材為相管理华北和中原地区,在这些地区稍微改变了战后屠城作風,保存不少金朝遺民和政治制度;同時又依耶律楚材建議,提拔漢人為官,整頓內治,安定了蒙古在華北地區的统治,使华北地区经济在戰後得到一定程度的恢复性發展,為日後忽必烈称帝滅南宋打下基礎。

灭金取中原:1229年登基的时候,大蒙古国在东亚部分的东南部大体以黄河为界,金朝领土基本上只剩下黄河以南的河南、陕西等地(当时的黄河取道江苏北部的淮河入海)。

1231年,窝阔台与其四弟拖雷分道进攻金朝,1232年初,拖雷率蒙古军在河南三峰山战胜金军,尽歼金军精锐。其后,拖雷与自白坡渡河南下的窝阔台军会合,一同北返蒙古草原,1232年农历九月,拖雷於北返途中病死之後,托雷四子忽必烈继承了他在华北地区的势力。

1232年春,蒙古军队继续南下,抵达金朝首都燕京(今北京市)附近,因此周围州县难民纷纷逃入汴京(今河南开封市),城中人口激增,而入夏后瘟疫流行,死者达九十餘万人。1232年秋,蒙古派使者入城要求金朝投降,被金朝将士所杀,蒙古军于是不再议和,击溃金朝援军,围困汴京城。

1233年2月6日(农历十二月二十六日),金哀宗和后妃们分别离开汴京,一路向南。1233年2月26日(农历正月十六日),金哀宗抵达归德(今河南商丘市),随后又出走;8月3日(农历六月二十六日),金哀宗逃到蔡州(今河南汝南县),在此地稳定下来。

1233年3月5日(农历正月二十三日),金朝汴京西面元帅崔立率军队杀死汴京的留守將領完颜奴申和完颜习捏阿不,控制全城,派使者向蒙古军统帅速不台投降。

1233年3月10日(农历正月二十八日),速不台向汴京进兵。速不台得知崔立同意投降后,因为之前进攻汴京时金人抗拒持久導致军队死伤甚多,便向窝阔台奏报建议軍隊入城后屠城泄愤。中书令耶律楚材坚决反对,他认为将士辛苦奋战为的就是土地和人民,屠城会导致得地无民,而且“奇巧之工,厚藏之家”都集中在汴京,屠城会导致一无所获,没有人民就没有人向朝廷交纳赋税,军队会白辛苦一场,最后窝阔台采纳了耶律楚材的意见,只关押了金朝宗室,其他人一概赦免。当时在汴京城中躲避兵祸的147万名居民因为耶律楚材的建议得以免于兵祸。

1233年5月29日(农历四月十九日),崔立将汴京城中的金朝宗室梁王完颜从恪、荆王完颜守纯以及其他宗室男女五百余人送到速不台军队驻地青城,速不台将他们送到漠北草原窝阔台的行銮驻跸之处,窝阔台为报祖先之仇(金熙宗当年曾将蒙古俺巴孩汗钉死在木驴上),将他们全部处死。同一天,崔立面见速不台,正式归降大蒙古国,速不台率军进入汴京,维护城中秩序,并将城中的金朝后妃和宗庙宝器也送到漠北草原窝阔台的行銮驻跸之处。

1234年2月9日(农历正月十日),大蒙古国军队與南宋军队联合攻入蔡州(今河南汝南县),金哀宗自杀,金末帝死于乱军之中,金朝灭亡。整个北方中原地区并入大蒙古国版图。

自1234年窝阔台汗灭金朝,到1368年烏哈噶圖汗 (元惠宗)逃离大都回到草原,由蒙古族建立的蒙古汗国、元帝國两政权,總共统治北方中原黄河流域长达134年。

端平入洛与蒙宋开战:1233年5月29日蒙古军队取得汴京(今河南开封市)后,继续进攻蔡州(金哀宗所在地),由于金朝军队抵抗顽强,为了减少损失,窝阔台决定联合南宋政權攻克蔡州灭亡金朝。

按照蒙宋双方协议,蒙宋联军攻克蔡州后,南宋可以取得蔡州未破前尚在金朝控制的河南土地,也就是唐、邓、蔡、颍、宿、泗、徐、邳等州(均位于河南南部)。这些州位于金朝和南宋的交界地带,属于金朝领土最南端的州。

在1234年2月9日蒙宋联军攻克蔡州灭亡金朝后,因为河南一带久经战火,田地荒芜,缺乏粮食,当时又正值冬季,天气严寒,于是把当地大部分居民暂时迁往河北一带,准备等天气转暖后将居民再陆续迁回河南,并恢复农业生产。同时军队久经战事,也需要休整,大部分军队撤到黄河以北。

宋理宗在部分大臣的怂恿下违背当初的蒙宋协议,1234年六月,宋军分二路出兵北伐,准备收复当年被金朝攻取的三京:西京河南府(今河南洛阳市)、东京开封府(今河南开封市)、南京应天府(亦称之为归德,今河南商丘市),这三京均位于河南北部,在蒙宋协议之前就已经被蒙古军队攻取,自然不属于当初蒙宋协议中灭金后南宋可以得到的领土。

由于宋军北上攻取三京发生在宋理宗端平年间,史称“端平入洛”。端平入洛揭开了蒙古与南宋对峙,连续四十余年不断战争的序幕,直到忽必烈渡过长江、灭亡南宋。

中國南宋违背蒙宋协议,大举进兵,但因为蒙古灭金后,大部分金朝军队和居民都已经撤到黄河以北,南宋军队最初进展顺利,一个月后顺利占领几乎是空城的三京。由于三京缺乏粮草,宋军携带粮草较少又缺乏后勤补给,蒙古军队又随后发起反击,宋军很快撤离三京,并撤回南宋境内。

蒙古军队随后追至原金朝和南宋的边界线一带,并向南宋边界的州县发起进攻,因为蒙古军队并不是很适合南方河流密布的地形作战,在取得一定战果后撤回中原。

自南宋违约进攻蒙古,端平入洛以后,南宋天灾人祸接连不断,国力逐渐衰弱直至灭亡。在军事上,收复三京失败,损兵折将,士气不振,将心不稳,成为南宋守边士兵面临的严重问题。

在中国北方实施“以儒治国”:1230年,有近臣别迭等人向窝阔台上奏,认为“汉人无补于国,可悉空其人以为牧地。”主张将汉人驱逐,把汉地的耕地变为牧场,耶律楚材则上奏请求均定中原地税、商税、盐、酒、铁冶、山泽之利,每年可得赋税白银50万两、帛8万匹、粟40余万石,足以支持窝阔台南征金朝的军队所需,窝阔台同意由耶律楚材试行。

1230年农历十一月,耶律楚材奏请在大蒙古国统治的黄河以北的河北、山西、山东等地(当时金朝尚未灭亡,黄河取道江苏北部的淮河入海)设立燕京等地设立十路征收课税使,并选用有名的儒士作为课税官员,得到窝阔台批准。

1231年农历八月,窝阔台到达云中(今山西大同市),十路征收课税使将当年征收到的汉地赋税簿册和金帛陈于廷中,窝阔台大悦,当日设立中书省,改侍从官名,以耶律楚材为中书令,粘合重山为左丞相,镇海为右丞相。

1235年春,窝阔台决定在哈拉和林建都城,修建万安宫;并部署伐南宋、征高丽和再次西征;1236年正月,万安宫建成。窝阔台大宴群臣,同月,窝阔台下诏发行纸币交钞。

1234年正月灭金朝后,窝阔台下诏括编汉地户籍,他接受耶律楚材的建议,以按户为单位收取赋税。由中州断事官失吉忽秃忽主持。1236年八月,括户完成,括得汉地民户110余万户。

1236年括户完成后,失吉忽秃忽主张按以往风俗在中原对诸王和有功之臣进行分封,窝阔台表示同意。耶律楚材力陈“裂土分民”的弊害,使窝阔台同意封地的官吏须朝廷任命,除常定赋役外,诸王勋臣不得擅自征敛,以限制诸王勋臣在封地的权力。

括户完成后,耶律楚材制订了中原赋税制度:每两户出丝一斤,上交朝廷,以供中央政府使用,每五户出丝一斤,以与所赐之家;先由中央政府征收,然后赐予该受封贵族,除此之外贵族不得擅加征敛。上田每亩税三升半,中田三升,下田二升,水田五升;商税三十分之一;盐每银一两四十斤。

这个赋税的定额是比较轻的,有利于当时已遭破坏的中原地区休养生息。在遇到大的灾情时,楚材还采取免征的措施。如果部分地区出现逃亡浮客,他们的赋税要由留下的主户负担,这些主户负担的赋税会重一些。此外,民户们也要负担一些随意性很大的杂泛差役。总的来说,民户们的负担还是相对比较轻的。

在耶律楚材的努力下,中原及北方的经济得到了恢复和保存。

1230年耶律楚材制定课税格,1231年收取的各种赋税中,白银为50万两,1234年灭金朝取得河南等地,赋税收入一直在增加,到了1238年,朝廷在中原汉地收取的各种赋税中,白银为110万两。丝和米等赋税也有显著增加。

1233年,为了培养蒙汉双语翻译类人材,窝阔台下诏在燕京(今北京市)建国子学,派遣蒙古人子弟18人学习汉语;汉人子弟12人,学习蒙古语和弓箭,并选儒士为教读。规定受业学生不仅要学习汉人文书,还要“兼谙匠艺,事及药材所用、彩色所出、地理州郡所纪,下至酒醴麴蘖、水银之造,饮食烹饪之制,皆欲周览旁通”。当时,全真教在燕京势力很大,儒家士大夫有很多托庇于全真教。燕京的学宫也是如此,学宫的主持者除杨惟中之外,葛志先、李志常均为当时有名的全真道士。

1233年农历四月,蒙古军队进入汴京城(今河南开封市),中书令耶律楚材向窝阔台奏请遣人入城,求孔子家族後代,得五十一代孙元措,奏袭封衍圣公,付以孔林庙地。耶律楚材又派人入汴京,挑选了大量的人才。

1233年农历六月,窝阔台下诏,以孔子五十一世孙孔元措袭封衍圣公。

1233年冬天,窝阔台敕修燕京孔子庙及浑天仪。

1236年农历三月,复修孔子庙及司天台。

1236年农历六月,耶律楚材奏请窝阔台同意后,在燕京(今北京市)建立编修所,在平阳(今山西临汾市)建立经籍所,主持经史类书籍的编纂和刊行,召儒士梁陟充长官,以王万庆、赵著副之。让他们直释九经,进讲东宫。又率大臣子孙,执经解义,使他们知道圣人之道。

1237年,窝阔台下旨蠲免孔子、孟子、颜子等儒教圣人子孙的差发杂役。

1237年,耶律楚材奏请对儒士举行科举考试,这就是1238年举行的戊戌选试,共录取4030人,皆当时的名士。

1238年,耶律楚材又支持杨惟中和姚枢在燕京建立太极书院,请赵复等人为师教授儒家的经典。南宋名士赵复的讲学,使程朱理学在北方中原地区传播开来。

戊戌选试:1234年2月9日,蒙古帝国灭金朝,夺取中原地区后,急需人才治理国家。

元太宗九年农历八月二十五日(1237年9月15日),根据中书令耶律楚材的建议,窝阔台下诏书命断事官术忽德和山西东路课税所长官刘中,历诸路考试,试诸路儒士,开科取士,并对考试内容和参加考试者的身份要求以及中选者的优厚待遇作了详细说明。

北方中原地区的诸路考试,均于1238年(戊戌年)举行,史称“戊戌选试”。

1238年的这次考试共录取东平杨奂等4030人,皆为一时名士,使得朝廷及时得到了加强统治所需要的各方面的人才。但后来“当世或以为非便,事复中止”。

直到元仁宗1313年下诏恢复科举,此时距离元太宗1238年的“戊戌选试”已经有75年,天下读书的士人至此再次获得以科举方式晋身做官的途径。

西征欧洲:1234年2月9日金朝灭亡后,由於大蒙古国與南宋接壤,使雙方的衝突日漸加劇,也拉開了雙方往後45年不斷爭戰的序幕。在南方戰線僵持不下之時,蒙古大軍的鐵蹄轉往東方的高麗,並使之臣服,西線方面,以拔都為首的欽察汗國,完全控制了罗斯,並繼續西進,佔領了除诺夫哥罗德以外俄羅斯的领土,以及波蘭和匈牙利的一部。

1241年12月11日(农历十一月八日),窝阔台因為酗酒而突然暴斃,使他的西征進程被逼中止。當時大軍正朝往維也納推進,但為了趕返參加位於蒙古的库里尔台大会而急忙撤軍,自此以後,蒙古大軍再也沒有踏足這片土地。

1241年年底,在窝阔台去世后不久,他的二哥察合台去世。

窝阔台去世后,1242年春天,皇后乃马真后开始称制,处理朝政,直到1246年8月24日窝阔台之子貴由繼任大汗為止。乃马真后临朝称制期间,朝政比较混乱,中书令耶律楚材力争而不能有效果,于1244年农历五月忧愤而死。

元朝重臣郝经在中统元年(1260年)农历八月给元世祖忽必烈的上书《立政议》中对元太宗窝阔台的評價是:“当太宗皇帝临御之时,耶律楚材为相,定税赋,立造作,榷宣课,分郡县,籍户口,理狱讼,别军民,设科举,推恩肆赦,方有志于天下,而一二不逞之人,投隙抵罅,相与排摈,百计攻讦,乘宫闱违豫之际,恣为矫诬,卒使楚材愤悒以死。”(说明:元太宗窝阔台在世之时,耶律楚材还是深受重用的,1241年元太宗去世,帝位空缺,皇后乃马真后开始临朝称制,朝政比较混乱,中书令耶律楚材力争而无效果,他于1244年忧愤而死)

明朝官修正史《元史》宋濂等的評價是:“帝有宽弘之量,忠恕之心,量时度力,举无过事,华夏富庶,羊马成群,旅不赍粮,时称治平。”

清朝史学家邵远平《元史类编》的評價是:“册曰:嗣业恢基,缵绪立制;五载灭金,十路命使;定赋崇儒,用昌厥世;仁厚恭俭,时称平治。”

清朝史学家毕沅《续资治通鉴》的評價是:“太宗性宽恕,量时度力,举无过事。境内富庶,旅不赍粮,时称治平。”

清朝史学家魏源《元史新编》的評價是:“帝有宽宏之量,淳朴之质,乘开国之运,师武臣力,继志述事,席卷西域,奄有中原。惟知诸子不材,又知宪宗之克荷,而储位不早定,致身后政擅宫闱,大业几沦,有余憾焉。”

清朝史学家曾廉《元书》的評價是:“论曰:太宗时金人已弱,然犹足阻河为固也。太宗遵遗令戡凤翔,道兴元,以达唐邓,而汴梁墟,可谓闻斯行之矣。当是时,操持国政,耶律楚材郁为时栋。然太宗之用楚材,以利也。太宗言利,楚材即以其利利天下,而纪纲粗立矣。用相违也,而相成也,岂非天哉!故开国之运,云龙风虎,非雷同也。”

清末民初史学家屠寄《蒙兀儿史记》的評價是:“论曰:财者,一国所公有也。语曰:百姓足,君孰与不足?人君以国用困乏,多取于民,然且不可。况可纵奸人异类,恣其侵夺乎?斡歌歹汗初得金,许奥都剌合蛮扑买中原银课,举国家财政大权授之贾胡之手,公利而私取之,上下交损焉。封建之制,始于自然,强并弱,众暴寡。自天子以至食采之大夫,各私其土地人民。古圣王不得以而仍之。秦汉以降,此制渐废,偶一行之,罔不召乱。自非至无识者,不轻议复也。汗括汉户,分赐诸王贵戚,其视无辜之民与奴虏奚择。彼固不知封建为何事,然斯制若行,弊且甚于封建。微耶律楚材言,纵虎豹而食人肉矣。前史称汗有宽仁之量,忠恕之心,度时量力,动无过举。迹其立站赤、选税使、试儒士、释俘囚,诏免旱蝗之租,代偿羊羔之息,固非无志于民者,惜乎不达怡体,而左右之人将顺其美者,又寡也。”

民国官修正史《新元史》柯劭忞的評價是:“太宗宽平仁恕,有人君之量。常谓即位之后,有四功、四过:灭金,立站赤,设诸路探马赤,无水处使百姓凿井,朕之四功;饮酒,括叔父斡赤斤部女子,筑围墙妨兄弟之射猎,以私撼杀功臣朵豁勒,朕之四过也。然信任奥都拉合蛮,始终不悟其奸,尤为帝知人之累云。”

\subsection{定宗貴由生平}

貴由汗(1206年-1248年4月),大蒙古国第三任大汗,孛儿只斤氏,窩闊台長子,乃馬真后所生,1246年8月24日—1248年在位,计2年。

至元三年(1266年)十月,太庙建成,追尊庙号定宗,谥简平皇帝,在宗庙中列祭于第七室,排在忽必烈之父托雷后、忽必烈之兄蒙哥前。

早年參加征伐金朝,俘虜了其親王;又曾經参与西征欧洲。蒙古帝国第三任大汗贵由、第四任大汗蒙哥,以及后来的元朝开国皇帝忽必烈,堂兄弟三人都是蒙古第二次西征时拔都的部下。

1241年12月11日,窝阔台去世,汗位虚悬,贵由的母亲乃马真脱列哥那称制,法纪混乱,很多宗王贵族滥发牌符征敛财物,唯有拖雷家族没有这样做,赢得了声誉。乃马真后欲立长子贵由为大汗,拔都与贵由不和,一直不肯参加选汗大会,后来,成吉思汗幼弟铁木哥斡赤斤也领兵来争位,帝国面临汗位争夺战和混乱的危险。拖雷的遗孀唆鲁禾帖尼决定率诸子参加忽里勒台大会,1246年8月24日,宗王大臣们拥立贵由登基,贵由成为大蒙古国大汗,“全体宗王们脱帽,解开宽腰带,把贵由扶上金王位,以汗号称呼他,到会者对新君九拜表示归顺,在帐外的藩王及外国使臣等也同时跪拜称贺。”

贵由登基后,虽然本人很有权威,但是因沉湎酒色、手足痉挛,并没有什么作为,且不理政事,多委于下臣。

1248年春,貴由親率大軍西征拔都,至橫相乙兒(今新疆青河縣東南)病死。一說被拔都系势力毒殺。

1246年8月24日至1248年4月20日在位,在位仅一年零八个月。

和罗马教宗的交往:贵由在位期间和罗马教宗有交往。歐洲諸國傳言蒙古大汗信仰基督教,因此教宗诺森四世派遣若望·柏郎嘉宾出使,希望勸說蒙古大汗不要傷害基督徒,同時要他深入了解蒙古人的風土民情、作戰方式等。1245年4月16日从法国里昂出发,途经神圣罗马帝国、波兰王国和基辅罗斯等国(他于1246年2月3日离开基辅)。1246年4月4日,他在伏尔加河下游的萨莱(今伏尔加河下游阿斯特拉罕附近)受到钦察汗拔都的接见。拔都派他去蒙古草原见大汗,他经过讹答剌、伊犁河下游、叶密立河—翻越阿尔泰山,向东抵达蒙古草原。

1246年7月22日,他抵达距离哈拉和林只有半天路程的地方,选举大汗的忽里勒台大会正在此召开。他目睹了1246年8月24日贵由的当选,并留下了对贵由的生动描述:“在他当选时,约有四十,最多四十五岁。他是中等身材,非常聪明.极为精明,举止极为严肃庄重。从来没有看见他放声大笑,或者是寻欢作乐。” 最後他未能说服贵由皈依天主教,得到贵由的回信后,于1246年11月13日离开蒙古草原,向西踏上归途,经伏尔加河下游的拔都驻地返回西方,1247年9月5日他到达拔都驻地,又经基辅返回西方。

凉州会盟与吐蕃归附:1247年,吐蕃诸部宗教界领袖萨迦班智达·贡嘎坚赞(简称萨班)同大蒙古国皇子西凉王阔端(贵由之弟,窝阔台之子,成吉思汗之孙)在凉州(今中国甘肃武威市)议定了吐蕃归附的条件,其中包括呈献图册,交纳贡物,接受派官设治,吐蕃地区纳入大蒙古国(蒙古帝国)治下,史称“凉州会盟”。

窝阔台家族的衰落:根据《新元史》记载,1248年农历三月(1248年4月),贵由以养病为名带兵西巡,途中病逝于横相乙儿(今新疆青河东南),距離別失八里一天路程。

贵由死后,其遗孀斡兀立海迷失临朝称制,由於贵由与拔都早年不和,拔都拒絕奔喪。为了对抗窝阔台家族,拔都以长支宗王的身份遣使邀请宗王、大臣到他在中亚草原的驻地召开忽里台,商议推举新大汗。窝阔台系和察合台系的宗王们多数拒绝前往,海迷失后只派大臣八剌为代表到会。唆鲁禾帖尼则命长子蒙哥率诸弟及家臣应召前往。

1250年,庫力臺大會在中亚地区拔都的驻地召开,拔都在会上极力称赞蒙哥能力出众,又有西征大功,应当即位,并指出贵由之立违背了窝阔台遗命(窝阔台遗命失烈门即位),窝阔台后人无继承汗位的资格。大会通过了拔都的提议,推举蒙哥为大汗。窝阔台、察合台两家拒不承认,唆鲁禾帖尼和蒙哥又遣使邀集各支宗王到斡难河畔召开忽里台,拔都派其弟别儿哥率大军随同蒙哥前往斡难河畔,但窝阔台、察合台两家很多宗王仍不肯应召,大会拖延了很长时间。

由于蒙哥的母亲唆鲁禾帖尼的威望甚高,并且善于笼络宗王贵族,多数宗王大臣最终应召前来,1251年农历六月在蒙古草原斡难河畔举行庫力臺大會,元宪宗元年农历六月十一日(1251年7月1日),宗王大臣们共同拥戴蒙哥即大汗位。此后,为了巩固汗位,唆鲁禾帖尼在镇压反对者时毫不留情,并亲自下令处死贵由的皇后斡兀立海迷失。

自此汗位繼承,便由窝阔台家族转移到了拖雷家族,皇族内部的分裂,为后来大蒙古国的彻底分裂,埋下伏筆。

明朝官修正史《元史》宋濂等的評價是:“三年戊申春三月,帝崩于横相乙儿之地。……是岁大旱,河水尽涸,野草自焚,牛马十死八九,人不聊生。诸王及各部又遣使于燕京迤南诸郡,征求货财、弓矢、鞍辔之物,或于西域回鹘索取珠玑,或于海东楼取鹰鹘,驲骑络绎,昼夜不绝,民力益困。然自壬寅以来,法度不一,内外离心,而太宗之政衰矣。”

清朝史学家毕沅《续资治通鉴》的評價是:“自太宗皇后称制以来,法度不一,内外离心。至是国内大旱,河内尽涸,野草自焚,牛马死者十八九,人不聊生。诸王及各部,又遣使于诸郡征求货财,或于西蕃、回鹘索取珠玑,或于东海搜取鹰、鹘、驿骑络绎,昼夜不绝,民力益困。皇后立库春子实勒们听政,诸王大臣多不服。”

清朝史学家魏源《元史新编》的評價是:“连岁大旱,河水尽涸,野草自焚,牛马十死八九,人不聊生。诸王及各部又遣使于燕京迤南诸郡,征求货财,或于西域、回鹘索取珠玑,或于海东搜取鹰鹘,驿骑不绝,内外离心,故无可纪。然自太祖崩后,拖雷监国者一年,太宗崩后,六皇后称制者四年,定宗之后,皇后临朝者又几四年,前后凡九载无君而国不乱,卒能创业垂统,上竝漢、唐者,则皆宗王宿将维持拱卫,根干蟠据之力。”

清朝史学家曾廉《元书》的評價是:“论曰:定宗之世,事多缺漏,而前史曰:‘ 帝崩之岁大旱,河水尽涸,野草自焚,牛马十死八九,人不聊生。诸王及各部又遣使于燕京迤南诸部,征求货财、弓矢、鞍辔,或于西域回鹘索取珠玑,海东索取鹰鹘,驿骑络绎,昼夜不绝,民力益困。然自壬寅以来,法度不一,内外离心,而太宗之政衰矣。’其言壬寅,盖以昭慈皇后称制时言之也。夫定宗即位时,年四十矣,而不能辑诸王侯大将,纪解威亵,此太宗之不谋付以匕图者乎?然在于汉亦孝惠之亚也。惟无良臣为之辅弼,而宗藩党羽遂成,以夺皇阼。炎异之丛,兴其足信耶?而失烈门则太宗遗诏所立也。前史复曰:定宗崩后,三岁无君。蒙哥之党之不欲以为君,非蒙古之无君也。窜之北陲,并逐太宗皇后而弑定宗皇后,可不谓之逆哉!自是而太宗子孙亦不欲以蒙哥兄弟为君,逮于海都,而中原震矣。”

中華民国史学家屠寄《蒙兀儿史记》的評價是:“汗严重有威,临御未久,不及设施,惟乃蛮真可敦称制时,威福下移,汗既亲政,纲纪粗立,君权复尊,自幼多疾,成吉思汗尝命亦鲁王之祖忽鲁扎克为之主膳。中年性好酒色,手足有拘挛之病,在位之日,常以疾不视事,事多决于大臣镇海、合答二人云。”

中華民国官修正史《新元史》柯劭忞的評價是:“史臣曰:定宗诛奥部拉合蛮,用镇海、耶律铸,赏罚之明,非太宗所及。又乃马真皇后之弊政,皆为帝所铲革。旧史不详考其事,谓前人之业自帝而衰,诬莫其矣。” 

\subsection{宪宗蒙哥生平}

蒙哥汗(1209年1月10日-1259年8月11日),大蒙古国第四任大汗,也是大蒙古国分裂前最後一個受普遍承認的大汗。他是成吉思汗幼子拖雷的长子、窝阔台的养子,由窝阔台的昂灰皇后抚养长大。

1251年7月1日登基,在位8年零2个月。其间长期主持对南宋、大理的战争,为其弟忽必烈最终建立元朝奠定坚实基础。至元三年(1266年)十月,太庙成,元廷追尊蒙哥庙号为宪宗,谥桓肃皇帝 。

潜邸岁月:1209年1月10日(农历戊辰年十二月三日),蒙哥生于漠北草原,是成吉思汗之孙,拖雷的长子,拖雷正妻唆鲁禾帖尼所生的嫡长子(元世祖忽必烈是嫡次子,旭烈兀是嫡三子,阿里不哥是嫡四子)。窝阔台汗即位之前,以蒙哥为养子,让昂灰皇后抚育蒙哥,并在他长大后,为他娶火鲁剌部女子火里差为妃、分给他部民。至1232年拖雷去世后,蒙哥才回去继承拖雷的封地。蒙哥多次跟随窝阔台参加征伐,屡立奇功。蒙哥沉默寡言、不好侈靡,喜歡打獵。1235年,蒙哥参加第二次蒙古西征,與拔都、貴由西征欧洲的不里阿耳、欽察、斡羅思等地,屢立戰功,在里海附近,活捉钦察首领八赤蛮。

拖雷家族争得大汗之位:1248年农历三月贵由汗去世后,由皇后斡兀立海迷失临朝称制;由於与贵由早年不和,拔都(铁木真长子术赤之子)拒絕奔喪。为了对抗窝阔台家族,拔都以长支宗王的身份遣使邀请宗王、大臣到他的驻地(在中亚草原)召开忽里台(蒙古的军政会议),商议推举新大汗。窝阔台系和察合台系的宗王们多数拒绝前往,贵由汗的皇后斡兀立海迷失只派大臣八剌为代表與会。托雷之妻唆鲁禾帖尼则命长子蒙哥率诸弟及家臣应召前往。

1250年,忽里台大会在拔都的驻地(中亚地区)召开,拔都在会上极力称赞蒙哥能力出众,又有西征大功,应当即位,并指出贵由之立违背了窝阔台遗命(窝阔台遗命失烈门即位),窝阔台后人不当有继承汗位的资格。大会通过了拔都的提议,推举蒙哥为大汗。窝阔台、察合台两家拒不承认,唆鲁禾帖尼和蒙哥又遣使邀集各支宗王到斡难河畔召开忽里台,拔都派其弟别儿哥率大军随同蒙哥前往斡难河畔,但窝阔台、察合台两家的很多宗王仍不肯应召,大会拖延了很长时间。

由于唆鲁禾帖尼威望甚高,并且善于笼络宗王贵族,最终多数宗王大臣应召前来,1251年农历六月在蒙古草原斡难河畔举行忽里台,宗王大臣们于7月1日(农历六月十一日)共同拥戴蒙哥登基,蒙哥成为大蒙古国大汗;蒙哥即位的当日,尊母亲唆鲁禾帖尼为皇太后。此后,为了巩固汗位,皇太后唆鲁禾帖尼镇压反对者毫不留情,并亲自下令处死贵由汗的皇后斡兀立海迷失。

自此“大汗”之位的繼承,便由窝阔台家族转移到了拖雷家族,为后来大蒙古国分裂埋下伏筆。

1251年7月1日,蒙哥即位后,窩闊台系諸宗王拒絕承認,被蒙哥率兵鎮壓;蒙哥又以其弟忽必烈统領漠南漢地軍政事務,同时指挥向南(东亚)、向西(西亚)两个方向的征服战争。

征服大理:1252年农历六月,命弟忽必烈南征大理国,次月,忽必烈率军出发。1253年农历八月,忽必烈军至陕西,开始进攻位于今云南等地的大理国。1254年1月2日(元宪宗三年农历十二月十二日),忽必烈攻克大理城,大理国王段兴智投降,大理国灭亡,并入大蒙古国版图。1256年,段兴智前往漠北和林觐见蒙哥汗,被任命為大理總管,子孙世襲。

从1254年大蒙古国忽必烈奉命灭大理国、大理国王战败投降,到1382年驻守云南的元朝梁王把匝剌瓦尔密兵败自杀、元朝大理总管段世战败归降明军,蒙古族建立的政权统治云南地区长达128年。

远征西亚:元宪宗三年(1253年)六月,蒙哥命弟旭烈兀率大军十万西征。旭烈兀的西征军从漠北草原出发,1256年大军渡过阿姆河后所向披靡,先攻灭波斯南部的卢尔人政权,1256年攻灭位于波斯西部的木剌夷国(阿萨辛派),1258年灭亡巴格达的阿拔斯王朝,1260年3月1日,灭亡叙利亚的阿尤布王朝,并派兵攻占了小亚细亚大部分地区。

攻占叙利亚后,旭烈兀西征军兵锋抵达今天地中海东岸的的巴勒斯坦地区,即将与埃及的马木留克王朝交战,此时旭烈兀得到使者带来的帝国最高统治者蒙哥在四川去世的消息,于是只派先锋怯的不花率不到一万军队驻守叙利亚,自己率大军开始东返。1260年9月3日,埃及马木留克王朝趁着旭烈兀攻率主力东返,攻占叙利亚,杀怯的不花,旭烈兀愤怒至极,本想率军继续西征,但此时他和钦察汗国的别儿哥汗因为争夺阿塞拜疆爆发了战争,只好结束西征。

旭烈兀东返途中得到忽必烈和阿里不哥争位的消息,于是留在西亚,自据一方,并宣布支持忽必烈,后来被忽必烈封为“伊儿汗”,西亚的伊儿汗国从此建立。

征伐南宋:1258年,蒙哥、其弟忽必烈和大将兀良合台分三路大举进攻南宋。1258年农历七月,蒙哥亲率主力进攻四川,所向披靡,攻克四川北部大部分地区,直到1259年初在合州(今重庆合川区)釣魚城下攻势受阻,战事胶着数月,蒙哥死前最终未能完成此次战役;而蒙哥死后,忽必烈得知忽里台大会选举阿里不哥即位,匆匆率军赶回漠北争夺汗位,对南宋的征伐计划暂时搁置。

蒙哥的去世原因,至今史学界尚无明确结论。主要有以下几说:

战争中受伤不治身亡:《合州志》記載,1259年8月11日(农历七月二十一日),蒙哥在合州钓鱼山一役,被南宋軍投石機的巨石打中,六天後傷重而亡。《馬可波羅游記》和明萬歷《合州志》则记载蒙哥在攻打合州时被釣魚城守城武器矢石擊中而重傷后去世。翦伯赞主编的《中国史纲要》采取了这种说法,书:“蒙古军因军中痢疾盛行,死伤极多,蒙哥汗又为宋军的飞矢射中身死”。《古今圖書集成》中的《釣魚城記》则记載:“炮風所震,因成疾。班師至愁軍山,病甚……次過金劍山溫湯峽(今重慶市北碚北溫泉)而歿”,謝士元在《遊釣魚山詩序》亦說蒙哥是“炮風致疾”而死。

病逝:《元史》则称天气多雨,蒙哥身体不适,于农历七月癸亥日死在钓鱼山 蒙古帝国伊儿汗国宰相拉施特的《史集》也推斷當時正值酷暑季节,军中痢疾流行,蒙哥亦染病身亡。畢沅在《續資治通鑑》稱蒙哥死於痢疾。

其他说法有:黃震的《古今紀要逸編》認為蒙哥因為屢攻合州釣魚城不克,致憂憤死;《海屯紀年》說是落水死。

据传蒙哥臨終前留下遺言,將來若攻下釣魚城,必屠殺全部軍民百姓;然而此事《元史》、《新元史》、《史集》均无记载(此三本史书记载蒙哥病逝,和钓鱼城的战斗无关)。後來釣魚城於1279年投降時,忽必烈赦免了所有軍民。

蒙哥的去世,对当时的蒙古帝国政局乃至世界格局都有極大的影響:蒙哥去世导致了旭烈兀统帅的第三次蒙古西征被迫中止;随后爆发了其弟忽必烈与阿里不哥争夺汗位之战,最终导致大蒙古国(蒙古帝国)的分裂。

法国国王路易九世派遣传教士卢布鲁克前往东方觐见蒙古大汗商讨传教和结盟对抗阿拉伯人事宜。卢布鲁克于1253年从地中海东岸阿克拉城(今以色列海法北)出发,于1253年5月7日离开君士坦丁堡,一路东行,渡过黑海,秋天到达伏尔加河畔,谒见拔都汗。拔都认为自己无权准许他在蒙古人中传教,便派他去东方觐见大汗蒙哥。卢布鲁克觐见拔都后,留下了对拔都的生动描述:“拔都坐在一金色的高椅上,或者说坐在像床一样大小的王位上,须上三级才能登上宝座,他的一个妻子坐在他旁边。其余的人坐他的右边和这位妻子的左边。”

1253年12月,卢布鲁克到达哈拉和林南部蒙哥冬季营地。1254年1月4日觐见蒙哥,并留下了对蒙哥的生动描述:“我们被领入帐殿,当挂在门前的毛毡卷起时,我们走进去,唱起赞美诗。整个帐幕的内壁全都以金布覆盖着。在帐幕中央,有一个小炉,里面用树枝、苦艾草的根和牛粪生着火。大汗坐在一张小床上,穿着一件皮袍,皮袍像海豹皮一样有光泽。他中等身材,约莫45岁,鼻子扁平。大汗吩咐给我们一些米酒,像白葡萄酒一样清澈甜润。然后,他又命拿来许多种猎鹰,把它们放在他的拳头上,观赏了好一会。此后他吩咐我们说话。他有一位聂思托里安教(景教)徒作为他的译员。”

1254年4月5日,随同蒙哥来到大蒙古国首都哈拉和林。8月18日带着蒙哥致路易九世的国书西归,信中写道:“这是长生天的命令。天上只有一个上帝,地上只有一个君主,即天子成吉思汗。”蒙哥以长生天以及它在地上的代表“大汗”的名义命令法兰西国王承认是他的属臣。

他于1255年回到地中海东岸。一年后,他用拉丁文写成的出使报告交给路易九世,即《东方行记》,又称《卢布鲁克游记》。

小亚美尼亚国王海屯一世于1244年归附大蒙古国,成为属国。1254年春,海屯一世遵从拔都汗之命亲自前往蒙古草原觐见大汗蒙哥。他与随臣一路东行,5月至拔都营帐(伏尔加河下游)谒见,然后继续东行,9月13日到达蒙哥汗廷(哈拉和林)朝见、献贡,得到蒙哥颁赐的诏书;“诏书上盖有蒙哥的御玺,不许人欺凌他及他的国家。还给他一纸敕令,允许各地教堂拥有自治权。”在哈拉和林停留50天后,他离开汗廷西还。

返回途中在中亚河中地区觐见蒙哥汗之弟弟旭烈兀,行程8个月,1255年7月返抵小亚美尼亚。回国后撰写《海屯行纪》。

父亲:拖雷,1227年—1229年帝位空缺时担任大蒙古国监国,1232年去世。《元史·睿宗本纪》载,蒙哥即位后追尊拖雷为皇帝,为拖雷上庙号睿宗、谥号英武皇帝,1266年忽必烈改谥其为景襄皇帝,1310年元武宗海山加谥为仁圣景襄皇帝。(元朝由忽必烈建立于1271年;然而在元朝建立之前,随着蒙古对金国、西夏等沿袭了中原礼制的王朝的征服,蒙古在中国地区的统治也受到了汉文化的影响,包括任用契丹人、汉人为官,尊重儒学等,为逝者上庙号、谥号等,或是出于对汉文化的吸收,而非意味着大蒙古国时期的“大汗”等同于元朝时期的“皇帝”。元史所载宗室,在忽必烈的同辈以及先辈中,除了宗庙里奉祭的重要祖先被予以“追赠尊谥”,均没有汉式的封号;而忽必烈建立元朝之后,宗室贵族才渐渐有了诸如“鲁国公主”之类的汉式封号,可见“大蒙古国”与“元朝”实为两个政权,不过后者宣称对前者继承耳。)

母亲:唆鲁禾帖尼,是蒙哥,忽必烈,旭烈兀,阿里不哥四人的生母,1251年蒙哥汗即位后尊其为皇太后,1252年去世。1266年元世祖忽必烈为其上谥号庄圣皇后,1310年元武宗海山加谥为显懿庄圣皇后。她的四个儿子皆曾称汗称帝,被后世史学家尊称为“四帝之母”。

元朝重臣郝经在中统元年(1260年)农历八月给元世祖忽必烈的上书《立政议》中对元宪宗蒙哥的評價是:“先皇帝初践宝位,皆以为致治之主,不世出也。既而下令鸠括符玺,督察邮传,遣使四出,究核徭赋,以来民瘼,污吏滥官,黜责殆遍,其愿治之心亦切也。惜其授任皆前日害民之尤者,旧弊未去,新弊复生,其为烦扰,又益剧甚,而致治之几又失也。”

明朝官修正史《元史》宋濂等的評價是:“帝刚明雄毅,沉断而寡言,不乐燕饮,不好侈靡,虽后妃不许之过制。初,太宗朝,群臣擅权,政出多门。至是,凡有诏旨,帝必亲起草,更易数四,然后行之。御群臣甚严,尝谕旨曰:‘尔辈若得朕奖谕之言,即志气骄逸,志气骄逸,而灾祸有不随至者乎?尔辈其戒之。’性喜畋猎,自谓遵祖宗之法,不蹈袭他国所为。然酷信巫觋卜筮之术,凡行事必谨叩之,殆无虚日,终不自厌也。”

清朝史学家邵远平《元史类编》的評價是:“册曰:天象知祥,众心戴主;遐辟西南,深入中土;未究厥勳,亦振乃武;友弟因心,终昌时绪。”

清朝史学家毕沅《续资治通鉴》的評價是:“宪宗沉断寡言,不乐宴饮,不好侈靡,虽后妃亦不许之过制。初,定宗朝,群臣擅权,政出多门,帝即位,凡有诏旨,必亲起草,更易数四,然后行之。御群臣甚严,尝曰:‘尔辈每得朕奖谕之言,即志气骄逸。志气骄逸,而灾祸有不随至者乎?尔辈其戒之!’性喜畋猎,自谓遵祖宗之法,不蹈袭他国所为。然酷信巫觋、卜筮之术,凡行事必谨叩之,殆无虚日。”

清朝史学家魏源《元史新编》的評價是:“帝早亲军旅,刚明沉断,威著中外。即位以后,不乐燕饮,不好侈靡,虽后妃不许之过制。初,太宗崩后,旷纪无君,黄裳御统,政出多门,阿柄几于旁落。至是,凡有诏旨,帝必亲起草,更易数四,然后行之。御臣下甚严,尝谓:‘臣下奖谕太过,即志气骄溢,过咎随之,是害之也。’承开国师武臣力之后,西平印度,南并大理,东取巴蜀,所向无敌。惟遵其国俗,喜田猎,信巫觋卜筮,是其小蔽。使太宗即世,早承大业,则伐宋之役,不俟末年而南北混一矣。天未既宋,暑雨老师,景命不延,故大勳重集于世祖皇帝。”

清朝史学家曾廉《元书》的評價是:“论曰:宪宗之立,有遗议焉。前史袭《元史》旧文,未为允也。史又称宪宗能辑士卒,皇子阿速歹猎骑伤稼,责之,复挞其近侍。卒拔民葱,即斩以徇。在蒙古治军可谓肃矣。夫古今称强汉、弱宋,然王坚以孤城罢卒,抗毳旃之劲族,卒乃师老解退。虽宪宗不晏驾,庸必克乎?盖自平金以来,中汉人之习,锦衣玉食,肌骨疏懈。故金以是亡,而元人兵势亦自是遂稍衰矣。历观史策,暾欲谷之言,有以哉!”

民国史学家屠寄《蒙兀儿史记》的評價是:“汗刚明雄毅,沉断而寡言,不乐燕饮,不好侈靡,虽后妃不许逾制。尝有西域商胡献水晶盆,珍珠伞等物,价值银三万余锭,汗曰:‘今百姓疲弊,所急者钱耳。朕独有此何为?’却之。赛典赤以为言,乃稍偿其值,且禁嗣后勿献。初,古余克汗朝群臣擅权,政出多门。至是,凡有诏旨,汗必亲起草,更易数四,然后行之。御群下甚严,尝谕旨曰:‘汝曹若得朕奖谕,即志气骄逸,志气骄逸,灾祸有不随至者乎?汝曹戒之。’性喜畋猎,自谓遵祖宗之法,不蹈袭他国所为。然酷信巫觋卜筮之术,凡行事必谨叩之,殆无虚日,终不自厌也。”

民国官修正史《新元史》柯劭忞的評價是:“帝沉断寡言,不喜侈靡。太宗朝群臣擅权,政出多门。至是,凡诏令皆帝手书,更易数四,然后行之。御群臣甚严,尝谕左右曰:“汝辈得朕奖谕,即志气骄逸,灾祸有不立至者乎?汝辈其戒之。”然酷信巫觋卜笨之术,凡行事必谨叩之无虚日,终不自厌也。史臣曰:“宪宗聪明果毅,内修政事,外辟土地,亲总六师,壁于坚城之下,虽天未厌宋,赍志而殂,抑亦不世之英主矣。然帝天资凉薄,猜嫌骨肉,失烈门诸王既宥之而复诛之。拉施特有言:蒙古之内乱,自此而萌,隳成吉思汗睦族田本这训。呜呼,知言哉!”


%%% Local Variables:
%%% mode: latex
%%% TeX-engine: xetex
%%% TeX-master: "../Main"
%%% End:

%% -*- coding: utf-8 -*-
%% Time-stamp: <Chen Wang: 2019-10-18 15:36:16>

\section{世祖\tiny(1260-1294)}

元世祖忽必烈,清代乾隆晚期乾隆帝命改譯为呼必赉。孛儿只斤氏,為父親拖雷的第四子,母親唆鲁禾帖尼的第二子,蒙古帝国大汗,元王朝的建立者。

1260年5月5日在自己的弟弟旭烈兀的支持和封地属臣的拥立下,自立为大蒙古国大汗,称大蒙古国皇帝。1271年12月18日,忽必烈改国号为“大元”,建立元朝,成为元朝首任皇帝。忽必烈于1260年5月5日至1276年2月4日自立为汗期间实际统治中国北方及蒙古高原地区属于蒙古大汗的直辖领地,于1271年12月18日至1294年2月18日作为元朝皇帝统治中国,前后在位34年,作为全中国皇帝在位18年。

1276年2月4日,元军攻入南宋行都临安,宋恭帝奉上传国玉玺和降表,元朝成为全国性政权,但南宋遗臣建立小朝廷继续抗元。1279年3月19日,南宋海上政权残余的最后一支抵抗力量被消灭,元朝统一全中国。

1276年2月4日,宋恭帝在降表中为忽必烈上尊号大元仁明神武皇帝。1284年1月24日,群臣为忽必烈上尊号宪天述道仁文义武大光孝皇帝。

去世后,获諡號聖德神功文武皇帝,廟號世祖,蒙古語尊號薛禪皇帝。

成吉思汗十年八月二十八日(1215年9月23日),忽必烈生于漠北草原。忽必烈是成吉思汗第四子拖雷與正妻唆鲁禾帖尼所生的嫡次子(蒙哥是嫡长子,旭烈兀是嫡三子,阿里不哥是嫡四子)。忽必烈长大后,“仁明英睿,事太后至孝,尤善抚下。”忽必烈年少有大志、重视汉地的治理,早在1244年,年轻的忽必烈便招揽了搜罗了各方的文人、儒生、旧臣等,形成了一个属于自己的幕僚团

1251年7月1日(农历辛亥年六月十一日),忽必烈長兄蒙哥经忽里台选举成为大蒙古国大汗(于1264年被忽必烈追尊为元宪宗),即位后不久即任命忽必烈負責總領漠南漢地事務。忽必烈设置金莲川幕府,并在这段时间内任用了大批漢族幕僚和儒士,如劉秉忠、許衡、姚樞、郝经、张文谦、窦默、趙璧等等,并提出了“行汉法”的主张。儒士元好问和张德辉还请求忽必烈接受“儒教大宗师”的称号,忽必烈悦而受之。忽必烈尊崇儒学,“圣度优宏,开白炳烺,好儒术,喜衣冠,崇礼让。”

1252年六月,忽必烈前往草原觐见蒙哥汗,奉命率军征云南地区的大理国,为继续进攻南宋作跳板。1253年八月,忽必烈率军从陕西出发,于1254年1月2日(农历十二月十二日)攻克大理城,國王段兴智投降,大理国灭,云南地区并入大蒙古国版图。1256年,段兴智前往漠北和林皇宫觐见蒙哥,被蒙哥任命為大理總管,子孙世襲。从1254年忽必烈奉蒙哥之命灭大理国,到1382年驻守云南的元朝梁王把匝剌瓦尔密兵败自杀、大理总管段世战败归降明军,蒙古族建立的政权统治云南地区长达128年。

1256年夏天,蒙哥以南宋扣押蒙古使者为理由,對南宋宣戰,并布置了三路大军,亲自率领西路军,以忽必烈为中路军统帅。忽必烈率军抵达河南汝南,继续向南宋进发,并派命杨惟中、郝经宣抚江淮。1259年9月3日(农历八月十五日),忽必烈统领中路军渡过淮河,攻入南宋境内,随后一路向南,在湖北开辟新的战场,进攻长江中游的鄂州。

1259年8月11日,蒙哥在四川合州钓鱼山病逝。1259年9月19日,在四川的忽必烈异母弟末哥派来的使者向忽必烈宣布蒙哥去世的消息,并请忽必烈北归参与忽里台大会,以便争取汗位继承权。忽必烈则认为“吾奉命南来,岂可无功遽还?”于是进攻南宋,并多次获胜,后来,忽必烈的正妻察必派使者密报,阿里不哥已经派阿蓝答兒在开平附近调兵,脱里赤在燕京附近征集民兵,催促忽必烈早日北还。1259年11月17日,儒臣郝经上《班师议》,陈述必须立即退兵的理由,坚定了忽必烈退兵北返的决心。

忽必烈声称要进攻南宋首都临安,留大将继续对鄂州的围攻,增加对南宋的军事压力,元宪宗九年闰十一月二日(1259年12月17日),南宋丞相贾似道派使者请和,约定南宋割地求和,并且送岁币,忽必烈于是在当日撤兵北返,元宪宗九年闰十一月二十日(1260年1月4日),忽必烈率军抵达燕京(今北京市),解散了脱里赤征集的民兵,“民心大悦”。忽必烈率军在燕京近郊驻扎,度过整个冬天,并积极和诸王联络,准备在1260年春天召开庫力台大會,举行登基大典。

庚申年三月二十四日(1260年5月5日),忽必烈在部分宗王和大臣擁立下于自己的封地开平(后称上都,今内蒙古多伦县北石别苏木)自立为“大蒙古国皇帝”(即蒙古帝国大汗的汉语称谓),庚申年四月四日(1260年5月15日),忽必烈发布称帝的即位诏书《皇帝登宝位诏》,在诏书中,他自称为“朕”,称他的哥哥元宪宗蒙哥(1251—1259年在位)为“先皇”。

中统元年五月十九日(1260年6月29日),忽必烈发布《中统建元诏》,正式建年号“中统”。

庚申年(1260年)农历四月,其弟阿里不哥在哈拉和林城西按坦河被部分宗王和大臣拥立為大蒙古国大汗。幼弟阿里不哥與忽必烈為此發動戰爭爭奪汗位,双方战争时断时续,一共持续了四年之久。忽必烈于庚申年三月二十四日(1260年5月5日)自立为汗,又称汉文的“皇帝”,以招揽汉族知识分子归心,一部分汉族知识分子果然对此表示赞许,赞美忽必烈“既以正立,一时豪杰云从景附,全制本国,奄有中夏,挟辅辽右、白霫、乐浪、玄菟、秽貊、朝鲜,面左燕云、常代,控引西夏、秦陇、吐蕃、云南,则玉烛金瓯,未为玷缺。藩墙不穴,根本强固,倍半于金源,五倍于契丹。”

1260年忽必烈称帝后,控制了漠南草原,以及原金朝和西夏故地,吐蕃,云南,西域东部等地区,对阿里不哥实施经济控制。阿里不哥控制的则是漠北草原和西域西北部地区,面对匮乏的物资最终无以为继。1264年忽必烈最终迫使阿里不哥投降,完全控制蒙古帝国的东部、原本属于大汗直辖领地的大部分地区。阿里不哥归降忽必烈后,忽必烈赦免了他和跟随的诸王,只是处死了他的众多谋臣。。阿里不哥失败后郁郁寡欢,于1266年去世。

1264年8月21日(忽必烈中统五年七月二十八日)阿里不哥投降后,忽必烈实际管辖的政治版圖包括(古今地名对照):中原地区(位于长城以南、秦岭淮河以北)、东北地区(包括整个黑龙江流域)、朝鲜半岛北部、漠南漠北蒙古草原全境(内蒙古和外蒙古地区),西伯利亚南部地区、西域大部分地区(今新疆東部和南部)、吐蕃地区(包括今青海、西藏、四川西部等地)、以及云南地区等地。

至元元年八月十六日(1264年9月7日),忽必烈发布《至元改元诏》,取《易经》“至哉坤元”之义,改“中统五年”为“至元元年”。

庚申年四月初一日(1260年5月12日),忽必烈立中书省,以中书省为最高行政机关,行使宰相职权,以王文統为平章政事,张文谦为中书左丞。

中统四年五月六日(1263年6月13日),忽必烈立枢密院,以枢密院为中央最高军事管理机关,以燕王真金守中书令,兼判枢密院事。

至元元年(1264年),忽必烈立总制院,以总制院统领全国宗教事务并管辖吐蕃地区,以国师八思巴领之。至元二十五年(1288年),尚书省右丞相桑哥认为总制院职责重大,故向忽必烈奏请根据唐朝时期在宣政殿接待吐蕃使者的缘故,改名为宣政院。忽必烈同意,并任命桑哥和脱因为宣政院使。

至元五年七月四日(1268年8月13日),忽必烈立御史台,以御史台为最高监察机关,以右丞相塔察兒为御史大夫,以張雄飛为侍御史。

至元八年十一月十五日(1271年12月18日),因劉秉忠之勸,忽必烈发布《建国号诏》,取《易经》“大哉乾元”之义,建立“大元”国号,其自身亦从大蒙古国皇帝(大汗)变为大元皇帝,元朝正式建立。

元军延续自1268年秋天以来的攻势继续围困襄阳,将襄阳和樊城分隔开来,至元十年正月九日(1273年1月29日),在回回炮的助攻下,元军将领阿里海牙攻克樊城,襄阳彻底成为孤城,元世祖降诏谕襄阳守将吕文焕,阿里海牙亲自到城下劝降吕文焕,保证吕文焕和城中军民的安全,吕文焕犹疑未决。于是阿里海牙和吕文焕折箭为誓担保,吕文焕感泣,至元十年二月二十四日(1273年3月14日),吕文焕和儿子出城投降,归顺元朝。元军经过接近五年時间包围,最终取得襄阳。但是以後的进展则相当顺利。

至元十一年六月十五日(1274年7月20日),忽必烈向行中书省及蒙古、汉军万户千户军士发布问罪于宋的诏书《兴师征南诏》。

至元十一年(1274年)农历七月,忽必烈发布《下江南檄》,派伯颜统率大军讨伐南宋,并告诫伯颜要学习曹彬不杀平江南。伯颜后来取临安,的确做到了忽必烈的要求。

至元十三年正月十八日(1276年2月4日),伯颜率领大军攻陷南宋首都临安(今杭州),宋恭帝派遣使者给元军统帅伯颜奉上传国玉玺和降表,在降表中宋恭帝为忽必烈上尊号大元仁明神武皇帝,元军俘虏5岁的宋恭帝和谢太皇太后,以及南宋宗室和大臣,灭南宋。

至元十三年二月十一日(1276年2月27日),忽必烈发布《归附安民诏》,诏谕江南一带新附府州司县官吏士民军卒人等,稳定江南社会秩序,安定江南士人和百姓之心。

逃离临安的部分大臣陆秀夫等人,先后扶持宋端宗,宋帝昺,建立海上流亡政权,在东南沿海一带继续和元军对抗。至元十六年二月六日(1279年3月19日),在厓山海战中,元军将领张弘范击败南宋海军,南宋丞相陆秀夫挟8岁的小皇帝“宋帝昺”跳海而死,不少後宮和大臣亦相繼跳海自殺。《宋史》記載七日後,十餘萬具屍體浮海。南宋残余的最后一支抵抗力量选择了惨烈的终结,至此,元朝统一海内,结束了中国自安史之乱以来520多年的分裂局面。

1281年3月20日,忽必烈愛妻察必皇后去世。1286年1月5日,皇太子真金去世,连续几年的时间里,爱妻和爱子的先后去世,使忽必烈悲痛不已。此外,忽必烈晚年飽受肥胖與痛風病痛之苦。過度飲酒也损害了他的健康。

至元三十一年正月二十二日(1294年2月18日),忽必烈於大都皇宮紫檀殿去世,享壽七十九岁,在位三十五年。忽必烈葬于起辇谷。

忽必烈去世后,在顾命大臣伯颜等人的拥戴下,其孙铁穆耳于1294年5月10日在上都继承皇位,是为元成宗。1303年,元成宗与西北诸王达成和议,西北的四大汗国重新承认元朝的宗主国地位。

因为1260年忽必烈和阿里不哥争位导致蒙古帝国表面上维持统一,实际上已经分裂,帝国西部為四大汗国实际控制,而帝国东部為忽必烈实际控制。趁着忽必烈和阿里不哥的内战,西北地区的钦察汗国、察合台汗国、窝阔台汗国纷纷自立,此时尚在西亚进行西征的旭烈兀也准备自帝一方,不论忽必烈还是阿里不哥都只得到一部分宗王支持,没有召开成吉思汗四子嫡系后裔参加的「忽里勒臺」(決定繼承人的大會),忽必烈不被广泛承认,于是,忽必烈将大汗在西亚的直辖地(阿姆河以西直到埃及边境)封给旭烈兀换取旭烈兀的支持,旭烈兀建立伊儿汗国(其实旭烈兀留在西亚,忽必烈也没办法,但忽必烈给了旭烈兀统治的合法性)。忽必烈将大汗在中亚的直辖地(阿尔泰山以西直到阿姆河的农耕和城郭地区)封给察合台汗阿鲁忽换取阿鲁忽的支持。而钦察汗国早在元定宗贵由和元宪宗蒙哥统治时期已经取得实际上基本独立的地位。

1264年8月21日,阿里不哥向忽必烈投降。胜利之后忽必烈立即向各系兀鲁思派去急使,召他们东赴蒙古草原,重新召开忽里台大会。忽必烈重开忽里台的目的,是因为考虑到中统元年三月二十四日仓促即位于开平,没有四大兀鲁思的代表参加,不符合成吉思汗的扎撒(蒙古语“军律”、“法规”之意),故而准备依照传统惯例,在祖先发祥地斡难---怯绿涟之域召开由各系宗王参加的忽里台,重新确立自己的大汗地位,并借这次大会扼制帝国分裂的趋势。

钦察汗别儿哥、察合台汗阿鲁忽和伊兒汗旭烈兀(忽必烈之弟)一致同意东来赴会。元世祖也向窝阔台汗海都派去了急使,但海都拒绝前来。当然,这次原定于至元四年(1267年)召开的忽里台没能如约举行,主要是因为各汗国之间随后爆发战争,以及在此后一年多时间里原本同意参加忽里台的阿鲁忽、旭烈兀、别儿哥三位汗王先后去世(旭烈兀1265年去世,别儿哥、阿鲁忽1266年去世,他们不可能参加1267年的忽里台)。但窝阔台汗海都的抗命已经明白无误地表明了分裂意图,忽必烈声称的大汗之位未获公认,成吉思汗及窝阔台汗创立的蒙古帝国处于分崩离析的边缘。

1269年,钦察汗国、窝阔台汗国与察合台汗国召开塔剌思忽里台,达成了协议,共同反对拖雷家族控制的大汗直辖地(即忽必烈的实际控制区)和伊儿汗国(旭烈兀家族控制区,忽必烈的唯一支持者),并协议划分了各自在阿姆河以北地区的势力范围。塔剌思大会标志着大蒙古国的实质分裂和解体,从此察合台汗国和窝阔台汗国脱离了大蒙古国,与掌控蒙古帝国东部的拖雷系家族分头發展。察合台汗国和窝阔台汗国对此后数十年中亚和西亚历史的发展产生了深远的影响。

窝阔台汗海都一直和忽必烈敵對,企图确立自己为大汗之位的继承人。终元世祖忽必烈一朝,元朝和窝阔台汗国、察合台汗国征战不休,直到元成宗时期才彻底解决西北问题。

大蒙古国时期的历任大汗,虽然经由对辽、金故地的征服,与汉文明一直有接触,也往往对汉文化表示接纳,蒙古贵族却大多数反对建立一个汉式的政府;忽必烈对其在汉地的领地则相当重视,并且花费了时间去了解汉人的治国思想和儒家文化,最终以自己的领地开平为中心,建立起了一个汉式的行政中心,其后忽必烈在试图争取整个蒙古帝国统治权的同时,一直没有放弃尝试让汉人接受他作为一个中国皇帝,并为此做了一系列汉化努力。

忽必烈赢取汉人接受其统治的第一个措施便是效仿汉人的典章制度,将“大蒙古国”的历史和皇族“汉化”,其中一个显著做法就是建立太庙,按照中原王朝的传统为大蒙古国的历任大汗确立庙号,追尊谥号。

中统四年(1263年)农历三月,忽必烈下诏在燕京(后来改称大都)建立太庙。至元元年(1264年)十月,初定太庙七室神主。至元二年农历十月十四日(1265年11月23日),忽必烈祭祀太庙,为皇祖成吉思汗上庙号太祖。至元三年(1266年)九月,太庙始作八室神主。十月,太庙建成。丞相安童、伯颜建议制定尊谥庙号,忽必烈命平章政事趙璧等集议,制尊谥庙号,定为八室,为大蒙古国的前四位帝王成吉思汗、窝阔台(元太宗)、贵由(元定宗)、蒙哥(元宪宗)上庙号和谥号,为他们的皇后上谥号;并追尊也速该、术赤、察合台三人为皇帝,也为他们上庙号和谥号,并为拖雷(已经于1251年被追尊为皇帝)改谥号为景襄皇帝,并将他们四人的正妻追谥为皇后,也上谥号。太庙八室,这八位和他们的妻子的神主各居一室。这些做法有效地吸引了汉族谋士和儒生参与忽必烈的新政权,《剑桥中国史——辽宋夏金元》认为,这一系列做法极大地帮助忽必烈巩固了蒙古族政权在汉地的统治。

蒙古帝国的首都,大汗的汗庭处于蒙古高原上的和林哈拉。忽必烈掌控蒙古帝国东部以后,逐步建立了两都制,并最终定都大都,将政权的统治中心移到了汉地文化更加发达的地区,有利于取得汉族谋士和蒙古贵族之间的平衡。

1215年5月31日,成吉思汗率大军攻克金中都(今北京市)。1217年,太师、国王木华黎改中都为燕京。燕京即为后来两都制中的中都。

1256年,忽必烈命刘秉忠在开平(今中国内蒙古自治区锡林郭勒盟正蓝旗多伦县西北闪电河畔)建立王府,忽必烈在此建立了著名的“金莲川幕府”。中统四年五月九日(1263年6月16日),忽必烈下诏升开平府为上都。

中统五年八月十四日(1264年9月5日),忽必烈发布《建国都诏》,改燕京(今北京市)为中都,定为陪都,两都制正式形成。

至元四年正月三十日(1267年2月25日),忽必烈由上都迁都到中都,定中都为首都,忽必烈迁都中都后,居住于中都城外的金代离宫——大宁宫内,并随即在中都的东北部,以大宁宫所在的琼华岛为中心开始了新宫殿和都城的规划兴建工作,上都成为陪都。

至元九年二月三日(1272年3月4日),忽必烈将中都改名为大都(突厥语称汗八里,帝都之意),元大都包括南城(金中都旧城)和北城(元大都新城),两者的城墙“仅隔一水”。

至元十一年正月初一(1274年2月9日),宫阙告成,元世祖忽必烈首次在大都皇宫正殿大明殿举行朝会,接受皇太子、诸王、百官以及高丽国王王禃所派使节的朝贺。

至元二十八年五月二十一日(1291年6月18日),忽必烈下诏颁布元朝第一部全国性的法律典籍《至元新格》。

忽必烈统一中国后,元朝疆域空前辽阔,远超汉唐盛世。“自封建变为郡县,有天下者,汉、隋、唐、宋为盛,然幅员之广,咸不逮元。汉梗于北狄,隋不能服东夷,唐患在西戎,宋患常在西北。若元,则起朔漠,并西域,平西夏,灭女真,臣高丽,定南诏,遂下江南,而天下为一,故其地北逾阴山,西极流沙,东尽辽左,南越海表。盖汉东西九千三百二里,南北一万三千三百六十八里,唐东西九千五百一十一里,南北一万六千九百一十八里,元东南所至不下汉、唐,而西北则过之,有难以里数限者矣。”

元朝不仅在疆域面积上远迈汉唐,而且在东北、西北、西南等边疆地区的控制程度上也远超汉唐盛世。“盖岭北、辽阳与甘肃、四川、云南、湖广之边,唐所谓羁縻之州,往往在是,今皆赋役之,比于内地;而高丽守东藩,执臣礼惟谨,亦古所未见。”

元世祖至元十七年(1280年)元朝的疆域范围:东北至外兴安岭、鄂霍次克海、日本海,包括库页岛,并到达朝鲜半岛中部的铁岭和慈悲岭一带,北到西伯利亚南部(谭其骧版地图认为北到北冰洋),到达贝加尔湖以北的鄂毕河和叶尼塞河上游地区,西北至今新疆大部分地区,西南包括今西藏、云南、以及缅甸北部,南到南海,东南到达东海中的澎湖列岛。

在灭南宋前后,元政府曾要求周边一些国家或地区(包括日本、安南、占城、缅甸、爪哇、琉求國)臣服,接受与元朝的朝贡关系,但遭到拒绝,故派遣军队进攻攻打这些国家或地区,例如緬甸蒲甘王朝拒絕朝貢,元軍入侵蒲甘並攻破蒲甘城,令缅甸臣服於元朝。其中以入侵日本国最为著名,也最惨烈。

忽必烈在位时期和中亚的察合台汗国,窝阔台汗国多次交战,双方互有胜负,1289年,窝阔台汗国夺取元朝控制下的新疆南部塔里木盆地大部分地区,元朝只控制塔里木盆地东部的且末、焉耆等地区。终忽必烈一朝,元朝始终控制新疆北部的别失八里(今乌鲁木齐东北)一带和新疆东部的吐鲁番、哈密等地。

对日战争 至元十一年(1274年)元军發動第一次侵日戰爭,日本史書稱之為“文永之役”,以三萬二千餘人,東征日本。至元十八年(1281年)七月,忽必烈又發動第二次侵日戰爭,史稱“弘安之役”,由范文虎、李庭率江南軍十餘萬人,到達次能、志賀二島,卻碰到颱風,溺死近半。通常认为台风(日本人称之为“神风”)是这两次征日造成失败的最大原因。亦有观点认为,忽必烈担心归附军的忠诚,故而借东征日本而一举消除隐患。

元朝重臣郝经在中统元年(1260年)农历四月奉元世祖忽必烈之命出使南宋南北议和,在九月到达南宋后被扣留软禁于真州15年,直到至元十二年(1275年)农历二月才被南宋送归元朝境内,他在被软禁期间十余次给南宋君臣上书,希望元宋缔结和约,均无任何回复。郝经在中统元年(1260年)农历十一月给南宋两淮制置使李庭芝的书信《再与宋国两淮制置使书》中对元世祖忽必烈的評價是:“今主上应期开运,资赋英明,喜衣冠,崇礼乐,乐贤下士,甚得中土之心,久为诸王推戴。稽诸气数,观其德度,汉高帝、唐太宗、魏孝文之流也。” (“汉高帝”指的是汉太祖刘邦,“太祖”为庙号,“高帝”为谥号,《史记》中常谓“高祖”,因此人多以为其庙号为高祖,其实乃庙号谥号混称。“唐太宗”指的是李世民。“魏孝文”指的是北魏孝文帝拓跋宏。)

元朝重臣郝经在中统二年(1261年)给南宋丞相贾似道的第三封书信《复与宋国丞相论本朝兵乱书》中对元世祖忽必烈的評價是:“夫主上之立,固其所也。太母有与贤之意,先帝无立子之诏。主上虽在潜邸,久符人望,而又以亲则尊,以德则厚,以功则大,以理则顺,爱养中国,宽仁爱人,乐贤下士,甚得夷夏之心,有汉、唐英主之风。加以地广众盛,将猛兵强,神断威灵,风蜚雷厉,其为天下主无疑也。”

明朝官修正史《元史》宋濂等的評價是:“世祖度量弘广,知人善任使,信用儒术,用能以夏变夷,立经陈纪,所以为一代之制者,规模宏远矣。”

明朝官修正史《元史》宋濂等的評價是:“世称元之治以至元、大德为首。……。故终世祖之世,家给人足。”

明朝官修皇帝实录《明太祖实录》记载,明太祖朱元璋在洪武七年八月初一日(1374年9月7日),亲自前往南京历代帝王庙祭祀三皇、五帝、夏禹王、商汤王、周武王、汉太祖、汉光武帝、隋文帝、唐太宗、宋太祖、元世祖一共十七位帝王,其中对元世祖忽必烈的祝文是:“惟神昔自朔土,来主中国,治安之盛,生餋之繁,功被人民者矣。夫何传及后世不遵前训,怠政致乱,天下云扰,莫能拯救。元璋本元之农民,遭时多艰,悯烝黎于涂炭,建义聚兵,图以保全生灵,初无黄屋左纛之意,岂期天佑人助,来归者众,事不能已,取天下于群雄之手,六师北征,遂定于一。乃不揆菲德,继承正统,此天命人心所致,非智力所能。且自古立君,在乎安民,所以唐虞择人禅授,汤武用兵征伐,因时制宜,其理昭然。神灵在天不昧,想自知之。今念历代帝王开基创业、有功德于民者,乃于京师肇新庙宇,列序圣像,每岁祀以春、秋仲月,永为常典,礼奠之初,谨奉牲醴致祭,伏惟神鉴。尚享!”

明朝官修皇帝实录《明太祖实录》记载,洪武二十二年(1389年)十二月,明太祖朱元璋给北元兀纳失里大王的信中,对元太祖和元世祖的评价如下:“昔中国大宋皇帝主天下三百一十余年,后其子孙不能敬天爱民,故天生元朝太祖皇帝,起于漠北,凡达达、回回、诸番君长尽平定之,太祖之孙以仁德著称,为世祖皇帝,混一天下,九夷八蛮、海外番国归于一统,百年之间,其恩德孰不思慕,号令孰不畏惧,是时四方无虞,民康物阜。”

邵远平《元史类编》的評價是:“册曰:遂辟雄图,混一中外;德威所指,无远弗届;建号立制,垂模一代;崇儒察奸,旋用旋败;英明克断,用无祗悔。”

叶子奇《草木子》卷三上: “元朝自世祖(忽必烈)混一之后,天下治平者六、七十年,轻刑薄赋,兵革罕用;生者有养,死者有葬;行旅万里,宿泊如家,诚所谓盛也亦!”

毕沅《续资治通鉴》的評價是:“帝度量恢廓,知人善任使,故能混一区宇,扩前古所未有。惟以亟于财用,中间为阿哈玛特、卢世荣、僧格所蔽,卒能知其罪而正之。立纲陈纪,殷然欲被以文德,规模亦已弘远矣。”(“阿哈玛特”指的是阿合马,“僧格”指的是桑哥,不同的人对他们的名字进行汉语音译时,有一定差别。)

魏源《元史新编》的評價是:“论曰:元之初入中国,震荡飘突,惟以杀伐攻虏为事,不知法度纪纲为何物,其去突厥、回纥者无几。及世祖兴,始延揽姚枢、窦默、刘秉忠、许衡之徒,以汉法治中夏,变夷为华,立纲陈纪,遂乃并吞东南,中外一统。加以享国长久,垂统创业,轶遼、金而媲漢、唐,赫矣哉!且其天性宽宏,包帡无外。阿里不哥及海都、笃哇诸王,皆亲犯乘舆。对垒血战,力屈势穷,一朝归命,则皆以太祖子孙,大朝会于上都,恩礼宴赉如初。当南北锋焰血战之余,或离间以侍郎张天悦通宋而不信。敕南儒被掠卖为奴者,官赎为民。所获宋商、宋谍私入境者,皆纵遣之而不诛。置榷场于樊城,通宋互市,弛沿边军器之禁。其长驾远驭如是。宋幼主母子至通州,命大宴十日,小宴十日,然后赴上都。除弘吉剌皇后厚待之事别详《皇后传》外,其母子在江南庄田,听为世业。其后文宗时市故全太后田为大承天寺永业,市故瀛国公田为大翔龙寺永业,直至顺帝末,始夺和尚赵完普之田归官,直与元相终始。宋之宗室如福王与芮等,随宋主来归,授平原郡公,其家赀在江南者,取至京赐之。此外宗室多类此。即奸民冒称赵氏作乱者,从不以累及宋后,其优礼亡国也如是。思创业艰难,移漠北和林青草丛植殿隅,俾后世无忘草地。又留所御裘带于大安阁以示子孙。武宗至大中尝诣阁中发故箧阅之,则皆大练之服。西域贾胡屡献牙忽大珠,价值数万而不受。宫闱肃穆,无豔宠奇闻。至元八年,平滦路昌黎县民生男,夜中有光,或奏请除之,帝曰:‘何幸天生一好人,奈何反生妒忌!’命有司加恩养。伯颜伐宋,谆谆命以曹彬取江南不戮一人为法。其俭慈也又如是,非命世天纵而何?惟功利之习不能自胜于中,故日本、爪哇之师远覆于海岛,王、阿、桑、卢掊克之臣相仍于覆辙,盖质有余而学不足欤!”(“王、阿、桑、卢”指的分别是王文统、阿合马、桑哥、卢世荣。四人均为元世祖朝不同时期的理财大臣。)

曾廉《元书》的評價是:“论曰:世祖崇儒重道,而特进言利之臣,三进三乱而讫不悟,岂非其明有所蔽耶?然其不欲剥民亦审矣。殆以为自我作则,将上下均足,堪为后世经制也。呜呼!以世祖之仁,乘开国之运,而言利之弊,若此,然则利其有可言者耶?至其任中书枢密而重台纲,法纪立矣。国治民安是在知人哉!”

中華民国史学家屠寄《蒙兀儿史记》的評價是:“汗目有威稜,而度量弘广,知人善任,群下畏而怀之,虽生长漠北,中年分藩用兵,多在汉地,知非汉法不足治汉民。故即位后,引用儒臣,参决大政,诸所设施,一变祖父诸兄武断之风,渐开文明之治。惟志勤远略,平宋之后,不知息民,东兴日本之役,南起占城、交趾、缅甸、爪哇之师,北御海都、昔里吉、乃颜之乱。而又盛作宫室,造寺观,干戈土木,岁月不休。国用既匮,乃亟于理财,中间颇为阿合马、卢世荣、桑哥之徒所蔽,虽知其罪而正之,闾阎受患已深矣。”

中華民国官修正史《新元史》柯劭忞的評價是:“唐太宗承隋季之乱,魏徵劝以行王道、敦教化。封德彝驳之曰:‘书生不知时务,听其虚论,必误国家。’太宗黜德彝而用徵,卒致贞观之治。蒙古之兴,无异于匈奴、突厥。至世祖独崇儒向学,召姚枢、许衡、窦默等敷陈仁义道德之说,岂非所谓书生之虚论者哉?然践阼之后,混壹南北,纪纲法度灿然明备,致治之隆,庶几贞观。由此言之,时儿今古,治无夷夏,未有舍先王之道,而能保世长民者也。至于日本之役,弃师十万犹图再举;阿合马已败,复用桑哥;以世祖之仁明,而吝于改过。如此,不能不为之叹息焉。”

\subsection{中统}

\begin{longtable}{|>{\centering\scriptsize}m{2em}|>{\centering\scriptsize}m{1.3em}|>{\centering}m{8.8em}|}
  % \caption{秦王政}\
  \toprule
  \SimHei \normalsize 年数 & \SimHei \scriptsize 公元 & \SimHei 大事件 \tabularnewline
  % \midrule
  \endfirsthead
  \toprule
  \SimHei \normalsize 年数 & \SimHei \scriptsize 公元 & \SimHei 大事件 \tabularnewline
  \midrule
  \endhead
  \midrule
  元年 & 1260 & \tabularnewline\hline
  二年 & 1261 & \tabularnewline\hline
  三年 & 1262 & \tabularnewline\hline
  四年 & 1263 & \tabularnewline\hline
  五年 & 1264 & \tabularnewline
  \bottomrule
\end{longtable}

\subsection{至元}

\begin{longtable}{|>{\centering\scriptsize}m{2em}|>{\centering\scriptsize}m{1.3em}|>{\centering}m{8.8em}|}
  % \caption{秦王政}\
  \toprule
  \SimHei \normalsize 年数 & \SimHei \scriptsize 公元 & \SimHei 大事件 \tabularnewline
  % \midrule
  \endfirsthead
  \toprule
  \SimHei \normalsize 年数 & \SimHei \scriptsize 公元 & \SimHei 大事件 \tabularnewline
  \midrule
  \endhead
  \midrule
  元年 & 1264 & \tabularnewline\hline
  二年 & 1265 & \tabularnewline\hline
  三年 & 1266 & \tabularnewline\hline
  四年 & 1267 & \tabularnewline\hline
  五年 & 1268 & \tabularnewline\hline
  六年 & 1269 & \tabularnewline\hline
  七年 & 1270 & \tabularnewline\hline
  八年 & 1271 & \tabularnewline\hline
  九年 & 1272 & \tabularnewline\hline
  十年 & 1273 & \tabularnewline\hline
  十一年 & 1274 & \tabularnewline\hline
  十二年 & 1275 & \tabularnewline\hline
  十三年 & 1276 & \tabularnewline\hline
  十四年 & 1277 & \tabularnewline\hline
  十五年 & 1278 & \tabularnewline\hline
  十六年 & 1279 & \tabularnewline\hline
  十七年 & 1280 & \tabularnewline\hline
  十八年 & 1281 & \tabularnewline\hline
  十九年 & 1282 & \tabularnewline\hline
  二十年 & 1283 & \tabularnewline\hline
  二一年 & 1284 & \tabularnewline\hline
  二二年 & 1285 & \tabularnewline\hline
  二三年 & 1286 & \tabularnewline\hline
  二四年 & 1287 & \tabularnewline\hline
  二五年 & 1288 & \tabularnewline\hline
  二六年 & 1289 & \tabularnewline\hline
  二七年 & 1290 & \tabularnewline\hline
  二八年 & 1291 & \tabularnewline\hline
  二九年 & 1292 & \tabularnewline\hline
  三十年 & 1293 & \tabularnewline\hline
  三一年 & 1294 & \tabularnewline
  \bottomrule
\end{longtable}


%%% Local Variables:
%%% mode: latex
%%% TeX-engine: xetex
%%% TeX-master: "../Main"
%%% End:

%% -*- coding: utf-8 -*-
%% Time-stamp: <Chen Wang: 2019-12-26 14:53:44>

\section{成宗\tiny(1294-1307)}

\subsection{生平}

元成宗铁穆耳,是元朝第二位皇帝,蒙古帝国第六位大汗,1294年5月10日—1307年2月10日在位,在位14年。元世祖孙、太子真金第三子。清乾隆帝命改譯遼、金、元三史中的音譯專名,改譯特穆爾,今日學界已無人使用。

他去世后,谥号钦明广孝皇帝,庙号成宗,蒙古語号完澤篤可汗。

至元二十二年农历十二月十日(1286年1月5日),皇太子真金去世,元世祖欲立真金次子答剌麻八剌為皇太子,但1292年答剌麻八剌因病去世。至元三十年(1293年)真金三子铁穆耳受皇太子宝,总兵镇守漠北和林。至元三十一年农历正月二十二日(1294年2月18日),元世祖忽必烈去世,被封為晉王的真金長子甘麻剌決定要繼續鎮撫北方,铁穆耳得以在其母阔阔真與大臣伯顏等人的支持下,於至元三十一年农历四月十四日(1294年5月10日)在上都大安阁即位,是为元成宗。

铁穆耳即位後停止对外战争,罷征日本、安南,专力整顿国内军政,減免江南部分賦稅。並推行限制诸王势力、新编律令等措施,使社会矛盾暂时缓和。

在位期间基本维持守成局面,但滥增赏赐,入不敷出,国库资财匮乏,「向之所儲,散之殆盡」,中统钞迅速贬值。曾发兵征讨八百媳妇(在今泰国北部),引起云南、贵州地区动乱。晚年患病,委任皇后卜鲁罕和色目人大臣,朝政日渐衰败。

大德九年六月初五(1305年6月27日),元成宗冊立皇子德寿為皇太子,元成宗有数子,只有德寿皇太子为伯牙吾·卜鲁罕皇后所生。 同年十二月十八日(1306年1月3日),德寿因病去世。德寿去世後,成宗在生前未再立皇太子。

大德十一年农历正月初八日(1307年2月10日),成宗在大都玉德殿病逝,享年42岁 , 在位14年。

晚年患病,委任皇后卜鲁罕和色目人大臣,朝政日渐衰败。铁穆耳后继无人,埋下了元朝中期皇位争夺战的隐患。庙号成宗,谥号钦明广孝皇帝。蒙古汗号完泽笃可汗。

大德十一年九月十一日(1307年10月7日),元武宗为铁穆耳上谥号钦明广孝皇帝,庙号成宗,蒙古语称号完澤篤皇帝。

大德五年(1301年)秋,元军与窝阔台汗国的海都和察合台汗国的笃哇会战于金山附近的铁坚古山。元军先败海都。笃哇后至,两军再战。双方互有胜负,但都受到重创。海都、笃哇在会战中负伤,海都于1302年去世。

钦察汗国的东部藩属术赤长子斡儿答家族白帐汗封地原先与大汗的直辖地相连。窝阔台汗国的海都兴起后,隔断了元朝与术赤家族领地的直接联系。与海都接壤的白帐汗系宗王古亦鲁克为争夺汗位,投靠海都、笃哇。古亦鲁克的对手伯颜汗曾遣使元朝,要求双方联合作战。元朝的军队攻击海都,从谦州深入钦察汗国控制下的亦必儿·失必儿之地(今俄罗斯鄂毕河中游地区)。

大德六年(1302年),钦察汗国脱脱汗和白帐汗伯颜汗出兵2万,与元成宗的军队联合进攻笃哇和察八儿。此后钦察汗国承认元朝的宗主地位,长期与元朝维持友好关系。

1301年的铁坚古山之战对于元与西北宗藩的关系有决定性的影响,1302年海都去世,到了大德七年(1303年),笃哇扶立察八儿为窝阔台兀鲁思汗。笃哇暗中向元朝驻守在哈剌和林边境的安西王阿难答派出使臣,向元成宗表示臣服,请求朝廷罢兵。成宗同意约和。获得元廷支持后,笃哇与察八儿等聚会,到会诸王一致认识到,与朝廷进行长达数十年的战争是“自伤祖宗之业”。

大德七年(1303年)秋,笃哇以及海都之子察八儿约和使臣到达元廷。元廷与西北诸王达成和议,西北诸王承认元朝的宗主地位,设驿路,开关塞。自从1260年忽必烈与阿里不哥争位以来,元朝西北边境的战火终于基本平息,元朝的宗主地位得到四大汗国的正式承认。

接着,他们又联合遣使到伊儿汗国、钦察汗国王庭,大德八年(1304年)秋,伊儿汗完者都在木干草原会见钦察汗脱脱的使臣,西北四大汗国彼此之间的约和也至此完成,整个蒙古帝国境内再次迎来了和平。

明朝官修正史《元史》宋濂等的評價是:“成宗承天下混壹之后,垂拱而治,可谓善于守成者矣。惟其末年,连岁寝疾,凡国家政事,内则决于宫壸,外则委于宰臣;然其不致于废坠者,则以去世祖为未远,成宪具在故也。”

明朝官修正史《元史》宋濂等的評價是:“世称元之治以至元、大德为首。……。故终世祖之世,家给人足。……。大德之治,几于至元。”

清朝史家邵远平《元史类编》的評價是:“册曰:豢业以治,垂拱用成;中年奋武,启衅南征;末婴寝疾,壼柄廼萌;赖斯贤辅,镇侧弭倾。”

清朝史家毕沅《续资治通鉴》的評價是:“帝承世祖混一之后,善于守成;惟末年连岁寝疾,凡国家政事,内则决于宫壼,外则委于宰臣,幸去世祖未远,守其成宪,不至废坠。”

清朝史家曾廉《元书》的評價是:“论曰:成宗号为能守法度,而为病虐,前星弗耀,牝鸡司晨,而内难作矣。然非成宗之过也,成宗早任合剌合孙,资为羽翼,自古未有贤人在位而乱其国者也。股肱之寄,要在忠良,唐宗之言,信夫!”

民国史家屠寄《蒙兀儿史记》的評價是:“始汗为太孙时,好饮无节。忽必烈汗常戒之,不悛。以此受杖者三次,忽必烈汗至命医官监其饮食。有近侍司太孙节沐者,私置酒于盥器,代水以进,忽必烈汗闻之,大怒,谪戍其人远方,杀之于道。汗既登极,深以前事为非,力自节饮。其勇于改过如此。汗仁惠聪睿,承天下混一之后,信用老成,垂拱而治。一革至元中叶以来聚敛之政,冗设之官。约束诸王、妃、主、驸马扰民,禁滥请赏赐。性又谦冲,不好虚誉。群臣、皇后一再请上徽号,卒不允。可谓守成之令主矣。虽晚婴末疾,政出中宫,而举错无大过失。固由委任贤相之效,亦未始非内助之得人也。”

民国私修正史《新元史》柯劭忞的評價是:“成宗席前人之业,因其成法而损益之,析薪克荷,帝无使焉。晚年寝疾,不早决计计传位武宗,使易世之后,亲贵相夷,祸延母后。悲夫!以天子之尊,而不能保其妃匹,岂非后世之殷鉴哉。”

\subsection{元贞}

\begin{longtable}{|>{\centering\scriptsize}m{2em}|>{\centering\scriptsize}m{1.3em}|>{\centering}m{8.8em}|}
  % \caption{秦王政}\
  \toprule
  \SimHei \normalsize 年数 & \SimHei \scriptsize 公元 & \SimHei 大事件 \tabularnewline
  % \midrule
  \endfirsthead
  \toprule
  \SimHei \normalsize 年数 & \SimHei \scriptsize 公元 & \SimHei 大事件 \tabularnewline
  \midrule
  \endhead
  \midrule
  元年 & 1295 & \tabularnewline\hline
  二年 & 1296 & \tabularnewline\hline
  三年 & 1297 & \tabularnewline
  \bottomrule
\end{longtable}

\subsection{大德}

\begin{longtable}{|>{\centering\scriptsize}m{2em}|>{\centering\scriptsize}m{1.3em}|>{\centering}m{8.8em}|}
  % \caption{秦王政}\
  \toprule
  \SimHei \normalsize 年数 & \SimHei \scriptsize 公元 & \SimHei 大事件 \tabularnewline
  % \midrule
  \endfirsthead
  \toprule
  \SimHei \normalsize 年数 & \SimHei \scriptsize 公元 & \SimHei 大事件 \tabularnewline
  \midrule
  \endhead
  \midrule
  元年 & 1297 & \tabularnewline\hline
  二年 & 1298 & \tabularnewline\hline
  三年 & 1299 & \tabularnewline\hline
  四年 & 1300 & \tabularnewline\hline
  五年 & 1301 & \tabularnewline\hline
  六年 & 1302 & \tabularnewline\hline
  七年 & 1303 & \tabularnewline\hline
  八年 & 1304 & \tabularnewline\hline
  九年 & 1305 & \tabularnewline\hline
  十年 & 1306 & \tabularnewline\hline
  十一年 & 1307 & \tabularnewline
  \bottomrule
\end{longtable}


%%% Local Variables:
%%% mode: latex
%%% TeX-engine: xetex
%%% TeX-master: "../Main"
%%% End:

%% -*- coding: utf-8 -*-
%% Time-stamp: <Chen Wang: 2019-10-18 15:41:12>

\section{武宗\tiny(1307-1311)}

元武宗海山,是元朝第三位皇帝,蒙古帝国第七位大汗,在位4年,自1307年6月21日至1311年1月27日。乃元世祖之曾孫、太子真金之孫、答剌麻八剌之子、元成宗之侄。

1309年2月17日,群臣为海山上汉文尊号统天继圣钦文英武大章孝皇帝。

他去世后,謚號仁惠宣孝皇帝,廟號武宗,蒙古语称曲律皇帝。

武宗為真金次子答剌麻八剌之次子,嫡長子,1299年,海山接受元成宗的命令统兵漠北,负责同西北窝阔台汗国的君主海都和察合台汗国君主笃哇作战,多立戰功,为元朝结束和西北宗王的战争,以及1303年四大汗国全部承认元朝宗主地位做出了重要贡献。因为战功被封为懷寧王。

大德十一年正月初八(1307年2月10日),元成宗鐵穆耳病逝,儲位虛懸。成宗的伯牙吾·卜鲁罕皇后下命垂簾聽政,命安西王阿難答輔政。海山回大都奔喪,其弟愛育黎拔力八達與右丞相哈剌哈孫合謀发动政变,囚禁伯牙吾·卜鲁罕皇后和安西王阿難答,宣布擁立在外拥有重兵的海山為帝,是為元武宗,海山即位后追封其父答剌麻八剌為元順宗。

大德十一年五月二十一日(1307年6月21日),武宗在上都大安阁即位,之後处死伯牙吾·卜鲁罕皇后和阿難答,并更換了成宗时期的大臣,封其弟愛育黎拔力八達為皇太弟。在位只得四年,大興土木,建筑中都城,派军士千餘人及大量民工修建五台山華佛寺,又令喇嘛翻譯佛經,并曾想规定凡毆打西僧者截其手,罵西僧者斷其舌(但在其弟即后来的元仁宗愛育黎拔力八達劝告下取消)。

大德十一年七月十九日(1307年8月17日),元武宗下诏加封“至圣文宣王”孔子为“大成至圣文宣王”。

至大元年(1308年)五月,白蓮教被禁止。

至大元年(1308年),元武宗派遣月鲁出使钦察汗国,册封钦察汗脱脱为宁肃王。

至大二年(1309年),元朝和察合台汗国联手灭亡窝阔台汗国,元朝取得窝阔台汗国北部,察合台汗国取得窝阔台汗国南部。

至大二年(1309年)九月,为摆脱财政危机,印發至大銀鈔,导致至元钞大为贬值,從二釐到二兩分為十三等,並在各路、府、州、縣設常平倉平抑物價。將中書省宣敕、用人的權力劃歸尚書省。

至大四年正月初八日(1311年1月27日),因沉耽淫乐、酗酒过度,武宗病逝於大都玉德殿,享年三十岁,葬於起輦谷。

至大四年三月十八日(1311年4月7日),其弟愛育黎拔力八達(元仁宗)以皇太弟身份即位,廢除一切新政。

至大四年六月二十四日(1311年7月10日),元仁宗为海山上謚號仁惠宣孝皇帝,廟號武宗,蒙古语称号曲律皇帝。

至大二年正月初七日(1309年2月17日),皇太子、诸王、百官为元武宗上尊号统天继圣钦文英武大章孝皇帝。

由于日本拒绝向元朝称臣,元朝下令增加日货税收,日本不满,后来虽然减少关税,但仍然对日商检查甚严。

至大元年(1308年)日本商船焚掠庆元,官军不能敌。

至大四年(1311年)十月,以江浙省尝言:“两浙沿海濒江隘口,地接诸番,海寇出没,兼收附江南之后,三十余年,承平日久,将骄卒情,帅领不得其人,军马安量不当,乞斟酌冲要去处,迁调镇遏。“枢密院官议:“庆元与日本相接,且为倭商焚毁,宜如所请,其余迁调军马,事关机务,别议行之。”由此可见,此时元朝在东南沿海一带的军队战斗力很差(草原的军队因为世祖朝和成宗朝经常在西北作战,战斗力还可以)。

明朝官修正史《元史》宋濂等的評價是:“武宗当富有之大业,慨然欲创治改法而有为,故其封爵太盛,而遥授之官众,锡赉太隆,而泛赏之恩溥,至元、大德之政,于是稍有变更云。”

清朝史学家邵远平《元史类编》的評價是:“册曰:北藩入嗣,三宫协和;慨然创治,爵滥赏阿;貮省乱政,令教繁讹;有为何裨,变政已多。”

清朝史学家毕沅《续资治通鉴》的評價是:“帝承世祖、成宗承平之业,慨然欲创制改法;而封爵太盛,多遥授之官,锡赉太优,泛赏无节。至元、大德之政,于是乎变。”

清朝史学家魏源《元史新编》的評價是:“武宗始以怀宁王总兵漠北和林,与叛王海都劲敌对垒,屡摧其锋,中间几濒险危,披坚陷阵,威震遐荒,可谓天潢之杰出,天授之雄武矣。入绍大统,谓有宏图,而始终误听宵人,以立尚书省为营利之府,何哉?夫世祖立制,以天下大政归于中书省,任相任贤,责无旁贷。故小人欲变法,忌中书不便于己,则必别立尚书省以夺其权。阿合马、桑哥之徒相继乱政,毒流海内,是以世祖深戒前辙,不复再蹈。乃当席丰履厚之余,慨然欲变更至元、大德之旧。封爵太盛,而遥授之官多;锡赉太侈,而滥赏之卮漏。母后市恩左右,挠其恭俭,于是言利之臣迎合攘袂,以争利权。虽柄操自上,不至如阿合马、桑哥之甚,而仁心仁闻渐蔽于功利,几同于宋之熙、丰。故仁宗绍统,翻然诛殛,尽复旧章。盖变法不得其人,则不如勿薬之尚得中医也。又攷陶九仪《元氏掖庭记》,则琼岛水嬉之华,月殿霓裳之豔,亦自帝大滥其觞,而《本纪》讳之,不载一字,亦英雄酒色之通病欤!惟授受之际,坚守金匮传弟之盟,虽有内侍李邦宁,怂恿离间,帝言:‘朕志已定,汝自往东宫言之。’斯则磊落光明,胜宋太宗万万。综计始末,固不失为一代之英主焉。”

清朝史学家曾廉《元书》的評價是:“论曰:武宗擐甲临边,至登大位,宜有雄武之风,而颓然晏安,惟鞠蘖芗泽之为乐,元业自是衰矣。遂至鼎鼐充庭,名器之贱如履。而欲后人惜其敝袴,得乎?易日负且乘致寇至,武宗启之矣。”

民国史学家屠寄《蒙兀儿史记》的評價是:“海山汗滥赏淫威,非恭俭之主也。明知尚书省貮政病民,排众议而立之。更钞铸钱,将以理财,而财政愈紊,前史称其慨然欲有所为,然郊天、祀孔、亲享太庙,诸虚文外,无足纪者。惟终身远铁木迭儿,虽以母后之命,不使得预朝政。由后校之,殆有所先见矣。若乃三宫协和,始终不受谗慝。其自处骨肉之间,盖亦有道焉尔。”

民国官修正史《新元史》柯劭忞的評價是:“武宗舍其子而立仁宗。与宋宣公舍与夷而立穆公无以异。公羊子曰:朱之乱,宣公为之。然则英宗之弑,文宗之篡夺,亦帝为之欤!《春秋》贵让而不贵争,公羊子之言过矣。帝享国日浅,滥恩幸赏无一善之可书。独传位仁宗,不愧孝友。其流祚于子孙宜哉。”

\subsection{至大}

\begin{longtable}{|>{\centering\scriptsize}m{2em}|>{\centering\scriptsize}m{1.3em}|>{\centering}m{8.8em}|}
  % \caption{秦王政}\
  \toprule
  \SimHei \normalsize 年数 & \SimHei \scriptsize 公元 & \SimHei 大事件 \tabularnewline
  % \midrule
  \endfirsthead
  \toprule
  \SimHei \normalsize 年数 & \SimHei \scriptsize 公元 & \SimHei 大事件 \tabularnewline
  \midrule
  \endhead
  \midrule
  元年 & 1308 & \tabularnewline\hline
  二年 & 1309 & \tabularnewline\hline
  三年 & 1310 & \tabularnewline\hline
  四年 & 1311 & \tabularnewline
  \bottomrule
\end{longtable}


%%% Local Variables:
%%% mode: latex
%%% TeX-engine: xetex
%%% TeX-master: "../Main"
%%% End:

%% -*- coding: utf-8 -*-
%% Time-stamp: <Chen Wang: 2019-12-26 14:53:56>

\section{仁宗\tiny(1311-1320)}

\subsection{生平}

元仁宗愛育黎拔力八達是元朝第四位皇帝,蒙古帝国第八位大汗,1311年4月7日—1320年3月1日在位,一共在位9年。清代乾隆晚期乾隆帝命改譯遼、金、元三史中的音譯專名,改譯阿裕爾巴里巴特喇,今日學界已無人使用。

早年助兄长海山即位,被海山立为皇太子(元朝的皇位继承人一律称皇太子),相约兄终弟及,叔侄相传。后嗣位,年號皇慶、延祐。

他去世后,諡號聖文欽孝皇帝,廟號仁宗,蒙古語稱號普顏篤皇帝,又譯巴顏圖可汗。

至大四年正月初八日(1311年1月27日),元武宗病逝。至大四年三月十八日(1311年4月7日),元仁宗在大都大明殿即位。

仁宗自幼熟讀儒籍,傾心釋典。他从十几岁起就师从著名儒士李孟,儒家的伦理和政治观念对他有很强的影响。 他在登基称帝之前,先后在身边任用的有王约、赵孟頫、张养浩等汉儒和很多艺术家以及翻译家和散曲作家。

仁宗不仅能够读、写汉文,还能鉴赏中国书法与绘画,此外他还非常熟悉儒家学说和中国历史。

仁宗下诏下令將《貞觀政要》、《帝範》、《資治通鑒》和儒家经典《尚书》、《大学衍义》等書翻译成蒙古文并刊行天下,令蒙古人、色目人誦習。 仁宗支持下刊行天下的汉文著作包括:儒家经典《孝经》、《烈女傳》、《春秋纂例》、《辨疑》、《微旨》以及元朝官修农书《农桑辑要》。

1234年,蒙古帝国灭金朝、控制中原地区后需大量人才治理国家,根据中书令耶律楚材的建议,1237年秋八月,元太宗窝阔台下诏开科取士。诸路考试,均于1238年(戊戌年)举行,史称“戊戌选试”。这次考试共录取东平杨奂等4030人,皆为一时名士,朝廷得到了需要的各方面的人才。但后来“当世或以为非便,事复中止”。

后来的定宗(贵由)、宪宗(蒙哥)、世祖、成宗、武宗等朝,朝廷多次讨论恢复科举,但因为多种原因,一直没能实现。

皇庆改元(1312年)仁宗将其儒师王约特拜集贤大学士,并将王约“兴科举”的建议“著为令甲”(《元史》列传第六十五王约)。 皇庆二年(1313年)农历十月,仁宗要求中书省议行科举。中书省官员建议只设德行明经一科取士,仁宗同意。

皇庆二年农历十一月十八日(1313年12月6日),元仁宗下诏恢复科举,以朱熹集注的《四书》为所有科举考试者的指定用书,并以朱熹和其他宋儒注释的《五经》为汉人科举考试者增试科目的指定用书。,

这一变化最终确定了程朱理学在今后600年里的国家正统学说地位,因为后来的明清两朝的科举取士基本沿袭元朝的科举制度及其实施办法,并在其基础上进一步加以发展、充实和完善。

元仁宗1313年下诏恢复科举距离元太宗窝阔台1238年的“戊戌选试”已经有75年,天下读书的士人至此再次获得以科举方式晉身做官的途徑,方便了不同社会阶层之间的流动,缓和了社会矛盾。

中书省对于乡试、会试(“会试”之名亦始见于金朝)、殿试的举行时间,每次考试的录取人数、考试内容、考官来源、各行省的乡试录取名额分配、考试过程中的考场纪律等都做了详细的规定。

乡试,每三年一次,都是在八月二十日举行,全国共在17个省级区域设17处乡试科场,按照不同的地方的人口和民族进行名额分配,从赴试者中选300名合格者次年二月到大都参加会试。值得注意的是,高麗王朝所在的征东行省也有乡试科场,并在300名乡试中选者中有3人的名额。

延祐元年(1314年)农历八月二十日,全国举行乡试,一共录取三百人。

延祐二年(1315年)农历二月初一日,三百名乡试合格者在大都举行会试第一场,初三日第二场,初五日第三场,取中选者一百人。

延祐二年(1315年)农历三月七日,一百名会试中选者在大都皇宫举行殿试(廷试),最终录取护都答儿、张起岩等五十六人为进士。

1238年的“戊戌选试”之后,科举考试中断了75年,元仁宗延祐年间恢复科举取士,史稱“延祐復科”。

从元仁宗1315年开科取士到1368年元惠宗逃离大都、元朝灭亡为止,科举每三年一次,元朝一共举行了16次科举考试,考中进士的共计1139人(中间因为因为元惠宗时期丞相伯颜擅权,执意废科举,1336年科举和1339年科举停办。)国子学积分及格生员参加廷试录取正副榜284人,总计为1423人。

延祐元年(1314年),元仁宗下诏在江浙、江西、河南等三行省地进行田产登记,清查田亩,以增加国家税收,但是当1314年农历十月经理正式实行时,由于官吏的上下其手导致的执行不力,很多富民通过贿赂官吏隐瞒田产,很多贫苦农民和有田富民则被官吏乱加亩数,广大农民深受其害,最终导致1315年江西赣州蔡五九起义,虽然两个月中就被平定,但是元仁宗迫于形势,不得不停止经理,并减免所查出的漏隐田亩租税。「延祐经理」以失败告终。

元仁宗即位后,“以格例条画有关于风纪者,类集成书,”编修成一部专门的监察法规《风宪宏纲》。 并命监察御史马祖常作《风宪宏纲序》。

元惠宗至元二年(1336年),在增订《风宪宏纲》的基础上,将有关御史台典章制度汇编为《宪台通纪》。

至大四年(1311年)三月元仁宗即位不久,允中书所奏,“择耆旧之贤、明练之士,时则若中书右丞伯杭、平章政事商议中书刘正等,由开创以来政制法程可著为令者,类集折衷,以示所司,”分为制诏、条格、断例三部分:此外将介于《条格》、《断例》之间的内容编成成别类。

延祐三年(1316年)五月,书成。书成之后,又命“枢密、御史、翰林、国史、集贤之臣相与正是,凡经八年而是事未克果。”

至治三年二月十九日(1323年3月26日),元英宗最终审定,命名《大元通制》,颁行天下。全书共88卷,2539条。

《大元通制》是继《至元新格》之后元朝的第二部法典,现在只有条格的一部分(22卷,653条)流传下来,称为《通制条格》。

在位期間,减裁冗员,整顿朝政,推行“以儒治國”政策。又出兵西北,击败察合台后王也先不花。

元朝历代皇帝中,仁宗是对元朝较有贡献和有一番作为的其中一位(其他几位較有作為的分别是元世祖、元成宗、元英宗和元文宗)。

仁宗后将武宗之長子和世㻋徙居云南,立自己兒子碩德八剌为皇太子,打破叔侄相传的誓約。這個做法導致後來元朝長達二十年的政治混亂及宮廷鬥爭。

根据史实,仁宗生平好酒,延祐七年正月二十一日(1320年3月1日),元仁宗在大都光天宫病逝,享年三十六岁,他的逝世可能和喝酒伤身有关系。

延祐七年八月初十日(1320年9月12日),元英宗为父亲愛育黎拔力八達上諡號聖文欽孝皇帝,廟號仁宗,蒙古語稱號普顏篤皇帝。

明朝官修正史《元史》宋濂等的評價是:“仁宗天性慈孝,聪明恭俭,通达儒术,妙悟释典,尝曰:‘明心见性,佛教为深;修身治国,儒道为切。’又曰:‘儒者可尚,以能维持三纲五常之道也。’平居服御质素,澹然无欲,不事游畋,不喜征伐,不崇货利。事皇太后,终身不违颜色;待宗戚勋旧,始终以礼。大臣亲老,时加恩赉;太官进膳,必分赐贵近。有司奏大辟,每惨恻移时。其孜孜为治,一遵世祖之成宪云。”

清朝史学家邵远平《元史类编》的評價是:“册曰:立极电扫,稗政悉除;设科辍猎,屏言利徒;澹然无欲,十年罔渝;是惟令主,信史用书。”

清朝史学家毕沅《续资治通鉴》的評價是:“帝天性恭俭,通达儒术,兼晓释典,每曰:‘明心见性,佛教为深;修身治国,儒道为大。’在位十年,不事游畋,不喜征伐,尊贤重士,待宗戚勋旧,始终有礼。有司奏大辟,每惨恻移时。其孜孜为治,一遵世祖成宪云。”

清朝史学家魏源《元史新编》的評價是:“武仁授受之际,无可议者,仁宗初政,首革尚书省敝政,在位九年,仁心仁闻,恭俭慈厚,有汉文帝之风。惟武宗初约,由帝传位己子和世㻋而后及于英宗。及武宗崩,仁宗立,乃出封和世㻋于云南,而立子硕德八剌为太子。虽迫于皇太后之命,而已不守初约矣。和世㻋不之云南而举兵赴漠北,又不予以总兵和林之任,于是英宗被弑而泰定以晋王入绍大统,武宗旧臣燕帖木儿不服,遂于泰定殂后迎立周王于漠北,迎立怀王于江陵。怀王先立,周王后至,岂肯让于兄,于是弑之于中途,而国乱者数世。使当初即立周王,何至于此。至铁木迭儿奸贪不法,已经言官列款弹劾,而犹碍于皇太后,不敢质问,遂贻英宗以奸党谋逆之祸,不得谓非仁宗贻谋不臧有以致之也。”

清朝史学家曾廉《元书》的評價是:“论曰:元代科举之议久矣,至延祐而后行之,何其难乎?夫元代文学之盛,亦不须科举也。然儒风以振矣。天下啧啧以盛事归之。仁宗不亦宜乎?”

清末民初史学家屠寄《蒙兀儿史记》的評價是:“汗事兴圣太后。终身不违颜色,手勘内难,迎奉海山汗,退处东宫,不矜不伐,及海山汗升遐,哀恸不已。居丧再逾月,而后践阼。其孝友盖天性也。通达儒术,妙悟释典,尝曰:‘明心见性,佛教为深;修身治国,儒道为切。’又曰:‘儒者可尚,以能维持三纲五常之道也。’居东宫日,即有志兴学,以铁穆耳汗朝建国子监未成,趋台臣奏毕其功。既即位,一再增广国子生额,行科举取士之法。又尝遣使四方,旁求经籍。得秘笈,辄识以小玉印,命近侍掌之。承旨忽都鲁都儿迷失、刘赓进宋儒真德秀《大学衍义》,汗觉而善之,谓侍臣曰:‘治天下此一书足矣。’命翰林学士阿邻铁木儿并《贞观政要》皆译以国语,与图象《孝经》、《列女传》同刊印,以赐蒙兀、色目诸臣。平居服御,质素澹然,无欲不事游畋,不喜征伐,不崇货利,不受虚誉。待宗戚勋旧始终以礼,太官进膳,必分赐贵近;有司奏大辟,每惨恻移时。尝谓札鲁忽赤买闾曰:‘札鲁忽赤,人命所系,其详阅狱辞,事无大小,必谋诸同寮,疑不能决,与省台臣集议以闻。’又顾谓侍臣曰:‘卿等以朕居帝位为安耶?朕惟太祖创业艰难,世祖混一不易,兢业守成,恒惧不能当天心,绳祖武,使万方百姓各得其所,朕念虑在兹,卿等固不知也。’其孜孜为治,一遵忽必烈汗成宪。 惟饮酒无度,或其短祚之由欤。”

民国官修正史《新元史》柯劭忞的評價是:“仁宗孝慈恭俭,不迩声色,不殖货利。侍宗戚勋旧,始终以礼,大臣亲老,时加恩赍。有司奏大辟,辄恻怛移时,晋宁侯甲兄弟五人,俱坐法死,帝悯之,宥一人以养其父母。崇尚儒学,兴科举之法,得士为多,可谓元之令主矣。然受制母后,嬖幸之臣见权用事,虽稔知其恶,犹曲贷之。常问右丞相阿散曰:‘卿日行何事。’对曰:‘臣等奉行诏旨而已。’帝曰:‘祖宗遣训,朝廷大法,卿辈犹不遵守,况朕之诏旨乎。’其切责宰相如此。有君而无臣,惜哉!”

\subsection{皇庆}

\begin{longtable}{|>{\centering\scriptsize}m{2em}|>{\centering\scriptsize}m{1.3em}|>{\centering}m{8.8em}|}
  % \caption{秦王政}\
  \toprule
  \SimHei \normalsize 年数 & \SimHei \scriptsize 公元 & \SimHei 大事件 \tabularnewline
  % \midrule
  \endfirsthead
  \toprule
  \SimHei \normalsize 年数 & \SimHei \scriptsize 公元 & \SimHei 大事件 \tabularnewline
  \midrule
  \endhead
  \midrule
  元年 & 1312 & \tabularnewline\hline
  二年 & 1313 & \tabularnewline
  \bottomrule
\end{longtable}

\subsection{延祐}

\begin{longtable}{|>{\centering\scriptsize}m{2em}|>{\centering\scriptsize}m{1.3em}|>{\centering}m{8.8em}|}
  % \caption{秦王政}\
  \toprule
  \SimHei \normalsize 年数 & \SimHei \scriptsize 公元 & \SimHei 大事件 \tabularnewline
  % \midrule
  \endfirsthead
  \toprule
  \SimHei \normalsize 年数 & \SimHei \scriptsize 公元 & \SimHei 大事件 \tabularnewline
  \midrule
  \endhead
  \midrule
  元年 & 1314 & \tabularnewline\hline
  二年 & 1315 & \tabularnewline\hline
  三年 & 1316 & \tabularnewline\hline
  四年 & 1317 & \tabularnewline\hline
  五年 & 1318 & \tabularnewline\hline
  六年 & 1319 & \tabularnewline\hline
  七年 & 1320 & \tabularnewline
  \bottomrule
\end{longtable}


%%% Local Variables:
%%% mode: latex
%%% TeX-engine: xetex
%%% TeX-master: "../Main"
%%% End:

%% -*- coding: utf-8 -*-
%% Time-stamp: <Chen Wang: 2019-10-18 15:46:02>

\section{英宗\tiny(1320-1323)}

元英宗硕德八剌,是元朝第五位皇帝,蒙古帝国第九位大汗,1320年4月19日—1323年9月4日在位,在位3年零5个月,是元仁宗之子。

1321年11月28日,群臣为硕德八剌上汉语尊号继天体道敬文仁武大昭孝皇帝。

去世后,谥号睿圣文孝皇帝,庙号英宗,蒙古语称号格坚皇帝。

延祐七年农历正月二十一日(1320年3月1日),元仁宗去世。延祐七年农历三月十一日(1320年4月19日),18岁的硕德八剌在太皇太后答己及右丞相铁木迭儿等人的扶持下,在大都大明殿登基称帝,是为元英宗,改元“至治”。英宗自幼受儒學薰陶,登基后推行“以儒治國”政策,但是前期英宗的权力受到太皇太后答己和权臣铁木迭儿的很大限制。

延祐七年农历五月十一日(1320年6月17日),元英宗任命拜住为左丞相,以遏制太皇太后和铁木迭儿的权力扩张。

至治元年农历十一月九日(1321年11月28日),群臣为元英宗上尊号继天体道敬文仁武大昭孝皇帝。

1322年10月6日右丞相铁木迭儿去世,1322年11月1日太皇太后去世 ,元英宗终于得以亲政。

至治二年农历十月二十五日(1322年12月4日),元英宗任命拜住为中书右丞相,并且不设左丞相,以拜住为唯一的丞相。在右丞相拜住、中书省平章政事张珪等的帮助下,元英宗进行改革,并实施了一些新政,比如裁减冗官,监督官员不法行为,颁布新法律,采用“助役法”以减轻人民的差役负担,等等。史称“至治改革”。

英宗在位后期,官修政書《大元圣政国朝典章》(《元典章》),内容包括元太宗六年(1234年)到元英宗至治二年(1322年)大约90年的政治、经济、军事、法律等方面官方资料,具有极高的史料价值。

至治三年农历二月十九日(1323年3月26日),元英宗颁布了继《至元新格》之后元朝第二部法律典籍—《大元通制》,一共有二千五百三十九条,其中断例七百一十七、条格千一百五十一、诏赦九十四、令类五百七十七。

元英宗曾经想把征东行省(高丽王国)郡县化,罢征东行省,改立三韩行省,完全和元朝的其他行省一个待遇,“制式如他省,诏下中书杂议”,因为集贤大学士王约说:“高丽去京师四千里,地瘠民贫,夷俗杂尚,非中原比,万一梗化,疲力治之,非幸事也,不如守祖宗旧制。”得到丞相的赞同,设立三韩行省奏议没有实行。最终高丽国祚得以存续,高丽人知道后,为王约画像带回高丽,为之立生祠,并说:“不绝国祀者,王公也。”

元英宗的新政使得元朝国势大有起色,但新政却触及到了蒙古保守贵族的利益,引起了他们的不满,而且英宗下令清除朝中铁木迭儿的势力,随着清理的扩大化,铁木迭儿的义子铁失在至治三年八月初四(1323年9月4日)趁着英宗从上都避暑结束南返大都途中,在上都以南15公里的地方南坡的刺杀了英宗及右丞相拜住等人。史称南坡之变。英宗去世时年仅21岁。

泰定元年农历二月十六日(1324年3月11日),元泰定帝为硕德八剌上谥号睿圣文孝皇帝,庙号英宗。

泰定元年农历四月八日(1324年5月1日),元泰定帝为硕德八剌上蒙古文稱号“格坚皇帝”。

明朝官修正史《元史》宋濂等的評價是:“英宗性刚明,尝以地震减膳、彻乐、避正殿,有近臣称觞以贺,问:‘何为贺?朕方修德不暇,汝为大臣,不能匡辅,反为谄耶?’斥出之。拜住进曰:‘地震乃臣等失职,宜求贤以代。’曰:‘毋多逊,此朕之过也。’尝戒群臣曰:‘卿等居高位,食厚禄,当勉力图报。苟或贫乏,朕不惜赐汝;若为不法,则必刑无赦。’八思吉思下狱,谓左右曰:‘法者,祖宗所制,非朕所得私。八思吉思虽事朕日久,今其有罪,当论如法。’尝御鹿顶殿,谓拜住曰:‘朕以幼冲,嗣承大业,锦衣玉食,何求不得。惟我祖宗栉风沐雨,戡定万方,曾有此乐邪?卿元勋之裔,当体朕至怀,毋忝尔祖。’拜住顿首对曰:‘创业惟艰,守成不易,陛下睿思及此,亿兆之福也。’又谓大臣曰:‘中书选人署事未旬日,御史台即改除之。台除者,中书亦然。今山林之下,遗逸良多,卿等不能尽心求访,惟以亲戚故旧更相引用邪?’其明断如此。然以果于刑戮,奸党畏诛,遂构大变云。”

清朝史学家邵远平《元史类编》的評價是:“册曰:三载承乾,庶务锐始;大飨躬亲,致哀尽礼;刚过鲜终,肘腋祸起;不察几先,励精徒尔。”

清朝史学家毕沅《续资治通鉴》的評價是:“帝性刚明,尝以地震,减膳,彻乐,避正殿,有近臣称觞以贺,问:‘何为贺?朕方修德不暇,汝为大臣,不能匡辅,反为谄耶?’斥出之。尝戒群臣曰:‘卿等居高位,食厚禄,当勉力图报。苟或贫乏,朕不惜赐汝;若为不法,则必刑无赦。’巴尔济苏下狱,谓左右曰:‘法者,祖宗所制,非朕所得私。巴尔济苏虽事朕日久,今有罪,当论如法。’尝御鹿顶殿,谓拜珠曰:‘朕以幼冲,嗣承大业,锦衣玉食,何求不得!惟我祖宗栉风沐雨,戡定万方,曾有此乐耶?卿元勋之裔,当体朕至怀,毋忝尔祖!’拜珠顿首谢曰:‘创业维艰,守成不易,陛下言及此,亿兆之福也。’又谓大臣曰:‘中书选人署事未旬日,御史台即改除之。台除亦然。今山林之士,遗逸良多,卿等不能尽心求访,惟以亲戚故旧更相引用耶?’其明断如此。然以果于刑戮,奸党惧诛,遂构大变云。”

清朝史学家魏源《元史新编》的評價是:“旧史谓英宗果于诛戮,奸党畏惧,遂构大变。乌乎!是何言与?以铁木迭儿之奸,不明正其诛,但疏远俾得善终于位,已为漏网,而复任用其子,曲贷其子,酿成枭獍。此失之果乎?失之不果乎?拜住于铁木迭儿引其党参政张思明自助时,或告拜住为备,拜住反以大臣不和,彼仇我报,非国家之利。及铁木迭儿死,又往哭之痛,此皆失之果乎?失之不果乎?且除奸莫要于夺兵权,乃以宿卫新兵掌于铁失之手。司徒刘夔冒卖浙田之案,真人蔡道泰杀人赇逭之案,皆奸赃巨万。拜住既平反其狱,独赦铁失不问。中书参议谏以除奸不可犹豫,犹豫恐生他变,拜住是其言而不能用。大抵安童、拜住皆木华黎之孙,木华黎用兵所过,动辄屠戮。安童从许衡受学,故其子孙皆出于宽容,以水懦救火猛,德量有余,机警不足,所谓君子之过过于厚也。乃胡粹中因旧史之言,谓英宗在位数载,除诛戮外无一善政可纪,甚至皇太后以嬖孽失势之故,郁郁而终,胡氏并指为英宗不孝祖母之罪。乌乎!其性与人殊,乃至此乎?”

清朝史学家曾廉《元书》的評價是:“论曰:英宗知赵世炎之非辜,抑亦汉昭之流亚也。然汉昭能诛燕王、上官桀,而专任霍光,英宗不能诛铁木迭儿诸权倖之徒,独任拜住也。抑考元时蒙古人横不可悉裁以法度,以拜住之世旧勋贵而不能骤正也。夫自古无无小人之朝,在振纪纲而已。自世祖好货开倖进之门,安童不能与阿合马、桑哥争,况幼沖在位乎?使霍光处此,则必射隼,于高墉藏器,儃回操刀,弗割明君贤相,胥受其祸,悲夫!”

清末民初史学家屠寄《蒙兀儿史记》的評價是:“汗性刚明,励精图治,尝御上都大安阁,见太祖、世祖遗衣,皆以缣素木绵为之,重加补缀。嗟叹良久,谓侍臣曰:‘祖宗草昧经营,服御节俭乃尔,朕焉敢顷刻忘之。’敕画《蚕麦图》于鹿顶殿,以时观之,藉知民事。一日御殿,谓拜住曰:‘朕冲龄嗣祚,锦衣玉食,何求不得。惟我祖宗节风沐雨,戡定大难,曾有此乐耶?卿元勋之裔,当体朕至怀,毋忝尔祖。’拜住顿首对曰:‘创业惟艰,守成亦不易,陛下睿思及此,亿兆之福也。’汗承延祐宽政之后,思济之以猛,御下甚严,在谅闇中。中书参议乞失监坐鬻官,刑部议法当杖,太后欲改笞,汗不可,曰:‘法者,天下之公,徇私而轻重之,非所以示民也。’卒从部议。每戒群臣曰:‘卿等居高位,食厚禄,当勉力图报。苟或贫乏,朕不惜赐汝;若为不法,则必刑无赦。’八思吉思下狱时,汗谓左右曰:‘法者,祖宗之制,非朕所得私。八思吉思虽事朕日久,今既有罪,当论如法。’其明决如此。然过信喇嘛,大起山寺,不受忠谏,饮酒逾量,有时至失常度云。”

民国官修正史《新元史》柯劭忞的評價是:“英宗诛兴圣太后幸臣失列门等,太后坐视而不能救,其严明过仁宗远甚。然蔽于铁木迭儿,既死始悟其奸,又置其逆党于肘腋之地。故南坡之祸。由于帝之失刑,非由于杀戮也。旧史所讥殆不然矣。”

\subsection{志治}

\begin{longtable}{|>{\centering\scriptsize}m{2em}|>{\centering\scriptsize}m{1.3em}|>{\centering}m{8.8em}|}
  % \caption{秦王政}\
  \toprule
  \SimHei \normalsize 年数 & \SimHei \scriptsize 公元 & \SimHei 大事件 \tabularnewline
  % \midrule
  \endfirsthead
  \toprule
  \SimHei \normalsize 年数 & \SimHei \scriptsize 公元 & \SimHei 大事件 \tabularnewline
  \midrule
  \endhead
  \midrule
  元年 & 1321 & \tabularnewline\hline
  二年 & 1322 & \tabularnewline\hline
  三年 & 1323 & \tabularnewline
  \bottomrule
\end{longtable}


%%% Local Variables:
%%% mode: latex
%%% TeX-engine: xetex
%%% TeX-master: "../Main"
%%% End:

%% -*- coding: utf-8 -*-
%% Time-stamp: <Chen Wang: 2019-10-18 15:52:31>

\section{泰定帝\tiny(1323-1328)}

元泰定帝也孙铁木儿是元朝第六位皇帝,蒙古帝国第十位大汗,在位5年,自1323年10月4日至1328年8月15日。清代乾隆晚期乾隆帝命改譯遼、金、元三史中的音譯專名,改譯伊蘇特穆爾,今日學界已無人使用。

他去世后不久,叔父之孫元文宗打敗其子元天顺帝,亦使他沒被授與谥号和庙号,因此历史上以其年号称之为泰定帝。

关于泰定帝的出生年,《元史》中的说法互相矛盾,在《元史·泰定帝一》中称“至元十三年十月二十九日,帝生于晋邸。”至元十三年是1276年,但在《元史·泰定帝二》中又说“庚午,帝崩,寿三十六”,按这个说法他应该是1293年(至元三十年)出生的。很可能作者误把“三十”写成了“十三”。泰定帝“生于晋邸”,而1292年甘麻剌被封为晉王,而且1328年他的长子阿剌吉八當時只有8岁,所以泰定帝应该是生于1293年。他的父亲甘麻剌是元世祖太子真金的長子,1292年被封为晉王,出镇嶺北。1302年甘麻剌死后也孙铁木儿袭晋王位。

至治三年(1323年)三月也孙铁木儿在元英宗附近的亲信向他告密说英宗将对也孙铁木儿不利。同年八月二日,也孙铁木儿获得英宗将被刺杀、自己将被迎立为皇帝的消息。

至治三年八月初四(1323年9月4日),铁木迭儿的义子铁失趁着元英宗从上都避暑结束南返大都途中,在上都以南15公里的地方南坡的刺杀了元英宗及右丞相拜住等人。史称南坡之变。

元英宗被刺后也孙铁木儿果然被擁立为皇帝,至治三年九月初四日(1323年10月4日),也孙铁木儿在漠北草原的龙居河(今克鲁伦河)河畔登基称帝。虽然也孙铁木儿是知情人,但他登基后就下令将刺杀英宗的人都處死了。

至治三年十一月十三日(1323年12月11日),泰定帝到达大都。1323年12月17日,泰定帝在大都大明殿接受诸王和百官朝贺。

至治三年十二月十一日(1324年1月7日),泰定帝追尊其父亲甘麻剌為皇帝,为甘麻剌上庙号显宗,汉文谥号光圣仁孝皇帝;追尊其母亲普顏怯里迷失为皇后,为普顏怯里迷失上谥号宣懿淑圣皇后。

泰定元年三月二十日(1324年4月14日),泰定帝立八八罕氏为皇后,立阿速吉八为太子。

从1325年开始,泰定帝因国库收入少于支出,开始减少国家支出。七月,他下令不允许汉人收藏和携带兵器。

泰定二年九月初一日(1325年10月8日),泰定帝改革全国的行政区划,将全国划分为18个道,分别为:两浙道、江东道、江西道、福建道、江南道、湖广道、河南道、江北道、燕南道、山东道、河东道、陕西道、山北道、辽东道、云南道、甘肃道、四川道、京畿道。

泰定帝还下达了一系列命令禁止和尚和道士购买民间的土地,克制僧院的过分富有。

在泰定帝统治期间,广西、四川、湖南、云南等少数民族地区经常爆发反抗元朝统治的暴乱,泰定帝一般使用软硬兼施的手段来平息这些暴乱。但从整体来说整个国家基本上比较安宁。

致和元年七月初十日(1328年8月15日),元泰定帝在上都病逝,享年36岁。

元泰定帝七月去世后,九月,他的儿子元天顺帝在上都登基,改元天顺,九月十三日,元武宗之子元文宗在大都登基,改元天历,双方交战一个月,最终以元文宗获胜告终,元天顺帝失败后下落不明,不知所终。

也孙铁木儿无庙号和谥号,故以年号史称为泰定帝。

明朝宋濂等官修正史《元史》的評價是:“泰定之世,灾异数见,君臣之间,亦未见其引咎责躬之实,然能知守祖宗之法以行,天下无事,号称治平,兹其所以为足称也。”

清朝史学家邵远平《元史类编》的評價是:“册曰:长子世嫡,嗣统允宜;武仁先立,泽承人思;忽焉不世,电灭云移;或曰南坡,其蛮与知;故史具在,其又谁欺?”

清朝史学家毕沅《续资治通鉴》的評價是:“帝在位,灾异数见,然能守祖宗之法,天下号称治平。”

清朝史学家魏源《元史新编》的評價是:“一代统绪之传,有正统即有公论,岂一时私意所能傎倒磔裂者哉!世祖明孝太子早卒,皇孙成宗立,追谥裕宗。成宗本裕宗第三子,其同母二兄,一为晋王甘麻剌,一为怀王答剌麻八剌,本无嫡庶,而晋邸居长。成宗崩后无嗣,晋王之子泰定帝即可嗣立,乃因仁宗自怀庆入,先靖内难,迎立其兄怀宁王于漠北,是为武宗。所谓先入关者王之,非晋王子不当立而必立怀王子也。及再传至英宗遇弑,晋王复出自漠北入靖内难,讨贼嗣位,是为泰定。与武、仁之事相埒,非武、仁有功宗社,而泰定无功也。泰定践阼,即以和林兵柄授周王使代己任,屡通朝贡。又召怀王自海南入朝京师,锡封藩国,移近江陵,屡赐金币,是泰定于文宗兄弟有德而无怨也。泰定太子册立已五载,父终子继,名正言顺,怀王、周王安得入干大统乎!若谓武、仁当日原有传位周王,嗣及英宗之约,则仁宗实背约在前,可以责仁宗,不可以责泰定也。乃文宗篡立之诏,谓泰定以旁支入继,正统遂偏,甚至诬其与贼臣铁失潜通阴谋,冒干宝位,追毁晋王显宗庙室。乌乎!以讨贼之主,而诬以通贼之罪,是何言哉!若谓武宗二子为人心所归,泰定当舍子而传侄,则何以天历颁诏至关中、至四川、至辽东,皆焚书斩使,起兵拒命,则人心归泰定之子,而不归武宗之子,明如星日。是则燕帖木儿之为逆臣,怀王之为逆立,亦明如星日。固不待鲁桓弑隐夺国,已无所逃于《春秋》之责,况欲宽其罪于中途弑逆之后哉!斯非难定之案,而数百年尚无定论。请断之,以折曲沃桓叔之徒,假托正谊者。”

清朝史学家曾廉《元书》的評價是:“论曰:周太王以国传王季,设季而无后,则泰伯之子孙遂不可以复承周祀乎?美哉晋王之让,而泰定之立,亦不可不畏之正也。上都告变,惜已无及,然大节亦明矣。故诸凶迁官非有他也,仓卒之间,形格势禁,度权力未足以制其命也。荣宠以诱之,俾喜而懈,稍缓须臾,成备而出,而疾雷不及掩耳矣。呜呼!此帝之所以为权,然岂不果哉!至后纪纲弗振,由不纳张珪、宋本之言,而乱是用长也,累受佛戒,亦梁武之俦乎?”

清末民初史学家屠寄《蒙兀儿史记》的評價是:“至元六年,诏称英宗遇害,正统遂偏,于戏!此惠宗一人之私言也。太子真金嫡子三人,泰定之父晋王甘麻剌最长,次则武仁之父答剌麻八剌,又次为成宗。成宗之立,非世祖本意也。向使储闱符玺之归,果足为大统继嗣之证,则当世祖宾天,诸王大会,成宗曷不径遵遗诏,即位梓宫之前,出受群臣之贺。顾乃迟回三月,必得晋王北面愿事之一言,而大策始定,何也?盖成宗以皇孙出抚北军时,既无王号,又未赐印,世祖用玉昔帖木儿之请,濒行仓卒,授以故太子宝,代一时行军印之用而已。非有告庙册立之礼也。晋王不让,成宗不得立。则所谓正统,宜属晋王之子孙。明史臣王祎言:武宗约继世子孙,兄弟相及。而仁宗不守宿诺,传位英宗,仍使武宗二子出居于外。及英宗遇弑,而明宗在北,文宗在南。嗣晋王于世祖为嫡长曾孙,则求所当立,舍嗣晋王谁归?旧传英宗之弑,晋邸与闻,考之宝录,不得其证。传闻之缪,殊不足信。邵阳魏氏源亦言:成宗无嗣,大统当归,泰定徒以仁宗自怀先入,靖内难而迎立武宗,所谓先入关者王之,非晋王子不当立,而必立答剌麻八剌子也。泰定能讨贼,胜于武、仁杀疑似之宗亲,非武、仁有功社稷,而泰定无功也。泰定践阼,即归周王之妃八不沙于漠北,召图帖睦尔汗于海南,既至京师,厚加赐予。封为怀王,妻以主女。初镇建康,六朝都会;及移江陵,益据上游。泰定之于怀王,有德而无怨也。阿速吉八太子册立已五载,父终子继,名正言顺,大统所在,孰得干之?若谓武宗当日原有传位周王以及英宗之约,则仁宗实背约在前,可以责仁宗,不可以责泰定也。若谓武宗二子,人心所归,泰定当舍子而传侄,何以山后、辽东、关陇、滇、蜀,先后为上都起兵,即河南、湖广,犹必执杀省官,易置郡县长吏,强之而后从。当日讴歌讼狱,不之武宗之子,而之泰定之子,明矣。然则燕帖木儿之为逆臣,怀王之为篡立。不待鲁桓弑隐,已无所逃于《春秋》之诛。况可宽其罪于旺兀察都推刃天伦之后哉!斯狱县之六百年,请断之,以折曲沃桓叔之徒,假托名义者。”(至元六年为1338年,此处至元为元惠宗年号。)

民国柯劭忞官修正史《新元史》的評價是:“孔子称叔孙昭子之不劳。泰定帝讨铁失等弑君之罪,虽叔孙昭子何以尚之。文宗篡立,欲厌天下之人心,诬蔑之辞无所不至。惜乎后世之君子,不引孔子之言,以论定其事也。”

\subsection{泰定}

\begin{longtable}{|>{\centering\scriptsize}m{2em}|>{\centering\scriptsize}m{1.3em}|>{\centering}m{8.8em}|}
  % \caption{秦王政}\
  \toprule
  \SimHei \normalsize 年数 & \SimHei \scriptsize 公元 & \SimHei 大事件 \tabularnewline
  % \midrule
  \endfirsthead
  \toprule
  \SimHei \normalsize 年数 & \SimHei \scriptsize 公元 & \SimHei 大事件 \tabularnewline
  \midrule
  \endhead
  \midrule
  元年 & 1324 & \tabularnewline\hline
  二年 & 1325 & \tabularnewline\hline
  三年 & 1326 & \tabularnewline\hline
  四年 & 1327 & \tabularnewline\hline
  五年 & 1328 & \tabularnewline
  \bottomrule
\end{longtable}

\subsection{致和}

\begin{longtable}{|>{\centering\scriptsize}m{2em}|>{\centering\scriptsize}m{1.3em}|>{\centering}m{8.8em}|}
  % \caption{秦王政}\
  \toprule
  \SimHei \normalsize 年数 & \SimHei \scriptsize 公元 & \SimHei 大事件 \tabularnewline
  % \midrule
  \endfirsthead
  \toprule
  \SimHei \normalsize 年数 & \SimHei \scriptsize 公元 & \SimHei 大事件 \tabularnewline
  \midrule
  \endhead
  \midrule
  元年 & 1328 & \tabularnewline
  \bottomrule
\end{longtable}


%%% Local Variables:
%%% mode: latex
%%% TeX-engine: xetex
%%% TeX-master: "../Main"
%%% End:

%% -*- coding: utf-8 -*-
%% Time-stamp: <Chen Wang: 2019-10-18 15:53:31>

\section{天顺帝\tiny(1328)}

元天顺帝阿剌吉八,是元朝第七位皇帝,蒙古帝国第十一位大汗,元泰定帝之子。1328年10月3日至1328年11月14日在位,在位一个月十一天。

致和元年七月初十日(1328年8月15日),元泰定帝也孙铁木儿在上都病逝,丞相倒剌沙专权自用,过了一个多月仍迟迟不立9岁的太子阿剌吉八即位。

致和元年九月十三日(1328年10月16日),知樞密院事燕帖木儿在大都(今北京)拥立元武宗之子图帖睦尔即位,改元“天历”,图帖睦尔是为元文宗。

致和元年九月,丞相倒剌沙在上都拥立太子阿剌吉八为皇帝,改元“天顺”。

上都的天顺帝朝廷由丞相倒剌沙派兵进攻大都的文宗朝廷,元文宗派燕帖木儿率军迎战,双方经过多次战争,一开始双方互有胜负,后来大都朝廷逐渐占据军事优势。

天顺元年十月十三日(1328年11月14日),大都朝廷的军队包围上都,丞相倒剌沙等大臣奉皇帝宝出降,天顺年号被元文宗废除,倒剌沙在投降一个月后被杀。

倒剌沙投降后,天顺帝下落不明,不知所终,在位大约一個月;其後他沒被授與谥号和庙号,因此历史上以其年号称之为天顺帝。

清朝史学家曾廉《元书》的評價是:“论曰:曾子以托孤寄命,临大节而不可夺,斯为君子人也。故山有猛虎,樵采不入。前史称泰定帝能守祖宗之法,故天下无事。呜呼!徒法不能以自行也,向使汉武不委裘于霍光、金日磾,而倚上官桀、桑弘羊,则孝昭岂得晏然南面?况又弗如孝昭者乎?狙于近习而不知求天下之贤以佐佑之,贵为天子,富有天下,而不能庇其妻孥,若敖之鬼佞焉咎安在哉!君子是以不多子孟,而乐道孝武之善付托也。”

\subsection{天顺}

\begin{longtable}{|>{\centering\scriptsize}m{2em}|>{\centering\scriptsize}m{1.3em}|>{\centering}m{8.8em}|}
  % \caption{秦王政}\
  \toprule
  \SimHei \normalsize 年数 & \SimHei \scriptsize 公元 & \SimHei 大事件 \tabularnewline
  % \midrule
  \endfirsthead
  \toprule
  \SimHei \normalsize 年数 & \SimHei \scriptsize 公元 & \SimHei 大事件 \tabularnewline
  \midrule
  \endhead
  \midrule
  元年 & 1328 & \tabularnewline
  \bottomrule
\end{longtable}


%%% Local Variables:
%%% mode: latex
%%% TeX-engine: xetex
%%% TeX-master: "../Main"
%%% End:

%% -*- coding: utf-8 -*-
%% Time-stamp: <Chen Wang: 2021-11-01 17:07:22>

\section{文宗图帖睦尔\tiny(1328-1332)}

\subsection{生平}

元文宗图帖睦尔,是元朝第八位皇帝,蒙古帝国第十二位大汗,两次在位,第一次在位时间为1328年10月16日—1329年4月3日;後復位,第二次在位时间为1329年9月8日—1332年9月2日,在位时间共4年,他是元武宗的次子。清代乾隆晚期乾隆帝命改譯遼、金、元三史中的音譯專名,改譯圖卜特穆爾,今日學界已無人使用。

1330年5月25日,群臣为图帖睦尔上汉语尊号钦天统圣至德诚功大文孝皇帝。

他去世后,谥号圣明元孝皇帝,庙号文宗,蒙古語称札牙篤皇帝。

致和元年七月十日(1328年8月15日),元泰定帝在上都去世。八月,在大都(今北京)的燕帖木儿等大臣决定立元武宗的长子周王和世㻋为帝,但是因为路远而先迎周王之弟怀王图帖睦尔(元文宗)。九月,在上都的倒剌沙等大臣則立太子阿速吉八为帝,是為天顺帝,并发兵攻大都。

天曆元年九月十三日(1328年10月16日),知樞密院事燕帖木儿在大都拥立图帖睦尔在大都大明殿即位称帝,并在即位诏中改致和元年为天曆元年。燕帖木儿经过多次战争,于1328年11月14日打败位于上都的天顺帝朝廷,天下安定。

元文宗采納燕帖木儿的建议,照原本的安排立自己的兄長周王和世㻋为帝,是为元明宗。1329年2月27日,元明宗在漠北草原和宁之北即位,并派遣撒迪等人前往大都通知元文宗;但直到1329年4月3日,在大都的元文宗才派遣燕铁木儿和众多官员奉皇帝宝玺前往元明宗行在所,正式让出皇位。5月5日,燕铁木儿率百官将皇帝宝玺献给元明宗。5月15日,元明宗正式立图帖睦尔为皇太子(實應為皇太弟)。8月16日,图帖睦尔受皇太子宝。8月25日,元明宗抵达元武宗时建为中都的王忽察都。8月26日,皇太子图帖睦尔入见,两兄弟会面,元明宗宴请皇太子及诸王、大臣于行殿。1329年8月30日,燕帖木儿毒死元明宗。

天曆二年八月十五日(1329年9月8日),在燕帖木儿等官员的拥戴下,元文宗于上都大安阁再次即位称帝,并发布第二次即位诏;因該年的年号是天曆,史称天曆之变。

元文宗第一次在位期间,於天历二年二月二十七日(1329年3月27日)設立了奎章閣學士院,掌進講經史之書,考察歷代治亂,又令所有勛貴大臣的子孫都要到奎章閣學習;奎章閣下設藝文監,專門負責將儒家典籍譯成蒙古文,以及校勘。同年下令編纂《經世大典》,兩年後修成,為元代一部重要的記述典章制度的巨著。元文宗第二次登基后亦大兴文治。

至顺元年五月八日(1330年5月25日),丞相燕帖木儿率文武百官及僧道、耆老,奉玉册、玉宝,为元文宗上尊号钦天统圣至德诚功大文孝皇帝。

元文宗在位期间,丞相燕帖木儿自恃有功,玩弄朝廷,元朝朝政更加腐败,国势更加衰落。文宗在位期间国内多次爆发民变,大动乱正在酝酿之中。

至顺三年八月十二日(1332年9月2日),元文宗在上都病逝,终年28岁。

元统元年十一月二十一日(1333年12月28日),侄子元順帝为图帖睦尔上谥号圣明元孝皇帝、庙号文宗,蒙古语称札牙笃皇帝。

文宗頗具漢文化修養,喜愛作詩。《宋元詩會》記載:文宗怡情詞翰,雅喜登臨。居金陵潛邸時,常屏從官,獨造鍾山冶亭,吟賞竟日,惜現存詩作僅有數首而已。又精於書畫。《元史》記載,文宗的書法受趙孟頫影響而宗晉人,落筆過人,得唐太宗晉祠碑風,遂益超旨。文宗曾命近臣房大年畫《京都萬歲山圖》,房大年以為自己火候未到而請辭。文宗於是索紙運筆,先作一稿,大年驚服,謂格法周匝停勻,雖積學專工,莫能及也。文宗的書畫作品在今日極為罕見,僅有《相馬圖》一幅。

清朝史学家邵远平《元史类编》的評價是:“册曰:应变戡乱,莫匪尔劳;玺绶虽去,太阿已操;前车所鉴,烛影斧声;从来疑案,多在弟兄。”

清朝史学家魏源《元史新编》的評價是:“元代诸帝不习汉文,凡有章奏,皆由翻译。其读汉书而不用翻译者,前惟太子真金,从王恽、王恂受学。后惟文宗潜邸,自通汉文而已。《书画谱》言,文宗在潜邸时,召画师房大年,俾图京师万岁山。大年以未至其地辞,文宗遂取笔布画位置,顷刻立就,命大年按稿图上。大年得稿敬藏之,意匠经营,虽积学专工,有所未及。始知文宗之多材多艺也。及践阼后,开奎章阁,招集儒臣,撰备《经世大典》数百卷,宏纲巨目,礼乐兵农,灿然开一代文明之治。即其声色俭澹,亦远胜武宗,此岂庸主所希及哉!使其迎立明宗之日,亦如仁宗之退处东宫,他日明宗复如武宗之传仁庙,则一代而胜事再见,虽殷人弟兄世及,何以过此!《易》曰:‘开国承家,小人勿用。’文宗之得大位也,以燕帖木儿;其得罪万世也,亦以燕帖木儿。语曰:‘治世之能臣,乱世之奸雄。’文宗之不陨于太平王手者,亦幸矣哉!”(魏源说“元代诸帝不习汉文,凡有章奏,皆由翻译。”此事并不符合历史事实,这和他了解的相关书籍不多有关。事实上,真金太子和元文宗的汉文学修养的确很高,除此之外,还有很多位元朝帝王有很高的汉文学修养。根据史料, 元世祖、元成宗、元仁宗、元英宗、元文宗、元順帝、元昭宗,均有很高的汉文化修养,其中,元世祖、元文宗、元順帝、元昭宗这四位帝王有汉文诗传世。元仁宗、元英宗、元憲宗和元文宗都受到过良好的汉学教育,都有很高的汉文学修养。

清朝史学家曾廉《元书》的評價是:“论曰:元自文宗,始亲郊祀,礼彬彬焉。尊崇圣贤之典,至是益隆,而开奎章阁以致儒臣,考文章,论治道,勤于延访,可以为文矣。然几沉而气锐,抑亦吴闾庭之流也。其言泰定帝通贼臣,阴谋冒干宝位,呜呼!文宗将毋其自道之也!兴且晋邸,日有盟书,周王可必其终为泰伯乎?文宗之深心乃以让,济其忍,然后足固其威福也,岂不险哉!生则欺人,死而犹饰,故地碎其主,春秋震夷伯之庙,所谓有隐慝者乎?”

清末民初史学家屠寄《蒙兀儿史记》的評價是:“汗旧劳于外,多艺好文。在建康潜邸时,忽忆京师万岁山,召画师房大年图之,大年以未至其地辞,汗自取笔,布画位置,顷刻立就,命大年按稿图上。大年得稿敬藏之,意匠经营,虽积学专工,有所未及。即位后首建奎章阁,御制记文,集儒臣阁中备顾问,敕编《经世大典》,保存一代制度。性爱典礼,欲革蒙兀腥膻本俗,则躬服衮冕,虔祀郊庙。又慎于用刑,行枢密院尝当云南逃军二人死罪,汗谓:‘临阵而逃,死宜也。彼非逃战,辄当以死,何视人命之易耶?’杖而流之。天历初抗命诸王大臣,临事故多诛杀,其它窜黜者,事后多蒙召还,或仍录用。至于严惩赃吏,尊信老成,节诸王驸马朝会刍粟赏赐之财,汰宿卫鹰坊饔人僧徒冗食之数。诸所设施,实一代恭俭守文之令主也。惟得国不正,隐亏天伦,且授权燕铁木儿太甚,未能大有为。”

民国官修正史《新元史》柯劭忞的評價是:“燕铁木儿挟震主之威,专权用事。文宗垂拱于上,无所可否,日与文字之士从容翰墨而已。昔汉灵帝好词赋,召乐松等待诏鸿都门,蔡邕露章极谏,斥为俳优。况区区书画之玩乎?君子以是知元祚之哀也。”

\subsection{天历}

\begin{longtable}{|>{\centering\scriptsize}m{2em}|>{\centering\scriptsize}m{1.3em}|>{\centering}m{8.8em}|}
  % \caption{秦王政}\
  \toprule
  \SimHei \normalsize 年数 & \SimHei \scriptsize 公元 & \SimHei 大事件 \tabularnewline
  % \midrule
  \endfirsthead
  \toprule
  \SimHei \normalsize 年数 & \SimHei \scriptsize 公元 & \SimHei 大事件 \tabularnewline
  \midrule
  \endhead
  \midrule
  元年 & 1328 & \tabularnewline\hline
  二年 & 1329 & \tabularnewline\hline
  三年 & 1330 & \tabularnewline
  \bottomrule
\end{longtable}

\subsection{志顺}

\begin{longtable}{|>{\centering\scriptsize}m{2em}|>{\centering\scriptsize}m{1.3em}|>{\centering}m{8.8em}|}
  % \caption{秦王政}\
  \toprule
  \SimHei \normalsize 年数 & \SimHei \scriptsize 公元 & \SimHei 大事件 \tabularnewline
  % \midrule
  \endfirsthead
  \toprule
  \SimHei \normalsize 年数 & \SimHei \scriptsize 公元 & \SimHei 大事件 \tabularnewline
  \midrule
  \endhead
  \midrule
  元年 & 1330 & \tabularnewline\hline
  二年 & 1331 & \tabularnewline\hline
  三年 & 1332 & \tabularnewline\hline
  四年 & 1333 & \tabularnewline
  \bottomrule
\end{longtable}


%%% Local Variables:
%%% mode: latex
%%% TeX-engine: xetex
%%% TeX-master: "../Main"
%%% End:

%% -*- coding: utf-8 -*-
%% Time-stamp: <Chen Wang: 2019-12-26 14:54:31>

\section{明宗\tiny(1329)}

\subsection{生平}

元明宗和世㻋,是元朝第九位皇帝,蒙古帝国第十三位大汗,1329年2月27日至1329年8月30日在位,在位185天。元武宗長子。清代乾隆晚期乾隆帝命改譯遼、金、元三史中的音譯專名,改譯和實拉,今日學界已無人使用。

他去世后,谥号翼獻景孝皇帝,庙号明宗,蒙古语称忽都篤皇帝。

1340年10月25日,元惠宗為元明宗上汉语尊号順天立道睿文智武大聖孝皇帝。

根据《元史》,天曆二年正月丙戌(儒略曆1329年2月27日),和世琜在漠北草原的和宁之北即位,继续使用年号“天曆”,是为元明宗,1329年4月3日,元文宗图帖睦尔派人将皇帝宝玺献给明宗,正式禪讓帝位,5月15日,元明宗正式立图帖睦尔为皇太子,8月16日,图帖睦尔受皇太子宝,8月25日,元明宗抵达元武宗时建为中都的王忽察都,8月26日,皇太子图帖睦尔入见,两兄弟会面,元明宗宴请皇太子及诸王、大臣于行殿。

天曆二年八月六日(1329年8月30日),元明宗和世㻋被燕帖木儿毒死,明宗去世时享年30岁。

1329年9月8日,燕帖木儿重新拥戴元文宗復辟,因为1329年的年号是天曆,史称天曆之变。

天曆二年十月十三日(1329年11月4日),元文宗为兄長和世㻋上谥号翼獻景孝皇帝,庙号明宗,蒙古文称忽都篤皇帝。

元明宗的两个儿子元宁宗懿璘质班和元惠宗妥懽帖睦尔在1332年9月2日元文宗去世后相继登基称帝。

至元六年十月四日(1340年10月25日),元惠宗给元明宗上尊号順天立道睿文智武大聖孝皇帝。

清朝史学家邵远平《元史类编》的評價是:“册曰:艰艰备尝,人望所属;何嫌何疑,推肝置腹;人心不同,天命反覆;论定千秋,此直彼曲。”

清朝史学家曾廉《元书》的評價是:“论曰:昔曹子臧、吴季札,贤者也。其君国子民也宜哉!然而义不受者,非独远情,亦知负飞及光之不厌,其欲将无以善其后也,闇哉明宗!焉有人披衮执玉,穆穆然位乎天位而肯北面俯首为人臣者乎?呜呼!此唐明皇不敢以望肃宗,父子且然,况兄弟哉!文宗盖惧北陲,复有海都、笃哇之流,托名拥戴,其言也顺而为患也。深抑亦私心,窃望周王之效法晋邸也。己则非夷,而以齐期人。不亦难乎?悠悠南行,甘咽其饵,悲夫!”

清末民初史学家屠寄《蒙兀儿史记》的評價是:“和世㻋汗年未弱冠,远逊金山,耕牧十有三年。所谓旧劳于外,知民情伪者也。观其论台纲,谕百司,斤斤于先世成宪,是殆有心救弊者乎?然以此言论风采,自曝于风尘道路之间,致令傲弟权相闻而生心,遂有旺兀察都之变。《易》曰:‘君不密,则失臣。’此之谓矣。怀抱盛意,未见设施,惜哉!”

民国官修正史《新元史》柯劭忞的評價是:“燕铁木儿立文宗,文宗固让于兄,犹仁宗之奉武宗也。明宗之弑,盖出于燕铁木儿,非文宗之本意。然与闻乎弑,是亦文宗弑之而已。”

\subsection{天历}

\begin{longtable}{|>{\centering\scriptsize}m{2em}|>{\centering\scriptsize}m{1.3em}|>{\centering}m{8.8em}|}
  % \caption{秦王政}\
  \toprule
  \SimHei \normalsize 年数 & \SimHei \scriptsize 公元 & \SimHei 大事件 \tabularnewline
  % \midrule
  \endfirsthead
  \toprule
  \SimHei \normalsize 年数 & \SimHei \scriptsize 公元 & \SimHei 大事件 \tabularnewline
  \midrule
  \endhead
  \midrule
  二年 & 1329 & \tabularnewline
  \bottomrule
\end{longtable}


%%% Local Variables:
%%% mode: latex
%%% TeX-engine: xetex
%%% TeX-master: "../Main"
%%% End:

%% -*- coding: utf-8 -*-
%% Time-stamp: <Chen Wang: 2021-11-01 17:07:58>

\section{宁宗懿璘质班\tiny(1332)}

\subsection{生平}

元寧宗懿璘质班是元朝第十位皇帝,蒙古帝国第十四位大汗。元明宗次子。1332年10月23日—1332年12月14日在位,在位2个月。

他去世后,谥号冲圣嗣孝皇帝,庙号寧宗。

《元史》记载,元宁宗于泰定三年三月二十九癸酉日(1326年5月1日)生于北方草原。

至顺三年八月十二日(1332年9月2日),元文宗崩。据杂史,元文宗在死前下诏让元明宗之子继承皇位。文宗死后,把持朝政的燕铁木儿为了继续专权,就请求元文宗皇后卜答失里立她的儿子古納答剌为帝。卜答失里为了执行丈夫的遗诏,予以拒绝。由于当时元明宗的长子妥懽贴睦尔(后来的元惠宗)远在广西静江(今广西桂林),而次子懿璘质班却深得文宗宠爱,受封为鄜王,留在文宗身边。

至顺三年十月初四(1332年10月23日),卜答失里皇后遂奉文宗遗诏拥立年仅7岁的懿璘质班在大都大明殿登上皇位,是为元宁宗。因为皇帝年幼,卜答失里皇后临朝称制,成了元朝的实际统治者。

懿璘质班即位后未改元,年号仍旧是“至顺”,至顺三年十一月二十六日(1332年12月14日),元宁宗在大都病逝,年仅7岁,在位仅53天。

至元三年正月十日(1337年2月10日),元惠宗为懿璘质班上谥号冲圣嗣孝皇帝、庙号宁宗。

清朝史学家魏源《元史新编》的評價是:“乌乎!《春秋》未逾年之君称子,故子般不与闵公并立庙谥。宁宗以负扆匝月之殇,而入庙称宗,立后媲谥,无一人引大谊以匡正之,斯元代礼臣博士之陋也。修史者又踵其失而立《本纪》,斯又明臣之陋也。今以附诸《文宗本纪》之末。”

清朝史学家曾廉《元书》的評價是:“论曰:文宗杀明宗皇后,播告天下,言妥懽帖睦尔非明宗子,既出之于静江,乃立皇子阿剌忒答剌为皇太子,公私之情见矣。皇天弗佑,元良夭丧,及大惭,而爱其少子之弱,非妥懽帖睦尔不能延其祚,而不可为之辞矣。则亦曰立明宗子,一似以明其固让之初志也者。任后人之拥戴,犹武宗之孙也。惟宁宗亦弗永年而大位卒,归于向所猜忌之兄子,天也!人岂有为哉!”

清末民初史学家屠寄《蒙兀儿史记》的評價是:“鄜王之立,不再月而殇。既未逾年改元,又未有所建设,顾乃追尊上谥,立庙称宗,甚乖《春秋》鲁般书子卒之义。蒙兀君臣瞢不知经,诚无足责,而明初脩胜国之史,仍立之本纪,不加裁正,宜乎魏源讥其陋也。退附《文宗本纪》,自邵远平始。”

民国官修正史《新元史》柯劭忞的評價是:“《春秋》之义,未逾年之君称子。宁宗即位匝月而殇,乃入庙称宗;其廷臣不学如此,岂非失礼之大者哉。” 

\subsection{志顺}

\begin{longtable}{|>{\centering\scriptsize}m{2em}|>{\centering\scriptsize}m{1.3em}|>{\centering}m{8.8em}|}
  % \caption{秦王政}\
  \toprule
  \SimHei \normalsize 年数 & \SimHei \scriptsize 公元 & \SimHei 大事件 \tabularnewline
  % \midrule
  \endfirsthead
  \toprule
  \SimHei \normalsize 年数 & \SimHei \scriptsize 公元 & \SimHei 大事件 \tabularnewline
  \midrule
  \endhead
  \midrule
  三年 & 1332 & \tabularnewline
  \bottomrule
\end{longtable}


%%% Local Variables:
%%% mode: latex
%%% TeX-engine: xetex
%%% TeX-master: "../Main"
%%% End:

\input{20_Yuan/11_HuiZOng}
%% -*- coding: utf-8 -*-
%% Time-stamp: <Chen Wang: 2021-11-01 17:09:23>

\section{北元\tiny(1368-1388)}

\subsection{简介}

北元指明朝建立并遣徐达大军攻陷元朝首都大都(汗八里)后,退居蒙古高原的原元朝宗室的政權,因国号仍叫大元,以其地处塞北,故稱“北元”[1]。 北元始于元惠宗至正二十八年(1368年,明太祖洪武元年),终于脱古思帖木儿天元十年(明朝洪武二十一年,1388年),为蒙古(明人稱鞑靼)所代替。

元惠宗至正二十八年正月初四日(1368年1月23日)明太祖建立明朝,統一南方,令徐达北伐,徐达率领的军队逼近大都,闰七月二十八日(1368年9月10日),元惠宗夜半开大都的健德门北奔,率太子愛猷識理答臘、后妃、臣僚等撤离大都,八月初二日(1368年9月14日),明军从大都的齐化门攻城而入,元朝对中国的统治结束,回到本土蒙古草原。

元惠宗撤离大都后,继续使用“大元”国号,当时高丽人叫北元。當時政治形势是除了元惠宗據有漠南漠北的蒙古本土,關中還有元將擴廓帖木兒(王保保)駐守甘肅定西,此外元廷還領有东北地区與雲南行中书省地區。

明太祖為了驱逐位于蒙古的元廷势力,採取兵分二路,各個擊破的方式,此即第一次北伐。至正二十八年八月初四日(1368年9月16日),元惠宗到达上都。至正二十九年六月十三日(1369年7月16日),明军逼近上都,元惠宗撤离上都,当天到达应昌。六月十七日(1369年7月20日),明将常遇春攻克上都。

元惠宗在上都和应昌那里曾两次组织兵力试图收复大都,但都被明军击败。至正三十年(洪武三年)四月二十八日(1370年5月23日)元惠宗因痢疾在应昌去世,享年51岁。皇太子愛猷識理答臘在应昌繼承皇位,是为元昭宗,并于1371年改元宣光。至正三十年五月十六日(1370年6月10日),明将李文忠攻克应昌,元昭宗撤至哈拉和林,并坚持抵抗明军。

擴廓帖木兒仍然在漠北多地与明将徐达等人作戰。明太祖曾多次寫信詔降,但擴廓帖木兒從不理会,被朱元璋稱為「當世奇男子」。元昭宗宣光二年(1372年)正月,徐达从雁门出发,向哈拉和林进发。三月,明将蓝玉在土拉河大败扩廓帖木儿。五月,扩廓帖木儿在草原击败明将徐达的这支明军。自此之后,明军十几年不再进攻漠北,直到1388年,蓝玉才再次进攻漠北草原。

宣光二年六月初三(1372年7月3日),明将冯胜大败元军,明朝从元朝治下收取甘肃行中书省地区。

宣光八年(1378年)四月,元昭宗去世,继位的北元后主脫古思帖木兒在1379年六月改年号为天元,继续和明军对抗,屢次侵犯明境[2]。

1371年,元朝辽阳行省平章刘益降明,明朝控制今辽宁南部。然而之外的辽阳行省地区仍由元朝太尉纳哈出控制,纳哈出屯兵二十万于金山(今辽宁省昌图金山堡以北辽河南岸一带),自恃畜牧丰盛,与明军对峙了十几年,多次拒绝明太祖的招抚。1387年,冯胜、傅友德、蓝玉等人發動第五次北伐,目标是攻占纳哈出的金山。经过多次战争,1387年10月,纳哈出投降蓝玉,明朝控制原辽阳等处行中书省的东北地区。

鎮守雲南的元朝梁王把匝剌瓦尔密,在元朝对中国的统治结束,撤到老家蒙古草原后依然繼續忠效之。1371年明太祖派湯和等人領兵攻灭據有四川的明玉珍的明夏政权,並且勸降梁王未果。1381年12月,明军的沐英和傅友德兵分二路攻入雲南,天元三年十二月二十二日(1382年1月6日),梁王把匝剌瓦尔密自杀,数月后,元朝云南大理总管段氏投降明军,明軍征服雲南地區,元朝对云南的统治结束。[2]。

1388年,蓝玉率领明军十五万發動第六次北伐,明军穿越过戈壁沙漠到达草原东部,天元十年四月十二日(1388年5月18日),蓝玉在捕鱼儿海(今贝尔湖)附近大败元军,俘虏北元后主次子地保奴及妃主五十余人、渠率三千、男女七万余,马驼牛羊十万。脱古思帖木儿和长子天保奴、知院捏怯来、丞相失烈门等数十骑逃走。

至此北元国力衰落。天元十年十月,脫古思帖木兒被也速迭尔(阿里不哥后裔)杀害,从1388年开始,蒙古不再使用年号,帝号、大元国号被废弃,北元时期结束。

在中外蒙古史学者的论著中,屡见“北元”一词,但是长期以来,对于这一史学概念的使用范畴却众说不一。争论的焦点就是“北元”是指1368-1388年这20年间的蒙古还是指1368-1635年这260多年间的蒙古。传统说法是1402年鬼力赤杀坤帖木儿汗,为北元时期结束的时间(《明史·鞑靼传》)。 关于这个问题,蔡美彪先生和曹永年先生曾作过深入探讨,认为“北元”应适用于脱古思帖木儿败亡而止,即1388年,此后大元国号已取消,仍称蒙古。

“北元”仅指大蒙古国的一个阶段,其根据是:脱古思帖木儿败亡后,蒙语文献中不再见大元国号的使用。思帖木儿败亡后,元朝传统的帝号、谥号、年号均不再见(也先汗与达延汗时期除外)。

1388年,北元皇帝、大汗脱古思帖木儿被叛臣也速迭儿弑杀。关于这位弑汗自立的也速迭儿,《华夷译语》中所载降明的蒙古知院捏怯来的奏报称是阿里不哥的子孙。这是一条很重要的史料。当年,蒙哥汗去世,镇守漠北的阿里不哥与控制中原的忽必烈发生汗位之争,结果阿里不哥失败,忽必烈做了蒙古大汗。随后迁都北京,仿汉族王朝模式定国号为“大元”,实行“汉法”,当上了元朝的皇帝。在这个时期,阿里不哥也被忽必烈杀害。忽必烈的所做所为无疑引起了阿里不哥子孙和漠北守旧的蒙古贵族的仇恨。在他们看来,大元是他们不共戴天的仇敌。有元一代,尽管忽必烈及其子孙在祖宗根本之地设立行省,实行宗王出镇制度,但这块龙飞之地却从未平静过。阿里不哥一系为首的反元斗争持续不断,这就是阿里不哥一派地方势力与元朝中央势不两立的明证。在这种心态驱使下,一旦元朝衰落,对蒙古草原的控制减弱,他们就会奋起反元。

在脱古思帖木儿败亡后的很长一段时间里,“部帅纷擎”,战乱频仍,与外界的联系基本中断。当时的明朝在捕鱼儿海战役胜利后,重点亦转向了对内部事务的处理。1398年明太祖去世,翌年太祖四子朱棣与建文帝同室操戈,是为“靖难之役”。这段时间《明实录》基本上没有关于蒙古的记载。直到朱棣“靖难”成功,当上了皇帝,才又重新开始了对北部边防的经略,致书蒙古大汗,要求“遣使往来通好,同为一家”,而此时已是1403年了。这时的蒙古大汗已是鬼力赤(他称汗在1402年左右)。当明朝方面获悉蒙古已去大元国号后,遂有明史「鬼力赤篡立,称可汗,去国号,遂称鞑靼」的误载。事实上,“去国号”的不是鬼力赤,而是也速迭儿。

在以后的蒙古历史上,大元国号仍出现,也先汗、达延汗时期即如此。但是,他们恢复大元国号的举动给汉蒙双方都带来了巨大的震动,这恰好反映出明代蒙古在大多数时期已取消了大元国号这一事实。

大元国号的废弃一定意义上意味着蒙古政权放弃了争夺中原的目标,转为立足于蒙古本身。

“北元”(1368年-1388年)仅代表一个时期的结束,其后进入《明史》所说的鞑靼时期(为明人所称,蒙方一直以蒙古自称)。但是从成吉思汗开始的“大蒙古国”政权仍然继续,鞑靼政权长期沿用元朝时代的汉制职官(如也先官职为太师淮王),至满都海夫人时才基本取消。“大蒙古国”政权延续至1635年察哈尔部为满洲的后金-清所灭亡。

故大蒙古国(1206年-1635年)依照中国名称的划分,可划为蒙古(1206年-1271年)、元朝(1271年-1368年)、北元(1368年-1388年)、鞑靼(1388年-1635年)。有时“元朝”可泛指从1206年至1368年这段时期。

\subsection{昭宗愛猷識理達臘\tiny(1370-1368)}

\subsubsection{生平}

元昭宗愛猷識理達臘,是北元的第二位君主,第十六位蒙古大汗,蒙古文称号必里克圖汗。他的在位時間是從1370年5月27日至1378年5月10日,在位8年,年號宣光。父為元順帝妥懽帖睦爾,母親是高麗貢女奇皇后。

明代王世貞《北虜始末志》稱愛猷識理達臘為“昭宗”。清代乾隆朝《蒙古世系譜》則稱愛猷識理達臘為“哲宗”。“哲宗”一說未被後人接受。

愛猷識理答臘生于元惠宗至元四年或五年的十二月二十四日。他的生母奇氏因为生育皇子,母凭子贵,至元六年(1340年)被元惠宗封为第二皇后,就是奇皇后。

至正十三年(1353年)六月,愛猷識理答臘被元惠宗(元顺帝)封为太子,他做太子之后,元朝内部党争日益激烈。愛猷識理答臘自己试图夺取帝位,提前登基,这样就造成了他和他父亲的关系紧张。至正二十四年(1364年),他的政敵將軍孛罗帖木儿帶兵闖入大都,愛猷識理答臘被迫流亡到王保保(扩廓帖木儿)的控制區太原,并以此为基地,召集各省军阀准备反攻孛罗。与此同时,元惠宗也对孛罗的专权产生不满,遂派人将其刺死,将人头送到太原,召回了愛猷識理達臘并与其和解。

至正二十八年(明朝洪武元年)闰七月二十八日(1368年9月10日),明朝军队逼近大都,元惠宗率太子愛猷識理達臘、后妃、臣僚等北走,前往上都,至正二十八年八月二日(1368年9月14日),明太祖的將軍徐達攻克大都。

至正二十八年八月四日(1368年9月16日),元惠宗和太子愛猷識理答臘等人到达上都。至正二十九年六月十三日(1369年7月16日),明军逼近上都,元惠宗和太子等人离开上都,当天到达应昌(今内蒙古克什克腾旗达里诺尔西南古城)。至正二十九年六月十七日(1369年7月20日),明军将领常遇春攻克上都。

至正三十年农历五月二日(1370年5月27日),元惠宗因痢疾去世于应昌,皇太子愛猷識理答臘在应昌繼承了皇位,並次年改元宣光。至正三十年农历五月十六日(1370年6月10日),明军将领李文忠攻克应昌,昭宗逃往和林,身边仅有一小股随从陪同,他的众多妃子以及儿子买的里八剌被明军俘虏,还有五万余元军投降明军。

宣光二年六月初三(1372年7月3日),明军将领冯胜大败元军,明朝从元朝手中取得甘肃地区。

北元在當時仍保持一定的勢力,在宣光二年(1372年)的戰事中,在王保保指挥下,元朝對於明朝贏得了一個局部勝利。

元昭宗於宣光八年(1378年5月10日)农历四月十三日逝世,在位8年,享年40岁。

元昭宗死後由弟北元后主脱古思帖木儿繼位,脱古思帖木儿年号为天元,又称为天元帝。

民国官定正史《新元史》柯劭忞的評價是:“昭宗以下,文献无徵。惟宣光八年之事,间存一二,故附载于本纪云。”

\subsubsection{宣光}

\begin{longtable}{|>{\centering\scriptsize}m{2em}|>{\centering\scriptsize}m{1.3em}|>{\centering}m{8.8em}|}
  % \caption{秦王政}\
  \toprule
  \SimHei \normalsize 年数 & \SimHei \scriptsize 公元 & \SimHei 大事件 \tabularnewline
  % \midrule
  \endfirsthead
  \toprule
  \SimHei \normalsize 年数 & \SimHei \scriptsize 公元 & \SimHei 大事件 \tabularnewline
  \midrule
  \endhead
  \midrule
  元年 & 1371 & \tabularnewline\hline
  二年 & 1372 & \tabularnewline\hline
  三年 & 1373 & \tabularnewline\hline
  四年 & 1374 & \tabularnewline\hline
  五年 & 1375 & \tabularnewline\hline
  六年 & 1376 & \tabularnewline\hline
  七年 & 1377 & \tabularnewline\hline
  八年 & 1378 & \tabularnewline\hline
  九年 & 1379 & \tabularnewline
  \bottomrule
\end{longtable}

\subsection{益宗脱古思帖木儿\tiny(1378-1388)}

\subsubsection{生平}

元天元帝脱古思帖木儿是北元第三位君主,第十七位蒙古大汗。史称北元后主,或以他的年号天元称为天元帝。或根据明朝史籍记载,他是愛猷識理達臘的弟弟。明代王世貞《北虜始末志》記載,脫古思帖木兒繼位前是益王。1378年5月13日—1388年11月1日在位,在位10年。

根据继承的次序推断,脱古思帖木儿应该就是蒙古语史料中的兀思哈勒可汗或烏薩哈爾汗。《蒙古源流》和《新元史》等史料记载他是必里克图可汗(愛猷識理達臘)的弟弟,但是这和《元史》中愛猷識理達臘弟弟早亡的记载不符。他的蒙古文称号是烏薩哈爾汗,无汉文廟號与諡號。

脱古思帖木儿的生年不详,父元順帝乌哈噶图汗。《蒙古源流》记载兀思哈勒可汗生于壬午年(1342年)。《黄史》记载他三十岁即位,照此推算生年约为1349年。但这些记载都和兀思哈勒是愛猷識理達臘(生于1339或1340年)之子的推断矛盾。

他於1378年5月即位,1379年农历六月改年号为天元。

1381年12月,明军进攻云南,天元三年十二月二十二日(1382年1月6日),镇守云南的元梁王把匝剌瓦尔密兵败自杀,天元四年闰二月二十三日(1382年4月7日),明将蓝玉、沐英攻克大理城,元朝大理总管段世投降明军,明朝平定云南,元朝在云南的统治结束。

从1254年元宪宗的皇弟忽必烈(后继位为元世祖)灭大理国,到1382年明军击败元军夺取云南,元朝统治云南地区长达128年。

1371年,元朝辽阳行省平章刘益降明,明朝占領辽宁南部。然而其餘东北地区仍由元朝太尉纳哈出控制,纳哈出屯兵二十万于金山(今辽宁省昌图金山堡以北辽河南岸一带),自持畜牧丰盛,与明军对峙了十几年,多次拒绝明朝的招抚。1387年冯胜、傅友德、蓝玉等人發動第五次北伐,目标是攻占纳哈出的金山。经过多次战争,1387年10月,纳哈出投降蓝玉,明朝占領东北地区。

天元十年四月十二日(1388年5月18日),明军将领蓝玉在捕鱼儿海(今贝尔湖)附近大败元军,俘虏脱古思帖木儿次子地保奴及妃主五十余人、渠率三千、男女七万余,马驼牛羊十万。脱古思帖木儿和长子天保奴、知院捏怯来、丞相失烈门等数十骑逃走。

1388年农历十月,脱古思帖木儿去世,次子恩克卓里克图继位。一说脱古思帖木儿遭阿里不哥後裔也速迭兒襲殺篡位。

\subsubsection{天元}

\begin{longtable}{|>{\centering\scriptsize}m{2em}|>{\centering\scriptsize}m{1.3em}|>{\centering}m{8.8em}|}
  % \caption{秦王政}\
  \toprule
  \SimHei \normalsize 年数 & \SimHei \scriptsize 公元 & \SimHei 大事件 \tabularnewline
  % \midrule
  \endfirsthead
  \toprule
  \SimHei \normalsize 年数 & \SimHei \scriptsize 公元 & \SimHei 大事件 \tabularnewline
  \midrule
  \endhead
  \midrule
  元年 & 1379 & \tabularnewline\hline
  二年 & 1380 & \tabularnewline\hline
  三年 & 1381 & \tabularnewline\hline
  四年 & 1382 & \tabularnewline\hline
  五年 & 1383 & \tabularnewline\hline
  六年 & 1384 & \tabularnewline\hline
  七年 & 1385 & \tabularnewline\hline
  八年 & 1386 & \tabularnewline\hline
  九年 & 1387 & \tabularnewline\hline
  十年 & 1388 & \tabularnewline
  \bottomrule
\end{longtable}


%%% Local Variables:
%%% mode: latex
%%% TeX-engine: xetex
%%% TeX-master: "../Main"
%%% End:



%%% Local Variables:
%%% mode: latex
%%% TeX-engine: xetex
%%% TeX-master: "../Main"
%%% End:
 % 元
% %% -*- coding: utf-8 -*-
%% Time-stamp: <Chen Wang: 2019-12-26 15:06:05>

\chapter{明\tiny(1368-1644)}

\section{简介}

明朝(1368年1月23日-1644年4月25日)是中國歷史上最後一個由漢人建立的大一统王朝,歷經十二世、十六位皇帝,國祚二百七十六年。

元朝末年政治腐败,天灾不断,民不聊生,农民起义屡禁不止,朱元璋加入红巾军并在其中乘势崛起,跟隨佔據濠州的郭子興。郭子興死後,朱元璋率部眾攻佔集慶(今江蘇南京),採取李善长所建议的「高築牆,廣積糧,緩稱王」的政策,鞏固根據地,讓士兵屯田積糧減少百姓負擔,以示自己為仁義之師而避免受敵。1368年,在扫灭陈友谅、張士誠和方国珍等群雄勢力后,朱元璋于当年农历正月初四日登基称帝,立国号为大明,并定都應天府(今南京市),其轄區稱為京師,由因皇室姓朱,因此又稱朱明。後以「驅逐胡虜,恢復中華」為號召北伐中原,少數民族政權統治四百年的燕云十六州也被漢族政權收回,結束蒙元在中國漢地的統治,并最終消滅陳友諒、張士誠和方國珍等各地群雄勢力,统一天下。明初天下大定,经过朱元璋的休养生息,社会经济得以恢复和发展,国力迅速恢复,史称洪武之治。朱元璋去世后,其孙朱允炆即位,但其在靖难之役中败于驻守燕京的朱元璋第四子朱棣,也自此失蹤。朱棣登基后遷都至順天府(今北京市),将北平布政司升為京師,原京師改稱南京。成祖朱棣时期,开疆拓土,又派遣鄭和七下西洋,此後許多漢人遠赴海外,国势达到顶峰,史称永乐盛世。其後的仁宗和宣宗时期国家仍处于兴盛时期,史称仁宣之治。英宗和代宗時期,遭遇土木之变,国力中衰,经于谦等人抗敌,最终解除国家危机。宪宗和孝宗相继与民休息,孝宗则力行节俭,减免税赋,百姓安居乐业,史称弘治中兴。武宗时期爆发了南巡之争和寧王之亂。世宗即位初,引发大礼议之争,他清除宦官和权臣势力后总揽朝纲,实现嘉靖中兴,并于屯门海战与西草湾之战中击退葡萄牙殖民侵略,任用胡宗宪和俞大猷等将领平定东南沿海的倭患。世宗驾崩后经过隆庆新政国力得到恢复,神宗前期任用张居正,推行万历新政,国家收入大增,商品经济空前繁荣、科学巨匠迭出、社会风尚呈现出活泼开放的新鲜气息,史称万历中兴。后经过万历三大征平定内忧外患,粉碎丰臣秀吉攻占朝鮮进而入明的計劃,然而因為国本之争,皇帝逐渐疏于朝政,史稱萬曆怠政,同时东林党争也带来了明中期的政治混乱。

萬曆一朝成為明朝由盛轉衰的轉折期。光宗继位不久因红丸案暴毙,熹宗继承大统改元天啟,天启年间魏忠贤阉党祸乱朝纲,至明思宗即位後铲除阉党。然而因其重用東林黨治國導致政治腐败以及连年天灾,导致国力衰退,最终爆发大规模民变。1644年4月25日(舊曆三月十九),李自成所建立的大順军攻破北京,思宗自缢於煤山,是為甲申之變。隨後吴三桂倒戈相向,满清入主中原。明朝宗室於江南地区相繼成立南明诸政权,原本反明的農民軍加入南明陣營,這些政權被清朝統治者以「为君父报仇」为名各个歼灭,1662年,明朝宗室最後政權被剷除,永曆帝被俘殺,滿清又击败各地农民军,以及進攻由漢人首次管理的台湾,直到1683年清朝攻占奉大明為正朔的明郑方止。

明代的核心領土囊括汉地,东北到外興安嶺及黑龍江流域,後縮為遼河流域;初年北達戈壁沙漠一帶,後改為今長城;西北至新疆哈密,後改為嘉峪關;西南临孟加拉湾,后折回约今云南境;曾經在今中国东北、新疆東部及西藏等地設有羈縻機構。不過,明朝是否實際統治了西藏國際上存在有一定的爭議。明成祖時期曾短暫征服及統治安南,永乐二十二年(1424年),明朝国土面积达到极盛,在东南亚设置旧港宣慰司等行政机构,加强对东南洋一带的管理。

明代商品经济繁荣,出现商业集镇,而手工业及文化艺术呈现世俗化趋势。根據《明实录》所载的人口峰值于成化十五年(1479年)达七千余万人,不过许多学者考虑到当时存在大量隐匿户口,故认为明朝人口峰值实际上逾亿,还有学者认为晚明人口峰值接近2亿。这一时期,其GDP总量所占的世界比例在中国古代史上也是最高的,1600年明朝GDP总量为960亿美元,占世界经济总量的29.2\%,晚明中国人均GDP在600美元。

明朝政治中央废除丞相,六部直接对皇帝負責,後来设置内阁;地方上由承宣布政使司、提刑按察使司、都指挥使司分掌权力,加强地方管理。仁宗、宣宗之后,文官治国的思想逐渐浓厚,行政权向内阁和六部转移。同时还设有都察院等监察机构,為加強對全國臣民的監視,明太祖設立特務機構錦衣衛,明成祖設立東廠,明憲宗时再設西廠(後取消),明武宗又設內廠(後取消),合稱「廠衛」。但到了后期出现了皇帝怠政,宦官行使大權的陋習,然而决策权始终集中在皇帝手里,不是全由皇帝独断独行。有许多事还必须经过经廷推、廷议、廷鞫的,同时还有能将原旨退还的给事中,另到了明代中晚期文官集團的集體意見足以與皇帝抗衡,在遇到事情決斷兩相僵持不下時,也容易產生一種類似於「憲法危機」的情況,因此「名義上他是天子,實際上他受制於廷臣。」。但明朝皇權受制於廷臣主要是基於道德上而非法理上,因為明朝當時風氣普遍注重名節,受儒家教育的皇帝往往要避免受到「昏君」之名。但是,皇帝隨時可以任意動用皇權,例如明世宗「大禮議」事件最後以廷杖朝臣多人的方式結束。

有学者认为明代是继汉唐之后的黄金时期。清代張廷玉等修的官修《明史》评价明朝为「治隆唐宋」、「遠邁漢唐」。

朱元璋早期给新的王朝定名为大中,后正式定国号为“大明”,是元朝以来中国历史上第二个把“大”字加于正式国号之中的大一统王朝,又称皇明,后世称为明朝或明代,又因皇室姓朱,又称朱明。部分人認為明朝之号承袭自小明王韩林儿之号,但韓林兒的國號為宋,而朱元璋部的大旗“山河奄有中華地,日月重開大宋天”、“ 九天日月開黃道,宋國江山複寶圖”反而有些關係。朱元璋手下有一部分明教徒,以“大明”為国号以表示自己的正统地位,亦同时应和明教中的“明王出世”预言。其次,以明喻火,根据五德终始说,表示明朝取代元朝,是以火剋金。

但七十年代,學界開始有人質疑“明王”是否出於明教(摩尼教) 。八十年代初,楊訥閱讀現存所有元代白蓮教史料後,否定吳晗學說。他除指出吳晗論文方法上的錯誤,及引証史料之疏漏外,並以傳世史料,証實元末起事者所提“彌勒佛下生”與“明王出世”口號,均與明教無涉,而出於佛教經典。但不論吳唅或楊訥,都是從宗教角度來探究。直到2014年,北京大學博士生杜洪濤突破了吳晗學說窠臼,循元明承續的思路,參照趙翼大元國號出自《易經•乾卦》“大哉乾元” 文義,而主張大明國號亦出自《易經•乾卦》“大明終始”這一字句,為大明此一國號的源由又增添了一種說法。

1644年4月24日(舊曆三月十八),明朝首都沦陷后,明朝宗室在江南地区建立政权仍沿用大明国号,别称南明或后明,清廷則称为伪明,一直坚持到1662年。而郑芝龙、郑成功等郑家势力在台湾建立了政权,史称東寧王國。

还有人指出,明之得号出于明教。明教在唐朝武则天延载年间,传到中国,但是一直保持神秘,因为明教宣传的是“弥勒降生,明王下世”。一些反抗朝廷的人经常借助于明教来号召群众,为了保护自己,明教就跟佛教拉上关系,和佛教的白莲宗拉上关系,最后就形成了白莲教。所以从唐朝、宋朝、元朝明教是时而浮出,时而潜入地下,但是常常用作反抗朝廷的武器。

元朝末期,官員貪污,貴族靡爛,朝政腐敗。為消除赤字,元廷加重賦稅,並且大量濫印新鈔「至正寶鈔」,隨之產生的通貨膨脹加上荒災、黃河氾濫等天災比以往任何時候發生得都要頻繁,使得民不聊生。1351年元順帝派賈魯治理黃河,徵調各地百姓二十萬人。同年五月,白蓮教韓山童與劉福通煽動飽受天災與督工苛待的百姓叛元起事。他自稱明王,建立紅巾軍,據有河南與安徽等地。紅巾軍與各地義軍陸續起事,勢力擴張到華中、華南地區。隔年,紅巾軍的郭子興聚眾起義,攻佔濠州(今安徽鳳陽)。不久,貧苦農民出身的安徽鳳陽人朱元璋投奔郭子興,屢立戰功,得到郭子興的器重和信任,並娶郭子興養女為妻。之後,朱元璋離開濠州,發展自己的勢力。

1356年朱元璋率兵佔領集慶(今江蘇省南京市),改名為應天府,並攻下周圍一些軍事要地,獲得一塊立足的基地。朱元璋採納謀士朱升「高築牆,廣積糧,緩稱王」的建議,經過幾年努力,其軍事和經濟實力迅速壯大。1360年,陳朱雙方在集慶西北的龍灣展開惡戰,陳友諒勢力遭到巨大打擊,逃至江州,史稱洪都之戰(今江西省九江市)。1363年,通過鄱陽湖水戰,陳友諒勢力基本被消滅。1367年朱元璋自稱吳王,率軍攻下平江(今江蘇省蘇州市),滅張士誠,同年又消滅割據浙江沿海的方國珍。

1368年正月,朱元璋於南京稱帝,即明太祖,年號洪武,明朝建立。之後趁元朝內訌之際乘機北伐和西征,同年攻佔元大都(今北京),元廷撤出中原,史稱北元。之後於1371年消滅位於四川的明玉珍勢力,於1381年消滅據守雲南的元朝梁王。最後,於1388年深入漠北進攻北元。天下至此初定。而朱元璋对于不願效忠新朝的蒙古人和色目人,则表示愿意归顺的可以在大明,不愿意的可以自行离开。

明初不願仕官和不願效忠新朝廷的地主文人為了逃避徵辟而採取自殺、自殘、逃往漠北、 隱居深山等方法,誓不出仕(中國古代銓選,有「身言書判」四方面標準,身體有殘疾者不能任官)。為應對元遺民對明政權的鄙夷與漠視,朱元璋設立新刑罰,宣告「士大夫不為君用」律,大規模徵辟前朝遺老、搜羅岩穴隱士,並且殺害不願效忠明朝以及為新朝當官的學者,表示「率土之濱,莫非王臣。寰中士大夫不為君用,是自外其教者,誅其身而沒其家,不為之過」,導致「才能之士,數年來倖存者百無一二,今所任率迂儒俗吏」。

由于幼年对于元末吏治痛苦记忆,明太祖即位后一方面減輕農民負擔,恢復社會的經濟生產,改革元朝遺留的吏治,懲治貪官,社會經濟從戰亂中得到恢復和發展,史稱洪武之治。明太祖確立里甲制,配合賦役黃冊戶籍登記簿冊和魚鱗圖冊的施行,落實賦稅勞役的徵收及地方治安的維持。同时对外加强海外交流,恢复中华宗主国地位。

平定天下後,明太祖大封功臣。但随后基于巩固皇权的考虑,加之不少功臣或骄纵或横行乡里或僭越等,明太祖兴起胡惟庸案和藍玉案,幾乎將功臣及权贵誅殺。廖永忠成为最先被处置的功臣。丞相胡惟庸深得朱元璋寵信,但之后日益跋扈,朝中奏章大事須先經其手,若不利於其的奏章就予以隱匿,並且大肆收取賄賂。1380年明太祖以擅權枉法之罪名殺胡惟庸,又殺御史大夫陳寧、御史中丞塗節等人。1390年有人告發李善長與胡惟庸關係密切,李善長因此被賜死,家屬七十餘人被殺,總計株連者達三萬餘人,史稱胡惟庸案,明太祖更藉此案廢除中書省和相職。此後,明太祖又借大將軍藍玉張狂跋扈之名對其誅殺,連坐被族誅的有一萬五千餘人,史稱藍玉案。加上空印案與郭桓案合稱明初四大案。此時除湯和與耿炳文外功臣几乎全数被杀。明太祖通過打擊权臣、特務監視等一系列方式加強皇權,使明初的皇帝專制程度與中國歷代各朝相比更為嚴重。

明太祖分封诸子為王,以加強邊防,藩屏皇室。諸王之中,以北方諸王勢力較強,又以秦王朱樉、晉王朱棡與燕王朱棣的勢力最大。為防止朝中奸臣不軌,明太祖規定諸王可移文中央捉拿奸臣,必要時得奉天子密詔,領兵「靖難」(意为“平定國難”)。同時為防止諸王尾大不掉,明太祖也允許今後的皇帝在必要時可下令「削藩」。

洪武三十一年(1398年)明太祖驾崩,由於皇太子朱標於七年前因巡视陕西而病薨逝,遗诏由皇太孫朱允炆即位。改年號建文,即明惠宗(亦稱建文帝、明惠帝)。明惠宗為鞏固皇權,與親信大臣齊泰、黃子澄等密謀削藩。周王、代王、齊王、湘王等先後或被廢為庶人,或被殺。同時以邊防為名調離燕王的精兵,準備削除燕王。結果燕王朱棣在姚廣孝的建議下以「清君側,靖內難」的名義起兵,最後迂迴南下,佔領京師,是為靖難之變。朱棣即位,即明成祖,年號永樂。明惠宗在宮城大火中下落不明。明成祖對支持明惠宗者大肆殺戮,諸如黃子澄、齊泰等。

繼洪武之治,明成祖、明仁宗與明宣宗相繼興起永樂盛世與仁宣之治,這是明朝的興盛時期之一。明成祖時期武功昌盛,明成祖先是出擊安南,将安南纳入明朝版图,设立交趾布政司。明成祖之后又親自五入漠北攻打北元分裂後的韃靼與瓦剌。明成祖冊封瓦剌三王,使與韃靼對立,等到瓦剌興盛後又助韃靼討伐瓦剌,不使任何一方独大。同时,明成祖撤去大宁都司,将宁王朱权内迁南昌,授予兀良哈蒙古的朵颜、泰宁和福余三个卫所自治权,但不允许三卫蒙古人南迁到大宁地区驻牧。明成祖还于1406年和1422年对兀良哈蒙古进行镇压,以维持这一地区的稳定。明成祖為安撫東北女真各部,在歸附的海西女真(位於松花江上游)與建州女真(位於松花江、牡丹江之間)設置衛所,並派亦失哈安撫位於黑龍江下游的野人女真。1407年亦失哈在黑龙江下游东岸奴儿干地方(元朝征东元帅府旧地)設置奴兒干都司,擴大明朝東疆,亦失哈并于1413年视察库页岛,宣示明朝对此地的宗主权。明成祖一改明太祖閉關自守的外交策略,自1405年開始派宦官鄭和下西洋,向各國交往、宣示威德以及建立朝貢體制,也有為圍堵西亞帖木兒帝國的說法。鄭和下西洋前後七次,前六次均在永乐年间由明成祖派遣,郑和船队足迹遍佈東南亞與南亞地區,還於滿剌加建有基地。其規模空前,最遠到達東非索馬利亞地區,擴大明朝對南洋、西洋各國的影響力。

文治方面,明成祖修大型類書《永樂大典》,在三年時間內即告完成。《永樂大典》有22877卷,其中凡例、目錄60卷,全書分裝為11095冊,引書達七八千種,字數約有三億七千多萬, 且未有任何刪節,《永樂大典》在編成後即被深鎖皇宮數百年,以至當時有多人認為《大典》已在戰火中被毀。根據記載,明朝年間僅有明孝宗和明世宗二帝閱《大典》。此外,明成祖并未将《永乐大典》复写刊刻,且决定只制作一份抄本,並于1409年完成。1405年明成祖將北平改名北京,稱行在,並設立北平國子監等衙門。1409年,明成祖巡幸北京,在北京設立六部與都察院,並在北京為逝世的徐皇后設立陵寢,已經顯示遷都的跡象。經過十幾年的經營,北京初步得到繁榮。1416年明成祖公佈遷都的想法,得到認同,隔年開始大規模營造北京。1420年宣告完工,隔年正式永樂遷都。因為永樂年間天下大治,並且大力開拓海外交流,史稱為永樂盛世,有學者將這段時期稱為永樂盛世,亦有史學家評價成祖遷都北京之舉是“天子守國門”,或称天子戍边、天子守边。

明成祖驾崩後,其長子朱高熾即位,即明仁宗,年號洪熙。明仁宗年齡已經偏高,即位僅一年就駕崩。其統治偏向保守固本,任用「三楊」(楊士奇、楊榮、楊溥)等賢臣輔佐朝政,停止鄭和下西洋和對外戰爭以積蓄民力,鼓勵生產,寬行省獄,力行節儉。明仁宗驾崩後長子朱瞻基即位,是為明宣宗,年號宣德。他基本繼承父親的路線,實行德政治國,並且發起最後一次下西洋。明宣宗同樣熱愛美術,有畫作傳世。但是,其執政期間也並非毫無弊端。由於明宣宗喜好養蟋蟀,許多官吏因此競相拍馬,被稱為「促織天子」。同時,明宣宗打破明太祖留下的宦官不得干政的規矩,一些太監如王振等人開始干政,為明英宗時期的太監專權埋下隱患。1435年明宣宗去世,九歲的朱祁鎮繼位,即明英宗,年號正統。

明英宗自小寵信服侍左右的宦官王振,自此開始明朝的宦官嚴重專權行為。1442年限制王振權勢的張太皇太后去世,當時明英宗僅十五歲,王振更加攬權。元老重臣「三楊」死後,王振專橫跋扈,將明太祖留下的禁止宦官干政的敕命鐵牌撤下,舉朝稱其為「翁父」,明英宗對他信任有加。王振擅權七年,家產計有金銀六十餘庫,其受賄程度可想而知。

1435年蒙古西部的瓦剌逐漸強大,經常在明朝邊境一帶生事。1449年瓦剌首領也先率軍南下伐明。王振聳使明英宗領兵二十萬御駕親征。大軍離燕京後,兵士乏糧勞頓。八月初大軍才至大同。王振得報前線各路潰敗,懼不敢戰,又令返回。回師至土木堡(今日河北省張家口懷來縣),被瓦剌軍追上,士兵死傷過半,隨從大臣有五十餘人陣亡。明英宗突圍不成被俘,王振為將軍樊忠所怒殺,史稱土木堡之變,是明朝由盛轉衰的一個轉捩點。

土木堡之變的消息來到北京後,朝中混亂。一些大臣要求遷都南京應天府,被兵部侍郎于謙駁斥。同年,大臣擁戴明英宗弟朱祁鈺即位,以求長君,即明景帝(又稱明代宗),年號景泰。于謙升兵部尚書,整頓邊防積極備戰,同時決定堅守北京,隨後京師、南京、河南、山東等地勤王部隊陸續趕到。同年十月,瓦剌軍直逼北京城下,也先安置明英宗於德勝門外土關。于謙率領各路明軍奮勇抗擊,屢次大破瓦剌軍,也先率軍撤退。明朝取得北京保衛戰的勝利,于謙力排眾議,加緊鞏固國防,拒絕求和,並於次年擊退瓦剌多次侵犯。

也先認為綁架明英宗已無意義,於1450年釋放之。然而明景帝因為皇權問題,不願意接受明英宗,先是不願遣使迎駕,又把明英宗困於南宮(今南池子)軟禁,並廢皇太子朱見深(明英宗之子,後來繼位為明憲宗),立自己的兒子朱見濟為太子。不久見濟病死,沒有兒子的景帝也遲遲不肯再立朱見深為太子,儼然有奪正之貌,英宗、景帝兄弟因而嚴重對立。

1457年石亨、徐有貞等人聯盟,欲擁戴明英宗復辟。趁著景帝重病之際發動兵變。徐有貞率軍攻入紫禁城,石亨等人占領東華門,立明英宗於奉天殿,改元天順。他們禁錮了景帝,並且捕殺了于謙及大學士王文,史稱奪門之變。由於兩次即位之故,明英宗也成為明清皇帝中,唯一使用兩個年號的皇帝。明英宗復辟後,略有新政,廢除自明太祖時殘酷的殉葬制度。之後因為內部政變流放徐有貞,因為曹石之變誅殺石亨、曹吉祥等人,並且以李賢等賢臣掌政。1464年明英宗去世後,兒子朱見深即位,即明憲宗,年號成化。

明憲宗為于謙冤昭雪,恢復景帝的帝號,平反奪門一案,人多稱快。而初年勵精圖治,任用賢臣,體諒民情,蠲賦省刑,善政史不絕書,又在武功有屢有建樹,如在丁亥之役中與朝鮮進攻屢次進犯的建州女真等,儼然為一代明君,史稱成化新風,堪稱與仁宣之治媲美。但明憲宗口吃內向,因此很少廷見大臣,終日沉溺於亦妻亦母的萬貴妃,寵信宦官汪直、梁芳等人,晚年好方術。以至奸佞當權,西廠橫恣,盜竊威柄。明憲宗直接頒詔封官,是為傳奉官。這使得傳奉官氾濫,舞弊成風,直到明孝宗才全被裁撤。他也是皇莊的始置者。該舉措事實上鼓勵豪強門閥兼併土地,危害不淺。宦官汪直受到明憲宗的寵信,張狂跋扈,透過西廠大肆冤殺普通民眾與官員。不久後由於民憤四起,西廠被罷,但汪直依然握有大權。直到1482年汪直因言官彈劾才被貶。成化一朝羣小當道:女寵、外戚、佞幸、奸宦、僧道共聚一堂,朋比為奸,濁亂朝政。1487年明憲宗去世,其子朱祐樘繼位,即明孝宗,年號弘治。

明孝宗自幼於貧寒出身,曾有被萬貴妃加害的危險。其在位期間「更新庶政,言路大開」,使得自明英宗以來的陋習得以去除,被譽為「中興之令主」。明孝宗先是將明憲宗時期留下的一批奸佞冗官盡數罷去,逮捕治罪。並選賢舉能,將能臣委以重任。明孝宗勤於政事,每日兩次視朝。明孝宗對宦官嚴加節制,錦衣衛與東廠也謹慎行事,用刑寬鬆。明孝宗力行節儉,不大興土木,減免稅賦。他本身踐行一夫一妻制,一生除張皇后外沒有任何妃嬪。明孝宗的勵精圖治,使得弘治時期成為明朝中期以來形勢最好的時期,明史稱明孝宗「恭儉有制,勤政愛民」,被稱為弘治中興,然而在弘治中後期明孝宗不再認真聽從諫諍,並且開始揮霍無度,導致國家步入了「一歲所入,不足以供一歲支用」,「太倉無儲,內府殫絀」以及邊備日弛的狀況,在弘治初期革除的弊政不僅全部恢復,而且還更加惡化<,其次,有明一代,以弘治對外臣最為縱容厚待,動則大肆對外戚藩王賞賜房屋,田地,造成嚴重的土地兼併問題。

1505年明孝宗去世,其子朱厚照即位,是為明武宗,年號正德。

及至明武宗一朝,宦官势力重新抬头,其归因于武宗精于游乐,怠于政事。不过,其祸患本身并未危及皇权,虽有刘瑾、谷大用等八虎为非作歹,但始终未曾如唐朝末年的宦官擅权情况,并且刘瑾等人最终仍被武宗处以极刑。武宗的喜好游逸,最终导致孝宗一脉绝嗣。并且致使大明统系发生第二次小宗入为大宗的情况。明武宗的荒游逸樂導致正德年間戰事頻生,先後發生韃靼達延汗(明史稱韃靼小王子)進犯、寧夏安化王朱寘鐇謀反、山東劉六劉七民變、江西寧王朱宸濠謀反等重大事件。1520年明武宗假藉出征江西寧王為由而南下遊玩,以大將軍朱壽為名前往南京,親自俘虜已被王守仁擊敗的寧王。班師回京途中,於南直隸清江浦(江蘇淮安)泛舟取樂時落水染病,1521年於豹房驾崩。

明武宗驾崩后,明孝宗之侄,兴献王之子朱厚熜入嗣大统,是為明世宗,年號嘉靖。登基前后,因时任内阁首辅杨廷和、礼部尚书毛澄等权臣引宋濮安事强令明世宗尊亲生父母为皇叔父母,引起明世宗的反感,是为大礼议之争。最终明世宗在張璁等不服权臣此举的朝官支持下得以尊父母為皇帝與皇后、立太廟在明武宗之上、修皇帝實錄。這次政治風波使反對者被罷官或被入獄,受杖者一百八十餘人,杖死者十七人。在清除權臣與宦官後,明世宗開始實行自己的政治抱負,任用張璁等賢臣,英明苛察,嚴以馭官,整頓朝綱,鼓勵耕織和減輕租銀,又勘查皇室莊園和勛戚莊園,減輕土地兼併,在軍事上大力提拔將才征剿倭寇,清除外患,整頓邊防,以解除邊疆危機,史稱「嘉靖中興」。

1534年後明世宗即不視朝,但仍悉知帝国事务,事无巨细仍出于明世宗决断。明世宗信奉道教,信用方士,在宮中日夜祈禱。先是將道士邵元節入京,封為真人及禮部尚書。邵死後又大寵方士陶仲文。1542年十月,乾清宮發生楊金英、邢翠蓮等宮女十餘人與寧嬪王氏趁明世宗熟睡之際企圖將其勒死,但未成功,此即壬寅宮變。此事后,直至明世宗驾崩前一晚,明世宗迁离大内移居西内。明世宗寵信權臣嚴嵩,他借此排斥異己,結黨營私。其子嚴世蕃協助其父作惡。朝臣雖然不斷有人彈劾嚴嵩結黨營私,但均以失敗告終。世宗晚期,嚴嵩年事已高,朝臣徐階開始取代嚴嵩之位。1562年徐階策動言官彈劾首輔大臣嚴嵩。嚴嵩辭去官職回鄉。1565年嚴世蕃以通倭罪被判斬刑、嚴嵩被削為民,兩年後病死。

嘉靖一朝,國家外患不斷。北方韃靼趁明朝衰弱而佔據河套。1550年韃靼首領俺答進犯大同,宣大總兵仇鸞重金收買俺答,讓其轉向其他目標。結果俺答轉而直攻北京,在北京城郊大肆搶掠之後西去,明朝軍隊在追擊過程中戰敗,此為庚戌之變。由於世宗時期明朝宣布海禁,由日本浪人與中國海盜組成的倭寇與沿海居民合作走私,先並且後襲擾山東、浙江、福建與廣東等地區。朱紈、張經等將領受明廷干擾而未能平定倭寇。而後兵部尚書胡宗憲署理浙江巡撫兼浙直總督全力剿倭,招撫浙江勢力最強的汪直(後被明廷殺害)。戚繼光與俞大猷平定浙閩粵等地的倭寇,為後來隆慶開關建立好背景。另外葡萄牙人在1557年開始移民澳門,但及至明亡,葡萄牙人及澳门始终为广东布政司香山县管辖。1566年明世宗驾崩,皇太子朱載坖即位,即明穆宗,年號隆慶。

明穆宗即位後,先後任用徐階、高拱與張居正等名臣。1567年位處執政之首的明世宗舊臣徐階策動朝官彈劾高拱,迫高拱辭官回鄉。高拱亦不甘示弱,一年後策動朝官彈劾徐階。徐階也被迫正式退休。朝廷的實際政務漸漸落到張居正的手上。隆慶末年,高拱回朝出任內閣首輔。隆慶朝名臣名將薈萃,陸上與韃靼首領俺答汗達成和議,史稱俺答封貢;海上開放民間貿易,史稱隆慶開關;因為這兩項措施與其他改革措施,明朝開始進入中興時期,史稱隆慶新政。1572年,明穆宗因中風突然駕崩,年僅九歲的皇太子朱翊鈞繼位,即明神宗,年號萬曆。

由於明神宗年幼,於是由太后攝政。重臣高拱由於與太后信任的宦官馮保對抗而被罷官;相反的,張居正得到馮保的鼎力支持。張居正輔政十年,推行改革,在內政方面,提出「尊主權,課吏職,行賞罰,一號令」,推行考成法,裁撤政府機構中的冗官冗員,整頓郵傳和銓政。經濟上,清丈全國土地,抑制豪強地主,改革賦役制度,推行一條鞭法,減輕農民負擔。1393年明太祖時期,全國耕種田地有三百六十六萬零七千七頃,到1502年明孝宗時期也只上升到四百廿二萬八千零五十八頃。經過張居正的治理後於1581年達到七百零一萬三千九百七十六頃。軍事上,加強武備整頓,平定西南騷亂,以名將戚繼光守衛北京的重鎮薊州、以遼東李成梁安撫東北女真、以宣大王崇古、方逢時安撫韃靼,其他重臣如四川的劉顯、兩廣的殷正茂、凌雲翼、浙江的張佳胤,張居正也十分信任他們。張居正還啟用潘季馴治理黃河,變水患為水利。同時張居正嚴懲貪官污吏,裁汰冗員。張居正整頓朝正,改革體制,史稱萬曆中興。

1577年張居正父親去世,按常理他需要丁憂,但張居正以為改革事業未竟,不願丁憂。他的政敵借此大做文章,此即為奪情之爭。最後在明神宗和兩太后的力挺下張居正被奪情起復,使得其改革並未被中斷。但是,這成為他的政敵之借口。同時,張居正利用自己的職權讓自己的兒子順利通過科舉進入翰林院。除此之外,張居正的私德也有問題,各種聚斂財物的情事被揭露,張居正也迫害了大量的政敵,好同惡異,為政專擅,他一死,立刻在萬曆的支持下,被昔年結怨的大臣清算,張居正家被抄家。張府一些來不及退出的人被囚禁於內,餓死十餘口。生前官爵也被剝奪。

張居正死後,明神宗親政,勵精圖治,勤於朝政,更新庶政,繁榮經濟,廢黜考成法等張居正改革中弊政,安撫流民,減少徭稅,有勤勉明君之風範,維持了中興。然后后来发生的国本之争,拉来了明末党争的纷乱和明朝没落的序幕。國本之爭是贯穿于明神宗中期至晚期的重大政治事件。主要是圍繞著皇長子朱常洛與福王朱常洵(鄭貴妃所生)繼承皇位之爭。由於皇后无嗣,明神宗偏愛皇三子朱常洵,不願立皇長子朱常洛為太子,令群臣憂心如焚,朝中的大臣也藉此開始黨爭。直到1601年在皇太后的強迫下,朱常洛才被封為太子,而朱常洵被封為福王,封地為洛城,卻遲遲不離京就任藩王。直到梃擊案發生,輿論對鄭貴妃不利後,福王才離京就藩,太子朱常洛的地位也因而穩固。

明神宗於國本之爭對大臣極度不滿,采取以不上朝作為報復,僅偶爾批閱奏摺,以處理一些重要事件,但如明世宗一樣,悉知帝国事务,事无巨细仍出于其之决断。大理寺左評事上疏,稱明神宗沉湎於酒、色、財、氣,結果被貶為民。明神宗中後期财政困难,因此明神宗派太監為天下礦監和稅監以充實內庫,然而矿监税使大多假借名義搜刮民間財產,擾亂天下。由於明神宗不理朝政,缺官現象非常嚴重。1602年,南北兩京共缺尚書三名,侍郎十名;各地缺巡撫三名,布政使、按察使等六十六名,知府廿五名。明神宗委頓於上,百官黨爭於下,明廷完全陷入空轉之中。因此明史言:「論者謂:明之亡,實亡於神宗。」,部分史學家認為明朝自此开始走向滅亡。

由於朝政混亂,部分中下階官吏在政治上受到排斥,紛紛要求政治改革,並強調道德標準。1593年癸巳京察促成東林黨的形成,其名稱源自顧憲成重修的東林書院。主持京察的孫鑨、李世達和趙南星,利用京察將不符他们標準和不属于东林党的官吏降職解雇。經過多次京察後,引起眾多反對黨如宣黨、崑黨、齊黨、浙黨等興起並與東林黨互相傾軋。自此門戶之禍堅固而不可拔,圖使朝政空轉內耗。明熹宗時反對黨在東廠魏忠賢的羽翼下成為閹黨,開始專權,並且迫害東林黨人,東林黨受到嚴重打擊,有所謂東林六君子、東林七賢等被閹黨殺害,直到明思宗即位,才整肅了閹黨。

在對外軍事方面,以萬曆三大征最為顯著,分別為平定蒙古哱拜叛變的寧夏之役、抗擊日本豐臣政權入侵朝鮮王朝的朝鮮之役,以及平定苗疆土司楊應龍叛變的播州之役,這三場戰爭幾乎同時發生,其性質均不相同。明朝於三戰皆勝以鞏固明朝邊疆、守護朝鮮王朝,但也消耗大量人力物力,成為國庫空虛、財政拮据的重要原因之一。粗略統計出這八年間國家的軍事開支高達一千一百六十餘萬兩白銀。1617年後金努爾哈赤以「七大恨」為由反明,兩年後在薩爾滸之戰中大敗明軍,明朝至此對後金改以防禦為主的戰略。

1620年明神宗去世。其長子朱常洛登基,即明光宗,年號泰昌,在位僅一個月。他發內帑賞賜在遼東前線明軍,重用東林黨人使朝政轉危為安,並且罷除天下礦監稅使。福王生母鄭貴妃為了攏絡明光宗,獻上四位美女。明光宗縱慾過度不久病倒,太監崔文升進以瀉藥而狂瀉,又因服用李可灼的紅丸而猝死,史稱紅丸案。明光宗逝世後,其寵妃李選侍欲居乾清宮,以挾皇長子朱由校自重。都給事中楊漣、御史左光斗等,為防其干預朝事,逼迫李選侍移到仁壽殿哕鸞宮,此即移宮案。皇長子朱由校最後得以繼位,即明熹宗,年號天啟。梃擊案、紅丸案與移宮案合稱明末三大案,是萬曆晚期國本之爭的延續,使得明廷的政治鬥爭更加劇烈,也是標誌著明末衰亡的開始。

明朝末年,明朝的对外贸易陷入低谷,白银输入大量减少,由于农民缴税需要用到白银,但是一般农民只有铜钱,造成白银价格暴涨,农民无法缴税,大量逃亡,造成民变。

明熹宗在位期間,政治更加腐敗黑暗。熹宗幼年喪母,對乳母客氏有特殊感情。客氏與宦官魏忠賢狼狽為奸。熹宗早期,倚賴東林黨人力爭,方能登基,故大量啟用東林黨人,結果導致東林黨與其他黨鬥爭不斷,明熹宗因此對朝政失去耐心,魏忠賢借此機會干預政治,將齊楚浙黨的勢力集結,號為閹黨。1624年閹黨控制內閣,魏忠賢更加張狂,其爪牙遍佈中央與地方。在權勢最盛時,魏忠賢的養子竟能代替皇帝祭太廟。全國遍佈他的生祠,並號為九千歲後又稱九千九百歲。更有閹黨的國子監生提出魏忠賢配孔子,魏忠賢父配啟聖公。魏忠賢並大肆打擊東林黨,借「梃擊、紅丸、移宮」三案為由,唆使其黨羽偽造《東林黨點將錄》上報朝廷,1625年明熹宗下詔,燒燬全國書院。大量東林黨人入獄,甚至處死。由於閹黨水準低下,政理不修。國家內部饑荒頻傳,民變不斷,外患持續,明朝已經陷入風雨飄搖之境地。1626年北京西南隅的工部王恭廠火藥庫發生王恭廠大爆炸,造成2萬多人死傷。1627年明熹宗不慎落水病重,不久因霍維華之藥而去世,其五弟信王朱由檢繼位,即明思宗,年號崇禎。

明思宗即位後,銳意剷除魏忠賢的勢力以改革朝政。他下令停建生祠,逼奉聖夫人客氏移居宮外,最後押到浣衣局處死。下令魏忠賢去鳳陽守陵,魏忠賢於途中與黨羽李朝欽一起自縊,明思宗將其首級懸於河間老家,閹黨其他分子也被貶黜或處死。然而黨爭內鬥激烈,明思宗不信任百官,他剛愎自用,加強集權。當時東北方的後金(即後來的清朝)占領遼東地區,袁崇煥等人於遼西寧遠、錦州等抵禦後金可汗皇太極的入侵。1629年皇太極改採繞道長城以入侵北京,袁崇煥緊急回軍與皇太極對峙於北京廣渠門。经六部九卿会审,最後殺袁崇煥,史稱己巳之變。其後皇太極多番遠征蒙古,終於在六年後徹底擊敗林丹汗,取得了傳國玉璽,1636年在盛京稱帝,改國號為大清,即清朝。並且陸續發起五次經長城入侵明朝直隸、山東等地區,史稱清兵入塞。當時直隸連年災荒疫疾,民不聊生。遼西局勢亦日益惡化,清軍多次與明軍作戰,最後於1640年占領錦州等地,明軍主力洪承疇等人投降,明朝勢力退縮至山海關。

明朝中期之後時常發生農民起事,崇禎雖勵精圖治,但其任人不得法(崇禎一朝撤換過五十個大學士,號稱「崇禎五十相」,為歷朝之最),朝政混亂與官員貪污昏庸;與後金的戰爭帶來大量遼餉的需求以及清兵的掠奪;以及因為小冰期氣候變冷,在當時連海南島都出現下雪氣候,農業減產帶來全國性饑荒,這些都加重明朝百姓的負擔。1627年,陝西澄城饑民暴動,拉開明末民變的序幕,隨後王自用、高迎祥、李自成、張獻忠等農民起事,最後發展成雄踞陝西、河南的李自成與先後占領湖廣、四川的張獻忠(最後成立大西政權)。1644年李自成建國大順,三月,李自成率軍北伐攻陷大同、宣府、居庸關,最後於1644年4月24日(舊曆三月十八)攻克北京。明思宗在煤山自縊,史稱甲申之變。後世有史學家評價思宗在社稷危難之時沒有逃跑是“君主死社稷”,但亦有學者指出崇祯多次迁都南京的計劃。

李自成攻克北京後,縱容部將在京城內大肆搜刮遂失民心。原為明將、鎮守山海關的吳三桂帶領清軍入關,並於一片石戰役擊敗大順軍。清朝攝政王多爾袞與順治帝入關,北京成為清朝的首都。李自成退回陝西,最後被清軍圍殲於湖北,大順亡。

甲申之變後,明朝在南方尚有勢力,史稱南明。南明主要勢力有四系王,分別是福王弘光帝朱由崧、魯王監國朱以海、唐王隆武帝朱聿鍵與紹武帝朱聿𨮁、桂王永曆帝朱由榔等。當南明滅亡後,又有鄭成功建立的明鄭與夔東十三家軍抗清。1644年北京被李自成攻陷後,南明大臣意圖擁護皇族北伐。經過多次討論後由鳳陽總督馬士英與江北四鎮高傑、黃得功、劉澤清與劉良佐擁護明思宗的堂兄弟福王朱由崧稱帝,即弘光帝,史稱南明。1645年清朝派多鐸率大軍南下南京,此時弘光帝昏庸,大權由閹黨餘孽掌握,江北四鎮各自為營,最後陸續瓦解。清軍攻破史可法死守的揚州,弘光帝逃至蕪湖被逮,送到北京殺害。此期間清軍發起揚州十日、江陰八十一日與嘉定三屠等大屠殺以鎮壓反抗的漢人。同时明朝数十万皇族也惨遭清廷和農民軍的屠杀。

弘光帝死後,魯王朱以海於浙江紹興監國;而唐王朱聿鍵在鄭芝龍等人的擁立下,於福建福州稱帝,即隆武帝。然而這兩個南明主要勢力互不承認彼此地位而互相攻打。1651年在舟山群島淪陷後,魯王朱以海在張名振、張煌言陪同下,赴廈門依靠鄭成功,不久病死在金門。隆武帝屢議出師北伐,然而得不到鄭芝龍的支持而終無所成。1646年,清軍分別占領浙江與福建,魯王朱以海逃亡海上,隆武帝於汀州逃往江西時被俘而死。鄭芝龍向清軍投降,由於其子鄭成功起兵反清而被清廷囚禁。朱聿鍵死後,其弟朱聿𨮁在廣州受蘇觀生及廣東布政司顧元鏡擁立稱帝,即紹武帝,於同年年底被清將李成棟攻滅。同時間桂王朱由榔於廣東肇慶稱帝,即永曆帝。

1646年永曆帝獲得瞿式耜、張獻忠餘部李定國、孫可望等勢力以及福建鄭成功勢力的支援之下展開反攻。同時各地降清的原明軍將領先後反正,例如1648年江西金聲桓、廣東李成棟、廣西耿獻忠與楊有光率部反正,一時之間南明收服華南各省。然而於同年,清將尚可喜率軍再度入侵,先後占領湖南、廣東等地。兩年後,李定國、孫可望與鄭成功發動第二次反攻,其中鄭成功一度包圍南京。然而,各路明軍因為距離互相難以照應,內部又發生孫可望等人的叛變,第二次反攻以節節敗退告終。1661年,清軍三路攻入云南,永曆帝流亡缅甸首都曼德勒,被缅甸王莽達收留。後吴三桂攻入缅甸,莽達之弟莽白乘机发动政变,杀死其兄後继8月12日,莽白發動咒水之难,杀盡永曆帝侍從近衛,永曆帝最後被吴三桂以弓弦絞死,南明亡。

此時反清勢力只剩夔東十三家軍與在金廈的鄭成功(史稱明鄭)。李自成余部在湖南抗清失敗後,轉移到川、鄂山區進行活動,在夔州府以東地區繼續抗清,稱為夔東十三家軍。1662年清軍開始攻打之,到1664年首領李來亨被殺而亡。鄭成功在南京之戰失敗後退回金廈,於1661年率軍遠征荷蘭人占据的台灣岛成功,明鄭領有台灣,定都東寧(今台灣台南)。其子鄭經曾參與三藩之亂,率軍参与反攻失利。1683年,清朝康熙帝命施琅為水師提督進攻台灣。明鄭主鄭克塽率眾投降,明鄭亡。

明初武功强盛,多次對北元和隨後的韃靼和瓦剌作戰,並在與漠南一帶設置四十餘個衛所防衛,包括東勝衛、雲川衛、官山衛、全寧衛、老哈河衛等,這些都是明廷的邊防重地。其走向大致為陰山-大青山南麓-西拉木倫河一線。15世紀30年代後,由於天氣轉寒,農耕不濟,靖難之役時邊塞軍隊被燕王抽調。因此期間邊境略有南移。在明成祖永樂年間,明軍多次北伐,邊境形勢一度改觀。但在明中葉以後,隨著蒙古的再次崛起,邊境再次南移。並修建長城以防禦蒙古,在長城沿線設置九邊重鎮加強防禦。長城也成為明中後期的北邊,同時也是農耕區與遊牧區的界線。

明太祖設置遼東都司以經營遼東。並多次進軍黑龍江流域,招撫當地土著部落,明廷勢力一度達到外興安嶺與黑龍江口,甚至庫頁島。明成祖永樂七年(1409年)於黑龍江地區設置奴兒干都司,然此都司並非常設機構,與東北130多個衛所不相轄屬,明宣宗宣德九年(1434年)廢棄之,撤回在奴儿干的流官驻军,不过之后女真仍奉明朝为主,原設於此處的各衛所及遼東都司仍然存在,至万历年间卫所增加至384个,以對當地實行羈縻統治。明英宗正統年間後,韃靼兀良哈與建州女真部南遷,並不斷侵犯遼東都司。明憲宗成化五年(1469年),明廷修建遼東邊牆。16世紀末開始,建州女真酋長努爾哈赤開始興起,統一女真部,明廷設置的衛所逐漸消亡。明神宗萬曆四十四年(1616年)努爾哈赤稱汗,建國後金。明神宗萬曆四十七年(1619年)薩爾滸之戰後,後金軍隊破遼東邊牆,佔領遼東都司大部土地。

明成祖永乐年間,西北疆界達到今新疆東部哈密地區,並設置沙州、安定、阿端、曲先、赤斤蒙古、罕东左一系列衛所。15世紀30年代之後,西北吐魯番與青海蒙古部日益強大。1472年,哈密衛城一度被吐魯番攻破,衛內遷,後復,1514年再度被並。16世紀後半期後,西北諸衛全部喪失,明軍退守嘉峪關。

明朝在1381年才將云贵地區完全劃入疆域,並設置一系列土司、宣慰司管轄之,除正式府州外另设有三宣六慰,永乐年间増设底兀刺、大古刺、底马撒三个宣慰司。邊界達到緬甸中北部、老撾北部、泰國北部一線。但後期這些地區多被周邊國家所並。

明成祖永樂四年(1406年)明軍進攻安南,南線達到日南州一帶。次年設置安南布政使司,下設十五府、卅六州、兩百餘縣。後因當地人民反抗激烈,明廷於明宣宗宣德二年(1427年)放棄,安南恢復黎氏王朝。

明初吐蕃宣慰使何锁南普等率吐蕃诸部归降,后明廷在青藏高原地区设乌思藏、朵甘卫指挥使司,采取广行招谕、多封众建、因俗以治的治藏政策。在完成藏区的统一后,明太祖要求藏民输马作赋、承担徭役,或蒸造乌茶、输纳租米,强调“民之有庸,土之有赋,必不可少”。永乐五年(1407年),明成祖派遣刘昭、何铭等人前往藏区设置驿站,永乐十二年(1414年),又遣中官杨三宝往藏区招谕各土官恢复驿站,经多年努力终使往来西番的驿道安全畅通。万历以后,明朝对边疆控制日益松弛,蒙古人攻占了整个青海草原,朵甘都司遂废弃。

1553年葡萄牙人獲得在澳門停泊船隻權,1557年取得居留權,在清光緒十三年(1887年)中葡签署《中葡和好通商条约》前,中国法律上一直拥有澳门主权。

1624年荷兰人进入台湾南部,筑热兰遮城。1626年西班牙人进入台湾北部。1642年荷兰赶走西班牙,占领台湾大部。1661年,郑成功进攻台湾,次年驱逐荷兰人,攻占台湾。

明成祖永乐年间,积极开展对外联系,特别是派遣郑和七下西洋,并积极对南海诸岛进行勘察和经营。多次往返南海诸岛的航行中而次次必登、必书南海诸岛。《郑和航海图》以“石塘”、“石星石塘”、“万生石塘屿”为今之西沙、东沙、中沙和南沙群岛之名。

永乐四年(1406年),郑和船队剿灭盘踞在旧港(今印尼巴邻旁)的海盗陈祖义,在其地设立旧港宣慰司,首任宣慰使施进卿即由郑和亲自前往册封。旧港宣慰司是为明朝最南方疆土,以控制南洋核心要冲地带,也确保了明朝在南洋的权威,令海外贸易大兴,还开启了华人大规模开发南洋的时代。

明朝大致上繼承元朝行政區劃,其一級地方行政區分置承宣布政使司(布政司)、提刑按察使司(按察司)與都指挥使司(都司)的都布按三司制度,分別掌管行政、司法與軍事等三種治權,防止地方權力集中。布政司通稱省,底下依序有道、府州與縣。道是明朝特別設置介於省和府州之間的行政單位,分為分守道和分巡道兩種,分守道为布政司的派出机构,负责监督协调府州行政,分巡道为按察司的派出机构,负责监督协调府州司法治安。府为明朝最主要的统县政区,原為元朝的路,以稅糧多寡為劃分標準,糧廿萬石以上為上府,廿萬以下十萬以上為中府,十萬以下為下府。州与府同样是统县政区,但人口税收比府少,地位也比府低。州按照其行政隶属分为两类,直辖于布政司的州称直隶州,隶属于府的称散州或属州。軍事區劃有衛、所兩级,但部分位于少数民族聚居区或边疆军屯区的卫所具有类似内地州县的行政职能,行政上分别相当于府与县。明代宗、明英宗時設有中央派出管理行政的巡撫與管理軍事的總督,地位在布政司與都司之上。為限制巡撫與總督的權力,又設有都御史制衡之。明朝最後有140府,193州,1138縣,493衛,359所。

承宣布政使司(布政司)主管地方行政,地位等同元朝的行中書省。明太祖原沿襲行中書省的稱呼,1376年時改為布政使司,通稱行省。明初設有十三個布政司與京師(非城市,地位等同布政司,轄現今江蘇與安徽兩省)。1380年胡惟庸案後撤廢中書省,京師及布政司直屬於六部之下。明成祖時期,於1407年到1428年間設置交阯布政司。於1413年設貴州布政司。為遷都北京,1403年將北平布政司升格為行在,1421年遷都北京後稱為京師(北直隸),原京師改稱南京(南直隸),形成「兩京十三省」的行政區劃。两京為明朝首都北京與南京的正式稱呼順天府與應天府,其与其周边州府分别合称北直隶与南直隶,不设布政司,十三布政司为陝西、山西、山東、河南、浙江、江西、湖广、四川、廣東、福建、廣西、貴州、雲南。明朝行政區劃設置大體符合山川形便之處,但仍有一些不合理之處,如南直隸地跨淮北、淮南、江南三個地區,语言文化上属于太湖吴越区的苏松地区归入南直而非传统上的浙江,秦岭以南的汉中等地归入陕西而非传统上的四川,河南也佔據局部的黃河以北土地。貴州省呈現中間窄兩邊寬的蝴蝶狀。

都指挥使司(都司)主管地方軍事,明太祖採用衛所制,於1370年於各省設置一都衛,1375年才設置都司管理。都司原隸屬大都督府,於胡惟庸案後析大都督府為五,分統諸軍司衛所。明朝一共設置十六個都司、五個行都司與兩個留守司。其中十三個是與布政使司同名的的都司,其他三個是萬全都司、大寧都司和遼東都司。五行都司是陝西(治甘州衛,今張掖)、四川(治建昌衛,今西昌)、湖廣(治鄖陽衛,今湖北鄖縣)、福建(治建寧府,今建甌市)、山西(治大同府)。兩留守司是洪武年間設置的中都留守司(今鳳陽)和嘉靖年間置於承天府(今湖北鍾祥)的興都留守司。屬羈縻性質的都司中,最有名的有統轄黑龍江、松花江流域和庫頁島的奴兒干都司,在政教合一的青藏地區設置有烏斯藏、朵甘二都司(但这是否代表當時的西藏受到了明朝的统治存在较大的争议,請參詳明朝治藏歷史),另有置於今甘肅、青海交界地區的哈密、曲先等衛。這些具羈縻性質的行政區劃與內地的都司、行都司性質不同。

巡撫主理民政,原本是明宣宗時期派六部、都察院大臣以此為名義督撫地方行政,到明代宗時正式形成一級行政區。總督於明英宗時設置,分短期與長期兩種,管轄數個布政司的軍務。而巡撫與布政司的轄屬關係不一,有的巡撫轄有有一個到兩個布政司,如正統年間的山西河南巡撫。有的是一個布政司上面有數個巡撫,如北直隸有順天巡撫(駐遵化)、保定巡撫(駐真定,今河北正定)、宣府巡撫(駐宣府鎮,今河北宣化,一度兼領山西大同府)三巡撫;南直隸有兩巡撫:應天巡撫(駐蘇州府,今江蘇蘇州)、鳳陽巡撫(駐淮安府,今江蘇淮安市淮安區)。有的巡撫管轄布政司與布政司之間的交界處,如南贛韶汀巡撫就跨越江西、廣東、福建三個布政司。

洪武十三年(1380年),明太祖以丞相胡惟庸謀反伏誅,於是廢去中書省和丞相一職。秦、漢以降實行一千六百餘年的宰相制度自此廢除,六部直接向皇帝負責,相權與君權合而為一,大權獨攬,施行軍權、行政權、監察權三權分立的國家體制。由於國家事務繁多,皇帝無法處理,而明太祖也一度深感疲憊,於是設立四輔制度來輔佐政事。但這項制度效能不彰。洪武十七年(1384年)後被廢。之後朱元璋請來幾位翰林學士幫忙輔佐,這些翰林學士的官職效仿唐宋馆阁学士旧制,被命為「某某殿(阁)大學士」,官階只有正五品。明成祖登基后,特派解缙、胡广、杨荣等入午门值文渊阁,参预机务,由此始設內閣。

內閣最初只是皇帝的諮詢機構,相當於今日秘書或幕僚的職務,奏章的批答為皇帝的專責。到後來成為明朝實際上最高決策機構,首輔地位有時可比丞相,有票擬之權明朝內閣由始至終都不是明朝中樞的一級行政機構,所謂內閣只是文淵閣的別稱。內閣大學士一職多以碩德宿儒或朝中大臣擔任,只照皇帝的意旨寫出,稱「傳旨當筆」,權力及地位遠遠不及過去的宰相,只有有实无名之地位,而沒有法定地位。宣宗時期,由於楊溥、楊士奇、楊榮等三楊入閣,宣宗批准內閣在奏章上以條旨陳述己見,稱為「票擬」制度,又授予宦官機構司禮監「批紅」。票擬之法補救可君主不願面見閣臣之弊,但內閣大臣與皇帝溝通,全賴司禮監(宦官)。由是開啟明朝宦官專政之大門。

明朝在中央設置吏、戶、禮、工、刑、兵六部,與前代相比,明朝最初在每部增加尚書、侍郎各一。胡惟庸案之後,朱元璋廢丞相之職,取消中書省。六部因此地位得到提高。每部只設一個尚書,兩個侍郎,原有的各科尚書降為郎中。各部尚書和侍郎的官階也上升。其中以禮部(主管教育,負責領導儒家學術,以及祭祀,外交等)和吏部(主管文官陞遷)最為重要,戶部(主管財政,土地和人口)人員最多。兵部(主管國防),刑部(主管司法,有對較大刑事案件的審判權)與工部(主管公共建設)地位較低。

在拟诏审议机构上,明朝開始只設給事中与中书舍人,不复设中书门下二省。明朝的审议机构为六科给事中,到洪武廿四年,設都給事中六人,分吏、戶、禮、工、刑、兵六科,每科一人,每科都给事中下设左右给事中各一人及给事中若干。六科给事中制度基本是繼承唐朝的門下省制度,但官位下降,机构更为精简,也失去了自魏晋以来皇帝内臣(皇室的收发站)和礼官的职责。六科官職品級雖低,然職權很高,他們可以批驳皇帝的意旨, 也能充当谏官的职责,对六部吏僚则具有分科对应的监察权,故該制度也發揮一定的改善朝政作用。明朝的拟诏机构为中书舍人官署,因其制度源流源于与门下并立的中书,故与六科相对俗称 「中书科」,但是其地位大为下降,职能也大幅削弱,事实上只是内阁与翰林院的誊抄机构。中央的重要事务执行机构为五寺,包括大理寺、太常寺、光祿寺、太僕寺、鴻臚寺,与唐宋相比,减省了四寺:宗正寺被并入宗人府,卫尉寺被并入兵部,司农寺与太府寺被并入户部。大理寺與刑部和都察院合為三法司,负责重大刑事案件的复审与复核。大理寺的首長稱為大理寺卿,也是九卿之一。其餘四個寺的卿職權較低。太常寺負責祭祀;太僕寺管理馬匹与全国牧政;光祿寺負責壽宴;鴻臚寺負責接待外賓。

在洪武十三年前,明朝還沿襲元的監察制度,設立御史台,有左右御史大夫各一名。洪武十三年後,朱元璋廢御史台。兩年之後,朱元璋設立新的監察機構—都察院。都察院下面設立監察御史若干人,分巡全國各省,稱為十二道監察御史。每道有監察御史三至五人,範圍大體為一省。但監察御史都駐在京師,有事帶印出巡,事畢回京繳印。到明末,監察御史分為十三道,共有一百一十人。都察院与六科同样具有谏官的职能和风闻言事的职责,故合称「科道言官」。

明初还實行特務機構,主要包括錦衣衛、東廠和西廠,武宗時期還一度設有內行廠。錦衣衛設立於洪武十五年,直接聽命於皇上,可以逮捕任何人,並進行不公開的審訊。但是朱元璋晚年逐步废除了锦衣卫及其特权,还有一些比较残酷的刑法。

在東廠設立後,錦衣衛權力受到削弱。東廠成立於永樂十八年,是明成祖為鎮壓政治上的反對力量而成立。地點位於京師東安門北。東廠的主要職責就是監視政府官員、社會名流、學者等各種政治力量,並有權將監視結果直接向皇帝匯報。依據監視得到的情報,對於那些地位較低的政治反對派,東廠可以直接逮捕、審訊;而對於擔任政府高級官員或者有皇室貴族身份的反對派,東廠在得到皇帝的授權後也能夠對其執行逮捕、審訊。東廠在設立之初,就由宦官擔任提督,後來通常以司禮監秉筆太監中位居第二、第三者擔任。西廠設立於憲宗時期,首領為汪直。1482年後被廢。其後又被武宗短暫恢復。內廠設置於武宗時期,首領為宦官劉瑾,劉瑾伏誅後,內廠與西廠同時被廢除,僅留東廠。

公孤官包括三公与三孤,是名义上的诸臣之首,但這些官職都是虛銜,一般授予功勞相當大的大臣以示榮耀。三公为太师、太傅、太保,三孤则是辅弼他们的少师、少傅、少保。其中太保和太傅名義上是太子的老師,而太師則是皇帝名義上的老師,但實際上輔導太子的機構是詹事府。詹事府下設兩坊、一局、一廳。此外還有太醫院,專門負責皇室人員的健康和醫療。太醫院附屬有生藥庫和惠民藥局。翰林院作為政府的官方學術最高機構,地位相當重要,甚至在政府中都有相當大的影響力。翰林院首長是翰林大學士,此職位者經常會同時兼任內閣大臣。

诸司指不屬於各部院的司。主要指通政司和行人司。通政司負責傳遞公文,公告周知。行人司負責到地方上頒詔諭及赴外國作使臣。

外三监包括國子監、欽天監、上林苑監。欽天監負責觀測星象。國子監是最高官方教育機構,也是全国官学的领导机构,有祭酒一人,司業一人,監丞一人,博士五人,助教十五人,學正十人,學錄七人,典簿一人,典籍一人,典饌兩人。上林苑監負責掌管皇帝的御花園,畜牧場與菜圃。

内十二监為宦官衙門。事實上只有在這些衙門工作的宦官才是太監。包括司禮監、內宮監、御用監、司設監、御馬監、神宮監、尚膳監、尚寶監、印綬監、直殿監、尚衣監、都知監。以司禮監最為重要,監內的提督太監主管宮內一切宦官禮儀刑名。而秉筆太監在宦官極端專權時竟代替皇帝批公文。此外宫内還設有四個司(惜薪、鐘鼓、寶鈔、混堂),八個局(兵仗、銀作、浣衣、巾帽、針工、內織染、酒醋面,司苑),合為內官廿四衙門。宮女也有六個局(尚宮、尚儀、尚服、尚食、尚寢、尚工),每個局下設四個司。

《大明律》,是明朝法令条例,由朱元璋总结历代法律施行的经验和教训制定而成,《大明律》为适应形势的发展,变通了体例,调整了刑名,肯定了明初人身地位的变化,注重了经济立法,在体例上表现了各部门法的相对独立性,并扩大了民法的范围,同时在“礼”与“法”的结合。

《大明律》共分30卷,篇目有名例一卷,包括五刑(笞、杖、徒、流、死)、十恶(谋反、谋大逆、谋叛、恶逆、不道、大不敬、不孝、不睦、不义、内乱)、八议(议亲、议故、议功、议贤、议能、议勤、议贵、议宾),以及吏律二卷、户律七卷、礼律二卷、兵律五卷、刑律十一卷、工律二卷,共460条。

有明一代比较重视法制的建设与实践,其中历经三次大规模修订的《大明律》。《大明律》在中国古代法典编纂史上具有革故鼎新的意义。不仅继承了明代以前的中国古代法律制定的优良传统,也是中国明代以前各个朝代法典文献编纂的历史总结,而且还开启了清代乃至近代中国立法活动的发展。《大明律》在明代实施的过程中,虽然也不断受到“朕言即法”的干扰,但这些干扰始终未能影响它的正统法典的地位。

而《大明律》对惩治贪财枉法者,严厉程度超过了历史上任何一个朝代。

明代早期軍隊的來源,有諸將原有之兵,即所謂從征,有元兵及群雄兵歸附的,有獲罪而謫發的,而最主要的來源則是籍選,亦即垛集軍,是由戶籍中抽丁而來。除此之外尚有簡拔、投充及收集等方式。此外,明朝中期以後又有強使民為軍的方式,不過都屬於少數,整體而言,衛所制仍然是最主要的軍制。衛所制為在全國各地軍事要地設立衛所駐軍,衛有軍隊五千六百人,其下依序有千戶所、百戶所、總旗及小旗等單位,各衛所都隸屬於五軍都督府,亦隸屬於兵部,有事從徵調發,無事則還歸衛所。軍隊來源為世襲的軍戶,由每戶派一人為正丁至衛所當兵,軍人在衛所中輪流戊守以及屯田,屯田所得以供給軍隊及將官等所需。其目標在養兵而不耗國家財力,但明宣宗以後漸無法維持,軍人生活水準及社會地位日漸低下,逃兵也逐漸增加,軍備因此逐漸廢馳。

在嘉靖年間,應付倭寇之亂時,將領戚繼光在浙江地區採用招募民兵加以訓練的方式,來取代不堪的衛所兵。正因為明朝正規軍衛所軍的不堪用,故這些民兵,在明朝後期逐漸擔負起維持明朝有效統治的作戰部隊,而其中最為有名的就是戚繼光的召募以浙江人為主戚家軍,李如松的私人部隊遼東鐵騎,及袁崇煥所召募以遼東人為主的關寧鐵騎。

发端于唐宋时期的中国火器制造技术,在明朝发展到了很高的水平。这时的火器不仅仅种类多,而且制造技术以及性能均有极大提高。火箭与鸟枪是明朝军队的主要轻型火器,地雷在明朝也很盛行,管形火器的发展尤为显著。明朝中后期,随着经济的发展、科技的进步以及国防需要的强化,火器技术得到迅速发展。火器技术的勃兴引发了一场火药时代的军事变革。佛郎机以及红夷大炮等西洋火器在此时期传入,使得明朝得以汲取其瞄准器的长处,以改良自产的火器性能。当时中国的冷兵器时代即将终结,火器时代正在来到,亦認為中国有机会赶上西方的火器技术水平,但这一过程却随着明朝的灭亡而中断。

學者梁柏力指出,中國雖然比西方早兩個世紀使用熱兵器,但到了15世紀技術開始被葡萄牙人超越,但是差距还不是很大,後來清軍利用了明朝和西方的技術和經驗,多次改良並製造出比明朝更有威力的火器,到了三藩之亂期間,中國的熱兵器技術回升接近西歐國家的水平,这也是郑成功能驱逐台湾荷兰人,以及清康熙時期的清軍能夠擊退入侵黑龍江的俄羅斯士兵的原因之一。在簽署尼布楚條約其後的150多年內,清朝境內大致升平,直到鸦片战争前夕,还停留在三藩之乱的技術水平上。可見中國火器的技術發展,與國內是否長時期出現大規模軍事對峙的局面有關。

在萬曆年間,日本人亦在火器技術上領先中國,以致日本火器的優勢在萬曆援朝戰爭中一度令日軍佔於上風。明朝軍事家戚繼光亦批評當時多種形式的火器實際上並不實用,故一切禁之,以節靡費。亦有學者批評宋元明清年代在政權穩定期間往往封鎖火器的研究成果,並且對研制者新的發明創造也不予以重視,甚至棄置不用,如明朝的趙士楨、畢懋康、薄玉和清朝的戴梓,他們的貢獻和成果都沒被恰當重視。

16、17世紀間,明代曾是世界上手工業與經濟最繁榮的國家之一。明代初期推行的海禁政策,使得商業受到一定的壓制,但明穆宗隆慶元年(1567年)廢除海禁後,海外貿易重新活躍起來,全盛時遠洋船舶噸位高達18000噸,占當時世界總量的18\%,推进了中国与国际市场的联系,促使晚明中国白银货币化的最终完成。明代手工业和商品经济繁荣,出现商业集镇,中国大陆学界认为出现「资本主义萌芽」,此说法仍然处于争议之中。

明朝初期,由於多年的戰爭加上通貨膨脹,且前朝元惠宗為治水加重徭役,經濟近乎在崩潰的邊緣。明太祖洪武年間實行休養生息的政策與移民墾荒,也實行屯田政策,軍屯面積佔全國耕地的近十分之一。此外,商屯也相當盛行,政府以買賣食鹽的專賣證(稱之為鹽引)作為交換,利用商人將糧食運往邊疆,以確保邊防的糧食需求,然而此方式並非以以物易物方式,而是要求鹽商先交錢再等曬鹽季再給鹽,卻又為稅收不足而將新產出的鹽另行外賣,延後交鹽給正規鹽商的時間,致使鹽商交了錢卻要三五年甚至十年後才拿得到鹽,卻又因身份管制而無法拋棄鹽商身份另行謀生,因此而家破人亡,私鹽亦大為流行。

明朝農業無論是產量還是生產工具,都高於前一朝代,番薯、南瓜、蠶豆、土豆、玉米、棉花等美洲高產作物在16世紀中葉時陸續傳到中國,尤其是棉花,已在全國普遍栽種。此外,較容易栽種的蕃薯和玉米,可以種植於土壤相對較貧瘠的地區,對於糧食需求日增的明清兩代尤其重要。

萬曆年間,耕地總面積超過七百萬頃,為明神宗萬曆年間開始的人口穩步增長提供堅實的基礎。而在南宋時流行的俗諺「蘇常熟,天下足」,由於長江下游地區城市居民的快速增加,及長江中游地區的快速開發,中晚明時,已經轉變為「湖廣熟,天下足」,意即當時主要的米糧生產區已經轉移到湖廣地區,也就是現在的湖北省和湖南省一帶。

隨着商業性農業的出現而發展起来的長途交通,有利於工商業的發展。晚明以後,湖廣的米開始被長途運送至江浙、閩廣等地區販售,使當地農民開始改種經濟作物。明太祖也曾派遣國子監下鄉督導水利建設,並以減免稅賦獎勵耕作。這些措施使得過去很多飽受戰亂損毀的地區恢復生氣,使明朝的經濟得到快速的恢復。

明朝无论是铁、造船、建筑,还是丝绸、纺织、瓷器、印刷等方面,在世界都是遥遥领先,产量占全世界的2/3以上,比农业产量在全世界的比例还要高得多。明朝民间的手工业不断壮大,而官营却不断萎缩,明朝后期,除了盐业等少数几个行业还在实行以商人为主体的盐引制外,一些手工业都摆脱了官府的控制,成为民间手工业。

自明初年起,以江南地區為代表的手工業高度發展,松江潞安府全盛時有織機一萬三千張,促進市場經濟化和城市化,南京、臨清等城市「周圍逾三十里,而一城之中,無論南北財貨,即紳士商民近百萬口」。南京一地有眾多的陶瓷廠,每年可生產100萬件瓷器。景德鎮成為世界瓷都。制瓷使用旋坯车,不但提高生产效率,還使旋出的瓷坯更为精细和规格化。施釉方式以吹釉法代替刷釉法,使施釉更加均匀光泽。並且發展出彩色瓷器。冶铁技术也有明显的提高,由灌钢冶炼法发展到苏钢冶炼法,是一种效率较高的炼钢方法。

明初期奉行「重本抑末」政策,甚至規定禁止商賈之家穿綢紗。明穆宗隆慶三年(1569年),大學士高拱上疏《議處商人錢法以蘇京邑民困疏》,反映商人的愁苦和商業的窘困,並奏請隆慶帝採取措施,革除宿弊。之後張居正提出農商榮枯相因,進一步肯定商人的作用。明代中後期商人地位有所提高,部分士大夫認為經商有成,在價值上也等同於讀書有得,「亦賈亦儒」「棄儒就賈」的現象也開始出現。此外,商業用的書也開始出現。商人為實用目的而編寫此類書籍,內容介紹貿易路徑沿途的交通、習俗及商品行情等。此類書籍現存最早者為《一統路程圖記》。此外,由於商業的發達,各地紛紛開始大量生產具有當地特色的商品,運銷他處,使得區域分工日益明顯。

明朝初期,明太祖洪武年間尝试使用「大明寶鈔」的紙幣,这种货币同样经历了迅速的通货膨胀,它在1450年暂停发行,但是直到1573年仍在流通。直到明朝晚期李自成威胁北京时,这种纸币才在1643年和1644年重新印刷。在明朝大部分时期,中国有一个包括所有重要交易的纯私人货币体系。而整個貨幣體系轉向為以銀本位為主。从海外流入的白银, 开始在南部省广东作为货币使用,并在1423年传到长江下游地区成为纳税的法定货币。各省税收自1465年起以白银的形式上交首都,灶户从1475年起开始使用白银支付,徭役豁免费从1485年起使用白银支付。中国对白银的需求部分通过西班牙人从美洲的进口得到满足,特别是秘鲁的波托西和墨西哥,在西班牙人1571年建立马尼拉之后。但这时的白银还没有被铸造。它们以重量为一个标准两(约36克)的银锭(被称为元宝)流通,尽管其纯度和重量在地区与地区间略有不同。

16世紀中葉之後日本和拉丁美洲的白銀大量流入也進一步促進中晚明經濟的發展,當時明朝佔有世界白銀需求量三成左右。明代經濟的另一個特色是城鎮經濟的繁榮,運河沿線由於往來商船不斷,周邊城市如濟寧、淮安、揚州等都非常發達。東南地區由於商品經濟繁榮,成為全國的經濟集散地。由於商品經濟的繁榮,明代形成按籍貫區分的商人集團,稱為「商幫」,如徽州商幫、晉陝商幫、廣東商幫、福建商幫、蘇州洞庭商幫、江西商幫等。這些商幫以「會館」為聯繫場所,互相支持,越做越大。

明嘉靖、萬曆間,各地出賣絲綢、酒肉、蔬果、煙草、農作物、瓷器等商品不計其數,大量外銷賺取外匯所得;外國的不少東西在中國城市都有賣,如歐洲的西洋鐘、美洲的煙草,當時商業大都會以江南的商業城市最多,有南京、儀征、揚州、瓜洲、蘇州、松江、杭州與嘉兴等,華中其他商業城市尚有汉口、南昌、淮安、蕪湖與景德鎮等,西南内陆有成都,華北有北京、濟寧與臨清等,而華南則有福州與廣州等。

晚明至清朝这一时期,明朝生產總量所占的世界比例在中国三千年历史上也是最高的。据英国经济学家安格斯·麦迪森的研究,1600年明朝生產總量为960亿美元,占世界经济总量的29.2\%,晚明中国人均GDP在600美元。事实上据他研究,1500年中国生產總量(618亿美元)已首次超过印度(605亿美元)。这里仅表明购买力平价,与所谓财政收入 (Government revenue) 是不同的概念,大多数中国学者如刘逖认为麦迪森已经高估了中国历史上的生產總量和人均生產總量。据刘逖指出国际公认的生存水平线是400美元,因此刘逖对麦迪森明朝数据做了调整,认为若以1990年的美元價值換算,1600年中国人均GDP在388美元、1610年在386美元、1620年在391美元、1630年在344美元、1640年在367美元,而非麦迪森说的一直维持在600美元。

而隆庆年间的开关,进一步促进了当时中国经济的增长。

元惠宗至正年間(1341年-1370年)全國發生多次大規模的災荒饑饉疾病和瘟疫,並最終促使紅巾軍起義爆發,期間造成人口大量減少。大明建立並統一全國後,明太祖實行休養生息政策,全國的農業生產在蒙元时代長期大規模战争而遭受極大破壞的背景下得到很大程度的恢復,加上洪武年間大規模向淮河以北和四川的荒無之地、墾荒填充移民,使人口得以穩定增長。到明太祖洪武廿六年(1393年)全國有6500萬人,其中民戶佔6175萬人,軍戶佔325萬人。北五省(北平、山西,山東、河南、陝西)人口有1755萬人,佔全國27\%,其中山東最多,有5,462,850人,以下依次為山西(3,790,760人)、河南(2,825,300人)、陝西(2,646,450人)、北平(2,619,500人)。中五省(京師、浙江、江西、湖廣、四川)人口總數為3380萬,佔全國52\%。其中,南直隸有11,291,460人;人口密度最高的蘇南太湖流域人口達6,320,300人,平均每平方公里220人;其次為浙江省,有9,959,270人;江西有7,260,000人,湖廣有4,318,420人,四川最少,僅1,314,260人。南五省(福建、廣東、廣西、雲南、貴州)總人口有1040萬人,佔全國的16\%。

明朝户口的峰值出現在明朝后期,但對於具体时间與人口數量,不同學者有不同說法。易中天认为,明末人口六千余万;赵文林、谢淑君认为1626年明朝达到人口峰值,实际人口大约有99,873,000人;王育民认为万历年间明朝人口达到峰值,实际人口在130,000,000人至150,000,000人之间;葛剑雄认为1600年明朝实际人口大约有197,000,000人,明朝人口峰值接近2亿;曹树基认为1630年明朝达到人口峰值,实际人口大约有192,510,000人,1644年实际人口大约有152,470,000人;而英国经济学家安格斯·麦迪森则认为1600年明朝实际人口大约有160,000,000人。

根據南开大学王泉伟在《明代男女比例的统计分析——根据地方志数据的分析》一文當中的研究,明朝的性別比相當不平衡,明朝中期後,全國範圍內的性別比曾一度達到每100個女性中有150個男性的狀況,有些地區甚至出現每一百個女性中有大約300個男性的狀況;復旦大學中國歷史地理研究所的曹樹基也曾在《墓誌銘中所見明代人口結構》一文中提及明代性別比失衡、出現男性遠多於女性的狀況。

明世宗嘉靖末年美洲高產作物傳入後開始在明代人口最為稠密的江浙和嶺南地區普及和推廣,尤其是經過萬曆中興過後以較快速度穩定成長,到明神宗万历四十八年(1620年)根據當代學者研究估計達到前所未有的150,000,000人,分佈格局基本未變。明思宗崇禎十三年(1640年)到清世祖順治七年(1650年),由於农民战争、饥荒和瘟疫等造成中原地區死亡加大,特別是由於北方鼠疫和旱灾的爆發、以及八旗入關掠殺和為防範漢人而進行有計劃的遷移,造成人口大量減少,只有原先人口總數一半不到,特別是經歷鼠疫大爆發的北方,人口降到不足20\%。

明代沿襲元代,將人戶分為民戶、軍戶、匠戶三等。手工業者為匠籍。也就是規定全國技術好的手工業工人必須於官營手工業部門服務的制度。匠籍、軍籍比一般民戶地位低,不得應試,並要世代承襲。若想脫離原戶籍極為困難,需經皇帝特旨批准方可。

明代定以前的匠戶為匠籍,並規定這些入匠籍的手工、工人子孫世代承襲,不得脫籍改行,但不同點在於明代時,他們不需永遠在王朝服役,而只要依規定每隔幾年輪班到京城服役一次即可,稱之為輪班匠。輪班匠的勞動是無償的,还由於到京城的路途遙遠,輪班匠仍然常常發生逃役的狀況,於是在明宪宗成化二十一年(1485年),朝廷便下令輪班匠可繳交銀兩折抵役期,稱為「匠班銀」。嘉靖四十一年(1562年)起,朝廷進一步改革匠役制度,輪班匠一律征銀,以銀代役,政府則以銀雇工。人身束縛大為削弱。但仍有部分工人留在官營手工業單位服務,匠籍制並未完全廢除。

明初为了恢复生产和发展经济,政府有组织的把山西一带的民众迁到中原等人口较少的地区,史称洪洞树大移民。17世纪开始的全国性大旱灾直接导致了全国性的大蝗灾。也引发了波及差不多整个华北地区的鼠疫。人口大量死亡,灾民大量离乡。但因明末的动乱很快结束,而灾民除死亡外,不久也回到了原籍,并未形成大规模的移民。

明代时期的教育发达,学校兴盛,唐宋所不及。明朝初期實行「科舉必由學校」的政策,明太祖多次強調:「古昔帝王育人材,正風俗,莫不先於學校。」並將學校列為「郡邑六事之首」,以官學結合科舉制度推行程朱理學,而不重視書院,書院因此沉寂近百年之久。也因此,明朝中早期最重要的教育機構也就是國子監。而各府、州、縣政府也皆立學校。府、州、縣學的學生稱為生員,俗稱秀才或相公。明初生員數目有定額,大致府學四十人,州學卅人,縣學廿人。明代中後期,地方官員六事皆舉者極少,「學校之政之修也久矣」,因此傳統書院再次承擔起培養科舉人才的重任。明代書院的創辦,以嘉靖年間為最多,據統計,明代書院共有1239所。書院的經費來源,大體上可以區分為官方撥置、和私人捐贈。由於政治上的牽連,書院屢遭劫難,歷史上共有四次禁毀書院的記載,但官方越是禁止,民間開辦的書院就越多。

科舉在明朝是正式的選拔官吏制度。科舉考試分為兩級,每三年舉行一次,稱為大比。科舉考試的內容主要是四書五經,考生必須用八股文做答。所謂股,即對偶之意。八股文萌芽於宋朝,形成於明成化以後。由於八股取士的制度,讀書人既不通經史,又不諳實際,嚴重束縛民眾智慧的進步。

文學方面,中國小說史上的四大名著中的《西遊記》、《水滸傳》、《三國演義》(原本是《金瓶梅》後被三國代替)就是出於明朝。馮夢龍加工編輯的三部白話短篇小說集「三言」(即《喻世明言》《警世通言》《醒世恆言》)每部四十篇,共一百二十篇,主要是描寫青年愛情故事以及平民市井生活,最著名的如《杜十娘怒沉百寶箱》、《金玉奴棒打薄情郎》、《轉運漢巧遇洞庭紅》等;與「三言」類似每部四十篇的短篇小說集還有凌蒙初編著的「二拍」以及1987年才被發現的《型世言》(陸人龍編著)。傳統雅文學的發展在明代繼續發展,著名文人有劉基、宋濂、高啟、方孝孺、唐寅、歸有光、徐渭、王世貞、袁宏道、錢謙益、張岱、吳偉業等人。散曲家則有王磐、馮維敏、薛論道、陳譯、康海等人。

萬曆時期,猛烈反對前後七子的擬古主義,有以公安袁宗道、袁宏道與袁中道為代表的公安派。他們認為文學是隨著時代的變化而變化的,有各個不同的時代,即有各種不同的文學。竟陵鍾惺、譚元春為代表的竟陵派主張獨抒性靈,並且乞靈於古人,目的為“引古人之精神以接後人之心目,使其心目有所止焉,如是而已矣”。

明朝時期,傳統雜劇逐漸衰落,而傳奇劇走向繁榮。在嘉靖後期到萬曆初期出現三部優秀的傳奇作品,即《寶劍記》、《浣紗記》及《鳴鳳記》。明代戲劇的集大成者是湯顯祖。他的代表作是臨川四夢(即《南柯記》、《邯鄲夢》、《紫釵記》及《牡丹亭·還魂記》)。南戲在明朝也進入最繁盛的時期。明朝的文學與戲劇在對「情慾」的描寫上是較為開放的,例如《牡丹亭》一劇中就充滿許多對少女情懷的正面刻寫。

明朝朝廷極力推崇書法,明朝書法以行書和草書為主。明初書法陷於台閣體泥沼,沈度學粲兄弟推波助瀾將工穩的小楷推向極致,“凡金版玉冊,用之朝廷,藏秘府,頒屬國,必命之書“。二沈書法被推為科舉楷則,於是台閣體盛行。明中期吳中四家崛起,書法開始朝尚態方向發展。祝允明、文征明、王寵與唐寅是這個時期的代表,書法開始邁入倡導個性化的新境域。晚明書壇興起一股批判思潮,書法上追求大尺幅,震盪的視覺效果,有名的有張瑞圖、黃道周、王鐸與倪元瑞等,而帖學殿軍董其昌仍堅持傳統立場。

明代的繪畫成就巨大,大致偏重於文人畫派,往上承襲唐、宋、元三代的體系,再經過充分發揮後而自成一家的。明代画风迭变,画派繁兴。在绘画的门类、题材方面,传统的人物画、山水画、花鳥画盛行,文人墨戏画的梅、兰、竹及杂画等也相当发达。其中最興盛的山水畫派可分為氣勢恢弘的浙派、蒼勁活潑的院派與清麗縝密的吳派三種。著名的書畫家如擅長花鳥的徐渭、擅長人物畫的陳洪綬,「明四家」沈周、文征明、唐寅和仇英,山水畫大師董其昌。明朝繪畫以山水和花鳥為主。人物畫和社會風俗畫相對較弱。明朝的雕像較多為城隍、孔子、關公、岳飛等為主。

明代晚期由于传教士纷纷来华,西方近代绘画也传入中国,开始了东西方艺术的第一次正面交流。但由于东西方审美观的差异,西方艺术的影响主要体现在西洋版画艺术方面,尤其是坤舆图、西方原版图书以及圣迹画对明代晚期的绘画产生了重要影响。

明代的建築工藝創下新成就。南京和北京城池都是偉大的建築作品。应天府京城城墙營建於洪武二年,完成於洪武十九年,城牆周長達66里,一般寬10-18米,高12-15米,是世界上最長的城牆。南京城突破方正的格局,而是按照地理形勢修建。皇城位於東部,市肆和居民區位於南部,西北則是軍營。洪武二十三年起,明政府開始修建京师外郭城(即南京外城),周圍120里,開十六門,將雨花台和鍾山都包入其中。而北京城池則較為方正,體現皇權至上的思想。明朝的宮殿建築也十分宏偉,故宮即為例證。明朝各種歷制建築也十分嚴謹工整。天壇、太廟、社稷壇、孔廟都是十分巍峨莊嚴的建築。明代帝陵工程浩大,可謂歷代之最。而在明代被重建的萬里長城(明長城)更是舉世無雙的巨作,保衛明朝的邊疆,至今依然聳立。

明朝的興起與元末信奉明教與白蓮教的紅巾軍息息相關,所以明太祖建立明朝後對宗教採取抑制和利用兼並的政策。他主要希望阻斷摩尼教、白蓮教與彌勒宗等宗教組織再度變成反朝廷的起事軍,並且希望利用佛教、道教等宗教的力量來維護社會秩序。結果,得到「皇糧」全面保障的佛教與道教演變成缺乏精神上的創新追求,亦脫離廣大信眾,民眾轉而尋求民間宗教作為慰藉。

明朝流行對不同宗教兼容並取傾向,民間宗教性信仰、習俗多樣而活躍。基本精神在信仰自由主義、保持國家政治世俗性質、維持社會穩定和國家對社會的控制。集中體現這些政策精神的仍是儒家政治社會理念並倚賴士大夫群體的努力。其變動因素和矛盾來源,則在諸教向國家政權機關的滲透、皇室特殊化行為、民間泛神論多元信仰傾向、部分士大夫的信仰綜合主義。在此期間,回族的形成與猶太教的消亡,表現出作為外因的社會環境與作為內在動力的宗教本土化、世俗化運動,對宗教發展有著至關重要的影響。

明朝中期以後,佛教受皇室宗教活動加強的刺激與儒家的矛盾尖銳起來。這種矛盾促使部分士大夫強烈反對寺院修建並發表闢佛言論。明朝政府對藏傳佛教政策與對漢傳佛教政策有同有異。其重要差異之一是,明朝對藏傳佛教政策與對西部邊疆政策緊密相關,而對漢地佛教政策則於周邊關係政策基本無關。此外,部分士大夫以藏傳佛教為「番教」,認同程度遜於內地佛教。明朝一些皇帝因喇嘛多擅長某些「法術」,對其有特殊興趣,並因而導致士大夫針對相關政策的批評。道教起源於本土民間信仰,在明代與儒家士大夫的衝突比較和緩。但明朝君主中信奉道教者多,既影響到國家政治,也影響到士大夫與君主的關係。士大夫在反覆重申儒家原旨的同時,對道教的批評也日趨尖銳。民間宗教以最貼近下層百姓生活的組織形式和內容,滿足中下層民眾的宗教需求,甚至部分地滿足他們的政治要求和經濟要求。這是明朝中葉之後,民間宗教如火如荼發展起來的重要原因。明朝政府將民間宗教基本看作民俗,一般無干預,對視為「陋俗」者加以排斥,在涉及秘密社會活動時則嚴厲禁止。

明朝還是信仰伊斯蘭教諸民族、藏傳佛教黃教的形成和發展,以及天主教在中國傳播的重要時期。伊斯蘭教在社會生活中相對封閉,在明代政策中大體上表現為一個民族政策問題而不是一個宗教問題,基本與國家以及其他社會成分相安無事。明朝中期以後,天主教再度傳入中國,當時士大夫尋求改革,明朝對天主教大致寬容。

哲學思想上,王陽明繼承陸九淵的「心學」並發揚光大,他的思想強調「致良知」及「知行合一」,並且肯定人的主體性地位,將「人」的主動性放在學說的重心。而王陽明的弟子王艮更進一部的強化此方面的論述,提出「百姓日用即道」,肯定平民百姓日常生活的意義。而李贄則更肯定「人欲」的價值,認為人的道德觀念系源自於對日常生活的需求,表現追求個體價值的思想。而隨著西學的傳入,科學精神與實學風尚也開始流行。明末之際,伴隨著朝代的更替與異族的入主,哲學家開始更多的思考現實問題與政治改良,如王船山、黃梨洲、顧亭林等。

而明代晚期書院的興盛,衝擊官學的地位。許多知識分子利用在書院講學之際藉機批評時政,例如曾講學於東林書院的顧憲成及高攀龍,就常諷刺時政,也使東林書院成為與當權派對抗的中心,進而造成東林黨爭。當時學者也會借用寺廟周邊的空地舉行「講會」,倡導新的思想價值與人生觀。

以顧炎武、黃宗羲、王夫之並為明末清初三大儒。顧炎武提倡「經學即理學」,提出以「實學」代替宋明理學,要學者直接研習六經。提倡“天下興亡,匹夫有責”,著有《日知錄》、《音學五書》等。黃宗羲有「中國思想啟蒙之父」之譽稱,著有《明儒學案》、《宋元學案》,是中國學術史之祖。他保護陽明學,排斥宋明理學,力主誠意慎獨之說,蔚為浙東學派。王夫之強調實際行動是知識的基礎,認為歷史發展具有規律性,是「理勢相成」。其思想發展成船山學,後人編為《船山遺書》。

以民為天下之主的思想於明末清初亦有所流行,例如生活在明末又经历清初时期的黃宗羲和顧炎武、王夫之提倡民權,所著的《明夷待訪錄》攻擊君主專制體制,提倡天下為主,君為客的觀點,倍受清末革命黨的推崇。部分學者認爲黃宗羲的思想是近代民主主義的思想,有西方學者稱黃宗羲為「中國自由主義的先驅」。

西歐進入大航海時代後,葡萄牙意圖在中國建立貿易據點。1513年,葡萄牙國王曼努埃爾一世為想要與明廷通商,派出使節團前往中國。使節團本來想在廣州登陸,但被拒絕入境。他們改以武力佔據屯門,與明朝爆發屯門海戰、西草灣之戰,結果戰敗。最後明世宗嘉靖皇帝同意入境,並且讓葡萄牙人在澳門開設商行,允許他們每年來廣州「越冬」。其後西班牙、荷蘭、英國等國相繼派使團東來,使得不少西洋事物傳入中國。1582年,天主教傳教士利瑪竇奉命前往中國教區工作。利瑪竇很快學會中文,並穿儒服、通儒書,頗得明朝士大夫好感。後來他被舉薦到北京,頗得明神宗信任。他向中國進獻聖母瑪莉亞像、十字架、坤輿萬國全圖、西洋自鳴鐘、西洋大炮、西洋式望遠鏡、西洋式火槍、西藥等貢品,先後在北京、肇慶等地展出。

學者李正焕認為明朝是中国科技发展的极其重要的时期,涌现出一大批集大成的科学家和许多不朽的科技名著,而金觀濤、樊洪業、劉青峰則認為自宋元兩代以後,中國的科學發展日益趨於停滯狀態。煉鐵量是用來評估國家產力的重要指數。在宋朝,中國每年的煉鐵量總和相當於18世紀的全歐州總煉鐵量,在明朝,許多煉鐵廠被荒廢。

明朝初期的大統歷一直沿用元代授時歷,不准民间研究,下詣「国初学天文有厉禁,習曆者遺戍,造曆者誅死」,但天文导航、冶炼钢铁、商业数学等实用科技仍有许多重要成就。到了後期禁令被放寬後有學者編寫了一部天文著作,可是無人問津、不被重視且「未曾用之」,《大明律》規定:“造讖諱、妖書妖言及傳用惑眾者,皆斬。若私有妖書隱藏不送官者,杖一百,徒三年”,嚴厲處置撰寫、刊行、銷售或使用“妖書”的人。被送官燒毀的「妖書」名目有《換天圖》、《飛天歷》、《聚寶經》、《太上玄元鏡》等共計88種。明朝的大統歷是承襲元朝的授時歷,對日月蝕的預報早已不准,明朝開國一百多年後陸續有人建議改曆,被禮部以“古法未可輕變,請仍舊法”和“祖制不可變”的理由反對。明代欽天監的天文官們已無人能掌握元代郭守敬等製訂授時歷時所依據的原理和方法。利瑪竇憑藉西洋書本上的知識即可預測日月蝕,而欽天監的官員們卻一籌莫展。當徐光啟、李之藻等人打算用西法改曆而發動宣傳攻勢時亦引起了守舊勢力的反感。後來由於士大夫攻擊傳教活動,並謂私習天文為違反大明律,政府下令嚴禁,並將所有耶穌會士逐往澳門。

明太祖亦禁止人民進行科學研究,且鄙薄科學技術,認為皆是「无益」之物並加以毀壞:“明太祖平元,司天监进水晶刻漏,中设二木偶人,能按时自击钲鼓。太祖以其无益而碎之”。

作明中晚期學術著作眾多,如李時珍的《本草綱目》、宋應星的《天工開物》、徐光啟的《農政全書》、方以智的《物理小識》、程大位的《算法統宗》、吳有性的《瘟疫論》、徐霞客的《徐霞客遊記》,這些科學家幾乎都是明朝有功名的士子。1637年宋应星在《论气·气声》中对声音的产生和传播作出合乎科学的解释,认为声音是由于物体振动或急速运动冲击空气而产生的,并通过空气传播,同水波相类似。方以智在《物理小识》中提出:“宙(时间)轮于宇(空间),则宇中有宙,宙中有宇。”提出时间和空间不能彼此独立存在的时空观。在《物理小识》中正确地解释蒙气差(即大气折射)现象。民间光学仪器制造家孙云球制造放大镜、显微镜等几十种光学仪器,并著《镜史》。

明朝宗室在技術上也有贡献,朱載堉在世界上第一次正確地提出十二平均律,並在數學、天文學方面亦多有建樹;明初周王朱橚把四百余种植物种于府内,并让王府画工将植物绘图编制成书,名为《救荒本草》。《救荒本草》共记有植物414种,并详细描述各种植物的形态、产地、生境、可食用部位和食用方法,是生物学历史上的重要书籍。中晚明的軍事科技也有所進步,各種新式火器大量湧現,但也被當時的军事家批評不實用。西方傳入的佛郎機火炮和紅夷大炮都在中國製造和使用。還有一些專門的火器論著出現,如茅元儀所著之《武備志》。

明朝末期,隨著耶穌會傳教士和西學的傳入,中晚明的科學技術出現新的進步。在他們傳播教義的同時,也大量傳入西方的科學技術。當時中國的科學發展趨於緩慢,落後于歐洲。隨著西學傳入,使得中國的少數士大夫開始認識到西方學問之中有其優於中國之處,但這並未造成中國人對於中西學的基本高下看法有所改變。西學中主要受到注意的仍是技術方面如天文曆法、測量以及所謂的「西洋奇器」等,對於中國學術本身的影響衝擊亦不大。而當時傳入中國的學問非常多樣,也有一些士大夫著手與傳教士合作翻譯西方書籍或著書介紹西學,例如徐光啟就曾與利瑪竇合譯幾何原本。李之藻與利瑪竇合譯同文算指。在中西文化交流的同時,基於雙方文化的歧異及認知方面的不同,也引發一些衝突,例如南京教案等。

明朝數學的發展停滯,且遠比宋元落後,明朝中葉的著名數學家顧應祥與唐順之對「天元術」的茫昧不解,被認為是中國數學在十四世紀之後由盛而衰的一個見證。在明朝年間失傳了宋元兩代累積的數學知識,後來經過清代學者梅穀成等人重新發現並加以研究。駱祖英認為,整體而言,明代數學的整體水平並不比同期西方數學滯後,當時東西方數學水平相當。

明朝时期各少数民族政权得到了迅猛发展,民族关系形势也非常复杂。明前期,退居漠北的北元政权伺机南下扰明,企图东山再起,成为明朝的心腹大害;明中晚期,白山黑水的女真族在首领努尔哈赤的带领下,建立后金政权,并最终取代明政权。

明初武功实力最强,具开拓进取精神。在“大一统”思想的指引下,明朝以实力为后盾,注意使用军事打击和政治招抚相结合的策略,积极经略周围边疆地区,对后期民族关系思想的形成产生了重大的影响。同时,儒家知识分子刘基的夷夏观在华夷易代之际也表现出开明与宽容的特色。后来仁宣之治时在民族关系上做出来调整,南北一同放弃大规模军事征伐,采取“顺则抚之,逆则御之”的守成求安思想。

土木之变后明朝实力由盛转衰,对周边少数民族也由进攻态势全面转向防御,形成了“守备”为主的民族关系思想。随着西南地区麓川土司势力大增,大臣门对于是“剿”与“抚”展开争论。到了孝宗期间,面对国计日艰、边防日蹙,和北方的蒙古、女真等民族关系更加复杂的情况,明孝宗想在民族关系处理上想有番作为,让边臣献策,比如马文升的“抚安东夷”、“收复哈密”,杨一清的“关中奏议”,王鳌的“上议边八事”以及丘浚的“严武备”、“驭夷狄”等;另外随着明朝国力的衰微以及土鲁番势力的强大,哈密卫的“弃”与“守”成为当朝大臣讨论争锋的焦点。世宗和穆宗统治时期边患增多,北虜南倭使明朝疲于应付,特别是面对套寇屡屡犯边,边疆祸事不断,曾铣等有识之士就收复河套问题多次上疏。穆宗在位期间实现了明蒙之间具有里程碑意义的隆庆和议,结束了蒙古各部与中原王朝近二百年兵戈交战的局面。

神宗在位时爆发了万历三大征,虽然取得胜利,但是耗费了明朝人力物力财力,使国家日趋衰败。内阁首辅张居正启用大将李成梁和戚继光,在辽东和蓟镇取得大捷。熹宗明思宗时期明朝衰落到不可收拾的地步,东北女真建立后金政权,不断扰明。因此,朝廷任用辽东总兵熊廷弼、袁崇焕等人和女真对抗。同时在明清易代之际,明末思想家王夫之表现出特有的悲壮情怀和对华夷问题的反思,成为近代民族主义思想的滥觞。

明朝承袭传统的华夷之辨民族思想,尊崇汉族,鄙视少数民族,并进一步强化。而明朝民族关系思想基本上是对传统儒家民族观“大一统”和“华夷之辨”思想的继承和发展,同时又受到蒙元政权的影响,表现出“华夷一家”与“华夷之防”思想的矛盾与统一。但是消极、保守的边疆政策不仅影响了民族关系的发展,对于一个整体的统一多民族国家的形成、发展,具有重大影响。

学术界一般认为明朝是回族最终形成的时期。元朝灭亡后,不断有归附明朝政府的回人,明初政府曾禁止蒙古、色目人更易姓氏,限制回族内部通婚,後來明廷支持對回民的漢化政策,讓回民改易漢姓。朱元璋“御制至圣百字赞”以及明皇室关于修建清真寺和保护清真寺宗教职业人员的谕旨,在一定程度上肯定了回族的宗教生活,史學家陳垣指出:「明人對於回教,多致好評。政府亦從未有禁止回教之事,與佛教、摩尼教、耶穌教之屢受政府禁止者,其歷史特異也。」。明代學者陸容說:「回教門異於中國者,不供佛,不祭神,不拜屍。所尊敬者惟一天字。最敬孔聖人。故其言云:僧言佛子在西空,道說蓬萊往東海,唯有孔門真實事,眼前無日不春風。見中國人修齋記醮,則笑之。」大约经历了200多年,在伊斯兰教影响下,以回回人为主体,融合了国内汉、维、蒙等多种民族成分逐渐形成为新的民族共同体。在明末农民起义中,陕北和甘肃东部的回民在马守应的率领下,成为当时张献忠、李自成軍隊的主力之一。明末清初时期,米剌印、丁国栋在「反清复明」的口号下,率领了持续两年的甘州起义。到了清代,回族社会政治地位降到了历史上的最低点。

明朝邊境上最大的兩個威脅明朝安全的部族是蒙古和女真,時人稱其為東虜和西虜。在明朝初年武功強盛時,一度將蒙古驅至漠北,蒙古也因內亂分裂成韃靼、瓦剌等部而無力南侵。之後伴隨明朝的衰落,蒙古諸部中最有實力者稱霸於族內後,也多次進攻明朝,諸如瓦剌發起的土木之變和土默特部發起的庚戌之變,明朝的疆界因此內縮,也大大消耗明朝的國力。俺答汗後期開始於明朝通好,受封為順義王,其後的三娘子繼承和平的政策。明蒙之間邊境安寧和平,互通有無。這種情況直到後金控制蒙古後才告結束。明朝早期曾經設置奴兒干都司來管理東北諸部,這一階段女真人作為明朝於東北地區排除北元殘餘勢力的盟友,雙方關係處於蜜月期,但中後期明朝採取「犁庭掃穴」等一些列不適當政策,對女真人進行歧視、限制、挑撥、分化甚至屠戮,激化當地矛盾。隨著東北的蒙古部和女真部日益強大,奴兒干都司被廢,明朝在東北的控制力更是進一步下降。17世紀後,建州女真首領努爾哈赤統一女真各部,降服蒙古,於1616年建國後金,與明朝分庭抗禮。後金佔領遼東大部土地,曾對當地的漢人進行屠殺,並有入主中原的野心,嚴重威脅明朝的安全。1636年改國號大清,建立清朝,最終於1644年明朝滅亡後接替明朝統治中國276年的歷史。

苦兀或称苦夷,是明代对库页岛上土著居民的称呼。永乐七年(1409年),明朝在黑龙江下游东岸特林设奴儿干都司,管理今东三省。《敕修奴儿干永宁寺碑记》、《重建永宁寺碑记》载:明钦差亦失哈等多次巡视奴儿干地方,曾对“海外苦夷诸民,赐男妇以衣服、器用,给以谷米,宴以酒食”。他们表示,“世世臣服,永无异意”。清代亦曾在此设姓长以统之。有人认为,“海外苦夷”(库页人)是指库页岛上的阿伊努人。

中國學者對於明朝對藏政策的主流見解是「因俗以治」、「多封眾建」、「羈縻懷柔」。明朝对西南藏族地区的治理基本承袭元朝统治管理的办法。对西藏地区推行“多封众建”的政策,先后分封三大法王和五大地方之王。同时,通过朝贡和回赐,互通有无,体现西藏与中央政治上的隶属关系。明以来,藏族地区社会安定,经济发展迅速,文化艺术繁荣,与中國内地的交往更趋广泛和密切。美國漢學家莫里斯·羅西比(英语:Morris Rossabi)認為,永樂帝是第一名積極尋求擴大與西藏關係的明朝統治者。

明朝时期傣族被称为“百夷”,而且经营百夷地区主要通过土司制度,明朝还制定了其他政策、采取了其他措施加强明朝对百夷的统治。百夷地处西南边疆地区,因此,明朝经营百夷的政策与明朝的西南边疆的形势发展息息相关。但由于明朝统治者的短视与误判,以“析解麓川地”的错误政策经略这一地区,最终导致明末缅甸洞吾王朝对中缅边界中方一侧领土的侵扰和“蚕食”,造成明朝西南边界大幅内缩。

明朝初年,實施朝貢體制,朝貢貿易薄來厚往,許多日本人冒充朝貢使者來賺取好處。日本實際上是處於割據狀態,沒有統一的中央政權,很多到中國來冒充朝貢使者的日本人沒有日本政府的管轄,朝貢後他們滯留在中國沿海搶劫。這是明初的倭寇。為防止倭寇朱元璋就頒布海禁政策。從此之後,如果要來中國做生意,必需朝貢兼貿易,否則不予,這就是所謂的「朝貢貿易」,兼具有懷柔拉攏周圍國家的用途。明朝嚴格的貿易管制政策的影響導致正常貿易地下化,轉為走私貿易。貿易港集中地由廣東、福建轉往已成為殖民地的菲律賓、印尼。而海上的維持秩序角色由於中國官方的消失而導致海盜集團猖獗。由於海上貿易仍在暗處進行,美洲銀器又大量流入中國,銀開始成為流行的通貨。

明初鉴于倭寇的猖獗,明初曾实施海禁政策,永樂年間,明成祖派遣航海家三寶太監鄭和率遠洋船隊七下西洋,最遠到達非洲東海岸,又派遣吏部驗封司員外郎陳子魯出使撒馬兒罕、吐魯番、火州等西域十八國,加強明王朝同世界各國的經濟政治上的往來,為中國走向世界做出貢獻,體現永樂王朝的鼎盛和開放,也能表现出明朝海洋政策具有外向型海权意识。後来明仁宗聽從朝中一些大臣的意見,認為下西洋過於浪費,收效不大,宣佈停止下西洋的活動。不到一年,仁宗病逝,宣宗朱瞻基繼位,改年號宣德。宣德五年閏十二月初六(1431年1月19日),派鄭和第七次也是最後一次下西洋。明憲宗年間,曾有太監向憲宗提議再次下西洋,於是皇帝下詔到兵部索要鄭和出使的海圖等資料。但由於劉大夏等官員認為下西洋為一大弊政,有害無益,因此將當年鄭和出海地圖等資料藏匿起來(一說銷毀),兵部尚書項忠命吏入庫搜索無果,再次下西洋一事於是作罷。

而相当长时段内领先于世界的明朝海军,随着保守海洋政策的施行,海军实力迅速衰落。自唐宋以來中國的大航海事業,在明代出現衰退。儘管也有“鄭和下西洋”的驚世盛舉,但總的來說,海外貿易在整個明代的經濟體系中所佔比重不大。明代海禁約持續了兩百年的時間,其結果是關閉了民間對外貿易的通道。私人下海販易被視為違法,海外商船來華貿易也受到嚴格的控制。朝貢貿易則是唯一留下的貿易孔道,由官方壟斷專營海外貿易,並與朝貢制度嚴密掛鉤,從而形成朝貢與貿易合二為一的「貢市一體化」格局。明代學者王圻在《續文獻通考》中記述:「凡外夷貢者,我朝皆設市舶司以領之⋯⋯其來也,許帶方物,官設牙行與民貿易,謂之互市。是有貢舶即有互市,非入貢即不許其互市明矣。」日本學者內田直作認為:「明代之朝貢貿易,不論從貿易政策上或財政政策上講,都沒有重大的價值,只是舉揚所謂朝貢禮的服從關係而已。」由於朝貢貿易無視經濟法則,幾乎全靠國力的強盛來維持,因此在明初明太祖和明成祖之後,由於國力漸衰以及時勢發生變化,朝貢貿易也走向衰落,代之而起的是走私貿易。

後來倭寇橫行,明朝加大海禁的力度,直到明穆宗隆慶元年(1567年)之後,倭寇逐漸平息,朝廷有鑒於對外貿易對沿海居民的重要性,才逐步有限度地對外開放,並開放福建月港為中國商民出洋貿易的唯一口岸,允許民間商船出洋遠販東南亞各地,惟日本不在通商範圍之內,去日貿易仍被視為「通倭」之舉,史称「隆庆开关」。

唐朝以来秉持着中华正统史观的朝鲜一直都是以“藩国”自居,尊中原王朝为宗主国,但在历代王朝中,朝鲜最为心悦诚服的却是明朝。1392年,高丽王朝大将李成桂发动政变,建立了李氏朝鮮。上书朱元璋要求赐予“国号”,朱元璋认为“朝鲜”是古名,而且“朝日鲜明”出处文雅,因此裁定朝鲜为新国名。 朝中关系进入了近三百年的相对稳定时期。明亡之后,朝鲜君臣无不思念明朝,最后修建了大报坛来纪念明朝皇帝,尽管此时朝鲜官方文书的纪年在明亡后早已采用清朝的年号,无论是私人文书,还是皇室的祭祀中,私下里一概都是延用明朝纪年,以至于出现了“崇祯两百多年”事情。

清朝基本上不干涉朝鮮的尊明之舉,朝鮮對明朝的崇拜不僅沒有影響到對清朝的忠誠,反而讓清朝感到朝鮮是一个知恩圖報、講情重義的國度。康熙帝曾說:「觀朝鮮国王,凡事极其敬慎,其国人亦皆感戴。」

倭寇對明朝的海疆構成嚴重威脅。但是倭寇的主要構成並非日本人,而是中國沿海一帶的破產流民。期間雖有朱紈和張經的抗倭,但最後都未能取得完全的成功。為防止倭寇的侵擾,世宗時期實行海禁,斷絕對日貿易。直到戚繼光等名將力行抗倭,倭寇才被剿清,海疆形勢才趨於平靜。豐臣秀吉統一日本後,意欲佔領朝鮮。萬曆廿年,日本進攻朝鮮,朝鮮國王逃到義州並派使節向明朝求救。明朝一度取得戰爭的勝利。中日一度進行和談。但萬曆廿五年後,日本再次進攻朝鮮,戰爭進入僵局狀態。萬曆廿六年,豐臣秀吉逝世,日本軍心動搖,結果撤軍。此即為壬辰衛國戰爭。這次戰爭嚴重削弱明朝與朝鮮兩國,明朝在張居正期間積蓄的國力大量被消耗,日本復又陷入分裂,女真部落成為相對的得益者。

1377年,朱元璋册封阿瑜陀耶国王为“暹罗国王”,“暹罗”这一名称正式固定下来,称为中文语境下对泰国的称呼。有明一代,阿瑜陀耶遣使臣访问中国达112次,而中国也派使臣访问阿瑜陀耶19次。

歐洲進入大航海時代後,葡萄牙人持續開拓前往印度、中國的航路,1511年葡萄牙佔領馬六甲(約今馬來亞地區)後,就意圖在中國建立貿易據點。明武宗正德七年(1513年),葡萄牙國王曼努埃爾一世為想要與明廷通商,派出使節團前往中國。使節團本來想在廣州登陸,但被拒絕入境。他們改以武力佔據屯門,與明朝爆發屯門海戰、西草灣之戰,結果葡萄牙戰敗。最後明世宗同意葡方入境,並且讓葡萄牙人在澳門開設洋行,修建洋房,允許他們每年來廣州「越冬」,這是西方列強第一次正式性的登陸中國。其後西班牙、荷蘭、英國等歐洲國家相繼派使團東來,使得不少西洋事物傳入中國。

明神宗萬曆十年(1582年),利瑪竇奉命前往中國教區工作。利瑪竇很快學會中文,並穿儒服、通儒書,頗得明朝士大夫好感。後來他被舉薦到北京,頗得明神宗信任。他向中國進獻坤輿萬國全圖、自鳴鐘、日晷、西洋大炮、望遠鏡、火槍、西藥、聖母瑪莉亞像、十字架等貢品,先後在北京、肇慶等地展出。利瑪竇不僅傳播天主教,還啟發徐光啟、李之藻等人學習西學。另外他還將中國各種文化傳入歐洲,如儒家思想、佛道學說、圍棋等,可謂「貫通中西第一人」。另外,在明末時期有不少明朝軍隊曾裝備火器,尤其是西洋大炮。

明代早期,社會風氣比較節儉。後期伴隨著商品經濟的發達以及政府控制力的下降,社會風氣轉向浮華與奢靡,不論士大夫或百姓,在飲食、居住、穿著、娛樂各方面都更為講究,甚至贫穷人家也追慕仿效,與過去儒家崇尚簡樸的風氣有很大的差別。商人的地位也明顯提高。時人张瀚曾言:“今之世风,上下俱损矣!”明初朱元璋認為「元以寬失天下」,因此要「救之以猛」,一改元朝優容江南士人的政策,採取各種措施打壓及迫害江南文人。有明一代,明廷便擬定江南重賦,「官、民田視他地方倍蓗」,並且規定「浙江,江西,蘇松人毋得任戶部」。仕宦的江南士人,或因黨案,或因文字獄之故,動輒獲罪橫死。

明朝的另一項重要社會風氣就是藏書之風。無論官方與民間皆好藏書。私家藏書尤為發達。天一閣是中國目前現存的最早的私家藏書樓。其創建者是范欽。在范欽去世時,天一閣藏書的總數達到七萬卷。天一閣對藏書嚴加保管,水火不入。也嚴禁外借。明代重要的藏書樓還有汲古閣、絳雲樓等。而私人刻書也逐漸發達,出現的彩印的套印等新工藝,印製的書籍量更是達到一個新的高峰,也使得書籍的讀者群更為擴大,各種通俗小說的出現也為平民百姓提供另一種娛樂。裝幀方法也得到改進,出現對後世影響深遠的線裝書。

貞節旌表的制度在明朝成為固定持續的制度,使得女性守貞守節從原本的典範理想成為一般性的風氣甚至規範。而纏足也在明朝逐漸成為社會上較普遍的習俗。此外,晚明社會風氣的開放,使當時成為中國歷史上才女文化最發達的時代之一。

16世纪的欧洲城市规模较小,1519年至1558年时期,拥有2万至3万人口即可称为“大城市”。从城市规模和人口比例看,晚明中国的城市化程度反倒稍高一些。据伊懋可的数据,中国城市人口在明末占总人口的6\%至7.5\%。而学者曹树基估计,1630年时中国城市化率已达到8\%,略高于清代城市化率的7.4\%。

明代百姓的娛樂風尚發達,「旅遊」一詞在中国歷史上首次出現。明代傢俱的樣式也進入一個新的階段,風格典雅,流傳至今者不在少數。園林藝術在明朝也非常興盛,代表著作是明代造园家計成的《園冶》一書,這是第一部全面總結私家園林的專著。

明朝是中国古代社会福利最好的时期,在平定天下驱除胡虏之后,朱元璋一方面实施“与民休息”的经济政策,另一方面推出了中国最早的福利政策。明朝的福利政策完备且有特色,对当时经济的发展起到了非常积极的推动作用。明代出现了免费养老院、免费医院和免费公墓等,而且对于60岁以上的老人,明朝政府制定了较为完备的养老政策。

明代的茶文化與酒文化也十分發達,民間盛行飲酒之風,酒令進入成熟的階段。各種新式茶色紛紛出現,紫砂壺也開始流行。酒樓茶館成為城市居民的重要休閒場所。

《乌青镇志》记载万历年间,市井之家的宴席:“万历年间,牙人以招商为业。初至,牙主人丰其款待,割鹅开宴,招妓演戏,以为常。”

万历进士顾起元在《客座赘语》中记述南京风俗民情说:“今则服舍违式,婚宴无节,白屋之家,侈僭无忌。”

张岱在《陶庵梦忆》中记载了许多美食:“越中清馋,无过余者,喜啖方物。北京则苹婆果、黄、马牙松;山东则羊肚梨、文官果、甜子;福建则福桔、福桔饼、牛皮糖、红腐乳;江西则青根丰城脯;山西则天花菜;苏州则带骨鲍螺、山楂丁、山楂糕、松子糖、白圆、橄榄脯;嘉兴则马交鱼脯、陶庄黄;南京则套樱桃、桃门枣、地栗团、窝笋团、山楂糖;杭州则西瓜、鸡豆子、花下藕、韭菜、元笋、塘栖蜜桔;萧山则杨梅、莼菜、鸠鸟、青鲫、方柿;诸暨则香狸、樱桃、虎栗;嵊则蕨粉、细榧、龙游糖;临海则枕头瓜;台州则瓦楞蚶、江瑶柱;浦江则火肉;东阳则南枣;山阴则破塘笋、谢桔、独山菱、河蟹、三江屯怪、白蛤、江鱼、鲥鱼、里河。远则岁致之,近则月致之,日致之。耽耽逐逐,的为口腹谋。”

叶梦珠在《阅世编》记述明末宴会:“肆筵设席,吴下向来丰盛。缙绅之家,或宴官长,一席之间水陆珍馐多至数十品。即庶士中人之家,新亲严席,有多至二三十品者。若十余品则是寻常之会矣。然品必用木漆果山如浮屠样,蔬用小瓷碟添案,小品用攒盒,俱以木漆架架高,取其适观而已。即食前方丈,盘中之餐,为物有限。崇祯初,始废果山碟架,用高装水果,严席则列五色,以饭盂盛之。相知之会则一大瓯而兼间数色,蔬用大铙碗,制渐大矣。”

明代笔记记载:“昔有一人,善制鹅掌。每豢养肥鹅将杀,先熬沸油一盂,投以鹅足,鹅痛欲绝,则纵之池中,任其跳跃。已而复擒复纵,炮瀹如初。若是者数回,则其为掌也,丰美甘甜,厚可经寸,是食中异品也。”。

明朝服饰继承了宋元两代的式样,但亦有一定程度的胡化,例如明代流行的曳撒就是继承于元代蒙古人的腰线袄。中后期更出现了前代未见的形制款式如立领,以及于一件衣服的显眼处大量使用钮扣。至清朝期间逐渐被禁止,但仍有少数款式和特征流传至今。近代至现代朝鲜族、琉球族、京族的民族服饰(韩服、琉装、越服)亦深受明朝服饰影响。

明代婦女的服裝,主要有衫、襖、霞帔、褙子、披風、比甲及裙子等,明中期出現立領。比甲的名稱,見於宋元以後,但這種服飾的基本樣式,卻早已存在。比甲為對襟、無袖,左右兩側開衩。隋唐時期的半臂,就是與比甲有著一定淵源關係。明代比甲大多為年輕婦女所穿,而且多流行在士庶妻女及奴婢之間[原創研究?]。成年女性多戴狄髻,並於上面插上成套的飾物,稱為頭面。明代上襦下裙的服裝形式,與唐宋時期的襦裙最大差別在於明代的上衣並不束在裙外,這種款式稱為襖裙。比如立领、宽衣大袖紧袖口与大褶裙装等,都是大明服饰的特色。勞動時常加一條短小的腰裙,以便活動,有些侍女丫環也喜歡這種裝束。上襦除傳統的交領外,到明中後期還出現立領。裙子除繼承前代的百褶裙、褶襉裙外,還出現了馬面裙。裙的顏色,初尚淺淡,雖有紋飾,但並不明顯。至中期則多飾以膝襴,有刺繡、織金、燙金等形式的裙襴。崇禎初年,裙子多為素白,即使刺繡紋樣,也僅在裙幅下邊一、二寸部位綴以一條花邊,作為壓腳。裙幅初為六幅,即所謂「裙拖六幅湘江水」;後用八幅,腰間有很多細褶,行動輒如水紋。到了明末,裙子的裝飾日益講究,裙幅也增至十幅,腰間的褶襉越來越密,此時出現一種裙子,每褶都有一種顏色,微風吹來,色如月華,故稱「月華裙」。腰間多掛上荷包、事件(小工具組合)等物品,裝飾與實用性兼備。明代出現一種以各色零碎錦料拼合縫制成的服裝,稱為水田衣,形似僧人所穿的袈裟,因整件服裝織料色彩互相交錯形如水田而得名。它具有其它服飾所無法具備的特殊效果,簡單而別致,水田衣的制作,在開始時還比較注意勻稱,各種錦緞料都事先裁成長方形,然後再有規律地編排縫制成衣。到了後來就不再那樣拘泥,織錦料子大小不一,參差不齊,形狀也各不相同,與戲台上的「百衲衣」(又稱富貴衣)十分相似。

明代男子常服、吉服、常禮服等,多用袍衫,有直身、直裰、道袍、道服、行衣、深衣等形制。上層社會及富家男子的便服面料以綢緞為主,上繪有紋樣,也有用織錦緞制作的,其制為大襟、右衽、寬袖,下長過膝。常服及吉服道袍、直裰、直身等,配以絲縧,勞動者多穿上衣下褲組成的裋褐。巾帽有多款,常見有幅巾、大帽、東坡巾、儒巾、飄飄巾等。

明太祖朱元璋詔令「衣冠制度悉如唐宋之舊」,因此明朝漢族男子服式沿襲大襟右衽交領和圓領這兩種傳統服飾式樣,又大量吸收元代服飾特點,發展出曳撒、兵笠等特色服飾。明代婦女的服裝,主要有衫、襖、霞帔、褙子、比甲及裙子等,衣服的多變與款式做工達到一個高峰。


%% -*- coding: utf-8 -*-
%% Time-stamp: <Chen Wang: 2021-11-01 17:10:49>

\section{太祖朱元璋\tiny(1368-1398)}

\subsection{生平}

明太祖朱元璋(1328年10月29日-1398年6月24日),或稱洪武帝,明朝開國皇帝,原名朱重八,曾改名朱興宗,投军被郭子兴取名元璋,字国瑞,生於濠州钟离县。廟號「太祖」,谥號「开天行道肇纪立极大圣至神仁文义武俊德成功高皇帝」,統稱「太祖高皇帝」。在位三十一年,因年号洪武也俗稱洪武帝。太祖之後的皇帝除明英宗(二度在位),皆實行一世一元制。

朱元璋出身贫农家庭,幼时贫穷,曾为地主放牛。後因災變,曾一度剃髮出家,四出流浪,化緣為生,25岁(1352年)时,参加郭子兴领导的红巾军反抗蒙元政权。先後击败了陈友谅、张士诚等其他諸侯軍閥,统一南方,後北伐灭元,建立大一統的封建皇朝政權,国号“大明”。

明太祖在位期间,為其家族能夠長期統治平民用殘酷方法殺害了許多人, 自著大誥三編宣揚部份經過。據臣下劉辰所著國初事跡他又發明使用多種殘酷殺人方法。

明太祖下令农民归耕,奖励垦荒;大興移民屯田和军屯;组织各地农民兴修水利;大力提倡种植桑、麻、棉等经济作物和果木作物;下令解放奴婢;减免賦稅。派人到全国各地丈量土地,清查户口等等。经过洪武时期的努力,社会生产逐渐恢复和发展,史称「洪武之治」。同时立《大明律》,用严刑峻法管理百姓与官僚,禁止百姓自由迁徙,严厉打击官吏的贪污腐败,设立锦衣卫等特务机构,整肅顯貴的势力及他認為對他的朝廷有威脅的人、並废中书省,由皇帝直領各部,进一步加强了中央集权。驾崩後传位于嫡长孙朱允炆為明惠宗。

明太祖的生活儉樸、工作勤奮,在南京的皇宮內,沒有設立“御花園”,只有“御菜園”,其中種滿蔬菜,使得皇宮自給自足。大封宗籓,令世世皆食歲祿,不授職任事。至明朝中后期,朱元璋子孫人口繁殖至近百萬人。洪武元年令:「凡孝子順孫、義夫節婦、志行卓異者,有司正官舉名,監察御史、按察司體覆,轉達上司,旌表門閭。又令:民間寡婦,三十以前,夫亡守制,五十以後,不改節者,旌表門閭(貞節牌坊),除免本家差役。」洪武二十六年令:「凡婦人因夫、子得封者,不許再嫁。如不遵守,將所授誥赦追奪,斷罪離異。其有追奪為事官誥赦,具本奏繳內府,會同吏科給事中、中書舍人,於勘合低簿內,附寫為事緣由,眼同燒毀。」明朝婦女守寡盛行。又創立明朝入宮婦女生殉制度。

元文宗天曆元年九月十八日(1328年10月29日)未時,朱元璋出生於濠州钟离县东乡(今安徽省凤阳县小溪河镇燃灯寺村),排行第三。朱元璋先世家沛(今江苏沛县),後徙句容(今江苏省句容市)达百年之久。祖辈生活在古泗州(今江苏省盱眙县)。父親朱五四(後改為世珍),母親陳氏为濠州钟离县(今安徽省凤阳县)人。

朱元璋幼時甚貧困,並無法讀書,曾為地主放牛。牧童伙伴多人都奉朱为领袖,且日后成朱起义将领多人,至正四年四月(1344年)淮北大旱,引發饑荒,朱元璋初六父崩,初九兄薨,廿二日母崩,與仲兄極力營葬後秋九月入皇覺寺當行童。入寺五十日,因荒年寺租難收,寺主封倉遣散眾僧,朱元璋只得離鄉為遊方僧雲遊淮西潁州。

元至正八年(1348年),朱元璋游歷淮西、汝潁、泗等州完畢,返回皇覺寺并逐渐讀書识字。至正十二年(1352年)二月辛丑,身在皇覺寺多年的朱元璋受好友湯和來信勸說,到濠州投靠郭子興,參加紅巾軍。由於指揮有方,不久便成為郭子興身旁一名親兵并赐名元璋字国瑞,並娶郭子興養女马氏(即後來的馬皇后)。後來朱元璋見郭子興與其他濠州紅巾軍領袖如孫德崖、趙均用不和,屢有衝突,朱元璋不願涉及濠州內鬥,故主動要求返家鄉招募新兵,徐達、湯和等朱元璋兒時好友獲准隨行,不久朱元璋的部隊已有結集了數千人。次年,朱元璋部隊攻下滁州,成為他首個據點,同時也在攻佔滁州期間,李善長加入朱元璋部隊,成為他一個重要幕僚。此時,濠州的郭子興被孫德崖及趙均用迫走,前來滁州投靠朱元璋,由於朱元璋名義上仍是郭子興部下,朱元璋乃將滁州兵權交予郭子興。

至正十四年(1354年),張士誠據高郵,自稱為誠王,十五年,元朝丞相脫脫率軍進攻高郵,分兵攻六合,六合乃滁州屏障,故朱元璋領兵援六合,幸好脫脫被誣陷而被迫交出兵權,元軍不戰自潰,滁州也轉危為安。朱元璋見滁州地小,建議進攻長江北岸的和州。朱元璋攻下和州不久,郭子興病故,郭子興次子郭天敘被立為都元帥,朱元璋與郭子興妻弟張天祐為副元帥,遥奉韩林儿的大宋龙凤政权。同年夏,常遇春、廖永安、俞通海歸附朱元璋,使得其軍著手渡江攻入采石、太平路,並計劃攻取集庆路(今南京市)。此時,元軍降將陳野先願協助紅巾軍攻集慶,郭天敘與張天祐感軍功不及朱元璋,故決定在陳野先引領下,親自領軍攻打集慶。結果紅巾軍攻集慶時陳野先叛變,郭、張二人被殺,陳野先也死於亂軍中。郭天敘與張天祐死後,朱元璋成為都元帥,盡領郭子興舊部。至正十六年(1356年),朱元璋領軍再次攻打集慶,結果集慶被朱元璋部隊一舉攻陷,朱元璋將這裡作為自己的根據地,並改名為應天府。至此,朱元璋以應天府為中心,與元朝軍隊、張士誠、徐壽輝等部形成犬牙交錯之勢。

朱元璋攻佔應天後,開始攻佔應天周邊地區以鞏固防務。至正十六年,遣徐達攻佔鎮江、鄧愈克廣德,次年,遣耿炳文克長興,徐達克常州,而朱元璋親自率眾攻取寧國。隨後趙繼祖克江陰、徐達克常熟。胡大海克徽州、常遇春克池州,繆大亨克揚州。至正十八年,朱元璋親取婺州。明年,朱元璋陸續攻佔浙東餘下各地,常遇春克衢州、胡大海克處州,至此朱元璋部控制江左、浙右各地,向西與陳友諒部相鄰。朱元璋攻下浙東後,小明王升朱元璋為儀同三司江南等處行中書省左丞相,同時朱元璋也得浙東名士如朱升、劉基相助,朱元璋採取朱升「高築牆、廣積糧、緩稱王」的建議,採取穩健的進攻措施;並且遵照劉基「先漢後周」之策略,着手對江南各勢力進行對抗。

至正二十年,陳友諒攻陷太平路,隨後弒主徐壽輝、稱帝建國,國號漢,之後傾全軍攻應天府。朱元璋與劉基設計,先命胡大海進攻信州,斷陳友諒後援,再命部下康茂才詐降作陳友諒的內應,引漢軍主力進入朱元璋在應天城外龍灣設下的埋伏中,結果漢軍被朱元璋軍隊大敗,隨後朱元璋攻取太平、安慶、信州等地。。至正二十一年,朱元璋改樞密院為大都督府,重新整理軍制。北結察罕帖木兒、密通方國珍,而與正面的陳友諒部進行會戰。同年攻克江州、南康、建昌、撫州等地。次年,佔領龍興,改洪都府(今江西南昌)。

至正二十三年(1363年),张士诚派部将吕珍围攻退守安丰的小明王韓林兒及丞相劉福通,朱元璋不顧劉基反對,派軍北上解安豐之圍,結果刘福通战死,韩林儿被朱元璋救出。此后,韩林儿被朱元璋安置在滁州,仍然被奉为皇帝。陳友諒趁朱元璋主力軍北上,率六十萬水軍進攻朱元璋根據地,首先圍攻洪都,但朱元璋姪朱文正堅守洪都兩個多月,待朱元璋親率二十萬部隊馳援,陳友諒大軍改往鄱陽湖與朱元璋大軍交戰,史稱“鄱陽湖之戰”。陳友諒自恃巨艦出戰,採用炮攻,朱元璋險些負傷被擒。隨後,朱元璋利用東北風而改用火攻,致使陳友諒部大量受損。之後朱元璋利用鄱陽湖水位降低便於小舟活動,改為分兵水路圍攻陳友諒。陳友諒中箭身亡,漢軍潰敗。隨後朱元璋圍攻武昌,并盡佔湖北各地。次年,朱元璋自立為「吳王」,以李善長為右相國,徐達為左相國,常遇春、俞通海為平章政事,立子朱標為世子。次月再次親征武昌,陳友諒之子陳理舉降。隨後吳軍相繼攻克廬州、吉安、衡州。至正二十五年,吳軍繼續攻佔寶慶、贛州、浦城、襄陽,同年冬,下令討張士誠。次年,吳軍再次攻破湖州、杭州。再一年,徐達克平江,張士誠被俘,至此朱元璋一統江南。至正二十六年(1366年),朱元璋派廖永忠迎接韩林儿至金陵應天府,途中在瓜步渡长江时,韩林儿所乘船只沉没,韩遇难。

至正二十七年(1367年),朱元璋命湯和為征南將軍,討伐割據浙東多年的方國珍。隨後制定北伐战略:先攻取山東,其次進攻河南,再次攻佔陝西潼关,最後再進軍元大都。隨後命徐達為征虜大將軍,常遇春為副將軍,帥師二十五萬,由淮河進入,北取中原。并命胡廷瑞為征南將軍,何文輝為副將軍,進攻福建。同年,方國珍投降,徐達攻破山東濟南,胡廷瑞下邵武,湯和、廖永忠由海道攻克福建福州。北伐一直持續到洪武年間,徐達、常遇春隨後攻佔整個河南、山西,最終直取元大都(今北京)。

至正二十八年正月初四(1368年1月23日),朱元璋在應天府登基即位,建國號大明,年號洪武,是為「明太祖」。以應天為「南京」,開封為「北京」。同年八月初二(9月14日),大將徐達攻克大都,元朝覆亡。由于幼年对于元末吏治痛苦记忆,即位后一方面減輕農民負擔,恢復社會的經濟生產,改革元朝留下的糟糕吏治,懲治貪污的官吏,社會經濟得到恢復和發展,史稱洪武之治。明太祖確立了里甲制,配合賦役黃冊戶籍登記簿冊和魚鱗圖冊的施行,落實賦稅勞役的徵收及地方治安的維持。

太祖平定天下後,大封諸將為公侯,部份追封為王。初封六公,其中以五大將、一大臣為開國元勳。分別為:韓國公李善長、魏國公徐達、鄭國公常遇春、曹國公李文忠、宋國公馮勝、衛國公鄧愈。而後又追封胡大海為越国公、戰死的丁德興為濟國公,湯和為信國公、馮國用封郢國公。次年,明太祖於雞鳴山立功臣廟,六月初三日廟成,太祖親定功臣位次,以徐達為首,次常遇春、李文忠、鄧愈、湯和、沐英、胡大海、馮國用、趙德勝、耿再成、華高、丁德興、俞通海、張德勝、吳良、吳禎、曹良臣、康茂才、吳復、茅成、孫興祖凡二十一人。死者像祀,生者虛位。又以廖永安、俞通海、張德勝、桑世杰、耿再成、胡大海、丁德興七人配享太廟。此位序屡经删汰,已非洪武二年所定名单位次。

随後,太祖进一步加强中央集权。洪武三年(1370年),杀中书左丞杨宪。洪武四年七月十一(1371年8月21日),傅友德攻克成都,明朝平定四川。洪武五年四月二十三日(1372年5月26日),廖永忠率明军平定广西,洪武五年六月初三(1372年7月3日),傅友德大败元军,明朝平定甘肃。洪武六年(1373年),太祖鑒於開國元勛多倚功犯法,虐暴鄉閭,特命工部制造鐵榜,鑄上申戒公侯的條令,類似戰國時代的「鑄刑鼎」。洪武八年(1375年),德庆侯廖永忠因僭用龙凤诸不法事,赐死。洪武十二年(1379年),贬右丞相汪广洋于广南,旋赐死。洪武十三年(1380年),胡惟庸案发,左丞相胡惟庸被诛,太祖罢中书省,分中书省之权归于六部,直接归皇帝掌管。洪武十五年(1382年),设立锦衣卫,加强明朝特务统治。1382年1月6日,明军在云南昆明附近大败元朝军队,元梁王自杀,1382年4月7日,蓝玉、沐英攻克大理,段氏投降,明朝平定雲南。洪武十八年(1385年),郭桓案发,由于涉案人员甚多,太祖將六部左右侍郎以下官员皆處死,各省官吏死於獄中達數萬人以上。

洪武二十三年(1390年),李善長的家奴盧仲謙告發李善長與胡惟庸往來勾結,以「狐疑觀望懷兩端,大逆不道」見誅,接續又誅殺陸仲亨與唐勝宗、費聚、趙庸三名侯爵,株連被殺的功臣及其家屬共計達三萬餘人,連「浙東四先生」(刘基、宋濂、章溢、叶琛)亦不能免,并頒布《昭示奸黨錄》。洪武二十六年(1393年),藍玉被錦衣衛指揮蔣瓛密告謀反,史称“藍玉案”。此案牵连到十三侯、二伯,連坐族誅達一萬五千人,明朝建国功臣因此案幾乎全亡。此時太祖又頒布《逆臣錄》,詔示一公、十三侯、二伯。洪武二十七年(1394年),太祖杀江夏侯周德兴以及颖国公傅友德,在捕鱼儿海战役中立功的定远侯王弼亦被赐死。洪武二十八年(1395年),开国六公爵最後一位僅存者冯胜被杀。

在处理内政同时,太祖亦多次籌劃北伐蒙古以保障北方邊塞的安寧,大勝。並曾成功在甘肅擊敗王保保(1372年)、在东北逼降納哈出(1387年)、在蒙古高原幾乎活捉元主脫古思帖木兒(1388年)。同时太祖进军辽东,使朝鮮王朝等归顺(1388年)。

洪武三十一年閏五月初十日(1398年6月24日),朱元璋崩逝於南京皇宮內,享壽七十歲,在位三十一年。與已故的妻子馬皇后兩人一起長眠於南京紫金山明孝陵。《明朝小史·卷三》載,責殉諸妃,強迫伺寝宫人尽数殉葬。《彤史拾遺記》記載,太祖以四十六妃陪葬孝陵,其中所殉,惟宮人十數人。

新任皇帝惠宗遵照遺命。洪武三十一年六月甲辰,上謚曰“欽明啟運俊德成功統天大孝高皇帝”,廟號太祖。永樂元年六月十一日丁巳,增諡“聖神文武欽明啟運俊德成功統天大孝高皇帝”。嘉靖十七年十一月朔,改諡“開天行道肇紀立極大聖至神仁文義武俊德成功高皇帝”。到了清朝,康熙帝历次南巡必跪拜孝陵,曾立碑「治隆唐宋」赞誉其功。中華民國建立初,孫文至孝陵祭告朱元璋。

朱元璋一直以來都是以猛治国。持正面評價者通常都是從其大力打擊貪污,恢復經濟著眼,歷史記載朱元璋是少數極力勤政的皇帝;而持負面評價者,則多從其高壓統治著眼,以猛著称,他的“重典治国”思想不只為遏制官僚腐败。亦顯現在清洗权贵势力、以特務錦衣衛控制政治、又用文字獄及廷杖大臣,以立帝王權威。

明初沿袭元朝制度,设立中書省,置左、右丞相。甲辰正月,初置左、右相國,其中李善長為右相國,徐達為左相國。洪武元年(1368年),改為左、右丞相。由中书省统六部,但不设置中書令。

洪武十三年(1380年),胡惟庸案之后,太祖罢中书省,分中书省之权归于六部。原中書省官屬盡革,惟存中書舍人。至此,秦、漢以降實行一千六百餘年的宰相制度自此廢除,相權與君權合而為一,施行軍權、行政權、監察權三權分立的國家體制。

由於國家事務繁多,皇帝無法處理,洪武十五年九月罷四輔官,仿宋殿閣制設內閣。內閣只為皇帝的顧問,雖無宰相之名,但有宰相之實。此外他仍沿用元朝制度,在中央設置吏、戶、禮、工、刑、兵六部。并設立都給事中六人,分吏、戶、禮、工、刑、兵六科,每科一人;此外建立五寺包括大理寺、太常寺、光祿寺、太僕寺、鴻臚寺等五寺制度。此外他還沿襲元的監察制度,設立御史台,有左右御史大夫各一名;不久改為都察院,下設若干監察御史,負責監督各級官吏。除此他还颁布《大明律》等,对官吏管理进行规制。

为了加强对臣民的控制和监视,太祖设置了巡检司和锦衣卫。巡检司主要是负责全国各地的关津要冲的把关盘查,缉捕盗贼,盘诘伪奸;锦衣卫则负责秘密侦察大小官吏活动,随时向皇帝报告不公不法之徒。同时太祖还授予锦衣卫侦察、缉捕、审判、处罚罪犯等一切大权,锦衣卫正式成为直屬皇帝的情报机构。

太祖出身貧寒,對政治貪污尤其憎惡,其對貪污腐敗官員處以極嚴厲的處罰。太祖在政期間,大批不法貪官被處死,包括開國將領朱亮祖,女婿駙馬都尉歐陽倫,其中甚至因為郭桓案、空印案殺死數萬名官員。由於太祖的吏治嚴厲,在明初相當長一段時間,官員腐敗的情況得到有效遏制。然而,随着大明江山逐步稳定,再加上军事和皇室贵族战功大,享有很高的社会特权,不少人迅速腐化变质。。朱元璋开展雷厉风行的肃贪运动,历时之久、措施之严、手段之狠、刑罚之酷、杀人之多,为几千年历史所罕见。尽管朱元璋反贪决心大、力度猛、出奇招,使腐败现象得到一定程度的遏制,也一度取得了“阶段性的成果”,但還是未能達到徹底清除人類貪欲權位腐敗的本性。

太祖性格多疑,對功臣有所猜忌,恐其居功枉法,圖謀不軌。这些特权阶级杀人伤人、霸占土地、逃税漏税、恃强凌弱、奸淫妇女、吃喝嫖赌、贪污纳贿,甚至造刀枪、穿龙袍的都有。面对这种对王朝的长治久安构成严重威胁局面,太祖把这些特权阶级无情地清洗。廖永忠和 朱亮祖 先後死於非命。隨後太祖以擅權枉法之罪名殺胡惟庸,又殺御史大夫陳寧、御史中丞塗節等人。之後李善長亦被牽連,家屬七十餘人被殺,總計株連者達三萬餘人。此後的藍玉案中,連坐被族誅的有一萬五千餘人。但紀非錄所記載太祖的兒子諸藩王犯有很多暴行,太祖則只是輕微勸戒了事。太祖還通過設立錦衣衛(洪武二十年废除)、詔獄、廷杖等機構或制度,打擊功臣、特務監視等一系列方式加強皇權控制。

太祖遵古制,王命法:三十受兵、六十歸兵。國有三軍,所以誡非常,伐無道,尊宗廟,重社稷,安不忘危。太祖令諸藩鎮守天下,又各領兵權,這固然是親親之情,信任無以復加,卻也未必就沒有帝王心術。強藩林立,能做皇帝的卻始終只有一個,諸藩勢力犬牙交錯,必然相互牽制,相互監視,除非朝廷中樞衰弱之極。當中樞真的衰弱至極時,就算沒有藩王,也會被權臣取而代之。自三皇五帝,以一介布衣而成天子者,唯漢高祖與太祖,其他帝王,大都是前朝重臣或一方豪強而黃袍加身。所以由自己子孫取代無能之君,也勝過將江山付與外人之手,如此可保朱家數百年江山。

建国伊始,太祖就在《大明律令》的基础上制订颁行《大明律》,紧接着又亲自编定《明大诰》。1397年,太祖下詔正式颁布了《大明律》。《大明律》一共四百六十卷,分吏、户、礼、刑、兵、工六律,简于《唐律》,严于《宋律》。《大明律》规定:“谋反”、“谋大逆”者,不管主犯还是从犯,一律凌迟,祖父,父、子、孙、兄、弟以及同居的人,只要是年满十六岁的都要处决。太祖立法一为治民,二为治吏,尤其是《明大诰》对贪官污吏的处决也十分严厉,可以视为反贪刑事特别法。只要是犯有贪污的官吏,一经查实,一律发配北方荒漠中充军,赃至六十两以上者枭首示众,仍剥皮实草。

太祖十分重視法律宣传,寫了大誥三編和大誥武臣,让臣民熟悉法律,不去犯禁。他還經常法外施刑,動輒凌遲。

早在朱元璋起兵时,他就多次强调军纪。他认为「攻克城池用武力,平定混乱用仁政」,杀人并非「勇猛」。要求部队不许滥杀无辜,还给予俘虏优待;同时还要求部队爱护百姓,不得随意焚烧抢掠乱杀百姓,他严令:「掠夺老百姓财物者处死,拆毁老百姓住房的处死。」由于朱元璋部队的军纪严明,朱元璋赢得了部属的尊重,也赢得了民众的支持。

明代早期軍隊的來源,有諸將原有之兵,有元兵及群雄兵歸附的,有獲罪而謫發的,而最主要的來源則是籍選,是由戶籍中抽丁而來。除此之外尚有簡拔、投充及收集等方式。洪武十三年(1380年),太祖廢除大都督府,並改为中军、左军、右軍、前军、后军等五军都督府。洪武十七年(1384年),太祖在全國的各軍事要地,設立軍衛,由都督府管理。一衛有軍隊五千六百人,其下依序有千戶所、百戶所、總旗及小旗等單位,各衛所都隸屬於五軍都督府,亦隸屬於兵部,有事從征調發,無事則還歸衛所。軍隊來源為世襲的軍戶,由每戶派一人為正丁至衛所當兵,軍人在衛所中輪流戊守以及屯田,屯田所得以供給軍隊及將官等所需。五军都督府有统兵权但无调兵权,兵部有调兵权而无统兵权,兩者互相制衡,互不統轄,各自與兵部直接聯繫,最後奏請皇帝裁定,以避免權力過大。

明代軍户是世襲制,一旦列入軍籍,世代都是軍人,朝廷有事要為朝廷作戰。軍丁一旦逃亡、病故、老疾或被虜,就要按軍籍所造之册,到該軍丁原籍追補本身或其親屬,以補足原數。

元朝初期,元世祖曾经遠征日本,导致日本念念不忘,于是终元之世,日本不与中国同好。明朝开国以后,太祖就派使臣持国书去日本、高丽、安南、占城四国,宣告元朝已经灭亡,现在的稱霸中国是大明,應奉大明为“正朔”来朝贡。高丽、安南、占城三国太祖使赴明称臣朝贺,惟独日本没有任何反应。令太祖更为恼火的是,不但日本人不来朝称臣,而且“乘中国未定,日本率以零服寇掠沿海”。同时,被太祖消灭的张士诚、方明珍等残部多逃亡海上,占据岛屿,勾結倭寇出没海上掳掠财货,辽宁、山东、福建、浙江、广东,“滨海之地,无岁不受其害”。

後来太祖喝令“日本国王”處理倭寇,结果使者被日本人殺害。消息傳回中國後,太祖大為怒火,批日本是“国王无道民为贼”的“跳梁小丑”。面对日本,太祖忍下了恶气,从此以后对日本使者一概驅逐處理,朝贡也一概拒绝接受,与日本不相往来。同时,太祖把朝鲜、日本、大琉球、小琉球、安南、真腊、暹罗、占城、苏门答腊、西洋、爪哇、彭亨、百花、三佛齐、勃泥等15国列为“不征诸夷”,写入《皇明祖训》,告诫子孙这些「蛮夷国家」如果不主动挑衅,就不许征伐。

公元1370年(洪武三年)太祖派遣莱州知府赵秩远赴日本。懷良親王经过赵秩的阐释明处外交政策打消了顾虑。不久懷良派遣僧人祖来跟随赵秩回明朝向进表笺。公元1371年(洪武四年)太祖派遣僧人祖阐、克勒等八人送日使归国,从此明朝和日本建立了外交关系。

公元1392年(洪武二十五年)七月,高丽大将李成桂发动兵变掌控高丽局势以后遣知密直司事赵胖至明朝礼部上表:“定昌府院君瑶权署国事,及今四年。瑶又昏迷不法,疏斥忠正,昵比谗邪,变乱是非,谋陷勋旧,谄惑佛神,妄兴土木,靡费无度,民不堪苦;子奭痴佁无知,纵于酒色,聚会群小,谋害忠直。又其臣郑梦周等潜成奸计,欲生乱阶,乃将勋臣李成桂、赵浚、郑道传、南訚等谮于权署国事,令有司论劾以致谋害,国人愤怨,共诛梦周。权署国事尚不悛改,又谋杀戮。举国臣民实虑社稷生灵俱被其害,惶惧失措,无可奈何,咸以为若所为难以主斯民奉社稷。洪武二十五年七月十二日,以恭愍王妃安氏之命,退居私第。窃念军国之务不可一日无统,择于宗亲,无有可当舆望者,惟门下侍中李成桂泽被生灵,功在社稷,中外之心夙皆归附。于是一国大小臣僚、闲良、耆老、军民臣等咸愿推戴,令知密直司事赵胖,前赴朝廷奏达,伏启照验,烦为闻奏,俯从舆意,以安一国之民。”太祖通过礼部传达圣旨:“三韩臣民既尊李氏,民无兵祸,人各乐天之乐,乃帝命也。虽然,自今以后慎守封疆,毋生谲诈,福愈增焉。尔礼部以示朕意。”李成桂遣门下侍郎赞成事郑道传赴京谢恩,并献马六十匹。

当年八月,李成桂又遣前密直使赵琳赴京进表:“权知高丽国事臣李成桂言:伏惟小邦自恭愍王无嗣薨逝之后,辛旽子禑冒姓窃位者十有五年矣。迄至戊辰春,妄兴师旅,将犯辽东,以臣为都统使,率兵至鸭绿江。臣窃自念小邦不可以犯上国之境,谕诸将以大义,即与还师,禑乃自知其罪,逊位子昌。昌亦暗弱,难以莅位,国人启奉恭愍王妃安氏之命,以定昌府院君王瑶权署国事。瑶乃昏迷不法,紊乱刑政,狎昵谗佞,贬斥忠良,臣民愤怨,无所控告。恭愍王妃安氏深虑其然,命归私邸。于是一国大小臣僚、闲良、耆老、军民等以为军国之务不可一日无统,推戴臣权知军国事。臣素无才德,辞至再三,而迫于众情,未获逃避,惊惶战栗,不知所措。伏望皇帝陛下以乾坤之量、日月之明,察众志之不可违、微臣之不获已,裁自圣心,以定民志。”朱元璋再通过礼部复旨:“高丽限山隔海,天造东夷,非我中国所治。尔礼部回文书,声教自由,果能顺天意合人心,以妥东夷之民,不生边衅,则使命往来,实彼国之福也。文书到日,国更何号,星驰来报。”

当年十一月,李成桂再遣艺文馆学士韩尚质至明朝上表:“窃念小邦王氏之裔瑶,昏迷不道,自底于亡,一国臣民推戴臣权监国事。惊惶战栗,措躬无地间,钦蒙圣慈许臣权知国事,仍问国号,臣与国人感喜尤切。臣窃思惟,有国立号诚非小臣所敢擅便。谨将“朝鲜”(箕子所建古国名)、“和宁”(李成桂诞生之地)等号闻达天聪,伏望取自圣裁。”太祖再通过礼部复旨:“东夷之号,惟朝鲜之称美,且其来远,可以本其名而祖之。体天牧民,永昌后嗣。”李成桂遣门下侍郎赞成事崔永沚谢恩,又遣政堂文学李恬送明朝颁赐的给前朝的高丽国王之印,并请更己名为“李旦”。

公元1394年(洪武二十七年)帖木儿帝国向明朝贡马,而且致国书。第二年,明朝派遣兵科给事中傅安率领使团往报。但当傅安等抵达帖木儿帝国国都撒马尔罕时,帖木儿打算要向东兴兵,攻打明朝了,于是扣押了傅安等人,而且百般的诱惑傅安等人归顺帖木儿,傅安被扣押十三年,坚贞不屈,维护明朝的尊严。一直到了帖木儿死了以后,他的孙子哈里嗣位,想和明朝和好,于是才放傅安等人回国。傅安回国以后又出使了中亚诸国。

公元1395年(洪武二十八年)十一月,李成桂遣艺文春秋馆太学士郑总赴京请诰命印章:“洪武二十五年七月十五日,差知密直司事赵胖奏达天庭,继差门下评理赵琳奉表陈奏,钦奉圣旨,许允权知国事。准奉礼部来咨內云:‘国更何号,星驰来报。准此。’即差知密直司事韩尚质赍擎奏本赴京,钦奉圣旨节该:‘东夷之号,惟朝鲜之称美,且其来远矣,可以本其名而祖之。钦此。’除钦遵外,洪武二十六年三月初九日,差门下评理李恬送纳前朝高丽国王金印,又于当年十二月初八日准奉左军都督督府咨,钦奉圣旨內一款节该:‘即合正名。今既改号朝鲜,表文仍称权知国事,未审何谋?钦此。’一国臣民战栗惶惧,咸请国王钦遵施行。见今虽称国王名号,窃缘未蒙颁降诰命及朝鲜国印信,一国臣民日夜颙望,仰天吁呼。伏请照验,烦为闻奏,乞赐颁降国王诰命及朝鲜印信施行。”朱元璋通过礼部下旨拒绝:“今朝鲜在当王之国,性相好而来王,顽嚣狡诈,听其自然,其来文关请印信诰命,未可轻与。朝鲜限山隔海,天造地设,东夷之邦也,风殊俗异。朕若赐与印信诰命,令彼臣妾,鬼神监见,无乃贪之甚欤?较之上古圣人,约束一节决不可为。朕数年前曾敕彼仪从本俗,法守旧章,令听其自为声教。喜则来王,怒则绝行,亦听其自然。尔礼部移文李成桂,使知朕意。”

明朝立国后日本因进入南北朝的大分裂时期后出现的、大量外出掠夺的武士阶层为主的倭寇骚扰入侵的恐惧,明政府立国后采取了一系列针对海患的闭关锁国政策:洪武三年(1370),明政府“罢太仓黄渡市舶司”;洪武七年(1374),明政府下令撤销自唐以来即存在的、负责海外贸易的福建泉州、浙江明州、广东广州三市舶司,中国对外贸易遂告断绝;洪武十四年(1381),太祖以倭寇仍不稍敛足迹,又下令禁濒海民私通海外诸国,此后每隔一两年即将该海禁政策再次昭示天下。自此,连与明朝素来交好的东南亚诸国也不能来华进行贸易和文化交流。

整个海禁政策从太祖开始,到了明穆宗在位期間被以“市通则寇转而为商,市禁则商转而为寇”为由实行开关(隆庆开关);至清初又开始一連串的闭关,清高宗時更推行“一口通商”政策、直至鸦片战争后,通行整个明清二代的海禁政策才被彻底打破。

元末之际,中國發生多次大規模的災荒饑饉疾病和瘟疫,以及連年戰爭,期間生产遭到严重破坏,人口也大量減少,经济全面崩溃,人民处在流离失所的过程中。大明建立並統一全國後,面对哀鸿遍野、饿殍满路的凄凉局面,太祖實行黃老治術治國,太祖说:天下初定,百姓财力困难,就像刚刚会飞的鸟不可拔羽,才种的树不可摇根一样。现在必须采取这种政策,同时主张藏富于民。

农业是明代社会最主要的生产部门。太祖在恢复和发展社会经济中,把发展农业放在了首位,为了保证农业第一线有足够的劳力资源。太祖通令全國,地主不得蓄养奴婢,所养的奴婢一律释放为良民。凡因饥饿而典卖为奴者,由朝廷代为赎身;嚴格控制寺院的發展,明令各州府县只能有一个大寺院,禁止四十歲以下的妇女当尼姑,严禁寺院收养童僧,二十岁以上的青年如果要是出家,必须得到父母和官方同意,出家后三年内还要赴京考試,不合格者潜发为民。這些政策的实施,使得社會增加了一只庞大的劳动力大軍。

全國的農業生產在大規模战争而遭受極大破壞的背景下得到很大程度的恢復,加上太祖在位期間大規模向淮河以北和四川的荒無之地、墾荒填充移民,使人口得以穩定增長。

此外他也實行屯田政策,軍屯面積佔全國耕地的近十分之一。此外,商屯也相當盛行,政府以買賣食鹽的專賣證(稱之為鹽引)作為交換,利誘商人將糧食運往邊疆,以確保邊防的糧食需求。明太祖也曾派遣國子監下鄉督導水利建設、赈灾,並以減免稅賦獎勵耕作。這些措施使得過去很多飽受戰亂損毀的地區恢復了生氣,使明朝的經濟得到了快速的恢復。

到洪武二十六年(1393年),全國有6500萬人,其中民戶佔6175萬人,軍戶佔325萬人。另外,其為了動員全社會,明太祖十分重視戶口普查,每個人有固定的義務。人民分為軍戶(弓兵、校尉、力士)、匠戶、民戶(马户、陵戶、茶戶、柴戶、阴阳戶、医戶)、灶戶,不允许隨便轉換工作,匠籍、軍籍比一般民戶地位低,不得應試,並要世代承襲。若想脫離原戶籍極為困難,須經皇帝特旨批准方可。各种活动也要引憑才合法。编成里甲,规 定了路引制度,也就是通行证制度。普通百姓只要走出出生地百里之外,就得持有官府开具的通行证,否则就以逃犯论处。

明朝初期實行「科舉必由學校」的政策,太祖多次強調:「古昔帝王育人材,正風俗,莫不先於學校。」明代洪武元年(1368年),詔開科舉,對制度、文體都有了明確要求。命令刘三吾等人刪節《孟子》中民貴君輕的內容,課試不以命題,科舉不以取士。。洪武年間,太祖共主持举办六次科考,七次发榜,共取一甲21名、二甲223名、三甲686名,合930名,平均每科取士155人,為明朝選拔輸送了大量有學識的官員,包括練子寧、黃子澄、解縉等一代名相。洪武三十年科舉時,因中進士者均為南方籍。太祖将试官二十餘人指為胡黨藍黨凌遲殺害,并自阅试卷,取中六十一人,皆为北方人,并于六月廷试。此外,他並將學校列為「郡邑六事之首」,以官學結合科舉制度推行程朱理學,并設立國子監等重要教育機構。由於太祖在位期間實行高壓的吏治政策,明初诗文三大家不得善終,後世不乏有學者主張太祖曾實行過一些文字獄。也有學者指出關於朱元璋嗜殺之事例,存有穿鑿附會的問題。

太祖崇尚简朴,也希望老百姓也勤俭节约。他规定靴子上不能有任何装饰。同时对于全国人民怎么穿衣;每个阶层佩戴什么样的首饰;盖什么样的房子;出行坐什么样的车子以及人们的行动举止也是朱元璋关注的焦点,因而制定了一系列规章制度,包括了生活的方方面面,其细致入微,可谓空前绝后。“洪武二十二年三月二十五日奉聖旨:“在京但有軍官軍人學唱的,割了舌頭;下棋打雙陸的,斷手;蹴圓的,卸脚;作買賣的,發邊遠充軍。”府軍衛千戶虞讓男虞端故違吹簫唱曲,將上脣連鼻尖割了。又龍江衛指揮伏顒與本衛小旗姚晏保蹴圓,卸了右脚,全家發赴雲南。又二十五年九月十九日,禮部榜文一款:“內使剃一搭頭,官民之家兒童剃留一搭頭者,閹割,全家發邊遠充軍。剃頭之人,不分老幼罪同。””(《客座贅語》卷十)

太祖对天下老年人施以尊重,颁布《存恤高年诏》。洪武二十年,太祖怕有关部门执行不力,就又叮嘱礼部尚书,要以皇帝的名义再次重申一下这项政策。在朝廷的要求和带动下,各地形成了尊老养老的风气,赡养老人的要求也渗透到各地家法族规之中。

对于社会的救济朱元璋也十分重视,洪武时期,荒政则受到朝廷高度重视。朝廷除了拨付救灾济贫款项,还侧重加强民众抗灾自救能力。面对天灾侵袭,朱元璋积极作为,既树立了朝廷的负责任形象,又增强了政府的凝聚力,赢得了民心。救灾济贫实为获取民心、形成治世的重要前提,为“洪武之治”的出现夯实了经济社会基础。

为了贬抑商人,太祖他特意规定,农民可以穿绸、纱、绢、布四种衣料。而商人却只能穿绢、布两种料子的衣服。商人考学、当官,都会受到种种刁难和限制。

太祖建立明朝前后,十分重视宗教问题,通过协调儒释道三者的关系,既稳定了局面,又争取了人心,为巩固明朝政权奠定了思想和群众基础。通过有效的宗教管理措施,把宗教的发展始终控制在适合自己的政治需要范围内,并利用宗教教化番荑,不断扩大自己的势力范围,为明政权创造了良好的国际环境。

在政治上,太祖推重儒释道三教并举的政策。他说:“尝闻天下无二道,圣人无两心。三教之立,虽持身荣俭之不同,其所济给之理一。”他极为重视佛教的辅政作用,将佛教事务视为朝中大事,对佛教制度、僧寺清规多方整饬,期望以此整顿僧团,去淤除垢,“振扬佛法以善世”。

洪武六年(1373年),太祖下诏对出家的僧尼免费发放度牒,才使得唐朝年间流传下来使的“度牒银”制度全部废除。

整顿僧团秩序,防止僧俗混淆,洪武二十四年,朱元璋还制定颁布了影响深广的《申明佛教榜册》,要求各地僧司查验清理天下僧寺,欲还俗者听其还俗,使出家僧人恪受戒律清规,禅、讲、瑜伽,各归本宗。

太祖亲自制定的“御制至圣百字赞”以及明皇室关于修建清真寺和保护清真寺宗教职业人员的谕旨,在一定程度上肯定了回族的宗教生活。

殉葬制度,在西漢初以後,逐漸在中原政權消失。朱元璋二子秦王對人民暴行(見御製紀非錄)被宮人殺死,即連坐迫秦王諸妃自殺。明朝時期明孝陵以四十六妃陪葬,其中有太祖死時殺死殉葬十几名侍寢宮人,這一制度沿襲至成祖、仁宗、宣宗、代宗。而“節烈從殉”的風氣,並向下廣為延伸至宗室公侯、官宦之家、以至民間,直至近百年之後其五世孫英宗死前指出殉葬非古禮,仁者所不忍,才禁殉葬于遺詔,永著為典。按朱元璋創立的制度,嬪妃殉葬由皇帝親臨作別。正統初,明英宗目睹皇父嬪妃殉葬,受很大刺激。天順年間下詔廢止。殺死從殉婦女的方法為將她們縊死,或勒死,或灌以水银毒死。这些生殉的妇女被称为“朝天女”,她們的家屬稱為“朝天女戶”,並給予一定待遇。關於朝天女記載主要依賴朝鮮的第一手資料《李朝實錄金黑口述》。寶慶公主生母張玄妙,以其女幼,得免殉葬。

《明清史事沉思录》中记载,“传谓男子宫刑,妇人幽闭,皆不知幽闭之义。今得之,乃是于牝(阴户)去其筋,如制马、豕之类,使欲火消减。国初常用此,而女往往多死,故不可行也。”对这种灭绝人性的手术,这本书的作者王春瑜评论道:“将人等同畜生处置,始作俑者其无后乎!”

明孝陵康熙題碑:“治隆唐宋”。

清朝官修正史《明史》张廷玉等对明太祖朱元璋最终能够成就帝业的评价是:“帝天授智勇,统一方夏,纬武经文,为汉、唐、宋诸君所未及。当其肇造之初,能沉几观变,次第经略,绰有成算。尝与诸臣论取天下之略,曰:‘朕遭时丧乱,初起乡土,本图自全。及渡江以来,观群雄所为,徒为生民之患,而张士诚、陈友谅尤为巨蠹。士诚恃富,友谅恃强,朕独无所恃。惟不嗜杀人,布信义,行节俭,与卿等同心共济。初与二寇相持,士诚尤逼近。或谓宜先击之。朕以友谅志骄,士诚器小,志骄则好生事,器小则无远圖,故先攻友谅。鄱阳之役,士诚卒不能出姑苏一步以为之援。向使先攻士诚,浙西负固坚守,友谅必空国而来,吾腹背受敌矣。二寇既除,北定中原,所以先山东、次河洛,止潼关之兵不遽取秦、陇者,盖扩廓帖木儿、李思齐、张思道皆百战之余,未肯遽下,急之则并力一隅,猝未易定,故出其不意,反旆而北。燕都既举,然后西征。张、李望绝势穷,不战而克,然扩廓犹力抗不屈。向令未下燕都,骤与角力,胜负未可知也。’帝之雄才大略,料敌制胜,率类此。故能戡定祸乱,以有天下。语云‘天道后起者胜’,岂偶然哉。” 清朝官修正史《明史》张廷玉等对明太祖朱元璋一生事业的评价是:“赞曰:太祖以聪明神武之资,抱济世安民之志,乘时应运,豪杰景从,戡乱摧强,十五载而成帝业。崛起布衣,奄奠海宇,西汉以后所未有也。惩元政废弛,治尚严峻。而能礼致耆儒,考礼定乐,昭揭经义,尊崇正学,加恩胜国,澄清吏治,修人纪,崇凤都,正后宫名义,内治肃清,禁宦竖不得干政,五府六部官职相维,置卫屯田,兵食俱足。武定祸乱,文致太平,太祖实身兼之。至于雅尚志节,听蔡子英北归。晚岁忧民益切,尝以一岁开支河暨塘堰数万以利农桑、备旱潦。用此子孙承业二百余年,士重名义,闾阎充实。至今苗裔蒙泽,尚如东楼、白马,世承先祀,有以哉。”

毛泽东在1964年3月24日,在一次听取汇报时的插话中对明太祖朱元璋、汉高祖刘邦、元太祖成吉思汗的治国能力评价如下:“可不要看不起老粗。”“知识分子是比较最没有知识的,历史上当皇帝的,有许多是知识分子,是没有出息的:隋炀帝,就是一个会做文章、诗词的人;陈后主、李后主,都是能诗善赋的人;宋徽宗,既能写诗又能绘画。一些老粗能办大事:成吉思汗,是不识字的老粗;刘邦,也不认识几个字,是老粗;朱元璋也不识字,是个放牛的。” 毛泽东对明太祖朱元璋的军事才能评价如下:“自古能君无出李世民之右者,其次则朱元璋耳。” 給吳晗提意見:“朱元璋是农民起义领袖,是应该肯定的,应该写的(得)好点,不要写的(得)那么坏。”

趙翼曾説:“藉諸功臣以取天下,及天下既定,即盡取天下之人而殺之,其殘忍實千古所未有。”“蓋明祖之性,實帝王,豪傑,盗賊兼而且也。”

商传评价朱元璋:「朱元璋出身于一个贫苦家庭,从社会最底层的放牛娃、四处讨饭的小和尚,全靠自己的奋斗成了一个统一王朝的开国皇帝。这是中国历史上,乃至世界历史上绝无仅有的事情。另外,朱元璋当上皇帝后,也没有停止步伐,他在位三十多年,成功地建立一个强大统一的明帝国」。

\subsection{洪武}

\begin{longtable}{|>{\centering\scriptsize}m{2em}|>{\centering\scriptsize}m{1.3em}|>{\centering}m{8.8em}|}
  % \caption{秦王政}\
  \toprule
  \SimHei \normalsize 年数 & \SimHei \scriptsize 公元 & \SimHei 大事件 \tabularnewline
  % \midrule
  \endfirsthead
  \toprule
  \SimHei \normalsize 年数 & \SimHei \scriptsize 公元 & \SimHei 大事件 \tabularnewline
  \midrule
  \endhead
  \midrule
  元年 & 1368 & \tabularnewline\hline
  二年 & 1369 & \tabularnewline\hline
  三年 & 1370 & \tabularnewline\hline
  四年 & 1371 & \tabularnewline\hline
  五年 & 1372 & \tabularnewline\hline
  六年 & 1373 & \tabularnewline\hline
  七年 & 1374 & \tabularnewline\hline
  八年 & 1375 & \tabularnewline\hline
  九年 & 1376 & \tabularnewline\hline
  十年 & 1377 & \tabularnewline\hline
  十一年 & 1378 & \tabularnewline\hline
  十二年 & 1379 & \tabularnewline\hline
  十三年 & 1380 & \tabularnewline\hline
  十四年 & 1381 & \tabularnewline\hline
  十五年 & 1382 & \tabularnewline\hline
  十六年 & 1383 & \tabularnewline\hline
  十七年 & 1384 & \tabularnewline\hline
  十八年 & 1385 & \tabularnewline\hline
  十九年 & 1386 & \tabularnewline\hline
  二十年 & 1387 & \tabularnewline\hline
  二一年 & 1388 & \tabularnewline\hline
  二二年 & 1389 & \tabularnewline\hline
  二三年 & 1390 & \tabularnewline\hline
  二四年 & 1391 & \tabularnewline\hline
  二五年 & 1392 & \tabularnewline\hline
  二六年 & 1393 & \tabularnewline\hline
  二七年 & 1394 & \tabularnewline\hline
  二八年 & 1395 & \tabularnewline\hline
  二九年 & 1396 & \tabularnewline\hline
  三十年 & 1397 & \tabularnewline\hline
  三一年 & 1398 & \tabularnewline
  \bottomrule
\end{longtable}


%%% Local Variables:
%%% mode: latex
%%% TeX-engine: xetex
%%% TeX-master: "../Main"
%%% End:

%% -*- coding: utf-8 -*-
%% Time-stamp: <Chen Wang: 2021-11-01 17:11:06>

\section{惠宗朱允炆\tiny(1398-1402)}

\subsection{生平}

建文帝朱允炆(1377年12月5日-?),或稱明惠宗,是明朝第二代皇帝,年號“建文”,明太祖朱元璋之孫。在位期間進行一系列寬政、削藩的改革,史稱“建文改制”。由於燕王朱棣發動靖難之變攻入南京應天府,是為明成祖,朱允炆下落不明。大臣梅殷私諡其為「神宗孝愍皇帝」但成祖不承認,故不使用,甚至明成祖不認為朱允炆是合法皇帝,故明朝人大多稱之為建文君。直到南明時,弘光帝追谥其為“嗣天章道诚懿渊功观文扬武克仁笃孝让皇帝”,庙号“惠宗”。清高宗乾隆元年,高宗追谥其為「恭閔惠皇帝」,故也作「明惠帝」。

朱允炆是懿文太子朱标第二子,嫡母太子妃常氏所生長子朱雄英早故,另有一子朱允熥为其弟。嫡母常氏在1378年逝世后,朱允炆生母吕氏成为继任太子妃,所以明太祖朱元璋就視朱允炆為嫡長孫。

洪武二十五年(1392年),父亲朱标病死,朱允炆被祖父朱元璋立为皇太孙。由於自幼熟讀儒家經書,所近之人多懷理想主義,性情因此與其父同樣溫文儒雅,即長皆以寬大著稱。洪武二十九年,朱允炆曾向太祖請求修改《大明律》,他參考《禮經》及歷朝刑法,修改《大明律》中七十三條過份嚴苛的條文,深得人心。

朱允炆出生时脑袋长得颇偏,朱元璋用手摸着说:“半边月儿。”一年除夕,他与父亲朱标陪同朱元璋,朱元璋叫他父子作詠月诗,朱允炆作诗曰:“谁将玉指甲,掐作天上痕。影落江湖里,蛟龙不敢吞。”朱元璋看后默然不语。

明洪武三十一年(1398年)閏五月,明太祖朱元璋去世,死前密命驸马梅殷辅佐新君。朱允炆在同月(6月30日)即位,定次年(從1399年2月6日开始)為建文元年。建文帝在六月晉用齊泰為兵部尚書、黃子澄為太常寺卿,七月召方孝孺為翰林院侍講,在國事上倚重三人。建文帝的年號“建文”有別於其祖父的洪武,他不想仿效祖父以嚴刑峻法治國,即位後改行寬政,囚犯人數減至洪武時期的三成左右。

建文帝能虚心纳谏。一次他因病上朝晚了,监察御史尹昌隆对此提出批评,左右建议他说出自己染病,建文帝却认为这样的谏言难得,不但没有自辩,还表扬了尹昌隆,公开了他的奏疏。

明太祖為鞏固皇室,大封宗室為藩王,各擁私人護衛軍隊。對建文帝來說,諸藩王大多為其叔輩,且在封地掌握兵權,心中由是不安。建文帝為皇太孫時曾問黃子澄曰:「諸王尊屬擁重兵,多不法,奈何?」子澄回答說諸王軍力不足以抗衡朝廷。建文帝即位後,下令各王國的地方文武官員聽朝廷節制,採取削藩政策,先後废黜周王、湘王、齐王、代王及岷王。在部署對付年齡最長、軍功最多、武力最强大的燕王朱棣時,由於建文帝身邊的謀士多缺乏實際的政治經驗,以致打草驚蛇,引發了燕王先發制人的念頭。朱棣在權衡利害之後,於建文元年(1399年)七月在封地北平起兵反叛。他以“靖难”為名,向京師進軍。

建文元年,明建文帝下詔討伐燕軍。命吳傑、吳高、耿瓛、盛庸、潘忠、楊松、顧成、徐凱、李友、陳暉、平安分道併進,并在河北真定設立平燕布政使司,兵部尚書暴昭掌司事。隨後,耿炳文率大軍抵達,與燕軍交戰后失利退守。明建文帝臨時換將,撤換耿炳文,由李景隆代替。隨後,朱棣獲得寧王朱權及朵顏三衛,實力大增。而李景隆在率軍圍困北平后,仍然無法破城,并在鄭村壩潰敗。燕王因此向明惠帝上書,明惠帝不得不罷免齊泰、黃子澄。

建文二年,燕軍與中央軍在白溝河大戰,李景隆再次潰敗并逃亡濟南,隨後再在濟南潰敗。然而,朱棣卻無法攻破山東參政鐵鉉、都督盛庸的濟南城,不得不撤軍。明惠帝隨後封賞鐵鉉、盛庸,但卻不誅殺李景隆。同年冬,燕軍再次進犯濟寧,盛庸擊敗并斬殺燕將張玉,并接連獲勝。建文三年,兩軍在河北山東一帶屢次交戰,并互有勝負,最後燕軍攻入真定。

建文四年,何福、平安率領的中央軍在小河大勝燕軍,并斬其將陳文;而徐輝祖亦在齊眉山獲得大捷。燕軍恐懼后計劃北歸。恰逢建文帝誤以為燕軍已經北撤,召徐輝祖班師,致使何福孤軍奮戰。隨後,靈璧之戰中,燕軍大勝,陳暉、平安、陳性善、彭與明被執。盛庸軍亦在淮河之戰中潰敗,燕軍遂渡過淮河,抵達六合。建文帝不得不下詔要求各地勤王,并遣使割地罷兵。同年六月,盛庸在浦子口與燕軍交戰不利,都督僉事陳瑄率水軍附燕。隨後,朱棣率燕軍渡江,最終逼近南京應天府。谷王朱橞與李景隆開金川門變節,致使燕軍進入都城。宮中起火,建文帝不知下落。

燕王朱棣入京师应天府后,建文帝在宫中举火,皇后焚死,建文帝本人及其太子朱文奎则不知所踪,至今其下落仍是未定論的历史之谜。有稱其從地道逃亡,也有別史稱其離宮後出家為僧。

朱棣入京後,先捕殺齊泰、黃子澄、方孝孺及大批忠于建文帝的官員後,方稱皇帝,是為明成祖。當時駙馬都尉梅殷在軍中,從黃彥清之議,為建文帝發喪,諡「孝愍皇帝」,廟號「神宗」,但是不被成祖承認。

雖然朱棣宣稱在宮中找到建文帝的屍體,並為他舉行葬禮,但朱棣對建文帝未死的傳言不敢掉以輕心。建文帝年仅2岁的幼子朱文圭被废为庶人,并囚禁于凤阳广安宫。建文帝的三个弟弟原本封为亲王,尚未就藩,朱棣将他们降为郡王;年长的朱允熥和朱允熞先被封至福建漳州和江西建昌,旋被召回京师(南京),以“不能匡正建文帝”为由废为庶人,并囚禁于凤阳,只留下年幼的朱允熙给朱标奉祀,而不久之后朱允熙也于永乐四年死于火灾。

溥洽是建文帝主錄僧,當時傳聞他知道建文帝出逃的事,朱棣遂以其它罪名囚禁溥洽長達十餘年,直到姚廣孝病危時請求朱棣釋放溥洽,溥洽才獲釋。

明成祖即位后,不承认建文帝的正统性,下令销毁建文朝史料,并先后三次修改明太祖实录。成祖还下令作《奉天靖难记》,对懿文太子及建文帝多加诋毁。

正統五年,有僧自雲南至廣西,詭稱建文皇帝。隨後被逮捕調查,乃是鈞州人楊行祥,隨後下獄而死。同行十二位僧侶,皆戍遼東。隨後,雲南、貴州、四川等地均相傳有帝為僧時往來跡。正德、萬曆、崇禎年間,諸位大臣請求續封建文帝,及加廟諡,均未成行。虽然《太宗实录》(成祖原廟號太宗),称建文帝被朱棣以天子礼下葬,但崇祯帝在位时却亲口承认建文并无陵墓。崇禎十七年(1644年)五月,南明的弘光帝在南京即位,於同年七月為建文帝君臣平反,上庙号「惠宗」,谥号为「嗣天章道诚懿渊功观文扬武克仁笃孝让皇帝」。清朝乾隆元年,乾隆帝詔廷臣集議,追諡曰「恭閔惠皇帝」,故後世也稱建文帝為「明惠帝」。

2008年1月,福建省宁德市金涵乡上金贝村发现的一个和尚墓被认为是惠宗的墓葬所在,这个墓穴也是迄今为止福建发现的最大的和尚墓。然而,建文帝的最終下落至今仍是不解之謎,一说建文帝藏身于湖南省永州市新田县。

明建文帝在登基后不久,即重新選拔六部官員,其中大量官员在靖難之役中死亡;在战事中陣亡、拒絕與燕王朱棣合作而自殺或不屈而亡,其中包括禮部尚書陳廸,兵部尚書齊泰、鐵鉉,刑部尚書暴昭、侯泰,左都御史景清,右都御史練子寧、翰林院方孝孺等。



\subsection{建文}

\begin{longtable}{|>{\centering\scriptsize}m{2em}|>{\centering\scriptsize}m{1.3em}|>{\centering}m{8.8em}|}
  % \caption{秦王政}\
  \toprule
  \SimHei \normalsize 年数 & \SimHei \scriptsize 公元 & \SimHei 大事件 \tabularnewline
  % \midrule
  \endfirsthead
  \toprule
  \SimHei \normalsize 年数 & \SimHei \scriptsize 公元 & \SimHei 大事件 \tabularnewline
  \midrule
  \endhead
  \midrule
  元年 & 1399 & \tabularnewline\hline
  二年 & 1400 & \tabularnewline\hline
  三年 & 1401 & \tabularnewline\hline
  四年 & 1402 & \tabularnewline
  \bottomrule
\end{longtable}


%%% Local Variables:
%%% mode: latex
%%% TeX-engine: xetex
%%% TeX-master: "../Main"
%%% End:

%% -*- coding: utf-8 -*-
%% Time-stamp: <Chen Wang: 2021-11-01 17:11:20>

\section{成祖朱棣\tiny(1402-1424)}

\subsection{生平}

明成祖朱棣(1360年5月2日-1424年8月12日),或稱永樂帝,是明朝第三代皇帝,公元1402年至1424年在位,在位二十二年,年号永乐。

明太祖皇四子,安徽凤阳人,生于应天府(今江苏南京),時事征伐,並受封為燕王。洪武三十二年或建文元年(1399年)建文帝削藩,燕王遂發動靖难之役,起兵奪位,經過三年的战争,最終胜利,殺害方孝孺,驅逐其姪建文帝奪權篡位自封為帝。明成祖在位期间,改善明朝政治制度,发展经济,开拓疆域,迁都北京,使北京至此成為中國的政治中心至今。此外他编修《永乐大典》,派遣鄭和下西洋,北征蒙古,南平安南。明成祖的统治时期被称为永乐盛世,明成祖也被后世称为「永乐大帝」。另外,他加強太祖以來的專制統治,強化錦衣衛並成立東廠,此外,他在位期間重用宦官,也促成明朝中葉後宦官專政的禍根。

明成祖崩逝后谥号「体天弘道高明广运圣武神功纯仁至孝文皇帝」,庙号「太宗」,葬于长陵。嘉靖十七年(1538)九月,嘉靖帝改谥为「启天弘道高明肇运圣武神功纯仁至孝文皇帝」,改上庙号为「成祖」。

(1360年)四月十七日(5月2日),朱棣生于应天府(今南京)。

明太祖洪武三年(1370年),朱棣十岁,受封燕王。曾居鳳陽,对民情颇有所知。洪武十三年(1380年),朱棣就藩燕京北平,之后多次受命参与北方军事活动,两次率师北征,曾招降蒙古乃兒不花,並曾生擒北元大將索林帖木兒,加强了他在北方军队中的影响。朱元璋晚年,長子太子朱标、次子秦王朱樉、三子晋王朱棡皆早於朱元璋去世,故朱元璋於洪武三十一年閒五月駕崩後,四子朱棣不仅在军事实力上,而且在家族尊序上都成为诸王之首。

建文帝朱允炆登基後,為了提防燕王造反,於洪武三十一年十二月派工部侍郎張昺為北平布政使,都指揮使謝貴、張信為北平都指揮使。隨後又命都督宋忠屯兵駐開平,并調走北平原屬燕王管轄的軍隊。

建文元年(1399年),朱棣裝病,使建文帝把作為人質的朱棣三子朱高熾、朱高煦、朱高燧回燕國;之後由於屬下被朝廷處死,遂裝瘋。由於王府長史葛誠告知朝廷,裝瘋被發覺。

時燕王遣使入金陵奏事,使者被齊泰等審訊,被迫供出燕王的異狀,於是朝廷下密旨,令張昺、謝貴逮捕燕王府的官屬,張信逮捕燕王本人。但張信經過考慮,將此事告知朱棣。於是朱棣和僧人姚道衍等進行舉兵的謀劃,令張玉、朱能將八百勇士帶入府中潛伏,以待變故。

張昺、謝貴得到皇帝密詔后,七月初四帶兵包圍了燕王府。朱棣假意將官屬全部捆縛,請二人進王府查驗。二人進府后,朱棣派出府內的死士將其擒獲,并連同府內叛變的葛誠、盧振一同斬殺。當日夜裡,朱棣攻下北平九門,遂控制北平城。

燕王朱棣起兵,援引《皇明祖訓》,號稱清君側,指惠帝身邊的齊泰和黃子澄為奸臣(謀害皇室親族),需要鏟除,稱自己的舉動為「靖難」(意為「平定災難」),并上書於惠帝朱允炆。

燕軍控制北平后,七月初六,通州主動歸附;七月初八,攻破薊州,遵化、密雲歸附;七月十一,攻破居庸關;七月十六,攻破懷來,擒殺宋忠等;七月十八,永平府(今河北盧龍縣,屬秦皇島市)歸附。七月二十七,為防止大寧軍隊從松亭關偷襲北平,用反間計使松亭關內訌,守將卜萬下獄。至此,北平周圍全部掃清。燕軍兵力增至數萬。

燕軍攻破懷來後,由於領地相距太近,七月二十四日,谷王朱橞逃離封地宣府(今屬張家口,距北京約150公里,距懷來約60公里),奔京師。八月,齊泰等顧慮遼王、寧王幫助燕王,建議召還京師;遼王從海路返京,而寧王不從,遂削寧王護衛。宋忠失敗後,部將陳質退守大同。代王本欲起兵呼應朱棣,被陳質所控制,未果。

七月,朱棣反書到京,朱允炆削朱棣屬籍,廢為庶人。決定起兵討燕。在真定(今河北正定)置平燕布政使司。

耿炳文率軍在八月十三日到達真定,并分兵於河間、鄚州(河北任丘北约30里)、雄縣,為犄角之勢。在經過觀察後,八月十五日,燕軍趁中秋夜敵軍不備,偷襲雄縣;成功後又利用伏擊擊敗了鄚州的援兵,遂攻克鄚州,收編剩餘的部隊。八月二十四日,燕軍到達無極縣。從樵夫和中央軍被俘士兵處得知敵情,於是燕軍發動決戰。

二十五日,燕軍趁耿炳文送使臣出城時偷襲中央軍,炳文逃回城中后,怒而迎戰。在燕軍主力與耿炳文軍相持時,朱棣親自率軍襲擊其側翼,耿炳文大敗潰逃,中央軍投降三千多人。中央軍狼狽逃回城中,城池差點失守。部將李堅、甯忠、顧成等被俘;士兵被殺、被俘數萬人(后放還)。耿炳文率殘部不到十萬人在真定堅守不出,燕軍攻城三天不克。八月二十九日,燕軍返回北平。顧成降燕之後,留在北平協助燕世子朱高熾守城。

耿炳文戰敗,朱允炆開始擔憂戰事,考慮換將。黃子澄說曹國公李景隆是名將李文忠之子,建議他接任;齊泰反對,但惠帝不聽。八月三十日,拜李景隆為大將軍,誓師出征,并召回耿炳文。李景隆以德州為大本營,調集各路兵馬包括耿炳文敗兵,增兵至五十萬人,九月十一日進至河間。

朱棣聽說朝廷以五十萬傾國之兵交付李景隆,大喜過望,說:「李景隆不會用兵,給他五十萬大軍,根本是自取滅亡。趙括之失必然重演,我軍必勝。」

九月初一江陰侯吳高率辽东兵攻打永平郡,九月廿五,攻陷永平郡,決定趁勢偷襲大寧(今內蒙古寧城)以獲得其精銳部隊;另一方面利而誘之,將中央軍引至「空城」北平下。九月廿八,出師。。十月初六,燕軍經小路到達大寧城下。朱棣單騎入城),見寧王朱權,向朱權求救。在居大寧期間,朱棣令手下吏士入城結交并賄賂大寧的軍官等。十月十三,朱棣提出告辭,朱權在郊外送行,伏兵盡起,大寧軍紛紛叛變,歸附朱棣。於是朱權與王妃、世子等一同隨朱棣前往北平,而大寧的全部軍隊(包括其騎兵精銳朵顏三衛)都被朱棣收編。大寧成為空城。朱棣實力大增。十月十九,燕軍在會州整編,分立五軍(中前左右後)。十月廿一,入松亭關。

十一月初五,渡白河(時已結冰,渡河處在今北京順義區東),打敗李景隆的哨探陳暉部隊萬餘人。李景隆大敗。李景隆令鄭村壩所有軍隊輕裝撤退。。燕軍輕易擊潰城下的敵軍,獲得大量物資。。此戰中央軍喪師十餘萬。十一月初九,朱棣回到北平城,再次上書,惠帝不應。。十二月十九日,朱棣出師攻打大同。十二月廿四,抵達廣昌,守將楊宗投降。建文二年(1400年)正月初一,燕軍抵達蔚州,守將王忠、李遠投降。二月初二,燕軍攻大同。李景隆前来救援。李景隆走出紫荊關后,燕軍從居庸關返回北平。中央軍兵力、裝備大量損失,士氣受到重創。

建文二年四月,李景隆從德州,郭英、吳傑等從真定誓師北伐兵力增至六十萬。燕軍亦出。四月二十日,燕軍渡過玉馬河。四月廿四,燕軍戰鬥失利。。次日(四月廿五),再次交战。。。四月廿七,燕軍進攻德州。初九,燕軍進入德州。五月十五,燕軍攻濟南,李景隆逃走。燕軍遂圍濟南。十月,朝廷召李景隆回南京。黃子澄、練子寧、葉希賢等上書,請求立斬李景隆。朱允炆不聽。。鄭村壩之戰和白溝河之戰,使得两军攻守形勢逆轉。

燕軍圍濟南。右參政鐵鉉、盛庸堅守。朱棣射信入城招降,未果。五月十七,燕軍掘開河堤,放水灌城。鐵鉉決定派千人詐降,誘朱棣進城。朱棣圍城攻打三個月。六月,惠帝遣使求和,朱棣不聽。七月,平安進軍河間,擾亂燕軍糧道。八月十六,朱棣撤兵回北平。盛庸、鐵鉉追擊,大敗燕軍,收復德州。

建文二年十月,朱棣決定再度南下,十月廿七到達滄州。燕軍僅用兩天就攻下滄州,徐凱等投降。燕軍自長蘆渡河,十一月初四到達德州。朱棣招降盛庸未果,遂南下。十一月,燕軍到達臨清,焚其糧船。燕軍從館陶渡河,先後到達東阿、東平,威脅南方,迫使盛庸南下。盛庸在東昌(今山東聊城)決戰。十二月廿五,燕軍至東昌。朱棣仍然親自率軍衝鋒,盛庸開陣將朱棣誘入,然後合圍,張玉被中央軍包圍戰死。次日,燕軍再次戰敗,遂北還。在擊退中央軍的阻截后,建文三年正月十六,燕軍返回北平。

朱棣與姚廣孝商議,姚廣孝強烈支持再次出兵。二月十六,朱棣再次出師。三月二十日,燕軍探知盛庸在夾河(今河北省衡水市武邑縣附近,漳河支流)駐紮,於是駐紮在距對方四十里的地方。三月廿二,燕軍進兵夾河。。朱棣率領一萬騎兵和五千步兵攻擊盛庸軍左翼,不能入。此時燕將譚淵望見已經開戰,於是主動出兵攻打。朱棣、朱能等則趁中央軍調動產生的混亂,趁暮色向中央軍後方猛攻,斬殺莊得。此戰殺傷相當,但燕軍損失了大將譚淵。當夜,朱棣率領十餘人在盛庸營地附近露宿;次日(三月廿三)清晨,發現被中央軍包圍。朱棣再次利用禁殺之旨,引馬鳴角,穿過敵軍,揚長而去。中央軍愕然,不敢射箭。

朱棣回到營中,鼓勵眾將「兩軍相當,將勇者勝」,於是再次會戰,雙方互有勝負。戰鬥打了七八個小時后,盛庸大敗,損失了數萬人,退回德州。吳傑、平安引兵準備會合盛庸,聞庸已敗,退回真定。夾河之戰結束。夾河之戰重新確立了燕軍的優勢。閏三月初四,朱允炆因夾河之敗,再次罷免齊泰、黃子澄,謫出京城,暗中令其募兵。

擊敗盛庸后,朱棣進軍真定。。閏三月初九,兩軍會於藳城交戰。。次日,復戰,南軍不能支,大敗而去。。朱棣將射成刺猬的軍旗送回北平,令世子朱高熾妥善保存,以警示後人。從白溝河、夾河到藳城,燕軍三次得大風相助而勝,朱棣認為這是天命所在,非人力所能為。夾藳之戰再次使南軍損失慘重,正面戰場戰事稍緩和。南軍改為通過談判、反間、襲擊後方等方式間接作戰。擊敗平安後,燕軍南下,先後經過順德、廣平、大名,并駐紮於大名。諸郡縣望風而降。

朱棣聽說齊黃被貶,上書和談,表示「奸臣竄逐而其計實行,不敢撤兵」。朱允炆得書,與方孝孺討論,方孝孺表示可以借此機會遣使回報,拖延時間,并懈怠其軍心;同時令遼東等軍隊攻其後方,以備夾攻。於是(四月)惠帝令大理寺少卿薛嵓去見朱棣,傳詔并秘密在軍中散佈相關消息。薛嵓見朱棣,說「朝廷言殿下旦釋甲,暮即旋師。」朱棣表示這連三尺小兒也騙不過。薛嵓無言以對。五月初一,盛庸、吳傑、平安等分兵騷擾燕軍餉道。朱棣遣使者進京表示盛庸等不肯罷兵,必有主使。惠帝聽從方孝孺的意見,將其下獄(一說誅殺),和談破裂。

朱棣見和談破裂,從濟寧南下,成功焚燒大量中央軍糧船,京師大震,德州陷入窘境。

七月,燕軍進攻彰德,林縣投降。七月初十,平安自真定趁虛攻北平,擾其耕牧。朱高熾固守。朱棣分兵回援;(九月十八)平安與戰不利,退回真定。由於河北戰事不利,方孝孺想出了反間計,利用朱高熾(長子)和朱高煦(次子)的矛盾,先寫一封信給守北平的高熾,令其歸順朝廷,許以燕王之位;然後派人告訴朱棣和高煦(隨軍)世子密通朝廷,以使燕軍北還。但朱高熾得到信後,根本沒有拆開,將朝廷使者連人帶信一起送往朱棣處。反間計失敗。

七月十五,盛庸令大同守將房昭入紫荊關威脅保定,據易州西水寨以窺北平。朱棣回兵救援。朱棣分兵守保定,并包圍房昭的山寨。十月初二,燕軍與真定援兵和房昭軍決戰,房昭退回大同。十月廿四,燕軍回到北平。之後又擊敗了襲永平的遼東敵軍。

建文三年冬,南京有宦官因犯錯被處罰,逃到朱棣處,告知南京守備空虛。朱棣遂決定直接率兵南下,臨江一決。道衍亦支持不再與盛庸、平安等糾纏,直趨京師。

1401年(建文三年十二月初二),燕師復出。十二月十二,到達蠡縣(約在保定以南50公里)。建文四年(1402年)正月,燕軍南下至館陶渡河,長驅直入。正月十四,陷東阿;正月十五,陷東平;正月十七,陷汶上;正月廿七,陷沛縣(進江蘇);正月三十,到達徐州。惠帝見燕軍再次出動,三年十二月令駙馬都尉梅殷(惠帝的姑父,顧命大臣)任總兵官,鎮淮安。建文四年正月初一,將遷往蒙化的朱橚(廢周王)召回南京。命魏國公徐輝祖率兵援山東。

二月初一,何福、平安、陳暉進兵濟寧,盛庸進兵淮上。二月廿一,朱棣擊敗徐州的出戰軍隊,徐州自此閉城死守。朱棣繼續南下。三月初一,燕軍進逼安徽宿州。三月初九,抵達渦河(今安徽蚌埠市懷遠縣以北)。平安帶兵來追;但三月十四日在淝河中了朱棣所設的伏兵,只得退回宿州。三月廿三,朱棣遣將斷徐州餉道,鐵鉉等率兵圍攻,互有勝負。四月十四,燕軍進達睢水之小河,搭浮橋。次日,平安、何福領軍奪橋,雙方隔河僵持。數日後,中央軍糧盡,朱棣決定偷襲。半夜,渡河繞至敵後;四月廿二,雙方戰於齊眉山(靈壁縣西南三十里),中央軍大勝,斬燕將李斌。

燕軍陷入窘境。四月廿三,燕軍眾將要求北返,朱棣不同意,說「欲渡河者左,不欲者右。」大部份人站於左側,朱棣怒。朱能這時強力支持朱棣,表示「漢高祖十戰九不勝,卒有天下」,堅定了燕軍堅持的決心。

這時,朝廷訛傳燕軍已兵敗,京師不可無良將,遂召回徐輝祖。四月廿五,考慮到在河邊不易防守,何福移營,與平安在靈壁(一作靈璧)深溝高壘作長遠之計。由於糧道被燕軍阻礙,平安親自率兵六萬護衛糧草。四月廿七,朱棣率精銳襲擊平安,將其一分為二;何福全軍出動救援,朱高煦也率伏兵出現,何福敗走。

中央軍缺糧,何福與平安決定次日(廿九)突圍而出,在淮河取得給養,號令為三聲炮響;次日,燕軍攻打靈壁墻壘,進攻信號正巧也是三聲炮響。於是中央軍以為是己方號炮,紛紛奪路而逃;燕軍趁勢進攻,中央軍全軍覆沒。靈壁之戰就此意外結束。此戰燕軍生擒了陳暉、平安、馬溥、徐真、孫成等三十七員敵將,四名內官(宦官),一百五十員朝廷大臣,獲馬二萬餘匹,降者不計其數。只有何福單騎逃走。

靈璧之戰後,燕軍向東南方向直線前進。五月初七下泗州,朱棣謁祖陵。盛庸在淮河設下防線阻礙燕軍渡河,朱棣在嘗試取道淮安、鳳陽受阻後,遣朱能、丘福率士兵數百人繞道上游乘漁船渡河,五月初九從後方突襲盛庸,盛庸敗走。燕軍遂克盱眙。

五月十一,燕軍向揚州方向前進,五月十七到達天長(揚州西北50公里)。守揚州的監察御史王彬本想抵抗,但屬下反叛,趁其沐浴時綁縛之。五月十八,揚州不戰而降。隨後高郵歸降。

揚州失陷,金陵震動。朱允炆驚慌不已,與方孝孺商議後,先後定下如下幾個救急方法:下罪己詔;號召天下勤王;派練子寧、黃觀、王叔英等外出募兵;召回被貶黜的齊泰、黃子澄;遣人許以割地求和,拖延時間。。

五月廿二,朱允炆遣慶成郡主(朱元璋的侄女、朱棣的堂姐)與朱棣談判,表示願意割地。朱棣說「此奸臣欲姑緩我,以俟遠方之兵耳。」郡主無言以對,遂返。

六月初一,燕軍準備從浦子口渡江,但遇到了盛庸最後的抵抗。燕軍戰不利,此時朱高煦引兵來援,殊死力戰,擊敗盛庸。隨後南軍的一支水軍部隊降燕,燕軍遂於六月初三自瓜洲渡江,并再次擊敗退守此地的盛庸。六月初六,燕軍至鎮江,守將率城投降。

六月初八,燕軍駐紮於龍潭(距京師金陵東約30公里),朝廷大震。朱允炆徘徊殿間,召方孝孺問計。方孝孺表示城中尚有二十萬兵,應堅守待援;即使真戰敗,國君為社稷而死,是理所應當的。可以再派大臣、在京諸王前往談判以拖延時間。於是六月初九,派李景隆、茹瑺等見朱棣,再次談判;朱棣表示割地無名,只要奸臣。六月初十,遣谷王朱橞(建文元年逃回京城)、安王朱楹等第三次前往談判,無果。

六月十二,外出募兵的大臣們仍未返回,朱允炆只得派在京諸王和武臣們守衛各門。時左都督徐增壽(徐達子,輝祖弟)謀內應,被一群文官圍毆。

次日(1402年7月13日),燕軍抵金陵。徐增壽作內應,事敗,被朱允炆親自誅殺於左順門。守衛金川門(位於南京城西北面)的朱橞和李景隆望見朱棣麾蓋,開門迎降。

燕軍進南京,朱允炆見事不可為,遂在皇宮放火。馬皇后死於大火,朱允炆本人不知所終;此後其下落成為謎團。朱棣入城。

朱棣进入南京,出榜安民,成为了明朝第三位皇帝。朱棣进城之时,翰林院編修楊榮迎於馬首,說:「殿下先謁陵乎?先即位乎?」一语點醒朱棣。次日(建文四年六月十四日)起,諸王及文武群臣多次上表勸進,朱棣不允。

數日後(七月十七日),朱棣謁明孝陵,并於當日登基即位,改元永樂,是為明成祖。明成祖重建奉天殿(舊殿被朱允炆所焚),刻玉璽。同年十一月十三日,封王妃徐氏為皇后。

朱棣登基称帝后,对靖难功臣进行了封赏。封王两人,为:朱能(东平武烈王);张玉(河间忠武王)。封公二十二人,为:丘福(淇国公);徐增寿(定国公);陈亨(泾国公);郭亮(兴国公);李彬(茂国公);李遠(莒国公);柳升(融国公);徐忠(蔡国公);袁容(沂国公);郑亨(漳国公);姚广孝(荣国公);张信(郧国公);王聪(漳国公);顾成(夏国公);张武(潞国公);陳珪(靖国公);薛禄(鄞国公);王真(宁国公);吴允诚(凉国公);李讓(景国公);孟善(滕国公);張輔(英国公)。封侯十五人,为:陳瑄(平江侯);何福(宁远侯)李濬(襄城侯);孙岩(应成侯);房宽(思恩侯);王友(清远侯);王忠(靖安侯);劉榮(广宁侯);火真(同安侯);王寧(永春侯);宋晟(西宁侯);郭义(安阳侯);谭渊(崇安侯);柳升(安远侯);薛绶。封伯十八人,为:陈贤(荣昌伯);陈旭(云阳伯);刘才(广恩伯);张兴(安乡伯);房胜(富昌伯);徐理(武康伯);徐祥(兴安伯);金玉(会安伯);高士文(建平伯);陈志(遂安伯);唐云(新昌伯);茹瑺(忠诚伯);王佐(顺昌伯);许诚(永新伯);薛斌(永顺伯);薛贵(安顺伯);赵彝(忻城伯);朱荣(武进伯)。

明成祖登基后不承認建文年號,七月初一(一說六月十八日),將建文元、二、三、四年改為洪武三十二至三十五年,次年改元永乐元年。凡建文年間貶斥的官員,一律恢復職務(如靖難初期因離間被貶的江陰侯吳高被再次起用,守大同);建文年間的各項改革一律取消;建文年間制定的各項法律規定,凡與太祖相悖的,一律廢除。但一些有利於民生的規定也被廢除,如建文二年下令減輕洪武年間浙西一帶的極重的田賦,至此又變重。

明成祖在靖難之役結束后,为了佐证他“清君侧”的起兵宣言,向金陵軍民發布公告:「諭知在京師的軍民人等,我先前一向守望我藩的封地,卻因奸臣弄權作威作福,導致我家骨肉被其殘害,所以不得不起兵誅殺他們,乃是要扶持社稷和保安宗親、藩王。今次研擬安定京城,有罪的奸臣我不敢赦免,無罪者我也不敢濫殺,如有小人藉機報復,擅作綁縛、放縱、掠奪等事情因而禍及無辜,並非我的本意。」

建文四年六月廿五,明成祖誅殺齊泰、黃子澄、方孝孺等建文帝大臣,滅其族。其中據記載,方孝孺被誅十族(九族加朋友門生),受牽連而死者共873人,充軍等罪者千餘人,當中被救的倖存者有假借余姓逃過一劫的方孝孺的幼子方德宗。而因黃子澄受牽連的有345人。景清降後密謀行刺,事敗,八月十二被殺,滅九族;後屠其家鄉,謂「瓜蔓抄」。

此外,眾多建文舊臣如卓敬、暴昭、練子寧、毛泰、郭任、盧植、戴德彝、王艮、王叔英、謝升、丁志方、甘霖、董鏞、陳繼之、韓永、葉福、劉端、黃觀、侯泰、茅大芳、陳迪、鐵鉉等等也都被酷刑處死或自盡,史稱:「忠憤激發,視刀鋸鼎鑊甘之若飴,百世而下,凜凜猶有生氣。」他們的家屬和親人也被牽連,死者甚眾。被流放、逼作妓女及被其它方式懲罰的人也不少。明仁宗即位後,大部份人始獲赦免,而餘下的人的後代卻遲至明神宗時始獲赦免。建文帝被朱棣篡位後,朝野為之盡忠死節者甚眾,不及備載。

在大肆誅殺之外,當月,明成祖將忠於建文帝的魏國公徐輝祖下獄,但顧及其父是中山王徐達,其姊即成祖仁孝文皇后,還是釋放了他,僅削其爵位。輝祖死後,其子嗣魏國公爵。黃觀被明成祖所嫉恨,其狀元的身份被革去,故明代保持三元及第記錄的只有商輅一人。耿炳文、盛庸、平安(靈壁之戰降)、何福、梅殷等将领投降後都受到迫害自杀身亡。

永乐初,明成祖为了安抚诸位藩王,稳定国内局势,同时表示自己和建文帝的不同,曾先后复周、齐、代、岷诸親王旧封;建文帝的弟弟吴王朱允熥、衡王朱允熞、徐王朱允𤐤尚未就藩,明成祖皆降为郡王,同年又将已就藩的朱允熥、朱允熞召到燕京,以不能匡正建文帝为由废为庶人,软禁于凤阳,仅留朱允𤐤奉祀懿文太子,而朱允𤐤不久也于永乐四年死于火灾。当其皇位较巩固时,继续实行削藩。周、齐、代、岷诸王再次遭到削夺;迁宁王于南昌;徙谷王于长沙,旋废为庶人;削辽王护卫。

在政治上,明成祖继续实行太祖的徙富民政策,以加强对豪强地主的控制。明成祖时期,完善了文官制度,在朝廷中逐渐形成了后来内阁制度的雏形。永乐初开始设置內閣,选资历较浅的官僚入阁参与机务,解决了废罢中书省后行政机构的空缺。朱棣重视监察机构的作用,设立分遣御史巡按天下的制度,鼓励官吏互相讦告。他善利用宦官出使、专征、监军、分镇、刺臣民隐事。

明成祖即位之初,对洪武、建文两朝政策进行了某些调整,提出“为治之道在宽猛适中”的原则。他利用科举制及编修书籍等笼络地主、士人,宣扬儒家思想以改变明初過事佛、道教之风,选择官吏力求因才而用,为当时政治、经济、军事、文化等方面的发展奠定了思想和组织基础。

在全国局势稳定之后,明成祖为了加强对大臣的监控,恢复洪武时废罢的锦衣卫。同时,明成祖又设置镇守内臣的东厂衙门,厂卫合势,强化专制统治。

永乐十八年(1420年),明成祖為了鎮壓政治上的反對力量,觉得锦衣卫不足以达成目的,決定設立一個稱為「東緝事廠」,簡稱“東廠”的新衙門,地點位於燕京(今北京)東安門之北,一說東華門旁。(今北京东城区东厂胡同,據說系原东厂所在地。)

東廠的行政長官為欽差掌印太監,全稱職銜為:欽差總督東廠官校辦事太監,簡稱提督東廠,尊稱為「廠公」或「督主」。初設時由司禮監掌印太監兼任,後因事務繁雜,改由司禮監秉筆太監中位居第二、第三者擔任。東廠的屬官有掌刑千戶、理刑百戶各一員,由錦衣衛千戶、百戶來擔任,稱貼刑官。隸役(稱掌班、領班、司房,共四十餘人)、緝事(稱役長和番役)等軍官由锦衣卫撥給。

明初《大明律》明令:「凡樂人搬做雜劇戲文,不許妝爾扮帝王后妃、忠臣節烈、先聖先賢神像,違者杖一百。官民之家容扮者與同罪」,以壓迫雜劇創作,明成祖即變本加厲,以極刑來禁止此類雜劇的印賣:「但有褻瀆帝王聖賢之詞曲、駕頭雜劇,非該律所載者,敢有收藏、傳誦、印賣,一時拿送法司究治」,「但這等詞曲,出榜後,限他五日,都要乾淨,將赴官燒毀了,敢有收藏的,全家殺了」。

明成祖十分重視經營北方,加之自己兴起于北平(今北京),明成祖在南京即位后,于永乐元年改北平為行在,設六部,增設北京周圍衛所,逐漸建立起北方新的政治、軍事中心。永乐七年(1409年),明成祖开始了營建北京天壽山長陵,以示立足北方的決心。與此同時,爭取與蒙古族建立友好關係。韃靼、瓦剌各部先後接受明政府封號。永乐八年(1410年)至二十二年(1424年),朱棣親自率兵五次北征,鞏固了北部邊防。永乐十四年(1416年)開工修建北京宮殿也就是紫禁城(但後來部分宮殿被李自成放火燒毀,清初又重新修復)。永乐十九年(1421年)正式遷都,定鼎北京。

明成祖注意社会经济的恢复与发展,认为“家给人足”、“斯民小康”是天下治平的根本。他大力发展和完善军事屯田制度和盐商开中则例,保证军粮和边饷的供给。在中原各地鼓励垦种荒闲田土,实行迁民宽乡,督民耕作等方法以促进生产,并注意蠲免赈济等措施,防止农民破产,保证了赋役征派。

明成祖对各地方官吏要求极为严格,要求凡地方官吏必须深入了解民情,随时向朝廷反映民间疾苦。永乐十年(1412年),朱棣命令入朝觐见的地方官吏五百余人各自陈述当地的民情,还规定“不言者罪之,言有不当者勿问’。之后,永乐帝宣布“谕户部,凡郡县有司及朝使目击民艰不言者,悉逮治。”即地方官或中央派出的民情观察员,如果看到民间疾苦而不实报的,要逮捕法办。对民间发生了灾情,地方上要及时赈济,做到“水旱朝告夕振,无有雍塞”。通过这些措施,永乐时“赋入盈羡”,达到有明一代最高峰,史称永乐盛世。

西南边疆,永乐十一年(1413年),平定思南、思州土司叛亂後,設立貴州布政使司。為加強對烏思藏(今西藏)地區的控制,朱棣派遣官吏迎番僧入京,給予封賜,尊為帝師。不過,史學界對明朝是否實際統治了西藏存在較大的爭議。

永乐年间,明朝在藏区建立一套僧官制度,僧官分教王、西天佛子、大国师、国师、禅师、都纲、喇嘛等,每级依受封者的身份、地位进行分封。如明成祖即位的当年,即派侯显前往乌思藏迎请噶玛噶举派的第五世噶玛巴活佛,后封其为“大宝法王”。1406年,明成祖又遣使入藏封乌思藏帕竹第五任第悉扎巴坚赞为“阐化王”。明封八王中的两大法王、五大教王都是永乐时期封授的。此外,明成祖依僧官制度还进行了大规模的分封,由此明朝对藏区的各政教势力由上至下各级首领的分封基本完成。但明朝并未在烏思藏等地区驻军。亦有学者通过对比元朝对于西藏的实际管辖,认为明朝上面这些对藏人名义上的封授并不能被认为拥有在西藏的实际政治权力。《劍橋中國明代史》亦指出:「無論是在經濟領域,還是在政治領域,西藏人都未覺得他們是明朝廷臣民。另外,他們無須中國(明朝)居中調解而維持著與其他國家和民族的關係。」

东北边疆,永乐七年(1409年)在女真地區,設立奴儿干都司。明成祖永乐元年(1403年)派邢枢等传谕奴儿干,正式招抚诸部,擴大明朝東疆。永乐二年(1404年),置奴儿干等卫所,其后在当地相继建卫所达一百三十餘。永乐七年(1409年)明政府设置奴儿干都指挥使司管辖奴儿干地区的所有军事建制机构。永乐九年(1411年)正式开始行政管辖权。都司的主要官员初为派駐數年而輪調的流官,后为當地部落領袖所世袭。明成祖為了安撫東北女真各部,在歸附的海西女真(位於松花江上游)與建州女真(位於松花江、牡丹江之間)設置衛所,並派宦官亦失哈安撫位於黑龍江下游的野人女真。亦失哈并于1413年视察了库页岛,宣示了明朝对此地的主权。在奴儿干都司官衙所在地附近建有永宁寺,立有永宁寺碑,清代曹廷杰于1885年曾拓回碑文。同时,明成祖撤去大宁都司,将宁王朱权内迁南昌,授予兀良哈蒙古的朵颜、泰宁和福余三个卫所自治权,但不允许三卫蒙古人南迁到大宁地区驻牧。明成祖还于1406年和1422年对兀良哈蒙古进行镇压,以维持这一地区的稳定。

辖区内主要居民为蒙古、女真、吉里迷(尼夫赫人)、苦夷(阿伊努人)、达斡尔等族人民,分置卫所,以各族首领为各卫所都督、都指挥、指挥、千户、百户、镇抚等职,给予印信。据《明史》记载,奴儿干都司有卫三百八十四,所二十四,站七,地面七,寨一。都司治所奴儿干城(元朝征东元帅府旧地,今俄罗斯尼古拉耶夫斯克特林),在黑龙江下游东岸,下距黑龙江口约两百公里,上距吉林船厂约两千五百公里。明宣宗即位后,奴儿干都司于宣德九年(1434年)正式废弃,共持续25年。

西北边疆,永乐四年(1406年)設立哈密衛。此前,察合台的后裔肃王兀纳失里於明洪武十三年(1380年),开始向明朝纳贡,被明太祖封为哈密国王。其子脱脱向明成祖朝贡,永乐四年(1406年)三月,明成祖宣布设立哈密卫,以其头目马哈麻火者等为指挥、千百户等官,又以周安为忠顺王长史,刘行为纪善,辅导。之后,哈密国成为设有明朝羁縻卫所的王国,忠顺王是哈密国王,哈密卫指挥使掌握哈密兵权,另有汉人长史。

同时,明成祖还多次派遣吏部驗封司員外郎陳誠、中官李達等官員出使西域。隨後西域的帖木兒帝國、吐魯番、失剌斯、俺都準、火州也與明朝多次互派使者往來,加強了政治、駐軍和貿易往來,全國統一形勢得到進一步發展和鞏固。

明成祖很重视河工,永乐九年(1411年)朱棣於疏浚會通河為保證北京糧食與各項物資的需要。朱棣命開漕運。漕運在元朝至元年間即有,然而卻因會通河一段水淺而無法大量載運物資,於是元朝均以海運為主。明朝初期,傳餉遼東、北平的途徑也均以海運為主。洪武二十四年,黃河在原武絕口,會通河於是被淤。

永乐年间,明成祖遷都北京,採用河路、海路并運。當時海運危險且多有損失;而河運卻經過淮河轉沙河,然後經過黃河進入衛河,於此轉入北京,陸運須經過八個衛所,勞民傷財。濟寧州同知潘叔正上疏建議浚通會通河,使得元朝運河恢復。於是,朱棣命宋禮、刑部侍郎金純、都督周長前往治理。會通河首要問題為水源不足,宋禮採用汶上老人白英的建議,修築埋城與戴村坝,橫截汶水向南,經河面最高端南旺分水,流入運河,且使黃河不會影響漕運。同年八月還京,論首功,受上賞。

次年,因御史許堪進言衛河水患,朱棣再命宋禮前往治理。宋禮在魏家灣分支黃河,泄水入土河,於是從德州西北開一支支流,到海豐、大沽流入大海。此時,宋禮以海運損失巨大、勞民傷財,上言請求停止海運,而恰逢平江伯陳瑄治理長江、淮河等告竣。於是河運從此昌盛,可運大型物資。永樂十三年,朱棣遂終止海運。

永乐十三年(1415年)鑿清江浦,使大運河重新暢通,對南北經濟文化交流與發展起了重要的作用。

永乐年间,明成祖还派派夏原吉治水江南,疏浚吴淞。

在政治稳定、经济繁荣、边疆稳定的局面下,为整理知识,明成祖令解縉等人修书。編撰宗旨:「凡书契以來经史子集百家之书,至於天文、地志、阴阳、医卜、僧道、技艺之言,备辑为一书,毋厌浩繁!」,召集一百四十七人,首次成书于永乐二年(1404年),初名《文献集成》;明成祖過目後認為「所纂尚多未備」,不甚滿意。永樂三年(1405年)再命姚廣孝、鄭賜、劉季篪、解縉等人重修,這次動用編寫人員朝野上下共二千一百六十九人,啟用了南京文淵閣的全部藏書,永樂五年(1407年)定稿進呈,明成祖看了十分滿意,親自為序,並命名為《永樂大典》,清抄至永樂六年(1408年)冬天才正式成書。

《永乐大典》由解縉、太子少傅姚廣孝和禮部尚書鄭賜監修,組織上設監修、總裁、副總裁、都總裁等職,負責各方面工作。監修:解縉、姚廣孝、鄭賜;總裁:副總裁:蔣用文、趙同友;都總裁:陳濟。

《永乐大典》修書過程對所收錄的書籍沒有做任何修改,採用兼收並取的方式,保持了書籍原始的內容。明成祖修大型類書《永樂大典》,在三年時間內即告完成。《永樂大典》有22877卷,其中凡例、目錄60卷,全書分裝為11095冊,引書達七八千種,字數約有三億七千多萬,且未有任何刪節,這是清朝《四庫全書》無法相提並論的。但成祖并未将《永乐大典》复写刊刻,而决定只制作一份抄本,并于1409年完成。永乐年間修訂的《永樂大典》原書只有一部,現今存世的都是嘉靖年間的抄本。

明成祖时期,为了开展对外交流,扩大明朝的影响,同时确立自己即位的正统性,从永乐三年起,朱棣派三宝太监郑和(初名馬三寶)率领船队六次出使西洋(第七次在明宣宗宣德年间),所历三十余国,成为明初盛事。永乐时派使臣来朝者亦达三十余国。浡泥王和苏禄东王亲自率使臣来中国,不幸病故,分别葬于南京(浡泥国王墓)和德州(苏禄国王墓)。

永乐三年六月十五(1405年7月11日)明成祖命郑和为正使,王景弘为副使率士兵二万八千余人出使西洋,造长44丈广18丈大船62艘,从苏州刘家河泛海到福建,再由福建五虎门杨帆,先到占城(今越南中南部地區),后向爪哇方向南航,次年6月30日在爪哇三宝垄登陆,进行贸易。时西爪哇与东爪哇内战,西爪哇灭东爪哇,西爪哇兵杀郑和士兵170人,西王畏惧,献黄金6万两,补偿郑和死难士兵。随后到三佛齐旧港,时旧港广东侨领施进卿来报,海盗陈祖义凶横,郑和兴兵剿灭贼党五千多人,烧贼船十艘,获贼船五艘,生擒海盗陈祖义等三贼首。郑和船队后到过苏门答腊、满刺加、锡兰、古里等国家。在古里赐其王国王诰命银印,并起建碑亭,立石碑“去中国十万余里,民物咸若,熙嗥同风,刻石于兹,永示万世”。

永乐五年九月初二(1407年10月2日),郑和回国,押陈祖义等献上,陈祖义等被问斩。施进卿被封为旧港宣慰使。旧港擒贼有功将士获赏:指挥官钞一百锭,彩币四表里,千户钞八十锭,彩币三表里,百户钞六十锭,彩币二表里;医士,番火长钞五十锭,彩币一表里,锦布三匹。

永乐六年正月,明成祖命工部造宝船四十八艘。永乐六年九月十三日(1407年10月13日),命太监郑和、王景弘,王贵通等出使古里,满剌加,苏门答剌,阿鲁,加异勒,爪哇,暹罗,占城,柯枝,阿拔把丹,小柯兰,南巫里,甘巴里等国,赐其国王锦绮纱罗,永乐七年夏(1409年)回国。第二次下西洋人数据载有27000人。

永乐七年九月(1409年10月),明成祖命正使太监郑和、副使王景弘、候显率领官兵二万七千余人,驾驶海舶四十八艘,从太仓浏家港启航,敕使占城,宾童龙,真腊,暹罗,假里马丁,交阑山,爪哇,重迦罗,吉里闷地,古里,满剌加,彭亨,东西竺,龙牙迦邈,淡洋,苏门答剌,花面,龙涎屿,翠兰屿,阿鲁,锡兰,小葛兰,柯枝,榜葛剌,不剌哇,竹步,木骨都束,苏禄等国。費信、馬歡等人會同前往。满剌加当时是暹罗属国,正使郑和奉帝命招敕,赐双台银印,冠带袍服,建碑封域为满剌加国,暹罗不敢扰。满剌加九洲山盛产沉香,黄熟香;太监郑和等差官兵入山采香,得直径八九尺,长八九丈的标本6株。永乐七年,皇上命正使太监郑和等赍捧诏敕金银供器等到锡兰山寺布施,并建立《布施锡兰山佛寺碑》此碑現存于科倫坡博物館。郑和访问锡兰山国时,锡兰山国王亞烈苦奈兒“負固不恭,謀害舟師”,被郑和觉察,离开锡兰山前往他国。回程时再次访问锡兰山国,亚烈苦奈儿诱骗郑和到国中,发兵五万围攻郑和船队,又伐木阻断郑和归路。郑和趁贼兵倾巢而出,国中空虚,带领随从二千官兵,取小道出其不意突袭亚烈苦奈儿王城,破城而入,生擒亚烈苦奈儿并家属。

永乐九年六月十六(1411年7月6日),郑和回国獻亚烈苦奈儿与永樂帝,朝臣齐奏诛杀,永樂帝怜悯亚烈苦奈儿无知,释放亚烈苦奈儿和妻子,给予衣食,命礼部商议,选其国人中贤者为王。选贤者邪把乃耶,遣使赍引,诰封为锡兰山国王,并遣返亚烈苦奈儿。永乐九年(1411年)满剌加国王拜里米苏剌,率领妻子陪臣540多人来朝,朝廷赐海船回国守卫疆土。从此“海外诸番,益服天子威德”。八月,礼部、兵部议奏,对锡兰战役有功将士754人,按奇功,奇功次等,头功,头功次等,各有升职,并赏赐钞银,彩币锦布等。

永乐十一年十一月(1413年11月),明成祖命正使太监郑和,副使王景弘等奉命统军二万七千余人,驾海舶四十,出使满剌加,爪哇,占城,苏门答剌,柯枝,古里,南渤里,彭亨,吉兰丹,加异勒,勿鲁谟斯,比剌,溜山,孙剌等国。郑和使团中包括官员868人,兵26800人,指挥93人,都指挥2人,书手140人,百户430人,户部郎中1人,阴阳官1人,教谕1人,舍人2人,医官医士180人,正使太监7人,监丞5人,少监10人,内官内使53人其中包括翻译官马欢,陕西西安羊市大街清真寺掌教哈三,指挥唐敬,王衡,林子宣,胡俊,哈同等。郑和先到占城,奉帝命赐占城王冠带。1413年郑和船队到苏门答剌,当时伪王苏干剌窃国,郑和奉帝命统率官兵追剿,生擒苏干剌送京伏诛。1413年郑和舰队在三宝垄停留一个月整休,郑和费信常在当地华人回教堂祈祷。郑和命哈芝黄达京掌管占婆华人回教徒。首次繞過阿拉伯半島,航行東非麻林迪(肯尼亚),永乐十三年七月初八(1415年8月12日)回国。同年11月,麻林迪特使來中國進獻“麒麟”(即長頸鹿)。

永乐十五年五月十五日(1417年6月)总兵太监郑和受明成祖命,在泉州回教先贤墓行香,往西洋忽鲁谟斯等国公干,永乐十五年五月(1417年6月)出发,护送古里、爪哇、满剌加、占城、锡兰山、木骨都束、溜山、喃渤里、卜剌哇、苏门答剌、麻林、剌撒、忽鲁谟斯、柯枝、南巫里、沙里湾泥、彭亨各国使者及旧港宣慰使归国。隨行有僧人慧信,将领朱真、唐敬等。郑和奉命在柯枝诏赐国王印诰,封国中大山为镇国山,并立碑铭文。忽鲁谟斯进贡狮子,金钱豹,西马;阿丹国进贡麒麟,祖法尔进贡长角马,木骨都束进贡花福鹿、狮子;卜剌哇进贡千里骆驼、鸵鸡;爪哇、古里进贡麾里羔兽。永乐十七年七月十七(1419年8月8日)回国。

宋末泉州市舶司提举蒲寿庚之侄蒲日和,也与太监郑和,奉敕往西洋寻玉玺,有功,加封泉州卫镇抚。

永乐十九年正月三十日(1421年3月3日),郑和奉明成祖命出发,往榜葛剌(孟加拉),史載“於鎮東洋中,官舟遭大風,掀翻欲溺,舟中喧泣,急叩神求佑,言未畢,……風恬浪靜”,中道返回,永乐二十年八月十八(1422年9月2日)回国。永樂二十二年,明成祖去世,仁宗朱高熾即位,以經濟空虛,下令停止下西洋的行動。

永乐二十二年七月十七日(1424年8月12日),明成祖去世,太子朱高炽即位,改元洪熙,是为明仁宗,于洪熙元年五月辛巳(1425年5月29日)去世,太子朱瞻基即位,改元宣德,是为明宣宗。宣德五年闰十二月初六(1430年1月),郑和奉明宣宗命率领二万七千余官兵,驾驶宝船61艘,从龙江关(今南京下关)启航,进行了第七次下西洋。开始返航后,郑和因劳累过度于宣德八年(1433年)四月初在印度西海岸古里去世,遺體埋葬於古里,船队由太监王景弘率领返航,宣德八年七月初六(1433年7月22日)返回南京。第七次下西洋人数据载有27550人。

明太祖朱元璋為與鄰近國家保持長久的和睦關係,便在其所主編的《皇明祖訓》中開列十五個「不征諸夷國名」,以警戒後世子孫切勿「倚中國富強,貪一時戰功,無故興兵,致傷人命」,越南(安南國)便是其中之一。1400年,安南陳朝權臣胡季犛篡位,建立胡朝,改國號為「大虞」。不久後自稱太上皇,由兒子胡漢蒼(即胡𡗨)即皇帝位。由於前朝陳氏原是向明朝稱臣,世世受明冊封,憑著篡奪得國的胡氏為免惹起明朝猜疑,便於1403年農曆四月丁未(西曆4月21日)遣使赴明,向剛起兵奪位的明成祖聲稱陳氏「宗嗣繼絕,支庶淪滅,無可紹承。臣,陳氏之甥,為眾所推」,欲藉此聲稱自己具有統治資格,要求明朝冊封。明成祖派楊渤到越南觀察後,當地陪臣耆老跟隨他向成祖上奏稱「眾人誠心推𡗨權理國事」,明廷一時再沒有懷疑的理由,便封胡漢蒼為「安南國王」。

1404年農曆八月乙亥(西曆9月10日),陳朝遺臣裴伯耆到明廷,控訴胡季犛父子「弒主篡位,屠害忠臣」,要求明朝出兵「擒滅此賊,蕩除奸凶,復立陳氏子孫」 八月丁酉日(西曆10月2日),有一位自稱陳氏子孫,名叫陳天平的人(越南史籍寫作「陳添平」,《大越史記全書》稱他的身份本是「陳元輝家奴阮康」),從老撾入明,亦向明帝訴說胡氏篡位的經過,要求恢復陳氏統治。 其後,明成祖當著胡朝的來使面前,安排陳天平與他們會見,使一眾來使都錯愕下拜,甚至涕泣,適值裴伯耆在場,向來使責以大義,場面緊張。 明廷於是對越南政局多所干涉,派員查核實情,胡朝明白勢不得已,唯有承認責任,要求「迎歸天平」。

另外,明越兩國又因領土問題出現外交風波。1405年,廣西省思明土官及雲南省寧遠州土官向明廷控訴,轄境猛慢、祿州等地被越南所佔。為此,明廷於該年農曆二月,遣使責難胡朝,要求取得祿州,胡朝便被迫將古樓等五十九村交給明朝政府。

胡朝雖然願意息事寧人,但兩國關係仍然緊張。其後,胡朝所派到明廷的使節,都遭扣留,不許回國。明廷又派員入越,查探山川道路險要之地,以為日後南征的準備。 另外,胡朝的南鄰占城,曾於1404年遣使入明,聲稱遭到胡氏「攻擾地方,殺掠人畜」,並進一步「請吏治之」, 這亦引起了明廷的注意。

不過,明成祖仍未敢輕言出兵。1405年年底,雲南將領沐晟建議出兵,卻遭明成祖反駁說:「爾又言欲發兵向安南。朕方以布恩信,懷遠人為務。胡𡗨雖擾我邊境,令已遣人詰問,若能攄誠順命,則亦當弘包荒之量。」 至於陳天平的處置,明廷則決定送歸越南,並要求越人「以君事之」,奉為國主。 越南方面,胡朝有感於對明關係緊張,亦積極防備,重編軍制,在多邦城(陳仲金說位於山西省先豐縣古法社)加強防守,於各個河海要處裝插木樁陷阱,整頓軍庫,招募人民有巧藝者入伍。但胡朝君臣對明主戰或主和,意見分歧甚大,有官員認為只好「從他(明朝)所好,以緩師可也」,左相國胡元澄則認為只決定於「民心之從違耳」,對明作戰並無十足把握。

1406年,明朝派鎮守廣西都督僉事黃中領五千士兵(《大越史記全書》稱領兵十萬),護送陳朝王孫陳天平(陳添平)回越南(《明實錄》把事件列在該年農曆三月丙午,即西曆4月4日;《大越史記全書》則列入農曆四月八日,即西曆4月26日)。當進入越南境內的支棱隘時,遇上胡軍截擊,明軍不敵,陳天平及部份士兵被俘。陳天平經胡朝審訊後,被「處陵遲罪」。明成祖得悉後大怒,便「決意興師」。

同年年中,明成祖派總兵官朱能加封「征夷大將軍」,配印信。後來在行軍時病卒,由副將張輔代替)、左副將軍沐晟、右副將軍張輔、左參將李彬、右參將陳旭等領兵(《大越史記全書》稱共有八十萬人,中國學者郭振鐸、張笑梅認為可能有誇大),分兵兩路,開進越南的白鶴江會師,一邊向越南腹地步步推進,一邊發出檄文向越人呼籲胡季犛父子的行為是「肆逞凶暴,虐于一國」,並列出胡氏「兩弒前安南國王以據其國」、「賊殺陳氏子孫宗族殆盡」、「淫刑峻法,暴殺無辜,重斂煩徵,剝削不已」等二十款大罪,又稱明軍的到來是「吊爾民之困苦,復陳氏之宗祀」,以使民心動搖。果然,不少越人「厭胡氏苛政,罔有戰心」,有助明軍前進更為順利。農曆十二月丙申十一日(西曆1407年1月19日),胡軍的主力退守多邦城,明軍亦看準該城位於河邊,有較大面積的沙灘可供搶灘,於是分兵進攻,成功以火銃擊退胡軍象兵。其後,明軍攻入越南的重要城市東都昇龍,並大肆掠奪,「擄掠女子玉帛,會計粮儲,分官辦事,招集流民。為久居計,多閹割童男,及收各處銅錢,驛送金陵」。

1407年年初,明軍攻破昇龍後,向胡朝的首都清化繼續前進,胡氏皇子胡元澄領軍退守黃江(在今越南河南省的一段紅河),與胡季犛、胡漢蒼會合。明將沐晟則進駐木凡江(在今越南河内市,與黃江相接)預備出擊。農曆二月,沐晟沿江兩岸擊敗胡元澄軍,追擊至悶海口(在今越南南定省),因軍中爆發疾疫,明軍移師到鹹子關立塞備戰。農曆三月,胡軍集合水步大軍七萬,號稱二十一萬,與明軍爆發鹹子關之戰。結果胡軍潰敗,大批兵士溺斃於該處河流,無數船隻及軍糧沉沒,胡氏父子敗逃,最終在農曆五月十一日(西曆6月16日)在奇羅海口(在今越南河靜省奇英縣)被明軍俘獲,胡朝滅亡,領土被明朝佔領。據當時的統計,越南土地人口物產資料為:府州四十八、縣一百六十八、戶三百一十二萬九千五百、象一百一十二、馬四百二十、牛三萬五千七百五十、船八千八百六十五。(※此一統計數字,按《明實錄》記載的1408年農曆六月的計算,則是「安撫人民三百一十二萬有奇;獲蠻人二百八萬七千五百有奇,糧儲一千三百六十萬石,象、馬、牛共二十三萬五千九百餘隻,船八千六百七十七艘,軍器二百五十三萬九千八百五十二件。」)

胡朝亡後,明成祖在農曆六月癸未朔(西曆7月5日)下詔,聲稱這次軍事行動是為了越南原本的陳氏王室著想,「期伐罪(指胡朝)以吊民,將興滅而繼絕」,並打算對「久染夷俗」的越人「設官兼治,教以中國禮法」,以達致「廣施一視之仁,永樂太平之治」。明廷又以陳朝子孫被胡氏殺戮殆盡,無可繼承,於是在越南設置交址都指揮使司、交址等處承宣布政使司及交址等處提刑按察使司等官署,將之直接管轄。

安南内属后,安南人民不断进行反抗,明军多次进行镇压。永乐二十二年(1424年),明成祖去世,太子朱高炽明仁宗即位,次年明仁宗去世,太子朱瞻基即位,是为明宣宗。宣宗考慮到「數年以來,一方不靖,屢勤王師」, 便允許撤兵。黎利得勝後,就發佈阮廌所起草的《平吳大誥》,稱他自己的抗明鬥爭是「仁義之舉,要在安民,吊伐之師,莫先去暴」;提出中越兩國是「山川之封域既殊,南北之風俗亦異」,因而有必要脫離明朝統治,自行建國,於是建立後黎朝。

其後,1431年農曆正月五日(西曆2月12日),明封黎利為安南國王,從此朝貢不絕。

为了稳定北方边境,对付蒙古势力。永乐七年(1409年),明成祖朱棣派淇国公丘福率十万大军征讨鞑靼,由于轻敌,孤军深入,中埋伏,全军覆没。为消除边患,明成祖决心亲征。明永乐八年(1410年)二月,明成祖调集50万大军。五月八日,明军行至胪朐河(今克鲁伦河,朱棣将之更名为“饮马河”)流域,询得鞑靼可汗本雅失里率军向西逃往瓦剌部,丞相阿鲁台则向东逃。朱棣亲率将士向西追击本雅失里,五月十三日,明军在斡难河(位于今蒙俄边境)大败本雅失里。朱棣打败本雅失里后,挥师向东攻击阿鲁台,双方在今蒙俄边境之斡难河东北方向交战,明军杀敌无数,阿鲁台坠马逃遁。此时天气炎热,缺水,且粮草不济,朱棣下令班师。鞑靼部经过明军的这次打击,臣服了明朝,当年向明成祖进贡马匹。成祖亦给予优厚的赏赐,其部臣阿鲁台接受了成祖给他“和宁王”的封号。

明军在永乐八年(1410年)第一次出征鞑靼后,瓦剌部趁机迅速发展壮大,1413年,瓦剌军进驻胪朐河(今克鲁伦河),窥视中原。明成祖决心再次亲征,调集兵力,筹集粮饷。永乐十二年(1414年)二月,明军从北京出发,六月初三,明军在三峡口(今蒙古乌兰巴托东南)击败了瓦剌部的一股游兵,杀敌数十騎;初七日,明军行至勿兰忽失温(今蒙古乌兰巴托东南),瓦剌军3万之众,依托山势,分三路阻抗,朱棣派骑兵冲击,引诱敌兵离开山势,遂命柳升发炮轰击,自己亦亲率铁骑杀入敌阵,瓦剌军败退,朱棣乘势追击,兵分几路夹击瓦剌军的所扑,杀敌数千,瓦剌军纷纷败逃。此役,瓦剌受到了重创,此后多年不敢犯边,同时,明军也伤亡惨重。

瓦剌被明成祖打败,鞑靼趁此机会经过几年的发展,势力日益强盛起来,从而改变对明朝的依附政策,并侮辱或拘留没明朝派去的使节,还时常对明朝边境进行骚扰的劫掠。永乐十九年(1421年)冬初,鞑靼围攻明朝北方重镇兴和,杀死了明军指挥官王祥,对此,朱棣决定第三次亲征漠北。永乐二十年(1422年)三月,明成祖率軍从北京出发,出击鞑靼。其主力部队至宣府(今河北宣化)东南的鸡鸣山时,鞑靼首领阿鲁台得知明军来袭,乘夜逃离兴和,避而不战。七月,明军到达煞胡原,俘获鞑靼的部属,得知阿鲁台已逃走,朱棣下令停止追击。明军在回师途中,击败兀良哈部,九月,回师北京。明成祖第三次出击漠北,虽对鞑靼部有一定的打击,但成效不大,并没彻底解决盘据漠北的蒙古三个部落对明朝边境的滋扰。

永乐二十一年(1423年),鞑靼首领阿鲁台再次率部滋扰明朝边境,明成祖闻悉后决定再次亲征。明军八月初出征,九月上旬,明军到达沙城(今河北张北以北)时,阿鲁台的部下阿失贴木儿率部投降明军,并得知阿鲁台被瓦剌打败,其部已溃散,明军暂时驻扎不前;十月,明军继续北上,在黄河以北击败鞑靼西部的军队,鞑靼王子也先土干率部众来降明,明成祖朱棣随即封也先土干为忠勇王,十一月,明军班师回京。

永乐时全国形势相对缓和,但由于国家支出过大,赋役征派繁重,使有些地区发生了农民流亡与起义,十八年山东发生的唐赛儿起义是其中规模较大的一支。明永乐二十二年(1424年)正月至七月,明軍对蒙古鞑靼部的作戰。是年正月,鞑靼部首领阿鲁台率軍進犯明山西大同、开平(今内蒙古正兰旗东北)等地。明成祖朱棣遂调集山西、山东、河南、陕西、辽东5都司之兵于京师(今北京)和宣府(今河北宣化)待命。四月三日,以安远侯柳升、遂安伯陈英为中军;武安侯郑亨、保定侯盂瑛为左哨,阳武侯薛禄、新宁伯谭忠为右哨;英国公张辅、成国公朱勇为左掖,成山侯王通、兴安伯徐亨为右掖;宁阳侯陈懋、忠勇王金忠又名也先土干为前锋,出兵北征。出征前戶部尚書夏元吉以國庫虛耗,曾勸他勿起戰事,但他不聽,反繫之大獄。二十五日,进至隰宁(今河北沽源南),获悉阿鲁台逃往答兰纳木儿河(今蒙古境内之哈剌哈河下游),明成祖令全军急速追击。六月十七日,进至答兰纳木儿河,周围300余里不见阿鲁台部踪影,遂下令班师。

明成祖為填補太祖廢除丞相後導致六部之首的空缺,但又希望強化皇權,他设立内阁,内阁大學士计有解縉、黃淮、胡廣、楊榮、金幼孜、楊士奇、胡儼。明成祖时期涌现许多著名大臣,包括蹇义、郁新、刘观、郑赐、宋礼、金纯、夏原吉、吕震、金忠、沐春、沐晟、沐昂。

明成祖任用酷吏强化自己的统治,著名的包括陳瑛和紀綱。

明成祖时期的著名太监包括:鄭和:三宝太監七下西洋;王景弘:鄭和的副手;侯顯:有才辨,強力敢任,五使絕域,勞績與鄭和亞;亦失哈:鞏固北方邊防,晚年研究改造武器,如改造步槍(裝槍頭-為安裝刺刀的先驅);王彦:原名王狗兒,尚寶監太監;昌盛:神宫監太監,貴州人。歷洪武-建文-永樂-洪熙-宣德五朝。

永樂二十二年(1424年)七月,明成祖率領北征大軍班師返京。七月十五日,明成祖病重。十六日,行至榆木川(今内蒙古多伦),昏迷不醒。十八日,明成祖朱棣崩逝於榆木川(今中國內蒙古自治區錫林郭勒盟多倫縣),享壽六十四岁,在位二十二年。遗诏传位皇太子。大學士楊榮、太監馬去等秘不發喪,暗中派御馬監少監海壽秘密回京,“奉遗命,驰讣皇太子”。太子朱高熾立即派皇太孫前往虎帐。八月十一日,皇太孫到達軍營後,始發佈帝崩消息。太子朱高炽即位,宣布次年改元洪熙,是为明仁宗。明成祖驾崩后,殉葬的有30余位宫女,其中包括成祖的16位嫔妃。

明成祖驾崩后,谥体天弘道高明广运圣武神功纯仁至孝文皇帝,庙号太宗,十二月十九日,明成祖与仁孝文皇后徐氏合葬于长陵。嘉靖十七年(1538年)九月,明世宗朱厚熜改谥明成祖为启天弘道高明肇运圣武神功纯仁至孝文皇帝,改上庙号为成祖。

《明史·成祖本纪》中评价明成祖:文皇少长习兵,据幽燕形胜之地,乘建文孱弱,长驱内向,奄有四海。即位以后,躬行节俭,水旱朝告夕振,无有壅蔽。知人善任,表里洞达,雄武之略,同符高祖。六师屡出,漠北尘清。至其季年,威德遐被,四方宾服,明命而入贡者殆三十国。幅陨之广,远迈汉、唐。成功骏烈,卓乎盛矣。然而革除之际,倒行逆施,惭德亦曷可掩哉。

蔡石山在其著作《永乐大帝:一个中国帝王的精神肖像》的开篇评价明成祖“明朝的永乐皇帝,驾崩于1424年8月12日,自从1402年7月17日登极以来——近乎八千零六十二天的在位期间——而且所有的证据也显示,他从未浪费过一天”。在书末,他再次评价明成祖“毋庸置疑,永乐有过多的自我,而且拥有很多的美德:他是自信、直率的,能够甄别和牢记有很强能力之人的贡献,而且保护依靠他的那些人,尤其是他的家人。不过,他也有黑暗面,特征就是不必要又未经思考的侵犯性,而这类侵犯性经常产生了暴虐和消耗”。

《朝鲜王朝实录·世宗庄宪大王实录》中评价明成祖「使臣言:"前後選獻韓氏等女,皆殉大行皇帝。" 先是,賈人子呂氏入皇帝宮中,與本國呂氏以同姓,欲結好,呂氏不從,賈呂蓄憾。 及權妃卒,誣告呂氏點毒藥於茶進之,帝怒,誅呂氏及宮人宦官數百餘人。 後賈呂與宮人魚氏私宦者,帝頗覺,然寵二人不發,二人自懼縊死。 帝怒,事起賈呂,鞫賈呂侍婢,皆誣服云:"欲行弑逆。" 凡連坐者二千八百人,皆親臨剮之,或有面詬帝曰:"自家陽衰,故私年少寺人,何咎之有?" 後帝命畫工圖,賈呂與小宦相抱之狀,欲令後世見之,然思魚氏不置,令藏於壽陵之側。 及仁宗卽位,掘棄之。 亂之初起,本國任氏、鄭氏自經而死,黃氏、李氏被鞫處斬。 黃氏援引他人甚多,李氏曰:"等死耳,何引他人爲? 我當獨死。" 終不誣一人而死。 於是,本國諸女皆被誅,獨崔氏曾在南京,帝召宮女之在南京者,崔氏以病未至,及亂作,殺宮人殆盡,以後至獲免。 韓氏當亂,幽閉空室,不給飮食者累日,守門宦者哀之,或時置食於門內,故得不死。 然其從婢皆逮死,乳媪金黑亦繫獄,事定乃特赦之。 初,黃氏之未赴京也,兄夫金德章坐於所在房窓外,黃儼見之大怒,責之,及其入朝,在道得腹痛之疾,醫用諸藥,皆無效,思食汁菹。 儼問元閔生曰:"此何物耶?" 閔生備言沈造之方,儼變色曰:"欲食人肉,吾可割股而進,如此草地,何得此物?" 黃氏腹痛不已,每夜使從婢以手磨動其腹,到一夜小便時,陰出一物,大如茄子許,皮裹肉塊也。 婢棄諸廁中,一行衆婢,皆知而喧說。 又黃氏婢潛說:"初出行也,德章贈一木梳。" 欽差皆不知之。 帝以黃氏非處女詰之,乃云:"曾與姊夫金德章、隣人皂隷通焉。" 帝怒,將責本國,勑已成,有宮人楊氏者方寵,知之,語韓氏其故,韓氏泣乞哀于帝曰:"黃氏在家私人,豈我王之所知也?" 帝感悟,遂命韓氏罰之,韓氏乃批黃氏之頰。 明年戊戌,欽差善才謂我太宗曰:"黃氏性險無溫色,正類負債之女。" 歲癸卯,欽差海壽謂上曰:"黃氏行路之時,腹痛至甚,吾等見則以鄕言言腹痛,必慙而入內。" 及帝之崩,宮人殉葬者,三十餘人,當死之日,皆餉之於庭。 餉輟,俱引升堂,哭聲震殿閣。 堂上置木小床,使立其上,掛繩圍於其上,以頭納其中,遂去其床,皆雉經而死。 韓氏臨死,顧謂金黑曰:"娘吾去! 娘吾去!" 語未竟,旁有宦者去床,乃與崔氏俱死。 諸死者之初升堂也,仁宗親入辭訣,韓氏泣謂仁宗曰:"吾母年老,願歸本國。" 仁宗許之丁寧,及韓氏旣死,仁宗欲送還金黑,宮中諸女秀才曰:"近日魚、呂之亂,曠古所無。 朝鮮國大君賢,中國亞匹也。 且古書有之,初佛之排布諸國也,朝鮮幾爲中華,以一小故,不得爲中華。 又遼東以東,前世屬朝鮮,今若得之,中國不得抗衡必矣。 如此之亂,不可使知之。" 仁宗召尹鳳問曰:"欲還金黑,恐洩近日事也,如何?" 鳳曰:"人各有心,奴何敢知之?" 遂不送金黑,特封爲恭人。 初,帝寵王氏,欲立以爲后,及王氏薨,帝甚痛悼,遂病風喪心,自後處事錯謬,用刑慘酷。 魚、呂之亂方殷,雷震奉天、華蓋、謹身三殿俱燼。 宮中皆喜以爲:"帝必懼天變,止誅戮。" 帝不以爲戒,恣行誅戮,無異平日。 後尹鳳奉使而來,粗傳梗槪,金黑之還,乃得其詳。」

\subsection{洪武}

\begin{longtable}{|>{\centering\scriptsize}m{2em}|>{\centering\scriptsize}m{1.3em}|>{\centering}m{8.8em}|}
  % \caption{秦王政}\
  \toprule
  \SimHei \normalsize 年数 & \SimHei \scriptsize 公元 & \SimHei 大事件 \tabularnewline
  % \midrule
  \endfirsthead
  \toprule
  \SimHei \normalsize 年数 & \SimHei \scriptsize 公元 & \SimHei 大事件 \tabularnewline
  \midrule
  \endhead
  \midrule
  三五年 & 1402 & \tabularnewline
  \bottomrule
\end{longtable}

\subsection{永乐}

\begin{longtable}{|>{\centering\scriptsize}m{2em}|>{\centering\scriptsize}m{1.3em}|>{\centering}m{8.8em}|}
  % \caption{秦王政}\
  \toprule
  \SimHei \normalsize 年数 & \SimHei \scriptsize 公元 & \SimHei 大事件 \tabularnewline
  % \midrule
  \endfirsthead
  \toprule
  \SimHei \normalsize 年数 & \SimHei \scriptsize 公元 & \SimHei 大事件 \tabularnewline
  \midrule
  \endhead
  \midrule
  元年 & 1403 & \tabularnewline\hline
  二年 & 1404 & \tabularnewline\hline
  三年 & 1405 & \tabularnewline\hline
  四年 & 1406 & \tabularnewline\hline
  五年 & 1407 & \tabularnewline\hline
  六年 & 1408 & \tabularnewline\hline
  七年 & 1409 & \tabularnewline\hline
  八年 & 1410 & \tabularnewline\hline
  九年 & 1411 & \tabularnewline\hline
  十年 & 1412 & \tabularnewline\hline
  十一年 & 1413 & \tabularnewline\hline
  十二年 & 1414 & \tabularnewline\hline
  十三年 & 1415 & \tabularnewline\hline
  十四年 & 1416 & \tabularnewline\hline
  十五年 & 1417 & \tabularnewline\hline
  十六年 & 1418 & \tabularnewline\hline
  十七年 & 1419 & \tabularnewline\hline
  十八年 & 1420 & \tabularnewline\hline
  十九年 & 1421 & \tabularnewline\hline
  二十年 & 1422 & \tabularnewline\hline
  二一年 & 1423 & \tabularnewline\hline
  二二年 & 1424 & \tabularnewline
  \bottomrule
\end{longtable}


%%% Local Variables:
%%% mode: latex
%%% TeX-engine: xetex
%%% TeX-master: "../Main"
%%% End:

%% -*- coding: utf-8 -*-
%% Time-stamp: <Chen Wang: 2019-12-26 15:06:42>

\section{仁宗\tiny(1424-1425)}

\subsection{生平}

明仁宗朱高熾(1378年8月16日-1425年5月29日),俗稱洪熙帝,明成祖長子,其母为仁孝文皇后,中山王徐達外孫,明朝第四代皇帝。

洪武年間,被封為燕世子。靖難之役中,仁宗負責鎮守北平,并成功抵禦李景隆率領的中央軍圍攻。永樂二年(1404年),立為皇太子,并在明成祖屢次北伐中,擔任監國職位,實際負責國家政事。永樂二十二年(1424年),繼承皇位,年號“洪熙”,在位期間,採取一系列政治、經濟、軍事改革與調整,國家富足。仁宗與子明宣宗在政治用人、行政處理上,均為後世所称善,史稱“仁宣之治”。

朱高熾年幼端重沉靜,善於言辭,且擅长射箭,喜愛與儒臣講論。洪武二十八年闰九月壬午(1395年11月4日),他被冊封為燕世子,後守衛北平,由於心性較爲溫良,體諒官員、士卒,深受祖父明太祖朱元璋喜愛。

靖難之役中,燕王朱棣起兵,朱高熾則鎮守北平,期間以一萬兵力,阻擋李景隆率領的五十萬中央軍圍攻。由於朱高熾身型肥胖而且有腳病,不良於行,不曾隨父親朱棣征戰,且性格相對較爲溫和,向來不獲父親寵愛。反而常隨朱棣征戰的次子朱高煦、三子朱高燧均受朱棣喜愛,而朱高煦則更因屢有戰功,於是出言詆毀朱高炽以奪嫡。當時,建文帝施離間計,下「賜世子書」;朱高燧的人馬得知此事,向朱棣建言「世子勾結朝廷」,沒想到朱高熾不予啟封,直接呈上朱棣,方破此計。朱棣即位後,改北平為北京,仍命朱高熾居守。

朱棣成功奪位為帝後,是為明成祖。永乐元年春正月丙戌,群臣上表请立皇太子,不允;三月戊寅朔,文武百官复上表,请立皇太子,敕“姑缓之”。成祖本想立自己喜愛的次子朱高煦為太子,但礙於長子朱高熾的世子地位是明太祖確立,而且朱高熾並無過失,又得一眾文官支持,最後於永樂二年四月甲戌(1404年5月12日),朱高熾被召入南京應天府,被立為皇太子。明成祖屢次北伐,均命其擔任監國,負責國事。當時全國經战争影響,水旱饑荒嚴重,他派遣官員賑災撫恤,仁政受到贊許。然而,失落太子地位的朱高煦心有不甘,聯同弟朱高燧及其他黨羽加緊離間明成祖與朱高熾的關係。明成祖問太子是否知悉有人離間,朱高熾則答稱不知情,“知盡子職而已”。

永樂十年,朱棣北伐歸還,朱高熾遣使誤期,加上書奏失辭,太子一系官員,如黃淮等人均下詔獄。次年,朱高燧黨羽黃儼等誣陷朱高熾擅自釋放罪人,其官僚多因連坐而亡。禮部侍郎胡濙奉命調查后,密奏朱棣稱太子誠敬孝謹等七事,明成祖才釋除疑慮。之後,朱高燧黨羽黃儼策劃謀立,后被發覺,伏法。太子朱高熾則力請免朱高燧罪,至此朱高熾地位方穩。

永樂二十二年(1424年)七月,明成祖在北征班師途中崩於榆木川。当时京师诸卫军皆随行,只有赵府三护卫留京师,随驾北征诸臣浮议籍籍,大学士杨荣、金幼孜等人顾虑赵府护卫闻讯发动政变,遂秘不发丧。杨荣与少监海寿持遗诏驰奔京师。朱高熾遣皇太孙朱瞻基出居庸关迎驾。同年八月己酉,皇太孙至雕鹗堡,入于军中,遂发丧。八月丁巳(1424年9月7日),朱高熾繼帝位,大赦天下,并取次年年號為洪熙。明仁宗登基後,褒奖直言,虚怀纳谏,減轻刑法。朱高熾與子朱瞻基在政治用人、行政處理上,均為後世所称善,史稱“仁宣之治”。

經濟方面,他下令中止鄭和下西洋,并取消官方在雲南、交阯的採辦活動、将首都迁回南京,以節省國家財政支出。政治方面,他恢復夏原吉、吳中官職,恢復三公、三孤等官職,命楊榮為太常寺卿,金幼孜為戶部侍郎,兼大學士,楊士奇為禮部左侍郎兼華蓋殿大學士,黃淮為通政使兼武英殿大學士,楊溥為翰林學士,進一步提升明朝內閣地位。軍事方面,他重新調整大同、交阯、山海關、遼東的邊疆總兵大臣,并建立南京守備制度。

同年冬天,朱高熾進一步對政治進行調整,加強戶部管理、以及城池防禦的同時,冊封張氏為皇后,立長子朱瞻基為皇太子、其餘八子分別為王。隨後下詔,赦免了建文帝的舊臣和永樂朝時遭連坐流放邊境的官員家屬,并免除受災地的稅糧。

外交方面,于闐、琉球、占城、哈密、古麻剌朗、滿剌加、蘇祿、瓦剌等國稱臣入貢。

洪熙元年春,因顯日食,朱高熾罷免宴樂。他進一步對政治進行調整,包括建立弘文閣,命楊溥掌管內閣;屢次求官員直言并納言,并對太祖時期的法外用刑制度進行修正,減少刑罰,實行寬政。

仁宗体弱多病,登基后不到十个月,遭李时勉當廷勸諫,龍顏大怒,雖命武士以金瓜錘將李时勉打斷三根肋骨,並拘入詔獄,仁宗仍不解恨,數日後一病不起,于洪熙元年五月辛巳(1425年5月29日)崩于钦安殿,廟號仁宗,葬于明獻陵(今北京昌平)。朱高熾延續了太祖和成祖的殉葬制度,死時生殉五名妃嬪。

\subsection{洪熙}

\begin{longtable}{|>{\centering\scriptsize}m{2em}|>{\centering\scriptsize}m{1.3em}|>{\centering}m{8.8em}|}
  % \caption{秦王政}\
  \toprule
  \SimHei \normalsize 年数 & \SimHei \scriptsize 公元 & \SimHei 大事件 \tabularnewline
  % \midrule
  \endfirsthead
  \toprule
  \SimHei \normalsize 年数 & \SimHei \scriptsize 公元 & \SimHei 大事件 \tabularnewline
  \midrule
  \endhead
  \midrule
  元年 & 1425 & \tabularnewline
  \bottomrule
\end{longtable}


%%% Local Variables:
%%% mode: latex
%%% TeX-engine: xetex
%%% TeX-master: "../Main"
%%% End:

%% -*- coding: utf-8 -*-
%% Time-stamp: <Chen Wang: 2019-12-26 15:06:47>

\section{宣宗\tiny(1425-1435)}

\subsection{生平}

明宣宗朱瞻基(1399年3月16日-1435年1月31日),或稱宣德帝,明仁宗皇长子,永樂九年(1411年)立為皇太孫;永乐二十二年(1424年)十月立為皇太子。洪熙元年(1425年)即位,年號宣德,明朝第5位皇帝,在位十年,享年37歲。宣德元年(1426年)平定高煦之亂,和其父仁宗一样,比较能倾听臣下的意见,聽從閣臣楊士奇、楊榮、楊溥等建議,停止對交阯用兵,与明仁宗并称「仁宣之治」,宣宗时君臣关系融洽,经济也稳步发展。不過,他也開啟此後宦官干政的局面。

明成祖時,朱瞻基父親朱高熾(仁宗)為太子,生性仁厚端重,但有時不免失之於懦怯。成祖最喜愛次子漢王朱高煦,覺得他最像自己,有心廢太子立漢王,但徐皇后和大臣們一直阻攔。而且朱瞻基自幼聰慧好學,與生母張氏皆深得成祖的喜愛,所以最終才沒有廢太子,並對朱瞻基悉心栽培。永樂九年(1411年)十一月立為皇太孫,數度隨成祖征討。永乐二十二年(1424年)仁宗即位,十月朱瞻基被立為皇太子。洪熙元年(1425年)四月,因南京地震多發,奉旨前往居守;同年六月仁宗駕崩,宣宗繼位。

明宣宗在位十年,重点在治理内政方面。宣德元年(1426年)平定汉王朱高煦的叛乱,宣宗原先只將他禁錮,仍前往探视,却被朱高煦使腿将其绊倒,宣宗一怒,将朱高煦用鼎扣住,烧烤至死,諸子全部處死。为了休兵养民,宣宗一改永乐时期的讨伐政策,主动从交阯撤兵。

宣宗整顿统治机构,罢免「贪津不律」、「不达政体」、「年老体疾」的官员,实行精简和裁冗措施,以振朝风。而在用人方面限制入仕人数,实行保举和欠任。宣宗实行一些减轻民困的措施,减免税粮、复业流民、赈灾救荒等。宣德三年出塞,并修建永寧、隆慶諸城。

在宦官问题上,因明代初期宦官多由藩屬國進貢或沒入各地罪犯家屬,在語言溝通上發生很大問題,言不同語只好以書同文來解決,宣德元年(1426年),明宣宗下令設置內書堂,教導宦官們讀書。不過,明太祖苦心謀劃的女官制度雖經成祖時期略加破壞,在此時仍發揮其防制閹黨之禍的功用,可是宣宗下令容許教導宦官讀書一舉,无意中卻開啟了明代宦官干政之先兆,尤其在明神宗後,因氣候變遷造成北方官話區大量貧困百姓自宮入朝廷謀職,萬曆至崇禎(1573-1644年)這71年間自宮入廷的閹宦總計高達三萬人,使得教導宦官成為明朝覆滅的其中原因,也是最受後世批評之處。不過與唐朝相比,明代皇帝極權之盛, 使終明一朝皇帝亦不至受宦官控制,一般而言亦只是通過宦官來處理政務及制約大臣的權力。

宣德五年(1431年1月),宣宗以外番多不來朝貢為由,命令鄭和再次出航。返航期間,鄭和因勞累過度於宣德八年(1433年)四月初在印度西海岸古里去世。船隊由太監王景弘率領返航,宣德八年七月初六(1433年7月22日)返回南京。第七次下西洋人數據載有27550人。這也是最後一次下西洋。

宣德十年(1435年)正月初三,皇帝崩于乾清宫,时年37岁,谥号宪天崇道英明神圣钦文昭武宽仁纯孝章皇帝。庙号宣宗。宣德十年六月廿一日,梓宫葬入景陵。

安南人黎利反叛,屡次打败官军。黎利请示朝廷,请求重新立陈氏之后为安南国王。朱瞻基认为国中疲惫,远征无益,于是答应了他,册封陈暠为安南国王,罢征南兵。后来黎利篡夺陈暠之位而自立为王。派人入朝纳贡谢罪,请求皇帝册封群臣。有人请求皇帝讨伐黎利,朱瞻基不许,册封黎利为安南国王。安南国也就是交趾国,自此以后朝贡不绝。

朱瞻基担心秋高马肥时蒙古人侵犯边疆,于是整顿兵马,驻扎喜峰口以待敌军。守将奏报兀良哈率领万名铁骑骚扰边疆,朱瞻基精选铁骑兵三千飞奔前往。敌军望见远处来军,以为是戍守边疆之兵,即以全军来迎战。朱瞻基命令将铁骑分为两路夹攻敌军,并且亲自射杀敌军先锋,杀死三人。两翼飞矢如云,敌人不敢前进。继而,朱瞻基又命连续发射神机铳,敌军人马死伤大半,剩下的全部溃逃。朱瞻基用数百铁骑直驱前行,敌人看到黄龙旗,才知道是皇帝亲征,于是全部下马拜倒在地请降,朱瞻基将这些人捆缚抓获,大胜而归。

《明史》赞誉宣宗:“仁宗为太子,失爱于成祖。其危而复安,太孙盖有力焉。即位以后,吏称其职,政得其平,纲纪修明,仓庾充羡,闾阎乐业。岁不能灾。盖明兴至是历年六十,民气渐舒,蒸然有治平之象矣。若乃强藩猝起,旋即削平,扫荡边尘,狡寇震慑,帝之英姿睿略,庶几克绳祖武者欤。”

《國榷》:“谈迁曰:国初严御,每重囚岁械入京辄千百,簿尉巡檄之任,辄烦圣虑,盖详极矣。宣宗幼侍文皇帝出入塞垣,深谙民事。及即位,遽有乐安之驾,非素才武,畴克灭此而朝食也者?然兵不轻试,惓惓以生灵为念。水旱朝奏,赈贷午曁。亲阅囚牍,多所释遣。好文学之士,一才一技,皆被甄录。盖睿质天纵,文翰并美,而不矜其能,尝有自下之色。国家之治,宽严有制,烦简有则,帝实始之。而於废胡后,弃南交,孰为帝谅者?呜呼!废后非盛德事也,其弃南交,比於汉之朱崖矣。”

《名山藏》:“高皇帝承胡元縱弛之弊,宏振威武以儆天下,成祖以英達之資纘緒大服,海內竦然,振厲者五十餘年。昭皇帝(明仁宗)至德深仁不久於位,章帝(明宣宗)繼之,乃涵濡以醇懿陶埴,以德義聞四方。”

《朝鮮文宗實錄》:“上(朝鮮文宗)謂代言等曰: "尹鳳率爾告予曰: 「洪熙皇帝及今(宣德)皇帝, 皆好戲事。 洪熙嘗聞安南叛, 終夜不寐, 甚無膽氣之主也。’」知申事鄭欽之對曰:“尹鳳謂予曰: 「洪熙沈于酒色,聽政無時,百官莫知早暮。 今皇帝燕于宮中,長作雜戲。 永樂皇帝, 雖有失節之事, 然勤於聽政, 有威可畏。」 鳳常慕太宗皇帝, 意以今皇帝爲不足矣。”上曰:「人主興居無節, 豈美事乎?」”

宣德皇帝既是一个有较高文化素養的皇帝,又是一个喜欢射猎、美食、鬥促织(蟋蟀)的皇帝。《聊齋誌異》裡的名篇《促織》裡的皇帝正是明宣宗,人稱“促织天子”,吳偉業有《明宣宗御用戧金蟋蟀盆歌》。

\subsection{宣德}

\begin{longtable}{|>{\centering\scriptsize}m{2em}|>{\centering\scriptsize}m{1.3em}|>{\centering}m{8.8em}|}
  % \caption{秦王政}\
  \toprule
  \SimHei \normalsize 年数 & \SimHei \scriptsize 公元 & \SimHei 大事件 \tabularnewline
  % \midrule
  \endfirsthead
  \toprule
  \SimHei \normalsize 年数 & \SimHei \scriptsize 公元 & \SimHei 大事件 \tabularnewline
  \midrule
  \endhead
  \midrule
  元年 & 1426 & \tabularnewline\hline
  二年 & 1427 & \tabularnewline\hline
  三年 & 1428 & \tabularnewline\hline
  四年 & 1429 & \tabularnewline\hline
  五年 & 1430 & \tabularnewline\hline
  六年 & 1431 & \tabularnewline\hline
  七年 & 1432 & \tabularnewline\hline
  八年 & 1433 & \tabularnewline\hline
  九年 & 1434 & \tabularnewline\hline
  十年 & 1435 & \tabularnewline
  \bottomrule
\end{longtable}


%%% Local Variables:
%%% mode: latex
%%% TeX-engine: xetex
%%% TeX-master: "../Main"
%%% End:

%% -*- coding: utf-8 -*-
%% Time-stamp: <Chen Wang: 2019-10-18 16:57:57>

\section{英宗\tiny(1435-1449)}

明英宗朱祁鎮(1427年11月29日-1464年2月23日),明宣宗朱瞻基長子,生母孝恭章皇后,明代宗朱祁鈺異母兄,明憲宗朱見深之父,是明朝的第6位和第8位皇帝;最初使用正統(1436年-1449年)年號,復位後使用天順(1457年-1464年)年號,在位22年。謚號「法天立道仁明誠敬昭文憲武至德廣孝睿皇帝」。

宣德二年(1427年),貴妃孫氏為明宣宗朱瞻基產下長子朱祁鎮(但《明史》記孫氏生平則說她暗中取宮女之子為己子)。出生四個月的朱祁鎮隨即被立為皇太子,其母孫氏為皇后。

宣德十年(1435年)正月,宣宗崩,時年7歲的朱祁鎮即位,是為英宗,改次年為正統元年。英宗在位初期由太皇太后張氏輔政,內閣由三楊(楊士奇、楊榮和楊溥)主持,仁宣之治得以延續。

正統六年(1441年),正式親政,同年定首都為北京,結束南京名義上的首都地位。

正統七年(1442年),張太后卒,三楊以年老淡出政壇,宦官王振開始專權,其黨羽遍天下,百官為之側目,這是明朝第一次宦官專權。

正統十四年(1449年),瓦剌蒙古大舉南侵,英宗以五十萬大軍親征,沿途鋪張。返師途中,八月十五(1449年9月1日)行至土木堡被瓦剌太師也先所敗,明軍「死者數十萬」,英宗被俘虜,附和英宗的太監王振被明英宗之護衛將軍樊忠殺死,樊忠殺死王振前曰:「吾為天下誅此賊!」以所持棰擊殺王振,力圖突圍,殺數十人後戰死。史稱土木堡之變,簡稱土木之變。

隨後,也先挾持英宗南下進攻北京,皇太后孫氏命英宗之弟郕王朱祁鈺監國,不久郕王即帝位,是為明代宗,改次年為景泰元年,尊英宗為太上皇。

于謙領導的北京保衛戰勝利後,瓦剌倡議和談,欲送還英宗。景帝不欲英宗還鑾。景泰元年(1450年),鴻臚卿楊善變賣家產,孤身出使瓦剌,又在景帝不同意的情況下,說服瓦剌太師也先,將英宗迎回燕京。

英宗回國後,代宗怕失去即位不久的帝位,將其兄長英宗軟禁於南內崇質宮,令錦衣衛防守嚴密。景泰三年,又廢原立為太子的英宗長子朱見深為沂王,另立己子朱見濟為儲君。但朱見濟在次年去世。後太子朱見濟死,但代宗仍不同意復立朱見深為太子。

景泰八年(1457年)正月,代宗病重,不能臨朝,手握重兵的武清侯石亨、副都御史徐有貞等人聯合太監曹吉祥,率死士攻入南宮,擁英宗復辟。十六日晚上,英宗自東華門入宮,於奉天殿即位,黎明時開宮門,諭令百官,改元天順,史稱「奪門之變」。代宗被禁於西內。不久死亡,死因不明,有謂乃英宗使宦官蔣安以布帛縊死。死後追貶為郕王,謚戾,葬於西郊金山(玉泉山北)。

英宗奪門之變復辟後,即以謀逆罪將兵部尚書于謙及大學士王文等人下獄,初尚言「于謙實有功」,徐有貞言「不殺于謙,今日之事無名」,遂於五日後斬殺于謙和王文於西市。天下冤之。大學士李賢告知英宗背後秘密,「奪門之變」沒有用處。因為郕王無子,擁立朱祁鎮的孫太后仍在世上,所以帝位遲早是英宗的,不需要奪門。奪門只是小人们的一齣戲,目的是求自己的升官發財。英宗下令宮中不得再使用「奪門」一詞,並且罷除因奪門之變而晉升的一切官職(計四千餘人),疏遠了徐​​有貞等,後來曹吉祥與石亨等人勾結,先設法中傷徐​​有貞,讓徐被流放。而後石亨與曹吉祥因圖謀叛亂發動曹石之變,石亨被囚至死,曹吉祥則被凌遲處死。

天順一朝,英宗勤於理政,並任用李賢、彭時等賢臣,先後懲治石亨、徐有貞、曹吉祥等人,政治尚算清明。又不顧左右反對,釋放建庶人(明惠宗幼子朱文圭,明成祖發動靖難後被幽禁宮中逾五十年,已豬狗不識),並提供飲食住行;聽錢皇后之言恢復前朝胡廢后的位號;病危遺言,取消了自明太祖以來的宮妃殉葬制度。《明史》讚譽道:「罷宮妃殉葬,則盛德之事可法後世者矣。」王世貞在《弇州山人別集》中亦稱:「此誠千​​古帝王之盛節。」

天順八年(1464年)正月英宗駕崩,享年38歲。葬入明十三陵中的裕陵。英宗與錢皇后感情頗深,錢皇后無子;因周妃專橫,英宗擔心死後嗣子明憲宗(周氏所生)不尊崇她的地位,所以遺命「皇后他日壽終,宜合葬」後來錢皇后死時,周太后果然不欲其祔葬裕陵,由於有英宗的遺詔,經過大臣力爭方得與英宗合葬。此後,在周太后的壓力下,不得已改變英宗的陵寢設計,周太后也得以附葬裕陵,開始出現一帝兩后或多后的格局。

\subsection{正统}

\begin{longtable}{|>{\centering\scriptsize}m{2em}|>{\centering\scriptsize}m{1.3em}|>{\centering}m{8.8em}|}
  % \caption{秦王政}\
  \toprule
  \SimHei \normalsize 年数 & \SimHei \scriptsize 公元 & \SimHei 大事件 \tabularnewline
  % \midrule
  \endfirsthead
  \toprule
  \SimHei \normalsize 年数 & \SimHei \scriptsize 公元 & \SimHei 大事件 \tabularnewline
  \midrule
  \endhead
  \midrule
  元年 & 1436 & \tabularnewline\hline
  二年 & 1437 & \tabularnewline\hline
  三年 & 1438 & \tabularnewline\hline
  四年 & 1439 & \tabularnewline\hline
  五年 & 1440 & \tabularnewline\hline
  六年 & 1441 & \tabularnewline\hline
  七年 & 1442 & \tabularnewline\hline
  八年 & 1443 & \tabularnewline\hline
  九年 & 1444 & \tabularnewline\hline
  十年 & 1445 & \tabularnewline\hline
  十一年 & 1446 & \tabularnewline\hline
  十二年 & 1447 & \tabularnewline\hline
  十三年 & 1448 & \tabularnewline\hline
  十四年 & 1449 & \tabularnewline
  \bottomrule
\end{longtable}


%%% Local Variables:
%%% mode: latex
%%% TeX-engine: xetex
%%% TeX-master: "../Main"
%%% End:

%% -*- coding: utf-8 -*-
%% Time-stamp: <Chen Wang: 2019-10-18 17:00:32>

\section{代宗\tiny(1449-1457)}

明代宗朱祁鈺(1428年9月21日-1457年3月14日),或稱景泰帝,年號景泰,明憲宗追諡其為「恭仁康定景皇帝」,弘光帝上庙号「代宗」,谥号「符天建道恭仁康定隆文布武显德崇孝景皇帝」,明朝第7位皇帝(1449年9月22日—1457年2月24日在位)。明宣宗皇次子,母親是賢妃吳氏。

生于宣德三年(1428年),他是明宣宗次子,母吴贤妃。据《明史》称吴贤妃为明宣宗为皇太孙时的侍女。

兄长明英宗即位後封他為郕王。1449年,明英宗在“土木堡之变”被瓦剌太師也先所俘后,郕王被于謙等大臣拥立,是为代宗,年号景泰,尊英宗為太上皇。

代宗即位后,用于謙为兵部尚书,北京保衛戰粉碎了瓦剌的进攻。景泰元年(1450年)八月,鴻臚寺卿楊善出使瓦剌,靠著三寸巧舌說服了也先,英宗返回北京,代宗並沒有迎回兄長的意思,又害怕他复辟,故将其软禁於宮中,以錦衣衛嚴密控管,宮門上鎖並且灌鉛,食物僅能由小洞遞入。

景泰三年,代宗廢去英宗長子朱見深的太子之位,改立自己兒子朱见济為太子,但朱見濟在次年去世。

景泰八年(1457年)正月,代宗病危,十六日曹吉祥、石亨、徐有貞等人謀復立英宗,十七日清晨,發動奪門之變,率領武士攻入紫禁城奉天殿,英宗復辟。代宗被软禁在西苑,一个多月後去世,得年三十岁。代宗死因不明,陸釴的《病逸漫記》說代宗是被英宗謀殺的,查繼佐的《罪惟錄》則表示代宗病愈,英宗為怕代宗復起,令太監蔣安用帛扼死景泰帝。代宗死后,葬于西郊金山(玉泉山北)的景泰陵。英宗令廷臣议王妃之殉葬。议及汪皇后,被李賢及太子谏止。后以皇贵妃唐氏殉葬。

英宗恨代宗薄待,谥为戾王,称郕戾王。明宪宗成化时期上谥号「恭仁康定景皇帝」。明崇禎十七年(1644年)七月乙丑,弘光帝上庙号代宗,谥号「符天建道恭仁康定隆文布武显德崇孝景皇帝」。清朝复称其谥号为「恭仁康定景皇帝」。明清史书多称明代宗为景帝。

明代宗是未安葬在明十三陵的皇帝(另外明太祖朱元璋葬于南京明孝陵,明惠帝因最後失踪故無陵墓)。

\subsection{景泰}

\begin{longtable}{|>{\centering\scriptsize}m{2em}|>{\centering\scriptsize}m{1.3em}|>{\centering}m{8.8em}|}
  % \caption{秦王政}\
  \toprule
  \SimHei \normalsize 年数 & \SimHei \scriptsize 公元 & \SimHei 大事件 \tabularnewline
  % \midrule
  \endfirsthead
  \toprule
  \SimHei \normalsize 年数 & \SimHei \scriptsize 公元 & \SimHei 大事件 \tabularnewline
  \midrule
  \endhead
  \midrule
  元年 & 1450 & \tabularnewline\hline
  二年 & 1451 & \tabularnewline\hline
  三年 & 1452 & \tabularnewline\hline
  四年 & 1453 & \tabularnewline\hline
  五年 & 1454 & \tabularnewline\hline
  六年 & 1455 & \tabularnewline\hline
  七年 & 1456 & \tabularnewline\hline
  八年 & 1457 & \tabularnewline
  \bottomrule
\end{longtable}


%%% Local Variables:
%%% mode: latex
%%% TeX-engine: xetex
%%% TeX-master: "../Main"
%%% End:

%% -*- coding: utf-8 -*-
%% Time-stamp: <Chen Wang: 2021-11-01 17:12:25>

\section{英宗朱祁鎮复辟\tiny(1457-1464)}

\subsection{生平}

奪門之變,又稱南宮復辟,是明代宗朱祁鈺景泰八年(1457年)正月,发生的一場政變,太上皇朱祁鎮成功復辟,奪回皇位。

正統十四年 (1449年) 發生土木堡之變,明英宗被瓦剌俘虜,其弟郕王朱祁鈺被眾大臣推舉為皇帝,是為明景帝(南明尊稱為代宗),改元景泰。孫太后亦要求景帝即位後立英宗兩歲兒子朱見深為太子,表示大明帝位仍由英宗一脈繼承。

景泰元年(1450年),兵部侍郎于谦成功抗敵,並與瓦剌議和,經過使臣楊善個人的斡旋,瓦剌首領也先見新君已立,英宗已經無利用價值,反而不想因英宗為虜之事成為與大明修好的障礙,於是同意放回英宗。但朱祁鈺對大臣說:「我並不是貪戀帝位,當初擁立我的是你們啊。」不願英宗返國,經大臣陳述其利弊後,朱祁鈺将英宗迎接回京,置於南宮,尊為太上皇。並以錦衣衛對英宗加以軟禁,嚴密控管,宮門不但上鎖,並且灌鉛,食物僅能由小洞遞入。其後景帝在景泰三年 (1452年)廢原太子朱見深,並立自己的獨子朱見濟為新太子。景泰五年 (1454年),朱见济夭折后,朱祁钰已无亲子,却也没有复立朱见深,储位空悬。

景泰七年(1456年),朱祁鈺病重,在對抗瓦剌時立下大功的將領石亨為了自身利益,有意協助英宗奪回帝位。在拉攏身邊人商討後,與宦官曹吉祥、都督張軏、都察院左都御史楊善、太常卿許彬以及左副都御史徐有貞等人行事。

景泰八年(1457年)正月,朱祁鈺病重。十六日夜,石亨、徐有贞等大臣带一千餘士兵偷襲紫禁城,撞开南宮宫门,接出英宗直奔东华门。守门的武士不开门,英宗上前说道:“朕乃太上皇帝也。”武士只好打开城门。

黎明时分,众大臣到了「奉天殿」,只见英宗坐于龙椅之上,徐有贞高喊:“太上皇帝復位。”史称「奪門之變」或「南宮復辟」。

英宗復辟後,朱祁鈺被遷至西宮,不久去世。

談遷評論:“于少保最留心兵事,爪牙四布,若奪門之謀,懵然不少聞,何貴本兵哉!或聞之倉卒,不及發耳!”

明英宗復辟後,于謙以謀逆罪名被處死,而曾助英宗回復帝位的功臣,如石亨、徐元玉、許彬、楊善、張軏與曹吉祥等人都被封為大官。其中,曹吉祥等在朝中橫行霸道,後期更發生了曹吉祥企圖弒位的曹石之變。

值得一提的是,景泰八年春正月,明英宗重登大寶後,废景泰年号,改景泰八年为天顺元年,但倉促之中忘記罷黜朱祁鈺,直到同年二月乙未才將朱祁鈺廢為郕王。因此,在這幾天之內,名義上英宗和景帝兩位合法的皇帝同時並存,成為中國帝制史上絕無僅有的奇觀。

曹石之变前,英宗在李贤提醒下,意识到朱祁钰时日无多,没有在世的儿子,也没有立储,一旦朱祁钰去世,自己复位顺理成章,夺门功臣其实是投机以求自己获益,一旦事败,英宗自己反而要受到牵连;于是开始罢黜夺门功臣的爵位。楊善、張軏已去世,爵位已分别由儿子杨宗、张瑾继承。明宪宗初年,罢黜杨宗、张瑾,因夺门之功所授爵位至此全部收回。

\subsection{天顺}

\begin{longtable}{|>{\centering\scriptsize}m{2em}|>{\centering\scriptsize}m{1.3em}|>{\centering}m{8.8em}|}
  % \caption{秦王政}\
  \toprule
  \SimHei \normalsize 年数 & \SimHei \scriptsize 公元 & \SimHei 大事件 \tabularnewline
  % \midrule
  \endfirsthead
  \toprule
  \SimHei \normalsize 年数 & \SimHei \scriptsize 公元 & \SimHei 大事件 \tabularnewline
  \midrule
  \endhead
  \midrule
  元年 & 1457 & \tabularnewline\hline
  二年 & 1458 & \tabularnewline\hline
  三年 & 1459 & \tabularnewline\hline
  四年 & 1460 & \tabularnewline\hline
  五年 & 1461 & \tabularnewline\hline
  六年 & 1462 & \tabularnewline\hline
  七年 & 1463 & \tabularnewline\hline
  八年 & 1464 & \tabularnewline
  \bottomrule
\end{longtable}


%%% Local Variables:
%%% mode: latex
%%% TeX-engine: xetex
%%% TeX-master: "../Main"
%%% End:

%% -*- coding: utf-8 -*-
%% Time-stamp: <Chen Wang: 2019-12-26 15:07:10>

\section{宪宗\tiny(1464-1487)}

\subsection{生平}

明憲宗,或稱成化帝,原名朱見深,後改名朱見濡(1447年12月9日-1487年9月9日),為明英宗皇長子,明朝第9代皇帝。明憲宗在位二十三年,期間恢復其叔朱祁鈺的帝號,又為于謙等忠臣平反,初年勵精圖治,體恤民情,任用李賢、商輅、彭時等賢臣,頗為時人所傳誦;在軍事方面,整飭戎政,對內平定荊襄群盜和西南傜蠻,對外抵禦抵禦韃靼女真、經略哈密,擁有不少功績。但憲宗寵嬖萬氏、中晚年信用汪直、梁芳、萬安等宦官奸臣,又以“皇莊”大肆侵占土地,使明朝政治日壞;而頻繁的內外用兵亦使明朝國力大損。成化朝是明朝自仁宣以來文治武功較卓越的時期,但是與此並存的弊政不得不說有所缺憾。谥号「繼天凝道誠明仁敬崇文肅武宏德聖孝纯皇帝。」

正統十四年(1449年)土木堡之變,英宗被瓦剌擄去,兵部侍郎于謙等立皇弟朱祁钰即位,是為景帝,改元景泰,同時立見深為太子。到景泰三年(1452年),朱祁鈺將見深廢為沂王,改立自己的儿子朱见济为太子。

五年后(1457年),英宗因奪門之變而復辟,見深重被立為太子。萬曆野獲編中記載憲宗皇帝玉音微吃,而臨朝宣旨,則瑯瑯如貫珠,其本人可能或多或少有口吃的情況。

原名朱見濬(《明史》誤載憲宗即位前名為朱見浚,即位後為見深),因英宗復辟後重立太子,將憲宗之名誤寫為見濡,憲宗於天顺八年(1464年)登基後遂改稱見濡。憲宗宽仁英明,即位之初就為于謙平冤昭雪,當時曾有大臣追論景泰廢立事往,憲宗切責說:「景泰事已往,朕不介意,且非臣下所當言。」另䆁放了浣衣局婦女和願歸宮人,又恢復明景帝帝號。文治上憲宗體諒民情,蠲賦省刑,任用賢臣,考察官吏,勵精圖治,善政史不絕書,儼然為一代明君,當其時朝廷多名贤俊彦,百姓得以休养生息,史稱成化新風,堪稱與仁宣之治媲美,朝鲜、琉球、哈密、烏斯藏、暹羅、吐魯番、撒馬兒罕、日本、蘇門答剌等國紛紛入貢。人口方面在成化十五年(1479年)中成為終明一代的人口峰值,達9,496,265戶,71,850,132人,反映當時明朝仍然處於盛世。

武功上憲宗恢復十二團營制度,幾次親閱騎射於西苑,巡查禁軍,整飭軍備,考試士兵訓練,還任用王越、余子俊、秦紘、朱永、朱英等能臣處理軍務,修建邊牆,并從不斷南下入侵盤踞河套的韃靼部手裡,一舉收復河套地區,使得套寇問題基本解決。在紅鹽池大捷中,明軍大破韃靼大營,擒斬三百五十人,獲駝馬器械不可勝計,史书記載「虏自是不敢复居套内者二十年,则此捷为所震慑故也。」「自是不复居河套,边患少弭;间盗边,弗敢大入,亦数遣使朝贡。」甚至在後來威宁海大捷中夜行晝伏直捣蒙古可汗王庭,生擒幼男婦女一百七十,斩首四百三十七级,獲旗纛十二面,馬駝牛羊六千餘,盔甲弓箭皮襖之類又萬餘,达延汗巴图蒙克仅以身逃。另外自從明英宗以來,盤踞在建州的李满住、董山屢寇掠辽东,逐漸成為邊患,明憲宗在多次招撫不果後決定用兵撻伐,先後於成化三年與成化十五年,明軍與朝鮮聯手進攻屢次犯邊的建州女真,生擒數百人,斩首千餘級,破四百五十餘寨,夺回被掳人口數千人,擒斬罪魁禍首的董山,史稱成化犁庭或丁亥之役。

明朝皇帝多擅畫像,作字運筆,憲宗亦擅畫神像,曾為張三豐畫像,神采生動,超然塵表,又曾親筆御製一團和氣和歲朝佳兆等畫流世,畫法老練嫻熟,頓挫自如。成化十八年,憲宗又親自編寫了《文華大訓》一書,以教導太子人倫治國之道,垂訓子孫。而《貞觀政要》自唐流傳至明,版本注釋繁亂,明憲宗即位後,立即組織儒臣對其進行校定,把宋元史纂輯的綱目皆寫入書中,頒示天下,即流傳至今的成化本,又為重修的孔子廟碑和《貞觀政要》作親自序。憲宗在《貞觀政要序》中寫道「朕萬幾之暇,悅情經史,偶及是編...太宗在唐為一代英明之君,其濟世康民,偉有成烈,卓乎不可及己,所可惜者,正心修身二帝三王之道,而治未純也。朕將遠師往聖,允迪大酋,以宏其治。」足見他的治國抱負和文化素質。

憲宗在位中后期,好方術,沉溺後宮,极度宠信大他19歲的万贵妃,又生活奢靡,取國庫填內帑并擴置皇莊,同时又任用太监汪直、梁芳等奸佞當權,以致西廠橫恣,朝紳諂附,且明憲宗直接頒詔封官,是為傳奉官,這使得傳奉官氾濫,舞弊成風,朝政荒芜。但整體而言,成化晚年,朝廷依然能有條不紊地對天災人禍有迅速的應對,因此仍幸稱歌舞升平,太平無事。

成化初年,土地兼併嚴重,造成大量流民依山據險,光是荊州、襄州、安州、沔州之間,“流民不下百萬”。湖廣荊襄地區成為流民的聚居區,賊盜嘯聚。成化元年(1465)三月劉通、石龍、馮子龍等於房縣大石廠立黃旗起義,擁眾數十萬。成化六年十一月,又有劉通舊部李原、小王洪起義,流民附和者達百萬人。史稱鄖陽民變。

成化二十三年(1487年)春,萬貴妃去世,憲宗過於悲痛而患病,長歎說:「萬氏長去了,我亦將去矣。」日漸消瘦,最終於同年八月廿二日駕崩,享年39歲(虚龄四十一)。葬於北京昌平茂陵。臨終前誨示太子要敬天法祖,勤政愛民,太子頓首受命,他的三子朱祐樘繼位,即后来的明孝宗。

明宪宗即位後任用李贤、彭时、商辂等人,下诏為于谦平反,派人去為于谦扫墓,并让其子于冕袭为千户,于谦的女婿朱翼等人,也被归还家产。

荆襄刘通造反,命抚宁伯朱永讨伐,将之平定。又有陕西周原土官满四占据石城,荆襄復反,憲宗力排众议,命项忠平定,荆襄贼平,明军击斩万人,首领刘通、苗龙等四十人被生擒献俘京师。宪宗又专门派出了杨璇抚治荆、襄、南阳流民,史載「大会湖广、河南、陕西抚、按、藩、臬之臣,籍流民得十一万三千余户,遣归故土者一万六千余户,其愿留者九万六千余户,许各自占旷土,官为计丁力限给之,令开垦为永业,以供赋役,置郡县统之。 」。此後流人得所,四境乂安,直至明未,荆襄再也沒有出現大亂了。

蠲賦省刑是成化一朝最為後人津津樂道的善政之一,史記憲宗「一聞四方水旱,蹙然不樂,亟下所司賑濟,或輦內帑以給之;重惜人命,斷死刑必累日乃下,稍有矜疑,輒從寬宥。」「憲宗好生,每奏讞大辟(死刑奏章),多所寬宥,或不得已而行刑。其日必卻八珍之奉,默坐焚香。哀矜之意,惻然見於玉色。」自他即位自駕崩唯止,僅在官田減免稅糧一項則已達一千九百多萬石,在民田稅額的蠲免和下內帑賑濟更是不計其數,僅以成化二十一年為例,實錄記載當年減免天下官田等項稅糧一百零八萬五千九百石,然而憲宗除此之外在該年正月從內庫中撥帑二十萬五兩賑濟災民,四月又撥漕糧四十萬賑災,同月與十月又免山東濟南、山西平陽、四川成都、河南開封、南直隸鳳陽等州府稅糧,總計連同官田稅賦該年蠲免三百萬石,相當全國稅額十之二一,可見憲宗不吝恤民。因此儘管成化一朝水旱災變不斷,在荆襄流民問題處理完後,再也沒有出現較大的社會波場動。

橫觀成化年間的最值得稱道的善政,除了處理荊襄流民與蠲賦省刑外,其次莫過於改革漕運,自明成祖永樂遷都以來,北京便依賴南糧北運,其中需要每年徵集大量民伕運糧,路途波折,時常耽誤農時,自成化七年後,朝廷減省少了民伕的運輸路程,改由官兵漕軍長運,雖然朝廷的加耗增加了,但節約了百姓的農時,有利農業生產,同時又制定了各類考課規條,自此以後明代的漕運才有了完備的制度,此制一直沿用至明末。

手工業者在成化年間身份有了進一步的自由,明太祖建國時,分天下百姓為軍民匠灶四類,手工業者便被歸類在匠户中,他們各分「住坐」和「輪班」,他們必須義務定期(通常五年一班,每班服役三個月)為朝廷工作,有時還要無償服役,於是逃役者越來越多。成化二十一年起,朝廷允許輪班匠不願服役者可以每月出錢免役,改由朝廷直接雇工造作,這不但令朝廷毋須再終年追捕工匠,勞官擾民,手工業者只要付出二三月的銀子,便可以免除三月的工役之苦和回來花費的時間,也換來四年的人身自由。

在位初期,天下称颂其统治;但宠信万贵妃后,朝政转向晦暗,万安开始得势。又设置西厂,命太监汪直提督外事,于是汪直便随意罗织罪名生事。汪直仗势将陈钺,威宁伯王越变为自己的羽翼,依附自己之人便任用,不听自己话的人就排挤打击,权势极为显赫,天下都惧之三分。汪直又想在外立功,胡乱进行边界挑衅。宪宗命汪直掌管十二团营。当时有个名叫阿丑的中官,善演诙谐幽默戏,经常在宪宗面前表演,颇有汉朝东方朔用滑稽方法进谏之风。一天阿丑假装喝醉酒,旁边一个人在佯装说:“某官到!”阿丑任装醉意大骂,人又说:“皇驾到!”阿丑还是醉骂如故,那人又说:“汪太监来了。”阿丑所装的醉人赶紧起来惊恐的站在一边。旁边的人问到:“天子驾到都不害怕,为什么害怕汪太监?”阿丑说:“我只知有汪太监,不知有天子。”自此以后汪直逐步失宠。此时王越和陈钺讨好汪直,三人结为死党。阿丑一日有在做戏,自己扮演汪直手持双斧向前前行,有人问其缘故,答说:“这双斧是王越和陈钺。”宪宗听后微笑了一下。御史徐鏞等人弹劾汪直欺君枉法,擅开边衅,宪宗后渐疏远汪直。

被宪宗先后任用的宰輔有:李賢,陳文,彭時,呂原,商輅,劉定之,萬安,劉珝,劉吉,彭華,尹直。对成化一朝,世有“紙糊三閣老,泥塑六尚書”之謠,三閣老指萬安、劉吉和劉珝,六尚書指尹禕、殷謙、周洪謨、張鵬、張鎣和劉昭,意讽这些朝廷重臣不作为,私德不佳,但也有意見認為他們之所以被抨擊,并非庸懦無能,貪贓枉法,而是因為對明憲宗專寵萬貴妃,內批傳奉官的行為沒有進行有力勸諫,使明憲宗符合傳統儒家人君規範,其實從成化後期對災區和地方事務的應對裁決,可見他們還是各有所長、恪盡職守的,因而即使同萬安這世稱的奸倖之臣,卻也見容於當其時彭時商輅等名臣官員中。

明憲宗本人曾經向兒子朱祐樘概括自己的一生作为:「修文史而究武略,饬内治以攘外侮,戡靖僭窃,应宁邦家,犹宵旰靡遑,惧功业未茂,德惠未周,而治平之效未臻也。」

《明實錄》:「葢上以守成之君,值重熙之運,兵革不試,萬民樂業,垂拱而天下大治矣。」

《名山藏》何乔远:上聪明仁恕,渊默勤恭,孝事母后如古帝王。郊庙斋祭,必极诚敬。景皇帝尝有封沂之命,未尝一语及之。委任大臣,略无猜忌,或即干纪,屏斥无疑。一闻四方水旱,戚戚然下所司赈济,或辇内帑给之。重惜人命,断死刑累日乃下。夙兴视朝,但遇雨雪辄放常参官而不废奏引。隆寒盛暑,或减奏事,以恤卫士侍立之劳。间有游豫,不出大内,如南囿祖宗时不废游猎,上未尝一幸焉。时御翰墨,作为诗赋,以赐大臣。诸司章奏,手自披阅,字画差错,亦蒙清问。臣下益兢业职事,莫敢或欺。葢上以守成之君,值重熙之运,兵革不试,万民乐业,垂拱而天下大治矣。

《国榷》谈迁:恤饥察冤,求言课吏,先后史不绝书,而于胡僧幸阉斜封墨敕之滥,亦不能为帝掩也。当其时,朝多耆德,士敦践履,上恬下熙,风淳政简,称明治者,首推成弘焉。而或有遗议,则在汪直、李孜省、繼曉辈蚀其一二,于全照无大损也。尺璧之瑕,乌足玷帝德哉!末谕太子以敬天法祖、勤政爱民之道,俨然成周之遗训也。说者谓帝初欲易储,以泰山屡震而止。噫!帝能尊钱后,复景帝,俱事出常情之外,而乃轻视东宫?必不然也。

《国榷》郑晓:帝仁恕英明,少更多难,练达情理。临政莅人,不刚不柔,有张有弛。进贤不骤而任之必专,远邪不亟而御之有法。值虏寇数侵边,惟遣将薄伐,不勤兵以竭我财力,虏亦离散,内外宁辑。荆襄岭海,时有寇窃,推毂之际,戒勿妄杀,或不用命,赏罚兼行。崇上理学,褒封儒贤。江淮大祲,截漕赈饥。星文示变,侧身省过。臣僚进谏,即涉浮伪,时有干忤,薄示谴谪,旋蒙牵复。若乃尊礼孝庄,尊景帝,保护汪后,褒恤于谦,其于爱憎恩怨,绝无芥蒂,帝谆然于天理彝伦者也。以故虽屡有彗孛之灾,而国家康靖,有繇然矣。

《国榷》李维桢:詩有之,“靡不有初,鮮克有終”,人情哉!純帝初載,亦何其斤斤也。中官幸,禱祠繁,而治隳矣。錢後之祔廟食,景帝之復位號,此兩者,雖甚盛德蔑以加已。

《明史》贊曰:「憲宗早正儲位,中更多故,而踐阼之后,上景帝尊號,恤于謙之冤,抑黎淳而召商輅,恢恢有人君之度矣。時際休明,朝多耆彥,帝能篤于任人,謹于天戒,蠲賦省刑,閭里日益充足,仁、宣之治于斯复見。顧以任用汪直,西厂橫恣,盜竊威柄,稔惡弄兵。夫明斷如帝而為所蔽惑,久而后覺,婦寺之禍固可畏哉。 」

《朝鮮成宗實錄》:上(成宗)御宣政殿, 引見明澮等, 謂曰: 「中國有何事?」 明澮對曰: 「(憲宗)皇帝勤於聽政, 天下太平, 民物富庶。」(時成化十一年)

《剑桥中国明代史》中写道:「朱见深与他的有军事头脑的祖父和父亲相同,向往他们的生气勃勃的、甚至具有侵略性的军事姿态,并且厚赏有成就的军事将领。」

負面事蹟主要與其大19歲的妃子萬貞兒的感情和鬆散的管理有關。

《罪惟錄》論曰:災異之警,無有酷於此二十三年者也。宮中位一女戎,而群小相緣益進,惑匿導誘,顛例黜陟,以致傳升無己,監督四出,閣輔阿循,廠衛搜射。而帝又旋悟旋迷,嘉言罔入,邊釁苗殘,幾無寧歲。天乃至仁,歷以所警,貫耳而呼,而其如溺柔聽者,袖不聞也。祗幸蠲賑免租,無少稽吝,猶不致啟中原之怒。且內外寡大故,無所藉以起,幸稱小康。嗟乎!哲婦傾城,危矣哉!

《明史講義》:凡此皆成化時朝政之穢濁,而國無大亂,《史》稱其時為太平,惟其不擾民生之故。

《朝鮮成宗實錄》:(司憲府掌令李琚)更啓曰: 「臣於丙午年往中國, 中國人言, 成化皇帝非賢君也, 然一用《大明律》, 故朝廷寧謐, 四方無虞矣。 臣今所啓, 別無他意, 欲殿下遵守舊章而已。」(朝鮮成宗) 傳曰: 「爾陪臣也, 而褒貶天子, 則我諸侯也, 何不褒貶我乎? 爾非新進之儒, 曾經弘文館, 爾不知予心而如此言之耶?」

《明朝時代上卷第38章陳獻章和他的心學》:成化王朝是明王朝歷史上的一個轉折點,正是在這個時期基本結束了朱元璋一百年來禁錮帝國的政策,從此帝國又重新恢復到唐宋元的那種自由、奔放的年代,商業開始復甦、城市開始繁華、思想文化開始活躍、士紳的生活開始奢靡,在這個社會整體鬆動下,起到穩定、凝聚作用的理學思想也開始搖搖欲墜,它必將被更能適應社會發展的思想所代替。

《成化皇帝大傳》:成化朝君臣们是预测不到的,他们留给弘治朝君臣的,乃是一个外无强敌,内无大敌,百业兴旺,万民乐业的太平世道。

\subsection{成化}

\begin{longtable}{|>{\centering\scriptsize}m{2em}|>{\centering\scriptsize}m{1.3em}|>{\centering}m{8.8em}|}
  % \caption{秦王政}\
  \toprule
  \SimHei \normalsize 年数 & \SimHei \scriptsize 公元 & \SimHei 大事件 \tabularnewline
  % \midrule
  \endfirsthead
  \toprule
  \SimHei \normalsize 年数 & \SimHei \scriptsize 公元 & \SimHei 大事件 \tabularnewline
  \midrule
  \endhead
  \midrule
  元年 & 1465 & \tabularnewline\hline
  二年 & 1466 & \tabularnewline\hline
  三年 & 1467 & \tabularnewline\hline
  四年 & 1468 & \tabularnewline\hline
  五年 & 1469 & \tabularnewline\hline
  六年 & 1470 & \tabularnewline\hline
  七年 & 1471 & \tabularnewline\hline
  八年 & 1472 & \tabularnewline\hline
  九年 & 1473 & \tabularnewline\hline
  十年 & 1474 & \tabularnewline\hline
  十一年 & 1475 & \tabularnewline\hline
  十二年 & 1476 & \tabularnewline\hline
  十三年 & 1477 & \tabularnewline\hline
  十四年 & 1478 & \tabularnewline\hline
  十五年 & 1479 & \tabularnewline\hline
  十六年 & 1480 & \tabularnewline\hline
  十七年 & 1481 & \tabularnewline\hline
  十八年 & 1482 & \tabularnewline\hline
  十九年 & 1483 & \tabularnewline\hline
  二十年 & 1484 & \tabularnewline\hline
  二一年 & 1485 & \tabularnewline\hline
  二二年 & 1486 & \tabularnewline\hline
  二三年 & 1487 & \tabularnewline
  \bottomrule
\end{longtable}


%%% Local Variables:
%%% mode: latex
%%% TeX-engine: xetex
%%% TeX-master: "../Main"
%%% End:

%% -*- coding: utf-8 -*-
%% Time-stamp: <Chen Wang: 2021-11-01 17:12:45>

\section{孝宗朱祐樘\tiny(1487-1505)}

\subsection{生平}

明孝宗朱祐樘(1470年7月30日-1505年6月9日),或稱弘治帝,是明宪宗皇三子。明朝第10代皇帝(1487年-1505年在位),他在位18年,年号弘治。孝宗“恭俭有制,勤政爱民”,又能信用贤臣、广开言路,在位期间“朝序清宁,民物康阜”,明朝出现中兴局面,史称“弘治中兴”。但在位后期對朝政有所懈怠,又縱容外戚,沉迷方術,使宦官李广、蒋琮等人乘机弄权,以致弘治晚年軍備弛廢,國用匱乏,弊政颇多,故不能谓之全美。明孝宗崩逝後谥号「建天明道诚纯中正圣文神武至仁大德敬皇帝」,庙号「孝宗」,葬于泰陵。

根据《明史》记载:“孝宗达(实为“建”,《明史》误)天明道纯诚中正圣文神武至仁大德敬皇帝,讳祐樘,宪宗第三子也。母淑妃纪氏,大明成化六年七月生帝于西宫。时万贵妃专宠,宫中莫敢言。悼恭太子薨后,宪宗始知之,育周太后宫中。十一年,敕礼部命名,大学士商辂等因以建储请。是年六月,淑妃暴薨,帝年六岁,哀慕如成人。十一月,立为皇太子。”民間則傳說:孝宗出生时,为免被当时的寵妃萬貴妃害死而藏在民間,在憲宗死前才由宮內太監於民間迎回即位。

孝宗出生後,廢后吳氏貶居西內,與紀氏謫居的安樂堂相近,頗知消息,往來就哺,才得保全孝宗生命,由吳氏用心撫養過一段日子。

弘治帝在位初期,励精图治、整肃朝纲、改革弊政,罢逐了朝中奸佞之臣、重用贤士,为于谦建祠平冤,减轻赋税、停征徭役、兴修水利、发展农业、繁荣经济,史稱“弘治中兴”。

弘治帝在位期间“更新庶政,言路大开”,启用了刘健、丘濬、李东阳、谢迁、王恕、马文升、刘大夏等能臣,使明憲宗成化朝晚年以来,奸佞当道的局面,得以大为改观。

此外,弘治帝重視司法,他令天下諸司審錄重囚,慎重處理刑事案件。弘治十三年(1500年),制定《問刑條例》。又於弘治十五年(1502年),編成《大明會典》。

弘治帝在治理水患方面亦頗有效果,曾委任白昂、劉大夏修治黃河,以改善河道流向、築堤等方法抑制黃河水患,此後二十餘年間,再無大患發生;另外,蘇松於弘治年間,曾因河道淤塞而泛濫成災,孝宗即命徐貫主持治理,歷時三年,消除了蘇松水患。

弘治帝在位初期的經濟成就也比較突出,賦稅收入比成化年間增加了一百多萬石,達二千七百萬石,成為明中葉的賦入高峰;而且,人口方面也有穩定的增長。從弘治元年(1488年)到弘治十七年(1504年)間,人口增加了一千多萬,達到六千萬口。

惟自弘治十五年起(1502年),「一歲所入,不足以供一歲支用」,國家財政邁進了入不敷出的狀況,戶部呂鈡指出:『常入之賦,以蠲色漸減,常出之費,以請乞漸增,入不足當出。正純以前軍國費省,小民輸正賦而已。自景泰至今,用度雜辦,皆昔所無。民已重困,無可復增。往時四方豐登,邊境無調發,州縣無流移。今太倉無儲,內府殫絀,而冗食冗費日加於前。』對此下廷臣議,廷臣作出多項建議,但僅觸及成效不大的修補政策。

此外,孝宗也常以京營禁軍投入繁重的工作,監察御史劉芳曾上奏說,“京師根本之地而軍士逃亡者過半”,“其錦衣騰驤等衛軍士不下十餘萬人,又不繫操練之數,近年雖立營營,而役佔賣放者多。”,另外又常縱容邊臣,邊臣冒報功次皆得升賞,而敗軍失律者往往令之戴罪殺賊,使邊備日弛,對於北虜入侵能有效抵禦的戰役寥寥無幾,如弘治十四年秋七月,孝宗令保國公掛征虜大將軍總兵官領十萬大軍夜襲韃靼於河套,韃靼早察覺徙家北遁,朝廷用銀八十餘萬,只斬首三級以還,而將士奏報功次竟一萬有餘,“不能禦”,“坐虜入境”,“議者恥之”之類的描述比比皆是。

再者,弘治中期,皇帝自己漸漸迷上了齋醮,從此內庫開銷劇增,孝宗開始不斷地命戶部將太倉庫的銀子納入內庫,至將河西務鈔關關船料改擬折銀進納。如弘治十五年(1502年)十月,戶部指出“銀承備庫先前進,金止備成造金冊支用;銀止備軍官折俸及兵荒支給,近年累稱不足。金則以稅糧折納及於京市買過八千三百八十六兩有奇,五次取太倉銀共一百九十五萬,”而從戶部納入內庫的銀兩,全部都被孝宗挪用來大興土木,又妝造武當山神像,各寺觀修齋賞賜,修齋設醮等,恣意浪費,以致府藏空竭,國庫捉襟见肘。而且孝宗在統治中期(1500年)後,漸漸不如當初勤政,且開始縱容外戚,措置乖方,如內閣輔臣劉健,徐溥就曾批評孝宗說「切見數月以來視朝漸遲多至日出」,「近年以来用度太侈,光禄寺支费增数十倍,各处织造降出新样动千百匹,显灵朝天等宫泰山武当等处修斋设醮费用累千万两,太仓官银存积无几,不勾给边而取入内府至四五十万,宗藩贵戚求讨田土占夺盐利动亦数十万。」,「事涉於近幸貴戚,牢不可破,或旨從中出,略不預聞,或有所議擬,徑行改易。」,而閣臣李東陽也曾直言弘治後期「冗食太眾,國用無經,差役頻煩,科派重疊。京城土木繁興,供役軍士財力交殫,每遇班操,寧死不赴;勢家巨族,田連郡縣,猶請乞不已。親王之藩,供億至二三十萬。」「天津一路,夏麥已枯,秋禾未種,挽舟者無完衣,荷鋤者有菜色。盜賊縱橫,青州尤甚。南來人言,江南、浙東流亡載道,戶口消耗,軍伍空虛,庫無旬日之儲,官缺累歲之俸。」「今天下民窮財盡,其勢已極。姑以三者言之,山東之地草根樹皮掘食殆盡,繼以人肉,荊沔諸湖水竭魚荒,河泊諸課率多折納,易州山廠林木已空,漸出關外一二百里,其他賦稅大抵皆然,天下之地無一處而不貧」。朝中大臣如禮部尚書倪岳也上疏極言道「(孝宗)近日視朝頗晏,聽納頗難,經筵稀,御用度漸侈,游幸漸頻,進貢之止者複來,樂戲之斥者複取。」但孝宗也不願意聽納,而名臣劉大夏請辭時也言「臣老且病,窃见天下民穷财尽,脱有不虞,责在兵部,自度力不办,故辞耳。」,而吏部右侍郎周經則言「(孝宗)幸賞齋醮屢修,游宴無節,內帑空虗多由於此。」,南京戶科給事中張宦也上書道「近來(孝宗)費出無經,或橫恩濫賜之溢出,或修飾繕造之泛興,或祈禱遊玩之紛舉,偶因內帑稍闕即命太倉支取,耗散財物莫此為極」 「今四海民窮財盡,三邊將寡兵疲,糧草空虗,馬匹倒死而黠虜跳粱之勢,貪狼之心視昔尤勝」,禮科左給事中葉紳也言「邇來(孝宗)經筵稀御日講不舉,畫工琴士承恩於便殿,教坊雜劇呈技於左右..少滯視朝時,晏鰲山觀燈或徹曉不休宮中燕享或竟日乃已。」,兵科给事中王廷相奏「今天下大可忧者,在于民穷财尽,其势渐不可为。然所以致此者有四,风俗奢侈也,官职冗滥也,征赋太繁也,酒酿无节也」。可見弘治中晚年皇帝倦勤,國家敗政拮据,百姓困苦的情況。

在統治的十八年中,召見閣臣的次數總共有九次,比成化帝二十三年來召見一次為多。明孝宗即位之初,會聽進閣臣的諫諍,但是後來用各種方法來搪塞閣臣和科道官的建議,使弘治初年所革除的弊政,不僅全部恢復,尚且有惡化之勢,如憲宗晚年的傳奉官號稱弊政,弘治初盡行革除,到了弘治十二(1499年)年五月,傳升乞升文職至八百四十餘員,武職至二百六十餘員,比成化末年增一倍。其次,在軍事方面,從弘治一朝起亦開始糜爛,邊備日弛,人浮於事,有效抵禦的入侵寥寥無幾,也不復當年成化一朝了。另外,有明一代,以弘治對外臣最為縱容厚待,動則大肆外戚藩王賞賜房屋和田地,甚至在一宗貴戚莊崎糾紛案中,偏幫小舅子張延齡,一次就得地一萬六千七百零五頃;又如曾在弘治十三年(1500年)二月,賜興王湖廣京山縣近湖淤地一千三百五十餘頃,旋在七月又賜岐王德安府田六百一十二頃等等,賞地史不絕書,引起嚴重的土地兼併問題。

弘治十八年(1505年)五月初七日,因偶染风寒,误服药物,鼻血不止而死,

當時“深山穷谷,闻之无不哀痛”。有遗命:“东宫年幼,好逸乐,先生辈善辅之。”是年十月葬於泰陵。長子明武宗繼位。

孝宗即位时所面临的政治局面混乱不堪,由于他父亲明宪宗在位后期重用宦官和奸佞,造成了“朝中皮秕政”的状况。为了振兴帝业,肃清吏治,他在人事上的改革和整顿,可謂大刀阔斧。对太监梁芳、礼部右侍郎李孜省等前朝奸佞惩罚严厉。将冒领官俸、总计三千多人的艺人、僧徒等一概除名。在清理过程中,朱祐樘注意方式、方法,没有大开杀戒,斬殺的只有罪大恶极的僧人继晓 。与此并举,孝宗开始任用贤能之士。1492年三月,孝宗下令吏、兵两部将两京文武大臣、在外知府守备以上的官吏姓名,全部抄录下来,贴在文华殿的墙壁上,遇有迁罢之人,随时更改。他还多次向吏部、都察院指出,提拔和罢免官吏的主要标准,是看此人有無实绩。由于孝宗注意任用贤能,明朝中期出现了许多名臣,形成了“朝多君子”的盛况。

朱祐樘即位初年,广开言路。上台不久,就出现了臣子纷纷上书的局面,连尚未做官的太学生也跃跃欲试,上书提出各种建议。孝宗也有奢侈的想法,于是计划在万寿山建造一座棕棚,以备登临眺望。太学生虎臣得知此事,力谏不可,负责这项工程的朝中官员担心獲罪,抓住虎臣。孝宗闻知此事,先取消了工程,且授予虎臣七品官,派往云南做了知县。孝宗还采纳了除早朝之外,再在便殿召见大臣,谋议政事,当面阅读奏章,下发指令的建议,开始增加“午朝”,每天在左顺门接见大臣,倾听他们对政事的见解。

有說法認為:孝宗统治期间所实行的一系列的政策,都自始至终地得以贯彻执行,然而有學者指出,在弘治十四年,孝宗因朝廷財政拮據,以及軍餉籌措有困難而下詔群臣商議辦法,大學士劉健上奏要求改革弊端,並絕無益之費,躬行節儉,孝宗卻未採取措施。至弘治十五年,國家財政入不敷出:「常入之賦,以蠲色漸減,常出之費,以請乞漸增,入不足當出。正純以前軍國費省,小民輸正賦而已。自景泰至今,用度雜辦,皆昔所無。民已重困,無可復增。往時四方豐登,邊境無調發,州縣無流移。今太倉無儲,內府殫絀,而冗食冗費日加於前。」但僅作出成效不大的修補政策。

1489年,内阁大臣刘吉数兴大狱,迫害了一批官员;信任太监李广,开始修炼斋蘸之术。孝宗對此自我检讨。

據美國牙醫學會的資料表示,明孝宗於1498年把短硬的豬猔毛插進一支骨製手把上成為牙刷。

1501年,崛起的鞑靼部落以十万骑兵从花马池、盐池杀入固原、宁夏境内,这一事件震惊了孝宗。为了加强军事力量,1502年,孝宗将刘大夏提升为兵部尚书,负责军事整顿。刘大夏核查了军队虚额人手,补进了大量壮丁,并请朱祐樘停办了不少“织造”和斋蘸。

作为改良,孝宗没有从制度上对百姓的税赋负担进行突出的改变,而在减轻百姓负担上,减免灾区的赋税征收。从1490年,河南因灾免秋粮始,他对每年奏报来的因灾免税要求,几乎是无一例外地表示同意。

清修《明史》高度评价明孝宗:明有天下,传世十六,太祖、成祖而外,可称者仁宗、宣宗、孝宗而已。仁、宣之际,国势初张,纲纪修立,淳朴未漓。至成化以来,号为太平无事,而晏安则易耽怠玩,富盛则渐启骄奢。孝宗独能恭俭有制,勤政爱民,兢兢于保泰持盈之道,用使朝序清宁,民物康阜。《易》曰:“无平不陂,无往不复,艰贞无咎。”知此道者,其惟孝宗乎!

《国榷》:孝宗在东宫,久稔知其习。首罢幸相,次第厘革,改步之初,中外鼓舞,晓然诵明圣,识上意所向也。优容言路,汇吁良士,六卿之长皆民誉,三事之登皆儒英。讲幄平台,天听日卑,老臣造膝之语,不漏属垣,少年恸哭之谈,尝为动色。故良楛鉴断,刑赏恬肃。虽寿宁之戚,天下艳之,然宠若窦宪,尚难泌水之园,骄即武安,未请考工之宅,则帝心端可知矣。

方志远在其著作《明代国家权力机构及运行机制》中对明孝宗持否定态度,称其“弱智”并详细解释道:“弘治时代夹在成化、正德之间,前有万贵妃、汪直与西厂,后有刘瑾、八虎及内行厂,加之成化帝的内向和正德帝的荒唐,故弘治帝被明人称为‘中兴之主’。清人作《明史·孝宗纪》,其赞曰:‘明有天下,传世十六,太祖、成祖而外,可称者仁宗、宣宗、孝宗而已。仁、宣之际,国势初张,纲纪修立,淳朴未漓。至成化以来,号为太平无事,而晏安则易耽怠玩,富盛则渐启骄奢。孝宗独能恭俭有制,勤政爱民,兢兢于保泰持盈之道,用使朝序清宁,民物康阜。’并称唯有孝宗知《易》所说的‘无平不陂,无往不复,艰贞无咎’之道。但黄仁宇在《万历十五年》中指出,孝宗之为文臣所称道,就是因为他比较愿意听文臣的摆布。而实际上,孝宗不仅为文臣摆布,更受内臣摆布,从其种种行事,应该是个智商较低或者说是一个相对弱智的皇帝。”方志远在书中表示将‘另具文考证’,但相关文章尚未问世,因此,关于这个评价也存在一定争议。

郭厚安在其著作《弘治皇帝大传》中称明孝宗“盛名之下,其实难副”。他表示“从总体上,他(明孝宗)比其祖父英宗、其父宪宗以及其子武宗、侄世宗等都要略高一筹,坏的方面也没有他们突出。因此可以说,他之所以受到赞颂,是与前后诸帝比较的结果”;“朱祐樘不过是一个‘中主’而已”;“总之,朱祐樘绝不是雄才大略、大有作为之君,当然也不是荒淫的昏君,而是平庸的、力求维持现状的‘太平天子’。”

查继佐的《罪惟录》中,对明孝宗的成就和不足如此评价:“帝业几于光昌矣。群贤辐辏,任用得宜,暖阁商量,尤堪口法。斥妖淫,辟冗异,停采献,罢传升,革仓差,正抽分,种种明断外,尤莫难于孝穆、孝肃之别祀,万贵妃之免议,于肃愍之旌功。所谓情而安之于义,又列辟之所不能忘也。升遐之日,万姓哀号,岂偶然哉!若夫待外戚过厚,赐予颇滥,冗员尚多,中贵太盛,或移心斋醮,纷费,盖积渐者久,未能遽革也。夫果深有得于《太极》、《西铭》诸图书,即何难骑龙而上仙哉!”查继佐尽管也为弘治辩解,但与上述史家不同的是,究竟委婉地指出了明孝宗的不足。

《朝鮮成宗實錄》上(朝鮮成宗)曰:“常慮建州野人邀截於中路,今卿好還,甚可喜也。中國太平乎?” 自貞曰:“太平。但聞皇帝不豫,朝會望見,天顔殊瘦,皇帝初卽位,皆稱明斷,今紀綱不嚴,雨暘不若,年穀不登,民甚困窮。向者朝會,朝臣各以位次序立,莫敢私語,今則或聚立私語,以此知紀綱不嚴也。”

《明朝時代上卷 第42章 弘治王朝的老生常談》:“後世史學家多將弘治王朝稱作“弘治中興”,但從更寬廣的歷史視野來看,這些其實都經不住推敲,從宣德王朝開始,文人們所認為的明朝衰敗,實際上並不存在。皇帝不臨朝、宦官跋扈、軍屯被破壞、京畿部分民田被侵占,這些在士大夫看起來,好像不可理喻的事情,實際上無關這個大明王朝的痛癢,正統、成化年間,我們的大明王朝仍舊是平穩、正常運行的,不僅如此,從中可以看出三個趨勢,那就是政治依賴日益成熟、穩定的官僚集團運作,商業貿易開始興起,哲學文化思想領域開始鬆動,這都是值得正面看待的事情。大歷史觀,對於歷史的觀察,不應該再是只從《是否符合儒家行為規範》來看待,如果繼續這樣看待歷史,就會使我們中國人陷入一種狹隘束縛的歷史發展桎梏中。 仔細分析正統、成化王朝的所謂衰敗,是因為史學家們以當時的君主統治行為,不符合儒家行為規範而已,而弘治王朝的所謂中興,也是因為弘治皇帝遵循了文人士大夫們的儒家王道意識,因為前朝感覺衰敗,才會存在後來的感覺中興。以弘治皇帝努力將自己塑造成一個仁君形象,這是值得嘉許的。但最後這些都無濟於事,皇帝的人性與權力,超越了士大夫們的儒家王道意識與封建禮法,這衝突使一代明君轉眼變為昏君,史學評論家立即改觀,對弘治王朝前面與後面的一個總結,就是不完美。”

有人根據清修《明史》、《明書》等資料記載,認為孝宗僅娶妻孝康敬皇后張氏一人,沒有其他妃嬪或妾室。並且孝宗的泰陵只葬有夫妻兩人。而實際上根據《勝朝彤史拾遺記》及《罪惟錄》所載,孝宗至少還有沈璚蓮、鄭金蓮(《罪惟錄》稱其小字黃兒)兩位選侍。因為各種史書中對於妃嬪傳記因有事跡可記、有立傳價值,取捨各有不同,參見《萬曆官修本朝正史研究》中「八種史書關於明太祖等十位皇帝后妃立傳情況表」。而大部分的妃嬪因為地位的關係都不能葬入明帝陵中。

至於孝宗宫中有五名夫人:敬順夫人邵氏,安和夫人周氏,安順夫人劉氏,榮順夫人孟氏及榮善夫人項氏。夫人在明朝制度並非妃嬪稱號,而是命婦的封號,如外命婦(公侯伯及一二品官正室)或內命婦(資深宮人或乳母褓姆)等,內命婦中,以皇帝的乳母最常在年老後因乳帝之功而被加封為夫人(如明孝宗的保姆封为佐圣夫人、天启帝的乳母奉圣夫人客氏、仁宗褓姆衛聖夫人楊氏等,皆是有夫有家的妇人)。另,榮善夫人項氏年龄比孝宗大四十四岁,比孝宗的祖父明英宗还大一岁。因此这五名夫人实际上不是明孝宗的妃嫔。

\subsection{弘治}

\begin{longtable}{|>{\centering\scriptsize}m{2em}|>{\centering\scriptsize}m{1.3em}|>{\centering}m{8.8em}|}
  % \caption{秦王政}\
  \toprule
  \SimHei \normalsize 年数 & \SimHei \scriptsize 公元 & \SimHei 大事件 \tabularnewline
  % \midrule
  \endfirsthead
  \toprule
  \SimHei \normalsize 年数 & \SimHei \scriptsize 公元 & \SimHei 大事件 \tabularnewline
  \midrule
  \endhead
  \midrule
  元年 & 1488 & \tabularnewline\hline
  二年 & 1489 & \tabularnewline\hline
  三年 & 1490 & \tabularnewline\hline
  四年 & 1491 & \tabularnewline\hline
  五年 & 1492 & \tabularnewline\hline
  六年 & 1493 & \tabularnewline\hline
  七年 & 1494 & \tabularnewline\hline
  八年 & 1495 & \tabularnewline\hline
  九年 & 1496 & \tabularnewline\hline
  十年 & 1497 & \tabularnewline\hline
  十一年 & 1498 & \tabularnewline\hline
  十二年 & 1499 & \tabularnewline\hline
  十三年 & 1500 & \tabularnewline\hline
  十四年 & 1501 & \tabularnewline\hline
  十五年 & 1502 & \tabularnewline\hline
  十六年 & 1503 & \tabularnewline\hline
  十七年 & 1504 & \tabularnewline\hline
  十八年 & 1505 & \tabularnewline
  \bottomrule
\end{longtable}


%%% Local Variables:
%%% mode: latex
%%% TeX-engine: xetex
%%% TeX-master: "../Main"
%%% End:

%% -*- coding: utf-8 -*-
%% Time-stamp: <Chen Wang: 2019-12-26 15:07:22>

\section{武宗\tiny(1505-1521)}

\subsection{生平}

明武宗朱厚照(1491年10月27日-1521年4月20日),或稱正德帝,明朝第11代皇帝(1505年-1521年在位),享年 31歲,年号「正德」。

武宗是明朝极具争议性的统治者。他任情恣性,為人嬉乐胡鬧,荒淫无度。寵信宦官、建立豹房,強徵處女、娈童入宮,有時也搶奪有夫之婦,逸遊無度。施政荒誕不經,朝廷乱象四起。給自己化名為朱壽,自封為「鎮國公、總督軍務威武大將軍、總兵官」。又信仰密宗、伊斯蘭教等,自稱忽必烈(蒙古名,元世祖之名)、沙吉熬爛(波斯語,伊斯蘭教蘇菲派的蘇菲師)、大寶法王(藏密名,白教首領)。

另一方面,他為人刚毅果断,任内诛灭刘瑾,平定安化王、寧王之亂,在应州之役中击败達延汗,令鞑靼多年不敢深入,并积极学习他国文化,促进中外交流,体现出有为之君的素质,是一位功过参半的皇帝。

明武宗朱厚照为明孝宗嫡长子,生于1491年10月26日(弘治四年九月二十四日申时)。两岁被立为皇太子。唯一的弟弟朱厚炜又早夭,是孝宗唯一长大成人的儿子。弘治十一年春,皇太子出阁读书。他天性聪颖,讲筵时极为认真,面对讲师则恭敬对待。几个月后,便已知晓翰林院与左春坊所有讲师的姓名,以致有讲师缺席便会问询左右“某先生今日安在邪?”這讓孝宗极为喜爱,出游必带上皇太子。同时孝宗听闻皇太子闲暇时喜好兵戎事,认为他安不忘危,所以也不予以干涉。

弘治十八年五月初八日,孝宗皇帝驾崩。在完成文武百官军民耆老劝进的固定程序后,五月十八日,皇太子朱厚照即位,是为明武宗。

明正德九年正月,後來反叛的寧王朱宸濠獻新樣元宵四時花燈數百,窮極奇巧,內附火藥,明武宗命獻者入懸。时值冬季,宫中按例在檐下设有毡幕御寒。以致火星觸及氊幕,引發大火,自二鼓时分一直烧至天明。火势最大时,武宗正在前往豹房的途中,望见乾清宫的火灾,武宗向左右开玩笑称这是「好一棚大烟火也」兩天後壬午日,武宗以乾清宫灾御奉天門視朝,撤寶座不設,遂下詔罪己,並諭文武百官,同加修省。後又常常离开帝都燕京四处巡游。

住在京師期间,又不愿住在紫禁城,在宫外建了一座“豹房”居住,並甄選大量美女於其中供其淫樂。其男宠也不计其数,名曰“老儿当”,但也有學者稱,因為正德帝喜歡各地宗教,這些人主要是通曉漢文、蒙文、藏文或波斯文,作為宗教人士的翻譯官。

正德帝不喜上朝,起初宠信刘瑾、張永、丘聚、谷大用等号称“八虎”的宦官,1510年平定安化王之乱朱寘鐇后,下令将刘瑾凌迟处死,后又宠信武士江彬等人。

正德帝喜好宗教靈異、怪力亂神,终日与来自西域、回回、蒙古、乌斯藏(西藏)、朝鲜半島的异域法師、番僧相伴。正德帝曾学习蒙古语,自称忽必烈,也学藏传佛教,自称大宝法王。正德帝還曾亲自接见第一位来华的葡萄牙使者皮莱资。正德帝並因為自己生肖屬豬,曾一度敕令全国禁食猪肉,但他自己仍食用猪肉「内批仍用豕」;旋即在大學士杨廷和的反對下,降敕廢除。

正德帝“奋然欲以武功自雄”。正德十二年(1517年)10月,在江彬的怂恿下,自封为“镇国公總督軍務威武大將軍總兵官朱寿”,到边地宣府(今张家口宣化区)亲征,击溃蒙古鞑靼小王子(即达延汗巴图蒙克),回去后又给自己加封太师。史称“应州大捷”。

正德十四年(1519年)六月十四日,宁王朱宸濠在封藩江西南昌叛乱,是為宁王之乱,不過四十三天,就被贛南巡撫王陽明及吉安知府伍文定募集散兵游勇平定,斬殺三萬餘人,朱宸濠被擒。八月二十二日,武宗离开北京亲征。二十六日,武宗抵达涿州,此時王陽明平定叛乱的奏报送达,但武宗仍决定继续南幸。十二月十一日,武宗传谕内阁,以正德十五年(1520年)元旦於南京朝贺、祭祀天地。十二月二十六日,武宗御驾抵应天府。次日,祭祀南京太庙,武宗成为自永乐以后重新驾临南京的皇帝。正德十五年闰八月初八日,武宗於南京受宁王降。八月十二日,武宗离京返回北京。

正德八年(1513年)起在江南全面推行的賦稅改革,既減輕了江南當地百姓的負擔,更使從弘治晚期開始,江南地區拖欠中央累積十年之久的賦稅,僅經兩年時間就全部還清。

武宗御驾南征返回北京途中,於淮安清江浦上学渔夫撒网,作為遊戲,卻失足落入水中,并因此患病「燥熱難退」。正德十五年十二月初十,大驾回到北京,文武百官出至正阳桥外迎接。十三日,皇帝於南郊祭祀天地,祭拜过程中突然呕血,随即送入斋宫休养。次日,返回大内,仅在奉天殿举行庆成礼。此后,立春日的朝贺一同免去。正德十六年(1521年)正月初九日,监察御史郑本公鉴于武宗身体状况不乐观,上奏武宗,望能於宗室间過繼一人主掌东宫,但后来武宗身体略有好转。三月十三日晚间,武宗突然向身边的太监陈敬和苏进表示自己可能無法痊癒,让其召司礼监并禀告皇太后,由太后与内阁议处天下事,并表示自己耽误子嗣。十四日,武宗於豹房驾崩,得年29歲。

由于武宗無子嗣,因此遵照《皇明祖训》,由武宗堂弟、孝宗弟兴献王朱祐杬之子兴王朱厚熜入嗣大统。正德十六年五月,朱厚熜抵达京师,上谥号为承天达道英肃睿哲昭德显功弘文思孝毅皇帝,上庙号为武宗。九月,武宗入葬天寿山陵区的康陵。

明武宗的生辰为弘治四年九月二十四日,八字为辛亥年,戊戌月,丁酉日,戊申时出生。其中,八字地支分别为申酉戌亥,这种排列方法被称为连如贯珠。在此以前仅太祖朱元璋的八字与此类似。

賜自己的替僧為漢地噶瑪巴,正德五年封大慶法王,鑄大慶法王西天覺道圆明自在大定慧佛金印,兼给誥命,藏名為「領占班丹」,並曾邀請藏地八代噶瑪巴至北京(七代噶瑪巴曾說:「將現身兩位噶瑪巴」);蒙古名為忽必烈;波斯名為沙吉熬爛,即蘇菲師(Shaykh,回教蘇菲派長者、教長),並擁有一群伊斯蘭火者,稱為老兒當。對道教亦多有了解,可能曾號錦堂老人。

正德十五年(1520年)闰八月,武宗御驾自南京返回时,途径镇江,适逢退休居家的原内阁大臣靳贵病逝,于是亲临靳贵家中吊唁。但是随行大臣代皇帝撰写的祭文皆不能称意,明武宗遂亲自写道:“朕居东宫,先生为傅。朕登大宝,先生为辅。朕今南游,先生已矣。呜呼哀哉!”左右的侍从文学之臣看后都敛手称服。

山西应县木塔顶层有一方明武宗皇帝御匾“天下奇观”。

2004年,在美國德州一位華僑手中發現由明朝正德皇帝親筆所書的聖旨,內容敘述做人應如何有進取心以及如何為忠君之臣與正人君子。此文物的發現造成了史學家對歷史記載正德皇帝人格的爭議。

史学界对正德帝的评价不一, 有人认为正德帝雖荒淫無行,行徑胡鬧,不理國政,造成叛變日起,且自身壯年即因為逸樂而死;但是亦有人认为他頗能容忍大臣,不罪勸諫之人。君臣之間,相安無事,知错能改,诛灭奸佞。

张廷玉等《明史》贊曰:「明自正統以來,國勢浸弱。毅皇手除逆瑾,躬禦邊寇,奮然欲以武功自雄。然耽樂嬉遊,暱近群小,至自署官號,冠履之分蕩然矣。猶幸用人之柄躬自操持,而秉鈞諸臣補苴匡救,是以朝綱紊亂,而不底於危亡。假使承孝宗之遺澤,制節謹度,有中主之操,則國泰而名完,豈至重後人之訾議哉!」

談遷《國榷》論曰:「武宗少即警敏,好佚樂。……而武宗又不罪一諫臣,元相呵護,群吏奉法。……夜半出片紙縛(劉)瑾,……錢寧俛首受罪。」

吳熾昌《續客窗閒話》論曰:「……遊戲中確有主裁,但好行小慧,為儒尚且不可,況九五之尊耶?今之讀史者直以帝比之桀紂,無乃過甚。當初諡曰武宗毅皇帝,毅者果決之謂,可見遇事實能決斷,非盡阿諛可知矣。」

\subsection{正德}

\begin{longtable}{|>{\centering\scriptsize}m{2em}|>{\centering\scriptsize}m{1.3em}|>{\centering}m{8.8em}|}
  % \caption{秦王政}\
  \toprule
  \SimHei \normalsize 年数 & \SimHei \scriptsize 公元 & \SimHei 大事件 \tabularnewline
  % \midrule
  \endfirsthead
  \toprule
  \SimHei \normalsize 年数 & \SimHei \scriptsize 公元 & \SimHei 大事件 \tabularnewline
  \midrule
  \endhead
  \midrule
  元年 & 1506 & \tabularnewline\hline
  二年 & 1507 & \tabularnewline\hline
  三年 & 1508 & \tabularnewline\hline
  四年 & 1509 & \tabularnewline\hline
  五年 & 1510 & \tabularnewline\hline
  六年 & 1511 & \tabularnewline\hline
  七年 & 1512 & \tabularnewline\hline
  八年 & 1513 & \tabularnewline\hline
  九年 & 1514 & \tabularnewline\hline
  十年 & 1515 & \tabularnewline\hline
  十一年 & 1516 & \tabularnewline\hline
  十二年 & 1517 & \tabularnewline\hline
  十三年 & 1518 & \tabularnewline\hline
  十四年 & 1519 & \tabularnewline\hline
  十五年 & 1520 & \tabularnewline\hline
  十六年 & 1521 & \tabularnewline
  \bottomrule
\end{longtable}


%%% Local Variables:
%%% mode: latex
%%% TeX-engine: xetex
%%% TeX-master: "../Main"
%%% End:

%% -*- coding: utf-8 -*-
%% Time-stamp: <Chen Wang: 2021-11-01 17:13:00>

\section{世宗朱厚熜\tiny(1521-1566)}

\subsection{生平}

明世宗朱厚熜(1507年9月16日-1567年1月23日),或稱嘉靖帝,明朝第12位皇帝,庙号世宗,年號嘉靖,正德十六年(1521年),明武宗駕崩無嗣,內閣首輔楊廷和立朱厚熜入繼大統,即明世宗。谥号“钦天履道英毅神圣宣文广武洪仁大孝肃皇帝”。

世宗前期进行改革,銳意圖治,颇有作為,他说:“今天下诸司官员,比旧过多。我太祖初无许多,后来增添冗滥,以致百姓艰窘,日甚一日。”下令革除先朝蠹政,又嚴以馭下,史稱其“世宗習見正德時宦侍之禍,即位後御近侍甚严,有罪挞之至死,或陈尸示戒...又盡撤天下鎮守內臣及典京營倉場者,終四十餘年不復設,故內臣之勢,惟嘉靖朝少殺雲。”,先後裁革錦衣衛十七萬餘人。且寸斬前朝王綸、钱宁和江彬等奸臣,天下翕然稱治,時稱嘉靖中興。

但世宗受人詬病處更多,如他為了追封生父興獻王的問題,與楊廷和等朝臣引發嚴重衝突,即大禮議事件,世宗為了此事,對大臣們進行了嚴重的大清洗。世宗在位中後期也漸無心朝政,深居不出,沉迷方術,只通過內閣掌控朝局,使得嚴嵩嚴世蕃父子專權逐漸形成,又因營建繁興而濫用民力,導致府藏告匱,民眾起義無數。在宮中,世宗也暴虐無道,因為虐待宮女,導致宮女發動壬寅宮變,險些喪命。

明世宗朱厚熜是明宪宗第四子兴献王朱祐杬次子,是明孝宗之姪,明武宗之堂弟;明武宗正德二年(1507)生,母兴王妃蒋氏。

正德十六年(1521年),明武宗驾崩,無子嗣,内阁首辅吏部尚书、武英殿大学士杨廷和定策,援引《皇明祖训》,推找皇位繼承人,而武宗唯一弟弟朱厚煒幼年夭折,於是上推至武宗父明孝宗一輩,孝宗是明憲宗的第三子,兩名兄長皆早逝無子嗣,四弟興王朱祐杬雖已薨,但有二子,興王長子(朱厚熙)已薨,遂以“兄終弟及”的原則,徵在服喪的興國世子朱厚熜入京即位。朱厚熜先繼承興王頭銜,後即帝位,改元“嘉靖”,是为明世宗。

朱厚熜十四歲入繼大統,因想追封親生父母「皇帝、皇后」的尊號,但首辅杨廷和等旧臣要求他改以明孝宗為義父,而引發了長達三年半的大禮議之爭,期間廷杖打死十六人;世宗不顧朝臣反對,追尊生父為興獻帝、生母為興國皇太后,改稱孝宗曰“皇伯考”。嘉靖十七年(1538年)九月興獻帝被追尊為「睿宗知天守道洪德淵仁寬穆純聖恭簡敬文獻皇帝」,並將睿宗的牌位升袝太廟,排序在明武宗之上,改興獻王墓為顯陵,大禮議事件至此最終結束。

嘉靖帝前期推行了改革,成效显著。河南道御史刘安说:“今明天子综核于上,百执事振于下,丛蠹之弊,十去其九,所少者元气耳。”张居正在万历三年(1575)以自己少年时的亲身体验对嘉靖前期整顿学政的成就予以极高的评价。他说:“臣等幼时,犹及见提学官多海内名流,类能以道自重,不苟徇人,人亦无敢干以私者。士习儒风,犹为近古。”

隆庆二年(1568)进士李乐对嘉靖前期革除镇守中官的积极作用给予的评价,言道:“世宗皇帝继统,年龄虽小,英断夙成,待此辈不少假借。又得张公孚敬以正佐之,尽革各省镇守内臣,司礼监不得干预章奏。往瑾时,公卿大臣相见,无敢抗礼,甚有拜伏者。自张公当国,司礼以下各监局巨珰,见公竦息敬畏,不敢并行并坐,至以『张爷』呼之,不动声色,而潜消其骄悍之心。盖自汉唐宋元以来,宦官敛戢,士气得伸,国体尊严,未有如今日者,诚千载一时哉!”

因應外戚为害天下,嘉靖帝和张璁、方献夫在革除外戚世封的问题上达到了共识,下令永远废除此制,《明通鉴》编纂者说:“安昌伯钱维圻卒,其庶兄维垣请嗣爵,下吏部议。尚书方献夫等言:‘外戚之封,不当世及。’历引汉、唐、宋事以证。璁以为然,力主之。上善其言,诏:”自今外戚封爵者,但终其身,毋得请袭。’自是,外戚遂永绝世封。”

明代史学家何乔远《名山藏》总结嘉靖前期“励精化理,湔濯海内观听,挈清政本,杜塞旁落,奋武揆文,网罗才实。至于稽古礼典,取次厘毖一切,创必表章,轶往宪来,赫然中兴,多孚敬(张璁)所翼赞”。何乔远认为嘉靖前期出现的国家中兴是得益于內閣首辅张璁推行的改革。

而在嘉靖中后期,海瑞于嘉靖四十五年亦言:世宗“二十余年不视朝,法纪弛矣”。

世宗濫用夫役與國家財政之力大事興建,迷信方士、尊崇道教,好長生不老之術,每年不斷修設齋醮,造成巨大的靡費。

世宗好房中術秘方,多採處女之經血煉丹,方士陶仲文與佞臣顾可学、盛端明等进献媚药得以倖進,世宗為人暴躁兇殘,朝鮮國使臣的著作,也稱他對宮女:「若有微過,多不容恕,輒加箠楚。因此殞命者,多至二百餘人。」嘉靖二十一年(1542年)十月爆發“壬寅宮變”,幾死於宮女之手。明朝的太醫許紳用“虎狼之药”救活世宗,但是,由于他在急救世宗皇帝时,承受着“不效必杀身”的巨大压力,不多久,许绅得了病,卧床不起,嘉靖帝来看望他。他说:“吾不起矣,曩者宫变,吾自度不效必杀身,因此惊悸,非药石所能疗。”病卒,赐谥恭僖。此後世宗相繼遷居西苑萬壽宮及玉熙宮謹身精舍,至死不曾回到紫禁城大內居住,直至瀕死前才在徐階以明武宗死在宮外為例子勸說下回到大內居住。首輔严嵩專國二十年,殘害忠良,楊繼盛、沈鍊等朝臣慘遭殺害。

嘉靖朝吏治敗壞,爆发多起农民起义,如:山東礦工起義、陳卿起義、蔡伯貫起義、浙贛礦工起義、李亞元起義、賴清規起義,邊事廢弛,1524年以後爆發多起大同兵變,1535年爆發遼東兵變,1560年爆發振武營兵變,長城北方蒙古鞑靼俺答汗寇邊,倭寇侵略中国東南沿海,就是“北虜南倭”的問題,後賴朱紈、戚繼光、俞大猷等人率軍肅清倭寇。世宗在位之時,葡萄牙人遠航當時屬广东省香山县管辖的澳門,並“借地晾晒水浸货物”为借口開始於澳門定居,從而在澳門展開了接近450年的葡萄牙佔領及殖民時期。

嘉靖四十四年(1565年)正月,方士王金等伪造《诸品仙方》、《养老新书》,制长生妙药献世宗。嘉靖四十五年二月,(1566年)戶部主事海瑞上《治安疏》,世宗初大怒,擲疏於地,並下詔讓錦衣衛及三法司論罪。但后重置御案上數日內再三閱讀。后法司擬處大辟的刑罰,但世宗審閱後卻留中不發,以致海瑞終未獲刑。

嘉靖四十五年十二月初八,世宗免去臘宴。十四日,世宗病篤,時隔二十多年重新住回大內。當日午時,於乾清宮駕崩,享壽六十岁。徐階請裕王入宮主持大行皇帝喪禮。裕王自東安門入,至乾清宮御榻前發喪。次日,大行皇帝小殮,并發佈遺詔。十六日,大殮,并上廟號世宗。

隆慶元年三月十一日,世宗梓宮及祔葬孝洁皇后、孝恪皇后梓宫離開北京。十六日,世宗及孝潔皇后、孝恪皇后梓宮抵達永陵。次日,世宗入葬永陵。

嘉靖皇帝醉心于西苑修仙斋醮,直到他最后死去,却一直是“虽深居渊穆而威柄不移”,虽数十年不见朝臣,仍能做到“大张弛、大封拜、大诛赏,皆出独断,至不可测度。”明世宗非常聪明,也十分勤奋,批阅奏书票拟经常到后半夜。但嘉靖后期,朝中官员贪污纳贿、奢侈靡费,確已成普遍的现象。

《明史·世宗本紀》:“贊曰:世宗御極之初,力除一切弊政,天下翕然稱治。顧迭議大禮,輿論沸騰,幸臣假托,尋興大獄。夫天性至情,君親大義,追尊立廟,禮亦宜之;然升祔太廟,而躋於武宗之上,不已過乎!若其時紛紜多故,將疲於邊,賊訌於內,而崇尚道教,享祀弗經,營建繁興,府藏告匱,百餘年富庶治平之業,因以漸替。雖剪剔權奸,威柄在御,要亦中材之主也矣。”

《国榷》:“世庙起正德之衰,厘革积习,诚雄主也。因议礼自裁,好稽古右文之事,诸臣迎附,祗诤于仪节,反实政略焉。”

《名山藏》:“臣喬遠曰:臣每見故縉紳父老,若為郎時尚接先朝皆御之臣,多好言嘉靖時事,其謨猷合聖賢,動作掀天地,真中興之主矣。晚節西苑崇玄,帝心固以為敬天,雖萬幾在宥而精神無時不運,於天下者四十餘年如一日,所以饗世獨久歟。”

\subsection{嘉靖}

\begin{longtable}{|>{\centering\scriptsize}m{2em}|>{\centering\scriptsize}m{1.3em}|>{\centering}m{8.8em}|}
  % \caption{秦王政}\
  \toprule
  \SimHei \normalsize 年数 & \SimHei \scriptsize 公元 & \SimHei 大事件 \tabularnewline
  % \midrule
  \endfirsthead
  \toprule
  \SimHei \normalsize 年数 & \SimHei \scriptsize 公元 & \SimHei 大事件 \tabularnewline
  \midrule
  \endhead
  \midrule
  元年 & 1522 & \tabularnewline\hline
  二年 & 1523 & \tabularnewline\hline
  三年 & 1524 & \tabularnewline\hline
  四年 & 1525 & \tabularnewline\hline
  五年 & 1526 & \tabularnewline\hline
  六年 & 1527 & \tabularnewline\hline
  七年 & 1528 & \tabularnewline\hline
  八年 & 1529 & \tabularnewline\hline
  九年 & 1530 & \tabularnewline\hline
  十年 & 1531 & \tabularnewline\hline
  十一年 & 1532 & \tabularnewline\hline
  十二年 & 1533 & \tabularnewline\hline
  十三年 & 1534 & \tabularnewline\hline
  十四年 & 1535 & \tabularnewline\hline
  十五年 & 1536 & \tabularnewline\hline
  十六年 & 1537 & \tabularnewline\hline
  十七年 & 1538 & \tabularnewline\hline
  十八年 & 1539 & \tabularnewline\hline
  十九年 & 1540 & \tabularnewline\hline
  二十年 & 1541 & \tabularnewline\hline
  二一年 & 1542 & \tabularnewline\hline
  二二年 & 1543 & \tabularnewline\hline
  二三年 & 1544 & \tabularnewline\hline
  二四年 & 1545 & \tabularnewline\hline
  二五年 & 1546 & \tabularnewline\hline
  二六年 & 1547 & \tabularnewline\hline
  二七年 & 1548 & \tabularnewline\hline
  二八年 & 1549 & \tabularnewline\hline
  二九年 & 1550 & \tabularnewline\hline
  三十年 & 1551 & \tabularnewline\hline
  三一年 & 1552 & \tabularnewline\hline
  三二年 & 1553 & \tabularnewline\hline
  三三年 & 1554 & \tabularnewline\hline
  三四年 & 1555 & \tabularnewline\hline
  三五年 & 1556 & \tabularnewline\hline
  三六年 & 1557 & \tabularnewline\hline
  三七年 & 1558 & \tabularnewline\hline
  三八年 & 1559 & \tabularnewline\hline
  三九年 & 1560 & \tabularnewline\hline
  四十年 & 1561 & \tabularnewline\hline
  四一年 & 1562 & \tabularnewline\hline
  四二年 & 1563 & \tabularnewline\hline
  四三年 & 1564 & \tabularnewline\hline
  四四年 & 1565 & \tabularnewline\hline
  四五年 & 1566 & \tabularnewline
  \bottomrule
\end{longtable}


%%% Local Variables:
%%% mode: latex
%%% TeX-engine: xetex
%%% TeX-master: "../Main"
%%% End:

%% -*- coding: utf-8 -*-
%% Time-stamp: <Chen Wang: 2019-12-26 15:07:33>

\section{穆宗\tiny(1567-1572)}

\subsection{生平}

明穆宗朱載坖(“坖”音“jì”,1537年3月4日-1572年7月5日),或稱隆慶帝,明朝第13位皇帝,庙号“穆宗”,谥号“契天隆道渊懿宽仁显文光武纯德弘孝莊皇帝”。

朱载坖的名讳在万历年间被武纬子误记为朱載塈(“塈”音“jì/ㄐㄧˋ”),崇祯年间被朱国祯等误记为朱載垕(“垕”音“hòu/ㄏㄡˋ”),导致清代文献、越南文献、朝鲜文献对穆宗名讳记载的混乱。

明穆宗是明世宗第三子,嘉靖十六年(1537)生,母亲是康妃杜氏。嘉靖十八年(1539)二月,明世宗册立次子朱载壡为太子、三子朱载坖为裕王、四子朱载圳为景王。嘉靖二十八年(1549)三月,太子朱载壡薨,裕王朱载坖以次序当为太子。由于明世宗次子朱载壡早逝,所以迟迟未予册立。时景王朱载圳年少,服色与裕王朱载坖无别,引起朝野议论。嘉靖四十(1561)年二月,明世宗命景王朱载圳出居封国,以杜绝其觊觎之心和朝野议论。嘉靖四十四年(1565)正月,景王朱载圳薨,明世宗对内阁首辅建极殿大学士徐阶说:“此子素谋夺嫡,今死矣。”

嘉靖四十五年十二月(公元1567年1月),明世宗驾崩,裕王朱载坖即位,改元隆庆,是为明穆宗。明穆宗立即纠正其父的弊政,之前以言获罪的诸臣全部召用,已死之臣抚恤并录用其后,方士交付有司论罪,以前的道教仪式全部停止,免除次年一半田赋及嘉靖四十三年以前的所有欠赋;又停止明世宗为博孝名强行施行的明睿宗(即明世宗本生父兴献王)明堂配享之礼(即秋季祭天,要以在位皇帝之父合祭,为此导致明太宗庙号被改为明成祖)。

隆庆帝重用徐阶、李春芳、高拱等内阁辅臣,致力于解决困扰朝局多年的“北虏南倭”问题,隆庆元年(1567年),采纳内阁大学士高拱、张居正的建议,与蒙古俺答议和,結束與蒙古長達二百年的戰爭,並有俺答封贡。同年宣布废除海禁,允许民间私人远贩东西二洋,史称隆庆开关。隆庆新政是明穆宗统治时期所出现的承平时期。

明穆宗力行节俭,信用内阁辅臣,并不加以掣肘,但也不能制止内阁辅臣之间的倾轧,这也与其本人仁厚而平庸的性格有关,即位后,首先宣告天下,将废除明世宗时期的所有弊政,一时间朝廷内外都希望新君能有所作为。但是,革弊施新取得实效没多久,他開始宠信太监膝祥等人,挥霍无度,纵情声色,荒废朝政。即位后不久,很快就将权力交给了以高拱为首的内阁,以后只召见过两次阁臣,而他自己就在后宫享乐,广修宫苑,犬马歌舞。

坊间传闻明穆宗特别好色,整天在后宫里忙来忙去,被人比做后宫中辛勤的蜜蜂。他長期服用春药,每天要数名美女陪伴。宫中的用品,小到茶杯,大到龙床,全部都有男欢女爱的雕刻和彩绘。对此,很多大臣都曾上书进谏,竭力劝阻,但他总是很温和地说:“国事有先生我就放心了,家事就不劳先生费心了”。

由于明穆宗贪于女色,纵情声色,加上长期服食春药,他的身体每况日下,难以支撑,萬曆野獲編称其“阳物昼夜不仆,遂不能视朝”。

隆庆六年(1572年)闰三月,宫中传出了明穆宗病危的消息。在休养了两个月之后,他又上朝视事,却又突然头晕目眩,支持不住而回宫。他自知病情不轻,急召高拱、张居正及高仪三人接受顾命,吩咐由太子继位,后崩于乾清宮,终年三十六岁,后被谥为庄皇帝,庙号穆宗,葬于北京昌平明昭陵。

《明穆宗实录》:“上即位,承之以宽厚,躬修玄默,不降阶序而运天下,务在属任大臣,引大体,不烦苛,无为自化,好静自正,故六年之间,海内翕然,称太平天子云。”

《明史》:“穆宗在位六载,端拱寡营,躬行俭约,尚食岁省巨万。许俺答封贡,减赋息民,边陲宁谧。继体守文,可称令主矣。第柄臣相轧,门户渐开,而帝未能振肃乾纲,矫除积习,盖亦宽恕有余,而刚明不足者欤!”

《国榷》:“迹帝之终始,宽大如仁庙,而精勤不若也。安豫如宪朝,而控纵不若也”

《名山藏》:“上端凝靜密,不殺自威,不察自智,優崇輔弼,假借臣僚用能守祖宗之法以致中國乂寧,外夷向風之盛,蓋清靜合軌漢帝寬仁,比跡宋宗矣。上在潛邸時,食驢腸而甘,及即位間問左右,左右請詔光祿,上不忍曰「若爾,則光祿日宰一驢矣。」歲時游吳行幸,諸供膳光祿先期請上旨為豐約,上常裁取最約者焉。”

\subsection{隆庆}

\begin{longtable}{|>{\centering\scriptsize}m{2em}|>{\centering\scriptsize}m{1.3em}|>{\centering}m{8.8em}|}
  % \caption{秦王政}\
  \toprule
  \SimHei \normalsize 年数 & \SimHei \scriptsize 公元 & \SimHei 大事件 \tabularnewline
  % \midrule
  \endfirsthead
  \toprule
  \SimHei \normalsize 年数 & \SimHei \scriptsize 公元 & \SimHei 大事件 \tabularnewline
  \midrule
  \endhead
  \midrule
  元年 & 1567 & \tabularnewline\hline
  二年 & 1568 & \tabularnewline\hline
  三年 & 1569 & \tabularnewline\hline
  四年 & 1570 & \tabularnewline\hline
  五年 & 1571 & \tabularnewline\hline
  六年 & 1572 & \tabularnewline
  \bottomrule
\end{longtable}


%%% Local Variables:
%%% mode: latex
%%% TeX-engine: xetex
%%% TeX-master: "../Main"
%%% End:

%% -*- coding: utf-8 -*-
%% Time-stamp: <Chen Wang: 2021-11-01 17:13:44>

\section{神宗朱翊鈞\tiny(1572-1620)}

\subsection{生平}

明神宗朱翊鈞(1563年9月4日-1620年8月18日),或稱萬曆帝,為明朝第14代皇帝,年号万历,是明穆宗朱载坖的第三子。隆慶六年(1572年),穆宗駕崩,九岁的朱翊鈞登基,是为明神宗。在位48年,是明代在位時間最長的皇帝,谥号為「範天合道哲肃敦简光文章武安仁止孝显皇帝」。

明神宗在位前十五年,明朝一度呈現中興景象,史稱萬曆中興,而在位中期亦主持万历三大征,保護藩屬,巩固疆土。在張居正死後始親政,因國本之爭等問題而倦於朝政,自此不上朝,國家機器運轉幾乎停擺,徵礦稅亦被評一大病。萬曆年間也走向活潑和開放,利瑪竇覲見萬曆帝,開始西學東漸,但同時朝廷內東林黨爭開始萌芽、塞外又有後金勢力虎視眈眈,在其晚年佔領明朝東北大部分地區,使明朝退守山海關,終走向滅亡的局面。

明神宗是明穆宗的第三子。出生时,父亲尚为裕王,母親李氏为王府宮女出身。父亲裕王的第一任王妃李氏所生二子──朱翊鈴、朱翊釴均早夭。他实际上成为裕王的长子。另,嫡母继妃陈氏无子。

在其父继位后的隆慶二年(1568年),他被立為皇太子,明穆宗對其很有期望,改名钧,意思是「夫钧者,言圣王制驭天下犹制器者之转钧也」。幼時朱翊钧就十分聰惠,明穆宗在宮中騎馬時,年幼的朱翊钧就大叫道「父皇為天下之主,獨騎疾騁,萬一馬驚,卻如何是好?」穆宗聽後恩喜萬分,就更喜愛朱翊钧了,馬上下馬過來摟朱翊鈞在懷裡褒賞一番。其母李贵妃教子非常嚴格,隔三差五就把兒子叫到面前諄諄教誡一番,每次經筵結束以後,都少不得督促考問他今天所學的內容。朱翊鈞小時候稍有懈怠,李贵妃就將其召至面前長跪。

隆慶六年,父亲明穆宗駕崩,朱翊鈞即位,改元萬曆,堅持按照祖宗舊制,舉日講,御經筵,讀經傳、史書。而他每天读书亦十分用功,朝章典故都读很多遍,即使是隆冬盛暑亦从不间断,以後隨朱翊钧年渐长而学愈进,他自己后来也常常十分得意地说:“朕五岁即能读书。”另外他的書法也十分出色,筆劃遒勁,經常親自賜墨寶給大臣,連張居正仔細端詳作品,也不得不承認皇帝的書法是「揮瀚灑墨,初若並不經意,而鋒穎所落卻是奇秀天成」,但張居正終究認為他應該成為一位聖君而非書法家,便劈頭蓋臉奏訓一頓,自此直到張居正死後朱翊鈞才重新接觸書法。

神宗在位之初十年尚處年幼,由母親李太后代為聽政。即位之初內閣紛爭傾軋,閣臣之間關係惡劣,時高拱以主幼國危,痛哭時偶然說了一句:「十歲太子如何治天下」,引起朱翊鈞極為不滿,最後在張居正與馮保添油加醋下罷免了高拱。太后將一切內務大事交由馮保,而大柄悉以委居正,軍政皆由張居正主持裁決,独握大权。

在小皇帝朱翊钧以及李太后全力的支持下,張居正大刀阔斧地實行了一条鞭法等一系列改革措施,清丈田畝,改革赋税,整飭軍備,考察官吏,使社会经济有很大的发展,人民生活也有所提高,一改前弊。萬曆初年太倉的積粟達1300萬石,可支用十年,僅僅是太僕寺的銀兩儲蓄便多達四百餘萬,而太倉庫更是有超過千萬兩的積蓄,國家繁荣昌盛,扭轉明中業以來的頹勢,是為「萬曆中興」。後人在論及此段發展情況時,多歸功於張居正的鞠躬盡瘁,而對朱翊鈞的傾心委任卻往往忽視,實際上,隨朱翊鈞年紀長大,他也不再是名義上的擺設,張居正可以勸導、利用他幹什麼,卻不能強迫他做出違心之事,因此張居正也有無可奈何之時。

神宗幼年,太后及張居正都希望其成為儒家所倡導的皇帝典範。萬曆八年,神宗因和太監孫海、容用出遊行為輕浮不檢,太監馮保告知李太后。太后大怒,數落道「天下大器豈獨爾可承耶」,並拿出以霍光罷黜昌邑王之事威脅神宗,帝師張居正又乘機捉刀,寫下罪己詔,言詞犀利,以警惕皇帝。雖然保住皇位,但也因此使神宗認為顏面盡失。一次神宗在讀《論語》時,誤將「色勃如也」之「勃」字讀作「背」音,張居正厲聲糾正:「當作勃字!」聲音太大,嚇得神宗驚惶失措,在朝的大臣無不大驚。沈德符在《萬曆野獲編》中說:「(張居正輔政)宮府一體,百辟從風,相權之重,本朝罕儷,部臣拱手受成,比於威君嚴父,又有加焉。」「江陵(張居正)以天下為己任,客有諛其相業者,輒曰我非相,乃攝也。」晚年張居正的權勢之大,威權赫奕,連神宗都有所忌憚,曾經有丘岳由亞卿左遷藩參,曾以黃金製對聯饋張居正「日月並明,萬國仰大明天子;丘山為岳,四方頌太岳相公。」張居正奉旨歸喪時,地方大員行長跪禮,撫按大吏越界迎送,空前絕後。而奪情以後,張居正也日益偏恣,好同惡異,左右用事之人多通賄賂,時人益惡之,神宗亦意識到張居正的權力過大,“幾乎震主”,為後期清算張居正埋下伏筆。張居正死後,二十歲的神宗始親政。

古籍文獻記載,神宗親政後勵精圖治,虛心納諫,屢蠲賦稅,生活節儉,如僅在萬曆十一年間,蠲免並災傷織造議留就已達銀一百七十六萬一千兩。北京乾旱,神宗關心民瘓,親自以旱詔中外理冤抑,釋鳳陽輕犯及禁錮年久的犯人。另親自步行至天壇祈雨,皇上齋戒,親躬步行將近二十里的路程而不乘車輦出,且絲毫沒有因驕陽酷日而為難的樣子,其舉止從容不迫,表現的肅穆得體,百姓能一睹天顏,紛紛舉首加額高呼「聖德爾」,另外又敕六部都察院等曰:「天旱雖由朕不德,亦天下有司貪婪,剝害小民,以致上乾天和,今後宜慎選有司。」蠲天下被災田租一年。

朝鮮使者於《朝天記》、《朝天日記》中記載神宗年輕時儀容莊嚴穩重,額頭廣闊、下巴飽滿,步伐矯健、神采威嚴,目光炯炯有神、舉手投足之間使人敬畏,而帝王氣度更是深不可測,是中外一至認為都有道明君。他在位的前十五年被評價「勤於朝政,勵精圖治,大有作為,足以稱道,儼然如一代賢君」。

万历帝的老师、第一任内阁首辅兼万历新政的策划与执行人张居正過世後第二年,万历帝斥逐馮保,下詔追奪張居正的封號和諡號,並查抄張家,平反劉臺冤案,起用因反對張居正而遭懲處的官員。万历十七年起(1588年),万历帝開始怠慢朝政(一說沉湎於酒色之中,一說是染上鴉片煙癮),万历十七年十二月大理寺左评事雒于仁写《酒色财气四箴疏》:“皇上之恙,病在酒色财气也。夫纵酒则溃胃,好色则耗精,贪财则乱神,尚气则损肝”。邹漪《启祯野乘》卷一《冯恭定传》中也说到明神宗荒于酒色:“因曲蘖而驩饮长夜,娱窈窕而晏眠终日。”《明史鈔略》記載萬曆二十一年皇太后萬壽時,神宗在暖閣召見王錫爵:……上曰:“朕知道了。”錫爵又奏:“今日見了皇上,不知再見何時?”上曰:“朕也要先生每常相見,不料朕體不時動火。”爵對:“動火原是小疾,望皇上清心寡欲,保養聖躬,以遂群臣願見之望。”而明神宗也開始奢侈靡费,斂財揮霍,又屢屢從國庫提銀,史稱「傳索帑金」,并任用張鯨等奸倖。後因立太子的國本之爭与内阁爭執長達十餘年,最後索性三十年不出宫门、不郊、不廟、不朝。1589年,神宗不再接見朝臣,內閣出现了“人滞于官”和“曹署多空”的现象。

万历二十五年,右副都御史謝杰批评神宗荒于政事,亲政后政不如初:「陛下孝亲、尊祖、好学、勤政、敬天、爱民、节用、听言、亲亲、贤贤,皆不克如初矣。」萬曆三十四年,禮科左給事中孫善繼也極陳時弊說:「惟願皇上修萬曆十五年以前之勵精,複萬曆十五年以前之政體,收萬曆十五年之人心,庶平明之治成,垂拱之理得。」以至於朱翊鈞在位中期以後,方入內閣的廷臣不知皇帝长相如何,于慎行、赵志皋、张位和沈一贯等四位国家重臣虽对政事忧心如焚,卻無計可施,僅能以数太阳影子长短来打发值班的时间。

萬曆四十年(1612年),南京各道御史上疏:「臺省空虛,諸務廢墮,上深居二十餘年,未嘗一接見大臣,天下將有陸沉之憂。」首輔葉向高卻說皇帝一日可接見福王兩次,但明神宗不承認,并表示他已經沒有傳召福王很久了,若真的每日接見,福王出入禁門,隨從這麼多,人所共見,必然耳目難掩。万历四十五年(1617年)十一月,「部、寺大僚十缺六、七,风宪重地空署数年,六科止存四人,十三道止存五人。」而明緬戰爭也因為明朝方面忽視而先勝後敗,被緬甸東吁王朝蠶食孟養在內國土。

囚犯們關在監獄裡,有長達二十年之久還沒有審問過一句話的,他們在獄中用磚頭砸自己,輾轉在血泊中呼冤。臨江知府錢若賡被神宗投入詔獄達三十七年之久,直至其子錢敬忠上疏:「臣父三十七年之中……氣血盡衰……膿血淋漓,四肢臃腫,瘡毒滿身,更患腳瘤,步立俱廢。耳既無聞,目既無見,手不能運,足不能行,喉中尚稍有氣,謂之未死,實與死一間耳。」萬曆帝才以「汝不負父,將來必不負朕。」將其釋放。首輔李廷機有病,連續上了一百二十次辭呈都得不到消息,最後不辭而去。萬曆四十年(1612年),吏部尚書孫丕揚上二十餘疏請辭不得,最後也拜疏自去。四十一年(1613年),吏部尚書趙煥也因數請去職還鄉不得,於是稱疾不出,逾月才終於請辭成功。吴亮嗣于万历末年的奏疏中说:「皇上每晚必饮,每饮必醉,每醉必怒。酒醉之后,左右近侍一言稍违,即毙杖下。」

樊樹志的《萬曆傳》考究裏,中允地解釋了明神宗怠政原因,源於健康狀況惡化非子虛烏有,追溯萬曆十四年九月十八日以後,皇帝因病免朝,言「頭昏眼黑,力乏不興」,對祭享太廟活動也只能權讓勛貴代理,并無奈地說道「非朕敢偷逸,恐弗成禮」,後來又遣內使對內閣傳諭「聖體連日動火,時作眩暈」,「聖體偶因動火,服涼藥過多,下注於足,搔破貼藥,朝講暫免。」與定陵发掘後查證神宗左足有疾互相引證。且當萬曆十五年三月初六,聖體初安以後,神宗旋即上朝聽政,隨后又與三輔臣見面,并打招呼說「朕偶有微疾,不得出朝,先生每憂心。」十六年二月初一又如常參與文華殿經筵,并興致勃勃地與閣臣討論《貞觀政要》,唐太宗與魏徵。萬曆十八年正月初一時,收到雒于仁奏疏的神宗召見首輔申時行入見,當申時行向他提出皇上有病需要静攝,也當一月之間至少數次視朝,神宗并沒有惱怒,只是解釋道「朕病愈,豈不欲出!即如祖宗廟祀大典,也要親行,聖母生身大恩,也要時常定省。只是腰痛腳軟,行立不便。」次年病情稍好後神宗與閣臣談起病情,也是真情流露地說起自己久病的心情「朕近年以來,因痰火之疾,不時舉發,朝政久缺,心神煩亂」。乃至神宗在位中期王家屏,王鍚爵輔政期間仍是「面目發腫,行步艱難」,以致連嫡母仁聖皇太后陳氏病逝,一向孝順聞名的神宗也因病動彈不得,只能遣人代理,而遭受到朝臣猛烈的評擊責難,有苦難言,此後神宗病情反復,在萬曆三十年病情之差甚至要一度立下遺旨,向沈一貫託孤。可見神宗在位期間的「動履不便」「身體虛弱」以致在位期間怠政,實不是推諉託辭。

萬曆中期後雖然不上朝,但是並沒有出現英宗以來的宦官之亂,也沒有外戚干政,也沒有嚴嵩這樣的奸臣,朝內黨爭也有所控制,萬曆對於日軍攻打朝鮮、女真入侵和梃擊案都有迅速的反應,如萬曆二十四年,乾清坤寧兩宮大火,神宗下罪己詔書,表示雖然忽略一般朝政庶務,但還是關心國家大事,而處理政事的主要方法多是在九重宫阙下通過諭旨的形式向下面傳遞,並透過一定的方式控制朝局。

此外礦稅之弊,即神宗在位期間的賦稅措施,一般被是認為萬曆中年後弊政的一部分,萬曆擺脫張居正的束縛之後,開始通過向各地徵收礦稅銀的方式,增加內庫的內帑,大多數學者認為這是一項弊政,也有許多的反對意見,認為礦稅也有相當的好處,如礦稅入內帑後大多用于国家救灾,餉軍救急等。

神宗在軍事上任用幹練將校,先後主持發兵平定了播州(遵义)杨应龙之亂的播州之役、平宁夏哱拜之乱的寧夏之役、抵抗日本丰臣秀吉發兵侵略朝鮮以及奴兒干都司的朝鮮之役,维护了明朝的內部统一及宗主國的權威。此三場戰爭合稱萬曆三大征。后世有说明軍雖均獲勝,但軍費消耗甚鉅,如僅朝鮮一役消耗國庫便高達银八百八十三万五千两,米数十万斛,對晚明的財政造成重大負擔。但实际上明代晚期仅对后金的战事,耗费就高达六千万两之巨,远超三大征,且三大征都是不得不打之戰,如朝鮮一國勢拱神京,地牽關海,薊、遼之外藩,東江之咽噎,一或失守,重險撤焉,如若不打甚至打败了,明朝都有亡国之危。而三大征实际军费则由内帑和太仓库银足额拨发,三大征结束后,内帑和太仓库仍有存银,而面對萨尔浒之战的大败,朱翊钧用熊廷弼守辽东,屯兵筑城,才稍稍将东北局势扭转。

萬曆皇帝指揮的萬曆朝鮮之役使朝鮮保全了國家,避免了亡國的巨大危險,儘管朝鮮人對萬曆皇帝有著深厚的感情,但是在朝鮮使臣的記錄中,更多的還是對萬曆帝消極怠政、貪婪奢侈等惡劣行徑的批評。而朝鮮使臣塑造的萬曆皇帝形象,也反映出明中葉之後朝鮮對中國社會集體想像的轉變,大明國的形像已經由朝鮮前期塑造的天朝上國,逐步褪去了耀目的光環,而走向了沒落。但在明清鼎革後,朝鮮對明朝的推崇思念又走向一個新的巔峰,朝鮮君王設大報壇,萬東廟祭祀明太祖,明神宗和明思宗。朝鮮孝宗甚至一度打算北伐清廷,朝鮮士子儒生暗中使用崇禎年號幾近三百年,鄙視清朝,并以宋時烈等為首推崇「尊周思明」「春秋大義」,稱自己是「皇明遺民」,那怕隱居山中,一生不出仕為大明守節者也大有人在,甚至到近代朝鮮高宗稱帝時,大明滅亡已超過二百餘年,其即位時諸臣勸進仍是「神宗皇帝再造土宇, 則義雖君臣, 恩實父子...嗚乎! 天命靡常, 皇社旣屋, 帝統墜地, 獨大報一壇, 乃皇春一胍之所寄...陛下聖德大業,宜承大明之統緒」,一切礼节皆取自《大明会典》。

神宗在位期间,西方传教士纷纷来华,其中以利玛窦为代表。利玛窦还在万历二十八年(1601年)觐见了神宗,向神宗进呈《万国图志》、自鸣钟、大西洋琴等西方方物,获得了神宗的信任。

利玛窦还与进士出身的翰林徐光启交情最好。除利玛窦来华外,来中国的传教士还有意大利的熊三拔、艾儒略,日耳曼人汤若望等人。

西方传教士来到中国,把西方数学、天文、地理等科学技术知识还有西方文化传到中国,在一定程度上促进了当时中国社会经济文化的发展,而中国士大夫阶层中的少数先进分子,同时起了一种唤醒的作用。

萬曆九年,神宗在向太后請安時,一時衝動,臨幸一名宮女,生下了長子朱常洛(後來的明光宗泰昌皇帝)。因為朱常洛是宮女所生,神宗不喜歡他,且有意立愛妃鄭氏所生的朱常洵為太子。萬曆十四年群臣上奏請神宗即立常洛為太子,萬曆以常洛尚年幼體虛未定,拖延不決。

萬曆二十一年,明神宗變本加厲,下手詔要將皇長子朱常洛、三子朱常洵和皇五子朱常浩一同封為藩王,以後再選擇其中適合人選為太子。朝臣聽聞一片譁然,紛紛上奏神宗。如雪片般飛來的痛批奏摺,使神宗倍感壓力,迫於眾議只好不得已收回前命。直到萬曆二十九年,朱常洛已年滿二十歲,立儲一事已不可拖延,神宗才立其為皇太子。

而長久以來的國本之爭引發出了兩次妖書案,這些案件即是朝廷大臣內鬨的縮影,可說是東林黨爭。

此时东北女真族興起,成为日後中原帝國的隐患。万历四十六年(1618年)四月十三日,女真酋長努爾哈赤自稱“覆育列國英明汗”,凑“七大恨”,以掀起叛乱,并僭称国号为後金。四十六年四月,女真兵克抚顺,殺死遼東總兵官張承胤,朝野震惊。為了應付女真,把努爾哈赤「务期歼灭,以奠封疆」,自萬曆四十六年九月起,朝廷先後三次下令除了畿內八府及貴州以外,加派全國田賦九厘,合共增賦五百二十萬,時稱遼餉,明末三饷之始,而神宗有鑑於地方官員在遼餉外可能會額外徵收火耗剝削百姓,特別下旨嚴禁。万历四十七年(1619年),遼東經略楊鎬領尚方劍,調兵遣將,并以李如柏、杜松、劉綎、馬林四將分兵進攻後金,結果在薩爾滸之戰大敗,死四萬餘人,開原和鐵嶺淪陷,首都燕京震動。

戰爭中,明神宗多有佈置方略,但一直吝惜內庫帑銀,不願撥內帑充餉,直至朝臣再三請求而後才勉強發了帑銀十萬,但其中多黑如漆或脆如土,致使師老餉匱。待四路殞將覆師後,神宗才又警愦振聋,發了近四十萬兩內帑銀解赴辽東,并任用熊廷弼守辽东,並給予其大力支持,屯兵筑城,振飭軍備,才稍稍将東北局勢扭轉。虽然明神宗多年未正式上朝,但大到朝鲜之役,小到顺天府祈雨,均由皇帝在内宫作出,并发各部门直接执行。

薩爾滸之戰後,遼東失陷,神宗鬱鬱寡歡,焦勞國事。隔年萬曆四十八年(1620年)四月,皇后王喜姐病逝,神宗心力交瘁,過了三個月,万历四十八年七月二十一日(1620年8月18日),明神宗駕崩於紫禁城弘德殿,享年五十七歲,在位四十八年。臨終前遺詔指出大臣應勉以用心辦事,以及廢礦稅,起用建言而得罪的官員等。

朝鮮一國為此舉哀。太子朱常洛立即发内帑(皇帝私房钱)百万犒赏边关将士。停止所有矿税,召回以言得罪的诸臣。不久,再发内帑百万犒边。八月即位,改元泰昌,是为明光宗,光宗即位後,內閣先是為萬曆帝擬謚上廟號顯宗恭皇帝,但後來朝臣認為諡號的「恭」是晉恭帝,隋恭帝兩位末代皇帝的諡號,先帝聖謨不可殫述,而帝堯運乃神之德,於是後改成為其上廟號神宗,諡號顯皇帝。九月,在位不足三十天的明光宗便在红丸案之中暴毙。因光宗即位不到一個月即告駕崩,孫子熹宗即位後於十月丙午(10月27日)葬神宗於定陵。

万历帝的定陵1958年被发掘,万历帝尸骨復原,“生前体形上部为驼背”、左腳略右腳短。文革時期的1966年8月24日,遗骨被紅衛兵付之一炬。因此,萬曆皇帝之所以三十年不上朝的原因,有一說是認為自己身形不正,感到自卑,所以不敢見人。

1955年10月4日,郭沫若、沈雁冰、吴晗、邓拓、范文澜、张苏等人联名提交《关于发掘明长陵的请示报告》给国务院秘书长习仲勋。报告转给主管文化工作的国务院副总理陈毅,并呈报国务院总理周恩来。文化部文物局局长郑振铎、中国科学院考古研究所副所长夏鼐得知后认为条件不成熟,强烈反对贸然发掘,高层形成一场争论。周恩来向毛泽东作了汇报,毛泽东点头后,周恩来批下“原则同意”四字。长陵发掘委员会委员夏鼐负责发掘的技术指导,便让其学生赵其昌(后任首都博物馆馆长)做前期调研。赵其昌带探工在长陵未找到发掘线索。在向夏鼐、吴晗等人汇报后,经商讨决定先试掘献陵,积累经验再发掘长陵。后来吴晗和夏鼐认为试掘献陵对长陵的发掘参考价值不大,吴晗提议试掘永陵,遭夏鼐强烈反对,认为这与发掘长陵无异;试掘思陵,吴晗认为太小,是妃子墓改建。此后吴晗和夏鼐才想到定陵。杨仕、岳南合著的《定陵地下玄宫洞开记》认为,吴晗和夏鼐想到定陵的原因有二,“第一,定陵是十三陵中营建年代较晚的一个,地面建筑保存得比较完整,将来修复起来也容易些。第二,万历是明朝统治时间最长的一个,做了48年皇帝,可能史料会多一些。” 定陵的開挖始末,《風雪定陵》一書有詳細的介紹。

1956年5月开始试掘,历时一年试掘成功,1957年打开玄宫。其玄宫由前室、中室、后室、左配室、右配室组成,石条起券,前室前面有隧道券,总面积1195平方米,出土文物3000多件。1959年9月30日,就定陵原址建为“定陵博物馆”,郭沫若题写馆名。1959年10月1日正式对外开放。由于技术水平落后,出土的大批文物无法保存,发掘出土的丝织品变硬腐化。郑振铎、夏鼐为此上书国务院,请求立即停止再批准发掘帝王陵墓的申请,国务院总理周恩来同意了他们的意见。不主动发掘帝王陵墓自此成为中国考古界的定规。

1966年文化大革命爆发后,定陵遭到嚴重破壞,保存在定陵文物仓库中的萬曆帝、后的屍骨被紅衛兵以「打倒地主階級的頭子萬曆」的口號被揪出。1966年8月24日,萬曆帝、后的三具尸骨以及一箱帝、后画像、资料照片等被抬到定陵博物馆重门前的广场上接受批斗并焚毁。

明朝官修的編年體史書《明神宗顯皇帝實錄》總評萬曆皇帝一生說:“蓋上仁孝聖神,逈絕千古,享國愈久,聖德彌隆,無挽近綜核之煩,而自臻治古幾康之理。海內沐浴玄化幾五十年,國祚靈長,永永無極,所培毓遠矣。先是因秉軸者懲操切之過,不無稍劑以寬大,而上明習政事,乾綱獨攬,予奪進退,莫可測識。晚頗厭言官章奏,概置不報,然每遇大事,未嘗不折衷群議,歸之聖裁。中外振聳,四封宴如,雖以憂勤之主極意治平而不得者,上獨以深居靜攝得之,周之成康,漢之文景,未足況也。至慈護先考,終始無間,尤非草野所得窺,而為堯為舜之旨,更諄諄以期。 ……廟號曰神,殆真如神雲。”

黃汝良:“仓箱红朽无忧岁,南北敉宁不用兵。北塞称臣四十年,封疆无数获生全。”

姚希孟(1579—1636):“缅怀祖德岂难跻,八柄河魁手自持。凤诏未闻传墨敕,貂珰只许贡朱提。兵符细柳将军令,国计元和宰相稽。蝉鬓秀才垂紫袖,批红不改旧标题。”

丁耀亢(1600—1669):“憶昔村民千百家,門前榆柳蔭桑麻。鳴雞犬吠滿深巷,男舂婦汲聲歡嘩。神宗在位多豐歲,鬥粟文錢物不貴。門少催科人晝眠,四十八載人如醉。”

钱谦益(1582—1664):“国家修明昌大之运,自世庙以迄神庙,比及百年,可谓极盛矣。”“万历中,正国家日中豫泰之候。”“当盛明日中,君臣大有为之日。”“呜呼,我神宗显皇帝,丕承谟烈,久道化成,制科取士,人物滋茂。”

王时敏(1592—1680):“神宗之世,海内乂安,生民不见兵火。”

谈迁(1593—1657):“今吏民嗷嗷,追念宽政,讴吟思慕,虽改代讵一日忘之哉?”

夏允彝(1596—1645):“神庙冲龄践祚,睿质夙成……士大夫以气节相矜,虽无姚、宋之辅,亦无愧开元之盛时也。”“神庙睿圣非常,虽御朝日希,而柄不旁落,止以鄙夷群臣之故,置庶务于不理。士大夫益纵横于下,而国事大坏。”

陈洪绶(1599—1652):“枫溪梅雨山楼醉,竹坞香茶佛屋眠。得福不知今日想,神宗皇帝太平年。”

吴伟业(1609—1671):“余尝惟国家当神宗皇帝时,天下平治。”“以余所闻,神宗皇帝时,士大夫以读书讲学相高。”“余生也晚,犹见神宗皇帝之世,江南土安俗阜,风习最为近古。”

顾炎武(1613—1682):“昔在神宗之世,一人无为,四海少事。”“老人尚記為兒時,煙火萬里連江畿。斗米三十穀如土,春花秋月同遊嬉。定陵(即神宗,神宗葬於定陵)龍馭歸蒼昊,國事人情亦草草事。”

彭孙贻(1615—1673):“眼见万历年,朝野穆清昊。”“风光漫思江南乐,父老还思万历年。”

方孝标(1617—?):“此时神庙正垂衣,四海烽清禾黍肥。”

吴嘉纪(1618—1684):“酒人一见皆垂泪,乃是先朝万历钱。”

林古度(1580年—1666年):“陸離彷彿五銖光,筆畫分明萬曆字。座客傳看盡黯然,還將一縷為君穿。且共開顏傾濁釀,不須滴淚憶當年。”

徐枋(1622—1694):“神宗朝正当国家全盛。”

杜濬(1611年-1687年):“萬曆年間,……九州富庶無旌麾,揚州之域尤稀奇。。”

李邺嗣(1622—1680):“神宗全盛日,海内一愁无。尚及闻遗老,今犹哭鼎湖。”

汪琬(1624—1691):“琬尝追溯神宗之世,国家方承平无事。”“神宗德泽犹在人心。”

曾灿(1625—1688):“神宗乙巳年,中原边辅无烽烟。圣人御极贤者出,粟米流脂贯朽钱。”

陈维嵩(1625—1682):“先朝神宗御宇五十余载,六服休畅,被润泽而大丰美。”

吕留良(1629—1683):「生逢神廟間,貌古性亦淳。海宇忘兵革,冠佩何彬彬。當時不知好,今憶真天神。三十後少年,語之笑且嗔。」

魏世效(1659—?):“万历之四十六年,天下熙暤。当斯时也,物安其性,民安其业,濡染涵育,莫不知立身爱君之道。而敦庞之风,谦下之节,亦惟此时人能有之。”

朝鮮貢使李睟光(1563年—1628年):“巍功赫業五帝六,冠帶車書四海一。商周禮樂漢文物,鼓舞堯天歌舜日。”“聖主天地千年德,嗚呼!聖主天地千年德。”

朝鮮大臣朴淳:: “皇上年方十岁, 圣资英睿, 自四岁已能读书, 以方在谅阴, 未安于逐日视事, 故礼部奏, 惟每旬内三六九日视朝。 仍诣文华馆, 御经筵, 四书及《近思录》、《性理大全》, 皆毕读。 自近日, 始讲《左传》, 百司奏帖, 亲自历览, 取笔批之, 大小臣工, 莫不称庆。”

朝鮮使臣對萬曆皇帝執政前期的勤政是極為稱道的:“因聞皇上講學之勤,三六九日,則無不視朝,其餘日則雖寒暑之極,不輟經筵。四書則方講孟子,綱目至於唐紀,日出坐殿,則講官立講。講迄,各陳時務。又書額字,書敬畏二字以賜閣老,又以責難陳善四字,賜經筵官,以正己率屬四字,賜六部尚書,虛心好問,而 聖學日進於高明。下懷盡達,而庶政無不修,至午乃罷,仍賜宴於講臣,寵禮優渥雲。嗚呼!聖年才至十二,而君德已著如此。若於後日長進不已,則四海萬姓之得受其福者。”

《宣祖實錄》:“今皇帝沖年卽位, 資質英明, 時無過誤, 朝野無事, 人情似有喜悅之意。”

成书于清初的小说《樵史通俗演义》开篇说:“传至万历,不要说别的好处,只说柴米油盐鸡鹅鱼肉诸般食用之类,哪一件不贱?假如数口之家,每日大鱼大肉,所费不过二三钱,这是极算丰富的了。还有那小户人家,肩挑步担的,每日赚得二三十文,就可过得一日了。到晚还要吃些酒,醉醺醺说笑话,唱吴歌,听说书,冬天烘火夏乘凉,百般玩耍。那时节大家小户好不快活,南北两京十三省皆然。皇帝不常常坐朝,大小官员都上本激聒,也不震怒。人都说神宗皇帝,真是个尧舜了。一时贤想如张居正,去位后有申时行、王锡爵,一班儿肯做事又不生事,有权柄又不弄权柄的,坐镇太平。至今父老说到那时节,好不感叹思慕。”

《乱离见闻录》作者陈舜回憶说:“予生萬曆四十六年戊午八月廿六日卯時,父母俱廿三歲,時丁昇平,四方樂利,又家海角,魚米之鄉。鬥米錢未二十,斤魚錢一二,檳榔十顆錢二文,著十束錢一文,斤肉,只鴨錢六七文,鬥鹽錢三文,百般平易。窮者幸托安生,差徭省,賦役輕,石米歲輸千錢。每年兩熟,耕者鼓腹,士好詞章,工賈九流熙熙自適,何樂如之。”

成书于天启四年的小说《警世通言》,第三十二章說:“自永樂爺九傳至於萬曆爺,此乃我朝第十一代的天子了。這位天子,聰明神武,德福兼全,十歲登基,在位四十八年,削平了三處寇亂。那三處?日本關白平秀吉,西夏承恩,播州楊應龍。平秀吉侵犯朝鮮,承恩、楊應龍是土官謀叛,先後削平。遠夷莫不畏服,爭來朝貢。真個是:一人有慶民安樂,四海無虞國太平。”

成书于萬曆四十七年的《萬曆野獲編》,編輯小引說:“今上御極已垂五十年。德符幸生堯舜之世,雖果處菰蘆,然詠歌太平,無非聖朝佳話。間有稍關時事者,其涇渭自明,藿食者但能粗憶梗概而已。”

清世祖(1643-1661):“當明之初,取民有制,休養生息。萬曆年間,海內殷富,家給人足。天啟,崇禎之世,因兵增餉,加派繁興,貪吏綠以為奸,民不堪命,國祚隨之,良足深鑒。”

崔瑞德《剑桥中国明代史》:万历皇帝聪明而敏锐;他自称早慧似乎是有根据的。他博览群书;甚至在他最后的日子里,在他已深居宫廷几十年,并已完全和他的官吏们疏远了时,按照他时代的标准,他仍然博闻广识。

《明史·神宗本紀》:“贊曰:神宗沖齡踐阼,江陵秉政,綜核名實,國勢幾於富強。繼乃因循牽制,晏處深宮,綱紀廢弛,君臣否隔。於是小人好權趨利者馳騖追逐,與名節之士為仇讎,門戶紛然角立。馴至悊、愍,邪黨滋蔓。在廷正類無深識遠慮以折其機牙,而不勝忿激,交相攻訐。以致人主蓄疑,賢奸雜用,潰敗決裂,不可振救。故論者謂明之亡,實亡於神宗,豈不諒歟。”“神皇乘運,豫大豐亨,征徭既繁,百工叢脞,揆厥亂源,所自來爾。”

趙翼《廿二史劄記·萬曆中礦稅之害》:“論者謂明之亡,不亡於崇禎而亡於萬曆。”

谷應泰《明史紀事本末·第六十五卷礦稅之弊》:“神宗奕葉昇平,邊圉封貢,海內乂安,家給人足...逮至萬曆二十四年,張位主謀,仲春建策,而礦稅始起...當斯時也,瓦解土崩,民流政散,其不亡者幸耳”

清高宗在《明長陵神功聖德碑》則道:“明之亡非亡於流寇,而亡於神宗之荒唐,及天啟時閹宦之專橫,大臣志在祿位金錢,百官專務鑽營阿諛。及思宗即位,逆閹雖誅,而天下之勢,已如河決不可復塞,魚爛不可復收矣。而又苛察太甚,人懷自免之心。小民疾苦而無告,故相聚為盜,闖賊乘之,而明社遂屋。嗚呼!有天下者,可不知所戒懼哉?”

宋浚吉: “不怨暗君, 天啓皇帝不可怨之君, 而萬曆皇帝以初年英豪之主, 臨御四十年, 未嘗引接臣僚, 此可爲戒者也。”

黃仁宇在《萬曆十五年》一書將萬曆皇帝的荒怠,聯繫到萬曆皇帝與文官群體在“立儲之爭”觀念上的對抗。怠政則是萬曆皇帝對文官集團的報復。黃仁宇說:「他(即萬曆皇帝)身上的巨大變化發生在什麼時候,沒有人可以做出確切的答復。但是追溯皇位繼承問題的發生,以及一連串使皇帝感到大為不快的問題的出現,那麼1587年丁亥,即萬曆十五年,可以作為一條界線。這一年表面上並無重大的動蕩,但是對本朝的歷史卻有它特別重要之處。」在《萬曆十五年》文末總結,「1587年,是為萬曆15年,歲次丁亥,表面上似乎是四海昇平,無事可記,實際上我們的大明帝國卻已經走到了它發展的盡頭。在這個時候,皇帝的勵精圖治或者晏安耽樂,首輔的獨裁或者調和,高級將領的富於創造或者習於苟安,文官的廉潔奉公或者貪污舞弊,思想家的極端進步或者絕對保守,最後的結果,都是無分善惡,統統不能在事實上取得有意義的發展。因此我們的故事只好在這裡作悲劇性的結束。萬曆丁亥年的年鑑,是為歷史上一部失敗的總記錄」。

在黄仁宇等的著作中也表达出中国明代中后期,皇帝只是一个牌位,而事实上万历的个人行为对基层的国家的习惯轨迹并无大的影响。

萬曆元年十月八日,是日講的日子,朱翊鈞在文華殿聽張居正進講《帝鑑圖說》。當張居正講到宋仁宗不喜珠飾,值得效法時,朱翊鈞立即表示同感:“賢臣才是寶,珠玉又有何益!”張居正接著說:“聖明的君主貴五穀而賤珠玉,五穀可以養人,而金玉飢不可食,寒不可衣,《書經》稱不作無益害有益,不貴異物賤用物,道理也就在這裡。”“是啊!宮裡的人喜歡裝飾,我在年賜時每每節省,宮人們都有意見,我說國庫的積蓄又有多少呢?”朱翊鈞又回答說。張居正便誇獎道“皇上能這樣說,真是社稷生靈的福氣啊!”當時朱翊鈞才不過十歲。

萬曆二年,朝鮮使臣許篈,趙憲前來朝貢。許篈在其前往中國記錄見聞的《朝天記》對年幼的萬曆天子的形象進行了描寫,記載其「聲甚清朗」「天威甚邇,龍顏壯大,語聲鏗然,(我)不勝歡欣之極」同行的另一位使臣趙憲則更生動地記錄地在《朝天日記》道「上(萬曆皇帝)年僅十二歲,而注視別人時十分老成,端坐在龍椅上也不曾搖動,並不會叫太監內臣傳達他的旨意,反而是親自對臣工下聖諭,而聲音玉質淵秀,金聲清暢。(我)一聽到年幼天子的聲音,就感動起來,對以後天下太平萬歲的希望,也更加愈切了。」,而趙憲甚至把年幼的萬曆天子與其父明穆宗作比較,卻指出其父上朝時精神不集中、時常東張西望,而且聲音微弱,需要宦官再去大聲宣旨,儀態形像不佳。

自從張居正去世以後,萬曆終於能擺脫出翰林學士的羈絆;而自從他成為父親以來,李太后也不再乾預他的生活。但是,皇帝自幼聰惠,在這個時候確實已經成年了,他已經不再有興趣和小宦官去打鬧,而是變成了一個喜歡讀書的人。他命令大學士把本朝諸祖宗的“實錄”抄出副本供他閱讀,又命令宦官在北京城內收買新出版的各種書籍,包括詩歌、論議、醫藥、劇本、小說等各個方面。

萬曆十四年三月,一次君臣召對中,因京師陰霾蔽空,皇帝決定減免一些稅賦,並認為或許最近開水田太過擾民,而致上天警示,應當停止,閣臣申時行委婉地說道:「京東地方,田地荒蕪,廢棄可惜,相應開墾。」皇帝復說道:「南方地下,北方地高。南地濕潤,北地鹼燥。且如去歲天旱,井泉都乾竭了。這水田怎能做得?」於是申時行頓時認為聖裁允當,拜首執行。

明朝遺民李長祥在“天問閣集”的“劉宮人傳”中也對萬曆皇帝有過高度評價,甚至認為萬曆皇帝比起東漢光武帝,唐太宗來,品德更在其上。

明末流離出宮的一個老宮女劉氏曾在萬曆年間任職。他與李長祥講述當年的事情「一天內官(太監)持朱筆寫的傳票給萬曆皇帝看,皇帝看完不說話,太監說:「連皇帝內侍的左右內官都容不下,還敢來捉拿。」皇帝沉默了一回,便回答說:「用朱票捉拿人是巡城御史的職責,怎麼能奪他權柄,阻礙他執法,況且你們一定是幹了些什麼壞事。這事朕不管,人就隨他捉拿吧。」這時候皇帝還不知道當時發生了什麼事。

後來李長祥覽神宗遺事,原來是當年有一人告內官於御史,御史不知道他已經進宮了,即出朱票拿人。手持朱票去捉人的也不是有經驗的人,直接走到午門去索問。一眾內官馬上就大怒並把票奪走,走到皇帝面前奏上此事,皇帝說的話就跟老宮女劉氏一模一樣,居然兩事能互相對證。

李長祥也不禁大加讚許:「嗚呼聖人哉,聖人哉......考當日所為,亦飾語耳,若神宗乃真有其實,雖唐虞三代之令主,何以加此。其能使海內家給人足,道不拾遺,夜不閉戶者四十八年,有以哉!」

明神宗屍骨被發掘後,發現其駝背後左右腳短,但學者認為神宗生前並不適用。一說神宗生前從未走出過紫禁城,也不符史實,《明神宗實錄》均載,祭先皇陵、祭天、祈雨、祭孔、祭先農等重大儀式均由皇帝主持,且亦有參與騎馬、步行,均不見有載其殘頹之說,屍體上發現的殘缺應該是年老時造成的,而非先天疾病,且三十年不上朝的神宗,其實都有在內廷批奏摺、發令等,並非完全不事朝政。

英国女王伊丽莎白一世在万历二十四年(1596年)给当时中国在位的神宗皇帝写了一封亲笔信,希望英中两国开展贸易往来以及在其他领域交流的愿望。同时还派使者约翰·纽伯莱出使明朝,将这封亲笔信递交给神宗。然而使者在途中遇难,但是这封亲笔信却没有丢失,伊丽莎白一世无奈与此,称为她的终身遗憾。现在这封亲笔信被英国国家博物馆收藏。



\subsection{万历}

\begin{longtable}{|>{\centering\scriptsize}m{2em}|>{\centering\scriptsize}m{1.3em}|>{\centering}m{8.8em}|}
  % \caption{秦王政}\
  \toprule
  \SimHei \normalsize 年数 & \SimHei \scriptsize 公元 & \SimHei 大事件 \tabularnewline
  % \midrule
  \endfirsthead
  \toprule
  \SimHei \normalsize 年数 & \SimHei \scriptsize 公元 & \SimHei 大事件 \tabularnewline
  \midrule
  \endhead
  \midrule
  元年 & 1573 & \tabularnewline\hline
  二年 & 1574 & \tabularnewline\hline
  三年 & 1575 & \tabularnewline\hline
  四年 & 1576 & \tabularnewline\hline
  五年 & 1577 & \tabularnewline\hline
  六年 & 1578 & \tabularnewline\hline
  七年 & 1579 & \tabularnewline\hline
  八年 & 1580 & \tabularnewline\hline
  九年 & 1581 & \tabularnewline\hline
  十年 & 1582 & \tabularnewline\hline
  十一年 & 1583 & \tabularnewline\hline
  十二年 & 1584 & \tabularnewline\hline
  十三年 & 1585 & \tabularnewline\hline
  十四年 & 1586 & \tabularnewline\hline
  十五年 & 1587 & \tabularnewline\hline
  十六年 & 1588 & \tabularnewline\hline
  十七年 & 1589 & \tabularnewline\hline
  十八年 & 1590 & \tabularnewline\hline
  十九年 & 1591 & \tabularnewline\hline
  二十年 & 1592 & \tabularnewline\hline
  二一年 & 1593 & \tabularnewline\hline
  二二年 & 1594 & \tabularnewline\hline
  二三年 & 1595 & \tabularnewline\hline
  二四年 & 1596 & \tabularnewline\hline
  二五年 & 1597 & \tabularnewline\hline
  二六年 & 1598 & \tabularnewline\hline
  二七年 & 1599 & \tabularnewline\hline
  二八年 & 1600 & \tabularnewline\hline
  二九年 & 1601 & \tabularnewline\hline
  三十年 & 1602 & \tabularnewline\hline
  三一年 & 1603 & \tabularnewline\hline
  三二年 & 1604 & \tabularnewline\hline
  三三年 & 1605 & \tabularnewline\hline
  三四年 & 1606 & \tabularnewline\hline
  三五年 & 1607 & \tabularnewline\hline
  三六年 & 1608 & \tabularnewline\hline
  三七年 & 1609 & \tabularnewline\hline
  三八年 & 1610 & \tabularnewline\hline
  三九年 & 1611 & \tabularnewline\hline
  四十年 & 1612 & \tabularnewline\hline
  四一年 & 1613 & \tabularnewline\hline
  四二年 & 1614 & \tabularnewline\hline
  四三年 & 1615 & \tabularnewline\hline
  四四年 & 1616 & \tabularnewline\hline
  四五年 & 1617 & \tabularnewline\hline
  四六年 & 1618 & \tabularnewline\hline
  四七年 & 1619 & \tabularnewline\hline
  四八年 & 1620 & \tabularnewline
  \bottomrule
\end{longtable}


%%% Local Variables:
%%% mode: latex
%%% TeX-engine: xetex
%%% TeX-master: "../Main"
%%% End:

%% -*- coding: utf-8 -*-
%% Time-stamp: <Chen Wang: 2021-11-01 17:13:48>

\section{光宗朱常洛\tiny(1620)}

\subsection{生平}

明光宗朱常洛(1582年8月28日-1620年9月26日),或稱泰昌帝,明朝第15代皇帝,年号泰昌,庙号「光宗」,谥号“崇天契道英睿恭纯宪文景武渊仁懿孝贞皇帝”。

明神宗长子,万历十年(1582年)八月生,母恭妃王氏原是祖母李太后身边的宫人。不久,明神宗郑贵妃生三子朱常洵,深得宠爱。长子朱常洛一直受到冷遇,群臣纷纷上书要求立储,是為國本之爭,明神宗要不是贬斥群臣,就是虚与委蛇地敷衍應付。祖母李太后以为不妥。一日,李太后询问神宗未立朱常洛为太子的缘故。神宗说:他是宫人所生。李太后大怒:你也是宫人所生(李太后亦是宫人出身)。神宗听后惶恐,伏地不敢起。

万历二十九年(1601年)十月,明神宗被迫册立长子朱常洛为太子,同时,立三子朱常洵为福王、五子朱常浩为瑞王、六子朱常润为惠王、七子朱常瀛为桂王。太子朱常洛以仁厚著称,朝野皆认为其将来可为明君。但常洛的地位不穩固,郑贵妃時時刻刻想要為朱常洵爭奪儲君之位,引發了兩次妖書案,牽連眾多大臣。而後,甚至有郑贵妃手下的兩名宦官指使刺客,欲以木梃刺殺朱常洛,是為梃击案,神宗為了不牽連郑贵妃,將該刺客、宦官等三人全部殺死。

朱常洛被立为太子后,就移居慈庆宫,从此与其母王恭妃被隔绝不得相见。万历三十四年(1606年),朱常洛的妾侍王氏生下皇长孙朱由校(日后的明熹宗),神宗为表庆祝,为李太后加尊号,又进封王恭妃为皇贵妃,赐金册金宝,但仍将其屏居景阳宫。万历三十九年九月十三日(1611年10月18日),王恭妃病笃,朱常洛闻言急往景阳宫探视,见景阳宫门深锁,于是破坏门锁入内探视。当时王恭妃已双眼失明,于是以手代眼,拉着朱常洛的衣角:“儿长大如此,我死何恨!”言毕王恭妃便与世长辞。《酌中志》则记载为王恭妃病重时太子每日从苍震门入内问安;《先拨志始》更记载王恭妃察觉到郑贵妃家人偷听,提醒太子,结果母子俩直到王恭妃去世也没有说话。大学士叶向高说:“皇太子母妃薨,礼宜从厚。”神宗不应,复请,才得到允准。

万历四十八年(1620年)七月二十一日,明神宗驾崩。太子朱常洛立即发内帑(皇帝私房钱)百万犒赏边关将士。停止所有矿税,召回以言得罪的诸臣。不久,再发内帑百万犒边。八月即位,改元泰昌,是为明光宗。福王生母鄭貴妃為了攏絡明光宗,獻上四位美女。明光宗縱慾過度不久病倒,太監崔文升進以瀉藥而狂瀉。在位不足三十天的明光宗在九月初一因服用李可灼的紅丸而猝死駕崩,史稱紅丸案。

在短短的一个月,明光宗在群臣的帮助下,也做了不少实事,比如:废矿税、饷边防、补官缺。

首先下令罢免全国范围内的矿监、税使,停止任何形式的的采榷活动。矿税早为人们所深感厌恶,所以诏书一颁布,朝野欢腾。

其次是饷边防。明光宗下令由大内银库调拨二百万两银子,发给辽东经略熊廷弼和九边巡抚按官,让他们犒赏将士;并拨给运费五千两白银,沿途支用。明光宗还专门强调,银子解到后,立刻派人下发,不得擅自入库挪为它用。

第三件事是补充官缺。朱常洛先命令礼部右侍郎、南京吏部侍郎二人为礼部尚书兼内阁大学士;随后,将何宗彦等四人均升为礼部尚书兼内阁大学士;启用卸官归田的旧辅臣叶向高,同意将因为“上疏”爭國本获罪的三十三人和为矿税等获罪的十一人一概录用。因此有人感慨明光宗矫枉过正,造成了前所未有的“官满为患”的局面。

因光宗即位一個月即告駕崩,该陵墓原为景泰帝所建,因景泰帝為英宗所贬,葬于西郊金山,所以空出一处皇陵。由于明光宗在位时间仅29天,来不及修建陵墓,故继位的長子明熹宗朱由校将光宗安葬于此陵墓。

《明實錄》:“自古帝皇仁心仁聞洽于天下,未有不須久道而後成者,必世後仁聖人言之矣。乃光宗貞皇帝在位僅三旬,升遐之日,深山窮谷莫不奔走悲號,何?聖化之神感孚若是速也。蓋帝睿質夙成,蚤親師傳,養德青宮已洞悉四海之難艱。故當神皇晏駕時,遺詔未頒,德音據播 ;大寶初嗣,仁政沛施。捐朽蠹而九塞飽騰,撤狐蟊而廛勸動政。地廣股肱之助,諫垣充耳目之司。黃髮並升于公庭,白駒不滯于空谷。至于虛懷延接一月,而三召臣工銳意圖。幾浹旬而兩蠲而稅額 。德意獨行,獨斷爕理,莫施其功,威權自攬。自綜執月,御不參其柄。鑠乎盛矣,曠千古而僅見者也,乃其尤難者以何思何慮之天,處若危若疑之地。冲齡出講,已歷艱辛,而容色溫然,動止泰然。內庭有菀枯之形,若勿知也者;外庭有羽翼之激,若勿聞也者。即冊立,尋常事耳。時而舉碁,時而反汗。大臣去,小臣譴,宜何如動于耳目者。 而帝也,有夔夔無慄慄。潛之又潛,巧伺者不能窺,善孽者不能中。福藩就國,慟哭抱持。張差發難,帝侍神皇。左右親傳睿旨,曉諭百官羣囂遂息,所全實多。登極後即遵遺命進封皇貴妃,廷臣力爭,竟不忍奪以戚畹,哀請而後止,毫不芥蔕于前事也。此即虞舜大孝何以加茲?以舜之孝,擴堯之仁,然則帝之所以感動人心又自有在,而非僅僅更張注措之迹者矣。夫官天下者,壽在令名;家天下者,壽在長世。神皇即不豫,何難四十日留也。使帝之出震未及而幹蠱,莫施天下之事將不可知。然則我國家億萬年無疆之祚,皆帝四十日之所延也。帝之功德又豈但在普天之思慕已哉,天眷宗社不虗也。”

\subsection{泰昌}

\begin{longtable}{|>{\centering\scriptsize}m{2em}|>{\centering\scriptsize}m{1.3em}|>{\centering}m{8.8em}|}
  % \caption{秦王政}\
  \toprule
  \SimHei \normalsize 年数 & \SimHei \scriptsize 公元 & \SimHei 大事件 \tabularnewline
  % \midrule
  \endfirsthead
  \toprule
  \SimHei \normalsize 年数 & \SimHei \scriptsize 公元 & \SimHei 大事件 \tabularnewline
  \midrule
  \endhead
  \midrule
  元年 & 1620 & \tabularnewline
  \bottomrule
\end{longtable}


%%% Local Variables:
%%% mode: latex
%%% TeX-engine: xetex
%%% TeX-master: "../Main"
%%% End:

%% -*- coding: utf-8 -*-
%% Time-stamp: <Chen Wang: 2021-11-01 17:13:53>

\section{熹宗朱由校\tiny(1620-1627)}

\subsection{生平}

明熹宗朱由校(1605年12月23日-1627年9月30日;校,居效切,拼音「jiào」、注音「ㄐㄧㄠˋ」),或稱天啟帝,光宗長子,明朝第16代皇帝。在位時間為1620年-1627年,年號天啟。光宗即位僅一個月而亡,使朱由校匆匆登位為帝,朱由校當時僅十四歲,未曾被立为太子,甚至未接受正規教育,政事皆賴宦官輔佐,後來造就太監魏忠賢等人的干政,與閹黨、東林黨之黨爭。

泰昌元年(1620年),其父明光宗在位不足三十天便在紅丸案之中暴斃。九月初六,由長子朱由校繼任。值得一提的是,其父明光宗朱常洛一向不為祖父明神宗所喜,故朱由校亦沒有被神宗重視。神宗駕崩後,大臣代言的遺囑:「皇長孫宜即時冊立、進學。」故顯示當時已十四歲的朱由校從未進學。明光宗即位後原擇九月初九冊立朱由校為東宮,惟來不及冊封,光宗於九月初一駕崩,故明熹宗連一天正式教育都未接受,便登上大寶,此為有明一代第一人,其情況比其父光宗勉強隨其他皇子出閣讀書,而非正統的太子教育方式,還要更加惡劣,且父子倆在繼位前都未監國輔政經驗,制造內宦干政的土壤,神宗亦無留下良好輔臣,國運衰退的因素在萬曆時國本之爭時即已種下。

泰昌元年(1620年),是明朝立國以來所遇到前所未有的情況。明熹宗的祖母孝端顯皇后、祖父明神宗與父親明光宗相繼在同一年駕崩,明神宗駕崩距孝端顯皇后駕崩才兩個多月,而明光宗駕崩時距明神宗駕崩不到一個月,實屬罕見。而明神宗與孝端顯皇后的大葬尚未完成,因此明廷在討論後,決定先為明神宗與孝端顯皇后辦理大葬,結束後再為明光宗辦理大葬。

熹宗繼位後,撫養皇帝的李選侍利用皇帝年少無知,佔據乾清宮,意圖把持朝政,東林黨左光斗、楊漣等反對,不讓李選侍與皇帝同住,迫使她移居他處,是為移宮案,此事後內侍魏忠賢被提拔為司禮監秉筆太監,魏忠賢與熹宗是皇孫時代即結識的舊識,魏忠賢乘機結交朱由校乳母客氏,兩人遂狼狽為奸。熹宗有感東林黨黨人從龍之功,大加提拔任用,又召回葉向高等先朝老臣擔任內閣首輔,時稱「眾正益朝」「群賢滿朝」,天下欣欣望治。另外熹宗也屢發內帑犒勞將士,補發九邊欠餉,如即位之初便發派一百八十萬帑金以勞邊,派帑金五十萬以給光宗陵工,準發帑五十萬作解發以發兵餉,又答允兵部再發帑金一百萬以佐急需,接著不足一年又因首輔葉向高所請而發帑金二百萬為東西兵餉之用。

朱由校喜歡木工,亦沉迷於刀鋸斧鑿,魏忠賢總是趁他木工做得全神貫注時,拿重要的奏章去請他批閱,熹宗隨口說:「朕已悉矣!汝輩好為之。」魏忠賢遂逐漸專權,竊奪威福,魏忠賢閹黨誣陷忠良,殺死包括東林六君子、東林七賢等正直的士大夫,致使朝政敗壞。

同時期,女真首領努爾哈赤則起於白山黑水之間,趁機攻佔瀋陽,奪取遼東地區,聲勢日隆。

天啟六年(1626年)北京發生「王恭廠大爆炸」,死傷2萬餘人,原因不明,朝野震驚,中外駭然,熹宗下了一道罪己詔,表示要痛加省醒,告誡大小臣工「務要竭慮洗心辦事,痛加反省」,并下旨發府庫萬兩黃金賑災。

天啟七年(1627年)八月,熹宗又與宦官魏忠賢、王體乾等去西苑深水處泛舟,卻因風強,小舟翻覆,皇帝落水,雖然隨即被救,但從此驚豫不堪,逐漸病重,尚書霍維華獻「靈露飲」,以五穀蒸餾而成,清甜可口,但幾個月後病情加劇,渾身浮腫,八月十一日,召見信王朱由檢,即行駕崩,時年23歲,廟號熹宗。熹宗諸子皆早夭,遺詔立五弟朱由檢為皇帝,即後來的明思宗。禮部定謚號曰「哲皇帝」,思宗宸墨改為「悊」。

《明實錄》:「上念光皇大業未究,雅志繼述,踐祚之初委任老成,摉羅遺逸,振鷺充庭,稱盛理焉。時四方多故,上宵旴靡遑,遼左及滇黔相繼請帑,無不立應,大臣行邊恩禮優渥,將士陷陣恤典立頒,又慮加派苦累,每有詔諭諄諄戒守令,加意撫字毋重困吾民。其軫念民碞如此,故能收拾人心,挽回天步,雖有煬灶假叢之奸而得人付托,社稷永固於苞桑。廟號曰熹,蓋稱有功安人云。」

《明史》:「自世宗而後,綱紀日以陵夷,神宗末年,廢壞極矣。雖有剛明英武之君,已難復振。而重以帝之庸懦,婦寺竊柄,濫賞淫刑,忠良慘禍,億兆離心,雖欲不亡,何可得哉。」

明朝劉若愚《酌中志》对熹宗评价较高,“先帝(明熹宗)生性虽不好静坐读书,然能留心大体,每一言一字,迥出臣子意表”;熹宗在宁锦大战中“日夜焦思,未遑自安”,王永光的题疏中曾有“要将宁远城中红夷大炮撤归山海关”,明熹宗批示:“此炮如撤,人心必摇”,表明他是有一定的政治决断力。当后金军队再犯锦州、宁远之时,“更愤激深虑”,对魏忠贤和乳母客氏也怒骂咒恨,形于颜色。同时,熹宗又「又極好作水戲,用大木桶大銅缸之類,鑿孔創機,啟閉灌輸,或湧瀉如噴珠,或澌流如瀑布,或使伏機於下,借水力沖擁園木球如核桃大者,於水湧之,大小盤旋宛轉 ,隨高隨下,久而不墮,視為嬉笑,皆出人意表。」。他曾親自在庭院中造了一座小宮殿,形式仿乾清宮,高不過三四尺,卻曲折微妙,巧奪天工。可见刘若愚对明熹宗的评价颇高。

明末清初談遷認為「閹尹之禍,劇於熹廟,并边徽而二之。……疵德多矣」。將閹黨及滿清視為天啟年間兩大威脅,可見其嚴重性。

清道光年間抱陽生《甲申朝事小紀》,認為朱由校沈迷於木工,放任魏忠賢矯詔、管理朝政的行為視為貪玩而不長進,史載「又好油漆,凡手用器具,皆自為之。性又急躁,有所為,朝起夕即期成。成而喜,不久而棄;棄而又成,不厭倦也。且不愛成器,不惜改毀,唯快一時之意。」「朝夕營造」,「每營造得意,即膳飲可忘,寒暑罔覺」。

民國直系將領吳佩孚認為明熹宗寵信閹黨,濫殺東林六君子、東林七賢,才是明朝亡國的主因,更甚於萬曆。其恩師王紹勛,與吳佩孚提及明神宗怠政三秩時,感歎曰:「無為而治兮不必生一神宗三秩」,吳佩孚居然立刻應聲對仗:「有明之亡矣莫非殺六君子七賢。」

《從萬曆到永曆》一書認為,魏忠賢不可能屢屢矯詔,故而天啟一朝的政治,包括鎮壓東林的決策,還是與熹宗相關,熹宗遭到了後來主編明史的東林和復社人士抹黑。此外,即便是明史也明確記載了熹宗對於朝政的參與,不可謂無自相矛盾之處。例如王士禛所謂老宮監刘若愚的原話是:(先帝)且不爱成器,不惜天物,任暴殄改毁,惟快圣意片时之适。当其斤斫刀削,解服磐礴,非素昵近者不得窥视,或有紧切本章,体乾等奏文书,一边经管鄙事,一边倾耳注听。奏请毕,玉音即曰:「尔们用心行去,我知道了」。這和所謂勤政的清朝皇帝批示:「知道了。」是差不多的作為。

此外,朱由校所謂沈迷於木工,很有可能是因為對於宮殿藝術有所追求,由於前兩次主要工程人員如蒯祥皆過世,為了三大殿能復原,朱由校特別注重木作部分等,事出有因,並非只因為個人興趣而不理會朝政。當時因南京三大殿早已燒失,北京紫禁城三大殿於萬曆年間亦燒燬,朱由校效法太祖親自監督三大殿重建計畫,聽從御史王大年节俭的建议,才會鑽研木匠手藝。

\subsection{天启}

\begin{longtable}{|>{\centering\scriptsize}m{2em}|>{\centering\scriptsize}m{1.3em}|>{\centering}m{8.8em}|}
  % \caption{秦王政}\
  \toprule
  \SimHei \normalsize 年数 & \SimHei \scriptsize 公元 & \SimHei 大事件 \tabularnewline
  % \midrule
  \endfirsthead
  \toprule
  \SimHei \normalsize 年数 & \SimHei \scriptsize 公元 & \SimHei 大事件 \tabularnewline
  \midrule
  \endhead
  \midrule
  元年 & 1621 & \tabularnewline\hline
  二年 & 1622 & \tabularnewline\hline
  三年 & 1623 & \tabularnewline\hline
  四年 & 1624 & \tabularnewline\hline
  五年 & 1625 & \tabularnewline\hline
  六年 & 1626 & \tabularnewline\hline
  七年 & 1627 & \tabularnewline
  \bottomrule
\end{longtable}


%%% Local Variables:
%%% mode: latex
%%% TeX-engine: xetex
%%% TeX-master: "../Main"
%%% End:

%% -*- coding: utf-8 -*-
%% Time-stamp: <Chen Wang: 2021-11-01 17:14:00>

\section{思宗朱由檢\tiny(1627-1644)}

\subsection{生平}

明思宗朱由檢(1611年2月6日-1644年4月25日),或稱崇禎帝,明朝第17代、末代皇帝。

思宗为明光宗第五子,明熹宗异母弟。五歲時,其母劉氏獲罪,被時為太子的光宗下令杖殺,朱由检交由庶母西李撫養,數年後改由另一庶母东李撫養至成人。於天启二年(1622年)年被兄長明熹宗册封為信王。明熹宗於天啟七年(公元1627年8月)駕崩,由于没有子嗣,朱由检受遗命于同月丁巳日登基,时年十八歲。次年改元崇禎,是为明思宗。

思宗一生操勞,日以繼夜的批閱奏章,节俭自律,不近女色。崇祯年間,与萬曆、天啟相较,朝政有了明显改观。即位之初就大力铲除阉党,曾六度下诏罪己,惜其生性多疑,无法挽救衰微的明朝。明朝末年农民起义不断,关外后金政权虎视眈眈,已处于内忧外患的境地。崇祯十七年(1644年)發生甲申之變,李自成攻破北京,思宗在煤山一树上吊身亡,终年三十三岁,在位十五年。

南明予其庙号「思宗」,后改「毅宗」、「威宗」,南明弘光帝上谥号「绍天绎道刚明恪俭揆文奋武敦仁懋孝烈皇帝」。清朝追谥「钦天守道敏毅敦俭弘文襄武体仁致孝端皇帝」,庙号「怀宗」;后去庙号,改谥为「庄烈愍皇帝」,葬于十三陵思陵。

生於萬曆庚戌十二月二十四日 ( 1611年2月6日 ) 寅時。崇祯帝之父為明光宗朱常洛,朱常洛雖早在萬曆廿九年 ( 1601年 ) 被立為太子,但其父親明神宗其實一心想立三子朱常洵為太子,是因為群臣國本之爭,才勉強保住了朱常洛儲君的寶座,故朱常洛一直得不到明神宗歡心。朱由检母亲刘氏則是朱常洛的婢女,亦不得朱常洛的歡心。祖父討厭父親,父親討厭母親,所以朱由检幼年并不幸福。五岁时,朱由檢母親劉氏得罪,被父親朱常洛下令杖杀,之後將朱由检交由庶母西李抚养。数年后西李生了女儿,照管不过来,改由另一庶母东李抚养至成人。及至朱由检长大,被當時已繼位為帝的哥哥朱由校封为信王,刘氏追封为贤妃。

天启七年(1627年),年僅廿二歲的明熹宗朱由校駕崩,由於朱由校三名兒子皆早夭,他唯一在世的弟弟朱由檢繼承皇位,當時朱由檢年僅十六歲,是為崇禎帝。朱由檢即位后,勤于政务,事必躬亲。崇祯十五年(1642年)七月初九,因“偶感微恙”而临时传免早朝,遭辅臣批评,崇禎連忙自我檢討。

天启七年十一月(1627年),崇祯帝在铲除魏忠贤的羽翼崔呈秀之后,再将其贬至凤阳。途至直隶阜城,魏忠贤得知大勢已去,遂与一名太监自缢而亡。此后崇祯帝又殺客氏,崔呈秀自盡,其阉党二百六十餘人或处死、或发配、或终身禁锢。与此同时,平反冤狱,重新启用天启年间被罢黜的官员。起用袁崇焕为兵部尚书,赐予尚方宝剑,託付他收复全辽的重任。

自崇禎元年(1628年)起,中國北方大旱,赤地千里,寸草不生,《汉南续郡志》记,“崇祯元年,全陕天赤如血。五年大饥,六年大水,七年秋蝗、大饥,八年九月西乡旱,略阳水涝,民舍全没。九年旱蝗,十年秋禾全无,十一年夏飞蝗蔽天……十三年大旱……十四年旱”。崇祯朝以來,陕西年年有大旱,百姓多流離失所。崇祯二年五月正式议裁陕北驛站,驛站兵士李自成失业。崇祯三年(1630年)陝西又大饑,陝西巡按馬懋才在《備陳大饑疏》上說百姓爭食山中的蓬草,蓬草吃完,剝樹皮吃,樹皮吃完,只能吃觀音土,最後腹脹而死,六年,“全陕旱蝗,耀州、澄城县一带,百姓死亡过半”。

崇祯七年,家住河南的前兵部尚书吕维祺上書朝廷:“盖数年来,臣乡无岁不苦荒,无月不苦兵,无日不苦輓输。庚午(崇祯三年)旱;辛未旱;壬申大旱。野无青草,十室九空。……村无吠犬,尚敲催征之门;树有啼鹃,尽洒鞭扑之血。黄埃赤地,乡乡几断人烟;白骨青燐,夜夜似闻鬼哭。欲使穷民之不化为盗,不可得也”。旱災又引起蝗災,使得災情更加擴大。河南於崇禎十年、十一年、十二年、十三年皆有蝗旱,“人相食,草木俱盡,土寇並起”,其飢民多從“闖王”李自成。崇祯十三、十四年,“南北俱大荒……死人弃孩,盈河塞路。”

十四年,左懋第督催漕運,道中馳疏言:“臣自靜海抵臨清,見人民飢死者三,疫死者三,為盜者四。米石銀二十四兩,人死取以食。惟聖明垂念。”保定巡撫徐標被召入京時說:“臣自江推來數千里,見城陷處固蕩然一空,即有完城,亦僅餘四壁城隍,物力已盡,蹂躪無餘,蓬蒿滿路,雞犬無音,未遇一耕者,成何世界”這時華北各省又疫疾大起,朝發夕死。“至一夜之內,百姓驚逃,城為之空”,崇禎十四年七月,疫疾從河北地区傳染至北京,崇祯十六年,北京人口死亡近四成。十室九空。

江南在崇祯十三年遭大水,十四年有旱蝗并灾,十五年持续发生旱灾和流行大疫。地方社会处在了十分脆弱的状态,盗匪与流民並起,各地民变不断爆发。

為剿流寇,崇祯帝先用楊鶴主撫,後用洪承疇,再用曹文詔,再用陳奇瑜,復用洪承疇,再用盧象昇,再用楊嗣昌,再用熊文燦,又用楊嗣昌,十三年中頻繁更換圍闖軍的將領。這其中除熊文燦外,其他都表現出了出色的才幹。然皆功虧一簣。李自成數次大難不死,後往河南聚眾發展。

此时北方皇太极又不断骚扰入侵,明廷苦於两线作战,每年的军费「三餉」开支高达两千万两以上,国家财政早已入不敷出,缺饷的情況普遍,常导致明军内部骚乱哗变。加上崇祯帝求治心切,生性多疑,刚愎自用,因此在朝政中屡铸大错:前期铲除专权宦官,后期又重用宦官,《春明梦余录》记述:“崇祯二年十一月,以司礼监太监沈良住提督九门及皇城门,以司礼监太监李凤翔总督忠勇营”崇祯帝說:“朕禦極之初,攝還內鎮,舉天下大事悉以委大小臣工,比者多營私圖,因協民艱,廉通者又遷疏無通。己已之冬,京城被攻,宗社震驚,此士大夫負國家也。清寫明史崇祯帝中后金反间计,自毁长城,冤杀袁崇焕;世傳皇太極施反間計,捕捉兩名明宮太監,然後故意讓兩人以為聽見滿清將軍之間的耳語,謂袁崇煥與滿人有密約,皇太極再放其中一名太監回京。崇祯帝中計,以為袁崇煥謀反。這種講法終明之世並無所本,僅流行於乾隆之後。一些學者傾向於相信崇祯帝殺袁崇煥,並非是皇太極的反間計得逞。由於袁崇煥是囚禁半年後才被處死的,不大可能是因一時激憤誤殺。事實上,崇祯帝生性多疑,所以僅擅殺毛文龍一事,便足以使崇祯帝心存忌憚。再者毛文龍舊部大都誤認為是皇帝要殺毛文龍,於是把怨恨轉移到皇帝身上,大舉譁變,造成日後一連串悲劇事件的發生,終於致使前線態勢一發不可收拾。袁崇煥不能不為此負責。

隨著局勢的日益嚴峻,崇祯帝的濫殺也日趨嚴重,總想以重典治世,總督中被誅者七人,巡撫被戮者十一人,連擁有崇高地位的內閣首辅也不能幸免,被殺二人,而其他各級文官武將更是多不勝數,不能詳列。崇祯帝亦知不能兩面作戰,私底下同意議和,但被明朝士大夫鑒於南宋的教訓,皆以為與滿人和談為恥。因此崇祯帝對於和議之事,始終左右為難,他暗中同意杨嗣昌的议和主张,但一旁的盧象昇立即告訴皇帝說:「陛下命臣督师,臣只知战斗而已!」,崇祯帝只能辯称根本就没有议和之事,盧象昇最後戰死沙場。明朝末年就在和戰兩難之間,走入滅亡之途。

崇禎十五年(1642年),松山、锦州失守,洪承畴降清,崇祯又想和满清议和而和兵部尚書陳新甲暗中商議計劃,後來陳新甲因泄漏議和之事被崇祯诿过處死,與清兵最後議和的機會也破滅了。崇禎十七年(1644年)明王朝面临没顶之灾,崇祯帝召見閣臣時悲嘆道:“吾非亡国之君,汝皆亡国之臣。吾待士亦不薄,今日至此,群臣何无一人相从?”在陳演、光時亨等反对和不情願負責之下未能下决心迁都南京。事後崇禎帝指責光時亨:“阻朕南遷,本應處斬,姑饒這遭。”後來,崇禎再次跟李明睿和左都御李邦華復議南遷的計劃,並要大學士陳演擔當責任,陳演不情願,於是在不久後被罷職。第二次南遷計劃失敗後,崇禎讓駙馬鞏永固代口要求重臣守京師,並以“聖駕南巡,征兵親討」為由出京,諸臣唯恐自己因皇帝不在京城而變成農民军發泄怒火的替死鬼,故依然不讓崇禎離京。

至此,农民军起义已经十多年,从北京向南,南京向北,纵横数千里之间,白骨满地,人烟断绝,行人稀少。崇祯帝召保定巡抚徐标入京觐见,徐标说:“臣从江淮而来,数千里地内荡然一空,即使有城池的地方,也仅存四周围墙,一眼望去都是杂草丛生,听不见鸡鸣狗叫。看不见一个耕田种地之人,像这样陛下将怎么治理天下呢?”崇祯帝听后,潸然泪下,叹息不止。于是,为了祭祀阵亡将士、罹难难民和殉國的各亲王,崇祯帝便在宫中大作佛事来祈求天下太平,并下诏罪己,催促督师孙传庭赶快围剿农民军。

崇禎十六年正月,李自成部克襄陽、荊州、德安、承天等府,張獻忠部陷蘄州,明將左良玉逃至安徽池州。崇禎十七年(1644年)三月一日,大同失陷,北京危急,初四日,崇禎任吳三桂為平西伯,飛檄三桂入衛京師,起用吳襄提督京營。六日,李自成陷宣府,太監杜勳投降,十五日,大學士李建泰投降,李自成部開始包圍北京,太監曹化淳說:「忠賢若在,時事必不至此。」三月十六日,昌平失守,十七日,圍攻北京城。三月十八日,李自成軍以飛梯攻西直、平則、德勝諸門,守軍或逃、或降。下午,曹化淳開彰儀門(一說是十九日王相堯開宣武門,另張縉彥守正陽門,朱純臣守朝陽門,一時俱開,二臣迎門拜賊,賊登城,殺兵部侍郎王家彥於城樓,刑部侍郎孟兆祥死於城門下),李自成軍攻入北京。太監王廉急告皇帝,思宗在宫中饮酒长叹:“苦我民尔!”太監張殷勸皇帝投降,被一劍刺死。崇祯帝命人分送太子、永王、定王到勳戚周奎、田弘遇家。又逼周后自杀,手刃袁妃(未死)、長平公主(未死)、昭仁公主。

然後思宗手執三眼槍與數十名太監騎馬出東華門,被亂箭所阻,再跑到齊化門(朝陽門),成國公朱純臣閉門不納,後轉向安定門,此地守軍已經星散,大門深鎖,太監以利斧亦無法劈開。三月十九日拂曉,大火四起,重返皇宮,城外已经是火光映天。此時天色将明,崇祯在前殿鸣钟召集百官,却无一人前来,崇祯帝說:“诸臣误朕也,国君死社稷,二百七十七年之天下,一旦弃之,皆为奸臣所误,以至于此。”最後在景山老歪脖子树上自缢身亡,死时光着左脚,右脚穿着一只红鞋。死於崇禎甲申三月十九日丑時,时年33岁。身边仅有提督太监王承恩陪同。上吊死前于蓝色袍服上大书其遺書:

“朕自登極(或作登基)十有七年,虽朕凉德藐躬(或作薄德匪躬),上干天咎(或作天譴、天怒),致逆贼直逼京师,然皆诸臣之误朕也。朕死无面目见祖宗于地下,自去冠冕,以髮覆面。任贼分裂朕尸,勿伤百姓一人。”

三月二十一日屍體被發現,大順軍將崇祯帝與周皇后的屍棺移出宮禁,在東華門示眾,也允許投降的諸臣前往送葬,只是人數不多,“諸臣哭拜者三十人,拜而不哭者六十人,餘皆睥睨過之”,只有主事劉養貞極其悲痛,梓宮暫厝在紫禁城北面的河邊。

崇祯帝死後,自杀官員有户部尚书倪元璐、工部尚书范景文、左都御史李邦华、左副都御史施邦曜、协理京营兵部右侍郎王家彦、大理寺卿凌义渠、太常寺卿吳麟徵、左中允刘理顺、刑部右侍郎孟兆祥、前戶科都給事中吳甘來、武庫主事成德、兵部主事金鉉、左諭德马世奇、檢討汪偉、右庶子周鳳翔、太僕寺丞申佳胤、吏部員外郎許直、戶部員外郎寧承烈、光禄寺署丞于腾雲、副兵馬使姚成、中書舍人宋天顯,滕之所、阮文貴、監察御史王章、陳良謨、陳纯德、經歷張應選,順天府知事陈貞達等、外戚如驸马都尉巩永固、新樂伯劉文炳、惠安伯張慶臻、宣城伯衛時春,錦衣衛都指揮使王國興自殺,太监自杀者以百计,战死在千人以上。宫女自杀者三百余人。绅生生员等七百多家举家自杀。四月四日,昌平州吏趙一桂等人將崇禎與皇后葬入昌平縣田貴妃的墓穴之中,清朝以“帝禮改葬,令臣民為服喪三日,諡曰莊烈愍皇帝,陵曰思陵”。

崇禎十七年五月初六日,多爾袞以李明睿為禮部侍郎,負責大行皇帝的諡號祭葬事宜,李擬上先帝諡號欽天守道敏毅敦儉弘文襄武體仁致孝端皇帝,廟號懷宗,并建議改葬梓宮。後因思宗梓宮已入葬恭淑端惠靜懷皇貴妃的園寢,便不再遷葬,改田貴妃園寢為思陵。

順治十六年十一月,以“興朝諡前代之君禮,不稱數、不稱宗”為由,[原創研究?]去懷宗廟號,改諡莊烈愍皇帝,因而清代史書多簡稱為莊烈帝或明愍帝。

《欽定古今圖書集成·方輿彙編·職方典·順天府部雜錄十一》、《欽定日下舊聞考·卷一百三十七》、《讀禮通考·卷九十三》三書均引《肅松錄》和《北游紀方》,稱思陵神牌題為“大明欽天守道敏毅敦儉弘文襄武體仁致孝莊烈愍皇帝”,又引《北游紀方》稱思陵神主題為“大明懷宗欽天守道敏毅敦儉弘文襄武體仁致孝莊烈端皇帝”,又引《肅松錄》稱思陵立有“莊烈愍皇帝之陵”的石碑。《明詩綜·卷一》則稱神牌是由順治初年定的“一十六字”加上改書的“莊烈愍皇帝”組合而成。神主甚至又改“愍”字為“端”,並仍題廟號“懷宗”二字,可見康熙年間的思陵神牌和神主是由順治年間兩次加諡崇禎帝的廟諡號混雜而成。《崇禎長編·卷一》作“果毅敦儉弘文襄武體仁致孝莊烈愍皇帝”,當是清廷所給諡號在傳抄中產生了訛誤。

南明安宗之大臣張慎言初議崇禎帝之廟諡號為“烈宗敏皇帝”,高弘图拟庙号“思宗”,顧錫疇議廟號“乾宗”。赵之龙上疏弹劾高弘图议庙号之失,称“思为下谥”。顧錫疇又拟庙号正宗,但未被採用。最終在崇禎十七年六月定先帝谥號為紹天繹道剛明恪儉揆文奮武敦仁懋孝烈皇帝,庙号思宗。 按《逸周書·諡法解》:“道德純一曰思。大省兆民曰思。外內思索曰思。追悔前過曰思。……有功安民曰烈。以武立功。秉德尊業曰烈。”

弘光元年李青上疏请改思宗庙号,多次上疏皆被駁回。管紹寧擬“敬宗”和“毅宗”兩號備選,同時又有人上疏請求改為“烈宗正皇帝”。弘光元年二月丙子改上廟號毅宗,谥号未改。唐王监国,谥思宗為威宗。

與其他朝代的亡國之君不同,崇祯帝是一個被普遍同情的皇帝,崇祯帝一直勤政,以挽救過去祖輩皇帝的過失。崇祯帝即位,正值國家內憂外患之際,內有黃土高原上百萬農民造反大軍,外有滿洲鐵騎,虎視耽耽,崇祯元年(1628年)陕西镇的兵饷积欠到30多月,次年二月延绥、宁夏、固原三镇皆告缺饷达36月之久。

推翻明朝的李自成《登極詔》也說“君非甚闇(崇禎皇帝不算太糟),孤立而煬灶恆多(孤立於上,而受到奸臣的蒙蔽);臣盡行私,比黨而公忠絕少。”

思宗的性格相當複雜,在去除魏忠賢時,崇禎表現得極為機智,但在處理袁崇煥一事,卻又表現得相當愚蠢,《明史》說他:「性多疑而任察,好刚而尚气。任察则苛刻寡恩,尚气则急遽失措。」

张岱认为「思宗焦心求治,旰食宵衣,恭俭辛勤,万机无旷。即古之中兴令主无以过之。」然而,他「惟务节省」,以至「九边军士数年无饷,体无完衣」;又「渴于用人,骤于行法」,以至「天下之人,无所不用。及至危及存亡之秋,并无一人为之分忧宣力。」

《明史》評價思宗:「帝承神、熹之後,慨然有為。即位之初,沈機獨斷,刈除奸逆,天下想望治平。惜乎大勢已傾,積習難挽。在廷則門戶糾紛,疆埸則將驕卒惰。兵荒四告,流寇蔓延。遂至潰爛而莫可救,可謂不幸也已。然在位十有七年,不邇聲色,憂勸惕勵,殫心治理。臨朝浩歎,慨然思得非常之材,而用匪其人,益以僨事。乃復信任宦官,布列要地,舉措失當,制置乖方。祚訖運移,身罹禍變,豈非氣數使然哉。迨至大命有歸,妖氛盡掃,而帝得加諡建陵,典禮優厚。是則聖朝盛德,度越千古,亦可以知帝之蒙難而不辱其身,為亡國之義烈矣。」

顺治帝評價思宗:「本朝入关定鼎,首为崇祯帝、后发丧,营建幽宫,为万古未闻之义举。」1657年,顺治谕工部曰:「朕念明崇祯帝孜孜求治,身殉社稷。若不急为阐扬,恐于千载之下,竟与失德亡国者同类并观,朕用是特制碑文一道,以昭悯恻。」谒崇祯陵的时候,顺治大呼说:「大哥大哥,我与若皆有君无臣。」顺治对崇祯的书法更是高度赞赏。史书记载,僧弘觉向顺治索字,顺治说:「朕字何足尚,崇祯帝乃佳耳。」说完叫人一并拿来八九十幅崇祯的字,一一展示,“上容惨戚,默然不语”。看完了,顺治说:「如此明君,身婴巨祸,使人不觉酸楚耳。」又说:「近修《明史》,朕敕群工不得妄议崇祯帝。」顺治的话,连弘觉都给感动了:「先帝何修得我皇为异世知己哉!」顺治写给崇祯的碑文是:「庄烈悯皇帝励精图治,宵旰焦心,原非失德之主。良由有君无臣,孤立于上,将帅拥兵而不战,文吏噂沓而营私。……逮逆渠犯阙,国势莫支,帝遂捐生以殉社稷。……」

談遷《國榷》稱:“先帝(崇禎)之患,在於好名而不根于實,名愛民而適痡之,名聽言而適拒之,名亟才而適市之;聰于始,愎于終,視舉朝無一人足任者,柄托奄尹,自貽伊戚,非淫虐,非昏懦,而卒與桀、紂、秦、隋、平、獻、恭、昭並日而語也,可勝痛哉!”

歷史學家孟森說:“思宗而在萬曆以前,非亡國之君;在天啟之後,則必亡而已矣!”。思宗雖有心為治,卻無治國良方,以致釀成亡國悲劇,未必無過。孟森也說思宗“苛察自用,無知人之明”、“不知恤民”。思宗用人不彰、疑心過重、馭下太嚴,史稱“崇禎五十相”(在位十七年,更換五十位內閣大學士、首輔),卻加速了明王朝的覆亡。

鎖綠山人在《明亡述略》中評價崇禎,“莊烈帝勇於求治,自異此前亡國之君。然承神宗、熹宗之失德,又好自用,無知人之識。君子修身齊家,宜防好惡之癖,而況平天下乎?雖當時無流賊之蹂躪海內,而明之亡也決矣。”

南明大臣則把思宗抬舉到千古聖主的地步,如禮部侍郎余煜在議改思宗廟號時說:“先帝(崇禎)英明神武,人所共欽,而內無聲色狗馬之好,外無神仙土木之營,臨難慷慨,合國君死社稷之義。千古未有之聖主,宜尊以千古未有之徽稱。”

\subsection{崇祯}

\begin{longtable}{|>{\centering\scriptsize}m{2em}|>{\centering\scriptsize}m{1.3em}|>{\centering}m{8.8em}|}
  % \caption{秦王政}\
  \toprule
  \SimHei \normalsize 年数 & \SimHei \scriptsize 公元 & \SimHei 大事件 \tabularnewline
  % \midrule
  \endfirsthead
  \toprule
  \SimHei \normalsize 年数 & \SimHei \scriptsize 公元 & \SimHei 大事件 \tabularnewline
  \midrule
  \endhead
  \midrule
  元年 & 1628 & \tabularnewline\hline
  二年 & 1629 & \tabularnewline\hline
  三年 & 1630 & \tabularnewline\hline
  四年 & 1631 & \tabularnewline\hline
  五年 & 1632 & \tabularnewline\hline
  六年 & 1633 & \tabularnewline\hline
  七年 & 1634 & \tabularnewline\hline
  八年 & 1635 & \tabularnewline\hline
  九年 & 1636 & \tabularnewline\hline
  十年 & 1637 & \tabularnewline\hline
  十一年 & 1638 & \tabularnewline\hline
  十二年 & 1639 & \tabularnewline\hline
  十三年 & 1640 & \tabularnewline\hline
  十四年 & 1641 & \tabularnewline\hline
  十五年 & 1642 & \tabularnewline\hline
  十六年 & 1643 & \tabularnewline\hline
  十七年 & 1644 & \tabularnewline
  \bottomrule
\end{longtable}


%%% Local Variables:
%%% mode: latex
%%% TeX-engine: xetex
%%% TeX-master: "../Main"
%%% End:

%% -*- coding: utf-8 -*-
%% Time-stamp: <Chen Wang: 2019-12-26 15:09:40>

\section{南明\tiny(1644-1662)}

\subsection{生平}

南明(1644年-1662年),中國朝代,是甲申之變後,明朝皇族與官員在中國南方相繼成立的明朝政權,為時十八年[註 1]。南明主要勢力有四系王,分別是福王弘光帝朱由崧、魯王監國朱以海、唐王隆武帝朱聿鍵與紹武帝朱聿鐭、桂王永曆帝朱由榔等。

1644年明朝首都北京被李自成攻陷[1][2],南明大臣意圖擁護皇族北伐。經過多次討論後由鳳陽總督馬士英與江北四鎮高傑、黃得功、劉澤清與劉良佐擁護明思宗的堂兄弟福王朱由崧稱帝,即弘光帝,国号依旧为大明,史称南明或后明。1645年清軍攻破揚州[3][4][5],弘光帝逃至蕪湖被逮,後被送到北京殺害[6]。弘光帝死後,魯王朱以海於浙江紹興監國;而唐王朱聿鍵在鄭芝龍等人的擁立下,於福建福州稱帝,即隆武帝。然而這兩個南明主要勢力互不承認彼此地位,而互相攻打。1651年在舟山群島淪陷後,魯王朱以海在張名振、張煌言陪同下,赴廈門依靠鄭成功,不久病死在金門。隆武帝屢議出師北伐,然而得不到鄭芝龍的支持而終無所成。1646年,清軍分別占領浙江與福建,魯王朱以海逃亡海上,隆武帝於汀州逃往江西時被俘而死。鄭芝龍向清軍投降,由於其子鄭成功起兵反清而被清廷囚禁。朱聿鍵死後,其弟朱聿鐭在廣州受蘇觀生及廣東布政司顧元鏡擁立稱帝,即紹武帝,於同年年底被清將李成棟攻滅。同時間桂王朱由榔於廣東肇慶稱帝,即永曆帝[6]。

1646年永曆帝獲得瞿式耜、張獻忠餘部李定國、孫可望等勢力的加入以及福建鄭成功勢力的支援之下展開反攻。同時各地降清的原明軍將領先後反正,例如1648年江西金聲桓、廣東李成棟、廣西耿獻忠與楊有光率部反正,一時之間南明收服華南各省。然而於同年,清將尚可喜率軍再度入侵,先後占領湖南、廣東等地。兩年後,李定國、孫可望與鄭成功發動第二次反攻,其中鄭成功一度包圍南京。然而,各路明軍因為距離互相難以照應,內部又發生孫可望等人的叛變,第二次反攻以節節敗退告終。1661年,清軍三路攻入云南,永曆帝流亡缅甸首都曼德勒,被缅甸王莽達收留。後吴三桂攻入缅甸,莽達之弟莽白乘机发动政变,杀死其兄後继8月12日,莽白發動咒水之难,杀盡永曆帝侍從近衛[7],永曆帝最後被吴三桂以弓弦絞死,南明正式滅亡[6]。此時反清勢力只剩夔東十三家軍與在金廈及台灣的明鄭王朝。

明崇禎十七年(1644年)正月,李自成在西安稱帝,建國「大順」,之後向北京進兵,三月十九攻克北京,崇禎皇帝朱由檢殉國,明朝宗室及遺留大臣多輾轉向南遁走。此时李自成的「大順」政權大体據有淮河以北原明朝故地,張獻忠於八月成立的「大西」政權則據四川一帶,清朝政權則據有山海关外的现今东北地区,且控制蒙古诸部落,而明朝的殘餘勢力則據有淮河以南的半壁江山。

此时明朝留都南京的一些文臣武將決計擁立朱家王室的藩王,重建明朝,然後揮師北上;但具体拥立何人则发生争议。根据“皇明祖训”,有嫡立嫡、无嫡立长,在当时明神宗长子光宗一脉(其後繼者是熹宗天啟皇帝和思宗崇禎皇帝)已无人能继位,而次子朱常漵甫生即死,三子朱常洵虽已亡故,但其长子朱由崧仍健在的情况下,按照兄終弟及的順序,第一人選為福王朱由崧;而钱谦益等东林党人由於之前的「國本之爭」事件,心存芥蒂,违背了东林党在国本之争中的立场,以立贤为名擁立神宗弟弟朱翊镠之子潞王朱常淓[8][9];史可法则主张既要立贤也要立亲,拥立神宗七子桂王朱常瀛。但最终福王朱由崧在卢九德的帮助下,获得了南京政权主要武装力量江北四镇高杰、黄得功、刘良佐和刘泽清,以及中都凤阳总督马士英的支持,成为最终的胜利者。五月初三,朱由崧監國于南京,五月十五 (1644年6月19日) 日即皇帝位,改次年為弘光,是為明安宗。南明時代自此開始。弘光帝的基本國策以「聯虜平寇」為主,謀求與清軍連合,一起消滅以李自成、张献忠为代表的農民軍。

明朝南渡前後,大顺已被多爾袞與吳三桂的聯軍击溃,李自成先后丢失北京和西安,退往湖北。弘光元年(1645年)三月,多尔衮将军事重心东移,命多铎移师南征。此时弘光政權內部正進行著激烈的黨爭,爆发太子案,駐守武昌的左良玉不愿与李自成正面交战,以「清君側」为名,順长江東下争夺南明政权。馬士英被迫急調江北四鎮迎擊左軍,致使面对清军的江淮防線陷入空虛。史可法时在揚州虽有督師之名,却实无法调动四镇之兵。一月之中,清軍破徐州,渡淮河,兵臨揚州城下。四月廿五,揚州城陷,史可法不屈遇害。隨後,清軍渡過長江,攻克鎮江。弘光帝出奔蕪湖。五月十五众大臣獻南京投降清兵;五月廿二弘光帝被虜獲,送往北京處死,弘光帝在位仅一年,即覆滅。

南京失陷後,又有杭州的潞王朱常淓(1645年)、金陵的崇禎太子朱慈烺(可能是貌似太子的王之明。1645年)、撫州的益王朱慈炲(1645年)、福州的唐王朱聿鍵(1645-1646年)、紹興的魯王朱以海(1645-1653年)、桂林的靖江王朱亨嘉(1645年)等監國政權先後建立,其中唐王朱聿鍵受鄭芝龍等人在福州擁立,登極稱帝,改元隆武,是為明紹宗。這時清朝再次宣佈薙髮令,江南一帶掀起了反薙髮的抗清鬥爭,清軍後方發生動亂,一時無力繼續南進。但南明內部嚴重的黨派鬥爭與地方勢力跋扈自雄,且隆武帝與魯王政權不但沒有利用這種有利形勢,發展抗清鬥爭,反而在自己之間為爭正統地位而形同水火,各自為戰,所以當1646年清軍再度南下時,先後為清軍所各各擊滅。魯王在張煌言等保護下逃亡海上,在沿海一帶繼續抗清;隆武帝則被清軍俘殺。

11月,在廣州和肇慶又成立了兩個南明政權:隆武帝之弟唐王朱聿鐭(1646年)繼位於廣州,改明年为紹武元年;桂王朱由榔(1646-1662年)稱帝於肇慶,改元永曆,是為明昭宗。紹武、永曆二帝也不能團結,甚至大動干戈,互相攻伐。紹武政權僅存在40天就被清軍消滅。揭陽的益王朱由榛(1647年)、夔州的楚王朱容藩(1649年)稱監國與永曆帝爭立。鄭成功也在南澳一度立淮王朱常清(1648年)為監國,後廢。永曆帝在清軍進逼下逃入廣西。

正當南明政權一個接一個地覆亡,形勢萬分危急之際,大順農民軍餘部出現在抗清鬥爭最前線,挽救了危局。自李自成于1645年战死于九宫山後,他的餘部分為二支,分別由郝搖旗、劉體純和李過、高一功率領,先後進入湖南,與明湖廣總督何騰蛟、湖北巡撫堵胤錫聯合抗清。1647年,郝搖旗部護衛逃來廣西的永曆帝居柳州,並出擊桂林。年底,大敗清軍於全州,進入湖南。次年,大順軍餘部又同何騰蛟、瞿式耜的部隊一起,在湖南連連取得勝利,幾乎收復了湖南全境。這時,廣東、四川等地的抗清鬥爭再起,清江西提督金聲桓、清广东提督李成棟、清广西巡抚耿献忠、清大同总兵姜镶、清延安营参将王永强、清甘州副将米喇印先後反正回归明朝,清軍後方的抗清力量也發動了廣泛的攻勢。一時間,永曆政權名义控制的區域擴大到了雲南、貴州、廣東、廣西、湖南、江西、四川七省,还包括北方山西、陕西、甘肃三省一部以及东南福建和浙江两省的沿海岛屿,出現了南明時期第一次抗清鬥爭的高潮。

但永曆政權內部仍然矛盾重重,各派政治勢力互相攻訐,農民軍也倍受排擠打擊,不能團結對敵,這就給了清軍以喘息之機。1649-1650年,何騰蛟、瞿式耜先後在湘潭、桂林的戰役中被俘杀,清軍重新佔領湖南、廣西;其他剛剛收復的失地也相繼丟掉了。不久,李過之子李來亨等農民軍將領率部脫離南明政府,轉移到巴東荊襄地區組成夔東十三家軍,獨立抗清。這支部隊一直堅持到1664年。

綜觀1645-1651年間,南明軍與清軍作戰中,敗多勝少,大批南明的軍隊先後降清。先後丟失了江蘇、安徽、浙江、江西、福建、兩廣、兩湖等等領地,地盤盡失。直到以孫可望為主的大西軍加入,再次改變了整個局勢。

張獻忠于1646年战死後,以其义子孫可望、李定國、刘文秀、艾能奇等人為主的大西軍残部自1647年進佔雲南、貴州二省。1652年,南明永曆政权接受孫可望和李定國的建议聯合抗清建議,定都安龙。不久,以大西军餘部为主体的南明軍對清軍展開了全面反擊。李定國率軍8萬東出湖南,取得靖州大捷,收复湖南大部;随后南下广西,取得桂林大捷,击毙清定南王孔有德,收复广西全省;然后又北上湖南取得衡阳大捷,击毙清敬谨亲王尼堪,天下震动。同時,劉文秀亦出擊四川,取得叙州大捷、停溪大捷,克復川南、川东。孙可望也亲自率军在湖南取得辰州大捷。東南沿海的張煌言、郑成功等的抗清軍隊也乘机發動攻勢,接连取得磁灶大捷、钱山大捷、小盈岭大捷、江东桥大捷、崇武大捷、海澄大捷的一连串胜利,並接受了永曆封號。一時間,永曆政權名义控制的區域恢复到了雲南、貴州、廣西三省全部,湖南、四川两省大部,廣東、江西、福建、湖北四省一部,出現了南明時期第二次抗清鬥爭的高潮。

之后,劉文秀於四川用兵失利,在保宁战役中被吳三桂侥幸取胜。而孫可望妒嫉李定國桂林、衡州大捷之大功,逼走李定國,自己统兵却在宝庆战役中失利。东南沿海的郑成功也在漳州战役中失利。所以明军在四川、湖南、福建三个战场上没能扩大战果,陷入了与清军相持的局面。之後李定國與鄭成功聯絡,於1653年、1654年率軍兩次進軍廣東,約定与鄭會師廣州,一舉收復廣東,但鄭軍屢誤約期,加上瘟疫流行,导致肇庆战役和新会战役没能成功。但郑成功部队并没有闲着,1656年,郑军取得泉州大捷,1657年又取得护国岭大捷。

永曆十年(1656年),孫可望祕謀篡位,引發了南明內部一場内讧,李定國擁永曆帝至雲南,次年大敗孫可望,孫可望勢窮降清。孫可望降清後,西南軍事情報盡供清廷,雲貴虛實盡為清軍所知。永曆十二年(1658年)四月,清軍主力從湖南、四川、廣西三路進攻貴州。年底吳三桂攻入雲南,次年正月,下昆明,進入雲南,永曆帝狼狽西奔,進入緬甸(東吁王朝)。李定國率全軍設伏於磨盤山,企圖一舉殲滅敵人追兵,結果因內奸洩密导致未能大获全胜,南明軍精銳損失殆盡,此即磨盘山血战。這時鄭成功趁清軍主力大舉攻擊西南之際,率領十餘萬大軍北伐,接连取得定海关大捷、瓜州大捷、镇江大捷的胜利,一度兵临南京城下,然而鄭軍中了清軍緩兵之計,最终失败,撤回廈門。清军派大军围攻厦门,企图一举歼灭郑成功,但郑成功沉着应战,取得厦门大捷的胜利,稳定了东南沿海局势。永曆十五年(1661年),吳三桂率清軍入緬,索求永曆帝,十二月緬甸東吁王朝國王平達力(莽達)將永曆交予清軍,次年四月永曆帝與其子哀愍太子朱慈煊等被吳三桂處死于昆明。七月,李定國在真臘得知永曆帝死訊,亦憂憤而死。而同年五月,鄭成功亦於臺灣急病而亡。

此后郑氏政权未再拥立皇帝或朱氏监国,而是继续奉永历为正朔。1683年,延平郡王鄭克塽降清,清军占领台湾,宁靖王朱术桂自杀殉国,标志着大明最后一个政权的覆灭。

南明時期,安南、日本、琉球、呂宋、占城也曾派使者入貢[10]。隆武元年也曾頒登基詔書予琉球,並記載於琉球《歷代寶案》一書。

南明弘光帝曾以對等的禮儀派使者左懋第詔諭,並稱順治帝為清國可汗。在詔書中,弘光帝提出四件事:要安葬崇禎帝及崇禎皇后、以山海關為界,關外土地給予清朝、每年十萬歲幣,並「犒金千兩、銀十萬兩、絲緞萬匹、犒銀三萬兩」、建國任便。[10]意圖令南明和清朝共存,通好議和。不過左懋第到北京被囚,使事失敗。

\subsection{安宗\tiny(1644-1645)}

\subsubsection{生平}

明安宗朱由崧(1607年9月5日-1646年7月1日),又稱弘光帝,為南明首位皇帝,原為福王。朱由崧是明神宗朱翊钧之孙,福忠王朱常洵之子。他是明熹宗朱由校、明思宗朱由檢的堂兄弟。思宗殉国後,朱由崧在南京即位,改元弘光,在位僅一年。弘光元年清军南攻,朱由崧被俘,押往北京,翌年被處決。南明永历帝为其上庙号安宗,谥号奉天遵道宽和静穆修文布武溫恭仁孝簡皇帝。

朱由崧小字福八,明神宗孙,福忠王朱常洵庶长子。万历三十五年七月乙巳生于福王京邸,生母姚氏。万历四十二年随福王朱常洵就藩于洛阳。万历四十八年七月甲辰封德昌王,后进封福王世子。

崇祯十四年正月,流賊李自成陷洛阳,福王常洵缒城出,藏匿于迎恩寺,后被搜出,遇害。朱由崧缒城逃脱,前往怀庆避难,崇祯十六年五月袭封福王。崇祯帝手择宫中玉带,遣内使赐之。

崇祯十七年正月,怀庆闻警,朱由崧逃亡卫辉,投奔潞王朱常淓。三月初四卫辉闻警,朱由崧随潞王逃往淮安,与南逃的周王、崇王一同寓居于湖嘴舟中。三月十一日周王朱恭枵薨于舟上,三月十八日福王上岸,住在杜光绍园中。三月十九日李自成陷北京,崇禎帝自縊,是為甲申之變。廿九日,消息传至淮安。

四月崇祯帝自盡的消息,传至南京,北京沦陷後,南京以及南方各省仍在明朝的控制之下。南京诸臣皆認為國不可一日無君,议立新帝。但對大寶誰屬,則有一番論戰。

从血统上来说,崇祯帝殉国,其子太子朱慈烺及永王朱慈炤、定王朱慈炯陷入清军之手,而崇禎帝父明光宗朱常洛仅有天啟帝、崇祯帝二子,天啟帝無子,而故应从崇禎帝祖父明神宗之子、光宗诸弟中选择。明神宗福王常洵为第三子,瑞王常浩为第五子,惠王常润为第六子,桂王常瀛为第七子,以常洵居长。朱由崧为朱常洵长子,因此在崇祯太子及定、永二王无法至南京继位的情况下,福王本为第一順位。然而東林黨人卻持相反意見,他們恐朱由崧即位后追究昔日“三案”及國本之爭攻讦郑贵妃(朱由崧祖母)之事,主张立明神宗之侄潞王朱常淓。史可法并称福王“在藩不忠不孝,恐难主天下”。四月二十六日,张慎言、高弘图、姜曰广、李沾、郭维经、诚意伯刘孔昭、司礼太监韩赞周等在朝中会议,李沾、刘孔昭、韩赞周议立福王,议遂定以福王继统,告庙并修武英殿。鳳陽總督馬士英與江北四鎮黃得功、高傑、劉良佐、劉澤清等人前往淮安迎接朱由崧。四月二十七日甲申,南京礼部率百司迎福王于儀真。

崇祯十七年四月二十八日乙酉,朱由崧至浦口,魏国公徐弘基等渡江迎接。翌日舟泊观音门燕子矶。四月三十日丁亥,南京百官迎见朱由崧于龙江关舟中,请其為監國。朱由崧身穿角巾葛衣,坐于卧榻之上,推说自己未携宫眷一人,准备避难浙东。众臣力劝,朱由崧乃同意。

五月初一戊子,朱由崧骑马自三山门环城而东,拜谒孝陵和懿文太子陵,随后经朝阳门入东华门,谒奉先殿,出西华门,以南京内守备府为行宫。五月初二群臣至行宫劝进,朱由崧以太子及定王、永王不知下落,且瑞王、惠王、桂王均为叔父行,应择贤迎立。诸臣再三劝进,乃依明代宗故事监国。五月初三庚寅自大明门入大内,至武英殿行监国礼。是日吴三桂引清摄政王多尔衮入北京。

崇祯十七年五月十五日壬寅,朱由崧即皇帝位于武英殿,以次年为弘光元年。其国号依旧为“大明”,史称“南明”。

朱由崧即位后,于六月戊午追封祖母郑贵妃为孝宁太皇太后,父福忠王朱常洵为贞纯肃哲圣敬仁毅恭皇帝(后改谥孝皇帝),立庙于南京,墓园称熙陵。上嫡母邹氏尊号为恪贞仁寿皇太后,生母姚氏为孝诚端惠慈顺贞穆皇太后。追封洛阳城陷时遇害的胞弟颍上王朱由榘为颍王,谥曰冲。六月辛酉上崇祯帝庙号为思宗,谥号烈皇帝。七月己丑追复懿文太子帝号,追崇建文帝、景泰帝庙号谥号。

东林党人编撰的史书说朱由崧生性暗弱,不忠不孝,荒淫无耻,政事则悉委于马士英、阮大铖。马、阮二人日以卖官鬻爵、报撼私仇为事,导致南明政事萎靡,不断发生内讧;而名臣李清则力为弘光辩冤,说这些记载都是谣言,又说弘光帝很少接近女色。在外以史可法督师江北,设淮、扬、凤、庐四镇,以黄得功、刘良佐、刘泽清、高杰为总兵统领,南明出现军阀化的趋势。前線將領不但因爭權而互相攻擊,也有掠奪平民的行為。

朱由崧即位后,下令选淑女入宫,派宦官于南京城中四出搜巷,凡是有女之家,必以黄纸贴额,持之而去,南京城中骚动。朱由崧又下令修西宫西一路为慈禧殿,以安置继母邹太后。当年八月邹太后自河南至南京,八月十四日谕户、兵、工三部“太后光临,限三日内搜括万金,以备赏赐”。八月十六日御用监又令造龙凤床座、床顶架、宫殿陈设金玉等项,越数十万两。造皇后冠,命内臣采购猫眼石、祖母绿及大珠重一钱以上者百余颗。崇祯十七年除夕,弘光帝独坐兴宁宫中,愀然不乐。太监韩赞周问道:“宫殿新落成,皇上应当欢喜,而闷闷不乐,是思念皇兄吗?”弘光帝不应,继而回答说:“梨园殊少佳者”。弘光元年(1645年)正月,弘光帝又下令修南京奉先殿、午门及左右掖门,并派太监田成至杭州、嘉兴二府选淑女。

崇祯十七年九月初三,弘光帝下令为北京殉难诸臣上谥号,计文臣二十一人、勋臣二人、戚臣一人。随后又给郢国公冯国用、宋国公冯胜、济国公丁德兴、德庆侯廖永忠、长兴侯耿炳文等开国功臣追上谥号;给方孝孺、齐泰、黄子澄、陈迪、景清、卓敬、练子宁等建文朝死难诸臣,蒋钦、陆震等正德朝死谏诸臣,左光斗、周朝瑞、周宗建、袁化中、顾大章、周起元等天启朝死珰难诸臣上谥号。

弘光元年三月初一甲申,有自称崇祯太子朱慈烺者至南京,朱由崧命令将其关入兵马司监狱,后命百官审北来太子于午门外,终裁断为伪太子王之明,是為崇禎太子案。三月庚申,宁南侯左良玉乃举兵于武昌,以“救太子、诛士英”为名顺流而下,黄得功、阮大铖率兵御之,南明发生内讧。正值此时,清军在豫王多铎率领下大举南下,攻陷归德、颍州、太和、泗州等地。

弘光元年四月辛未,清军围攻江北重镇扬州。督師江北的兵部尚書史可法率城中百姓抵御清军,清军围困百日,损失惨重。史可法急忙向朝廷求援,但卻因為鎮將們個個擁兵自重、意圖觀望,最終揚州在被围五天后沦陷。清军攻破扬州之後进行了十天屠杀,史称“扬州十日”。四月甲子,弘光帝在南京贡院选淑女,七十人中选中一人,即阮大铖的侄女。四月壬戌,杭州送来淑女五十人,弘光帝选中周姓一人,王姓一人。

弘光元年五月初八己丑,清军自瓜洲渡江,镇江巡抚杨文骢逃奔苏州,靖虏伯郑鸿逵逃入東海,总兵蒋云台投降。南京闭城门。五月初十辛卯,朱由崧传旨放归所选淑女,当天午夜尤召梨园入宫演剧。翌日凌晨二漏时,朱由崧率内官四五十人骑马出通济门,莫知所踪。天亮后百官入朝,见宫女、内臣、优伶杂沓逃奔西华门外,方知弘光帝已出逃。南京城内大哗,马士英携邹太后出奔,市民救北来太子出狱,扶其入宫,在武英殿即位。五月十二日癸巳,朱由崧至太平府,以按察院为行宫,寻即移驾芜湖,投奔靖国公黄得功军营。五月十五日丙申,清军入南京,魏国公徐文爵、保国公朱国弼、灵璧侯汤国祚、定远侯邓文郁,及尚书钱谦益、大学士王铎、都御史唐世济等人剃髮降清。

清军攻克南京后,多铎命降将刘良佐带清兵追击弘光帝。五月二十二日癸卯,总兵田雄、马得功、丘钺、张杰、黄名、陈献策冲上御舟,劫持弘光帝,将其献给清军。豫王多铎命去锁链,以红绳捆绑。五月二十五日丙午,朱由崧乘无幔小轿入南京聚宝门,头蒙缁素帕,身衣蓝布袍,以油扇掩面,两妃乘驴随后,夹路百姓唾骂,有投瓦砾者。多铎在灵璧侯府设宴,命朱由崧居于北来太子之下。宴罢,拘弘光帝于江宁县署。

弘光元年闰六月,唐王朱聿鍵即位于福州,改元隆武,遥上朱由崧尊号为「上皇圣安皇帝」。当年九月甲寅,朱由崧与皇太后邹氏、潞王朱常淓等人被押送至燕京,安置居住。由滿清太医院,日时馈宴,朱由崧酣饮极乐。

顺治三年(1646年,隆武二年)四月九日,有人向清摄政王多尔衮告发,称燕京居住的故明衡王、荆王欲谋反。五月甲子,弘光帝与秦王朱存極、晋王朱審烜、潞王朱常淓、荆王朱慈煃、徳王朱由栎、衡王朱由棷等十七人被斬首於菜市口(一说弘光帝以弓弦絞死)。

朱由崧王妃黄氏之弟黄調鼎购得棺木,与黄妃合葬于河南孟津县东山头村。

弘光帝凶讯南传后,监国魯王朱以海上谥号为赧皇帝,不久又上庙谥为质宗安皇帝。永曆帝立,于永历十一年四月改弘光帝廟號曰安宗,谥号奉天遵道宽和静穆修文布武溫恭仁孝簡皇帝。

根据明末清初笔记记载,朱由崧是个十分昏庸腐朽的君主,整日只知吃喝玩乐,沉湎于酒色之中,不理朝政。在其即位之前,史可法曾寫信給馬士英說明「福王七不可立」──貪、淫、酗酒、不孝、虐下、無知和專橫。由史可法、張慎言、高弘圖等17人簽名送與馬士英。後人称其为明朝及南明最昏庸的帝王,唯知享樂,不問政事,沉湎酒色,荒淫透頂。然而細檢史籍可知竟傳聞難據,推其緣由多為東林黨人因國本之爭對福王藩一系的成見所致。而其本來的經歷顯現的是並非昏庸且頗有個性的政治家形象。如曾任弘光朝給事中李清《三垣筆記》、《南渡錄》及《甲申日記》對荒淫縱欲之事,且加辯誣。此外,朱由崧替靖難之變殉難的明惠帝一系君臣予以平反,並貶抑當時擴大迫害的陳瑛。因此其政治得失尚有爭議。

钱海岳《南明史》评价弘光帝“北京颠覆,上膺鼎籙,丰芑奠磐,徵用俊耆。卷阿翙羽,相得益彰。故初政有客观者。性素宽厚,马、阮欲以《三朝要典》起大狱,屡请不允。观其谕解良玉,委任继咸,词婉处当;拒纳银赎罪之议,禁武臣罔利之非,皆非武、熹昏騃之比。顾少读书,章奏未能亲裁,政事一出士英,不从中制,坐是狐鸣虎噬,咆哮恣睢,纪纲倒持。及大铖得志,众正去朝,罗罻高张,党祸益烈。上燕居神功,辄顿足谓士英误我,而太阿旁落,无可如何,遂日饮火酒,亲伶官优人为乐,卒至触蛮之争,清收渔利。时未一朞,柱折维缺。故虽遗爱足以感其遗民,而卒不能保社稷云。”

\subsubsection{弘光}

\begin{longtable}{|>{\centering\scriptsize}m{2em}|>{\centering\scriptsize}m{1.3em}|>{\centering}m{8.8em}|}
  % \caption{秦王政}\
  \toprule
  \SimHei \normalsize 年数 & \SimHei \scriptsize 公元 & \SimHei 大事件 \tabularnewline
  % \midrule
  \endfirsthead
  \toprule
  \SimHei \normalsize 年数 & \SimHei \scriptsize 公元 & \SimHei 大事件 \tabularnewline
  \midrule
  \endhead
  \midrule
  元年 & 1645 & \tabularnewline
  \bottomrule
\end{longtable}

\subsection{绍宗\tiny(1645-1646)}

\subsubsection{生平}

明紹宗朱聿鍵(1602年5月25日-1646年10月6日),又稱隆武帝,小字長壽,南明第二代皇帝,原為唐王,為明太祖朱元璋二十三子唐王朱桱的八世孫(與明神宗同輩份),祖父唐端王朱碩熿,父為唐王之子朱器墭,母宣皇后毛氏。1644年,明思宗在北京自缢,1645年弘光帝被俘,鄭芝龍、黃道周等人扶朱聿鍵於福州登基称帝,改元為隆武並與同年開鑄「隆武通寶」,而弘光帝在翌年才被清廷所殺。

1646年,清军入福建,隆武帝在汀州被擄殺,享年44岁。永曆帝即位后初上尊谥思文皇帝,永历十一年上廟號紹宗,改谥号為配天至道弘毅肅穆思文烈武敏仁廣孝襄皇帝。朱聿键自奉甚俭,品格在南明诸君中是少見的優良。黄道周描述了隆武帝的为人:“今上不饮酒,精吏事,洞达古今,想亦高、光而下之所未见也。”

朱聿键为明太祖第二十三子唐定王朱桱的后裔,系太祖九世孙。万历三十四年四月丙申生于南阳唐王府,母妃毛氏。其祖父唐端王朱碩熿惑于嬖妾,不喜愛朱聿键的父親世子朱器墭,把朱器墭父子一起囚禁在承奉司內,欲立爱子。崇祯二年(1629年),朱器墭疑似被其弟福山王朱器塽、安陽王朱器埈毒死,朱碩熿讳言其事,但经守道陈奇瑜奏请,朱聿鍵被明廷立为唐國世孙,不再被囚禁,同年朱碩熿也去世。

崇禎五年(1632年)朱聿鍵繼為唐王,封地南阳。崇祯帝赐其《皇明祖训》、《大明会典》、《四书》、《五经》、《二十一史》、《資治通鑒綱目》、《孝经》、《忠經》等书。朱聿鍵在王府内起高明楼,延请四方名士。

崇祯九年(1636年)七月初一,朱聿鍵杖殺叔父福山王朱器塽、杖傷叔父安阳王朱器埈,为其父朱器墭当年被毒死一事报仇。当年八月,清兵入塞,克宝坻,直逼北京,京师戒严。朱聿鍵上疏请勤王,不许,乃自率护军千人北上勤王。行至裕州,巡抚杨绳武上奏,崇祯帝勒令其返回,后朱聿键因与农民军相遇交锋,两名太监被杀,乃班师回南阳。冬十一月下部议,废为庶人,幽禁在凤阳之高墙。崇禎帝改封其弟朱聿鏼为唐王。

朱聿键高墙圈禁期间,凤阳守陵太监石应诏索贿不得,用墩锁之法折磨之,朱聿键病苦几殆。后凤阳巡抚路振飞入高墙见之,向崇祯帝上疏陈高墙监吏凌虐宗室之状,请加恩于宗室。乃下旨誅殺石应诏。

崇禎十四年(1641年),李自成攻陷南阳,杀死朱聿鏼。

崇禎十七年(1644年),李自成攻陷北京,即甲申之變,崇祯帝自缢,南京諸臣拥从洛阳逃出的福王子朱由崧为帝,在南京即位,改年號弘光,实行大赦。在广昌伯刘良佐奏请下,囚於鳳陽的朱聿键也被释,并改封为南阳王。南京礼部请恢复唐王故爵,朱由崧不允,并令朱聿鍵迁至广西平乐(今桂林南),但朱聿鍵贫病不能行。

清朝順治二年、南明弘光元年(1645年)五月,朱聿鍵赴平乐途中,在苏州闻清军已破南京,俘虜了弘光帝朱由崧,朱聿鍵遂至嘉兴避难。六月辛酉,朱聿键至杭州,遇潞王朱常淓,奏请其监国,不听;请朝陈方略,不允。当时鎮江總兵官鄭鴻逵、戶部郎中蘇觀生至杭州,与朱聿键谈及国难,泣下沾襟。后朱聿键被郑鸿逵护送,前往福建。途中在浙江衢州闻得潞王朱常淓已在杭州降清,于是南安伯鄭芝龍、巡撫都御史張肯堂與禮部尚書黃道周等商议奉朱聿鍵为監國。

弘光元年六月己卯(二十八日),朱聿鍵在福建建宁,以唐王的身分监国。闰六月丁亥(初七)至福州,以南安伯府为行宫。

闰六月丁未,朱聿鍵於福州称帝,遙尊朱由崧為「上皇聖安皇帝」,宣布從七月初一起,改弘光年号為隆武元年,改福建布政司称福京行在,改福州府為天興府,改布政司为行殿,建行在太庙、社稷及唐国宗庙。升鄭芝龍为平虏侯、鄭鴻逵為定虏侯,封鄭芝豹为澄济伯、鄭彩為永胜伯。以何吾驺为首辅,以黄道周为吏部尚书、武英殿大学士,蒋德璟为户部尚书、文渊阁大学士,朱继祚为礼部尚书、东阁大学士,曾樱为工部尚书、东阁大学士,黄鸣俊、李光春、蘇觀生等人为礼、兵各部左右侍郎兼东阁大学士。

朱聿键即帝位后,上高曾祖父四代帝号,高祖唐敬王朱宇温为惠皇帝,曾祖唐顺王朱宙栐为顺皇帝,祖父唐端王朱碩熿为端皇帝,父唐裕王(追封)朱器墭为宣皇帝。四代祖妣皆追封皇后。封弟朱聿𨮁为唐王,封国南宁;升叔德安王朱器䵺为邓王;追封弟朱聿𨧨为陈王,子朱琳渼为陈王世子。遥上弘光帝尊号“圣安皇帝”。隆武元年七月下令将嘉靖年间皇极殿、中极殿、建极殿三殿之名恢复为奉天殿、华盖殿、谨身殿,各衙门前加“行在”二字。

当时,在绍兴还有鲁王朱以海建立的小朝廷,亦自稱「監國」。清军攻绍兴,朱以海派使者前来福州向朱聿鍵求援兵。信上称朱聿键为“皇伯父”,而未称“陛下”,朱聿键怒,令杀鲁王信使。

隆武二年/清顺治三年(1646年)五月,清将博洛贝勒率兵征浙、闽。七月庚申清兵陷金华,八月甲申陷建宁,乙未过仙霞关,武毅伯施天福、武功伯陈秀、靖安伯郭熺降清。郑芝龙向清軍投降,隆武政权很快灭亡。楊鳳苞稱“福京之亡,亡于鄭芝龍之通款”。

隆武二年八月甲午,隆武帝率宫嫔自延平出狩,欲逃往江西避難。八月庚申至汀州,以府署为行宫。八月辛丑五鼓,有清军八十三骑伪装成扈跸者叩城,守城者开汀州丽春门。骑兵突袭行宫,杀福清伯周之藩、总兵王凉武等人。时隆武帝腹饥,命内官市二汤圆以进,方举箸,清兵发矢,隆武帝后背中箭,崩,年四十五。百姓敛葬于罗汉岭。另有说法称隆武帝被俘后不食而死,或称崩于福京天兴府,或称崩于建宁。

八月壬戌福京天兴府陷落,阳曲王朱敏渡、松滋王朱俨𨫃、翼城王朱弘橺、奉新王朱常涟遇害。十月辛卯漳州陷落。十一月,侍郎蘇觀生立隆武帝之弟朱聿𨮁於廣東省廣州府番禺縣,改元紹武,觀生自為宰相。當時已經稱帝的永曆帝,希望紹武帝取消帝號,蘇觀生大怒,以新歸降的海盜加上四處捕捉來的民兵征討永曆,大勝,誰知滿清將領佟養甲、李成棟已取潮州、惠州,兵臨廣州,蘇觀生死於戰事,清兵隨即俘獲了紹武帝,紹武自縊。

永曆帝即位后,一直聽到謠言說隆武帝化妝隱居不出,上尊号「上皇思文皇帝」,遣間諜打聽隆武帝消息,傳言隆武帝潜至安溪縣妙峯为僧,或称在汀州府单骑逃出,藏于乡民蒋氏家中,清兵離開以後,前往大帽山出家。永历五年曾遣侍郎王命璿探訪,又不得,永历十一年乃确信隆武帝已死,立廟號绍宗,諡號配天至道弘毅肃穆思文烈武敏仁广孝襄皇帝。

隆武帝死后百姓敛葬于福州罗汉岭,一说葬于汀州。

\subsubsection{隆武}

\begin{longtable}{|>{\centering\scriptsize}m{2em}|>{\centering\scriptsize}m{1.3em}|>{\centering}m{8.8em}|}
  % \caption{秦王政}\
  \toprule
  \SimHei \normalsize 年数 & \SimHei \scriptsize 公元 & \SimHei 大事件 \tabularnewline
  % \midrule
  \endfirsthead
  \toprule
  \SimHei \normalsize 年数 & \SimHei \scriptsize 公元 & \SimHei 大事件 \tabularnewline
  \midrule
  \endhead
  \midrule
  元年 & 1645 & \tabularnewline\hline
  二年 & 1646 & \tabularnewline
  \bottomrule
\end{longtable}

\subsection{绍武帝\tiny(1646-1647)}

\subsubsection{生平}

明紹武帝朱聿{\fzk 𨮁}(1605年-1647年1月20日),年號紹武。1646年—1647年在位,南明第三任君主。朱聿𨮁又稱小唐王,是明绍宗(唐王)之弟,明太祖二十三子唐定王朱桱的八世孙,祖父唐端王朱碩熿,父為唐王之子朱器墭。

明紹宗即位後封朱聿{\fzk 𨮁}為唐王,主祀唐國,幾天後紹宗出征,留他和邓王朱器䵺監國。

1646年(隆武二年),南明重臣郑芝龙拒不发兵,以致清軍隊长驱直入福京,並於长汀俘虜明紹宗,紹宗殉國,時為唐王的朱聿{\fzk 𨮁}和隆武朝的宫员逃到廣東省廣州府番禺縣,而其他南明勢力則在肇庆府推举明神宗之孙、明思宗堂弟桂王朱由榔为监国。同年十月十六日,江西赣州失守后,朱由榔政權大驚,于十月二十一仓皇从肇庆逃往广西梧州,置廣東全省於不顧。於是,大学士苏观生,在廣東權力真空與一眾明朝藩王已由海路到達广州的情況之下,聯同大学士何吾驺、广东布政使顾元镜,侍郎王应华、曾道唯等拥立朱聿{\fzk 𨮁}为监国,以都司署为行宫。隆武二年十一月五日,四十一歲的朱聿{\fzk 𨮁}按兄終弟及的皇明祖訓,繼位称帝,以明年为绍武元年。苏观生因拥戴有功,被命为首輔,封建明伯,掌兵部。由於朱聿{\fzk 𨮁}仓促稱帝,登極時的龍袍與百官官服都要假借于粵劇伶人的戏服。

十一月初八,紹武称帝的消息传到梧州,朱由榔政權大驚大怒,四日後回到肇庆,再於十八日登極稱帝,改元永曆,是為明昭宗。永曆帝立刻派遣兵科给事中彭耀、兵部主事陈嘉谟前往广州,拜见紹武帝,稱其為「殿下」,規勸其取消帝号。首輔苏观生大怒,以大不敬斬彭、陈二人,再令陈际泰督师攻打肇庆。永曆帝派兵部右侍郎林佳鼎、夏四敷率兵,在十一月二十九日於三水县城西,與紹武軍展開內戰,並將對方擊退。苏观生再令广东总兵林察聯同新降的海盗等数万人反擊,並且大敗永曆軍隊。大捷消息传到广州,苏观生下令广州张灯结彩粉饰太平。正当紹武、永曆二帝自相殘殺之時,由佟养甲、李成栋率领的清兵已取潮州、惠州,臨近广州附近,並用缴获的南明地方官印,向紹武帝发出太平的錯誤信息。

十二月十五日,绍武帝幸武学,百官聚集,而此時,清兵已经偷偷兵臨城下,内应脫去头上的伪装,露出辫子。有人向苏观生報告,反遭斬首。苏观生说:“潮州昨尚有报,安得遽至此。妄言惑众,斩之!”不久,清军壓境的戰況得到證實,苏观生遂率領部隊与清兵激战一晝夜,清兵本有撤退之意,但內奸谢尚政旋引清兵入城,广州即陷落。苏观生見大勢已去,写下“大明忠臣义固当死”八个大字后,自縊死亡。已易服的紹武帝,打算爬城墙逃走,但被追骑赶上抓获,囚于东察院。李成栋派人送来饮食,紹武帝拒絕,說:“我若饮汝一勺水,何以见先人地下!”後自缢而殉國,結束其四十日的統治。绍武朝的主要官員如何吾驺、王应华、顾元镜等降清,而广州內的二十四個明朝藩王則全數被殺。紹武帝死後,永曆帝成為南明唯一的皇帝。

後人將紹武、蘇觀生等十五人,葬於廣州城北象岗山北麓。1954年因基建,迁葬于越秀公园木壳岗;1981年再迁葬于公园南秀湖畔。墓坐东向西,封土呈覆竹形,正面竖墓碑,中刻“明绍武君臣冢”,上款为“光绪癸未(1883年)孟冬吉旦”,下款为“粤东绅士重修”。1963年3月广州市政府公布为市级文物保护单位。

\subsubsection{绍武}

\begin{longtable}{|>{\centering\scriptsize}m{2em}|>{\centering\scriptsize}m{1.3em}|>{\centering}m{8.8em}|}
  % \caption{秦王政}\
  \toprule
  \SimHei \normalsize 年数 & \SimHei \scriptsize 公元 & \SimHei 大事件 \tabularnewline
  % \midrule
  \endfirsthead
  \toprule
  \SimHei \normalsize 年数 & \SimHei \scriptsize 公元 & \SimHei 大事件 \tabularnewline
  \midrule
  \endhead
  \midrule
  元年 & 1646 & \tabularnewline
  \bottomrule
\end{longtable}


\subsection{昭帝\tiny(1646-1662)}

\subsubsection{生平}

明昭宗朱由榔(1623年11月1日-1662年6月1日),或又稱永曆帝,南明第四位也是最後一位皇帝(1646年12月24日-1662年6月1日在位)。原為桂王。

1646年隆武帝被俘死,本為桂王的朱由榔自稱監國。不久,隆武帝弟唐王朱聿{\fzk 𨮁}在廣東廣州繼位,以次年為紹武元年,是為紹武帝。數日後,朱由榔在廣東肇庆亦登基稱帝,年號永曆。紹武、永曆二帝為爭正統,隨即開戰,後永曆軍大敗。1647年,清軍攻陷廣州,紹武帝兵敗殉國,永曆帝自此成為南明唯一的統治者。1659年,清军攻陷昆明后流亡缅甸東吁王朝,永曆十五年(1661年)夏历十二月初三日被送交吴三桂,永曆十六年四月十五日(1662年6月1日)遭縊死。死後,台灣的明鄭政權仍沿用永曆年號至1683年清朝佔領台灣為止。

朱由榔是明神宗之孙,明思宗堂弟,生於天啟三年(1623年)。崇禎年間封永明王,其父為桂端王朱常瀛,是明神宗第七子,封湖南衡阳,天启七年九月二十六日就藩,弘光元年(1645年)十一月初四日病死於梧州。第三子安仁王朱由𣜬承嗣。隆武帝称帝後不久病重。不久朱由榔被封桂王,在1646年隆武帝被俘後,於当年十月初十(一说十四日)称监国於廣東肇庆。

朱由榔於1646年(清顺治三年)農曆十一月十二日东返肇庆,十八日在肇庆正式稱帝,年号永曆,史称永曆帝。曾道唯、顾元镜、王应华等人都入阁,洪朝钟在十天之内升官三次。

永曆帝在中後期倚仗张献忠之餘部李定国、孙可望等人在西南一带抵抗满清,并且得到包括延平郡王郑成功在内的各反清力量的支持,是为反清的精神领袖和天下共主。1652年,李定国在桂林逼死定南王孔有德,又在衡州斩杀敬谨亲王尼堪,取得大捷,一度收复湖南西部、四川(除了保宁)、廣東(李成栋反正取得全部地区,后来仅保有沿海)、江西(金声桓、王得仁反正)等地。

1660年,清军攻入云南,永曆帝流亡缅甸東吁王朝首都瓦城,獲國王莽達(平達力)收留。後来,吴三桂攻入缅甸,莽達之弟莽白乘机发动兵变,杀死其兄奪位。1661年8月12日,莽白發動咒水之难,杀盡永曆帝侍從近衛。

永历帝得到清军进入缅境的消息后,曾寫信给吴三桂,到1662年1月22日(永历十五年十二月初三),莽白将永曆帝献给吴三桂,南明灭亡。

1662年6月1日(永历十六年四月十五望日,清康熙元年),永曆帝父子及眷属25人在昆明篦子坡遭弓弦勒死,终年40岁。其身亡處時人稱為逼死坡,即今天的昆明市五华区的华山西路,辛亥革命後蔡鍔等人在當地豎立「明永曆帝殉國處」石碑。死后庙号昭宗,谥号應天推道敏毅恭儉經文緯武體仁克孝匡皇帝。

至今未发现永历帝之墓。仅贵州都匀大坪镇有永历帝的衣冠冢。当地扶姓人家说,是他们先人明朝大学士扶纲派人搜集衣冠而葬的,为隐其真,只传是桂王坟,不留碑记。扶纲是因明亡不愿降清而回乡隐居的。帝墓左边是编修涂宏猷的 髮冢,右边是节愍侯邬昌期的衣带冢。民国十年都匀县奉令修史,查实桂王坟乃永历墓,才为其树碑立传,省长任可澄、省志总 陈炬、知县窦全曾都为之写了碑记,碑文“大明永历皇帝陵”几个字,墓碑及碑记是时任四川綦江县县长张瑞徵写的(张系都匀人),还修了些亭阁楹联,帝墓才初显规模。墓高3米、径6米,碑高1.62米,宽0.81米、厚0.13米,碑字阴刻正楷,字笔工整秀丽。涂宏猷和邬昌期二人,是咒水之难42大臣之二,坟比帝坟小得多。“文革”中被盗,帝坟从前到后挖了一个大坑,碑断为两截仰卧坟前土中。1996年都匀市人民政府公布大明永历皇帝陵为市级文物保护单位,着手修复帝陵。坟用青石砌边,水泥勾缝,碑文由书法家芦如平书写,前边加修了上下山的双向百级石阶,供游人参观。

\subsubsection{永历}
\begin{longtable}{|>{\centering\scriptsize}m{2em}|>{\centering\scriptsize}m{1.3em}|>{\centering}m{8.8em}|}
  % \caption{秦王政}\
  \toprule
  \SimHei \normalsize 年数 & \SimHei \scriptsize 公元 & \SimHei 大事件 \tabularnewline
  % \midrule
  \endfirsthead
  \toprule
  \SimHei \normalsize 年数 & \SimHei \scriptsize 公元 & \SimHei 大事件 \tabularnewline
  \midrule
  \endhead
  \midrule
  元年 & 1647 & \tabularnewline\hline
  二年 & 1648 & \tabularnewline\hline
  三年 & 1649 & \tabularnewline\hline
  四年 & 1650 & \tabularnewline\hline
  五年 & 1651 & \tabularnewline\hline
  六年 & 1652 & \tabularnewline\hline
  七年 & 1653 & \tabularnewline\hline
  八年 & 1654 & \tabularnewline\hline
  九年 & 1655 & \tabularnewline\hline
  十年 & 1656 & \tabularnewline\hline
  十一年 & 1657 & \tabularnewline\hline
  十二年 & 1658 & \tabularnewline\hline
  十三年 & 1659 & \tabularnewline\hline
  十四年 & 1660 & \tabularnewline\hline
  十五年 & 1661 & \tabularnewline\hline
  十六年 & 1662 & \tabularnewline\hline
  十七年 & 1663 & \tabularnewline\hline
  十八年 & 1664 & \tabularnewline\hline
  十九年 & 1665 & \tabularnewline\hline
  二十年 & 1666 & \tabularnewline\hline
  二一年 & 1667 & \tabularnewline\hline
  二二年 & 1668 & \tabularnewline\hline
  二三年 & 1669 & \tabularnewline\hline
  二四年 & 1670 & \tabularnewline\hline
  二五年 & 1671 & \tabularnewline\hline
  二六年 & 1672 & \tabularnewline\hline
  二七年 & 1673 & \tabularnewline\hline
  二八年 & 1674 & \tabularnewline\hline
  二九年 & 1675 & \tabularnewline\hline
  三十年 & 1676 & \tabularnewline\hline
  三一年 & 1677 & \tabularnewline\hline
  三二年 & 1678 & \tabularnewline\hline
  三三年 & 1679 & \tabularnewline\hline
  三四年 & 1680 & \tabularnewline\hline
  三五年 & 1681 & \tabularnewline\hline
  三六年 & 1682 & \tabularnewline\hline
  三七年 & 1683 & \tabularnewline
  \bottomrule
\end{longtable}


%%% Local Variables:
%%% mode: latex
%%% TeX-engine: xetex
%%% TeX-master: "../Main"
%%% End:



%%% Local Variables:
%%% mode: latex
%%% TeX-engine: xetex
%%% TeX-master: "../Main"
%%% End:
 % 明
% %% -*- coding: utf-8 -*-
%% Time-stamp: <Chen Wang: 2019-10-15 11:23:21>

\chapter{清\tiny(1636-1912)}

%% -*- coding: utf-8 -*-
%% Time-stamp: <Chen Wang: 2018-07-12 22:07:49>

\section{后金\tiny(1616-1636)}

\subsection{努尔哈赤\tiny(1616-1626)}

\subsubsection{天命}

\begin{longtable}{|>{\centering\scriptsize}m{2em}|>{\centering\scriptsize}m{1.3em}|>{\centering}m{8.8em}|}
  % \caption{秦王政}\
  \toprule
  \SimHei \normalsize 年数 & \SimHei \scriptsize 公元 & \SimHei 大事件 \tabularnewline
  % \midrule
  \endfirsthead
  \toprule
  \SimHei \normalsize 年数 & \SimHei \scriptsize 公元 & \SimHei 大事件 \tabularnewline
  \midrule
  \endhead
  \midrule
  元年 & 1616 & \tabularnewline\hline
  二年 & 1617 & \tabularnewline\hline
  三年 & 1618 & \tabularnewline\hline
  四年 & 1619 & \tabularnewline\hline
  五年 & 1620 & \tabularnewline\hline
  六年 & 1621 & \tabularnewline\hline
  七年 & 1622 & \tabularnewline\hline
  八年 & 1623 & \tabularnewline\hline
  九年 & 1624 & \tabularnewline\hline
  十年 & 1625 & \tabularnewline\hline
  十一年 & 1626 & \tabularnewline
  \bottomrule
\end{longtable}

\subsection{皇太极\tiny(1626-1636)}

\subsubsection{天聪}

\begin{longtable}{|>{\centering\scriptsize}m{2em}|>{\centering\scriptsize}m{1.3em}|>{\centering}m{8.8em}|}
  % \caption{秦王政}\
  \toprule
  \SimHei \normalsize 年数 & \SimHei \scriptsize 公元 & \SimHei 大事件 \tabularnewline
  % \midrule
  \endfirsthead
  \toprule
  \SimHei \normalsize 年数 & \SimHei \scriptsize 公元 & \SimHei 大事件 \tabularnewline
  \midrule
  \endhead
  \midrule
  元年 & 1627 & \tabularnewline\hline
  二年 & 1628 & \tabularnewline\hline
  三年 & 1629 & \tabularnewline\hline
  四年 & 1630 & \tabularnewline\hline
  五年 & 1631 & \tabularnewline\hline
  六年 & 1632 & \tabularnewline\hline
  七年 & 1633 & \tabularnewline\hline
  八年 & 1634 & \tabularnewline\hline
  九年 & 1635 & \tabularnewline\hline
  十年 & 1636 & \tabularnewline
  \bottomrule
\end{longtable}


%%% Local Variables:
%%% mode: latex
%%% TeX-engine: xetex
%%% TeX-master: "../Main"
%%% End:

%% -*- coding: utf-8 -*-
%% Time-stamp: <Chen Wang: 2018-07-12 22:09:47>

\section{太宗\tiny(1626-1643)}

\subsection{崇德}

\begin{longtable}{|>{\centering\scriptsize}m{2em}|>{\centering\scriptsize}m{1.3em}|>{\centering}m{8.8em}|}
  % \caption{秦王政}\
  \toprule
  \SimHei \normalsize 年数 & \SimHei \scriptsize 公元 & \SimHei 大事件 \tabularnewline
  % \midrule
  \endfirsthead
  \toprule
  \SimHei \normalsize 年数 & \SimHei \scriptsize 公元 & \SimHei 大事件 \tabularnewline
  \midrule
  \endhead
  \midrule
  元年 & 1636 & \tabularnewline\hline
  二年 & 1637 & \tabularnewline\hline
  三年 & 1638 & \tabularnewline\hline
  四年 & 1639 & \tabularnewline\hline
  五年 & 1640 & \tabularnewline\hline
  六年 & 1641 & \tabularnewline\hline
  七年 & 1642 & \tabularnewline\hline
  八年 & 1643 & \tabularnewline
  \bottomrule
\end{longtable}


%%% Local Variables:
%%% mode: latex
%%% TeX-engine: xetex
%%% TeX-master: "../Main"
%%% End:

%% -*- coding: utf-8 -*-
%% Time-stamp: <Chen Wang: 2018-07-12 22:14:50>

\section{世祖\tiny(1643-1661)}

\subsection{顺治}

\begin{longtable}{|>{\centering\scriptsize}m{2em}|>{\centering\scriptsize}m{1.3em}|>{\centering}m{8.8em}|}
  % \caption{秦王政}\
  \toprule
  \SimHei \normalsize 年数 & \SimHei \scriptsize 公元 & \SimHei 大事件 \tabularnewline
  % \midrule
  \endfirsthead
  \toprule
  \SimHei \normalsize 年数 & \SimHei \scriptsize 公元 & \SimHei 大事件 \tabularnewline
  \midrule
  \endhead
  \midrule
  元年 & 1644 & \tabularnewline\hline
  二年 & 1645 & \tabularnewline\hline
  三年 & 1646 & \tabularnewline\hline
  四年 & 1647 & \tabularnewline\hline
  五年 & 1648 & \tabularnewline\hline
  六年 & 1649 & \tabularnewline\hline
  七年 & 1650 & \tabularnewline\hline
  八年 & 1651 & \tabularnewline\hline
  九年 & 1652 & \tabularnewline\hline
  十年 & 1653 & \tabularnewline\hline
  十一年 & 1654 & \tabularnewline\hline
  十二年 & 1655 & \tabularnewline\hline
  十三年 & 1656 & \tabularnewline\hline
  十四年 & 1657 & \tabularnewline\hline
  十五年 & 1658 & \tabularnewline\hline
  十六年 & 1659 & \tabularnewline\hline
  十七年 & 1660 & \tabularnewline\hline
  十八年 & 1661 & \tabularnewline
  \bottomrule
\end{longtable}


%%% Local Variables:
%%% mode: latex
%%% TeX-engine: xetex
%%% TeX-master: "../Main"
%%% End:

%% -*- coding: utf-8 -*-
%% Time-stamp: <Chen Wang: 2018-07-12 22:16:11>

\section{圣祖\tiny(1661-1722)}

\subsection{康熙}

\begin{longtable}{|>{\centering\scriptsize}m{2em}|>{\centering\scriptsize}m{1.3em}|>{\centering}m{8.8em}|}
  % \caption{秦王政}\
  \toprule
  \SimHei \normalsize 年数 & \SimHei \scriptsize 公元 & \SimHei 大事件 \tabularnewline
  % \midrule
  \endfirsthead
  \toprule
  \SimHei \normalsize 年数 & \SimHei \scriptsize 公元 & \SimHei 大事件 \tabularnewline
  \midrule
  \endhead
  \midrule
  元年 & 1662 & \tabularnewline\hline
  二年 & 1663 & \tabularnewline\hline
  三年 & 1664 & \tabularnewline\hline
  四年 & 1665 & \tabularnewline\hline
  五年 & 1666 & \tabularnewline\hline
  六年 & 1667 & \tabularnewline\hline
  七年 & 1668 & \tabularnewline\hline
  八年 & 1669 & \tabularnewline\hline
  九年 & 1670 & \tabularnewline\hline
  十年 & 1671 & \tabularnewline\hline
  十一年 & 1672 & \tabularnewline\hline
  十二年 & 1673 & \tabularnewline\hline
  十三年 & 1674 & \tabularnewline\hline
  十四年 & 1675 & \tabularnewline\hline
  十五年 & 1676 & \tabularnewline\hline
  十六年 & 1677 & \tabularnewline\hline
  十七年 & 1678 & \tabularnewline\hline
  十八年 & 1679 & \tabularnewline\hline
  十九年 & 1680 & \tabularnewline\hline
  二十年 & 1681 & \tabularnewline\hline
  二一年 & 1682 & \tabularnewline\hline
  二二年 & 1683 & \tabularnewline\hline
  二三年 & 1684 & \tabularnewline\hline
  二四年 & 1685 & \tabularnewline\hline
  二五年 & 1686 & \tabularnewline\hline
  二六年 & 1687 & \tabularnewline\hline
  二七年 & 1688 & \tabularnewline\hline
  二八年 & 1689 & \tabularnewline\hline
  二九年 & 1690 & \tabularnewline\hline
  三十年 & 1691 & \tabularnewline\hline
  三一年 & 1692 & \tabularnewline\hline
  三二年 & 1693 & \tabularnewline\hline
  三三年 & 1694 & \tabularnewline\hline
  三四年 & 1695 & \tabularnewline\hline
  三五年 & 1696 & \tabularnewline\hline
  三六年 & 1697 & \tabularnewline\hline
  三七年 & 1698 & \tabularnewline\hline
  三八年 & 1699 & \tabularnewline\hline
  三九年 & 1700 & \tabularnewline\hline
  四十年 & 1701 & \tabularnewline\hline
  四一年 & 1702 & \tabularnewline\hline
  四二年 & 1703 & \tabularnewline\hline
  四三年 & 1704 & \tabularnewline\hline
  四四年 & 1705 & \tabularnewline\hline
  四五年 & 1706 & \tabularnewline\hline
  四六年 & 1707 & \tabularnewline\hline
  四七年 & 1708 & \tabularnewline\hline
  四八年 & 1709 & \tabularnewline\hline
  四九年 & 1710 & \tabularnewline\hline
  五十年 & 1711 & \tabularnewline\hline
  五一年 & 1712 & \tabularnewline\hline
  五二年 & 1713 & \tabularnewline\hline
  五三年 & 1714 & \tabularnewline\hline
  五四年 & 1715 & \tabularnewline\hline
  五五年 & 1716 & \tabularnewline\hline
  五六年 & 1717 & \tabularnewline\hline
  五七年 & 1718 & \tabularnewline\hline
  五八年 & 1719 & \tabularnewline\hline
  五九年 & 1720 & \tabularnewline\hline
  六十年 & 1721 & \tabularnewline\hline
  六一年 & 1722 & \tabularnewline
  \bottomrule
\end{longtable}


%%% Local Variables:
%%% mode: latex
%%% TeX-engine: xetex
%%% TeX-master: "../Main"
%%% End:

%% -*- coding: utf-8 -*-
%% Time-stamp: <Chen Wang: 2018-07-12 22:16:54>

\section{世宗\tiny(1722-1735)}

\subsection{雍正}

\begin{longtable}{|>{\centering\scriptsize}m{2em}|>{\centering\scriptsize}m{1.3em}|>{\centering}m{8.8em}|}
  % \caption{秦王政}\
  \toprule
  \SimHei \normalsize 年数 & \SimHei \scriptsize 公元 & \SimHei 大事件 \tabularnewline
  % \midrule
  \endfirsthead
  \toprule
  \SimHei \normalsize 年数 & \SimHei \scriptsize 公元 & \SimHei 大事件 \tabularnewline
  \midrule
  \endhead
  \midrule
  元年 & 1723 & \tabularnewline\hline
  二年 & 1724 & \tabularnewline\hline
  三年 & 1725 & \tabularnewline\hline
  四年 & 1726 & \tabularnewline\hline
  五年 & 1727 & \tabularnewline\hline
  六年 & 1728 & \tabularnewline\hline
  七年 & 1729 & \tabularnewline\hline
  八年 & 1730 & \tabularnewline\hline
  九年 & 1731 & \tabularnewline\hline
  十年 & 1732 & \tabularnewline\hline
  十一年 & 1733 & \tabularnewline\hline
  十二年 & 1734 & \tabularnewline\hline
  十三年 & 1735 & \tabularnewline
  \bottomrule
\end{longtable}


%%% Local Variables:
%%% mode: latex
%%% TeX-engine: xetex
%%% TeX-master: "../Main"
%%% End:

%% -*- coding: utf-8 -*-
%% Time-stamp: <Chen Wang: 2018-07-12 22:17:39>

\section{高宗\tiny(1736-1795)}

\subsection{乾隆}

\begin{longtable}{|>{\centering\scriptsize}m{2em}|>{\centering\scriptsize}m{1.3em}|>{\centering}m{8.8em}|}
  % \caption{秦王政}\
  \toprule
  \SimHei \normalsize 年数 & \SimHei \scriptsize 公元 & \SimHei 大事件 \tabularnewline
  % \midrule
  \endfirsthead
  \toprule
  \SimHei \normalsize 年数 & \SimHei \scriptsize 公元 & \SimHei 大事件 \tabularnewline
  \midrule
  \endhead
  \midrule
  元年 & 1736 & \tabularnewline\hline
  二年 & 1737 & \tabularnewline\hline
  三年 & 1738 & \tabularnewline\hline
  四年 & 1739 & \tabularnewline\hline
  五年 & 1740 & \tabularnewline\hline
  六年 & 1741 & \tabularnewline\hline
  七年 & 1742 & \tabularnewline\hline
  八年 & 1743 & \tabularnewline\hline
  九年 & 1744 & \tabularnewline\hline
  十年 & 1745 & \tabularnewline\hline
  十一年 & 1746 & \tabularnewline\hline
  十二年 & 1747 & \tabularnewline\hline
  十三年 & 1748 & \tabularnewline\hline
  十四年 & 1749 & \tabularnewline\hline
  十五年 & 1750 & \tabularnewline\hline
  十六年 & 1751 & \tabularnewline\hline
  十七年 & 1752 & \tabularnewline\hline
  十八年 & 1753 & \tabularnewline\hline
  十九年 & 1754 & \tabularnewline\hline
  二十年 & 1755 & \tabularnewline\hline
  二一年 & 1756 & \tabularnewline\hline
  二二年 & 1757 & \tabularnewline\hline
  二三年 & 1758 & \tabularnewline\hline
  二四年 & 1759 & \tabularnewline\hline
  二五年 & 1760 & \tabularnewline\hline
  二六年 & 1761 & \tabularnewline\hline
  二七年 & 1762 & \tabularnewline\hline
  二八年 & 1763 & \tabularnewline\hline
  二九年 & 1764 & \tabularnewline\hline
  三十年 & 1765 & \tabularnewline\hline
  三一年 & 1766 & \tabularnewline\hline
  三二年 & 1767 & \tabularnewline\hline
  三三年 & 1768 & \tabularnewline\hline
  三四年 & 1769 & \tabularnewline\hline
  三五年 & 1770 & \tabularnewline\hline
  三六年 & 1771 & \tabularnewline\hline
  三七年 & 1772 & \tabularnewline\hline
  三八年 & 1773 & \tabularnewline\hline
  三九年 & 1774 & \tabularnewline\hline
  四十年 & 1775 & \tabularnewline\hline
  四一年 & 1776 & \tabularnewline\hline
  四二年 & 1777 & \tabularnewline\hline
  四三年 & 1778 & \tabularnewline\hline
  四四年 & 1779 & \tabularnewline\hline
  四五年 & 1780 & \tabularnewline\hline
  四六年 & 1781 & \tabularnewline\hline
  四七年 & 1782 & \tabularnewline\hline
  四八年 & 1783 & \tabularnewline\hline
  四九年 & 1784 & \tabularnewline\hline
  五十年 & 1785 & \tabularnewline\hline
  五一年 & 1786 & \tabularnewline\hline
  五二年 & 1787 & \tabularnewline\hline
  五三年 & 1788 & \tabularnewline\hline
  五四年 & 1789 & \tabularnewline\hline
  五五年 & 1790 & \tabularnewline\hline
  五六年 & 1791 & \tabularnewline\hline
  五七年 & 1792 & \tabularnewline\hline
  五八年 & 1793 & \tabularnewline\hline
  五九年 & 1794 & \tabularnewline\hline
  六十年 & 1795 & \tabularnewline
  \bottomrule
\end{longtable}


%%% Local Variables:
%%% mode: latex
%%% TeX-engine: xetex
%%% TeX-master: "../Main"
%%% End:

%% -*- coding: utf-8 -*-
%% Time-stamp: <Chen Wang: 2018-07-12 22:18:34>

\section{仁宗\tiny(1795-1820)}

\subsection{嘉庆}

\begin{longtable}{|>{\centering\scriptsize}m{2em}|>{\centering\scriptsize}m{1.3em}|>{\centering}m{8.8em}|}
  % \caption{秦王政}\
  \toprule
  \SimHei \normalsize 年数 & \SimHei \scriptsize 公元 & \SimHei 大事件 \tabularnewline
  % \midrule
  \endfirsthead
  \toprule
  \SimHei \normalsize 年数 & \SimHei \scriptsize 公元 & \SimHei 大事件 \tabularnewline
  \midrule
  \endhead
  \midrule
  元年 & 1796 & \tabularnewline\hline
  二年 & 1797 & \tabularnewline\hline
  三年 & 1798 & \tabularnewline\hline
  四年 & 1799 & \tabularnewline\hline
  五年 & 1800 & \tabularnewline\hline
  六年 & 1801 & \tabularnewline\hline
  七年 & 1802 & \tabularnewline\hline
  八年 & 1803 & \tabularnewline\hline
  九年 & 1804 & \tabularnewline\hline
  十年 & 1805 & \tabularnewline\hline
  十一年 & 1806 & \tabularnewline\hline
  十二年 & 1807 & \tabularnewline\hline
  十三年 & 1808 & \tabularnewline\hline
  十四年 & 1809 & \tabularnewline\hline
  十五年 & 1810 & \tabularnewline\hline
  十六年 & 1811 & \tabularnewline\hline
  十七年 & 1812 & \tabularnewline\hline
  十八年 & 1813 & \tabularnewline\hline
  十九年 & 1814 & \tabularnewline\hline
  二十年 & 1815 & \tabularnewline\hline
  二一年 & 1816 & \tabularnewline\hline
  二二年 & 1817 & \tabularnewline\hline
  二三年 & 1818 & \tabularnewline\hline
  二四年 & 1819 & \tabularnewline\hline
  二五年 & 1820 & \tabularnewline
  \bottomrule
\end{longtable}


%%% Local Variables:
%%% mode: latex
%%% TeX-engine: xetex
%%% TeX-master: "../Main"
%%% End:

%% -*- coding: utf-8 -*-
%% Time-stamp: <Chen Wang: 2018-07-12 22:19:20>

\section{宣宗\tiny(1821-1850)}

\subsection{道光}

\begin{longtable}{|>{\centering\scriptsize}m{2em}|>{\centering\scriptsize}m{1.3em}|>{\centering}m{8.8em}|}
  % \caption{秦王政}\
  \toprule
  \SimHei \normalsize 年数 & \SimHei \scriptsize 公元 & \SimHei 大事件 \tabularnewline
  % \midrule
  \endfirsthead
  \toprule
  \SimHei \normalsize 年数 & \SimHei \scriptsize 公元 & \SimHei 大事件 \tabularnewline
  \midrule
  \endhead
  \midrule
  元年 & 1821 & \tabularnewline\hline
  二年 & 1822 & \tabularnewline\hline
  三年 & 1823 & \tabularnewline\hline
  四年 & 1824 & \tabularnewline\hline
  五年 & 1825 & \tabularnewline\hline
  六年 & 1826 & \tabularnewline\hline
  七年 & 1827 & \tabularnewline\hline
  八年 & 1828 & \tabularnewline\hline
  九年 & 1829 & \tabularnewline\hline
  十年 & 1830 & \tabularnewline\hline
  十一年 & 1831 & \tabularnewline\hline
  十二年 & 1832 & \tabularnewline\hline
  十三年 & 1833 & \tabularnewline\hline
  十四年 & 1834 & \tabularnewline\hline
  十五年 & 1835 & \tabularnewline\hline
  十六年 & 1836 & \tabularnewline\hline
  十七年 & 1837 & \tabularnewline\hline
  十八年 & 1838 & \tabularnewline\hline
  十九年 & 1839 & \tabularnewline\hline
  二十年 & 1840 & \tabularnewline\hline
  二一年 & 1841 & \tabularnewline\hline
  二二年 & 1842 & \tabularnewline\hline
  二三年 & 1843 & \tabularnewline\hline
  二四年 & 1844 & \tabularnewline\hline
  二五年 & 1845 & \tabularnewline\hline
  二六年 & 1846 & \tabularnewline\hline
  二七年 & 1847 & \tabularnewline\hline
  二八年 & 1848 & \tabularnewline\hline
  二九年 & 1849 & \tabularnewline\hline
  三十年 & 1850 & \tabularnewline
  \bottomrule
\end{longtable}


%%% Local Variables:
%%% mode: latex
%%% TeX-engine: xetex
%%% TeX-master: "../Main"
%%% End:

%% -*- coding: utf-8 -*-
%% Time-stamp: <Chen Wang: 2018-07-12 22:20:19>

\section{文宗\tiny(1850-1861)}

\subsection{咸丰}

\begin{longtable}{|>{\centering\scriptsize}m{2em}|>{\centering\scriptsize}m{1.3em}|>{\centering}m{8.8em}|}
  % \caption{秦王政}\
  \toprule
  \SimHei \normalsize 年数 & \SimHei \scriptsize 公元 & \SimHei 大事件 \tabularnewline
  % \midrule
  \endfirsthead
  \toprule
  \SimHei \normalsize 年数 & \SimHei \scriptsize 公元 & \SimHei 大事件 \tabularnewline
  \midrule
  \endhead
  \midrule
  元年 & 1851 & \tabularnewline\hline
  二年 & 1852 & \tabularnewline\hline
  三年 & 1853 & \tabularnewline\hline
  四年 & 1854 & \tabularnewline\hline
  五年 & 1855 & \tabularnewline\hline
  六年 & 1856 & \tabularnewline\hline
  七年 & 1857 & \tabularnewline\hline
  八年 & 1858 & \tabularnewline\hline
  九年 & 1859 & \tabularnewline\hline
  十年 & 1860 & \tabularnewline\hline
  十一年 & 1861 & \tabularnewline
  \bottomrule
\end{longtable}


%%% Local Variables:
%%% mode: latex
%%% TeX-engine: xetex
%%% TeX-master: "../Main"
%%% End:

%% -*- coding: utf-8 -*-
%% Time-stamp: <Chen Wang: 2018-07-12 22:21:05>

\section{穆宗\tiny(1861-1875)}

\subsection{同治}

\begin{longtable}{|>{\centering\scriptsize}m{2em}|>{\centering\scriptsize}m{1.3em}|>{\centering}m{8.8em}|}
  % \caption{秦王政}\
  \toprule
  \SimHei \normalsize 年数 & \SimHei \scriptsize 公元 & \SimHei 大事件 \tabularnewline
  % \midrule
  \endfirsthead
  \toprule
  \SimHei \normalsize 年数 & \SimHei \scriptsize 公元 & \SimHei 大事件 \tabularnewline
  \midrule
  \endhead
  \midrule
  元年 & 1862 & \tabularnewline\hline
  二年 & 1863 & \tabularnewline\hline
  三年 & 1864 & \tabularnewline\hline
  四年 & 1865 & \tabularnewline\hline
  五年 & 1866 & \tabularnewline\hline
  六年 & 1867 & \tabularnewline\hline
  七年 & 1868 & \tabularnewline\hline
  八年 & 1869 & \tabularnewline\hline
  九年 & 1870 & \tabularnewline\hline
  十年 & 1871 & \tabularnewline\hline
  十一年 & 1872 & \tabularnewline\hline
  十二年 & 1873 & \tabularnewline\hline
  十三年 & 1874 & \tabularnewline
  \bottomrule
\end{longtable}


%%% Local Variables:
%%% mode: latex
%%% TeX-engine: xetex
%%% TeX-master: "../Main"
%%% End:

%% -*- coding: utf-8 -*-
%% Time-stamp: <Chen Wang: 2018-07-12 22:21:55>

\section{德宗\tiny(1875-1908)}

\subsection{光绪}

\begin{longtable}{|>{\centering\scriptsize}m{2em}|>{\centering\scriptsize}m{1.3em}|>{\centering}m{8.8em}|}
  % \caption{秦王政}\
  \toprule
  \SimHei \normalsize 年数 & \SimHei \scriptsize 公元 & \SimHei 大事件 \tabularnewline
  % \midrule
  \endfirsthead
  \toprule
  \SimHei \normalsize 年数 & \SimHei \scriptsize 公元 & \SimHei 大事件 \tabularnewline
  \midrule
  \endhead
  \midrule
  元年 & 1875 & \tabularnewline\hline
  二年 & 1876 & \tabularnewline\hline
  三年 & 1877 & \tabularnewline\hline
  四年 & 1878 & \tabularnewline\hline
  五年 & 1879 & \tabularnewline\hline
  六年 & 1880 & \tabularnewline\hline
  七年 & 1881 & \tabularnewline\hline
  八年 & 1882 & \tabularnewline\hline
  九年 & 1883 & \tabularnewline\hline
  十年 & 1884 & \tabularnewline\hline
  十一年 & 1885 & \tabularnewline\hline
  十二年 & 1886 & \tabularnewline\hline
  十三年 & 1887 & \tabularnewline\hline
  十四年 & 1888 & \tabularnewline\hline
  十五年 & 1889 & \tabularnewline\hline
  十六年 & 1890 & \tabularnewline\hline
  十七年 & 1891 & \tabularnewline\hline
  十八年 & 1892 & \tabularnewline\hline
  十九年 & 1893 & \tabularnewline\hline
  二十年 & 1894 & \tabularnewline\hline
  二一年 & 1895 & \tabularnewline\hline
  二二年 & 1896 & \tabularnewline\hline
  二三年 & 1897 & \tabularnewline\hline
  二四年 & 1898 & \tabularnewline\hline
  二五年 & 1899 & \tabularnewline\hline
  二六年 & 1900 & \tabularnewline\hline
  二七年 & 1901 & \tabularnewline\hline
  二八年 & 1902 & \tabularnewline\hline
  二九年 & 1903 & \tabularnewline\hline
  三十年 & 1904 & \tabularnewline\hline
  三一年 & 1905 & \tabularnewline\hline
  三二年 & 1906 & \tabularnewline\hline
  三三年 & 1907 & \tabularnewline\hline
  三四年 & 1908 & \tabularnewline
  \bottomrule
\end{longtable}


%%% Local Variables:
%%% mode: latex
%%% TeX-engine: xetex
%%% TeX-master: "../Main"
%%% End:

%% -*- coding: utf-8 -*-
%% Time-stamp: <Chen Wang: 2018-07-12 22:22:42>

\section{溥仪\tiny(1909-1912)}

\subsection{宣统}

\begin{longtable}{|>{\centering\scriptsize}m{2em}|>{\centering\scriptsize}m{1.3em}|>{\centering}m{8.8em}|}
  % \caption{秦王政}\
  \toprule
  \SimHei \normalsize 年数 & \SimHei \scriptsize 公元 & \SimHei 大事件 \tabularnewline
  % \midrule
  \endfirsthead
  \toprule
  \SimHei \normalsize 年数 & \SimHei \scriptsize 公元 & \SimHei 大事件 \tabularnewline
  \midrule
  \endhead
  \midrule
  元年 & 1909 & \tabularnewline\hline
  二年 & 1910 & \tabularnewline\hline
  三年 & 1911 & \tabularnewline\hline
  四年 & 1912 & \tabularnewline
  \bottomrule
\end{longtable}


%%% Local Variables:
%%% mode: latex
%%% TeX-engine: xetex
%%% TeX-master: "../Main"
%%% End:



%%% Local Variables:
%%% mode: latex
%%% TeX-engine: xetex
%%% TeX-master: "../Main"
%%% End:
 % 清

% %% -*- coding: utf-8 -*-
%% Time-stamp: <Chen Wang: 2019-10-15 11:23:36>

\chapter{附录}

%% -*- coding: utf-8 -*-
%% Time-stamp: <Chen Wang: 2019-10-15 11:23:51>

\section{名人简介}

常见于各种诗话、词话,多以名人之字、号、尊称、谥号等,外加小传。

%% -*- coding: utf-8 -*-
%% Time-stamp: <Chen Wang: 2018-10-30 16:26:19>

\subsection{先秦}

\begin{longtable}{|>{\centering\namefont\heiti}m{2em}|>{\centering\tiny}m{3.0em}|>{\xzfont\kaiti}m{7em}|}
    % \caption{秦王政}\
    \toprule
    \SimHei \normalsize 姓名 & \SimHei \normalsize 异名 & \SimHei \normalsize \hspace{2.5em}小传 \tabularnewline
    % \midrule
    \endfirsthead
    \toprule
    \SimHei \normalsize 姓名 & \SimHei \normalsize 异名 & \SimHei \normalsize \hspace{2.5em}小传 \tabularnewline 
    \midrule
    \endhead
    \midrule
    屈平 & \begin{description}
    \item[字] 屈原
    \item[号] 屈子
    \item[谥] 
    \item[尊] 三闾大夫
    \item[生] 楚国
    \end{description} & 屈原(约前340年-约前278年6月6日),芈姓,屈氏,名平,字原,楚国人(今湖北秭归),是古帝高阳氏的后裔,其自作词曰:“帝高阳之苗裔兮,朕皇考曰伯庸。”,其先祖屈瑕受楚武王封于屈地,因以屈为氏,名平。屈,昭,景为楚国大姓,官拜左徒,左徒多以贵族近臣任之,左徒任务有四 “议国事”、“出号令”、“接遇宾客”、“应对诸侯”。 \tabularnewline
    \bottomrule
\end{longtable}


%%% Local Variables:
%%% mode: latex
%%% TeX-engine: xetex
%%% TeX-master: "../../Main"
%%% End:

%% -*- coding: utf-8 -*-
%% Time-stamp: <Chen Wang: 2018-10-30 16:25:48>

\subsection{秦汉}

\begin{longtable}{|>{\centering\namefont\heiti}m{2em}|>{\centering\tiny}m{3.0em}|>{\xzfont\kaiti}m{7em}|}
    % \caption{秦王政}\
    \toprule
    \SimHei \normalsize 姓名 & \SimHei \normalsize 异名 & \SimHei \normalsize \hspace{2.5em}小传 \tabularnewline
    % \midrule
    \endfirsthead
    \toprule
    \SimHei \normalsize 姓名 & \SimHei \normalsize 异名 & \SimHei \normalsize \hspace{2.5em}小传 \tabularnewline 
    \midrule
    \endhead
    \midrule
    司马迁 & \begin{description}
    \item[字] 子长
    \item[号] 
    \item[谥] 
    \item[尊] 太史公
    \item[生] 龙门
    \end{description} & 司马迁(前145年(景帝五年)-约前86年(昭帝始元元年)),字子长,左冯翊夏阳(今山西河津)人(一说陕西韩城人),是中国西汉时期著名的史学家和文学家。司马迁所撰写的《史记》被公认为是中国史书的典范,首创的纪传体撰史方法为后来历代正史所传承,被后世尊称为史迁,又因曾任太史令,故自称太史公。 \tabularnewline\hline
    班固 & \begin{description}
    \item[字] 孟坚
    \item[号] 
    \item[谥] 
    \item[尊] 
    \item[生] 陕西咸阳
    \end{description} & 班固(东汉光武帝建武十(公元32)年-东汉和帝永元四(公元92)年),字孟坚,扶风安陵(今陕西咸阳)人,东汉史学家班彪之子,东汉历史学家,《汉书》的作者。 \tabularnewline\hline
    张衡 & \begin{description}
    \item[字] 平子
    \item[号] 
    \item[谥] 
    \item[尊] 
    \item[生] 南阳西鄂
    \end{description} &  张衡(78年-139年),字平子,南阳西鄂人,东汉士大夫、天文学家、地理学家、数学家、科学家、发明家及文学家,官至太史令、侍中、尚书。张衡一生成就不凡,曾制作以水力推动的浑天仪、发明能够探测震源方向的地动仪和指南车、发现月蚀的原因、绘制记录2,500颗星体的星图、计算圆周率准确至小数点后一个位、解释和确立浑天说的宇宙论;在文学方面,他创作了《二京赋》及《归田赋》等辞赋名篇,拓展了汉赋的文体与题材,被列为“汉赋四大家”之一。他开创了七言古诗的诗歌体裁,对中华文化有巨大贡献。张衡为备受尊崇的伟大科学家,成就与西方同时期的托勒密媲美。此外,他的地位也被现代天文学界所肯定。\tabularnewline\hline
    王粲 & \begin{description}
    \item[字] 仲宣
    \item[号] 
    \item[谥] 
    \item[尊] 
    \item[生] 山东微山
    \end{description} & 王粲(177年-217年2月17日),字仲宣,东汉山阳高平(今山东省微山县)人。擅长辞赋,建安七子之一,被誉为“七子之冠冕”。少有才名,为著名学者蔡邕所赏识。初平二年(192年),因关中骚乱,前往荆州依靠刘表,客居荆州十余年,有志不伸,心怀颇郁郁。建安十三年(208年),曹操南征荆州,不久,刘表病逝,其子刘琮举州投降,王粲也归曹操,深得曹氏父子信赖,赐爵关内侯。建安十八年(213年),魏王国建立,王粲任侍中。建安二十二年(216年),王粲随曹操南征孙权,于北还途中病逝,终年四十一岁。王粲善属文,其诗赋为建安七子之冠,又与曹植并称“曹王”。著《英雄记》,《三国志》记王粲著诗、赋、论、议近60篇,《隋书·经籍志》著录有文集十一卷。明人张溥辑有《王侍中集》。 \tabularnewline\hline

    \bottomrule
\end{longtable}


%%% Local Variables:
%%% mode: latex
%%% TeX-engine: xetex
%%% TeX-master: "../../Main"
%%% End:

%% -*- coding: utf-8 -*-
%% Time-stamp: <Chen Wang: 2018-10-30 16:26:11>

\subsection{魏晋南北朝}

\begin{longtable}{|>{\centering\namefont\heiti}m{2em}|>{\centering\tiny}m{3.0em}|>{\xzfont\kaiti}m{7em}|}
  % \caption{秦王政}\
  \toprule
  \SimHei \normalsize 姓名 & \SimHei \normalsize 异名 & \SimHei \normalsize \hspace{2.5em}小传 \tabularnewline
  % \midrule
  \endfirsthead
  \toprule
  \SimHei \normalsize 姓名 & \SimHei \normalsize 异名 & \SimHei \normalsize \hspace{2.5em}小传 \tabularnewline 
  \midrule
  \endhead
  \midrule
  张协 & \begin{description}
  \item[字] 景阳
  \item[号] 
  \item[谥] 
  \item[尊] 
  \item[生] 河北安平
  \end{description} & 张协(?~307?),字景阳。 西晋文学家,安平(今属河北省)人。父亲张收,蜀郡太守。张协少有俊才,与兄长张载齐名。曾任公府掾、秘书郎、华阳令等职。永宁元年(301年),为成都王、征北将军司马颖的从事中郎,后迁中书侍郎,转河间内史,治郡清简。惠帝末年,天下纷乱,他辞官隐居,以吟咏自娱。永嘉初,复征为黄门侍郎,托病不就。后逝于家。与其兄张载、其弟张亢,都是西晋著名的文学家,时称“三张”。 \tabularnewline\hline
  潘岳 & \begin{description}
  \item[字] 安仁
  \item[号] 
  \item[谥] 
  \item[尊] 
  \item[生] 河南中牟
  \end{description} & 潘安(公元247年―公元300年),即潘岳,字安仁。河南中牟人。西晋著名文学家、政治家,潘安之名始于杜甫《花底》诗“恐是潘安县,堪留卫玠车。”后世遂以潘安称焉。美姿仪,少以才名闻世,他性轻躁,趋于世利,与石崇等谄事贾谧,每候其出,辄望尘而拜。与石崇、陆机、刘琨、左思等并为“贾谧二十四友”,潘安为首。潘安被誉为“古代第一美男”。潘岳在文学上与陆机并称“潘江陆海”,钟嵘《诗品》称“陆才如海,潘才如江”,王勃《滕王阁序》“请洒潘江,各倾陆海云尔。” \tabularnewline\hline
  潘尼 & \begin{description}
  \item[字] 正叔
  \item[号] 
  \item[谥] 
  \item[尊] 
  \item[生] 荥阳中牟
  \end{description} & 潘尼(约250~约311年),字正叔,荥阳中牟人(在今河南城关镇大潘庄),西晋文学家。祖父潘勖,中国东汉东海相。父亲潘满,平原内史。潘岳之侄,少有才,与潘岳俱以文章知名,并称“两潘”。潘尼生情稳静恬淡,不与人争利,安心研读,专志著述。 \tabularnewline\hline
  陆机 & \begin{description}
  \item[字] 士衡
  \item[号] 
  \item[谥] 
  \item[尊] 
  \item[生] 江苏苏州
  \end{description} & 陆机(261年-303年),字士衡,吴郡吴县(今江苏苏州)人。西晋著名文学家、书法家。出身吴郡陆氏,为孙吴丞相陆逊之孙、大司马陆抗第四子,与其弟陆云合称“二陆”,又与顾荣、陆云并称“洛阳三俊”。陆机在孙吴时曾任牙门将,吴亡后出仕西晋,太康十年(289年),陆机兄弟来到洛阳,文才倾动一时,受太常张华赏识,此后名气大振。时有“二陆入洛,三张减价”之说。历任太傅祭酒、吴国郎中令、著作郎等职,与贾谧等结为“金谷二十四友”。 \tabularnewline\hline
  陆云 & \begin{description}
  \item[字] 士龙
  \item[号] 
  \item[谥] 
  \item[尊] 陆清河
  \item[生] 江苏苏州
  \end{description} & 陆云(262年-303年),字士龙,吴郡吴县(今江苏苏州)人,西晋官员、文学家,东吴丞相陆逊之孙,大司马陆抗第五子。与其兄陆机合称“二陆”,曾任清河内史,故世称“陆清河”。陆机死于“八王之乱”而被夷三族后,陆云也为之牵连入狱。尽管许多人上疏司马颖请求不要株连陆云,但他最终还是遇害了。时年四十二岁,无子,生有二女。由门生故吏迎葬于清河。 \tabularnewline\hline
  左思 & \begin{description}
  \item[字] 泰冲
  \item[号] 
  \item[谥] 
  \item[尊] 
  \item[生] 山东淄博
  \end{description} & 左思(约250~305),字泰冲,齐国临淄(今山东淄博)人。西晋著名文学家,其《三都赋》颇被当时称颂,造成“洛阳纸贵”。另外,其《咏史诗》《娇女诗》也很有名。其诗文语言质朴凝练。后人辑有《左太冲集》。 \tabularnewline\hline
  卢谌 & \begin{description}
  \item[字] 子谅
  \item[号] 
  \item[谥] 
  \item[尊] 
  \item[生] 河北涿县
  \end{description} & 卢谌(284─351),字子谅,范阳涿(今属河北涿县)人,晋代文学家。曹魏司空卢毓曾孙。西晋卫尉卿卢珽之孙,尚书卢志长子。晋朝历任司空主簿、从事中郎、幽州别驾。后赵、冉魏时官至侍中、中书监。卢谌最初担任太尉椽。311年,洛阳失陷,随父北依刘琨,途中被刘粲所掳。312年,辗转归于姨父刘琨,受到青睐。315年,刘琨为司空,任卢谌为主簿,继转任从事中郎。316年,并州失守,随刘琨投奔幽州刺史段匹磾,匹磾以卢谌幽州别驾。318年,刘琨为匹磾所拘。期间,卢谌与刘琨以诗相互赠答,写有《答刘琨诗二首》《赠刘琨诗二十首》。刘琨被害,卢谌前往辽西依附段末波。朝廷不敢为其吊祭,后卢谌等上表申理,文旨甚是切恳。石虎攻取辽西后,进入后赵,历任中书侍郎、国子祭酒、侍中、中书监等职。350年,冉闵诛石氏、灭后赵,卢谌在冉魏任中书监,后在襄国遇害。时年67岁。卢谌为人清敏、才思敏捷,喜读老庄,又善于写文章。他著有《祭法》《庄子注》及文集十卷,其中有些诗篇流传至今。 \tabularnewline\hline
  孙绰 & \begin{description}
  \item[字] 兴公
  \item[号] 
  \item[谥] 
  \item[尊] 
  \item[生] 山西平遥
  \end{description} & 孙绰(314—371),字兴公,东晋玄言诗人。中都(今山西平遥)人,后迁会稽(今浙江绍兴)。曾任临海章安令,在任时写过著名的《天台山赋》。其善书博学,是参加王羲之兰亭修禊的诗人和书法家。 \tabularnewline\hline
    颜延之 & \begin{description}
    \item[字] 延年
    \item[号] 
    \item[谥] 
    \item[尊] 
    \item[生] 山东临沂
    \end{description} & 颜延之(384~456年),字延年,南朝宋文学家。琅邪临沂(今山东临沂)人。曾祖含,右光禄大夫。祖约,零陵太守。父显,护军司马。少孤贫,居陋室,好读书,无所不览,文章之美,冠绝当时,与谢灵运并称“颜谢”。 \tabularnewline\hline
    谢灵运 & \begin{description}
    \item[字] 灵运
    \item[号] 
    \item[谥] 
    \item[尊] 谢客
    \item[生] 河南太康
    \end{description} & 谢灵运(385年—433年),原名公义,字灵运,以字行于世,小名客儿,世称谢客。南北朝时期杰出的诗人、文学家、旅行家、道家。谢灵运出身陈郡谢氏,祖籍陈郡阳夏(今河南太康县),生于会稽始宁(今绍兴市嵊州市三界镇)。为东晋名将谢玄之孙、秘书郎谢瑍之子。东晋时世袭为康乐公,世称谢康乐。曾出任大司马行军参军、抚军将军记室参军、太尉参军等职。刘宋代晋后,降封康乐侯,历任永嘉太守、秘书监、临川内史,元嘉十年(433年)被宋文帝刘义隆以“叛逆”罪名杀害,年四十九。谢灵运少即好学,博览群书,工诗善文。其诗与颜延之齐名,并称“颜谢”,开创了中国文学史上的山水诗派,他还兼通史学,擅书法,曾翻译外来佛经,并奉诏撰《晋书》。明人辑有《谢康乐集》。 \tabularnewline\hline
    鲍照 & \begin{description}
    \item[字] 明远
    \item[号] 
    \item[谥] 
    \item[尊] 
    \item[生] 山工临沂
    \end{description} & 鲍照(414年-466年),字明远,东海郡人(今属山东临沂市兰陵县长城镇),中国南朝宋杰出的文学家、诗人。宋元嘉中,临川王刘义庆“招聚文学之士,近远必至”,鲍照以辞章之美而被看重,遂引为“佐史国臣”。元嘉十六年因献诗而被宋文帝用为中书令、秣稜令。大明五年出任前军参军,故世称“鲍参军”。泰始二年刘子顼起兵反明帝失败,鲍照死于乱军中。鲍照与颜延之、谢灵运同为宋元嘉时代的著名诗人,合称“元嘉三大家”,其诗歌注意描写山水,讲究对仗和辞藻。他长于乐府诗,其七言诗对唐代诗歌的发展起了重要作用。世称“元嘉体”,现有《鲍参军集》传世。鲍照和庾信合称“南照北信”。 \tabularnewline\hline
    谢朓 & \begin{description}
    \item[字] 玄晖
    \item[号] 
    \item[谥] 
    \item[尊] 
    \item[生] 河南太康
    \end{description} & 谢朓(464—499),字玄晖,汉族,陈郡阳夏(今河南太康县)人。南朝齐杰出的山水诗人,出身高门士族,与“大谢”谢灵运同族,世称“小谢”。19岁解褐豫章王太尉行参军。永明五年(487),与竟陵王萧子良西邸之游,初任其功曹、文学,为“竟陵八友”之一。永明九年(491),随随王萧子隆至荆州,十一年还京,为骠骑咨议、领记室。建武二年(495),出为宣城太守。两年后,复返京为中书郎。之后,又出为南东海太守,寻迁尚书吏部郎,又称谢宣城、谢吏部。东昏侯永元元年(499)遭始安王萧遥光诬陷,死狱中,时年36岁。曾与沈约等共创“永明体”。今存诗二百余首,多描写自然景物,间亦直抒怀抱,诗风清新秀丽,圆美流转,善于发端,时有佳句;又平仄协调,对偶工整,开启唐代律绝之先河。 \tabularnewline\hline
    江淹 & \begin{description}
    \item[字] 文通
    \item[号] 
    \item[谥] 
    \item[尊] 
    \item[生]河南商丘
    \end{description} & 江淹(444年—505年),字文通,南朝著名政治家、文学家,历仕三朝,宋州济阳考城(今河南省商丘市民权县程庄镇江集村)人。江淹少时孤贫好学,六岁能诗。文章华著,十三岁丧父。二十岁左右在新安王刘子鸾幕下任职,开始其政治生涯,齐高帝闻其才,召授尚书驾部郎,骠骑参军事;明帝时为御史中丞,先后弹劾中书令谢朏等人;武帝时任骠骑将军兼尚书左丞,历仕南朝宋、齐、梁三代。 \tabularnewline\hline
    丘迟 & \begin{description}
    \item[字] 希范
    \item[号] 
    \item[谥] 
    \item[尊] 
    \item[生] 浙江湖州
    \end{description} & 丘迟(464年-508年),字希范,中国南朝文学家,吴兴乌程(今属浙江省湖州市)人。父丘灵鞠,南齐太中大夫,亦为当时知名文人。丘迟八岁能文,初仕南齐,官至殿中郎、车骑录事参军。后投入萧衍幕中,为其所重,其后萧衍代齐为帝建立南梁的一应劝进文书均为丘迟所作。天监四年(505年)随萧宏北伐,为其记室,以一封《与陈伯之书》成功招降投奔北魏的原南齐将领陈伯之来降,后历任永嘉太守、拜中书郎,再升任司徒从事中郎。天监七年,以四十五岁卒于官。 \tabularnewline\hline
    钟嵘 & \begin{description}
    \item[字] 仲伟
    \item[号] 
    \item[谥] 
    \item[尊] 
    \item[生] 不详
    \end{description} & 钟嵘(约468—约518), 中国南朝文学批评家。字仲伟。颍川长社(今河南许昌长葛市)人。齐代官至司徒行参军。入梁,历任中军临川王行参军、西中郎将晋安王记室。梁武帝天监十二年(513)以后,仿汉代“九品论人,七略裁士”的著作先例,写成诗歌评论专著《诗品》。以五言诗为主,全书将两汉至梁作家122人,分为上、中、下三品进行评论,故名为《诗品》。《隋书·经籍志》著录此书,书名为《诗评》,这是因为除品第之外,还就作品评论其优劣。后以《诗品》定名。在《诗品》中,钟嵘提倡风力,反对玄言;主张音韵自然和谐,反对人为的声病说;主张“直寻”,反对用典,提出了一套比较系统的诗歌品评的标准。钟嵘(约468—约518), 中国南朝文学批评家。字仲伟。颍川长社(今河南许昌长葛市)人。齐代官至司徒行参军。入梁,历任中军临川王行参军、西中郎将晋安王记室。梁武帝天监十二年(513)以后,仿汉代“九品论人,七略裁士”的著作先例,写成诗歌评论专著《诗品》。以五言诗为主,全书将两汉至梁作家122人,分为上、中、下三品进行评论,故名为《诗品》。《隋书·经籍志》著录此书,书名为《诗评》,这是因为除品第之外,还就作品评论其优劣。后以《诗品》定名。在《诗品》中,钟嵘提倡风力,反对玄言;主张音韵自然和谐,反对人为的声病说;主张“直寻”,反对用典,提出了一套比较系统的诗歌品评的标准。 \tabularnewline\hline

  \bottomrule
\end{longtable}


%%% Local Variables:
%%% mode: latex
%%% TeX-engine: xetex
%%% TeX-master: "../../Main"
%%% End:

%% -*- coding: utf-8 -*-
%% Time-stamp: <Chen Wang: 2018-10-30 16:26:04>

\subsection{唐五代}

\begin{longtable}{|>{\centering\namefont\heiti}m{2em}|>{\centering\tiny}m{3.0em}|>{\xzfont\kaiti}m{7em}|}
  % \caption{秦王政}\
  \toprule
  \SimHei \normalsize 姓名 & \SimHei \normalsize 异名 & \SimHei \normalsize \hspace{2.5em}小传 \tabularnewline
  % \midrule
  \endfirsthead
  \toprule
  \SimHei \normalsize 姓名 & \SimHei \normalsize 异名 & \SimHei \normalsize \hspace{2.5em}小传 \tabularnewline 
  \midrule
  \endhead
  \midrule
    宋之问 & \begin{description}
    \item[字] 延清
    \item[号] 
    \item[谥] 
    \item[尊] 宋考功
    \item[生] 山西汾阳
    \end{description} & 宋之问(约656 — 约712),字延清,名少连,汉族,汾州隰城人(今山西汾阳市)人,初唐时期的诗人,与沈佺期并称“沈宋”。与陈子昂、卢藏用、司马承祯、王适、毕构、李白、孟浩然、王维、贺知章称为仙宗十友。 \tabularnewline\hline
    沈佺期 & \begin{description}
    \item[字] 云卿
    \item[号] 
    \item[谥] 
    \item[尊] 
    \item[生] 安阳
    \end{description} & 沈佺期(约656 — 约715),字云卿,相州内黄(今安阳市内黄县)人,祖籍吴兴(今浙江湖州)。唐代诗人。与宋之问齐名,称“ 沈宋 ”。善属文,尤长七言之作。擢进士第。长安中,累迁通事舍人,预修《三教珠英》,转考功郎给事中。坐交张易之,流驩州。稍迁台州录事参军。神龙中,召见,拜起居郎,修文馆直学士,历中书舍人,太子少詹事。开元初卒。建安后,讫江左,诗律屡变,至沈约、庾信,以音韵相婉附,属对精密,及沈佺期与宋之问,尤加靡丽。回忌声病,约句准篇,如锦绣成文,学者宗之,号为“沈宋”。语曰:苏李居前,沈宋比肩。有集十卷,今编诗三卷。 \tabularnewline\hline
    陈子昂 & \begin{description}
    \item[字] 伯玉
    \item[号] 
    \item[谥] 
    \item[尊] 陈拾遗
    \item[生] 四川射洪
    \end{description} & 陈子昂(公元659~公元702),字伯玉,梓州射洪(今四川省遂宁市射洪县)人,唐代诗人,初唐诗文革新人物之一。因曾任右拾遗,后世称陈拾遗。青少年时轻财好施,慷慨任侠,24岁举进士,以上书论政得到女皇武则天重视,授麟台正字。后升右拾遗,直言敢谏,曾因“逆党”反对武后而株连下狱。在26岁、36岁时两次从军边塞,对边防颇有些远见。38岁(圣历元年698)时,因父老解官回乡,不久父死。陈子昂居丧期间,权臣武三思指使射洪县令段简罗织罪名,加以迫害,冤死狱中。   其存诗共100多首,其诗风骨峥嵘,寓意深远,苍劲有力。其中最有代表性的有组诗《感遇》38首,《蓟丘览古》7首和《登幽州台歌》、《登泽州城北楼宴》等。 \tabularnewline\hline
  张说 & \begin{description}
  \item[字] 道济\\说之
  \item[号] 
  \item[谥] 
  \item[尊] 张燕公
  \item[生] 河北涿州
  \end{description} & 张说(667年-730年),字道济,一字说之,原籍范阳(今河北涿州市),世居河东(今山西永济),后徙洛阳。唐玄宗宰相,封燕国公。擅长文学,当时朝廷重要辞章多出其手,尤长于碑文墓志,与许国公苏颋齐名,并称“燕许大手笔”。 \tabularnewline\hline
  李颀 & \begin{description}
  \item[字] 
  \item[号] 
  \item[谥] 
  \item[尊] 
  \item[生] 河北赵县
  \end{description} & 李颀(690年-751年),唐代赵郡(今河北赵县)人,后长居颖阳(今河南登封),唐代诗人。李颀出身于唐朝士族赵郡李氏,常服饵丹砂,“甚有好颜色”,因结识富豪轻薄子弟,倾财破产。后立志刻苦读书,隐居颍阳苦读十年,于唐玄宗开元二十三年(735年)贾季邻榜进士及第,曾为新乡县尉,始终未得迁调,天宝十载前即辞官归隐。余事不详。李颀性格超脱,厌薄世俗,以写诗称著,与诗人王维、王昌龄、高适等来往密切。他的诗秀丽雄浑,内容与体裁颇为广泛。又以五言、七言歌行和七言律诗见长,清代王士祯评:“盛唐七言诗,老杜外,王维、李颀、岑参耳。”。他尤以边塞诗著称,格调雄浑奔放,慷慨激昂。李颀的代表作有《古从军行》、《古意》、《塞下曲》、《听董大弹胡笳兼寄语房给事》。著有《李颀集》。《全唐诗》录其诗3卷,共127首。 \tabularnewline\hline
  王昌龄 & \begin{description}
  \item[字] 少伯
  \item[号] 
  \item[谥] 
  \item[尊] 王江宁\\龙标
  \item[生] 山西太原
  \end{description} & 王昌龄(698年-756年),字少伯,山西太原人,盛唐著名边塞诗人。他的诗和高适、王之涣齐名,因其善写场面雄阔的边塞诗,而有“诗家天子”(或作“诗家夫子”)、 “七绝圣手”、“开天圣手”、“诗天子”的美誉。世称“王江宁”。著有文集六卷,今编诗四卷。代表作有《从军行七首》、《出塞》、《闺怨》等。 \tabularnewline\hline
  崔颢 & \begin{description}
  \item[字] 
  \item[号] 
  \item[谥] 
  \item[尊] 崔思勋\\崔郎中
  \item[生] 河南开封
  \end{description} & 崔颢(约704年-754年)是中国唐朝诗人,汴州(今河南开封)人。开元十一年(723年)中进士,开元二十九年 ,担任扶沟县尉,官位一直不显,后游历天下。天宝九载前后曾任监察御史,官至司勋员外郎。天宝十三载卒。现存诗仅四十二首,最有名的一首莫过于《黄鹤楼》,乃千古绝唱。少年时作的诗多写闺情,流于浮艳,后历边塞,诗风变得雄浑奔放、风骨凛然。崔颢四处游历,吟诗甚勤,其友人笑他吟诗吟得人也瘦(非子病如此,乃苦吟诗瘦耳)。明人辑有《崔颢集》。 \tabularnewline\hline
  高适 & \begin{description}
  \item[字] 达夫
  \item[号] 
  \item[谥] 
  \item[尊] 高常侍
  \item[生] 河北景县
  \end{description} & 高适(706年-765年2月17日),字达夫,沧州渤海人(今河北景县)。唐朝边塞诗人,诗词语言质朴,风格雄浑,与岑参并称“高岑”。 \tabularnewline\hline
  刘长卿 & \begin{description}
  \item[字] 文房
  \item[号] 
  \item[谥] 
  \item[尊] 刘随州
  \item[生] 安徽宣城
  \end{description} & 刘长卿(?-约790年),字文房,宣城(今属安徽)人,郡望河间(今属河北),唐代诗人。年轻时在嵩山读书,唐玄宗开元进士,曾任监察御史,常因性情刚烈而冒犯他人,至德三年(乾元元年,758年)正月,摄海盐令。因事由苏州长洲尉贬为潘州南巴(今广东电白县)尉,代宗时任转运使判官,知淮西、鄂岳转运留后,大历年间,又因得罪了鄂岳观察使吴仲孺,被诬为贪赃,贬为睦州(今浙江淳安)司马。终官随州(今湖北随县)刺史,世称“刘随州”。贞元元年,淮西节度使李希烈割据随州称王,时局动荡,刘长卿离开随州,晚年流寓江州,曾入淮南节度使幕。约卒于贞元六年。 \tabularnewline\hline
  岑参 & \begin{description}
  \item[字] 
  \item[号] 
  \item[谥] 
  \item[尊] 岑嘉州
  \item[生] 荆州江陵
  \end{description} & 岑参(715年-770年),荆州江陵县人,郡望南阳,唐朝诗人,宰相岑文本曾孙,边塞诗代表人物,与高适并称高岑。曾任嘉州(今四川省乐山市)刺史,后人因称“岑嘉州”。 \tabularnewline\hline
  元结 & \begin{description}
  \item[字] 次山
  \item[号] 漫郎\\猗玕子
  \item[谥] 
  \item[尊] 
  \item[生] 河南鲁山
  \end{description} & 元结(723年-772年5月26日),字次山,号漫郎、猗玕子,河南鲁山人。唐朝进士、官员。有《元次山集》。 \tabularnewline\hline
  钱起 & \begin{description}
  \item[字] 仲文
  \item[号] 
  \item[谥] 
  \item[尊] 员外郎
  \item[生] 浙江湖州
  \end{description} & 钱起(710年-782年),字仲文,吴兴(今浙江湖州)人。唐代诗人,诗风清奇,与郎士元、司空曙、李益、李端、卢纶、李嘉祐等称大历十才子。 \tabularnewline\hline
  李泌 & \begin{description}
  \item[字] 长源
  \item[号] 
  \item[谥] 
  \item[尊] 
  \item[生] 京兆
  \end{description} & 李泌(722年-789年),字长源,唐朝宰相,京兆人,祖籍辽东襄平。李泌是西魏八柱国李弼的六代孙,父亲李承休是吴房县令,娶汝南周氏为妻,聚书两万余卷,并告诫子孙不得卖书。李泌幼居长安,七岁能文,张九龄奇之,玄宗召令供奉东宫,写诗讽刺杨国忠,有“青青东门柳,岁晏复憔悴。”之句,隐居颍阳。肃宗时,参预军国大议,拜银青光禄大夫,隐居衡山(今湖南省),修练道教,刘昫说:“居相位而从事鬼神,乃见狂妄浮薄之踪。”代宗时,召为翰林学士,不久因得罪权臣元载,被代宗外放为杭州刺史以避祸。德宗时,元载失势,复召回朝廷并授散骑常侍。贞元中,拜中书侍郎平章事,封邺县侯。李泌以虚诞自任,辅佐四朝天子。贞元五年(789年)三月,辞世。有文集二十卷。 \tabularnewline\hline
  司空曙 & \begin{description}
  \item[字] 文明
  \item[号] 
  \item[谥] 
  \item[尊] 
  \item[生] 河北永年
  \end{description} & 司空曙(720年-790年),字文明,或作文初。广平郡(治所在今河北省永年县东南)人。唐代官员、诗人。中年因安史之乱避居南方,数年后北归长安,曾中进士,曾任洛阳主簿,后任左拾遗,因事贬长林(今湖北荆门西北)县丞 。贞元年间在剑南节度使韦皋下作幕府,官检校水部郎中。终官虞部郎中。余事不可确考,生卒年亦不详。司空曙经历安史之乱,他的诗作以写自然景色和乡情旅思为主,擅长五律,共有集三卷,是大历十才子之一。 \tabularnewline\hline
  戴叔伦 & \begin{description}
  \item[字] 幼公\\次公
  \item[号] 
  \item[谥] 
  \item[尊] 戴容州
  \item[生] 江苏常州
  \end{description} & 戴叔伦(732年-789年),字幼公,一字次公,润州金坛南瑶村(今属江苏常州市)人,唐朝著名诗人。远祖戴安道。戴于唐代宗广德初年任秘书省正字,大历元年(766年),在户部尚书充诸道盐铁使刘晏幕下任职,经刘晏表奏,授监察御史衔。唐德宗建中初年出任东阳县令,后又入江西观察使幕府,授大理寺司直衔,兴元元年(784年)出任抚州刺史,贞元元年(785年)十一月撰有《贺平贼赦表》,授吏部郎中衔。贞元二年(786)辞官还乡,四年出任容州刺史兼容管经略使,贞元五年,卒于任所。 \tabularnewline\hline
  韦应物 & \begin{description}
  \item[字] 
  \item[号] 
  \item[谥] 
  \item[尊] 韦左司\\韦苏州
  \item[生] 陕西西安
  \end{description} & 韦应物(737年-791年),京兆郡杜陵县(今陕西省西安市长安区)人。唐代诗人。《唐诗三百首》收录韦应物诗12首。 \tabularnewline\hline
  李益 & \begin{description}
  \item[字] 君虞
  \item[号] 
  \item[谥] 
  \item[尊] 
  \item[生] 河南郑州
  \end{description} & 李益(746年-829年),字君虞,郑州人,祖籍陇西狄道(今甘肃省临洮县),中唐诗人,以边塞诗作名世,擅长绝句,尤其工于七绝。李益出自陇西李氏姑臧房,是唐朝给事、赠兵部尚书李亶的曾孙,虞部郎中李成绩的孙子,大理司直、赠太子少师李存的儿子,擅长写作诗歌,成名于贞元末年,与唐朝宗室大郑王房“诗鬼”之称的李贺齐名。年轻时的他颇负文名,每写成一篇诗作,宫中都会有乐工名伶争相出价,希望买下他的作品,编排乐曲,让皇帝欣赏。李益所创作的《征人》、《早行》等名篇,更被当时人绘成图赞,流传天下。可是李益为人多疑善妒,相当执著,对于妻妾的德操管治非常严苛,不许妻妾与帮闲坊众接触,因此世人都戏称那些善妒者是患上“李益疾”,更有“妒痴尚书李十郎”之语。 \tabularnewline\hline
  张籍 & \begin{description}
  \item[字] 文昌
  \item[号] 诗肠
  \item[谥] 
  \item[尊] 张水部\\张司业
  \item[生] 江苏苏州
  \end{description} & 张籍(约767年—约830年),唐朝诗人。字文昌,又称“诗肠”。原籍吴郡(今江苏苏州),后迁居和州乌江(今安徽和县)。唐德宗贞元十四年北游,经孟郊介绍,在汴州(今河南开封)认识韩愈。贞元十五年进士。历任太常寺太祝,因患目疾,自称“草色遥看近却无”,孟郊称他为“穷瞎张太祝”。元和十一年,转任国子监助教,目疾稍愈。迁秘书郎。藩镇李师道仰慕张籍的学识,想网罗入幕,张籍婉拒,写了一首〈节妇吟〉寄给了李司徒。长庆元年(821年),韩愈荐为国子博士,历任水部员外郎、主客郎中,终国子司业。时称“张水部”或“张司业”。因其出身贫寒,官职低微,能较多地接触社会底层的民众,故其所作乐府诗多批判社会,同情百姓的遭遇,颇为白居易等人所推崇,白居易称赞为“尤工乐府诗,举代少其伦”。与王建齐名,号称“张王乐府”。《彦周诗话》论道:“张籍,乐府、宫辞皆杰出”。其与白居易,孟郊等所作的诗歌被称为“元和体”。著有《张司业集》,编为五卷。南唐张洎收集其诗400多首编为《木铎集》12卷。明代嘉靖万历间刻本《唐张司业诗集》8卷,收诗450多首。 \tabularnewline\hline
  刘禹锡 & \begin{description}
  \item[字] 梦得
  \item[号] 
  \item[谥] 
  \item[尊] 刘宾客\\刘尚书
  \item[生] 河南洛阳
  \end{description} & 刘禹锡(772年-842年),河南洛阳人,字梦得,祖先来自北方,自言出于中山(今河北省定州市)(又自称“家本荥上,籍占洛阳”)。唐朝著名诗人,中唐文学的代表人物之一。因曾任太子宾客,故称刘宾客,晚年曾加检校礼部尚书、秘书监等虚衔,故又称秘书刘尚书。 \tabularnewline\hline
  李绅 & \begin{description}
  \item[字] 公垂
  \item[号] 
  \item[谥] 
  \item[尊] 
  \item[生] 安徽亳州
  \end{description} & 李绅(772年-846年),字公垂,中唐诗人。亳州(今属安徽)人,生于乌程(今浙江湖州),长于润州无锡(今属江苏)。李绅生于唐大历七年(772年),曾祖父李敬玄,祖父定李守一籍安徽亳州。父李晤,历任金坛、乌程(今浙江吴兴)、晋陵(今江苏常州)等县令,携家来无锡,定居梅里抵陀里(今江苏无锡东亭长大厦村)。15岁时读书于惠山。与元稹、白居易共倡“新乐府” 诗体,史称“新乐府运动”。元和元年(806年)进士,补国子监助教。润州观察使李锜聘为从事,不随其叛乱,拜右拾遗。元和七年担任校书郎。历官翰林学士,转任右补阙,与李德裕、元稹同时号“三俊”,后卷入牛李党争。长庆元年(821)三月,改为司勋员外郎、知制诰。二年二月,破格升任中书舍人,入中书省。长庆四年(824年)李党失势,受李逢吉排挤被贬为端州(今广东肇庆)司马,宝历元年(825年)改任江州(今江西九江市)刺史,不久迁滁州、寿州刺史,又改授太子宾客分司东都。太和七年,李德裕为相,任浙东观察使,开成元年(836年)任河南尹,历任汴州刺史、宣武军节度使、宋亳汴颖观察使。开成五年(840年)任淮南节度使。不久入京拜相,官至尚书右仆射门下侍郎,封赵国公。 \tabularnewline\hline
  孟郊 & \begin{description}
  \item[字] 东野
  \item[号] 
  \item[谥] 
  \item[尊] 
  \item[生] 浙江德清
  \end{description} & 孟郊(751年-814年),字东野,唐朝湖州武康(今浙江德清)人。现存诗歌500多首,以短篇的五言古诗最多,没有一首律诗。代表作有<游子吟>。祖籍平昌(今山东临邑东北)。先世居洛阳(今属河南),孟郊早年生活贫困,曾游历湖北、湖南、广西等地,无所遇合,屡试不第。贞元中张建封镇守徐州时,孟郊曾往谒见。46岁(一说45岁),始登进士第,有诗《登科后》:“昔日龌龊不足夸,今朝放荡思无涯;春风得意马蹄疾,一日看尽长安花(成语“走马看花”由来)。”。然后东归,旅游汴州(今河南开封)、越州(今浙江绍兴)。贞元十七年(801年),任为溧阳尉。在任不事曹务,常以作诗为乐,被罚半俸。韩愈称他为“酸寒溧阳尉”。元和元年(806年),河南尹郑余庆奏为河南水陆转运从事,试协律郎,定居洛阳。元和三年(808年)为检校兵部尚书,兼东都留守。60岁时,因母死去官。九年三月,郑余庆转任山南西道节度使,镇守兴元,又奏孟郊为参谋、试大理评事。郊应邀前往,到阌乡(今河南灵宝),不幸以暴病去世,孟郊的朋友韩愈等人凑了100贯为他营葬,郑余庆派人送300贯,“为遗孀永久之赖”。张籍私谥为贞曜先生。 \tabularnewline\hline
  贾岛 & \begin{description}
  \item[字] 浪仙\\阆仙
  \item[号] 
  \item[谥] 
  \item[尊] 
  \item[生] 河北涿州
  \end{description} & 贾岛(779年-843年),字浪仙(亦作阆仙),范阳(今河北省涿州市)人,唐朝诗人,与韩愈同时。贾岛贫寒,曾经做过和尚,法号无本。元和五年(810年)冬,至长安,见张籍。据说在洛阳的时候后因当时有命令禁止和尚午后外出,贾岛做诗发牢骚,被韩愈发现其才华。后受教于韩愈,并还俗参加科举,但累举不中第。元和十四年(819年),韩愈抵潮州(今广东潮州),致信贾岛,贾岛作《寄韩潮州愈》诗给韩愈。长庆二年(822年)举进士,以“僻涩之才无所用”。唐文宗的时候被排挤,贬做长江主簿。唐武宗会昌年初由普州司仓参军改任司户,未任病逝。《新唐书》将贾岛附名于《韩愈传》之后。 \tabularnewline\hline
  卢仝 & \begin{description}
  \item[字] 
  \item[号] 玉川子
  \item[谥] 
  \item[尊] 
  \item[生] 河南济源
  \end{description} & 卢仝(795年-835年),自号玉川子,河南济源人,中国唐朝中期诗人。诗风奇诡险怪,人称“卢仝体”,有《玉川子诗集》传世。后为韩愈赏识,韩愈的《月蚀诗效玉川子作》是对卢仝《月蚀诗》进行繁删,体现他对卢仝体的推崇。后卢仝迁居洛阳。元和六年,卢仝在洛阳里仁坊购宅。他好饮茶,一首《走笔谢孟谏议寄新茶》人称“玉川茶歌”,与陆羽茶经齐名。 \tabularnewline\hline
  杜牧 & \begin{description}
  \item[字] 牧之
  \item[号] 樊川
  \item[谥] 
  \item[尊] 杜紫薇
  \item[生] 陕西西安
  \end{description} & 杜牧(803年-852年),字牧之,号樊川,京兆府万年县(今陕西省西安市)人。晚唐著名诗人和古文家。擅长长篇五言古诗和七律。曾任中书舍人(中书省别名紫微省),人称杜紫微。其诗英发俊爽,为文尤纵横奥衍,多切经世之务,在晚唐成就颇高,时人称其为“小杜”,以别于杜甫;又与李商隐齐名,人称“小李杜”。 \tabularnewline\hline
  陆龟蒙 & \begin{description}
  \item[字] 鲁望
  \item[号] 江湖散人\\甫里先生\\天随子
  \item[谥] 
  \item[尊] 
  \item[生] 江苏苏州
  \end{description} & 陆龟蒙(?-881年),字鲁望,唐朝苏州吴县(今属江苏)人,自号江湖散人、甫里先生,又号天随子。陆元方七世孙,其父陆宾虞曾任御史之职。开成元年(836)前后出生,进士不第,曾在湖州、苏州从事幕僚。随湖州刺史张博游历,后来回到了故乡苏州甫里(今江苏吴县东南甪直镇),过着隐居耕读的生活,自号天随子;由于甫里地低下,常苦水潦,乃至饥馑,著有《耒耜经》,是一本农学书;喜爱品茗,在顾渚山下辟一茶园,耕读之余,则喜好垂钓。与皮日休为友,时常在一起游山玩水,饮酒吟诗,世称“皮陆”,二人唱和之作编为《松陵集》十卷。 \tabularnewline\hline
  司空图 & \begin{description}
  \item[字] 表圣
  \item[号] 
  \item[谥] 
  \item[尊] 
  \item[生] 山西永济
  \end{description} & 司空图(837年-908年),字表圣,中国唐朝末年诗人、文学评论家,河中郡虞乡(今山西省永济县)人。司空图早年为王凝赏识,在其推荐下于唐懿宗咸通10年(869年)中进士,后为报恩,放弃在朝中为官的机会,长期居于王凝幕府中。878年,被任命为光禄寺主簿,分司洛阳。在洛阳期间得到卢携的赏识,后卢携回朝复相,司空图被任命为礼部员外郎,不久升任郎中。唐僖宗广明元年(880年),黄巢入长安,司空图拒绝其招揽,逃往凤翔投奔唐僖宗,被任命为知制诰、中书舍人。次年,唐僖宗迁往宝鸡,司空图与其失散,回乡隐居中条山王官谷。唐昭宗及宰相朱温屡次征召其为侍郎、尚书等职,他均坚辞不受,最后接受了宰相柳璨的要求为官,却故意装作衰老的样子,在朝堂上失手坠落笏板,得以放还本乡中条山。907年,朱温废去唐哀帝,建立后梁,次年又将哀帝刺杀。司空图闻信后,绝食而死。 \tabularnewline\hline
  郑谷 & \begin{description}
  \item[字] 守愚
  \item[号] 
  \item[谥] 
  \item[尊] 郑鹧鸪\\郑都官
  \item[生] 江西宜春
  \end{description} & 郑谷(849年-911年),字守愚,江西袁州(今宜春)人,唐代诗人。父郑史为永州刺史。七岁能诗,光启三年进士,官右拾遗,历都官郎中。生逢乱世,际遇坎坷。郑谷与许棠、任涛等九人时相唱和,时称“芳林十哲”,“尝赋鹧鸪,警绝”,故有“郑鹧鸪”的称号。郑谷隐居仰山,有一诗僧齐己以一首《早梅》诗求教,郑谷将诗中“前村深雪里,昨夜数枝开”的“数枝”改为“一枝”,齐己下拜,当时士子又称郑谷称为齐己的“一字之师”。乾宁三年(896年),昭宗避难华州,郑谷亦赴华州,“寓居云台道舍”,因而自称诗集为《云台编》。其作品有《云台编》三卷、《宜阳集》三卷以及《国风正诀》一卷等。 \tabularnewline\hline
  韦庄 & \begin{description}
  \item[字] 端己
  \item[号] 
  \item[谥] 文靖
  \item[尊] 
  \item[生] 陕西西安
  \end{description} & 韦庄(836年-910年),字端己,京兆杜陵(今陕西省西安市)人。晚唐政治家,诗人。广明元年(880年)韦庄在长安应举,黄巢攻占长安以后,与弟妹失散,浪迹天涯。中和三年(883年)三月,在洛阳写有长篇歌行《秦妇吟》。昭宗乾宁元年(894年)进士,曾任校书郎、左补阙等职。乾宁四年(897年),李询为两川宣谕和协使,聘用他为判官。在四川时为王建掌书记,蜀开国制度皆庄所定,官至吏部尚书,同平章事,武成三年(910年)八月,卒于成都花林坊。葬白沙之阳。谥文靖。韦庄是唐朝花间派词人,词风清丽,与温庭筠并称“温韦”。韦庄的弟弟韦蔼所编之《浣花词》流传。《花间集》收四十八首。杨慎《升庵外集》评韦庄词“明白如画,蕴情深至”。况周颐《蕙风词话》称他“尤能运密如疏、寓浓于淡,花间群贤,殆鲜其匹”。近人王国维谓之“骨秀也”,评价更在温庭筠之上。 \tabularnewline\hline
  冯延巳 & \begin{description}
  \item[字] 正中
  \item[号] 
  \item[谥] 忠肃
  \item[尊] 
  \item[生] 广陵
  \end{description} & 冯延巳(903年-960年),原名冯延嗣,是五代时词人,广陵人。字正中,一说名延己,但支持延巳的较多。南唐时官至宰相,是南唐中主李璟的老师。他是南唐吏部尚书冯令额的儿子,弟弟冯延鲁亦是著名文人。死后谥号忠肃,有《阳春集》传世。冯延巳词风清丽,善写离情别绪,有很高的艺术成就,对李煜影响很大。冯延巳、李煜被认为直接影响了北宋以来的词风。有“吹皱一池春水”名句。 \tabularnewline\hline
  毛文锡 & \begin{description}
  \item[字] 平珪
  \item[号] 
  \item[谥] 
  \item[尊] 毛司徒
  \item[生] 不详
  \end{description} & 毛文锡,字平珪,五代十国时期前蜀国人。曾任翰林学士承旨、礼部尚书。曾经力谏前蜀帝王建不要决坝淹没江陵,挽救无数百姓性命。 \tabularnewline\hline
  顾夐 & \begin{description}
  \item[字] 
  \item[号] 
  \item[谥] 
  \item[尊] 顾太尉
  \item[生] 不详
  \end{description} & 顾敻(xìong),五代词人。生卒年、籍贯及字号均不详。前蜀王建通正(916)时,以小臣给事内廷,见秃鹫翔摩诃池上,作诗刺之,几遭不测之祸。后擢茂州刺史。入后蜀,累官至太尉。顾夐能诗善词。 《花间集》收其词55首,全部写男女艳情。 \tabularnewline\hline

  \bottomrule
\end{longtable}


%%% Local Variables:
%%% mode: latex
%%% TeX-engine: xetex
%%% TeX-master: "../../Main"
%%% End:

%% -*- coding: utf-8 -*-
%% Time-stamp: <Chen Wang: 2018-10-30 16:25:56>

\subsection{南北两宋}

\begin{longtable}{|>{\centering\namefont\heiti}m{2em}|>{\centering\tiny}m{3.0em}|>{\xzfont\kaiti}m{7em}|}
  % \caption{秦王政}\
  \toprule
  \SimHei \normalsize 姓名 & \SimHei \normalsize 异名 & \SimHei \normalsize \hspace{2.5em}小传 \tabularnewline
  % \midrule
  \endfirsthead
  \toprule
  \SimHei \normalsize 姓名 & \SimHei \normalsize 异名 & \SimHei \normalsize \hspace{2.5em}小传 \tabularnewline 
  \midrule
  \endhead
  \midrule
  王禹偁 & \begin{description}
  \item[字] 元之
  \item[号] 
  \item[谥] 
  \item[尊] 王黄州
  \item[生] 山东菏泽
  \end{description} & 王禹偁(954年-1001年),字元之,济州钜野(今山东菏泽市钜野县)人,北宋文学家。王禹偁出身清寒,家庭世代务农。从小发愤求学,五岁便能够写诗。宋太宗太平兴国八年(983年)中进士,最初担任成武县主簿。他对仕途充满抱负,曾在《吾志》诗中表白:“吾生非不辰,吾志复不卑,致君望尧舜,学业根孔姬”。端拱元年(988年),他被召见入京,担任右拾遗、直史馆。他旋即进谏,以《端拱箴》来批评皇宫的奢侈生活。后来历任左司谏、知制诰、翰林学士。为人刚直,敢直言进谏,誓言要“兼磨断佞剑,拟树直言旗”。曾三次被贬职:于淳化二年(991年),一贬商州,于至道元年,二贬滁州,于咸平元年(998年),三贬黄州。故有“王黄州”之称。谪居黄州期间,以骈散相间之"黄州新建小竹楼记"抒发虽遭贬谪却心地坦荡,具达观旷逸之胸怀。 \tabularnewline\hline
  寇准 & \begin{description}
  \item[字] 平仲
  \item[号] 
  \item[谥] 忠愍
  \item[尊] 寇莱公
  \item[生] 陕西渭南
  \end{description} & 寇准(961年-1023年10月24日),北宋名相。字平仲,华州下邽(今陕西渭南)人。寇凖系春秋司寇苏氏裔孙,其父寇湘,后晋开运中,很有学问,应辟为魏王府记室参军。寇准出生于山西太谷县,年少时期豪爽嗜酒,性格大方,喜欢在家里大摆筵席。[注 1] 。皇祐四年,诏翰林学士孙抃撰神道碑,帝为篆其首曰“旌忠”。寇凖善诗能文,七绝尤有韵味,今传《寇忠湣诗集》三卷。 \tabularnewline\hline
  丁谓 & \begin{description}
  \item[字] 谓之\\公言
  \item[号] 
  \item[谥] 
  \item[尊] 丁晋公
  \item[生] 江苏苏州
  \end{description} & 丁谓(966年-1037年),字谓之,后更字公言,北宋时期苏州长州(今江苏苏州)人。善言谈,喜欢作诗,于图书、博奕、音律无一不精。出自寇准门下,太宗淳化三年(992年)进士,授大理寺评事、通判饶州事。真宗咸平初除三司户部判官,大中祥符初,权三司使。咸平五年(1012年)任户部侍郎。官至参政知事。当政后极力排斥寇准,乾兴元年(1022年)二月,再贬寇准为雷州司户参军。丁谓同党雷允恭因先帝陵寝工程事故,坐“擅移皇堂”罪,丁谓受牵连,贬为太子太保。后以“丁谓前后欺罔”罪,被贬崖州(今海南省琼山县)司户参军。以秘书省监致仕归里。景祐四年(1037年)病卒。著有《丁谓集》8卷、《虎丘集》50卷、《刀笔集》2卷、《青衿集》3卷、《知命集》1卷,皆佚。所著《天香传》,则是中国最早系统性针对沉香尤其是海南沉香所写的专著,书中记载沉香自古便为人所用,最早用于祭天礼地的场合,焚沉香祝祷。丁谓实地考察海南沉香,提出了四名十二状的分类法,被后世多人藉鉴。 \tabularnewline\hline
  陈尧佐 & \begin{description}
  \item[字] 希元
  \item[号] 
  \item[谥] 文惠
  \item[尊] 
  \item[生] 四川南充
  \end{description} & 陈尧佐(963年—1044年),字希元,阆州阆中新井县(今四川省南充市南部县)人。北宋大臣,官至同中书门下平章事、集贤殿大学士,太子太师致仕,追赠司空兼侍中,谥号“文惠”。 陈尧佐,字希元,号知余子。阆州阆中人。北宋大臣、水利专家、书法家、诗人。左谏议大夫陈省华次子,枢密使陈尧叟之弟。陈尧佐与长兄陈尧叟、弟陈尧咨皆中状元。端拱元年(988年),陈尧佐进士及第,授魏县、中牟县尉。咸平初年,任潮州通判。历官翰林学士、枢密副使、参知政事。宋仁宗时官至宰相,景祐四年(1037年),拜同中书门下平章事。康定元年(1040年),以太子太师致仕。庆历四年(1044年),陈尧佐去世,年八十二,赠司空兼侍中,谥号“文惠”。陈尧佐明吏事,工书法,喜欢写特大的隶书字,著有《潮阳编》、《野庐编》、《遣兴集》、《愚邱集》等。 \tabularnewline\hline
  林逋 & \begin{description}
  \item[字] 君复
  \item[号] 
  \item[谥] 和靖先生
  \item[尊] 
  \item[生] 浙江杭州
  \end{description} & 林逋(967年或968年─1028年),汉族,北宋诗人。字君复,后人称为和靖先生,钱塘人(今浙江杭州)。出生于儒学世家,恬淡好古,早年曾游历于江淮等地,隐居于西湖孤山,终身不仕,未娶妻,与梅花、仙鹤作伴,称为“梅妻鹤子”。宋真宗闻其名,赐粟帛,诏长吏岁时劳问。性孤高自好,喜恬淡,不趋名利,自谓:“然吾志之所适,非室家也,非功名富贵也,只觉青山绿水与我情相宜。”林逋善为诗,其词澄浃峭特,多奇句。其诗大都反映隐居生活,描写梅花尤其入神,苏轼高度赞扬林逋之诗、书及人品,并诗跋其书:“诗如东野不言寒,书似留台差少肉。”宋仁宗天圣六年(1028年)去世,享寿六十二岁,仁宗赐谥“和靖先生”。留有《林和靖诗集》。宋代桑世昌著有《林逋传》。 \tabularnewline\hline
  杨亿 & \begin{description}
  \item[字] 大年
  \item[号] 
  \item[谥] 文
  \item[尊] 杨文公
  \item[生] 福建浦城
  \end{description} & 杨亿(974年-1020年),字大年,人称杨文公。建州浦城(今属福建浦城县)人。北宋文学家。自幼是个神童,博览强记,太宗雍熙元年(984年),十一岁受宋太宗召试,授秘书省正字(掌管图书秘籍的次长),淳化三年(992年)赐进士及第,迁光禄寺丞。淳化四年,直集贤院。至道二年(996年)迁著作佐郎。大中祥符六年(1013年)以太常少卿分司西京。天禧二年(1018年)拜工部侍郎。官至工部侍郎。以“秉清节”自许,“性特刚劲寡合”,为“忠清鲠亮之士”。又好谈禅。又好写诗,善于西昆体,朱熹评之为“巧中犹有混成底意思,便巧得来不觉”,与刘筠、钱惟演等诗歌唱和,其编著《西昆酬唱集》,收录十七位诗人作品,共250首,多言学李商隐,不喜杜工部诗,谓为村夫子。杨亿曾为翰林学士兼史馆修撰。长于典章制度,真宗即位初,曾参预修《太宗实录》,咸平元年(998年)书成,景德二年(1005年)与王钦若主修《册府元龟》。在政治上支持丞相寇准抵抗辽兵入侵。又反对宋真宗大兴土木。卒谥文,故称杨文公。著作多佚,今存《武夷新集》20卷。《宋史》卷三○五有传。清人全祖望有《杨文公论》。 \tabularnewline\hline
  张先 & \begin{description}
  \item[字] 子野
  \item[号] 
  \item[谥] 
  \item[尊] 张三影\\张郎中
  \item[生] 浙江湖州
  \end{description} & 张先(990年-1078年),字子野,湖州乌程(今浙江湖州吴兴)人,因张先曾在安陆郡(今湖北省安陆市)任职多年,人亦称张安陆,为北宋著名婉约派词人。父张维,好读书。张先是天圣八年进士,授汉阳军司理参军,调河南法曹参军,改著作佐郎知阆中县,代还,拜秘书丞,知亳州鹿邑县。与欧阳修友好。官至尚书都官郎中,著有《安陆集》一卷。清代侯文灿、葛鸣阳、鲍廷博等人根据宋人手抄本编纂《张子野词》。宝元二年卒,年八十八。 \tabularnewline\hline
  李冠 & \begin{description}
  \item[字] 世英
  \item[号] 
  \item[谥] 
  \item[尊] 
  \item[生] 山东济南
  \end{description} & 约公元1019年前后在世,字世英,齐州历城(今山东济南)人。生卒年均不详,约宋真宗天禧中前后在世。与王樵、贾同齐名;又与刘潜同时以文学称京东。举进士不第,得同三礼出身,调乾宁主。冠著有《东皋集》二十卷,不传。存词五首。《宋史本传》传于世。 沈谦《填词杂说》赞其《蝶恋花》“数点雨声风约住,朦胧淡月云来去”句,以为“‘红杏枝头春意闹’,‘云破月来花弄影’俱不及”。 \tabularnewline\hline
  晏殊 & \begin{description}
  \item[字] 同叔
  \item[号] 
  \item[谥] 元献
  \item[尊] 晏元献
  \item[生] 江西南昌
  \end{description} & 晏殊(991年-1055年2月27日),字同叔,抚州临川文港乡(今南昌进贤县)人。北宋著名词人晏几道父亲,世称晏殊为大晏,晏几道为小晏。为北宋前期著名婉约派词人,与欧阳修并称“晏欧”。晏殊自幼聪颖,七岁能文,十四岁时因宰相张知白推荐,以神童召试,被朝廷赐同进士出身,之后到秘书省做正字。宋仁宗康定初(1040年),官至同平章事兼枢密使,位同宰相,掌军政大权。仁宗至和二年(1055年)正月二十八日病逝,年六十五,封临淄公,谥号元献,世称晏元献。性刚简,自奉清俭,好燕饮。能荐拔人才,号称贤相,王安石、范仲淹、欧阳修均出其门下。 \tabularnewline\hline
  石延年 & \begin{description}
  \item[字] 曼卿
  \item[号] 
  \item[谥] 
  \item[尊] 
  \item[生] 河南商丘
  \end{description} & 石延年(994年-1041年),字曼卿,宋代的文学家和书法家。先世幽州(今河北省涿县)人,后迁宋州宋城(今河南省商丘市)。不拘礼法,不慕名利。屡试不中,宋真宗时,因为三举进士不中,最后补三班奉职(从九品下,俸钱七百文)。历官知金乡县,累迁大理寺丞。好饮酒,有时披头散发,双手要带着枷锁,称“囚饮”;有时爬到树上去饮,曰“巢饮”;有时用稻麦杆束身,伸出头来与人对饮,称作“鳖饮”;有时和朋友摸黑饮酒,称作“鬼饮”。在海州任通判时,与刘潜曾在王氏酒楼喝酒,从早饮到晚,不发一言,隔日,京城传出昨日王氏酒楼有二神仙来饮酒。和杜默、欧阳修合称“三豪”。 \tabularnewline\hline
  宋祁 & \begin{description}
  \item[字] 子京
  \item[号] 
  \item[谥] 景文
  \item[尊] 红杏尚书
  \item[生] 河南杞县
  \end{description} & 宋祁(998年-1061年),字子京,安陆(今属湖北)人,徙居开封雍丘(今河南杞县),中国北宋文学家、史学家。与其兄宋庠诗文齐名,时呼“小宋”、“大宋”,合称“二宋”。著有《宋景文公集》。宋祁处于北宋阶级矛盾的时期,宝元二年(1038年)时任同判礼院,上疏认为国用不足在于“三冗三费”,三冗即冗官、冗兵、冗僧,三费是道场斋醮、多建寺观、靡费公用。主张裁减官员,节省经费,宰相吕夷简指责他是朋党,并加以打击。 \tabularnewline\hline
  梅尧臣 & \begin{description}
  \item[字] 圣俞
  \item[号] 
  \item[谥] 
  \item[尊] 宛陵先生
  \item[生] 安徽宣城
  \end{description} & 梅尧臣(1002年-1060年),字圣俞,宣城(今安徽宣城)人,世称宛陵先生。北宋著名现实主义诗人。50岁后,始得宋仁宗召试,赐同进士出身,后任授国子监直讲,迁尚书屯田都官员外郎,故时称“梅直讲”、“梅都官”。梅尧臣少即能诗,与苏舜钦齐名,世人美称“苏梅”,同被誉为宋诗“开山祖师”。与欧阳修为挚友,同为宋诗革新推动者。晚年曾参与编撰《新唐书》。嘉祐五年(1060年)京师有大疫,四月以疾卒。 \tabularnewline\hline
  苏舜钦 & \begin{description}
  \item[字] 子美
  \item[号] 
  \item[谥] 
  \item[尊] 苏学士
  \item[生] 河南开封
  \end{description} & 苏舜钦(1009年-1049年),字子美,开封(今属河南)人,曾祖父苏协由梓州铜山(今四川中江)移家开封(今属河南)。父亲苏耆,母亲为王雍、王冲、王素之姐。宋仁宗景祐元年(1035年)进士。历任蒙城、长垣县令,庆历三年因母丧守制,后入大理评事、集贤校理、监进奏院等职。杜衍以女嫁之,进奏院祠神,售废纸公钱宴会。因参加范仲淹为首的革新集团,为人所弹劾,以“监守自盗罪”削职为民,闲居苏州沧浪亭。后再起用为湖州长史,庆历八年(1048年)十二月卒。 \tabularnewline\hline
  邵雍 & \begin{description}
  \item[字] 尧夫
  \item[号] 安乐先生
  \item[谥] 康节
  \item[尊] 百源先生
  \item[生] 河南辉县
  \end{description} & 邵雍(1011年1月21日-1077年7月27日[1]),字尧夫,自号安乐先生,人又称百源先生,谥康节,后世称邵康节,北宋五子之一,易学家、思想家、诗人。雍青年时期即有好学之名,《宋史》记载:“雍少时,自雄其才,慷慨欲树功名。于书无所不读,始为学,即坚苦刻厉,寒不炉,暑不扇,夜不就席者数年。已而叹曰:‘昔人尚友于古,而吾独未及四方。’于是逾河、汾,涉淮、汉,周流齐、鲁、宋、郑之墟,久之,幡然来归,曰:‘道在是矣。’遂不复出。”雍后居洛阳,与司马光、二程、吕公著等交游甚密。邵雍与二程、周敦颐、张载,合称为“北宋五子”。 \tabularnewline\hline
  曾巩 & \begin{description}
  \item[字] 子固
  \item[号] 
  \item[谥] 文定
  \item[尊] 曾南丰
  \item[生] 江西南丰
  \end{description} & 曾巩(1019年9月30日-1083年4月30日),字子固,建昌南丰(今江西南丰)人,汉族江右民系,北宋散文家,被誉为“唐宋八大家”之一。曾巩的文体风格为“古雅平正”,擅长引经据典;结构则平易理醇,章法开阖、承转、起伏、回环都有一定约束法度、严密、规矩。正因为其文章易于模仿和学习,他成为了唐宋文派和桐城派学习的首要对象。 \tabularnewline\hline
  晏几道 & \begin{description}
  \item[字] 叔原
  \item[号] 小山
  \item[谥] 
  \item[尊] 小晏
  \item[生] 南昌进贤
  \end{description} & 晏几道(1037年-1110年),字叔原,号小山,晏殊第七子。北宋婉约派词人。抚州临川文港乡(今属南昌进贤县)人。以父荫赐进士出身,历官开封府判官、颍昌府许田镇监、乾宁军通判等。一般讲到北宋词人时,称晏殊为大晏,称晏几道为小晏。《雪浪斋日记》云:“晏叔原工小词,不愧六朝宫掖体。”《鹧鸪天》中“舞低杨柳楼心月,歌尽桃花扇底风。”两句受人赞赏。晏几道的词经常都是多愁善感。可能与他晚年家道中落有关,他在《小山词自序》中回忆说:“追惟往昔过从饮酒之人,或垅木已长,或病不偶。考其篇中所记悲欢离合之事,如幻,如电,如昨梦前尘,但能掩卷怃然,感光阴之易迁,叹境缘之无实也!”《全宋词》存录有二百六十余首。 \tabularnewline\hline
  晁炯 & \begin{description}
  \item[字] 
  \item[号] 
  \item[谥] 文元
  \item[尊] 
  \item[生] 山东
  \end{description} & 真宗时晁炯声名显赫,此后,“晁氏自迥以来,家传文学,几于人人有集”。 \tabularnewline\hline
  刘攽 & \begin{description}
  \item[字] 贡父
  \item[号] 公非
  \item[谥] 
  \item[尊] 
  \item[生] 樟树市
  \end{description} & 刘攽(1022年-1088年),字贡父(一作戆父,或赣父),号公非。樟树市黄土岗镇荻斜刘家人。北宋史学家,著有《彭城集》。《资治通鉴》副主编之一。先世为彭城人,西晋末年,避胡兵乱,迁居江南,又迁庐陵。刘攽好谐谑,庆历六年(1046年)贾黯榜进士。历任汝州推官,至和二年乙未(1055年)调江阴县主簿,嘉祐二年丁酉(1057年)担任庐州推官等。历州县官二十年,嘉祐八年癸卯(1063年),入京为国子监直讲,迁馆阁校勘。宋神宗熙宁初年同知太常礼院,以反对新法出知曹州。博览群书,精于史学,助司马光修《资治通鉴》,专治汉史部分。元丰八年(1085年),由衡州盐仓起知襄州,元祐初年召拜中书舍人。四年卒,年六十七。 \tabularnewline\hline
  沈括 & \begin{description}
  \item[字] 存中
  \item[号] 梦溪丈人
  \item[谥] 
  \item[尊] 
  \item[生] 浙江杭州
  \end{description} & 沈括(1031年-1095年),字存中,号梦溪丈人,是中国北宋科学家、杭州钱塘县(今浙江省杭州市)人,随母寿昌县太君许氏入籍苏州吴县(今江苏省苏州市)。沈括在物理学、数学、天文学、地学、生物医学等方面都有重要的成就和贡献,在化学、工程技术等方面也有相当的成就。此外,沈括在文学、音乐、艺术、史学等方面都有一定的造诣。《宋史·沈括传》称他“博学善文,于天文、方志、律历、音乐、医药、卜算无所不通,皆有所论著”。沈括突出的成就主要集中在《梦溪笔谈》中。 \tabularnewline\hline
  文同 & \begin{description}
  \item[字] 与可
  \item[号] 笑笑先生
  \item[谥] 
  \item[尊] 石室先生\\文湖州
  \item[生] 四川盐亭
  \end{description} & 文同(1018年-1079年),字与可,自号笑笑先生或笑笑居士,人称石室先生,四川梓州永泰(今四川盐亭县东北面)人。文同历官邛州、洋州等知州,元丰初出知湖州,未到任而死,人称“文湖州”。曾参与校对《新唐书》。善画墨竹,他的表弟苏轼曾称赞他为诗、词、画、草书四绝,苏轼画竹受其影响,学他的人很多,有“湖州竹派”之称。成语“胸有成竹”正是从他画竹而来。 \tabularnewline\hline
  贺铸 & \begin{description}
  \item[字] 方回
  \item[号] 庆湖遗老
  \item[谥] 
  \item[尊] 贺鬼头\\贺梅子
  \item[生] 浙江绍兴
  \end{description} & 贺铸(1052年-1125年),字方回,号庆湖遗老。越州山阴(今浙江绍兴)人,生长于卫州(治今河南卫辉)。北宋词人。著有《东山词》2卷,《东山词补》1卷,今存词200余首。其词风格多样,字句锤炼,常借用古乐府、及唐人诗句入词,作品多写艳情及闺情离思,也描写世间沧桑,嗟叹功名不就,亦有个人闲愁、纵酒狂放之作。代表作为《青玉案》、《六州歌头》。其词风婉约而豪放。 \tabularnewline\hline
  潘大临 & \begin{description}
  \item[字] 君孚\\邠老
  \item[号] 
  \item[谥] 
  \item[尊] 
  \item[生] 湖北黄州
  \end{description} & 潘大临(生卒年不详),字君孚,一字邠老,原籍长乐三溪,黄州(今属湖北)吝安镇人。北宋著名诗人。大临与其弟潘大观皆有诗名。元丰三年(1080年)苏东坡贬黄州,二月与潘鲠、潘丙有交往;张耒谪黄州时,多有交往。 \tabularnewline\hline
  陈师道 & \begin{description}
  \item[字] 履常\\无己
  \item[号] 后山居士
  \item[谥] 
  \item[尊] 
  \item[生] 江苏徐州
  \end{description} & 陈师道(1053年-1101年),字履常,一字无己,别号后山居士,彭城(今江苏徐州)人,北宋诗人。师道一生淡薄名利,闭门苦吟,有“闭门觅句陈无己”之称。苏门六君子之一,常与苏轼、黄庭坚等唱和,见黄庭坚之诗,爱不释手,把自己的旧作全部烧掉,重学黄诗,后致力于学杜甫;方回的《瀛奎律髓》有“一祖三宗”之说,即以杜甫为祖,三宗便是黄庭坚、陈师道和陈与义。著有《后山集》、《后山谈丛》、《后山诗话》等。门人魏衍编有《彭城陈先生集》二十卷。 \tabularnewline\hline
  张耒 & \begin{description}
  \item[字] 文潜
  \item[号] 柯山
  \item[谥] 
  \item[尊] 
  \item[生] 江苏淮安
  \end{description} & 张耒(1054年-1114年),字文潜,号柯山,生于楚州淮阴(今江苏淮安市),祖籍亳州谯县(今安徽亳县)。北宋诗人。早年游学陈州,受到当时学官苏辙厚爱,从学于苏轼,苏轼说他的文章类似苏辙,“汪洋淡泊,有一唱三叹之声”。张耒嗜酒,晚年有疾。其诗学白居易、张籍,如:《田家》、《海州道中》、《输麦行》多反映下层人民的生活以及自己的生活感受,风格平易晓畅。他与黄庭坚、秦观、晁补之三人一同被时人誉为“苏门四学士”。编《苏门六君子文粹》,有四库全书版本。 \tabularnewline\hline
  蔡绦 & \begin{description}
  \item[字] 约之
  \item[号] 无为子\\百衲居士
  \item[谥] 
  \item[尊] 
  \item[生] 不详
  \end{description} & 蔡絛,字约之,别号无为子、百衲居士。蔡京之季子。徽宗宣和六年(1124年),蔡京担任太师,起领三省,因年老不能事事,奏判悉取决于蔡絛。宣和七年,赐进士出身,不久勒令停止,官至徽猷阁待制。靖康元年(1126年),蔡京垮台后,其子孙二十三人被流放,蔡絛亦遭到流放邵州,再改白州(广西博白),死于戌所。著有《国史后补》、《北征纪实》、《铁围山丛谈》、《西清诗话》及《蔡百衲诗评》等。 \tabularnewline\hline
  魏泰 & \begin{description}
  \item[字] 道辅
  \item[号] 汉上丈人
  \item[谥] 
  \item[尊] 
  \item[生] 湖北襄樊
  \end{description} & 魏泰,字道辅,号汉上丈人,晚号临汉隐居,北宋襄阳邓城(今湖北省襄樊市)人,生卒年不详,约活动于宋神宗、哲宗、徽宗时期。出生世族,为北宋著名女词人魏芷之弟。著有《东轩笔录》十一卷、《临汉隐居集》二十卷、《临汉隐居诗话》一卷、《东轩笔录》十五卷、《襄阳形胜赋》、《续录》一卷,是一本记载宋太祖至神宗六朝旧事的笔记,今存者唯笔录、诗话及诗四首。 \tabularnewline\hline
  范温 & \begin{description}
  \item[字] 元实
  \item[号] 
  \item[谥] 
  \item[尊] 
  \item[生] 不详
  \end{description} & 范温,字元实,范祖禹之子,秦少游之婿,吕居仁之表叔,曾学诗于黄庭坚。著有《潜溪诗眼》一卷。 \tabularnewline\hline
  唐庚 & \begin{description}
  \item[字] 子西
  \item[号] 
  \item[谥] 
  \item[尊] 
  \item[生] 四川眉山
  \end{description} & 唐庚(1071年-1121年),字子西,眉州丹棱(今属四川)人。唐庚是哲宗绍圣元年(1094年)进士,利州司法参军,为宰相张商英所赏识。绍圣四年(1110年),除京畿路提举常平。张商英罢相后,被贬惠州。政和七年(1117年)还京,提举上清太平宫。宣和三年(1121年)归返四川,卒于途中。著有《眉山诗集》、《眉山文集》,清四库全书集部存录本。 \tabularnewline\hline
  惠洪 & \begin{description}
  \item[字] 德洪
  \item[号] 觉范
  \item[谥] 
  \item[尊] 
  \item[生] 江西高安
  \end{description} & 惠洪(1071年-1128年),名德洪,号觉范,俗姓彭。北宋筠州(今江西高安)人。出生于今宜丰县桥西盐岭下竹园彭家,族叔彭几,官至协律郎。元丰七年(1084年)父母双亡,至县城北郊三峰山宝云寺为童子,元祐四年(1089年),参加东京天王寺佛经考试,冒惠洪名得剃度为僧。后依真净禅师,迁往洪州石门寺。后还俗。黄庭坚曾教他读书,与尚书右仆射张商英和节度使郭天信有往来。政和元年(1111年),因张商英一案牵连,流放朱崖(广东海口市)。三年后赦还,居筠州。建炎二年(1128年)逝世于新昌。惠洪长于诗文,“觉范斯须立就”,《彦周诗话》说:“颇似文章巨公所作,殊不类衲子。”被推为“宋僧之冠”,王安石女儿称其“浪子和尚”。 \tabularnewline\hline
  韩驹 & \begin{description}
  \item[字] 子苍
  \item[号] 牟阳
  \item[谥] 
  \item[尊] 陵阳先生
  \item[生] 四川井研
  \end{description} & 韩驹(1080年~1135年),字子苍,号牟阳,陵阳仙井(今四川井研)人。南宋初诗人,世称陵阳先生。少有文名,黄庭坚称其诗“超轶绝尘”。韩驹于元丰三年(1080年)出生,早年在许下从苏辙学,苏辙称读其诗“恍然重见储光羲”。韩驹是江西诗派人物,曾季狸《艇斋诗话》:“后山(陈师道)论诗说换骨,东湖(徐俯)论诗说中的,东莱(吕本中)论诗说活法,子苍论诗说饱参。”。晚年以为“学古人尚恐不至,况学今人哉!”。绍兴五年(1135年)卒于抚州(今江西临川),得年五十六岁。今存《陵阳先生诗》四卷。《宋史》卷四四五有传。 \tabularnewline\hline
  周紫芝 & \begin{description}
  \item[字] 小隐
  \item[号] 竹坡居士
  \item[谥] 
  \item[尊] 
  \item[生] 安徽宣城
  \end{description} & 周紫芝(1082年-1155年),字小隐,号竹坡居士。宣城(今属安徽)人, 南宋文学家、官员。少时家贫,勤学不辍,绍兴十二年(1142年)进士。历官枢密院编修, 绍兴十七年(1147年)为右迪功郎敕令所删定官。二十一年四月出京知兴国军(今湖北阳新县),为政简静,晚年隐居九江庐山。工于诗,不引典故,谀颂秦桧父子,为时论所嘲。约卒于绍兴末年。著有《太仓稊米集》、《竹坡诗话》、《竹坡词》。 \tabularnewline\hline
  吕本中 & \begin{description}
  \item[字] 居仁
  \item[号] 紫薇
  \item[谥] 
  \item[尊] 
  \item[生] 东莱先生
  \end{description} & 吕本中(1084年-1145年),初名大中,字居仁,号紫微、东莱,寿州(今安徽寿县)人。生于宋神宗元丰七年(1084年),是道学家,学者称之为“东莱先生”。著有《东莱先生诗集》、《江西诗社宗派图》、《紫微诗话》及《童蒙诗训》等。绍兴八年(1145年),卒于上饶。《宋史》卷376有传。 \tabularnewline\hline
  葛立方 & \begin{description}
  \item[字] 常之
  \item[号] 懒真子
  \item[谥] 
  \item[尊] 
  \item[生] 江苏江阴
  \end{description} & 葛立方(-1165年),字常之,号懒真子。南宋江阴人。葛密之孙,葛胜仲之子,母张濩。随父徙居吴兴。绍兴八年进士及第,历官左奉议郎、诸王宫大小学教授、太常博士。十七年,除秘书省正字。十九年,迁校书郎。二十一年,为尚书考功员外郎兼中书舍人。官至吏部侍郎。因得罪秦桧,被逼退出官场。绍兴二十六年归休于吴兴汛金溪上。他“博极群书,以文章名一世”,曾著有《韵语阳秋》二十卷、《西畴笔耕》五十卷、《方舆别志》二十卷、《归愚集》一卷,今存《韵语阳秋》与《归愚集》。《四库全书总目提要》评:“多平实铺叙,少清新宛转之思,然大致不失宋人规格。”隆兴二年卒。 \tabularnewline\hline
  周邦彦 & \begin{description}
  \item[字] 美成
  \item[号] 清真居士
  \item[谥] 
  \item[尊] 
  \item[生] 浙江杭州
  \end{description} & 周邦彦(1056年-1121年),中国北宋末期著名的词人,音乐家,字美成,号清真居士,钱塘(今浙江杭州)人。据记载他少年时期个性比较疏散,但相当喜欢读书,宋神宗时,他写了一篇《汴都赋》,赞扬新法,因此由诸生擢为太学正,任教太学。当上学正后,常有积极作为,但在仕途上并没有得意的成果,长期在州县间担任小官职。倒是词愈写愈受世人喜爱,加上精通音律,能自创新曲,词名愈来愈大。到宋徽宗时,周邦彦升为徽猷阁待制,并提举大晟府,任命周邦彦担任主管,从事审订古调,讨论古音,并创设许多音律,影响后世很大。徽宗时期是他作品最多的时期,大部分都带有他华美、轻狂的特质。长期被后人尊为“词家之冠”。 \tabularnewline\hline
  李清照 & \begin{description}
  \item[字] 
  \item[号] 易安居士
  \item[谥] 
  \item[尊] 
  \item[生] 山东济南
  \end{description} & 李清照(1084年3月13日-1155年5月12日),北宋齐州(今山东省济南市)人,为中国历史上最著名的女词人。自号易安居士,与辛幼安并称“济南二安”;又因其词有“新来瘦,非干病酒,不是悲秋”《凤凰台上忆吹箫》、“知否?知否?应是绿肥红瘦”《如梦令》、“莫道不销魂。帘卷西风,人比黄花瘦”《醉花阴》三句,故人称“李三瘦”。有《易安居士文集》七卷、《易安词》八卷,皆佚散。现有《漱玉词》的辑本,存其作约五十首。 \tabularnewline\hline
  徐俯 & \begin{description}
  \item[字] 师川
  \item[号] 东湖居士
  \item[谥] 
  \item[尊] 
  \item[生] 江西修水
  \end{description} & 徐俯(1075年-1141年),字师川,号东湖居士,洪州分宁(今江西修水)人。黄庭坚之甥,父徐禧死于宋夏战争。元丰末年,袭父爵授通直郎,后升司门郎,累官右谏议大夫。靖康元年(1126年),金人围汴京(今河南开封),次年攻陷东京,靖康二年(1127年)张邦昌僭位,徐俯辞归。入江西诗社,与董颖、韩驹等有往来。吕本中《江西诗社宗派图》列其名。建炎初年,内侍郑谌极赏识徐俯文才,向高宗荐举,胡直儒、汪藻等亦荐之,绍兴二年(1132年)赐进士出身。绍兴三年(1133年)升迁为翰林学士,再擢拔为端明殿学士。官至参知政事。因与赵鼎不合去职。绍兴九年,知信州,被劾不理郡事,又被罢免。晚年提举洞霄宫,绍兴十一年终老德兴天门村。著有《东湖诗集》六卷。 \tabularnewline\hline
  陈与义 & \begin{description}
  \item[字] 去非
  \item[号] 简斋
  \item[谥] 
  \item[尊] 
  \item[生] 河南洛阳
  \end{description} & 陈与义(1090年-1138年),字去非,号简斋,洛阳(今属河南)人。徽宗政和三年(1113年)甲科进士,授开德府教授。宣和四年(1122年)擢太学博士、著作佐郎。宋室南渡后,避乱于襄汉。高宗建炎四年(1130年),召为兵部员外郎。绍兴元年(1131年)迁中书舍人。绍兴五年(1135年),召为给事中。绍兴六年(1136年),拜翰林学士。绍兴八年(1138年),以资政殿学士知湖州,因病卒。有《简斋集》三十卷。 \tabularnewline\hline
  王铚 & \begin{description}
  \item[字] 性之
  \item[号] 汝阴老民
  \item[谥] 
  \item[尊] 
  \item[生] 安徽阜阳
  \end{description} & 王铚(?-1144年),字性之,自号汝阴老民。汝阴(今安徽阜阳)人。北宋学者王昭素五世孙,父王萃师事欧阳修。王铚约生于元祐初年,幼而博学,读书一目十行,尝从欧阳修学习。大观元年,王铚访曾布于京口,曾布以三子曾纡之女儿嫁之。南渡后寓居剡中,绍兴初年,官迪功郎,高宗建炎四年(1130年),权枢密院编修官。绍兴四年(1134年)撰成《枢庭备检》,为右承事郎。绍兴五年乙卯(1135年),以右承事郎主管江州庐山太平观。绍兴七年,遭秦桧排挤,避居剡溪山,以诗词自娱。世称雪溪先生,绍兴九年(1139)正月,献《元祐八年补录》及《七朝史》,由右承郎迁右宣义郎。绍兴十三年癸亥(1143年),献《太玄经解义》,绍兴十四年(1144年)卒。著有《默记》一卷、《杂纂续》一卷、《侍儿小名录》一卷、《国老谈苑》二卷等书。 \tabularnewline\hline
  吴沆 & \begin{description}
  \item[字] 德远
  \item[号] 环溪
  \item[谥] 
  \item[尊] 文通先生
  \item[生] 江西抚州
  \end{description} & 吴沆,字德远,抚州崇仁(今属江西)人。兄弟吴涛、吴澥皆有文名。吴沆少年学《易经》,绍兴十六年(1146年)与弟吴澥献著作《易璇玑》﹑《三坟训义》,入国子监,太学博士王之望驳其《三坟训义》之说。后以书法犯下庙讳罢归。隐居环溪,好读杜甫诗,认为杜诗最明显的特色是一句说多件事。卒后其弟子私谥文通先生。后人辑有《环溪诗话》一卷。 \tabularnewline\hline
  范成大 & \begin{description}
  \item[字] 致能
  \item[号] 石湖居士
  \item[谥] 文穆
  \item[尊] 
  \item[生] 江苏苏州
  \end{description} & 范成大(1126年-1193年),字致能,号石湖居士,谥文穆,吴郡(今江苏苏州)人。宋代绍兴二十四年(1154年)中进士,初授司户参军,历官监“和剂局”、检讨、编修、正字、校书郎、处州知州、礼部员外郎、祈请国信使、集英殿修撰、出知静江府、广西经略使、敷文阁待制、四川制置使、礼部尚书、资政殿学士等,官至参知政事,追赠少师、崇国公。范成大曾出使金国,在金国气节不屈,撼动了金世宗,有日记《揽辔录》。范成大与杨万里、尤袤、陆游号称“南宋四大诗人”。范成大的诗作在宋代即有显著影响,到清初则影响尤大,有“家剑南而户石湖”(“剑南”指陆游)之说,其诗风格轻巧,但好用僻典、佛经。范成大同时还是著名的词作家、旅游作家,另有《石湖诗集》、《石湖词》、《桂海虞衡志》、《骖鸾录》、《吴船录》、《吴郡志》等著作传世。 \tabularnewline\hline
  杨万里 & \begin{description}
  \item[字] 廷秀
  \item[号] 诚斋
  \item[谥] 
  \item[尊] 
  \item[生] 江西吉水
  \end{description} & 杨万里(1127年10月29日-1206年6月15日),字廷秀,号诚斋,吉水(今江西省吉水县)人,官至宝谟阁学士。一生力主抗金,与尤袤、范成大、陆游合称南宋“中兴四大诗人”。与欧阳修、杨邦乂、胡铨、周必大、文天祥,合称庐陵“五忠一节”。其诗起初模仿江西诗派,后尽焚少时千余首作品,而另辟蹊径。他在《荆溪集自序》自述:“余之诗,始学江西诸君子,既又学后山(陈师道)五字律,既又学半山老人(王安石)七字绝句,晚乃学绝句于唐人。……戊戌作诗,忽若有悟,于是辞谢唐人及王、陈、江西诸君子皆不敢学,而后欣如也。”终于自成一家,即严羽《沧浪诗话》所谓“诚斋体”。诚斋体的特色是富于幽默诙谐、活泼自然,一反“江西诗派”的生硬槎桠。此对当时诗坛风气之转变,颇起作用。 \tabularnewline\hline
  朱熹 & \begin{description}
  \item[字] 元晦\\仲晦
  \item[号] 晦庵\\考亭\\晦翁
  \item[谥] 文
  \item[尊] 朱子
  \item[生] 江西上饶
  \end{description} & 朱熹(1130年10月22日-1200年4月23日),字元晦,一字仲晦,斋号晦庵、考亭,晚称晦翁,又称紫阳先生、紫阳夫子、沧州病叟、云谷老人,行五十二,小名沋郎,小字季延,谥文,又称朱文公。南宋江南东路徽州婺源(今江西上饶市婺源县)人,生于福建路尤溪县(今福建三明市尤溪县)。南宋理学家,程朱理学集大成者,学者尊称朱子。朱熹家境穷困,但自幼聪颖,绍兴十八年(1148年)中进士,年仅十九岁,历高宗、孝宗、光宗、甯宗四朝。于建阳云谷结草堂名“晦庵”,在此讲学,宋理宗赐名“考亭书院”,故世称“考亭学派”,又因朱熹别号“紫阳”,故世称“紫阳学派”。朱熹是程颢、程颐的三传弟子李侗的学生,承北宋周敦颐与二程学说,创立宋代研究哲理的学风,称为理学。其著作甚多,辑定《大学》、《中庸》、《论语》、《孟子》为四书作为教本,也成为后代科举应试的科目,在中国,有专家认为他确立了完整的客观唯心主义体系。 \tabularnewline\hline
  张栻 & \begin{description}
  \item[字] 敬甫
  \item[号] 南轩
  \item[谥] 宣
  \item[尊] 
  \item[生] 四川绵竹
  \end{description} & 张栻(1133年-1180年) 南宋时理学学者。字敬甫,号南轩,汉州绵竹县(今属四川省)人,仕至右文殿修撰。张栻十三岁写“连州八景”诗,与吕祖谦和朱熹齐名,时称“东南三贤”。张栻曾师从胡宏,被誉为“圣门有人,吾道幸矣”。学成归长沙,先后主讲岳麓书院、城南书院。张栻为“湖湘学派”代表人物,与朱熹的“闽学”,吕祖谦的“婺学”鼎足而三。张栻政治上誓不与秦桧为伍,力主抗金,学术上虽承二程,但有别于二程。《宋史·道学传序》称:“张栻之学,亦出程氏,既见朱熹,相与博约,又大进焉!”主要著作有:《论语解》、《孟子说》、《洙泗言仁》、《诸葛忠武侯传》、《经世编年》等。 \tabularnewline\hline
  陈亮 & \begin{description}
  \item[字] 同甫
  \item[号] 龙川先生
  \item[谥] 
  \item[尊] 
  \item[生] 浙江金华
  \end{description} & 陈亮(1143年10月16日-1194年),南宋两浙东路婺州永康县(今浙江金华永康市)人,字同甫,号龙川先生,南宋政治家、哲学家、词人。反对以朱熹为代表的理学。著有《龙川先生集》。中又以上孝宗皇帝四书、《酌古论》最知名。龙川先生是朴素唯物主义思想的哲学家。创立永康学派,主“事功”。陈亮词风以豪迈雄健为主,有慷慨悲歌,“自负以经济之意具在。”。辛弃疾曾称赞陈亮,“同父之才,落笔千言,俊丽雄伟,珠明玉坚,文方窘步,我独沛然。”南宋著名词人,风格豪放激昂,是豪放派代表。词句中常抒发自己的政治抱负和爱国激情。有《龙川文集》、《龙川词》传世。 \tabularnewline\hline
  王楙 & \begin{description}
  \item[字] 勉夫
  \item[号] 分定居士
  \item[谥] 
  \item[尊] 
  \item[生] 平江吴县
  \end{description} & 王楙,宋福州福清人,徙居平江吴县,字勉夫,号分定居士。生于绍兴二十一年,少失父,事母以孝闻。宽厚诚实,刻苦嗜书。功名不偶,杜门著述,当时称为讲书君。客湖南仓使张頠门三十年,宾主相欢如一日。所著《野客丛书》三十卷,分门类聚,钩隐抉微,考证经史百家,下至骚人墨客,佚草佚事,细大不捐。另有《巢睫稿笔》。宋宁宗嘉定六年卒,年六十三。事见《野客丛书》附《宋王勉夫圹铭》。 \tabularnewline\hline
  刘过 & \begin{description}
  \item[字] 改之
  \item[号] 龙洲道人
  \item[谥] 
  \item[尊] 
  \item[生] 江西吉安
  \end{description} & 刘过(1154年-1206年),字改之,号龙洲道人,太和(今江西吉安市泰和县)人,一作庐陵(今江西吉安市)人。南宋词人。喜言兵事,早年流落江湖,重义气,力主恢复北土,与岳珂友好,与辛弃疾有唱和,词风亦相近,“赡逸有思致”。刘熙载说“刘改之词狂逸中自饶俊致”。与刘仙伦齐名,世称庐陵二布衣。有《龙洲集》、《龙洲词》。代表作有《唐多令》等。 \tabularnewline\hline
  赵蕃 & \begin{description}
  \item[字] 昌父
  \item[号] 章泉
  \item[谥] 文节
  \item[尊] 
  \item[生] 郑州
  \end{description} & 赵蕃(1143年-1229年),字昌父,号章泉,其先祖为郑州人。靖康之变后,居信州玉山(今属江西)。师从刘清之,以曾祖赵旸恩荫补州文学,调浮梁(今景德镇)尉、连江(今福建连江)主簿,皆不赴任,又曾为太和(今江西泰和县)主簿。后调辰州司理参军,因与知州争狱罢官。居家三十三年,五十岁问学于朱熹。能诗,宗黄庭坚,与韩淲(号涧泉)合称“二泉先生”。理宗绍定二年,以直秘阁致仕,不久卒,享寿八十七。 \tabularnewline\hline
  史达祖 & \begin{description}
  \item[字] 邦卿
  \item[号] 梅溪
  \item[谥] 
  \item[尊] 
  \item[生] 河南开封
  \end{description} & 史达祖字邦卿,号梅溪,汴京(河南开封)人。寓居杭州。早年师事张磁,但屡试不中,只好当韩侂胄的幕僚,任“省吏”,负责撰拟文稿,“奉行文字,拟帖撰旨,俱出其手”,颇得韩的倚重。开禧三年(1207年)韩侂胄因北伐事败被杀,达祖遭到牵连,被处以黥刑。流放到江汉。晚年困顿而死。达祖工于填词,姜夔称其词风“奇秀清逸”,善咏物,精于描写刻画,有《梅溪词》传世。王士祯在《花草蒙恰》中说:“仆每读史邦卿‘咏燕’词,以为咏物至此,人巧极天工矣。”。 \tabularnewline\hline
  姜夔 & \begin{description}
  \item[字] 尧章
  \item[号] 白石道人\\石帚
  \item[谥] 
  \item[尊] 
  \item[生] 江西鄱阳
  \end{description} & 姜夔(1155年-1209年),字尧章,号白石道人,饶州鄱阳(今江西省鄱阳)人。中国南宋词人。一生没有做过官,家贫,无立锥之地。精通音乐,会为诗,初学山谷之江西诗派,后被归类为江湖诗派。亦善填词,自度十七曲传世。范成大称其:“翰墨人品,皆似晋宋之雅士。”他的词对于南宋后期词坛的格律化有巨大的影响,姜夔和张炎并称为“姜张”。曾与杨万里、范成大、辛弃疾等交游。约卒于嘉定二年(1209年)。 \tabularnewline\hline
  韩淲 & \begin{description}
  \item[字] 仲止
  \item[号] 涧泉
  \item[谥] 
  \item[尊] 
  \item[生] 河南杞县
  \end{description} & 韩淲(biāo)(1159年-1224年),字仲止,一作子仲,号涧泉。开封雍丘(今河南杞县)人。吏部尚书韩元吉之子。生于绍兴二十九年(1159年),早年以父荫入仕,为平江府属官,嘉泰元年(1201年)曾入吴应试。不久被斥。后家居二十年。南渡后,落籍信州上饶(今属江西),与赵蕃(号章泉)合称“上饶二泉”。嘉定十七年(1224年),得疾而卒,得年六十六。著有《涧泉集》二十卷、《涧泉日记》三卷、《涧泉诗馀》一卷。 \tabularnewline\hline
  施岳 & \begin{description}
  \item[字] 仲山
  \item[号] 梅川
  \item[谥] 
  \item[尊] 
  \item[生] 江苏苏州
  \end{description} & 施岳,字仲山,号梅川。吴(今苏州)人。生卒年均不详,约宋理宗淳佑中前后在世。精于音律,死后由杨缵为树梅作亭,薛梦珪为作墓志,李彭老书,周密题,葬于西湖虎头岩下。生平事迹因无相关记载已经不可考。 \tabularnewline\hline
  严羽 & \begin{description}
  \item[字] 丹邱\\仪卿
  \item[号] 沧浪逋客
  \item[谥] 
  \item[尊] 
  \item[生] 福建邵武
  \end{description} & 严羽,字丹邱,号沧浪逋客。宋邵武(今属福建)人。生卒年不详。早年就学于邻县光泽县学教授包恢门下,包恢之父包扬曾受学于朱熹。他与严仁、严参并称“邵武三严”。并且受到司空图的影响,而有“妙悟说”。宋亡后隐居不仕,曾浪迹江、楚等地。严羽教人学诗,必先熟读《楚辞》,乃至于盛唐名家作品,并且反对苏轼、黄庭坚的诗风,称其为诗虽工,“盖于一唱三叹之音有所歉焉”,同时批评四灵派和江湖派。戴复古《祝二严》称:“羽也天资高,不肯事科举,风雅与骚些,历历在肺腑。持论伤太高,与世或龃龉。”。严氏事不见《宋史》,《福建通志》有载。著有《沧浪诗话》。 \tabularnewline\hline
  翁卷 & \begin{description}
  \item[字] 续古\\灵舒
  \item[号] 
  \item[谥] 
  \item[尊] 
  \item[生] 浙江乐清
  \end{description} & 翁卷,字续古,一字灵舒,乐清(今属浙江)人,南宋时期诗人。曾领乡荐,但一生布衣。工于诗,与徐照、徐玑、赵师秀合称“永嘉四灵”。中年以后迁居永嘉县城。有《四岩集》,《苇碧轩集》。 \tabularnewline\hline
  戴复古 & \begin{description}
  \item[字] 式之
  \item[号] 石屏
  \item[谥] 
  \item[尊] 
  \item[生] 浙江温岭
  \end{description} & 复古(1167年-1248年),字式之。天台黄岩南塘(今属浙江省温岭市新河镇)人。常居南塘石屏山,故自号石屏,南宋著名的江湖派诗人。南宋孝宗乾道三年(1167年)出生于天台道黄岩县南塘屏山,终身布衣,浪游江湖,“凡空迥奇特荒怪古僻之迹,靡不登历”。曾从陆游学诗,作品受晚唐诗风影响,兼具江西诗派风格。部分作品抒发爱国思想,反映人民疾苦,具有现实意义。其诗词格调高朗,诗笔俊爽,清健轻捷,工整自然。“往往作豪放语,锦丽是其本色。”(况周颐语)。他以诗鸣江湖间,楼钥称其“尤笃意古律……又登三山陆放翁之门,而诗益进”,真德秀《石屏词跋》云:“戴复古诗词,高处不减孟浩然。”。传世有《石屏集》六卷,《石屏长短句》一卷。 \tabularnewline\hline
  刘克庄 & \begin{description}
  \item[字] 潜夫
  \item[号] 后村居士
  \item[谥] 
  \item[尊] 
  \item[生] 福建莆田
  \end{description} & 刘克庄(1187年-1269年)初名灼,字潜夫,号后村居士,莆田城厢(今属福建)人。吏部侍郎刘弥正之子。宋朝爱国诗词家,为江湖诗派人物。理宗淳佑六年(1246年)以“文名久著,史学尤精”,赐进士,历任枢密院编修、中书舍人、兵部侍郎等,官至龙图阁直学士。其间因弹劾宰相史嵩之而先后五次贬官。惜晚节不保,晚年与奸臣贾似道交好,为人所讥。诗学晚唐,刻琢精丽,为江湖诗派中的领军人物。创作大量的爱国诗词,著作有《后村别调》和《后村先生大全集》,有诗5000多首,词200多首。 \tabularnewline\hline
  谢枋得 & \begin{description}
  \item[字] 君直
  \item[号] 叠山
  \item[谥] 
  \item[尊] 
  \item[生] 江西信州
  \end{description} & 谢枋得(1226年3月23日-1289年4月25日),字君直,号叠山,远祖居会稽,信州弋阳(今属江西)人,南宋移民、文学家,隐居福建建宁、泉州安溪、被元朝征调至燕京,不降,绝食而死,门人私谥文节。南宋灭亡后,枋得隐居于福建建宁县,又至泉州府安溪县唐石山,流寓槐植村,以卜卦、教书度日,不索钱财,惟取米、屦(白米和草鞋)而已。曾到武夷山拜访遗民熊禾。元朝先后五次征聘,坚辞不应,并写《却聘书》:“人莫不有一死,或重于泰山,或轻于鸿毛,若逼我降元,我必慷慨赴死,决不失志。”著《叠山集》16卷。他评点的《文章轨范》,是科举考试的范本,以文章类别编选文章,是南宋一部重要的评注选本,被誉为集合宋人评点学之大成。 《千家诗》原名《分门纂类唐宋时贤千家诗选》,刘克庄编辑。谢枋得对原有《千家诗》有所整理增删,成为谢枋得编辑《千家诗》。从此《千家诗》有两种版本并行与世。 \tabularnewline\hline
  吴文英 & \begin{description}
  \item[字] 君特
  \item[号] 梦窗\\觉翁
  \item[谥] 
  \item[尊] 
  \item[生] 浙江宁波
  \end{description} & 吴文英(1200年-1260年),字君特,号梦窗,晚年号觉翁,四明(今浙江宁波鄞县)人,南宋词人。吴文英本姓翁,后来过继给姓吴的人改姓吴。终生未仕。早年居苏州,后入杭州,与当朝的达官贵人交接甚密,比如丞相吴潜、史弥远等。其词多为恋情怀旧之作。当时由于他写词奉承贾似道,被当时的人所鄙视。晚年他寄居荣王赵与芮门下。有《梦窗词集》一部,存词三百四十余首,分四卷本与一卷本。其词作数量丰沃,风格雅致,多酬答、伤时与忆悼之作,号“词中李商隐”。而后世品评却甚有争论。 \tabularnewline\hline
  刘辰翁 & \begin{description}
  \item[字] 会孟
  \item[号] 须溪
  \item[谥] 
  \item[尊] 
  \item[生] 江西吉安
  \end{description} & 刘辰翁(1232年-1297年),南宋诗人,字会孟,号须溪。吉州庐陵(今江西吉安)人。生于绍定五年(1232年),早年入太学,理宗景定三年(1262年)进士,因对策忤权奸贾似道,被评丙等。曾任濂溪书院山长,咸淳元年(1265年),授临安府学教授、参江东转运幕,后荐入史馆,除太学博士。宣传《庄子》思想,与江万里友好。德佑元年(1275年),文天祥勤王,辰翁参与江西幕府。宋亡元后不仕,隐居以终。卒于元大德元年(1297年)。有《须溪集》10卷、《须溪词》3卷。 \tabularnewline\hline
  周密 & \begin{description}
  \item[字] 公谨
  \item[号] 草窗\\四水潜夫\\弁阳老人
  \item[谥] 
  \item[尊] 
  \item[生] 山东
  \end{description} & 周密(1232年-1298年),宋末元初人,字公谨,号草窗,又号四水潜夫、弁阳老人、弁阳啸翁。著有《齐东野语》等书。周密为南宋末年雅词词派领袖,有词集《萍洲渔笛谱》,词选《绝妙好词》流传于世。周密曾作有《三姝媚》送王沂孙,王沂孙也赋词相和。周密、张炎,和王沂孙、蒋捷并称宋末四大词家。他虽出身望族,却无意仕进,一生中大部分时间为平民,可谓一个“职业江湖雅人”,从其自号“草窗”便可见端倪,其词风格在姜夔、吴文英之间,与吴文英并称“二窗”。 \tabularnewline\hline
  仇远 & \begin{description}
  \item[字] 仁近
  \item[号] 
  \item[谥] 
  \item[尊] 
  \item[生] 浙江杭州
  \end{description} & 仇远(1247年~1326年),字仁近,一字仁父,钱塘(今浙江杭州)人。因居余杭溪上之仇山,自号山村、山村民,人称山村先生。元代文学家、书法家。元大德年间(1297~1307)五十八岁的他任溧阳儒学教授,不久罢归,遂在忧郁中游山河以终。著有《金渊集》六卷,皆官溧阳时所作,清人从《永乐大典》中辑出。另有《兴观集》、《山村遗集》,是清项梦昶所编,残缺不全。 \tabularnewline\hline
  唐珏 & \begin{description}
  \item[字] 玉潜
  \item[号] 菊山
  \item[谥] 
  \item[尊] 
  \item[生] 浙江绍兴
  \end{description} & 唐珏(1247-?),字玉潜,号菊山,南宋词人、义士。会稽山阴(今浙江绍兴)人。于《宋史翼》、《新元史》有传。亦记载于《宋人轶事汇编》。今存词四首,《全宋词》据《乐府补题》辑录。 \tabularnewline\hline
  文天祥 & \begin{description}
  \item[字] 宋瑞\\履善
  \item[号] 浮休道人\\文山
  \item[谥] 
  \item[尊] 
  \item[生] 江西吉安
  \end{description} & 文天祥(1236年6月6日-1283年1月9日),初名云孙,字宋瑞,一字履善。道号浮休道人、文山。江西吉州庐陵(今江西省吉安市青原区富田镇   )人,南宋末政治家、文学家,爱国诗人,抗元名臣、民族英雄,与陆秀夫、张世杰并称为“宋末三杰”。元至元十九年十二月初九(1283年1月9日),于大都就义。终年47岁。 著有《文山诗集》、《指南录》、《指南后录》、《正气歌》等。 \tabularnewline\hline
  王沂孙 & \begin{description}
  \item[字] 圣与
  \item[号] 碧山\\中仙
  \item[谥] 
  \item[尊] 竹笥山人
  \item[生] 浙江绍兴
  \end{description} & 王沂孙,生卒年不详,字圣与,又字咏道,号碧山,又号中仙,因家住玉笥山,故又号玉笥山人,南宋会稽(今浙江绍兴)人,大约生活在1230年至1291年之间,曾任庆元路(路治今宁波鄞州)学正。王沂孙工词,风格接近周邦彦,含蓄深婉,如《花犯·苔梅》之类。其清峭处,又颇似姜夔,张炎说他“琢语峭拔,有(姜)白石意度”。尤以咏物为工,如《齐天乐·蝉》、《水龙吟·白莲》等,皆善于体会物象以寄托感慨。其词章法缜密,在宋末格律派词人中是一位有显著艺术个性的词家,与周密、张炎、蒋捷并称“宋末词坛四大家”。词集《碧山乐府》,一称《花外集》,收词60余首。 \tabularnewline
  \bottomrule
\end{longtable}


%%% Local Variables:
%%% mode: latex
%%% TeX-engine: xetex
%%% TeX-master: "../../Main"
%%% End:

%% -*- coding: utf-8 -*-
%% Time-stamp: <Chen Wang: 2018-10-30 16:03:11>

\subsection{辽金元}

\begin{longtable}{|>{\centering\namefont\heiti}m{2em}|>{\centering\tiny}m{3.0em}|>{\xzfont\kaiti}m{7em}|}
 % \caption{秦王政}\
 \toprule
 \SimHei \normalsize 姓名 & \SimHei \normalsize 异名 & \SimHei \normalsize \hspace{2.5em}小传 \tabularnewline
 % \midrule
 \endfirsthead
 \toprule
 \SimHei \normalsize 姓名 & \SimHei \normalsize 异名 & \SimHei \normalsize \hspace{2.5em}小传 \tabularnewline 
 \midrule
 \endhead
 \midrule
 元好问 & \begin{description}
 \item[字] 裕之
 \item[号] 遗山
 \item[谥] 
 \item[尊] 遗山先生
 \item[生] 山西忻州
 \end{description} & 元好(hào)问(1190年8月10日—1257年10月12日),字裕之,号遗山,世称遗山先生。太原秀容(今山西忻州)人。金末至大蒙古国时期著名文学家、历史学家。元好问自幼聪慧,有“神童”之誉。金宣宗兴定五年(1221年),元好问进士及第。正大元年(1224年),又以宏词科登第后,授权国史院编修,官至知制诰。金朝灭亡后,元好问被囚数年。晚年重回故乡,隐居不仕,于家中潜心著述。元宪宗七年(1257年),元好问逝世,年六十八。元好问是宋金对峙时期北方文学的主要代表、文坛盟主,又是金元之际在文学上承前启后的桥梁,被尊为“北方文雄”、“一代文宗”。他擅作诗、文、词、曲。其中以诗作成就最高,其“丧乱诗”尤为有名;其词为金代一朝之冠,可与两宋名家媲美;其散曲虽传世不多,但当时影响很大,有倡导之功。有《元遗山先生全集》、《中州集》。 \tabularnewline\hline
 方回 & \begin{description}
 \item[字] 万里\\渊甫
 \item[号] 虚谷\\紫阳山人
 \item[谥] 
 \item[尊] 
 \item[生] 安徽歙县
 \end{description} & 方回(1227年-1307年),字万里,一字渊甫,号虚谷,别号紫阳山人,徽州歙县(今属安徽)人。长期寓居钱塘,与宋遗民往来。元成宗大德十一年(1307年)卒。有《桐江集》六十五卷。另有《瀛奎律髓》49卷,其中本集卷二四《送丘子正以能书入都……》阿谀元廷为“今日朝廷贞观同”,为周密《癸辛杂识》别集卷上所深诋。 \tabularnewline\hline
 范梈 & \begin{description}
 \item[字] 亨父\\德机
 \item[号] 
 \item[谥] 
 \item[尊] 
 \item[生] 江西樟树
 \end{description} & 范梈(pēng)(1272年~1330年),字亨父,一名德机,清江(今江西樟树)人。元代诗人,生于宋度宗咸淳八年(1272年),幼孤贫,过目成诵,作文师宗颜延年、谢灵运,大德十一年(1307),至京师,在中丞董士选家担任家教。被荐为左卫教授,历官海南海北道廉访司照磨、翰林应奉、福建闽海道知事等,官至翰林院编修,以疾归里。工于诗,同时代的虞集称范椁诗“如唐临晋帖”,与虞集、杨载、揭傒斯齐名,被誉为“元诗四大家”之一。天历二年(1329年),以母病辞归,不久母卒。天历三年(1330年)范梈亦卒。人称“文白先生”。著有《木天禁语》、《诗学禁脔》。 \tabularnewline\hline
 乔吉 & \begin{description}
 \item[字] 梦符
 \item[号] 笙鹤翁\\惺惺道人
 \item[谥] 
 \item[尊] 
 \item[生] 山西太原
 \end{description} & 乔吉(1280年-1345年),又名乔吉甫,字梦符,号笙鹤翁,又号惺惺道人。中国元朝杂剧(元曲)家、散曲作家。乔吉为山西太原人,寓居杭州。他与张可久并称双璧。 \tabularnewline\hline
 杨维桢 & \begin{description}
 \item[字] 廉夫
 \item[号] 铁崖\\东维子
 \item[谥] 
 \item[尊] 
 \item[生] 浙江绍兴
 \end{description} & 杨维桢(1296年-1370年),又作维祯,字廉夫,号铁崖、东维子会稽(今浙江绍兴)人。元末明初政治人物。杨维翰之弟。成宗元贞二年(1296年)生,少时读书于铁崖山,其父杨宏在铁崖山麓筑楼,楼上藏书万卷,周围种数百株梅树,将梯子撤去,令其专心攻读,杨维桢苦读五年,每日用辘皿传递食物。泰定四年(1327年)中进士,授天台县尹,杭州四务提举。维桢为人倔强,诗文奇诡,喜做翻案文章,如《炮烙辞》一诗支持纣王。又以拟古乐府见称于时,是当时诗坛领袖,因“诗名擅一时,号铁崖体”,独领风骚。元末天下大乱,维桢避寓富春江一带,张士诚屡召不仕,迁苏州、松江等地,隐居不出,和文人“笔墨纵横,铅粉狼藉”,沉溺声色。与陆居仁、钱惟善被称为“元末三高士”。著有《东维子文集》、《铁崖先生古乐府》等。 \tabularnewline\hline
 萨都剌 & \begin{description}
 \item[字] 天锡
 \item[号] 直斋
 \item[谥] 
 \item[尊] 
 \item[生] 蒙古
 \end{description} & 萨都剌,又作萨都拉,(1272年或1300年-1355年),字天锡,庵号直斋,元代著名诗人、画家、书法家。蒙古化色目人(一说回回人)。出身将门,但据其《溪行中秋玩月》诗自序,幼年家贫。早年科举不顺,以经商为业。到泰定四年(1327年)才中进士,一生只做过一些卑微的官职,包括京口录事司达鲁花赤、江南行御史台掾史、燕南河北道肃政廉访司照磨、闽海福建道肃政廉访司知事、燕南河北道肃政廉访司经历等职等。为官清廉,有政绩,不趋炎附势,因得罪权贵而被贬。 \tabularnewline\hline
 赵汸 & \begin{description}
 \item[字] 子常
 \item[号] 
 \item[谥] 
 \item[尊] 东山先生
 \item[生] 安徽休宁
 \end{description} & 赵汸(1319年-1369年),字子常。安徽休宁人。生于元仁宗延佑六年(1319年),读朱子《四书》,多所疑难,乃尽取朱子书读之。师事黄泽,专攻《春秋》《易》象之学。后复从临川虞集游,获闻吴澄之学,思想可见于《对江右六君子策略》,主张“澄心默坐,涵养本源,以为致思之地”,而后“凡所得于师之指及文字奥义有未通者,必用向上功夫以求之”。赵汸生于乱世,淡泊名利,隐居著述,作“东山精舍”以奉母,学者称东山先生,邑人建商山书院,聘赵汸、朱升为书院山长。洪武二年(1369年)召修《元史》,完成初稿159卷。半年后乞还东山。未几,以病卒。著有《葬书问对》、《东山存槁》、《周易文诠》、《春秋集传》等。《明史·儒林传》有传。 \tabularnewline
 \bottomrule
\end{longtable}


%%% Local Variables:
%%% mode: latex
%%% TeX-engine: xetex
%%% TeX-master: "../../Main"
%%% End:

%% -*- coding: utf-8 -*-
%% Time-stamp: <Chen Wang: 2018-10-30 16:03:57>

\subsection{明}

\begin{longtable}{|>{\centering\namefont\heiti}m{2em}|>{\centering\tiny}m{3.0em}|>{\xzfont\kaiti}m{7em}|}
  % \caption{秦王政}\
  \toprule
  \SimHei \normalsize 姓名 & \SimHei \normalsize 异名 & \SimHei \normalsize \hspace{2.5em}小传 \tabularnewline
  % \midrule
  \endfirsthead
  \toprule
  \SimHei \normalsize 姓名 & \SimHei \normalsize 异名 & \SimHei \normalsize \hspace{2.5em}小传 \tabularnewline 
  \midrule
  \endhead
  \midrule
  高棅 & \begin{description}
  \item[字] 廷礼\\彦恢
  \item[号] 漫士
  \item[谥] 
  \item[尊] 
  \item[生] 福建长乐
  \end{description} & 高棅(1350年-1423年),又名廷礼,字彦恢,号漫士,福建长乐人。闽中十才子之一。高棅为明朝初年研究唐诗的重要学者,所著的《唐诗品汇》为明初诗歌复古的里程碑,也是中国文学的重要评论著作。高棅著有《啸台集》、《水天清气集》、《唐诗品汇》、《唐诗拾遗》、《唐诗正声》。 \tabularnewline\hline
  高启 & \begin{description}
  \item[字] 季迪
  \item[号] 青丘子
  \item[谥] 
  \item[尊] 
  \item[生] 江苏苏州
  \end{description} & 高启(1336年-1373年,37岁),字季迪,号青丘,元末明初平江路(明改苏州府)长洲县(今江苏省苏州市)人,明初十才子之一。和宋濂、刘基合称“明初诗文三大家”。因得罪明太祖,以魏观案累文字狱,处腰斩。高启有诗才,其诗清新超拔,雄健豪迈,尤擅长于七言歌行,《四库全书总目提要》称:“拟汉魏似汉魏,拟六朝似六朝,拟唐似唐,拟宋似宋,凡古人所长,无不兼之。”与杨基、张羽、徐贲合称“吴中四杰”。景泰元年(1450年),徐庸搜集《缶鸣集》等遗篇,编为《高太史大全集》18卷。 \tabularnewline\hline
  陈献章 & \begin{description}
  \item[字] 公甫
  \item[号] 实斋
  \item[谥] 
  \item[尊] 白沙先生
  \item[生] 广东江门
  \end{description} & 陈献章(1428年-1500年),字公甫,号实斋,广东新会县会城都会乡(今江门市新会区会城街道)人,后迁居白沙乡,世称白沙先生。明代著名的书法家、诗人、教育家、思想家,为岭南学派创始人。是岭南唯一诏准从祀孔庙的学者,有“岭南第一人”、“广东第一大儒”的称誉。曾自制以新会圭峰山长成的硬朗的茅草为材料的茅龙笔,字体苍劲有力,别具风格。陈献章遭逢明朝中叶的乱象,历经王振弄权(1435年)、土木之变(1449年)、明英宗夺门之变复辟(1457年)等社会动乱。一生清贫,都御史邓廷缵曾令番禺县每月给他米一石,陈拒不接受,说自己“有田二顷,耕之足矣”。又有按察使花了巨金买园林豪宅送他,他亦不受。陈献章的入学法门是“以静为主”,“端坐澄心,于静中养出端倪。”献章创立了岭南第一个颇具影响的学术流派——岭南学派。其弟子有湛若水、梁储、李承箕、林缉熙、张廷实、贺钦、陈茂烈、容一之、罗服周、潘汉、叶宏、谢佑、林廷{\fzk 𤩽}等。 \tabularnewline\hline
  唐寅 & \begin{description}
  \item[字] 伯虎\\子谓
  \item[号] 六如居士\\桃花庵主
  \item[谥] 
  \item[尊] 
  \item[生] 江苏苏州
  \end{description} & 唐寅(1470年3月6日-1524年1月4日),明代著名画家、文学家。字伯虎,又字子畏,以字行,号六如居士、桃花庵主、逃禅仙吏等,直隶吴县人,吴中四才子之一。在画史上又与沈周、文徵明、仇英合称“明四家”或“吴门四家”。民间有很多关于唐伯虎的传说,最为人熟悉的《唐伯虎点秋香》曾多次被改编成戏剧,以及拍成电视剧及电影,也宣传、加深了唐伯虎在民间的形象。唐寅出生于世商家庭,有一妹一弟,父亲唐广德,经营一家唐记酒店。唐寅作品以山水画、人物画闻名于世,其创作的多幅春宫图也为他个人添加了“风流才子”的名声。 \tabularnewline\hline
  沈周 & \begin{description}
  \item[字] 启南
  \item[号] 石田\\白石翁\\玉田生
  \item[谥] 
  \item[尊] 
  \item[生] 江苏苏州
  \end{description} & 沈周(1427年-1509年)字启南、号石田、白石翁、玉田生、有竹居主人等,明朝画家,吴门画派的创始人,明四家之一,长洲(今属江苏苏州市)人。沈周的书画流传很广,真伪混杂,较难分辨。文征明因此称他为飘然世外的“神仙中人”。 \tabularnewline\hline
  李梦阳 & \begin{description}
  \item[字] 恩赐
  \item[号] 空同子
  \item[谥] 
  \item[尊] 
  \item[生] 甘肃庆阳
  \end{description} & 李梦阳(1472年-1529年),又名献吉,字恩赐,号空同子,祖籍河南扶沟,出生于陕西庆阳(今甘肃),后又还归故里。明朝政治人物,文坛前七子之一。著有《空同集》。李梦阳为明朝初年研究唐诗的重要学者,乐府﹑歌行有相当成就,郭卓茂评论道:“有明一代研究唐诗的重要学者,中国古代文坛上胆大包天的诗人。”王维祯认为:“七言律自杜甫以后﹐善用顿挫倒插之法﹐惟梦阳一人。”他主要贡献在于诗歌理论批评,他所提出的“古体学习汉魏,近体学唐诗”的观念,相当具有指标性。他还提出“真诗乃在民间”的观点。他与何景明“倡导复古﹐文自西京﹑诗自中唐而下﹐一切吐弃。操觚谈艺之士﹐翁然宗之”。 \tabularnewline\hline
  李东阳 & \begin{description}
  \item[字] 宾之
  \item[号] 西涯
  \item[谥] 文正
  \item[尊] 
  \item[生] 湖南茶陵
  \end{description} & 李东阳(1447年-1516年),字宾之,号西涯,谥文正,明朝中叶重臣,文学家,书法家,茶陵诗派的核心人物。湖广茶陵县(今湖南茶陵)人,金吾左卫军籍。李东阳入阁多年,在朝时间长,地位高,不仅自己才学渊博,又能奖励后学,推荐隽才,因此不少文学之士都围聚在他周围,形成了一个颇有影响的诗人派别。李东阳也就在明中期一度领导文坛。因而《明史》中写道:“弘治时,宰相李东阳主文柄,天下翕然宗之。” \tabularnewline\hline
  何景明 & \begin{description}
  \item[字] 仲默
  \item[号] 白坡\\大复山人
  \item[谥] 
  \item[尊] 
  \item[生] 河南信阳
  \end{description} & 何景明(1483年-1521年),字仲默,号白坡,又号大复山人。河南信阳人。明朝作家。明朝文学家前七子之一,官至陕西提学副使。何景明工诗古文,与李梦阳皆提倡复古之学,天下从之,文体一变。在“七子”中,地位仅次于李梦阳,“天下语诗文,必并称何、李”(《明史‧何景明传》)。他也主张文宗秦、汉,古诗宗汉、魏,近体诗宗盛唐。 \tabularnewline\hline
  杨慎 & \begin{description}
  \item[字] 用修
  \item[号] 升庵
  \item[谥] 文宪
  \item[尊] 
  \item[生] 四川新都
  \end{description} & 杨慎(1488年12月8日-1559年8月8日),字用修,号升庵,别号博南山人、博南戍史,谥文宪,四川新都县(今成都市新都区马家镇升庵村)人,祖籍江西庐陵,为内阁首辅杨廷和之子,正德年间状元,官至翰林院修撰。大礼议事件中,因率领百官在左顺门求世宗改变皇考,而遭贬云南,终老于戍地。现成都市新都区仍存有其私家园林升庵桂湖。杨慎与解缙、徐渭合称“明朝三才子”。主要著作有《滇程记》、《丹铅总录》、《丹铅杂录》、《南诏野史》、《古音猎要》、《全蜀艺文志》、《春秋地名考》等。 \tabularnewline\hline
  薛蕙 & \begin{description}
  \item[字] 君采
  \item[号] 西原
  \item[谥] 
  \item[尊] 
  \item[生] 亳州
  \end{description} & 薛蕙(1489年-1541年),字君采,号西原,直隶亳州人,明朝政治人物。正德九年(1514年)登甲戌科进士,授刑部主事。明武宗南巡之争中,因进谏劝阻而受杖夺俸,随后引疾归乡。此后起用恢复官职,改吏部,历任吏部考功司郎中。嘉靖二年(1523年)大礼议事件中,廷臣数次进谏,薛蕙亦上疏劝阻。世宗读后大怒,夺俸三月,此后因事诬陷而归乡。薛蕙一生著有《西原集》10卷,《补遗》1卷,《五经杂录》、《大宁斋日录》五卷、《老子集解》、《庄子注》、《考功集》、《约言》和《西原遗书》二卷。 \tabularnewline\hline
  李攀龙 & \begin{description}
  \item[字] 于鳞
  \item[号] 沧溟
  \item[谥] 
  \item[尊] 
  \item[生] 山东济南
  \end{description} & 李攀龙(1514年-1570年),字于鳞,号沧溟,山东历城(今济南)人,明朝官员、文学家,“后七子”之首。是明朝知名作家,也是知名的文学评论家。他对于秦汉文学抱甚高的评价,并对唐诗颇多与他学者不同的贬抑看法。他所著的《答冯通书》就提到:“秦汉以后无文矣”。著有《沧溟集》。 \tabularnewline\hline
  王世贞 & \begin{description}
  \item[字] 元美
  \item[号] 凤洲\\弇州山人
  \item[谥] 
  \item[尊] 
  \item[生] 江苏太仓
  \end{description} & 王世贞(1526年-1590年),字元美,号凤洲,又号弇州山人,直隶太仓州(今江苏太仓)人,明朝文学家、史学家。“后七子”之一。世贞早年与李攀龙同为“后七子”领袖。攀龙死后,他独主诗坛二十年。“一时士大夫及山人、词客、衲子、羽流,莫不奔走门下。片言褒赏,声价骤起”。善诗,尤擅律,绝,倡导文学复古运动,有“文必秦汉,诗必盛唐”的主张。有《弇州山人四部稿》一百七十四卷、《弇山堂别集》一百卷(多载史事杂考)、《艺苑卮言》十二卷(南北曲源流与评论)、《鸣凤记》(剧本,以批严嵩为题材。王世贞之父被严嵩陷害死,作品大斥严氏罪行。)、《史乘考误》传世。不少学者认为《金瓶梅》作者兰陵笑笑生的真实身份便是王世贞。 \tabularnewline\hline
  王世懋 & \begin{description}
  \item[字] 敬美
  \item[号] 麟州\\损斋\\墙东生
  \item[谥] 
  \item[尊] 
  \item[生] 江苏太仓
  \end{description} & 王世懋(1536年-1588年),字敬美,号麟州,又号损斋,或曰墙东生。直隶太仓(今属江苏省)人。明朝政治人物。南京刑部尚书、史学家王世贞之弟。著有《王仪部集》、《二酋委谭摘录》、《名山游记》、《奉常集词》、《窥天外乘》、《艺圃撷余》等。《明史》附其传于王世贞传后。 \tabularnewline\hline
  胡应麟 & \begin{description}
  \item[字] 元瑞
  \item[号] 少室山人\\石羊生
  \item[谥] 
  \item[尊] 
  \item[生] 浙江金华
  \end{description} & 胡应麟(1551年-1602年),字元瑞,一字明瑞,号“少室山人”,又号“石羊生”,浙江金华兰溪人。他的《四部正伪》一书,上承宋濂的“诸子辩”,扩大检讨重要的古书,为古书辨伪。古书辨伪工作早发于刘知几、柳宗元,由胡应麟与姚际恒等续作。另著有《诗薮》、《华阳博议》、《九流绪论》、《经籍会通》、《史书占毕》、《庄岳委谈》、《唐同姓名录》、《二酉山房歌》、《少室山房笔丛》等。 \tabularnewline\hline
  钟惺 & \begin{description}
  \item[字] 伯敬
  \item[号] 退谷
  \item[谥] 
  \item[尊] 
  \item[生] 湖北天门
  \end{description} & 钟惺(1574—1625), 明代文学家。字伯敬,号退谷,湖广竟陵(今湖北天门市)人。万历三十八年(1610)进士。曾任工部主事,万历四十四年(1616)与林古度登泰山。后官至福建提学佥事。不久辞官归乡,闭户读书,晚年入寺院。其为人严冷,不喜接俗客,由此得谢人事,研读史书。他与同里谭元春共选《唐诗归》和《古诗归》(见《诗归》),名扬一时,形成“竟陵派”,世称“钟谭”。 \tabularnewline\hline
  谭元春 & \begin{description}
  \item[字] 友夏
  \item[号] 鹄湾\\蓑翁
  \item[谥] 
  \item[尊] 
  \item[生] 湖北天门
  \end{description} & 谭元春(1586~1637),湖广竟陵(今湖北天门市)人,字友夏,号鹄湾,别号蓑翁。明代文学家,天启间乡试第一,与同里钟惺同为“竟陵派”创始人,论文重视性灵,反对摹古,提倡幽深孤峭的风格,所作亦流于僻奥冷涩,有《谭友夏合集》。复社兴起后,他又加入了复社,被列为“复社四十八友”之一。 \tabularnewline\hline
  钱澄之 & \begin{description}
  \item[字] 幼光
  \item[号] 田间\\西顽道人
  \item[谥] 
  \item[尊] 
  \item[生] 安徽桐城
  \end{description} & 钱澄之(1612年-1693年)是一名明朝末年的诗人、官员。安徽桐城人。初名秉镫,字幼光;后改名澄之,字饮光,号田间,又号西顽道人。自小随父读书,十一岁能写文章,崇祯时中秀才。明崇祯初年,与方以智、孙临、方文、周岐等人成立诗社泽社。曾参加抗清活动,兵败后游历于江浙一带著述。王夫之推崇他“诗体整健”。著有《田间集》、《田间诗集》、《田间文集》、《藏山阁集》等。 \tabularnewline
  \bottomrule
\end{longtable}


%%% Local Variables:
%%% mode: latex
%%% TeX-engine: xetex
%%% TeX-master: "../../Main"
%%% End:

%% -*- coding: utf-8 -*-
%% Time-stamp: <Chen Wang: 2018-10-30 16:25:41>

\subsection{清}

\begin{longtable}{|>{\centering\namefont\heiti}m{2em}|>{\centering\tiny}m{3.0em}|>{\xzfont\kaiti}m{7em}|}
  % \caption{秦王政}\
  \toprule
  \SimHei \normalsize 姓名 & \SimHei \normalsize 异名 & \SimHei \normalsize \hspace{2.5em}小传 \tabularnewline
  % \midrule
  \endfirsthead
  \toprule
  \SimHei \normalsize 姓名 & \SimHei \normalsize 异名 & \SimHei \normalsize \hspace{2.5em}小传 \tabularnewline 
  \midrule
  \endhead
  \midrule
  吴伟业 & \begin{description}
  \item[字] 骏公
  \item[号] 梅村
  \item[谥] 
  \item[尊] 
  \item[生] 江苏昆山
  \end{description} & 吴伟业(1609年6月21日-1672年1月23日),字骏公,号梅村,祖籍南直隶苏州府昆山县(今江苏省昆山市),祖父始迁居太仓州(今江苏省苏州市太仓市),明末清初著名诗人、政治人物,长于七言歌行,初学“长庆体”,后自成新吟,后人称之为“梅村体”。与钱谦益、龚鼎孳并称为江左三大家。吴伟业著有《梅村家藏稿》、《梅村诗馀》,传奇《秣陵春》,杂剧《通天台》、《临春阁》,史料《绥寇纪略》等。其诗情深文丽,宫商和谐,敷衍成长篇七言,蔚然可观,在清朝被称为“本朝词家之领袖”。 \tabularnewline\hline
  钱谦益 & \begin{description}
  \item[字] 受之
  \item[号] 木斋\\绛云楼主
  \item[谥] 
  \item[尊] 虞山\\宗伯
  \item[生] 江苏常熟
  \end{description} & 钱谦益(1582年10月22日-1664年6月17日),字受之,号牧斋,晚号绛云楼主人、蒙叟、东涧老人,又因其住址而称虞山、因其职位而称宗伯,直隶常熟县(今江苏省苏州市常熟市)人。作为明末清初时期文学领域的集大成者,钱谦益领导这一时期的文坛长达五十年。在政治上钱视为东林党或复社人士。明朝时四次出仕,官至礼部尚书。后在南京降清,任礼部侍郎五个月,被视作“贰臣”。辞官后投入反清复明运动,为遗民义士接纳,更成为联络东南与西南抗清复明势力的总枢纽。后钱谦益的诗文被乾隆帝下诏禁毁。陈寅恪认为其是“复国之英雄”,“应恕其前此失节之愆,而嘉其后来赎罪之意,始可称为平心之论”,并称钱与其妻柳如是的诗文足以“表彰我民族独立之精神,自由之思想”。钱谦益学问渊博,反对竟陵派“尖新”、“鬼趣”的文风,倡言“情真”、“情至”,主张具“独至之性,旁出之情,偏诣之学”。 \tabularnewline\hline
  李渔 & \begin{description}
  \item[字] 谪凡
  \item[号] 笠翁\\蟹仙
  \item[谥] 
  \item[尊] 
  \item[生] 浙江兰溪
  \end{description} & 李渔(1611年9月13日-1680年2月12日),初名仙侣,后改名渔,字谪凡,号笠翁,后人常称之蟹仙。明末清初文学家、戏曲家,曾经评定《四大奇书》,祖籍浙江省兰溪县(今浙江省兰溪市)夏李村,后来祖父随“兰溪帮”到了江苏如皋做种药材生意。著有《凰求凤》、《玉搔头》等戏剧,《肉蒲团》、《觉世名言十二楼》、《无声戏》、《连城璧》等小说,与《闲情偶寄》等书。 \tabularnewline\hline
  冯班 & \begin{description}
  \item[字] 定远
  \item[号] 钝吟老人
  \item[谥] 
  \item[尊] 
  \item[生] 江苏常熟
  \end{description} & 冯班(1602年-1671年),字定远,号钝吟老人,江苏常熟人。生于万历三十年(1602)。早年为诸生。从钱谦益学诗,与兄冯舒齐名,称为“海虞二冯”。明亡后不仕,常常就座中恸哭,人称其为“二痴”。冯班是虞山诗派的重要人物,钱谦益称冯班之诗“沈酣六代,出入于义山、牧之、庭筠之间”,论诗讲究“无字无来历气”,反对严羽《沧浪诗话》的妙悟说,其《钝吟杂录》卷五《严氏纠谬》专驳严说。吴乔推崇贺裳、冯班,称《载酒园诗话》、《钝吟杂录》与自己的《围炉诗话》为“谈诗三绝”,书中多引贺、冯之语。康熙十年(1671)卒。赵执信尝谒其墓,写“私淑门人”刺焚冢前。有《钝吟集》、《钝吟杂录》、《钝吟书要》和《钝吟诗文稿》等。 \tabularnewline\hline
  贺裳 & \begin{description}
  \item[字] 黄公
  \item[号] 白凤词人\\檗斋
  \item[谥] 
  \item[尊] 
  \item[生] 丹阳
  \end{description} & 贺裳,字黄公,号檗斋,别号白凤词人。丹阳人。生卒年不详。明末入太学,崇祯二年加入复社。入清为诸生。工于词,长于批评,“于诗有深得,而又能详读宋人之诗,持论至当。”吴乔推崇贺裳与冯班,称贺的《载酒园诗话》、冯的《钝吟杂录》与自著《围炉诗话》为“谈诗三绝”,书中多引贺、冯之语。著有《载酒园诗话》三卷、《红牙词》、《史折》等。 \tabularnewline\hline
  吴乔 & \begin{description}
  \item[字] 修龄
  \item[号] 
  \item[谥] 
  \item[尊] 
  \item[生] 江苏常熟
  \end{description} & 吴乔(1611~1695),原名殳,字修龄,江南太仓(今属江苏)人,入赘昆山。明崇祯十一年诸生,寻被斥;字不详,生卒年不详,属蜀汉至成汉期间,蜀车骑将军吴壹之孙。有《载酒园诗话》、《古宫词》、《托物草》、《好山诗》、《舒拂集》等。 \tabularnewline\hline
  王夫之 & \begin{description}
  \item[字] 而农
  \item[号] 姜斋
  \item[谥] 
  \item[尊] 船山先生
  \item[生] 湖南衡阳
  \end{description} & 王夫之(1619年-1692年,即万历四十七年-康熙三十一年),湖广衡阳县人,杰出的思想家、哲学家、明末清初大儒。字而农,号姜斋、又号夕堂,或署一瓢道人、双髻外史,自署船山病叟、南岳遗民,晚年隐居于石船山麓,世称遂称船山先生,主要著作有《周易外传》、《读通鉴论》等,后汇编为《船山遗书》。与顾炎武、黄宗羲并称明清之际三大思想家。王夫之生前著有《周易外传》、《黄书》、《尚书引义》、《永历实录》、《春秋世论》、《噩梦》、《读通鉴论》、《宋论》等书。 \tabularnewline\hline
  邓汉仪 & \begin{description}
  \item[字] 孝威
  \item[号] 旧山
  \item[谥] 
  \item[尊] 
  \item[生] 江苏苏州
  \end{description} & 邓汉仪(1617年-1689年),字孝威,号旧山,别号旧山农、钵叟。江南苏州府吴县洞庭琦里人。邓旭之弟。清顺治元年(1644年),迁居泰州,不仕清,与吴梅村、龚鼎孳友好,早负诗名,有《题息夫人庙》诗: “千古艰难惟一死,伤心岂独息夫人。”。曾纂有《江南通志》。康熙十八年(1679年),召试博学鸿儒,不第,以年老授中书舍人。著有《淮阴集》、《官梅集》、《过岭集》、《甬东集》、《濠梁集》、《燕薹集》、《被征集》、《慎墨堂笔记》一卷,《诗观》四集,《箫楼集》等。 \tabularnewline\hline
  周在浚 & \begin{description}
  \item[字] 雪客
  \item[号] 犁庄\\仓谷
  \item[谥] 
  \item[尊] 
  \item[生] 河南开封
  \end{description} & 清藏书家。字雪客,号梨庄,一号苍谷,又号耐龛。祥符(今河南开封)人。约公元一六七五年前后在世,周亮工之子。和著名藏书家黄虞稷合编纂目录《征刻唐宋秘本书目》1卷、附《考证》1卷。《征刻书启五先生事略》1卷。著有《云烟过眼录》、《晋碑》、《南唐书注》、《大梁守城志》、《黎庄集》、《遗谷集》、《天发神谶碑考》、《秋水轩集》等。 \tabularnewline\hline
  毛奇龄 & \begin{description}
  \item[字] 大可
  \item[号] 西河
  \item[谥] 
  \item[尊] 
  \item[生] 浙江萧山
  \end{description} & 毛奇龄(1629年10月28日-1713年),字大可,又字于一,号西河,又号河右、初晴、晚晴。浙江萧山人。明末清初经学家、文学家。毛奇龄之文章,“纵横博辨,傲睨一世”,[4]他反对朱子学,他的弟子收集其旧文编撰《四书改错》以攻击朱熹《四书集注》。清初《四库全书》收录其著作二十八种,见于《存目》的三十五种,为《四库全书》中个人著作被收录最多的一位。 \tabularnewline\hline
  邹祗谟 & \begin{description}
  \item[字] 訏士
  \item[号] 程村
  \item[谥] 
  \item[尊] 
  \item[生] 江南武进
  \end{description} & 邹祇(zhǐ)谟(1627年-1670年),字𬣙士,号程村,江南武进人。清朝文学家。同进士出身。天启七年(1627年)出生。读书过目不忘,顺治十五年(1658年)登戊戌科孙承恩榜进士,顺治十八年以逋粮案黜职,遂不复仕。著有《丽农词》二卷,与王士祯《衍波词》、彭孙遹《延露词》并称“三名家词”。工于诗,与陈维崧、黄永、 董以宁号“毗陵四子”。又与王士祯编《倚声初集》,收集一千九百余首,于清初词风影响甚巨。 康熙九年卒。此外著有《远志斋集》。 \tabularnewline\hline
  王士祯 & \begin{description}
  \item[字] 贻上
  \item[号] 阮亭\\渔洋山人
  \item[谥] 文简
  \item[尊] 王渔洋
  \item[生] 山东桓台
  \end{description} & 王士禛(1634年9月17日-1711年6月26日),赐名士祯,小名豫孙,字贻上,号阮亭,别号渔洋山人,人称王渔洋,谥文简。山东新城(今山东桓台)人,清代著名文人,进士出身,康熙年间官至刑部尚书。工诗文,勤著述,著作有《渔洋山人精华录》、《池北偶谈》等五百余种。渔洋与长兄王士禄、二兄王士禧、三兄王士祜皆有诗名。其一生著述达500余种,作诗4000余首,主要有《渔洋山人精华录》、《蚕尾集》、《池北偶谈》、《香祖笔记》、《居易录》、《古夫于亭杂录》、《分甘余话》、《渔洋文略》、《渔洋诗集》、《带经堂集》、《感旧集》等。作中间有明季入清之家事。 \tabularnewline\hline
  邵长蘅 & \begin{description}
  \item[字] 子湘
  \item[号] 青门山人
  \item[谥] 
  \item[尊] 
  \item[生] 江苏常州
  \end{description} & 邵长蘅(1637年-1704年),字子湘,号青门山人,江苏武进人。生于明思宗崇祯十年(1637年),读书一目数行,十岁补诸生,康熙中曾应博学鸿词科。江苏巡抚宋荦聘致幕中。善写文章,为王士禛、汪琬所称道,主张为文必多读书[1]。卒于清圣祖康熙四十三年(1704年)。著有《青门集》、《八大山人传》。 \tabularnewline\hline
  李光地 & \begin{description}
  \item[字] 晋卿
  \item[号] 厚庵\\榕村
  \item[谥] 
  \item[尊] 安溪先生
  \item[生] 福建泉州
  \end{description} & 李光地(1642年-1718年),字晋卿,号厚庵,又号榕村,福建泉州安溪湖头人,闽南人。清圣祖康熙九年(1670年)登进士第五名,官至直隶巡抚、吏部尚书、文渊阁大学士。1681年并推保荐施琅领军,结束明郑;是清初著名的政治人物与理学家。同时代的学者尊称为“安溪先生”,或“安溪李相国”。李光地研究理学,倡言礼乐,实行海禁措施,导致近海百里无人烟,限制了农耕渔矿多种产业的发展,对康熙中年的决策有决定性的影响。晚年的李光地仍大受康熙宠信,出任吏部尚书、文渊阁大学士等职。康熙称他“谨慎清勤,始终一节,学问渊博。朕知之最真,知朕亦无过光地者”。太子允礽被废后,李光地开始辅助后来的雍正帝。雍正帝称李光地为“一代之完人”。 \tabularnewline\hline
  阎若璩 & \begin{description}
  \item[字] 百诗
  \item[号] 潜丘
  \item[谥] 
  \item[尊] 
  \item[生] 山西太原
  \end{description} & 阎若璩(1636年-1704年)字百诗,号潜丘。清初经学家、学者。山西太原人。一生勤奋著书,著有《尚书古文疏证》、《四书释地》、《潜邱札记》、《困学记闻注》、《孟子生逐年月考》、《眷西堂集》等。又曾为顾炎武《日知录》订正错误。其中《尚书古文疏证》八卷,引经据典,确定《古文尚书》为东晋梅赜所伪著。 \tabularnewline\hline
  赵执信 & \begin{description}
  \item[字] 伸符
  \item[号] 秋谷\\饴山老人
  \item[谥] 
  \item[尊] 
  \item[生] 山东淄博
  \end{description} & 赵执信[shēn](1662~1744)清代诗人、诗论家、书法家。字伸符,号秋谷,晚号饴山老人、知如老人。山东省淄博市博山人。十四岁中秀才,十七岁中举人,十八岁中进士,后任右春坊右赞善兼翰林院检讨。二十八岁因佟皇后丧葬期间观看洪升所作《长生殿》戏剧,被劾革职。此后五十年间,终身不仕,徜徉林壑。赵执信为王士祯甥婿,然论诗与其异趣,强调“文意为主,言语为役”。所作诗文深沉峭拔,亦不乏反映民生疾苦的篇目。赵执信的著作已经刊行的有《饴山诗集》十九卷,《饴山文集》十二卷,《诗余》一卷,《谈龙录》一卷,《声调谱》一卷,《礼俗权衡》两卷等。 \tabularnewline\hline
  沈德潜 & \begin{description}
  \item[字] 碻士
  \item[号] 归愚
  \item[谥] 
  \item[尊] 
  \item[生] 江苏苏州
  \end{description} & 沈德潜(1673年-1769年),字碻士(碻读音què),号归愚,江苏苏州人,清代政治人物、诗人。他在诗歌理论方面主张格调说,反对钱谦益之后的重视宋元诗的风潮,也与袁枚的性灵说相对立。编有《唐宋八家文读本》,另外他所编辑的隋代以前古诗选集《古诗源》、唐诗选集《唐诗别裁》、唐明清诗选集《国朝诗别裁集》代表了他的诗歌创作观念,广受欢迎。 \tabularnewline\hline
  王琦 & \begin{description}
  \item[字] 琢崖
  \item[号] 
  \item[谥] 
  \item[尊] 
  \item[生] 浙江杭州
  \end{description} & 王琦,字琢崖,清代钱塘人,乾隆时期的有名学者。曾注《李太白文集》三十六卷、《李长吉歌诗汇解》五卷,并帮助赵殿成注释《王右丞集》中的佛教典故。 \tabularnewline\hline
  袁枚 & \begin{description}
  \item[字] 子才
  \item[号] 简斋\\随园老人
  \item[谥] 
  \item[尊] 
  \item[生] 浙江杭州
  \end{description} & 袁枚(1716年-1797年),清代诗人,散文家。字子才,号简斋,别号随园老人,时称随园先生,浙江钱塘县(今浙江杭州)人,祖籍浙江慈谿[1][2],年廿四中进士,曾官溧水、江浦、沭阳、江宁等地知县,不到卅八岁即辞官还乡,致仕之后因投资地产有道,家财万贯。袁枚擅长诗、赋、制艺,能写骈文、小品文、笔记,乾隆时期为诗坛盟主,又为“清代骈文八大家”、“江右三大家”之一,文笔亦与大学士直隶纪昀齐名,时称“南袁北纪”。其喜好广泛,甚至编写食谱、志怪小说,著有《小仓山房文集》、《随园诗话》、《子不语》、《祭妹文》等。书信亦有名,其《小仓山房尺牍》与许葭村《秋水轩尺牍》、龚未斋《雪鸿轩尺牍》,人称“清代三大尺牍”。袁枚生平喜称人善、奖掖士类,也提倡女性文学,广收女弟子。不喜理学、汉学,追求自由,反对统一思想,他说“物之不齐,物之情也,天亦不能做主,而况于人乎?”,故被当时的许多文人严厉批判,袁枚依然悠哉度日,在文坛享有盛名。 \tabularnewline\hline
  纪昀 & \begin{description}
  \item[字] 晓岚
  \item[号] 石云
  \item[谥] 文达
  \item[尊] 
  \item[生] 河北献县
  \end{description} & 纪昀(雍正2年六月十五日-嘉庆10年二月十四日,即1724年7月26日-1805年3月14日),字晓岚,又字春帆,晚号石云,又号观弈道人、孤石老人、河间才子,在文学作品、通俗评论中,常被称为纪晓岚。清代直隶献县(今河北献县)人,乾隆年间的著名学者,政治人物。官至礼部尚书、协办大学士,曾任《四库全书》总纂修官。卒谥文达。纪昀文采超群,与同时代江南的袁枚齐名,时称“北纪南袁”。纪昀反对理学[2],《阅微草堂笔记》和《四库全书总目提要》中有相当深刻的反映。 \tabularnewline\hline
  张惠言 & \begin{description}
  \item[字] 皋文
  \item[号] 
  \item[谥] 
  \item[尊] 
  \item[生] 江苏常州
  \end{description} & 张惠言(1761年-1802年),原名一鸣,字皋文,江苏武进(今常州)人,清代政治人物,经学家、词学家。生于清高宗乾隆二十六年(1761年),幼年贫困[1],清仁宗嘉庆四年(1799年)中进士,授庶吉士,充实录馆纂修官,卒于嘉庆七年(1802年)。著有《茗柯文》五卷。张惠言提出“比兴寄托”,主张“意内言外”,人称常州词派始祖。 \tabularnewline\hline
  周济 & \begin{description}
  \item[字] 介存\\保绪
  \item[号] 未斋
  \item[谥] 
  \item[尊] 
  \item[生] 江苏宜兴
  \end{description} & 周济(1781年-1839年),清朝词人及词论家。字保绪,一字介存,号未斋,晚号止庵。江苏荆溪(今江苏宜兴)人。周济是董士锡的弟子,继承了张惠言的词论传统,一般被称为常州词派的集大成者。他论词强调寄托;自作词意旨较为隐晦。著有《味隽斋词》、《词辨》、《介存斋论词杂著》、《晋略》,编有《宋四家词选》。 \tabularnewline\hline
  康有为 & \begin{description}
  \item[字] 广厦
  \item[号] 长素
  \item[谥] 
  \item[尊] 康南海
  \item[生] 广东南海
  \end{description} & 康有为(1858年3月19日-1927年3月31日),清末维新变法派主要发起者,原名祖诒,字广厦,号长素,又号明夷、更生、西樵山人、游存叟、天游化人,广东省南海县丹灶苏村人,人称康南海,光绪廿一年(1895年)进士,曾与弟子梁启超合作戊戌变法,变法失败后,被慈禧太后通缉而出逃。1912年宣统退位后,康有为继续反对共和,1917年曾与张勋合作,发动兵变,拥立宣统帝,是为辫军复辟,但十二日之内就被段祺瑞讨平。1927年在一场宴会后病逝,被质疑是政敌下毒。康有为的理想和政治主张主要在他撰写的《大同书》中得到体现。 \tabularnewline\hline
  王国维 & \begin{description}
  \item[字] 静安
  \item[号] 观堂
  \item[谥] 忠悫
  \item[尊] 
  \item[生] 浙江杭州
  \end{description} & 王国维(1877年12月3日-1927年6月2日),字静安,又字伯隅,晚号观堂(甲骨四堂之一),谥忠悫(què)。浙江杭州府海宁人,国学大师。王国维与梁启超、陈寅恪、和赵元任号称清华国学研究院的“四大导师”。中国新学术的开拓者,连接中西美学的大家,在文学、美学、史学、哲学、金石学、甲骨文、考古学等领域成就卓著。甲骨四堂之一。王国维精通英文、德文、日文,使他在研究宋元戏曲史时独树一帜,成为用西方文学原理批评中国旧文学的第一人。陈寅恪认为王国维的学术成就“几若无涯岸之可望、辙迹之可寻”。著述甚丰,有《海宁王静安先生遗书》、《红楼梦评论》、《宋元戏曲考》、《人间词话》、《观堂集林》、《古史新证》、《曲录》、《殷周制度论》、《流沙坠简》等62种。 \tabularnewline
  \bottomrule
\end{longtable}


%%% Local Variables:
%%% mode: latex
%%% TeX-engine: xetex
%%% TeX-master: "../../Main"
%%% End:


%%% Local Variables:
%%% mode: latex
%%% TeX-engine: xetex
%%% TeX-master: "../../Main"
%%% End:



%%% Local Variables:
%%% mode: latex
%%% TeX-engine: xetex
%%% TeX-master: "../Main"
%%% End:
 % 附录

\end{document}

%%% Local Variables:
%%% mode: latex
%%% TeX-engine: xetex
%%% TeX-master: t
%%% End:
