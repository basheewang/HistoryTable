%% -*- coding: utf-8 -*-
%% Time-stamp: <Chen Wang: 2018-07-10 00:58:17>

\documentclass[zihao=-4]{ctexbook}
\ctexset{
  chapter = {
    name = {第,卷},
    format = \centering\Large\bfseries\heiti,
    beforeskip = 10pt,
    afterskip = 20pt,
    titleformat = \chaptertitleformat
  },
  section = {
    name = {第,章},
    number = \chinese{section},
    format = \newpage\large\heiti,
    afterskip = 10pt,
    beforeskip = 10pt,
  },
  subsection = {
    name = {第,节},
    number = \chinese{subsection},
    format = \heiti,
    afterskip = 0pt,
    beforeskip = 0pt,
  }
}

\usepackage{varwidth}
\newcommand{\chaptertitleformat}[1]{%%
  \begin{varwidth}[t]{.7\linewidth}#1\end{varwidth}}

% Some extra packages
\usepackage{listings}
\lstset{
  basicstyle=\ttfamily,
  escapeinside={||},
  mathescape=true
}

\usepackage{fancyvrb}
\newsavebox{\FVerbBox}
\newenvironment{FVerbatim}
{\VerbatimEnvironment
  \begin{center}
\begin{BVerbatim}[commandchars=\\\{\}]}
 {\end{BVerbatim}
   \end{center}}

\usepackage{color}
\usepackage[perpage,hang,flushmargin]{footmisc}

\usepackage[hidelinks]{hyperref}
% \hypersetup{
%   colorlinks,
%   citecolor=black,
%   filecolor=black,
%   linkcolor=black,
%   urlcolor=black
% }

% For Meizu Pro5
\usepackage[
    % showframe,
    paperwidth=2.75in,
    paperheight=4.9in,
    left=0.1in,
    right=0.1in,
    top=0.1in,
    bottom=0.18in,
    footskip=10pt
]
{geometry}

% For Kindle 6"
% \usepackage[
% % showframe,
% paperwidth=3.6in,
% paperheight=4.8in,
% left=0.1in,
% right=0.1in,
% top=0.1in,
% bottom=0.18in,
% % footskip=10pt
% ]
% {geometry}

\setCJKmainfont{FZLanTingSong}

\newCJKfontfamily[fzsong]\fzsong{FZLanTingSong}
\newCJKfontfamily[kaiti]\kaiti{KaiTi}
\newCJKfontfamily[hkxm]\hkxm{FZBeiWeiKaiShu-S19_GB18030}
\newCJKfontfamily[song]\pml{PMingLiU}
\newCJKfontfamily[song]\hnm{HanaMin}
\newCJKfontfamily[fzk]\fzk{FZKaiS-Extended(SIP)}
\newCJKfontfamily[HeiTi]\SimHei{SimHei}

\usepackage{booktabs}
\usepackage{array}

\usepackage{enumitem}
% \setlength{\parindent}{0pt}
\setlist[description]{leftmargin=\parindent,labelindent=\parindent}
\setlist[enumerate]{
  nosep,
  itemsep=0pt,
  parsep=0pt,
  topsep=3pt,
  partopsep=0pt,
  leftmargin=0.8em,
  after=\vspace{-3.2em},
  before=\vspace{-1.6em},  
  }

\usepackage{longtable}

%% 改变页数字体大小
\renewcommand*{\thepage}{\scriptsize\arabic{page}}

%% 改变字体大小
\renewcommand{\footnotesize}{\fontsize{5pt}{6pt}\selectfont}
\renewcommand{\threept}{\fontsize{3pt}{3pt}\selectfont}

%% 使每页的脚注的数字为黑色圆圈数字
\usepackage{pifont}
\makeatletter
\newcommand*{\circnum}[1]{%
  \expandafter\@circnum\csname c@#1\endcsname
}
\newcommand*{\@circnum}[1]{%
  \ifnum#1<1 %
  \@ctrerr
  \else
  \ifnum#1>20 %
  \@ctrerr
  \else
  \ding{\the\numexpr 181+(#1)\relax}%
  \fi
  \fi
}
\makeatother

\renewcommand*{\thefootnote}{\circnum{footnote}}
\setlength{\footskip}{0pt} 
\setlength{\footnotesep}{0pt}

% \usepackage{footnote}
% \makesavenoteenv{tabular}
% \makesavenoteenv{table}

%% title & author
\title{\huge 历\quad{}代\quad{}年\quad{}表}
\author{}
\date{}

\begin{document}
\clearpage\maketitle
\clearpage\tableofcontents

\thispagestyle{empty}

\setcounter{page}{1}

%% -*- coding: utf-8 -*-
%% Time-stamp: <Chen Wang: 2018-07-08 23:41:09>

\chapter{前言}

本书包括历代君王年号。

%%% Local Variables:
%%% mode: latex
%%% TeX-engine: xetex
%%% TeX-master: "../Main"
%%% End:
%% -*- coding: utf-8 -*-
%% Time-stamp: <Chen Wang: 2018-07-09 21:16:31>

\chapter{战国{\tiny(BC402-BC221)}}

%% -*- coding: utf-8 -*-
%% Time-stamp: <Chen Wang: 2018-07-10 15:07:54>

\section{秦}

%% -*- coding: utf-8 -*-
%% Time-stamp: <Chen Wang: 2018-07-10 17:30:49>

\subsection{昭襄王{\tiny(BC306-BC251)}}


% \centering
\begin{longtable}{|>{\centering\scriptsize}m{2em}|>{\centering\scriptsize}m{1.3em}|>{\centering}m{8.8em}|}
  % \caption{秦王政}\\
  \toprule
  \SimHei \normalsize 年数 & \SimHei \scriptsize 公元 & \SimHei 大事件 \tabularnewline
  % \midrule
  \endfirsthead
  \toprule
  \SimHei \normalsize 年数 & \SimHei \scriptsize 公元 & \SimHei 大事件 \tabularnewline
  \midrule
  \endhead
  \midrule
  元年 & -306 & \tabularnewline\hline
  二年 & -305 & \tabularnewline\hline
  三年 & -304 & \tabularnewline\hline
  四年 & -303 & \tabularnewline\hline
  五年 & -302 & \tabularnewline\hline
  六年 & -301 & \tabularnewline\hline
  七年 & -300 & \tabularnewline\hline
  八年 & -299 & \tabularnewline\hline
  九年 & -298 & \tabularnewline\hline
  十年 & -297 & \tabularnewline\hline
  十一年 & -296 & \tabularnewline\hline
  十二年 & -295 & \tabularnewline\hline
  十三年 & -294 & \tabularnewline\hline
  十四年 & -293 & \tabularnewline\hline
  十五年 & -292 & \tabularnewline\hline
  十六年 & -291 & \tabularnewline\hline
  十七年 & -290 & \tabularnewline\hline
  十八年 & -289 & \tabularnewline\hline
  十九年 & -288 & \tabularnewline\hline
  二十年 & -287 & \tabularnewline\hline
  二一年 & -286 & \tabularnewline\hline
  二二年 & -285 & \tabularnewline\hline
  二三年 & -284 & \tabularnewline\hline
  二四年 & -283 & \tabularnewline\hline
  二五年 & -282 & \tabularnewline\hline
  二六年 & -281 & \tabularnewline\hline
  二七年 & -280 & \tabularnewline\hline
  二八年 & -279 & \tabularnewline\hline
  二九年 & -278 & \tabularnewline\hline
  三十年 & -277 & \tabularnewline\hline
  三一年 & -276 & \tabularnewline\hline
  三二年 & -275 & \tabularnewline\hline
  三三年 & -274 & \tabularnewline\hline
  三四年 & -273 & \tabularnewline\hline
  三五年 & -272 & \tabularnewline\hline
  三六年 & -271 & \tabularnewline\hline
  三七年 & -270 & \tabularnewline\hline
  三八年 & -269 & \tabularnewline\hline
  三九年 & -268 & \tabularnewline\hline
  四十年 & -267 & \tabularnewline\hline
  四一年 & -266 & \tabularnewline\hline
  四二年 & -265 & \tabularnewline\hline
  四三年 & -264 & \tabularnewline\hline
  四四年 & -263 & \tabularnewline\hline
  四五年 & -262 & \tabularnewline\hline
  四六年 & -261 & \tabularnewline\hline
  四七年 & -260 & \tabularnewline\hline
  四八年 & -259 & \tabularnewline\hline
  四九年 & -258 & \tabularnewline\hline
  五十年 & -257 & \tabularnewline\hline
  五一年 & -256 & \tabularnewline\hline
  五二年 & -255 & \tabularnewline\hline
  五三年 & -254 & \tabularnewline\hline
  五四年 & -253 & \tabularnewline\hline
  五五年 & -252 & \tabularnewline\hline
  五六年 & -251 & \tabularnewline
  \bottomrule
\end{longtable}

%%% Local Variables:
%%% mode: latex
%%% TeX-engine: xetex
%%% TeX-master: "../../Main"
%%% End:

%% -*- coding: utf-8 -*-
%% Time-stamp: <Chen Wang: 2018-07-10 17:30:44>

\subsection{孝文王{\tiny(BC250-BC250)}}


% \centering
\begin{longtable}{|>{\centering\scriptsize}m{2em}|>{\centering\scriptsize}m{1.3em}|>{\centering}m{8.8em}|}
  % \caption{秦王政}\\
  \toprule
  \SimHei \normalsize 年数 & \SimHei \scriptsize 公元 & \SimHei 大事件 \tabularnewline
  % \midrule
  \endfirsthead
  \toprule
  \SimHei \normalsize 年数 & \SimHei \scriptsize 公元 & \SimHei 大事件 \tabularnewline
  \midrule
  \endhead
  \midrule
  元年 & -250 & \tabularnewline
  \bottomrule
\end{longtable}

%%% Local Variables:
%%% mode: latex
%%% TeX-engine: xetex
%%% TeX-master: "../../Main"
%%% End:

%% -*- coding: utf-8 -*-
%% Time-stamp: <Chen Wang: 2018-07-10 17:30:54>

\subsection{庄襄王{\tiny(BC249-BC247)}}


% \centering
\begin{longtable}{|>{\centering\scriptsize}m{2em}|>{\centering\scriptsize}m{1.3em}|>{\centering}m{8.8em}|}
  % \caption{秦王政}\\
  \toprule
  \SimHei \normalsize 年数 & \SimHei \scriptsize 公元 & \SimHei 大事件 \tabularnewline
  % \midrule
  \endfirsthead
  \toprule
  \SimHei \normalsize 年数 & \SimHei \scriptsize 公元 & \SimHei 大事件 \tabularnewline
  \midrule
  \endhead
  \midrule
  元年 & -249 & \tabularnewline\hline
  二年 & -248 & \tabularnewline\hline
  三年 & -247 & \tabularnewline
  \bottomrule
\end{longtable}

%%% Local Variables:
%%% mode: latex
%%% TeX-engine: xetex
%%% TeX-master: "../../Main"
%%% End:

%% -*- coding: utf-8 -*-
%% Time-stamp: <Chen Wang: 2018-07-10 03:03:58>

\subsection{赢政{\tiny(BC246-BC221)}}


% \centering
\begin{longtable}{|>{\centering\scriptsize}m{2em}|>{\centering\scriptsize}m{1.3em}|>{\centering}m{9em}|}
  % \caption{秦王政}\\
  \toprule
  \SimHei \normalsize 年数 & \SimHei \scriptsize 公元 & \SimHei 大事件 \tabularnewline
  % \midrule
  \endfirsthead
  \toprule
  \SimHei \normalsize 年数 & \SimHei \scriptsize 公元 & \SimHei 大事件 \tabularnewline
  \midrule
  \endhead
  \midrule
  元年 & -246 & \begin{enumerate}
    \tiny
  \item 韩国水工郑国开始建造郑国渠,约十年后完工。
  \item 秦晋阳反,蒙骜击平之。
  \end{enumerate} \tabularnewline\hline
  二年 & -245 & \begin{enumerate}
    \tiny
  \item 秦麃公将卒攻卷,斩首三万。
  \item 赵以廉颇为假相国,伐魏,取繁阳。赵孝成王薨,子赵悼襄王偃立。
  \end{enumerate} \tabularnewline\hline
  三年 & -244 & \begin{enumerate}
    \tiny
  \item 秦蒙骜攻韩,取12城。
  \end{enumerate} \tabularnewline\hline
  四年 & -243 & \begin{enumerate}
    \tiny
  \item 春,秦蒙骜伐魏,取旸、有诡。三月,军罢。
  \item 秦质子归自赵;赵太子出归国。
  \item 七月,秦国蝗,疫。令百姓纳粟千石,拜爵一级。
  \item 魏安釐王薨,子魏景湣王增立。
  \item 赵悼襄王以李牧为将,伐燕,取武遂、方城。
  \item 逝世:魏安釐王、信陵君魏无忌。
  \end{enumerate} \tabularnewline\hline
  五年 & -242 & \begin{enumerate}
    \tiny
  \item 秦蒙骜伐魏,取酸枣、燕、虚、长平、雍丘、山阳等二十城;初置东郡。
  \item 燕王使剧辛将而伐赵。
  \end{enumerate} \tabularnewline\hline
  六年 & -241 & \begin{enumerate}
    \tiny
  \item 函谷关之战。
  \item 秦拔魏朝歌,及卫濮阳。
  \end{enumerate} \tabularnewline\hline
  七年 & -240 & \begin{enumerate}
    \tiny
  \item 秦置濮阳县,属东郡,并定其为东郡治所。
  \item 逝世:蒙骜、邹衍。
  \item 出生:陆贾。
  \item 天象:彗星光出东方,见北方,五月见西方。
  \end{enumerate} \tabularnewline\hline
  八年 & -239 & \begin{enumerate}
    \tiny
  \item 北扶余王国建立。
  \item 嫪毐封长信侯。
  \item 魏与赵邺。
  \item 文学:吕氏春秋编成。
  \item 逝世:长安君成蟜、韩桓惠王。
  \end{enumerate} \tabularnewline\hline
  九年 & -238 & \begin{enumerate}
    \tiny
  \item 嬴政亲政。
  \item 嫪毐叛乱,被秦王政夷灭三族。
  \item 秦伐魏,取垣、浦。
  \item 逝世:荀子、楚春申君黄歇、楚考烈王。
  \end{enumerate} \tabularnewline\hline
  十年 & -237 & \begin{enumerate}
    \tiny
  \item 齐王建拜会秦王政。
  \item 吕不韦免相。
  \item 秦王政下令驱除异邦客卿,李斯上书劝秦始皇收回逐客令。
  \end{enumerate} \tabularnewline\hline
  十一年 & -236 & \begin{enumerate}
    \tiny
  \item 郑国渠建成。
  \item 秦攻赵,赵攻燕\footnote{公元前236年,秦乘攻取赵的阏与、橑阳、邺、安阳等城,后又大举攻赵,遭到顽强抵抗。赵虽两次打败秦军,但兵力耗损殆尽。秦国西出太行山,突袭赵国邯郸拉开了统一战的的序幕。 赵国和燕国激战正酣,他想将秦国造成的领土损失在燕国身上补回来。这时秦国乘虚而入。赵国急忙命令大将李牧率军南下应敌。}。
  \end{enumerate} \tabularnewline\hline
  十二年 & -235 & \begin{enumerate}
    \tiny
  \item 秦攻楚国\footnote{秦继攻赵之后,即命辛梧率四郡兵,会同魏国,对楚国发起攻击。}。
  \item 吕不韦卒\footnote{因嫪毐集团叛乱事受牵连,被免除相邦职务,出居河南封地。不久,秦王政下令将其流放至蜀地(今四川),不韦忧惧交加,于是在三川郡(今河南洛阳)自鸩而亡。}。
  \end{enumerate} \tabularnewline\hline
  十三年 & -234 & \begin{enumerate}
    \tiny
  \item 秦攻赵\footnote{公元前234年,秦再度向赵南部进攻。桓龁避开正面渡河,改由漳河下游渡河迂回赵扈辄军的侧后,攻击邯郸东南的平阳。两军于平阳展开交战,赵军被击破,被斩10万人,赵将扈辄阵亡。赵王启用北部边疆名将李牧为统帅。李牧军曾歼灭匈奴入侵军10万之众,威震边疆,战斗力最强。李牧率军回赵,立即同秦桓龁军交战于宜安肥下地区,给秦军几乎全军覆灭的沉重打击,只有统帅桓龁带领少数护卫突围逃走。}。
  \item 韩非\footnote{韩非(约前281年-前233年),生活于战国末期时期的韩国(今属河南省新郑市)的思想家,为中国古代著名法家思想的代表人物,认为应该要“法”、“术”、“势”三者并重,是法家的集大成者。韩非出身韩国公族,与李斯均是荀子学生,后因其学识渊博,被秦始皇召唤入秦,正欲重用,却遭到妒忌的同窗李斯害死,在韩非死后,秦始皇在韩非的思想指引下,完成统一六国的帝业。韩非其学出于荀子,源于儒家,而成为法家,又推究老子思想,归本于道家。司马迁指出韩非喜好“刑名法术”且归本于道家的“黄老之学”,一套由“道”、“法”共同完善的政治统治理论。}作为韩国的使臣来到秦国,上书秦王,劝其先伐赵而缓伐韩。
  \end{enumerate} \tabularnewline\hline
  十四年 & -233 & \begin{enumerate}
    \tiny
  \item 韩非子卒。
  \item 燕抗秦\footnote{公元前233年,秦将樊於期叛逃至燕国后,太子丹的师傅鞠武害怕秦国以此借口攻燕,便策划送樊於期到头曼那里,利用熟悉秦国虚实的樊於期结连匈奴攻秦。可惜性急的太子丹等不得这种长远之计凑效,他决定派出荆轲刺杀自己的童年好友嬴政,为了能够解除嬴政的戒备,荆轲提出要携带两样礼物:樊於期的人头和燕国督亢地图(割地求和)。嬴政在逃过刺杀威胁后更以迅雷不及掩耳之势统一六国。}。
  \item 赵将李牧大败秦将桓齮\footnote{桓齮(yǐ)(?-前227年),战国末年秦国将军。杨宽的《战国史》认为桓齮就是樊於期。始皇十一年(前237年),桓齮与王翦和杨端和攻赵,取邺九城。秦始皇十四年,也就是赵王迁二年(前233年),桓齮从上党越太行山进攻赵的赤丽、宜安(石家庄东南),与赵将李牧战于肥下(宜安东北),为李牧所败,逃至燕国(《战国策》说是战败被杀,《资治通鉴》记载“秦师败绩,桓齮奔还”)后无相关记载。}于肥。
  \end{enumerate} \tabularnewline\hline
  十五年 & -232 & \begin{enumerate}
    \tiny
  \item 项羽出生。
  \item 太子丹回燕。
  \end{enumerate} \tabularnewline\hline
  十六年 & -231 & \begin{enumerate}
    \tiny
  \item 秦攻韩。
  \item 魏献丽邑。
  \item 赵国地震。
  \item 韩信出生。
  \end{enumerate} \tabularnewline\hline
  十七年 & -230 & \begin{enumerate}
    \tiny
  \item 韩国灭亡。
  \end{enumerate} \tabularnewline\hline
  十八年 & -229 & \begin{enumerate}
    \tiny
  \item 秦攻赵国。
  \item 李牧被杀。
  \end{enumerate} \tabularnewline\hline
  十九年 & -228 & \begin{enumerate}
    \tiny
  \item 秦破赵得和氏璧。
  \item 赵国灭亡。
  \end{enumerate} \tabularnewline\hline
  二十年 & -227 & \begin{enumerate}
    \tiny
  \item 荆轲刺秦王。
  \item 王翦、辛胜在易水西败燕、代联军。
  \end{enumerate} \tabularnewline\hline
  二一年 & -226 & \begin{enumerate}
    \tiny
  \item 秦军攻燕都。
  \item 秦攻蓟城。
  \end{enumerate} \tabularnewline\hline
  二二年 & -225 & \begin{enumerate}
    \tiny
  \item 魏国灭亡。
  \item 秦置砀郡,立浚仪(大梁)、启封两县。
  \end{enumerate} \tabularnewline\hline
  二三年 & -224 & \begin{enumerate}
    \tiny
  \item 秦楚之战。
  \item 秦置修武县。
  \end{enumerate} \tabularnewline\hline
  二四年 & -223 & \begin{enumerate}
    \tiny
  \item 楚将项燕自杀。
  \item 秦灭楚。
  \end{enumerate} \tabularnewline\hline
  二五年 & -222 & \begin{enumerate}
    \tiny
  \item 秦灭代。
  \item 秦灭燕。
  \end{enumerate} \tabularnewline
  \bottomrule
\end{longtable}

%%% Local Variables:
%%% mode: latex
%%% TeX-engine: xetex
%%% TeX-master: "../../Main"
%%% End:


%%% Local Variables:
%%% mode: latex
%%% TeX-engine: xetex
%%% TeX-master: "../../Main"
%%% End:


%%% Local Variables:
%%% mode: latex
%%% TeX-engine: xetex
%%% TeX-master: "../Main"
%%% End:

%% -*- coding: utf-8 -*-
%% Time-stamp: <Chen Wang: 2018-07-10 15:12:24>

\chapter{秦\tiny(BC221-BC207)}

%% -*- coding: utf-8 -*-
%% Time-stamp: <Chen Wang: 2018-07-10 17:29:45>

\section{始皇帝\tiny(BC221-BC210)}

\begin{longtable}{|>{\centering\scriptsize}m{2em}|>{\centering\scriptsize}m{1.3em}|>{\centering}m{8.8em}|}
  % \caption{秦王政}\
  \toprule
  \SimHei \normalsize 年数 & \SimHei \scriptsize 公元 & \SimHei 大事件 \tabularnewline
  % \midrule
  \endfirsthead
  \toprule
  \SimHei \normalsize 年数 & \SimHei \scriptsize 公元 & \SimHei 大事件 \tabularnewline
  \midrule
  \endhead
  \midrule
  二六年 & -221 & \begin{enumerate}
    \tiny
  \item 秦将王贲率军灭齐。
  \item 始皇统一中国。
  \item 秦攻百越\footnote{公元前221年,秦始皇统一后,令50万大军准备征服南方百越各部。秦军分5路南下,在越城岭遭到南方越人的顽强抵抗。}。
  \item 秦始凿灵渠\footnote{灵渠,建于秦始皇执政时期,是中国,也是世界上最早的运河之一。对中国岭南地区的开发起了重要作用。对今天的水利工程建设,仍然据有很好的参考价值}。
  \end{enumerate} \tabularnewline\hline
  二七年 & -220 & \begin{enumerate}
    \tiny
  \item 秦规划咸阳\footnote{公元前220年,秦始皇下令,将秦的东门由黄河延伸到上朐,并以咸阳和东门为中轴线规划新版图。}。
  \end{enumerate} \tabularnewline\hline
  二八年 & -219 & \begin{enumerate}
    \tiny
  \item 徐福\footnote{徐福,即徐巿”(在秦始皇本纪中称“徐巿”,在淮南衡山列传中称“徐福”)。(注意,是“巿”〔fú〕而不是“市”〔shì 〕),字君房,秦朝时齐地人,当时的著名方士。}出海。
  \item 始皇泰山封禅。
  \end{enumerate} \tabularnewline\hline
  二九年 & -218 & \begin{enumerate}
    \tiny
  \item 秦始皇第三次巡游,张良在博浪沙击始皇未中。
  \item 秦征岭南\footnote{尉佗真定人。公元前218年,奉秦始皇命令征岭南,略定南越后,任为南海龙川令。高后五年自立, 僭号“南越武帝”。 尉佗(?-前137年),真定(今石家庄市东古城)人。公元前218年,奉秦始皇命令征岭南,略定南越后,任为南海郡(治所在今广州市)龙川(今广档龙川县)令。秦二世时,赵佗受南海尉任嚣托,行南海尉事。秦亡后,出兵击并桂林郡( 治所在今广西桂平县西南古城)、象郡(治所在今广西崇左县),自立为南越王, 实行“和揖百越”的民族平等政策,采取一系列措施发展当地经济文化。}。
  \item 西瓯国反秦\footnote{公元前218年,西江中部的“西瓯国”起兵反秦,秦始皇派50万大军征讨。又派史禄在海阳山开凿灵渠,将湘江与漓江沟通,以保证军事上的运输。灵渠便成为中原汉人进入岭南的第一条主要通道。秦始皇灭了西瓯国,战争告一段落,秦“发诸尝捕亡人、赘婿、贾人略取陆梁地,为桂林、象郡、南海,以适遣戍。 ”(《史记.秦始皇本纪》)“五十万人守五岭。”(《集解》)这50万人,便是第一批汉族移民。秦始皇搞大迁徙,目的在于铲除六国的地方势力,把族人和故土分开,交叉汇编,徙到南蛮之地戍边,也就连根拔起,使之不能在秦的京城附近形成威胁,兹生复国复旧之梦。}。
  \end{enumerate} \tabularnewline\hline
  三十年 & -217 & \begin{enumerate}
    \tiny
  \item 始修建长城\footnote{秦灭六国之后,即开始北筑长城,每年征发民夫四十余万。全长7000多千米的长城,称作“九边重镇”,每镇设总兵官作为这一段长城的军事长官,受兵部的指挥,负责所辖军区内的防务或奉命支援相邻军区的防务。}。
  \end{enumerate} \tabularnewline\hline
  三一年 & -216 & \begin{enumerate}
    \tiny
  \item 秦改革屯田制\footnote{平民自报所占土地面积,自报耕地面积、土地产量及大小人丁。所报内容由乡出人审查核实,并统一评定产量,计算每户应纳税额,最后登记入册,上报到县,经批准后,即按登记数征收。此前著名的改革家商鞅还在秦国推行了包括土地制度在内的改革。提出了“算地”和“定分”的主张。“算地”就是对土地进行全面的调查核算,以作为制定土地政策的客观依据;“定分”就是用法律形式确认地主或平民对土地占有的“名分”,确认土地所有权。这些实际上都是土地登记的内容。}。
  \item 始皇微行咸阳,兰池遇盗,武士击杀之。大索二十日。
  \item 西汉七国之乱主谋,刘邦之侄,吴王刘濞出生。
  \end{enumerate} \tabularnewline\hline
  三二年 & -215 & \begin{enumerate}
    \tiny
  \item 始皇在今广西等地建立了桂林郡和象郡。
  \item 始皇东巡到达蓟城。
  \item 秦将蒙恬筑马邑城池,置马邑县。
  \end{enumerate} \tabularnewline\hline
  三三年 & -214 & \begin{enumerate}
    \tiny
  \item 灵渠建成。
  \item 秦设龙川县。
  \item 秦设南海郡。
  \item 秦占岭南,夺高阙、阳山、北假\footnote{公元前214年,秦始皇派遣50万军队分5路攻占岭南,任命任嚣为南海尉。派蒙恬渡过黄河去夺取高阙、阳山、北假一带地方,筑起堡垒以驱逐戎狄。迁移被贬谪的人,让他们充实新设置的县。}。
  \end{enumerate} \tabularnewline\hline
  三四年 & -213 & \begin{enumerate}
    \tiny
  \item 李斯任左丞相。
  \item 淳于越谏秦。
  \item 焚书事件。
  \item 秦颁行《挟书令》。
  \item 秦在五岭开山道筑三关,即横浦关、阳山关、湟鸡谷关。
  \item 秦始修筑驰道。
  \end{enumerate} \tabularnewline\hline
  三五年 & -212 & \begin{enumerate}
    \tiny
  \item 修建阿房宫。
  \item 扶苏被派往上郡(今天的陕西绥德)做大将蒙恬的监军。
  \item 焚书坑儒。
  \item 蒙恬率领大军修建了一条从咸阳到九原(今内蒙古包头市)的直道。
  \end{enumerate} \tabularnewline\hline
  三六年 & -211 & \begin{enumerate}
    \tiny
  \item 陨石事件\footnote{秦始皇三十六年,一颗流星坠落到了东郡。东郡是在秦始皇即位之初吕不韦主政时攻打下来的,当时此郡是齐、秦两国的交界地。现在已是大秦帝国的一个东方大郡。陨石落地还不可怕,可怕的是陨石上面刻的字“始皇帝死而地分”。这七个字非同小可!它代表了上天的旨意,预示着秦始皇将死,同时也预告了大秦帝国将亡。}。
  \item 汉惠帝刘盈出生。
  \item 秦置皮氏县。
  \end{enumerate} \tabularnewline\hline
  三七年 & -210 & \begin{enumerate}
    \tiny
  \item 始皇卒\footnote{秦始皇三十七年(公元前210年),秦始皇出巡至平原津(今德州平原县南六十里有张公故城,城东有水津)而病,秦始皇不愿意听到“死”,所以群臣莫敢言死事。8月28日行至沙丘(沙丘台在邢州平乡县东北二十里)病死。}。
  \item 扶苏被害。
  \item 胡亥\footnote{秦二世胡亥(前230年—前207年,在位时间前209年—前207年),也称二世皇帝。是秦始皇第二十六子,公子扶苏的弟弟。秦始皇出游南方病死途中时,在赵高与李斯的帮助下,杀害哥哥扶苏当上秦朝的二世皇帝。贾谊《过秦论》曰:“始皇既没,胡亥极愚,郦山未毕,复作阿房,以遂前策。云“凡所为贵有天下者,肆意极欲,大臣至欲罢先君所为”。诛斯、去疾,任用赵高。痛哉言乎!人头畜鸣。不威不伐恶,不笃不虚亡。距之不得留,残虐以促期,虽居形便之国,犹不得存。”}称帝,是为秦二世。
  \end{enumerate} \tabularnewline
  \bottomrule
\end{longtable}


%%% Local Variables:
%%% mode: latex
%%% TeX-engine: xetex
%%% TeX-master: "../Main"
%%% End:

%% -*- coding: utf-8 -*-
%% Time-stamp: <Chen Wang: 2018-07-10 17:29:32>

\section{秦二世\tiny(BC209-BC207)}

\begin{longtable}{|>{\centering\scriptsize}m{2em}|>{\centering\scriptsize}m{1.3em}|>{\centering}m{8.8em}|}
  % \caption{秦王政}\
  \toprule
  \SimHei \normalsize 年数 & \SimHei \scriptsize 公元 & \SimHei 大事件 \tabularnewline
  % \midrule
  \endfirsthead
  \toprule
  \SimHei \normalsize 年数 & \SimHei \scriptsize 公元 & \SimHei 大事件 \tabularnewline
  \midrule
  \endhead
  \midrule
  元年 & -209 & \begin{enumerate}
    \tiny
  \item 大泽乡起义。
  \item 刘邦起义。
  \item 项羽反秦。
  \item 冒顿即位。
  \end{enumerate} \tabularnewline\hline
  二年 & -208 & \begin{enumerate}
    \tiny
  \item 秦灭项梁。
  \item 孔鲋逝世。
  \item 陈胜卒。
  \item 李斯卒。
  \item 薛地会议。
  \item 统一越南。
  \end{enumerate} \tabularnewline\hline
  三年 & -207 & \begin{enumerate}
    \tiny
  \item 指鹿为马。
  \item 破釜沉舟。
  \item 胡亥被弑。
  \item 子婴即位,诛赵高,在位47天被废。
  \end{enumerate} \tabularnewline
  \bottomrule
\end{longtable}


%%% Local Variables:
%%% mode: latex
%%% TeX-engine: xetex
%%% TeX-master: "../Main"
%%% End:

%% -*- coding: utf-8 -*-
%% Time-stamp: <Chen Wang: 2018-07-10 17:29:53>

\section{子婴\tiny(BC206-BC206)}

\begin{longtable}{|>{\centering\scriptsize}m{2em}|>{\centering\scriptsize}m{1.3em}|>{\centering}m{8.8em}|}
  % \caption{秦王政}\
  \toprule
  \SimHei \normalsize 年数 & \SimHei \scriptsize 公元 & \SimHei 大事件 \tabularnewline
  % \midrule
  \endfirsthead
  \toprule
  \SimHei \normalsize 年数 & \SimHei \scriptsize 公元 & \SimHei 大事件 \tabularnewline
  \midrule
  \endhead
  \midrule
  元年 & -206 & \tabularnewline
  \bottomrule
\end{longtable}


%%% Local Variables:
%%% mode: latex
%%% TeX-engine: xetex
%%% TeX-master: "../Main"
%%% End:


%%% Local Variables:
%%% mode: latex
%%% TeX-engine: xetex
%%% TeX-master: "../Main"
%%% End:

%% -*- coding: utf-8 -*-
%% Time-stamp: <Chen Wang: 2018-07-10 00:39:09>

\chapter{西汉\tiny(BC202-8)}

%% -*- coding: utf-8 -*-
%% Time-stamp: <Chen Wang: 2018-07-10 00:44:24>

\section{汉高祖\tiny(BC206-BC195)}

\begin{longtable}{|>{\centering\scriptsize}m{2em}|>{\centering\small}m{2em}|>{\centering}m{8.3em}|}
  % \caption{秦王政}\
  \toprule
  \SimHei \normalsize 年数 & \SimHei \normalsize 公元 & \SimHei 大事件 \tabularnewline
  % \midrule
  \endfirsthead
  \toprule
  \SimHei 年数 & \SimHei 公元 & \SimHei 大事件 \tabularnewline
  \midrule
  \endhead
  \midrule
  元年 & -206 & \begin{enumerate}
    \tiny
  \item 秦朝灭亡。
  \item 鸿门宴。
  \item 项羽建立西楚王朝,自称西楚霸王。
  \end{enumerate} \tabularnewline\hline
  二年 & -205 & \begin{enumerate}
    \tiny
  \item 彭城之战。
  \item 成皋之战。
  \item 韩信破代、赵。
  \item 韩信灭燕、齐。
  \end{enumerate} \tabularnewline\hline
  三年 & -204 & \begin{enumerate}
    \tiny
  \item 背水一战。
  \item 南越国建立。
  \item 成皋之战。
  \end{enumerate} \tabularnewline\hline
  四年 & -203 & \begin{enumerate}
    \tiny
  \item 英布封王。
  \item 张耳封王。
  \end{enumerate} \tabularnewline\hline
  五年 & -202 & \begin{enumerate}
    \tiny
  \item 十二月垓下之战,汉灭楚统一天下,汉王刘邦即皇帝位。
  \item 汉置长安县、无锡县。
  \item 七月,燕王臧荼起兵反汉。
  \item 十月,刘邦率军亲征灭燕,俘杀臧荼。刘邦立卢绾为燕王。
  \item 汉高祖册封无诸为闽越王,封国闽越,首都冶城位于今之福州。
  \end{enumerate} \tabularnewline\hline
  
  \bottomrule
\end{longtable}


%%% Local Variables:
%%% mode: latex
%%% TeX-engine: xetex
%%% TeX-master: "../Main"
%%% End:

% %% -*- coding: utf-8 -*-
%% Time-stamp: <Chen Wang: 2018-07-10 17:29:32>

\section{秦二世\tiny(BC209-BC207)}

\begin{longtable}{|>{\centering\scriptsize}m{2em}|>{\centering\scriptsize}m{1.3em}|>{\centering}m{8.8em}|}
  % \caption{秦王政}\
  \toprule
  \SimHei \normalsize 年数 & \SimHei \scriptsize 公元 & \SimHei 大事件 \tabularnewline
  % \midrule
  \endfirsthead
  \toprule
  \SimHei \normalsize 年数 & \SimHei \scriptsize 公元 & \SimHei 大事件 \tabularnewline
  \midrule
  \endhead
  \midrule
  元年 & -209 & \begin{enumerate}
    \tiny
  \item 大泽乡起义。
  \item 刘邦起义。
  \item 项羽反秦。
  \item 冒顿即位。
  \end{enumerate} \tabularnewline\hline
  二年 & -208 & \begin{enumerate}
    \tiny
  \item 秦灭项梁。
  \item 孔鲋逝世。
  \item 陈胜卒。
  \item 李斯卒。
  \item 薛地会议。
  \item 统一越南。
  \end{enumerate} \tabularnewline\hline
  三年 & -207 & \begin{enumerate}
    \tiny
  \item 指鹿为马。
  \item 破釜沉舟。
  \item 胡亥被弑。
  \item 子婴即位,诛赵高,在位47天被废。
  \end{enumerate} \tabularnewline
  \bottomrule
\end{longtable}


%%% Local Variables:
%%% mode: latex
%%% TeX-engine: xetex
%%% TeX-master: "../Main"
%%% End:

% \input{Han/XiChu}

%%% Local Variables:
%%% mode: latex
%%% TeX-engine: xetex
%%% TeX-master: "../Main"
%%% End:


\end{document}

%%% Local Variables:
%%% mode: latex
%%% TeX-engine: xetex
%%% TeX-master: t
%%% End:
