%% -*- coding: utf-8 -*-
%% Time-stamp: <Chen Wang: 2018-07-10 01:37:37>

\documentclass[zihao=-4]{ctexbook}
\ctexset{
  chapter = {
    name = {第,卷},
    format = \centering\Large\bfseries\heiti,
    beforeskip = 10pt,
    afterskip = 20pt,
    titleformat = \chaptertitleformat
  },
  section = {
    name = {第,章},
    number = \chinese{section},
    format = \newpage\large\heiti,
    afterskip = 10pt,
    beforeskip = 10pt,
  },
  subsection = {
    name = {第,节},
    number = \chinese{subsection},
    format = \heiti,
    afterskip = 0pt,
    beforeskip = 0pt,
  }
}

\usepackage{varwidth}
\newcommand{\chaptertitleformat}[1]{%%
  \begin{varwidth}[t]{.7\linewidth}#1\end{varwidth}}

% Some extra packages
\usepackage{listings}
\lstset{
  basicstyle=\ttfamily,
  escapeinside={||},
  mathescape=true
}

\usepackage{fancyvrb}
\newsavebox{\FVerbBox}
\newenvironment{FVerbatim}
{\VerbatimEnvironment
  \begin{center}
\begin{BVerbatim}[commandchars=\\\{\}]}
 {\end{BVerbatim}
   \end{center}}

\usepackage{color}
\usepackage[perpage,hang,flushmargin]{footmisc}

\usepackage[hidelinks]{hyperref}
% \hypersetup{
%   colorlinks,
%   citecolor=black,
%   filecolor=black,
%   linkcolor=black,
%   urlcolor=black
% }

% For Meizu Pro5
\usepackage[
    % showframe,
    paperwidth=2.75in,
    paperheight=4.9in,
    left=0.1in,
    right=0.1in,
    top=0.1in,
    bottom=0.18in,
    footskip=10pt
]
{geometry}

% For Kindle 6"
% \usepackage[
% % showframe,
% paperwidth=3.6in,
% paperheight=4.8in,
% left=0.1in,
% right=0.1in,
% top=0.1in,
% bottom=0.18in,
% % footskip=10pt
% ]
% {geometry}

\setCJKmainfont{FZLanTingSong}

\newCJKfontfamily[fzsong]\fzsong{FZLanTingSong}
\newCJKfontfamily[kaiti]\kaiti{KaiTi}
\newCJKfontfamily[hkxm]\hkxm{FZBeiWeiKaiShu-S19_GB18030}
\newCJKfontfamily[song]\pml{PMingLiU}
\newCJKfontfamily[song]\hnm{HanaMin}
\newCJKfontfamily[fzk]\fzk{FZKaiS-Extended(SIP)}
\newCJKfontfamily[HeiTi]\SimHei{SimHei}

\usepackage{booktabs}
\usepackage{array}

\usepackage{enumitem}
% \setlength{\parindent}{0pt}
\setlist[description]{leftmargin=\parindent,labelindent=\parindent}
\setlist[enumerate]{
  nosep,
  itemsep=0pt,
  parsep=0pt,
  topsep=3pt,
  partopsep=0pt,
  leftmargin=0.8em,
  after=\vspace{-3.2em},
  before=\vspace{-1.6em},  
  }

\usepackage{longtable}

%% 改变页数字体大小
\renewcommand*{\thepage}{\scriptsize\arabic{page}}

%% 改变字体大小
\renewcommand{\footnotesize}{\fontsize{5pt}{6pt}\selectfont}
\newcommand{\threept}{\fontsize{3pt}{3pt}\selectfont}

%% 使每页的脚注的数字为黑色圆圈数字
\usepackage{pifont}
\makeatletter
\newcommand*{\circnum}[1]{%
  \expandafter\@circnum\csname c@#1\endcsname
}
\newcommand*{\@circnum}[1]{%
  \ifnum#1<1 %
  \@ctrerr
  \else
  \ifnum#1>20 %
  \@ctrerr
  \else
  \ding{\the\numexpr 181+(#1)\relax}%
  \fi
  \fi
}
\makeatother

\renewcommand*{\thefootnote}{\circnum{footnote}}
\setlength{\footskip}{0pt} 
\setlength{\footnotesep}{0pt}

% \usepackage{footnote}
% \makesavenoteenv{tabular}
% \makesavenoteenv{table}

%% title & author
\title{\huge 历\quad{}代\quad{}年\quad{}表}
\author{}
\date{}

\begin{document}
\clearpage\maketitle
\clearpage\tableofcontents

\thispagestyle{empty}

\setcounter{page}{1}

%% -*- coding: utf-8 -*-
%% Time-stamp: <Chen Wang: 2018-07-11 16:06:39>

\chapter{前言}

本书包括历代君王年号及大事件。下面为一些较为有用的链接:

\begin{itemize}
  \small \kaiti
  \item 十六国时期的其他割据势力参考此\href{https://zh.wikipedia.org/wiki/%E4%BA%94%E8%83%A1%E5%8D%81%E5%85%AD%E5%9B%BD#%E5%85%B6%E4%BB%96}{链接}。
  \item 中国年号、纪元、帝王查询\href{http://www.chinese-artists.net/year/}{网站}。
\end{itemize}

%%% Local Variables:
%%% mode: latex
%%% TeX-engine: xetex
%%% TeX-master: "../Main"
%%% End:
%% -*- coding: utf-8 -*-
%% Time-stamp: <Chen Wang: 2018-07-10 09:06:06>

\chapter{战国{\tiny(BC402-BC221)}}

%% -*- coding: utf-8 -*-
%% Time-stamp: <Chen Wang: 2018-07-10 13:38:16>

\section{东周}

%% -*- coding: utf-8 -*-
%% Time-stamp: <Chen Wang: 2018-07-13 01:01:27>

\subsection{威烈王{\tiny(BC425-BC402)}}


% \centering
\begin{longtable}{|>{\centering\scriptsize}m{2em}|>{\centering\scriptsize}m{1.3em}|>{\centering}m{8.8em}|}
  % \caption{秦王政}\\
  \toprule
  \SimHei \normalsize 年数 & \SimHei \scriptsize 公元 & \SimHei 大事件 \tabularnewline
  % \midrule
  \endfirsthead
  \toprule
  \SimHei \normalsize 年数 & \SimHei \scriptsize 公元 & \SimHei 大事件 \tabularnewline
  \midrule
  \endhead
  \midrule
  二三年 & -403 & \begin{enumerate}
    \tiny
  \item 命晉大夫魏斯\footnote{魏文侯(?-前396年),安邑(今山西夏县)人。中国战国时魏国统治者。姬姓,魏氏,名斯。周贞定王二十四年(前445年)继魏桓子位,周威烈王二年(前424年)称侯改元,威烈王二十三年(前403年)与韩、赵两家一起被周威烈王册封为诸侯,是为三家分晋,周安王六年(前396年)卒。}、趙籍\footnote{赵烈侯(?-前400年),是中国战国时期赵国的君主,原名赵籍,赵献侯之子。在位时用公仲连、牛畜、荀欣、徐越等人,为政待以仁义,约以王道。}、韓虔\footnote{韩景侯(?-前400年),名虔,韩武子之子。}爲諸侯\footnote{晋国(首府新田【山西省侯马市】)长期以来,在魏、赵、韩三大家族控制之下,国君不过空拥虚名,只在形式上,看起来晋国仍是一个完整的独立封国。本年(前四〇三年),周王国(首都洛阳【河南省洛阳市白马寺东】)国王(三十八任威烈王)姬午,下令擢升三大家族族长,亦即晋国三位国务官(大夫):魏斯当魏国(首府安邑【山西省夏县】)国君、赵籍当赵国(首府晋阳【山西省太原市】)国君、韩虔当韩国(首府平阳【山西省临汾市】)国君。晋国被三国瓜分后,只剩下一小片国土。}。
  \item 魏文侯使乐羊\footnote{乐羊,中山国人,战国时魏国的大将。是乐毅先祖。}伐中山\footnote{中山国,姬姓,春秋战国时白狄的一支——鲜虞仿照东周各诸侯国于公元前507年建立的国家,位于今河北省中部太行山东麓一带,中山国当时位于赵国和燕国之间,都于顾,后迁都于灵寿(今中国河北省灵寿县),因城中有山得国名。},克之。
  \item 吴起\footnote{吴起(前440年-前381年),中国战国初期军事家、政治家、改革家,兵家代表人物。卫国左氏(今山东省定陶县,一说山东省曹县东北)人。}杀妻以求为鲁将,大破齐师。
  \item 燕愍公\footnote{燕国(首府蓟城【北京市】)国君(三十四任)。}薨,子僖公立。
  \end{enumerate} \tabularnewline\hline
  二四年 & -402 & \tabularnewline
  \bottomrule
\end{longtable}

%%% Local Variables:
%%% mode: latex
%%% TeX-engine: xetex
%%% TeX-master: "../../Main"
%%% End:

%% -*- coding: utf-8 -*-
%% Time-stamp: <Chen Wang: 2018-07-10 12:11:58>

\subsection{安王{\tiny(BC401-BC376)}}


% \centering
\begin{longtable}{|>{\centering\scriptsize}m{2em}|>{\centering\scriptsize}m{1.3em}|>{\centering}m{9em}|}
  % \caption{秦王政}\\
  \toprule
  \SimHei \normalsize 年数 & \SimHei \scriptsize 公元 & \SimHei 大事件 \tabularnewline
  % \midrule
  \endfirsthead
  \toprule
  \SimHei \normalsize 年数 & \SimHei \scriptsize 公元 & \SimHei 大事件 \tabularnewline
  \midrule
  \endhead
  \midrule
  元年 & -401 & \tabularnewline\hline
  二年 & -400 & \tabularnewline\hline
  三年 & -399 & \tabularnewline\hline
  四年 & -398 & \tabularnewline\hline
  五年 & -397 & \tabularnewline\hline
  六年 & -396 & \tabularnewline\hline
  七年 & -395 & \tabularnewline\hline
  八年 & -394 & \tabularnewline\hline
  九年 & -393 & \tabularnewline\hline
  十年 & -392 & \tabularnewline\hline
  十一年 & -391 & \tabularnewline\hline
  十二年 & -390 & \tabularnewline\hline
  十三年 & -389 & \tabularnewline\hline
  十四年 & -388 & \tabularnewline\hline
  十五年 & -387 & \tabularnewline\hline
  十六年 & -386 & \tabularnewline\hline
  十七年 & -385 & \tabularnewline\hline
  十八年 & -384 & \tabularnewline\hline
  十九年 & -383 & \tabularnewline\hline
  二十年 & -382 & \tabularnewline\hline
  二一年 & -381 & \tabularnewline\hline
  二二年 & -380 & \tabularnewline\hline
  二三年 & -379 & \tabularnewline\hline
  二四年 & -378 & \tabularnewline\hline
  二五年 & -377 & \tabularnewline\hline
  二六年 & -376 & \tabularnewline
  \bottomrule
\end{longtable}

%%% Local Variables:
%%% mode: latex
%%% TeX-engine: xetex
%%% TeX-master: "../../Main"
%%% End:

%% -*- coding: utf-8 -*-
%% Time-stamp: <Chen Wang: 2018-07-16 22:54:57>

\subsection{烈王{\tiny(BC375-BC369)}}

周烈王(?-前369年),又称周夷烈王,姓姬,名喜,中国东周君主,在位7年。他是周安王之子。周烈王在位期间,秦献公迁都栎阳(今陕西省临潼市),开启秦国强盛的序幕。周烈王五年(庚戌,前371年),秦献公发兵攻占韩国六座城市。烈王六年(前370年)齐威王朝见周天子,威王贤名更盛。

% \centering
\begin{longtable}{|>{\centering\scriptsize}m{2em}|>{\centering\scriptsize}m{1.3em}|>{\centering}m{8.8em}|}
  % \caption{秦王政}\\
  \toprule
  \SimHei \normalsize 年数 & \SimHei \scriptsize 公元 & \SimHei 大事件 \tabularnewline
  % \midrule
  \endfirsthead
  \toprule
  \SimHei \normalsize 年数 & \SimHei \scriptsize 公元 & \SimHei 大事件 \tabularnewline
  \midrule
  \endhead
  \midrule
  元年 & -375 & \begin{enumerate}
    \tiny
  \item 日有食之。
  \item 韓滅鄭,因徙都之\footnote{韓本都平陽,其地屬漢之河東郡;中間徙都陽翟。鄭都新鄭,其地屬漢之河南郡。鄭桓公始封於鄭,其地屬漢之京兆;後滅虢、鄶而國於溱、洧之間,故曰新鄭,«左傳»鄭莊公所謂「吾先君新邑於此」是也。今韓旣滅鄭,自陽翟徙都之。韓旣都鄭,故時人亦謂韓王爲鄭王,考之«戰國策»、«韓非子»可見。}。
  \item 趙敬侯薨,子成侯種立。
  \end{enumerate} \tabularnewline\hline
  二年 & -374 & \tiny \kaiti 无记载 \tabularnewline\hline
  三年 & -373 & \begin{enumerate}
    \tiny
  \item 燕敗齊師於林狐。
  \item 魯伐齊,入陽關。
  \item 魏伐齊,至博陵。
  \item 燕僖公薨,子桓公立。
  \item 宋休公薨,子辟公立。
  \item 衞愼公\footnote{«諡法»︰敏以敬曰愼。«戴記»︰思慮深遠曰愼。}薨,子聲公訓立。
  \end{enumerate} \tabularnewline\hline
  四年 & -372 & \begin{enumerate}
    \tiny
  \item 趙伐衞,取都鄙\footnote{«周禮»︰太宰以八則治都鄙。«註»云︰都之所居曰鄙。都鄙,卿大夫之采邑。蓋周之制,四縣爲都,方四十里,一千六百井,積一萬四千四百夫;五酇爲鄙,鄙五百家也。此時衞國褊小,若都鄙七十三,以成周之制率之,其地廣矣,盡衞之提封,未必能及此數也。更俟博考。}七十三。
  \item 魏敗趙師于北藺。
  \end{enumerate} \tabularnewline\hline
  五年 & -371 & \begin{enumerate}
    \tiny
  \item 魏伐楚,取魯陽。
  \item 韓嚴遂弑哀侯,國人立其子懿侯\footnote{哀侯任命韩廆当宰相,但对严遂却更亲信。韩廆跟严遂之间,结仇至深,已不可解,互相想置对方于死地。严遂雇请杀手行刺韩廆。韩廆急奔哀侯身旁,哀侯为了保护他,把他抱住。然而杀手并不停止,仍刺杀韩廆;刀锋所及,哀侯也中刃而亡。(《战国策》认为聂政杀侠累和严遂杀哀侯是一件事,《史记》认为是两件事,《资治通鉴》根据《史记》。然而,二十六年间,韩国政府发生两次重大凶案,一次杀宰相,一次除了杀宰相外,还顺手杀了国君,太过突出。所以司马光对此并不敢十分肯定,在给刘道原信中,也曾表示他的怀疑。)}。
  \item 魏武侯薨,不立太子,子罃與公中緩爭立,國內亂。
  \end{enumerate} \tabularnewline\hline
  六年 & -370 & \begin{enumerate}
    \tiny
  \item 齊威王來朝。是時周室微弱,諸侯莫朝,而齊獨朝之,天下以此益賢威王。
  \item 趙伐齊,至鄄。
  \item 魏敗趙師于懷。
  \item 齊威王奖卽墨大夫,惩阿大夫,羣臣聳懼,莫敢飾詐,務盡其情,齊國大治,強於天下\footnote{齐国国君(四任)田因齐把即墨(山东省平度市东南)城主(大夫)召到首府临淄(山东省淄博市东临淄镇),对他说:“自从命你前去即墨,我每天都接到控告你的报告。然而我派人去即墨秘密调查,发现你开荒辟田,农作物遍野,人民生活富庶,官员清廉,齐国东部,得到平安。你之所以口碑不好,我了解,是你没有巴结我左右那些当权派而已。”于是,增加他一万户人家的封邑,作为奖励。又把阿邑(山东省东阿县)城主(大夫)召到首府临淄,对他说:“自从命你前去阿邑,我几乎每天都听到对你的赞扬。可是,我派人去阿邑秘密调查,发现完全不是那么回事,那里田野荒芜,农民贫困。前些时,赵国攻击鄄城(山东省鄄城县),你不率军救援。卫国占领薛陵(山东省阳谷县东北,薛陵跟阿邑之间航空距离不到十千米),你却假装不知道。我了解,我所听到的那些捧你场的话,都是你拿钱买来的。”于是下令把阿邑城主以及平常赞扬阿邑城主的一批官员,全都用大锅烹杀。全国大为震动,官员悚然戒惧,不敢再弄玄虚,大家改变态度,认真做事。齐国大治,成为强国。}。
  \item 楚肅王薨,無子,立其弟良夫,是爲宣王。
  \item 宋辟公薨,子剔成立。
  \end{enumerate} \tabularnewline\hline \newpage
  七年 & -369 & \begin{enumerate}
    \tiny
  \item 日有食之。
  \item 王崩,弟扁(音篇)立,是爲顯王。
  \item 魏大夫王錯出奔韓,韩懿侯乃與趙成侯合兵伐魏\footnote{魏国(首府安邑【山西省夏县】)内乱(参考前三七一年),已历时三年,国务官(大夫)王错,投奔韩国(首府新郑【河南省新郑县】)。韩国国务官(大夫)公孙颀,向国君(五任懿侯)韩若山建议说:“魏国已经腐烂,亡在眉睫,我们应该把它吞并。”韩若山遂跟赵国(首府邯郸【河北省邯郸市】)国君(四任成侯)赵种结盟,联合攻击魏国,在浊泽(山西省永济县西,与安邑航空距离五十千米)会战,魏军大败,韩、赵联军遂包围魏国首府安邑(山西省夏县)。赵种主张:“杀掉魏罃,立公中缓当魏国国君,割一部分士地给我们,我们就退兵。”韩若山说:“杀掉魏罃,我们落得一个残暴的名声。割让土地,又落得一个贪心的名声。不如把魏国一分为二,分成两个国家,使他们二人都当国君。魏国一分为二之后,就跟宋国、卫国一样,成了一个小国,我们就可永远摆脱魏国的压力。”赵种不同意,韩若山大不高兴,在夜晚撤军而去。赵种人单势孤,也只好撤军而去。魏罃遂乘机袭杀他的对头公中缓,继任国君(三任)。}。
  \end{enumerate} \tabularnewline
  \bottomrule
\end{longtable}

%%% Local Variables:
%%% mode: latex
%%% TeX-engine: xetex
%%% TeX-master: "../../Main"
%%% End:

%% -*- coding: utf-8 -*-
%% Time-stamp: <Chen Wang: 2018-07-10 17:30:31>

\subsection{显王{\tiny(BC368-BC321)}}


% \centering
\begin{longtable}{|>{\centering\scriptsize}m{2em}|>{\centering\scriptsize}m{1.3em}|>{\centering}m{8.8em}|}
  % \caption{秦王政}\\
  \toprule
  \SimHei \normalsize 年数 & \SimHei \scriptsize 公元 & \SimHei 大事件 \tabularnewline
  % \midrule
  \endfirsthead
  \toprule
  \SimHei \normalsize 年数 & \SimHei \scriptsize 公元 & \SimHei 大事件 \tabularnewline
  \midrule
  \endhead
  \midrule
  元年 & -368 & \tabularnewline\hline
  二年 & -367 & \tabularnewline\hline
  三年 & -366 & \tabularnewline\hline
  四年 & -365 & \tabularnewline\hline
  五年 & -364 & \tabularnewline\hline
  六年 & -363 & \tabularnewline\hline
  七年 & -362 & \tabularnewline\hline
  八年 & -361 & \tabularnewline\hline
  九年 & -360 & \tabularnewline\hline
  十年 & -359 & \tabularnewline\hline
  十一年 & -358 & \tabularnewline\hline
  十二年 & -357 & \tabularnewline\hline
  十三年 & -356 & \tabularnewline\hline
  十四年 & -355 & \tabularnewline\hline
  十五年 & -354 & \tabularnewline\hline
  十六年 & -353 & \tabularnewline\hline
  十七年 & -352 & \tabularnewline\hline
  十八年 & -351 & \tabularnewline\hline
  十九年 & -350 & \tabularnewline\hline
  二十年 & -349 & \tabularnewline\hline
  二一年 & -348 & \tabularnewline\hline
  二二年 & -347 & \tabularnewline\hline
  二三年 & -346 & \tabularnewline\hline
  二四年 & -345 & \tabularnewline\hline
  二五年 & -344 & \tabularnewline\hline
  二六年 & -343 & \tabularnewline\hline
  二七年 & -342 & \tabularnewline\hline
  二八年 & -341 & \tabularnewline\hline
  二九年 & -340 & \tabularnewline\hline
  三十年 & -339 & \tabularnewline\hline
  三一年 & -338 & \tabularnewline\hline
  三二年 & -337 & \tabularnewline\hline
  三三年 & -336 & \tabularnewline\hline
  三四年 & -335 & \tabularnewline\hline
  三五年 & -334 & \tabularnewline\hline
  三六年 & -333 & \tabularnewline\hline
  三七年 & -332 & \tabularnewline\hline
  三八年 & -331 & \tabularnewline\hline
  三九年 & -330 & \tabularnewline\hline
  四十年 & -329 & \tabularnewline\hline
  四一年 & -328 & \tabularnewline\hline
  四二年 & -327 & \tabularnewline\hline
  四三年 & -326 & \tabularnewline\hline
  四四年 & -325 & \tabularnewline\hline
  四五年 & -324 & \tabularnewline\hline
  四六年 & -323 & \tabularnewline\hline
  四七年 & -322 & \tabularnewline\hline
  四八年 & -321 & \tabularnewline
  \bottomrule
\end{longtable}

%%% Local Variables:
%%% mode: latex
%%% TeX-engine: xetex
%%% TeX-master: "../../Main"
%%% End:

%% -*- coding: utf-8 -*-
%% Time-stamp: <Chen Wang: 2018-07-10 17:30:19>

\subsection{慎靓王{\tiny(BC320-BC315)}}


% \centering
\begin{longtable}{|>{\centering\scriptsize}m{2em}|>{\centering\scriptsize}m{1.3em}|>{\centering}m{8.8em}|}
  % \caption{秦王政}\\
  \toprule
  \SimHei \normalsize 年数 & \SimHei \scriptsize 公元 & \SimHei 大事件 \tabularnewline
  % \midrule
  \endfirsthead
  \toprule
  \SimHei \normalsize 年数 & \SimHei \scriptsize 公元 & \SimHei 大事件 \tabularnewline
  \midrule
  \endhead
  \midrule
  元年 & -320 & \tabularnewline\hline
  二年 & -319 & \tabularnewline\hline
  三年 & -318 & \tabularnewline\hline
  四年 & -317 & \tabularnewline\hline
  五年 & -316 & \tabularnewline\hline
  六年 & -315 & \tabularnewline
  \bottomrule
\end{longtable}

%%% Local Variables:
%%% mode: latex
%%% TeX-engine: xetex
%%% TeX-master: "../../Main"
%%% End:

%% -*- coding: utf-8 -*-
%% Time-stamp: <Chen Wang: 2018-07-10 17:30:15>

\subsection{赧王{\tiny(BC314-BC256)}}


% \centering
\begin{longtable}{|>{\centering\scriptsize}m{2em}|>{\centering\scriptsize}m{1.3em}|>{\centering}m{8.8em}|}
  % \caption{秦王政}\\
  \toprule
  \SimHei \normalsize 年数 & \SimHei \scriptsize 公元 & \SimHei 大事件 \tabularnewline
  % \midrule
  \endfirsthead
  \toprule
  \SimHei \normalsize 年数 & \SimHei \scriptsize 公元 & \SimHei 大事件 \tabularnewline
  \midrule
  \endhead
  \midrule
  元年 & -314 & \tabularnewline\hline
  二年 & -313 & \tabularnewline\hline
  三年 & -312 & \tabularnewline\hline
  四年 & -311 & \tabularnewline\hline
  五年 & -310 & \tabularnewline\hline
  六年 & -309 & \tabularnewline\hline
  七年 & -308 & \tabularnewline\hline
  八年 & -307 & \tabularnewline\hline
  九年 & -306 & \tabularnewline\hline
  十年 & -305 & \tabularnewline\hline
  十一年 & -304 & \tabularnewline\hline
  十二年 & -303 & \tabularnewline\hline
  十三年 & -302 & \tabularnewline\hline
  十四年 & -301 & \tabularnewline\hline
  十五年 & -300 & \tabularnewline\hline
  十六年 & -299 & \tabularnewline\hline
  十七年 & -298 & \tabularnewline\hline
  十八年 & -297 & \tabularnewline\hline
  十九年 & -296 & \tabularnewline\hline
  二十年 & -295 & \tabularnewline\hline
  二一年 & -294 & \tabularnewline\hline
  二二年 & -293 & \tabularnewline\hline
  二三年 & -292 & \tabularnewline\hline
  二四年 & -291 & \tabularnewline\hline
  二五年 & -290 & \tabularnewline\hline
  二六年 & -289 & \tabularnewline\hline
  二七年 & -288 & \tabularnewline\hline
  二八年 & -287 & \tabularnewline\hline
  二九年 & -286 & \tabularnewline\hline
  三十年 & -285 & \tabularnewline\hline
  三一年 & -284 & \tabularnewline\hline
  三二年 & -283 & \tabularnewline\hline
  三三年 & -282 & \tabularnewline\hline
  三四年 & -281 & \tabularnewline\hline
  三五年 & -280 & \tabularnewline\hline
  三六年 & -279 & \tabularnewline\hline
  三七年 & -278 & \tabularnewline\hline
  三八年 & -277 & \tabularnewline\hline
  三九年 & -276 & \tabularnewline\hline
  四十年 & -275 & \tabularnewline\hline
  四一年 & -274 & \tabularnewline\hline
  四二年 & -273 & \tabularnewline\hline
  四三年 & -272 & \tabularnewline\hline
  四四年 & -271 & \tabularnewline\hline
  四五年 & -270 & \tabularnewline\hline
  四六年 & -269 & \tabularnewline\hline
  四七年 & -268 & \tabularnewline\hline
  四八年 & -267 & \tabularnewline\hline
  四九年 & -266 & \tabularnewline\hline
  五十年 & -265 & \tabularnewline\hline
  五一年 & -264 & \tabularnewline\hline
  五二年 & -263 & \tabularnewline\hline
  五三年 & -262 & \tabularnewline\hline
  五四年 & -261 & \tabularnewline\hline
  五五年 & -260 & \tabularnewline\hline
  五六年 & -259 & \tabularnewline\hline
  五七年 & -258 & \tabularnewline\hline
  五八年 & -257 & \tabularnewline\hline
  五九年 & -256 & \tabularnewline
  \bottomrule
\end{longtable}

%%% Local Variables:
%%% mode: latex
%%% TeX-engine: xetex
%%% TeX-master: "../../Main"
%%% End:


%%% Local Variables:
%%% mode: latex
%%% TeX-engine: xetex
%%% TeX-master: "../../Main"
%%% End:

%% -*- coding: utf-8 -*-
%% Time-stamp: <Chen Wang: 2018-07-10 15:07:54>

\section{秦}

%% -*- coding: utf-8 -*-
%% Time-stamp: <Chen Wang: 2018-07-10 17:30:49>

\subsection{昭襄王{\tiny(BC306-BC251)}}


% \centering
\begin{longtable}{|>{\centering\scriptsize}m{2em}|>{\centering\scriptsize}m{1.3em}|>{\centering}m{8.8em}|}
  % \caption{秦王政}\\
  \toprule
  \SimHei \normalsize 年数 & \SimHei \scriptsize 公元 & \SimHei 大事件 \tabularnewline
  % \midrule
  \endfirsthead
  \toprule
  \SimHei \normalsize 年数 & \SimHei \scriptsize 公元 & \SimHei 大事件 \tabularnewline
  \midrule
  \endhead
  \midrule
  元年 & -306 & \tabularnewline\hline
  二年 & -305 & \tabularnewline\hline
  三年 & -304 & \tabularnewline\hline
  四年 & -303 & \tabularnewline\hline
  五年 & -302 & \tabularnewline\hline
  六年 & -301 & \tabularnewline\hline
  七年 & -300 & \tabularnewline\hline
  八年 & -299 & \tabularnewline\hline
  九年 & -298 & \tabularnewline\hline
  十年 & -297 & \tabularnewline\hline
  十一年 & -296 & \tabularnewline\hline
  十二年 & -295 & \tabularnewline\hline
  十三年 & -294 & \tabularnewline\hline
  十四年 & -293 & \tabularnewline\hline
  十五年 & -292 & \tabularnewline\hline
  十六年 & -291 & \tabularnewline\hline
  十七年 & -290 & \tabularnewline\hline
  十八年 & -289 & \tabularnewline\hline
  十九年 & -288 & \tabularnewline\hline
  二十年 & -287 & \tabularnewline\hline
  二一年 & -286 & \tabularnewline\hline
  二二年 & -285 & \tabularnewline\hline
  二三年 & -284 & \tabularnewline\hline
  二四年 & -283 & \tabularnewline\hline
  二五年 & -282 & \tabularnewline\hline
  二六年 & -281 & \tabularnewline\hline
  二七年 & -280 & \tabularnewline\hline
  二八年 & -279 & \tabularnewline\hline
  二九年 & -278 & \tabularnewline\hline
  三十年 & -277 & \tabularnewline\hline
  三一年 & -276 & \tabularnewline\hline
  三二年 & -275 & \tabularnewline\hline
  三三年 & -274 & \tabularnewline\hline
  三四年 & -273 & \tabularnewline\hline
  三五年 & -272 & \tabularnewline\hline
  三六年 & -271 & \tabularnewline\hline
  三七年 & -270 & \tabularnewline\hline
  三八年 & -269 & \tabularnewline\hline
  三九年 & -268 & \tabularnewline\hline
  四十年 & -267 & \tabularnewline\hline
  四一年 & -266 & \tabularnewline\hline
  四二年 & -265 & \tabularnewline\hline
  四三年 & -264 & \tabularnewline\hline
  四四年 & -263 & \tabularnewline\hline
  四五年 & -262 & \tabularnewline\hline
  四六年 & -261 & \tabularnewline\hline
  四七年 & -260 & \tabularnewline\hline
  四八年 & -259 & \tabularnewline\hline
  四九年 & -258 & \tabularnewline\hline
  五十年 & -257 & \tabularnewline\hline
  五一年 & -256 & \tabularnewline\hline
  五二年 & -255 & \tabularnewline\hline
  五三年 & -254 & \tabularnewline\hline
  五四年 & -253 & \tabularnewline\hline
  五五年 & -252 & \tabularnewline\hline
  五六年 & -251 & \tabularnewline
  \bottomrule
\end{longtable}

%%% Local Variables:
%%% mode: latex
%%% TeX-engine: xetex
%%% TeX-master: "../../Main"
%%% End:

\input{Zhanguo/Qin/QinXiaoWenWang}
\input{Zhanguo/Qin/QinZhuangXiangWang}
%% -*- coding: utf-8 -*-
%% Time-stamp: <Chen Wang: 2018-07-09 23:26:42>

\section{始皇帝\tiny(BC221-BC210)}

\begin{longtable}{|>{\centering\scriptsize}m{2em}|>{\centering\small}m{2em}|>{\centering}m{8.3em}|}
  % \caption{秦王政}\
  \toprule
  \SimHei \normalsize 年数 & \SimHei \normalsize 公元 & \SimHei 大事件 \tabularnewline
  % \midrule
  \endfirsthead
  \toprule
  \SimHei 年数 & \SimHei 公元 & \SimHei 大事件 \tabularnewline
  \midrule
  \endhead
  \midrule
  二六年 & -221 & \begin{enumerate}
    \tiny
  \item 秦将\CJKunderline{王贲}率军灭齐。
  \item \CJKunderline{始皇}统一中国。
  \item 秦攻百越\footnote{公元前221年,秦始皇统一后,令50万大军准备征服南方百越各部。秦军分5路南下,在越城岭遭到南方越人的顽强抵抗。}。
  \item 秦始凿灵渠\footnote{灵渠,建于秦始皇执政时期,是中国,也是世界上最早的运河之一。对中国岭南地区的开发起了重要作用。对今天的水利工程建设,仍然据有很好的参考价值}。
  \end{enumerate} \tabularnewline\hline
  二七年 & -220 & \begin{enumerate}
    \tiny
  \item 秦规划咸阳\footnote{公元前220年,秦始皇下令,将秦的东门由黄河延伸到上朐,并以咸阳和东门为中轴线规划新版图。}。
  \end{enumerate} \tabularnewline\hline
  二八年 & -219 & \begin{enumerate}
    \tiny
  \item \CJKunderline{徐福}\footnote{徐福,即徐巿”(在秦始皇本纪中称“徐巿”,在淮南衡山列传中称“徐福”)。(注意,是“巿”〔fú〕而不是“市”〔shì 〕),字君房,秦朝时齐地人,当时的著名方士。}出海。
  \item \CJKunderline{始皇}泰山封禅。
  \end{enumerate} \tabularnewline\hline
  二九年 & -218 & \begin{enumerate}
    \tiny
  \item \CJKunderline{秦始皇}第三次巡游,\CJKunderline{张良}在博浪沙击始皇未中。
  \item 秦征岭南\footnote{尉佗真定人。公元前218年,奉秦始皇命令征岭南,略定南越后,任为南海龙川令。高后五年自立, 僭号“南越武帝”。 尉佗(?-前137年),真定(今石家庄市东古城)人。公元前218年,奉秦始皇命令征岭南,略定南越后,任为南海郡(治所在今广州市)龙川(今广档龙川县)令。秦二世时,赵佗受南海尉任嚣托,行南海尉事。秦亡后,出兵击并桂林郡( 治所在今广西桂平县西南古城)、象郡(治所在今广西崇左县),自立为南越王, 实行“和揖百越”的民族平等政策,采取一系列措施发展当地经济文化。}。
  \item 西瓯国反秦\footnote{公元前218年,西江中部的“西瓯国”起兵反秦,秦始皇派50万大军征讨。又派史禄在海阳山开凿灵渠,将湘江与漓江沟通,以保证军事上的运输。灵渠便成为中原汉人进入岭南的第一条主要通道。秦始皇灭了西瓯国,战争告一段落,秦“发诸尝捕亡人、赘婿、贾人略取陆梁地,为桂林、象郡、南海,以适遣戍。 ”(《史记.秦始皇本纪》)“五十万人守五岭。”(《集解》)这50万人,便是第一批汉族移民。秦始皇搞大迁徙,目的在于铲除六国的地方势力,把族人和故土分开,交叉汇编,徙到南蛮之地戍边,也就连根拔起,使之不能在秦的京城附近形成威胁,兹生复国复旧之梦。}。
  \end{enumerate} \tabularnewline\hline
  三十年 & -217 & \begin{enumerate}
    \tiny
  \item 始修建长城\footnote{秦灭六国之后,即开始北筑长城,每年征发民夫四十余万。全长7000多千米的长城,称作“九边重镇”,每镇设总兵官作为这一段长城的军事长官,受兵部的指挥,负责所辖军区内的防务或奉命支援相邻军区的防务。}。
  \end{enumerate} \tabularnewline\hline
  三一年 & -216 & \begin{enumerate}
    \tiny
  \item 秦改革屯田制\footnote{平民自报所占土地面积,自报耕地面积、土地产量及大小人丁。所报内容由乡出人审查核实,并统一评定产量,计算每户应纳税额,最后登记入册,上报到县,经批准后,即按登记数征收。此前著名的改革家商鞅还在秦国推行了包括土地制度在内的改革。提出了“算地”和“定分”的主张。“算地”就是对土地进行全面的调查核算,以作为制定土地政策的客观依据;“定分”就是用法律形式确认地主或平民对土地占有的“名分”,确认土地所有权。这些实际上都是土地登记的内容。}。
  \item 始皇微行咸阳,兰池遇盗,武士击杀之。大索二十日。
  \item 西汉七国之乱主谋,刘邦之侄,吴王刘濞出生。
  \end{enumerate} \tabularnewline\hline
  三二年 & -215 & \begin{enumerate}
    \tiny
  \item 始皇在今广西等地建立了桂林郡和象郡。
  \item 始皇东巡到达蓟城。
  \item 秦将蒙恬筑马邑城池,置马邑县。
  \end{enumerate} \tabularnewline\hline
  三三年 & -214 & \begin{enumerate}
    \tiny
  \item 灵渠建成。
  \item 秦设龙川县。
  \item 秦设南海郡。
  \item 秦占岭南,夺高阙、阳山、北假\footnote{公元前214年,秦始皇派遣50万军队分5路攻占岭南,任命任嚣为南海尉。派蒙恬渡过黄河去夺取高阙、阳山、北假一带地方,筑起堡垒以驱逐戎狄。迁移被贬谪的人,让他们充实新设置的县。}。
  \end{enumerate} \tabularnewline\hline
  三四年 & -213 & \begin{enumerate}
    \tiny
  \item 李斯任左丞相。
  \item 淳于越谏秦。
  \item 焚书事件。
  \item 秦颁行《挟书令》。
  \item 秦在五岭开山道筑三关,即横浦关、阳山关、湟鸡谷关。
  \item 秦始修筑驰道。
  \end{enumerate} \tabularnewline\hline
  三五年 & -212 & \begin{enumerate}
    \tiny
  \item 修建阿房宫。
  \item 扶苏被派往上郡(今天的陕西绥德)做大将蒙恬的监军。
  \item 焚书坑儒。
  \item 蒙恬率领大军修建了一条从咸阳到九原(今内蒙古包头市)的直道。
  \end{enumerate} \tabularnewline\hline
  三六年 & -211 & \begin{enumerate}
    \tiny
  \item 陨石事件\footnote{秦始皇三十六年,一颗流星坠落到了东郡。东郡是在秦始皇即位之初吕不韦主政时攻打下来的,当时此郡是齐、秦两国的交界地。现在已是大秦帝国的一个东方大郡。陨石落地还不可怕,可怕的是陨石上面刻的字“始皇帝死而地分”。这七个字非同小可!它代表了上天的旨意,预示着秦始皇将死,同时也预告了大秦帝国将亡。}。
  \item 汉惠帝刘盈出生。
  \item 秦置皮氏县。
  \end{enumerate} \tabularnewline\hline
  三七年 & -210 & \begin{enumerate}
    \tiny
  \item 始皇卒\footnote{秦始皇三十七年(公元前210年),秦始皇出巡至平原津(今德州平原县南六十里有张公故城,城东有水津)而病,秦始皇不愿意听到“死”,所以群臣莫敢言死事。8月28日行至沙丘(沙丘台在邢州平乡县东北二十里)病死。}。
  \item 扶苏被害。
  \item 胡亥\footnote{秦二世胡亥(前230年—前207年,在位时间前209年—前207年),也称二世皇帝。是秦始皇第二十六子,公子扶苏的弟弟。秦始皇出游南方病死途中时,在赵高与李斯的帮助下,杀害哥哥扶苏当上秦朝的二世皇帝。贾谊《过秦论》曰:“始皇既没,胡亥极愚,郦山未毕,复作阿房,以遂前策。云“凡所为贵有天下者,肆意极欲,大臣至欲罢先君所为”。诛斯、去疾,任用赵高。痛哉言乎!人头畜鸣。不威不伐恶,不笃不虚亡。距之不得留,残虐以促期,虽居形便之国,犹不得存。”}称帝,是为秦二世。
  \end{enumerate} \tabularnewline
  \bottomrule
\end{longtable}


%%% Local Variables:
%%% mode: latex
%%% TeX-engine: xetex
%%% TeX-master: "../Main"
%%% End:


%%% Local Variables:
%%% mode: latex
%%% TeX-engine: xetex
%%% TeX-master: "../../Main"
%%% End:


%%% Local Variables:
%%% mode: latex
%%% TeX-engine: xetex
%%% TeX-master: "../Main"
%%% End:

%% -*- coding: utf-8 -*-
%% Time-stamp: <Chen Wang: 2018-07-10 15:12:24>

\chapter{秦\tiny(BC221-BC207)}

\input{Qin/ShiHuangDi}
%% -*- coding: utf-8 -*-
%% Time-stamp: <Chen Wang: 2018-07-10 17:29:32>

\section{秦二世\tiny(BC209-BC207)}

\begin{longtable}{|>{\centering\scriptsize}m{2em}|>{\centering\scriptsize}m{1.3em}|>{\centering}m{8.8em}|}
  % \caption{秦王政}\
  \toprule
  \SimHei \normalsize 年数 & \SimHei \scriptsize 公元 & \SimHei 大事件 \tabularnewline
  % \midrule
  \endfirsthead
  \toprule
  \SimHei \normalsize 年数 & \SimHei \scriptsize 公元 & \SimHei 大事件 \tabularnewline
  \midrule
  \endhead
  \midrule
  元年 & -209 & \begin{enumerate}
    \tiny
  \item 大泽乡起义。
  \item 刘邦起义。
  \item 项羽反秦。
  \item 冒顿即位。
  \end{enumerate} \tabularnewline\hline
  二年 & -208 & \begin{enumerate}
    \tiny
  \item 秦灭项梁。
  \item 孔鲋逝世。
  \item 陈胜卒。
  \item 李斯卒。
  \item 薛地会议。
  \item 统一越南。
  \end{enumerate} \tabularnewline\hline
  三年 & -207 & \begin{enumerate}
    \tiny
  \item 指鹿为马。
  \item 破釜沉舟。
  \item 胡亥被弑。
  \item 子婴即位,诛赵高,在位47天被废。
  \end{enumerate} \tabularnewline
  \bottomrule
\end{longtable}


%%% Local Variables:
%%% mode: latex
%%% TeX-engine: xetex
%%% TeX-master: "../Main"
%%% End:

%% -*- coding: utf-8 -*-
%% Time-stamp: <Chen Wang: 2018-07-16 22:13:17>

\subsection{子婴{\tiny(BC693-BC680)}}

% \centering
\begin{longtable}{|>{\centering\scriptsize}m{2em}|>{\centering\scriptsize}m{1.3em}|>{\centering}m{8.8em}|}
  % \caption{秦王政}\\
  \toprule
  \SimHei \normalsize 年数 & \SimHei \scriptsize 公元 & \SimHei 大事件 \tabularnewline
  % \midrule
  \endfirsthead
  \toprule
  \SimHei \normalsize 年数 & \SimHei \scriptsize 公元 & \SimHei 大事件 \tabularnewline
  \midrule
  \endhead
  \midrule
  元年 & -693 & \tabularnewline\hline
  二年 & -692 & \tabularnewline\hline
  三年 & -691 & \tabularnewline\hline
  四年 & -690 & \tabularnewline\hline
  五年 & -689 & \tabularnewline\hline
  六年 & -688 & \tabularnewline\hline
  七年 & -687 & \tabularnewline\hline
  八年 & -686 & \tabularnewline\hline
  九年 & -685 & \tabularnewline\hline
  十年 & -684 & \tabularnewline\hline
  十一年 & -683 & \tabularnewline\hline
  十二年 & -682 & \tabularnewline\hline
  十三年 & -681 & \tabularnewline\hline
  十四年 & -680 & \tabularnewline
  \bottomrule
\end{longtable}

%%% Local Variables:
%%% mode: latex
%%% TeX-engine: xetex
%%% TeX-master: "../../Main"
%%% End:


%%% Local Variables:
%%% mode: latex
%%% TeX-engine: xetex
%%% TeX-master: "../Main"
%%% End:

%% -*- coding: utf-8 -*-
%% Time-stamp: <Chen Wang: 2018-07-10 17:45:04>

\chapter{西汉\tiny(BC202-8)}

\input{Han/ChuHanZhiZheng}
%% -*- coding: utf-8 -*-
%% Time-stamp: <Chen Wang: 2018-07-11 22:59:02>

\subsection{高祖\tiny(947-948)}

\subsubsection{天福}

\begin{longtable}{|>{\centering\scriptsize}m{2em}|>{\centering\scriptsize}m{1.3em}|>{\centering}m{8.8em}|}
  % \caption{秦王政}\
  \toprule
  \SimHei \normalsize 年数 & \SimHei \scriptsize 公元 & \SimHei 大事件 \tabularnewline
  % \midrule
  \endfirsthead
  \toprule
  \SimHei \normalsize 年数 & \SimHei \scriptsize 公元 & \SimHei 大事件 \tabularnewline
  \midrule
  \endhead
  \midrule
  元年 & 947 & \tabularnewline
  \bottomrule
\end{longtable}

\subsubsection{乾祐}

\begin{longtable}{|>{\centering\scriptsize}m{2em}|>{\centering\scriptsize}m{1.3em}|>{\centering}m{8.8em}|}
  % \caption{秦王政}\
  \toprule
  \SimHei \normalsize 年数 & \SimHei \scriptsize 公元 & \SimHei 大事件 \tabularnewline
  % \midrule
  \endfirsthead
  \toprule
  \SimHei \normalsize 年数 & \SimHei \scriptsize 公元 & \SimHei 大事件 \tabularnewline
  \midrule
  \endhead
  \midrule
  元年 & 948 & \tabularnewline\hline
  \bottomrule
\end{longtable}


%%% Local Variables:
%%% mode: latex
%%% TeX-engine: xetex
%%% TeX-master: "../../Main"
%%% End:

%% -*- coding: utf-8 -*-
%% Time-stamp: <Chen Wang: 2018-07-10 17:29:04>

\section{孝惠帝\tiny(BC195-BC188)}

\begin{longtable}{|>{\centering\scriptsize}m{2em}|>{\centering\scriptsize}m{1.3em}|>{\centering}m{8.8em}|}
  % \caption{秦王政}\
  \toprule
  \SimHei \normalsize 年数 & \SimHei \scriptsize 公元 & \SimHei 大事件 \tabularnewline
  % \midrule
  \endfirsthead
  \toprule
  \SimHei \normalsize 年数 & \SimHei \scriptsize 公元 & \SimHei 大事件 \tabularnewline
  \midrule
  \endhead
  \midrule
  元年 & -194 & \tabularnewline\hline
  二年 & -193 & \tabularnewline\hline
  三年 & -192 & \tabularnewline\hline
  四年 & -191 & \tabularnewline\hline
  五年 & -190 & \tabularnewline\hline
  六年 & -189 & \tabularnewline\hline
  七年 & -188 & \tabularnewline
  \bottomrule
\end{longtable}


%%% Local Variables:
%%% mode: latex
%%% TeX-engine: xetex
%%% TeX-master: "../Main"
%%% End:

\input{Han/QianShaoDi}
\input{Han/HouShaoDi}
%% -*- coding: utf-8 -*-
%% Time-stamp: <Chen Wang: 2018-07-10 17:28:15>

\section{孝文帝\tiny(BC179-BC157)}

\subsection{前元}

\begin{longtable}{|>{\centering\scriptsize}m{2em}|>{\centering\scriptsize}m{1.3em}|>{\centering}m{8.8em}|}
  % \caption{秦王政}\
  \toprule
  \SimHei \normalsize 年数 & \SimHei \scriptsize 公元 & \SimHei 大事件 \tabularnewline
  % \midrule
  \endfirsthead
  \toprule
  \SimHei \normalsize 年数 & \SimHei \scriptsize 公元 & \SimHei 大事件 \tabularnewline
  \midrule
  \endhead
  \midrule
  元年 & -179 & \tabularnewline\hline
  二年 & -178 & \tabularnewline\hline
  三年 & -177 & \tabularnewline\hline
  四年 & -176 & \tabularnewline\hline
  五年 & -175 & \tabularnewline\hline
  六年 & -174 & \tabularnewline\hline
  七年 & -173 & \tabularnewline\hline
  八年 & -172 & \tabularnewline\hline
  九年 & -171 & \tabularnewline\hline
  十年 & -170 & \tabularnewline\hline
  十一年 & -169 & \tabularnewline\hline
  十二年 & -168 & \tabularnewline\hline
  十三年 & -167 & \tabularnewline\hline
  十四年 & -166 & \tabularnewline\hline
  十五年 & -165 & \tabularnewline\hline
  十六年 & -164 & \tabularnewline
  \bottomrule
\end{longtable}


\subsection{后元}

\begin{longtable}{|>{\centering\scriptsize}m{2em}|>{\centering\scriptsize}m{1.3em}|>{\centering}m{8.8em}|}
  % \caption{秦王政}\
  \toprule
  \SimHei \normalsize 年数 & \SimHei \scriptsize 公元 & \SimHei 大事件 \tabularnewline
  % \midrule
  \endfirsthead
  \toprule
  \SimHei \normalsize 年数 & \SimHei \scriptsize 公元 & \SimHei 大事件 \tabularnewline
  \midrule
  \endhead
  \midrule
  元年 & -163 & \tabularnewline\hline
  二年 & -162 & \tabularnewline\hline
  三年 & -161 & \tabularnewline\hline
  四年 & -160 & \tabularnewline\hline
  五年 & -159 & \tabularnewline\hline
  六年 & -158 & \tabularnewline\hline
  七年 & -157 & \tabularnewline
  \bottomrule
\end{longtable}


%%% Local Variables:
%%% mode: latex
%%% TeX-engine: xetex
%%% TeX-master: "../Main"
%%% End:

%% -*- coding: utf-8 -*-
%% Time-stamp: <Chen Wang: 2018-07-11 19:39:50>

\subsection{孝静帝\tiny(534-550)}

\subsubsection{天平}

\begin{longtable}{|>{\centering\scriptsize}m{2em}|>{\centering\scriptsize}m{1.3em}|>{\centering}m{8.8em}|}
  % \caption{秦王政}\
  \toprule
  \SimHei \normalsize 年数 & \SimHei \scriptsize 公元 & \SimHei 大事件 \tabularnewline
  % \midrule
  \endfirsthead
  \toprule
  \SimHei \normalsize 年数 & \SimHei \scriptsize 公元 & \SimHei 大事件 \tabularnewline
  \midrule
  \endhead
  \midrule
  元年 & 534 & \tabularnewline\hline
  二年 & 535 & \tabularnewline\hline
  三年 & 536 & \tabularnewline\hline
  四年 & 537 & \tabularnewline
  \bottomrule
\end{longtable}

\subsubsection{元象}

\begin{longtable}{|>{\centering\scriptsize}m{2em}|>{\centering\scriptsize}m{1.3em}|>{\centering}m{8.8em}|}
  % \caption{秦王政}\
  \toprule
  \SimHei \normalsize 年数 & \SimHei \scriptsize 公元 & \SimHei 大事件 \tabularnewline
  % \midrule
  \endfirsthead
  \toprule
  \SimHei \normalsize 年数 & \SimHei \scriptsize 公元 & \SimHei 大事件 \tabularnewline
  \midrule
  \endhead
  \midrule
  元年 & 538 & \tabularnewline\hline
  二年 & 539 & \tabularnewline
  \bottomrule
\end{longtable}

\subsubsection{兴和}

\begin{longtable}{|>{\centering\scriptsize}m{2em}|>{\centering\scriptsize}m{1.3em}|>{\centering}m{8.8em}|}
  % \caption{秦王政}\
  \toprule
  \SimHei \normalsize 年数 & \SimHei \scriptsize 公元 & \SimHei 大事件 \tabularnewline
  % \midrule
  \endfirsthead
  \toprule
  \SimHei \normalsize 年数 & \SimHei \scriptsize 公元 & \SimHei 大事件 \tabularnewline
  \midrule
  \endhead
  \midrule
  元年 & 539 & \tabularnewline\hline
  二年 & 540 & \tabularnewline\hline
  三年 & 541 & \tabularnewline\hline
  四年 & 542 & \tabularnewline
  \bottomrule
\end{longtable}

\subsubsection{武定}

\begin{longtable}{|>{\centering\scriptsize}m{2em}|>{\centering\scriptsize}m{1.3em}|>{\centering}m{8.8em}|}
  % \caption{秦王政}\
  \toprule
  \SimHei \normalsize 年数 & \SimHei \scriptsize 公元 & \SimHei 大事件 \tabularnewline
  % \midrule
  \endfirsthead
  \toprule
  \SimHei \normalsize 年数 & \SimHei \scriptsize 公元 & \SimHei 大事件 \tabularnewline
  \midrule
  \endhead
  \midrule
  元年 & 543 & \tabularnewline\hline
  二年 & 544 & \tabularnewline\hline
  三年 & 545 & \tabularnewline\hline
  四年 & 546 & \tabularnewline\hline
  五年 & 547 & \tabularnewline\hline
  六年 & 548 & \tabularnewline\hline
  七年 & 549 & \tabularnewline\hline
  八年 & 550 & \tabularnewline
  \bottomrule
\end{longtable}


%%% Local Variables:
%%% mode: latex
%%% TeX-engine: xetex
%%% TeX-master: "../../Main"
%%% End:


%%% Local Variables:
%%% mode: latex
%%% TeX-engine: xetex
%%% TeX-master: "../Main"
%%% End:


\end{document}

%%% Local Variables:
%%% mode: latex
%%% TeX-engine: xetex
%%% TeX-master: t
%%% End:
