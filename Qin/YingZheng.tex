%% -*- coding: utf-8 -*-
%% Time-stamp: <Chen Wang: 2018-07-09 23:26:42>

\section{始皇帝\tiny(BC221-BC210)}

\begin{longtable}{|>{\centering\scriptsize}m{2em}|>{\centering\small}m{2em}|>{\centering}m{8.3em}|}
  % \caption{秦王政}\
  \toprule
  \SimHei \normalsize 年数 & \SimHei \normalsize 公元 & \SimHei 大事件 \tabularnewline
  % \midrule
  \endfirsthead
  \toprule
  \SimHei 年数 & \SimHei 公元 & \SimHei 大事件 \tabularnewline
  \midrule
  \endhead
  \midrule
  二六年 & -221 & \begin{enumerate}
    \tiny
  \item 秦将\CJKunderline{王贲}率军灭齐。
  \item \CJKunderline{始皇}统一中国。
  \item 秦攻百越\footnote{公元前221年,秦始皇统一后,令50万大军准备征服南方百越各部。秦军分5路南下,在越城岭遭到南方越人的顽强抵抗。}。
  \item 秦始凿灵渠\footnote{灵渠,建于秦始皇执政时期,是中国,也是世界上最早的运河之一。对中国岭南地区的开发起了重要作用。对今天的水利工程建设,仍然据有很好的参考价值}。
  \end{enumerate} \tabularnewline\hline
  二七年 & -220 & \begin{enumerate}
    \tiny
  \item 秦规划咸阳\footnote{公元前220年,秦始皇下令,将秦的东门由黄河延伸到上朐,并以咸阳和东门为中轴线规划新版图。}。
  \end{enumerate} \tabularnewline\hline
  二八年 & -219 & \begin{enumerate}
    \tiny
  \item \CJKunderline{徐福}\footnote{徐福,即徐巿”(在秦始皇本纪中称“徐巿”,在淮南衡山列传中称“徐福”)。(注意,是“巿”〔fú〕而不是“市”〔shì 〕),字君房,秦朝时齐地人,当时的著名方士。}出海。
  \item \CJKunderline{始皇}泰山封禅。
  \end{enumerate} \tabularnewline\hline
  二九年 & -218 & \begin{enumerate}
    \tiny
  \item \CJKunderline{秦始皇}第三次巡游,\CJKunderline{张良}在博浪沙击始皇未中。
  \item 秦征岭南\footnote{尉佗真定人。公元前218年,奉秦始皇命令征岭南,略定南越后,任为南海龙川令。高后五年自立, 僭号“南越武帝”。 尉佗(?-前137年),真定(今石家庄市东古城)人。公元前218年,奉秦始皇命令征岭南,略定南越后,任为南海郡(治所在今广州市)龙川(今广档龙川县)令。秦二世时,赵佗受南海尉任嚣托,行南海尉事。秦亡后,出兵击并桂林郡( 治所在今广西桂平县西南古城)、象郡(治所在今广西崇左县),自立为南越王, 实行“和揖百越”的民族平等政策,采取一系列措施发展当地经济文化。}。
  \item 西瓯国反秦\footnote{公元前218年,西江中部的“西瓯国”起兵反秦,秦始皇派50万大军征讨。又派史禄在海阳山开凿灵渠,将湘江与漓江沟通,以保证军事上的运输。灵渠便成为中原汉人进入岭南的第一条主要通道。秦始皇灭了西瓯国,战争告一段落,秦“发诸尝捕亡人、赘婿、贾人略取陆梁地,为桂林、象郡、南海,以适遣戍。 ”(《史记.秦始皇本纪》)“五十万人守五岭。”(《集解》)这50万人,便是第一批汉族移民。秦始皇搞大迁徙,目的在于铲除六国的地方势力,把族人和故土分开,交叉汇编,徙到南蛮之地戍边,也就连根拔起,使之不能在秦的京城附近形成威胁,兹生复国复旧之梦。}。
  \end{enumerate} \tabularnewline\hline
  三十年 & -217 & \begin{enumerate}
    \tiny
  \item 始修建长城\footnote{秦灭六国之后,即开始北筑长城,每年征发民夫四十余万。全长7000多千米的长城,称作“九边重镇”,每镇设总兵官作为这一段长城的军事长官,受兵部的指挥,负责所辖军区内的防务或奉命支援相邻军区的防务。}。
  \end{enumerate} \tabularnewline\hline
  三一年 & -216 & \begin{enumerate}
    \tiny
  \item 秦改革屯田制\footnote{平民自报所占土地面积,自报耕地面积、土地产量及大小人丁。所报内容由乡出人审查核实,并统一评定产量,计算每户应纳税额,最后登记入册,上报到县,经批准后,即按登记数征收。此前著名的改革家商鞅还在秦国推行了包括土地制度在内的改革。提出了“算地”和“定分”的主张。“算地”就是对土地进行全面的调查核算,以作为制定土地政策的客观依据;“定分”就是用法律形式确认地主或平民对土地占有的“名分”,确认土地所有权。这些实际上都是土地登记的内容。}。
  \item 始皇微行咸阳,兰池遇盗,武士击杀之。大索二十日。
  \item 西汉七国之乱主谋,刘邦之侄,吴王刘濞出生。
  \end{enumerate} \tabularnewline\hline
  三二年 & -215 & \begin{enumerate}
    \tiny
  \item 始皇在今广西等地建立了桂林郡和象郡。
  \item 始皇东巡到达蓟城。
  \item 秦将蒙恬筑马邑城池,置马邑县。
  \end{enumerate} \tabularnewline\hline
  三三年 & -214 & \begin{enumerate}
    \tiny
  \item 灵渠建成。
  \item 秦设龙川县。
  \item 秦设南海郡。
  \item 秦占岭南,夺高阙、阳山、北假\footnote{公元前214年,秦始皇派遣50万军队分5路攻占岭南,任命任嚣为南海尉。派蒙恬渡过黄河去夺取高阙、阳山、北假一带地方,筑起堡垒以驱逐戎狄。迁移被贬谪的人,让他们充实新设置的县。}。
  \end{enumerate} \tabularnewline\hline
  三四年 & -213 & \begin{enumerate}
    \tiny
  \item 李斯任左丞相。
  \item 淳于越谏秦。
  \item 焚书事件。
  \item 秦颁行《挟书令》。
  \item 秦在五岭开山道筑三关,即横浦关、阳山关、湟鸡谷关。
  \item 秦始修筑驰道。
  \end{enumerate} \tabularnewline\hline
  三五年 & -212 & \begin{enumerate}
    \tiny
  \item 修建阿房宫。
  \item 扶苏被派往上郡(今天的陕西绥德)做大将蒙恬的监军。
  \item 焚书坑儒。
  \item 蒙恬率领大军修建了一条从咸阳到九原(今内蒙古包头市)的直道。
  \end{enumerate} \tabularnewline\hline
  三六年 & -211 & \begin{enumerate}
    \tiny
  \item 陨石事件\footnote{秦始皇三十六年,一颗流星坠落到了东郡。东郡是在秦始皇即位之初吕不韦主政时攻打下来的,当时此郡是齐、秦两国的交界地。现在已是大秦帝国的一个东方大郡。陨石落地还不可怕,可怕的是陨石上面刻的字“始皇帝死而地分”。这七个字非同小可!它代表了上天的旨意,预示着秦始皇将死,同时也预告了大秦帝国将亡。}。
  \item 汉惠帝刘盈出生。
  \item 秦置皮氏县。
  \end{enumerate} \tabularnewline\hline
  三七年 & -210 & \begin{enumerate}
    \tiny
  \item 始皇卒\footnote{秦始皇三十七年(公元前210年),秦始皇出巡至平原津(今德州平原县南六十里有张公故城,城东有水津)而病,秦始皇不愿意听到“死”,所以群臣莫敢言死事。8月28日行至沙丘(沙丘台在邢州平乡县东北二十里)病死。}。
  \item 扶苏被害。
  \item 胡亥\footnote{秦二世胡亥(前230年—前207年,在位时间前209年—前207年),也称二世皇帝。是秦始皇第二十六子,公子扶苏的弟弟。秦始皇出游南方病死途中时,在赵高与李斯的帮助下,杀害哥哥扶苏当上秦朝的二世皇帝。贾谊《过秦论》曰:“始皇既没,胡亥极愚,郦山未毕,复作阿房,以遂前策。云“凡所为贵有天下者,肆意极欲,大臣至欲罢先君所为”。诛斯、去疾,任用赵高。痛哉言乎!人头畜鸣。不威不伐恶,不笃不虚亡。距之不得留,残虐以促期,虽居形便之国,犹不得存。”}称帝,是为秦二世。
  \end{enumerate} \tabularnewline
  \bottomrule
\end{longtable}


%%% Local Variables:
%%% mode: latex
%%% TeX-engine: xetex
%%% TeX-master: "../Main"
%%% End:
