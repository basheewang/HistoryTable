%% -*- coding: utf-8 -*-
%% Time-stamp: <Chen Wang: 2019-12-24 15:58:07>

\chapter{五代\tiny(907-960)}

\section{简介}

五代十國(907年-979年),是中國歷史上的一个時期,该时期由唐朝滅亡開始,至宋朝統一大部分汉地為止。五代十國实質上是唐末藩鎮割據和唐朝后期政治的延續。唐朝滅亡後,各地藩鎮紛紛自立為國,其中位在華北地區,軍力強盛的國家即五代,部分则是由沙陀所建立。虽然五代实力较为強大,但起扔無力控制整个中國本土,只是割据的地方朝廷。而其他割據一方的藩鎮,或自立為帝,或奉中原王朝為正统,其中十個國齡較長、國力較強的国家被統稱為十國。本時期時常發生地方大员叛變奪位的情况,使得戰亂连年不休,統治者多重武抑文。中原的內亂,也帶給契丹國南侵的機會,遼朝得以建立。五代十国是中国历史上的重要时期,其间河西和交趾地區逐渐托离中央,交趾(越南)最终脱离中国独立。

從唐朝滅亡至北宋建立半個多世紀期間,中原地區依次出現梁、唐、晉、漢、周五個朝代,史稱後梁、後唐、後晉、後漢、後周。907年梁王朱溫篡唐建立後梁标志着五代十國的开始。之后,晉王李克用之子李存勗滅后梁,建國後唐。後唐之后五代君王均出自李克用的子孫與其部屬。後唐歷經後唐明宗的擴張與整頓,國力日渐強盛,但因后来发生內亂,被河東節度使石敬瑭引契丹軍攻滅,随后後晉建立。不久,石敬瑭死后,契丹与晉的關係惡化,契丹軍南下滅後晉,建立遼朝。同时后晉河東節度使劉知遠在太原府稱帝建立後漢,收復中原。之后,後漢樞密使郭威篡后汉建後周并将皇位传给其样子柴荣,柴荣苦心經營,使得后周日渐拥有有统一中原的能力,但柴荣在北伐燕云十六州时不幸病亡。後周随后被殿前司都點檢趙匡胤所篡,五代結束。

十國方面,江南以吳國最強,而後被齊王李昪夺位,建國南唐,其次有吳越與閩國等。湖廣则被荊南、楚國與南漢等占据。南唐國力較盛,先後攻滅閩國、楚國,但多次用兵使得國力衰退,曾被後周所敗。蜀地有前蜀、後蜀,國家富強,是僅次於南唐的強國。北漢是惟一在北方的十國,是後漢劉氏後裔所建。趙匡胤建立宋朝(史稱北宋)後,與其弟宋太宗相繼掃蕩其他中原諸國及地區,最後於979年統一中國本土地區,十國結束。

五代十國大體上延續唐朝后期的政治體制,但是以「使」名官者很多。其中五代的變化很多,官職廢置不常,主要設有主管行政的三省六部、主管財政的三司與主管軍事的樞密院,這個制度後由宋朝繼承。十國其中一些政权臣服於五代,其政治架構大致上与五代等同。由於地方節度使不受管制,時常背叛中央,所以朝廷紛紛加強禁軍軍力以壓制地方實力派。為了抵制五代以来的武人干政现象,宋朝采取強幹弱枝政策。外交方面,唐朝時胡漢融合,外族陸續入住中國四周。唐朝崩潰進入五代十國後,出現一些外族國家,如沙陀建立五代後唐、後晉與後漢等。契丹先建立契丹國,南下滅後晉後建立遼朝。其他還有党項的定難軍,於北宋時建國西夏。這些都对宋朝的國際局勢造成深遠影響。

由于北方戰亂、外族入侵與天災不斷,南方十國在人口、經濟、文化與科技方面皆勝於北方五代。這也是華南經濟再度勝過華北的時刻,此後这一局面再也沒有逆轉。十國為擴展經濟,重視興修水利與經濟作物,发展貿易業、茶葉、紡織,錢塘江海塘也是在這個時期興建。由於關中經濟崩潰,五代大多定都于隋唐大運河的樞紐開封,都城的因素與燕雲十六州被外族統治影響宋朝的軍事與經濟。文化方面,本時期是詞發展的關鍵時期,禪宗也在本時期進入全興期。五代推行雕版印刷《九經》,保存許多儒學經典。繪畫方面,不論南方北方都有獨到之處。

安史之亂後,唐朝陸續出現許多不受控制的藩鎮。雖然藩鎮在唐憲宗時期大致臣服,但直到唐朝中后期,唐廷仍然因為宦官專政與牛李党争而無法根除藩鎮问题。由于河北被藩鎮控制、中原戰亂不堪,唐廷十分依賴江南的財富。然而江南地区在龐勛之變和黃巢之亂中遭到破壞,严重影响朝廷经济收入,唐朝逐漸走向滅亡。唐朝后期出現三個重要的藩鎮:沙陀人李國昌(原名朱邪赤心)因平亂有功受封河東節度使,治所為太原;原黃巢部将朱全忠(原名朱溫)因平亂有功,受封宣武軍節度使,治汴州(今河南開封);鳳翔節度使李茂貞(後封岐王)、護國軍節度使王重盈與鎮國軍節度使韓建為首的關中藩鎮勢力盛大,時常威脅唐廷。黃巢之亂平定後,黃巢降將秦宗權叛變,率軍在中原地區四處攻掠,一度攻陷東都(今河南洛陽),造成了「極目千里、無復煙火」的局面。亂事波及兩淮江南地區,當地群雄纷起抗敵,十國中的吳國與楚國得以建立。秦宗權之亂直到唐昭宗時才在朱全忠的努力下平定。

梁(朱全忠)、晉(李克用)與岐(李茂貞)這三派藩鎮影響了唐朝后期、五代前期的政治,而李克用的子孫與部屬,更成為五代後唐、後晉、後漢與後周的君主。朱全忠與李克用因故不合,雙方上至朝廷,下至藩鎮,都鬥爭不斷。朱全忠利用朝中的勢力打壓李克用,並且趁李克用與李茂貞等人抗衡之際威服河北各藩鎮,併吞護國軍、淄青軍等節度使領地。地盤的擴充使得朱全忠的勢力遠較李克用大。而關中的李茂貞雖然威脅唐廷,但因李克用與朱全忠的干涉而失败。888年唐昭宗繼位後,宰相崔胤与宦官韓全誨争权。唐昭宗被宦官韓全誨幽禁,崔胤緊急招喚朱全忠入援。而韓全誨强迫唐昭宗投靠李茂貞,朱全忠于是率軍圍困鳳翔。隔年,鳳翔軍糧草耗盡,李茂貞只好殺宦官韓全誨等人,與朱全忠和解。朱全忠趁機掌控朝中大權,還屠杀宦官數百人,派兵控制長安。崔胤後悔不已,有意擺脫朱全忠的威脅,暗中召募六軍十二衛,被朱全忠在長安的眼線所察觉。904年朱全忠殺崔胤,逼迫唐昭宗遷都洛陽,同年8月弑帝,另立昭宗子李柷为帝,即唐哀帝。朱全忠本想等統一後再夺取帝位,但因征淮南失利,於907年逼迫唐哀帝禅位,建國後梁,都開封(升汴州為開封府),唐朝亡,五代十國時期开始。

五代各朝雖然掌控中原與關中地區,但沒有像唐朝一樣成為藩鎮認可的共主,主要勢力範圍也不出華北地區,只能說是一個藩鎮型的朝廷。各地藩鎮在唐朝滅亡後,有的奉五代為宗主、有些依舊擁護唐室,其他則是保境固守或稱帝爭天下。不管他們的外交策略是如何,這些藩鎮都已獨立自主,其中十個具代表性(並非同時出現)的國家被史家稱為十國。中原地區歸附後梁的有義武軍節度使-北平王王處直、成德軍節度使-趙王王鎔等,半獨立的有鳳翔節度使岐王李茂貞、盧龍軍節度使燕王刘守光,獨立的是河東節度使晉王李克用。蜀地方面,蜀王王建建立前蜀;湖廣一帶,占據江陵周圍的荊南節度使高季興在後唐時建荊南,武安軍節度使馬殷建楚國,兩廣(嶺南)清海軍節度使劉隱建南漢;江南地區,占據兩淮吳地的淮南節度使楊行密建吳國,鎮海軍節度使錢鏐建吳越國,威武軍節度使王審知建閩國。另外,交趾地區,静海节度使曲承裕自立,在越南歷史稱為曲家,是越南地區脫離中國歷史的開端。党項族组成的定難軍也在陝北夏州一帶割據自立。位於河西瓜州的歸義軍一度建立金山國。

後梁太祖針對唐朝后期的弊端做出不少強烈改革。他極度厭惡宦官,曾拒絕在南方避難的宦官返回京城;討厭唐廷高級官員,啟用失意士人如李振、敬翔等人,並且聽從李振建議,屠殺宰相裴樞、崔遠等三十名高官,史稱白馬之禍。這些失意士人重實際而輕名義,是五代政治人物的代表。經濟方面重視農業發展,致力減輕賦稅;對軍隊十分嚴厲,如大將戰死,所部士卒一律斬首,稱「跋隊斬」。然而後梁太祖晚年荒淫無度,甚至不顧倫理,經常召諸子之妻入宮陪侍。外交方面,後梁立國之初,幾乎所有國家與藩鎮都表示臣服,只有晉、岐、前蜀與吳敵視後梁,依舊奉唐室年號。其中晉王李克用更是後梁太祖的死敌,自開國起後梁太祖就北伐晉國,在潞州(今山西長治)與李克用僵持不下,史稱潞州之戰。李克用因憂勞去世后,其子李存勗在李克寧、張承業的輔佐下而獲得勝利。910年後梁太祖懷疑成德軍與晉密結而率軍進伐,迫使成德軍王鎔與義武軍王處直倒向李存勗。李存勗率軍於柏鄉(今河北柏鄉)擊潰後梁軍,成功救援成德軍,而梁軍元氣大傷,史稱柏鄉之戰。北方之雄劉守光為人殘暴,且是晉的強敵。他於909年被後梁封為燕王,三年後稱帝建國燕國。隔年,李存勗派周德威攻打劉守光,後梁太祖親自率軍救援,但被晉軍擊潰。劉守光最後於隔年被李存勗攻滅。

後梁太祖敗退洛陽後病危,次子朱友珪不滿後梁太祖有意立義子朱友文為太子,趁機刺殺父皇而繼位。然而朱友珪荒淫无度,不得人心,913年其弟朱友貞聯合天雄軍節度使(即魏博軍)楊師厚進伐奪位,史稱後梁末帝。楊師厚去世後,天雄軍等河北諸鎮都陸續歸附晉國,再加上916年魏州之戰中梁軍慘敗於晉軍,後梁北疆只能勉強維持在黃河以南。918年李存勗率軍南征,與梁軍相持於濮州一帶。梁軍慘敗,但晉將盧龍軍節度使周德威戰死,梁晉戰爭沉寂了一段時期。921年張文禮殺成德軍節度使王鎔,控制成德军,聯合契丹國與後梁,對抗晉國。然而李存勗率軍於鎮州擊潰趙梁聯軍,又夺得河北三镇後。923年,李存勗在魏州稱帝(即後唐莊宗),以光復唐朝為由建國後唐,不久又二度南征。梁北面招討使王彥章採取牽制鄆州(今河南東平)的方式,成功圍堵唐軍於楊劉(今河南東阿)附近。雙方對峙良久,唐軍軍糧不足,有即将撤退的跡象。然而梁戶部尚書趙巖、張漢傑等人進讒言,使王彥章被撤換,後唐莊宗又率軍經鄆州迂迴攻入空虛無兵的開封府。城破之日,後梁末帝无奈命控鶴軍都指揮使皇甫麟殺死他,後梁亡。

後唐莊宗滅後梁後,定都洛陽。此時河北三鎮已定,後唐國力強盛。岐國李茂貞對後唐稱臣,後唐莊宗封他為秦王。924年李茂貞去世,後唐莊宗的長子李繼岌擔任鳳翔節度使,吞并了岐国。前蜀王建在建國後注重農桑、興修水利,使得前蜀在經濟與軍事都十分強盛。但918年王建去世後,其子王衍奢侈無度,殘暴昏庸。925年後唐莊宗派郭崇韬、李繼岌率军攻入成都,王衍投降,前蜀灭亡。

後唐對外強盛,但是內憂積重。後唐莊宗定都洛陽後,招回宦官以任樞要之職,任用李襲吉等保守派,一切與唐朝后期政治相同,朝政日益敗壞。後唐莊宗自認基業已固,不務政事,肆情縱慾,自取藝名「李天下」,寵信伶人敬新磨、伶官景進等人。當時军队龐大,国库吃緊,然而其妻刘皇后干預朝政、贪婪爱财,将税收一半归后宫,使得朝廷還要暫扣军粮以補其他支出,形成極大的隱憂,不久征蜀唐軍即因故兵變。

郭崇韬雖然完成滅蜀任務,但李繼岌對於不能深入參與軍務而感到不滿。他密報朝廷,意圖陷害郭崇韜。後唐莊宗有意先調查再決定,但刘皇后自行命李繼岌處決之。926年郭崇韜被殺,唐軍军心涣散,兵变四起。刘皇后又不願將自己的財物用於劳军,使局面更加恶化。不久,天雄軍士兵皇甫暉因賭博輸錢而煽動士兵造反,殺主帥楊仁晸,立偏將趙在禮為留後,是為鄴城之變,唐將李绍荣平定失敗,後唐莊宗只好派养兄李嗣源前往平定。李嗣源於魏州受部众與叛軍擁護,反而率軍南征後唐莊宗。各地唐軍不願為後唐莊宗作戰,汴州被陷,後唐莊宗因洛阳內乱中流箭而死。李嗣源攻入洛陽後殺盡叛臣而稱帝,即後唐明宗,後唐莊宗的長子李繼岌自殺於長安。

後唐明宗執政期間革除後唐莊宗時的弊政,朝政逐渐安定。他诛除宦官,任用士人;撤銷不少冗餘機關,建立三司等財政機關;提倡節儉,興修水利,關心百姓疾苦;加強中央軍力,建立侍衛親軍以壓制藩鎮。這是五代少見的穩定時期之一,史家稱後唐明宗是五代時期僅次於後周世宗的明君,他制定的一些制度也被宋朝所繼承。然而到晚年後唐再度走入內亂。933年後唐明宗重病,其子李從榮奪位被殺,幼子李從厚继位,即後唐閔帝。此時後唐明宗的兩位大將養子李從珂任鳳翔節度使、女婿石敬瑭任河東節度使,均擁重兵。宰相朱弘昭、馮贇本想以調動節度使的方式來分離軍權,反而激起叛變。934年李從珂以清君側為由攻入洛陽,後唐閔帝在逃往魏州途中被石敬瑭俘虜,最後被李從珂所殺。李從珂稱帝,即後唐末帝。內亂期間發生後蜀獨立之事。原來在前蜀滅亡後,後唐莊宗以孟知祥為四川節度使。不久後唐明宗叛變奪位,孟知祥練兵意圖獨立。932年孟知祥在併吞東川軍後被後唐明宗封為蜀王,並於後唐末帝發動內亂時稱帝建國後蜀。同年孟知祥去世,其子孟昶繼位。孟昶勵精圖治,擴展疆土,讓後蜀維持了三十年的和平局面。

後唐末帝與石敬瑭早在後唐明宗時就彼此不合。唐末帝繼位後十分猜忌石敬瑭,而石敬瑭也因畏懼而懷有叛變之心。936年後唐末帝把石敬瑭調任天平軍,並命張敬達、楊光遠率軍催促。石敬瑭聽从桑維翰與劉知遠的建議向契丹國借兵叛變,並且對遼太宗耶律德光稱兒,事後割讓燕云十六州给契丹,每年还要輸帛三十萬匹。張敬達聞知叛變之事後,率軍圍攻太原,石敬瑭堅守不下。當時卢龙軍的赵德钧和契丹可汗耶律德光有意合作共谋中原,石敬瑭大为惊惧,急令桑维翰见耶律德光。桑维翰跪于契丹帐前,苦苦哀求,才使耶律德光放弃与赵德钧合作的打算。耶律德光率軍解圍,帮助石敬瑭於太原建國後晉,即後晉高祖。937年晉軍與契丹軍大舉南下,楊光遠、趙德鈞等諸鎮陸續投降。晉軍獨自攻入洛陽,後唐末帝自焚而死,後唐滅亡。後晉高祖定都汴州,依約將燕云十六州割讓給契丹國,此後契丹國對五代的影響力達到最大。

五代进入后晋时期,國力大不如前,時常被契丹國威脅。江淮地區的吳與後繼的南唐国势強盛,他们採取聯合北方契丹國制约中原的策略,屢次征討周邊國家壯大勢力,成為中原王朝的一大威脅。吳國是淮南節度使楊行密建立的。早在秦宗權之亂時,秦的部屬畢師鐸率軍攻打揚州,楊行密在抗敌过程中发展割據势力,最後建立吳國。902年楊行密被唐廷封為吳王,建都廣陵,稱江都府。執政期間鼓勵農桑,穩定經濟,使江淮地區逐漸復甦。對外則擁護唐室,與宣武軍朱全忠(後梁建立者)敵對。905年杨行密去世,其子楊渥繼位。隔年江西钟传去世,諸子內亂,楊渥趁機派秦裴攻占江西,統一江淮。然而楊渥喜好玩樂,又猜忌功臣,大臣张颢、徐溫發動兵變,殺死楊渥。908年徐溫擁立楊渥之弟杨隆演為主,除掉想自立的張顥,徹底掌握吳國大權。

徐溫掌握大權後屢次攻伐吳越國未果,至後梁末期才和談。唐朝滅亡後,吳國不承認後梁宗主地位,仍然沿用唐哀帝年號,直到919年吳國改元,才正式與唐朝切斷關係。對內則逐步翦除楊氏舊將以穩固其勢力,然而專政的長子徐知训驕橫恣肆,曾因欺負吳王楊隆演而引來兵變,最後被部下朱瑾杀死。徐温养子徐知诰平定亂事,而徐知诰事徐溫甚孝謹,最後成為徐溫政權的繼承者。923年,楊隆演鬱鬱而終,其弟楊溥繼位,並於927年稱帝,即吳睿帝。927年徐溫去世,追封齊王,養子徐知誥繼位,成為吳國實際統治者。徐知诰生活俭朴,尊重吳帝和將領而頗得民心。937年徐知誥奪吳睿帝之位,吳亡,建國齊,都金陵,稱江寧府(今江蘇南京)。同年後唐滅亡,兩年後徐知誥自稱唐室後裔,改姓名李昪,建國南唐,即南唐烈祖。李昪建國後採取与民休息、與鄰国友好的政策,使國力持續强盛。943年李昪去世後,其子李璟繼位,即南唐元宗。李璟在位初期,南唐國力仍然強盛,對外聯合遼朝壓制後周,對四周國家也採取見機入侵的方式,陸續滅閩國與楚國。

在南唐東南方有吳越國與閩國。吳越國的建立者為鎮海、鎮東軍節度使錢鏐,都杭州,其疆域約同今浙江省。907年錢鏐被後梁封為吳越王,即吳越太祖。在位期間促進經濟發展,保境安民;對外奉中原王朝為宗主國,與吳和南唐為死對頭,這個策略一直維持到亡國為止。另外曾經派使冊封新羅、渤海國等國王,海中諸國皆奉他為君長。閩國是由福建觀察使王潮所建立,其與其弟王審知控制福州一帶,後為威武軍節度使,其疆域約同今福建省。王審知執政後於909年被後梁封為閩王,即閩太祖。在位期間也是提倡節儉,與民休息,並向五代稱臣,使閩國迅速發展。925年閩太祖去世後,其繼位者與宗室、大臣互相猜忌、鬥爭而使閩國逐漸衰弱。

943年閩景宗王延羲之弟王延政於建州(今福建建甌)稱帝,國號殷。隔年閩景宗被大臣所殺,國內大亂。945年王延政改國號為閩。同年南唐元宗趁機伐閩國,攻下建州,閩亡。然而吳越趁機介入,閩將李仁達以福州附吳越,泉州、漳州又為清源軍留從效所據,南唐最後只獲得建州與汀州(今福建西北部)等,與吳越國的關係持續惡化。不久南唐元宗趁楚內亂之際於951年派邊鎬攻滅,但隔年因楚將劉言起兵反抗,使得南唐又失去湖南一地。南唐連年用兵使國力受到很大消耗,所得之地也大半喪失。再加上南唐元宗為人柔和、好谀恶直,以是群小競進,政事日非。後周趁機於957年發兵南征南唐。南唐元宗戰敗,割讓江北十四州給後周,並且去掉自己的帝號,只稱江南國主,南唐元氣大傷。而比較有軍事才能的太子李弘冀,他在毒死意圖奪位的叔父李景遂後也去世。南唐元宗只好改立五子李煜為太子,但是李煜的書生氣質較重。南唐元宗為了避周軍與吳越軍聯合入侵金陵而遷都洪州,即南昌府(今江西南昌)。961年唐元宗去世後,由李煜即位,即南唐後主,還都金陵府。至此南唐無力威脅五代,只能保境安民。

而湖廣一帶有荊南、楚國、南漢。荆南又称南平、北楚,其疆域約為今湖北省西部。建立者高季兴为後梁梁太祖的將領,907年被封为荆南节度使,首府为江陵。荆南地小国弱,因而向四周各國稱臣。其國君高季兴和高從誨貪圖各國貢品而攔截搶奪,遭各國發兵威脅才願歸還,被稱為「高赖子」。後梁滅亡後,高季兴改向后唐称臣,在924年被後唐唐莊宗封為南平王,即武信王。后唐滅前蜀时,高季兴表示願意協助伐蜀,但並未實際行動,而後又向後唐索要前蜀土地。這些使後唐唐明宗大怒而發兵南征,所幸江南雨季使唐軍粮草不济而退。荊南與後唐的關係直到其子文獻王高從誨繼位後才和好。

楚國則由武安軍節度使馬殷所建立。秦宗權之亂時,秦的部署孫儒攻打兩淮楊行密,孫儒部將馬殷帶部分人馬經江西至湖南割據。907年後梁建立後,馬殷向後梁稱臣而被封為楚王,即武穆王。其勢力涵蓋今湖南與廣西省北部,對外臣服五代各朝,對內平定亂軍、強藩,並且採取保境安民的政策,使楚國國勢強盛。927年後唐封馬殷為楚國王,定都潭州,即長沙府。楚文昭王馬希範時期擴地至今廣西省東北部,國勢頗盛。然而馬希範在947年去世後國勢大亂,楚將擁護次子馬希廣繼立,使長子馬希萼不滿而叛變。950年馬希萼成功攻下長沙,即楚恭孝王。然而他縱酒荒淫,使得楚將王逵、周行逢舉兵叛變。他們擁護马殷嫡长孙馬光惠為武平節度使,以劉言為武平軍留後,率軍佔據朗州(今湖南常德)。不久徐威也擁護馬希崇為武安軍留後,放逐馬希萼。而馬希萼則於衡山再度被擁立。楚國分裂成馬光惠、馬希崇與馬希萼三派後,南唐元宗趁機於951年派邊鎬攻佔長沙,馬希崇與馬希萼先後投降,楚國亡。同時間南漢北取桂州(今廣西桂林)一帶,占據全嶺南地區。隔年武平軍留後劉言不願降唐,派王逵、周行逢攻下潭州,至此南唐全面退出湖南地區。劉言被後周封為武平節度使,由於與王逵對立而被王逵與周行逢所廢而死。而王逵貪得無厭,也被部下潘叔嗣所殺,武平軍之位最後由周行逢繼承。周行逢革除楚國劣政,愛護百姓,提倡廉潔。對將領用法嚴厲,果斷誅殺。湖南地區又恢復平穩,直至962年周行逢去世為止。

南漢是由清海軍節度使劉隱所建立,907年被後梁封為彭郡王,最後為南海王。劉隱穩固嶺南後重用當地士人,為日後建國打下基礎。911年劉隱去世後,由其弟劉龑繼位。劉龑在統一嶺南後於917年稱帝,即南漢高祖。國號大越,都番禺,號興王府。隔年改國號為漢,即南漢。南漢高祖與鄰國和好,推廣科舉制度。然而本身殘酷奢侈,每視殺人則喜,寵幸宦官,以至政事不寧。942年南漢高祖去世,其子劉玢繼位,即南漢殤帝。南漢殤帝貪圖享樂,當時有張遇賢叛亂,隔年被其兄劉晟所殺。劉晟自立為帝,即南漢中宗。在位期間,雖然奪取楚國容州(今廣西北流)、邕州(今廣西南寧),但是提倡嚴刑立威,為人殘暴,大肆屠殺皇族和大臣將領,南漢只剩宦官、宮女執政。958年去世後,由其子劉鋹繼位,即南漢後主。其间越南开始脱离中国统治。

當十國陸續衰弱或自保時,北方的後晉也因為契丹國的威脅而屢屢不安。當時後晉新立,財政匱乏,契丹貪求無厭,藩鎮多不願服從。為解決財政危機,後晉高祖採納桑維翰的建議,採取安撫藩鎮、恭謹契丹的方式,並且重視農業、商業以提升經濟。雖然契丹國得以安撫,但原燕雲十六州官員如吳巒、郭崇威恥臣於契丹,不願投降。各地藩鎮幾乎不服晉廷,有些甚至有意拉攏契丹國以奪位,此時有賴杜重威、李守貞等人平定。937年天雄軍(即魏博軍)范廷光反於魏州,前去平亂的張從賓也向他投降,並且殺後晉高祖之子石重信和石重乂。最後在范張聯軍逼近開封時,有賴侯益與杜重威率軍擊潰而平定。楊光遠自恃重兵而干預朝政,後晉高祖常屈從之,後來勾結契丹國叛變而被李守貞打敗而死。942年成德軍安重榮指斥後晉高祖父事契丹,要求出並討伐契丹國。但實際上卻是暗通契丹,意圖奪位。後晉高祖派杜重威率軍擊斬安重榮,史稱宗城之戰,並將其頭送與契丹國。同年位於代北的吐谷浑部,因為不願意投降契丹國,首領白承福率部投奔河東節度使劉知遠,契丹國派使問罪。後晉高祖最後在這些憂憤之中去世,其大臣馮道、景延廣以國家多難,宜立長君,就以侄子石重貴繼位於鄴都(今河北大名),即後晉出帝。

由於後晉的將領與百姓對屈尊異族而感到強烈不滿,後晉出帝听从景延廣建议,放棄對契丹國稱臣而改稱孫以洗刷屈辱。景延廣對契丹人的敵意十分強烈,他殺害契丹商人,逮捕契丹使者出氣,屢次對契丹挑釁。此舉引來契丹可汗耶律德光的憤怒,他於944年率軍南征。當時河北大旱,蝗蟲侵襲,契丹軍攻掠貝州(今河北清河)等地而還。隔年後晉出帝派杜重威率軍北伐,耶律德光聞之率大軍南下,最後杜重威成功的在白溝(河北定興、新城間)擊潰契丹軍。然而,後晉出帝於白溝之戰後日益驕奢,又以馮玉執政,賄絡公行,朝政敗壞。946年後晉出帝再以杜重威率軍北伐,與耶律德光在滹沱河會戰。此時杜重威有意奪位,反而向耶律德光投降。耶律德光趁機率聯軍直逼開封,後晉大將李守貞、張彥澤陸續投降,最後後晉出帝開城投降,後晉亡,史稱遼滅晉之戰。隔年耶律德光將國號改為「大遼」,即遼太宗,正式建立遼朝。遼太宗本來對經營中國地區很有信心,然而「打草穀」與掠人為奴的掠奪政策使中原百姓群聚反抗。其中河東軍劉知遠聽從張彥威的建議,以中原無主為由於太原稱帝,建國後漢,即後漢高祖。遼太宗壓制不了此局面,以天氣炎熱為由率軍北返。他命蕭翰留守開封,杜重威留守鄴都。最後於殺胡林(今河北欒城)去世,其兄子耶律兀欲繼位,即遼世宗。

後漢高祖在遼軍北返後開始收復中原。蕭翰得知消息後,劫持後唐宗室李從益稱帝於開封,而後北返。後漢高祖聞之派使殺李從益以定都開封,並派高行周與慕容彥超在魏州之戰戰役降服杜重威,諸鎮相繼歸附。948年後漢高祖去世,其子劉承祐继位,是为後汉隐帝,並以楊邠、郭威、史弘肇與王章為輔國大臣。當時河中節度使李守貞叛亂,有賴郭威平定。後汉隐帝年長後猜忌輔國大臣,與郭允明協議後於950年以遼軍寇河北為由派郭威鎮守鄴都,隨後大殺楊、史與王等大臣,又殺郭威一家,並召泰寧軍慕容彥超等急速入京。郭威聽從魏仁浦建議起兵南下,並派養子柴榮鎮守鄴都。隔年擊潰慕容彥超,攻入开封,後汉隐帝最後为郭允明等所杀。郭威本有意立劉崇子徐州軍劉贇為帝,先以李太后臨朝。當時恰巧遼軍入侵,郭威出師禦敵,但大軍至澶州(今河南濮陽)時,軍士擁護郭威稱帝,大軍返回開封。951年郭威称帝,建國後周,即後周太祖,後漢亡。

後周太祖登基後減除若干苛政,勵行節儉,使南流的人口再度有流回中原的傾向。然而劉贇被殺,使後舊漢將不服周廷。河東軍劉崇(後漢高祖劉知遠之弟)得知郭威称帝後,自立為帝,建國北漢。他依遼人為援,自稱侄皇帝,並且伺機伐周。舊漢將徐州鞏廷美與泰寧軍慕容彥超意圖叛變,有賴後周太祖陸續平定。

954年後周太祖去世,由养子柴榮繼位,即後周世宗。後周世宗是五代十國中的第一明君,於繼位之初遭遇北漢帝劉崇與遼將楊袞聯合南下。當時周廷驚恐,大多主張穩重行事,然而後周世宗親自擊潰漢遼聯軍,並斬臨陣後退的無能將領,史稱高平之戰。此後改革軍事制度,精簡中央禁軍,補充強健之士,形成「殿前諸班」的禁軍。內政方面,他招撫流亡,減少賦稅,穩定國內經濟。整頓吏治,延聘文人,打壓武人政治,使後周政治清明。955年又廢天下佛寺,獲取大量銅器以整頓經濟。軍事與經濟的提升都為日後統一中國本土而建立重要的基礎。

後周世宗在穩定國內後即意圖統一天下,他以「十年开拓天下,十年养百姓,十年致太平」為目標。955年率軍擊潰後蜀,占秦州漢中一帶。956年率兵擊潰南唐,獲得江北之地,迫南唐稱臣。959年後周世宗率軍北伐遼朝以收復燕云十六州,周军陸續攻陷瀛洲、莫州等地。當他准备收复幽州時,却突然生病,被迫班师。不久去世,其幼子柴宗訓即位,即後周恭帝。960年,殿前司都點檢趙匡胤以鎮定二州遭北漢、遼朝入侵為由率軍北禦,而後在開封的陳橋驛發生陳橋兵變,受禁軍擁護為帝。趙匡胤回師開封,廢黜後周恭帝,後周灭亡,五代結束。他建立宋朝,即宋太祖。

宋太祖繼位之時,十國仍有後蜀、北漢、南唐、吳越、南漢、荊南與湖南武平軍周行逢、閩南清源軍留從效等,這些國家或藩鎮大多奉宋朝為宗主或臣服之。宋太祖面對遼朝的威脅,採趙普「先易後難,先南後北」的策略統一中國地區。962年荊南主高保勗去世,同年湖南周行逢去世,兩國新主年幼無能。宋太祖趁機於隔年以平湖南之亂為由派兵南下併湖南,途中假道伐虢,併吞荊南。後蜀後主孟昶聞知荊南與湖南被併吞後,聯合北漢以拒宋師。然而其晚年奢侈逸樂,朝政不修,軍隊皆無戰鬥力。965年宋太祖派王全斌、崔彥進出鳳州(今陝西鳳縣)、劉光義、曹彬出歸州(今湖北秭歸),北東兩路同時入蜀。結果不出六十多日,後蜀帝孟昶投降,後蜀亡。其寵妃花蕊夫人在亡國後寫下:「君王城上豎降旗,妾在深宮哪得知,十四萬人齊解甲,更無一個是男兒。」南漢後主劉鋹將政事交給宦官龔澄樞及女侍中盧瓊仙等人。由於只信宦官,官員都需閹割才能進用。970年宋廷派潘美伐南漢,由於南漢將領大臣宗室皆死光,只有宦官領軍,隔年南漢帝劉鋹投降,南漢亡。

南唐後主李煜是詞壇高手,雖然終日以外患為憂,但不擅政事。當時有賴其弟李從善、大臣潘佑與將領林仁肇等人,尚且與後周得以對峙。宋朝建立後,李煜親近小人,濫殺大臣,終日與臣酣宴、愁思悲歌,南唐國勢混亂。975年宋太祖以南唐帝李煜稱病不入朝為由,派曹彬南征,並以吳越軍為輔夾攻。最後李煜投降,南唐亡。吳越國方面,雖然忠獻王錢弘佐時趁閩國內亂獲得福州,但是本身課稅繁重,民不堪苦。到忠懿王錢俶時,因為對宋朝十分恭順,宋太祖沒有奪取地。而閩南清源軍留從效割據一方,去世後多人爭位,最後由陳洪進奪得。978年錢俶與陳洪進納土歸順宋朝,吳越國與閩南清源軍亡。而十國最後一個國家北漢的末主是英武帝劉繼元。979年北宋宋太宗派潘美圍攻北漢都城太原,擊退遼國援兵,劉繼元投降,北漢滅亡。至此十國時期結束,正式進入宋朝時期。然而,尚有燕雲十六州還未收復。宋太宗滅亡北漢後不久,他不顧大臣反對,從太原北伐遼朝以圖收復燕雲十六州。起初宋軍攻下易州和涿州,但在燕京的高梁河之戰慘敗而退,至此進入宋遼對峙的時代。

後梁時五代十國局勢圖。圖中只有岐國、晉國、前蜀與吳國依舊奉唐室為尊,不承認後梁的地位。其他各地藩鎮大都向後梁臣服,其中盧龍軍(後建燕國)、成德軍(趙王)與義武軍(北平王)都具有獨立的地位,並向後梁稱臣,而荊南節度使由後梁直轄。

後唐時五代十國局勢圖。圖中後唐已經統一河北,直接面對契丹國的威脅,並且西進滅岐國、前蜀。但是在攻打定難軍時失敗,蜀地也在孟知祥的努力下建國後蜀,荊南也正式建國。

五代十國的疆域大抵區分成五代與十國。五代諸朝的大抵是華北地區與關中地區,一度領有燕雲十六州、河東(今山西省)、蜀地(四川省)與淮北地區。十國與其他藩鎮大多分佈在五代的週圍如華南、湖廣、蜀地、甘肅、河東與河北等地區,其中江南、湖廣地區分裂為六國,這顯示江南地區遠較三國時期更為開化,故以可以用小地域形成自立的地盤。五代十國的各國疆域在宋朝統一後仍然被沿用為路的行政區劃,現在則成為省界。而且,被細分化的疆域仍然不能自給自足,各國只能發展自身產業,並越界進行經濟交流以互通有無,最後促使宋朝商業的發達。

唐朝后期和五代時,政治核心因為戰亂與經濟因素,由長安、洛陽過渡到開封。當時關中因戰亂而荒廢,較強的藩鎮只有岐國李茂貞與定難軍,而河隴地區也持續衰退,回鶻、吐蕃等外族紛紛割據河西走廊;而華中、華南地區經濟強盛,所割據藩鎮繁多,是十國勢力範圍。而開封處於隋唐大運河中樞地位,負責轉運河北、關中、江南與湖廣地區的貨物,是天下糧食、貨物的轉運站。當關中因戰亂而荒廢時,聚集天下財富的開封就成為五代的首選地位,這也促使宋朝之後的中國朝代選擇以大運河城市如北京、開封、南京與杭州等為首都。另外,五代的戰爭大多以開封的宣武節度使與太原的河東節度使對峙為主,例如李克用的晉與後梁、後晉與後唐、後漢與佔據中原的遼朝、北漢與後周等都是如此。

五代十國的範圍與唐朝后期相比,萎縮明显,比中國本土的範圍略小,外族大舉占領中國本土的周圍,最後建立遼朝與西夏。河西地區被歸義軍、甘州回鶻與吐蕃諸部所占領。燕雲十六州在938年被後晉高祖石敬瑭割讓給契丹國(後為遼朝),使漢、唐以來北方的國防線全部後退,黃河北岸幾乎沒有屏障。再加上中國政治核心東移,使得五代、宋朝備受遼朝的壓力。而安南地區被靜海軍的曲家所割據,並在吳權於白藤江之戰擊敗後漢軍後,使越南地區正式脫離中國歷史。而在陝北夏州割據的定難軍,也在宋朝時獨立成西夏。

後晉、後漢時五代十國局勢圖。圖中燕雲十六州已經割讓給契丹國。不久契丹南下滅後晉,建國遼朝,而後後漢收復華北。南方的吳國被南唐取代,而南唐雖然攻滅閩國、楚國,但奪得的土地也被四周國家占據而獲益不大。

後周時五代十國局勢圖。圖中後周國勢最強,南奪南唐江北諸州,西占後蜀隴西之地,北奪遼朝瀛洲、莫州等地。十國除北漢外都臣服後周,而北漢是後漢皇室所建,連遼抗周。

五代十國的行政區劃繼承唐朝后期的形式,即道(節度使)、州(府)、縣三級行政區劃。五代注重對地方官的考課,令其忠於職守,後梁、後唐皇帝都詔諭吏部注意州縣官不得“姑徇私情,靡求才實”。

節度使成為地方行政區劃是由唐朝中期才開始設置的,又稱藩鎮,主管地方軍事、行政與財政,位高權重。安史之亂期間,唐廷成立許多地方節度使以圍堵叛軍。平亂後,唐廷也冊封大量降將為地方節度使以安撫,造就這些擁兵自重,割據為王的藩鎮,形成唐末藩鎮割據的局面。五代時節度使的授任更爲冗濫,有的節度使以親王遙領,如後唐末帝之子李重美遙領成德軍節度使,后汉高祖之弟劉勳領山南西道節度使;或以宰相遙領,如後唐莊宗時以侍中、監修國史郭崇韜兼領成德軍節度使。掌握兵權的節度使往往專横至極,爲所欲爲。其中,權重者稱節度使、權輕者稱防禦使(後稱觀察使),安史之亂後的道,即是節度使的轄區。當藩主有異心時,往往趁機舉兵以圖推翻中央,這個現象自唐朝后期开始出现,五代各朝或地方十國的內部也時常發生,這是五代十國動盪不安的起因。此外還在某些地方設“軍”,成爲一級行政機構。如907年後梁在輝州碭山縣置崇德軍,939年後晉改舊威州爲清遠軍,954年後周以萊蕪監爲廣利軍等,其軍使委命本道差補。到宋朝時,節度使被路行政區取代,並且分割地方的行政、財政與軍事權以防止擁兵自重的局勢出現。

第二行政區為州,州設刺史,第三行政區縣則設縣令。部分州因首都地位或地勢重要而升級為府、例如五代在汴州設有東京開封府、長安設有西京京兆府、魏州設有大名府,有些重要的府在宋朝形成五京制。而十國與各地藩鎮也在其首都或重要州設府,如吳國的揚州江都府、南唐的昇州金陵府與洪州南昌府、楚國的潭州長沙府、南漢的廣州興王府、北漢的太原府、前蜀與後蜀的成都府與興元府、荊南的江陵府等等。並於軍事要地設大都督府,如後梁在宋州、福州均設大都督府。後唐在全國設十大都督府。本時期南方的州縣數量,因為政局穩定、經濟發展與人口增加而增加。《太平寰宇記》所載五代十國時期全國新置五十九縣,絕大部分是在南方,如蜀置五縣,吳越設五縣,閩增設十三縣,南唐新置二十六縣。

五代十國的政治制度大體沿用唐朝制度,但是各朝變化很多,官職時常廢置不常,其制度比較混亂。朝廷設有主管行政的三省六部、主管財政的三司與主管軍事的樞密院。由於五代十國戰亂不斷,樞密院的權力往往比三省來得大,所以時常以宰相兼領樞密使。五代十國以“使”名官者很多,據《五代會要》記載有崇政院使、宣徽院使、飛龍使、翰林使、五坊使等等三十種之多。十國諸國中雖然有臣服於五代各朝,在制度上仍然是獨立的國家,政治架構等同五代。由於五代十國大多是從節度使起家,對支持他們的幕僚往往擔任新朝廷的職位,而前朝遺老則給予三師、三公或台省官等虛職。而將士有功時,為了攏絡他們,也以官爵名號爲賞賜。這些狀況成為後來宋朝冗官煩多的源頭。

中央行政機構有三省六部。三省為尚書省、門下省與中書省,下設六部尚書,並分司辦事。後梁重新設置唐朝空置的尚書令,並且定爲正一品,改唐朝的尚書左右丞爲左右司侍郎。後唐時恢複唐朝舊制,並多設左右僕射,與尚書左右丞均爲正四品。後梁又設中書門下省,置“中書門下平章事”,改司政殿爲金鑾殿,設大學士一員,以崇政院使敬翔爲金鑾殿大學士。中書省和門下省方面,其官員品級也比唐朝高,其長官侍中在唐代宗以前均爲正三品,後晉時中書令和侍中均爲正二品,左右常侍從三品升爲正三品,門下侍郎從正四品升爲正三品。十國方面,有設有等同宰相的官職,如吳國的大丞相,楚、吳的左右丞相,吳、南漢的參知政事,吳越的參相府事等都等同宰相的職稱。

三司使專管財務,至五代時才確定。早在唐朝時就有户部、度支、鹽鐵等三司分管租稅、財務收支和鹽鐵專賣、物資轉運事務。唐昭宗以宰相崔胤兼領三司使,至此出現三司使。後唐曾設置租庸使以管轄三司,最後正式設置三司使和副使以管理朝廷財務,地方財政也需聽從三司使的命令。以後歷朝相承不廢,宋朝設置的三司就是緣自五代的。

五代十國還設有樞密院以掌管軍事,又大多為武將。樞密使掌握軍事,其實權往往超過宰相,可直接下令任免藩鎮。所以通常由皇帝最親信的臣僚充當,有時又以宰臣兼任樞密使。例如959年後周世宗命司徒平章事范質與禮部尚書平章事王溥參知樞密院事,藉此以加強文人官僚制度。早在唐代宗時就以宦官掌樞密,所統領的左右神策軍護軍中尉與兩樞密使共稱「四貴」。此後宦官往往侵奪相權,甚至廢立皇帝。唐朝后期,朱溫大殺宦官,至此開始用朝臣充任樞密使。朱晃建立後梁後,改樞密院爲崇政院,改樞密使爲崇政使。923年後唐莊宗又復稱樞密院,並設樞密使與副使。後晉曾以宣徽使代之,但不久又恢複。中書和樞密對掌文武二柄的方式,最後由宋朝所繼承之。而十國等各國或地方藩鎮也大抵置有樞密使或相當於樞密使的官職。

五代十國時的刑法,基本沿用唐朝的律令格式和編敕,但因歷朝又都有新頒的敕條,彙編附益,使得格敕前後重複矛盾。957年後周世宗令大臣們進行整理,唐律條文難解的,加上注釋,格敕繁雜的,加以刪除,彙編為《大周刑統》二十一卷。北宋初年所編的《宋刑統》即就此書略加增刪而成。

早在唐朝時因為外族紛紛進入中原内附定居,在安史之亂後河北地區、陝北與河西走廊陸續成為外族的勢力範圍,使得中原政局更容易受外族的影響。例如沙陀族、党項族受唐朝冊封為節度使,而沙陀領有的河東軍於五代建立後唐、後晉與後漢。而契丹族的影響最大,多次成為篡立者的外援。建立契丹國後於946年入主中原,建國遼朝。雖然遼朝最後返回燕雲地區,但仍然對中原地區有一定的影響力。

五代時,北方以契丹最強。契丹族原唐朝受封為松漠都督府。唐朝后期,契丹迭剌部的首領耶律阿保機崛起並征服各部,取代痕德堇可汗後於907年即可汗位。他先後鎮壓了契丹貴族的叛亂,征服漠北地區奚、室韋、黠嘎斯、阻卜等部落,在軍事與經濟方面都十分強盛。915年耶律阿保機出征室韋得勝回國,但被迫交出汗位。不久他在灤河邊建城,於隔年建立契丹國,即遼太祖。契丹國掠奪中原的人口,收留因河北戰爭而逃亡的流民,任用韓延徽、韓知古、康默記與盧文進等漢人為佐命功臣。925年東征渤海國後即有意南下中原。遼太宗繼位後趁後唐發生內亂之際,接受石敬瑭的請求,出兵協助攻滅後唐,扶持兒皇帝建國後晉,並且獲得燕雲十六州。石重貴繼位後不願稱臣,並濫殺契丹商人。遼太宗為此多次出兵南征未果,後來有賴後晉大將杜重威投降,而攻陷汴州,建遼朝。而後因為遼軍打草穀濫殺漢人,使中原人人舉兵抗遼,遼太宗也在北返之際於殺胡林去世,遼朝政局動盪不安。雖然遼廷扶持北漢,但最後於959年被後周世宗奪下瀛洲、莫州等地。宋朝建立並滅北漢統一中國本土後,於同年北伐遼朝。此時有賴耶律沙、耶律休哥、耶律斜軫等名將與宋軍大戰高梁河(今北京西直門外),成功擊敗宋軍,此時也進入宋遼對峙時期。

五代時北方與東北還有奚、吐谷浑、室韋、渤海國等國。奚國於唐朝受封為饒樂都督府,於唐朝中後期多次入侵邊疆。遼朝建立後,奚國被契丹征服,契丹還建立遼中京以統治之,並且逐步同化奚族。而唐朝后期的另一強國渤海國,也在926年亡於契丹,並於原地成立東丹國,以契丹太子耶律倍任人皇王。遼太宗繼位後廢除東丹國,建立遼東京以管理之。而吐谷浑部,原本定居青海一帶,被吐蕃攻滅後,東遷到朔方、河东一帶。五代时散处蔚州等地。936年後晉割讓燕云十六州給契丹國,使得部分吐谷浑臣服于契丹,但仍有不少逃回太原,投奔河東節度使劉知遠。

唐朝后期,西方吐蕃最強,但因內部分裂而衰。五代時,河西走廊被回鶻、吐蕃與党項等許多民族所割據,有甘州回鶻、吐蕃六谷部、黃頭回鶻、位於蘭州一帶的党項族、陝北的定難軍(党項統領)等。此時漢人政權只有沙洲、瓜州的歸義節度使與五代屬地的朔方節度使與河西節度使(管制涼州、蘭州等)。而定難節度使是西夏的前身,本體是陝北夏州的党項族。其領袖拓跋思恭因平亂有功,被唐僖宗所冊封。雖然定難軍獨立自主,對外仍然臣服五代各朝與北漢。五代時,後唐明宗意圖併吞定難軍,將庭州軍安從進與定難軍李彝超對調,最後有賴李彝超成功擊退安重進的唐軍才穩固之。最後在宋朝時吞併靈州、河西等地,建國西夏。西域地區則有西州回鶻、高昌回鶻、龜茲回鶻、于闐與喀喇汗國等。其中于阗和喀喇汗國是西域大國,其範圍涵蓋整個西域地區。拥有塔里木南部广大领土的于阗是盛行佛教的塞种人国度,统治者尉迟家族自汉朝起就掌握该国的政权。唐朝时,于阗属于毗沙都督府,由当时的国王尉迟伏阇雄兼任都督。五代时的于阗趋于汉化,国王李圣天自称“唐之宗属”,国内实行唐朝旧俗,并派人向中原朝廷进贡,后晋封其为大宝于阗国王。喀喇汗國则为伊斯兰势力东进的主要势力,同于阗多次发生战争,但皆以失败告终。十一世纪初,于阗与归义军在喀喇汗國和西夏夹攻中相继灭亡。1212年时,喀剌汗國亡於由耶律大石率領的契丹族,後者建國西遼。

南方有交趾、大理國與牂柯蠻。曲承裕擔任靜海節度使後,曲家長期割據交趾,907年去世後由曲顥繼位。930年曲主曲承美被南漢高祖攻滅,不久楊廷藝舉兵並攻下大羅城,靜海軍再度建立。937年矯公羨奪位並向南漢稱臣,隔年楊廷藝的女婿吳權自愛州舉兵,南漢高祖派其子劉弘操率軍入援。吳權在殺死矯公羨後於白藤江之戰擊潰劉弘操率領的漢軍,至此南漢再也沒有南征交趾的意願。939年吳權稱王,建立吳朝,都城古螺,越南地區开始脫離中國歷史,至968年丁部領統一十二使君後建立丁朝,越南正式走上独立发展道路。大理國源自唐朝時的強國南詔,由於長期與唐朝戰爭,國力日趨衰弱,於南詔末年多次發生權臣篡位事件。902年,世襲清平官的權臣鄭買嗣迫南詔帝蒙舜化貞退位,建國大長和。928年大長和的東川節度使楊干貞和清平官趙善政殺死大長和帝鄭隆亶,趙善政建國大天興。次年,楊干貞廢趙善政自立,建國大義寧,而楊干貞之弟楊詔認為海通節度使段思平有異心,促使楊干貞派兵追殺。段思平即向高方尋求庇護。而後,段思平向東方的黑爨借兵,與其弟段思良和軍師董迦羅舉兵反抗。937年滅大義寧,建立大理國,都城大理。

五代十國的軍事制度繼承唐朝后期節度使的制度,當時地方藩鎮時常舉兵意圖推翻中央朝廷。為了解決此問題,朝廷逐漸加強中央禁軍以打壓地方,到宋朝更發展成強幹弱枝的政策。而五代十國最高的中央單位是樞密院,大多為武將擔任。由於五代重軍事、輕文人,為了鞏固政權也以宰相兼任樞密使。

唐朝中期開始,節度使擁有強大兵力,掌握地方軍事、民政、财政。他們位高權重、專横至極,時常發生舉兵意圖推翻朝廷之事,史稱藩鎮割據。而唐朝最後也被宣武節度使朱溫所篡,五代十國的建國者也大多是唐朝后期的節度使。到五代十國時,舉兵篡位之事更多,在後晉高祖割讓燕雲十六州後更加劇烈。也使得五代君王時常替換,最後形成九姓十五君之多的亂世。為此君主紛紛採行建立禁軍、調動地方節度使等強幹弱枝的政策,以削弱地方實力派。禁軍負責守衛首都與皇宮,有時會駐防各地以壓制地方藩鎮,例如後梁、後唐就以禁軍壓制、削弱河北三鎮。後來宋朝宋太祖更以禁軍「殿前諸班」統一天下。此外,朝廷還頻繁調動節度使,更換其駐地,以防止他們長期佔據一方,形成割據勢力。

而五代各朝時常擴充禁軍,軍事官制也繁多易變。後梁太祖最親近的軍隊是「廳子都」,此軍裝備精良,兇悍異常,太原晉軍十分畏懼。立國後擴編宣武軍為禁軍,取禁軍的精銳以成立侍衛親軍。在首都設左右龍虎軍、左右羽林軍、左右神武軍、左右龍驤軍,均以親王爲軍使,後來名稱時有變動。後唐禁軍的前身是河東軍,李克用以眾多養子為骨幹建立「義兒軍」,是其最精銳的軍隊。征戰的主力部隊是915年收編的魏博銀槍效節軍,在滅梁時發揮重大作用。建國後在首都設立嚴衛左右軍、捧聖左右軍等。在後唐明宗時成立侍衛親軍為禁軍,以其鄴都起事的兵士為骨幹,又稱隨駕軍。其中石敬塘還擔任侍衛親軍馬步軍都指揮使兼六軍都衛副使。後晉又在首都設護聖左右軍,其本部軍源自石敬塘在河東起事的軍隊,其部屬劉知遠還擔任侍衛馬步軍都指揮使。後漢軍制沿襲後晉,沒有很大的改變。後周在首都設龍捷左右軍、虎捷左右軍。後周世宗時改革軍事制度,實施練選制度,精簡中央禁軍,補充強健之士,設有殿前都指揮使、水陸都部署、殿前都點檢等高級軍官,形成「殿前諸班」的禁軍。其中殿前都點檢掌握軍事實權,後來擔任此職的趙匡胤在後周世宗去世後發動陳橋兵變,篡位建國宋朝。其次是嚴明軍紀,命兵部尚書張昭遠制定新的軍法。最後是限制藩鎮權力,例如禁止造軍器、干預民政等等。

由於戰爭頻繁,兵役負擔沉重。當時為了防止士兵逃亡,特在士兵臉上刺字記其軍號,以便各地關津識認、追捕逃兵。另外,各地都徵派男女從事運輸,無數人畜累斃途中。後梁太祖攻打青州王師範時,甚至把徵發來堆積攻城土山的民丁、牛驢一起掩埋在土山中。劉仁恭在幽燕徵發十五歲以上、七十歲以下的男子自備軍糧從軍,共得二十萬人。北漢規定十七歲以上的男子皆入兵籍為兵。南唐曾強令老弱以外的人全部從軍。吳越錢俶「盡括國中丁民」為兵。湖南馬希萼調發朗州全部丁壯為鄉兵。閩國後期發民為兵,力役無節。除了兵役,還有各種名目的土木修建勞役。後唐莊宗盛暑修建營樓,「日役萬人」。荊南修理江陵外郭,驅兵民萬餘人從役。閩主建築寺觀宮殿,「百役繁興」。賦役嚴重,使戰亂破壞嚴重的北方社會經濟難以復甦,也大大阻礙了南方經濟發展的進程。

唐朝后期至五代十國時期,中原地區的經濟因為長期的戰亂、天災而殘破不堪。黃巢起義後,長達六七十年內,大小戰事不停。華北地區的兵役和各種勞役異常繁重。統治者視人命如草芥,無辜群眾常遭慘殺。戰爭破壞和苛重賦役促使人民數以萬計餓死或流徙他處。例如唐朝后期蔡州秦宗權四處肆虐,一度攻陷東都(今河南洛陽),形成「極目千里,無復煙火」的情況。朱溫與徐州時溥的戰爭,破壞徐、泗、濠三州的農業。朱溫與河北劉仁恭的戰爭又破壞魏州至滄州之地,於定州之戰更死傷六萬多人。唐朝的精華區長安、洛陽一帶也被朱溫東遷唐昭宗期間,強制遷移人民,並且拆毀房屋,焚燒一切,在籍的民戶還不滿一百戶。而後後梁與晉的戰事,使晉南豫北不少地方「里無麥禾,邑無煙火」。為了抵擋晉軍,後梁數度決開黃河,使得河南、山東一帶洪水氾濫不堪。到後唐、後晉期間,華北地區被受契丹國襲擾,盧龍、燕州之地屢次被契丹騎兵焚掠一空,千里內「民物殆盡」。尤其在契丹軍南下攻陷汴州後,開封至洛陽數百裡間人煙稀少,相州百姓有十餘萬人被殺死。而後河中與鳳翔等鎮在後漢時發動叛亂,戰死餓死的屍體有二十萬具以上。北漢的十二州,盛唐時有二十八萬戶,而在北漢亡國時僅有三萬餘戶,約為盛唐時戶口的八分之一。839年唐文宗時期,戶口有4,990,000戶,到宋朝再度統一時,全國戶口只剩3,790,000戶,在這一百四十年間減少達一百二十萬戶,可以想見唐末五代戰亂的劇烈和民生的痛苦。

雖然唐朝后期南方也受到龐勛之變與黃巢之亂的影响,但在十國時期,重大戰事較少,政局比較穩定,有利於社會經濟的恢復和發展。再加上唐朝中衰以后,中原动荡不安,不少人紛紛南下江南、湖廣與巴蜀一帶,最遠達兩廣之地,關內道、河南道、河北道都減少很多。而南方如蘇州、鄂州、洪州、饒州、吉州、襄州、郢州、唐州、衡州、廣州等都大幅提升。據說當時的蘇州戶口中,自北方遷來的占原來人數的三分之一,武昌在兩年內戶口增加了三倍,這都反映了南徙人口的眾多,使得人口分布則以南方地區比較密集。長期安定的環境有利於發展生產,使得十國府庫逐漸充實。五代十國末期,後周與北漢的戶口不過一百萬戶,南方諸國則多達兩百七十餘萬戶。在這些國家中,以南唐的六十五萬戶最多,其次是吳越的五十五萬戶居次;再次為後蜀的五十三萬餘戶。這三國的人口總合,差不多是當時中國地區總人口的一半。北宋統一南北時,原後周和北漢所在的華北地區約一百萬戶,而南方九國所在地區已有二百三十萬戶。南方人口超過北方的態勢至此已經定型,北宋年初的南北人口比例大約是6:4。

唐朝後期因為安史之亂、藩鎮割據與黃巢之亂的因素,使得北方戰亂不堪,人口流移南方,田園荒蕪。到五代十國時期,五代時交迭頻繁,北方戰火始終未能平熄,北方經濟比較落後,人口持續大減。直至後周後期才逐漸恢復,但經濟力始終不如南方。而南方則較為安定,持續吸收來自北方的流民,替南方帶來大批的勞動力及先進的耕織技術,加速了南方經濟的發展。到五代十國期間,由於南方十國國家林立,擺脫北方經濟負擔,而且君王重視生產發展,發展出若干個以大城市為中心的經濟區域。蜀地是農業、工商業發達地區,倉庫飽滿。江南兩淮重農桑、茶葉、水利與商業貿易,其中吳越、閩國與南漢的貿易最為興盛。湖廣要靠賣茶和通商,運茶到黃河一帶,交換衣料和戰馬以獲利。這些區域彼此互通有無,並與華北、外國通商貿易,商業十分興盛。所以,南方至此已完全代替北方成為全中國地區的經濟中心。

唐朝后期至五代十國時期,中原地區的經濟因為長期的戰亂、天災而殘破不堪,河北、河南、山東與關中一帶都是戰亂區。例如943年後晉出帝時,春夏裏有早災,秋冬有水災,蝗蟲大起,境內竹木葉都被蝗蟲吃光;再加上軍事上人為的決黃河水,水浸汴、鄂各州,使北方的生産遭到極大破壞。相對的,自漢魏六朝以來,比較平穩的江南、湖廣與巴蜀地區的經濟持續發展而十分興盛,成為中原人民投奔的地方。在加上華南地區被細劃分數國,各國為了提升經濟力莫不細心經營,這使得十國的經濟力遠勝於重武力的五代。

雖然五代戰亂不堪,但仍有不少君王提振經濟。後梁太祖稱帝後重視農業,他任張全義為河南尹,以恢復河南地區的生產。908年又令諸州滅蝗以利農桑。後唐明宗執政期間,提倡節儉,興修水利,關心百姓疾苦,使百姓得以喘息。到後周時,後周太祖郭威為了減輕農民壓力,於952年直接將兵屯的營田賜給佃戶,以提升稅收;並且廢除後梁太祖朱溫實行的「牛租」,使農民免除牛死租存的負擔。到後周世祖時,建立均田制,按實際占有田畝徵稅。這不同於隋唐前期的舊制,而是同兩稅法之後普遍實行稅產是一致的。

南方十國提倡經濟發展,並且重视兴修水利,防水治害。例如吳越、南唐獎勵農桑;閩及南漢促進海外貿易;前蜀和後蜀亦能發展農耕絲織,此均能令南方的經濟得到發展。巴蜀地區在唐朝就十分富庶,有天府之國之譽稱。經歷戰亂後,在前蜀王建與後蜀孟知祥、孟昶父子的經營下,政治相對穩定。他們又注重興修水利,廣泛耕墾,在褒中一帶還興辦了屯田,使得農業生產比較發達。後蜀時期,「百姓富庶」、「斗米三錢」,米便宜到一斗三文錢。而兩廣地區也讓不少人遷居,五十年來,南嶺以南無事,使得南漢府庫逐漸充實。

吳與南唐、吳越所在的兩淮、江南與太湖地區在隋唐時期十分繁榮,是唐朝的粮食重鎮。歷經龐勛之變與黃巢之亂後也逐漸回復,當地朝廷支持大規模開墾荒地,並且修築水道。吴和南唐在丹阳疏浚练湖,在句容疏浚绛岩湖,在楚州筑白水塘,在寿州筑安丰塘,少者溉田数千顷,多者溉田万顷以上。吳越王錢鏐在錢塘江修築錢塘江石塘以防海潮侵襲,並且疏浚西湖、太湖和鑒湖等,又募民開墾荒田,免徵田稅,使杭州一帶成為江南富裕之城。而福建地區在唐朝后期經濟力不強,王潮、王審知兄弟領有閩國後,他們勸民農桑,在连江县车湖周围筑堤,可溉田四万余顷。南唐和吴越的農民还修建了一种圩田,即围田。旱则开闸引水灌田,涝则关闸拒水,把低洼的涝地变成良好的耕田。而湖廣之地,在東晉南朝以來也十分興盛。馬殷據湖南建國楚國後,不斷提升湘中、湘西的糧食產量。在周行逢據有湖南時,人民「率務稼穡,四五年間,倉廩充實」。這些都使得長江中下游一帶成為「賦出於天下,江南居十九」的餘糧區,到宋朝更有「蘇常(或湖)熟、天下足」的說法。

南方除了糧食作物興盛之外,茶葉、絲綢與棉花等經濟作物也十分興盛,且進入專業化的地步。當時茶葉除了種於山區之外,也有建立於平地丘陵制之上。根據《四時纂要》記載當時江南茶園十分發達。楚國发展茶葉、植桑養蠶與棉花十分興盛,閩國發展經濟產物茶葉,又獎勵海上貿易,大舉提升當地的經濟。

諸國混戰雖然嚴重破壞了社會經濟,但社會生產仍未中斷。即使在華北地區,後梁太祖建國初期和後唐明宗在位時,都曾分別採取某些恢復生產的措施。後周時,手工業如紡織、造紙、制茶、曬煮鹽等生產也有所發展。而南方十國的紡織業更是凌駕在北方之上。

雕版印刷最初是在民間流行,在五代十國時期尤為突出,其中以江南和巴蜀两地比較发达,不仅有民间书肆出售的佛经和日用各书,而且士大夫阶级所读的儒家经典也用雕版印刷發行。雕版印刷較發達的前蜀,印刷品主要是占卜書、字書等。到後蜀時專門印製,導致「蜀中文學復盛」。932年後唐宰相馮道提議官方採行雕版印製《詩經》、《書經》、《禮記》等等九經,出現官方大規模印刷。這個計畫由國子監實現,沒有因為戰亂而中止,直到953年後周時期才刻印完畢,共二十二年。從此,刻本「九經」廣為流傳。此後朝廷刊印經書的數量增加,這個任務也交給國子監負責,書版也收藏在國子監內,被稱為「監本」。

五代亦為陶藝的重要蛻變期,也是由民間走向官方製窯。民窯與官窯分道揚鑣,爭奇鬥艷,成為一色釉瓷器盛行的時期。官方創設官窯,專門供應皇室和官員所用。在北方有後唐、後周的御窯,在南方有吳越國的秘色窯,西南方有前蜀、後蜀的官窯。而民間也保持優良的傳統,例如位於河北的定窯即十分興盛。而吳越國的越器,其燒製技術優良,十分有名。五代的陶瓷匠師更是創造出「雨過天青」的傳世之寶,成為中國古代陶瓷發展史上的一大創舉。製窯技術也遠傳國外,918年後梁時期時,高麗便學會中國的造瓷技術,並在康津設立了窯廠,此後又陸續傳到了日本及西方各國。

澄心堂紙是五代時期的名紙之一。五代南唐後主擅寫詩詞,喜歡收藏書籍和紙張,為此將金陵官府的一幢房子命名為澄心堂,作為作詩藏書之地。南唐後主還特地令四川造紙工匠來到澄心堂,仿照蜀紙製成一種質地優良的新紙,並命名為「澄心堂紙」。因為澄心堂紙的質量非常好,以至一紙值百金,是紙品中的佼佼者。此後宋朝、清朝也都學習南唐的技術,生產並使用了這種紙。

商業方面,由於北方五代戰亂不斷,農業遭受破壞,連帶商業也難以發展。而華南經濟未遭受很大破壞,南方十國的政局相對於北方也比較穩定,除了盛產糧食之外,部分國家還大量出產茶葉、絲綢與棉花等經濟作物。由於國家林立,長江水運與海上運輸都很便利。各國紛紛互通有無,有的還與外國貿易,使得商業貿易十分發達。華北需要大量的茶葉,而楚國、南唐與閩國等南方茶商也需要運送至河南、河北等地,使得荊南成為茶葉轉運中心,商人販賣茶葉,買回繒纊、戰馬。而江南所需的食鹽一部分也依賴華北供應。北方五代與北漢、燕、岐等勢力從契丹、回鶻、党項買馬,前後蜀向西邊各少數族買馬。南方諸臣服國都以進貢方式和北方進行貿易。南唐、吳越、閩國與北方的貿易主要是通過海路,東自高麗、日本,西至大食,南及占城、三佛齊等都有商業往來。當時有不少貿易大港如揚州、明州、廣州等等,其中杭州、福州與泉州都是這個時期擴建成長。例如吳越錢鏐擴建杭州城、閩國王審知擴建福州城、閩南留從效擴建泉州城等。吳越、吳國和南唐從海外輸入「猛火油」使用,還從海道再輸往遼朝。但從全國範圍來說,由於政治不統一,交通阻隔,經濟很少進步,所以商業的發展也受到了限制。如前蜀法令規定:「不許奇貨東出」。後周規定販運食鹽不得逾越漳河。但是,通商貿易仍然十分興盛。

五代於中国文化史上有重要地位,主要表现为印刷事业的发展、火药在战场上的出现、词的兴起等几个方面。由於南方較北方富庶安定,因此,文學、繪畫、金屬工藝、浮雕、紡織、陶藝等均盛行於南方。

五代儒家學說雖然還是國政的基本依據,但是對社會、政治的影響力已經大大降低。因為儒學對於官方及正常秩序的依賴,要比佛、道二家大得多。當五代政權屢變,儒學備遭破壞,其思想影響大大降低。後唐明宗在敕旨中指出了學校多廢、典籍罕傳的狀況。後周世宗時,更做了一些恢復儒學的努力,使儒學的傳統不至於中斷。而民間私人講學的風氣也很盛行,培育了不少的儒學人才,這都成為宋代儒學興發的重要養分。

由於社會動蕩和時代短促的緣故,時常發生叛變弒君事,而君王大多重武輕文,士人也重實輕虛,使得本時期比較少有傑出的學者和思想家,有名的儒者只有馮道。馮道為五代政治家,大規模官刻儒學《九經》,侍奉九姓十五君,「累朝不離將相、三公、三師之位」,前後為官四十多年。其行為事君不忠,但是事親濟民、其主政提攜賢良,在五代有著「當世之士無賢愚,皆仰道為元老,而喜為之偁譽」的聲望,晚年著有《長樂老自敘》。

由於亂世災禍,人們對前途深感無從把握,大多採行消極避世的思想,部分儒者與百姓轉向佛教與道教。有研讀道籍者,也有隱遁山林者,名利之心淡漠,注重個人養生,而有「五季之亂,避世宜多」。其中道教學者譚峭繼承老子「道」為世界本源的說法。他認為天地間萬物均是由道演化而來,而道的本質則是虛,許多觀點蘊含著人民性、民主性因素。譚峭著有《化書》,其中《道化篇》云:「道之委也,虛化神,神化氣,氣化形,形生而萬物所以塞也。道之用也,形化氣,氣化神,神化虛,虛明而萬物所以通也」。

唐朝后期的皮日休繼承王充以來的「氣」一元論,將氣看作天地萬物的本原。他不認同相命術數等迷信,也不認為有「天」,主張無神論。並且具有民本思想,重視人民生活,批評唐廷政治腐敗無能。

五代的文人饱经沧桑,诗文也透露着沉痛的气息。其中以吳越國诗人罗隐的五七言诗比较优秀,著有《羅隱甲乙集》,收其詩作,今已不傳。五代前期时期流亡四方的文人学士颇多,司空图、韦庄、杜光庭等,都是非常有文学成就的人物。

然而五代十國的文學是詞的重要發展時期。其詞風的前期繼承晚唐風格,主要描寫皇室貴族的享樂生活。其題材庸俗,境界狹窄,風格柔靡,以花間派的作品為代表。到後期出現清晰深沉的描述,情感生動,使人回味無窮,對宋詞的影響極大。花間派起源於晚唐溫庭筠、晚唐前蜀的韋莊,其中溫庭筠被後人稱為「花間鼻祖」,有名的有〈菩薩蠻〉、〈夢江南〉等,而韋莊有〈女冠子〉、〈菩薩蠻〉等,其風格較為清新。而後繁榮於五代,以蜀地和南唐詞人較多,水平也較高,從而成為兩個中心。蜀地有晚唐前蜀的韋莊與後蜀的歐陽炯等人,他們的作品後來由趙崇祚收入《花間集》。歐陽炯詞作風極委婉之致,有名的有〈南鄉子〉。

另一個中心的南唐有馮延巳、中主李璟、後主李煜等人。馮延巳的作品有〈採桑子〉、〈謁金門〉等,詞風細膩深沉,影響北宋詞人晏殊、歐陽修等,遺有《陽春集》。李璟的作品以〈攤破浣溪紗〉最具代表,內容深動,沒有艷麗虛浮感,李璟父子的作品被後人集刻為《南唐二主詞》。李煜是五代十國中最重要的詞人。其前期的作品也是如同花間派,以〈玉樓春〉、〈菩薩蠻〉等宮廷艷麗生活為主。但在國亡被俘後所寫的詞,或慨嘆身世,或懷戀往昔,形像鮮明,語言生動,把傷感之情寫得很深摯,以〈虞美人〉、〈浪淘沙〉、〈烏夜啼〉等最具代表。突破了晚唐以來專寫風花雪月、男女之情的窠臼。在內容和意境兩方面都有創新,為北宋詞的發展開拓了新的領域。

史學早在唐朝就十分興盛,到五代十國時仍然盛行,並且重視唐朝史料的部分。其中,以《舊唐書》和《唐會要》最有名,對於記述唐代史事、人物、典制、興亡盛衰等都具有特別重要的價值。《舊唐書》即《唐書》。為了累積豐厚的資料,早在後梁末帝即下詔徵集唐代家傳以及公私章疏,到後唐明宗又設三川搜訪圖籍使到成都一帶搜尋唐實錄,並明令保護碑碣。最後後晉高祖在914年後晉天福六年下令撰寫,到945年開運二年完成。本書先後由張昭遠、賈緯、趙熙等等人撰寫,監修一開始是趙熙,最後是劉昫。《舊唐書》保存豐富的人物、事件等原始史料,例如李密的〈討隋煬帝檄文〉,受到後世史學家的重視。但是成書倉促,對原始材料缺乏加工,唐憲宗以前多照抄國史、實錄,而唐穆宗以後系編纂雜說、傳記,所以到宋朝又出現《新唐書》。

五代另一個鉅作《唐會要》是由後周的王溥所著,其分門別類的敘述唐代各項典章制度與文物的沿革变迁,再現唐代風貌,是中國历史上第一部《会要》专著。五代的歷史筆記也十分興盛,主要也是敘述唐代事物。王仁裕的《開元天寶遺事》記載唐玄宗時的朝野逸事、王定保的《唐摭言》詳述唐代貢舉制度、劉崇遠撰《金華子》記敘唐末朝野故事、孫光憲的《北夢瑣言》記載唐及五代士人逸事等。這些筆記內容真切,在時代特點與社會風貌方面的敘述也比史書多元。

唐朝后期、五代時期政局混亂、戰亂不堪,使得儒學衰退、許多士人百姓紛紛尋求宗教上的撫慰。宗教方面依舊延續唐朝中期以来的政策,趨向崇道貶佛,但是佛教在南方逐漸生根发展。五代各朝推行限制賞賜名僧和度僧人數的政策以限制佛教。但是南方十國崇尚佛教,並沒有強制限制,而南方又以中國化的禪宗為主。道教在五代十國時期所受限制較少,許多五代皇帝推崇道教,使得道教比較興盛。然而佛教在民間的影響力仍然勝過道教。

武宗灭佛後,佛教各宗陸續衰退,只有禅宗南宗逐漸興盛,並且在唐朝后期开始分為五宗。禪宗在南頓北漸後分成神秀的北宗與惠能的南宗。惠能主張頓悟、見性成佛,在南遷嶺南後廣為流傳。其弟子神會北返洛陽,並在明定南北總是非大會上擊敗北宗,使禪宗南宗成為禪宗主流。禪宗南宗而後分成神會的荷澤宗、行思的青原宗與懷讓的南岳宗。南岳宗至百丈懷海時,其弟子靈佑、慧濟創建溈仰宗,在五代時期十分興盛,至北宋四代而亡。百丈懷海的弟子希連、義玄創建臨濟宗,到北宋成為禪宗最興盛的派系。文偃創立雲門宗,其思想可概括為三句,即:「函蓋乾坤句,截斷眾流句,隨波逐浪句」。文益開創法眼宗,認為「三界唯心,萬物唯識」,所以主張「不著他求,盡由心造」。青原宗至曹山本寂時創建曹洞宗,在唐末五代時期形成五宗七派,即溈仰宗、臨濟宗、雲門宗、法眼宗、曹洞宗等五宗,再加上臨濟門下分出的黃龍、楊岐兩派等。然而禪宗到後期過度推廣頓悟,反而流於形式與神祕主義,甚至出現「呵佛罵祖」之事。使得佛教走向世俗化、制度解體化。

佛教其他宗在武宗灭佛後大半衰亡,天台宗、唯識宗的典籍亡失。五代十國時,十國吳國皇帝邀請高麗諦觀應攜天台典籍,諦觀著《四教儀》,使得天台宗因而復興。而淨土宗滲入民間,並且向上流傳至士人,到後期與禪宗融合,一度有「禪淨一致」的思潮。佛教在後周時又發生大規模排佛運動,後周世宗以寺僧浮濫,直接影響國家賦稅、兵役為由整飭寺院,沙汰僧尼,與前次滅佛運動合稱三武一宗。至此中國北方的佛教日益衰落,而南方佛教仍繼續發展。

唐末五代的道教仍然十分興盛,逐漸以外丹道走向內丹道。五代十國時有不少崇道的君王,例如後周世宗就是抑佛揚道。而道教則在五代多朝皇帝的扶持下,以此較盛行的勢態延續至宋朝,為道教在宋朝的鼎盛奠定基礎。五代有名的道士有杜光庭、譚峭、彭曉、谭紫霞與劉海等。杜光庭主張以道為本,納儒、佛入道,著有《道德真經廣聖義》、《常清靜經住》等等。他主張修道之人都要「因元氣而成」,其方法是「安神去欲、保守三元」。他繼承唐玄宗時道士吳筠的作法,認為三教應該融合無別。他說:「凡學仙之士,若悟真理,則不以西竺東土,為名分別,六合之內,天上地下,道化一也。若悟解之者,亦不以至道為尊,亦不以象教為異,亦不以儒宗為別也。三教聖人,所說各異,其理一也」。杜光庭的清淨之道,可以說是道教融合佛儒的典型代表。此外,他還將茅山宗和天師道兩派的齋醮儀式統一起來,並加以規範與制度化,廣受後世道教所採用。

譚峭自幼愛好黃老、諸子及列仙傳記,立志修道學仙。擅長辟穀養氣之術,著有《化書》(《譚子化書》),認為萬事萬物皆源於虛,「虛化神,神化氣,氣化形」,後復歸於虛,「其化無窮」。彭曉著有《周易參同契分章通真義》,都比較有影響力。

摩尼教、景教與祅教等宗教也因武宗灭佛而大半衰退,其中摩尼教走入地下化,並在五代復受五代朝廷禮遇。摩尼教化為民間秘密宗教後,成為農民起事的凝聚力量來源。例如920年後梁時期的陳州毋乙、董乙等人就利用摩尼教起事。而伊斯兰教沒有在武宗灭佛被取締,其主要以僑寓中國沿海的阿拉伯、波斯商人的後裔為主,這些人大都沿習父輩的信仰。在西域一帶的民族也因為伊斯蘭教的東傳而逐漸放棄原本的摩尼教、景教、祅教、萨满教與佛教,成為新一代的穆斯林。

五代十國的繪畫主要繼承唐朝繪畫,並且有所創新。南唐、前蜀後蜀與吳越等國經濟強盛,皇室和士人生活優裕,產生宮廷畫院,使繪畫藝術走向觀賞性、集中性,其中還產生許多以家族為單位的創作群體。935年後蜀創設翰林圖畫院,這是中國有正式的宮廷畫院之始,而後南唐也設立圖畫院。圖畫院聚集了一批著名畫家,互相討論研究,造就一批頗有成就的畫家。他們在人物畫、山水畫、花鳥畫都有一定的發展,特別山水畫和花鳥畫對宋代的畫風影響很大。

五代十國時期因為中原戰亂不堪,許多畫家由中原轉向西南和東南,並隱居深山,造成山水畫的迅速發展以及花鳥畫的興起。水墨山水畫在五代進入成熟階段,畫家體味生活,將所見自然環境的特色,用不同技法加以再現,形成北方荊關與南方董巨兩派。北方山水畫以後梁的荊浩、關仝師徒最有名,荊浩陶情林泉,寄趣丹青,人稱「洪谷子」。他擅長畫崇山峻岭,其所繪的《匡廬圖》有「全景山水」之稱。關仝擅長畫關河之勢,雄渾之中平添北方山水蕭索蒼涼之氣,繪有《關山行旅圖》、《秋山晚翠圖》等。南方山水畫以南唐的董源、巨然師徒著稱,他們皆擅用水墨描繪江南景色。董源擅長用披麻皴,好以淡墨輕嵐寫出江南平淡天真之趣,以《洞天山堂》、《寒林重汀圖》最有名。巨然直接承襲董源的畫法,更在山頂上常鉤畫一些明淨的卵石,即「礬頭」。以《蕭翼賺蘭亭圖》、《層巖叢樹圖》等著稱。而花鳥畫以南唐的徐熙與後蜀的黃筌等人最有名。黃筌擅畫宮廷的珍禽異卉,徐熙擅畫江湖上的水鳥汀花,兩人並稱為「黃、徐」,當時有「黃家富貴,徐熙野逸」的諺語,有名的有《寫生珍禽圖》、《雪竹圖》等。人物畫皆繼承唐朝周昉和張萱的宮廷人物畫風,有名的有顧閎中、周文矩與石恪。顧閎中所畫《韓熙載夜宴圖》線條細膩,色彩華麗鮮豔,為傳世的藝術珍品。周文矩對人物的刻畫表情生動,對形體與姿勢掌握的深厚功力,繪有《蘇李別意》與《按樂宮女圖》等。而後蜀石恪擅繪人物鬼神,形象多作丑怪奇詭之狀,繪有《二祖調心圖》。五代時的道士張素卿擅長道畫,「曾於青城山丈人觀,畫五嶽四瀆真形,並十二溪女數壁,筆跡遒健,神彩欲活。見之者心驚神悸,足不能進,實畫之極至者也」。

五代楊凝式兼具顏柳的專長。上蒴二王,側鋒取態,鋪毫著力,遂於離亂之際獨饒承平之象,也為唐書之回光,以《夏熱帖》、《神仙起居法》、《草堂十志圖跋》傳世。五代之際,狂禪之風大熾,此亦影響到書壇,『狂禪書法』雖未在五代一顯規模,然對宋代書法影響不小。

由於戰亂與天災,五代十國的科技發展不如唐朝,而南方十國的科技發展較北方五代興盛。然而整體而言,在製瓷雕版、農業水利與火器方面仍有發展的地方。朝廷的曆書因为藩镇割据不能遍及全國,人民為了能有曆書使用,紛紛採用民間曆書。其中唐朝曹士所編的《符天曆》流行於唐朝后期、五代與北宋民間,有好幾百年之久。《符天曆》是以顯慶五年為曆元,以雨水為氣首,以一萬為基本天文數據的分母,從而大大減輕了計算工作。由於不是官方頒布的曆書,被貶稱為小曆。在醫學方面,五代出現了官方醫官,後唐於清泰年間增設翰林醫官之職。北宋後改太醫署為太醫局,並設翰林醫官院。後蜀的韓保升是本草學家,他詳察藥品,深知药性,施药辄神效。在後蜀帝孟昶的支持下,他以唐朝《新修本草》為藍本,重新編著成《蜀重廣英公本草》,史稱《蜀本草》,後散失。

在農業與水利方面,五代也有很高的發展。韓鄂一般被認為是唐末五代人,他參考唐朝以前的農書如《齊民要術》等,撰寫出《四時纂要》,是繼《齊民要術》之後,又一本重要的農書。書中採用了「月令」的形式,按月編排農民每月應作的事宜,其中以農業為主體。此外也記錄許多當代的農業技術,其中還首次記載茶樹、棉花、香菇和薯蕷等作物的栽培技術以及人工養蜂。唐朝的茶葉研究到五代時仍然盛行,其中前蜀毛文錫著有《茶譜》等。由於五代時期的河患增多,治河規模和次數都較前代為多。再加上南方十國極力發展經濟,一些沿海堤防或河道工程也積極建設。五代時已有遙堤出現。924年後唐時修築酸棗河堤,於隔年由符習成功修復。五代還使用「帚工」來護岸、堵口、護堤的水工建築物。主要是將薪柴、竹木、軟草等夾以土石捆紮成帚捆,然後連接起來,具有很好的抗水衝擊作用。最後在宋朝成熟並被普遍推廣使用。五代時,江南吳國、吳越國與南唐重修江河,引湖水濟運,持續發展唐朝的塘浦圩田系統,並且還修築錢塘江石塘以防海潮侵襲。

五代的吳國、南唐常將火药、猛火油等使用在戰爭上。904年楊行密軍圍攻洪州﹙今江西南昌﹚,部將鄭璠命所部「發機飛火,燒龍沙門,率壯士突火先登入城,焦灼被體」。975年北宋大軍南征南唐,南唐將領朱全贇用猛火油縱火攻宋軍,但最後因為風向改變,火焰反燃己軍而大潰。

%% -*- coding: utf-8 -*-
%% Time-stamp: <Chen Wang: 2019-12-24 16:08:49>


\section{后梁\tiny(907-923)}

\subsection{简介}

後梁(907年-923年)是中国历史上五代十国时期建立的第一个皇朝。自907年梁太祖朱温篡唐称帝建國至923年11月19日梁末帝亡国于后唐,历时17年。因为皇帝为朱姓,为与南北朝时的南梁区别,故又称朱梁。

朱全忠原为唐僖宗在位时代爆发的黄巢之乱将领,原名朱溫。降唐后赐名全忠,任宣武军节度使,据汴州(今河南开封),渐渐成为唐末最强大的藩镇,并受封为梁王。天祐元年(904年)闰四月,自西京长安劫持唐昭宗李晔至东都洛阳,又于八月加害,并另立年仅13岁的李祝为帝,即唐哀帝,并发起白马驿之祸清洗官僚及杀害皇室成员,准备篡位。

唐天祐四年四月十八日(907年6月1日),朱全忠迫使哀帝禅位,以其封爵“梁王”(“梁”之得名来自其祖居砀山为战国时梁国故地)改国号为梁,建元开平,都汴(号开封府),史称后梁,是为梁太祖。根據五行相生的規律,唐朝的「土」德之後為「金」德,因此後梁以「金」為王朝德運。后梁疆土虽是中原五个王朝中最小的一个,辖地仅今河南、山东两省,陕西、湖北大部,河北、宁夏、山西、江苏、安徽等省的一部,但也是存在最久的五代朝,并得到契丹及一些藩镇的尊奉。

朱温篡唐后,很多藩镇均不承认后梁,仍用唐年号。次年(908年)蜀王王建也称帝,建立了前蜀。当时有些割据势力表示归顺后梁,朱温遂晋封湖南马殷为楚王,两浙钱镠为吴越王,广东刘隐为靜海軍節度使,福建王审知为闽王。同年,太祖杀唐哀帝。開平三年(909年),幽州刘守光自稱幽州節度使,建立刘燕。而位於江浙一帶的淮南節度使楊行密則早在天復二年(902年)就已受封為吳王,建立吳。连同后梁,十一个割据势力并存。

由于过去朱温与李克用在公私事上均有有恩怨,所以自建国起,后梁与晋王李克用、李存勗持续战斗,直至亡国。后梁建立后发兵8万,打算收复被李克用占据的潞州,但围攻半年不下。次年初李克用死,李存勗继为晋王,亲率晋军为潞州解围,大获全胜。

梁太祖疑功臣,迫使镇州(今河北正定)王鎔和定州(今属河北)王处直,于开平四年(910年)起兵反梁,并向晋王求援。乾化元年(911年)初,李存勗率晋军击梁军于柏乡(今属河北),经过一日激战后,梁军大败。晋军追击150余里,直至邢州(今河北邢台),又连克澶州(今河南濮阳)、新乡(今属河南)等地。梁太祖亦亲自率军前往洛阳设防。柏乡之战梁军主力受损,后梁处于劣势。

次年,梁太祖趁晋攻燕,晋、赵、定州,带病亲率军北上,号称50万大军。昼夜兼行,至下博(今河北深州),率军5万转攻蓨县(今景县)。其时晋军主力北攻幽州,南方空虚,僅有符存審的少量兵力鎮守赵州(今河北赵县),但他用計以小部队骚扰梁军,又派数百骑兵伪装梁军,趁夜袭了梁太祖营寨,外加被晋军释放的梁军士兵,归来后传言晋王李存勗亲率大军来攻,梁太祖惊惶失措,遂烧营夜遁。

乾化二年(912年)五月,梁太祖退至洛阳,病入膏肓,同年六月,为次子朱友珪所杀。次年,朱友珪又为禁軍所杀,梁太祖四子东都留守朱友贞遂在东都开封继位,是为梁末帝。後梁内乱相继,只有大将杨师厚率军与晋、赵周旋于河北。

贞明元年(915年)春,杨师厚死,梁朝廷密謀把杨师厚的領地一分為二,魏州(今河北大名县东北)军士叛降于晋,晋王李存勗乃亲征出兵太行黄泽岭(今山西左权东南),又袭德州(今属山东)、澶州,梁军连战皆败。次年春,梁末帝命王檀率军3万北上,直奔太原,企图袭取晋军基地,但为守城军击败。

贞明四年(918年)八月,晋王李存勗举兵魏州南下,想要灭梁,与梁军相持于濮州一带。十二月下旬,晋军至胡柳陂(今濮阳西南),贺瓌率梁军跟踪而至,两军激战,梁军骑军王彦章败,西逃时冲散了晋军的西线部队,晋名将周德威战死。另一名將李存審與李嗣昭、王建及率領骑兵反攻冲击梁軍步兵,后梁惨败,伤亡近3万。但晋军也因此战元气大伤,梁晋战争沉寂了一段时期。

龙德元年(921年)春,晋王李存勗正拟称帝之际,镇州王鎔为部下张文礼所杀。张文礼勾结後梁与契丹。晋军进围镇州时,梁军袭击晋军,却反为晋军所败,死伤2万多人。

龙德三年(923年),晋王李存勗称帝,国号大唐,史称後唐。

龙德三年闰四月末,后唐乘后梁西攻泽州(今山西晋城),派将李嗣源率骑5000袭郓州(今山东东平),次日清晨占之。

后梁启用王彦章为帅,段凝为副帅,调集精兵10万北讨后唐。李存勗亲自率军与梁军苦战于杨刘(今东阿)。后王彦章兵败中都县(今山东汶上)被俘斩。923年11月19日后唐军达开封城下,开封随即降唐,后梁亡。梁末帝自杀。


%% -*- coding: utf-8 -*-
%% Time-stamp: <Chen Wang: 2019-12-24 16:36:46>

\subsection{太祖\tiny(907-912)}

\subsubsection{生平}

梁太祖朱温(852年12月5日-912年7月18日),五代時期後梁開國皇帝,曾参与黄巢之亂,后降唐为将,唐僖宗賜名朱全忠。但又密谋杀害唐昭宗,立唐哀帝,后废哀帝自立,建立“后梁”,称帝后改名朱晃。晚年大肆荒淫,強姦兒媳。後為三子朱友珪所殺,終年59歲。

朱全忠出生于大中六年(852年)12月5日。朱全忠原名朱温,宋州砀山午沟里(今安徽省砀山县)人。年幼喪父,年少时是一個游手好閒、不务正业的无赖。

乾符四年(877年)朱溫参加黃巢軍,反抗朝廷,屡立战功,很快升为大将。大齐政权建立后,任同州防御使,率军攻打河中。由于屡战屡败,怕受责罰,于是叛变降唐,投歸河中節度使王重榮。唐僖宗任朱溫為左金吾衛大將軍,充河中行营副招讨使,並赐名“全忠”。中和三年(883年)又被授以宣武节度使,随后击败黄巢。

龍紀元年(889年)斬黃巢餘部蔡州節度使秦宗權,被封為東平王。黃巢覆亡後,唐帝國已名存實亡,各藩鎮擁兵自重,其中以宣武節度使朱全忠、河東節度使李克用、鳳翔節度使李茂貞、盧龍節度使劉仁恭、鎮海節度使錢镠、淮南節度副大使楊行密等人勢力最大,史載“郡將自擅,常賦殆絕,藩鎮廢置,不自朝廷”,“王室日卑,號令不出國門”。

天復元年(901年)昭宗被宦官韓全誨幽禁,宰相崔胤乃召朱全忠救駕。韓全誨不得已投靠鳳翔節度使李茂貞,朱全忠進攻鳳翔,鳳翔食盡待援。天復三年(903年),節度使李茂貞殺宦官韓全誨等七十餘人,與朱全忠和解,護送昭宗出城,昭宗又回到長安。崔胤指責宦官“大則構扇藩鎮,傾危國家;小則賣官鬻爵,蠹害朝政”,不久朱全忠盡殺宦官數百人,廢神策軍,完全控制皇室。天復元年(901年)封为梁王。天祐元年(904年),朱全忠殺宰相崔胤,逼迫昭宗遷都洛陽,八月壬寅夜,指使朱友恭、氏叔琮、枢密使蒋玄晖等人殺昭宗,另立其子李柷為帝,是為唐哀帝。天祐二年(905年),在親信李振鼓動下,於滑州白馬驛(今河南滑縣境)一夕殺盡杀宰相裴樞、崔遠等朝臣三十餘人,投屍於河,史稱“白馬之禍”。年末,预备篡位称帝,让宰相柳璨、蒋玄晖谋划受九锡,蒋玄晖与太常卿张廷范认为天下未定不可过急,朱全忠不悦。宣徽副使蒋殷、赵殷衡素与蒋玄晖、张廷范不和,趁机诬告他们与柳璨对何太后盟誓复唐,朱全忠怒,遣使杀蒋玄晖,密令蒋殷、赵殷衡在积善宫缢杀何太后,迫哀帝下诏称太后系秽乱宫闱自杀谢罪,追废为庶人,停新年郊礼。又贬杀柳璨、张廷范。

朱全忠為節度使時,用法苛嚴,大軍交戰時,如將軍戰死,所部士卒則一律斬首,稱「跋隊斬」,自是戰無不勝。而且士卒逃匿州郡,不歸者甚眾,為防士卒逃亡,朱全忠命軍士紋面以記軍號。

開平元年(907年)廢唐哀帝,自行称帝,改名为晃,建都開封,国号为“梁”,史称“后梁”,后人称为后梁太祖。封李柷為濟陰王,次年又殺李柷,自此唐朝結束289年的統治,中國進入五代十國的紛亂時期。

朱全忠在位時頗重視農業發展,下令兩稅法之外不得妄有科配,并曾因侄子朱友谅不恤灾民却进献瑞麦怒罢其官;但因連年戰事,民不聊生,開平四年(910年)發生柏鄉之戰并战败,與晉王李存勗矛盾加劇。晚年宮廷內陷入權力鬥爭。朱溫生性殘暴,殺人如草芥。夫人在世時尚能勸止,死後卻大肆淫亂,甚至亂倫,包括兒媳都得入宮侍寢。乾化二年(912年)被三子朱友珪刺杀,终年61岁,在位6年。

毛泽东评价他说:“朱温处四战之地,与曹操略同,而狡猾过之。”


\subsubsection{开平}

\begin{longtable}{|>{\centering\scriptsize}m{2em}|>{\centering\scriptsize}m{1.3em}|>{\centering}m{8.8em}|}
  % \caption{秦王政}\
  \toprule
  \SimHei \normalsize 年数 & \SimHei \scriptsize 公元 & \SimHei 大事件 \tabularnewline
  % \midrule
  \endfirsthead
  \toprule
  \SimHei \normalsize 年数 & \SimHei \scriptsize 公元 & \SimHei 大事件 \tabularnewline
  \midrule
  \endhead
  \midrule
  元年 & 907 & \tabularnewline\hline
  二年 & 908 & \tabularnewline\hline
  三年 & 909 & \tabularnewline\hline
  四年 & 910 & \tabularnewline\hline
  五年 & 911 & \tabularnewline
  \bottomrule
\end{longtable}

\subsubsection{乾化}

\begin{longtable}{|>{\centering\scriptsize}m{2em}|>{\centering\scriptsize}m{1.3em}|>{\centering}m{8.8em}|}
  % \caption{秦王政}\
  \toprule
  \SimHei \normalsize 年数 & \SimHei \scriptsize 公元 & \SimHei 大事件 \tabularnewline
  % \midrule
  \endfirsthead
  \toprule
  \SimHei \normalsize 年数 & \SimHei \scriptsize 公元 & \SimHei 大事件 \tabularnewline
  \midrule
  \endhead
  \midrule
  元年 & 911 & \tabularnewline\hline
  二年 & 912 & \tabularnewline\hline
  三年 & 913 & \tabularnewline
  \bottomrule
\end{longtable}


%%% Local Variables:
%%% mode: latex
%%% TeX-engine: xetex
%%% TeX-master: "../../Main"
%%% End:

%% -*- coding: utf-8 -*-
%% Time-stamp: <Chen Wang: 2019-12-24 16:37:49>

\subsection{郢王\tiny(912-913)}

\subsubsection{生平}

朱友珪(885年-913年),小字遙喜,五代時期後梁皇帝,為後梁太祖朱全忠之第三子,弑父自立。登基后不得民心,为袁象先所杀。

朱友珪之母為亳州營妓,唐僖宗光啟年間(885年-888年),朱温有一次率軍經過亳州,召其母陪侍,並且使之懷孕,朱温離去後,其母差人告以生男,朱温大喜,遂名遙喜,後來為朱温接回。朱温篡唐後,將他封為郢王。後梁開平四年(910年)被任命為左右控鶴都指揮使。

朱温晚年,長子郴王朱友裕已死;次子博王朱友文本名康勤,是朱温的義子;三子即朱友珪,時為實際上的長子;四子均王朱友貞。朱温自妻子张氏過世後,就開始縱情聲色,荒淫無度,甚至不顧倫理,經常召諸子之妻入宮陪侍。朱友珪之妻張氏貌美,亦被朱温召去同寝。但後来義子朱友文之妻王氏也入宮和朱温通姦,并特别得到朱温寵愛,在王氏的煽动下,朱温有了改以朱友文繼位的打算。

乾化二年(912年),朱温病重,命王氏召朱友文託付後事,张氏急忙把這件事告訴朱友珪。朱友珪遂率所部政變,由僕夫馮廷諤殺朱温,並假傳遺詔,自登帝位。第二年(913年)改年號為鳳曆。

朱友珪登帝位後,雖然大量賞賜將兵以圖收買人心,然而很多老將還是頗為不平,而朱友珪本人又荒淫無度,因此人心沸騰。鳳曆元年(913年),朱温之婿趙巖、朱温之甥袁象先、均王朱友貞、將領楊師厚等人密謀政變。袁象先首先發難,率禁軍數千人殺入宮中,朱友珪無法逃脫,遂命馮廷諤將他及张皇后都殺死。死後被追廢為庶人。


\subsubsection{凤历}

\begin{longtable}{|>{\centering\scriptsize}m{2em}|>{\centering\scriptsize}m{1.3em}|>{\centering}m{8.8em}|}
  % \caption{秦王政}\
  \toprule
  \SimHei \normalsize 年数 & \SimHei \scriptsize 公元 & \SimHei 大事件 \tabularnewline
  % \midrule
  \endfirsthead
  \toprule
  \SimHei \normalsize 年数 & \SimHei \scriptsize 公元 & \SimHei 大事件 \tabularnewline
  \midrule
  \endhead
  \midrule
  元年 & 913 & \tabularnewline
  \bottomrule
\end{longtable}


%%% Local Variables:
%%% mode: latex
%%% TeX-engine: xetex
%%% TeX-master: "../../Main"
%%% End:

%% -*- coding: utf-8 -*-
%% Time-stamp: <Chen Wang: 2021-11-01 15:20:36>

\subsection{末帝朱友贞\tiny(913-923)}

\subsubsection{生平}

朱友貞(888年-923年11月18日),後改名朱瑱、朱鍠,五代時期後梁皇帝,為後梁太祖朱全忠之第四子,也是朱全忠的嫡子,母亲张夫人。朱友珪异母弟。亡国后,被后唐追废为庶人。朱友貞在位10年,在五代諸帝中在位年期最長。

朱友貞在朱全忠篡唐後,被封為均王。後梁開平四年(910年)被任命為東京馬步軍都指揮使。

後梁乾化二年(912年),郢王朱友珪弒朱全忠自立,大量封賞將兵以圖收買人心,朱友貞當時亦被任命為東京(大梁,今河南開封)留守,開封府尹。然而包括朱友貞在內的眾多官員、將領仍然對朱友珪的行為十分不滿。次年(913年),朱友貞與朱全忠之婿趙巖、朱全忠之甥袁象先、將領楊師厚等人密謀政變。袁象先首先發難,率禁軍數千入殺入宮中,朱友珪無法逃脫,由左右將其殺死。朱友貞遂在大梁稱帝,取消朱友珪的鳳曆年號,仍使用朱全忠的乾化年號。

朱友貞雖然登上帝位,但是他接手的是一個外患內亂即將不斷引爆的帝國。北方的晉王國,於朱友貞登位的同年(913年),在晉王李存勗的率領之下,滅桀燕。915年,朱友貞改元貞明。同年,天雄節度使(魏博節度使)楊師厚去世,魏博自唐末即以地廣兵強著稱,朱友貞藉此機會分割天雄軍,不料卻引起天雄軍官兵的叛變,歸降晉王國。

該年(915年),朱友貞亦被康王朱友敬(一作朱友孜)派人行刺,此事件之後,朱友貞漸漸疏遠宗室,只信任心腹的幾個人。

貞明二年(916年),在與晉的數場會戰敗北後,後梁黃河以北之地幾乎全部喪失。之後的數年間,後梁與晉持續爭戰,然而勝少敗多,領土不斷地被蠶食。

貞明七年(921年),朱友貞改元龍德。龍德三年(923年),已即後唐帝位的李存勗率軍對後梁發動總攻,勢如破竹,朱友貞在後唐軍攻入大梁的前夕,命控鶴都將皇甫麟將他殺死,後梁亦隨之亡國。

\subsubsection{乾化}

\begin{longtable}{|>{\centering\scriptsize}m{2em}|>{\centering\scriptsize}m{1.3em}|>{\centering}m{8.8em}|}
  % \caption{秦王政}\
  \toprule
  \SimHei \normalsize 年数 & \SimHei \scriptsize 公元 & \SimHei 大事件 \tabularnewline
  % \midrule
  \endfirsthead
  \toprule
  \SimHei \normalsize 年数 & \SimHei \scriptsize 公元 & \SimHei 大事件 \tabularnewline
  \midrule
  \endhead
  \midrule
  元年 & 913 & \tabularnewline\hline
  二年 & 914 & \tabularnewline\hline
  三年 & 915 & \tabularnewline
  \bottomrule
\end{longtable}

\subsubsection{贞明}

\begin{longtable}{|>{\centering\scriptsize}m{2em}|>{\centering\scriptsize}m{1.3em}|>{\centering}m{8.8em}|}
  % \caption{秦王政}\
  \toprule
  \SimHei \normalsize 年数 & \SimHei \scriptsize 公元 & \SimHei 大事件 \tabularnewline
  % \midrule
  \endfirsthead
  \toprule
  \SimHei \normalsize 年数 & \SimHei \scriptsize 公元 & \SimHei 大事件 \tabularnewline
  \midrule
  \endhead
  \midrule
  元年 & 915 & \tabularnewline\hline
  二年 & 916 & \tabularnewline\hline
  三年 & 917 & \tabularnewline\hline
  四年 & 918 & \tabularnewline\hline
  五年 & 919 & \tabularnewline\hline
  六年 & 920 & \tabularnewline\hline
  七年 & 921 & \tabularnewline
  \bottomrule
\end{longtable}

\subsubsection{龙德}

\begin{longtable}{|>{\centering\scriptsize}m{2em}|>{\centering\scriptsize}m{1.3em}|>{\centering}m{8.8em}|}
  % \caption{秦王政}\
  \toprule
  \SimHei \normalsize 年数 & \SimHei \scriptsize 公元 & \SimHei 大事件 \tabularnewline
  % \midrule
  \endfirsthead
  \toprule
  \SimHei \normalsize 年数 & \SimHei \scriptsize 公元 & \SimHei 大事件 \tabularnewline
  \midrule
  \endhead
  \midrule
  元年 & 921 & \tabularnewline\hline
  二年 & 922 & \tabularnewline\hline
  三年 & 923 & \tabularnewline
  \bottomrule
\end{longtable}


%%% Local Variables:
%%% mode: latex
%%% TeX-engine: xetex
%%% TeX-master: "../../Main"
%%% End:


%%% Local Variables:
%%% mode: latex
%%% TeX-engine: xetex
%%% TeX-master: "../../Main"
%%% End:

%% -*- coding: utf-8 -*-
%% Time-stamp: <Chen Wang: 2019-12-24 16:41:33>


\section{后唐\tiny(923-937)}

\subsection{简介}

后唐(923年-937年)是中國五代時期的政權之一。923年,唐朝的赐姓沙陀人李存勖消灭後梁,重建唐朝。在魏州(河北大名县西)称帝,以“復興唐朝”为名,不久迁都洛阳。後為石敬瑭勾結契丹入侵而滅亡。史学家為了區別由李淵所建立的唐朝,因而稱之為後唐,歷時十四年。

另外,雖然后唐統治者的祖源是沙陀族,但當時的後唐統治者被視為漢族,後唐也被稱為「漢國」。

後唐之建立可追溯到唐朝末年龐勛之亂。龐勛之亂發生在唐懿宗咸通年間,唐政府命沙陀族的朱邪赤心領兵平亂,並賜名李國昌,编入唐朝宗籍,成为唐朝宗室。後來李國昌病故,其子李克用又協助唐室平定黃巢之亂和王行瑜之乱。李克用平亂有功,被封河東節度使,駐守太原,受封晉王。

李克用追擊黃巢時,曾被當時的宣武節度使朱溫邀請入汴梁作客,因为某些不明原因(据朱温党羽的说法,李克用酒後侮辱朱溫),朱溫决意趁李克用酒醉殺害他,李克用突圍而出才脫身,自此他與朱溫誓不兩立。朱溫後來篡唐建立後梁,李克用作为唐朝宗室,仍用唐天祐年號,以唐朝北都(今天的山西太原)为基地,发誓兴复唐朝,事实上成为中国南北各地反梁势力的盟主。故晉王集团成為後梁北方最大的威脅。

李克用死後,兒子李存勖繼承晉王爵位,屡败梁军。915年(唐天祐十二年,后梁貞明元年),梁在河北鎮守的鄴王楊師厚死,河北发生反梁兵变,李存勖乘機占据河北,晉與後梁在黄河爭峙。李存勖在923年于魏州(今河北大名县)稱帝,宣布继承唐朝皇统,史稱後唐,改元同光。同年唐軍直迫汴京滅了後梁,定都洛陽,李存勖即位,成為後唐莊宗。後唐不承認後梁為正統,在其五行德運的選取上選擇恢復唐朝的「土」德。

後唐莊宗定都洛陽後,力图恢复大唐的荣光。他見前蜀王氏無道,925年派郭崇韜攻入成都,不出七十日就滅了前蜀。至此,汉地诸藩国臣服,後唐莊宗成为长城以南整个汉地公认的唯一皇帝。當時南方諸藩国深感恐慌,認為中原很快就會派大軍來削平南方的割据。

但另一方面,後唐莊宗漸自大貪逸,宠信宦官、伶人,疏远旧将,内部矛盾激化。皇后刘氏和宦官與郭崇韜不和,向莊宗進讒言,結果郭崇韜被误殺,引來政局动荡和軍人反叛。一個半月內,各地起兵反叛,後唐大亂。

莊宗不得已,派李克用的養子李嗣源,往河北討伐反叛,在嗣源的女婿石敬瑭的策動下,河北軍立嗣源為帝,反攻洛陽。莊宗被伶人郭從謙所殺,史稱興教門之變,以後,李嗣源即皇帝位,是為後唐明宗。

後唐明宗李嗣源即位後,也有相當治績,朝政漸為安定。但軍人安重誨專權,未能处理好与孟知祥、董璋的关系,两人发生内斗,孟知祥取胜,结果为后来后蜀脱离后唐独立埋下了祸根。

明宗晚年病倒在床時,秦王李從榮以為明宗已死,起兵企圖攻入皇宮,結果事敗被殺。明宗得知秦王被殺,震驚之下駕崩,大臣妃子擁立宋王李從厚,是為後唐閔帝。

閔帝即位後,採用削藩政策,引起潞王李從珂的叛亂,叛軍攻入京師,而閔帝夫婦逃往河北,被姐夫石敬瑭設計除去其手下將士,為從珂軍士擒殺,從珂即帝位,是為後唐末帝。

末帝與鎮守太原的河東節度使石敬瑭不和,936年末帝下詔把石敬瑭調任,引來石敬瑭的叛亂。末帝發兵攻太原,石敬瑭向契丹借兵,遼太宗耶律德光親率大軍南下,唐軍大敗,937年,契丹與石敬瑭的大軍攻入洛陽,1月11日末帝自焚而死,後唐滅亡。歷四帝共十四年。

後唐据有今河南、山东、山西、河北、陕西关中、甘肃东部、湖北北部、安徽北部。

\subsection{武帝生平}

李克用(856年10月24日-908年2月24日),字翼圣,神武川之新城(今山西雁门)人,後唐莊宗李存勗之父,本姓朱邪(又作朱耶),其父受唐朝天子賜李姓。綽號鴉兒、三郎、獨眼龍、飛虎子,沙陀族人,唐大中十年(856年)生于神武川之新城(在今山西雁門北部)。是中國唐朝末年最強大的藩鎮節度使之一,後受唐封為晉王。后唐建立後,尊稱其為唐太祖武皇帝。

李克用為沙陀人,本姓朱邪(又作朱耶),其父朱邪赤心因鎮壓龐勛兵變有功,受唐懿宗賜姓名為李國昌,因此克用亦以「李」為姓。李克用是國昌第三子,家族內暱稱「三郎」。因天生一目較為細小,而被人稱為獨眼龍;作戰驍勇,又因其衝鋒陷陣為諸將之冠,軍中也稱其為飛虎子。

李克用驍勇善騎射,年15歲即從軍,後來被唐朝廷任命為沙陀副兵馬使。唐僖宗乾符五年(878年),當時代北(今山西省北部)饑荒,漕運不繼。大同(今山西省大同市)防禦使段文楚大量縮減軍士衣物和米糧的供應,而執法嚴厲,士卒怨恨。李克用為下屬所擁,殺段文楚而起事。廣明元年(880年),再殺河東(今山西省太原市)節度使康傳圭,佔領太原,不久為唐軍所敗,與父逃入韃靼部落。

唐僖宗中和元年(881年),公家赦李氏父子之罪,命李克用率沙陀軍南下助戰,以鎮壓佔領兩都、自稱大齊皇帝的黃巢。克用途經太原,因無法得到犒賞,遂縱容沙陀軍剽掠河東居民,引起百姓驚駭,不久北返,並繼續剽掠北部邊境一帶。中和二年(882年)李克用二次受敕勤王,此次沙陀軍南下,正式面對齊軍。中和三年(883年),屢敗齊軍,黃巢退出長安,由於李克用在長安收復戰中功勞最大,因此被命為河東節度使,河東也成為他後來的根據地。其時黃巢兵勢仍強,宣武節度使朱溫等各鎮皆無法抵擋,遂請河東軍來援。中和四年(884年),李克用再自河東南下大敗齊軍,最終使得黃巢在狼虎谷(在今山東省萊蕪市)自殺。

河東節度使李克用擊敗黃巢後回師河東,途經宣武節度使首府汴州(今河南省開封市),受節度使朱溫(朱全忠)邀請入城。朱溫因李克用酒後言語中多有侮辱,趁李克用酒醉之際夜襲,李克用幾乎被殺,狼狽逃回太原(上源驿之变),從此宣武朱氏與河東李氏兩家結下深仇。光啟元年(885年),朝廷讨伐河中节度使王重荣。李克用根据王重荣的假情报,指責盤踞關中的静难节度使朱玫、凤翔节度使李昌符結交朱溫欲滅己,因此進軍關中擊敗二人。僖宗逃往鳳翔。战乱中,各方军队进入長安,縱火大掠,雖然不久即行退去,然而在黃巢亂後稍有恢復的京師長安再度毀於一旦。唐昭宗大順元年(890年),朱溫與宰相張濬力主討伐河東,乃削李克用之官爵,聯合諸鎮之兵進攻,卻反為河東軍所敗,副都统孙揆被擒杀,朝廷禁军大损,只好恢復李克用的官爵。乾寧二年(895年),李茂貞、王行瑜及韓建三帥進京挾持唐昭宗,李克用不念旧恶,再度率軍勤王,敗三帥,救出昭宗;王行瑜被迫出走,被手下杀死。因功,十二月(合896年)李克用被封為晉王,但昭宗怕其日后难制,没有同意他继续消灭李茂贞、韩建。其後數年,李克用持續與朱溫爭戰,相互間成為爭奪天下的最大對手。但李克用性格直率,往往得罪人而不察,又对一些险诈之徒缺乏防范,多次遭到背叛,特别是幽州劉仁恭叛變,使李克用元气大伤。而宣武不斷併吞鄰鎮,兵勢日盛;李克用漸居下風,甚至无法救援自己的女婿河中节度使王珂,致使其被迫投降朱温。900年,黃河以北藩镇多附朱溫。

天復元年、二年(901年、902年),朱溫兩次率軍圍攻太原,李克用顽强抵抗,迫使敌军撤退。此後,朱全忠挾天子以令諸侯,焚毁长安,弑杀唐昭宗,唐室如風中殘燭。李克用坚决反对朱温,多次发起勤王行动,但均未能成功。天祐三年(906年),李克用收复潞州,粉碎了朱溫精心策划的总攻势。天祐四年(907年),朱溫篡唐稱帝,建立後梁,改元開平,李克用仍用唐天祐年號,以復興唐朝為名與後梁展開另一個階段的爭鬥。次年(908年),在与後梁的战争中,李克用因积劳成疾去世,子李存勗繼立。後來,李存勗滅後梁、建後唐,李克用被追諡為武皇帝,廟號太祖。

身為一名軍閥,李克用年輕時確有不少违抗唐政府的事蹟,但或許因為其年少英雄,後來又打著「復興唐室」的名號來對抗惡名昭彰的朱溫,因此歷史上有諸多關於他正面的傳說。而有意思的是,這些傳說不少與他所擅長之武器「箭」有關。

李克用一目失明,號「獨眼龍」。《五代史補》中記有一則故事:在李克用佔據河東,聲威大振後,盤據淮南的另一個軍閥楊行密很想見見李克用甚麼模樣。於是楊行密找了一個畫家,假扮商人到河東伺機偷畫李克用面貌。不料畫家到了河東,立刻為事先得到情報的河東武士所俘虜。克用頗怒,就對親信說:「我少了一隻眼睛,看他要怎麼畫我。」等到畫家一到殿內,李克用立刻大怒說道:「淮南派你來畫我,想必你是畫家中最好的,如果今天畫得不好,那麼這裏就是你的死地!」當時是炎夏,所以畫家最初畫李克用手拿扇子搧風,而扇角正好遮住了李克用失明的眼晴,非常巧妙。但克用不喜歡,說畫家「諂媚」,命其重畫。畫家這次重畫,就畫李克用彎弓射箭,一隻眼睛瞇了起來,好像就在瞄準目標,克用大喜,於是重賞畫家銀兩,並送之回淮南。

這則故事見於罗贯中的《殘唐五代史演義傳》第九回《克用箭服周德威》;另外,白樸的元曲也有《李克用箭射雙鵰》的折子;除此之外,京劇《珠簾寨》中也有這個橋段。大意是李克用在受唐政府所託出軍討伐黃巢時,兵至珠簾寨,遇見一將周德威擋路,李克用素聞周德威之名,躍馬向前迎戰,二人交手一百餘合不分勝負,遂以射鵰為賭注,若李克用能箭射飛鵰,周德威即下馬受降。只見李克用彎弓一射,弓弦響處,鵰已落地,周德威甘服而降。這則故事的原型,很可能是出自《舊五代史·武皇紀上》的記載。而《新五代史·唐本紀第四》亦有其善射的敘事。

這可能是李克用和李存勗父子間最有名的故事:宋初王禹偁的《五代史闕文》記傳傳說李克用臨終時,將三支箭交給李存勗,說道:「劉仁恭父子背叛我,契丹耶律阿保機違背與我們的盟約,朱溫和我們是世仇,我給你三支箭,第一支箭要你討伐劉仁恭,第二支箭要你打敗契丹,第三支箭要消滅朱溫,希望你完成我這三個願望。」李存勗把三支箭供奉在宗廟裏,逢出征時依次派人取箭,帶上戰場,後來併桀燕、敗契丹、滅後梁,得勝後分別將箭送回宗廟,表示完成了李克用的願望。

不過,歷代都有人懷疑這則故事的真實性。司馬光《資治通鑑考異》認為,依《舊五代史·契丹傳》記載:李存勗剛繼位時,後梁兵圍潞州(今山西省長治市),因此尚且對契丹「遣使告哀,賂以金繒,求騎軍以救潞州」。可見根本沒有和契丹結仇的事。另外,有人亦指出朱全忠掃蕩群雄,華北只剩李克用與劉仁恭二者,李克用父子深知唇亡齒寒之理,因此劉仁恭及其子劉守光被朱溫圍攻,甚至劉守光被其兄劉守文攻擊,李克用、李存勗還屢次派兵來救,因此至少在李克用去世之時,河東也沒有與桀燕劉氏對立一事。


%% -*- coding: utf-8 -*-
%% Time-stamp: <Chen Wang: 2019-12-24 16:43:09>

\subsection{庄宗\tiny(923-926)}

\subsubsection{生平}


唐莊宗李存勗xù(885年12月2日-926年5月15日),山西应县人,沙陀族,本姓朱邪,因其父是河東節度使李克用受唐懿宗赐以李姓,而改姓李,諱存勗,唐光启元年正月(885年12月)生于山西应县,五代時期后唐开国皇帝。小名“亚子”,藝名“李天下”,以勇猛闻名。

923年5月13日在魏州(河北大名府)称帝,国号唐,史称后唐。後因義兄李嗣源被軍士擁戴造反,揮軍直取洛陽。宮中指揮使郭从谦為報仇,趁機發動兵變——興教門之變,將存勗殺害。

《旧五代史》记载:庄宗光圣神闵孝皇帝,讳存勖,武皇帝之长子也。母曰贞简皇后曹氏,以唐光启元年岁在乙巳,冬十月二十二日癸亥,生帝于晋阳宫。《册府元龟》记载:后唐庄宗以光启元年十月癸亥生于晋阳宫。《旧唐书》记载:十二月辛亥朔。李存勖以唐光启元年十月二十二日(885年12月2日)生于晋阳宫,而十月二十二日不是癸亥是癸酉,十月十二日才是癸亥。同光元年至三年的万寿节(李存勖生日)皆在十月二十二日,李存勖的生日就是十月二十二日。光启元年十月壬子朔,癸亥是十二日,癸酉是二十二日,李存勖若生于十二日何以过二十二日生日,此当是编纂者失察误抄所致。

李存勗是李克用與貞簡皇后曹氏的長子。他自幼擅長騎馬射箭,膽力過人,為李克用所寵愛。少年時隨父作戰,11歲就與父親到長安向唐朝朝廷報功,得到唐昭宗的賞賜和誇獎。

李存勗成年後狀貌雄偉瑰麗,得習《春秋》,豁達而且通大義,並勇敢善戰,熟知戰略要術。他又喜愛音樂、歌舞、俳優之戲,旁人多有異談。當時,軍閥割據混戰、佔據河東的李克用常被控制河南的朱溫牽制圍困,兵力不足,地盤狹小,非常悲觀。李存勗勸說其父:“朱全忠恃其武力,吞滅四鄰,想篡奪帝位,這是自取滅亡。我們千萬不可灰心喪氣,要積蓄力量,等待時機”。李克用聽後大為高興,重新振作起來,與朱全忠對抗。

後梁開平二年(908年)正月,李克用病死,李存勗於同月襲晉王位。但是當時的兵馬大權歸於其叔父李克寧,軍民之事皆由李克寧決定,權柄既重,令眾人皆攀附李克寧。當辦完喪事後,李存勗與張承業、李存璋設計,要除去勢力龐大的叔父李克寧。同年二月二十日,當諸將於府第時,乃伏兵於府中,置酒大會,李克寧既至,於席間擒下李存顥、李克寧二人,李存勗哭著責備李克寧:「姪兒一開始就打算把軍隊、政權都讓給叔父,叔父不願意背棄我父親的遺命,怎麼現在又把我跟我母親丟給豺狼虎豹?叔父怎麼忍心?」李克寧泣對:「這是讒言啊,我還能說甚麼?」當日,李克寧與李存顥俱伏法。

其後,李存勗認為潞州(今山西上黨)是河東屏障,沒有潞州對河東不利,所以他立即率軍從晉陽出發,直取上黨,乘大霧突襲圍潞州的梁軍,大獲全勝。李存勗的用兵之奇使梁太祖朱溫大驚,他說:「生兒子就要生李存勗一樣的兒子,李克用不會滅亡了啊!至於我的兒子,豬狗之輩而已!」

當潞州之圍解決後,河東威振,控制鎮州的王鎔和控制定州的王處直見形勢驟變,也動搖了依附後梁的信心,竟然和李存勗結成聯盟共同對付後梁。後梁為了保護河北之地,不惜一切,出兵再戰,於是雙方在柏鄉又展開了一場血戰。柏鄉之戰中,晉軍有周德威等三千騎兵和鎮州、定州兵;對方梁軍有王景仁率領的禁軍和魏博兵八萬。梁軍守衛柏鄉、以逸待勞,在地形、兵力、裝備幾方面處於優勢;而晉軍是騎兵,機動性和進攻能力大,對梁軍構成威脅。戰役開始,李存勗採用周德威建議,引誘梁兵出城,聚而殲之,晉軍主動後撤。梁軍主將王景仁果然上當,傾巢而出。晉軍抓住機會,以騎兵猛烈突擊梁軍,周德威攻右翼,李嗣源攻左翼,鼓譟而進。這時晉軍李存璋率領的騎兵大隊也趕上,梁軍丟盔棄甲,死傷殆盡。這一仗,使梁軍喪失了對河北的控制權,之後,朱溫一聽晉軍就談虎色變。而李存勗卻進一步安定了河東局勢,他息兵行賞,任用賢才,懲治貪官惡吏,寬刑減賦,一時河東大治。

李克用臨死時,交給李存勗三支箭,囑咐他要完成三件大事:一是討伐劉仁恭,攻克幽州;二是征討契丹,解除北方邊境的威脅;第三件大事就是要消滅世敵朱溫。他將三支箭供奉在家廟裡,每臨出征就派人取來,放在精製的絲套裡,帶著上陣,打了勝仗,又送回家廟,表示完成了任務。其後李存勗達成李克用遺志,打敗契丹,攻破燕地,並且攻滅劉守光與劉仁恭父子割據的桀燕政權,並且於923年,在魏州(河北大名縣西)稱帝,國號為唐,史稱後唐,其後攻滅後梁,統一北方。李存勗還收降了李茂貞建立的岐,並攻滅王建所建立的前蜀。

李存勗以唐朝赐姓为李的合法继承人身份,打起中兴唐朝的旗号,并为唐朝皇帝立庙。又以诛灭唐朝逆臣之名,族灭了后梁宰相敬翔、李振等人,将帮助朱溫篡唐的旧臣11人贬官。

但李存勗到了晚年自認為已經拚命一生,應該好好享樂,遂荒廢朝政。李存勗自幼喜歡看戲、演戲,常粉墨登場,並自號藝名“李天下”。伶人大受皇帝寵幸,以至于伶人景进干预朝政。士大夫皆气愤,又不敢出气。李存勗又派伶人、宦官搶民女入宮,強擄魏博士卒們妻女千餘人,怨聲四起。同光二年,李存勗恢复旧唐宦官的势力,本来已经消失的监军又凌驾于藩镇之上,导致诸将更大的不满。同光三年(925年),李存勗派遣兒子魏王李继岌、侍中郭崇韬,攻滅前蜀。但是其後继岌、崇韬互相猜疑。郭崇韬又得罪宦官,李存勗於是对崇韬起疑,下命孟知祥入蜀,见机行事。翌年,李存勗被宦官的谗言所迷惑,诛杀了朱友谦、李存乂。后唐朝廷人心惶惶。

後唐同光四年(926年),魏博士兵皇甫暉在鄴城叛亂,是為鄴城之亂,李存勗命李绍荣前往討伐,久不能下,无奈命李嗣源攻鄴城,李嗣源命其女婿石敬瑭同征。兵進魏州時,李嗣源卻被叛軍擁戴,恭迎入城,李嗣源百口莫辯,石敬瑭表示就算不造反也無法免責,李嗣源因而擁兵自立,與魏博的叛軍合兵造反。李嗣源占據汴州(今河南開封),進軍洛邑,先鋒石敬瑭則帶兵逼進汜水關(河南滎陽汜水鎮),李存勗決定親征反擊。

這時擔任指揮使的伶人郭从谦不知李存乂已被莊宗殺死,欲奉李存乂之名作乱,火燒興教門。蕃汉马步使朱守殷见危不救。李存勗當時僅有符彥卿及王全斌等少數將領效忠他。郭从谦率兵攻入皇城。李存勗被流箭射中。王全斌將其扶至絳霄殿。李存勗失血過多,渴懑求飲,經宦官奉進酪漿,喝完一杯,遽爾殞命。王全斌大慟而去。一名伶人揀丟棄的樂器放在李存勗屍體上,點火焚屍。史稱興教門之變。李嗣源入洛陽杀尽叛臣,葬李存勗屍骨于雍陵,進廟號莊宗,李嗣源在汴州稱帝,是為後唐明宗。

李存勗稱帝即位之前,和后梁血战十餘年,大小百餘战,作战英勇异常。但打了天下,却不懂得治天下,宠幸伶人,重用宦官,又吝於銀錢,不抚恤士卒,三年後因兵变被杀,失败之速,亦是罕见。

北宋歐陽修寫《新五代史·伶官傳序》便是討論李存勗沉溺逸樂、寵信樂官而致亡國的史實,叹惜李存勗“方其盛也,举天下之豪杰复能与之争;及其衰也,数十伶人困之,而身死国灭,为天下笑。”,說明“憂勞可以興國,逸豫可以亡身”的歷史規律 。

《旧五代史》则称赞李存勗是“中兴之主”,是唐朝的合法继承者,但語鋒一轉,隨即批評他“忘櫛沐之艱難,徇色禽之荒樂”、“伶人亂政、靳吝貨財、大臣無罪以獲誅、眾口吞聲而避禍” 。

朱温评价李存勗说“生子当如李亚子,克用为不亡矣!至如吾儿,豚犬耳!”(生儿子就要生像李存勖这样的,李克用的大业不会灭亡了!至于说我的儿子,猪狗之辈而已!),中國共產黨中央委員會主席毛澤東也同意這個看法。


\subsubsection{同光}

\begin{longtable}{|>{\centering\scriptsize}m{2em}|>{\centering\scriptsize}m{1.3em}|>{\centering}m{8.8em}|}
  % \caption{秦王政}\
  \toprule
  \SimHei \normalsize 年数 & \SimHei \scriptsize 公元 & \SimHei 大事件 \tabularnewline
  % \midrule
  \endfirsthead
  \toprule
  \SimHei \normalsize 年数 & \SimHei \scriptsize 公元 & \SimHei 大事件 \tabularnewline
  \midrule
  \endhead
  \midrule
  元年 & 923 & \tabularnewline\hline
  二年 & 924 & \tabularnewline\hline
  三年 & 925 & \tabularnewline\hline
  四年 & 926 & \tabularnewline
  \bottomrule
\end{longtable}


%%% Local Variables:
%%% mode: latex
%%% TeX-engine: xetex
%%% TeX-master: "../../Main"
%%% End:

%% -*- coding: utf-8 -*-
%% Time-stamp: <Chen Wang: 2019-12-24 16:45:22>

\subsection{明宗\tiny(926-933)}

\subsubsection{生平}

唐明宗李亶(867年10月10日-933年12月15日),初名嗣源,小名邈佶烈。应州金城(今山西省应县)人,沙陀族,五代十国时期后唐第二位皇帝(926年6月3日-933年12月15日在位),在位8年。唐河东节度使李克用之养子,生父為李霓。

李嗣源初以騎射為河東節度使李克用效命,李克用則以李嗣源為養子,寵愛有加。克用死後,923年克用之子李存勗在魏博稱帝,國號唐,史稱後唐庄宗。同光元年二月,后唐潞州(今山西長治)、卫州(今河南汲縣)情势危殆,李嗣源以奇兵夜袭后梁东部重镇郓州(今山東東平),攻克。因功就任天平节度使。十月,李嗣源作为前锋进击大梁(今河南開封)。随即攻克。

926年魏博軍士兵皇甫暉乘人心不安聚眾作亂於貝州(今河北清河),斬殺主帥楊仁晸,立偏將趙在禮為帥,攻破鄴都(今河北臨漳),莊宗雖忌李嗣源,卻因乏人,只好命嗣源前往討伐,嗣源到魏博(今河北邯鄲)時,發生兵變,擁嗣源入城與叛軍會合,嗣源女婿石敬瑭勸告,莊宗不可能不追究此事,只好決心謀反。嗣源南下據汴京(今河南開封),西攻洛邑(今河南洛陽),史稱鄴都之變。此時伶人出身的禁軍將領郭從謙為了報仇,在洛邑乘機發動興教門之變,率兵攻入皇城,莊宗中流箭而死。群臣拥戴李嗣源为监国。李嗣源杀死宫中所有伶人,接着于926年四月丙午日称帝,年号“天成”。927年1月改名亶。

李嗣源即位后,革除莊宗时的弊政,励精图治,兴修水利,誅滅宦官,关心百姓疾苦,並撤銷不少有名無實的機關,后唐趋于强盛。

明宗是文盲,完全不识字,全国的上奏都得交由安重诲读给他听。天成三年,义武节度使王都叛乱,明宗削去其官爵,命王晏球讨伐。契丹遣兵救援王都,为后唐大败,回到本国境内的不过数十人,自此不敢轻易寇边。长兴元年(930年),明宗养子李从珂被安重诲陷害,但明宗本人没有听信谗言,亦没有追责安重诲。八月,明宗诛杀了朝中制造混乱,诬陷大臣的李行德、张俭等。他在位八年,兵革粗定,連年豐收,多次率领军队打败契丹。放觀整個五代只有他與後周兩位君王堪稱是有作為的皇帝。他在位期間的大事,是董璋、孟知祥割據兩川。此时安重诲逐渐失宠,明宗派女婿石敬瑭等讨伐两川时责令安重诲督运粮食去两川前线,安重诲未至前线即被凤翔节度使朱弘昭弹劾意欲夺取石敬瑭军权,于是又被召回,明宗又在途中突然改任安重诲为护国军节度使,后又将其杀死。石敬瑭讨伐两川最终无果,孟知祥又在董璋来伐时先败后胜消灭了董璋的东川势力,最终在明宗死后建立后蜀。

長興四年,李嗣源因聽聞定難節度使李彝超欲叛變,貿然任命李彝超為彰武留後,令安從進為定難節度使,李彝超抗命,明宗派兵進攻,竟久攻不下;李彝超上表謝罪,明宗只好授李彝超為檢校司徒,定難軍節度使,既而定難軍朝貢如初。明宗久年未出兵,一朝用兵卻無功而返,無論是兩川(今四川)或是夏州(今陝西榆林),軍中就有了對明宗不利的謠言,明宗懼,下令賞賜軍士,這個賞賜無任何理由,士卒於是日益驕縱。

933年明宗病危,数日不见臣下,原本被内定为继承人的秦王李从荣担心有变,引兵入宫。枢密使朱弘昭、冯赟以讨逆为名,派兵抵抗,將李從榮誅殺,后更杀死养在李嗣源身边的李从荣之子。李嗣源得知消息,悲痛过度,病重去世,庙号明宗,谥号圣德和武钦孝皇帝,葬于徽陵。由宋王李從厚繼位。

後唐明宗為人純質,寬仁愛人,在位時力除莊宗時的弊病、廣納建言、關心民生、嚴懲貪官、不邇聲色、不樂遊畋,鮮少發生戰事,百姓得以喘息;是五代十國數十位帝王中少數稱得上明主的,可是他也有許多不明智的地方:無法解決任圜與安重誨的矛盾,使得任圜被冤殺;無法解決兩川問題,使孟知祥反叛建立後蜀;聽信他人讒言,使安重誨夫婦於自宅被處死;久不立太子,使李從榮生疑並最終被殺。其後續位的後唐閔帝昏庸無能,周遭又是一些無用之輩,後唐很快也跟著滅亡了。


\subsubsection{天成}

\begin{longtable}{|>{\centering\scriptsize}m{2em}|>{\centering\scriptsize}m{1.3em}|>{\centering}m{8.8em}|}
  % \caption{秦王政}\
  \toprule
  \SimHei \normalsize 年数 & \SimHei \scriptsize 公元 & \SimHei 大事件 \tabularnewline
  % \midrule
  \endfirsthead
  \toprule
  \SimHei \normalsize 年数 & \SimHei \scriptsize 公元 & \SimHei 大事件 \tabularnewline
  \midrule
  \endhead
  \midrule
  元年 & 926 & \tabularnewline\hline
  二年 & 927 & \tabularnewline\hline
  三年 & 928 & \tabularnewline\hline
  四年 & 929 & \tabularnewline\hline
  五年 & 930 & \tabularnewline
  \bottomrule
\end{longtable}

\subsubsection{长兴}

\begin{longtable}{|>{\centering\scriptsize}m{2em}|>{\centering\scriptsize}m{1.3em}|>{\centering}m{8.8em}|}
  % \caption{秦王政}\
  \toprule
  \SimHei \normalsize 年数 & \SimHei \scriptsize 公元 & \SimHei 大事件 \tabularnewline
  % \midrule
  \endfirsthead
  \toprule
  \SimHei \normalsize 年数 & \SimHei \scriptsize 公元 & \SimHei 大事件 \tabularnewline
  \midrule
  \endhead
  \midrule
  元年 & 930 & \tabularnewline\hline
  二年 & 931 & \tabularnewline\hline
  三年 & 932 & \tabularnewline\hline
  四年 & 933 & \tabularnewline
  \bottomrule
\end{longtable}


%%% Local Variables:
%%% mode: latex
%%% TeX-engine: xetex
%%% TeX-master: "../../Main"
%%% End:

%% -*- coding: utf-8 -*-
%% Time-stamp: <Chen Wang: 2019-12-24 16:48:51>

\subsection{闵帝\tiny(933-934)}

\subsubsection{生平}

唐閔帝李從厚(914年-934年),小字菩薩奴,五代時期後唐皇帝,為後唐明宗李嗣源之子,母昭懿皇后夏氏,有一胞兄李从荣。李嗣源因李從厚與自己相像,特別喜歡他。

李從厚於李嗣源在位時,原被封為宋王。後唐長興四年(933年),李嗣源病重,本為繼承人的秦王李從榮誤以為李嗣源已死,為確保能夠繼位,遂帶兵入宮,事敗被殺。李嗣源不得已,召時任天雄節度使的李從厚回京。不久,李嗣源去世,李從厚繼位。次年(934年),改年號應順。

李從厚即帝位後,信任朱弘昭、馮贇等人,二人於應順元年(934年),調動各重要節度使,準備削藩,鳳翔節度使潞王李從珂恐懼,遂反,攻入京師洛陽,李從厚出逃魏州,途经卫州,遇到河东节度使石敬瑭。石敬瑭無意救之。李从厚的亲随不满石敬瑭,抽刀要杀石敬瑭,结果反被石敬瑭的侍卫杀死。石敬瑭的部將劉知遠尽杀闵帝亲随,把闵帝安置在卫州後,不久離去。皇太后下令降閔帝為鄂王。不久闵帝為潞王李從珂派人所殺。後晉高祖石敬瑭稱帝後,將他諡為閔皇帝。

李從厚個性仁慈,對兄弟很和睦,雖遭李從榮忌恨,卻能坦誠相待,所以當時才能逃過一劫。本來與李從珂也沒有過節,只因輕易地聽信周遭人的讒言,才會招來大禍。

\subsubsection{应顺}

\begin{longtable}{|>{\centering\scriptsize}m{2em}|>{\centering\scriptsize}m{1.3em}|>{\centering}m{8.8em}|}
  % \caption{秦王政}\
  \toprule
  \SimHei \normalsize 年数 & \SimHei \scriptsize 公元 & \SimHei 大事件 \tabularnewline
  % \midrule
  \endfirsthead
  \toprule
  \SimHei \normalsize 年数 & \SimHei \scriptsize 公元 & \SimHei 大事件 \tabularnewline
  \midrule
  \endhead
  \midrule
  元年 & 934 & \tabularnewline
  \bottomrule
\end{longtable}


%%% Local Variables:
%%% mode: latex
%%% TeX-engine: xetex
%%% TeX-master: "../../Main"
%%% End:

%% -*- coding: utf-8 -*-
%% Time-stamp: <Chen Wang: 2019-12-24 16:49:22>

\subsection{李从珂\tiny(934-937)}

\subsubsection{生平}

李從珂(885年2月11日-937年1月11日),鎮州(今河北正定)人,五代時期後唐皇帝,史稱後唐末帝或後唐廢帝,本姓王,小字二十三,因此又被叫做阿三。

李從珂十餘歲時,其母魏氏被當時仍是將領的後唐明宗李嗣源所擄,李從珂不久就被李嗣源改名並收為養子。長大後身形雄偉健壯,又驍勇善戰,常隨李嗣源南征北討,頗得其喜愛。莊宗也曾赞道:“阿三不惟与我同齿,敢战亦相类。”

同光三年(925年),李嗣源因家在太原,上表请求让时任卫州刺史的李从珂为北京(太原)内牙马步都指挥使,以便相聚,却导致莊宗大怒,李从珂被黜为突骑指挥使,率数百人戍石门镇。次年李嗣源平定邺都之乱时被迫造反,李从珂率军与之会合,助其夺位。

李嗣源即帝位後,李從珂曾任河中節度使之職,然因與權臣樞密使安重誨之前有過節,在長興元年(930年),被安重誨設計解除軍權,回京師洛陽居住。次年(931年),安重誨失勢,李從珂再受重用,被任命為左衛大將軍、西京(長安)留守。長興三年(932年),被改命為鳳翔節度使。長興四年(933年),封潞王。

後唐應順元年(934年),閔帝李從厚聽信大臣的建議,調動各重要節度使之職,準備削弱藩鎮的實力,李從珂恐懼,遂反。李從厚命王思同率大軍討伐,王思同围攻凤翔城(今陝西鳳翔)。凤翔城墙低,护城河窄浅,根本无法固守。眼看鳳翔即將陷落,未料討伐軍將兵驕橫,貪圖賞賜,李從珂抓住這點誘使討伐軍叛變,反敗為勝,不久以摧枯拉朽之勢攻入京師洛陽,即帝位,改元清泰,並派人將逃亡的李從厚殺害。

李嗣源之婿石敬瑭時任重鎮河東節度使之職,李從珂與他二人當初在李嗣源手下皆以勇力過人著稱,彼此存有競爭之心。因此李從珂即位後,對石敬瑭愈發猜忌,而石敬瑭亦有謀反之意。

清泰三年(936年),石敬瑭以調鎮他處試探,而李從珂果真將石敬瑭改任天平節度使,石敬瑭因此叛變,同時向契丹乞援。李從珂命各鎮聯合討伐,不料因聯軍各懷鬼胎,致大敗於團柏谷,石敬瑭與契丹大軍得以順利南下進逼京師洛陽,李從珂無計可施,於閏十一月二十六日(陽曆為937年1月11日)自焚而死。死後無諡號及廟號,史家稱之為末帝或廢帝。传国玉玺亦在此时遗失不知所踪。

李從珂是一個矛盾的人,時運順他之時,他竟敢潛伏到敵軍陣營內,殺死敵軍並且砍下對方望桿扛回去;時運離他而去時,整天消沉著在宮中飲酒哭泣,最後自焚,並不是大臣沒有給他好的意見,而是他自己優柔寡斷,無法及時下決定,最終誤國。


\subsubsection{清泰}

\begin{longtable}{|>{\centering\scriptsize}m{2em}|>{\centering\scriptsize}m{1.3em}|>{\centering}m{8.8em}|}
  % \caption{秦王政}\
  \toprule
  \SimHei \normalsize 年数 & \SimHei \scriptsize 公元 & \SimHei 大事件 \tabularnewline
  % \midrule
  \endfirsthead
  \toprule
  \SimHei \normalsize 年数 & \SimHei \scriptsize 公元 & \SimHei 大事件 \tabularnewline
  \midrule
  \endhead
  \midrule
  元年 & 934 & \tabularnewline\hline
  二年 & 935 & \tabularnewline\hline
  三年 & 936 & \tabularnewline
  \bottomrule
\end{longtable}


%%% Local Variables:
%%% mode: latex
%%% TeX-engine: xetex
%%% TeX-master: "../../Main"
%%% End:


%%% Local Variables:
%%% mode: latex
%%% TeX-engine: xetex
%%% TeX-master: "../../Main"
%%% End:

%% -*- coding: utf-8 -*-
%% Time-stamp: <Chen Wang: 2019-12-24 16:50:13>


\section{后晋\tiny(936-947)}

\subsection{简介}

后晋(936年-947年)是中国历史上五代十国时期的一个朝代,从后晋高祖石敬瑭936年灭后唐开国到契丹947年灭后晋一共经历了两个皇帝,總計12年。为与司马氏的晋朝相区别,又别称为石晋。

后晋的开国皇帝沙陀人石敬瑭是后唐开国的功臣,他曾经多次在危难中救护后唐开国皇帝李存勖和明宗李嗣源。李存勖和李嗣源都十分器重他,李嗣源甚至将自己的女儿嫁给了他。后唐建立后石敬瑭任河东节度使(今山西),石敬瑭成为当地军民最高指挥官。石敬瑭在河东政绩很高,而且生活清廉,很受当地人的欢迎。但李嗣源死后后唐内部互相倾轧,石敬瑭受李从珂的猜忌,因此渐渐产生了反唐的计划。当李从珂决定将石敬瑭调离河东时,石敬瑭决定反唐。

石敬瑭在河东的兵力不足以抵挡后唐的进攻,因此石敬瑭决定求救于契丹。作为条件,他同意割让燕雲十六州(此十六个州,属今河北和山西)给契丹,并对辽太宗耶律德光称「儿」。在这种情况下,耶律德光决定帮助石敬瑭。

契丹和石敬瑭的联军打败了后唐,攻入后唐首都洛阳。后唐灭亡,石敬瑭称帝,国号晋,史称后晋。依據五行相生的順序,後唐的「土」德之後為「金」德,因此後晉以「金」為王朝德運。後晉移都开封,并按约将16州让给契丹。这16州是:幽(今北京市)、蓟(今天津蓟县)、瀛(今河北河间)、莫(今河北任丘)、涿(今河北涿州)、檀(今北京密云)、顺(今北京顺义)、新(今河北涿鹿)、妫(原属河北怀来,今为官厅水库库区)、儒(今北京延庆)、武(今河北宣化)、蔚(今河北蔚县)、云(今山西大同)、应(今山西应县)、寰(今山西朔县东马邑镇)、朔(今山西朔县),并向契丹称儿皇帝,契丹封其为“晋帝”。

石敬瑭割让燕雲十六州为辽国和金国后来对宋朝长江以北地区的威胁打开了门户。

后晋建国后一直处于动盪中,石敬瑭割地称儿的做法受到许多人的反对,包括他自己过去的亲信。石敬瑭本人到死没有改变依附契丹的政策,但国家多處发生叛乱,石敬瑭的两个儿子在这些叛乱中被杀,種種事情给他带来了极大的打击。为了对付叛乱,石敬瑭加重嚴刑峻法,同时非常猜忌自己的手下。

石敬瑭死时,立石重贵为继承人。石重贵是他的侄子,因为在战场上立战功获得石敬瑭的赏识。但石重贵仅是一勇之夫,根本无法在國家面對困境下應付各種政治问题。石重贵登基后决定渐渐脱离对契丹的依附,他首先宣称对耶律德光称孙,但不称臣。

契丹对此当然不能坐视。944年契丹伐晋,双方在澶州(今河南濮阳南)交战,互有胜负。945年契丹再次南征,石重贵亲征,再次战败契丹。947年,契丹第三次南下,后晋重臣杜重威、李守貞和張彥澤率軍向契丹投降,后晋丧失主力,契丹派張彥澤率先部入開封。石重贵被迫投降,全家被俘虏到契丹。后晋灭亡。

后晋亡后,河东节度使北平王刘知远在太原称帝,建立后汉。

%% -*- coding: utf-8 -*-
%% Time-stamp: <Chen Wang: 2021-11-01 15:21:45>

\subsection{高祖石敬瑭\tiny(936-942)}

\subsubsection{生平}

晉高祖石敬瑭(892年3月30日-942年7月28日),五代十国時期的后晋开国皇帝(936年11月28日–942年7月28日在位)。庙号高祖,谥号圣文章武明德孝皇帝。他把燕雲十六州割讓给契丹,使中原地區丧失了北方屏障,並向辽太宗自称儿皇帝。

《新五代史》指石敬瑭的祖先为中亚人,从沙陀移居太原,但發挖出土的石重貴墓誌銘則指他是後趙石勒之後裔。

父石紹雍,母何氏。石紹雍从李克用父子征战,官至洺州刺史。

石敬瑭自少为李嗣源(日後的唐明宗)赏识,为其亲兵将领,被招为女婿。後唐莊宗同光四年(926年),邺都之变,石敬瑭力劝李嗣源入汴京,转攻洛阳。李嗣源即位后,石敬瑭历任保义、宣武、河東诸镇节度使。

934年,閔帝李從厚徙石敬瑭為成德節度使。閔帝討伐潞王李从珂失敗,逃到衛州向石敬瑭求援,可是石敬瑭的部下把閔帝隨從殺盡,石敬瑭把閔帝安置在衛州,最後閔帝被李从珂派人殺死。

末帝李从珂继位后,任石敬瑭為河東節度使,後來開始對石敬瑭起疑,石敬瑭也暗中謀自保。石敬瑭以多病為理由,上表請求朝廷調他往其它藩鎮,借此試探朝廷對他的態度。末帝在清泰三年(936年)五月改授石敬瑭為天平節度使,並降旨催促赴任。石敬瑭懷疑末帝對他起疑心,便举兵叛变。後唐派兵討伐,石敬瑭被圍,向契丹求援。九月契丹軍南下,擊敗唐軍。

石敬瑭的岳父是唐明宗李嗣源。李嗣源的义父是李克用。李克用曾和辽太宗耶律阿保机结为兄弟,故石敬瑭按辈份称比他小10歲的耶律阿宝机的儿子耶律德光为亚父,并在国书中稱自己為“兒皇帝”,耶律德光为“父皇帝”。

石敬瑭在十一月受契丹冊封為大晉皇帝,然後向洛陽進軍,後唐末帝在閏十一月(937年1月)自焚,後唐遂亡。

石敬瑭滅後唐後,按约定将燕雲十六州献给契丹,其结果使中原地區丧失了北方屏障。另外晋国向辽国每岁奉绢三十万匹。

石敬瑭在位期間,各地將領魏博节度使范延光、西京留守张从宾、成德节度使安重荣、山南东道节度使安从进等引发的叛變事件不斷,他的兒子石重信和石重乂亦遭叛軍殺害。后因成德節度使安重榮及河东节度使劉知遠先後接受吐谷浑部族投降,石敬瑭屡遭契丹责问,乃忧愤而死。

《舊五代史》稱讚石敬瑭的謙虛、節儉;「旰食宵衣,禮賢從諫」、「以絁為衣,以麻為履」,後又責怪他向契丹乞兵,反而使得百姓陷入連年戰火;「強鄰來援,契丹自茲而孔熾,黔黎由是以罹殃。」「兵連禍結、舉族為俘」,這無疑是決鯨海以救焚,結果自己溺死了、飲鴆漿而止渴,結果毒死自己。《舊五代史》最後為他惋惜,如果他是靠自己的力量取得帝位,以他的節儉、謙卑、公正的態度,即使功德不超過前人,亦可謂仁慈恭儉之主。


\subsubsection{天福}

\begin{longtable}{|>{\centering\scriptsize}m{2em}|>{\centering\scriptsize}m{1.3em}|>{\centering}m{8.8em}|}
  % \caption{秦王政}\
  \toprule
  \SimHei \normalsize 年数 & \SimHei \scriptsize 公元 & \SimHei 大事件 \tabularnewline
  % \midrule
  \endfirsthead
  \toprule
  \SimHei \normalsize 年数 & \SimHei \scriptsize 公元 & \SimHei 大事件 \tabularnewline
  \midrule
  \endhead
  \midrule
  元年 & 936 & \tabularnewline\hline
  二年 & 937 & \tabularnewline\hline
  三年 & 938 & \tabularnewline\hline
  四年 & 939 & \tabularnewline\hline
  五年 & 940 & \tabularnewline\hline
  六年 & 941 & \tabularnewline\hline
  七年 & 942 & \tabularnewline\hline
  八年 & 943 & \tabularnewline\hline
  九年 & 944 & \tabularnewline
  \bottomrule
\end{longtable}


%%% Local Variables:
%%% mode: latex
%%% TeX-engine: xetex
%%% TeX-master: "../../Main"
%%% End:

%% -*- coding: utf-8 -*-
%% Time-stamp: <Chen Wang: 2021-11-01 15:21:55>

\subsection{出帝石重贵\tiny(942-946)}

\subsubsection{生平}

晋出帝石重贵(914年-974年),又稱少帝,942年-946年在位。天福七年(942年),后晋高祖石敬瑭死,重贵繼位,沿用高祖天福年号,天福九年(944年)七月改元开运。石重贵不肯向契丹称臣,契丹攻后晋,開運三年十二月(947年1月)佔開封,石重贵投降,后晋亡。

石重貴出生於太原,是石敬瑭的姪兒,父親是石敬瑭兄敬儒,母安氏。敬儒早逝,敬瑭以其子重貴為子。石敬瑭雖有六子,但有五子早死,而餘下的石重睿年幼,所以石敬瑭便選擇重貴作為繼承人。石敬瑭起兵反後唐時,以石重貴為金紫光祿大夫,行太原尹、北京留守,知河東節度事。

天福二年(937年)九月,升為左金吾衛上將軍。天福三年冬,為開封尹,封鄭王,加太尉,同中書門下平章事。天福六年,改為廣晉尹,徙封齊王。天福七年六月,石敬瑭去世,石重貴繼位。

天福八年間,二十七個州郡發生蝗災,數十萬人餓死。次年飢荒仍然嚴重,在隴州有五萬六千人餓死。

石重貴依從重臣景延廣之言,放棄高祖時期對契丹恭順的政策,对耶律德光称孙但不称臣,兩国關係惡化。天福九年(944年)正月契丹軍開始入侵,三年間雙方互有勝負。開運三年十二月將領杜重威、李守贞、張彥澤率軍向契丹軍投降,契丹派張彥澤率領先頭部隊入開封,石重貴投降,後晉滅亡。

遼太宗耶律德光在947年正月把石重貴降為光祿大夫、檢校太尉,封「負義侯」,後晉正式滅亡。石重貴被安置在黃龍府,後來遷往建州。《舊五代史》引范質《晉朝陷蕃記》稱石重貴「凡十八年而卒」,即在北宋乾德二年(964年)去世。石重貴墓誌銘(現藏於遼寧省博物館)稱他在遼保寧六年(974年)六月十八日病逝。


\subsubsection{开运}

\begin{longtable}{|>{\centering\scriptsize}m{2em}|>{\centering\scriptsize}m{1.3em}|>{\centering}m{8.8em}|}
  % \caption{秦王政}\
  \toprule
  \SimHei \normalsize 年数 & \SimHei \scriptsize 公元 & \SimHei 大事件 \tabularnewline
  % \midrule
  \endfirsthead
  \toprule
  \SimHei \normalsize 年数 & \SimHei \scriptsize 公元 & \SimHei 大事件 \tabularnewline
  \midrule
  \endhead
  \midrule
  元年 & 944 & \tabularnewline\hline
  二年 & 945 & \tabularnewline\hline
  三年 & 946 & \tabularnewline
  \bottomrule
\end{longtable}


%%% Local Variables:
%%% mode: latex
%%% TeX-engine: xetex
%%% TeX-master: "../../Main"
%%% End:



%%% Local Variables:
%%% mode: latex
%%% TeX-engine: xetex
%%% TeX-master: "../../Main"
%%% End:

%% -*- coding: utf-8 -*-
%% Time-stamp: <Chen Wang: 2019-12-24 16:54:27>


\section{后汉\tiny(947-951)}

\subsection{简介}

後漢(947年-951年)是五代十國的第四個朝代,同时也是最后一个由沙陀人建立的中原王朝。亦称刘汉。後漢承自後晉,根據五行相生的順序,後晉的「金」德之後是「水」德,因此後漢以「水」為王朝德運。

後漢政權存在只有兩朝共四年,是五代十國中历时最短的政權,也被一些学者(例如钱文忠)认为是中国历史上最短命的“中央政權”。主要原因是劉知遠不懂治國,朝中爭鬥激烈、動則族誅,最後其子隱帝懷疑大臣郭威想要造反,派郭崇前往魏州殺死郭威,導致郭威起兵討伐隱帝,隱帝為部屬郭允明所殺。郭威進入都城開封后,本想立劉知遠侄子劉赟為帝,後來反悔,殺死劉赟,自己稱帝,建立後周。后汉被后周郭威所篡后,後漢高祖劉知遠的弟弟、鎮守晉陽的河东节度使刘崇在太原继位称帝,自称延续汉祚,然政权之疆域及地位均已改变,史家一般作为新政权或殘餘政權定位,列為「十國」之一,称为北汉。

五代史記述:「五代亂世,本無刑章,視人命如草芥,動以族誅為事。是族誅之法,凡罪人之父兄妻妾子孫並女之出嫁者,無一得免,非法之刑,於茲極矣!而尤莫如漢代之濫。然不問罪之輕重,理之是非,但云有犯,即處極刑。枉濫之家,莫敢上訴。軍吏因之為奸,嫁禍脅人,不可勝數。而此毒痛四海,殃及萬方。後漢劉氏父子二帝,享國不及四年。杨邠、史弘肇、蘇逢吉、劉銖等諸人亦皆被橫禍,無一善終者。此固天道之報施昭然,而民之生於是時,不知如何措手足也。」

统治范围包括今河南、山东、山西、河北南部、湖北北部、陕西北部、安徽北部。

開國君主劉知遠為世居太原的沙陀人,原為五代後晉河東節度使。947年,乘契丹陷開封而於太原稱帝,国号漢,史家稱“後漢”,以别于汉朝。自称为东汉显宗八子淮阳王刘昞之后,继承汉朝,宗庙内祭祀刘邦和刘秀。后攻克中原,定都汴京(今开封)。948年刘知远二子劉承祐嗣位,即后汉隐帝。950年李守贞等藩镇发生叛乱,汉隐帝命郭威平之,但汉隐帝猜忌郭威,欲杀之,郭威进而反叛。同年十一月二十一日(951年1月1日)刘承祐被杀,后汉亡。


%% -*- coding: utf-8 -*-
%% Time-stamp: <Chen Wang: 2019-12-24 16:55:29>

\subsection{高祖\tiny(947-948)}

\subsubsection{生平}

汉高祖劉知遠(895年-948年),晋阳(今山西太原)人,五代時期後漢開國皇帝,沙陀族,即帝位後改名劉暠,947年—948年在位,死後諡睿文聖武昭肅孝皇帝。劉知遠在太原出生,祖先是沙陀人,父名琠,冒姓刘氏。

劉知遠最初是李嗣源(後來的後唐明宗)的部下,鄴城之變,李嗣源登基後,劉知遠在石敬瑭帳下任牙門都校。934年,閔帝出逃到衛州時,與石敬瑭議事未決,後唐閔帝隨從欲動武,劉知遠把閔帝隨從殺盡,石敬瑭便捨閔帝而去。

936年,後唐末帝下詔調河東節度使石敬瑭為天平節度使。劉知遠勸石敬瑭起兵。石敬瑭便舉兵叛唐。劉知遠不贊成石敬瑭以割讓燕雲十六州向契丹借兵。石敬瑭不從。劉知遠任馬步軍都指揮使,同年任保義軍節度使。石敬瑭滅後唐,建後晉,937年任劉知遠為忠武軍節度使,941年任大名府留守兼河東節度使。出帝繼位後,在943年升劉知遠為中書令,944年任幽州道行营招讨使,封太原王,次年改封北平王。

出帝开运四年(947年),契丹滅后晋。河东行军司马张彦威等人以中原无主為由,勸劉知遠称帝,劉知遠在推搪一番後便在太原称帝,沿用后晋高祖年号天福,称天福十二年,同年六月入汴京,自称为东汉明帝八子淮阳王劉昞之后,改國號為「漢」。天福十三年(948年)正月改元乾祐,劉知遠又改名為劉暠,同月病逝。

劉知遠沒有治世的能力,為人也不守信,對於一些叛將,他先誘降而後殺之,例如張璉,他對民間百姓的刑罰也是以嚴厲聞名,他曾規定民間如有牛死,由官府收納牛皮,犯令者死;後又規定“民有犯鹽、礬、酒曲者,無多少皆抵死”,在他統治的時代,百姓苦不堪言;劉知遠在收復幽州時曾答應歸降契丹的軍民無罪,爾後卻殺光他們;又曾殺害吐谷渾部落百姓四百口,顯示出他殘忍、不講信用的一面;晉出帝受難之時,他居然隱兵不發,代表他早有異志,企圖稱帝。

歷史對於劉知遠多是批評居多,《舊五代史》即說劉知遠“乘虛而取神器,因亂而有帝圖”;“雖有應運之名,而未睹為君之德”。


\subsubsection{天福}

\begin{longtable}{|>{\centering\scriptsize}m{2em}|>{\centering\scriptsize}m{1.3em}|>{\centering}m{8.8em}|}
  % \caption{秦王政}\
  \toprule
  \SimHei \normalsize 年数 & \SimHei \scriptsize 公元 & \SimHei 大事件 \tabularnewline
  % \midrule
  \endfirsthead
  \toprule
  \SimHei \normalsize 年数 & \SimHei \scriptsize 公元 & \SimHei 大事件 \tabularnewline
  \midrule
  \endhead
  \midrule
  元年 & 947 & \tabularnewline
  \bottomrule
\end{longtable}

\subsubsection{乾祐}

\begin{longtable}{|>{\centering\scriptsize}m{2em}|>{\centering\scriptsize}m{1.3em}|>{\centering}m{8.8em}|}
  % \caption{秦王政}\
  \toprule
  \SimHei \normalsize 年数 & \SimHei \scriptsize 公元 & \SimHei 大事件 \tabularnewline
  % \midrule
  \endfirsthead
  \toprule
  \SimHei \normalsize 年数 & \SimHei \scriptsize 公元 & \SimHei 大事件 \tabularnewline
  \midrule
  \endhead
  \midrule
  元年 & 948 & \tabularnewline\hline
  \bottomrule
\end{longtable}


%%% Local Variables:
%%% mode: latex
%%% TeX-engine: xetex
%%% TeX-master: "../../Main"
%%% End:

%% -*- coding: utf-8 -*-
%% Time-stamp: <Chen Wang: 2019-12-24 17:03:41>

\subsection{隐帝\tiny(948-950)}

\subsubsection{生平}

汉隐帝劉承祐(931年-951年),並州晉陽(今山西太原)人,沙陀族,後漢第二個皇帝,948年—951年在位。汉高祖乾祐元年(948年)正月,劉知遠死后,承祐即位,年十八歲。沿用后汉高祖年号乾祐。

堂侄刘继文墓志载其为“少帝承翰”。

當時國事完全取決於重臣楊邠、郭威、史弘肇、王章之手。楊邠總機政,郭威主征伐,史弘肇典宿衛,王章掌財賦,權臣相爭,承祐寢食不安。他派郭威鎮压起义,平河中節度使李守貞之亂。乾祐三年(950年)初夏,契丹寇河北,命郭威鎮守鄴都(今河北大名)。

然承祐性多猜忌,私下與茶酒使郭允明計畫殺大臣,十一月將宰相楊邠、史弘肇、王章等砍死在東廂之下,又将郭威一家全部斩杀。復下詔誅殺正在鄴都留守的郭威,召泰寧節度使慕容彥超等急速入京。但郭威舉兵南下,十六日,抵澶州,十八日駐滑州。二十日,郭威至封丘(今属河南),擊敗慕容彦超於刘子陂(今河南封丘南)。二十一日(951年1月1日)攻入开封。刘承祐意欲出城亲征,李太后劝他不要莽撞,刘承祐不听。承祐出戰兵敗,連同苏逢吉、聂文进和郭允明等人向西北奔逃,二十二日(951年1月2日),至趙村,為郭允明所殺。郭威迎立劉崇子劉贇。廣順元年(951年)正月,李太后將傳國璽交給郭威,郭威在崇元殿登極,改年號廣順,是為後周代漢。

明末學者王夫之於《讀通鑑論》中直接以姓名稱呼劉承祐,寓意其不足為中國之主,又指劉承祐以為用一紙檄書可以殺盡權臣是以國家大事為遊戲,愚蠢至極;然而,王夫之亦讚揚他誅殺史弘肇、王章、楊邠等人之舉導致「風氣以移」、「內難不生」、「天下漸寧」,使代之而起的後周和北宋得以與民休息,進而統一中國。

\subsubsection{乾祐}

\begin{longtable}{|>{\centering\scriptsize}m{2em}|>{\centering\scriptsize}m{1.3em}|>{\centering}m{8.8em}|}
  % \caption{秦王政}\
  \toprule
  \SimHei \normalsize 年数 & \SimHei \scriptsize 公元 & \SimHei 大事件 \tabularnewline
  % \midrule
  \endfirsthead
  \toprule
  \SimHei \normalsize 年数 & \SimHei \scriptsize 公元 & \SimHei 大事件 \tabularnewline
  \midrule
  \endhead
  \midrule
  元年 & 948 & \tabularnewline\hline
  二年 & 949 & \tabularnewline\hline
  三年 & 950 & \tabularnewline
  \bottomrule
\end{longtable}

\subsection{湘阴公生平}

劉贇(10世纪?-951年),本名刘承赟,五代時期後漢宗室,生父劉崇,養父劉知遠。權臣郭威假意立之為漢帝,卻在赴京途中貶之為湘陰公,後被郭威所殺。

其生父劉崇是後漢高祖劉知遠的弟弟,劉贇十分受伯父劉知遠喜愛,因此被過繼為劉知遠的養子。

後漢隱帝乾祐三年(950年),任武寧節度使(駐守今江苏省徐州市)。

漢隱帝為了親自掌權,殺死樞密使郭威一家,郭威叛變,漢隱帝被亂兵所殺。郭威欲改立其弟刘承勋,承勋卧病不能胜任,于是郭威迎劉贇至首都開封府(今河南省开封市)即位。未料不久郭威就假借黃旗加身,在士兵的擁戴之下自立為君,而劉贇當時正行至中途,處於尷尬的困境。刘崇以为儿子将被拥立为帝,没有听从手下的建议,按兵不动。劉贇在其衛兵投降郭威後遭軟禁,并被李太后下詔,废為湘陰公。數日後,郭威正式稱帝,建立後周,時為951年正月。不久,劉贇就被郭威派人殺害。其后,其父刘崇在太原登基,建立北汉。

%%% Local Variables:
%%% mode: latex
%%% TeX-engine: xetex
%%% TeX-master: "../../Main"
%%% End:


%%% Local Variables:
%%% mode: latex
%%% TeX-engine: xetex
%%% TeX-master: "../../Main"
%%% End:

%% -*- coding: utf-8 -*-
%% Time-stamp: <Chen Wang: 2019-12-24 17:04:43>


\section{后周\tiny(951-960)}

\subsection{简介}

后周(951年-960年)是中国历史上五代十国时期的最后一个朝代,它从951年正月后周太祖郭威灭后汉开国到960年北宋太祖赵匡胤陈桥兵变被取代共经历了三个皇帝,9年。后周的首都是开封。

统治地区包括今河南、山东、山西南部、河北中南部、陕西中部、甘肃东部、湖北北部、以及安徽、江苏的长江以北地区。

郭威自称为周朝虢叔后裔,因此以「周」为国号,史称「后周」,以别于其他以周为国号的政权,又以郭威之姓,别称「郭周」。

后周的开国皇帝郭威是后汉的开国功臣,受后汉高祖刘知远重任。刘知远临死时郭威是他指定的顾命大臣之一,他奉后汉隐帝刘承祐命,平定多次反汉叛变,同时他又能体恤手下,因此深受军队的热爱。刘承祐感到自己受顾命大臣的控制太多,因此开始杀这些大臣。郭威当时领兵在外,闻讯后以清君侧的名义起兵。刘承祐为此将郭威在开封的所有亲属杀害。郭威在仅仅数日内就进入开封。刘承祐死于非命。郭威的军队在开封大掠。郭威首先名义上迎在徐州的刘赟做新皇帝,自己却以攻契丹为名北上,同时他却派部下将刘赟在路上杀死,然后又让自己的士兵拥护自己做皇帝,做出迫不得已的样子。就在这种情况下他依然首先以“监国”为名上任,一个月后才正式持皇帝名。依據五行相生的順序,後漢的「水」德之後為「木」德,因此後周以「木」為王朝德運。

郭威登基后著手进行一系列的改革。首先他减轻和免除了许多徭役,同时也整顿军纪和管理机构内部的腐败和贿赂。

由于刘承祐将郭威在开封的所有亲属杀害,其中包括他兩个儿子与柴荣的三个儿子。郭威死后由其养子(本身是其内侄)柴荣继位,是為后周世宗。柴荣继续郭威的政策,使得后周所控制的地区的经济得到了很大的发展,同时也使得后周的军事得到了强大的发展。

后周的两位皇帝也是中国历史上少有的非常节俭的皇帝,比如郭威死后陵前仅立石碑一块,其陵寢本身也非常简单,连守陵的宦官都没有。

柴荣继位后不久北汉就联合辽朝乘机攻打後周,打算乘后周内部未稳打击其力量,柴荣决定亲征抵禦进攻。他在高平之战中亲临战场,在战役开初不利,己方右翼溃退的情况下扭转战势,击溃北汉。战后后周军队乘胜追击,一直攻到太原。

此后柴荣开始南征,从955年到958年三次亲征南唐,迫使南唐取消皇帝称号,并割讓几乎所有长江以北的地区予後周。

959年柴荣在解除后顾之忧之后再次北上攻辽,在两个月內几乎攻到幽州,但就在此时他突然患病,不得不中止北伐。柴荣此后不久病逝。

后周在这八年內基本上统一了长江以北的中原地区,向北收复了许多被后晋让给契丹的地区。其統治地区恢复和发展了经济生产,為日后北宋统一中国打下了基础。

柴荣死后,其七岁的儿子柴宗训登基。殿前都点检赵匡胤谎稱辽国和北汉进犯,借口率兵到陈桥驿发动陈桥兵变,夺取后周帝位建立北宋。


%% -*- coding: utf-8 -*-
%% Time-stamp: <Chen Wang: 2019-12-24 17:05:56>

\subsection{太祖\tiny(951-954)}

\subsubsection{生平}

周太祖郭威(904年9月10日-954年2月21日),邢州堯山(今天河北省隆堯),汉族,字文仲,小名“郭雀兒”。五代時期後周開國皇帝(951年—954年)。原為五代後漢的樞密使,卻因隐帝疑忌之下,全家被殺。怒而起兵,隐帝死於亂軍之中,郭威不久發動黃旗加身的兵變,建立後周。

《舊五代史》說,有記載其本名為常威,隨母改嫁入郭家,改用郭姓。《新五代史》的紀錄卻剛好相反,曰其母本來是郭姓之妻,後來改嫁常氏。

唐天祐元年七月二十八日(904年9月10日)郭威生於堯山,父為後晉時的順州刺史郭简,母王氏。或說本姓常,幼时随母亲改嫁郭简,故改姓郭。3歲時徙家太原,不久郭简被殺,郭威成為孤兒,由姨母韓氏撫養。他身材魁梧,習武好鬥。其時李繼韜在潞州招募兵勇,郭威前去投軍,得到李繼韜的賞識。郭威在鬧市與一名欺壓市場的屠戶爭執,醉酒而殺之,原應斬首,李继韬怜其才勇,暗中将其释放,而後又召之,任為幕僚。

947年契丹滅後晉,沙陀人劉知遠起兵太原,建國後漢,郭威为邺都(今河北大名县)留守。劉知遠称帝不到一年即死去,其子劉承祐继位,拜郭威為樞密副使。乾祐元年(948年)三月河中(今山西永济)李守贞、永兴(今陕西西安)赵思绾及凤翔(今陕西凤翔)王景崇相继反汉,郭威相繼平定亂事,李守贞自焚,赵思绾投降,王景崇自焚。

乾祐三年(950年)四月,隐帝疑忌大臣,乘郭威在外時下诏將开封城内郭威(当时郭威已有成年的儿子)、柴荣和王峻的全家屠殺殆盡。

清趙翼謂之:“五代亂世,本無刑章,視人命如草芥,動以族誅為事”。 枢密使院吏魏仁浦劝郭威先发制人。同年十一月,郭威发动兵变,隱帝死於亂軍之中。郭威假意迎宗室劉贇為帝,又派部下在路上刺殺劉贇,再演出一番戲碼,讓士兵擁護自己稱帝,建立后周,建都汴京(今河南省開封市),改元广顺。他廣招人才、励精图治,得魏仁浦、李穀、王溥、范質等輔臣。广顺三年(953年),封義子柴榮為晉王。

广顺四年(954年),周太祖郭威去世,享年50歲。因亲生儿子全都被刘承祐杀害,妻侄(外甥)柴榮繼位。


\subsubsection{广顺}

\begin{longtable}{|>{\centering\scriptsize}m{2em}|>{\centering\scriptsize}m{1.3em}|>{\centering}m{8.8em}|}
  % \caption{秦王政}\
  \toprule
  \SimHei \normalsize 年数 & \SimHei \scriptsize 公元 & \SimHei 大事件 \tabularnewline
  % \midrule
  \endfirsthead
  \toprule
  \SimHei \normalsize 年数 & \SimHei \scriptsize 公元 & \SimHei 大事件 \tabularnewline
  \midrule
  \endhead
  \midrule
  元年 & 951 & \tabularnewline\hline
  二年 & 952 & \tabularnewline\hline
  三年 & 953 & \tabularnewline\hline
  四年 & 954 & \tabularnewline
  \bottomrule
\end{longtable}

\subsubsection{显德}

\begin{longtable}{|>{\centering\scriptsize}m{2em}|>{\centering\scriptsize}m{1.3em}|>{\centering}m{8.8em}|}
  % \caption{秦王政}\
  \toprule
  \SimHei \normalsize 年数 & \SimHei \scriptsize 公元 & \SimHei 大事件 \tabularnewline
  % \midrule
  \endfirsthead
  \toprule
  \SimHei \normalsize 年数 & \SimHei \scriptsize 公元 & \SimHei 大事件 \tabularnewline
  \midrule
  \endhead
  \midrule
  元年 & 954 & \tabularnewline
  \bottomrule
\end{longtable}


%%% Local Variables:
%%% mode: latex
%%% TeX-engine: xetex
%%% TeX-master: "../../Main"
%%% End:

%% -*- coding: utf-8 -*-
%% Time-stamp: <Chen Wang: 2021-11-01 15:22:29>

\subsection{世宗柴榮\tiny(954-959)}

\subsubsection{生平}

周世宗柴榮(921年10月27日-959年7月27日),五代時期後周皇帝,於954年2月26日-959年7月27日在位,在位6年。邢州堯山柴家莊(今河北省邢台市隆堯縣)人,是周太祖郭威的養子(柴榮本身是郭威正室柴皇后的侄子),是中國少数由外戚继承宗室的皇帝,廟號世宗,諡號睿武孝文皇帝。根據史書記錄,在此時期在政治、軍事、經濟上都有建樹,號稱英主,他初步奠定了後來北宋的勢力。

父柴守禮,祖父柴翁是當地望族,柴榮年輕時曾隨商人頡跌氏在江陵販賣茶葉,對社會積弊有所體驗。史載其「器貌英奇,善騎射,略通書、史、黃老,性沉重寡言」。

广顺元年(951年),周太祖郭威即位,柴荣授澶州节度使、检校太保,封太原郡侯。史载其境“为政清肃,盗不犯境”。二年,检校太傅为相。又一年,封晋王。

显德元年(954年)正月,判内外兵马事,总揽兵权。同月太祖崩,即帝位。

柴榮即位後,立刻下令招撫流亡,減少賦稅,恢復中原經濟。當時,北方及中原經過了一系列的戰爭,百姓痛苦不堪,柴榮的舉動正好使中原開始復甦,他整頓吏治,使後周政治清明,百姓富庶,經濟開始繁榮。

顯德二年(955年)推行顯德毀佛,以佛寺銅材鑄行「周元通寶」,錢質與鑄量均居五代之冠。因为其毀佛行為,後周世宗被列入毀佛的「三武(北魏太武帝、北周武帝和唐武宗)一宗」。司馬光評述周世宗「毀佛」:「不愛己身而愛民,不以無益廢有益,周世宗算得是仁愛明理之人。」

柴榮對內進行改革,對外則積極開拓疆土。顯德元年(954年)二月,北漢主劉崇乘其新立,勾結遼兵4萬攻後周,柴榮力排馮道勸阻,率軍迎戰,於高平(今山西)南大破北漢軍,穩定政局。戰後整軍練卒,裁汰冗弱,於是軍威大振。顯德二年詔令群臣獻《為君難為臣不易論》、《平邊策》,確定王樸提出的「先南後北」的統一方略;命兵部撰集兵法,名《制旨兵法》。他擊敗後蜀的孟昶,取得秦、鳳、成、階四州,孟昶大懼,「致書請和」;又先後三次征南唐,創建水軍,攻下淮南十四州。

顯德六年三月,圖舉收復燕雲十六州,一連攻陷瀛洲、莫州二州(今河北),莫州刺史劉楚信、瀛洲刺史高彥暉投降,再向北挺進,又連陷益津關、瓦橋關、高陽關三關。五月在議取幽州(今北京)時,柴榮病倒,只好撤退[註 1]。

後周顯德六年(959年)六月,柴榮去世,年僅39歲。由年仅7岁的儿子柴宗训即位,是为周恭帝。

柴榮是五代十國時期最英明的君主,為北宋之疆土奠定了基礎。《舊五代史》稱:“世宗頃在仄微,尤務韜晦……不日破高平之陣,逾年復秦、鳳之封,江北、燕南,取之如拾芥,神武雄略,乃一代之英主也……而降年不永,美志不就,悲夫!”

北汉进犯,柴荣对军士们说:“昔唐太宗定天下,未尝不自行,朕何敢偷安?”两军遇于高平之南,周将士俱,柴荣“介马自临阵督战”、“自引亲兵犯矢石督战”,当晚与将士“宿于野次”。

选武艺超绝者为殿前诸班。使得“征伐四方,所向皆捷”,自中唐以来的冗兵积弊,一扫而光。

“十年开拓天下,十年养百姓,十年致太平。”

针对“私度僧尼,日益猥杂”、“乡村之中,其弊转盛”,下诏:“近览诸州奏闻……私度僧尼,日增猥杂,创修寺院,渐至繁多……宜举旧章,以革前弊……诸道州县镇村坊,应有敕额寺院,一切仍旧,其无敕额者,并抑停废。”诏旨颁布后,废佛之风席卷全国,当年就废去寺院30,336所,僧尼还俗者大约6万人。除重点保护寺院外,一律停废。禁私度僧尼,禁僧俗舍身,并下诏毁铜佛像以铸钱。柴荣说:“卿辈勿以毁佛为疑。夫佛,以善道化人,苟志于善,斯奉佛矣。彼铜像岂所谓佛耶?且吾闻佛在利人,虽头目犹舍以布施。若朕身可以济民,亦非所惜也。”

定税征,保边民,解决了逃户庄田的问题。

均田租,抑豪强。

浚卞口,导河流,江淮舟楫始通。


\subsubsection{显德}

\begin{longtable}{|>{\centering\scriptsize}m{2em}|>{\centering\scriptsize}m{1.3em}|>{\centering}m{8.8em}|}
  % \caption{秦王政}\
  \toprule
  \SimHei \normalsize 年数 & \SimHei \scriptsize 公元 & \SimHei 大事件 \tabularnewline
  % \midrule
  \endfirsthead
  \toprule
  \SimHei \normalsize 年数 & \SimHei \scriptsize 公元 & \SimHei 大事件 \tabularnewline
  \midrule
  \endhead
  \midrule
  元年 & 954 & \tabularnewline\hline
  二年 & 955 & \tabularnewline\hline
  三年 & 956 & \tabularnewline\hline
  四年 & 957 & \tabularnewline\hline
  五年 & 958 & \tabularnewline\hline
  六年 & 959 & \tabularnewline
  \bottomrule
\end{longtable}


%%% Local Variables:
%%% mode: latex
%%% TeX-engine: xetex
%%% TeX-master: "../../Main"
%%% End:

%% -*- coding: utf-8 -*-
%% Time-stamp: <Chen Wang: 2021-11-01 15:22:38>

\subsection{恭帝柴宗训\tiny(959-960)}

\subsubsection{生平}

周恭帝柴宗训(953年9月14日-973年4月6日),五代时期后周皇帝,周世宗第四子。显德六年(959年)封为梁王。世宗于同年六月病死,他于同月甲午日继位,沿用周世宗年号“显德”。

柴宗训即位时,年仅七岁,由符太后垂帘听政,范质、王溥等主持军国大事。柴宗训在位期间,特别重用其父親任命的赵匡胤負責軍事事務。

显德七年(960年)正月元旦,群臣正在朝贺柴宗训时,镇(今河北省正定县)、定(今河北省定县)两州遣人来报,辽国和北汉合兵南侵。范质命令殿前都點檢赵匡胤率领禁军北上抵御。禁军到达开封东北部的陈桥驿后,突然发动兵变,拥赵匡胤为帝,黃袍加身。赵匡胤回师开封,朝中大臣范质等人被挟迫拜见“新天子”,显德七年,后周恭帝柴宗训禅让帝位于赵匡胤,降封郑王,符太后改称周太后,郭威柴荣的宗族被迁往房州(今湖北省房县)居住。

同年赵匡胤建立宋朝,改元“建隆”。柴宗训在位前后仅六个月。后周亡。

宋太祖頒布聖旨優待帝母子,賜柴氏“丹書鐵券”,保柴氏子孫永享富貴,可免除小罪,宽赦大罪,若犯十恶不赦则钦赐死于牢中(主要是为了维护贵族的体面,不被当街斩首)。

建隆三年(962年)柴宗训被迁往房陵(今湖北省房县)居住,开宝六年(973年)三月卒于当地,终年仅21岁。赵匡胤“闻之震恸”,「素服發哀,輟朝十日」,谥曰“恭皇帝”,归葬于世宗庆陵之侧。

\subsubsection{显德}

\begin{longtable}{|>{\centering\scriptsize}m{2em}|>{\centering\scriptsize}m{1.3em}|>{\centering}m{8.8em}|}
  % \caption{秦王政}\
  \toprule
  \SimHei \normalsize 年数 & \SimHei \scriptsize 公元 & \SimHei 大事件 \tabularnewline
  % \midrule
  \endfirsthead
  \toprule
  \SimHei \normalsize 年数 & \SimHei \scriptsize 公元 & \SimHei 大事件 \tabularnewline
  \midrule
  \endhead
  \midrule
  元年 & 959 & \tabularnewline\hline
  二年 & 960 & \tabularnewline
  \bottomrule
\end{longtable}


%%% Local Variables:
%%% mode: latex
%%% TeX-engine: xetex
%%% TeX-master: "../../Main"
%%% End:


%%% Local Variables:
%%% mode: latex
%%% TeX-engine: xetex
%%% TeX-master: "../../Main"
%%% End:



%%% Local Variables:
%%% mode: latex
%%% TeX-engine: xetex
%%% TeX-master: "../Main"
%%% End:
