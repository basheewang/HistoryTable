%% -*- coding: utf-8 -*-
%% Time-stamp: <Chen Wang: 2019-12-24 16:54:27>


\section{后汉\tiny(947-951)}

\subsection{简介}

後漢(947年-951年)是五代十國的第四個朝代,同时也是最后一个由沙陀人建立的中原王朝。亦称刘汉。後漢承自後晉,根據五行相生的順序,後晉的「金」德之後是「水」德,因此後漢以「水」為王朝德運。

後漢政權存在只有兩朝共四年,是五代十國中历时最短的政權,也被一些学者(例如钱文忠)认为是中国历史上最短命的“中央政權”。主要原因是劉知遠不懂治國,朝中爭鬥激烈、動則族誅,最後其子隱帝懷疑大臣郭威想要造反,派郭崇前往魏州殺死郭威,導致郭威起兵討伐隱帝,隱帝為部屬郭允明所殺。郭威進入都城開封后,本想立劉知遠侄子劉赟為帝,後來反悔,殺死劉赟,自己稱帝,建立後周。后汉被后周郭威所篡后,後漢高祖劉知遠的弟弟、鎮守晉陽的河东节度使刘崇在太原继位称帝,自称延续汉祚,然政权之疆域及地位均已改变,史家一般作为新政权或殘餘政權定位,列為「十國」之一,称为北汉。

五代史記述:「五代亂世,本無刑章,視人命如草芥,動以族誅為事。是族誅之法,凡罪人之父兄妻妾子孫並女之出嫁者,無一得免,非法之刑,於茲極矣!而尤莫如漢代之濫。然不問罪之輕重,理之是非,但云有犯,即處極刑。枉濫之家,莫敢上訴。軍吏因之為奸,嫁禍脅人,不可勝數。而此毒痛四海,殃及萬方。後漢劉氏父子二帝,享國不及四年。杨邠、史弘肇、蘇逢吉、劉銖等諸人亦皆被橫禍,無一善終者。此固天道之報施昭然,而民之生於是時,不知如何措手足也。」

统治范围包括今河南、山东、山西、河北南部、湖北北部、陕西北部、安徽北部。

開國君主劉知遠為世居太原的沙陀人,原為五代後晉河東節度使。947年,乘契丹陷開封而於太原稱帝,国号漢,史家稱“後漢”,以别于汉朝。自称为东汉显宗八子淮阳王刘昞之后,继承汉朝,宗庙内祭祀刘邦和刘秀。后攻克中原,定都汴京(今开封)。948年刘知远二子劉承祐嗣位,即后汉隐帝。950年李守贞等藩镇发生叛乱,汉隐帝命郭威平之,但汉隐帝猜忌郭威,欲杀之,郭威进而反叛。同年十一月二十一日(951年1月1日)刘承祐被杀,后汉亡。


%% -*- coding: utf-8 -*-
%% Time-stamp: <Chen Wang: 2019-12-24 16:55:29>

\subsection{高祖\tiny(947-948)}

\subsubsection{生平}

汉高祖劉知遠(895年-948年),晋阳(今山西太原)人,五代時期後漢開國皇帝,沙陀族,即帝位後改名劉暠,947年—948年在位,死後諡睿文聖武昭肅孝皇帝。劉知遠在太原出生,祖先是沙陀人,父名琠,冒姓刘氏。

劉知遠最初是李嗣源(後來的後唐明宗)的部下,鄴城之變,李嗣源登基後,劉知遠在石敬瑭帳下任牙門都校。934年,閔帝出逃到衛州時,與石敬瑭議事未決,後唐閔帝隨從欲動武,劉知遠把閔帝隨從殺盡,石敬瑭便捨閔帝而去。

936年,後唐末帝下詔調河東節度使石敬瑭為天平節度使。劉知遠勸石敬瑭起兵。石敬瑭便舉兵叛唐。劉知遠不贊成石敬瑭以割讓燕雲十六州向契丹借兵。石敬瑭不從。劉知遠任馬步軍都指揮使,同年任保義軍節度使。石敬瑭滅後唐,建後晉,937年任劉知遠為忠武軍節度使,941年任大名府留守兼河東節度使。出帝繼位後,在943年升劉知遠為中書令,944年任幽州道行营招讨使,封太原王,次年改封北平王。

出帝开运四年(947年),契丹滅后晋。河东行军司马张彦威等人以中原无主為由,勸劉知遠称帝,劉知遠在推搪一番後便在太原称帝,沿用后晋高祖年号天福,称天福十二年,同年六月入汴京,自称为东汉明帝八子淮阳王劉昞之后,改國號為「漢」。天福十三年(948年)正月改元乾祐,劉知遠又改名為劉暠,同月病逝。

劉知遠沒有治世的能力,為人也不守信,對於一些叛將,他先誘降而後殺之,例如張璉,他對民間百姓的刑罰也是以嚴厲聞名,他曾規定民間如有牛死,由官府收納牛皮,犯令者死;後又規定“民有犯鹽、礬、酒曲者,無多少皆抵死”,在他統治的時代,百姓苦不堪言;劉知遠在收復幽州時曾答應歸降契丹的軍民無罪,爾後卻殺光他們;又曾殺害吐谷渾部落百姓四百口,顯示出他殘忍、不講信用的一面;晉出帝受難之時,他居然隱兵不發,代表他早有異志,企圖稱帝。

歷史對於劉知遠多是批評居多,《舊五代史》即說劉知遠“乘虛而取神器,因亂而有帝圖”;“雖有應運之名,而未睹為君之德”。


\subsubsection{天福}

\begin{longtable}{|>{\centering\scriptsize}m{2em}|>{\centering\scriptsize}m{1.3em}|>{\centering}m{8.8em}|}
  % \caption{秦王政}\
  \toprule
  \SimHei \normalsize 年数 & \SimHei \scriptsize 公元 & \SimHei 大事件 \tabularnewline
  % \midrule
  \endfirsthead
  \toprule
  \SimHei \normalsize 年数 & \SimHei \scriptsize 公元 & \SimHei 大事件 \tabularnewline
  \midrule
  \endhead
  \midrule
  元年 & 947 & \tabularnewline
  \bottomrule
\end{longtable}

\subsubsection{乾祐}

\begin{longtable}{|>{\centering\scriptsize}m{2em}|>{\centering\scriptsize}m{1.3em}|>{\centering}m{8.8em}|}
  % \caption{秦王政}\
  \toprule
  \SimHei \normalsize 年数 & \SimHei \scriptsize 公元 & \SimHei 大事件 \tabularnewline
  % \midrule
  \endfirsthead
  \toprule
  \SimHei \normalsize 年数 & \SimHei \scriptsize 公元 & \SimHei 大事件 \tabularnewline
  \midrule
  \endhead
  \midrule
  元年 & 948 & \tabularnewline\hline
  \bottomrule
\end{longtable}


%%% Local Variables:
%%% mode: latex
%%% TeX-engine: xetex
%%% TeX-master: "../../Main"
%%% End:

%% -*- coding: utf-8 -*-
%% Time-stamp: <Chen Wang: 2019-12-24 17:03:41>

\subsection{隐帝\tiny(948-950)}

\subsubsection{生平}

汉隐帝劉承祐(931年-951年),並州晉陽(今山西太原)人,沙陀族,後漢第二個皇帝,948年—951年在位。汉高祖乾祐元年(948年)正月,劉知遠死后,承祐即位,年十八歲。沿用后汉高祖年号乾祐。

堂侄刘继文墓志载其为“少帝承翰”。

當時國事完全取決於重臣楊邠、郭威、史弘肇、王章之手。楊邠總機政,郭威主征伐,史弘肇典宿衛,王章掌財賦,權臣相爭,承祐寢食不安。他派郭威鎮压起义,平河中節度使李守貞之亂。乾祐三年(950年)初夏,契丹寇河北,命郭威鎮守鄴都(今河北大名)。

然承祐性多猜忌,私下與茶酒使郭允明計畫殺大臣,十一月將宰相楊邠、史弘肇、王章等砍死在東廂之下,又将郭威一家全部斩杀。復下詔誅殺正在鄴都留守的郭威,召泰寧節度使慕容彥超等急速入京。但郭威舉兵南下,十六日,抵澶州,十八日駐滑州。二十日,郭威至封丘(今属河南),擊敗慕容彦超於刘子陂(今河南封丘南)。二十一日(951年1月1日)攻入开封。刘承祐意欲出城亲征,李太后劝他不要莽撞,刘承祐不听。承祐出戰兵敗,連同苏逢吉、聂文进和郭允明等人向西北奔逃,二十二日(951年1月2日),至趙村,為郭允明所殺。郭威迎立劉崇子劉贇。廣順元年(951年)正月,李太后將傳國璽交給郭威,郭威在崇元殿登極,改年號廣順,是為後周代漢。

明末學者王夫之於《讀通鑑論》中直接以姓名稱呼劉承祐,寓意其不足為中國之主,又指劉承祐以為用一紙檄書可以殺盡權臣是以國家大事為遊戲,愚蠢至極;然而,王夫之亦讚揚他誅殺史弘肇、王章、楊邠等人之舉導致「風氣以移」、「內難不生」、「天下漸寧」,使代之而起的後周和北宋得以與民休息,進而統一中國。

\subsubsection{乾祐}

\begin{longtable}{|>{\centering\scriptsize}m{2em}|>{\centering\scriptsize}m{1.3em}|>{\centering}m{8.8em}|}
  % \caption{秦王政}\
  \toprule
  \SimHei \normalsize 年数 & \SimHei \scriptsize 公元 & \SimHei 大事件 \tabularnewline
  % \midrule
  \endfirsthead
  \toprule
  \SimHei \normalsize 年数 & \SimHei \scriptsize 公元 & \SimHei 大事件 \tabularnewline
  \midrule
  \endhead
  \midrule
  元年 & 948 & \tabularnewline\hline
  二年 & 949 & \tabularnewline\hline
  三年 & 950 & \tabularnewline
  \bottomrule
\end{longtable}

\subsection{湘阴公生平}

劉贇(10世纪?-951年),本名刘承赟,五代時期後漢宗室,生父劉崇,養父劉知遠。權臣郭威假意立之為漢帝,卻在赴京途中貶之為湘陰公,後被郭威所殺。

其生父劉崇是後漢高祖劉知遠的弟弟,劉贇十分受伯父劉知遠喜愛,因此被過繼為劉知遠的養子。

後漢隱帝乾祐三年(950年),任武寧節度使(駐守今江苏省徐州市)。

漢隱帝為了親自掌權,殺死樞密使郭威一家,郭威叛變,漢隱帝被亂兵所殺。郭威欲改立其弟刘承勋,承勋卧病不能胜任,于是郭威迎劉贇至首都開封府(今河南省开封市)即位。未料不久郭威就假借黃旗加身,在士兵的擁戴之下自立為君,而劉贇當時正行至中途,處於尷尬的困境。刘崇以为儿子将被拥立为帝,没有听从手下的建议,按兵不动。劉贇在其衛兵投降郭威後遭軟禁,并被李太后下詔,废為湘陰公。數日後,郭威正式稱帝,建立後周,時為951年正月。不久,劉贇就被郭威派人殺害。其后,其父刘崇在太原登基,建立北汉。

%%% Local Variables:
%%% mode: latex
%%% TeX-engine: xetex
%%% TeX-master: "../../Main"
%%% End:


%%% Local Variables:
%%% mode: latex
%%% TeX-engine: xetex
%%% TeX-master: "../../Main"
%%% End:
