%% -*- coding: utf-8 -*-
%% Time-stamp: <Chen Wang: 2021-11-01 15:23:28>

\subsection{隐帝劉承祐\tiny(948-950)}

\subsubsection{生平}

汉隐帝劉承祐(931年-951年),並州晉陽(今山西太原)人,沙陀族,後漢第二個皇帝,948年—951年在位。汉高祖乾祐元年(948年)正月,劉知遠死后,承祐即位,年十八歲。沿用后汉高祖年号乾祐。

堂侄刘继文墓志载其为“少帝承翰”。

當時國事完全取決於重臣楊邠、郭威、史弘肇、王章之手。楊邠總機政,郭威主征伐,史弘肇典宿衛,王章掌財賦,權臣相爭,承祐寢食不安。他派郭威鎮压起义,平河中節度使李守貞之亂。乾祐三年(950年)初夏,契丹寇河北,命郭威鎮守鄴都(今河北大名)。

然承祐性多猜忌,私下與茶酒使郭允明計畫殺大臣,十一月將宰相楊邠、史弘肇、王章等砍死在東廂之下,又将郭威一家全部斩杀。復下詔誅殺正在鄴都留守的郭威,召泰寧節度使慕容彥超等急速入京。但郭威舉兵南下,十六日,抵澶州,十八日駐滑州。二十日,郭威至封丘(今属河南),擊敗慕容彦超於刘子陂(今河南封丘南)。二十一日(951年1月1日)攻入开封。刘承祐意欲出城亲征,李太后劝他不要莽撞,刘承祐不听。承祐出戰兵敗,連同苏逢吉、聂文进和郭允明等人向西北奔逃,二十二日(951年1月2日),至趙村,為郭允明所殺。郭威迎立劉崇子劉贇。廣順元年(951年)正月,李太后將傳國璽交給郭威,郭威在崇元殿登極,改年號廣順,是為後周代漢。

明末學者王夫之於《讀通鑑論》中直接以姓名稱呼劉承祐,寓意其不足為中國之主,又指劉承祐以為用一紙檄書可以殺盡權臣是以國家大事為遊戲,愚蠢至極;然而,王夫之亦讚揚他誅殺史弘肇、王章、楊邠等人之舉導致「風氣以移」、「內難不生」、「天下漸寧」,使代之而起的後周和北宋得以與民休息,進而統一中國。

\subsubsection{乾祐}

\begin{longtable}{|>{\centering\scriptsize}m{2em}|>{\centering\scriptsize}m{1.3em}|>{\centering}m{8.8em}|}
  % \caption{秦王政}\
  \toprule
  \SimHei \normalsize 年数 & \SimHei \scriptsize 公元 & \SimHei 大事件 \tabularnewline
  % \midrule
  \endfirsthead
  \toprule
  \SimHei \normalsize 年数 & \SimHei \scriptsize 公元 & \SimHei 大事件 \tabularnewline
  \midrule
  \endhead
  \midrule
  元年 & 948 & \tabularnewline\hline
  二年 & 949 & \tabularnewline\hline
  三年 & 950 & \tabularnewline
  \bottomrule
\end{longtable}

\subsection{湘阴公劉贇生平}

劉贇(10世纪?-951年),本名刘承赟,五代時期後漢宗室,生父劉崇,養父劉知遠。權臣郭威假意立之為漢帝,卻在赴京途中貶之為湘陰公,後被郭威所殺。

其生父劉崇是後漢高祖劉知遠的弟弟,劉贇十分受伯父劉知遠喜愛,因此被過繼為劉知遠的養子。

後漢隱帝乾祐三年(950年),任武寧節度使(駐守今江苏省徐州市)。

漢隱帝為了親自掌權,殺死樞密使郭威一家,郭威叛變,漢隱帝被亂兵所殺。郭威欲改立其弟刘承勋,承勋卧病不能胜任,于是郭威迎劉贇至首都開封府(今河南省开封市)即位。未料不久郭威就假借黃旗加身,在士兵的擁戴之下自立為君,而劉贇當時正行至中途,處於尷尬的困境。刘崇以为儿子将被拥立为帝,没有听从手下的建议,按兵不动。劉贇在其衛兵投降郭威後遭軟禁,并被李太后下詔,废為湘陰公。數日後,郭威正式稱帝,建立後周,時為951年正月。不久,劉贇就被郭威派人殺害。其后,其父刘崇在太原登基,建立北汉。

%%% Local Variables:
%%% mode: latex
%%% TeX-engine: xetex
%%% TeX-master: "../../Main"
%%% End:
