%% -*- coding: utf-8 -*-
%% Time-stamp: <Chen Wang: 2021-11-01 15:24:24>

\subsection{高祖劉知遠\tiny(947-948)}

\subsubsection{生平}

汉高祖劉知遠(895年-948年),晋阳(今山西太原)人,五代時期後漢開國皇帝,沙陀族,即帝位後改名劉暠,947年—948年在位,死後諡睿文聖武昭肅孝皇帝。劉知遠在太原出生,祖先是沙陀人,父名琠,冒姓刘氏。

劉知遠最初是李嗣源(後來的後唐明宗)的部下,鄴城之變,李嗣源登基後,劉知遠在石敬瑭帳下任牙門都校。934年,閔帝出逃到衛州時,與石敬瑭議事未決,後唐閔帝隨從欲動武,劉知遠把閔帝隨從殺盡,石敬瑭便捨閔帝而去。

936年,後唐末帝下詔調河東節度使石敬瑭為天平節度使。劉知遠勸石敬瑭起兵。石敬瑭便舉兵叛唐。劉知遠不贊成石敬瑭以割讓燕雲十六州向契丹借兵。石敬瑭不從。劉知遠任馬步軍都指揮使,同年任保義軍節度使。石敬瑭滅後唐,建後晉,937年任劉知遠為忠武軍節度使,941年任大名府留守兼河東節度使。出帝繼位後,在943年升劉知遠為中書令,944年任幽州道行营招讨使,封太原王,次年改封北平王。

出帝开运四年(947年),契丹滅后晋。河东行军司马张彦威等人以中原无主為由,勸劉知遠称帝,劉知遠在推搪一番後便在太原称帝,沿用后晋高祖年号天福,称天福十二年,同年六月入汴京,自称为东汉明帝八子淮阳王劉昞之后,改國號為「漢」。天福十三年(948年)正月改元乾祐,劉知遠又改名為劉暠,同月病逝。

劉知遠沒有治世的能力,為人也不守信,對於一些叛將,他先誘降而後殺之,例如張璉,他對民間百姓的刑罰也是以嚴厲聞名,他曾規定民間如有牛死,由官府收納牛皮,犯令者死;後又規定“民有犯鹽、礬、酒曲者,無多少皆抵死”,在他統治的時代,百姓苦不堪言;劉知遠在收復幽州時曾答應歸降契丹的軍民無罪,爾後卻殺光他們;又曾殺害吐谷渾部落百姓四百口,顯示出他殘忍、不講信用的一面;晉出帝受難之時,他居然隱兵不發,代表他早有異志,企圖稱帝。

歷史對於劉知遠多是批評居多,《舊五代史》即說劉知遠“乘虛而取神器,因亂而有帝圖”;“雖有應運之名,而未睹為君之德”。


\subsubsection{天福}

\begin{longtable}{|>{\centering\scriptsize}m{2em}|>{\centering\scriptsize}m{1.3em}|>{\centering}m{8.8em}|}
  % \caption{秦王政}\
  \toprule
  \SimHei \normalsize 年数 & \SimHei \scriptsize 公元 & \SimHei 大事件 \tabularnewline
  % \midrule
  \endfirsthead
  \toprule
  \SimHei \normalsize 年数 & \SimHei \scriptsize 公元 & \SimHei 大事件 \tabularnewline
  \midrule
  \endhead
  \midrule
  元年 & 947 & \tabularnewline
  \bottomrule
\end{longtable}

\subsubsection{乾祐}

\begin{longtable}{|>{\centering\scriptsize}m{2em}|>{\centering\scriptsize}m{1.3em}|>{\centering}m{8.8em}|}
  % \caption{秦王政}\
  \toprule
  \SimHei \normalsize 年数 & \SimHei \scriptsize 公元 & \SimHei 大事件 \tabularnewline
  % \midrule
  \endfirsthead
  \toprule
  \SimHei \normalsize 年数 & \SimHei \scriptsize 公元 & \SimHei 大事件 \tabularnewline
  \midrule
  \endhead
  \midrule
  元年 & 948 & \tabularnewline
  \bottomrule
\end{longtable}


%%% Local Variables:
%%% mode: latex
%%% TeX-engine: xetex
%%% TeX-master: "../../Main"
%%% End:
