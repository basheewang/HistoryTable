%% -*- coding: utf-8 -*-
%% Time-stamp: <Chen Wang: 2019-12-24 16:52:18>

\subsection{高祖\tiny(936-942)}

\subsubsection{生平}

晉高祖石敬瑭(892年3月30日-942年7月28日),五代十国時期的后晋开国皇帝(936年11月28日–942年7月28日在位)。庙号高祖,谥号圣文章武明德孝皇帝。他把燕雲十六州割讓给契丹,使中原地區丧失了北方屏障,並向辽太宗自称儿皇帝。

《新五代史》指石敬瑭的祖先为中亚人,从沙陀移居太原,但發挖出土的石重貴墓誌銘則指他是後趙石勒之後裔。

父石紹雍,母何氏。石紹雍从李克用父子征战,官至洺州刺史。

石敬瑭自少为李嗣源(日後的唐明宗)赏识,为其亲兵将领,被招为女婿。後唐莊宗同光四年(926年),邺都之变,石敬瑭力劝李嗣源入汴京,转攻洛阳。李嗣源即位后,石敬瑭历任保义、宣武、河東诸镇节度使。

934年,閔帝李從厚徙石敬瑭為成德節度使。閔帝討伐潞王李从珂失敗,逃到衛州向石敬瑭求援,可是石敬瑭的部下把閔帝隨從殺盡,石敬瑭把閔帝安置在衛州,最後閔帝被李从珂派人殺死。

末帝李从珂继位后,任石敬瑭為河東節度使,後來開始對石敬瑭起疑,石敬瑭也暗中謀自保。石敬瑭以多病為理由,上表請求朝廷調他往其它藩鎮,借此試探朝廷對他的態度。末帝在清泰三年(936年)五月改授石敬瑭為天平節度使,並降旨催促赴任。石敬瑭懷疑末帝對他起疑心,便举兵叛变。後唐派兵討伐,石敬瑭被圍,向契丹求援。九月契丹軍南下,擊敗唐軍。

石敬瑭的岳父是唐明宗李嗣源。李嗣源的义父是李克用。李克用曾和辽太宗耶律阿保机结为兄弟,故石敬瑭按辈份称比他小10歲的耶律阿宝机的儿子耶律德光为亚父,并在国书中稱自己為“兒皇帝”,耶律德光为“父皇帝”。

石敬瑭在十一月受契丹冊封為大晉皇帝,然後向洛陽進軍,後唐末帝在閏十一月(937年1月)自焚,後唐遂亡。

石敬瑭滅後唐後,按约定将燕雲十六州献给契丹,其结果使中原地區丧失了北方屏障。另外晋国向辽国每岁奉绢三十万匹。

石敬瑭在位期間,各地將領魏博节度使范延光、西京留守张从宾、成德节度使安重荣、山南东道节度使安从进等引发的叛變事件不斷,他的兒子石重信和石重乂亦遭叛軍殺害。后因成德節度使安重榮及河东节度使劉知遠先後接受吐谷浑部族投降,石敬瑭屡遭契丹责问,乃忧愤而死。

《舊五代史》稱讚石敬瑭的謙虛、節儉;「旰食宵衣,禮賢從諫」、「以絁為衣,以麻為履」,後又責怪他向契丹乞兵,反而使得百姓陷入連年戰火;「強鄰來援,契丹自茲而孔熾,黔黎由是以罹殃。」「兵連禍結、舉族為俘」,這無疑是決鯨海以救焚,結果自己溺死了、飲鴆漿而止渴,結果毒死自己。《舊五代史》最後為他惋惜,如果他是靠自己的力量取得帝位,以他的節儉、謙卑、公正的態度,即使功德不超過前人,亦可謂仁慈恭儉之主。


\subsubsection{天福}

\begin{longtable}{|>{\centering\scriptsize}m{2em}|>{\centering\scriptsize}m{1.3em}|>{\centering}m{8.8em}|}
  % \caption{秦王政}\
  \toprule
  \SimHei \normalsize 年数 & \SimHei \scriptsize 公元 & \SimHei 大事件 \tabularnewline
  % \midrule
  \endfirsthead
  \toprule
  \SimHei \normalsize 年数 & \SimHei \scriptsize 公元 & \SimHei 大事件 \tabularnewline
  \midrule
  \endhead
  \midrule
  元年 & 936 & \tabularnewline\hline
  二年 & 937 & \tabularnewline\hline
  三年 & 938 & \tabularnewline\hline
  四年 & 939 & \tabularnewline\hline
  五年 & 940 & \tabularnewline\hline
  六年 & 941 & \tabularnewline\hline
  七年 & 942 & \tabularnewline\hline
  八年 & 943 & \tabularnewline\hline
  九年 & 944 & \tabularnewline
  \bottomrule
\end{longtable}


%%% Local Variables:
%%% mode: latex
%%% TeX-engine: xetex
%%% TeX-master: "../../Main"
%%% End:
