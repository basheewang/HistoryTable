%% -*- coding: utf-8 -*-
%% Time-stamp: <Chen Wang: 2019-12-24 16:53:42>

\subsection{出帝\tiny(942-946)}

\subsubsection{生平}

晋出帝石重贵(914年-974年),又稱少帝,942年-946年在位。天福七年(942年),后晋高祖石敬瑭死,重贵繼位,沿用高祖天福年号,天福九年(944年)七月改元开运。石重贵不肯向契丹称臣,契丹攻后晋,開運三年十二月(947年1月)佔開封,石重贵投降,后晋亡。

石重貴出生於太原,是石敬瑭的姪兒,父親是石敬瑭兄敬儒,母安氏。敬儒早逝,敬瑭以其子重貴為子。石敬瑭雖有六子,但有五子早死,而餘下的石重睿年幼,所以石敬瑭便選擇重貴作為繼承人。石敬瑭起兵反後唐時,以石重貴為金紫光祿大夫,行太原尹、北京留守,知河東節度事。

天福二年(937年)九月,升為左金吾衛上將軍。天福三年冬,為開封尹,封鄭王,加太尉,同中書門下平章事。天福六年,改為廣晉尹,徙封齊王。天福七年六月,石敬瑭去世,石重貴繼位。

天福八年間,二十七個州郡發生蝗災,數十萬人餓死。次年飢荒仍然嚴重,在隴州有五萬六千人餓死。

石重貴依從重臣景延廣之言,放棄高祖時期對契丹恭順的政策,对耶律德光称孙但不称臣,兩国關係惡化。天福九年(944年)正月契丹軍開始入侵,三年間雙方互有勝負。開運三年十二月將領杜重威、李守贞、張彥澤率軍向契丹軍投降,契丹派張彥澤率領先頭部隊入開封,石重貴投降,後晉滅亡。

遼太宗耶律德光在947年正月把石重貴降為光祿大夫、檢校太尉,封「負義侯」,後晉正式滅亡。石重貴被安置在黃龍府,後來遷往建州。《舊五代史》引范質《晉朝陷蕃記》稱石重貴「凡十八年而卒」,即在北宋乾德二年(964年)去世。石重貴墓誌銘(現藏於遼寧省博物館)稱他在遼保寧六年(974年)六月十八日病逝。


\subsubsection{开运}

\begin{longtable}{|>{\centering\scriptsize}m{2em}|>{\centering\scriptsize}m{1.3em}|>{\centering}m{8.8em}|}
  % \caption{秦王政}\
  \toprule
  \SimHei \normalsize 年数 & \SimHei \scriptsize 公元 & \SimHei 大事件 \tabularnewline
  % \midrule
  \endfirsthead
  \toprule
  \SimHei \normalsize 年数 & \SimHei \scriptsize 公元 & \SimHei 大事件 \tabularnewline
  \midrule
  \endhead
  \midrule
  元年 & 944 & \tabularnewline\hline
  二年 & 945 & \tabularnewline\hline
  三年 & 946 & \tabularnewline
  \bottomrule
\end{longtable}


%%% Local Variables:
%%% mode: latex
%%% TeX-engine: xetex
%%% TeX-master: "../../Main"
%%% End:
