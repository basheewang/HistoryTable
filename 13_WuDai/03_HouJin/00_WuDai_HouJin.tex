%% -*- coding: utf-8 -*-
%% Time-stamp: <Chen Wang: 2019-12-24 16:50:13>


\section{后晋\tiny(936-947)}

\subsection{简介}

后晋(936年-947年)是中国历史上五代十国时期的一个朝代,从后晋高祖石敬瑭936年灭后唐开国到契丹947年灭后晋一共经历了两个皇帝,總計12年。为与司马氏的晋朝相区别,又别称为石晋。

后晋的开国皇帝沙陀人石敬瑭是后唐开国的功臣,他曾经多次在危难中救护后唐开国皇帝李存勖和明宗李嗣源。李存勖和李嗣源都十分器重他,李嗣源甚至将自己的女儿嫁给了他。后唐建立后石敬瑭任河东节度使(今山西),石敬瑭成为当地军民最高指挥官。石敬瑭在河东政绩很高,而且生活清廉,很受当地人的欢迎。但李嗣源死后后唐内部互相倾轧,石敬瑭受李从珂的猜忌,因此渐渐产生了反唐的计划。当李从珂决定将石敬瑭调离河东时,石敬瑭决定反唐。

石敬瑭在河东的兵力不足以抵挡后唐的进攻,因此石敬瑭决定求救于契丹。作为条件,他同意割让燕雲十六州(此十六个州,属今河北和山西)给契丹,并对辽太宗耶律德光称「儿」。在这种情况下,耶律德光决定帮助石敬瑭。

契丹和石敬瑭的联军打败了后唐,攻入后唐首都洛阳。后唐灭亡,石敬瑭称帝,国号晋,史称后晋。依據五行相生的順序,後唐的「土」德之後為「金」德,因此後晉以「金」為王朝德運。後晉移都开封,并按约将16州让给契丹。这16州是:幽(今北京市)、蓟(今天津蓟县)、瀛(今河北河间)、莫(今河北任丘)、涿(今河北涿州)、檀(今北京密云)、顺(今北京顺义)、新(今河北涿鹿)、妫(原属河北怀来,今为官厅水库库区)、儒(今北京延庆)、武(今河北宣化)、蔚(今河北蔚县)、云(今山西大同)、应(今山西应县)、寰(今山西朔县东马邑镇)、朔(今山西朔县),并向契丹称儿皇帝,契丹封其为“晋帝”。

石敬瑭割让燕雲十六州为辽国和金国后来对宋朝长江以北地区的威胁打开了门户。

后晋建国后一直处于动盪中,石敬瑭割地称儿的做法受到许多人的反对,包括他自己过去的亲信。石敬瑭本人到死没有改变依附契丹的政策,但国家多處发生叛乱,石敬瑭的两个儿子在这些叛乱中被杀,種種事情给他带来了极大的打击。为了对付叛乱,石敬瑭加重嚴刑峻法,同时非常猜忌自己的手下。

石敬瑭死时,立石重贵为继承人。石重贵是他的侄子,因为在战场上立战功获得石敬瑭的赏识。但石重贵仅是一勇之夫,根本无法在國家面對困境下應付各種政治问题。石重贵登基后决定渐渐脱离对契丹的依附,他首先宣称对耶律德光称孙,但不称臣。

契丹对此当然不能坐视。944年契丹伐晋,双方在澶州(今河南濮阳南)交战,互有胜负。945年契丹再次南征,石重贵亲征,再次战败契丹。947年,契丹第三次南下,后晋重臣杜重威、李守貞和張彥澤率軍向契丹投降,后晋丧失主力,契丹派張彥澤率先部入開封。石重贵被迫投降,全家被俘虏到契丹。后晋灭亡。

后晋亡后,河东节度使北平王刘知远在太原称帝,建立后汉。

%% -*- coding: utf-8 -*-
%% Time-stamp: <Chen Wang: 2019-12-24 16:52:18>

\subsection{高祖\tiny(936-942)}

\subsubsection{生平}

晉高祖石敬瑭(892年3月30日-942年7月28日),五代十国時期的后晋开国皇帝(936年11月28日–942年7月28日在位)。庙号高祖,谥号圣文章武明德孝皇帝。他把燕雲十六州割讓给契丹,使中原地區丧失了北方屏障,並向辽太宗自称儿皇帝。

《新五代史》指石敬瑭的祖先为中亚人,从沙陀移居太原,但發挖出土的石重貴墓誌銘則指他是後趙石勒之後裔。

父石紹雍,母何氏。石紹雍从李克用父子征战,官至洺州刺史。

石敬瑭自少为李嗣源(日後的唐明宗)赏识,为其亲兵将领,被招为女婿。後唐莊宗同光四年(926年),邺都之变,石敬瑭力劝李嗣源入汴京,转攻洛阳。李嗣源即位后,石敬瑭历任保义、宣武、河東诸镇节度使。

934年,閔帝李從厚徙石敬瑭為成德節度使。閔帝討伐潞王李从珂失敗,逃到衛州向石敬瑭求援,可是石敬瑭的部下把閔帝隨從殺盡,石敬瑭把閔帝安置在衛州,最後閔帝被李从珂派人殺死。

末帝李从珂继位后,任石敬瑭為河東節度使,後來開始對石敬瑭起疑,石敬瑭也暗中謀自保。石敬瑭以多病為理由,上表請求朝廷調他往其它藩鎮,借此試探朝廷對他的態度。末帝在清泰三年(936年)五月改授石敬瑭為天平節度使,並降旨催促赴任。石敬瑭懷疑末帝對他起疑心,便举兵叛变。後唐派兵討伐,石敬瑭被圍,向契丹求援。九月契丹軍南下,擊敗唐軍。

石敬瑭的岳父是唐明宗李嗣源。李嗣源的义父是李克用。李克用曾和辽太宗耶律阿保机结为兄弟,故石敬瑭按辈份称比他小10歲的耶律阿宝机的儿子耶律德光为亚父,并在国书中稱自己為“兒皇帝”,耶律德光为“父皇帝”。

石敬瑭在十一月受契丹冊封為大晉皇帝,然後向洛陽進軍,後唐末帝在閏十一月(937年1月)自焚,後唐遂亡。

石敬瑭滅後唐後,按约定将燕雲十六州献给契丹,其结果使中原地區丧失了北方屏障。另外晋国向辽国每岁奉绢三十万匹。

石敬瑭在位期間,各地將領魏博节度使范延光、西京留守张从宾、成德节度使安重荣、山南东道节度使安从进等引发的叛變事件不斷,他的兒子石重信和石重乂亦遭叛軍殺害。后因成德節度使安重榮及河东节度使劉知遠先後接受吐谷浑部族投降,石敬瑭屡遭契丹责问,乃忧愤而死。

《舊五代史》稱讚石敬瑭的謙虛、節儉;「旰食宵衣,禮賢從諫」、「以絁為衣,以麻為履」,後又責怪他向契丹乞兵,反而使得百姓陷入連年戰火;「強鄰來援,契丹自茲而孔熾,黔黎由是以罹殃。」「兵連禍結、舉族為俘」,這無疑是決鯨海以救焚,結果自己溺死了、飲鴆漿而止渴,結果毒死自己。《舊五代史》最後為他惋惜,如果他是靠自己的力量取得帝位,以他的節儉、謙卑、公正的態度,即使功德不超過前人,亦可謂仁慈恭儉之主。


\subsubsection{天福}

\begin{longtable}{|>{\centering\scriptsize}m{2em}|>{\centering\scriptsize}m{1.3em}|>{\centering}m{8.8em}|}
  % \caption{秦王政}\
  \toprule
  \SimHei \normalsize 年数 & \SimHei \scriptsize 公元 & \SimHei 大事件 \tabularnewline
  % \midrule
  \endfirsthead
  \toprule
  \SimHei \normalsize 年数 & \SimHei \scriptsize 公元 & \SimHei 大事件 \tabularnewline
  \midrule
  \endhead
  \midrule
  元年 & 936 & \tabularnewline\hline
  二年 & 937 & \tabularnewline\hline
  三年 & 938 & \tabularnewline\hline
  四年 & 939 & \tabularnewline\hline
  五年 & 940 & \tabularnewline\hline
  六年 & 941 & \tabularnewline\hline
  七年 & 942 & \tabularnewline\hline
  八年 & 943 & \tabularnewline\hline
  九年 & 944 & \tabularnewline
  \bottomrule
\end{longtable}


%%% Local Variables:
%%% mode: latex
%%% TeX-engine: xetex
%%% TeX-master: "../../Main"
%%% End:

%% -*- coding: utf-8 -*-
%% Time-stamp: <Chen Wang: 2019-12-24 16:53:42>

\subsection{出帝\tiny(942-946)}

\subsubsection{生平}

晋出帝石重贵(914年-974年),又稱少帝,942年-946年在位。天福七年(942年),后晋高祖石敬瑭死,重贵繼位,沿用高祖天福年号,天福九年(944年)七月改元开运。石重贵不肯向契丹称臣,契丹攻后晋,開運三年十二月(947年1月)佔開封,石重贵投降,后晋亡。

石重貴出生於太原,是石敬瑭的姪兒,父親是石敬瑭兄敬儒,母安氏。敬儒早逝,敬瑭以其子重貴為子。石敬瑭雖有六子,但有五子早死,而餘下的石重睿年幼,所以石敬瑭便選擇重貴作為繼承人。石敬瑭起兵反後唐時,以石重貴為金紫光祿大夫,行太原尹、北京留守,知河東節度事。

天福二年(937年)九月,升為左金吾衛上將軍。天福三年冬,為開封尹,封鄭王,加太尉,同中書門下平章事。天福六年,改為廣晉尹,徙封齊王。天福七年六月,石敬瑭去世,石重貴繼位。

天福八年間,二十七個州郡發生蝗災,數十萬人餓死。次年飢荒仍然嚴重,在隴州有五萬六千人餓死。

石重貴依從重臣景延廣之言,放棄高祖時期對契丹恭順的政策,对耶律德光称孙但不称臣,兩国關係惡化。天福九年(944年)正月契丹軍開始入侵,三年間雙方互有勝負。開運三年十二月將領杜重威、李守贞、張彥澤率軍向契丹軍投降,契丹派張彥澤率領先頭部隊入開封,石重貴投降,後晉滅亡。

遼太宗耶律德光在947年正月把石重貴降為光祿大夫、檢校太尉,封「負義侯」,後晉正式滅亡。石重貴被安置在黃龍府,後來遷往建州。《舊五代史》引范質《晉朝陷蕃記》稱石重貴「凡十八年而卒」,即在北宋乾德二年(964年)去世。石重貴墓誌銘(現藏於遼寧省博物館)稱他在遼保寧六年(974年)六月十八日病逝。


\subsubsection{开运}

\begin{longtable}{|>{\centering\scriptsize}m{2em}|>{\centering\scriptsize}m{1.3em}|>{\centering}m{8.8em}|}
  % \caption{秦王政}\
  \toprule
  \SimHei \normalsize 年数 & \SimHei \scriptsize 公元 & \SimHei 大事件 \tabularnewline
  % \midrule
  \endfirsthead
  \toprule
  \SimHei \normalsize 年数 & \SimHei \scriptsize 公元 & \SimHei 大事件 \tabularnewline
  \midrule
  \endhead
  \midrule
  元年 & 944 & \tabularnewline\hline
  二年 & 945 & \tabularnewline\hline
  三年 & 946 & \tabularnewline
  \bottomrule
\end{longtable}


%%% Local Variables:
%%% mode: latex
%%% TeX-engine: xetex
%%% TeX-master: "../../Main"
%%% End:



%%% Local Variables:
%%% mode: latex
%%% TeX-engine: xetex
%%% TeX-master: "../../Main"
%%% End:
