%% -*- coding: utf-8 -*-
%% Time-stamp: <Chen Wang: 2021-11-01 15:22:29>

\subsection{世宗柴榮\tiny(954-959)}

\subsubsection{生平}

周世宗柴榮(921年10月27日-959年7月27日),五代時期後周皇帝,於954年2月26日-959年7月27日在位,在位6年。邢州堯山柴家莊(今河北省邢台市隆堯縣)人,是周太祖郭威的養子(柴榮本身是郭威正室柴皇后的侄子),是中國少数由外戚继承宗室的皇帝,廟號世宗,諡號睿武孝文皇帝。根據史書記錄,在此時期在政治、軍事、經濟上都有建樹,號稱英主,他初步奠定了後來北宋的勢力。

父柴守禮,祖父柴翁是當地望族,柴榮年輕時曾隨商人頡跌氏在江陵販賣茶葉,對社會積弊有所體驗。史載其「器貌英奇,善騎射,略通書、史、黃老,性沉重寡言」。

广顺元年(951年),周太祖郭威即位,柴荣授澶州节度使、检校太保,封太原郡侯。史载其境“为政清肃,盗不犯境”。二年,检校太傅为相。又一年,封晋王。

显德元年(954年)正月,判内外兵马事,总揽兵权。同月太祖崩,即帝位。

柴榮即位後,立刻下令招撫流亡,減少賦稅,恢復中原經濟。當時,北方及中原經過了一系列的戰爭,百姓痛苦不堪,柴榮的舉動正好使中原開始復甦,他整頓吏治,使後周政治清明,百姓富庶,經濟開始繁榮。

顯德二年(955年)推行顯德毀佛,以佛寺銅材鑄行「周元通寶」,錢質與鑄量均居五代之冠。因为其毀佛行為,後周世宗被列入毀佛的「三武(北魏太武帝、北周武帝和唐武宗)一宗」。司馬光評述周世宗「毀佛」:「不愛己身而愛民,不以無益廢有益,周世宗算得是仁愛明理之人。」

柴榮對內進行改革,對外則積極開拓疆土。顯德元年(954年)二月,北漢主劉崇乘其新立,勾結遼兵4萬攻後周,柴榮力排馮道勸阻,率軍迎戰,於高平(今山西)南大破北漢軍,穩定政局。戰後整軍練卒,裁汰冗弱,於是軍威大振。顯德二年詔令群臣獻《為君難為臣不易論》、《平邊策》,確定王樸提出的「先南後北」的統一方略;命兵部撰集兵法,名《制旨兵法》。他擊敗後蜀的孟昶,取得秦、鳳、成、階四州,孟昶大懼,「致書請和」;又先後三次征南唐,創建水軍,攻下淮南十四州。

顯德六年三月,圖舉收復燕雲十六州,一連攻陷瀛洲、莫州二州(今河北),莫州刺史劉楚信、瀛洲刺史高彥暉投降,再向北挺進,又連陷益津關、瓦橋關、高陽關三關。五月在議取幽州(今北京)時,柴榮病倒,只好撤退[註 1]。

後周顯德六年(959年)六月,柴榮去世,年僅39歲。由年仅7岁的儿子柴宗训即位,是为周恭帝。

柴榮是五代十國時期最英明的君主,為北宋之疆土奠定了基礎。《舊五代史》稱:“世宗頃在仄微,尤務韜晦……不日破高平之陣,逾年復秦、鳳之封,江北、燕南,取之如拾芥,神武雄略,乃一代之英主也……而降年不永,美志不就,悲夫!”

北汉进犯,柴荣对军士们说:“昔唐太宗定天下,未尝不自行,朕何敢偷安?”两军遇于高平之南,周将士俱,柴荣“介马自临阵督战”、“自引亲兵犯矢石督战”,当晚与将士“宿于野次”。

选武艺超绝者为殿前诸班。使得“征伐四方,所向皆捷”,自中唐以来的冗兵积弊,一扫而光。

“十年开拓天下,十年养百姓,十年致太平。”

针对“私度僧尼,日益猥杂”、“乡村之中,其弊转盛”,下诏:“近览诸州奏闻……私度僧尼,日增猥杂,创修寺院,渐至繁多……宜举旧章,以革前弊……诸道州县镇村坊,应有敕额寺院,一切仍旧,其无敕额者,并抑停废。”诏旨颁布后,废佛之风席卷全国,当年就废去寺院30,336所,僧尼还俗者大约6万人。除重点保护寺院外,一律停废。禁私度僧尼,禁僧俗舍身,并下诏毁铜佛像以铸钱。柴荣说:“卿辈勿以毁佛为疑。夫佛,以善道化人,苟志于善,斯奉佛矣。彼铜像岂所谓佛耶?且吾闻佛在利人,虽头目犹舍以布施。若朕身可以济民,亦非所惜也。”

定税征,保边民,解决了逃户庄田的问题。

均田租,抑豪强。

浚卞口,导河流,江淮舟楫始通。


\subsubsection{显德}

\begin{longtable}{|>{\centering\scriptsize}m{2em}|>{\centering\scriptsize}m{1.3em}|>{\centering}m{8.8em}|}
  % \caption{秦王政}\
  \toprule
  \SimHei \normalsize 年数 & \SimHei \scriptsize 公元 & \SimHei 大事件 \tabularnewline
  % \midrule
  \endfirsthead
  \toprule
  \SimHei \normalsize 年数 & \SimHei \scriptsize 公元 & \SimHei 大事件 \tabularnewline
  \midrule
  \endhead
  \midrule
  元年 & 954 & \tabularnewline\hline
  二年 & 955 & \tabularnewline\hline
  三年 & 956 & \tabularnewline\hline
  四年 & 957 & \tabularnewline\hline
  五年 & 958 & \tabularnewline\hline
  六年 & 959 & \tabularnewline
  \bottomrule
\end{longtable}


%%% Local Variables:
%%% mode: latex
%%% TeX-engine: xetex
%%% TeX-master: "../../Main"
%%% End:
