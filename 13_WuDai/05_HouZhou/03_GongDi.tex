%% -*- coding: utf-8 -*-
%% Time-stamp: <Chen Wang: 2019-12-24 17:08:22>

\subsection{恭帝\tiny(959-960)}

\subsubsection{生平}

周恭帝柴宗训(953年9月14日-973年4月6日),五代时期后周皇帝,周世宗第四子。显德六年(959年)封为梁王。世宗于同年六月病死,他于同月甲午日继位,沿用周世宗年号“显德”。

柴宗训即位时,年仅七岁,由符太后垂帘听政,范质、王溥等主持军国大事。柴宗训在位期间,特别重用其父親任命的赵匡胤負責軍事事務。

显德七年(960年)正月元旦,群臣正在朝贺柴宗训时,镇(今河北省正定县)、定(今河北省定县)两州遣人来报,辽国和北汉合兵南侵。范质命令殿前都點檢赵匡胤率领禁军北上抵御。禁军到达开封东北部的陈桥驿后,突然发动兵变,拥赵匡胤为帝,黃袍加身。赵匡胤回师开封,朝中大臣范质等人被挟迫拜见“新天子”,显德七年,后周恭帝柴宗训禅让帝位于赵匡胤,降封郑王,符太后改称周太后,郭威柴荣的宗族被迁往房州(今湖北省房县)居住。

同年赵匡胤建立宋朝,改元“建隆”。柴宗训在位前后仅六个月。后周亡。

宋太祖頒布聖旨優待帝母子,賜柴氏“丹書鐵券”,保柴氏子孫永享富貴,可免除小罪,宽赦大罪,若犯十恶不赦则钦赐死于牢中(主要是为了维护贵族的体面,不被当街斩首)。

建隆三年(962年)柴宗训被迁往房陵(今湖北省房县)居住,开宝六年(973年)三月卒于当地,终年仅21岁。赵匡胤“闻之震恸”,「素服發哀,輟朝十日」,谥曰“恭皇帝”,归葬于世宗庆陵之侧。

\subsubsection{显德}

\begin{longtable}{|>{\centering\scriptsize}m{2em}|>{\centering\scriptsize}m{1.3em}|>{\centering}m{8.8em}|}
  % \caption{秦王政}\
  \toprule
  \SimHei \normalsize 年数 & \SimHei \scriptsize 公元 & \SimHei 大事件 \tabularnewline
  % \midrule
  \endfirsthead
  \toprule
  \SimHei \normalsize 年数 & \SimHei \scriptsize 公元 & \SimHei 大事件 \tabularnewline
  \midrule
  \endhead
  \midrule
  元年 & 959 & \tabularnewline\hline
  二年 & 960 & \tabularnewline
  \bottomrule
\end{longtable}


%%% Local Variables:
%%% mode: latex
%%% TeX-engine: xetex
%%% TeX-master: "../../Main"
%%% End:
