%% -*- coding: utf-8 -*-
%% Time-stamp: <Chen Wang: 2019-12-24 17:04:43>


\section{后周\tiny(951-960)}

\subsection{简介}

后周(951年-960年)是中国历史上五代十国时期的最后一个朝代,它从951年正月后周太祖郭威灭后汉开国到960年北宋太祖赵匡胤陈桥兵变被取代共经历了三个皇帝,9年。后周的首都是开封。

统治地区包括今河南、山东、山西南部、河北中南部、陕西中部、甘肃东部、湖北北部、以及安徽、江苏的长江以北地区。

郭威自称为周朝虢叔后裔,因此以「周」为国号,史称「后周」,以别于其他以周为国号的政权,又以郭威之姓,别称「郭周」。

后周的开国皇帝郭威是后汉的开国功臣,受后汉高祖刘知远重任。刘知远临死时郭威是他指定的顾命大臣之一,他奉后汉隐帝刘承祐命,平定多次反汉叛变,同时他又能体恤手下,因此深受军队的热爱。刘承祐感到自己受顾命大臣的控制太多,因此开始杀这些大臣。郭威当时领兵在外,闻讯后以清君侧的名义起兵。刘承祐为此将郭威在开封的所有亲属杀害。郭威在仅仅数日内就进入开封。刘承祐死于非命。郭威的军队在开封大掠。郭威首先名义上迎在徐州的刘赟做新皇帝,自己却以攻契丹为名北上,同时他却派部下将刘赟在路上杀死,然后又让自己的士兵拥护自己做皇帝,做出迫不得已的样子。就在这种情况下他依然首先以“监国”为名上任,一个月后才正式持皇帝名。依據五行相生的順序,後漢的「水」德之後為「木」德,因此後周以「木」為王朝德運。

郭威登基后著手进行一系列的改革。首先他减轻和免除了许多徭役,同时也整顿军纪和管理机构内部的腐败和贿赂。

由于刘承祐将郭威在开封的所有亲属杀害,其中包括他兩个儿子与柴荣的三个儿子。郭威死后由其养子(本身是其内侄)柴荣继位,是為后周世宗。柴荣继续郭威的政策,使得后周所控制的地区的经济得到了很大的发展,同时也使得后周的军事得到了强大的发展。

后周的两位皇帝也是中国历史上少有的非常节俭的皇帝,比如郭威死后陵前仅立石碑一块,其陵寢本身也非常简单,连守陵的宦官都没有。

柴荣继位后不久北汉就联合辽朝乘机攻打後周,打算乘后周内部未稳打击其力量,柴荣决定亲征抵禦进攻。他在高平之战中亲临战场,在战役开初不利,己方右翼溃退的情况下扭转战势,击溃北汉。战后后周军队乘胜追击,一直攻到太原。

此后柴荣开始南征,从955年到958年三次亲征南唐,迫使南唐取消皇帝称号,并割讓几乎所有长江以北的地区予後周。

959年柴荣在解除后顾之忧之后再次北上攻辽,在两个月內几乎攻到幽州,但就在此时他突然患病,不得不中止北伐。柴荣此后不久病逝。

后周在这八年內基本上统一了长江以北的中原地区,向北收复了许多被后晋让给契丹的地区。其統治地区恢复和发展了经济生产,為日后北宋统一中国打下了基础。

柴荣死后,其七岁的儿子柴宗训登基。殿前都点检赵匡胤谎稱辽国和北汉进犯,借口率兵到陈桥驿发动陈桥兵变,夺取后周帝位建立北宋。


%% -*- coding: utf-8 -*-
%% Time-stamp: <Chen Wang: 2019-12-24 17:05:56>

\subsection{太祖\tiny(951-954)}

\subsubsection{生平}

周太祖郭威(904年9月10日-954年2月21日),邢州堯山(今天河北省隆堯),汉族,字文仲,小名“郭雀兒”。五代時期後周開國皇帝(951年—954年)。原為五代後漢的樞密使,卻因隐帝疑忌之下,全家被殺。怒而起兵,隐帝死於亂軍之中,郭威不久發動黃旗加身的兵變,建立後周。

《舊五代史》說,有記載其本名為常威,隨母改嫁入郭家,改用郭姓。《新五代史》的紀錄卻剛好相反,曰其母本來是郭姓之妻,後來改嫁常氏。

唐天祐元年七月二十八日(904年9月10日)郭威生於堯山,父為後晉時的順州刺史郭简,母王氏。或說本姓常,幼时随母亲改嫁郭简,故改姓郭。3歲時徙家太原,不久郭简被殺,郭威成為孤兒,由姨母韓氏撫養。他身材魁梧,習武好鬥。其時李繼韜在潞州招募兵勇,郭威前去投軍,得到李繼韜的賞識。郭威在鬧市與一名欺壓市場的屠戶爭執,醉酒而殺之,原應斬首,李继韬怜其才勇,暗中将其释放,而後又召之,任為幕僚。

947年契丹滅後晉,沙陀人劉知遠起兵太原,建國後漢,郭威为邺都(今河北大名县)留守。劉知遠称帝不到一年即死去,其子劉承祐继位,拜郭威為樞密副使。乾祐元年(948年)三月河中(今山西永济)李守贞、永兴(今陕西西安)赵思绾及凤翔(今陕西凤翔)王景崇相继反汉,郭威相繼平定亂事,李守贞自焚,赵思绾投降,王景崇自焚。

乾祐三年(950年)四月,隐帝疑忌大臣,乘郭威在外時下诏將开封城内郭威(当时郭威已有成年的儿子)、柴荣和王峻的全家屠殺殆盡。

清趙翼謂之:“五代亂世,本無刑章,視人命如草芥,動以族誅為事”。 枢密使院吏魏仁浦劝郭威先发制人。同年十一月,郭威发动兵变,隱帝死於亂軍之中。郭威假意迎宗室劉贇為帝,又派部下在路上刺殺劉贇,再演出一番戲碼,讓士兵擁護自己稱帝,建立后周,建都汴京(今河南省開封市),改元广顺。他廣招人才、励精图治,得魏仁浦、李穀、王溥、范質等輔臣。广顺三年(953年),封義子柴榮為晉王。

广顺四年(954年),周太祖郭威去世,享年50歲。因亲生儿子全都被刘承祐杀害,妻侄(外甥)柴榮繼位。


\subsubsection{广顺}

\begin{longtable}{|>{\centering\scriptsize}m{2em}|>{\centering\scriptsize}m{1.3em}|>{\centering}m{8.8em}|}
  % \caption{秦王政}\
  \toprule
  \SimHei \normalsize 年数 & \SimHei \scriptsize 公元 & \SimHei 大事件 \tabularnewline
  % \midrule
  \endfirsthead
  \toprule
  \SimHei \normalsize 年数 & \SimHei \scriptsize 公元 & \SimHei 大事件 \tabularnewline
  \midrule
  \endhead
  \midrule
  元年 & 951 & \tabularnewline\hline
  二年 & 952 & \tabularnewline\hline
  三年 & 953 & \tabularnewline\hline
  四年 & 954 & \tabularnewline
  \bottomrule
\end{longtable}

\subsubsection{显德}

\begin{longtable}{|>{\centering\scriptsize}m{2em}|>{\centering\scriptsize}m{1.3em}|>{\centering}m{8.8em}|}
  % \caption{秦王政}\
  \toprule
  \SimHei \normalsize 年数 & \SimHei \scriptsize 公元 & \SimHei 大事件 \tabularnewline
  % \midrule
  \endfirsthead
  \toprule
  \SimHei \normalsize 年数 & \SimHei \scriptsize 公元 & \SimHei 大事件 \tabularnewline
  \midrule
  \endhead
  \midrule
  元年 & 954 & \tabularnewline
  \bottomrule
\end{longtable}


%%% Local Variables:
%%% mode: latex
%%% TeX-engine: xetex
%%% TeX-master: "../../Main"
%%% End:

%% -*- coding: utf-8 -*-
%% Time-stamp: <Chen Wang: 2021-11-01 15:22:29>

\subsection{世宗柴榮\tiny(954-959)}

\subsubsection{生平}

周世宗柴榮(921年10月27日-959年7月27日),五代時期後周皇帝,於954年2月26日-959年7月27日在位,在位6年。邢州堯山柴家莊(今河北省邢台市隆堯縣)人,是周太祖郭威的養子(柴榮本身是郭威正室柴皇后的侄子),是中國少数由外戚继承宗室的皇帝,廟號世宗,諡號睿武孝文皇帝。根據史書記錄,在此時期在政治、軍事、經濟上都有建樹,號稱英主,他初步奠定了後來北宋的勢力。

父柴守禮,祖父柴翁是當地望族,柴榮年輕時曾隨商人頡跌氏在江陵販賣茶葉,對社會積弊有所體驗。史載其「器貌英奇,善騎射,略通書、史、黃老,性沉重寡言」。

广顺元年(951年),周太祖郭威即位,柴荣授澶州节度使、检校太保,封太原郡侯。史载其境“为政清肃,盗不犯境”。二年,检校太傅为相。又一年,封晋王。

显德元年(954年)正月,判内外兵马事,总揽兵权。同月太祖崩,即帝位。

柴榮即位後,立刻下令招撫流亡,減少賦稅,恢復中原經濟。當時,北方及中原經過了一系列的戰爭,百姓痛苦不堪,柴榮的舉動正好使中原開始復甦,他整頓吏治,使後周政治清明,百姓富庶,經濟開始繁榮。

顯德二年(955年)推行顯德毀佛,以佛寺銅材鑄行「周元通寶」,錢質與鑄量均居五代之冠。因为其毀佛行為,後周世宗被列入毀佛的「三武(北魏太武帝、北周武帝和唐武宗)一宗」。司馬光評述周世宗「毀佛」:「不愛己身而愛民,不以無益廢有益,周世宗算得是仁愛明理之人。」

柴榮對內進行改革,對外則積極開拓疆土。顯德元年(954年)二月,北漢主劉崇乘其新立,勾結遼兵4萬攻後周,柴榮力排馮道勸阻,率軍迎戰,於高平(今山西)南大破北漢軍,穩定政局。戰後整軍練卒,裁汰冗弱,於是軍威大振。顯德二年詔令群臣獻《為君難為臣不易論》、《平邊策》,確定王樸提出的「先南後北」的統一方略;命兵部撰集兵法,名《制旨兵法》。他擊敗後蜀的孟昶,取得秦、鳳、成、階四州,孟昶大懼,「致書請和」;又先後三次征南唐,創建水軍,攻下淮南十四州。

顯德六年三月,圖舉收復燕雲十六州,一連攻陷瀛洲、莫州二州(今河北),莫州刺史劉楚信、瀛洲刺史高彥暉投降,再向北挺進,又連陷益津關、瓦橋關、高陽關三關。五月在議取幽州(今北京)時,柴榮病倒,只好撤退[註 1]。

後周顯德六年(959年)六月,柴榮去世,年僅39歲。由年仅7岁的儿子柴宗训即位,是为周恭帝。

柴榮是五代十國時期最英明的君主,為北宋之疆土奠定了基礎。《舊五代史》稱:“世宗頃在仄微,尤務韜晦……不日破高平之陣,逾年復秦、鳳之封,江北、燕南,取之如拾芥,神武雄略,乃一代之英主也……而降年不永,美志不就,悲夫!”

北汉进犯,柴荣对军士们说:“昔唐太宗定天下,未尝不自行,朕何敢偷安?”两军遇于高平之南,周将士俱,柴荣“介马自临阵督战”、“自引亲兵犯矢石督战”,当晚与将士“宿于野次”。

选武艺超绝者为殿前诸班。使得“征伐四方,所向皆捷”,自中唐以来的冗兵积弊,一扫而光。

“十年开拓天下,十年养百姓,十年致太平。”

针对“私度僧尼,日益猥杂”、“乡村之中,其弊转盛”,下诏:“近览诸州奏闻……私度僧尼,日增猥杂,创修寺院,渐至繁多……宜举旧章,以革前弊……诸道州县镇村坊,应有敕额寺院,一切仍旧,其无敕额者,并抑停废。”诏旨颁布后,废佛之风席卷全国,当年就废去寺院30,336所,僧尼还俗者大约6万人。除重点保护寺院外,一律停废。禁私度僧尼,禁僧俗舍身,并下诏毁铜佛像以铸钱。柴荣说:“卿辈勿以毁佛为疑。夫佛,以善道化人,苟志于善,斯奉佛矣。彼铜像岂所谓佛耶?且吾闻佛在利人,虽头目犹舍以布施。若朕身可以济民,亦非所惜也。”

定税征,保边民,解决了逃户庄田的问题。

均田租,抑豪强。

浚卞口,导河流,江淮舟楫始通。


\subsubsection{显德}

\begin{longtable}{|>{\centering\scriptsize}m{2em}|>{\centering\scriptsize}m{1.3em}|>{\centering}m{8.8em}|}
  % \caption{秦王政}\
  \toprule
  \SimHei \normalsize 年数 & \SimHei \scriptsize 公元 & \SimHei 大事件 \tabularnewline
  % \midrule
  \endfirsthead
  \toprule
  \SimHei \normalsize 年数 & \SimHei \scriptsize 公元 & \SimHei 大事件 \tabularnewline
  \midrule
  \endhead
  \midrule
  元年 & 954 & \tabularnewline\hline
  二年 & 955 & \tabularnewline\hline
  三年 & 956 & \tabularnewline\hline
  四年 & 957 & \tabularnewline\hline
  五年 & 958 & \tabularnewline\hline
  六年 & 959 & \tabularnewline
  \bottomrule
\end{longtable}


%%% Local Variables:
%%% mode: latex
%%% TeX-engine: xetex
%%% TeX-master: "../../Main"
%%% End:

%% -*- coding: utf-8 -*-
%% Time-stamp: <Chen Wang: 2021-11-01 15:22:38>

\subsection{恭帝柴宗训\tiny(959-960)}

\subsubsection{生平}

周恭帝柴宗训(953年9月14日-973年4月6日),五代时期后周皇帝,周世宗第四子。显德六年(959年)封为梁王。世宗于同年六月病死,他于同月甲午日继位,沿用周世宗年号“显德”。

柴宗训即位时,年仅七岁,由符太后垂帘听政,范质、王溥等主持军国大事。柴宗训在位期间,特别重用其父親任命的赵匡胤負責軍事事務。

显德七年(960年)正月元旦,群臣正在朝贺柴宗训时,镇(今河北省正定县)、定(今河北省定县)两州遣人来报,辽国和北汉合兵南侵。范质命令殿前都點檢赵匡胤率领禁军北上抵御。禁军到达开封东北部的陈桥驿后,突然发动兵变,拥赵匡胤为帝,黃袍加身。赵匡胤回师开封,朝中大臣范质等人被挟迫拜见“新天子”,显德七年,后周恭帝柴宗训禅让帝位于赵匡胤,降封郑王,符太后改称周太后,郭威柴荣的宗族被迁往房州(今湖北省房县)居住。

同年赵匡胤建立宋朝,改元“建隆”。柴宗训在位前后仅六个月。后周亡。

宋太祖頒布聖旨優待帝母子,賜柴氏“丹書鐵券”,保柴氏子孫永享富貴,可免除小罪,宽赦大罪,若犯十恶不赦则钦赐死于牢中(主要是为了维护贵族的体面,不被当街斩首)。

建隆三年(962年)柴宗训被迁往房陵(今湖北省房县)居住,开宝六年(973年)三月卒于当地,终年仅21岁。赵匡胤“闻之震恸”,「素服發哀,輟朝十日」,谥曰“恭皇帝”,归葬于世宗庆陵之侧。

\subsubsection{显德}

\begin{longtable}{|>{\centering\scriptsize}m{2em}|>{\centering\scriptsize}m{1.3em}|>{\centering}m{8.8em}|}
  % \caption{秦王政}\
  \toprule
  \SimHei \normalsize 年数 & \SimHei \scriptsize 公元 & \SimHei 大事件 \tabularnewline
  % \midrule
  \endfirsthead
  \toprule
  \SimHei \normalsize 年数 & \SimHei \scriptsize 公元 & \SimHei 大事件 \tabularnewline
  \midrule
  \endhead
  \midrule
  元年 & 959 & \tabularnewline\hline
  二年 & 960 & \tabularnewline
  \bottomrule
\end{longtable}


%%% Local Variables:
%%% mode: latex
%%% TeX-engine: xetex
%%% TeX-master: "../../Main"
%%% End:


%%% Local Variables:
%%% mode: latex
%%% TeX-engine: xetex
%%% TeX-master: "../../Main"
%%% End:
