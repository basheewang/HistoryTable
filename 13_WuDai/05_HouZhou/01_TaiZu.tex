%% -*- coding: utf-8 -*-
%% Time-stamp: <Chen Wang: 2019-12-24 17:05:56>

\subsection{太祖\tiny(951-954)}

\subsubsection{生平}

周太祖郭威(904年9月10日-954年2月21日),邢州堯山(今天河北省隆堯),汉族,字文仲,小名“郭雀兒”。五代時期後周開國皇帝(951年—954年)。原為五代後漢的樞密使,卻因隐帝疑忌之下,全家被殺。怒而起兵,隐帝死於亂軍之中,郭威不久發動黃旗加身的兵變,建立後周。

《舊五代史》說,有記載其本名為常威,隨母改嫁入郭家,改用郭姓。《新五代史》的紀錄卻剛好相反,曰其母本來是郭姓之妻,後來改嫁常氏。

唐天祐元年七月二十八日(904年9月10日)郭威生於堯山,父為後晉時的順州刺史郭简,母王氏。或說本姓常,幼时随母亲改嫁郭简,故改姓郭。3歲時徙家太原,不久郭简被殺,郭威成為孤兒,由姨母韓氏撫養。他身材魁梧,習武好鬥。其時李繼韜在潞州招募兵勇,郭威前去投軍,得到李繼韜的賞識。郭威在鬧市與一名欺壓市場的屠戶爭執,醉酒而殺之,原應斬首,李继韬怜其才勇,暗中将其释放,而後又召之,任為幕僚。

947年契丹滅後晉,沙陀人劉知遠起兵太原,建國後漢,郭威为邺都(今河北大名县)留守。劉知遠称帝不到一年即死去,其子劉承祐继位,拜郭威為樞密副使。乾祐元年(948年)三月河中(今山西永济)李守贞、永兴(今陕西西安)赵思绾及凤翔(今陕西凤翔)王景崇相继反汉,郭威相繼平定亂事,李守贞自焚,赵思绾投降,王景崇自焚。

乾祐三年(950年)四月,隐帝疑忌大臣,乘郭威在外時下诏將开封城内郭威(当时郭威已有成年的儿子)、柴荣和王峻的全家屠殺殆盡。

清趙翼謂之:“五代亂世,本無刑章,視人命如草芥,動以族誅為事”。 枢密使院吏魏仁浦劝郭威先发制人。同年十一月,郭威发动兵变,隱帝死於亂軍之中。郭威假意迎宗室劉贇為帝,又派部下在路上刺殺劉贇,再演出一番戲碼,讓士兵擁護自己稱帝,建立后周,建都汴京(今河南省開封市),改元广顺。他廣招人才、励精图治,得魏仁浦、李穀、王溥、范質等輔臣。广顺三年(953年),封義子柴榮為晉王。

广顺四年(954年),周太祖郭威去世,享年50歲。因亲生儿子全都被刘承祐杀害,妻侄(外甥)柴榮繼位。


\subsubsection{广顺}

\begin{longtable}{|>{\centering\scriptsize}m{2em}|>{\centering\scriptsize}m{1.3em}|>{\centering}m{8.8em}|}
  % \caption{秦王政}\
  \toprule
  \SimHei \normalsize 年数 & \SimHei \scriptsize 公元 & \SimHei 大事件 \tabularnewline
  % \midrule
  \endfirsthead
  \toprule
  \SimHei \normalsize 年数 & \SimHei \scriptsize 公元 & \SimHei 大事件 \tabularnewline
  \midrule
  \endhead
  \midrule
  元年 & 951 & \tabularnewline\hline
  二年 & 952 & \tabularnewline\hline
  三年 & 953 & \tabularnewline\hline
  四年 & 954 & \tabularnewline
  \bottomrule
\end{longtable}

\subsubsection{显德}

\begin{longtable}{|>{\centering\scriptsize}m{2em}|>{\centering\scriptsize}m{1.3em}|>{\centering}m{8.8em}|}
  % \caption{秦王政}\
  \toprule
  \SimHei \normalsize 年数 & \SimHei \scriptsize 公元 & \SimHei 大事件 \tabularnewline
  % \midrule
  \endfirsthead
  \toprule
  \SimHei \normalsize 年数 & \SimHei \scriptsize 公元 & \SimHei 大事件 \tabularnewline
  \midrule
  \endhead
  \midrule
  元年 & 954 & \tabularnewline
  \bottomrule
\end{longtable}


%%% Local Variables:
%%% mode: latex
%%% TeX-engine: xetex
%%% TeX-master: "../../Main"
%%% End:
