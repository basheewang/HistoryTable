%% -*- coding: utf-8 -*-
%% Time-stamp: <Chen Wang: 2019-12-24 16:36:46>

\subsection{太祖\tiny(907-912)}

\subsubsection{生平}

梁太祖朱温(852年12月5日-912年7月18日),五代時期後梁開國皇帝,曾参与黄巢之亂,后降唐为将,唐僖宗賜名朱全忠。但又密谋杀害唐昭宗,立唐哀帝,后废哀帝自立,建立“后梁”,称帝后改名朱晃。晚年大肆荒淫,強姦兒媳。後為三子朱友珪所殺,終年59歲。

朱全忠出生于大中六年(852年)12月5日。朱全忠原名朱温,宋州砀山午沟里(今安徽省砀山县)人。年幼喪父,年少时是一個游手好閒、不务正业的无赖。

乾符四年(877年)朱溫参加黃巢軍,反抗朝廷,屡立战功,很快升为大将。大齐政权建立后,任同州防御使,率军攻打河中。由于屡战屡败,怕受责罰,于是叛变降唐,投歸河中節度使王重榮。唐僖宗任朱溫為左金吾衛大將軍,充河中行营副招讨使,並赐名“全忠”。中和三年(883年)又被授以宣武节度使,随后击败黄巢。

龍紀元年(889年)斬黃巢餘部蔡州節度使秦宗權,被封為東平王。黃巢覆亡後,唐帝國已名存實亡,各藩鎮擁兵自重,其中以宣武節度使朱全忠、河東節度使李克用、鳳翔節度使李茂貞、盧龍節度使劉仁恭、鎮海節度使錢镠、淮南節度副大使楊行密等人勢力最大,史載“郡將自擅,常賦殆絕,藩鎮廢置,不自朝廷”,“王室日卑,號令不出國門”。

天復元年(901年)昭宗被宦官韓全誨幽禁,宰相崔胤乃召朱全忠救駕。韓全誨不得已投靠鳳翔節度使李茂貞,朱全忠進攻鳳翔,鳳翔食盡待援。天復三年(903年),節度使李茂貞殺宦官韓全誨等七十餘人,與朱全忠和解,護送昭宗出城,昭宗又回到長安。崔胤指責宦官“大則構扇藩鎮,傾危國家;小則賣官鬻爵,蠹害朝政”,不久朱全忠盡殺宦官數百人,廢神策軍,完全控制皇室。天復元年(901年)封为梁王。天祐元年(904年),朱全忠殺宰相崔胤,逼迫昭宗遷都洛陽,八月壬寅夜,指使朱友恭、氏叔琮、枢密使蒋玄晖等人殺昭宗,另立其子李柷為帝,是為唐哀帝。天祐二年(905年),在親信李振鼓動下,於滑州白馬驛(今河南滑縣境)一夕殺盡杀宰相裴樞、崔遠等朝臣三十餘人,投屍於河,史稱“白馬之禍”。年末,预备篡位称帝,让宰相柳璨、蒋玄晖谋划受九锡,蒋玄晖与太常卿张廷范认为天下未定不可过急,朱全忠不悦。宣徽副使蒋殷、赵殷衡素与蒋玄晖、张廷范不和,趁机诬告他们与柳璨对何太后盟誓复唐,朱全忠怒,遣使杀蒋玄晖,密令蒋殷、赵殷衡在积善宫缢杀何太后,迫哀帝下诏称太后系秽乱宫闱自杀谢罪,追废为庶人,停新年郊礼。又贬杀柳璨、张廷范。

朱全忠為節度使時,用法苛嚴,大軍交戰時,如將軍戰死,所部士卒則一律斬首,稱「跋隊斬」,自是戰無不勝。而且士卒逃匿州郡,不歸者甚眾,為防士卒逃亡,朱全忠命軍士紋面以記軍號。

開平元年(907年)廢唐哀帝,自行称帝,改名为晃,建都開封,国号为“梁”,史称“后梁”,后人称为后梁太祖。封李柷為濟陰王,次年又殺李柷,自此唐朝結束289年的統治,中國進入五代十國的紛亂時期。

朱全忠在位時頗重視農業發展,下令兩稅法之外不得妄有科配,并曾因侄子朱友谅不恤灾民却进献瑞麦怒罢其官;但因連年戰事,民不聊生,開平四年(910年)發生柏鄉之戰并战败,與晉王李存勗矛盾加劇。晚年宮廷內陷入權力鬥爭。朱溫生性殘暴,殺人如草芥。夫人在世時尚能勸止,死後卻大肆淫亂,甚至亂倫,包括兒媳都得入宮侍寢。乾化二年(912年)被三子朱友珪刺杀,终年61岁,在位6年。

毛泽东评价他说:“朱温处四战之地,与曹操略同,而狡猾过之。”


\subsubsection{开平}

\begin{longtable}{|>{\centering\scriptsize}m{2em}|>{\centering\scriptsize}m{1.3em}|>{\centering}m{8.8em}|}
  % \caption{秦王政}\
  \toprule
  \SimHei \normalsize 年数 & \SimHei \scriptsize 公元 & \SimHei 大事件 \tabularnewline
  % \midrule
  \endfirsthead
  \toprule
  \SimHei \normalsize 年数 & \SimHei \scriptsize 公元 & \SimHei 大事件 \tabularnewline
  \midrule
  \endhead
  \midrule
  元年 & 907 & \tabularnewline\hline
  二年 & 908 & \tabularnewline\hline
  三年 & 909 & \tabularnewline\hline
  四年 & 910 & \tabularnewline\hline
  五年 & 911 & \tabularnewline
  \bottomrule
\end{longtable}

\subsubsection{乾化}

\begin{longtable}{|>{\centering\scriptsize}m{2em}|>{\centering\scriptsize}m{1.3em}|>{\centering}m{8.8em}|}
  % \caption{秦王政}\
  \toprule
  \SimHei \normalsize 年数 & \SimHei \scriptsize 公元 & \SimHei 大事件 \tabularnewline
  % \midrule
  \endfirsthead
  \toprule
  \SimHei \normalsize 年数 & \SimHei \scriptsize 公元 & \SimHei 大事件 \tabularnewline
  \midrule
  \endhead
  \midrule
  元年 & 911 & \tabularnewline\hline
  二年 & 912 & \tabularnewline\hline
  三年 & 913 & \tabularnewline
  \bottomrule
\end{longtable}


%%% Local Variables:
%%% mode: latex
%%% TeX-engine: xetex
%%% TeX-master: "../../Main"
%%% End:
