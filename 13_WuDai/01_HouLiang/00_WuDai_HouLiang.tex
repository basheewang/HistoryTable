%% -*- coding: utf-8 -*-
%% Time-stamp: <Chen Wang: 2019-12-24 16:08:49>


\section{后梁\tiny(907-923)}

\subsection{简介}

後梁(907年-923年)是中国历史上五代十国时期建立的第一个皇朝。自907年梁太祖朱温篡唐称帝建國至923年11月19日梁末帝亡国于后唐,历时17年。因为皇帝为朱姓,为与南北朝时的南梁区别,故又称朱梁。

朱全忠原为唐僖宗在位时代爆发的黄巢之乱将领,原名朱溫。降唐后赐名全忠,任宣武军节度使,据汴州(今河南开封),渐渐成为唐末最强大的藩镇,并受封为梁王。天祐元年(904年)闰四月,自西京长安劫持唐昭宗李晔至东都洛阳,又于八月加害,并另立年仅13岁的李祝为帝,即唐哀帝,并发起白马驿之祸清洗官僚及杀害皇室成员,准备篡位。

唐天祐四年四月十八日(907年6月1日),朱全忠迫使哀帝禅位,以其封爵“梁王”(“梁”之得名来自其祖居砀山为战国时梁国故地)改国号为梁,建元开平,都汴(号开封府),史称后梁,是为梁太祖。根據五行相生的規律,唐朝的「土」德之後為「金」德,因此後梁以「金」為王朝德運。后梁疆土虽是中原五个王朝中最小的一个,辖地仅今河南、山东两省,陕西、湖北大部,河北、宁夏、山西、江苏、安徽等省的一部,但也是存在最久的五代朝,并得到契丹及一些藩镇的尊奉。

朱温篡唐后,很多藩镇均不承认后梁,仍用唐年号。次年(908年)蜀王王建也称帝,建立了前蜀。当时有些割据势力表示归顺后梁,朱温遂晋封湖南马殷为楚王,两浙钱镠为吴越王,广东刘隐为靜海軍節度使,福建王审知为闽王。同年,太祖杀唐哀帝。開平三年(909年),幽州刘守光自稱幽州節度使,建立刘燕。而位於江浙一帶的淮南節度使楊行密則早在天復二年(902年)就已受封為吳王,建立吳。连同后梁,十一个割据势力并存。

由于过去朱温与李克用在公私事上均有有恩怨,所以自建国起,后梁与晋王李克用、李存勗持续战斗,直至亡国。后梁建立后发兵8万,打算收复被李克用占据的潞州,但围攻半年不下。次年初李克用死,李存勗继为晋王,亲率晋军为潞州解围,大获全胜。

梁太祖疑功臣,迫使镇州(今河北正定)王鎔和定州(今属河北)王处直,于开平四年(910年)起兵反梁,并向晋王求援。乾化元年(911年)初,李存勗率晋军击梁军于柏乡(今属河北),经过一日激战后,梁军大败。晋军追击150余里,直至邢州(今河北邢台),又连克澶州(今河南濮阳)、新乡(今属河南)等地。梁太祖亦亲自率军前往洛阳设防。柏乡之战梁军主力受损,后梁处于劣势。

次年,梁太祖趁晋攻燕,晋、赵、定州,带病亲率军北上,号称50万大军。昼夜兼行,至下博(今河北深州),率军5万转攻蓨县(今景县)。其时晋军主力北攻幽州,南方空虚,僅有符存審的少量兵力鎮守赵州(今河北赵县),但他用計以小部队骚扰梁军,又派数百骑兵伪装梁军,趁夜袭了梁太祖营寨,外加被晋军释放的梁军士兵,归来后传言晋王李存勗亲率大军来攻,梁太祖惊惶失措,遂烧营夜遁。

乾化二年(912年)五月,梁太祖退至洛阳,病入膏肓,同年六月,为次子朱友珪所杀。次年,朱友珪又为禁軍所杀,梁太祖四子东都留守朱友贞遂在东都开封继位,是为梁末帝。後梁内乱相继,只有大将杨师厚率军与晋、赵周旋于河北。

贞明元年(915年)春,杨师厚死,梁朝廷密謀把杨师厚的領地一分為二,魏州(今河北大名县东北)军士叛降于晋,晋王李存勗乃亲征出兵太行黄泽岭(今山西左权东南),又袭德州(今属山东)、澶州,梁军连战皆败。次年春,梁末帝命王檀率军3万北上,直奔太原,企图袭取晋军基地,但为守城军击败。

贞明四年(918年)八月,晋王李存勗举兵魏州南下,想要灭梁,与梁军相持于濮州一带。十二月下旬,晋军至胡柳陂(今濮阳西南),贺瓌率梁军跟踪而至,两军激战,梁军骑军王彦章败,西逃时冲散了晋军的西线部队,晋名将周德威战死。另一名將李存審與李嗣昭、王建及率領骑兵反攻冲击梁軍步兵,后梁惨败,伤亡近3万。但晋军也因此战元气大伤,梁晋战争沉寂了一段时期。

龙德元年(921年)春,晋王李存勗正拟称帝之际,镇州王鎔为部下张文礼所杀。张文礼勾结後梁与契丹。晋军进围镇州时,梁军袭击晋军,却反为晋军所败,死伤2万多人。

龙德三年(923年),晋王李存勗称帝,国号大唐,史称後唐。

龙德三年闰四月末,后唐乘后梁西攻泽州(今山西晋城),派将李嗣源率骑5000袭郓州(今山东东平),次日清晨占之。

后梁启用王彦章为帅,段凝为副帅,调集精兵10万北讨后唐。李存勗亲自率军与梁军苦战于杨刘(今东阿)。后王彦章兵败中都县(今山东汶上)被俘斩。923年11月19日后唐军达开封城下,开封随即降唐,后梁亡。梁末帝自杀。


%% -*- coding: utf-8 -*-
%% Time-stamp: <Chen Wang: 2021-11-01 15:19:50>

\subsection{太祖朱温\tiny(907-912)}

\subsubsection{生平}

梁太祖朱温(852年12月5日-912年7月18日),五代時期後梁開國皇帝,曾参与黄巢之亂,后降唐为将,唐僖宗賜名朱全忠。但又密谋杀害唐昭宗,立唐哀帝,后废哀帝自立,建立“后梁”,称帝后改名朱晃。晚年大肆荒淫,強姦兒媳。後為三子朱友珪所殺,終年59歲。

朱全忠出生于大中六年(852年)12月5日。朱全忠原名朱温,宋州砀山午沟里(今安徽省砀山县)人。年幼喪父,年少时是一個游手好閒、不务正业的无赖。

乾符四年(877年)朱溫参加黃巢軍,反抗朝廷,屡立战功,很快升为大将。大齐政权建立后,任同州防御使,率军攻打河中。由于屡战屡败,怕受责罰,于是叛变降唐,投歸河中節度使王重榮。唐僖宗任朱溫為左金吾衛大將軍,充河中行营副招讨使,並赐名“全忠”。中和三年(883年)又被授以宣武节度使,随后击败黄巢。

龍紀元年(889年)斬黃巢餘部蔡州節度使秦宗權,被封為東平王。黃巢覆亡後,唐帝國已名存實亡,各藩鎮擁兵自重,其中以宣武節度使朱全忠、河東節度使李克用、鳳翔節度使李茂貞、盧龍節度使劉仁恭、鎮海節度使錢镠、淮南節度副大使楊行密等人勢力最大,史載“郡將自擅,常賦殆絕,藩鎮廢置,不自朝廷”,“王室日卑,號令不出國門”。

天復元年(901年)昭宗被宦官韓全誨幽禁,宰相崔胤乃召朱全忠救駕。韓全誨不得已投靠鳳翔節度使李茂貞,朱全忠進攻鳳翔,鳳翔食盡待援。天復三年(903年),節度使李茂貞殺宦官韓全誨等七十餘人,與朱全忠和解,護送昭宗出城,昭宗又回到長安。崔胤指責宦官“大則構扇藩鎮,傾危國家;小則賣官鬻爵,蠹害朝政”,不久朱全忠盡殺宦官數百人,廢神策軍,完全控制皇室。天復元年(901年)封为梁王。天祐元年(904年),朱全忠殺宰相崔胤,逼迫昭宗遷都洛陽,八月壬寅夜,指使朱友恭、氏叔琮、枢密使蒋玄晖等人殺昭宗,另立其子李柷為帝,是為唐哀帝。天祐二年(905年),在親信李振鼓動下,於滑州白馬驛(今河南滑縣境)一夕殺盡杀宰相裴樞、崔遠等朝臣三十餘人,投屍於河,史稱“白馬之禍”。年末,预备篡位称帝,让宰相柳璨、蒋玄晖谋划受九锡,蒋玄晖与太常卿张廷范认为天下未定不可过急,朱全忠不悦。宣徽副使蒋殷、赵殷衡素与蒋玄晖、张廷范不和,趁机诬告他们与柳璨对何太后盟誓复唐,朱全忠怒,遣使杀蒋玄晖,密令蒋殷、赵殷衡在积善宫缢杀何太后,迫哀帝下诏称太后系秽乱宫闱自杀谢罪,追废为庶人,停新年郊礼。又贬杀柳璨、张廷范。

朱全忠為節度使時,用法苛嚴,大軍交戰時,如將軍戰死,所部士卒則一律斬首,稱「跋隊斬」,自是戰無不勝。而且士卒逃匿州郡,不歸者甚眾,為防士卒逃亡,朱全忠命軍士紋面以記軍號。

開平元年(907年)廢唐哀帝,自行称帝,改名为晃,建都開封,国号为“梁”,史称“后梁”,后人称为后梁太祖。封李柷為濟陰王,次年又殺李柷,自此唐朝結束289年的統治,中國進入五代十國的紛亂時期。

朱全忠在位時頗重視農業發展,下令兩稅法之外不得妄有科配,并曾因侄子朱友谅不恤灾民却进献瑞麦怒罢其官;但因連年戰事,民不聊生,開平四年(910年)發生柏鄉之戰并战败,與晉王李存勗矛盾加劇。晚年宮廷內陷入權力鬥爭。朱溫生性殘暴,殺人如草芥。夫人在世時尚能勸止,死後卻大肆淫亂,甚至亂倫,包括兒媳都得入宮侍寢。乾化二年(912年)被三子朱友珪刺杀,终年61岁,在位6年。

毛泽东评价他说:“朱温处四战之地,与曹操略同,而狡猾过之。”


\subsubsection{开平}

\begin{longtable}{|>{\centering\scriptsize}m{2em}|>{\centering\scriptsize}m{1.3em}|>{\centering}m{8.8em}|}
  % \caption{秦王政}\
  \toprule
  \SimHei \normalsize 年数 & \SimHei \scriptsize 公元 & \SimHei 大事件 \tabularnewline
  % \midrule
  \endfirsthead
  \toprule
  \SimHei \normalsize 年数 & \SimHei \scriptsize 公元 & \SimHei 大事件 \tabularnewline
  \midrule
  \endhead
  \midrule
  元年 & 907 & \tabularnewline\hline
  二年 & 908 & \tabularnewline\hline
  三年 & 909 & \tabularnewline\hline
  四年 & 910 & \tabularnewline\hline
  五年 & 911 & \tabularnewline
  \bottomrule
\end{longtable}

\subsubsection{乾化}

\begin{longtable}{|>{\centering\scriptsize}m{2em}|>{\centering\scriptsize}m{1.3em}|>{\centering}m{8.8em}|}
  % \caption{秦王政}\
  \toprule
  \SimHei \normalsize 年数 & \SimHei \scriptsize 公元 & \SimHei 大事件 \tabularnewline
  % \midrule
  \endfirsthead
  \toprule
  \SimHei \normalsize 年数 & \SimHei \scriptsize 公元 & \SimHei 大事件 \tabularnewline
  \midrule
  \endhead
  \midrule
  元年 & 911 & \tabularnewline\hline
  二年 & 912 & \tabularnewline\hline
  三年 & 913 & \tabularnewline
  \bottomrule
\end{longtable}


%%% Local Variables:
%%% mode: latex
%%% TeX-engine: xetex
%%% TeX-master: "../../Main"
%%% End:

%% -*- coding: utf-8 -*-
%% Time-stamp: <Chen Wang: 2021-11-01 15:20:00>

\subsection{郢王朱友珪\tiny(912-913)}

\subsubsection{生平}

朱友珪(885年-913年),小字遙喜,五代時期後梁皇帝,為後梁太祖朱全忠之第三子,弑父自立。登基后不得民心,为袁象先所杀。

朱友珪之母為亳州營妓,唐僖宗光啟年間(885年-888年),朱温有一次率軍經過亳州,召其母陪侍,並且使之懷孕,朱温離去後,其母差人告以生男,朱温大喜,遂名遙喜,後來為朱温接回。朱温篡唐後,將他封為郢王。後梁開平四年(910年)被任命為左右控鶴都指揮使。

朱温晚年,長子郴王朱友裕已死;次子博王朱友文本名康勤,是朱温的義子;三子即朱友珪,時為實際上的長子;四子均王朱友貞。朱温自妻子张氏過世後,就開始縱情聲色,荒淫無度,甚至不顧倫理,經常召諸子之妻入宮陪侍。朱友珪之妻張氏貌美,亦被朱温召去同寝。但後来義子朱友文之妻王氏也入宮和朱温通姦,并特别得到朱温寵愛,在王氏的煽动下,朱温有了改以朱友文繼位的打算。

乾化二年(912年),朱温病重,命王氏召朱友文託付後事,张氏急忙把這件事告訴朱友珪。朱友珪遂率所部政變,由僕夫馮廷諤殺朱温,並假傳遺詔,自登帝位。第二年(913年)改年號為鳳曆。

朱友珪登帝位後,雖然大量賞賜將兵以圖收買人心,然而很多老將還是頗為不平,而朱友珪本人又荒淫無度,因此人心沸騰。鳳曆元年(913年),朱温之婿趙巖、朱温之甥袁象先、均王朱友貞、將領楊師厚等人密謀政變。袁象先首先發難,率禁軍數千人殺入宮中,朱友珪無法逃脫,遂命馮廷諤將他及张皇后都殺死。死後被追廢為庶人。


\subsubsection{凤历}

\begin{longtable}{|>{\centering\scriptsize}m{2em}|>{\centering\scriptsize}m{1.3em}|>{\centering}m{8.8em}|}
  % \caption{秦王政}\
  \toprule
  \SimHei \normalsize 年数 & \SimHei \scriptsize 公元 & \SimHei 大事件 \tabularnewline
  % \midrule
  \endfirsthead
  \toprule
  \SimHei \normalsize 年数 & \SimHei \scriptsize 公元 & \SimHei 大事件 \tabularnewline
  \midrule
  \endhead
  \midrule
  元年 & 913 & \tabularnewline
  \bottomrule
\end{longtable}


%%% Local Variables:
%%% mode: latex
%%% TeX-engine: xetex
%%% TeX-master: "../../Main"
%%% End:

%% -*- coding: utf-8 -*-
%% Time-stamp: <Chen Wang: 2021-11-01 15:20:36>

\subsection{末帝朱友贞\tiny(913-923)}

\subsubsection{生平}

朱友貞(888年-923年11月18日),後改名朱瑱、朱鍠,五代時期後梁皇帝,為後梁太祖朱全忠之第四子,也是朱全忠的嫡子,母亲张夫人。朱友珪异母弟。亡国后,被后唐追废为庶人。朱友貞在位10年,在五代諸帝中在位年期最長。

朱友貞在朱全忠篡唐後,被封為均王。後梁開平四年(910年)被任命為東京馬步軍都指揮使。

後梁乾化二年(912年),郢王朱友珪弒朱全忠自立,大量封賞將兵以圖收買人心,朱友貞當時亦被任命為東京(大梁,今河南開封)留守,開封府尹。然而包括朱友貞在內的眾多官員、將領仍然對朱友珪的行為十分不滿。次年(913年),朱友貞與朱全忠之婿趙巖、朱全忠之甥袁象先、將領楊師厚等人密謀政變。袁象先首先發難,率禁軍數千入殺入宮中,朱友珪無法逃脫,由左右將其殺死。朱友貞遂在大梁稱帝,取消朱友珪的鳳曆年號,仍使用朱全忠的乾化年號。

朱友貞雖然登上帝位,但是他接手的是一個外患內亂即將不斷引爆的帝國。北方的晉王國,於朱友貞登位的同年(913年),在晉王李存勗的率領之下,滅桀燕。915年,朱友貞改元貞明。同年,天雄節度使(魏博節度使)楊師厚去世,魏博自唐末即以地廣兵強著稱,朱友貞藉此機會分割天雄軍,不料卻引起天雄軍官兵的叛變,歸降晉王國。

該年(915年),朱友貞亦被康王朱友敬(一作朱友孜)派人行刺,此事件之後,朱友貞漸漸疏遠宗室,只信任心腹的幾個人。

貞明二年(916年),在與晉的數場會戰敗北後,後梁黃河以北之地幾乎全部喪失。之後的數年間,後梁與晉持續爭戰,然而勝少敗多,領土不斷地被蠶食。

貞明七年(921年),朱友貞改元龍德。龍德三年(923年),已即後唐帝位的李存勗率軍對後梁發動總攻,勢如破竹,朱友貞在後唐軍攻入大梁的前夕,命控鶴都將皇甫麟將他殺死,後梁亦隨之亡國。

\subsubsection{乾化}

\begin{longtable}{|>{\centering\scriptsize}m{2em}|>{\centering\scriptsize}m{1.3em}|>{\centering}m{8.8em}|}
  % \caption{秦王政}\
  \toprule
  \SimHei \normalsize 年数 & \SimHei \scriptsize 公元 & \SimHei 大事件 \tabularnewline
  % \midrule
  \endfirsthead
  \toprule
  \SimHei \normalsize 年数 & \SimHei \scriptsize 公元 & \SimHei 大事件 \tabularnewline
  \midrule
  \endhead
  \midrule
  元年 & 913 & \tabularnewline\hline
  二年 & 914 & \tabularnewline\hline
  三年 & 915 & \tabularnewline
  \bottomrule
\end{longtable}

\subsubsection{贞明}

\begin{longtable}{|>{\centering\scriptsize}m{2em}|>{\centering\scriptsize}m{1.3em}|>{\centering}m{8.8em}|}
  % \caption{秦王政}\
  \toprule
  \SimHei \normalsize 年数 & \SimHei \scriptsize 公元 & \SimHei 大事件 \tabularnewline
  % \midrule
  \endfirsthead
  \toprule
  \SimHei \normalsize 年数 & \SimHei \scriptsize 公元 & \SimHei 大事件 \tabularnewline
  \midrule
  \endhead
  \midrule
  元年 & 915 & \tabularnewline\hline
  二年 & 916 & \tabularnewline\hline
  三年 & 917 & \tabularnewline\hline
  四年 & 918 & \tabularnewline\hline
  五年 & 919 & \tabularnewline\hline
  六年 & 920 & \tabularnewline\hline
  七年 & 921 & \tabularnewline
  \bottomrule
\end{longtable}

\subsubsection{龙德}

\begin{longtable}{|>{\centering\scriptsize}m{2em}|>{\centering\scriptsize}m{1.3em}|>{\centering}m{8.8em}|}
  % \caption{秦王政}\
  \toprule
  \SimHei \normalsize 年数 & \SimHei \scriptsize 公元 & \SimHei 大事件 \tabularnewline
  % \midrule
  \endfirsthead
  \toprule
  \SimHei \normalsize 年数 & \SimHei \scriptsize 公元 & \SimHei 大事件 \tabularnewline
  \midrule
  \endhead
  \midrule
  元年 & 921 & \tabularnewline\hline
  二年 & 922 & \tabularnewline\hline
  三年 & 923 & \tabularnewline
  \bottomrule
\end{longtable}


%%% Local Variables:
%%% mode: latex
%%% TeX-engine: xetex
%%% TeX-master: "../../Main"
%%% End:


%%% Local Variables:
%%% mode: latex
%%% TeX-engine: xetex
%%% TeX-master: "../../Main"
%%% End:
