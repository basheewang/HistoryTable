%% -*- coding: utf-8 -*-
%% Time-stamp: <Chen Wang: 2021-11-01 15:20:00>

\subsection{郢王朱友珪\tiny(912-913)}

\subsubsection{生平}

朱友珪(885年-913年),小字遙喜,五代時期後梁皇帝,為後梁太祖朱全忠之第三子,弑父自立。登基后不得民心,为袁象先所杀。

朱友珪之母為亳州營妓,唐僖宗光啟年間(885年-888年),朱温有一次率軍經過亳州,召其母陪侍,並且使之懷孕,朱温離去後,其母差人告以生男,朱温大喜,遂名遙喜,後來為朱温接回。朱温篡唐後,將他封為郢王。後梁開平四年(910年)被任命為左右控鶴都指揮使。

朱温晚年,長子郴王朱友裕已死;次子博王朱友文本名康勤,是朱温的義子;三子即朱友珪,時為實際上的長子;四子均王朱友貞。朱温自妻子张氏過世後,就開始縱情聲色,荒淫無度,甚至不顧倫理,經常召諸子之妻入宮陪侍。朱友珪之妻張氏貌美,亦被朱温召去同寝。但後来義子朱友文之妻王氏也入宮和朱温通姦,并特别得到朱温寵愛,在王氏的煽动下,朱温有了改以朱友文繼位的打算。

乾化二年(912年),朱温病重,命王氏召朱友文託付後事,张氏急忙把這件事告訴朱友珪。朱友珪遂率所部政變,由僕夫馮廷諤殺朱温,並假傳遺詔,自登帝位。第二年(913年)改年號為鳳曆。

朱友珪登帝位後,雖然大量賞賜將兵以圖收買人心,然而很多老將還是頗為不平,而朱友珪本人又荒淫無度,因此人心沸騰。鳳曆元年(913年),朱温之婿趙巖、朱温之甥袁象先、均王朱友貞、將領楊師厚等人密謀政變。袁象先首先發難,率禁軍數千人殺入宮中,朱友珪無法逃脫,遂命馮廷諤將他及张皇后都殺死。死後被追廢為庶人。


\subsubsection{凤历}

\begin{longtable}{|>{\centering\scriptsize}m{2em}|>{\centering\scriptsize}m{1.3em}|>{\centering}m{8.8em}|}
  % \caption{秦王政}\
  \toprule
  \SimHei \normalsize 年数 & \SimHei \scriptsize 公元 & \SimHei 大事件 \tabularnewline
  % \midrule
  \endfirsthead
  \toprule
  \SimHei \normalsize 年数 & \SimHei \scriptsize 公元 & \SimHei 大事件 \tabularnewline
  \midrule
  \endhead
  \midrule
  元年 & 913 & \tabularnewline
  \bottomrule
\end{longtable}


%%% Local Variables:
%%% mode: latex
%%% TeX-engine: xetex
%%% TeX-master: "../../Main"
%%% End:
