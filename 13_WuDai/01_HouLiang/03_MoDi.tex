%% -*- coding: utf-8 -*-
%% Time-stamp: <Chen Wang: 2019-12-24 16:38:20>

\subsection{朱友贞\tiny(913-923)}

\subsubsection{生平}

朱友貞(888年-923年11月18日),後改名朱瑱、朱鍠,五代時期後梁皇帝,為後梁太祖朱全忠之第四子,也是朱全忠的嫡子,母亲张夫人。朱友珪异母弟。亡国后,被后唐追废为庶人。朱友貞在位10年,在五代諸帝中在位年期最長。

朱友貞在朱全忠篡唐後,被封為均王。後梁開平四年(910年)被任命為東京馬步軍都指揮使。

後梁乾化二年(912年),郢王朱友珪弒朱全忠自立,大量封賞將兵以圖收買人心,朱友貞當時亦被任命為東京(大梁,今河南開封)留守,開封府尹。然而包括朱友貞在內的眾多官員、將領仍然對朱友珪的行為十分不滿。次年(913年),朱友貞與朱全忠之婿趙巖、朱全忠之甥袁象先、將領楊師厚等人密謀政變。袁象先首先發難,率禁軍數千入殺入宮中,朱友珪無法逃脫,由左右將其殺死。朱友貞遂在大梁稱帝,取消朱友珪的鳳曆年號,仍使用朱全忠的乾化年號。

朱友貞雖然登上帝位,但是他接手的是一個外患內亂即將不斷引爆的帝國。北方的晉王國,於朱友貞登位的同年(913年),在晉王李存勗的率領之下,滅桀燕。915年,朱友貞改元貞明。同年,天雄節度使(魏博節度使)楊師厚去世,魏博自唐末即以地廣兵強著稱,朱友貞藉此機會分割天雄軍,不料卻引起天雄軍官兵的叛變,歸降晉王國。

該年(915年),朱友貞亦被康王朱友敬(一作朱友孜)派人行刺,此事件之後,朱友貞漸漸疏遠宗室,只信任心腹的幾個人。

貞明二年(916年),在與晉的數場會戰敗北後,後梁黃河以北之地幾乎全部喪失。之後的數年間,後梁與晉持續爭戰,然而勝少敗多,領土不斷地被蠶食。

貞明七年(921年),朱友貞改元龍德。龍德三年(923年),已即後唐帝位的李存勗率軍對後梁發動總攻,勢如破竹,朱友貞在後唐軍攻入大梁的前夕,命控鶴都將皇甫麟將他殺死,後梁亦隨之亡國。

\subsubsection{乾化}

\begin{longtable}{|>{\centering\scriptsize}m{2em}|>{\centering\scriptsize}m{1.3em}|>{\centering}m{8.8em}|}
  % \caption{秦王政}\
  \toprule
  \SimHei \normalsize 年数 & \SimHei \scriptsize 公元 & \SimHei 大事件 \tabularnewline
  % \midrule
  \endfirsthead
  \toprule
  \SimHei \normalsize 年数 & \SimHei \scriptsize 公元 & \SimHei 大事件 \tabularnewline
  \midrule
  \endhead
  \midrule
  元年 & 913 & \tabularnewline\hline
  二年 & 914 & \tabularnewline\hline
  三年 & 915 & \tabularnewline
  \bottomrule
\end{longtable}

\subsubsection{贞明}

\begin{longtable}{|>{\centering\scriptsize}m{2em}|>{\centering\scriptsize}m{1.3em}|>{\centering}m{8.8em}|}
  % \caption{秦王政}\
  \toprule
  \SimHei \normalsize 年数 & \SimHei \scriptsize 公元 & \SimHei 大事件 \tabularnewline
  % \midrule
  \endfirsthead
  \toprule
  \SimHei \normalsize 年数 & \SimHei \scriptsize 公元 & \SimHei 大事件 \tabularnewline
  \midrule
  \endhead
  \midrule
  元年 & 915 & \tabularnewline\hline
  二年 & 916 & \tabularnewline\hline
  三年 & 917 & \tabularnewline\hline
  四年 & 918 & \tabularnewline\hline
  五年 & 919 & \tabularnewline\hline
  六年 & 920 & \tabularnewline\hline
  七年 & 921 & \tabularnewline
  \bottomrule
\end{longtable}

\subsubsection{龙德}

\begin{longtable}{|>{\centering\scriptsize}m{2em}|>{\centering\scriptsize}m{1.3em}|>{\centering}m{8.8em}|}
  % \caption{秦王政}\
  \toprule
  \SimHei \normalsize 年数 & \SimHei \scriptsize 公元 & \SimHei 大事件 \tabularnewline
  % \midrule
  \endfirsthead
  \toprule
  \SimHei \normalsize 年数 & \SimHei \scriptsize 公元 & \SimHei 大事件 \tabularnewline
  \midrule
  \endhead
  \midrule
  元年 & 921 & \tabularnewline\hline
  二年 & 922 & \tabularnewline\hline
  三年 & 923 & \tabularnewline
  \bottomrule
\end{longtable}


%%% Local Variables:
%%% mode: latex
%%% TeX-engine: xetex
%%% TeX-master: "../../Main"
%%% End:
