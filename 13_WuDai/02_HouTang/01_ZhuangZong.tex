%% -*- coding: utf-8 -*-
%% Time-stamp: <Chen Wang: 2019-12-24 16:43:09>

\subsection{庄宗\tiny(923-926)}

\subsubsection{生平}


唐莊宗李存勗xù(885年12月2日-926年5月15日),山西应县人,沙陀族,本姓朱邪,因其父是河東節度使李克用受唐懿宗赐以李姓,而改姓李,諱存勗,唐光启元年正月(885年12月)生于山西应县,五代時期后唐开国皇帝。小名“亚子”,藝名“李天下”,以勇猛闻名。

923年5月13日在魏州(河北大名府)称帝,国号唐,史称后唐。後因義兄李嗣源被軍士擁戴造反,揮軍直取洛陽。宮中指揮使郭从谦為報仇,趁機發動兵變——興教門之變,將存勗殺害。

《旧五代史》记载:庄宗光圣神闵孝皇帝,讳存勖,武皇帝之长子也。母曰贞简皇后曹氏,以唐光启元年岁在乙巳,冬十月二十二日癸亥,生帝于晋阳宫。《册府元龟》记载:后唐庄宗以光启元年十月癸亥生于晋阳宫。《旧唐书》记载:十二月辛亥朔。李存勖以唐光启元年十月二十二日(885年12月2日)生于晋阳宫,而十月二十二日不是癸亥是癸酉,十月十二日才是癸亥。同光元年至三年的万寿节(李存勖生日)皆在十月二十二日,李存勖的生日就是十月二十二日。光启元年十月壬子朔,癸亥是十二日,癸酉是二十二日,李存勖若生于十二日何以过二十二日生日,此当是编纂者失察误抄所致。

李存勗是李克用與貞簡皇后曹氏的長子。他自幼擅長騎馬射箭,膽力過人,為李克用所寵愛。少年時隨父作戰,11歲就與父親到長安向唐朝朝廷報功,得到唐昭宗的賞賜和誇獎。

李存勗成年後狀貌雄偉瑰麗,得習《春秋》,豁達而且通大義,並勇敢善戰,熟知戰略要術。他又喜愛音樂、歌舞、俳優之戲,旁人多有異談。當時,軍閥割據混戰、佔據河東的李克用常被控制河南的朱溫牽制圍困,兵力不足,地盤狹小,非常悲觀。李存勗勸說其父:“朱全忠恃其武力,吞滅四鄰,想篡奪帝位,這是自取滅亡。我們千萬不可灰心喪氣,要積蓄力量,等待時機”。李克用聽後大為高興,重新振作起來,與朱全忠對抗。

後梁開平二年(908年)正月,李克用病死,李存勗於同月襲晉王位。但是當時的兵馬大權歸於其叔父李克寧,軍民之事皆由李克寧決定,權柄既重,令眾人皆攀附李克寧。當辦完喪事後,李存勗與張承業、李存璋設計,要除去勢力龐大的叔父李克寧。同年二月二十日,當諸將於府第時,乃伏兵於府中,置酒大會,李克寧既至,於席間擒下李存顥、李克寧二人,李存勗哭著責備李克寧:「姪兒一開始就打算把軍隊、政權都讓給叔父,叔父不願意背棄我父親的遺命,怎麼現在又把我跟我母親丟給豺狼虎豹?叔父怎麼忍心?」李克寧泣對:「這是讒言啊,我還能說甚麼?」當日,李克寧與李存顥俱伏法。

其後,李存勗認為潞州(今山西上黨)是河東屏障,沒有潞州對河東不利,所以他立即率軍從晉陽出發,直取上黨,乘大霧突襲圍潞州的梁軍,大獲全勝。李存勗的用兵之奇使梁太祖朱溫大驚,他說:「生兒子就要生李存勗一樣的兒子,李克用不會滅亡了啊!至於我的兒子,豬狗之輩而已!」

當潞州之圍解決後,河東威振,控制鎮州的王鎔和控制定州的王處直見形勢驟變,也動搖了依附後梁的信心,竟然和李存勗結成聯盟共同對付後梁。後梁為了保護河北之地,不惜一切,出兵再戰,於是雙方在柏鄉又展開了一場血戰。柏鄉之戰中,晉軍有周德威等三千騎兵和鎮州、定州兵;對方梁軍有王景仁率領的禁軍和魏博兵八萬。梁軍守衛柏鄉、以逸待勞,在地形、兵力、裝備幾方面處於優勢;而晉軍是騎兵,機動性和進攻能力大,對梁軍構成威脅。戰役開始,李存勗採用周德威建議,引誘梁兵出城,聚而殲之,晉軍主動後撤。梁軍主將王景仁果然上當,傾巢而出。晉軍抓住機會,以騎兵猛烈突擊梁軍,周德威攻右翼,李嗣源攻左翼,鼓譟而進。這時晉軍李存璋率領的騎兵大隊也趕上,梁軍丟盔棄甲,死傷殆盡。這一仗,使梁軍喪失了對河北的控制權,之後,朱溫一聽晉軍就談虎色變。而李存勗卻進一步安定了河東局勢,他息兵行賞,任用賢才,懲治貪官惡吏,寬刑減賦,一時河東大治。

李克用臨死時,交給李存勗三支箭,囑咐他要完成三件大事:一是討伐劉仁恭,攻克幽州;二是征討契丹,解除北方邊境的威脅;第三件大事就是要消滅世敵朱溫。他將三支箭供奉在家廟裡,每臨出征就派人取來,放在精製的絲套裡,帶著上陣,打了勝仗,又送回家廟,表示完成了任務。其後李存勗達成李克用遺志,打敗契丹,攻破燕地,並且攻滅劉守光與劉仁恭父子割據的桀燕政權,並且於923年,在魏州(河北大名縣西)稱帝,國號為唐,史稱後唐,其後攻滅後梁,統一北方。李存勗還收降了李茂貞建立的岐,並攻滅王建所建立的前蜀。

李存勗以唐朝赐姓为李的合法继承人身份,打起中兴唐朝的旗号,并为唐朝皇帝立庙。又以诛灭唐朝逆臣之名,族灭了后梁宰相敬翔、李振等人,将帮助朱溫篡唐的旧臣11人贬官。

但李存勗到了晚年自認為已經拚命一生,應該好好享樂,遂荒廢朝政。李存勗自幼喜歡看戲、演戲,常粉墨登場,並自號藝名“李天下”。伶人大受皇帝寵幸,以至于伶人景进干预朝政。士大夫皆气愤,又不敢出气。李存勗又派伶人、宦官搶民女入宮,強擄魏博士卒們妻女千餘人,怨聲四起。同光二年,李存勗恢复旧唐宦官的势力,本来已经消失的监军又凌驾于藩镇之上,导致诸将更大的不满。同光三年(925年),李存勗派遣兒子魏王李继岌、侍中郭崇韬,攻滅前蜀。但是其後继岌、崇韬互相猜疑。郭崇韬又得罪宦官,李存勗於是对崇韬起疑,下命孟知祥入蜀,见机行事。翌年,李存勗被宦官的谗言所迷惑,诛杀了朱友谦、李存乂。后唐朝廷人心惶惶。

後唐同光四年(926年),魏博士兵皇甫暉在鄴城叛亂,是為鄴城之亂,李存勗命李绍荣前往討伐,久不能下,无奈命李嗣源攻鄴城,李嗣源命其女婿石敬瑭同征。兵進魏州時,李嗣源卻被叛軍擁戴,恭迎入城,李嗣源百口莫辯,石敬瑭表示就算不造反也無法免責,李嗣源因而擁兵自立,與魏博的叛軍合兵造反。李嗣源占據汴州(今河南開封),進軍洛邑,先鋒石敬瑭則帶兵逼進汜水關(河南滎陽汜水鎮),李存勗決定親征反擊。

這時擔任指揮使的伶人郭从谦不知李存乂已被莊宗殺死,欲奉李存乂之名作乱,火燒興教門。蕃汉马步使朱守殷见危不救。李存勗當時僅有符彥卿及王全斌等少數將領效忠他。郭从谦率兵攻入皇城。李存勗被流箭射中。王全斌將其扶至絳霄殿。李存勗失血過多,渴懑求飲,經宦官奉進酪漿,喝完一杯,遽爾殞命。王全斌大慟而去。一名伶人揀丟棄的樂器放在李存勗屍體上,點火焚屍。史稱興教門之變。李嗣源入洛陽杀尽叛臣,葬李存勗屍骨于雍陵,進廟號莊宗,李嗣源在汴州稱帝,是為後唐明宗。

李存勗稱帝即位之前,和后梁血战十餘年,大小百餘战,作战英勇异常。但打了天下,却不懂得治天下,宠幸伶人,重用宦官,又吝於銀錢,不抚恤士卒,三年後因兵变被杀,失败之速,亦是罕见。

北宋歐陽修寫《新五代史·伶官傳序》便是討論李存勗沉溺逸樂、寵信樂官而致亡國的史實,叹惜李存勗“方其盛也,举天下之豪杰复能与之争;及其衰也,数十伶人困之,而身死国灭,为天下笑。”,說明“憂勞可以興國,逸豫可以亡身”的歷史規律 。

《旧五代史》则称赞李存勗是“中兴之主”,是唐朝的合法继承者,但語鋒一轉,隨即批評他“忘櫛沐之艱難,徇色禽之荒樂”、“伶人亂政、靳吝貨財、大臣無罪以獲誅、眾口吞聲而避禍” 。

朱温评价李存勗说“生子当如李亚子,克用为不亡矣!至如吾儿,豚犬耳!”(生儿子就要生像李存勖这样的,李克用的大业不会灭亡了!至于说我的儿子,猪狗之辈而已!),中國共產黨中央委員會主席毛澤東也同意這個看法。


\subsubsection{同光}

\begin{longtable}{|>{\centering\scriptsize}m{2em}|>{\centering\scriptsize}m{1.3em}|>{\centering}m{8.8em}|}
  % \caption{秦王政}\
  \toprule
  \SimHei \normalsize 年数 & \SimHei \scriptsize 公元 & \SimHei 大事件 \tabularnewline
  % \midrule
  \endfirsthead
  \toprule
  \SimHei \normalsize 年数 & \SimHei \scriptsize 公元 & \SimHei 大事件 \tabularnewline
  \midrule
  \endhead
  \midrule
  元年 & 923 & \tabularnewline\hline
  二年 & 924 & \tabularnewline\hline
  三年 & 925 & \tabularnewline\hline
  四年 & 926 & \tabularnewline
  \bottomrule
\end{longtable}


%%% Local Variables:
%%% mode: latex
%%% TeX-engine: xetex
%%% TeX-master: "../../Main"
%%% End:
