%% -*- coding: utf-8 -*-
%% Time-stamp: <Chen Wang: 2019-12-24 16:49:22>

\subsection{李从珂\tiny(934-937)}

\subsubsection{生平}

李從珂(885年2月11日-937年1月11日),鎮州(今河北正定)人,五代時期後唐皇帝,史稱後唐末帝或後唐廢帝,本姓王,小字二十三,因此又被叫做阿三。

李從珂十餘歲時,其母魏氏被當時仍是將領的後唐明宗李嗣源所擄,李從珂不久就被李嗣源改名並收為養子。長大後身形雄偉健壯,又驍勇善戰,常隨李嗣源南征北討,頗得其喜愛。莊宗也曾赞道:“阿三不惟与我同齿,敢战亦相类。”

同光三年(925年),李嗣源因家在太原,上表请求让时任卫州刺史的李从珂为北京(太原)内牙马步都指挥使,以便相聚,却导致莊宗大怒,李从珂被黜为突骑指挥使,率数百人戍石门镇。次年李嗣源平定邺都之乱时被迫造反,李从珂率军与之会合,助其夺位。

李嗣源即帝位後,李從珂曾任河中節度使之職,然因與權臣樞密使安重誨之前有過節,在長興元年(930年),被安重誨設計解除軍權,回京師洛陽居住。次年(931年),安重誨失勢,李從珂再受重用,被任命為左衛大將軍、西京(長安)留守。長興三年(932年),被改命為鳳翔節度使。長興四年(933年),封潞王。

後唐應順元年(934年),閔帝李從厚聽信大臣的建議,調動各重要節度使之職,準備削弱藩鎮的實力,李從珂恐懼,遂反。李從厚命王思同率大軍討伐,王思同围攻凤翔城(今陝西鳳翔)。凤翔城墙低,护城河窄浅,根本无法固守。眼看鳳翔即將陷落,未料討伐軍將兵驕橫,貪圖賞賜,李從珂抓住這點誘使討伐軍叛變,反敗為勝,不久以摧枯拉朽之勢攻入京師洛陽,即帝位,改元清泰,並派人將逃亡的李從厚殺害。

李嗣源之婿石敬瑭時任重鎮河東節度使之職,李從珂與他二人當初在李嗣源手下皆以勇力過人著稱,彼此存有競爭之心。因此李從珂即位後,對石敬瑭愈發猜忌,而石敬瑭亦有謀反之意。

清泰三年(936年),石敬瑭以調鎮他處試探,而李從珂果真將石敬瑭改任天平節度使,石敬瑭因此叛變,同時向契丹乞援。李從珂命各鎮聯合討伐,不料因聯軍各懷鬼胎,致大敗於團柏谷,石敬瑭與契丹大軍得以順利南下進逼京師洛陽,李從珂無計可施,於閏十一月二十六日(陽曆為937年1月11日)自焚而死。死後無諡號及廟號,史家稱之為末帝或廢帝。传国玉玺亦在此时遗失不知所踪。

李從珂是一個矛盾的人,時運順他之時,他竟敢潛伏到敵軍陣營內,殺死敵軍並且砍下對方望桿扛回去;時運離他而去時,整天消沉著在宮中飲酒哭泣,最後自焚,並不是大臣沒有給他好的意見,而是他自己優柔寡斷,無法及時下決定,最終誤國。


\subsubsection{清泰}

\begin{longtable}{|>{\centering\scriptsize}m{2em}|>{\centering\scriptsize}m{1.3em}|>{\centering}m{8.8em}|}
  % \caption{秦王政}\
  \toprule
  \SimHei \normalsize 年数 & \SimHei \scriptsize 公元 & \SimHei 大事件 \tabularnewline
  % \midrule
  \endfirsthead
  \toprule
  \SimHei \normalsize 年数 & \SimHei \scriptsize 公元 & \SimHei 大事件 \tabularnewline
  \midrule
  \endhead
  \midrule
  元年 & 934 & \tabularnewline\hline
  二年 & 935 & \tabularnewline\hline
  三年 & 936 & \tabularnewline
  \bottomrule
\end{longtable}


%%% Local Variables:
%%% mode: latex
%%% TeX-engine: xetex
%%% TeX-master: "../../Main"
%%% End:
