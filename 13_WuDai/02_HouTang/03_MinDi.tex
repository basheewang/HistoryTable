%% -*- coding: utf-8 -*-
%% Time-stamp: <Chen Wang: 2021-11-01 15:21:11>

\subsection{闵帝李從厚\tiny(933-934)}

\subsubsection{生平}

唐閔帝李從厚(914年-934年),小字菩薩奴,五代時期後唐皇帝,為後唐明宗李嗣源之子,母昭懿皇后夏氏,有一胞兄李从荣。李嗣源因李從厚與自己相像,特別喜歡他。

李從厚於李嗣源在位時,原被封為宋王。後唐長興四年(933年),李嗣源病重,本為繼承人的秦王李從榮誤以為李嗣源已死,為確保能夠繼位,遂帶兵入宮,事敗被殺。李嗣源不得已,召時任天雄節度使的李從厚回京。不久,李嗣源去世,李從厚繼位。次年(934年),改年號應順。

李從厚即帝位後,信任朱弘昭、馮贇等人,二人於應順元年(934年),調動各重要節度使,準備削藩,鳳翔節度使潞王李從珂恐懼,遂反,攻入京師洛陽,李從厚出逃魏州,途经卫州,遇到河东节度使石敬瑭。石敬瑭無意救之。李从厚的亲随不满石敬瑭,抽刀要杀石敬瑭,结果反被石敬瑭的侍卫杀死。石敬瑭的部將劉知遠尽杀闵帝亲随,把闵帝安置在卫州後,不久離去。皇太后下令降閔帝為鄂王。不久闵帝為潞王李從珂派人所殺。後晉高祖石敬瑭稱帝後,將他諡為閔皇帝。

李從厚個性仁慈,對兄弟很和睦,雖遭李從榮忌恨,卻能坦誠相待,所以當時才能逃過一劫。本來與李從珂也沒有過節,只因輕易地聽信周遭人的讒言,才會招來大禍。

\subsubsection{应顺}

\begin{longtable}{|>{\centering\scriptsize}m{2em}|>{\centering\scriptsize}m{1.3em}|>{\centering}m{8.8em}|}
  % \caption{秦王政}\
  \toprule
  \SimHei \normalsize 年数 & \SimHei \scriptsize 公元 & \SimHei 大事件 \tabularnewline
  % \midrule
  \endfirsthead
  \toprule
  \SimHei \normalsize 年数 & \SimHei \scriptsize 公元 & \SimHei 大事件 \tabularnewline
  \midrule
  \endhead
  \midrule
  元年 & 934 & \tabularnewline
  \bottomrule
\end{longtable}


%%% Local Variables:
%%% mode: latex
%%% TeX-engine: xetex
%%% TeX-master: "../../Main"
%%% End:
