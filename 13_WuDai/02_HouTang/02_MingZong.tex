%% -*- coding: utf-8 -*-
%% Time-stamp: <Chen Wang: 2021-11-01 15:21:01>

\subsection{明宗李亶\tiny(926-933)}

\subsubsection{生平}

唐明宗李亶(867年10月10日-933年12月15日),初名嗣源,小名邈佶烈。应州金城(今山西省应县)人,沙陀族,五代十国时期后唐第二位皇帝(926年6月3日-933年12月15日在位),在位8年。唐河东节度使李克用之养子,生父為李霓。

李嗣源初以騎射為河東節度使李克用效命,李克用則以李嗣源為養子,寵愛有加。克用死後,923年克用之子李存勗在魏博稱帝,國號唐,史稱後唐庄宗。同光元年二月,后唐潞州(今山西長治)、卫州(今河南汲縣)情势危殆,李嗣源以奇兵夜袭后梁东部重镇郓州(今山東東平),攻克。因功就任天平节度使。十月,李嗣源作为前锋进击大梁(今河南開封)。随即攻克。

926年魏博軍士兵皇甫暉乘人心不安聚眾作亂於貝州(今河北清河),斬殺主帥楊仁晸,立偏將趙在禮為帥,攻破鄴都(今河北臨漳),莊宗雖忌李嗣源,卻因乏人,只好命嗣源前往討伐,嗣源到魏博(今河北邯鄲)時,發生兵變,擁嗣源入城與叛軍會合,嗣源女婿石敬瑭勸告,莊宗不可能不追究此事,只好決心謀反。嗣源南下據汴京(今河南開封),西攻洛邑(今河南洛陽),史稱鄴都之變。此時伶人出身的禁軍將領郭從謙為了報仇,在洛邑乘機發動興教門之變,率兵攻入皇城,莊宗中流箭而死。群臣拥戴李嗣源为监国。李嗣源杀死宫中所有伶人,接着于926年四月丙午日称帝,年号“天成”。927年1月改名亶。

李嗣源即位后,革除莊宗时的弊政,励精图治,兴修水利,誅滅宦官,关心百姓疾苦,並撤銷不少有名無實的機關,后唐趋于强盛。

明宗是文盲,完全不识字,全国的上奏都得交由安重诲读给他听。天成三年,义武节度使王都叛乱,明宗削去其官爵,命王晏球讨伐。契丹遣兵救援王都,为后唐大败,回到本国境内的不过数十人,自此不敢轻易寇边。长兴元年(930年),明宗养子李从珂被安重诲陷害,但明宗本人没有听信谗言,亦没有追责安重诲。八月,明宗诛杀了朝中制造混乱,诬陷大臣的李行德、张俭等。他在位八年,兵革粗定,連年豐收,多次率领军队打败契丹。放觀整個五代只有他與後周兩位君王堪稱是有作為的皇帝。他在位期間的大事,是董璋、孟知祥割據兩川。此时安重诲逐渐失宠,明宗派女婿石敬瑭等讨伐两川时责令安重诲督运粮食去两川前线,安重诲未至前线即被凤翔节度使朱弘昭弹劾意欲夺取石敬瑭军权,于是又被召回,明宗又在途中突然改任安重诲为护国军节度使,后又将其杀死。石敬瑭讨伐两川最终无果,孟知祥又在董璋来伐时先败后胜消灭了董璋的东川势力,最终在明宗死后建立后蜀。

長興四年,李嗣源因聽聞定難節度使李彝超欲叛變,貿然任命李彝超為彰武留後,令安從進為定難節度使,李彝超抗命,明宗派兵進攻,竟久攻不下;李彝超上表謝罪,明宗只好授李彝超為檢校司徒,定難軍節度使,既而定難軍朝貢如初。明宗久年未出兵,一朝用兵卻無功而返,無論是兩川(今四川)或是夏州(今陝西榆林),軍中就有了對明宗不利的謠言,明宗懼,下令賞賜軍士,這個賞賜無任何理由,士卒於是日益驕縱。

933年明宗病危,数日不见臣下,原本被内定为继承人的秦王李从荣担心有变,引兵入宫。枢密使朱弘昭、冯赟以讨逆为名,派兵抵抗,將李從榮誅殺,后更杀死养在李嗣源身边的李从荣之子。李嗣源得知消息,悲痛过度,病重去世,庙号明宗,谥号圣德和武钦孝皇帝,葬于徽陵。由宋王李從厚繼位。

後唐明宗為人純質,寬仁愛人,在位時力除莊宗時的弊病、廣納建言、關心民生、嚴懲貪官、不邇聲色、不樂遊畋,鮮少發生戰事,百姓得以喘息;是五代十國數十位帝王中少數稱得上明主的,可是他也有許多不明智的地方:無法解決任圜與安重誨的矛盾,使得任圜被冤殺;無法解決兩川問題,使孟知祥反叛建立後蜀;聽信他人讒言,使安重誨夫婦於自宅被處死;久不立太子,使李從榮生疑並最終被殺。其後續位的後唐閔帝昏庸無能,周遭又是一些無用之輩,後唐很快也跟著滅亡了。


\subsubsection{天成}

\begin{longtable}{|>{\centering\scriptsize}m{2em}|>{\centering\scriptsize}m{1.3em}|>{\centering}m{8.8em}|}
  % \caption{秦王政}\
  \toprule
  \SimHei \normalsize 年数 & \SimHei \scriptsize 公元 & \SimHei 大事件 \tabularnewline
  % \midrule
  \endfirsthead
  \toprule
  \SimHei \normalsize 年数 & \SimHei \scriptsize 公元 & \SimHei 大事件 \tabularnewline
  \midrule
  \endhead
  \midrule
  元年 & 926 & \tabularnewline\hline
  二年 & 927 & \tabularnewline\hline
  三年 & 928 & \tabularnewline\hline
  四年 & 929 & \tabularnewline\hline
  五年 & 930 & \tabularnewline
  \bottomrule
\end{longtable}

\subsubsection{长兴}

\begin{longtable}{|>{\centering\scriptsize}m{2em}|>{\centering\scriptsize}m{1.3em}|>{\centering}m{8.8em}|}
  % \caption{秦王政}\
  \toprule
  \SimHei \normalsize 年数 & \SimHei \scriptsize 公元 & \SimHei 大事件 \tabularnewline
  % \midrule
  \endfirsthead
  \toprule
  \SimHei \normalsize 年数 & \SimHei \scriptsize 公元 & \SimHei 大事件 \tabularnewline
  \midrule
  \endhead
  \midrule
  元年 & 930 & \tabularnewline\hline
  二年 & 931 & \tabularnewline\hline
  三年 & 932 & \tabularnewline\hline
  四年 & 933 & \tabularnewline
  \bottomrule
\end{longtable}


%%% Local Variables:
%%% mode: latex
%%% TeX-engine: xetex
%%% TeX-master: "../../Main"
%%% End:
