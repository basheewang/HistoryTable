%% -*- coding: utf-8 -*-
%% Time-stamp: <Chen Wang: 2019-12-24 16:41:33>


\section{后唐\tiny(923-937)}

\subsection{简介}

后唐(923年-937年)是中國五代時期的政權之一。923年,唐朝的赐姓沙陀人李存勖消灭後梁,重建唐朝。在魏州(河北大名县西)称帝,以“復興唐朝”为名,不久迁都洛阳。後為石敬瑭勾結契丹入侵而滅亡。史学家為了區別由李淵所建立的唐朝,因而稱之為後唐,歷時十四年。

另外,雖然后唐統治者的祖源是沙陀族,但當時的後唐統治者被視為漢族,後唐也被稱為「漢國」。

後唐之建立可追溯到唐朝末年龐勛之亂。龐勛之亂發生在唐懿宗咸通年間,唐政府命沙陀族的朱邪赤心領兵平亂,並賜名李國昌,编入唐朝宗籍,成为唐朝宗室。後來李國昌病故,其子李克用又協助唐室平定黃巢之亂和王行瑜之乱。李克用平亂有功,被封河東節度使,駐守太原,受封晉王。

李克用追擊黃巢時,曾被當時的宣武節度使朱溫邀請入汴梁作客,因为某些不明原因(据朱温党羽的说法,李克用酒後侮辱朱溫),朱溫决意趁李克用酒醉殺害他,李克用突圍而出才脫身,自此他與朱溫誓不兩立。朱溫後來篡唐建立後梁,李克用作为唐朝宗室,仍用唐天祐年號,以唐朝北都(今天的山西太原)为基地,发誓兴复唐朝,事实上成为中国南北各地反梁势力的盟主。故晉王集团成為後梁北方最大的威脅。

李克用死後,兒子李存勖繼承晉王爵位,屡败梁军。915年(唐天祐十二年,后梁貞明元年),梁在河北鎮守的鄴王楊師厚死,河北发生反梁兵变,李存勖乘機占据河北,晉與後梁在黄河爭峙。李存勖在923年于魏州(今河北大名县)稱帝,宣布继承唐朝皇统,史稱後唐,改元同光。同年唐軍直迫汴京滅了後梁,定都洛陽,李存勖即位,成為後唐莊宗。後唐不承認後梁為正統,在其五行德運的選取上選擇恢復唐朝的「土」德。

後唐莊宗定都洛陽後,力图恢复大唐的荣光。他見前蜀王氏無道,925年派郭崇韜攻入成都,不出七十日就滅了前蜀。至此,汉地诸藩国臣服,後唐莊宗成为长城以南整个汉地公认的唯一皇帝。當時南方諸藩国深感恐慌,認為中原很快就會派大軍來削平南方的割据。

但另一方面,後唐莊宗漸自大貪逸,宠信宦官、伶人,疏远旧将,内部矛盾激化。皇后刘氏和宦官與郭崇韜不和,向莊宗進讒言,結果郭崇韜被误殺,引來政局动荡和軍人反叛。一個半月內,各地起兵反叛,後唐大亂。

莊宗不得已,派李克用的養子李嗣源,往河北討伐反叛,在嗣源的女婿石敬瑭的策動下,河北軍立嗣源為帝,反攻洛陽。莊宗被伶人郭從謙所殺,史稱興教門之變,以後,李嗣源即皇帝位,是為後唐明宗。

後唐明宗李嗣源即位後,也有相當治績,朝政漸為安定。但軍人安重誨專權,未能处理好与孟知祥、董璋的关系,两人发生内斗,孟知祥取胜,结果为后来后蜀脱离后唐独立埋下了祸根。

明宗晚年病倒在床時,秦王李從榮以為明宗已死,起兵企圖攻入皇宮,結果事敗被殺。明宗得知秦王被殺,震驚之下駕崩,大臣妃子擁立宋王李從厚,是為後唐閔帝。

閔帝即位後,採用削藩政策,引起潞王李從珂的叛亂,叛軍攻入京師,而閔帝夫婦逃往河北,被姐夫石敬瑭設計除去其手下將士,為從珂軍士擒殺,從珂即帝位,是為後唐末帝。

末帝與鎮守太原的河東節度使石敬瑭不和,936年末帝下詔把石敬瑭調任,引來石敬瑭的叛亂。末帝發兵攻太原,石敬瑭向契丹借兵,遼太宗耶律德光親率大軍南下,唐軍大敗,937年,契丹與石敬瑭的大軍攻入洛陽,1月11日末帝自焚而死,後唐滅亡。歷四帝共十四年。

後唐据有今河南、山东、山西、河北、陕西关中、甘肃东部、湖北北部、安徽北部。

\subsection{武帝生平}

李克用(856年10月24日-908年2月24日),字翼圣,神武川之新城(今山西雁门)人,後唐莊宗李存勗之父,本姓朱邪(又作朱耶),其父受唐朝天子賜李姓。綽號鴉兒、三郎、獨眼龍、飛虎子,沙陀族人,唐大中十年(856年)生于神武川之新城(在今山西雁門北部)。是中國唐朝末年最強大的藩鎮節度使之一,後受唐封為晉王。后唐建立後,尊稱其為唐太祖武皇帝。

李克用為沙陀人,本姓朱邪(又作朱耶),其父朱邪赤心因鎮壓龐勛兵變有功,受唐懿宗賜姓名為李國昌,因此克用亦以「李」為姓。李克用是國昌第三子,家族內暱稱「三郎」。因天生一目較為細小,而被人稱為獨眼龍;作戰驍勇,又因其衝鋒陷陣為諸將之冠,軍中也稱其為飛虎子。

李克用驍勇善騎射,年15歲即從軍,後來被唐朝廷任命為沙陀副兵馬使。唐僖宗乾符五年(878年),當時代北(今山西省北部)饑荒,漕運不繼。大同(今山西省大同市)防禦使段文楚大量縮減軍士衣物和米糧的供應,而執法嚴厲,士卒怨恨。李克用為下屬所擁,殺段文楚而起事。廣明元年(880年),再殺河東(今山西省太原市)節度使康傳圭,佔領太原,不久為唐軍所敗,與父逃入韃靼部落。

唐僖宗中和元年(881年),公家赦李氏父子之罪,命李克用率沙陀軍南下助戰,以鎮壓佔領兩都、自稱大齊皇帝的黃巢。克用途經太原,因無法得到犒賞,遂縱容沙陀軍剽掠河東居民,引起百姓驚駭,不久北返,並繼續剽掠北部邊境一帶。中和二年(882年)李克用二次受敕勤王,此次沙陀軍南下,正式面對齊軍。中和三年(883年),屢敗齊軍,黃巢退出長安,由於李克用在長安收復戰中功勞最大,因此被命為河東節度使,河東也成為他後來的根據地。其時黃巢兵勢仍強,宣武節度使朱溫等各鎮皆無法抵擋,遂請河東軍來援。中和四年(884年),李克用再自河東南下大敗齊軍,最終使得黃巢在狼虎谷(在今山東省萊蕪市)自殺。

河東節度使李克用擊敗黃巢後回師河東,途經宣武節度使首府汴州(今河南省開封市),受節度使朱溫(朱全忠)邀請入城。朱溫因李克用酒後言語中多有侮辱,趁李克用酒醉之際夜襲,李克用幾乎被殺,狼狽逃回太原(上源驿之变),從此宣武朱氏與河東李氏兩家結下深仇。光啟元年(885年),朝廷讨伐河中节度使王重荣。李克用根据王重荣的假情报,指責盤踞關中的静难节度使朱玫、凤翔节度使李昌符結交朱溫欲滅己,因此進軍關中擊敗二人。僖宗逃往鳳翔。战乱中,各方军队进入長安,縱火大掠,雖然不久即行退去,然而在黃巢亂後稍有恢復的京師長安再度毀於一旦。唐昭宗大順元年(890年),朱溫與宰相張濬力主討伐河東,乃削李克用之官爵,聯合諸鎮之兵進攻,卻反為河東軍所敗,副都统孙揆被擒杀,朝廷禁军大损,只好恢復李克用的官爵。乾寧二年(895年),李茂貞、王行瑜及韓建三帥進京挾持唐昭宗,李克用不念旧恶,再度率軍勤王,敗三帥,救出昭宗;王行瑜被迫出走,被手下杀死。因功,十二月(合896年)李克用被封為晉王,但昭宗怕其日后难制,没有同意他继续消灭李茂贞、韩建。其後數年,李克用持續與朱溫爭戰,相互間成為爭奪天下的最大對手。但李克用性格直率,往往得罪人而不察,又对一些险诈之徒缺乏防范,多次遭到背叛,特别是幽州劉仁恭叛變,使李克用元气大伤。而宣武不斷併吞鄰鎮,兵勢日盛;李克用漸居下風,甚至无法救援自己的女婿河中节度使王珂,致使其被迫投降朱温。900年,黃河以北藩镇多附朱溫。

天復元年、二年(901年、902年),朱溫兩次率軍圍攻太原,李克用顽强抵抗,迫使敌军撤退。此後,朱全忠挾天子以令諸侯,焚毁长安,弑杀唐昭宗,唐室如風中殘燭。李克用坚决反对朱温,多次发起勤王行动,但均未能成功。天祐三年(906年),李克用收复潞州,粉碎了朱溫精心策划的总攻势。天祐四年(907年),朱溫篡唐稱帝,建立後梁,改元開平,李克用仍用唐天祐年號,以復興唐朝為名與後梁展開另一個階段的爭鬥。次年(908年),在与後梁的战争中,李克用因积劳成疾去世,子李存勗繼立。後來,李存勗滅後梁、建後唐,李克用被追諡為武皇帝,廟號太祖。

身為一名軍閥,李克用年輕時確有不少违抗唐政府的事蹟,但或許因為其年少英雄,後來又打著「復興唐室」的名號來對抗惡名昭彰的朱溫,因此歷史上有諸多關於他正面的傳說。而有意思的是,這些傳說不少與他所擅長之武器「箭」有關。

李克用一目失明,號「獨眼龍」。《五代史補》中記有一則故事:在李克用佔據河東,聲威大振後,盤據淮南的另一個軍閥楊行密很想見見李克用甚麼模樣。於是楊行密找了一個畫家,假扮商人到河東伺機偷畫李克用面貌。不料畫家到了河東,立刻為事先得到情報的河東武士所俘虜。克用頗怒,就對親信說:「我少了一隻眼睛,看他要怎麼畫我。」等到畫家一到殿內,李克用立刻大怒說道:「淮南派你來畫我,想必你是畫家中最好的,如果今天畫得不好,那麼這裏就是你的死地!」當時是炎夏,所以畫家最初畫李克用手拿扇子搧風,而扇角正好遮住了李克用失明的眼晴,非常巧妙。但克用不喜歡,說畫家「諂媚」,命其重畫。畫家這次重畫,就畫李克用彎弓射箭,一隻眼睛瞇了起來,好像就在瞄準目標,克用大喜,於是重賞畫家銀兩,並送之回淮南。

這則故事見於罗贯中的《殘唐五代史演義傳》第九回《克用箭服周德威》;另外,白樸的元曲也有《李克用箭射雙鵰》的折子;除此之外,京劇《珠簾寨》中也有這個橋段。大意是李克用在受唐政府所託出軍討伐黃巢時,兵至珠簾寨,遇見一將周德威擋路,李克用素聞周德威之名,躍馬向前迎戰,二人交手一百餘合不分勝負,遂以射鵰為賭注,若李克用能箭射飛鵰,周德威即下馬受降。只見李克用彎弓一射,弓弦響處,鵰已落地,周德威甘服而降。這則故事的原型,很可能是出自《舊五代史·武皇紀上》的記載。而《新五代史·唐本紀第四》亦有其善射的敘事。

這可能是李克用和李存勗父子間最有名的故事:宋初王禹偁的《五代史闕文》記傳傳說李克用臨終時,將三支箭交給李存勗,說道:「劉仁恭父子背叛我,契丹耶律阿保機違背與我們的盟約,朱溫和我們是世仇,我給你三支箭,第一支箭要你討伐劉仁恭,第二支箭要你打敗契丹,第三支箭要消滅朱溫,希望你完成我這三個願望。」李存勗把三支箭供奉在宗廟裏,逢出征時依次派人取箭,帶上戰場,後來併桀燕、敗契丹、滅後梁,得勝後分別將箭送回宗廟,表示完成了李克用的願望。

不過,歷代都有人懷疑這則故事的真實性。司馬光《資治通鑑考異》認為,依《舊五代史·契丹傳》記載:李存勗剛繼位時,後梁兵圍潞州(今山西省長治市),因此尚且對契丹「遣使告哀,賂以金繒,求騎軍以救潞州」。可見根本沒有和契丹結仇的事。另外,有人亦指出朱全忠掃蕩群雄,華北只剩李克用與劉仁恭二者,李克用父子深知唇亡齒寒之理,因此劉仁恭及其子劉守光被朱溫圍攻,甚至劉守光被其兄劉守文攻擊,李克用、李存勗還屢次派兵來救,因此至少在李克用去世之時,河東也沒有與桀燕劉氏對立一事。


%% -*- coding: utf-8 -*-
%% Time-stamp: <Chen Wang: 2019-12-24 16:43:09>

\subsection{庄宗\tiny(923-926)}

\subsubsection{生平}


唐莊宗李存勗xù(885年12月2日-926年5月15日),山西应县人,沙陀族,本姓朱邪,因其父是河東節度使李克用受唐懿宗赐以李姓,而改姓李,諱存勗,唐光启元年正月(885年12月)生于山西应县,五代時期后唐开国皇帝。小名“亚子”,藝名“李天下”,以勇猛闻名。

923年5月13日在魏州(河北大名府)称帝,国号唐,史称后唐。後因義兄李嗣源被軍士擁戴造反,揮軍直取洛陽。宮中指揮使郭从谦為報仇,趁機發動兵變——興教門之變,將存勗殺害。

《旧五代史》记载:庄宗光圣神闵孝皇帝,讳存勖,武皇帝之长子也。母曰贞简皇后曹氏,以唐光启元年岁在乙巳,冬十月二十二日癸亥,生帝于晋阳宫。《册府元龟》记载:后唐庄宗以光启元年十月癸亥生于晋阳宫。《旧唐书》记载:十二月辛亥朔。李存勖以唐光启元年十月二十二日(885年12月2日)生于晋阳宫,而十月二十二日不是癸亥是癸酉,十月十二日才是癸亥。同光元年至三年的万寿节(李存勖生日)皆在十月二十二日,李存勖的生日就是十月二十二日。光启元年十月壬子朔,癸亥是十二日,癸酉是二十二日,李存勖若生于十二日何以过二十二日生日,此当是编纂者失察误抄所致。

李存勗是李克用與貞簡皇后曹氏的長子。他自幼擅長騎馬射箭,膽力過人,為李克用所寵愛。少年時隨父作戰,11歲就與父親到長安向唐朝朝廷報功,得到唐昭宗的賞賜和誇獎。

李存勗成年後狀貌雄偉瑰麗,得習《春秋》,豁達而且通大義,並勇敢善戰,熟知戰略要術。他又喜愛音樂、歌舞、俳優之戲,旁人多有異談。當時,軍閥割據混戰、佔據河東的李克用常被控制河南的朱溫牽制圍困,兵力不足,地盤狹小,非常悲觀。李存勗勸說其父:“朱全忠恃其武力,吞滅四鄰,想篡奪帝位,這是自取滅亡。我們千萬不可灰心喪氣,要積蓄力量,等待時機”。李克用聽後大為高興,重新振作起來,與朱全忠對抗。

後梁開平二年(908年)正月,李克用病死,李存勗於同月襲晉王位。但是當時的兵馬大權歸於其叔父李克寧,軍民之事皆由李克寧決定,權柄既重,令眾人皆攀附李克寧。當辦完喪事後,李存勗與張承業、李存璋設計,要除去勢力龐大的叔父李克寧。同年二月二十日,當諸將於府第時,乃伏兵於府中,置酒大會,李克寧既至,於席間擒下李存顥、李克寧二人,李存勗哭著責備李克寧:「姪兒一開始就打算把軍隊、政權都讓給叔父,叔父不願意背棄我父親的遺命,怎麼現在又把我跟我母親丟給豺狼虎豹?叔父怎麼忍心?」李克寧泣對:「這是讒言啊,我還能說甚麼?」當日,李克寧與李存顥俱伏法。

其後,李存勗認為潞州(今山西上黨)是河東屏障,沒有潞州對河東不利,所以他立即率軍從晉陽出發,直取上黨,乘大霧突襲圍潞州的梁軍,大獲全勝。李存勗的用兵之奇使梁太祖朱溫大驚,他說:「生兒子就要生李存勗一樣的兒子,李克用不會滅亡了啊!至於我的兒子,豬狗之輩而已!」

當潞州之圍解決後,河東威振,控制鎮州的王鎔和控制定州的王處直見形勢驟變,也動搖了依附後梁的信心,竟然和李存勗結成聯盟共同對付後梁。後梁為了保護河北之地,不惜一切,出兵再戰,於是雙方在柏鄉又展開了一場血戰。柏鄉之戰中,晉軍有周德威等三千騎兵和鎮州、定州兵;對方梁軍有王景仁率領的禁軍和魏博兵八萬。梁軍守衛柏鄉、以逸待勞,在地形、兵力、裝備幾方面處於優勢;而晉軍是騎兵,機動性和進攻能力大,對梁軍構成威脅。戰役開始,李存勗採用周德威建議,引誘梁兵出城,聚而殲之,晉軍主動後撤。梁軍主將王景仁果然上當,傾巢而出。晉軍抓住機會,以騎兵猛烈突擊梁軍,周德威攻右翼,李嗣源攻左翼,鼓譟而進。這時晉軍李存璋率領的騎兵大隊也趕上,梁軍丟盔棄甲,死傷殆盡。這一仗,使梁軍喪失了對河北的控制權,之後,朱溫一聽晉軍就談虎色變。而李存勗卻進一步安定了河東局勢,他息兵行賞,任用賢才,懲治貪官惡吏,寬刑減賦,一時河東大治。

李克用臨死時,交給李存勗三支箭,囑咐他要完成三件大事:一是討伐劉仁恭,攻克幽州;二是征討契丹,解除北方邊境的威脅;第三件大事就是要消滅世敵朱溫。他將三支箭供奉在家廟裡,每臨出征就派人取來,放在精製的絲套裡,帶著上陣,打了勝仗,又送回家廟,表示完成了任務。其後李存勗達成李克用遺志,打敗契丹,攻破燕地,並且攻滅劉守光與劉仁恭父子割據的桀燕政權,並且於923年,在魏州(河北大名縣西)稱帝,國號為唐,史稱後唐,其後攻滅後梁,統一北方。李存勗還收降了李茂貞建立的岐,並攻滅王建所建立的前蜀。

李存勗以唐朝赐姓为李的合法继承人身份,打起中兴唐朝的旗号,并为唐朝皇帝立庙。又以诛灭唐朝逆臣之名,族灭了后梁宰相敬翔、李振等人,将帮助朱溫篡唐的旧臣11人贬官。

但李存勗到了晚年自認為已經拚命一生,應該好好享樂,遂荒廢朝政。李存勗自幼喜歡看戲、演戲,常粉墨登場,並自號藝名“李天下”。伶人大受皇帝寵幸,以至于伶人景进干预朝政。士大夫皆气愤,又不敢出气。李存勗又派伶人、宦官搶民女入宮,強擄魏博士卒們妻女千餘人,怨聲四起。同光二年,李存勗恢复旧唐宦官的势力,本来已经消失的监军又凌驾于藩镇之上,导致诸将更大的不满。同光三年(925年),李存勗派遣兒子魏王李继岌、侍中郭崇韬,攻滅前蜀。但是其後继岌、崇韬互相猜疑。郭崇韬又得罪宦官,李存勗於是对崇韬起疑,下命孟知祥入蜀,见机行事。翌年,李存勗被宦官的谗言所迷惑,诛杀了朱友谦、李存乂。后唐朝廷人心惶惶。

後唐同光四年(926年),魏博士兵皇甫暉在鄴城叛亂,是為鄴城之亂,李存勗命李绍荣前往討伐,久不能下,无奈命李嗣源攻鄴城,李嗣源命其女婿石敬瑭同征。兵進魏州時,李嗣源卻被叛軍擁戴,恭迎入城,李嗣源百口莫辯,石敬瑭表示就算不造反也無法免責,李嗣源因而擁兵自立,與魏博的叛軍合兵造反。李嗣源占據汴州(今河南開封),進軍洛邑,先鋒石敬瑭則帶兵逼進汜水關(河南滎陽汜水鎮),李存勗決定親征反擊。

這時擔任指揮使的伶人郭从谦不知李存乂已被莊宗殺死,欲奉李存乂之名作乱,火燒興教門。蕃汉马步使朱守殷见危不救。李存勗當時僅有符彥卿及王全斌等少數將領效忠他。郭从谦率兵攻入皇城。李存勗被流箭射中。王全斌將其扶至絳霄殿。李存勗失血過多,渴懑求飲,經宦官奉進酪漿,喝完一杯,遽爾殞命。王全斌大慟而去。一名伶人揀丟棄的樂器放在李存勗屍體上,點火焚屍。史稱興教門之變。李嗣源入洛陽杀尽叛臣,葬李存勗屍骨于雍陵,進廟號莊宗,李嗣源在汴州稱帝,是為後唐明宗。

李存勗稱帝即位之前,和后梁血战十餘年,大小百餘战,作战英勇异常。但打了天下,却不懂得治天下,宠幸伶人,重用宦官,又吝於銀錢,不抚恤士卒,三年後因兵变被杀,失败之速,亦是罕见。

北宋歐陽修寫《新五代史·伶官傳序》便是討論李存勗沉溺逸樂、寵信樂官而致亡國的史實,叹惜李存勗“方其盛也,举天下之豪杰复能与之争;及其衰也,数十伶人困之,而身死国灭,为天下笑。”,說明“憂勞可以興國,逸豫可以亡身”的歷史規律 。

《旧五代史》则称赞李存勗是“中兴之主”,是唐朝的合法继承者,但語鋒一轉,隨即批評他“忘櫛沐之艱難,徇色禽之荒樂”、“伶人亂政、靳吝貨財、大臣無罪以獲誅、眾口吞聲而避禍” 。

朱温评价李存勗说“生子当如李亚子,克用为不亡矣!至如吾儿,豚犬耳!”(生儿子就要生像李存勖这样的,李克用的大业不会灭亡了!至于说我的儿子,猪狗之辈而已!),中國共產黨中央委員會主席毛澤東也同意這個看法。


\subsubsection{同光}

\begin{longtable}{|>{\centering\scriptsize}m{2em}|>{\centering\scriptsize}m{1.3em}|>{\centering}m{8.8em}|}
  % \caption{秦王政}\
  \toprule
  \SimHei \normalsize 年数 & \SimHei \scriptsize 公元 & \SimHei 大事件 \tabularnewline
  % \midrule
  \endfirsthead
  \toprule
  \SimHei \normalsize 年数 & \SimHei \scriptsize 公元 & \SimHei 大事件 \tabularnewline
  \midrule
  \endhead
  \midrule
  元年 & 923 & \tabularnewline\hline
  二年 & 924 & \tabularnewline\hline
  三年 & 925 & \tabularnewline\hline
  四年 & 926 & \tabularnewline
  \bottomrule
\end{longtable}


%%% Local Variables:
%%% mode: latex
%%% TeX-engine: xetex
%%% TeX-master: "../../Main"
%%% End:

%% -*- coding: utf-8 -*-
%% Time-stamp: <Chen Wang: 2019-12-24 16:45:22>

\subsection{明宗\tiny(926-933)}

\subsubsection{生平}

唐明宗李亶(867年10月10日-933年12月15日),初名嗣源,小名邈佶烈。应州金城(今山西省应县)人,沙陀族,五代十国时期后唐第二位皇帝(926年6月3日-933年12月15日在位),在位8年。唐河东节度使李克用之养子,生父為李霓。

李嗣源初以騎射為河東節度使李克用效命,李克用則以李嗣源為養子,寵愛有加。克用死後,923年克用之子李存勗在魏博稱帝,國號唐,史稱後唐庄宗。同光元年二月,后唐潞州(今山西長治)、卫州(今河南汲縣)情势危殆,李嗣源以奇兵夜袭后梁东部重镇郓州(今山東東平),攻克。因功就任天平节度使。十月,李嗣源作为前锋进击大梁(今河南開封)。随即攻克。

926年魏博軍士兵皇甫暉乘人心不安聚眾作亂於貝州(今河北清河),斬殺主帥楊仁晸,立偏將趙在禮為帥,攻破鄴都(今河北臨漳),莊宗雖忌李嗣源,卻因乏人,只好命嗣源前往討伐,嗣源到魏博(今河北邯鄲)時,發生兵變,擁嗣源入城與叛軍會合,嗣源女婿石敬瑭勸告,莊宗不可能不追究此事,只好決心謀反。嗣源南下據汴京(今河南開封),西攻洛邑(今河南洛陽),史稱鄴都之變。此時伶人出身的禁軍將領郭從謙為了報仇,在洛邑乘機發動興教門之變,率兵攻入皇城,莊宗中流箭而死。群臣拥戴李嗣源为监国。李嗣源杀死宫中所有伶人,接着于926年四月丙午日称帝,年号“天成”。927年1月改名亶。

李嗣源即位后,革除莊宗时的弊政,励精图治,兴修水利,誅滅宦官,关心百姓疾苦,並撤銷不少有名無實的機關,后唐趋于强盛。

明宗是文盲,完全不识字,全国的上奏都得交由安重诲读给他听。天成三年,义武节度使王都叛乱,明宗削去其官爵,命王晏球讨伐。契丹遣兵救援王都,为后唐大败,回到本国境内的不过数十人,自此不敢轻易寇边。长兴元年(930年),明宗养子李从珂被安重诲陷害,但明宗本人没有听信谗言,亦没有追责安重诲。八月,明宗诛杀了朝中制造混乱,诬陷大臣的李行德、张俭等。他在位八年,兵革粗定,連年豐收,多次率领军队打败契丹。放觀整個五代只有他與後周兩位君王堪稱是有作為的皇帝。他在位期間的大事,是董璋、孟知祥割據兩川。此时安重诲逐渐失宠,明宗派女婿石敬瑭等讨伐两川时责令安重诲督运粮食去两川前线,安重诲未至前线即被凤翔节度使朱弘昭弹劾意欲夺取石敬瑭军权,于是又被召回,明宗又在途中突然改任安重诲为护国军节度使,后又将其杀死。石敬瑭讨伐两川最终无果,孟知祥又在董璋来伐时先败后胜消灭了董璋的东川势力,最终在明宗死后建立后蜀。

長興四年,李嗣源因聽聞定難節度使李彝超欲叛變,貿然任命李彝超為彰武留後,令安從進為定難節度使,李彝超抗命,明宗派兵進攻,竟久攻不下;李彝超上表謝罪,明宗只好授李彝超為檢校司徒,定難軍節度使,既而定難軍朝貢如初。明宗久年未出兵,一朝用兵卻無功而返,無論是兩川(今四川)或是夏州(今陝西榆林),軍中就有了對明宗不利的謠言,明宗懼,下令賞賜軍士,這個賞賜無任何理由,士卒於是日益驕縱。

933年明宗病危,数日不见臣下,原本被内定为继承人的秦王李从荣担心有变,引兵入宫。枢密使朱弘昭、冯赟以讨逆为名,派兵抵抗,將李從榮誅殺,后更杀死养在李嗣源身边的李从荣之子。李嗣源得知消息,悲痛过度,病重去世,庙号明宗,谥号圣德和武钦孝皇帝,葬于徽陵。由宋王李從厚繼位。

後唐明宗為人純質,寬仁愛人,在位時力除莊宗時的弊病、廣納建言、關心民生、嚴懲貪官、不邇聲色、不樂遊畋,鮮少發生戰事,百姓得以喘息;是五代十國數十位帝王中少數稱得上明主的,可是他也有許多不明智的地方:無法解決任圜與安重誨的矛盾,使得任圜被冤殺;無法解決兩川問題,使孟知祥反叛建立後蜀;聽信他人讒言,使安重誨夫婦於自宅被處死;久不立太子,使李從榮生疑並最終被殺。其後續位的後唐閔帝昏庸無能,周遭又是一些無用之輩,後唐很快也跟著滅亡了。


\subsubsection{天成}

\begin{longtable}{|>{\centering\scriptsize}m{2em}|>{\centering\scriptsize}m{1.3em}|>{\centering}m{8.8em}|}
  % \caption{秦王政}\
  \toprule
  \SimHei \normalsize 年数 & \SimHei \scriptsize 公元 & \SimHei 大事件 \tabularnewline
  % \midrule
  \endfirsthead
  \toprule
  \SimHei \normalsize 年数 & \SimHei \scriptsize 公元 & \SimHei 大事件 \tabularnewline
  \midrule
  \endhead
  \midrule
  元年 & 926 & \tabularnewline\hline
  二年 & 927 & \tabularnewline\hline
  三年 & 928 & \tabularnewline\hline
  四年 & 929 & \tabularnewline\hline
  五年 & 930 & \tabularnewline
  \bottomrule
\end{longtable}

\subsubsection{长兴}

\begin{longtable}{|>{\centering\scriptsize}m{2em}|>{\centering\scriptsize}m{1.3em}|>{\centering}m{8.8em}|}
  % \caption{秦王政}\
  \toprule
  \SimHei \normalsize 年数 & \SimHei \scriptsize 公元 & \SimHei 大事件 \tabularnewline
  % \midrule
  \endfirsthead
  \toprule
  \SimHei \normalsize 年数 & \SimHei \scriptsize 公元 & \SimHei 大事件 \tabularnewline
  \midrule
  \endhead
  \midrule
  元年 & 930 & \tabularnewline\hline
  二年 & 931 & \tabularnewline\hline
  三年 & 932 & \tabularnewline\hline
  四年 & 933 & \tabularnewline
  \bottomrule
\end{longtable}


%%% Local Variables:
%%% mode: latex
%%% TeX-engine: xetex
%%% TeX-master: "../../Main"
%%% End:

%% -*- coding: utf-8 -*-
%% Time-stamp: <Chen Wang: 2019-12-24 16:48:51>

\subsection{闵帝\tiny(933-934)}

\subsubsection{生平}

唐閔帝李從厚(914年-934年),小字菩薩奴,五代時期後唐皇帝,為後唐明宗李嗣源之子,母昭懿皇后夏氏,有一胞兄李从荣。李嗣源因李從厚與自己相像,特別喜歡他。

李從厚於李嗣源在位時,原被封為宋王。後唐長興四年(933年),李嗣源病重,本為繼承人的秦王李從榮誤以為李嗣源已死,為確保能夠繼位,遂帶兵入宮,事敗被殺。李嗣源不得已,召時任天雄節度使的李從厚回京。不久,李嗣源去世,李從厚繼位。次年(934年),改年號應順。

李從厚即帝位後,信任朱弘昭、馮贇等人,二人於應順元年(934年),調動各重要節度使,準備削藩,鳳翔節度使潞王李從珂恐懼,遂反,攻入京師洛陽,李從厚出逃魏州,途经卫州,遇到河东节度使石敬瑭。石敬瑭無意救之。李从厚的亲随不满石敬瑭,抽刀要杀石敬瑭,结果反被石敬瑭的侍卫杀死。石敬瑭的部將劉知遠尽杀闵帝亲随,把闵帝安置在卫州後,不久離去。皇太后下令降閔帝為鄂王。不久闵帝為潞王李從珂派人所殺。後晉高祖石敬瑭稱帝後,將他諡為閔皇帝。

李從厚個性仁慈,對兄弟很和睦,雖遭李從榮忌恨,卻能坦誠相待,所以當時才能逃過一劫。本來與李從珂也沒有過節,只因輕易地聽信周遭人的讒言,才會招來大禍。

\subsubsection{应顺}

\begin{longtable}{|>{\centering\scriptsize}m{2em}|>{\centering\scriptsize}m{1.3em}|>{\centering}m{8.8em}|}
  % \caption{秦王政}\
  \toprule
  \SimHei \normalsize 年数 & \SimHei \scriptsize 公元 & \SimHei 大事件 \tabularnewline
  % \midrule
  \endfirsthead
  \toprule
  \SimHei \normalsize 年数 & \SimHei \scriptsize 公元 & \SimHei 大事件 \tabularnewline
  \midrule
  \endhead
  \midrule
  元年 & 934 & \tabularnewline
  \bottomrule
\end{longtable}


%%% Local Variables:
%%% mode: latex
%%% TeX-engine: xetex
%%% TeX-master: "../../Main"
%%% End:

%% -*- coding: utf-8 -*-
%% Time-stamp: <Chen Wang: 2021-11-01 15:21:28>

\subsection{末帝李从珂\tiny(934-937)}

\subsubsection{生平}

李從珂(885年2月11日-937年1月11日),鎮州(今河北正定)人,五代時期後唐皇帝,史稱後唐末帝或後唐廢帝,本姓王,小字二十三,因此又被叫做阿三。

李從珂十餘歲時,其母魏氏被當時仍是將領的後唐明宗李嗣源所擄,李從珂不久就被李嗣源改名並收為養子。長大後身形雄偉健壯,又驍勇善戰,常隨李嗣源南征北討,頗得其喜愛。莊宗也曾赞道:“阿三不惟与我同齿,敢战亦相类。”

同光三年(925年),李嗣源因家在太原,上表请求让时任卫州刺史的李从珂为北京(太原)内牙马步都指挥使,以便相聚,却导致莊宗大怒,李从珂被黜为突骑指挥使,率数百人戍石门镇。次年李嗣源平定邺都之乱时被迫造反,李从珂率军与之会合,助其夺位。

李嗣源即帝位後,李從珂曾任河中節度使之職,然因與權臣樞密使安重誨之前有過節,在長興元年(930年),被安重誨設計解除軍權,回京師洛陽居住。次年(931年),安重誨失勢,李從珂再受重用,被任命為左衛大將軍、西京(長安)留守。長興三年(932年),被改命為鳳翔節度使。長興四年(933年),封潞王。

後唐應順元年(934年),閔帝李從厚聽信大臣的建議,調動各重要節度使之職,準備削弱藩鎮的實力,李從珂恐懼,遂反。李從厚命王思同率大軍討伐,王思同围攻凤翔城(今陝西鳳翔)。凤翔城墙低,护城河窄浅,根本无法固守。眼看鳳翔即將陷落,未料討伐軍將兵驕橫,貪圖賞賜,李從珂抓住這點誘使討伐軍叛變,反敗為勝,不久以摧枯拉朽之勢攻入京師洛陽,即帝位,改元清泰,並派人將逃亡的李從厚殺害。

李嗣源之婿石敬瑭時任重鎮河東節度使之職,李從珂與他二人當初在李嗣源手下皆以勇力過人著稱,彼此存有競爭之心。因此李從珂即位後,對石敬瑭愈發猜忌,而石敬瑭亦有謀反之意。

清泰三年(936年),石敬瑭以調鎮他處試探,而李從珂果真將石敬瑭改任天平節度使,石敬瑭因此叛變,同時向契丹乞援。李從珂命各鎮聯合討伐,不料因聯軍各懷鬼胎,致大敗於團柏谷,石敬瑭與契丹大軍得以順利南下進逼京師洛陽,李從珂無計可施,於閏十一月二十六日(陽曆為937年1月11日)自焚而死。死後無諡號及廟號,史家稱之為末帝或廢帝。传国玉玺亦在此时遗失不知所踪。

李從珂是一個矛盾的人,時運順他之時,他竟敢潛伏到敵軍陣營內,殺死敵軍並且砍下對方望桿扛回去;時運離他而去時,整天消沉著在宮中飲酒哭泣,最後自焚,並不是大臣沒有給他好的意見,而是他自己優柔寡斷,無法及時下決定,最終誤國。


\subsubsection{清泰}

\begin{longtable}{|>{\centering\scriptsize}m{2em}|>{\centering\scriptsize}m{1.3em}|>{\centering}m{8.8em}|}
  % \caption{秦王政}\
  \toprule
  \SimHei \normalsize 年数 & \SimHei \scriptsize 公元 & \SimHei 大事件 \tabularnewline
  % \midrule
  \endfirsthead
  \toprule
  \SimHei \normalsize 年数 & \SimHei \scriptsize 公元 & \SimHei 大事件 \tabularnewline
  \midrule
  \endhead
  \midrule
  元年 & 934 & \tabularnewline\hline
  二年 & 935 & \tabularnewline\hline
  三年 & 936 & \tabularnewline
  \bottomrule
\end{longtable}


%%% Local Variables:
%%% mode: latex
%%% TeX-engine: xetex
%%% TeX-master: "../../Main"
%%% End:


%%% Local Variables:
%%% mode: latex
%%% TeX-engine: xetex
%%% TeX-master: "../../Main"
%%% End:
