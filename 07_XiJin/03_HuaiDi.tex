%% -*- coding: utf-8 -*-
%% Time-stamp: <Chen Wang: 2019-12-18 13:21:01>

\section{怀帝\tiny(306-313)}

\subsection{生平}

晉懷帝司馬熾(284年-313年),字豐度,西晉的第三代皇帝,司馬炎的第二十五子,其正式諡號為「孝懷皇帝」,後世省略「孝」字稱「晋怀帝」。

司馬熾生於太康五年(284年),生母為晉武帝中才人王媛姬,王媛姬死后由晋武帝继后杨芷抚养。武帝太熙元年(290年)司馬熾被封為豫章王,四月,司馬炎病死。太子司馬衷即位,是為晉惠帝,在晉惠帝在位期間爆發的八王之亂中,司馬熾並未加入亂事,並且行事低調,不太熱衷於交結賓客,愛好鑽研史籍。司馬熾本人並無雄才大略,最初擔任散騎常侍,永康二年(301年)趙王司馬倫廢晉惠帝時,司馬熾的散騎常侍也被罷黜,同年四月晉惠帝復位後,改元永寧元年,熾任射聲校尉。永寧三年(304年)出任鎮北大將軍,同年被立為皇太弟。但是立司馬熾為皇太弟,是由於成都王司馬穎和河間王司馬顒對立之下的結果,其實司馬熾本人並沒有權力的野心。

光熙元年十一月庚午(307年1月8日)東海王司馬越毒死惠帝,1月11日,司馬熾即位,改元永嘉,司馬越为太傅辅政,政局為司馬越把持。司马炽亦改葬追谥先前被废的养母杨芷。在此期間,匈奴等少數民族也開始建立獨立的政權,其中劉淵已建漢國,但是晉朝內部的權力鬥爭也日漸嚴重。永嘉五年(311年)正月,晉懷帝密詔苟晞討司馬越,三月發佈詔書討伐,司馬越於同月病死,眾共推王衍為元帥。四月王衍遣軍隊在護送司馬越靈柩回到東海封國時,與漢國鎮東大將軍石勒的二萬軍隊於苦縣(河南鹿邑)寧平城(河南省鄲城縣東寧平鄉)作戰,晋军全被殲滅,石勒焚燒司馬越的靈樞。王衍被擒時,勸石勒建國稱帝,以求苟活,但仍被石勒活埋。西晉最後一支主要兵力被消滅,已無可戰之兵。

311年六月劉淵之子劉聰的軍隊攻入洛陽,晉懷帝在逃往長安途中被俘,太子司馬詮被殺,史稱永嘉之乱。晉懷帝被送往平陽,劉聰告訴他:“卿為豫章王時,朕嘗與王武子(濟)相造,武子示朕於卿,卿言聞其名久矣……”後封為會稽公,並被囚禁。313年正月,晉懷帝在朝會上被命令為斟酒的僕人,有晉朝舊臣號哭,令劉聰反感,不久用毒酒毒殺懷帝,得年30歲,葬處不明。

\subsection{永嘉}

\begin{longtable}{|>{\centering\scriptsize}m{2em}|>{\centering\scriptsize}m{1.3em}|>{\centering}m{8.8em}|}
  % \caption{秦王政}\
  \toprule
  \SimHei \normalsize 年数 & \SimHei \scriptsize 公元 & \SimHei 大事件 \tabularnewline
  % \midrule
  \endfirsthead
  \toprule
  \SimHei \normalsize 年数 & \SimHei \scriptsize 公元 & \SimHei 大事件 \tabularnewline
  \midrule
  \endhead
  \midrule
  元年 & 307 & \tabularnewline\hline
  二年 & 308 & \tabularnewline\hline
  三年 & 309 & \tabularnewline\hline
  四年 & 310 & \tabularnewline\hline
  五年 & 311 & \tabularnewline\hline
  六年 & 312 & \tabularnewline\hline
  七年 & 313 & \tabularnewline
  \bottomrule
\end{longtable}


%%% Local Variables:
%%% mode: latex
%%% TeX-engine: xetex
%%% TeX-master: "../Main"
%%% End:
