%% -*- coding: utf-8 -*-
%% Time-stamp: <Chen Wang: 2021-11-01 11:38:22>

\section{惠帝司马衷\tiny(290-306)}

\subsection{生平}

晋惠帝司马衷(259年2月13日-307年1月8日),字正度,是晋武帝司马炎的次子,西晋的第二位皇帝,290年至307年在位,其正式諡號為「孝惠皇帝」,後世省略「孝」字稱「晉惠帝」。在他的统治期间发生八王之乱,西晋走向灭亡。

晋惠帝生于259年2月13日。晋惠帝于泰始三年(267年)被立为太子,他的母亲是晋武帝的皇后楊艷。作为次子他被立为太子是因为他的哥哥司马轨很早就死了,也有說是晋武帝為了將來傳位給他寵愛的聰明孫子湣懷太子司马遹。

晋惠帝一般被評價为“甚愚”,王夫之說他是“土木偶人”,《晋书》中記載武帝担忧晋惠帝的能力,多次对他进行考验,而惠帝則在太子妃賈南風及謀臣的獻策下通過了这些考验。即位后,他显然无法解决他统治时期的政治困难,造成了八王之乱,成为了傀儡,最后被东海王司马越毒死。

根據刘驰的研究认为,虽然惠帝昏庸,政治才能低下,无法应对朝中局面,但是从今日的医学概念来判斷,他不能夠算作智能障礙。

272年惠帝奉武帝命娶贾充之女贾南风为太子妃。

290年武帝去世,惠帝登基,尊繼母杨芷(楊艷的堂妹)为皇太后,妻子贾南风为皇后,司马遹为太子。惠帝当政后非常信任他的皇后。因此贾氏专权,甚至假造惠帝的诏书。291年迫害皇太后,废其太后位,后又杀大臣如太宰司马亮。291年賈皇后又杀皇太后。

294年和296年匈奴和其他少数民族反叛,氐人齐万年称帝,一直到299年这次反叛才被消灭。

同年賈后开始迫害太子遹,首先废他的太子地位。次年杀太子。这个举动成为许多反对贾后专政的皇族开始行动的起点。赵王司马伦假造诏书废杀贾后,杀大臣如司空张华等,自领相国位,这是八王之乱的开始。恢复原太子的地位,立故太子之子司马臧为皇太孫。300年八月淮南王司马允举兵讨伐司马伦,兵败被杀。同年十二月,益州刺史赵廞协同从中原逃到四川的流民在成都造反。

301年司马伦篡位,自立为皇帝,惠帝被改为太上皇,太孫司马臧被杀。三月,齐王司马冏起兵反司马伦,受到成都王司马颖、河间王司马颙、常山王司马乂等的支持。司马伦兵败。淮陵王司马漼杀司马伦的党羽,驱逐司马伦,引惠帝复位。司马伦被杀。五月,立襄阳王司马尚为皇太孙,並以羊獻容为皇后。六月,东莱王司马蕤谋推翻司马冏的专权,事漏被废。十二月,李特开始在四川反晋,这是成汉的起点。

302年初皇太孙司马尚夭折,司马覃被立为太子。五月,李特在四川击败了司马颙派去讨伐他的军队,杀广汉太守张微,自立为大将军。十二月,司马颖、司马颙、新野王司马歆和范阳王司马虓在洛阳聚会反司马冏的专政。司马乂乘机杀司马冏,成为朝内的权臣。

303年三月李特在攻成都时被杀,但四月他的儿子李雄就占领了成都,到年末,李雄几乎占领了整个四川盆地。五月张昌、丘沈反,建国汉,杀司马歆。八月,司马颖和司马颙讨伐司马乂。十月,司马颙的军队攻擊首都洛陽,在此后的洗劫中上万人死亡。此后两军在洛陽城外对阵,连十三岁的少年都被征军,同时两军都征募匈奴等的军队。最后司马乂兵败被杀。司马颙成为晋朝举足轻重的人物。

304年初惠帝感到受到司马颙的威胁越来越大,因此下密诏给刘沈和皇甫重攻司马颙,但没有成功。司马颙的军队在洛阳大肆抢劫。二月废皇后羊氏,废皇太子司马覃,立司马颖为皇太弟。司马颖和司马颙专政。但六月京城又发生政变,司马颖被逐,羊氏復位为皇后,司马覃復位为太子。七月,惠帝率军讨伐司马颖,在荡阴被司马颖的军队战败,惠帝面部中伤,身中三箭,被司马颖俘虏。羊氏和司马覃再次被废。八月,司马颖被安北将军王浚战败,他挟持惠帝逃亡到洛阳。一路上只有粗米为饭。十一月,惠帝又被司马颙的将军张方劫持到长安,张方的军队抢劫皇宫,将皇宫内的宝藏洗劫一空。到年末司马颙再次在长安一揽大权,司马越成为太傅。同年李雄在成都称成都王,成汉建国,刘渊自称汉王,建立前赵。305年司马颙和张方的军队、司马颖的军队、司马越的军队和范阳王司马虓的军队在中原混战,基本上中央政府已经不存在,中国边缘的地区纷纷独立。到305年末,司马越战胜,司马颙杀张方向司马越请和,但无效。

306年司马越手下的鲜卑军队攻入长安,大肆抢劫,二万多人被杀。九月,司马颖被俘,后被杀。

307年光熙元年十一月庚午(307年1月8日),惠帝于长安显阳殿去世,可能是被司马越毒死的。惠帝死后葬太阳陵。

因為他在位期間全國先後經歷八王之亂和五胡亂華等重大事件的影響,連帶使他在中國歷史上非自願地創造紀錄:最多皇儲(?,以上太弟、太孫加上太子,待另考)。最多皇太弟(2個),司馬穎、司馬熾。最多皇太孫(2個),司馬臧、司馬尚。被自己的長輩尊為太上皇(司馬倫為惠帝的叔公)。其第二任皇后羊獻容曾任2個不同政權君主的皇后,除晉惠帝外,後來又為前趙帝劉曜之后。主政18年,用了9個年號。除了時間最長的元康年號(9年),9年間用了8個年號,更換年號的頻率和武則天相當,使用過的年號數量則居武則天之後(武則天為帝的16年使用15個年號,為帝前握有實權的6年使用3個年號),中國眾多皇帝的第二位。

由於晉惠帝的昏庸愚蠢名動天下,使原本不帶貶意的諡號“惠”在他用過之後成了昏君代名詞,故後來如唐宋這類長期統治中原的主流大朝代皆避免用惠字當皇帝諡號。

有人向晉惠帝报告老百姓无食物吃(天下荒饥,百姓饿死),他反问:“何不食肉糜?”(為何不吃肉粥?)。

还有一次,晋惠帝游上林苑,听到蛤蟆叫,问身旁的人:“此鸣者,为官乎?为私乎?”(在叫的蛤蟆是官家的还是私人的啊?)

戰亂時,嵇康之子嵇紹以身捍衛,血噴滿惠帝整身 《晋书·忠义传·嵇绍》:“绍以天子蒙尘,承诏驰诣行在所。值王师败绩于荡阴,百官及侍卫莫不散溃,唯绍俨然端冕,以身捍卫,兵交御辇,飞箭雨集。绍遂被害于帝侧,血溅御服,天子深哀叹之。及事定,左右欲浣衣,帝曰:‘此嵇侍中血,勿去。’(這是嵇侍中的血,不能去掉)”

\subsection{永熙}

\begin{longtable}{|>{\centering\scriptsize}m{2em}|>{\centering\scriptsize}m{1.3em}|>{\centering}m{8.8em}|}
  % \caption{秦王政}\
  \toprule
  \SimHei \normalsize 年数 & \SimHei \scriptsize 公元 & \SimHei 大事件 \tabularnewline
  % \midrule
  \endfirsthead
  \toprule
  \SimHei \normalsize 年数 & \SimHei \scriptsize 公元 & \SimHei 大事件 \tabularnewline
  \midrule
  \endhead
  \midrule
  元年 & 290 & \tabularnewline
  \bottomrule
\end{longtable}

\subsection{永平}

\begin{longtable}{|>{\centering\scriptsize}m{2em}|>{\centering\scriptsize}m{1.3em}|>{\centering}m{8.8em}|}
  % \caption{秦王政}\
  \toprule
  \SimHei \normalsize 年数 & \SimHei \scriptsize 公元 & \SimHei 大事件 \tabularnewline
  % \midrule
  \endfirsthead
  \toprule
  \SimHei \normalsize 年数 & \SimHei \scriptsize 公元 & \SimHei 大事件 \tabularnewline
  \midrule
  \endhead
  \midrule
  元年 & 291 & \tabularnewline
  \bottomrule
\end{longtable}

\subsection{元康}

\begin{longtable}{|>{\centering\scriptsize}m{2em}|>{\centering\scriptsize}m{1.3em}|>{\centering}m{8.8em}|}
  % \caption{秦王政}\
  \toprule
  \SimHei \normalsize 年数 & \SimHei \scriptsize 公元 & \SimHei 大事件 \tabularnewline
  % \midrule
  \endfirsthead
  \toprule
  \SimHei \normalsize 年数 & \SimHei \scriptsize 公元 & \SimHei 大事件 \tabularnewline
  \midrule
  \endhead
  \midrule
  元年 & 291 & \tabularnewline\hline
  二年 & 292 & \tabularnewline\hline
  三年 & 293 & \tabularnewline\hline
  四年 & 294 & \tabularnewline\hline
  五年 & 295 & \tabularnewline\hline
  六年 & 296 & \tabularnewline\hline
  七年 & 297 & \tabularnewline\hline
  八年 & 298 & \tabularnewline\hline
  九年 & 299 & \tabularnewline
  \bottomrule
\end{longtable}

\subsection{永康}

\begin{longtable}{|>{\centering\scriptsize}m{2em}|>{\centering\scriptsize}m{1.3em}|>{\centering}m{8.8em}|}
  % \caption{秦王政}\
  \toprule
  \SimHei \normalsize 年数 & \SimHei \scriptsize 公元 & \SimHei 大事件 \tabularnewline
  % \midrule
  \endfirsthead
  \toprule
  \SimHei \normalsize 年数 & \SimHei \scriptsize 公元 & \SimHei 大事件 \tabularnewline
  \midrule
  \endhead
  \midrule
  元年 & 300 & \tabularnewline\hline
  二年 & 301 & \tabularnewline
  \bottomrule
\end{longtable}

\subsection{永宁}

\begin{longtable}{|>{\centering\scriptsize}m{2em}|>{\centering\scriptsize}m{1.3em}|>{\centering}m{8.8em}|}
  % \caption{秦王政}\
  \toprule
  \SimHei \normalsize 年数 & \SimHei \scriptsize 公元 & \SimHei 大事件 \tabularnewline
  % \midrule
  \endfirsthead
  \toprule
  \SimHei \normalsize 年数 & \SimHei \scriptsize 公元 & \SimHei 大事件 \tabularnewline
  \midrule
  \endhead
  \midrule
  元年 & 301 & \tabularnewline\hline
  二年 & 302 & \tabularnewline
  \bottomrule
\end{longtable}

\subsection{太安}

\begin{longtable}{|>{\centering\scriptsize}m{2em}|>{\centering\scriptsize}m{1.3em}|>{\centering}m{8.8em}|}
  % \caption{秦王政}\
  \toprule
  \SimHei \normalsize 年数 & \SimHei \scriptsize 公元 & \SimHei 大事件 \tabularnewline
  % \midrule
  \endfirsthead
  \toprule
  \SimHei \normalsize 年数 & \SimHei \scriptsize 公元 & \SimHei 大事件 \tabularnewline
  \midrule
  \endhead
  \midrule
  元年 & 302 & \tabularnewline\hline
  二年 & 303 & \tabularnewline
  \bottomrule
\end{longtable}

\subsection{永安}

\begin{longtable}{|>{\centering\scriptsize}m{2em}|>{\centering\scriptsize}m{1.3em}|>{\centering}m{8.8em}|}
  % \caption{秦王政}\
  \toprule
  \SimHei \normalsize 年数 & \SimHei \scriptsize 公元 & \SimHei 大事件 \tabularnewline
  % \midrule
  \endfirsthead
  \toprule
  \SimHei \normalsize 年数 & \SimHei \scriptsize 公元 & \SimHei 大事件 \tabularnewline
  \midrule
  \endhead
  \midrule
  元年 & 304 & \tabularnewline
  \bottomrule
\end{longtable}

\subsection{建武}

\begin{longtable}{|>{\centering\scriptsize}m{2em}|>{\centering\scriptsize}m{1.3em}|>{\centering}m{8.8em}|}
  % \caption{秦王政}\
  \toprule
  \SimHei \normalsize 年数 & \SimHei \scriptsize 公元 & \SimHei 大事件 \tabularnewline
  % \midrule
  \endfirsthead
  \toprule
  \SimHei \normalsize 年数 & \SimHei \scriptsize 公元 & \SimHei 大事件 \tabularnewline
  \midrule
  \endhead
  \midrule
  元年 & 304 & \tabularnewline
  \bottomrule
\end{longtable}

\subsection{永兴}

\begin{longtable}{|>{\centering\scriptsize}m{2em}|>{\centering\scriptsize}m{1.3em}|>{\centering}m{8.8em}|}
  % \caption{秦王政}\
  \toprule
  \SimHei \normalsize 年数 & \SimHei \scriptsize 公元 & \SimHei 大事件 \tabularnewline
  % \midrule
  \endfirsthead
  \toprule
  \SimHei \normalsize 年数 & \SimHei \scriptsize 公元 & \SimHei 大事件 \tabularnewline
  \midrule
  \endhead
  \midrule
  元年 & 304 & \tabularnewline\hline
  二年 & 305 & \tabularnewline\hline
  三年 & 306 & \tabularnewline
  \bottomrule
\end{longtable}

\subsection{光熙}

\begin{longtable}{|>{\centering\scriptsize}m{2em}|>{\centering\scriptsize}m{1.3em}|>{\centering}m{8.8em}|}
  % \caption{秦王政}\
  \toprule
  \SimHei \normalsize 年数 & \SimHei \scriptsize 公元 & \SimHei 大事件 \tabularnewline
  % \midrule
  \endfirsthead
  \toprule
  \SimHei \normalsize 年数 & \SimHei \scriptsize 公元 & \SimHei 大事件 \tabularnewline
  \midrule
  \endhead
  \midrule
  元年 & 306 & \tabularnewline
  \bottomrule
\end{longtable}


%%% Local Variables:
%%% mode: latex
%%% TeX-engine: xetex
%%% TeX-master: "../Main"
%%% End:
