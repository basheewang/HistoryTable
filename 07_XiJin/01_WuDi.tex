%% -*- coding: utf-8 -*-
%% Time-stamp: <Chen Wang: 2021-11-01 11:38:14>

\section{武帝司馬炎\tiny(266-290)}

\subsection{生平}

晉武帝司馬炎(236年-290年5月16日),字安世,河内郡温縣(今河南省焦作市温县)人,曹魏权臣司马昭长子,晉朝開國皇帝,諡號武皇帝,在位二十五年。

魏咸熙二年(266年2月8日)十二月丙寅,晋王、相国司馬炎逼迫魏元帝禪讓,即位為帝,定有天下之号曰晉,改年號泰始。在位期间,封同姓诸王,以郡为国,置军士,希望互相维系,拱卫中央。晋武帝采取一系列经济措施以发展生产,屡次责令郡县官劝课农桑,并严禁私募佃客。又招募原吴、蜀地区人民北来,充实北方,并废屯田制,使屯田民成为州郡编户。

太康元年(280年),颁行户调式,包括占田制、户调制和品官占田荫客制。太康年间出现一片繁荣景象。晋武帝鉴于曹魏末期为政严苛,风俗颓废,生活豪奢,乃“矫以仁俭”,鳏寡孤独不能自存者赐穀人五斛,免逋债宿负,诏郡国守相巡行属县,并能容纳直言。还重视法律,亲自向百姓讲解賈充等人上所刊修律令,并亲身听讼录囚。但灭吳后,逐渐怠惰政事,沉迷女色,扩充后宫,荒淫无度。

鑒於曹魏宗室力量薄弱才讓其父祖有機可乘,因此他巩固皇权而大封宗室。然而诸王统率兵马各据一方,晋武帝死后,诸王为争夺权力,内讧不已,形成16年的内战,史称八王之乱。

司馬炎出生于236年,為司馬昭長子(母亲王元姬是經学家王肅女儿),曾出任中撫軍等要职。司马昭曾想为他求娶阮籍女,但阮籍连续醉酒六十多天,司马昭找不到提出求亲的机会,只得作罢。

司馬昭次子司馬攸是司马师的過繼養子,司馬昭曾因认为自己的權位来自於司马师,有意讓司馬攸繼承晋王位,以報答司馬師,但因重臣反對,只好於咸熙二年(265年)五月立司馬炎為世子。同年八月司馬昭過世之後,司馬炎繼承晉王的爵位。

咸熙二年(265年),司马昭病死,享年55岁。司马炎继承相国晋王位,掌握全国军政大权。经过精心准备,同年十二月,仿效当年曹丕篡汉的故事,为自己登基做准备。在司马炎接任相国后,就有一些人受司马炎指使劝说曹奂早点让位。不久,曹奂下诏书说:“晋王,你家世代辅佐皇帝,功勋高过上天,四海蒙受司马家族的恩泽,上天要我把皇帝之位让给你,请顺应天命,不要推辞!”司马炎却假意多次推让。司马炎的心腹太尉何曾、卫将军賈充等人,带领满朝文武官员再三劝谏。司马炎多次推让后,才接受曹奂禅让,封曹奂为陈留王。司马炎登上帝位,改国号为「晋」,史称为「西晋」,晋王司马炎成了晋武帝。

晋武帝施行了一系列进步政策增强国力,发展生产。此时孙吴局势混乱,吴帝孙皓不修内政又穷极奢侈,民心不附。为了防御吴国,司马炎派羊祜镇守襄阳与吴将陆抗对峙,派王濬于益州大造船舰。泰始十年(274年)陆抗去世,二年後羊祜提议伐吴,遭群臣反对而作罢。咸寧四年(278年)羊祜病故,临终推荐杜预镇守荆州。咸寧五年(279年)西北秃发树机能之乱始平,王濬、杜预上书司马炎,认为是伐吴之时,賈充、荀勖等认为西北未定而反对。最后司马炎决定于该年十二月进攻吴国,史称晋灭吴之战。他以賈充为大都督,上游王濬、唐彬军,中游杜预、胡奋、王戎军,下游王浑、司马伷军多路并进。于隔年三月逼近建业,孙皓见大势已去而投降,孙吴灭亡,西晋统一天下,三国时期结束。

泰始六年(270年),河西鲜卑领主秃发树机能叛,次年匈奴刘猛也随之出关。泰始八年(272年),司马炎派何桢招降李恪平定刘猛叛乱。咸寧元年(275年),司马炎释放奴婢替代士兵屯田,树机能归降,拓跋部沙漠汗出使晋朝,马循平定鲜卑。咸寧三年(277年)树机能复叛,司马骏帅文鸯等败树机能,降鲜卑二十万。沙漠汗被鲜卑旧贵族杀害,卫瓘平定拓跋部内乱。咸寧五年(279年)司马炎派马隆前往凉州平叛,秃发部众杀禿髮樹機能降。

建国后采取一系列经济措施以发展生产,屡次责令郡县官劝课农桑,并严禁私募佃客。又招募原吴、蜀地区人民北来,充实北方,并废屯田制,使屯田民成为州郡编户,太康三年(282年)户达到377万户。《晋书·食货志》说:“平吴之后,……天下无事,赋税平均,人咸安其业而乐其事。”太康元年,颁行户调式,包括占田制、户调制和品官占田荫客制。太康年间出现一片繁荣景象,史称“太康之治”。

司马炎鉴于魏宗室衰微,帝室孤弱,终致灭亡之教训乃大封皇族为藩王,以对抗士族。始则封王不就国,官于京师以辅皇室,继则分遣诸王就国,都督诸军事,后又出使镇要害地。此举目的,是为对抗士族中野心家。但“八王之乱”证明,这种政策反而使这些手握重兵的诸王中涌现了许多野心家。

西晋之所以重任宗室,实际上与其政权的结构有关。晋是以皇室司马氏为首门阀贵族联合统治,皇室作为一个家族驾于其它家族之上,皇帝是这个第一家族的代表,因而其家族家成员有资格也有必要取得更大权势,以保持其优越地位。[來源請求]

全国统一后,司马炎下诏:“悉去州郡兵,大郡置武吏百人,小郡五十人”,即规定:诸州无事者罢其兵。刺史只作为监司,罢将军名号,不领兵,也不兼领兵的校尉官。实行军民分治,都督校尉治军,刺史太守治民。罢州郡兵,一方面可使地方官专心民事,另一是扩大承担赋役的课丁。兵役是汉灵帝光和七年(184年)以后农民最沉重的负担,免除这负担,对恢复生产意义重大,但也因悉去州郡兵,连治安都没办法维持,因此到永寧元年(301年),天下大乱时,无力控制局面。

西晋的皇族和贵族都有优裕的经济基础,政治的安定与统一更帮助他们累积了大量的财富,于是纵情享受,过着豪华奢侈的生活。晋武帝领先作了荒淫奢纵的表率,《晋书·胡贵嫔传》称:晋武“多内宠,平吴后,复纳吴王孙皓宫人数千,自此掖庭殆将万人,而并宠者甚众,帝莫知所适,常乘羊车,恣其所之,至使宴寝”,奢侈浪费,风气日渐败坏。公卿贵游也跟着竞富争豪,大臣何曾每天吃饭用一万钱,还“无处下箸”,他的儿子何劭一定要吃四方畛异,一天膳费二万钱。王恺是武帝的母舅,曾与当时首富石崇比赛炫耀财富,争夸豪丽。为维持这种奢靡腐化的生活,必然加紧聚敛,因此贪污纳贿,习以为常,当时有人指:“奢侈之费,甚于天灾”,可见为害之大。間接養成了其子的生活態度,當有人向晉惠帝報告老百姓無食物吃(天下荒飢,百姓餓死),晉惠帝卻反問:「何不食肉糜?」。

太熙元年(290年)四月己酉(5月16日),晋武帝司马炎驾崩于含章殿,享年五十四歲。五月辛未(6月12日),葬於峻陽陵。其次子司马衷即位,是为晋惠帝。不过一年後,皇后贾南风发动政变,杀死总揽朝政的大臣杨骏;接着又发生了“八王之乱”。建兴四年(316年),刘渊的侄子刘曜攻破长安,俘获末代皇帝司马邺,西晋亡国。时距司马炎之死只有26年。

司马炎最突出的性格特点正如《晋书》论赞所言,“宇量弘厚”,性情温和宽厚而较能包容异己者:不杀退位的曹魏皇室和投降的蜀漢、孙吴皇室,封曹奂为陈留王、孙皓为归命侯,并允许刘协和刘禅的子孙继续承袭山阳公和安乐公的爵位(相对来说这与后继的南朝君主们的作风不太一样)。王恺、石崇斗富,石崇公然砸碎司马炎赐给王恺的珊瑚,却没有因此而受到司马炎的责罚。泰始九年,司马炎下诏为邓艾平反,厚葬邓艾尸首、归还籍没的田宅并赠给其孙邓朗官职。司马炎纳山涛之议,允许被司马昭处死的嵇康之子嵇绍入朝为官。前蜀汉官员李密上《陈情表》,以侍奉祖母为由拒绝出仕晋朝,司马炎同意并赐给李密的祖母粮米布帛。王浑上表弹劾王濬在灭吴战争中违背诏令,不听调遣,司马炎只是下诏责备王濬而未予以任何处罚。冯紞在反对张华拜相时,当着司马炎的面说“臣窃谓锺会之衅,颇由太祖。”亦即钟会的叛乱是由于司马昭的失策造成的。司马炎虽当场发怒“变色”但最终仍采纳冯紞的意见,改任张华为太常。司隶校尉刘毅在南郊祭典上批评司马炎卖官敛财,并说司马炎连汉末的桓灵二帝都不如:“桓、灵卖官,钱入官库;陛下卖官,钱入私门。以此言之,殆不如也。”司马炎只是“大笑”而并未对刘毅有何不利。刘毅在之后还升任尚书左仆射。

何曾:“聪明神武,有超世之才。”“主上开创大业,吾每宴见,未尝闻经国远图,惟说平生常事,非贻厥孙谋之道也,及身而已,后嗣其殆乎!”(《资治通鉴·卷第八十七·晋纪九》)

刘毅:“桓、灵卖官,钱入官库;陛下卖官,钱入私门。以此言之,殆不如也。”(《晋书·列传第十五》)

陆云:“世祖武皇帝临朝拱默,训世以俭,即位二十有六载,宫室台榭无所新营,屡发明诏,厚戒丰奢。”(《晋书·列传第二十四》)

曹毗:“于穆武皇,允龚钦明。应期登禅,龙飞紫庭。百揆时序,听断以情。殊域既宾,伪吴亦平。晨流甘露,宵映朗星。野有击壤,路垂颂声。”(宋书·卷二十◎志第十◎乐二)

干宝:“至于世祖,遂享皇极,仁以厚下,俭以足用,和而不弛,宽而能断,掩唐、虞之旧域,班正朔于八荒,于时有“天下无穷人”之谚,虽太平未洽,亦足以明民乐其生矣。武皇既崩,山陵未干而变难继起。宗子无维城之助,师尹无具瞻之贵,朝为伊、周,夕成桀、跖;国政迭移于乱人,禁兵外散于四方,方岳无钧石之镇,关门无结草之固。戎、羯称制,二帝失尊,何哉?树立失权,托付非才,四维不张,而苟且之政多也。”(《资治通鉴·卷第八十九》)

谢灵运:“世祖受命,祯祥屡臻,苛慝不作,万国欣戴。远至迩安,德足以彰,天启其运,民乐其功矣。反古之道,当以美事为先。今五等罔刑,井田王制,凡诸礼律,未能定正,而采择嫔媛,不拘华门者。昔武王伐纣,归倾宫之女,不以助纣为虐。而世祖平皓,纳吴妓五千,是同皓之弊。妇人之封,六国乱政。如追赠外曾祖母,违古之道。凡此非事,并见前书,诚有点於徽猷,史氏所不敢蔽也。”(《太平御览·卷九十六》)

虞世南:“武帝平一天下,谁曰不然,至於创业垂统,其道则阙矣。夫帝王者,必立德立功,可大可久,经之以仁义,纬之以文武,深根固蒂,贻厥子孙,一言一行,以为轨范,垂之万代,为不可易。武帝平吴之後,怠於政事,蔽惑邪佞,留心内宠,用冯紞之谗言,拒和峤之正谏,智士永叹,有识寒心。以此国风,传之庸子,遂使坟土未乾,四海鼎沸,衣冠殄灭,县宇星分,何曾之言,於是信矣。其去明主,不亦远乎?”(《唐文拾遗》卷十三)

房玄龄《晋书》:“帝宇量弘厚,造次必于仁恕;容纳谠正,未尝失色于人;明达善谋,能断大事,故得抚宁万国,绥静四方。承魏氏奢侈革弊之后,百姓思古之遗风,乃厉以恭俭,敦以寡欲。有司尝奏御牛青丝纼断,诏以青麻代之。临朝宽裕,法度有恒。高阳许允既为文帝所杀,允子奇为太常丞。帝将有事于太庙,朝议以奇受害之门,不欲接近左右,请出为长史。帝乃追述允夙望,称奇之才,擢为祠部郎,时论称其夷旷。平吴之后,天下乂安,遂怠于政术,耽于游宴,宠爱后党,亲贵当权,旧臣不得专任,彝章紊废,请谒行矣。爰至未年,知惠帝弗克负荷,然恃皇孙聪睿,故无废立之心。复虑非贾后所生,终致危败,遂与腹心共图后事。说者纷然,久而不定,竟用王佑之谋,遣太子母弟秦王柬都督关中,楚王玮、淮南王允并镇守要害,以强帝室。又恐杨氏之逼,复以佑为北军中候,以典禁兵。既而寝疾弥留,至于大渐,佐命元勋,皆已先没,群臣惶惑,计无所从。会帝小差,有诏以汝南王亮辅政,又欲令朝士之有名望年少者数人佐之,杨骏秘而不宣。帝复寻至迷乱,杨后辄为诏以骏辅政,促亮进发。帝寻小间,问汝南王来未,意欲见之,有所付托。左右答言未至,帝遂困笃。中朝之乱,实始于斯矣。”

周昙《晋门晋武帝》:“汉贪金帛鬻公卿,财赡羸军冀国宁。晋武鬻官私室富,是知犹不及桓灵。”

李世民:“武皇承基,诞膺天命,握图御宇,敷化导民,以佚代劳。以治易乱。绝缣绝之贡,去雕琢之饰,制奢俗以变俭约,止浇风而反淳朴。雅好直言,留心采擢,刘毅、裴楷以质直见容,嵇绍、许奇虽仇雠不弃。仁以御物,宽而得众,宏略大度,有帝王之量焉。于是民和俗静,家给人足,聿修武用,思启封疆。决神算于深衷,断雄图于议表。马隆西伐,王濬南征,师不延时,獯虏削迹,兵无血刃,扬越为墟。通上代之不通,服前王之未服。祯祥显应,风教肃清,天人之功成矣,霸王之业大矣。虽登封之礼,让而不为,骄泰之心,因斯而起。见土地之广,谓万弃而无虞;睹天下之安,谓千年而永治。不知处广以思狭,则广可长广;居治而忘危,则治无常治。加之建立非所,委寄失才,志欲就于升平,行先迎于祸乱。是犹将适越者指沙漠以遵途,欲登山者涉舟航而觅路,所趣逾远,所尚转难,南北倍殊,高下相反,求其至也,不亦难乎!况以新集易动之基,而久安难拔之虑,故贾充凶竖,怀奸志以拥权;杨骏豺狼,苞祸心以专辅。及乎宫车晚出,谅闇未周,籓翰变亲以成疏,连兵竞灭其本;栋梁回忠而起伪,拥众各举其威。曾未数年,网纪大乱,海内版荡,宗庙播迁。帝道王猷,反居文身之俗;神州赤县,翻成被发之乡。弃所大以资人,掩其小而自托,为天下笑,其故何哉?良由失慎于前,所以贻患于后。且知子者贤父,知臣者明君;子不肖则家亡,臣不忠则国乱;国乱不可以安也,家亡不可以全也。是以君子防其始,圣人闲其端。而世祖惑荀勖之奸谋,迷王浑之伪策,心屡移于众口,事不定于己图。元海当除而不除,卒令扰乱区夏;惠帝可废而不废,终使倾覆洪基。夫全一人者德之轻,拯天下者功之重,弃一子者忍之小,安社稷者孝之大;况乎资三世而成业,延二孽以丧之,所谓取轻德而舍重功,畏小忍而忘大孝。圣贤之道,岂若斯乎!虽则善始于初,而乖令终于末,所以殷勤史策,不能无慷慨焉。”(《晋书·帝纪三》)

徐惠:“昔秦皇并吞六国,反速危亡之基;晋武奄有三方,翻成覆败之业。岂非矜功恃大,弃德而轻邦;图利忘害,肆情而纵欲?遂使悠悠六合,虽广不救其亡;嗷嗷黎庶,因弊以成其祸。”

刘仁轨:“晋代平吴,史籍具载。内有武帝、张华,外有羊祜、杜预,筹谋策画,经纬谘询。王濬之徒,折冲万里,楼船战舰,已到石头。贾充、王浑之辈,犹欲斩张华以谢天下。武帝报云:‘平吴之计,出自朕意,张华同朕见耳,非其本心。’是非不同,乖乱如此。平吴之后,犹欲苦绳王濬,赖武帝拥护,始得保全。不逢武帝圣明,王濬不存首领。”(《旧唐书·列传第三十四》)

苏辙:“武帝之为人,好善而不择人,苟安而无远虑,虽贤人满朝,而贾充、荀勖之流以为腹心,使吴尚在,相持而不敢肆,虽为贤君可也。吴亡之后,荒于女色,蔽于庸子,疏贤臣,近小人,去武备,崇藩国,所以兆亡国之祸者,不可胜数,此则灭吴之所从致也。”(《栾城集·卷五十》)

司马光:“至于晋武独以天性矫而行之,可谓不世之贤君。”(《资治通鉴·卷第七十九》)

陈普:“宫中掷戟又飞刀,谢玖兢兢命若毛。岂是君王轻社稷,天教炽业谢芳髦。”

邓林:“秋风铜爵曲池平,吴主宫娃满掖庭。凭仗皇孙聪慧早,不知祸在夕阳亭。”

孙承恩:“帝资弘裕,明达好谋。纂述先志,混一九州。礼优三恪,忠厚之道。贻谋弗臧,识者所少。”(《文简集·卷三十八》)

李慈铭:“晋武帝纯孝性成,三代以下不多得。”(《越缦堂读书记》)

蔡东藩:“彼如马隆之得平树机能,未始非晋初名将,观晋武之倚重两人,乃知开国之主,必有所长,不得以外此瑕疵,遽掩其知人之明也。”“武帝既知太子不聪,复恨贾妃之奇悍,废之锢之,何必多疑,乃被欺于狡吏而不之知,牵情于皇孙而不之断,受朦于宫帟而不之觉,卒至一误再误,身死而天下乱,名为开国,实是覆宗,王之不明,宁足福哉?”

《满江红》:“承祖余威。曹魏末、受禅称帝。扫吴越,多方齐下,摧枯之势。 吴主凶残民向背,晋公仁厚人归帜。施仁政,创社稷荣繁。功难讳。 世袭制,官僚继。田与土,官伸挤。痛卖官鬻爵,竟奢豪气。 罢却州兵难稳序,助长王势终遭戏。叹武帝,开国未几年。皇权弃。”

\subsection{泰始}

\begin{longtable}{|>{\centering\scriptsize}m{2em}|>{\centering\scriptsize}m{1.3em}|>{\centering}m{8.8em}|}
  % \caption{秦王政}\
  \toprule
  \SimHei \normalsize 年数 & \SimHei \scriptsize 公元 & \SimHei 大事件 \tabularnewline
  % \midrule
  \endfirsthead
  \toprule
  \SimHei \normalsize 年数 & \SimHei \scriptsize 公元 & \SimHei 大事件 \tabularnewline
  \midrule
  \endhead
  \midrule
  元年 & 265 & \tabularnewline\hline
  二年 & 266 & \tabularnewline\hline
  三年 & 267 & \tabularnewline\hline
  四年 & 268 & \tabularnewline\hline
  五年 & 269 & \tabularnewline\hline
  六年 & 270 & \tabularnewline\hline
  七年 & 271 & \tabularnewline\hline
  八年 & 272 & \tabularnewline\hline
  九年 & 273 & \tabularnewline\hline
  十年 & 274 & \tabularnewline
  \bottomrule
\end{longtable}

\subsection{咸宁}


\begin{longtable}{|>{\centering\scriptsize}m{2em}|>{\centering\scriptsize}m{1.3em}|>{\centering}m{8.8em}|}
  % \caption{秦王政}\
  \toprule
  \SimHei \normalsize 年数 & \SimHei \scriptsize 公元 & \SimHei 大事件 \tabularnewline
  % \midrule
  \endfirsthead
  \toprule
  \SimHei \normalsize 年数 & \SimHei \scriptsize 公元 & \SimHei 大事件 \tabularnewline
  \midrule
  \endhead
  \midrule
  元年 & 275 & \tabularnewline\hline
  二年 & 276 & \tabularnewline\hline
  三年 & 277 & \tabularnewline\hline
  四年 & 278 & \tabularnewline\hline
  五年 & 279 & \tabularnewline\hline
  六年 & 280 & \tabularnewline
  \bottomrule
\end{longtable}

\subsection{太康}

\begin{longtable}{|>{\centering\scriptsize}m{2em}|>{\centering\scriptsize}m{1.3em}|>{\centering}m{8.8em}|}
  % \caption{秦王政}\
  \toprule
  \SimHei \normalsize 年数 & \SimHei \scriptsize 公元 & \SimHei 大事件 \tabularnewline
  % \midrule
  \endfirsthead
  \toprule
  \SimHei \normalsize 年数 & \SimHei \scriptsize 公元 & \SimHei 大事件 \tabularnewline
  \midrule
  \endhead
  \midrule
  元年 & 280 & \tabularnewline\hline
  二年 & 281 & \tabularnewline\hline
  三年 & 282 & \tabularnewline\hline
  四年 & 283 & \tabularnewline\hline
  五年 & 284 & \tabularnewline\hline
  六年 & 285 & \tabularnewline\hline
  七年 & 286 & \tabularnewline\hline
  八年 & 287 & \tabularnewline\hline
  九年 & 288 & \tabularnewline\hline
  十年 & 289 & \tabularnewline
  \bottomrule
\end{longtable}

\subsection{太熙}

\begin{longtable}{|>{\centering\scriptsize}m{2em}|>{\centering\scriptsize}m{1.3em}|>{\centering}m{8.8em}|}
  % \caption{秦王政}\
  \toprule
  \SimHei \normalsize 年数 & \SimHei \scriptsize 公元 & \SimHei 大事件 \tabularnewline
  % \midrule
  \endfirsthead
  \toprule
  \SimHei \normalsize 年数 & \SimHei \scriptsize 公元 & \SimHei 大事件 \tabularnewline
  \midrule
  \endhead
  \midrule
  元年 & 290 & \tabularnewline
  \bottomrule
\end{longtable}


%%% Local Variables:
%%% mode: latex
%%% TeX-engine: xetex
%%% TeX-master: "../Main"
%%% End:
