%% -*- coding: utf-8 -*-
%% Time-stamp: <Chen Wang: 2019-12-18 13:21:59>

\section{愍帝\tiny(313-316)}

\subsection{生平}

晉愍帝司馬鄴[1](300年-318年2月7日),字彥旗,西晉的第五任也是最後一任皇帝。司馬鄴為吳王司馬晏之子,晉武帝之孫,外祖父是荀勖。

司馬鄴首先過繼於秦王司馬柬而被封為秦王。308年被封為散騎常侍與撫軍將軍。晉懷帝於洛陽被俘之後司馬鄴逃亡許昌,後在雍州刺史賈疋的護送下逃入長安,之後被封為皇太子。313年晉懷帝於平陽遇害之後,司馬鄴於長安即帝位,改元建興,是为晉愍帝。

晉愍帝即位時,西晉已經沒有可以作戰的能力,而且長安也沒有可與前趙作戰的物資。316年8月劉曜發兵攻打長安,並切斷長安的糧運。晉愍帝在食斷糧絕的情況之下於建兴四年十一月十一日(316年12月11日)投降前趙,之後被送往平陽,封為懷平侯,並且承受為狩獵隊伍前導以及为宴會洗杯子等雜役的屈辱。建兴五年十二月二十日(318年2月7日)被殺。

\subsection{建兴}

\begin{longtable}{|>{\centering\scriptsize}m{2em}|>{\centering\scriptsize}m{1.3em}|>{\centering}m{8.8em}|}
  % \caption{秦王政}\
  \toprule
  \SimHei \normalsize 年数 & \SimHei \scriptsize 公元 & \SimHei 大事件 \tabularnewline
  % \midrule
  \endfirsthead
  \toprule
  \SimHei \normalsize 年数 & \SimHei \scriptsize 公元 & \SimHei 大事件 \tabularnewline
  \midrule
  \endhead
  \midrule
  元年 & 313 & \tabularnewline\hline
  二年 & 314 & \tabularnewline\hline
  三年 & 315 & \tabularnewline\hline
  四年 & 316 & \tabularnewline\hline
  五年 & 317 & \tabularnewline
  \bottomrule
\end{longtable}


%%% Local Variables:
%%% mode: latex
%%% TeX-engine: xetex
%%% TeX-master: "../Main"
%%% End:
