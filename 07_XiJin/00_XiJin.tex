%% -*- coding: utf-8 -*-
%% Time-stamp: <Chen Wang: 2021-11-01 11:40:10>

\chapter{西晋\tiny(265-316)}

\section{简介}

西晋(266年2月4日-316年12月11日),是中国歷史上魏晋南北朝时期的一個大一統的时期,乃於265年由晋武帝司马炎取代曹魏政权而建立。晉武帝靠父祖餘蔭和世族支持而得位。國號為「晉」,定都洛陽,後世稱「西晉」,為晉朝的一部分。這大一統時期為時僅51年,倘由滅吳到五胡十六國開始計,則僅24年。西晉先定都洛陽,後以長安為陪都。實行两京制。

西晉的開國君主司馬炎出身於一個名為河内司馬氏的世族,祖父司馬懿歷任三國時曹魏的大将军、太尉、太傅,其伯父司马师官至大将军、父亲司馬昭官至相国,都是曹魏時權傾一時的權臣,权势超過君主。而在西晉建立前,曹魏已先於霸府司馬氏控制下在263年滅了蜀漢。265年司馬炎篡魏自立改國號「晉」。西晉代魏後,期間發生西陵之戰。西晋於279年發動滅吳之戰,280年滅孫吳,結束了三國鼎立的分裂局面,重新統一。

晉朝本身承接了東漢晚期至曹魏期間的割據局面,地方上世族影響力遠超帝王。如司馬氏一家,是形成霸府政治的世族權臣,在高平陵之變後,武力控制曹魏朝廷,並篡魏自立。所以司馬炎在篡魏得手後,為免其他世族、權臣效法,便分封了各宗室成員為王,在地方上作為維護皇室的力量。同時又頒布「佔田令」,限制世族擁有田地的面積和數量。

西晉乃魏晉南北朝中唯一处于統一的时期。由於魏晉以來世族在地方上的影响力不断扩大,地位遠超帝王(如司馬氏篡曹魏正是),長期以來都令中國處於分裂局面。晉武帝時憑藉其威望,又先後分封宗室郡國並都督諸州和實行佔田制、蔭客制,稍微限制世族的無限擴張。當晉武帝死后,发生八王之亂,失去了維繫統一的重心,又再一次分裂。

同時西晉另一特色是大量游牧部落內遷。东汉以来,大量游牧民族因各种方式被遷入,到西晉時關中和凉州一帶的外族已佔當地人口一半。這些外族本身被世族收作奴婢(五胡十六國時君主之一的石勒為例子)。由於遷入人口數目相當多,與關中一帶晋人相差不遠,形成割據勢力,為西晉亡國和五胡十六国埋下伏笔。

西晉時期以仿铸青銅器高溫燒製的青瓷聞名。西晉墓穴中除了青瓷,還發現有墓穴模型、銅鏡等。

太熙元年(290年),晋武帝病卒,太子衷嗣位,是为晋惠帝。惠帝甚愚,无法担当治国重任。之前很多大臣鉴于太子“不慧”希望武帝传位于其弟、素有贤能之名的齐王攸。武帝也一度考虑废黜太子,但在皇后和一些宠臣的劝阻下改变了主意,并勒令齐王攸离京前往封国,攸发愤病卒。惠帝终于即位。

动乱的前期表现为宫廷政变。晋武帝临终,命其岳父杨骏辅政。惠帝皇后贾南风夙有干政野心,与宗室楚王司马玮合谋,于元康元年(291年)发动政变杀杨骏及其家属亲党,以辈份较高的宗室汝南王亮辅政。不久,贾后唆使楚王玮杀亮,然后又以专杀之罪杀玮,这样大权就落到了贾后手中。此后数年,尽管地方上连续出现流民及内迁诸民族的暴动,朝廷尚相对稳定。元康九年(299年),贾后废黜惠帝后宫所生的太子司马遹,并于次年将他杀害,此举使西晋统治集团的内部冲突大为激化。统领禁军的赵王司马伦发动政变,杀死贾后,随后又废黜惠帝,自即帝位。赵王伦的篡位引起了宗室诸王的普遍反对,政变开始演变为内战。在12外任都督的齐王冏(时镇许昌)、成都王颖(时镇邺,今河北临漳西南)、河间王颙(时镇关中)起兵讨伐赵王伦,拥惠帝复位,随后三王又互相厮杀,长沙王乂、东海王越也卷入了战争。

诸王各引效忠于自己的地方官乃至内迁的民族参战使北方社会陷入严重的动荡和混乱。自惠帝即位至此卷入政变和内战的主要有汝南、楚、赵、齐、成都、河间、长沙和东海八位宗王,故史称“八王之乱”。至光熙元年(306年),前七王皆已败死,东海王最终控制了朝政,毒死惠帝,立怀帝。八王之乱遂结束。

西晋晋怀帝、晋愍帝时期漢族北部地区大规模战争不断,内徙的周边外族相继建立君主制政权,强大起来威胁到西晋政权,并最终酿成永嘉之乱。晋建武年间,中原汉族臣民南渡,史称衣冠南渡。晋渡江后,在建康(今南京)定都,史称东晋。

\section{宣帝司馬懿生平}

司馬懿(179年-251年9月7日),字仲達,河內郡溫縣(今河南省焦作市溫縣)人,三國時期魏國權臣、政治家、軍事家。曾抵禦蜀漢丞相諸葛亮的北伐軍,堅守疆土。歷經曹操、曹丕、曹叡、曹芳四代君主,晚年發動高平陵之變,奪取曹魏的政權。

嘉平三年(251年),魏朝諡舞陽宣文侯;次子司馬昭稱晉王後,追尊谥為晉宣王;孫司馬炎稱帝後,追尊為高祖宣皇帝,故也稱晉高祖、晉宣帝。

司馬懿家族數代為官,高祖父司馬钧为汉安帝时的征西将军,曾祖父司馬量为豫章太守,祖父司馬儁为颍川太守,父亲司馬防为京兆尹。司馬防有八子,因字中都有一个“达”字,当时号称司馬八达。司馬懿是司馬防的次子,出生于汉灵帝光和二年(179年)。司馬懿少年时期就胸怀谋略,當時正是东汉末年乱世,司馬懿「常慨然有忧天下心」。素以知人善任著称的南阳太守杨俊曾见过少年司馬懿,他说司馬懿绝非寻常之子。尚书崔琰与司馬懿的兄长司馬朗交好,曾对司馬朗说:「你弟弟聪明懂事,做事果断,英姿不凡,不是你所能比得上的。」

建安六年(201年),郡中推举他为计掾。时曹操正任司空,听到他的名声后,派人召他到府中任职。司馬懿见汉朝国运已微,不想在曹操手下,便借口自己有风痹病,身体不能起居。曹操不信,派人夜间去刺探消息,司馬懿躺在那里,一动不动,像真染上风痹一般。

建安十三年(208年),曹操为丞相以后,使用强制手段辟司馬懿为文学掾。曹操对使者说,「若复盘桓,便收之」。司馬懿畏惧,只得就职。曹操让他与公子往来游处,历任黄门侍郎、议郎、丞相东曹属、丞相主簿等职。

《魏略》记载召辟过程有所不同。说是司馬懿好学,曹洪自以为才疏,想让司馬懿去帮助他,司馬懿耻于和曹洪来往,借口病到拄拐杖的地步无法去见曹洪。曹洪记恨司馬懿,去跟曹操打小报告,曹操征召司馬懿,司馬懿立刻扔了拐杖去见曹操为其效命。

《晉書》記載曹操逐渐察觉司馬懿“有雄豪志”,又发现他有“狼顧之相”,心里很忌讳。因此对曹丕说,司馬懿不是甘为臣下的人,必会干预我们的家族之事。但因曹丕和司馬懿关系很好,而得以无事。

司馬懿先後擔任黄门侍郎、议郎、丞相东曹属和丞相主簿,并跟隨曹操征討張魯。司馬懿見劉備剛剛奪取益州,腳跟未穩,勸曹操接著攻打劉備,曹操不聽。曹操班師后司馬懿跟隨曹操參加第三次濡須口之戰。

建安二十四年(219年),司馬懿任丞相军司馬,指出荆州刺史胡修粗暴、南乡太守傅方骄奢,都不应派任驻守边防,曹操未予重视;七月,关羽攻荆襄,围魏将曹仁,水淹于禁七军;胡、傅二人果然乘机降劉。

因汉献帝在许都,距樊城很近,曹操感到威胁,为避关羽锋芒,一度准备迁都河北。蒋济、司馬懿等人当时劝阻:“于禁是被洪水所淹,不是战败失守,所以国家大计并没受损,现在迁都既示弱于敌,又使人心不安;刘备、孙权外亲内疏,现在关羽得意,孙权必定更不高兴,把这事告之孙权,坐山观虎斗,则樊城之围自解”。曹操听从他的计策,孙权果然派吕蒙袭取江陵,用假冒商船之計連破關羽設下的烽火台,攻佔眾多城池,关羽只好回援,魏軍趁勢追殺,關羽損失大量兵馬,最終於麥城附近中東吳埋伏而敗亡。曹孫聯盟勝利,樊城之危解除。

徐晃击退关羽后,曹操嫌恶荆州及附近百姓,想把他们都迁走。司馬懿认为:“荆楚轻脱,易动难安。关羽新破,诸为恶者藏窜观望。今徙其善者,既伤其意,将令去者不敢复还。”曹操从之,没有移民。藏窜逃亡者果然都复出归化。

曹丕继魏王位后,封河津亭侯,转丞相长史。当时孙权率兵向西,朝议以樊城、襄阳无谷,不可以御,请召当时镇襄阳的曹仁回宛。司馬懿说:“孙权新破关羽,这是想和魏结交的时候,必不敢为患。襄阳是水陆之冲,御寇要害,不可弃。”不被听从。于是曹仁焚弃樊城、襄阳,孙权果然没有来犯,曹丕悔之。曹魏代汉后,以司馬懿为尚书。不久,转督军、御史中丞,封安国乡侯。黄初二年(221年),迁侍中、尚书右仆射。

黃初七年(226年),曹丕逝世。孙权得知后,于八月出兵攻魏。命吴左将军诸葛瑾部兵分两路进攻襄阳,亲自率军进攻江夏郡。孙权一路为魏军所败,撤兵而走。司馬懿击败诸葛瑾,并斩杀吴将张霸,斩首千余级。十二月,升任骠骑将军。

司馬懿的軍事才略還表現在孟達叛魏事件上。孟達原為蜀漢降將,曹丕命他守新城(今湖北房縣)。曹丕死後,孟達欲叛魏歸蜀。司馬懿偵知消息後,一方面寫信麻痹孟達,一方面遣軍進討,八日內行軍一千二百里。孟達給諸葛亮信中認為司馬氏率軍來討,至少需要一個月時間,所以當司馬懿提前二十餘日趕來時,完全打亂了他的叛亂計劃。司馬懿前後總共只用了十六日,即破上庸,斬殺孟達。

魏太和四年(230年),曹魏派司馬懿由西城,張郃由子午道,曹真由斜谷,三路進攻漢中,諸葛亮率軍西征,命李嚴率軍2萬赴漢中。合攻漢中,但因天雨而罷兵。

魏太和五年(231年)春,諸葛亮第四次北伐。魏明帝让司馬懿驻扎长安。诸葛亮先是圍祁山,又招鮮卑軻比能響應。司馬懿命令費曜、戴淩等留四千兵守上邽,其餘兵力皆往祁山救援,司馬懿親率大軍前往隃麋。就在司馬懿以為諸葛亮主攻祁山時,不料諸葛亮卻反其道而行,得知司馬懿大軍要來,分兵留攻祁山,自己卻率主力奇襲上邽,迎擊司馬懿。郭淮、費曜等出击诸葛亮,却被诸葛亮大败。司馬懿與諸葛亮遇於上邽之東,諸葛亮不斷後撤將司馬懿引誘至鹵城祁山。司馬懿旣至,登山掘營,不肯戰。軍將領數次請戰,司馬懿都不准,被譏笑:「公畏蜀如虎,奈天下笑何!」,在眾將的一再要求下,五月十八日,司馬懿命襲擊蜀將王平所部,却反遭諸葛亮逆襲,魏军為魏延、高翔、吳班等人所破,退保營寨。

六月,諸葛亮向漢中撤退,司馬懿命張郃率軍追擊,張郃中箭身亡。(《魏略》)

魏青龍二年四月(234年),諸葛亮占據渭河南武功五丈原,想北渡渭河,被大將郭淮阻攔而失敗。諸葛亮只能在渭河南與魏國大都督司馬懿對峙。由于戰術上的敗勢,司馬懿據險堅守,對峙百餘日后,蜀軍糧盡而自退。諸葛亮曾多次向司馬懿挑戰。司馬懿寫奏章向魏明帝請戰,明帝下令不許出戰,並派衛尉辛毗持符節制止魏軍。姜維對諸葛亮說:「辛佐治仗節而到,賊不復出矣。」諸葛亮說:「彼本無戰情,所以固請戰者,以示武於其眾耳。將在軍,君命有所不受,苟能制吾,豈千里而請戰邪!」諸葛亮再派人至司馬懿軍中,司馬懿只向漢使詢問諸葛亮寢食和管理政事的繁雜,不過問軍旅之事。諸葛亮使者對司馬懿說:「諸葛公夙興夜寐,罰二十以上,皆親攬焉;所啖食不至三升。」司馬懿說:「亮將死矣。」其後,兩軍對峙一百多天。諸葛亮的北伐都以失敗告終。八月,諸葛亮終於病倒。劉禪派尚書僕射李福到前線去探望,回到成都時,諸葛亮在五丈原前線病逝,享年五十四岁。當楊儀等人整軍而還時,司馬懿得知蜀漢撤退,親自率兵來追;姜維見司馬懿追來,反旗鳴鼓,司馬懿不知諸葛亮已死,下令停止追擊,全軍返回。於是楊儀結陣而去,從容退進斜谷,百姓流傳一諺語:「死諸葛驚走生仲達。」司馬懿聽到後,說:「吾能料生,不便料死也。」(我能預料他活著的時候想做什麼,不能預料他死後想怎麼做。)司馬懿到諸葛亮駐軍營壘處所察看,感嘆說:「真是天下的奇才啊!」。追到赤岸,沒有追上蜀軍而退軍。此後,司馬懿成為曹魏政權的中流砥柱,地位日益提高。次年正月戊子日(235年2月13日)升遷為太尉。

青龍五年(237年),原魏遼東太守公孫淵自立為燕王,置列百官,對抗朝廷。景初二年(238年)正月,魏明帝曹叡遣59歲的司馬懿率軍4萬,前往征討,途中司馬懿路过故乡温县,会见父老乡亲,感慨万千,做诗一首:「天地开辟,日月重光。遭遇际会,毕力遐方。将扫群秽,还过故乡。肃清万里,总齐八荒。告成归老,待罪舞阳。」六月進至遼水。

公孫淵急令大將軍卑衍、楊祚等人率步騎數萬,依遼水圍塹20餘里,堅壁高壘,阻擊魏軍。司馬懿採用聲東擊西之計,佯攻圍塹,而以主力隱蔽渡過遼水,直取公孫淵根據地襄平,迫使敵軍放棄圍塹回軍救援襄平。司馬懿督軍回首山,迎戰敵援軍,三戰皆捷,遂乘勝進圍襄平。

当初公孙渊闻魏军来攻,曾求救于孙权,孙权也出兵为其声援,并写信给公孙渊说:“司馬懿善用兵所向无前,深为弟忧也。”

適逢連降大雨,遼水暴漲,平地數尺,魏軍恐懼,諸將思欲遷營。司馬懿下令有敢言遷營者斬,都督令史張靜違令被斬首,軍心始安。公孫淵軍乘雨出城,魏將領請求出擊,司馬懿不予採納。一月有餘,雨停,水漸退去。魏軍完成對襄平的包圍,起土山、挖地道、造樓車、鉤梯等攻城器具,晝夜強攻。守軍糧食將盡,軍心動搖,楊祚等先降。

八月,襄平城破,公孫淵率數百騎兵突圍,被殺于梁水。司馬懿把襄平城内平民男子年十五以上七千餘人及公孙渊政权官员将军等二千餘人全部杀死,并将尸体堆积起来,号称“京觀”。遼東四郡為魏所據。

曹操封魏王後,以司馬懿為太子中庶子以辅佐曹丕。曹丕臨終時,令司馬懿與曹真等為輔政大臣,輔佐魏明帝曹叡。明帝時,司馬懿屢遷撫軍大將軍、大將軍、太尉等要職。明帝崩,托孤幼帝曹芳以司馬懿和曹爽。

曹芳繼位後,曹爽最初因司馬懿德高望重,像对父亲一样对他,不敢自专。但不久曹爽听亲信丁谧之谋,排挤司馬懿,明升暗降,迁他為無實權的最高虚衔太傅,自己专擅朝政。正始十年(249年),司馬懿趁曹爽陪曹芳離洛陽至高平陵掃墓,發動政變並控制京師,將曹爽族滅。自此曹魏軍權政權落入司馬氏手中,史稱高平陵之變,又稱正始之變。司馬懿時年70歲。

事后,司馬懿被册命为丞相,进爵安平郡公,增食邑万户,群臣奏事不得称名,但他上书十余次辞让,故其官爵仍为太傅、舞阳侯。朝廷加他九锡,仍然辞而不受。雖然表面謙遜,司馬懿的架子十分大。奪權之後司馬懿藉口生病連皇帝曹芳都不去朝見,遇到國事,皇帝曹芳反而要去司馬懿宅邸親自詢問司馬懿。

嘉平二年(250年),皇帝曹芳下诏在曹魏首都洛阳设立司馬懿的家庙。

嘉平三年(251年)太尉王凌與外甥兗州刺史令狐愚密謀廢立,迎曹操子曹彪登帝位。陰謀泄漏,司馬懿夷王凌、令狐愚三族,並賜死57歲的曹彪,為史載壽春三叛之一。之後司馬懿把曹魏宗室遷至鄴監視,不准他們結交他人。

嘉平三年(251年),司馬懿病逝于洛阳,享年72岁。承其遗愿,辞让郡公和殊礼,不树不坟,不设明器,葬于首阳山。谥文(舞阳文侯),后改宣文(舞阳宣文侯)。264年,子司馬昭获封晋王,旋即追尊司馬懿为晋王,谥号宣王。265年,孫司馬炎稱帝,建立西晉,追尊為皇帝,庙号高祖,谥号宣皇帝。

司馬懿在上邽兴屯田,京兆、天水、南安兴冶铁,穿成国渠,筑临晋坡使雍凉足食,并有余力供给关中不足。後来又大兴屯田於淮北,穿广槽渠。

司馬懿曾向曹操建議军事屯田制,而曹操採納後不久便病逝了(曹操死前已經有屯田制)。管理军屯的主要官吏——度支中郎将、度支校尉、度支都尉等官员,都是在曹丕称帝后的黄初年间(220—226年)设置的。所以,事實上軍屯的推廣在曹丕時期。曹魏軍屯其主要基地設置在和孫吳、蜀漢對立的地帶(淮河南北以及上邽、長安、槐里、陳倉等地),而這裡基地的開創都經過司馬懿的督工:

太和四年,司馬懿上表建上邽军屯。主持具体事宜的是度支尚书、司馬懿的三弟司馬孚。

青龍元年(233年),司馬懿興修水利,而「開成國渠,自陳倉至槐裡築臨晉陂,引汧洛溉舄鹵之地三千餘頃」使得‘国以充实’。

青龍三年(236年),關東饑荒,司馬懿調5百萬斛粟運往洛陽,足見關中存有大量糧食。

曹操时魏国就曾“开募屯田于淮南”(《三国志·魏书·仓慈传》),但仅是民屯。正始二年(241),司馬懿主持对吴作战时,始与邓艾筹划在淮南淮北创建军屯。。正始四年(243),司馬懿又‘在于积谷,乃大兴屯田,广开淮阳、百尺二渠,又修诸陂于颍之南北,万余顷。自是淮北仓庾相望,寿阳至于京师,农官屯兵连属焉。’‘因欲广田积谷,为兼并之计,乃使邓艾行陈、项以东,至寿春地’(《晋书·食货志》),使得‘遂北临淮水,自钟离而南横石以西,尽沘水四百余里,五里置一营,营六十人,且佃且守。兼修广淮阳、百尺二渠,上引河流,下通淮颍,大治诸陂于颍南、颍北,穿渠三百余里,溉田二万顷,淮南、淮北皆相连接。自寿春到京师,农官兵田,鸡犬之声,阡陌相属’。

在识拔人才上,司馬懿从寒门中提拔邓艾、王基、州泰、贾越等人,如虞预所说,经略之才可谓远矣。

景初二年(238年),司馬懿远征辽东。时曹魏大修宫室,加上军用物资,以至百姓饥弊。出征前,司馬懿劝明帝说:“昔周公营洛邑,萧何造未央,今宫室未备,臣之责也。然自河以北,百姓困穷,外内有役,势不并兴,宜假绝内务,以救时急。”司馬懿劝谏曹叡不要大修宫殿劳民伤财,以保国之根本。曹叡死後他奏请罢修宫室,雕玩物之人力,节用务农,使天下欣赖。

《晉書·宣帝紀》記載,司馬懿心里猜忌而表面寬和,內心多疑但能靈活應對。曹操察覺司馬懿有雄心豪志,且聽聞司馬懿有狼顧的模樣。曹操打算驗明,所以先叫司馬懿在前頭走,再故意喊司馬懿,只見司馬懿身體不動,只有頭轉向後看。曹操又曾經發了一個噩夢,三匹馬同在一個槽里進食,醒來後十分不快。曹操感到不好,召曹丕說道:“司馬懿不會甘做人臣,將來必干預你的朝政家事。”這是三馬同槽的典故,此槽不是馬槽,是暗指着司馬氏吃掉曹氏,而且按照后来的史实走向来看,“三马”还可以解读为司馬懿、司馬师和司馬昭父子三人,他们共同为之后的司馬炎代魏创造了有利条件。但曹丕和司馬懿关系很好,总是維护他而得以无事。而司馬懿更加勤于职守,废寝忘食,甚至喂马之类的小事都亲力亲为,也终于使曹操安心。

司馬懿年少时和三国时的著名隐士胡昭关系很好,同郡的周生要加害司馬懿,胡昭听说后涉险赶来搭救,“邀生于崤、渑之间,止生,生不肯。昭泣与结诚,生感其义,乃止”。

《三国志》卷16《仓慈传》注引《魏略》载:青龙中,司馬宣王在长安立军市,而军中吏士多侵侮县民,裴以白宣王,宣王乃发怒召军市候,便于斐前杖一百……宣王遂严持吏士,自是之后,军营、郡县各得其分。

南陽太守同郡楊俊:「此非常之人也。」认为有非常之器(《三国志·杨俊传》、《晋書·宣帝紀第一》)

崔琰對司馬懿兄長司馬朗說:「君弟聰亮明允,剛斷英特,非子所及也。」(《晋書·宣帝紀第一》)

曹操:「有雄豪志,聞有狼顾相。」因而對曹丕說:「司馬懿非人臣也,必預汝家事。」(《晋書·宣帝紀第一》)

曹植曾形容司馬懿:「魁杰雄特,秉心平直。威严允惮,风行草靡。在朝则匡赞时俗,百僚仪一;临事则戎昭果毅,折冲厌难者,司馬骠骑也。」(《全三国文·卷十八·魏十八》)

曹叡:「西方有事,非君(司馬懿)莫可付者。」(《晋書·宣帝紀第一》)“司馬懿临危制变,擒渊可计日待也。”“吾得司馬懿二人,复何忧哉!”(《晋书·卷三十七·列传第七》)

曹芳:「太尉体道正直,尽忠三世,南擒孟达,西破蜀虏,东灭公孙渊,功盖海内。」(《《三國志·魏書四·三少帝紀第四》》)

夏侯霸:“彼自吻家,非人臣也”(《太平御览·卷四百四十二 ◎人事部八十三○知人上》引 孙盛《魏氏春秋》)

吴质:“骠骑将军司馬懿,忠智至公,社稷之臣也。”

陈矫:“朝廷之望;社稷,未知也。”(《三国志·卷二十二·魏书二十二·桓二陈徐卫卢传第二十二》裴松之注引《世语》)

孙权:“司馬公善用兵,变化若神,所向无前。”

毌丘俭、文钦起兵反对司馬师时,依然对去世的司馬懿评价很高:“故相国懿,匡辅魏室,历事忠贞,故烈祖明皇帝授以寄讬之任。懿戮力尽节,以宁华夏。又以齐王聪明,无有秽德,乃心勤尽忠以辅上,天下赖之。懿欲讨灭二虏以安宇内,始分军粮,克时同举,未成而薨。齐王以懿有辅己大功,故遂使师承统懿业,委以大事。而师以盛年在职,无疾讬病,坐拥强兵,无有臣礼,朝臣非之,义士讥之,天下所闻,其罪一也。懿造计取贼,多舂军粮,克期有日。师为大臣,当除国难,又为人子,当卒父业。哀声未绝而便罢息,为臣不忠,为子不孝,其罪二也。”

《晉書》評司馬懿平定公孫淵時殺戮太甚,誅殺曹爽時甚至連女子都殺害,然後篡奪魏國:「及平公孫文懿,大行殺戮。誅曹爽之際,支黨皆夷及三族,男女無少長,姑姊妹女子之適人者皆殺之,既而竟遷魏鼎云。」

晋明帝時,王導侍坐。明帝問王導晉朝何以得天下,王導乃陳述司馬懿創業之始,及司馬昭末年弑高貴鄉公之事。明帝以面覆在牀上說:「若如公言,晉祚復安得長遠!」迹司馬懿之猜忍,蓋有符於狼顧。(《晋書·宣帝紀第一》)

虞预:“服膺文艺,以儒素立德,而雅有雄霸之量。值魏氏短祚,内外多难,谋而鲜过,举必独克,知人拔善,显用仄陋。王基、邓艾、周秦、贾越之徒,皆起自寒门,而著绩于朝,经略之才,可谓远矣。”

张悌:“诸葛、司馬二相,遭值际会,讬身明主,或收功于蜀汉,或册名于伊、洛。丕、备既没,后嗣继统,各受保阿之任,辅翼幼主,不负然诺之诚,亦一国之宗臣,霸王之贤佐也。”

关于司馬氏代魏而仍能天下一统的原因,当时東吴屯騎校尉(官職)张悌的话,也许是很好的解释:

永安六年,魏伐蜀,吴人问悌曰:“司馬氏得政以来,大难屡作,智力虽丰,而百姓未服也。今又竭其资力,远征巴蜀,兵劳民疲而不知恤,败於不暇,何以能济?昔夫差伐齐,非不克胜,所以危亡,不忧其本也,况彼之争地乎!”悌曰:“不然。曹操虽功盖中夏,威震四海,崇诈杖术,征伐无已,民畏其威,而不怀其德也。丕、叡承之,系以惨虐,内兴宫室,外惧雄豪,东西驰驱,无岁获安,彼之失民,为日久矣。司馬懿父子,自握其柄,累有大功,除其烦苛而布其平惠,为之谋主而救其疾,民心归之,亦已久矣。故淮南三叛而腹心不扰,曹髦之死,四方不动,摧坚敌如折枯,荡异同如反掌,任贤使能,各尽其心,非智勇兼人,孰能如之?其威武张矣,本根固矣,群情服矣,奸计立矣。今蜀阉宦专朝,国无政令,而玩戎黩武,民劳卒弊,竞於外利,不脩守备。彼强弱不同,智算亦胜,因危而伐,殆其克乎!若其不克,不过无功,终无退北之忧,覆军之虑也,何为不可哉?昔楚剑利而秦昭惧,孟明用而晋人忧,彼之得志,故我之大患也。”吴人笑其言,而蜀果降于魏。

石勒:「大丈夫行事,當磊磊落落,如日月皎然,終不能如曹孟德、司馬仲達父子,欺他孤兒寡婦,狐媚以取天下也。」(《晉書·載記第五·石勒下》)

李世民曾为《晋书·宣帝纪》作史论,原文如下:「夫天地之大,黎元为本。邦国之贵,元首为先。治乱无常,兴亡有运。是故五帝之上,居万乘以为忧;三王已来,处其忧而为乐。竞智力,争利害,大小相吞,强弱相袭。逮乎魏室,三方鼎峙,干戈不息,氛雾交飞。宣皇以天挺之姿,应期佐命,文以缵治,武以棱威。用人如在己,求贤若不及;情深阻而莫测,性宽绰而能容,和光同尘,与时舒卷,戢鳞潜翼,思属风云。饰忠于已诈之心,延安于将危之命。观其雄略内断,英猷外决,殄公孙于百日,擒孟达于盈旬,自以兵动若神,谋无再计矣。既而拥众西举,与诸葛相持。抑其甲兵,本无斗志,遗其巾帼,方发愤心。杖节当门,雄图顿屈,请战千里,诈欲示威。且秦蜀之人,勇懦非敌,夷险之路,劳逸不同,以此争功,其利可见。而返闭军固垒,莫敢争锋,生怯实而未前,死疑虚而犹遁,良将之道,失在斯乎!文帝之世,辅翼权重,许昌同萧何之委,崇华甚霍光之寄。当谓竭诚尽节,伊傅可齐。及明帝将终,栋梁是属,受遗二主,佐命三朝,既承忍死之托,曾无殉生之报。天子在外,内起甲兵,陵土未乾,遽相诛戮,贞臣之体,宁若此乎!尽善之方,以斯为惑。夫征讨之策,岂东智而西愚?辅佐之心,何前忠而后乱?故晋明掩面,耻欺伪以成功;石勒肆言,笑奸回以定业。古人有云:“积善三年,知之者少,为恶一日,闻于天下。”可不谓然乎!虽自隐过当年,而终见嗤后代。亦犹窃钟掩耳,以众人为不闻;锐意盗金,谓市中为莫睹。故知贪于近者则遗远,溺于利者则伤名;若不损己以益人,则当祸人而福己。顺理而举易为力,背时而动难为功。况以未成之晋基,逼有余之魏祚?虽复道格区宇,德被苍生,而天未启时,宝位犹阻,非可以智竞,不可以力争,虽则庆流后昆,而身终于北面矣。」(《晋書·宣帝紀第一》)

房玄龄:“少有奇节,聪明多大略,博学洽闻,伏膺儒教。汉末大乱,常慨然有忧天下心。”“帝内忌而外宽,猜忌多权变。”(《晋書·宣帝紀第一》)

杜牧:“周有齐太公,秦有王翦,两汉有韩信、赵充国、耿恭、虞诩、段颎,魏有司馬懿,吴有周瑜,蜀有诸葛武侯,晋有羊祜、杜公元凯,梁有韦睿,元魏有崔浩,周有韦孝宽,隋有杨素,国朝有李靖、李勣、裴行俭、郭元振。如此人者,当此一时,其所出计画,皆考古校今,奇秘长远,策先定於内,功后成於外。”(《注孙子序》)

司馬光:“司馬懿,少聪达,多大略。”(《资治通鉴·卷六十五》)

而宋代何去非认为司馬懿能忍耐,又有足够的度量。不与诸葛亮正面对战并非胆怯,而是高明的消耗敌军之策:“仲达提秦、雍之劲卒,以不应而老其师者,岂徒然哉!将求全于一胜也。然而,孔明既死,蜀师引还,而仲达不穷追之者,盖不虞孔明之死,其士尚饱,而军未有变,蜀道阻而易伏,疑其伪退以诱我也。向使孔明之不死,而弊于相持,则仲达之志得矣。或者谓仲达之权诡,不足以当孔明之节制,此腐儒守经之谈,不足为晓机者道也。”,评其“制其兵,出奇应变,奄忽若神,无往不殄,虽曹公有所不逮焉”。

黄道周:“司馬魏人,从讨张鲁。备争江陵,请乘蜀土。言虽不从,大志已睹。关羽震樊,魏欲避许。懿请结孙,因而斩羽。孟达虽降,意犹首鼠。八日往擒,尽惊神武。诸葛出祁,以懿御悔。利则急驱,屯则守伍。巾帼相加,亦不妄举。食少事烦,早知其苦。五丈秋风,更辈无补。料死料生,功已足数。文懿反辽,视鱼游釜。计日攻虚,破之若取。后晋帝基,皆懿遗祜。”(《广名将传》)

罗贯中:“开言崇圣典,用武若通神。三国英雄士,四朝经济臣。屯兵驱虎豹,养子得麒麟。诸葛常谈羡,能回天地春!”(《战徐塘吴魏交兵》)

毛宗岗:“今人将曹操、司馬懿并称。及观司馬懿临终之语,而懿之与操则有别矣。操之事,皆懿之子为之,而懿则终其身未敢为操之事也。操之忌先主,是欲除宗室之贤者;懿之谋曹爽,是特杀宗室之不贤者。至于弑主后,害皇嗣,僭皇号,受九锡,但见之于操,而未见之于懿。故君子于懿有恕辞焉。”(《汇评三国志演义》)

蔡东藩:“从前王莽、曹操、司馬懿、刘裕诸奸雄,其险恶犹不若温也。”(《五代史演义》)

毛泽东:“司馬懿是个了不起的人物,历来说他坏,我看有几手比曹操高明。”他认为曹操攻下张鲁以后应该听司馬懿和刘晔的建议进攻四川;评其“出身士族,多谋略,善权变,为魏国重臣”。曾在《三国志·陆逊传》中评注:“此司馬懿敌孔明之智也。”(《毛泽东读书笔记解析》)

柏杨:“就史料记载而言,真实的司馬懿,跟世人印象中的司馬懿不同,诸葛亮先生受托孤后,并没有曹爽般的政敌,李严的窃弄威权,一纸命令便告解决,而曹爽先生确是将司馬懿整个排除,司馬懿对曹爽虽然不满,但一直到244年,曹爽先生攻击蜀汉帝国,大军被雨困在峡谷,司馬懿仍忧虑他会失败,劝告退军,假使他心怀不轨,大可闭口不言,等曹爽覆灭后,由他来收拾残局。司馬懿先生当初最大的目的,不过是反击曹爽,夺官夺权。249年的政变,受到朝野一致爱戴,251年王凌起兵时,司馬懿不但没有任何叛逆迹象,而且声望正值高峰。……就司馬懿本身而言,他所受到的诟骂与诅咒,并不公平。”(《柏杨版资治通鉴》)

司馬懿在世时,在魏国威望相当高,即使文钦在其死后讨伐司馬师时,在檄文中依然对司馬懿有‘故相国懿,匡辅魏室,历事忠贞’等相当多的赞誉。史书上说他做到了使‘天下大悦’‘天下欣赖’。晋朝初年,司馬炎有「天下无穷人」的太康之治,人民对促成三国统一的奠基者司馬还是相当推崇。甚至因司馬炎之孙司馬遹长得像司馬懿就寄予厚望。

司馬懿名声毁败,是从永嘉之祸以後开始的。其不肖子孙自相残杀,带来五胡乱华的浩劫,出身奴隶的石勒恨透司馬家。而东晋在南方,又长期被世家大族把持。晋室南渡,情况与当年蜀汉类似,以蜀汉为正统呼声日高;东晋灭亡後,情况更明显。到隋唐时代,民间说书戏曲日盛。唐太宗主导官修的《晋书》,以封建帝王立场更不鼓励臣下效法。到明代《三国演义》后,火熄上方谷、见木雕魏都督丧胆等的形象也就被演义流传民间。

呂思勉認為,司馬懿之深謀祕計,還有許多不為後人所知。而且他很為暴虐,他之政敵被殺,都是夷及三族,連已出嫁女亦不能得免,所以《晉書》作者房玄齡等人,也說他猜忌殘忍。他一生用盡深刻心計,暴虐手段,全是為一己之地位,絲毫沒有曹操匡扶漢室、平定天下之意。封建時代之道德,是公忠、正直、勇敢,是犠牲一己以利天下,司馬懿卻件件和他相反。魏武帝曹操亡歿後,司馬氏父子繼而得志,忠君愛民之心地,光明磊落之行為,全都看不見,只剩下自私自利之心地,狡詐刻毒之行為。

\section{景帝司馬師生平}

司馬師(208年-255年3月23日),字子元,河内温(今河南省温县)人,三國時期魏國後期權臣,官至大將軍。西晉開國君主晉武帝司馬炎的伯父,司馬懿與張春華的長子,司馬昭的兄長。繼承父權后,先肅清內亂,又击敗東吳的諸葛恪,並逐渐控制魏國政權。以高平陵之变有功而获封長平鄉侯,司马懿故后袭封舞陽侯。正元二年(255年),司马师在平定毌丘俭之乱后病逝,魏国谥舞阳忠武侯。二弟司馬昭受封晉王後,追谥为晉景王;其侄司馬炎登基後,追谥为世宗景皇帝。

司马师出生于建安十三年(208年)。魏景初年間,司馬師拜散騎常侍,累遷中護軍。為選用之法,舉不越功,吏無私焉。

249年正月,司馬師參與高平陵之變,事变前夜,当司马懿将计划告诉司马师与司马昭后,司马昭担心得整晚都睡不着,而司马师却像平常一样安睡。司马师率兵驻守司马门,控制洛阳,推翻當時僭越乱政的宗室曹爽有功,封長平鄉侯,食邑千戶,不久加封衛將軍。

251年八月,司馬懿去世,司馬師成為撫軍大將軍,執掌魏國軍政大權。

嘉平四年正月癸卯日(252年1月30日),司馬師升為大將軍,意欲攻取東吳新建的堤壩。十二月,發兵攻擊東興,并聽取諸葛誕的計策;分軍三路進攻,王昶进攻南郡,毌丘俭进攻武昌,諸葛誕自己的軍隊和胡遵的軍隊合共七萬作主力,并架起浮橋。東興之戰卻被諸葛恪所擊敗,朝臣眾議要把諸葛誕等參戰的武將貶降職位。司馬師把戰敗歸咎於自己,并說到:“我不聽公休(諸葛誕字),以至於此。此我過也,諸將何罪?”司馬昭因為是監軍所以被削弱稱號,其他武將都沒有過大的懲罰,只是將防區對調而已。

253年五月,吳國太傅諸葛恪派遣二十萬大軍包圍合肥新城,朝廷紛紛議論,擔心分兵攻打淮泗,并有守住各個水口的打算。司馬師說道:“諸葛恪剛得到吳國的權力,欲使用一時的權力,合兵於合肥新城,希望再次成功,沒有空餘時間攻擊青州和徐州。況且水路口岸不單只一個,多防的地方則用多兵,少防的地方則不足以抵禦。”諸葛恪把兵力全都集中在合肥新城,就如司馬師所料,并派司馬孚督諸軍二十萬防禦,鎮東將軍毌丘儉與揚州刺史文欽等請戰。司馬師説:“諸葛恪軍隊輕裝深入,投兵於絕境,兵力的鋒芒難以抵禦。況且新城小而堅固,攻擊也未能馬上攻克下來。”命令諸將以高壘來對待,諸葛恪相持數月,攻城兵力力竭,死傷過半。司馬師命文欽督遣精銳部隊會合,要其斷諸葛恪的退路,毌丘儉等將斷後。諸葛恪懼怕而遁逃,文欽上前大敗敵軍,斬首萬餘人。

254年二月,中書令李豐、光祿大夫張緝等人图谋废掉司马师,改立太常夏侯玄为大将军,事情败露,被司马师灭族。司馬師對曹芳有所猜疑,同年九月联合大臣上奏郭太后,列举曹芳年长不亲政、沉迷女色、废弃讲学、弃辱儒士、与优人淫乱作乐,弹打进谏的清商令令狐景、清商丞庞熙,太后丧母时不尽礼等罪名,请依照霍光故事废掉曹芳的帝位,封为齐王,改立东海王曹霖之子高贵乡公曹髦為皇帝。

255年正月,鎮東將軍毌丘儉与揚州刺史文欽听说司马师擅行废立之事,起兵征讨司马师,並且把自己的四個兒子當成人質送到東吳,向孫亮討好,卻並未得到東吳的大力支援。毌丘儉、文欽渡過淮河由壽春向西進發,沒有辦法搗洛陽佔許昌,走到了項縣就停住了。司馬師吩咐監軍王基帶領前鋒部隊紮在“南頓”监视毌丘儉、文欽,派遣諸葛誕帶領豫州的兵,進攻壽春,派遣胡遵帶領青州、徐州的兵,斜出譙縣與今日的商丘之間,斷絕毌丘儉、文欽從項縣回歸壽春的路。

司馬師自己親率主力,屯聚在汝陽。又叫鄧艾帶一萬多名“泰山諸軍”部隊到樂嘉縣,做出不堪一擊的樣子,引誘毌丘儉、文欽出擊。毌丘儉果然就叫文欽來打鄧艾,司馬師就指揮大股騎兵,從後面襲擊文欽,文欽大敗。毌丘儉在項縣城裡聽到消息,慌忙棄城而走。毌丘儉走到慎縣以後,躲在河旁的草叢裡被老百姓射死。文欽一口氣逃往了東吳。毌丘氏與文氏兩家的人,凡是留在魏國的都被司馬師所诛殺。

255年閏正月,文欽之子文鴦帶兵襲營,司馬師驚嚇過度,再加上本來眼睛上就有瘤疾,經常流膿,致使眼睛震出眼眶,病情加重,最後在辛亥日(3月23日)死於許昌,终年47岁。二月,曹髦素服臨弔,諡忠武侯。後來司馬昭受封晉王,追尊司馬師為晉景王。司馬炎稱帝後,尊司馬師為晉景帝,陵曰峻平,廟號世宗。

毌丘儉:“師為大臣,當除國難,又為人子,當卒父業。哀聲未絕而便罷息,為臣不忠,為子不孝。”(《三国志·魏书·王毌丘诸葛邓钟传第二十八》)

何晏:“惟幾也能成天下之務,司馬子元是也。”

文钦:“司马师滔天作逆,废害二主,辛、癸、高、莽,恶不足喻。”(《全三国文》)

张悌:“司马懿父子,自握其柄,累有大功,除其烦苛而布其平惠,为之谋主而救其疾,民心归之,亦已久矣。故淮南三叛而腹心不扰,曹髦之死,四方不动,摧坚敌如折枯,荡异同如反掌,任贤使能,各尽其心,非智勇兼人,孰能如之?其威武张矣,本根固矣,群情服矣,奸计立矣。”(《三国志·卷四十八·吴书三·三嗣主传第三》)

孙盛:“初,夏侯玄、何晏等名盛于时,司马景王亦预焉。”

曹毗:“景皇承运,纂隆洪绪。皇维重抗,天晖再举。蠢矣二寇,扰我扬楚。乃整元戎,以膏齐斧。亹亹神算,赫赫王旅。鲸鲵既平,功冠帝宇。”(《宋书·卷二十·志第十》)

虞世南:“少有名节,见重当时。王佐之才,著於往日,及诛爽之际,智略已宣,钦、俭称兵,全军独克,此足以见其英图矣。虽道盛三分,而终身北面,威名震主,而臣节不亏,侯服归全,於斯为美。”(《唐文拾遗·卷十三》)

房玄龄:“世宗以睿略創基,太祖以雄才成務。事殷之跡空存,翦商之志彌遠,三分天下,功業在焉。及逾劍銷氛,浮淮靜亂,桐宮胥怨,或所不堪。若乃體以名臣,格之端揆,周公流連於此歲,魏武得意於茲日。軒懸之樂,大啟南陽,師摯之圖,於焉北面。壯矣哉,包舉天人者也!為帝之主,不亦難乎。”“ 世宗繼文,邦權未分。三千之士,其從如雲。” (《晉書·景帝紀》)

胡三省:“王莽、司马师、萧鸾同是心也。国之奸贼,必有羽翼,有天下者,其戒之哉!”(注《资治通鉴》)

王夫之:“所恶于强臣者,唯其很耳。戆者,很之徒也。无所忌而函之心,乃可无所忌而矢诸口,遂以无所忌而见之事。司马师、高澄、朱温、李茂贞唯其言之无忌者,有以震慑乎人心,而天下且诧之曰:‘此英雄之无隐也。’”

赵翼:“司马氏当魏室未衰,乘机窃权,废一帝、弑一帝而夺其位,比之于操,其功罪不可同日语矣!”(《廿二史札记:魏晋禅代不同》)

蔡东藩:“有曹操之废伏后,乃有司马师之废张后。操废后而止,至废帝一事,留待其子曹丕;而师独以一身兼之,既废张后,复废魏主芳,乱贼效尤,比前为甚。无怪后事之愈出愈凶。然使前无曹操父子,后亦必无司马师兄弟;天鉴不远,加倍相偿,世人欲为子孙计,亦何勿稍留余地乎?”(《后汉演义》)

司馬師沉著堅強,并且有雄才大略,與夏侯玄、何晏齊名,雖言齊名但司馬師不論帶兵作戰或政治能力皆遠勝上述二人。

司馬師為人沈穩,司馬懿將“高平陵之變”計畫告訴司馬師與司馬昭後,司馬昭擔心得整晚都睡不著,而司馬師卻像平常一樣安睡。事變當時,司馬師親自率兵屯司馬門,控制京師洛阳。事後,論功封為長平鄉侯,加衛將軍之職。

\section{文帝司马昭生平}

司马昭(211年-265年9月6日),字子上(小说《三国演义》为子尚),河内郡温县(今河南省温县)人。司马懿與張春華的次子,司马师之弟,西晋开国皇帝司马炎之父,曹魏後期著名政治家和军事家。司马昭繼承其兄權位,消灭蜀汉,掌握曹魏。滅蜀後一年逝世,其子司馬炎逼曹奐禪讓後稱帝,追司馬昭為太祖文皇帝。

239年司馬昭被封為新城鄉侯。后迁散骑常侍。244年作为征蜀将军随曹爽伐蜀,副于夏侯玄。蜀将王平夜袭司马昭营,司马昭躺着不动,王平退去。司马昭对夏侯玄说:“费祎据险距守,进军不能与之交战,攻之也不行,应该立刻回师,以后再作打算。”加上司马懿也给夏侯玄写信劝说,夏侯玄最终说服曹爽回师。(興勢之戰)

249年司馬懿密谋發動高平陵之變,在政变前一天晚上才告诉司马昭,司马昭非常紧张。高平陵之變后,族滅曹爽一家,司马懿開始專權國政。

曲城之战时,司马昭任安西将军,进兵駱谷迫使姜维退兵,句安投降。之后转任安東將軍、持節,鎮守許昌。王凌之亂时,督淮北諸軍,率军和司马懿會于項。

251年司馬懿死後,他的長子司馬師任抚军大将军位,东关之战时,司马昭战败失侯。

254年姜維出隴西,以司马昭行征西將軍,驻扎长安。击破叛乱的新平羌,以功復封新城鄉侯,後廢魏帝曹芳,立曹髦,因此功進爵高都縣侯(屬并州上黨郡),增邑兩千戶。

司马师率军镇压毌丘儉之亂时,司马昭兼任中領軍,留鎮洛陽。司马师回军时在许昌病重,司马昭前往探病,拜衛將軍。司马师去世后,曹髦下令司马昭留镇许昌,让尚书傅嘏带大军回洛阳,意图从司马家夺权。司马昭听从傅嘏和钟会计谋,亲自率军回到洛阳,曹髦夺权失败。之后司马昭进位大将军、加侍中,都督中外诸军、录尚书事,辅佐国政,上殿时可以不解佩剑,不脱鞋。

256年又進為大都督、進爵高都縣公、加九錫,後又加封三縣食邑。

256年,司馬昭偕同曹魏皇帝曹髦一起平定鎮東大將軍諸葛誕的反叛。司馬昭則率軍征伐諸葛誕。東吳派文欽與全懌、全端、唐諮和王祚等領兵救援,趁王基包圍圈未完成而領兵進入壽春城。司马昭指揮魏軍主力包圍了壽春的近十八萬吳軍和叛軍,一面設計讓他們誤以為救兵將至,不再節約糧食,一面招降瓦解敵軍,諸葛誕的部下大量歸降,突圍攻擊魏軍防禦工事未成,內部矛盾激化,諸葛誕殺文欽,文欽的兒子文鴦、文虎向司馬昭投降,繼而繞城勸降,城中將士見文欽之子都被赦免,皆有降意,司馬昭見城上守軍都持弓而不放箭,就下令說可以攻城了,魏軍攻入壽春,胡奮臨陣斬諸葛誕,戰役結束。這次戰役吳軍直接投入戰鬥的士兵高達十幾萬,和諸葛誕的十八萬軍隊相加超過三十萬,超過了司馬昭指揮的二十六萬人,而魏國還要在西部抵禦蜀國的進攻,最終司馬昭以少勝多,贏得戰爭。司馬昭因功被晉陞為相國,但拒受。

魏甘露五年(260年)四月魏帝曹髦見王沈、王經、王業等人,憤慨說道:「司馬昭之心,路人皆知也!朕不能坐受廢辱,今日當與卿等自出討之。」並且率領宮人300餘人討伐。有近臣先行向司馬昭通風報信,司馬昭馬上派兵入宮鎮壓,雙方在宮內東止車門相遇,中護軍賈充在南闕下率軍迎戰曹髦,賈充命令成濟殺曹髦,成濟一劍從曹髦胸部刺穿,曹髦立即死在車上。後來司馬昭以罪誅殺成濟一族。司馬昭立曹奐為魏元帝。

263年夏,權握曹魏實權的司馬昭決定對蜀漢發動最後的戰役,处死反对伐蜀的将军邓敦,派遣鍾會、鄧艾、諸葛緒等分東、中、西三路進攻漢中。蜀漢則以姜維為首組成抵抗軍,據劍閣天險與鍾會的軍隊相持,鍾會不能前進。鄧艾遂率精兵偷渡陰平攻佔涪城,并在綿竹擊殺諸葛瞻,進逼成都。蜀漢後主劉禪出降,姜維聞訊後帶部眾向鍾會投降,蜀漢滅亡,開始三國統一的序幕。

魏元帝於264年5月2日(景元五年三月三十日)再次下詔拜司馬昭為相國,並晉陞為晉王,加九錫。郭太后已于此前不久去世,司马氏代魏之势已成。

265年司馬昭死後葬在崇陽陵,數月後司馬昭被謚為文王。他的兒子司馬炎代魏稱帝,國號晉,史稱西晉。西晉建立後他被追尊為文帝,廟號太祖。

晉明帝:「若如公言,祚安得長!」

曹髦:「司馬昭之心,路人皆知也。」 (起源自《三國志·魏書四·三少帝紀》,後來成為知名俚語。)

王经:「朝廷四方皆为之效死。」(《三國志·魏書四·三少帝紀》)

毌丘儉:「弟昭,忠肅寬明,樂善好士,有高世君子之度,忠誠為國,不與師同。」

张悌:「摧坚敌如折枯,荡异同如反掌,任贤使能,各尽其心,非智勇兼人,孰能如之?」(《三国志·卷四十八·吴书三·三嗣主传第三》)

羊祜:「先帝(司馬昭)順天應時,西平巴、蜀,南和吳會,海內得以休息,兆庶有樂安之心。而吳複背信,使邊事更興。夫期運雖天所授,而功業必由人而成,不一大舉掃滅,則役無時得安。亦所以隆先帝之勳,成無為之化也。」

習鑿齒:「自是天下畏威懷德矣。君子謂司馬大將軍於是役也,可謂能以德攻矣。夫建業者異矣,各有所尚,而不能兼併也。故窮武之雄斃於不仁,存義之國喪於懦退,今一征而禽三叛,大虜吳眾,席捲淮浦,俘馘十萬,可謂壯矣。而未及安坐,种惠吴人,结异类之情,宠鸯葬钦,忘畴昔之隙,不咎诞众,使扬土怀愧,功高而人乐其成,业广而敌怀其德,武昭既敷,文算又洽,推是道也,天下其孰能当之哉!」 

虞世南:「克宁祸乱,南定淮海,西平庸蜀,役不逾时,厥功为重。及高贵纂位,聪明夙智,朝野欣欣,方之文武,不能竭忠叶赞,拟迹伊周,遂乃伪杀彦士,委罪成济,自贻逆节,终享恶名。斯言之玷,不可为也。」(《唐文拾遗》卷十三)

房玄齡:「世宗以睿略創基,太祖以雄才成務。事殷之跡空存,翦商之志彌遠,三分天下,功業在焉。及逾劍銷氛,浮淮靜亂,桐宮胥怨,或所不堪。若乃體以名臣,格之端揆,周公流連於此歲,魏武得意於茲日。軒懸之樂,大啟南陽,師摯之圖,於焉北面。壯矣哉,包舉天人者也!為帝之主,不亦難乎。」「世宗继文,邦权未分。三千之士,其从如云。世祖無外,靈關靜氛。反雖討賊,終為弑君。」(《晉書》)

王应麟:「司马师引二败以为己过,司马昭怒王仪责在元帅之言。昭之恶,甚于师。」(《卷十三考史》)

王夫之:「司马昭、郭威虽逆,而固非朱温之暴,可以理夺者也。」(《读通鉴论·卷十·三国》)「使司马昭杀贾充以谢天下,天下其可谢,而天其弗亟绝之邪?己谋逆而人成之,事成而恶其人,心之不昧者也。」(《读通鉴论·卷十七·梁武帝》)

赵翼:「司马氏当魏室未衰,乘机窃权,废一帝、弑一帝而夺其位,比之于操,其功罪不可同日语矣!」(《廿二史札记:魏晋禅代不同》)

罗贯中:「假意投身强哭尸,公然弑主待推谁?欲诛成济瞒天下,天下人人已尽知!」「司马当年命贾充,弑君南阙赭袍红。却将成济夷三族,欲使军民耳尽聋!」

%% -*- coding: utf-8 -*-
%% Time-stamp: <Chen Wang: 2019-12-18 13:16:51>

\section{武帝\tiny(266-290)}

\subsection{生平}

晉武帝司馬炎(236年-290年5月16日),字安世,河内郡温縣(今河南省焦作市温县)人,曹魏权臣司马昭长子,晉朝開國皇帝,諡號武皇帝,在位二十五年。

魏咸熙二年(266年2月8日)十二月丙寅,晋王、相国司馬炎逼迫魏元帝禪讓,即位為帝,定有天下之号曰晉,改年號泰始。在位期间,封同姓诸王,以郡为国,置军士,希望互相维系,拱卫中央。晋武帝采取一系列经济措施以发展生产,屡次责令郡县官劝课农桑,并严禁私募佃客。又招募原吴、蜀地区人民北来,充实北方,并废屯田制,使屯田民成为州郡编户。

太康元年(280年),颁行户调式,包括占田制、户调制和品官占田荫客制。太康年间出现一片繁荣景象。晋武帝鉴于曹魏末期为政严苛,风俗颓废,生活豪奢,乃“矫以仁俭”,鳏寡孤独不能自存者赐穀人五斛,免逋债宿负,诏郡国守相巡行属县,并能容纳直言。还重视法律,亲自向百姓讲解賈充等人上所刊修律令,并亲身听讼录囚。但灭吳后,逐渐怠惰政事,沉迷女色,扩充后宫,荒淫无度。

鑒於曹魏宗室力量薄弱才讓其父祖有機可乘,因此他巩固皇权而大封宗室。然而诸王统率兵马各据一方,晋武帝死后,诸王为争夺权力,内讧不已,形成16年的内战,史称八王之乱。

司馬炎出生于236年,為司馬昭長子(母亲王元姬是經学家王肅女儿),曾出任中撫軍等要职。司马昭曾想为他求娶阮籍女,但阮籍连续醉酒六十多天,司马昭找不到提出求亲的机会,只得作罢。

司馬昭次子司馬攸是司马师的過繼養子,司馬昭曾因认为自己的權位来自於司马师,有意讓司馬攸繼承晋王位,以報答司馬師,但因重臣反對,只好於咸熙二年(265年)五月立司馬炎為世子。同年八月司馬昭過世之後,司馬炎繼承晉王的爵位。

咸熙二年(265年),司马昭病死,享年55岁。司马炎继承相国晋王位,掌握全国军政大权。经过精心准备,同年十二月,仿效当年曹丕篡汉的故事,为自己登基做准备。在司马炎接任相国后,就有一些人受司马炎指使劝说曹奂早点让位。不久,曹奂下诏书说:“晋王,你家世代辅佐皇帝,功勋高过上天,四海蒙受司马家族的恩泽,上天要我把皇帝之位让给你,请顺应天命,不要推辞!”司马炎却假意多次推让。司马炎的心腹太尉何曾、卫将军賈充等人,带领满朝文武官员再三劝谏。司马炎多次推让后,才接受曹奂禅让,封曹奂为陈留王。司马炎登上帝位,改国号为「晋」,史称为「西晋」,晋王司马炎成了晋武帝。

晋武帝施行了一系列进步政策增强国力,发展生产。此时孙吴局势混乱,吴帝孙皓不修内政又穷极奢侈,民心不附。为了防御吴国,司马炎派羊祜镇守襄阳与吴将陆抗对峙,派王濬于益州大造船舰。泰始十年(274年)陆抗去世,二年後羊祜提议伐吴,遭群臣反对而作罢。咸寧四年(278年)羊祜病故,临终推荐杜预镇守荆州。咸寧五年(279年)西北秃发树机能之乱始平,王濬、杜预上书司马炎,认为是伐吴之时,賈充、荀勖等认为西北未定而反对。最后司马炎决定于该年十二月进攻吴国,史称晋灭吴之战。他以賈充为大都督,上游王濬、唐彬军,中游杜预、胡奋、王戎军,下游王浑、司马伷军多路并进。于隔年三月逼近建业,孙皓见大势已去而投降,孙吴灭亡,西晋统一天下,三国时期结束。

泰始六年(270年),河西鲜卑领主秃发树机能叛,次年匈奴刘猛也随之出关。泰始八年(272年),司马炎派何桢招降李恪平定刘猛叛乱。咸寧元年(275年),司马炎释放奴婢替代士兵屯田,树机能归降,拓跋部沙漠汗出使晋朝,马循平定鲜卑。咸寧三年(277年)树机能复叛,司马骏帅文鸯等败树机能,降鲜卑二十万。沙漠汗被鲜卑旧贵族杀害,卫瓘平定拓跋部内乱。咸寧五年(279年)司马炎派马隆前往凉州平叛,秃发部众杀禿髮樹機能降。

建国后采取一系列经济措施以发展生产,屡次责令郡县官劝课农桑,并严禁私募佃客。又招募原吴、蜀地区人民北来,充实北方,并废屯田制,使屯田民成为州郡编户,太康三年(282年)户达到377万户。《晋书·食货志》说:“平吴之后,……天下无事,赋税平均,人咸安其业而乐其事。”太康元年,颁行户调式,包括占田制、户调制和品官占田荫客制。太康年间出现一片繁荣景象,史称“太康之治”。

司马炎鉴于魏宗室衰微,帝室孤弱,终致灭亡之教训乃大封皇族为藩王,以对抗士族。始则封王不就国,官于京师以辅皇室,继则分遣诸王就国,都督诸军事,后又出使镇要害地。此举目的,是为对抗士族中野心家。但“八王之乱”证明,这种政策反而使这些手握重兵的诸王中涌现了许多野心家。

西晋之所以重任宗室,实际上与其政权的结构有关。晋是以皇室司马氏为首门阀贵族联合统治,皇室作为一个家族驾于其它家族之上,皇帝是这个第一家族的代表,因而其家族家成员有资格也有必要取得更大权势,以保持其优越地位。[來源請求]

全国统一后,司马炎下诏:“悉去州郡兵,大郡置武吏百人,小郡五十人”,即规定:诸州无事者罢其兵。刺史只作为监司,罢将军名号,不领兵,也不兼领兵的校尉官。实行军民分治,都督校尉治军,刺史太守治民。罢州郡兵,一方面可使地方官专心民事,另一是扩大承担赋役的课丁。兵役是汉灵帝光和七年(184年)以后农民最沉重的负担,免除这负担,对恢复生产意义重大,但也因悉去州郡兵,连治安都没办法维持,因此到永寧元年(301年),天下大乱时,无力控制局面。

西晋的皇族和贵族都有优裕的经济基础,政治的安定与统一更帮助他们累积了大量的财富,于是纵情享受,过着豪华奢侈的生活。晋武帝领先作了荒淫奢纵的表率,《晋书·胡贵嫔传》称:晋武“多内宠,平吴后,复纳吴王孙皓宫人数千,自此掖庭殆将万人,而并宠者甚众,帝莫知所适,常乘羊车,恣其所之,至使宴寝”,奢侈浪费,风气日渐败坏。公卿贵游也跟着竞富争豪,大臣何曾每天吃饭用一万钱,还“无处下箸”,他的儿子何劭一定要吃四方畛异,一天膳费二万钱。王恺是武帝的母舅,曾与当时首富石崇比赛炫耀财富,争夸豪丽。为维持这种奢靡腐化的生活,必然加紧聚敛,因此贪污纳贿,习以为常,当时有人指:“奢侈之费,甚于天灾”,可见为害之大。間接養成了其子的生活態度,當有人向晉惠帝報告老百姓無食物吃(天下荒飢,百姓餓死),晉惠帝卻反問:「何不食肉糜?」。

太熙元年(290年)四月己酉(5月16日),晋武帝司马炎驾崩于含章殿,享年五十四歲。五月辛未(6月12日),葬於峻陽陵。其次子司马衷即位,是为晋惠帝。不过一年後,皇后贾南风发动政变,杀死总揽朝政的大臣杨骏;接着又发生了“八王之乱”。建兴四年(316年),刘渊的侄子刘曜攻破长安,俘获末代皇帝司马邺,西晋亡国。时距司马炎之死只有26年。

司马炎最突出的性格特点正如《晋书》论赞所言,“宇量弘厚”,性情温和宽厚而较能包容异己者:不杀退位的曹魏皇室和投降的蜀漢、孙吴皇室,封曹奂为陈留王、孙皓为归命侯,并允许刘协和刘禅的子孙继续承袭山阳公和安乐公的爵位(相对来说这与后继的南朝君主们的作风不太一样)。王恺、石崇斗富,石崇公然砸碎司马炎赐给王恺的珊瑚,却没有因此而受到司马炎的责罚。泰始九年,司马炎下诏为邓艾平反,厚葬邓艾尸首、归还籍没的田宅并赠给其孙邓朗官职。司马炎纳山涛之议,允许被司马昭处死的嵇康之子嵇绍入朝为官。前蜀汉官员李密上《陈情表》,以侍奉祖母为由拒绝出仕晋朝,司马炎同意并赐给李密的祖母粮米布帛。王浑上表弹劾王濬在灭吴战争中违背诏令,不听调遣,司马炎只是下诏责备王濬而未予以任何处罚。冯紞在反对张华拜相时,当着司马炎的面说“臣窃谓锺会之衅,颇由太祖。”亦即钟会的叛乱是由于司马昭的失策造成的。司马炎虽当场发怒“变色”但最终仍采纳冯紞的意见,改任张华为太常。司隶校尉刘毅在南郊祭典上批评司马炎卖官敛财,并说司马炎连汉末的桓灵二帝都不如:“桓、灵卖官,钱入官库;陛下卖官,钱入私门。以此言之,殆不如也。”司马炎只是“大笑”而并未对刘毅有何不利。刘毅在之后还升任尚书左仆射。

何曾:“聪明神武,有超世之才。”“主上开创大业,吾每宴见,未尝闻经国远图,惟说平生常事,非贻厥孙谋之道也,及身而已,后嗣其殆乎!”(《资治通鉴·卷第八十七·晋纪九》)

刘毅:“桓、灵卖官,钱入官库;陛下卖官,钱入私门。以此言之,殆不如也。”(《晋书·列传第十五》)

陆云:“世祖武皇帝临朝拱默,训世以俭,即位二十有六载,宫室台榭无所新营,屡发明诏,厚戒丰奢。”(《晋书·列传第二十四》)

曹毗:“于穆武皇,允龚钦明。应期登禅,龙飞紫庭。百揆时序,听断以情。殊域既宾,伪吴亦平。晨流甘露,宵映朗星。野有击壤,路垂颂声。”(宋书·卷二十◎志第十◎乐二)

干宝:“至于世祖,遂享皇极,仁以厚下,俭以足用,和而不弛,宽而能断,掩唐、虞之旧域,班正朔于八荒,于时有“天下无穷人”之谚,虽太平未洽,亦足以明民乐其生矣。武皇既崩,山陵未干而变难继起。宗子无维城之助,师尹无具瞻之贵,朝为伊、周,夕成桀、跖;国政迭移于乱人,禁兵外散于四方,方岳无钧石之镇,关门无结草之固。戎、羯称制,二帝失尊,何哉?树立失权,托付非才,四维不张,而苟且之政多也。”(《资治通鉴·卷第八十九》)

谢灵运:“世祖受命,祯祥屡臻,苛慝不作,万国欣戴。远至迩安,德足以彰,天启其运,民乐其功矣。反古之道,当以美事为先。今五等罔刑,井田王制,凡诸礼律,未能定正,而采择嫔媛,不拘华门者。昔武王伐纣,归倾宫之女,不以助纣为虐。而世祖平皓,纳吴妓五千,是同皓之弊。妇人之封,六国乱政。如追赠外曾祖母,违古之道。凡此非事,并见前书,诚有点於徽猷,史氏所不敢蔽也。”(《太平御览·卷九十六》)

虞世南:“武帝平一天下,谁曰不然,至於创业垂统,其道则阙矣。夫帝王者,必立德立功,可大可久,经之以仁义,纬之以文武,深根固蒂,贻厥子孙,一言一行,以为轨范,垂之万代,为不可易。武帝平吴之後,怠於政事,蔽惑邪佞,留心内宠,用冯紞之谗言,拒和峤之正谏,智士永叹,有识寒心。以此国风,传之庸子,遂使坟土未乾,四海鼎沸,衣冠殄灭,县宇星分,何曾之言,於是信矣。其去明主,不亦远乎?”(《唐文拾遗》卷十三)

房玄龄《晋书》:“帝宇量弘厚,造次必于仁恕;容纳谠正,未尝失色于人;明达善谋,能断大事,故得抚宁万国,绥静四方。承魏氏奢侈革弊之后,百姓思古之遗风,乃厉以恭俭,敦以寡欲。有司尝奏御牛青丝纼断,诏以青麻代之。临朝宽裕,法度有恒。高阳许允既为文帝所杀,允子奇为太常丞。帝将有事于太庙,朝议以奇受害之门,不欲接近左右,请出为长史。帝乃追述允夙望,称奇之才,擢为祠部郎,时论称其夷旷。平吴之后,天下乂安,遂怠于政术,耽于游宴,宠爱后党,亲贵当权,旧臣不得专任,彝章紊废,请谒行矣。爰至未年,知惠帝弗克负荷,然恃皇孙聪睿,故无废立之心。复虑非贾后所生,终致危败,遂与腹心共图后事。说者纷然,久而不定,竟用王佑之谋,遣太子母弟秦王柬都督关中,楚王玮、淮南王允并镇守要害,以强帝室。又恐杨氏之逼,复以佑为北军中候,以典禁兵。既而寝疾弥留,至于大渐,佐命元勋,皆已先没,群臣惶惑,计无所从。会帝小差,有诏以汝南王亮辅政,又欲令朝士之有名望年少者数人佐之,杨骏秘而不宣。帝复寻至迷乱,杨后辄为诏以骏辅政,促亮进发。帝寻小间,问汝南王来未,意欲见之,有所付托。左右答言未至,帝遂困笃。中朝之乱,实始于斯矣。”

周昙《晋门晋武帝》:“汉贪金帛鬻公卿,财赡羸军冀国宁。晋武鬻官私室富,是知犹不及桓灵。”

李世民:“武皇承基,诞膺天命,握图御宇,敷化导民,以佚代劳。以治易乱。绝缣绝之贡,去雕琢之饰,制奢俗以变俭约,止浇风而反淳朴。雅好直言,留心采擢,刘毅、裴楷以质直见容,嵇绍、许奇虽仇雠不弃。仁以御物,宽而得众,宏略大度,有帝王之量焉。于是民和俗静,家给人足,聿修武用,思启封疆。决神算于深衷,断雄图于议表。马隆西伐,王濬南征,师不延时,獯虏削迹,兵无血刃,扬越为墟。通上代之不通,服前王之未服。祯祥显应,风教肃清,天人之功成矣,霸王之业大矣。虽登封之礼,让而不为,骄泰之心,因斯而起。见土地之广,谓万弃而无虞;睹天下之安,谓千年而永治。不知处广以思狭,则广可长广;居治而忘危,则治无常治。加之建立非所,委寄失才,志欲就于升平,行先迎于祸乱。是犹将适越者指沙漠以遵途,欲登山者涉舟航而觅路,所趣逾远,所尚转难,南北倍殊,高下相反,求其至也,不亦难乎!况以新集易动之基,而久安难拔之虑,故贾充凶竖,怀奸志以拥权;杨骏豺狼,苞祸心以专辅。及乎宫车晚出,谅闇未周,籓翰变亲以成疏,连兵竞灭其本;栋梁回忠而起伪,拥众各举其威。曾未数年,网纪大乱,海内版荡,宗庙播迁。帝道王猷,反居文身之俗;神州赤县,翻成被发之乡。弃所大以资人,掩其小而自托,为天下笑,其故何哉?良由失慎于前,所以贻患于后。且知子者贤父,知臣者明君;子不肖则家亡,臣不忠则国乱;国乱不可以安也,家亡不可以全也。是以君子防其始,圣人闲其端。而世祖惑荀勖之奸谋,迷王浑之伪策,心屡移于众口,事不定于己图。元海当除而不除,卒令扰乱区夏;惠帝可废而不废,终使倾覆洪基。夫全一人者德之轻,拯天下者功之重,弃一子者忍之小,安社稷者孝之大;况乎资三世而成业,延二孽以丧之,所谓取轻德而舍重功,畏小忍而忘大孝。圣贤之道,岂若斯乎!虽则善始于初,而乖令终于末,所以殷勤史策,不能无慷慨焉。”(《晋书·帝纪三》)

徐惠:“昔秦皇并吞六国,反速危亡之基;晋武奄有三方,翻成覆败之业。岂非矜功恃大,弃德而轻邦;图利忘害,肆情而纵欲?遂使悠悠六合,虽广不救其亡;嗷嗷黎庶,因弊以成其祸。”

刘仁轨:“晋代平吴,史籍具载。内有武帝、张华,外有羊祜、杜预,筹谋策画,经纬谘询。王濬之徒,折冲万里,楼船战舰,已到石头。贾充、王浑之辈,犹欲斩张华以谢天下。武帝报云:‘平吴之计,出自朕意,张华同朕见耳,非其本心。’是非不同,乖乱如此。平吴之后,犹欲苦绳王濬,赖武帝拥护,始得保全。不逢武帝圣明,王濬不存首领。”(《旧唐书·列传第三十四》)

苏辙:“武帝之为人,好善而不择人,苟安而无远虑,虽贤人满朝,而贾充、荀勖之流以为腹心,使吴尚在,相持而不敢肆,虽为贤君可也。吴亡之后,荒于女色,蔽于庸子,疏贤臣,近小人,去武备,崇藩国,所以兆亡国之祸者,不可胜数,此则灭吴之所从致也。”(《栾城集·卷五十》)

司马光:“至于晋武独以天性矫而行之,可谓不世之贤君。”(《资治通鉴·卷第七十九》)

陈普:“宫中掷戟又飞刀,谢玖兢兢命若毛。岂是君王轻社稷,天教炽业谢芳髦。”

邓林:“秋风铜爵曲池平,吴主宫娃满掖庭。凭仗皇孙聪慧早,不知祸在夕阳亭。”

孙承恩:“帝资弘裕,明达好谋。纂述先志,混一九州。礼优三恪,忠厚之道。贻谋弗臧,识者所少。”(《文简集·卷三十八》)

李慈铭:“晋武帝纯孝性成,三代以下不多得。”(《越缦堂读书记》)

蔡东藩:“彼如马隆之得平树机能,未始非晋初名将,观晋武之倚重两人,乃知开国之主,必有所长,不得以外此瑕疵,遽掩其知人之明也。”“武帝既知太子不聪,复恨贾妃之奇悍,废之锢之,何必多疑,乃被欺于狡吏而不之知,牵情于皇孙而不之断,受朦于宫帟而不之觉,卒至一误再误,身死而天下乱,名为开国,实是覆宗,王之不明,宁足福哉?”

《满江红》:“承祖余威。曹魏末、受禅称帝。扫吴越,多方齐下,摧枯之势。 吴主凶残民向背,晋公仁厚人归帜。施仁政,创社稷荣繁。功难讳。 世袭制,官僚继。田与土,官伸挤。痛卖官鬻爵,竟奢豪气。 罢却州兵难稳序,助长王势终遭戏。叹武帝,开国未几年。皇权弃。”

\subsection{泰始}

\begin{longtable}{|>{\centering\scriptsize}m{2em}|>{\centering\scriptsize}m{1.3em}|>{\centering}m{8.8em}|}
  % \caption{秦王政}\
  \toprule
  \SimHei \normalsize 年数 & \SimHei \scriptsize 公元 & \SimHei 大事件 \tabularnewline
  % \midrule
  \endfirsthead
  \toprule
  \SimHei \normalsize 年数 & \SimHei \scriptsize 公元 & \SimHei 大事件 \tabularnewline
  \midrule
  \endhead
  \midrule
  元年 & 265 & \tabularnewline\hline
  二年 & 266 & \tabularnewline\hline
  三年 & 267 & \tabularnewline\hline
  四年 & 268 & \tabularnewline\hline
  五年 & 269 & \tabularnewline\hline
  六年 & 270 & \tabularnewline\hline
  七年 & 271 & \tabularnewline\hline
  八年 & 272 & \tabularnewline\hline
  九年 & 273 & \tabularnewline\hline
  十年 & 274 & \tabularnewline
  \bottomrule
\end{longtable}

\subsection{咸宁}


\begin{longtable}{|>{\centering\scriptsize}m{2em}|>{\centering\scriptsize}m{1.3em}|>{\centering}m{8.8em}|}
  % \caption{秦王政}\
  \toprule
  \SimHei \normalsize 年数 & \SimHei \scriptsize 公元 & \SimHei 大事件 \tabularnewline
  % \midrule
  \endfirsthead
  \toprule
  \SimHei \normalsize 年数 & \SimHei \scriptsize 公元 & \SimHei 大事件 \tabularnewline
  \midrule
  \endhead
  \midrule
  元年 & 275 & \tabularnewline\hline
  二年 & 276 & \tabularnewline\hline
  三年 & 277 & \tabularnewline\hline
  四年 & 278 & \tabularnewline\hline
  五年 & 279 & \tabularnewline\hline
  六年 & 280 & \tabularnewline
  \bottomrule
\end{longtable}

\subsection{太康}

\begin{longtable}{|>{\centering\scriptsize}m{2em}|>{\centering\scriptsize}m{1.3em}|>{\centering}m{8.8em}|}
  % \caption{秦王政}\
  \toprule
  \SimHei \normalsize 年数 & \SimHei \scriptsize 公元 & \SimHei 大事件 \tabularnewline
  % \midrule
  \endfirsthead
  \toprule
  \SimHei \normalsize 年数 & \SimHei \scriptsize 公元 & \SimHei 大事件 \tabularnewline
  \midrule
  \endhead
  \midrule
  元年 & 280 & \tabularnewline\hline
  二年 & 281 & \tabularnewline\hline
  三年 & 282 & \tabularnewline\hline
  四年 & 283 & \tabularnewline\hline
  五年 & 284 & \tabularnewline\hline
  六年 & 285 & \tabularnewline\hline
  七年 & 286 & \tabularnewline\hline
  八年 & 287 & \tabularnewline\hline
  九年 & 288 & \tabularnewline\hline
  十年 & 289 & \tabularnewline
  \bottomrule
\end{longtable}

\subsection{太熙}

\begin{longtable}{|>{\centering\scriptsize}m{2em}|>{\centering\scriptsize}m{1.3em}|>{\centering}m{8.8em}|}
  % \caption{秦王政}\
  \toprule
  \SimHei \normalsize 年数 & \SimHei \scriptsize 公元 & \SimHei 大事件 \tabularnewline
  % \midrule
  \endfirsthead
  \toprule
  \SimHei \normalsize 年数 & \SimHei \scriptsize 公元 & \SimHei 大事件 \tabularnewline
  \midrule
  \endhead
  \midrule
  元年 & 290 & \tabularnewline
  \bottomrule
\end{longtable}


%%% Local Variables:
%%% mode: latex
%%% TeX-engine: xetex
%%% TeX-master: "../Main"
%%% End:

%% -*- coding: utf-8 -*-
%% Time-stamp: <Chen Wang: 2021-11-01 11:38:22>

\section{惠帝司马衷\tiny(290-306)}

\subsection{生平}

晋惠帝司马衷(259年2月13日-307年1月8日),字正度,是晋武帝司马炎的次子,西晋的第二位皇帝,290年至307年在位,其正式諡號為「孝惠皇帝」,後世省略「孝」字稱「晉惠帝」。在他的统治期间发生八王之乱,西晋走向灭亡。

晋惠帝生于259年2月13日。晋惠帝于泰始三年(267年)被立为太子,他的母亲是晋武帝的皇后楊艷。作为次子他被立为太子是因为他的哥哥司马轨很早就死了,也有說是晋武帝為了將來傳位給他寵愛的聰明孫子湣懷太子司马遹。

晋惠帝一般被評價为“甚愚”,王夫之說他是“土木偶人”,《晋书》中記載武帝担忧晋惠帝的能力,多次对他进行考验,而惠帝則在太子妃賈南風及謀臣的獻策下通過了这些考验。即位后,他显然无法解决他统治时期的政治困难,造成了八王之乱,成为了傀儡,最后被东海王司马越毒死。

根據刘驰的研究认为,虽然惠帝昏庸,政治才能低下,无法应对朝中局面,但是从今日的医学概念来判斷,他不能夠算作智能障礙。

272年惠帝奉武帝命娶贾充之女贾南风为太子妃。

290年武帝去世,惠帝登基,尊繼母杨芷(楊艷的堂妹)为皇太后,妻子贾南风为皇后,司马遹为太子。惠帝当政后非常信任他的皇后。因此贾氏专权,甚至假造惠帝的诏书。291年迫害皇太后,废其太后位,后又杀大臣如太宰司马亮。291年賈皇后又杀皇太后。

294年和296年匈奴和其他少数民族反叛,氐人齐万年称帝,一直到299年这次反叛才被消灭。

同年賈后开始迫害太子遹,首先废他的太子地位。次年杀太子。这个举动成为许多反对贾后专政的皇族开始行动的起点。赵王司马伦假造诏书废杀贾后,杀大臣如司空张华等,自领相国位,这是八王之乱的开始。恢复原太子的地位,立故太子之子司马臧为皇太孫。300年八月淮南王司马允举兵讨伐司马伦,兵败被杀。同年十二月,益州刺史赵廞协同从中原逃到四川的流民在成都造反。

301年司马伦篡位,自立为皇帝,惠帝被改为太上皇,太孫司马臧被杀。三月,齐王司马冏起兵反司马伦,受到成都王司马颖、河间王司马颙、常山王司马乂等的支持。司马伦兵败。淮陵王司马漼杀司马伦的党羽,驱逐司马伦,引惠帝复位。司马伦被杀。五月,立襄阳王司马尚为皇太孙,並以羊獻容为皇后。六月,东莱王司马蕤谋推翻司马冏的专权,事漏被废。十二月,李特开始在四川反晋,这是成汉的起点。

302年初皇太孙司马尚夭折,司马覃被立为太子。五月,李特在四川击败了司马颙派去讨伐他的军队,杀广汉太守张微,自立为大将军。十二月,司马颖、司马颙、新野王司马歆和范阳王司马虓在洛阳聚会反司马冏的专政。司马乂乘机杀司马冏,成为朝内的权臣。

303年三月李特在攻成都时被杀,但四月他的儿子李雄就占领了成都,到年末,李雄几乎占领了整个四川盆地。五月张昌、丘沈反,建国汉,杀司马歆。八月,司马颖和司马颙讨伐司马乂。十月,司马颙的军队攻擊首都洛陽,在此后的洗劫中上万人死亡。此后两军在洛陽城外对阵,连十三岁的少年都被征军,同时两军都征募匈奴等的军队。最后司马乂兵败被杀。司马颙成为晋朝举足轻重的人物。

304年初惠帝感到受到司马颙的威胁越来越大,因此下密诏给刘沈和皇甫重攻司马颙,但没有成功。司马颙的军队在洛阳大肆抢劫。二月废皇后羊氏,废皇太子司马覃,立司马颖为皇太弟。司马颖和司马颙专政。但六月京城又发生政变,司马颖被逐,羊氏復位为皇后,司马覃復位为太子。七月,惠帝率军讨伐司马颖,在荡阴被司马颖的军队战败,惠帝面部中伤,身中三箭,被司马颖俘虏。羊氏和司马覃再次被废。八月,司马颖被安北将军王浚战败,他挟持惠帝逃亡到洛阳。一路上只有粗米为饭。十一月,惠帝又被司马颙的将军张方劫持到长安,张方的军队抢劫皇宫,将皇宫内的宝藏洗劫一空。到年末司马颙再次在长安一揽大权,司马越成为太傅。同年李雄在成都称成都王,成汉建国,刘渊自称汉王,建立前赵。305年司马颙和张方的军队、司马颖的军队、司马越的军队和范阳王司马虓的军队在中原混战,基本上中央政府已经不存在,中国边缘的地区纷纷独立。到305年末,司马越战胜,司马颙杀张方向司马越请和,但无效。

306年司马越手下的鲜卑军队攻入长安,大肆抢劫,二万多人被杀。九月,司马颖被俘,后被杀。

307年光熙元年十一月庚午(307年1月8日),惠帝于长安显阳殿去世,可能是被司马越毒死的。惠帝死后葬太阳陵。

因為他在位期間全國先後經歷八王之亂和五胡亂華等重大事件的影響,連帶使他在中國歷史上非自願地創造紀錄:最多皇儲(?,以上太弟、太孫加上太子,待另考)。最多皇太弟(2個),司馬穎、司馬熾。最多皇太孫(2個),司馬臧、司馬尚。被自己的長輩尊為太上皇(司馬倫為惠帝的叔公)。其第二任皇后羊獻容曾任2個不同政權君主的皇后,除晉惠帝外,後來又為前趙帝劉曜之后。主政18年,用了9個年號。除了時間最長的元康年號(9年),9年間用了8個年號,更換年號的頻率和武則天相當,使用過的年號數量則居武則天之後(武則天為帝的16年使用15個年號,為帝前握有實權的6年使用3個年號),中國眾多皇帝的第二位。

由於晉惠帝的昏庸愚蠢名動天下,使原本不帶貶意的諡號“惠”在他用過之後成了昏君代名詞,故後來如唐宋這類長期統治中原的主流大朝代皆避免用惠字當皇帝諡號。

有人向晉惠帝报告老百姓无食物吃(天下荒饥,百姓饿死),他反问:“何不食肉糜?”(為何不吃肉粥?)。

还有一次,晋惠帝游上林苑,听到蛤蟆叫,问身旁的人:“此鸣者,为官乎?为私乎?”(在叫的蛤蟆是官家的还是私人的啊?)

戰亂時,嵇康之子嵇紹以身捍衛,血噴滿惠帝整身 《晋书·忠义传·嵇绍》:“绍以天子蒙尘,承诏驰诣行在所。值王师败绩于荡阴,百官及侍卫莫不散溃,唯绍俨然端冕,以身捍卫,兵交御辇,飞箭雨集。绍遂被害于帝侧,血溅御服,天子深哀叹之。及事定,左右欲浣衣,帝曰:‘此嵇侍中血,勿去。’(這是嵇侍中的血,不能去掉)”

\subsection{永熙}

\begin{longtable}{|>{\centering\scriptsize}m{2em}|>{\centering\scriptsize}m{1.3em}|>{\centering}m{8.8em}|}
  % \caption{秦王政}\
  \toprule
  \SimHei \normalsize 年数 & \SimHei \scriptsize 公元 & \SimHei 大事件 \tabularnewline
  % \midrule
  \endfirsthead
  \toprule
  \SimHei \normalsize 年数 & \SimHei \scriptsize 公元 & \SimHei 大事件 \tabularnewline
  \midrule
  \endhead
  \midrule
  元年 & 290 & \tabularnewline
  \bottomrule
\end{longtable}

\subsection{永平}

\begin{longtable}{|>{\centering\scriptsize}m{2em}|>{\centering\scriptsize}m{1.3em}|>{\centering}m{8.8em}|}
  % \caption{秦王政}\
  \toprule
  \SimHei \normalsize 年数 & \SimHei \scriptsize 公元 & \SimHei 大事件 \tabularnewline
  % \midrule
  \endfirsthead
  \toprule
  \SimHei \normalsize 年数 & \SimHei \scriptsize 公元 & \SimHei 大事件 \tabularnewline
  \midrule
  \endhead
  \midrule
  元年 & 291 & \tabularnewline
  \bottomrule
\end{longtable}

\subsection{元康}

\begin{longtable}{|>{\centering\scriptsize}m{2em}|>{\centering\scriptsize}m{1.3em}|>{\centering}m{8.8em}|}
  % \caption{秦王政}\
  \toprule
  \SimHei \normalsize 年数 & \SimHei \scriptsize 公元 & \SimHei 大事件 \tabularnewline
  % \midrule
  \endfirsthead
  \toprule
  \SimHei \normalsize 年数 & \SimHei \scriptsize 公元 & \SimHei 大事件 \tabularnewline
  \midrule
  \endhead
  \midrule
  元年 & 291 & \tabularnewline\hline
  二年 & 292 & \tabularnewline\hline
  三年 & 293 & \tabularnewline\hline
  四年 & 294 & \tabularnewline\hline
  五年 & 295 & \tabularnewline\hline
  六年 & 296 & \tabularnewline\hline
  七年 & 297 & \tabularnewline\hline
  八年 & 298 & \tabularnewline\hline
  九年 & 299 & \tabularnewline
  \bottomrule
\end{longtable}

\subsection{永康}

\begin{longtable}{|>{\centering\scriptsize}m{2em}|>{\centering\scriptsize}m{1.3em}|>{\centering}m{8.8em}|}
  % \caption{秦王政}\
  \toprule
  \SimHei \normalsize 年数 & \SimHei \scriptsize 公元 & \SimHei 大事件 \tabularnewline
  % \midrule
  \endfirsthead
  \toprule
  \SimHei \normalsize 年数 & \SimHei \scriptsize 公元 & \SimHei 大事件 \tabularnewline
  \midrule
  \endhead
  \midrule
  元年 & 300 & \tabularnewline\hline
  二年 & 301 & \tabularnewline
  \bottomrule
\end{longtable}

\subsection{永宁}

\begin{longtable}{|>{\centering\scriptsize}m{2em}|>{\centering\scriptsize}m{1.3em}|>{\centering}m{8.8em}|}
  % \caption{秦王政}\
  \toprule
  \SimHei \normalsize 年数 & \SimHei \scriptsize 公元 & \SimHei 大事件 \tabularnewline
  % \midrule
  \endfirsthead
  \toprule
  \SimHei \normalsize 年数 & \SimHei \scriptsize 公元 & \SimHei 大事件 \tabularnewline
  \midrule
  \endhead
  \midrule
  元年 & 301 & \tabularnewline\hline
  二年 & 302 & \tabularnewline
  \bottomrule
\end{longtable}

\subsection{太安}

\begin{longtable}{|>{\centering\scriptsize}m{2em}|>{\centering\scriptsize}m{1.3em}|>{\centering}m{8.8em}|}
  % \caption{秦王政}\
  \toprule
  \SimHei \normalsize 年数 & \SimHei \scriptsize 公元 & \SimHei 大事件 \tabularnewline
  % \midrule
  \endfirsthead
  \toprule
  \SimHei \normalsize 年数 & \SimHei \scriptsize 公元 & \SimHei 大事件 \tabularnewline
  \midrule
  \endhead
  \midrule
  元年 & 302 & \tabularnewline\hline
  二年 & 303 & \tabularnewline
  \bottomrule
\end{longtable}

\subsection{永安}

\begin{longtable}{|>{\centering\scriptsize}m{2em}|>{\centering\scriptsize}m{1.3em}|>{\centering}m{8.8em}|}
  % \caption{秦王政}\
  \toprule
  \SimHei \normalsize 年数 & \SimHei \scriptsize 公元 & \SimHei 大事件 \tabularnewline
  % \midrule
  \endfirsthead
  \toprule
  \SimHei \normalsize 年数 & \SimHei \scriptsize 公元 & \SimHei 大事件 \tabularnewline
  \midrule
  \endhead
  \midrule
  元年 & 304 & \tabularnewline
  \bottomrule
\end{longtable}

\subsection{建武}

\begin{longtable}{|>{\centering\scriptsize}m{2em}|>{\centering\scriptsize}m{1.3em}|>{\centering}m{8.8em}|}
  % \caption{秦王政}\
  \toprule
  \SimHei \normalsize 年数 & \SimHei \scriptsize 公元 & \SimHei 大事件 \tabularnewline
  % \midrule
  \endfirsthead
  \toprule
  \SimHei \normalsize 年数 & \SimHei \scriptsize 公元 & \SimHei 大事件 \tabularnewline
  \midrule
  \endhead
  \midrule
  元年 & 304 & \tabularnewline
  \bottomrule
\end{longtable}

\subsection{永兴}

\begin{longtable}{|>{\centering\scriptsize}m{2em}|>{\centering\scriptsize}m{1.3em}|>{\centering}m{8.8em}|}
  % \caption{秦王政}\
  \toprule
  \SimHei \normalsize 年数 & \SimHei \scriptsize 公元 & \SimHei 大事件 \tabularnewline
  % \midrule
  \endfirsthead
  \toprule
  \SimHei \normalsize 年数 & \SimHei \scriptsize 公元 & \SimHei 大事件 \tabularnewline
  \midrule
  \endhead
  \midrule
  元年 & 304 & \tabularnewline\hline
  二年 & 305 & \tabularnewline\hline
  三年 & 306 & \tabularnewline
  \bottomrule
\end{longtable}

\subsection{光熙}

\begin{longtable}{|>{\centering\scriptsize}m{2em}|>{\centering\scriptsize}m{1.3em}|>{\centering}m{8.8em}|}
  % \caption{秦王政}\
  \toprule
  \SimHei \normalsize 年数 & \SimHei \scriptsize 公元 & \SimHei 大事件 \tabularnewline
  % \midrule
  \endfirsthead
  \toprule
  \SimHei \normalsize 年数 & \SimHei \scriptsize 公元 & \SimHei 大事件 \tabularnewline
  \midrule
  \endhead
  \midrule
  元年 & 306 & \tabularnewline
  \bottomrule
\end{longtable}


%%% Local Variables:
%%% mode: latex
%%% TeX-engine: xetex
%%% TeX-master: "../Main"
%%% End:

%% -*- coding: utf-8 -*-
%% Time-stamp: <Chen Wang: 2021-11-01 11:38:45>

\section{怀帝司馬熾\tiny(306-313)}

\subsection{生平}

晉懷帝司馬熾(284年-313年),字豐度,西晉的第三代皇帝,司馬炎的第二十五子,其正式諡號為「孝懷皇帝」,後世省略「孝」字稱「晋怀帝」。

司馬熾生於太康五年(284年),生母為晉武帝中才人王媛姬,王媛姬死后由晋武帝继后杨芷抚养。武帝太熙元年(290年)司馬熾被封為豫章王,四月,司馬炎病死。太子司馬衷即位,是為晉惠帝,在晉惠帝在位期間爆發的八王之亂中,司馬熾並未加入亂事,並且行事低調,不太熱衷於交結賓客,愛好鑽研史籍。司馬熾本人並無雄才大略,最初擔任散騎常侍,永康二年(301年)趙王司馬倫廢晉惠帝時,司馬熾的散騎常侍也被罷黜,同年四月晉惠帝復位後,改元永寧元年,熾任射聲校尉。永寧三年(304年)出任鎮北大將軍,同年被立為皇太弟。但是立司馬熾為皇太弟,是由於成都王司馬穎和河間王司馬顒對立之下的結果,其實司馬熾本人並沒有權力的野心。

光熙元年十一月庚午(307年1月8日)東海王司馬越毒死惠帝,1月11日,司馬熾即位,改元永嘉,司馬越为太傅辅政,政局為司馬越把持。司马炽亦改葬追谥先前被废的养母杨芷。在此期間,匈奴等少數民族也開始建立獨立的政權,其中劉淵已建漢國,但是晉朝內部的權力鬥爭也日漸嚴重。永嘉五年(311年)正月,晉懷帝密詔苟晞討司馬越,三月發佈詔書討伐,司馬越於同月病死,眾共推王衍為元帥。四月王衍遣軍隊在護送司馬越靈柩回到東海封國時,與漢國鎮東大將軍石勒的二萬軍隊於苦縣(河南鹿邑)寧平城(河南省鄲城縣東寧平鄉)作戰,晋军全被殲滅,石勒焚燒司馬越的靈樞。王衍被擒時,勸石勒建國稱帝,以求苟活,但仍被石勒活埋。西晉最後一支主要兵力被消滅,已無可戰之兵。

311年六月劉淵之子劉聰的軍隊攻入洛陽,晉懷帝在逃往長安途中被俘,太子司馬詮被殺,史稱永嘉之乱。晉懷帝被送往平陽,劉聰告訴他:“卿為豫章王時,朕嘗與王武子(濟)相造,武子示朕於卿,卿言聞其名久矣……”後封為會稽公,並被囚禁。313年正月,晉懷帝在朝會上被命令為斟酒的僕人,有晉朝舊臣號哭,令劉聰反感,不久用毒酒毒殺懷帝,得年30歲,葬處不明。

\subsection{永嘉}

\begin{longtable}{|>{\centering\scriptsize}m{2em}|>{\centering\scriptsize}m{1.3em}|>{\centering}m{8.8em}|}
  % \caption{秦王政}\
  \toprule
  \SimHei \normalsize 年数 & \SimHei \scriptsize 公元 & \SimHei 大事件 \tabularnewline
  % \midrule
  \endfirsthead
  \toprule
  \SimHei \normalsize 年数 & \SimHei \scriptsize 公元 & \SimHei 大事件 \tabularnewline
  \midrule
  \endhead
  \midrule
  元年 & 307 & \tabularnewline\hline
  二年 & 308 & \tabularnewline\hline
  三年 & 309 & \tabularnewline\hline
  四年 & 310 & \tabularnewline\hline
  五年 & 311 & \tabularnewline\hline
  六年 & 312 & \tabularnewline\hline
  七年 & 313 & \tabularnewline
  \bottomrule
\end{longtable}


%%% Local Variables:
%%% mode: latex
%%% TeX-engine: xetex
%%% TeX-master: "../Main"
%%% End:

%% -*- coding: utf-8 -*-
%% Time-stamp: <Chen Wang: 2021-11-01 11:39:01>

\section{愍帝司馬鄴\tiny(313-316)}

\subsection{生平}

晉愍帝司馬鄴(300年-318年2月7日),字彥旗,西晉的第五任也是最後一任皇帝。司馬鄴為吳王司馬晏之子,晉武帝之孫,外祖父是荀勖。

司馬鄴首先過繼於秦王司馬柬而被封為秦王。308年被封為散騎常侍與撫軍將軍。晉懷帝於洛陽被俘之後司馬鄴逃亡許昌,後在雍州刺史賈疋的護送下逃入長安,之後被封為皇太子。313年晉懷帝於平陽遇害之後,司馬鄴於長安即帝位,改元建興,是为晉愍帝。

晉愍帝即位時,西晉已經沒有可以作戰的能力,而且長安也沒有可與前趙作戰的物資。316年8月劉曜發兵攻打長安,並切斷長安的糧運。晉愍帝在食斷糧絕的情況之下於建兴四年十一月十一日(316年12月11日)投降前趙,之後被送往平陽,封為懷平侯,並且承受為狩獵隊伍前導以及为宴會洗杯子等雜役的屈辱。建兴五年十二月二十日(318年2月7日)被殺。

\subsection{建兴}

\begin{longtable}{|>{\centering\scriptsize}m{2em}|>{\centering\scriptsize}m{1.3em}|>{\centering}m{8.8em}|}
  % \caption{秦王政}\
  \toprule
  \SimHei \normalsize 年数 & \SimHei \scriptsize 公元 & \SimHei 大事件 \tabularnewline
  % \midrule
  \endfirsthead
  \toprule
  \SimHei \normalsize 年数 & \SimHei \scriptsize 公元 & \SimHei 大事件 \tabularnewline
  \midrule
  \endhead
  \midrule
  元年 & 313 & \tabularnewline\hline
  二年 & 314 & \tabularnewline\hline
  三年 & 315 & \tabularnewline\hline
  四年 & 316 & \tabularnewline\hline
  五年 & 317 & \tabularnewline
  \bottomrule
\end{longtable}


%%% Local Variables:
%%% mode: latex
%%% TeX-engine: xetex
%%% TeX-master: "../Main"
%%% End:


%%% Local Variables:
%%% mode: latex
%%% TeX-engine: xetex
%%% TeX-master: "../Main"
%%% End:
