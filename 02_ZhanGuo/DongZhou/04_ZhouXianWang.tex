%% -*- coding: utf-8 -*-
%% Time-stamp: <Chen Wang: 2021-11-02 15:50:33>

\subsection{显王扁{\tiny(BC368-BC321)}}

\subsubsection{生平}

周显王(?-前321年),又称周显圣王或周显声王,姓姬,名扁,中国东周君主,在位48年,为周烈王之弟。

周显王五年(前364年)发生河西之战,秦献公亲率主力进至河东,秦将章蟜在石门山(今山西省运城市西南)大败魏军,斩首6万。由于赵国出兵救援魏国,秦才退兵。此战为秦国对魏国的首次重大胜利,诸侯震动,周显王亦祝贺“献公称伯”,并颁赏他绣着黼黻图案的服饰。

周显王十三年(前356年),秦国商鞅变法。周显王十六年(前353年)发生桂陵之战,周显王二十八年(前341年)发生马陵之战。

\subsubsection{年表}

% \centering
\begin{longtable}{|>{\centering\scriptsize}m{2em}|>{\centering\scriptsize}m{1.3em}|>{\centering}m{8.8em}|}
  % \caption{秦王政}\\
  \toprule
  \SimHei \normalsize 年数 & \SimHei \scriptsize 公元 & \SimHei 大事件 \tabularnewline
  % \midrule
  \endfirsthead
  \toprule
  \SimHei \normalsize 年数 & \SimHei \scriptsize 公元 & \SimHei 大事件 \tabularnewline
  \midrule
  \endhead
  \midrule
  元年 & -368 & \begin{enumerate}
    \tiny
  \item 齊伐魏,取觀津。
  \item 趙侵齊,取長城。
  \end{enumerate} \tabularnewline\hline
  二年 & -367 & \tiny \kaiti 无记载 \tabularnewline\hline
  三年 & -366 & \begin{enumerate}
    \tiny
  \item 魏、韓會于宅陽。
  \item 秦敗魏師、韓師于洛陽。
  \end{enumerate} \tabularnewline\hline
  四年 & -365 & \begin{enumerate}
    \tiny
  \item 魏伐宋。
  \end{enumerate} \tabularnewline\hline
  五年 & -364 & \begin{enumerate}
    \tiny
  \item 秦獻公敗三晉之師于石門,斬首六萬。王賜以黼黻\footnote{黼者,刺繡爲斧形;黻者,刺繡爲兩「己」相背。孔穎達曰:白與黑謂之黼,黑與青謂之黻。}之服。
  \end{enumerate} \tabularnewline\hline
  六年 & -363 & \tiny \kaiti 无记载 \tabularnewline\hline
  七年 & -362 & \begin{enumerate}
    \tiny
  \item 魏敗韓師、趙師于澮。
  \item 秦、魏戰于少梁,魏師敗績;獲魏公孫痤。
  \item 衞聲公薨,子成侯速立。
  \item 燕桓公薨,子文公立。
  \item 秦獻公薨,子孝公立\footnote{这个时候,黄河和华山(陕西省华阴市南)以东,有六个强国(齐国【首府临淄·山东省淄博市东临淄镇】、韩国【首府新郑】、赵国【首府邯郸】、魏国【首府安邑】、燕国【首府蓟城】、楚王国【首都郢都·湖北省江陵县】)。淮河、泗水之间的小封国还有十余国。楚王国、魏国都跟秦国接壤。魏国为了防御秦国,从郑县(陕西省华县)沿着洛河(纵贯陕西省中部,在陕西省大荔县东南注入渭河),直到上郡(陕西省延安市),修筑长城。楚王国自汉中(陕西省汉中市),经巴城(重庆市),南到黔中(湖南省沅陵县),分别拥有广大领土,都把秦国看作落后地区的蛮族部落,中国境内(中原)各种国际会议,一向拒绝秦国参加。这种歧视使嬴渠梁深感羞辱,决心整顿内政,提高文化水准,追求强大。}。
  \end{enumerate} \tabularnewline\hline
  八年 & -361 & \begin{enumerate}
    \tiny
  \item 衞公孫鞅西入秦\footnote{秦国(首府栎阳【陕西省临潼县】)国君(二十五任孝公)嬴渠梁颁布招贤令:“从前我们的国君穆公(九任嬴任好),在岐山(陕西省岐山县东北)、雍县(陕西省凤翔县),励精图治。东方与晋国以黄河为界,协助他们,削平内乱。西方称霸夷狄,地广千里,天子封为盟主,封国国君们都来祝贺,开辟后世万年基业。不幸出现一连串不肖的国君,如厉公(十七任嬴刺)、躁公(十八任,名不详)、简公(二十一任嬴悼子)、出公(二十三任,名不详),国家动乱,无力顾及外事。于是,晋国占领我祖先的河西领土(陕西省合阳县、大荔县一带,魏长城至黄河之间),使我们丢丑。我父亲献公(二十四任嬴师隰)即位,把首府迁到栎阳(陕西省临潼县),准备东征,收复失地,复兴当年声势。可惜壮志未遂,即与世长辞,每一思及,万分痛心。现在我们公开征聘贤才,无论是本国人民,或外国宾客,只要有谋略可以使秦国强大,我愿任命他当高官,分封采邑。”卫国(首府濮阳【河南省濮阳市】)贵族公孙鞅,听到消息,西行投奔。公孙鞅既到秦国,通过宠臣景监的推荐,晋见嬴渠梁,提出富国强兵的具体方案,嬴渠梁喜出望外,要求公孙鞅负责执行。}。
  \end{enumerate} \tabularnewline\hline
  九年 & -360 & \tiny \kaiti 无记载 \tabularnewline\hline
  十年 & -359 & \begin{enumerate}
    \tiny
  \item 衞鞅于秦變法。
  \end{enumerate} \tabularnewline\hline
  十一年 & -358 & \tabularnewline\hline
  十二年 & -357 & \tabularnewline\hline
  十三年 & -356 & \tabularnewline\hline
  十四年 & -355 & \tabularnewline\hline
  十五年 & -354 & \tabularnewline\hline
  十六年 & -353 & \tabularnewline\hline
  十七年 & -352 & \tabularnewline\hline
  十八年 & -351 & \tabularnewline\hline
  十九年 & -350 & \tabularnewline\hline
  二十年 & -349 & \tabularnewline\hline
  二一年 & -348 & \tabularnewline\hline
  二二年 & -347 & \tabularnewline\hline
  二三年 & -346 & \tabularnewline\hline
  二四年 & -345 & \tabularnewline\hline
  二五年 & -344 & \tabularnewline\hline
  二六年 & -343 & \tabularnewline\hline
  二七年 & -342 & \tabularnewline\hline
  二八年 & -341 & \tabularnewline\hline
  二九年 & -340 & \tabularnewline\hline
  三十年 & -339 & \tabularnewline\hline
  三一年 & -338 & \tabularnewline\hline
  三二年 & -337 & \tabularnewline\hline
  三三年 & -336 & \tabularnewline\hline
  三四年 & -335 & \tabularnewline\hline
  三五年 & -334 & \tabularnewline\hline
  三六年 & -333 & \tabularnewline\hline
  三七年 & -332 & \tabularnewline\hline
  三八年 & -331 & \tabularnewline\hline
  三九年 & -330 & \tabularnewline\hline
  四十年 & -329 & \tabularnewline\hline
  四一年 & -328 & \tabularnewline\hline
  四二年 & -327 & \tabularnewline\hline
  四三年 & -326 & \tabularnewline\hline
  四四年 & -325 & \tabularnewline\hline
  四五年 & -324 & \tabularnewline\hline
  四六年 & -323 & \tabularnewline\hline
  四七年 & -322 & \tabularnewline\hline
  四八年 & -321 & \tabularnewline
  \bottomrule
\end{longtable}

%%% Local Variables:
%%% mode: latex
%%% TeX-engine: xetex
%%% TeX-master: "../../Main"
%%% End:
