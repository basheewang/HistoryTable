%% -*- coding: utf-8 -*-
%% Time-stamp: <Chen Wang: 2021-11-02 15:50:45>

\subsection{慎靓王定{\tiny(BC320-BC315)}}

\subsubsection{生平}

周慎靚王(?-前315年),姓姬,名定,又名順,中國東周君主,在位6年,為周显王之子。

當時戰國七雄為了壯大自己,各自找尋盟友,有不少弱小的國家聯合起來對抗一個強國,稱為“合縱”,以蘇秦為首;也有一些強國相互結合,攻打較弱的國家,史稱“連橫”,以張儀為首。前316年,秦軍攻滅了巴、蜀兩個小國,大量移民巴、蜀,佔有對抗長江中下游的楚國的戰略優勢。前315年,周王定病死,諡號為慎靚王。

\subsubsection{年表}

% \centering
\begin{longtable}{|>{\centering\scriptsize}m{2em}|>{\centering\scriptsize}m{1.3em}|>{\centering}m{8.8em}|}
  % \caption{秦王政}\\
  \toprule
  \SimHei \normalsize 年数 & \SimHei \scriptsize 公元 & \SimHei 大事件 \tabularnewline
  % \midrule
  \endfirsthead
  \toprule
  \SimHei \normalsize 年数 & \SimHei \scriptsize 公元 & \SimHei 大事件 \tabularnewline
  \midrule
  \endhead
  \midrule
  元年 & -320 & \tabularnewline\hline
  二年 & -319 & \tabularnewline\hline
  三年 & -318 & \tabularnewline\hline
  四年 & -317 & \tabularnewline\hline
  五年 & -316 & \tabularnewline\hline
  六年 & -315 & \tabularnewline
  \bottomrule
\end{longtable}

%%% Local Variables:
%%% mode: latex
%%% TeX-engine: xetex
%%% TeX-master: "../../Main"
%%% End:
