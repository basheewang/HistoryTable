%% -*- coding: utf-8 -*-
%% Time-stamp: <Chen Wang: 2021-11-02 15:50:57>

\subsection{赧王延{\tiny(BC314-BC256)}}

\subsubsection{生平}

周赧王(?-前256年),姓姬,名延,或名赧,皇甫謐说名诞。《竹書紀年》作周隱王,為周慎靚王之子。史文常作王赧,《史記》誤為諡號。据传,他即位于前314年。赧王在位59年,是周朝在位最长的君主,也是周朝的最後一位君主。

他在位时期,周王室的影响力仅限于王畿(现在的洛阳附近,当时是東周的首都)。早在他的祖父周显王在位期间,秦国的势力迅速膨胀,以西戎霸主自居。赧王五十九年,驾崩。是年,秦昭襄王迁九鼎,占王畿,灭周朝。

其父周慎靚王在位时住在东周国,受东周公保护供养,慎靚王在位六年后去世,东周公表示不愿继续供养周天子,于是赧王只得求助西周公接纳,遷至西周国居住。

周赧王五十六年(前259年),邯郸之战,秦军围攻邯郸,魏国和楚国都起兵相救,大破秦军。楚考烈王決定组织各国合纵。同时派人到赧王处请求赧王以天下共主名义下达组建联军的命令,赧王为此求助于西周公。西周公倾其国力组建起一支五六千人的小军队,与诸侯商定在伊阙会师。但最後只有楚國和燕國軍隊到達,合縱失敗。當時周赧王為了聯軍的開銷,向當地富人借貸,並答應班師之日以戰利品歸還。事後當地富人向周赧王討債,他只好躲到宮內一座高台,此台後被稱為「避债台」,成語「債台高築」因此而來。

周赧王五十九年,秦昭王令将军摎进攻西周国,西周武公奔往秦国谢罪投降。同年,周赧王与西周武公卒。西周百姓东逃,秦国取九鼎,迁西周文公于憚狐。七年后,秦国灭东周国。

\subsubsection{年表}

% \centering
\begin{longtable}{|>{\centering\scriptsize}m{2em}|>{\centering\scriptsize}m{1.3em}|>{\centering}m{8.8em}|}
  % \caption{秦王政}\\
  \toprule
  \SimHei \normalsize 年数 & \SimHei \scriptsize 公元 & \SimHei 大事件 \tabularnewline
  % \midrule
  \endfirsthead
  \toprule
  \SimHei \normalsize 年数 & \SimHei \scriptsize 公元 & \SimHei 大事件 \tabularnewline
  \midrule
  \endhead
  \midrule
  元年 & -314 & \tabularnewline\hline
  二年 & -313 & \tabularnewline\hline
  三年 & -312 & \tabularnewline\hline
  四年 & -311 & \tabularnewline\hline
  五年 & -310 & \tabularnewline\hline
  六年 & -309 & \tabularnewline\hline
  七年 & -308 & \tabularnewline\hline
  八年 & -307 & \tabularnewline\hline
  九年 & -306 & \tabularnewline\hline
  十年 & -305 & \tabularnewline\hline
  十一年 & -304 & \tabularnewline\hline
  十二年 & -303 & \tabularnewline\hline
  十三年 & -302 & \tabularnewline\hline
  十四年 & -301 & \tabularnewline\hline
  十五年 & -300 & \tabularnewline\hline
  十六年 & -299 & \tabularnewline\hline
  十七年 & -298 & \tabularnewline\hline
  十八年 & -297 & \tabularnewline\hline
  十九年 & -296 & \tabularnewline\hline
  二十年 & -295 & \tabularnewline\hline
  二一年 & -294 & \tabularnewline\hline
  二二年 & -293 & \tabularnewline\hline
  二三年 & -292 & \tabularnewline\hline
  二四年 & -291 & \tabularnewline\hline
  二五年 & -290 & \tabularnewline\hline
  二六年 & -289 & \tabularnewline\hline
  二七年 & -288 & \tabularnewline\hline
  二八年 & -287 & \tabularnewline\hline
  二九年 & -286 & \tabularnewline\hline
  三十年 & -285 & \tabularnewline\hline
  三一年 & -284 & \tabularnewline\hline
  三二年 & -283 & \tabularnewline\hline
  三三年 & -282 & \tabularnewline\hline
  三四年 & -281 & \tabularnewline\hline
  三五年 & -280 & \tabularnewline\hline
  三六年 & -279 & \tabularnewline\hline
  三七年 & -278 & \tabularnewline\hline
  三八年 & -277 & \tabularnewline\hline
  三九年 & -276 & \tabularnewline\hline
  四十年 & -275 & \tabularnewline\hline
  四一年 & -274 & \tabularnewline\hline
  四二年 & -273 & \tabularnewline\hline
  四三年 & -272 & \tabularnewline\hline
  四四年 & -271 & \tabularnewline\hline
  四五年 & -270 & \tabularnewline\hline
  四六年 & -269 & \tabularnewline\hline
  四七年 & -268 & \tabularnewline\hline
  四八年 & -267 & \tabularnewline\hline
  四九年 & -266 & \tabularnewline\hline
  五十年 & -265 & \tabularnewline\hline
  五一年 & -264 & \tabularnewline\hline
  五二年 & -263 & \tabularnewline\hline
  五三年 & -262 & \tabularnewline\hline
  五四年 & -261 & \tabularnewline\hline
  五五年 & -260 & \tabularnewline\hline
  五六年 & -259 & \tabularnewline\hline
  五七年 & -258 & \tabularnewline\hline
  五八年 & -257 & \tabularnewline\hline
  五九年 & -256 & \tabularnewline
  \bottomrule
\end{longtable}

%%% Local Variables:
%%% mode: latex
%%% TeX-engine: xetex
%%% TeX-master: "../../Main"
%%% End:
