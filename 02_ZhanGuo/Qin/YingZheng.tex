%% -*- coding: utf-8 -*-
%% Time-stamp: <Chen Wang: 2018-07-10 17:31:00>

\subsection{赢政{\tiny(BC246-BC221)}}


% \centering
\begin{longtable}{|>{\centering\scriptsize}m{2em}|>{\centering\scriptsize}m{1.3em}|>{\centering}m{8.8em}|}
  % \caption{秦王政}\\
  \toprule
  \SimHei \normalsize 年数 & \SimHei \scriptsize 公元 & \SimHei 大事件 \tabularnewline
  % \midrule
  \endfirsthead
  \toprule
  \SimHei \normalsize 年数 & \SimHei \scriptsize 公元 & \SimHei 大事件 \tabularnewline
  \midrule
  \endhead
  \midrule
  元年 & -246 & \begin{enumerate}
    \tiny
  \item 韩国水工郑国开始建造郑国渠,约十年后完工。
  \item 秦晋阳反,蒙骜击平之。
  \end{enumerate} \tabularnewline\hline
  二年 & -245 & \begin{enumerate}
    \tiny
  \item 秦麃公将卒攻卷,斩首三万。
  \item 赵以廉颇为假相国,伐魏,取繁阳。赵孝成王薨,子赵悼襄王偃立。
  \end{enumerate} \tabularnewline\hline
  三年 & -244 & \begin{enumerate}
    \tiny
  \item 秦蒙骜攻韩,取12城。
  \end{enumerate} \tabularnewline\hline
  四年 & -243 & \begin{enumerate}
    \tiny
  \item 春,秦蒙骜伐魏,取旸、有诡。三月,军罢。
  \item 秦质子归自赵;赵太子出归国。
  \item 七月,秦国蝗,疫。令百姓纳粟千石,拜爵一级。
  \item 魏安釐王薨,子魏景湣王增立。
  \item 赵悼襄王以李牧为将,伐燕,取武遂、方城。
  \item 逝世:魏安釐王、信陵君魏无忌。
  \end{enumerate} \tabularnewline\hline
  五年 & -242 & \begin{enumerate}
    \tiny
  \item 秦蒙骜伐魏,取酸枣、燕、虚、长平、雍丘、山阳等二十城;初置东郡。
  \item 燕王使剧辛将而伐赵。
  \end{enumerate} \tabularnewline\hline
  六年 & -241 & \begin{enumerate}
    \tiny
  \item 函谷关之战。
  \item 秦拔魏朝歌,及卫濮阳。
  \end{enumerate} \tabularnewline\hline
  七年 & -240 & \begin{enumerate}
    \tiny
  \item 秦置濮阳县,属东郡,并定其为东郡治所。
  \item 逝世:蒙骜、邹衍。
  \item 出生:陆贾。
  \item 天象:彗星光出东方,见北方,五月见西方。
  \end{enumerate} \tabularnewline\hline
  八年 & -239 & \begin{enumerate}
    \tiny
  \item 北扶余王国建立。
  \item 嫪毐封长信侯。
  \item 魏与赵邺。
  \item 文学:吕氏春秋编成。
  \item 逝世:长安君成蟜、韩桓惠王。
  \end{enumerate} \tabularnewline\hline
  九年 & -238 & \begin{enumerate}
    \tiny
  \item 嬴政亲政。
  \item 嫪毐叛乱,被秦王政夷灭三族。
  \item 秦伐魏,取垣、浦。
  \item 逝世:荀子、楚春申君黄歇、楚考烈王。
  \end{enumerate} \tabularnewline\hline
  十年 & -237 & \begin{enumerate}
    \tiny
  \item 齐王建拜会秦王政。
  \item 吕不韦免相。
  \item 秦王政下令驱除异邦客卿,李斯上书劝秦始皇收回逐客令。
  \end{enumerate} \tabularnewline\hline
  十一年 & -236 & \begin{enumerate}
    \tiny
  \item 郑国渠建成。
  \item 秦攻赵,赵攻燕\footnote{公元前236年,秦乘攻取赵的阏与、橑阳、邺、安阳等城,后又大举攻赵,遭到顽强抵抗。赵虽两次打败秦军,但兵力耗损殆尽。秦国西出太行山,突袭赵国邯郸拉开了统一战的的序幕。 赵国和燕国激战正酣,他想将秦国造成的领土损失在燕国身上补回来。这时秦国乘虚而入。赵国急忙命令大将李牧率军南下应敌。}。
  \end{enumerate} \tabularnewline\hline
  十二年 & -235 & \begin{enumerate}
    \tiny
  \item 秦攻楚国\footnote{秦继攻赵之后,即命辛梧率四郡兵,会同魏国,对楚国发起攻击。}。
  \item 吕不韦卒\footnote{因嫪毐集团叛乱事受牵连,被免除相邦职务,出居河南封地。不久,秦王政下令将其流放至蜀地(今四川),不韦忧惧交加,于是在三川郡(今河南洛阳)自鸩而亡。}。
  \end{enumerate} \tabularnewline\hline
  十三年 & -234 & \begin{enumerate}
    \tiny
  \item 秦攻赵\footnote{公元前234年,秦再度向赵南部进攻。桓龁避开正面渡河,改由漳河下游渡河迂回赵扈辄军的侧后,攻击邯郸东南的平阳。两军于平阳展开交战,赵军被击破,被斩10万人,赵将扈辄阵亡。赵王启用北部边疆名将李牧为统帅。李牧军曾歼灭匈奴入侵军10万之众,威震边疆,战斗力最强。李牧率军回赵,立即同秦桓龁军交战于宜安肥下地区,给秦军几乎全军覆灭的沉重打击,只有统帅桓龁带领少数护卫突围逃走。}。
  \item 韩非\footnote{韩非(约前281年-前233年),生活于战国末期时期的韩国(今属河南省新郑市)的思想家,为中国古代著名法家思想的代表人物,认为应该要“法”、“术”、“势”三者并重,是法家的集大成者。韩非出身韩国公族,与李斯均是荀子学生,后因其学识渊博,被秦始皇召唤入秦,正欲重用,却遭到妒忌的同窗李斯害死,在韩非死后,秦始皇在韩非的思想指引下,完成统一六国的帝业。韩非其学出于荀子,源于儒家,而成为法家,又推究老子思想,归本于道家。司马迁指出韩非喜好“刑名法术”且归本于道家的“黄老之学”,一套由“道”、“法”共同完善的政治统治理论。}作为韩国的使臣来到秦国,上书秦王,劝其先伐赵而缓伐韩。
  \end{enumerate} \tabularnewline\hline
  十四年 & -233 & \begin{enumerate}
    \tiny
  \item 韩非子卒。
  \item 燕抗秦\footnote{公元前233年,秦将樊於期叛逃至燕国后,太子丹的师傅鞠武害怕秦国以此借口攻燕,便策划送樊於期到头曼那里,利用熟悉秦国虚实的樊於期结连匈奴攻秦。可惜性急的太子丹等不得这种长远之计凑效,他决定派出荆轲刺杀自己的童年好友嬴政,为了能够解除嬴政的戒备,荆轲提出要携带两样礼物:樊於期的人头和燕国督亢地图(割地求和)。嬴政在逃过刺杀威胁后更以迅雷不及掩耳之势统一六国。}。
  \item 赵将李牧大败秦将桓齮\footnote{桓齮(yǐ)(?-前227年),战国末年秦国将军。杨宽的《战国史》认为桓齮就是樊於期。始皇十一年(前237年),桓齮与王翦和杨端和攻赵,取邺九城。秦始皇十四年,也就是赵王迁二年(前233年),桓齮从上党越太行山进攻赵的赤丽、宜安(石家庄东南),与赵将李牧战于肥下(宜安东北),为李牧所败,逃至燕国(《战国策》说是战败被杀,《资治通鉴》记载“秦师败绩,桓齮奔还”)后无相关记载。}于肥。
  \end{enumerate} \tabularnewline\hline
  十五年 & -232 & \begin{enumerate}
    \tiny
  \item 项羽出生。
  \item 太子丹回燕。
  \end{enumerate} \tabularnewline\hline
  十六年 & -231 & \begin{enumerate}
    \tiny
  \item 秦攻韩。
  \item 魏献丽邑。
  \item 赵国地震。
  \item 韩信出生。
  \end{enumerate} \tabularnewline\hline
  十七年 & -230 & \begin{enumerate}
    \tiny
  \item 韩国灭亡。
  \end{enumerate} \tabularnewline\hline
  十八年 & -229 & \begin{enumerate}
    \tiny
  \item 秦攻赵国。
  \item 李牧被杀。
  \end{enumerate} \tabularnewline\hline
  十九年 & -228 & \begin{enumerate}
    \tiny
  \item 秦破赵得和氏璧。
  \item 赵国灭亡。
  \end{enumerate} \tabularnewline\hline
  二十年 & -227 & \begin{enumerate}
    \tiny
  \item 荆轲刺秦王。
  \item 王翦、辛胜在易水西败燕、代联军。
  \end{enumerate} \tabularnewline\hline
  二一年 & -226 & \begin{enumerate}
    \tiny
  \item 秦军攻燕都。
  \item 秦攻蓟城。
  \end{enumerate} \tabularnewline\hline
  二二年 & -225 & \begin{enumerate}
    \tiny
  \item 魏国灭亡。
  \item 秦置砀郡,立浚仪(大梁)、启封两县。
  \end{enumerate} \tabularnewline\hline
  二三年 & -224 & \begin{enumerate}
    \tiny
  \item 秦楚之战。
  \item 秦置修武县。
  \end{enumerate} \tabularnewline\hline
  二四年 & -223 & \begin{enumerate}
    \tiny
  \item 楚将项燕自杀。
  \item 秦灭楚。
  \end{enumerate} \tabularnewline\hline
  二五年 & -222 & \begin{enumerate}
    \tiny
  \item 秦灭代。
  \item 秦灭燕。
  \end{enumerate} \tabularnewline
  \bottomrule
\end{longtable}

%%% Local Variables:
%%% mode: latex
%%% TeX-engine: xetex
%%% TeX-master: "../../Main"
%%% End:
