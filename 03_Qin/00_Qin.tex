%% -*- coding: utf-8 -*-
%% Time-stamp: <Chen Wang: 2019-12-17 12:03:45>

\chapter{秦\tiny(BC221-BC207)}

\section{简介}

秦朝(上古漢語IPA读音:/dzin/,前221年-前207年),是中國歷史上首个大一統中央集权的帝國。秦朝皇室為嬴姓,所以史書又称嬴秦。秦朝源自周朝諸侯國秦國。公元前905年,秦非子因善于养马,得到周孝王的赏识,受爵获封秦地,成为秦国始封君,建立秦國,号称秦嬴。前770年,秦襄公在東周周平王東遷時有功,受封於关中平原,成为一方诸侯。战国時期,秦國根据礼记总结地“今天下车同轨,书同文,行同輪”的社会现状,推行车辆统一道路,书籍统一文字,行为统一伦理,并在政治、軍事、經濟、交通方面,实施商鞅变法,成為天下第一強國。前230年至前221年,到秦王嬴政陸續攻滅其他六個主要諸侯國,一統中原,史称秦朝。

秦王政建立秦朝後自稱「始皇帝」(即秦始皇),從此中國有了皇帝的稱號,並且發起一系列的改革以鞏固帝國。而後南征百越、北伐匈奴,使得領土約等同中國本土。疆域為東起遼東,西至臨洮,北抵長城,南達象郡(今越南北部)。雖然秦朝外表十分強盛,但由於秦始皇集權、過度發展、嚴重勞役百姓,所以秦朝之統治不免帶有苛急、暴虐之特點,讓天下百姓飽受苛政之苦而想要叛變。秦二世繼位後,秦廷被掌權的趙高掌控而混亂不堪。此時秦末民變爆發,六國有力的軍人各自復國,雖然秦將章邯努力平亂,但秦將王離於鉅鹿之戰被楚將項羽擊敗,秦軍主力投降。前207年十月,新任秦王子嬰於咸陽向劉邦投降,後項羽率軍抵達關中,殺死秦王子嬰,焚燒咸陽宮,秦朝於秦始皇死後3年滅亡。而後至西漢統一全國之期間稱秦漢之際,又稱西楚時期。西楚霸王項羽重建封國制體系,分封十八諸侯。但項羽分封不公,劉邦、田榮等諸侯繼而起兵反抗,演變成楚漢戰爭。前202年漢王劉邦擊滅項羽,受諸侯擁戴為皇帝,開啟漢朝。

秦朝為了鞏固遼闊但各地文化不一的帝國,在政治、軍事、經濟、交通方面,都推行政策與改革。政治制度建立有皇帝制度、三公九卿制與郡縣制為基礎的中央集權制,取代过去不同诸侯豪门的爵位制度。法律基本延續秦國法律,增加了皇帝握有生殺大權,丞相僅僅是輔佐。防务方面,興建長城以鞏固北方,毀六國長城與城墩等防禦工事,沒收天下百姓武器鑄成十二金人,先後五次東巡以安定關東各地區。交通方面,興建馳道與靈渠等交通要道以便軍隊前往各地平亂,也有利各地區運輸物資。經濟方面,秦朝延續商鞅變法的政策,推行重農輕商,鼓勵農民增產糧食,甚至有機會獲爵位。工商業方面統一貨幣(秦半兩)與度量衡,實行鹽鐵專賣,但不完全禁止商業,也鼓勵如經營畜牧業的烏氏倮與丹砂的巴地寡婦清等商人。

在文化思想方面,秦朝的政治思想以法家為尊,推行融合「法、術、勢」的君主專制,另有發揚陰陽家的五德始終說以神化皇帝。宗教以傳統崇拜上帝、祖先、山神、河神等眾神以及巫術、占卜與占星等,而神仙方術之術受秦始皇所迷信。為了箝制人民叛亂思想、统一中央集权思想、報復欺騙秦始皇的方士們,先後發生焚書坑儒事件,這是先秦文化與諸子百家思想的一大浩劫。

秦朝的建立使中國由一個抽象的地理名稱轉為具體的大一統帝國,為融合中原文化、秦文化、荊楚文化與齊魯文化等文化打下基礎,使得「秦」成為中國文化的象徵之一。其遺留下來的驪山秦始皇陵與秦始皇兵馬俑也受後世史學家注目,其中兵馬俑被列入《世界遺產名錄》。

秦朝的前身為周朝諸侯國秦國,位於關中平原。春秋时期的諸侯國因為種種原因,時常發生強國兼併弱國之事。到了戰國時期,各國變法圖強,競爭更加激烈。其中秦國在商鞅變法後由弱轉強,歷經多次戰爭而擊敗各國。在長平之戰後,幾乎占有一半的天下。然而長平之戰的主將白起在前257年被秦昭襄王賜死,雄才大略的秦昭襄王於前251年去世,繼位的秦孝文王、秦莊襄王也相繼在三年內去世。前247年幼主秦王政繼位,朝政大權掌握在秦太后與她的舊情人相邦呂不韋手中。呂不韋領有秦國大權,他招攬食客,編撰《呂氏春秋》,並於前249年率軍滅東周國(而東周早於前256年亡於秦)。

此時六國以趙楚尚能抗衡秦國。趙國士卒精銳,有良將李牧,楚國領土與士兵尚多。為了抵制秦國侵略,楚春申君於前241年策劃,聯合楚、趙、魏、韓、燕聯軍直逼秦函谷關,但被秦國擊敗,楚國遷都壽春。秦國趁機猛攻魏國,魏國形勢危險,最後秦國因為國內發生大事而暫緩攻擊。起因是秦王政日漸長大,呂不韋恐懼繼續與秦太后幽會會出事,便改以舍人嫪毐入宮服侍太后。嫪毐得寵,太后為他生兩子,被封為長信侯,並且專決國政。前239年秦王政之弟成蟜於伐趙時,就因畏懼嫪毐而降趙。前238年秦王政發現此事,將意圖作亂反抗的嫪毐處死,並且幽禁太后、驅除呂不韋入蜀(呂不韋於途中去世)。秦王政掌權之後,聽從宗室大臣的建議,驅除呂不韋的食客(大多是三晉人)。但客卿李斯上書《諫逐客書》,說明外卿對秦國的貢獻。秦王政再度重用李斯,並且招攬尉繚與韓非等外國人。尉繚提出建議,以李斯派辯士、間諜賄絡六國名人政要,以為秦用。如果不受用,則暗殺之。這個辦法使六國離心離德,再配合軍事行動,使六國在十年內亡於秦國。而韓非提倡採用法家「法、術、勢」三派論點為一體來治國,受秦王政的重視,成為秦廷施政的原則。然而韓非最後被李斯害死,未能執政秦廷。

秦王政在李斯、尉繚等人的協助下制定「籠絡燕齊,穩住魏楚,消滅韓趙;遠交近攻,逐個擊破」的方針,展開秦滅六國之戰。首先是秦國東鄰的三晉,三晉以韓國最弱,趙國最強。前230年秦王政趁趙國飢荒,派内史腾率秦軍攻滅韓國,俘韓王安,韓國亡。接著是趙國,趙國名將李牧為秦之勁敵,曾於前233年擊敗秦國,使秦將桓齮投奔燕國。前229年秦王政派秦將王翦攻趙國,賄絡趙王倿臣郭開,以反間計讓趙王遷殺李牧、廢司馬尚。隔年邯鄲陷,俘趙王遷,趙國亡,趙嘉率宗族逃到代城稱代王,即代王嘉。最後為魏國,然而前227年燕國太子丹派荊軻假借獻樊於期(疑為桓齮)頭顱與燕地地圖而刺殺秦王政,最後失敗。為此,秦王政馬上派王翦領兵攻燕,於隔年破燕都薊,燕王喜退守遼東,殺太子丹以求和。

秦王政接著於前225年同時伐魏楚:命秦將王賁率十萬大軍伐魏、命李信、蒙武率二十萬大軍伐楚。王賁率十萬大軍於同年包圍魏都大梁,引黃河、大溝水灌城而破,魏王假投降,魏國亡。李信軍方面,因為據守郢陳的昌平君叛秦歸楚,李信軍只好回頭平亂,在路途上被楚國名將項燕襲擊而大敗。秦王政只好請老將王翦率領六十萬大軍,再度伐楚。王翦入楚境後屯兵練武、堅壁不戰,與楚軍對峙。前224年項燕因糧草不足、楚軍鬥志渙散,只好率軍南撤。王翦趁機追擊,渡淮水,攻陷楚都壽春,俘虜楚王負芻。項燕退守長江以南,立昌平君為楚王。223年王翦造戰船,渡江擊潰楚軍,昌平君與項燕身死,楚國亡。隔年王翦攻百越,降越君。為了平定燕趙之地,前222年再派王賁攻滅東北的遼東燕王喜與代城代王嘉,燕國亡。此時六國只剩齊國,由於齊相國后勝被秦國收買,當秦國滅他國時齊國只冷眼旁觀,等到五國均滅後齊王建才慌張派兵守齊國西部。而王賁軍在平定燕趙後,於隔年由燕地奇襲南下,攻下齊都臨淄,齊王建投降,齊國亡。齐人十分怨恨齐王建,但已於事無補。秦王政已平定六國,一統天下,建秦朝。

秦朝的建立,讓中國地區從一個抽象的地理概念變成一個具體的大一統帝國。秦王政自認其功業「功過三皇,德兼五帝」,自號『始皇帝』,史稱秦始皇,連鬼神也該對他敬畏。他認為諡號乃是「子論父,臣論君」,大為不妥而廢除之。改用世數尊號,宣布子孫稱二世、三世,以至萬世,代代承襲。其他像是自稱「朕」、「命」改為「制」、「令」改為「詔」等皆源自秦始皇開始。

秦始皇日理萬機,實行一系列政治措施以加強中央集權。皇帝擁有決策權,大臣只有議事權。實行三公九卿制,「三公」即是丞相、御史大夫、太尉,分掌行政、監察和軍事。在地方實行郡縣制,廢除周朝的封建制,地方首長均可輪替,其下還有鄉里制度。為了鞏固帝國,統一度量衡、貨幣、文字與車軌。戰國各國使用的刀貨、貝貨、鏟錢與圓錢,至此統一成秦半兩。由於各國文字不一,李斯修改秦國籀文,建立小篆並通行天下,而後程邈簡化小篆為隸書,流行於漢朝。為了「銷兵」以避免地方叛亂,沒收民間所有武器,於咸陽鑄大鐘與十二金人。強遷全國豪族富戶至首都咸陽,以便集中管理。仿照六國宮室規格,大肆營建;以咸陽為中心,沿涇渭二水沿岸興建離宮別館。為了讓軍隊便利前往帝國各地,於前220年建立馳道,以咸陽為中心通往四方。

文教思想方面,秦始皇兼用儒生與法吏,祿養七十位博士,已備顧問。但秦始皇不許任何人批評他的政治措施。前213年,博士淳于越向他建議恢復封國制,遭到李斯反對。李斯還認為儒生非議朝廷法令,恐怕會擾亂人心,所以進一步要求燒毀政治性書籍,只留醫藥、吏書、卜筮與種樹等書,史稱「焚書」。秦始皇有請方士於海外求神仙與長生不老,但都未果。隔年,由於方士侯生與盧生二人批評秦始皇的行為,而後逃去。秦始皇大怒,審問咸陽所有儒生(秦朝儒生大多兼方士),但都不願承罪,於是都被活埋,史稱「坑儒」。由此可見,晚年的秦始皇想藉焚書坑儒來加強思想的控制。

對外方面,秦朝北方有匈奴、東胡與月氏等民族,其中匈奴與東胡對秦朝威脅極大。秦始皇三十三年(前214年),因方士所言「亡秦者胡也」的讖語,秦始皇命蒙恬率三十萬大軍北伐匈奴,占領河套地區,即九原郡,並且移民建長城。為了防範匈奴入侵,在前214年至前213年間,興建由九原通往咸陽西北雲揚(今陝西淳化)的「直道」;連接燕、趙與秦長城,完成西起臨洮、東至遼東的萬里長城。而南方以討伐百越為主,當時百越族在浙江南部有東甌國、福建有閩越國、廣東與越南有古南越國、廣西有西甌國與雒越。秦滅六國期間,降伏東甌、閩越以設置郡縣。統一全國後,派兵討伐南越西甌,即秦攻百越之戰。他先後派屠睢(前221年)、任囂與趙佗(前214年)南征。並且命史祿於湘江和珠江建立靈渠,以便運糧給南征軍。平定百越後,移民罪犯五十萬人至南方,以控制嶺南地區。

秦始皇在統一六國之後於渭水畔興建阿房宮(未完工)、於驪山修建秦始皇陵(未完工)。並且四處封禪、五次出巡。在鄒嶧山(在今山東鄒城)、泰山、芝罘山、琅邪、會稽山、碣石(在今河北昌黎)等地留下刻石,以表彰自己的功德。又依古代帝王慣例,於泰山祭告天地,以表示受命於天,謂之封禪。前210年最後一次東巡,他帶幼子胡亥同行,南巡雲夢、長江,途中於琅琊(今山東青岛)命徐福東尋仙丹,在返途於平原(今山東平原)生病,最後於沙丘(今河北平鄉)去世。

秦始皇死後,宦官趙高與李斯密不發喪,密謀奪權。趙高為胡亥的師傅,與秦始皇長子扶蘇和蒙恬之弟蒙毅不合,而李斯也對蒙氏兄弟不滿。他們擁立胡亥為太子,假造詔令以賜死在九原督軍的扶蘇,史稱沙丘之變。他們等到東巡大軍返抵咸陽才發喪,胡亥也繼位為帝,即秦二世。秦二世受趙高擺佈,他賜死蒙氏兄弟,並殺害兄弟姐妹數十人。蒙恬的九原軍也由王離接管。秦二世縱情於聲色犬馬,圈養無數狗馬禽獸;徵招罪犯與百姓,續建阿房宮與秦始皇陵。秦朝政治趨向混亂,趙高把持朝廷,蒙蔽皇帝,控制群臣言論,指鹿為馬。前208年,趙高陷害李斯,腰斬於咸陽,夷三族。另外役民過甚,當時力役三十倍於古以及田賦二十倍於古。舊六國百姓不願受嚴刻的秦法箝制與無數的勞役折磨,於同年爆發秦末民變。

前209年陳勝、吳廣運送士兵逾期,為避免全體處斬,就在泗水郡的大澤鄉(今安徽宿縣)「揭竿起義」。大軍攻下陳(今河南淮陽)後,陳勝稱王,建立張楚。他們四處擴張,沒多久秦朝關東地區陷入動亂:北路軍武臣、張耳、陳餘北略趙地,周市略魏地。西征軍吳廣攻榮陽(今河南榮陽)、周文攻函谷關、宋留攻武關。武臣占領邯鄲後自立趙王,以張耳陳餘為輔佐。他派韓廣略燕地,而後韓廣於燕地自稱燕王。周市與復興齊國的齊王田儋作戰失敗後,擁立魏咎為魏王。而周文也越函古關逼近咸陽。此時關中空虛,擁兵50萬秦軍的南海郡趙佗不北上救援,自立南越武王,建南越國。秦二世聽從章邯建議,緊徵在秦始皇陵工作的刑徒與奴隸的兒子為兵,以章邯率領應戰,最後擊潰周文軍。而吳廣的部下田臧殺吳廣,並率軍與東進的章邯軍作戰,但也被擊潰。前208年初,章邯趁勝東進,擊潰陳勝軍於陳以西,陳勝被殺,宋留軍也投降。張楚餘部擁立景駒為楚王。

然而,在陳勝吳廣起事後的兩個月,江東會稽郡吳縣(今江蘇吳縣)的項梁、項羽舉兵起事。項梁軍精實,於前208年初渡長江北上,接納陳嬰、英布等人,軍隊逐漸擴大,最後連張楚景駒軍與沛縣(今江蘇沛縣)的劉邦軍也被吸收。項梁聽從范增建議,立楚懷王之孫熊心為楚懷王,定都盱眙,以激發楚人鬥志。聽從張良建議,立韓王後裔韓成為韓王。韓王成與張良游擊於潁川。此時章邯軍攻魏地,擊潰魏王咎、齊王儋與楚將項它,魏齊二王均死。項梁立魏王咎弟魏豹為魏王。而混亂的齊地最後被齊王儋之弟田榮領有,立齊王儋之子田市為王,齊王田假(齊王建之弟)投奔項梁。項梁東擊秦軍,接連勝利。九月秦將王離率九原軍支援章邯,章邯於於定陶(今山東定陶)擊潰楚軍,項梁去世。章邯認為楚軍不足憂,率軍北上擊趙。趙王武臣因內亂而死,張耳陳餘擁立趙國後裔趙王歇,定都信都(今河北冀縣)。章邯擊潰趙軍,命王離圍困趙王歇於鉅鹿(今河北平鄉)。楚懷王命宋義、項羽與范增北上救趙,命劉邦西進伐秦,並言道先入關中者為「關中王」。前207年宋義駐軍安陽(今山東曹縣)不前。項羽不服,殺宋義奪軍權,率領楚軍渡河,破釜沉舟。項羽軍奇襲鉅鹿,擊潰秦軍主力,王離被俘虜,而蘇角與涉間皆死,史稱鉅鹿之戰。此後秦朝無兵平亂,諸侯拜服項羽,奉為聯軍領袖。而章邯因為戰敗,被趙高威脅,最後同司馬欣、董翳投降楚軍,被封為雍王。

而劉邦緩慢地迂迴西進,收編散落的陳勝、項梁軍。前207年與張良會合於洛陽以東,然後奪南陽、定武關,入關中。此時趙高擔任丞相,因畏懼劉邦,命閻樂殺秦二世于望夷宫,改立子婴為秦王。同時,劉邦軍於藍田擊潰秦軍,駐軍霸上。秦王子婴發动政变,誅殺趙高,並於軹道向刘邦投降,秦朝亡。劉邦入主關中後,廢除嚴刻的秦法,推行約法三章,受秦人擁戴。並聽張良建議,退出秦宮,靜等諸侯共同處理財富。項羽得知劉邦入關,率楚軍、諸侯軍與秦降軍急行。途中得知秦降軍有怨言,於新安(今河南新安)活埋大量降軍。劉邦擔憂項羽命雍王章邯主管關中,就封鎖函谷關。項羽大怒,攻下函谷關後進駐鴻門(陝西臨潼東)。劉邦前往開鴻門宴,雙方和解。項羽入咸陽後,殺子嬰,燒毀咸陽宮,掠奪財寶婦女東歸。此時的項羽,已然成為天下的主宰者。

秦亡後到西漢統一天下之間的歷史,史學家稱為「秦漢之際」或西楚時期。項羽成為天下的主宰者後,採取封建制,重新分封領土。他對楚懷王不滿,前203年尊為楚義帝,但只給湘江上游的小地,並於楚義帝就國途中殺掉。項羽自立為西楚霸王,領有梁楚九郡,都彭城。他分封十八諸侯,因為鄙視諸國原諸侯滅秦無功,而遷封或削弱領土,分裂諸國如齊分為三、秦分為四或楚分為四等等,並且大封諸國有功將領為王,這些措施讓部分諸侯與將領不滿,為項羽製造大量敵人。

當諸侯陸續就國時,項羽以韓王成無功、張良又輔佐劉邦為由廢為侯。齊相田榮有再造齊國之功,但因田假事件與項氏不合,也沒有封地。前204年田榮反叛項羽,獲得彭越軍的協助,率軍統一三齊,稱齊王。關中方面,劉邦被改封於漢中而非關中,章邯等三位秦降將入主關中,這讓劉邦與秦人十分不滿(秦人怪罪三位降將讓數十萬秦軍死在關外,又將災禍帶至關中,不適合在關中稱王)。劉邦在就國途中燒毀棧道,宣示無歸心。趁項羽無心關注關中之際,於前206年派韓信暗渡陳倉,包圍雍國首都廢丘(今陝西興平),占領三秦領地(然而雍國首都要到隔年才攻下)。同年,原燕王韓廣不願就國遼東,被燕王臧荼所殺,燕國統一。趙將陳餘之功等同張耳,但只封侯而非王,這讓陳餘不滿。前205年陳餘獲田榮之助,趕走張耳,迎趙歇為趙王,自封為代王並輔佐趙王歇,命夏說為代相,趙國統一。漢王劉邦統一關中後向外發展,河南王申陽降漢,立韓王信於韓地,並且遷都櫟陽(陝西臨潼東)。面對各地叛亂,項羽北上擊潰齊王田榮於城陽(山東莒縣)。田榮後死,楚軍持續北上沿路燒掠,齊人擁立田榮弟田橫續與西楚作戰。劉邦於前205年三月降西魏王魏豹、俘虜殷王司馬卬。而後以為楚義帝發喪為由,號召諸侯東征。四月,劉邦率諸侯聯軍五十六萬人攻入西楚首都彭城。項羽急率軍自齊返楚,奇襲聯軍,滅二十萬人,諸侯又叛漢附楚。而田橫也趁機立田榮子田廣為齊王。

戰事轉入楚漢對峙。劉邦收集散兵,固守榮陽、成皋(今河南汜水),賴蕭何供給關中的人力與物資,以抵禦西楚項羽的攻勢。面對僵局,劉邦派韓信、張耳側擊河北地區,九月韓信攻滅親附西楚的魏王豹於安邑(今山西夏縣),魏國亡。接著擒代相夏說於閼與(今山西和順)。隔年,韓信於井陘之戰擊潰趙王歇與陳餘的趙軍,趙國亡。同時間,劉邦派隨何拉攏九江王英布叛變。而後項羽派龍且擊潰,英布投奔劉邦。項羽還中反間計,猜忌范增,讓范增憤而告退。然而西楚對漢防線的攻勢仍然強烈。五月楚軍攻下榮陽、成皋,劉邦北逃至趙地,奪韓信軍權。劉邦命韓信東征齊國,一面率軍南下與楚作戰,又封彭越為魏相,負責截斷楚軍補給線。彭越攻下睢陽、外黃等城,項羽不得以返梁地平亂。前203年年初,項羽暫時平定梁地,但成皋也被劉邦占領。同年,韓信聽李左車建議,降伏燕國。同時間劉邦派酈食其說服齊王投降。韓信聽蒯通建議,趁機奇襲齊國首都臨淄。齊王田廣殺酈食其,東奔高密,向楚求救。而項羽派龍且北援,韓信於濰水之戰擊潰齊楚聯軍,齊國亡。韓信坐鎮齊國,聽蒯通建議自封齊王。十月,項羽北有韓信,內有彭越,西有劉邦,局勢轉壞。項羽只好與劉邦和談,以鴻溝為界,約成後項羽領兵東歸。

而劉邦聽張良、陳平建議,率軍追擊楚軍於固陵(今河南太康南),並以封王裂土為誘因招韓信、彭越圍攻項羽。前202年年初,劉邦聯軍包圍楚軍於垓下(今安徽靈壁),史稱垓下之戰。項羽率軍突圍,南至長江西岸的烏江(今安徽和縣),從人將盡,項羽就自刎而死。而後,劉邦罷韓信兵權,令灌嬰率軍渡江定江東。臨江王共尉不願投降,劉邦派盧綰、劉賈、靳歙先後攻打而下。前202年二月漢王劉邦受諸侯擁戴為漢帝,建國西漢,開啟漢朝,天下再歸一統。

秦朝的疆域於前214年達到最大,約等同近代中國的關內地區。其疆域西起臨洮、東至遼東,北疆為萬里長城,西南至巴蜀地,正南至兩廣、越南北部,東南至東海、台灣海峽,正東至海,這是中原王朝的傳統疆域範圍。所有的邊疆地區都站有險要的地形,構成天然的國防線。當時的氣候與現在相比,比較溫暖潮濕,森林草原的覆蓋率也比較高。使得淮水以南的江南地區都屬於潮濕的熱帶叢林氣候,關中地區的森林也十分茂密,畏寒的竹類在當時也分佈在黃河渭水一帶。而長城一線約等同十五吋等雨線,成為遊牧和農耕的邊界,長城的北疆意義對後世影響甚為巨大。

關於疆域的變遷,秦朝的前身為秦國,其疆域在春秋時期領有關中地區,在戰國時期大幅擴張。到秦昭襄王時,秦國領有關中、漢中、巴蜀、河東、南陽與南郡等地,幾乎佔有一半中原地區。於秦王政時發起秦滅六國之戰。每滅一國即設置若干郡,如滅韓設置潁川郡,滅趙設置邯鄲郡、鉅鹿郡、太原郡,滅燕與代分別設遼東郡與代郡。前230年-前225年滅韓趙魏等國、並破燕國,秦國勢力擴張到河北、河南之地。前225年-前223年間滅楚國與越族,秦國勢力擴大到長江下游。前222年滅燕國與代國。最後於前221年滅齊。秦朝建立後,疆域由中原地區向四方擴張,前215年派蒙恬北伐匈奴,奪河南地,建九原郡。興建萬里長城,成為中原王朝的北界。南方方面,降伏閩越後設置閩中郡,並於前214年發起秦攻百越之戰,將疆域由長江流域延伸到嶺南地區、南海海邊與越南,設置南海郡、桂林郡與象郡,至此疆域達到最大。秦朝滅亡後,由於項羽分封十八諸侯,出現三秦、三齊等地域範圍,與三楚、三晉等成為秦漢之際的地域範圍。

秦朝的行政區劃為郡縣制,即郡(內史)、縣(道)制。早在戰國晚期,秦國與其他六國就推行郡縣制,只是秦國的封君食邑在嫪毐事變後完全被剷除。秦朝曾有兩次提議加設封國制,一次是立國之初丞相王绾建議如同周朝,增封秦皇諸子到東方新征服之地為王,一方面管制地方,另一方面拱衛中央。但李斯認為過數代後列國尾大不掉,反而成為帝國內患。第二次是儒生淳于越同周青臣辯論,他希望同商周時期設置封國,認為這樣才能如同商周常保國祚,遭到李斯的強力反對,埋下了焚書之禍。秦郡的特色是根據自然區域而劃分的,每郡多半以平原或盆地為中心,邊原為山地或高地。據《史记》記載,立國之初有三十六郡,后来增加至四十八郡,此外还有四十郡、五十四郡的说法。由于秦末民變和楚漢戰爭的严重破壞,导致详细资料的不足。

秦朝對地方控制十分嚴謹細膩。首都咸陽及附近關中平原由內史直轄。闽中郡由原君長閩越王無諸管轄。部分郡底下有為少數民族地區設「道」,這是因為秦廷只能掌握該地區的交通要道而故。每郡設有主管民政的郡守、主管軍事的郡尉、主管監察的郡監,郡守下設郡丞為副手。內郡郡守沒兵權,但邊疆與外族接壤,需要屯駐軍隊,所以郡守也有管理軍隊的權力。而邊郡郡承改稱長史,掌管兵馬事物。而縣置縣令,人數不滿萬戶的縣稱縣長,縣令也有縣承、縣尉,職務與郡承郡尉相似。縣以下有鄉。鄉官有掌教化的三老、掌訴訟賦稅的嗇夫(大鄉為有秩)、掌巡查緝盜的游徼等。一鄉轄十亭,亭有亭長。一亭十里,里設里正。一里轄百家,五家為伍,十家為什。

秦四十八郡分別如下(內史非郡,粗體為立朝後設置,斜體為不確定):
秦地:隴西郡、北地郡、漢中郡、巴郡、蜀郡
魏地:上郡、河東郡、東郡、碭郡、河內郡
趙地:雲中郡、代郡、鉅鹿郡、邯鄲郡、太原郡、雁門郡、常山郡
韓地:上黨郡、三川郡、潁川郡
楚地西部:南郡、黔中郡、南陽郡、長沙郡
楚地南部:九江郡、會稽郡、衡山郡、鄣郡 、廬江郡
楚地北部:泗水郡、薛郡、陳郡(楚郡)、東海郡
齊地:齊郡(臨淄郡)、琅邪郡、濟北郡、膠東郡
燕地:廣陽郡、上谷郡、漁陽郡、右北平郡、遼西郡、遼東郡
越地:閩中郡、南海郡、桂林郡、象郡
匈奴地:九原郡
另外,根據里耶秦簡,可能還有洞庭郡和蒼梧郡。

秦朝的政治體制為皇帝制與三公九卿制。秦國的政治制度由商鞅打下基礎,而李斯是加以發展、延伸。秦朝建立以皇帝為首的中央集團制度,皇帝擁有決定權與人事任免權,被後世中國各朝代所採用,而三公九卿掌握參政權。

三公分別是:處理全國政務、選用百官的丞相,負責監察百官的御史大夫(兼副丞相),掌管全國軍隊的太尉等。三公待遇相等,但丞相權力最大,居三公之首,御史大夫與太尉為輔佐,而皇帝委託政事給丞相處理。三公是從周朝的官職變遷而來。丞相源自掌握禮節的相,到春秋戰國逐漸參與政事,被國君提拔以制衡公卿,最後成為一國最高的行政首長。丞相的職稱也非秦國獨創,六國皆有丞相之官。秦朝丞相始於前309年,秦武王設置,並分左右丞相。前275年曾以魏冉為相國(比丞相更為尊崇),秦王政時期呂不韋也擔任此職。前208年李斯被殺後,趙高以宦官擔任,稱中丞相。丞相的幕僚有侍中、尚書與舍人等。太尉源自戰國的尉,本為君主的侍衛,後為代替貴族掌管軍事之官。地位最高的為國尉,趙國也有此職務。秦昭襄王時白起就擔任過國尉,秦王政時尉繚也擔任此職。但秦朝建立後太尉為虛職,軍權由秦始皇直接掌控。御史大夫為副丞相,為御史們的領袖,所以稱為大夫。御史本為君王近臣,掌管記事之職,戰國時為君主的耳目,替君主監察百官。秦國設大夫為御史領袖,到秦朝時,地位才提升到僅次丞相。重要幕僚有御史承(對內事物)、御史中承(專管監察)及侍御史十五人,另有符節令、領符璽郎數人。

九卿為中央政府下轄的九個官職,通常也表示整個朝廷。九卿也是源自周朝的官職,共有:衛尉(皇宮保衛)、郎中令(警衛)、太僕(宮廷車馬)、廷尉(司法)、典客(外交)、奉常(宗廟禮儀)、宗正(皇室內部事務)、少府(山河湖海稅收和製造業)與治粟內史(財政稅收)等,另有內史(管理首都與關中地區)與將作少府(宮室興建)等。除了九卿,秦廷還設有中尉(巡視首都、緝察盜賊)、詹事(管理皇后與太子家事)、將行(皇后之卿,由宦官擔任)與主爵中尉(管理列侯事物)等。秦朝國家機構齊全,在中國歷史上的建立中央集權制度的特點。這種「職臣遵分,各知所行」的管理方式,使不同行政機構並立,不相統屬,只對皇帝負責。由於強化官僚的行政職能,進一步削弱宗法貴族對朝政的影響力。

秦廷重視法家,所以秦律為中國古代較為完善的法律體系。秦律的許多法律刑責都極端殘暴和酷烈,導致「奸邪並生,囚徒塞路,牢獄成市,天下愁怨,潰而叛之」。

秦律主要是刑法,刑責分為死刑、肉刑、笞刑、徒刑、遷刑、髠刑與耐刑、罰金、贖刑、連坐與族刑、剝奪政治權力與誶刑等。死刑分為九類,有梟首(斬首示眾)、棄市(處死於鬧市)、斬(砍頭與腰斬)、車裂(五馬分屍)、磔(分裂肢體)、戮(處死後陳屍示眾)、定殺(淹死)、生埋(活埋)、絞(絞死)等。據《漢書‧刑法志》,商鞅時期還有鑿頭顱、烹殺等等慘忍刑罰。肉刑分為四種,有黥刑(臉上刺字)、劓刑(割鼻)、刖刑(断足)、宫刑(去除生殖器)等,這四種肉刑是自夏商周三代流傳下來的,在推崇重刑的秦廷中十分常用,例如受過宮刑的犯人就有七十餘萬人。笞刑就是用竹木板責打犯人。徒刑就是服勞役,分為城旦舂(男修城牆、女舂米)、鬼薪(男砍山柴)、白粲(女擇米祭廟)、司寇(至邊疆勞役)、罰作復作(男至邊疆戍守勞役、女至官府勞役)、侯(從事嘹望、防禦)、下吏(官吏被罰服勞役)、隸臣妾(罰作官奴婢)。遷刑是強制犯人遷至指定地區服勞役而不得返回原地。髠刑是剃光犯人頭髮,耐刑是剃光犯人鬍鬚與鬢毛。罰金是犯人繳納罰金或有價物給政府。贖刑是用罰金來贖免其被判處刑罰。連坐是連帶犯人的家人、鄰里和相關人等均受刑,族刑則是滅絕犯人全族。剝奪政治權力共有奪爵、廢、削籍等。誶刑則是責罵犯人。

總之,按照輕罪重刑原則,秦律可以被稱為苛政嚴刑。因為戰爭的利益與統一天下的慾望,人民尚可承受此嚴刑。當秦始皇統一天下後,未能輕徭役、省刑罰,使得山東六國遺民不能接受此等苛法,而關中秦民也希望能緩和徭役,埋下秦朝動盪不安的後果。

秦朝政治另一個特色是連坐法。對於各級官僚,當某處底層發生社會動蕩等嚴重違反秦律的事情,則從最下級官員一直到中級官員全部撤職,只需要審判最底層的官員,定罪後則他的幾層上級無須審判就可全部定罪。在地方,則有什伍連坐法,按軍事組織把全國吏民編製起來,將農戶的住宅集中在田邊聯排建設,戶戶相鄰,並以「五家為伍,十家為什」編為單元,便於互相監視,互相檢舉,每組單元均設立負責人,若單元內有人犯法或舉報,負責人要向上級彙報並依法律處理。農戶不准擅自遷居,不得私下鬥毆,不留宿未登記之陌生人,不得私售糧食等等。這種嚴苛的法律把農民牢牢束縛在土地上,國家直接控制了全國的勞動力,極大提高糧食賦稅收入。

秦朝統一天下後,北方的外族威脅較大,有匈奴、東胡與月氏等,其中以匈奴最強盛。匈奴分佈在山西、陝西北部到漠南地區,可能是源自西周的玁狁,在戰國時期擊敗燕國、趙國兩國,這兩國興建長城以禦北疆。在秦滅六國之戰時,列國忙於內戰,匈奴趁機坐大。到前215年,因方士提出「亡秦者胡」,秦始皇就派蒙恬率三十萬大軍北伐匈奴,頭曼單于率眾北遁,遷都頭曼城(今內蒙古陰山山脈北)。秦朝占領河南地,設九原郡。而後在前214年至前213年,秦廷連接秦長城、趙長城與燕長城而成萬里長城,長城成為游牧民族與農耕民族的分界線。蒙恬守北防五年,匈奴懾其威猛,不敢再犯。然而到秦二世時,冒頓單于奪得單于之位,在他的經營下匈奴走向強盛。東胡分佈在河北、遼寧北部與內蒙古東部一帶。春秋時期兼併山戎,戰國時期勢力擴大,號稱「控弦之士二十萬」,多次南下侵入中原,後被燕將秦開擊敗。在秦漢之際衰退,而被匈奴擊潰。最後發展出鮮卑、烏桓等東北民族。而月氏分佈在甘肅、內蒙古西部一帶,經常與匈奴發生衝突。秦末民變時期,月氏實力強大,與東胡一同脅迫匈奴,匈奴曾送人質(時任太子的冒頓)於月氏。而後於前205年—前202年間被匈奴擊敗。月氏一分為二,向西遷徙,最後在西域立國的為大月氏,遷移到青海北部的為小月氏。

而南方以百越為主,在浙江南部有東甌國、福建有閩越國、廣東與越南有古南越國、廣西與越南北部有西甌國與雒越國。秦滅六國之戰時,於滅楚平越後,降伏東甌國與閩越國。於前222年在于越之地設會稽郡,於東甌閩越之地設置閩中郡,派原閩越王無諸為閩中郡君長。秦朝建立後,秦始皇基於統一南方與獲得象牙、犀角、珍珠等等南方財富,發動秦攻百越之戰。約在前221年-前219年間,派屠睢率領五路大軍南下攻打西甌國與古南越國。雖然大軍攻下古南越國首都番禺,但是地廣人稀,時常有游擊襲擊糧道,戰事拖延三年。而後派史祿於湘江和珠江支流灕江建立靈渠,以便運糧給南征軍。這樣南征軍得以深入到西甌國內,攻殺其君長吁宋。然而游擊依舊未止,屠睢最後也被擊殺。前214年秦始皇再派任囂與趙佗率樓船士入援,並且先後派士兵、罪犯或百姓武裝移民,嶺南地區得以粗安。此後,於番禺設置南海尉以控管嶺南三郡(南海郡、桂林郡與象郡),於嶺南西北的靈渠與灕江交接口設置堡壘(今廣西興安秦城遺址),史稱「置東南一尉,西北一侯」。秦二世時,秦末民變爆發,南海尉趙佗閉關自守,不派軍隊北援平亂。而後自稱南越武王,建立南越國。

而東北部分,大多為東夷族的後代,位於今遼寧的有濊貊,位於今吉林的有肅慎,位於今朝鮮半島北部的有箕子朝鲜。箕子朝鲜於秦漢之際被衛滿率東逃之民征服,建立衛滿朝鮮。秦始皇為了長生不老,聽信方士徐福之言,認為東海的蓬萊、方丈、瀛洲等三座仙山上有仙丹,於是派徐福於琅岈準備尋覓。前210年秦始皇東巡琅岈,徐福推託說海上有大魚,而無法出巡。秦始皇就派人射殺之,並迫徐福率童男童女遠航。據說徐福可能抵達日本九州或大阪一帶建國。

西方则有西戎、西羌等族。西戎聚於關中隴西一帶,於春秋戰國時期與秦國時常戰爭,最後被秦國同化。西羌聚居於今陝西、甘肅、青海一帶。秦國向西擴張時曾與羌人發生過戰爭。西南方的國家統稱西南夷,有位於滇東黔西的夜郎國、且蘭、句町等國,與滇中的滇國(據說為楚將莊蹻所建)、勞浸、靡莫等國,滇西的昆明國、斯榆、桐师、哀牢等國。因交通道路不便,所以秦廷只有修建五尺道,此道經僰道(今四川宜宾)、朱提(今云南昭通)到滇池,以溝通夜郎、滇等國為主。

秦朝的軍事制度源自秦國商鞅變法的基礎上發展起來的。秦始皇兼任太尉以掌控軍權。平時太尉掌握全國兵事,戰時則另置將軍,作為軍隊的最高統帥。將軍有上、前、後、左、右將軍之分,其下有裨將、都尉與司馬等。還有監軍以監督出征軍隊。戰爭結束後,軍隊交還國家。

秦朝採取徵兵制,主要是全國男丁,一半務農,一半當兵。自二十三歲開始服役。先在所屬郡縣服役一月,即「更卒」;再赴首都任戍衛一年,即「正卒」;最後屯邊一年,即「戍卒」。秦朝建立前夕,約有軍隊五十萬人。然而統一後兵力反而擴充,例如北駐五原與南戍南越即調動九十萬,修築阿房宮與驪山秦始皇陵又動員七十萬人。因兵員濫用而不足,又動員商人、罪犯、贅婿與閭左(貧弱的人)為兵,戰鬥力也大為稀釋。到秦末農民戰爭就有兵力不足與素質不齊的問題。

秦朝的軍隊組織,可分為正卒(京師兵)、更卒(地方兵)和戍卒(邊兵)三部分。京師兵由郎官、衛士和正卒組成。郎官由郎中令統領,衛士由衛尉統領,負責宮廷內外的警衛。負責守衛京城的屯兵由中尉統領。地方兵置於郡、縣,一般由郡尉、縣尉協助郡守或縣令統率,平時維持地方治安,戰時聽需以皇帝「虎符」為憑才能徵調。邊兵主要負責邊郡戍守,由邊郡郡守統領,下轄都尉和部都尉。軍種分為輕車(車兵)、材官(步兵)、騎士(騎兵)、樓船(水兵)四個基本兵種。大抵平原諸郡多編練騎士、輕車,山地諸郡多編練材官,沿江、海諸郡多編練樓船。秦軍以步兵為主,騎兵、車兵為輔。车兵虽已不是军队的主体,战斗时远则以弩箭射击,近则以矛钺格斗。步兵是秦代军队构成中的主体,分有重装步兵和轻装步兵两种。騎兵方面,用於北击匈奴和镇压农民起义的战争。樓船士主要分布在巴蜀地區,而後擴展到江南,在南征南越時被用於支援步兵。

秦朝軍制的一個特色是商鞅制定的二十等爵制度,共二十等。軍士要憑斬殺敵人首級而升爵,但這應該限於低階爵位。升爵一等者,有罪可以減免,五十六歲退役。無爵者稱為「士伍」,六十歲才能退役。最高為徹侯,可有食邑,大者食縣、小者食鄉、亭。由於這個制度,使得秦人善戰,被稱為「首功之國」。

秦軍的武器裝備大多採用銅器,直到秦朝建立後,獲得六國大量鐵兵器,以及中原冶鐵技術,才完全以鐵器為主。秦軍的遠射武器十分先進,不論步兵、騎兵或車兵,都裝備有大量的弓、弩、箭。其箭大多採用飛行穩定、準度高的三出刃簇。還有一種特大的簇,專門用於強弩。秦軍的近戰武器大多以銅器為主,有長柄的戈、矛、戟、鈹,短柄的彎刀和劍,還有某些過了時的鏢、殳、鉞等等。秦式铜剑不仅长,而且很锋利。一些剑出土时毫无锈蚀,光洁如新,锋刃锐利。经试验,一次尚能划透18层纸。秦剑多数为双手使用,少数剑茎较短者可能是单手剑。秦軍的鎧甲已經制式化,均用金屬製成,式樣因兵種及職位不同而有所區別。

秦朝人口没有可靠的数据,历代学者只能根据各种的条件推算。葛剑雄在其著作《中国人口发展史》中推算为前213年有2,500万人口左右。范文瀾在其著作《中國通史》中認為秦時中國人口約為兩千萬左右。秦末民變和楚漢戰爭造成期間大量人口死亡。秦始皇時代的人口總數在2000萬~3000萬之間,但在經過戰國末年長期的戰爭之后,人口數量已經有了很大幅度的下降,所以當時普遍存在的是勞動力的不足,而不是人口壓力。正因為如此,秦始皇在大規模征調民眾服勞役和兵役時不得不採取殘暴的強制手段,而且已顯得捉襟見肘,如對征南越的軍隊派不出更多的增援和補給,在設置新政區后也無法遷入更多的移民。秦亡以后和西漢初年,秦朝的新領土喪失殆盡,西南和南方全部為當地民族奪回,或建立了實際上獨立的政權,一個重要的原因就是來自秦朝的駐軍、行政人員和移民數量太少。就是在中原移民集團掌握了政權的南越國,還得依靠當地的部族首領,沿用百越習俗。秦朝曾有一些人從山東半島及東南沿海地區遷往朝鮮半島和日本列島,但他們大多是出於逃避秦朝統治的目的,或者是長期海洋遷移傳統的延續,並不是遷出地人口過多的結果。

秦廷為了穩固帝國,透過遷移六國富商貴族入關中的方式以嚴格控管透過武裝移民的方式鞏固新征服之地。第一類即是「實關中」,如前221年秦始皇「徙天下豪富於咸陽十二萬戶」,目的在於加強統治,削弱關東地區的經濟力量,把關中發展成為帝國的政治與經濟中心。第二類是武裝移民邊區,例如移民瑯邪郡三萬人;在北方經營「新秦」,遷罪人以實邊疆。還曾向麗邑(今陝西臨潼)移民三萬人、向雲陽(今陝西泾阳)移民五萬人等等。其中最有名的遷移有兩次,分別是前214年秦始皇派蒙恬率30萬大軍,奪取河套地區,設九原郡,遷徙內地人民以發展生產,加強邊防。以及秦攻百越之戰後期,徵集「諸嘗逋亡人、贅婿、賈人為兵」(大概是商人和囚犯等人)近10萬加上原先剩下的20萬秦軍部隊南征成功,設置桂林、南海、象郡等,派官進行治理,遷徙50萬中原人民與百越族雜居。這些都對長城沿線和華南的開發起了重要作用。

秦朝經濟以商鞅變法的制度為基礎,即耕戰、重农抑商、盐铁国营政策,主要以提升土地的粮食产出與迅速增加人口為目標。統一天下後,秦始皇將此制度推廣到關東地區,不合理的以關中奴役關東地區的區域經濟方針。秦始皇期望「諸產得宜」,在謀求「男樂其疇,女修其業」,即民眾都積極在社會生產的積點上,形成新的經濟秩序,實現所謂「黥首是富」,「諸產繁殖」。由於秦朝經濟的史料有限,後世對其管理方式大多採負面評價,所以有賴出土文獻如《睡虎地秦簡》、《里耶秦簡》釐清。總之,秦朝經濟是採取精密又嚴格的軍事化管理,以極端苛急的政策傾向為特徵。秦朝工商業的發展戰國以來社會生產力的飛躍發展,秦統一國家的建立,以及秦始皇推行的有關經濟政策,促進了秦代手工業和商業的發展。經濟的發展,使得秦漢時形成四個經濟區域,分別是山西(即關中巴蜀)、山東(即關東平原)、江南(淮水以南)與龍門、碣石以北(燕山、晉北、陝北與隴西等長城一帶)等四大經濟區。

早在战国初期的秦國,商鞅變法推行「國家授田制」、「什伍連坐法制」與「重農抑商」政策以提倡農業。此後秦國的經濟體制轉入「耕戰」,即重視農業生產和對外戰爭。以農業生產支持對外戰爭,平民可以由二條路徑獲得國家爵位和官位,分別是:參軍作戰,透過獲得敵方首級折換國家爵位;農戶可以將家裡的存糧按一定比例折換國家爵位。關於國家授田制,秦廷將新開墾土地、無主土地、罰沒土地、戰爭奪得土地,通過法律形式授予以農戶為單位的家庭。授予土地不可出售或出租,若農戶家中無男性繼承者,則土地由朝廷收回並重新授予其他農戶。秦律禁止父子兄弟同室而居,凡民有二男勞力以上的都必須分居,獨立編戶,並授予新的土地。前216年秦始皇推行「使黔首自實田」,即讓人民向官府實報領有田畝數量,使秦朝進入土地私有制,國家授田制逐漸式微。什伍連坐法把農民牢牢束縛在土地上,國家直接控制了全國的勞動力,極大提高糧食賦稅收入。由〈田律〉、〈廊莞律〉可知秦廷與郡縣政府要負責各地區地籍、戶籍的登記與管理、耕牛的統一飼養與出租、鐵器的出租、種子的出租,並興修區域水利工程。秦廷十分重視降雨即時撥種,並且要上報農田生長狀況、天災危害狀況等等,讓秦廷進行具體的規劃與評估。

秦朝對水利工程十分重視,先後建立都江堰、鄭國渠與靈渠。鄭國於關中興建的鄭國渠為秦國開闢大量良田,提升經濟力,助長秦國吞併六國的能力。秦國李冰及其子為了解決蜀地岷江防洪與引水灌溉問題,建立的都江堰。《史記·河渠書》記載,「穿二江成都之中,此渠皆可行舟,有餘利用溉浸,百姓饗其利。」岷江和長江因之得以通航,岷江上游盛產的木材還可以漂運成都,使得成都從秦朝時起便成為蜀地交通的中心。靈渠則偏向運輸為主,為打通長江與珠江的重要渠道,溝通、運輸中原與嶺南地區的貨物。

秦漢農業結構為「主穀式」,主穀種植比例比大豆、小豆、芝麻、葵花子、麻等還要普遍。江南經濟區與山東水力豐富之處以水稻為主,山東經濟區以大麥、小麥、椹麥、春麥、栗、黍、稷、菽等農作物。這些在先秦時期早已種植,但在秦漢時期,麥的地位逐漸超越其他農作物。至於秦朝農作物產量,根據《睡虎地秦簡‧倉律》:「種:稻、麻畝用兩斗大半斗,禾、麥畝一斗,黍、苔畝大半斗,菽畝半斗。」。

由於軍事後勤的需求,秦始皇先後在全國各地興建許多大糧倉。縱觀秦糧倉所設之處,可以分作三組:第一組太倉、敖倉、陳留倉、櫟陽倉、咸陽倉、龍岩倉等,與漕運關東糧食以供應關中有關。因為陳留倉地處鴻溝,是江淮運輸的必經之地,而龍岩倉、敖倉居滎陽附近,臨近黃河,正是鴻溝轉入黃河之處。霸上倉、櫟陽倉則是從黃河沿渭水運抵咸陽的終點站,太倉則是漕運的起點。第二組為督道等邊郡諸倉,專為北方軍事征戰而設的,是內地糧食運往邊境的貯藏地。第三組成都倉建置時間較早,也是與軍事行動有關。

秦朝手工業分為官營和民營兩種,官營負責生產朝廷、皇室與官府的需求。民營製造民用產品,以投入銷售,與商業發展關係密切。根據不同的手工技術又分為工官、鐵官、將作等獨立的生產部門,負責生產和管理的有監造官、工師、丞等。從事生產的工人,有的是工匠,大部分是刑徒。官府手工產品必須刻上製造官署、負責人和製造人的姓名,以便考核質量。   

秦朝最重要的手工業是冶鐵業,在戰國後期已相當發展。秦國官府專門設有管理鐵器生產和使用的官吏,如秦簡中的左採鐵、右採鐵等。秦代民營冶鐵業也很發達,司馬遷的四世祖司馬昌曾為秦主鐵官,當時鐵官大概既管理官營冶鐵業,又負責向民營冶鐵業收取鐵稅。秦廷曾把一批六國的冶鐵手工業者遷到巴蜀、南陽等地。蜀郡卓氏、程氏、南陽孔氏等,都是這樣發展起來的冶鐵家,促進了冶鐵業的發展和商品市場的活躍。秦朝的青銅業也具有地位,秦始皇曾經沒收六國遺民武器,將銅製武器鑄成十二金人,可想而知這樣的官營冶銅作坊規模更大。在秦始皇陵東側大型陶俑陪葬坑中,發現的青銅兵器劍、矛、鏃等,皆製造精良,並經過鉻化處理,不蝕不銹,鋒利如新,說明秦代青銅鑄造技術非常高超。

制陶業也很發達,當時官府制陶作坊為宮廷燒制磚瓦和陪葬用的陶質明器。而秦始皇兵馬俑的陶俑十分逼真、精美。另有官營性質的市亭制陶作坊,還製造部分民用生活用器。而民營制陶作坊全部製造生活用器。在秦都咸陽附近的咸亭,是民營制陶作坊的聚居區,咸亭所屬的17個裡都是密集的民間獨立制陶作坊,可見當時私營制陶業的發達。另外,漆器業也很發達。關中是中國古代生漆的重要產地,漆的產量相當多。據《睡虎地秦簡‧工律》,秦朝有專門的漆工和髤工從事漆器生產。在秦墓中也不斷出土各種漆器。此外,秦代的紡織業、製革業、煮鹽業等都較發達,特別是造船業和建築業相當進步。 

秦朝自商鞅變法推行「重農抑商」後,嚴格限制商業的發展,主要措施有:在集市收取高額的市場租金;在主要道路關卡收取高額的關稅;對商人編商籍(類似今天的工商登記);若商人破產則將被收編為國家苦役。這些措施實施後,使得商人的可預期利潤遠低於農戶,於是自由商人自行消亡。然而隨著農業和手工業的發展,秦朝商業也開始活躍起來,而且秦始皇對大商人就比較尊重。如烏氏倮以買賣畜牧起家,馬牛數量之多要用山谷來計量。他曾經以絲綢等珍奇物品贈獻給戎王(部落酋長),獲得十倍的酬償。秦始皇也封他為君。巴地寡婦清維護先人經營的丹砂而致富,秦始皇帝因她是貞婦以客相待。為她修築「女懷清台」。這些作法實際鼓勵商業的發展。同時,秦始皇統一貨幣、統一度量衡及水陸交通建設,也為商業的發展創造了條件。鹽鐵部分,主要由官府經營鹽、鐵的開採和販賣,民間商人不得從事此類行業。

秦朝的商業都會大多在交通便利之處,有商品集散中心的作用。關中地區物豐民富,且西南控巴蜀、西北控天水、隴西、北地與上郡,東通三晉,而咸陽成為關中平原的唯一大都會,也是全國最大的商業中心。蜀郡成都,市張列肆,與咸陽同制,城內有繁華的商業區,其他還有雍(今陝西鳳翔)、櫟陽(今陕西阎良)、烏氏(今甘肅平涼西北)等。自秦朝新發展起來的城市有麗邑(今陝西臨潼)、雲陽(今陝西涇陽)、臨邛(今四川邛崍)等,都有新興商業市場。其他自戰國以來就發達的城市的部分:關東地區被黃河分割成兩部分,溫(今河南溫縣)與軹(今河南濟源)等城市,西接關中,北連薊(今北京市)、邯鄲(今河北邯鄲)等燕趙地區。黃河以南的商業活動更盛,主要都會有陶(今山東定陶)、臨淄(今山東臨淄)與宛(今河南南陽)等。江淮地區的吳(今江蘇蘇州)、壽春(今安徽壽縣)也都是地區性都會,與關東地區有密切的商業關係。嶺南唯一的大都會為番禺(今廣東廣州),與海外貿易犀象、玳瑁、銀銅之物等。秦朝的馳道與江河水運把這些都市、集市連接起來,形成龐大的經濟整體。

秦始皇統一中國地區後,推行書同文、車同軌等統一制度。其中與經濟有關的為貨幣與度量衡。秦始皇廢止戰國時各國形制貨幣,改以黃金為上幣,以鎰(二十兩)為單位。以秦國舊行的方孔圓錢銅幣為下幣,文曰秦半兩,重如其文,銘文用小篆。其改革詳細如下:法定銅錢制定為較晚發行但較先進的方孔半兩有廓圓錢。廢除秦國較早發行的圓孔無廓圓錢、圓肩圓足有孔布錢。廢除六國使用的布錢、刀錢、蟻鼻錢以及郢爰(金幣)等貨幣。珠玉、龜貝、銀錫等物不得視作貨幣而流通使用。貨幣的統一,減少貨幣流通時不必要的換算,提升方便性,對於商業經濟發展有很大的幫助。然而,由於鑄造技術的局限,每枚錢的質量不會完全一樣,所以要求使用錢幣時不得挑選,不論重輕一律使用。

前211年秦始皇頒布的統一度量衡的詔書,用商鞅時制定的度量衡標準器為全國標準,用法律規定了度量衡器誤差的允許限度。此外還規定六尺為步,二百四十步為畝。不過二百四十步為畝的制度實際上只行於舊秦,東方許多地區仍以百步為畝。統一貨幣、度量衡為經濟的發展提供了便利條件,促進了統一國家的發展。

秦朝交通要道秦馳道、秦棧道、五尺道與嶺南新道(缺秦直道),與國防建設秦長城。

秦始皇統一全國後,為了溝通關中與其他地區以避免地方叛亂,興建秦馳道與靈渠,行「車同軌」(統一車軌間距)。為了抵禦北方胡族,完成了秦長城。為了避免六國遺民舉兵叛亂,夷平城墩、長城。沒收天下百姓武器,將其中銅製武器鑄成十二金人,置於咸陽,又興建阿房宮、興樂宮等離宮與驪山秦始皇陵,包括著名的秦始皇兵馬俑。但據目前的考古證據顯示,阿房宮並沒有建成,因此被燒毀亦無從談起。此外,為了安定六國遺民,秦始皇五度出巡關東,並且封禪與刻石。

交通建設有秦馳道、秦直道、秦棧道、五尺道與嶺南新道,以及溝通中原與嶺南的靈渠。秦馳道以首都咸阳为中心向四面八方延展開來,為溝通關中與關東、江南各地區。當六國遺民舉兵叛亂時,秦軍可以迅速前往平定。馳道在平坦之處,道寬五十步,隔三丈栽一棵樹,道兩旁用金屬錐夯築厚實。靈渠打通長江的湘江與珠江的灕江,溝通中原與嶺南之地。當百越遺民叛亂時,秦軍可以經靈渠、灕江、珠江直抵嶺南重鎮番禺(今廣東廣州)平亂。

秦馳道有如下數條:
西方道:為溝通關中與隴西地區,由咸阳西北至北地,再西至甘肅临洮,最後經渭水返咸陽。
上郡道:為溝通關中與代郡、九原地區,咸陽經高陵北上至上郡(今陝西陝北一帶),再到雲中,銜接北方道。
北方道:為溝通長城一帶的地區,由九原沿长城东行如雁門、代郡、上谷、漁陽、右北平至遼西郡的碣石。
東方道:為溝通關中與整個關東地區,其中洛陽、陳留與定陶都是重要的大都會。咸陽經函谷關沿黄河经洛陽、陳留、定陶、泰山、临淄、膠西至山東半島最東邊的成山頭。
東北支道:由洛陽北上經孟津、邯鄲、鉅鹿、恆山至河北廣陽的薊城。
臨晉支道:渡黃河後,蒲州經安邑、上黨至太原,貫穿山西後再接邯鄲。
濟北支道:鉅鹿東通临淄。
中山支道:恆山西通太原、雁門。
膠東支道:临淄東南通瑯琊後分兩支,一支再北接膠西至山東半島北邊的芝罘山,另一支南通東海接江南道。
江南道:為溝通關中與西楚、東楚地區(詳見三楚),洛陽東抵泗水、東海,至此分兩支。
東楚支道:南渡長江、經吳縣、錢塘至會稽山,並由錢塘通丹陽。
西楚支道:走彭城、壽春、南渡長江、經衡山郡至江陵。
武關道:為溝通關中與荊楚地區,咸陽經武关走東南方,經河南南阳至湖北江陵。
秦直道:為溝通關中與九原地區,由咸陽西北的云阳(今陕西淳化)的秦林光宫(即汉甘泉宫)為起點,循子午嶺、榆林定邊縣,再東北渡鄂爾多斯草原、於今呼和浩特昭君墓渡黃河而達九原。秦直道可代替較迂迴的上郡道,當匈奴侵犯九原地區時,秦軍可立即自關中北上支援。
秦棧道:為關中越秦嶺與巴蜀地區溝通。
五尺道:為蜀地通西南夷地區的通道。
嶺南新道:溝通荊楚與嶺南地區,為秦攻百越之戰期間所修道路。主要有四條道路,接武關道,越南嶺後可達番禺、桂林與象郡。
關於國防建設,秦朝於前214年北伐擊敗匈奴後佔據河套。為了保衛建九原以抵禦北方胡族如匈奴、東胡等,秦始皇將秦、趙、燕三國長城連接起來,並興建九原長城,形成秦長城。秦長城從甘肅臨洮到遼東,大致分為西段和北段。為了避免六國遺民舉兵叛亂,秦始皇將六國遺留城墩、長城悉數破壞、夷平。沒收天下百姓武器,將其中銅製武器鑄成十二金人,置於咸陽。

秦朝是法家理論得以全面實踐的時代,以較急促的方式完成高度集權的專制統治,與東方六國的政治風格大不相同。在思想文化謀求統一,通過強制性的專制手段推行政策。例如焚書坑儒就是企圖摒棄東方傳統,以秦文化為主體實行強制性的文化統一,導致諸子百家之學出現歷史的斷層。另外一個特徵則是「以吏為師」,由官吏來承擔思想文化的傳遞與教學,廢除先秦時期就十分活燿的「私學」。這表現出秦廷重「法」輕「學」的文化導向。

秦朝建立後,秦始皇持續採用自商鞅以來行之有效的法家學說來治理國家,並且採用鄒衍陰陽家的「五德始終說」,以維護朝廷的統治。李斯為當時法家的代表人物,他是荀子的學生,繼承商鞅的思想傳統,推行法治,受秦始皇重用,先後任長史、廷尉與丞相。此時秦始皇與李斯所採用的政治學說,為韓非提倡的「法、術、勢」(以法律為根本、以愚民為手段、法術勢相結合)三合一的君主專制,以及以郡縣制為其基礎的中央集權制。首先加強皇權與尊君的措施,秦始皇改「王」而自稱「皇帝」,規定皇帝之命稱「制」、令稱「詔」、印稱「璽」、皇帝自稱「朕」。以臣下不能議論皇帝為由,廢除諡號,而以始皇帝、二世、乃至萬世等。皇帝為全國最高統治者,獨攬軍政大權。此外為了長久統治人民,採納陰陽家的「五德始終說」,用五行解釋朝代更替之因,以歸之於天意,認為每一種德興起時,天下會出現祥瑞。秦始皇統一後,齊人把這個學說上奏,認為「秦以周為火德,能滅火者,水也」,所以秦朝為水德。因此秦始皇推行相關制定:冬天屬於水德,所以規定十月為歲首。水德尚黑,以黑為正色,衣服旗幟皆改用黑色。水德與「五數」中的「六」相應,所以各種器具都用六來計數。按陰陽家所說「水主陰,陰刑殺」,結合法家思想,演變成重嚴刑峻法,不講「仁恩和義」。

對控制人民思想方面,秦始皇藉由「五德始終說」,塑造自己神化,為受命於天的「真命天子」,他的政制措施為天意,不可違抗,使得人民服從他的統治。其次,不崇古,廢除封國制,建立郡縣制為基礎的中央集權制,就如李斯所說的「海內為郡縣,法令由一統」。還有,禁止「私學」,加強思想控制。到秦始皇後期,因為焚書坑儒事件,更是排斥儒家、道家,獨遵法家。焚书事件源自前213年,太子师博士齊人淳于越提出恢愎周朝的封國制。秦始皇召集延議來商量,丞相李斯反對,並且認為各人用自己的意見批評政府的法令,既影響政府的威信,又容易造成宗派。為了禁天下百姓「以古非今者」,秦始皇下令各郡、縣立即查禁所有《詩》、《書》和諸子百家的書籍,30天内全部焚燒,只留醫藥、卜筮、種樹之書即可。由咸陽博士官的藏書保留諸子百家之書,以嚴格控制思想。坑儒事件為秦始皇向方士盧生等人求仙受騙,又有方士儒生議論朝政。秦始皇大怒,下令搜捕咸陽城内的方士儒生,後來的審問過程中,方士儒生互相告發,共有460餘人受到株連,秦始皇下令將這些人全部坑殺。此即“坑儒”。這個作法是源自商鞅、韓非主張的人民不需要任何教育,詳知國家法令即可;不需要先王的教誨,人民以官吏為師即可;不應該私鬥為強悍,以斬殺敵人為勇敢。焚書坑儒對當時中華文化的一次严重破壞,春秋戰國的百家言論被官方嚴格控制,而天下百姓也被控制成愚民。由於秦朝重賦稅、嚴苛法、濫用民力的政策,遠遠超出了農民所能承受的程度,最後導致秦朝的早亡。

秦代文學並不發達,這是受到獨尊法家與思想箝制(即焚書坑儒)的關係,對文學發展並不重視,再加上秦朝國祚不長,使得文學發展沒有獲得很大的成就。在秦國統一天下前,較重要的著作有《呂氏春秋》,包括八覽、六論、十二紀,後世又稱「呂覽」,是秦國丞相呂不韋率領其門下門客所著成的著作。其內容涉及甚廣,以道家黃老思想為主,兼收儒、名、法、墨、農和陰陽各先秦諸子百家言論,被《漢書·文藝志》列為雜家。書中保存了大量的先秦時代的文獻和遺聞佚事,在《蕩兵》、《順說》、《察今》等篇。《呂氏春秋》以寓言故事表提其含意,說理文,層層深入,最見條理,是中國古代類書的起源。

秦代文學的主要作家是楚人李斯,為法家荀子的學生,後入秦國投靠呂不韋,是建立秦朝政治體制與地方制度的重要人物之一。呂不韋因嫪毐事變被廢後,秦王政(即秦始皇)聽從宗室大臣建議驅除外國客卿,而李斯作《諫逐客書》反駁此意見。文中敘述秦國自秦穆公以來,用人惟材,不限於本土人才,這才獲得東方六國之士投靠,使秦國比六國強盛。並且以秦始皇獲得的天下寶物皆非秦國出土為譬喻,希望秦始皇能繼續使用客卿。文章排比鋪張,有戰國縱橫辭說的習氣。而文辭修飾整齊,音節和諧流暢,與漢初散文和漢代辭賦也頗接近。

秦始皇東巡時時常刻石,以歌頌秦皇功德。如泰山、琅牙、之罘、會稽等處的刻石文,大都也出於李斯之手。例如《秦泰山刻石》、《琅琊台刻石》、《嶧山刻石》和《會稽刻石》等。內容為四言韻文,以三句為一韻。這是最古的碑文,對後世碑誌文有影響。民間歌謠方面,根據《漢書·賈捐之傳》有「長城之歌,至今未絕」的話,顯示秦代勞役繁重的情況。類似這樣的民謠當時一定不少,可惜沒有流傳下來。

秦朝文字採用小篆與隸書,主要用途是记录历史事件、律法、生产工艺。戰國時期各國使用文字有異,秦始皇統一後即要求“書同文”,並且命秦丞相李斯執行這項工作。李斯簡化秦國原本使用的西周大篆而建立小篆,並且推行到全國各地,所以小篆又被稱為「秦篆」。為了推廣全國,李斯、趙高、胡毋敬等人用小篆來編寫識字課本,比較著名的是《倉頡篇》、《爰歷篇》、《博學篇》等,成為了兒童的啟蒙教材。而秦人程邈因得罪秦始皇進入牢獄。他整理民間各字體,簡化小篆書寫方式,改良為隸書。最後被秦始皇賞識,被任命為御史。「隸」字有「附屬」的含意,可能意旨其為篆字之衍生。漢字從小篆到隸書的演變過程稱為隸變,主要將篆書圓轉的筆劃改為方折,書寫速度更快,後來被漢朝採用,而成漢隸。

秦朝國祚較短又偏重法制,沒有建立縝密、系統化的宗教。主要以上帝崇拜和祭祖先為主,民間尚存在上古鬼神之说、巫術、占卜與占星,以及起於先秦的神仙方術。其中後者深受秦始皇崇拜。

上帝崇拜源自商朝,秦國於東周時居於歧西,秦國君王按五行崇拜上帝之一的白帝,又因秦族源自東方而崇拜少昊。到秦始皇統一中國地區後,秦人有白、青、黃、炎四天帝之祭(無北方黑帝),其中因地理關係特別重視西方白帝。秦朝皇帝每年都会去祭天,以祈求来年风调雨顺。秦室以崇拜上帝為主,而秦人除了上帝還以陳寶為貴。陳寶為寶雞地區之神(即陳寶祠),為秦地所特有。至於雍地諸祠,有日、月、星、辰、南斗北斗、熒惑(即火星)、太白(即金星)、歲星(即木星)、土星(即金星)、辰星(即水星)、二十八宿、風伯、雨師、四海、九臣、十四臣、諸布、諸嚴、諸逑之屬,百有餘廟。而杜伯是西周右將軍,在雍地也有祠(即右將軍廟),被秦人視為最靈驗者。以上諸天神、人鬼與地衹為古代傳統信仰,為上古鬼神之说。祭祖先也是主要信仰之一,表現對宗廟的重示。秦始皇統一全國時不謝鬼神,而言有賴先祖諸王的庇護。秦先王廟或在雍地、或在咸陽。秦二世時立史皇廟,尊為祖廟,並按古廟制立七廟。

秦始皇是靠武力併吞六國,他本身迷信個人威勢,對鬼神之說不是很熱切,所以在他的諸多訟德碑文中只強調皇帝本身的權威與武功。但為了以神權鞏固政權,以其秦朝傳統宗教信仰,他仍然依循傳通祭祀上帝,並且實現當初春秋齊桓公想要做的封禪之舉。秦始皇於泰山封禪,招集齊魯儒生博士討論封禪禮儀。但當初齊桓公的封禪禮儀未定,最後以故秦祭上帝之禮行封禪。他在泰山還有封藏,且密不示人,這個作法被漢朝漢武帝所效法。秦始皇認為自己上應天命,大地諸神必須尊崇自己。如南巡湘江時因舟船浮動而遷怒湘神,下令伐盡湘山樹木,火燒湘山祠。此外,秦始皇將五德始終說引入官方宗教,用以論證秦代周而立的合法性。

神仙傳說可追溯到戰國時期,一出荊楚文化,一出齊魯文化。楚人好幻想,如《莊子》、《楚辭》都有提到神仙的美妙形象,也就是長生不老、逍遙自在與神通廣大。而燕齊濱臨大海,海天明滅變幻,海島迷茫隱約,都引發當地人的聯想遐思。所以有渤海有蓬萊、方丈與瀛洲三神山的傳說,山上有仙人和不死之藥。實現個體永生的神仙信仰吸引了貴族與君王,也產生不少方士如徐福、韓終、侯生、盧生、石生等。秦始皇為求不死之藥,最後親自東巡黃海、渤海,到成山頭、登芝罘、遊碣石、至錢塘、上會稽。徐福謊稱受大海怪獸所隔,難出航尋不死之藥,秦始皇就親自芝罘出海射殺鯨魚。盧生稱尋仙藥不遇,因為惡鬼所害,建議人主應微行以避惡鬼,所居宮殿不應讓人知道,然後真人可至,不死之藥可得。秦始皇因此隱密其行,自稱真人。後來侯生等人不滿秦皇專橫而逃逸,秦始皇也因為他們的誹謗之言而坑儒報復。最後,神仙方術並未成為官方宗教,但大批方士的出現與活動,帶動方術仙道的發展,以及航海與文學的發展。

秦朝延续了春秋战国时期土木混合结构高台建筑的传统,可確認為秦代的建筑遗迹与雕塑实物十分有限,大型的高台宫室建筑遗址以陕西咸阳的咸阳宫遗址和辽宁绥中的姜女石遗址为代表。但最為人知,尤其能够表现秦朝建筑與雕塑群氣勢與魅力的实例为臨潼的驪山秦始皇陵與秦始皇兵馬俑。當時修築秦始皇陵的有七十餘萬人,前后延续三十余年。根據《史记》記載,司馬遷認為秦始皇陵應是「鑿穿了三層地下水,灌注銅水來填補縫隙,又修造宮殿,設置百官位置,放置奇珍異寶。用水銀做成百川江河大海,用機械來模擬江河的流動,頂壁裝有天文圖象,下面置有地理圖形,用娃娃魚的油脂做成長明燈。陵墓中還有許多機關。」。根据勘查,确实在秦始皇陵陵区周围发现汞含量异常,学者认为如果地宫打开过,汞会很快挥发。因此初步证实了陵内含大量水银的说法。

而秦始皇兵馬俑為形似戰陣之陶馬與陶製士兵,所有馬匹及士兵之大小長矮,皆與實物相同。士兵之高度均在 175 公分至 185 公分之間,可推測古代中國人的身材可能甚為高大。陶製之馬匹係每四隻一組,馬身頗為高大肥短,與漢唐時所見之戰馬有類似之處。初期發掘出土之兵俑約有六千之數,其後亦相繼發現銅製品及武器。士兵陶俑之手足均為實心,而頭、身及臂部則為空心,部分且塗有彩色。尤值一提者為所有陶馬及士兵均非模型所製,且表情不同、神態生動,可見秦代塑匠已有頗高之造詣,以及築墓耗工之巨。兵馬俑是秦軍的精確複製。由於兵俑完美的再現了秦軍的風範,軍事學者可以通過兵俑增加對「強秦」的了解。由於陶俑的原形來源於真實的秦軍將士,所以有研究者嘗試通過他們的容貌了解秦代的風土人情。

在湖北省發現之秦代墳墓內,保存完整者有毛筆兩支、竹簡一千一百片,並有漆盤一個。每片竹簡有上、中、下三孔,以供串連之用,其內容有關法律,字體為隸書。據此判斷,該墓的主人應為秦朝官吏,生前或與法律工作有關。尤以漆盤上之鳳凰與魚,與仰韶文化陶盤上之魚形裝飾,以及漢代鳳凰之流行,均有密切之關連。此一漆盤之出土,顯示漢代漆器風格仍襲秦代,鳳凰之流行亦早見於秦代。

秦代篆刻發展也比先秦時期更趨加成熟,具有典雅而靈動、平整而率意的藝術效果,至今仍受許多篆刻家模仿。另外有名的是「刻石」,有《嶧山刻石》、《泰山刻石》、《琅琊台刻石》、《芝罘刻石》、《東顴刻石》、《碣石刻石》、《會稽刻石》等七種。這些立石有政治意義,也有極高的藝術價值。



%% -*- coding: utf-8 -*-
%% Time-stamp: <Chen Wang: 2019-10-22 11:24:50>

\section{始皇帝\tiny(BC221-BC210)}

秦始皇(前259年1月10日-前210年8月10日),嬴姓,趙氏,名政,時稱趙政,史書多作秦王政或始皇帝,祖籍嬴国(今山东济南市莱芜区),生於趙國首都邯鄲(今河北邯鄲市),是秦莊襄王之子,商朝重臣惡來的第35世孫。出土《北京大學藏西漢竹書》第三卷中稱其為趙正。唐代司馬貞在《史記索隱》引述《世本》稱其為趙政。曹植《文帝诔》最早称始皇帝为嬴政,後世通稱嬴政,亦被某些文學作品稱為「祖龍」。他是戰國末期秦國君主,十三歲即位,先後鏟除嫪毐與呂不韋,並重用李斯、尉繚,三十九歲時滅亡六國建立秦朝,自稱「始皇帝」,五十歲出巡時駕崩,在位三十七年。

秦始皇是中國史上第一位使用「皇帝」稱號的君主。統一天下後,秦始皇繼承了商鞅變法的郡縣制度和中央集權,統一度量衡,「車同軌,書同文,行同倫」及典章法制,奠定了中國政治史上兩千餘年之專制政治格局,他被明代思想家李贄譽為「千古一帝」。但另一方面,秦始皇在位期間亦進行多項大型工程,包括修築長城、阿房宮、驪山陵等,施政急躁,令人民徭役過重,是秦朝在他死後3年迅速衰亡的重要原因。

秦始皇統一六國後,認為過去「 皇」、「帝」、「王」等稱號都不足以顯示自己崇高的地位,因而創造出「皇帝」這個新頭銜授予自己,自稱「秦始皇帝」,簡稱「始皇帝」:

「始」有最初、首次的意思,秦始皇希望自己的後繼者沿稱「二世皇帝」、「三世皇帝」,以至萬世傳之無窮。「皇帝」一詞主要引用「三皇五帝」的神話傳說,從中抽取「皇」字和「帝」字結合而成。秦始皇顯然希望透過這個頭銜,以示自己不遜於黃帝的地位和威望。「皇」的意思指「大」或「輝煌」,通常是古代中國人對「天庭」的稱謂,如皇天;皇穹(指天帝)等。「帝」的意思指「天帝」、「上帝」,古人想像中宇宙萬物的主宰。自称“皇帝”,是為了顯示其至高无上的地位和权威,是上天给予的,即“受命於天”,反映了他並不滿足仅仅做一個人间的统治者。

「秦始皇帝」和「始皇帝」的稱謂,首見於西漢太史公司馬遷著作的《太史公书》,即《史記》,其中「秦始皇帝」出自《秦本紀》,而「始皇帝」則出自《秦本紀》及《秦始皇本紀》。由於秦始皇首度将“皇”和“帝”两个字结合起来,故此秦始皇的正式稱謂應為「秦始皇帝」。

嬴政在秦昭王四十八年正月生於邯鄲,父親是秦公子異人(有一說其真實生父為呂不韋),母親是趙姬。

由於秦昭襄王採用了范睢「遠交近攻」的戰略,將近攻的對象選定為鄰國韓國和魏國,而和較遠的趙國停戰。按照當時的慣例,秦趙兩國需要互換人質(質子)以示真誠。秦國派到趙國的質子是秦始皇的父親異人(又名子楚,即秦莊襄王),異人當時是被秦昭襄王立為太子的安國君(即秦孝文王)的其中一個庶子。异人有20多個兄弟,加上他的母親夏姬不受安國君寵愛,因此他繼承王位的機會很低。更不幸的是,由於秦趙兩國已發生數次的軍事衝突,因此異人在趙國的待遇並不好。

異人就這樣成為趙國的人質,但是濮陽大商人呂不韋卻改變他晚年的命運。呂不韋有着遠大的政治抱負,覺得異人「奇貨可居」,很快成了異人的朋友。當時異人的父親安國君已即位為秦王,華陽夫人被立為王后,但華陽夫人無子,呂不韋便親自遊說華陽夫人之弟陽泉君,指秦王年事已高,如無意外子傒便會登位,必重用士倉,陽泉君的權勢就會煙消雲散。陽泉君大驚下問計於呂,呂不韋巧施簧舌,指异人才德兼备,可惜没有母亲在宫中庇护。王后倘若能立异人为太子,異人肯定会感念华阳夫人的恩德,作為國舅的陽泉君,在權勢方面也就得到保障。

陽泉君被呂不韋說服,同意遊說王后,王后便要求赵国将公子异人遣返秦国。

異人的返秦之路並非一帆風順,由於趙國不允異人返國,呂不韋不得不親自游說趙孝成王,指秦國不會因為一個質子而放棄攻趙,但若厚禮相送異人返秦登上王位,便可以贏得未來秦王的友誼,而且呂不韋更稱現時秦孝文王年事已高,一旦駕崩,趙國尽管以異人為質,秦國也隨時可以另立秦王,趙國只會一無所得,最終趙王同意異人返國。

自返秦後,吕不韋便让異人穿着楚服拜見王后,王后大悅並替異人改名為“楚”,史稱「子楚」。其後子楚在秦王面前展現自己的才華,秦孝文王大悅,在丞相面前稱兒子們沒人比得上子楚,同時在華陽王后勸說下將子楚立為太子。

秦孝文王在位時間很短,先是服喪一年,正式稱王後僅僅三天便駕崩,享年54歲。隨後太子子楚即位,也就是秦莊襄王,他任命呂不韋為丞相、兼封文信侯,賜食邑藍田十二縣,並繼續秉承「遠交近攻」的國策,對三晉(韓趙魏三國)開戰,以確立秦國的強勢地位。不過莊襄王在位時間也不長,三年後便駕崩,享年35歲,時年13歲的嬴政因而正式坐上秦王的寶座。

据《史記·秦始皇本紀》记载,秦王政是秦庄襄王子楚与赵姬所生。但《史記·呂不韋列傳》卻记载,呂不韋將趙姬餽贈給子楚時,已經知道她懷孕,亦即是說嬴政很有可能是吕不韦的私生子。東漢史家班固亦引用此說法,在他的著作《漢書》中稱嬴政為呂不韋之私生子。

歷代史家均相信《史記》所載,其中以清人梁玉繩為之辯論最力,但梁卻跟很多文人,包括明人王世貞及湯聘尹一樣,質疑「不韋獻匿身姬」的說法;在他所著的《史記志疑》中,根據唐初孔穎達《春秋左傳注疏》所云:「十月而產,婦人大期」,但《史記》卻以十二月作大期,此外亦指出若趙姬未足月就生下嬴政,子楚絕無不知的道理,而且修史很多時候服務於政治,為了證明本朝得位「順天應人」,對前朝故事總有曲筆、污衊之辭,可信性存疑,由於漢承秦祚、惡盡歸秦,司馬遷總不能公然在史書中翻案,故此特意記下生秦始皇的年月,至於一般人相信「不韋獻匿身姬」一事應為誤讀《史記》的緣故。翻譯《呂氏春秋》的外國學者约翰·诺伯洛克(John Knoblock)教授及杰弗瑞·瑞杰尔(Jeffrey Riegel)教授認為這種說法是用以誹謗秦始皇及呂不韋,而按照现代医学理论,“呂不韦生父论”中所谓的赵姬“怀胎十二個月”,根本就不可能成立,這是因為人類的妊娠時間只有266天(不足九個月)。

近代治先秦史名家馬非百不但質疑「不韋獻匿身姬」,更指《史記》中「奇貨可居」的版本實屬偽造,其中最重要一點是呂不韋封邑的記載。據《漢書·地理志》所言,秦代並無河南之名,莊襄王元年初置三川郡,至漢高祖時,始改名河南郡。呂不韋之采邑顯然不會是河南郡,而是《戰國策》所記載的「藍田十二縣」。此外,馬非百指司馬遷在記述六國史事時,多數引用《戰國策》,惟獨「奇貨可居」一事別據他說,令人奇怪。此外,「不韋獻匿身姬」一事只在《史記》記載,並沒有其他同時代的文獻記錄,按照郭沫若在《十批判书》的說法,在史學上屬於「孤證」,可信性成疑。

另一方面,郭沫若亦在《十批判书》提及其他疑点:

「不韋獻匿身姬」一事與春申君將懷有身孕的李園之妹,獻給無子的楚考烈王的情節雷同,過於巧合。《史記·吕不韦列传》又有「子楚夫人赵豪家女」之说,显然与前文「趙姬乃邯鄲諸姬絕好、善舞者」自相矛盾。

嬴政即位當年,晉陽發生叛亂,後被秦將蒙驁迅速平定。

由於年齡尚幼的關係,嬴政尊稱相國呂不韋為仲父主持國政,以蒙驁、王齮、麃公等為將軍。

當時各諸侯國的貴族為了鞏固其政治地位,都會專門招收人才。凡是投奔到其門下的,皆收留下來並供養他們,這些被供養的人一般稱為「食客」,供養食客眾多名揚天下的,則以戰國四公子(魏國信陵君、趙國平原君、齊國孟嘗君及楚國春申君)為首。呂不韋自地位鞏固後,感歎以秦國之強,居然在養士方面不如戰國四公子,因此亦大力招攬食客,並重金禮聘許多有學問的人,其門下食客一度達三千人。他更授意自己的食客編撰《呂氏春秋》,寫成八覽、六論、十二紀,共二十萬言。

據日本就实大学綜合歷史學系李開元教授的分析,當時秦國的國政,完全掌握在三大外戚勢力手中,分別是以嬴政的養祖母華陽夫人為代表的楚系外戚,嬴政真正祖母夏姬為代表的韓系外戚,以及嬴政生母趙姬為代表的趙系外戚,這三大外戚勢力的傾軋,深刻地影響嬴政即位初期的歷史。

成蟜为秦庄襄王之子,嬴政之弟,其生母有可能是祖母夏姬為代表的韓國系外戚的族人,被封为长安君。

公元前239年(秦王政八年),嬴政命成蟜率军攻打赵国,成蟜在屯留(今山西省屯留县)叛秦降赵。秦军攻占屯留后,成蟜的部下皆因连坐被斩首处死,屯留的百姓被流放到临洮(今甘肃省岷县)。成蟜投降赵国后,被赵悼襄王封于饶(今河北省饶阳县东北)。

對於成蟜叛變的原因,據李開元分析,有可能因為其主要靠山,韓系外戚的夏姬於秦王政七年去世,嫪毐在趙姬授意下,彻底清洗韩系外戚势力,领兵在外的成蟜被迫叛亂。吴裕垂、黄式三等认为是饶是成蟜生前所受封,也可作为其通赵的证据,更怀疑其系在叔父们支持下认为兄长秦王政非亲生而投靠赵国意欲在叔父们支持下夺位,但辛德勇结合了钱大昕、许宗彦、李慈铭的部分观点,指成蟜在深受兄长倚重的情况下,没有理由为了饶这个封地放弃长安这个封地并背弃家国,故认为被赵悼襄王封以饶的是另一个长安君,即赵悼襄王的叔父。辛德勇认为“反”应作“返”解,即成蟜是在出征时突然染病不得不班师,途中在屯留去世;成蟜死于军营中后,因屯留原属韩国,不乐属秦,故屯留卒“蒲鶮”趁机作乱并获得一定支持,失败后遭清算,即成蟜是没有叛秦的,这也就能解释他为何没得到赵国接应、死后也没有作为罪魁被戮尸。

有一說法指秦朝末代君主子嬰是成蟜的後裔。

自秦莊襄王亡故後,呂不韋跟太后趙姬(嬴政的母親)的關係一直藕斷絲連,不過這段不倫關係似乎是趙姬主動單方面展開的。隨著嬴政年齡漸長,吕不韦唯恐事情败露,灾祸降临在自己头上,就暗地寻求了一个陽具特别大的人嫪毐(làoǎi,音涝矮,粵音路藹)作为门客,並借機讓嫪毐在一場盛大的宴會中,表演陽具驅使桐木車輪轉動的把戲,故意傳到太后趙姬耳中。

正如呂不韋所料,太后果然對嫪毐大感興趣,於是呂不韋假裝不知情地將嫪毐帶進宮中,並找人假意告發嫪毐有犯腐罪之嫌。另一方面,呂不韋又暗中教唆太后,替嫪毐安排「給事中」的職務留在宮中。太后深以為然,暗中收買行腐刑的役人,不對嫪毐執行腐刑,但要他們對外宣稱嫪毐已受刑,並將嫪毐的鬍鬚拔除。嫪毐就這樣成為服侍太后趙姬身邊的侍宦。

太后完全為嫪毐的「巨陰」而痴狂,其間竟意外懷孕。為了避人耳目,太后與嫪毐捏造卜辭,指居於咸陽不利,雙雙移至秦國舊都雍(今陝西省鳳翔縣南),其間太后與嫪毐誕下兩子。

公元前239年(秦王政八年),嫪毐獲封長信侯,以山陽郡(今河南焦作市東南)為其食邑,又以河西、太原等郡為其封田。嫪毐門下最多時有家僮數千人,希望做官而自愿成为嫪毐门客的,也達到千餘人。

不過按照秦國的規矩,封侯可謂相當困難,例如王翦在滅楚前,曾向嬴政提到自己為將多年,仍未得封侯之賞,而王翦當時已經有消滅趙國,重創燕國的戰績。令人訝異的是,《史記》也沒有提到嫪毐封侯的原因。有學者推斷,嫪毐封侯不外乎軍功或告奸(商鞅法:告奸者與殺敵同賞),正是在嫪毐封侯的同一年,成蟜與蒲鶮在屯留發動叛亂,有可能是嫪毐平定成蟜之亂有功,他才有封侯的資格。

公元前238年(秦王政九年),22歲的嬴政按照慣例到秦國舊都雍舉行冠禮,其間有人向嬴政告發嫪毐為假宦,並與太后趙姬淫亂,甚至還試圖以其與太后所生之子為秦王,嬴政下令徹查。嫪毐決心孤注一擲,先發制人,遂偽造秦王與太后印信,引導其童僕門客和少數受騙軍隊發動政變,攻擊蘄年宮。

嬴政令相國呂不韋及有楚系外戚背景的昌平君、昌文君率兵平叛。嫪毐軍本是烏合之眾,不堪一擊,加之不得人心,很快就被擊潰。在懸紅銅錢百萬的重賞下,嫪毐被生擒,被送至咸陽後,處以車裂之刑,「夷三族」,其和太后所生的兩個兒子也被殺,其童僕門客皆被流放蜀地,太后趙姬則被囚在雍都。

可是在當時,禁錮親母始終有悖孝道,嬴政餘怒未消,下令凡為太后求情的,先用蒺藜责打,然后杀掉,为此有27位进谏者被殺。這時齊國人茅焦勸說嬴政,指出幽禁母親有損嬴政聲名,難以讓天下人信服;杀害进献忠言的大臣,會寒了天下人才之心,对收买六國人心、统一天下大业不利。嬴政頓時茅塞頓開,采纳了茅焦的建议,厚葬被杀的大臣,又亲自率领车队,前往雍地把太后接回咸阳,復居甘泉宮,母子关系得以恢复。茅焦因此事被尊为上卿。

不過亦有學者認為茅焦的一番話未足於令嬴政改變初衷,真正讓嬴政擔憂的,是趙系外戚勢力在嫪毐之亂中幾乎煙消雲散,讓華陽夫人為首的楚系外戚勢力成為最終的勝利者,出於帝王平衡的需要,嬴政才寬大對待嫪毐流放蜀地的門客,以及將趙姬迎歸咸陽,避免楚系外戚獨霸朝堂。

公元前229年(秦王政十九年),太后趙姬去世,谥号为帝太后,与庄襄王合葬在茝陽。

嫪毐叛亂最終牵连到相国吕不韦,嬴政打算誅殺呂不韋,但呂不韋畢竟令嬴政父親得以登上秦王之位,有擁立之功,而且有眾多呂不韋門客求情,於是打消了判處呂不韋死刑的念頭。公元前237年(秦王政十年)十月,嬴政以失職為名罷免吕不韦的相國职务,並把呂不韋放逐到其領地。

不過一年過後,吕不韦仍然名聲顯赫,有不少來自各诸侯国的宾客使者,專程拜訪呂不韋。嬴政惟恐呂不韋发动叛乱,在公元前235年(秦王政12年)向他賜下一封指責他的敕書,大致內容如下:

君何功於秦。秦封君河南,食十萬戶。君何親於秦。號稱仲父。其與家屬徙處蜀!
—《史記·呂不韋列傳》
呂不韋被下詔,命他與其族人遷往蜀地,想到嬴政不會放過自己,惟有服毒酒自殺,遺體被其食客偷偷安葬在洛邑北邙山。

嬴政对于呂不韋的舊部,與參加呂不韋葬禮的賓客,採取下列措施:

對於出身自六國的呂不韋門客,一律驅逐出境。對於俸禄在六百石以上的秦國官員,參加呂不韋葬禮者,即剝奪其官爵,流放至房陵。對於俸祿在五百石以下的秦國官員,不參加呂不韋葬禮者,同樣流放至房陵,但不剝奪其爵位。自嫪毐及呂不韋相繼死去後,嬴政怒氣稍歛,就让流放到蜀地的嫪毐门客都回到京城咸陽,並警告其臣下若膽敢像呂不韋、嫪毐等人不遵從正道處理國事的話,就會剝奪其官職,家人充當為奴。

京都大學名譽教授吉川忠夫推測,在嫪毐叛亂事件當中,嬴政查出背后跟呂不韋有關,實際已出乎嬴政的預料。至於陳舜臣則推測少年嬴政早有剷除呂不韋,獨攬大權的念頭。另一方面,從嬴政在嫪毐叛亂之後三年,才對呂不韋作出較嚴厲處分,以及對出席呂不韋葬禮賓客的不同處分,都顯示嬴政在政治上的慎重。

至此,嬴政徹底掃清了威脅自己王位的因素,大權獨攬,成為秦國名實相符的君主。

儘管嬴政即位初年年齡尚幼,但其麾下秦軍並未停止擴張的步伐,例如秦國將領蒙驁分別在秦王政三年(公元前244年)攻取韓國十三城,以及在秦王政五年(公元前242年)攻取魏國二十城,並在該地設置東郡。

鄭國渠在公元前246年(秦王政元年)开始建造,位於今日中國陝西省涇陽縣上然村涇出口一帶。建議者為來自韓國的水利專家鄭國,其真正身份是韓國的細作。

當時三晉之一的韓國聽說秦國喜歡大興土木,就想以建渠消耗秦國的國力,使秦國無法向東用兵,韓國便讓水工鄭國找機會遊說秦國,讓秦國鑿通涇水,從中山以西到瓠口修一條水渠,出北山向東流入洛水長三百餘里,用來灌溉農田。工程進行途中,鄭國的陰謀被發覺,嬴政打算殺掉鄭國。鄭國指自己雖然是為韓國做細作而來,但建渠不會為韓國延續多少國祚,而渠建成以後的確會對秦國大為有利。

實際上以秦國之強,也免不了受到自然災害的烕脅,秦王政四年(公元前243年)蝗災嚴重導致瘟疫流行,甚至逼使嬴政下詔鼓勵老百姓納粟受爵,在某程度上說明在關中地區建渠,以抵禦自然災害的必要性,因此嬴政最終命令鄭國繼續把渠修成。

自渠成後,淤積混濁的涇河水被引至灌溉兩岸低洼的鹽鹼地,面積達四萬多頃,畝產達到了六石四斗。從此關中成為沃野,再沒有饑荒年份,為併吞六國打下堅實基礎,該渠也因此被命名為「鄭國渠」。

雖然鄭國渠的建設本身對秦國利大於弊,但其企圖疲弊秦國的意圖,郤引起秦國本土大臣的警覺,認為山東六國出身的客卿根本不值得信任,在他們的慫恿下,嬴政頒布「逐客令」,驅逐一切出身六國的客卿。

據《史記·秦始皇本紀》記載,嬴政頒布「逐客令」的時間為秦王政十年(公元前237年),正值呂不韋因嫪毐之亂被罷免的時間,因此「逐客令」很有可能針對的是「養士三千」的相國呂不韋,以削弱呂不韋的勢力。呂不韋的門客之一,出身自楚國上蔡的李斯也在被逐之列。為了避免被逐的命运,李斯主動向嬴政上書,這就是著名的「諫逐客書」,內容大致如下:

臣聞吏議逐客,竊以為過矣。昔穆公求士,西取由余於戎,東得百里奚於宛,迎蹇叔於宋,求丕豹、公孫支於晉;此五子者,不產于秦,而穆公用之,并國二十,遂霸西戎。孝公用商鞅之法,移風易俗,民以殷盛,國以富強,百姓樂用,諸侯親服,獲楚、魏之師,舉地千里,至今治強。惠王用張儀之計,拔三川之地,西并巴、蜀,北收上郡,南取漢中,包九夷,制鄢、郢,東據成皋之險,割膏腴之壤,遂散六國之從,使之西面事秦,功施到今。昭王得范睢,廢穰侯,逐華陽,彊公室,杜私門,蠶食諸侯,使秦成帝業。此四君者,皆以客之功。由此觀之,客何負於秦哉?向使四君卻客而不內,疏士而不用;是使國無富利之實,而秦無強大之名也。

嬴政讀了李斯的上書,就廢除了“逐客令”,命人追回李斯,並恢復其官職。

韓非出身韓國公族,有口吃的毛病,與李斯都是戰國著名思想家荀子的學生,李斯自覺才學不如韓非,他們都信奉「性惡論」,認為人的思想容易受到環境左右。

自商鞅變法後使秦國強大後,秦國的統治階級便相當重視法家思想,並用之作為治國方針。另一方面,韓非自荀子處學成歸來後,多次上書韓王,但不為所用,於是寫下《五蠹》、《孤憤》、《顯學》、《難言》等著作。

韓非的著作流傳到秦國後,嬴政相當欣賞其所闡述的治國思想,曾说:“寡人得见此人与之游,死不恨矣!”,當從李斯口中得知作者為韓非後,即以戰爭作要脅,逼韓王命韓非出使秦國。

公元前233年(秦王政14年),韓非出使秦國,受到嬴政的欣賞,準備加以重用,李斯與姚賈怕嬴政重用韓非,私下誣陷韓非,說韓非是韓國宗室公子,必定不會效忠秦國,既然不能為嬴政所用,那就是一個禍患,勸嬴政把韓非禁錮在雲陽,嬴政深以為然。不過李姚兩人害怕嬴政反悔,為免夜長夢多,於是派人给韩非送去毒药,让他自杀。韩非想向嬴政自陈心迹,却又不能进见。

后来嬴政後悔,命人赦了韓非的罪名,但為時已晚,韓非已被毒殺。

在眾多韓非著作中,嬴政則比較欣賞韓非在《孤憤》、《五蠹》的理論闡述,例如韓非在《孤憤》中稱有才有谋的人,一定有远见并且能明察秋毫,不能明察,就不能照亮私暗处的奸邪;能执法之人,一定性格坚毅并且為人刚劲正直,不刚劲正直,就不能矫正奸邪。另一方面,韓非亦在《五蠹》論述,指作為明君,應不用有关学术的文献典籍(指詩經、書經),而該以法令为教本;禁绝先王的言论,而以吏為师;不提倡游侠刺客的凶悍,而只以杀敌立功为勇敢。这样,国内民众的一切言论都必须遵循法令,—切行动都必须归于为国立功,一切勇力都必须用到从军打仗上,才能奠定称王天下的资本

韓非亦毫不客氣稱學者(儒生)、言議者(縱橫家)、帶劍者(墨家俠者與俠客)、患御者(怕被徵調作戰的人)、工商買賣者等,為擾亂君王法治的五種人(五蠹),指這些人无益于耕战,就像蛀虫那样有害于社会。這些論調也間接影響嬴政日後作出「焚書」、「坑儒」的決定。

自鄭國渠建成後,关中变成了肥沃之地。至此,秦国的三大粮仓——巴蜀、汉中、关中就此全部建成,嬴政因而發動了歷時十年的統一六國戰爭。

李斯早在擔任呂不韋門客的時候,很快就得到面見嬴政的機會,他不動聲色地向嬴政獻上消滅六國之策,指現時正是吞併六國最好的時候,如果不把握這個時機的話,一旦六國中興,訂立合縱的盟約對抗秦國的時候,秦國要席卷中原,吞併六國就要付上相當沉重的代價。另一方面,魏國大梁人尉繚亦指六國與秦國相比,六國諸侯就像郡县的首脑,但六國合縱對付秦國卻是一桩大麻煩,為免重蹈智伯、夫差、齊湣王因敵手聯合攻擊而亡的覆轍,希望嬴政不要吝惜财物,给各国权贵大臣送礼,利用受賄的大臣,打乱六國诸侯的合縱计划,这样雖然損失些許財物,但卻可以消滅所有諸侯。

嬴政深以為然,听从了尉繚的计谋,为了显示恩宠,嬴政还让尉缭享受同自己一样的衣服饮食,每次见到他,总是表现得很谦卑,並任命李斯为长史,負責暗中派遣谋士带着金銀珍宝去各国游说。对各国能收买的六國權貴大臣,就多送礼物加以收买;不能收买的,就把他们杀掉;成功离间六国君臣关系後,嬴政隨即派良将随后攻打。由於秦軍戰勝所付出的代價甚少,故嬴政任命李斯为客卿。

嬴政在李斯、尉繚等人的協助下制定了「滅諸侯,成帝業,為天下一統」的策略。具體的措施是:籠絡燕齊,穩住魏楚,消滅韓趙;遠交近攻,逐個擊破。

嬴政首先选择的攻击目标为韩国,因为韩国的实力在六国中最弱,但是韩国还没有到不堪一击的地步,而且三晉唇齒相依,嬴政擔心秦滅韓時,趙國仍有助韓的可能,所以在滅韓之前,必需大幅削弱趙國。

公元前236年(秦王政11年),趙軍將領龐煖率領主力北上攻打燕國,想脅迫燕國一起進攻秦國。嬴政乘趙國進攻燕國之際,以救援燕國為由,派王翦、桓齮、杨端和率军两路攻打赵国,拉開了消滅六國的序幕。

當趙軍攻取了趙燕邊境的勺梁(今河北省定州市北)時,王翦出兵攻打赵国的上党郡;當趙軍攻取了燕國的貍(今河北省任丘市東北)時,王翦已攻克了赵国的阏与(今山西省和顺县)、橑杨(今山西省左权县)等六座城池。王翦统率军队十八天,让军中年俸禄不满百石的小官回家,每十人當中挑选二人留在军队。当趙軍攻取了燕國的陽城(今河北省保定市西南)时,桓齮攻克了赵国的邺城(今河北省磁县邺镇)和安阳(即新宁中,今河南省安阳市西南);當龐煖聞訊揮師南下救援時,秦軍已經將漳水流域全數吞併。秦軍與趙軍同步行動,趙國只奪得北方邊境的幾座城池,卻丟失了南方、西方的九座城池,實力大減。

公元前234年(秦王政13年),秦軍再度进攻趙國的平陽(今河北磁縣東南)、武城(磁縣西南),斬首10萬,大敗趙軍,並殺死趙將扈輒。趙國經過秦國這次攻擊后,國力大衰,僅能退守邯鄲自保。

不過秦軍並沒有打算罷手,同年十月,秦將桓齮又率秦軍東出上黨,越太行山深入趙國後方,大破趙軍,攻佔了赤麗、宜安(今河北省葶城西南)。公元前233年初(秦王政14年),秦軍進逼邯鄲,趙王遷急命北部邊防名將李牧為將軍,率領他的部隊南下,指揮全部趙軍抗擊秦軍。

李牧率邊防軍主力與邯鄲派出的趙軍會合後,在宜安附近與秦軍對峙。經激烈戰鬥後,秦軍大敗。桓齮僅率少量親兵衝出重圍,奔回秦國,史稱「肥之戰」。趙國奪回了被秦國佔領的土地,李牧因此戰受封為「武安君」。

儘管李牧在肥之戰挫敗了秦軍的兵鋒,但趙國只能僅僅自保,消除了趙國援韓的可能,秦國可算是達到了目的。

趙國國力被大幅削弱後,韓王安被逼在公元前233年(秦王政14年)向秦國稱臣。公元前232年(秦王政15年),秦軍分兩路進攻趙國,一軍至鄴城,一軍至太原,取狼孟,但其後在番吾再被李牧擊敗,史稱「番吾之戰」。

秦國雖然進攻趙國失敗,但並不能挽回韓國危如累卵的形勢。秦王政16年,韓國被迫割讓南陽一帶土地給秦國。當年九月,嬴政派內史騰去接受韓國所獻之地,由他代理南陽守之位,並开始命令所有秦國國內男子登记年龄,以便征发兵卒、徭役,為大規模的滅國戰爭作好準備。

公元前230年(秦王政17年),嬴政借著趙國發生大地震與飢荒,無力援助韓國的大好機會,命內史騰攻打韓國,並擒獲韓王安,韓國滅亡,將其國之地設置潁川郡,建郡治於陽翟(今河南禹州),但嬴政下令將「天下不軌之民」遷於南陽的舉措,卻造成韓國故地局勢不穩,秦國一度陷入新鄭騷亂及李信之敗的窘境。

嬴政自然不會因為滅韓而滿足,趙國同時發生大地震與饑荒,可說是消滅趙國的大好機會,於是大舉出兵,命王翦、楊端和為將,兵分兩路,南北合擊趙都邯鄲。赵国派李牧、司马尚率兵抵御,兩軍相持。

吸取了番吾之戰失敗的教訓,秦國遂派間諜賄賂趙國權臣郭開,要郭開離間李牧和趙王。郭開其後向趙王遷進言,指李牧、司馬尚欲謀反。趙王遷乃使趙蔥及齊國出身的將領顏聚,取代李牧為將。李牧拒不受命,更使趙王遷倍加相信李牧試圖謀反,因此使人暗中拘捕李牧並將他處決,並免除司馬尚的將軍之位。

公元前228年(秦王政十九年),秦國王翦軍破趙軍,殺趙蔥,俘顏聚,佔邯鄲,趙王遷被俘虜,趙國滅亡。赵公子嘉在邯鄲被攻佔後,率领他的宗族几百人到代地(今河北蔚縣東北)收拾殘部,並自立为代王,向东与燕国的军队会合。至於秦軍則暫停軍事行動,驻扎在中山,做好攻打燕國的準備。

王翦攻佔邯鄲後,嬴政親自到邯郸,找到当初与他在赵国时,與母親有仇的人,把他们全部活埋,其後经太原、上郡返回都城咸陽。

韓國遺民並未因為韓國淪亡而變得安於現狀,反而該地正醞釀著反秦的餘波。

雖然韓王安自被虜後,嬴政並沒有將他流放至蠻荒之地,目的是顯示寬容態度,以此怀柔韩国遗民,对其他国家的君王示以姿态,有利於秦國各個擊破,但自從荊軻刺秦事件後,嬴政對山東六國的態度大變,韓王安也被牽連,在秦王政二十年被迫离开韩国本土,以割断他和韩国遺民之间的联系。同时由于韩国并没有对秦国作殊死的抵抗,嬴政只是将韓王安迁徙至附近的郢陈(今河南省周口市淮陽縣),留下了温和的余地。

然而事情的发展与嬴政的预料相反,就在韩王安被迁徙的次年(秦王政二十一年),韓國遺民憤於國破王遷,在新郑爆发了大规模的反秦騷亂。雖然這次騷亂被鎮壓,但韓王安卻因新鄭騷亂受牽連而亡。

早在秦王政16年韓國獻南陽地的時候,迫于秦国强大的軍力,魏國亦主动向秦國献地求和,秦國在該地設置麗邑。此时嬴政正调集主力全力攻趙,不想分散兵力攻魏,就接受了献地,使魏国苟延殘喘。

公元前226年(秦王政21年),嬴政藉口楚王背棄獻出青陽(今湖南長沙)以西土地的承諾,並襲擊秦國南郡為理由,派王賁率大軍出函谷關,攻佔了楚國北部的十幾座城。在保障了攻魏秦軍側翼安全後,王賁旋即回軍北上突襲並圍困住魏國國都大梁(今河南省開封市西北)。大梁居於睢水、潁水、鴻溝的交匯之地,護城河十分遼闊,五座城門皆備吊橋,地形易守難攻。魏軍依託大梁的城防工事死守,秦軍強攻毫無奏效,王賁竟引黃河、鴻溝(汴渠)水灌入城內。

公元前225年(秦王政22年),大梁城被水浸近三個月,城牆崩壞,魏王假投降,魏國滅亡。嬴政在魏國地區設立碭郡,又建置泗水郡。

楚国是南方大國,疆域辽阔,山林茂密,物产丰富,号称拥有甲士百万。不過楚国的内政一直不振,总是贵族争权夺利,这种状况到战国末期尤为严重。儘管如此,三晉滅亡後,僅存的楚燕齊三國當中,以楚國最為強大。楚國亦隨之成為繼趙國後,統一戰爭中最大的絆腳石。

秦滅楚國的戰爭,史記記載得十分簡略,予人的印象是秋風掃落葉一般,但從《雲夢秦簡》(亦稱睡虎地秦簡)的《編年記》的記載來看,情況不完全是這樣,至少秦滅楚之戰是相當艱難的,而其中的關鍵人物就是昌平君。

昌平君是楚考烈王熊元的庶子,名啟,其父熊元返楚即位為楚王後,他與其生母滯留在秦國。由於他與華陽夫人同屬楚國王族,故此成為楚系外戚的主力,活躍於秦國政壇,更與嬴政聯手平定嫪毐之亂。呂不韋去相後,昌平君繼任為丞相。

當嬴政決定消滅楚國後,隨即召開廷議,參與者為秦國主要大臣,包括昌平君、李信及王翦,就攻楚方略提供意見(見《史記·白起王翦列傳》)。不過對於滅楚所需兵力,李信與王翦卻產生分歧。李信認為需二十萬人滅楚,但王翦卻稱滅楚需要六十萬人。嬴政傾向採用李信的方略,這是因為王賁滅魏國前,曾試探式攻擊楚國,輕易地取得楚國十餘城,讓嬴政得出滅楚不難的結論。王翦的话不被采用,就推托稱病,回到频阳家乡养老。另一方面,昌平君因對攻楚頗有微辭,故在秦王政21年被貶至郢陳。

郢陳原屬陳國國都,被楚國所滅後稱為陳縣。公元前278年(秦昭襄王29年),秦國名將白起攻陷楚國國都郢(今湖北省荊州市),在該地設置南郡。楚頃襄王被迫遷都於陳縣,故此亦稱為郢陳。由於郢陳以西與韓國國境相連,故此郢陳一旦發生糾葛,往往牽動秦韓楚三國。

郢陳一地雖然最終被秦國所佔,但其楚人勢力並沒有被消滅,反而一直成為楚人反秦的溫床,從秦滅楚之戰至陳勝吳廣的大澤鄉起義,楚人反秦的重要事件幾乎都與郢陳之地有關。因此嬴政將昌平君貶至郢陳,是有深刻的政治用意。由於昌平君長年仕秦,並協助平定嫪毐之亂,加上他沒有跟故國楚國聯繫,因而得到嬴政的信任。昌平君徙至郢陳的表面理由是負責監管韓王安,但實際上,嬴政卻希望利用昌平君楚國公子的身份,安抚郢陈地区的楚人,為滅楚作準備。

公元前225年(秦王政22年),嬴政遣李信、蒙武等將兵二十萬討伐楚國,朝東南方向深入楚國腹地,攻擊平輿(今河南汝南縣東南)和寢(今安徽省阜陽市臨泉縣),大勝楚軍,兵鋒指向楚國首都壽春(今安徽寿县),但是昌平君卻在這個時候據郢陳叛秦歸楚,截斷了南征秦軍的後路,郢陳楚人紛紛響應,而潁川郡的韓國遺民亦聞風而叛,秦軍形勢危急。

李信的征楚大軍隨即引兵向西攻佔鄢郢(即郢陳),然後接到潁川郡父城(今河南省平頂山市寶豐縣)告急的消息,於是李信、蒙武相約會師父城,以平韓人之叛。昌平君叛軍雖未能守住郢陳,但實力未損,他們緊緊追击秦軍,连着三天三夜不休息,结果在父城附近與韓人叛軍大败李信部队,攻入两个军营,杀死七个都尉,秦军大败而逃。

李信之敗,主要是因為王賁滅魏前,對楚國進行的軍事行動過於容易,讓他產生誤判,以為郢陳地區及附近郡縣的反秦力量已被消滅,沒有考慮到郢陳地區的楚人仍有相當可觀的反抗力量,雖然壽春楚王負芻的楚軍力量薄弱,容易攻取,但長年事秦的昌平君在關鍵時刻叛秦,卻給予李信軍致命一擊,終至一敗塗地。

楚國收復楚國故都郢陳為中心的失地,更趁勢西進深入至原韓國境內,再加上李信慘敗的消息,都令嬴政感到驚恐,更感大失面子,因為啟用李信及贬斥昌平君出京都是由他作決定。他不得不亲自前往频阳,登门造访被贬斥出京、还乡养老的王翦,強行徵召王翦攻楚。王翦本來以自己年老多病拒絕攻楚,但見嬴政面色不豫,擔心自己重蹈武安君白起的覆轍,不得已答應嬴政的請求,並稱攻楚需六十萬人,他要求全權指揮六十萬兵力,嬴政一一答應,並親自到到灞上為王翦的大軍送行。為了消除嬴政的疑心,王翦稱自己征戰多年仍未得以封侯,故希望嬴政赐予數量眾多的良田、美宅、园林池苑等,為子孫掙下一份家業,大軍出征期間,又不厭其煩地连续五次派使者回朝廷,请求赐予良田。

公元前224年(秦王政23年),王翦率領六十萬大軍沿著之前李信攻楚的行軍路線,直撲郢陳,苦战久攻不下。秦王政23年4月,與昌平君一樣同為楚國公子而仕於秦的昌文君戰死。同年另一路秦軍南至平輿,攻陷壽春,俘楚王負芻。

秦王政24年約3月,王翦軍攻破郢陳,嬴政親自出巡郢陳,以鎮懾反秦勢力,而昌平君在收到楚王負芻被俘的消息後,撤出郢陳,被楚军大将项燕拥立为楚王,在淮南地區繼續進行反秦事業。王翦和蒙武统领秦军消滅楚軍餘孽,昌平君戰死,项燕在淮北的蕲县(今安徽省宿州市蕲县鎮)兵敗自杀。

公元前222年(秦王政25年),王翦大軍平定了长江以南一带,降服了越族的首领,设置了会稽郡,楚國徹底滅亡。該年五月,秦国为庆祝灭掉楚国而下令特许天下聚饮。

燕太子丹為燕王喜之子,过去曾在赵国作质子,由於嬴政幼時在趙國生活,故與太子丹相當要好。其後嬴政被立为秦王,太子丹又到秦国作质子,但嬴政苛待太子丹,太子丹心生怨恨,最終逃归燕國。

太子丹逃返燕國後,打算報復嬴政對他的無禮,但燕国弱小,力不能及。他的太傅鞠武指秦國國土遼闊,敢戰之士眾多,且有山川之固,勸誡太子丹不要因為自己被嬴政欺侮的怨恨,而去触动嬴政的逆鳞。

正在此時,秦将樊於期(即桓齮)得罪了嬴政,逃到燕国,被太子丹收留。不過鞠武十分反對太子丹收留樊於期,他指嬴政肯定會借此事大造文章,遷怒於燕國,為了避免給嬴政藉詞攻燕,應該讓樊於期流亡匈奴,他提倡六國合縱,與北面的匈奴和好,才有對付秦國的可能。不過太子丹認為這個策略所需的時間太長,而且樊於期已走投無路,讓他投奔匈奴即是讓他送死,自己總不能夠因為害怕秦國,而摒棄樊於期的投奔,因此不接納鞠武的建議。

鞠武認為太子丹過於短視,為了結交樊於期而不顧國家大禍,心灰意冷下向太子丹推薦田光,而田光亦接受了鞠武的遊說,親自面見太子丹,再向太子丹推薦衛國人荊軻,最後田光自戕以激勵荊軻為太子丹賣命。

知道田光自戕的太子丹表現得相當悲痛,當時秦軍已經消滅趙國,駐兵中山,兵鋒直指燕國,燕国君臣唯恐大祸临头,因而與逃至代地的趙公子嘉結盟,共同防禦秦國。太子丹認為魏楚齊燕四國合縱已經沒有太大意義,打算派勇士前往秦國,像曹沫劫持齐桓公般,逼嬴政归还侵占各国的土地,若不答應則將嬴政殺死,使秦國國內陷入混亂。荊軻正是執行這件事的最佳人選,因此太子丹尊奉荆軻为上卿。

荊軻幫助太子丹完善刺殺嬴政的計劃,知道嬴政悬赏黄金千斤、封邑万户来购买樊於期的脑袋,就勸樊於期自殺。公元前227年(秦王政20年),燕國以荊軻為正使,秦舞陽為副使,帶同樊於期的脑袋和燕国督亢的地图,出使秦國献给嬴政。

荆轲带着价值千金的礼物,厚赠嬴政宠幸的臣子中庶子蒙嘉。蒙嘉在嬴政面前說盡好話,令嬴政大為高興,安排了外交上极为隆重的九宾仪式,親自在咸阳宫接見荊軻與秦舞陽兩人。荆轲捧着樊于期的首级,秦舞阳捧着地图匣子,按照正、副使的次序前进。走到殿前台阶下的秦舞阳脸色突变,害怕得发抖,大臣们都感到奇怪。荆轲以「秦舞陽為蛮夷,因未見過天子威嚴而心生恐懼」作解釋。嬴政令荊軻递上秦舞陽的地圖,荆轲展开地图尽头,一把淬毒的匕首露出来。

荊軻左手抓住嬴政袖子,右手用匕首刺向嬴政。嬴政大驚,站了起來,掙斷衣袖想要拔劍,卻因為劍身太長,拔不出來。荆轲追赶嬴政,嬴政绕柱奔跑。由於事情發生得太過突然,大臣们吓得發呆,大家都失去常态。而秦国的法律规定,殿上侍从大臣不允许携带任何兵器;武士都在殿下,沒有詔諭不能上殿。這時侍醫夏無且把一個藥囊向荊軻扔去,荊軻伸手擋了一下。嬴政趁這時把劍轉到背後拔出,回頭砍斷荊軻的左腿。荊軻倒地,將匕首扔向嬴政,惜撞在銅柱。嬴政向荊軻連砍八劍,武士衝上殿來,殺掉荊軻。事后嬴政评论功过,赏赐群臣及处置罪官都各有差别,其中以夏無且攔截荊軻有功,賞赐黄金二百镒。

嬴政大发雷霆,於公元前227年(秦王政20年)命令王翦、辛勝的军队跨過易水,大敗燕、代兩軍於易水之西。其後一年,秦軍攻陷燕國都城蓟。燕王喜、太子丹、代王嘉等人率领全部精锐部队向东退守辽东,王翦以稱病為由歸秦。秦将李信紧紧追击燕王與太子丹,太子丹隐藏在衍水河中,代王嘉則建議燕王喜殺掉太子丹,以求嬴政的寬恕。燕王喜接納代王嘉的建議,派使者杀了太子丹,把他的人头献给嬴政。當時秦军主力幾乎调往南线进攻楚国,燕王喜、代王嘉兩人得以在遼東苟延殘喘。

不過嬴政並未因而罷手,五年後(秦王政廿五年),王賁奉嬴政之命,掃除燕国在辽东的残余势力,俘虏了燕王喜及代王嘉,燕國徹底滅亡。

公元前264年,齊王田建即位,在位達44年,即位初期由母親君王后輔佐。

齊國因处在东部海滨,秦国頻頻进攻三晋及楚國,这四国面对秦国的进攻只有分别谋求自救,因此齊王建在位時期,齊國境內並没有遭受太大战祸,但這種和平卻是建基於其餘四國的犧牲,因為有這四國為田齊阻擋強秦,田齊才可以享有和平。另一方面,君王后生前對秦國處處忍讓,對四國不施以援手,甚至趙國在長平之戰戰敗後,也沒有勸誡齊王建及時援助趙國,結果隨後秦軍輕易地包圍邯鄲,趙國國力大削,為六國滅亡種下遠因。

君王后逝世後,秦國重金收買了齊國丞相,出身自君王后家族的后勝,使齊國即不合縱抗秦,也不加強戰備,齊王建甚至在公元前237年(秦王政十年)親自到秦国朝拜,嬴政在咸阳设酒宴款待。

秦國滅五國後,齊王建才頓感秦國的威脅,慌忙將軍隊集結到西部邊境,並断绝和秦国的来往,但為時已晚。公元前221年(秦王政26年),嬴政以齊國拒絕秦使者訪齊為由,命王賁在滅燕之後率軍南下攻齊,而蒙恬由于出身将门,剛擔任秦国的将军,亦有參與滅齊戰事,作戰勝利後被授予內史。

秦軍避開了齊軍西部主力,由燕國南部南下,一路勢如破竹,幾乎沒有抵抗,大軍直抵齊都臨淄(今山東淄博北),齊軍措手不及。齊王建聽從后勝的建議,不戰而降,後被迁到共城。秦军攻入临淄後,百姓居然没人敢反抗。不過事後齊國百姓反而埋怨齊王建不早與諸侯合縱攻秦,僅對秦國言聽計從,以致亡國,諷刺道:「松耶柏耶?住建共者客耶?」

齊國的滅亡也標誌著戰國時代的落幕,结束500多年來諸侯長期割據紛爭的局面,最终建立了中国历史上第一个中央集权君主統治国家——秦帝国。這一年秦王嬴政為三十九歲。

公元前242年(秦王政五年),秦軍併吞魏地二十城,在該地設置東郡,第二年(秦王政6年),韩、魏、赵、卫、楚五國組成聯軍进攻秦国,攻占了寿陵邑。秦国派出军队,五国停止了进军。秦国繼而攻下卫国,卫君角惟有率领他的宗族迁居到野王,而魏國的河內郡因山势险阻而得以保全,但朝歌卻被秦國攻佔,至於衛國原本的首都濮陽則合併至秦國的東郡。

秦滅六國後,衛國仍然被嬴政保留。直至秦二世元年,才下令將其解國,廢衛君角為庶人,衛國從而成為最後一個被秦國所滅的關東國家。因此有學者認為,這可能連繫著秦始皇皇后身分的歷史之謎,也就是说,嬴政的皇后(或王后)有可能出自姬姓卫国公族,而公子扶苏有可能为卫国来的皇后(或王后)姬氏所生(有另一說指扶蘇生母為楚國公主),这也解释了秦二世胡亥即位後立即废掉卫君角的原因,即秦二世矫诏杀死其兄扶苏后,剪除东方起义军利用扶苏和其外戚卫国的影响上的威胁。

歷代不少文人都曾經探討過六國被秦國所滅的原因,其中以位列唐宋八大家的三蘇父子(蘇洵、蘇軾、蘇轍)的六國論較為著名。蘇洵認為六國破滅的原因是「弊在賂秦」。蘇軾則認為秦國因養士而輕易地滅六國,而統一後又因視養士無用而速亡,大概是出於宋朝士子張元、吳昊兩人因累試不第而叛宋投夏的感慨。至於蘇轍則以地理戰略的角度著手,指山東六國必需保有韓魏兩國作屏障,韓魏兩國位處中原,當兩國不保,其餘四國只能被秦國各個擊破。

事實上,六國之亡是主要受到內因外因等多個因素影響,三蘇的論點合起來,才比較接近秦能統一六國的真相,但仍然有不足之處。孔子曾說過「足食,足兵,民信之矣。」,秦人能做到足食、足兵、國人信服這三點,只要上台的不是昏君,自然能兼併六國。例如秦人有關中沃野之利,又有巴蜀的鹽鐵之利,不懼山東六國的鹽鐵封鎖,北有鄭國渠,南有都江堰,已做到「足食」一點;軍功授爵,首級易功,士卒敢戰,已做到「足兵」一點,而商鞅徙木示信,則做到「國人信服」一點,因此才對山東六國造成壓倒性優勢。

除此之外,秦国武器制造的流程与制度十分嚴謹,每一件兵器从生产工人、仓库保管、工场的责任人,一直到中央政府的总监制者,都要实名在产品上记录。这种制度稱為「勒名工官」,既是产品质量监管制度,也是产品流通监管制度,可以从头到尾追踪每一件武器产品的行踪,因此秦國兵器精良無比,1982年發現的秦代銅戈「十七年丞相啟狀戈」正是具體例子(現收藏在天津市博物館)。

此外,秦國地理位置優越,其西南北三個方向都沒有強大,或者是勢均力敵的敵人,故可專心一致向東方擴張。即使征東失敗,有崤山、函谷關之固,亦可足以自守,休養生息,坐待山東六國互相攻伐。相比起三晉的地狹人眾來說,秦國可說是地廣人稀,故此秦國經常招徠三晉百姓到秦國種地,並賜田宅,免兵役,專事耕織,這樣秦國本土百姓就可完全投入兵役,輪番作戰。

史書記載當時秦國百姓勇悍,卻甚為單純,以兵戎來說則秦最強悍,三晉次之,齊人最怯。秦國為免民風變得柔弱,故採取措施以盡量減少秦國百姓與關東百姓的接觸。如公元前325年秦惠文王遣張儀取魏國陝縣後,即將當地人遣回魏國。秦國即使招徠三晉百姓開墾,也只把他們安置在新闢地方,不與秦國百姓接觸。保持強悍民風,亦是秦國消滅六國的主要因素。

史家杜正勝在其著作指出,戰國時代各國都透過稅制及戶籍制度對百姓直接支配,稱為「編戶齊民」(編入戶籍的農民),承擔國家賦稅,作為國家的主要經濟支柱。不過到了戰國後期,六國農民朝不保夕,經濟破產,無法支持戰爭的開支及徵召,六國的賦稅基礎受到破壞,因而敗亡。

六國合縱本是嬴政在統一戰爭最大的絆腳石,但六國本身各懷鬼胎,且山東六國之間亦時常發生戰事,例如趙燕兩國已經因連年戰爭,而成為生死大敵,在公元前242年(秦王政五年),燕王喜因趙國長期遭受秦國攻擊,主將廉頗又出奔魏國,燕王喜不但沒有打算援趙抗秦,反而有意趁火打劫進攻趙國。出征前燕王喜曾詢問劇辛關於龐煖的情況,劇辛說龐煖容易對付。燕王喜於是以劇辛為將攻打趙國,趙國派龐煖迎戰。最終趙軍大勝,俘虜燕國兩萬人,劇辛被擒殺。從上述例子可見六國合縱關係十分脆弱,終被秦國逐一輕易擊破。

嬴政用武力統一六國後,推行了一系列的政治措施,在政治、經濟及文化方面,均對後世造成巨大的影響。

西周初年,只有周天子才可稱為「王」,但自平王東遷後,周室衰落,楚吳越三國分別僭越稱王,而到了戰國時代,周天子權威更形低落,其間發生「五國相王事件」,各諸侯不但各自稱王,還互相承認對方君主的王位,一時間「王」的稱號亦大幅貶值,因此「皇、帝」之名開始形成。當時各國諸侯為了合理化自己的政權及統一的依據,紛紛從上古史中找出根源,甚至為自己王族編造古代帝王譜系。如戰國後期齊湣王及秦昭襄王互稱東西兩帝。

嬴政統一六國後,下令说:“寡人以眇眇之身,兴兵诛暴乱,赖宗庙之灵,六王咸服其辜,天下大定。今名号不更,无以称成功,传后世。其议帝号。”於是丞相王綰、御史大夫馮劫、及廷尉李斯等人商議说,五帝的土地雖然廣闊,外面还划分有侯服、夷服等地区,諸侯是否覲見不由天子控制,不少諸侯更加是聽宣不聽調,嬴政的功業已經超越三皇五帝,古代有天皇、地皇、泰皇等稱謂,以泰皇最尊贵,因此向嬴政獻上「泰皇」的尊號。頒布律令的稱為「制书」,詔告天下事件的称为「诏书」,印章稱「璽」,所說的話稱「諭」,群臣稱其為「陛下」。天子不再自稱為「寡人」,改称为「朕」。

嬴政接受王綰等人的大多數建議,但對於群臣向自己獻上的尊號「泰皇」,則將其泰字去掉,采用上古「帝」一字,称为「皇帝」,並追尊其父莊襄王为太上皇,废除谥法。自稱「始皇帝」,后代則称二世、三世直到万世。百姓則稱為「黔首」。

另一方面,嬴政又命李斯將和氏璧(一說是藍田玉)磨成玉璽,亦即是後世相傳的「傳國玉璽」,玉璽大小為四寸方形,由咸阳玉工王孙寿将和氏璧精研细磨,玉璽上方雕著五條龍,上寫八個蟲鳥篆字,根據記載和現存拓片有「昊天之命、皇帝壽昌」、「受命於天,既壽永昌(見《三國志·吳書》)」、「受命於天,既壽且康(見《應氏漢官》、《皇甫世紀》)」以及「受天之命,皇帝壽昌(史家裴松之說法)」四種說法。

秦始皇採用戰國陰陽家鄒衍的五德終始說,認為黃帝屬土德,有黄龙和大蚯蚓出现。夏朝得木德,有青龙降落在都城郊外,草木长得格外茁壮茂盛。商朝得金德,所以才从山中流出银子来。周朝屬火德,所以出現赤烏的祥端。秦繼周而興,加上秦始皇的先祖秦文公在出獵時遇上黑龍,故秦應屬水德。為配合水德的特性,以十月為歲首,年初朝賀改至十月一日進行,色尚黑、終數六,因而規定衣服旄旌節旗皆尚黑,符傳、法冠、輿乘(天子乘輿六尺、車駕六馬)等制度都以「六」為數,並更改黃河的名稱為「德水」。其他水德的特性包括方向尚「北」,季節尚「冬」等。因為水主陰,陰代表刑殺,秦始皇以此作為其加重嚴刑酷法的依據。

由於秦代國祚短暫,加上記載殘缺,對於秦代的官僚制度只能作出間接推斷,但無論是司馬遷的《史記》,還是東漢班固所著《漢書》,都指漢承秦制,因此對於秦始皇時期的官僚體制,主要是根據《漢書·百官公卿表》的記載。不過無論如何,秦始皇確立百官體制,稱「三公九卿」,並沿用後世的說法,仍為人廣泛接受。

關於「三公」一詞出現的時間,應在春秋之末。至於「九卿」则自周初至战国初期,未曾发现各國採用此官制的證據,但可以確定「九卿」一詞出現在春秋魯定公、魯哀公在位期間,且与‘三公’连在一起。先秦文献中关于九卿之说的确切记载是《吕氏春秋》,另外《呂氏春秋·十二纪》中也有类似的制度描述,可以视为‘九卿’一词的最早出处。

秦始皇統一六國後,確立百官體制,以丞相總理國政·太尉掌軍事;御史大夫掌糾察和监察工作,統稱為三公。「三公」之下設「九卿」掌控宮廷事務,包括郎中令(宮中保安)、治粟內史(掌財政)、奉常(掌宗廟禮儀)、太僕(掌皇室輿馬)、廷尉(掌刑獄)、少府(掌山海池澤)、衛尉(掌宮門衛兵)、典客(掌「蠻夷」事務)及宗正(掌宗室事務),但因為文獻記載的缺乏,故未能確定太尉及御史大夫是否真正履行實務。至於九卿的數目开始时只是象征式,只是观念上之官制,并未尝为「九」之数字所拘。直至东汉才将观念上之九卿,坐实为事实上之九卿。

另一方面,秦代官制亦設博士七十二人,以備諮詢,又設將軍(掌征討)、將作少府(治宮室),所有官員都由朝廷委任,隨時調動任免,並須每年考績。秦始皇建立的官僚制度,成為歷朝政治制度的典範。

自秦國消滅六國後,朝堂上對於如何管治六國故地,究竟是採用沿用已久的郡縣制,還是復行封建展開討論。當時丞相王綰認為,燕国、齐国、楚国地处偏远,若果不在此三地分封宗室,此三地就無法真正被朝廷控制,因此建議秦始皇分封諸子,以屏藩皇室。這個建議儘管得到大部分大臣支持,但當時擔任廷尉的李斯卻力排眾議。

李斯以周王室作例子,指周王室亦分封不少姬姓公族,但他們的後代卻因為血緣疏遠,紛紛割據,互視對方為寇仇,周天子根本無力阻止諸候之間的爭端,而且諸侯根本不再將王室放在眼內,最终周朝分崩离析。分封諸子只會導致日後諸侯割據,兵甲不息。要使天下安寧,只能推行郡縣制,对于皇子功臣,用公家的赋税重重赏赐,這樣才可以消除天下人的野心。

秦始皇認同李斯的意見,指諸侯王才是擾亂天下的禍端,正是因為他們的存在,天下人才苦于连年战争无止无休,渴求和平。如果重新分封諸王,只會重演春秋戰國時,各國互相攻伐的苦況,因此決定把天下分为三十六郡,直至秦亡為止,共置五十四郡,每郡都设置郡守(掌民政)、郡尉(掌軍政)、郡监(掌監察),由中央朝廷委任,不得世襲,並須向朝廷匯報租稅、戶口及治安情況。以下是秦代初期三十六郡的資枓:

據劉宋裴駰《史記集解》記載,秦初三十六郡分別是:

關中巴蜀地區:內史、上郡、隴西郡、北地郡、雲中郡、九原郡、漢中郡、巴郡、蜀郡
河南地區:三川郡、南陽郡、潁川郡、碭郡、薛郡、東郡、琅邪郡、齊郡
河北地區:上谷郡、漁陽郡、右北平郡、遼西郡、遼東郡、代郡、鉅鹿郡、邯鄲郡、上黨郡、太原郡、雁門郡、河東郡
淮南地區:南郡、九江郡、鄣郡、會稽郡、泗水郡、黔中郡、長沙郡

秦始皇決定在地方治理上採用三級行政區劃制度,構建中央—郡—縣—鄉—亭—里—什—伍—戶的縱向金字塔式的控制體系,以加強對地方的控制,主要在郡設郡守,郡下設縣,縣設縣令,但秦代的国家行政机构只下设到县級。縣之下實行「鄉亭制」(亦稱乡里制度),設「鄉」,具體辦法是在「鄉」級設「三老」掌教化,「嗇夫」聽訟和徵稅,「游徼」偵緝盜賊。此外,鄉級以下則利用民間力量來治理,一鄉轄十亭,設亭長;一亭轄十里,設「里魁」(亦稱里正);一里轄百家,五家為伍,十家為什。這種制度為秦朝帶來了巨大的動員能力,即使三百多年前的波斯居鲁士大帝创建帝国之时,阿契美尼德王朝唯有省(萨特拉庇)和县作为行政区划机构。。

除此之外,歷來盤踞在淮河流域的淮夷、泗夷早已變為民戶,而秦始皇統一天下後,在今福建省一帶地方設置閩中郡。由於秦始皇認為閩中郡遠離中原,是「荒服之國」,地處偏遠,山高路險,而且越人強悍,難以統治,故實際上並未派遣官吏往閩中,只是廢去當地酋長無諸及騶搖的王號,改稱他們為「君長」,並讓其繼續統治該地。

郡縣制在全國推行後,標誌著六國故地不會再有擁兵割據,自選官吏,財政獨立的諸侯。此外,百姓的籍貫不再用以前所屬的諸侯國,例如楚國人、齊國人等作為識別,而是用其所處的郡名以咨識別。

不過自秦滅漢興後,漢高祖劉邦認為秦王室推行郡縣制後,因缺乏宗室屏藩,過於孤立而亡;又害怕諸侯王太強盛,造成皇室的衰弱,於是推行郡國並行制,施行郡縣制,但也封劉氏宗室子弟在各大要地為王,作為折衷。劉邦的作為,導致日後的吳楚七國之亂。因此歷代統治者都試圖在郡縣制至分封制之間,取得一定的平衡,以維持國祚。

為了防止六國遺民作亂,秦始皇下令收集天下的兵器,聚集到咸阳熔化,铸成大钟及十二个铜人,每個銅人个重达二十四萬斤,放置在宫廷里。

公元前215年(秦始皇32年),秦始皇進行第二次巡遊,途經碣石,宣布拆除原關東六國建造的城廓及堤防,稱為「墜城廓、決堤防、夷險阻」,並在碣石山门刻石立碑,碑文內容如下:

遂興師旅,誅戮無道,為逆滅息。武殄暴逆,文復無罪,庶心咸服。惠論功勞,賞及牛馬,恩肥土域。皇帝奮威,德幷諸侯,初一泰平。墮壞城郭,決通川防,夷去險阻。地勢既定,黎庶無繇,天下咸撫。男樂其疇,女修其業,事各有序。惠被諸產,久並來田,莫不安所。羣臣誦烈,請刻此石,垂著儀矩。

譯文:皇帝兴师用兵,诛灭无道之君,要把反叛平息。武力消灭暴徒,依法平反良民,民心全都归服。论功行赏众臣,惠泽施及牛马,皇恩遍布全国。皇帝振奋神威,以德兼并诸侯,天下统一太平。拆除关东旧城,挖通河川堤防,夷平各处险阻。地势既已平坦,众民不服徭役,天下都得安抚。男子欣喜耕作,女子修治女红,事事井然有序。皇恩覆盖百业,合力勤勉耕田,无不乐业安居。群臣敬颂伟业,敬请镌刻此石,永留典范规矩。

至於拆除原關東六國所建造堤防,原因是為了消除地方割據,這是由於各國堤防的設計不合理。根據《孟子·告子篇》的記載,魏國著名水利專家及商人白圭曾向孟子指,自己治理水患的水平勝過大禹,但孟子卻不以為然,指責白圭只是將鄰國當作蓄水的溝壑,即是將洪水引向別國,令百姓厭惡。

此外,據《漢書·溝洫志》的記載,当时齐國和赵魏兩國是以黄河为界,赵魏两国位處黃河上游,地势较高,齐国的地势低下,黄河泛滥时齐国所遭受的灾害就较严重,因而齐国首先沿着黄河建筑了一条离河二十五里的堤防,以防止黄河的泛滥。自从齐国沿黄河筑了堤防,令黄河泛滥的水流冲向赵魏两国,于是赵魏两国也沿着黄河建筑了一条离河二十五里的堤防。從以上記載可見,戰國諸侯都有利用水利設施,作為削弱敵國的手段,因此秦始皇才強調採取「決通川防」的措施。

「墜城廓、決堤防、夷險阻」等措施,除了防止六國遺民據險作亂外,還有方便交通,促進貿易的作用。

秦始皇自消滅六國後,便下令遷徙關東六國富豪十二萬戶入咸陽,其中以齊楚兩地出身的富豪為主。這是因為齊魯之地崇尚「齊魯學」,崇尚聚眾講學,追求典雅,與主張謀富強,尚功利,務實際的秦國主流思想「三晉學」截然不同,學術思想的衝突很容易形成反秦輿論,齊魯儒生譏諷秦始皇封禪一事正是明證。不過相比起齊國來說,秦始皇更為擔心楚國故地,例如他往楚國故地巡視期間,史籍可見其在江東金陵、丹徒、曲阿等地掘地厭天子氣的記載,而且「亡秦必楚」一說,亦反映楚人強烈的復仇慾望。因此徙民以齊楚之地為主,也就變得理所當然。

另一方面,遷徙富豪還有其他政治用意,除了可繁榮首都外,更可避免富豪與六國貴族互相勾結。此外,富豪們在本地兼併土地,放高利貸,造成地方不安定因素,將他們遷徙至咸陽,亦有助消除地方勢力。

秦始皇為統一人們的思想,大造天神,統一文字,強行教化。使小篆和隸書為全國通行的字體,對中國文化、政治發展產生了深遠的影響。

秦始皇统一天下后,得悉六國文字各不相同,决定推行「書同文」政策,按照當時任廷尉的李斯所奏,廢除關東六國原有文字,將史籀大篆簡化为小篆(亦稱秦篆),作為全國通用字體,于是令李斯作《仓颉》七章、赵高作《爰历》六章、太史令胡毋敬作《博学》七章作为全国规范字帖,皆取材于周宣王时期的大篆《史籀》十五篇,但現時世人皆稱李斯為小篆的鼻祖,其相傳書跡有《泰山刻石》、《琅琊台刻石》、《嶧山刻石》和《會稽刻石》等。

西汉时期,闾里书师将三篇以六十字为一章合并为《蒼颉篇》,共五十五章。西汉时期又称《仓颉》、《爰历》、《博学》为三苍,但因多种原因,《苍颉篇》文字大都失传。

另一方面,據唐朝張懷瓘所著《書斷》記載,秦朝下邽(今陕西省渭南市)人程邈(字元岑)在獄中花了十年時間,創造隸書,得到秦始皇賞識,被任命為御史。由於小篆難以在奏事繁多的環境下使用,故此程邈發明的字體頗受歡迎,而因為程邈字體起初专供隶役应用,所以把这一书体称之为隶书,但正史沒有關於程邈的生平記載。

度量衡傳統上是计量长度、体积、重量單位的统称。度是用來计算长短,量是用來測量體積,衡則用來計算物件重量。戰國時代,各國都採用不同的度量衡標準,致令在換算過程中顯得十分混亂。

以計算物件體積的「量」來說,各國都有計算「量」的標準,例如魏国的量制以益、斗、斛为单位,齐国的量制以升、豆、区、釜、钟为单位。據《左傳》記載,姜齊的定制為四升为一豆,四豆为一区,四区为一釜,十釜为一钟。齊國田氏門閥为了夺取姜氏的齐国政权,收买民心,故改变量制,以五升为豆,五豆为区,五区为釜,十釜为钟。另一方面,以量度物件重量的「衡」來說,赵国的衡制以釿、镒为单位。楚国以铢、两、斤为单位。据楚墓出土的砝码测量,楚制一斤平均值是260.798克。

各地度量衡制度的不同,顯然不利統一,故此秦始皇在公元前221年(秦始皇26年)下令統一全國的度量衡,以商鞅變法時制定的秦度量衡作為標準,規定標準度量衡器具須由官府負責監製,民間不得私鑄,亦規定六尺為步,二百四十步為一畝。秦代度量衡器具的测算标准为:一尺为23.1厘米,一升为201毫升,一斗为2010毫升,一斤为256.25克,一石(120斤)为30.75公斤,每件標準度量衡器具均刻上以小篆寫成的銘文。

不過秦始皇也明白,要在短時間內推廣單一的度量衡標準,可說是相當困難,甚至有可能引起一場大混亂。因此一般推斷,秦始皇會在統一度量衡的詔書下達前,先設立一個過渡期,暫時承認各地的度量衡標準,並製定嚴格的換算率作為折衷辦法。

统一货币可說是秦始皇在经济领域方面的重大举措。秦始皇三十七年,重新發行錢幣,把原來的圓形方孔錢推廣全國,廢除原來全國各地相異的貨幣,克服過去商品流通使用和換算的困難,使貨幣在全國範圍內有更佳的流通、支付和儲蓄的價值作用。

战国时期各国使用的货币,無論在形狀、價值都不尽相同。当时通行的货币主要有四种形态,布币是三晉地區主要的流通貨幣,形状像农具中的镈(bó,博);此外,齊燕趙三國亦流行使用呈小刀狀的「刀币」,有些刀幣更可長達十六公分;秦地、西周东周、赵魏沿河(靠近秦国)之地則流行「圓錢」;至於郢爰与铜贝,則只在楚國流通。當時貨幣經濟仍未確立,各國自鑄貨幣,甚至私人鑄幣,貨幣價值則取決於其中的含銅量,含銅量愈高,貨幣的價值愈高。

秦始皇因各地幣制紊亂,遂廢止關東六國原本通行的珠玉、龟贝、银锡等貨幣,改以黃金為上幣,以鎰(二十兩)為單位,圓形方孔的銅錢為下幣,文曰「半兩」,直徑一寸二分,重十二銖。

需要注意的是,「半兩錢」只是對秦國銅錢的統稱,不是指銅錢的實際重量。據《史記·平準書》記載,秦錢既重,又輕重不一,故此難以使用,並非一種理想的貨幣。另一方面,「半兩錢」圓形方孔的設計,亦非秦始皇首創,班固《漢書·食貨志下》記載秦錢「質如周錢」,可見在秦始皇統一六國前,已經有類似的貨幣設計。不過無論如何,圓形方孔的銅錢設計可以用繩索綑綁成串,攜帶方便,遂成為東亞各國貨幣的原形。

秦始皇在統一六國後,大力推行重農抑商的措施,他在瑯琊台刻石明白寫著:「皇帝之功,勸勞本事。上農除末,黔首是富。」,其具體措施是把商人和罪犯、奴隸作為首先「謫戍」(充軍)的對象,讓他們長期地出外遠征,並到新征服的荒涼地方開墾,如河套地及桂林、南海等地,使邊地得到開發。當時秦徭役法規定,首先徵召有罪吏、贅婿及賈人(即商人);其次徵召曾為商賈的人;再其次徵召祖父母或父母曾為商賈的人。此外,富人也得先服徭役,稱「發閭右」,最後才徵召貧弱人家,稱「發閭左」。

不過秦始皇並非鄙視商人本身,相反他對某些富商大賈,他仍給予一定的尊重,例如當時有個經營畜牧業的大亨乌氏倮,牲畜多到以山谷为单位来计算数量。秦始皇诏令乌氏倮位与封君同列,按规定时间同诸大臣进宫朝拜。另一方面,巴蜀有個名叫「清」的寡妇,其先祖自得到朱砂矿後,竟独揽其利达好几代人,家产也多到不计其数,更以弱女子之身守住先人的家業。秦始皇认为寡妇清是个贞妇,除了對她以客礼相待外,為了表揚她还为她修筑了「女怀清台」。

據劉宋裴駟《史記集解》引東晉徐廣敍述,指在公元前216年(秦始皇31年),秦始皇下詔「使黔首自实田」,但對於此詔令的真正意思,史學界眾說紛芸。有學者在其著作稱,此詔令是指按国家规定数额,让黔首自己设法占有足额土地,不再保证按规定或階級授田,并认为这是战国授田制的崩溃,此外更指西漢初年的「名田制」正是「使黔首自实田」政策的延續。

除此以外,秦始皇31年時的糧價高得離奇,米价每石為一千六百钱(平常年份糧價為每石30錢),但該年卻沒有任何大型自然災害的記載。由於史料缺乏,故無從得知糧價高昂與「使黔首自实田」的因果關係。

另一方意見則指,秦始皇不太可能讓百姓自行占有土地,因為自商鞅變法以來,秦國一直以授田制把农民束缚在土地上,保证官府对农民的人身控制,而土地原則上是國有的,由國家授田給百姓耕種。从控制百姓的角度來说,給予百姓自行占有土地的自由,即等于削弱自身對百姓的控制力,而控制力一旦削弱,国家就会引發动乱,这恰恰是秦始皇所不愿看到的。

因此,「使黔首自实田」一句未必指廢棄秦國原有的土地制度(國家授田制),土地国有的政策并未根本改变,該詔令可能是要求新近得到授田的百姓,去「充实」從國家领到的土地,努力垦荒,专心农耕,不要弃农经商。

秦始皇在位期间大兴土木,主要是为了国家的安全和建设,但也给当时百姓带來繁重的徭役负担。

公元前214年(秦始皇33年),秦始皇派將軍蒙恬率领三十万人攻擊匈奴,占据河套。為了抵御北方游牧民族的侵略,秦始皇下令將秦赵燕三国修筑的旧城牆連接起來,从临洮到辽东绵延万里,成為“万里长城”的前身。在修築北方长城的同时,原诸侯国用以「互防」的城牆被拆毁。

秦长城可大致分为西段和北段。西段起于今甘肃省岷县,循洮河北至临洮县,经定西县向东北至宁夏固原县、甘肃环县、陕西靖边、横山、榆林、神木,然后向北折至今内蒙古托克托南,抵黄河南岸。北段即黄河以北的长城沿阴山西段的狼山,向东至大青山北麓,再向东经今内蒙集宁、兴和至河北尚义,再向东北经今河北张北、围场,再向东经抚顺、本溪后向东南,终點在漢樂浪郡遂城縣(今朝鲜清川江入海处)。與現代留存的明長城相比,秦長城的工程規模遠遠不及。現時靠近九原的內蒙古自治區固陽縣內,便殘存著秦朝建造的長城遺跡。

不過后世不少人都質疑秦始皇修長城的意義。蒙恬死后没多久,匈奴便轻松跨过秦长城,不但收复了原来的失地,并且侵入至燕郡、代郡。可见长城本身在抵禦匈奴的作用並不若想象中大。

秦始皇從公元前220年(秦始皇27年)開始,大幅修築以京師咸陽為中心,向四面八方延伸出去的馳道,类似现代的高速公路,將秦故地和原六国境内的旧道连接起来,并加以扩建。李斯正是其中一名負責馳道修築工程的大臣。

馳道本來是天子專用的道路,修築馳道最初只是為了方便始皇巡遊,軍事用途不過是附帶的功能。據《漢書·賈山傳》記載,秦馳道宽50步(合今6.9米),约隔三丈(合今7米)载一颗树,用来计算道路的里程。驰道两边根据当地情况,种植杨,柳,槐,榆等树。驰道的路基加厚,呈“龟背形”,形成一个缓坡,有利于排水,著名的馳道包括上郡道、臨晉道、東方道、武關道、西方道、秦棧道及秦直道。

在眾多馳道中,以秦直道最為著名。據《史記》記述,秦始皇下令修秦直道的主要目的是為了巡遊天下,故在秦始皇35年(公元前212年)任命蒙恬興建由甘泉宮(位於雲陽,今陝西省咸陽市淳化縣西北)直達大漠深处的九原郡直道,達一千八百里,工程相當浩大,堪称两千多年前的军用高速公路,不但要削平高山,還需要劈山填谷,而且所經之处地势险恶、人迹罕至,甚至越过海拔1800米的子午岭而不回避,然而这条直道没能完成,但在今天陕西省北部的大山中,直道的遗迹依旧清晰可见。

據考證,秦直道是完全純人工打造,以每六至七厘米为单位,將黃土固定打硬後,再舖上一層相同厚度的黃土,然後將之又打硬,這種建築方法稱為「版築法」,以版築法打硬過的土壤,會變得非常堅實,植物無法在這種土壤上發芽生根。以每六至七厘米打硬黃土,從而成為一千八百里的直道,可以想像耗用的人力及錢糧必然是十分驚人。

可是,秦代道路並不像后世的水泥路般堅實,下雨時道路即成為一片泥濘,戰車會在泥濘地面留下車轍的痕跡,天晴後路面就會留下堅硬的車輪溝痕。車輪相距(即車軌)不同的戰車,會因與道路上留下的溝痕不同,而有行進困難的情況,這正是秦始皇推行「車同軌」(統一車距)的背景。

秦始皇為方便運送征討嶺南所需的軍隊和物資,下令開鑿南通百越的運河。經過實地檢查,發現流往東北注入長江水系的湘江,及由桂林往南注入南海的灕江,二條河川之間距離很短,以現時單位計算只有約四十公里。如果將這二條河川鑿通,長江水系便可與南海相通。

然而,湘江與灕江之間的水位差距甚大,工程接連失敗,直到秦始皇命使監「祿」(史書稱史祿)開鑿運河,終於在公元前214年(秦始皇33年)以三十六道水門,完成溝通長江水系與珠江水系的工程,被稱為秦鑿渠,後因灕江的上游為零水,故又稱零渠、澪渠。唐代以後,方改名為靈渠,俗稱陡河。

靈渠位於廣西省桂林市興安縣境內,全長36.4公里,寬十米,由鏵嘴、大小天平、泄水天平、南渠、北渠、秦堤和陡門等子工程組成。鏵嘴是靈渠最主要的分水設施,位於興安縣城東南2.5公里的湘江之中。建造時以長石疊砌四周,中間用砂卵石回填而成,高約6米,寬23米,長90米,前銳後鈍,狀似犁鏵。鏵嘴將湘江水分為兩股,其中七分水被分水垻所阻,因而沿著大天平,經北渠流到湘江,三分水經小天平和南渠注入灕江,即所謂的「湘七灕三」。自貫通後二千多年來,一直是嶺南與中原地區之間的水路交通要道,此項工程在1988年被定為全國重點文物保護單位之一。

近代以來,隨着粵漢鐵路和湘桂鐵路的通車,靈渠內的航運逐漸停止,但直至現在,靈渠仍被當地居民用作生活用水及灌溉用途。另一方面,現時不再以水門,而以作「堰」的方式,提高湘江的水位,以解決灕江與湘江之間水位高低差的問題。

早在秦始皇統一六國期間,便已開始為自己的宮室大興土木,每逢灭掉一个诸侯,都在咸阳北面的山坡上,仿造該諸侯國的宮室,範圍从雍门往东直到泾、渭二水交会处,殿屋之间有天桥和环行长廊互相连接起来,並把虜來的美人和鐘鼓樂器放在裡面。

據《史記·秦始皇本紀》記載,在建造阿房宮期間,秦始皇已打算在关中建造三百座宮殿,关外建四百座宮殿。另一方面,史記亦記載在秦始皇35年(公元前212年)時,咸陽城方圓二百里內已經有二百七十座宫观,以天橋、甬道相互连接起来,並把帷帐、钟鼓和美人都安置在里边。

阿房宮亦稱阿城,位於渭水南岸,雍州長安縣(今西安市)西北十四里,本來在秦惠文王統治期間建造,但惠文王逝世卻令宮殿的建造擱置。

秦始皇統一天下後,認為現時所居的宮殿狹小,不符合自己皇帝的身份,而且位於渭水北岸的咸陽人煙稠密,擴展宮殿規模受到限制。當得悉周文王、周武王分別建都的豐、鎬两城,都是位於渭水南岸後,認為該地才是所謂的帝王之都,故打算在渭水南岸的上林苑中建造「朝宮」,首先建造前殿,稱為「阿房」,這正是「阿房宮」的名稱由來。

不過以上只是阿房宮名稱由來的其中一種說法,唐朝顏師古為漢書作注時,指「阿房」指宮殿之四阿,皆稱為「房」。另有說法指「阿房宮」座落在地势高峻的丘陵上,「大陵若阿」,亦是阿房宮名稱的由來。除此之外,「阿」有「近」的意思,「房」與「旁」相通,「阿房」即近旁之意,因該宮靠近咸陽,所以稱為阿房宮。另有人指「阿房」一詞乃秦始皇寵妾的名字。可是,司馬遷在《史記·秦始皇本紀》記載,指當時秦始皇對這個名稱並不滿意,準備等阿房宮修成後,改為更好的名稱,結果未能如願,阿房宮這個名稱便一直沿用下來。

對於阿房宮開始建造的時間,《史記》卻有兩個不同的記載。《史記·六國年表》記載開始建造阿房宮的時間為秦始皇28年(公元前219年),但同書的《史記·秦始皇本紀》卻記載為秦始皇三十五年(公元前212年),多數人把後者當作阿房宮的始建時間,較合理的解釋是,秦始皇28年是嬴政意欲新建「朝宮」的時間,其間醞釀和規劃用去幾年時間,到秦始皇35年才下令動工。

阿房宮的建造規模相當宏大,雖然阿房宮實際上只是渭南「朝宮」的其中一個部分(即前殿),但據《史記》所述,其面積達東西五百步(693米),南北五十丈(116.5米),高數十仞(約11.65米),上面可以坐上万人,下可建為五丈旗,在里面运送酒菜要用车和马才行;四周架有天桥可供驰走,从宫殿之下一直通到南山。在南山的顶峰修建门阙作为标志。此外亦修造天桥,从阿房跨过渭水,与咸阳连接起来,以象征天上的北极星、阁道星跨过银河抵达营室星。另據《三輔舊事》所述,阿房宮的宮門以磁石製造,亦稱「却胡門」,四夷朝拜時若有兵刃藏身,必然被發現。

為了完成如此宏大的工程,秦始皇下令徵集隱官刑徒70餘萬人,並从渭北山嶽开采山石,从蜀地、荆地运来木料,分別修建驪山陵和阿房宮,为此迁徙三万家到骊邑,五万家到云阳(今陝西省咸陽市淳化縣西北),都免除十年的赋税和徭役。兩年之後,秦始皇在東巡途中病逝,九月被埋葬在驪山。由於覆土驪山的需要,遂將阿房宮建設工地上的勞力徵調到驪山陵。第二年即四月復建阿房宮,但當年冬天(公元前209年),數十萬起義軍洶湧而至,修建隊伍被迫停止。除去因覆土驪山暫停的7個月,共施工了2年7個月。

可是,由中国社会科学院考古研究所和西安市文物保护考古所联合组成的阿房宫考古队,对阿房宫遗址进行的考古工作发现,阿房宫从来就没有建成,仅是完成地基而已,而項羽入關中後,焚燒的是位於渭水北岸的秦咸陽宮,並不是一般認為的阿房宮(位於渭水南岸),考古學家在發掘原咸陽宮遺址時,發現了大量的灰燼和紅焦土,證明項羽確實曾縱火焚燒過咸陽宮。

秦始皇的另一个宏大工程就是花了三十餘年時間,修建自己的陵墓,後世稱為秦始皇陵(亦稱驪山陵),與阿房宮等大型工程一起,共動員了近七十萬人,陵墓高五十餘丈,周回五里,從渭水北岸的山嶽運取石料。由於驪山一帶盛產黃金,南面的藍田以盛產美玉聞名,故此被秦始皇視為風水寶地,因而定此為自己的長眠之地。

自秦以後,秦始皇陵即被人認識到其特異性,東漢蔡邕在其著作《獨斷》一書中,指古時並沒有祭拜墳墓的風俗,因為當時人們相信,人死後靈魂仍永遠存在宇宙間,所以並不將靈魂脫離後的遺體視為祭拜的對象。對死者的祭拜是在「宗廟」內舉行,直到秦始皇將「寢」從宗廟脫離,改置在陵墓旁,世人才有在陵墓外祭拜的風俗。漢承秦制,也在陵墓旁安置寢殿,還準備了讓死者生活舒適的各式各樣設備。

據《史記·秦始皇本紀》記載,秦始皇嬴政自從登基為王開始,便已著手在驪山營造陵墓,統一天下後,即徵召天下罪犯為他營造帝陵。鑿穿三泉,以銅製外槨(棺),墓內設計有如宮中一樣,表現出百官就位的模樣,至於絕品器皿及珍禽異獸等也從宮中移至墓室。墓室內點燃著以人魚油脂製成的蠟燭,祈願其永遠明亮不灭。

為了防止盜墓者進入,秦始皇下令工匠在墓室中裝設可自動發射的弩弓,並以水銀模擬天下河川及大海,以機關使之流動,如真實世界一樣。據推斷,由於水銀易揮發的特性,在墓室內注入水銀,彌漫的水銀蒸氣不但可令入葬的屍體和隨葬品保持長久不腐爛,而且水銀蒸氣具劇毒,大量吸入可導致死亡,因此地宮中的水銀還可毒死盜墓者。

除此以外,《史記·秦始皇本紀》亦記載秦二世命人在秦始皇墓外栽种草木,从外边看上去好像一座山,並下令凡沒有子女的先帝(秦始皇)後宮妃嫔,都要殉葬;為了防止工匠盜墓,貪取墓室財寶,所有參與修造墓室的工匠,不待他們出來,就封閉墓門,被活埋在陵墓裏。據考證,秦陵西侧发现大量暴露在外的修墓人骸骨。

秦始皇兵馬俑博物館的鎮館之寶——「銅馬車」,出土時間為1980年11月,發現地點為陕西省临潼縣秦始皇陵西侧通往地宫的甬道中,由數千件零件組成,雖只有兵馬俑實際大小的二分之一,但考古學家要用近十三年時間(1980年至1993年)修復兩輛「銅馬車」,可見秦朝時期的工藝水平已經達到相當高的水準。
秦始皇兵馬俑博物館的鎮館之寶——「銅馬車」,出土時間為1980年11月,發現地點為陕西省临潼縣秦始皇陵西侧通往地宫的甬道中,由數千件零件組成,雖只有兵馬俑實際大小的二分之一,但考古學家要用近十三年時間(1980年至1993年)修復兩輛「銅馬車」,可見秦朝時期的工藝水平已經達到相當高的水準。
《古今圖書集成·坤輿典》引述史料《漢舊儀》指,李斯是秦始皇陵工程的主持者,他曾向秦始皇报告,称其带了72万人修筑骊山陵墓,已经挖得很深了,好像到了地底一样。秦始皇听后,下令「再旁行三百丈乃至」。關於「旁行三百丈」的意思,專家作出解釋,指修陵人从地宫向南挖巡游通道时,遇到了大砾石,最后不得不顺着砾石层改向挖掘,即所谓的「旁行三百丈」。

2006年,秦陵考古队队长段清波率領一眾考古學家,用遥感和物探的方法分别對秦始皇陵进行探测,查明地宫就在封土堆下,距离地平面35米深,东西长170米,南北宽145米,與墓室均呈矩形状。至於墓室則位于地宫中央,高15米,大小相当于一个标准足球场,周圍建了一圈很厚的細夯土墻,即所謂的宮墻,東西長約168米,南北長約141米,南墻寬16米,北墻寬22米。

此外,考古學家亦發現秦始皇陵周圍地下存在規模巨大的阻排水渠,底部由厚達17米的防水性強的清膏泥夯成,上部由84米寬的黃土夯成,規模大到難以想象。秦始皇陵考古隊隊長段清波指,《史記·秦始皇本紀》中記載的“穿三泉”中,“三”其實是個概數,其實應該是指在施工過程期間遇到水淹,所以才修建阻排水渠,正好擋住了地下水由高向低滲透,有效保護墓室不遭水浸。

從商周到漢代,帝王的墓道通常都為4條,分別貫穿東南西北4個方向,這是尊貴身份和地位的象徵,而普通官員和百姓的墓道為一條或兩條,但考古學家發現,秦始皇陵只有東、西兩條墓道,這出乎考古學家的意料之外。

除此以外,秦始皇陵的封土堆體積龐大,堪称国内之最,但封土从何而来則尚無定论。儘管有文獻記載指封土堆「取土魚池」,但秦陵考古队队长段清波质疑這個說法,指封土堆的土壤樣本含雜大量沙石,但取自鱼池里的土却是纯净的黄土,且粘性甚强,极少含有沙石,而且認為秦陵陵区地势南高北低,且落差很大,从山下的鱼池取土显然要费力的多。因此,取土于鱼池一说值得商榷。

另有專家根據《史記·秦始皇本紀》「復土驪山」的說法,指封土堆其實是從墓穴中挖来,但經秦陵陵区高光谱遥感探测,在秦陵南部的骊山脚下发现了一处南北走向串珠状的巨型凹陷。经实地勘查,这个深达30米的巨型凹陷有明显人工挖掘的迹象,而凹陷的土质也与封土相同,這也是對封土堆由來的新推斷。

另一方面,中國地質調查研究院研究員劉士毅指,秦始皇地宮內水銀含量的確存在異常,如果以水銀的分布代表江海的話,正好與渤海、黃海的分布位置相符。若查明屬實,正好說明秦朝時期已經有對中國地理作出調查和研究,可說是新的歷史發現。

目前中國政府並沒有對秦始皇陵動土發掘的計劃,主要考慮到著手發掘後,極有可能出土龐大數量的歷史文物,而在尚未完全確立妥善的保存方法下,實在不宜發掘。

秦兵馬俑坑位於秦始皇陵封土以東約1.5公里處,有戍衛陵寢的含義,是秦始皇陵其中一部分。

兵馬俑是在1974年3月29日被陝西省臨潼縣村民楊志發、楊彥信、楊泉義等人發現,當時考古學家一致認為此遺跡規不會太大,但很快發現兵馬俑的規模比想象中要大得多。專家推測兵馬俑遺址呈長方形狀,東西長二百三十米,南北長六十二米,約埋藏七千多座與實物等身的兵馬俑。不過超乎預期的發掘成果,卻引來保存兵馬俑文物方面的困難。

當時的中國國務院副總理李先念得悉發現兵馬俑坑的消息,即指示國家文物局與陝西省政府合作,迅速採取相關措施保護文物。因此挖掘工程暫時中止,在不損及遺跡的前提下,耗費近兩年時間在遺跡上矗立著一座橢圓形的巨蛋建築,並在1979年10月正式對公眾開放,命名為「秦始皇兵馬俑博物館」。

按照中國古代建築講求的對稱原則,在陵園的西門、北門和南門應也有兵馬俑坑,但考古人員多番實地勘查,一無所獲,對於兵馬俑坑置於陵園東側的原因,目前有多種說法,現摘錄如下,但以下說法目前尚無定論。

秦始皇在位期间还扩大了国家的疆土,疆域異常遼闊,北至河套和阴山,南至南越(即古時日南郡,今越南中部),西至陇西臨洮,东则延伸到了辽东及朝鮮。

正當戰國七雄相互攻伐的時候,北方草原的匈奴、東胡及月氏亦相互牽制,未能全力南下干涉中原局勢,直到秦始皇統一六國後,居住在中國北方與西北草原的遊牧民族,隨即成為新生秦帝國的最大敵人。

公元前215年(秦始皇32年),燕國人盧生向秦始皇獻上一本名為《錄圖書》的預言書,上面寫著「亡秦者胡」。根據這個「預言」,秦始皇認為最終滅亡大秦帝國的,是北方的胡人。當時北方草原的胡人當中,月氏和東胡都比匈奴強大,但因匈奴人的聚居之地是河套地區,對定都咸陽的秦帝國來說是最大的威脅,因此秦始皇決定遣兵征伐匈奴。

不過李斯卻指出,匈奴人逐水草而居,居无定所,也从来不储蓄粮食,極难征服。如果派兵輕裝前進,容易導致軍糧斷絕而全軍覆沒;如果携带大量粮食进军,物资沉重难运,也是无济于事。對秦國而言,即使得到匈奴的土地也无利可圖,只會令中原百姓疲惫。遇到匈奴百姓,亦因擔心他們的忠誠而不敢役使他們。如果殺掉全部匈奴百姓,這又與君父自許的「君王」作為相悖,表明反對征伐匈奴。

秦始皇沒有接納李斯的意見,他下令將領蒙恬率兵三十萬人,沿著今日蘭州市至鄂爾多斯市一段的黃河,北伐匈奴,企圖奪取黃河河套以南之地。次年蒙恬攻佔河套以南地區,从榆中(今甘肅省蘭州市)沿黄河往东一直连接到阴山,划分成四十四个县,並設置九原郡,沿著黃河修筑長城,於險阻設置要塞。

其後,秦始皇又遣蒙恬渡过黄河去夺取高阙、阳山(即陰山)、北假一带地方,筑起堡垒,逐次向北推進勢力。當時的匈奴單于頭曼被迫北遷,秦軍在外奔波達十餘年,並屯駐在上郡一地,蒙恬之名威震匈奴。

據考證,秦軍迅速取勝的最大原因,就是普遍使用弩弓作遠程武器。持弩的秦騎兵射擊的準確度是匈奴人的弓無法比擬的,匈奴人的皮甲也抵擋不住弩箭強大的穿透力。對馬背上的匈奴騎手而言,弩是最致命的武器。中國著名的兵書經典《武經七書》指出,弩是對付古代遊牧民族的最有效武器。由於弩的結構過於複雜,對匈奴人來說,他們很難裝配或仿製。

為了充實河套以南新設郡縣的人口,秦始皇下令將罪犯遷移至新郡縣,當地被稱為「新秦」。當年(秦始皇33年下半年)在西方出現彗星,為不吉之兆,故此秦始皇再下令貶謫執法不公的獄吏往新闢之地,以修筑长城及戌守南越地区。

據主父偃在《諌伐匈奴書》所言,蒙恬所戍之地地勢低窪,而且是無法種植五谷的盐碱地,實在未能供給三十萬戍卒的兵糧,故此必須仰賴內地的補給。為了從黃腄(今山東省烟台市)和琅琊等沿海之地運送糧食至北方的黃河,原本僅一石的東西,必須花費三十鍾(192石)。正因如此,秦朝百姓迫不得已肩負著沉重的負擔。即使男子拚命耕種,也生產不及軍役所需的食糧;女子日以繼夜紡織,也無法滿足軍用帳幕所需。百姓疲憊,孤兒寡婦及老弱病殘者無法生活,倒在道路兩旁死去的人,屢見不鮮。直到蒙恬死後,在河套以南戍邊的勞役因陳勝起義,關東大亂而军心不稳,很快散去,從而迫使秦帝國放棄河套以南之地。

現存史料提及有關秦國南征百越的歷史,可說是相當簡略。對於秦軍南征百越的經過,軍事部署、作戰環境以及秦軍面對的軍事困難,《史記》的記載甚至可以用「一鱗半爪」來形容,但可以肯定南征百越的時間,實際上遠比北伐匈奴要早。

根據漢文帝元年(公元前179年)南越王趙佗所述,他已經在嶺南地區生活了49年。由於《史記》記載趙佗籍貫真定,並非嶺南當地人,故此可以參考他在嶺南生活的時間,從而推算出秦始皇何時開始南征百越。自文帝元年往上推朔49年,為秦王政20年(公元前227年),正值趙國被消滅的次年,可見在趙國滅亡後不久,趙佗隨即因為趙人的身份而被徵召入伍,成為征越大軍的其中一名士兵。因此可以肯定秦始皇在消滅六國的同時,便開始進行征服五嶺以南的軍事行動,而且更持續了十多年之久。

嶺南自古屬於「瘴疠之地」,亦稱「百越」,意指當地生活著眾多部落。漢初文學家賈誼自被貶為長沙王太傅後,亦聽聞長沙氣候潮濕多雨,以為自己會早死。漢人畏長沙如此,比長沙更南的嶺南,其環境之惡劣可想而知。即使是驰骋疆场半生,未尝言苦的伏波將軍馬援,自受命討伐在交阯地區發動叛亂的徵氏姐妹以來,在平定叛亂後亦自言後悔未有听从弟弟少游的话,做个平民百姓,反而從軍平越。當時漢朝已統治嶺南地區達百餘年,土地半垦,道路粗通,郡、县、乡、里四级政权已建立,已有不少越人漢化,其自然环境和人文环境已有巨大变化。馬援在提及征越之戰時,仍猶有餘悸,恐怕客死異鄉,可以想像到秦朝時期,嶺南地區的自然及人文环境,较二百年後的马援时代恶劣十倍,而現存的史料均指,秦始皇對號稱「瘴疠之地」的嶺南採取軍事行動的最大目的,是看中當地出產的犀角、象牙、翡翠及珠璣等物,以及揚威海外。

秦征百越的經過,綜合現存史料可分為四个阶段:

約公元前227年(秦王政20年),秦始皇下令使尉屠睢以南郡為後勤重地,在當地征发“吏卒”、“新黔首”、“甲兵”及各种军输物资等所謂「樓船之士」,沿湘江水道為主要補給線,深入百越之地。南征秦軍一路勢如破竹,越人紛紛逃入深山密林,逃避秦人的統治。

秦征百越之前,由於百越地區的部落們已經矛盾重重,甚至多數相互攻擊,難以共同抵禦秦軍的攻勢,但問題是百越地區部族眾多,儘管各自為戰,但越人逃入深山叢林的舉措,令秦軍難以徹底消滅其有生力量,加上秦軍水土不服,當地瘴癘橫行,致使軍中大疫,非戰鬥減員情況日益嚴重,戰事陷入長期化,正如《淮南子》記載的「三年不解甲弛弩」。

在戰事中期,秦軍擊殺了當地最大部落的首領,西甌國首領譯吁宋,但越人並未屈服,反而推舉桀駿為新的首領,這個新推舉的首領在一次夜襲中大破秦人,並擊殺征越秦軍最高統帥屠睢。

秦滅六國,百戰百勝,但在百越地區的征戰居然令秦軍「伏屍流血數十萬」(《淮南子.人間訓》語),甚至連主帥也被擊斃。秦始皇大怒,為了迅速結束戰爭,惟有下令對百越增兵,而楚國之亡又讓秦軍兵力得到釋放,秦始皇乃「发卒五十万」,分為五軍:一軍塞鐔城(今湖南靖州縣境)之嶺,一軍守九嶷(今湖南寧遠南)之塞,一軍處番禺(今廣東廣州)之都,一軍守南野(今江西南康境),一軍結餘干(今江西余干境)之水。五路并进,企图一举攻下百越全境。

不過兵力的增加,再加上越人反抗,利用熟悉地形的優勢,不斷襲擾秦軍勉強維持的補給線,令秦軍幾乎陷於崩潰,因此秦始皇下令使監祿(即史祿)負責轉運糧餉。

為了支持長期战争的物资消耗,史祿決定派遣士兵,開鑿連接湘江與灕江之間的運河,後世稱為靈渠,使中原和五岭以北地区之兵员和粮草,能通过水运进入百越地區。由於开凿灵渠屬於龐大工程,按当时技术条件,至少也得三至五年时间才能完成,故此戰事陷入僵持。

“灵渠”的建成只是为战争提供較好的后勤保障,并不能保證最後的勝利。秦征南越统帥尉屠睢战死後,继任统帅尉任嚣,起初亦无良策对付越人之丛林战,最後才想出了军队屯垦和“以谪徙民,与越杂处”之持久战和移民同化的策略。

屯垦和移民战术是控制百越地區最有效的方法,但见效慢,至少须五六年时间才能初见成效。由於當時秦始皇已消滅六國,故有充分條件遷移關東六國的「新黔首」(即所謂「中縣之民」)往百越地區「移民戍邊」。對於这些“新黔首”,秦始皇及秦國官吏根本不把他们当人看,把他们留在百越地區,与“越人杂居”,对秦始皇说有双重好处:一是消灭了六国“乱民”或“惰民”;二是稳定了百越地區。

公元前214年(秦始皇33年),自下詔徵發嘗逋亡人、贅婿及賈人等往百越地區後,秦始皇認為嶺南「粗定」,故在當地設置桂林、南海及象郡三郡。次年(秦始皇34年)始皇又下令貶謫執法不公的獄吏戌守百越地區。自秦王政20年令屠睢攻百越,至秦始皇33年嶺南「粗定」,對百越地區的戰事共進行了十三年。

儘管百越地區粗定,令嶺南地區自此納入秦國版圖,但秦國卻為此付出極其沉重代價,長期的征越戰事導致百姓靡敝,民不聊生,而時任南海尉的任囂得悉朝廷因對越戰事的緣故,中原疲敝已极,就留居南越,称王不归,并派人上书,要求朝廷征集无婆家的妇女三万人,来替士兵缝补衣裳。秦始皇只同意给他一万五千人。于是百姓人心离散、土崩瓦解,密謀造反的十家有七。

另一方面,移民「戍越」的黔首,多數出身自六國故地,他們為了逃避「戍越」,紛紛逃入深山密林,淪為盜賊,從而成為始皇崩後,六國故地相繼叛亂的導火線。

秦始皇一統天下,十年期間六次巡遊近全土三分之一,以示強、威服海内,但亦採懷柔手段舒緩昔六國反抗未平之情緒;如襲齊、魯之禮行封禪,依齊人八神信仰登成山祭日主。《嶧山刻石》曰:「登於繹山,群臣從者,咸思攸長。追念亂世,分土建邦,以開爭理。功戰日作,流血於野。自泰古始,世無萬數,陀及五帝,莫能禁止。廼今皇帝,壹家天下,兵不復起。」

秦始皇本紀第六記載「...維二十八年,皇帝作始。....齊人徐市等上書,言海中有三神山,名曰蓬萊、方丈、瀛洲,仙人居之。請得齋戒,與童男女求之。於是遣徐市發童男女數千人,入海求仙人。」

隨後的記載「...三十二年,....因使韓終、侯公、石生求仙人不死之藥。」,顯示秦始皇深信不疑且陸續遣人尋求不死之藥。三十三年焚書"所不去者,醫藥卜筮種樹之書。

秦始皇去世前一年的記載「三十七年十月癸丑,始皇出遊。....還過吳,從江乘渡。並海上,北至琅邪。方士徐市等入海求神藥,數歲不得,費多,恐譴,乃詐曰:「蓬萊藥可得,然常為大鮫魚所苦,故不得至....」,表明確方士徐市尋求長生不老之藥,而且無功歸返。 次一年「七月丙寅,始皇崩於沙丘平臺。」

秦始皇自消滅六國後,引起無數六國貴族的仇視,故遭到暗殺並不罕見,除了荊軻刺秦較為人所熟悉外,《史記》還記載三宗針對秦始皇的暗殺事件。

战国末燕国人高渐离,擅长击筑(古代的一種擊弦樂器,頸細肩圓,中空,十三弦),是荆轲好友。荊軻死後,秦始皇下令通緝太子丹和荆轲的门客,门客们都潜逃了。高渐离遂改名換姓给人家当酒保,隐藏在宋子这个地方作工。過了不久,高渐离考虑到长久隐姓埋名,担惊受怕地躲藏下去恐怕没有尽头,就不再隱藏自己的身份,應邀往宋子城里人家輪流作客,表演擊筑,聽眾都讚不絕口,並向高漸離賜酒以示讚賞。后来因高渐离击筑技艺高超,被秦始皇传进宫中表演。但被秦始皇的某位臣子告发,秦始皇雖然特赦他的死罪。但让人先弄瞎他的双眼以防行刺。不想高渐离双目虽瞎,却灌铅于筑中,在始皇听其击筑着迷不留意时,奋起用灌铅的筑击打始皇。高渐离暗殺秦始皇最終失敗,并以身殉,自此秦始皇再也不敢接近以前曾為東方六国的百姓。

公元前218年(秦始皇29年),秦始皇在第3次巡遊途中途經博浪沙(今河南省新鄉市原陽縣),隨行車隊突然遭到一個120斤(相當於現在的30公斤)重的大鐵椎撞擊,但大铁椎砸在了另一辆车上,行刺失敗,秦始皇沒有受傷。

張良是博浪沙行刺事件的主謀,他的家族五代仕韓,出於滅韓之恨,幾乎散盡家財寻求勇士谋划刺殺秦始皇,後來找到一個大力士,以大鐵椎撞擊秦始皇的車駕,惜誤中副車,秦始皇為此大索十日,追捕甚急。张良惟有改名换姓,逃到下邳躲藏起来。

公元前216年(秦始皇31年)一個晚上,秦始皇與四名武士一起,在咸阳一帶微服出行,但在兰池宮附近遇上一眾强盗襲擊,情势危急,幸而最終擊斃企圖襲擊秦始皇的強盜。由於懷疑事件另有主謀,故在关中地區大索二十天。

秦始皇崇尚法家治国理念,他以秦国原有的法律令为基础,吸收六国法律的某些条文,制定和颁行全国统一的法律。秦朝对于官吏的管理是很严格的,制定了很多处罚官吏的方法

秦始皇的严酷法律引起了士人的不满,各种指责纷纷而来,不同于秦始皇法家的种种学说不绝于耳。由於當時社會上百家爭鳴,嚴重的阻礙了秦始皇對征服的原六國民眾思想的統一。丞相李斯认为这威脅到了秦朝的統治,主张严厉镇压这些士人,秦始皇支持李斯的看法,并发动了焚书坑儒事件:

所谓“焚书”,就是秦始皇為了統一原六國人民的思想和巩固秦朝的统治,開始销毁除《秦记》之外的其他六国的史书,保留关于农业、技术、卜筮和医药的书籍。除了博士官所职,天下敢私藏诗、书、百家语的人,都命令守、尉将其藏书烧掉。有敢聚论《诗》《书》者弃市,以古非今者族诛。如果官吏知情隐瞒不报,也和藏书的人同等治罪(《史記·秦始皇本紀》)。这一政策从前213年一直执行到前206年秦朝滅亡。但也有人认为,秦始皇曾下令将一些禁书保存于皇家图书馆,直到西楚霸王项羽攻破咸阳后纵火,这些书籍才彻底消失。
始皇三十五年(前212年),盧生說始皇曰:「臣等求芝奇藥仙者常弗遇,類物有害之者。方中,人主時為微行以辟惡鬼,惡鬼辟,真人至。人主所居而人臣知之,則害於神。真人者,入水不濡,入火不爇,陵雲氣,與天地久長。今上治天下,未能恬倓。原上所居宮毋令人知,然後不死之藥殆可得也。」於是始皇曰:「吾慕真人,自謂『真人』,不稱『朕』。」,秦始皇相信不死之藥的真實存在。將皆誦法孔子的儒生四百六十餘人,皆阬之咸陽。《史记·儒林列传》中说秦始皇焚书坑儒之后,六艺从此缺失。但若非大規模頂撞皇帝,始皇對士人也不是那樣不友善。如博士之廷議制度,或前述行封禪之禮,即召魯儒士議之。

據《史記》記載,公元前211年(秦始皇36年)的一年之內,连续发生三件怪事。

首先是當年出現「荧惑守心」的天文現象,古人把「火星」称作「荧惑」,二十八宿中的「心宿」简称为「心」,「心宿」就是现代天文学的「天蝎座」,而荧惑守心的出現則被稱為大凶之兆,轻者天子要失位,严重的情況就是皇帝駕崩。

同年有一塊隕石在東郡地區(今河南省濮陽市)墜落,其間有人在隕石上刻上「始皇帝死而地分」七字,傳入秦始皇耳中。秦始皇大怒,下令御史前去挨家查问,但没有人认罪,于是把居住在那块石头周围的人全部抓来杀了,並焚毁了那块陨石。這次事件讓秦始皇心情不快,故此他让博士作了一首《仙真人诗》,等到巡行天下时,走到一处就传令乐师弹奏唱歌。

同年秋天,有一位使者从关东走夜路经过华阴平舒道,其間有人手持玉璧拦住使者,要求使者將王璧送至滈池君,並稱「今年祖龙死」。使者问他缘由,那人隨即失去蹤影,並留下玉璧,使者惟有捧回玉璧,向秦始皇敍述自己的奇遇。秦始皇沉默了好一会,稱山鬼最多只能预知一年的事,並輕描淡寫地指現時已是秋季,这话未必能应验。他在遣退使者後,又稱「祖龙」的意思是「人的祖先」,(意指「祖龙死」与他无关)。

其後,秦始皇让御府察看那块玉璧,竟發現該玉璧是始皇二十八年出外巡视渡江时,沉入水中祭拜水神的那块,故為此占卜,占卜的结果是迁徙才吉利(史記記載的卜辭稱「游徙吉」),因而下令迁移三万户人家到北河、榆中地区,每户授给二十等爵一级,而秦始皇則在公元前210年(秦始皇37年)進行第五次巡遊,亦是他有生之年最後一次巡遊。

公元前210年年尾(秦始皇三十七年十月),秦始皇踏上第四次全國巡遊之旅,隨行的包括他的幼子胡亥,左丞相李斯及執掌中車府令的趙高,右丞相冯去疾則負責留守京師,其具體巡行路線如下:咸陽 → 雲夢(今湖北省雲夢縣) → 海渚(今安徽省桐城市)→ 丹陽(今江蘇省南京市) → 錢塘(今浙江省杭州市)→ 會稽(今浙江省紹興市) → 吳(今江蘇省蘇州市) → 琅琊→ 荣成山(今山東省榮成市) → 之罘(今山東省烟台市芝罘區)→ 平原津(今山東省德州市平原縣)→ 沙丘(今河北省邢台市广宗县)

秦始皇一行在該年十一月到達云梦,在九疑山遥祭虞舜,然后乘船沿长江而下,观览籍柯,渡过海渚,经丹阳到达钱塘。到浙江边上的时候,水波凶险,就向西走了一百二十里,从江面狭窄的地方渡过,最後登上会稽山祭祀大禹,遥望南海,在那里刻石立碑。碑文除了颂扬秦朝的功德外,還有整頓吳越之地風俗的用意。據明末清初學者顧炎武的著作《日知錄》載,春秋時越王勾踐為反擊吳國,鼓勵生育以增加人口,因此養成吳越地區百姓不重貞節的風氣,故命人在碑文刻下「飾省宣義,有子而嫁,倍死不貞。防隔內外,禁止淫泆,男女絜誠。夫為寄豭,殺之無罪,男秉義程。妻為逃嫁,子不得母,咸化廉清」一段。碑文全篇如下:
皇帝休烈,平一宇內,德惠修長。三十有七年,親巡天下,周覽遠方。遂登會稽,宣省習俗,黔首齋莊。
群臣誦功,本原事跡,追首高明。秦聖臨國,始定刑名,顯陳舊章。初平法式,審別職任,以立恒常。
六王專倍,貪戾傲猛,率眾自彊。暴虐恣行,負力而驕,數動甲兵。陰通閒使,以事合從,行為辟方。
內飾詐謀,外來侵邊,遂起禍殃。義威誅之,殄熄暴悖,亂賊滅亡。聖德廣密,六合之中,被澤無疆。
皇帝并宇,兼聽萬事,遠近畢清。運理群物,考驗事實,各載其名。貴賤并通,善否陳前,靡有隱情。
飾省宣義,有子而嫁,倍死不貞。防隔內外,禁止淫泆,男女絜誠。夫為寄豭,殺之無罪,男秉義程。
妻為逃嫁,子不得母,咸化廉清。大治濯俗,天下承風,蒙被休經。皆遵度軌,和安敦勉,莫不順令。
黔首修絜,人樂同則,嘉保太平。後敬奉法,常治無極,輿舟不傾。從臣誦烈,請刻此石,光垂休銘。

秦始皇之後從會稽山折回,途经吴地,聲勢浩大,避居當地的項梁、項籍(即項羽)叔侄也看到,項籍更稱「吾可取而代之」。儘管項梁擔心項籍禍從口出,惹來族誅之禍,立即叫項籍噤聲,但自此項梁對項籍開始另眼相看。

橫渡長江後,秦始皇沿海岸北上到达琅琊台。這是秦始皇第三次到達琅琊台,並在當地召見方士徐巿,距離他初會徐巿的時間,已經有九年的時光。

根據《史記·秦始皇本紀》的描述,徐巿當時仍未渡海向東求取仙藥,故推斷與秦始皇會面後,徐巿隨即東渡大海。由於徐巿在尋找仙丹的過程中花費大量錢財,害怕面對秦始皇的雷霆之怒,故托辭稱蓬莱仙药雖可能取得,卻總被大鮫魚阻擋,因此無法抵達仙人島,要求增派神箭手及裝備連弩對付鮫魚。不久秦始皇在晚上夢見与海神交战,海神的形状好像人,大惑不解下邀請某博士解梦,博士指海神一向以大鱼蛟龙作斥侯,一般人是不會夢見海神的,不過秦始皇在祭祀方面對神靈恭敬周到,却出现这种恶神,应当除掉它,然后真正的善神才會迎來。

因此秦始皇應允徐巿的要求,派遣射手捕殺鮫魚。从琅琊向北直到荣成山(今山東省榮成市),都不曾遇见鮫魚,直到到达之罘(今山東省烟台市芝罘區)的时候,才射殺了一頭大魚,接着又沿海向西进发。

沿著山東半島沿岸繞行,秦始皇在抵逹黃河下游的平原津(今山東省德州市平原縣)時,突发重病倒下。群臣知道秦始皇忌諱「死」一字,因此不敢將死字掛在嘴邊,但秦始皇的病情仍沒有好轉。

秦始皇知道自己時日無多,故命人写了一封盖上御印的信,给當時正在上郡(今陝西榆林附近)監視蒙恬部隊的長子扶蘇,命扶蘇回咸阳参與丧事,但是這封遺書卻存放在中車府令兼掌印玺事务赵高的办公处,沒有送達扶蘇手上。公元前210年(秦始皇三十七年七月),秦始皇在沙丘平台(今河北省邢台市广宗县)逝世,巧合的是,秦始皇的出生地點與逝世地點,僅相距一百多公里。

《史記·秦本紀》載:「始皇帝立十一年而崩。」但因後來版本傳抄將「立」錯寫「五」,造成後世以為秦始皇活到五十一歲,清代史学家钱大昕指出錯誤,但没有找到相关证据,直到南京师范大学團隊在日本高山寺所藏抄本中找到版本依据,再加上旁证材料,才將错误改过。

秦始皇统一六国后,先后进行5次大规模的巡游,在名山胜地刻石記功,炫耀声威。始皇三十七年(前210年),秦始皇開始最后一次巡游,返至平原津得病。行至沙丘平台,秦始皇驾崩。而秦始皇崩日有二條文獻記錄,如下:《史記》卷六〈秦始皇本紀〉:「〔三十七年〕七月丙寅,始皇崩於沙丘平臺。」此年七月丙子朔,無丙寅日。《開元占經》卷90引伏生《洪範五行傳》:「始皇以〔三十七年〕六月乙丑死於沙丘。」此年六月丁未朔,乙丑十九日(儒曆7月11日)。若《史記》「七月丙寅」之「七」為「六」之誤,則丙寅為二十日(儒曆7月12日)。另有一說「七月丙寅」是依殷曆日期記載,換算成秦曆實際是八月丙寅,八月丙午朔,丙寅二十一日(儒曆9月10日)。世人只知丙寅日为发丧日,却不知秦始皇死之具体日期。由于秦末战火,秦代史料多亡佚,司马迁多方寻访,得知秦始皇死于七月丙寅,将其写入《史记》。

丞相李斯恐天下有变,乃秘不发丧,棺载辒凉车中。所至,百官奏事如故,宦者辄从车中可其奏事,独胡亥、赵高及宦者五六人知。诈为丞相李斯受始皇遗诏于沙丘,立子胡亥为太子。更为书赐公子扶苏、蒙恬,赐死。是為沙丘之變。該年秋季,从直道至咸阳,发丧。太子胡亥袭位,为二世皇帝,天下始知秦始皇驾崩。九月,葬始皇帝於酈山。

秦始皇本作书命在上郡(今陝西榆林附近)監軍的扶苏送葬,并继嗣帝位。但赵高勾结始皇少子胡亥及丞相李斯,在秦始皇死后,伪造了遗诏立胡亥为皇帝,是为秦二世,并赐死扶苏。

秦二世即位后不久,前209年七月, 庶民陈胜 、吴广率领戍卒在大泽乡起事反秦并迅速壮大。 随后,天下大乱。尽管陈胜、吴广攻打关中失利,被秦军反扑击败,但原东方六国相继复国,秦朝走向崩溃。 前207年,秦始皇驾崩後三年, 楚将项燕之孙项羽在巨鹿之战中歼灭秦将章邯统帅的秦军主力。同年九月,赵高杀秦二世,欲自立为帝,但遭到大臣反对,遂立子婴为秦王。即位当日,子婴即设计诛杀赵高,但对秦朝灭亡于事无补。十月,秦王子嬰在灞上向楚將劉邦投降,秦朝灭亡。随后项羽与刘邦争夺天下,爆发了持续五年的楚汉战争,以刘邦胜利建立汉朝而告终。

陳勝、吳廣起義後不久,南海尉任囂病重,臨終前把時任龍川令的恆山郡真定縣(今河北省正定縣)人趙佗召來,向趙佗指南海郡雖位處偏僻,遠離中原,但佔地數千里,傍山靠海,有險可據,可以憑此成為一州之主,並提到自己有切斷通往中原道路的打算,以抵抗中原流寇的侵犯。由於信任趙佗的能力,除了趙佗外,任囂並沒有將他的打算告訴他人,其後更将任命文书交给赵佗,让他代行南海郡郡尉的职务。

任囂不久病亡,繼任為南海尉的趙佗向嶺南的橫浦關(今廣東省南雄市北)、陽山關(今廣東省陽山縣北)、湟溪關(今廣東省英德縣境內連江入北江處)等各關口的軍隊傳達了據險防守的指令,防止中原的起義軍進犯,並藉機殺了秦朝安置在南海郡的官吏們,換上自己的親信。

公元前206年,秦朝滅亡,此時桂林和象兩郡的越人也趁機紛紛獨立,古蜀王族後裔蜀泮在象郡擊敗其它駱越族的部落首領,以螺城(今越南河內市東英縣內)為都城建立了甌駱國,自稱「安陽王」。隨後趙佗出兵攻打安陽王並將其趕走,接着還兼并了桂林郡和象郡。趙佗考慮到象郡越人部落勢力過大,將象郡分拆為交趾、九真兩郡,僅派兩使者管理重大事務,日常事務仍由各部落首領自己管理。約公元前203年(漢高祖三年),趙佗以番禺為都城在嶺南地區建立南越國,自稱「南越武王」。

據《史記·刺客列傳》記載,燕國使者荊軻企圖行刺嬴政時,嬴政由於不能及時拔出其佩劍,結果弄到要繞柱奔逃,十分狼狽。

在一般的認知中,嬴政佩帶的應屬青銅劍,而青銅劍容易折斷,故此一般青銅劍都是短劍,最負盛名的青銅劍——越王勾踐劍全長僅55.6厘米,這種長度的劍隨手就可以拔出。對於司馬遷這段記載,歷史學家一直都顯得很困惑。

1974年,考古學家發現一把接近1米的青銅長劍,現收藏在秦始皇兵馬俑博物館,這次考古發現推翻了以往「青銅劍不能鑄得太長」的認知。可以推斷,當年嬴政就是佩帶這種加長版的青銅劍,所以在被行刺時因劍身過長,而不能及時拔出佩劍,也是可以理解的。

此外,歷史學家發現拔劍困難還有另一個原因,蓋因戰國時期,帝王及官員都習慣將佩劍「負」在身後。他們做過實驗:身高在1.90米以上的人才能自行拔出背在身後達1米的青銅劍。根據這個實驗,秦始皇的身高應該低於1.90米,因此秦始皇才無法及時拔出身後的佩劍。

據《復活的軍團:秦軍秘史》一書所述,隨身佩劍是親政的重要標誌,因此有說法指秦始皇在親政後曾經讓人打造兩柄青銅劍,並在劍上刻了兩個字「定秦」。一柄隨身攜帶,一柄埋在觀台下。秦始皇死後,隨身佩劍很有可能與主人一同入葬。

此外,西漢劉向所著的《說苑·至公》亦記載,秦始皇曾經與博士們討論「禪讓」的問題:

事緣六國破滅之後,秦始皇在某日召開廷議,以三王(即夏禹、商湯、周武王)行世襲制,而五帝行禪讓制,詢問群臣何者為優,然後擇而從之。由於這涉及皇位繼承的敏感議題,博士們都對此保持緘默,只有博士鲍白令之稱五帝讓賢,是視天下為官天下;三王世襲,是視天下為家天下。

秦始皇認為自己德行繼承自五帝,故欲「官天下」,將天下留給賢德之人,並反問鲍白令之,何人能承繼自己的事業。鮑白令之當堂向秦始皇澆了冷水,指秦始皇行「桀紂之道」,在任期間推行多項工程,濫用民力過甚,與五帝的德行相比差得遠。面對鲍白令之的一番話,秦始皇沉默了好一會,面露慚色,最終打消了「禪讓」的念頭。

事實上,由於戰國時早有燕王噲「禪讓」王位給子之的前科,而且秦始皇又自認為「功過三皇,德兼五帝」,有「禪讓」的念頭也不足為奇。只不過《史記》是研究秦始皇生平的最權威資料,《史記》沒有記載該事,意味著「秦始皇談禪讓」一事只能作為軼事參考。

为了寻求长生不老之药,秦始皇派遣方士徐福率童男女6,000人渡东海求神仙。《楚義六帖》記載,徐福和童男女们在到达目的地瀛洲(即今日本)之后一去不返,日本秦氏為其後代,但是有後代學者對此提出了懷疑。

秦始皇是中国历史上一位极富传奇色彩的雄才大略的划时代人物。他是中国歷史上第一位皇帝,是皇帝尊号的创立者,是中国皇帝制度创立者,也是使中国进入了中央集权帝制时代的第一人。他一生并天下、称皇帝、废分封、置郡县、征百越、逐匈奴、修长城、通沟渠、销兵器、迁富豪、车同轨、书同文、钱同币、币同形、度同尺、权同衡、行同伦、一法度、以法治国、焚书坑儒,对于中国之大一统、对于中国政制之创建、对于中国版图之确立、对于中国民族之传承,都起到了不可磨灭的关键作用,对后世的中国和世界产生了不可估量的深远影响。但自古以来,秦始皇一直是一个备受争议的人物,誉之者称其为首创统一局面的“千古一帝”,毁之者则称其为专制独裁的“一代暴君”。



\begin{longtable}{|>{\centering\scriptsize}m{2em}|>{\centering\scriptsize}m{1.3em}|>{\centering}m{8.8em}|}
  % \caption{秦王政}\
  \toprule
  \SimHei \normalsize 年数 & \SimHei \scriptsize 公元 & \SimHei 大事件 \tabularnewline
  % \midrule
  \endfirsthead
  \toprule
  \SimHei \normalsize 年数 & \SimHei \scriptsize 公元 & \SimHei 大事件 \tabularnewline
  \midrule
  \endhead
  \midrule
  二六年 & -221 & \begin{enumerate}
    \tiny
  \item 秦将王贲率军灭齐。
  \item 始皇统一中国。
  \item 秦攻百越\footnote{公元前221年,秦始皇统一后,令50万大军准备征服南方百越各部。秦军分5路南下,在越城岭遭到南方越人的顽强抵抗。}。
  \item 秦始凿灵渠\footnote{灵渠,建于秦始皇执政时期,是中国,也是世界上最早的运河之一。对中国岭南地区的开发起了重要作用。对今天的水利工程建设,仍然据有很好的参考价值}。
  \end{enumerate} \tabularnewline\hline
  二七年 & -220 & \begin{enumerate}
    \tiny
  \item 秦规划咸阳\footnote{公元前220年,秦始皇下令,将秦的东门由黄河延伸到上朐,并以咸阳和东门为中轴线规划新版图。}。
  \end{enumerate} \tabularnewline\hline
  二八年 & -219 & \begin{enumerate}
    \tiny
  \item 徐福\footnote{徐福,即徐巿”(在秦始皇本纪中称“徐巿”,在淮南衡山列传中称“徐福”)。(注意,是“巿”〔fú〕而不是“市”〔shì 〕),字君房,秦朝时齐地人,当时的著名方士。}出海。
  \item 始皇泰山封禅。
  \end{enumerate} \tabularnewline\hline
  二九年 & -218 & \begin{enumerate}
    \tiny
  \item 秦始皇第三次巡游,张良在博浪沙击始皇未中。
  \item 秦征岭南\footnote{尉佗真定人。公元前218年,奉秦始皇命令征岭南,略定南越后,任为南海龙川令。高后五年自立, 僭号“南越武帝”。 尉佗(?-前137年),真定(今石家庄市东古城)人。公元前218年,奉秦始皇命令征岭南,略定南越后,任为南海郡(治所在今广州市)龙川(今广档龙川县)令。秦二世时,赵佗受南海尉任嚣托,行南海尉事。秦亡后,出兵击并桂林郡( 治所在今广西桂平县西南古城)、象郡(治所在今广西崇左县),自立为南越王, 实行“和揖百越”的民族平等政策,采取一系列措施发展当地经济文化。}。
  \item 西瓯国反秦\footnote{公元前218年,西江中部的“西瓯国”起兵反秦,秦始皇派50万大军征讨。又派史禄在海阳山开凿灵渠,将湘江与漓江沟通,以保证军事上的运输。灵渠便成为中原汉人进入岭南的第一条主要通道。秦始皇灭了西瓯国,战争告一段落,秦“发诸尝捕亡人、赘婿、贾人略取陆梁地,为桂林、象郡、南海,以适遣戍。 ”(《史记.秦始皇本纪》)“五十万人守五岭。”(《集解》)这50万人,便是第一批汉族移民。秦始皇搞大迁徙,目的在于铲除六国的地方势力,把族人和故土分开,交叉汇编,徙到南蛮之地戍边,也就连根拔起,使之不能在秦的京城附近形成威胁,兹生复国复旧之梦。}。
  \end{enumerate} \tabularnewline\hline
  三十年 & -217 & \begin{enumerate}
    \tiny
  \item 始修建长城\footnote{秦灭六国之后,即开始北筑长城,每年征发民夫四十余万。全长7000多千米的长城,称作“九边重镇”,每镇设总兵官作为这一段长城的军事长官,受兵部的指挥,负责所辖军区内的防务或奉命支援相邻军区的防务。}。
  \end{enumerate} \tabularnewline\hline
  三一年 & -216 & \begin{enumerate}
    \tiny
  \item 秦改革屯田制\footnote{平民自报所占土地面积,自报耕地面积、土地产量及大小人丁。所报内容由乡出人审查核实,并统一评定产量,计算每户应纳税额,最后登记入册,上报到县,经批准后,即按登记数征收。此前著名的改革家商鞅还在秦国推行了包括土地制度在内的改革。提出了“算地”和“定分”的主张。“算地”就是对土地进行全面的调查核算,以作为制定土地政策的客观依据;“定分”就是用法律形式确认地主或平民对土地占有的“名分”,确认土地所有权。这些实际上都是土地登记的内容。}。
  \item 始皇微行咸阳,兰池遇盗,武士击杀之。大索二十日。
  \item 西汉七国之乱主谋,刘邦之侄,吴王刘濞出生。
  \end{enumerate} \tabularnewline\hline
  三二年 & -215 & \begin{enumerate}
    \tiny
  \item 始皇在今广西等地建立了桂林郡和象郡。
  \item 始皇东巡到达蓟城。
  \item 秦将蒙恬筑马邑城池,置马邑县。
  \end{enumerate} \tabularnewline\hline
  三三年 & -214 & \begin{enumerate}
    \tiny
  \item 灵渠建成。
  \item 秦设龙川县。
  \item 秦设南海郡。
  \item 秦占岭南,夺高阙、阳山、北假\footnote{公元前214年,秦始皇派遣50万军队分5路攻占岭南,任命任嚣为南海尉。派蒙恬渡过黄河去夺取高阙、阳山、北假一带地方,筑起堡垒以驱逐戎狄。迁移被贬谪的人,让他们充实新设置的县。}。
  \end{enumerate} \tabularnewline\hline
  三四年 & -213 & \begin{enumerate}
    \tiny
  \item 李斯任左丞相。
  \item 淳于越谏秦。
  \item 焚书事件。
  \item 秦颁行《挟书令》。
  \item 秦在五岭开山道筑三关,即横浦关、阳山关、湟鸡谷关。
  \item 秦始修筑驰道。
  \end{enumerate} \tabularnewline\hline
  三五年 & -212 & \begin{enumerate}
    \tiny
  \item 修建阿房宫。
  \item 扶苏被派往上郡(今天的陕西绥德)做大将蒙恬的监军。
  \item 焚书坑儒。
  \item 蒙恬率领大军修建了一条从咸阳到九原(今内蒙古包头市)的直道。
  \end{enumerate} \tabularnewline\hline
  三六年 & -211 & \begin{enumerate}
    \tiny
  \item 陨石事件\footnote{秦始皇三十六年,一颗流星坠落到了东郡。东郡是在秦始皇即位之初吕不韦主政时攻打下来的,当时此郡是齐、秦两国的交界地。现在已是大秦帝国的一个东方大郡。陨石落地还不可怕,可怕的是陨石上面刻的字“始皇帝死而地分”。这七个字非同小可!它代表了上天的旨意,预示着秦始皇将死,同时也预告了大秦帝国将亡。}。
  \item 汉惠帝刘盈出生。
  \item 秦置皮氏县。
  \end{enumerate} \tabularnewline\hline
  三七年 & -210 & \begin{enumerate}
    \tiny
  \item 始皇卒\footnote{秦始皇三十七年(公元前210年),秦始皇出巡至平原津(今德州平原县南六十里有张公故城,城东有水津)而病,秦始皇不愿意听到“死”,所以群臣莫敢言死事。8月28日行至沙丘(沙丘台在邢州平乡县东北二十里)病死。}。
  \item 扶苏被害。
  \item 胡亥\footnote{秦二世胡亥(前230年—前207年,在位时间前209年—前207年),也称二世皇帝。是秦始皇第二十六子,公子扶苏的弟弟。秦始皇出游南方病死途中时,在赵高与李斯的帮助下,杀害哥哥扶苏当上秦朝的二世皇帝。贾谊《过秦论》曰:“始皇既没,胡亥极愚,郦山未毕,复作阿房,以遂前策。云“凡所为贵有天下者,肆意极欲,大臣至欲罢先君所为”。诛斯、去疾,任用赵高。痛哉言乎!人头畜鸣。不威不伐恶,不笃不虚亡。距之不得留,残虐以促期,虽居形便之国,犹不得存。”}称帝,是为秦二世。
  \end{enumerate} \tabularnewline
  \bottomrule
\end{longtable}


%%% Local Variables:
%%% mode: latex
%%% TeX-engine: xetex
%%% TeX-master: "../Main"
%%% End:

%% -*- coding: utf-8 -*-
%% Time-stamp: <Chen Wang: 2019-12-17 12:04:40>

\section{秦二世\tiny(BC209-BC207)}

\subsection{生平}


秦二世(前230年3月17日-前207年10月1日),是秦朝第二位皇帝,嬴姓,名胡亥,是秦始皇第十八子[2][3]。从中车府令赵高学习狱法。西元前210年,始皇出游南方,病死沙丘宫,胡亥在赵高与宰相李斯的帮助下,秘不发丧,發動沙丘之變,賜死長兄扶蘇,而即位為二世皇帝。秦二世即位后,殺兄弟姐妹二十餘人,赵高掌权,实行残暴统治,终于激起陈胜、吴广的大澤之變,山東六國旧公室也乘機各自复国。西元前207年赵高的女婿阎乐強逼胡亥自刎于望夷宫,卒年24岁。

胡亥奉始皇帝敕令,從中車府令趙高學習法律。秦始皇三十七年(前210年)秋七月,始皇崩於沙丘平台,丞相李斯恐天下有变,乃秘不发丧,棺载辒辌车中。所至,百官奏事如故,宦者辄从车中可其奏事,独胡亥、赵高及宦者五六人知。诈为丞相李斯受始皇遗诏于沙丘,立子胡亥为太子。更为书赐死公子扶苏、蒙恬,是為沙丘之變。八月丙寅(前210年9月10日)从直道至咸阳,发丧。太子胡亥袭位,为二世皇帝。

秦二世即位後,下令秦始皇後宮無子者皆令殉葬,在埋葬秦始皇時封死了全部工匠在驪山陵墓裏。徵調武士五萬人屯衛咸陽,令教射狗馬禽獸。

秦朝的暴政激起了前209年陳勝、吳廣的大泽乡起義。左丞相李斯與右丞相馮去疾、大將軍馮劫上書請求停止修建阿房宮,減輕各種苛捐雜稅。秦二世聽信趙高讒言,誅殺李斯,賜死馮去疾和馮劫。李斯死后,秦二世拜赵高为中丞相,事无大小皆决於赵高。

秦二世三年七月,章邯投降西楚軍項羽,劉邦攻下武關,趙高惶恐。八月己亥(前207年9月27日),中丞相赵高欲为乱,恐群臣不听,乃先设验,持鹿献于秦二世曰:“马也。”秦二世笑曰:“丞相误邪,谓鹿为马!”群臣皆畏赵高,莫敢言其过。成語“指鹿为马”由此而來。之后,秦二世乃出居望夷宫。过三日,因秦二世派使者责问赵高关东盗贼的事情,赵高心中大为恐惧,遂與其婿咸陽令閻樂合謀,派赵成作为内应,声称有盗贼作乱,命阎乐发兵抓捕盗贼。阎乐率吏卒一千多人包围望夷宫,杀死卫令后攻入宫中,逼胡亥自殺,史称望夷宫之变。臨死前秦二世說寜願只當一位万户侯或平民百姓,阎乐皆不准,秦二世只可自殺,時年24岁,以平民之禮葬。墓地在今西安市雁塔區曲江鄉曲江池村南緣台地上,稱胡亥墓。


秦二世即位年齡有兩種說法:

一是《史記·秦始皇本紀》云「二世皇帝元年,年二十一」,即秦王政十八年(前230年10月29日十月初一 — 前229年11月15日后九月廿九)[4]出生。二是《秦記》云「二世生十二年而立」,以始皇三十七年八月立,即始皇二十六年(前222年10月31日十月初一 — 前221年11月17日后九月三十)出生。時至今日,秦二世二十一歲即位說影響甚廣,馬非百[5]、王蘧常[6],英人杜希德[7]等均從此說,杜希德還明確考辨「他當時二十一歲,《史記》卷六的結尾誤作十二歲」。

\subsection{年表}


\begin{longtable}{|>{\centering\scriptsize}m{2em}|>{\centering\scriptsize}m{1.3em}|>{\centering}m{8.8em}|}
  % \caption{秦王政}\
  \toprule
  \SimHei \normalsize 年数 & \SimHei \scriptsize 公元 & \SimHei 大事件 \tabularnewline
  % \midrule
  \endfirsthead
  \toprule
  \SimHei \normalsize 年数 & \SimHei \scriptsize 公元 & \SimHei 大事件 \tabularnewline
  \midrule
  \endhead
  \midrule
  元年 & -209 & \begin{enumerate}
    \tiny
  \item 大泽乡起义。
  \item 刘邦起义。
  \item 项羽反秦。
  \item 冒顿即位。
  \end{enumerate} \tabularnewline\hline
  二年 & -208 & \begin{enumerate}
    \tiny
  \item 秦灭项梁。
  \item 孔鲋逝世。
  \item 陈胜卒。
  \item 李斯卒。
  \item 薛地会议。
  \item 统一越南。
  \end{enumerate} \tabularnewline\hline
  三年 & -207 & \begin{enumerate}
    \tiny
  \item 指鹿为马。
  \item 破釜沉舟。
  \item 胡亥被弑。
  \item 子婴即位,诛赵高,在位47天被废。
  \end{enumerate} \tabularnewline
  \bottomrule
\end{longtable}


%%% Local Variables:
%%% mode: latex
%%% TeX-engine: xetex
%%% TeX-master: "../Main"
%%% End:

%% -*- coding: utf-8 -*-
%% Time-stamp: <Chen Wang: 2019-10-22 11:28:30>

\section{子婴\tiny(BC206-BC206)}

嬴嬰(前242年5月4日-前206年1月17日),嬴姓,史称秦忖王子婴,又稱秦三世,是為秦朝最後一位君主,出身秦國宗室。其性格仁愛且節制,秦二世皇帝胡亥在位时,嬴婴曾就二世杀死兄弟、功臣之事向二世进谏。胡亥被弑後,中丞相趙高迎立子嬰繼位。子婴即位五日便設計謀殺趙高,族誅。《史記》指趙高企圖招引義軍到咸陽及承諾殺死所有秦朝宗室,嬴婴得知就先下手殺死他。《赵政书》则指赵高为章邯所杀。劉邦進入關中,子嬰向劉邦軍投降,秦朝正式滅亡。不久,项羽亦率军抵達關中,於咸陽城殺死嬴婴及秦诸公子宗族。

關於嬴嬰的身世有多個說法,可能是胡亥姪子,或是秦始皇嬴政姪子、成蟜長子。

秦始皇三十七年(前210年)始皇驾崩,趙高與李斯發動沙丘之變,賜死始皇的長子扶蘇,也處決中尉蒙恬、蒙毅等人,立幼子胡亥为皇帝,即秦二世,二世并任命赵高为郎中令。秦二世为巩固自己的统治,又欲铲除自己的兄弟姐妹,焚毁律令及故世之藏。又欲起属车万乘以抚天下,说:“且与天下更始。”子婴进谏秦二世说:“不可。臣闻之:芬茝未根而生凋香同,天地相去远而阴阳气合,五国十二诸侯,民之嗜欲不同而意不异。夫赵王迁杀其良将李牧而用顏聚,燕王喜而用轲之谋而背秦之约,齐王建遂杀其故世之忠臣而用后胜之议。此三君者,皆终以失其国而殃其身。是皆大臣之谋,而社稷之神零福也。今王欲一日而弃去之,臣窃以为不可。臣闻之:轻虑不可以治固,独勇不可以存将,同力可以举重,比心壹智可以胜眾,而弱胜强者,上下调而多力壹也。今国危敌比,斗士在外,而内自夷宗族,诛群忠臣,而立无节行之人,是内使群臣不相信,而外使斗士之意离也。臣窃以為不可。”秦二世最终未能听从,仍按照自己原有的想法,杀死了中尉蒙恬等人,并立赵高为郎中令,出游天下。

秦二世三年,秦二世又对左丞相李斯起了杀机。李斯为表无罪多次向二世上书,但秦二世都没有听从,仍想要杀死李斯。子婴进谏曰:“不可。夫变俗而易法令,诛群忠臣,而立无节行之人,使以法纵其欲,而行不义於天下臣,臣恐其有后咎。大臣外谋而百姓内怨。今将军章邯兵居外,卒士劳苦,委输不给,外毋敌而内有争臣之志,故曰危。”但秦二世仍未听从,李斯最终被腰斩,夷灭三族。赵高被秦二世任命为中丞相。

秦二世三年,中丞相赵高知道自己已位高权重,以指鹿为马排除异己后,派女婿咸阳令阎乐发动望夷宫之变,逼迫秦二世胡亥自杀。

胡亥自杀后,趙高拿皇帝印璽打算自立,左右伙伴、百官都不服從;要上殿登基時,大殿地震三次,彷彿要崩塌,趙高於是知道天意、群臣都不服。同時,趙高也派使者联络攻破武关的关东起义军首领刘邦,约定与关东诸侯共同瓜分秦国,自立为王,但遭到刘邦拒绝。赵高于是召集秦朝群臣和诸公子,打算立公子子婴为王。赵高认为山东六国都已经复国,秦国疆域日益缩小,称帝徒有虚名,应像过去一样称王,子婴于是在赵高和群臣的迎立下登基,为秦王子婴,随后秦二世以庶民之礼,埋葬在杜县南面的宜春苑中。

赵高让子婴斋戒,然后準備到宗庙祭拜祖先,接受传国玉玺。子婴斋戒了五日和其二子商量说:“丞相赵高在望夷宫杀死秦二世,恐怕群臣诛杀他,就假装以大义为名立我为王。我听聞赵高和关东诸侯有约定,由他消灭秦国宗室,然后在关中称王。现在使我斋戒,参拜祖庙,此是想趁我在祖庙时杀死我。我就稱病不行,丞相一定亲自来我此處,来就趁机杀死他。”赵高多次派人催促子婴前往,子婴坚决推辞,赵高果然亲自来请子婴,说:“国家大事,你奈何不亲自前往呢?”子婴就在斋戒的宫室裡派宦官韩谈刺杀了赵高,同时下令诛灭赵高三族,在咸阳示众。

秦王子婴元年十月,刘邦的军队攻破武关,到达灞上,兵临咸阳时,派人劝子婴投降。此时群臣百官都背叛了秦国,子嬰于是和妻子和儿子用繩綁縛自己,坐上由白馬駕駛的白色馬車,身著死者葬禮所穿的白裝束,並攜同包括玉璽和兵符在内的皇帝御用物品,在軹道親自到劉邦的軍前投降。子婴在位46日,秦朝灭亡。子婴投降后,刘邦的部將提议處決子婴,但刘邦拒絕,將子婴交给了随行的吏员看管。

一個多月後,項羽亦率領大軍到達關中。劉邦的部將曹無傷向項羽稱劉邦將以子嬰為相而自立為關中王,結果項羽設下了鴻門宴。項羽进入咸阳后,殺死子婴及秦诸公子宗族,並在咸阳屠杀,火烧秦宫室,擄掠秦宫的女子,分了秦国的珍宝貨財給诸侯。

子嬰死後,其埋葬地點一直不詳。2007年,被譽為“秦兵馬俑之父”的考古學家袁仲一指出,在秦始皇陵園旁新發現的“秦陵第二大墓”,是座相對獨立的陪葬墓園,其墓主很可能是子嬰。

袁仲一認為,子嬰是秦二世之兄、秦始皇之子,被殺後埋于始皇陵園附近,是符合古代的喪葬禮制和一般常理。放子嬰墓在秦始皇陵園西北隅也是迫不得已。子嬰是亡國之君,在位時間僅46日,倉促選址埋葬,致使墓上沒堆築封土,沒築城垣,連墓的方向都有違傳統,遂致其葬地長期不明。

陝西咸陽、西安等地區皆有奉祀子嬰的廟宇。一般從祀於扶蘇,稱「秦王廟」。

根據《史記》所述,賈誼認為子嬰是使秦朝完全滅亡的人物。他在《過秦論》中認為:只要子嬰有「庸主之材」,加上中規中矩的輔佐,秦仍可保守關中地區。司馬遷本人在《秦始皇本紀》亦贊同賈誼之論文。

東漢史家班固則持不同看法:他認為秦二世駕崩時,秦朝僅餘數日,即使有周公旦之材,秦朝已不可救;子嬰雖然無能為力,但誅殺趙高已證明他已盡力完成自己能做之事,應該同情其志和尊重其義:「吾讀《秦紀》,至於子嬰車裂趙高,未嘗不健其決、憐其志。嬰死生之義備矣。」。

唐書道宣和尚所著的《廣弘明集》引用南朝梁道士陶弘景《陶隱居年紀》記載,子嬰被諡為殤,是為秦殤帝。然而秦始皇帝已經廢除諡法,以二世皇帝、三世皇帝稱之;所以此稱呼僅為後世假設秦朝如果有諡號制度之見解。

最早記載子嬰事跡的《史記》,對子嬰其人,有幾種不同的說法:

一是胡亥的姪子。《秦始皇本紀》「立二世之兄子公子嬰為秦王。」(《六國年表》作「高立二世兄子嬰」)這種說法认为「兄子」就是兄長的兒子。二是秦始皇的弟弟。《李斯列傳》:「高自知天弗與,群臣弗許,乃召始皇弟,授之璽。子嬰即位,患之,乃稱疾不聽事,與宦者韓談及其子謀殺高。」三是秦二世胡亥的哥哥。这一派认为《六国年表》的有關章句:「高立二世兄子婴」 应该理解为「趙高拥立秦二世的兄長子婴为秦王。」四是始皇弟成蟜之子。有歷史學家表示《李斯列傳》集解引徐廣說「一本曰『召始皇弟子嬰,授之璽』」中的「弟子」應理解為「弟弟的兒子」。這幾種說法當中,以第一說「二世兄子」較為流行。迄今為止,從東漢班固到近現代,多採用這一說法。學界中多數亦支持这一說。就連近幾年修訂出版的《辭海》和《辭源》這兩部著名的大辭典,也都一致認為子嬰是二世兄子,並指出是扶蘇之子。

亦有論者如楊善群、王蘧常等人支持第二說。論點包括:

子嬰的遭遇、才幹及影響力絕非秦二世或同輩所能及。據《秦始皇本紀》、《李斯列傳》記載,胡亥對待自己的兄弟絕不手軟,子嬰若為胡亥的兄長,為何能存活下來。秦始皇死時年僅50歲,扶蘇年齡大約為30歲左右。而《秦始皇本紀》中敘述子嬰與「其子二人」謀殺趙高(與《李斯列傳》中所述殺趙高過程不同),其子年齡至少有15-20歲左右,推斷子嬰年齡當為35-40歲左右,比秦始皇小10-15歲左右,與扶蘇年紀大致相當。在兩漢時期的史書《史記》、《漢書》原文及《史記》三家注、顏師古注並無提及子嬰為扶蘇之子。尚有学者李开元、马非百等人提出第四说,论点如下:

有关《李斯列传》集解引徐广说「一本曰『召始皇弟子婴,授之玺』」中的「弟子嬰」,应理解为「弟弟的兒子嬰」。但秦始皇的弟兄见于文献记载的只有成蟜、母赵姬与嫪毐所生二子。所以被認為是成蟜的兒子。《释名‧释长幼》:「人始生曰婴」。「婴」之名,有初生儿,年幼儿的含义。据有关史料推测,成蟜大约出生于前256年,子婴大约出生于前240年。成蟜于前239年降赵时,其子此时约为2岁左右,并且可能留在秦国。因与胡亥同辈且年龄较大,所以《六国年表》「高立二世兄子婴」中的「二世兄」应理解为「秦二世的从兄」。与胡亥无皇位争夺的利害关系,所以不在二世所欲清除的兄弟姐妹中,反而能站出来劝谏二世不要滥施诛杀。

\begin{longtable}{|>{\centering\scriptsize}m{2em}|>{\centering\scriptsize}m{1.3em}|>{\centering}m{8.8em}|}
  % \caption{秦王政}\
  \toprule
  \SimHei \normalsize 年数 & \SimHei \scriptsize 公元 & \SimHei 大事件 \tabularnewline
  % \midrule
  \endfirsthead
  \toprule
  \SimHei \normalsize 年数 & \SimHei \scriptsize 公元 & \SimHei 大事件 \tabularnewline
  \midrule
  \endhead
  \midrule
  元年 & -206 & \tabularnewline
  \bottomrule
\end{longtable}


%%% Local Variables:
%%% mode: latex
%%% TeX-engine: xetex
%%% TeX-master: "../Main"
%%% End:


%%% Local Variables:
%%% mode: latex
%%% TeX-engine: xetex
%%% TeX-master: "../Main"
%%% End:
