%% -*- coding: utf-8 -*-
%% Time-stamp: <Chen Wang: 2019-10-22 11:28:30>

\section{子婴\tiny(BC206-BC206)}

嬴嬰(前242年5月4日-前206年1月17日),嬴姓,史称秦忖王子婴,又稱秦三世,是為秦朝最後一位君主,出身秦國宗室。其性格仁愛且節制,秦二世皇帝胡亥在位时,嬴婴曾就二世杀死兄弟、功臣之事向二世进谏。胡亥被弑後,中丞相趙高迎立子嬰繼位。子婴即位五日便設計謀殺趙高,族誅。《史記》指趙高企圖招引義軍到咸陽及承諾殺死所有秦朝宗室,嬴婴得知就先下手殺死他。《赵政书》则指赵高为章邯所杀。劉邦進入關中,子嬰向劉邦軍投降,秦朝正式滅亡。不久,项羽亦率军抵達關中,於咸陽城殺死嬴婴及秦诸公子宗族。

關於嬴嬰的身世有多個說法,可能是胡亥姪子,或是秦始皇嬴政姪子、成蟜長子。

秦始皇三十七年(前210年)始皇驾崩,趙高與李斯發動沙丘之變,賜死始皇的長子扶蘇,也處決中尉蒙恬、蒙毅等人,立幼子胡亥为皇帝,即秦二世,二世并任命赵高为郎中令。秦二世为巩固自己的统治,又欲铲除自己的兄弟姐妹,焚毁律令及故世之藏。又欲起属车万乘以抚天下,说:“且与天下更始。”子婴进谏秦二世说:“不可。臣闻之:芬茝未根而生凋香同,天地相去远而阴阳气合,五国十二诸侯,民之嗜欲不同而意不异。夫赵王迁杀其良将李牧而用顏聚,燕王喜而用轲之谋而背秦之约,齐王建遂杀其故世之忠臣而用后胜之议。此三君者,皆终以失其国而殃其身。是皆大臣之谋,而社稷之神零福也。今王欲一日而弃去之,臣窃以为不可。臣闻之:轻虑不可以治固,独勇不可以存将,同力可以举重,比心壹智可以胜眾,而弱胜强者,上下调而多力壹也。今国危敌比,斗士在外,而内自夷宗族,诛群忠臣,而立无节行之人,是内使群臣不相信,而外使斗士之意离也。臣窃以為不可。”秦二世最终未能听从,仍按照自己原有的想法,杀死了中尉蒙恬等人,并立赵高为郎中令,出游天下。

秦二世三年,秦二世又对左丞相李斯起了杀机。李斯为表无罪多次向二世上书,但秦二世都没有听从,仍想要杀死李斯。子婴进谏曰:“不可。夫变俗而易法令,诛群忠臣,而立无节行之人,使以法纵其欲,而行不义於天下臣,臣恐其有后咎。大臣外谋而百姓内怨。今将军章邯兵居外,卒士劳苦,委输不给,外毋敌而内有争臣之志,故曰危。”但秦二世仍未听从,李斯最终被腰斩,夷灭三族。赵高被秦二世任命为中丞相。

秦二世三年,中丞相赵高知道自己已位高权重,以指鹿为马排除异己后,派女婿咸阳令阎乐发动望夷宫之变,逼迫秦二世胡亥自杀。

胡亥自杀后,趙高拿皇帝印璽打算自立,左右伙伴、百官都不服從;要上殿登基時,大殿地震三次,彷彿要崩塌,趙高於是知道天意、群臣都不服。同時,趙高也派使者联络攻破武关的关东起义军首领刘邦,约定与关东诸侯共同瓜分秦国,自立为王,但遭到刘邦拒绝。赵高于是召集秦朝群臣和诸公子,打算立公子子婴为王。赵高认为山东六国都已经复国,秦国疆域日益缩小,称帝徒有虚名,应像过去一样称王,子婴于是在赵高和群臣的迎立下登基,为秦王子婴,随后秦二世以庶民之礼,埋葬在杜县南面的宜春苑中。

赵高让子婴斋戒,然后準備到宗庙祭拜祖先,接受传国玉玺。子婴斋戒了五日和其二子商量说:“丞相赵高在望夷宫杀死秦二世,恐怕群臣诛杀他,就假装以大义为名立我为王。我听聞赵高和关东诸侯有约定,由他消灭秦国宗室,然后在关中称王。现在使我斋戒,参拜祖庙,此是想趁我在祖庙时杀死我。我就稱病不行,丞相一定亲自来我此處,来就趁机杀死他。”赵高多次派人催促子婴前往,子婴坚决推辞,赵高果然亲自来请子婴,说:“国家大事,你奈何不亲自前往呢?”子婴就在斋戒的宫室裡派宦官韩谈刺杀了赵高,同时下令诛灭赵高三族,在咸阳示众。

秦王子婴元年十月,刘邦的军队攻破武关,到达灞上,兵临咸阳时,派人劝子婴投降。此时群臣百官都背叛了秦国,子嬰于是和妻子和儿子用繩綁縛自己,坐上由白馬駕駛的白色馬車,身著死者葬禮所穿的白裝束,並攜同包括玉璽和兵符在内的皇帝御用物品,在軹道親自到劉邦的軍前投降。子婴在位46日,秦朝灭亡。子婴投降后,刘邦的部將提议處決子婴,但刘邦拒絕,將子婴交给了随行的吏员看管。

一個多月後,項羽亦率領大軍到達關中。劉邦的部將曹無傷向項羽稱劉邦將以子嬰為相而自立為關中王,結果項羽設下了鴻門宴。項羽进入咸阳后,殺死子婴及秦诸公子宗族,並在咸阳屠杀,火烧秦宫室,擄掠秦宫的女子,分了秦国的珍宝貨財給诸侯。

子嬰死後,其埋葬地點一直不詳。2007年,被譽為“秦兵馬俑之父”的考古學家袁仲一指出,在秦始皇陵園旁新發現的“秦陵第二大墓”,是座相對獨立的陪葬墓園,其墓主很可能是子嬰。

袁仲一認為,子嬰是秦二世之兄、秦始皇之子,被殺後埋于始皇陵園附近,是符合古代的喪葬禮制和一般常理。放子嬰墓在秦始皇陵園西北隅也是迫不得已。子嬰是亡國之君,在位時間僅46日,倉促選址埋葬,致使墓上沒堆築封土,沒築城垣,連墓的方向都有違傳統,遂致其葬地長期不明。

陝西咸陽、西安等地區皆有奉祀子嬰的廟宇。一般從祀於扶蘇,稱「秦王廟」。

根據《史記》所述,賈誼認為子嬰是使秦朝完全滅亡的人物。他在《過秦論》中認為:只要子嬰有「庸主之材」,加上中規中矩的輔佐,秦仍可保守關中地區。司馬遷本人在《秦始皇本紀》亦贊同賈誼之論文。

東漢史家班固則持不同看法:他認為秦二世駕崩時,秦朝僅餘數日,即使有周公旦之材,秦朝已不可救;子嬰雖然無能為力,但誅殺趙高已證明他已盡力完成自己能做之事,應該同情其志和尊重其義:「吾讀《秦紀》,至於子嬰車裂趙高,未嘗不健其決、憐其志。嬰死生之義備矣。」。

唐書道宣和尚所著的《廣弘明集》引用南朝梁道士陶弘景《陶隱居年紀》記載,子嬰被諡為殤,是為秦殤帝。然而秦始皇帝已經廢除諡法,以二世皇帝、三世皇帝稱之;所以此稱呼僅為後世假設秦朝如果有諡號制度之見解。

最早記載子嬰事跡的《史記》,對子嬰其人,有幾種不同的說法:

一是胡亥的姪子。《秦始皇本紀》「立二世之兄子公子嬰為秦王。」(《六國年表》作「高立二世兄子嬰」)這種說法认为「兄子」就是兄長的兒子。二是秦始皇的弟弟。《李斯列傳》:「高自知天弗與,群臣弗許,乃召始皇弟,授之璽。子嬰即位,患之,乃稱疾不聽事,與宦者韓談及其子謀殺高。」三是秦二世胡亥的哥哥。这一派认为《六国年表》的有關章句:「高立二世兄子婴」 应该理解为「趙高拥立秦二世的兄長子婴为秦王。」四是始皇弟成蟜之子。有歷史學家表示《李斯列傳》集解引徐廣說「一本曰『召始皇弟子嬰,授之璽』」中的「弟子」應理解為「弟弟的兒子」。這幾種說法當中,以第一說「二世兄子」較為流行。迄今為止,從東漢班固到近現代,多採用這一說法。學界中多數亦支持这一說。就連近幾年修訂出版的《辭海》和《辭源》這兩部著名的大辭典,也都一致認為子嬰是二世兄子,並指出是扶蘇之子。

亦有論者如楊善群、王蘧常等人支持第二說。論點包括:

子嬰的遭遇、才幹及影響力絕非秦二世或同輩所能及。據《秦始皇本紀》、《李斯列傳》記載,胡亥對待自己的兄弟絕不手軟,子嬰若為胡亥的兄長,為何能存活下來。秦始皇死時年僅50歲,扶蘇年齡大約為30歲左右。而《秦始皇本紀》中敘述子嬰與「其子二人」謀殺趙高(與《李斯列傳》中所述殺趙高過程不同),其子年齡至少有15-20歲左右,推斷子嬰年齡當為35-40歲左右,比秦始皇小10-15歲左右,與扶蘇年紀大致相當。在兩漢時期的史書《史記》、《漢書》原文及《史記》三家注、顏師古注並無提及子嬰為扶蘇之子。尚有学者李开元、马非百等人提出第四说,论点如下:

有关《李斯列传》集解引徐广说「一本曰『召始皇弟子婴,授之玺』」中的「弟子嬰」,应理解为「弟弟的兒子嬰」。但秦始皇的弟兄见于文献记载的只有成蟜、母赵姬与嫪毐所生二子。所以被認為是成蟜的兒子。《释名‧释长幼》:「人始生曰婴」。「婴」之名,有初生儿,年幼儿的含义。据有关史料推测,成蟜大约出生于前256年,子婴大约出生于前240年。成蟜于前239年降赵时,其子此时约为2岁左右,并且可能留在秦国。因与胡亥同辈且年龄较大,所以《六国年表》「高立二世兄子婴」中的「二世兄」应理解为「秦二世的从兄」。与胡亥无皇位争夺的利害关系,所以不在二世所欲清除的兄弟姐妹中,反而能站出来劝谏二世不要滥施诛杀。

\begin{longtable}{|>{\centering\scriptsize}m{2em}|>{\centering\scriptsize}m{1.3em}|>{\centering}m{8.8em}|}
  % \caption{秦王政}\
  \toprule
  \SimHei \normalsize 年数 & \SimHei \scriptsize 公元 & \SimHei 大事件 \tabularnewline
  % \midrule
  \endfirsthead
  \toprule
  \SimHei \normalsize 年数 & \SimHei \scriptsize 公元 & \SimHei 大事件 \tabularnewline
  \midrule
  \endhead
  \midrule
  元年 & -206 & \tabularnewline
  \bottomrule
\end{longtable}


%%% Local Variables:
%%% mode: latex
%%% TeX-engine: xetex
%%% TeX-master: "../Main"
%%% End:
