%% -*- coding: utf-8 -*-
%% Time-stamp: <Chen Wang: 2021-10-29 17:22:00>

\section{秦二世胡亥\tiny(BC209-BC207)}

\subsection{生平}


秦二世(前230年3月17日-前207年10月1日),是秦朝第二位皇帝,嬴姓,名胡亥,是秦始皇第十八子[2][3]。从中车府令赵高学习狱法。西元前210年,始皇出游南方,病死沙丘宫,胡亥在赵高与宰相李斯的帮助下,秘不发丧,發動沙丘之變,賜死長兄扶蘇,而即位為二世皇帝。秦二世即位后,殺兄弟姐妹二十餘人,赵高掌权,实行残暴统治,终于激起陈胜、吴广的大澤之變,山東六國旧公室也乘機各自复国。西元前207年赵高的女婿阎乐強逼胡亥自刎于望夷宫,卒年24岁。

胡亥奉始皇帝敕令,從中車府令趙高學習法律。秦始皇三十七年(前210年)秋七月,始皇崩於沙丘平台,丞相李斯恐天下有变,乃秘不发丧,棺载辒辌车中。所至,百官奏事如故,宦者辄从车中可其奏事,独胡亥、赵高及宦者五六人知。诈为丞相李斯受始皇遗诏于沙丘,立子胡亥为太子。更为书赐死公子扶苏、蒙恬,是為沙丘之變。八月丙寅(前210年9月10日)从直道至咸阳,发丧。太子胡亥袭位,为二世皇帝。

秦二世即位後,下令秦始皇後宮無子者皆令殉葬,在埋葬秦始皇時封死了全部工匠在驪山陵墓裏。徵調武士五萬人屯衛咸陽,令教射狗馬禽獸。

秦朝的暴政激起了前209年陳勝、吳廣的大泽乡起義。左丞相李斯與右丞相馮去疾、大將軍馮劫上書請求停止修建阿房宮,減輕各種苛捐雜稅。秦二世聽信趙高讒言,誅殺李斯,賜死馮去疾和馮劫。李斯死后,秦二世拜赵高为中丞相,事无大小皆决於赵高。

秦二世三年七月,章邯投降西楚軍項羽,劉邦攻下武關,趙高惶恐。八月己亥(前207年9月27日),中丞相赵高欲为乱,恐群臣不听,乃先设验,持鹿献于秦二世曰:“马也。”秦二世笑曰:“丞相误邪,谓鹿为马!”群臣皆畏赵高,莫敢言其过。成語“指鹿为马”由此而來。之后,秦二世乃出居望夷宫。过三日,因秦二世派使者责问赵高关东盗贼的事情,赵高心中大为恐惧,遂與其婿咸陽令閻樂合謀,派赵成作为内应,声称有盗贼作乱,命阎乐发兵抓捕盗贼。阎乐率吏卒一千多人包围望夷宫,杀死卫令后攻入宫中,逼胡亥自殺,史称望夷宫之变。臨死前秦二世說寜願只當一位万户侯或平民百姓,阎乐皆不准,秦二世只可自殺,時年24岁,以平民之禮葬。墓地在今西安市雁塔區曲江鄉曲江池村南緣台地上,稱胡亥墓。


秦二世即位年齡有兩種說法:

一是《史記·秦始皇本紀》云「二世皇帝元年,年二十一」,即秦王政十八年(前230年10月29日十月初一 — 前229年11月15日后九月廿九)[4]出生。二是《秦記》云「二世生十二年而立」,以始皇三十七年八月立,即始皇二十六年(前222年10月31日十月初一 — 前221年11月17日后九月三十)出生。時至今日,秦二世二十一歲即位說影響甚廣,馬非百[5]、王蘧常[6],英人杜希德[7]等均從此說,杜希德還明確考辨「他當時二十一歲,《史記》卷六的結尾誤作十二歲」。

\subsection{年表}


\begin{longtable}{|>{\centering\scriptsize}m{2em}|>{\centering\scriptsize}m{1.3em}|>{\centering}m{8.8em}|}
  % \caption{秦王政}\
  \toprule
  \SimHei \normalsize 年数 & \SimHei \scriptsize 公元 & \SimHei 大事件 \tabularnewline
  % \midrule
  \endfirsthead
  \toprule
  \SimHei \normalsize 年数 & \SimHei \scriptsize 公元 & \SimHei 大事件 \tabularnewline
  \midrule
  \endhead
  \midrule
  元年 & -209 & \begin{enumerate}
    \tiny
  \item 大泽乡起义。
  \item 刘邦起义。
  \item 项羽反秦。
  \item 冒顿即位。
  \end{enumerate} \tabularnewline\hline
  二年 & -208 & \begin{enumerate}
    \tiny
  \item 秦灭项梁。
  \item 孔鲋逝世。
  \item 陈胜卒。
  \item 李斯卒。
  \item 薛地会议。
  \item 统一越南。
  \end{enumerate} \tabularnewline\hline
  三年 & -207 & \begin{enumerate}
    \tiny
  \item 指鹿为马。
  \item 破釜沉舟。
  \item 胡亥被弑。
  \item 子婴即位,诛赵高,在位47天被废。
  \end{enumerate} \tabularnewline
  \bottomrule
\end{longtable}


%%% Local Variables:
%%% mode: latex
%%% TeX-engine: xetex
%%% TeX-master: "../Main"
%%% End:
