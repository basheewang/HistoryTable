%% -*- coding: utf-8 -*-
%% Time-stamp: <Chen Wang: 2019-12-26 15:06:30>

\section{惠宗\tiny(1398-1402)}

\subsection{生平}

建文帝朱允炆(1377年12月5日-?),或稱明惠宗,是明朝第二代皇帝,年號“建文”,明太祖朱元璋之孫。在位期間進行一系列寬政、削藩的改革,史稱“建文改制”。由於燕王朱棣發動靖難之變攻入南京應天府,是為明成祖,朱允炆下落不明。大臣梅殷私諡其為「神宗孝愍皇帝」但成祖不承認,故不使用,甚至明成祖不認為朱允炆是合法皇帝,故明朝人大多稱之為建文君。直到南明時,弘光帝追谥其為“嗣天章道诚懿渊功观文扬武克仁笃孝让皇帝”,庙号“惠宗”。清高宗乾隆元年,高宗追谥其為「恭閔惠皇帝」,故也作「明惠帝」。

朱允炆是懿文太子朱标第二子,嫡母太子妃常氏所生長子朱雄英早故,另有一子朱允熥为其弟。嫡母常氏在1378年逝世后,朱允炆生母吕氏成为继任太子妃,所以明太祖朱元璋就視朱允炆為嫡長孫。

洪武二十五年(1392年),父亲朱标病死,朱允炆被祖父朱元璋立为皇太孙。由於自幼熟讀儒家經書,所近之人多懷理想主義,性情因此與其父同樣溫文儒雅,即長皆以寬大著稱。洪武二十九年,朱允炆曾向太祖請求修改《大明律》,他參考《禮經》及歷朝刑法,修改《大明律》中七十三條過份嚴苛的條文,深得人心。

朱允炆出生时脑袋长得颇偏,朱元璋用手摸着说:“半边月儿。”一年除夕,他与父亲朱标陪同朱元璋,朱元璋叫他父子作詠月诗,朱允炆作诗曰:“谁将玉指甲,掐作天上痕。影落江湖里,蛟龙不敢吞。”朱元璋看后默然不语。

明洪武三十一年(1398年)閏五月,明太祖朱元璋去世,死前密命驸马梅殷辅佐新君。朱允炆在同月(6月30日)即位,定次年(從1399年2月6日开始)為建文元年。建文帝在六月晉用齊泰為兵部尚書、黃子澄為太常寺卿,七月召方孝孺為翰林院侍講,在國事上倚重三人。建文帝的年號“建文”有別於其祖父的洪武,他不想仿效祖父以嚴刑峻法治國,即位後改行寬政,囚犯人數減至洪武時期的三成左右。

建文帝能虚心纳谏。一次他因病上朝晚了,监察御史尹昌隆对此提出批评,左右建议他说出自己染病,建文帝却认为这样的谏言难得,不但没有自辩,还表扬了尹昌隆,公开了他的奏疏。

明太祖為鞏固皇室,大封宗室為藩王,各擁私人護衛軍隊。對建文帝來說,諸藩王大多為其叔輩,且在封地掌握兵權,心中由是不安。建文帝為皇太孫時曾問黃子澄曰:「諸王尊屬擁重兵,多不法,奈何?」子澄回答說諸王軍力不足以抗衡朝廷。建文帝即位後,下令各王國的地方文武官員聽朝廷節制,採取削藩政策,先後废黜周王、湘王、齐王、代王及岷王。在部署對付年齡最長、軍功最多、武力最强大的燕王朱棣時,由於建文帝身邊的謀士多缺乏實際的政治經驗,以致打草驚蛇,引發了燕王先發制人的念頭。朱棣在權衡利害之後,於建文元年(1399年)七月在封地北平起兵反叛。他以“靖难”為名,向京師進軍。

建文元年,明建文帝下詔討伐燕軍。命吳傑、吳高、耿瓛、盛庸、潘忠、楊松、顧成、徐凱、李友、陳暉、平安分道併進,并在河北真定設立平燕布政使司,兵部尚書暴昭掌司事。隨後,耿炳文率大軍抵達,與燕軍交戰后失利退守。明建文帝臨時換將,撤換耿炳文,由李景隆代替。隨後,朱棣獲得寧王朱權及朵顏三衛,實力大增。而李景隆在率軍圍困北平后,仍然無法破城,并在鄭村壩潰敗。燕王因此向明惠帝上書,明惠帝不得不罷免齊泰、黃子澄。

建文二年,燕軍與中央軍在白溝河大戰,李景隆再次潰敗并逃亡濟南,隨後再在濟南潰敗。然而,朱棣卻無法攻破山東參政鐵鉉、都督盛庸的濟南城,不得不撤軍。明惠帝隨後封賞鐵鉉、盛庸,但卻不誅殺李景隆。同年冬,燕軍再次進犯濟寧,盛庸擊敗并斬殺燕將張玉,并接連獲勝。建文三年,兩軍在河北山東一帶屢次交戰,并互有勝負,最後燕軍攻入真定。

建文四年,何福、平安率領的中央軍在小河大勝燕軍,并斬其將陳文;而徐輝祖亦在齊眉山獲得大捷。燕軍恐懼后計劃北歸。恰逢建文帝誤以為燕軍已經北撤,召徐輝祖班師,致使何福孤軍奮戰。隨後,靈璧之戰中,燕軍大勝,陳暉、平安、陳性善、彭與明被執。盛庸軍亦在淮河之戰中潰敗,燕軍遂渡過淮河,抵達六合。建文帝不得不下詔要求各地勤王,并遣使割地罷兵。同年六月,盛庸在浦子口與燕軍交戰不利,都督僉事陳瑄率水軍附燕。隨後,朱棣率燕軍渡江,最終逼近南京應天府。谷王朱橞與李景隆開金川門變節,致使燕軍進入都城。宮中起火,建文帝不知下落。

燕王朱棣入京师应天府后,建文帝在宫中举火,皇后焚死,建文帝本人及其太子朱文奎则不知所踪,至今其下落仍是未定論的历史之谜。有稱其從地道逃亡,也有別史稱其離宮後出家為僧。

朱棣入京後,先捕殺齊泰、黃子澄、方孝孺及大批忠于建文帝的官員後,方稱皇帝,是為明成祖。當時駙馬都尉梅殷在軍中,從黃彥清之議,為建文帝發喪,諡「孝愍皇帝」,廟號「神宗」,但是不被成祖承認。

雖然朱棣宣稱在宮中找到建文帝的屍體,並為他舉行葬禮,但朱棣對建文帝未死的傳言不敢掉以輕心。建文帝年仅2岁的幼子朱文圭被废为庶人,并囚禁于凤阳广安宫。建文帝的三个弟弟原本封为亲王,尚未就藩,朱棣将他们降为郡王;年长的朱允熥和朱允熞先被封至福建漳州和江西建昌,旋被召回京师(南京),以“不能匡正建文帝”为由废为庶人,并囚禁于凤阳,只留下年幼的朱允熙给朱标奉祀,而不久之后朱允熙也于永乐四年死于火灾。

溥洽是建文帝主錄僧,當時傳聞他知道建文帝出逃的事,朱棣遂以其它罪名囚禁溥洽長達十餘年,直到姚廣孝病危時請求朱棣釋放溥洽,溥洽才獲釋。

明成祖即位后,不承认建文帝的正统性,下令销毁建文朝史料,并先后三次修改明太祖实录。成祖还下令作《奉天靖难记》,对懿文太子及建文帝多加诋毁。

正統五年,有僧自雲南至廣西,詭稱建文皇帝。隨後被逮捕調查,乃是鈞州人楊行祥,隨後下獄而死。同行十二位僧侶,皆戍遼東。隨後,雲南、貴州、四川等地均相傳有帝為僧時往來跡。正德、萬曆、崇禎年間,諸位大臣請求續封建文帝,及加廟諡,均未成行。虽然《太宗实录》(成祖原廟號太宗),称建文帝被朱棣以天子礼下葬,但崇祯帝在位时却亲口承认建文并无陵墓。崇禎十七年(1644年)五月,南明的弘光帝在南京即位,於同年七月為建文帝君臣平反,上庙号「惠宗」,谥号为「嗣天章道诚懿渊功观文扬武克仁笃孝让皇帝」。清朝乾隆元年,乾隆帝詔廷臣集議,追諡曰「恭閔惠皇帝」,故後世也稱建文帝為「明惠帝」。

2008年1月,福建省宁德市金涵乡上金贝村发现的一个和尚墓被认为是惠宗的墓葬所在,这个墓穴也是迄今为止福建发现的最大的和尚墓。然而,建文帝的最終下落至今仍是不解之謎,一说建文帝藏身于湖南省永州市新田县。

明建文帝在登基后不久,即重新選拔六部官員,其中大量官员在靖難之役中死亡;在战事中陣亡、拒絕與燕王朱棣合作而自殺或不屈而亡,其中包括禮部尚書陳廸,兵部尚書齊泰、鐵鉉,刑部尚書暴昭、侯泰,左都御史景清,右都御史練子寧、翰林院方孝孺等。



\subsection{建文}

\begin{longtable}{|>{\centering\scriptsize}m{2em}|>{\centering\scriptsize}m{1.3em}|>{\centering}m{8.8em}|}
  % \caption{秦王政}\
  \toprule
  \SimHei \normalsize 年数 & \SimHei \scriptsize 公元 & \SimHei 大事件 \tabularnewline
  % \midrule
  \endfirsthead
  \toprule
  \SimHei \normalsize 年数 & \SimHei \scriptsize 公元 & \SimHei 大事件 \tabularnewline
  \midrule
  \endhead
  \midrule
  元年 & 1399 & \tabularnewline\hline
  二年 & 1400 & \tabularnewline\hline
  三年 & 1401 & \tabularnewline\hline
  四年 & 1402 & \tabularnewline
  \bottomrule
\end{longtable}


%%% Local Variables:
%%% mode: latex
%%% TeX-engine: xetex
%%% TeX-master: "../Main"
%%% End:
