%% -*- coding: utf-8 -*-
%% Time-stamp: <Chen Wang: 2019-10-18 16:51:41>

\section{宣宗\tiny(1425-1435)}

明宣宗朱瞻基(1399年3月16日-1435年1月31日),或稱宣德帝,明仁宗皇长子,永樂九年(1411年)立為皇太孫;永乐二十二年(1424年)十月立為皇太子。洪熙元年(1425年)即位,年號宣德,明朝第5位皇帝,在位十年,享年37歲。宣德元年(1426年)平定高煦之亂,和其父仁宗一样,比较能倾听臣下的意见,聽從閣臣楊士奇、楊榮、楊溥等建議,停止對交阯用兵,与明仁宗并称「仁宣之治」,宣宗时君臣关系融洽,经济也稳步发展。不過,他也開啟此後宦官干政的局面。

明成祖時,朱瞻基父親朱高熾(仁宗)為太子,生性仁厚端重,但有時不免失之於懦怯。成祖最喜愛次子漢王朱高煦,覺得他最像自己,有心廢太子立漢王,但徐皇后和大臣們一直阻攔。而且朱瞻基自幼聰慧好學,與生母張氏皆深得成祖的喜愛,所以最終才沒有廢太子,並對朱瞻基悉心栽培。永樂九年(1411年)十一月立為皇太孫,數度隨成祖征討。永乐二十二年(1424年)仁宗即位,十月朱瞻基被立為皇太子。洪熙元年(1425年)四月,因南京地震多發,奉旨前往居守;同年六月仁宗駕崩,宣宗繼位。

明宣宗在位十年,重点在治理内政方面。宣德元年(1426年)平定汉王朱高煦的叛乱,宣宗原先只將他禁錮,仍前往探视,却被朱高煦使腿将其绊倒,宣宗一怒,将朱高煦用鼎扣住,烧烤至死,諸子全部處死。为了休兵养民,宣宗一改永乐时期的讨伐政策,主动从交阯撤兵。

宣宗整顿统治机构,罢免「贪津不律」、「不达政体」、「年老体疾」的官员,实行精简和裁冗措施,以振朝风。而在用人方面限制入仕人数,实行保举和欠任。宣宗实行一些减轻民困的措施,减免税粮、复业流民、赈灾救荒等。宣德三年出塞,并修建永寧、隆慶諸城。

在宦官问题上,因明代初期宦官多由藩屬國進貢或沒入各地罪犯家屬,在語言溝通上發生很大問題,言不同語只好以書同文來解決,宣德元年(1426年),明宣宗下令設置內書堂,教導宦官們讀書。不過,明太祖苦心謀劃的女官制度雖經成祖時期略加破壞,在此時仍發揮其防制閹黨之禍的功用,可是宣宗下令容許教導宦官讀書一舉,无意中卻開啟了明代宦官干政之先兆,尤其在明神宗後,因氣候變遷造成北方官話區大量貧困百姓自宮入朝廷謀職,萬曆至崇禎(1573-1644年)這71年間自宮入廷的閹宦總計高達三萬人,使得教導宦官成為明朝覆滅的其中原因,也是最受後世批評之處。不過與唐朝相比,明代皇帝極權之盛, 使終明一朝皇帝亦不至受宦官控制,一般而言亦只是通過宦官來處理政務及制約大臣的權力。

宣德五年(1431年1月),宣宗以外番多不來朝貢為由,命令鄭和再次出航。返航期間,鄭和因勞累過度於宣德八年(1433年)四月初在印度西海岸古里去世。船隊由太監王景弘率領返航,宣德八年七月初六(1433年7月22日)返回南京。第七次下西洋人數據載有27550人。這也是最後一次下西洋。

宣德十年(1435年)正月初三,皇帝崩于乾清宫,时年37岁,谥号宪天崇道英明神圣钦文昭武宽仁纯孝章皇帝。庙号宣宗。宣德十年六月廿一日,梓宫葬入景陵。

安南人黎利反叛,屡次打败官军。黎利请示朝廷,请求重新立陈氏之后为安南国王。朱瞻基认为国中疲惫,远征无益,于是答应了他,册封陈暠为安南国王,罢征南兵。后来黎利篡夺陈暠之位而自立为王。派人入朝纳贡谢罪,请求皇帝册封群臣。有人请求皇帝讨伐黎利,朱瞻基不许,册封黎利为安南国王。安南国也就是交趾国,自此以后朝贡不绝。

朱瞻基担心秋高马肥时蒙古人侵犯边疆,于是整顿兵马,驻扎喜峰口以待敌军。守将奏报兀良哈率领万名铁骑骚扰边疆,朱瞻基精选铁骑兵三千飞奔前往。敌军望见远处来军,以为是戍守边疆之兵,即以全军来迎战。朱瞻基命令将铁骑分为两路夹攻敌军,并且亲自射杀敌军先锋,杀死三人。两翼飞矢如云,敌人不敢前进。继而,朱瞻基又命连续发射神机铳,敌军人马死伤大半,剩下的全部溃逃。朱瞻基用数百铁骑直驱前行,敌人看到黄龙旗,才知道是皇帝亲征,于是全部下马拜倒在地请降,朱瞻基将这些人捆缚抓获,大胜而归。

《明史》赞誉宣宗:“仁宗为太子,失爱于成祖。其危而复安,太孙盖有力焉。即位以后,吏称其职,政得其平,纲纪修明,仓庾充羡,闾阎乐业。岁不能灾。盖明兴至是历年六十,民气渐舒,蒸然有治平之象矣。若乃强藩猝起,旋即削平,扫荡边尘,狡寇震慑,帝之英姿睿略,庶几克绳祖武者欤。”

《國榷》:“谈迁曰:国初严御,每重囚岁械入京辄千百,簿尉巡檄之任,辄烦圣虑,盖详极矣。宣宗幼侍文皇帝出入塞垣,深谙民事。及即位,遽有乐安之驾,非素才武,畴克灭此而朝食也者?然兵不轻试,惓惓以生灵为念。水旱朝奏,赈贷午曁。亲阅囚牍,多所释遣。好文学之士,一才一技,皆被甄录。盖睿质天纵,文翰并美,而不矜其能,尝有自下之色。国家之治,宽严有制,烦简有则,帝实始之。而於废胡后,弃南交,孰为帝谅者?呜呼!废后非盛德事也,其弃南交,比於汉之朱崖矣。”

《名山藏》:“高皇帝承胡元縱弛之弊,宏振威武以儆天下,成祖以英達之資纘緒大服,海內竦然,振厲者五十餘年。昭皇帝(明仁宗)至德深仁不久於位,章帝(明宣宗)繼之,乃涵濡以醇懿陶埴,以德義聞四方。”

《朝鮮文宗實錄》:“上(朝鮮文宗)謂代言等曰: "尹鳳率爾告予曰: 「洪熙皇帝及今(宣德)皇帝, 皆好戲事。 洪熙嘗聞安南叛, 終夜不寐, 甚無膽氣之主也。’」知申事鄭欽之對曰:“尹鳳謂予曰: 「洪熙沈于酒色,聽政無時,百官莫知早暮。 今皇帝燕于宮中,長作雜戲。 永樂皇帝, 雖有失節之事, 然勤於聽政, 有威可畏。」 鳳常慕太宗皇帝, 意以今皇帝爲不足矣。”上曰:「人主興居無節, 豈美事乎?」”

宣德皇帝既是一个有较高文化素養的皇帝,又是一个喜欢射猎、美食、鬥促织(蟋蟀)的皇帝。《聊齋誌異》裡的名篇《促織》裡的皇帝正是明宣宗,人稱“促织天子”,吳偉業有《明宣宗御用戧金蟋蟀盆歌》。

\subsection{宣德}

\begin{longtable}{|>{\centering\scriptsize}m{2em}|>{\centering\scriptsize}m{1.3em}|>{\centering}m{8.8em}|}
  % \caption{秦王政}\
  \toprule
  \SimHei \normalsize 年数 & \SimHei \scriptsize 公元 & \SimHei 大事件 \tabularnewline
  % \midrule
  \endfirsthead
  \toprule
  \SimHei \normalsize 年数 & \SimHei \scriptsize 公元 & \SimHei 大事件 \tabularnewline
  \midrule
  \endhead
  \midrule
  元年 & 1426 & \tabularnewline\hline
  二年 & 1427 & \tabularnewline\hline
  三年 & 1428 & \tabularnewline\hline
  四年 & 1429 & \tabularnewline\hline
  五年 & 1430 & \tabularnewline\hline
  六年 & 1431 & \tabularnewline\hline
  七年 & 1432 & \tabularnewline\hline
  八年 & 1433 & \tabularnewline\hline
  九年 & 1434 & \tabularnewline\hline
  十年 & 1435 & \tabularnewline
  \bottomrule
\end{longtable}


%%% Local Variables:
%%% mode: latex
%%% TeX-engine: xetex
%%% TeX-master: "../Main"
%%% End:
