%% -*- coding: utf-8 -*-
%% Time-stamp: <Chen Wang: 2019-12-26 15:07:43>

\section{光宗\tiny(1620)}

\subsection{生平}

明光宗朱常洛(1582年8月28日-1620年9月26日),或稱泰昌帝,明朝第15代皇帝,年号泰昌,庙号「光宗」,谥号“崇天契道英睿恭纯宪文景武渊仁懿孝贞皇帝”。

明神宗长子,万历十年(1582年)八月生,母恭妃王氏原是祖母李太后身边的宫人。不久,明神宗郑贵妃生三子朱常洵,深得宠爱。长子朱常洛一直受到冷遇,群臣纷纷上书要求立储,是為國本之爭,明神宗要不是贬斥群臣,就是虚与委蛇地敷衍應付。祖母李太后以为不妥。一日,李太后询问神宗未立朱常洛为太子的缘故。神宗说:他是宫人所生。李太后大怒:你也是宫人所生(李太后亦是宫人出身)。神宗听后惶恐,伏地不敢起。

万历二十九年(1601年)十月,明神宗被迫册立长子朱常洛为太子,同时,立三子朱常洵为福王、五子朱常浩为瑞王、六子朱常润为惠王、七子朱常瀛为桂王。太子朱常洛以仁厚著称,朝野皆认为其将来可为明君。但常洛的地位不穩固,郑贵妃時時刻刻想要為朱常洵爭奪儲君之位,引發了兩次妖書案,牽連眾多大臣。而後,甚至有郑贵妃手下的兩名宦官指使刺客,欲以木梃刺殺朱常洛,是為梃击案,神宗為了不牽連郑贵妃,將該刺客、宦官等三人全部殺死。

朱常洛被立为太子后,就移居慈庆宫,从此与其母王恭妃被隔绝不得相见。万历三十四年(1606年),朱常洛的妾侍王氏生下皇长孙朱由校(日后的明熹宗),神宗为表庆祝,为李太后加尊号,又进封王恭妃为皇贵妃,赐金册金宝,但仍将其屏居景阳宫。万历三十九年九月十三日(1611年10月18日),王恭妃病笃,朱常洛闻言急往景阳宫探视,见景阳宫门深锁,于是破坏门锁入内探视。当时王恭妃已双眼失明,于是以手代眼,拉着朱常洛的衣角:“儿长大如此,我死何恨!”言毕王恭妃便与世长辞。《酌中志》则记载为王恭妃病重时太子每日从苍震门入内问安;《先拨志始》更记载王恭妃察觉到郑贵妃家人偷听,提醒太子,结果母子俩直到王恭妃去世也没有说话。大学士叶向高说:“皇太子母妃薨,礼宜从厚。”神宗不应,复请,才得到允准。

万历四十八年(1620年)七月二十一日,明神宗驾崩。太子朱常洛立即发内帑(皇帝私房钱)百万犒赏边关将士。停止所有矿税,召回以言得罪的诸臣。不久,再发内帑百万犒边。八月即位,改元泰昌,是为明光宗。福王生母鄭貴妃為了攏絡明光宗,獻上四位美女。明光宗縱慾過度不久病倒,太監崔文升進以瀉藥而狂瀉。在位不足三十天的明光宗在九月初一因服用李可灼的紅丸而猝死駕崩,史稱紅丸案。

在短短的一个月,明光宗在群臣的帮助下,也做了不少实事,比如:废矿税、饷边防、补官缺。

首先下令罢免全国范围内的矿监、税使,停止任何形式的的采榷活动。矿税早为人们所深感厌恶,所以诏书一颁布,朝野欢腾。

其次是饷边防。明光宗下令由大内银库调拨二百万两银子,发给辽东经略熊廷弼和九边巡抚按官,让他们犒赏将士;并拨给运费五千两白银,沿途支用。明光宗还专门强调,银子解到后,立刻派人下发,不得擅自入库挪为它用。

第三件事是补充官缺。朱常洛先命令礼部右侍郎、南京吏部侍郎二人为礼部尚书兼内阁大学士;随后,将何宗彦等四人均升为礼部尚书兼内阁大学士;启用卸官归田的旧辅臣叶向高,同意将因为“上疏”爭國本获罪的三十三人和为矿税等获罪的十一人一概录用。因此有人感慨明光宗矫枉过正,造成了前所未有的“官满为患”的局面。

因光宗即位一個月即告駕崩,该陵墓原为景泰帝所建,因景泰帝為英宗所贬,葬于西郊金山,所以空出一处皇陵。由于明光宗在位时间仅29天,来不及修建陵墓,故继位的長子明熹宗朱由校将光宗安葬于此陵墓。

《明實錄》:“自古帝皇仁心仁聞洽于天下,未有不須久道而後成者,必世後仁聖人言之矣。乃光宗貞皇帝在位僅三旬,升遐之日,深山窮谷莫不奔走悲號,何?聖化之神感孚若是速也。蓋帝睿質夙成,蚤親師傳,養德青宮已洞悉四海之難艱。故當神皇晏駕時,遺詔未頒,德音據播 ;大寶初嗣,仁政沛施。捐朽蠹而九塞飽騰,撤狐蟊而廛勸動政。地廣股肱之助,諫垣充耳目之司。黃髮並升于公庭,白駒不滯于空谷。至于虛懷延接一月,而三召臣工銳意圖。幾浹旬而兩蠲而稅額 。德意獨行,獨斷爕理,莫施其功,威權自攬。自綜執月,御不參其柄。鑠乎盛矣,曠千古而僅見者也,乃其尤難者以何思何慮之天,處若危若疑之地。冲齡出講,已歷艱辛,而容色溫然,動止泰然。內庭有菀枯之形,若勿知也者;外庭有羽翼之激,若勿聞也者。即冊立,尋常事耳。時而舉碁,時而反汗。大臣去,小臣譴,宜何如動于耳目者。 而帝也,有夔夔無慄慄。潛之又潛,巧伺者不能窺,善孽者不能中。福藩就國,慟哭抱持。張差發難,帝侍神皇。左右親傳睿旨,曉諭百官羣囂遂息,所全實多。登極後即遵遺命進封皇貴妃,廷臣力爭,竟不忍奪以戚畹,哀請而後止,毫不芥蔕于前事也。此即虞舜大孝何以加茲?以舜之孝,擴堯之仁,然則帝之所以感動人心又自有在,而非僅僅更張注措之迹者矣。夫官天下者,壽在令名;家天下者,壽在長世。神皇即不豫,何難四十日留也。使帝之出震未及而幹蠱,莫施天下之事將不可知。然則我國家億萬年無疆之祚,皆帝四十日之所延也。帝之功德又豈但在普天之思慕已哉,天眷宗社不虗也。”

\subsection{泰昌}

\begin{longtable}{|>{\centering\scriptsize}m{2em}|>{\centering\scriptsize}m{1.3em}|>{\centering}m{8.8em}|}
  % \caption{秦王政}\
  \toprule
  \SimHei \normalsize 年数 & \SimHei \scriptsize 公元 & \SimHei 大事件 \tabularnewline
  % \midrule
  \endfirsthead
  \toprule
  \SimHei \normalsize 年数 & \SimHei \scriptsize 公元 & \SimHei 大事件 \tabularnewline
  \midrule
  \endhead
  \midrule
  元年 & 1620 & \tabularnewline
  \bottomrule
\end{longtable}


%%% Local Variables:
%%% mode: latex
%%% TeX-engine: xetex
%%% TeX-master: "../Main"
%%% End:
