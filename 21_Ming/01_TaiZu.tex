%% -*- coding: utf-8 -*-
%% Time-stamp: <Chen Wang: 2019-12-26 15:06:23>

\section{太祖\tiny(1368-1398)}

\subsection{生平}

明太祖朱元璋(1328年10月29日-1398年6月24日),或稱洪武帝,明朝開國皇帝,原名朱重八,曾改名朱興宗,投军被郭子兴取名元璋,字国瑞,生於濠州钟离县。廟號「太祖」,谥號「开天行道肇纪立极大圣至神仁文义武俊德成功高皇帝」,統稱「太祖高皇帝」。在位三十一年,因年号洪武也俗稱洪武帝。太祖之後的皇帝除明英宗(二度在位),皆實行一世一元制。

朱元璋出身贫农家庭,幼时贫穷,曾为地主放牛。後因災變,曾一度剃髮出家,四出流浪,化緣為生,25岁(1352年)时,参加郭子兴领导的红巾军反抗蒙元政权。先後击败了陈友谅、张士诚等其他諸侯軍閥,统一南方,後北伐灭元,建立大一統的封建皇朝政權,国号“大明”。

明太祖在位期间,為其家族能夠長期統治平民用殘酷方法殺害了許多人, 自著大誥三編宣揚部份經過。據臣下劉辰所著國初事跡他又發明使用多種殘酷殺人方法。

明太祖下令农民归耕,奖励垦荒;大興移民屯田和军屯;组织各地农民兴修水利;大力提倡种植桑、麻、棉等经济作物和果木作物;下令解放奴婢;减免賦稅。派人到全国各地丈量土地,清查户口等等。经过洪武时期的努力,社会生产逐渐恢复和发展,史称「洪武之治」。同时立《大明律》,用严刑峻法管理百姓与官僚,禁止百姓自由迁徙,严厉打击官吏的贪污腐败,设立锦衣卫等特务机构,整肅顯貴的势力及他認為對他的朝廷有威脅的人、並废中书省,由皇帝直領各部,进一步加强了中央集权。驾崩後传位于嫡长孙朱允炆為明惠宗。

明太祖的生活儉樸、工作勤奮,在南京的皇宮內,沒有設立“御花園”,只有“御菜園”,其中種滿蔬菜,使得皇宮自給自足。大封宗籓,令世世皆食歲祿,不授職任事。至明朝中后期,朱元璋子孫人口繁殖至近百萬人。洪武元年令:「凡孝子順孫、義夫節婦、志行卓異者,有司正官舉名,監察御史、按察司體覆,轉達上司,旌表門閭。又令:民間寡婦,三十以前,夫亡守制,五十以後,不改節者,旌表門閭(貞節牌坊),除免本家差役。」洪武二十六年令:「凡婦人因夫、子得封者,不許再嫁。如不遵守,將所授誥赦追奪,斷罪離異。其有追奪為事官誥赦,具本奏繳內府,會同吏科給事中、中書舍人,於勘合低簿內,附寫為事緣由,眼同燒毀。」明朝婦女守寡盛行。又創立明朝入宮婦女生殉制度。

元文宗天曆元年九月十八日(1328年10月29日)未時,朱元璋出生於濠州钟离县东乡(今安徽省凤阳县小溪河镇燃灯寺村),排行第三。朱元璋先世家沛(今江苏沛县),後徙句容(今江苏省句容市)达百年之久。祖辈生活在古泗州(今江苏省盱眙县)。父親朱五四(後改為世珍),母親陳氏为濠州钟离县(今安徽省凤阳县)人。

朱元璋幼時甚貧困,並無法讀書,曾為地主放牛。牧童伙伴多人都奉朱为领袖,且日后成朱起义将领多人,至正四年四月(1344年)淮北大旱,引發饑荒,朱元璋初六父崩,初九兄薨,廿二日母崩,與仲兄極力營葬後秋九月入皇覺寺當行童。入寺五十日,因荒年寺租難收,寺主封倉遣散眾僧,朱元璋只得離鄉為遊方僧雲遊淮西潁州。

元至正八年(1348年),朱元璋游歷淮西、汝潁、泗等州完畢,返回皇覺寺并逐渐讀書识字。至正十二年(1352年)二月辛丑,身在皇覺寺多年的朱元璋受好友湯和來信勸說,到濠州投靠郭子興,參加紅巾軍。由於指揮有方,不久便成為郭子興身旁一名親兵并赐名元璋字国瑞,並娶郭子興養女马氏(即後來的馬皇后)。後來朱元璋見郭子興與其他濠州紅巾軍領袖如孫德崖、趙均用不和,屢有衝突,朱元璋不願涉及濠州內鬥,故主動要求返家鄉招募新兵,徐達、湯和等朱元璋兒時好友獲准隨行,不久朱元璋的部隊已有結集了數千人。次年,朱元璋部隊攻下滁州,成為他首個據點,同時也在攻佔滁州期間,李善長加入朱元璋部隊,成為他一個重要幕僚。此時,濠州的郭子興被孫德崖及趙均用迫走,前來滁州投靠朱元璋,由於朱元璋名義上仍是郭子興部下,朱元璋乃將滁州兵權交予郭子興。

至正十四年(1354年),張士誠據高郵,自稱為誠王,十五年,元朝丞相脫脫率軍進攻高郵,分兵攻六合,六合乃滁州屏障,故朱元璋領兵援六合,幸好脫脫被誣陷而被迫交出兵權,元軍不戰自潰,滁州也轉危為安。朱元璋見滁州地小,建議進攻長江北岸的和州。朱元璋攻下和州不久,郭子興病故,郭子興次子郭天敘被立為都元帥,朱元璋與郭子興妻弟張天祐為副元帥,遥奉韩林儿的大宋龙凤政权。同年夏,常遇春、廖永安、俞通海歸附朱元璋,使得其軍著手渡江攻入采石、太平路,並計劃攻取集庆路(今南京市)。此時,元軍降將陳野先願協助紅巾軍攻集慶,郭天敘與張天祐感軍功不及朱元璋,故決定在陳野先引領下,親自領軍攻打集慶。結果紅巾軍攻集慶時陳野先叛變,郭、張二人被殺,陳野先也死於亂軍中。郭天敘與張天祐死後,朱元璋成為都元帥,盡領郭子興舊部。至正十六年(1356年),朱元璋領軍再次攻打集慶,結果集慶被朱元璋部隊一舉攻陷,朱元璋將這裡作為自己的根據地,並改名為應天府。至此,朱元璋以應天府為中心,與元朝軍隊、張士誠、徐壽輝等部形成犬牙交錯之勢。

朱元璋攻佔應天後,開始攻佔應天周邊地區以鞏固防務。至正十六年,遣徐達攻佔鎮江、鄧愈克廣德,次年,遣耿炳文克長興,徐達克常州,而朱元璋親自率眾攻取寧國。隨後趙繼祖克江陰、徐達克常熟。胡大海克徽州、常遇春克池州,繆大亨克揚州。至正十八年,朱元璋親取婺州。明年,朱元璋陸續攻佔浙東餘下各地,常遇春克衢州、胡大海克處州,至此朱元璋部控制江左、浙右各地,向西與陳友諒部相鄰。朱元璋攻下浙東後,小明王升朱元璋為儀同三司江南等處行中書省左丞相,同時朱元璋也得浙東名士如朱升、劉基相助,朱元璋採取朱升「高築牆、廣積糧、緩稱王」的建議,採取穩健的進攻措施;並且遵照劉基「先漢後周」之策略,着手對江南各勢力進行對抗。

至正二十年,陳友諒攻陷太平路,隨後弒主徐壽輝、稱帝建國,國號漢,之後傾全軍攻應天府。朱元璋與劉基設計,先命胡大海進攻信州,斷陳友諒後援,再命部下康茂才詐降作陳友諒的內應,引漢軍主力進入朱元璋在應天城外龍灣設下的埋伏中,結果漢軍被朱元璋軍隊大敗,隨後朱元璋攻取太平、安慶、信州等地。。至正二十一年,朱元璋改樞密院為大都督府,重新整理軍制。北結察罕帖木兒、密通方國珍,而與正面的陳友諒部進行會戰。同年攻克江州、南康、建昌、撫州等地。次年,佔領龍興,改洪都府(今江西南昌)。

至正二十三年(1363年),张士诚派部将吕珍围攻退守安丰的小明王韓林兒及丞相劉福通,朱元璋不顧劉基反對,派軍北上解安豐之圍,結果刘福通战死,韩林儿被朱元璋救出。此后,韩林儿被朱元璋安置在滁州,仍然被奉为皇帝。陳友諒趁朱元璋主力軍北上,率六十萬水軍進攻朱元璋根據地,首先圍攻洪都,但朱元璋姪朱文正堅守洪都兩個多月,待朱元璋親率二十萬部隊馳援,陳友諒大軍改往鄱陽湖與朱元璋大軍交戰,史稱“鄱陽湖之戰”。陳友諒自恃巨艦出戰,採用炮攻,朱元璋險些負傷被擒。隨後,朱元璋利用東北風而改用火攻,致使陳友諒部大量受損。之後朱元璋利用鄱陽湖水位降低便於小舟活動,改為分兵水路圍攻陳友諒。陳友諒中箭身亡,漢軍潰敗。隨後朱元璋圍攻武昌,并盡佔湖北各地。次年,朱元璋自立為「吳王」,以李善長為右相國,徐達為左相國,常遇春、俞通海為平章政事,立子朱標為世子。次月再次親征武昌,陳友諒之子陳理舉降。隨後吳軍相繼攻克廬州、吉安、衡州。至正二十五年,吳軍繼續攻佔寶慶、贛州、浦城、襄陽,同年冬,下令討張士誠。次年,吳軍再次攻破湖州、杭州。再一年,徐達克平江,張士誠被俘,至此朱元璋一統江南。至正二十六年(1366年),朱元璋派廖永忠迎接韩林儿至金陵應天府,途中在瓜步渡长江时,韩林儿所乘船只沉没,韩遇难。

至正二十七年(1367年),朱元璋命湯和為征南將軍,討伐割據浙東多年的方國珍。隨後制定北伐战略:先攻取山東,其次進攻河南,再次攻佔陝西潼关,最後再進軍元大都。隨後命徐達為征虜大將軍,常遇春為副將軍,帥師二十五萬,由淮河進入,北取中原。并命胡廷瑞為征南將軍,何文輝為副將軍,進攻福建。同年,方國珍投降,徐達攻破山東濟南,胡廷瑞下邵武,湯和、廖永忠由海道攻克福建福州。北伐一直持續到洪武年間,徐達、常遇春隨後攻佔整個河南、山西,最終直取元大都(今北京)。

至正二十八年正月初四(1368年1月23日),朱元璋在應天府登基即位,建國號大明,年號洪武,是為「明太祖」。以應天為「南京」,開封為「北京」。同年八月初二(9月14日),大將徐達攻克大都,元朝覆亡。由于幼年对于元末吏治痛苦记忆,即位后一方面減輕農民負擔,恢復社會的經濟生產,改革元朝留下的糟糕吏治,懲治貪污的官吏,社會經濟得到恢復和發展,史稱洪武之治。明太祖確立了里甲制,配合賦役黃冊戶籍登記簿冊和魚鱗圖冊的施行,落實賦稅勞役的徵收及地方治安的維持。

太祖平定天下後,大封諸將為公侯,部份追封為王。初封六公,其中以五大將、一大臣為開國元勳。分別為:韓國公李善長、魏國公徐達、鄭國公常遇春、曹國公李文忠、宋國公馮勝、衛國公鄧愈。而後又追封胡大海為越国公、戰死的丁德興為濟國公,湯和為信國公、馮國用封郢國公。次年,明太祖於雞鳴山立功臣廟,六月初三日廟成,太祖親定功臣位次,以徐達為首,次常遇春、李文忠、鄧愈、湯和、沐英、胡大海、馮國用、趙德勝、耿再成、華高、丁德興、俞通海、張德勝、吳良、吳禎、曹良臣、康茂才、吳復、茅成、孫興祖凡二十一人。死者像祀,生者虛位。又以廖永安、俞通海、張德勝、桑世杰、耿再成、胡大海、丁德興七人配享太廟。此位序屡经删汰,已非洪武二年所定名单位次。

随後,太祖进一步加强中央集权。洪武三年(1370年),杀中书左丞杨宪。洪武四年七月十一(1371年8月21日),傅友德攻克成都,明朝平定四川。洪武五年四月二十三日(1372年5月26日),廖永忠率明军平定广西,洪武五年六月初三(1372年7月3日),傅友德大败元军,明朝平定甘肃。洪武六年(1373年),太祖鑒於開國元勛多倚功犯法,虐暴鄉閭,特命工部制造鐵榜,鑄上申戒公侯的條令,類似戰國時代的「鑄刑鼎」。洪武八年(1375年),德庆侯廖永忠因僭用龙凤诸不法事,赐死。洪武十二年(1379年),贬右丞相汪广洋于广南,旋赐死。洪武十三年(1380年),胡惟庸案发,左丞相胡惟庸被诛,太祖罢中书省,分中书省之权归于六部,直接归皇帝掌管。洪武十五年(1382年),设立锦衣卫,加强明朝特务统治。1382年1月6日,明军在云南昆明附近大败元朝军队,元梁王自杀,1382年4月7日,蓝玉、沐英攻克大理,段氏投降,明朝平定雲南。洪武十八年(1385年),郭桓案发,由于涉案人员甚多,太祖將六部左右侍郎以下官员皆處死,各省官吏死於獄中達數萬人以上。

洪武二十三年(1390年),李善長的家奴盧仲謙告發李善長與胡惟庸往來勾結,以「狐疑觀望懷兩端,大逆不道」見誅,接續又誅殺陸仲亨與唐勝宗、費聚、趙庸三名侯爵,株連被殺的功臣及其家屬共計達三萬餘人,連「浙東四先生」(刘基、宋濂、章溢、叶琛)亦不能免,并頒布《昭示奸黨錄》。洪武二十六年(1393年),藍玉被錦衣衛指揮蔣瓛密告謀反,史称“藍玉案”。此案牵连到十三侯、二伯,連坐族誅達一萬五千人,明朝建国功臣因此案幾乎全亡。此時太祖又頒布《逆臣錄》,詔示一公、十三侯、二伯。洪武二十七年(1394年),太祖杀江夏侯周德兴以及颖国公傅友德,在捕鱼儿海战役中立功的定远侯王弼亦被赐死。洪武二十八年(1395年),开国六公爵最後一位僅存者冯胜被杀。

在处理内政同时,太祖亦多次籌劃北伐蒙古以保障北方邊塞的安寧,大勝。並曾成功在甘肅擊敗王保保(1372年)、在东北逼降納哈出(1387年)、在蒙古高原幾乎活捉元主脫古思帖木兒(1388年)。同时太祖进军辽东,使朝鮮王朝等归顺(1388年)。

洪武三十一年閏五月初十日(1398年6月24日),朱元璋崩逝於南京皇宮內,享壽七十歲,在位三十一年。與已故的妻子馬皇后兩人一起長眠於南京紫金山明孝陵。《明朝小史·卷三》載,責殉諸妃,強迫伺寝宫人尽数殉葬。《彤史拾遺記》記載,太祖以四十六妃陪葬孝陵,其中所殉,惟宮人十數人。

新任皇帝惠宗遵照遺命。洪武三十一年六月甲辰,上謚曰“欽明啟運俊德成功統天大孝高皇帝”,廟號太祖。永樂元年六月十一日丁巳,增諡“聖神文武欽明啟運俊德成功統天大孝高皇帝”。嘉靖十七年十一月朔,改諡“開天行道肇紀立極大聖至神仁文義武俊德成功高皇帝”。到了清朝,康熙帝历次南巡必跪拜孝陵,曾立碑「治隆唐宋」赞誉其功。中華民國建立初,孫文至孝陵祭告朱元璋。

朱元璋一直以來都是以猛治国。持正面評價者通常都是從其大力打擊貪污,恢復經濟著眼,歷史記載朱元璋是少數極力勤政的皇帝;而持負面評價者,則多從其高壓統治著眼,以猛著称,他的“重典治国”思想不只為遏制官僚腐败。亦顯現在清洗权贵势力、以特務錦衣衛控制政治、又用文字獄及廷杖大臣,以立帝王權威。

明初沿袭元朝制度,设立中書省,置左、右丞相。甲辰正月,初置左、右相國,其中李善長為右相國,徐達為左相國。洪武元年(1368年),改為左、右丞相。由中书省统六部,但不设置中書令。

洪武十三年(1380年),胡惟庸案之后,太祖罢中书省,分中书省之权归于六部。原中書省官屬盡革,惟存中書舍人。至此,秦、漢以降實行一千六百餘年的宰相制度自此廢除,相權與君權合而為一,施行軍權、行政權、監察權三權分立的國家體制。

由於國家事務繁多,皇帝無法處理,洪武十五年九月罷四輔官,仿宋殿閣制設內閣。內閣只為皇帝的顧問,雖無宰相之名,但有宰相之實。此外他仍沿用元朝制度,在中央設置吏、戶、禮、工、刑、兵六部。并設立都給事中六人,分吏、戶、禮、工、刑、兵六科,每科一人;此外建立五寺包括大理寺、太常寺、光祿寺、太僕寺、鴻臚寺等五寺制度。此外他還沿襲元的監察制度,設立御史台,有左右御史大夫各一名;不久改為都察院,下設若干監察御史,負責監督各級官吏。除此他还颁布《大明律》等,对官吏管理进行规制。

为了加强对臣民的控制和监视,太祖设置了巡检司和锦衣卫。巡检司主要是负责全国各地的关津要冲的把关盘查,缉捕盗贼,盘诘伪奸;锦衣卫则负责秘密侦察大小官吏活动,随时向皇帝报告不公不法之徒。同时太祖还授予锦衣卫侦察、缉捕、审判、处罚罪犯等一切大权,锦衣卫正式成为直屬皇帝的情报机构。

太祖出身貧寒,對政治貪污尤其憎惡,其對貪污腐敗官員處以極嚴厲的處罰。太祖在政期間,大批不法貪官被處死,包括開國將領朱亮祖,女婿駙馬都尉歐陽倫,其中甚至因為郭桓案、空印案殺死數萬名官員。由於太祖的吏治嚴厲,在明初相當長一段時間,官員腐敗的情況得到有效遏制。然而,随着大明江山逐步稳定,再加上军事和皇室贵族战功大,享有很高的社会特权,不少人迅速腐化变质。。朱元璋开展雷厉风行的肃贪运动,历时之久、措施之严、手段之狠、刑罚之酷、杀人之多,为几千年历史所罕见。尽管朱元璋反贪决心大、力度猛、出奇招,使腐败现象得到一定程度的遏制,也一度取得了“阶段性的成果”,但還是未能達到徹底清除人類貪欲權位腐敗的本性。

太祖性格多疑,對功臣有所猜忌,恐其居功枉法,圖謀不軌。这些特权阶级杀人伤人、霸占土地、逃税漏税、恃强凌弱、奸淫妇女、吃喝嫖赌、贪污纳贿,甚至造刀枪、穿龙袍的都有。面对这种对王朝的长治久安构成严重威胁局面,太祖把这些特权阶级无情地清洗。廖永忠和 朱亮祖 先後死於非命。隨後太祖以擅權枉法之罪名殺胡惟庸,又殺御史大夫陳寧、御史中丞塗節等人。之後李善長亦被牽連,家屬七十餘人被殺,總計株連者達三萬餘人。此後的藍玉案中,連坐被族誅的有一萬五千餘人。但紀非錄所記載太祖的兒子諸藩王犯有很多暴行,太祖則只是輕微勸戒了事。太祖還通過設立錦衣衛(洪武二十年废除)、詔獄、廷杖等機構或制度,打擊功臣、特務監視等一系列方式加強皇權控制。

太祖遵古制,王命法:三十受兵、六十歸兵。國有三軍,所以誡非常,伐無道,尊宗廟,重社稷,安不忘危。太祖令諸藩鎮守天下,又各領兵權,這固然是親親之情,信任無以復加,卻也未必就沒有帝王心術。強藩林立,能做皇帝的卻始終只有一個,諸藩勢力犬牙交錯,必然相互牽制,相互監視,除非朝廷中樞衰弱之極。當中樞真的衰弱至極時,就算沒有藩王,也會被權臣取而代之。自三皇五帝,以一介布衣而成天子者,唯漢高祖與太祖,其他帝王,大都是前朝重臣或一方豪強而黃袍加身。所以由自己子孫取代無能之君,也勝過將江山付與外人之手,如此可保朱家數百年江山。

建国伊始,太祖就在《大明律令》的基础上制订颁行《大明律》,紧接着又亲自编定《明大诰》。1397年,太祖下詔正式颁布了《大明律》。《大明律》一共四百六十卷,分吏、户、礼、刑、兵、工六律,简于《唐律》,严于《宋律》。《大明律》规定:“谋反”、“谋大逆”者,不管主犯还是从犯,一律凌迟,祖父,父、子、孙、兄、弟以及同居的人,只要是年满十六岁的都要处决。太祖立法一为治民,二为治吏,尤其是《明大诰》对贪官污吏的处决也十分严厉,可以视为反贪刑事特别法。只要是犯有贪污的官吏,一经查实,一律发配北方荒漠中充军,赃至六十两以上者枭首示众,仍剥皮实草。

太祖十分重視法律宣传,寫了大誥三編和大誥武臣,让臣民熟悉法律,不去犯禁。他還經常法外施刑,動輒凌遲。

早在朱元璋起兵时,他就多次强调军纪。他认为「攻克城池用武力,平定混乱用仁政」,杀人并非「勇猛」。要求部队不许滥杀无辜,还给予俘虏优待;同时还要求部队爱护百姓,不得随意焚烧抢掠乱杀百姓,他严令:「掠夺老百姓财物者处死,拆毁老百姓住房的处死。」由于朱元璋部队的军纪严明,朱元璋赢得了部属的尊重,也赢得了民众的支持。

明代早期軍隊的來源,有諸將原有之兵,有元兵及群雄兵歸附的,有獲罪而謫發的,而最主要的來源則是籍選,是由戶籍中抽丁而來。除此之外尚有簡拔、投充及收集等方式。洪武十三年(1380年),太祖廢除大都督府,並改为中军、左军、右軍、前军、后军等五军都督府。洪武十七年(1384年),太祖在全國的各軍事要地,設立軍衛,由都督府管理。一衛有軍隊五千六百人,其下依序有千戶所、百戶所、總旗及小旗等單位,各衛所都隸屬於五軍都督府,亦隸屬於兵部,有事從征調發,無事則還歸衛所。軍隊來源為世襲的軍戶,由每戶派一人為正丁至衛所當兵,軍人在衛所中輪流戊守以及屯田,屯田所得以供給軍隊及將官等所需。五军都督府有统兵权但无调兵权,兵部有调兵权而无统兵权,兩者互相制衡,互不統轄,各自與兵部直接聯繫,最後奏請皇帝裁定,以避免權力過大。

明代軍户是世襲制,一旦列入軍籍,世代都是軍人,朝廷有事要為朝廷作戰。軍丁一旦逃亡、病故、老疾或被虜,就要按軍籍所造之册,到該軍丁原籍追補本身或其親屬,以補足原數。

元朝初期,元世祖曾经遠征日本,导致日本念念不忘,于是终元之世,日本不与中国同好。明朝开国以后,太祖就派使臣持国书去日本、高丽、安南、占城四国,宣告元朝已经灭亡,现在的稱霸中国是大明,應奉大明为“正朔”来朝贡。高丽、安南、占城三国太祖使赴明称臣朝贺,惟独日本没有任何反应。令太祖更为恼火的是,不但日本人不来朝称臣,而且“乘中国未定,日本率以零服寇掠沿海”。同时,被太祖消灭的张士诚、方明珍等残部多逃亡海上,占据岛屿,勾結倭寇出没海上掳掠财货,辽宁、山东、福建、浙江、广东,“滨海之地,无岁不受其害”。

後来太祖喝令“日本国王”處理倭寇,结果使者被日本人殺害。消息傳回中國後,太祖大為怒火,批日本是“国王无道民为贼”的“跳梁小丑”。面对日本,太祖忍下了恶气,从此以后对日本使者一概驅逐處理,朝贡也一概拒绝接受,与日本不相往来。同时,太祖把朝鲜、日本、大琉球、小琉球、安南、真腊、暹罗、占城、苏门答腊、西洋、爪哇、彭亨、百花、三佛齐、勃泥等15国列为“不征诸夷”,写入《皇明祖训》,告诫子孙这些「蛮夷国家」如果不主动挑衅,就不许征伐。

公元1370年(洪武三年)太祖派遣莱州知府赵秩远赴日本。懷良親王经过赵秩的阐释明处外交政策打消了顾虑。不久懷良派遣僧人祖来跟随赵秩回明朝向进表笺。公元1371年(洪武四年)太祖派遣僧人祖阐、克勒等八人送日使归国,从此明朝和日本建立了外交关系。

公元1392年(洪武二十五年)七月,高丽大将李成桂发动兵变掌控高丽局势以后遣知密直司事赵胖至明朝礼部上表:“定昌府院君瑶权署国事,及今四年。瑶又昏迷不法,疏斥忠正,昵比谗邪,变乱是非,谋陷勋旧,谄惑佛神,妄兴土木,靡费无度,民不堪苦;子奭痴佁无知,纵于酒色,聚会群小,谋害忠直。又其臣郑梦周等潜成奸计,欲生乱阶,乃将勋臣李成桂、赵浚、郑道传、南訚等谮于权署国事,令有司论劾以致谋害,国人愤怨,共诛梦周。权署国事尚不悛改,又谋杀戮。举国臣民实虑社稷生灵俱被其害,惶惧失措,无可奈何,咸以为若所为难以主斯民奉社稷。洪武二十五年七月十二日,以恭愍王妃安氏之命,退居私第。窃念军国之务不可一日无统,择于宗亲,无有可当舆望者,惟门下侍中李成桂泽被生灵,功在社稷,中外之心夙皆归附。于是一国大小臣僚、闲良、耆老、军民臣等咸愿推戴,令知密直司事赵胖,前赴朝廷奏达,伏启照验,烦为闻奏,俯从舆意,以安一国之民。”太祖通过礼部传达圣旨:“三韩臣民既尊李氏,民无兵祸,人各乐天之乐,乃帝命也。虽然,自今以后慎守封疆,毋生谲诈,福愈增焉。尔礼部以示朕意。”李成桂遣门下侍郎赞成事郑道传赴京谢恩,并献马六十匹。

当年八月,李成桂又遣前密直使赵琳赴京进表:“权知高丽国事臣李成桂言:伏惟小邦自恭愍王无嗣薨逝之后,辛旽子禑冒姓窃位者十有五年矣。迄至戊辰春,妄兴师旅,将犯辽东,以臣为都统使,率兵至鸭绿江。臣窃自念小邦不可以犯上国之境,谕诸将以大义,即与还师,禑乃自知其罪,逊位子昌。昌亦暗弱,难以莅位,国人启奉恭愍王妃安氏之命,以定昌府院君王瑶权署国事。瑶乃昏迷不法,紊乱刑政,狎昵谗佞,贬斥忠良,臣民愤怨,无所控告。恭愍王妃安氏深虑其然,命归私邸。于是一国大小臣僚、闲良、耆老、军民等以为军国之务不可一日无统,推戴臣权知军国事。臣素无才德,辞至再三,而迫于众情,未获逃避,惊惶战栗,不知所措。伏望皇帝陛下以乾坤之量、日月之明,察众志之不可违、微臣之不获已,裁自圣心,以定民志。”朱元璋再通过礼部复旨:“高丽限山隔海,天造东夷,非我中国所治。尔礼部回文书,声教自由,果能顺天意合人心,以妥东夷之民,不生边衅,则使命往来,实彼国之福也。文书到日,国更何号,星驰来报。”

当年十一月,李成桂再遣艺文馆学士韩尚质至明朝上表:“窃念小邦王氏之裔瑶,昏迷不道,自底于亡,一国臣民推戴臣权监国事。惊惶战栗,措躬无地间,钦蒙圣慈许臣权知国事,仍问国号,臣与国人感喜尤切。臣窃思惟,有国立号诚非小臣所敢擅便。谨将“朝鲜”(箕子所建古国名)、“和宁”(李成桂诞生之地)等号闻达天聪,伏望取自圣裁。”太祖再通过礼部复旨:“东夷之号,惟朝鲜之称美,且其来远,可以本其名而祖之。体天牧民,永昌后嗣。”李成桂遣门下侍郎赞成事崔永沚谢恩,又遣政堂文学李恬送明朝颁赐的给前朝的高丽国王之印,并请更己名为“李旦”。

公元1394年(洪武二十七年)帖木儿帝国向明朝贡马,而且致国书。第二年,明朝派遣兵科给事中傅安率领使团往报。但当傅安等抵达帖木儿帝国国都撒马尔罕时,帖木儿打算要向东兴兵,攻打明朝了,于是扣押了傅安等人,而且百般的诱惑傅安等人归顺帖木儿,傅安被扣押十三年,坚贞不屈,维护明朝的尊严。一直到了帖木儿死了以后,他的孙子哈里嗣位,想和明朝和好,于是才放傅安等人回国。傅安回国以后又出使了中亚诸国。

公元1395年(洪武二十八年)十一月,李成桂遣艺文春秋馆太学士郑总赴京请诰命印章:“洪武二十五年七月十五日,差知密直司事赵胖奏达天庭,继差门下评理赵琳奉表陈奏,钦奉圣旨,许允权知国事。准奉礼部来咨內云:‘国更何号,星驰来报。准此。’即差知密直司事韩尚质赍擎奏本赴京,钦奉圣旨节该:‘东夷之号,惟朝鲜之称美,且其来远矣,可以本其名而祖之。钦此。’除钦遵外,洪武二十六年三月初九日,差门下评理李恬送纳前朝高丽国王金印,又于当年十二月初八日准奉左军都督督府咨,钦奉圣旨內一款节该:‘即合正名。今既改号朝鲜,表文仍称权知国事,未审何谋?钦此。’一国臣民战栗惶惧,咸请国王钦遵施行。见今虽称国王名号,窃缘未蒙颁降诰命及朝鲜国印信,一国臣民日夜颙望,仰天吁呼。伏请照验,烦为闻奏,乞赐颁降国王诰命及朝鲜印信施行。”朱元璋通过礼部下旨拒绝:“今朝鲜在当王之国,性相好而来王,顽嚣狡诈,听其自然,其来文关请印信诰命,未可轻与。朝鲜限山隔海,天造地设,东夷之邦也,风殊俗异。朕若赐与印信诰命,令彼臣妾,鬼神监见,无乃贪之甚欤?较之上古圣人,约束一节决不可为。朕数年前曾敕彼仪从本俗,法守旧章,令听其自为声教。喜则来王,怒则绝行,亦听其自然。尔礼部移文李成桂,使知朕意。”

明朝立国后日本因进入南北朝的大分裂时期后出现的、大量外出掠夺的武士阶层为主的倭寇骚扰入侵的恐惧,明政府立国后采取了一系列针对海患的闭关锁国政策:洪武三年(1370),明政府“罢太仓黄渡市舶司”;洪武七年(1374),明政府下令撤销自唐以来即存在的、负责海外贸易的福建泉州、浙江明州、广东广州三市舶司,中国对外贸易遂告断绝;洪武十四年(1381),太祖以倭寇仍不稍敛足迹,又下令禁濒海民私通海外诸国,此后每隔一两年即将该海禁政策再次昭示天下。自此,连与明朝素来交好的东南亚诸国也不能来华进行贸易和文化交流。

整个海禁政策从太祖开始,到了明穆宗在位期間被以“市通则寇转而为商,市禁则商转而为寇”为由实行开关(隆庆开关);至清初又开始一連串的闭关,清高宗時更推行“一口通商”政策、直至鸦片战争后,通行整个明清二代的海禁政策才被彻底打破。

元末之际,中國發生多次大規模的災荒饑饉疾病和瘟疫,以及連年戰爭,期間生产遭到严重破坏,人口也大量減少,经济全面崩溃,人民处在流离失所的过程中。大明建立並統一全國後,面对哀鸿遍野、饿殍满路的凄凉局面,太祖實行黃老治術治國,太祖说:天下初定,百姓财力困难,就像刚刚会飞的鸟不可拔羽,才种的树不可摇根一样。现在必须采取这种政策,同时主张藏富于民。

农业是明代社会最主要的生产部门。太祖在恢复和发展社会经济中,把发展农业放在了首位,为了保证农业第一线有足够的劳力资源。太祖通令全國,地主不得蓄养奴婢,所养的奴婢一律释放为良民。凡因饥饿而典卖为奴者,由朝廷代为赎身;嚴格控制寺院的發展,明令各州府县只能有一个大寺院,禁止四十歲以下的妇女当尼姑,严禁寺院收养童僧,二十岁以上的青年如果要是出家,必须得到父母和官方同意,出家后三年内还要赴京考試,不合格者潜发为民。這些政策的实施,使得社會增加了一只庞大的劳动力大軍。

全國的農業生產在大規模战争而遭受極大破壞的背景下得到很大程度的恢復,加上太祖在位期間大規模向淮河以北和四川的荒無之地、墾荒填充移民,使人口得以穩定增長。

此外他也實行屯田政策,軍屯面積佔全國耕地的近十分之一。此外,商屯也相當盛行,政府以買賣食鹽的專賣證(稱之為鹽引)作為交換,利誘商人將糧食運往邊疆,以確保邊防的糧食需求。明太祖也曾派遣國子監下鄉督導水利建設、赈灾,並以減免稅賦獎勵耕作。這些措施使得過去很多飽受戰亂損毀的地區恢復了生氣,使明朝的經濟得到了快速的恢復。

到洪武二十六年(1393年),全國有6500萬人,其中民戶佔6175萬人,軍戶佔325萬人。另外,其為了動員全社會,明太祖十分重視戶口普查,每個人有固定的義務。人民分為軍戶(弓兵、校尉、力士)、匠戶、民戶(马户、陵戶、茶戶、柴戶、阴阳戶、医戶)、灶戶,不允许隨便轉換工作,匠籍、軍籍比一般民戶地位低,不得應試,並要世代承襲。若想脫離原戶籍極為困難,須經皇帝特旨批准方可。各种活动也要引憑才合法。编成里甲,规 定了路引制度,也就是通行证制度。普通百姓只要走出出生地百里之外,就得持有官府开具的通行证,否则就以逃犯论处。

明朝初期實行「科舉必由學校」的政策,太祖多次強調:「古昔帝王育人材,正風俗,莫不先於學校。」明代洪武元年(1368年),詔開科舉,對制度、文體都有了明確要求。命令刘三吾等人刪節《孟子》中民貴君輕的內容,課試不以命題,科舉不以取士。。洪武年間,太祖共主持举办六次科考,七次发榜,共取一甲21名、二甲223名、三甲686名,合930名,平均每科取士155人,為明朝選拔輸送了大量有學識的官員,包括練子寧、黃子澄、解縉等一代名相。洪武三十年科舉時,因中進士者均為南方籍。太祖将试官二十餘人指為胡黨藍黨凌遲殺害,并自阅试卷,取中六十一人,皆为北方人,并于六月廷试。此外,他並將學校列為「郡邑六事之首」,以官學結合科舉制度推行程朱理學,并設立國子監等重要教育機構。由於太祖在位期間實行高壓的吏治政策,明初诗文三大家不得善終,後世不乏有學者主張太祖曾實行過一些文字獄。也有學者指出關於朱元璋嗜殺之事例,存有穿鑿附會的問題。

太祖崇尚简朴,也希望老百姓也勤俭节约。他规定靴子上不能有任何装饰。同时对于全国人民怎么穿衣;每个阶层佩戴什么样的首饰;盖什么样的房子;出行坐什么样的车子以及人们的行动举止也是朱元璋关注的焦点,因而制定了一系列规章制度,包括了生活的方方面面,其细致入微,可谓空前绝后。“洪武二十二年三月二十五日奉聖旨:“在京但有軍官軍人學唱的,割了舌頭;下棋打雙陸的,斷手;蹴圓的,卸脚;作買賣的,發邊遠充軍。”府軍衛千戶虞讓男虞端故違吹簫唱曲,將上脣連鼻尖割了。又龍江衛指揮伏顒與本衛小旗姚晏保蹴圓,卸了右脚,全家發赴雲南。又二十五年九月十九日,禮部榜文一款:“內使剃一搭頭,官民之家兒童剃留一搭頭者,閹割,全家發邊遠充軍。剃頭之人,不分老幼罪同。””(《客座贅語》卷十)

太祖对天下老年人施以尊重,颁布《存恤高年诏》。洪武二十年,太祖怕有关部门执行不力,就又叮嘱礼部尚书,要以皇帝的名义再次重申一下这项政策。在朝廷的要求和带动下,各地形成了尊老养老的风气,赡养老人的要求也渗透到各地家法族规之中。

对于社会的救济朱元璋也十分重视,洪武时期,荒政则受到朝廷高度重视。朝廷除了拨付救灾济贫款项,还侧重加强民众抗灾自救能力。面对天灾侵袭,朱元璋积极作为,既树立了朝廷的负责任形象,又增强了政府的凝聚力,赢得了民心。救灾济贫实为获取民心、形成治世的重要前提,为“洪武之治”的出现夯实了经济社会基础。

为了贬抑商人,太祖他特意规定,农民可以穿绸、纱、绢、布四种衣料。而商人却只能穿绢、布两种料子的衣服。商人考学、当官,都会受到种种刁难和限制。

太祖建立明朝前后,十分重视宗教问题,通过协调儒释道三者的关系,既稳定了局面,又争取了人心,为巩固明朝政权奠定了思想和群众基础。通过有效的宗教管理措施,把宗教的发展始终控制在适合自己的政治需要范围内,并利用宗教教化番荑,不断扩大自己的势力范围,为明政权创造了良好的国际环境。

在政治上,太祖推重儒释道三教并举的政策。他说:“尝闻天下无二道,圣人无两心。三教之立,虽持身荣俭之不同,其所济给之理一。”他极为重视佛教的辅政作用,将佛教事务视为朝中大事,对佛教制度、僧寺清规多方整饬,期望以此整顿僧团,去淤除垢,“振扬佛法以善世”。

洪武六年(1373年),太祖下诏对出家的僧尼免费发放度牒,才使得唐朝年间流传下来使的“度牒银”制度全部废除。

整顿僧团秩序,防止僧俗混淆,洪武二十四年,朱元璋还制定颁布了影响深广的《申明佛教榜册》,要求各地僧司查验清理天下僧寺,欲还俗者听其还俗,使出家僧人恪受戒律清规,禅、讲、瑜伽,各归本宗。

太祖亲自制定的“御制至圣百字赞”以及明皇室关于修建清真寺和保护清真寺宗教职业人员的谕旨,在一定程度上肯定了回族的宗教生活。

殉葬制度,在西漢初以後,逐漸在中原政權消失。朱元璋二子秦王對人民暴行(見御製紀非錄)被宮人殺死,即連坐迫秦王諸妃自殺。明朝時期明孝陵以四十六妃陪葬,其中有太祖死時殺死殉葬十几名侍寢宮人,這一制度沿襲至成祖、仁宗、宣宗、代宗。而“節烈從殉”的風氣,並向下廣為延伸至宗室公侯、官宦之家、以至民間,直至近百年之後其五世孫英宗死前指出殉葬非古禮,仁者所不忍,才禁殉葬于遺詔,永著為典。按朱元璋創立的制度,嬪妃殉葬由皇帝親臨作別。正統初,明英宗目睹皇父嬪妃殉葬,受很大刺激。天順年間下詔廢止。殺死從殉婦女的方法為將她們縊死,或勒死,或灌以水银毒死。这些生殉的妇女被称为“朝天女”,她們的家屬稱為“朝天女戶”,並給予一定待遇。關於朝天女記載主要依賴朝鮮的第一手資料《李朝實錄金黑口述》。寶慶公主生母張玄妙,以其女幼,得免殉葬。

《明清史事沉思录》中记载,“传谓男子宫刑,妇人幽闭,皆不知幽闭之义。今得之,乃是于牝(阴户)去其筋,如制马、豕之类,使欲火消减。国初常用此,而女往往多死,故不可行也。”对这种灭绝人性的手术,这本书的作者王春瑜评论道:“将人等同畜生处置,始作俑者其无后乎!”

明孝陵康熙題碑:“治隆唐宋”。

清朝官修正史《明史》张廷玉等对明太祖朱元璋最终能够成就帝业的评价是:“帝天授智勇,统一方夏,纬武经文,为汉、唐、宋诸君所未及。当其肇造之初,能沉几观变,次第经略,绰有成算。尝与诸臣论取天下之略,曰:‘朕遭时丧乱,初起乡土,本图自全。及渡江以来,观群雄所为,徒为生民之患,而张士诚、陈友谅尤为巨蠹。士诚恃富,友谅恃强,朕独无所恃。惟不嗜杀人,布信义,行节俭,与卿等同心共济。初与二寇相持,士诚尤逼近。或谓宜先击之。朕以友谅志骄,士诚器小,志骄则好生事,器小则无远圖,故先攻友谅。鄱阳之役,士诚卒不能出姑苏一步以为之援。向使先攻士诚,浙西负固坚守,友谅必空国而来,吾腹背受敌矣。二寇既除,北定中原,所以先山东、次河洛,止潼关之兵不遽取秦、陇者,盖扩廓帖木儿、李思齐、张思道皆百战之余,未肯遽下,急之则并力一隅,猝未易定,故出其不意,反旆而北。燕都既举,然后西征。张、李望绝势穷,不战而克,然扩廓犹力抗不屈。向令未下燕都,骤与角力,胜负未可知也。’帝之雄才大略,料敌制胜,率类此。故能戡定祸乱,以有天下。语云‘天道后起者胜’,岂偶然哉。” 清朝官修正史《明史》张廷玉等对明太祖朱元璋一生事业的评价是:“赞曰:太祖以聪明神武之资,抱济世安民之志,乘时应运,豪杰景从,戡乱摧强,十五载而成帝业。崛起布衣,奄奠海宇,西汉以后所未有也。惩元政废弛,治尚严峻。而能礼致耆儒,考礼定乐,昭揭经义,尊崇正学,加恩胜国,澄清吏治,修人纪,崇凤都,正后宫名义,内治肃清,禁宦竖不得干政,五府六部官职相维,置卫屯田,兵食俱足。武定祸乱,文致太平,太祖实身兼之。至于雅尚志节,听蔡子英北归。晚岁忧民益切,尝以一岁开支河暨塘堰数万以利农桑、备旱潦。用此子孙承业二百余年,士重名义,闾阎充实。至今苗裔蒙泽,尚如东楼、白马,世承先祀,有以哉。”

毛泽东在1964年3月24日,在一次听取汇报时的插话中对明太祖朱元璋、汉高祖刘邦、元太祖成吉思汗的治国能力评价如下:“可不要看不起老粗。”“知识分子是比较最没有知识的,历史上当皇帝的,有许多是知识分子,是没有出息的:隋炀帝,就是一个会做文章、诗词的人;陈后主、李后主,都是能诗善赋的人;宋徽宗,既能写诗又能绘画。一些老粗能办大事:成吉思汗,是不识字的老粗;刘邦,也不认识几个字,是老粗;朱元璋也不识字,是个放牛的。” 毛泽东对明太祖朱元璋的军事才能评价如下:“自古能君无出李世民之右者,其次则朱元璋耳。” 給吳晗提意見:“朱元璋是农民起义领袖,是应该肯定的,应该写的(得)好点,不要写的(得)那么坏。”

趙翼曾説:“藉諸功臣以取天下,及天下既定,即盡取天下之人而殺之,其殘忍實千古所未有。”“蓋明祖之性,實帝王,豪傑,盗賊兼而且也。”

商传评价朱元璋:「朱元璋出身于一个贫苦家庭,从社会最底层的放牛娃、四处讨饭的小和尚,全靠自己的奋斗成了一个统一王朝的开国皇帝。这是中国历史上,乃至世界历史上绝无仅有的事情。另外,朱元璋当上皇帝后,也没有停止步伐,他在位三十多年,成功地建立一个强大统一的明帝国」。

\subsection{洪武}

\begin{longtable}{|>{\centering\scriptsize}m{2em}|>{\centering\scriptsize}m{1.3em}|>{\centering}m{8.8em}|}
  % \caption{秦王政}\
  \toprule
  \SimHei \normalsize 年数 & \SimHei \scriptsize 公元 & \SimHei 大事件 \tabularnewline
  % \midrule
  \endfirsthead
  \toprule
  \SimHei \normalsize 年数 & \SimHei \scriptsize 公元 & \SimHei 大事件 \tabularnewline
  \midrule
  \endhead
  \midrule
  元年 & 1368 & \tabularnewline\hline
  二年 & 1369 & \tabularnewline\hline
  三年 & 1370 & \tabularnewline\hline
  四年 & 1371 & \tabularnewline\hline
  五年 & 1372 & \tabularnewline\hline
  六年 & 1373 & \tabularnewline\hline
  七年 & 1374 & \tabularnewline\hline
  八年 & 1375 & \tabularnewline\hline
  九年 & 1376 & \tabularnewline\hline
  十年 & 1377 & \tabularnewline\hline
  十一年 & 1378 & \tabularnewline\hline
  十二年 & 1379 & \tabularnewline\hline
  十三年 & 1380 & \tabularnewline\hline
  十四年 & 1381 & \tabularnewline\hline
  十五年 & 1382 & \tabularnewline\hline
  十六年 & 1383 & \tabularnewline\hline
  十七年 & 1384 & \tabularnewline\hline
  十八年 & 1385 & \tabularnewline\hline
  十九年 & 1386 & \tabularnewline\hline
  二十年 & 1387 & \tabularnewline\hline
  二一年 & 1388 & \tabularnewline\hline
  二二年 & 1389 & \tabularnewline\hline
  二三年 & 1390 & \tabularnewline\hline
  二四年 & 1391 & \tabularnewline\hline
  二五年 & 1392 & \tabularnewline\hline
  二六年 & 1393 & \tabularnewline\hline
  二七年 & 1394 & \tabularnewline\hline
  二八年 & 1395 & \tabularnewline\hline
  二九年 & 1396 & \tabularnewline\hline
  三十年 & 1397 & \tabularnewline\hline
  三一年 & 1398 & \tabularnewline
  \bottomrule
\end{longtable}


%%% Local Variables:
%%% mode: latex
%%% TeX-engine: xetex
%%% TeX-master: "../Main"
%%% End:
