%% -*- coding: utf-8 -*-
%% Time-stamp: <Chen Wang: 2021-11-01 17:11:57>

\section{代宗朱祁鈺\tiny(1449-1457)}

\subsection{生平}

明代宗朱祁鈺(1428年9月21日-1457年3月14日),或稱景泰帝,年號景泰,明憲宗追諡其為「恭仁康定景皇帝」,弘光帝上庙号「代宗」,谥号「符天建道恭仁康定隆文布武显德崇孝景皇帝」,明朝第7位皇帝(1449年9月22日—1457年2月24日在位)。明宣宗皇次子,母親是賢妃吳氏。

生于宣德三年(1428年),他是明宣宗次子,母吴贤妃。据《明史》称吴贤妃为明宣宗为皇太孙时的侍女。

兄长明英宗即位後封他為郕王。1449年,明英宗在“土木堡之变”被瓦剌太師也先所俘后,郕王被于謙等大臣拥立,是为代宗,年号景泰,尊英宗為太上皇。

代宗即位后,用于謙为兵部尚书,北京保衛戰粉碎了瓦剌的进攻。景泰元年(1450年)八月,鴻臚寺卿楊善出使瓦剌,靠著三寸巧舌說服了也先,英宗返回北京,代宗並沒有迎回兄長的意思,又害怕他复辟,故将其软禁於宮中,以錦衣衛嚴密控管,宮門上鎖並且灌鉛,食物僅能由小洞遞入。

景泰三年,代宗廢去英宗長子朱見深的太子之位,改立自己兒子朱见济為太子,但朱見濟在次年去世。

景泰八年(1457年)正月,代宗病危,十六日曹吉祥、石亨、徐有貞等人謀復立英宗,十七日清晨,發動奪門之變,率領武士攻入紫禁城奉天殿,英宗復辟。代宗被软禁在西苑,一个多月後去世,得年三十岁。代宗死因不明,陸釴的《病逸漫記》說代宗是被英宗謀殺的,查繼佐的《罪惟錄》則表示代宗病愈,英宗為怕代宗復起,令太監蔣安用帛扼死景泰帝。代宗死后,葬于西郊金山(玉泉山北)的景泰陵。英宗令廷臣议王妃之殉葬。议及汪皇后,被李賢及太子谏止。后以皇贵妃唐氏殉葬。

英宗恨代宗薄待,谥为戾王,称郕戾王。明宪宗成化时期上谥号「恭仁康定景皇帝」。明崇禎十七年(1644年)七月乙丑,弘光帝上庙号代宗,谥号「符天建道恭仁康定隆文布武显德崇孝景皇帝」。清朝复称其谥号为「恭仁康定景皇帝」。明清史书多称明代宗为景帝。

明代宗是未安葬在明十三陵的皇帝(另外明太祖朱元璋葬于南京明孝陵,明惠帝因最後失踪故無陵墓)。

\subsection{景泰}

\begin{longtable}{|>{\centering\scriptsize}m{2em}|>{\centering\scriptsize}m{1.3em}|>{\centering}m{8.8em}|}
  % \caption{秦王政}\
  \toprule
  \SimHei \normalsize 年数 & \SimHei \scriptsize 公元 & \SimHei 大事件 \tabularnewline
  % \midrule
  \endfirsthead
  \toprule
  \SimHei \normalsize 年数 & \SimHei \scriptsize 公元 & \SimHei 大事件 \tabularnewline
  \midrule
  \endhead
  \midrule
  元年 & 1450 & \tabularnewline\hline
  二年 & 1451 & \tabularnewline\hline
  三年 & 1452 & \tabularnewline\hline
  四年 & 1453 & \tabularnewline\hline
  五年 & 1454 & \tabularnewline\hline
  六年 & 1455 & \tabularnewline\hline
  七年 & 1456 & \tabularnewline\hline
  八年 & 1457 & \tabularnewline
  \bottomrule
\end{longtable}


%%% Local Variables:
%%% mode: latex
%%% TeX-engine: xetex
%%% TeX-master: "../Main"
%%% End:
