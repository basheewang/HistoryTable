%% -*- coding: utf-8 -*-
%% Time-stamp: <Chen Wang: 2019-12-26 15:07:52>

\section{熹宗\tiny(1620-1627)}

\subsection{生平}

明熹宗朱由校(1605年12月23日-1627年9月30日;校,居效切,拼音「jiào」、注音「ㄐㄧㄠˋ」),或稱天啟帝,光宗長子,明朝第16代皇帝。在位時間為1620年-1627年,年號天啟。光宗即位僅一個月而亡,使朱由校匆匆登位為帝,朱由校當時僅十四歲,未曾被立为太子,甚至未接受正規教育,政事皆賴宦官輔佐,後來造就太監魏忠賢等人的干政,與閹黨、東林黨之黨爭。

泰昌元年(1620年),其父明光宗在位不足三十天便在紅丸案之中暴斃。九月初六,由長子朱由校繼任。值得一提的是,其父明光宗朱常洛一向不為祖父明神宗所喜,故朱由校亦沒有被神宗重視。神宗駕崩後,大臣代言的遺囑:「皇長孫宜即時冊立、進學。」故顯示當時已十四歲的朱由校從未進學。明光宗即位後原擇九月初九冊立朱由校為東宮,惟來不及冊封,光宗於九月初一駕崩,故明熹宗連一天正式教育都未接受,便登上大寶,此為有明一代第一人,其情況比其父光宗勉強隨其他皇子出閣讀書,而非正統的太子教育方式,還要更加惡劣,且父子倆在繼位前都未監國輔政經驗,制造內宦干政的土壤,神宗亦無留下良好輔臣,國運衰退的因素在萬曆時國本之爭時即已種下。

泰昌元年(1620年),是明朝立國以來所遇到前所未有的情況。明熹宗的祖母孝端顯皇后、祖父明神宗與父親明光宗相繼在同一年駕崩,明神宗駕崩距孝端顯皇后駕崩才兩個多月,而明光宗駕崩時距明神宗駕崩不到一個月,實屬罕見。而明神宗與孝端顯皇后的大葬尚未完成,因此明廷在討論後,決定先為明神宗與孝端顯皇后辦理大葬,結束後再為明光宗辦理大葬。

熹宗繼位後,撫養皇帝的李選侍利用皇帝年少無知,佔據乾清宮,意圖把持朝政,東林黨左光斗、楊漣等反對,不讓李選侍與皇帝同住,迫使她移居他處,是為移宮案,此事後內侍魏忠賢被提拔為司禮監秉筆太監,魏忠賢與熹宗是皇孫時代即結識的舊識,魏忠賢乘機結交朱由校乳母客氏,兩人遂狼狽為奸。熹宗有感東林黨黨人從龍之功,大加提拔任用,又召回葉向高等先朝老臣擔任內閣首輔,時稱「眾正益朝」「群賢滿朝」,天下欣欣望治。另外熹宗也屢發內帑犒勞將士,補發九邊欠餉,如即位之初便發派一百八十萬帑金以勞邊,派帑金五十萬以給光宗陵工,準發帑五十萬作解發以發兵餉,又答允兵部再發帑金一百萬以佐急需,接著不足一年又因首輔葉向高所請而發帑金二百萬為東西兵餉之用。

朱由校喜歡木工,亦沉迷於刀鋸斧鑿,魏忠賢總是趁他木工做得全神貫注時,拿重要的奏章去請他批閱,熹宗隨口說:「朕已悉矣!汝輩好為之。」魏忠賢遂逐漸專權,竊奪威福,魏忠賢閹黨誣陷忠良,殺死包括東林六君子、東林七賢等正直的士大夫,致使朝政敗壞。

同時期,女真首領努爾哈赤則起於白山黑水之間,趁機攻佔瀋陽,奪取遼東地區,聲勢日隆。

天啟六年(1626年)北京發生「王恭廠大爆炸」,死傷2萬餘人,原因不明,朝野震驚,中外駭然,熹宗下了一道罪己詔,表示要痛加省醒,告誡大小臣工「務要竭慮洗心辦事,痛加反省」,并下旨發府庫萬兩黃金賑災。

天啟七年(1627年)八月,熹宗又與宦官魏忠賢、王體乾等去西苑深水處泛舟,卻因風強,小舟翻覆,皇帝落水,雖然隨即被救,但從此驚豫不堪,逐漸病重,尚書霍維華獻「靈露飲」,以五穀蒸餾而成,清甜可口,但幾個月後病情加劇,渾身浮腫,八月十一日,召見信王朱由檢,即行駕崩,時年23歲,廟號熹宗。熹宗諸子皆早夭,遺詔立五弟朱由檢為皇帝,即後來的明思宗。禮部定謚號曰「哲皇帝」,思宗宸墨改為「悊」。

《明實錄》:「上念光皇大業未究,雅志繼述,踐祚之初委任老成,摉羅遺逸,振鷺充庭,稱盛理焉。時四方多故,上宵旴靡遑,遼左及滇黔相繼請帑,無不立應,大臣行邊恩禮優渥,將士陷陣恤典立頒,又慮加派苦累,每有詔諭諄諄戒守令,加意撫字毋重困吾民。其軫念民碞如此,故能收拾人心,挽回天步,雖有煬灶假叢之奸而得人付托,社稷永固於苞桑。廟號曰熹,蓋稱有功安人云。」

《明史》:「自世宗而後,綱紀日以陵夷,神宗末年,廢壞極矣。雖有剛明英武之君,已難復振。而重以帝之庸懦,婦寺竊柄,濫賞淫刑,忠良慘禍,億兆離心,雖欲不亡,何可得哉。」

明朝劉若愚《酌中志》对熹宗评价较高,“先帝(明熹宗)生性虽不好静坐读书,然能留心大体,每一言一字,迥出臣子意表”;熹宗在宁锦大战中“日夜焦思,未遑自安”,王永光的题疏中曾有“要将宁远城中红夷大炮撤归山海关”,明熹宗批示:“此炮如撤,人心必摇”,表明他是有一定的政治决断力。当后金军队再犯锦州、宁远之时,“更愤激深虑”,对魏忠贤和乳母客氏也怒骂咒恨,形于颜色。同时,熹宗又「又極好作水戲,用大木桶大銅缸之類,鑿孔創機,啟閉灌輸,或湧瀉如噴珠,或澌流如瀑布,或使伏機於下,借水力沖擁園木球如核桃大者,於水湧之,大小盤旋宛轉 ,隨高隨下,久而不墮,視為嬉笑,皆出人意表。」。他曾親自在庭院中造了一座小宮殿,形式仿乾清宮,高不過三四尺,卻曲折微妙,巧奪天工。可见刘若愚对明熹宗的评价颇高。

明末清初談遷認為「閹尹之禍,劇於熹廟,并边徽而二之。……疵德多矣」。將閹黨及滿清視為天啟年間兩大威脅,可見其嚴重性。

清道光年間抱陽生《甲申朝事小紀》,認為朱由校沈迷於木工,放任魏忠賢矯詔、管理朝政的行為視為貪玩而不長進,史載「又好油漆,凡手用器具,皆自為之。性又急躁,有所為,朝起夕即期成。成而喜,不久而棄;棄而又成,不厭倦也。且不愛成器,不惜改毀,唯快一時之意。」「朝夕營造」,「每營造得意,即膳飲可忘,寒暑罔覺」。

民國直系將領吳佩孚認為明熹宗寵信閹黨,濫殺東林六君子、東林七賢,才是明朝亡國的主因,更甚於萬曆。其恩師王紹勛,與吳佩孚提及明神宗怠政三秩時,感歎曰:「無為而治兮不必生一神宗三秩」,吳佩孚居然立刻應聲對仗:「有明之亡矣莫非殺六君子七賢。」

《從萬曆到永曆》一書認為,魏忠賢不可能屢屢矯詔,故而天啟一朝的政治,包括鎮壓東林的決策,還是與熹宗相關,熹宗遭到了後來主編明史的東林和復社人士抹黑。此外,即便是明史也明確記載了熹宗對於朝政的參與,不可謂無自相矛盾之處。例如王士禛所謂老宮監刘若愚的原話是:(先帝)且不爱成器,不惜天物,任暴殄改毁,惟快圣意片时之适。当其斤斫刀削,解服磐礴,非素昵近者不得窥视,或有紧切本章,体乾等奏文书,一边经管鄙事,一边倾耳注听。奏请毕,玉音即曰:「尔们用心行去,我知道了」。這和所謂勤政的清朝皇帝批示:「知道了。」是差不多的作為。

此外,朱由校所謂沈迷於木工,很有可能是因為對於宮殿藝術有所追求,由於前兩次主要工程人員如蒯祥皆過世,為了三大殿能復原,朱由校特別注重木作部分等,事出有因,並非只因為個人興趣而不理會朝政。當時因南京三大殿早已燒失,北京紫禁城三大殿於萬曆年間亦燒燬,朱由校效法太祖親自監督三大殿重建計畫,聽從御史王大年节俭的建议,才會鑽研木匠手藝。

\subsection{天启}

\begin{longtable}{|>{\centering\scriptsize}m{2em}|>{\centering\scriptsize}m{1.3em}|>{\centering}m{8.8em}|}
  % \caption{秦王政}\
  \toprule
  \SimHei \normalsize 年数 & \SimHei \scriptsize 公元 & \SimHei 大事件 \tabularnewline
  % \midrule
  \endfirsthead
  \toprule
  \SimHei \normalsize 年数 & \SimHei \scriptsize 公元 & \SimHei 大事件 \tabularnewline
  \midrule
  \endhead
  \midrule
  元年 & 1621 & \tabularnewline\hline
  二年 & 1622 & \tabularnewline\hline
  三年 & 1623 & \tabularnewline\hline
  四年 & 1624 & \tabularnewline\hline
  五年 & 1625 & \tabularnewline\hline
  六年 & 1626 & \tabularnewline\hline
  七年 & 1627 & \tabularnewline
  \bottomrule
\end{longtable}


%%% Local Variables:
%%% mode: latex
%%% TeX-engine: xetex
%%% TeX-master: "../Main"
%%% End:
