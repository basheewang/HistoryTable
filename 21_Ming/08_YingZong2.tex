%% -*- coding: utf-8 -*-
%% Time-stamp: <Chen Wang: 2019-10-21 17:06:32>

\section{英宗复辟\tiny(1457-1464)}

奪門之變,又稱南宮復辟,是明代宗朱祁鈺景泰八年(1457年)正月,发生的一場政變,太上皇朱祁鎮成功復辟,奪回皇位。

正統十四年 (1449年) 發生土木堡之變,明英宗被瓦剌俘虜,其弟郕王朱祁鈺被眾大臣推舉為皇帝,是為明景帝(南明尊稱為代宗),改元景泰。孫太后亦要求景帝即位後立英宗兩歲兒子朱見深為太子,表示大明帝位仍由英宗一脈繼承。

景泰元年(1450年),兵部侍郎于谦成功抗敵,並與瓦剌議和,經過使臣楊善個人的斡旋,瓦剌首領也先見新君已立,英宗已經無利用價值,反而不想因英宗為虜之事成為與大明修好的障礙,於是同意放回英宗。但朱祁鈺對大臣說:「我並不是貪戀帝位,當初擁立我的是你們啊。」不願英宗返國,經大臣陳述其利弊後,朱祁鈺将英宗迎接回京,置於南宮,尊為太上皇。並以錦衣衛對英宗加以軟禁,嚴密控管,宮門不但上鎖,並且灌鉛,食物僅能由小洞遞入。其後景帝在景泰三年 (1452年)廢原太子朱見深,並立自己的獨子朱見濟為新太子。景泰五年 (1454年),朱见济夭折后,朱祁钰已无亲子,却也没有复立朱见深,储位空悬。

景泰七年(1456年),朱祁鈺病重,在對抗瓦剌時立下大功的將領石亨為了自身利益,有意協助英宗奪回帝位。在拉攏身邊人商討後,與宦官曹吉祥、都督張軏、都察院左都御史楊善、太常卿許彬以及左副都御史徐有貞等人行事。

景泰八年(1457年)正月,朱祁鈺病重。十六日夜,石亨、徐有贞等大臣带一千餘士兵偷襲紫禁城,撞开南宮宫门,接出英宗直奔东华门。守门的武士不开门,英宗上前说道:“朕乃太上皇帝也。”武士只好打开城门。

黎明时分,众大臣到了「奉天殿」,只见英宗坐于龙椅之上,徐有贞高喊:“太上皇帝復位。”史称「奪門之變」或「南宮復辟」。

英宗復辟後,朱祁鈺被遷至西宮,不久去世。

談遷評論:“于少保最留心兵事,爪牙四布,若奪門之謀,懵然不少聞,何貴本兵哉!或聞之倉卒,不及發耳!”

明英宗復辟後,于謙以謀逆罪名被處死,而曾助英宗回復帝位的功臣,如石亨、徐元玉、許彬、楊善、張軏與曹吉祥等人都被封為大官。其中,曹吉祥等在朝中橫行霸道,後期更發生了曹吉祥企圖弒位的曹石之變。

值得一提的是,景泰八年春正月,明英宗重登大寶後,废景泰年号,改景泰八年为天顺元年,但倉促之中忘記罷黜朱祁鈺,直到同年二月乙未才將朱祁鈺廢為郕王。因此,在這幾天之內,名義上英宗和景帝兩位合法的皇帝同時並存,成為中國帝制史上絕無僅有的奇觀。

曹石之变前,英宗在李贤提醒下,意识到朱祁钰时日无多,没有在世的儿子,也没有立储,一旦朱祁钰去世,自己复位顺理成章,夺门功臣其实是投机以求自己获益,一旦事败,英宗自己反而要受到牵连;于是开始罢黜夺门功臣的爵位。楊善、張軏已去世,爵位已分别由儿子杨宗、张瑾继承。明宪宗初年,罢黜杨宗、张瑾,因夺门之功所授爵位至此全部收回。

\subsection{天顺}

\begin{longtable}{|>{\centering\scriptsize}m{2em}|>{\centering\scriptsize}m{1.3em}|>{\centering}m{8.8em}|}
  % \caption{秦王政}\
  \toprule
  \SimHei \normalsize 年数 & \SimHei \scriptsize 公元 & \SimHei 大事件 \tabularnewline
  % \midrule
  \endfirsthead
  \toprule
  \SimHei \normalsize 年数 & \SimHei \scriptsize 公元 & \SimHei 大事件 \tabularnewline
  \midrule
  \endhead
  \midrule
  元年 & 1457 & \tabularnewline\hline
  二年 & 1458 & \tabularnewline\hline
  三年 & 1459 & \tabularnewline\hline
  四年 & 1460 & \tabularnewline\hline
  五年 & 1461 & \tabularnewline\hline
  六年 & 1462 & \tabularnewline\hline
  七年 & 1463 & \tabularnewline\hline
  八年 & 1464 & \tabularnewline
  \bottomrule
\end{longtable}


%%% Local Variables:
%%% mode: latex
%%% TeX-engine: xetex
%%% TeX-master: "../Main"
%%% End:
