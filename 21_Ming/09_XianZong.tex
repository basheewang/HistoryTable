%% -*- coding: utf-8 -*-
%% Time-stamp: <Chen Wang: 2021-11-01 17:12:40>

\section{宪宗朱見深\tiny(1464-1487)}

\subsection{生平}

明憲宗,或稱成化帝,原名朱見深,後改名朱見濡(1447年12月9日-1487年9月9日),為明英宗皇長子,明朝第9代皇帝。明憲宗在位二十三年,期間恢復其叔朱祁鈺的帝號,又為于謙等忠臣平反,初年勵精圖治,體恤民情,任用李賢、商輅、彭時等賢臣,頗為時人所傳誦;在軍事方面,整飭戎政,對內平定荊襄群盜和西南傜蠻,對外抵禦抵禦韃靼女真、經略哈密,擁有不少功績。但憲宗寵嬖萬氏、中晚年信用汪直、梁芳、萬安等宦官奸臣,又以“皇莊”大肆侵占土地,使明朝政治日壞;而頻繁的內外用兵亦使明朝國力大損。成化朝是明朝自仁宣以來文治武功較卓越的時期,但是與此並存的弊政不得不說有所缺憾。谥号「繼天凝道誠明仁敬崇文肅武宏德聖孝纯皇帝。」

正統十四年(1449年)土木堡之變,英宗被瓦剌擄去,兵部侍郎于謙等立皇弟朱祁钰即位,是為景帝,改元景泰,同時立見深為太子。到景泰三年(1452年),朱祁鈺將見深廢為沂王,改立自己的儿子朱见济为太子。

五年后(1457年),英宗因奪門之變而復辟,見深重被立為太子。萬曆野獲編中記載憲宗皇帝玉音微吃,而臨朝宣旨,則瑯瑯如貫珠,其本人可能或多或少有口吃的情況。

原名朱見濬(《明史》誤載憲宗即位前名為朱見浚,即位後為見深),因英宗復辟後重立太子,將憲宗之名誤寫為見濡,憲宗於天顺八年(1464年)登基後遂改稱見濡。憲宗宽仁英明,即位之初就為于謙平冤昭雪,當時曾有大臣追論景泰廢立事往,憲宗切責說:「景泰事已往,朕不介意,且非臣下所當言。」另䆁放了浣衣局婦女和願歸宮人,又恢復明景帝帝號。文治上憲宗體諒民情,蠲賦省刑,任用賢臣,考察官吏,勵精圖治,善政史不絕書,儼然為一代明君,當其時朝廷多名贤俊彦,百姓得以休养生息,史稱成化新風,堪稱與仁宣之治媲美,朝鲜、琉球、哈密、烏斯藏、暹羅、吐魯番、撒馬兒罕、日本、蘇門答剌等國紛紛入貢。人口方面在成化十五年(1479年)中成為終明一代的人口峰值,達9,496,265戶,71,850,132人,反映當時明朝仍然處於盛世。

武功上憲宗恢復十二團營制度,幾次親閱騎射於西苑,巡查禁軍,整飭軍備,考試士兵訓練,還任用王越、余子俊、秦紘、朱永、朱英等能臣處理軍務,修建邊牆,并從不斷南下入侵盤踞河套的韃靼部手裡,一舉收復河套地區,使得套寇問題基本解決。在紅鹽池大捷中,明軍大破韃靼大營,擒斬三百五十人,獲駝馬器械不可勝計,史书記載「虏自是不敢复居套内者二十年,则此捷为所震慑故也。」「自是不复居河套,边患少弭;间盗边,弗敢大入,亦数遣使朝贡。」甚至在後來威宁海大捷中夜行晝伏直捣蒙古可汗王庭,生擒幼男婦女一百七十,斩首四百三十七级,獲旗纛十二面,馬駝牛羊六千餘,盔甲弓箭皮襖之類又萬餘,达延汗巴图蒙克仅以身逃。另外自從明英宗以來,盤踞在建州的李满住、董山屢寇掠辽东,逐漸成為邊患,明憲宗在多次招撫不果後決定用兵撻伐,先後於成化三年與成化十五年,明軍與朝鮮聯手進攻屢次犯邊的建州女真,生擒數百人,斩首千餘級,破四百五十餘寨,夺回被掳人口數千人,擒斬罪魁禍首的董山,史稱成化犁庭或丁亥之役。

明朝皇帝多擅畫像,作字運筆,憲宗亦擅畫神像,曾為張三豐畫像,神采生動,超然塵表,又曾親筆御製一團和氣和歲朝佳兆等畫流世,畫法老練嫻熟,頓挫自如。成化十八年,憲宗又親自編寫了《文華大訓》一書,以教導太子人倫治國之道,垂訓子孫。而《貞觀政要》自唐流傳至明,版本注釋繁亂,明憲宗即位後,立即組織儒臣對其進行校定,把宋元史纂輯的綱目皆寫入書中,頒示天下,即流傳至今的成化本,又為重修的孔子廟碑和《貞觀政要》作親自序。憲宗在《貞觀政要序》中寫道「朕萬幾之暇,悅情經史,偶及是編...太宗在唐為一代英明之君,其濟世康民,偉有成烈,卓乎不可及己,所可惜者,正心修身二帝三王之道,而治未純也。朕將遠師往聖,允迪大酋,以宏其治。」足見他的治國抱負和文化素質。

憲宗在位中后期,好方術,沉溺後宮,极度宠信大他19歲的万贵妃,又生活奢靡,取國庫填內帑并擴置皇莊,同时又任用太监汪直、梁芳等奸佞當權,以致西廠橫恣,朝紳諂附,且明憲宗直接頒詔封官,是為傳奉官,這使得傳奉官氾濫,舞弊成風,朝政荒芜。但整體而言,成化晚年,朝廷依然能有條不紊地對天災人禍有迅速的應對,因此仍幸稱歌舞升平,太平無事。

成化初年,土地兼併嚴重,造成大量流民依山據險,光是荊州、襄州、安州、沔州之間,“流民不下百萬”。湖廣荊襄地區成為流民的聚居區,賊盜嘯聚。成化元年(1465)三月劉通、石龍、馮子龍等於房縣大石廠立黃旗起義,擁眾數十萬。成化六年十一月,又有劉通舊部李原、小王洪起義,流民附和者達百萬人。史稱鄖陽民變。

成化二十三年(1487年)春,萬貴妃去世,憲宗過於悲痛而患病,長歎說:「萬氏長去了,我亦將去矣。」日漸消瘦,最終於同年八月廿二日駕崩,享年39歲(虚龄四十一)。葬於北京昌平茂陵。臨終前誨示太子要敬天法祖,勤政愛民,太子頓首受命,他的三子朱祐樘繼位,即后来的明孝宗。

明宪宗即位後任用李贤、彭时、商辂等人,下诏為于谦平反,派人去為于谦扫墓,并让其子于冕袭为千户,于谦的女婿朱翼等人,也被归还家产。

荆襄刘通造反,命抚宁伯朱永讨伐,将之平定。又有陕西周原土官满四占据石城,荆襄復反,憲宗力排众议,命项忠平定,荆襄贼平,明军击斩万人,首领刘通、苗龙等四十人被生擒献俘京师。宪宗又专门派出了杨璇抚治荆、襄、南阳流民,史載「大会湖广、河南、陕西抚、按、藩、臬之臣,籍流民得十一万三千余户,遣归故土者一万六千余户,其愿留者九万六千余户,许各自占旷土,官为计丁力限给之,令开垦为永业,以供赋役,置郡县统之。 」。此後流人得所,四境乂安,直至明未,荆襄再也沒有出現大亂了。

蠲賦省刑是成化一朝最為後人津津樂道的善政之一,史記憲宗「一聞四方水旱,蹙然不樂,亟下所司賑濟,或輦內帑以給之;重惜人命,斷死刑必累日乃下,稍有矜疑,輒從寬宥。」「憲宗好生,每奏讞大辟(死刑奏章),多所寬宥,或不得已而行刑。其日必卻八珍之奉,默坐焚香。哀矜之意,惻然見於玉色。」自他即位自駕崩唯止,僅在官田減免稅糧一項則已達一千九百多萬石,在民田稅額的蠲免和下內帑賑濟更是不計其數,僅以成化二十一年為例,實錄記載當年減免天下官田等項稅糧一百零八萬五千九百石,然而憲宗除此之外在該年正月從內庫中撥帑二十萬五兩賑濟災民,四月又撥漕糧四十萬賑災,同月與十月又免山東濟南、山西平陽、四川成都、河南開封、南直隸鳳陽等州府稅糧,總計連同官田稅賦該年蠲免三百萬石,相當全國稅額十之二一,可見憲宗不吝恤民。因此儘管成化一朝水旱災變不斷,在荆襄流民問題處理完後,再也沒有出現較大的社會波場動。

橫觀成化年間的最值得稱道的善政,除了處理荊襄流民與蠲賦省刑外,其次莫過於改革漕運,自明成祖永樂遷都以來,北京便依賴南糧北運,其中需要每年徵集大量民伕運糧,路途波折,時常耽誤農時,自成化七年後,朝廷減省少了民伕的運輸路程,改由官兵漕軍長運,雖然朝廷的加耗增加了,但節約了百姓的農時,有利農業生產,同時又制定了各類考課規條,自此以後明代的漕運才有了完備的制度,此制一直沿用至明末。

手工業者在成化年間身份有了進一步的自由,明太祖建國時,分天下百姓為軍民匠灶四類,手工業者便被歸類在匠户中,他們各分「住坐」和「輪班」,他們必須義務定期(通常五年一班,每班服役三個月)為朝廷工作,有時還要無償服役,於是逃役者越來越多。成化二十一年起,朝廷允許輪班匠不願服役者可以每月出錢免役,改由朝廷直接雇工造作,這不但令朝廷毋須再終年追捕工匠,勞官擾民,手工業者只要付出二三月的銀子,便可以免除三月的工役之苦和回來花費的時間,也換來四年的人身自由。

在位初期,天下称颂其统治;但宠信万贵妃后,朝政转向晦暗,万安开始得势。又设置西厂,命太监汪直提督外事,于是汪直便随意罗织罪名生事。汪直仗势将陈钺,威宁伯王越变为自己的羽翼,依附自己之人便任用,不听自己话的人就排挤打击,权势极为显赫,天下都惧之三分。汪直又想在外立功,胡乱进行边界挑衅。宪宗命汪直掌管十二团营。当时有个名叫阿丑的中官,善演诙谐幽默戏,经常在宪宗面前表演,颇有汉朝东方朔用滑稽方法进谏之风。一天阿丑假装喝醉酒,旁边一个人在佯装说:“某官到!”阿丑任装醉意大骂,人又说:“皇驾到!”阿丑还是醉骂如故,那人又说:“汪太监来了。”阿丑所装的醉人赶紧起来惊恐的站在一边。旁边的人问到:“天子驾到都不害怕,为什么害怕汪太监?”阿丑说:“我只知有汪太监,不知有天子。”自此以后汪直逐步失宠。此时王越和陈钺讨好汪直,三人结为死党。阿丑一日有在做戏,自己扮演汪直手持双斧向前前行,有人问其缘故,答说:“这双斧是王越和陈钺。”宪宗听后微笑了一下。御史徐鏞等人弹劾汪直欺君枉法,擅开边衅,宪宗后渐疏远汪直。

被宪宗先后任用的宰輔有:李賢,陳文,彭時,呂原,商輅,劉定之,萬安,劉珝,劉吉,彭華,尹直。对成化一朝,世有“紙糊三閣老,泥塑六尚書”之謠,三閣老指萬安、劉吉和劉珝,六尚書指尹禕、殷謙、周洪謨、張鵬、張鎣和劉昭,意讽这些朝廷重臣不作为,私德不佳,但也有意見認為他們之所以被抨擊,并非庸懦無能,貪贓枉法,而是因為對明憲宗專寵萬貴妃,內批傳奉官的行為沒有進行有力勸諫,使明憲宗符合傳統儒家人君規範,其實從成化後期對災區和地方事務的應對裁決,可見他們還是各有所長、恪盡職守的,因而即使同萬安這世稱的奸倖之臣,卻也見容於當其時彭時商輅等名臣官員中。

明憲宗本人曾經向兒子朱祐樘概括自己的一生作为:「修文史而究武略,饬内治以攘外侮,戡靖僭窃,应宁邦家,犹宵旰靡遑,惧功业未茂,德惠未周,而治平之效未臻也。」

《明實錄》:「葢上以守成之君,值重熙之運,兵革不試,萬民樂業,垂拱而天下大治矣。」

《名山藏》何乔远:上聪明仁恕,渊默勤恭,孝事母后如古帝王。郊庙斋祭,必极诚敬。景皇帝尝有封沂之命,未尝一语及之。委任大臣,略无猜忌,或即干纪,屏斥无疑。一闻四方水旱,戚戚然下所司赈济,或辇内帑给之。重惜人命,断死刑累日乃下。夙兴视朝,但遇雨雪辄放常参官而不废奏引。隆寒盛暑,或减奏事,以恤卫士侍立之劳。间有游豫,不出大内,如南囿祖宗时不废游猎,上未尝一幸焉。时御翰墨,作为诗赋,以赐大臣。诸司章奏,手自披阅,字画差错,亦蒙清问。臣下益兢业职事,莫敢或欺。葢上以守成之君,值重熙之运,兵革不试,万民乐业,垂拱而天下大治矣。

《国榷》谈迁:恤饥察冤,求言课吏,先后史不绝书,而于胡僧幸阉斜封墨敕之滥,亦不能为帝掩也。当其时,朝多耆德,士敦践履,上恬下熙,风淳政简,称明治者,首推成弘焉。而或有遗议,则在汪直、李孜省、繼曉辈蚀其一二,于全照无大损也。尺璧之瑕,乌足玷帝德哉!末谕太子以敬天法祖、勤政爱民之道,俨然成周之遗训也。说者谓帝初欲易储,以泰山屡震而止。噫!帝能尊钱后,复景帝,俱事出常情之外,而乃轻视东宫?必不然也。

《国榷》郑晓:帝仁恕英明,少更多难,练达情理。临政莅人,不刚不柔,有张有弛。进贤不骤而任之必专,远邪不亟而御之有法。值虏寇数侵边,惟遣将薄伐,不勤兵以竭我财力,虏亦离散,内外宁辑。荆襄岭海,时有寇窃,推毂之际,戒勿妄杀,或不用命,赏罚兼行。崇上理学,褒封儒贤。江淮大祲,截漕赈饥。星文示变,侧身省过。臣僚进谏,即涉浮伪,时有干忤,薄示谴谪,旋蒙牵复。若乃尊礼孝庄,尊景帝,保护汪后,褒恤于谦,其于爱憎恩怨,绝无芥蒂,帝谆然于天理彝伦者也。以故虽屡有彗孛之灾,而国家康靖,有繇然矣。

《国榷》李维桢:詩有之,“靡不有初,鮮克有終”,人情哉!純帝初載,亦何其斤斤也。中官幸,禱祠繁,而治隳矣。錢後之祔廟食,景帝之復位號,此兩者,雖甚盛德蔑以加已。

《明史》贊曰:「憲宗早正儲位,中更多故,而踐阼之后,上景帝尊號,恤于謙之冤,抑黎淳而召商輅,恢恢有人君之度矣。時際休明,朝多耆彥,帝能篤于任人,謹于天戒,蠲賦省刑,閭里日益充足,仁、宣之治于斯复見。顧以任用汪直,西厂橫恣,盜竊威柄,稔惡弄兵。夫明斷如帝而為所蔽惑,久而后覺,婦寺之禍固可畏哉。 」

《朝鮮成宗實錄》:上(成宗)御宣政殿, 引見明澮等, 謂曰: 「中國有何事?」 明澮對曰: 「(憲宗)皇帝勤於聽政, 天下太平, 民物富庶。」(時成化十一年)

《剑桥中国明代史》中写道:「朱见深与他的有军事头脑的祖父和父亲相同,向往他们的生气勃勃的、甚至具有侵略性的军事姿态,并且厚赏有成就的军事将领。」

負面事蹟主要與其大19歲的妃子萬貞兒的感情和鬆散的管理有關。

《罪惟錄》論曰:災異之警,無有酷於此二十三年者也。宮中位一女戎,而群小相緣益進,惑匿導誘,顛例黜陟,以致傳升無己,監督四出,閣輔阿循,廠衛搜射。而帝又旋悟旋迷,嘉言罔入,邊釁苗殘,幾無寧歲。天乃至仁,歷以所警,貫耳而呼,而其如溺柔聽者,袖不聞也。祗幸蠲賑免租,無少稽吝,猶不致啟中原之怒。且內外寡大故,無所藉以起,幸稱小康。嗟乎!哲婦傾城,危矣哉!

《明史講義》:凡此皆成化時朝政之穢濁,而國無大亂,《史》稱其時為太平,惟其不擾民生之故。

《朝鮮成宗實錄》:(司憲府掌令李琚)更啓曰: 「臣於丙午年往中國, 中國人言, 成化皇帝非賢君也, 然一用《大明律》, 故朝廷寧謐, 四方無虞矣。 臣今所啓, 別無他意, 欲殿下遵守舊章而已。」(朝鮮成宗) 傳曰: 「爾陪臣也, 而褒貶天子, 則我諸侯也, 何不褒貶我乎? 爾非新進之儒, 曾經弘文館, 爾不知予心而如此言之耶?」

《明朝時代上卷第38章陳獻章和他的心學》:成化王朝是明王朝歷史上的一個轉折點,正是在這個時期基本結束了朱元璋一百年來禁錮帝國的政策,從此帝國又重新恢復到唐宋元的那種自由、奔放的年代,商業開始復甦、城市開始繁華、思想文化開始活躍、士紳的生活開始奢靡,在這個社會整體鬆動下,起到穩定、凝聚作用的理學思想也開始搖搖欲墜,它必將被更能適應社會發展的思想所代替。

《成化皇帝大傳》:成化朝君臣们是预测不到的,他们留给弘治朝君臣的,乃是一个外无强敌,内无大敌,百业兴旺,万民乐业的太平世道。

\subsection{成化}

\begin{longtable}{|>{\centering\scriptsize}m{2em}|>{\centering\scriptsize}m{1.3em}|>{\centering}m{8.8em}|}
  % \caption{秦王政}\
  \toprule
  \SimHei \normalsize 年数 & \SimHei \scriptsize 公元 & \SimHei 大事件 \tabularnewline
  % \midrule
  \endfirsthead
  \toprule
  \SimHei \normalsize 年数 & \SimHei \scriptsize 公元 & \SimHei 大事件 \tabularnewline
  \midrule
  \endhead
  \midrule
  元年 & 1465 & \tabularnewline\hline
  二年 & 1466 & \tabularnewline\hline
  三年 & 1467 & \tabularnewline\hline
  四年 & 1468 & \tabularnewline\hline
  五年 & 1469 & \tabularnewline\hline
  六年 & 1470 & \tabularnewline\hline
  七年 & 1471 & \tabularnewline\hline
  八年 & 1472 & \tabularnewline\hline
  九年 & 1473 & \tabularnewline\hline
  十年 & 1474 & \tabularnewline\hline
  十一年 & 1475 & \tabularnewline\hline
  十二年 & 1476 & \tabularnewline\hline
  十三年 & 1477 & \tabularnewline\hline
  十四年 & 1478 & \tabularnewline\hline
  十五年 & 1479 & \tabularnewline\hline
  十六年 & 1480 & \tabularnewline\hline
  十七年 & 1481 & \tabularnewline\hline
  十八年 & 1482 & \tabularnewline\hline
  十九年 & 1483 & \tabularnewline\hline
  二十年 & 1484 & \tabularnewline\hline
  二一年 & 1485 & \tabularnewline\hline
  二二年 & 1486 & \tabularnewline\hline
  二三年 & 1487 & \tabularnewline
  \bottomrule
\end{longtable}


%%% Local Variables:
%%% mode: latex
%%% TeX-engine: xetex
%%% TeX-master: "../Main"
%%% End:
