%% -*- coding: utf-8 -*-
%% Time-stamp: <Chen Wang: 2021-11-01 17:13:06>

\section{穆宗朱載坖\tiny(1567-1572)}

\subsection{生平}

明穆宗朱載坖(“坖”音“jì”,1537年3月4日-1572年7月5日),或稱隆慶帝,明朝第13位皇帝,庙号“穆宗”,谥号“契天隆道渊懿宽仁显文光武纯德弘孝莊皇帝”。

朱载坖的名讳在万历年间被武纬子误记为朱載塈(“塈”音“jì/ㄐㄧˋ”),崇祯年间被朱国祯等误记为朱載垕(“垕”音“hòu/ㄏㄡˋ”),导致清代文献、越南文献、朝鲜文献对穆宗名讳记载的混乱。

明穆宗是明世宗第三子,嘉靖十六年(1537)生,母亲是康妃杜氏。嘉靖十八年(1539)二月,明世宗册立次子朱载壡为太子、三子朱载坖为裕王、四子朱载圳为景王。嘉靖二十八年(1549)三月,太子朱载壡薨,裕王朱载坖以次序当为太子。由于明世宗次子朱载壡早逝,所以迟迟未予册立。时景王朱载圳年少,服色与裕王朱载坖无别,引起朝野议论。嘉靖四十(1561)年二月,明世宗命景王朱载圳出居封国,以杜绝其觊觎之心和朝野议论。嘉靖四十四年(1565)正月,景王朱载圳薨,明世宗对内阁首辅建极殿大学士徐阶说:“此子素谋夺嫡,今死矣。”

嘉靖四十五年十二月(公元1567年1月),明世宗驾崩,裕王朱载坖即位,改元隆庆,是为明穆宗。明穆宗立即纠正其父的弊政,之前以言获罪的诸臣全部召用,已死之臣抚恤并录用其后,方士交付有司论罪,以前的道教仪式全部停止,免除次年一半田赋及嘉靖四十三年以前的所有欠赋;又停止明世宗为博孝名强行施行的明睿宗(即明世宗本生父兴献王)明堂配享之礼(即秋季祭天,要以在位皇帝之父合祭,为此导致明太宗庙号被改为明成祖)。

隆庆帝重用徐阶、李春芳、高拱等内阁辅臣,致力于解决困扰朝局多年的“北虏南倭”问题,隆庆元年(1567年),采纳内阁大学士高拱、张居正的建议,与蒙古俺答议和,結束與蒙古長達二百年的戰爭,並有俺答封贡。同年宣布废除海禁,允许民间私人远贩东西二洋,史称隆庆开关。隆庆新政是明穆宗统治时期所出现的承平时期。

明穆宗力行节俭,信用内阁辅臣,并不加以掣肘,但也不能制止内阁辅臣之间的倾轧,这也与其本人仁厚而平庸的性格有关,即位后,首先宣告天下,将废除明世宗时期的所有弊政,一时间朝廷内外都希望新君能有所作为。但是,革弊施新取得实效没多久,他開始宠信太监膝祥等人,挥霍无度,纵情声色,荒废朝政。即位后不久,很快就将权力交给了以高拱为首的内阁,以后只召见过两次阁臣,而他自己就在后宫享乐,广修宫苑,犬马歌舞。

坊间传闻明穆宗特别好色,整天在后宫里忙来忙去,被人比做后宫中辛勤的蜜蜂。他長期服用春药,每天要数名美女陪伴。宫中的用品,小到茶杯,大到龙床,全部都有男欢女爱的雕刻和彩绘。对此,很多大臣都曾上书进谏,竭力劝阻,但他总是很温和地说:“国事有先生我就放心了,家事就不劳先生费心了”。

由于明穆宗贪于女色,纵情声色,加上长期服食春药,他的身体每况日下,难以支撑,萬曆野獲編称其“阳物昼夜不仆,遂不能视朝”。

隆庆六年(1572年)闰三月,宫中传出了明穆宗病危的消息。在休养了两个月之后,他又上朝视事,却又突然头晕目眩,支持不住而回宫。他自知病情不轻,急召高拱、张居正及高仪三人接受顾命,吩咐由太子继位,后崩于乾清宮,终年三十六岁,后被谥为庄皇帝,庙号穆宗,葬于北京昌平明昭陵。

《明穆宗实录》:“上即位,承之以宽厚,躬修玄默,不降阶序而运天下,务在属任大臣,引大体,不烦苛,无为自化,好静自正,故六年之间,海内翕然,称太平天子云。”

《明史》:“穆宗在位六载,端拱寡营,躬行俭约,尚食岁省巨万。许俺答封贡,减赋息民,边陲宁谧。继体守文,可称令主矣。第柄臣相轧,门户渐开,而帝未能振肃乾纲,矫除积习,盖亦宽恕有余,而刚明不足者欤!”

《国榷》:“迹帝之终始,宽大如仁庙,而精勤不若也。安豫如宪朝,而控纵不若也”

《名山藏》:“上端凝靜密,不殺自威,不察自智,優崇輔弼,假借臣僚用能守祖宗之法以致中國乂寧,外夷向風之盛,蓋清靜合軌漢帝寬仁,比跡宋宗矣。上在潛邸時,食驢腸而甘,及即位間問左右,左右請詔光祿,上不忍曰「若爾,則光祿日宰一驢矣。」歲時游吳行幸,諸供膳光祿先期請上旨為豐約,上常裁取最約者焉。”

\subsection{隆庆}

\begin{longtable}{|>{\centering\scriptsize}m{2em}|>{\centering\scriptsize}m{1.3em}|>{\centering}m{8.8em}|}
  % \caption{秦王政}\
  \toprule
  \SimHei \normalsize 年数 & \SimHei \scriptsize 公元 & \SimHei 大事件 \tabularnewline
  % \midrule
  \endfirsthead
  \toprule
  \SimHei \normalsize 年数 & \SimHei \scriptsize 公元 & \SimHei 大事件 \tabularnewline
  \midrule
  \endhead
  \midrule
  元年 & 1567 & \tabularnewline\hline
  二年 & 1568 & \tabularnewline\hline
  三年 & 1569 & \tabularnewline\hline
  四年 & 1570 & \tabularnewline\hline
  五年 & 1571 & \tabularnewline\hline
  六年 & 1572 & \tabularnewline
  \bottomrule
\end{longtable}


%%% Local Variables:
%%% mode: latex
%%% TeX-engine: xetex
%%% TeX-master: "../Main"
%%% End:
