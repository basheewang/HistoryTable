%% -*- coding: utf-8 -*-
%% Time-stamp: <Chen Wang: 2019-12-26 15:06:05>

\chapter{明\tiny(1368-1644)}

\section{简介}

明朝(1368年1月23日-1644年4月25日)是中國歷史上最後一個由漢人建立的大一统王朝,歷經十二世、十六位皇帝,國祚二百七十六年。

元朝末年政治腐败,天灾不断,民不聊生,农民起义屡禁不止,朱元璋加入红巾军并在其中乘势崛起,跟隨佔據濠州的郭子興。郭子興死後,朱元璋率部眾攻佔集慶(今江蘇南京),採取李善长所建议的「高築牆,廣積糧,緩稱王」的政策,鞏固根據地,讓士兵屯田積糧減少百姓負擔,以示自己為仁義之師而避免受敵。1368年,在扫灭陈友谅、張士誠和方国珍等群雄勢力后,朱元璋于当年农历正月初四日登基称帝,立国号为大明,并定都應天府(今南京市),其轄區稱為京師,由因皇室姓朱,因此又稱朱明。後以「驅逐胡虜,恢復中華」為號召北伐中原,少數民族政權統治四百年的燕云十六州也被漢族政權收回,結束蒙元在中國漢地的統治,并最終消滅陳友諒、張士誠和方國珍等各地群雄勢力,统一天下。明初天下大定,经过朱元璋的休养生息,社会经济得以恢复和发展,国力迅速恢复,史称洪武之治。朱元璋去世后,其孙朱允炆即位,但其在靖难之役中败于驻守燕京的朱元璋第四子朱棣,也自此失蹤。朱棣登基后遷都至順天府(今北京市),将北平布政司升為京師,原京師改稱南京。成祖朱棣时期,开疆拓土,又派遣鄭和七下西洋,此後許多漢人遠赴海外,国势达到顶峰,史称永乐盛世。其後的仁宗和宣宗时期国家仍处于兴盛时期,史称仁宣之治。英宗和代宗時期,遭遇土木之变,国力中衰,经于谦等人抗敌,最终解除国家危机。宪宗和孝宗相继与民休息,孝宗则力行节俭,减免税赋,百姓安居乐业,史称弘治中兴。武宗时期爆发了南巡之争和寧王之亂。世宗即位初,引发大礼议之争,他清除宦官和权臣势力后总揽朝纲,实现嘉靖中兴,并于屯门海战与西草湾之战中击退葡萄牙殖民侵略,任用胡宗宪和俞大猷等将领平定东南沿海的倭患。世宗驾崩后经过隆庆新政国力得到恢复,神宗前期任用张居正,推行万历新政,国家收入大增,商品经济空前繁荣、科学巨匠迭出、社会风尚呈现出活泼开放的新鲜气息,史称万历中兴。后经过万历三大征平定内忧外患,粉碎丰臣秀吉攻占朝鮮进而入明的計劃,然而因為国本之争,皇帝逐渐疏于朝政,史稱萬曆怠政,同时东林党争也带来了明中期的政治混乱。

萬曆一朝成為明朝由盛轉衰的轉折期。光宗继位不久因红丸案暴毙,熹宗继承大统改元天啟,天启年间魏忠贤阉党祸乱朝纲,至明思宗即位後铲除阉党。然而因其重用東林黨治國導致政治腐败以及连年天灾,导致国力衰退,最终爆发大规模民变。1644年4月25日(舊曆三月十九),李自成所建立的大順军攻破北京,思宗自缢於煤山,是為甲申之變。隨後吴三桂倒戈相向,满清入主中原。明朝宗室於江南地区相繼成立南明诸政权,原本反明的農民軍加入南明陣營,這些政權被清朝統治者以「为君父报仇」为名各个歼灭,1662年,明朝宗室最後政權被剷除,永曆帝被俘殺,滿清又击败各地农民军,以及進攻由漢人首次管理的台湾,直到1683年清朝攻占奉大明為正朔的明郑方止。

明代的核心領土囊括汉地,东北到外興安嶺及黑龍江流域,後縮為遼河流域;初年北達戈壁沙漠一帶,後改為今長城;西北至新疆哈密,後改為嘉峪關;西南临孟加拉湾,后折回约今云南境;曾經在今中国东北、新疆東部及西藏等地設有羈縻機構。不過,明朝是否實際統治了西藏國際上存在有一定的爭議。明成祖時期曾短暫征服及統治安南,永乐二十二年(1424年),明朝国土面积达到极盛,在东南亚设置旧港宣慰司等行政机构,加强对东南洋一带的管理。

明代商品经济繁荣,出现商业集镇,而手工业及文化艺术呈现世俗化趋势。根據《明实录》所载的人口峰值于成化十五年(1479年)达七千余万人,不过许多学者考虑到当时存在大量隐匿户口,故认为明朝人口峰值实际上逾亿,还有学者认为晚明人口峰值接近2亿。这一时期,其GDP总量所占的世界比例在中国古代史上也是最高的,1600年明朝GDP总量为960亿美元,占世界经济总量的29.2\%,晚明中国人均GDP在600美元。

明朝政治中央废除丞相,六部直接对皇帝負責,後来设置内阁;地方上由承宣布政使司、提刑按察使司、都指挥使司分掌权力,加强地方管理。仁宗、宣宗之后,文官治国的思想逐渐浓厚,行政权向内阁和六部转移。同时还设有都察院等监察机构,為加強對全國臣民的監視,明太祖設立特務機構錦衣衛,明成祖設立東廠,明憲宗时再設西廠(後取消),明武宗又設內廠(後取消),合稱「廠衛」。但到了后期出现了皇帝怠政,宦官行使大權的陋習,然而决策权始终集中在皇帝手里,不是全由皇帝独断独行。有许多事还必须经过经廷推、廷议、廷鞫的,同时还有能将原旨退还的给事中,另到了明代中晚期文官集團的集體意見足以與皇帝抗衡,在遇到事情決斷兩相僵持不下時,也容易產生一種類似於「憲法危機」的情況,因此「名義上他是天子,實際上他受制於廷臣。」。但明朝皇權受制於廷臣主要是基於道德上而非法理上,因為明朝當時風氣普遍注重名節,受儒家教育的皇帝往往要避免受到「昏君」之名。但是,皇帝隨時可以任意動用皇權,例如明世宗「大禮議」事件最後以廷杖朝臣多人的方式結束。

有学者认为明代是继汉唐之后的黄金时期。清代張廷玉等修的官修《明史》评价明朝为「治隆唐宋」、「遠邁漢唐」。

朱元璋早期给新的王朝定名为大中,后正式定国号为“大明”,是元朝以来中国历史上第二个把“大”字加于正式国号之中的大一统王朝,又称皇明,后世称为明朝或明代,又因皇室姓朱,又称朱明。部分人認為明朝之号承袭自小明王韩林儿之号,但韓林兒的國號為宋,而朱元璋部的大旗“山河奄有中華地,日月重開大宋天”、“ 九天日月開黃道,宋國江山複寶圖”反而有些關係。朱元璋手下有一部分明教徒,以“大明”為国号以表示自己的正统地位,亦同时应和明教中的“明王出世”预言。其次,以明喻火,根据五德终始说,表示明朝取代元朝,是以火剋金。

但七十年代,學界開始有人質疑“明王”是否出於明教(摩尼教) 。八十年代初,楊訥閱讀現存所有元代白蓮教史料後,否定吳晗學說。他除指出吳晗論文方法上的錯誤,及引証史料之疏漏外,並以傳世史料,証實元末起事者所提“彌勒佛下生”與“明王出世”口號,均與明教無涉,而出於佛教經典。但不論吳唅或楊訥,都是從宗教角度來探究。直到2014年,北京大學博士生杜洪濤突破了吳晗學說窠臼,循元明承續的思路,參照趙翼大元國號出自《易經•乾卦》“大哉乾元” 文義,而主張大明國號亦出自《易經•乾卦》“大明終始”這一字句,為大明此一國號的源由又增添了一種說法。

1644年4月24日(舊曆三月十八),明朝首都沦陷后,明朝宗室在江南地区建立政权仍沿用大明国号,别称南明或后明,清廷則称为伪明,一直坚持到1662年。而郑芝龙、郑成功等郑家势力在台湾建立了政权,史称東寧王國。

还有人指出,明之得号出于明教。明教在唐朝武则天延载年间,传到中国,但是一直保持神秘,因为明教宣传的是“弥勒降生,明王下世”。一些反抗朝廷的人经常借助于明教来号召群众,为了保护自己,明教就跟佛教拉上关系,和佛教的白莲宗拉上关系,最后就形成了白莲教。所以从唐朝、宋朝、元朝明教是时而浮出,时而潜入地下,但是常常用作反抗朝廷的武器。

元朝末期,官員貪污,貴族靡爛,朝政腐敗。為消除赤字,元廷加重賦稅,並且大量濫印新鈔「至正寶鈔」,隨之產生的通貨膨脹加上荒災、黃河氾濫等天災比以往任何時候發生得都要頻繁,使得民不聊生。1351年元順帝派賈魯治理黃河,徵調各地百姓二十萬人。同年五月,白蓮教韓山童與劉福通煽動飽受天災與督工苛待的百姓叛元起事。他自稱明王,建立紅巾軍,據有河南與安徽等地。紅巾軍與各地義軍陸續起事,勢力擴張到華中、華南地區。隔年,紅巾軍的郭子興聚眾起義,攻佔濠州(今安徽鳳陽)。不久,貧苦農民出身的安徽鳳陽人朱元璋投奔郭子興,屢立戰功,得到郭子興的器重和信任,並娶郭子興養女為妻。之後,朱元璋離開濠州,發展自己的勢力。

1356年朱元璋率兵佔領集慶(今江蘇省南京市),改名為應天府,並攻下周圍一些軍事要地,獲得一塊立足的基地。朱元璋採納謀士朱升「高築牆,廣積糧,緩稱王」的建議,經過幾年努力,其軍事和經濟實力迅速壯大。1360年,陳朱雙方在集慶西北的龍灣展開惡戰,陳友諒勢力遭到巨大打擊,逃至江州,史稱洪都之戰(今江西省九江市)。1363年,通過鄱陽湖水戰,陳友諒勢力基本被消滅。1367年朱元璋自稱吳王,率軍攻下平江(今江蘇省蘇州市),滅張士誠,同年又消滅割據浙江沿海的方國珍。

1368年正月,朱元璋於南京稱帝,即明太祖,年號洪武,明朝建立。之後趁元朝內訌之際乘機北伐和西征,同年攻佔元大都(今北京),元廷撤出中原,史稱北元。之後於1371年消滅位於四川的明玉珍勢力,於1381年消滅據守雲南的元朝梁王。最後,於1388年深入漠北進攻北元。天下至此初定。而朱元璋对于不願效忠新朝的蒙古人和色目人,则表示愿意归顺的可以在大明,不愿意的可以自行离开。

明初不願仕官和不願效忠新朝廷的地主文人為了逃避徵辟而採取自殺、自殘、逃往漠北、 隱居深山等方法,誓不出仕(中國古代銓選,有「身言書判」四方面標準,身體有殘疾者不能任官)。為應對元遺民對明政權的鄙夷與漠視,朱元璋設立新刑罰,宣告「士大夫不為君用」律,大規模徵辟前朝遺老、搜羅岩穴隱士,並且殺害不願效忠明朝以及為新朝當官的學者,表示「率土之濱,莫非王臣。寰中士大夫不為君用,是自外其教者,誅其身而沒其家,不為之過」,導致「才能之士,數年來倖存者百無一二,今所任率迂儒俗吏」。

由于幼年对于元末吏治痛苦记忆,明太祖即位后一方面減輕農民負擔,恢復社會的經濟生產,改革元朝遺留的吏治,懲治貪官,社會經濟從戰亂中得到恢復和發展,史稱洪武之治。明太祖確立里甲制,配合賦役黃冊戶籍登記簿冊和魚鱗圖冊的施行,落實賦稅勞役的徵收及地方治安的維持。同时对外加强海外交流,恢复中华宗主国地位。

平定天下後,明太祖大封功臣。但随后基于巩固皇权的考虑,加之不少功臣或骄纵或横行乡里或僭越等,明太祖兴起胡惟庸案和藍玉案,幾乎將功臣及权贵誅殺。廖永忠成为最先被处置的功臣。丞相胡惟庸深得朱元璋寵信,但之后日益跋扈,朝中奏章大事須先經其手,若不利於其的奏章就予以隱匿,並且大肆收取賄賂。1380年明太祖以擅權枉法之罪名殺胡惟庸,又殺御史大夫陳寧、御史中丞塗節等人。1390年有人告發李善長與胡惟庸關係密切,李善長因此被賜死,家屬七十餘人被殺,總計株連者達三萬餘人,史稱胡惟庸案,明太祖更藉此案廢除中書省和相職。此後,明太祖又借大將軍藍玉張狂跋扈之名對其誅殺,連坐被族誅的有一萬五千餘人,史稱藍玉案。加上空印案與郭桓案合稱明初四大案。此時除湯和與耿炳文外功臣几乎全数被杀。明太祖通過打擊权臣、特務監視等一系列方式加強皇權,使明初的皇帝專制程度與中國歷代各朝相比更為嚴重。

明太祖分封诸子為王,以加強邊防,藩屏皇室。諸王之中,以北方諸王勢力較強,又以秦王朱樉、晉王朱棡與燕王朱棣的勢力最大。為防止朝中奸臣不軌,明太祖規定諸王可移文中央捉拿奸臣,必要時得奉天子密詔,領兵「靖難」(意为“平定國難”)。同時為防止諸王尾大不掉,明太祖也允許今後的皇帝在必要時可下令「削藩」。

洪武三十一年(1398年)明太祖驾崩,由於皇太子朱標於七年前因巡视陕西而病薨逝,遗诏由皇太孫朱允炆即位。改年號建文,即明惠宗(亦稱建文帝、明惠帝)。明惠宗為鞏固皇權,與親信大臣齊泰、黃子澄等密謀削藩。周王、代王、齊王、湘王等先後或被廢為庶人,或被殺。同時以邊防為名調離燕王的精兵,準備削除燕王。結果燕王朱棣在姚廣孝的建議下以「清君側,靖內難」的名義起兵,最後迂迴南下,佔領京師,是為靖難之變。朱棣即位,即明成祖,年號永樂。明惠宗在宮城大火中下落不明。明成祖對支持明惠宗者大肆殺戮,諸如黃子澄、齊泰等。

繼洪武之治,明成祖、明仁宗與明宣宗相繼興起永樂盛世與仁宣之治,這是明朝的興盛時期之一。明成祖時期武功昌盛,明成祖先是出擊安南,将安南纳入明朝版图,设立交趾布政司。明成祖之后又親自五入漠北攻打北元分裂後的韃靼與瓦剌。明成祖冊封瓦剌三王,使與韃靼對立,等到瓦剌興盛後又助韃靼討伐瓦剌,不使任何一方独大。同时,明成祖撤去大宁都司,将宁王朱权内迁南昌,授予兀良哈蒙古的朵颜、泰宁和福余三个卫所自治权,但不允许三卫蒙古人南迁到大宁地区驻牧。明成祖还于1406年和1422年对兀良哈蒙古进行镇压,以维持这一地区的稳定。明成祖為安撫東北女真各部,在歸附的海西女真(位於松花江上游)與建州女真(位於松花江、牡丹江之間)設置衛所,並派亦失哈安撫位於黑龍江下游的野人女真。1407年亦失哈在黑龙江下游东岸奴儿干地方(元朝征东元帅府旧地)設置奴兒干都司,擴大明朝東疆,亦失哈并于1413年视察库页岛,宣示明朝对此地的宗主权。明成祖一改明太祖閉關自守的外交策略,自1405年開始派宦官鄭和下西洋,向各國交往、宣示威德以及建立朝貢體制,也有為圍堵西亞帖木兒帝國的說法。鄭和下西洋前後七次,前六次均在永乐年间由明成祖派遣,郑和船队足迹遍佈東南亞與南亞地區,還於滿剌加建有基地。其規模空前,最遠到達東非索馬利亞地區,擴大明朝對南洋、西洋各國的影響力。

文治方面,明成祖修大型類書《永樂大典》,在三年時間內即告完成。《永樂大典》有22877卷,其中凡例、目錄60卷,全書分裝為11095冊,引書達七八千種,字數約有三億七千多萬, 且未有任何刪節,《永樂大典》在編成後即被深鎖皇宮數百年,以至當時有多人認為《大典》已在戰火中被毀。根據記載,明朝年間僅有明孝宗和明世宗二帝閱《大典》。此外,明成祖并未将《永乐大典》复写刊刻,且决定只制作一份抄本,並于1409年完成。1405年明成祖將北平改名北京,稱行在,並設立北平國子監等衙門。1409年,明成祖巡幸北京,在北京設立六部與都察院,並在北京為逝世的徐皇后設立陵寢,已經顯示遷都的跡象。經過十幾年的經營,北京初步得到繁榮。1416年明成祖公佈遷都的想法,得到認同,隔年開始大規模營造北京。1420年宣告完工,隔年正式永樂遷都。因為永樂年間天下大治,並且大力開拓海外交流,史稱為永樂盛世,有學者將這段時期稱為永樂盛世,亦有史學家評價成祖遷都北京之舉是“天子守國門”,或称天子戍边、天子守边。

明成祖驾崩後,其長子朱高熾即位,即明仁宗,年號洪熙。明仁宗年齡已經偏高,即位僅一年就駕崩。其統治偏向保守固本,任用「三楊」(楊士奇、楊榮、楊溥)等賢臣輔佐朝政,停止鄭和下西洋和對外戰爭以積蓄民力,鼓勵生產,寬行省獄,力行節儉。明仁宗驾崩後長子朱瞻基即位,是為明宣宗,年號宣德。他基本繼承父親的路線,實行德政治國,並且發起最後一次下西洋。明宣宗同樣熱愛美術,有畫作傳世。但是,其執政期間也並非毫無弊端。由於明宣宗喜好養蟋蟀,許多官吏因此競相拍馬,被稱為「促織天子」。同時,明宣宗打破明太祖留下的宦官不得干政的規矩,一些太監如王振等人開始干政,為明英宗時期的太監專權埋下隱患。1435年明宣宗去世,九歲的朱祁鎮繼位,即明英宗,年號正統。

明英宗自小寵信服侍左右的宦官王振,自此開始明朝的宦官嚴重專權行為。1442年限制王振權勢的張太皇太后去世,當時明英宗僅十五歲,王振更加攬權。元老重臣「三楊」死後,王振專橫跋扈,將明太祖留下的禁止宦官干政的敕命鐵牌撤下,舉朝稱其為「翁父」,明英宗對他信任有加。王振擅權七年,家產計有金銀六十餘庫,其受賄程度可想而知。

1435年蒙古西部的瓦剌逐漸強大,經常在明朝邊境一帶生事。1449年瓦剌首領也先率軍南下伐明。王振聳使明英宗領兵二十萬御駕親征。大軍離燕京後,兵士乏糧勞頓。八月初大軍才至大同。王振得報前線各路潰敗,懼不敢戰,又令返回。回師至土木堡(今日河北省張家口懷來縣),被瓦剌軍追上,士兵死傷過半,隨從大臣有五十餘人陣亡。明英宗突圍不成被俘,王振為將軍樊忠所怒殺,史稱土木堡之變,是明朝由盛轉衰的一個轉捩點。

土木堡之變的消息來到北京後,朝中混亂。一些大臣要求遷都南京應天府,被兵部侍郎于謙駁斥。同年,大臣擁戴明英宗弟朱祁鈺即位,以求長君,即明景帝(又稱明代宗),年號景泰。于謙升兵部尚書,整頓邊防積極備戰,同時決定堅守北京,隨後京師、南京、河南、山東等地勤王部隊陸續趕到。同年十月,瓦剌軍直逼北京城下,也先安置明英宗於德勝門外土關。于謙率領各路明軍奮勇抗擊,屢次大破瓦剌軍,也先率軍撤退。明朝取得北京保衛戰的勝利,于謙力排眾議,加緊鞏固國防,拒絕求和,並於次年擊退瓦剌多次侵犯。

也先認為綁架明英宗已無意義,於1450年釋放之。然而明景帝因為皇權問題,不願意接受明英宗,先是不願遣使迎駕,又把明英宗困於南宮(今南池子)軟禁,並廢皇太子朱見深(明英宗之子,後來繼位為明憲宗),立自己的兒子朱見濟為太子。不久見濟病死,沒有兒子的景帝也遲遲不肯再立朱見深為太子,儼然有奪正之貌,英宗、景帝兄弟因而嚴重對立。

1457年石亨、徐有貞等人聯盟,欲擁戴明英宗復辟。趁著景帝重病之際發動兵變。徐有貞率軍攻入紫禁城,石亨等人占領東華門,立明英宗於奉天殿,改元天順。他們禁錮了景帝,並且捕殺了于謙及大學士王文,史稱奪門之變。由於兩次即位之故,明英宗也成為明清皇帝中,唯一使用兩個年號的皇帝。明英宗復辟後,略有新政,廢除自明太祖時殘酷的殉葬制度。之後因為內部政變流放徐有貞,因為曹石之變誅殺石亨、曹吉祥等人,並且以李賢等賢臣掌政。1464年明英宗去世後,兒子朱見深即位,即明憲宗,年號成化。

明憲宗為于謙冤昭雪,恢復景帝的帝號,平反奪門一案,人多稱快。而初年勵精圖治,任用賢臣,體諒民情,蠲賦省刑,善政史不絕書,又在武功有屢有建樹,如在丁亥之役中與朝鮮進攻屢次進犯的建州女真等,儼然為一代明君,史稱成化新風,堪稱與仁宣之治媲美。但明憲宗口吃內向,因此很少廷見大臣,終日沉溺於亦妻亦母的萬貴妃,寵信宦官汪直、梁芳等人,晚年好方術。以至奸佞當權,西廠橫恣,盜竊威柄。明憲宗直接頒詔封官,是為傳奉官。這使得傳奉官氾濫,舞弊成風,直到明孝宗才全被裁撤。他也是皇莊的始置者。該舉措事實上鼓勵豪強門閥兼併土地,危害不淺。宦官汪直受到明憲宗的寵信,張狂跋扈,透過西廠大肆冤殺普通民眾與官員。不久後由於民憤四起,西廠被罷,但汪直依然握有大權。直到1482年汪直因言官彈劾才被貶。成化一朝羣小當道:女寵、外戚、佞幸、奸宦、僧道共聚一堂,朋比為奸,濁亂朝政。1487年明憲宗去世,其子朱祐樘繼位,即明孝宗,年號弘治。

明孝宗自幼於貧寒出身,曾有被萬貴妃加害的危險。其在位期間「更新庶政,言路大開」,使得自明英宗以來的陋習得以去除,被譽為「中興之令主」。明孝宗先是將明憲宗時期留下的一批奸佞冗官盡數罷去,逮捕治罪。並選賢舉能,將能臣委以重任。明孝宗勤於政事,每日兩次視朝。明孝宗對宦官嚴加節制,錦衣衛與東廠也謹慎行事,用刑寬鬆。明孝宗力行節儉,不大興土木,減免稅賦。他本身踐行一夫一妻制,一生除張皇后外沒有任何妃嬪。明孝宗的勵精圖治,使得弘治時期成為明朝中期以來形勢最好的時期,明史稱明孝宗「恭儉有制,勤政愛民」,被稱為弘治中興,然而在弘治中後期明孝宗不再認真聽從諫諍,並且開始揮霍無度,導致國家步入了「一歲所入,不足以供一歲支用」,「太倉無儲,內府殫絀」以及邊備日弛的狀況,在弘治初期革除的弊政不僅全部恢復,而且還更加惡化<,其次,有明一代,以弘治對外臣最為縱容厚待,動則大肆對外戚藩王賞賜房屋,田地,造成嚴重的土地兼併問題。

1505年明孝宗去世,其子朱厚照即位,是為明武宗,年號正德。

及至明武宗一朝,宦官势力重新抬头,其归因于武宗精于游乐,怠于政事。不过,其祸患本身并未危及皇权,虽有刘瑾、谷大用等八虎为非作歹,但始终未曾如唐朝末年的宦官擅权情况,并且刘瑾等人最终仍被武宗处以极刑。武宗的喜好游逸,最终导致孝宗一脉绝嗣。并且致使大明统系发生第二次小宗入为大宗的情况。明武宗的荒游逸樂導致正德年間戰事頻生,先後發生韃靼達延汗(明史稱韃靼小王子)進犯、寧夏安化王朱寘鐇謀反、山東劉六劉七民變、江西寧王朱宸濠謀反等重大事件。1520年明武宗假藉出征江西寧王為由而南下遊玩,以大將軍朱壽為名前往南京,親自俘虜已被王守仁擊敗的寧王。班師回京途中,於南直隸清江浦(江蘇淮安)泛舟取樂時落水染病,1521年於豹房驾崩。

明武宗驾崩后,明孝宗之侄,兴献王之子朱厚熜入嗣大统,是為明世宗,年號嘉靖。登基前后,因时任内阁首辅杨廷和、礼部尚书毛澄等权臣引宋濮安事强令明世宗尊亲生父母为皇叔父母,引起明世宗的反感,是为大礼议之争。最终明世宗在張璁等不服权臣此举的朝官支持下得以尊父母為皇帝與皇后、立太廟在明武宗之上、修皇帝實錄。這次政治風波使反對者被罷官或被入獄,受杖者一百八十餘人,杖死者十七人。在清除權臣與宦官後,明世宗開始實行自己的政治抱負,任用張璁等賢臣,英明苛察,嚴以馭官,整頓朝綱,鼓勵耕織和減輕租銀,又勘查皇室莊園和勛戚莊園,減輕土地兼併,在軍事上大力提拔將才征剿倭寇,清除外患,整頓邊防,以解除邊疆危機,史稱「嘉靖中興」。

1534年後明世宗即不視朝,但仍悉知帝国事务,事无巨细仍出于明世宗决断。明世宗信奉道教,信用方士,在宮中日夜祈禱。先是將道士邵元節入京,封為真人及禮部尚書。邵死後又大寵方士陶仲文。1542年十月,乾清宮發生楊金英、邢翠蓮等宮女十餘人與寧嬪王氏趁明世宗熟睡之際企圖將其勒死,但未成功,此即壬寅宮變。此事后,直至明世宗驾崩前一晚,明世宗迁离大内移居西内。明世宗寵信權臣嚴嵩,他借此排斥異己,結黨營私。其子嚴世蕃協助其父作惡。朝臣雖然不斷有人彈劾嚴嵩結黨營私,但均以失敗告終。世宗晚期,嚴嵩年事已高,朝臣徐階開始取代嚴嵩之位。1562年徐階策動言官彈劾首輔大臣嚴嵩。嚴嵩辭去官職回鄉。1565年嚴世蕃以通倭罪被判斬刑、嚴嵩被削為民,兩年後病死。

嘉靖一朝,國家外患不斷。北方韃靼趁明朝衰弱而佔據河套。1550年韃靼首領俺答進犯大同,宣大總兵仇鸞重金收買俺答,讓其轉向其他目標。結果俺答轉而直攻北京,在北京城郊大肆搶掠之後西去,明朝軍隊在追擊過程中戰敗,此為庚戌之變。由於世宗時期明朝宣布海禁,由日本浪人與中國海盜組成的倭寇與沿海居民合作走私,先並且後襲擾山東、浙江、福建與廣東等地區。朱紈、張經等將領受明廷干擾而未能平定倭寇。而後兵部尚書胡宗憲署理浙江巡撫兼浙直總督全力剿倭,招撫浙江勢力最強的汪直(後被明廷殺害)。戚繼光與俞大猷平定浙閩粵等地的倭寇,為後來隆慶開關建立好背景。另外葡萄牙人在1557年開始移民澳門,但及至明亡,葡萄牙人及澳门始终为广东布政司香山县管辖。1566年明世宗驾崩,皇太子朱載坖即位,即明穆宗,年號隆慶。

明穆宗即位後,先後任用徐階、高拱與張居正等名臣。1567年位處執政之首的明世宗舊臣徐階策動朝官彈劾高拱,迫高拱辭官回鄉。高拱亦不甘示弱,一年後策動朝官彈劾徐階。徐階也被迫正式退休。朝廷的實際政務漸漸落到張居正的手上。隆慶末年,高拱回朝出任內閣首輔。隆慶朝名臣名將薈萃,陸上與韃靼首領俺答汗達成和議,史稱俺答封貢;海上開放民間貿易,史稱隆慶開關;因為這兩項措施與其他改革措施,明朝開始進入中興時期,史稱隆慶新政。1572年,明穆宗因中風突然駕崩,年僅九歲的皇太子朱翊鈞繼位,即明神宗,年號萬曆。

由於明神宗年幼,於是由太后攝政。重臣高拱由於與太后信任的宦官馮保對抗而被罷官;相反的,張居正得到馮保的鼎力支持。張居正輔政十年,推行改革,在內政方面,提出「尊主權,課吏職,行賞罰,一號令」,推行考成法,裁撤政府機構中的冗官冗員,整頓郵傳和銓政。經濟上,清丈全國土地,抑制豪強地主,改革賦役制度,推行一條鞭法,減輕農民負擔。1393年明太祖時期,全國耕種田地有三百六十六萬零七千七頃,到1502年明孝宗時期也只上升到四百廿二萬八千零五十八頃。經過張居正的治理後於1581年達到七百零一萬三千九百七十六頃。軍事上,加強武備整頓,平定西南騷亂,以名將戚繼光守衛北京的重鎮薊州、以遼東李成梁安撫東北女真、以宣大王崇古、方逢時安撫韃靼,其他重臣如四川的劉顯、兩廣的殷正茂、凌雲翼、浙江的張佳胤,張居正也十分信任他們。張居正還啟用潘季馴治理黃河,變水患為水利。同時張居正嚴懲貪官污吏,裁汰冗員。張居正整頓朝正,改革體制,史稱萬曆中興。

1577年張居正父親去世,按常理他需要丁憂,但張居正以為改革事業未竟,不願丁憂。他的政敵借此大做文章,此即為奪情之爭。最後在明神宗和兩太后的力挺下張居正被奪情起復,使得其改革並未被中斷。但是,這成為他的政敵之借口。同時,張居正利用自己的職權讓自己的兒子順利通過科舉進入翰林院。除此之外,張居正的私德也有問題,各種聚斂財物的情事被揭露,張居正也迫害了大量的政敵,好同惡異,為政專擅,他一死,立刻在萬曆的支持下,被昔年結怨的大臣清算,張居正家被抄家。張府一些來不及退出的人被囚禁於內,餓死十餘口。生前官爵也被剝奪。

張居正死後,明神宗親政,勵精圖治,勤於朝政,更新庶政,繁榮經濟,廢黜考成法等張居正改革中弊政,安撫流民,減少徭稅,有勤勉明君之風範,維持了中興。然后后来发生的国本之争,拉来了明末党争的纷乱和明朝没落的序幕。國本之爭是贯穿于明神宗中期至晚期的重大政治事件。主要是圍繞著皇長子朱常洛與福王朱常洵(鄭貴妃所生)繼承皇位之爭。由於皇后无嗣,明神宗偏愛皇三子朱常洵,不願立皇長子朱常洛為太子,令群臣憂心如焚,朝中的大臣也藉此開始黨爭。直到1601年在皇太后的強迫下,朱常洛才被封為太子,而朱常洵被封為福王,封地為洛城,卻遲遲不離京就任藩王。直到梃擊案發生,輿論對鄭貴妃不利後,福王才離京就藩,太子朱常洛的地位也因而穩固。

明神宗於國本之爭對大臣極度不滿,采取以不上朝作為報復,僅偶爾批閱奏摺,以處理一些重要事件,但如明世宗一樣,悉知帝国事务,事无巨细仍出于其之决断。大理寺左評事上疏,稱明神宗沉湎於酒、色、財、氣,結果被貶為民。明神宗中後期财政困难,因此明神宗派太監為天下礦監和稅監以充實內庫,然而矿监税使大多假借名義搜刮民間財產,擾亂天下。由於明神宗不理朝政,缺官現象非常嚴重。1602年,南北兩京共缺尚書三名,侍郎十名;各地缺巡撫三名,布政使、按察使等六十六名,知府廿五名。明神宗委頓於上,百官黨爭於下,明廷完全陷入空轉之中。因此明史言:「論者謂:明之亡,實亡於神宗。」,部分史學家認為明朝自此开始走向滅亡。

由於朝政混亂,部分中下階官吏在政治上受到排斥,紛紛要求政治改革,並強調道德標準。1593年癸巳京察促成東林黨的形成,其名稱源自顧憲成重修的東林書院。主持京察的孫鑨、李世達和趙南星,利用京察將不符他们標準和不属于东林党的官吏降職解雇。經過多次京察後,引起眾多反對黨如宣黨、崑黨、齊黨、浙黨等興起並與東林黨互相傾軋。自此門戶之禍堅固而不可拔,圖使朝政空轉內耗。明熹宗時反對黨在東廠魏忠賢的羽翼下成為閹黨,開始專權,並且迫害東林黨人,東林黨受到嚴重打擊,有所謂東林六君子、東林七賢等被閹黨殺害,直到明思宗即位,才整肅了閹黨。

在對外軍事方面,以萬曆三大征最為顯著,分別為平定蒙古哱拜叛變的寧夏之役、抗擊日本豐臣政權入侵朝鮮王朝的朝鮮之役,以及平定苗疆土司楊應龍叛變的播州之役,這三場戰爭幾乎同時發生,其性質均不相同。明朝於三戰皆勝以鞏固明朝邊疆、守護朝鮮王朝,但也消耗大量人力物力,成為國庫空虛、財政拮据的重要原因之一。粗略統計出這八年間國家的軍事開支高達一千一百六十餘萬兩白銀。1617年後金努爾哈赤以「七大恨」為由反明,兩年後在薩爾滸之戰中大敗明軍,明朝至此對後金改以防禦為主的戰略。

1620年明神宗去世。其長子朱常洛登基,即明光宗,年號泰昌,在位僅一個月。他發內帑賞賜在遼東前線明軍,重用東林黨人使朝政轉危為安,並且罷除天下礦監稅使。福王生母鄭貴妃為了攏絡明光宗,獻上四位美女。明光宗縱慾過度不久病倒,太監崔文升進以瀉藥而狂瀉,又因服用李可灼的紅丸而猝死,史稱紅丸案。明光宗逝世後,其寵妃李選侍欲居乾清宮,以挾皇長子朱由校自重。都給事中楊漣、御史左光斗等,為防其干預朝事,逼迫李選侍移到仁壽殿哕鸞宮,此即移宮案。皇長子朱由校最後得以繼位,即明熹宗,年號天啟。梃擊案、紅丸案與移宮案合稱明末三大案,是萬曆晚期國本之爭的延續,使得明廷的政治鬥爭更加劇烈,也是標誌著明末衰亡的開始。

明朝末年,明朝的对外贸易陷入低谷,白银输入大量减少,由于农民缴税需要用到白银,但是一般农民只有铜钱,造成白银价格暴涨,农民无法缴税,大量逃亡,造成民变。

明熹宗在位期間,政治更加腐敗黑暗。熹宗幼年喪母,對乳母客氏有特殊感情。客氏與宦官魏忠賢狼狽為奸。熹宗早期,倚賴東林黨人力爭,方能登基,故大量啟用東林黨人,結果導致東林黨與其他黨鬥爭不斷,明熹宗因此對朝政失去耐心,魏忠賢借此機會干預政治,將齊楚浙黨的勢力集結,號為閹黨。1624年閹黨控制內閣,魏忠賢更加張狂,其爪牙遍佈中央與地方。在權勢最盛時,魏忠賢的養子竟能代替皇帝祭太廟。全國遍佈他的生祠,並號為九千歲後又稱九千九百歲。更有閹黨的國子監生提出魏忠賢配孔子,魏忠賢父配啟聖公。魏忠賢並大肆打擊東林黨,借「梃擊、紅丸、移宮」三案為由,唆使其黨羽偽造《東林黨點將錄》上報朝廷,1625年明熹宗下詔,燒燬全國書院。大量東林黨人入獄,甚至處死。由於閹黨水準低下,政理不修。國家內部饑荒頻傳,民變不斷,外患持續,明朝已經陷入風雨飄搖之境地。1626年北京西南隅的工部王恭廠火藥庫發生王恭廠大爆炸,造成2萬多人死傷。1627年明熹宗不慎落水病重,不久因霍維華之藥而去世,其五弟信王朱由檢繼位,即明思宗,年號崇禎。

明思宗即位後,銳意剷除魏忠賢的勢力以改革朝政。他下令停建生祠,逼奉聖夫人客氏移居宮外,最後押到浣衣局處死。下令魏忠賢去鳳陽守陵,魏忠賢於途中與黨羽李朝欽一起自縊,明思宗將其首級懸於河間老家,閹黨其他分子也被貶黜或處死。然而黨爭內鬥激烈,明思宗不信任百官,他剛愎自用,加強集權。當時東北方的後金(即後來的清朝)占領遼東地區,袁崇煥等人於遼西寧遠、錦州等抵禦後金可汗皇太極的入侵。1629年皇太極改採繞道長城以入侵北京,袁崇煥緊急回軍與皇太極對峙於北京廣渠門。经六部九卿会审,最後殺袁崇煥,史稱己巳之變。其後皇太極多番遠征蒙古,終於在六年後徹底擊敗林丹汗,取得了傳國玉璽,1636年在盛京稱帝,改國號為大清,即清朝。並且陸續發起五次經長城入侵明朝直隸、山東等地區,史稱清兵入塞。當時直隸連年災荒疫疾,民不聊生。遼西局勢亦日益惡化,清軍多次與明軍作戰,最後於1640年占領錦州等地,明軍主力洪承疇等人投降,明朝勢力退縮至山海關。

明朝中期之後時常發生農民起事,崇禎雖勵精圖治,但其任人不得法(崇禎一朝撤換過五十個大學士,號稱「崇禎五十相」,為歷朝之最),朝政混亂與官員貪污昏庸;與後金的戰爭帶來大量遼餉的需求以及清兵的掠奪;以及因為小冰期氣候變冷,在當時連海南島都出現下雪氣候,農業減產帶來全國性饑荒,這些都加重明朝百姓的負擔。1627年,陝西澄城饑民暴動,拉開明末民變的序幕,隨後王自用、高迎祥、李自成、張獻忠等農民起事,最後發展成雄踞陝西、河南的李自成與先後占領湖廣、四川的張獻忠(最後成立大西政權)。1644年李自成建國大順,三月,李自成率軍北伐攻陷大同、宣府、居庸關,最後於1644年4月24日(舊曆三月十八)攻克北京。明思宗在煤山自縊,史稱甲申之變。後世有史學家評價思宗在社稷危難之時沒有逃跑是“君主死社稷”,但亦有學者指出崇祯多次迁都南京的計劃。

李自成攻克北京後,縱容部將在京城內大肆搜刮遂失民心。原為明將、鎮守山海關的吳三桂帶領清軍入關,並於一片石戰役擊敗大順軍。清朝攝政王多爾袞與順治帝入關,北京成為清朝的首都。李自成退回陝西,最後被清軍圍殲於湖北,大順亡。

甲申之變後,明朝在南方尚有勢力,史稱南明。南明主要勢力有四系王,分別是福王弘光帝朱由崧、魯王監國朱以海、唐王隆武帝朱聿鍵與紹武帝朱聿𨮁、桂王永曆帝朱由榔等。當南明滅亡後,又有鄭成功建立的明鄭與夔東十三家軍抗清。1644年北京被李自成攻陷後,南明大臣意圖擁護皇族北伐。經過多次討論後由鳳陽總督馬士英與江北四鎮高傑、黃得功、劉澤清與劉良佐擁護明思宗的堂兄弟福王朱由崧稱帝,即弘光帝,史稱南明。1645年清朝派多鐸率大軍南下南京,此時弘光帝昏庸,大權由閹黨餘孽掌握,江北四鎮各自為營,最後陸續瓦解。清軍攻破史可法死守的揚州,弘光帝逃至蕪湖被逮,送到北京殺害。此期間清軍發起揚州十日、江陰八十一日與嘉定三屠等大屠殺以鎮壓反抗的漢人。同时明朝数十万皇族也惨遭清廷和農民軍的屠杀。

弘光帝死後,魯王朱以海於浙江紹興監國;而唐王朱聿鍵在鄭芝龍等人的擁立下,於福建福州稱帝,即隆武帝。然而這兩個南明主要勢力互不承認彼此地位而互相攻打。1651年在舟山群島淪陷後,魯王朱以海在張名振、張煌言陪同下,赴廈門依靠鄭成功,不久病死在金門。隆武帝屢議出師北伐,然而得不到鄭芝龍的支持而終無所成。1646年,清軍分別占領浙江與福建,魯王朱以海逃亡海上,隆武帝於汀州逃往江西時被俘而死。鄭芝龍向清軍投降,由於其子鄭成功起兵反清而被清廷囚禁。朱聿鍵死後,其弟朱聿𨮁在廣州受蘇觀生及廣東布政司顧元鏡擁立稱帝,即紹武帝,於同年年底被清將李成棟攻滅。同時間桂王朱由榔於廣東肇慶稱帝,即永曆帝。

1646年永曆帝獲得瞿式耜、張獻忠餘部李定國、孫可望等勢力以及福建鄭成功勢力的支援之下展開反攻。同時各地降清的原明軍將領先後反正,例如1648年江西金聲桓、廣東李成棟、廣西耿獻忠與楊有光率部反正,一時之間南明收服華南各省。然而於同年,清將尚可喜率軍再度入侵,先後占領湖南、廣東等地。兩年後,李定國、孫可望與鄭成功發動第二次反攻,其中鄭成功一度包圍南京。然而,各路明軍因為距離互相難以照應,內部又發生孫可望等人的叛變,第二次反攻以節節敗退告終。1661年,清軍三路攻入云南,永曆帝流亡缅甸首都曼德勒,被缅甸王莽達收留。後吴三桂攻入缅甸,莽達之弟莽白乘机发动政变,杀死其兄後继8月12日,莽白發動咒水之难,杀盡永曆帝侍從近衛,永曆帝最後被吴三桂以弓弦絞死,南明亡。

此時反清勢力只剩夔東十三家軍與在金廈的鄭成功(史稱明鄭)。李自成余部在湖南抗清失敗後,轉移到川、鄂山區進行活動,在夔州府以東地區繼續抗清,稱為夔東十三家軍。1662年清軍開始攻打之,到1664年首領李來亨被殺而亡。鄭成功在南京之戰失敗後退回金廈,於1661年率軍遠征荷蘭人占据的台灣岛成功,明鄭領有台灣,定都東寧(今台灣台南)。其子鄭經曾參與三藩之亂,率軍参与反攻失利。1683年,清朝康熙帝命施琅為水師提督進攻台灣。明鄭主鄭克塽率眾投降,明鄭亡。

明初武功强盛,多次對北元和隨後的韃靼和瓦剌作戰,並在與漠南一帶設置四十餘個衛所防衛,包括東勝衛、雲川衛、官山衛、全寧衛、老哈河衛等,這些都是明廷的邊防重地。其走向大致為陰山-大青山南麓-西拉木倫河一線。15世紀30年代後,由於天氣轉寒,農耕不濟,靖難之役時邊塞軍隊被燕王抽調。因此期間邊境略有南移。在明成祖永樂年間,明軍多次北伐,邊境形勢一度改觀。但在明中葉以後,隨著蒙古的再次崛起,邊境再次南移。並修建長城以防禦蒙古,在長城沿線設置九邊重鎮加強防禦。長城也成為明中後期的北邊,同時也是農耕區與遊牧區的界線。

明太祖設置遼東都司以經營遼東。並多次進軍黑龍江流域,招撫當地土著部落,明廷勢力一度達到外興安嶺與黑龍江口,甚至庫頁島。明成祖永樂七年(1409年)於黑龍江地區設置奴兒干都司,然此都司並非常設機構,與東北130多個衛所不相轄屬,明宣宗宣德九年(1434年)廢棄之,撤回在奴儿干的流官驻军,不过之后女真仍奉明朝为主,原設於此處的各衛所及遼東都司仍然存在,至万历年间卫所增加至384个,以對當地實行羈縻統治。明英宗正統年間後,韃靼兀良哈與建州女真部南遷,並不斷侵犯遼東都司。明憲宗成化五年(1469年),明廷修建遼東邊牆。16世紀末開始,建州女真酋長努爾哈赤開始興起,統一女真部,明廷設置的衛所逐漸消亡。明神宗萬曆四十四年(1616年)努爾哈赤稱汗,建國後金。明神宗萬曆四十七年(1619年)薩爾滸之戰後,後金軍隊破遼東邊牆,佔領遼東都司大部土地。

明成祖永乐年間,西北疆界達到今新疆東部哈密地區,並設置沙州、安定、阿端、曲先、赤斤蒙古、罕东左一系列衛所。15世紀30年代之後,西北吐魯番與青海蒙古部日益強大。1472年,哈密衛城一度被吐魯番攻破,衛內遷,後復,1514年再度被並。16世紀後半期後,西北諸衛全部喪失,明軍退守嘉峪關。

明朝在1381年才將云贵地區完全劃入疆域,並設置一系列土司、宣慰司管轄之,除正式府州外另设有三宣六慰,永乐年间増设底兀刺、大古刺、底马撒三个宣慰司。邊界達到緬甸中北部、老撾北部、泰國北部一線。但後期這些地區多被周邊國家所並。

明成祖永樂四年(1406年)明軍進攻安南,南線達到日南州一帶。次年設置安南布政使司,下設十五府、卅六州、兩百餘縣。後因當地人民反抗激烈,明廷於明宣宗宣德二年(1427年)放棄,安南恢復黎氏王朝。

明初吐蕃宣慰使何锁南普等率吐蕃诸部归降,后明廷在青藏高原地区设乌思藏、朵甘卫指挥使司,采取广行招谕、多封众建、因俗以治的治藏政策。在完成藏区的统一后,明太祖要求藏民输马作赋、承担徭役,或蒸造乌茶、输纳租米,强调“民之有庸,土之有赋,必不可少”。永乐五年(1407年),明成祖派遣刘昭、何铭等人前往藏区设置驿站,永乐十二年(1414年),又遣中官杨三宝往藏区招谕各土官恢复驿站,经多年努力终使往来西番的驿道安全畅通。万历以后,明朝对边疆控制日益松弛,蒙古人攻占了整个青海草原,朵甘都司遂废弃。

1553年葡萄牙人獲得在澳門停泊船隻權,1557年取得居留權,在清光緒十三年(1887年)中葡签署《中葡和好通商条约》前,中国法律上一直拥有澳门主权。

1624年荷兰人进入台湾南部,筑热兰遮城。1626年西班牙人进入台湾北部。1642年荷兰赶走西班牙,占领台湾大部。1661年,郑成功进攻台湾,次年驱逐荷兰人,攻占台湾。

明成祖永乐年间,积极开展对外联系,特别是派遣郑和七下西洋,并积极对南海诸岛进行勘察和经营。多次往返南海诸岛的航行中而次次必登、必书南海诸岛。《郑和航海图》以“石塘”、“石星石塘”、“万生石塘屿”为今之西沙、东沙、中沙和南沙群岛之名。

永乐四年(1406年),郑和船队剿灭盘踞在旧港(今印尼巴邻旁)的海盗陈祖义,在其地设立旧港宣慰司,首任宣慰使施进卿即由郑和亲自前往册封。旧港宣慰司是为明朝最南方疆土,以控制南洋核心要冲地带,也确保了明朝在南洋的权威,令海外贸易大兴,还开启了华人大规模开发南洋的时代。

明朝大致上繼承元朝行政區劃,其一級地方行政區分置承宣布政使司(布政司)、提刑按察使司(按察司)與都指挥使司(都司)的都布按三司制度,分別掌管行政、司法與軍事等三種治權,防止地方權力集中。布政司通稱省,底下依序有道、府州與縣。道是明朝特別設置介於省和府州之間的行政單位,分為分守道和分巡道兩種,分守道为布政司的派出机构,负责监督协调府州行政,分巡道为按察司的派出机构,负责监督协调府州司法治安。府为明朝最主要的统县政区,原為元朝的路,以稅糧多寡為劃分標準,糧廿萬石以上為上府,廿萬以下十萬以上為中府,十萬以下為下府。州与府同样是统县政区,但人口税收比府少,地位也比府低。州按照其行政隶属分为两类,直辖于布政司的州称直隶州,隶属于府的称散州或属州。軍事區劃有衛、所兩级,但部分位于少数民族聚居区或边疆军屯区的卫所具有类似内地州县的行政职能,行政上分别相当于府与县。明代宗、明英宗時設有中央派出管理行政的巡撫與管理軍事的總督,地位在布政司與都司之上。為限制巡撫與總督的權力,又設有都御史制衡之。明朝最後有140府,193州,1138縣,493衛,359所。

承宣布政使司(布政司)主管地方行政,地位等同元朝的行中書省。明太祖原沿襲行中書省的稱呼,1376年時改為布政使司,通稱行省。明初設有十三個布政司與京師(非城市,地位等同布政司,轄現今江蘇與安徽兩省)。1380年胡惟庸案後撤廢中書省,京師及布政司直屬於六部之下。明成祖時期,於1407年到1428年間設置交阯布政司。於1413年設貴州布政司。為遷都北京,1403年將北平布政司升格為行在,1421年遷都北京後稱為京師(北直隸),原京師改稱南京(南直隸),形成「兩京十三省」的行政區劃。两京為明朝首都北京與南京的正式稱呼順天府與應天府,其与其周边州府分别合称北直隶与南直隶,不设布政司,十三布政司为陝西、山西、山東、河南、浙江、江西、湖广、四川、廣東、福建、廣西、貴州、雲南。明朝行政區劃設置大體符合山川形便之處,但仍有一些不合理之處,如南直隸地跨淮北、淮南、江南三個地區,语言文化上属于太湖吴越区的苏松地区归入南直而非传统上的浙江,秦岭以南的汉中等地归入陕西而非传统上的四川,河南也佔據局部的黃河以北土地。貴州省呈現中間窄兩邊寬的蝴蝶狀。

都指挥使司(都司)主管地方軍事,明太祖採用衛所制,於1370年於各省設置一都衛,1375年才設置都司管理。都司原隸屬大都督府,於胡惟庸案後析大都督府為五,分統諸軍司衛所。明朝一共設置十六個都司、五個行都司與兩個留守司。其中十三個是與布政使司同名的的都司,其他三個是萬全都司、大寧都司和遼東都司。五行都司是陝西(治甘州衛,今張掖)、四川(治建昌衛,今西昌)、湖廣(治鄖陽衛,今湖北鄖縣)、福建(治建寧府,今建甌市)、山西(治大同府)。兩留守司是洪武年間設置的中都留守司(今鳳陽)和嘉靖年間置於承天府(今湖北鍾祥)的興都留守司。屬羈縻性質的都司中,最有名的有統轄黑龍江、松花江流域和庫頁島的奴兒干都司,在政教合一的青藏地區設置有烏斯藏、朵甘二都司(但这是否代表當時的西藏受到了明朝的统治存在较大的争议,請參詳明朝治藏歷史),另有置於今甘肅、青海交界地區的哈密、曲先等衛。這些具羈縻性質的行政區劃與內地的都司、行都司性質不同。

巡撫主理民政,原本是明宣宗時期派六部、都察院大臣以此為名義督撫地方行政,到明代宗時正式形成一級行政區。總督於明英宗時設置,分短期與長期兩種,管轄數個布政司的軍務。而巡撫與布政司的轄屬關係不一,有的巡撫轄有有一個到兩個布政司,如正統年間的山西河南巡撫。有的是一個布政司上面有數個巡撫,如北直隸有順天巡撫(駐遵化)、保定巡撫(駐真定,今河北正定)、宣府巡撫(駐宣府鎮,今河北宣化,一度兼領山西大同府)三巡撫;南直隸有兩巡撫:應天巡撫(駐蘇州府,今江蘇蘇州)、鳳陽巡撫(駐淮安府,今江蘇淮安市淮安區)。有的巡撫管轄布政司與布政司之間的交界處,如南贛韶汀巡撫就跨越江西、廣東、福建三個布政司。

洪武十三年(1380年),明太祖以丞相胡惟庸謀反伏誅,於是廢去中書省和丞相一職。秦、漢以降實行一千六百餘年的宰相制度自此廢除,六部直接向皇帝負責,相權與君權合而為一,大權獨攬,施行軍權、行政權、監察權三權分立的國家體制。由於國家事務繁多,皇帝無法處理,而明太祖也一度深感疲憊,於是設立四輔制度來輔佐政事。但這項制度效能不彰。洪武十七年(1384年)後被廢。之後朱元璋請來幾位翰林學士幫忙輔佐,這些翰林學士的官職效仿唐宋馆阁学士旧制,被命為「某某殿(阁)大學士」,官階只有正五品。明成祖登基后,特派解缙、胡广、杨荣等入午门值文渊阁,参预机务,由此始設內閣。

內閣最初只是皇帝的諮詢機構,相當於今日秘書或幕僚的職務,奏章的批答為皇帝的專責。到後來成為明朝實際上最高決策機構,首輔地位有時可比丞相,有票擬之權明朝內閣由始至終都不是明朝中樞的一級行政機構,所謂內閣只是文淵閣的別稱。內閣大學士一職多以碩德宿儒或朝中大臣擔任,只照皇帝的意旨寫出,稱「傳旨當筆」,權力及地位遠遠不及過去的宰相,只有有实无名之地位,而沒有法定地位。宣宗時期,由於楊溥、楊士奇、楊榮等三楊入閣,宣宗批准內閣在奏章上以條旨陳述己見,稱為「票擬」制度,又授予宦官機構司禮監「批紅」。票擬之法補救可君主不願面見閣臣之弊,但內閣大臣與皇帝溝通,全賴司禮監(宦官)。由是開啟明朝宦官專政之大門。

明朝在中央設置吏、戶、禮、工、刑、兵六部,與前代相比,明朝最初在每部增加尚書、侍郎各一。胡惟庸案之後,朱元璋廢丞相之職,取消中書省。六部因此地位得到提高。每部只設一個尚書,兩個侍郎,原有的各科尚書降為郎中。各部尚書和侍郎的官階也上升。其中以禮部(主管教育,負責領導儒家學術,以及祭祀,外交等)和吏部(主管文官陞遷)最為重要,戶部(主管財政,土地和人口)人員最多。兵部(主管國防),刑部(主管司法,有對較大刑事案件的審判權)與工部(主管公共建設)地位較低。

在拟诏审议机构上,明朝開始只設給事中与中书舍人,不复设中书门下二省。明朝的审议机构为六科给事中,到洪武廿四年,設都給事中六人,分吏、戶、禮、工、刑、兵六科,每科一人,每科都给事中下设左右给事中各一人及给事中若干。六科给事中制度基本是繼承唐朝的門下省制度,但官位下降,机构更为精简,也失去了自魏晋以来皇帝内臣(皇室的收发站)和礼官的职责。六科官職品級雖低,然職權很高,他們可以批驳皇帝的意旨, 也能充当谏官的职责,对六部吏僚则具有分科对应的监察权,故該制度也發揮一定的改善朝政作用。明朝的拟诏机构为中书舍人官署,因其制度源流源于与门下并立的中书,故与六科相对俗称 「中书科」,但是其地位大为下降,职能也大幅削弱,事实上只是内阁与翰林院的誊抄机构。中央的重要事务执行机构为五寺,包括大理寺、太常寺、光祿寺、太僕寺、鴻臚寺,与唐宋相比,减省了四寺:宗正寺被并入宗人府,卫尉寺被并入兵部,司农寺与太府寺被并入户部。大理寺與刑部和都察院合為三法司,负责重大刑事案件的复审与复核。大理寺的首長稱為大理寺卿,也是九卿之一。其餘四個寺的卿職權較低。太常寺負責祭祀;太僕寺管理馬匹与全国牧政;光祿寺負責壽宴;鴻臚寺負責接待外賓。

在洪武十三年前,明朝還沿襲元的監察制度,設立御史台,有左右御史大夫各一名。洪武十三年後,朱元璋廢御史台。兩年之後,朱元璋設立新的監察機構—都察院。都察院下面設立監察御史若干人,分巡全國各省,稱為十二道監察御史。每道有監察御史三至五人,範圍大體為一省。但監察御史都駐在京師,有事帶印出巡,事畢回京繳印。到明末,監察御史分為十三道,共有一百一十人。都察院与六科同样具有谏官的职能和风闻言事的职责,故合称「科道言官」。

明初还實行特務機構,主要包括錦衣衛、東廠和西廠,武宗時期還一度設有內行廠。錦衣衛設立於洪武十五年,直接聽命於皇上,可以逮捕任何人,並進行不公開的審訊。但是朱元璋晚年逐步废除了锦衣卫及其特权,还有一些比较残酷的刑法。

在東廠設立後,錦衣衛權力受到削弱。東廠成立於永樂十八年,是明成祖為鎮壓政治上的反對力量而成立。地點位於京師東安門北。東廠的主要職責就是監視政府官員、社會名流、學者等各種政治力量,並有權將監視結果直接向皇帝匯報。依據監視得到的情報,對於那些地位較低的政治反對派,東廠可以直接逮捕、審訊;而對於擔任政府高級官員或者有皇室貴族身份的反對派,東廠在得到皇帝的授權後也能夠對其執行逮捕、審訊。東廠在設立之初,就由宦官擔任提督,後來通常以司禮監秉筆太監中位居第二、第三者擔任。西廠設立於憲宗時期,首領為汪直。1482年後被廢。其後又被武宗短暫恢復。內廠設置於武宗時期,首領為宦官劉瑾,劉瑾伏誅後,內廠與西廠同時被廢除,僅留東廠。

公孤官包括三公与三孤,是名义上的诸臣之首,但這些官職都是虛銜,一般授予功勞相當大的大臣以示榮耀。三公为太师、太傅、太保,三孤则是辅弼他们的少师、少傅、少保。其中太保和太傅名義上是太子的老師,而太師則是皇帝名義上的老師,但實際上輔導太子的機構是詹事府。詹事府下設兩坊、一局、一廳。此外還有太醫院,專門負責皇室人員的健康和醫療。太醫院附屬有生藥庫和惠民藥局。翰林院作為政府的官方學術最高機構,地位相當重要,甚至在政府中都有相當大的影響力。翰林院首長是翰林大學士,此職位者經常會同時兼任內閣大臣。

诸司指不屬於各部院的司。主要指通政司和行人司。通政司負責傳遞公文,公告周知。行人司負責到地方上頒詔諭及赴外國作使臣。

外三监包括國子監、欽天監、上林苑監。欽天監負責觀測星象。國子監是最高官方教育機構,也是全国官学的领导机构,有祭酒一人,司業一人,監丞一人,博士五人,助教十五人,學正十人,學錄七人,典簿一人,典籍一人,典饌兩人。上林苑監負責掌管皇帝的御花園,畜牧場與菜圃。

内十二监為宦官衙門。事實上只有在這些衙門工作的宦官才是太監。包括司禮監、內宮監、御用監、司設監、御馬監、神宮監、尚膳監、尚寶監、印綬監、直殿監、尚衣監、都知監。以司禮監最為重要,監內的提督太監主管宮內一切宦官禮儀刑名。而秉筆太監在宦官極端專權時竟代替皇帝批公文。此外宫内還設有四個司(惜薪、鐘鼓、寶鈔、混堂),八個局(兵仗、銀作、浣衣、巾帽、針工、內織染、酒醋面,司苑),合為內官廿四衙門。宮女也有六個局(尚宮、尚儀、尚服、尚食、尚寢、尚工),每個局下設四個司。

《大明律》,是明朝法令条例,由朱元璋总结历代法律施行的经验和教训制定而成,《大明律》为适应形势的发展,变通了体例,调整了刑名,肯定了明初人身地位的变化,注重了经济立法,在体例上表现了各部门法的相对独立性,并扩大了民法的范围,同时在“礼”与“法”的结合。

《大明律》共分30卷,篇目有名例一卷,包括五刑(笞、杖、徒、流、死)、十恶(谋反、谋大逆、谋叛、恶逆、不道、大不敬、不孝、不睦、不义、内乱)、八议(议亲、议故、议功、议贤、议能、议勤、议贵、议宾),以及吏律二卷、户律七卷、礼律二卷、兵律五卷、刑律十一卷、工律二卷,共460条。

有明一代比较重视法制的建设与实践,其中历经三次大规模修订的《大明律》。《大明律》在中国古代法典编纂史上具有革故鼎新的意义。不仅继承了明代以前的中国古代法律制定的优良传统,也是中国明代以前各个朝代法典文献编纂的历史总结,而且还开启了清代乃至近代中国立法活动的发展。《大明律》在明代实施的过程中,虽然也不断受到“朕言即法”的干扰,但这些干扰始终未能影响它的正统法典的地位。

而《大明律》对惩治贪财枉法者,严厉程度超过了历史上任何一个朝代。

明代早期軍隊的來源,有諸將原有之兵,即所謂從征,有元兵及群雄兵歸附的,有獲罪而謫發的,而最主要的來源則是籍選,亦即垛集軍,是由戶籍中抽丁而來。除此之外尚有簡拔、投充及收集等方式。此外,明朝中期以後又有強使民為軍的方式,不過都屬於少數,整體而言,衛所制仍然是最主要的軍制。衛所制為在全國各地軍事要地設立衛所駐軍,衛有軍隊五千六百人,其下依序有千戶所、百戶所、總旗及小旗等單位,各衛所都隸屬於五軍都督府,亦隸屬於兵部,有事從徵調發,無事則還歸衛所。軍隊來源為世襲的軍戶,由每戶派一人為正丁至衛所當兵,軍人在衛所中輪流戊守以及屯田,屯田所得以供給軍隊及將官等所需。其目標在養兵而不耗國家財力,但明宣宗以後漸無法維持,軍人生活水準及社會地位日漸低下,逃兵也逐漸增加,軍備因此逐漸廢馳。

在嘉靖年間,應付倭寇之亂時,將領戚繼光在浙江地區採用招募民兵加以訓練的方式,來取代不堪的衛所兵。正因為明朝正規軍衛所軍的不堪用,故這些民兵,在明朝後期逐漸擔負起維持明朝有效統治的作戰部隊,而其中最為有名的就是戚繼光的召募以浙江人為主戚家軍,李如松的私人部隊遼東鐵騎,及袁崇煥所召募以遼東人為主的關寧鐵騎。

发端于唐宋时期的中国火器制造技术,在明朝发展到了很高的水平。这时的火器不仅仅种类多,而且制造技术以及性能均有极大提高。火箭与鸟枪是明朝军队的主要轻型火器,地雷在明朝也很盛行,管形火器的发展尤为显著。明朝中后期,随着经济的发展、科技的进步以及国防需要的强化,火器技术得到迅速发展。火器技术的勃兴引发了一场火药时代的军事变革。佛郎机以及红夷大炮等西洋火器在此时期传入,使得明朝得以汲取其瞄准器的长处,以改良自产的火器性能。当时中国的冷兵器时代即将终结,火器时代正在来到,亦認為中国有机会赶上西方的火器技术水平,但这一过程却随着明朝的灭亡而中断。

學者梁柏力指出,中國雖然比西方早兩個世紀使用熱兵器,但到了15世紀技術開始被葡萄牙人超越,但是差距还不是很大,後來清軍利用了明朝和西方的技術和經驗,多次改良並製造出比明朝更有威力的火器,到了三藩之亂期間,中國的熱兵器技術回升接近西歐國家的水平,这也是郑成功能驱逐台湾荷兰人,以及清康熙時期的清軍能夠擊退入侵黑龍江的俄羅斯士兵的原因之一。在簽署尼布楚條約其後的150多年內,清朝境內大致升平,直到鸦片战争前夕,还停留在三藩之乱的技術水平上。可見中國火器的技術發展,與國內是否長時期出現大規模軍事對峙的局面有關。

在萬曆年間,日本人亦在火器技術上領先中國,以致日本火器的優勢在萬曆援朝戰爭中一度令日軍佔於上風。明朝軍事家戚繼光亦批評當時多種形式的火器實際上並不實用,故一切禁之,以節靡費。亦有學者批評宋元明清年代在政權穩定期間往往封鎖火器的研究成果,並且對研制者新的發明創造也不予以重視,甚至棄置不用,如明朝的趙士楨、畢懋康、薄玉和清朝的戴梓,他們的貢獻和成果都沒被恰當重視。

16、17世紀間,明代曾是世界上手工業與經濟最繁榮的國家之一。明代初期推行的海禁政策,使得商業受到一定的壓制,但明穆宗隆慶元年(1567年)廢除海禁後,海外貿易重新活躍起來,全盛時遠洋船舶噸位高達18000噸,占當時世界總量的18\%,推进了中国与国际市场的联系,促使晚明中国白银货币化的最终完成。明代手工业和商品经济繁荣,出现商业集镇,中国大陆学界认为出现「资本主义萌芽」,此说法仍然处于争议之中。

明朝初期,由於多年的戰爭加上通貨膨脹,且前朝元惠宗為治水加重徭役,經濟近乎在崩潰的邊緣。明太祖洪武年間實行休養生息的政策與移民墾荒,也實行屯田政策,軍屯面積佔全國耕地的近十分之一。此外,商屯也相當盛行,政府以買賣食鹽的專賣證(稱之為鹽引)作為交換,利用商人將糧食運往邊疆,以確保邊防的糧食需求,然而此方式並非以以物易物方式,而是要求鹽商先交錢再等曬鹽季再給鹽,卻又為稅收不足而將新產出的鹽另行外賣,延後交鹽給正規鹽商的時間,致使鹽商交了錢卻要三五年甚至十年後才拿得到鹽,卻又因身份管制而無法拋棄鹽商身份另行謀生,因此而家破人亡,私鹽亦大為流行。

明朝農業無論是產量還是生產工具,都高於前一朝代,番薯、南瓜、蠶豆、土豆、玉米、棉花等美洲高產作物在16世紀中葉時陸續傳到中國,尤其是棉花,已在全國普遍栽種。此外,較容易栽種的蕃薯和玉米,可以種植於土壤相對較貧瘠的地區,對於糧食需求日增的明清兩代尤其重要。

萬曆年間,耕地總面積超過七百萬頃,為明神宗萬曆年間開始的人口穩步增長提供堅實的基礎。而在南宋時流行的俗諺「蘇常熟,天下足」,由於長江下游地區城市居民的快速增加,及長江中游地區的快速開發,中晚明時,已經轉變為「湖廣熟,天下足」,意即當時主要的米糧生產區已經轉移到湖廣地區,也就是現在的湖北省和湖南省一帶。

隨着商業性農業的出現而發展起来的長途交通,有利於工商業的發展。晚明以後,湖廣的米開始被長途運送至江浙、閩廣等地區販售,使當地農民開始改種經濟作物。明太祖也曾派遣國子監下鄉督導水利建設,並以減免稅賦獎勵耕作。這些措施使得過去很多飽受戰亂損毀的地區恢復生氣,使明朝的經濟得到快速的恢復。

明朝无论是铁、造船、建筑,还是丝绸、纺织、瓷器、印刷等方面,在世界都是遥遥领先,产量占全世界的2/3以上,比农业产量在全世界的比例还要高得多。明朝民间的手工业不断壮大,而官营却不断萎缩,明朝后期,除了盐业等少数几个行业还在实行以商人为主体的盐引制外,一些手工业都摆脱了官府的控制,成为民间手工业。

自明初年起,以江南地區為代表的手工業高度發展,松江潞安府全盛時有織機一萬三千張,促進市場經濟化和城市化,南京、臨清等城市「周圍逾三十里,而一城之中,無論南北財貨,即紳士商民近百萬口」。南京一地有眾多的陶瓷廠,每年可生產100萬件瓷器。景德鎮成為世界瓷都。制瓷使用旋坯车,不但提高生产效率,還使旋出的瓷坯更为精细和规格化。施釉方式以吹釉法代替刷釉法,使施釉更加均匀光泽。並且發展出彩色瓷器。冶铁技术也有明显的提高,由灌钢冶炼法发展到苏钢冶炼法,是一种效率较高的炼钢方法。

明初期奉行「重本抑末」政策,甚至規定禁止商賈之家穿綢紗。明穆宗隆慶三年(1569年),大學士高拱上疏《議處商人錢法以蘇京邑民困疏》,反映商人的愁苦和商業的窘困,並奏請隆慶帝採取措施,革除宿弊。之後張居正提出農商榮枯相因,進一步肯定商人的作用。明代中後期商人地位有所提高,部分士大夫認為經商有成,在價值上也等同於讀書有得,「亦賈亦儒」「棄儒就賈」的現象也開始出現。此外,商業用的書也開始出現。商人為實用目的而編寫此類書籍,內容介紹貿易路徑沿途的交通、習俗及商品行情等。此類書籍現存最早者為《一統路程圖記》。此外,由於商業的發達,各地紛紛開始大量生產具有當地特色的商品,運銷他處,使得區域分工日益明顯。

明朝初期,明太祖洪武年間尝试使用「大明寶鈔」的紙幣,这种货币同样经历了迅速的通货膨胀,它在1450年暂停发行,但是直到1573年仍在流通。直到明朝晚期李自成威胁北京时,这种纸币才在1643年和1644年重新印刷。在明朝大部分时期,中国有一个包括所有重要交易的纯私人货币体系。而整個貨幣體系轉向為以銀本位為主。从海外流入的白银, 开始在南部省广东作为货币使用,并在1423年传到长江下游地区成为纳税的法定货币。各省税收自1465年起以白银的形式上交首都,灶户从1475年起开始使用白银支付,徭役豁免费从1485年起使用白银支付。中国对白银的需求部分通过西班牙人从美洲的进口得到满足,特别是秘鲁的波托西和墨西哥,在西班牙人1571年建立马尼拉之后。但这时的白银还没有被铸造。它们以重量为一个标准两(约36克)的银锭(被称为元宝)流通,尽管其纯度和重量在地区与地区间略有不同。

16世紀中葉之後日本和拉丁美洲的白銀大量流入也進一步促進中晚明經濟的發展,當時明朝佔有世界白銀需求量三成左右。明代經濟的另一個特色是城鎮經濟的繁榮,運河沿線由於往來商船不斷,周邊城市如濟寧、淮安、揚州等都非常發達。東南地區由於商品經濟繁榮,成為全國的經濟集散地。由於商品經濟的繁榮,明代形成按籍貫區分的商人集團,稱為「商幫」,如徽州商幫、晉陝商幫、廣東商幫、福建商幫、蘇州洞庭商幫、江西商幫等。這些商幫以「會館」為聯繫場所,互相支持,越做越大。

明嘉靖、萬曆間,各地出賣絲綢、酒肉、蔬果、煙草、農作物、瓷器等商品不計其數,大量外銷賺取外匯所得;外國的不少東西在中國城市都有賣,如歐洲的西洋鐘、美洲的煙草,當時商業大都會以江南的商業城市最多,有南京、儀征、揚州、瓜洲、蘇州、松江、杭州與嘉兴等,華中其他商業城市尚有汉口、南昌、淮安、蕪湖與景德鎮等,西南内陆有成都,華北有北京、濟寧與臨清等,而華南則有福州與廣州等。

晚明至清朝这一时期,明朝生產總量所占的世界比例在中国三千年历史上也是最高的。据英国经济学家安格斯·麦迪森的研究,1600年明朝生產總量为960亿美元,占世界经济总量的29.2\%,晚明中国人均GDP在600美元。事实上据他研究,1500年中国生產總量(618亿美元)已首次超过印度(605亿美元)。这里仅表明购买力平价,与所谓财政收入 (Government revenue) 是不同的概念,大多数中国学者如刘逖认为麦迪森已经高估了中国历史上的生產總量和人均生產總量。据刘逖指出国际公认的生存水平线是400美元,因此刘逖对麦迪森明朝数据做了调整,认为若以1990年的美元價值換算,1600年中国人均GDP在388美元、1610年在386美元、1620年在391美元、1630年在344美元、1640年在367美元,而非麦迪森说的一直维持在600美元。

而隆庆年间的开关,进一步促进了当时中国经济的增长。

元惠宗至正年間(1341年-1370年)全國發生多次大規模的災荒饑饉疾病和瘟疫,並最終促使紅巾軍起義爆發,期間造成人口大量減少。大明建立並統一全國後,明太祖實行休養生息政策,全國的農業生產在蒙元时代長期大規模战争而遭受極大破壞的背景下得到很大程度的恢復,加上洪武年間大規模向淮河以北和四川的荒無之地、墾荒填充移民,使人口得以穩定增長。到明太祖洪武廿六年(1393年)全國有6500萬人,其中民戶佔6175萬人,軍戶佔325萬人。北五省(北平、山西,山東、河南、陝西)人口有1755萬人,佔全國27\%,其中山東最多,有5,462,850人,以下依次為山西(3,790,760人)、河南(2,825,300人)、陝西(2,646,450人)、北平(2,619,500人)。中五省(京師、浙江、江西、湖廣、四川)人口總數為3380萬,佔全國52\%。其中,南直隸有11,291,460人;人口密度最高的蘇南太湖流域人口達6,320,300人,平均每平方公里220人;其次為浙江省,有9,959,270人;江西有7,260,000人,湖廣有4,318,420人,四川最少,僅1,314,260人。南五省(福建、廣東、廣西、雲南、貴州)總人口有1040萬人,佔全國的16\%。

明朝户口的峰值出現在明朝后期,但對於具体时间與人口數量,不同學者有不同說法。易中天认为,明末人口六千余万;赵文林、谢淑君认为1626年明朝达到人口峰值,实际人口大约有99,873,000人;王育民认为万历年间明朝人口达到峰值,实际人口在130,000,000人至150,000,000人之间;葛剑雄认为1600年明朝实际人口大约有197,000,000人,明朝人口峰值接近2亿;曹树基认为1630年明朝达到人口峰值,实际人口大约有192,510,000人,1644年实际人口大约有152,470,000人;而英国经济学家安格斯·麦迪森则认为1600年明朝实际人口大约有160,000,000人。

根據南开大学王泉伟在《明代男女比例的统计分析——根据地方志数据的分析》一文當中的研究,明朝的性別比相當不平衡,明朝中期後,全國範圍內的性別比曾一度達到每100個女性中有150個男性的狀況,有些地區甚至出現每一百個女性中有大約300個男性的狀況;復旦大學中國歷史地理研究所的曹樹基也曾在《墓誌銘中所見明代人口結構》一文中提及明代性別比失衡、出現男性遠多於女性的狀況。

明世宗嘉靖末年美洲高產作物傳入後開始在明代人口最為稠密的江浙和嶺南地區普及和推廣,尤其是經過萬曆中興過後以較快速度穩定成長,到明神宗万历四十八年(1620年)根據當代學者研究估計達到前所未有的150,000,000人,分佈格局基本未變。明思宗崇禎十三年(1640年)到清世祖順治七年(1650年),由於农民战争、饥荒和瘟疫等造成中原地區死亡加大,特別是由於北方鼠疫和旱灾的爆發、以及八旗入關掠殺和為防範漢人而進行有計劃的遷移,造成人口大量減少,只有原先人口總數一半不到,特別是經歷鼠疫大爆發的北方,人口降到不足20\%。

明代沿襲元代,將人戶分為民戶、軍戶、匠戶三等。手工業者為匠籍。也就是規定全國技術好的手工業工人必須於官營手工業部門服務的制度。匠籍、軍籍比一般民戶地位低,不得應試,並要世代承襲。若想脫離原戶籍極為困難,需經皇帝特旨批准方可。

明代定以前的匠戶為匠籍,並規定這些入匠籍的手工、工人子孫世代承襲,不得脫籍改行,但不同點在於明代時,他們不需永遠在王朝服役,而只要依規定每隔幾年輪班到京城服役一次即可,稱之為輪班匠。輪班匠的勞動是無償的,还由於到京城的路途遙遠,輪班匠仍然常常發生逃役的狀況,於是在明宪宗成化二十一年(1485年),朝廷便下令輪班匠可繳交銀兩折抵役期,稱為「匠班銀」。嘉靖四十一年(1562年)起,朝廷進一步改革匠役制度,輪班匠一律征銀,以銀代役,政府則以銀雇工。人身束縛大為削弱。但仍有部分工人留在官營手工業單位服務,匠籍制並未完全廢除。

明初为了恢复生产和发展经济,政府有组织的把山西一带的民众迁到中原等人口较少的地区,史称洪洞树大移民。17世纪开始的全国性大旱灾直接导致了全国性的大蝗灾。也引发了波及差不多整个华北地区的鼠疫。人口大量死亡,灾民大量离乡。但因明末的动乱很快结束,而灾民除死亡外,不久也回到了原籍,并未形成大规模的移民。

明代时期的教育发达,学校兴盛,唐宋所不及。明朝初期實行「科舉必由學校」的政策,明太祖多次強調:「古昔帝王育人材,正風俗,莫不先於學校。」並將學校列為「郡邑六事之首」,以官學結合科舉制度推行程朱理學,而不重視書院,書院因此沉寂近百年之久。也因此,明朝中早期最重要的教育機構也就是國子監。而各府、州、縣政府也皆立學校。府、州、縣學的學生稱為生員,俗稱秀才或相公。明初生員數目有定額,大致府學四十人,州學卅人,縣學廿人。明代中後期,地方官員六事皆舉者極少,「學校之政之修也久矣」,因此傳統書院再次承擔起培養科舉人才的重任。明代書院的創辦,以嘉靖年間為最多,據統計,明代書院共有1239所。書院的經費來源,大體上可以區分為官方撥置、和私人捐贈。由於政治上的牽連,書院屢遭劫難,歷史上共有四次禁毀書院的記載,但官方越是禁止,民間開辦的書院就越多。

科舉在明朝是正式的選拔官吏制度。科舉考試分為兩級,每三年舉行一次,稱為大比。科舉考試的內容主要是四書五經,考生必須用八股文做答。所謂股,即對偶之意。八股文萌芽於宋朝,形成於明成化以後。由於八股取士的制度,讀書人既不通經史,又不諳實際,嚴重束縛民眾智慧的進步。

文學方面,中國小說史上的四大名著中的《西遊記》、《水滸傳》、《三國演義》(原本是《金瓶梅》後被三國代替)就是出於明朝。馮夢龍加工編輯的三部白話短篇小說集「三言」(即《喻世明言》《警世通言》《醒世恆言》)每部四十篇,共一百二十篇,主要是描寫青年愛情故事以及平民市井生活,最著名的如《杜十娘怒沉百寶箱》、《金玉奴棒打薄情郎》、《轉運漢巧遇洞庭紅》等;與「三言」類似每部四十篇的短篇小說集還有凌蒙初編著的「二拍」以及1987年才被發現的《型世言》(陸人龍編著)。傳統雅文學的發展在明代繼續發展,著名文人有劉基、宋濂、高啟、方孝孺、唐寅、歸有光、徐渭、王世貞、袁宏道、錢謙益、張岱、吳偉業等人。散曲家則有王磐、馮維敏、薛論道、陳譯、康海等人。

萬曆時期,猛烈反對前後七子的擬古主義,有以公安袁宗道、袁宏道與袁中道為代表的公安派。他們認為文學是隨著時代的變化而變化的,有各個不同的時代,即有各種不同的文學。竟陵鍾惺、譚元春為代表的竟陵派主張獨抒性靈,並且乞靈於古人,目的為“引古人之精神以接後人之心目,使其心目有所止焉,如是而已矣”。

明朝時期,傳統雜劇逐漸衰落,而傳奇劇走向繁榮。在嘉靖後期到萬曆初期出現三部優秀的傳奇作品,即《寶劍記》、《浣紗記》及《鳴鳳記》。明代戲劇的集大成者是湯顯祖。他的代表作是臨川四夢(即《南柯記》、《邯鄲夢》、《紫釵記》及《牡丹亭·還魂記》)。南戲在明朝也進入最繁盛的時期。明朝的文學與戲劇在對「情慾」的描寫上是較為開放的,例如《牡丹亭》一劇中就充滿許多對少女情懷的正面刻寫。

明朝朝廷極力推崇書法,明朝書法以行書和草書為主。明初書法陷於台閣體泥沼,沈度學粲兄弟推波助瀾將工穩的小楷推向極致,“凡金版玉冊,用之朝廷,藏秘府,頒屬國,必命之書“。二沈書法被推為科舉楷則,於是台閣體盛行。明中期吳中四家崛起,書法開始朝尚態方向發展。祝允明、文征明、王寵與唐寅是這個時期的代表,書法開始邁入倡導個性化的新境域。晚明書壇興起一股批判思潮,書法上追求大尺幅,震盪的視覺效果,有名的有張瑞圖、黃道周、王鐸與倪元瑞等,而帖學殿軍董其昌仍堅持傳統立場。

明代的繪畫成就巨大,大致偏重於文人畫派,往上承襲唐、宋、元三代的體系,再經過充分發揮後而自成一家的。明代画风迭变,画派繁兴。在绘画的门类、题材方面,传统的人物画、山水画、花鳥画盛行,文人墨戏画的梅、兰、竹及杂画等也相当发达。其中最興盛的山水畫派可分為氣勢恢弘的浙派、蒼勁活潑的院派與清麗縝密的吳派三種。著名的書畫家如擅長花鳥的徐渭、擅長人物畫的陳洪綬,「明四家」沈周、文征明、唐寅和仇英,山水畫大師董其昌。明朝繪畫以山水和花鳥為主。人物畫和社會風俗畫相對較弱。明朝的雕像較多為城隍、孔子、關公、岳飛等為主。

明代晚期由于传教士纷纷来华,西方近代绘画也传入中国,开始了东西方艺术的第一次正面交流。但由于东西方审美观的差异,西方艺术的影响主要体现在西洋版画艺术方面,尤其是坤舆图、西方原版图书以及圣迹画对明代晚期的绘画产生了重要影响。

明代的建築工藝創下新成就。南京和北京城池都是偉大的建築作品。应天府京城城墙營建於洪武二年,完成於洪武十九年,城牆周長達66里,一般寬10-18米,高12-15米,是世界上最長的城牆。南京城突破方正的格局,而是按照地理形勢修建。皇城位於東部,市肆和居民區位於南部,西北則是軍營。洪武二十三年起,明政府開始修建京师外郭城(即南京外城),周圍120里,開十六門,將雨花台和鍾山都包入其中。而北京城池則較為方正,體現皇權至上的思想。明朝的宮殿建築也十分宏偉,故宮即為例證。明朝各種歷制建築也十分嚴謹工整。天壇、太廟、社稷壇、孔廟都是十分巍峨莊嚴的建築。明代帝陵工程浩大,可謂歷代之最。而在明代被重建的萬里長城(明長城)更是舉世無雙的巨作,保衛明朝的邊疆,至今依然聳立。

明朝的興起與元末信奉明教與白蓮教的紅巾軍息息相關,所以明太祖建立明朝後對宗教採取抑制和利用兼並的政策。他主要希望阻斷摩尼教、白蓮教與彌勒宗等宗教組織再度變成反朝廷的起事軍,並且希望利用佛教、道教等宗教的力量來維護社會秩序。結果,得到「皇糧」全面保障的佛教與道教演變成缺乏精神上的創新追求,亦脫離廣大信眾,民眾轉而尋求民間宗教作為慰藉。

明朝流行對不同宗教兼容並取傾向,民間宗教性信仰、習俗多樣而活躍。基本精神在信仰自由主義、保持國家政治世俗性質、維持社會穩定和國家對社會的控制。集中體現這些政策精神的仍是儒家政治社會理念並倚賴士大夫群體的努力。其變動因素和矛盾來源,則在諸教向國家政權機關的滲透、皇室特殊化行為、民間泛神論多元信仰傾向、部分士大夫的信仰綜合主義。在此期間,回族的形成與猶太教的消亡,表現出作為外因的社會環境與作為內在動力的宗教本土化、世俗化運動,對宗教發展有著至關重要的影響。

明朝中期以後,佛教受皇室宗教活動加強的刺激與儒家的矛盾尖銳起來。這種矛盾促使部分士大夫強烈反對寺院修建並發表闢佛言論。明朝政府對藏傳佛教政策與對漢傳佛教政策有同有異。其重要差異之一是,明朝對藏傳佛教政策與對西部邊疆政策緊密相關,而對漢地佛教政策則於周邊關係政策基本無關。此外,部分士大夫以藏傳佛教為「番教」,認同程度遜於內地佛教。明朝一些皇帝因喇嘛多擅長某些「法術」,對其有特殊興趣,並因而導致士大夫針對相關政策的批評。道教起源於本土民間信仰,在明代與儒家士大夫的衝突比較和緩。但明朝君主中信奉道教者多,既影響到國家政治,也影響到士大夫與君主的關係。士大夫在反覆重申儒家原旨的同時,對道教的批評也日趨尖銳。民間宗教以最貼近下層百姓生活的組織形式和內容,滿足中下層民眾的宗教需求,甚至部分地滿足他們的政治要求和經濟要求。這是明朝中葉之後,民間宗教如火如荼發展起來的重要原因。明朝政府將民間宗教基本看作民俗,一般無干預,對視為「陋俗」者加以排斥,在涉及秘密社會活動時則嚴厲禁止。

明朝還是信仰伊斯蘭教諸民族、藏傳佛教黃教的形成和發展,以及天主教在中國傳播的重要時期。伊斯蘭教在社會生活中相對封閉,在明代政策中大體上表現為一個民族政策問題而不是一個宗教問題,基本與國家以及其他社會成分相安無事。明朝中期以後,天主教再度傳入中國,當時士大夫尋求改革,明朝對天主教大致寬容。

哲學思想上,王陽明繼承陸九淵的「心學」並發揚光大,他的思想強調「致良知」及「知行合一」,並且肯定人的主體性地位,將「人」的主動性放在學說的重心。而王陽明的弟子王艮更進一部的強化此方面的論述,提出「百姓日用即道」,肯定平民百姓日常生活的意義。而李贄則更肯定「人欲」的價值,認為人的道德觀念系源自於對日常生活的需求,表現追求個體價值的思想。而隨著西學的傳入,科學精神與實學風尚也開始流行。明末之際,伴隨著朝代的更替與異族的入主,哲學家開始更多的思考現實問題與政治改良,如王船山、黃梨洲、顧亭林等。

而明代晚期書院的興盛,衝擊官學的地位。許多知識分子利用在書院講學之際藉機批評時政,例如曾講學於東林書院的顧憲成及高攀龍,就常諷刺時政,也使東林書院成為與當權派對抗的中心,進而造成東林黨爭。當時學者也會借用寺廟周邊的空地舉行「講會」,倡導新的思想價值與人生觀。

以顧炎武、黃宗羲、王夫之並為明末清初三大儒。顧炎武提倡「經學即理學」,提出以「實學」代替宋明理學,要學者直接研習六經。提倡“天下興亡,匹夫有責”,著有《日知錄》、《音學五書》等。黃宗羲有「中國思想啟蒙之父」之譽稱,著有《明儒學案》、《宋元學案》,是中國學術史之祖。他保護陽明學,排斥宋明理學,力主誠意慎獨之說,蔚為浙東學派。王夫之強調實際行動是知識的基礎,認為歷史發展具有規律性,是「理勢相成」。其思想發展成船山學,後人編為《船山遺書》。

以民為天下之主的思想於明末清初亦有所流行,例如生活在明末又经历清初时期的黃宗羲和顧炎武、王夫之提倡民權,所著的《明夷待訪錄》攻擊君主專制體制,提倡天下為主,君為客的觀點,倍受清末革命黨的推崇。部分學者認爲黃宗羲的思想是近代民主主義的思想,有西方學者稱黃宗羲為「中國自由主義的先驅」。

西歐進入大航海時代後,葡萄牙意圖在中國建立貿易據點。1513年,葡萄牙國王曼努埃爾一世為想要與明廷通商,派出使節團前往中國。使節團本來想在廣州登陸,但被拒絕入境。他們改以武力佔據屯門,與明朝爆發屯門海戰、西草灣之戰,結果戰敗。最後明世宗嘉靖皇帝同意入境,並且讓葡萄牙人在澳門開設商行,允許他們每年來廣州「越冬」。其後西班牙、荷蘭、英國等國相繼派使團東來,使得不少西洋事物傳入中國。1582年,天主教傳教士利瑪竇奉命前往中國教區工作。利瑪竇很快學會中文,並穿儒服、通儒書,頗得明朝士大夫好感。後來他被舉薦到北京,頗得明神宗信任。他向中國進獻聖母瑪莉亞像、十字架、坤輿萬國全圖、西洋自鳴鐘、西洋大炮、西洋式望遠鏡、西洋式火槍、西藥等貢品,先後在北京、肇慶等地展出。

學者李正焕認為明朝是中国科技发展的极其重要的时期,涌现出一大批集大成的科学家和许多不朽的科技名著,而金觀濤、樊洪業、劉青峰則認為自宋元兩代以後,中國的科學發展日益趨於停滯狀態。煉鐵量是用來評估國家產力的重要指數。在宋朝,中國每年的煉鐵量總和相當於18世紀的全歐州總煉鐵量,在明朝,許多煉鐵廠被荒廢。

明朝初期的大統歷一直沿用元代授時歷,不准民间研究,下詣「国初学天文有厉禁,習曆者遺戍,造曆者誅死」,但天文导航、冶炼钢铁、商业数学等实用科技仍有许多重要成就。到了後期禁令被放寬後有學者編寫了一部天文著作,可是無人問津、不被重視且「未曾用之」,《大明律》規定:“造讖諱、妖書妖言及傳用惑眾者,皆斬。若私有妖書隱藏不送官者,杖一百,徒三年”,嚴厲處置撰寫、刊行、銷售或使用“妖書”的人。被送官燒毀的「妖書」名目有《換天圖》、《飛天歷》、《聚寶經》、《太上玄元鏡》等共計88種。明朝的大統歷是承襲元朝的授時歷,對日月蝕的預報早已不准,明朝開國一百多年後陸續有人建議改曆,被禮部以“古法未可輕變,請仍舊法”和“祖制不可變”的理由反對。明代欽天監的天文官們已無人能掌握元代郭守敬等製訂授時歷時所依據的原理和方法。利瑪竇憑藉西洋書本上的知識即可預測日月蝕,而欽天監的官員們卻一籌莫展。當徐光啟、李之藻等人打算用西法改曆而發動宣傳攻勢時亦引起了守舊勢力的反感。後來由於士大夫攻擊傳教活動,並謂私習天文為違反大明律,政府下令嚴禁,並將所有耶穌會士逐往澳門。

明太祖亦禁止人民進行科學研究,且鄙薄科學技術,認為皆是「无益」之物並加以毀壞:“明太祖平元,司天监进水晶刻漏,中设二木偶人,能按时自击钲鼓。太祖以其无益而碎之”。

作明中晚期學術著作眾多,如李時珍的《本草綱目》、宋應星的《天工開物》、徐光啟的《農政全書》、方以智的《物理小識》、程大位的《算法統宗》、吳有性的《瘟疫論》、徐霞客的《徐霞客遊記》,這些科學家幾乎都是明朝有功名的士子。1637年宋应星在《论气·气声》中对声音的产生和传播作出合乎科学的解释,认为声音是由于物体振动或急速运动冲击空气而产生的,并通过空气传播,同水波相类似。方以智在《物理小识》中提出:“宙(时间)轮于宇(空间),则宇中有宙,宙中有宇。”提出时间和空间不能彼此独立存在的时空观。在《物理小识》中正确地解释蒙气差(即大气折射)现象。民间光学仪器制造家孙云球制造放大镜、显微镜等几十种光学仪器,并著《镜史》。

明朝宗室在技術上也有贡献,朱載堉在世界上第一次正確地提出十二平均律,並在數學、天文學方面亦多有建樹;明初周王朱橚把四百余种植物种于府内,并让王府画工将植物绘图编制成书,名为《救荒本草》。《救荒本草》共记有植物414种,并详细描述各种植物的形态、产地、生境、可食用部位和食用方法,是生物学历史上的重要书籍。中晚明的軍事科技也有所進步,各種新式火器大量湧現,但也被當時的军事家批評不實用。西方傳入的佛郎機火炮和紅夷大炮都在中國製造和使用。還有一些專門的火器論著出現,如茅元儀所著之《武備志》。

明朝末期,隨著耶穌會傳教士和西學的傳入,中晚明的科學技術出現新的進步。在他們傳播教義的同時,也大量傳入西方的科學技術。當時中國的科學發展趨於緩慢,落後于歐洲。隨著西學傳入,使得中國的少數士大夫開始認識到西方學問之中有其優於中國之處,但這並未造成中國人對於中西學的基本高下看法有所改變。西學中主要受到注意的仍是技術方面如天文曆法、測量以及所謂的「西洋奇器」等,對於中國學術本身的影響衝擊亦不大。而當時傳入中國的學問非常多樣,也有一些士大夫著手與傳教士合作翻譯西方書籍或著書介紹西學,例如徐光啟就曾與利瑪竇合譯幾何原本。李之藻與利瑪竇合譯同文算指。在中西文化交流的同時,基於雙方文化的歧異及認知方面的不同,也引發一些衝突,例如南京教案等。

明朝數學的發展停滯,且遠比宋元落後,明朝中葉的著名數學家顧應祥與唐順之對「天元術」的茫昧不解,被認為是中國數學在十四世紀之後由盛而衰的一個見證。在明朝年間失傳了宋元兩代累積的數學知識,後來經過清代學者梅穀成等人重新發現並加以研究。駱祖英認為,整體而言,明代數學的整體水平並不比同期西方數學滯後,當時東西方數學水平相當。

明朝时期各少数民族政权得到了迅猛发展,民族关系形势也非常复杂。明前期,退居漠北的北元政权伺机南下扰明,企图东山再起,成为明朝的心腹大害;明中晚期,白山黑水的女真族在首领努尔哈赤的带领下,建立后金政权,并最终取代明政权。

明初武功实力最强,具开拓进取精神。在“大一统”思想的指引下,明朝以实力为后盾,注意使用军事打击和政治招抚相结合的策略,积极经略周围边疆地区,对后期民族关系思想的形成产生了重大的影响。同时,儒家知识分子刘基的夷夏观在华夷易代之际也表现出开明与宽容的特色。后来仁宣之治时在民族关系上做出来调整,南北一同放弃大规模军事征伐,采取“顺则抚之,逆则御之”的守成求安思想。

土木之变后明朝实力由盛转衰,对周边少数民族也由进攻态势全面转向防御,形成了“守备”为主的民族关系思想。随着西南地区麓川土司势力大增,大臣门对于是“剿”与“抚”展开争论。到了孝宗期间,面对国计日艰、边防日蹙,和北方的蒙古、女真等民族关系更加复杂的情况,明孝宗想在民族关系处理上想有番作为,让边臣献策,比如马文升的“抚安东夷”、“收复哈密”,杨一清的“关中奏议”,王鳌的“上议边八事”以及丘浚的“严武备”、“驭夷狄”等;另外随着明朝国力的衰微以及土鲁番势力的强大,哈密卫的“弃”与“守”成为当朝大臣讨论争锋的焦点。世宗和穆宗统治时期边患增多,北虜南倭使明朝疲于应付,特别是面对套寇屡屡犯边,边疆祸事不断,曾铣等有识之士就收复河套问题多次上疏。穆宗在位期间实现了明蒙之间具有里程碑意义的隆庆和议,结束了蒙古各部与中原王朝近二百年兵戈交战的局面。

神宗在位时爆发了万历三大征,虽然取得胜利,但是耗费了明朝人力物力财力,使国家日趋衰败。内阁首辅张居正启用大将李成梁和戚继光,在辽东和蓟镇取得大捷。熹宗明思宗时期明朝衰落到不可收拾的地步,东北女真建立后金政权,不断扰明。因此,朝廷任用辽东总兵熊廷弼、袁崇焕等人和女真对抗。同时在明清易代之际,明末思想家王夫之表现出特有的悲壮情怀和对华夷问题的反思,成为近代民族主义思想的滥觞。

明朝承袭传统的华夷之辨民族思想,尊崇汉族,鄙视少数民族,并进一步强化。而明朝民族关系思想基本上是对传统儒家民族观“大一统”和“华夷之辨”思想的继承和发展,同时又受到蒙元政权的影响,表现出“华夷一家”与“华夷之防”思想的矛盾与统一。但是消极、保守的边疆政策不仅影响了民族关系的发展,对于一个整体的统一多民族国家的形成、发展,具有重大影响。

学术界一般认为明朝是回族最终形成的时期。元朝灭亡后,不断有归附明朝政府的回人,明初政府曾禁止蒙古、色目人更易姓氏,限制回族内部通婚,後來明廷支持對回民的漢化政策,讓回民改易漢姓。朱元璋“御制至圣百字赞”以及明皇室关于修建清真寺和保护清真寺宗教职业人员的谕旨,在一定程度上肯定了回族的宗教生活,史學家陳垣指出:「明人對於回教,多致好評。政府亦從未有禁止回教之事,與佛教、摩尼教、耶穌教之屢受政府禁止者,其歷史特異也。」。明代學者陸容說:「回教門異於中國者,不供佛,不祭神,不拜屍。所尊敬者惟一天字。最敬孔聖人。故其言云:僧言佛子在西空,道說蓬萊往東海,唯有孔門真實事,眼前無日不春風。見中國人修齋記醮,則笑之。」大约经历了200多年,在伊斯兰教影响下,以回回人为主体,融合了国内汉、维、蒙等多种民族成分逐渐形成为新的民族共同体。在明末农民起义中,陕北和甘肃东部的回民在马守应的率领下,成为当时张献忠、李自成軍隊的主力之一。明末清初时期,米剌印、丁国栋在「反清复明」的口号下,率领了持续两年的甘州起义。到了清代,回族社会政治地位降到了历史上的最低点。

明朝邊境上最大的兩個威脅明朝安全的部族是蒙古和女真,時人稱其為東虜和西虜。在明朝初年武功強盛時,一度將蒙古驅至漠北,蒙古也因內亂分裂成韃靼、瓦剌等部而無力南侵。之後伴隨明朝的衰落,蒙古諸部中最有實力者稱霸於族內後,也多次進攻明朝,諸如瓦剌發起的土木之變和土默特部發起的庚戌之變,明朝的疆界因此內縮,也大大消耗明朝的國力。俺答汗後期開始於明朝通好,受封為順義王,其後的三娘子繼承和平的政策。明蒙之間邊境安寧和平,互通有無。這種情況直到後金控制蒙古後才告結束。明朝早期曾經設置奴兒干都司來管理東北諸部,這一階段女真人作為明朝於東北地區排除北元殘餘勢力的盟友,雙方關係處於蜜月期,但中後期明朝採取「犁庭掃穴」等一些列不適當政策,對女真人進行歧視、限制、挑撥、分化甚至屠戮,激化當地矛盾。隨著東北的蒙古部和女真部日益強大,奴兒干都司被廢,明朝在東北的控制力更是進一步下降。17世紀後,建州女真首領努爾哈赤統一女真各部,降服蒙古,於1616年建國後金,與明朝分庭抗禮。後金佔領遼東大部土地,曾對當地的漢人進行屠殺,並有入主中原的野心,嚴重威脅明朝的安全。1636年改國號大清,建立清朝,最終於1644年明朝滅亡後接替明朝統治中國276年的歷史。

苦兀或称苦夷,是明代对库页岛上土著居民的称呼。永乐七年(1409年),明朝在黑龙江下游东岸特林设奴儿干都司,管理今东三省。《敕修奴儿干永宁寺碑记》、《重建永宁寺碑记》载:明钦差亦失哈等多次巡视奴儿干地方,曾对“海外苦夷诸民,赐男妇以衣服、器用,给以谷米,宴以酒食”。他们表示,“世世臣服,永无异意”。清代亦曾在此设姓长以统之。有人认为,“海外苦夷”(库页人)是指库页岛上的阿伊努人。

中國學者對於明朝對藏政策的主流見解是「因俗以治」、「多封眾建」、「羈縻懷柔」。明朝对西南藏族地区的治理基本承袭元朝统治管理的办法。对西藏地区推行“多封众建”的政策,先后分封三大法王和五大地方之王。同时,通过朝贡和回赐,互通有无,体现西藏与中央政治上的隶属关系。明以来,藏族地区社会安定,经济发展迅速,文化艺术繁荣,与中國内地的交往更趋广泛和密切。美國漢學家莫里斯·羅西比(英语:Morris Rossabi)認為,永樂帝是第一名積極尋求擴大與西藏關係的明朝統治者。

明朝时期傣族被称为“百夷”,而且经营百夷地区主要通过土司制度,明朝还制定了其他政策、采取了其他措施加强明朝对百夷的统治。百夷地处西南边疆地区,因此,明朝经营百夷的政策与明朝的西南边疆的形势发展息息相关。但由于明朝统治者的短视与误判,以“析解麓川地”的错误政策经略这一地区,最终导致明末缅甸洞吾王朝对中缅边界中方一侧领土的侵扰和“蚕食”,造成明朝西南边界大幅内缩。

明朝初年,實施朝貢體制,朝貢貿易薄來厚往,許多日本人冒充朝貢使者來賺取好處。日本實際上是處於割據狀態,沒有統一的中央政權,很多到中國來冒充朝貢使者的日本人沒有日本政府的管轄,朝貢後他們滯留在中國沿海搶劫。這是明初的倭寇。為防止倭寇朱元璋就頒布海禁政策。從此之後,如果要來中國做生意,必需朝貢兼貿易,否則不予,這就是所謂的「朝貢貿易」,兼具有懷柔拉攏周圍國家的用途。明朝嚴格的貿易管制政策的影響導致正常貿易地下化,轉為走私貿易。貿易港集中地由廣東、福建轉往已成為殖民地的菲律賓、印尼。而海上的維持秩序角色由於中國官方的消失而導致海盜集團猖獗。由於海上貿易仍在暗處進行,美洲銀器又大量流入中國,銀開始成為流行的通貨。

明初鉴于倭寇的猖獗,明初曾实施海禁政策,永樂年間,明成祖派遣航海家三寶太監鄭和率遠洋船隊七下西洋,最遠到達非洲東海岸,又派遣吏部驗封司員外郎陳子魯出使撒馬兒罕、吐魯番、火州等西域十八國,加強明王朝同世界各國的經濟政治上的往來,為中國走向世界做出貢獻,體現永樂王朝的鼎盛和開放,也能表现出明朝海洋政策具有外向型海权意识。後来明仁宗聽從朝中一些大臣的意見,認為下西洋過於浪費,收效不大,宣佈停止下西洋的活動。不到一年,仁宗病逝,宣宗朱瞻基繼位,改年號宣德。宣德五年閏十二月初六(1431年1月19日),派鄭和第七次也是最後一次下西洋。明憲宗年間,曾有太監向憲宗提議再次下西洋,於是皇帝下詔到兵部索要鄭和出使的海圖等資料。但由於劉大夏等官員認為下西洋為一大弊政,有害無益,因此將當年鄭和出海地圖等資料藏匿起來(一說銷毀),兵部尚書項忠命吏入庫搜索無果,再次下西洋一事於是作罷。

而相当长时段内领先于世界的明朝海军,随着保守海洋政策的施行,海军实力迅速衰落。自唐宋以來中國的大航海事業,在明代出現衰退。儘管也有“鄭和下西洋”的驚世盛舉,但總的來說,海外貿易在整個明代的經濟體系中所佔比重不大。明代海禁約持續了兩百年的時間,其結果是關閉了民間對外貿易的通道。私人下海販易被視為違法,海外商船來華貿易也受到嚴格的控制。朝貢貿易則是唯一留下的貿易孔道,由官方壟斷專營海外貿易,並與朝貢制度嚴密掛鉤,從而形成朝貢與貿易合二為一的「貢市一體化」格局。明代學者王圻在《續文獻通考》中記述:「凡外夷貢者,我朝皆設市舶司以領之⋯⋯其來也,許帶方物,官設牙行與民貿易,謂之互市。是有貢舶即有互市,非入貢即不許其互市明矣。」日本學者內田直作認為:「明代之朝貢貿易,不論從貿易政策上或財政政策上講,都沒有重大的價值,只是舉揚所謂朝貢禮的服從關係而已。」由於朝貢貿易無視經濟法則,幾乎全靠國力的強盛來維持,因此在明初明太祖和明成祖之後,由於國力漸衰以及時勢發生變化,朝貢貿易也走向衰落,代之而起的是走私貿易。

後來倭寇橫行,明朝加大海禁的力度,直到明穆宗隆慶元年(1567年)之後,倭寇逐漸平息,朝廷有鑒於對外貿易對沿海居民的重要性,才逐步有限度地對外開放,並開放福建月港為中國商民出洋貿易的唯一口岸,允許民間商船出洋遠販東南亞各地,惟日本不在通商範圍之內,去日貿易仍被視為「通倭」之舉,史称「隆庆开关」。

唐朝以来秉持着中华正统史观的朝鲜一直都是以“藩国”自居,尊中原王朝为宗主国,但在历代王朝中,朝鲜最为心悦诚服的却是明朝。1392年,高丽王朝大将李成桂发动政变,建立了李氏朝鮮。上书朱元璋要求赐予“国号”,朱元璋认为“朝鲜”是古名,而且“朝日鲜明”出处文雅,因此裁定朝鲜为新国名。 朝中关系进入了近三百年的相对稳定时期。明亡之后,朝鲜君臣无不思念明朝,最后修建了大报坛来纪念明朝皇帝,尽管此时朝鲜官方文书的纪年在明亡后早已采用清朝的年号,无论是私人文书,还是皇室的祭祀中,私下里一概都是延用明朝纪年,以至于出现了“崇祯两百多年”事情。

清朝基本上不干涉朝鮮的尊明之舉,朝鮮對明朝的崇拜不僅沒有影響到對清朝的忠誠,反而讓清朝感到朝鮮是一个知恩圖報、講情重義的國度。康熙帝曾說:「觀朝鮮国王,凡事极其敬慎,其国人亦皆感戴。」

倭寇對明朝的海疆構成嚴重威脅。但是倭寇的主要構成並非日本人,而是中國沿海一帶的破產流民。期間雖有朱紈和張經的抗倭,但最後都未能取得完全的成功。為防止倭寇的侵擾,世宗時期實行海禁,斷絕對日貿易。直到戚繼光等名將力行抗倭,倭寇才被剿清,海疆形勢才趨於平靜。豐臣秀吉統一日本後,意欲佔領朝鮮。萬曆廿年,日本進攻朝鮮,朝鮮國王逃到義州並派使節向明朝求救。明朝一度取得戰爭的勝利。中日一度進行和談。但萬曆廿五年後,日本再次進攻朝鮮,戰爭進入僵局狀態。萬曆廿六年,豐臣秀吉逝世,日本軍心動搖,結果撤軍。此即為壬辰衛國戰爭。這次戰爭嚴重削弱明朝與朝鮮兩國,明朝在張居正期間積蓄的國力大量被消耗,日本復又陷入分裂,女真部落成為相對的得益者。

1377年,朱元璋册封阿瑜陀耶国王为“暹罗国王”,“暹罗”这一名称正式固定下来,称为中文语境下对泰国的称呼。有明一代,阿瑜陀耶遣使臣访问中国达112次,而中国也派使臣访问阿瑜陀耶19次。

歐洲進入大航海時代後,葡萄牙人持續開拓前往印度、中國的航路,1511年葡萄牙佔領馬六甲(約今馬來亞地區)後,就意圖在中國建立貿易據點。明武宗正德七年(1513年),葡萄牙國王曼努埃爾一世為想要與明廷通商,派出使節團前往中國。使節團本來想在廣州登陸,但被拒絕入境。他們改以武力佔據屯門,與明朝爆發屯門海戰、西草灣之戰,結果葡萄牙戰敗。最後明世宗同意葡方入境,並且讓葡萄牙人在澳門開設洋行,修建洋房,允許他們每年來廣州「越冬」,這是西方列強第一次正式性的登陸中國。其後西班牙、荷蘭、英國等歐洲國家相繼派使團東來,使得不少西洋事物傳入中國。

明神宗萬曆十年(1582年),利瑪竇奉命前往中國教區工作。利瑪竇很快學會中文,並穿儒服、通儒書,頗得明朝士大夫好感。後來他被舉薦到北京,頗得明神宗信任。他向中國進獻坤輿萬國全圖、自鳴鐘、日晷、西洋大炮、望遠鏡、火槍、西藥、聖母瑪莉亞像、十字架等貢品,先後在北京、肇慶等地展出。利瑪竇不僅傳播天主教,還啟發徐光啟、李之藻等人學習西學。另外他還將中國各種文化傳入歐洲,如儒家思想、佛道學說、圍棋等,可謂「貫通中西第一人」。另外,在明末時期有不少明朝軍隊曾裝備火器,尤其是西洋大炮。

明代早期,社會風氣比較節儉。後期伴隨著商品經濟的發達以及政府控制力的下降,社會風氣轉向浮華與奢靡,不論士大夫或百姓,在飲食、居住、穿著、娛樂各方面都更為講究,甚至贫穷人家也追慕仿效,與過去儒家崇尚簡樸的風氣有很大的差別。商人的地位也明顯提高。時人张瀚曾言:“今之世风,上下俱损矣!”明初朱元璋認為「元以寬失天下」,因此要「救之以猛」,一改元朝優容江南士人的政策,採取各種措施打壓及迫害江南文人。有明一代,明廷便擬定江南重賦,「官、民田視他地方倍蓗」,並且規定「浙江,江西,蘇松人毋得任戶部」。仕宦的江南士人,或因黨案,或因文字獄之故,動輒獲罪橫死。

明朝的另一項重要社會風氣就是藏書之風。無論官方與民間皆好藏書。私家藏書尤為發達。天一閣是中國目前現存的最早的私家藏書樓。其創建者是范欽。在范欽去世時,天一閣藏書的總數達到七萬卷。天一閣對藏書嚴加保管,水火不入。也嚴禁外借。明代重要的藏書樓還有汲古閣、絳雲樓等。而私人刻書也逐漸發達,出現的彩印的套印等新工藝,印製的書籍量更是達到一個新的高峰,也使得書籍的讀者群更為擴大,各種通俗小說的出現也為平民百姓提供另一種娛樂。裝幀方法也得到改進,出現對後世影響深遠的線裝書。

貞節旌表的制度在明朝成為固定持續的制度,使得女性守貞守節從原本的典範理想成為一般性的風氣甚至規範。而纏足也在明朝逐漸成為社會上較普遍的習俗。此外,晚明社會風氣的開放,使當時成為中國歷史上才女文化最發達的時代之一。

16世纪的欧洲城市规模较小,1519年至1558年时期,拥有2万至3万人口即可称为“大城市”。从城市规模和人口比例看,晚明中国的城市化程度反倒稍高一些。据伊懋可的数据,中国城市人口在明末占总人口的6\%至7.5\%。而学者曹树基估计,1630年时中国城市化率已达到8\%,略高于清代城市化率的7.4\%。

明代百姓的娛樂風尚發達,「旅遊」一詞在中国歷史上首次出現。明代傢俱的樣式也進入一個新的階段,風格典雅,流傳至今者不在少數。園林藝術在明朝也非常興盛,代表著作是明代造园家計成的《園冶》一書,這是第一部全面總結私家園林的專著。

明朝是中国古代社会福利最好的时期,在平定天下驱除胡虏之后,朱元璋一方面实施“与民休息”的经济政策,另一方面推出了中国最早的福利政策。明朝的福利政策完备且有特色,对当时经济的发展起到了非常积极的推动作用。明代出现了免费养老院、免费医院和免费公墓等,而且对于60岁以上的老人,明朝政府制定了较为完备的养老政策。

明代的茶文化與酒文化也十分發達,民間盛行飲酒之風,酒令進入成熟的階段。各種新式茶色紛紛出現,紫砂壺也開始流行。酒樓茶館成為城市居民的重要休閒場所。

《乌青镇志》记载万历年间,市井之家的宴席:“万历年间,牙人以招商为业。初至,牙主人丰其款待,割鹅开宴,招妓演戏,以为常。”

万历进士顾起元在《客座赘语》中记述南京风俗民情说:“今则服舍违式,婚宴无节,白屋之家,侈僭无忌。”

张岱在《陶庵梦忆》中记载了许多美食:“越中清馋,无过余者,喜啖方物。北京则苹婆果、黄、马牙松;山东则羊肚梨、文官果、甜子;福建则福桔、福桔饼、牛皮糖、红腐乳;江西则青根丰城脯;山西则天花菜;苏州则带骨鲍螺、山楂丁、山楂糕、松子糖、白圆、橄榄脯;嘉兴则马交鱼脯、陶庄黄;南京则套樱桃、桃门枣、地栗团、窝笋团、山楂糖;杭州则西瓜、鸡豆子、花下藕、韭菜、元笋、塘栖蜜桔;萧山则杨梅、莼菜、鸠鸟、青鲫、方柿;诸暨则香狸、樱桃、虎栗;嵊则蕨粉、细榧、龙游糖;临海则枕头瓜;台州则瓦楞蚶、江瑶柱;浦江则火肉;东阳则南枣;山阴则破塘笋、谢桔、独山菱、河蟹、三江屯怪、白蛤、江鱼、鲥鱼、里河。远则岁致之,近则月致之,日致之。耽耽逐逐,的为口腹谋。”

叶梦珠在《阅世编》记述明末宴会:“肆筵设席,吴下向来丰盛。缙绅之家,或宴官长,一席之间水陆珍馐多至数十品。即庶士中人之家,新亲严席,有多至二三十品者。若十余品则是寻常之会矣。然品必用木漆果山如浮屠样,蔬用小瓷碟添案,小品用攒盒,俱以木漆架架高,取其适观而已。即食前方丈,盘中之餐,为物有限。崇祯初,始废果山碟架,用高装水果,严席则列五色,以饭盂盛之。相知之会则一大瓯而兼间数色,蔬用大铙碗,制渐大矣。”

明代笔记记载:“昔有一人,善制鹅掌。每豢养肥鹅将杀,先熬沸油一盂,投以鹅足,鹅痛欲绝,则纵之池中,任其跳跃。已而复擒复纵,炮瀹如初。若是者数回,则其为掌也,丰美甘甜,厚可经寸,是食中异品也。”。

明朝服饰继承了宋元两代的式样,但亦有一定程度的胡化,例如明代流行的曳撒就是继承于元代蒙古人的腰线袄。中后期更出现了前代未见的形制款式如立领,以及于一件衣服的显眼处大量使用钮扣。至清朝期间逐渐被禁止,但仍有少数款式和特征流传至今。近代至现代朝鲜族、琉球族、京族的民族服饰(韩服、琉装、越服)亦深受明朝服饰影响。

明代婦女的服裝,主要有衫、襖、霞帔、褙子、披風、比甲及裙子等,明中期出現立領。比甲的名稱,見於宋元以後,但這種服飾的基本樣式,卻早已存在。比甲為對襟、無袖,左右兩側開衩。隋唐時期的半臂,就是與比甲有著一定淵源關係。明代比甲大多為年輕婦女所穿,而且多流行在士庶妻女及奴婢之間[原創研究?]。成年女性多戴狄髻,並於上面插上成套的飾物,稱為頭面。明代上襦下裙的服裝形式,與唐宋時期的襦裙最大差別在於明代的上衣並不束在裙外,這種款式稱為襖裙。比如立领、宽衣大袖紧袖口与大褶裙装等,都是大明服饰的特色。勞動時常加一條短小的腰裙,以便活動,有些侍女丫環也喜歡這種裝束。上襦除傳統的交領外,到明中後期還出現立領。裙子除繼承前代的百褶裙、褶襉裙外,還出現了馬面裙。裙的顏色,初尚淺淡,雖有紋飾,但並不明顯。至中期則多飾以膝襴,有刺繡、織金、燙金等形式的裙襴。崇禎初年,裙子多為素白,即使刺繡紋樣,也僅在裙幅下邊一、二寸部位綴以一條花邊,作為壓腳。裙幅初為六幅,即所謂「裙拖六幅湘江水」;後用八幅,腰間有很多細褶,行動輒如水紋。到了明末,裙子的裝飾日益講究,裙幅也增至十幅,腰間的褶襉越來越密,此時出現一種裙子,每褶都有一種顏色,微風吹來,色如月華,故稱「月華裙」。腰間多掛上荷包、事件(小工具組合)等物品,裝飾與實用性兼備。明代出現一種以各色零碎錦料拼合縫制成的服裝,稱為水田衣,形似僧人所穿的袈裟,因整件服裝織料色彩互相交錯形如水田而得名。它具有其它服飾所無法具備的特殊效果,簡單而別致,水田衣的制作,在開始時還比較注意勻稱,各種錦緞料都事先裁成長方形,然後再有規律地編排縫制成衣。到了後來就不再那樣拘泥,織錦料子大小不一,參差不齊,形狀也各不相同,與戲台上的「百衲衣」(又稱富貴衣)十分相似。

明代男子常服、吉服、常禮服等,多用袍衫,有直身、直裰、道袍、道服、行衣、深衣等形制。上層社會及富家男子的便服面料以綢緞為主,上繪有紋樣,也有用織錦緞制作的,其制為大襟、右衽、寬袖,下長過膝。常服及吉服道袍、直裰、直身等,配以絲縧,勞動者多穿上衣下褲組成的裋褐。巾帽有多款,常見有幅巾、大帽、東坡巾、儒巾、飄飄巾等。

明太祖朱元璋詔令「衣冠制度悉如唐宋之舊」,因此明朝漢族男子服式沿襲大襟右衽交領和圓領這兩種傳統服飾式樣,又大量吸收元代服飾特點,發展出曳撒、兵笠等特色服飾。明代婦女的服裝,主要有衫、襖、霞帔、褙子、比甲及裙子等,衣服的多變與款式做工達到一個高峰。


%% -*- coding: utf-8 -*-
%% Time-stamp: <Chen Wang: 2019-12-26 15:06:23>

\section{太祖\tiny(1368-1398)}

\subsection{生平}

明太祖朱元璋(1328年10月29日-1398年6月24日),或稱洪武帝,明朝開國皇帝,原名朱重八,曾改名朱興宗,投军被郭子兴取名元璋,字国瑞,生於濠州钟离县。廟號「太祖」,谥號「开天行道肇纪立极大圣至神仁文义武俊德成功高皇帝」,統稱「太祖高皇帝」。在位三十一年,因年号洪武也俗稱洪武帝。太祖之後的皇帝除明英宗(二度在位),皆實行一世一元制。

朱元璋出身贫农家庭,幼时贫穷,曾为地主放牛。後因災變,曾一度剃髮出家,四出流浪,化緣為生,25岁(1352年)时,参加郭子兴领导的红巾军反抗蒙元政权。先後击败了陈友谅、张士诚等其他諸侯軍閥,统一南方,後北伐灭元,建立大一統的封建皇朝政權,国号“大明”。

明太祖在位期间,為其家族能夠長期統治平民用殘酷方法殺害了許多人, 自著大誥三編宣揚部份經過。據臣下劉辰所著國初事跡他又發明使用多種殘酷殺人方法。

明太祖下令农民归耕,奖励垦荒;大興移民屯田和军屯;组织各地农民兴修水利;大力提倡种植桑、麻、棉等经济作物和果木作物;下令解放奴婢;减免賦稅。派人到全国各地丈量土地,清查户口等等。经过洪武时期的努力,社会生产逐渐恢复和发展,史称「洪武之治」。同时立《大明律》,用严刑峻法管理百姓与官僚,禁止百姓自由迁徙,严厉打击官吏的贪污腐败,设立锦衣卫等特务机构,整肅顯貴的势力及他認為對他的朝廷有威脅的人、並废中书省,由皇帝直領各部,进一步加强了中央集权。驾崩後传位于嫡长孙朱允炆為明惠宗。

明太祖的生活儉樸、工作勤奮,在南京的皇宮內,沒有設立“御花園”,只有“御菜園”,其中種滿蔬菜,使得皇宮自給自足。大封宗籓,令世世皆食歲祿,不授職任事。至明朝中后期,朱元璋子孫人口繁殖至近百萬人。洪武元年令:「凡孝子順孫、義夫節婦、志行卓異者,有司正官舉名,監察御史、按察司體覆,轉達上司,旌表門閭。又令:民間寡婦,三十以前,夫亡守制,五十以後,不改節者,旌表門閭(貞節牌坊),除免本家差役。」洪武二十六年令:「凡婦人因夫、子得封者,不許再嫁。如不遵守,將所授誥赦追奪,斷罪離異。其有追奪為事官誥赦,具本奏繳內府,會同吏科給事中、中書舍人,於勘合低簿內,附寫為事緣由,眼同燒毀。」明朝婦女守寡盛行。又創立明朝入宮婦女生殉制度。

元文宗天曆元年九月十八日(1328年10月29日)未時,朱元璋出生於濠州钟离县东乡(今安徽省凤阳县小溪河镇燃灯寺村),排行第三。朱元璋先世家沛(今江苏沛县),後徙句容(今江苏省句容市)达百年之久。祖辈生活在古泗州(今江苏省盱眙县)。父親朱五四(後改為世珍),母親陳氏为濠州钟离县(今安徽省凤阳县)人。

朱元璋幼時甚貧困,並無法讀書,曾為地主放牛。牧童伙伴多人都奉朱为领袖,且日后成朱起义将领多人,至正四年四月(1344年)淮北大旱,引發饑荒,朱元璋初六父崩,初九兄薨,廿二日母崩,與仲兄極力營葬後秋九月入皇覺寺當行童。入寺五十日,因荒年寺租難收,寺主封倉遣散眾僧,朱元璋只得離鄉為遊方僧雲遊淮西潁州。

元至正八年(1348年),朱元璋游歷淮西、汝潁、泗等州完畢,返回皇覺寺并逐渐讀書识字。至正十二年(1352年)二月辛丑,身在皇覺寺多年的朱元璋受好友湯和來信勸說,到濠州投靠郭子興,參加紅巾軍。由於指揮有方,不久便成為郭子興身旁一名親兵并赐名元璋字国瑞,並娶郭子興養女马氏(即後來的馬皇后)。後來朱元璋見郭子興與其他濠州紅巾軍領袖如孫德崖、趙均用不和,屢有衝突,朱元璋不願涉及濠州內鬥,故主動要求返家鄉招募新兵,徐達、湯和等朱元璋兒時好友獲准隨行,不久朱元璋的部隊已有結集了數千人。次年,朱元璋部隊攻下滁州,成為他首個據點,同時也在攻佔滁州期間,李善長加入朱元璋部隊,成為他一個重要幕僚。此時,濠州的郭子興被孫德崖及趙均用迫走,前來滁州投靠朱元璋,由於朱元璋名義上仍是郭子興部下,朱元璋乃將滁州兵權交予郭子興。

至正十四年(1354年),張士誠據高郵,自稱為誠王,十五年,元朝丞相脫脫率軍進攻高郵,分兵攻六合,六合乃滁州屏障,故朱元璋領兵援六合,幸好脫脫被誣陷而被迫交出兵權,元軍不戰自潰,滁州也轉危為安。朱元璋見滁州地小,建議進攻長江北岸的和州。朱元璋攻下和州不久,郭子興病故,郭子興次子郭天敘被立為都元帥,朱元璋與郭子興妻弟張天祐為副元帥,遥奉韩林儿的大宋龙凤政权。同年夏,常遇春、廖永安、俞通海歸附朱元璋,使得其軍著手渡江攻入采石、太平路,並計劃攻取集庆路(今南京市)。此時,元軍降將陳野先願協助紅巾軍攻集慶,郭天敘與張天祐感軍功不及朱元璋,故決定在陳野先引領下,親自領軍攻打集慶。結果紅巾軍攻集慶時陳野先叛變,郭、張二人被殺,陳野先也死於亂軍中。郭天敘與張天祐死後,朱元璋成為都元帥,盡領郭子興舊部。至正十六年(1356年),朱元璋領軍再次攻打集慶,結果集慶被朱元璋部隊一舉攻陷,朱元璋將這裡作為自己的根據地,並改名為應天府。至此,朱元璋以應天府為中心,與元朝軍隊、張士誠、徐壽輝等部形成犬牙交錯之勢。

朱元璋攻佔應天後,開始攻佔應天周邊地區以鞏固防務。至正十六年,遣徐達攻佔鎮江、鄧愈克廣德,次年,遣耿炳文克長興,徐達克常州,而朱元璋親自率眾攻取寧國。隨後趙繼祖克江陰、徐達克常熟。胡大海克徽州、常遇春克池州,繆大亨克揚州。至正十八年,朱元璋親取婺州。明年,朱元璋陸續攻佔浙東餘下各地,常遇春克衢州、胡大海克處州,至此朱元璋部控制江左、浙右各地,向西與陳友諒部相鄰。朱元璋攻下浙東後,小明王升朱元璋為儀同三司江南等處行中書省左丞相,同時朱元璋也得浙東名士如朱升、劉基相助,朱元璋採取朱升「高築牆、廣積糧、緩稱王」的建議,採取穩健的進攻措施;並且遵照劉基「先漢後周」之策略,着手對江南各勢力進行對抗。

至正二十年,陳友諒攻陷太平路,隨後弒主徐壽輝、稱帝建國,國號漢,之後傾全軍攻應天府。朱元璋與劉基設計,先命胡大海進攻信州,斷陳友諒後援,再命部下康茂才詐降作陳友諒的內應,引漢軍主力進入朱元璋在應天城外龍灣設下的埋伏中,結果漢軍被朱元璋軍隊大敗,隨後朱元璋攻取太平、安慶、信州等地。。至正二十一年,朱元璋改樞密院為大都督府,重新整理軍制。北結察罕帖木兒、密通方國珍,而與正面的陳友諒部進行會戰。同年攻克江州、南康、建昌、撫州等地。次年,佔領龍興,改洪都府(今江西南昌)。

至正二十三年(1363年),张士诚派部将吕珍围攻退守安丰的小明王韓林兒及丞相劉福通,朱元璋不顧劉基反對,派軍北上解安豐之圍,結果刘福通战死,韩林儿被朱元璋救出。此后,韩林儿被朱元璋安置在滁州,仍然被奉为皇帝。陳友諒趁朱元璋主力軍北上,率六十萬水軍進攻朱元璋根據地,首先圍攻洪都,但朱元璋姪朱文正堅守洪都兩個多月,待朱元璋親率二十萬部隊馳援,陳友諒大軍改往鄱陽湖與朱元璋大軍交戰,史稱“鄱陽湖之戰”。陳友諒自恃巨艦出戰,採用炮攻,朱元璋險些負傷被擒。隨後,朱元璋利用東北風而改用火攻,致使陳友諒部大量受損。之後朱元璋利用鄱陽湖水位降低便於小舟活動,改為分兵水路圍攻陳友諒。陳友諒中箭身亡,漢軍潰敗。隨後朱元璋圍攻武昌,并盡佔湖北各地。次年,朱元璋自立為「吳王」,以李善長為右相國,徐達為左相國,常遇春、俞通海為平章政事,立子朱標為世子。次月再次親征武昌,陳友諒之子陳理舉降。隨後吳軍相繼攻克廬州、吉安、衡州。至正二十五年,吳軍繼續攻佔寶慶、贛州、浦城、襄陽,同年冬,下令討張士誠。次年,吳軍再次攻破湖州、杭州。再一年,徐達克平江,張士誠被俘,至此朱元璋一統江南。至正二十六年(1366年),朱元璋派廖永忠迎接韩林儿至金陵應天府,途中在瓜步渡长江时,韩林儿所乘船只沉没,韩遇难。

至正二十七年(1367年),朱元璋命湯和為征南將軍,討伐割據浙東多年的方國珍。隨後制定北伐战略:先攻取山東,其次進攻河南,再次攻佔陝西潼关,最後再進軍元大都。隨後命徐達為征虜大將軍,常遇春為副將軍,帥師二十五萬,由淮河進入,北取中原。并命胡廷瑞為征南將軍,何文輝為副將軍,進攻福建。同年,方國珍投降,徐達攻破山東濟南,胡廷瑞下邵武,湯和、廖永忠由海道攻克福建福州。北伐一直持續到洪武年間,徐達、常遇春隨後攻佔整個河南、山西,最終直取元大都(今北京)。

至正二十八年正月初四(1368年1月23日),朱元璋在應天府登基即位,建國號大明,年號洪武,是為「明太祖」。以應天為「南京」,開封為「北京」。同年八月初二(9月14日),大將徐達攻克大都,元朝覆亡。由于幼年对于元末吏治痛苦记忆,即位后一方面減輕農民負擔,恢復社會的經濟生產,改革元朝留下的糟糕吏治,懲治貪污的官吏,社會經濟得到恢復和發展,史稱洪武之治。明太祖確立了里甲制,配合賦役黃冊戶籍登記簿冊和魚鱗圖冊的施行,落實賦稅勞役的徵收及地方治安的維持。

太祖平定天下後,大封諸將為公侯,部份追封為王。初封六公,其中以五大將、一大臣為開國元勳。分別為:韓國公李善長、魏國公徐達、鄭國公常遇春、曹國公李文忠、宋國公馮勝、衛國公鄧愈。而後又追封胡大海為越国公、戰死的丁德興為濟國公,湯和為信國公、馮國用封郢國公。次年,明太祖於雞鳴山立功臣廟,六月初三日廟成,太祖親定功臣位次,以徐達為首,次常遇春、李文忠、鄧愈、湯和、沐英、胡大海、馮國用、趙德勝、耿再成、華高、丁德興、俞通海、張德勝、吳良、吳禎、曹良臣、康茂才、吳復、茅成、孫興祖凡二十一人。死者像祀,生者虛位。又以廖永安、俞通海、張德勝、桑世杰、耿再成、胡大海、丁德興七人配享太廟。此位序屡经删汰,已非洪武二年所定名单位次。

随後,太祖进一步加强中央集权。洪武三年(1370年),杀中书左丞杨宪。洪武四年七月十一(1371年8月21日),傅友德攻克成都,明朝平定四川。洪武五年四月二十三日(1372年5月26日),廖永忠率明军平定广西,洪武五年六月初三(1372年7月3日),傅友德大败元军,明朝平定甘肃。洪武六年(1373年),太祖鑒於開國元勛多倚功犯法,虐暴鄉閭,特命工部制造鐵榜,鑄上申戒公侯的條令,類似戰國時代的「鑄刑鼎」。洪武八年(1375年),德庆侯廖永忠因僭用龙凤诸不法事,赐死。洪武十二年(1379年),贬右丞相汪广洋于广南,旋赐死。洪武十三年(1380年),胡惟庸案发,左丞相胡惟庸被诛,太祖罢中书省,分中书省之权归于六部,直接归皇帝掌管。洪武十五年(1382年),设立锦衣卫,加强明朝特务统治。1382年1月6日,明军在云南昆明附近大败元朝军队,元梁王自杀,1382年4月7日,蓝玉、沐英攻克大理,段氏投降,明朝平定雲南。洪武十八年(1385年),郭桓案发,由于涉案人员甚多,太祖將六部左右侍郎以下官员皆處死,各省官吏死於獄中達數萬人以上。

洪武二十三年(1390年),李善長的家奴盧仲謙告發李善長與胡惟庸往來勾結,以「狐疑觀望懷兩端,大逆不道」見誅,接續又誅殺陸仲亨與唐勝宗、費聚、趙庸三名侯爵,株連被殺的功臣及其家屬共計達三萬餘人,連「浙東四先生」(刘基、宋濂、章溢、叶琛)亦不能免,并頒布《昭示奸黨錄》。洪武二十六年(1393年),藍玉被錦衣衛指揮蔣瓛密告謀反,史称“藍玉案”。此案牵连到十三侯、二伯,連坐族誅達一萬五千人,明朝建国功臣因此案幾乎全亡。此時太祖又頒布《逆臣錄》,詔示一公、十三侯、二伯。洪武二十七年(1394年),太祖杀江夏侯周德兴以及颖国公傅友德,在捕鱼儿海战役中立功的定远侯王弼亦被赐死。洪武二十八年(1395年),开国六公爵最後一位僅存者冯胜被杀。

在处理内政同时,太祖亦多次籌劃北伐蒙古以保障北方邊塞的安寧,大勝。並曾成功在甘肅擊敗王保保(1372年)、在东北逼降納哈出(1387年)、在蒙古高原幾乎活捉元主脫古思帖木兒(1388年)。同时太祖进军辽东,使朝鮮王朝等归顺(1388年)。

洪武三十一年閏五月初十日(1398年6月24日),朱元璋崩逝於南京皇宮內,享壽七十歲,在位三十一年。與已故的妻子馬皇后兩人一起長眠於南京紫金山明孝陵。《明朝小史·卷三》載,責殉諸妃,強迫伺寝宫人尽数殉葬。《彤史拾遺記》記載,太祖以四十六妃陪葬孝陵,其中所殉,惟宮人十數人。

新任皇帝惠宗遵照遺命。洪武三十一年六月甲辰,上謚曰“欽明啟運俊德成功統天大孝高皇帝”,廟號太祖。永樂元年六月十一日丁巳,增諡“聖神文武欽明啟運俊德成功統天大孝高皇帝”。嘉靖十七年十一月朔,改諡“開天行道肇紀立極大聖至神仁文義武俊德成功高皇帝”。到了清朝,康熙帝历次南巡必跪拜孝陵,曾立碑「治隆唐宋」赞誉其功。中華民國建立初,孫文至孝陵祭告朱元璋。

朱元璋一直以來都是以猛治国。持正面評價者通常都是從其大力打擊貪污,恢復經濟著眼,歷史記載朱元璋是少數極力勤政的皇帝;而持負面評價者,則多從其高壓統治著眼,以猛著称,他的“重典治国”思想不只為遏制官僚腐败。亦顯現在清洗权贵势力、以特務錦衣衛控制政治、又用文字獄及廷杖大臣,以立帝王權威。

明初沿袭元朝制度,设立中書省,置左、右丞相。甲辰正月,初置左、右相國,其中李善長為右相國,徐達為左相國。洪武元年(1368年),改為左、右丞相。由中书省统六部,但不设置中書令。

洪武十三年(1380年),胡惟庸案之后,太祖罢中书省,分中书省之权归于六部。原中書省官屬盡革,惟存中書舍人。至此,秦、漢以降實行一千六百餘年的宰相制度自此廢除,相權與君權合而為一,施行軍權、行政權、監察權三權分立的國家體制。

由於國家事務繁多,皇帝無法處理,洪武十五年九月罷四輔官,仿宋殿閣制設內閣。內閣只為皇帝的顧問,雖無宰相之名,但有宰相之實。此外他仍沿用元朝制度,在中央設置吏、戶、禮、工、刑、兵六部。并設立都給事中六人,分吏、戶、禮、工、刑、兵六科,每科一人;此外建立五寺包括大理寺、太常寺、光祿寺、太僕寺、鴻臚寺等五寺制度。此外他還沿襲元的監察制度,設立御史台,有左右御史大夫各一名;不久改為都察院,下設若干監察御史,負責監督各級官吏。除此他还颁布《大明律》等,对官吏管理进行规制。

为了加强对臣民的控制和监视,太祖设置了巡检司和锦衣卫。巡检司主要是负责全国各地的关津要冲的把关盘查,缉捕盗贼,盘诘伪奸;锦衣卫则负责秘密侦察大小官吏活动,随时向皇帝报告不公不法之徒。同时太祖还授予锦衣卫侦察、缉捕、审判、处罚罪犯等一切大权,锦衣卫正式成为直屬皇帝的情报机构。

太祖出身貧寒,對政治貪污尤其憎惡,其對貪污腐敗官員處以極嚴厲的處罰。太祖在政期間,大批不法貪官被處死,包括開國將領朱亮祖,女婿駙馬都尉歐陽倫,其中甚至因為郭桓案、空印案殺死數萬名官員。由於太祖的吏治嚴厲,在明初相當長一段時間,官員腐敗的情況得到有效遏制。然而,随着大明江山逐步稳定,再加上军事和皇室贵族战功大,享有很高的社会特权,不少人迅速腐化变质。。朱元璋开展雷厉风行的肃贪运动,历时之久、措施之严、手段之狠、刑罚之酷、杀人之多,为几千年历史所罕见。尽管朱元璋反贪决心大、力度猛、出奇招,使腐败现象得到一定程度的遏制,也一度取得了“阶段性的成果”,但還是未能達到徹底清除人類貪欲權位腐敗的本性。

太祖性格多疑,對功臣有所猜忌,恐其居功枉法,圖謀不軌。这些特权阶级杀人伤人、霸占土地、逃税漏税、恃强凌弱、奸淫妇女、吃喝嫖赌、贪污纳贿,甚至造刀枪、穿龙袍的都有。面对这种对王朝的长治久安构成严重威胁局面,太祖把这些特权阶级无情地清洗。廖永忠和 朱亮祖 先後死於非命。隨後太祖以擅權枉法之罪名殺胡惟庸,又殺御史大夫陳寧、御史中丞塗節等人。之後李善長亦被牽連,家屬七十餘人被殺,總計株連者達三萬餘人。此後的藍玉案中,連坐被族誅的有一萬五千餘人。但紀非錄所記載太祖的兒子諸藩王犯有很多暴行,太祖則只是輕微勸戒了事。太祖還通過設立錦衣衛(洪武二十年废除)、詔獄、廷杖等機構或制度,打擊功臣、特務監視等一系列方式加強皇權控制。

太祖遵古制,王命法:三十受兵、六十歸兵。國有三軍,所以誡非常,伐無道,尊宗廟,重社稷,安不忘危。太祖令諸藩鎮守天下,又各領兵權,這固然是親親之情,信任無以復加,卻也未必就沒有帝王心術。強藩林立,能做皇帝的卻始終只有一個,諸藩勢力犬牙交錯,必然相互牽制,相互監視,除非朝廷中樞衰弱之極。當中樞真的衰弱至極時,就算沒有藩王,也會被權臣取而代之。自三皇五帝,以一介布衣而成天子者,唯漢高祖與太祖,其他帝王,大都是前朝重臣或一方豪強而黃袍加身。所以由自己子孫取代無能之君,也勝過將江山付與外人之手,如此可保朱家數百年江山。

建国伊始,太祖就在《大明律令》的基础上制订颁行《大明律》,紧接着又亲自编定《明大诰》。1397年,太祖下詔正式颁布了《大明律》。《大明律》一共四百六十卷,分吏、户、礼、刑、兵、工六律,简于《唐律》,严于《宋律》。《大明律》规定:“谋反”、“谋大逆”者,不管主犯还是从犯,一律凌迟,祖父,父、子、孙、兄、弟以及同居的人,只要是年满十六岁的都要处决。太祖立法一为治民,二为治吏,尤其是《明大诰》对贪官污吏的处决也十分严厉,可以视为反贪刑事特别法。只要是犯有贪污的官吏,一经查实,一律发配北方荒漠中充军,赃至六十两以上者枭首示众,仍剥皮实草。

太祖十分重視法律宣传,寫了大誥三編和大誥武臣,让臣民熟悉法律,不去犯禁。他還經常法外施刑,動輒凌遲。

早在朱元璋起兵时,他就多次强调军纪。他认为「攻克城池用武力,平定混乱用仁政」,杀人并非「勇猛」。要求部队不许滥杀无辜,还给予俘虏优待;同时还要求部队爱护百姓,不得随意焚烧抢掠乱杀百姓,他严令:「掠夺老百姓财物者处死,拆毁老百姓住房的处死。」由于朱元璋部队的军纪严明,朱元璋赢得了部属的尊重,也赢得了民众的支持。

明代早期軍隊的來源,有諸將原有之兵,有元兵及群雄兵歸附的,有獲罪而謫發的,而最主要的來源則是籍選,是由戶籍中抽丁而來。除此之外尚有簡拔、投充及收集等方式。洪武十三年(1380年),太祖廢除大都督府,並改为中军、左军、右軍、前军、后军等五军都督府。洪武十七年(1384年),太祖在全國的各軍事要地,設立軍衛,由都督府管理。一衛有軍隊五千六百人,其下依序有千戶所、百戶所、總旗及小旗等單位,各衛所都隸屬於五軍都督府,亦隸屬於兵部,有事從征調發,無事則還歸衛所。軍隊來源為世襲的軍戶,由每戶派一人為正丁至衛所當兵,軍人在衛所中輪流戊守以及屯田,屯田所得以供給軍隊及將官等所需。五军都督府有统兵权但无调兵权,兵部有调兵权而无统兵权,兩者互相制衡,互不統轄,各自與兵部直接聯繫,最後奏請皇帝裁定,以避免權力過大。

明代軍户是世襲制,一旦列入軍籍,世代都是軍人,朝廷有事要為朝廷作戰。軍丁一旦逃亡、病故、老疾或被虜,就要按軍籍所造之册,到該軍丁原籍追補本身或其親屬,以補足原數。

元朝初期,元世祖曾经遠征日本,导致日本念念不忘,于是终元之世,日本不与中国同好。明朝开国以后,太祖就派使臣持国书去日本、高丽、安南、占城四国,宣告元朝已经灭亡,现在的稱霸中国是大明,應奉大明为“正朔”来朝贡。高丽、安南、占城三国太祖使赴明称臣朝贺,惟独日本没有任何反应。令太祖更为恼火的是,不但日本人不来朝称臣,而且“乘中国未定,日本率以零服寇掠沿海”。同时,被太祖消灭的张士诚、方明珍等残部多逃亡海上,占据岛屿,勾結倭寇出没海上掳掠财货,辽宁、山东、福建、浙江、广东,“滨海之地,无岁不受其害”。

後来太祖喝令“日本国王”處理倭寇,结果使者被日本人殺害。消息傳回中國後,太祖大為怒火,批日本是“国王无道民为贼”的“跳梁小丑”。面对日本,太祖忍下了恶气,从此以后对日本使者一概驅逐處理,朝贡也一概拒绝接受,与日本不相往来。同时,太祖把朝鲜、日本、大琉球、小琉球、安南、真腊、暹罗、占城、苏门答腊、西洋、爪哇、彭亨、百花、三佛齐、勃泥等15国列为“不征诸夷”,写入《皇明祖训》,告诫子孙这些「蛮夷国家」如果不主动挑衅,就不许征伐。

公元1370年(洪武三年)太祖派遣莱州知府赵秩远赴日本。懷良親王经过赵秩的阐释明处外交政策打消了顾虑。不久懷良派遣僧人祖来跟随赵秩回明朝向进表笺。公元1371年(洪武四年)太祖派遣僧人祖阐、克勒等八人送日使归国,从此明朝和日本建立了外交关系。

公元1392年(洪武二十五年)七月,高丽大将李成桂发动兵变掌控高丽局势以后遣知密直司事赵胖至明朝礼部上表:“定昌府院君瑶权署国事,及今四年。瑶又昏迷不法,疏斥忠正,昵比谗邪,变乱是非,谋陷勋旧,谄惑佛神,妄兴土木,靡费无度,民不堪苦;子奭痴佁无知,纵于酒色,聚会群小,谋害忠直。又其臣郑梦周等潜成奸计,欲生乱阶,乃将勋臣李成桂、赵浚、郑道传、南訚等谮于权署国事,令有司论劾以致谋害,国人愤怨,共诛梦周。权署国事尚不悛改,又谋杀戮。举国臣民实虑社稷生灵俱被其害,惶惧失措,无可奈何,咸以为若所为难以主斯民奉社稷。洪武二十五年七月十二日,以恭愍王妃安氏之命,退居私第。窃念军国之务不可一日无统,择于宗亲,无有可当舆望者,惟门下侍中李成桂泽被生灵,功在社稷,中外之心夙皆归附。于是一国大小臣僚、闲良、耆老、军民臣等咸愿推戴,令知密直司事赵胖,前赴朝廷奏达,伏启照验,烦为闻奏,俯从舆意,以安一国之民。”太祖通过礼部传达圣旨:“三韩臣民既尊李氏,民无兵祸,人各乐天之乐,乃帝命也。虽然,自今以后慎守封疆,毋生谲诈,福愈增焉。尔礼部以示朕意。”李成桂遣门下侍郎赞成事郑道传赴京谢恩,并献马六十匹。

当年八月,李成桂又遣前密直使赵琳赴京进表:“权知高丽国事臣李成桂言:伏惟小邦自恭愍王无嗣薨逝之后,辛旽子禑冒姓窃位者十有五年矣。迄至戊辰春,妄兴师旅,将犯辽东,以臣为都统使,率兵至鸭绿江。臣窃自念小邦不可以犯上国之境,谕诸将以大义,即与还师,禑乃自知其罪,逊位子昌。昌亦暗弱,难以莅位,国人启奉恭愍王妃安氏之命,以定昌府院君王瑶权署国事。瑶乃昏迷不法,紊乱刑政,狎昵谗佞,贬斥忠良,臣民愤怨,无所控告。恭愍王妃安氏深虑其然,命归私邸。于是一国大小臣僚、闲良、耆老、军民等以为军国之务不可一日无统,推戴臣权知军国事。臣素无才德,辞至再三,而迫于众情,未获逃避,惊惶战栗,不知所措。伏望皇帝陛下以乾坤之量、日月之明,察众志之不可违、微臣之不获已,裁自圣心,以定民志。”朱元璋再通过礼部复旨:“高丽限山隔海,天造东夷,非我中国所治。尔礼部回文书,声教自由,果能顺天意合人心,以妥东夷之民,不生边衅,则使命往来,实彼国之福也。文书到日,国更何号,星驰来报。”

当年十一月,李成桂再遣艺文馆学士韩尚质至明朝上表:“窃念小邦王氏之裔瑶,昏迷不道,自底于亡,一国臣民推戴臣权监国事。惊惶战栗,措躬无地间,钦蒙圣慈许臣权知国事,仍问国号,臣与国人感喜尤切。臣窃思惟,有国立号诚非小臣所敢擅便。谨将“朝鲜”(箕子所建古国名)、“和宁”(李成桂诞生之地)等号闻达天聪,伏望取自圣裁。”太祖再通过礼部复旨:“东夷之号,惟朝鲜之称美,且其来远,可以本其名而祖之。体天牧民,永昌后嗣。”李成桂遣门下侍郎赞成事崔永沚谢恩,又遣政堂文学李恬送明朝颁赐的给前朝的高丽国王之印,并请更己名为“李旦”。

公元1394年(洪武二十七年)帖木儿帝国向明朝贡马,而且致国书。第二年,明朝派遣兵科给事中傅安率领使团往报。但当傅安等抵达帖木儿帝国国都撒马尔罕时,帖木儿打算要向东兴兵,攻打明朝了,于是扣押了傅安等人,而且百般的诱惑傅安等人归顺帖木儿,傅安被扣押十三年,坚贞不屈,维护明朝的尊严。一直到了帖木儿死了以后,他的孙子哈里嗣位,想和明朝和好,于是才放傅安等人回国。傅安回国以后又出使了中亚诸国。

公元1395年(洪武二十八年)十一月,李成桂遣艺文春秋馆太学士郑总赴京请诰命印章:“洪武二十五年七月十五日,差知密直司事赵胖奏达天庭,继差门下评理赵琳奉表陈奏,钦奉圣旨,许允权知国事。准奉礼部来咨內云:‘国更何号,星驰来报。准此。’即差知密直司事韩尚质赍擎奏本赴京,钦奉圣旨节该:‘东夷之号,惟朝鲜之称美,且其来远矣,可以本其名而祖之。钦此。’除钦遵外,洪武二十六年三月初九日,差门下评理李恬送纳前朝高丽国王金印,又于当年十二月初八日准奉左军都督督府咨,钦奉圣旨內一款节该:‘即合正名。今既改号朝鲜,表文仍称权知国事,未审何谋?钦此。’一国臣民战栗惶惧,咸请国王钦遵施行。见今虽称国王名号,窃缘未蒙颁降诰命及朝鲜国印信,一国臣民日夜颙望,仰天吁呼。伏请照验,烦为闻奏,乞赐颁降国王诰命及朝鲜印信施行。”朱元璋通过礼部下旨拒绝:“今朝鲜在当王之国,性相好而来王,顽嚣狡诈,听其自然,其来文关请印信诰命,未可轻与。朝鲜限山隔海,天造地设,东夷之邦也,风殊俗异。朕若赐与印信诰命,令彼臣妾,鬼神监见,无乃贪之甚欤?较之上古圣人,约束一节决不可为。朕数年前曾敕彼仪从本俗,法守旧章,令听其自为声教。喜则来王,怒则绝行,亦听其自然。尔礼部移文李成桂,使知朕意。”

明朝立国后日本因进入南北朝的大分裂时期后出现的、大量外出掠夺的武士阶层为主的倭寇骚扰入侵的恐惧,明政府立国后采取了一系列针对海患的闭关锁国政策:洪武三年(1370),明政府“罢太仓黄渡市舶司”;洪武七年(1374),明政府下令撤销自唐以来即存在的、负责海外贸易的福建泉州、浙江明州、广东广州三市舶司,中国对外贸易遂告断绝;洪武十四年(1381),太祖以倭寇仍不稍敛足迹,又下令禁濒海民私通海外诸国,此后每隔一两年即将该海禁政策再次昭示天下。自此,连与明朝素来交好的东南亚诸国也不能来华进行贸易和文化交流。

整个海禁政策从太祖开始,到了明穆宗在位期間被以“市通则寇转而为商,市禁则商转而为寇”为由实行开关(隆庆开关);至清初又开始一連串的闭关,清高宗時更推行“一口通商”政策、直至鸦片战争后,通行整个明清二代的海禁政策才被彻底打破。

元末之际,中國發生多次大規模的災荒饑饉疾病和瘟疫,以及連年戰爭,期間生产遭到严重破坏,人口也大量減少,经济全面崩溃,人民处在流离失所的过程中。大明建立並統一全國後,面对哀鸿遍野、饿殍满路的凄凉局面,太祖實行黃老治術治國,太祖说:天下初定,百姓财力困难,就像刚刚会飞的鸟不可拔羽,才种的树不可摇根一样。现在必须采取这种政策,同时主张藏富于民。

农业是明代社会最主要的生产部门。太祖在恢复和发展社会经济中,把发展农业放在了首位,为了保证农业第一线有足够的劳力资源。太祖通令全國,地主不得蓄养奴婢,所养的奴婢一律释放为良民。凡因饥饿而典卖为奴者,由朝廷代为赎身;嚴格控制寺院的發展,明令各州府县只能有一个大寺院,禁止四十歲以下的妇女当尼姑,严禁寺院收养童僧,二十岁以上的青年如果要是出家,必须得到父母和官方同意,出家后三年内还要赴京考試,不合格者潜发为民。這些政策的实施,使得社會增加了一只庞大的劳动力大軍。

全國的農業生產在大規模战争而遭受極大破壞的背景下得到很大程度的恢復,加上太祖在位期間大規模向淮河以北和四川的荒無之地、墾荒填充移民,使人口得以穩定增長。

此外他也實行屯田政策,軍屯面積佔全國耕地的近十分之一。此外,商屯也相當盛行,政府以買賣食鹽的專賣證(稱之為鹽引)作為交換,利誘商人將糧食運往邊疆,以確保邊防的糧食需求。明太祖也曾派遣國子監下鄉督導水利建設、赈灾,並以減免稅賦獎勵耕作。這些措施使得過去很多飽受戰亂損毀的地區恢復了生氣,使明朝的經濟得到了快速的恢復。

到洪武二十六年(1393年),全國有6500萬人,其中民戶佔6175萬人,軍戶佔325萬人。另外,其為了動員全社會,明太祖十分重視戶口普查,每個人有固定的義務。人民分為軍戶(弓兵、校尉、力士)、匠戶、民戶(马户、陵戶、茶戶、柴戶、阴阳戶、医戶)、灶戶,不允许隨便轉換工作,匠籍、軍籍比一般民戶地位低,不得應試,並要世代承襲。若想脫離原戶籍極為困難,須經皇帝特旨批准方可。各种活动也要引憑才合法。编成里甲,规 定了路引制度,也就是通行证制度。普通百姓只要走出出生地百里之外,就得持有官府开具的通行证,否则就以逃犯论处。

明朝初期實行「科舉必由學校」的政策,太祖多次強調:「古昔帝王育人材,正風俗,莫不先於學校。」明代洪武元年(1368年),詔開科舉,對制度、文體都有了明確要求。命令刘三吾等人刪節《孟子》中民貴君輕的內容,課試不以命題,科舉不以取士。。洪武年間,太祖共主持举办六次科考,七次发榜,共取一甲21名、二甲223名、三甲686名,合930名,平均每科取士155人,為明朝選拔輸送了大量有學識的官員,包括練子寧、黃子澄、解縉等一代名相。洪武三十年科舉時,因中進士者均為南方籍。太祖将试官二十餘人指為胡黨藍黨凌遲殺害,并自阅试卷,取中六十一人,皆为北方人,并于六月廷试。此外,他並將學校列為「郡邑六事之首」,以官學結合科舉制度推行程朱理學,并設立國子監等重要教育機構。由於太祖在位期間實行高壓的吏治政策,明初诗文三大家不得善終,後世不乏有學者主張太祖曾實行過一些文字獄。也有學者指出關於朱元璋嗜殺之事例,存有穿鑿附會的問題。

太祖崇尚简朴,也希望老百姓也勤俭节约。他规定靴子上不能有任何装饰。同时对于全国人民怎么穿衣;每个阶层佩戴什么样的首饰;盖什么样的房子;出行坐什么样的车子以及人们的行动举止也是朱元璋关注的焦点,因而制定了一系列规章制度,包括了生活的方方面面,其细致入微,可谓空前绝后。“洪武二十二年三月二十五日奉聖旨:“在京但有軍官軍人學唱的,割了舌頭;下棋打雙陸的,斷手;蹴圓的,卸脚;作買賣的,發邊遠充軍。”府軍衛千戶虞讓男虞端故違吹簫唱曲,將上脣連鼻尖割了。又龍江衛指揮伏顒與本衛小旗姚晏保蹴圓,卸了右脚,全家發赴雲南。又二十五年九月十九日,禮部榜文一款:“內使剃一搭頭,官民之家兒童剃留一搭頭者,閹割,全家發邊遠充軍。剃頭之人,不分老幼罪同。””(《客座贅語》卷十)

太祖对天下老年人施以尊重,颁布《存恤高年诏》。洪武二十年,太祖怕有关部门执行不力,就又叮嘱礼部尚书,要以皇帝的名义再次重申一下这项政策。在朝廷的要求和带动下,各地形成了尊老养老的风气,赡养老人的要求也渗透到各地家法族规之中。

对于社会的救济朱元璋也十分重视,洪武时期,荒政则受到朝廷高度重视。朝廷除了拨付救灾济贫款项,还侧重加强民众抗灾自救能力。面对天灾侵袭,朱元璋积极作为,既树立了朝廷的负责任形象,又增强了政府的凝聚力,赢得了民心。救灾济贫实为获取民心、形成治世的重要前提,为“洪武之治”的出现夯实了经济社会基础。

为了贬抑商人,太祖他特意规定,农民可以穿绸、纱、绢、布四种衣料。而商人却只能穿绢、布两种料子的衣服。商人考学、当官,都会受到种种刁难和限制。

太祖建立明朝前后,十分重视宗教问题,通过协调儒释道三者的关系,既稳定了局面,又争取了人心,为巩固明朝政权奠定了思想和群众基础。通过有效的宗教管理措施,把宗教的发展始终控制在适合自己的政治需要范围内,并利用宗教教化番荑,不断扩大自己的势力范围,为明政权创造了良好的国际环境。

在政治上,太祖推重儒释道三教并举的政策。他说:“尝闻天下无二道,圣人无两心。三教之立,虽持身荣俭之不同,其所济给之理一。”他极为重视佛教的辅政作用,将佛教事务视为朝中大事,对佛教制度、僧寺清规多方整饬,期望以此整顿僧团,去淤除垢,“振扬佛法以善世”。

洪武六年(1373年),太祖下诏对出家的僧尼免费发放度牒,才使得唐朝年间流传下来使的“度牒银”制度全部废除。

整顿僧团秩序,防止僧俗混淆,洪武二十四年,朱元璋还制定颁布了影响深广的《申明佛教榜册》,要求各地僧司查验清理天下僧寺,欲还俗者听其还俗,使出家僧人恪受戒律清规,禅、讲、瑜伽,各归本宗。

太祖亲自制定的“御制至圣百字赞”以及明皇室关于修建清真寺和保护清真寺宗教职业人员的谕旨,在一定程度上肯定了回族的宗教生活。

殉葬制度,在西漢初以後,逐漸在中原政權消失。朱元璋二子秦王對人民暴行(見御製紀非錄)被宮人殺死,即連坐迫秦王諸妃自殺。明朝時期明孝陵以四十六妃陪葬,其中有太祖死時殺死殉葬十几名侍寢宮人,這一制度沿襲至成祖、仁宗、宣宗、代宗。而“節烈從殉”的風氣,並向下廣為延伸至宗室公侯、官宦之家、以至民間,直至近百年之後其五世孫英宗死前指出殉葬非古禮,仁者所不忍,才禁殉葬于遺詔,永著為典。按朱元璋創立的制度,嬪妃殉葬由皇帝親臨作別。正統初,明英宗目睹皇父嬪妃殉葬,受很大刺激。天順年間下詔廢止。殺死從殉婦女的方法為將她們縊死,或勒死,或灌以水银毒死。这些生殉的妇女被称为“朝天女”,她們的家屬稱為“朝天女戶”,並給予一定待遇。關於朝天女記載主要依賴朝鮮的第一手資料《李朝實錄金黑口述》。寶慶公主生母張玄妙,以其女幼,得免殉葬。

《明清史事沉思录》中记载,“传谓男子宫刑,妇人幽闭,皆不知幽闭之义。今得之,乃是于牝(阴户)去其筋,如制马、豕之类,使欲火消减。国初常用此,而女往往多死,故不可行也。”对这种灭绝人性的手术,这本书的作者王春瑜评论道:“将人等同畜生处置,始作俑者其无后乎!”

明孝陵康熙題碑:“治隆唐宋”。

清朝官修正史《明史》张廷玉等对明太祖朱元璋最终能够成就帝业的评价是:“帝天授智勇,统一方夏,纬武经文,为汉、唐、宋诸君所未及。当其肇造之初,能沉几观变,次第经略,绰有成算。尝与诸臣论取天下之略,曰:‘朕遭时丧乱,初起乡土,本图自全。及渡江以来,观群雄所为,徒为生民之患,而张士诚、陈友谅尤为巨蠹。士诚恃富,友谅恃强,朕独无所恃。惟不嗜杀人,布信义,行节俭,与卿等同心共济。初与二寇相持,士诚尤逼近。或谓宜先击之。朕以友谅志骄,士诚器小,志骄则好生事,器小则无远圖,故先攻友谅。鄱阳之役,士诚卒不能出姑苏一步以为之援。向使先攻士诚,浙西负固坚守,友谅必空国而来,吾腹背受敌矣。二寇既除,北定中原,所以先山东、次河洛,止潼关之兵不遽取秦、陇者,盖扩廓帖木儿、李思齐、张思道皆百战之余,未肯遽下,急之则并力一隅,猝未易定,故出其不意,反旆而北。燕都既举,然后西征。张、李望绝势穷,不战而克,然扩廓犹力抗不屈。向令未下燕都,骤与角力,胜负未可知也。’帝之雄才大略,料敌制胜,率类此。故能戡定祸乱,以有天下。语云‘天道后起者胜’,岂偶然哉。” 清朝官修正史《明史》张廷玉等对明太祖朱元璋一生事业的评价是:“赞曰:太祖以聪明神武之资,抱济世安民之志,乘时应运,豪杰景从,戡乱摧强,十五载而成帝业。崛起布衣,奄奠海宇,西汉以后所未有也。惩元政废弛,治尚严峻。而能礼致耆儒,考礼定乐,昭揭经义,尊崇正学,加恩胜国,澄清吏治,修人纪,崇凤都,正后宫名义,内治肃清,禁宦竖不得干政,五府六部官职相维,置卫屯田,兵食俱足。武定祸乱,文致太平,太祖实身兼之。至于雅尚志节,听蔡子英北归。晚岁忧民益切,尝以一岁开支河暨塘堰数万以利农桑、备旱潦。用此子孙承业二百余年,士重名义,闾阎充实。至今苗裔蒙泽,尚如东楼、白马,世承先祀,有以哉。”

毛泽东在1964年3月24日,在一次听取汇报时的插话中对明太祖朱元璋、汉高祖刘邦、元太祖成吉思汗的治国能力评价如下:“可不要看不起老粗。”“知识分子是比较最没有知识的,历史上当皇帝的,有许多是知识分子,是没有出息的:隋炀帝,就是一个会做文章、诗词的人;陈后主、李后主,都是能诗善赋的人;宋徽宗,既能写诗又能绘画。一些老粗能办大事:成吉思汗,是不识字的老粗;刘邦,也不认识几个字,是老粗;朱元璋也不识字,是个放牛的。” 毛泽东对明太祖朱元璋的军事才能评价如下:“自古能君无出李世民之右者,其次则朱元璋耳。” 給吳晗提意見:“朱元璋是农民起义领袖,是应该肯定的,应该写的(得)好点,不要写的(得)那么坏。”

趙翼曾説:“藉諸功臣以取天下,及天下既定,即盡取天下之人而殺之,其殘忍實千古所未有。”“蓋明祖之性,實帝王,豪傑,盗賊兼而且也。”

商传评价朱元璋:「朱元璋出身于一个贫苦家庭,从社会最底层的放牛娃、四处讨饭的小和尚,全靠自己的奋斗成了一个统一王朝的开国皇帝。这是中国历史上,乃至世界历史上绝无仅有的事情。另外,朱元璋当上皇帝后,也没有停止步伐,他在位三十多年,成功地建立一个强大统一的明帝国」。

\subsection{洪武}

\begin{longtable}{|>{\centering\scriptsize}m{2em}|>{\centering\scriptsize}m{1.3em}|>{\centering}m{8.8em}|}
  % \caption{秦王政}\
  \toprule
  \SimHei \normalsize 年数 & \SimHei \scriptsize 公元 & \SimHei 大事件 \tabularnewline
  % \midrule
  \endfirsthead
  \toprule
  \SimHei \normalsize 年数 & \SimHei \scriptsize 公元 & \SimHei 大事件 \tabularnewline
  \midrule
  \endhead
  \midrule
  元年 & 1368 & \tabularnewline\hline
  二年 & 1369 & \tabularnewline\hline
  三年 & 1370 & \tabularnewline\hline
  四年 & 1371 & \tabularnewline\hline
  五年 & 1372 & \tabularnewline\hline
  六年 & 1373 & \tabularnewline\hline
  七年 & 1374 & \tabularnewline\hline
  八年 & 1375 & \tabularnewline\hline
  九年 & 1376 & \tabularnewline\hline
  十年 & 1377 & \tabularnewline\hline
  十一年 & 1378 & \tabularnewline\hline
  十二年 & 1379 & \tabularnewline\hline
  十三年 & 1380 & \tabularnewline\hline
  十四年 & 1381 & \tabularnewline\hline
  十五年 & 1382 & \tabularnewline\hline
  十六年 & 1383 & \tabularnewline\hline
  十七年 & 1384 & \tabularnewline\hline
  十八年 & 1385 & \tabularnewline\hline
  十九年 & 1386 & \tabularnewline\hline
  二十年 & 1387 & \tabularnewline\hline
  二一年 & 1388 & \tabularnewline\hline
  二二年 & 1389 & \tabularnewline\hline
  二三年 & 1390 & \tabularnewline\hline
  二四年 & 1391 & \tabularnewline\hline
  二五年 & 1392 & \tabularnewline\hline
  二六年 & 1393 & \tabularnewline\hline
  二七年 & 1394 & \tabularnewline\hline
  二八年 & 1395 & \tabularnewline\hline
  二九年 & 1396 & \tabularnewline\hline
  三十年 & 1397 & \tabularnewline\hline
  三一年 & 1398 & \tabularnewline
  \bottomrule
\end{longtable}


%%% Local Variables:
%%% mode: latex
%%% TeX-engine: xetex
%%% TeX-master: "../Main"
%%% End:

%% -*- coding: utf-8 -*-
%% Time-stamp: <Chen Wang: 2019-12-26 15:06:30>

\section{惠宗\tiny(1398-1402)}

\subsection{生平}

建文帝朱允炆(1377年12月5日-?),或稱明惠宗,是明朝第二代皇帝,年號“建文”,明太祖朱元璋之孫。在位期間進行一系列寬政、削藩的改革,史稱“建文改制”。由於燕王朱棣發動靖難之變攻入南京應天府,是為明成祖,朱允炆下落不明。大臣梅殷私諡其為「神宗孝愍皇帝」但成祖不承認,故不使用,甚至明成祖不認為朱允炆是合法皇帝,故明朝人大多稱之為建文君。直到南明時,弘光帝追谥其為“嗣天章道诚懿渊功观文扬武克仁笃孝让皇帝”,庙号“惠宗”。清高宗乾隆元年,高宗追谥其為「恭閔惠皇帝」,故也作「明惠帝」。

朱允炆是懿文太子朱标第二子,嫡母太子妃常氏所生長子朱雄英早故,另有一子朱允熥为其弟。嫡母常氏在1378年逝世后,朱允炆生母吕氏成为继任太子妃,所以明太祖朱元璋就視朱允炆為嫡長孫。

洪武二十五年(1392年),父亲朱标病死,朱允炆被祖父朱元璋立为皇太孙。由於自幼熟讀儒家經書,所近之人多懷理想主義,性情因此與其父同樣溫文儒雅,即長皆以寬大著稱。洪武二十九年,朱允炆曾向太祖請求修改《大明律》,他參考《禮經》及歷朝刑法,修改《大明律》中七十三條過份嚴苛的條文,深得人心。

朱允炆出生时脑袋长得颇偏,朱元璋用手摸着说:“半边月儿。”一年除夕,他与父亲朱标陪同朱元璋,朱元璋叫他父子作詠月诗,朱允炆作诗曰:“谁将玉指甲,掐作天上痕。影落江湖里,蛟龙不敢吞。”朱元璋看后默然不语。

明洪武三十一年(1398年)閏五月,明太祖朱元璋去世,死前密命驸马梅殷辅佐新君。朱允炆在同月(6月30日)即位,定次年(從1399年2月6日开始)為建文元年。建文帝在六月晉用齊泰為兵部尚書、黃子澄為太常寺卿,七月召方孝孺為翰林院侍講,在國事上倚重三人。建文帝的年號“建文”有別於其祖父的洪武,他不想仿效祖父以嚴刑峻法治國,即位後改行寬政,囚犯人數減至洪武時期的三成左右。

建文帝能虚心纳谏。一次他因病上朝晚了,监察御史尹昌隆对此提出批评,左右建议他说出自己染病,建文帝却认为这样的谏言难得,不但没有自辩,还表扬了尹昌隆,公开了他的奏疏。

明太祖為鞏固皇室,大封宗室為藩王,各擁私人護衛軍隊。對建文帝來說,諸藩王大多為其叔輩,且在封地掌握兵權,心中由是不安。建文帝為皇太孫時曾問黃子澄曰:「諸王尊屬擁重兵,多不法,奈何?」子澄回答說諸王軍力不足以抗衡朝廷。建文帝即位後,下令各王國的地方文武官員聽朝廷節制,採取削藩政策,先後废黜周王、湘王、齐王、代王及岷王。在部署對付年齡最長、軍功最多、武力最强大的燕王朱棣時,由於建文帝身邊的謀士多缺乏實際的政治經驗,以致打草驚蛇,引發了燕王先發制人的念頭。朱棣在權衡利害之後,於建文元年(1399年)七月在封地北平起兵反叛。他以“靖难”為名,向京師進軍。

建文元年,明建文帝下詔討伐燕軍。命吳傑、吳高、耿瓛、盛庸、潘忠、楊松、顧成、徐凱、李友、陳暉、平安分道併進,并在河北真定設立平燕布政使司,兵部尚書暴昭掌司事。隨後,耿炳文率大軍抵達,與燕軍交戰后失利退守。明建文帝臨時換將,撤換耿炳文,由李景隆代替。隨後,朱棣獲得寧王朱權及朵顏三衛,實力大增。而李景隆在率軍圍困北平后,仍然無法破城,并在鄭村壩潰敗。燕王因此向明惠帝上書,明惠帝不得不罷免齊泰、黃子澄。

建文二年,燕軍與中央軍在白溝河大戰,李景隆再次潰敗并逃亡濟南,隨後再在濟南潰敗。然而,朱棣卻無法攻破山東參政鐵鉉、都督盛庸的濟南城,不得不撤軍。明惠帝隨後封賞鐵鉉、盛庸,但卻不誅殺李景隆。同年冬,燕軍再次進犯濟寧,盛庸擊敗并斬殺燕將張玉,并接連獲勝。建文三年,兩軍在河北山東一帶屢次交戰,并互有勝負,最後燕軍攻入真定。

建文四年,何福、平安率領的中央軍在小河大勝燕軍,并斬其將陳文;而徐輝祖亦在齊眉山獲得大捷。燕軍恐懼后計劃北歸。恰逢建文帝誤以為燕軍已經北撤,召徐輝祖班師,致使何福孤軍奮戰。隨後,靈璧之戰中,燕軍大勝,陳暉、平安、陳性善、彭與明被執。盛庸軍亦在淮河之戰中潰敗,燕軍遂渡過淮河,抵達六合。建文帝不得不下詔要求各地勤王,并遣使割地罷兵。同年六月,盛庸在浦子口與燕軍交戰不利,都督僉事陳瑄率水軍附燕。隨後,朱棣率燕軍渡江,最終逼近南京應天府。谷王朱橞與李景隆開金川門變節,致使燕軍進入都城。宮中起火,建文帝不知下落。

燕王朱棣入京师应天府后,建文帝在宫中举火,皇后焚死,建文帝本人及其太子朱文奎则不知所踪,至今其下落仍是未定論的历史之谜。有稱其從地道逃亡,也有別史稱其離宮後出家為僧。

朱棣入京後,先捕殺齊泰、黃子澄、方孝孺及大批忠于建文帝的官員後,方稱皇帝,是為明成祖。當時駙馬都尉梅殷在軍中,從黃彥清之議,為建文帝發喪,諡「孝愍皇帝」,廟號「神宗」,但是不被成祖承認。

雖然朱棣宣稱在宮中找到建文帝的屍體,並為他舉行葬禮,但朱棣對建文帝未死的傳言不敢掉以輕心。建文帝年仅2岁的幼子朱文圭被废为庶人,并囚禁于凤阳广安宫。建文帝的三个弟弟原本封为亲王,尚未就藩,朱棣将他们降为郡王;年长的朱允熥和朱允熞先被封至福建漳州和江西建昌,旋被召回京师(南京),以“不能匡正建文帝”为由废为庶人,并囚禁于凤阳,只留下年幼的朱允熙给朱标奉祀,而不久之后朱允熙也于永乐四年死于火灾。

溥洽是建文帝主錄僧,當時傳聞他知道建文帝出逃的事,朱棣遂以其它罪名囚禁溥洽長達十餘年,直到姚廣孝病危時請求朱棣釋放溥洽,溥洽才獲釋。

明成祖即位后,不承认建文帝的正统性,下令销毁建文朝史料,并先后三次修改明太祖实录。成祖还下令作《奉天靖难记》,对懿文太子及建文帝多加诋毁。

正統五年,有僧自雲南至廣西,詭稱建文皇帝。隨後被逮捕調查,乃是鈞州人楊行祥,隨後下獄而死。同行十二位僧侶,皆戍遼東。隨後,雲南、貴州、四川等地均相傳有帝為僧時往來跡。正德、萬曆、崇禎年間,諸位大臣請求續封建文帝,及加廟諡,均未成行。虽然《太宗实录》(成祖原廟號太宗),称建文帝被朱棣以天子礼下葬,但崇祯帝在位时却亲口承认建文并无陵墓。崇禎十七年(1644年)五月,南明的弘光帝在南京即位,於同年七月為建文帝君臣平反,上庙号「惠宗」,谥号为「嗣天章道诚懿渊功观文扬武克仁笃孝让皇帝」。清朝乾隆元年,乾隆帝詔廷臣集議,追諡曰「恭閔惠皇帝」,故後世也稱建文帝為「明惠帝」。

2008年1月,福建省宁德市金涵乡上金贝村发现的一个和尚墓被认为是惠宗的墓葬所在,这个墓穴也是迄今为止福建发现的最大的和尚墓。然而,建文帝的最終下落至今仍是不解之謎,一说建文帝藏身于湖南省永州市新田县。

明建文帝在登基后不久,即重新選拔六部官員,其中大量官员在靖難之役中死亡;在战事中陣亡、拒絕與燕王朱棣合作而自殺或不屈而亡,其中包括禮部尚書陳廸,兵部尚書齊泰、鐵鉉,刑部尚書暴昭、侯泰,左都御史景清,右都御史練子寧、翰林院方孝孺等。



\subsection{建文}

\begin{longtable}{|>{\centering\scriptsize}m{2em}|>{\centering\scriptsize}m{1.3em}|>{\centering}m{8.8em}|}
  % \caption{秦王政}\
  \toprule
  \SimHei \normalsize 年数 & \SimHei \scriptsize 公元 & \SimHei 大事件 \tabularnewline
  % \midrule
  \endfirsthead
  \toprule
  \SimHei \normalsize 年数 & \SimHei \scriptsize 公元 & \SimHei 大事件 \tabularnewline
  \midrule
  \endhead
  \midrule
  元年 & 1399 & \tabularnewline\hline
  二年 & 1400 & \tabularnewline\hline
  三年 & 1401 & \tabularnewline\hline
  四年 & 1402 & \tabularnewline
  \bottomrule
\end{longtable}


%%% Local Variables:
%%% mode: latex
%%% TeX-engine: xetex
%%% TeX-master: "../Main"
%%% End:

%% -*- coding: utf-8 -*-
%% Time-stamp: <Chen Wang: 2019-12-26 15:06:36>

\section{成祖\tiny(1402-1424)}

\subsection{生平}

明成祖朱棣(1360年5月2日-1424年8月12日),或稱永樂帝,是明朝第三代皇帝,公元1402年至1424年在位,在位二十二年,年号永乐。

明太祖皇四子,安徽凤阳人,生于应天府(今江苏南京),時事征伐,並受封為燕王。洪武三十二年或建文元年(1399年)建文帝削藩,燕王遂發動靖难之役,起兵奪位,經過三年的战争,最終胜利,殺害方孝孺,驅逐其姪建文帝奪權篡位自封為帝。明成祖在位期间,改善明朝政治制度,发展经济,开拓疆域,迁都北京,使北京至此成為中國的政治中心至今。此外他编修《永乐大典》,派遣鄭和下西洋,北征蒙古,南平安南。明成祖的统治时期被称为永乐盛世,明成祖也被后世称为「永乐大帝」。另外,他加強太祖以來的專制統治,強化錦衣衛並成立東廠,此外,他在位期間重用宦官,也促成明朝中葉後宦官專政的禍根。

明成祖崩逝后谥号「体天弘道高明广运圣武神功纯仁至孝文皇帝」,庙号「太宗」,葬于长陵。嘉靖十七年(1538)九月,嘉靖帝改谥为「启天弘道高明肇运圣武神功纯仁至孝文皇帝」,改上庙号为「成祖」。

(1360年)四月十七日(5月2日),朱棣生于应天府(今南京)。

明太祖洪武三年(1370年),朱棣十岁,受封燕王。曾居鳳陽,对民情颇有所知。洪武十三年(1380年),朱棣就藩燕京北平,之后多次受命参与北方军事活动,两次率师北征,曾招降蒙古乃兒不花,並曾生擒北元大將索林帖木兒,加强了他在北方军队中的影响。朱元璋晚年,長子太子朱标、次子秦王朱樉、三子晋王朱棡皆早於朱元璋去世,故朱元璋於洪武三十一年閒五月駕崩後,四子朱棣不仅在军事实力上,而且在家族尊序上都成为诸王之首。

建文帝朱允炆登基後,為了提防燕王造反,於洪武三十一年十二月派工部侍郎張昺為北平布政使,都指揮使謝貴、張信為北平都指揮使。隨後又命都督宋忠屯兵駐開平,并調走北平原屬燕王管轄的軍隊。

建文元年(1399年),朱棣裝病,使建文帝把作為人質的朱棣三子朱高熾、朱高煦、朱高燧回燕國;之後由於屬下被朝廷處死,遂裝瘋。由於王府長史葛誠告知朝廷,裝瘋被發覺。

時燕王遣使入金陵奏事,使者被齊泰等審訊,被迫供出燕王的異狀,於是朝廷下密旨,令張昺、謝貴逮捕燕王府的官屬,張信逮捕燕王本人。但張信經過考慮,將此事告知朱棣。於是朱棣和僧人姚道衍等進行舉兵的謀劃,令張玉、朱能將八百勇士帶入府中潛伏,以待變故。

張昺、謝貴得到皇帝密詔后,七月初四帶兵包圍了燕王府。朱棣假意將官屬全部捆縛,請二人進王府查驗。二人進府后,朱棣派出府內的死士將其擒獲,并連同府內叛變的葛誠、盧振一同斬殺。當日夜裡,朱棣攻下北平九門,遂控制北平城。

燕王朱棣起兵,援引《皇明祖訓》,號稱清君側,指惠帝身邊的齊泰和黃子澄為奸臣(謀害皇室親族),需要鏟除,稱自己的舉動為「靖難」(意為「平定災難」),并上書於惠帝朱允炆。

燕軍控制北平后,七月初六,通州主動歸附;七月初八,攻破薊州,遵化、密雲歸附;七月十一,攻破居庸關;七月十六,攻破懷來,擒殺宋忠等;七月十八,永平府(今河北盧龍縣,屬秦皇島市)歸附。七月二十七,為防止大寧軍隊從松亭關偷襲北平,用反間計使松亭關內訌,守將卜萬下獄。至此,北平周圍全部掃清。燕軍兵力增至數萬。

燕軍攻破懷來後,由於領地相距太近,七月二十四日,谷王朱橞逃離封地宣府(今屬張家口,距北京約150公里,距懷來約60公里),奔京師。八月,齊泰等顧慮遼王、寧王幫助燕王,建議召還京師;遼王從海路返京,而寧王不從,遂削寧王護衛。宋忠失敗後,部將陳質退守大同。代王本欲起兵呼應朱棣,被陳質所控制,未果。

七月,朱棣反書到京,朱允炆削朱棣屬籍,廢為庶人。決定起兵討燕。在真定(今河北正定)置平燕布政使司。

耿炳文率軍在八月十三日到達真定,并分兵於河間、鄚州(河北任丘北约30里)、雄縣,為犄角之勢。在經過觀察後,八月十五日,燕軍趁中秋夜敵軍不備,偷襲雄縣;成功後又利用伏擊擊敗了鄚州的援兵,遂攻克鄚州,收編剩餘的部隊。八月二十四日,燕軍到達無極縣。從樵夫和中央軍被俘士兵處得知敵情,於是燕軍發動決戰。

二十五日,燕軍趁耿炳文送使臣出城時偷襲中央軍,炳文逃回城中后,怒而迎戰。在燕軍主力與耿炳文軍相持時,朱棣親自率軍襲擊其側翼,耿炳文大敗潰逃,中央軍投降三千多人。中央軍狼狽逃回城中,城池差點失守。部將李堅、甯忠、顧成等被俘;士兵被殺、被俘數萬人(后放還)。耿炳文率殘部不到十萬人在真定堅守不出,燕軍攻城三天不克。八月二十九日,燕軍返回北平。顧成降燕之後,留在北平協助燕世子朱高熾守城。

耿炳文戰敗,朱允炆開始擔憂戰事,考慮換將。黃子澄說曹國公李景隆是名將李文忠之子,建議他接任;齊泰反對,但惠帝不聽。八月三十日,拜李景隆為大將軍,誓師出征,并召回耿炳文。李景隆以德州為大本營,調集各路兵馬包括耿炳文敗兵,增兵至五十萬人,九月十一日進至河間。

朱棣聽說朝廷以五十萬傾國之兵交付李景隆,大喜過望,說:「李景隆不會用兵,給他五十萬大軍,根本是自取滅亡。趙括之失必然重演,我軍必勝。」

九月初一江陰侯吳高率辽东兵攻打永平郡,九月廿五,攻陷永平郡,決定趁勢偷襲大寧(今內蒙古寧城)以獲得其精銳部隊;另一方面利而誘之,將中央軍引至「空城」北平下。九月廿八,出師。。十月初六,燕軍經小路到達大寧城下。朱棣單騎入城),見寧王朱權,向朱權求救。在居大寧期間,朱棣令手下吏士入城結交并賄賂大寧的軍官等。十月十三,朱棣提出告辭,朱權在郊外送行,伏兵盡起,大寧軍紛紛叛變,歸附朱棣。於是朱權與王妃、世子等一同隨朱棣前往北平,而大寧的全部軍隊(包括其騎兵精銳朵顏三衛)都被朱棣收編。大寧成為空城。朱棣實力大增。十月十九,燕軍在會州整編,分立五軍(中前左右後)。十月廿一,入松亭關。

十一月初五,渡白河(時已結冰,渡河處在今北京順義區東),打敗李景隆的哨探陳暉部隊萬餘人。李景隆大敗。李景隆令鄭村壩所有軍隊輕裝撤退。。燕軍輕易擊潰城下的敵軍,獲得大量物資。。此戰中央軍喪師十餘萬。十一月初九,朱棣回到北平城,再次上書,惠帝不應。。十二月十九日,朱棣出師攻打大同。十二月廿四,抵達廣昌,守將楊宗投降。建文二年(1400年)正月初一,燕軍抵達蔚州,守將王忠、李遠投降。二月初二,燕軍攻大同。李景隆前来救援。李景隆走出紫荊關后,燕軍從居庸關返回北平。中央軍兵力、裝備大量損失,士氣受到重創。

建文二年四月,李景隆從德州,郭英、吳傑等從真定誓師北伐兵力增至六十萬。燕軍亦出。四月二十日,燕軍渡過玉馬河。四月廿四,燕軍戰鬥失利。。次日(四月廿五),再次交战。。。四月廿七,燕軍進攻德州。初九,燕軍進入德州。五月十五,燕軍攻濟南,李景隆逃走。燕軍遂圍濟南。十月,朝廷召李景隆回南京。黃子澄、練子寧、葉希賢等上書,請求立斬李景隆。朱允炆不聽。。鄭村壩之戰和白溝河之戰,使得两军攻守形勢逆轉。

燕軍圍濟南。右參政鐵鉉、盛庸堅守。朱棣射信入城招降,未果。五月十七,燕軍掘開河堤,放水灌城。鐵鉉決定派千人詐降,誘朱棣進城。朱棣圍城攻打三個月。六月,惠帝遣使求和,朱棣不聽。七月,平安進軍河間,擾亂燕軍糧道。八月十六,朱棣撤兵回北平。盛庸、鐵鉉追擊,大敗燕軍,收復德州。

建文二年十月,朱棣決定再度南下,十月廿七到達滄州。燕軍僅用兩天就攻下滄州,徐凱等投降。燕軍自長蘆渡河,十一月初四到達德州。朱棣招降盛庸未果,遂南下。十一月,燕軍到達臨清,焚其糧船。燕軍從館陶渡河,先後到達東阿、東平,威脅南方,迫使盛庸南下。盛庸在東昌(今山東聊城)決戰。十二月廿五,燕軍至東昌。朱棣仍然親自率軍衝鋒,盛庸開陣將朱棣誘入,然後合圍,張玉被中央軍包圍戰死。次日,燕軍再次戰敗,遂北還。在擊退中央軍的阻截后,建文三年正月十六,燕軍返回北平。

朱棣與姚廣孝商議,姚廣孝強烈支持再次出兵。二月十六,朱棣再次出師。三月二十日,燕軍探知盛庸在夾河(今河北省衡水市武邑縣附近,漳河支流)駐紮,於是駐紮在距對方四十里的地方。三月廿二,燕軍進兵夾河。。朱棣率領一萬騎兵和五千步兵攻擊盛庸軍左翼,不能入。此時燕將譚淵望見已經開戰,於是主動出兵攻打。朱棣、朱能等則趁中央軍調動產生的混亂,趁暮色向中央軍後方猛攻,斬殺莊得。此戰殺傷相當,但燕軍損失了大將譚淵。當夜,朱棣率領十餘人在盛庸營地附近露宿;次日(三月廿三)清晨,發現被中央軍包圍。朱棣再次利用禁殺之旨,引馬鳴角,穿過敵軍,揚長而去。中央軍愕然,不敢射箭。

朱棣回到營中,鼓勵眾將「兩軍相當,將勇者勝」,於是再次會戰,雙方互有勝負。戰鬥打了七八個小時后,盛庸大敗,損失了數萬人,退回德州。吳傑、平安引兵準備會合盛庸,聞庸已敗,退回真定。夾河之戰結束。夾河之戰重新確立了燕軍的優勢。閏三月初四,朱允炆因夾河之敗,再次罷免齊泰、黃子澄,謫出京城,暗中令其募兵。

擊敗盛庸后,朱棣進軍真定。。閏三月初九,兩軍會於藳城交戰。。次日,復戰,南軍不能支,大敗而去。。朱棣將射成刺猬的軍旗送回北平,令世子朱高熾妥善保存,以警示後人。從白溝河、夾河到藳城,燕軍三次得大風相助而勝,朱棣認為這是天命所在,非人力所能為。夾藳之戰再次使南軍損失慘重,正面戰場戰事稍緩和。南軍改為通過談判、反間、襲擊後方等方式間接作戰。擊敗平安後,燕軍南下,先後經過順德、廣平、大名,并駐紮於大名。諸郡縣望風而降。

朱棣聽說齊黃被貶,上書和談,表示「奸臣竄逐而其計實行,不敢撤兵」。朱允炆得書,與方孝孺討論,方孝孺表示可以借此機會遣使回報,拖延時間,并懈怠其軍心;同時令遼東等軍隊攻其後方,以備夾攻。於是(四月)惠帝令大理寺少卿薛嵓去見朱棣,傳詔并秘密在軍中散佈相關消息。薛嵓見朱棣,說「朝廷言殿下旦釋甲,暮即旋師。」朱棣表示這連三尺小兒也騙不過。薛嵓無言以對。五月初一,盛庸、吳傑、平安等分兵騷擾燕軍餉道。朱棣遣使者進京表示盛庸等不肯罷兵,必有主使。惠帝聽從方孝孺的意見,將其下獄(一說誅殺),和談破裂。

朱棣見和談破裂,從濟寧南下,成功焚燒大量中央軍糧船,京師大震,德州陷入窘境。

七月,燕軍進攻彰德,林縣投降。七月初十,平安自真定趁虛攻北平,擾其耕牧。朱高熾固守。朱棣分兵回援;(九月十八)平安與戰不利,退回真定。由於河北戰事不利,方孝孺想出了反間計,利用朱高熾(長子)和朱高煦(次子)的矛盾,先寫一封信給守北平的高熾,令其歸順朝廷,許以燕王之位;然後派人告訴朱棣和高煦(隨軍)世子密通朝廷,以使燕軍北還。但朱高熾得到信後,根本沒有拆開,將朝廷使者連人帶信一起送往朱棣處。反間計失敗。

七月十五,盛庸令大同守將房昭入紫荊關威脅保定,據易州西水寨以窺北平。朱棣回兵救援。朱棣分兵守保定,并包圍房昭的山寨。十月初二,燕軍與真定援兵和房昭軍決戰,房昭退回大同。十月廿四,燕軍回到北平。之後又擊敗了襲永平的遼東敵軍。

建文三年冬,南京有宦官因犯錯被處罰,逃到朱棣處,告知南京守備空虛。朱棣遂決定直接率兵南下,臨江一決。道衍亦支持不再與盛庸、平安等糾纏,直趨京師。

1401年(建文三年十二月初二),燕師復出。十二月十二,到達蠡縣(約在保定以南50公里)。建文四年(1402年)正月,燕軍南下至館陶渡河,長驅直入。正月十四,陷東阿;正月十五,陷東平;正月十七,陷汶上;正月廿七,陷沛縣(進江蘇);正月三十,到達徐州。惠帝見燕軍再次出動,三年十二月令駙馬都尉梅殷(惠帝的姑父,顧命大臣)任總兵官,鎮淮安。建文四年正月初一,將遷往蒙化的朱橚(廢周王)召回南京。命魏國公徐輝祖率兵援山東。

二月初一,何福、平安、陳暉進兵濟寧,盛庸進兵淮上。二月廿一,朱棣擊敗徐州的出戰軍隊,徐州自此閉城死守。朱棣繼續南下。三月初一,燕軍進逼安徽宿州。三月初九,抵達渦河(今安徽蚌埠市懷遠縣以北)。平安帶兵來追;但三月十四日在淝河中了朱棣所設的伏兵,只得退回宿州。三月廿三,朱棣遣將斷徐州餉道,鐵鉉等率兵圍攻,互有勝負。四月十四,燕軍進達睢水之小河,搭浮橋。次日,平安、何福領軍奪橋,雙方隔河僵持。數日後,中央軍糧盡,朱棣決定偷襲。半夜,渡河繞至敵後;四月廿二,雙方戰於齊眉山(靈壁縣西南三十里),中央軍大勝,斬燕將李斌。

燕軍陷入窘境。四月廿三,燕軍眾將要求北返,朱棣不同意,說「欲渡河者左,不欲者右。」大部份人站於左側,朱棣怒。朱能這時強力支持朱棣,表示「漢高祖十戰九不勝,卒有天下」,堅定了燕軍堅持的決心。

這時,朝廷訛傳燕軍已兵敗,京師不可無良將,遂召回徐輝祖。四月廿五,考慮到在河邊不易防守,何福移營,與平安在靈壁(一作靈璧)深溝高壘作長遠之計。由於糧道被燕軍阻礙,平安親自率兵六萬護衛糧草。四月廿七,朱棣率精銳襲擊平安,將其一分為二;何福全軍出動救援,朱高煦也率伏兵出現,何福敗走。

中央軍缺糧,何福與平安決定次日(廿九)突圍而出,在淮河取得給養,號令為三聲炮響;次日,燕軍攻打靈壁墻壘,進攻信號正巧也是三聲炮響。於是中央軍以為是己方號炮,紛紛奪路而逃;燕軍趁勢進攻,中央軍全軍覆沒。靈壁之戰就此意外結束。此戰燕軍生擒了陳暉、平安、馬溥、徐真、孫成等三十七員敵將,四名內官(宦官),一百五十員朝廷大臣,獲馬二萬餘匹,降者不計其數。只有何福單騎逃走。

靈璧之戰後,燕軍向東南方向直線前進。五月初七下泗州,朱棣謁祖陵。盛庸在淮河設下防線阻礙燕軍渡河,朱棣在嘗試取道淮安、鳳陽受阻後,遣朱能、丘福率士兵數百人繞道上游乘漁船渡河,五月初九從後方突襲盛庸,盛庸敗走。燕軍遂克盱眙。

五月十一,燕軍向揚州方向前進,五月十七到達天長(揚州西北50公里)。守揚州的監察御史王彬本想抵抗,但屬下反叛,趁其沐浴時綁縛之。五月十八,揚州不戰而降。隨後高郵歸降。

揚州失陷,金陵震動。朱允炆驚慌不已,與方孝孺商議後,先後定下如下幾個救急方法:下罪己詔;號召天下勤王;派練子寧、黃觀、王叔英等外出募兵;召回被貶黜的齊泰、黃子澄;遣人許以割地求和,拖延時間。。

五月廿二,朱允炆遣慶成郡主(朱元璋的侄女、朱棣的堂姐)與朱棣談判,表示願意割地。朱棣說「此奸臣欲姑緩我,以俟遠方之兵耳。」郡主無言以對,遂返。

六月初一,燕軍準備從浦子口渡江,但遇到了盛庸最後的抵抗。燕軍戰不利,此時朱高煦引兵來援,殊死力戰,擊敗盛庸。隨後南軍的一支水軍部隊降燕,燕軍遂於六月初三自瓜洲渡江,并再次擊敗退守此地的盛庸。六月初六,燕軍至鎮江,守將率城投降。

六月初八,燕軍駐紮於龍潭(距京師金陵東約30公里),朝廷大震。朱允炆徘徊殿間,召方孝孺問計。方孝孺表示城中尚有二十萬兵,應堅守待援;即使真戰敗,國君為社稷而死,是理所應當的。可以再派大臣、在京諸王前往談判以拖延時間。於是六月初九,派李景隆、茹瑺等見朱棣,再次談判;朱棣表示割地無名,只要奸臣。六月初十,遣谷王朱橞(建文元年逃回京城)、安王朱楹等第三次前往談判,無果。

六月十二,外出募兵的大臣們仍未返回,朱允炆只得派在京諸王和武臣們守衛各門。時左都督徐增壽(徐達子,輝祖弟)謀內應,被一群文官圍毆。

次日(1402年7月13日),燕軍抵金陵。徐增壽作內應,事敗,被朱允炆親自誅殺於左順門。守衛金川門(位於南京城西北面)的朱橞和李景隆望見朱棣麾蓋,開門迎降。

燕軍進南京,朱允炆見事不可為,遂在皇宮放火。馬皇后死於大火,朱允炆本人不知所終;此後其下落成為謎團。朱棣入城。

朱棣进入南京,出榜安民,成为了明朝第三位皇帝。朱棣进城之时,翰林院編修楊榮迎於馬首,說:「殿下先謁陵乎?先即位乎?」一语點醒朱棣。次日(建文四年六月十四日)起,諸王及文武群臣多次上表勸進,朱棣不允。

數日後(七月十七日),朱棣謁明孝陵,并於當日登基即位,改元永樂,是為明成祖。明成祖重建奉天殿(舊殿被朱允炆所焚),刻玉璽。同年十一月十三日,封王妃徐氏為皇后。

朱棣登基称帝后,对靖难功臣进行了封赏。封王两人,为:朱能(东平武烈王);张玉(河间忠武王)。封公二十二人,为:丘福(淇国公);徐增寿(定国公);陈亨(泾国公);郭亮(兴国公);李彬(茂国公);李遠(莒国公);柳升(融国公);徐忠(蔡国公);袁容(沂国公);郑亨(漳国公);姚广孝(荣国公);张信(郧国公);王聪(漳国公);顾成(夏国公);张武(潞国公);陳珪(靖国公);薛禄(鄞国公);王真(宁国公);吴允诚(凉国公);李讓(景国公);孟善(滕国公);張輔(英国公)。封侯十五人,为:陳瑄(平江侯);何福(宁远侯)李濬(襄城侯);孙岩(应成侯);房宽(思恩侯);王友(清远侯);王忠(靖安侯);劉榮(广宁侯);火真(同安侯);王寧(永春侯);宋晟(西宁侯);郭义(安阳侯);谭渊(崇安侯);柳升(安远侯);薛绶。封伯十八人,为:陈贤(荣昌伯);陈旭(云阳伯);刘才(广恩伯);张兴(安乡伯);房胜(富昌伯);徐理(武康伯);徐祥(兴安伯);金玉(会安伯);高士文(建平伯);陈志(遂安伯);唐云(新昌伯);茹瑺(忠诚伯);王佐(顺昌伯);许诚(永新伯);薛斌(永顺伯);薛贵(安顺伯);赵彝(忻城伯);朱荣(武进伯)。

明成祖登基后不承認建文年號,七月初一(一說六月十八日),將建文元、二、三、四年改為洪武三十二至三十五年,次年改元永乐元年。凡建文年間貶斥的官員,一律恢復職務(如靖難初期因離間被貶的江陰侯吳高被再次起用,守大同);建文年間的各項改革一律取消;建文年間制定的各項法律規定,凡與太祖相悖的,一律廢除。但一些有利於民生的規定也被廢除,如建文二年下令減輕洪武年間浙西一帶的極重的田賦,至此又變重。

明成祖在靖難之役結束后,为了佐证他“清君侧”的起兵宣言,向金陵軍民發布公告:「諭知在京師的軍民人等,我先前一向守望我藩的封地,卻因奸臣弄權作威作福,導致我家骨肉被其殘害,所以不得不起兵誅殺他們,乃是要扶持社稷和保安宗親、藩王。今次研擬安定京城,有罪的奸臣我不敢赦免,無罪者我也不敢濫殺,如有小人藉機報復,擅作綁縛、放縱、掠奪等事情因而禍及無辜,並非我的本意。」

建文四年六月廿五,明成祖誅殺齊泰、黃子澄、方孝孺等建文帝大臣,滅其族。其中據記載,方孝孺被誅十族(九族加朋友門生),受牽連而死者共873人,充軍等罪者千餘人,當中被救的倖存者有假借余姓逃過一劫的方孝孺的幼子方德宗。而因黃子澄受牽連的有345人。景清降後密謀行刺,事敗,八月十二被殺,滅九族;後屠其家鄉,謂「瓜蔓抄」。

此外,眾多建文舊臣如卓敬、暴昭、練子寧、毛泰、郭任、盧植、戴德彝、王艮、王叔英、謝升、丁志方、甘霖、董鏞、陳繼之、韓永、葉福、劉端、黃觀、侯泰、茅大芳、陳迪、鐵鉉等等也都被酷刑處死或自盡,史稱:「忠憤激發,視刀鋸鼎鑊甘之若飴,百世而下,凜凜猶有生氣。」他們的家屬和親人也被牽連,死者甚眾。被流放、逼作妓女及被其它方式懲罰的人也不少。明仁宗即位後,大部份人始獲赦免,而餘下的人的後代卻遲至明神宗時始獲赦免。建文帝被朱棣篡位後,朝野為之盡忠死節者甚眾,不及備載。

在大肆誅殺之外,當月,明成祖將忠於建文帝的魏國公徐輝祖下獄,但顧及其父是中山王徐達,其姊即成祖仁孝文皇后,還是釋放了他,僅削其爵位。輝祖死後,其子嗣魏國公爵。黃觀被明成祖所嫉恨,其狀元的身份被革去,故明代保持三元及第記錄的只有商輅一人。耿炳文、盛庸、平安(靈壁之戰降)、何福、梅殷等将领投降後都受到迫害自杀身亡。

永乐初,明成祖为了安抚诸位藩王,稳定国内局势,同时表示自己和建文帝的不同,曾先后复周、齐、代、岷诸親王旧封;建文帝的弟弟吴王朱允熥、衡王朱允熞、徐王朱允𤐤尚未就藩,明成祖皆降为郡王,同年又将已就藩的朱允熥、朱允熞召到燕京,以不能匡正建文帝为由废为庶人,软禁于凤阳,仅留朱允𤐤奉祀懿文太子,而朱允𤐤不久也于永乐四年死于火灾。当其皇位较巩固时,继续实行削藩。周、齐、代、岷诸王再次遭到削夺;迁宁王于南昌;徙谷王于长沙,旋废为庶人;削辽王护卫。

在政治上,明成祖继续实行太祖的徙富民政策,以加强对豪强地主的控制。明成祖时期,完善了文官制度,在朝廷中逐渐形成了后来内阁制度的雏形。永乐初开始设置內閣,选资历较浅的官僚入阁参与机务,解决了废罢中书省后行政机构的空缺。朱棣重视监察机构的作用,设立分遣御史巡按天下的制度,鼓励官吏互相讦告。他善利用宦官出使、专征、监军、分镇、刺臣民隐事。

明成祖即位之初,对洪武、建文两朝政策进行了某些调整,提出“为治之道在宽猛适中”的原则。他利用科举制及编修书籍等笼络地主、士人,宣扬儒家思想以改变明初過事佛、道教之风,选择官吏力求因才而用,为当时政治、经济、军事、文化等方面的发展奠定了思想和组织基础。

在全国局势稳定之后,明成祖为了加强对大臣的监控,恢复洪武时废罢的锦衣卫。同时,明成祖又设置镇守内臣的东厂衙门,厂卫合势,强化专制统治。

永乐十八年(1420年),明成祖為了鎮壓政治上的反對力量,觉得锦衣卫不足以达成目的,決定設立一個稱為「東緝事廠」,簡稱“東廠”的新衙門,地點位於燕京(今北京)東安門之北,一說東華門旁。(今北京东城区东厂胡同,據說系原东厂所在地。)

東廠的行政長官為欽差掌印太監,全稱職銜為:欽差總督東廠官校辦事太監,簡稱提督東廠,尊稱為「廠公」或「督主」。初設時由司禮監掌印太監兼任,後因事務繁雜,改由司禮監秉筆太監中位居第二、第三者擔任。東廠的屬官有掌刑千戶、理刑百戶各一員,由錦衣衛千戶、百戶來擔任,稱貼刑官。隸役(稱掌班、領班、司房,共四十餘人)、緝事(稱役長和番役)等軍官由锦衣卫撥給。

明初《大明律》明令:「凡樂人搬做雜劇戲文,不許妝爾扮帝王后妃、忠臣節烈、先聖先賢神像,違者杖一百。官民之家容扮者與同罪」,以壓迫雜劇創作,明成祖即變本加厲,以極刑來禁止此類雜劇的印賣:「但有褻瀆帝王聖賢之詞曲、駕頭雜劇,非該律所載者,敢有收藏、傳誦、印賣,一時拿送法司究治」,「但這等詞曲,出榜後,限他五日,都要乾淨,將赴官燒毀了,敢有收藏的,全家殺了」。

明成祖十分重視經營北方,加之自己兴起于北平(今北京),明成祖在南京即位后,于永乐元年改北平為行在,設六部,增設北京周圍衛所,逐漸建立起北方新的政治、軍事中心。永乐七年(1409年),明成祖开始了營建北京天壽山長陵,以示立足北方的決心。與此同時,爭取與蒙古族建立友好關係。韃靼、瓦剌各部先後接受明政府封號。永乐八年(1410年)至二十二年(1424年),朱棣親自率兵五次北征,鞏固了北部邊防。永乐十四年(1416年)開工修建北京宮殿也就是紫禁城(但後來部分宮殿被李自成放火燒毀,清初又重新修復)。永乐十九年(1421年)正式遷都,定鼎北京。

明成祖注意社会经济的恢复与发展,认为“家给人足”、“斯民小康”是天下治平的根本。他大力发展和完善军事屯田制度和盐商开中则例,保证军粮和边饷的供给。在中原各地鼓励垦种荒闲田土,实行迁民宽乡,督民耕作等方法以促进生产,并注意蠲免赈济等措施,防止农民破产,保证了赋役征派。

明成祖对各地方官吏要求极为严格,要求凡地方官吏必须深入了解民情,随时向朝廷反映民间疾苦。永乐十年(1412年),朱棣命令入朝觐见的地方官吏五百余人各自陈述当地的民情,还规定“不言者罪之,言有不当者勿问’。之后,永乐帝宣布“谕户部,凡郡县有司及朝使目击民艰不言者,悉逮治。”即地方官或中央派出的民情观察员,如果看到民间疾苦而不实报的,要逮捕法办。对民间发生了灾情,地方上要及时赈济,做到“水旱朝告夕振,无有雍塞”。通过这些措施,永乐时“赋入盈羡”,达到有明一代最高峰,史称永乐盛世。

西南边疆,永乐十一年(1413年),平定思南、思州土司叛亂後,設立貴州布政使司。為加強對烏思藏(今西藏)地區的控制,朱棣派遣官吏迎番僧入京,給予封賜,尊為帝師。不過,史學界對明朝是否實際統治了西藏存在較大的爭議。

永乐年间,明朝在藏区建立一套僧官制度,僧官分教王、西天佛子、大国师、国师、禅师、都纲、喇嘛等,每级依受封者的身份、地位进行分封。如明成祖即位的当年,即派侯显前往乌思藏迎请噶玛噶举派的第五世噶玛巴活佛,后封其为“大宝法王”。1406年,明成祖又遣使入藏封乌思藏帕竹第五任第悉扎巴坚赞为“阐化王”。明封八王中的两大法王、五大教王都是永乐时期封授的。此外,明成祖依僧官制度还进行了大规模的分封,由此明朝对藏区的各政教势力由上至下各级首领的分封基本完成。但明朝并未在烏思藏等地区驻军。亦有学者通过对比元朝对于西藏的实际管辖,认为明朝上面这些对藏人名义上的封授并不能被认为拥有在西藏的实际政治权力。《劍橋中國明代史》亦指出:「無論是在經濟領域,還是在政治領域,西藏人都未覺得他們是明朝廷臣民。另外,他們無須中國(明朝)居中調解而維持著與其他國家和民族的關係。」

东北边疆,永乐七年(1409年)在女真地區,設立奴儿干都司。明成祖永乐元年(1403年)派邢枢等传谕奴儿干,正式招抚诸部,擴大明朝東疆。永乐二年(1404年),置奴儿干等卫所,其后在当地相继建卫所达一百三十餘。永乐七年(1409年)明政府设置奴儿干都指挥使司管辖奴儿干地区的所有军事建制机构。永乐九年(1411年)正式开始行政管辖权。都司的主要官员初为派駐數年而輪調的流官,后为當地部落領袖所世袭。明成祖為了安撫東北女真各部,在歸附的海西女真(位於松花江上游)與建州女真(位於松花江、牡丹江之間)設置衛所,並派宦官亦失哈安撫位於黑龍江下游的野人女真。亦失哈并于1413年视察了库页岛,宣示了明朝对此地的主权。在奴儿干都司官衙所在地附近建有永宁寺,立有永宁寺碑,清代曹廷杰于1885年曾拓回碑文。同时,明成祖撤去大宁都司,将宁王朱权内迁南昌,授予兀良哈蒙古的朵颜、泰宁和福余三个卫所自治权,但不允许三卫蒙古人南迁到大宁地区驻牧。明成祖还于1406年和1422年对兀良哈蒙古进行镇压,以维持这一地区的稳定。

辖区内主要居民为蒙古、女真、吉里迷(尼夫赫人)、苦夷(阿伊努人)、达斡尔等族人民,分置卫所,以各族首领为各卫所都督、都指挥、指挥、千户、百户、镇抚等职,给予印信。据《明史》记载,奴儿干都司有卫三百八十四,所二十四,站七,地面七,寨一。都司治所奴儿干城(元朝征东元帅府旧地,今俄罗斯尼古拉耶夫斯克特林),在黑龙江下游东岸,下距黑龙江口约两百公里,上距吉林船厂约两千五百公里。明宣宗即位后,奴儿干都司于宣德九年(1434年)正式废弃,共持续25年。

西北边疆,永乐四年(1406年)設立哈密衛。此前,察合台的后裔肃王兀纳失里於明洪武十三年(1380年),开始向明朝纳贡,被明太祖封为哈密国王。其子脱脱向明成祖朝贡,永乐四年(1406年)三月,明成祖宣布设立哈密卫,以其头目马哈麻火者等为指挥、千百户等官,又以周安为忠顺王长史,刘行为纪善,辅导。之后,哈密国成为设有明朝羁縻卫所的王国,忠顺王是哈密国王,哈密卫指挥使掌握哈密兵权,另有汉人长史。

同时,明成祖还多次派遣吏部驗封司員外郎陳誠、中官李達等官員出使西域。隨後西域的帖木兒帝國、吐魯番、失剌斯、俺都準、火州也與明朝多次互派使者往來,加強了政治、駐軍和貿易往來,全國統一形勢得到進一步發展和鞏固。

明成祖很重视河工,永乐九年(1411年)朱棣於疏浚會通河為保證北京糧食與各項物資的需要。朱棣命開漕運。漕運在元朝至元年間即有,然而卻因會通河一段水淺而無法大量載運物資,於是元朝均以海運為主。明朝初期,傳餉遼東、北平的途徑也均以海運為主。洪武二十四年,黃河在原武絕口,會通河於是被淤。

永乐年间,明成祖遷都北京,採用河路、海路并運。當時海運危險且多有損失;而河運卻經過淮河轉沙河,然後經過黃河進入衛河,於此轉入北京,陸運須經過八個衛所,勞民傷財。濟寧州同知潘叔正上疏建議浚通會通河,使得元朝運河恢復。於是,朱棣命宋禮、刑部侍郎金純、都督周長前往治理。會通河首要問題為水源不足,宋禮採用汶上老人白英的建議,修築埋城與戴村坝,橫截汶水向南,經河面最高端南旺分水,流入運河,且使黃河不會影響漕運。同年八月還京,論首功,受上賞。

次年,因御史許堪進言衛河水患,朱棣再命宋禮前往治理。宋禮在魏家灣分支黃河,泄水入土河,於是從德州西北開一支支流,到海豐、大沽流入大海。此時,宋禮以海運損失巨大、勞民傷財,上言請求停止海運,而恰逢平江伯陳瑄治理長江、淮河等告竣。於是河運從此昌盛,可運大型物資。永樂十三年,朱棣遂終止海運。

永乐十三年(1415年)鑿清江浦,使大運河重新暢通,對南北經濟文化交流與發展起了重要的作用。

永乐年间,明成祖还派派夏原吉治水江南,疏浚吴淞。

在政治稳定、经济繁荣、边疆稳定的局面下,为整理知识,明成祖令解縉等人修书。編撰宗旨:「凡书契以來经史子集百家之书,至於天文、地志、阴阳、医卜、僧道、技艺之言,备辑为一书,毋厌浩繁!」,召集一百四十七人,首次成书于永乐二年(1404年),初名《文献集成》;明成祖過目後認為「所纂尚多未備」,不甚滿意。永樂三年(1405年)再命姚廣孝、鄭賜、劉季篪、解縉等人重修,這次動用編寫人員朝野上下共二千一百六十九人,啟用了南京文淵閣的全部藏書,永樂五年(1407年)定稿進呈,明成祖看了十分滿意,親自為序,並命名為《永樂大典》,清抄至永樂六年(1408年)冬天才正式成書。

《永乐大典》由解縉、太子少傅姚廣孝和禮部尚書鄭賜監修,組織上設監修、總裁、副總裁、都總裁等職,負責各方面工作。監修:解縉、姚廣孝、鄭賜;總裁:副總裁:蔣用文、趙同友;都總裁:陳濟。

《永乐大典》修書過程對所收錄的書籍沒有做任何修改,採用兼收並取的方式,保持了書籍原始的內容。明成祖修大型類書《永樂大典》,在三年時間內即告完成。《永樂大典》有22877卷,其中凡例、目錄60卷,全書分裝為11095冊,引書達七八千種,字數約有三億七千多萬,且未有任何刪節,這是清朝《四庫全書》無法相提並論的。但成祖并未将《永乐大典》复写刊刻,而决定只制作一份抄本,并于1409年完成。永乐年間修訂的《永樂大典》原書只有一部,現今存世的都是嘉靖年間的抄本。

明成祖时期,为了开展对外交流,扩大明朝的影响,同时确立自己即位的正统性,从永乐三年起,朱棣派三宝太监郑和(初名馬三寶)率领船队六次出使西洋(第七次在明宣宗宣德年间),所历三十余国,成为明初盛事。永乐时派使臣来朝者亦达三十余国。浡泥王和苏禄东王亲自率使臣来中国,不幸病故,分别葬于南京(浡泥国王墓)和德州(苏禄国王墓)。

永乐三年六月十五(1405年7月11日)明成祖命郑和为正使,王景弘为副使率士兵二万八千余人出使西洋,造长44丈广18丈大船62艘,从苏州刘家河泛海到福建,再由福建五虎门杨帆,先到占城(今越南中南部地區),后向爪哇方向南航,次年6月30日在爪哇三宝垄登陆,进行贸易。时西爪哇与东爪哇内战,西爪哇灭东爪哇,西爪哇兵杀郑和士兵170人,西王畏惧,献黄金6万两,补偿郑和死难士兵。随后到三佛齐旧港,时旧港广东侨领施进卿来报,海盗陈祖义凶横,郑和兴兵剿灭贼党五千多人,烧贼船十艘,获贼船五艘,生擒海盗陈祖义等三贼首。郑和船队后到过苏门答腊、满刺加、锡兰、古里等国家。在古里赐其王国王诰命银印,并起建碑亭,立石碑“去中国十万余里,民物咸若,熙嗥同风,刻石于兹,永示万世”。

永乐五年九月初二(1407年10月2日),郑和回国,押陈祖义等献上,陈祖义等被问斩。施进卿被封为旧港宣慰使。旧港擒贼有功将士获赏:指挥官钞一百锭,彩币四表里,千户钞八十锭,彩币三表里,百户钞六十锭,彩币二表里;医士,番火长钞五十锭,彩币一表里,锦布三匹。

永乐六年正月,明成祖命工部造宝船四十八艘。永乐六年九月十三日(1407年10月13日),命太监郑和、王景弘,王贵通等出使古里,满剌加,苏门答剌,阿鲁,加异勒,爪哇,暹罗,占城,柯枝,阿拔把丹,小柯兰,南巫里,甘巴里等国,赐其国王锦绮纱罗,永乐七年夏(1409年)回国。第二次下西洋人数据载有27000人。

永乐七年九月(1409年10月),明成祖命正使太监郑和、副使王景弘、候显率领官兵二万七千余人,驾驶海舶四十八艘,从太仓浏家港启航,敕使占城,宾童龙,真腊,暹罗,假里马丁,交阑山,爪哇,重迦罗,吉里闷地,古里,满剌加,彭亨,东西竺,龙牙迦邈,淡洋,苏门答剌,花面,龙涎屿,翠兰屿,阿鲁,锡兰,小葛兰,柯枝,榜葛剌,不剌哇,竹步,木骨都束,苏禄等国。費信、馬歡等人會同前往。满剌加当时是暹罗属国,正使郑和奉帝命招敕,赐双台银印,冠带袍服,建碑封域为满剌加国,暹罗不敢扰。满剌加九洲山盛产沉香,黄熟香;太监郑和等差官兵入山采香,得直径八九尺,长八九丈的标本6株。永乐七年,皇上命正使太监郑和等赍捧诏敕金银供器等到锡兰山寺布施,并建立《布施锡兰山佛寺碑》此碑現存于科倫坡博物館。郑和访问锡兰山国时,锡兰山国王亞烈苦奈兒“負固不恭,謀害舟師”,被郑和觉察,离开锡兰山前往他国。回程时再次访问锡兰山国,亚烈苦奈儿诱骗郑和到国中,发兵五万围攻郑和船队,又伐木阻断郑和归路。郑和趁贼兵倾巢而出,国中空虚,带领随从二千官兵,取小道出其不意突袭亚烈苦奈儿王城,破城而入,生擒亚烈苦奈儿并家属。

永乐九年六月十六(1411年7月6日),郑和回国獻亚烈苦奈儿与永樂帝,朝臣齐奏诛杀,永樂帝怜悯亚烈苦奈儿无知,释放亚烈苦奈儿和妻子,给予衣食,命礼部商议,选其国人中贤者为王。选贤者邪把乃耶,遣使赍引,诰封为锡兰山国王,并遣返亚烈苦奈儿。永乐九年(1411年)满剌加国王拜里米苏剌,率领妻子陪臣540多人来朝,朝廷赐海船回国守卫疆土。从此“海外诸番,益服天子威德”。八月,礼部、兵部议奏,对锡兰战役有功将士754人,按奇功,奇功次等,头功,头功次等,各有升职,并赏赐钞银,彩币锦布等。

永乐十一年十一月(1413年11月),明成祖命正使太监郑和,副使王景弘等奉命统军二万七千余人,驾海舶四十,出使满剌加,爪哇,占城,苏门答剌,柯枝,古里,南渤里,彭亨,吉兰丹,加异勒,勿鲁谟斯,比剌,溜山,孙剌等国。郑和使团中包括官员868人,兵26800人,指挥93人,都指挥2人,书手140人,百户430人,户部郎中1人,阴阳官1人,教谕1人,舍人2人,医官医士180人,正使太监7人,监丞5人,少监10人,内官内使53人其中包括翻译官马欢,陕西西安羊市大街清真寺掌教哈三,指挥唐敬,王衡,林子宣,胡俊,哈同等。郑和先到占城,奉帝命赐占城王冠带。1413年郑和船队到苏门答剌,当时伪王苏干剌窃国,郑和奉帝命统率官兵追剿,生擒苏干剌送京伏诛。1413年郑和舰队在三宝垄停留一个月整休,郑和费信常在当地华人回教堂祈祷。郑和命哈芝黄达京掌管占婆华人回教徒。首次繞過阿拉伯半島,航行東非麻林迪(肯尼亚),永乐十三年七月初八(1415年8月12日)回国。同年11月,麻林迪特使來中國進獻“麒麟”(即長頸鹿)。

永乐十五年五月十五日(1417年6月)总兵太监郑和受明成祖命,在泉州回教先贤墓行香,往西洋忽鲁谟斯等国公干,永乐十五年五月(1417年6月)出发,护送古里、爪哇、满剌加、占城、锡兰山、木骨都束、溜山、喃渤里、卜剌哇、苏门答剌、麻林、剌撒、忽鲁谟斯、柯枝、南巫里、沙里湾泥、彭亨各国使者及旧港宣慰使归国。隨行有僧人慧信,将领朱真、唐敬等。郑和奉命在柯枝诏赐国王印诰,封国中大山为镇国山,并立碑铭文。忽鲁谟斯进贡狮子,金钱豹,西马;阿丹国进贡麒麟,祖法尔进贡长角马,木骨都束进贡花福鹿、狮子;卜剌哇进贡千里骆驼、鸵鸡;爪哇、古里进贡麾里羔兽。永乐十七年七月十七(1419年8月8日)回国。

宋末泉州市舶司提举蒲寿庚之侄蒲日和,也与太监郑和,奉敕往西洋寻玉玺,有功,加封泉州卫镇抚。

永乐十九年正月三十日(1421年3月3日),郑和奉明成祖命出发,往榜葛剌(孟加拉),史載“於鎮東洋中,官舟遭大風,掀翻欲溺,舟中喧泣,急叩神求佑,言未畢,……風恬浪靜”,中道返回,永乐二十年八月十八(1422年9月2日)回国。永樂二十二年,明成祖去世,仁宗朱高熾即位,以經濟空虛,下令停止下西洋的行動。

永乐二十二年七月十七日(1424年8月12日),明成祖去世,太子朱高炽即位,改元洪熙,是为明仁宗,于洪熙元年五月辛巳(1425年5月29日)去世,太子朱瞻基即位,改元宣德,是为明宣宗。宣德五年闰十二月初六(1430年1月),郑和奉明宣宗命率领二万七千余官兵,驾驶宝船61艘,从龙江关(今南京下关)启航,进行了第七次下西洋。开始返航后,郑和因劳累过度于宣德八年(1433年)四月初在印度西海岸古里去世,遺體埋葬於古里,船队由太监王景弘率领返航,宣德八年七月初六(1433年7月22日)返回南京。第七次下西洋人数据载有27550人。

明太祖朱元璋為與鄰近國家保持長久的和睦關係,便在其所主編的《皇明祖訓》中開列十五個「不征諸夷國名」,以警戒後世子孫切勿「倚中國富強,貪一時戰功,無故興兵,致傷人命」,越南(安南國)便是其中之一。1400年,安南陳朝權臣胡季犛篡位,建立胡朝,改國號為「大虞」。不久後自稱太上皇,由兒子胡漢蒼(即胡𡗨)即皇帝位。由於前朝陳氏原是向明朝稱臣,世世受明冊封,憑著篡奪得國的胡氏為免惹起明朝猜疑,便於1403年農曆四月丁未(西曆4月21日)遣使赴明,向剛起兵奪位的明成祖聲稱陳氏「宗嗣繼絕,支庶淪滅,無可紹承。臣,陳氏之甥,為眾所推」,欲藉此聲稱自己具有統治資格,要求明朝冊封。明成祖派楊渤到越南觀察後,當地陪臣耆老跟隨他向成祖上奏稱「眾人誠心推𡗨權理國事」,明廷一時再沒有懷疑的理由,便封胡漢蒼為「安南國王」。

1404年農曆八月乙亥(西曆9月10日),陳朝遺臣裴伯耆到明廷,控訴胡季犛父子「弒主篡位,屠害忠臣」,要求明朝出兵「擒滅此賊,蕩除奸凶,復立陳氏子孫」 八月丁酉日(西曆10月2日),有一位自稱陳氏子孫,名叫陳天平的人(越南史籍寫作「陳添平」,《大越史記全書》稱他的身份本是「陳元輝家奴阮康」),從老撾入明,亦向明帝訴說胡氏篡位的經過,要求恢復陳氏統治。 其後,明成祖當著胡朝的來使面前,安排陳天平與他們會見,使一眾來使都錯愕下拜,甚至涕泣,適值裴伯耆在場,向來使責以大義,場面緊張。 明廷於是對越南政局多所干涉,派員查核實情,胡朝明白勢不得已,唯有承認責任,要求「迎歸天平」。

另外,明越兩國又因領土問題出現外交風波。1405年,廣西省思明土官及雲南省寧遠州土官向明廷控訴,轄境猛慢、祿州等地被越南所佔。為此,明廷於該年農曆二月,遣使責難胡朝,要求取得祿州,胡朝便被迫將古樓等五十九村交給明朝政府。

胡朝雖然願意息事寧人,但兩國關係仍然緊張。其後,胡朝所派到明廷的使節,都遭扣留,不許回國。明廷又派員入越,查探山川道路險要之地,以為日後南征的準備。 另外,胡朝的南鄰占城,曾於1404年遣使入明,聲稱遭到胡氏「攻擾地方,殺掠人畜」,並進一步「請吏治之」, 這亦引起了明廷的注意。

不過,明成祖仍未敢輕言出兵。1405年年底,雲南將領沐晟建議出兵,卻遭明成祖反駁說:「爾又言欲發兵向安南。朕方以布恩信,懷遠人為務。胡𡗨雖擾我邊境,令已遣人詰問,若能攄誠順命,則亦當弘包荒之量。」 至於陳天平的處置,明廷則決定送歸越南,並要求越人「以君事之」,奉為國主。 越南方面,胡朝有感於對明關係緊張,亦積極防備,重編軍制,在多邦城(陳仲金說位於山西省先豐縣古法社)加強防守,於各個河海要處裝插木樁陷阱,整頓軍庫,招募人民有巧藝者入伍。但胡朝君臣對明主戰或主和,意見分歧甚大,有官員認為只好「從他(明朝)所好,以緩師可也」,左相國胡元澄則認為只決定於「民心之從違耳」,對明作戰並無十足把握。

1406年,明朝派鎮守廣西都督僉事黃中領五千士兵(《大越史記全書》稱領兵十萬),護送陳朝王孫陳天平(陳添平)回越南(《明實錄》把事件列在該年農曆三月丙午,即西曆4月4日;《大越史記全書》則列入農曆四月八日,即西曆4月26日)。當進入越南境內的支棱隘時,遇上胡軍截擊,明軍不敵,陳天平及部份士兵被俘。陳天平經胡朝審訊後,被「處陵遲罪」。明成祖得悉後大怒,便「決意興師」。

同年年中,明成祖派總兵官朱能加封「征夷大將軍」,配印信。後來在行軍時病卒,由副將張輔代替)、左副將軍沐晟、右副將軍張輔、左參將李彬、右參將陳旭等領兵(《大越史記全書》稱共有八十萬人,中國學者郭振鐸、張笑梅認為可能有誇大),分兵兩路,開進越南的白鶴江會師,一邊向越南腹地步步推進,一邊發出檄文向越人呼籲胡季犛父子的行為是「肆逞凶暴,虐于一國」,並列出胡氏「兩弒前安南國王以據其國」、「賊殺陳氏子孫宗族殆盡」、「淫刑峻法,暴殺無辜,重斂煩徵,剝削不已」等二十款大罪,又稱明軍的到來是「吊爾民之困苦,復陳氏之宗祀」,以使民心動搖。果然,不少越人「厭胡氏苛政,罔有戰心」,有助明軍前進更為順利。農曆十二月丙申十一日(西曆1407年1月19日),胡軍的主力退守多邦城,明軍亦看準該城位於河邊,有較大面積的沙灘可供搶灘,於是分兵進攻,成功以火銃擊退胡軍象兵。其後,明軍攻入越南的重要城市東都昇龍,並大肆掠奪,「擄掠女子玉帛,會計粮儲,分官辦事,招集流民。為久居計,多閹割童男,及收各處銅錢,驛送金陵」。

1407年年初,明軍攻破昇龍後,向胡朝的首都清化繼續前進,胡氏皇子胡元澄領軍退守黃江(在今越南河南省的一段紅河),與胡季犛、胡漢蒼會合。明將沐晟則進駐木凡江(在今越南河内市,與黃江相接)預備出擊。農曆二月,沐晟沿江兩岸擊敗胡元澄軍,追擊至悶海口(在今越南南定省),因軍中爆發疾疫,明軍移師到鹹子關立塞備戰。農曆三月,胡軍集合水步大軍七萬,號稱二十一萬,與明軍爆發鹹子關之戰。結果胡軍潰敗,大批兵士溺斃於該處河流,無數船隻及軍糧沉沒,胡氏父子敗逃,最終在農曆五月十一日(西曆6月16日)在奇羅海口(在今越南河靜省奇英縣)被明軍俘獲,胡朝滅亡,領土被明朝佔領。據當時的統計,越南土地人口物產資料為:府州四十八、縣一百六十八、戶三百一十二萬九千五百、象一百一十二、馬四百二十、牛三萬五千七百五十、船八千八百六十五。(※此一統計數字,按《明實錄》記載的1408年農曆六月的計算,則是「安撫人民三百一十二萬有奇;獲蠻人二百八萬七千五百有奇,糧儲一千三百六十萬石,象、馬、牛共二十三萬五千九百餘隻,船八千六百七十七艘,軍器二百五十三萬九千八百五十二件。」)

胡朝亡後,明成祖在農曆六月癸未朔(西曆7月5日)下詔,聲稱這次軍事行動是為了越南原本的陳氏王室著想,「期伐罪(指胡朝)以吊民,將興滅而繼絕」,並打算對「久染夷俗」的越人「設官兼治,教以中國禮法」,以達致「廣施一視之仁,永樂太平之治」。明廷又以陳朝子孫被胡氏殺戮殆盡,無可繼承,於是在越南設置交址都指揮使司、交址等處承宣布政使司及交址等處提刑按察使司等官署,將之直接管轄。

安南内属后,安南人民不断进行反抗,明军多次进行镇压。永乐二十二年(1424年),明成祖去世,太子朱高炽明仁宗即位,次年明仁宗去世,太子朱瞻基即位,是为明宣宗。宣宗考慮到「數年以來,一方不靖,屢勤王師」, 便允許撤兵。黎利得勝後,就發佈阮廌所起草的《平吳大誥》,稱他自己的抗明鬥爭是「仁義之舉,要在安民,吊伐之師,莫先去暴」;提出中越兩國是「山川之封域既殊,南北之風俗亦異」,因而有必要脫離明朝統治,自行建國,於是建立後黎朝。

其後,1431年農曆正月五日(西曆2月12日),明封黎利為安南國王,從此朝貢不絕。

为了稳定北方边境,对付蒙古势力。永乐七年(1409年),明成祖朱棣派淇国公丘福率十万大军征讨鞑靼,由于轻敌,孤军深入,中埋伏,全军覆没。为消除边患,明成祖决心亲征。明永乐八年(1410年)二月,明成祖调集50万大军。五月八日,明军行至胪朐河(今克鲁伦河,朱棣将之更名为“饮马河”)流域,询得鞑靼可汗本雅失里率军向西逃往瓦剌部,丞相阿鲁台则向东逃。朱棣亲率将士向西追击本雅失里,五月十三日,明军在斡难河(位于今蒙俄边境)大败本雅失里。朱棣打败本雅失里后,挥师向东攻击阿鲁台,双方在今蒙俄边境之斡难河东北方向交战,明军杀敌无数,阿鲁台坠马逃遁。此时天气炎热,缺水,且粮草不济,朱棣下令班师。鞑靼部经过明军的这次打击,臣服了明朝,当年向明成祖进贡马匹。成祖亦给予优厚的赏赐,其部臣阿鲁台接受了成祖给他“和宁王”的封号。

明军在永乐八年(1410年)第一次出征鞑靼后,瓦剌部趁机迅速发展壮大,1413年,瓦剌军进驻胪朐河(今克鲁伦河),窥视中原。明成祖决心再次亲征,调集兵力,筹集粮饷。永乐十二年(1414年)二月,明军从北京出发,六月初三,明军在三峡口(今蒙古乌兰巴托东南)击败了瓦剌部的一股游兵,杀敌数十騎;初七日,明军行至勿兰忽失温(今蒙古乌兰巴托东南),瓦剌军3万之众,依托山势,分三路阻抗,朱棣派骑兵冲击,引诱敌兵离开山势,遂命柳升发炮轰击,自己亦亲率铁骑杀入敌阵,瓦剌军败退,朱棣乘势追击,兵分几路夹击瓦剌军的所扑,杀敌数千,瓦剌军纷纷败逃。此役,瓦剌受到了重创,此后多年不敢犯边,同时,明军也伤亡惨重。

瓦剌被明成祖打败,鞑靼趁此机会经过几年的发展,势力日益强盛起来,从而改变对明朝的依附政策,并侮辱或拘留没明朝派去的使节,还时常对明朝边境进行骚扰的劫掠。永乐十九年(1421年)冬初,鞑靼围攻明朝北方重镇兴和,杀死了明军指挥官王祥,对此,朱棣决定第三次亲征漠北。永乐二十年(1422年)三月,明成祖率軍从北京出发,出击鞑靼。其主力部队至宣府(今河北宣化)东南的鸡鸣山时,鞑靼首领阿鲁台得知明军来袭,乘夜逃离兴和,避而不战。七月,明军到达煞胡原,俘获鞑靼的部属,得知阿鲁台已逃走,朱棣下令停止追击。明军在回师途中,击败兀良哈部,九月,回师北京。明成祖第三次出击漠北,虽对鞑靼部有一定的打击,但成效不大,并没彻底解决盘据漠北的蒙古三个部落对明朝边境的滋扰。

永乐二十一年(1423年),鞑靼首领阿鲁台再次率部滋扰明朝边境,明成祖闻悉后决定再次亲征。明军八月初出征,九月上旬,明军到达沙城(今河北张北以北)时,阿鲁台的部下阿失贴木儿率部投降明军,并得知阿鲁台被瓦剌打败,其部已溃散,明军暂时驻扎不前;十月,明军继续北上,在黄河以北击败鞑靼西部的军队,鞑靼王子也先土干率部众来降明,明成祖朱棣随即封也先土干为忠勇王,十一月,明军班师回京。

永乐时全国形势相对缓和,但由于国家支出过大,赋役征派繁重,使有些地区发生了农民流亡与起义,十八年山东发生的唐赛儿起义是其中规模较大的一支。明永乐二十二年(1424年)正月至七月,明軍对蒙古鞑靼部的作戰。是年正月,鞑靼部首领阿鲁台率軍進犯明山西大同、开平(今内蒙古正兰旗东北)等地。明成祖朱棣遂调集山西、山东、河南、陕西、辽东5都司之兵于京师(今北京)和宣府(今河北宣化)待命。四月三日,以安远侯柳升、遂安伯陈英为中军;武安侯郑亨、保定侯盂瑛为左哨,阳武侯薛禄、新宁伯谭忠为右哨;英国公张辅、成国公朱勇为左掖,成山侯王通、兴安伯徐亨为右掖;宁阳侯陈懋、忠勇王金忠又名也先土干为前锋,出兵北征。出征前戶部尚書夏元吉以國庫虛耗,曾勸他勿起戰事,但他不聽,反繫之大獄。二十五日,进至隰宁(今河北沽源南),获悉阿鲁台逃往答兰纳木儿河(今蒙古境内之哈剌哈河下游),明成祖令全军急速追击。六月十七日,进至答兰纳木儿河,周围300余里不见阿鲁台部踪影,遂下令班师。

明成祖為填補太祖廢除丞相後導致六部之首的空缺,但又希望強化皇權,他设立内阁,内阁大學士计有解縉、黃淮、胡廣、楊榮、金幼孜、楊士奇、胡儼。明成祖时期涌现许多著名大臣,包括蹇义、郁新、刘观、郑赐、宋礼、金纯、夏原吉、吕震、金忠、沐春、沐晟、沐昂。

明成祖任用酷吏强化自己的统治,著名的包括陳瑛和紀綱。

明成祖时期的著名太监包括:鄭和:三宝太監七下西洋;王景弘:鄭和的副手;侯顯:有才辨,強力敢任,五使絕域,勞績與鄭和亞;亦失哈:鞏固北方邊防,晚年研究改造武器,如改造步槍(裝槍頭-為安裝刺刀的先驅);王彦:原名王狗兒,尚寶監太監;昌盛:神宫監太監,貴州人。歷洪武-建文-永樂-洪熙-宣德五朝。

永樂二十二年(1424年)七月,明成祖率領北征大軍班師返京。七月十五日,明成祖病重。十六日,行至榆木川(今内蒙古多伦),昏迷不醒。十八日,明成祖朱棣崩逝於榆木川(今中國內蒙古自治區錫林郭勒盟多倫縣),享壽六十四岁,在位二十二年。遗诏传位皇太子。大學士楊榮、太監馬去等秘不發喪,暗中派御馬監少監海壽秘密回京,“奉遗命,驰讣皇太子”。太子朱高熾立即派皇太孫前往虎帐。八月十一日,皇太孫到達軍營後,始發佈帝崩消息。太子朱高炽即位,宣布次年改元洪熙,是为明仁宗。明成祖驾崩后,殉葬的有30余位宫女,其中包括成祖的16位嫔妃。

明成祖驾崩后,谥体天弘道高明广运圣武神功纯仁至孝文皇帝,庙号太宗,十二月十九日,明成祖与仁孝文皇后徐氏合葬于长陵。嘉靖十七年(1538年)九月,明世宗朱厚熜改谥明成祖为启天弘道高明肇运圣武神功纯仁至孝文皇帝,改上庙号为成祖。

《明史·成祖本纪》中评价明成祖:文皇少长习兵,据幽燕形胜之地,乘建文孱弱,长驱内向,奄有四海。即位以后,躬行节俭,水旱朝告夕振,无有壅蔽。知人善任,表里洞达,雄武之略,同符高祖。六师屡出,漠北尘清。至其季年,威德遐被,四方宾服,明命而入贡者殆三十国。幅陨之广,远迈汉、唐。成功骏烈,卓乎盛矣。然而革除之际,倒行逆施,惭德亦曷可掩哉。

蔡石山在其著作《永乐大帝:一个中国帝王的精神肖像》的开篇评价明成祖“明朝的永乐皇帝,驾崩于1424年8月12日,自从1402年7月17日登极以来——近乎八千零六十二天的在位期间——而且所有的证据也显示,他从未浪费过一天”。在书末,他再次评价明成祖“毋庸置疑,永乐有过多的自我,而且拥有很多的美德:他是自信、直率的,能够甄别和牢记有很强能力之人的贡献,而且保护依靠他的那些人,尤其是他的家人。不过,他也有黑暗面,特征就是不必要又未经思考的侵犯性,而这类侵犯性经常产生了暴虐和消耗”。

《朝鲜王朝实录·世宗庄宪大王实录》中评价明成祖「使臣言:"前後選獻韓氏等女,皆殉大行皇帝。" 先是,賈人子呂氏入皇帝宮中,與本國呂氏以同姓,欲結好,呂氏不從,賈呂蓄憾。 及權妃卒,誣告呂氏點毒藥於茶進之,帝怒,誅呂氏及宮人宦官數百餘人。 後賈呂與宮人魚氏私宦者,帝頗覺,然寵二人不發,二人自懼縊死。 帝怒,事起賈呂,鞫賈呂侍婢,皆誣服云:"欲行弑逆。" 凡連坐者二千八百人,皆親臨剮之,或有面詬帝曰:"自家陽衰,故私年少寺人,何咎之有?" 後帝命畫工圖,賈呂與小宦相抱之狀,欲令後世見之,然思魚氏不置,令藏於壽陵之側。 及仁宗卽位,掘棄之。 亂之初起,本國任氏、鄭氏自經而死,黃氏、李氏被鞫處斬。 黃氏援引他人甚多,李氏曰:"等死耳,何引他人爲? 我當獨死。" 終不誣一人而死。 於是,本國諸女皆被誅,獨崔氏曾在南京,帝召宮女之在南京者,崔氏以病未至,及亂作,殺宮人殆盡,以後至獲免。 韓氏當亂,幽閉空室,不給飮食者累日,守門宦者哀之,或時置食於門內,故得不死。 然其從婢皆逮死,乳媪金黑亦繫獄,事定乃特赦之。 初,黃氏之未赴京也,兄夫金德章坐於所在房窓外,黃儼見之大怒,責之,及其入朝,在道得腹痛之疾,醫用諸藥,皆無效,思食汁菹。 儼問元閔生曰:"此何物耶?" 閔生備言沈造之方,儼變色曰:"欲食人肉,吾可割股而進,如此草地,何得此物?" 黃氏腹痛不已,每夜使從婢以手磨動其腹,到一夜小便時,陰出一物,大如茄子許,皮裹肉塊也。 婢棄諸廁中,一行衆婢,皆知而喧說。 又黃氏婢潛說:"初出行也,德章贈一木梳。" 欽差皆不知之。 帝以黃氏非處女詰之,乃云:"曾與姊夫金德章、隣人皂隷通焉。" 帝怒,將責本國,勑已成,有宮人楊氏者方寵,知之,語韓氏其故,韓氏泣乞哀于帝曰:"黃氏在家私人,豈我王之所知也?" 帝感悟,遂命韓氏罰之,韓氏乃批黃氏之頰。 明年戊戌,欽差善才謂我太宗曰:"黃氏性險無溫色,正類負債之女。" 歲癸卯,欽差海壽謂上曰:"黃氏行路之時,腹痛至甚,吾等見則以鄕言言腹痛,必慙而入內。" 及帝之崩,宮人殉葬者,三十餘人,當死之日,皆餉之於庭。 餉輟,俱引升堂,哭聲震殿閣。 堂上置木小床,使立其上,掛繩圍於其上,以頭納其中,遂去其床,皆雉經而死。 韓氏臨死,顧謂金黑曰:"娘吾去! 娘吾去!" 語未竟,旁有宦者去床,乃與崔氏俱死。 諸死者之初升堂也,仁宗親入辭訣,韓氏泣謂仁宗曰:"吾母年老,願歸本國。" 仁宗許之丁寧,及韓氏旣死,仁宗欲送還金黑,宮中諸女秀才曰:"近日魚、呂之亂,曠古所無。 朝鮮國大君賢,中國亞匹也。 且古書有之,初佛之排布諸國也,朝鮮幾爲中華,以一小故,不得爲中華。 又遼東以東,前世屬朝鮮,今若得之,中國不得抗衡必矣。 如此之亂,不可使知之。" 仁宗召尹鳳問曰:"欲還金黑,恐洩近日事也,如何?" 鳳曰:"人各有心,奴何敢知之?" 遂不送金黑,特封爲恭人。 初,帝寵王氏,欲立以爲后,及王氏薨,帝甚痛悼,遂病風喪心,自後處事錯謬,用刑慘酷。 魚、呂之亂方殷,雷震奉天、華蓋、謹身三殿俱燼。 宮中皆喜以爲:"帝必懼天變,止誅戮。" 帝不以爲戒,恣行誅戮,無異平日。 後尹鳳奉使而來,粗傳梗槪,金黑之還,乃得其詳。」

\subsection{洪武}

\begin{longtable}{|>{\centering\scriptsize}m{2em}|>{\centering\scriptsize}m{1.3em}|>{\centering}m{8.8em}|}
  % \caption{秦王政}\
  \toprule
  \SimHei \normalsize 年数 & \SimHei \scriptsize 公元 & \SimHei 大事件 \tabularnewline
  % \midrule
  \endfirsthead
  \toprule
  \SimHei \normalsize 年数 & \SimHei \scriptsize 公元 & \SimHei 大事件 \tabularnewline
  \midrule
  \endhead
  \midrule
  三五年 & 1402 & \tabularnewline
  \bottomrule
\end{longtable}

\subsection{永乐}

\begin{longtable}{|>{\centering\scriptsize}m{2em}|>{\centering\scriptsize}m{1.3em}|>{\centering}m{8.8em}|}
  % \caption{秦王政}\
  \toprule
  \SimHei \normalsize 年数 & \SimHei \scriptsize 公元 & \SimHei 大事件 \tabularnewline
  % \midrule
  \endfirsthead
  \toprule
  \SimHei \normalsize 年数 & \SimHei \scriptsize 公元 & \SimHei 大事件 \tabularnewline
  \midrule
  \endhead
  \midrule
  元年 & 1403 & \tabularnewline\hline
  二年 & 1404 & \tabularnewline\hline
  三年 & 1405 & \tabularnewline\hline
  四年 & 1406 & \tabularnewline\hline
  五年 & 1407 & \tabularnewline\hline
  六年 & 1408 & \tabularnewline\hline
  七年 & 1409 & \tabularnewline\hline
  八年 & 1410 & \tabularnewline\hline
  九年 & 1411 & \tabularnewline\hline
  十年 & 1412 & \tabularnewline\hline
  十一年 & 1413 & \tabularnewline\hline
  十二年 & 1414 & \tabularnewline\hline
  十三年 & 1415 & \tabularnewline\hline
  十四年 & 1416 & \tabularnewline\hline
  十五年 & 1417 & \tabularnewline\hline
  十六年 & 1418 & \tabularnewline\hline
  十七年 & 1419 & \tabularnewline\hline
  十八年 & 1420 & \tabularnewline\hline
  十九年 & 1421 & \tabularnewline\hline
  二十年 & 1422 & \tabularnewline\hline
  二一年 & 1423 & \tabularnewline\hline
  二二年 & 1424 & \tabularnewline
  \bottomrule
\end{longtable}


%%% Local Variables:
%%% mode: latex
%%% TeX-engine: xetex
%%% TeX-master: "../Main"
%%% End:

%% -*- coding: utf-8 -*-
%% Time-stamp: <Chen Wang: 2019-12-26 15:06:42>

\section{仁宗\tiny(1424-1425)}

\subsection{生平}

明仁宗朱高熾(1378年8月16日-1425年5月29日),俗稱洪熙帝,明成祖長子,其母为仁孝文皇后,中山王徐達外孫,明朝第四代皇帝。

洪武年間,被封為燕世子。靖難之役中,仁宗負責鎮守北平,并成功抵禦李景隆率領的中央軍圍攻。永樂二年(1404年),立為皇太子,并在明成祖屢次北伐中,擔任監國職位,實際負責國家政事。永樂二十二年(1424年),繼承皇位,年號“洪熙”,在位期間,採取一系列政治、經濟、軍事改革與調整,國家富足。仁宗與子明宣宗在政治用人、行政處理上,均為後世所称善,史稱“仁宣之治”。

朱高熾年幼端重沉靜,善於言辭,且擅长射箭,喜愛與儒臣講論。洪武二十八年闰九月壬午(1395年11月4日),他被冊封為燕世子,後守衛北平,由於心性較爲溫良,體諒官員、士卒,深受祖父明太祖朱元璋喜愛。

靖難之役中,燕王朱棣起兵,朱高熾則鎮守北平,期間以一萬兵力,阻擋李景隆率領的五十萬中央軍圍攻。由於朱高熾身型肥胖而且有腳病,不良於行,不曾隨父親朱棣征戰,且性格相對較爲溫和,向來不獲父親寵愛。反而常隨朱棣征戰的次子朱高煦、三子朱高燧均受朱棣喜愛,而朱高煦則更因屢有戰功,於是出言詆毀朱高炽以奪嫡。當時,建文帝施離間計,下「賜世子書」;朱高燧的人馬得知此事,向朱棣建言「世子勾結朝廷」,沒想到朱高熾不予啟封,直接呈上朱棣,方破此計。朱棣即位後,改北平為北京,仍命朱高熾居守。

朱棣成功奪位為帝後,是為明成祖。永乐元年春正月丙戌,群臣上表请立皇太子,不允;三月戊寅朔,文武百官复上表,请立皇太子,敕“姑缓之”。成祖本想立自己喜愛的次子朱高煦為太子,但礙於長子朱高熾的世子地位是明太祖確立,而且朱高熾並無過失,又得一眾文官支持,最後於永樂二年四月甲戌(1404年5月12日),朱高熾被召入南京應天府,被立為皇太子。明成祖屢次北伐,均命其擔任監國,負責國事。當時全國經战争影響,水旱饑荒嚴重,他派遣官員賑災撫恤,仁政受到贊許。然而,失落太子地位的朱高煦心有不甘,聯同弟朱高燧及其他黨羽加緊離間明成祖與朱高熾的關係。明成祖問太子是否知悉有人離間,朱高熾則答稱不知情,“知盡子職而已”。

永樂十年,朱棣北伐歸還,朱高熾遣使誤期,加上書奏失辭,太子一系官員,如黃淮等人均下詔獄。次年,朱高燧黨羽黃儼等誣陷朱高熾擅自釋放罪人,其官僚多因連坐而亡。禮部侍郎胡濙奉命調查后,密奏朱棣稱太子誠敬孝謹等七事,明成祖才釋除疑慮。之後,朱高燧黨羽黃儼策劃謀立,后被發覺,伏法。太子朱高熾則力請免朱高燧罪,至此朱高熾地位方穩。

永樂二十二年(1424年)七月,明成祖在北征班師途中崩於榆木川。当时京师诸卫军皆随行,只有赵府三护卫留京师,随驾北征诸臣浮议籍籍,大学士杨荣、金幼孜等人顾虑赵府护卫闻讯发动政变,遂秘不发丧。杨荣与少监海寿持遗诏驰奔京师。朱高熾遣皇太孙朱瞻基出居庸关迎驾。同年八月己酉,皇太孙至雕鹗堡,入于军中,遂发丧。八月丁巳(1424年9月7日),朱高熾繼帝位,大赦天下,并取次年年號為洪熙。明仁宗登基後,褒奖直言,虚怀纳谏,減轻刑法。朱高熾與子朱瞻基在政治用人、行政處理上,均為後世所称善,史稱“仁宣之治”。

經濟方面,他下令中止鄭和下西洋,并取消官方在雲南、交阯的採辦活動、将首都迁回南京,以節省國家財政支出。政治方面,他恢復夏原吉、吳中官職,恢復三公、三孤等官職,命楊榮為太常寺卿,金幼孜為戶部侍郎,兼大學士,楊士奇為禮部左侍郎兼華蓋殿大學士,黃淮為通政使兼武英殿大學士,楊溥為翰林學士,進一步提升明朝內閣地位。軍事方面,他重新調整大同、交阯、山海關、遼東的邊疆總兵大臣,并建立南京守備制度。

同年冬天,朱高熾進一步對政治進行調整,加強戶部管理、以及城池防禦的同時,冊封張氏為皇后,立長子朱瞻基為皇太子、其餘八子分別為王。隨後下詔,赦免了建文帝的舊臣和永樂朝時遭連坐流放邊境的官員家屬,并免除受災地的稅糧。

外交方面,于闐、琉球、占城、哈密、古麻剌朗、滿剌加、蘇祿、瓦剌等國稱臣入貢。

洪熙元年春,因顯日食,朱高熾罷免宴樂。他進一步對政治進行調整,包括建立弘文閣,命楊溥掌管內閣;屢次求官員直言并納言,并對太祖時期的法外用刑制度進行修正,減少刑罰,實行寬政。

仁宗体弱多病,登基后不到十个月,遭李时勉當廷勸諫,龍顏大怒,雖命武士以金瓜錘將李时勉打斷三根肋骨,並拘入詔獄,仁宗仍不解恨,數日後一病不起,于洪熙元年五月辛巳(1425年5月29日)崩于钦安殿,廟號仁宗,葬于明獻陵(今北京昌平)。朱高熾延續了太祖和成祖的殉葬制度,死時生殉五名妃嬪。

\subsection{洪熙}

\begin{longtable}{|>{\centering\scriptsize}m{2em}|>{\centering\scriptsize}m{1.3em}|>{\centering}m{8.8em}|}
  % \caption{秦王政}\
  \toprule
  \SimHei \normalsize 年数 & \SimHei \scriptsize 公元 & \SimHei 大事件 \tabularnewline
  % \midrule
  \endfirsthead
  \toprule
  \SimHei \normalsize 年数 & \SimHei \scriptsize 公元 & \SimHei 大事件 \tabularnewline
  \midrule
  \endhead
  \midrule
  元年 & 1425 & \tabularnewline
  \bottomrule
\end{longtable}


%%% Local Variables:
%%% mode: latex
%%% TeX-engine: xetex
%%% TeX-master: "../Main"
%%% End:

%% -*- coding: utf-8 -*-
%% Time-stamp: <Chen Wang: 2019-10-18 16:51:41>

\section{宣宗\tiny(1425-1435)}

明宣宗朱瞻基(1399年3月16日-1435年1月31日),或稱宣德帝,明仁宗皇长子,永樂九年(1411年)立為皇太孫;永乐二十二年(1424年)十月立為皇太子。洪熙元年(1425年)即位,年號宣德,明朝第5位皇帝,在位十年,享年37歲。宣德元年(1426年)平定高煦之亂,和其父仁宗一样,比较能倾听臣下的意见,聽從閣臣楊士奇、楊榮、楊溥等建議,停止對交阯用兵,与明仁宗并称「仁宣之治」,宣宗时君臣关系融洽,经济也稳步发展。不過,他也開啟此後宦官干政的局面。

明成祖時,朱瞻基父親朱高熾(仁宗)為太子,生性仁厚端重,但有時不免失之於懦怯。成祖最喜愛次子漢王朱高煦,覺得他最像自己,有心廢太子立漢王,但徐皇后和大臣們一直阻攔。而且朱瞻基自幼聰慧好學,與生母張氏皆深得成祖的喜愛,所以最終才沒有廢太子,並對朱瞻基悉心栽培。永樂九年(1411年)十一月立為皇太孫,數度隨成祖征討。永乐二十二年(1424年)仁宗即位,十月朱瞻基被立為皇太子。洪熙元年(1425年)四月,因南京地震多發,奉旨前往居守;同年六月仁宗駕崩,宣宗繼位。

明宣宗在位十年,重点在治理内政方面。宣德元年(1426年)平定汉王朱高煦的叛乱,宣宗原先只將他禁錮,仍前往探视,却被朱高煦使腿将其绊倒,宣宗一怒,将朱高煦用鼎扣住,烧烤至死,諸子全部處死。为了休兵养民,宣宗一改永乐时期的讨伐政策,主动从交阯撤兵。

宣宗整顿统治机构,罢免「贪津不律」、「不达政体」、「年老体疾」的官员,实行精简和裁冗措施,以振朝风。而在用人方面限制入仕人数,实行保举和欠任。宣宗实行一些减轻民困的措施,减免税粮、复业流民、赈灾救荒等。宣德三年出塞,并修建永寧、隆慶諸城。

在宦官问题上,因明代初期宦官多由藩屬國進貢或沒入各地罪犯家屬,在語言溝通上發生很大問題,言不同語只好以書同文來解決,宣德元年(1426年),明宣宗下令設置內書堂,教導宦官們讀書。不過,明太祖苦心謀劃的女官制度雖經成祖時期略加破壞,在此時仍發揮其防制閹黨之禍的功用,可是宣宗下令容許教導宦官讀書一舉,无意中卻開啟了明代宦官干政之先兆,尤其在明神宗後,因氣候變遷造成北方官話區大量貧困百姓自宮入朝廷謀職,萬曆至崇禎(1573-1644年)這71年間自宮入廷的閹宦總計高達三萬人,使得教導宦官成為明朝覆滅的其中原因,也是最受後世批評之處。不過與唐朝相比,明代皇帝極權之盛, 使終明一朝皇帝亦不至受宦官控制,一般而言亦只是通過宦官來處理政務及制約大臣的權力。

宣德五年(1431年1月),宣宗以外番多不來朝貢為由,命令鄭和再次出航。返航期間,鄭和因勞累過度於宣德八年(1433年)四月初在印度西海岸古里去世。船隊由太監王景弘率領返航,宣德八年七月初六(1433年7月22日)返回南京。第七次下西洋人數據載有27550人。這也是最後一次下西洋。

宣德十年(1435年)正月初三,皇帝崩于乾清宫,时年37岁,谥号宪天崇道英明神圣钦文昭武宽仁纯孝章皇帝。庙号宣宗。宣德十年六月廿一日,梓宫葬入景陵。

安南人黎利反叛,屡次打败官军。黎利请示朝廷,请求重新立陈氏之后为安南国王。朱瞻基认为国中疲惫,远征无益,于是答应了他,册封陈暠为安南国王,罢征南兵。后来黎利篡夺陈暠之位而自立为王。派人入朝纳贡谢罪,请求皇帝册封群臣。有人请求皇帝讨伐黎利,朱瞻基不许,册封黎利为安南国王。安南国也就是交趾国,自此以后朝贡不绝。

朱瞻基担心秋高马肥时蒙古人侵犯边疆,于是整顿兵马,驻扎喜峰口以待敌军。守将奏报兀良哈率领万名铁骑骚扰边疆,朱瞻基精选铁骑兵三千飞奔前往。敌军望见远处来军,以为是戍守边疆之兵,即以全军来迎战。朱瞻基命令将铁骑分为两路夹攻敌军,并且亲自射杀敌军先锋,杀死三人。两翼飞矢如云,敌人不敢前进。继而,朱瞻基又命连续发射神机铳,敌军人马死伤大半,剩下的全部溃逃。朱瞻基用数百铁骑直驱前行,敌人看到黄龙旗,才知道是皇帝亲征,于是全部下马拜倒在地请降,朱瞻基将这些人捆缚抓获,大胜而归。

《明史》赞誉宣宗:“仁宗为太子,失爱于成祖。其危而复安,太孙盖有力焉。即位以后,吏称其职,政得其平,纲纪修明,仓庾充羡,闾阎乐业。岁不能灾。盖明兴至是历年六十,民气渐舒,蒸然有治平之象矣。若乃强藩猝起,旋即削平,扫荡边尘,狡寇震慑,帝之英姿睿略,庶几克绳祖武者欤。”

《國榷》:“谈迁曰:国初严御,每重囚岁械入京辄千百,簿尉巡檄之任,辄烦圣虑,盖详极矣。宣宗幼侍文皇帝出入塞垣,深谙民事。及即位,遽有乐安之驾,非素才武,畴克灭此而朝食也者?然兵不轻试,惓惓以生灵为念。水旱朝奏,赈贷午曁。亲阅囚牍,多所释遣。好文学之士,一才一技,皆被甄录。盖睿质天纵,文翰并美,而不矜其能,尝有自下之色。国家之治,宽严有制,烦简有则,帝实始之。而於废胡后,弃南交,孰为帝谅者?呜呼!废后非盛德事也,其弃南交,比於汉之朱崖矣。”

《名山藏》:“高皇帝承胡元縱弛之弊,宏振威武以儆天下,成祖以英達之資纘緒大服,海內竦然,振厲者五十餘年。昭皇帝(明仁宗)至德深仁不久於位,章帝(明宣宗)繼之,乃涵濡以醇懿陶埴,以德義聞四方。”

《朝鮮文宗實錄》:“上(朝鮮文宗)謂代言等曰: "尹鳳率爾告予曰: 「洪熙皇帝及今(宣德)皇帝, 皆好戲事。 洪熙嘗聞安南叛, 終夜不寐, 甚無膽氣之主也。’」知申事鄭欽之對曰:“尹鳳謂予曰: 「洪熙沈于酒色,聽政無時,百官莫知早暮。 今皇帝燕于宮中,長作雜戲。 永樂皇帝, 雖有失節之事, 然勤於聽政, 有威可畏。」 鳳常慕太宗皇帝, 意以今皇帝爲不足矣。”上曰:「人主興居無節, 豈美事乎?」”

宣德皇帝既是一个有较高文化素養的皇帝,又是一个喜欢射猎、美食、鬥促织(蟋蟀)的皇帝。《聊齋誌異》裡的名篇《促織》裡的皇帝正是明宣宗,人稱“促织天子”,吳偉業有《明宣宗御用戧金蟋蟀盆歌》。

\subsection{宣德}

\begin{longtable}{|>{\centering\scriptsize}m{2em}|>{\centering\scriptsize}m{1.3em}|>{\centering}m{8.8em}|}
  % \caption{秦王政}\
  \toprule
  \SimHei \normalsize 年数 & \SimHei \scriptsize 公元 & \SimHei 大事件 \tabularnewline
  % \midrule
  \endfirsthead
  \toprule
  \SimHei \normalsize 年数 & \SimHei \scriptsize 公元 & \SimHei 大事件 \tabularnewline
  \midrule
  \endhead
  \midrule
  元年 & 1426 & \tabularnewline\hline
  二年 & 1427 & \tabularnewline\hline
  三年 & 1428 & \tabularnewline\hline
  四年 & 1429 & \tabularnewline\hline
  五年 & 1430 & \tabularnewline\hline
  六年 & 1431 & \tabularnewline\hline
  七年 & 1432 & \tabularnewline\hline
  八年 & 1433 & \tabularnewline\hline
  九年 & 1434 & \tabularnewline\hline
  十年 & 1435 & \tabularnewline
  \bottomrule
\end{longtable}


%%% Local Variables:
%%% mode: latex
%%% TeX-engine: xetex
%%% TeX-master: "../Main"
%%% End:

%% -*- coding: utf-8 -*-
%% Time-stamp: <Chen Wang: 2019-12-26 15:06:53>

\section{英宗\tiny(1435-1449)}

\subsection{生平}

明英宗朱祁鎮(1427年11月29日-1464年2月23日),明宣宗朱瞻基長子,生母孝恭章皇后,明代宗朱祁鈺異母兄,明憲宗朱見深之父,是明朝的第6位和第8位皇帝;最初使用正統(1436年-1449年)年號,復位後使用天順(1457年-1464年)年號,在位22年。謚號「法天立道仁明誠敬昭文憲武至德廣孝睿皇帝」。

宣德二年(1427年),貴妃孫氏為明宣宗朱瞻基產下長子朱祁鎮(但《明史》記孫氏生平則說她暗中取宮女之子為己子)。出生四個月的朱祁鎮隨即被立為皇太子,其母孫氏為皇后。

宣德十年(1435年)正月,宣宗崩,時年7歲的朱祁鎮即位,是為英宗,改次年為正統元年。英宗在位初期由太皇太后張氏輔政,內閣由三楊(楊士奇、楊榮和楊溥)主持,仁宣之治得以延續。

正統六年(1441年),正式親政,同年定首都為北京,結束南京名義上的首都地位。

正統七年(1442年),張太后卒,三楊以年老淡出政壇,宦官王振開始專權,其黨羽遍天下,百官為之側目,這是明朝第一次宦官專權。

正統十四年(1449年),瓦剌蒙古大舉南侵,英宗以五十萬大軍親征,沿途鋪張。返師途中,八月十五(1449年9月1日)行至土木堡被瓦剌太師也先所敗,明軍「死者數十萬」,英宗被俘虜,附和英宗的太監王振被明英宗之護衛將軍樊忠殺死,樊忠殺死王振前曰:「吾為天下誅此賊!」以所持棰擊殺王振,力圖突圍,殺數十人後戰死。史稱土木堡之變,簡稱土木之變。

隨後,也先挾持英宗南下進攻北京,皇太后孫氏命英宗之弟郕王朱祁鈺監國,不久郕王即帝位,是為明代宗,改次年為景泰元年,尊英宗為太上皇。

于謙領導的北京保衛戰勝利後,瓦剌倡議和談,欲送還英宗。景帝不欲英宗還鑾。景泰元年(1450年),鴻臚卿楊善變賣家產,孤身出使瓦剌,又在景帝不同意的情況下,說服瓦剌太師也先,將英宗迎回燕京。

英宗回國後,代宗怕失去即位不久的帝位,將其兄長英宗軟禁於南內崇質宮,令錦衣衛防守嚴密。景泰三年,又廢原立為太子的英宗長子朱見深為沂王,另立己子朱見濟為儲君。但朱見濟在次年去世。後太子朱見濟死,但代宗仍不同意復立朱見深為太子。

景泰八年(1457年)正月,代宗病重,不能臨朝,手握重兵的武清侯石亨、副都御史徐有貞等人聯合太監曹吉祥,率死士攻入南宮,擁英宗復辟。十六日晚上,英宗自東華門入宮,於奉天殿即位,黎明時開宮門,諭令百官,改元天順,史稱「奪門之變」。代宗被禁於西內。不久死亡,死因不明,有謂乃英宗使宦官蔣安以布帛縊死。死後追貶為郕王,謚戾,葬於西郊金山(玉泉山北)。

英宗奪門之變復辟後,即以謀逆罪將兵部尚書于謙及大學士王文等人下獄,初尚言「于謙實有功」,徐有貞言「不殺于謙,今日之事無名」,遂於五日後斬殺于謙和王文於西市。天下冤之。大學士李賢告知英宗背後秘密,「奪門之變」沒有用處。因為郕王無子,擁立朱祁鎮的孫太后仍在世上,所以帝位遲早是英宗的,不需要奪門。奪門只是小人们的一齣戲,目的是求自己的升官發財。英宗下令宮中不得再使用「奪門」一詞,並且罷除因奪門之變而晉升的一切官職(計四千餘人),疏遠了徐​​有貞等,後來曹吉祥與石亨等人勾結,先設法中傷徐​​有貞,讓徐被流放。而後石亨與曹吉祥因圖謀叛亂發動曹石之變,石亨被囚至死,曹吉祥則被凌遲處死。

天順一朝,英宗勤於理政,並任用李賢、彭時等賢臣,先後懲治石亨、徐有貞、曹吉祥等人,政治尚算清明。又不顧左右反對,釋放建庶人(明惠宗幼子朱文圭,明成祖發動靖難後被幽禁宮中逾五十年,已豬狗不識),並提供飲食住行;聽錢皇后之言恢復前朝胡廢后的位號;病危遺言,取消了自明太祖以來的宮妃殉葬制度。《明史》讚譽道:「罷宮妃殉葬,則盛德之事可法後世者矣。」王世貞在《弇州山人別集》中亦稱:「此誠千​​古帝王之盛節。」

天順八年(1464年)正月英宗駕崩,享年38歲。葬入明十三陵中的裕陵。英宗與錢皇后感情頗深,錢皇后無子;因周妃專橫,英宗擔心死後嗣子明憲宗(周氏所生)不尊崇她的地位,所以遺命「皇后他日壽終,宜合葬」後來錢皇后死時,周太后果然不欲其祔葬裕陵,由於有英宗的遺詔,經過大臣力爭方得與英宗合葬。此後,在周太后的壓力下,不得已改變英宗的陵寢設計,周太后也得以附葬裕陵,開始出現一帝兩后或多后的格局。

\subsection{正统}

\begin{longtable}{|>{\centering\scriptsize}m{2em}|>{\centering\scriptsize}m{1.3em}|>{\centering}m{8.8em}|}
  % \caption{秦王政}\
  \toprule
  \SimHei \normalsize 年数 & \SimHei \scriptsize 公元 & \SimHei 大事件 \tabularnewline
  % \midrule
  \endfirsthead
  \toprule
  \SimHei \normalsize 年数 & \SimHei \scriptsize 公元 & \SimHei 大事件 \tabularnewline
  \midrule
  \endhead
  \midrule
  元年 & 1436 & \tabularnewline\hline
  二年 & 1437 & \tabularnewline\hline
  三年 & 1438 & \tabularnewline\hline
  四年 & 1439 & \tabularnewline\hline
  五年 & 1440 & \tabularnewline\hline
  六年 & 1441 & \tabularnewline\hline
  七年 & 1442 & \tabularnewline\hline
  八年 & 1443 & \tabularnewline\hline
  九年 & 1444 & \tabularnewline\hline
  十年 & 1445 & \tabularnewline\hline
  十一年 & 1446 & \tabularnewline\hline
  十二年 & 1447 & \tabularnewline\hline
  十三年 & 1448 & \tabularnewline\hline
  十四年 & 1449 & \tabularnewline
  \bottomrule
\end{longtable}


%%% Local Variables:
%%% mode: latex
%%% TeX-engine: xetex
%%% TeX-master: "../Main"
%%% End:

%% -*- coding: utf-8 -*-
%% Time-stamp: <Chen Wang: 2019-12-26 15:06:58>

\section{代宗\tiny(1449-1457)}

\subsection{生平}

明代宗朱祁鈺(1428年9月21日-1457年3月14日),或稱景泰帝,年號景泰,明憲宗追諡其為「恭仁康定景皇帝」,弘光帝上庙号「代宗」,谥号「符天建道恭仁康定隆文布武显德崇孝景皇帝」,明朝第7位皇帝(1449年9月22日—1457年2月24日在位)。明宣宗皇次子,母親是賢妃吳氏。

生于宣德三年(1428年),他是明宣宗次子,母吴贤妃。据《明史》称吴贤妃为明宣宗为皇太孙时的侍女。

兄长明英宗即位後封他為郕王。1449年,明英宗在“土木堡之变”被瓦剌太師也先所俘后,郕王被于謙等大臣拥立,是为代宗,年号景泰,尊英宗為太上皇。

代宗即位后,用于謙为兵部尚书,北京保衛戰粉碎了瓦剌的进攻。景泰元年(1450年)八月,鴻臚寺卿楊善出使瓦剌,靠著三寸巧舌說服了也先,英宗返回北京,代宗並沒有迎回兄長的意思,又害怕他复辟,故将其软禁於宮中,以錦衣衛嚴密控管,宮門上鎖並且灌鉛,食物僅能由小洞遞入。

景泰三年,代宗廢去英宗長子朱見深的太子之位,改立自己兒子朱见济為太子,但朱見濟在次年去世。

景泰八年(1457年)正月,代宗病危,十六日曹吉祥、石亨、徐有貞等人謀復立英宗,十七日清晨,發動奪門之變,率領武士攻入紫禁城奉天殿,英宗復辟。代宗被软禁在西苑,一个多月後去世,得年三十岁。代宗死因不明,陸釴的《病逸漫記》說代宗是被英宗謀殺的,查繼佐的《罪惟錄》則表示代宗病愈,英宗為怕代宗復起,令太監蔣安用帛扼死景泰帝。代宗死后,葬于西郊金山(玉泉山北)的景泰陵。英宗令廷臣议王妃之殉葬。议及汪皇后,被李賢及太子谏止。后以皇贵妃唐氏殉葬。

英宗恨代宗薄待,谥为戾王,称郕戾王。明宪宗成化时期上谥号「恭仁康定景皇帝」。明崇禎十七年(1644年)七月乙丑,弘光帝上庙号代宗,谥号「符天建道恭仁康定隆文布武显德崇孝景皇帝」。清朝复称其谥号为「恭仁康定景皇帝」。明清史书多称明代宗为景帝。

明代宗是未安葬在明十三陵的皇帝(另外明太祖朱元璋葬于南京明孝陵,明惠帝因最後失踪故無陵墓)。

\subsection{景泰}

\begin{longtable}{|>{\centering\scriptsize}m{2em}|>{\centering\scriptsize}m{1.3em}|>{\centering}m{8.8em}|}
  % \caption{秦王政}\
  \toprule
  \SimHei \normalsize 年数 & \SimHei \scriptsize 公元 & \SimHei 大事件 \tabularnewline
  % \midrule
  \endfirsthead
  \toprule
  \SimHei \normalsize 年数 & \SimHei \scriptsize 公元 & \SimHei 大事件 \tabularnewline
  \midrule
  \endhead
  \midrule
  元年 & 1450 & \tabularnewline\hline
  二年 & 1451 & \tabularnewline\hline
  三年 & 1452 & \tabularnewline\hline
  四年 & 1453 & \tabularnewline\hline
  五年 & 1454 & \tabularnewline\hline
  六年 & 1455 & \tabularnewline\hline
  七年 & 1456 & \tabularnewline\hline
  八年 & 1457 & \tabularnewline
  \bottomrule
\end{longtable}


%%% Local Variables:
%%% mode: latex
%%% TeX-engine: xetex
%%% TeX-master: "../Main"
%%% End:

%% -*- coding: utf-8 -*-
%% Time-stamp: <Chen Wang: 2019-10-21 17:06:32>

\section{英宗复辟\tiny(1457-1464)}

奪門之變,又稱南宮復辟,是明代宗朱祁鈺景泰八年(1457年)正月,发生的一場政變,太上皇朱祁鎮成功復辟,奪回皇位。

正統十四年 (1449年) 發生土木堡之變,明英宗被瓦剌俘虜,其弟郕王朱祁鈺被眾大臣推舉為皇帝,是為明景帝(南明尊稱為代宗),改元景泰。孫太后亦要求景帝即位後立英宗兩歲兒子朱見深為太子,表示大明帝位仍由英宗一脈繼承。

景泰元年(1450年),兵部侍郎于谦成功抗敵,並與瓦剌議和,經過使臣楊善個人的斡旋,瓦剌首領也先見新君已立,英宗已經無利用價值,反而不想因英宗為虜之事成為與大明修好的障礙,於是同意放回英宗。但朱祁鈺對大臣說:「我並不是貪戀帝位,當初擁立我的是你們啊。」不願英宗返國,經大臣陳述其利弊後,朱祁鈺将英宗迎接回京,置於南宮,尊為太上皇。並以錦衣衛對英宗加以軟禁,嚴密控管,宮門不但上鎖,並且灌鉛,食物僅能由小洞遞入。其後景帝在景泰三年 (1452年)廢原太子朱見深,並立自己的獨子朱見濟為新太子。景泰五年 (1454年),朱见济夭折后,朱祁钰已无亲子,却也没有复立朱见深,储位空悬。

景泰七年(1456年),朱祁鈺病重,在對抗瓦剌時立下大功的將領石亨為了自身利益,有意協助英宗奪回帝位。在拉攏身邊人商討後,與宦官曹吉祥、都督張軏、都察院左都御史楊善、太常卿許彬以及左副都御史徐有貞等人行事。

景泰八年(1457年)正月,朱祁鈺病重。十六日夜,石亨、徐有贞等大臣带一千餘士兵偷襲紫禁城,撞开南宮宫门,接出英宗直奔东华门。守门的武士不开门,英宗上前说道:“朕乃太上皇帝也。”武士只好打开城门。

黎明时分,众大臣到了「奉天殿」,只见英宗坐于龙椅之上,徐有贞高喊:“太上皇帝復位。”史称「奪門之變」或「南宮復辟」。

英宗復辟後,朱祁鈺被遷至西宮,不久去世。

談遷評論:“于少保最留心兵事,爪牙四布,若奪門之謀,懵然不少聞,何貴本兵哉!或聞之倉卒,不及發耳!”

明英宗復辟後,于謙以謀逆罪名被處死,而曾助英宗回復帝位的功臣,如石亨、徐元玉、許彬、楊善、張軏與曹吉祥等人都被封為大官。其中,曹吉祥等在朝中橫行霸道,後期更發生了曹吉祥企圖弒位的曹石之變。

值得一提的是,景泰八年春正月,明英宗重登大寶後,废景泰年号,改景泰八年为天顺元年,但倉促之中忘記罷黜朱祁鈺,直到同年二月乙未才將朱祁鈺廢為郕王。因此,在這幾天之內,名義上英宗和景帝兩位合法的皇帝同時並存,成為中國帝制史上絕無僅有的奇觀。

曹石之变前,英宗在李贤提醒下,意识到朱祁钰时日无多,没有在世的儿子,也没有立储,一旦朱祁钰去世,自己复位顺理成章,夺门功臣其实是投机以求自己获益,一旦事败,英宗自己反而要受到牵连;于是开始罢黜夺门功臣的爵位。楊善、張軏已去世,爵位已分别由儿子杨宗、张瑾继承。明宪宗初年,罢黜杨宗、张瑾,因夺门之功所授爵位至此全部收回。

\subsection{天顺}

\begin{longtable}{|>{\centering\scriptsize}m{2em}|>{\centering\scriptsize}m{1.3em}|>{\centering}m{8.8em}|}
  % \caption{秦王政}\
  \toprule
  \SimHei \normalsize 年数 & \SimHei \scriptsize 公元 & \SimHei 大事件 \tabularnewline
  % \midrule
  \endfirsthead
  \toprule
  \SimHei \normalsize 年数 & \SimHei \scriptsize 公元 & \SimHei 大事件 \tabularnewline
  \midrule
  \endhead
  \midrule
  元年 & 1457 & \tabularnewline\hline
  二年 & 1458 & \tabularnewline\hline
  三年 & 1459 & \tabularnewline\hline
  四年 & 1460 & \tabularnewline\hline
  五年 & 1461 & \tabularnewline\hline
  六年 & 1462 & \tabularnewline\hline
  七年 & 1463 & \tabularnewline\hline
  八年 & 1464 & \tabularnewline
  \bottomrule
\end{longtable}


%%% Local Variables:
%%% mode: latex
%%% TeX-engine: xetex
%%% TeX-master: "../Main"
%%% End:

%% -*- coding: utf-8 -*-
%% Time-stamp: <Chen Wang: 2021-11-01 17:12:40>

\section{宪宗朱見深\tiny(1464-1487)}

\subsection{生平}

明憲宗,或稱成化帝,原名朱見深,後改名朱見濡(1447年12月9日-1487年9月9日),為明英宗皇長子,明朝第9代皇帝。明憲宗在位二十三年,期間恢復其叔朱祁鈺的帝號,又為于謙等忠臣平反,初年勵精圖治,體恤民情,任用李賢、商輅、彭時等賢臣,頗為時人所傳誦;在軍事方面,整飭戎政,對內平定荊襄群盜和西南傜蠻,對外抵禦抵禦韃靼女真、經略哈密,擁有不少功績。但憲宗寵嬖萬氏、中晚年信用汪直、梁芳、萬安等宦官奸臣,又以“皇莊”大肆侵占土地,使明朝政治日壞;而頻繁的內外用兵亦使明朝國力大損。成化朝是明朝自仁宣以來文治武功較卓越的時期,但是與此並存的弊政不得不說有所缺憾。谥号「繼天凝道誠明仁敬崇文肅武宏德聖孝纯皇帝。」

正統十四年(1449年)土木堡之變,英宗被瓦剌擄去,兵部侍郎于謙等立皇弟朱祁钰即位,是為景帝,改元景泰,同時立見深為太子。到景泰三年(1452年),朱祁鈺將見深廢為沂王,改立自己的儿子朱见济为太子。

五年后(1457年),英宗因奪門之變而復辟,見深重被立為太子。萬曆野獲編中記載憲宗皇帝玉音微吃,而臨朝宣旨,則瑯瑯如貫珠,其本人可能或多或少有口吃的情況。

原名朱見濬(《明史》誤載憲宗即位前名為朱見浚,即位後為見深),因英宗復辟後重立太子,將憲宗之名誤寫為見濡,憲宗於天顺八年(1464年)登基後遂改稱見濡。憲宗宽仁英明,即位之初就為于謙平冤昭雪,當時曾有大臣追論景泰廢立事往,憲宗切責說:「景泰事已往,朕不介意,且非臣下所當言。」另䆁放了浣衣局婦女和願歸宮人,又恢復明景帝帝號。文治上憲宗體諒民情,蠲賦省刑,任用賢臣,考察官吏,勵精圖治,善政史不絕書,儼然為一代明君,當其時朝廷多名贤俊彦,百姓得以休养生息,史稱成化新風,堪稱與仁宣之治媲美,朝鲜、琉球、哈密、烏斯藏、暹羅、吐魯番、撒馬兒罕、日本、蘇門答剌等國紛紛入貢。人口方面在成化十五年(1479年)中成為終明一代的人口峰值,達9,496,265戶,71,850,132人,反映當時明朝仍然處於盛世。

武功上憲宗恢復十二團營制度,幾次親閱騎射於西苑,巡查禁軍,整飭軍備,考試士兵訓練,還任用王越、余子俊、秦紘、朱永、朱英等能臣處理軍務,修建邊牆,并從不斷南下入侵盤踞河套的韃靼部手裡,一舉收復河套地區,使得套寇問題基本解決。在紅鹽池大捷中,明軍大破韃靼大營,擒斬三百五十人,獲駝馬器械不可勝計,史书記載「虏自是不敢复居套内者二十年,则此捷为所震慑故也。」「自是不复居河套,边患少弭;间盗边,弗敢大入,亦数遣使朝贡。」甚至在後來威宁海大捷中夜行晝伏直捣蒙古可汗王庭,生擒幼男婦女一百七十,斩首四百三十七级,獲旗纛十二面,馬駝牛羊六千餘,盔甲弓箭皮襖之類又萬餘,达延汗巴图蒙克仅以身逃。另外自從明英宗以來,盤踞在建州的李满住、董山屢寇掠辽东,逐漸成為邊患,明憲宗在多次招撫不果後決定用兵撻伐,先後於成化三年與成化十五年,明軍與朝鮮聯手進攻屢次犯邊的建州女真,生擒數百人,斩首千餘級,破四百五十餘寨,夺回被掳人口數千人,擒斬罪魁禍首的董山,史稱成化犁庭或丁亥之役。

明朝皇帝多擅畫像,作字運筆,憲宗亦擅畫神像,曾為張三豐畫像,神采生動,超然塵表,又曾親筆御製一團和氣和歲朝佳兆等畫流世,畫法老練嫻熟,頓挫自如。成化十八年,憲宗又親自編寫了《文華大訓》一書,以教導太子人倫治國之道,垂訓子孫。而《貞觀政要》自唐流傳至明,版本注釋繁亂,明憲宗即位後,立即組織儒臣對其進行校定,把宋元史纂輯的綱目皆寫入書中,頒示天下,即流傳至今的成化本,又為重修的孔子廟碑和《貞觀政要》作親自序。憲宗在《貞觀政要序》中寫道「朕萬幾之暇,悅情經史,偶及是編...太宗在唐為一代英明之君,其濟世康民,偉有成烈,卓乎不可及己,所可惜者,正心修身二帝三王之道,而治未純也。朕將遠師往聖,允迪大酋,以宏其治。」足見他的治國抱負和文化素質。

憲宗在位中后期,好方術,沉溺後宮,极度宠信大他19歲的万贵妃,又生活奢靡,取國庫填內帑并擴置皇莊,同时又任用太监汪直、梁芳等奸佞當權,以致西廠橫恣,朝紳諂附,且明憲宗直接頒詔封官,是為傳奉官,這使得傳奉官氾濫,舞弊成風,朝政荒芜。但整體而言,成化晚年,朝廷依然能有條不紊地對天災人禍有迅速的應對,因此仍幸稱歌舞升平,太平無事。

成化初年,土地兼併嚴重,造成大量流民依山據險,光是荊州、襄州、安州、沔州之間,“流民不下百萬”。湖廣荊襄地區成為流民的聚居區,賊盜嘯聚。成化元年(1465)三月劉通、石龍、馮子龍等於房縣大石廠立黃旗起義,擁眾數十萬。成化六年十一月,又有劉通舊部李原、小王洪起義,流民附和者達百萬人。史稱鄖陽民變。

成化二十三年(1487年)春,萬貴妃去世,憲宗過於悲痛而患病,長歎說:「萬氏長去了,我亦將去矣。」日漸消瘦,最終於同年八月廿二日駕崩,享年39歲(虚龄四十一)。葬於北京昌平茂陵。臨終前誨示太子要敬天法祖,勤政愛民,太子頓首受命,他的三子朱祐樘繼位,即后来的明孝宗。

明宪宗即位後任用李贤、彭时、商辂等人,下诏為于谦平反,派人去為于谦扫墓,并让其子于冕袭为千户,于谦的女婿朱翼等人,也被归还家产。

荆襄刘通造反,命抚宁伯朱永讨伐,将之平定。又有陕西周原土官满四占据石城,荆襄復反,憲宗力排众议,命项忠平定,荆襄贼平,明军击斩万人,首领刘通、苗龙等四十人被生擒献俘京师。宪宗又专门派出了杨璇抚治荆、襄、南阳流民,史載「大会湖广、河南、陕西抚、按、藩、臬之臣,籍流民得十一万三千余户,遣归故土者一万六千余户,其愿留者九万六千余户,许各自占旷土,官为计丁力限给之,令开垦为永业,以供赋役,置郡县统之。 」。此後流人得所,四境乂安,直至明未,荆襄再也沒有出現大亂了。

蠲賦省刑是成化一朝最為後人津津樂道的善政之一,史記憲宗「一聞四方水旱,蹙然不樂,亟下所司賑濟,或輦內帑以給之;重惜人命,斷死刑必累日乃下,稍有矜疑,輒從寬宥。」「憲宗好生,每奏讞大辟(死刑奏章),多所寬宥,或不得已而行刑。其日必卻八珍之奉,默坐焚香。哀矜之意,惻然見於玉色。」自他即位自駕崩唯止,僅在官田減免稅糧一項則已達一千九百多萬石,在民田稅額的蠲免和下內帑賑濟更是不計其數,僅以成化二十一年為例,實錄記載當年減免天下官田等項稅糧一百零八萬五千九百石,然而憲宗除此之外在該年正月從內庫中撥帑二十萬五兩賑濟災民,四月又撥漕糧四十萬賑災,同月與十月又免山東濟南、山西平陽、四川成都、河南開封、南直隸鳳陽等州府稅糧,總計連同官田稅賦該年蠲免三百萬石,相當全國稅額十之二一,可見憲宗不吝恤民。因此儘管成化一朝水旱災變不斷,在荆襄流民問題處理完後,再也沒有出現較大的社會波場動。

橫觀成化年間的最值得稱道的善政,除了處理荊襄流民與蠲賦省刑外,其次莫過於改革漕運,自明成祖永樂遷都以來,北京便依賴南糧北運,其中需要每年徵集大量民伕運糧,路途波折,時常耽誤農時,自成化七年後,朝廷減省少了民伕的運輸路程,改由官兵漕軍長運,雖然朝廷的加耗增加了,但節約了百姓的農時,有利農業生產,同時又制定了各類考課規條,自此以後明代的漕運才有了完備的制度,此制一直沿用至明末。

手工業者在成化年間身份有了進一步的自由,明太祖建國時,分天下百姓為軍民匠灶四類,手工業者便被歸類在匠户中,他們各分「住坐」和「輪班」,他們必須義務定期(通常五年一班,每班服役三個月)為朝廷工作,有時還要無償服役,於是逃役者越來越多。成化二十一年起,朝廷允許輪班匠不願服役者可以每月出錢免役,改由朝廷直接雇工造作,這不但令朝廷毋須再終年追捕工匠,勞官擾民,手工業者只要付出二三月的銀子,便可以免除三月的工役之苦和回來花費的時間,也換來四年的人身自由。

在位初期,天下称颂其统治;但宠信万贵妃后,朝政转向晦暗,万安开始得势。又设置西厂,命太监汪直提督外事,于是汪直便随意罗织罪名生事。汪直仗势将陈钺,威宁伯王越变为自己的羽翼,依附自己之人便任用,不听自己话的人就排挤打击,权势极为显赫,天下都惧之三分。汪直又想在外立功,胡乱进行边界挑衅。宪宗命汪直掌管十二团营。当时有个名叫阿丑的中官,善演诙谐幽默戏,经常在宪宗面前表演,颇有汉朝东方朔用滑稽方法进谏之风。一天阿丑假装喝醉酒,旁边一个人在佯装说:“某官到!”阿丑任装醉意大骂,人又说:“皇驾到!”阿丑还是醉骂如故,那人又说:“汪太监来了。”阿丑所装的醉人赶紧起来惊恐的站在一边。旁边的人问到:“天子驾到都不害怕,为什么害怕汪太监?”阿丑说:“我只知有汪太监,不知有天子。”自此以后汪直逐步失宠。此时王越和陈钺讨好汪直,三人结为死党。阿丑一日有在做戏,自己扮演汪直手持双斧向前前行,有人问其缘故,答说:“这双斧是王越和陈钺。”宪宗听后微笑了一下。御史徐鏞等人弹劾汪直欺君枉法,擅开边衅,宪宗后渐疏远汪直。

被宪宗先后任用的宰輔有:李賢,陳文,彭時,呂原,商輅,劉定之,萬安,劉珝,劉吉,彭華,尹直。对成化一朝,世有“紙糊三閣老,泥塑六尚書”之謠,三閣老指萬安、劉吉和劉珝,六尚書指尹禕、殷謙、周洪謨、張鵬、張鎣和劉昭,意讽这些朝廷重臣不作为,私德不佳,但也有意見認為他們之所以被抨擊,并非庸懦無能,貪贓枉法,而是因為對明憲宗專寵萬貴妃,內批傳奉官的行為沒有進行有力勸諫,使明憲宗符合傳統儒家人君規範,其實從成化後期對災區和地方事務的應對裁決,可見他們還是各有所長、恪盡職守的,因而即使同萬安這世稱的奸倖之臣,卻也見容於當其時彭時商輅等名臣官員中。

明憲宗本人曾經向兒子朱祐樘概括自己的一生作为:「修文史而究武略,饬内治以攘外侮,戡靖僭窃,应宁邦家,犹宵旰靡遑,惧功业未茂,德惠未周,而治平之效未臻也。」

《明實錄》:「葢上以守成之君,值重熙之運,兵革不試,萬民樂業,垂拱而天下大治矣。」

《名山藏》何乔远:上聪明仁恕,渊默勤恭,孝事母后如古帝王。郊庙斋祭,必极诚敬。景皇帝尝有封沂之命,未尝一语及之。委任大臣,略无猜忌,或即干纪,屏斥无疑。一闻四方水旱,戚戚然下所司赈济,或辇内帑给之。重惜人命,断死刑累日乃下。夙兴视朝,但遇雨雪辄放常参官而不废奏引。隆寒盛暑,或减奏事,以恤卫士侍立之劳。间有游豫,不出大内,如南囿祖宗时不废游猎,上未尝一幸焉。时御翰墨,作为诗赋,以赐大臣。诸司章奏,手自披阅,字画差错,亦蒙清问。臣下益兢业职事,莫敢或欺。葢上以守成之君,值重熙之运,兵革不试,万民乐业,垂拱而天下大治矣。

《国榷》谈迁:恤饥察冤,求言课吏,先后史不绝书,而于胡僧幸阉斜封墨敕之滥,亦不能为帝掩也。当其时,朝多耆德,士敦践履,上恬下熙,风淳政简,称明治者,首推成弘焉。而或有遗议,则在汪直、李孜省、繼曉辈蚀其一二,于全照无大损也。尺璧之瑕,乌足玷帝德哉!末谕太子以敬天法祖、勤政爱民之道,俨然成周之遗训也。说者谓帝初欲易储,以泰山屡震而止。噫!帝能尊钱后,复景帝,俱事出常情之外,而乃轻视东宫?必不然也。

《国榷》郑晓:帝仁恕英明,少更多难,练达情理。临政莅人,不刚不柔,有张有弛。进贤不骤而任之必专,远邪不亟而御之有法。值虏寇数侵边,惟遣将薄伐,不勤兵以竭我财力,虏亦离散,内外宁辑。荆襄岭海,时有寇窃,推毂之际,戒勿妄杀,或不用命,赏罚兼行。崇上理学,褒封儒贤。江淮大祲,截漕赈饥。星文示变,侧身省过。臣僚进谏,即涉浮伪,时有干忤,薄示谴谪,旋蒙牵复。若乃尊礼孝庄,尊景帝,保护汪后,褒恤于谦,其于爱憎恩怨,绝无芥蒂,帝谆然于天理彝伦者也。以故虽屡有彗孛之灾,而国家康靖,有繇然矣。

《国榷》李维桢:詩有之,“靡不有初,鮮克有終”,人情哉!純帝初載,亦何其斤斤也。中官幸,禱祠繁,而治隳矣。錢後之祔廟食,景帝之復位號,此兩者,雖甚盛德蔑以加已。

《明史》贊曰:「憲宗早正儲位,中更多故,而踐阼之后,上景帝尊號,恤于謙之冤,抑黎淳而召商輅,恢恢有人君之度矣。時際休明,朝多耆彥,帝能篤于任人,謹于天戒,蠲賦省刑,閭里日益充足,仁、宣之治于斯复見。顧以任用汪直,西厂橫恣,盜竊威柄,稔惡弄兵。夫明斷如帝而為所蔽惑,久而后覺,婦寺之禍固可畏哉。 」

《朝鮮成宗實錄》:上(成宗)御宣政殿, 引見明澮等, 謂曰: 「中國有何事?」 明澮對曰: 「(憲宗)皇帝勤於聽政, 天下太平, 民物富庶。」(時成化十一年)

《剑桥中国明代史》中写道:「朱见深与他的有军事头脑的祖父和父亲相同,向往他们的生气勃勃的、甚至具有侵略性的军事姿态,并且厚赏有成就的军事将领。」

負面事蹟主要與其大19歲的妃子萬貞兒的感情和鬆散的管理有關。

《罪惟錄》論曰:災異之警,無有酷於此二十三年者也。宮中位一女戎,而群小相緣益進,惑匿導誘,顛例黜陟,以致傳升無己,監督四出,閣輔阿循,廠衛搜射。而帝又旋悟旋迷,嘉言罔入,邊釁苗殘,幾無寧歲。天乃至仁,歷以所警,貫耳而呼,而其如溺柔聽者,袖不聞也。祗幸蠲賑免租,無少稽吝,猶不致啟中原之怒。且內外寡大故,無所藉以起,幸稱小康。嗟乎!哲婦傾城,危矣哉!

《明史講義》:凡此皆成化時朝政之穢濁,而國無大亂,《史》稱其時為太平,惟其不擾民生之故。

《朝鮮成宗實錄》:(司憲府掌令李琚)更啓曰: 「臣於丙午年往中國, 中國人言, 成化皇帝非賢君也, 然一用《大明律》, 故朝廷寧謐, 四方無虞矣。 臣今所啓, 別無他意, 欲殿下遵守舊章而已。」(朝鮮成宗) 傳曰: 「爾陪臣也, 而褒貶天子, 則我諸侯也, 何不褒貶我乎? 爾非新進之儒, 曾經弘文館, 爾不知予心而如此言之耶?」

《明朝時代上卷第38章陳獻章和他的心學》:成化王朝是明王朝歷史上的一個轉折點,正是在這個時期基本結束了朱元璋一百年來禁錮帝國的政策,從此帝國又重新恢復到唐宋元的那種自由、奔放的年代,商業開始復甦、城市開始繁華、思想文化開始活躍、士紳的生活開始奢靡,在這個社會整體鬆動下,起到穩定、凝聚作用的理學思想也開始搖搖欲墜,它必將被更能適應社會發展的思想所代替。

《成化皇帝大傳》:成化朝君臣们是预测不到的,他们留给弘治朝君臣的,乃是一个外无强敌,内无大敌,百业兴旺,万民乐业的太平世道。

\subsection{成化}

\begin{longtable}{|>{\centering\scriptsize}m{2em}|>{\centering\scriptsize}m{1.3em}|>{\centering}m{8.8em}|}
  % \caption{秦王政}\
  \toprule
  \SimHei \normalsize 年数 & \SimHei \scriptsize 公元 & \SimHei 大事件 \tabularnewline
  % \midrule
  \endfirsthead
  \toprule
  \SimHei \normalsize 年数 & \SimHei \scriptsize 公元 & \SimHei 大事件 \tabularnewline
  \midrule
  \endhead
  \midrule
  元年 & 1465 & \tabularnewline\hline
  二年 & 1466 & \tabularnewline\hline
  三年 & 1467 & \tabularnewline\hline
  四年 & 1468 & \tabularnewline\hline
  五年 & 1469 & \tabularnewline\hline
  六年 & 1470 & \tabularnewline\hline
  七年 & 1471 & \tabularnewline\hline
  八年 & 1472 & \tabularnewline\hline
  九年 & 1473 & \tabularnewline\hline
  十年 & 1474 & \tabularnewline\hline
  十一年 & 1475 & \tabularnewline\hline
  十二年 & 1476 & \tabularnewline\hline
  十三年 & 1477 & \tabularnewline\hline
  十四年 & 1478 & \tabularnewline\hline
  十五年 & 1479 & \tabularnewline\hline
  十六年 & 1480 & \tabularnewline\hline
  十七年 & 1481 & \tabularnewline\hline
  十八年 & 1482 & \tabularnewline\hline
  十九年 & 1483 & \tabularnewline\hline
  二十年 & 1484 & \tabularnewline\hline
  二一年 & 1485 & \tabularnewline\hline
  二二年 & 1486 & \tabularnewline\hline
  二三年 & 1487 & \tabularnewline
  \bottomrule
\end{longtable}


%%% Local Variables:
%%% mode: latex
%%% TeX-engine: xetex
%%% TeX-master: "../Main"
%%% End:

%% -*- coding: utf-8 -*-
%% Time-stamp: <Chen Wang: 2019-10-21 17:15:57>

\section{孝宗\tiny(1487-1505)}

明孝宗朱祐樘(1470年7月30日-1505年6月9日),或稱弘治帝,是明宪宗皇三子。明朝第10代皇帝(1487年-1505年在位),他在位18年,年号弘治。孝宗“恭俭有制,勤政爱民”,又能信用贤臣、广开言路,在位期间“朝序清宁,民物康阜”,明朝出现中兴局面,史称“弘治中兴”。但在位后期對朝政有所懈怠,又縱容外戚,沉迷方術,使宦官李广、蒋琮等人乘机弄权,以致弘治晚年軍備弛廢,國用匱乏,弊政颇多,故不能谓之全美。明孝宗崩逝後谥号「建天明道诚纯中正圣文神武至仁大德敬皇帝」,庙号「孝宗」,葬于泰陵。

根据《明史》记载:“孝宗达(实为“建”,《明史》误)天明道纯诚中正圣文神武至仁大德敬皇帝,讳祐樘,宪宗第三子也。母淑妃纪氏,大明成化六年七月生帝于西宫。时万贵妃专宠,宫中莫敢言。悼恭太子薨后,宪宗始知之,育周太后宫中。十一年,敕礼部命名,大学士商辂等因以建储请。是年六月,淑妃暴薨,帝年六岁,哀慕如成人。十一月,立为皇太子。”民間則傳說:孝宗出生时,为免被当时的寵妃萬貴妃害死而藏在民間,在憲宗死前才由宮內太監於民間迎回即位。

孝宗出生後,廢后吳氏貶居西內,與紀氏謫居的安樂堂相近,頗知消息,往來就哺,才得保全孝宗生命,由吳氏用心撫養過一段日子。

弘治帝在位初期,励精图治、整肃朝纲、改革弊政,罢逐了朝中奸佞之臣、重用贤士,为于谦建祠平冤,减轻赋税、停征徭役、兴修水利、发展农业、繁荣经济,史稱“弘治中兴”。

弘治帝在位期间“更新庶政,言路大开”,启用了刘健、丘濬、李东阳、谢迁、王恕、马文升、刘大夏等能臣,使明憲宗成化朝晚年以来,奸佞当道的局面,得以大为改观。

此外,弘治帝重視司法,他令天下諸司審錄重囚,慎重處理刑事案件。弘治十三年(1500年),制定《問刑條例》。又於弘治十五年(1502年),編成《大明會典》。

弘治帝在治理水患方面亦頗有效果,曾委任白昂、劉大夏修治黃河,以改善河道流向、築堤等方法抑制黃河水患,此後二十餘年間,再無大患發生;另外,蘇松於弘治年間,曾因河道淤塞而泛濫成災,孝宗即命徐貫主持治理,歷時三年,消除了蘇松水患。

弘治帝在位初期的經濟成就也比較突出,賦稅收入比成化年間增加了一百多萬石,達二千七百萬石,成為明中葉的賦入高峰;而且,人口方面也有穩定的增長。從弘治元年(1488年)到弘治十七年(1504年)間,人口增加了一千多萬,達到六千萬口。

惟自弘治十五年起(1502年),「一歲所入,不足以供一歲支用」,國家財政邁進了入不敷出的狀況,戶部呂鈡指出:『常入之賦,以蠲色漸減,常出之費,以請乞漸增,入不足當出。正純以前軍國費省,小民輸正賦而已。自景泰至今,用度雜辦,皆昔所無。民已重困,無可復增。往時四方豐登,邊境無調發,州縣無流移。今太倉無儲,內府殫絀,而冗食冗費日加於前。』對此下廷臣議,廷臣作出多項建議,但僅觸及成效不大的修補政策。

此外,孝宗也常以京營禁軍投入繁重的工作,監察御史劉芳曾上奏說,“京師根本之地而軍士逃亡者過半”,“其錦衣騰驤等衛軍士不下十餘萬人,又不繫操練之數,近年雖立營營,而役佔賣放者多。”,另外又常縱容邊臣,邊臣冒報功次皆得升賞,而敗軍失律者往往令之戴罪殺賊,使邊備日弛,對於北虜入侵能有效抵禦的戰役寥寥無幾,如弘治十四年秋七月,孝宗令保國公掛征虜大將軍總兵官領十萬大軍夜襲韃靼於河套,韃靼早察覺徙家北遁,朝廷用銀八十餘萬,只斬首三級以還,而將士奏報功次竟一萬有餘,“不能禦”,“坐虜入境”,“議者恥之”之類的描述比比皆是。

再者,弘治中期,皇帝自己漸漸迷上了齋醮,從此內庫開銷劇增,孝宗開始不斷地命戶部將太倉庫的銀子納入內庫,至將河西務鈔關關船料改擬折銀進納。如弘治十五年(1502年)十月,戶部指出“銀承備庫先前進,金止備成造金冊支用;銀止備軍官折俸及兵荒支給,近年累稱不足。金則以稅糧折納及於京市買過八千三百八十六兩有奇,五次取太倉銀共一百九十五萬,”而從戶部納入內庫的銀兩,全部都被孝宗挪用來大興土木,又妝造武當山神像,各寺觀修齋賞賜,修齋設醮等,恣意浪費,以致府藏空竭,國庫捉襟见肘。而且孝宗在統治中期(1500年)後,漸漸不如當初勤政,且開始縱容外戚,措置乖方,如內閣輔臣劉健,徐溥就曾批評孝宗說「切見數月以來視朝漸遲多至日出」,「近年以来用度太侈,光禄寺支费增数十倍,各处织造降出新样动千百匹,显灵朝天等宫泰山武当等处修斋设醮费用累千万两,太仓官银存积无几,不勾给边而取入内府至四五十万,宗藩贵戚求讨田土占夺盐利动亦数十万。」,「事涉於近幸貴戚,牢不可破,或旨從中出,略不預聞,或有所議擬,徑行改易。」,而閣臣李東陽也曾直言弘治後期「冗食太眾,國用無經,差役頻煩,科派重疊。京城土木繁興,供役軍士財力交殫,每遇班操,寧死不赴;勢家巨族,田連郡縣,猶請乞不已。親王之藩,供億至二三十萬。」「天津一路,夏麥已枯,秋禾未種,挽舟者無完衣,荷鋤者有菜色。盜賊縱橫,青州尤甚。南來人言,江南、浙東流亡載道,戶口消耗,軍伍空虛,庫無旬日之儲,官缺累歲之俸。」「今天下民窮財盡,其勢已極。姑以三者言之,山東之地草根樹皮掘食殆盡,繼以人肉,荊沔諸湖水竭魚荒,河泊諸課率多折納,易州山廠林木已空,漸出關外一二百里,其他賦稅大抵皆然,天下之地無一處而不貧」。朝中大臣如禮部尚書倪岳也上疏極言道「(孝宗)近日視朝頗晏,聽納頗難,經筵稀,御用度漸侈,游幸漸頻,進貢之止者複來,樂戲之斥者複取。」但孝宗也不願意聽納,而名臣劉大夏請辭時也言「臣老且病,窃见天下民穷财尽,脱有不虞,责在兵部,自度力不办,故辞耳。」,而吏部右侍郎周經則言「(孝宗)幸賞齋醮屢修,游宴無節,內帑空虗多由於此。」,南京戶科給事中張宦也上書道「近來(孝宗)費出無經,或橫恩濫賜之溢出,或修飾繕造之泛興,或祈禱遊玩之紛舉,偶因內帑稍闕即命太倉支取,耗散財物莫此為極」 「今四海民窮財盡,三邊將寡兵疲,糧草空虗,馬匹倒死而黠虜跳粱之勢,貪狼之心視昔尤勝」,禮科左給事中葉紳也言「邇來(孝宗)經筵稀御日講不舉,畫工琴士承恩於便殿,教坊雜劇呈技於左右..少滯視朝時,晏鰲山觀燈或徹曉不休宮中燕享或竟日乃已。」,兵科给事中王廷相奏「今天下大可忧者,在于民穷财尽,其势渐不可为。然所以致此者有四,风俗奢侈也,官职冗滥也,征赋太繁也,酒酿无节也」。可見弘治中晚年皇帝倦勤,國家敗政拮据,百姓困苦的情況。

在統治的十八年中,召見閣臣的次數總共有九次,比成化帝二十三年來召見一次為多。明孝宗即位之初,會聽進閣臣的諫諍,但是後來用各種方法來搪塞閣臣和科道官的建議,使弘治初年所革除的弊政,不僅全部恢復,尚且有惡化之勢,如憲宗晚年的傳奉官號稱弊政,弘治初盡行革除,到了弘治十二(1499年)年五月,傳升乞升文職至八百四十餘員,武職至二百六十餘員,比成化末年增一倍。其次,在軍事方面,從弘治一朝起亦開始糜爛,邊備日弛,人浮於事,有效抵禦的入侵寥寥無幾,也不復當年成化一朝了。另外,有明一代,以弘治對外臣最為縱容厚待,動則大肆外戚藩王賞賜房屋和田地,甚至在一宗貴戚莊崎糾紛案中,偏幫小舅子張延齡,一次就得地一萬六千七百零五頃;又如曾在弘治十三年(1500年)二月,賜興王湖廣京山縣近湖淤地一千三百五十餘頃,旋在七月又賜岐王德安府田六百一十二頃等等,賞地史不絕書,引起嚴重的土地兼併問題。

弘治十八年(1505年)五月初七日,因偶染风寒,误服药物,鼻血不止而死,

當時“深山穷谷,闻之无不哀痛”。有遗命:“东宫年幼,好逸乐,先生辈善辅之。”是年十月葬於泰陵。長子明武宗繼位。

孝宗即位时所面临的政治局面混乱不堪,由于他父亲明宪宗在位后期重用宦官和奸佞,造成了“朝中皮秕政”的状况。为了振兴帝业,肃清吏治,他在人事上的改革和整顿,可謂大刀阔斧。对太监梁芳、礼部右侍郎李孜省等前朝奸佞惩罚严厉。将冒领官俸、总计三千多人的艺人、僧徒等一概除名。在清理过程中,朱祐樘注意方式、方法,没有大开杀戒,斬殺的只有罪大恶极的僧人继晓 。与此并举,孝宗开始任用贤能之士。1492年三月,孝宗下令吏、兵两部将两京文武大臣、在外知府守备以上的官吏姓名,全部抄录下来,贴在文华殿的墙壁上,遇有迁罢之人,随时更改。他还多次向吏部、都察院指出,提拔和罢免官吏的主要标准,是看此人有無实绩。由于孝宗注意任用贤能,明朝中期出现了许多名臣,形成了“朝多君子”的盛况。

朱祐樘即位初年,广开言路。上台不久,就出现了臣子纷纷上书的局面,连尚未做官的太学生也跃跃欲试,上书提出各种建议。孝宗也有奢侈的想法,于是计划在万寿山建造一座棕棚,以备登临眺望。太学生虎臣得知此事,力谏不可,负责这项工程的朝中官员担心獲罪,抓住虎臣。孝宗闻知此事,先取消了工程,且授予虎臣七品官,派往云南做了知县。孝宗还采纳了除早朝之外,再在便殿召见大臣,谋议政事,当面阅读奏章,下发指令的建议,开始增加“午朝”,每天在左顺门接见大臣,倾听他们对政事的见解。

有說法認為:孝宗统治期间所实行的一系列的政策,都自始至终地得以贯彻执行,然而有學者指出,在弘治十四年,孝宗因朝廷財政拮據,以及軍餉籌措有困難而下詔群臣商議辦法,大學士劉健上奏要求改革弊端,並絕無益之費,躬行節儉,孝宗卻未採取措施。至弘治十五年,國家財政入不敷出:「常入之賦,以蠲色漸減,常出之費,以請乞漸增,入不足當出。正純以前軍國費省,小民輸正賦而已。自景泰至今,用度雜辦,皆昔所無。民已重困,無可復增。往時四方豐登,邊境無調發,州縣無流移。今太倉無儲,內府殫絀,而冗食冗費日加於前。」但僅作出成效不大的修補政策。

1489年,内阁大臣刘吉数兴大狱,迫害了一批官员;信任太监李广,开始修炼斋蘸之术。孝宗對此自我检讨。

據美國牙醫學會的資料表示,明孝宗於1498年把短硬的豬猔毛插進一支骨製手把上成為牙刷。

1501年,崛起的鞑靼部落以十万骑兵从花马池、盐池杀入固原、宁夏境内,这一事件震惊了孝宗。为了加强军事力量,1502年,孝宗将刘大夏提升为兵部尚书,负责军事整顿。刘大夏核查了军队虚额人手,补进了大量壮丁,并请朱祐樘停办了不少“织造”和斋蘸。

作为改良,孝宗没有从制度上对百姓的税赋负担进行突出的改变,而在减轻百姓负担上,减免灾区的赋税征收。从1490年,河南因灾免秋粮始,他对每年奏报来的因灾免税要求,几乎是无一例外地表示同意。

清修《明史》高度评价明孝宗:明有天下,传世十六,太祖、成祖而外,可称者仁宗、宣宗、孝宗而已。仁、宣之际,国势初张,纲纪修立,淳朴未漓。至成化以来,号为太平无事,而晏安则易耽怠玩,富盛则渐启骄奢。孝宗独能恭俭有制,勤政爱民,兢兢于保泰持盈之道,用使朝序清宁,民物康阜。《易》曰:“无平不陂,无往不复,艰贞无咎。”知此道者,其惟孝宗乎!

《国榷》:孝宗在东宫,久稔知其习。首罢幸相,次第厘革,改步之初,中外鼓舞,晓然诵明圣,识上意所向也。优容言路,汇吁良士,六卿之长皆民誉,三事之登皆儒英。讲幄平台,天听日卑,老臣造膝之语,不漏属垣,少年恸哭之谈,尝为动色。故良楛鉴断,刑赏恬肃。虽寿宁之戚,天下艳之,然宠若窦宪,尚难泌水之园,骄即武安,未请考工之宅,则帝心端可知矣。

方志远在其著作《明代国家权力机构及运行机制》中对明孝宗持否定态度,称其“弱智”并详细解释道:“弘治时代夹在成化、正德之间,前有万贵妃、汪直与西厂,后有刘瑾、八虎及内行厂,加之成化帝的内向和正德帝的荒唐,故弘治帝被明人称为‘中兴之主’。清人作《明史·孝宗纪》,其赞曰:‘明有天下,传世十六,太祖、成祖而外,可称者仁宗、宣宗、孝宗而已。仁、宣之际,国势初张,纲纪修立,淳朴未漓。至成化以来,号为太平无事,而晏安则易耽怠玩,富盛则渐启骄奢。孝宗独能恭俭有制,勤政爱民,兢兢于保泰持盈之道,用使朝序清宁,民物康阜。’并称唯有孝宗知《易》所说的‘无平不陂,无往不复,艰贞无咎’之道。但黄仁宇在《万历十五年》中指出,孝宗之为文臣所称道,就是因为他比较愿意听文臣的摆布。而实际上,孝宗不仅为文臣摆布,更受内臣摆布,从其种种行事,应该是个智商较低或者说是一个相对弱智的皇帝。”方志远在书中表示将‘另具文考证’,但相关文章尚未问世,因此,关于这个评价也存在一定争议。

郭厚安在其著作《弘治皇帝大传》中称明孝宗“盛名之下,其实难副”。他表示“从总体上,他(明孝宗)比其祖父英宗、其父宪宗以及其子武宗、侄世宗等都要略高一筹,坏的方面也没有他们突出。因此可以说,他之所以受到赞颂,是与前后诸帝比较的结果”;“朱祐樘不过是一个‘中主’而已”;“总之,朱祐樘绝不是雄才大略、大有作为之君,当然也不是荒淫的昏君,而是平庸的、力求维持现状的‘太平天子’。”

查继佐的《罪惟录》中,对明孝宗的成就和不足如此评价:“帝业几于光昌矣。群贤辐辏,任用得宜,暖阁商量,尤堪口法。斥妖淫,辟冗异,停采献,罢传升,革仓差,正抽分,种种明断外,尤莫难于孝穆、孝肃之别祀,万贵妃之免议,于肃愍之旌功。所谓情而安之于义,又列辟之所不能忘也。升遐之日,万姓哀号,岂偶然哉!若夫待外戚过厚,赐予颇滥,冗员尚多,中贵太盛,或移心斋醮,纷费,盖积渐者久,未能遽革也。夫果深有得于《太极》、《西铭》诸图书,即何难骑龙而上仙哉!”查继佐尽管也为弘治辩解,但与上述史家不同的是,究竟委婉地指出了明孝宗的不足。

《朝鮮成宗實錄》上(朝鮮成宗)曰:“常慮建州野人邀截於中路,今卿好還,甚可喜也。中國太平乎?” 自貞曰:“太平。但聞皇帝不豫,朝會望見,天顔殊瘦,皇帝初卽位,皆稱明斷,今紀綱不嚴,雨暘不若,年穀不登,民甚困窮。向者朝會,朝臣各以位次序立,莫敢私語,今則或聚立私語,以此知紀綱不嚴也。”

《明朝時代上卷 第42章 弘治王朝的老生常談》:“後世史學家多將弘治王朝稱作“弘治中興”,但從更寬廣的歷史視野來看,這些其實都經不住推敲,從宣德王朝開始,文人們所認為的明朝衰敗,實際上並不存在。皇帝不臨朝、宦官跋扈、軍屯被破壞、京畿部分民田被侵占,這些在士大夫看起來,好像不可理喻的事情,實際上無關這個大明王朝的痛癢,正統、成化年間,我們的大明王朝仍舊是平穩、正常運行的,不僅如此,從中可以看出三個趨勢,那就是政治依賴日益成熟、穩定的官僚集團運作,商業貿易開始興起,哲學文化思想領域開始鬆動,這都是值得正面看待的事情。大歷史觀,對於歷史的觀察,不應該再是只從《是否符合儒家行為規範》來看待,如果繼續這樣看待歷史,就會使我們中國人陷入一種狹隘束縛的歷史發展桎梏中。 仔細分析正統、成化王朝的所謂衰敗,是因為史學家們以當時的君主統治行為,不符合儒家行為規範而已,而弘治王朝的所謂中興,也是因為弘治皇帝遵循了文人士大夫們的儒家王道意識,因為前朝感覺衰敗,才會存在後來的感覺中興。以弘治皇帝努力將自己塑造成一個仁君形象,這是值得嘉許的。但最後這些都無濟於事,皇帝的人性與權力,超越了士大夫們的儒家王道意識與封建禮法,這衝突使一代明君轉眼變為昏君,史學評論家立即改觀,對弘治王朝前面與後面的一個總結,就是不完美。”

有人根據清修《明史》、《明書》等資料記載,認為孝宗僅娶妻孝康敬皇后張氏一人,沒有其他妃嬪或妾室。並且孝宗的泰陵只葬有夫妻兩人。而實際上根據《勝朝彤史拾遺記》及《罪惟錄》所載,孝宗至少還有沈璚蓮、鄭金蓮(《罪惟錄》稱其小字黃兒)兩位選侍。因為各種史書中對於妃嬪傳記因有事跡可記、有立傳價值,取捨各有不同,參見《萬曆官修本朝正史研究》中「八種史書關於明太祖等十位皇帝后妃立傳情況表」。而大部分的妃嬪因為地位的關係都不能葬入明帝陵中。

至於孝宗宫中有五名夫人:敬順夫人邵氏,安和夫人周氏,安順夫人劉氏,榮順夫人孟氏及榮善夫人項氏。夫人在明朝制度並非妃嬪稱號,而是命婦的封號,如外命婦(公侯伯及一二品官正室)或內命婦(資深宮人或乳母褓姆)等,內命婦中,以皇帝的乳母最常在年老後因乳帝之功而被加封為夫人(如明孝宗的保姆封为佐圣夫人、天启帝的乳母奉圣夫人客氏、仁宗褓姆衛聖夫人楊氏等,皆是有夫有家的妇人)。另,榮善夫人項氏年龄比孝宗大四十四岁,比孝宗的祖父明英宗还大一岁。因此这五名夫人实际上不是明孝宗的妃嫔。

\subsection{弘治}

\begin{longtable}{|>{\centering\scriptsize}m{2em}|>{\centering\scriptsize}m{1.3em}|>{\centering}m{8.8em}|}
  % \caption{秦王政}\
  \toprule
  \SimHei \normalsize 年数 & \SimHei \scriptsize 公元 & \SimHei 大事件 \tabularnewline
  % \midrule
  \endfirsthead
  \toprule
  \SimHei \normalsize 年数 & \SimHei \scriptsize 公元 & \SimHei 大事件 \tabularnewline
  \midrule
  \endhead
  \midrule
  元年 & 1488 & \tabularnewline\hline
  二年 & 1489 & \tabularnewline\hline
  三年 & 1490 & \tabularnewline\hline
  四年 & 1491 & \tabularnewline\hline
  五年 & 1492 & \tabularnewline\hline
  六年 & 1493 & \tabularnewline\hline
  七年 & 1494 & \tabularnewline\hline
  八年 & 1495 & \tabularnewline\hline
  九年 & 1496 & \tabularnewline\hline
  十年 & 1497 & \tabularnewline\hline
  十一年 & 1498 & \tabularnewline\hline
  十二年 & 1499 & \tabularnewline\hline
  十三年 & 1500 & \tabularnewline\hline
  十四年 & 1501 & \tabularnewline\hline
  十五年 & 1502 & \tabularnewline\hline
  十六年 & 1503 & \tabularnewline\hline
  十七年 & 1504 & \tabularnewline\hline
  十八年 & 1505 & \tabularnewline
  \bottomrule
\end{longtable}


%%% Local Variables:
%%% mode: latex
%%% TeX-engine: xetex
%%% TeX-master: "../Main"
%%% End:

%% -*- coding: utf-8 -*-
%% Time-stamp: <Chen Wang: 2019-10-21 17:18:46>

\section{武宗\tiny(1505-1521)}

明武宗朱厚照(1491年10月27日-1521年4月20日),或稱正德帝,明朝第11代皇帝(1505年-1521年在位),享年 31歲,年号「正德」。

武宗是明朝极具争议性的统治者。他任情恣性,為人嬉乐胡鬧,荒淫无度。寵信宦官、建立豹房,強徵處女、娈童入宮,有時也搶奪有夫之婦,逸遊無度。施政荒誕不經,朝廷乱象四起。給自己化名為朱壽,自封為「鎮國公、總督軍務威武大將軍、總兵官」。又信仰密宗、伊斯蘭教等,自稱忽必烈(蒙古名,元世祖之名)、沙吉熬爛(波斯語,伊斯蘭教蘇菲派的蘇菲師)、大寶法王(藏密名,白教首領)。

另一方面,他為人刚毅果断,任内诛灭刘瑾,平定安化王、寧王之亂,在应州之役中击败達延汗,令鞑靼多年不敢深入,并积极学习他国文化,促进中外交流,体现出有为之君的素质,是一位功过参半的皇帝。

明武宗朱厚照为明孝宗嫡长子,生于1491年10月26日(弘治四年九月二十四日申时)。两岁被立为皇太子。唯一的弟弟朱厚炜又早夭,是孝宗唯一长大成人的儿子。弘治十一年春,皇太子出阁读书。他天性聪颖,讲筵时极为认真,面对讲师则恭敬对待。几个月后,便已知晓翰林院与左春坊所有讲师的姓名,以致有讲师缺席便会问询左右“某先生今日安在邪?”這讓孝宗极为喜爱,出游必带上皇太子。同时孝宗听闻皇太子闲暇时喜好兵戎事,认为他安不忘危,所以也不予以干涉。

弘治十八年五月初八日,孝宗皇帝驾崩。在完成文武百官军民耆老劝进的固定程序后,五月十八日,皇太子朱厚照即位,是为明武宗。

明正德九年正月,後來反叛的寧王朱宸濠獻新樣元宵四時花燈數百,窮極奇巧,內附火藥,明武宗命獻者入懸。时值冬季,宫中按例在檐下设有毡幕御寒。以致火星觸及氊幕,引發大火,自二鼓时分一直烧至天明。火势最大时,武宗正在前往豹房的途中,望见乾清宫的火灾,武宗向左右开玩笑称这是「好一棚大烟火也」兩天後壬午日,武宗以乾清宫灾御奉天門視朝,撤寶座不設,遂下詔罪己,並諭文武百官,同加修省。後又常常离开帝都燕京四处巡游。

住在京師期间,又不愿住在紫禁城,在宫外建了一座“豹房”居住,並甄選大量美女於其中供其淫樂。其男宠也不计其数,名曰“老儿当”,但也有學者稱,因為正德帝喜歡各地宗教,這些人主要是通曉漢文、蒙文、藏文或波斯文,作為宗教人士的翻譯官。

正德帝不喜上朝,起初宠信刘瑾、張永、丘聚、谷大用等号称“八虎”的宦官,1510年平定安化王之乱朱寘鐇后,下令将刘瑾凌迟处死,后又宠信武士江彬等人。

正德帝喜好宗教靈異、怪力亂神,终日与来自西域、回回、蒙古、乌斯藏(西藏)、朝鲜半島的异域法師、番僧相伴。正德帝曾学习蒙古语,自称忽必烈,也学藏传佛教,自称大宝法王。正德帝還曾亲自接见第一位来华的葡萄牙使者皮莱资。正德帝並因為自己生肖屬豬,曾一度敕令全国禁食猪肉,但他自己仍食用猪肉「内批仍用豕」;旋即在大學士杨廷和的反對下,降敕廢除。

正德帝“奋然欲以武功自雄”。正德十二年(1517年)10月,在江彬的怂恿下,自封为“镇国公總督軍務威武大將軍總兵官朱寿”,到边地宣府(今张家口宣化区)亲征,击溃蒙古鞑靼小王子(即达延汗巴图蒙克),回去后又给自己加封太师。史称“应州大捷”。

正德十四年(1519年)六月十四日,宁王朱宸濠在封藩江西南昌叛乱,是為宁王之乱,不過四十三天,就被贛南巡撫王陽明及吉安知府伍文定募集散兵游勇平定,斬殺三萬餘人,朱宸濠被擒。八月二十二日,武宗离开北京亲征。二十六日,武宗抵达涿州,此時王陽明平定叛乱的奏报送达,但武宗仍决定继续南幸。十二月十一日,武宗传谕内阁,以正德十五年(1520年)元旦於南京朝贺、祭祀天地。十二月二十六日,武宗御驾抵应天府。次日,祭祀南京太庙,武宗成为自永乐以后重新驾临南京的皇帝。正德十五年闰八月初八日,武宗於南京受宁王降。八月十二日,武宗离京返回北京。

正德八年(1513年)起在江南全面推行的賦稅改革,既減輕了江南當地百姓的負擔,更使從弘治晚期開始,江南地區拖欠中央累積十年之久的賦稅,僅經兩年時間就全部還清。

武宗御驾南征返回北京途中,於淮安清江浦上学渔夫撒网,作為遊戲,卻失足落入水中,并因此患病「燥熱難退」。正德十五年十二月初十,大驾回到北京,文武百官出至正阳桥外迎接。十三日,皇帝於南郊祭祀天地,祭拜过程中突然呕血,随即送入斋宫休养。次日,返回大内,仅在奉天殿举行庆成礼。此后,立春日的朝贺一同免去。正德十六年(1521年)正月初九日,监察御史郑本公鉴于武宗身体状况不乐观,上奏武宗,望能於宗室间過繼一人主掌东宫,但后来武宗身体略有好转。三月十三日晚间,武宗突然向身边的太监陈敬和苏进表示自己可能無法痊癒,让其召司礼监并禀告皇太后,由太后与内阁议处天下事,并表示自己耽误子嗣。十四日,武宗於豹房驾崩,得年29歲。

由于武宗無子嗣,因此遵照《皇明祖训》,由武宗堂弟、孝宗弟兴献王朱祐杬之子兴王朱厚熜入嗣大统。正德十六年五月,朱厚熜抵达京师,上谥号为承天达道英肃睿哲昭德显功弘文思孝毅皇帝,上庙号为武宗。九月,武宗入葬天寿山陵区的康陵。

明武宗的生辰为弘治四年九月二十四日,八字为辛亥年,戊戌月,丁酉日,戊申时出生。其中,八字地支分别为申酉戌亥,这种排列方法被称为连如贯珠。在此以前仅太祖朱元璋的八字与此类似。

賜自己的替僧為漢地噶瑪巴,正德五年封大慶法王,鑄大慶法王西天覺道圆明自在大定慧佛金印,兼给誥命,藏名為「領占班丹」,並曾邀請藏地八代噶瑪巴至北京(七代噶瑪巴曾說:「將現身兩位噶瑪巴」);蒙古名為忽必烈;波斯名為沙吉熬爛,即蘇菲師(Shaykh,回教蘇菲派長者、教長),並擁有一群伊斯蘭火者,稱為老兒當。對道教亦多有了解,可能曾號錦堂老人。

正德十五年(1520年)闰八月,武宗御驾自南京返回时,途径镇江,适逢退休居家的原内阁大臣靳贵病逝,于是亲临靳贵家中吊唁。但是随行大臣代皇帝撰写的祭文皆不能称意,明武宗遂亲自写道:“朕居东宫,先生为傅。朕登大宝,先生为辅。朕今南游,先生已矣。呜呼哀哉!”左右的侍从文学之臣看后都敛手称服。

山西应县木塔顶层有一方明武宗皇帝御匾“天下奇观”。

2004年,在美國德州一位華僑手中發現由明朝正德皇帝親筆所書的聖旨,內容敘述做人應如何有進取心以及如何為忠君之臣與正人君子。此文物的發現造成了史學家對歷史記載正德皇帝人格的爭議。

史学界对正德帝的评价不一, 有人认为正德帝雖荒淫無行,行徑胡鬧,不理國政,造成叛變日起,且自身壯年即因為逸樂而死;但是亦有人认为他頗能容忍大臣,不罪勸諫之人。君臣之間,相安無事,知错能改,诛灭奸佞。

张廷玉等《明史》贊曰:「明自正統以來,國勢浸弱。毅皇手除逆瑾,躬禦邊寇,奮然欲以武功自雄。然耽樂嬉遊,暱近群小,至自署官號,冠履之分蕩然矣。猶幸用人之柄躬自操持,而秉鈞諸臣補苴匡救,是以朝綱紊亂,而不底於危亡。假使承孝宗之遺澤,制節謹度,有中主之操,則國泰而名完,豈至重後人之訾議哉!」

談遷《國榷》論曰:「武宗少即警敏,好佚樂。……而武宗又不罪一諫臣,元相呵護,群吏奉法。……夜半出片紙縛(劉)瑾,……錢寧俛首受罪。」

吳熾昌《續客窗閒話》論曰:「……遊戲中確有主裁,但好行小慧,為儒尚且不可,況九五之尊耶?今之讀史者直以帝比之桀紂,無乃過甚。當初諡曰武宗毅皇帝,毅者果決之謂,可見遇事實能決斷,非盡阿諛可知矣。」

\subsection{正德}

\begin{longtable}{|>{\centering\scriptsize}m{2em}|>{\centering\scriptsize}m{1.3em}|>{\centering}m{8.8em}|}
  % \caption{秦王政}\
  \toprule
  \SimHei \normalsize 年数 & \SimHei \scriptsize 公元 & \SimHei 大事件 \tabularnewline
  % \midrule
  \endfirsthead
  \toprule
  \SimHei \normalsize 年数 & \SimHei \scriptsize 公元 & \SimHei 大事件 \tabularnewline
  \midrule
  \endhead
  \midrule
  元年 & 1506 & \tabularnewline\hline
  二年 & 1507 & \tabularnewline\hline
  三年 & 1508 & \tabularnewline\hline
  四年 & 1509 & \tabularnewline\hline
  五年 & 1510 & \tabularnewline\hline
  六年 & 1511 & \tabularnewline\hline
  七年 & 1512 & \tabularnewline\hline
  八年 & 1513 & \tabularnewline\hline
  九年 & 1514 & \tabularnewline\hline
  十年 & 1515 & \tabularnewline\hline
  十一年 & 1516 & \tabularnewline\hline
  十二年 & 1517 & \tabularnewline\hline
  十三年 & 1518 & \tabularnewline\hline
  十四年 & 1519 & \tabularnewline\hline
  十五年 & 1520 & \tabularnewline\hline
  十六年 & 1521 & \tabularnewline
  \bottomrule
\end{longtable}


%%% Local Variables:
%%% mode: latex
%%% TeX-engine: xetex
%%% TeX-master: "../Main"
%%% End:

%% -*- coding: utf-8 -*-
%% Time-stamp: <Chen Wang: 2021-11-01 17:13:00>

\section{世宗朱厚熜\tiny(1521-1566)}

\subsection{生平}

明世宗朱厚熜(1507年9月16日-1567年1月23日),或稱嘉靖帝,明朝第12位皇帝,庙号世宗,年號嘉靖,正德十六年(1521年),明武宗駕崩無嗣,內閣首輔楊廷和立朱厚熜入繼大統,即明世宗。谥号“钦天履道英毅神圣宣文广武洪仁大孝肃皇帝”。

世宗前期进行改革,銳意圖治,颇有作為,他说:“今天下诸司官员,比旧过多。我太祖初无许多,后来增添冗滥,以致百姓艰窘,日甚一日。”下令革除先朝蠹政,又嚴以馭下,史稱其“世宗習見正德時宦侍之禍,即位後御近侍甚严,有罪挞之至死,或陈尸示戒...又盡撤天下鎮守內臣及典京營倉場者,終四十餘年不復設,故內臣之勢,惟嘉靖朝少殺雲。”,先後裁革錦衣衛十七萬餘人。且寸斬前朝王綸、钱宁和江彬等奸臣,天下翕然稱治,時稱嘉靖中興。

但世宗受人詬病處更多,如他為了追封生父興獻王的問題,與楊廷和等朝臣引發嚴重衝突,即大禮議事件,世宗為了此事,對大臣們進行了嚴重的大清洗。世宗在位中後期也漸無心朝政,深居不出,沉迷方術,只通過內閣掌控朝局,使得嚴嵩嚴世蕃父子專權逐漸形成,又因營建繁興而濫用民力,導致府藏告匱,民眾起義無數。在宮中,世宗也暴虐無道,因為虐待宮女,導致宮女發動壬寅宮變,險些喪命。

明世宗朱厚熜是明宪宗第四子兴献王朱祐杬次子,是明孝宗之姪,明武宗之堂弟;明武宗正德二年(1507)生,母兴王妃蒋氏。

正德十六年(1521年),明武宗驾崩,無子嗣,内阁首辅吏部尚书、武英殿大学士杨廷和定策,援引《皇明祖训》,推找皇位繼承人,而武宗唯一弟弟朱厚煒幼年夭折,於是上推至武宗父明孝宗一輩,孝宗是明憲宗的第三子,兩名兄長皆早逝無子嗣,四弟興王朱祐杬雖已薨,但有二子,興王長子(朱厚熙)已薨,遂以“兄終弟及”的原則,徵在服喪的興國世子朱厚熜入京即位。朱厚熜先繼承興王頭銜,後即帝位,改元“嘉靖”,是为明世宗。

朱厚熜十四歲入繼大統,因想追封親生父母「皇帝、皇后」的尊號,但首辅杨廷和等旧臣要求他改以明孝宗為義父,而引發了長達三年半的大禮議之爭,期間廷杖打死十六人;世宗不顧朝臣反對,追尊生父為興獻帝、生母為興國皇太后,改稱孝宗曰“皇伯考”。嘉靖十七年(1538年)九月興獻帝被追尊為「睿宗知天守道洪德淵仁寬穆純聖恭簡敬文獻皇帝」,並將睿宗的牌位升袝太廟,排序在明武宗之上,改興獻王墓為顯陵,大禮議事件至此最終結束。

嘉靖帝前期推行了改革,成效显著。河南道御史刘安说:“今明天子综核于上,百执事振于下,丛蠹之弊,十去其九,所少者元气耳。”张居正在万历三年(1575)以自己少年时的亲身体验对嘉靖前期整顿学政的成就予以极高的评价。他说:“臣等幼时,犹及见提学官多海内名流,类能以道自重,不苟徇人,人亦无敢干以私者。士习儒风,犹为近古。”

隆庆二年(1568)进士李乐对嘉靖前期革除镇守中官的积极作用给予的评价,言道:“世宗皇帝继统,年龄虽小,英断夙成,待此辈不少假借。又得张公孚敬以正佐之,尽革各省镇守内臣,司礼监不得干预章奏。往瑾时,公卿大臣相见,无敢抗礼,甚有拜伏者。自张公当国,司礼以下各监局巨珰,见公竦息敬畏,不敢并行并坐,至以『张爷』呼之,不动声色,而潜消其骄悍之心。盖自汉唐宋元以来,宦官敛戢,士气得伸,国体尊严,未有如今日者,诚千载一时哉!”

因應外戚为害天下,嘉靖帝和张璁、方献夫在革除外戚世封的问题上达到了共识,下令永远废除此制,《明通鉴》编纂者说:“安昌伯钱维圻卒,其庶兄维垣请嗣爵,下吏部议。尚书方献夫等言:‘外戚之封,不当世及。’历引汉、唐、宋事以证。璁以为然,力主之。上善其言,诏:”自今外戚封爵者,但终其身,毋得请袭。’自是,外戚遂永绝世封。”

明代史学家何乔远《名山藏》总结嘉靖前期“励精化理,湔濯海内观听,挈清政本,杜塞旁落,奋武揆文,网罗才实。至于稽古礼典,取次厘毖一切,创必表章,轶往宪来,赫然中兴,多孚敬(张璁)所翼赞”。何乔远认为嘉靖前期出现的国家中兴是得益于內閣首辅张璁推行的改革。

而在嘉靖中后期,海瑞于嘉靖四十五年亦言:世宗“二十余年不视朝,法纪弛矣”。

世宗濫用夫役與國家財政之力大事興建,迷信方士、尊崇道教,好長生不老之術,每年不斷修設齋醮,造成巨大的靡費。

世宗好房中術秘方,多採處女之經血煉丹,方士陶仲文與佞臣顾可学、盛端明等进献媚药得以倖進,世宗為人暴躁兇殘,朝鮮國使臣的著作,也稱他對宮女:「若有微過,多不容恕,輒加箠楚。因此殞命者,多至二百餘人。」嘉靖二十一年(1542年)十月爆發“壬寅宮變”,幾死於宮女之手。明朝的太醫許紳用“虎狼之药”救活世宗,但是,由于他在急救世宗皇帝时,承受着“不效必杀身”的巨大压力,不多久,许绅得了病,卧床不起,嘉靖帝来看望他。他说:“吾不起矣,曩者宫变,吾自度不效必杀身,因此惊悸,非药石所能疗。”病卒,赐谥恭僖。此後世宗相繼遷居西苑萬壽宮及玉熙宮謹身精舍,至死不曾回到紫禁城大內居住,直至瀕死前才在徐階以明武宗死在宮外為例子勸說下回到大內居住。首輔严嵩專國二十年,殘害忠良,楊繼盛、沈鍊等朝臣慘遭殺害。

嘉靖朝吏治敗壞,爆发多起农民起义,如:山東礦工起義、陳卿起義、蔡伯貫起義、浙贛礦工起義、李亞元起義、賴清規起義,邊事廢弛,1524年以後爆發多起大同兵變,1535年爆發遼東兵變,1560年爆發振武營兵變,長城北方蒙古鞑靼俺答汗寇邊,倭寇侵略中国東南沿海,就是“北虜南倭”的問題,後賴朱紈、戚繼光、俞大猷等人率軍肅清倭寇。世宗在位之時,葡萄牙人遠航當時屬广东省香山县管辖的澳門,並“借地晾晒水浸货物”为借口開始於澳門定居,從而在澳門展開了接近450年的葡萄牙佔領及殖民時期。

嘉靖四十四年(1565年)正月,方士王金等伪造《诸品仙方》、《养老新书》,制长生妙药献世宗。嘉靖四十五年二月,(1566年)戶部主事海瑞上《治安疏》,世宗初大怒,擲疏於地,並下詔讓錦衣衛及三法司論罪。但后重置御案上數日內再三閱讀。后法司擬處大辟的刑罰,但世宗審閱後卻留中不發,以致海瑞終未獲刑。

嘉靖四十五年十二月初八,世宗免去臘宴。十四日,世宗病篤,時隔二十多年重新住回大內。當日午時,於乾清宮駕崩,享壽六十岁。徐階請裕王入宮主持大行皇帝喪禮。裕王自東安門入,至乾清宮御榻前發喪。次日,大行皇帝小殮,并發佈遺詔。十六日,大殮,并上廟號世宗。

隆慶元年三月十一日,世宗梓宮及祔葬孝洁皇后、孝恪皇后梓宫離開北京。十六日,世宗及孝潔皇后、孝恪皇后梓宮抵達永陵。次日,世宗入葬永陵。

嘉靖皇帝醉心于西苑修仙斋醮,直到他最后死去,却一直是“虽深居渊穆而威柄不移”,虽数十年不见朝臣,仍能做到“大张弛、大封拜、大诛赏,皆出独断,至不可测度。”明世宗非常聪明,也十分勤奋,批阅奏书票拟经常到后半夜。但嘉靖后期,朝中官员贪污纳贿、奢侈靡费,確已成普遍的现象。

《明史·世宗本紀》:“贊曰:世宗御極之初,力除一切弊政,天下翕然稱治。顧迭議大禮,輿論沸騰,幸臣假托,尋興大獄。夫天性至情,君親大義,追尊立廟,禮亦宜之;然升祔太廟,而躋於武宗之上,不已過乎!若其時紛紜多故,將疲於邊,賊訌於內,而崇尚道教,享祀弗經,營建繁興,府藏告匱,百餘年富庶治平之業,因以漸替。雖剪剔權奸,威柄在御,要亦中材之主也矣。”

《国榷》:“世庙起正德之衰,厘革积习,诚雄主也。因议礼自裁,好稽古右文之事,诸臣迎附,祗诤于仪节,反实政略焉。”

《名山藏》:“臣喬遠曰:臣每見故縉紳父老,若為郎時尚接先朝皆御之臣,多好言嘉靖時事,其謨猷合聖賢,動作掀天地,真中興之主矣。晚節西苑崇玄,帝心固以為敬天,雖萬幾在宥而精神無時不運,於天下者四十餘年如一日,所以饗世獨久歟。”

\subsection{嘉靖}

\begin{longtable}{|>{\centering\scriptsize}m{2em}|>{\centering\scriptsize}m{1.3em}|>{\centering}m{8.8em}|}
  % \caption{秦王政}\
  \toprule
  \SimHei \normalsize 年数 & \SimHei \scriptsize 公元 & \SimHei 大事件 \tabularnewline
  % \midrule
  \endfirsthead
  \toprule
  \SimHei \normalsize 年数 & \SimHei \scriptsize 公元 & \SimHei 大事件 \tabularnewline
  \midrule
  \endhead
  \midrule
  元年 & 1522 & \tabularnewline\hline
  二年 & 1523 & \tabularnewline\hline
  三年 & 1524 & \tabularnewline\hline
  四年 & 1525 & \tabularnewline\hline
  五年 & 1526 & \tabularnewline\hline
  六年 & 1527 & \tabularnewline\hline
  七年 & 1528 & \tabularnewline\hline
  八年 & 1529 & \tabularnewline\hline
  九年 & 1530 & \tabularnewline\hline
  十年 & 1531 & \tabularnewline\hline
  十一年 & 1532 & \tabularnewline\hline
  十二年 & 1533 & \tabularnewline\hline
  十三年 & 1534 & \tabularnewline\hline
  十四年 & 1535 & \tabularnewline\hline
  十五年 & 1536 & \tabularnewline\hline
  十六年 & 1537 & \tabularnewline\hline
  十七年 & 1538 & \tabularnewline\hline
  十八年 & 1539 & \tabularnewline\hline
  十九年 & 1540 & \tabularnewline\hline
  二十年 & 1541 & \tabularnewline\hline
  二一年 & 1542 & \tabularnewline\hline
  二二年 & 1543 & \tabularnewline\hline
  二三年 & 1544 & \tabularnewline\hline
  二四年 & 1545 & \tabularnewline\hline
  二五年 & 1546 & \tabularnewline\hline
  二六年 & 1547 & \tabularnewline\hline
  二七年 & 1548 & \tabularnewline\hline
  二八年 & 1549 & \tabularnewline\hline
  二九年 & 1550 & \tabularnewline\hline
  三十年 & 1551 & \tabularnewline\hline
  三一年 & 1552 & \tabularnewline\hline
  三二年 & 1553 & \tabularnewline\hline
  三三年 & 1554 & \tabularnewline\hline
  三四年 & 1555 & \tabularnewline\hline
  三五年 & 1556 & \tabularnewline\hline
  三六年 & 1557 & \tabularnewline\hline
  三七年 & 1558 & \tabularnewline\hline
  三八年 & 1559 & \tabularnewline\hline
  三九年 & 1560 & \tabularnewline\hline
  四十年 & 1561 & \tabularnewline\hline
  四一年 & 1562 & \tabularnewline\hline
  四二年 & 1563 & \tabularnewline\hline
  四三年 & 1564 & \tabularnewline\hline
  四四年 & 1565 & \tabularnewline\hline
  四五年 & 1566 & \tabularnewline
  \bottomrule
\end{longtable}


%%% Local Variables:
%%% mode: latex
%%% TeX-engine: xetex
%%% TeX-master: "../Main"
%%% End:

%% -*- coding: utf-8 -*-
%% Time-stamp: <Chen Wang: 2021-11-01 17:13:06>

\section{穆宗朱載坖\tiny(1567-1572)}

\subsection{生平}

明穆宗朱載坖(“坖”音“jì”,1537年3月4日-1572年7月5日),或稱隆慶帝,明朝第13位皇帝,庙号“穆宗”,谥号“契天隆道渊懿宽仁显文光武纯德弘孝莊皇帝”。

朱载坖的名讳在万历年间被武纬子误记为朱載塈(“塈”音“jì/ㄐㄧˋ”),崇祯年间被朱国祯等误记为朱載垕(“垕”音“hòu/ㄏㄡˋ”),导致清代文献、越南文献、朝鲜文献对穆宗名讳记载的混乱。

明穆宗是明世宗第三子,嘉靖十六年(1537)生,母亲是康妃杜氏。嘉靖十八年(1539)二月,明世宗册立次子朱载壡为太子、三子朱载坖为裕王、四子朱载圳为景王。嘉靖二十八年(1549)三月,太子朱载壡薨,裕王朱载坖以次序当为太子。由于明世宗次子朱载壡早逝,所以迟迟未予册立。时景王朱载圳年少,服色与裕王朱载坖无别,引起朝野议论。嘉靖四十(1561)年二月,明世宗命景王朱载圳出居封国,以杜绝其觊觎之心和朝野议论。嘉靖四十四年(1565)正月,景王朱载圳薨,明世宗对内阁首辅建极殿大学士徐阶说:“此子素谋夺嫡,今死矣。”

嘉靖四十五年十二月(公元1567年1月),明世宗驾崩,裕王朱载坖即位,改元隆庆,是为明穆宗。明穆宗立即纠正其父的弊政,之前以言获罪的诸臣全部召用,已死之臣抚恤并录用其后,方士交付有司论罪,以前的道教仪式全部停止,免除次年一半田赋及嘉靖四十三年以前的所有欠赋;又停止明世宗为博孝名强行施行的明睿宗(即明世宗本生父兴献王)明堂配享之礼(即秋季祭天,要以在位皇帝之父合祭,为此导致明太宗庙号被改为明成祖)。

隆庆帝重用徐阶、李春芳、高拱等内阁辅臣,致力于解决困扰朝局多年的“北虏南倭”问题,隆庆元年(1567年),采纳内阁大学士高拱、张居正的建议,与蒙古俺答议和,結束與蒙古長達二百年的戰爭,並有俺答封贡。同年宣布废除海禁,允许民间私人远贩东西二洋,史称隆庆开关。隆庆新政是明穆宗统治时期所出现的承平时期。

明穆宗力行节俭,信用内阁辅臣,并不加以掣肘,但也不能制止内阁辅臣之间的倾轧,这也与其本人仁厚而平庸的性格有关,即位后,首先宣告天下,将废除明世宗时期的所有弊政,一时间朝廷内外都希望新君能有所作为。但是,革弊施新取得实效没多久,他開始宠信太监膝祥等人,挥霍无度,纵情声色,荒废朝政。即位后不久,很快就将权力交给了以高拱为首的内阁,以后只召见过两次阁臣,而他自己就在后宫享乐,广修宫苑,犬马歌舞。

坊间传闻明穆宗特别好色,整天在后宫里忙来忙去,被人比做后宫中辛勤的蜜蜂。他長期服用春药,每天要数名美女陪伴。宫中的用品,小到茶杯,大到龙床,全部都有男欢女爱的雕刻和彩绘。对此,很多大臣都曾上书进谏,竭力劝阻,但他总是很温和地说:“国事有先生我就放心了,家事就不劳先生费心了”。

由于明穆宗贪于女色,纵情声色,加上长期服食春药,他的身体每况日下,难以支撑,萬曆野獲編称其“阳物昼夜不仆,遂不能视朝”。

隆庆六年(1572年)闰三月,宫中传出了明穆宗病危的消息。在休养了两个月之后,他又上朝视事,却又突然头晕目眩,支持不住而回宫。他自知病情不轻,急召高拱、张居正及高仪三人接受顾命,吩咐由太子继位,后崩于乾清宮,终年三十六岁,后被谥为庄皇帝,庙号穆宗,葬于北京昌平明昭陵。

《明穆宗实录》:“上即位,承之以宽厚,躬修玄默,不降阶序而运天下,务在属任大臣,引大体,不烦苛,无为自化,好静自正,故六年之间,海内翕然,称太平天子云。”

《明史》:“穆宗在位六载,端拱寡营,躬行俭约,尚食岁省巨万。许俺答封贡,减赋息民,边陲宁谧。继体守文,可称令主矣。第柄臣相轧,门户渐开,而帝未能振肃乾纲,矫除积习,盖亦宽恕有余,而刚明不足者欤!”

《国榷》:“迹帝之终始,宽大如仁庙,而精勤不若也。安豫如宪朝,而控纵不若也”

《名山藏》:“上端凝靜密,不殺自威,不察自智,優崇輔弼,假借臣僚用能守祖宗之法以致中國乂寧,外夷向風之盛,蓋清靜合軌漢帝寬仁,比跡宋宗矣。上在潛邸時,食驢腸而甘,及即位間問左右,左右請詔光祿,上不忍曰「若爾,則光祿日宰一驢矣。」歲時游吳行幸,諸供膳光祿先期請上旨為豐約,上常裁取最約者焉。”

\subsection{隆庆}

\begin{longtable}{|>{\centering\scriptsize}m{2em}|>{\centering\scriptsize}m{1.3em}|>{\centering}m{8.8em}|}
  % \caption{秦王政}\
  \toprule
  \SimHei \normalsize 年数 & \SimHei \scriptsize 公元 & \SimHei 大事件 \tabularnewline
  % \midrule
  \endfirsthead
  \toprule
  \SimHei \normalsize 年数 & \SimHei \scriptsize 公元 & \SimHei 大事件 \tabularnewline
  \midrule
  \endhead
  \midrule
  元年 & 1567 & \tabularnewline\hline
  二年 & 1568 & \tabularnewline\hline
  三年 & 1569 & \tabularnewline\hline
  四年 & 1570 & \tabularnewline\hline
  五年 & 1571 & \tabularnewline\hline
  六年 & 1572 & \tabularnewline
  \bottomrule
\end{longtable}


%%% Local Variables:
%%% mode: latex
%%% TeX-engine: xetex
%%% TeX-master: "../Main"
%%% End:

%% -*- coding: utf-8 -*-
%% Time-stamp: <Chen Wang: 2021-11-01 17:13:44>

\section{神宗朱翊鈞\tiny(1572-1620)}

\subsection{生平}

明神宗朱翊鈞(1563年9月4日-1620年8月18日),或稱萬曆帝,為明朝第14代皇帝,年号万历,是明穆宗朱载坖的第三子。隆慶六年(1572年),穆宗駕崩,九岁的朱翊鈞登基,是为明神宗。在位48年,是明代在位時間最長的皇帝,谥号為「範天合道哲肃敦简光文章武安仁止孝显皇帝」。

明神宗在位前十五年,明朝一度呈現中興景象,史稱萬曆中興,而在位中期亦主持万历三大征,保護藩屬,巩固疆土。在張居正死後始親政,因國本之爭等問題而倦於朝政,自此不上朝,國家機器運轉幾乎停擺,徵礦稅亦被評一大病。萬曆年間也走向活潑和開放,利瑪竇覲見萬曆帝,開始西學東漸,但同時朝廷內東林黨爭開始萌芽、塞外又有後金勢力虎視眈眈,在其晚年佔領明朝東北大部分地區,使明朝退守山海關,終走向滅亡的局面。

明神宗是明穆宗的第三子。出生时,父亲尚为裕王,母親李氏为王府宮女出身。父亲裕王的第一任王妃李氏所生二子──朱翊鈴、朱翊釴均早夭。他实际上成为裕王的长子。另,嫡母继妃陈氏无子。

在其父继位后的隆慶二年(1568年),他被立為皇太子,明穆宗對其很有期望,改名钧,意思是「夫钧者,言圣王制驭天下犹制器者之转钧也」。幼時朱翊钧就十分聰惠,明穆宗在宮中騎馬時,年幼的朱翊钧就大叫道「父皇為天下之主,獨騎疾騁,萬一馬驚,卻如何是好?」穆宗聽後恩喜萬分,就更喜愛朱翊钧了,馬上下馬過來摟朱翊鈞在懷裡褒賞一番。其母李贵妃教子非常嚴格,隔三差五就把兒子叫到面前諄諄教誡一番,每次經筵結束以後,都少不得督促考問他今天所學的內容。朱翊鈞小時候稍有懈怠,李贵妃就將其召至面前長跪。

隆慶六年,父亲明穆宗駕崩,朱翊鈞即位,改元萬曆,堅持按照祖宗舊制,舉日講,御經筵,讀經傳、史書。而他每天读书亦十分用功,朝章典故都读很多遍,即使是隆冬盛暑亦从不间断,以後隨朱翊钧年渐长而学愈进,他自己后来也常常十分得意地说:“朕五岁即能读书。”另外他的書法也十分出色,筆劃遒勁,經常親自賜墨寶給大臣,連張居正仔細端詳作品,也不得不承認皇帝的書法是「揮瀚灑墨,初若並不經意,而鋒穎所落卻是奇秀天成」,但張居正終究認為他應該成為一位聖君而非書法家,便劈頭蓋臉奏訓一頓,自此直到張居正死後朱翊鈞才重新接觸書法。

神宗在位之初十年尚處年幼,由母親李太后代為聽政。即位之初內閣紛爭傾軋,閣臣之間關係惡劣,時高拱以主幼國危,痛哭時偶然說了一句:「十歲太子如何治天下」,引起朱翊鈞極為不滿,最後在張居正與馮保添油加醋下罷免了高拱。太后將一切內務大事交由馮保,而大柄悉以委居正,軍政皆由張居正主持裁決,独握大权。

在小皇帝朱翊钧以及李太后全力的支持下,張居正大刀阔斧地實行了一条鞭法等一系列改革措施,清丈田畝,改革赋税,整飭軍備,考察官吏,使社会经济有很大的发展,人民生活也有所提高,一改前弊。萬曆初年太倉的積粟達1300萬石,可支用十年,僅僅是太僕寺的銀兩儲蓄便多達四百餘萬,而太倉庫更是有超過千萬兩的積蓄,國家繁荣昌盛,扭轉明中業以來的頹勢,是為「萬曆中興」。後人在論及此段發展情況時,多歸功於張居正的鞠躬盡瘁,而對朱翊鈞的傾心委任卻往往忽視,實際上,隨朱翊鈞年紀長大,他也不再是名義上的擺設,張居正可以勸導、利用他幹什麼,卻不能強迫他做出違心之事,因此張居正也有無可奈何之時。

神宗幼年,太后及張居正都希望其成為儒家所倡導的皇帝典範。萬曆八年,神宗因和太監孫海、容用出遊行為輕浮不檢,太監馮保告知李太后。太后大怒,數落道「天下大器豈獨爾可承耶」,並拿出以霍光罷黜昌邑王之事威脅神宗,帝師張居正又乘機捉刀,寫下罪己詔,言詞犀利,以警惕皇帝。雖然保住皇位,但也因此使神宗認為顏面盡失。一次神宗在讀《論語》時,誤將「色勃如也」之「勃」字讀作「背」音,張居正厲聲糾正:「當作勃字!」聲音太大,嚇得神宗驚惶失措,在朝的大臣無不大驚。沈德符在《萬曆野獲編》中說:「(張居正輔政)宮府一體,百辟從風,相權之重,本朝罕儷,部臣拱手受成,比於威君嚴父,又有加焉。」「江陵(張居正)以天下為己任,客有諛其相業者,輒曰我非相,乃攝也。」晚年張居正的權勢之大,威權赫奕,連神宗都有所忌憚,曾經有丘岳由亞卿左遷藩參,曾以黃金製對聯饋張居正「日月並明,萬國仰大明天子;丘山為岳,四方頌太岳相公。」張居正奉旨歸喪時,地方大員行長跪禮,撫按大吏越界迎送,空前絕後。而奪情以後,張居正也日益偏恣,好同惡異,左右用事之人多通賄賂,時人益惡之,神宗亦意識到張居正的權力過大,“幾乎震主”,為後期清算張居正埋下伏筆。張居正死後,二十歲的神宗始親政。

古籍文獻記載,神宗親政後勵精圖治,虛心納諫,屢蠲賦稅,生活節儉,如僅在萬曆十一年間,蠲免並災傷織造議留就已達銀一百七十六萬一千兩。北京乾旱,神宗關心民瘓,親自以旱詔中外理冤抑,釋鳳陽輕犯及禁錮年久的犯人。另親自步行至天壇祈雨,皇上齋戒,親躬步行將近二十里的路程而不乘車輦出,且絲毫沒有因驕陽酷日而為難的樣子,其舉止從容不迫,表現的肅穆得體,百姓能一睹天顏,紛紛舉首加額高呼「聖德爾」,另外又敕六部都察院等曰:「天旱雖由朕不德,亦天下有司貪婪,剝害小民,以致上乾天和,今後宜慎選有司。」蠲天下被災田租一年。

朝鮮使者於《朝天記》、《朝天日記》中記載神宗年輕時儀容莊嚴穩重,額頭廣闊、下巴飽滿,步伐矯健、神采威嚴,目光炯炯有神、舉手投足之間使人敬畏,而帝王氣度更是深不可測,是中外一至認為都有道明君。他在位的前十五年被評價「勤於朝政,勵精圖治,大有作為,足以稱道,儼然如一代賢君」。

万历帝的老师、第一任内阁首辅兼万历新政的策划与执行人张居正過世後第二年,万历帝斥逐馮保,下詔追奪張居正的封號和諡號,並查抄張家,平反劉臺冤案,起用因反對張居正而遭懲處的官員。万历十七年起(1588年),万历帝開始怠慢朝政(一說沉湎於酒色之中,一說是染上鴉片煙癮),万历十七年十二月大理寺左评事雒于仁写《酒色财气四箴疏》:“皇上之恙,病在酒色财气也。夫纵酒则溃胃,好色则耗精,贪财则乱神,尚气则损肝”。邹漪《启祯野乘》卷一《冯恭定传》中也说到明神宗荒于酒色:“因曲蘖而驩饮长夜,娱窈窕而晏眠终日。”《明史鈔略》記載萬曆二十一年皇太后萬壽時,神宗在暖閣召見王錫爵:……上曰:“朕知道了。”錫爵又奏:“今日見了皇上,不知再見何時?”上曰:“朕也要先生每常相見,不料朕體不時動火。”爵對:“動火原是小疾,望皇上清心寡欲,保養聖躬,以遂群臣願見之望。”而明神宗也開始奢侈靡费,斂財揮霍,又屢屢從國庫提銀,史稱「傳索帑金」,并任用張鯨等奸倖。後因立太子的國本之爭与内阁爭執長達十餘年,最後索性三十年不出宫门、不郊、不廟、不朝。1589年,神宗不再接見朝臣,內閣出现了“人滞于官”和“曹署多空”的现象。

万历二十五年,右副都御史謝杰批评神宗荒于政事,亲政后政不如初:「陛下孝亲、尊祖、好学、勤政、敬天、爱民、节用、听言、亲亲、贤贤,皆不克如初矣。」萬曆三十四年,禮科左給事中孫善繼也極陳時弊說:「惟願皇上修萬曆十五年以前之勵精,複萬曆十五年以前之政體,收萬曆十五年之人心,庶平明之治成,垂拱之理得。」以至於朱翊鈞在位中期以後,方入內閣的廷臣不知皇帝长相如何,于慎行、赵志皋、张位和沈一贯等四位国家重臣虽对政事忧心如焚,卻無計可施,僅能以数太阳影子长短来打发值班的时间。

萬曆四十年(1612年),南京各道御史上疏:「臺省空虛,諸務廢墮,上深居二十餘年,未嘗一接見大臣,天下將有陸沉之憂。」首輔葉向高卻說皇帝一日可接見福王兩次,但明神宗不承認,并表示他已經沒有傳召福王很久了,若真的每日接見,福王出入禁門,隨從這麼多,人所共見,必然耳目難掩。万历四十五年(1617年)十一月,「部、寺大僚十缺六、七,风宪重地空署数年,六科止存四人,十三道止存五人。」而明緬戰爭也因為明朝方面忽視而先勝後敗,被緬甸東吁王朝蠶食孟養在內國土。

囚犯們關在監獄裡,有長達二十年之久還沒有審問過一句話的,他們在獄中用磚頭砸自己,輾轉在血泊中呼冤。臨江知府錢若賡被神宗投入詔獄達三十七年之久,直至其子錢敬忠上疏:「臣父三十七年之中……氣血盡衰……膿血淋漓,四肢臃腫,瘡毒滿身,更患腳瘤,步立俱廢。耳既無聞,目既無見,手不能運,足不能行,喉中尚稍有氣,謂之未死,實與死一間耳。」萬曆帝才以「汝不負父,將來必不負朕。」將其釋放。首輔李廷機有病,連續上了一百二十次辭呈都得不到消息,最後不辭而去。萬曆四十年(1612年),吏部尚書孫丕揚上二十餘疏請辭不得,最後也拜疏自去。四十一年(1613年),吏部尚書趙煥也因數請去職還鄉不得,於是稱疾不出,逾月才終於請辭成功。吴亮嗣于万历末年的奏疏中说:「皇上每晚必饮,每饮必醉,每醉必怒。酒醉之后,左右近侍一言稍违,即毙杖下。」

樊樹志的《萬曆傳》考究裏,中允地解釋了明神宗怠政原因,源於健康狀況惡化非子虛烏有,追溯萬曆十四年九月十八日以後,皇帝因病免朝,言「頭昏眼黑,力乏不興」,對祭享太廟活動也只能權讓勛貴代理,并無奈地說道「非朕敢偷逸,恐弗成禮」,後來又遣內使對內閣傳諭「聖體連日動火,時作眩暈」,「聖體偶因動火,服涼藥過多,下注於足,搔破貼藥,朝講暫免。」與定陵发掘後查證神宗左足有疾互相引證。且當萬曆十五年三月初六,聖體初安以後,神宗旋即上朝聽政,隨后又與三輔臣見面,并打招呼說「朕偶有微疾,不得出朝,先生每憂心。」十六年二月初一又如常參與文華殿經筵,并興致勃勃地與閣臣討論《貞觀政要》,唐太宗與魏徵。萬曆十八年正月初一時,收到雒于仁奏疏的神宗召見首輔申時行入見,當申時行向他提出皇上有病需要静攝,也當一月之間至少數次視朝,神宗并沒有惱怒,只是解釋道「朕病愈,豈不欲出!即如祖宗廟祀大典,也要親行,聖母生身大恩,也要時常定省。只是腰痛腳軟,行立不便。」次年病情稍好後神宗與閣臣談起病情,也是真情流露地說起自己久病的心情「朕近年以來,因痰火之疾,不時舉發,朝政久缺,心神煩亂」。乃至神宗在位中期王家屏,王鍚爵輔政期間仍是「面目發腫,行步艱難」,以致連嫡母仁聖皇太后陳氏病逝,一向孝順聞名的神宗也因病動彈不得,只能遣人代理,而遭受到朝臣猛烈的評擊責難,有苦難言,此後神宗病情反復,在萬曆三十年病情之差甚至要一度立下遺旨,向沈一貫託孤。可見神宗在位期間的「動履不便」「身體虛弱」以致在位期間怠政,實不是推諉託辭。

萬曆中期後雖然不上朝,但是並沒有出現英宗以來的宦官之亂,也沒有外戚干政,也沒有嚴嵩這樣的奸臣,朝內黨爭也有所控制,萬曆對於日軍攻打朝鮮、女真入侵和梃擊案都有迅速的反應,如萬曆二十四年,乾清坤寧兩宮大火,神宗下罪己詔書,表示雖然忽略一般朝政庶務,但還是關心國家大事,而處理政事的主要方法多是在九重宫阙下通過諭旨的形式向下面傳遞,並透過一定的方式控制朝局。

此外礦稅之弊,即神宗在位期間的賦稅措施,一般被是認為萬曆中年後弊政的一部分,萬曆擺脫張居正的束縛之後,開始通過向各地徵收礦稅銀的方式,增加內庫的內帑,大多數學者認為這是一項弊政,也有許多的反對意見,認為礦稅也有相當的好處,如礦稅入內帑後大多用于国家救灾,餉軍救急等。

神宗在軍事上任用幹練將校,先後主持發兵平定了播州(遵义)杨应龙之亂的播州之役、平宁夏哱拜之乱的寧夏之役、抵抗日本丰臣秀吉發兵侵略朝鮮以及奴兒干都司的朝鮮之役,维护了明朝的內部统一及宗主國的權威。此三場戰爭合稱萬曆三大征。后世有说明軍雖均獲勝,但軍費消耗甚鉅,如僅朝鮮一役消耗國庫便高達银八百八十三万五千两,米数十万斛,對晚明的財政造成重大負擔。但实际上明代晚期仅对后金的战事,耗费就高达六千万两之巨,远超三大征,且三大征都是不得不打之戰,如朝鮮一國勢拱神京,地牽關海,薊、遼之外藩,東江之咽噎,一或失守,重險撤焉,如若不打甚至打败了,明朝都有亡国之危。而三大征实际军费则由内帑和太仓库银足额拨发,三大征结束后,内帑和太仓库仍有存银,而面對萨尔浒之战的大败,朱翊钧用熊廷弼守辽东,屯兵筑城,才稍稍将东北局势扭转。

萬曆皇帝指揮的萬曆朝鮮之役使朝鮮保全了國家,避免了亡國的巨大危險,儘管朝鮮人對萬曆皇帝有著深厚的感情,但是在朝鮮使臣的記錄中,更多的還是對萬曆帝消極怠政、貪婪奢侈等惡劣行徑的批評。而朝鮮使臣塑造的萬曆皇帝形象,也反映出明中葉之後朝鮮對中國社會集體想像的轉變,大明國的形像已經由朝鮮前期塑造的天朝上國,逐步褪去了耀目的光環,而走向了沒落。但在明清鼎革後,朝鮮對明朝的推崇思念又走向一個新的巔峰,朝鮮君王設大報壇,萬東廟祭祀明太祖,明神宗和明思宗。朝鮮孝宗甚至一度打算北伐清廷,朝鮮士子儒生暗中使用崇禎年號幾近三百年,鄙視清朝,并以宋時烈等為首推崇「尊周思明」「春秋大義」,稱自己是「皇明遺民」,那怕隱居山中,一生不出仕為大明守節者也大有人在,甚至到近代朝鮮高宗稱帝時,大明滅亡已超過二百餘年,其即位時諸臣勸進仍是「神宗皇帝再造土宇, 則義雖君臣, 恩實父子...嗚乎! 天命靡常, 皇社旣屋, 帝統墜地, 獨大報一壇, 乃皇春一胍之所寄...陛下聖德大業,宜承大明之統緒」,一切礼节皆取自《大明会典》。

神宗在位期间,西方传教士纷纷来华,其中以利玛窦为代表。利玛窦还在万历二十八年(1601年)觐见了神宗,向神宗进呈《万国图志》、自鸣钟、大西洋琴等西方方物,获得了神宗的信任。

利玛窦还与进士出身的翰林徐光启交情最好。除利玛窦来华外,来中国的传教士还有意大利的熊三拔、艾儒略,日耳曼人汤若望等人。

西方传教士来到中国,把西方数学、天文、地理等科学技术知识还有西方文化传到中国,在一定程度上促进了当时中国社会经济文化的发展,而中国士大夫阶层中的少数先进分子,同时起了一种唤醒的作用。

萬曆九年,神宗在向太后請安時,一時衝動,臨幸一名宮女,生下了長子朱常洛(後來的明光宗泰昌皇帝)。因為朱常洛是宮女所生,神宗不喜歡他,且有意立愛妃鄭氏所生的朱常洵為太子。萬曆十四年群臣上奏請神宗即立常洛為太子,萬曆以常洛尚年幼體虛未定,拖延不決。

萬曆二十一年,明神宗變本加厲,下手詔要將皇長子朱常洛、三子朱常洵和皇五子朱常浩一同封為藩王,以後再選擇其中適合人選為太子。朝臣聽聞一片譁然,紛紛上奏神宗。如雪片般飛來的痛批奏摺,使神宗倍感壓力,迫於眾議只好不得已收回前命。直到萬曆二十九年,朱常洛已年滿二十歲,立儲一事已不可拖延,神宗才立其為皇太子。

而長久以來的國本之爭引發出了兩次妖書案,這些案件即是朝廷大臣內鬨的縮影,可說是東林黨爭。

此时东北女真族興起,成为日後中原帝國的隐患。万历四十六年(1618年)四月十三日,女真酋長努爾哈赤自稱“覆育列國英明汗”,凑“七大恨”,以掀起叛乱,并僭称国号为後金。四十六年四月,女真兵克抚顺,殺死遼東總兵官張承胤,朝野震惊。為了應付女真,把努爾哈赤「务期歼灭,以奠封疆」,自萬曆四十六年九月起,朝廷先後三次下令除了畿內八府及貴州以外,加派全國田賦九厘,合共增賦五百二十萬,時稱遼餉,明末三饷之始,而神宗有鑑於地方官員在遼餉外可能會額外徵收火耗剝削百姓,特別下旨嚴禁。万历四十七年(1619年),遼東經略楊鎬領尚方劍,調兵遣將,并以李如柏、杜松、劉綎、馬林四將分兵進攻後金,結果在薩爾滸之戰大敗,死四萬餘人,開原和鐵嶺淪陷,首都燕京震動。

戰爭中,明神宗多有佈置方略,但一直吝惜內庫帑銀,不願撥內帑充餉,直至朝臣再三請求而後才勉強發了帑銀十萬,但其中多黑如漆或脆如土,致使師老餉匱。待四路殞將覆師後,神宗才又警愦振聋,發了近四十萬兩內帑銀解赴辽東,并任用熊廷弼守辽东,並給予其大力支持,屯兵筑城,振飭軍備,才稍稍将東北局勢扭轉。虽然明神宗多年未正式上朝,但大到朝鲜之役,小到顺天府祈雨,均由皇帝在内宫作出,并发各部门直接执行。

薩爾滸之戰後,遼東失陷,神宗鬱鬱寡歡,焦勞國事。隔年萬曆四十八年(1620年)四月,皇后王喜姐病逝,神宗心力交瘁,過了三個月,万历四十八年七月二十一日(1620年8月18日),明神宗駕崩於紫禁城弘德殿,享年五十七歲,在位四十八年。臨終前遺詔指出大臣應勉以用心辦事,以及廢礦稅,起用建言而得罪的官員等。

朝鮮一國為此舉哀。太子朱常洛立即发内帑(皇帝私房钱)百万犒赏边关将士。停止所有矿税,召回以言得罪的诸臣。不久,再发内帑百万犒边。八月即位,改元泰昌,是为明光宗,光宗即位後,內閣先是為萬曆帝擬謚上廟號顯宗恭皇帝,但後來朝臣認為諡號的「恭」是晉恭帝,隋恭帝兩位末代皇帝的諡號,先帝聖謨不可殫述,而帝堯運乃神之德,於是後改成為其上廟號神宗,諡號顯皇帝。九月,在位不足三十天的明光宗便在红丸案之中暴毙。因光宗即位不到一個月即告駕崩,孫子熹宗即位後於十月丙午(10月27日)葬神宗於定陵。

万历帝的定陵1958年被发掘,万历帝尸骨復原,“生前体形上部为驼背”、左腳略右腳短。文革時期的1966年8月24日,遗骨被紅衛兵付之一炬。因此,萬曆皇帝之所以三十年不上朝的原因,有一說是認為自己身形不正,感到自卑,所以不敢見人。

1955年10月4日,郭沫若、沈雁冰、吴晗、邓拓、范文澜、张苏等人联名提交《关于发掘明长陵的请示报告》给国务院秘书长习仲勋。报告转给主管文化工作的国务院副总理陈毅,并呈报国务院总理周恩来。文化部文物局局长郑振铎、中国科学院考古研究所副所长夏鼐得知后认为条件不成熟,强烈反对贸然发掘,高层形成一场争论。周恩来向毛泽东作了汇报,毛泽东点头后,周恩来批下“原则同意”四字。长陵发掘委员会委员夏鼐负责发掘的技术指导,便让其学生赵其昌(后任首都博物馆馆长)做前期调研。赵其昌带探工在长陵未找到发掘线索。在向夏鼐、吴晗等人汇报后,经商讨决定先试掘献陵,积累经验再发掘长陵。后来吴晗和夏鼐认为试掘献陵对长陵的发掘参考价值不大,吴晗提议试掘永陵,遭夏鼐强烈反对,认为这与发掘长陵无异;试掘思陵,吴晗认为太小,是妃子墓改建。此后吴晗和夏鼐才想到定陵。杨仕、岳南合著的《定陵地下玄宫洞开记》认为,吴晗和夏鼐想到定陵的原因有二,“第一,定陵是十三陵中营建年代较晚的一个,地面建筑保存得比较完整,将来修复起来也容易些。第二,万历是明朝统治时间最长的一个,做了48年皇帝,可能史料会多一些。” 定陵的開挖始末,《風雪定陵》一書有詳細的介紹。

1956年5月开始试掘,历时一年试掘成功,1957年打开玄宫。其玄宫由前室、中室、后室、左配室、右配室组成,石条起券,前室前面有隧道券,总面积1195平方米,出土文物3000多件。1959年9月30日,就定陵原址建为“定陵博物馆”,郭沫若题写馆名。1959年10月1日正式对外开放。由于技术水平落后,出土的大批文物无法保存,发掘出土的丝织品变硬腐化。郑振铎、夏鼐为此上书国务院,请求立即停止再批准发掘帝王陵墓的申请,国务院总理周恩来同意了他们的意见。不主动发掘帝王陵墓自此成为中国考古界的定规。

1966年文化大革命爆发后,定陵遭到嚴重破壞,保存在定陵文物仓库中的萬曆帝、后的屍骨被紅衛兵以「打倒地主階級的頭子萬曆」的口號被揪出。1966年8月24日,萬曆帝、后的三具尸骨以及一箱帝、后画像、资料照片等被抬到定陵博物馆重门前的广场上接受批斗并焚毁。

明朝官修的編年體史書《明神宗顯皇帝實錄》總評萬曆皇帝一生說:“蓋上仁孝聖神,逈絕千古,享國愈久,聖德彌隆,無挽近綜核之煩,而自臻治古幾康之理。海內沐浴玄化幾五十年,國祚靈長,永永無極,所培毓遠矣。先是因秉軸者懲操切之過,不無稍劑以寬大,而上明習政事,乾綱獨攬,予奪進退,莫可測識。晚頗厭言官章奏,概置不報,然每遇大事,未嘗不折衷群議,歸之聖裁。中外振聳,四封宴如,雖以憂勤之主極意治平而不得者,上獨以深居靜攝得之,周之成康,漢之文景,未足況也。至慈護先考,終始無間,尤非草野所得窺,而為堯為舜之旨,更諄諄以期。 ……廟號曰神,殆真如神雲。”

黃汝良:“仓箱红朽无忧岁,南北敉宁不用兵。北塞称臣四十年,封疆无数获生全。”

姚希孟(1579—1636):“缅怀祖德岂难跻,八柄河魁手自持。凤诏未闻传墨敕,貂珰只许贡朱提。兵符细柳将军令,国计元和宰相稽。蝉鬓秀才垂紫袖,批红不改旧标题。”

丁耀亢(1600—1669):“憶昔村民千百家,門前榆柳蔭桑麻。鳴雞犬吠滿深巷,男舂婦汲聲歡嘩。神宗在位多豐歲,鬥粟文錢物不貴。門少催科人晝眠,四十八載人如醉。”

钱谦益(1582—1664):“国家修明昌大之运,自世庙以迄神庙,比及百年,可谓极盛矣。”“万历中,正国家日中豫泰之候。”“当盛明日中,君臣大有为之日。”“呜呼,我神宗显皇帝,丕承谟烈,久道化成,制科取士,人物滋茂。”

王时敏(1592—1680):“神宗之世,海内乂安,生民不见兵火。”

谈迁(1593—1657):“今吏民嗷嗷,追念宽政,讴吟思慕,虽改代讵一日忘之哉?”

夏允彝(1596—1645):“神庙冲龄践祚,睿质夙成……士大夫以气节相矜,虽无姚、宋之辅,亦无愧开元之盛时也。”“神庙睿圣非常,虽御朝日希,而柄不旁落,止以鄙夷群臣之故,置庶务于不理。士大夫益纵横于下,而国事大坏。”

陈洪绶(1599—1652):“枫溪梅雨山楼醉,竹坞香茶佛屋眠。得福不知今日想,神宗皇帝太平年。”

吴伟业(1609—1671):“余尝惟国家当神宗皇帝时,天下平治。”“以余所闻,神宗皇帝时,士大夫以读书讲学相高。”“余生也晚,犹见神宗皇帝之世,江南土安俗阜,风习最为近古。”

顾炎武(1613—1682):“昔在神宗之世,一人无为,四海少事。”“老人尚記為兒時,煙火萬里連江畿。斗米三十穀如土,春花秋月同遊嬉。定陵(即神宗,神宗葬於定陵)龍馭歸蒼昊,國事人情亦草草事。”

彭孙贻(1615—1673):“眼见万历年,朝野穆清昊。”“风光漫思江南乐,父老还思万历年。”

方孝标(1617—?):“此时神庙正垂衣,四海烽清禾黍肥。”

吴嘉纪(1618—1684):“酒人一见皆垂泪,乃是先朝万历钱。”

林古度(1580年—1666年):“陸離彷彿五銖光,筆畫分明萬曆字。座客傳看盡黯然,還將一縷為君穿。且共開顏傾濁釀,不須滴淚憶當年。”

徐枋(1622—1694):“神宗朝正当国家全盛。”

杜濬(1611年-1687年):“萬曆年間,……九州富庶無旌麾,揚州之域尤稀奇。。”

李邺嗣(1622—1680):“神宗全盛日,海内一愁无。尚及闻遗老,今犹哭鼎湖。”

汪琬(1624—1691):“琬尝追溯神宗之世,国家方承平无事。”“神宗德泽犹在人心。”

曾灿(1625—1688):“神宗乙巳年,中原边辅无烽烟。圣人御极贤者出,粟米流脂贯朽钱。”

陈维嵩(1625—1682):“先朝神宗御宇五十余载,六服休畅,被润泽而大丰美。”

吕留良(1629—1683):「生逢神廟間,貌古性亦淳。海宇忘兵革,冠佩何彬彬。當時不知好,今憶真天神。三十後少年,語之笑且嗔。」

魏世效(1659—?):“万历之四十六年,天下熙暤。当斯时也,物安其性,民安其业,濡染涵育,莫不知立身爱君之道。而敦庞之风,谦下之节,亦惟此时人能有之。”

朝鮮貢使李睟光(1563年—1628年):“巍功赫業五帝六,冠帶車書四海一。商周禮樂漢文物,鼓舞堯天歌舜日。”“聖主天地千年德,嗚呼!聖主天地千年德。”

朝鮮大臣朴淳:: “皇上年方十岁, 圣资英睿, 自四岁已能读书, 以方在谅阴, 未安于逐日视事, 故礼部奏, 惟每旬内三六九日视朝。 仍诣文华馆, 御经筵, 四书及《近思录》、《性理大全》, 皆毕读。 自近日, 始讲《左传》, 百司奏帖, 亲自历览, 取笔批之, 大小臣工, 莫不称庆。”

朝鮮使臣對萬曆皇帝執政前期的勤政是極為稱道的:“因聞皇上講學之勤,三六九日,則無不視朝,其餘日則雖寒暑之極,不輟經筵。四書則方講孟子,綱目至於唐紀,日出坐殿,則講官立講。講迄,各陳時務。又書額字,書敬畏二字以賜閣老,又以責難陳善四字,賜經筵官,以正己率屬四字,賜六部尚書,虛心好問,而 聖學日進於高明。下懷盡達,而庶政無不修,至午乃罷,仍賜宴於講臣,寵禮優渥雲。嗚呼!聖年才至十二,而君德已著如此。若於後日長進不已,則四海萬姓之得受其福者。”

《宣祖實錄》:“今皇帝沖年卽位, 資質英明, 時無過誤, 朝野無事, 人情似有喜悅之意。”

成书于清初的小说《樵史通俗演义》开篇说:“传至万历,不要说别的好处,只说柴米油盐鸡鹅鱼肉诸般食用之类,哪一件不贱?假如数口之家,每日大鱼大肉,所费不过二三钱,这是极算丰富的了。还有那小户人家,肩挑步担的,每日赚得二三十文,就可过得一日了。到晚还要吃些酒,醉醺醺说笑话,唱吴歌,听说书,冬天烘火夏乘凉,百般玩耍。那时节大家小户好不快活,南北两京十三省皆然。皇帝不常常坐朝,大小官员都上本激聒,也不震怒。人都说神宗皇帝,真是个尧舜了。一时贤想如张居正,去位后有申时行、王锡爵,一班儿肯做事又不生事,有权柄又不弄权柄的,坐镇太平。至今父老说到那时节,好不感叹思慕。”

《乱离见闻录》作者陈舜回憶说:“予生萬曆四十六年戊午八月廿六日卯時,父母俱廿三歲,時丁昇平,四方樂利,又家海角,魚米之鄉。鬥米錢未二十,斤魚錢一二,檳榔十顆錢二文,著十束錢一文,斤肉,只鴨錢六七文,鬥鹽錢三文,百般平易。窮者幸托安生,差徭省,賦役輕,石米歲輸千錢。每年兩熟,耕者鼓腹,士好詞章,工賈九流熙熙自適,何樂如之。”

成书于天启四年的小说《警世通言》,第三十二章說:“自永樂爺九傳至於萬曆爺,此乃我朝第十一代的天子了。這位天子,聰明神武,德福兼全,十歲登基,在位四十八年,削平了三處寇亂。那三處?日本關白平秀吉,西夏承恩,播州楊應龍。平秀吉侵犯朝鮮,承恩、楊應龍是土官謀叛,先後削平。遠夷莫不畏服,爭來朝貢。真個是:一人有慶民安樂,四海無虞國太平。”

成书于萬曆四十七年的《萬曆野獲編》,編輯小引說:“今上御極已垂五十年。德符幸生堯舜之世,雖果處菰蘆,然詠歌太平,無非聖朝佳話。間有稍關時事者,其涇渭自明,藿食者但能粗憶梗概而已。”

清世祖(1643-1661):“當明之初,取民有制,休養生息。萬曆年間,海內殷富,家給人足。天啟,崇禎之世,因兵增餉,加派繁興,貪吏綠以為奸,民不堪命,國祚隨之,良足深鑒。”

崔瑞德《剑桥中国明代史》:万历皇帝聪明而敏锐;他自称早慧似乎是有根据的。他博览群书;甚至在他最后的日子里,在他已深居宫廷几十年,并已完全和他的官吏们疏远了时,按照他时代的标准,他仍然博闻广识。

《明史·神宗本紀》:“贊曰:神宗沖齡踐阼,江陵秉政,綜核名實,國勢幾於富強。繼乃因循牽制,晏處深宮,綱紀廢弛,君臣否隔。於是小人好權趨利者馳騖追逐,與名節之士為仇讎,門戶紛然角立。馴至悊、愍,邪黨滋蔓。在廷正類無深識遠慮以折其機牙,而不勝忿激,交相攻訐。以致人主蓄疑,賢奸雜用,潰敗決裂,不可振救。故論者謂明之亡,實亡於神宗,豈不諒歟。”“神皇乘運,豫大豐亨,征徭既繁,百工叢脞,揆厥亂源,所自來爾。”

趙翼《廿二史劄記·萬曆中礦稅之害》:“論者謂明之亡,不亡於崇禎而亡於萬曆。”

谷應泰《明史紀事本末·第六十五卷礦稅之弊》:“神宗奕葉昇平,邊圉封貢,海內乂安,家給人足...逮至萬曆二十四年,張位主謀,仲春建策,而礦稅始起...當斯時也,瓦解土崩,民流政散,其不亡者幸耳”

清高宗在《明長陵神功聖德碑》則道:“明之亡非亡於流寇,而亡於神宗之荒唐,及天啟時閹宦之專橫,大臣志在祿位金錢,百官專務鑽營阿諛。及思宗即位,逆閹雖誅,而天下之勢,已如河決不可復塞,魚爛不可復收矣。而又苛察太甚,人懷自免之心。小民疾苦而無告,故相聚為盜,闖賊乘之,而明社遂屋。嗚呼!有天下者,可不知所戒懼哉?”

宋浚吉: “不怨暗君, 天啓皇帝不可怨之君, 而萬曆皇帝以初年英豪之主, 臨御四十年, 未嘗引接臣僚, 此可爲戒者也。”

黃仁宇在《萬曆十五年》一書將萬曆皇帝的荒怠,聯繫到萬曆皇帝與文官群體在“立儲之爭”觀念上的對抗。怠政則是萬曆皇帝對文官集團的報復。黃仁宇說:「他(即萬曆皇帝)身上的巨大變化發生在什麼時候,沒有人可以做出確切的答復。但是追溯皇位繼承問題的發生,以及一連串使皇帝感到大為不快的問題的出現,那麼1587年丁亥,即萬曆十五年,可以作為一條界線。這一年表面上並無重大的動蕩,但是對本朝的歷史卻有它特別重要之處。」在《萬曆十五年》文末總結,「1587年,是為萬曆15年,歲次丁亥,表面上似乎是四海昇平,無事可記,實際上我們的大明帝國卻已經走到了它發展的盡頭。在這個時候,皇帝的勵精圖治或者晏安耽樂,首輔的獨裁或者調和,高級將領的富於創造或者習於苟安,文官的廉潔奉公或者貪污舞弊,思想家的極端進步或者絕對保守,最後的結果,都是無分善惡,統統不能在事實上取得有意義的發展。因此我們的故事只好在這裡作悲劇性的結束。萬曆丁亥年的年鑑,是為歷史上一部失敗的總記錄」。

在黄仁宇等的著作中也表达出中国明代中后期,皇帝只是一个牌位,而事实上万历的个人行为对基层的国家的习惯轨迹并无大的影响。

萬曆元年十月八日,是日講的日子,朱翊鈞在文華殿聽張居正進講《帝鑑圖說》。當張居正講到宋仁宗不喜珠飾,值得效法時,朱翊鈞立即表示同感:“賢臣才是寶,珠玉又有何益!”張居正接著說:“聖明的君主貴五穀而賤珠玉,五穀可以養人,而金玉飢不可食,寒不可衣,《書經》稱不作無益害有益,不貴異物賤用物,道理也就在這裡。”“是啊!宮裡的人喜歡裝飾,我在年賜時每每節省,宮人們都有意見,我說國庫的積蓄又有多少呢?”朱翊鈞又回答說。張居正便誇獎道“皇上能這樣說,真是社稷生靈的福氣啊!”當時朱翊鈞才不過十歲。

萬曆二年,朝鮮使臣許篈,趙憲前來朝貢。許篈在其前往中國記錄見聞的《朝天記》對年幼的萬曆天子的形象進行了描寫,記載其「聲甚清朗」「天威甚邇,龍顏壯大,語聲鏗然,(我)不勝歡欣之極」同行的另一位使臣趙憲則更生動地記錄地在《朝天日記》道「上(萬曆皇帝)年僅十二歲,而注視別人時十分老成,端坐在龍椅上也不曾搖動,並不會叫太監內臣傳達他的旨意,反而是親自對臣工下聖諭,而聲音玉質淵秀,金聲清暢。(我)一聽到年幼天子的聲音,就感動起來,對以後天下太平萬歲的希望,也更加愈切了。」,而趙憲甚至把年幼的萬曆天子與其父明穆宗作比較,卻指出其父上朝時精神不集中、時常東張西望,而且聲音微弱,需要宦官再去大聲宣旨,儀態形像不佳。

自從張居正去世以後,萬曆終於能擺脫出翰林學士的羈絆;而自從他成為父親以來,李太后也不再乾預他的生活。但是,皇帝自幼聰惠,在這個時候確實已經成年了,他已經不再有興趣和小宦官去打鬧,而是變成了一個喜歡讀書的人。他命令大學士把本朝諸祖宗的“實錄”抄出副本供他閱讀,又命令宦官在北京城內收買新出版的各種書籍,包括詩歌、論議、醫藥、劇本、小說等各個方面。

萬曆十四年三月,一次君臣召對中,因京師陰霾蔽空,皇帝決定減免一些稅賦,並認為或許最近開水田太過擾民,而致上天警示,應當停止,閣臣申時行委婉地說道:「京東地方,田地荒蕪,廢棄可惜,相應開墾。」皇帝復說道:「南方地下,北方地高。南地濕潤,北地鹼燥。且如去歲天旱,井泉都乾竭了。這水田怎能做得?」於是申時行頓時認為聖裁允當,拜首執行。

明朝遺民李長祥在“天問閣集”的“劉宮人傳”中也對萬曆皇帝有過高度評價,甚至認為萬曆皇帝比起東漢光武帝,唐太宗來,品德更在其上。

明末流離出宮的一個老宮女劉氏曾在萬曆年間任職。他與李長祥講述當年的事情「一天內官(太監)持朱筆寫的傳票給萬曆皇帝看,皇帝看完不說話,太監說:「連皇帝內侍的左右內官都容不下,還敢來捉拿。」皇帝沉默了一回,便回答說:「用朱票捉拿人是巡城御史的職責,怎麼能奪他權柄,阻礙他執法,況且你們一定是幹了些什麼壞事。這事朕不管,人就隨他捉拿吧。」這時候皇帝還不知道當時發生了什麼事。

後來李長祥覽神宗遺事,原來是當年有一人告內官於御史,御史不知道他已經進宮了,即出朱票拿人。手持朱票去捉人的也不是有經驗的人,直接走到午門去索問。一眾內官馬上就大怒並把票奪走,走到皇帝面前奏上此事,皇帝說的話就跟老宮女劉氏一模一樣,居然兩事能互相對證。

李長祥也不禁大加讚許:「嗚呼聖人哉,聖人哉......考當日所為,亦飾語耳,若神宗乃真有其實,雖唐虞三代之令主,何以加此。其能使海內家給人足,道不拾遺,夜不閉戶者四十八年,有以哉!」

明神宗屍骨被發掘後,發現其駝背後左右腳短,但學者認為神宗生前並不適用。一說神宗生前從未走出過紫禁城,也不符史實,《明神宗實錄》均載,祭先皇陵、祭天、祈雨、祭孔、祭先農等重大儀式均由皇帝主持,且亦有參與騎馬、步行,均不見有載其殘頹之說,屍體上發現的殘缺應該是年老時造成的,而非先天疾病,且三十年不上朝的神宗,其實都有在內廷批奏摺、發令等,並非完全不事朝政。

英国女王伊丽莎白一世在万历二十四年(1596年)给当时中国在位的神宗皇帝写了一封亲笔信,希望英中两国开展贸易往来以及在其他领域交流的愿望。同时还派使者约翰·纽伯莱出使明朝,将这封亲笔信递交给神宗。然而使者在途中遇难,但是这封亲笔信却没有丢失,伊丽莎白一世无奈与此,称为她的终身遗憾。现在这封亲笔信被英国国家博物馆收藏。



\subsection{万历}

\begin{longtable}{|>{\centering\scriptsize}m{2em}|>{\centering\scriptsize}m{1.3em}|>{\centering}m{8.8em}|}
  % \caption{秦王政}\
  \toprule
  \SimHei \normalsize 年数 & \SimHei \scriptsize 公元 & \SimHei 大事件 \tabularnewline
  % \midrule
  \endfirsthead
  \toprule
  \SimHei \normalsize 年数 & \SimHei \scriptsize 公元 & \SimHei 大事件 \tabularnewline
  \midrule
  \endhead
  \midrule
  元年 & 1573 & \tabularnewline\hline
  二年 & 1574 & \tabularnewline\hline
  三年 & 1575 & \tabularnewline\hline
  四年 & 1576 & \tabularnewline\hline
  五年 & 1577 & \tabularnewline\hline
  六年 & 1578 & \tabularnewline\hline
  七年 & 1579 & \tabularnewline\hline
  八年 & 1580 & \tabularnewline\hline
  九年 & 1581 & \tabularnewline\hline
  十年 & 1582 & \tabularnewline\hline
  十一年 & 1583 & \tabularnewline\hline
  十二年 & 1584 & \tabularnewline\hline
  十三年 & 1585 & \tabularnewline\hline
  十四年 & 1586 & \tabularnewline\hline
  十五年 & 1587 & \tabularnewline\hline
  十六年 & 1588 & \tabularnewline\hline
  十七年 & 1589 & \tabularnewline\hline
  十八年 & 1590 & \tabularnewline\hline
  十九年 & 1591 & \tabularnewline\hline
  二十年 & 1592 & \tabularnewline\hline
  二一年 & 1593 & \tabularnewline\hline
  二二年 & 1594 & \tabularnewline\hline
  二三年 & 1595 & \tabularnewline\hline
  二四年 & 1596 & \tabularnewline\hline
  二五年 & 1597 & \tabularnewline\hline
  二六年 & 1598 & \tabularnewline\hline
  二七年 & 1599 & \tabularnewline\hline
  二八年 & 1600 & \tabularnewline\hline
  二九年 & 1601 & \tabularnewline\hline
  三十年 & 1602 & \tabularnewline\hline
  三一年 & 1603 & \tabularnewline\hline
  三二年 & 1604 & \tabularnewline\hline
  三三年 & 1605 & \tabularnewline\hline
  三四年 & 1606 & \tabularnewline\hline
  三五年 & 1607 & \tabularnewline\hline
  三六年 & 1608 & \tabularnewline\hline
  三七年 & 1609 & \tabularnewline\hline
  三八年 & 1610 & \tabularnewline\hline
  三九年 & 1611 & \tabularnewline\hline
  四十年 & 1612 & \tabularnewline\hline
  四一年 & 1613 & \tabularnewline\hline
  四二年 & 1614 & \tabularnewline\hline
  四三年 & 1615 & \tabularnewline\hline
  四四年 & 1616 & \tabularnewline\hline
  四五年 & 1617 & \tabularnewline\hline
  四六年 & 1618 & \tabularnewline\hline
  四七年 & 1619 & \tabularnewline\hline
  四八年 & 1620 & \tabularnewline
  \bottomrule
\end{longtable}


%%% Local Variables:
%%% mode: latex
%%% TeX-engine: xetex
%%% TeX-master: "../Main"
%%% End:

%% -*- coding: utf-8 -*-
%% Time-stamp: <Chen Wang: 2019-12-26 15:07:43>

\section{光宗\tiny(1620)}

\subsection{生平}

明光宗朱常洛(1582年8月28日-1620年9月26日),或稱泰昌帝,明朝第15代皇帝,年号泰昌,庙号「光宗」,谥号“崇天契道英睿恭纯宪文景武渊仁懿孝贞皇帝”。

明神宗长子,万历十年(1582年)八月生,母恭妃王氏原是祖母李太后身边的宫人。不久,明神宗郑贵妃生三子朱常洵,深得宠爱。长子朱常洛一直受到冷遇,群臣纷纷上书要求立储,是為國本之爭,明神宗要不是贬斥群臣,就是虚与委蛇地敷衍應付。祖母李太后以为不妥。一日,李太后询问神宗未立朱常洛为太子的缘故。神宗说:他是宫人所生。李太后大怒:你也是宫人所生(李太后亦是宫人出身)。神宗听后惶恐,伏地不敢起。

万历二十九年(1601年)十月,明神宗被迫册立长子朱常洛为太子,同时,立三子朱常洵为福王、五子朱常浩为瑞王、六子朱常润为惠王、七子朱常瀛为桂王。太子朱常洛以仁厚著称,朝野皆认为其将来可为明君。但常洛的地位不穩固,郑贵妃時時刻刻想要為朱常洵爭奪儲君之位,引發了兩次妖書案,牽連眾多大臣。而後,甚至有郑贵妃手下的兩名宦官指使刺客,欲以木梃刺殺朱常洛,是為梃击案,神宗為了不牽連郑贵妃,將該刺客、宦官等三人全部殺死。

朱常洛被立为太子后,就移居慈庆宫,从此与其母王恭妃被隔绝不得相见。万历三十四年(1606年),朱常洛的妾侍王氏生下皇长孙朱由校(日后的明熹宗),神宗为表庆祝,为李太后加尊号,又进封王恭妃为皇贵妃,赐金册金宝,但仍将其屏居景阳宫。万历三十九年九月十三日(1611年10月18日),王恭妃病笃,朱常洛闻言急往景阳宫探视,见景阳宫门深锁,于是破坏门锁入内探视。当时王恭妃已双眼失明,于是以手代眼,拉着朱常洛的衣角:“儿长大如此,我死何恨!”言毕王恭妃便与世长辞。《酌中志》则记载为王恭妃病重时太子每日从苍震门入内问安;《先拨志始》更记载王恭妃察觉到郑贵妃家人偷听,提醒太子,结果母子俩直到王恭妃去世也没有说话。大学士叶向高说:“皇太子母妃薨,礼宜从厚。”神宗不应,复请,才得到允准。

万历四十八年(1620年)七月二十一日,明神宗驾崩。太子朱常洛立即发内帑(皇帝私房钱)百万犒赏边关将士。停止所有矿税,召回以言得罪的诸臣。不久,再发内帑百万犒边。八月即位,改元泰昌,是为明光宗。福王生母鄭貴妃為了攏絡明光宗,獻上四位美女。明光宗縱慾過度不久病倒,太監崔文升進以瀉藥而狂瀉。在位不足三十天的明光宗在九月初一因服用李可灼的紅丸而猝死駕崩,史稱紅丸案。

在短短的一个月,明光宗在群臣的帮助下,也做了不少实事,比如:废矿税、饷边防、补官缺。

首先下令罢免全国范围内的矿监、税使,停止任何形式的的采榷活动。矿税早为人们所深感厌恶,所以诏书一颁布,朝野欢腾。

其次是饷边防。明光宗下令由大内银库调拨二百万两银子,发给辽东经略熊廷弼和九边巡抚按官,让他们犒赏将士;并拨给运费五千两白银,沿途支用。明光宗还专门强调,银子解到后,立刻派人下发,不得擅自入库挪为它用。

第三件事是补充官缺。朱常洛先命令礼部右侍郎、南京吏部侍郎二人为礼部尚书兼内阁大学士;随后,将何宗彦等四人均升为礼部尚书兼内阁大学士;启用卸官归田的旧辅臣叶向高,同意将因为“上疏”爭國本获罪的三十三人和为矿税等获罪的十一人一概录用。因此有人感慨明光宗矫枉过正,造成了前所未有的“官满为患”的局面。

因光宗即位一個月即告駕崩,该陵墓原为景泰帝所建,因景泰帝為英宗所贬,葬于西郊金山,所以空出一处皇陵。由于明光宗在位时间仅29天,来不及修建陵墓,故继位的長子明熹宗朱由校将光宗安葬于此陵墓。

《明實錄》:“自古帝皇仁心仁聞洽于天下,未有不須久道而後成者,必世後仁聖人言之矣。乃光宗貞皇帝在位僅三旬,升遐之日,深山窮谷莫不奔走悲號,何?聖化之神感孚若是速也。蓋帝睿質夙成,蚤親師傳,養德青宮已洞悉四海之難艱。故當神皇晏駕時,遺詔未頒,德音據播 ;大寶初嗣,仁政沛施。捐朽蠹而九塞飽騰,撤狐蟊而廛勸動政。地廣股肱之助,諫垣充耳目之司。黃髮並升于公庭,白駒不滯于空谷。至于虛懷延接一月,而三召臣工銳意圖。幾浹旬而兩蠲而稅額 。德意獨行,獨斷爕理,莫施其功,威權自攬。自綜執月,御不參其柄。鑠乎盛矣,曠千古而僅見者也,乃其尤難者以何思何慮之天,處若危若疑之地。冲齡出講,已歷艱辛,而容色溫然,動止泰然。內庭有菀枯之形,若勿知也者;外庭有羽翼之激,若勿聞也者。即冊立,尋常事耳。時而舉碁,時而反汗。大臣去,小臣譴,宜何如動于耳目者。 而帝也,有夔夔無慄慄。潛之又潛,巧伺者不能窺,善孽者不能中。福藩就國,慟哭抱持。張差發難,帝侍神皇。左右親傳睿旨,曉諭百官羣囂遂息,所全實多。登極後即遵遺命進封皇貴妃,廷臣力爭,竟不忍奪以戚畹,哀請而後止,毫不芥蔕于前事也。此即虞舜大孝何以加茲?以舜之孝,擴堯之仁,然則帝之所以感動人心又自有在,而非僅僅更張注措之迹者矣。夫官天下者,壽在令名;家天下者,壽在長世。神皇即不豫,何難四十日留也。使帝之出震未及而幹蠱,莫施天下之事將不可知。然則我國家億萬年無疆之祚,皆帝四十日之所延也。帝之功德又豈但在普天之思慕已哉,天眷宗社不虗也。”

\subsection{泰昌}

\begin{longtable}{|>{\centering\scriptsize}m{2em}|>{\centering\scriptsize}m{1.3em}|>{\centering}m{8.8em}|}
  % \caption{秦王政}\
  \toprule
  \SimHei \normalsize 年数 & \SimHei \scriptsize 公元 & \SimHei 大事件 \tabularnewline
  % \midrule
  \endfirsthead
  \toprule
  \SimHei \normalsize 年数 & \SimHei \scriptsize 公元 & \SimHei 大事件 \tabularnewline
  \midrule
  \endhead
  \midrule
  元年 & 1620 & \tabularnewline
  \bottomrule
\end{longtable}


%%% Local Variables:
%%% mode: latex
%%% TeX-engine: xetex
%%% TeX-master: "../Main"
%%% End:

%% -*- coding: utf-8 -*-
%% Time-stamp: <Chen Wang: 2021-11-01 17:13:53>

\section{熹宗朱由校\tiny(1620-1627)}

\subsection{生平}

明熹宗朱由校(1605年12月23日-1627年9月30日;校,居效切,拼音「jiào」、注音「ㄐㄧㄠˋ」),或稱天啟帝,光宗長子,明朝第16代皇帝。在位時間為1620年-1627年,年號天啟。光宗即位僅一個月而亡,使朱由校匆匆登位為帝,朱由校當時僅十四歲,未曾被立为太子,甚至未接受正規教育,政事皆賴宦官輔佐,後來造就太監魏忠賢等人的干政,與閹黨、東林黨之黨爭。

泰昌元年(1620年),其父明光宗在位不足三十天便在紅丸案之中暴斃。九月初六,由長子朱由校繼任。值得一提的是,其父明光宗朱常洛一向不為祖父明神宗所喜,故朱由校亦沒有被神宗重視。神宗駕崩後,大臣代言的遺囑:「皇長孫宜即時冊立、進學。」故顯示當時已十四歲的朱由校從未進學。明光宗即位後原擇九月初九冊立朱由校為東宮,惟來不及冊封,光宗於九月初一駕崩,故明熹宗連一天正式教育都未接受,便登上大寶,此為有明一代第一人,其情況比其父光宗勉強隨其他皇子出閣讀書,而非正統的太子教育方式,還要更加惡劣,且父子倆在繼位前都未監國輔政經驗,制造內宦干政的土壤,神宗亦無留下良好輔臣,國運衰退的因素在萬曆時國本之爭時即已種下。

泰昌元年(1620年),是明朝立國以來所遇到前所未有的情況。明熹宗的祖母孝端顯皇后、祖父明神宗與父親明光宗相繼在同一年駕崩,明神宗駕崩距孝端顯皇后駕崩才兩個多月,而明光宗駕崩時距明神宗駕崩不到一個月,實屬罕見。而明神宗與孝端顯皇后的大葬尚未完成,因此明廷在討論後,決定先為明神宗與孝端顯皇后辦理大葬,結束後再為明光宗辦理大葬。

熹宗繼位後,撫養皇帝的李選侍利用皇帝年少無知,佔據乾清宮,意圖把持朝政,東林黨左光斗、楊漣等反對,不讓李選侍與皇帝同住,迫使她移居他處,是為移宮案,此事後內侍魏忠賢被提拔為司禮監秉筆太監,魏忠賢與熹宗是皇孫時代即結識的舊識,魏忠賢乘機結交朱由校乳母客氏,兩人遂狼狽為奸。熹宗有感東林黨黨人從龍之功,大加提拔任用,又召回葉向高等先朝老臣擔任內閣首輔,時稱「眾正益朝」「群賢滿朝」,天下欣欣望治。另外熹宗也屢發內帑犒勞將士,補發九邊欠餉,如即位之初便發派一百八十萬帑金以勞邊,派帑金五十萬以給光宗陵工,準發帑五十萬作解發以發兵餉,又答允兵部再發帑金一百萬以佐急需,接著不足一年又因首輔葉向高所請而發帑金二百萬為東西兵餉之用。

朱由校喜歡木工,亦沉迷於刀鋸斧鑿,魏忠賢總是趁他木工做得全神貫注時,拿重要的奏章去請他批閱,熹宗隨口說:「朕已悉矣!汝輩好為之。」魏忠賢遂逐漸專權,竊奪威福,魏忠賢閹黨誣陷忠良,殺死包括東林六君子、東林七賢等正直的士大夫,致使朝政敗壞。

同時期,女真首領努爾哈赤則起於白山黑水之間,趁機攻佔瀋陽,奪取遼東地區,聲勢日隆。

天啟六年(1626年)北京發生「王恭廠大爆炸」,死傷2萬餘人,原因不明,朝野震驚,中外駭然,熹宗下了一道罪己詔,表示要痛加省醒,告誡大小臣工「務要竭慮洗心辦事,痛加反省」,并下旨發府庫萬兩黃金賑災。

天啟七年(1627年)八月,熹宗又與宦官魏忠賢、王體乾等去西苑深水處泛舟,卻因風強,小舟翻覆,皇帝落水,雖然隨即被救,但從此驚豫不堪,逐漸病重,尚書霍維華獻「靈露飲」,以五穀蒸餾而成,清甜可口,但幾個月後病情加劇,渾身浮腫,八月十一日,召見信王朱由檢,即行駕崩,時年23歲,廟號熹宗。熹宗諸子皆早夭,遺詔立五弟朱由檢為皇帝,即後來的明思宗。禮部定謚號曰「哲皇帝」,思宗宸墨改為「悊」。

《明實錄》:「上念光皇大業未究,雅志繼述,踐祚之初委任老成,摉羅遺逸,振鷺充庭,稱盛理焉。時四方多故,上宵旴靡遑,遼左及滇黔相繼請帑,無不立應,大臣行邊恩禮優渥,將士陷陣恤典立頒,又慮加派苦累,每有詔諭諄諄戒守令,加意撫字毋重困吾民。其軫念民碞如此,故能收拾人心,挽回天步,雖有煬灶假叢之奸而得人付托,社稷永固於苞桑。廟號曰熹,蓋稱有功安人云。」

《明史》:「自世宗而後,綱紀日以陵夷,神宗末年,廢壞極矣。雖有剛明英武之君,已難復振。而重以帝之庸懦,婦寺竊柄,濫賞淫刑,忠良慘禍,億兆離心,雖欲不亡,何可得哉。」

明朝劉若愚《酌中志》对熹宗评价较高,“先帝(明熹宗)生性虽不好静坐读书,然能留心大体,每一言一字,迥出臣子意表”;熹宗在宁锦大战中“日夜焦思,未遑自安”,王永光的题疏中曾有“要将宁远城中红夷大炮撤归山海关”,明熹宗批示:“此炮如撤,人心必摇”,表明他是有一定的政治决断力。当后金军队再犯锦州、宁远之时,“更愤激深虑”,对魏忠贤和乳母客氏也怒骂咒恨,形于颜色。同时,熹宗又「又極好作水戲,用大木桶大銅缸之類,鑿孔創機,啟閉灌輸,或湧瀉如噴珠,或澌流如瀑布,或使伏機於下,借水力沖擁園木球如核桃大者,於水湧之,大小盤旋宛轉 ,隨高隨下,久而不墮,視為嬉笑,皆出人意表。」。他曾親自在庭院中造了一座小宮殿,形式仿乾清宮,高不過三四尺,卻曲折微妙,巧奪天工。可见刘若愚对明熹宗的评价颇高。

明末清初談遷認為「閹尹之禍,劇於熹廟,并边徽而二之。……疵德多矣」。將閹黨及滿清視為天啟年間兩大威脅,可見其嚴重性。

清道光年間抱陽生《甲申朝事小紀》,認為朱由校沈迷於木工,放任魏忠賢矯詔、管理朝政的行為視為貪玩而不長進,史載「又好油漆,凡手用器具,皆自為之。性又急躁,有所為,朝起夕即期成。成而喜,不久而棄;棄而又成,不厭倦也。且不愛成器,不惜改毀,唯快一時之意。」「朝夕營造」,「每營造得意,即膳飲可忘,寒暑罔覺」。

民國直系將領吳佩孚認為明熹宗寵信閹黨,濫殺東林六君子、東林七賢,才是明朝亡國的主因,更甚於萬曆。其恩師王紹勛,與吳佩孚提及明神宗怠政三秩時,感歎曰:「無為而治兮不必生一神宗三秩」,吳佩孚居然立刻應聲對仗:「有明之亡矣莫非殺六君子七賢。」

《從萬曆到永曆》一書認為,魏忠賢不可能屢屢矯詔,故而天啟一朝的政治,包括鎮壓東林的決策,還是與熹宗相關,熹宗遭到了後來主編明史的東林和復社人士抹黑。此外,即便是明史也明確記載了熹宗對於朝政的參與,不可謂無自相矛盾之處。例如王士禛所謂老宮監刘若愚的原話是:(先帝)且不爱成器,不惜天物,任暴殄改毁,惟快圣意片时之适。当其斤斫刀削,解服磐礴,非素昵近者不得窥视,或有紧切本章,体乾等奏文书,一边经管鄙事,一边倾耳注听。奏请毕,玉音即曰:「尔们用心行去,我知道了」。這和所謂勤政的清朝皇帝批示:「知道了。」是差不多的作為。

此外,朱由校所謂沈迷於木工,很有可能是因為對於宮殿藝術有所追求,由於前兩次主要工程人員如蒯祥皆過世,為了三大殿能復原,朱由校特別注重木作部分等,事出有因,並非只因為個人興趣而不理會朝政。當時因南京三大殿早已燒失,北京紫禁城三大殿於萬曆年間亦燒燬,朱由校效法太祖親自監督三大殿重建計畫,聽從御史王大年节俭的建议,才會鑽研木匠手藝。

\subsection{天启}

\begin{longtable}{|>{\centering\scriptsize}m{2em}|>{\centering\scriptsize}m{1.3em}|>{\centering}m{8.8em}|}
  % \caption{秦王政}\
  \toprule
  \SimHei \normalsize 年数 & \SimHei \scriptsize 公元 & \SimHei 大事件 \tabularnewline
  % \midrule
  \endfirsthead
  \toprule
  \SimHei \normalsize 年数 & \SimHei \scriptsize 公元 & \SimHei 大事件 \tabularnewline
  \midrule
  \endhead
  \midrule
  元年 & 1621 & \tabularnewline\hline
  二年 & 1622 & \tabularnewline\hline
  三年 & 1623 & \tabularnewline\hline
  四年 & 1624 & \tabularnewline\hline
  五年 & 1625 & \tabularnewline\hline
  六年 & 1626 & \tabularnewline\hline
  七年 & 1627 & \tabularnewline
  \bottomrule
\end{longtable}


%%% Local Variables:
%%% mode: latex
%%% TeX-engine: xetex
%%% TeX-master: "../Main"
%%% End:

%% -*- coding: utf-8 -*-
%% Time-stamp: <Chen Wang: 2019-12-26 15:08:00>

\section{思宗\tiny(1627-1644)}

\subsection{生平}

明思宗朱由檢(1611年2月6日-1644年4月25日),或稱崇禎帝,明朝第17代、末代皇帝。

思宗为明光宗第五子,明熹宗异母弟。五歲時,其母劉氏獲罪,被時為太子的光宗下令杖殺,朱由检交由庶母西李撫養,數年後改由另一庶母东李撫養至成人。於天启二年(1622年)年被兄長明熹宗册封為信王。明熹宗於天啟七年(公元1627年8月)駕崩,由于没有子嗣,朱由检受遗命于同月丁巳日登基,时年十八歲。次年改元崇禎,是为明思宗。

思宗一生操勞,日以繼夜的批閱奏章,节俭自律,不近女色。崇祯年間,与萬曆、天啟相较,朝政有了明显改观。即位之初就大力铲除阉党,曾六度下诏罪己,惜其生性多疑,无法挽救衰微的明朝。明朝末年农民起义不断,关外后金政权虎视眈眈,已处于内忧外患的境地。崇祯十七年(1644年)發生甲申之變,李自成攻破北京,思宗在煤山一树上吊身亡,终年三十三岁,在位十五年。

南明予其庙号「思宗」,后改「毅宗」、「威宗」,南明弘光帝上谥号「绍天绎道刚明恪俭揆文奋武敦仁懋孝烈皇帝」。清朝追谥「钦天守道敏毅敦俭弘文襄武体仁致孝端皇帝」,庙号「怀宗」;后去庙号,改谥为「庄烈愍皇帝」,葬于十三陵思陵。

生於萬曆庚戌十二月二十四日 ( 1611年2月6日 ) 寅時。崇祯帝之父為明光宗朱常洛,朱常洛雖早在萬曆廿九年 ( 1601年 ) 被立為太子,但其父親明神宗其實一心想立三子朱常洵為太子,是因為群臣國本之爭,才勉強保住了朱常洛儲君的寶座,故朱常洛一直得不到明神宗歡心。朱由检母亲刘氏則是朱常洛的婢女,亦不得朱常洛的歡心。祖父討厭父親,父親討厭母親,所以朱由检幼年并不幸福。五岁时,朱由檢母親劉氏得罪,被父親朱常洛下令杖杀,之後將朱由检交由庶母西李抚养。数年后西李生了女儿,照管不过来,改由另一庶母东李抚养至成人。及至朱由检长大,被當時已繼位為帝的哥哥朱由校封为信王,刘氏追封为贤妃。

天启七年(1627年),年僅廿二歲的明熹宗朱由校駕崩,由於朱由校三名兒子皆早夭,他唯一在世的弟弟朱由檢繼承皇位,當時朱由檢年僅十六歲,是為崇禎帝。朱由檢即位后,勤于政务,事必躬亲。崇祯十五年(1642年)七月初九,因“偶感微恙”而临时传免早朝,遭辅臣批评,崇禎連忙自我檢討。

天启七年十一月(1627年),崇祯帝在铲除魏忠贤的羽翼崔呈秀之后,再将其贬至凤阳。途至直隶阜城,魏忠贤得知大勢已去,遂与一名太监自缢而亡。此后崇祯帝又殺客氏,崔呈秀自盡,其阉党二百六十餘人或处死、或发配、或终身禁锢。与此同时,平反冤狱,重新启用天启年间被罢黜的官员。起用袁崇焕为兵部尚书,赐予尚方宝剑,託付他收复全辽的重任。

自崇禎元年(1628年)起,中國北方大旱,赤地千里,寸草不生,《汉南续郡志》记,“崇祯元年,全陕天赤如血。五年大饥,六年大水,七年秋蝗、大饥,八年九月西乡旱,略阳水涝,民舍全没。九年旱蝗,十年秋禾全无,十一年夏飞蝗蔽天……十三年大旱……十四年旱”。崇祯朝以來,陕西年年有大旱,百姓多流離失所。崇祯二年五月正式议裁陕北驛站,驛站兵士李自成失业。崇祯三年(1630年)陝西又大饑,陝西巡按馬懋才在《備陳大饑疏》上說百姓爭食山中的蓬草,蓬草吃完,剝樹皮吃,樹皮吃完,只能吃觀音土,最後腹脹而死,六年,“全陕旱蝗,耀州、澄城县一带,百姓死亡过半”。

崇祯七年,家住河南的前兵部尚书吕维祺上書朝廷:“盖数年来,臣乡无岁不苦荒,无月不苦兵,无日不苦輓输。庚午(崇祯三年)旱;辛未旱;壬申大旱。野无青草,十室九空。……村无吠犬,尚敲催征之门;树有啼鹃,尽洒鞭扑之血。黄埃赤地,乡乡几断人烟;白骨青燐,夜夜似闻鬼哭。欲使穷民之不化为盗,不可得也”。旱災又引起蝗災,使得災情更加擴大。河南於崇禎十年、十一年、十二年、十三年皆有蝗旱,“人相食,草木俱盡,土寇並起”,其飢民多從“闖王”李自成。崇祯十三、十四年,“南北俱大荒……死人弃孩,盈河塞路。”

十四年,左懋第督催漕運,道中馳疏言:“臣自靜海抵臨清,見人民飢死者三,疫死者三,為盜者四。米石銀二十四兩,人死取以食。惟聖明垂念。”保定巡撫徐標被召入京時說:“臣自江推來數千里,見城陷處固蕩然一空,即有完城,亦僅餘四壁城隍,物力已盡,蹂躪無餘,蓬蒿滿路,雞犬無音,未遇一耕者,成何世界”這時華北各省又疫疾大起,朝發夕死。“至一夜之內,百姓驚逃,城為之空”,崇禎十四年七月,疫疾從河北地区傳染至北京,崇祯十六年,北京人口死亡近四成。十室九空。

江南在崇祯十三年遭大水,十四年有旱蝗并灾,十五年持续发生旱灾和流行大疫。地方社会处在了十分脆弱的状态,盗匪与流民並起,各地民变不断爆发。

為剿流寇,崇祯帝先用楊鶴主撫,後用洪承疇,再用曹文詔,再用陳奇瑜,復用洪承疇,再用盧象昇,再用楊嗣昌,再用熊文燦,又用楊嗣昌,十三年中頻繁更換圍闖軍的將領。這其中除熊文燦外,其他都表現出了出色的才幹。然皆功虧一簣。李自成數次大難不死,後往河南聚眾發展。

此时北方皇太极又不断骚扰入侵,明廷苦於两线作战,每年的军费「三餉」开支高达两千万两以上,国家财政早已入不敷出,缺饷的情況普遍,常导致明军内部骚乱哗变。加上崇祯帝求治心切,生性多疑,刚愎自用,因此在朝政中屡铸大错:前期铲除专权宦官,后期又重用宦官,《春明梦余录》记述:“崇祯二年十一月,以司礼监太监沈良住提督九门及皇城门,以司礼监太监李凤翔总督忠勇营”崇祯帝說:“朕禦極之初,攝還內鎮,舉天下大事悉以委大小臣工,比者多營私圖,因協民艱,廉通者又遷疏無通。己已之冬,京城被攻,宗社震驚,此士大夫負國家也。清寫明史崇祯帝中后金反间计,自毁长城,冤杀袁崇焕;世傳皇太極施反間計,捕捉兩名明宮太監,然後故意讓兩人以為聽見滿清將軍之間的耳語,謂袁崇煥與滿人有密約,皇太極再放其中一名太監回京。崇祯帝中計,以為袁崇煥謀反。這種講法終明之世並無所本,僅流行於乾隆之後。一些學者傾向於相信崇祯帝殺袁崇煥,並非是皇太極的反間計得逞。由於袁崇煥是囚禁半年後才被處死的,不大可能是因一時激憤誤殺。事實上,崇祯帝生性多疑,所以僅擅殺毛文龍一事,便足以使崇祯帝心存忌憚。再者毛文龍舊部大都誤認為是皇帝要殺毛文龍,於是把怨恨轉移到皇帝身上,大舉譁變,造成日後一連串悲劇事件的發生,終於致使前線態勢一發不可收拾。袁崇煥不能不為此負責。

隨著局勢的日益嚴峻,崇祯帝的濫殺也日趨嚴重,總想以重典治世,總督中被誅者七人,巡撫被戮者十一人,連擁有崇高地位的內閣首辅也不能幸免,被殺二人,而其他各級文官武將更是多不勝數,不能詳列。崇祯帝亦知不能兩面作戰,私底下同意議和,但被明朝士大夫鑒於南宋的教訓,皆以為與滿人和談為恥。因此崇祯帝對於和議之事,始終左右為難,他暗中同意杨嗣昌的议和主张,但一旁的盧象昇立即告訴皇帝說:「陛下命臣督师,臣只知战斗而已!」,崇祯帝只能辯称根本就没有议和之事,盧象昇最後戰死沙場。明朝末年就在和戰兩難之間,走入滅亡之途。

崇禎十五年(1642年),松山、锦州失守,洪承畴降清,崇祯又想和满清议和而和兵部尚書陳新甲暗中商議計劃,後來陳新甲因泄漏議和之事被崇祯诿过處死,與清兵最後議和的機會也破滅了。崇禎十七年(1644年)明王朝面临没顶之灾,崇祯帝召見閣臣時悲嘆道:“吾非亡国之君,汝皆亡国之臣。吾待士亦不薄,今日至此,群臣何无一人相从?”在陳演、光時亨等反对和不情願負責之下未能下决心迁都南京。事後崇禎帝指責光時亨:“阻朕南遷,本應處斬,姑饒這遭。”後來,崇禎再次跟李明睿和左都御李邦華復議南遷的計劃,並要大學士陳演擔當責任,陳演不情願,於是在不久後被罷職。第二次南遷計劃失敗後,崇禎讓駙馬鞏永固代口要求重臣守京師,並以“聖駕南巡,征兵親討」為由出京,諸臣唯恐自己因皇帝不在京城而變成農民军發泄怒火的替死鬼,故依然不讓崇禎離京。

至此,农民军起义已经十多年,从北京向南,南京向北,纵横数千里之间,白骨满地,人烟断绝,行人稀少。崇祯帝召保定巡抚徐标入京觐见,徐标说:“臣从江淮而来,数千里地内荡然一空,即使有城池的地方,也仅存四周围墙,一眼望去都是杂草丛生,听不见鸡鸣狗叫。看不见一个耕田种地之人,像这样陛下将怎么治理天下呢?”崇祯帝听后,潸然泪下,叹息不止。于是,为了祭祀阵亡将士、罹难难民和殉國的各亲王,崇祯帝便在宫中大作佛事来祈求天下太平,并下诏罪己,催促督师孙传庭赶快围剿农民军。

崇禎十六年正月,李自成部克襄陽、荊州、德安、承天等府,張獻忠部陷蘄州,明將左良玉逃至安徽池州。崇禎十七年(1644年)三月一日,大同失陷,北京危急,初四日,崇禎任吳三桂為平西伯,飛檄三桂入衛京師,起用吳襄提督京營。六日,李自成陷宣府,太監杜勳投降,十五日,大學士李建泰投降,李自成部開始包圍北京,太監曹化淳說:「忠賢若在,時事必不至此。」三月十六日,昌平失守,十七日,圍攻北京城。三月十八日,李自成軍以飛梯攻西直、平則、德勝諸門,守軍或逃、或降。下午,曹化淳開彰儀門(一說是十九日王相堯開宣武門,另張縉彥守正陽門,朱純臣守朝陽門,一時俱開,二臣迎門拜賊,賊登城,殺兵部侍郎王家彥於城樓,刑部侍郎孟兆祥死於城門下),李自成軍攻入北京。太監王廉急告皇帝,思宗在宫中饮酒长叹:“苦我民尔!”太監張殷勸皇帝投降,被一劍刺死。崇祯帝命人分送太子、永王、定王到勳戚周奎、田弘遇家。又逼周后自杀,手刃袁妃(未死)、長平公主(未死)、昭仁公主。

然後思宗手執三眼槍與數十名太監騎馬出東華門,被亂箭所阻,再跑到齊化門(朝陽門),成國公朱純臣閉門不納,後轉向安定門,此地守軍已經星散,大門深鎖,太監以利斧亦無法劈開。三月十九日拂曉,大火四起,重返皇宮,城外已经是火光映天。此時天色将明,崇祯在前殿鸣钟召集百官,却无一人前来,崇祯帝說:“诸臣误朕也,国君死社稷,二百七十七年之天下,一旦弃之,皆为奸臣所误,以至于此。”最後在景山老歪脖子树上自缢身亡,死时光着左脚,右脚穿着一只红鞋。死於崇禎甲申三月十九日丑時,时年33岁。身边仅有提督太监王承恩陪同。上吊死前于蓝色袍服上大书其遺書:

“朕自登極(或作登基)十有七年,虽朕凉德藐躬(或作薄德匪躬),上干天咎(或作天譴、天怒),致逆贼直逼京师,然皆诸臣之误朕也。朕死无面目见祖宗于地下,自去冠冕,以髮覆面。任贼分裂朕尸,勿伤百姓一人。”

三月二十一日屍體被發現,大順軍將崇祯帝與周皇后的屍棺移出宮禁,在東華門示眾,也允許投降的諸臣前往送葬,只是人數不多,“諸臣哭拜者三十人,拜而不哭者六十人,餘皆睥睨過之”,只有主事劉養貞極其悲痛,梓宮暫厝在紫禁城北面的河邊。

崇祯帝死後,自杀官員有户部尚书倪元璐、工部尚书范景文、左都御史李邦华、左副都御史施邦曜、协理京营兵部右侍郎王家彦、大理寺卿凌义渠、太常寺卿吳麟徵、左中允刘理顺、刑部右侍郎孟兆祥、前戶科都給事中吳甘來、武庫主事成德、兵部主事金鉉、左諭德马世奇、檢討汪偉、右庶子周鳳翔、太僕寺丞申佳胤、吏部員外郎許直、戶部員外郎寧承烈、光禄寺署丞于腾雲、副兵馬使姚成、中書舍人宋天顯,滕之所、阮文貴、監察御史王章、陳良謨、陳纯德、經歷張應選,順天府知事陈貞達等、外戚如驸马都尉巩永固、新樂伯劉文炳、惠安伯張慶臻、宣城伯衛時春,錦衣衛都指揮使王國興自殺,太监自杀者以百计,战死在千人以上。宫女自杀者三百余人。绅生生员等七百多家举家自杀。四月四日,昌平州吏趙一桂等人將崇禎與皇后葬入昌平縣田貴妃的墓穴之中,清朝以“帝禮改葬,令臣民為服喪三日,諡曰莊烈愍皇帝,陵曰思陵”。

崇禎十七年五月初六日,多爾袞以李明睿為禮部侍郎,負責大行皇帝的諡號祭葬事宜,李擬上先帝諡號欽天守道敏毅敦儉弘文襄武體仁致孝端皇帝,廟號懷宗,并建議改葬梓宮。後因思宗梓宮已入葬恭淑端惠靜懷皇貴妃的園寢,便不再遷葬,改田貴妃園寢為思陵。

順治十六年十一月,以“興朝諡前代之君禮,不稱數、不稱宗”為由,[原創研究?]去懷宗廟號,改諡莊烈愍皇帝,因而清代史書多簡稱為莊烈帝或明愍帝。

《欽定古今圖書集成·方輿彙編·職方典·順天府部雜錄十一》、《欽定日下舊聞考·卷一百三十七》、《讀禮通考·卷九十三》三書均引《肅松錄》和《北游紀方》,稱思陵神牌題為“大明欽天守道敏毅敦儉弘文襄武體仁致孝莊烈愍皇帝”,又引《北游紀方》稱思陵神主題為“大明懷宗欽天守道敏毅敦儉弘文襄武體仁致孝莊烈端皇帝”,又引《肅松錄》稱思陵立有“莊烈愍皇帝之陵”的石碑。《明詩綜·卷一》則稱神牌是由順治初年定的“一十六字”加上改書的“莊烈愍皇帝”組合而成。神主甚至又改“愍”字為“端”,並仍題廟號“懷宗”二字,可見康熙年間的思陵神牌和神主是由順治年間兩次加諡崇禎帝的廟諡號混雜而成。《崇禎長編·卷一》作“果毅敦儉弘文襄武體仁致孝莊烈愍皇帝”,當是清廷所給諡號在傳抄中產生了訛誤。

南明安宗之大臣張慎言初議崇禎帝之廟諡號為“烈宗敏皇帝”,高弘图拟庙号“思宗”,顧錫疇議廟號“乾宗”。赵之龙上疏弹劾高弘图议庙号之失,称“思为下谥”。顧錫疇又拟庙号正宗,但未被採用。最終在崇禎十七年六月定先帝谥號為紹天繹道剛明恪儉揆文奮武敦仁懋孝烈皇帝,庙号思宗。 按《逸周書·諡法解》:“道德純一曰思。大省兆民曰思。外內思索曰思。追悔前過曰思。……有功安民曰烈。以武立功。秉德尊業曰烈。”

弘光元年李青上疏请改思宗庙号,多次上疏皆被駁回。管紹寧擬“敬宗”和“毅宗”兩號備選,同時又有人上疏請求改為“烈宗正皇帝”。弘光元年二月丙子改上廟號毅宗,谥号未改。唐王监国,谥思宗為威宗。

與其他朝代的亡國之君不同,崇祯帝是一個被普遍同情的皇帝,崇祯帝一直勤政,以挽救過去祖輩皇帝的過失。崇祯帝即位,正值國家內憂外患之際,內有黃土高原上百萬農民造反大軍,外有滿洲鐵騎,虎視耽耽,崇祯元年(1628年)陕西镇的兵饷积欠到30多月,次年二月延绥、宁夏、固原三镇皆告缺饷达36月之久。

推翻明朝的李自成《登極詔》也說“君非甚闇(崇禎皇帝不算太糟),孤立而煬灶恆多(孤立於上,而受到奸臣的蒙蔽);臣盡行私,比黨而公忠絕少。”

思宗的性格相當複雜,在去除魏忠賢時,崇禎表現得極為機智,但在處理袁崇煥一事,卻又表現得相當愚蠢,《明史》說他:「性多疑而任察,好刚而尚气。任察则苛刻寡恩,尚气则急遽失措。」

张岱认为「思宗焦心求治,旰食宵衣,恭俭辛勤,万机无旷。即古之中兴令主无以过之。」然而,他「惟务节省」,以至「九边军士数年无饷,体无完衣」;又「渴于用人,骤于行法」,以至「天下之人,无所不用。及至危及存亡之秋,并无一人为之分忧宣力。」

《明史》評價思宗:「帝承神、熹之後,慨然有為。即位之初,沈機獨斷,刈除奸逆,天下想望治平。惜乎大勢已傾,積習難挽。在廷則門戶糾紛,疆埸則將驕卒惰。兵荒四告,流寇蔓延。遂至潰爛而莫可救,可謂不幸也已。然在位十有七年,不邇聲色,憂勸惕勵,殫心治理。臨朝浩歎,慨然思得非常之材,而用匪其人,益以僨事。乃復信任宦官,布列要地,舉措失當,制置乖方。祚訖運移,身罹禍變,豈非氣數使然哉。迨至大命有歸,妖氛盡掃,而帝得加諡建陵,典禮優厚。是則聖朝盛德,度越千古,亦可以知帝之蒙難而不辱其身,為亡國之義烈矣。」

顺治帝評價思宗:「本朝入关定鼎,首为崇祯帝、后发丧,营建幽宫,为万古未闻之义举。」1657年,顺治谕工部曰:「朕念明崇祯帝孜孜求治,身殉社稷。若不急为阐扬,恐于千载之下,竟与失德亡国者同类并观,朕用是特制碑文一道,以昭悯恻。」谒崇祯陵的时候,顺治大呼说:「大哥大哥,我与若皆有君无臣。」顺治对崇祯的书法更是高度赞赏。史书记载,僧弘觉向顺治索字,顺治说:「朕字何足尚,崇祯帝乃佳耳。」说完叫人一并拿来八九十幅崇祯的字,一一展示,“上容惨戚,默然不语”。看完了,顺治说:「如此明君,身婴巨祸,使人不觉酸楚耳。」又说:「近修《明史》,朕敕群工不得妄议崇祯帝。」顺治的话,连弘觉都给感动了:「先帝何修得我皇为异世知己哉!」顺治写给崇祯的碑文是:「庄烈悯皇帝励精图治,宵旰焦心,原非失德之主。良由有君无臣,孤立于上,将帅拥兵而不战,文吏噂沓而营私。……逮逆渠犯阙,国势莫支,帝遂捐生以殉社稷。……」

談遷《國榷》稱:“先帝(崇禎)之患,在於好名而不根于實,名愛民而適痡之,名聽言而適拒之,名亟才而適市之;聰于始,愎于終,視舉朝無一人足任者,柄托奄尹,自貽伊戚,非淫虐,非昏懦,而卒與桀、紂、秦、隋、平、獻、恭、昭並日而語也,可勝痛哉!”

歷史學家孟森說:“思宗而在萬曆以前,非亡國之君;在天啟之後,則必亡而已矣!”。思宗雖有心為治,卻無治國良方,以致釀成亡國悲劇,未必無過。孟森也說思宗“苛察自用,無知人之明”、“不知恤民”。思宗用人不彰、疑心過重、馭下太嚴,史稱“崇禎五十相”(在位十七年,更換五十位內閣大學士、首輔),卻加速了明王朝的覆亡。

鎖綠山人在《明亡述略》中評價崇禎,“莊烈帝勇於求治,自異此前亡國之君。然承神宗、熹宗之失德,又好自用,無知人之識。君子修身齊家,宜防好惡之癖,而況平天下乎?雖當時無流賊之蹂躪海內,而明之亡也決矣。”

南明大臣則把思宗抬舉到千古聖主的地步,如禮部侍郎余煜在議改思宗廟號時說:“先帝(崇禎)英明神武,人所共欽,而內無聲色狗馬之好,外無神仙土木之營,臨難慷慨,合國君死社稷之義。千古未有之聖主,宜尊以千古未有之徽稱。”

\subsection{崇祯}

\begin{longtable}{|>{\centering\scriptsize}m{2em}|>{\centering\scriptsize}m{1.3em}|>{\centering}m{8.8em}|}
  % \caption{秦王政}\
  \toprule
  \SimHei \normalsize 年数 & \SimHei \scriptsize 公元 & \SimHei 大事件 \tabularnewline
  % \midrule
  \endfirsthead
  \toprule
  \SimHei \normalsize 年数 & \SimHei \scriptsize 公元 & \SimHei 大事件 \tabularnewline
  \midrule
  \endhead
  \midrule
  元年 & 1628 & \tabularnewline\hline
  二年 & 1629 & \tabularnewline\hline
  三年 & 1630 & \tabularnewline\hline
  四年 & 1631 & \tabularnewline\hline
  五年 & 1632 & \tabularnewline\hline
  六年 & 1633 & \tabularnewline\hline
  七年 & 1634 & \tabularnewline\hline
  八年 & 1635 & \tabularnewline\hline
  九年 & 1636 & \tabularnewline\hline
  十年 & 1637 & \tabularnewline\hline
  十一年 & 1638 & \tabularnewline\hline
  十二年 & 1639 & \tabularnewline\hline
  十三年 & 1640 & \tabularnewline\hline
  十四年 & 1641 & \tabularnewline\hline
  十五年 & 1642 & \tabularnewline\hline
  十六年 & 1643 & \tabularnewline\hline
  十七年 & 1644 & \tabularnewline
  \bottomrule
\end{longtable}


%%% Local Variables:
%%% mode: latex
%%% TeX-engine: xetex
%%% TeX-master: "../Main"
%%% End:

%% -*- coding: utf-8 -*-
%% Time-stamp: <Chen Wang: 2021-11-01 17:16:19>

\section{南明\tiny(1644-1662)}

\subsection{生平}

南明(1644年-1662年),中國朝代,是甲申之變後,明朝皇族與官員在中國南方相繼成立的明朝政權,為時十八年[註 1]。南明主要勢力有四系王,分別是福王弘光帝朱由崧、魯王監國朱以海、唐王隆武帝朱聿鍵與紹武帝朱聿鐭、桂王永曆帝朱由榔等。

1644年明朝首都北京被李自成攻陷[1][2],南明大臣意圖擁護皇族北伐。經過多次討論後由鳳陽總督馬士英與江北四鎮高傑、黃得功、劉澤清與劉良佐擁護明思宗的堂兄弟福王朱由崧稱帝,即弘光帝,国号依旧为大明,史称南明或后明。1645年清軍攻破揚州[3][4][5],弘光帝逃至蕪湖被逮,後被送到北京殺害[6]。弘光帝死後,魯王朱以海於浙江紹興監國;而唐王朱聿鍵在鄭芝龍等人的擁立下,於福建福州稱帝,即隆武帝。然而這兩個南明主要勢力互不承認彼此地位,而互相攻打。1651年在舟山群島淪陷後,魯王朱以海在張名振、張煌言陪同下,赴廈門依靠鄭成功,不久病死在金門。隆武帝屢議出師北伐,然而得不到鄭芝龍的支持而終無所成。1646年,清軍分別占領浙江與福建,魯王朱以海逃亡海上,隆武帝於汀州逃往江西時被俘而死。鄭芝龍向清軍投降,由於其子鄭成功起兵反清而被清廷囚禁。朱聿鍵死後,其弟朱聿鐭在廣州受蘇觀生及廣東布政司顧元鏡擁立稱帝,即紹武帝,於同年年底被清將李成棟攻滅。同時間桂王朱由榔於廣東肇慶稱帝,即永曆帝[6]。

1646年永曆帝獲得瞿式耜、張獻忠餘部李定國、孫可望等勢力的加入以及福建鄭成功勢力的支援之下展開反攻。同時各地降清的原明軍將領先後反正,例如1648年江西金聲桓、廣東李成棟、廣西耿獻忠與楊有光率部反正,一時之間南明收服華南各省。然而於同年,清將尚可喜率軍再度入侵,先後占領湖南、廣東等地。兩年後,李定國、孫可望與鄭成功發動第二次反攻,其中鄭成功一度包圍南京。然而,各路明軍因為距離互相難以照應,內部又發生孫可望等人的叛變,第二次反攻以節節敗退告終。1661年,清軍三路攻入云南,永曆帝流亡缅甸首都曼德勒,被缅甸王莽達收留。後吴三桂攻入缅甸,莽達之弟莽白乘机发动政变,杀死其兄後继8月12日,莽白發動咒水之难,杀盡永曆帝侍從近衛[7],永曆帝最後被吴三桂以弓弦絞死,南明正式滅亡[6]。此時反清勢力只剩夔東十三家軍與在金廈及台灣的明鄭王朝。

明崇禎十七年(1644年)正月,李自成在西安稱帝,建國「大順」,之後向北京進兵,三月十九攻克北京,崇禎皇帝朱由檢殉國,明朝宗室及遺留大臣多輾轉向南遁走。此时李自成的「大順」政權大体據有淮河以北原明朝故地,張獻忠於八月成立的「大西」政權則據四川一帶,清朝政權則據有山海关外的现今东北地区,且控制蒙古诸部落,而明朝的殘餘勢力則據有淮河以南的半壁江山。

此时明朝留都南京的一些文臣武將決計擁立朱家王室的藩王,重建明朝,然後揮師北上;但具体拥立何人则发生争议。根据“皇明祖训”,有嫡立嫡、无嫡立长,在当时明神宗长子光宗一脉(其後繼者是熹宗天啟皇帝和思宗崇禎皇帝)已无人能继位,而次子朱常漵甫生即死,三子朱常洵虽已亡故,但其长子朱由崧仍健在的情况下,按照兄終弟及的順序,第一人選為福王朱由崧;而钱谦益等东林党人由於之前的「國本之爭」事件,心存芥蒂,违背了东林党在国本之争中的立场,以立贤为名擁立神宗弟弟朱翊镠之子潞王朱常淓[8][9];史可法则主张既要立贤也要立亲,拥立神宗七子桂王朱常瀛。但最终福王朱由崧在卢九德的帮助下,获得了南京政权主要武装力量江北四镇高杰、黄得功、刘良佐和刘泽清,以及中都凤阳总督马士英的支持,成为最终的胜利者。五月初三,朱由崧監國于南京,五月十五 (1644年6月19日) 日即皇帝位,改次年為弘光,是為明安宗。南明時代自此開始。弘光帝的基本國策以「聯虜平寇」為主,謀求與清軍連合,一起消滅以李自成、张献忠为代表的農民軍。

明朝南渡前後,大顺已被多爾袞與吳三桂的聯軍击溃,李自成先后丢失北京和西安,退往湖北。弘光元年(1645年)三月,多尔衮将军事重心东移,命多铎移师南征。此时弘光政權內部正進行著激烈的黨爭,爆发太子案,駐守武昌的左良玉不愿与李自成正面交战,以「清君側」为名,順长江東下争夺南明政权。馬士英被迫急調江北四鎮迎擊左軍,致使面对清军的江淮防線陷入空虛。史可法时在揚州虽有督師之名,却实无法调动四镇之兵。一月之中,清軍破徐州,渡淮河,兵臨揚州城下。四月廿五,揚州城陷,史可法不屈遇害。隨後,清軍渡過長江,攻克鎮江。弘光帝出奔蕪湖。五月十五众大臣獻南京投降清兵;五月廿二弘光帝被虜獲,送往北京處死,弘光帝在位仅一年,即覆滅。

南京失陷後,又有杭州的潞王朱常淓(1645年)、金陵的崇禎太子朱慈烺(可能是貌似太子的王之明。1645年)、撫州的益王朱慈炲(1645年)、福州的唐王朱聿鍵(1645-1646年)、紹興的魯王朱以海(1645-1653年)、桂林的靖江王朱亨嘉(1645年)等監國政權先後建立,其中唐王朱聿鍵受鄭芝龍等人在福州擁立,登極稱帝,改元隆武,是為明紹宗。這時清朝再次宣佈薙髮令,江南一帶掀起了反薙髮的抗清鬥爭,清軍後方發生動亂,一時無力繼續南進。但南明內部嚴重的黨派鬥爭與地方勢力跋扈自雄,且隆武帝與魯王政權不但沒有利用這種有利形勢,發展抗清鬥爭,反而在自己之間為爭正統地位而形同水火,各自為戰,所以當1646年清軍再度南下時,先後為清軍所各各擊滅。魯王在張煌言等保護下逃亡海上,在沿海一帶繼續抗清;隆武帝則被清軍俘殺。

11月,在廣州和肇慶又成立了兩個南明政權:隆武帝之弟唐王朱聿鐭(1646年)繼位於廣州,改明年为紹武元年;桂王朱由榔(1646-1662年)稱帝於肇慶,改元永曆,是為明昭宗。紹武、永曆二帝也不能團結,甚至大動干戈,互相攻伐。紹武政權僅存在40天就被清軍消滅。揭陽的益王朱由榛(1647年)、夔州的楚王朱容藩(1649年)稱監國與永曆帝爭立。鄭成功也在南澳一度立淮王朱常清(1648年)為監國,後廢。永曆帝在清軍進逼下逃入廣西。

正當南明政權一個接一個地覆亡,形勢萬分危急之際,大順農民軍餘部出現在抗清鬥爭最前線,挽救了危局。自李自成于1645年战死于九宫山後,他的餘部分為二支,分別由郝搖旗、劉體純和李過、高一功率領,先後進入湖南,與明湖廣總督何騰蛟、湖北巡撫堵胤錫聯合抗清。1647年,郝搖旗部護衛逃來廣西的永曆帝居柳州,並出擊桂林。年底,大敗清軍於全州,進入湖南。次年,大順軍餘部又同何騰蛟、瞿式耜的部隊一起,在湖南連連取得勝利,幾乎收復了湖南全境。這時,廣東、四川等地的抗清鬥爭再起,清江西提督金聲桓、清广东提督李成棟、清广西巡抚耿献忠、清大同总兵姜镶、清延安营参将王永强、清甘州副将米喇印先後反正回归明朝,清軍後方的抗清力量也發動了廣泛的攻勢。一時間,永曆政權名义控制的區域擴大到了雲南、貴州、廣東、廣西、湖南、江西、四川七省,还包括北方山西、陕西、甘肃三省一部以及东南福建和浙江两省的沿海岛屿,出現了南明時期第一次抗清鬥爭的高潮。

但永曆政權內部仍然矛盾重重,各派政治勢力互相攻訐,農民軍也倍受排擠打擊,不能團結對敵,這就給了清軍以喘息之機。1649-1650年,何騰蛟、瞿式耜先後在湘潭、桂林的戰役中被俘杀,清軍重新佔領湖南、廣西;其他剛剛收復的失地也相繼丟掉了。不久,李過之子李來亨等農民軍將領率部脫離南明政府,轉移到巴東荊襄地區組成夔東十三家軍,獨立抗清。這支部隊一直堅持到1664年。

綜觀1645-1651年間,南明軍與清軍作戰中,敗多勝少,大批南明的軍隊先後降清。先後丟失了江蘇、安徽、浙江、江西、福建、兩廣、兩湖等等領地,地盤盡失。直到以孫可望為主的大西軍加入,再次改變了整個局勢。

張獻忠于1646年战死後,以其义子孫可望、李定國、刘文秀、艾能奇等人為主的大西軍残部自1647年進佔雲南、貴州二省。1652年,南明永曆政权接受孫可望和李定國的建议聯合抗清建議,定都安龙。不久,以大西军餘部为主体的南明軍對清軍展開了全面反擊。李定國率軍8萬東出湖南,取得靖州大捷,收复湖南大部;随后南下广西,取得桂林大捷,击毙清定南王孔有德,收复广西全省;然后又北上湖南取得衡阳大捷,击毙清敬谨亲王尼堪,天下震动。同時,劉文秀亦出擊四川,取得叙州大捷、停溪大捷,克復川南、川东。孙可望也亲自率军在湖南取得辰州大捷。東南沿海的張煌言、郑成功等的抗清軍隊也乘机發動攻勢,接连取得磁灶大捷、钱山大捷、小盈岭大捷、江东桥大捷、崇武大捷、海澄大捷的一连串胜利,並接受了永曆封號。一時間,永曆政權名义控制的區域恢复到了雲南、貴州、廣西三省全部,湖南、四川两省大部,廣東、江西、福建、湖北四省一部,出現了南明時期第二次抗清鬥爭的高潮。

之后,劉文秀於四川用兵失利,在保宁战役中被吳三桂侥幸取胜。而孫可望妒嫉李定國桂林、衡州大捷之大功,逼走李定國,自己统兵却在宝庆战役中失利。东南沿海的郑成功也在漳州战役中失利。所以明军在四川、湖南、福建三个战场上没能扩大战果,陷入了与清军相持的局面。之後李定國與鄭成功聯絡,於1653年、1654年率軍兩次進軍廣東,約定与鄭會師廣州,一舉收復廣東,但鄭軍屢誤約期,加上瘟疫流行,导致肇庆战役和新会战役没能成功。但郑成功部队并没有闲着,1656年,郑军取得泉州大捷,1657年又取得护国岭大捷。

永曆十年(1656年),孫可望祕謀篡位,引發了南明內部一場内讧,李定國擁永曆帝至雲南,次年大敗孫可望,孫可望勢窮降清。孫可望降清後,西南軍事情報盡供清廷,雲貴虛實盡為清軍所知。永曆十二年(1658年)四月,清軍主力從湖南、四川、廣西三路進攻貴州。年底吳三桂攻入雲南,次年正月,下昆明,進入雲南,永曆帝狼狽西奔,進入緬甸(東吁王朝)。李定國率全軍設伏於磨盤山,企圖一舉殲滅敵人追兵,結果因內奸洩密导致未能大获全胜,南明軍精銳損失殆盡,此即磨盘山血战。這時鄭成功趁清軍主力大舉攻擊西南之際,率領十餘萬大軍北伐,接连取得定海关大捷、瓜州大捷、镇江大捷的胜利,一度兵临南京城下,然而鄭軍中了清軍緩兵之計,最终失败,撤回廈門。清军派大军围攻厦门,企图一举歼灭郑成功,但郑成功沉着应战,取得厦门大捷的胜利,稳定了东南沿海局势。永曆十五年(1661年),吳三桂率清軍入緬,索求永曆帝,十二月緬甸東吁王朝國王平達力(莽達)將永曆交予清軍,次年四月永曆帝與其子哀愍太子朱慈煊等被吳三桂處死于昆明。七月,李定國在真臘得知永曆帝死訊,亦憂憤而死。而同年五月,鄭成功亦於臺灣急病而亡。

此后郑氏政权未再拥立皇帝或朱氏监国,而是继续奉永历为正朔。1683年,延平郡王鄭克塽降清,清军占领台湾,宁靖王朱术桂自杀殉国,标志着大明最后一个政权的覆灭。

南明時期,安南、日本、琉球、呂宋、占城也曾派使者入貢[10]。隆武元年也曾頒登基詔書予琉球,並記載於琉球《歷代寶案》一書。

南明弘光帝曾以對等的禮儀派使者左懋第詔諭,並稱順治帝為清國可汗。在詔書中,弘光帝提出四件事:要安葬崇禎帝及崇禎皇后、以山海關為界,關外土地給予清朝、每年十萬歲幣,並「犒金千兩、銀十萬兩、絲緞萬匹、犒銀三萬兩」、建國任便。[10]意圖令南明和清朝共存,通好議和。不過左懋第到北京被囚,使事失敗。

\subsection{安宗朱由崧\tiny(1644-1645)}

\subsubsection{生平}

明安宗朱由崧(1607年9月5日-1646年7月1日),又稱弘光帝,為南明首位皇帝,原為福王。朱由崧是明神宗朱翊钧之孙,福忠王朱常洵之子。他是明熹宗朱由校、明思宗朱由檢的堂兄弟。思宗殉国後,朱由崧在南京即位,改元弘光,在位僅一年。弘光元年清军南攻,朱由崧被俘,押往北京,翌年被處決。南明永历帝为其上庙号安宗,谥号奉天遵道宽和静穆修文布武溫恭仁孝簡皇帝。

朱由崧小字福八,明神宗孙,福忠王朱常洵庶长子。万历三十五年七月乙巳生于福王京邸,生母姚氏。万历四十二年随福王朱常洵就藩于洛阳。万历四十八年七月甲辰封德昌王,后进封福王世子。

崇祯十四年正月,流賊李自成陷洛阳,福王常洵缒城出,藏匿于迎恩寺,后被搜出,遇害。朱由崧缒城逃脱,前往怀庆避难,崇祯十六年五月袭封福王。崇祯帝手择宫中玉带,遣内使赐之。

崇祯十七年正月,怀庆闻警,朱由崧逃亡卫辉,投奔潞王朱常淓。三月初四卫辉闻警,朱由崧随潞王逃往淮安,与南逃的周王、崇王一同寓居于湖嘴舟中。三月十一日周王朱恭枵薨于舟上,三月十八日福王上岸,住在杜光绍园中。三月十九日李自成陷北京,崇禎帝自縊,是為甲申之變。廿九日,消息传至淮安。

四月崇祯帝自盡的消息,传至南京,北京沦陷後,南京以及南方各省仍在明朝的控制之下。南京诸臣皆認為國不可一日無君,议立新帝。但對大寶誰屬,則有一番論戰。

从血统上来说,崇祯帝殉国,其子太子朱慈烺及永王朱慈炤、定王朱慈炯陷入清军之手,而崇禎帝父明光宗朱常洛仅有天啟帝、崇祯帝二子,天啟帝無子,而故应从崇禎帝祖父明神宗之子、光宗诸弟中选择。明神宗福王常洵为第三子,瑞王常浩为第五子,惠王常润为第六子,桂王常瀛为第七子,以常洵居长。朱由崧为朱常洵长子,因此在崇祯太子及定、永二王无法至南京继位的情况下,福王本为第一順位。然而東林黨人卻持相反意見,他們恐朱由崧即位后追究昔日“三案”及國本之爭攻讦郑贵妃(朱由崧祖母)之事,主张立明神宗之侄潞王朱常淓。史可法并称福王“在藩不忠不孝,恐难主天下”。四月二十六日,张慎言、高弘图、姜曰广、李沾、郭维经、诚意伯刘孔昭、司礼太监韩赞周等在朝中会议,李沾、刘孔昭、韩赞周议立福王,议遂定以福王继统,告庙并修武英殿。鳳陽總督馬士英與江北四鎮黃得功、高傑、劉良佐、劉澤清等人前往淮安迎接朱由崧。四月二十七日甲申,南京礼部率百司迎福王于儀真。

崇祯十七年四月二十八日乙酉,朱由崧至浦口,魏国公徐弘基等渡江迎接。翌日舟泊观音门燕子矶。四月三十日丁亥,南京百官迎见朱由崧于龙江关舟中,请其為監國。朱由崧身穿角巾葛衣,坐于卧榻之上,推说自己未携宫眷一人,准备避难浙东。众臣力劝,朱由崧乃同意。

五月初一戊子,朱由崧骑马自三山门环城而东,拜谒孝陵和懿文太子陵,随后经朝阳门入东华门,谒奉先殿,出西华门,以南京内守备府为行宫。五月初二群臣至行宫劝进,朱由崧以太子及定王、永王不知下落,且瑞王、惠王、桂王均为叔父行,应择贤迎立。诸臣再三劝进,乃依明代宗故事监国。五月初三庚寅自大明门入大内,至武英殿行监国礼。是日吴三桂引清摄政王多尔衮入北京。

崇祯十七年五月十五日壬寅,朱由崧即皇帝位于武英殿,以次年为弘光元年。其国号依旧为“大明”,史称“南明”。

朱由崧即位后,于六月戊午追封祖母郑贵妃为孝宁太皇太后,父福忠王朱常洵为贞纯肃哲圣敬仁毅恭皇帝(后改谥孝皇帝),立庙于南京,墓园称熙陵。上嫡母邹氏尊号为恪贞仁寿皇太后,生母姚氏为孝诚端惠慈顺贞穆皇太后。追封洛阳城陷时遇害的胞弟颍上王朱由榘为颍王,谥曰冲。六月辛酉上崇祯帝庙号为思宗,谥号烈皇帝。七月己丑追复懿文太子帝号,追崇建文帝、景泰帝庙号谥号。

东林党人编撰的史书说朱由崧生性暗弱,不忠不孝,荒淫无耻,政事则悉委于马士英、阮大铖。马、阮二人日以卖官鬻爵、报撼私仇为事,导致南明政事萎靡,不断发生内讧;而名臣李清则力为弘光辩冤,说这些记载都是谣言,又说弘光帝很少接近女色。在外以史可法督师江北,设淮、扬、凤、庐四镇,以黄得功、刘良佐、刘泽清、高杰为总兵统领,南明出现军阀化的趋势。前線將領不但因爭權而互相攻擊,也有掠奪平民的行為。

朱由崧即位后,下令选淑女入宫,派宦官于南京城中四出搜巷,凡是有女之家,必以黄纸贴额,持之而去,南京城中骚动。朱由崧又下令修西宫西一路为慈禧殿,以安置继母邹太后。当年八月邹太后自河南至南京,八月十四日谕户、兵、工三部“太后光临,限三日内搜括万金,以备赏赐”。八月十六日御用监又令造龙凤床座、床顶架、宫殿陈设金玉等项,越数十万两。造皇后冠,命内臣采购猫眼石、祖母绿及大珠重一钱以上者百余颗。崇祯十七年除夕,弘光帝独坐兴宁宫中,愀然不乐。太监韩赞周问道:“宫殿新落成,皇上应当欢喜,而闷闷不乐,是思念皇兄吗?”弘光帝不应,继而回答说:“梨园殊少佳者”。弘光元年(1645年)正月,弘光帝又下令修南京奉先殿、午门及左右掖门,并派太监田成至杭州、嘉兴二府选淑女。

崇祯十七年九月初三,弘光帝下令为北京殉难诸臣上谥号,计文臣二十一人、勋臣二人、戚臣一人。随后又给郢国公冯国用、宋国公冯胜、济国公丁德兴、德庆侯廖永忠、长兴侯耿炳文等开国功臣追上谥号;给方孝孺、齐泰、黄子澄、陈迪、景清、卓敬、练子宁等建文朝死难诸臣,蒋钦、陆震等正德朝死谏诸臣,左光斗、周朝瑞、周宗建、袁化中、顾大章、周起元等天启朝死珰难诸臣上谥号。

弘光元年三月初一甲申,有自称崇祯太子朱慈烺者至南京,朱由崧命令将其关入兵马司监狱,后命百官审北来太子于午门外,终裁断为伪太子王之明,是為崇禎太子案。三月庚申,宁南侯左良玉乃举兵于武昌,以“救太子、诛士英”为名顺流而下,黄得功、阮大铖率兵御之,南明发生内讧。正值此时,清军在豫王多铎率领下大举南下,攻陷归德、颍州、太和、泗州等地。

弘光元年四月辛未,清军围攻江北重镇扬州。督師江北的兵部尚書史可法率城中百姓抵御清军,清军围困百日,损失惨重。史可法急忙向朝廷求援,但卻因為鎮將們個個擁兵自重、意圖觀望,最終揚州在被围五天后沦陷。清军攻破扬州之後进行了十天屠杀,史称“扬州十日”。四月甲子,弘光帝在南京贡院选淑女,七十人中选中一人,即阮大铖的侄女。四月壬戌,杭州送来淑女五十人,弘光帝选中周姓一人,王姓一人。

弘光元年五月初八己丑,清军自瓜洲渡江,镇江巡抚杨文骢逃奔苏州,靖虏伯郑鸿逵逃入東海,总兵蒋云台投降。南京闭城门。五月初十辛卯,朱由崧传旨放归所选淑女,当天午夜尤召梨园入宫演剧。翌日凌晨二漏时,朱由崧率内官四五十人骑马出通济门,莫知所踪。天亮后百官入朝,见宫女、内臣、优伶杂沓逃奔西华门外,方知弘光帝已出逃。南京城内大哗,马士英携邹太后出奔,市民救北来太子出狱,扶其入宫,在武英殿即位。五月十二日癸巳,朱由崧至太平府,以按察院为行宫,寻即移驾芜湖,投奔靖国公黄得功军营。五月十五日丙申,清军入南京,魏国公徐文爵、保国公朱国弼、灵璧侯汤国祚、定远侯邓文郁,及尚书钱谦益、大学士王铎、都御史唐世济等人剃髮降清。

清军攻克南京后,多铎命降将刘良佐带清兵追击弘光帝。五月二十二日癸卯,总兵田雄、马得功、丘钺、张杰、黄名、陈献策冲上御舟,劫持弘光帝,将其献给清军。豫王多铎命去锁链,以红绳捆绑。五月二十五日丙午,朱由崧乘无幔小轿入南京聚宝门,头蒙缁素帕,身衣蓝布袍,以油扇掩面,两妃乘驴随后,夹路百姓唾骂,有投瓦砾者。多铎在灵璧侯府设宴,命朱由崧居于北来太子之下。宴罢,拘弘光帝于江宁县署。

弘光元年闰六月,唐王朱聿鍵即位于福州,改元隆武,遥上朱由崧尊号为「上皇圣安皇帝」。当年九月甲寅,朱由崧与皇太后邹氏、潞王朱常淓等人被押送至燕京,安置居住。由滿清太医院,日时馈宴,朱由崧酣饮极乐。

顺治三年(1646年,隆武二年)四月九日,有人向清摄政王多尔衮告发,称燕京居住的故明衡王、荆王欲谋反。五月甲子,弘光帝与秦王朱存極、晋王朱審烜、潞王朱常淓、荆王朱慈煃、徳王朱由栎、衡王朱由棷等十七人被斬首於菜市口(一说弘光帝以弓弦絞死)。

朱由崧王妃黄氏之弟黄調鼎购得棺木,与黄妃合葬于河南孟津县东山头村。

弘光帝凶讯南传后,监国魯王朱以海上谥号为赧皇帝,不久又上庙谥为质宗安皇帝。永曆帝立,于永历十一年四月改弘光帝廟號曰安宗,谥号奉天遵道宽和静穆修文布武溫恭仁孝簡皇帝。

根据明末清初笔记记载,朱由崧是个十分昏庸腐朽的君主,整日只知吃喝玩乐,沉湎于酒色之中,不理朝政。在其即位之前,史可法曾寫信給馬士英說明「福王七不可立」──貪、淫、酗酒、不孝、虐下、無知和專橫。由史可法、張慎言、高弘圖等17人簽名送與馬士英。後人称其为明朝及南明最昏庸的帝王,唯知享樂,不問政事,沉湎酒色,荒淫透頂。然而細檢史籍可知竟傳聞難據,推其緣由多為東林黨人因國本之爭對福王藩一系的成見所致。而其本來的經歷顯現的是並非昏庸且頗有個性的政治家形象。如曾任弘光朝給事中李清《三垣筆記》、《南渡錄》及《甲申日記》對荒淫縱欲之事,且加辯誣。此外,朱由崧替靖難之變殉難的明惠帝一系君臣予以平反,並貶抑當時擴大迫害的陳瑛。因此其政治得失尚有爭議。

钱海岳《南明史》评价弘光帝“北京颠覆,上膺鼎籙,丰芑奠磐,徵用俊耆。卷阿翙羽,相得益彰。故初政有客观者。性素宽厚,马、阮欲以《三朝要典》起大狱,屡请不允。观其谕解良玉,委任继咸,词婉处当;拒纳银赎罪之议,禁武臣罔利之非,皆非武、熹昏騃之比。顾少读书,章奏未能亲裁,政事一出士英,不从中制,坐是狐鸣虎噬,咆哮恣睢,纪纲倒持。及大铖得志,众正去朝,罗罻高张,党祸益烈。上燕居神功,辄顿足谓士英误我,而太阿旁落,无可如何,遂日饮火酒,亲伶官优人为乐,卒至触蛮之争,清收渔利。时未一朞,柱折维缺。故虽遗爱足以感其遗民,而卒不能保社稷云。”

\subsubsection{弘光}

\begin{longtable}{|>{\centering\scriptsize}m{2em}|>{\centering\scriptsize}m{1.3em}|>{\centering}m{8.8em}|}
  % \caption{秦王政}\
  \toprule
  \SimHei \normalsize 年数 & \SimHei \scriptsize 公元 & \SimHei 大事件 \tabularnewline
  % \midrule
  \endfirsthead
  \toprule
  \SimHei \normalsize 年数 & \SimHei \scriptsize 公元 & \SimHei 大事件 \tabularnewline
  \midrule
  \endhead
  \midrule
  元年 & 1645 & \tabularnewline
  \bottomrule
\end{longtable}

\subsection{绍宗朱聿鍵\tiny(1645-1646)}

\subsubsection{生平}

明紹宗朱聿鍵(1602年5月25日-1646年10月6日),又稱隆武帝,小字長壽,南明第二代皇帝,原為唐王,為明太祖朱元璋二十三子唐王朱桱的八世孫(與明神宗同輩份),祖父唐端王朱碩熿,父為唐王之子朱器墭,母宣皇后毛氏。1644年,明思宗在北京自缢,1645年弘光帝被俘,鄭芝龍、黃道周等人扶朱聿鍵於福州登基称帝,改元為隆武並與同年開鑄「隆武通寶」,而弘光帝在翌年才被清廷所殺。

1646年,清军入福建,隆武帝在汀州被擄殺,享年44岁。永曆帝即位后初上尊谥思文皇帝,永历十一年上廟號紹宗,改谥号為配天至道弘毅肅穆思文烈武敏仁廣孝襄皇帝。朱聿键自奉甚俭,品格在南明诸君中是少見的優良。黄道周描述了隆武帝的为人:“今上不饮酒,精吏事,洞达古今,想亦高、光而下之所未见也。”

朱聿键为明太祖第二十三子唐定王朱桱的后裔,系太祖九世孙。万历三十四年四月丙申生于南阳唐王府,母妃毛氏。其祖父唐端王朱碩熿惑于嬖妾,不喜愛朱聿键的父親世子朱器墭,把朱器墭父子一起囚禁在承奉司內,欲立爱子。崇祯二年(1629年),朱器墭疑似被其弟福山王朱器塽、安陽王朱器埈毒死,朱碩熿讳言其事,但经守道陈奇瑜奏请,朱聿鍵被明廷立为唐國世孙,不再被囚禁,同年朱碩熿也去世。

崇禎五年(1632年)朱聿鍵繼為唐王,封地南阳。崇祯帝赐其《皇明祖训》、《大明会典》、《四书》、《五经》、《二十一史》、《資治通鑒綱目》、《孝经》、《忠經》等书。朱聿鍵在王府内起高明楼,延请四方名士。

崇祯九年(1636年)七月初一,朱聿鍵杖殺叔父福山王朱器塽、杖傷叔父安阳王朱器埈,为其父朱器墭当年被毒死一事报仇。当年八月,清兵入塞,克宝坻,直逼北京,京师戒严。朱聿鍵上疏请勤王,不许,乃自率护军千人北上勤王。行至裕州,巡抚杨绳武上奏,崇祯帝勒令其返回,后朱聿键因与农民军相遇交锋,两名太监被杀,乃班师回南阳。冬十一月下部议,废为庶人,幽禁在凤阳之高墙。崇禎帝改封其弟朱聿鏼为唐王。

朱聿键高墙圈禁期间,凤阳守陵太监石应诏索贿不得,用墩锁之法折磨之,朱聿键病苦几殆。后凤阳巡抚路振飞入高墙见之,向崇祯帝上疏陈高墙监吏凌虐宗室之状,请加恩于宗室。乃下旨誅殺石应诏。

崇禎十四年(1641年),李自成攻陷南阳,杀死朱聿鏼。

崇禎十七年(1644年),李自成攻陷北京,即甲申之變,崇祯帝自缢,南京諸臣拥从洛阳逃出的福王子朱由崧为帝,在南京即位,改年號弘光,实行大赦。在广昌伯刘良佐奏请下,囚於鳳陽的朱聿键也被释,并改封为南阳王。南京礼部请恢复唐王故爵,朱由崧不允,并令朱聿鍵迁至广西平乐(今桂林南),但朱聿鍵贫病不能行。

清朝順治二年、南明弘光元年(1645年)五月,朱聿鍵赴平乐途中,在苏州闻清军已破南京,俘虜了弘光帝朱由崧,朱聿鍵遂至嘉兴避难。六月辛酉,朱聿键至杭州,遇潞王朱常淓,奏请其监国,不听;请朝陈方略,不允。当时鎮江總兵官鄭鴻逵、戶部郎中蘇觀生至杭州,与朱聿键谈及国难,泣下沾襟。后朱聿键被郑鸿逵护送,前往福建。途中在浙江衢州闻得潞王朱常淓已在杭州降清,于是南安伯鄭芝龍、巡撫都御史張肯堂與禮部尚書黃道周等商议奉朱聿鍵为監國。

弘光元年六月己卯(二十八日),朱聿鍵在福建建宁,以唐王的身分监国。闰六月丁亥(初七)至福州,以南安伯府为行宫。

闰六月丁未,朱聿鍵於福州称帝,遙尊朱由崧為「上皇聖安皇帝」,宣布從七月初一起,改弘光年号為隆武元年,改福建布政司称福京行在,改福州府為天興府,改布政司为行殿,建行在太庙、社稷及唐国宗庙。升鄭芝龍为平虏侯、鄭鴻逵為定虏侯,封鄭芝豹为澄济伯、鄭彩為永胜伯。以何吾驺为首辅,以黄道周为吏部尚书、武英殿大学士,蒋德璟为户部尚书、文渊阁大学士,朱继祚为礼部尚书、东阁大学士,曾樱为工部尚书、东阁大学士,黄鸣俊、李光春、蘇觀生等人为礼、兵各部左右侍郎兼东阁大学士。

朱聿键即帝位后,上高曾祖父四代帝号,高祖唐敬王朱宇温为惠皇帝,曾祖唐顺王朱宙栐为顺皇帝,祖父唐端王朱碩熿为端皇帝,父唐裕王(追封)朱器墭为宣皇帝。四代祖妣皆追封皇后。封弟朱聿𨮁为唐王,封国南宁;升叔德安王朱器䵺为邓王;追封弟朱聿𨧨为陈王,子朱琳渼为陈王世子。遥上弘光帝尊号“圣安皇帝”。隆武元年七月下令将嘉靖年间皇极殿、中极殿、建极殿三殿之名恢复为奉天殿、华盖殿、谨身殿,各衙门前加“行在”二字。

当时,在绍兴还有鲁王朱以海建立的小朝廷,亦自稱「監國」。清军攻绍兴,朱以海派使者前来福州向朱聿鍵求援兵。信上称朱聿键为“皇伯父”,而未称“陛下”,朱聿键怒,令杀鲁王信使。

隆武二年/清顺治三年(1646年)五月,清将博洛贝勒率兵征浙、闽。七月庚申清兵陷金华,八月甲申陷建宁,乙未过仙霞关,武毅伯施天福、武功伯陈秀、靖安伯郭熺降清。郑芝龙向清軍投降,隆武政权很快灭亡。楊鳳苞稱“福京之亡,亡于鄭芝龍之通款”。

隆武二年八月甲午,隆武帝率宫嫔自延平出狩,欲逃往江西避難。八月庚申至汀州,以府署为行宫。八月辛丑五鼓,有清军八十三骑伪装成扈跸者叩城,守城者开汀州丽春门。骑兵突袭行宫,杀福清伯周之藩、总兵王凉武等人。时隆武帝腹饥,命内官市二汤圆以进,方举箸,清兵发矢,隆武帝后背中箭,崩,年四十五。百姓敛葬于罗汉岭。另有说法称隆武帝被俘后不食而死,或称崩于福京天兴府,或称崩于建宁。

八月壬戌福京天兴府陷落,阳曲王朱敏渡、松滋王朱俨𨫃、翼城王朱弘橺、奉新王朱常涟遇害。十月辛卯漳州陷落。十一月,侍郎蘇觀生立隆武帝之弟朱聿𨮁於廣東省廣州府番禺縣,改元紹武,觀生自為宰相。當時已經稱帝的永曆帝,希望紹武帝取消帝號,蘇觀生大怒,以新歸降的海盜加上四處捕捉來的民兵征討永曆,大勝,誰知滿清將領佟養甲、李成棟已取潮州、惠州,兵臨廣州,蘇觀生死於戰事,清兵隨即俘獲了紹武帝,紹武自縊。

永曆帝即位后,一直聽到謠言說隆武帝化妝隱居不出,上尊号「上皇思文皇帝」,遣間諜打聽隆武帝消息,傳言隆武帝潜至安溪縣妙峯为僧,或称在汀州府单骑逃出,藏于乡民蒋氏家中,清兵離開以後,前往大帽山出家。永历五年曾遣侍郎王命璿探訪,又不得,永历十一年乃确信隆武帝已死,立廟號绍宗,諡號配天至道弘毅肃穆思文烈武敏仁广孝襄皇帝。

隆武帝死后百姓敛葬于福州罗汉岭,一说葬于汀州。

\subsubsection{隆武}

\begin{longtable}{|>{\centering\scriptsize}m{2em}|>{\centering\scriptsize}m{1.3em}|>{\centering}m{8.8em}|}
  % \caption{秦王政}\
  \toprule
  \SimHei \normalsize 年数 & \SimHei \scriptsize 公元 & \SimHei 大事件 \tabularnewline
  % \midrule
  \endfirsthead
  \toprule
  \SimHei \normalsize 年数 & \SimHei \scriptsize 公元 & \SimHei 大事件 \tabularnewline
  \midrule
  \endhead
  \midrule
  元年 & 1645 & \tabularnewline\hline
  二年 & 1646 & \tabularnewline
  \bottomrule
\end{longtable}

\subsection{文宗朱聿{\fzk 𨮁}\tiny(1646-1647)}

\subsubsection{生平}

明紹武帝朱聿{\fzk 𨮁}(1605年-1647年1月20日),年號紹武。1646年—1647年在位,南明第三任君主。朱聿𨮁又稱小唐王,是明绍宗(唐王)之弟,明太祖二十三子唐定王朱桱的八世孙,祖父唐端王朱碩熿,父為唐王之子朱器墭。

明紹宗即位後封朱聿{\fzk 𨮁}為唐王,主祀唐國,幾天後紹宗出征,留他和邓王朱器䵺監國。

1646年(隆武二年),南明重臣郑芝龙拒不发兵,以致清軍隊长驱直入福京,並於长汀俘虜明紹宗,紹宗殉國,時為唐王的朱聿{\fzk 𨮁}和隆武朝的宫员逃到廣東省廣州府番禺縣,而其他南明勢力則在肇庆府推举明神宗之孙、明思宗堂弟桂王朱由榔为监国。同年十月十六日,江西赣州失守后,朱由榔政權大驚,于十月二十一仓皇从肇庆逃往广西梧州,置廣東全省於不顧。於是,大学士苏观生,在廣東權力真空與一眾明朝藩王已由海路到達广州的情況之下,聯同大学士何吾驺、广东布政使顾元镜,侍郎王应华、曾道唯等拥立朱聿{\fzk 𨮁}为监国,以都司署为行宫。隆武二年十一月五日,四十一歲的朱聿{\fzk 𨮁}按兄終弟及的皇明祖訓,繼位称帝,以明年为绍武元年。苏观生因拥戴有功,被命为首輔,封建明伯,掌兵部。由於朱聿{\fzk 𨮁}仓促稱帝,登極時的龍袍與百官官服都要假借于粵劇伶人的戏服。

十一月初八,紹武称帝的消息传到梧州,朱由榔政權大驚大怒,四日後回到肇庆,再於十八日登極稱帝,改元永曆,是為明昭宗。永曆帝立刻派遣兵科给事中彭耀、兵部主事陈嘉谟前往广州,拜见紹武帝,稱其為「殿下」,規勸其取消帝号。首輔苏观生大怒,以大不敬斬彭、陈二人,再令陈际泰督师攻打肇庆。永曆帝派兵部右侍郎林佳鼎、夏四敷率兵,在十一月二十九日於三水县城西,與紹武軍展開內戰,並將對方擊退。苏观生再令广东总兵林察聯同新降的海盗等数万人反擊,並且大敗永曆軍隊。大捷消息传到广州,苏观生下令广州张灯结彩粉饰太平。正当紹武、永曆二帝自相殘殺之時,由佟养甲、李成栋率领的清兵已取潮州、惠州,臨近广州附近,並用缴获的南明地方官印,向紹武帝发出太平的錯誤信息。

十二月十五日,绍武帝幸武学,百官聚集,而此時,清兵已经偷偷兵臨城下,内应脫去头上的伪装,露出辫子。有人向苏观生報告,反遭斬首。苏观生说:“潮州昨尚有报,安得遽至此。妄言惑众,斩之!”不久,清军壓境的戰況得到證實,苏观生遂率領部隊与清兵激战一晝夜,清兵本有撤退之意,但內奸谢尚政旋引清兵入城,广州即陷落。苏观生見大勢已去,写下“大明忠臣义固当死”八个大字后,自縊死亡。已易服的紹武帝,打算爬城墙逃走,但被追骑赶上抓获,囚于东察院。李成栋派人送来饮食,紹武帝拒絕,說:“我若饮汝一勺水,何以见先人地下!”後自缢而殉國,結束其四十日的統治。绍武朝的主要官員如何吾驺、王应华、顾元镜等降清,而广州內的二十四個明朝藩王則全數被殺。紹武帝死後,永曆帝成為南明唯一的皇帝。

後人將紹武、蘇觀生等十五人,葬於廣州城北象岗山北麓。1954年因基建,迁葬于越秀公园木壳岗;1981年再迁葬于公园南秀湖畔。墓坐东向西,封土呈覆竹形,正面竖墓碑,中刻“明绍武君臣冢”,上款为“光绪癸未(1883年)孟冬吉旦”,下款为“粤东绅士重修”。1963年3月广州市政府公布为市级文物保护单位。

\subsubsection{绍武}

\begin{longtable}{|>{\centering\scriptsize}m{2em}|>{\centering\scriptsize}m{1.3em}|>{\centering}m{8.8em}|}
  % \caption{秦王政}\
  \toprule
  \SimHei \normalsize 年数 & \SimHei \scriptsize 公元 & \SimHei 大事件 \tabularnewline
  % \midrule
  \endfirsthead
  \toprule
  \SimHei \normalsize 年数 & \SimHei \scriptsize 公元 & \SimHei 大事件 \tabularnewline
  \midrule
  \endhead
  \midrule
  元年 & 1646 & \tabularnewline
  \bottomrule
\end{longtable}


\subsection{昭宗朱由榔\tiny(1646-1662)}

\subsubsection{生平}

明昭宗朱由榔(1623年11月1日-1662年6月1日),或又稱永曆帝,南明第四位也是最後一位皇帝(1646年12月24日-1662年6月1日在位)。原為桂王。

1646年隆武帝被俘死,本為桂王的朱由榔自稱監國。不久,隆武帝弟唐王朱聿{\fzk 𨮁}在廣東廣州繼位,以次年為紹武元年,是為紹武帝。數日後,朱由榔在廣東肇庆亦登基稱帝,年號永曆。紹武、永曆二帝為爭正統,隨即開戰,後永曆軍大敗。1647年,清軍攻陷廣州,紹武帝兵敗殉國,永曆帝自此成為南明唯一的統治者。1659年,清军攻陷昆明后流亡缅甸東吁王朝,永曆十五年(1661年)夏历十二月初三日被送交吴三桂,永曆十六年四月十五日(1662年6月1日)遭縊死。死後,台灣的明鄭政權仍沿用永曆年號至1683年清朝佔領台灣為止。

朱由榔是明神宗之孙,明思宗堂弟,生於天啟三年(1623年)。崇禎年間封永明王,其父為桂端王朱常瀛,是明神宗第七子,封湖南衡阳,天启七年九月二十六日就藩,弘光元年(1645年)十一月初四日病死於梧州。第三子安仁王朱由𣜬承嗣。隆武帝称帝後不久病重。不久朱由榔被封桂王,在1646年隆武帝被俘後,於当年十月初十(一说十四日)称监国於廣東肇庆。

朱由榔於1646年(清顺治三年)農曆十一月十二日东返肇庆,十八日在肇庆正式稱帝,年号永曆,史称永曆帝。曾道唯、顾元镜、王应华等人都入阁,洪朝钟在十天之内升官三次。

永曆帝在中後期倚仗张献忠之餘部李定国、孙可望等人在西南一带抵抗满清,并且得到包括延平郡王郑成功在内的各反清力量的支持,是为反清的精神领袖和天下共主。1652年,李定国在桂林逼死定南王孔有德,又在衡州斩杀敬谨亲王尼堪,取得大捷,一度收复湖南西部、四川(除了保宁)、廣東(李成栋反正取得全部地区,后来仅保有沿海)、江西(金声桓、王得仁反正)等地。

1660年,清军攻入云南,永曆帝流亡缅甸東吁王朝首都瓦城,獲國王莽達(平達力)收留。後来,吴三桂攻入缅甸,莽達之弟莽白乘机发动兵变,杀死其兄奪位。1661年8月12日,莽白發動咒水之难,杀盡永曆帝侍從近衛。

永历帝得到清军进入缅境的消息后,曾寫信给吴三桂,到1662年1月22日(永历十五年十二月初三),莽白将永曆帝献给吴三桂,南明灭亡。

1662年6月1日(永历十六年四月十五望日,清康熙元年),永曆帝父子及眷属25人在昆明篦子坡遭弓弦勒死,终年40岁。其身亡處時人稱為逼死坡,即今天的昆明市五华区的华山西路,辛亥革命後蔡鍔等人在當地豎立「明永曆帝殉國處」石碑。死后庙号昭宗,谥号應天推道敏毅恭儉經文緯武體仁克孝匡皇帝。

至今未发现永历帝之墓。仅贵州都匀大坪镇有永历帝的衣冠冢。当地扶姓人家说,是他们先人明朝大学士扶纲派人搜集衣冠而葬的,为隐其真,只传是桂王坟,不留碑记。扶纲是因明亡不愿降清而回乡隐居的。帝墓左边是编修涂宏猷的 髮冢,右边是节愍侯邬昌期的衣带冢。民国十年都匀县奉令修史,查实桂王坟乃永历墓,才为其树碑立传,省长任可澄、省志总 陈炬、知县窦全曾都为之写了碑记,碑文“大明永历皇帝陵”几个字,墓碑及碑记是时任四川綦江县县长张瑞徵写的(张系都匀人),还修了些亭阁楹联,帝墓才初显规模。墓高3米、径6米,碑高1.62米,宽0.81米、厚0.13米,碑字阴刻正楷,字笔工整秀丽。涂宏猷和邬昌期二人,是咒水之难42大臣之二,坟比帝坟小得多。“文革”中被盗,帝坟从前到后挖了一个大坑,碑断为两截仰卧坟前土中。1996年都匀市人民政府公布大明永历皇帝陵为市级文物保护单位,着手修复帝陵。坟用青石砌边,水泥勾缝,碑文由书法家芦如平书写,前边加修了上下山的双向百级石阶,供游人参观。

\subsubsection{永历}
\begin{longtable}{|>{\centering\scriptsize}m{2em}|>{\centering\scriptsize}m{1.3em}|>{\centering}m{8.8em}|}
  % \caption{秦王政}\
  \toprule
  \SimHei \normalsize 年数 & \SimHei \scriptsize 公元 & \SimHei 大事件 \tabularnewline
  % \midrule
  \endfirsthead
  \toprule
  \SimHei \normalsize 年数 & \SimHei \scriptsize 公元 & \SimHei 大事件 \tabularnewline
  \midrule
  \endhead
  \midrule
  元年 & 1647 & \tabularnewline\hline
  二年 & 1648 & \tabularnewline\hline
  三年 & 1649 & \tabularnewline\hline
  四年 & 1650 & \tabularnewline\hline
  五年 & 1651 & \tabularnewline\hline
  六年 & 1652 & \tabularnewline\hline
  七年 & 1653 & \tabularnewline\hline
  八年 & 1654 & \tabularnewline\hline
  九年 & 1655 & \tabularnewline\hline
  十年 & 1656 & \tabularnewline\hline
  十一年 & 1657 & \tabularnewline\hline
  十二年 & 1658 & \tabularnewline\hline
  十三年 & 1659 & \tabularnewline\hline
  十四年 & 1660 & \tabularnewline\hline
  十五年 & 1661 & \tabularnewline\hline
  十六年 & 1662 & \tabularnewline\hline
  十七年 & 1663 & \tabularnewline\hline
  十八年 & 1664 & \tabularnewline\hline
  十九年 & 1665 & \tabularnewline\hline
  二十年 & 1666 & \tabularnewline\hline
  二一年 & 1667 & \tabularnewline\hline
  二二年 & 1668 & \tabularnewline\hline
  二三年 & 1669 & \tabularnewline\hline
  二四年 & 1670 & \tabularnewline\hline
  二五年 & 1671 & \tabularnewline\hline
  二六年 & 1672 & \tabularnewline\hline
  二七年 & 1673 & \tabularnewline\hline
  二八年 & 1674 & \tabularnewline\hline
  二九年 & 1675 & \tabularnewline\hline
  三十年 & 1676 & \tabularnewline\hline
  三一年 & 1677 & \tabularnewline\hline
  三二年 & 1678 & \tabularnewline\hline
  三三年 & 1679 & \tabularnewline\hline
  三四年 & 1680 & \tabularnewline\hline
  三五年 & 1681 & \tabularnewline\hline
  三六年 & 1682 & \tabularnewline\hline
  三七年 & 1683 & \tabularnewline
  \bottomrule
\end{longtable}


%%% Local Variables:
%%% mode: latex
%%% TeX-engine: xetex
%%% TeX-master: "../Main"
%%% End:



%%% Local Variables:
%%% mode: latex
%%% TeX-engine: xetex
%%% TeX-master: "../Main"
%%% End:
