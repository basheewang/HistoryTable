%% -*- coding: utf-8 -*-
%% Time-stamp: <Chen Wang: 2019-12-26 15:07:16>

\section{孝宗\tiny(1487-1505)}

\subsection{生平}

明孝宗朱祐樘(1470年7月30日-1505年6月9日),或稱弘治帝,是明宪宗皇三子。明朝第10代皇帝(1487年-1505年在位),他在位18年,年号弘治。孝宗“恭俭有制,勤政爱民”,又能信用贤臣、广开言路,在位期间“朝序清宁,民物康阜”,明朝出现中兴局面,史称“弘治中兴”。但在位后期對朝政有所懈怠,又縱容外戚,沉迷方術,使宦官李广、蒋琮等人乘机弄权,以致弘治晚年軍備弛廢,國用匱乏,弊政颇多,故不能谓之全美。明孝宗崩逝後谥号「建天明道诚纯中正圣文神武至仁大德敬皇帝」,庙号「孝宗」,葬于泰陵。

根据《明史》记载:“孝宗达(实为“建”,《明史》误)天明道纯诚中正圣文神武至仁大德敬皇帝,讳祐樘,宪宗第三子也。母淑妃纪氏,大明成化六年七月生帝于西宫。时万贵妃专宠,宫中莫敢言。悼恭太子薨后,宪宗始知之,育周太后宫中。十一年,敕礼部命名,大学士商辂等因以建储请。是年六月,淑妃暴薨,帝年六岁,哀慕如成人。十一月,立为皇太子。”民間則傳說:孝宗出生时,为免被当时的寵妃萬貴妃害死而藏在民間,在憲宗死前才由宮內太監於民間迎回即位。

孝宗出生後,廢后吳氏貶居西內,與紀氏謫居的安樂堂相近,頗知消息,往來就哺,才得保全孝宗生命,由吳氏用心撫養過一段日子。

弘治帝在位初期,励精图治、整肃朝纲、改革弊政,罢逐了朝中奸佞之臣、重用贤士,为于谦建祠平冤,减轻赋税、停征徭役、兴修水利、发展农业、繁荣经济,史稱“弘治中兴”。

弘治帝在位期间“更新庶政,言路大开”,启用了刘健、丘濬、李东阳、谢迁、王恕、马文升、刘大夏等能臣,使明憲宗成化朝晚年以来,奸佞当道的局面,得以大为改观。

此外,弘治帝重視司法,他令天下諸司審錄重囚,慎重處理刑事案件。弘治十三年(1500年),制定《問刑條例》。又於弘治十五年(1502年),編成《大明會典》。

弘治帝在治理水患方面亦頗有效果,曾委任白昂、劉大夏修治黃河,以改善河道流向、築堤等方法抑制黃河水患,此後二十餘年間,再無大患發生;另外,蘇松於弘治年間,曾因河道淤塞而泛濫成災,孝宗即命徐貫主持治理,歷時三年,消除了蘇松水患。

弘治帝在位初期的經濟成就也比較突出,賦稅收入比成化年間增加了一百多萬石,達二千七百萬石,成為明中葉的賦入高峰;而且,人口方面也有穩定的增長。從弘治元年(1488年)到弘治十七年(1504年)間,人口增加了一千多萬,達到六千萬口。

惟自弘治十五年起(1502年),「一歲所入,不足以供一歲支用」,國家財政邁進了入不敷出的狀況,戶部呂鈡指出:『常入之賦,以蠲色漸減,常出之費,以請乞漸增,入不足當出。正純以前軍國費省,小民輸正賦而已。自景泰至今,用度雜辦,皆昔所無。民已重困,無可復增。往時四方豐登,邊境無調發,州縣無流移。今太倉無儲,內府殫絀,而冗食冗費日加於前。』對此下廷臣議,廷臣作出多項建議,但僅觸及成效不大的修補政策。

此外,孝宗也常以京營禁軍投入繁重的工作,監察御史劉芳曾上奏說,“京師根本之地而軍士逃亡者過半”,“其錦衣騰驤等衛軍士不下十餘萬人,又不繫操練之數,近年雖立營營,而役佔賣放者多。”,另外又常縱容邊臣,邊臣冒報功次皆得升賞,而敗軍失律者往往令之戴罪殺賊,使邊備日弛,對於北虜入侵能有效抵禦的戰役寥寥無幾,如弘治十四年秋七月,孝宗令保國公掛征虜大將軍總兵官領十萬大軍夜襲韃靼於河套,韃靼早察覺徙家北遁,朝廷用銀八十餘萬,只斬首三級以還,而將士奏報功次竟一萬有餘,“不能禦”,“坐虜入境”,“議者恥之”之類的描述比比皆是。

再者,弘治中期,皇帝自己漸漸迷上了齋醮,從此內庫開銷劇增,孝宗開始不斷地命戶部將太倉庫的銀子納入內庫,至將河西務鈔關關船料改擬折銀進納。如弘治十五年(1502年)十月,戶部指出“銀承備庫先前進,金止備成造金冊支用;銀止備軍官折俸及兵荒支給,近年累稱不足。金則以稅糧折納及於京市買過八千三百八十六兩有奇,五次取太倉銀共一百九十五萬,”而從戶部納入內庫的銀兩,全部都被孝宗挪用來大興土木,又妝造武當山神像,各寺觀修齋賞賜,修齋設醮等,恣意浪費,以致府藏空竭,國庫捉襟见肘。而且孝宗在統治中期(1500年)後,漸漸不如當初勤政,且開始縱容外戚,措置乖方,如內閣輔臣劉健,徐溥就曾批評孝宗說「切見數月以來視朝漸遲多至日出」,「近年以来用度太侈,光禄寺支费增数十倍,各处织造降出新样动千百匹,显灵朝天等宫泰山武当等处修斋设醮费用累千万两,太仓官银存积无几,不勾给边而取入内府至四五十万,宗藩贵戚求讨田土占夺盐利动亦数十万。」,「事涉於近幸貴戚,牢不可破,或旨從中出,略不預聞,或有所議擬,徑行改易。」,而閣臣李東陽也曾直言弘治後期「冗食太眾,國用無經,差役頻煩,科派重疊。京城土木繁興,供役軍士財力交殫,每遇班操,寧死不赴;勢家巨族,田連郡縣,猶請乞不已。親王之藩,供億至二三十萬。」「天津一路,夏麥已枯,秋禾未種,挽舟者無完衣,荷鋤者有菜色。盜賊縱橫,青州尤甚。南來人言,江南、浙東流亡載道,戶口消耗,軍伍空虛,庫無旬日之儲,官缺累歲之俸。」「今天下民窮財盡,其勢已極。姑以三者言之,山東之地草根樹皮掘食殆盡,繼以人肉,荊沔諸湖水竭魚荒,河泊諸課率多折納,易州山廠林木已空,漸出關外一二百里,其他賦稅大抵皆然,天下之地無一處而不貧」。朝中大臣如禮部尚書倪岳也上疏極言道「(孝宗)近日視朝頗晏,聽納頗難,經筵稀,御用度漸侈,游幸漸頻,進貢之止者複來,樂戲之斥者複取。」但孝宗也不願意聽納,而名臣劉大夏請辭時也言「臣老且病,窃见天下民穷财尽,脱有不虞,责在兵部,自度力不办,故辞耳。」,而吏部右侍郎周經則言「(孝宗)幸賞齋醮屢修,游宴無節,內帑空虗多由於此。」,南京戶科給事中張宦也上書道「近來(孝宗)費出無經,或橫恩濫賜之溢出,或修飾繕造之泛興,或祈禱遊玩之紛舉,偶因內帑稍闕即命太倉支取,耗散財物莫此為極」 「今四海民窮財盡,三邊將寡兵疲,糧草空虗,馬匹倒死而黠虜跳粱之勢,貪狼之心視昔尤勝」,禮科左給事中葉紳也言「邇來(孝宗)經筵稀御日講不舉,畫工琴士承恩於便殿,教坊雜劇呈技於左右..少滯視朝時,晏鰲山觀燈或徹曉不休宮中燕享或竟日乃已。」,兵科给事中王廷相奏「今天下大可忧者,在于民穷财尽,其势渐不可为。然所以致此者有四,风俗奢侈也,官职冗滥也,征赋太繁也,酒酿无节也」。可見弘治中晚年皇帝倦勤,國家敗政拮据,百姓困苦的情況。

在統治的十八年中,召見閣臣的次數總共有九次,比成化帝二十三年來召見一次為多。明孝宗即位之初,會聽進閣臣的諫諍,但是後來用各種方法來搪塞閣臣和科道官的建議,使弘治初年所革除的弊政,不僅全部恢復,尚且有惡化之勢,如憲宗晚年的傳奉官號稱弊政,弘治初盡行革除,到了弘治十二(1499年)年五月,傳升乞升文職至八百四十餘員,武職至二百六十餘員,比成化末年增一倍。其次,在軍事方面,從弘治一朝起亦開始糜爛,邊備日弛,人浮於事,有效抵禦的入侵寥寥無幾,也不復當年成化一朝了。另外,有明一代,以弘治對外臣最為縱容厚待,動則大肆外戚藩王賞賜房屋和田地,甚至在一宗貴戚莊崎糾紛案中,偏幫小舅子張延齡,一次就得地一萬六千七百零五頃;又如曾在弘治十三年(1500年)二月,賜興王湖廣京山縣近湖淤地一千三百五十餘頃,旋在七月又賜岐王德安府田六百一十二頃等等,賞地史不絕書,引起嚴重的土地兼併問題。

弘治十八年(1505年)五月初七日,因偶染风寒,误服药物,鼻血不止而死,

當時“深山穷谷,闻之无不哀痛”。有遗命:“东宫年幼,好逸乐,先生辈善辅之。”是年十月葬於泰陵。長子明武宗繼位。

孝宗即位时所面临的政治局面混乱不堪,由于他父亲明宪宗在位后期重用宦官和奸佞,造成了“朝中皮秕政”的状况。为了振兴帝业,肃清吏治,他在人事上的改革和整顿,可謂大刀阔斧。对太监梁芳、礼部右侍郎李孜省等前朝奸佞惩罚严厉。将冒领官俸、总计三千多人的艺人、僧徒等一概除名。在清理过程中,朱祐樘注意方式、方法,没有大开杀戒,斬殺的只有罪大恶极的僧人继晓 。与此并举,孝宗开始任用贤能之士。1492年三月,孝宗下令吏、兵两部将两京文武大臣、在外知府守备以上的官吏姓名,全部抄录下来,贴在文华殿的墙壁上,遇有迁罢之人,随时更改。他还多次向吏部、都察院指出,提拔和罢免官吏的主要标准,是看此人有無实绩。由于孝宗注意任用贤能,明朝中期出现了许多名臣,形成了“朝多君子”的盛况。

朱祐樘即位初年,广开言路。上台不久,就出现了臣子纷纷上书的局面,连尚未做官的太学生也跃跃欲试,上书提出各种建议。孝宗也有奢侈的想法,于是计划在万寿山建造一座棕棚,以备登临眺望。太学生虎臣得知此事,力谏不可,负责这项工程的朝中官员担心獲罪,抓住虎臣。孝宗闻知此事,先取消了工程,且授予虎臣七品官,派往云南做了知县。孝宗还采纳了除早朝之外,再在便殿召见大臣,谋议政事,当面阅读奏章,下发指令的建议,开始增加“午朝”,每天在左顺门接见大臣,倾听他们对政事的见解。

有說法認為:孝宗统治期间所实行的一系列的政策,都自始至终地得以贯彻执行,然而有學者指出,在弘治十四年,孝宗因朝廷財政拮據,以及軍餉籌措有困難而下詔群臣商議辦法,大學士劉健上奏要求改革弊端,並絕無益之費,躬行節儉,孝宗卻未採取措施。至弘治十五年,國家財政入不敷出:「常入之賦,以蠲色漸減,常出之費,以請乞漸增,入不足當出。正純以前軍國費省,小民輸正賦而已。自景泰至今,用度雜辦,皆昔所無。民已重困,無可復增。往時四方豐登,邊境無調發,州縣無流移。今太倉無儲,內府殫絀,而冗食冗費日加於前。」但僅作出成效不大的修補政策。

1489年,内阁大臣刘吉数兴大狱,迫害了一批官员;信任太监李广,开始修炼斋蘸之术。孝宗對此自我检讨。

據美國牙醫學會的資料表示,明孝宗於1498年把短硬的豬猔毛插進一支骨製手把上成為牙刷。

1501年,崛起的鞑靼部落以十万骑兵从花马池、盐池杀入固原、宁夏境内,这一事件震惊了孝宗。为了加强军事力量,1502年,孝宗将刘大夏提升为兵部尚书,负责军事整顿。刘大夏核查了军队虚额人手,补进了大量壮丁,并请朱祐樘停办了不少“织造”和斋蘸。

作为改良,孝宗没有从制度上对百姓的税赋负担进行突出的改变,而在减轻百姓负担上,减免灾区的赋税征收。从1490年,河南因灾免秋粮始,他对每年奏报来的因灾免税要求,几乎是无一例外地表示同意。

清修《明史》高度评价明孝宗:明有天下,传世十六,太祖、成祖而外,可称者仁宗、宣宗、孝宗而已。仁、宣之际,国势初张,纲纪修立,淳朴未漓。至成化以来,号为太平无事,而晏安则易耽怠玩,富盛则渐启骄奢。孝宗独能恭俭有制,勤政爱民,兢兢于保泰持盈之道,用使朝序清宁,民物康阜。《易》曰:“无平不陂,无往不复,艰贞无咎。”知此道者,其惟孝宗乎!

《国榷》:孝宗在东宫,久稔知其习。首罢幸相,次第厘革,改步之初,中外鼓舞,晓然诵明圣,识上意所向也。优容言路,汇吁良士,六卿之长皆民誉,三事之登皆儒英。讲幄平台,天听日卑,老臣造膝之语,不漏属垣,少年恸哭之谈,尝为动色。故良楛鉴断,刑赏恬肃。虽寿宁之戚,天下艳之,然宠若窦宪,尚难泌水之园,骄即武安,未请考工之宅,则帝心端可知矣。

方志远在其著作《明代国家权力机构及运行机制》中对明孝宗持否定态度,称其“弱智”并详细解释道:“弘治时代夹在成化、正德之间,前有万贵妃、汪直与西厂,后有刘瑾、八虎及内行厂,加之成化帝的内向和正德帝的荒唐,故弘治帝被明人称为‘中兴之主’。清人作《明史·孝宗纪》,其赞曰:‘明有天下,传世十六,太祖、成祖而外,可称者仁宗、宣宗、孝宗而已。仁、宣之际,国势初张,纲纪修立,淳朴未漓。至成化以来,号为太平无事,而晏安则易耽怠玩,富盛则渐启骄奢。孝宗独能恭俭有制,勤政爱民,兢兢于保泰持盈之道,用使朝序清宁,民物康阜。’并称唯有孝宗知《易》所说的‘无平不陂,无往不复,艰贞无咎’之道。但黄仁宇在《万历十五年》中指出,孝宗之为文臣所称道,就是因为他比较愿意听文臣的摆布。而实际上,孝宗不仅为文臣摆布,更受内臣摆布,从其种种行事,应该是个智商较低或者说是一个相对弱智的皇帝。”方志远在书中表示将‘另具文考证’,但相关文章尚未问世,因此,关于这个评价也存在一定争议。

郭厚安在其著作《弘治皇帝大传》中称明孝宗“盛名之下,其实难副”。他表示“从总体上,他(明孝宗)比其祖父英宗、其父宪宗以及其子武宗、侄世宗等都要略高一筹,坏的方面也没有他们突出。因此可以说,他之所以受到赞颂,是与前后诸帝比较的结果”;“朱祐樘不过是一个‘中主’而已”;“总之,朱祐樘绝不是雄才大略、大有作为之君,当然也不是荒淫的昏君,而是平庸的、力求维持现状的‘太平天子’。”

查继佐的《罪惟录》中,对明孝宗的成就和不足如此评价:“帝业几于光昌矣。群贤辐辏,任用得宜,暖阁商量,尤堪口法。斥妖淫,辟冗异,停采献,罢传升,革仓差,正抽分,种种明断外,尤莫难于孝穆、孝肃之别祀,万贵妃之免议,于肃愍之旌功。所谓情而安之于义,又列辟之所不能忘也。升遐之日,万姓哀号,岂偶然哉!若夫待外戚过厚,赐予颇滥,冗员尚多,中贵太盛,或移心斋醮,纷费,盖积渐者久,未能遽革也。夫果深有得于《太极》、《西铭》诸图书,即何难骑龙而上仙哉!”查继佐尽管也为弘治辩解,但与上述史家不同的是,究竟委婉地指出了明孝宗的不足。

《朝鮮成宗實錄》上(朝鮮成宗)曰:“常慮建州野人邀截於中路,今卿好還,甚可喜也。中國太平乎?” 自貞曰:“太平。但聞皇帝不豫,朝會望見,天顔殊瘦,皇帝初卽位,皆稱明斷,今紀綱不嚴,雨暘不若,年穀不登,民甚困窮。向者朝會,朝臣各以位次序立,莫敢私語,今則或聚立私語,以此知紀綱不嚴也。”

《明朝時代上卷 第42章 弘治王朝的老生常談》:“後世史學家多將弘治王朝稱作“弘治中興”,但從更寬廣的歷史視野來看,這些其實都經不住推敲,從宣德王朝開始,文人們所認為的明朝衰敗,實際上並不存在。皇帝不臨朝、宦官跋扈、軍屯被破壞、京畿部分民田被侵占,這些在士大夫看起來,好像不可理喻的事情,實際上無關這個大明王朝的痛癢,正統、成化年間,我們的大明王朝仍舊是平穩、正常運行的,不僅如此,從中可以看出三個趨勢,那就是政治依賴日益成熟、穩定的官僚集團運作,商業貿易開始興起,哲學文化思想領域開始鬆動,這都是值得正面看待的事情。大歷史觀,對於歷史的觀察,不應該再是只從《是否符合儒家行為規範》來看待,如果繼續這樣看待歷史,就會使我們中國人陷入一種狹隘束縛的歷史發展桎梏中。 仔細分析正統、成化王朝的所謂衰敗,是因為史學家們以當時的君主統治行為,不符合儒家行為規範而已,而弘治王朝的所謂中興,也是因為弘治皇帝遵循了文人士大夫們的儒家王道意識,因為前朝感覺衰敗,才會存在後來的感覺中興。以弘治皇帝努力將自己塑造成一個仁君形象,這是值得嘉許的。但最後這些都無濟於事,皇帝的人性與權力,超越了士大夫們的儒家王道意識與封建禮法,這衝突使一代明君轉眼變為昏君,史學評論家立即改觀,對弘治王朝前面與後面的一個總結,就是不完美。”

有人根據清修《明史》、《明書》等資料記載,認為孝宗僅娶妻孝康敬皇后張氏一人,沒有其他妃嬪或妾室。並且孝宗的泰陵只葬有夫妻兩人。而實際上根據《勝朝彤史拾遺記》及《罪惟錄》所載,孝宗至少還有沈璚蓮、鄭金蓮(《罪惟錄》稱其小字黃兒)兩位選侍。因為各種史書中對於妃嬪傳記因有事跡可記、有立傳價值,取捨各有不同,參見《萬曆官修本朝正史研究》中「八種史書關於明太祖等十位皇帝后妃立傳情況表」。而大部分的妃嬪因為地位的關係都不能葬入明帝陵中。

至於孝宗宫中有五名夫人:敬順夫人邵氏,安和夫人周氏,安順夫人劉氏,榮順夫人孟氏及榮善夫人項氏。夫人在明朝制度並非妃嬪稱號,而是命婦的封號,如外命婦(公侯伯及一二品官正室)或內命婦(資深宮人或乳母褓姆)等,內命婦中,以皇帝的乳母最常在年老後因乳帝之功而被加封為夫人(如明孝宗的保姆封为佐圣夫人、天启帝的乳母奉圣夫人客氏、仁宗褓姆衛聖夫人楊氏等,皆是有夫有家的妇人)。另,榮善夫人項氏年龄比孝宗大四十四岁,比孝宗的祖父明英宗还大一岁。因此这五名夫人实际上不是明孝宗的妃嫔。

\subsection{弘治}

\begin{longtable}{|>{\centering\scriptsize}m{2em}|>{\centering\scriptsize}m{1.3em}|>{\centering}m{8.8em}|}
  % \caption{秦王政}\
  \toprule
  \SimHei \normalsize 年数 & \SimHei \scriptsize 公元 & \SimHei 大事件 \tabularnewline
  % \midrule
  \endfirsthead
  \toprule
  \SimHei \normalsize 年数 & \SimHei \scriptsize 公元 & \SimHei 大事件 \tabularnewline
  \midrule
  \endhead
  \midrule
  元年 & 1488 & \tabularnewline\hline
  二年 & 1489 & \tabularnewline\hline
  三年 & 1490 & \tabularnewline\hline
  四年 & 1491 & \tabularnewline\hline
  五年 & 1492 & \tabularnewline\hline
  六年 & 1493 & \tabularnewline\hline
  七年 & 1494 & \tabularnewline\hline
  八年 & 1495 & \tabularnewline\hline
  九年 & 1496 & \tabularnewline\hline
  十年 & 1497 & \tabularnewline\hline
  十一年 & 1498 & \tabularnewline\hline
  十二年 & 1499 & \tabularnewline\hline
  十三年 & 1500 & \tabularnewline\hline
  十四年 & 1501 & \tabularnewline\hline
  十五年 & 1502 & \tabularnewline\hline
  十六年 & 1503 & \tabularnewline\hline
  十七年 & 1504 & \tabularnewline\hline
  十八年 & 1505 & \tabularnewline
  \bottomrule
\end{longtable}


%%% Local Variables:
%%% mode: latex
%%% TeX-engine: xetex
%%% TeX-master: "../Main"
%%% End:
