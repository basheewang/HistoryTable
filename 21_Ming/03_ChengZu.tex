%% -*- coding: utf-8 -*-
%% Time-stamp: <Chen Wang: 2019-12-26 15:06:36>

\section{成祖\tiny(1402-1424)}

\subsection{生平}

明成祖朱棣(1360年5月2日-1424年8月12日),或稱永樂帝,是明朝第三代皇帝,公元1402年至1424年在位,在位二十二年,年号永乐。

明太祖皇四子,安徽凤阳人,生于应天府(今江苏南京),時事征伐,並受封為燕王。洪武三十二年或建文元年(1399年)建文帝削藩,燕王遂發動靖难之役,起兵奪位,經過三年的战争,最終胜利,殺害方孝孺,驅逐其姪建文帝奪權篡位自封為帝。明成祖在位期间,改善明朝政治制度,发展经济,开拓疆域,迁都北京,使北京至此成為中國的政治中心至今。此外他编修《永乐大典》,派遣鄭和下西洋,北征蒙古,南平安南。明成祖的统治时期被称为永乐盛世,明成祖也被后世称为「永乐大帝」。另外,他加強太祖以來的專制統治,強化錦衣衛並成立東廠,此外,他在位期間重用宦官,也促成明朝中葉後宦官專政的禍根。

明成祖崩逝后谥号「体天弘道高明广运圣武神功纯仁至孝文皇帝」,庙号「太宗」,葬于长陵。嘉靖十七年(1538)九月,嘉靖帝改谥为「启天弘道高明肇运圣武神功纯仁至孝文皇帝」,改上庙号为「成祖」。

(1360年)四月十七日(5月2日),朱棣生于应天府(今南京)。

明太祖洪武三年(1370年),朱棣十岁,受封燕王。曾居鳳陽,对民情颇有所知。洪武十三年(1380年),朱棣就藩燕京北平,之后多次受命参与北方军事活动,两次率师北征,曾招降蒙古乃兒不花,並曾生擒北元大將索林帖木兒,加强了他在北方军队中的影响。朱元璋晚年,長子太子朱标、次子秦王朱樉、三子晋王朱棡皆早於朱元璋去世,故朱元璋於洪武三十一年閒五月駕崩後,四子朱棣不仅在军事实力上,而且在家族尊序上都成为诸王之首。

建文帝朱允炆登基後,為了提防燕王造反,於洪武三十一年十二月派工部侍郎張昺為北平布政使,都指揮使謝貴、張信為北平都指揮使。隨後又命都督宋忠屯兵駐開平,并調走北平原屬燕王管轄的軍隊。

建文元年(1399年),朱棣裝病,使建文帝把作為人質的朱棣三子朱高熾、朱高煦、朱高燧回燕國;之後由於屬下被朝廷處死,遂裝瘋。由於王府長史葛誠告知朝廷,裝瘋被發覺。

時燕王遣使入金陵奏事,使者被齊泰等審訊,被迫供出燕王的異狀,於是朝廷下密旨,令張昺、謝貴逮捕燕王府的官屬,張信逮捕燕王本人。但張信經過考慮,將此事告知朱棣。於是朱棣和僧人姚道衍等進行舉兵的謀劃,令張玉、朱能將八百勇士帶入府中潛伏,以待變故。

張昺、謝貴得到皇帝密詔后,七月初四帶兵包圍了燕王府。朱棣假意將官屬全部捆縛,請二人進王府查驗。二人進府后,朱棣派出府內的死士將其擒獲,并連同府內叛變的葛誠、盧振一同斬殺。當日夜裡,朱棣攻下北平九門,遂控制北平城。

燕王朱棣起兵,援引《皇明祖訓》,號稱清君側,指惠帝身邊的齊泰和黃子澄為奸臣(謀害皇室親族),需要鏟除,稱自己的舉動為「靖難」(意為「平定災難」),并上書於惠帝朱允炆。

燕軍控制北平后,七月初六,通州主動歸附;七月初八,攻破薊州,遵化、密雲歸附;七月十一,攻破居庸關;七月十六,攻破懷來,擒殺宋忠等;七月十八,永平府(今河北盧龍縣,屬秦皇島市)歸附。七月二十七,為防止大寧軍隊從松亭關偷襲北平,用反間計使松亭關內訌,守將卜萬下獄。至此,北平周圍全部掃清。燕軍兵力增至數萬。

燕軍攻破懷來後,由於領地相距太近,七月二十四日,谷王朱橞逃離封地宣府(今屬張家口,距北京約150公里,距懷來約60公里),奔京師。八月,齊泰等顧慮遼王、寧王幫助燕王,建議召還京師;遼王從海路返京,而寧王不從,遂削寧王護衛。宋忠失敗後,部將陳質退守大同。代王本欲起兵呼應朱棣,被陳質所控制,未果。

七月,朱棣反書到京,朱允炆削朱棣屬籍,廢為庶人。決定起兵討燕。在真定(今河北正定)置平燕布政使司。

耿炳文率軍在八月十三日到達真定,并分兵於河間、鄚州(河北任丘北约30里)、雄縣,為犄角之勢。在經過觀察後,八月十五日,燕軍趁中秋夜敵軍不備,偷襲雄縣;成功後又利用伏擊擊敗了鄚州的援兵,遂攻克鄚州,收編剩餘的部隊。八月二十四日,燕軍到達無極縣。從樵夫和中央軍被俘士兵處得知敵情,於是燕軍發動決戰。

二十五日,燕軍趁耿炳文送使臣出城時偷襲中央軍,炳文逃回城中后,怒而迎戰。在燕軍主力與耿炳文軍相持時,朱棣親自率軍襲擊其側翼,耿炳文大敗潰逃,中央軍投降三千多人。中央軍狼狽逃回城中,城池差點失守。部將李堅、甯忠、顧成等被俘;士兵被殺、被俘數萬人(后放還)。耿炳文率殘部不到十萬人在真定堅守不出,燕軍攻城三天不克。八月二十九日,燕軍返回北平。顧成降燕之後,留在北平協助燕世子朱高熾守城。

耿炳文戰敗,朱允炆開始擔憂戰事,考慮換將。黃子澄說曹國公李景隆是名將李文忠之子,建議他接任;齊泰反對,但惠帝不聽。八月三十日,拜李景隆為大將軍,誓師出征,并召回耿炳文。李景隆以德州為大本營,調集各路兵馬包括耿炳文敗兵,增兵至五十萬人,九月十一日進至河間。

朱棣聽說朝廷以五十萬傾國之兵交付李景隆,大喜過望,說:「李景隆不會用兵,給他五十萬大軍,根本是自取滅亡。趙括之失必然重演,我軍必勝。」

九月初一江陰侯吳高率辽东兵攻打永平郡,九月廿五,攻陷永平郡,決定趁勢偷襲大寧(今內蒙古寧城)以獲得其精銳部隊;另一方面利而誘之,將中央軍引至「空城」北平下。九月廿八,出師。。十月初六,燕軍經小路到達大寧城下。朱棣單騎入城),見寧王朱權,向朱權求救。在居大寧期間,朱棣令手下吏士入城結交并賄賂大寧的軍官等。十月十三,朱棣提出告辭,朱權在郊外送行,伏兵盡起,大寧軍紛紛叛變,歸附朱棣。於是朱權與王妃、世子等一同隨朱棣前往北平,而大寧的全部軍隊(包括其騎兵精銳朵顏三衛)都被朱棣收編。大寧成為空城。朱棣實力大增。十月十九,燕軍在會州整編,分立五軍(中前左右後)。十月廿一,入松亭關。

十一月初五,渡白河(時已結冰,渡河處在今北京順義區東),打敗李景隆的哨探陳暉部隊萬餘人。李景隆大敗。李景隆令鄭村壩所有軍隊輕裝撤退。。燕軍輕易擊潰城下的敵軍,獲得大量物資。。此戰中央軍喪師十餘萬。十一月初九,朱棣回到北平城,再次上書,惠帝不應。。十二月十九日,朱棣出師攻打大同。十二月廿四,抵達廣昌,守將楊宗投降。建文二年(1400年)正月初一,燕軍抵達蔚州,守將王忠、李遠投降。二月初二,燕軍攻大同。李景隆前来救援。李景隆走出紫荊關后,燕軍從居庸關返回北平。中央軍兵力、裝備大量損失,士氣受到重創。

建文二年四月,李景隆從德州,郭英、吳傑等從真定誓師北伐兵力增至六十萬。燕軍亦出。四月二十日,燕軍渡過玉馬河。四月廿四,燕軍戰鬥失利。。次日(四月廿五),再次交战。。。四月廿七,燕軍進攻德州。初九,燕軍進入德州。五月十五,燕軍攻濟南,李景隆逃走。燕軍遂圍濟南。十月,朝廷召李景隆回南京。黃子澄、練子寧、葉希賢等上書,請求立斬李景隆。朱允炆不聽。。鄭村壩之戰和白溝河之戰,使得两军攻守形勢逆轉。

燕軍圍濟南。右參政鐵鉉、盛庸堅守。朱棣射信入城招降,未果。五月十七,燕軍掘開河堤,放水灌城。鐵鉉決定派千人詐降,誘朱棣進城。朱棣圍城攻打三個月。六月,惠帝遣使求和,朱棣不聽。七月,平安進軍河間,擾亂燕軍糧道。八月十六,朱棣撤兵回北平。盛庸、鐵鉉追擊,大敗燕軍,收復德州。

建文二年十月,朱棣決定再度南下,十月廿七到達滄州。燕軍僅用兩天就攻下滄州,徐凱等投降。燕軍自長蘆渡河,十一月初四到達德州。朱棣招降盛庸未果,遂南下。十一月,燕軍到達臨清,焚其糧船。燕軍從館陶渡河,先後到達東阿、東平,威脅南方,迫使盛庸南下。盛庸在東昌(今山東聊城)決戰。十二月廿五,燕軍至東昌。朱棣仍然親自率軍衝鋒,盛庸開陣將朱棣誘入,然後合圍,張玉被中央軍包圍戰死。次日,燕軍再次戰敗,遂北還。在擊退中央軍的阻截后,建文三年正月十六,燕軍返回北平。

朱棣與姚廣孝商議,姚廣孝強烈支持再次出兵。二月十六,朱棣再次出師。三月二十日,燕軍探知盛庸在夾河(今河北省衡水市武邑縣附近,漳河支流)駐紮,於是駐紮在距對方四十里的地方。三月廿二,燕軍進兵夾河。。朱棣率領一萬騎兵和五千步兵攻擊盛庸軍左翼,不能入。此時燕將譚淵望見已經開戰,於是主動出兵攻打。朱棣、朱能等則趁中央軍調動產生的混亂,趁暮色向中央軍後方猛攻,斬殺莊得。此戰殺傷相當,但燕軍損失了大將譚淵。當夜,朱棣率領十餘人在盛庸營地附近露宿;次日(三月廿三)清晨,發現被中央軍包圍。朱棣再次利用禁殺之旨,引馬鳴角,穿過敵軍,揚長而去。中央軍愕然,不敢射箭。

朱棣回到營中,鼓勵眾將「兩軍相當,將勇者勝」,於是再次會戰,雙方互有勝負。戰鬥打了七八個小時后,盛庸大敗,損失了數萬人,退回德州。吳傑、平安引兵準備會合盛庸,聞庸已敗,退回真定。夾河之戰結束。夾河之戰重新確立了燕軍的優勢。閏三月初四,朱允炆因夾河之敗,再次罷免齊泰、黃子澄,謫出京城,暗中令其募兵。

擊敗盛庸后,朱棣進軍真定。。閏三月初九,兩軍會於藳城交戰。。次日,復戰,南軍不能支,大敗而去。。朱棣將射成刺猬的軍旗送回北平,令世子朱高熾妥善保存,以警示後人。從白溝河、夾河到藳城,燕軍三次得大風相助而勝,朱棣認為這是天命所在,非人力所能為。夾藳之戰再次使南軍損失慘重,正面戰場戰事稍緩和。南軍改為通過談判、反間、襲擊後方等方式間接作戰。擊敗平安後,燕軍南下,先後經過順德、廣平、大名,并駐紮於大名。諸郡縣望風而降。

朱棣聽說齊黃被貶,上書和談,表示「奸臣竄逐而其計實行,不敢撤兵」。朱允炆得書,與方孝孺討論,方孝孺表示可以借此機會遣使回報,拖延時間,并懈怠其軍心;同時令遼東等軍隊攻其後方,以備夾攻。於是(四月)惠帝令大理寺少卿薛嵓去見朱棣,傳詔并秘密在軍中散佈相關消息。薛嵓見朱棣,說「朝廷言殿下旦釋甲,暮即旋師。」朱棣表示這連三尺小兒也騙不過。薛嵓無言以對。五月初一,盛庸、吳傑、平安等分兵騷擾燕軍餉道。朱棣遣使者進京表示盛庸等不肯罷兵,必有主使。惠帝聽從方孝孺的意見,將其下獄(一說誅殺),和談破裂。

朱棣見和談破裂,從濟寧南下,成功焚燒大量中央軍糧船,京師大震,德州陷入窘境。

七月,燕軍進攻彰德,林縣投降。七月初十,平安自真定趁虛攻北平,擾其耕牧。朱高熾固守。朱棣分兵回援;(九月十八)平安與戰不利,退回真定。由於河北戰事不利,方孝孺想出了反間計,利用朱高熾(長子)和朱高煦(次子)的矛盾,先寫一封信給守北平的高熾,令其歸順朝廷,許以燕王之位;然後派人告訴朱棣和高煦(隨軍)世子密通朝廷,以使燕軍北還。但朱高熾得到信後,根本沒有拆開,將朝廷使者連人帶信一起送往朱棣處。反間計失敗。

七月十五,盛庸令大同守將房昭入紫荊關威脅保定,據易州西水寨以窺北平。朱棣回兵救援。朱棣分兵守保定,并包圍房昭的山寨。十月初二,燕軍與真定援兵和房昭軍決戰,房昭退回大同。十月廿四,燕軍回到北平。之後又擊敗了襲永平的遼東敵軍。

建文三年冬,南京有宦官因犯錯被處罰,逃到朱棣處,告知南京守備空虛。朱棣遂決定直接率兵南下,臨江一決。道衍亦支持不再與盛庸、平安等糾纏,直趨京師。

1401年(建文三年十二月初二),燕師復出。十二月十二,到達蠡縣(約在保定以南50公里)。建文四年(1402年)正月,燕軍南下至館陶渡河,長驅直入。正月十四,陷東阿;正月十五,陷東平;正月十七,陷汶上;正月廿七,陷沛縣(進江蘇);正月三十,到達徐州。惠帝見燕軍再次出動,三年十二月令駙馬都尉梅殷(惠帝的姑父,顧命大臣)任總兵官,鎮淮安。建文四年正月初一,將遷往蒙化的朱橚(廢周王)召回南京。命魏國公徐輝祖率兵援山東。

二月初一,何福、平安、陳暉進兵濟寧,盛庸進兵淮上。二月廿一,朱棣擊敗徐州的出戰軍隊,徐州自此閉城死守。朱棣繼續南下。三月初一,燕軍進逼安徽宿州。三月初九,抵達渦河(今安徽蚌埠市懷遠縣以北)。平安帶兵來追;但三月十四日在淝河中了朱棣所設的伏兵,只得退回宿州。三月廿三,朱棣遣將斷徐州餉道,鐵鉉等率兵圍攻,互有勝負。四月十四,燕軍進達睢水之小河,搭浮橋。次日,平安、何福領軍奪橋,雙方隔河僵持。數日後,中央軍糧盡,朱棣決定偷襲。半夜,渡河繞至敵後;四月廿二,雙方戰於齊眉山(靈壁縣西南三十里),中央軍大勝,斬燕將李斌。

燕軍陷入窘境。四月廿三,燕軍眾將要求北返,朱棣不同意,說「欲渡河者左,不欲者右。」大部份人站於左側,朱棣怒。朱能這時強力支持朱棣,表示「漢高祖十戰九不勝,卒有天下」,堅定了燕軍堅持的決心。

這時,朝廷訛傳燕軍已兵敗,京師不可無良將,遂召回徐輝祖。四月廿五,考慮到在河邊不易防守,何福移營,與平安在靈壁(一作靈璧)深溝高壘作長遠之計。由於糧道被燕軍阻礙,平安親自率兵六萬護衛糧草。四月廿七,朱棣率精銳襲擊平安,將其一分為二;何福全軍出動救援,朱高煦也率伏兵出現,何福敗走。

中央軍缺糧,何福與平安決定次日(廿九)突圍而出,在淮河取得給養,號令為三聲炮響;次日,燕軍攻打靈壁墻壘,進攻信號正巧也是三聲炮響。於是中央軍以為是己方號炮,紛紛奪路而逃;燕軍趁勢進攻,中央軍全軍覆沒。靈壁之戰就此意外結束。此戰燕軍生擒了陳暉、平安、馬溥、徐真、孫成等三十七員敵將,四名內官(宦官),一百五十員朝廷大臣,獲馬二萬餘匹,降者不計其數。只有何福單騎逃走。

靈璧之戰後,燕軍向東南方向直線前進。五月初七下泗州,朱棣謁祖陵。盛庸在淮河設下防線阻礙燕軍渡河,朱棣在嘗試取道淮安、鳳陽受阻後,遣朱能、丘福率士兵數百人繞道上游乘漁船渡河,五月初九從後方突襲盛庸,盛庸敗走。燕軍遂克盱眙。

五月十一,燕軍向揚州方向前進,五月十七到達天長(揚州西北50公里)。守揚州的監察御史王彬本想抵抗,但屬下反叛,趁其沐浴時綁縛之。五月十八,揚州不戰而降。隨後高郵歸降。

揚州失陷,金陵震動。朱允炆驚慌不已,與方孝孺商議後,先後定下如下幾個救急方法:下罪己詔;號召天下勤王;派練子寧、黃觀、王叔英等外出募兵;召回被貶黜的齊泰、黃子澄;遣人許以割地求和,拖延時間。。

五月廿二,朱允炆遣慶成郡主(朱元璋的侄女、朱棣的堂姐)與朱棣談判,表示願意割地。朱棣說「此奸臣欲姑緩我,以俟遠方之兵耳。」郡主無言以對,遂返。

六月初一,燕軍準備從浦子口渡江,但遇到了盛庸最後的抵抗。燕軍戰不利,此時朱高煦引兵來援,殊死力戰,擊敗盛庸。隨後南軍的一支水軍部隊降燕,燕軍遂於六月初三自瓜洲渡江,并再次擊敗退守此地的盛庸。六月初六,燕軍至鎮江,守將率城投降。

六月初八,燕軍駐紮於龍潭(距京師金陵東約30公里),朝廷大震。朱允炆徘徊殿間,召方孝孺問計。方孝孺表示城中尚有二十萬兵,應堅守待援;即使真戰敗,國君為社稷而死,是理所應當的。可以再派大臣、在京諸王前往談判以拖延時間。於是六月初九,派李景隆、茹瑺等見朱棣,再次談判;朱棣表示割地無名,只要奸臣。六月初十,遣谷王朱橞(建文元年逃回京城)、安王朱楹等第三次前往談判,無果。

六月十二,外出募兵的大臣們仍未返回,朱允炆只得派在京諸王和武臣們守衛各門。時左都督徐增壽(徐達子,輝祖弟)謀內應,被一群文官圍毆。

次日(1402年7月13日),燕軍抵金陵。徐增壽作內應,事敗,被朱允炆親自誅殺於左順門。守衛金川門(位於南京城西北面)的朱橞和李景隆望見朱棣麾蓋,開門迎降。

燕軍進南京,朱允炆見事不可為,遂在皇宮放火。馬皇后死於大火,朱允炆本人不知所終;此後其下落成為謎團。朱棣入城。

朱棣进入南京,出榜安民,成为了明朝第三位皇帝。朱棣进城之时,翰林院編修楊榮迎於馬首,說:「殿下先謁陵乎?先即位乎?」一语點醒朱棣。次日(建文四年六月十四日)起,諸王及文武群臣多次上表勸進,朱棣不允。

數日後(七月十七日),朱棣謁明孝陵,并於當日登基即位,改元永樂,是為明成祖。明成祖重建奉天殿(舊殿被朱允炆所焚),刻玉璽。同年十一月十三日,封王妃徐氏為皇后。

朱棣登基称帝后,对靖难功臣进行了封赏。封王两人,为:朱能(东平武烈王);张玉(河间忠武王)。封公二十二人,为:丘福(淇国公);徐增寿(定国公);陈亨(泾国公);郭亮(兴国公);李彬(茂国公);李遠(莒国公);柳升(融国公);徐忠(蔡国公);袁容(沂国公);郑亨(漳国公);姚广孝(荣国公);张信(郧国公);王聪(漳国公);顾成(夏国公);张武(潞国公);陳珪(靖国公);薛禄(鄞国公);王真(宁国公);吴允诚(凉国公);李讓(景国公);孟善(滕国公);張輔(英国公)。封侯十五人,为:陳瑄(平江侯);何福(宁远侯)李濬(襄城侯);孙岩(应成侯);房宽(思恩侯);王友(清远侯);王忠(靖安侯);劉榮(广宁侯);火真(同安侯);王寧(永春侯);宋晟(西宁侯);郭义(安阳侯);谭渊(崇安侯);柳升(安远侯);薛绶。封伯十八人,为:陈贤(荣昌伯);陈旭(云阳伯);刘才(广恩伯);张兴(安乡伯);房胜(富昌伯);徐理(武康伯);徐祥(兴安伯);金玉(会安伯);高士文(建平伯);陈志(遂安伯);唐云(新昌伯);茹瑺(忠诚伯);王佐(顺昌伯);许诚(永新伯);薛斌(永顺伯);薛贵(安顺伯);赵彝(忻城伯);朱荣(武进伯)。

明成祖登基后不承認建文年號,七月初一(一說六月十八日),將建文元、二、三、四年改為洪武三十二至三十五年,次年改元永乐元年。凡建文年間貶斥的官員,一律恢復職務(如靖難初期因離間被貶的江陰侯吳高被再次起用,守大同);建文年間的各項改革一律取消;建文年間制定的各項法律規定,凡與太祖相悖的,一律廢除。但一些有利於民生的規定也被廢除,如建文二年下令減輕洪武年間浙西一帶的極重的田賦,至此又變重。

明成祖在靖難之役結束后,为了佐证他“清君侧”的起兵宣言,向金陵軍民發布公告:「諭知在京師的軍民人等,我先前一向守望我藩的封地,卻因奸臣弄權作威作福,導致我家骨肉被其殘害,所以不得不起兵誅殺他們,乃是要扶持社稷和保安宗親、藩王。今次研擬安定京城,有罪的奸臣我不敢赦免,無罪者我也不敢濫殺,如有小人藉機報復,擅作綁縛、放縱、掠奪等事情因而禍及無辜,並非我的本意。」

建文四年六月廿五,明成祖誅殺齊泰、黃子澄、方孝孺等建文帝大臣,滅其族。其中據記載,方孝孺被誅十族(九族加朋友門生),受牽連而死者共873人,充軍等罪者千餘人,當中被救的倖存者有假借余姓逃過一劫的方孝孺的幼子方德宗。而因黃子澄受牽連的有345人。景清降後密謀行刺,事敗,八月十二被殺,滅九族;後屠其家鄉,謂「瓜蔓抄」。

此外,眾多建文舊臣如卓敬、暴昭、練子寧、毛泰、郭任、盧植、戴德彝、王艮、王叔英、謝升、丁志方、甘霖、董鏞、陳繼之、韓永、葉福、劉端、黃觀、侯泰、茅大芳、陳迪、鐵鉉等等也都被酷刑處死或自盡,史稱:「忠憤激發,視刀鋸鼎鑊甘之若飴,百世而下,凜凜猶有生氣。」他們的家屬和親人也被牽連,死者甚眾。被流放、逼作妓女及被其它方式懲罰的人也不少。明仁宗即位後,大部份人始獲赦免,而餘下的人的後代卻遲至明神宗時始獲赦免。建文帝被朱棣篡位後,朝野為之盡忠死節者甚眾,不及備載。

在大肆誅殺之外,當月,明成祖將忠於建文帝的魏國公徐輝祖下獄,但顧及其父是中山王徐達,其姊即成祖仁孝文皇后,還是釋放了他,僅削其爵位。輝祖死後,其子嗣魏國公爵。黃觀被明成祖所嫉恨,其狀元的身份被革去,故明代保持三元及第記錄的只有商輅一人。耿炳文、盛庸、平安(靈壁之戰降)、何福、梅殷等将领投降後都受到迫害自杀身亡。

永乐初,明成祖为了安抚诸位藩王,稳定国内局势,同时表示自己和建文帝的不同,曾先后复周、齐、代、岷诸親王旧封;建文帝的弟弟吴王朱允熥、衡王朱允熞、徐王朱允𤐤尚未就藩,明成祖皆降为郡王,同年又将已就藩的朱允熥、朱允熞召到燕京,以不能匡正建文帝为由废为庶人,软禁于凤阳,仅留朱允𤐤奉祀懿文太子,而朱允𤐤不久也于永乐四年死于火灾。当其皇位较巩固时,继续实行削藩。周、齐、代、岷诸王再次遭到削夺;迁宁王于南昌;徙谷王于长沙,旋废为庶人;削辽王护卫。

在政治上,明成祖继续实行太祖的徙富民政策,以加强对豪强地主的控制。明成祖时期,完善了文官制度,在朝廷中逐渐形成了后来内阁制度的雏形。永乐初开始设置內閣,选资历较浅的官僚入阁参与机务,解决了废罢中书省后行政机构的空缺。朱棣重视监察机构的作用,设立分遣御史巡按天下的制度,鼓励官吏互相讦告。他善利用宦官出使、专征、监军、分镇、刺臣民隐事。

明成祖即位之初,对洪武、建文两朝政策进行了某些调整,提出“为治之道在宽猛适中”的原则。他利用科举制及编修书籍等笼络地主、士人,宣扬儒家思想以改变明初過事佛、道教之风,选择官吏力求因才而用,为当时政治、经济、军事、文化等方面的发展奠定了思想和组织基础。

在全国局势稳定之后,明成祖为了加强对大臣的监控,恢复洪武时废罢的锦衣卫。同时,明成祖又设置镇守内臣的东厂衙门,厂卫合势,强化专制统治。

永乐十八年(1420年),明成祖為了鎮壓政治上的反對力量,觉得锦衣卫不足以达成目的,決定設立一個稱為「東緝事廠」,簡稱“東廠”的新衙門,地點位於燕京(今北京)東安門之北,一說東華門旁。(今北京东城区东厂胡同,據說系原东厂所在地。)

東廠的行政長官為欽差掌印太監,全稱職銜為:欽差總督東廠官校辦事太監,簡稱提督東廠,尊稱為「廠公」或「督主」。初設時由司禮監掌印太監兼任,後因事務繁雜,改由司禮監秉筆太監中位居第二、第三者擔任。東廠的屬官有掌刑千戶、理刑百戶各一員,由錦衣衛千戶、百戶來擔任,稱貼刑官。隸役(稱掌班、領班、司房,共四十餘人)、緝事(稱役長和番役)等軍官由锦衣卫撥給。

明初《大明律》明令:「凡樂人搬做雜劇戲文,不許妝爾扮帝王后妃、忠臣節烈、先聖先賢神像,違者杖一百。官民之家容扮者與同罪」,以壓迫雜劇創作,明成祖即變本加厲,以極刑來禁止此類雜劇的印賣:「但有褻瀆帝王聖賢之詞曲、駕頭雜劇,非該律所載者,敢有收藏、傳誦、印賣,一時拿送法司究治」,「但這等詞曲,出榜後,限他五日,都要乾淨,將赴官燒毀了,敢有收藏的,全家殺了」。

明成祖十分重視經營北方,加之自己兴起于北平(今北京),明成祖在南京即位后,于永乐元年改北平為行在,設六部,增設北京周圍衛所,逐漸建立起北方新的政治、軍事中心。永乐七年(1409年),明成祖开始了營建北京天壽山長陵,以示立足北方的決心。與此同時,爭取與蒙古族建立友好關係。韃靼、瓦剌各部先後接受明政府封號。永乐八年(1410年)至二十二年(1424年),朱棣親自率兵五次北征,鞏固了北部邊防。永乐十四年(1416年)開工修建北京宮殿也就是紫禁城(但後來部分宮殿被李自成放火燒毀,清初又重新修復)。永乐十九年(1421年)正式遷都,定鼎北京。

明成祖注意社会经济的恢复与发展,认为“家给人足”、“斯民小康”是天下治平的根本。他大力发展和完善军事屯田制度和盐商开中则例,保证军粮和边饷的供给。在中原各地鼓励垦种荒闲田土,实行迁民宽乡,督民耕作等方法以促进生产,并注意蠲免赈济等措施,防止农民破产,保证了赋役征派。

明成祖对各地方官吏要求极为严格,要求凡地方官吏必须深入了解民情,随时向朝廷反映民间疾苦。永乐十年(1412年),朱棣命令入朝觐见的地方官吏五百余人各自陈述当地的民情,还规定“不言者罪之,言有不当者勿问’。之后,永乐帝宣布“谕户部,凡郡县有司及朝使目击民艰不言者,悉逮治。”即地方官或中央派出的民情观察员,如果看到民间疾苦而不实报的,要逮捕法办。对民间发生了灾情,地方上要及时赈济,做到“水旱朝告夕振,无有雍塞”。通过这些措施,永乐时“赋入盈羡”,达到有明一代最高峰,史称永乐盛世。

西南边疆,永乐十一年(1413年),平定思南、思州土司叛亂後,設立貴州布政使司。為加強對烏思藏(今西藏)地區的控制,朱棣派遣官吏迎番僧入京,給予封賜,尊為帝師。不過,史學界對明朝是否實際統治了西藏存在較大的爭議。

永乐年间,明朝在藏区建立一套僧官制度,僧官分教王、西天佛子、大国师、国师、禅师、都纲、喇嘛等,每级依受封者的身份、地位进行分封。如明成祖即位的当年,即派侯显前往乌思藏迎请噶玛噶举派的第五世噶玛巴活佛,后封其为“大宝法王”。1406年,明成祖又遣使入藏封乌思藏帕竹第五任第悉扎巴坚赞为“阐化王”。明封八王中的两大法王、五大教王都是永乐时期封授的。此外,明成祖依僧官制度还进行了大规模的分封,由此明朝对藏区的各政教势力由上至下各级首领的分封基本完成。但明朝并未在烏思藏等地区驻军。亦有学者通过对比元朝对于西藏的实际管辖,认为明朝上面这些对藏人名义上的封授并不能被认为拥有在西藏的实际政治权力。《劍橋中國明代史》亦指出:「無論是在經濟領域,還是在政治領域,西藏人都未覺得他們是明朝廷臣民。另外,他們無須中國(明朝)居中調解而維持著與其他國家和民族的關係。」

东北边疆,永乐七年(1409年)在女真地區,設立奴儿干都司。明成祖永乐元年(1403年)派邢枢等传谕奴儿干,正式招抚诸部,擴大明朝東疆。永乐二年(1404年),置奴儿干等卫所,其后在当地相继建卫所达一百三十餘。永乐七年(1409年)明政府设置奴儿干都指挥使司管辖奴儿干地区的所有军事建制机构。永乐九年(1411年)正式开始行政管辖权。都司的主要官员初为派駐數年而輪調的流官,后为當地部落領袖所世袭。明成祖為了安撫東北女真各部,在歸附的海西女真(位於松花江上游)與建州女真(位於松花江、牡丹江之間)設置衛所,並派宦官亦失哈安撫位於黑龍江下游的野人女真。亦失哈并于1413年视察了库页岛,宣示了明朝对此地的主权。在奴儿干都司官衙所在地附近建有永宁寺,立有永宁寺碑,清代曹廷杰于1885年曾拓回碑文。同时,明成祖撤去大宁都司,将宁王朱权内迁南昌,授予兀良哈蒙古的朵颜、泰宁和福余三个卫所自治权,但不允许三卫蒙古人南迁到大宁地区驻牧。明成祖还于1406年和1422年对兀良哈蒙古进行镇压,以维持这一地区的稳定。

辖区内主要居民为蒙古、女真、吉里迷(尼夫赫人)、苦夷(阿伊努人)、达斡尔等族人民,分置卫所,以各族首领为各卫所都督、都指挥、指挥、千户、百户、镇抚等职,给予印信。据《明史》记载,奴儿干都司有卫三百八十四,所二十四,站七,地面七,寨一。都司治所奴儿干城(元朝征东元帅府旧地,今俄罗斯尼古拉耶夫斯克特林),在黑龙江下游东岸,下距黑龙江口约两百公里,上距吉林船厂约两千五百公里。明宣宗即位后,奴儿干都司于宣德九年(1434年)正式废弃,共持续25年。

西北边疆,永乐四年(1406年)設立哈密衛。此前,察合台的后裔肃王兀纳失里於明洪武十三年(1380年),开始向明朝纳贡,被明太祖封为哈密国王。其子脱脱向明成祖朝贡,永乐四年(1406年)三月,明成祖宣布设立哈密卫,以其头目马哈麻火者等为指挥、千百户等官,又以周安为忠顺王长史,刘行为纪善,辅导。之后,哈密国成为设有明朝羁縻卫所的王国,忠顺王是哈密国王,哈密卫指挥使掌握哈密兵权,另有汉人长史。

同时,明成祖还多次派遣吏部驗封司員外郎陳誠、中官李達等官員出使西域。隨後西域的帖木兒帝國、吐魯番、失剌斯、俺都準、火州也與明朝多次互派使者往來,加強了政治、駐軍和貿易往來,全國統一形勢得到進一步發展和鞏固。

明成祖很重视河工,永乐九年(1411年)朱棣於疏浚會通河為保證北京糧食與各項物資的需要。朱棣命開漕運。漕運在元朝至元年間即有,然而卻因會通河一段水淺而無法大量載運物資,於是元朝均以海運為主。明朝初期,傳餉遼東、北平的途徑也均以海運為主。洪武二十四年,黃河在原武絕口,會通河於是被淤。

永乐年间,明成祖遷都北京,採用河路、海路并運。當時海運危險且多有損失;而河運卻經過淮河轉沙河,然後經過黃河進入衛河,於此轉入北京,陸運須經過八個衛所,勞民傷財。濟寧州同知潘叔正上疏建議浚通會通河,使得元朝運河恢復。於是,朱棣命宋禮、刑部侍郎金純、都督周長前往治理。會通河首要問題為水源不足,宋禮採用汶上老人白英的建議,修築埋城與戴村坝,橫截汶水向南,經河面最高端南旺分水,流入運河,且使黃河不會影響漕運。同年八月還京,論首功,受上賞。

次年,因御史許堪進言衛河水患,朱棣再命宋禮前往治理。宋禮在魏家灣分支黃河,泄水入土河,於是從德州西北開一支支流,到海豐、大沽流入大海。此時,宋禮以海運損失巨大、勞民傷財,上言請求停止海運,而恰逢平江伯陳瑄治理長江、淮河等告竣。於是河運從此昌盛,可運大型物資。永樂十三年,朱棣遂終止海運。

永乐十三年(1415年)鑿清江浦,使大運河重新暢通,對南北經濟文化交流與發展起了重要的作用。

永乐年间,明成祖还派派夏原吉治水江南,疏浚吴淞。

在政治稳定、经济繁荣、边疆稳定的局面下,为整理知识,明成祖令解縉等人修书。編撰宗旨:「凡书契以來经史子集百家之书,至於天文、地志、阴阳、医卜、僧道、技艺之言,备辑为一书,毋厌浩繁!」,召集一百四十七人,首次成书于永乐二年(1404年),初名《文献集成》;明成祖過目後認為「所纂尚多未備」,不甚滿意。永樂三年(1405年)再命姚廣孝、鄭賜、劉季篪、解縉等人重修,這次動用編寫人員朝野上下共二千一百六十九人,啟用了南京文淵閣的全部藏書,永樂五年(1407年)定稿進呈,明成祖看了十分滿意,親自為序,並命名為《永樂大典》,清抄至永樂六年(1408年)冬天才正式成書。

《永乐大典》由解縉、太子少傅姚廣孝和禮部尚書鄭賜監修,組織上設監修、總裁、副總裁、都總裁等職,負責各方面工作。監修:解縉、姚廣孝、鄭賜;總裁:副總裁:蔣用文、趙同友;都總裁:陳濟。

《永乐大典》修書過程對所收錄的書籍沒有做任何修改,採用兼收並取的方式,保持了書籍原始的內容。明成祖修大型類書《永樂大典》,在三年時間內即告完成。《永樂大典》有22877卷,其中凡例、目錄60卷,全書分裝為11095冊,引書達七八千種,字數約有三億七千多萬,且未有任何刪節,這是清朝《四庫全書》無法相提並論的。但成祖并未将《永乐大典》复写刊刻,而决定只制作一份抄本,并于1409年完成。永乐年間修訂的《永樂大典》原書只有一部,現今存世的都是嘉靖年間的抄本。

明成祖时期,为了开展对外交流,扩大明朝的影响,同时确立自己即位的正统性,从永乐三年起,朱棣派三宝太监郑和(初名馬三寶)率领船队六次出使西洋(第七次在明宣宗宣德年间),所历三十余国,成为明初盛事。永乐时派使臣来朝者亦达三十余国。浡泥王和苏禄东王亲自率使臣来中国,不幸病故,分别葬于南京(浡泥国王墓)和德州(苏禄国王墓)。

永乐三年六月十五(1405年7月11日)明成祖命郑和为正使,王景弘为副使率士兵二万八千余人出使西洋,造长44丈广18丈大船62艘,从苏州刘家河泛海到福建,再由福建五虎门杨帆,先到占城(今越南中南部地區),后向爪哇方向南航,次年6月30日在爪哇三宝垄登陆,进行贸易。时西爪哇与东爪哇内战,西爪哇灭东爪哇,西爪哇兵杀郑和士兵170人,西王畏惧,献黄金6万两,补偿郑和死难士兵。随后到三佛齐旧港,时旧港广东侨领施进卿来报,海盗陈祖义凶横,郑和兴兵剿灭贼党五千多人,烧贼船十艘,获贼船五艘,生擒海盗陈祖义等三贼首。郑和船队后到过苏门答腊、满刺加、锡兰、古里等国家。在古里赐其王国王诰命银印,并起建碑亭,立石碑“去中国十万余里,民物咸若,熙嗥同风,刻石于兹,永示万世”。

永乐五年九月初二(1407年10月2日),郑和回国,押陈祖义等献上,陈祖义等被问斩。施进卿被封为旧港宣慰使。旧港擒贼有功将士获赏:指挥官钞一百锭,彩币四表里,千户钞八十锭,彩币三表里,百户钞六十锭,彩币二表里;医士,番火长钞五十锭,彩币一表里,锦布三匹。

永乐六年正月,明成祖命工部造宝船四十八艘。永乐六年九月十三日(1407年10月13日),命太监郑和、王景弘,王贵通等出使古里,满剌加,苏门答剌,阿鲁,加异勒,爪哇,暹罗,占城,柯枝,阿拔把丹,小柯兰,南巫里,甘巴里等国,赐其国王锦绮纱罗,永乐七年夏(1409年)回国。第二次下西洋人数据载有27000人。

永乐七年九月(1409年10月),明成祖命正使太监郑和、副使王景弘、候显率领官兵二万七千余人,驾驶海舶四十八艘,从太仓浏家港启航,敕使占城,宾童龙,真腊,暹罗,假里马丁,交阑山,爪哇,重迦罗,吉里闷地,古里,满剌加,彭亨,东西竺,龙牙迦邈,淡洋,苏门答剌,花面,龙涎屿,翠兰屿,阿鲁,锡兰,小葛兰,柯枝,榜葛剌,不剌哇,竹步,木骨都束,苏禄等国。費信、馬歡等人會同前往。满剌加当时是暹罗属国,正使郑和奉帝命招敕,赐双台银印,冠带袍服,建碑封域为满剌加国,暹罗不敢扰。满剌加九洲山盛产沉香,黄熟香;太监郑和等差官兵入山采香,得直径八九尺,长八九丈的标本6株。永乐七年,皇上命正使太监郑和等赍捧诏敕金银供器等到锡兰山寺布施,并建立《布施锡兰山佛寺碑》此碑現存于科倫坡博物館。郑和访问锡兰山国时,锡兰山国王亞烈苦奈兒“負固不恭,謀害舟師”,被郑和觉察,离开锡兰山前往他国。回程时再次访问锡兰山国,亚烈苦奈儿诱骗郑和到国中,发兵五万围攻郑和船队,又伐木阻断郑和归路。郑和趁贼兵倾巢而出,国中空虚,带领随从二千官兵,取小道出其不意突袭亚烈苦奈儿王城,破城而入,生擒亚烈苦奈儿并家属。

永乐九年六月十六(1411年7月6日),郑和回国獻亚烈苦奈儿与永樂帝,朝臣齐奏诛杀,永樂帝怜悯亚烈苦奈儿无知,释放亚烈苦奈儿和妻子,给予衣食,命礼部商议,选其国人中贤者为王。选贤者邪把乃耶,遣使赍引,诰封为锡兰山国王,并遣返亚烈苦奈儿。永乐九年(1411年)满剌加国王拜里米苏剌,率领妻子陪臣540多人来朝,朝廷赐海船回国守卫疆土。从此“海外诸番,益服天子威德”。八月,礼部、兵部议奏,对锡兰战役有功将士754人,按奇功,奇功次等,头功,头功次等,各有升职,并赏赐钞银,彩币锦布等。

永乐十一年十一月(1413年11月),明成祖命正使太监郑和,副使王景弘等奉命统军二万七千余人,驾海舶四十,出使满剌加,爪哇,占城,苏门答剌,柯枝,古里,南渤里,彭亨,吉兰丹,加异勒,勿鲁谟斯,比剌,溜山,孙剌等国。郑和使团中包括官员868人,兵26800人,指挥93人,都指挥2人,书手140人,百户430人,户部郎中1人,阴阳官1人,教谕1人,舍人2人,医官医士180人,正使太监7人,监丞5人,少监10人,内官内使53人其中包括翻译官马欢,陕西西安羊市大街清真寺掌教哈三,指挥唐敬,王衡,林子宣,胡俊,哈同等。郑和先到占城,奉帝命赐占城王冠带。1413年郑和船队到苏门答剌,当时伪王苏干剌窃国,郑和奉帝命统率官兵追剿,生擒苏干剌送京伏诛。1413年郑和舰队在三宝垄停留一个月整休,郑和费信常在当地华人回教堂祈祷。郑和命哈芝黄达京掌管占婆华人回教徒。首次繞過阿拉伯半島,航行東非麻林迪(肯尼亚),永乐十三年七月初八(1415年8月12日)回国。同年11月,麻林迪特使來中國進獻“麒麟”(即長頸鹿)。

永乐十五年五月十五日(1417年6月)总兵太监郑和受明成祖命,在泉州回教先贤墓行香,往西洋忽鲁谟斯等国公干,永乐十五年五月(1417年6月)出发,护送古里、爪哇、满剌加、占城、锡兰山、木骨都束、溜山、喃渤里、卜剌哇、苏门答剌、麻林、剌撒、忽鲁谟斯、柯枝、南巫里、沙里湾泥、彭亨各国使者及旧港宣慰使归国。隨行有僧人慧信,将领朱真、唐敬等。郑和奉命在柯枝诏赐国王印诰,封国中大山为镇国山,并立碑铭文。忽鲁谟斯进贡狮子,金钱豹,西马;阿丹国进贡麒麟,祖法尔进贡长角马,木骨都束进贡花福鹿、狮子;卜剌哇进贡千里骆驼、鸵鸡;爪哇、古里进贡麾里羔兽。永乐十七年七月十七(1419年8月8日)回国。

宋末泉州市舶司提举蒲寿庚之侄蒲日和,也与太监郑和,奉敕往西洋寻玉玺,有功,加封泉州卫镇抚。

永乐十九年正月三十日(1421年3月3日),郑和奉明成祖命出发,往榜葛剌(孟加拉),史載“於鎮東洋中,官舟遭大風,掀翻欲溺,舟中喧泣,急叩神求佑,言未畢,……風恬浪靜”,中道返回,永乐二十年八月十八(1422年9月2日)回国。永樂二十二年,明成祖去世,仁宗朱高熾即位,以經濟空虛,下令停止下西洋的行動。

永乐二十二年七月十七日(1424年8月12日),明成祖去世,太子朱高炽即位,改元洪熙,是为明仁宗,于洪熙元年五月辛巳(1425年5月29日)去世,太子朱瞻基即位,改元宣德,是为明宣宗。宣德五年闰十二月初六(1430年1月),郑和奉明宣宗命率领二万七千余官兵,驾驶宝船61艘,从龙江关(今南京下关)启航,进行了第七次下西洋。开始返航后,郑和因劳累过度于宣德八年(1433年)四月初在印度西海岸古里去世,遺體埋葬於古里,船队由太监王景弘率领返航,宣德八年七月初六(1433年7月22日)返回南京。第七次下西洋人数据载有27550人。

明太祖朱元璋為與鄰近國家保持長久的和睦關係,便在其所主編的《皇明祖訓》中開列十五個「不征諸夷國名」,以警戒後世子孫切勿「倚中國富強,貪一時戰功,無故興兵,致傷人命」,越南(安南國)便是其中之一。1400年,安南陳朝權臣胡季犛篡位,建立胡朝,改國號為「大虞」。不久後自稱太上皇,由兒子胡漢蒼(即胡𡗨)即皇帝位。由於前朝陳氏原是向明朝稱臣,世世受明冊封,憑著篡奪得國的胡氏為免惹起明朝猜疑,便於1403年農曆四月丁未(西曆4月21日)遣使赴明,向剛起兵奪位的明成祖聲稱陳氏「宗嗣繼絕,支庶淪滅,無可紹承。臣,陳氏之甥,為眾所推」,欲藉此聲稱自己具有統治資格,要求明朝冊封。明成祖派楊渤到越南觀察後,當地陪臣耆老跟隨他向成祖上奏稱「眾人誠心推𡗨權理國事」,明廷一時再沒有懷疑的理由,便封胡漢蒼為「安南國王」。

1404年農曆八月乙亥(西曆9月10日),陳朝遺臣裴伯耆到明廷,控訴胡季犛父子「弒主篡位,屠害忠臣」,要求明朝出兵「擒滅此賊,蕩除奸凶,復立陳氏子孫」 八月丁酉日(西曆10月2日),有一位自稱陳氏子孫,名叫陳天平的人(越南史籍寫作「陳添平」,《大越史記全書》稱他的身份本是「陳元輝家奴阮康」),從老撾入明,亦向明帝訴說胡氏篡位的經過,要求恢復陳氏統治。 其後,明成祖當著胡朝的來使面前,安排陳天平與他們會見,使一眾來使都錯愕下拜,甚至涕泣,適值裴伯耆在場,向來使責以大義,場面緊張。 明廷於是對越南政局多所干涉,派員查核實情,胡朝明白勢不得已,唯有承認責任,要求「迎歸天平」。

另外,明越兩國又因領土問題出現外交風波。1405年,廣西省思明土官及雲南省寧遠州土官向明廷控訴,轄境猛慢、祿州等地被越南所佔。為此,明廷於該年農曆二月,遣使責難胡朝,要求取得祿州,胡朝便被迫將古樓等五十九村交給明朝政府。

胡朝雖然願意息事寧人,但兩國關係仍然緊張。其後,胡朝所派到明廷的使節,都遭扣留,不許回國。明廷又派員入越,查探山川道路險要之地,以為日後南征的準備。 另外,胡朝的南鄰占城,曾於1404年遣使入明,聲稱遭到胡氏「攻擾地方,殺掠人畜」,並進一步「請吏治之」, 這亦引起了明廷的注意。

不過,明成祖仍未敢輕言出兵。1405年年底,雲南將領沐晟建議出兵,卻遭明成祖反駁說:「爾又言欲發兵向安南。朕方以布恩信,懷遠人為務。胡𡗨雖擾我邊境,令已遣人詰問,若能攄誠順命,則亦當弘包荒之量。」 至於陳天平的處置,明廷則決定送歸越南,並要求越人「以君事之」,奉為國主。 越南方面,胡朝有感於對明關係緊張,亦積極防備,重編軍制,在多邦城(陳仲金說位於山西省先豐縣古法社)加強防守,於各個河海要處裝插木樁陷阱,整頓軍庫,招募人民有巧藝者入伍。但胡朝君臣對明主戰或主和,意見分歧甚大,有官員認為只好「從他(明朝)所好,以緩師可也」,左相國胡元澄則認為只決定於「民心之從違耳」,對明作戰並無十足把握。

1406年,明朝派鎮守廣西都督僉事黃中領五千士兵(《大越史記全書》稱領兵十萬),護送陳朝王孫陳天平(陳添平)回越南(《明實錄》把事件列在該年農曆三月丙午,即西曆4月4日;《大越史記全書》則列入農曆四月八日,即西曆4月26日)。當進入越南境內的支棱隘時,遇上胡軍截擊,明軍不敵,陳天平及部份士兵被俘。陳天平經胡朝審訊後,被「處陵遲罪」。明成祖得悉後大怒,便「決意興師」。

同年年中,明成祖派總兵官朱能加封「征夷大將軍」,配印信。後來在行軍時病卒,由副將張輔代替)、左副將軍沐晟、右副將軍張輔、左參將李彬、右參將陳旭等領兵(《大越史記全書》稱共有八十萬人,中國學者郭振鐸、張笑梅認為可能有誇大),分兵兩路,開進越南的白鶴江會師,一邊向越南腹地步步推進,一邊發出檄文向越人呼籲胡季犛父子的行為是「肆逞凶暴,虐于一國」,並列出胡氏「兩弒前安南國王以據其國」、「賊殺陳氏子孫宗族殆盡」、「淫刑峻法,暴殺無辜,重斂煩徵,剝削不已」等二十款大罪,又稱明軍的到來是「吊爾民之困苦,復陳氏之宗祀」,以使民心動搖。果然,不少越人「厭胡氏苛政,罔有戰心」,有助明軍前進更為順利。農曆十二月丙申十一日(西曆1407年1月19日),胡軍的主力退守多邦城,明軍亦看準該城位於河邊,有較大面積的沙灘可供搶灘,於是分兵進攻,成功以火銃擊退胡軍象兵。其後,明軍攻入越南的重要城市東都昇龍,並大肆掠奪,「擄掠女子玉帛,會計粮儲,分官辦事,招集流民。為久居計,多閹割童男,及收各處銅錢,驛送金陵」。

1407年年初,明軍攻破昇龍後,向胡朝的首都清化繼續前進,胡氏皇子胡元澄領軍退守黃江(在今越南河南省的一段紅河),與胡季犛、胡漢蒼會合。明將沐晟則進駐木凡江(在今越南河内市,與黃江相接)預備出擊。農曆二月,沐晟沿江兩岸擊敗胡元澄軍,追擊至悶海口(在今越南南定省),因軍中爆發疾疫,明軍移師到鹹子關立塞備戰。農曆三月,胡軍集合水步大軍七萬,號稱二十一萬,與明軍爆發鹹子關之戰。結果胡軍潰敗,大批兵士溺斃於該處河流,無數船隻及軍糧沉沒,胡氏父子敗逃,最終在農曆五月十一日(西曆6月16日)在奇羅海口(在今越南河靜省奇英縣)被明軍俘獲,胡朝滅亡,領土被明朝佔領。據當時的統計,越南土地人口物產資料為:府州四十八、縣一百六十八、戶三百一十二萬九千五百、象一百一十二、馬四百二十、牛三萬五千七百五十、船八千八百六十五。(※此一統計數字,按《明實錄》記載的1408年農曆六月的計算,則是「安撫人民三百一十二萬有奇;獲蠻人二百八萬七千五百有奇,糧儲一千三百六十萬石,象、馬、牛共二十三萬五千九百餘隻,船八千六百七十七艘,軍器二百五十三萬九千八百五十二件。」)

胡朝亡後,明成祖在農曆六月癸未朔(西曆7月5日)下詔,聲稱這次軍事行動是為了越南原本的陳氏王室著想,「期伐罪(指胡朝)以吊民,將興滅而繼絕」,並打算對「久染夷俗」的越人「設官兼治,教以中國禮法」,以達致「廣施一視之仁,永樂太平之治」。明廷又以陳朝子孫被胡氏殺戮殆盡,無可繼承,於是在越南設置交址都指揮使司、交址等處承宣布政使司及交址等處提刑按察使司等官署,將之直接管轄。

安南内属后,安南人民不断进行反抗,明军多次进行镇压。永乐二十二年(1424年),明成祖去世,太子朱高炽明仁宗即位,次年明仁宗去世,太子朱瞻基即位,是为明宣宗。宣宗考慮到「數年以來,一方不靖,屢勤王師」, 便允許撤兵。黎利得勝後,就發佈阮廌所起草的《平吳大誥》,稱他自己的抗明鬥爭是「仁義之舉,要在安民,吊伐之師,莫先去暴」;提出中越兩國是「山川之封域既殊,南北之風俗亦異」,因而有必要脫離明朝統治,自行建國,於是建立後黎朝。

其後,1431年農曆正月五日(西曆2月12日),明封黎利為安南國王,從此朝貢不絕。

为了稳定北方边境,对付蒙古势力。永乐七年(1409年),明成祖朱棣派淇国公丘福率十万大军征讨鞑靼,由于轻敌,孤军深入,中埋伏,全军覆没。为消除边患,明成祖决心亲征。明永乐八年(1410年)二月,明成祖调集50万大军。五月八日,明军行至胪朐河(今克鲁伦河,朱棣将之更名为“饮马河”)流域,询得鞑靼可汗本雅失里率军向西逃往瓦剌部,丞相阿鲁台则向东逃。朱棣亲率将士向西追击本雅失里,五月十三日,明军在斡难河(位于今蒙俄边境)大败本雅失里。朱棣打败本雅失里后,挥师向东攻击阿鲁台,双方在今蒙俄边境之斡难河东北方向交战,明军杀敌无数,阿鲁台坠马逃遁。此时天气炎热,缺水,且粮草不济,朱棣下令班师。鞑靼部经过明军的这次打击,臣服了明朝,当年向明成祖进贡马匹。成祖亦给予优厚的赏赐,其部臣阿鲁台接受了成祖给他“和宁王”的封号。

明军在永乐八年(1410年)第一次出征鞑靼后,瓦剌部趁机迅速发展壮大,1413年,瓦剌军进驻胪朐河(今克鲁伦河),窥视中原。明成祖决心再次亲征,调集兵力,筹集粮饷。永乐十二年(1414年)二月,明军从北京出发,六月初三,明军在三峡口(今蒙古乌兰巴托东南)击败了瓦剌部的一股游兵,杀敌数十騎;初七日,明军行至勿兰忽失温(今蒙古乌兰巴托东南),瓦剌军3万之众,依托山势,分三路阻抗,朱棣派骑兵冲击,引诱敌兵离开山势,遂命柳升发炮轰击,自己亦亲率铁骑杀入敌阵,瓦剌军败退,朱棣乘势追击,兵分几路夹击瓦剌军的所扑,杀敌数千,瓦剌军纷纷败逃。此役,瓦剌受到了重创,此后多年不敢犯边,同时,明军也伤亡惨重。

瓦剌被明成祖打败,鞑靼趁此机会经过几年的发展,势力日益强盛起来,从而改变对明朝的依附政策,并侮辱或拘留没明朝派去的使节,还时常对明朝边境进行骚扰的劫掠。永乐十九年(1421年)冬初,鞑靼围攻明朝北方重镇兴和,杀死了明军指挥官王祥,对此,朱棣决定第三次亲征漠北。永乐二十年(1422年)三月,明成祖率軍从北京出发,出击鞑靼。其主力部队至宣府(今河北宣化)东南的鸡鸣山时,鞑靼首领阿鲁台得知明军来袭,乘夜逃离兴和,避而不战。七月,明军到达煞胡原,俘获鞑靼的部属,得知阿鲁台已逃走,朱棣下令停止追击。明军在回师途中,击败兀良哈部,九月,回师北京。明成祖第三次出击漠北,虽对鞑靼部有一定的打击,但成效不大,并没彻底解决盘据漠北的蒙古三个部落对明朝边境的滋扰。

永乐二十一年(1423年),鞑靼首领阿鲁台再次率部滋扰明朝边境,明成祖闻悉后决定再次亲征。明军八月初出征,九月上旬,明军到达沙城(今河北张北以北)时,阿鲁台的部下阿失贴木儿率部投降明军,并得知阿鲁台被瓦剌打败,其部已溃散,明军暂时驻扎不前;十月,明军继续北上,在黄河以北击败鞑靼西部的军队,鞑靼王子也先土干率部众来降明,明成祖朱棣随即封也先土干为忠勇王,十一月,明军班师回京。

永乐时全国形势相对缓和,但由于国家支出过大,赋役征派繁重,使有些地区发生了农民流亡与起义,十八年山东发生的唐赛儿起义是其中规模较大的一支。明永乐二十二年(1424年)正月至七月,明軍对蒙古鞑靼部的作戰。是年正月,鞑靼部首领阿鲁台率軍進犯明山西大同、开平(今内蒙古正兰旗东北)等地。明成祖朱棣遂调集山西、山东、河南、陕西、辽东5都司之兵于京师(今北京)和宣府(今河北宣化)待命。四月三日,以安远侯柳升、遂安伯陈英为中军;武安侯郑亨、保定侯盂瑛为左哨,阳武侯薛禄、新宁伯谭忠为右哨;英国公张辅、成国公朱勇为左掖,成山侯王通、兴安伯徐亨为右掖;宁阳侯陈懋、忠勇王金忠又名也先土干为前锋,出兵北征。出征前戶部尚書夏元吉以國庫虛耗,曾勸他勿起戰事,但他不聽,反繫之大獄。二十五日,进至隰宁(今河北沽源南),获悉阿鲁台逃往答兰纳木儿河(今蒙古境内之哈剌哈河下游),明成祖令全军急速追击。六月十七日,进至答兰纳木儿河,周围300余里不见阿鲁台部踪影,遂下令班师。

明成祖為填補太祖廢除丞相後導致六部之首的空缺,但又希望強化皇權,他设立内阁,内阁大學士计有解縉、黃淮、胡廣、楊榮、金幼孜、楊士奇、胡儼。明成祖时期涌现许多著名大臣,包括蹇义、郁新、刘观、郑赐、宋礼、金纯、夏原吉、吕震、金忠、沐春、沐晟、沐昂。

明成祖任用酷吏强化自己的统治,著名的包括陳瑛和紀綱。

明成祖时期的著名太监包括:鄭和:三宝太監七下西洋;王景弘:鄭和的副手;侯顯:有才辨,強力敢任,五使絕域,勞績與鄭和亞;亦失哈:鞏固北方邊防,晚年研究改造武器,如改造步槍(裝槍頭-為安裝刺刀的先驅);王彦:原名王狗兒,尚寶監太監;昌盛:神宫監太監,貴州人。歷洪武-建文-永樂-洪熙-宣德五朝。

永樂二十二年(1424年)七月,明成祖率領北征大軍班師返京。七月十五日,明成祖病重。十六日,行至榆木川(今内蒙古多伦),昏迷不醒。十八日,明成祖朱棣崩逝於榆木川(今中國內蒙古自治區錫林郭勒盟多倫縣),享壽六十四岁,在位二十二年。遗诏传位皇太子。大學士楊榮、太監馬去等秘不發喪,暗中派御馬監少監海壽秘密回京,“奉遗命,驰讣皇太子”。太子朱高熾立即派皇太孫前往虎帐。八月十一日,皇太孫到達軍營後,始發佈帝崩消息。太子朱高炽即位,宣布次年改元洪熙,是为明仁宗。明成祖驾崩后,殉葬的有30余位宫女,其中包括成祖的16位嫔妃。

明成祖驾崩后,谥体天弘道高明广运圣武神功纯仁至孝文皇帝,庙号太宗,十二月十九日,明成祖与仁孝文皇后徐氏合葬于长陵。嘉靖十七年(1538年)九月,明世宗朱厚熜改谥明成祖为启天弘道高明肇运圣武神功纯仁至孝文皇帝,改上庙号为成祖。

《明史·成祖本纪》中评价明成祖:文皇少长习兵,据幽燕形胜之地,乘建文孱弱,长驱内向,奄有四海。即位以后,躬行节俭,水旱朝告夕振,无有壅蔽。知人善任,表里洞达,雄武之略,同符高祖。六师屡出,漠北尘清。至其季年,威德遐被,四方宾服,明命而入贡者殆三十国。幅陨之广,远迈汉、唐。成功骏烈,卓乎盛矣。然而革除之际,倒行逆施,惭德亦曷可掩哉。

蔡石山在其著作《永乐大帝:一个中国帝王的精神肖像》的开篇评价明成祖“明朝的永乐皇帝,驾崩于1424年8月12日,自从1402年7月17日登极以来——近乎八千零六十二天的在位期间——而且所有的证据也显示,他从未浪费过一天”。在书末,他再次评价明成祖“毋庸置疑,永乐有过多的自我,而且拥有很多的美德:他是自信、直率的,能够甄别和牢记有很强能力之人的贡献,而且保护依靠他的那些人,尤其是他的家人。不过,他也有黑暗面,特征就是不必要又未经思考的侵犯性,而这类侵犯性经常产生了暴虐和消耗”。

《朝鲜王朝实录·世宗庄宪大王实录》中评价明成祖「使臣言:"前後選獻韓氏等女,皆殉大行皇帝。" 先是,賈人子呂氏入皇帝宮中,與本國呂氏以同姓,欲結好,呂氏不從,賈呂蓄憾。 及權妃卒,誣告呂氏點毒藥於茶進之,帝怒,誅呂氏及宮人宦官數百餘人。 後賈呂與宮人魚氏私宦者,帝頗覺,然寵二人不發,二人自懼縊死。 帝怒,事起賈呂,鞫賈呂侍婢,皆誣服云:"欲行弑逆。" 凡連坐者二千八百人,皆親臨剮之,或有面詬帝曰:"自家陽衰,故私年少寺人,何咎之有?" 後帝命畫工圖,賈呂與小宦相抱之狀,欲令後世見之,然思魚氏不置,令藏於壽陵之側。 及仁宗卽位,掘棄之。 亂之初起,本國任氏、鄭氏自經而死,黃氏、李氏被鞫處斬。 黃氏援引他人甚多,李氏曰:"等死耳,何引他人爲? 我當獨死。" 終不誣一人而死。 於是,本國諸女皆被誅,獨崔氏曾在南京,帝召宮女之在南京者,崔氏以病未至,及亂作,殺宮人殆盡,以後至獲免。 韓氏當亂,幽閉空室,不給飮食者累日,守門宦者哀之,或時置食於門內,故得不死。 然其從婢皆逮死,乳媪金黑亦繫獄,事定乃特赦之。 初,黃氏之未赴京也,兄夫金德章坐於所在房窓外,黃儼見之大怒,責之,及其入朝,在道得腹痛之疾,醫用諸藥,皆無效,思食汁菹。 儼問元閔生曰:"此何物耶?" 閔生備言沈造之方,儼變色曰:"欲食人肉,吾可割股而進,如此草地,何得此物?" 黃氏腹痛不已,每夜使從婢以手磨動其腹,到一夜小便時,陰出一物,大如茄子許,皮裹肉塊也。 婢棄諸廁中,一行衆婢,皆知而喧說。 又黃氏婢潛說:"初出行也,德章贈一木梳。" 欽差皆不知之。 帝以黃氏非處女詰之,乃云:"曾與姊夫金德章、隣人皂隷通焉。" 帝怒,將責本國,勑已成,有宮人楊氏者方寵,知之,語韓氏其故,韓氏泣乞哀于帝曰:"黃氏在家私人,豈我王之所知也?" 帝感悟,遂命韓氏罰之,韓氏乃批黃氏之頰。 明年戊戌,欽差善才謂我太宗曰:"黃氏性險無溫色,正類負債之女。" 歲癸卯,欽差海壽謂上曰:"黃氏行路之時,腹痛至甚,吾等見則以鄕言言腹痛,必慙而入內。" 及帝之崩,宮人殉葬者,三十餘人,當死之日,皆餉之於庭。 餉輟,俱引升堂,哭聲震殿閣。 堂上置木小床,使立其上,掛繩圍於其上,以頭納其中,遂去其床,皆雉經而死。 韓氏臨死,顧謂金黑曰:"娘吾去! 娘吾去!" 語未竟,旁有宦者去床,乃與崔氏俱死。 諸死者之初升堂也,仁宗親入辭訣,韓氏泣謂仁宗曰:"吾母年老,願歸本國。" 仁宗許之丁寧,及韓氏旣死,仁宗欲送還金黑,宮中諸女秀才曰:"近日魚、呂之亂,曠古所無。 朝鮮國大君賢,中國亞匹也。 且古書有之,初佛之排布諸國也,朝鮮幾爲中華,以一小故,不得爲中華。 又遼東以東,前世屬朝鮮,今若得之,中國不得抗衡必矣。 如此之亂,不可使知之。" 仁宗召尹鳳問曰:"欲還金黑,恐洩近日事也,如何?" 鳳曰:"人各有心,奴何敢知之?" 遂不送金黑,特封爲恭人。 初,帝寵王氏,欲立以爲后,及王氏薨,帝甚痛悼,遂病風喪心,自後處事錯謬,用刑慘酷。 魚、呂之亂方殷,雷震奉天、華蓋、謹身三殿俱燼。 宮中皆喜以爲:"帝必懼天變,止誅戮。" 帝不以爲戒,恣行誅戮,無異平日。 後尹鳳奉使而來,粗傳梗槪,金黑之還,乃得其詳。」

\subsection{洪武}

\begin{longtable}{|>{\centering\scriptsize}m{2em}|>{\centering\scriptsize}m{1.3em}|>{\centering}m{8.8em}|}
  % \caption{秦王政}\
  \toprule
  \SimHei \normalsize 年数 & \SimHei \scriptsize 公元 & \SimHei 大事件 \tabularnewline
  % \midrule
  \endfirsthead
  \toprule
  \SimHei \normalsize 年数 & \SimHei \scriptsize 公元 & \SimHei 大事件 \tabularnewline
  \midrule
  \endhead
  \midrule
  三五年 & 1402 & \tabularnewline
  \bottomrule
\end{longtable}

\subsection{永乐}

\begin{longtable}{|>{\centering\scriptsize}m{2em}|>{\centering\scriptsize}m{1.3em}|>{\centering}m{8.8em}|}
  % \caption{秦王政}\
  \toprule
  \SimHei \normalsize 年数 & \SimHei \scriptsize 公元 & \SimHei 大事件 \tabularnewline
  % \midrule
  \endfirsthead
  \toprule
  \SimHei \normalsize 年数 & \SimHei \scriptsize 公元 & \SimHei 大事件 \tabularnewline
  \midrule
  \endhead
  \midrule
  元年 & 1403 & \tabularnewline\hline
  二年 & 1404 & \tabularnewline\hline
  三年 & 1405 & \tabularnewline\hline
  四年 & 1406 & \tabularnewline\hline
  五年 & 1407 & \tabularnewline\hline
  六年 & 1408 & \tabularnewline\hline
  七年 & 1409 & \tabularnewline\hline
  八年 & 1410 & \tabularnewline\hline
  九年 & 1411 & \tabularnewline\hline
  十年 & 1412 & \tabularnewline\hline
  十一年 & 1413 & \tabularnewline\hline
  十二年 & 1414 & \tabularnewline\hline
  十三年 & 1415 & \tabularnewline\hline
  十四年 & 1416 & \tabularnewline\hline
  十五年 & 1417 & \tabularnewline\hline
  十六年 & 1418 & \tabularnewline\hline
  十七年 & 1419 & \tabularnewline\hline
  十八年 & 1420 & \tabularnewline\hline
  十九年 & 1421 & \tabularnewline\hline
  二十年 & 1422 & \tabularnewline\hline
  二一年 & 1423 & \tabularnewline\hline
  二二年 & 1424 & \tabularnewline
  \bottomrule
\end{longtable}


%%% Local Variables:
%%% mode: latex
%%% TeX-engine: xetex
%%% TeX-master: "../Main"
%%% End:
