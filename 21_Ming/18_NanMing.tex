%% -*- coding: utf-8 -*-
%% Time-stamp: <Chen Wang: 2019-12-26 15:09:40>

\section{南明\tiny(1644-1662)}

\subsection{生平}

南明(1644年-1662年),中國朝代,是甲申之變後,明朝皇族與官員在中國南方相繼成立的明朝政權,為時十八年[註 1]。南明主要勢力有四系王,分別是福王弘光帝朱由崧、魯王監國朱以海、唐王隆武帝朱聿鍵與紹武帝朱聿鐭、桂王永曆帝朱由榔等。

1644年明朝首都北京被李自成攻陷[1][2],南明大臣意圖擁護皇族北伐。經過多次討論後由鳳陽總督馬士英與江北四鎮高傑、黃得功、劉澤清與劉良佐擁護明思宗的堂兄弟福王朱由崧稱帝,即弘光帝,国号依旧为大明,史称南明或后明。1645年清軍攻破揚州[3][4][5],弘光帝逃至蕪湖被逮,後被送到北京殺害[6]。弘光帝死後,魯王朱以海於浙江紹興監國;而唐王朱聿鍵在鄭芝龍等人的擁立下,於福建福州稱帝,即隆武帝。然而這兩個南明主要勢力互不承認彼此地位,而互相攻打。1651年在舟山群島淪陷後,魯王朱以海在張名振、張煌言陪同下,赴廈門依靠鄭成功,不久病死在金門。隆武帝屢議出師北伐,然而得不到鄭芝龍的支持而終無所成。1646年,清軍分別占領浙江與福建,魯王朱以海逃亡海上,隆武帝於汀州逃往江西時被俘而死。鄭芝龍向清軍投降,由於其子鄭成功起兵反清而被清廷囚禁。朱聿鍵死後,其弟朱聿鐭在廣州受蘇觀生及廣東布政司顧元鏡擁立稱帝,即紹武帝,於同年年底被清將李成棟攻滅。同時間桂王朱由榔於廣東肇慶稱帝,即永曆帝[6]。

1646年永曆帝獲得瞿式耜、張獻忠餘部李定國、孫可望等勢力的加入以及福建鄭成功勢力的支援之下展開反攻。同時各地降清的原明軍將領先後反正,例如1648年江西金聲桓、廣東李成棟、廣西耿獻忠與楊有光率部反正,一時之間南明收服華南各省。然而於同年,清將尚可喜率軍再度入侵,先後占領湖南、廣東等地。兩年後,李定國、孫可望與鄭成功發動第二次反攻,其中鄭成功一度包圍南京。然而,各路明軍因為距離互相難以照應,內部又發生孫可望等人的叛變,第二次反攻以節節敗退告終。1661年,清軍三路攻入云南,永曆帝流亡缅甸首都曼德勒,被缅甸王莽達收留。後吴三桂攻入缅甸,莽達之弟莽白乘机发动政变,杀死其兄後继8月12日,莽白發動咒水之难,杀盡永曆帝侍從近衛[7],永曆帝最後被吴三桂以弓弦絞死,南明正式滅亡[6]。此時反清勢力只剩夔東十三家軍與在金廈及台灣的明鄭王朝。

明崇禎十七年(1644年)正月,李自成在西安稱帝,建國「大順」,之後向北京進兵,三月十九攻克北京,崇禎皇帝朱由檢殉國,明朝宗室及遺留大臣多輾轉向南遁走。此时李自成的「大順」政權大体據有淮河以北原明朝故地,張獻忠於八月成立的「大西」政權則據四川一帶,清朝政權則據有山海关外的现今东北地区,且控制蒙古诸部落,而明朝的殘餘勢力則據有淮河以南的半壁江山。

此时明朝留都南京的一些文臣武將決計擁立朱家王室的藩王,重建明朝,然後揮師北上;但具体拥立何人则发生争议。根据“皇明祖训”,有嫡立嫡、无嫡立长,在当时明神宗长子光宗一脉(其後繼者是熹宗天啟皇帝和思宗崇禎皇帝)已无人能继位,而次子朱常漵甫生即死,三子朱常洵虽已亡故,但其长子朱由崧仍健在的情况下,按照兄終弟及的順序,第一人選為福王朱由崧;而钱谦益等东林党人由於之前的「國本之爭」事件,心存芥蒂,违背了东林党在国本之争中的立场,以立贤为名擁立神宗弟弟朱翊镠之子潞王朱常淓[8][9];史可法则主张既要立贤也要立亲,拥立神宗七子桂王朱常瀛。但最终福王朱由崧在卢九德的帮助下,获得了南京政权主要武装力量江北四镇高杰、黄得功、刘良佐和刘泽清,以及中都凤阳总督马士英的支持,成为最终的胜利者。五月初三,朱由崧監國于南京,五月十五 (1644年6月19日) 日即皇帝位,改次年為弘光,是為明安宗。南明時代自此開始。弘光帝的基本國策以「聯虜平寇」為主,謀求與清軍連合,一起消滅以李自成、张献忠为代表的農民軍。

明朝南渡前後,大顺已被多爾袞與吳三桂的聯軍击溃,李自成先后丢失北京和西安,退往湖北。弘光元年(1645年)三月,多尔衮将军事重心东移,命多铎移师南征。此时弘光政權內部正進行著激烈的黨爭,爆发太子案,駐守武昌的左良玉不愿与李自成正面交战,以「清君側」为名,順长江東下争夺南明政权。馬士英被迫急調江北四鎮迎擊左軍,致使面对清军的江淮防線陷入空虛。史可法时在揚州虽有督師之名,却实无法调动四镇之兵。一月之中,清軍破徐州,渡淮河,兵臨揚州城下。四月廿五,揚州城陷,史可法不屈遇害。隨後,清軍渡過長江,攻克鎮江。弘光帝出奔蕪湖。五月十五众大臣獻南京投降清兵;五月廿二弘光帝被虜獲,送往北京處死,弘光帝在位仅一年,即覆滅。

南京失陷後,又有杭州的潞王朱常淓(1645年)、金陵的崇禎太子朱慈烺(可能是貌似太子的王之明。1645年)、撫州的益王朱慈炲(1645年)、福州的唐王朱聿鍵(1645-1646年)、紹興的魯王朱以海(1645-1653年)、桂林的靖江王朱亨嘉(1645年)等監國政權先後建立,其中唐王朱聿鍵受鄭芝龍等人在福州擁立,登極稱帝,改元隆武,是為明紹宗。這時清朝再次宣佈薙髮令,江南一帶掀起了反薙髮的抗清鬥爭,清軍後方發生動亂,一時無力繼續南進。但南明內部嚴重的黨派鬥爭與地方勢力跋扈自雄,且隆武帝與魯王政權不但沒有利用這種有利形勢,發展抗清鬥爭,反而在自己之間為爭正統地位而形同水火,各自為戰,所以當1646年清軍再度南下時,先後為清軍所各各擊滅。魯王在張煌言等保護下逃亡海上,在沿海一帶繼續抗清;隆武帝則被清軍俘殺。

11月,在廣州和肇慶又成立了兩個南明政權:隆武帝之弟唐王朱聿鐭(1646年)繼位於廣州,改明年为紹武元年;桂王朱由榔(1646-1662年)稱帝於肇慶,改元永曆,是為明昭宗。紹武、永曆二帝也不能團結,甚至大動干戈,互相攻伐。紹武政權僅存在40天就被清軍消滅。揭陽的益王朱由榛(1647年)、夔州的楚王朱容藩(1649年)稱監國與永曆帝爭立。鄭成功也在南澳一度立淮王朱常清(1648年)為監國,後廢。永曆帝在清軍進逼下逃入廣西。

正當南明政權一個接一個地覆亡,形勢萬分危急之際,大順農民軍餘部出現在抗清鬥爭最前線,挽救了危局。自李自成于1645年战死于九宫山後,他的餘部分為二支,分別由郝搖旗、劉體純和李過、高一功率領,先後進入湖南,與明湖廣總督何騰蛟、湖北巡撫堵胤錫聯合抗清。1647年,郝搖旗部護衛逃來廣西的永曆帝居柳州,並出擊桂林。年底,大敗清軍於全州,進入湖南。次年,大順軍餘部又同何騰蛟、瞿式耜的部隊一起,在湖南連連取得勝利,幾乎收復了湖南全境。這時,廣東、四川等地的抗清鬥爭再起,清江西提督金聲桓、清广东提督李成棟、清广西巡抚耿献忠、清大同总兵姜镶、清延安营参将王永强、清甘州副将米喇印先後反正回归明朝,清軍後方的抗清力量也發動了廣泛的攻勢。一時間,永曆政權名义控制的區域擴大到了雲南、貴州、廣東、廣西、湖南、江西、四川七省,还包括北方山西、陕西、甘肃三省一部以及东南福建和浙江两省的沿海岛屿,出現了南明時期第一次抗清鬥爭的高潮。

但永曆政權內部仍然矛盾重重,各派政治勢力互相攻訐,農民軍也倍受排擠打擊,不能團結對敵,這就給了清軍以喘息之機。1649-1650年,何騰蛟、瞿式耜先後在湘潭、桂林的戰役中被俘杀,清軍重新佔領湖南、廣西;其他剛剛收復的失地也相繼丟掉了。不久,李過之子李來亨等農民軍將領率部脫離南明政府,轉移到巴東荊襄地區組成夔東十三家軍,獨立抗清。這支部隊一直堅持到1664年。

綜觀1645-1651年間,南明軍與清軍作戰中,敗多勝少,大批南明的軍隊先後降清。先後丟失了江蘇、安徽、浙江、江西、福建、兩廣、兩湖等等領地,地盤盡失。直到以孫可望為主的大西軍加入,再次改變了整個局勢。

張獻忠于1646年战死後,以其义子孫可望、李定國、刘文秀、艾能奇等人為主的大西軍残部自1647年進佔雲南、貴州二省。1652年,南明永曆政权接受孫可望和李定國的建议聯合抗清建議,定都安龙。不久,以大西军餘部为主体的南明軍對清軍展開了全面反擊。李定國率軍8萬東出湖南,取得靖州大捷,收复湖南大部;随后南下广西,取得桂林大捷,击毙清定南王孔有德,收复广西全省;然后又北上湖南取得衡阳大捷,击毙清敬谨亲王尼堪,天下震动。同時,劉文秀亦出擊四川,取得叙州大捷、停溪大捷,克復川南、川东。孙可望也亲自率军在湖南取得辰州大捷。東南沿海的張煌言、郑成功等的抗清軍隊也乘机發動攻勢,接连取得磁灶大捷、钱山大捷、小盈岭大捷、江东桥大捷、崇武大捷、海澄大捷的一连串胜利,並接受了永曆封號。一時間,永曆政權名义控制的區域恢复到了雲南、貴州、廣西三省全部,湖南、四川两省大部,廣東、江西、福建、湖北四省一部,出現了南明時期第二次抗清鬥爭的高潮。

之后,劉文秀於四川用兵失利,在保宁战役中被吳三桂侥幸取胜。而孫可望妒嫉李定國桂林、衡州大捷之大功,逼走李定國,自己统兵却在宝庆战役中失利。东南沿海的郑成功也在漳州战役中失利。所以明军在四川、湖南、福建三个战场上没能扩大战果,陷入了与清军相持的局面。之後李定國與鄭成功聯絡,於1653年、1654年率軍兩次進軍廣東,約定与鄭會師廣州,一舉收復廣東,但鄭軍屢誤約期,加上瘟疫流行,导致肇庆战役和新会战役没能成功。但郑成功部队并没有闲着,1656年,郑军取得泉州大捷,1657年又取得护国岭大捷。

永曆十年(1656年),孫可望祕謀篡位,引發了南明內部一場内讧,李定國擁永曆帝至雲南,次年大敗孫可望,孫可望勢窮降清。孫可望降清後,西南軍事情報盡供清廷,雲貴虛實盡為清軍所知。永曆十二年(1658年)四月,清軍主力從湖南、四川、廣西三路進攻貴州。年底吳三桂攻入雲南,次年正月,下昆明,進入雲南,永曆帝狼狽西奔,進入緬甸(東吁王朝)。李定國率全軍設伏於磨盤山,企圖一舉殲滅敵人追兵,結果因內奸洩密导致未能大获全胜,南明軍精銳損失殆盡,此即磨盘山血战。這時鄭成功趁清軍主力大舉攻擊西南之際,率領十餘萬大軍北伐,接连取得定海关大捷、瓜州大捷、镇江大捷的胜利,一度兵临南京城下,然而鄭軍中了清軍緩兵之計,最终失败,撤回廈門。清军派大军围攻厦门,企图一举歼灭郑成功,但郑成功沉着应战,取得厦门大捷的胜利,稳定了东南沿海局势。永曆十五年(1661年),吳三桂率清軍入緬,索求永曆帝,十二月緬甸東吁王朝國王平達力(莽達)將永曆交予清軍,次年四月永曆帝與其子哀愍太子朱慈煊等被吳三桂處死于昆明。七月,李定國在真臘得知永曆帝死訊,亦憂憤而死。而同年五月,鄭成功亦於臺灣急病而亡。

此后郑氏政权未再拥立皇帝或朱氏监国,而是继续奉永历为正朔。1683年,延平郡王鄭克塽降清,清军占领台湾,宁靖王朱术桂自杀殉国,标志着大明最后一个政权的覆灭。

南明時期,安南、日本、琉球、呂宋、占城也曾派使者入貢[10]。隆武元年也曾頒登基詔書予琉球,並記載於琉球《歷代寶案》一書。

南明弘光帝曾以對等的禮儀派使者左懋第詔諭,並稱順治帝為清國可汗。在詔書中,弘光帝提出四件事:要安葬崇禎帝及崇禎皇后、以山海關為界,關外土地給予清朝、每年十萬歲幣,並「犒金千兩、銀十萬兩、絲緞萬匹、犒銀三萬兩」、建國任便。[10]意圖令南明和清朝共存,通好議和。不過左懋第到北京被囚,使事失敗。

\subsection{安宗\tiny(1644-1645)}

\subsubsection{生平}

明安宗朱由崧(1607年9月5日-1646年7月1日),又稱弘光帝,為南明首位皇帝,原為福王。朱由崧是明神宗朱翊钧之孙,福忠王朱常洵之子。他是明熹宗朱由校、明思宗朱由檢的堂兄弟。思宗殉国後,朱由崧在南京即位,改元弘光,在位僅一年。弘光元年清军南攻,朱由崧被俘,押往北京,翌年被處決。南明永历帝为其上庙号安宗,谥号奉天遵道宽和静穆修文布武溫恭仁孝簡皇帝。

朱由崧小字福八,明神宗孙,福忠王朱常洵庶长子。万历三十五年七月乙巳生于福王京邸,生母姚氏。万历四十二年随福王朱常洵就藩于洛阳。万历四十八年七月甲辰封德昌王,后进封福王世子。

崇祯十四年正月,流賊李自成陷洛阳,福王常洵缒城出,藏匿于迎恩寺,后被搜出,遇害。朱由崧缒城逃脱,前往怀庆避难,崇祯十六年五月袭封福王。崇祯帝手择宫中玉带,遣内使赐之。

崇祯十七年正月,怀庆闻警,朱由崧逃亡卫辉,投奔潞王朱常淓。三月初四卫辉闻警,朱由崧随潞王逃往淮安,与南逃的周王、崇王一同寓居于湖嘴舟中。三月十一日周王朱恭枵薨于舟上,三月十八日福王上岸,住在杜光绍园中。三月十九日李自成陷北京,崇禎帝自縊,是為甲申之變。廿九日,消息传至淮安。

四月崇祯帝自盡的消息,传至南京,北京沦陷後,南京以及南方各省仍在明朝的控制之下。南京诸臣皆認為國不可一日無君,议立新帝。但對大寶誰屬,則有一番論戰。

从血统上来说,崇祯帝殉国,其子太子朱慈烺及永王朱慈炤、定王朱慈炯陷入清军之手,而崇禎帝父明光宗朱常洛仅有天啟帝、崇祯帝二子,天啟帝無子,而故应从崇禎帝祖父明神宗之子、光宗诸弟中选择。明神宗福王常洵为第三子,瑞王常浩为第五子,惠王常润为第六子,桂王常瀛为第七子,以常洵居长。朱由崧为朱常洵长子,因此在崇祯太子及定、永二王无法至南京继位的情况下,福王本为第一順位。然而東林黨人卻持相反意見,他們恐朱由崧即位后追究昔日“三案”及國本之爭攻讦郑贵妃(朱由崧祖母)之事,主张立明神宗之侄潞王朱常淓。史可法并称福王“在藩不忠不孝,恐难主天下”。四月二十六日,张慎言、高弘图、姜曰广、李沾、郭维经、诚意伯刘孔昭、司礼太监韩赞周等在朝中会议,李沾、刘孔昭、韩赞周议立福王,议遂定以福王继统,告庙并修武英殿。鳳陽總督馬士英與江北四鎮黃得功、高傑、劉良佐、劉澤清等人前往淮安迎接朱由崧。四月二十七日甲申,南京礼部率百司迎福王于儀真。

崇祯十七年四月二十八日乙酉,朱由崧至浦口,魏国公徐弘基等渡江迎接。翌日舟泊观音门燕子矶。四月三十日丁亥,南京百官迎见朱由崧于龙江关舟中,请其為監國。朱由崧身穿角巾葛衣,坐于卧榻之上,推说自己未携宫眷一人,准备避难浙东。众臣力劝,朱由崧乃同意。

五月初一戊子,朱由崧骑马自三山门环城而东,拜谒孝陵和懿文太子陵,随后经朝阳门入东华门,谒奉先殿,出西华门,以南京内守备府为行宫。五月初二群臣至行宫劝进,朱由崧以太子及定王、永王不知下落,且瑞王、惠王、桂王均为叔父行,应择贤迎立。诸臣再三劝进,乃依明代宗故事监国。五月初三庚寅自大明门入大内,至武英殿行监国礼。是日吴三桂引清摄政王多尔衮入北京。

崇祯十七年五月十五日壬寅,朱由崧即皇帝位于武英殿,以次年为弘光元年。其国号依旧为“大明”,史称“南明”。

朱由崧即位后,于六月戊午追封祖母郑贵妃为孝宁太皇太后,父福忠王朱常洵为贞纯肃哲圣敬仁毅恭皇帝(后改谥孝皇帝),立庙于南京,墓园称熙陵。上嫡母邹氏尊号为恪贞仁寿皇太后,生母姚氏为孝诚端惠慈顺贞穆皇太后。追封洛阳城陷时遇害的胞弟颍上王朱由榘为颍王,谥曰冲。六月辛酉上崇祯帝庙号为思宗,谥号烈皇帝。七月己丑追复懿文太子帝号,追崇建文帝、景泰帝庙号谥号。

东林党人编撰的史书说朱由崧生性暗弱,不忠不孝,荒淫无耻,政事则悉委于马士英、阮大铖。马、阮二人日以卖官鬻爵、报撼私仇为事,导致南明政事萎靡,不断发生内讧;而名臣李清则力为弘光辩冤,说这些记载都是谣言,又说弘光帝很少接近女色。在外以史可法督师江北,设淮、扬、凤、庐四镇,以黄得功、刘良佐、刘泽清、高杰为总兵统领,南明出现军阀化的趋势。前線將領不但因爭權而互相攻擊,也有掠奪平民的行為。

朱由崧即位后,下令选淑女入宫,派宦官于南京城中四出搜巷,凡是有女之家,必以黄纸贴额,持之而去,南京城中骚动。朱由崧又下令修西宫西一路为慈禧殿,以安置继母邹太后。当年八月邹太后自河南至南京,八月十四日谕户、兵、工三部“太后光临,限三日内搜括万金,以备赏赐”。八月十六日御用监又令造龙凤床座、床顶架、宫殿陈设金玉等项,越数十万两。造皇后冠,命内臣采购猫眼石、祖母绿及大珠重一钱以上者百余颗。崇祯十七年除夕,弘光帝独坐兴宁宫中,愀然不乐。太监韩赞周问道:“宫殿新落成,皇上应当欢喜,而闷闷不乐,是思念皇兄吗?”弘光帝不应,继而回答说:“梨园殊少佳者”。弘光元年(1645年)正月,弘光帝又下令修南京奉先殿、午门及左右掖门,并派太监田成至杭州、嘉兴二府选淑女。

崇祯十七年九月初三,弘光帝下令为北京殉难诸臣上谥号,计文臣二十一人、勋臣二人、戚臣一人。随后又给郢国公冯国用、宋国公冯胜、济国公丁德兴、德庆侯廖永忠、长兴侯耿炳文等开国功臣追上谥号;给方孝孺、齐泰、黄子澄、陈迪、景清、卓敬、练子宁等建文朝死难诸臣,蒋钦、陆震等正德朝死谏诸臣,左光斗、周朝瑞、周宗建、袁化中、顾大章、周起元等天启朝死珰难诸臣上谥号。

弘光元年三月初一甲申,有自称崇祯太子朱慈烺者至南京,朱由崧命令将其关入兵马司监狱,后命百官审北来太子于午门外,终裁断为伪太子王之明,是為崇禎太子案。三月庚申,宁南侯左良玉乃举兵于武昌,以“救太子、诛士英”为名顺流而下,黄得功、阮大铖率兵御之,南明发生内讧。正值此时,清军在豫王多铎率领下大举南下,攻陷归德、颍州、太和、泗州等地。

弘光元年四月辛未,清军围攻江北重镇扬州。督師江北的兵部尚書史可法率城中百姓抵御清军,清军围困百日,损失惨重。史可法急忙向朝廷求援,但卻因為鎮將們個個擁兵自重、意圖觀望,最終揚州在被围五天后沦陷。清军攻破扬州之後进行了十天屠杀,史称“扬州十日”。四月甲子,弘光帝在南京贡院选淑女,七十人中选中一人,即阮大铖的侄女。四月壬戌,杭州送来淑女五十人,弘光帝选中周姓一人,王姓一人。

弘光元年五月初八己丑,清军自瓜洲渡江,镇江巡抚杨文骢逃奔苏州,靖虏伯郑鸿逵逃入東海,总兵蒋云台投降。南京闭城门。五月初十辛卯,朱由崧传旨放归所选淑女,当天午夜尤召梨园入宫演剧。翌日凌晨二漏时,朱由崧率内官四五十人骑马出通济门,莫知所踪。天亮后百官入朝,见宫女、内臣、优伶杂沓逃奔西华门外,方知弘光帝已出逃。南京城内大哗,马士英携邹太后出奔,市民救北来太子出狱,扶其入宫,在武英殿即位。五月十二日癸巳,朱由崧至太平府,以按察院为行宫,寻即移驾芜湖,投奔靖国公黄得功军营。五月十五日丙申,清军入南京,魏国公徐文爵、保国公朱国弼、灵璧侯汤国祚、定远侯邓文郁,及尚书钱谦益、大学士王铎、都御史唐世济等人剃髮降清。

清军攻克南京后,多铎命降将刘良佐带清兵追击弘光帝。五月二十二日癸卯,总兵田雄、马得功、丘钺、张杰、黄名、陈献策冲上御舟,劫持弘光帝,将其献给清军。豫王多铎命去锁链,以红绳捆绑。五月二十五日丙午,朱由崧乘无幔小轿入南京聚宝门,头蒙缁素帕,身衣蓝布袍,以油扇掩面,两妃乘驴随后,夹路百姓唾骂,有投瓦砾者。多铎在灵璧侯府设宴,命朱由崧居于北来太子之下。宴罢,拘弘光帝于江宁县署。

弘光元年闰六月,唐王朱聿鍵即位于福州,改元隆武,遥上朱由崧尊号为「上皇圣安皇帝」。当年九月甲寅,朱由崧与皇太后邹氏、潞王朱常淓等人被押送至燕京,安置居住。由滿清太医院,日时馈宴,朱由崧酣饮极乐。

顺治三年(1646年,隆武二年)四月九日,有人向清摄政王多尔衮告发,称燕京居住的故明衡王、荆王欲谋反。五月甲子,弘光帝与秦王朱存極、晋王朱審烜、潞王朱常淓、荆王朱慈煃、徳王朱由栎、衡王朱由棷等十七人被斬首於菜市口(一说弘光帝以弓弦絞死)。

朱由崧王妃黄氏之弟黄調鼎购得棺木,与黄妃合葬于河南孟津县东山头村。

弘光帝凶讯南传后,监国魯王朱以海上谥号为赧皇帝,不久又上庙谥为质宗安皇帝。永曆帝立,于永历十一年四月改弘光帝廟號曰安宗,谥号奉天遵道宽和静穆修文布武溫恭仁孝簡皇帝。

根据明末清初笔记记载,朱由崧是个十分昏庸腐朽的君主,整日只知吃喝玩乐,沉湎于酒色之中,不理朝政。在其即位之前,史可法曾寫信給馬士英說明「福王七不可立」──貪、淫、酗酒、不孝、虐下、無知和專橫。由史可法、張慎言、高弘圖等17人簽名送與馬士英。後人称其为明朝及南明最昏庸的帝王,唯知享樂,不問政事,沉湎酒色,荒淫透頂。然而細檢史籍可知竟傳聞難據,推其緣由多為東林黨人因國本之爭對福王藩一系的成見所致。而其本來的經歷顯現的是並非昏庸且頗有個性的政治家形象。如曾任弘光朝給事中李清《三垣筆記》、《南渡錄》及《甲申日記》對荒淫縱欲之事,且加辯誣。此外,朱由崧替靖難之變殉難的明惠帝一系君臣予以平反,並貶抑當時擴大迫害的陳瑛。因此其政治得失尚有爭議。

钱海岳《南明史》评价弘光帝“北京颠覆,上膺鼎籙,丰芑奠磐,徵用俊耆。卷阿翙羽,相得益彰。故初政有客观者。性素宽厚,马、阮欲以《三朝要典》起大狱,屡请不允。观其谕解良玉,委任继咸,词婉处当;拒纳银赎罪之议,禁武臣罔利之非,皆非武、熹昏騃之比。顾少读书,章奏未能亲裁,政事一出士英,不从中制,坐是狐鸣虎噬,咆哮恣睢,纪纲倒持。及大铖得志,众正去朝,罗罻高张,党祸益烈。上燕居神功,辄顿足谓士英误我,而太阿旁落,无可如何,遂日饮火酒,亲伶官优人为乐,卒至触蛮之争,清收渔利。时未一朞,柱折维缺。故虽遗爱足以感其遗民,而卒不能保社稷云。”

\subsubsection{弘光}

\begin{longtable}{|>{\centering\scriptsize}m{2em}|>{\centering\scriptsize}m{1.3em}|>{\centering}m{8.8em}|}
  % \caption{秦王政}\
  \toprule
  \SimHei \normalsize 年数 & \SimHei \scriptsize 公元 & \SimHei 大事件 \tabularnewline
  % \midrule
  \endfirsthead
  \toprule
  \SimHei \normalsize 年数 & \SimHei \scriptsize 公元 & \SimHei 大事件 \tabularnewline
  \midrule
  \endhead
  \midrule
  元年 & 1645 & \tabularnewline
  \bottomrule
\end{longtable}

\subsection{绍宗\tiny(1645-1646)}

\subsubsection{生平}

明紹宗朱聿鍵(1602年5月25日-1646年10月6日),又稱隆武帝,小字長壽,南明第二代皇帝,原為唐王,為明太祖朱元璋二十三子唐王朱桱的八世孫(與明神宗同輩份),祖父唐端王朱碩熿,父為唐王之子朱器墭,母宣皇后毛氏。1644年,明思宗在北京自缢,1645年弘光帝被俘,鄭芝龍、黃道周等人扶朱聿鍵於福州登基称帝,改元為隆武並與同年開鑄「隆武通寶」,而弘光帝在翌年才被清廷所殺。

1646年,清军入福建,隆武帝在汀州被擄殺,享年44岁。永曆帝即位后初上尊谥思文皇帝,永历十一年上廟號紹宗,改谥号為配天至道弘毅肅穆思文烈武敏仁廣孝襄皇帝。朱聿键自奉甚俭,品格在南明诸君中是少見的優良。黄道周描述了隆武帝的为人:“今上不饮酒,精吏事,洞达古今,想亦高、光而下之所未见也。”

朱聿键为明太祖第二十三子唐定王朱桱的后裔,系太祖九世孙。万历三十四年四月丙申生于南阳唐王府,母妃毛氏。其祖父唐端王朱碩熿惑于嬖妾,不喜愛朱聿键的父親世子朱器墭,把朱器墭父子一起囚禁在承奉司內,欲立爱子。崇祯二年(1629年),朱器墭疑似被其弟福山王朱器塽、安陽王朱器埈毒死,朱碩熿讳言其事,但经守道陈奇瑜奏请,朱聿鍵被明廷立为唐國世孙,不再被囚禁,同年朱碩熿也去世。

崇禎五年(1632年)朱聿鍵繼為唐王,封地南阳。崇祯帝赐其《皇明祖训》、《大明会典》、《四书》、《五经》、《二十一史》、《資治通鑒綱目》、《孝经》、《忠經》等书。朱聿鍵在王府内起高明楼,延请四方名士。

崇祯九年(1636年)七月初一,朱聿鍵杖殺叔父福山王朱器塽、杖傷叔父安阳王朱器埈,为其父朱器墭当年被毒死一事报仇。当年八月,清兵入塞,克宝坻,直逼北京,京师戒严。朱聿鍵上疏请勤王,不许,乃自率护军千人北上勤王。行至裕州,巡抚杨绳武上奏,崇祯帝勒令其返回,后朱聿键因与农民军相遇交锋,两名太监被杀,乃班师回南阳。冬十一月下部议,废为庶人,幽禁在凤阳之高墙。崇禎帝改封其弟朱聿鏼为唐王。

朱聿键高墙圈禁期间,凤阳守陵太监石应诏索贿不得,用墩锁之法折磨之,朱聿键病苦几殆。后凤阳巡抚路振飞入高墙见之,向崇祯帝上疏陈高墙监吏凌虐宗室之状,请加恩于宗室。乃下旨誅殺石应诏。

崇禎十四年(1641年),李自成攻陷南阳,杀死朱聿鏼。

崇禎十七年(1644年),李自成攻陷北京,即甲申之變,崇祯帝自缢,南京諸臣拥从洛阳逃出的福王子朱由崧为帝,在南京即位,改年號弘光,实行大赦。在广昌伯刘良佐奏请下,囚於鳳陽的朱聿键也被释,并改封为南阳王。南京礼部请恢复唐王故爵,朱由崧不允,并令朱聿鍵迁至广西平乐(今桂林南),但朱聿鍵贫病不能行。

清朝順治二年、南明弘光元年(1645年)五月,朱聿鍵赴平乐途中,在苏州闻清军已破南京,俘虜了弘光帝朱由崧,朱聿鍵遂至嘉兴避难。六月辛酉,朱聿键至杭州,遇潞王朱常淓,奏请其监国,不听;请朝陈方略,不允。当时鎮江總兵官鄭鴻逵、戶部郎中蘇觀生至杭州,与朱聿键谈及国难,泣下沾襟。后朱聿键被郑鸿逵护送,前往福建。途中在浙江衢州闻得潞王朱常淓已在杭州降清,于是南安伯鄭芝龍、巡撫都御史張肯堂與禮部尚書黃道周等商议奉朱聿鍵为監國。

弘光元年六月己卯(二十八日),朱聿鍵在福建建宁,以唐王的身分监国。闰六月丁亥(初七)至福州,以南安伯府为行宫。

闰六月丁未,朱聿鍵於福州称帝,遙尊朱由崧為「上皇聖安皇帝」,宣布從七月初一起,改弘光年号為隆武元年,改福建布政司称福京行在,改福州府為天興府,改布政司为行殿,建行在太庙、社稷及唐国宗庙。升鄭芝龍为平虏侯、鄭鴻逵為定虏侯,封鄭芝豹为澄济伯、鄭彩為永胜伯。以何吾驺为首辅,以黄道周为吏部尚书、武英殿大学士,蒋德璟为户部尚书、文渊阁大学士,朱继祚为礼部尚书、东阁大学士,曾樱为工部尚书、东阁大学士,黄鸣俊、李光春、蘇觀生等人为礼、兵各部左右侍郎兼东阁大学士。

朱聿键即帝位后,上高曾祖父四代帝号,高祖唐敬王朱宇温为惠皇帝,曾祖唐顺王朱宙栐为顺皇帝,祖父唐端王朱碩熿为端皇帝,父唐裕王(追封)朱器墭为宣皇帝。四代祖妣皆追封皇后。封弟朱聿𨮁为唐王,封国南宁;升叔德安王朱器䵺为邓王;追封弟朱聿𨧨为陈王,子朱琳渼为陈王世子。遥上弘光帝尊号“圣安皇帝”。隆武元年七月下令将嘉靖年间皇极殿、中极殿、建极殿三殿之名恢复为奉天殿、华盖殿、谨身殿,各衙门前加“行在”二字。

当时,在绍兴还有鲁王朱以海建立的小朝廷,亦自稱「監國」。清军攻绍兴,朱以海派使者前来福州向朱聿鍵求援兵。信上称朱聿键为“皇伯父”,而未称“陛下”,朱聿键怒,令杀鲁王信使。

隆武二年/清顺治三年(1646年)五月,清将博洛贝勒率兵征浙、闽。七月庚申清兵陷金华,八月甲申陷建宁,乙未过仙霞关,武毅伯施天福、武功伯陈秀、靖安伯郭熺降清。郑芝龙向清軍投降,隆武政权很快灭亡。楊鳳苞稱“福京之亡,亡于鄭芝龍之通款”。

隆武二年八月甲午,隆武帝率宫嫔自延平出狩,欲逃往江西避難。八月庚申至汀州,以府署为行宫。八月辛丑五鼓,有清军八十三骑伪装成扈跸者叩城,守城者开汀州丽春门。骑兵突袭行宫,杀福清伯周之藩、总兵王凉武等人。时隆武帝腹饥,命内官市二汤圆以进,方举箸,清兵发矢,隆武帝后背中箭,崩,年四十五。百姓敛葬于罗汉岭。另有说法称隆武帝被俘后不食而死,或称崩于福京天兴府,或称崩于建宁。

八月壬戌福京天兴府陷落,阳曲王朱敏渡、松滋王朱俨𨫃、翼城王朱弘橺、奉新王朱常涟遇害。十月辛卯漳州陷落。十一月,侍郎蘇觀生立隆武帝之弟朱聿𨮁於廣東省廣州府番禺縣,改元紹武,觀生自為宰相。當時已經稱帝的永曆帝,希望紹武帝取消帝號,蘇觀生大怒,以新歸降的海盜加上四處捕捉來的民兵征討永曆,大勝,誰知滿清將領佟養甲、李成棟已取潮州、惠州,兵臨廣州,蘇觀生死於戰事,清兵隨即俘獲了紹武帝,紹武自縊。

永曆帝即位后,一直聽到謠言說隆武帝化妝隱居不出,上尊号「上皇思文皇帝」,遣間諜打聽隆武帝消息,傳言隆武帝潜至安溪縣妙峯为僧,或称在汀州府单骑逃出,藏于乡民蒋氏家中,清兵離開以後,前往大帽山出家。永历五年曾遣侍郎王命璿探訪,又不得,永历十一年乃确信隆武帝已死,立廟號绍宗,諡號配天至道弘毅肃穆思文烈武敏仁广孝襄皇帝。

隆武帝死后百姓敛葬于福州罗汉岭,一说葬于汀州。

\subsubsection{隆武}

\begin{longtable}{|>{\centering\scriptsize}m{2em}|>{\centering\scriptsize}m{1.3em}|>{\centering}m{8.8em}|}
  % \caption{秦王政}\
  \toprule
  \SimHei \normalsize 年数 & \SimHei \scriptsize 公元 & \SimHei 大事件 \tabularnewline
  % \midrule
  \endfirsthead
  \toprule
  \SimHei \normalsize 年数 & \SimHei \scriptsize 公元 & \SimHei 大事件 \tabularnewline
  \midrule
  \endhead
  \midrule
  元年 & 1645 & \tabularnewline\hline
  二年 & 1646 & \tabularnewline
  \bottomrule
\end{longtable}

\subsection{绍武帝\tiny(1646-1647)}

\subsubsection{生平}

明紹武帝朱聿{\fzk 𨮁}(1605年-1647年1月20日),年號紹武。1646年—1647年在位,南明第三任君主。朱聿𨮁又稱小唐王,是明绍宗(唐王)之弟,明太祖二十三子唐定王朱桱的八世孙,祖父唐端王朱碩熿,父為唐王之子朱器墭。

明紹宗即位後封朱聿{\fzk 𨮁}為唐王,主祀唐國,幾天後紹宗出征,留他和邓王朱器䵺監國。

1646年(隆武二年),南明重臣郑芝龙拒不发兵,以致清軍隊长驱直入福京,並於长汀俘虜明紹宗,紹宗殉國,時為唐王的朱聿{\fzk 𨮁}和隆武朝的宫员逃到廣東省廣州府番禺縣,而其他南明勢力則在肇庆府推举明神宗之孙、明思宗堂弟桂王朱由榔为监国。同年十月十六日,江西赣州失守后,朱由榔政權大驚,于十月二十一仓皇从肇庆逃往广西梧州,置廣東全省於不顧。於是,大学士苏观生,在廣東權力真空與一眾明朝藩王已由海路到達广州的情況之下,聯同大学士何吾驺、广东布政使顾元镜,侍郎王应华、曾道唯等拥立朱聿{\fzk 𨮁}为监国,以都司署为行宫。隆武二年十一月五日,四十一歲的朱聿{\fzk 𨮁}按兄終弟及的皇明祖訓,繼位称帝,以明年为绍武元年。苏观生因拥戴有功,被命为首輔,封建明伯,掌兵部。由於朱聿{\fzk 𨮁}仓促稱帝,登極時的龍袍與百官官服都要假借于粵劇伶人的戏服。

十一月初八,紹武称帝的消息传到梧州,朱由榔政權大驚大怒,四日後回到肇庆,再於十八日登極稱帝,改元永曆,是為明昭宗。永曆帝立刻派遣兵科给事中彭耀、兵部主事陈嘉谟前往广州,拜见紹武帝,稱其為「殿下」,規勸其取消帝号。首輔苏观生大怒,以大不敬斬彭、陈二人,再令陈际泰督师攻打肇庆。永曆帝派兵部右侍郎林佳鼎、夏四敷率兵,在十一月二十九日於三水县城西,與紹武軍展開內戰,並將對方擊退。苏观生再令广东总兵林察聯同新降的海盗等数万人反擊,並且大敗永曆軍隊。大捷消息传到广州,苏观生下令广州张灯结彩粉饰太平。正当紹武、永曆二帝自相殘殺之時,由佟养甲、李成栋率领的清兵已取潮州、惠州,臨近广州附近,並用缴获的南明地方官印,向紹武帝发出太平的錯誤信息。

十二月十五日,绍武帝幸武学,百官聚集,而此時,清兵已经偷偷兵臨城下,内应脫去头上的伪装,露出辫子。有人向苏观生報告,反遭斬首。苏观生说:“潮州昨尚有报,安得遽至此。妄言惑众,斩之!”不久,清军壓境的戰況得到證實,苏观生遂率領部隊与清兵激战一晝夜,清兵本有撤退之意,但內奸谢尚政旋引清兵入城,广州即陷落。苏观生見大勢已去,写下“大明忠臣义固当死”八个大字后,自縊死亡。已易服的紹武帝,打算爬城墙逃走,但被追骑赶上抓获,囚于东察院。李成栋派人送来饮食,紹武帝拒絕,說:“我若饮汝一勺水,何以见先人地下!”後自缢而殉國,結束其四十日的統治。绍武朝的主要官員如何吾驺、王应华、顾元镜等降清,而广州內的二十四個明朝藩王則全數被殺。紹武帝死後,永曆帝成為南明唯一的皇帝。

後人將紹武、蘇觀生等十五人,葬於廣州城北象岗山北麓。1954年因基建,迁葬于越秀公园木壳岗;1981年再迁葬于公园南秀湖畔。墓坐东向西,封土呈覆竹形,正面竖墓碑,中刻“明绍武君臣冢”,上款为“光绪癸未(1883年)孟冬吉旦”,下款为“粤东绅士重修”。1963年3月广州市政府公布为市级文物保护单位。

\subsubsection{绍武}

\begin{longtable}{|>{\centering\scriptsize}m{2em}|>{\centering\scriptsize}m{1.3em}|>{\centering}m{8.8em}|}
  % \caption{秦王政}\
  \toprule
  \SimHei \normalsize 年数 & \SimHei \scriptsize 公元 & \SimHei 大事件 \tabularnewline
  % \midrule
  \endfirsthead
  \toprule
  \SimHei \normalsize 年数 & \SimHei \scriptsize 公元 & \SimHei 大事件 \tabularnewline
  \midrule
  \endhead
  \midrule
  元年 & 1646 & \tabularnewline
  \bottomrule
\end{longtable}


\subsection{昭帝\tiny(1646-1662)}

\subsubsection{生平}

明昭宗朱由榔(1623年11月1日-1662年6月1日),或又稱永曆帝,南明第四位也是最後一位皇帝(1646年12月24日-1662年6月1日在位)。原為桂王。

1646年隆武帝被俘死,本為桂王的朱由榔自稱監國。不久,隆武帝弟唐王朱聿{\fzk 𨮁}在廣東廣州繼位,以次年為紹武元年,是為紹武帝。數日後,朱由榔在廣東肇庆亦登基稱帝,年號永曆。紹武、永曆二帝為爭正統,隨即開戰,後永曆軍大敗。1647年,清軍攻陷廣州,紹武帝兵敗殉國,永曆帝自此成為南明唯一的統治者。1659年,清军攻陷昆明后流亡缅甸東吁王朝,永曆十五年(1661年)夏历十二月初三日被送交吴三桂,永曆十六年四月十五日(1662年6月1日)遭縊死。死後,台灣的明鄭政權仍沿用永曆年號至1683年清朝佔領台灣為止。

朱由榔是明神宗之孙,明思宗堂弟,生於天啟三年(1623年)。崇禎年間封永明王,其父為桂端王朱常瀛,是明神宗第七子,封湖南衡阳,天启七年九月二十六日就藩,弘光元年(1645年)十一月初四日病死於梧州。第三子安仁王朱由𣜬承嗣。隆武帝称帝後不久病重。不久朱由榔被封桂王,在1646年隆武帝被俘後,於当年十月初十(一说十四日)称监国於廣東肇庆。

朱由榔於1646年(清顺治三年)農曆十一月十二日东返肇庆,十八日在肇庆正式稱帝,年号永曆,史称永曆帝。曾道唯、顾元镜、王应华等人都入阁,洪朝钟在十天之内升官三次。

永曆帝在中後期倚仗张献忠之餘部李定国、孙可望等人在西南一带抵抗满清,并且得到包括延平郡王郑成功在内的各反清力量的支持,是为反清的精神领袖和天下共主。1652年,李定国在桂林逼死定南王孔有德,又在衡州斩杀敬谨亲王尼堪,取得大捷,一度收复湖南西部、四川(除了保宁)、廣東(李成栋反正取得全部地区,后来仅保有沿海)、江西(金声桓、王得仁反正)等地。

1660年,清军攻入云南,永曆帝流亡缅甸東吁王朝首都瓦城,獲國王莽達(平達力)收留。後来,吴三桂攻入缅甸,莽達之弟莽白乘机发动兵变,杀死其兄奪位。1661年8月12日,莽白發動咒水之难,杀盡永曆帝侍從近衛。

永历帝得到清军进入缅境的消息后,曾寫信给吴三桂,到1662年1月22日(永历十五年十二月初三),莽白将永曆帝献给吴三桂,南明灭亡。

1662年6月1日(永历十六年四月十五望日,清康熙元年),永曆帝父子及眷属25人在昆明篦子坡遭弓弦勒死,终年40岁。其身亡處時人稱為逼死坡,即今天的昆明市五华区的华山西路,辛亥革命後蔡鍔等人在當地豎立「明永曆帝殉國處」石碑。死后庙号昭宗,谥号應天推道敏毅恭儉經文緯武體仁克孝匡皇帝。

至今未发现永历帝之墓。仅贵州都匀大坪镇有永历帝的衣冠冢。当地扶姓人家说,是他们先人明朝大学士扶纲派人搜集衣冠而葬的,为隐其真,只传是桂王坟,不留碑记。扶纲是因明亡不愿降清而回乡隐居的。帝墓左边是编修涂宏猷的 髮冢,右边是节愍侯邬昌期的衣带冢。民国十年都匀县奉令修史,查实桂王坟乃永历墓,才为其树碑立传,省长任可澄、省志总 陈炬、知县窦全曾都为之写了碑记,碑文“大明永历皇帝陵”几个字,墓碑及碑记是时任四川綦江县县长张瑞徵写的(张系都匀人),还修了些亭阁楹联,帝墓才初显规模。墓高3米、径6米,碑高1.62米,宽0.81米、厚0.13米,碑字阴刻正楷,字笔工整秀丽。涂宏猷和邬昌期二人,是咒水之难42大臣之二,坟比帝坟小得多。“文革”中被盗,帝坟从前到后挖了一个大坑,碑断为两截仰卧坟前土中。1996年都匀市人民政府公布大明永历皇帝陵为市级文物保护单位,着手修复帝陵。坟用青石砌边,水泥勾缝,碑文由书法家芦如平书写,前边加修了上下山的双向百级石阶,供游人参观。

\subsubsection{永历}
\begin{longtable}{|>{\centering\scriptsize}m{2em}|>{\centering\scriptsize}m{1.3em}|>{\centering}m{8.8em}|}
  % \caption{秦王政}\
  \toprule
  \SimHei \normalsize 年数 & \SimHei \scriptsize 公元 & \SimHei 大事件 \tabularnewline
  % \midrule
  \endfirsthead
  \toprule
  \SimHei \normalsize 年数 & \SimHei \scriptsize 公元 & \SimHei 大事件 \tabularnewline
  \midrule
  \endhead
  \midrule
  元年 & 1647 & \tabularnewline\hline
  二年 & 1648 & \tabularnewline\hline
  三年 & 1649 & \tabularnewline\hline
  四年 & 1650 & \tabularnewline\hline
  五年 & 1651 & \tabularnewline\hline
  六年 & 1652 & \tabularnewline\hline
  七年 & 1653 & \tabularnewline\hline
  八年 & 1654 & \tabularnewline\hline
  九年 & 1655 & \tabularnewline\hline
  十年 & 1656 & \tabularnewline\hline
  十一年 & 1657 & \tabularnewline\hline
  十二年 & 1658 & \tabularnewline\hline
  十三年 & 1659 & \tabularnewline\hline
  十四年 & 1660 & \tabularnewline\hline
  十五年 & 1661 & \tabularnewline\hline
  十六年 & 1662 & \tabularnewline\hline
  十七年 & 1663 & \tabularnewline\hline
  十八年 & 1664 & \tabularnewline\hline
  十九年 & 1665 & \tabularnewline\hline
  二十年 & 1666 & \tabularnewline\hline
  二一年 & 1667 & \tabularnewline\hline
  二二年 & 1668 & \tabularnewline\hline
  二三年 & 1669 & \tabularnewline\hline
  二四年 & 1670 & \tabularnewline\hline
  二五年 & 1671 & \tabularnewline\hline
  二六年 & 1672 & \tabularnewline\hline
  二七年 & 1673 & \tabularnewline\hline
  二八年 & 1674 & \tabularnewline\hline
  二九年 & 1675 & \tabularnewline\hline
  三十年 & 1676 & \tabularnewline\hline
  三一年 & 1677 & \tabularnewline\hline
  三二年 & 1678 & \tabularnewline\hline
  三三年 & 1679 & \tabularnewline\hline
  三四年 & 1680 & \tabularnewline\hline
  三五年 & 1681 & \tabularnewline\hline
  三六年 & 1682 & \tabularnewline\hline
  三七年 & 1683 & \tabularnewline
  \bottomrule
\end{longtable}


%%% Local Variables:
%%% mode: latex
%%% TeX-engine: xetex
%%% TeX-master: "../Main"
%%% End:
