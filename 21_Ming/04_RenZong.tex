%% -*- coding: utf-8 -*-
%% Time-stamp: <Chen Wang: 2021-11-01 17:11:25>

\section{仁宗朱高熾\tiny(1424-1425)}

\subsection{生平}

明仁宗朱高熾(1378年8月16日-1425年5月29日),俗稱洪熙帝,明成祖長子,其母为仁孝文皇后,中山王徐達外孫,明朝第四代皇帝。

洪武年間,被封為燕世子。靖難之役中,仁宗負責鎮守北平,并成功抵禦李景隆率領的中央軍圍攻。永樂二年(1404年),立為皇太子,并在明成祖屢次北伐中,擔任監國職位,實際負責國家政事。永樂二十二年(1424年),繼承皇位,年號“洪熙”,在位期間,採取一系列政治、經濟、軍事改革與調整,國家富足。仁宗與子明宣宗在政治用人、行政處理上,均為後世所称善,史稱“仁宣之治”。

朱高熾年幼端重沉靜,善於言辭,且擅长射箭,喜愛與儒臣講論。洪武二十八年闰九月壬午(1395年11月4日),他被冊封為燕世子,後守衛北平,由於心性較爲溫良,體諒官員、士卒,深受祖父明太祖朱元璋喜愛。

靖難之役中,燕王朱棣起兵,朱高熾則鎮守北平,期間以一萬兵力,阻擋李景隆率領的五十萬中央軍圍攻。由於朱高熾身型肥胖而且有腳病,不良於行,不曾隨父親朱棣征戰,且性格相對較爲溫和,向來不獲父親寵愛。反而常隨朱棣征戰的次子朱高煦、三子朱高燧均受朱棣喜愛,而朱高煦則更因屢有戰功,於是出言詆毀朱高炽以奪嫡。當時,建文帝施離間計,下「賜世子書」;朱高燧的人馬得知此事,向朱棣建言「世子勾結朝廷」,沒想到朱高熾不予啟封,直接呈上朱棣,方破此計。朱棣即位後,改北平為北京,仍命朱高熾居守。

朱棣成功奪位為帝後,是為明成祖。永乐元年春正月丙戌,群臣上表请立皇太子,不允;三月戊寅朔,文武百官复上表,请立皇太子,敕“姑缓之”。成祖本想立自己喜愛的次子朱高煦為太子,但礙於長子朱高熾的世子地位是明太祖確立,而且朱高熾並無過失,又得一眾文官支持,最後於永樂二年四月甲戌(1404年5月12日),朱高熾被召入南京應天府,被立為皇太子。明成祖屢次北伐,均命其擔任監國,負責國事。當時全國經战争影響,水旱饑荒嚴重,他派遣官員賑災撫恤,仁政受到贊許。然而,失落太子地位的朱高煦心有不甘,聯同弟朱高燧及其他黨羽加緊離間明成祖與朱高熾的關係。明成祖問太子是否知悉有人離間,朱高熾則答稱不知情,“知盡子職而已”。

永樂十年,朱棣北伐歸還,朱高熾遣使誤期,加上書奏失辭,太子一系官員,如黃淮等人均下詔獄。次年,朱高燧黨羽黃儼等誣陷朱高熾擅自釋放罪人,其官僚多因連坐而亡。禮部侍郎胡濙奉命調查后,密奏朱棣稱太子誠敬孝謹等七事,明成祖才釋除疑慮。之後,朱高燧黨羽黃儼策劃謀立,后被發覺,伏法。太子朱高熾則力請免朱高燧罪,至此朱高熾地位方穩。

永樂二十二年(1424年)七月,明成祖在北征班師途中崩於榆木川。当时京师诸卫军皆随行,只有赵府三护卫留京师,随驾北征诸臣浮议籍籍,大学士杨荣、金幼孜等人顾虑赵府护卫闻讯发动政变,遂秘不发丧。杨荣与少监海寿持遗诏驰奔京师。朱高熾遣皇太孙朱瞻基出居庸关迎驾。同年八月己酉,皇太孙至雕鹗堡,入于军中,遂发丧。八月丁巳(1424年9月7日),朱高熾繼帝位,大赦天下,并取次年年號為洪熙。明仁宗登基後,褒奖直言,虚怀纳谏,減轻刑法。朱高熾與子朱瞻基在政治用人、行政處理上,均為後世所称善,史稱“仁宣之治”。

經濟方面,他下令中止鄭和下西洋,并取消官方在雲南、交阯的採辦活動、将首都迁回南京,以節省國家財政支出。政治方面,他恢復夏原吉、吳中官職,恢復三公、三孤等官職,命楊榮為太常寺卿,金幼孜為戶部侍郎,兼大學士,楊士奇為禮部左侍郎兼華蓋殿大學士,黃淮為通政使兼武英殿大學士,楊溥為翰林學士,進一步提升明朝內閣地位。軍事方面,他重新調整大同、交阯、山海關、遼東的邊疆總兵大臣,并建立南京守備制度。

同年冬天,朱高熾進一步對政治進行調整,加強戶部管理、以及城池防禦的同時,冊封張氏為皇后,立長子朱瞻基為皇太子、其餘八子分別為王。隨後下詔,赦免了建文帝的舊臣和永樂朝時遭連坐流放邊境的官員家屬,并免除受災地的稅糧。

外交方面,于闐、琉球、占城、哈密、古麻剌朗、滿剌加、蘇祿、瓦剌等國稱臣入貢。

洪熙元年春,因顯日食,朱高熾罷免宴樂。他進一步對政治進行調整,包括建立弘文閣,命楊溥掌管內閣;屢次求官員直言并納言,并對太祖時期的法外用刑制度進行修正,減少刑罰,實行寬政。

仁宗体弱多病,登基后不到十个月,遭李时勉當廷勸諫,龍顏大怒,雖命武士以金瓜錘將李时勉打斷三根肋骨,並拘入詔獄,仁宗仍不解恨,數日後一病不起,于洪熙元年五月辛巳(1425年5月29日)崩于钦安殿,廟號仁宗,葬于明獻陵(今北京昌平)。朱高熾延續了太祖和成祖的殉葬制度,死時生殉五名妃嬪。

\subsection{洪熙}

\begin{longtable}{|>{\centering\scriptsize}m{2em}|>{\centering\scriptsize}m{1.3em}|>{\centering}m{8.8em}|}
  % \caption{秦王政}\
  \toprule
  \SimHei \normalsize 年数 & \SimHei \scriptsize 公元 & \SimHei 大事件 \tabularnewline
  % \midrule
  \endfirsthead
  \toprule
  \SimHei \normalsize 年数 & \SimHei \scriptsize 公元 & \SimHei 大事件 \tabularnewline
  \midrule
  \endhead
  \midrule
  元年 & 1425 & \tabularnewline
  \bottomrule
\end{longtable}


%%% Local Variables:
%%% mode: latex
%%% TeX-engine: xetex
%%% TeX-master: "../Main"
%%% End:
