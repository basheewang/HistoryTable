%% -*- coding: utf-8 -*-
%% Time-stamp: <Chen Wang: 2019-10-18 16:57:57>

\section{英宗\tiny(1435-1449)}

明英宗朱祁鎮(1427年11月29日-1464年2月23日),明宣宗朱瞻基長子,生母孝恭章皇后,明代宗朱祁鈺異母兄,明憲宗朱見深之父,是明朝的第6位和第8位皇帝;最初使用正統(1436年-1449年)年號,復位後使用天順(1457年-1464年)年號,在位22年。謚號「法天立道仁明誠敬昭文憲武至德廣孝睿皇帝」。

宣德二年(1427年),貴妃孫氏為明宣宗朱瞻基產下長子朱祁鎮(但《明史》記孫氏生平則說她暗中取宮女之子為己子)。出生四個月的朱祁鎮隨即被立為皇太子,其母孫氏為皇后。

宣德十年(1435年)正月,宣宗崩,時年7歲的朱祁鎮即位,是為英宗,改次年為正統元年。英宗在位初期由太皇太后張氏輔政,內閣由三楊(楊士奇、楊榮和楊溥)主持,仁宣之治得以延續。

正統六年(1441年),正式親政,同年定首都為北京,結束南京名義上的首都地位。

正統七年(1442年),張太后卒,三楊以年老淡出政壇,宦官王振開始專權,其黨羽遍天下,百官為之側目,這是明朝第一次宦官專權。

正統十四年(1449年),瓦剌蒙古大舉南侵,英宗以五十萬大軍親征,沿途鋪張。返師途中,八月十五(1449年9月1日)行至土木堡被瓦剌太師也先所敗,明軍「死者數十萬」,英宗被俘虜,附和英宗的太監王振被明英宗之護衛將軍樊忠殺死,樊忠殺死王振前曰:「吾為天下誅此賊!」以所持棰擊殺王振,力圖突圍,殺數十人後戰死。史稱土木堡之變,簡稱土木之變。

隨後,也先挾持英宗南下進攻北京,皇太后孫氏命英宗之弟郕王朱祁鈺監國,不久郕王即帝位,是為明代宗,改次年為景泰元年,尊英宗為太上皇。

于謙領導的北京保衛戰勝利後,瓦剌倡議和談,欲送還英宗。景帝不欲英宗還鑾。景泰元年(1450年),鴻臚卿楊善變賣家產,孤身出使瓦剌,又在景帝不同意的情況下,說服瓦剌太師也先,將英宗迎回燕京。

英宗回國後,代宗怕失去即位不久的帝位,將其兄長英宗軟禁於南內崇質宮,令錦衣衛防守嚴密。景泰三年,又廢原立為太子的英宗長子朱見深為沂王,另立己子朱見濟為儲君。但朱見濟在次年去世。後太子朱見濟死,但代宗仍不同意復立朱見深為太子。

景泰八年(1457年)正月,代宗病重,不能臨朝,手握重兵的武清侯石亨、副都御史徐有貞等人聯合太監曹吉祥,率死士攻入南宮,擁英宗復辟。十六日晚上,英宗自東華門入宮,於奉天殿即位,黎明時開宮門,諭令百官,改元天順,史稱「奪門之變」。代宗被禁於西內。不久死亡,死因不明,有謂乃英宗使宦官蔣安以布帛縊死。死後追貶為郕王,謚戾,葬於西郊金山(玉泉山北)。

英宗奪門之變復辟後,即以謀逆罪將兵部尚書于謙及大學士王文等人下獄,初尚言「于謙實有功」,徐有貞言「不殺于謙,今日之事無名」,遂於五日後斬殺于謙和王文於西市。天下冤之。大學士李賢告知英宗背後秘密,「奪門之變」沒有用處。因為郕王無子,擁立朱祁鎮的孫太后仍在世上,所以帝位遲早是英宗的,不需要奪門。奪門只是小人们的一齣戲,目的是求自己的升官發財。英宗下令宮中不得再使用「奪門」一詞,並且罷除因奪門之變而晉升的一切官職(計四千餘人),疏遠了徐​​有貞等,後來曹吉祥與石亨等人勾結,先設法中傷徐​​有貞,讓徐被流放。而後石亨與曹吉祥因圖謀叛亂發動曹石之變,石亨被囚至死,曹吉祥則被凌遲處死。

天順一朝,英宗勤於理政,並任用李賢、彭時等賢臣,先後懲治石亨、徐有貞、曹吉祥等人,政治尚算清明。又不顧左右反對,釋放建庶人(明惠宗幼子朱文圭,明成祖發動靖難後被幽禁宮中逾五十年,已豬狗不識),並提供飲食住行;聽錢皇后之言恢復前朝胡廢后的位號;病危遺言,取消了自明太祖以來的宮妃殉葬制度。《明史》讚譽道:「罷宮妃殉葬,則盛德之事可法後世者矣。」王世貞在《弇州山人別集》中亦稱:「此誠千​​古帝王之盛節。」

天順八年(1464年)正月英宗駕崩,享年38歲。葬入明十三陵中的裕陵。英宗與錢皇后感情頗深,錢皇后無子;因周妃專橫,英宗擔心死後嗣子明憲宗(周氏所生)不尊崇她的地位,所以遺命「皇后他日壽終,宜合葬」後來錢皇后死時,周太后果然不欲其祔葬裕陵,由於有英宗的遺詔,經過大臣力爭方得與英宗合葬。此後,在周太后的壓力下,不得已改變英宗的陵寢設計,周太后也得以附葬裕陵,開始出現一帝兩后或多后的格局。

\subsection{正统}

\begin{longtable}{|>{\centering\scriptsize}m{2em}|>{\centering\scriptsize}m{1.3em}|>{\centering}m{8.8em}|}
  % \caption{秦王政}\
  \toprule
  \SimHei \normalsize 年数 & \SimHei \scriptsize 公元 & \SimHei 大事件 \tabularnewline
  % \midrule
  \endfirsthead
  \toprule
  \SimHei \normalsize 年数 & \SimHei \scriptsize 公元 & \SimHei 大事件 \tabularnewline
  \midrule
  \endhead
  \midrule
  元年 & 1436 & \tabularnewline\hline
  二年 & 1437 & \tabularnewline\hline
  三年 & 1438 & \tabularnewline\hline
  四年 & 1439 & \tabularnewline\hline
  五年 & 1440 & \tabularnewline\hline
  六年 & 1441 & \tabularnewline\hline
  七年 & 1442 & \tabularnewline\hline
  八年 & 1443 & \tabularnewline\hline
  九年 & 1444 & \tabularnewline\hline
  十年 & 1445 & \tabularnewline\hline
  十一年 & 1446 & \tabularnewline\hline
  十二年 & 1447 & \tabularnewline\hline
  十三年 & 1448 & \tabularnewline\hline
  十四年 & 1449 & \tabularnewline
  \bottomrule
\end{longtable}


%%% Local Variables:
%%% mode: latex
%%% TeX-engine: xetex
%%% TeX-master: "../Main"
%%% End:
