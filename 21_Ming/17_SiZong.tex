%% -*- coding: utf-8 -*-
%% Time-stamp: <Chen Wang: 2019-10-21 17:37:10>

\section{思宗\tiny(1627-1644)}

明思宗朱由檢(1611年2月6日-1644年4月25日),或稱崇禎帝,明朝第17代、末代皇帝。

思宗为明光宗第五子,明熹宗异母弟。五歲時,其母劉氏獲罪,被時為太子的光宗下令杖殺,朱由检交由庶母西李撫養,數年後改由另一庶母东李撫養至成人。於天启二年(1622年)年被兄長明熹宗册封為信王。明熹宗於天啟七年(公元1627年8月)駕崩,由于没有子嗣,朱由检受遗命于同月丁巳日登基,时年十八歲。次年改元崇禎,是为明思宗。

思宗一生操勞,日以繼夜的批閱奏章,节俭自律,不近女色。崇祯年間,与萬曆、天啟相较,朝政有了明显改观。即位之初就大力铲除阉党,曾六度下诏罪己,惜其生性多疑,无法挽救衰微的明朝。明朝末年农民起义不断,关外后金政权虎视眈眈,已处于内忧外患的境地。崇祯十七年(1644年)發生甲申之變,李自成攻破北京,思宗在煤山一树上吊身亡,终年三十三岁,在位十五年。

南明予其庙号「思宗」,后改「毅宗」、「威宗」,南明弘光帝上谥号「绍天绎道刚明恪俭揆文奋武敦仁懋孝烈皇帝」。清朝追谥「钦天守道敏毅敦俭弘文襄武体仁致孝端皇帝」,庙号「怀宗」;后去庙号,改谥为「庄烈愍皇帝」,葬于十三陵思陵。

生於萬曆庚戌十二月二十四日 ( 1611年2月6日 ) 寅時。崇祯帝之父為明光宗朱常洛,朱常洛雖早在萬曆廿九年 ( 1601年 ) 被立為太子,但其父親明神宗其實一心想立三子朱常洵為太子,是因為群臣國本之爭,才勉強保住了朱常洛儲君的寶座,故朱常洛一直得不到明神宗歡心。朱由检母亲刘氏則是朱常洛的婢女,亦不得朱常洛的歡心。祖父討厭父親,父親討厭母親,所以朱由检幼年并不幸福。五岁时,朱由檢母親劉氏得罪,被父親朱常洛下令杖杀,之後將朱由检交由庶母西李抚养。数年后西李生了女儿,照管不过来,改由另一庶母东李抚养至成人。及至朱由检长大,被當時已繼位為帝的哥哥朱由校封为信王,刘氏追封为贤妃。

天启七年(1627年),年僅廿二歲的明熹宗朱由校駕崩,由於朱由校三名兒子皆早夭,他唯一在世的弟弟朱由檢繼承皇位,當時朱由檢年僅十六歲,是為崇禎帝。朱由檢即位后,勤于政务,事必躬亲。崇祯十五年(1642年)七月初九,因“偶感微恙”而临时传免早朝,遭辅臣批评,崇禎連忙自我檢討。

天启七年十一月(1627年),崇祯帝在铲除魏忠贤的羽翼崔呈秀之后,再将其贬至凤阳。途至直隶阜城,魏忠贤得知大勢已去,遂与一名太监自缢而亡。此后崇祯帝又殺客氏,崔呈秀自盡,其阉党二百六十餘人或处死、或发配、或终身禁锢。与此同时,平反冤狱,重新启用天启年间被罢黜的官员。起用袁崇焕为兵部尚书,赐予尚方宝剑,託付他收复全辽的重任。

自崇禎元年(1628年)起,中國北方大旱,赤地千里,寸草不生,《汉南续郡志》记,“崇祯元年,全陕天赤如血。五年大饥,六年大水,七年秋蝗、大饥,八年九月西乡旱,略阳水涝,民舍全没。九年旱蝗,十年秋禾全无,十一年夏飞蝗蔽天……十三年大旱……十四年旱”。崇祯朝以來,陕西年年有大旱,百姓多流離失所。崇祯二年五月正式议裁陕北驛站,驛站兵士李自成失业。崇祯三年(1630年)陝西又大饑,陝西巡按馬懋才在《備陳大饑疏》上說百姓爭食山中的蓬草,蓬草吃完,剝樹皮吃,樹皮吃完,只能吃觀音土,最後腹脹而死,六年,“全陕旱蝗,耀州、澄城县一带,百姓死亡过半”。

崇祯七年,家住河南的前兵部尚书吕维祺上書朝廷:“盖数年来,臣乡无岁不苦荒,无月不苦兵,无日不苦輓输。庚午(崇祯三年)旱;辛未旱;壬申大旱。野无青草,十室九空。……村无吠犬,尚敲催征之门;树有啼鹃,尽洒鞭扑之血。黄埃赤地,乡乡几断人烟;白骨青燐,夜夜似闻鬼哭。欲使穷民之不化为盗,不可得也”。旱災又引起蝗災,使得災情更加擴大。河南於崇禎十年、十一年、十二年、十三年皆有蝗旱,“人相食,草木俱盡,土寇並起”,其飢民多從“闖王”李自成。崇祯十三、十四年,“南北俱大荒……死人弃孩,盈河塞路。”

十四年,左懋第督催漕運,道中馳疏言:“臣自靜海抵臨清,見人民飢死者三,疫死者三,為盜者四。米石銀二十四兩,人死取以食。惟聖明垂念。”保定巡撫徐標被召入京時說:“臣自江推來數千里,見城陷處固蕩然一空,即有完城,亦僅餘四壁城隍,物力已盡,蹂躪無餘,蓬蒿滿路,雞犬無音,未遇一耕者,成何世界”這時華北各省又疫疾大起,朝發夕死。“至一夜之內,百姓驚逃,城為之空”,崇禎十四年七月,疫疾從河北地区傳染至北京,崇祯十六年,北京人口死亡近四成。十室九空。

江南在崇祯十三年遭大水,十四年有旱蝗并灾,十五年持续发生旱灾和流行大疫。地方社会处在了十分脆弱的状态,盗匪与流民並起,各地民变不断爆发。

為剿流寇,崇祯帝先用楊鶴主撫,後用洪承疇,再用曹文詔,再用陳奇瑜,復用洪承疇,再用盧象昇,再用楊嗣昌,再用熊文燦,又用楊嗣昌,十三年中頻繁更換圍闖軍的將領。這其中除熊文燦外,其他都表現出了出色的才幹。然皆功虧一簣。李自成數次大難不死,後往河南聚眾發展。

此时北方皇太极又不断骚扰入侵,明廷苦於两线作战,每年的军费「三餉」开支高达两千万两以上,国家财政早已入不敷出,缺饷的情況普遍,常导致明军内部骚乱哗变。加上崇祯帝求治心切,生性多疑,刚愎自用,因此在朝政中屡铸大错:前期铲除专权宦官,后期又重用宦官,《春明梦余录》记述:“崇祯二年十一月,以司礼监太监沈良住提督九门及皇城门,以司礼监太监李凤翔总督忠勇营”崇祯帝說:“朕禦極之初,攝還內鎮,舉天下大事悉以委大小臣工,比者多營私圖,因協民艱,廉通者又遷疏無通。己已之冬,京城被攻,宗社震驚,此士大夫負國家也。清寫明史崇祯帝中后金反间计,自毁长城,冤杀袁崇焕;世傳皇太極施反間計,捕捉兩名明宮太監,然後故意讓兩人以為聽見滿清將軍之間的耳語,謂袁崇煥與滿人有密約,皇太極再放其中一名太監回京。崇祯帝中計,以為袁崇煥謀反。這種講法終明之世並無所本,僅流行於乾隆之後。一些學者傾向於相信崇祯帝殺袁崇煥,並非是皇太極的反間計得逞。由於袁崇煥是囚禁半年後才被處死的,不大可能是因一時激憤誤殺。事實上,崇祯帝生性多疑,所以僅擅殺毛文龍一事,便足以使崇祯帝心存忌憚。再者毛文龍舊部大都誤認為是皇帝要殺毛文龍,於是把怨恨轉移到皇帝身上,大舉譁變,造成日後一連串悲劇事件的發生,終於致使前線態勢一發不可收拾。袁崇煥不能不為此負責。

隨著局勢的日益嚴峻,崇祯帝的濫殺也日趨嚴重,總想以重典治世,總督中被誅者七人,巡撫被戮者十一人,連擁有崇高地位的內閣首辅也不能幸免,被殺二人,而其他各級文官武將更是多不勝數,不能詳列。崇祯帝亦知不能兩面作戰,私底下同意議和,但被明朝士大夫鑒於南宋的教訓,皆以為與滿人和談為恥。因此崇祯帝對於和議之事,始終左右為難,他暗中同意杨嗣昌的议和主张,但一旁的盧象昇立即告訴皇帝說:「陛下命臣督师,臣只知战斗而已!」,崇祯帝只能辯称根本就没有议和之事,盧象昇最後戰死沙場。明朝末年就在和戰兩難之間,走入滅亡之途。

崇禎十五年(1642年),松山、锦州失守,洪承畴降清,崇祯又想和满清议和而和兵部尚書陳新甲暗中商議計劃,後來陳新甲因泄漏議和之事被崇祯诿过處死,與清兵最後議和的機會也破滅了。崇禎十七年(1644年)明王朝面临没顶之灾,崇祯帝召見閣臣時悲嘆道:“吾非亡国之君,汝皆亡国之臣。吾待士亦不薄,今日至此,群臣何无一人相从?”在陳演、光時亨等反对和不情願負責之下未能下决心迁都南京。事後崇禎帝指責光時亨:“阻朕南遷,本應處斬,姑饒這遭。”後來,崇禎再次跟李明睿和左都御李邦華復議南遷的計劃,並要大學士陳演擔當責任,陳演不情願,於是在不久後被罷職。第二次南遷計劃失敗後,崇禎讓駙馬鞏永固代口要求重臣守京師,並以“聖駕南巡,征兵親討」為由出京,諸臣唯恐自己因皇帝不在京城而變成農民军發泄怒火的替死鬼,故依然不讓崇禎離京。

至此,农民军起义已经十多年,从北京向南,南京向北,纵横数千里之间,白骨满地,人烟断绝,行人稀少。崇祯帝召保定巡抚徐标入京觐见,徐标说:“臣从江淮而来,数千里地内荡然一空,即使有城池的地方,也仅存四周围墙,一眼望去都是杂草丛生,听不见鸡鸣狗叫。看不见一个耕田种地之人,像这样陛下将怎么治理天下呢?”崇祯帝听后,潸然泪下,叹息不止。于是,为了祭祀阵亡将士、罹难难民和殉國的各亲王,崇祯帝便在宫中大作佛事来祈求天下太平,并下诏罪己,催促督师孙传庭赶快围剿农民军。

崇禎十六年正月,李自成部克襄陽、荊州、德安、承天等府,張獻忠部陷蘄州,明將左良玉逃至安徽池州。崇禎十七年(1644年)三月一日,大同失陷,北京危急,初四日,崇禎任吳三桂為平西伯,飛檄三桂入衛京師,起用吳襄提督京營。六日,李自成陷宣府,太監杜勳投降,十五日,大學士李建泰投降,李自成部開始包圍北京,太監曹化淳說:「忠賢若在,時事必不至此。」三月十六日,昌平失守,十七日,圍攻北京城。三月十八日,李自成軍以飛梯攻西直、平則、德勝諸門,守軍或逃、或降。下午,曹化淳開彰儀門(一說是十九日王相堯開宣武門,另張縉彥守正陽門,朱純臣守朝陽門,一時俱開,二臣迎門拜賊,賊登城,殺兵部侍郎王家彥於城樓,刑部侍郎孟兆祥死於城門下),李自成軍攻入北京。太監王廉急告皇帝,思宗在宫中饮酒长叹:“苦我民尔!”太監張殷勸皇帝投降,被一劍刺死。崇祯帝命人分送太子、永王、定王到勳戚周奎、田弘遇家。又逼周后自杀,手刃袁妃(未死)、長平公主(未死)、昭仁公主。

然後思宗手執三眼槍與數十名太監騎馬出東華門,被亂箭所阻,再跑到齊化門(朝陽門),成國公朱純臣閉門不納,後轉向安定門,此地守軍已經星散,大門深鎖,太監以利斧亦無法劈開。三月十九日拂曉,大火四起,重返皇宮,城外已经是火光映天。此時天色将明,崇祯在前殿鸣钟召集百官,却无一人前来,崇祯帝說:“诸臣误朕也,国君死社稷,二百七十七年之天下,一旦弃之,皆为奸臣所误,以至于此。”最後在景山老歪脖子树上自缢身亡,死时光着左脚,右脚穿着一只红鞋。死於崇禎甲申三月十九日丑時,时年33岁。身边仅有提督太监王承恩陪同。上吊死前于蓝色袍服上大书其遺書:

“朕自登極(或作登基)十有七年,虽朕凉德藐躬(或作薄德匪躬),上干天咎(或作天譴、天怒),致逆贼直逼京师,然皆诸臣之误朕也。朕死无面目见祖宗于地下,自去冠冕,以髮覆面。任贼分裂朕尸,勿伤百姓一人。”

三月二十一日屍體被發現,大順軍將崇祯帝與周皇后的屍棺移出宮禁,在東華門示眾,也允許投降的諸臣前往送葬,只是人數不多,“諸臣哭拜者三十人,拜而不哭者六十人,餘皆睥睨過之”,只有主事劉養貞極其悲痛,梓宮暫厝在紫禁城北面的河邊。

崇祯帝死後,自杀官員有户部尚书倪元璐、工部尚书范景文、左都御史李邦华、左副都御史施邦曜、协理京营兵部右侍郎王家彦、大理寺卿凌义渠、太常寺卿吳麟徵、左中允刘理顺、刑部右侍郎孟兆祥、前戶科都給事中吳甘來、武庫主事成德、兵部主事金鉉、左諭德马世奇、檢討汪偉、右庶子周鳳翔、太僕寺丞申佳胤、吏部員外郎許直、戶部員外郎寧承烈、光禄寺署丞于腾雲、副兵馬使姚成、中書舍人宋天顯,滕之所、阮文貴、監察御史王章、陳良謨、陳纯德、經歷張應選,順天府知事陈貞達等、外戚如驸马都尉巩永固、新樂伯劉文炳、惠安伯張慶臻、宣城伯衛時春,錦衣衛都指揮使王國興自殺,太监自杀者以百计,战死在千人以上。宫女自杀者三百余人。绅生生员等七百多家举家自杀。四月四日,昌平州吏趙一桂等人將崇禎與皇后葬入昌平縣田貴妃的墓穴之中,清朝以“帝禮改葬,令臣民為服喪三日,諡曰莊烈愍皇帝,陵曰思陵”。

崇禎十七年五月初六日,多爾袞以李明睿為禮部侍郎,負責大行皇帝的諡號祭葬事宜,李擬上先帝諡號欽天守道敏毅敦儉弘文襄武體仁致孝端皇帝,廟號懷宗,并建議改葬梓宮。後因思宗梓宮已入葬恭淑端惠靜懷皇貴妃的園寢,便不再遷葬,改田貴妃園寢為思陵。

順治十六年十一月,以“興朝諡前代之君禮,不稱數、不稱宗”為由,[原創研究?]去懷宗廟號,改諡莊烈愍皇帝,因而清代史書多簡稱為莊烈帝或明愍帝。

《欽定古今圖書集成·方輿彙編·職方典·順天府部雜錄十一》、《欽定日下舊聞考·卷一百三十七》、《讀禮通考·卷九十三》三書均引《肅松錄》和《北游紀方》,稱思陵神牌題為“大明欽天守道敏毅敦儉弘文襄武體仁致孝莊烈愍皇帝”,又引《北游紀方》稱思陵神主題為“大明懷宗欽天守道敏毅敦儉弘文襄武體仁致孝莊烈端皇帝”,又引《肅松錄》稱思陵立有“莊烈愍皇帝之陵”的石碑。《明詩綜·卷一》則稱神牌是由順治初年定的“一十六字”加上改書的“莊烈愍皇帝”組合而成。神主甚至又改“愍”字為“端”,並仍題廟號“懷宗”二字,可見康熙年間的思陵神牌和神主是由順治年間兩次加諡崇禎帝的廟諡號混雜而成。《崇禎長編·卷一》作“果毅敦儉弘文襄武體仁致孝莊烈愍皇帝”,當是清廷所給諡號在傳抄中產生了訛誤。

南明安宗之大臣張慎言初議崇禎帝之廟諡號為“烈宗敏皇帝”,高弘图拟庙号“思宗”,顧錫疇議廟號“乾宗”。赵之龙上疏弹劾高弘图议庙号之失,称“思为下谥”。顧錫疇又拟庙号正宗,但未被採用。最終在崇禎十七年六月定先帝谥號為紹天繹道剛明恪儉揆文奮武敦仁懋孝烈皇帝,庙号思宗。 按《逸周書·諡法解》:“道德純一曰思。大省兆民曰思。外內思索曰思。追悔前過曰思。……有功安民曰烈。以武立功。秉德尊業曰烈。”

弘光元年李青上疏请改思宗庙号,多次上疏皆被駁回。管紹寧擬“敬宗”和“毅宗”兩號備選,同時又有人上疏請求改為“烈宗正皇帝”。弘光元年二月丙子改上廟號毅宗,谥号未改。唐王监国,谥思宗為威宗。

與其他朝代的亡國之君不同,崇祯帝是一個被普遍同情的皇帝,崇祯帝一直勤政,以挽救過去祖輩皇帝的過失。崇祯帝即位,正值國家內憂外患之際,內有黃土高原上百萬農民造反大軍,外有滿洲鐵騎,虎視耽耽,崇祯元年(1628年)陕西镇的兵饷积欠到30多月,次年二月延绥、宁夏、固原三镇皆告缺饷达36月之久。

推翻明朝的李自成《登極詔》也說“君非甚闇(崇禎皇帝不算太糟),孤立而煬灶恆多(孤立於上,而受到奸臣的蒙蔽);臣盡行私,比黨而公忠絕少。”

思宗的性格相當複雜,在去除魏忠賢時,崇禎表現得極為機智,但在處理袁崇煥一事,卻又表現得相當愚蠢,《明史》說他:「性多疑而任察,好刚而尚气。任察则苛刻寡恩,尚气则急遽失措。」

张岱认为「思宗焦心求治,旰食宵衣,恭俭辛勤,万机无旷。即古之中兴令主无以过之。」然而,他「惟务节省」,以至「九边军士数年无饷,体无完衣」;又「渴于用人,骤于行法」,以至「天下之人,无所不用。及至危及存亡之秋,并无一人为之分忧宣力。」

《明史》評價思宗:「帝承神、熹之後,慨然有為。即位之初,沈機獨斷,刈除奸逆,天下想望治平。惜乎大勢已傾,積習難挽。在廷則門戶糾紛,疆埸則將驕卒惰。兵荒四告,流寇蔓延。遂至潰爛而莫可救,可謂不幸也已。然在位十有七年,不邇聲色,憂勸惕勵,殫心治理。臨朝浩歎,慨然思得非常之材,而用匪其人,益以僨事。乃復信任宦官,布列要地,舉措失當,制置乖方。祚訖運移,身罹禍變,豈非氣數使然哉。迨至大命有歸,妖氛盡掃,而帝得加諡建陵,典禮優厚。是則聖朝盛德,度越千古,亦可以知帝之蒙難而不辱其身,為亡國之義烈矣。」

顺治帝評價思宗:「本朝入关定鼎,首为崇祯帝、后发丧,营建幽宫,为万古未闻之义举。」1657年,顺治谕工部曰:「朕念明崇祯帝孜孜求治,身殉社稷。若不急为阐扬,恐于千载之下,竟与失德亡国者同类并观,朕用是特制碑文一道,以昭悯恻。」谒崇祯陵的时候,顺治大呼说:「大哥大哥,我与若皆有君无臣。」顺治对崇祯的书法更是高度赞赏。史书记载,僧弘觉向顺治索字,顺治说:「朕字何足尚,崇祯帝乃佳耳。」说完叫人一并拿来八九十幅崇祯的字,一一展示,“上容惨戚,默然不语”。看完了,顺治说:「如此明君,身婴巨祸,使人不觉酸楚耳。」又说:「近修《明史》,朕敕群工不得妄议崇祯帝。」顺治的话,连弘觉都给感动了:「先帝何修得我皇为异世知己哉!」顺治写给崇祯的碑文是:「庄烈悯皇帝励精图治,宵旰焦心,原非失德之主。良由有君无臣,孤立于上,将帅拥兵而不战,文吏噂沓而营私。……逮逆渠犯阙,国势莫支,帝遂捐生以殉社稷。……」

談遷《國榷》稱:“先帝(崇禎)之患,在於好名而不根于實,名愛民而適痡之,名聽言而適拒之,名亟才而適市之;聰于始,愎于終,視舉朝無一人足任者,柄托奄尹,自貽伊戚,非淫虐,非昏懦,而卒與桀、紂、秦、隋、平、獻、恭、昭並日而語也,可勝痛哉!”

歷史學家孟森說:“思宗而在萬曆以前,非亡國之君;在天啟之後,則必亡而已矣!”。思宗雖有心為治,卻無治國良方,以致釀成亡國悲劇,未必無過。孟森也說思宗“苛察自用,無知人之明”、“不知恤民”。思宗用人不彰、疑心過重、馭下太嚴,史稱“崇禎五十相”(在位十七年,更換五十位內閣大學士、首輔),卻加速了明王朝的覆亡。

鎖綠山人在《明亡述略》中評價崇禎,“莊烈帝勇於求治,自異此前亡國之君。然承神宗、熹宗之失德,又好自用,無知人之識。君子修身齊家,宜防好惡之癖,而況平天下乎?雖當時無流賊之蹂躪海內,而明之亡也決矣。”

南明大臣則把思宗抬舉到千古聖主的地步,如禮部侍郎余煜在議改思宗廟號時說:“先帝(崇禎)英明神武,人所共欽,而內無聲色狗馬之好,外無神仙土木之營,臨難慷慨,合國君死社稷之義。千古未有之聖主,宜尊以千古未有之徽稱。”

\subsection{崇祯}

\begin{longtable}{|>{\centering\scriptsize}m{2em}|>{\centering\scriptsize}m{1.3em}|>{\centering}m{8.8em}|}
  % \caption{秦王政}\
  \toprule
  \SimHei \normalsize 年数 & \SimHei \scriptsize 公元 & \SimHei 大事件 \tabularnewline
  % \midrule
  \endfirsthead
  \toprule
  \SimHei \normalsize 年数 & \SimHei \scriptsize 公元 & \SimHei 大事件 \tabularnewline
  \midrule
  \endhead
  \midrule
  元年 & 1628 & \tabularnewline\hline
  二年 & 1629 & \tabularnewline\hline
  三年 & 1630 & \tabularnewline\hline
  四年 & 1631 & \tabularnewline\hline
  五年 & 1632 & \tabularnewline\hline
  六年 & 1633 & \tabularnewline\hline
  七年 & 1634 & \tabularnewline\hline
  八年 & 1635 & \tabularnewline\hline
  九年 & 1636 & \tabularnewline\hline
  十年 & 1637 & \tabularnewline\hline
  十一年 & 1638 & \tabularnewline\hline
  十二年 & 1639 & \tabularnewline\hline
  十三年 & 1640 & \tabularnewline\hline
  十四年 & 1641 & \tabularnewline\hline
  十五年 & 1642 & \tabularnewline\hline
  十六年 & 1643 & \tabularnewline\hline
  十七年 & 1644 & \tabularnewline
  \bottomrule
\end{longtable}


%%% Local Variables:
%%% mode: latex
%%% TeX-engine: xetex
%%% TeX-master: "../Main"
%%% End:
