%% -*- coding: utf-8 -*-
%% Time-stamp: <Chen Wang: 2021-11-01 17:13:00>

\section{世宗朱厚熜\tiny(1521-1566)}

\subsection{生平}

明世宗朱厚熜(1507年9月16日-1567年1月23日),或稱嘉靖帝,明朝第12位皇帝,庙号世宗,年號嘉靖,正德十六年(1521年),明武宗駕崩無嗣,內閣首輔楊廷和立朱厚熜入繼大統,即明世宗。谥号“钦天履道英毅神圣宣文广武洪仁大孝肃皇帝”。

世宗前期进行改革,銳意圖治,颇有作為,他说:“今天下诸司官员,比旧过多。我太祖初无许多,后来增添冗滥,以致百姓艰窘,日甚一日。”下令革除先朝蠹政,又嚴以馭下,史稱其“世宗習見正德時宦侍之禍,即位後御近侍甚严,有罪挞之至死,或陈尸示戒...又盡撤天下鎮守內臣及典京營倉場者,終四十餘年不復設,故內臣之勢,惟嘉靖朝少殺雲。”,先後裁革錦衣衛十七萬餘人。且寸斬前朝王綸、钱宁和江彬等奸臣,天下翕然稱治,時稱嘉靖中興。

但世宗受人詬病處更多,如他為了追封生父興獻王的問題,與楊廷和等朝臣引發嚴重衝突,即大禮議事件,世宗為了此事,對大臣們進行了嚴重的大清洗。世宗在位中後期也漸無心朝政,深居不出,沉迷方術,只通過內閣掌控朝局,使得嚴嵩嚴世蕃父子專權逐漸形成,又因營建繁興而濫用民力,導致府藏告匱,民眾起義無數。在宮中,世宗也暴虐無道,因為虐待宮女,導致宮女發動壬寅宮變,險些喪命。

明世宗朱厚熜是明宪宗第四子兴献王朱祐杬次子,是明孝宗之姪,明武宗之堂弟;明武宗正德二年(1507)生,母兴王妃蒋氏。

正德十六年(1521年),明武宗驾崩,無子嗣,内阁首辅吏部尚书、武英殿大学士杨廷和定策,援引《皇明祖训》,推找皇位繼承人,而武宗唯一弟弟朱厚煒幼年夭折,於是上推至武宗父明孝宗一輩,孝宗是明憲宗的第三子,兩名兄長皆早逝無子嗣,四弟興王朱祐杬雖已薨,但有二子,興王長子(朱厚熙)已薨,遂以“兄終弟及”的原則,徵在服喪的興國世子朱厚熜入京即位。朱厚熜先繼承興王頭銜,後即帝位,改元“嘉靖”,是为明世宗。

朱厚熜十四歲入繼大統,因想追封親生父母「皇帝、皇后」的尊號,但首辅杨廷和等旧臣要求他改以明孝宗為義父,而引發了長達三年半的大禮議之爭,期間廷杖打死十六人;世宗不顧朝臣反對,追尊生父為興獻帝、生母為興國皇太后,改稱孝宗曰“皇伯考”。嘉靖十七年(1538年)九月興獻帝被追尊為「睿宗知天守道洪德淵仁寬穆純聖恭簡敬文獻皇帝」,並將睿宗的牌位升袝太廟,排序在明武宗之上,改興獻王墓為顯陵,大禮議事件至此最終結束。

嘉靖帝前期推行了改革,成效显著。河南道御史刘安说:“今明天子综核于上,百执事振于下,丛蠹之弊,十去其九,所少者元气耳。”张居正在万历三年(1575)以自己少年时的亲身体验对嘉靖前期整顿学政的成就予以极高的评价。他说:“臣等幼时,犹及见提学官多海内名流,类能以道自重,不苟徇人,人亦无敢干以私者。士习儒风,犹为近古。”

隆庆二年(1568)进士李乐对嘉靖前期革除镇守中官的积极作用给予的评价,言道:“世宗皇帝继统,年龄虽小,英断夙成,待此辈不少假借。又得张公孚敬以正佐之,尽革各省镇守内臣,司礼监不得干预章奏。往瑾时,公卿大臣相见,无敢抗礼,甚有拜伏者。自张公当国,司礼以下各监局巨珰,见公竦息敬畏,不敢并行并坐,至以『张爷』呼之,不动声色,而潜消其骄悍之心。盖自汉唐宋元以来,宦官敛戢,士气得伸,国体尊严,未有如今日者,诚千载一时哉!”

因應外戚为害天下,嘉靖帝和张璁、方献夫在革除外戚世封的问题上达到了共识,下令永远废除此制,《明通鉴》编纂者说:“安昌伯钱维圻卒,其庶兄维垣请嗣爵,下吏部议。尚书方献夫等言:‘外戚之封,不当世及。’历引汉、唐、宋事以证。璁以为然,力主之。上善其言,诏:”自今外戚封爵者,但终其身,毋得请袭。’自是,外戚遂永绝世封。”

明代史学家何乔远《名山藏》总结嘉靖前期“励精化理,湔濯海内观听,挈清政本,杜塞旁落,奋武揆文,网罗才实。至于稽古礼典,取次厘毖一切,创必表章,轶往宪来,赫然中兴,多孚敬(张璁)所翼赞”。何乔远认为嘉靖前期出现的国家中兴是得益于內閣首辅张璁推行的改革。

而在嘉靖中后期,海瑞于嘉靖四十五年亦言:世宗“二十余年不视朝,法纪弛矣”。

世宗濫用夫役與國家財政之力大事興建,迷信方士、尊崇道教,好長生不老之術,每年不斷修設齋醮,造成巨大的靡費。

世宗好房中術秘方,多採處女之經血煉丹,方士陶仲文與佞臣顾可学、盛端明等进献媚药得以倖進,世宗為人暴躁兇殘,朝鮮國使臣的著作,也稱他對宮女:「若有微過,多不容恕,輒加箠楚。因此殞命者,多至二百餘人。」嘉靖二十一年(1542年)十月爆發“壬寅宮變”,幾死於宮女之手。明朝的太醫許紳用“虎狼之药”救活世宗,但是,由于他在急救世宗皇帝时,承受着“不效必杀身”的巨大压力,不多久,许绅得了病,卧床不起,嘉靖帝来看望他。他说:“吾不起矣,曩者宫变,吾自度不效必杀身,因此惊悸,非药石所能疗。”病卒,赐谥恭僖。此後世宗相繼遷居西苑萬壽宮及玉熙宮謹身精舍,至死不曾回到紫禁城大內居住,直至瀕死前才在徐階以明武宗死在宮外為例子勸說下回到大內居住。首輔严嵩專國二十年,殘害忠良,楊繼盛、沈鍊等朝臣慘遭殺害。

嘉靖朝吏治敗壞,爆发多起农民起义,如:山東礦工起義、陳卿起義、蔡伯貫起義、浙贛礦工起義、李亞元起義、賴清規起義,邊事廢弛,1524年以後爆發多起大同兵變,1535年爆發遼東兵變,1560年爆發振武營兵變,長城北方蒙古鞑靼俺答汗寇邊,倭寇侵略中国東南沿海,就是“北虜南倭”的問題,後賴朱紈、戚繼光、俞大猷等人率軍肅清倭寇。世宗在位之時,葡萄牙人遠航當時屬广东省香山县管辖的澳門,並“借地晾晒水浸货物”为借口開始於澳門定居,從而在澳門展開了接近450年的葡萄牙佔領及殖民時期。

嘉靖四十四年(1565年)正月,方士王金等伪造《诸品仙方》、《养老新书》,制长生妙药献世宗。嘉靖四十五年二月,(1566年)戶部主事海瑞上《治安疏》,世宗初大怒,擲疏於地,並下詔讓錦衣衛及三法司論罪。但后重置御案上數日內再三閱讀。后法司擬處大辟的刑罰,但世宗審閱後卻留中不發,以致海瑞終未獲刑。

嘉靖四十五年十二月初八,世宗免去臘宴。十四日,世宗病篤,時隔二十多年重新住回大內。當日午時,於乾清宮駕崩,享壽六十岁。徐階請裕王入宮主持大行皇帝喪禮。裕王自東安門入,至乾清宮御榻前發喪。次日,大行皇帝小殮,并發佈遺詔。十六日,大殮,并上廟號世宗。

隆慶元年三月十一日,世宗梓宮及祔葬孝洁皇后、孝恪皇后梓宫離開北京。十六日,世宗及孝潔皇后、孝恪皇后梓宮抵達永陵。次日,世宗入葬永陵。

嘉靖皇帝醉心于西苑修仙斋醮,直到他最后死去,却一直是“虽深居渊穆而威柄不移”,虽数十年不见朝臣,仍能做到“大张弛、大封拜、大诛赏,皆出独断,至不可测度。”明世宗非常聪明,也十分勤奋,批阅奏书票拟经常到后半夜。但嘉靖后期,朝中官员贪污纳贿、奢侈靡费,確已成普遍的现象。

《明史·世宗本紀》:“贊曰:世宗御極之初,力除一切弊政,天下翕然稱治。顧迭議大禮,輿論沸騰,幸臣假托,尋興大獄。夫天性至情,君親大義,追尊立廟,禮亦宜之;然升祔太廟,而躋於武宗之上,不已過乎!若其時紛紜多故,將疲於邊,賊訌於內,而崇尚道教,享祀弗經,營建繁興,府藏告匱,百餘年富庶治平之業,因以漸替。雖剪剔權奸,威柄在御,要亦中材之主也矣。”

《国榷》:“世庙起正德之衰,厘革积习,诚雄主也。因议礼自裁,好稽古右文之事,诸臣迎附,祗诤于仪节,反实政略焉。”

《名山藏》:“臣喬遠曰:臣每見故縉紳父老,若為郎時尚接先朝皆御之臣,多好言嘉靖時事,其謨猷合聖賢,動作掀天地,真中興之主矣。晚節西苑崇玄,帝心固以為敬天,雖萬幾在宥而精神無時不運,於天下者四十餘年如一日,所以饗世獨久歟。”

\subsection{嘉靖}

\begin{longtable}{|>{\centering\scriptsize}m{2em}|>{\centering\scriptsize}m{1.3em}|>{\centering}m{8.8em}|}
  % \caption{秦王政}\
  \toprule
  \SimHei \normalsize 年数 & \SimHei \scriptsize 公元 & \SimHei 大事件 \tabularnewline
  % \midrule
  \endfirsthead
  \toprule
  \SimHei \normalsize 年数 & \SimHei \scriptsize 公元 & \SimHei 大事件 \tabularnewline
  \midrule
  \endhead
  \midrule
  元年 & 1522 & \tabularnewline\hline
  二年 & 1523 & \tabularnewline\hline
  三年 & 1524 & \tabularnewline\hline
  四年 & 1525 & \tabularnewline\hline
  五年 & 1526 & \tabularnewline\hline
  六年 & 1527 & \tabularnewline\hline
  七年 & 1528 & \tabularnewline\hline
  八年 & 1529 & \tabularnewline\hline
  九年 & 1530 & \tabularnewline\hline
  十年 & 1531 & \tabularnewline\hline
  十一年 & 1532 & \tabularnewline\hline
  十二年 & 1533 & \tabularnewline\hline
  十三年 & 1534 & \tabularnewline\hline
  十四年 & 1535 & \tabularnewline\hline
  十五年 & 1536 & \tabularnewline\hline
  十六年 & 1537 & \tabularnewline\hline
  十七年 & 1538 & \tabularnewline\hline
  十八年 & 1539 & \tabularnewline\hline
  十九年 & 1540 & \tabularnewline\hline
  二十年 & 1541 & \tabularnewline\hline
  二一年 & 1542 & \tabularnewline\hline
  二二年 & 1543 & \tabularnewline\hline
  二三年 & 1544 & \tabularnewline\hline
  二四年 & 1545 & \tabularnewline\hline
  二五年 & 1546 & \tabularnewline\hline
  二六年 & 1547 & \tabularnewline\hline
  二七年 & 1548 & \tabularnewline\hline
  二八年 & 1549 & \tabularnewline\hline
  二九年 & 1550 & \tabularnewline\hline
  三十年 & 1551 & \tabularnewline\hline
  三一年 & 1552 & \tabularnewline\hline
  三二年 & 1553 & \tabularnewline\hline
  三三年 & 1554 & \tabularnewline\hline
  三四年 & 1555 & \tabularnewline\hline
  三五年 & 1556 & \tabularnewline\hline
  三六年 & 1557 & \tabularnewline\hline
  三七年 & 1558 & \tabularnewline\hline
  三八年 & 1559 & \tabularnewline\hline
  三九年 & 1560 & \tabularnewline\hline
  四十年 & 1561 & \tabularnewline\hline
  四一年 & 1562 & \tabularnewline\hline
  四二年 & 1563 & \tabularnewline\hline
  四三年 & 1564 & \tabularnewline\hline
  四四年 & 1565 & \tabularnewline\hline
  四五年 & 1566 & \tabularnewline
  \bottomrule
\end{longtable}


%%% Local Variables:
%%% mode: latex
%%% TeX-engine: xetex
%%% TeX-master: "../Main"
%%% End:
