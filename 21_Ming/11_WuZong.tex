%% -*- coding: utf-8 -*-
%% Time-stamp: <Chen Wang: 2021-11-01 17:12:49>

\section{武宗朱厚照\tiny(1505-1521)}

\subsection{生平}

明武宗朱厚照(1491年10月27日-1521年4月20日),或稱正德帝,明朝第11代皇帝(1505年-1521年在位),享年 31歲,年号「正德」。

武宗是明朝极具争议性的统治者。他任情恣性,為人嬉乐胡鬧,荒淫无度。寵信宦官、建立豹房,強徵處女、娈童入宮,有時也搶奪有夫之婦,逸遊無度。施政荒誕不經,朝廷乱象四起。給自己化名為朱壽,自封為「鎮國公、總督軍務威武大將軍、總兵官」。又信仰密宗、伊斯蘭教等,自稱忽必烈(蒙古名,元世祖之名)、沙吉熬爛(波斯語,伊斯蘭教蘇菲派的蘇菲師)、大寶法王(藏密名,白教首領)。

另一方面,他為人刚毅果断,任内诛灭刘瑾,平定安化王、寧王之亂,在应州之役中击败達延汗,令鞑靼多年不敢深入,并积极学习他国文化,促进中外交流,体现出有为之君的素质,是一位功过参半的皇帝。

明武宗朱厚照为明孝宗嫡长子,生于1491年10月26日(弘治四年九月二十四日申时)。两岁被立为皇太子。唯一的弟弟朱厚炜又早夭,是孝宗唯一长大成人的儿子。弘治十一年春,皇太子出阁读书。他天性聪颖,讲筵时极为认真,面对讲师则恭敬对待。几个月后,便已知晓翰林院与左春坊所有讲师的姓名,以致有讲师缺席便会问询左右“某先生今日安在邪?”這讓孝宗极为喜爱,出游必带上皇太子。同时孝宗听闻皇太子闲暇时喜好兵戎事,认为他安不忘危,所以也不予以干涉。

弘治十八年五月初八日,孝宗皇帝驾崩。在完成文武百官军民耆老劝进的固定程序后,五月十八日,皇太子朱厚照即位,是为明武宗。

明正德九年正月,後來反叛的寧王朱宸濠獻新樣元宵四時花燈數百,窮極奇巧,內附火藥,明武宗命獻者入懸。时值冬季,宫中按例在檐下设有毡幕御寒。以致火星觸及氊幕,引發大火,自二鼓时分一直烧至天明。火势最大时,武宗正在前往豹房的途中,望见乾清宫的火灾,武宗向左右开玩笑称这是「好一棚大烟火也」兩天後壬午日,武宗以乾清宫灾御奉天門視朝,撤寶座不設,遂下詔罪己,並諭文武百官,同加修省。後又常常离开帝都燕京四处巡游。

住在京師期间,又不愿住在紫禁城,在宫外建了一座“豹房”居住,並甄選大量美女於其中供其淫樂。其男宠也不计其数,名曰“老儿当”,但也有學者稱,因為正德帝喜歡各地宗教,這些人主要是通曉漢文、蒙文、藏文或波斯文,作為宗教人士的翻譯官。

正德帝不喜上朝,起初宠信刘瑾、張永、丘聚、谷大用等号称“八虎”的宦官,1510年平定安化王之乱朱寘鐇后,下令将刘瑾凌迟处死,后又宠信武士江彬等人。

正德帝喜好宗教靈異、怪力亂神,终日与来自西域、回回、蒙古、乌斯藏(西藏)、朝鲜半島的异域法師、番僧相伴。正德帝曾学习蒙古语,自称忽必烈,也学藏传佛教,自称大宝法王。正德帝還曾亲自接见第一位来华的葡萄牙使者皮莱资。正德帝並因為自己生肖屬豬,曾一度敕令全国禁食猪肉,但他自己仍食用猪肉「内批仍用豕」;旋即在大學士杨廷和的反對下,降敕廢除。

正德帝“奋然欲以武功自雄”。正德十二年(1517年)10月,在江彬的怂恿下,自封为“镇国公總督軍務威武大將軍總兵官朱寿”,到边地宣府(今张家口宣化区)亲征,击溃蒙古鞑靼小王子(即达延汗巴图蒙克),回去后又给自己加封太师。史称“应州大捷”。

正德十四年(1519年)六月十四日,宁王朱宸濠在封藩江西南昌叛乱,是為宁王之乱,不過四十三天,就被贛南巡撫王陽明及吉安知府伍文定募集散兵游勇平定,斬殺三萬餘人,朱宸濠被擒。八月二十二日,武宗离开北京亲征。二十六日,武宗抵达涿州,此時王陽明平定叛乱的奏报送达,但武宗仍决定继续南幸。十二月十一日,武宗传谕内阁,以正德十五年(1520年)元旦於南京朝贺、祭祀天地。十二月二十六日,武宗御驾抵应天府。次日,祭祀南京太庙,武宗成为自永乐以后重新驾临南京的皇帝。正德十五年闰八月初八日,武宗於南京受宁王降。八月十二日,武宗离京返回北京。

正德八年(1513年)起在江南全面推行的賦稅改革,既減輕了江南當地百姓的負擔,更使從弘治晚期開始,江南地區拖欠中央累積十年之久的賦稅,僅經兩年時間就全部還清。

武宗御驾南征返回北京途中,於淮安清江浦上学渔夫撒网,作為遊戲,卻失足落入水中,并因此患病「燥熱難退」。正德十五年十二月初十,大驾回到北京,文武百官出至正阳桥外迎接。十三日,皇帝於南郊祭祀天地,祭拜过程中突然呕血,随即送入斋宫休养。次日,返回大内,仅在奉天殿举行庆成礼。此后,立春日的朝贺一同免去。正德十六年(1521年)正月初九日,监察御史郑本公鉴于武宗身体状况不乐观,上奏武宗,望能於宗室间過繼一人主掌东宫,但后来武宗身体略有好转。三月十三日晚间,武宗突然向身边的太监陈敬和苏进表示自己可能無法痊癒,让其召司礼监并禀告皇太后,由太后与内阁议处天下事,并表示自己耽误子嗣。十四日,武宗於豹房驾崩,得年29歲。

由于武宗無子嗣,因此遵照《皇明祖训》,由武宗堂弟、孝宗弟兴献王朱祐杬之子兴王朱厚熜入嗣大统。正德十六年五月,朱厚熜抵达京师,上谥号为承天达道英肃睿哲昭德显功弘文思孝毅皇帝,上庙号为武宗。九月,武宗入葬天寿山陵区的康陵。

明武宗的生辰为弘治四年九月二十四日,八字为辛亥年,戊戌月,丁酉日,戊申时出生。其中,八字地支分别为申酉戌亥,这种排列方法被称为连如贯珠。在此以前仅太祖朱元璋的八字与此类似。

賜自己的替僧為漢地噶瑪巴,正德五年封大慶法王,鑄大慶法王西天覺道圆明自在大定慧佛金印,兼给誥命,藏名為「領占班丹」,並曾邀請藏地八代噶瑪巴至北京(七代噶瑪巴曾說:「將現身兩位噶瑪巴」);蒙古名為忽必烈;波斯名為沙吉熬爛,即蘇菲師(Shaykh,回教蘇菲派長者、教長),並擁有一群伊斯蘭火者,稱為老兒當。對道教亦多有了解,可能曾號錦堂老人。

正德十五年(1520年)闰八月,武宗御驾自南京返回时,途径镇江,适逢退休居家的原内阁大臣靳贵病逝,于是亲临靳贵家中吊唁。但是随行大臣代皇帝撰写的祭文皆不能称意,明武宗遂亲自写道:“朕居东宫,先生为傅。朕登大宝,先生为辅。朕今南游,先生已矣。呜呼哀哉!”左右的侍从文学之臣看后都敛手称服。

山西应县木塔顶层有一方明武宗皇帝御匾“天下奇观”。

2004年,在美國德州一位華僑手中發現由明朝正德皇帝親筆所書的聖旨,內容敘述做人應如何有進取心以及如何為忠君之臣與正人君子。此文物的發現造成了史學家對歷史記載正德皇帝人格的爭議。

史学界对正德帝的评价不一, 有人认为正德帝雖荒淫無行,行徑胡鬧,不理國政,造成叛變日起,且自身壯年即因為逸樂而死;但是亦有人认为他頗能容忍大臣,不罪勸諫之人。君臣之間,相安無事,知错能改,诛灭奸佞。

张廷玉等《明史》贊曰:「明自正統以來,國勢浸弱。毅皇手除逆瑾,躬禦邊寇,奮然欲以武功自雄。然耽樂嬉遊,暱近群小,至自署官號,冠履之分蕩然矣。猶幸用人之柄躬自操持,而秉鈞諸臣補苴匡救,是以朝綱紊亂,而不底於危亡。假使承孝宗之遺澤,制節謹度,有中主之操,則國泰而名完,豈至重後人之訾議哉!」

談遷《國榷》論曰:「武宗少即警敏,好佚樂。……而武宗又不罪一諫臣,元相呵護,群吏奉法。……夜半出片紙縛(劉)瑾,……錢寧俛首受罪。」

吳熾昌《續客窗閒話》論曰:「……遊戲中確有主裁,但好行小慧,為儒尚且不可,況九五之尊耶?今之讀史者直以帝比之桀紂,無乃過甚。當初諡曰武宗毅皇帝,毅者果決之謂,可見遇事實能決斷,非盡阿諛可知矣。」

\subsection{正德}

\begin{longtable}{|>{\centering\scriptsize}m{2em}|>{\centering\scriptsize}m{1.3em}|>{\centering}m{8.8em}|}
  % \caption{秦王政}\
  \toprule
  \SimHei \normalsize 年数 & \SimHei \scriptsize 公元 & \SimHei 大事件 \tabularnewline
  % \midrule
  \endfirsthead
  \toprule
  \SimHei \normalsize 年数 & \SimHei \scriptsize 公元 & \SimHei 大事件 \tabularnewline
  \midrule
  \endhead
  \midrule
  元年 & 1506 & \tabularnewline\hline
  二年 & 1507 & \tabularnewline\hline
  三年 & 1508 & \tabularnewline\hline
  四年 & 1509 & \tabularnewline\hline
  五年 & 1510 & \tabularnewline\hline
  六年 & 1511 & \tabularnewline\hline
  七年 & 1512 & \tabularnewline\hline
  八年 & 1513 & \tabularnewline\hline
  九年 & 1514 & \tabularnewline\hline
  十年 & 1515 & \tabularnewline\hline
  十一年 & 1516 & \tabularnewline\hline
  十二年 & 1517 & \tabularnewline\hline
  十三年 & 1518 & \tabularnewline\hline
  十四年 & 1519 & \tabularnewline\hline
  十五年 & 1520 & \tabularnewline\hline
  十六年 & 1521 & \tabularnewline
  \bottomrule
\end{longtable}


%%% Local Variables:
%%% mode: latex
%%% TeX-engine: xetex
%%% TeX-master: "../Main"
%%% End:
