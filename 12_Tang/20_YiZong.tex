%% -*- coding: utf-8 -*-
%% Time-stamp: <Chen Wang: 2021-10-29 16:48:01>

\section{懿宗李漼\tiny(859-873)}

\subsection{生平}

唐懿宗李漼(cuǐ,833年12月28日-873年8月15日),唐朝第20代皇帝(除去武则天),859年至873年在位,在位14年,终年41岁。

李漼初名温,唐宣宗李忱的長子。宣宗病死後,被宦官迎立為帝,是為唐懿宗,改元“咸通”。死後葬於簡陵,谥号昭聖恭惠孝皇帝。

大和七年十一月十四日(833年),生于籓邸,时父亲李忱为光王,母晁氏为其妾室。初名李温。父亲继位后,于会昌六年十月,封李温为郓王。当时,唐宣宗喜欢第三子夔王李滋,欲立为皇太子,而李温年长,久而不决。大中十三年(859年)八月,宣宗病逝,左神策护军中尉王宗实、副使丌元实矯詔立李温为皇太子。

唐懿宗「器度沈厚,形貌瑰偉」、「洞曉音律,猶如天縱」但遊宴無度、沉湎酒色,以致政治腐敗,藩鎮割據重新興起。

由於崇仰佛法,咸通十四年春(873年),不顧大臣反對,舉行最大規模的迎奉佛骨活動。

他將其父唐宣宗大中之治的成果損耗殆盡。翰林學士劉允章在《直諫書》用“國有九破”描繪過當時的局勢:“終年聚兵,一破也。蠻夷熾興,二破也。權豪奢僭,三破也。大將不朝,四破也。廣造佛寺,五破也。賂賄公行,六破也。長吏殘暴,七破也。賦役不等,八破也。食祿人多,輸稅人少,九破也。”此时唐朝已無可救藥,大動亂正在醞釀。當時赋税刻薄,百姓無法過活,更有人吃人惨劇,百姓无路可走,只好起义。859年,裘甫在浙東起兵;868年,庞勋领导徐泗地区的戍兵在桂林起兵。懿宗派遣王式、康承訓等镇压,但對人民的剥削並無停止。

懿宗死後隨即引发導致唐朝灭亡的黃巢之亂,因而被认为是唐朝間接的亡國之君。

《唐人傳奇》、《太平廣記》對於咸通年間有諸多著墨。

\subsection{咸通}

\begin{longtable}{|>{\centering\scriptsize}m{2em}|>{\centering\scriptsize}m{1.3em}|>{\centering}m{8.8em}|}
  % \caption{秦王政}\
  \toprule
  \SimHei \normalsize 年数 & \SimHei \scriptsize 公元 & \SimHei 大事件 \tabularnewline
  % \midrule
  \endfirsthead
  \toprule
  \SimHei \normalsize 年数 & \SimHei \scriptsize 公元 & \SimHei 大事件 \tabularnewline
  \midrule
  \endhead
  \midrule
  元年 & 860 & \tabularnewline\hline
  二年 & 861 & \tabularnewline\hline
  三年 & 862 & \tabularnewline\hline
  四年 & 863 & \tabularnewline\hline
  五年 & 864 & \tabularnewline\hline
  六年 & 865 & \tabularnewline\hline
  七年 & 866 & \tabularnewline\hline
  八年 & 867 & \tabularnewline\hline
  九年 & 868 & \tabularnewline\hline
  十年 & 869 & \tabularnewline\hline
  十一年 & 870 & \tabularnewline\hline
  十二年 & 871 & \tabularnewline\hline
  十三年 & 872 & \tabularnewline\hline
  十四年 & 873 & \tabularnewline\hline
  十五年 & 874 & \tabularnewline
  \bottomrule
\end{longtable}


%%% Local Variables:
%%% mode: latex
%%% TeX-engine: xetex
%%% TeX-master: "../Main"
%%% End:
