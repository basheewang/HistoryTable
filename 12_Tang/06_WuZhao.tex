%% -*- coding: utf-8 -*-
%% Time-stamp: <Chen Wang: 2021-10-29 15:29:04>

\section{周武曌\tiny(683-705)}

\subsection{生平}

武曌(624年2月17日-705年12月16日),唐高宗的皇后、武周開國皇帝,當代稱則天順聖皇后,或武后(神龍革命後成為皇太后,遺詔退稱皇后),後代通称武则天,并州文水县人,中国历史上因執掌君權而得到正史唯一承認的女性皇帝。十四歲入宮為唐太宗才人,十二年不得遷。唐高宗时復为昭儀,謀廢得到唐太宗託付于重臣褚遂良的“佳兒佳婦”元后與淑妃,得立为皇后(655年-683年)。一時尊号为天后,与唐高宗天皇李治并称“二圣”。由于唐高宗患风眩病,無力聽政,660年11月开始臨朝,史载“自此内辅国政数十年,威势与帝无异”,683年12月27日-690年10月16日作为唐中宗、唐睿宗的皇太后临朝称制,后利用酷吏集團屢次屠殺唐室諸王大臣以求立威,殺害嫌疑對象遍及子、女、媳、婿、孫、孫女、孫婿、庶子、嫡兄、親姊、親甥女、夫之伯叔姑嫂、堂兄,終於自立为武周皇帝(690年10月16日-705年2月21日在位),在位時間共14年4個月又5天。晚年惑于內寵,不知當立侄或立子。705年元月,被宰相狄仁傑舉薦的後任張柬之與禁衛軍背叛,被迫還位。退位以後,成為中國歷史上唯一一位女性太上皇,同年崩于洛陽上陽宮仙居殿。唐高宗死后从683年实际真正掌权前后22年。武則天是即位年龄最大(67岁即位)、寿命第三长的皇帝(终年82岁),僅次於清高宗(87歲)和梁武帝(86歲)。

武氏本名无记载,为唐开国勳舊武士彠次女,母亲杨氏為隋朝宗室楊達之女是武士彠繼室,不見禮于正室諸子。祖籍并州文水县(今山西省文水县),十四岁時(貞觀十一年)因貌美而入后宫为唐太宗的才人,唐太宗赐号武媚。高宗时为昭仪,后封为皇后,又上尊号为“天后”。高宗崩,中宗即位,武氏为皇太后,临朝称制后改名曌。武氏認為自己好像日、月一樣崇高,凌掛於天空之上。於称帝后上尊号“聖神皇帝”,退位后中宗上尊号“则天大圣皇帝”,武氏遗制去帝号,称“则天大圣皇后”。武氏另有废除的尊号“圣母神皇、圣神皇帝、金轮圣神皇帝、越古金轮圣神皇帝、慈氏越古金轮圣神皇帝、天册金轮圣神皇帝”等。在位期間喜土木作造,尤喜造國字改年號,一年一號。傳說洛陽龍門石窟的奉先寺大佛是模仿其面容而作。

武氏十四歲时,唐太宗聽聞她姿色豔美,將她納入宮中,据《资治通鉴》所载,时在貞觀十一年(637年)十一月。入宫后,武氏封為五品才人,賜號「武媚」后世讹称武媚娘。武氏入宮之前向寡居的母親楊氏告別時說:「侍奉聖明天子,豈知非福?為何還要哭哭啼啼、作兒女之態呢?」

武才人与太宗的三位嫔妃燕德妃、杨婕妤、巢王妃杨氏俱为表亲。而对于太宗时期武氏在宫中的生活细节,史书并没有详细的描述。仅见武氏在晚年时回忆自己为太宗驯马一事。当时,太宗有名马獅子骢,又肥又暴躁没有能调教牠的人。武氏在太宗身边侍候,对太宗说:「我能制服牠,但是须要三件东西:一是铁鞭,二是铁楇,三是匕首。用铁鞭打它不服,就用楇打牠的头;再不服,就用匕首割断牠的喉咙。」武氏稱太宗壮其之志。複自稱嘗侍太宗,得其書法之妙。

貞觀十七年(643年),太子李承乾被廢,晉王李治被立为太子。此後,在太子侍奉太宗湯藥之際,李治見到武才人並悅之。

贞观二十三年(649年),唐太宗駕崩。武才人依唐後宮之例,入感業寺剃髮出家。永徽元年(650年)五月,唐高宗在太宗週年忌日入感業寺進香之時,與身為比丘尼的武氏相遇。當時與蕭淑妃爭寵的王皇后知悉後,便主動向高宗請求將武氏納入宮中,企圖以此打擊蕭淑妃。唐高宗早有此意,當即應允。永徽二年(651年)五月,唐高宗的孝服已滿,二十七歲的武氏還俗,再度入宮。入宮前武氏已經懷孕,入宮後生下兒子李弘。次年五月,被拜為二品昭儀。

永徽六年(655年)六月,後宮中有人放出不利王皇后之謠言,流傳王皇后與其母柳氏(宰相柳奭之姊,柳宗元同族)請來巫師,企圖用魘鎮之術將武昭仪詛咒而死亡。這謠言在無證據下傳到高宗之耳,高宗大怒,並將其母柳氏趕出皇宮,而且欲將武昭仪陞為一品宸妃(唐朝後宮四夫人中本來並無宸妃此封號,而原本的四夫人名額已滿,唐高宗為了武氏,才創宸妃封號),受到宰相韓瑗和來濟的反對,最後不能成事。不久,中書舍人李義府等人勾結武氏,得知高宗欲行廢皇后而立武昭仪消息,聯絡本已貶官不得再進的許敬宗、崔義玄、袁公瑜等人向唐高宗不斷請求立武昭仪為后,造成群臣支持的表象,廢立之意遂再次萌生。

永徽六年(655年)十月十三日,唐高宗又在李世勣等朝廷武勛的模棱兩可下,終於頒下詔書:以「陰謀下毒」的罪名,將王皇后和蕭淑妃廢為庶人,並加囚禁;她們的父母、兄弟等也被削爵免官,流放嶺南。七天以後,唐高宗再次下詔,將武昭仪立為皇后;與此同時,又將反對最大的宰相褚遂良貶為外州都督。因为忌讳武氏曾为父亲太宗才人的事实,唐高宗在立后诏书中,称武氏为父亲所赐,“事同政君”。

顯慶四年(659年)四月,武后與唐高宗達成共識:將長孫無忌、于志寧、韓瑗、來濟等人削職免官,貶出京師。

顯慶五年(660年),高宗患上風疾之症,頭暈目眩,不能處理國家大事,遂命皇后武氏代理朝政。在麟德元年(664年),高宗与宰相上官儀商議,打算廢掉武氏皇后之位。但上官儀的廢后詔書還未草擬好,武后即已從宦官親信接到消息。她直接來到高宗面前追問此事,唐高宗不得已,便把責任推到上官儀身上。十二月,上官儀被逮捕入獄,不久,即被滅族。

乾封二年(667年)高宗因久疾,命太子弘監國。上元元年(674年)秋八月,武后和高宗並稱天皇天后,名為避先帝、先后之稱,實欲自尊。十二月,武后上表建議十二事:「一、勸農桑,薄賦徭。二、給復三輔地(免除長安及其附近地區之徭役)。三、息兵,以道德化天下。四、南、北中尚(政府手工工場)禁浮巧。五、省功費力役。六、廣言路。七、杜讒口。八、王公以降皆習《老子》。九、父在為母服齊衰(喪服)三年(過去是一年)。十、上元前勳官已給告身(委任狀)者,無追覈。十一、京官八品以上,益稟入(增加薪水)。十二、百官任事久,材高位下者,得進階(提級)申滯。」高宗詔皆施行之。武則天能夠重視農業生產,規定各州縣境內,「田疇墾闢,家有餘糧」者予以升獎;「為政苛濫,戶口流移」者必加懲罰。所編《兆人本業》農書,頒行天下,影響很大。而武則天執政期間,其宗教政策乃以佛教在道教之上。

上元二年(675年)三月,武后召集大批文人學士,大量修書,先後撰成《玄覽》、《古今內範》、《青宮紀要》、《少陽正範》、《維城典訓》、《紫樞要錄》、《鳳樓新誡》、《孝子傳》、《列女傳》、《內範要略》、《樂書要錄》、《百僚新誡》、《兆人本業》、《臣軌》等書。且密令這批學者參決朝廷奏議,以分宰相之權,時人謂之「北門學士」。時高宗風眩更甚,擬使武后攝政,宰相郝處俊說:「陛下奈何以高祖、太宗之天下,不傳之子孫而委之天后乎!」高宗才罷攝政之意。太子李弘深為高宗鍾愛,高宗欲禪位於太子,武后不滿;剛好太子因为蕭淑妃之女義陽、宣城二公主因母得罪武后而被幽禁掖庭宮中、年逾21而未嫁,奏請出降,高宗許之,武后甚怒。不久太子死於合璧宮,時人以為武后所毒殺,但亦有说法称李弘本来病弱而早夭。

弘道元年(683年)十二月,唐高宗病逝,臨終遺詔:太子李顯於柩前即位,軍國大事有不能裁決者,由武氏決定。四天以後,李顯即位,是為唐中宗。武后被尊為皇太后。

光宅元年(684年)二月,中宗欲以韋后父韋玄貞為侍中(宰相),裴炎力諫不聽,武后遂廢唐中宗為廬陵王,並遷於房州。立第四子豫王李旦為帝,是為唐睿宗,武后臨朝稱制,自專朝政。同年九月,徐敬業、徐敬猷兄弟聯合唐之奇、杜求仁等以扶支持廬陵王為號召,在揚州舉兵反武,十多天內就聚合了十萬部眾。武后當即以左玉鈐大將軍李孝逸為揚州道大總管,率兵三十萬,前往征討。十一月,徐敬業兵敗自殺。

垂拱二年(686年)三月,武后下令製造銅匭(銅製的小箱子),置於洛陽宮城之前,隨時接納臣下表疏。同時,又大開告密之門,規定任何人均可告密。凡屬告密之人,國家都要供給驛站車馬和飲食。即使是農夫樵人,武后都親自接見。所告之事,如果符合旨意,就可破格陞官。如所告並非事實,亦不會問罪。同時,武后又先後任用索元禮、周興、來俊臣、侯思止等一大批酷吏,掌管制獄,如果被告者一旦被投入此獄,酷吏們則使用各種酷刑審訊,能活著出獄的百無一二。這樣,隨著告密之風的日益興起,被酷吏刑訊拷打致死的人日漸增多。為獎勵告密,若有屬實,武后對告密者破例授官,以賣餅為生的侯思止,因舉發舒王李元名與恒州刺史裴貞謀反,被任命為游擊將軍、侍御史。王弘義,以無德行見稱,告鄉里謀反,擢授游擊將軍、殿中侍御史。

武后掌管李唐的社稷,翦除唐宗室,諸王不自安,欲起兵對抗。還未有共識的時候,博州刺史瑯邪王李沖,垂拱四年(688年)八月於博州(今山東聊城東北)舉兵。豫州刺史越王李貞起兵豫州(今河南汝南)呼應。武后分遣丘神勣、魏崇裕擊之。瑯邪王李沖起兵七日敗死;九月,越王李貞兵敗自殺。武后想盡除李氏諸王,使周興等審訊之,迫韓王李元嘉、魯王李靈夔、黃國公李譔、東莞郡公李融、常樂公主等自殺,親信等均被誅。

這年命令僧薛怀义率令萬多人,毀乾元殿,建明堂,花了近一年落成,高二百九十四尺,闊三百尺。共三層,上為圓蓋,有條九龍作捧著的姿態。上有鐵鳳,高一丈。飾以黃金,稱為「萬象神宮」。明堂既成,又命僧薛怀义鑄大像,大像的小指也可以容納數十人,於明堂北起五層高的天堂來收納這個大像。所花費用以萬億計,政府財政為之枯竭。是年武承嗣命人鑿白石為文曰:「聖母臨人,永昌帝業。」號稱在洛水中發現,獻給武后,武后大喜,命其石曰「寶圖」。之後武后加尊號為「聖母神皇」。

武后當政期間爲了打擊關中著姓預立的“九品中正制”官人法,造成其父系母系皆是“從龍入關”的世家門閥的歷史假象,進一步發展收攏民心的科舉制度。貞觀年間共錄取進士205人,高宗和武后統治期間共錄取一千餘人。平均每年錄取人數比貞觀時增加一倍以上。武后載初元年(690年)武后在洛城殿對貢士親發策問,是「殿試」之始。是年遣「存撫使」十人巡撫諸道,推舉有才之人,一年後共舉薦一百餘人,武后不問出身,全部加以接見,自稱量才任用,或為試鳳閣(中書省)舍人、給事中,或為試員外郎、侍御史、補闕、拾遺、校書郎,試官制度自此始,時人有「補闕連車載,拾遺平斗量,把推侍御史,腕脫校書郎。」之語。武后雖以官位收買人心,但對不稱己意的人亦會加以罷黜;號稱明察善斷,故當時一部份人亦樂於為武后效力。

次年(690年)七月,僧法明等撰《大雲經》四卷,說武后是彌勒菩薩化身下凡,應作為天下主人,武后下令頒行天下。命兩京諸州各置大雲寺一所,藏《大雲經》,命僧人講解,並提升佛教的地位在道教之上。是年九月侍御史傅游藝率關中百姓九百人上表,請改國號為周,賜皇帝姓武。於是百官及帝室宗戚、百姓、四夷酋長、沙門、道士共六萬餘人,亦上表請改國號。武后准所請,改唐為周。在神都则天门登基即位,改元天授,加尊號聖神皇帝,以睿宗為皇嗣,賜姓武氏,以皇太子為皇孫。立武氏祖宗七廟於神都洛陽,追尊周文王廟號曰始祖,諡號文皇帝。立武承嗣為魏王,武三思為梁王,其餘武氏多人為王及長公主。

同年九月,武則天派右鷹揚衛將軍王孝傑為武威軍總管,與武衛大將軍阿史那忠節率兵赴西域征討吐蕃。十月,唐軍大勝,連克于闐、疏勒、龜兹、碎葉等安西四鎮,仍置安西都護府於龜玆,發兵戍守。

长寿三年(694年)武三思率四夷首領請以銅鐵鑄天樞,立於端門外,以歌頌武則天的功德。武則天親題曰:「大周萬國頌德天樞」。天樞鑄造歷時八月而成,其形制若柱,高一百零五尺,直徑十二尺,八面,每面各五尺,下為鐵山,周一百七十尺,以銅為蟠龍、麒麟環繞之;上為騰雲承露盤直徑三丈,盤上四龍直立捧火珠,高一丈。工人毛婆羅造模,武三思為文,刻百官及四夷首領之名於其上。用銅鐵二百萬斤,「請胡聚錢百萬億,買銅鐵不能足,賦民間農器以足之。」

萬歲通天元年(696年)五月,契丹首領李盡忠和孫萬榮率兵起義,攻陷營州,殺都督趙文翽。武則天派將軍曹仁、張玄遇、李多祚等率兵征討。由於誤吐蕃伏兵,全軍覆沒。接著,武則天再派武攸宜、王孝傑等率兵討伐,均大敗而歸。神功元年(697年)四月,武則天又派武懿宗、婁師德、沙吒忠義率兵二十萬,討伐契丹。六月,孫萬榮兵敗被殺,契丹餘眾歸降於突厥。

神功元年(697年)武則天使武懿宗審訊劉思禮謀反事,武懿宗說只要劉思禮指出哪些朝士有分謀反,就免其死罪,於是劉思禮誣告宰相李元素、孫元亨等三十六家「海內名士」,皆遭滅族,親舊連坐流竄者千餘人。時人以為武懿宗之殘暴僅次於周興、來俊臣。

是年,來俊臣欲羅告武氏諸王及太平公主(中宗之妹,武則天唯一长大成人的親生女兒),又欲誣皇嗣李旦及廬陵王李顯與南北衙共同謀反,擬一網打盡。武氏諸王與太平公主都十分害怕,共同揭發其罪行,下獄處以極刑。仇家爭食其肉,不一會就食盡。來俊臣兇狡貪暴網羅無辜,織成反狀,殺人不可勝計。「贓賄如山,冤魂塞路」,武則天亦知天下憤怨,下令數他的罪狀,並沒收其家財。

聖歷元年(698年)武承嗣、武三思謀求當太子,幾次使人對武則天說:「自古天子未有以異姓為嗣者。」武則天猶豫未決,狄仁傑對武后說:「姑侄之與母子,哪個比較親近?(武承嗣、武三思皆武后之侄,中宗、睿宗則武后之子)陛下立子,則千秋萬歲後,祭祖於太廟;立侄則未聞侄為天子祭姑於太廟者」。又勸武則天召還廬陵王(中宗)。武后由是無立武承嗣、武三思之意。乃召廬陵王還東都,皇嗣(睿宗)請遜位於廬陵王,武后立廬陵王為皇太子,命為元帥,狄仁傑為副元帥率兵擊突厥。武則天信重狄仁傑,常謂之「國老」而不呼其名。狄仁傑好諍諫,武則天每屈意從之。狄仁傑死後,武則天泣曰:「朝堂空矣!」常嘆:「天奪吾國老何太早邪!」

武则天晚年张易之、张昌宗兄弟迅速崛起,成为武则天的新宠,張易之、張昌宗兄弟年少美姿容,入侍武則天。二人常傅朱粉、穿著華麗的衣服。武承嗣、武三思等都爭著追捧他們,甚至為他們執鞭牽馬。

中宗長子邵王李重润(中宗第二次為太子時封為邵王)與其妹永泰郡主及郡主婿武延基竊議張易之兄弟「何得任意入宮」,易之投訴於武則天,武則天敕李重润、永泰郡主、武延基皆賜死。

神龍元年(705年)正月,武則天病篤,卧床不起,只有寵臣張易之、張昌宗兄弟侍側。宰相張柬之、崔玄暐與大臣敬暉、桓彥範、袁恕己等,交結禁軍統領李多祚,佯稱張易之、張昌宗兄弟謀反,於是發動兵變,率禁軍五百餘人,衝入宮中,殺死二張兄弟,隨即包圍武則天寢宮,要求武則天退位,史稱「神龍革命」。

武氏被迫禪讓帝位予兒子李顯,是為唐中宗,迁居上阳宫。中宗上尊號為「則天大聖皇帝」。

神龍元年十一月十六日(705年12月16日),武曌崩逝於洛陽上陽宮仙居殿內,享壽八十一歲。遺制去帝號,稱「則天大聖皇后」。神龍二年(706年)五月,武則天與唐高宗李治合葬於唐乾陵,留无字碑。

历代对武则天有各种不同的评價。唐代前期,由于所有的皇帝都是她的直系子孙,所以当时对武则天的评價相对比較寬容,但唐國史通過對後宮嬪妃與諸王公主的描敘淒慘地予以了無情揭露。随着时间的推移,特別是司馬光所主編之《資治通鑑》,對武氏進行嚴正批判。到了南宋期间,程朱理学在中国思想上占据了主导地位,舆论決定了对武则天的長久評價。譬如明末清初的时候,著名的思想家王夫之,就曾评价武则天“鬼神之所不容,臣民之所共怨”。惟不可否认的是,武后重視延攬,首創鉗制文網式的考試,而且知人善任,能重用狄仁傑、張柬之、桓彥範、敬暉、姚崇等中兴名臣。國家在武則天主政期間,文化承貞觀之模、百姓尚稱富庶。故享「貞觀遺風」之譽,亦及于其孫唐玄宗(其母死於武后手)的開元盛世。

武则天对历史发展做出的第一个贡献是,她打击了保守的门阀世族。武则天被立为皇后以後,把反对她做皇后的长孙无忌、褚遂良等人一个一个的都赶出了朝廷,贬逐到边远地区。这对于武则天来说,是殺雞儆猴,但这些关陇集团和他们的依附者,在当时已经成为一种既得利益的保守力量。把他们赶出政治舞台标志着关陇集团从北周以来长达一个多世纪统治的终结,也为社会进步和经济发展创造了一个良好的条件。

第二是促进了经济的发展。武则天在建言十二事中就建议“劝农桑,薄赋役”。在她掌权以后,又编撰了《兆人本业记》颁发到州县,作为州县官劝农的参考。她还注意地方吏治,加强对地主官吏的监察。对于土地兼并和逃亡的农民,也采取比较寛容的政策。因此,武则天统治时期,社会是相當安定的,农业、手工业和商业都有了長足的发展,户口也由唐高宗永徽三年(652年)的380万户增加到唐中宗神龙元年(705年)的615万户,平均每年增长0.721\%。这在中古時代,是一个很高的增长率,也是反映武则天时期经济发展的客觀數據。

第三个贡献是推动了文化的发展。唐人沈既济在谈及科举制度时说到:“太后颇涉文史,好雕虫之艺。”“太后君临天下二十余年,当时公卿百辟,无不以文章达,因循日久,浸已成风”。一是当时进士科和制科考试主要都是考策问,也就是申論。文章的好坏是录取的主要标准。二是武则天用人不看门第,不问是否為高级官吏的子孙,而是看有否政治才能。因此特别注意从科举出身者中选拔高级官吏。科举出身做到高级官吏的越来越多。这就大大刺激了仕人参加科舉的积极性,更刺激了一般人读书学习的热情。这就是沈既济所说的“浸已成风”。开元、天宝年间“父教其子,兄教其弟”,“五尺童子耻不言文墨焉”的社会风气,就是从武则天时期开始的。正是文化的普及,推动了文化的全面发展。著名的诗人和文学家崔融、李乔都是这个时期涌现出来的。雕塑、绘画也达到了前所未有的水平。

另外武則天也有不少負面評價,岑仲勉说,“武后任事率性,好恶无定,终其临朝之日,计曾任宰相七十三人”。其主政初期,由於大興告密之風,重用酷吏周興、來俊臣等,加上後世史學家不齒於她違反傳統的禮教,身為女子,竟然擁有不少男性嬪妃(稱為「男寵」)。但趙翼為武則天的私生活辯護,說:“人主富有四海,妃嬪動千百,后既為女王,而所寵幸不過數人,固亦未足深怪,故后初不以為諱,而且不必諱也。”

武則天統治的缺失主要是丟失安北領土,她將大部份的精力用於對內,因此對外軍事,屢有失策。首先在686年一度丟棄了安西四鎮,在692年才派王孝杰收復; 696年任用郭元振使反间计令吐蕃内乱,除掉吐蕃名將论钦陵,削弱吐蕃實力。另外又在696年激起孫萬榮、李盡忠的叛亂,使武周期間契丹一度落入突厥人手中。安北都護府在高宗死時尚處在中國統治,而濫殺程務挺、棄用王方翼等名將更使東突厥復國; 但在她執政後期已大致平復邊疆對外用兵的不利局面,並留下後代一個國力尚強的唐朝。然而,唐太宗和唐高宗辛苦經營的安北領土始終沒有再收復過,使唐朝曾經過千萬平方公里的江山永遠丟失了五份之二,即使是唐玄宗開元之治也沒有恢復安北任何領土,而是被回紇取代。

《旧唐书》:“治乱,时也,存亡,势也。使桀、纣在上,虽十尧不能治;使尧、舜在上,虽十桀不能乱;使懦夫女子乘时得势,亦足坐制群生之命,肆行不义之威。观夫武氏称制之年,英才接轸,靡不痛心于家索,扼腕于朝危,竟不能报先帝之恩,卫吾君之子。俄至无辜被陷,引颈就诛,天地为笼,去将安所?初虽牝鸡司晨,终能复子明辟,飞语辩元忠之罪,善言慰仁杰之心,尊时宪而抑幸臣,听忠言而诛酷吏。有旨哉,有旨哉!”赞曰:“龙漦易貌,丙殿昌储。胡为穹昊,生此夔魖?夺攘神器,秽亵皇居。穷妖白首,降鉴何如。”

《新唐书》:“昔者孔子作《春秋》而乱臣贼子惧,其于杀君篡国之主,皆不黜绝之,岂以其盗而有之者,莫大之罪也,不没其实,所以著其大恶而不隐欤?自司马迁、班固皆作《高后纪》,吕氏虽非篡汉,而盗执其国政,遂不敢没其实,岂其得圣人之意欤?抑亦偶合于《春秋》之法也。唐之旧史因之,列武后于本纪,盖其所从来远矣。夫吉凶之于人,犹影响也,而为善者得吉常多,其不幸而罹于凶者有矣;为恶者未始不及于凶,其幸而免者亦时有焉。而小人之虑,遂以为天道难知,为善未必福,而为恶未必祸也。武后之恶,不及于大戮,所谓幸免者也。至中宗韦氏,则祸不旋踵矣。然其亲遭母后之难,而躬自蹈之,所谓下愚之不移者欤!”

沈既济:“太后颇涉文史,好雕虫之艺。”“太后君临天下二十余年,当时公卿百辟,无不以文章达,因循日久,浸已成风。”

崔融:“英才远略,鸿业大勋。雷霆其武,日月其文。洒以甘露,覆之庆云。制礼作乐,还淳返朴。宗礼明堂,崇儒太学。四海慕化,九夷禀朔。沈璧大河,泥金中岳。巍乎成功,翕然向风。”

鲁宗道:“唐之罪人也,几危社稷。”

洪邁《容齋隨筆》:「漢之武帝、唐之武后,不可謂不明」。

司马光:“虽滥以禄位收天下人心, 然不称职责,寻亦黜之,或加刑诛,挟刑赏之柄以驾御天下,政由己出,明察善断,故当时英贤亦竞为之。”

赵翼:“女中英主。”“人主富有四海,妃嫔动千百,后既为女王,而所宠幸不过数人,固亦未足深怪,故后初不以为讳,而且不必讳也。”

翟蔼:“武氏以一妇人君临天下二十余年,是不比於母后之称制者,而直自帝自王也,此其智有过人者。”

岑仲勉:“武后任事率性,好恶无定,终其临朝之日,计曾任宰相七十三人。”

郭沫若:“政启开元,治宏贞观;芳流剑阁,光被利州。”

宋庆龄:“武则天是封建时代杰出的女政治家。但就家庭角色而言,不难看出武则天也是个好妻子。”

毛泽东:“武则天确实是个治国之才,她既有容人之量,又有识人之智,还有用人之术。她提拔过不少人,也杀了不少人。刚刚提拔又杀了的也不少。”

翦伯赞:“武则天的打击门阀贵族和提拔普通地主做官的政策,是符合当时社会发展趋势的,因此她的作用是积极的……武则天在巩固封建国家的边疆方面,也做了不少工作。”

江青對武則天的評價很高,認爲武則天是中國婦女中最傑出的人物。

相传唐太宗在世時,曾請天文師算命,天文師認為,不出三十年,李氏皆亡於一個姓武的人手裡了。於是太宗屠殺武氏朝臣,沒想到所算者竟是他身邊的武才人。

相傳在感業寺時期,李治有次前往祭拜,看到武氏後便魂不守舍。當時,與蕭淑妃爭寵的王皇后,便趁此納武氏為自己派系,跟蕭淑妃對抗,沒想到兩人皆亡於武氏之手。

武則天稱帝後,亦有多名男寵,此段為妄說。其中最出名者乃馮小寶(薛怀义),武則天後派他在洛陽東的白馬寺出家,法名懷義,但仍與武則天私通。某年盂蘭盆節,當時已經逐漸失寵的懷義,為討武則天注意,火烧明堂,火勢蔓延整個洛陽。

武则天为了誇飾武周革命,创造了则天文字。部分的則天文字還傳到日本、韓國,甚至成為某些日本禮遇中國文化人的人名用字。

武则天著有《垂拱集》百卷,《金轮集》十卷,已散佚。今存诗四十六首,《全唐文》编其文为四卷。有《石榴裙》之思。

武則天稱帝前掌握實權的6年,使用了3個年號,稱帝的15年使用了16個年號,合19個年號,是中國皇帝中用年號最多和密度最高的皇帝。居於第二的是西晉皇帝晉惠帝司馬衷,除了時間最長的元康年號(9年),9年間用了8個年號。

佛經「開經偈」的撰寫者。"無上甚深微妙法,百千萬劫難遭遇,我今見聞得受持,願解如來真實義"。武則天是真正的大修行人,做此偈時感得天女散花,流傳迄今。很多評論均是不符合事實的妄說,因不符合統治者利益和倫理習俗之故。但武則天問心無愧,立無字碑任憑人解讀。

\subsection{天授}

\begin{longtable}{|>{\centering\scriptsize}m{2em}|>{\centering\scriptsize}m{1.3em}|>{\centering}m{8.8em}|}
  % \caption{秦王政}\
  \toprule
  \SimHei \normalsize 年数 & \SimHei \scriptsize 公元 & \SimHei 大事件 \tabularnewline
  % \midrule
  \endfirsthead
  \toprule
  \SimHei \normalsize 年数 & \SimHei \scriptsize 公元 & \SimHei 大事件 \tabularnewline
  \midrule
  \endhead
  \midrule
  元年 & 690 & \tabularnewline\hline
  二年 & 691 & \tabularnewline\hline
  三年 & 692 & \tabularnewline
  \bottomrule
\end{longtable}

\subsection{如意}

\begin{longtable}{|>{\centering\scriptsize}m{2em}|>{\centering\scriptsize}m{1.3em}|>{\centering}m{8.8em}|}
  % \caption{秦王政}\
  \toprule
  \SimHei \normalsize 年数 & \SimHei \scriptsize 公元 & \SimHei 大事件 \tabularnewline
  % \midrule
  \endfirsthead
  \toprule
  \SimHei \normalsize 年数 & \SimHei \scriptsize 公元 & \SimHei 大事件 \tabularnewline
  \midrule
  \endhead
  \midrule
  元年 & 692 & \tabularnewline
  \bottomrule
\end{longtable}

\subsection{长寿}

\begin{longtable}{|>{\centering\scriptsize}m{2em}|>{\centering\scriptsize}m{1.3em}|>{\centering}m{8.8em}|}
  % \caption{秦王政}\
  \toprule
  \SimHei \normalsize 年数 & \SimHei \scriptsize 公元 & \SimHei 大事件 \tabularnewline
  % \midrule
  \endfirsthead
  \toprule
  \SimHei \normalsize 年数 & \SimHei \scriptsize 公元 & \SimHei 大事件 \tabularnewline
  \midrule
  \endhead
  \midrule
  元年 & 692 & \tabularnewline\hline
  二年 & 693 & \tabularnewline\hline
  三年 & 694 & \tabularnewline
  \bottomrule
\end{longtable}

\subsection{延载}

\begin{longtable}{|>{\centering\scriptsize}m{2em}|>{\centering\scriptsize}m{1.3em}|>{\centering}m{8.8em}|}
  % \caption{秦王政}\
  \toprule
  \SimHei \normalsize 年数 & \SimHei \scriptsize 公元 & \SimHei 大事件 \tabularnewline
  % \midrule
  \endfirsthead
  \toprule
  \SimHei \normalsize 年数 & \SimHei \scriptsize 公元 & \SimHei 大事件 \tabularnewline
  \midrule
  \endhead
  \midrule
  元年 & 694 & \tabularnewline
  \bottomrule
\end{longtable}

\subsection{证圣}

\begin{longtable}{|>{\centering\scriptsize}m{2em}|>{\centering\scriptsize}m{1.3em}|>{\centering}m{8.8em}|}
  % \caption{秦王政}\
  \toprule
  \SimHei \normalsize 年数 & \SimHei \scriptsize 公元 & \SimHei 大事件 \tabularnewline
  % \midrule
  \endfirsthead
  \toprule
  \SimHei \normalsize 年数 & \SimHei \scriptsize 公元 & \SimHei 大事件 \tabularnewline
  \midrule
  \endhead
  \midrule
  元年 & 695 & \tabularnewline
  \bottomrule
\end{longtable}

\subsection{天册万岁}

\begin{longtable}{|>{\centering\scriptsize}m{2em}|>{\centering\scriptsize}m{1.3em}|>{\centering}m{8.8em}|}
  % \caption{秦王政}\
  \toprule
  \SimHei \normalsize 年数 & \SimHei \scriptsize 公元 & \SimHei 大事件 \tabularnewline
  % \midrule
  \endfirsthead
  \toprule
  \SimHei \normalsize 年数 & \SimHei \scriptsize 公元 & \SimHei 大事件 \tabularnewline
  \midrule
  \endhead
  \midrule
  元年 & 695 & \tabularnewline
  \bottomrule
\end{longtable}

\subsection{万岁登封}

\begin{longtable}{|>{\centering\scriptsize}m{2em}|>{\centering\scriptsize}m{1.3em}|>{\centering}m{8.8em}|}
  % \caption{秦王政}\
  \toprule
  \SimHei \normalsize 年数 & \SimHei \scriptsize 公元 & \SimHei 大事件 \tabularnewline
  % \midrule
  \endfirsthead
  \toprule
  \SimHei \normalsize 年数 & \SimHei \scriptsize 公元 & \SimHei 大事件 \tabularnewline
  \midrule
  \endhead
  \midrule
  元年 & 695 & \tabularnewline\hline
  二年 & 696 & \tabularnewline
  \bottomrule
\end{longtable}

\subsection{万岁通天}

\begin{longtable}{|>{\centering\scriptsize}m{2em}|>{\centering\scriptsize}m{1.3em}|>{\centering}m{8.8em}|}
  % \caption{秦王政}\
  \toprule
  \SimHei \normalsize 年数 & \SimHei \scriptsize 公元 & \SimHei 大事件 \tabularnewline
  % \midrule
  \endfirsthead
  \toprule
  \SimHei \normalsize 年数 & \SimHei \scriptsize 公元 & \SimHei 大事件 \tabularnewline
  \midrule
  \endhead
  \midrule
  元年 & 696 & \tabularnewline\hline
  二年 & 697 & \tabularnewline
  \bottomrule
\end{longtable}

\subsection{神功}

\begin{longtable}{|>{\centering\scriptsize}m{2em}|>{\centering\scriptsize}m{1.3em}|>{\centering}m{8.8em}|}
  % \caption{秦王政}\
  \toprule
  \SimHei \normalsize 年数 & \SimHei \scriptsize 公元 & \SimHei 大事件 \tabularnewline
  % \midrule
  \endfirsthead
  \toprule
  \SimHei \normalsize 年数 & \SimHei \scriptsize 公元 & \SimHei 大事件 \tabularnewline
  \midrule
  \endhead
  \midrule
  元年 & 697 & \tabularnewline
  \bottomrule
\end{longtable}

\subsection{圣历}

\begin{longtable}{|>{\centering\scriptsize}m{2em}|>{\centering\scriptsize}m{1.3em}|>{\centering}m{8.8em}|}
  % \caption{秦王政}\
  \toprule
  \SimHei \normalsize 年数 & \SimHei \scriptsize 公元 & \SimHei 大事件 \tabularnewline
  % \midrule
  \endfirsthead
  \toprule
  \SimHei \normalsize 年数 & \SimHei \scriptsize 公元 & \SimHei 大事件 \tabularnewline
  \midrule
  \endhead
  \midrule
  元年 & 698 & \tabularnewline\hline
  二年 & 699 & \tabularnewline\hline
  三年 & 700 & \tabularnewline
  \bottomrule
\end{longtable}

\subsection{久视}

\begin{longtable}{|>{\centering\scriptsize}m{2em}|>{\centering\scriptsize}m{1.3em}|>{\centering}m{8.8em}|}
  % \caption{秦王政}\
  \toprule
  \SimHei \normalsize 年数 & \SimHei \scriptsize 公元 & \SimHei 大事件 \tabularnewline
  % \midrule
  \endfirsthead
  \toprule
  \SimHei \normalsize 年数 & \SimHei \scriptsize 公元 & \SimHei 大事件 \tabularnewline
  \midrule
  \endhead
  \midrule
  元年 & 700 & \tabularnewline\hline
  二年 & 701 & \tabularnewline
  \bottomrule
\end{longtable}

\subsection{大足}

\begin{longtable}{|>{\centering\scriptsize}m{2em}|>{\centering\scriptsize}m{1.3em}|>{\centering}m{8.8em}|}
  % \caption{秦王政}\
  \toprule
  \SimHei \normalsize 年数 & \SimHei \scriptsize 公元 & \SimHei 大事件 \tabularnewline
  % \midrule
  \endfirsthead
  \toprule
  \SimHei \normalsize 年数 & \SimHei \scriptsize 公元 & \SimHei 大事件 \tabularnewline
  \midrule
  \endhead
  \midrule
  元年 & 701 & \tabularnewline
  \bottomrule
\end{longtable}

\subsection{长安}

\begin{longtable}{|>{\centering\scriptsize}m{2em}|>{\centering\scriptsize}m{1.3em}|>{\centering}m{8.8em}|}
  % \caption{秦王政}\
  \toprule
  \SimHei \normalsize 年数 & \SimHei \scriptsize 公元 & \SimHei 大事件 \tabularnewline
  % \midrule
  \endfirsthead
  \toprule
  \SimHei \normalsize 年数 & \SimHei \scriptsize 公元 & \SimHei 大事件 \tabularnewline
  \midrule
  \endhead
  \midrule
  元年 & 701 & \tabularnewline\hline
  二年 & 702 & \tabularnewline\hline
  三年 & 703 & \tabularnewline\hline
  四年 & 704 & \tabularnewline
  \bottomrule
\end{longtable}

\subsection{神龙}

\begin{longtable}{|>{\centering\scriptsize}m{2em}|>{\centering\scriptsize}m{1.3em}|>{\centering}m{8.8em}|}
  % \caption{秦王政}\
  \toprule
  \SimHei \normalsize 年数 & \SimHei \scriptsize 公元 & \SimHei 大事件 \tabularnewline
  % \midrule
  \endfirsthead
  \toprule
  \SimHei \normalsize 年数 & \SimHei \scriptsize 公元 & \SimHei 大事件 \tabularnewline
  \midrule
  \endhead
  \midrule
  元年 & 705 & \tabularnewline\hline
  二年 & 706 & \tabularnewline\hline
  三年 & 707 & \tabularnewline
  \bottomrule
\end{longtable}


%%% Local Variables:
%%% mode: latex
%%% TeX-engine: xetex
%%% TeX-master: "../Main"
%%% End:
