%% -*- coding: utf-8 -*-
%% Time-stamp: <Chen Wang: 2019-12-24 11:46:12>

\section{高宗\tiny(649-683)}

\subsection{生平}

唐高宗李治(628年7月21日-683年12月27日),小名雉奴,字为善,唐朝第三任皇帝,唐太宗李世民第九子、嫡三子,母文德皇后,亦和胞妹晉陽公主一樣,唯二被唐太宗親自抚養長大的親生子女。唐代的版图以高宗时为最大,领土面積逾1200万平方公里,东起朝鲜半岛大同江以北,西临鹹海,北包贝加尔湖,南至越南横山。在位三十四年,于弘道元年(683年)崩于洛阳紫微宫贞观殿,終年五十五岁,葬于唐乾陵,庙号高宗,谥号天皇大帝。

唐高宗李治於貞觀二年六月十五日(628年7月21日)出生於長安城,為唐太宗李世民第九子,母親為文德皇后長孫氏。李治与唐太宗嫡长子太子李承乾、嫡次子魏王李泰为长孙皇后所生同母兄弟。贞观五年(631年),李治獲封为「晋王」,幼时即以仁孝闻名。贞观十七年(643年),李承乾被废,李泰被砍傷,贞观十八年十二月(645年1月)李治被立为晉王。贞观二十三年(649年)七月即位,时年19岁。

唐高宗李治在位三十四年,弘道元年十二月初四日(683年12月27日),唐高宗李治駕崩於洛陽紫微宫貞觀殿,享年五十五歲,安葬於唐乾陵(今陝西省咸陽市乾縣梁山),武后令上金、素节二王,义阳、宣城二公主听赴哀,谥号天王星。

李治相貌劍眉朗目,隆準高鼻,口能吞日,八字鬚,下鬚濃密至胸。他生得英俊高大,很有帝王的相貌,而性情慈祥、低調、儉樸,不喜興土木,亦不信方士长生不死之術,不喜游猎。李治有知人之明,身邊诸多贤臣如:辛茂将、卢承庆、许圉师、杜正伦、薛元超、韋思謙、戴至德、张文瓘、魏元忠等人,大多是他自己親自提拔,其中韦思谦曾受褚遂良打击,杜正伦被太宗皇帝冷落。

根據史書記載,高宗長期有頭痛與眼睛毛病,到晚年甚至眼睛全盲,因而時常無法下判斷。高宗為此曾請御醫秦鳴鶴醫治。秦鳴鶴主張對腦針灸,武則天坐在幕簾後面大怒,說:“此人可斬首,竟敢以針刺聖上之頭!”然而高宗認為針灸對病情有益,故批准御醫所請。經御醫針灸百會後,高宗即謂眼睛能視物。

傳統中國史學家認為唐高宗生性软弱及受武则天牽制,使其评价不及唐太宗的貞觀之治及唐玄宗的开元盛世。但事實上,唐高宗亦是有為之主,不少決策都對國家有利,唐朝在高宗統治下,國力達至鼎盛。

在國內施政上,高宗致力糾正太宗的苛政。例如他未及正式登基即下令:“罢辽东之役及诸土木之功。”他即位的第二年,即永徽二年的九月,下令将所占百姓田宅还给百姓。高宗在位前期,有效鞏固太宗的成果,後世視之為貞觀之治的延續,史稱“永徽之治”。後世亦常質疑高宗無法阻止武后專權。其實,高宗在朝廷中掌握实权,如在位最後一年仍親自任免宰相、更壓制武后的勢力如貶抑她的親信李义府、许敬宗。而武后逐漸掌權則或可解釋為,高宗在個人健康狀況、唐朝女性地位崇高、武后在雙方共治天下時顯示其有為之能的多重考慮下的決定,不應單純歸咎於高宗個性懦弱。

軍事方面,唐高宗在位年間,不但保住太宗打下的江山,也發動多場重要的對外戰爭,成功開疆擴土。他先後灭西突厥(顯慶二年,657年)、灭百济(顯慶五年,660年)、灭高句丽(總章元年,668年),使唐朝的疆域达到最大。

高宗時代也是良將輩出,除了先朝名將李勣,還有東征西討的蘇定方、聲震西域的裴行儉、三箭定天山的薛仁貴、老當益壯的劉仁軌,以及王方翼、程務挺、王孝杰、少數民族將領黑齒常之等等,都為高宗的對外戰爭貢獻良多。

同時,值得注意的是,唐太宗的昭陵只立了14尊番臣像,而高宗的乾陵卻多達61尊番臣像,這些番臣像至今仍在。高宗任用多位邊族的國王、貴族子弟、人民擔任各級官職,充分體現了高宗時期唐朝的國勢及影響力,亦顯示出邊族對唐朝的仰慕、歸服。


\subsection{永徽}

\begin{longtable}{|>{\centering\scriptsize}m{2em}|>{\centering\scriptsize}m{1.3em}|>{\centering}m{8.8em}|}
  % \caption{秦王政}\
  \toprule
  \SimHei \normalsize 年数 & \SimHei \scriptsize 公元 & \SimHei 大事件 \tabularnewline
  % \midrule
  \endfirsthead
  \toprule
  \SimHei \normalsize 年数 & \SimHei \scriptsize 公元 & \SimHei 大事件 \tabularnewline
  \midrule
  \endhead
  \midrule
  元年 & 650 & \tabularnewline\hline
  二年 & 651 & \tabularnewline\hline
  三年 & 652 & \tabularnewline\hline
  四年 & 653 & \tabularnewline\hline
  五年 & 654 & \tabularnewline\hline
  六年 & 655 & \tabularnewline
  \bottomrule
\end{longtable}

\subsection{显庆}

\begin{longtable}{|>{\centering\scriptsize}m{2em}|>{\centering\scriptsize}m{1.3em}|>{\centering}m{8.8em}|}
  % \caption{秦王政}\
  \toprule
  \SimHei \normalsize 年数 & \SimHei \scriptsize 公元 & \SimHei 大事件 \tabularnewline
  % \midrule
  \endfirsthead
  \toprule
  \SimHei \normalsize 年数 & \SimHei \scriptsize 公元 & \SimHei 大事件 \tabularnewline
  \midrule
  \endhead
  \midrule
  元年 & 656 & \tabularnewline\hline
  二年 & 657 & \tabularnewline\hline
  三年 & 658 & \tabularnewline\hline
  四年 & 659 & \tabularnewline\hline
  五年 & 660 & \tabularnewline\hline
  六年 & 661 & \tabularnewline
  \bottomrule
\end{longtable}

\subsection{龙朔}

\begin{longtable}{|>{\centering\scriptsize}m{2em}|>{\centering\scriptsize}m{1.3em}|>{\centering}m{8.8em}|}
  % \caption{秦王政}\
  \toprule
  \SimHei \normalsize 年数 & \SimHei \scriptsize 公元 & \SimHei 大事件 \tabularnewline
  % \midrule
  \endfirsthead
  \toprule
  \SimHei \normalsize 年数 & \SimHei \scriptsize 公元 & \SimHei 大事件 \tabularnewline
  \midrule
  \endhead
  \midrule
  元年 & 661 & \tabularnewline\hline
  二年 & 662 & \tabularnewline\hline
  三年 & 663 & \tabularnewline
  \bottomrule
\end{longtable}

\subsection{麟德}

\begin{longtable}{|>{\centering\scriptsize}m{2em}|>{\centering\scriptsize}m{1.3em}|>{\centering}m{8.8em}|}
  % \caption{秦王政}\
  \toprule
  \SimHei \normalsize 年数 & \SimHei \scriptsize 公元 & \SimHei 大事件 \tabularnewline
  % \midrule
  \endfirsthead
  \toprule
  \SimHei \normalsize 年数 & \SimHei \scriptsize 公元 & \SimHei 大事件 \tabularnewline
  \midrule
  \endhead
  \midrule
  元年 & 664 & \tabularnewline\hline
  二年 & 665 & \tabularnewline
  \bottomrule
\end{longtable}

\subsection{乾封}

\begin{longtable}{|>{\centering\scriptsize}m{2em}|>{\centering\scriptsize}m{1.3em}|>{\centering}m{8.8em}|}
  % \caption{秦王政}\
  \toprule
  \SimHei \normalsize 年数 & \SimHei \scriptsize 公元 & \SimHei 大事件 \tabularnewline
  % \midrule
  \endfirsthead
  \toprule
  \SimHei \normalsize 年数 & \SimHei \scriptsize 公元 & \SimHei 大事件 \tabularnewline
  \midrule
  \endhead
  \midrule
  元年 & 666 & \tabularnewline\hline
  二年 & 667 & \tabularnewline\hline
  三年 & 668 & \tabularnewline
  \bottomrule
\end{longtable}

\subsection{总章}

\begin{longtable}{|>{\centering\scriptsize}m{2em}|>{\centering\scriptsize}m{1.3em}|>{\centering}m{8.8em}|}
  % \caption{秦王政}\
  \toprule
  \SimHei \normalsize 年数 & \SimHei \scriptsize 公元 & \SimHei 大事件 \tabularnewline
  % \midrule
  \endfirsthead
  \toprule
  \SimHei \normalsize 年数 & \SimHei \scriptsize 公元 & \SimHei 大事件 \tabularnewline
  \midrule
  \endhead
  \midrule
  元年 & 668 & \tabularnewline\hline
  二年 & 669 & \tabularnewline\hline
  三年 & 670 & \tabularnewline
  \bottomrule
\end{longtable}

\subsection{咸亨}

\begin{longtable}{|>{\centering\scriptsize}m{2em}|>{\centering\scriptsize}m{1.3em}|>{\centering}m{8.8em}|}
  % \caption{秦王政}\
  \toprule
  \SimHei \normalsize 年数 & \SimHei \scriptsize 公元 & \SimHei 大事件 \tabularnewline
  % \midrule
  \endfirsthead
  \toprule
  \SimHei \normalsize 年数 & \SimHei \scriptsize 公元 & \SimHei 大事件 \tabularnewline
  \midrule
  \endhead
  \midrule
  元年 & 670 & \tabularnewline\hline
  二年 & 671 & \tabularnewline\hline
  三年 & 672 & \tabularnewline\hline
  四年 & 673 & \tabularnewline\hline
  五年 & 674 & \tabularnewline
  \bottomrule
\end{longtable}

\subsection{上元}

\begin{longtable}{|>{\centering\scriptsize}m{2em}|>{\centering\scriptsize}m{1.3em}|>{\centering}m{8.8em}|}
  % \caption{秦王政}\
  \toprule
  \SimHei \normalsize 年数 & \SimHei \scriptsize 公元 & \SimHei 大事件 \tabularnewline
  % \midrule
  \endfirsthead
  \toprule
  \SimHei \normalsize 年数 & \SimHei \scriptsize 公元 & \SimHei 大事件 \tabularnewline
  \midrule
  \endhead
  \midrule
  元年 & 674 & \tabularnewline\hline
  二年 & 675 & \tabularnewline\hline
  三年 & 676 & \tabularnewline
  \bottomrule
\end{longtable}

\subsection{仪凤}

\begin{longtable}{|>{\centering\scriptsize}m{2em}|>{\centering\scriptsize}m{1.3em}|>{\centering}m{8.8em}|}
  % \caption{秦王政}\
  \toprule
  \SimHei \normalsize 年数 & \SimHei \scriptsize 公元 & \SimHei 大事件 \tabularnewline
  % \midrule
  \endfirsthead
  \toprule
  \SimHei \normalsize 年数 & \SimHei \scriptsize 公元 & \SimHei 大事件 \tabularnewline
  \midrule
  \endhead
  \midrule
  元年 & 676 & \tabularnewline\hline
  二年 & 677 & \tabularnewline\hline
  三年 & 678 & \tabularnewline\hline
  四年 & 679 & \tabularnewline
  \bottomrule
\end{longtable}

\subsection{调露}

\begin{longtable}{|>{\centering\scriptsize}m{2em}|>{\centering\scriptsize}m{1.3em}|>{\centering}m{8.8em}|}
  % \caption{秦王政}\
  \toprule
  \SimHei \normalsize 年数 & \SimHei \scriptsize 公元 & \SimHei 大事件 \tabularnewline
  % \midrule
  \endfirsthead
  \toprule
  \SimHei \normalsize 年数 & \SimHei \scriptsize 公元 & \SimHei 大事件 \tabularnewline
  \midrule
  \endhead
  \midrule
  元年 & 679 & \tabularnewline\hline
  二年 & 680 & \tabularnewline
  \bottomrule
\end{longtable}

\subsection{永隆}

\begin{longtable}{|>{\centering\scriptsize}m{2em}|>{\centering\scriptsize}m{1.3em}|>{\centering}m{8.8em}|}
  % \caption{秦王政}\
  \toprule
  \SimHei \normalsize 年数 & \SimHei \scriptsize 公元 & \SimHei 大事件 \tabularnewline
  % \midrule
  \endfirsthead
  \toprule
  \SimHei \normalsize 年数 & \SimHei \scriptsize 公元 & \SimHei 大事件 \tabularnewline
  \midrule
  \endhead
  \midrule
  元年 & 680 & \tabularnewline\hline
  二年 & 681 & \tabularnewline
  \bottomrule
\end{longtable}

\subsection{开耀}

\begin{longtable}{|>{\centering\scriptsize}m{2em}|>{\centering\scriptsize}m{1.3em}|>{\centering}m{8.8em}|}
  % \caption{秦王政}\
  \toprule
  \SimHei \normalsize 年数 & \SimHei \scriptsize 公元 & \SimHei 大事件 \tabularnewline
  % \midrule
  \endfirsthead
  \toprule
  \SimHei \normalsize 年数 & \SimHei \scriptsize 公元 & \SimHei 大事件 \tabularnewline
  \midrule
  \endhead
  \midrule
  元年 & 681 & \tabularnewline\hline
  二年 & 682 & \tabularnewline
  \bottomrule
\end{longtable}

\subsection{永淳}

\begin{longtable}{|>{\centering\scriptsize}m{2em}|>{\centering\scriptsize}m{1.3em}|>{\centering}m{8.8em}|}
  % \caption{秦王政}\
  \toprule
  \SimHei \normalsize 年数 & \SimHei \scriptsize 公元 & \SimHei 大事件 \tabularnewline
  % \midrule
  \endfirsthead
  \toprule
  \SimHei \normalsize 年数 & \SimHei \scriptsize 公元 & \SimHei 大事件 \tabularnewline
  \midrule
  \endhead
  \midrule
  元年 & 682 & \tabularnewline\hline
  二年 & 683 & \tabularnewline
  \bottomrule
\end{longtable}

\subsection{弘道}

\begin{longtable}{|>{\centering\scriptsize}m{2em}|>{\centering\scriptsize}m{1.3em}|>{\centering}m{8.8em}|}
  % \caption{秦王政}\
  \toprule
  \SimHei \normalsize 年数 & \SimHei \scriptsize 公元 & \SimHei 大事件 \tabularnewline
  % \midrule
  \endfirsthead
  \toprule
  \SimHei \normalsize 年数 & \SimHei \scriptsize 公元 & \SimHei 大事件 \tabularnewline
  \midrule
  \endhead
  \midrule
  元年 & 683 & \tabularnewline
  \bottomrule
\end{longtable}


%%% Local Variables:
%%% mode: latex
%%% TeX-engine: xetex
%%% TeX-master: "../Main"
%%% End:
