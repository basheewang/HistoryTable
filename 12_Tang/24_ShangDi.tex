%% -*- coding: utf-8 -*-
%% Time-stamp: <Chen Wang: 2021-10-29 17:15:43>

\section{殇帝李重茂\tiny(710)}

\subsection{生平}

唐殇帝李重茂(695年-714年),唐高宗和武则天孙子,唐中宗幼子,安乐公主同父异母弟弟,唐睿宗和太平公主侄子,唐玄宗堂弟。唐朝第七位皇帝,景龙四年(710年)六月在位,开元二年(714年)逝世,谥号殇皇帝。

李重茂生于武后延载元年(695年),圣历三年(700年)封为北海郡王,中宗神龙元年(705年)进封温王,授右卫大将军,兼遥领赠州大都督,没有到任。

景龙四年(710年)中宗病逝后,六月初四韦后临朝,改元唐隆。六月初七韦后矫诏立时年仅16岁的李重茂为帝,韦后临朝称制。李重茂即位不足一个月,六月二十日夜,睿宗三子临淄王李隆基、中宗妹妹太平公主等交结禁军将领,发兵入宫,将韦后与安乐公主等人杀死,是为唐隆之变。

六月二十二日,宫人、宦官请求中书舍人刘幽求草诏立太后,刘幽求意图复辟睿宗,拒绝了。六月二十四日,太平公主称李重茂有意让位睿宗,更亲自将其提下御座,睿宗复辟。李重茂仍封温王。

李重茂的庶兄谯王李重福随即发动政变,与其党羽郑愔等策划矫诏称得中宗传位,以李重茂为皇太弟,但很快被镇压。

睿宗景云二年改封襄王,李重茂离开长安,被迁到集州,睿宗令中郎将率武士五百人守备。

玄宗开元二年(714年),李重茂除房州刺史,不久死于房州。葬于武功西原,年仅19岁。

\subsection{唐隆}

\begin{longtable}{|>{\centering\scriptsize}m{2em}|>{\centering\scriptsize}m{1.3em}|>{\centering}m{8.8em}|}
  % \caption{秦王政}\
  \toprule
  \SimHei \normalsize 年数 & \SimHei \scriptsize 公元 & \SimHei 大事件 \tabularnewline
  % \midrule
  \endfirsthead
  \toprule
  \SimHei \normalsize 年数 & \SimHei \scriptsize 公元 & \SimHei 大事件 \tabularnewline
  \midrule
  \endhead
  \midrule
  元年 & 710 & \tabularnewline
  \bottomrule
\end{longtable}


%%% Local Variables:
%%% mode: latex
%%% TeX-engine: xetex
%%% TeX-master: "../Main"
%%% End:
