%% -*- coding: utf-8 -*-
%% Time-stamp: <Chen Wang: 2019-12-24 15:36:50>

\section{宣宗\tiny(846-859)}

\subsection{生平}

唐宣宗李忱(810年-859年),唐朝第19代皇帝(846年—859年在位,未算武周政权),初名怡,登基之前封為光王,在位13年。唐宪宗李纯十三子,母鄭宮人,元和五年(810年)六月廿二日生於大明宫,是唐穆宗李恒的弟弟,唐敬宗李湛、唐文宗李昂、唐武宗李炎的叔叔。

唐宣宗李忱原名李怡,母親郑氏原为镇海节度使李錡侍妾,李锜謀反失敗後,郑氏被送入宫后当郭貴妃的侍儿,后来被唐憲宗临幸,生下李忱,封為光王,故唐宣宗為唐穆宗之弟,唐敬宗、唐文宗、唐武宗之叔父。唐宣宗為光王時,居於十六宅,故作愚鈍,曾被唐文宗及其他宗室作弄。

傳說唐宣宗登基之前,为了逃避姪唐武宗的迫害而出家為僧,传说他在河南淅川香严寺避难,法名琼俊,齐安见其举止不凡。

会昌六年(846年),唐武宗被道士上供的长寿丹毒死。光王李怡被宦官擁立為帝,改名為李忱,是为唐宣宗,年号大中。擁立唐宣宗的宦官本以為他愚鈍容易控制,豈料他登基為帝後立即勵精圖治,並贬谪李德裕,结束牛李党爭。眾宦官、朝臣及宗室才驚覺唐宣宗以往是故作愚鈍,實際是非常賢明,有如其父唐憲宗。

唐宣宗登基后,唐朝国势已暮氣沉沉,藩鎮割據,牛李党争,农民起义,朝政腐败,官吏贪污,宦官专权,四夷不朝。唐宣宗致力于改变这种状况,宣宗勤俭治国,体贴百姓,减少赋税,注重人才选拔,唐朝国势有所起色,社会矛盾有所缓和,百姓日渐富裕,使十分腐败的唐朝呈现出“中兴”的小康局面,史稱大中之治。但只靠政府支支節節的改革,未能完全解決問題。宣宗是唐朝中期以後少數比较有作为的皇帝,另外,唐宣宗还趁吐蕃、回纥衰微,加上張議潮的歸義軍起事反吐蕃; 不僅出兵收复河湟,安定塞北,更一度大致重奪丟失多年的河西走廊(沙、瓜等河西十一州),國威稍振; 國內百姓亦有申冤之地,又宽减刑法,对百姓加以安抚。又抑制宦官,革除弊制,唐朝出现中兴的局面。

大中十三年(859年)八月,宣宗因丹藥中毒驾崩,时年49岁。

唐宣宗在位时,有一越州美女天姿國色。唐宣宗初見之,宠爱异常。不久,唐宣宗怕自己耽誤國事,居然把她賜死。

唐末的黃巢之亂和藩镇战争使宣宗朝的實錄散失,使後人難以追查當年發生過的事。

传说宣宗继位之前曾在淅川香严寺当过和尚,所以对佛教极力推崇,据说曾在大中七年(853年)大拜释迦牟尼的舍利,關於這些資料見諸韋昭度《讀皇室運尋》、令狐绹《偵陵遺事》、讚寧《宋高僧傳》及僧圓悟禪師《碧岩集》。

唐末西川變民韓秀升在被高仁厚征服後,就曾坦言唐宣宗在位時天下尚有公道,唐宣宗故去之後就是勝者才有公道;高仁厚聞言後為之側目。可惜唐朝當時已是病入膏肓之軀,再沒有人能有力回天。

歷史上評價宣宗在位曾經燒過三把火:“權豪斂跡”、“奸臣畏法”、“閽寺詟氣”,稱之為“明君”,有“小太宗”的外號。據說唐宣宗退朝后還會讀書到半夜,燭灺委積,近侍呼之為“老儒生”。

明末清初的大儒王夫之在《讀通鑑論》論“唐之亡,宣宗亡之”,“小說載宣宗之政,琅琅乎其言之,皆治像也,溫公亟取之登之於策,若有余美焉。自知治者觀之,則皆亡國之符也。”評價與《舊唐書》和《資治通鑑》的大力稱頌,實有天壤之別。

\subsection{大中}

\begin{longtable}{|>{\centering\scriptsize}m{2em}|>{\centering\scriptsize}m{1.3em}|>{\centering}m{8.8em}|}
  % \caption{秦王政}\
  \toprule
  \SimHei \normalsize 年数 & \SimHei \scriptsize 公元 & \SimHei 大事件 \tabularnewline
  % \midrule
  \endfirsthead
  \toprule
  \SimHei \normalsize 年数 & \SimHei \scriptsize 公元 & \SimHei 大事件 \tabularnewline
  \midrule
  \endhead
  \midrule
  元年 & 847 & \tabularnewline\hline
  二年 & 848 & \tabularnewline\hline
  三年 & 849 & \tabularnewline\hline
  四年 & 850 & \tabularnewline\hline
  五年 & 851 & \tabularnewline\hline
  六年 & 852 & \tabularnewline\hline
  七年 & 853 & \tabularnewline\hline
  八年 & 854 & \tabularnewline\hline
  九年 & 855 & \tabularnewline\hline
  十年 & 856 & \tabularnewline\hline
  十一年 & 857 & \tabularnewline\hline
  十二年 & 858 & \tabularnewline\hline
  十三年 & 859 & \tabularnewline\hline
  十四年 & 860 & \tabularnewline
  \bottomrule
\end{longtable}


%%% Local Variables:
%%% mode: latex
%%% TeX-engine: xetex
%%% TeX-master: "../Main"
%%% End:
