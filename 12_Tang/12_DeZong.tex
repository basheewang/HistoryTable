%% -*- coding: utf-8 -*-
%% Time-stamp: <Chen Wang: 2019-12-24 15:25:46>

\section{德宗\tiny(779-805)}

\subsection{生平}

唐德宗李适(音同括,742年5月27日-805年2月25日),唐代宗與睿真皇后所生的长子,唐朝第12代皇帝(除去武则天以外),779年6月12日―805年2月25日在位,在位26年,享壽62岁。谥号为神武孝文皇帝。

李适於唐玄宗天宝元年四月十九日(742年5月27日),生于长安大内宫中。当时,父亲唐代宗为广平郡王,母亲沈氏是他的妾室,出生后八个月便被封为奉节郡王。嫡母崔氏的母族在安史之乱中失势,自己的母亲沈氏亦在期间失踪。宝应元年(762年),父亲唐代宗继位。广德二年(764年)被立为皇太子。在一般认知中,唐德宗是以庶长子的身份被立为太子的,但实际上在《让皇太子表》中,唐德宗自称为唐代宗的嫡长子,在让表中曾向代宗表示不应该只因为嫡长身份就立自己为皇太子,唐代宗应该效法三皇五帝选择贤德之人。

大历十四年(779年)37歲即位。次年,为了改善财政,采纳宰相楊炎建议,废除庸调制,颁行“两税法”。

执政前期一改代宗朝姑息藩镇的弊政,坚决削弱藩镇割据,加强中央集权,但由于措置失当,镇抚乖方,往往在消灭旧叛之后又激起一批原本忠于朝廷的节度使的叛乱。

建中四年(783年),因朝廷赏赐有失公平,激起泾原兵变,德宗仓皇出逃到奉天(今陕西乾县),早年入朝面圣却被软禁在京城的前卢龙节度使朱泚在哗变军士的拥护下称帝,改国号为秦。泾原之变时,长期讨逆伐叛的朔方节度使李懷光因奸臣卢杞畏罪挑拨而见疑于德宗,自危之下铤而走险加入了叛乱阵营。唐中兴名将李晟经历艰难险阻,终于击破朱泚、李怀光联军,德宗才得以重返长安。 德宗前期,坚持信用文武百官,严禁宦官干政,颇有一番中兴气象;但泾原兵变后,文官武将的相继失节与宦官集团的忠心护驾所形成的强烈反差,使德宗彻底放弃了以往的观念,改元贞元后,德宗委任宦官为禁军统帅,掌握监军以防兵变,宦官监军的制度也开始确立。

晚年的德宗日益变得贪婪自私,不但经常把国库赋税收入划拨到自己的内帑,还纵容在外宦官强令地方官进奉贡物,甚至在长安施行宫市以充实自己的小金库。为弥补中央财政,德宗在全国范围内增收茶叶等杂税,导致民怨日深。

德宗在位时期,对外联合回纥、南诏,打击吐蕃,成功扭转对吐蕃的战略劣势,为唐宪宗的元和中兴创造了较为有利的外部环境。

唐德宗于贞元二十一年(805年)逝世,終年63歲。

德宗本人喜好理辩,往往喜欢就某件政务反复讨论诘难,无疑而后实行。因此他曾对李泌比较过前三任宰相的优劣:崔祐甫性格急躁,每當德宗責難他時便應對失態,但德宗因知其缺點而加以迴護;杨炎有才但為人倨傲,面對德宗責難時动辄忿然作色,“無復君臣之禮”,若德宗不批准楊炎的上奏,楊炎便以辭官要挾,数年后德宗还忿然回忆道:“楊炎視朕如三尺童子”;卢杞謹慎但沒有學識,面对德宗的质疑面色如常,唯唯称是,却無法與德宗反覆應對,使德宗無法理清心中疑惑。德宗认为李泌沒有三人的缺點,既能与皇帝平心静气地辩论,又有自己独特的政见,以事論事,往往让德宗真心信服“如此則理安,如彼則危亂”,不得不从其议。


\subsection{建中}

\begin{longtable}{|>{\centering\scriptsize}m{2em}|>{\centering\scriptsize}m{1.3em}|>{\centering}m{8.8em}|}
  % \caption{秦王政}\
  \toprule
  \SimHei \normalsize 年数 & \SimHei \scriptsize 公元 & \SimHei 大事件 \tabularnewline
  % \midrule
  \endfirsthead
  \toprule
  \SimHei \normalsize 年数 & \SimHei \scriptsize 公元 & \SimHei 大事件 \tabularnewline
  \midrule
  \endhead
  \midrule
  元年 & 780 & \tabularnewline\hline
  二年 & 781 & \tabularnewline\hline
  三年 & 782 & \tabularnewline\hline
  四年 & 783 & \tabularnewline
  \bottomrule
\end{longtable}

\subsection{兴元}

\begin{longtable}{|>{\centering\scriptsize}m{2em}|>{\centering\scriptsize}m{1.3em}|>{\centering}m{8.8em}|}
  % \caption{秦王政}\
  \toprule
  \SimHei \normalsize 年数 & \SimHei \scriptsize 公元 & \SimHei 大事件 \tabularnewline
  % \midrule
  \endfirsthead
  \toprule
  \SimHei \normalsize 年数 & \SimHei \scriptsize 公元 & \SimHei 大事件 \tabularnewline
  \midrule
  \endhead
  \midrule
  元年 & 784 & \tabularnewline
  \bottomrule
\end{longtable}

\subsection{贞元}

\begin{longtable}{|>{\centering\scriptsize}m{2em}|>{\centering\scriptsize}m{1.3em}|>{\centering}m{8.8em}|}
  % \caption{秦王政}\
  \toprule
  \SimHei \normalsize 年数 & \SimHei \scriptsize 公元 & \SimHei 大事件 \tabularnewline
  % \midrule
  \endfirsthead
  \toprule
  \SimHei \normalsize 年数 & \SimHei \scriptsize 公元 & \SimHei 大事件 \tabularnewline
  \midrule
  \endhead
  \midrule
  元年 & 785 & \tabularnewline\hline
  二年 & 786 & \tabularnewline\hline
  三年 & 787 & \tabularnewline\hline
  四年 & 788 & \tabularnewline\hline
  五年 & 789 & \tabularnewline\hline
  六年 & 790 & \tabularnewline\hline
  七年 & 791 & \tabularnewline\hline
  八年 & 792 & \tabularnewline\hline
  九年 & 793 & \tabularnewline\hline
  十年 & 794 & \tabularnewline\hline
  十一年 & 795 & \tabularnewline\hline
  十二年 & 796 & \tabularnewline\hline
  十三年 & 797 & \tabularnewline\hline
  十四年 & 798 & \tabularnewline\hline
  十五年 & 799 & \tabularnewline\hline
  十六年 & 800 & \tabularnewline\hline
  十七年 & 801 & \tabularnewline\hline
  十八年 & 802 & \tabularnewline\hline
  十九年 & 803 & \tabularnewline\hline
  二十年 & 804 & \tabularnewline\hline
  二一年 & 805 & \tabularnewline
  \bottomrule
\end{longtable}


%%% Local Variables:
%%% mode: latex
%%% TeX-engine: xetex
%%% TeX-master: "../Main"
%%% End:
