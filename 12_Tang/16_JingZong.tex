%% -*- coding: utf-8 -*-
%% Time-stamp: <Chen Wang: 2019-12-24 15:31:33>

\section{敬宗\tiny(824-826)}

\subsection{生平}

唐敬宗李湛(809年-827年),唐朝皇帝。唐穆宗长子。他是唐朝第16代皇帝(除去武则天以外),15歲即位,824年—827年在位,在位3年,得年17岁。

元和四年六月七日,生于东内(大明宫)之别殿,父亲唐穆宗时为遂王。母王氏是他的妾室。元和十五年,祖父唐宪宗逝世,父亲继位,是为唐穆宗。长庆元年(821年)三月,封景王。二年十二月,立为皇太子。四年正月壬申,父亲穆宗逝世。癸酉,李湛即位柩前,时年十五。

即位后,奢侈荒淫。沉迷擊鞠(古代馬球),喜歡半夜在宮中捉狐狸(打夜狐),史称“视朝月不再三,大臣罕得进见”。宦官王守澄把持朝政,勾结权臣李逢吉,排斥异己,败坏纲纪。导致官府工匠突起暴动攻入宫廷的事件。寶曆二年十二月初八为宦官刘克明等人杀害,死后谥号为睿武昭愍孝皇帝。


\subsection{宝历}

\begin{longtable}{|>{\centering\scriptsize}m{2em}|>{\centering\scriptsize}m{1.3em}|>{\centering}m{8.8em}|}
  % \caption{秦王政}\
  \toprule
  \SimHei \normalsize 年数 & \SimHei \scriptsize 公元 & \SimHei 大事件 \tabularnewline
  % \midrule
  \endfirsthead
  \toprule
  \SimHei \normalsize 年数 & \SimHei \scriptsize 公元 & \SimHei 大事件 \tabularnewline
  \midrule
  \endhead
  \midrule
  元年 & 825 & \tabularnewline\hline
  二年 & 826 & \tabularnewline\hline
  三年 & 827 & \tabularnewline
  \bottomrule
\end{longtable}


%%% Local Variables:
%%% mode: latex
%%% TeX-engine: xetex
%%% TeX-master: "../Main"
%%% End:
