%% -*- coding: utf-8 -*-
%% Time-stamp: <Chen Wang: 2021-10-29 17:19:48>

\chapter{唐\tiny(618-907)}

\section{简介}

唐朝(中古漢語IPA读音:/dang/,618年-907年),中國歷史上的朝代,國祚共历289年,23位皇帝。由唐高祖李淵所建立。李渊先祖李虎在南北朝的西魏是八柱国之一,封为唐国公。其後代李淵為隋朝晋阳(今山西省太原市)留守。唐国公李渊于617年在晋阳起兵,於618年攻入長安,接受隋恭帝楊侑禪位,建立唐朝,是為唐高祖。定都長安(今陕西省西安市),690年到705年为武周,定都神都洛阳。

唐朝歷史可以概略分成數個時期,大致上以安史之亂為界。初唐時軍事實力強盛,但人口處於中國历史上的人口低點之一。李淵建立唐朝,年號武德,是為唐高祖。其子秦王李世民在唐朝建立中立下赫赫戰功,号天策上将,權力膨脹。626年,李世民發動玄武門之變,射殺太子李建成、弟李元吉,逼迫高祖內禪帝位,即為唐太宗。太宗時期對內廣開言路、虛心納諫,成就中國歷史上最出名的治世貞觀之治;對外先後平定東突厥、薛延陀、回紇、高昌、吐谷渾等,受尊為「天可汗」。唐高宗時期击败西突厥、高句麗等強敵,史稱永徽之治,把唐朝版圖擴到最大。高宗去世後,其皇后武后先後擁立兒子中宗和睿宗當傀儡,最後於690年廢睿宗自立為皇帝,改國號曰「周」,即武周,人稱「武則天」。705年,以宰相張柬之為首的五大臣聯合李旦和太平公主發動神龍政變,擁立中宗為帝,唐朝國號得以恢復。唐中宗李顯昏庸,其皇后韋后與其女安樂公主意圖效仿武后。宗室李隆基主導唐隆政變,誅殺韋氏,擁立其父李旦為帝。712年,睿宗禅位于李隆基,是為唐玄宗。玄宗即位後便發動先天政变,賜死太平公主,取得國家最高統治權。李隆基前期任用姚崇、宋璟等能臣為相,勵精圖治,將唐朝帶入極盛時期。開元時期唐玄宗革除前朝弊端,政治開明,威服四周國家,史稱開元盛世。到天寶時期,政治逐漸混亂,於755年爆發安史之亂,唐朝盛極而衰。中唐時,唐朝受到河朔三鎮、吐蕃的侵擾、宦官專權與牛李黨爭等內憂外患的影響而衰退。其間雖然有唐憲宗的元和中興、唐武宗的會昌中興與唐宣宗的大中之治,但是都未能根治唐朝的內憂外患。在晚唐時因為政治腐敗,爆發唐末民變,其中黃巢之亂破壞江南經濟,使唐朝經濟完全瓦解,导致全国性的藩鎮割據。唐室最後被藩鎮朱全忠控制,他迫使唐昭宗迁都洛阳,並於907年逼唐哀帝禅位,唐灭亡,共289年。朱全忠建國梁,史称后梁,进入五代十国时期。

唐朝的疆域廣大,但時常變動,630年就超过隋朝极盛时的版图。唐朝也是自秦汉以来,第一个不使用前朝所筑长城及不筑长城的统一王朝。其鼎盛时期為7世纪,當時中亚的綠洲地帶受唐朝支配。其最大範圍南至罗伏州(今越南河静)、北括玄阙州(今俄罗斯安加拉河流域)、西及安息州(今乌兹别克斯坦布哈拉)、东临哥勿州(今吉林通化)的辽阔疆域,国土面积达1076万平方公里。盛唐時尚能保住和漢朝全盛同等的版圖,但中唐後漠北、西域的領地相繼失去,到晚唐時完全衰退到等同中國本部的大小,但歸義軍起事並歸唐使朝廷一度重奪河西走廊,但到黃巢之亂使唐朝失去甘州、肃州地区的控制權,唐亡时歸義軍仅能控制沙州、瓜州一带。河套地區到五代时期被契丹所占。天宝十三年(754年)户口统计为五千二百八十八万四百八十八人,不过许多学者考虑到当时统计不严,存在大量没有计入统计的瞒报户口,此外还有隐户、佃农、奴婢、士兵、僧道等人群不纳入户口统计,故大多数学者认为唐朝人口峰值在八千万左右。此时,京兆府辖区人口估算在200万人左右,而市区则是100万人。

唐朝在文化、科技、政治、经济、外交等方面都达到很高的成就。在中国历史上有大量的科技发明,四大发明中的火药即诞生于唐朝、雕版印刷开始广泛应用。其政治為三省六部制,前期中央權力在皇帝與宰相,中後期宦官影响力大增。同隋朝推行科舉制度,使得晉朝南朝的世族制度不再興起,中国历史上第一个状元、三元及第,都诞生于唐朝,即622年状元孙伏伽(一说651年的颜康成)。軍事制度前期採用府兵制,軍力強盛,多次擊敗外族。後期則出現節度使(藩鎮)的軍政制度,割據一方,到唐朝後期還出現四十八個藩鎮。唐朝是当时世界的强国,與突厥、高句丽、吐蕃、大食爭奪四方霸權。藉由羈縻制度控制回紇、契丹等北方各族,还调度漠北地区的突厥诸部军队攻打西突厥、高句丽,並且讓南诏、高昌、龟兹、粟特、吐蕃、新罗、渤海国和日本等國家吸收唐朝的文化與政治體制。唐朝的經濟富盛,結合華北、關中與江南的經濟,到後期更加依重江南赋税。土地、盐铁與赋税制度隨著社会改變而改革,由均田制與租庸調制轉向兩稅制,並且增加許多雜稅。其中兩稅制影響中國後半期的賦稅制度。唐朝文化兼容並蓄,接納各個民族與宗教,進行交流融合,成為開放的國際文化。其文學發展達到高峰,以詩最為興盛。當時有李白、杜甫等诗人,以及推行古文運動的韓愈,其史書與傳奇(小說的前身)也十分發達。由於吸收西域特徵與宗教色彩,唐朝藝術與前後朝代都迥然不同,其壁畫、雕刻、書法與音樂都很發達。唐朝声誉远及海外,其歷史地位深重,到明清時期海外多称中国人为“唐人”

隋朝建立後,隋文帝封李虎之子李昞為唐國公,後由李昞之子李淵繼承爵位。在建國之後,以唐为国号。

国号唐是晋國的古名,泛指今山西省的中南部地域。传说遠古帝堯号称陶唐氏,建都于现在山西中南部,后人遂称其所都为唐地。周成王分封其弟虞在古唐地上,为北唐国,后来改国号为晋国。原建国于唐地的帝尧后人则移封现湖北省枣阳市一带,为南唐国,恰好与隋朝国号来源的随国比邻。

唐朝時期漫長,大致上可以分成前期與後期。其分界點可按政治與經濟角度區分成安史之亂與兩稅法的頒布。安史之亂之前,唐朝國力強盛,經濟繁榮,武將四處開疆拓土,文臣穩定朝政,是唐朝的鼎盛時期。亂事發生後,唐朝遭遇許多問題,國力趨向衰退。從經濟的角度看,前期採取均田制與租庸調制,在唐德宗頒布兩稅法後,中國後期的土地制度和賦稅制度基本上以兩稅法為基礎。比較傳統的分法有四分法,即高祖至高宗的初唐、武則天至玄宗的「盛唐」、肅宗至文宗的「中唐」與武宗至哀帝的「晚唐」四個時期。

唐朝皇室自稱出自陇西李氏,陳寅恪经考證认為其为赵郡隆庆李氏之后,而朱希祖考證认为确系陇西李氏,屬於關隴集團之一,與北周皇室和隋朝皇室的關係密切。劉盼遂與王桐齡考據認為李淵家族應為拓跋氏後裔。劉盼遂之後雖取消了自己的观点,但其學說仍引發學界討論。

其先祖為南北朝時期的李虎,他因功被封為西魏北周的八柱国之一,封陇西郡公。其子李昞在隋时封唐国公。

唐朝皇室以老子后裔自居,于佛道之争时偏袒道教。为反击不利局面,和尚法琳声称李氏非老子李耳后裔,与陇西李氏无关,乃拓跋氏之后,因而触怒皇室,被流放益州而死。虽然如此,但李唐宗室确实与拓跋氏有联姻关系,李昞之妻為獨孤信之女,出自匈奴刘氏,李世民之妻長孫皇后为拓跋郁律之后裔。与拓跋部有密切关系。因此日本学者杉山正明创造了“拓跋国家”这一概念,将北齐北周隋朝唐朝这些虽非拓跋氏所建,但统治阶级互相通婚,且国家形态与政治制度相互影响的政权,统统归入到“拓跋国家”中。

唐朝皇室先祖為南北朝時期的李虎,他因功被封為西魏北周的八柱國之一。隋朝建立後,隋文帝封李虎之子李昞為唐國公,後由李昞之子李淵繼承爵位。李渊受隋炀帝重用,於616年被派為太原留守,但隋炀帝對他也不放心,派王威與高君雅監督之。隋朝在大業年间,由於隋炀帝過度使用國力與三征高句丽的失败,使得各地民變不止,史稱隋末民变。李渊見天下大亂,隋朝的灭亡不可扭转,便生起取而代之的念头。617年李淵殺王威、高君雅,在太原起兵造反。不久,李淵率諸子眾將攻破守備關中的屈突通,占領隋都大興城。李淵擁立楊侑為帝,是為隋恭帝,遙尊隋煬帝為太上皇,自任大丞相,進封唐王。而在揚州的隋煬帝,他心灰意冷,不願返回關中,最後於618年的江都政變中被宇文化及等叛軍殺害。李淵藉此機會,於同年五月迫使隋恭帝禅位,建國唐朝,即唐高祖。都城大兴改名為长安,封嫡長子李建成為太子、嫡次子李世民為秦王、嫡四子李元吉為齊王。

李淵建立唐朝后以关中为基地逐步统一天下。在入主關中前,先派使吹捧占據河南的瓦岡軍李密,使其成為東方的屏障。入主關中後,派李世民平定西北金城的薛舉、薛仁杲,派唐使安兴贵、安修仁生擒武威的李轨。620年派李世民击败入侵河东(今山西省)的刘武周、宋金刚。而後洛陽鄭帝王世充与河北夏帝窦建德宣布结盟,聯合抗唐。622年李世民擊潰聯軍,俘窦建德,王世充投降。窦建德的余部刘黑闼也被李建成擊潰,河北至此平定。623年辅公祏率杜伏威余部在丹阳反唐,隔年被唐军俘杀,江南平定。而兩湖地區也在621年唐将李靖於唐平蕭銑之戰獲勝,梁帝萧铣於江陵降唐。翌年,岭南馮盎降服,又虔州林士弘死,中國本部歸唐朝所有。依據五行相生順序,隋朝「火」德之後為「土」德,因此唐朝以「土」為皇朝德運並以與土德對應之黃色為正色。

唐朝的崛起有賴秦王李世民,他的军事才能突出,率军赢得多次关键胜利。掃平群雄後,太子李建成與李世民為了皇位而鬥爭,626年李世民發動玄武门之变,殺了哥哥太子李建成與弟弟齐王李元吉,控制長安。李淵深知形勢,於是禪讓帝位,成為太上皇。李世民繼位,即唐太宗。。

唐太宗勵精圖治、納諫如流,逐漸恢復唐朝的國力。在內政方面,唐太宗推行均田制與租庸調制,提升農業發展。在職官制度上,改良隋朝的制度,形成三省六部和科举選士制,限制皇权發展與貴族世襲等惡習。唐太宗不計出身,網羅一大批精明強幹的大臣,比如房玄龄、杜如晦、长孙无忌、魏徵、马周、高士廉和蕭瑀等文臣,尉遲敬德、李靖、侯君集、程知節、李世勣和秦叔寶等武將。此外,唐太宗派官員四處詢問百姓的生活情況,然後把各官員的功過寫在屏風上,以便褒貶。

對外方面,唐太宗採取積極防禦、以戰止戰的策略,以及用羈縻與武力的方式安撫四方。隋末唐初之際,北方東突厥汗國十分強大,時常南下侵擾,並且介入中原各勢力。唐朝初期百廢待舉,626年東突厥突然襲擊长安,率軍抵達距離長安不遠的泾阳(今陕西咸阳泾阳县)。對此唐太宗亲率高士廉、房玄龄等在渭水隔河與突厥可汗對峙,定下渭水之盟。之后,唐太宗積極對付突厥,挑拨颉利可汗與突利可汗的關係,以及突厥与週圍诸部的关系。627年东突厥的藩屬薛延陀、回纥、拔也古、同罗诸部因為不認同颉利可汗的政令與改革国俗,紛紛脫離,改立薛延陀部为可汗,突利可汗也歸降唐朝。628年朔方人梁洛仁杀盤據夏州的梁师都,归降唐朝。而东突厥在分裂後又遇到大雪侵襲,牲畜大多被冻死饿死。629年李靖率騎兵奇襲攻滅东突厥,隔年北方各族入貢長安,諸民族尊稱唐太宗為天可汗。635年派李靖攻占吐谷渾,657年派蘇定方西征攻下西突厥汗國,641年派文成公主與吐蕃贊普松贊干布通婚。這些都穩定唐朝與四方各國的關係。

贞观時期國家安定,經濟得到恢复和发展,史稱“贞观之治”。《資治通鑒》記載,貞觀四年(630年)一斗米不過三、四錢,全年死刑犯僅二十九人。成书于唐中宗时期的《贞观政要》中对于唐太宗政績的總結,成爲日本和新罗帝王的治国教科書,亦為後世君主模彷學習的對象。

唐太宗晚年,發生太子李承乾與魏王李泰內鬥的事件。所以唐太宗廢承乾,逐李泰,改立晉王李治為太子。唐太宗去世後,李治即位,即唐高宗。此時唐朝承繼贞观之治,國力鼎盛,史稱永徽之治。當時尚有宿將如李勣、蘇定方、薛仁貴等,名臣長孫無忌、褚遂良等。對內持續推行均田制,選用較低級但有才能的官吏。對外於659年消滅西突厥,疆域西擴至鹹海與阿姆河一帶,設立安西都護府於碎葉城(今吉爾吉斯托克馬克市)。並且於蔥嶺以西設置十六個都督府,讓吐火羅葉護、訶達羅支國王等等中亞君主兼任都督。在東方,与新罗联合滅掉東北強國高句丽和百济,並白江口之战擊敗日本援軍。唐朝在朝鮮半島建立安東都護府,也間接促使新羅統一朝鲜半岛。

高宗中期以后,任命皇后武氏協助理政。武則天原為太宗時期的才人,太宗死後被高宗招入宮中。她在權力鬥爭中獲勝,被立為皇后,史稱「素多智計,兼涉文史」。656年起,高宗因健康原因,許多政事都逐漸交給武后處理,武后成為最高統治者之一,與高宗並稱「二聖」(天皇與天后)。高宗去世後,太子李顯即位,是為唐中宗。因為與中宗不合,武太后不久將中宗廢為廬陵王,改立四子李旦為帝,是為唐睿宗。武后平定徐敬業的反叛後,於690年廢睿宗,即皇帝位,改國號為周,即武周,改東都洛陽為「神都」,上尊號「聖神皇帝」,人稱「武則天」,改立李旦為皇嗣,成為中國歷史上唯一的女皇帝。在武則天掌權與稱帝的期間,國家人口持續增長,但外戰不利,疆域大量萎縮。武则天执政期间,科舉制度得以進一步完善,開創出殿試和武舉,她大力提拔科舉出身的官員。這批官員中有許多在後世成為賢臣能吏、如狄仁傑、張柬之、張仁願、姚崇等。。然而由於武則天本人信仰佛教,她大量賞賜和尚尼姑田產,徵用農田建造佛寺,土地兼並嚴重、導致均田制崩潰。。武則天執政前期為了維持自己的統治,或殺害或流放了數位名將,啓用的「武三思」、「武懿宗」、「薛懷義」等人多是平庸之輩,導致其執政時期外戰敗多勝少,國家疆域大量萎縮。武則天執政的另一特點是強力控管,主要有嚴厲鎮壓徐敬業等反對派、屠殺唐宗室親王與支持唐朝的大臣將領。鼓勵告密,暗中監控官吏、諸侯,以及推廣酷吏制度。扶持武三思、上官婉儿等黨羽。這些在后世经常受到史學家的批评。

武則天晚年,聽從狄仁傑的勸告,重立李顯為太子,改立李旦為相王。705年武則天病重時,宰相張柬之與將領李多祚等人擁太子李顯發動政變,他們杀女皇的男宠张易之兄弟,逼武則天退位。中宗李顯重祚,唐朝復辟,封其弟李旦為安國相王,其妹太平公主為鎮國太平公主,史称神龍革命。中宗統治經驗頗為缺乏,在位時政治腐敗,貪墨成風。他受到韋后、女兒安乐公主和武氏黨羽武三思等人迷惑,將功臣張柬之和敬琿等人全部流放誅殺。韋后與安樂公主野心勃勃,想要成為武則天第二。他們與上官婉儿聯手迫使太子李重俊發動景龍之變,重俊最後事敗被殺。710年韋后和安樂公主疑似唆使他人在餅中下毒害死中宗,立溫王李重茂為帝,即殤帝,並且打算加害相王李旦。李旦之子李隆基在姑母太平公主的協助下發動唐隆之變,誅盡韋后與武氏勢力,擁立睿宗李旦復辟為帝。睿宗復位後,立其子李隆基為太子,同意其妹太平公主干預政局,雙方時常發生權力鬥爭。712年睿宗決定禪讓帝位,太子李隆基即位,即唐玄宗。但是玄宗宣稱太平公主又準備用羽林軍兵變。隔年,玄宗賜死太平公主,發兵誅殺與其黨羽,即先天之變,唐朝自武則天以來的女主政治,至此結束。

唐玄宗時期可分為開元與天寶兩個部分,其中開元時期的政治比較清明。當時唐朝政治日益敗壞,唐玄宗提出以武、韋為戒,以貞觀為榜樣,作為執政的指導思想。他先後任用姚崇、宋璟、盧懷慎、张九龄與韓休等賢臣,並且廣納諫言。例如采纳张九龄的建议,將京官中有能之士外调为都督刺史以训练行政能力,又將有為的都督刺史升为京官。增进中央与地方的沟通、了解和信任。裁減武周中宗時期的員外官等冗官,精簡機構以便節省開支與提升行政能力。嚴格執行法律,抑制權貴,就算是皇親國戚犯罪,也繩之以法。對於穩定社會秩序產生良好的影響。加強執行均田制,打擊土豪。發展農業生產,興修水利,擴大耕地面積,大大提升農業生產力。對外方面,改善與吐蕃、東突厥、契丹與奚的關係,推行和親政策。聽從姚崇與宋璟的建議,充實邊防軍務,並且避免與外族發生戰爭。這些措施使唐朝進入第二個全盛時期,人口大量增長,物產豐富,史稱開元之治。當時不僅中原地區、江淮地區以及成都平原經濟發達,連人口較少的隴右河西地區也逐漸繁榮。

天寶時期時,唐玄宗志得意滿,放縱享樂,不問國事,並且納兒媳楊貴妃。此時國政漸亂,唐玄宗罷免賢相张九龄,相繼以李林甫與楊國忠為相。李林甫有“口蜜腹劍”的惡名,,他蔽塞言路,排斥賢才,採取任用不擅文采的蕃將為邊將以杜絕「出將入相」之源,使得唐廷陸續任用高仙芝、哥舒翰與安祿山等邊將。此時宦官也逐漸崛起,高力士權勢炙手可熱。在軍事上,由於唐朝多年的戰爭使得府兵制崩潰,兵源逐漸改為募兵制,禁軍也進一步獲得擴大。唐玄宗為了便於管控遼闊的邊疆,於722年設置九個節度使與一個經略使。節度使不只負責軍事,之後還兼顧地方民政與財務,久之形成節度使尾大不掉的局面,也成為藩鎮割據的遠因。對外方面,唐玄宗好大喜功,為此邊將經常挑起對外戰事,以邀戰功。當時唐朝正與吐蕃、黑衣大食(即阿拉伯帝国的阿拔斯王朝)爭奪在西域與中亞的勢力,其中以751年的怛罗斯战役最有名。唐將高仙芝被阿拔斯王朝與石國聯軍擊潰而喪失在中亞的地位,而後因為中土爆發安史之亂,唐朝也沒有恢復地位的打算。

節度使的權力甚大,當與中央發生衝突時,就很有機會發生叛亂。當時又以身兼范阳、平盧、河東三鎮節度使的安祿山最有機會,他甚獲唐玄宗寵信,與丞相楊國忠勾心鬥角。755年十一月,安祿山以討伐楊國忠為由發動叛亂,史稱安史之亂。楊國忠與封常清認為敵軍不足憂慮,命郭子儀自朔方出兵河北、高仙芝提大軍出潼關戰關東。十二月,封高兩將皆敗,東都洛陽淪陷,唐軍退守潼關。封高二人被讒言所殺,改由哥舒翰堅守潼關。於河北舉兵的常山(今河北正定)太守顏杲卿也在隔年正月被叛將史思明擊潰,關東一帶盡數淪陷。然而郭子儀與河東李光弼進軍河北,會師恆州(今河北真定),擊敗叛军将领史思明,叛军軍心大亂。然而,唐玄宗與楊國忠急於平亂,強迫哥舒翰出兵。六月,哥舒翰將兵八萬與賊將崔乾祐戰於靈寶西原,官軍大敗,死者十六七。哥舒翰退至潼關,為其帳下火拔歸仁以左右數十騎執之降賊,關門不守,京師大駭,唐玄宗緊急南逃蜀地成都,途中發生马嵬驿之变,楊國忠與楊貴妃在憤怒士兵的要求下被殺。而太子李亨奉唐玄宗之命,前往西北靈武募兵。安祿山占据長安後建僭燕。七月,李亨抵達靈武後,在宦官李辅国擁立下稱帝,即唐肅宗,奉唐玄宗為太上皇。

唐肅宗命其子李俶統領諸將,以李泌輔佐,派僕固懷恩出使回紇請兵。當時唐將房琯反攻長安失敗,局勢一度危急。757年叛军內訌,安祿山之子安慶緒殺父奪位,史思明回守范陽,並掌握河北軍力。繼而郭子儀和李光弼率軍返回靈武,並聯合回紇,於年底收復長安。然而叛军早於十月攻克江淮重鎮睢陽(今河南商丘),張巡與許遠戰死。所幸郭子儀接著攻下洛陽,牽制叛军。不久,安慶緒退回鄴城(今河北臨漳),謀除史思明。史思明得知後投降唐朝,叛军勢力只剩鄴城一帶,758年郭子儀、李光弼等九節度使圍攻鄴城。然而唐廷想要消滅史思明之事外洩,史思明於隔年三月率叛军南下擊潰唐軍,史稱鄴城之戰。郭子儀被魚朝恩讒毀而返回長安;史思明殺安慶緒,併吞其部,自稱帝,以范陽為都;李光弼因叛军攻克洛陽而退守,局勢急轉直下。761年李光弼反攻洛陽失敗,史思明獲捷後居然被其子史朝義所殺,叛軍分崩離析。762年太上皇與唐肅宗相繼去世,太子李豫(原名李俶)繼位,即唐代宗。唐代宗派其子李适統領諸將,僕固懷恩為副,率唐軍與回紇軍攻克洛陽。史朝義北走范陽,僕固懷恩率軍追擊,河北叛將李懷仙也投降唐軍,並一同追擊。763年正月,史朝義在石頭(今河北唐山東北)自縊,八年的戰亂才告平定。

安史之亂成為唐朝歷史上的转折点。藩鎮割據、外族入侵、宦官專權與牛李黨爭等蜂擁而至,成為唐朝的內憂外患。唐室為了盡快結束戰事,將安史降將就地封為節度使以安撫之。為了提防降將復叛,又遍地安置节度使。由於节度使兼管地方军事、政治和经济,全國各地幾乎處於半獨立的狀態。戰後關東人丁锐减,土地大量荒蕪,河北之地逐漸胡化,人民好武輕文,與詩賦取士的關中之地相比,形成截然不同的文化區。由於邊防軍調回平亂,外族紛紛入侵。吐蕃占領隴西、攻入關中,长安一度淪陷。回紇的勒索也消耗國力。宦官專權,李輔國、程元振擁立唐代宗為帝,是唐朝第一個受宦官擁立的皇帝,宦官魚朝恩更被委任統領禁兵。779年唐代宗就在這些亂事中去世,長子李适繼位,即唐德宗。

唐德宗在初期頗能勵精圖治,堅決削滅藩鎮,藩鎮對其較為敬畏。他起用楊炎推行兩稅法,以劉晏改革漕運,修改鹽法,行常平法以改善財政。但是他為人剛愎忌刻,沒有任人之明。781年任用奸相盧杞後,政治日非。聽信盧杞讒言,誅殺楊劉兩臣。政治的敗壞使藩鎮逐漸輕視,最後爆發亂事。同年,成德李寶臣去世,其子李惟岳不被唐室同意繼任,他就聯合魏博田悅與淄青李納舉兵叛亂。唐室派馬燧、李晟擊敗叛軍,田悅被中央軍圍困於魏州(今河北大名),李惟岳則被部下王武俊所殺。另一方面,盧龍朱泚入朝後,由其弟朱滔繼任盧龍節度使。由於盧龍朱滔與成德王武俊對朝廷不滿,就聯合淄青李納、淮西節度使(約今河南省東南)李希烈叛亂,共推朱滔為盟主。調來抵抗淮西的涇原軍也因為對朝廷賞賜不滿,爆發涇原兵變,唐帝出逃奉天(今陝西乾縣)。涇原軍入長安後,共立朱泚為帝,並且包圍奉天。李晟緊急率中央軍回師關中,與朔方軍李懷光解奉天之圍。事後,唐德宗因盧杞讒言而不召見李懷光,雖然最後盧杞被貶,李懷光仍然怨恨唐帝。784年唐德宗採用陸贄之策,同意諸藩鎮的要求,只有朱泚不赦,並且廢除苛稅,諸藩鎮紛紛歸服。朱滔和李希烈不願投降,拉攏李懷光倒戈,唐德宗又逃到梁州(今陝西南鄭)。同年,李晟收復長安,朱泚於東逃之際被部下所殺,李懷光也在隔年被馬燧、渾瑊所滅,淮西李希烈也被部下所殺,至此亂事平定。然而,唐室承認藩鎮的統治權,加深割據局面。由於唐德宗不信任將領,禁軍轉由宦官掌控,宦官權勢薰天。唐德宗晚年任用奸臣裴延齡,並且親暱宦官貪吏,國政日衰。805年唐德宗去世,太子李诵繼位,即唐顺宗。

唐代宗遺留下來的問題越來越嚴重,唐順宗與唐憲宗都企圖解決,其中唐憲宗較為成功,實現元和中興。唐順宗以韋執誼為宰相,啟用以王叔文為首的改革派。他們廢除欺壓百姓的宮市和五坊小兒,減輕稅賦。任韓泰掌控神策軍,試圖奪取宦官軍權,史稱永贞革新。同年,唐順宗中風,宦官俱文珍利用太子李純想做皇帝的心理,聯合韋皋等等藩鎮迫使唐順宗讓位,藉此扳倒改革派,史稱永貞內禪。太子李純繼位,即唐憲宗。唐憲宗頗能駕馭宦官與外廷,做事勤勉政務,善於納諫。他採納杜黃裳的建議著手削藩。當時全國共有四十六處藩鎮,大都在半獨立的狀態,只剩浙江一帶還供應朝廷的財務來源。他採取擴寬財路,力行節儉的方式以穩固財力。由於藩鎮中以安史系最強,他先從較弱的藩鎮下手。806年劍南西川節度副使劉闢、夏綏留後楊惠琳與隔年的鎮海李錡先後叛變,被唐室一一平定。接著是牽制數十萬唐軍的安史系淮西節度使吳元濟,814年由於吳元濟四處掠奪且私自传位继承,唐憲宗先後派十六鎮的兵力討伐之,然而未能成功。期間淄青李師道與成德王承宗派人刺殺主戰派宰相武元衡,唐帝復以裴度代替,並以李愬(李晟之子)主討戰事。817年李愬採降將李祐之計,雪中奇襲吳元濟總部蔡州(今河南汝南),淮西平定。淄青李師道恐慌,唐憲宗派李光顏、李愬率軍討伐。兩年後李師道被部下所殺,淄青平定。河北方面,魏博田弘正支持唐室。王承宗曾經反叛唐室,淮西平定後與盧龍劉總歸順唐室。到819年,全國藩鎮在名義上都服從中央,派使納貢,史稱元和中興。然而唐憲宗對國事有點荒怠,喜好營建豪宅。他十分崇佛,曾經赴法門寺奉迎佛骨,韩愈勸諫而被貶。

820年唐憲宗在大明宮被宦官毒死,河北三鎮復叛,中興時期結束。821年盧龍劉總離職,唐室派張弘靖接管。張弘靖管理不當,盧龍兵擁護朱克融叛變。移鎮成德的田弘正被將領王庭湊奪位殺害。魏博田布(田正弘之子)被軍隊迫死,魏博軍擁護史憲誠叛變,至此河北三鎮復叛。當河北未叛之時,大臣蕭俛、段文昌建議國家裁軍。如今被裁之兵都投奔河北三鎮,助長其勢。然而此後的河北三鎮並非持續強盛,唐敬宗與唐武宗期間,河北三鎮大多受制其強兵,有時還被部下篡位,遠遠不如當初的跋扈。而各地藩鎮依然聽命於中央,直到黃巢之亂為止。

唐朝中央的政治大權大多由皇帝與宰相掌控,但在天寶之後轉變成皇帝與内廷宦官的聯合,外廷宰相變成政治上的二流角色。涇原兵變後,皇帝不再信任武臣,宦官更加把持者中央禁軍(即神策軍)。再加上唐朝中後期的皇帝普遍不立皇后,導致沒有外戚勢力可以平衡宦權,相權又低落,使得宦官勢力極度膨脹,永貞內禪事件更使宦官成功擊敗外廷與士大夫。

掌控軍政大權的宦官一躍成為中央的幕後掌控者,唐憲宗之後的皇帝大多被宦官任意廢立,例如宦官王守澄就是一個好例子。820年唐憲宗被宦官陳弘志毒死,這個事件背後是宦官王守澄指使。王守澄扶持太子李桓繼位,即唐穆宗。他即位後遊樂無度,河北三鎮復叛,宦官背後掌控的牛李党争亦愈演愈烈。唐穆宗即位三年就去世,其子李湛繼位,即唐敬宗,大權仍由王守澄掌控。唐敬宗同樣不理朝政,專好遊樂擊球擺宴。826年唐敬宗出去「打夜狐」,回宮後大擺宴席,被宦官劉克明所殺。劉克明有意奪王守澄權,擁立絳王李悟。王守澄得知後以兵迎立唐穆宗之子江王李函,並且殺死政敵。李函繼位,即唐文宗。當時王守澄權勢最大,其次為陳弘志、仇士良等。

唐文宗勤勉聽政、生活節儉,本身十分厭惡宦官,隨時想聯合外廷大臣扳倒宦官。831年與宰相宋申錫合謀失敗,宋申錫被殺。而後唐文宗與大臣李訓、鄭注聯手發動政變。他們都是王守澄推薦的,因此宦官毫不忌諱。李鄭二人先建議唐文宗提拔與王不合的仇士良,並且杖殺元和逆首陳弘志,貶死若干掌權宦官。835年,唐文宗以李訓為宰相、鄭注掌鳳翔節度使,內外呼應。接著密派中使毒殺王守澄,至此元和逆黨皆誅殺殆盡。李訓更擴充勢力與軍權,與只掌握神策軍的宦官尚可一拼。835年李訓發動甘露之變,意圖將皇帝從宦官手裡搶出,但宦官仇士良搶先奪回皇帝,並且以神策軍擊潰政敵,誅殺大臣。甘露之變後,宦官們團結一致對外,並且牢固地掌握軍政大權,皇帝與大臣徒具擺飾,即便是後期的唐武宗與唐宣宗也無法消滅宦官的勢力。而大臣只能借藩鎮對抗宦官權力,埋下晚唐藩鎮入關奪權的陰影。840年鬱鬱寡歡的唐文宗去世,其弟在宦官仇士良的擁立下繼位,即唐武宗。由於當時朝廷派系林立,仇士良只好讓唐武宗親自處理朝政。唐武宗重用李德裕以削減仇士良權力,也提出一連串振興朝廷的政績,史稱會昌中興。他大力推行滅佛,史稱會昌滅佛。唐武宗推行道教,希望长生不老,最後因為服金藥去世。

在唐憲宗到唐宣宗期間,發生較長的黨爭,即稱牛李党争。這兩派分成以經學為正統、大多是關東世族的李黨,主要有李吉甫、李德裕、鄭覃;以文彩華麗、高宗武后以來進士科出身的牛黨,主要有李宗閔、牛僧孺等。兩派士大夫背後都有宦官當後台,宦官有最終掌政權。兩派明爭暗鬥的很厲害,徒然消耗國力。政見方面,李党主張对藩鎮與吐蕃用兵,而牛党主张和平。牛党倾力拥护科举制度,李党極力要求改革。李党建议精简国家机构,牛党反之。黨爭起始於808年的科舉考試,當時宰相李吉甫(李德裕之父)主張對藩鎮用兵,舉人李宗閔、牛僧孺與皇甫湜在考卷裡批評朝政失當。李吉甫得知後打壓這些人,這引起朝野嘩然,李吉甫最後也失勢,朝中大臣也逐漸形成兩黨以互相鬥爭。然而當時主戰派宦官吐突承璀把持權力,所以李黨仍然得勢。唐穆宗時,由牛黨人物钱徽主持進士考試,卻被告徇私舞弊。在時任翰林學士的李德裕證實下,錢徽被降職,李宗閔也受牽連而被貶謫到外地。從此牛李兩黨各樹朋黨,互向傾軋。李黨有李德裕、裴度、李紳等,牛黨有李宗閔、牛僧孺與李逢吉等。然而,主和派宦官王守澄崛起,李黨失勢,時任宰相的牛僧孺與李宗閔、李逢吉聯手,牛黨勢大,李德裕被罷免外放。牛黨的優勢一直到823年,牛僧孺因為被唐文宗不滿而罷相,隔年由李黨的李德裕上台,這是顯然與王守澄放棄牛黨有關。之後王守澄支持李訓與鄭注,極力打壓牛李兩黨。甘露之變後李鄭勢力崩潰,宦官由仇士良掌權。唐武宗時任用李德裕為宰相,極力排斥牛黨。

846年唐武宗去世,宦官們發生權力鬥爭,其叔李忱在宦官马元贽的扶持之下即位,即唐宣宗。由於李黨失勢,李德裕被貶黜到崖州(今海南瓊山),至此長達40年的牛李黨爭結束。唐宣宗表面上是容易被宦官利用的君主,但即位以後勵精圖治,加強皇權、抑制宦官權力,是時唐朝又出現短暫的復興景象,史稱大中暫治。然而唐宣宗為人多疑苛察,使得上下莫不粉飾太平;他崇奉道教,一直希望能夠通過服用丹藥來長生不老。859年唐宣宗因服用丹藥過度而去世。實際上,大中暫治並不穩定。唐宣宗晚年,國內已有亂象,他死後不久就爆發寇亂。

唐宣宗去世後,相繼為帝的唐懿宗與唐僖宗是著名的无道昏君,使唐朝的國勢一直走下坡。政治敗壞、社會貧富差距過大,不少叛亂相繼發生,唐朝經濟命脈的江南地區也被破壞殆盡,徹底動搖這個政權,也產生李國昌、朱全忠等新藩鎮。859年唐懿宗繼位,他為人驕奢淫逸,寵信宦官;並且篤信佛教。為了崇佛,不惜削減軍費。860年後相繼發生裘甫之亂、庞勋之变與王郢之变(僖宗時期)。其中庞勋之变破壞關東地區的經濟,有賴沙陀軍首領朱邪赤心率軍助戰而定,朱邪赤心因功賜名李國昌。873年唐僖宗繼位,為人專好鬥雞打毬,寡顾朝政,更大的叛亂在北方誕生。由於關東連年水災,加上政治敗壞,盐价锐升,使得盜賊不斷。874年王仙芝聚眾於長垣(今河南長垣)起事,隔年攻陷山東西部、流竄於河南淮南一帶,聲勢益盛。878年王仙芝戰死於黃梅(今湖北黃梅),餘部潰散投奔黃巢。黃巢由毫州(今安徽毫州)南下掠奪江南與嶺南地區,沿路屠殺不斷,並且攻陷商業大城廣州,華南經濟幾乎全毀。879年因為軍隊遭遇瘟疫,黃巢率軍經桂州、沿湘江北上流竄回江南。隔年,黃巢正式西進,攻陷洛陽與潼關。掌權宦官田令孜帶唐僖宗逃往四川,黃巢入長安後建國齊。各地勤王之師也因為號令不整,收復的長安又被黃巢奪回。唐室只好赦免叛逃漠北的李國昌、李克用父子,李克用率沙陀軍協助唐軍克復長安。另一方面,黃巢部將朱溫投降,賜名朱全忠,受封宣武節度使(治汴州)。黃巢東走並且包圍朱全忠於陳州。884年李克用率軍解陳州之圍,並且追擊黃巢軍。黃巢於隔年被其甥林言斬殺投降,黄巢之乱平定。而後,黃巢降將秦宗權叛變,率軍在中原地區四處攻掠,一度攻陷東都(今河南洛陽),造成「極目千里、無復煙火」的局面,直到唐昭宗時才由朱全忠平定。

平定民變後的唐室因為國力衰退而被關中藩鎮反噬。而宦官與外廷為了政治鬥爭又拉攏藩鎮加入戰局,最後演變成各藩鎮爭奪朝廷。這些藩鎮以河東李國昌、宣武朱全忠與鳳翔李茂貞最強。885年唐僖宗返京後,仍然信任宦官田令孜。田令孜與河中節度使(轄今山西省南部)王重榮交惡,雙方都拉攏藩鎮並抗衡。王重榮與李克用聯軍成功的攻入長安,田令孜又帶唐僖宗出京避難。原本與田令孜合作的朱玫、李昌符也倒戈,率軍追擊田令孜。兩人奉襄王李熅監國,朱玫被任宰相,李昌符暗中不滿,在興元(今陝西南鄭)的唐室趁機說服王重榮、李克用與李昌符聯合收復長安。唐僖宗返京途中又與鳳翔李昌符發生衝突,當時王重榮被部下所殺,唐僖宗有賴李茂貞平定才得以返回長安,李茂貞也繼任鳳翔節度使。888年唐僖宗去世,其弟李曄被宦官楊復恭擁立,即唐昭宗。宣武朱全忠與河東李克用因故不合,雙方上至朝廷,下至藩鎮,都鬥爭不斷。當時張全義與李罕之爭奪河陽節度使(治河南孟縣),雙方分別拉朱全忠與李克用對戰。結果朱全忠獲勝,兼併河陽、洛陽,擊敗秦宗權後幾乎占領全河南省。當時宦官楊復恭與宰相张濬不和,雙方分別拉攏李克用與朱全忠。890年朱全忠與张濬攻河東軍失敗,张濬被貶。李克用趁機併吞昭義的潞州、澤州,約佔領今山西省地區。不久宦官楊復恭失勢,南依其兄子山南西道節度使楊守亮叛變,唐室以李茂貞等人平亂,李克用在朝廷的勢力衰退。鳳翔李茂貞因不能擴張地盤與唐帝不和,雙方發生戰爭。最後李茂貞與王行瑜戰勝,他們掌控關中地區,宦官與外廷受其管制,唐室只剩首都一地。

此時唐帝淪為各藩鎮角力的戰利品,最後被藩鎮擄走,取而代之。895年河中王重盈去世,王行瑜、李茂貞與韓建等人與河東李克用爭奪河中。王行瑜趁機入京殺宰相韋昭度等人,並謀廢唐昭宗。李克用緊急率軍入援,而王行瑜被部下所殺,唐室才得以安定。事後,唐室建立殿後四軍,李茂貞、韓建搶先於896年逼近長安,唐昭宗逃到華州,殿後四軍被廢。最後有賴李克用、朱全忠率軍入援,唐昭宗得以於898年返回長安。900年宦官劉季述立唐昭宗嫡长子皇太子李裕為皇帝(李縝,即德王),901年李縝被崔胤所廢,改回原名李裕并降封為德王,昭宗復辟。而後宰相崔胤与宦官韓全誨争权,韓全誨强迫唐昭宗投靠自己的盟友李茂貞,崔胤緊急招喚朱全忠入援,朱全忠于是率軍圍困鳳翔。隔年,鳳翔軍糧草耗盡,李茂貞只好殺宦官韓全誨等人,與朱全忠和解。朱全忠趁機掌控朝中大權,還屠杀宦官數百人,派兵控制長安。崔胤後悔不已,有意擺脫朱全忠的威脅,暗中召募六軍十二衛,被朱全忠在長安的眼線所察觉。904年朱全忠殺崔胤,逼迫唐昭宗遷都洛陽,長安城被毀。同年8月朱全忠弑帝,另立昭宗子李柷为帝,即唐哀帝。隔年,朱全忠杀李裕等昭宗年长九子,大肆貶逐朝官,並全部殺死於白馬驛,投屍於黄河,史稱白馬之禍,年末又听信诬告杀害哀帝母何太后。朱全忠本想等統一後再夺取帝位,但因征淮南失利,所以提早於907年逼迫唐哀帝禅让,建國後梁,唐朝亡,五代十國時期开始。923年,唐朝的赐姓宗室李克用之子李存勖消灭後梁,重建唐朝。在魏州(河北大名县西)称帝,復興唐朝,後為石敬瑭勾結契丹入侵而滅亡,歷時十四年,史稱後唐。

唐朝疆域變遷圖。本圖為完整呈現唐朝各時期的領土變遷,共分成貞觀元年(627年)、貞觀十四年(640年)、貞觀二十一年(647年)、顯慶五年(660年)、龍朔二年(662年)、麟德二年(665年)、總章元年(668年)、咸亨三年(672年)、儀鳳四年(679年)、開元三年(715年)、天寶十年(751年)、元和十五年(820年)、大中二年(848年)、大中三年(849年)、乾符二年(875年)。

唐初是唐朝武功興旺的时期。在漠南漠北方面,在唐高祖建立唐朝对突厥做出战略防守退让求和之后开始反击。貞觀四年(630年),唐军滅亡東突厥,漠南成為唐势力范围。貞觀二十年(646年),又联手铁勒部落一舉消滅薛延陀汗國,至此大漠南北广大地区皆為唐的势力范围。唐朝廷在漠北設立安北都護府,在漠南設立單于都護府,建立南至罗伏州(今越南河静)、北括玄阙州(後改名余吾州,今安加拉河地区)、西及安息州(今乌兹别克斯坦布哈拉)、东临哥勿州(今吉林通化)的辽阔疆域。但永淳元年(682年),突厥復國,漠北等地遂为其占,后直到后突厥灭亡为止唐朝的北方边患都很严峻。天寶三载(744年),回紇建國,占据漠南漠北。安史之乱后,边患再起,但唐朝与回纥并没有发生大规模的战争。

在西北,貞觀四年,唐朝廷在伊吾七城設立西伊州,開始經營西域。貞觀十九年(645年),唐朝廷移安西都护府到龟兹。显庆四年(659年),唐军又灭西突厥,势力及咸海到里海一带。但唐朝廷對蔥嶺以西地區的統治始终不稳固,乾封二年(662年),阿史那弥射死,阿史那步真统领西突厥十姓,此后蔥嶺以西一直为唐朝臣属国,尤其是吐火罗。。安史之亂爆发後的三十六年时间内,唐朝陆续失去原安西都护府所辖地区。

在东北,顯慶五年(660年),唐军联合新罗滅亡百济。總章元年(668年)八月,唐军与新罗又滅高句丽,並設安東都護府於平壤。但由於當地人民反抗激烈及新罗勢力的北進,咸亨元年(670年)安東都護府內遷遼東。开元元年(713年)安東都護府移到遼西。天寶年間(742年—756年)安東都護府廢,安史之乱后唐朝逐渐失去对辽东半岛的直接控制。武周圣历元年(698年)其首領大祚荣建立震國,唐朝称之为渤海国;號為「海東盛國」,但与唐朝的关系友好,大部分时间向唐朝称臣。

在青藏高原上,吐蕃日漸興起,至6世紀末與吐谷浑、蘇毗為高原上三大勢力。7世纪初,贊普松贊干布即位,統一高原,又征服位於西藏西部的蘇毗、阿裏地區的羊同和尼婆羅(今尼泊尔)。龍朔三年(663年),吐蕃滅吐谷渾,盡有其地。後又多次占领唐朝的安西四镇,為唐朝最大敵國。安史之亂後,由於大量河隴邊兵參與平亂(主要為隴右節度使、河西節度使所部)導致邊防空虛,吐蕃趁勢進逼,占领原属于唐朝的陇西,黃河以西甘、涼皆不可得,陇山以西为吐蕃占据。唐宣宗大中二年(848年),沙州(甘肃敦煌)人张议潮发动起义,唐人群起响应,很快占领沙州。接着,张议潮又派兵攻取瓜、伊、西、甘、肃、兰、鄯、河、岷、廓(以上地区在今甘肃、新疆、青海境内)等十州。大中五年(851年),张议潮遣其兄张议潭奉沙、瓜等十一州地图入朝,唐宣宗在沙州置归义军,以张议潮为节度使,河陇地区又重新为唐朝廷所控制。890年,河西、陇右又被党项族占据。但終唐之世已完全喪失對於敦煌以西的控制。

在西南云贵高原,天宝七载(748年)南诏建國,與唐時戰時和,也削弱唐朝的國力。同時,自汉武帝平南越後的相当长时间内是中国領土的安南(越南北部),唐代統治時先後設立「交州總管府」、「安南都護府」(唐肅宗改名鎮南,唐代宗復稱安南)、「靜海軍節度使」等官署,唐末时开始藩镇割据,土豪兴起,至北宋初完全脫離中原王朝而獨立。

隋朝前期實行州縣制,後期實行郡縣制。唐又改郡為州,恢復州縣二級制。貞觀元年,天下大定,又對州縣進行省並。唐朝還在州一級的行政區劃中設立「府」這一建制。先是開元元年設立京兆府和河南府。今後陸續升新的陪都和皇帝到過的地方為府。同時,唐朝根據山川形便將全國分為關內、河南、河東、河北、山南、隴右、淮南、江南、劍南、嶺南十道,是为贞观十道。神龍二年設立十道巡察使、十道存撫使和十道按察使。這些都是監察官,為中央臨時派遣,不常置,也无固定治所。開元廿一年又从关内道分立京畿道,从河南道分立都畿道,分山南道为东西两道,分江南道为江南东、江南西和黔中三道,共十五道,是为开元十五道,每道設立固定的監察官員(觀察使),有如汉朝的刺史,也设立了固定的治所(首府),正式成為十五個監察區,并逐渐向行政区转变。这十五道如下:

驻守各道的武将称都督,都督带使持节的称节度使。不带者不称。在安史之亂平定後,唐朝政府增加了許多节度使,而節度使管轄的地區稱為藩镇。唐政府本企圖可借節度使來平定一些叛亂,不料這些節度使擁兵自重。唐朝末期因此形成了道(方鎮)、州(府)、縣三級行政區劃。唐末年全國有四五十個鎮,除了京兆府和周圍幾個州以及河南府外,全國其他地方都是藩镇割据的局面。唐德宗時期,河朔一帶的藩鎮叛亂,佔領京師長安,德宗逃到漢中,用了四年的時間才平定,從此之後藩鎮之禍日益擴大。憲宗年間雖然平定了淮西吳元濟勢力,各地藩鎮繼歸順中央,但是卻未能除根。憲宗死後藩鎮割據的局面就又死灰復燃。最後唐朝終於亡在節度使朱溫的手中。唐朝後的五代十國實際上是藩鎮之禍的延續,只是一些藩鎮已經完全獨立而已。唐朝主要的地方官階如下:

州(郡)首领:刺史(太守);
別駕、长史、司馬;
錄事、參軍事;
六曹:司功、司倉、司戶、司兵、司法、司士。
縣:縣令;
縣丞、主簿;
縣尉、錄事、佐史。
鄉:耆老;
里:里正;
村:村正;
保:保長;
鄰:鄰長。

地方行政方面,唐从隋旧,分州县上下二级区划。州级政区多称“州”,有刺史,少数称“郡”,有郡守。县有县令。县级政区以下按照鄉里制设鄉设里。百户人家为一里,由里正管辖;四家为一鄰,由鄰長管辖,五鄰为一保,由保長管辖,五保为一里,由里正管辖,五里为一乡,由耆老管辖。一自然村為一村,设村正。在城市聚居区域以坊代替村,设坊正,和村正同級。在边疆、京畿、军事要塞等重要地区设立都督府,由武官都督兼管多个州郡的军事和民政。

唐朝沿用隋朝制訂的三省六部制,主要機構有三省、六部、一台、五監、九寺。三省即為中书省,門下省,尚書省。此外中央還有掌帝室器物车马的殿中省、掌帝室经史书籍的秘書省、掌宫官内侍的内侍省三个职权较小的省。尚書省為全國最高行政機構,其中枢称“尚书都省”,都省下設立吏、戶、禮、兵、刑、工六部,長官本為尚书令,但因唐太宗曾任尚書令,後以左、右僕射為首。中書省是皇帝頒佈大政文書的機構,長官為中书令,副手為中書侍郎,下有中書舍人六人,此外右散騎常侍、右諫議大夫等諫官。門下省則是審核大政文書之機構,长官为门下侍中,副手为黄门侍郎(又称门下侍郎),下有给事中四人,此外与中书省相似,有左散骑常侍、左谏议大夫等谏官,也有掌符策印玺的符宝郎、掌起居记录的起居郎等官员。由於尚書權力太大,因此後來設立左右僕射代行大權。左右僕射就是宰相。後來,此二職要加同中書門下的頭銜才是宰相。但中书令和門下侍中的名位很高,也不常設。於是,給其他管理加上參議朝政、參議得失、同中書門下三品等頭銜就為宰相。宰相平時在政事堂討論朝政,政事堂會議成為協助皇帝統治的最高決策機構。至玄宗,差遣制成为制度,特点是官位与职位的脱节。官仅代表官位与俸禄的高低,其实际职务完全由皇帝或上官灵活掌握。差遣官官衔中多有“使”字(如转运使、盐铁使、团练使等)。开元末年置翰林学士院,学士参与决奏议疏表,专掌内制,对中书省的权利产生少许威胁。

六部作为尚书省的分支机构,分管各種具體行政事務,按严耕望的研究,六部上承三省所布政令,下传寺监所行方案,主要负责具体事务的规划和监督,而非寺监的具体执行,故而官吏员数远少于寺监。六部有高低之分,吏、兵二部為前行,戶、刑二部為中行,禮、工二部為後行。其中吏部主管全國文官升遷,下設吏部、司封、司勳、考功四司;户部掌管全國土地、民眾、財賦,下設戶部、度支、金部、倉部四司;礼部掌管祭祀,下設禮部、祠部、膳部、主客四司;兵部負責武人選舉、地图、車馬、兵械等事務,下設兵部、職方、駕部、庫部四司。刑部主管律令刑事,下設刑部、都官、比部、司門四司;工部負責山澤、紙筆、屯田、工匠等事務,下設工部、屯田、虞都、水部四司。三省六部制在中國政治史上具有重要地位。

一台就是御史臺,其負責監察中央和地方管理,參與大獄的審訊。其長官為御史大夫,副長官是御史中丞。五監為国子监(掌文教学校);少府監(掌皇家工業生產);將作監(掌國家工程);軍器監(掌兵器製造);都水監(掌水利建設)。九寺有太常寺(掌禮儀祭祀);光禄寺(掌国家宴会);衛尉寺(掌兵器儀仗);宗正寺(掌皇室族譜);太仆寺(掌国家牧政);大理寺(掌刑狱审判);鸿胪寺(掌邦交典禮);司農寺(掌國家倉儲);太府寺(掌國家財政)。此外,唐朝還有三師(太師、太傅、太保),三公(太尉、司徒、司空)等榮譽職務。在盛唐时期還設立過如节度使、觀察使、樞密使等臨時職務,後來則成為定職。

隋代成立的科舉制度在唐初還不完善,朝中的政治仍然被關隴集團所壟斷。到了武則天執政後,她大力起用通過科舉進入朝廷的庶族地主官僚,貴族政治的局面至此開始衰落。玄宗朝以後,世族官僚不復存在,但是科舉士人卻進行牛李党争,這場黨爭持續長達四十年,嚴重敗壞朝政。

唐朝中後期也與東漢中後期和明朝後期成為中國歷史上三個宦官時代。早期,宦官並沒有什麼權力,自唐玄宗時代高力士得寵以來,宦官的地位步步高升,開始直接參與政治。後來伴隨着宦官對兵權的掌握,皇帝的廢立都掌握在宦官手中。這以「甘露之變」表現得最為突出。而在朱全忠誅滅全部宦官之後,唐朝也很快滅亡。顯示宦官已與皇帝形成命運共同體。

唐朝法律分為律、令、格、式四種。律是刑法典;令是指國家對各項制度所做出具體規定(如《戶令》);格是對律令式做出補充修改与对禁令的汇编;式則是各項行政法規(如《水部式》)。《唐律》是根據隋朝《开皇律》经过《武德律》、《贞观律》、《永徽律》三朝修正而來。自唐高祖時代開始制訂,在唐太宗時才宣告完成。至唐高宗永徽年間又對唐律進行全面解釋,写成《律疏》,與《唐律》合稱為《唐律疏议》。後世又稱呼為《唐律疏典》。唐律分十二篇,共五百零二條,刑為五刑。唐朝律法将谋反、谋叛等反对朝廷的行为定作不得赦免或赎免的「十惡」大罪,对朝廷的延续起到保障作用。又有一系列相关土地私有权的条例,维护经济基础。贵族、富人、官僚受到一定的不平等的法律保护,在与庶民触犯同样的法律下可减刑或免刑。

由於初唐时代武力比较兴旺,周边国家比较安分且與初唐的關係比較友好。唐高宗在位后期由於军事转向衰弱,關係也時戰時和反复不定。初唐时代在邊境上設立六個都護府,分別是:安西(640年設立,主要負責天山以南地區的守備);安北(647年設立,主要守衛漠北);單于(650年設立,主要守衛漠南);安東(668年設立,主要守护辽河以东);安南(679年設立,主要守衛今越南北部红河三角洲地區);北庭(701年設立,主要守衛天山以北地區)。

靺鞨人源自肃慎,隋唐交际时分为多部,其中有粟末、黑水、白山、伯咄、拂涅、号室、安车骨七部势力较大。698年,在東北邊境上,粟末靺鞨人大祚荣建立震国。713年大祚荣接受唐玄宗册封为渤海郡王,设立忽汗州,国名更为渤海国。渤海与唐「車書本一家」,之間一直互動頻繁,多名渤海贵族子弟曾到长安学习。726年又在黑水靺鞨之地设黑水都督府。唐朝與新罗關係一直密切。新羅派大量留學生到唐朝學習,其中的崔致远還中了進士。中國的文化也大量傳入新羅。兩國在邊境之間商貿往來非常頻繁。660年至668年间,新罗联合唐军先后灭百济与高句丽,统一朝鲜半岛,而后两国往来更加频繁。723年,旅唐新罗僧人慧超从广州渡海前往印度诸国巡礼,路径波斯、大食、突厥等国回到长安,撰写《往五天竺国传》。新罗留唐学生薛聰,整理吏讀表记法,方便书写新罗语虚词虚字,促进朝鲜文化发展。唐朝东部沿海城市多有新罗人聚集的“新罗坊”和接待新罗人的“新罗馆”,可见境内新罗人之多。

倭国武周时期改称日本,与唐朝来往密切。孝德天皇推行革新,效法唐制,走向中央集权。引入均田制和租庸調制,落实户籍和记账制度,参考《唐令》写成《大宝令》法典,遵照长安城布局规划平安、平城二京。日本先後派遣了十三次遣唐使,每次使團規模都在百人以上,团中除使臣、水手外,还有留学生、学问僧、医师、音声生、玉生、锻生、铸生、细工生等。著名的来唐日本人有留学生吉备真备和阿倍仲麻呂与僧人空海和圆仁。空海著有《文鏡秘府論》与日本的第一部汉字字典《篆隶万象名义》。圆仁寻觅佛法而走遍唐国多个道郡,带回日本大量佛学经文器具。百济艺僧味摩之将在唐学到的荊楚儺舞传至日本,称吴伎乐。日本的文字平假名和片假名也都是分別從中國的草书和楷书部首演變而來。鑒真和尚應日本僧人之邀,曾經六次東渡回日,最後終於成功。他带去了佛经,促进了中国文化向日本的流传以及佛教在日本的兴盛。

契丹源于东胡,自称青牛白马之后。唐初,契丹族部落联盟首领大贺摩会臣服于唐。648年,在羁縻制度下设松漠都督府,以大贺窟哥担任松漠都督兼左领军将军,赐姓李。武则天时期因受到營州都督赵文翙的凌辱而反抗数十年。开元初,松漠都督府得以复置,从此双方睦邻友好百余年,经济文化交流频繁,始终忠服于唐,直至唐王朝灭亡之后,耶律阿保机才在塞北称汗。

東突厥常年南下袭击中原,唐初北方割据政权纷纷联笼突厥抗唐,是唐建国初期的一大边害。高祖太宗积极抵御,貞觀三年(629年)遣李靖、李勣二将分路征讨,次年降服東突厥,小可汗突利可汗投降,大可汗頡利可汗被俘,東突厥汗国覆亡。大量突厥人遷到長安,太宗将降众左右安置在灵武至幽州地区,设羁縻府管辖。東突厥的滅亡與歸順震動了西突厥與西域各國,一些西域小國紛紛改投唐朝,尊称唐太宗為「天可汗」。西突厥西抵波斯,北并疏勒,控制了丝绸之路。唐于640年攻克高昌城(今新疆吐鲁番东南),设安西都护府。802年平定焉耆,806年平定龟兹,安西都护府迁至龟兹,统管于阗、高昌、焉耆、龟兹四镇。唐高宗显庆二年(657年),苏定方、萧嗣业大败西突厥。西突厥最终在唐军的数次打击下覆亡。西域至此成为唐朝的势力范围,期间唐军與當時的另一大帝國大食国开始交往。不过随着时间转移,天宝十载(751年),唐朝在與大食国阿拔斯王朝的怛罗斯战役中失敗,安史之乱後,唐朝勢力也基本退出了中亚地區。

东突厥灭亡后,常年臣服突厥的回紇又受到了薛延陀的控制。647年,回紇联合唐击溃薛延陀。唐高宗永淳二年(682年),阿史那骨咄禄在蒙古高原称汗,东突厥復國(史称后突厥),开始南迁。日频严峻的边患一直困扰武则天。武后通过册封、和亲的手段试图同化南迁的突厥人。唐玄宗天宝三载(744年)回紇又与唐联军灭亡后突厥,回紇建国。貞元五年(790年)更名回鹘。回鹘與唐朝关系一直比较良好,但在安史之乱期间曾趁机敲诈勒索唐朝,并再联合唐军攻入洛阳城之後,大肆烧杀掳掠。直到唐文宗開成五年(840年),回鹘因為统治无道而最终被黠戛斯所灭。被迫迁徒,有的南迁至塞内或近塞,有的西迁至甘州(甘州回鶻)、西州(高昌回鹘)、龟兹(龟兹回鹘)、葱岭融入葛逻禄(黑汗国)。

吐谷浑乃鲜卑支系,南北朝时期西迁至青藏高原东北端。曾被隋炀帝灭亡,隋末战乱年间复国。吐谷浑因夹处于吐蕃和唐两大势力之间,又与吐蕃同居青藏高原上,早年慕容伏允采取亲蕃疏唐的外交政策。唐太宗几进召见未能成功,634年开始派兵西征,次年,大将李靖击败吐谷浑,亲唐的慕容顺继位并对唐称臣。死后,子慕容诺曷钵继位,唐遣送弘化公主和亲。663年吐蕃灭吐谷浑,诺曷钵率众迁至唐安乐州(今宁夏中宁东南)。

在西部與唐對峙的另一大國是吐蕃。吐蕃赞普松贊干布在統一吐蕃後,以强大的武力为由,期间一直向唐朝廷提親。唐太宗貞觀十五年(641年),唐太宗派禮部尚書、江夏王李道宗護送文成公主入藏,松贊幹布到柏海迎接。文成公主将蚕等中原特有的事物带入吐蕃,中国的风俗同时也传入吐蕃,一些吐蕃的大臣改穿絲綢服飾。文成公主的嫁妆中还有一批工匠,这些工匠将中原的建筑形式混入吐蕃的建筑形式,大昭寺是其中代表。吐蕃的历法也参考了唐朝的历法。從此之後,唐蕃兩國维持了二十年的和平,此后军事争夺日渐剧烈。唐中宗神龍二年(706年),由於吐蕃軍事失利,便主動與唐修好,雙方使臣在长安會盟。史稱神龍會盟。唐中宗應允,將金城公主嫁給吐蕃贊普尺带珠丹,但實際上吐蕃也秣馬厲兵,積極備戰。714年,吐蕃向唐朝要求重劃邊界,修改盟書,被唐朝拒絕。兩國因此交戰,吐蕃兵敗,於是又主動求和談判。

唐玄宗開元廿年(732年),兩國再次會盟,兩國決定以赤嶺為界限。734年正式立碑。不久后发生的安史之亂使得唐朝走向衰落,吐蕃趁机大力扩张势力。唐德宗建中年間其要求與唐確立甥舅之國的關係,而不用臣國之禮。783年,兩國在清水會盟,這次會盟基本滿足吐蕃的要求,兩國改以賀蘭山為界。787年,唐蕃又會盟於平涼,吐蕃預備進行劫盟,結果唐朝除了主盟官員外,其餘六十多名官員都被扣押。唐軍死五百多人,被俘一千多人,史稱平涼劫盟。長慶元年,吐蕃內部分裂,國勢衰落,再次請求與唐会盟。後兩國在長安西郊進行會盟,以清水會盟確立的邊界为界。史稱長慶會盟,從此之後,两国关系趋于缓和,但是也被连年战争所困而無力再戰。

天宝七载(748年),南诏統一了西南的雲南,贵州西部,四川最南部和今缅甸北部地區。唐朝與南詔國的關係也是時好時壞。南詔一度長期與吐蕃合作,一同進攻唐朝。但大历十四年(779年)後,吐蕃、南詔聯軍攻唐失敗,南詔軍元氣大傷,吐蕃又遷怒南詔。兩國從此矛盾加深。794年,唐朝與南詔在點蒼山會盟,雙方建立了良好的關係。但是到820年代後,由於南詔王權旁落,兩國又開始爆發戰爭。829年,南詔傾全國之兵力進攻唐朝,在831年一度攻入成都外城廓,但是最後因為害怕唐朝報復而又修好。之後,兩國之間的關係依然是和戰相間,直到雙雙覆滅。

唐朝與东南亚和南亚的真腊(柬埔寨)、诃陵国(爪哇岛)、室利佛逝(苏门答腊岛)、林邑(越南中部)、驃(缅甸)、獅子國(僧伽罗)、天竺(印度)等國家都有经济文化方面的往来。

玄奘西域求法,从天竺携回佛经六百五十七部,还用梵文翻译了《道德经》赠送天竺,回到长安后将所见所闻写成《大唐西域記》。義淨渡海去天竺求法,携回经、律、论约四百部,将西域见闻写成《大唐西域求法高僧傳》和《南海寄归内法传》,都是唐代重要的中外關係史著作。唐代流行的婆罗门曲融合天竺中华乐舞为一体。唐朝的佛教建筑也吸收了天竺的风格。

西域地区有康国、安国、曹国、石国、米国、何国、火寻国、戊地国、史国九个全国以昭武为姓的小国,其使节商人频繁来往于唐。

651年大食与唐始建联系,之后通使多达三十六次。唐军在西域多次与大食交涉,在怛罗斯战役中被击败,大食俘虏了不少中国工匠,包括纸匠,造纸术等技术传入大食。唐初,大食国教伊斯兰教入华,大食的伦理学、语法学、天文学、算学、航海学等也随之传到中国。大食幅员广阔,势力遥及大西洋摩洛哥,唐朝的影响通过大食中介商人间接波及西亚、东非、北非等地。

波斯在唐初受到大食侵略,半世纪便被吞并,大食在波斯境内大肆屠杀,许多波斯非伊斯兰教徒、商人、贵族迁居西域塞内,以及东部沿海城市,从事商业。为后期色目人和回族的一个组成部分。由此,波斯的祆教、景教和摩尼教在唐地推广。从波斯又带来了波罗毬戏(擊鞠),深受唐皇贵族的喜爱。唐末,回回人李珣在《海药本草》中对波斯药物作了系统性介绍。唐朝與中西亞的吐火罗和東羅馬帝國之間也有往來。

唐朝统一中国之后,太宗、高宗、武后先后对外用兵,击败北方疆外和西北方疆外的敵國東突厥与西突厥,在西北佔領高昌、收其地為州縣,重新控制西域,在东北吞滅高句丽和百济,並在白江口戰役击败日本援軍。到玄宗时,唐朝对外扩张达到顶峰,势力甚至远达中亚与新興的黑衣大食(即阿拔斯王朝)相遇。但唐朝经安史之亂后一蹶不振,不仅无力保持前期开疆辟土的成果,还要依靠吐蕃、回纥的军事实力以对抗藩镇的割据势力。虽然唐憲宗時获得过对淮西、剑南等地藩镇的军事胜利,但是无法阻止地方割据的大势。唐朝就此衰落下去。京城长安甚至一度被吐蕃攻陷(763年),西南的南诏也曾联合吐蕃占领过成都(831年)。

唐初继承隋代制度實行府兵制,沿襲北周和北齊的府兵制,不過北周府兵是兵民合籍,隋唐的府兵則由當地丁男抽調服役,是兵民合一的徵兵制度。府兵制的基本單位是折沖府。府分三等。上府一千兩百人,中府一千人,下府八百人。軍府長官為折沖都尉,副職為左右果毅都尉。府兵稱衛士或侍官。軍府隸屬于十二衛和六率。軍府最多時有六百三十四個,其中三成以上駐紮在關中,保衛長安。府兵制是以均田制為基礎的農兵合一制度。兵士廿一歲入軍,六十歲免役,以每戶三丁抽一的比例服役。衛士平時在家生產,農閒時由軍府訓練。其經常性任務是輪流到長安宿衛,叫做番上。戰時則應徵作戰。服役期間免除自己的租調;但口糧和兵器都要自己負責。

府兵制实际上是士兵和农民的结合,減輕國家的負擔。平时为民,战时为兵;兵不识将,将不知兵。战事结束后,士兵回府,将领回朝,降低将领拥兵自重的危险。府兵制的主要缺点在于动员速度慢,用兵时间过长会影响农业,而且免除士兵的税赋对朝廷收入也是一个损失。因此,太宗、高宗及武后时已经采取过临时征募士兵的办法作为对府兵制的补充。太宗時,朝廷直接管轄全國約六百個軍府,一切軍事任務,不管是派往護衛戍京師、地方駐紮或出征,均由這支軍隊執行。然而,為了便於管理,仍然需要設置軍政首長,這也就是「節度使」的由來之一。而且當社會經濟改善時,人民經常會反抗兵役制度。另外也由於國家太平已久,府兵備而不用,政府對之也日益冷漠,其素質自然大為下降。

到玄宗时,朝廷對人口的掌握能力降低,府兵逃散。天宝年间,玄宗采纳張說的建议,正式以徵兵制和募兵制替代已经废坏的府兵制。为了满足他“领有四夷”的虚荣心,透過招募取得的士兵长期驻扎在边镇以进行对外战争,称为“健儿”。这些雇傭兵与土地没有联系,他们只渴望从边境战争中获得收益。边镇将领通过利益关系和部族关系(很多将领和士兵都来自依附的异族)大大加强对士兵的控制,埋下日后戰禍的种子。安史之亂后,唐朝廷在軍事上開始失勢:內有藩镇割据,外有回紇、吐蕃、南诏的入侵。例如唐朝需要借回紇兵來平定安史之亂,763年吐蕃軍曾經佔領長安達十五日,南詔軍一度攻打成都,並於咸通年間多次進侵安南,863年將之佔領,到866年才由唐將高駢收復。唐朝駐守在南詔的士兵不滿,導致庞勋之变。後來黃巢流寇叛亂導致朱全忠和沙陀人李克用的爭戰,各地職業軍人陸續佔據地,甚至自立政權,直至唐朝滅亡後仍未平息,後來五代十國各政權,大致上是唐代晚期藩鎮割據的延續。

唐玄宗時唐朝的勢力與來自現在阿拉伯、新興和信奉伊斯兰教的阿拔斯王朝(黑衣大食)的勢力在包含昭武九姓國、大勃律、小勃律、吐火罗在內的中亚諸國相遇;天宝十载(751年)怛罗斯战役,唐军失敗,經略中亞的進展遇挫,但是接踵而至的安史之亂和藩鎮割據導致華北地區經濟蕭條,使正重整旗鼓的唐朝大軍從此無暇顧及中亞,軍隊必須退回長安一帶平定內亂,致使在往後的一百五十年間吐蕃和回紇勢力興起並佔領原屬唐朝的西半部領土。

唐朝的眾多著名將領中,除了淩煙閣二十四功臣中的將領和郭子仪、李晟及其子李愬、高駢等汉族統帥外,異族將領也佔據重要地位:比較重要的有胡漢混血安祿山、突厥人史思明、百濟人黑齿常之、高句麗人高仙芝、突厥人阿史那社尔、契丹人李光弼、靺鞨人李懷光、突厥突騎施部人哥舒翰、鐵勒部的僕固懷恩、渾瑊和阿跌光進等。

唐朝自武德初至天寶末,其戶口與人口比隋朝低,有可能因為法令不行,戶口時常有隱漏不報,所以史書記載为虚数,其比實際數據尚少。根据《旧唐书》记载,唐武德元年(618年)有一百八十万户;唐武德七年(624年)有二百一十九万户,唐贞观十三年(639年)三百零四万户,唐太宗贞观二十二年(648年)三百六十万户,唐高宗永徽三年(652年)有三百八十万户,據《通典》卷七《食貨》載,到唐玄宗天寶十三载(754年),全國有9,069,154戶,52,880,488人,然唐朝户口统计不严多有隐漏,故大部分学者认为唐朝的人口峰值为八千万左右。

當時全國有十五道,秦嶺淮河以北有人口3000萬。人口最多的是河南、河北兩道及淮北地區,這些地區合計人口接近2000萬。首都京兆府长安人口達到196萬,東都河南府洛阳則有118萬人口。隋唐大運河沿岸的交通樞紐城市魏州也有人口110萬。河東道人口達372萬;關內道有150萬;隴右道人口最少,僅53萬。南方各道中,江南東道人口最多,有661萬。其次為劍南道,有409萬,其中成都府人口就有92萬。江南西道人口亦有372萬,淮南道227萬,嶺南道116萬。人口位居全國之末的是黔中道,僅16萬。

安史之亂时,社會生產遭受毁坏,安史之乱结束后根据史载的户口数只是安史之乱前的三分之一,此后的唐朝户口一蹶不振,估计唐朝中期的户口在四五百万户之间。全国人口分佈格局因此發生重大變化。五代十国時期,南方九國中除了吳和吳越兩國統治者是南方本地人,南汉是早期移民後裔外,其他六國統治者都是唐末北方移民。

唐朝是繁榮強盛的大朝代,經濟的發展與規模有長足的發展。隋朝末年因為戰亂的關係產生大量無主地,使得均田制可以持續推行,對於穩定農業有很大的幫助。而自孫吳、東晉等六朝發展的江南經濟持續提升,已經顯出超越黃河流域的趨勢。而唐朝掌握南北經濟使得經濟十分強盛。自隋唐開始,中國經濟進入更高的發展階段。

唐代農業生產工具比前代有所進步,开元年间发明曲辕犁,還出現新的灌溉工具水車和筒車。唐高祖武德七年(624年)统一全国,在之后稳定的一百三十年之中,僅見於記載的重要水利工程总计一百六十多項。其中著名的如玉梁渠、絳岩湖、安徽鏡湖、山东窦公渠、山西文水、河北三河、四川彭山、湖南武陵等。开元二十八年(740年),总耕地面積達到14,003,862頃(折合今市制为12.197亿市亩耕地)。农业工具的进步以及水利工程的发展促使糧食產量逐年提高。天寶八载(749年),官倉存糧達九千六百萬石。長安洛陽米價最低的唐玄宗开元十四年(726年)時,每斗僅十三文,青州、齊州每斗僅五文。五谷的丰盛直接体现在唐朝前期各地户口与垦田数量的增长。

唐朝中期之后,由於黄河中下游地区在安史之亂期间遭受破坏,而淮河以南地区遭受战争的破坏相对小得多,所以淮河以南地区的经济文化发展水平就在之后的发展之中超越黄河中下游地区,唐朝中期淮河以南的土地大量開墾及大修水利,插秧移植水稻,使江淮的糧產量大幅增加,成为全国重要的粮食产区。白糖的製造始於唐贞观二十一年(647年),宋以後長江以南各省種植甘蔗。种植贩运茶叶的发展形成南方经济的一大收入。饮茶的习俗,从南方传到北方,逐渐普及。南方的茶叶,通过大运河和陆路大批运往北方各地,至吐蕃渤海,甚至远及波斯大食。然因賦稅不足,國用匱乏,贞元九年(793年)正月,鹽鐵使張滂奏請在主要產茶州郡及交通要塞,委派鹽鐵度支巡院設置茶場,由主管官吏分三等定價,每十稅一,在唐朝中期以後成為國家的重要收入,因此在歷史上成為正式建立稅茶之始。

唐代手工業分官營和私營兩種。工部是主管官營手工業的最重要部門,直接管理的機構有少府監、將作監、軍器監。少府監主管精緻手工藝品;將作監主管土木工程的興建;軍器監負責兵器的建造。監下設署、署下設作坊。此外還有鑄錢監和冶監等。官營手工業的產品一般不對外銷售,只供皇室和衙門消費。工人則分為工匠、刑徒、官奴婢、官戶、雜戶等。私營手工業較官營手工業比不發達。唐前期主要手工業有紡織業、陶瓷業和礦冶業。丝、麻为主要纺织对象。河南道的绢,江淮的布都是其中的上等品种。唐朝的丝织品广泛沿用北朝的蜡缬法染色,并先后研发出夹缬、绞缬两种新染色法。织品图案亦受西域胡风影响体现出少许波斯风格。白瓷的精细,唐三彩的数量可以证实当时陶瓷业之发达。唐三彩以黄、绿、白三色为主,表现当时对施釉技术的熟练掌握,虽是随葬物品,但制作精致,取材涉及唐代社会上下的方方面面。金银器制造业汲取西域的一些技术,采用灰吹法达到很高的金银纯度。淮南扬州出产方丈镜、江心镜等上等铜镜。唐朝中期,南方手工業大幅進步,特別是絲織業、造紙業和造船業:民间普及饲养桑蚕,开辟用竹造纸,制造人力脚踏轮船。越州越窑烧制出的秘色瓷是唐朝后期南方陶瓷业的杰出代表。

唐代的城市商品经济处于成長的胚芽时期。长安(雍州、京兆府)、洛陽(洛州、河南府)、魏州、清河郡、齐州历城(济南郡)、睢阳(宋州)、楚州、蘇州、涿郡(幽州)、揚州(江都、广陵城)、成都(益州、成都府)、廣州、晋阳(并州、太原府)等都是一定地域內的商业中心。唐朝國內交通在当时世界上是十分發達的。陸路交通以長安為中心,道路遍佈全國。水路交通則是以洛陽為中心的南北大運河為主。全國共有驛站一千四百六十三所。其中陸驛一千二百九十七所,水驛一百六十六所。商人用于存放商物的邸店因其利润之高,在交通枢纽周边发展开来。唐朝中期开始,由于大批官僚士族与工匠南迁,长江流域商業城市發展快速,国家的经济财政亦仰赖南方的补给,当时有「揚一益二」的說法;而江南最大城市、江南东道治所苏州的繁华程度在中唐时期已逐渐开始超越扬州和洛阳,在全国仅次于长安,成为整个中国南方唯一的、最高等级的州——雄州,有「甲郡标天下」之说,即所谓“当今国用,多出江南。江南诸州,苏最为大”;此外杭州、湖州等地的经济也得到较快发展。而坊市分開的制度在苏州、揚州等商業城市被打破,還出現夜市。

大唐是世界上最早发行纸币的国家,飛錢是世界上最早的纸币。這是世界上最早的紙幣雏形,也是近代世界各国学者所公认和认可的最早纸币。唐代大城市中出現櫃坊和飛錢。櫃枋經營錢物寄付,在櫃枋存錢的客戶可以憑書貼(類似於支票)寄付錢財。這些都說明商業在唐朝中期的繁榮。唐末,因為黃巢之亂和藩镇戰爭,户数锐减,社会经济规模再也未能达到开元盛世的水平。

唐代,海外貿易開始興盛,西元八世紀下半期,從廣州經由麻六甲海峽進入印度洋,抵達印度、錫蘭、再西入波斯灣、亞丁及紅海地區的航路。將通往西方的海道與往新羅及日本的海道連接起來,唐代海外交通所能抵達的範圍,已及於新大陸發現之前舊世界的大部分地區,中東商人如猶太人、波斯人以及阿拉伯人紛紛東來。中國沿岸的交州、廣州、泉州、明州(今浙江宁波)、揚州等城市,因與蕃舶互動頻繁,如雨後春筍般興盛起來,成為重要的對外貿易港口。為因應海上貿易的新形勢,唐代還特別設置「市舶司」,用來管理蕃舶的進出以及徵稅事由。海外貿易的數量,自此不斷成長。

唐武德四年(621年)七月,“廢五銖錢,行開元通寶錢,徑八分,重二銖四絫,積十文重一兩,一千文重六斤四兩”,確立國家鑄幣的法幣地位。與此同時,又繼承魏晋南北朝時期以絹帛為貨幣的傳統,實行“錢帛兼行”的貨幣制度——錢即銅錢,帛則是絲織品的總稱,包括錦、繡、綾、羅、絹、絁、綺、縑、紬等,實際上是一種以實物貨幣和金屬貨幣兼而行之的多元的貨幣制度。

初期,社會經濟以自然经济為主,商品经济處於復蘇階段,水準很低。在這種情況下,錢帛兼行的貨幣制度較好地適應小額商品交易的需要。但隨著貞觀末期,尤其是唐高宗、武后及唐玄宗時期商品經濟的繼續發展,錢帛兼行的貨幣制度逐漸暴露出其落後的一面。首先表現在絹帛作為貨幣因體大物重、不便分割、難於運輸儲藏等缺點開始不受市場歡迎,絹帛作為貨幣的職能趨於衰退,商品交易趨向喜歡使用更高一級的銅錢作仲介,提出增加流通中銅錢投放量的要求,然而唐王朝的官營鑄幣不能滿足這種要求,於是造成流通中銅錢短缺的日益加劇,又進而引發嚴重的銅錢的私鑄和濫鑄,造成物價波動、貨幣流通不穩定以及經濟發展的混亂,對國家財政制度造成威脅。

唐政府不斷出臺嚴厲打擊私鑄和濫鑄等的法令,並禁斷使用惡錢,但是由於銅錢供應量嚴重短缺,幣值不斷上升堅挺,私鑄和濫鑄有暴利可圖,所以成效並不理想。

唐朝戶籍制度沿襲隋朝,行三等戶制。前期的賦稅制度,大提承襲隋朝,於624年頒行均田制與租庸調制。均田制是政府授田給人民而徵其租賦,分成公田與私田。身死後公田繳還政府重新分配,剩下可以傳後的私田即「永業田」。由於隋末民变產生大量無主土地,所以唐朝前期有充足的土地推行。除了人民之外,政府官員與王公貴族也各有額定的永業田。相较隋朝,唐朝对土地的買賣宽松许多,但仍有嚴格的限制。租庸調制方面,租是授田男丁每年繳固定的栗或稻,庸是每人每年要為國家服的勞役,調是每丁按照當地特產繳納絹麻之物,如果不產絹麻可用銀兩代替,庸和調也可用一定數量的絹免役。這種制度精神在於政府為民置產,其因其產而繳稅,即沒有重徵累民的問題,又可以防止兼併之風,自然是一種良制。唐朝前半叶,户税逐年上升,唐高宗時约收户税十五万余贯,至唐玄宗時已高达二百多万贯。

均田制與租庸調制對人民的經濟壓力不會很大,但是人口流動不能過大,戶籍和田籍需要齊全清楚。如果政治敗壞,田地過度兼併,閒田過少,人民過度避稅,這兩個制度就會走向瓦解。武周末年均田制开始形同虚设,政治漸不以往。加上突厥、契丹連年入侵,人民逃避徭役,逃亡者漸增。唐玄宗天寶後期,不課稅的戶約占全國總戶三分之一;不服役的人口約佔全國人口六分之五,逃税情况普遍存在。安史之亂後,戶口逃匿者增加,租庸調制無法繼續實行,所以在唐朝後期出現兩稅法。唐德宗時期,宰相楊炎制定兩稅法,並且廢除其餘名目的租稅。兩稅法即政府以當地現有的男丁與田地數為依據,劃分等級,規定分兩次於夏天、秋天納稅。而商人是以貨物總值的三十分之一,於所在的州縣納稅。其稅額,原本用錢為單位,到唐穆宗時以布代替。這樣,官僚、貴族、地主和商人都要合理納稅,減輕平民的負擔,也增加政府的收入。兩稅法雖然簡化賦稅方式,但是授田制度被廢除。使得戶籍持續陷入混亂,田地兼併的問題也都沒有解決。此後中國的賦稅制度,一直沿襲兩稅法的原則,沒有再恢復授田制度。

兩稅法未能阻擋官僚、地主、大商人利用特权手段减税、免税、逃税。唐朝后期随着物价上升,两税制对平民的剥削愈来愈严重。唐朝后期,为解决财政拮据的局面,先後對盐、铁、酒、礦等實行专卖制度,並且課茶稅與關稅等。结果导致物价飞腾,民怨四起,民间贩卖私盐者不在少数。而盐铁专卖制度也是黃巢之亂的直接原因之一。

唐代前期思想繼承魏晋南北朝的儒学,例如孔颖达编著的《五经正义》,五经正义中的思想大多由汉晋大儒完成,尤其是郑玄的功劳最大。唐初与明初比较类似,国家在做的是执行前哲的思想。唐朝中期以後,思想上的重大改进发生,韩愈、柳宗元、李翱、刘禹锡等人的思想创见,承前启后。还有,杜甫、白居易等人的思想价值同样不能被忽略,他们不仅仅是诗人。后世所谓经学,严格意义上应该叫做“汉晋唐经学”,后世所谓理学,应该叫做“唐宋明理学”。

韩愈和李翱的作品突出体现唯心主義思想,而柳宗元和刘禹锡更是唐代唯物主义思想的代表。韩愈在他著作《原道》和《原性》中复古崇儒、驳斥佛道,认为僧道不顾及生产,浪费社会财富,僧尼道士应当回乡还俗,焚烧佛经咒文,将寺庙观宇改为民居。他推崇孔子在《论语》中道述的道德观念,以其作为日常伦理的标准。他认为天生人性,并可划分为上中下三品。李翱在《复性书》发展孟子的性善论,认为人之性皆善,但在日常生活中受到喜怒哀乐之情的干扰,使得性无法发挥,要求恢复人的善性克制人的情欲,所谓“复性”。韩愈和李翱的思想是宋代理学的先声”。

柳宗元在他的《天说》、《天对》、《封建论》等哲理文章中指出人命与天命无关,天即自然元气,无法对人世赏功罚过,“功者自功,祸者自祸”,人的遭遇纯属自己创造。刘禹锡发展荀子的天论观点认为宇宙之内竟是物质,天本身同样是物质,虽有客观规律存在,但不能影响人事。他认为唯心理论的产生是因为人世间是非颠倒,人无能胜天,所以宣扬天命理论。

唐朝文學成就以詩歌最為發達。清人所編《全唐詩》共收錄兩千兩百多位詩人的四萬八千九百多首詩,這還不是全部。唐初詩人以「初唐四傑」最為著名(王勃、楊炯、盧照鄰、骆宾王)。盛唐時期詩人可分為以王维、孟浩然為代表的田園派和岑参、王昌龄為代表的邊塞派。其中集大成者為「詩仙」李白和「詩聖」杜甫最為出名。李白的詩,飄逸灑脫,感情澎湃,充滿浪漫主義的色彩。而杜甫的詩則更多體現現實主義之情懷。中唐時期最卓越的詩人是白居易,他的詩通俗易懂。此外還有元稹、韩愈、柳宗元、刘禹锡、李賀等。晚唐詩人以李商隱和杜牧最為出眾,被稱為「小李杜」。後世宋、明、清雖仍有傑出詩人出現,但總體水準都不如唐朝詩人,唐詩成為中國古詩不可逾越的巔峰。

散文方面,六朝以來,文壇盛行駢文這種文體形式,駢文講究聲韻、對偶、典故,辭藻華麗,以四字句和六字句為主。在唐初十分流行,以初唐四傑最為著名,但這種文體到唐朝時顯得形式僵化,內容空洞,故到了天寶年間,古文逐漸興起。古文運動在名義上是要恢復先秦兩漢的散文,實際上是要文章更有內容,也就是「文以載道」。韓愈是唐宋八大家之首,他的散文氣勢磅礴又思想深刻,號稱「文起八代之衰」;不過唐代的古文運動在韓柳去世後就逐漸衰退,唐末駢文又再度興起。

傳奇是中國的一種古典小說形式,出現在隋朝,興盛於唐朝。著名的傳奇包括《柳毅傳》、《鶯鶯傳》、《南柯太守傳》、《枕中記》和《長恨傳》等。有的傳奇在後代還被改編為戲劇和白話小說。唐朝變文在中國文學史上也有重要地位。所謂變文起初是指佛教僧侶宣傳佛教講唱佛經的底本。最初變文僅限於佛教經典,後來則開始講唱其他故事,講唱的人也不限於僧侶。變文對傳奇和後世的說唱文學都有很大影響。

唐朝史學開創國家正式開館修史這一風潮。贞观年间史館奉詔所修的正史有《晋书》、《梁书》、《陈书》、《北齐书》、《周书》、《隋书》六部。加上史家李延寿私撰的《南史》和《北史》,合計廿四史中有八部出在唐朝,占總數的三分之一。官修史书成书较快、收录详尽,丰富国家的历史档案,但因统治者直接控制修史工作,多少会根据编书时的政治需求出现删减夸大的行为。此外,唐朝還有杜佑扩寫《政典》的政書《通典》与刘知几的修史專著《史通》等。杜佑尤其重视财政经济与典章法令制度,认为历史多有现实政治中可以采纳效仿之处。刘知几强调史学家在修史的过程中要有独自创新的评论见解,是为中国历史理论学的开端。

道教遵奉老子李耳为本教創祖,由於唐朝皇帝乃李姓,因此道教自唐初就被规定居于佛教之上,在唐代上流社會也很流行。唐朝李氏家族认为其为老子之后,唐高祖特別在終南山建太和宮以祭老子,唐高宗追尊老子為太上玄元皇帝,並詔令王公百官研習老子的《道德經》。武则天上奏请令王公百官都学《道德經》,每年依《孝经》、《论语》例考士人。玄宗、代宗亦大力提倡道教,使其在中国的地位达到顶峰。玄宗亲自注解《道德经》,开元二十一年(733年)還在科舉考試中增設道舉與儒家經典,同列《明經》科舉人策試教本,明顯有將道家列為國學,頗有與儒家經學齊足並馳的意義。据《新唐书·百官志》记载,开元年间全国有宫观1687所,其中女观550所。当时主要有清经法派和正一派二宗,主要人物有王远知、潘师正、司马承祯、吴筠、张果等。道教之所以受皇室青睞,主要原因是他们多有炼丹,以求長生不老,但其成份可能有毒,故唐朝的许多皇帝亦因信之服用而丧生,例如唐武宗、唐宣宗。

宗教在社會上的地位與影響力,唐時可謂最高。唐朝時期佛教的主要宗派有天台宗、華嚴宗、法相宗、淨土宗和禅宗。唐代佛教的一大轉變,由出世轉向入世。天台宗奉《法華經》,故又稱為法華宗。華嚴宗奉《華嚴經》,參與政治較多。淨土宗則易於入門。禪宗分為南北二宗,北宗創立者是神秀,他主張漸悟說。南宗創立者是惠能。唐武宗因崇信道教,對佛教採取高壓政策,史稱會昌毀佛,使得除禅宗南宗等少數宗派外,其他佛教派別從此一蹶不振。佛教的政治地位虽不及道教,但其传播范围之广、经济实力之大、信徒人数之多都远在唐代道教之上。

除了佛道二教外,當時還有伊斯兰教、景教、拜火教與摩尼教等外來宗教,後三者合稱「唐代三夷教」,但社会影響力较小。唐代对外来宗教相对宽容,期间多有外来教士传授教法,其中以伊斯兰教和景教为最大。伊斯兰教是唐的敌国大食的国教,称作“大食法”。651年,先知穆罕默德的舅父沙德作为使节两次出使中国,得到高宗接见以及传教的准许,在广州筑建怀圣寺。以后的两个多世纪,伊斯兰教随着西域商人沿途陆海两条丝路入唐,在中国发展壮大。景教通过同一个路线传入中国,因被误认是大秦国(拜占庭帝国)的国教,所以称作“大秦景教”。638年为唐朝所认可并得到政府资助在长安兴建大秦寺,并立下石碑。然而會昌五年(845年)唐武宗大舉廢佛,因此景教也同時被禁,此後幾乎在中國絕跡。

摩尼教為西元242年創建於波斯國沙普爾一世時的摩尼,安史之亂後,回紇勢大,摩尼教憑著回紇的庇蔭下在中國傳教,不過後來受會昌毀佛影響,摩尼教勢力遭受沉重打擊,不過並未斷絕,該教信徒到了政治控制力較弱的南方並漸與其他宗教相結合,在今天的福建建立傳教據點,流傳到東南浙、閩沿海地區,從此轉為民間秘密宗教,也影響日後的明教、彌勒教、白蓮教等教派。

中国历史上第一个状元、三元及第,都诞生于唐朝,即武德五年(622年)状元孙伏伽(一说651年的颜康成), 建中二年(781年)三元状元崔元翰。 唐朝的學校以官辦為主。中央設國子監,下轄六學,為國子學、太學、四門學、律學、書學、算學。這些學校主要招收貴族官僚子弟,也招收少量平民子弟。由博士与助教授课,學生稱生徒。國子學、太學、四門學传授以九經为主的儒学经典,按生徒家中官位的高低分级招收。三品以上官员的子孙可入國子學,有生徒三百余人;五品以上官员子孙可进太学,有五百余生徒;四门学兼收五品以下官员及庶民子孙,生徒多达千人。律學、書學、算學教授实用学问,收纳八、九品官员及庶民子弟,名额限于十余人。地方設立州学、縣學,每校有學生十来人。

学校旨在培养官僚书吏,亦為科举考試服務。名望好的學校保送生徒參加科舉考試。科舉制度在唐朝進入逐渐完备期,分為常舉和制舉兩種。常舉每年舉辦考試,科目有明經、進士、明法、明書、明算等。此外還有秀才、道舉、童子、一史、三史等科目。常舉的應考舉子有兩個來源,一是保送的生徒;二是鄉貢選拔出來的自學者。應考舉子主要集中在明經和進士兩科。明經科主要考試儒家經典,難度較低。進士科主要考詩賦和政論,難度高,但其是主要的高官晉身之階,即“昔日齷齪不足跨,今朝放蕩思無涯。春風得意馬蹄疾,一日看盡長安花。”明經科的錄取率約為十分之一二,進士科不過百分之一二。時有諺曰:三十老明經,五十少進士。而制舉則是臨時考試,是為了網羅非常人才,不常舉行。因為科舉制度比較公平且機會相等,平民得以晉身,所以成為士族末落、門第消融的起點。

科举制度除外,还有门荫和流外入流两种入仕渠道。门荫即晚辈承接前辈职务。流外入流指九品以下的官员通过考验,升职为品官。唐初,以此二途入仕的为主流,后来唐太宗大力推广学府,科举制度逐渐推行。唐代教育的普及,削弱了传统世族的特权,加强了有效的行政管理,扩大了政权的社会基础;尤其是唐朝后期黄巢之乱对门阀士族的沉重打击,在后来的宋代中科举制度真正得到完善。盛唐时期,东亚多国遣送其贵族子弟来唐入学,又将儒家文化传授国外。

由於吸收了西域特徵與宗教色彩,唐朝藝術與前後朝代都迥然不同。唐初的阎立本、閻立德兄弟擅畫人物。吴道子則有「畫聖」之稱呼,他兼擅人物、山水,並吸收了西域畫派的技法,畫面富於立體感,有「吳帶當風」之說。张萱和周昉以畫侍女圖為主,他们的著名作品有「搗練圖」、「虢國夫人游春圖」和「簪花仕女圖」等,进一步发展人物画。魏晋南北朝时期,山水风景多为衬托人物主题的配景,而隋唐以来,山水风景成为主题,出现山水画这个重要分支。当时分南、北两派。詩人王维擅長水墨山水畫,是南派的代表,蘇軾评他「诗中有画,畫中有詩」。北派画家李思訓善用青绿画金碧山水。又有曹霸、韓幹善画马,韓滉善画牛,薛稷善画鹤,边鸾善画孔雀等。

唐朝的壁畫事業特別發達。莫高窟與墓室壁畫都是傳世精品。唐朝的雕刻藝術同樣出眾。敦煌、龍門、麥積山和炳靈寺石窟都是在唐朝時期步入全盛。龙门石窟的盧舍那大佛和四川乐山大佛都令人讚歎。昭陵六駿、墓葬三彩陶俑都非常精美。其中雕刻家楊惠之被稱為塑聖。唐朝時期,書法家輩出。欧阳询、虞世南都是唐初著名書法家。歐陽詢的楷書筆力嚴整,《九成宮醴泉銘》为其名作。虞世南楷書字體柔圓,代表作品有《孔子庙堂碑》、《汝南公主墓志》、《摹兰亭序》等。颜真卿和柳公權是唐朝中後期的著名書法家。顏真卿的楷書用筆肥厚,內含筋骨,勁健灑脫,其代表作有《多寶塔碑》、《颜氏家寶庙碑》、《麻姑仙坛记》等;柳公權的字體勁健,代表作有《玄秘塔碑》,世人稱顏柳二人書法為「顏筋柳骨」。張旭和懷素則是草書大家,後者奔放揮灑,深具個人風格及藝術性。

唐朝音樂舞蹈發達。唐太宗平高昌得高昌乐,并入原有的九部乐成为十部乐:燕樂、清商樂、西涼樂、天竺樂、高麗樂、龟兹乐、安國樂、疏勒乐、康國樂、高昌樂。唐高宗以後,十部樂開始衰落,音樂家開始研究新的樂舞,各部乐间的区别逐渐消失,至玄宗朝撤销。玄宗本人就是音樂家,爱好亲自演奏琵琶、羯鼓等多种乐器,擅长作曲,作有《霓裳羽衣曲》、《小破阵乐》等百余首乐曲;他非常重視雅乐事業,将十部乐分为坐部伎(坐在堂上演奏)和立部伎(立在堂下演奏),曾經親選坐部伎三百人,號為「皇帝梨园弟子」,李龜年和永新娘子都是名噪一時的歌唱家。唐朝的舞蹈則是以健舞和軟舞最為出名。健舞因其节奏明快、雄健豪爽而得名,有《阿辽》、《柘枝》、《拂林》、《大渭州》、《黄獐》、《阿连》、《剑器》、《胡旋》、《胡腾》、《杨柳枝》等多种。软舞即文舞,优美柔婉,节奏舒缓,有《垂手罗》、《回波乐》、《兰陵王》、《春莺啭》、《借席》、《乌夜啼》、《凉州》、《绿腰》、《屈柘枝》、《甘州》等。著名的舞蹈「七德舞」、「上元舞」、“九功舞”合称“三大舞”,流行于宫廷。舞蹈家則有楊玉環、公孫大娘、谢阿蛮等。晋朝永嘉之乱后西域舞乐东传中原,与华夏舞乐融合两个多世纪,至唐代已有很强的胡风特色。多种健舞软舞都采用一种昂首望上,双脚原地急转如旋风的动作,因来源西域,谓之“胡旋”。唐代散乐多含杂技,统称“百戏”,包括浑脱、寻撞、跳丸、吐火、吞刀、筋斗、踢毯等项目。


唐朝科技相对于前代有明顯進步。在中国历史上有大量的科技发明,所谓的四大发明之中有两个都诞生于唐朝,即火药和雕版印刷。尊称藥王的孙思邈撰写的《千金要方》和补本《千金翼方》,论及药物之本、诊治之诀、针灸之法、养生之术,都是不可多得的醫書。《新修本草》是中国最早的一本国家官修药书,成书于唐高宗显庆四年(659年)。天文學家僧一行在世界上首次測量子午線的長度,他還與梁令瓚合作,铜铸製成黄道游仪与水運渾天儀。他在《大衍历》历书中运用二次差内插法并创新近似三次差的内插公式,为王恂等后人奠定基础。初唐数学家王孝通著于武德九年〔626年〕的《缉古算经》在世界上首次系统地创立三次多项式方程,对代数学的发展,有重要意义。李淳風等人修订《算经十书》是唐朝算学的重要成果。

唐初大型地理志书《括地志》共550卷,内容丰富,对后世的地理研究影响深远。贾耽的《海内华夷图》绘有唐近邻的数百国家。此外还有李吉甫著的地方誌《元和郡縣圖志》,杜佑撰寫的政書《州郡典》,樊绰介绍云南南诏国的《蛮书》等。唐在隋大興城的基础上扩展修建首都長安城,東都洛陽規劃同样规划嚴整,规模宏大,是中国历史都城中规划最为严谨端正的两个。长安城在盛唐年间极盛时人口达到80—100万,成为当时世界上最大的城市,也為後世留下城市規劃的樣板。當時周邊國家的首都,如:日本平安京、新羅金城、高句丽平壤和渤海國上京龙泉府都是仿照長安建造。大明皇宫占地广大,现今遗址范围相当于明清紫禁城总面积的三倍之多。

唐朝的木結構建築規模雄渾,氣魄豪邁,建筑流程进一步定型化,提高施工速度。佛塔形式也融合中國與印度的造型,顯得千變萬化,多種多樣。868年,中國《金剛經》的印製是世界上已知最早的雕版印刷。在成都和敦煌都发现过雕版印制的《陀罗尼经咒》。雕版印刷为五代以后书籍的大量发行和普及创造条件。中國的造紙、紡織等技術在751年的怛罗斯战役之中传入大食国,之后在12世纪传入西班牙,到13世纪传入意大利,到14世纪初叶传遍整个欧洲。646年,甘蔗熬糖法也从摩揭陀传入唐朝。

唐代社會,雖然士族的勢力被削減,但仍然不是一個平等的社會。《唐律》中也明訂,人分為「良」、「賤」兩大類,賤民只能與賤民結婚;地主殺害部曲最多求刑一年,而部曲殺害地主必處斬。雖然科舉制度實行,但由於世族的生活條件較為優渥,其子弟的文化修養也就跟著較高,不論是否參加科舉,進入仕途都不是非常困難;唐代宰相出身世族者也就不在少數。唐代進士選拔,另有一些社會公評的含義,防弊措施並不嚴格,常有考生向主考官請託,自我吹噓的情形,但當時人並不視為舞弊,所以錄取進士的,有許多是權門子弟;而才氣縱橫的杜甫,兩次考試都落榜。

唐代是“胡風”盛行的時代。所謂“胡风”,特指流行於唐朝社會各階層的種種並非漢民族原有的社會風習而言,其中主要有當時從北方遊牧民族和西域等地傳來的風俗,也有由五胡十六国时期南下的遊牧民族遺留的社會風俗,諸因素共同作用的結果,形成唐朝胡風盛行的局面。如“胡樂”、“胡服”、“胡食”等在長安城是極其盛行的。西域外族服饰文化对唐朝服饰影响巨大。唐代元和以前,除百官公服外,士女的常服大都随意穿着,故有穿胡服的风气,至玄宗时达于极盛。元和以后,衣服渐尚宽长,此后,唐人渐厌胡风改从汉制,于此可以看出,唐人已颇有复古的倾向。

唐代妇女的地位较高,在服饰中也有体现。贵族及宫廷女子多为半裸胸的宽松罗裙。裙腰系得较高,在腰腋之间。歌女服饰染色醒目绚丽,贵族染色富丽高雅。按领子款式分为圆领、翻领、方领、斜领、直领和鸡心领等。隋文帝开创穿黄龙袍的习礼,唐高祖武德年间令臣民不得僭服黄色,黄袍成为皇室专用之服。

%% -*- coding: utf-8 -*-
%% Time-stamp: <Chen Wang: 2021-10-29 15:27:43>

\section{高祖李渊\tiny(618-626)}

\subsection{生平}

唐高祖李渊(566年4月7日-635年6月25日),字叔德,陇西成纪人,唐朝开国皇帝及奠基者,618年6月18日-626年9月4日在位共8年 。玄武门之变后不久禅位于唐太宗,称号“太上皇”。

据《旧唐书》高祖以周天和元年生於長安,太宗在貞觀八年三月甲戌(初二)上壽,推其应为天和元年三月初二(566年4月7日)生。据《册府元龟》记载,李渊以北周天和元年十一月丁酉(566年12月21日)生於長安,似有誤。父親李昞,北周安州總管、柱國大將軍,襲封唐國公。李渊七岁,父亲去世,李渊世袭为唐国公。

581年,隋文帝逼迫北周静帝禪让,李渊任千牛備身(皇帝的禁衛武官),因與皇室的姻親關係,589年隨隋文帝滅陳,后累任谯、陇、岐三州刺史,荥阳郡太守。604年,隋文帝駕崩,遷樓煩太守,隋煬帝大業元年(605年)遷殿内少监;大业二年,除郑州刺史;大業九年(613年),遷衛尉少卿。是年隋煬帝征高句麗,李淵在怀远镇(今辽宁朝阳附近)負責督運。楊玄感之亂,煬帝詔李淵為弘化留守,知關右諸軍事。可見李淵與隋朝宗室關係密切,參與了朝廷的衆多大事,他也趁此機會招納人才,引起煬帝猜忌,李淵懼而以酗酒、受賄等行爲“自污”。

大業十一年(615年)李渊任山西河东郡慰抚大使。大業十二年(616年)升为右骁卫将军。大業十三年(617年)正月遷太原郡留守,7月殺郡丞王威、武牙郎將高君雅,打着勤王定乱,迎回隋天子的旗号正式开始於晉陽縣起兵。晋阳起兵即得到李氏宗族及姻親的響應。他一邊招降叛軍、流寇,一邊派親族迅速進兵,並且借助突厥始毕可汗的500骑兵进攻隋大兴城,于12月攻克。

他拥代王杨侑做傀儡皇帝,遥尊隋煬帝為太上皇,受假黃鉞、使持節、大都督內外諸軍事、大丞相,進封唐王,不久進位相國,加九錫。

义宁二年(公元618年6月18日),隋炀帝在四月被叛军所弑后,李渊逼迫隋恭帝禅让称帝,建立唐朝,隋朝灭亡。李渊开始着手消灭其他原隋朝领土上的諸侯、軍閥,開展唐朝統一戰爭,他的儿女李世民、李建成、李元吉和平陽昭公主的征讨下,用了七年时间,先后消灭薛仁果、薛举、李轨、宋金刚、刘武周、王世充、窦建德、萧铣、杜伏威和梁师都等割据势力。

最后一个梁师都是在贞观二年(628年)被平定的,此时他早已经将皇帝位让给儿子李世民了。同时他又利用东突厥和西突厥之间的分裂,维持了北部的边界,这是他有力量能够夺取中原的主要条件(参见唐与突厥的战争)。

在官制上李渊于武德七年(624年)颁布了唐的官僚制度,基本使用了隋的制度。在农业方面他于武德七年(624年)颁布均田制;对稅捐他也做了调节,减轻了受田农民的负担。在法律上他废弃了隋煬帝的许多苛政,颁布了武德律。李渊对唐朝的措施,为唐太宗“贞观之治”打下了非常重要的基础。

高祖在位期間,没有能尽早確立及处理好繼承人問題,雖然他一早立長子李建成為太子,他眼見皇太子李建成与各儿子明争暗斗,他卻一再纵容,图讓众子互相制衡並未加以控制,同时李世民拥护者众多,导致太子李建成、李元吉和李世民之间的矛盾激化。

最终李世民先下手为强发动政變,史稱玄武门之变,李建成、李元吉被李世民所殺,李世民的軍隊控制了長安,声称二人系作乱伏诛,加上群臣的支持和擁戴,李渊被迫将军国大事交由李世民处理,而李建成、李元吉不但被追废为庶人,诸子也遭诛杀,皆除宗籍。

三天后,高祖立已掌握实权的李世民为皇太子,三個月後更将帝位內禪给李世民,自己退位为太上皇,贞观三年,移居弘义宫。

貞觀九年五月初六(635年),李渊去世,享年六十九歲。死后谥号太武皇帝,廟號高祖,葬在献陵。唐高宗上元元年(674年)八月,改上諡號為神堯皇帝。唐玄宗天寶十三載(754年)二月,上尊號神堯大聖大光孝皇帝。

根據《舊唐書》和《新唐書》,兩者都聲稱李淵是受到兒子李世民的唆擺才起兵反隋。根据这两部史书的记载,李世民通过裴寂把李渊引进隋炀帝的晋阳行官,灌醉了李渊,使得李渊酒后与宫女发生关系,迫使李渊起兵。

反隋應該是李淵本人的意思。孟宪实认为无论从政治影响、军事经验、经济实力还是从社会地位来比较,李世民都无法与李渊相提并论。即便是有人愿意结交李世民,也是因为看重了李世民背后的李渊。李世民要结交那些非法的豪杰大侠,没有背后李渊的政治经济资源几乎是不可能的。晋阳起兵的历史真相是,以李渊为首的军事政治集团,看到隋朝大势已去,于是开始谋划夺取最高权力。这个集团的核心人物当然是李渊,作为李渊的次子,李世民不过是李渊手下的一员得力干将而已。因为父子关系,李渊信任李世民,李世民很早就参与了晋阳起兵的谋划,并且承担某些具体任务。但是,只有李渊才是主谋,这个地位任何人都无法取代。”。

李渊的祖父李虎曾為尚书左仆射,封隴西郡公,賜姓大野氏,與宇文泰等共八位柱国大将军并称八柱国。宇文泰的家族建立北周後,李虎已經去世,获封陇西郡公。父親李昞,北周安州總管、柱國大將軍,襲封陇西郡公,于550年加封唐国公,是为唐仁公,追封李虎为唐襄公。李渊七岁丧父,襲封唐国公。

李渊是隋炀帝杨广的姨表兄弟。北周明帝的明敬皇后、李淵生母元貞太后、隋文帝的文獻皇后分别是西魏八大柱国之一独孤信的長女、四女、七女。

唐朝皇室自称出自隴西李氏,為李暠第二子李歆的后裔,多称陇西狄道(陇西郡狄道县)人,亦可称陇西成纪(陇西郡成纪县)人。

由于唐朝皇室以老子后裔自居,崇尚道教,唐初武德九年太史令傅奕上疏抬道抑佛,引发佛道论争。和尚法琳作《破邪论》《辨证论》反对傅奕。法琳反对唐朝皇室为老子李耳后裔之说,亦与隴西李氏无关,而是拓跋氏后裔,法琳因而触怒唐太宗,被流放益州而死。劉盼遂與王桐齡考據認為李淵家族應為拓跋氏後裔。劉盼遂之後雖取消了自己的观点,但其學說仍引發學界討論。

陳寅恪依据唐祖陵在今河北省境,認為李氏出身赵郡李氏,原為漢人寒門。在崛起之後,分別宣稱自己是隴西李氏或拓跋氏的後代,以抬高身份,实际上不是赵郡李氏破落户,就是广阿庶姓李氏的假冒牌。

朱希祖经考据认为李熙与李买得不是同一个人,李熙曾作为强宗子弟镇戍武川,后卒于武川。其子李天锡为避六镇兵乱,携父遺骨南迁于赵郡广阿,因以为家,不久亦卒。其子李虎将父祖合葬,即所谓唐祖陵。李氏并非出身赵郡李氏,而确系为隴西李氏。

李淵善於騎射,與其妻竇皇后的成親曾經為一時佳話,竇氏未嫁之時為貴族,樣貌豔麗,明艷照人,故其父北周大將竇毅不肯輕易許嫁女兒。故而舉辦射箭之賽,比武招親,要求來求親的公子們,在一「雀屏」(繪有孔雀的屏風)上射箭,以射中孔雀為標準,李淵憑藉準確的目力與勁道,於數步外射箭,竟然成功射中「孔雀的眼睛」,而娶得竇氏,這段佳話流傳後世成為成語“雀屏中選”。今陕西省西安市碑林区,一街道名为“窦府巷”,以窦姓府第在此而得名,亦有传说此“窦府”即窦毅之府。


\subsection{武德}

\begin{longtable}{|>{\centering\scriptsize}m{2em}|>{\centering\scriptsize}m{1.3em}|>{\centering}m{8.8em}|}
  % \caption{秦王政}\
  \toprule
  \SimHei \normalsize 年数 & \SimHei \scriptsize 公元 & \SimHei 大事件 \tabularnewline
  % \midrule
  \endfirsthead
  \toprule
  \SimHei \normalsize 年数 & \SimHei \scriptsize 公元 & \SimHei 大事件 \tabularnewline
  \midrule
  \endhead
  \midrule
  元年 & 618 & \tabularnewline\hline
  二年 & 619 & \tabularnewline\hline
  三年 & 620 & \tabularnewline\hline
  四年 & 621 & \tabularnewline\hline
  五年 & 622 & \tabularnewline\hline
  六年 & 623 & \tabularnewline\hline
  七年 & 624 & \tabularnewline\hline
  八年 & 625 & \tabularnewline\hline
  九年 & 626 & \tabularnewline
  \bottomrule
\end{longtable}


%%% Local Variables:
%%% mode: latex
%%% TeX-engine: xetex
%%% TeX-master: "../Main"
%%% End:

%% -*- coding: utf-8 -*-
%% Time-stamp: <Chen Wang: 2019-12-23 17:53:22>

\section{太宗\tiny(626-649)}

\subsection{生平}

唐太宗李世民(598年1月23日-649年7月10日),唐朝第二任皇帝。出自陇西成纪,626年至649年在位。父親是唐高祖李淵,母親是太穆皇后窦氏。竇皇后有四個兒子,一個女兒,按長幼顺序為:李建成、平陽昭公主、李世民、李玄霸、李元吉。吐火羅葉護稱唐太宗為「天可汗」。

李世民少年从军,曾于雁门关营救隋炀帝。唐朝建立后,李世民受封为秦公,后晋封为秦王,他是杰出的軍事家,率部平定了薛仁果、刘武周、竇建德、王世充等隋末群雄,在唐朝的建立与统一过程中立下赫赫战功,最終統一天下。

武德九年(626年)发动玄武門之變殺死自己的兄长太子李建成、四弟齐王李元吉二人及二人诸子,被立为太子,唐高祖李渊不久被迫退位,李世民即位,在位時間只使用一個年号贞观。

李世民登基後,積極聽取群臣的意見,以文治天下,并开疆拓土,成為中國史上著名的明君。他虛心納諫,在國內厲行節約,使百姓能够休養生息,終於使得社會出現了國泰民安的局面,開創了中国历史上著名的貞觀之治,為唐朝130年的盛世奠定重要基礎。李世民爱好文学与书法,其真迹今仅存晋祠之铭并序碑刻。649年7月10日(贞观二十三年五月己巳日),唐太宗李世民因病驾崩于含风殿,享年52岁,在位23年,庙号太宗 ,諡號「文皇帝」,葬于昭陵。

隋文帝開皇十七年十二月戊午日(598年1月23日),李世民出生於岐州武功縣,是當時擔任隋朝岐州刺史的漢人李淵與竇氏所生的嫡次子。

隋炀帝大业九年(613年),他的母親在涿郡(治今北京市西南)病逝,高士廉看中了李世民,把外甥女長孫氏(登基後稱長孫皇后)許配給李世民為妻。

隋大業十一年(615年),雲定興被授以左屯衛大將軍,奉命援救在雁門關被突厥始畢可汗所率大軍圍困的隋煬帝。雲定興向各地招募願意出征的軍士,李世民應募從軍,被劃歸雲定興的帳下,並建議使用疑兵之計攻突厥,雲定興採納,突厥軍遂撤退。

隋大業十二年(616年),父親李淵升任隋朝右驍衛將軍。

大業十三年(617年)正月,李淵又遷任太原郡留守。同年7月,殺郡丞王威、武牙郎將高君雅,李世民也跟隨到太原勸諫父親起兵反隋,接著打著“勤王定亂,迎回隋天子”的旗號,正式開始於晉陽縣起兵,並且得到李氏宗族、姻親的響應,是為晉陽起兵。李淵以李世民為敦煌公、右領軍都督、統右三軍。

李渊诛杀了隋炀帝派来监视他的王威、高君雅。派刘文静出使东突厥得到了始毕可汗的支持,派李建成、李世民夺取西河郡。六月,正式起兵。李渊自为大将军,以长子李建成、次子李世民为左右大都督,以四子李元吉留守太原,进兵大兴城(长安)。李渊在霍邑大破宋老生,从龙门渡黄河,开永丰仓赈济百姓。关中有其女李三娘等人起兵响应。

十一月,李渊攻克大興,以代王杨侑为皇帝,尊隋炀帝杨广为太上皇,李渊自为大丞相、唐王。次年三月,隋炀帝杨广在江都被宇文化及所杀,五月,李渊废黜杨侑,称帝,改国号为唐,定都大興,易名長安,唐朝建立。攻克隋大兴城後,李世民官拜京兆尹、受封秦國公。618年,遷趙國公。李淵稱帝后,李世民拜尚書令、晉為秦王。

唐朝建立後,疆土只限于关中和河东一带,尚未完全统治全国,因此,李世民經常出征,最终統一中國。自武德元年起,李世民親自参與四場大戰役。

其一,破薛舉,淺水原平定隴西薛仁果(薛舉之子),平定祖宗之地。其二,敗宋金剛、劉武周,收復並、汾失地,消灭北方地方军阀。其三,在虎牢之戰中,一舉殲滅中原兩大割據勢力—河南王世充和河北竇建德集團,消除河北、河南的地方势力。其四,重創竇建德餘部劉黑闥和山東的徐圓朗。

自此李世民威望日隆,尤其是在虎牢之戰後班師返京時,受到长安軍民的隆重歡迎。武德四年十月,封為天策上將,領司徒、陝東道大行臺尚書令,食邑增至二萬戶。李淵又下詔特許天策府自置官屬,李世民因此闢弘文館,收攬四方彥士入館備詢顧問,與秦王府相結合,儼然一個小內閣。

618年,李渊建立唐朝為唐高祖,并立世子李建成为太子。太原起兵是李世民的谋略,高祖曾答应他事成之后立他为太子,但天下平定后,李世民功名日盛,高祖却犹豫不决。太子李建成随即联合四弟齐王李元吉,共同排挤李世民。同时,高祖的优柔寡断,也使朝中政令相互冲突,加速了诸子的兵戎相见。

此後,長兄皇太子李建成知李世民終不肯屈為人臣,而李世民也認為是自己奠下唐朝開國的基業,遂與李建成、四弟齊王李元吉猜忌日深,兩派大臣之間互相傾軋。李世民曾在李建成东宫饮酒,吐血数升,怀疑李建成下毒。

其中宰相裴寂、謀士王珪、魏徵、東宮衛士將領薛萬徹等追隨李建成、李元吉。秦府謀士杜如晦、房玄齡,將領秦叔寶、尉遲敬德、段志玄、侯君集、王君廓等跟從李世民。宰相陳叔達、朝臣長孫無忌等暗中支援李世民。其餘將領李靖、李世勣,大臣宇文士及等保持中立。

武德九年,突厥侵犯唐边境,李建成向高祖建议由李元吉做统帅出征突厥。太子府率更丞王晊告诉了秦王:李建成想借此控制秦王的兵马,并准备在昆明池设伏兵杀李世民,于是李世民决定先发制人。武德九年六月初四庚申日(626年7月2日),李世民在首都長安城宮城的北门玄武门附近射殺皇太子李建成、齊王李元吉,史稱「玄武門之變」。

此後高祖讓出軍政大權予秦王,而建成、元吉则被宣布为作乱者,諸子则遭诛杀并从宗籍中除名。三天后(六月初七癸亥日,7月5日),李世民被立为皇太子,诏曰:“自今军国庶事,无大小悉委太子处决,然后闻奏”。八月初九甲子日(9月4日),高祖退位稱太上皇,禪位於李世民。李世民登基,是为唐太宗。当年十月,追封李建成为息隐王,李元吉为海陵剌王。次年改元貞觀。642年,追复李建成为隐太子,李元吉为巢剌王,并将皇子李福过继李建成为嗣(后来另一皇子李明也在唐高宗年间被出继李元吉为嗣)。

貞觀二年(628年),当时的人口已因隋末戰爭而銳減,此时唐朝只有290萬戶,經太宗君臣23年的努力,社會安定、經濟恢復並穩定發展,至唐高宗永徽三年(652年),人口達到380萬戶,奠下了高宗、武則天、玄宗年間大唐盛世的基礎,史稱貞觀之治。

貞觀二年四月二十六壬寅日(628年6月3日),朔方人梁洛仁杀夏州割据势力首领梁师都,归降唐朝,唐朝统一全国。貞觀四年(630年),太宗令李靖出師塞北,挑戰東突厥在東亞的霸主地位。唐軍在李靖的調遣下,滅亡東突厥,太宗因此被西域諸國尊為「天可汗」。在位期間,積極推行了府兵制、租庸調制和均田制,並加強科舉制等政策。

在民族政策方面,太宗一方面扩张疆土,另一方面又接受了拓跋魏、北齐、北周二百多年的历史现实,提出其蕃汉兼包、一视同仁的民族政策。李世民曾对他的左右大臣说:「自古帝王皆貴中華,賤夷狄,朕獨愛之如一,故其種落皆視朕如父母。」

太宗本身也是個既英武又善辯之人,但是有鑑於帝位得之不易,加上隋煬帝本人亦以雄健爾雅善辯聞名,隋卻因此鑄下滅亡的大錯,因此在位期間,太宗鼓勵群臣批評他的決策和風格。其中魏徵廷諫了200多次,在廷上直陳太宗的過失,在早朝時多次發生了使太宗尷尬、下不了臺的狀況。晚年的太宗因國富民強,納諫的器度不如初期,偶爾也發生誤殺大臣的遺憾,但是大致上仍克制、保有納言的風範。641年,唐室文成公主下嫁於吐蕃的松贊干布。

太宗即帝位不久,按秦王府文學館模式,新設弘文館,進一步儲備天下文才。另外,太宗精擅書法,以行書寫碑,稱「飛白」,聞名後世。著名作品有《溫泉銘》、《晉祠銘》等。晚年太宗著《帝範》一書以教戒太子李治,總結了他的施政經驗,同時自評一生功過。史家曾疑太宗生前,指定以東晉書法大家王羲之所作《蘭亭集序》為陪葬品。近年據考古學家和歷史學者研究,《蘭亭集序》應該不在太宗之昭陵,而在高宗、武則天所合葬的乾陵之中。

唐太宗與身邊大臣魏徵、王珪、房玄齡、杜如晦、虞世南、褚遂良等的對答也在開元十八(730年)、十九年間被吳兢輯為《貞觀政要》一書,以發揚太宗勵精圖治的治國精神。

武德九年(626年)八月,因唐朝發生玄武門之變,政局不穩,東突厥伺機入侵,攻至距首都長安僅40裡的涇陽(今陝西咸陽涇陽縣),京師震動。此時,長安兵力不過數萬,剛剛即位的太宗被迫設疑兵之計,親率高士廉、房玄齡等6騎在渭水隔河與頡利可汗對話,怒斥頡利、突利二可汗背約。《唐語林》記載太宗「空府庫」贈予頡利可汗金帛財物,以求突厥退軍,並與之結「渭水之盟」,突厥兵於是退去。之後,太宗勵精圖治,並且挑撥頡利、突利二可汗和突厥與鐵勒諸部的關係。627年,東突厥內部出現分裂。反對頡利可汗的薛延陀、回紇、拔也古、同羅諸部落對其變革國俗和推行的政令不滿,另立薛延陀為可汗。突利可汗也暗中與唐聯絡,並與頡利可汗決裂。同時東突厥又遇到大雪氣候,牲畜大多被凍死餓死,突厥勢力漸弱。太宗於629年八月任命李靖、李世勣、柴紹、李道宗等為行軍總管,出兵征討東突厥。630年三月頡利兵敗被俘,東突厥滅亡。唐朝在東突厥突利可汗故地設置順、祐、化、長四州都督府,頡利可汗故地置定襄都督府、雲中都督府。

東突厥滅亡後,薛延陀的真珠可汗乙失夷男接管了東突厥的故土。薛延陀表面臣服於唐朝,暗中却扩充自己的力量。639年,太宗試圖恢復東突厥,立俟力苾可汗阿史那思摩,以抗衡薛延陀的崛起,薛延陀为避免新恢复的东突厥站稳脚跟,與其進行了多次戰爭。為保住東突厥,李世勣在641年进攻薛延陀,并取得了胜利。但是644年,趁太宗征伐高句麗的機會,薛延陀部隊發起新一輪攻勢,擊敗東突厥,迫使阿史那思摩逃回云州。隨後,高句麗尋求薛延陀援助,但夷男希望避免與唐朝直接戰鬥。645年,夷男死後,他的兒子多彌可汗拔灼開始和唐軍作戰。646年,唐軍反擊並打敗拔灼後,薛延陀的附庸回紇、鐵勒等部落出兵,將他殺死。拔灼的堂兄伊特勿失可汗咄摩支向唐軍投降,薛延陀滅亡。太宗於鐵勒故地設六府七州:瀚海府(回紇)、金微府(僕骨)、燕然府(多濫葛)、盧山府(思結)、龜林府(同羅)、幽陵府(拔野古)。七州:皋蘭州(渾)、高闕州(斛薛)、雞鹿州(奚結)、雞田州(阿跌)、榆溪州(契苾)、蹛林州(思結別部)、窴顏州(白霫)。由燕然都護府管理,治所在陰山之麓(今內蒙古杭錦後旗),轄境東到大興安嶺、西到阿爾泰山、南到戈壁、北到貝加爾湖的整個蒙古高原。

634年,吐蕃贊普松贊干布遣使與唐朝修好,唐朝也派臣入蕃。636年,松贊干布派專使去長安請婚,唐朝不允許。638年,松贊干布遂借口唐朝屬國吐谷渾從中作梗,亲自指挥大约20万吐蕃军,开始攻击唐朝的松州(今四川阿坝藏族羌族自治州)。但同时松赞干布又派遣使者到唐朝国都长安再次请求,并宣称他们打算欢迎公主。唐太宗派侯君集为当弥道行军大总管指挥5万军队,执失思力、牛进达、刘简协助,援救松州。与此同时,吐蕃军正在围困松州的首县-嘉诚(今四川松潘),但唐军先遣部队在牛进达指挥下,打败了吐蕃军。唐軍在松州大勝吐蕃軍,但唐朝也見識到了吐蕃的力量。640年,松贊干布又派大臣祿東贊使唐求婚,唐太宗便以宗室之女文成公主許嫁於吐蕃贊普松贊干布,並派禮部尚書江夏王李道宗持節護送。641年文成公主入蕃,《新唐書》記載松贊干布親迎於柏海,文成公主進蕃時把各種漢地的生產技術轉入吐蕃。

唐太宗滅東突厥後,開始對西域(即現代新疆和中亞地區)的西突厥以及一些鬆散結盟綠洲國家的施加軍事實力,其主要針對西突厥,以恢復兩漢以來對西域的統治。高昌王麴文泰與西突厥欲谷設聯合,阻礙西域商路,進攻唐朝的伊州。639年冬,太宗以侯君集為交河道行軍大總管,率兵出擊高昌王麴文泰。640年,唐軍至磧口,麴文泰驚懼而病死。其子麴智盛即位後不久,侯君集圍城,麴智盛降唐軍。高昌國三州、五縣、二十二城,八千戶、三萬餘人歸屬唐朝,高昌國結束。唐朝在高昌設置西州。

吐谷渾可汗伏允聽信大臣天柱王的建議,屢次侵犯唐朝的西部邊境,634年,扣留唐朝使者趙德楷,六月,太宗以段志玄為行軍總管,討伐伏允,十二月,又以李靖、侯君集、李道宗等為行軍總管,大舉討吐谷渾。635年,伏允敗走,被部下所殺。伏允之子慕容順殺死天柱王,自立為可汗,投降唐朝,太宗冊封慕容順為吐谷渾可汗。慕容順死後,636年,太宗冊封慕容順之子諾曷缽為吐谷渾可汗。

640年,唐朝在交河城設安西都護府,用以針對西突厥和管理西域。644年,西突厥的盟友焉耆攻打西州,安西都護郭孝恪為西州道行軍總管,討伐依附西突厥的焉耆,佔領焉耆,俘虜國王龍突騎支,但後來焉耆再次脫離唐朝。648年,唐太宗派遣阿史那社爾、郭孝恪率軍討伐依附西突厥的焉耆和龜茲(今新疆阿克蘇),征服兩國。然後疏勒和于闐歸附唐朝,將安西都護府遷至龜茲,撫寧西域,統龜茲、焉耆、于闐、疏勒四國,史稱安西四鎮。

貞觀四年,西域諸國君主在長安尊稱太宗為「天可汗」,意為天下總皇帝或天下共主。「天可汗」除了是一種對唐朝皇帝的榮銜,還是一種有實質意義的國際組織體系,以維持當時各同盟國的集體安全。

642年,高句麗東部大人淵蓋蘇文殺死榮留王後立高藏為王並自封為「大莫離支」攝政。為征討淵蓋蘇文和保護唐朝的盟友新羅,唐太宗認為有必要對高句麗開戰。644年,太宗率領李世勣、李道宗、張亮和長孫無忌統軍10萬親征高句麗。645年,太宗衝破高句麗的防線準備攻打高句麗國都平壤,似乎大功在即。不料在安市(今遼寧鞍山)受阻,再也無法前行。在這之後,太宗對高句麗的進攻僅維持在一些小規模的突襲。646年,唐朝與回紇擊滅薛延陀後,647年,唐太宗令牛進達率兵從海上、李世勣率兵從陸路攻打遼東半島。648年,太宗再派薛萬徹率軍從海上攻打鴨綠江口。隨後,唐開始集結陸海部隊準備在649年再一次大規模攻高句麗。不過太宗於649年去世後,新皇帝唐高宗暫停東征的計劃。668年,高宗聯合新羅滅亡高句麗,載籍戶數69.7萬。並建立安東都護府等加以控制遼東。

在北方,貞觀四年(630年),唐军滅亡東突厥,漠南成為唐势力范围。貞觀二十年(646年),又一舉消滅了薛延陀汗國,至此大漠南北广大地区皆為唐的势力范围。唐朝廷在漠北設立安北都护府,在漠南設立单于都护府,建立了南至罗伏州(今越南河静)、北括玄阙州(後改名余吾州,今安加拉河地区)、西及安息州(今乌兹别克斯坦布哈拉)、东临哥勿州(今吉林通化)的辽阔疆域。

在西北,貞觀四年,唐朝廷在伊吾七城設立西伊州,開始經營西域。貞觀二十二年(648年),郭孝恪击败龟兹国,把安西都护府治所迁至龟兹。

在东北,644年唐太宗征讨高句丽未果,唐高宗在668年乃联合新罗灭高句丽,设立安东都护府。

在青藏高原上,吐蕃日漸興起,至六世紀末與吐谷浑、蘇毗為高原上三大勢力。七世纪初,贊普松贊干布即位,統一了高原,又征服了位於西藏西部的蘇毗、阿里地區的羊同和尼婆羅(今尼泊尔)。松贊干布於634年遣使與唐朝修好,唐朝也派臣入蕃。636年,松贊干布派專使去長安請婚,唐朝不允,638年,松贊干布遂借口唐朝屬國吐谷渾從中作梗,出兵入侵吐谷渾,唐軍在松州大勝吐蕃軍。640年,松贊干布又派大臣祿東贊使唐求婚,唐太宗便以宗室之女文成公主許嫁於吐蕃贊普松贊干布,並派禮部尚書江夏王李道宗持節護送,吐蕃赞普遂接受唐朝的册封。

面對自己空前的文治武功,太宗到晚年也出現一些過失。首先納諫不如貞觀早期積極,比如貞觀十年,魏徵發現他「漸惡直言」。其次奢侈之風日重。不過晚年他還是能反省自己過度奢靡的錯誤。司馬光說唐太宗:「好尚功名,不及禮樂,父子兄弟之間,慚德多矣」。同时,太宗晚年也由早年的清静转为奢纵,营建宫殿,计划封禅泰山等,并自辩“百姓无事则骄逸,劳役则易使”,魏征因此谏到“恐非兴邦之至言,岂安人之长算?”不过由于太宗晚年能够清醒认识自己的问题,所以也能进行调整,因此虽然太宗晚年存在这些过失,最终没有出现败亡的危机,“功大过微,故业不堕”,维持了贞观之治的局面。

《資治通鑑》記載,太宗貞觀十七年廢太子李承乾之後、改立李治為皇太子之前,李世民之三子一弟(長子李承乾、四子李泰、五子李祐、及七弟李元昌)俱謀取帝位,致太宗心灰意冷之曲折,史載:「承乾既廢,上御兩儀殿,群臣俱出,獨留長孫無忌、房玄齡、李世勣、褚遂良,謂曰:『我三子一弟,所為如是,我心誠無聊賴!』因自投於床,無忌等登前扶抱,上又抽佩刀欲自刎,遂良奪刀以授晉王治。」

貞觀二十二年(648年)正月,唐太宗撰寫《帝範》十二篇頒賜給太子李治。貞觀二十三年(649年),唐太宗得了痢疾(一種傳染病),醫治最終無效(一說是服用天竺长生藥无效),命李治到金掖門代理國事。貞觀二十三年五月二十六日(649年7月10日),唐太宗李世民崩逝于終南山翠微宮含風殿內,享年五十一岁,在位二十三年,初谥文皇帝,庙号太宗,唐高宗上元元年(674年)加谥文武圣皇帝,唐玄宗天寶八年(749年)加谥文武大圣皇帝,天寶十三年(754年)加谥文武大圣大广孝皇帝,安葬於唐昭陵(位于今中國陝西省禮泉縣東北50多里山峰上)。

《貞觀政要》贊貞觀之治:官吏多自清謹,王公妃主之家,大姓豪猾之伍,無敢侵欺細人。商旅野次,无复盗贼,囹圄常空,去年犯死者仅二十九人。又频致丰稔,米斗三钱,马牛布野,外户不闭,行旅自京师至于岭表,自山东至于沧海,皆不赍粮,取给于路。入山东村落,行客经过者,必厚加供待,或发时有赠遗。此皆古昔未有也。

后晋官修正史《旧唐书》刘昫等的評價是:“史臣曰:臣观文皇帝发迹多奇,聪明神武。拔人物则不私于党,负志业则咸尽其才。所以屈突、尉迟,由仇敌而愿倾心膂;马周、刘洎,自疏远而卒委钧衡。终平泰阶,谅由斯道。尝试论之:础润云兴,虫鸣螽跃。虽尧、舜之圣,不能用檮杌、穷奇而治平;伊、吕之贤,不能为夏桀、殷辛而昌盛。君臣之际,遭遇斯难,以至抉目剖心,虫流筋擢,良由遭值之异也。以房、魏之智,不逾于丘、轲,遂能尊主庇民者,遭时也。或曰:以太宗之贤,失爱于昆弟,失教于诸子,何也?曰:然,舜不能仁四罪,尧不能训丹朱,斯前志也。当神尧任谗之年,建成忌功之日,苟除畏逼,孰顾分崩,变故之兴,间不容发,方惧“毁巢”之祸,宁虞“尺布”之谣?承乾之愚,圣父不能移也。若文皇自定储于哲嗣,不骋志于高丽;用人如贞观之初,纳谏比魏徵之日。况周发、周成之世袭,我有遗妍;较汉文、汉武之恢弘,彼多惭德。迹其听断不惑,从善如流,千载可称,一人而已!赞曰:昌、发启国,一门三圣。文定高位,友于不令。管、蔡既诛,成、康道正。贞观之风,到今歌咏。”

北宋官修正史《新唐书》欧阳修、宋祁等的評價是:“甚矣,至治之君不世出也!禹有天下,传十有六王,而少康有中兴之业。汤有天下,传二十八王,而其甚盛者,号称三宗。武王有天下,传三十六王,而成、康之治与宣之功,其余无所称焉。虽《诗》、《书》所载,时有阙略,然三代千有七百余年,传七十余君,其卓然著见于后世者,此六七君而已。呜呼,可谓难得也!唐有天下,传世二十,其可称者三君,玄宗、宪宗皆不克其终,盛哉,太宗之烈也!其除隋之乱,比迹汤、武;致治之美,庶几成、康。自古功德兼隆,由汉以来未之有也。至其牵于多爱,复立浮图,好大喜功,勤兵于远,此中材庸主之所常为。然《春秋》之法,常责备于贤者,是以后世君子之欲成人之美者,莫不叹息于斯焉。”

《新唐书·北狄列传》:唐之德大矣!際天所覆,悉臣而屬之;薄海內外,無不州縣,遂尊天子曰“天可汗”。三王以來,未有以過之。至荒區君長,待唐璽纛乃能國;一為不賓,隨輒夷縛。故蠻琛夷寶,踵相逮於廷。

朱熹与陈亮书:“太宗之心,则吾恐其无一不出于人欲也。直以其能假仁假义,以行其私。而当时与之争者,才能知术既出其下,又不知有仁义之可饬。是以彼善于此,而得以成其功尔。”“论后世人,不当尽绳以古人礼法。毕竟高祖不当立建成。”“太宗功高,天下所系属,亦自无安顿处,只高祖不善处置了。”

文天祥《古代状元卷:文天祥殿试卷》:太宗全不知道、闺门之耻、将相之夸、末年辽东一行、终不能以克其血气之暴、其心也骄。

元朝戈直在《貞觀政要》集論中說:“夫太宗之於正心修身之道,齊家明倫之方,誠有愧於二帝三王之事矣。然其屈己而納諫,任賢而使能,恭儉而節用,寬厚而愛民,亦三代而下,絕無而僅有者也。後之人君,擇其善者而從之,其不善者而改之,豈不交有所益乎!”這裡所說,太宗在正心修身,齊家明倫方面,有愧于二帝三王之事,主要是指太宗與其兄李建成的皇位之爭。

明朝官修皇帝实录《明太祖实录》记载,明太祖朱元璋在洪武七年八月初一日(1374年9月7日),亲自前往南京历代帝王庙祭祀三皇、五帝、夏禹王、商汤王、周武王、汉高祖、汉光武帝、隋文帝,唐太宗、宋太祖、元世祖一共十七位帝王,其中对唐太宗李世民的祝文是:“惟唐太宗皇帝英姿盖世,武定四方,贞观之治,式昭文德。有君天下之德而安万世之功者也。元璋以菲德荷天佑人助,君临天下,继承中国帝王正统,伏念列圣去世已远,神灵在天,万古长存,崇报之礼,多未举行,故于祭祀有阙。是用肇新庙宇于京师,列序圣像及历代开基帝王,每岁祀以春、秋仲月,永为常典。今礼奠之初,谨奉牲醴、庶品致祭,伏惟神鉴。尚享!”

明憲宗在命儒臣訂正重刊《貞觀政要》時寫道:“太宗在唐為一代英明之君,其濟世康民,偉有成烈,卓乎不可及已。所可惜者,正心修身,有愧于二帝三王之道,而治未純也。”

毛泽东评价李世民说:“自古能军无出李世民之右者,其次则朱元璋耳。”

王仲荦《隋唐五代史》:“唐代的皇帝裡,唐太宗,早年的唐玄宗,唐宣宗,都是杰出的皇帝。”“我们认为旧日的封建歷史家对‘貞觀之治’是渲染得有點過分的。……固然,在唐太宗统治的二十多年间,人口有了较大的增长,但比之隋極盛时户数,还不到二分之一。”“魏徵疏文中也说到:‘今自伊洛以东,暨于海岱,灌莽巨泽,茫茫千里、人煙斷絕,鸡犬不闻。道路萧条,进退艰阻。’”“封建歷史家把貞觀時期當作理想的太平盛世,和實際情況是有很大距離的。”

吕思勉《隋唐五代史》:“唐太宗不过中材,论其恭俭之德,及忧深思远之资,实尚不如宋武帝,更无论梁武帝;其武略亦不如梁武帝,更无论宋武帝,陈武帝矣!”

据《贞观政要》李世民的生日是十二月癸丑,据《资治通鉴》李世民的生日是十二月癸未,据《旧唐书》李世民生于隋开皇十八年十二月戊午(599年1月23日),因此李世民的生日应为十二月份。据《旧唐书》李世民卒年五十二岁,其弟李玄霸无考;据《新唐书》李世民卒年五十三岁,其弟李玄霸年十六岁死于隋大业十年(614年),则李玄霸生卒年为公元599-614年,而李世民生卒年为公元597-649年;李世民以十二月出生,李世民生卒年月为598年1月-649年7月,与李玄霸(599-614)为同母兄弟。《新唐书》推翻了《旧唐书》关于李世民的生卒年月,增加了李玄霸的生卒年,使李世民与李玄霸的生卒更可信。胡如雷著《李世民传》即以《新唐书》为依据,考证李世民的出生年月为隋开皇十七年十二月戊午(598年1月28日)。

《新唐书》增加了李玄霸的生卒年岁,补正了李世民的生卒年岁,补充了《旧唐书》中没有的珍贵史料,《新唐书》与《旧唐书》同被列为《二十四史》之钦定官史。据胡如雷考证:“李世民生于开皇十八年十二月之说亦难成立,因窦氏在不到十三个月的时间里先后两次生子的可能性虽然不能完全排除,但就常情而言,这种可能性也不大”。根据李世民同母弟李玄霸十六岁时死于大业十年,而倒推出李玄霸生于开皇十九年,所以若李世民生于开皇十八年十二月,则李玄霸最迟生于开皇十九年十二月,两兄弟生辰过近,不太可能。

貞觀九年十月,即李淵死後五個月,李世民第一次要求觀覽《起居注》,未遂。

《貞觀政要·卷七·論文史第二十八》記載:貞觀十三年,褚遂良為諫議大夫,兼知太宗《起居註》。唐太宗欲查看起居註,褚遂良以「不聞帝王躬自觀史」為由拒絕了。唐太宗說:「朕有不善,卿必記耶?」褚遂良說:「臣職當載筆,何不書之?」黃門侍郎劉洎進言:「人君有過失,如日月之蝕,人皆見之。設令遂良不記,天下之人皆記之矣。」《舊唐書·褚遂良傳》和《資治通鑑·唐紀十二》也載有此事。

次年(640年),唐太宗再度要求看《起居注》,宰相房玄齡等人就刪減整理國史,撰寫成《高祖實錄》和《太宗實錄》各二十卷。當太宗見到「六月四日事,語多微文」——指史官對當年玄武門事變內容含糊其辭,多有隱諱文飾之語,太宗告訴房玄齡:不必替他遮遮掩掩,反正玄武門事件本來就是像「周公誅管、蔡,季友鴆叔牙」之義舉,目的是為了「安社稷、利萬民」,要求「削去浮詞,直書其事」。《資治通鑑·唐紀十三》亦有記載。

這一行為帶給史學考究極大困難,也遭到章太炎等學者指責:「太宗即立,懼於身後名,始以宰相監修國史,故兩朝《實錄》無信辭。」

《大唐新語·卷一》載,太宗繼位後曾在苑囿內狩獵,一群野豬從森林中衝出。太宗舉弓四箭射殺了四隻,但還是有一頭雄野豬向馬匹直衝而來。吏部尚書唐儉慌忙下馬,與之搏鬥。太宗拔劍砍死野豬,笑著對唐儉說,「天策長史,不見上將擊賊耶?何懼之甚!」唐儉當即回答道:「漢祖以馬上得之,不以馬上理之。陛下以神武定四方,豈復逞雄心於一獸!」太宗覺得唐儉說得有理,於是停止了狩獵。

由於唐太宗在即位前曾當過尚書令,故當太宗做皇帝後,大臣多不敢任其職,於是之後這個職務就幾乎不授人,尚書省的長官就只設左、右僕射,後用其他官員以「同中書門下三品」的頭銜參預朝政,執行宰相職務。至高宗時,又用低級官員以「同中書門下平章事」的頭銜參預朝政,執行宰相職務。左、右僕射成了聽令執行的官員,不能參加大政,唐中宗神龍革命復辟之後,僕射就非宰相職務。中書令、侍中在安史之亂後也不常設了。同中書門下平章事成了宰相最普遍的名稱。

《舊唐書·本紀第二:太宗上》记载,李世民四歲時,其父李淵任岐州刺史,有一書生自稱善相,拜訪李淵說:“公贵人也,且有贵子。”見到李世民時又說:“龙凤之姿,天日之表,年将二十,必能济世安民矣。”李淵害怕這話走漏,派人去追殺書生,書生卻忽然失蹤了。於是李淵就取“濟世安民”之意給李世民命名。

李世民酷愛書法,其書法以隸書見長,並且酷愛書法名品《蘭亭序》(即《蘭亭集序》,王羲之書法珍品,王羲之的字十分多變,就一「之」字就有十數種變化之多),相傳當年某大臣見太宗似有鬱結難紓,問之原因,知道其欲得《蘭亭序》,於是便與辯才和尚(王羲之當年墨寶輾轉傳至其七世孫智永,智永出家為僧,又將墨寶傳予其弟子辯才和尚)鬥智最後終於為李世民獲得。而王羲之本願並不想《蘭亭序》落入君王之手成為陪葬品。但最後結果事與願違,《蘭亭序》最終成為唐太宗的陪葬品。

據新舊唐書太宗本紀,李世民十六歲時參軍,跟隨隋將雲定興,一次隋煬帝楊廣被圍,雲定興軍負責救駕,李世民獻計,故佈疑陣,嚇退敵軍,救回天子。


\subsection{贞观}

\begin{longtable}{|>{\centering\scriptsize}m{2em}|>{\centering\scriptsize}m{1.3em}|>{\centering}m{8.8em}|}
  % \caption{秦王政}\
  \toprule
  \SimHei \normalsize 年数 & \SimHei \scriptsize 公元 & \SimHei 大事件 \tabularnewline
  % \midrule
  \endfirsthead
  \toprule
  \SimHei \normalsize 年数 & \SimHei \scriptsize 公元 & \SimHei 大事件 \tabularnewline
  \midrule
  \endhead
  \midrule
  元年 & 627 & \tabularnewline\hline
  二年 & 628 & \tabularnewline\hline
  三年 & 629 & \tabularnewline\hline
  四年 & 630 & \tabularnewline\hline
  五年 & 631 & \tabularnewline\hline
  六年 & 632 & \tabularnewline\hline
  七年 & 633 & \tabularnewline\hline
  八年 & 634 & \tabularnewline\hline
  九年 & 635 & \tabularnewline\hline
  十年 & 636 & \tabularnewline\hline
  十一年 & 637 & \tabularnewline\hline
  十二年 & 638 & \tabularnewline\hline
  十三年 & 639 & \tabularnewline\hline
  十四年 & 640 & \tabularnewline\hline
  十五年 & 641 & \tabularnewline\hline
  十六年 & 642 & \tabularnewline\hline
  十七年 & 643 & \tabularnewline\hline
  十八年 & 644 & \tabularnewline\hline
  十九年 & 645 & \tabularnewline\hline
  二十年 & 646 & \tabularnewline\hline
  二一年 & 647 & \tabularnewline\hline
  二二年 & 648 & \tabularnewline\hline
  二三年 & 649 & \tabularnewline
  \bottomrule
\end{longtable}


%%% Local Variables:
%%% mode: latex
%%% TeX-engine: xetex
%%% TeX-master: "../Main"
%%% End:

%% -*- coding: utf-8 -*-
%% Time-stamp: <Chen Wang: 2021-10-29 15:28:14>

\section{高宗李治\tiny(649-683)}

\subsection{生平}

唐高宗李治(628年7月21日-683年12月27日),小名雉奴,字为善,唐朝第三任皇帝,唐太宗李世民第九子、嫡三子,母文德皇后,亦和胞妹晉陽公主一樣,唯二被唐太宗親自抚養長大的親生子女。唐代的版图以高宗时为最大,领土面積逾1200万平方公里,东起朝鲜半岛大同江以北,西临鹹海,北包贝加尔湖,南至越南横山。在位三十四年,于弘道元年(683年)崩于洛阳紫微宫贞观殿,終年五十五岁,葬于唐乾陵,庙号高宗,谥号天皇大帝。

唐高宗李治於貞觀二年六月十五日(628年7月21日)出生於長安城,為唐太宗李世民第九子,母親為文德皇后長孫氏。李治与唐太宗嫡长子太子李承乾、嫡次子魏王李泰为长孙皇后所生同母兄弟。贞观五年(631年),李治獲封为「晋王」,幼时即以仁孝闻名。贞观十七年(643年),李承乾被废,李泰被砍傷,贞观十八年十二月(645年1月)李治被立为晉王。贞观二十三年(649年)七月即位,时年19岁。

唐高宗李治在位三十四年,弘道元年十二月初四日(683年12月27日),唐高宗李治駕崩於洛陽紫微宫貞觀殿,享年五十五歲,安葬於唐乾陵(今陝西省咸陽市乾縣梁山),武后令上金、素节二王,义阳、宣城二公主听赴哀,谥号天王星。

李治相貌劍眉朗目,隆準高鼻,口能吞日,八字鬚,下鬚濃密至胸。他生得英俊高大,很有帝王的相貌,而性情慈祥、低調、儉樸,不喜興土木,亦不信方士长生不死之術,不喜游猎。李治有知人之明,身邊诸多贤臣如:辛茂将、卢承庆、许圉师、杜正伦、薛元超、韋思謙、戴至德、张文瓘、魏元忠等人,大多是他自己親自提拔,其中韦思谦曾受褚遂良打击,杜正伦被太宗皇帝冷落。

根據史書記載,高宗長期有頭痛與眼睛毛病,到晚年甚至眼睛全盲,因而時常無法下判斷。高宗為此曾請御醫秦鳴鶴醫治。秦鳴鶴主張對腦針灸,武則天坐在幕簾後面大怒,說:“此人可斬首,竟敢以針刺聖上之頭!”然而高宗認為針灸對病情有益,故批准御醫所請。經御醫針灸百會後,高宗即謂眼睛能視物。

傳統中國史學家認為唐高宗生性软弱及受武则天牽制,使其评价不及唐太宗的貞觀之治及唐玄宗的开元盛世。但事實上,唐高宗亦是有為之主,不少決策都對國家有利,唐朝在高宗統治下,國力達至鼎盛。

在國內施政上,高宗致力糾正太宗的苛政。例如他未及正式登基即下令:“罢辽东之役及诸土木之功。”他即位的第二年,即永徽二年的九月,下令将所占百姓田宅还给百姓。高宗在位前期,有效鞏固太宗的成果,後世視之為貞觀之治的延續,史稱“永徽之治”。後世亦常質疑高宗無法阻止武后專權。其實,高宗在朝廷中掌握实权,如在位最後一年仍親自任免宰相、更壓制武后的勢力如貶抑她的親信李义府、许敬宗。而武后逐漸掌權則或可解釋為,高宗在個人健康狀況、唐朝女性地位崇高、武后在雙方共治天下時顯示其有為之能的多重考慮下的決定,不應單純歸咎於高宗個性懦弱。

軍事方面,唐高宗在位年間,不但保住太宗打下的江山,也發動多場重要的對外戰爭,成功開疆擴土。他先後灭西突厥(顯慶二年,657年)、灭百济(顯慶五年,660年)、灭高句丽(總章元年,668年),使唐朝的疆域达到最大。

高宗時代也是良將輩出,除了先朝名將李勣,還有東征西討的蘇定方、聲震西域的裴行儉、三箭定天山的薛仁貴、老當益壯的劉仁軌,以及王方翼、程務挺、王孝杰、少數民族將領黑齒常之等等,都為高宗的對外戰爭貢獻良多。

同時,值得注意的是,唐太宗的昭陵只立了14尊番臣像,而高宗的乾陵卻多達61尊番臣像,這些番臣像至今仍在。高宗任用多位邊族的國王、貴族子弟、人民擔任各級官職,充分體現了高宗時期唐朝的國勢及影響力,亦顯示出邊族對唐朝的仰慕、歸服。


\subsection{永徽}

\begin{longtable}{|>{\centering\scriptsize}m{2em}|>{\centering\scriptsize}m{1.3em}|>{\centering}m{8.8em}|}
  % \caption{秦王政}\
  \toprule
  \SimHei \normalsize 年数 & \SimHei \scriptsize 公元 & \SimHei 大事件 \tabularnewline
  % \midrule
  \endfirsthead
  \toprule
  \SimHei \normalsize 年数 & \SimHei \scriptsize 公元 & \SimHei 大事件 \tabularnewline
  \midrule
  \endhead
  \midrule
  元年 & 650 & \tabularnewline\hline
  二年 & 651 & \tabularnewline\hline
  三年 & 652 & \tabularnewline\hline
  四年 & 653 & \tabularnewline\hline
  五年 & 654 & \tabularnewline\hline
  六年 & 655 & \tabularnewline
  \bottomrule
\end{longtable}

\subsection{显庆}

\begin{longtable}{|>{\centering\scriptsize}m{2em}|>{\centering\scriptsize}m{1.3em}|>{\centering}m{8.8em}|}
  % \caption{秦王政}\
  \toprule
  \SimHei \normalsize 年数 & \SimHei \scriptsize 公元 & \SimHei 大事件 \tabularnewline
  % \midrule
  \endfirsthead
  \toprule
  \SimHei \normalsize 年数 & \SimHei \scriptsize 公元 & \SimHei 大事件 \tabularnewline
  \midrule
  \endhead
  \midrule
  元年 & 656 & \tabularnewline\hline
  二年 & 657 & \tabularnewline\hline
  三年 & 658 & \tabularnewline\hline
  四年 & 659 & \tabularnewline\hline
  五年 & 660 & \tabularnewline\hline
  六年 & 661 & \tabularnewline
  \bottomrule
\end{longtable}

\subsection{龙朔}

\begin{longtable}{|>{\centering\scriptsize}m{2em}|>{\centering\scriptsize}m{1.3em}|>{\centering}m{8.8em}|}
  % \caption{秦王政}\
  \toprule
  \SimHei \normalsize 年数 & \SimHei \scriptsize 公元 & \SimHei 大事件 \tabularnewline
  % \midrule
  \endfirsthead
  \toprule
  \SimHei \normalsize 年数 & \SimHei \scriptsize 公元 & \SimHei 大事件 \tabularnewline
  \midrule
  \endhead
  \midrule
  元年 & 661 & \tabularnewline\hline
  二年 & 662 & \tabularnewline\hline
  三年 & 663 & \tabularnewline
  \bottomrule
\end{longtable}

\subsection{麟德}

\begin{longtable}{|>{\centering\scriptsize}m{2em}|>{\centering\scriptsize}m{1.3em}|>{\centering}m{8.8em}|}
  % \caption{秦王政}\
  \toprule
  \SimHei \normalsize 年数 & \SimHei \scriptsize 公元 & \SimHei 大事件 \tabularnewline
  % \midrule
  \endfirsthead
  \toprule
  \SimHei \normalsize 年数 & \SimHei \scriptsize 公元 & \SimHei 大事件 \tabularnewline
  \midrule
  \endhead
  \midrule
  元年 & 664 & \tabularnewline\hline
  二年 & 665 & \tabularnewline
  \bottomrule
\end{longtable}

\subsection{乾封}

\begin{longtable}{|>{\centering\scriptsize}m{2em}|>{\centering\scriptsize}m{1.3em}|>{\centering}m{8.8em}|}
  % \caption{秦王政}\
  \toprule
  \SimHei \normalsize 年数 & \SimHei \scriptsize 公元 & \SimHei 大事件 \tabularnewline
  % \midrule
  \endfirsthead
  \toprule
  \SimHei \normalsize 年数 & \SimHei \scriptsize 公元 & \SimHei 大事件 \tabularnewline
  \midrule
  \endhead
  \midrule
  元年 & 666 & \tabularnewline\hline
  二年 & 667 & \tabularnewline\hline
  三年 & 668 & \tabularnewline
  \bottomrule
\end{longtable}

\subsection{总章}

\begin{longtable}{|>{\centering\scriptsize}m{2em}|>{\centering\scriptsize}m{1.3em}|>{\centering}m{8.8em}|}
  % \caption{秦王政}\
  \toprule
  \SimHei \normalsize 年数 & \SimHei \scriptsize 公元 & \SimHei 大事件 \tabularnewline
  % \midrule
  \endfirsthead
  \toprule
  \SimHei \normalsize 年数 & \SimHei \scriptsize 公元 & \SimHei 大事件 \tabularnewline
  \midrule
  \endhead
  \midrule
  元年 & 668 & \tabularnewline\hline
  二年 & 669 & \tabularnewline\hline
  三年 & 670 & \tabularnewline
  \bottomrule
\end{longtable}

\subsection{咸亨}

\begin{longtable}{|>{\centering\scriptsize}m{2em}|>{\centering\scriptsize}m{1.3em}|>{\centering}m{8.8em}|}
  % \caption{秦王政}\
  \toprule
  \SimHei \normalsize 年数 & \SimHei \scriptsize 公元 & \SimHei 大事件 \tabularnewline
  % \midrule
  \endfirsthead
  \toprule
  \SimHei \normalsize 年数 & \SimHei \scriptsize 公元 & \SimHei 大事件 \tabularnewline
  \midrule
  \endhead
  \midrule
  元年 & 670 & \tabularnewline\hline
  二年 & 671 & \tabularnewline\hline
  三年 & 672 & \tabularnewline\hline
  四年 & 673 & \tabularnewline\hline
  五年 & 674 & \tabularnewline
  \bottomrule
\end{longtable}

\subsection{上元}

\begin{longtable}{|>{\centering\scriptsize}m{2em}|>{\centering\scriptsize}m{1.3em}|>{\centering}m{8.8em}|}
  % \caption{秦王政}\
  \toprule
  \SimHei \normalsize 年数 & \SimHei \scriptsize 公元 & \SimHei 大事件 \tabularnewline
  % \midrule
  \endfirsthead
  \toprule
  \SimHei \normalsize 年数 & \SimHei \scriptsize 公元 & \SimHei 大事件 \tabularnewline
  \midrule
  \endhead
  \midrule
  元年 & 674 & \tabularnewline\hline
  二年 & 675 & \tabularnewline\hline
  三年 & 676 & \tabularnewline
  \bottomrule
\end{longtable}

\subsection{仪凤}

\begin{longtable}{|>{\centering\scriptsize}m{2em}|>{\centering\scriptsize}m{1.3em}|>{\centering}m{8.8em}|}
  % \caption{秦王政}\
  \toprule
  \SimHei \normalsize 年数 & \SimHei \scriptsize 公元 & \SimHei 大事件 \tabularnewline
  % \midrule
  \endfirsthead
  \toprule
  \SimHei \normalsize 年数 & \SimHei \scriptsize 公元 & \SimHei 大事件 \tabularnewline
  \midrule
  \endhead
  \midrule
  元年 & 676 & \tabularnewline\hline
  二年 & 677 & \tabularnewline\hline
  三年 & 678 & \tabularnewline\hline
  四年 & 679 & \tabularnewline
  \bottomrule
\end{longtable}

\subsection{调露}

\begin{longtable}{|>{\centering\scriptsize}m{2em}|>{\centering\scriptsize}m{1.3em}|>{\centering}m{8.8em}|}
  % \caption{秦王政}\
  \toprule
  \SimHei \normalsize 年数 & \SimHei \scriptsize 公元 & \SimHei 大事件 \tabularnewline
  % \midrule
  \endfirsthead
  \toprule
  \SimHei \normalsize 年数 & \SimHei \scriptsize 公元 & \SimHei 大事件 \tabularnewline
  \midrule
  \endhead
  \midrule
  元年 & 679 & \tabularnewline\hline
  二年 & 680 & \tabularnewline
  \bottomrule
\end{longtable}

\subsection{永隆}

\begin{longtable}{|>{\centering\scriptsize}m{2em}|>{\centering\scriptsize}m{1.3em}|>{\centering}m{8.8em}|}
  % \caption{秦王政}\
  \toprule
  \SimHei \normalsize 年数 & \SimHei \scriptsize 公元 & \SimHei 大事件 \tabularnewline
  % \midrule
  \endfirsthead
  \toprule
  \SimHei \normalsize 年数 & \SimHei \scriptsize 公元 & \SimHei 大事件 \tabularnewline
  \midrule
  \endhead
  \midrule
  元年 & 680 & \tabularnewline\hline
  二年 & 681 & \tabularnewline
  \bottomrule
\end{longtable}

\subsection{开耀}

\begin{longtable}{|>{\centering\scriptsize}m{2em}|>{\centering\scriptsize}m{1.3em}|>{\centering}m{8.8em}|}
  % \caption{秦王政}\
  \toprule
  \SimHei \normalsize 年数 & \SimHei \scriptsize 公元 & \SimHei 大事件 \tabularnewline
  % \midrule
  \endfirsthead
  \toprule
  \SimHei \normalsize 年数 & \SimHei \scriptsize 公元 & \SimHei 大事件 \tabularnewline
  \midrule
  \endhead
  \midrule
  元年 & 681 & \tabularnewline\hline
  二年 & 682 & \tabularnewline
  \bottomrule
\end{longtable}

\subsection{永淳}

\begin{longtable}{|>{\centering\scriptsize}m{2em}|>{\centering\scriptsize}m{1.3em}|>{\centering}m{8.8em}|}
  % \caption{秦王政}\
  \toprule
  \SimHei \normalsize 年数 & \SimHei \scriptsize 公元 & \SimHei 大事件 \tabularnewline
  % \midrule
  \endfirsthead
  \toprule
  \SimHei \normalsize 年数 & \SimHei \scriptsize 公元 & \SimHei 大事件 \tabularnewline
  \midrule
  \endhead
  \midrule
  元年 & 682 & \tabularnewline\hline
  二年 & 683 & \tabularnewline
  \bottomrule
\end{longtable}

\subsection{弘道}

\begin{longtable}{|>{\centering\scriptsize}m{2em}|>{\centering\scriptsize}m{1.3em}|>{\centering}m{8.8em}|}
  % \caption{秦王政}\
  \toprule
  \SimHei \normalsize 年数 & \SimHei \scriptsize 公元 & \SimHei 大事件 \tabularnewline
  % \midrule
  \endfirsthead
  \toprule
  \SimHei \normalsize 年数 & \SimHei \scriptsize 公元 & \SimHei 大事件 \tabularnewline
  \midrule
  \endhead
  \midrule
  元年 & 683 & \tabularnewline
  \bottomrule
\end{longtable}


%%% Local Variables:
%%% mode: latex
%%% TeX-engine: xetex
%%% TeX-master: "../Main"
%%% End:

%% -*- coding: utf-8 -*-
%% Time-stamp: <Chen Wang: 2021-10-29 15:28:25>

\section{中宗李顯\tiny(683-684)}

\subsection{生平}

唐中宗李顯(656年11月26日-710年7月3日),后改名李哲,是唐朝的第四和第六任皇帝,两次在位:第一次在位时间为684年1月3日-684年2月26日,第二次在位时间为705年2月23日-710年7月3日。唐中宗前后两次当政,共在位五年半,公元710年病逝,终年55岁,谥号大和大圣大昭孝皇帝(初谥孝和皇帝),葬于定陵。

李显是唐高宗和武则天的儿子。显庆元年(656年)十一月乙丑,生于长安。號為佛光王。一开始他被封为周王。娶祖姑母常樂公主的女儿赵氏为王妃。但赵氏为武则天所不喜,最终在675年幽禁于内侍省而死。仪凤二年(677年),李显为徙封英王,改名李哲。永隆元年(680年),兄長李賢的太子地位被废黜,李显繼立為太子。同年,他的庶长子李重福出生。而后,选韦玄贞女韦氏为太子妃。高宗去世后,李显于弘道元年十二月十一日(684年1月3日)继位。

李顯繼位後,打算建立一支自己的力量與母后 ( 武則天 ) 抗衡,他的主要支持人是他的妻子韋后外戚。他打算將國丈 ( 韋后之父 ) 韋玄貞提拔為侍中,遭到武太后親信裴炎的反對。李顯怒下說:「朕即使把天下都給韋玄貞,又有何不可?還在乎一個侍中嗎?」於是,武太后便以此為理由,將李顯貶為廬陵王,软禁在均州(今湖北丹江口市均州镇)和房州(今湖北十堰市房县)。嗣聖元年(684年),武太后廢李顯,改立李旦為帝。

698年,武则天将李显重新立为太子。701年李显的嫡长子李重润、女儿永泰公主因得罪武则天男宠张易之、张昌宗兄弟被赐死。705年2月22日,宰相張柬之、侍郎崔玄暐、左羽林將軍敬仲曄、右羽林將軍桓彥範、司刑少卿袁恕己等五人以禁軍發動兵變,史稱神龍革命,武则天被軍隊包圍,被迫下詔禪讓帝位給李显。2月23日,李显復辟,同年3月3日,復國號為唐。

李显对与他患难与共的韋后非常信任,与她同参朝政,将她已過世的父亲追封王爵,她的女儿安乐公主也得参政,获大权。安乐公主希望李显能将她立为皇太女,以继帝位,韦后这时也对他越来越不看在眼中了,希望學武則天般當皇帝。韦后认为李显庶长子李重福告密导致了李重润和永泰公主的死,对李显进谗,使得李重福被外放不得回京,而庶三子李重俊被立为太子。

707年,太子李重俊被安樂公主迫害,於是發動了重俊之變,率兵攻殺武三思、武崇訓父子於其門第,又攻玄武門,欲殺韋皇后與安樂公主等人。不幸被守城的衛兵攔阻,李显見狀呼喊,以重賞要求士兵歸順。於是軍官王歡喜倒戈,斬殺李多祚等於樓下,餘黨潰散。李重俊逃亡途中,被左右殺死。

710年李显病逝,終年55岁,諡號大和大聖大昭孝皇帝(初諡孝和皇帝)。死後不久,韦后矫诏立时年仅16岁的李重茂为帝,不久就被李隆基與太平公主發動兵變推翻,史稱唐隆之變。

因大臣们认为韦后有罪被剥夺皇室身份不再适合与李显合葬,李旦复登皇位后,追谥被李显追封为皇后的赵氏为和思皇后,因不知道她的墓地,就举行了招魂祔葬之礼,将皇后翟衣葬于陵所。

按照两《唐书》和《资治通鉴》的记载,唐中宗李顯是被毒死。按照这个说法,韋后的两个情人杨均和马秦客害怕和皇后私通的事情败露,韋后想当皇帝,而安乐公主想当皇太女,几方势力都觉得中宗碍手碍脚。于是,大家联合搞出了一碗毒汤饼。为了增强这个说法的合理性,《资治通鉴》在景龙四年的五月,也就是唐中宗去世的前一个月还特意加上一笔:“五月,丁卯,许州司兵参军偃师燕钦融复上言:‘皇后淫乱,干预国政,宗族强盛;安乐公主、武延秀、宗楚客图危宗社。” 唐中宗的死因还有以下几种可能。众所周知,李唐家族有心脑血管的遗传病史,唐高祖、唐太宗、长孙皇后、唐高宗统统患有“气疾”、“风疾”,这在古代都指心脑血管类疾病。正因为如此,李唐王朝的皇帝们并不长寿,李显五十五岁死亡尚属正常。另外,有的心脑血管疾病是以发病急、死亡率高为特征的,李显在事先没有表现出什么症状的情况下暴卒,也符合心脑血管疾病的一般规律。

唐中宗被母后武则天所废,其原因是多方面的,而其结果对后世的影响也颇大,直接导致了此后李唐王朝的一度中断和武周王朝的建立,是武则天发动武周革命的先奏。中宗继位后,由于为先皇守丧,政事都暂时取决于母后武则天,但丧期过后,母后仍无意还政,他不甘心受制于人,便提拔妻子韋皇后的娘家人试图向母后挑战,任人唯亲,更说出“让天下”的气话,却反而成为母后废他的把柄。

此外,武则天的政治野心也是重要因素。武则天自成为高宗皇后後就逐步掌握大唐朝廷的最高权力,高宗驾崩后,擅自临朝称制,无意还政,图谋自立为女皇帝,不愿意中宗成为自己称帝道路上的绊脚石,所以要废黜他,让睿宗来当傀儡皇帝。

有关裴炎历史上颇有争议,因为其人心术不正,曾经排挤多位名臣(如裴行儉),被人猜测他有政治野心,甚至想取李唐皇室而代之,所以他助武后成功地临朝称制(见弘道大事记)以后,又助武后废中宗,但终于被武后视为眼中钉而被杀(见光宅大事记),死后没多少人同情他,或认为他活该;但也有人认为他对唐室忠心不二;还有人认为他只是贪图权位的普通政客,并非想要篡权。因此,他之所以助武后废中宗也有三种不同的看法:

有篡位野心,废中宗是行动的一步,之后全面控制朝廷,效法魏晋南北朝以来的权臣(如曹操、司马懿、刘裕、高欢、宇文泰等),立傀儡皇帝,最后取而代之。

对大唐无二心,参与废中宗是因为不想让大唐的天下落入外戚(指韦后一族)手中。

贪恋权位的政治人物,参与废中宗是为防止宰相权力受到削弱,或者是希望自己支持的睿宗能够当上皇帝,权势更加巩固。

掌控军队力量:太后掌控羽林军,政变当天羽林军参与是政变成功的军事基础,即获得武将的支持。

宰相集团相助:政变时裴炎是中书令、宰相之首,宰相中人也多是裴炎门下或是武后心腹,有他们相助,便获得文官的支持。

突袭成功:太后发动政变当天是农历的双日,按理不上朝。太后在双日上朝出乎中宗的意料之外,使他不及准备,政变时只能束手就擒。

武后发动废中宗的政变是继李世民发动玄武门之变后大唐历史上第二次成功的政变,只是玄武门之变流血很多,而太后兵不血刃,发动的是一次极成功的不流血政变。

武后废黜中宗后,立他的弟弟李旦當傀儡皇帝,从此正式临朝称制,掌握了绝对的君权,在位二十二年,史称“则天朝”或“则天时代”的正式开始,为她成为千古第一女皇铺平道路。

武后废黜中宗也遭来一部分人的不服,其中以徐敬业反应最强烈,他于该年(仍然是684年,但已再改元为光宅)九月在扬州发动反对太后的兵变,号称“匡复”,复称“嗣圣元年”,但旋即被镇压。

李顯是自商朝太甲後的第一位兩朝天子,第一個復辟的「皇帝」,但他昏庸無能,親小人遠賢臣,無法控制宗室、權臣與皇后間的爭鬥,是一位評價中下的中國歷史人物。然而武功方面,他派张仁愿修建三受降城,巩固了河套,嫁金城公主给吐蕃赞普尺带珠丹,在巩固边疆有一定贡献。

\subsection{嗣圣}

\begin{longtable}{|>{\centering\scriptsize}m{2em}|>{\centering\scriptsize}m{1.3em}|>{\centering}m{8.8em}|}
  % \caption{秦王政}\
  \toprule
  \SimHei \normalsize 年数 & \SimHei \scriptsize 公元 & \SimHei 大事件 \tabularnewline
  % \midrule
  \endfirsthead
  \toprule
  \SimHei \normalsize 年数 & \SimHei \scriptsize 公元 & \SimHei 大事件 \tabularnewline
  \midrule
  \endhead
  \midrule
  元年 & 684 & \tabularnewline
  \bottomrule
\end{longtable}


%%% Local Variables:
%%% mode: latex
%%% TeX-engine: xetex
%%% TeX-master: "../Main"
%%% End:

%% -*- coding: utf-8 -*-
%% Time-stamp: <Chen Wang: 2019-12-24 11:49:23>

\section{睿宗\tiny(684-690)}

\subsection{生平}

唐睿宗李旦(662年6月22日-716年7月13日),唐高宗李治和武則天之子,唐朝的第五和第八任皇帝,曾用名李旭轮、李轮,他一生兩繼大統,兩度禪位。两次登基,第一次為天后武氏(登基前的武則天)廢唐中宗李顯而繼位,在位时间是文明元年至載初二年(684年2月27日-690年10月16日),後上表自行退位,禪讓予母親武則天;第二次是在唐隆之變誅除韋皇后及其黨羽後復辟,在位时间是景雲元年至延和元年(710年7月25日-712年9月8日),後退位,內禪予其子李隆基(唐玄宗)。李旦為唐高宗李治諸子之中排行第八,母为武则天,李弘、李贤、唐中宗李顯都是其同母兄长,太平公主則為其同母妹妹。

龍朔二年(662年)六月一日(6月22日)生於長安蓬萊宮含涼殿。当年,封殷王,遥领冀州大都督、单于大都护、右金吾卫大将军。睿宗“謙恭孝友,好學,工草隸,尤愛文字訓詁之書”。乾封元年(666年),徙封豫王。总章二年(669年),徙封冀王。上元二年(675年),徙封相王,拜右卫大将军。仪凤元年(676年),十四岁的李轮纳豆卢氏为孺人(妾室)。儀鳳三年(678年),迁洛牧;改名李旦,徙封豫王。在七月的一次宴会上,父亲唐高宗对叔祖父霍王李元轨说:因为他是最小的儿子,所以最为喜爱。679年,王妃刘氏生下他的长子李成器。

嗣聖元年(684年)二月七日,武则天废其兄中宗帝位,立他为帝,改元文明。不过,由於是武则天執政,“政事決于太后”,睿宗毫无实权,甚至連干预国家大政的权力都沒有,淪為傀儡。載初元年(690年),武则天废除睿宗後自登帝位,改國號周,睿宗被貶为「皇嗣」(候補性質的皇位繼承人。具儀一比皇太子,卻不給皇太子的名分),改名武轮,迁居东宫。

武则天聖曆元年(698年),武則天又改立中宗為儲君。睿宗則從「皇嗣」再被貶為親王,封號相王,他的五個兒子(李成器、李成義、李隆基、李隆範、李隆業)被封為郡王,唐睿宗從此重獲自由,擁有干预国家大政的权力。

神龍元年(705年),宰相張柬之等五人發動神龍革命,殺张易之、张昌宗兄弟,逼武則天退位,迎中宗復辟,不久武则天去世。此後中宗封其為安國相王,隨即辭去。景雲元年(710年),中宗駕崩(傳說是被韋皇后毒杀),(後在韋皇后矯詔下)由中宗幼子李重茂登位,改元唐隆,是為少帝。

睿宗的三子李隆基與太平公主等聯絡禁軍將領,擁兵入宮,將韦后誅殺,迫少帝李重茂遜位,史曰唐隆之變。六月二十四日睿宗復辟於承天門樓,大赦天下,与其子李隆基一起铲除了韋皇后一黨的势力。

延和元年(712年)七月二十五日,唐睿宗無法面對李隆基與太平公主的爭端,於是禪讓帝位於李隆基,是為唐玄宗,自称「太上皇」,每五天在太極殿接受群臣的朝賀,仍自称“朕”,三品已上除授及大刑狱仍然自决,命令称诰、令,而玄宗每日受朝于武德殿,自称“予”,决定三品已下除授及徒罪,命令称制、敕。后来玄宗发动先天政变消灭太平公主一党,睿宗才不得不彻底放权。

開元四年六月二十日(716年7月13日)李旦病逝,享年五十五歲。

\subsection{文明}

\begin{longtable}{|>{\centering\scriptsize}m{2em}|>{\centering\scriptsize}m{1.3em}|>{\centering}m{8.8em}|}
  % \caption{秦王政}\
  \toprule
  \SimHei \normalsize 年数 & \SimHei \scriptsize 公元 & \SimHei 大事件 \tabularnewline
  % \midrule
  \endfirsthead
  \toprule
  \SimHei \normalsize 年数 & \SimHei \scriptsize 公元 & \SimHei 大事件 \tabularnewline
  \midrule
  \endhead
  \midrule
  元年 & 684 & \tabularnewline
  \bottomrule
\end{longtable}

\subsection{光宅}

\begin{longtable}{|>{\centering\scriptsize}m{2em}|>{\centering\scriptsize}m{1.3em}|>{\centering}m{8.8em}|}
  % \caption{秦王政}\
  \toprule
  \SimHei \normalsize 年数 & \SimHei \scriptsize 公元 & \SimHei 大事件 \tabularnewline
  % \midrule
  \endfirsthead
  \toprule
  \SimHei \normalsize 年数 & \SimHei \scriptsize 公元 & \SimHei 大事件 \tabularnewline
  \midrule
  \endhead
  \midrule
  元年 & 684 & \tabularnewline
  \bottomrule
\end{longtable}

\subsection{垂拱}

\begin{longtable}{|>{\centering\scriptsize}m{2em}|>{\centering\scriptsize}m{1.3em}|>{\centering}m{8.8em}|}
  % \caption{秦王政}\
  \toprule
  \SimHei \normalsize 年数 & \SimHei \scriptsize 公元 & \SimHei 大事件 \tabularnewline
  % \midrule
  \endfirsthead
  \toprule
  \SimHei \normalsize 年数 & \SimHei \scriptsize 公元 & \SimHei 大事件 \tabularnewline
  \midrule
  \endhead
  \midrule
  元年 & 685 & \tabularnewline\hline
  二年 & 686 & \tabularnewline\hline
  三年 & 687 & \tabularnewline\hline
  四年 & 688 & \tabularnewline
  \bottomrule
\end{longtable}

\subsection{永昌}

\begin{longtable}{|>{\centering\scriptsize}m{2em}|>{\centering\scriptsize}m{1.3em}|>{\centering}m{8.8em}|}
  % \caption{秦王政}\
  \toprule
  \SimHei \normalsize 年数 & \SimHei \scriptsize 公元 & \SimHei 大事件 \tabularnewline
  % \midrule
  \endfirsthead
  \toprule
  \SimHei \normalsize 年数 & \SimHei \scriptsize 公元 & \SimHei 大事件 \tabularnewline
  \midrule
  \endhead
  \midrule
  元年 & 689 & \tabularnewline
  \bottomrule
\end{longtable}

\subsection{载初}

\begin{longtable}{|>{\centering\scriptsize}m{2em}|>{\centering\scriptsize}m{1.3em}|>{\centering}m{8.8em}|}
  % \caption{秦王政}\
  \toprule
  \SimHei \normalsize 年数 & \SimHei \scriptsize 公元 & \SimHei 大事件 \tabularnewline
  % \midrule
  \endfirsthead
  \toprule
  \SimHei \normalsize 年数 & \SimHei \scriptsize 公元 & \SimHei 大事件 \tabularnewline
  \midrule
  \endhead
  \midrule
  元年 & 689 & \tabularnewline\hline
  二年 & 690 & \tabularnewline
  \bottomrule
\end{longtable}



%%% Local Variables:
%%% mode: latex
%%% TeX-engine: xetex
%%% TeX-master: "../Main"
%%% End:

%% -*- coding: utf-8 -*-
%% Time-stamp: <Chen Wang: 2021-10-29 15:29:04>

\section{周武曌\tiny(683-705)}

\subsection{生平}

武曌(624年2月17日-705年12月16日),唐高宗的皇后、武周開國皇帝,當代稱則天順聖皇后,或武后(神龍革命後成為皇太后,遺詔退稱皇后),後代通称武则天,并州文水县人,中国历史上因執掌君權而得到正史唯一承認的女性皇帝。十四歲入宮為唐太宗才人,十二年不得遷。唐高宗时復为昭儀,謀廢得到唐太宗託付于重臣褚遂良的“佳兒佳婦”元后與淑妃,得立为皇后(655年-683年)。一時尊号为天后,与唐高宗天皇李治并称“二圣”。由于唐高宗患风眩病,無力聽政,660年11月开始臨朝,史载“自此内辅国政数十年,威势与帝无异”,683年12月27日-690年10月16日作为唐中宗、唐睿宗的皇太后临朝称制,后利用酷吏集團屢次屠殺唐室諸王大臣以求立威,殺害嫌疑對象遍及子、女、媳、婿、孫、孫女、孫婿、庶子、嫡兄、親姊、親甥女、夫之伯叔姑嫂、堂兄,終於自立为武周皇帝(690年10月16日-705年2月21日在位),在位時間共14年4個月又5天。晚年惑于內寵,不知當立侄或立子。705年元月,被宰相狄仁傑舉薦的後任張柬之與禁衛軍背叛,被迫還位。退位以後,成為中國歷史上唯一一位女性太上皇,同年崩于洛陽上陽宮仙居殿。唐高宗死后从683年实际真正掌权前后22年。武則天是即位年龄最大(67岁即位)、寿命第三长的皇帝(终年82岁),僅次於清高宗(87歲)和梁武帝(86歲)。

武氏本名无记载,为唐开国勳舊武士彠次女,母亲杨氏為隋朝宗室楊達之女是武士彠繼室,不見禮于正室諸子。祖籍并州文水县(今山西省文水县),十四岁時(貞觀十一年)因貌美而入后宫为唐太宗的才人,唐太宗赐号武媚。高宗时为昭仪,后封为皇后,又上尊号为“天后”。高宗崩,中宗即位,武氏为皇太后,临朝称制后改名曌。武氏認為自己好像日、月一樣崇高,凌掛於天空之上。於称帝后上尊号“聖神皇帝”,退位后中宗上尊号“则天大圣皇帝”,武氏遗制去帝号,称“则天大圣皇后”。武氏另有废除的尊号“圣母神皇、圣神皇帝、金轮圣神皇帝、越古金轮圣神皇帝、慈氏越古金轮圣神皇帝、天册金轮圣神皇帝”等。在位期間喜土木作造,尤喜造國字改年號,一年一號。傳說洛陽龍門石窟的奉先寺大佛是模仿其面容而作。

武氏十四歲时,唐太宗聽聞她姿色豔美,將她納入宮中,据《资治通鉴》所载,时在貞觀十一年(637年)十一月。入宫后,武氏封為五品才人,賜號「武媚」后世讹称武媚娘。武氏入宮之前向寡居的母親楊氏告別時說:「侍奉聖明天子,豈知非福?為何還要哭哭啼啼、作兒女之態呢?」

武才人与太宗的三位嫔妃燕德妃、杨婕妤、巢王妃杨氏俱为表亲。而对于太宗时期武氏在宫中的生活细节,史书并没有详细的描述。仅见武氏在晚年时回忆自己为太宗驯马一事。当时,太宗有名马獅子骢,又肥又暴躁没有能调教牠的人。武氏在太宗身边侍候,对太宗说:「我能制服牠,但是须要三件东西:一是铁鞭,二是铁楇,三是匕首。用铁鞭打它不服,就用楇打牠的头;再不服,就用匕首割断牠的喉咙。」武氏稱太宗壮其之志。複自稱嘗侍太宗,得其書法之妙。

貞觀十七年(643年),太子李承乾被廢,晉王李治被立为太子。此後,在太子侍奉太宗湯藥之際,李治見到武才人並悅之。

贞观二十三年(649年),唐太宗駕崩。武才人依唐後宮之例,入感業寺剃髮出家。永徽元年(650年)五月,唐高宗在太宗週年忌日入感業寺進香之時,與身為比丘尼的武氏相遇。當時與蕭淑妃爭寵的王皇后知悉後,便主動向高宗請求將武氏納入宮中,企圖以此打擊蕭淑妃。唐高宗早有此意,當即應允。永徽二年(651年)五月,唐高宗的孝服已滿,二十七歲的武氏還俗,再度入宮。入宮前武氏已經懷孕,入宮後生下兒子李弘。次年五月,被拜為二品昭儀。

永徽六年(655年)六月,後宮中有人放出不利王皇后之謠言,流傳王皇后與其母柳氏(宰相柳奭之姊,柳宗元同族)請來巫師,企圖用魘鎮之術將武昭仪詛咒而死亡。這謠言在無證據下傳到高宗之耳,高宗大怒,並將其母柳氏趕出皇宮,而且欲將武昭仪陞為一品宸妃(唐朝後宮四夫人中本來並無宸妃此封號,而原本的四夫人名額已滿,唐高宗為了武氏,才創宸妃封號),受到宰相韓瑗和來濟的反對,最後不能成事。不久,中書舍人李義府等人勾結武氏,得知高宗欲行廢皇后而立武昭仪消息,聯絡本已貶官不得再進的許敬宗、崔義玄、袁公瑜等人向唐高宗不斷請求立武昭仪為后,造成群臣支持的表象,廢立之意遂再次萌生。

永徽六年(655年)十月十三日,唐高宗又在李世勣等朝廷武勛的模棱兩可下,終於頒下詔書:以「陰謀下毒」的罪名,將王皇后和蕭淑妃廢為庶人,並加囚禁;她們的父母、兄弟等也被削爵免官,流放嶺南。七天以後,唐高宗再次下詔,將武昭仪立為皇后;與此同時,又將反對最大的宰相褚遂良貶為外州都督。因为忌讳武氏曾为父亲太宗才人的事实,唐高宗在立后诏书中,称武氏为父亲所赐,“事同政君”。

顯慶四年(659年)四月,武后與唐高宗達成共識:將長孫無忌、于志寧、韓瑗、來濟等人削職免官,貶出京師。

顯慶五年(660年),高宗患上風疾之症,頭暈目眩,不能處理國家大事,遂命皇后武氏代理朝政。在麟德元年(664年),高宗与宰相上官儀商議,打算廢掉武氏皇后之位。但上官儀的廢后詔書還未草擬好,武后即已從宦官親信接到消息。她直接來到高宗面前追問此事,唐高宗不得已,便把責任推到上官儀身上。十二月,上官儀被逮捕入獄,不久,即被滅族。

乾封二年(667年)高宗因久疾,命太子弘監國。上元元年(674年)秋八月,武后和高宗並稱天皇天后,名為避先帝、先后之稱,實欲自尊。十二月,武后上表建議十二事:「一、勸農桑,薄賦徭。二、給復三輔地(免除長安及其附近地區之徭役)。三、息兵,以道德化天下。四、南、北中尚(政府手工工場)禁浮巧。五、省功費力役。六、廣言路。七、杜讒口。八、王公以降皆習《老子》。九、父在為母服齊衰(喪服)三年(過去是一年)。十、上元前勳官已給告身(委任狀)者,無追覈。十一、京官八品以上,益稟入(增加薪水)。十二、百官任事久,材高位下者,得進階(提級)申滯。」高宗詔皆施行之。武則天能夠重視農業生產,規定各州縣境內,「田疇墾闢,家有餘糧」者予以升獎;「為政苛濫,戶口流移」者必加懲罰。所編《兆人本業》農書,頒行天下,影響很大。而武則天執政期間,其宗教政策乃以佛教在道教之上。

上元二年(675年)三月,武后召集大批文人學士,大量修書,先後撰成《玄覽》、《古今內範》、《青宮紀要》、《少陽正範》、《維城典訓》、《紫樞要錄》、《鳳樓新誡》、《孝子傳》、《列女傳》、《內範要略》、《樂書要錄》、《百僚新誡》、《兆人本業》、《臣軌》等書。且密令這批學者參決朝廷奏議,以分宰相之權,時人謂之「北門學士」。時高宗風眩更甚,擬使武后攝政,宰相郝處俊說:「陛下奈何以高祖、太宗之天下,不傳之子孫而委之天后乎!」高宗才罷攝政之意。太子李弘深為高宗鍾愛,高宗欲禪位於太子,武后不滿;剛好太子因为蕭淑妃之女義陽、宣城二公主因母得罪武后而被幽禁掖庭宮中、年逾21而未嫁,奏請出降,高宗許之,武后甚怒。不久太子死於合璧宮,時人以為武后所毒殺,但亦有说法称李弘本来病弱而早夭。

弘道元年(683年)十二月,唐高宗病逝,臨終遺詔:太子李顯於柩前即位,軍國大事有不能裁決者,由武氏決定。四天以後,李顯即位,是為唐中宗。武后被尊為皇太后。

光宅元年(684年)二月,中宗欲以韋后父韋玄貞為侍中(宰相),裴炎力諫不聽,武后遂廢唐中宗為廬陵王,並遷於房州。立第四子豫王李旦為帝,是為唐睿宗,武后臨朝稱制,自專朝政。同年九月,徐敬業、徐敬猷兄弟聯合唐之奇、杜求仁等以扶支持廬陵王為號召,在揚州舉兵反武,十多天內就聚合了十萬部眾。武后當即以左玉鈐大將軍李孝逸為揚州道大總管,率兵三十萬,前往征討。十一月,徐敬業兵敗自殺。

垂拱二年(686年)三月,武后下令製造銅匭(銅製的小箱子),置於洛陽宮城之前,隨時接納臣下表疏。同時,又大開告密之門,規定任何人均可告密。凡屬告密之人,國家都要供給驛站車馬和飲食。即使是農夫樵人,武后都親自接見。所告之事,如果符合旨意,就可破格陞官。如所告並非事實,亦不會問罪。同時,武后又先後任用索元禮、周興、來俊臣、侯思止等一大批酷吏,掌管制獄,如果被告者一旦被投入此獄,酷吏們則使用各種酷刑審訊,能活著出獄的百無一二。這樣,隨著告密之風的日益興起,被酷吏刑訊拷打致死的人日漸增多。為獎勵告密,若有屬實,武后對告密者破例授官,以賣餅為生的侯思止,因舉發舒王李元名與恒州刺史裴貞謀反,被任命為游擊將軍、侍御史。王弘義,以無德行見稱,告鄉里謀反,擢授游擊將軍、殿中侍御史。

武后掌管李唐的社稷,翦除唐宗室,諸王不自安,欲起兵對抗。還未有共識的時候,博州刺史瑯邪王李沖,垂拱四年(688年)八月於博州(今山東聊城東北)舉兵。豫州刺史越王李貞起兵豫州(今河南汝南)呼應。武后分遣丘神勣、魏崇裕擊之。瑯邪王李沖起兵七日敗死;九月,越王李貞兵敗自殺。武后想盡除李氏諸王,使周興等審訊之,迫韓王李元嘉、魯王李靈夔、黃國公李譔、東莞郡公李融、常樂公主等自殺,親信等均被誅。

這年命令僧薛怀义率令萬多人,毀乾元殿,建明堂,花了近一年落成,高二百九十四尺,闊三百尺。共三層,上為圓蓋,有條九龍作捧著的姿態。上有鐵鳳,高一丈。飾以黃金,稱為「萬象神宮」。明堂既成,又命僧薛怀义鑄大像,大像的小指也可以容納數十人,於明堂北起五層高的天堂來收納這個大像。所花費用以萬億計,政府財政為之枯竭。是年武承嗣命人鑿白石為文曰:「聖母臨人,永昌帝業。」號稱在洛水中發現,獻給武后,武后大喜,命其石曰「寶圖」。之後武后加尊號為「聖母神皇」。

武后當政期間爲了打擊關中著姓預立的“九品中正制”官人法,造成其父系母系皆是“從龍入關”的世家門閥的歷史假象,進一步發展收攏民心的科舉制度。貞觀年間共錄取進士205人,高宗和武后統治期間共錄取一千餘人。平均每年錄取人數比貞觀時增加一倍以上。武后載初元年(690年)武后在洛城殿對貢士親發策問,是「殿試」之始。是年遣「存撫使」十人巡撫諸道,推舉有才之人,一年後共舉薦一百餘人,武后不問出身,全部加以接見,自稱量才任用,或為試鳳閣(中書省)舍人、給事中,或為試員外郎、侍御史、補闕、拾遺、校書郎,試官制度自此始,時人有「補闕連車載,拾遺平斗量,把推侍御史,腕脫校書郎。」之語。武后雖以官位收買人心,但對不稱己意的人亦會加以罷黜;號稱明察善斷,故當時一部份人亦樂於為武后效力。

次年(690年)七月,僧法明等撰《大雲經》四卷,說武后是彌勒菩薩化身下凡,應作為天下主人,武后下令頒行天下。命兩京諸州各置大雲寺一所,藏《大雲經》,命僧人講解,並提升佛教的地位在道教之上。是年九月侍御史傅游藝率關中百姓九百人上表,請改國號為周,賜皇帝姓武。於是百官及帝室宗戚、百姓、四夷酋長、沙門、道士共六萬餘人,亦上表請改國號。武后准所請,改唐為周。在神都则天门登基即位,改元天授,加尊號聖神皇帝,以睿宗為皇嗣,賜姓武氏,以皇太子為皇孫。立武氏祖宗七廟於神都洛陽,追尊周文王廟號曰始祖,諡號文皇帝。立武承嗣為魏王,武三思為梁王,其餘武氏多人為王及長公主。

同年九月,武則天派右鷹揚衛將軍王孝傑為武威軍總管,與武衛大將軍阿史那忠節率兵赴西域征討吐蕃。十月,唐軍大勝,連克于闐、疏勒、龜兹、碎葉等安西四鎮,仍置安西都護府於龜玆,發兵戍守。

长寿三年(694年)武三思率四夷首領請以銅鐵鑄天樞,立於端門外,以歌頌武則天的功德。武則天親題曰:「大周萬國頌德天樞」。天樞鑄造歷時八月而成,其形制若柱,高一百零五尺,直徑十二尺,八面,每面各五尺,下為鐵山,周一百七十尺,以銅為蟠龍、麒麟環繞之;上為騰雲承露盤直徑三丈,盤上四龍直立捧火珠,高一丈。工人毛婆羅造模,武三思為文,刻百官及四夷首領之名於其上。用銅鐵二百萬斤,「請胡聚錢百萬億,買銅鐵不能足,賦民間農器以足之。」

萬歲通天元年(696年)五月,契丹首領李盡忠和孫萬榮率兵起義,攻陷營州,殺都督趙文翽。武則天派將軍曹仁、張玄遇、李多祚等率兵征討。由於誤吐蕃伏兵,全軍覆沒。接著,武則天再派武攸宜、王孝傑等率兵討伐,均大敗而歸。神功元年(697年)四月,武則天又派武懿宗、婁師德、沙吒忠義率兵二十萬,討伐契丹。六月,孫萬榮兵敗被殺,契丹餘眾歸降於突厥。

神功元年(697年)武則天使武懿宗審訊劉思禮謀反事,武懿宗說只要劉思禮指出哪些朝士有分謀反,就免其死罪,於是劉思禮誣告宰相李元素、孫元亨等三十六家「海內名士」,皆遭滅族,親舊連坐流竄者千餘人。時人以為武懿宗之殘暴僅次於周興、來俊臣。

是年,來俊臣欲羅告武氏諸王及太平公主(中宗之妹,武則天唯一长大成人的親生女兒),又欲誣皇嗣李旦及廬陵王李顯與南北衙共同謀反,擬一網打盡。武氏諸王與太平公主都十分害怕,共同揭發其罪行,下獄處以極刑。仇家爭食其肉,不一會就食盡。來俊臣兇狡貪暴網羅無辜,織成反狀,殺人不可勝計。「贓賄如山,冤魂塞路」,武則天亦知天下憤怨,下令數他的罪狀,並沒收其家財。

聖歷元年(698年)武承嗣、武三思謀求當太子,幾次使人對武則天說:「自古天子未有以異姓為嗣者。」武則天猶豫未決,狄仁傑對武后說:「姑侄之與母子,哪個比較親近?(武承嗣、武三思皆武后之侄,中宗、睿宗則武后之子)陛下立子,則千秋萬歲後,祭祖於太廟;立侄則未聞侄為天子祭姑於太廟者」。又勸武則天召還廬陵王(中宗)。武后由是無立武承嗣、武三思之意。乃召廬陵王還東都,皇嗣(睿宗)請遜位於廬陵王,武后立廬陵王為皇太子,命為元帥,狄仁傑為副元帥率兵擊突厥。武則天信重狄仁傑,常謂之「國老」而不呼其名。狄仁傑好諍諫,武則天每屈意從之。狄仁傑死後,武則天泣曰:「朝堂空矣!」常嘆:「天奪吾國老何太早邪!」

武则天晚年张易之、张昌宗兄弟迅速崛起,成为武则天的新宠,張易之、張昌宗兄弟年少美姿容,入侍武則天。二人常傅朱粉、穿著華麗的衣服。武承嗣、武三思等都爭著追捧他們,甚至為他們執鞭牽馬。

中宗長子邵王李重润(中宗第二次為太子時封為邵王)與其妹永泰郡主及郡主婿武延基竊議張易之兄弟「何得任意入宮」,易之投訴於武則天,武則天敕李重润、永泰郡主、武延基皆賜死。

神龍元年(705年)正月,武則天病篤,卧床不起,只有寵臣張易之、張昌宗兄弟侍側。宰相張柬之、崔玄暐與大臣敬暉、桓彥範、袁恕己等,交結禁軍統領李多祚,佯稱張易之、張昌宗兄弟謀反,於是發動兵變,率禁軍五百餘人,衝入宮中,殺死二張兄弟,隨即包圍武則天寢宮,要求武則天退位,史稱「神龍革命」。

武氏被迫禪讓帝位予兒子李顯,是為唐中宗,迁居上阳宫。中宗上尊號為「則天大聖皇帝」。

神龍元年十一月十六日(705年12月16日),武曌崩逝於洛陽上陽宮仙居殿內,享壽八十一歲。遺制去帝號,稱「則天大聖皇后」。神龍二年(706年)五月,武則天與唐高宗李治合葬於唐乾陵,留无字碑。

历代对武则天有各种不同的评價。唐代前期,由于所有的皇帝都是她的直系子孙,所以当时对武则天的评價相对比較寬容,但唐國史通過對後宮嬪妃與諸王公主的描敘淒慘地予以了無情揭露。随着时间的推移,特別是司馬光所主編之《資治通鑑》,對武氏進行嚴正批判。到了南宋期间,程朱理学在中国思想上占据了主导地位,舆论決定了对武则天的長久評價。譬如明末清初的时候,著名的思想家王夫之,就曾评价武则天“鬼神之所不容,臣民之所共怨”。惟不可否认的是,武后重視延攬,首創鉗制文網式的考試,而且知人善任,能重用狄仁傑、張柬之、桓彥範、敬暉、姚崇等中兴名臣。國家在武則天主政期間,文化承貞觀之模、百姓尚稱富庶。故享「貞觀遺風」之譽,亦及于其孫唐玄宗(其母死於武后手)的開元盛世。

武则天对历史发展做出的第一个贡献是,她打击了保守的门阀世族。武则天被立为皇后以後,把反对她做皇后的长孙无忌、褚遂良等人一个一个的都赶出了朝廷,贬逐到边远地区。这对于武则天来说,是殺雞儆猴,但这些关陇集团和他们的依附者,在当时已经成为一种既得利益的保守力量。把他们赶出政治舞台标志着关陇集团从北周以来长达一个多世纪统治的终结,也为社会进步和经济发展创造了一个良好的条件。

第二是促进了经济的发展。武则天在建言十二事中就建议“劝农桑,薄赋役”。在她掌权以后,又编撰了《兆人本业记》颁发到州县,作为州县官劝农的参考。她还注意地方吏治,加强对地主官吏的监察。对于土地兼并和逃亡的农民,也采取比较寛容的政策。因此,武则天统治时期,社会是相當安定的,农业、手工业和商业都有了長足的发展,户口也由唐高宗永徽三年(652年)的380万户增加到唐中宗神龙元年(705年)的615万户,平均每年增长0.721\%。这在中古時代,是一个很高的增长率,也是反映武则天时期经济发展的客觀數據。

第三个贡献是推动了文化的发展。唐人沈既济在谈及科举制度时说到:“太后颇涉文史,好雕虫之艺。”“太后君临天下二十余年,当时公卿百辟,无不以文章达,因循日久,浸已成风”。一是当时进士科和制科考试主要都是考策问,也就是申論。文章的好坏是录取的主要标准。二是武则天用人不看门第,不问是否為高级官吏的子孙,而是看有否政治才能。因此特别注意从科举出身者中选拔高级官吏。科举出身做到高级官吏的越来越多。这就大大刺激了仕人参加科舉的积极性,更刺激了一般人读书学习的热情。这就是沈既济所说的“浸已成风”。开元、天宝年间“父教其子,兄教其弟”,“五尺童子耻不言文墨焉”的社会风气,就是从武则天时期开始的。正是文化的普及,推动了文化的全面发展。著名的诗人和文学家崔融、李乔都是这个时期涌现出来的。雕塑、绘画也达到了前所未有的水平。

另外武則天也有不少負面評價,岑仲勉说,“武后任事率性,好恶无定,终其临朝之日,计曾任宰相七十三人”。其主政初期,由於大興告密之風,重用酷吏周興、來俊臣等,加上後世史學家不齒於她違反傳統的禮教,身為女子,竟然擁有不少男性嬪妃(稱為「男寵」)。但趙翼為武則天的私生活辯護,說:“人主富有四海,妃嬪動千百,后既為女王,而所寵幸不過數人,固亦未足深怪,故后初不以為諱,而且不必諱也。”

武則天統治的缺失主要是丟失安北領土,她將大部份的精力用於對內,因此對外軍事,屢有失策。首先在686年一度丟棄了安西四鎮,在692年才派王孝杰收復; 696年任用郭元振使反间计令吐蕃内乱,除掉吐蕃名將论钦陵,削弱吐蕃實力。另外又在696年激起孫萬榮、李盡忠的叛亂,使武周期間契丹一度落入突厥人手中。安北都護府在高宗死時尚處在中國統治,而濫殺程務挺、棄用王方翼等名將更使東突厥復國; 但在她執政後期已大致平復邊疆對外用兵的不利局面,並留下後代一個國力尚強的唐朝。然而,唐太宗和唐高宗辛苦經營的安北領土始終沒有再收復過,使唐朝曾經過千萬平方公里的江山永遠丟失了五份之二,即使是唐玄宗開元之治也沒有恢復安北任何領土,而是被回紇取代。

《旧唐书》:“治乱,时也,存亡,势也。使桀、纣在上,虽十尧不能治;使尧、舜在上,虽十桀不能乱;使懦夫女子乘时得势,亦足坐制群生之命,肆行不义之威。观夫武氏称制之年,英才接轸,靡不痛心于家索,扼腕于朝危,竟不能报先帝之恩,卫吾君之子。俄至无辜被陷,引颈就诛,天地为笼,去将安所?初虽牝鸡司晨,终能复子明辟,飞语辩元忠之罪,善言慰仁杰之心,尊时宪而抑幸臣,听忠言而诛酷吏。有旨哉,有旨哉!”赞曰:“龙漦易貌,丙殿昌储。胡为穹昊,生此夔魖?夺攘神器,秽亵皇居。穷妖白首,降鉴何如。”

《新唐书》:“昔者孔子作《春秋》而乱臣贼子惧,其于杀君篡国之主,皆不黜绝之,岂以其盗而有之者,莫大之罪也,不没其实,所以著其大恶而不隐欤?自司马迁、班固皆作《高后纪》,吕氏虽非篡汉,而盗执其国政,遂不敢没其实,岂其得圣人之意欤?抑亦偶合于《春秋》之法也。唐之旧史因之,列武后于本纪,盖其所从来远矣。夫吉凶之于人,犹影响也,而为善者得吉常多,其不幸而罹于凶者有矣;为恶者未始不及于凶,其幸而免者亦时有焉。而小人之虑,遂以为天道难知,为善未必福,而为恶未必祸也。武后之恶,不及于大戮,所谓幸免者也。至中宗韦氏,则祸不旋踵矣。然其亲遭母后之难,而躬自蹈之,所谓下愚之不移者欤!”

沈既济:“太后颇涉文史,好雕虫之艺。”“太后君临天下二十余年,当时公卿百辟,无不以文章达,因循日久,浸已成风。”

崔融:“英才远略,鸿业大勋。雷霆其武,日月其文。洒以甘露,覆之庆云。制礼作乐,还淳返朴。宗礼明堂,崇儒太学。四海慕化,九夷禀朔。沈璧大河,泥金中岳。巍乎成功,翕然向风。”

鲁宗道:“唐之罪人也,几危社稷。”

洪邁《容齋隨筆》:「漢之武帝、唐之武后,不可謂不明」。

司马光:“虽滥以禄位收天下人心, 然不称职责,寻亦黜之,或加刑诛,挟刑赏之柄以驾御天下,政由己出,明察善断,故当时英贤亦竞为之。”

赵翼:“女中英主。”“人主富有四海,妃嫔动千百,后既为女王,而所宠幸不过数人,固亦未足深怪,故后初不以为讳,而且不必讳也。”

翟蔼:“武氏以一妇人君临天下二十余年,是不比於母后之称制者,而直自帝自王也,此其智有过人者。”

岑仲勉:“武后任事率性,好恶无定,终其临朝之日,计曾任宰相七十三人。”

郭沫若:“政启开元,治宏贞观;芳流剑阁,光被利州。”

宋庆龄:“武则天是封建时代杰出的女政治家。但就家庭角色而言,不难看出武则天也是个好妻子。”

毛泽东:“武则天确实是个治国之才,她既有容人之量,又有识人之智,还有用人之术。她提拔过不少人,也杀了不少人。刚刚提拔又杀了的也不少。”

翦伯赞:“武则天的打击门阀贵族和提拔普通地主做官的政策,是符合当时社会发展趋势的,因此她的作用是积极的……武则天在巩固封建国家的边疆方面,也做了不少工作。”

江青對武則天的評價很高,認爲武則天是中國婦女中最傑出的人物。

相传唐太宗在世時,曾請天文師算命,天文師認為,不出三十年,李氏皆亡於一個姓武的人手裡了。於是太宗屠殺武氏朝臣,沒想到所算者竟是他身邊的武才人。

相傳在感業寺時期,李治有次前往祭拜,看到武氏後便魂不守舍。當時,與蕭淑妃爭寵的王皇后,便趁此納武氏為自己派系,跟蕭淑妃對抗,沒想到兩人皆亡於武氏之手。

武則天稱帝後,亦有多名男寵,此段為妄說。其中最出名者乃馮小寶(薛怀义),武則天後派他在洛陽東的白馬寺出家,法名懷義,但仍與武則天私通。某年盂蘭盆節,當時已經逐漸失寵的懷義,為討武則天注意,火烧明堂,火勢蔓延整個洛陽。

武则天为了誇飾武周革命,创造了则天文字。部分的則天文字還傳到日本、韓國,甚至成為某些日本禮遇中國文化人的人名用字。

武则天著有《垂拱集》百卷,《金轮集》十卷,已散佚。今存诗四十六首,《全唐文》编其文为四卷。有《石榴裙》之思。

武則天稱帝前掌握實權的6年,使用了3個年號,稱帝的15年使用了16個年號,合19個年號,是中國皇帝中用年號最多和密度最高的皇帝。居於第二的是西晉皇帝晉惠帝司馬衷,除了時間最長的元康年號(9年),9年間用了8個年號。

佛經「開經偈」的撰寫者。"無上甚深微妙法,百千萬劫難遭遇,我今見聞得受持,願解如來真實義"。武則天是真正的大修行人,做此偈時感得天女散花,流傳迄今。很多評論均是不符合事實的妄說,因不符合統治者利益和倫理習俗之故。但武則天問心無愧,立無字碑任憑人解讀。

\subsection{天授}

\begin{longtable}{|>{\centering\scriptsize}m{2em}|>{\centering\scriptsize}m{1.3em}|>{\centering}m{8.8em}|}
  % \caption{秦王政}\
  \toprule
  \SimHei \normalsize 年数 & \SimHei \scriptsize 公元 & \SimHei 大事件 \tabularnewline
  % \midrule
  \endfirsthead
  \toprule
  \SimHei \normalsize 年数 & \SimHei \scriptsize 公元 & \SimHei 大事件 \tabularnewline
  \midrule
  \endhead
  \midrule
  元年 & 690 & \tabularnewline\hline
  二年 & 691 & \tabularnewline\hline
  三年 & 692 & \tabularnewline
  \bottomrule
\end{longtable}

\subsection{如意}

\begin{longtable}{|>{\centering\scriptsize}m{2em}|>{\centering\scriptsize}m{1.3em}|>{\centering}m{8.8em}|}
  % \caption{秦王政}\
  \toprule
  \SimHei \normalsize 年数 & \SimHei \scriptsize 公元 & \SimHei 大事件 \tabularnewline
  % \midrule
  \endfirsthead
  \toprule
  \SimHei \normalsize 年数 & \SimHei \scriptsize 公元 & \SimHei 大事件 \tabularnewline
  \midrule
  \endhead
  \midrule
  元年 & 692 & \tabularnewline
  \bottomrule
\end{longtable}

\subsection{长寿}

\begin{longtable}{|>{\centering\scriptsize}m{2em}|>{\centering\scriptsize}m{1.3em}|>{\centering}m{8.8em}|}
  % \caption{秦王政}\
  \toprule
  \SimHei \normalsize 年数 & \SimHei \scriptsize 公元 & \SimHei 大事件 \tabularnewline
  % \midrule
  \endfirsthead
  \toprule
  \SimHei \normalsize 年数 & \SimHei \scriptsize 公元 & \SimHei 大事件 \tabularnewline
  \midrule
  \endhead
  \midrule
  元年 & 692 & \tabularnewline\hline
  二年 & 693 & \tabularnewline\hline
  三年 & 694 & \tabularnewline
  \bottomrule
\end{longtable}

\subsection{延载}

\begin{longtable}{|>{\centering\scriptsize}m{2em}|>{\centering\scriptsize}m{1.3em}|>{\centering}m{8.8em}|}
  % \caption{秦王政}\
  \toprule
  \SimHei \normalsize 年数 & \SimHei \scriptsize 公元 & \SimHei 大事件 \tabularnewline
  % \midrule
  \endfirsthead
  \toprule
  \SimHei \normalsize 年数 & \SimHei \scriptsize 公元 & \SimHei 大事件 \tabularnewline
  \midrule
  \endhead
  \midrule
  元年 & 694 & \tabularnewline
  \bottomrule
\end{longtable}

\subsection{证圣}

\begin{longtable}{|>{\centering\scriptsize}m{2em}|>{\centering\scriptsize}m{1.3em}|>{\centering}m{8.8em}|}
  % \caption{秦王政}\
  \toprule
  \SimHei \normalsize 年数 & \SimHei \scriptsize 公元 & \SimHei 大事件 \tabularnewline
  % \midrule
  \endfirsthead
  \toprule
  \SimHei \normalsize 年数 & \SimHei \scriptsize 公元 & \SimHei 大事件 \tabularnewline
  \midrule
  \endhead
  \midrule
  元年 & 695 & \tabularnewline
  \bottomrule
\end{longtable}

\subsection{天册万岁}

\begin{longtable}{|>{\centering\scriptsize}m{2em}|>{\centering\scriptsize}m{1.3em}|>{\centering}m{8.8em}|}
  % \caption{秦王政}\
  \toprule
  \SimHei \normalsize 年数 & \SimHei \scriptsize 公元 & \SimHei 大事件 \tabularnewline
  % \midrule
  \endfirsthead
  \toprule
  \SimHei \normalsize 年数 & \SimHei \scriptsize 公元 & \SimHei 大事件 \tabularnewline
  \midrule
  \endhead
  \midrule
  元年 & 695 & \tabularnewline
  \bottomrule
\end{longtable}

\subsection{万岁登封}

\begin{longtable}{|>{\centering\scriptsize}m{2em}|>{\centering\scriptsize}m{1.3em}|>{\centering}m{8.8em}|}
  % \caption{秦王政}\
  \toprule
  \SimHei \normalsize 年数 & \SimHei \scriptsize 公元 & \SimHei 大事件 \tabularnewline
  % \midrule
  \endfirsthead
  \toprule
  \SimHei \normalsize 年数 & \SimHei \scriptsize 公元 & \SimHei 大事件 \tabularnewline
  \midrule
  \endhead
  \midrule
  元年 & 695 & \tabularnewline\hline
  二年 & 696 & \tabularnewline
  \bottomrule
\end{longtable}

\subsection{万岁通天}

\begin{longtable}{|>{\centering\scriptsize}m{2em}|>{\centering\scriptsize}m{1.3em}|>{\centering}m{8.8em}|}
  % \caption{秦王政}\
  \toprule
  \SimHei \normalsize 年数 & \SimHei \scriptsize 公元 & \SimHei 大事件 \tabularnewline
  % \midrule
  \endfirsthead
  \toprule
  \SimHei \normalsize 年数 & \SimHei \scriptsize 公元 & \SimHei 大事件 \tabularnewline
  \midrule
  \endhead
  \midrule
  元年 & 696 & \tabularnewline\hline
  二年 & 697 & \tabularnewline
  \bottomrule
\end{longtable}

\subsection{神功}

\begin{longtable}{|>{\centering\scriptsize}m{2em}|>{\centering\scriptsize}m{1.3em}|>{\centering}m{8.8em}|}
  % \caption{秦王政}\
  \toprule
  \SimHei \normalsize 年数 & \SimHei \scriptsize 公元 & \SimHei 大事件 \tabularnewline
  % \midrule
  \endfirsthead
  \toprule
  \SimHei \normalsize 年数 & \SimHei \scriptsize 公元 & \SimHei 大事件 \tabularnewline
  \midrule
  \endhead
  \midrule
  元年 & 697 & \tabularnewline
  \bottomrule
\end{longtable}

\subsection{圣历}

\begin{longtable}{|>{\centering\scriptsize}m{2em}|>{\centering\scriptsize}m{1.3em}|>{\centering}m{8.8em}|}
  % \caption{秦王政}\
  \toprule
  \SimHei \normalsize 年数 & \SimHei \scriptsize 公元 & \SimHei 大事件 \tabularnewline
  % \midrule
  \endfirsthead
  \toprule
  \SimHei \normalsize 年数 & \SimHei \scriptsize 公元 & \SimHei 大事件 \tabularnewline
  \midrule
  \endhead
  \midrule
  元年 & 698 & \tabularnewline\hline
  二年 & 699 & \tabularnewline\hline
  三年 & 700 & \tabularnewline
  \bottomrule
\end{longtable}

\subsection{久视}

\begin{longtable}{|>{\centering\scriptsize}m{2em}|>{\centering\scriptsize}m{1.3em}|>{\centering}m{8.8em}|}
  % \caption{秦王政}\
  \toprule
  \SimHei \normalsize 年数 & \SimHei \scriptsize 公元 & \SimHei 大事件 \tabularnewline
  % \midrule
  \endfirsthead
  \toprule
  \SimHei \normalsize 年数 & \SimHei \scriptsize 公元 & \SimHei 大事件 \tabularnewline
  \midrule
  \endhead
  \midrule
  元年 & 700 & \tabularnewline\hline
  二年 & 701 & \tabularnewline
  \bottomrule
\end{longtable}

\subsection{大足}

\begin{longtable}{|>{\centering\scriptsize}m{2em}|>{\centering\scriptsize}m{1.3em}|>{\centering}m{8.8em}|}
  % \caption{秦王政}\
  \toprule
  \SimHei \normalsize 年数 & \SimHei \scriptsize 公元 & \SimHei 大事件 \tabularnewline
  % \midrule
  \endfirsthead
  \toprule
  \SimHei \normalsize 年数 & \SimHei \scriptsize 公元 & \SimHei 大事件 \tabularnewline
  \midrule
  \endhead
  \midrule
  元年 & 701 & \tabularnewline
  \bottomrule
\end{longtable}

\subsection{长安}

\begin{longtable}{|>{\centering\scriptsize}m{2em}|>{\centering\scriptsize}m{1.3em}|>{\centering}m{8.8em}|}
  % \caption{秦王政}\
  \toprule
  \SimHei \normalsize 年数 & \SimHei \scriptsize 公元 & \SimHei 大事件 \tabularnewline
  % \midrule
  \endfirsthead
  \toprule
  \SimHei \normalsize 年数 & \SimHei \scriptsize 公元 & \SimHei 大事件 \tabularnewline
  \midrule
  \endhead
  \midrule
  元年 & 701 & \tabularnewline\hline
  二年 & 702 & \tabularnewline\hline
  三年 & 703 & \tabularnewline\hline
  四年 & 704 & \tabularnewline
  \bottomrule
\end{longtable}

\subsection{神龙}

\begin{longtable}{|>{\centering\scriptsize}m{2em}|>{\centering\scriptsize}m{1.3em}|>{\centering}m{8.8em}|}
  % \caption{秦王政}\
  \toprule
  \SimHei \normalsize 年数 & \SimHei \scriptsize 公元 & \SimHei 大事件 \tabularnewline
  % \midrule
  \endfirsthead
  \toprule
  \SimHei \normalsize 年数 & \SimHei \scriptsize 公元 & \SimHei 大事件 \tabularnewline
  \midrule
  \endhead
  \midrule
  元年 & 705 & \tabularnewline\hline
  二年 & 706 & \tabularnewline\hline
  三年 & 707 & \tabularnewline
  \bottomrule
\end{longtable}


%%% Local Variables:
%%% mode: latex
%%% TeX-engine: xetex
%%% TeX-master: "../Main"
%%% End:

%% -*- coding: utf-8 -*-
%% Time-stamp: <Chen Wang: 2019-12-24 15:49:15>

\section{中宗复辟\tiny(705-710)}

\subsection{景龙}

\begin{longtable}{|>{\centering\scriptsize}m{2em}|>{\centering\scriptsize}m{1.3em}|>{\centering}m{8.8em}|}
  % \caption{秦王政}\
  \toprule
  \SimHei \normalsize 年数 & \SimHei \scriptsize 公元 & \SimHei 大事件 \tabularnewline
  % \midrule
  \endfirsthead
  \toprule
  \SimHei \normalsize 年数 & \SimHei \scriptsize 公元 & \SimHei 大事件 \tabularnewline
  \midrule
  \endhead
  \midrule
  元年 & 707 & \tabularnewline\hline
  二年 & 708 & \tabularnewline\hline
  三年 & 709 & \tabularnewline\hline
  四年 & 710 & \tabularnewline
  \bottomrule
\end{longtable}


\section{殇帝}

\subsection{生平}


唐恭宗李重茂(695年-714年),唐中宗幼子,唐朝第七位皇帝,景龍四年(710年)六月在位,開元二年(714年)逝世,諡曰殤皇帝。

李重茂生於武后延載元年(695年),聖曆三年(700年)封為北海郡王,中宗神龍元年(705年)進封温王,授右衛大將軍,兼遙領赠州大都督,沒有到任。

景龍四年(710年)中宗病逝后,六月初四韋后臨朝,改元唐隆。六月初七韋后矯詔立时年仅16岁的李重茂為帝,韋后臨朝稱制。李重茂即位不足一個月,六月二十日夜,睿宗三子臨淄王李隆基、中宗妹妹太平公主等交結禁軍將領,發兵入宮,將韦后與安樂公主等人杀死,是為唐隆之變。

六月二十二日,宫人、宦官请求中书舍人刘幽求草诏立太后,刘幽求意图复辟睿宗,拒绝了。六月二十四日,太平公主称李重茂有意让位睿宗,更亲自将其提下御座,睿宗復辟。李重茂仍封温王。

李重茂的庶兄谯王李重福随即发动政变,与其党羽郑愔等策划矫诏称得中宗传位,以李重茂为皇太弟,但很快被镇压。

睿宗景雲二年改封襄王,李重茂離開长安,被遷到集州,睿宗令中郎將率武士五百人守備。

玄宗開元二年(714年),李重茂除房州刺史,不久死於房州。葬於武功西原,得年二十。

\subsection{唐隆}

\begin{longtable}{|>{\centering\scriptsize}m{2em}|>{\centering\scriptsize}m{1.3em}|>{\centering}m{8.8em}|}
  % \caption{秦王政}\
  \toprule
  \SimHei \normalsize 年数 & \SimHei \scriptsize 公元 & \SimHei 大事件 \tabularnewline
  % \midrule
  \endfirsthead
  \toprule
  \SimHei \normalsize 年数 & \SimHei \scriptsize 公元 & \SimHei 大事件 \tabularnewline
  \midrule
  \endhead
  \midrule
  元年 & 710 & \tabularnewline
  \bottomrule
\end{longtable}


%%% Local Variables:
%%% mode: latex
%%% TeX-engine: xetex
%%% TeX-master: "../Main"
%%% End:

\input{12_Tang/08_RuiZong2}
%% -*- coding: utf-8 -*-
%% Time-stamp: <Chen Wang: 2021-10-29 15:50:47>

\section{玄宗李隆基\tiny(712-756)}

\subsection{生平}

唐玄宗李隆基(公元685年9月8日-762年5月3日),685年出生在神都洛阳,唐朝第九代皇帝(712年-756年在位);統治唐朝長達44年,是唐朝在位最久的皇帝,唐睿宗第三子,母窦德妃。庙号玄宗,谥号至道大圣大明孝皇帝。宋代避聖祖趙玄朗讳,清代避諱康熙帝「玄燁」,皆称为唐明皇,另有尊号“开元圣文神武皇帝”。性格英明果断,多才多艺,知晓音律,擅长书法,儀表雄伟俊丽。

唐隆元年(710年),李隆基与太平公主联手发动唐隆之变,诛韦皇后,擁立睿宗李旦,並掌握朝政與京師實際兵權。公元712年,李旦禅位于李隆基,是為玄宗,隨即發動先天之變,赐死可能爭奪大位的太平公主,取得了国家的最高统治权。玄宗在位44年間,前30年開元之治是唐朝的極盛之世,在位後期,由於其怠政加上政策失誤和重用安祿山等,導致了後來長達8年的安史之亂,逃往四川,為唐朝中衰埋下伏筆。756年李亨自立為帝,即肅宗,尊玄宗为太上皇。玄宗762年病逝,葬于泰陵。

武后垂拱元年秋八月五日戊寅(685年9月8日),李隆基生于东都(今河南省洛阳市),后以其日为千秋节、天长节。出生时其父李旦为帝,母窦氏为德妃。垂拱三年(687年)闰七月丁卯,封楚王。永昌元年(689年),祖母武則天命令李隆基過繼予孝敬帝為子,繼其香火。載初二年(690年),李隆基五岁时,父亲李旦被祖母武氏废除帝位,迁居东宫。

天授三年(692年)十月戊戌,李隆基出阁,开府置官属。李隆基英俊多藝,儀表堂堂,少年時代就顯出了極有膽識的性格。當他七歲時,正是武周时期,武懿宗自認為是武則天的姪子,趾高氣揚,根本不把李氏宗室放在眼裏。有一次,武氏諸王到朝堂參加每月朔望的兩次會見時,他看到李隆基的車騎儀仗威嚴而整齊,心中不悅,便利用自己金吾將軍糾察風紀的權力橫加阻撓。李隆基卻理直氣壯地責問:「我家的朝堂,干你甚麼事?竟敢挾迫我的車騎隨從!」祖母武則天知道此事後,不僅未加罪於他,反而更加宠爱他。虽然李隆基获得了祖母的宠爱,但在长寿二年(693年)正月,其母窦氏与嫡母皇嗣妃刘氏被武则天秘密杀害,尸骨无踪。根据史料可知,李旦的另一位妾室豆卢氏和李隆基的姨妈窦氏抚养、照料过年幼丧母的李隆基。同年腊月丁卯,李隆基由楚王改封临淄郡王。根据《唐会要》的记载,就在长寿二年,李隆基娶王氏为郡王妃。

圣历元年(698年),李隆基再度出阁,赐第于东都洛阳的积善坊。大足元年(701年),随祖母回到西京长安,赐宅于兴庆坊。长安中,历右卫郎将、尚辇奉御。神龙革命后,伯父唐中宗复位。景龙二年四月(708年),李隆基兼潞州別駕。

景龙四年(710年),李隆基從潞州(治所在今山西長治)回到長安。他暗中聚結才勇之士,在皇帝的親軍萬騎中發展勢力。太宗時,選官戶及蕃口中驍勇的武士穿虎紋衣,跨豹紋韉,從遊獵,於馬前射禽獸,謂之百騎。武則天時增加為千騎,中宗時發展為萬騎。李隆基非常重視萬騎的作用。

韋后想效法武則天自稱皇帝,但太平公主與上官婉兒密謀,以中宗遺制,立溫王李重茂(中宗少子)為皇太子,韋后知政事,父亲相王李旦參政。韋黨宗楚客、韋溫、紀處訥等人,極力反對相王參謀政事。相王不想捲入宮廷鬥爭,對事件採取迴避的態度,於是李隆基就主動地策劃了消滅韋黨的宮廷政變。

當時韋后想稱帝登基,對太平公主立李重茂為帝不滿;李隆基又借助太平公主的力量壯大自己。正當雙方劍拔弩張之際,原來親近韋氏的兵部侍郎崔日用改變態度,暗中向李隆基告密,勸其立即發動攻勢。於是,李隆基與太平公主的兒子薛崇簡,苑總監鍾紹京等,密謀策劃,欲先發制人。

有人建議,把發動政變的事先向相王報告,李隆基胸有成竹地說:「我是為了拯救社稷,為君主、父親救急,成功了福祉歸於宗廟與社稷,失敗了我因忠孝而死,不連累相王。怎可以報告,讓相王擔心呢!現在報告,相王若贊成,就是害他參與了危險的起事;若他不贊成,我計謀就失敗了。」於是,決定背著相王,立即行動。

唐隆元年(710年)六月庚子日申時,李隆基等人穿便服進入禁苑,到苑總監鍾紹京住處。這時,鍾紹京反悔,拒絕參加這次政變。但其妻許氏卻堅定地說:「忘身殉國,神必助之。既然參與同謀,即使不參加,勢難免罪。」鍾紹京明白,前往拜謁李隆基。入夜後,萬騎果毅李仙鳧、葛福順都先後來到,請李隆基發佈命令。

二更時分,葛福順拔劍直入羽林營,斬韋黨掌握軍隊的韋璿、韋播、高嵩,然後宣佈:「韋后毒死先帝,謀危社稷,今夕當共誅諸韋,身高有馬鞭之長者皆殺之,立相王為帝以安天下。敢有反對者,罪及三族。」羽林軍將士紛紛表示從命。李隆基率眾出禁苑南門,進攻宮城。葛福順率左萬騎攻玄德門,李仙鳧率右萬騎攻白獸門,相約在凌煙閣會見。李隆基率兵直入玄武門。韋后惶恐逃入飛騎營,被飛騎斬首獻於李隆基;安樂公主正在畫眉,也被斬首,其夫武延秀同時被殺(一作夫妇皆在内堂力战而死)。凡是諸韋及韋后親信均被逮捕斬首(但并未杀绝韦家人),史稱唐隆之變。

這時,李隆基才將唐隆之變的經過報告相王。相王抱著李隆基哭泣著說:「宗廟社稷的災禍是你平定的,神明與百姓也都仰賴你的力量了。」當日,隆基被改封為平王,兼殿中監,同中書門下三品、兼押左右萬騎。

李隆基與姑姑太平公主迫使李重茂禪讓於相王李旦。相王即位,是為睿宗。睿宗與大臣議立太子。按嫡長子繼承制度,兄长宋王李成器應為太子,但李成器堅決辭讓說:「國家安則先嫡長,國家危則先有功;平王有功於國,自己決不居平王之上。」參與消滅韋黨的功臣也多主張立李隆基為太子。睿宗順水推舟,遂在秋七月己巳,册立平王李隆基为皇太子,大赦天下,改元景云。九月庚戌,李隆基长子李嗣直封为许昌郡王,次子李嗣谦为真定郡王。

太平公主恃著擁立睿宗有功,經常干預政事。她又感到太子李隆基精明能幹,妨礙自己參政,總想另易太子。同时,太平公主在后宫中,包括李隆基的身边大量安插耳目。李隆基當然不願任人擺佈,亦想除掉太平公主。睿宗最初遇到困難先聽太平公主的意見,再徵求太子的意見。後來,愈來愈傾向太子。

景雲二年(711年)二月,睿宗命太子監國,六品以下除官及徒罪以下,由太子處分。九月,李隆基的一位妾室——楊良媛生下他的第三子李嗣升,即日后的唐肃宗。楊良媛怀孕时,东宫中依附于太平公主的耳目,“必阴伺察,事虽纤芥,皆闻于上”,李隆基心不自安,甚至因太平公主之故试图为楊良媛堕胎。在先天元年(712年)七月,睿宗禪讓於太子。太平公主雖力勸睿宗不要放棄處理大政的權力,但已無濟於事了。

李隆基於延和元年(712年)八月三日即位,是為唐玄宗,改元先天。當時,宰相多是太平公主之黨,文武大臣,也多依附她。於是,除掉太平公主就成了玄宗的當前要務。而太平公主的黨羽看到玄宗銳意親政,就想廢黜玄宗。

先天元年(713年)七月,玄宗與岐王李範、薛王李業、兵部尚書郭元振、龍武將軍王毛仲等決定起事。玄宗命王毛仲到閒廄取出御馬並調家兵三百餘人,親自率領太僕少卿李令向、王守一,內侍高力士,果毅李守德等親信十多人,先殺左、右羽林大將軍常元楷、李慈,又擒獲了太平公主的親信右散騎常侍賈膺福及中書舍人李猷,接著殺了宰相岑羲、蕭至忠;竇懷貞暫時走脫,最後自縊而死。太平公主驚恐萬狀,先逃入山寺,後被賜死於家,是為先天政變。自此以後,一切軍政大事玄宗完全可以自作主張了。

先天元年十月,玄宗到新豐(今陝西臨潼)閱兵於驪山下,調動二十萬人馬,旌旗連亙五十餘里,聲勢浩大。但由於軍容不整,欲斬兵部尚書郭元振,因宰相劉幽求、中書令張說求情,將其流於新州(今廣東新興)。接著,以制軍禮不肅罪殺了給事中、知禮儀事唐紹。本來,玄宗只是為了整頓軍紀,樹立自己的威信,並無意殺唐紹,但由於金吾將軍李邈倉促宣敕,無可挽回,故而玄宗罷了李邈的官。由於兩位大臣得罪,諸軍震動很大,秩序不穩,只有左軍節度薛訥、朔方道大總管解琬二軍穩定,玄宗讚嘆不已。

先天元年十二月,改元為開元。開元時期的三十年是唐朝的極盛時期。玄宗即位後,勵精圖治,重用姚崇,革新政治。姚崇建議:抑制權貴,重視爵賞,納諫諍,禁貢獻,他都採納。無關大局的具體問題,他都放手讓姚崇處理。有一次,姚崇奏請決定郎吏的任命問題,姚崇再三請求玄宗決定,玄宗只是仰視殿屋,置之不理。高力士提醒玄宗應置可否,他答曰:「朕委姚崇理政,大事應當與朕共議,郎吏小官的事,何須一一煩朕!」自此以後,群臣於是知道玄宗能尊重大臣的決定。

玄宗弟薛王李業母舅王仙童,凌辱百姓,被御史彈奏。薛王李業為其求情,玄宗命中書、門下復查。姚崇等奏曰:「王仙童罪狀明白,御史所言正確,不可縱容。」玄宗同意姚崇的意見。從此,所有貴族都不敢放肆。

為了糾正奢華的風氣,開元二年(714年)七月玄宗下令:「乘輿服御、金銀器玩,宜令有司銷毀,以供軍國之用;其珠玉、錦繡,焚於殿前;后妃以下,皆毋得服珠玉錦繡。」又下欶:「百官所服帶及酒器、馬銜、鐙,三品以上,聽飾以玉,四品以金,五品以銀,自餘皆禁之;婦人服飾從其夫、子。其舊成錦繡,聽染為皂。自今天下更毋得採珠玉,織錦繡等物,違者杖一百,工人減一等。」(《資治通鑑》卷二百二十一開元二年七月條)同時,還罷兩京織錦坊。他還反對厚葬,他認為厚葬無益於死者,有損於生者。於是,要求喪葬務遵簡儉,凡送終物品,均不得以金銀器為飾。如有違者,杖一百。州縣長官不能舉察者,一律貶官。

為了從歷史上總結經驗,汲取教訓,作為治理國家的借鑑,玄宗喜愛閱讀史書,讀到有關政事的問題,他特別留心。但常碰到不能解決的疑難問題,於是,他要宰相為他推薦侍讀,幫助他讀書。開元三年(715年)九月,馬懷素、褚無量被推薦為侍讀。玄宗對侍讀非常尊敬,親自迎送,待以師傅之禮。开元三年(715年)正月,玄宗次子李瑛被立为皇太子。

開元十三年(公元725年),唐朝在伯力(今俄羅斯哈巴羅夫斯克)設置黑水府,置黑水軍,對黑水靺鞨地區實施有效的行政管轄,並勘探了堪察加半島和千島群島。《新唐書·北狄傳》記載:“黑水西北又有思慕部,益北行十日得郡利部,東北行十日得窟說部,亦號屈設,稍東南行十日得莫曳皆部。”。

開元二十三年(735年)四月,玄宗與中書門下及禮官、學士宴於東都集仙殿。他說:「仙者憑虛之論,朕所不取。賢者能治理國家,朕與諸位合宴,宜更名曰:集賢殿。」「仙」、「賢」雖一字之差,卻反映了玄宗重視人才的態度。

隨著時間的流逝,玄宗自認為天下已經太平,逐漸喪失了積極進取的精神,以致生活奢華,減少過問政事。陳建平《中國通史一百講》:“開元二十三年的時候,他覺得國家太平,要表現國家的歡樂盛況,於是大宴五鳳樓,在五鳳樓的殿前,開了一個盛大的同樂會,各種音樂、舞蹈、戲劇,百劇雜陳,讓三百里之內的刺史縣令,都要帶領當地的樂舞伎人,集合到五鳳樓之下來表演,這種歡樂表演,熱鬧喧天,連續了五日之久。”

玄宗因所宠武惠妃谗言,将三个儿子太子李瑛、鄂王李瑶、光王李琚废为庶人并杀害,改立三子忠王李玙为太子;武惠妃不久也於開元二十五年(737年)去世,後宮雖多美人,但沒有一個能使他滿意。开元二十八年(740年)十月,玄宗以为逝世多年的母亲窦氏祈福的名义,敕书儿媳、第十四子寿王妃杨氏出家为女道士,道号“太真”。天寶四年(745年)八月,冊楊氏為貴妃。

楊貴妃不僅個人受寵,其三個姐姐也均賜府邸於京師,寵貴赫然;其遠堂兄楊國忠也因而飛黃騰達。楊貴妃每次乘馬,都有大宦官高力士親自執轡授鞭,貴妃院有織繡工七百人。嶺南經略史張九章、廣陵長史王翼,因所他們獻給楊貴妃的貢品精美,二人均被陞官。於是,官吏競相仿效。楊貴妃喜愛嶺南的荔枝,就有人千方百計急運新鮮荔枝到長安。在男尊女卑的社會裏,民間竟然流行歌謠日:「生男勿喜女勿悲,若今看女作門楣。」可見,玄宗寵愛楊貴妃的社會影響相當深遠。

生活的奢靡,隨之而來的是政治上的腐敗。天寶初年,口蜜腹劍的李林甫被重用為相。李林甫為了掌握大權,反對諫官有益的建議。他訓斥諸諫官道:「今明主在上,群臣將順之不暇,何須多言!」補闕杜璡上書言事,次日即被降為下邽(今陝西渭南東北)令。自此以後,沒有人敢再有諫諍之言了。

在用人方面,李林甫認為凡在德才方面超過自己者,他都設法將其除去。玄宗想重用兵部侍郎盧絢,他就把盧絢調任華州(治所在今陝西華縣)刺史,並欺騙玄宗說盧絢因病不能理事而棄而不用。玄宗又欲重用絳州(治所在今山西新絳)刺史嚴挺之,李林甫又欺騙玄宗說嚴挺之年老多病,宜授其散職,便於他養病。於是,嚴挺之又被送到東京(今河南洛陽)養病去了。李林甫雖然專權亂政,但其在位期間,政局尚穩。

李林甫欺上壓下並未引起玄宗注意,他反而仍然認為天下無事,把主要政事交由李林甫處理。高力士多次勸他不可使大權旁落以免失去君威,他還甚為不悅,致使高力士惶恐自責。天寶十一年(752年)李林甫死後,玄宗一方面重用擅權弄法的楊貴妃堂兄楊國忠為宰相,一方面信任居心叵測的邊將安祿山,以圖左右平衡。

楊國忠的專權亂政比李林甫更甚,重用親信,排斥異己。天寶十二年(753年),關中大饑,因京兆尹李峴不甚順從,遂以災氣歸罪於李峴,貶李峴為長沙(今湖南長沙)太守。後來霖雨成災,玄宗過問災情,楊國忠取最好的禾苗給玄宗看,掩蓋災情真象。扶風太守房琯反映了所管地區的災情,楊國忠就派御史去追究他的責任。因此,天寶十三年(754年)雖然關中災情嚴重,但無人敢如實上報。連玄宗身邊的宦官高力士也說,楊國忠大權在握,賞罰不公,連他也不敢說話了。

范陽(今北京附近)節度使安祿山為了和楊國忠在玄宗面前爭寵,二人互相詆譭。玄宗對此搖擺不定,認為主要政事交付宰相,邊防事務交付諸將,無可憂慮。這樣一來,蓄謀已久的安祿山終於發動了反唐的大叛亂。

唐玄宗雖然沒有發動過像唐太宗、唐高宗朝時那樣的大規模的開邊軍事行動,但是他在位期間中原周邊地區與鄰近少數族吐蕃、契丹、南詔等的戰事連綿不斷。在邊疆軍事勝利的刺激下,玄宗日益滋長了他好大喜功的思想,寵愛有戰功的邊將。邊將也因此不停對外族開戰,以邀功賞。特別是李林甫為遏制政敵而拉邊將牛仙客入相後,更開放了蕃將以邊功為手段,窺伺中央政權的機會。

唐玄宗晚年驕奢淫逸,終日只顧與楊貴妃遊樂。他罢免良相张九龄,任用奸相李林甫,朝政每况愈下。玄宗本不太相信鬼神之說,後來崇信方士張果,漸好神仙;並尊奉道教,企慕長生不老,以是朝野爭言符瑞。李林甫死后,又以楊貴妃之從兄楊國忠担任丞相,李林甫在位時尚可穩住朝政,楊國忠不仅没有李林甫的才幹,反而縱容贪污腐败,局面遂不可收拾。不久,楊國忠与手握兵權的范陽节度使安祿山发生冲突,安祿山决心先發制人,發動叛变。

天寶十四年(755年),安祿山趁唐朝內部空虛腐敗,發動兵變,於時承平日久,民不知戰,河北州縣,望風瓦解。史稱安史之乱。玄宗決定逃往四川,途中至馬嵬驛,士兵譁變,士兵砍殺楊國忠,又逼玄宗賜死楊貴妃,玄宗權衡輕重下後,為了保命及維持君威,不得已下令高力士把楊貴妃勒死。

對玄宗早有不滿的太子李亨與玄宗分道揚鑣;李亨率一部份禁軍北趨靈武(今寧夏靈武西南),七月即位,改元至德,是為唐肅宗。李隆基與陳玄禮率另一部份禁軍南逃成都,後被尊為太上皇,玄宗長達44年的統治告終。

至德二年(757年)十二月,隨著安祿山被殺,郭子仪收复长安,玄宗由成都返回長安,居興慶宮(南內),奉玄宗為太上皇。乾元三年(760年)七月,宦官李輔國奉承肅宗,離間玄宗與肅宗的關係,迫使玄宗軟禁於太極宮(西內)甘露殿。高力士、陈玄礼等人被贬谪,玄宗浸不自怿、憂鬱寡歡。

寶應元年农历四月初五日(762年5月3日),太上皇唐玄宗李隆基崩逝於長安城太極宮甘露殿內,享壽七十六歲,在位四十四年。同年四月十八日(762年5月16日),久病未癒的唐肅宗李亨亦駕崩于长生殿,享年五十一歲,在位僅短短的六年。广德元年(763年)三月,將唐玄宗李隆基安葬於唐泰陵(今陝西省渭南市蒲城縣東北15公里處)。廟號玄宗,諡號至道大聖大明孝皇帝

開元年間,玄宗勵精圖治,任用賢臣,革除弊害,鼓勵生產,經濟發展,史稱「開元之治」。开元十四年(726年)杜甫《憶昔》有詩證:「憶昔開元全盛日,小邑猶藏萬家室。稻米流脂粟米白,公私倉廩俱豐實。九州道路無豺狼,遠行不勞吉日出。齊紈魯縞車班班,男耕女桑不相失。」

雖然玄宗後期怠政,但直到他在位四十三年的天寶十三年(754年),仍是唐代的極盛之世,全國有三百二十一郡,一千五百三十八縣,一萬六千八百二十九鄉,九百零六萬九千一百五十四戶,五千二百八十八萬四百八十八口。史載:“戶口之盛,極於此”。

唐玄宗富有音樂才華,对唐朝音乐发展有重大影响,他愛好親自演奏琵琶、羯鼓,擅長作曲,作有《霓裳羽衣曲》,《小破陣樂》,《春光好》,《秋風高》等百餘首樂曲。他曾选乐工,宫女在禁院梨园中歌舞,这是后来称戏班为“梨园”的由来。


\subsection{先天}

\begin{longtable}{|>{\centering\scriptsize}m{2em}|>{\centering\scriptsize}m{1.3em}|>{\centering}m{8.8em}|}
  % \caption{秦王政}\
  \toprule
  \SimHei \normalsize 年数 & \SimHei \scriptsize 公元 & \SimHei 大事件 \tabularnewline
  % \midrule
  \endfirsthead
  \toprule
  \SimHei \normalsize 年数 & \SimHei \scriptsize 公元 & \SimHei 大事件 \tabularnewline
  \midrule
  \endhead
  \midrule
  元年 & 712 & \tabularnewline\hline
  二年 & 713 & \tabularnewline
  \bottomrule
\end{longtable}

\subsection{开元}

\begin{longtable}{|>{\centering\scriptsize}m{2em}|>{\centering\scriptsize}m{1.3em}|>{\centering}m{8.8em}|}
  % \caption{秦王政}\
  \toprule
  \SimHei \normalsize 年数 & \SimHei \scriptsize 公元 & \SimHei 大事件 \tabularnewline
  % \midrule
  \endfirsthead
  \toprule
  \SimHei \normalsize 年数 & \SimHei \scriptsize 公元 & \SimHei 大事件 \tabularnewline
  \midrule
  \endhead
  \midrule
  元年 & 713 & \tabularnewline\hline
  二年 & 714 & \tabularnewline\hline
  三年 & 715 & \tabularnewline\hline
  四年 & 716 & \tabularnewline\hline
  五年 & 717 & \tabularnewline\hline
  六年 & 718 & \tabularnewline\hline
  七年 & 719 & \tabularnewline\hline
  八年 & 720 & \tabularnewline\hline
  九年 & 721 & \tabularnewline\hline
  十年 & 722 & \tabularnewline\hline
  十一年 & 723 & \tabularnewline\hline
  十二年 & 724 & \tabularnewline\hline
  十三年 & 725 & \tabularnewline\hline
  十四年 & 726 & \tabularnewline\hline
  十五年 & 727 & \tabularnewline\hline
  十六年 & 728 & \tabularnewline\hline
  十七年 & 729 & \tabularnewline\hline
  十八年 & 730 & \tabularnewline\hline
  十九年 & 731 & \tabularnewline\hline
  二十年 & 732 & \tabularnewline\hline
  二一年 & 733 & \tabularnewline\hline
  二二年 & 734 & \tabularnewline\hline
  二三年 & 735 & \tabularnewline\hline
  二四年 & 736 & \tabularnewline\hline
  二五年 & 737 & \tabularnewline\hline
  二六年 & 738 & \tabularnewline\hline
  二七年 & 739 & \tabularnewline\hline
  二八年 & 740 & \tabularnewline\hline
  二九年 & 741 & \tabularnewline
  \bottomrule
\end{longtable}

\subsection{天宝}

\begin{longtable}{|>{\centering\scriptsize}m{2em}|>{\centering\scriptsize}m{1.3em}|>{\centering}m{8.8em}|}
  % \caption{秦王政}\
  \toprule
  \SimHei \normalsize 年数 & \SimHei \scriptsize 公元 & \SimHei 大事件 \tabularnewline
  % \midrule
  \endfirsthead
  \toprule
  \SimHei \normalsize 年数 & \SimHei \scriptsize 公元 & \SimHei 大事件 \tabularnewline
  \midrule
  \endhead
  \midrule
  元年 & 742 & \tabularnewline\hline
  二年 & 743 & \tabularnewline\hline
  三年 & 744 & \tabularnewline\hline
  四年 & 745 & \tabularnewline\hline
  五年 & 746 & \tabularnewline\hline
  六年 & 747 & \tabularnewline\hline
  七年 & 748 & \tabularnewline\hline
  八年 & 749 & \tabularnewline\hline
  九年 & 750 & \tabularnewline\hline
  十年 & 751 & \tabularnewline\hline
  十一年 & 752 & \tabularnewline\hline
  十二年 & 753 & \tabularnewline\hline
  十三年 & 754 & \tabularnewline\hline
  十四年 & 755 & \tabularnewline\hline
  十五年 & 756 & \tabularnewline
  \bottomrule
\end{longtable}


%%% Local Variables:
%%% mode: latex
%%% TeX-engine: xetex
%%% TeX-master: "../Main"
%%% End:

%% -*- coding: utf-8 -*-
%% Time-stamp: <Chen Wang: 2021-10-29 15:50:59>

\section{肃宗李亨\tiny(756-762)}

\subsection{生平}

唐肃宗李亨(711年1月21日-762年5月16日),為唐玄宗李隆基第三子,母親為追尊元獻皇后楊貴嬪,唐朝第10代皇帝(不計武则天),756年8月12日—762年5月16日在位,在位6年。在位期間,是唐朝安史之亂時期。

唐睿宗景云二年(711年)九月三日(711年10月)出生在东宫之别殿。初名李嗣升,始封陝王,母親楊良媛懷孕時,父親唐玄宗身為太子,據說太平公主不滿,故玄宗一度想要楊氏墮胎,但玄宗炮製墮胎藥時睡著,天神託夢勸阻,幕僚張說得知,於是請求玄宗收回成命,後來玄宗登基,終於發動先天之變,賜死太平公主。

开元十三年(725年),一日早朝时,玄宗见李亨早衰,就在罢朝后驾临李亨府,见府中庭宇无人打扫,也无宫女使唤,就令高力士去京兆尹官衙,亟选民间女子入王府。高力士认为现在选秀過甚导致民间喧嚣,御史弹劾,不如在掖庭选取。如此选出三人,章敬皇后吴氏是其中一人。开元十四年(726年),吴氏生下他的长子李俶(即唐代宗)。开元十五年(727年),後徙封忠王,初改名為浚,後改名為璵。

开元二十六年(738年)皇太子李瑛為武惠妃所讒,貶庶人廢死;其被立為太子。他的妾室孺人韦氏被立为太子妃。当初唐玄宗欲立太子时,李林甫曾推举寿王李瑁。李亨成为太子后,李林甫惧怕,阴谋推翻太子。李林甫为构陷太子,诬陷太子妃兄韋堅,使得韦氏被迫与李亨离异,出家为尼。另一位妾室杜良娣的父亲杜有邻被赐死,杜良娣亦被废为庶人。

天寶三载(744年)改名為亨。安史之亂爆發,天寶十五载(756年)六月,鎮守潼關之大將哥舒翰受杨国忠逼迫出兵討叛,結果大敗,潼关陷落,长安震動,玄宗攜太子、寵妃倉皇逃往成都,行經馬嵬驛(今陕西省兴平市西),軍士譁變殺楊國忠,並逼迫玄宗賜死杨贵妃。馬嵬民眾攔阻玄宗請留,玄宗不從。太子李亨留下,隨即往朔方節度使所在地靈武(今寧夏靈武西南),同年农历七月十二日即位,尊玄宗為太上皇,改元至德,時年四十六歲,是為肅宗。

肅宗將郭子仪和李光弼部從河北召至靈武,並聯合回紇,開展大規模的反攻,並約定「克城之日,土地、士庶歸唐,金帛、子女皆歸回紇。」(縱容強姦搶奪)。至德二载(757年)正月,安祿山被其子安慶緒殺死。九月,郭子儀率唐軍和回紇騎兵收復長安,十二月太上皇玄宗回到长安。乾元元年(758年)九月,肅宗調動各路大軍進攻圍攻相州安慶緒,命宦官魚朝恩為觀軍容宣慰處置使,總攬全局。

玄宗回到长安后,厭惡張皇后與李輔國,常勸肅宗不要寵信他們。李輔國趁機構諂,說玄宗預謀復辟,故軟禁玄宗於西宮甘露殿,並流放高力士到巫州。上元元年(760年),玄宗被迫遷居西內太極宮,並於寶應元年農曆四月初五日(762年5月3日)病逝於西內的甘露殿。不久肅宗也一病不起,頒布詔令讓太子李豫監國。

唐肅宗在位僅六年,死於寶應元年四月十八日(762年5月16日),享年五十一歲。廟號肅宗,謚號文明武德大聖大宣孝皇帝,安葬於唐建陵(今陝西省咸陽市禮泉縣西北15公里處)。

安史之亂事直至唐代宗時,方完全平定,歷八年。

\subsection{至德}

\begin{longtable}{|>{\centering\scriptsize}m{2em}|>{\centering\scriptsize}m{1.3em}|>{\centering}m{8.8em}|}
  % \caption{秦王政}\
  \toprule
  \SimHei \normalsize 年数 & \SimHei \scriptsize 公元 & \SimHei 大事件 \tabularnewline
  % \midrule
  \endfirsthead
  \toprule
  \SimHei \normalsize 年数 & \SimHei \scriptsize 公元 & \SimHei 大事件 \tabularnewline
  \midrule
  \endhead
  \midrule
  元年 & 756 & \tabularnewline\hline
  二年 & 757 & \tabularnewline\hline
  三年 & 758 & \tabularnewline
  \bottomrule
\end{longtable}

\subsection{乾元}

\begin{longtable}{|>{\centering\scriptsize}m{2em}|>{\centering\scriptsize}m{1.3em}|>{\centering}m{8.8em}|}
  % \caption{秦王政}\
  \toprule
  \SimHei \normalsize 年数 & \SimHei \scriptsize 公元 & \SimHei 大事件 \tabularnewline
  % \midrule
  \endfirsthead
  \toprule
  \SimHei \normalsize 年数 & \SimHei \scriptsize 公元 & \SimHei 大事件 \tabularnewline
  \midrule
  \endhead
  \midrule
  元年 & 758 & \tabularnewline\hline
  二年 & 759 & \tabularnewline\hline
  三年 & 760 & \tabularnewline
  \bottomrule
\end{longtable}

\subsection{上元}

\begin{longtable}{|>{\centering\scriptsize}m{2em}|>{\centering\scriptsize}m{1.3em}|>{\centering}m{8.8em}|}
  % \caption{秦王政}\
  \toprule
  \SimHei \normalsize 年数 & \SimHei \scriptsize 公元 & \SimHei 大事件 \tabularnewline
  % \midrule
  \endfirsthead
  \toprule
  \SimHei \normalsize 年数 & \SimHei \scriptsize 公元 & \SimHei 大事件 \tabularnewline
  \midrule
  \endhead
  \midrule
  元年 & 760 & \tabularnewline\hline
  二年 & 761 & \tabularnewline
  \bottomrule
\end{longtable}

\subsection{宝应}

\begin{longtable}{|>{\centering\scriptsize}m{2em}|>{\centering\scriptsize}m{1.3em}|>{\centering}m{8.8em}|}
  % \caption{秦王政}\
  \toprule
  \SimHei \normalsize 年数 & \SimHei \scriptsize 公元 & \SimHei 大事件 \tabularnewline
  % \midrule
  \endfirsthead
  \toprule
  \SimHei \normalsize 年数 & \SimHei \scriptsize 公元 & \SimHei 大事件 \tabularnewline
  \midrule
  \endhead
  \midrule
  元年 & 762 & \tabularnewline\hline
  二年 & 763 & \tabularnewline
  \bottomrule
\end{longtable}


%%% Local Variables:
%%% mode: latex
%%% TeX-engine: xetex
%%% TeX-master: "../Main"
%%% End:

%% -*- coding: utf-8 -*-
%% Time-stamp: <Chen Wang: 2019-12-24 15:23:33>

\section{代宗\tiny(762-779)}

\subsection{生平}

唐代宗李豫(726年11月11日-779年6月10日),唐肃宗李亨的嫡长子。初名俶,小名冬郎,原封广平郡王,後改封廣平王、楚王、成王,天可汗,唐朝第11代皇帝(不计武则天),在位17年(762年5月18日-779年6月10日)。

据《旧唐书》,代宗以开元十四年十二月十三日(727年1月9日)生于東都洛阳上阳宫。据《册府元龟》,代宗以开元十四年十月十三日(726年11月11日)生于东都上阳宫,且多年之中的十月份都有庆祝活动(上寿)。似以十月生日更可信。唐代宗十月十三日天兴节,见令狐绹文集。

开元十四年十月十三日(726年11月11日)生于东都洛阳上阳宫,初名李俶。当时,父亲唐肃宗李亨为藩王,母亲吴氏是他的妾室。开元二十六年(738年),父亲李亨被立为太子。数年后,十五岁的李俶封广平郡王。天宝元年(742年),长子李适出生。天宝五载(746年),娶崔氏为妃。

天宝十五载(756年),安禄山叛军攻占潼关,祖父玄宗逃至马嵬驿,当地民众揽留肃宗。于是李俶护送肃宗北上灵武即帝位。安史之乱中,以兵马元帅名义收复洛阳、长安两京。乾元元年(758年)三月改封成王,四月被立為皇太子。

宝应元年(762年),宦官李辅国杀张皇后,肃宗受惊吓而死,李俶于肃宗灵柩前依其遗诏即位,改名豫。

次年,安史之亂结束,大唐开始走向衰落。当时,东部有诸多藩镇割据,北方又有邻国回鶻不断勒索,西面有邻国吐蕃侵扰。吐蕃甚至在廣德元年(763年)佔領首都長安十五日,立李承宏為帝,河西走廊亦被吐蕃佔領。

代宗迷信佛教,“有寇至则令僧讲《仁王经》以禳之,寇去则厚加赏赐”,宰相元载、王缙、杜鸿渐三人都信佛,以王缙尤甚。寺院多占有田地,“造金閣寺於五台山,鑄銅塗金為瓦,所費巨億”,朝廷政治经济进一步恶化。

大历十四年(779年)五月初二,宫中传出代宗生病的消息,結果一病不起,不到十天就无法上朝。五月辛酉(6月10日),下达了令皇太子監國的制书,当晚即在紫宸殿驾崩,享年五十三歲。死後葬于元陵(今陕西省富平县西北三十里的檀山),谥号睿文孝武皇帝,廟號代宗。太子李适繼位,是為唐德宗。

由於李豫平定安史之亂,有不世之功,其廟號原議定為世宗,但為避太宗李世民諱,最終定為代宗意為世代。

\subsection{广德}

\begin{longtable}{|>{\centering\scriptsize}m{2em}|>{\centering\scriptsize}m{1.3em}|>{\centering}m{8.8em}|}
  % \caption{秦王政}\
  \toprule
  \SimHei \normalsize 年数 & \SimHei \scriptsize 公元 & \SimHei 大事件 \tabularnewline
  % \midrule
  \endfirsthead
  \toprule
  \SimHei \normalsize 年数 & \SimHei \scriptsize 公元 & \SimHei 大事件 \tabularnewline
  \midrule
  \endhead
  \midrule
  元年 & 763 & \tabularnewline\hline
  二年 & 764 & \tabularnewline
  \bottomrule
\end{longtable}

\subsection{永泰}

\begin{longtable}{|>{\centering\scriptsize}m{2em}|>{\centering\scriptsize}m{1.3em}|>{\centering}m{8.8em}|}
  % \caption{秦王政}\
  \toprule
  \SimHei \normalsize 年数 & \SimHei \scriptsize 公元 & \SimHei 大事件 \tabularnewline
  % \midrule
  \endfirsthead
  \toprule
  \SimHei \normalsize 年数 & \SimHei \scriptsize 公元 & \SimHei 大事件 \tabularnewline
  \midrule
  \endhead
  \midrule
  元年 & 765 & \tabularnewline\hline
  二年 & 766 & \tabularnewline
  \bottomrule
\end{longtable}

\subsection{大历}

\begin{longtable}{|>{\centering\scriptsize}m{2em}|>{\centering\scriptsize}m{1.3em}|>{\centering}m{8.8em}|}
  % \caption{秦王政}\
  \toprule
  \SimHei \normalsize 年数 & \SimHei \scriptsize 公元 & \SimHei 大事件 \tabularnewline
  % \midrule
  \endfirsthead
  \toprule
  \SimHei \normalsize 年数 & \SimHei \scriptsize 公元 & \SimHei 大事件 \tabularnewline
  \midrule
  \endhead
  \midrule
  元年 & 766 & \tabularnewline\hline
  二年 & 767 & \tabularnewline\hline
  三年 & 768 & \tabularnewline\hline
  四年 & 769 & \tabularnewline\hline
  五年 & 770 & \tabularnewline\hline
  六年 & 771 & \tabularnewline\hline
  七年 & 772 & \tabularnewline\hline
  八年 & 773 & \tabularnewline\hline
  九年 & 774 & \tabularnewline\hline
  十年 & 775 & \tabularnewline\hline
  十一年 & 776 & \tabularnewline\hline
  十二年 & 777 & \tabularnewline\hline
  十三年 & 778 & \tabularnewline\hline
  十四年 & 779 & \tabularnewline
  \bottomrule
\end{longtable}


%%% Local Variables:
%%% mode: latex
%%% TeX-engine: xetex
%%% TeX-master: "../Main"
%%% End:

%% -*- coding: utf-8 -*-
%% Time-stamp: <Chen Wang: 2021-10-29 16:30:20>

\section{德宗李适\tiny(779-805)}

\subsection{生平}

唐德宗李适(音同括,742年5月27日-805年2月25日),唐代宗與睿真皇后所生的长子,唐朝第12代皇帝(除去武则天以外),779年6月12日―805年2月25日在位,在位26年,享壽62岁。谥号为神武孝文皇帝。

李适於唐玄宗天宝元年四月十九日(742年5月27日),生于长安大内宫中。当时,父亲唐代宗为广平郡王,母亲沈氏是他的妾室,出生后八个月便被封为奉节郡王。嫡母崔氏的母族在安史之乱中失势,自己的母亲沈氏亦在期间失踪。宝应元年(762年),父亲唐代宗继位。广德二年(764年)被立为皇太子。在一般认知中,唐德宗是以庶长子的身份被立为太子的,但实际上在《让皇太子表》中,唐德宗自称为唐代宗的嫡长子,在让表中曾向代宗表示不应该只因为嫡长身份就立自己为皇太子,唐代宗应该效法三皇五帝选择贤德之人。

大历十四年(779年)37歲即位。次年,为了改善财政,采纳宰相楊炎建议,废除庸调制,颁行“两税法”。

执政前期一改代宗朝姑息藩镇的弊政,坚决削弱藩镇割据,加强中央集权,但由于措置失当,镇抚乖方,往往在消灭旧叛之后又激起一批原本忠于朝廷的节度使的叛乱。

建中四年(783年),因朝廷赏赐有失公平,激起泾原兵变,德宗仓皇出逃到奉天(今陕西乾县),早年入朝面圣却被软禁在京城的前卢龙节度使朱泚在哗变军士的拥护下称帝,改国号为秦。泾原之变时,长期讨逆伐叛的朔方节度使李懷光因奸臣卢杞畏罪挑拨而见疑于德宗,自危之下铤而走险加入了叛乱阵营。唐中兴名将李晟经历艰难险阻,终于击破朱泚、李怀光联军,德宗才得以重返长安。 德宗前期,坚持信用文武百官,严禁宦官干政,颇有一番中兴气象;但泾原兵变后,文官武将的相继失节与宦官集团的忠心护驾所形成的强烈反差,使德宗彻底放弃了以往的观念,改元贞元后,德宗委任宦官为禁军统帅,掌握监军以防兵变,宦官监军的制度也开始确立。

晚年的德宗日益变得贪婪自私,不但经常把国库赋税收入划拨到自己的内帑,还纵容在外宦官强令地方官进奉贡物,甚至在长安施行宫市以充实自己的小金库。为弥补中央财政,德宗在全国范围内增收茶叶等杂税,导致民怨日深。

德宗在位时期,对外联合回纥、南诏,打击吐蕃,成功扭转对吐蕃的战略劣势,为唐宪宗的元和中兴创造了较为有利的外部环境。

唐德宗于贞元二十一年(805年)逝世,終年63歲。

德宗本人喜好理辩,往往喜欢就某件政务反复讨论诘难,无疑而后实行。因此他曾对李泌比较过前三任宰相的优劣:崔祐甫性格急躁,每當德宗責難他時便應對失態,但德宗因知其缺點而加以迴護;杨炎有才但為人倨傲,面對德宗責難時动辄忿然作色,“無復君臣之禮”,若德宗不批准楊炎的上奏,楊炎便以辭官要挾,数年后德宗还忿然回忆道:“楊炎視朕如三尺童子”;卢杞謹慎但沒有學識,面对德宗的质疑面色如常,唯唯称是,却無法與德宗反覆應對,使德宗無法理清心中疑惑。德宗认为李泌沒有三人的缺點,既能与皇帝平心静气地辩论,又有自己独特的政见,以事論事,往往让德宗真心信服“如此則理安,如彼則危亂”,不得不从其议。


\subsection{建中}

\begin{longtable}{|>{\centering\scriptsize}m{2em}|>{\centering\scriptsize}m{1.3em}|>{\centering}m{8.8em}|}
  % \caption{秦王政}\
  \toprule
  \SimHei \normalsize 年数 & \SimHei \scriptsize 公元 & \SimHei 大事件 \tabularnewline
  % \midrule
  \endfirsthead
  \toprule
  \SimHei \normalsize 年数 & \SimHei \scriptsize 公元 & \SimHei 大事件 \tabularnewline
  \midrule
  \endhead
  \midrule
  元年 & 780 & \tabularnewline\hline
  二年 & 781 & \tabularnewline\hline
  三年 & 782 & \tabularnewline\hline
  四年 & 783 & \tabularnewline
  \bottomrule
\end{longtable}

\subsection{兴元}

\begin{longtable}{|>{\centering\scriptsize}m{2em}|>{\centering\scriptsize}m{1.3em}|>{\centering}m{8.8em}|}
  % \caption{秦王政}\
  \toprule
  \SimHei \normalsize 年数 & \SimHei \scriptsize 公元 & \SimHei 大事件 \tabularnewline
  % \midrule
  \endfirsthead
  \toprule
  \SimHei \normalsize 年数 & \SimHei \scriptsize 公元 & \SimHei 大事件 \tabularnewline
  \midrule
  \endhead
  \midrule
  元年 & 784 & \tabularnewline
  \bottomrule
\end{longtable}

\subsection{贞元}

\begin{longtable}{|>{\centering\scriptsize}m{2em}|>{\centering\scriptsize}m{1.3em}|>{\centering}m{8.8em}|}
  % \caption{秦王政}\
  \toprule
  \SimHei \normalsize 年数 & \SimHei \scriptsize 公元 & \SimHei 大事件 \tabularnewline
  % \midrule
  \endfirsthead
  \toprule
  \SimHei \normalsize 年数 & \SimHei \scriptsize 公元 & \SimHei 大事件 \tabularnewline
  \midrule
  \endhead
  \midrule
  元年 & 785 & \tabularnewline\hline
  二年 & 786 & \tabularnewline\hline
  三年 & 787 & \tabularnewline\hline
  四年 & 788 & \tabularnewline\hline
  五年 & 789 & \tabularnewline\hline
  六年 & 790 & \tabularnewline\hline
  七年 & 791 & \tabularnewline\hline
  八年 & 792 & \tabularnewline\hline
  九年 & 793 & \tabularnewline\hline
  十年 & 794 & \tabularnewline\hline
  十一年 & 795 & \tabularnewline\hline
  十二年 & 796 & \tabularnewline\hline
  十三年 & 797 & \tabularnewline\hline
  十四年 & 798 & \tabularnewline\hline
  十五年 & 799 & \tabularnewline\hline
  十六年 & 800 & \tabularnewline\hline
  十七年 & 801 & \tabularnewline\hline
  十八年 & 802 & \tabularnewline\hline
  十九年 & 803 & \tabularnewline\hline
  二十年 & 804 & \tabularnewline\hline
  二一年 & 805 & \tabularnewline
  \bottomrule
\end{longtable}


%%% Local Variables:
%%% mode: latex
%%% TeX-engine: xetex
%%% TeX-master: "../Main"
%%% End:

%% -*- coding: utf-8 -*-
%% Time-stamp: <Chen Wang: 2019-12-24 15:26:35>

\section{顺宗\tiny(805)}

唐顺宗李诵(761年2月21日-806年2月11日),唐德宗长子,唐朝第13代皇帝(除武则天以外),805年在位。

唐顺宗也是唐朝唯一一位以先帝嫡长子身份成为皇太子并最终继位的皇帝

上元二年(761年)陰曆正月十二日(阳历761年2月21日),生於長安之東內。初封宣城郡王。大历十四年(779年)唐代宗崩,唐德宗继位,六月,李诵进封宣王,十二月乙卯,立为皇太子。贞元二十一年(805年)正月即位,改元永贞。任用東宮舊人王伾、王叔文为翰林学士,在宰相韋執誼、韩泰、韩晔、柳宗元、刘禹锡、陈谏、凌准、程异等人支持下,从事改革德宗以来的弊政,贬斥贪官,废除宫市,停止盐铁进钱和地方进奉,并试图收回宦官兵权,史称“永贞革新”。

顺宗即位時已患中风,喑啞不能言,詔令皆出牛昭容手。同年八月,宦官俱文珍等勾结部分官僚和藩镇,逼其退位,传位于太子李纯,贬王伾等人,史称“永貞內禪”(顺宗朝僅歷七月,舊制逾年改元,內禪暨革新失敗之後,方以太上皇詔,改元永貞)。又贬斥韩泰等八人,史称“二王八司马”。次年元和元年陰曆正月十九日(806年2月11日),崩於興慶宮之咸寧殿。官方說法為病死。野史認為順宗是被宦官謀殺而死,其事透過唐人傳奇《辛公平上仙》的影射以流傳後世。

死后谥号为至德弘道大圣大安孝皇帝。

\subsection{永贞}

\begin{longtable}{|>{\centering\scriptsize}m{2em}|>{\centering\scriptsize}m{1.3em}|>{\centering}m{8.8em}|}
  % \caption{秦王政}\
  \toprule
  \SimHei \normalsize 年数 & \SimHei \scriptsize 公元 & \SimHei 大事件 \tabularnewline
  % \midrule
  \endfirsthead
  \toprule
  \SimHei \normalsize 年数 & \SimHei \scriptsize 公元 & \SimHei 大事件 \tabularnewline
  \midrule
  \endhead
  \midrule
  元年 & 805 & \tabularnewline
  \bottomrule
\end{longtable}


%%% Local Variables:
%%% mode: latex
%%% TeX-engine: xetex
%%% TeX-master: "../Main"
%%% End:

%% -*- coding: utf-8 -*-
%% Time-stamp: <Chen Wang: 2021-10-29 16:30:49>

\section{宪宗李纯\tiny(805-820)}

\subsection{生平}

唐宪宗李纯(778年3月17日-820年2月14日),原名李淳,唐朝第14代皇帝(除去武则天以外),805年—820年在位。唐憲宗透過宦官俱文珍等人的協助,迫使父親唐順宗讓位予自己,即“永貞內禪”。即位後,曾一度討平不服朝廷的藩鎮,短暫終結藩鎮割據,重新統一中國,史稱“元和中興”,也因此成為唐朝繼唐太宗、唐玄宗以後,歷史評價相當高的皇帝。

唐代宗大曆十三年(778年)陰曆二月十四日,李淳(唐憲宗)在長安之東內(大明宫)內出生。

當时的皇帝是曾祖父唐代宗,祖父唐德宗为皇太子。父唐顺宗为郡王,母為莊憲皇后王氏,为妾室。

貞元四年(788年),封為广陵郡王。

貞元九年(793年)十一月丁卯,纳妃郭氏。同年,长子李宁出生。

貞元二十一年(805年)初,父亲唐顺宗繼位,唐顺宗重用王叔文、韋執誼、柳宗元、劉禹錫等官員,王叔文集團试图进行永貞改革,抑制宦官勢力。

但當時唐順宗有中風、癱瘓的情況。不滿改革的宦官俱文珍、劉光琦等人聯合劍南節度使韋皋、荊南節度使裴均、河東節度使嚴綬等外藩,迫使唐順宗立李淳為太子,改名李純,七月,太子監國。八月,俱文珍又迫唐順宗退位为上皇,傳位予長子唐憲宗,史稱永貞內禪。王叔文等二王八司馬皆被貶謫。

元和元年正月,上皇駕崩,官方說法為病死。野史影射順宗是被宦官謀殺而死,可見於唐人傳奇《辛公平上仙》。

唐宪宗继位后,決心“以法度裁制藩鎮”,开始对割据的藩镇开展了一系列战争,他在继位的次年就开始对西川节度副使刘闢开战获胜,同年夏綏軍留後杨惠琳不肯交出他的兵权,宪宗也征討他,惠琳败死。

元和二年(807年),讨伐鎮海軍節度使李錡。

元和七年(812年),魏博节度使田兴歸服唐朝,同年他开始对抗拒唐朝的成德節度使王承宗作战,但没有能够获胜,从元和十年(815年)到元和十二年(817年)唐鄧節度使李愬平定了淮西吳元濟的叛乱。

元和十三年(818年),發五道兵討淄青節度使李師道。吴元济被平定後,全国所有藩镇至少名义上全部歸服唐朝。唐朝出現短暫統一,“至是盡遵朝廷約束。”史稱“元和中兴”。

元和十四年(819年)正月,憲宗遣使往鳳翔迎釋迦牟尼佛遺骨入宮供奉,刑部侍郎韩愈上「論佛骨表」勸諫,言語不敬,皇帝大怒,差點處死了韓愈,不過最後只將韓貶为潮州刺史。這次迎佛骨陣容浩大,《資治通鑒》載“中使迎佛骨至京師,上留禁中三日,乃歷送諸寺。王公士民瞻奉舍施,唯恐不及。有謁戶充施者,有燃香燒頂供養者。”

宪宗的帝位是由宦官擁立的,因此宪宗重用宦官,军队中许多将領與監軍由宦官担任,有些宦官拥有很高的军权,但宪宗對宦官亦不優待,其晚年好长生不老之術,多服金丹,“日加躁渴”,性情暴躁易怒,動輒責罰左右黃門,宦官們不堪鞭笞。

元和十五年(820年)陰曆正月二十七日,宪宗暴卒,据说是被宦官內常侍陈弘志和王守澄合謀毒死。享年42岁,在位15年,谥圣神章武孝皇帝。大中三年,加谥昭文章武大圣至神孝皇帝。

王夫之在其《讀通鑑論》中推理认为宪宗之暴毙实则是郭氏(穆宗生母)与穆宗纵逆之所为。

\subsection{元和}

\begin{longtable}{|>{\centering\scriptsize}m{2em}|>{\centering\scriptsize}m{1.3em}|>{\centering}m{8.8em}|}
  % \caption{秦王政}\
  \toprule
  \SimHei \normalsize 年数 & \SimHei \scriptsize 公元 & \SimHei 大事件 \tabularnewline
  % \midrule
  \endfirsthead
  \toprule
  \SimHei \normalsize 年数 & \SimHei \scriptsize 公元 & \SimHei 大事件 \tabularnewline
  \midrule
  \endhead
  \midrule
  元年 & 806 & \tabularnewline\hline
  二年 & 807 & \tabularnewline\hline
  三年 & 808 & \tabularnewline\hline
  四年 & 809 & \tabularnewline\hline
  五年 & 810 & \tabularnewline\hline
  六年 & 811 & \tabularnewline\hline
  七年 & 812 & \tabularnewline\hline
  八年 & 813 & \tabularnewline\hline
  九年 & 814 & \tabularnewline\hline
  十年 & 815 & \tabularnewline\hline
  十一年 & 816 & \tabularnewline\hline
  十二年 & 817 & \tabularnewline\hline
  十三年 & 818 & \tabularnewline\hline
  十四年 & 819 & \tabularnewline\hline
  十五年 & 820 & \tabularnewline
  \bottomrule
\end{longtable}


%%% Local Variables:
%%% mode: latex
%%% TeX-engine: xetex
%%% TeX-master: "../Main"
%%% End:

%% -*- coding: utf-8 -*-
%% Time-stamp: <Chen Wang: 2019-12-24 15:30:09>

\section{穆宗\tiny(820-824)}

\subsection{生平}

唐穆宗李恒(795年7月26日-824年2月25日),原名宥。唐宪宗第三子,母懿安皇后郭氏。他是唐朝第15代皇帝(除去武则天以外),在位4年。

唐德宗貞元十一年七月六日(795年7月26日)生於大明宮之別殿,初名李宥。当时的皇帝是他的曾祖父唐德宗,父亲唐宪宗时为广陵郡王,母亲郭氏为广陵郡王妃。贞元二十一年(805年)初,祖父唐顺宗即位后,三月,父亲李纯被立为太子。夏四月庚戌,李宥与其他兄弟五人同封郡王,食邑三千户。

其父唐宪宗在805年登基后,母亲郭氏做为唐宪宗的嫡妻却未能立为皇后,只是被封为贵妃。其后,在元和元年(806年)八月,李宥进封遂王。元和四年(809年),他的异母长兄邓王李宁被册为太子。元和五年(810年)三月,李宥领彰义军节度大使。元和六年闰十二月廿一日(812年2月7日),太子李宁逝世。元和七年(812年)十月,李宥被立为皇太子,改名恒。

元和十五年正月庚子(820年2月14日),唐宪宗暴卒,疑似被宦官陳弘志、王守澄下毒謀害。宦官梁守謙等擁立李恒,丙午(2月20日)登基,是为唐穆宗。唐穆宗在位期间“宴樂過多,畋遊無度”,“不留意天下之務”。任用的宰相萧俛、段文昌又无远见,认为藩镇已平,应当消兵。于是令天下军镇有兵处每年在100人中限八人或逃或死,消其兵籍。被取消兵籍的军士无处可去,又無法從事他業,只好藏於山林。不久河朔三鎮復叛,躲藏的军士纷纷归附三鎮。

朝廷内宦官权势日盛,官僚朋党斗争剧烈。使唐宪宗時期的“元和中兴”局面完全丧失。好服金石之藥,長慶二年十一月,一次他与宦官内臣打马球时,穆宗突然中風,长庆四年陰曆正月二十二日,崩於寢殿。同年十一月,葬於光陵。死后谥号为睿圣文惠孝皇帝。

\subsection{永新}

\begin{longtable}{|>{\centering\scriptsize}m{2em}|>{\centering\scriptsize}m{1.3em}|>{\centering}m{8.8em}|}
  % \caption{秦王政}\
  \toprule
  \SimHei \normalsize 年数 & \SimHei \scriptsize 公元 & \SimHei 大事件 \tabularnewline
  % \midrule
  \endfirsthead
  \toprule
  \SimHei \normalsize 年数 & \SimHei \scriptsize 公元 & \SimHei 大事件 \tabularnewline
  \midrule
  \endhead
  \midrule
  元年 & 820 & \tabularnewline
  \bottomrule
\end{longtable}

\subsection{长庆}

\begin{longtable}{|>{\centering\scriptsize}m{2em}|>{\centering\scriptsize}m{1.3em}|>{\centering}m{8.8em}|}
  % \caption{秦王政}\
  \toprule
  \SimHei \normalsize 年数 & \SimHei \scriptsize 公元 & \SimHei 大事件 \tabularnewline
  % \midrule
  \endfirsthead
  \toprule
  \SimHei \normalsize 年数 & \SimHei \scriptsize 公元 & \SimHei 大事件 \tabularnewline
  \midrule
  \endhead
  \midrule
  元年 & 821 & \tabularnewline\hline
  二年 & 822 & \tabularnewline\hline
  三年 & 823 & \tabularnewline\hline
  四年 & 824 & \tabularnewline
  \bottomrule
\end{longtable}


%%% Local Variables:
%%% mode: latex
%%% TeX-engine: xetex
%%% TeX-master: "../Main"
%%% End:

%% -*- coding: utf-8 -*-
%% Time-stamp: <Chen Wang: 2021-10-29 16:46:26>

\section{敬宗李湛\tiny(824-826)}

\subsection{生平}

唐敬宗李湛(809年-827年),唐朝皇帝。唐穆宗长子。他是唐朝第16代皇帝(除去武则天以外),15歲即位,824年—827年在位,在位3年,得年17岁。

元和四年六月七日,生于东内(大明宫)之别殿,父亲唐穆宗时为遂王。母王氏是他的妾室。元和十五年,祖父唐宪宗逝世,父亲继位,是为唐穆宗。长庆元年(821年)三月,封景王。二年十二月,立为皇太子。四年正月壬申,父亲穆宗逝世。癸酉,李湛即位柩前,时年十五。

即位后,奢侈荒淫。沉迷擊鞠(古代馬球),喜歡半夜在宮中捉狐狸(打夜狐),史称“视朝月不再三,大臣罕得进见”。宦官王守澄把持朝政,勾结权臣李逢吉,排斥异己,败坏纲纪。导致官府工匠突起暴动攻入宫廷的事件。寶曆二年十二月初八为宦官刘克明等人杀害,死后谥号为睿武昭愍孝皇帝。


\subsection{宝历}

\begin{longtable}{|>{\centering\scriptsize}m{2em}|>{\centering\scriptsize}m{1.3em}|>{\centering}m{8.8em}|}
  % \caption{秦王政}\
  \toprule
  \SimHei \normalsize 年数 & \SimHei \scriptsize 公元 & \SimHei 大事件 \tabularnewline
  % \midrule
  \endfirsthead
  \toprule
  \SimHei \normalsize 年数 & \SimHei \scriptsize 公元 & \SimHei 大事件 \tabularnewline
  \midrule
  \endhead
  \midrule
  元年 & 825 & \tabularnewline\hline
  二年 & 826 & \tabularnewline\hline
  三年 & 827 & \tabularnewline
  \bottomrule
\end{longtable}


%%% Local Variables:
%%% mode: latex
%%% TeX-engine: xetex
%%% TeX-master: "../Main"
%%% End:

%% -*- coding: utf-8 -*-
%% Time-stamp: <Chen Wang: 2021-10-29 16:47:00>

\section{文宗李昂\tiny(826-840)}

\subsection{生平}

唐文宗李昂(809年11月20日-840年2月10日),原名涵,唐穆宗第二子,母侍女萧氏。唐敬宗之弟。他是唐朝第17代皇帝(除去武则天以外),827年—840年在位,在位13年,享年30岁。

元和四年十月十日生,以其日為慶成節。長慶元年,封為江王。宝历二年(826年),唐敬宗荒淫無道,虐待宦官劉克明等人,劉克明於是刺殺敬宗,立絳王李悟為帝。宰相裴度、執金吾梁守謙、樞密使王守澄(宦官)等率神策軍攻入朝廷,殺李悟,改立江王李涵,李涵改名为李昂,是為唐文宗。

文宗在位期间勵精圖治,資遣宫女三千人,罷免官员一千二百餘人。朝臣朋党相互倾轧,官员调动频繁,牛李党争达到高潮。后起用李训、郑注等人,意欲铲除宦官。

太和五年(831年),唐文宗与宰相宋申锡暗中密谋除掉宦官,但是被宦官王守澄及其门客探听出来,诬告宋申锡谋立漳王李凑。唐文宗中计,宋申锡被赐死。太和九年(835年),文宗终于杀死王守澄。王守澄死后仅一个月,李训引诱仇士良等宦官往左金吾卫衙中取石榴树上的“甘露”,企图将其一举消灭,但事情败露,导致仇士良等宦官大肆屠杀朝臣一千餘人,史称“甘露之变”。事后,文宗更被宦官钳制,對當值學士周墀慨叹自己受制于家奴,境遇不如周赧王、汉献帝,不禁淒然淚下。周墀聽了也伏地流涕。

唐文宗时期,藩镇叛乱依旧频繁。

开成五年(840年)文宗抑郁成病,正月初四病死在大明宮中的太和殿,葬於章陵,死后谥号为元圣昭献孝皇帝。

太子李永死后,文宗曾立敬宗幼子陈王李成美为太子,但未行册礼就病重了,临终时托孤于宰相杨嗣复、李珏,但当权宦官仇士良、鱼弘志因太子不是自己力主所立,矫诏仍废太子为陈王,改立文宗弟颍王李瀍为皇太弟,文宗死后,二人说服李瀍逼令李成美自杀。李瀍继位,就是唐武宗。

唐文宗为庄恪太子李永选妃时,朝廷大臣的女儿们都进入了挑选名单之中,朝廷内外因此动荡不安。唐文宗得知后对宰相郑覃说:“我希望为太子求娶你们荥阳郑氏有礼数的女子为妻,听说在外的大臣们都不愿与我做亲戚,这是為甚麼呢?我家也是幾百年的世家大族,怎麼把神堯皇帝的後人當作佛家羅漢(不願締結親事)呢?”唐文宗于是放弃了选太子妃的计划。不久郑覃把孙女嫁给了一位「九品芝麻官」崔皋,唐文宗无可奈何地说:“民间缔结婚姻,不计较官品却崇尚门第。我家已做了二百年的天子,还比不上崔氏和卢氏吗?”

陈寅恪认为李唐数百年的天子门户还比不上山东旧族九品卫佐的崔皋,可以想见唐朝山东世族心目中两者社会价值的差距,李唐皇室出自关陇胡汉集团,与山东士族以礼法为门风的家法大有不同,李唐汉化程度较深后,与旧有的士族相比自觉相形见绌,越发仰慕,贵为天子也不能胜过山东世族九品卫佐的崔皋,说明山东旧族的自我高标准并非没有原因。


\subsection{大和}

\begin{longtable}{|>{\centering\scriptsize}m{2em}|>{\centering\scriptsize}m{1.3em}|>{\centering}m{8.8em}|}
  % \caption{秦王政}\
  \toprule
  \SimHei \normalsize 年数 & \SimHei \scriptsize 公元 & \SimHei 大事件 \tabularnewline
  % \midrule
  \endfirsthead
  \toprule
  \SimHei \normalsize 年数 & \SimHei \scriptsize 公元 & \SimHei 大事件 \tabularnewline
  \midrule
  \endhead
  \midrule
  元年 & 827 & \tabularnewline\hline
  二年 & 828 & \tabularnewline\hline
  三年 & 829 & \tabularnewline\hline
  四年 & 830 & \tabularnewline\hline
  五年 & 831 & \tabularnewline\hline
  六年 & 832 & \tabularnewline\hline
  七年 & 833 & \tabularnewline\hline
  八年 & 834 & \tabularnewline\hline
  九年 & 835 & \tabularnewline
  \bottomrule
\end{longtable}

\subsection{开成}

\begin{longtable}{|>{\centering\scriptsize}m{2em}|>{\centering\scriptsize}m{1.3em}|>{\centering}m{8.8em}|}
  % \caption{秦王政}\
  \toprule
  \SimHei \normalsize 年数 & \SimHei \scriptsize 公元 & \SimHei 大事件 \tabularnewline
  % \midrule
  \endfirsthead
  \toprule
  \SimHei \normalsize 年数 & \SimHei \scriptsize 公元 & \SimHei 大事件 \tabularnewline
  \midrule
  \endhead
  \midrule
  元年 & 836 & \tabularnewline\hline
  二年 & 837 & \tabularnewline\hline
  三年 & 838 & \tabularnewline\hline
  四年 & 839 & \tabularnewline\hline
  五年 & 840 & \tabularnewline
  \bottomrule
\end{longtable}


%%% Local Variables:
%%% mode: latex
%%% TeX-engine: xetex
%%% TeX-master: "../Main"
%%% End:

%% -*- coding: utf-8 -*-
%% Time-stamp: <Chen Wang: 2021-10-29 16:47:11>

\section{武宗李瀍\tiny(840-846)}

\subsection{生平}

唐武宗李瀍(814年7月2日-846年4月22日)(「瀍」,拼音:chán),临死前12天改名“炎”,唐穆宗的第五子和事实上的第九子,母韦贵妃。他是唐朝的第18代皇帝(除去武则天以外),在位时间是840年至846年,在位6年,享年31岁。

唐武宗本来是唐敬宗、唐文宗的弟弟,被封为颍王。在宦官仇士良的操纵下,趁文宗病,矫诏立他为皇太弟,废原来的太子敬宗子李成美为陈王,武宗由此得以登基,并赐死李成美、文宗杨贤妃和皇兄安王李溶。

唐武宗登基後,召李黨人物李德裕回朝,任為宰相,李德裕提倡“政歸中書”等政策。在李德裕執政下,國家漸漸回復元氣,被稱為會昌中興。而仇士良的權勢亦被壓抑,仇士良不得不退下政治舞台。

唐武宗外攘回纥,内平泽潞,威震中外;更严肃整顿吏治,裁汰冗官,制驭宦官,使朝政为之一新。

唐武宗信奉道教,从845年开始他大规模下令打击佛教,史称会昌灭法。除少数在長安的寺院外,全国所有寺院被拆毁,僧尼被迫还俗,寺院所有的田地被没收为国有。

这是中国历史上佛教受打击很激烈的一次。在唐朝历史上对于佛教势力不满的现象始终存在,武宗灭佛可能有多种原因。第一可能因为唐武宗本人更加信奉道教,因此打击佛教。此外当时佛教的势力非常强大,唐武宗在他的旨意中说,佛教寺院的规模比皇宫还要大,寺院不纳税,对国家财务稅收是一个重大损失。

最后有传说认为唐武宗继位后怕有人会另立他的叔叔光王李忱(即后来的唐宣宗)来威胁他的地位,李忱则逃入佛门,因此唐武宗灭佛是为了让李忱无处可藏。但这个说法可能只是传说,因为历史学家对于李忱是否真的做过和尚仍有争议。

唐武宗吃道士给他的长寿丹后中毒而死。死后葬于端陵,谥号为至道昭肃孝皇帝。

唐武宗虽有五子皆封王,但生前未确立继承人,宦官马元贽等遂矫诏立光王李忱为皇太叔并最终继位,即唐宣宗。武宗五子后事无载,一说皆被宣宗所害。


\subsection{会昌}

\begin{longtable}{|>{\centering\scriptsize}m{2em}|>{\centering\scriptsize}m{1.3em}|>{\centering}m{8.8em}|}
  % \caption{秦王政}\
  \toprule
  \SimHei \normalsize 年数 & \SimHei \scriptsize 公元 & \SimHei 大事件 \tabularnewline
  % \midrule
  \endfirsthead
  \toprule
  \SimHei \normalsize 年数 & \SimHei \scriptsize 公元 & \SimHei 大事件 \tabularnewline
  \midrule
  \endhead
  \midrule
  元年 & 841 & \tabularnewline\hline
  二年 & 842 & \tabularnewline\hline
  三年 & 843 & \tabularnewline\hline
  四年 & 844 & \tabularnewline\hline
  五年 & 845 & \tabularnewline\hline
  六年 & 846 & \tabularnewline
  \bottomrule
\end{longtable}


%%% Local Variables:
%%% mode: latex
%%% TeX-engine: xetex
%%% TeX-master: "../Main"
%%% End:

%% -*- coding: utf-8 -*-
%% Time-stamp: <Chen Wang: 2021-10-29 16:47:22>

\section{宣宗李忱\tiny(846-859)}

\subsection{生平}

唐宣宗李忱(810年-859年),唐朝第19代皇帝(846年—859年在位,未算武周政权),初名怡,登基之前封為光王,在位13年。唐宪宗李纯十三子,母鄭宮人,元和五年(810年)六月廿二日生於大明宫,是唐穆宗李恒的弟弟,唐敬宗李湛、唐文宗李昂、唐武宗李炎的叔叔。

唐宣宗李忱原名李怡,母親郑氏原为镇海节度使李錡侍妾,李锜謀反失敗後,郑氏被送入宫后当郭貴妃的侍儿,后来被唐憲宗临幸,生下李忱,封為光王,故唐宣宗為唐穆宗之弟,唐敬宗、唐文宗、唐武宗之叔父。唐宣宗為光王時,居於十六宅,故作愚鈍,曾被唐文宗及其他宗室作弄。

傳說唐宣宗登基之前,为了逃避姪唐武宗的迫害而出家為僧,传说他在河南淅川香严寺避难,法名琼俊,齐安见其举止不凡。

会昌六年(846年),唐武宗被道士上供的长寿丹毒死。光王李怡被宦官擁立為帝,改名為李忱,是为唐宣宗,年号大中。擁立唐宣宗的宦官本以為他愚鈍容易控制,豈料他登基為帝後立即勵精圖治,並贬谪李德裕,结束牛李党爭。眾宦官、朝臣及宗室才驚覺唐宣宗以往是故作愚鈍,實際是非常賢明,有如其父唐憲宗。

唐宣宗登基后,唐朝国势已暮氣沉沉,藩鎮割據,牛李党争,农民起义,朝政腐败,官吏贪污,宦官专权,四夷不朝。唐宣宗致力于改变这种状况,宣宗勤俭治国,体贴百姓,减少赋税,注重人才选拔,唐朝国势有所起色,社会矛盾有所缓和,百姓日渐富裕,使十分腐败的唐朝呈现出“中兴”的小康局面,史稱大中之治。但只靠政府支支節節的改革,未能完全解決問題。宣宗是唐朝中期以後少數比较有作为的皇帝,另外,唐宣宗还趁吐蕃、回纥衰微,加上張議潮的歸義軍起事反吐蕃; 不僅出兵收复河湟,安定塞北,更一度大致重奪丟失多年的河西走廊(沙、瓜等河西十一州),國威稍振; 國內百姓亦有申冤之地,又宽减刑法,对百姓加以安抚。又抑制宦官,革除弊制,唐朝出现中兴的局面。

大中十三年(859年)八月,宣宗因丹藥中毒驾崩,时年49岁。

唐宣宗在位时,有一越州美女天姿國色。唐宣宗初見之,宠爱异常。不久,唐宣宗怕自己耽誤國事,居然把她賜死。

唐末的黃巢之亂和藩镇战争使宣宗朝的實錄散失,使後人難以追查當年發生過的事。

传说宣宗继位之前曾在淅川香严寺当过和尚,所以对佛教极力推崇,据说曾在大中七年(853年)大拜释迦牟尼的舍利,關於這些資料見諸韋昭度《讀皇室運尋》、令狐绹《偵陵遺事》、讚寧《宋高僧傳》及僧圓悟禪師《碧岩集》。

唐末西川變民韓秀升在被高仁厚征服後,就曾坦言唐宣宗在位時天下尚有公道,唐宣宗故去之後就是勝者才有公道;高仁厚聞言後為之側目。可惜唐朝當時已是病入膏肓之軀,再沒有人能有力回天。

歷史上評價宣宗在位曾經燒過三把火:“權豪斂跡”、“奸臣畏法”、“閽寺詟氣”,稱之為“明君”,有“小太宗”的外號。據說唐宣宗退朝后還會讀書到半夜,燭灺委積,近侍呼之為“老儒生”。

明末清初的大儒王夫之在《讀通鑑論》論“唐之亡,宣宗亡之”,“小說載宣宗之政,琅琅乎其言之,皆治像也,溫公亟取之登之於策,若有余美焉。自知治者觀之,則皆亡國之符也。”評價與《舊唐書》和《資治通鑑》的大力稱頌,實有天壤之別。

\subsection{大中}

\begin{longtable}{|>{\centering\scriptsize}m{2em}|>{\centering\scriptsize}m{1.3em}|>{\centering}m{8.8em}|}
  % \caption{秦王政}\
  \toprule
  \SimHei \normalsize 年数 & \SimHei \scriptsize 公元 & \SimHei 大事件 \tabularnewline
  % \midrule
  \endfirsthead
  \toprule
  \SimHei \normalsize 年数 & \SimHei \scriptsize 公元 & \SimHei 大事件 \tabularnewline
  \midrule
  \endhead
  \midrule
  元年 & 847 & \tabularnewline\hline
  二年 & 848 & \tabularnewline\hline
  三年 & 849 & \tabularnewline\hline
  四年 & 850 & \tabularnewline\hline
  五年 & 851 & \tabularnewline\hline
  六年 & 852 & \tabularnewline\hline
  七年 & 853 & \tabularnewline\hline
  八年 & 854 & \tabularnewline\hline
  九年 & 855 & \tabularnewline\hline
  十年 & 856 & \tabularnewline\hline
  十一年 & 857 & \tabularnewline\hline
  十二年 & 858 & \tabularnewline\hline
  十三年 & 859 & \tabularnewline\hline
  十四年 & 860 & \tabularnewline
  \bottomrule
\end{longtable}


%%% Local Variables:
%%% mode: latex
%%% TeX-engine: xetex
%%% TeX-master: "../Main"
%%% End:

%% -*- coding: utf-8 -*-
%% Time-stamp: <Chen Wang: 2021-10-29 16:48:01>

\section{懿宗李漼\tiny(859-873)}

\subsection{生平}

唐懿宗李漼(cuǐ,833年12月28日-873年8月15日),唐朝第20代皇帝(除去武则天),859年至873年在位,在位14年,终年41岁。

李漼初名温,唐宣宗李忱的長子。宣宗病死後,被宦官迎立為帝,是為唐懿宗,改元“咸通”。死後葬於簡陵,谥号昭聖恭惠孝皇帝。

大和七年十一月十四日(833年),生于籓邸,时父亲李忱为光王,母晁氏为其妾室。初名李温。父亲继位后,于会昌六年十月,封李温为郓王。当时,唐宣宗喜欢第三子夔王李滋,欲立为皇太子,而李温年长,久而不决。大中十三年(859年)八月,宣宗病逝,左神策护军中尉王宗实、副使丌元实矯詔立李温为皇太子。

唐懿宗「器度沈厚,形貌瑰偉」、「洞曉音律,猶如天縱」但遊宴無度、沉湎酒色,以致政治腐敗,藩鎮割據重新興起。

由於崇仰佛法,咸通十四年春(873年),不顧大臣反對,舉行最大規模的迎奉佛骨活動。

他將其父唐宣宗大中之治的成果損耗殆盡。翰林學士劉允章在《直諫書》用“國有九破”描繪過當時的局勢:“終年聚兵,一破也。蠻夷熾興,二破也。權豪奢僭,三破也。大將不朝,四破也。廣造佛寺,五破也。賂賄公行,六破也。長吏殘暴,七破也。賦役不等,八破也。食祿人多,輸稅人少,九破也。”此时唐朝已無可救藥,大動亂正在醞釀。當時赋税刻薄,百姓無法過活,更有人吃人惨劇,百姓无路可走,只好起义。859年,裘甫在浙東起兵;868年,庞勋领导徐泗地区的戍兵在桂林起兵。懿宗派遣王式、康承訓等镇压,但對人民的剥削並無停止。

懿宗死後隨即引发導致唐朝灭亡的黃巢之亂,因而被认为是唐朝間接的亡國之君。

《唐人傳奇》、《太平廣記》對於咸通年間有諸多著墨。

\subsection{咸通}

\begin{longtable}{|>{\centering\scriptsize}m{2em}|>{\centering\scriptsize}m{1.3em}|>{\centering}m{8.8em}|}
  % \caption{秦王政}\
  \toprule
  \SimHei \normalsize 年数 & \SimHei \scriptsize 公元 & \SimHei 大事件 \tabularnewline
  % \midrule
  \endfirsthead
  \toprule
  \SimHei \normalsize 年数 & \SimHei \scriptsize 公元 & \SimHei 大事件 \tabularnewline
  \midrule
  \endhead
  \midrule
  元年 & 860 & \tabularnewline\hline
  二年 & 861 & \tabularnewline\hline
  三年 & 862 & \tabularnewline\hline
  四年 & 863 & \tabularnewline\hline
  五年 & 864 & \tabularnewline\hline
  六年 & 865 & \tabularnewline\hline
  七年 & 866 & \tabularnewline\hline
  八年 & 867 & \tabularnewline\hline
  九年 & 868 & \tabularnewline\hline
  十年 & 869 & \tabularnewline\hline
  十一年 & 870 & \tabularnewline\hline
  十二年 & 871 & \tabularnewline\hline
  十三年 & 872 & \tabularnewline\hline
  十四年 & 873 & \tabularnewline\hline
  十五年 & 874 & \tabularnewline
  \bottomrule
\end{longtable}


%%% Local Variables:
%%% mode: latex
%%% TeX-engine: xetex
%%% TeX-master: "../Main"
%%% End:

%% -*- coding: utf-8 -*-
%% Time-stamp: <Chen Wang: 2021-10-29 16:48:10>

\section{僖宗李儇\tiny(873-888)}

\subsection{生平}

唐僖宗李儇(「儇」,拼音:xuān,注音:ㄒㄩㄢ,粤拼:hyun1;862年6月8日-888年4月20日),唐朝第21代皇帝(除去武则天以外)。唐懿宗第五子,初名俨。873年-888年在位,在位15年,得年27岁,死后谥号为惠圣恭定孝皇帝。唐朝皇帝中出逃时间最长的一位,在位年间有8年不在京师长安。

咸通三年(862年)五月八日,生於东内(大明宫),母王贵妃。初名李俨,六年(865年)七月封普王。十一年(870年)遥领魏博节度使。

在咸通十四年(873年)由宦官拥立,时年十二岁。僖宗一共有五個年號:乾符(6年)、廣明(1年)、中和(4年)、光啟(3年)、文德(1年)。在位期間政事全交给宦官田令孜掌握,自己却肆無忌憚地遊樂,喜歡鬥雞、賭鵝、騎射、劍槊、法算、音樂、圍棋。他對打馬球十分迷戀,對身邊的優伶石野豬說:“朕若參加擊球進士科考試,應該中個狀元。”当时灾害连年,人民生活困苦,官员盘剥沉重。

乾符元年(874年),濮州王仙芝发动起兵。次年,黄巢也起兵于冤句(今山東曹縣西北),隨後兩軍會合,「黃巢之亂」爆发。王仙芝失败后,叛军由黄巢率领,南下進攻浙東,開山路700里入福建,克廣州,回師北上,克潭州,下江陵,直進中原。

广明元年(880年)十一月,黃巢軍攻克洛陽。十二月,下潼關,占领长安,宰相盧攜自殺,田令孜率五百神策軍帶僖宗自長安西門的金光門逃亡入四川,召沙陀族人李克用入援。李克用擊敗黃巢軍於田陂,黃巢退出關中。中和二年(882年)朱溫降唐,賜名朱全忠。

中和四年(884年),黃巢在山東泰安的虎狼谷中自殺(一說為部下林言所殺)。次年三月,唐僖宗返回长安,唐朝已接近灭亡的尾声。此時地方軍閥割據,秦彥據宣、歙,劉漢宏據浙東,朱全忠據汴、滑,李克用據太原、上黨,李昌符據鳳翔,諸葛爽據河陽、洛陽,秦宗權據許、蔡,王敬武據淄、青,高駢據淮南八州,各擅兵賦,迭相吞噬,唐朝中央政府無法節制,能夠控制的地區不過河西、山南、劍南、嶺南西道數十州。

中和五年(885年)三月,田令孜與河中節度使王重榮交惡,王重榮求救於太原李克用,大敗和田令孜结盟的静难节度使朱玫和李昌符,進逼長安。田令孜再領僖宗於光啟元年十二月逃亡到鳳翔(今陝西寶雞),這時諸道兵馬進入長安,燒殺搶掠,宮室坊裏被縱火燒焚者大半,“宮闕蕭條,鞠為茂草”。朱玫立襄王李熅為帝,改元“建貞”。僖宗以正統為號召,把王重榮和李克用爭取過來反攻朱玫,密詔朱玫的愛將王行瑜攻朱,王行瑜將朱玫及其黨羽數百人斬殺,縱兵大掠,時值寒冬,凍死的百姓橫屍蔽地。王重榮殺死襄王煴,田令孜被貶斥。光啟三年(887年)三月,僖宗到達鳳翔,節度使李昌符強留車隊,六月,李昌符進攻僖宗行宮,兵敗出逃隴州,扈駕都將李茂貞追擊,李昌符被斬。

光啟四年(888年)二月,僖宗又回到長安,舉行大赦,改元“文德”。文德元年(888年)三月六日,去世,葬於靖陵(位於今陝西乾縣)。

\subsection{乾符}

\begin{longtable}{|>{\centering\scriptsize}m{2em}|>{\centering\scriptsize}m{1.3em}|>{\centering}m{8.8em}|}
  % \caption{秦王政}\
  \toprule
  \SimHei \normalsize 年数 & \SimHei \scriptsize 公元 & \SimHei 大事件 \tabularnewline
  % \midrule
  \endfirsthead
  \toprule
  \SimHei \normalsize 年数 & \SimHei \scriptsize 公元 & \SimHei 大事件 \tabularnewline
  \midrule
  \endhead
  \midrule
  元年 & 874 & \tabularnewline\hline
  二年 & 875 & \tabularnewline\hline
  三年 & 876 & \tabularnewline\hline
  四年 & 877 & \tabularnewline\hline
  五年 & 878 & \tabularnewline\hline
  六年 & 879 & \tabularnewline
  \bottomrule
\end{longtable}

\subsection{广明}

\begin{longtable}{|>{\centering\scriptsize}m{2em}|>{\centering\scriptsize}m{1.3em}|>{\centering}m{8.8em}|}
  % \caption{秦王政}\
  \toprule
  \SimHei \normalsize 年数 & \SimHei \scriptsize 公元 & \SimHei 大事件 \tabularnewline
  % \midrule
  \endfirsthead
  \toprule
  \SimHei \normalsize 年数 & \SimHei \scriptsize 公元 & \SimHei 大事件 \tabularnewline
  \midrule
  \endhead
  \midrule
  元年 & 880 & \tabularnewline\hline
  二年 & 881 & \tabularnewline
  \bottomrule
\end{longtable}

\subsection{中和}

\begin{longtable}{|>{\centering\scriptsize}m{2em}|>{\centering\scriptsize}m{1.3em}|>{\centering}m{8.8em}|}
  % \caption{秦王政}\
  \toprule
  \SimHei \normalsize 年数 & \SimHei \scriptsize 公元 & \SimHei 大事件 \tabularnewline
  % \midrule
  \endfirsthead
  \toprule
  \SimHei \normalsize 年数 & \SimHei \scriptsize 公元 & \SimHei 大事件 \tabularnewline
  \midrule
  \endhead
  \midrule
  元年 & 881 & \tabularnewline\hline
  二年 & 882 & \tabularnewline\hline
  三年 & 883 & \tabularnewline\hline
  四年 & 884 & \tabularnewline\hline
  五年 & 885 & \tabularnewline
  \bottomrule
\end{longtable}

\subsection{光启}

\begin{longtable}{|>{\centering\scriptsize}m{2em}|>{\centering\scriptsize}m{1.3em}|>{\centering}m{8.8em}|}
  % \caption{秦王政}\
  \toprule
  \SimHei \normalsize 年数 & \SimHei \scriptsize 公元 & \SimHei 大事件 \tabularnewline
  % \midrule
  \endfirsthead
  \toprule
  \SimHei \normalsize 年数 & \SimHei \scriptsize 公元 & \SimHei 大事件 \tabularnewline
  \midrule
  \endhead
  \midrule
  元年 & 885 & \tabularnewline\hline
  二年 & 886 & \tabularnewline\hline
  三年 & 887 & \tabularnewline\hline
  四年 & 888 & \tabularnewline
  \bottomrule
\end{longtable}

\subsection{文德}

\begin{longtable}{|>{\centering\scriptsize}m{2em}|>{\centering\scriptsize}m{1.3em}|>{\centering}m{8.8em}|}
  % \caption{秦王政}\
  \toprule
  \SimHei \normalsize 年数 & \SimHei \scriptsize 公元 & \SimHei 大事件 \tabularnewline
  % \midrule
  \endfirsthead
  \toprule
  \SimHei \normalsize 年数 & \SimHei \scriptsize 公元 & \SimHei 大事件 \tabularnewline
  \midrule
  \endhead
  \midrule
  元年 & 888 & \tabularnewline
  \bottomrule
\end{longtable}


%%% Local Variables:
%%% mode: latex
%%% TeX-engine: xetex
%%% TeX-master: "../Main"
%%% End:

%% -*- coding: utf-8 -*-
%% Time-stamp: <Chen Wang: 2021-10-29 16:48:18>

\section{昭宗李晔\tiny(888-904)}

\subsection{生平}

唐昭宗李晔(867年-904年),姓李諱晔,原名傑,即位後改名為敏,又改名曄。是唐朝第22代皇帝(除去武则天以外),888年-904年在位,在位16年,享年38岁。他是唐懿宗第七子、唐僖宗的弟弟。在位年间曾三次出逃京师长安。

初名李杰,咸通十三年四月封寿王。乾符三年授开府仪同三司、幽州大都督、幽州卢龙等军节度、押奚契丹、管内观察处置等使。

唐昭宗是在唐僖宗死后由當時掌權的宦官楊復恭矫诏立为皇太弟所擁立,先改名李敏,再改名李晔。登帝位后,各藩镇趁着平定农民起兵的机会逐渐扩大。雖然昭宗曾圖增強軍備以增強中央朝廷的實力,在大顺元年(890年)已招募到十万军队,但他贸然宣布西川节度使陈敬瑄、河东节度使李克用为叛逆予以讨伐,反被李克用击败,不得不恢复李克用官爵。而对西川的讨伐战则陷入僵局,昭宗因河东之败决定终止西川战事并恢复陈敬瑄官爵,但讨伐陈敬瑄的永平军节度使王建却抗命继续攻打,最终取代陈敬瑄成为西川节度使。大顺二年(891年),唐昭宗下令逮捕杨复恭,杨复恭抵抗失败,出奔投靠养侄山南西道节度使杨守亮。

昭宗的举动也引起了藩镇的疑心。景福二年(893年),凤翔节度使李茂貞击败杨守亮后上表昭宗,求任山南西道节度使,期待兼领两镇。昭宗却想为朝廷收回一些领地,下诏任李茂贞为山南西道节度使和相邻的武定军节度使,而任中书侍郎宰相徐彦若为凤翔节度使。李茂贞失望,不奉诏。八月,昭宗命嗣覃王李嗣周率领新组建的三万禁军护送徐彦若就职。李茂贞与盟友静难节度使王行瑜调兵准备迎战,禁军望风溃逃,李、王兵临京城长安。宰相杜让能原本反对昭宗讨伐凤翔,昭宗不听,还专任他计划调度此战。李、王迫使昭宗赐死杜让能。昭宗也被迫召回徐彦若,同意李茂贞兼领凤翔和山南西道。

乾宁元年/宽平六年(894年),鉴于唐的弱化和内乱频发; 在菅原道真的建议下,当时的日本朝廷废止了遣唐使的职务,十余年后(907年),唐灭亡,遣唐使走入历史。

后来昭宗让宗室诸王嗣延王李戒丕、李嗣周、通王李滋、仪王(唐昭宗兄,未详)、丹王李允、嗣韩王李克良、睦王李倚、韶王、彭王李惕、陈王、济王等十一人掌兵,愈发让李茂贞生疑,遂带頭叛亂,兵至長安。昭宗出逃,本欲求助于李克用,派嗣延王李戒丕前往河东,但这时李克用无力相助。这时镇国节度使韩建表忠相请,昭宗君臣害怕长途跋涉,于是前往镇国军府华州。但韩建却控制昭宗,不让朝廷迎战李茂贞,还迫使昭宗解散宗室诸王的军队,斩护驾有功的捧日都头李筠。李戒丕从河东归来后,显然李克用不能发兵勤王,于是韩建又以诬以谋反的方式,未经昭宗许可即将被罢去兵权幽禁别第的十一王杀死。期间韩建又为了缓和与昭宗的紧张关系,请求昭宗立皇太子,于是昭宗立何淑妃所生皇长子李祐为皇太子,改名李裕,后来又立何淑妃为皇后,这是唐朝一百多年来第一次也是最后一次立皇后。898年,昭宗与李茂贞讲和复其官爵,宣武军节度使朱温又劝昭宗迁都洛阳,韩建和李茂贞担心朱温勤王救驾,就修复被李茂贞军烧毁的长安宫殿,昭宗才得以重返长安,但身边除了宦官控制的神策军外再无一兵一卒。

光化三年(900年)十一月,唐昭宗醉后亲手杀了几个宦官、侍女,引起了宦官神策军左中尉刘季述的剧烈反应。当天,昭宗打猎夜归,何皇后遣太子李裕还邸,李裕遇到刘季述,被留在紫廷院。第二天刘季述挟持李裕,带兵逼唐昭宗禅让帝位给李裕,昭宗意欲反抗,何皇后闻宫人报信趋至,出拜说:“军容长官护官家,勿使惊恐,有事与军容商量。”她怕伤害皇帝,说服昭宗听从刘季述安排,于御前取玉玺授予刘季述。宦官扶昭宗与何皇后同乘一辇,与嫔御侍从公主等十余人入东宫少阳院,刘季述亲手锁院门,将锁眼熔铁,软禁了他们。当时天大寒,嫔御公主没有衣被,号哭声闻于外。刘季述迎立李裕为皇帝,改名李缜,尊昭宗为太上皇,何皇后为太上皇后,改少阳院为问安宫,每日只从窗中送饭。十二月,忠于昭宗的神策军军官孙德昭、董彦弼、周承诲在宰相崔胤指使下发动反政变,杀刘季述、右护军中尉王仲先,赴少阳院叩门称逆贼已诛,请昭宗出劳。何皇后不信,要孙德昭等送上刘季述等首级。孙德昭献刘季述等首后,昭宗、何皇后才与宫人与孙德昭一同破坏门锁而出。昭宗复辟,复李缜名李裕,褫夺太子位,复为德王。

901年因崔胤谋诛宦官,宦官韓全誨強迫昭宗投奔鳳翔,崔胤召朱全忠圍攻鳳翔。903年李茂貞殺韓全誨、張彥弘,跟朱溫和解,送李曄回長安。這時唐朝公家已经名存实亡。李茂貞被朱温打敗,但反而使朱溫變成了最大的藩鎮,並控制着昭宗。朱温为了灭亡唐朝,自己做皇帝,先杀掉皇宫所有宦官五千餘人,朱溫下令,派往各軍區擔任監軍的宦官,一律就地處決(只是下令,但各地藩鎮並未徹底奉行,如李克用未杀张承业、卢龙节度使刘仁恭未杀张居翰等)。

因后来改封濮王的皇长子李裕年长俊秀,朱全忠厌恶他。崔胤也揣摩朱全忠的心意,在唐昭宗以朱全忠为天下兵马副都统,要以皇子为名义上的正都统时,明知昭宗因李裕年长而心向他,仍然坚持请求任命辉王李祚,最后李祚被任为都统。朱全忠又通过崔胤提出以李裕曾篡位为由处死李裕,昭宗大惊,问朱全忠,朱全忠否认有此请求。崔胤觉察到朱全忠的异心,想募兵对抗他,朱全忠先逼迫昭宗罢崔胤相位,再攻杀崔胤。

天祐元年(904年)正月,朱全忠不顧大臣反对,迁都洛邑,令長安居民按戶籍遷居,房屋被拆後的木材扔在渭河當中,長安城哭聲一片。昭宗无奈,自陕州出发,至谷水,身边已無禁军。至洛陽時,何皇后哭着对朱全忠说:“此后大家(唐、宋對皇帝的俗稱,或稱「官家」)夫妇,委身全忠了。”太原軍李克用、鳳翔軍李茂貞、西川軍王建、淮南軍杨行密等各藩镇起義,與朱全忠對抗,声称要出兵勤王救出萬歲。昭宗離開京師後,終日與皇后、內人“沉飲自寬”。朱全忠又借设宴为名将随同唐昭宗东行的供奉内园小儿二百余人缢死,选身形相似的宣武军人穿上他们的衣服回去。从此宫中事无论大小,朱全忠都能得知。朱全忠心腹蒋玄晖被任为枢密使监视唐昭宗。昭宗身边都是强横小人,只有何皇后依然照顾他,不离开他身边。朱全忠依然坚持要处死李裕,昭宗向蒋玄晖哭诉:“德王是朕爱子,为什么全忠坚持要杀他?”朱全忠得知后很不快。

朱全忠正要用兵讨伐李茂贞及其养子静难军节度使李继徽,担心昭宗英杰不群从中生变,决意弑君另立幼主。天祐元年(904年)八月十一日壬寅夜,朱全忠派左龍武統軍朱友恭、右龍武統軍氏叔琮、蒋玄晖弒殺唐昭宗。是夜朱友恭等率兵上百人闖入內門,玄暉每門留卒十人,至東都之椒殿院,斬殺河東夫人裴貞一,昭儀李漸榮在門外道:“院使(蔣玄暉)莫傷官家(唐、宋對皇帝的俗稱),寧殺我輩。”昭宗聞訊,身著睡衣繞著殿內的柱子逃命,被龙武衙官史太追上,李漸榮以身體護天子,一起被殺,唯獨何皇后求饶得免死。是年十月,朱全忠返回洛邑,得知昭宗已死,故意假装震驚,伏於棺材大哭说:“奴辈负我,令我受恶名于万代!”斬殺朱友恭等人。

唐昭宗在位十六載间,一直是藩镇手中的傀儡。在极度困窘之中,昭宗求才若渴,且急於大用,有可用之人,则立即提拔。昭宗朝共拜相二十五人,宰相更替頻繁,崔胤先后四次拜相。郑綮於乾宁元年二月拜相,三个多月後,即“以太子太保致仕。”陆扆在乾宁三年七月被拜为宰相,兩個月後贬峡州刺史。韦昭度、孔纬、徐彦若、崔远、裴枢等人都先后两次入相。昭宗死後,葬于和陵,他的第九子李柷被擁立即位,是為唐哀帝,不久,唐朝滅亡。

死后廟號昭宗,谥号圣穆景文孝皇帝,起居郎蘇楷在天祐二年(905年)以昭宗非功德,議改廟谥,後來太常卿張廷範將廟號改為襄宗,谥号恭靈莊閔孝皇帝,後唐同光年間恢復原有廟谥。

\subsection{龙纪}

\begin{longtable}{|>{\centering\scriptsize}m{2em}|>{\centering\scriptsize}m{1.3em}|>{\centering}m{8.8em}|}
  % \caption{秦王政}\
  \toprule
  \SimHei \normalsize 年数 & \SimHei \scriptsize 公元 & \SimHei 大事件 \tabularnewline
  % \midrule
  \endfirsthead
  \toprule
  \SimHei \normalsize 年数 & \SimHei \scriptsize 公元 & \SimHei 大事件 \tabularnewline
  \midrule
  \endhead
  \midrule
  元年 & 889 & \tabularnewline
  \bottomrule
\end{longtable}

\subsection{大顺}

\begin{longtable}{|>{\centering\scriptsize}m{2em}|>{\centering\scriptsize}m{1.3em}|>{\centering}m{8.8em}|}
  % \caption{秦王政}\
  \toprule
  \SimHei \normalsize 年数 & \SimHei \scriptsize 公元 & \SimHei 大事件 \tabularnewline
  % \midrule
  \endfirsthead
  \toprule
  \SimHei \normalsize 年数 & \SimHei \scriptsize 公元 & \SimHei 大事件 \tabularnewline
  \midrule
  \endhead
  \midrule
  元年 & 890 & \tabularnewline\hline
  二年 & 891 & \tabularnewline
  \bottomrule
\end{longtable}

\subsection{景福}

\begin{longtable}{|>{\centering\scriptsize}m{2em}|>{\centering\scriptsize}m{1.3em}|>{\centering}m{8.8em}|}
  % \caption{秦王政}\
  \toprule
  \SimHei \normalsize 年数 & \SimHei \scriptsize 公元 & \SimHei 大事件 \tabularnewline
  % \midrule
  \endfirsthead
  \toprule
  \SimHei \normalsize 年数 & \SimHei \scriptsize 公元 & \SimHei 大事件 \tabularnewline
  \midrule
  \endhead
  \midrule
  元年 & 892 & \tabularnewline\hline
  二年 & 893 & \tabularnewline
  \bottomrule
\end{longtable}

\subsection{乾宁}

\begin{longtable}{|>{\centering\scriptsize}m{2em}|>{\centering\scriptsize}m{1.3em}|>{\centering}m{8.8em}|}
  % \caption{秦王政}\
  \toprule
  \SimHei \normalsize 年数 & \SimHei \scriptsize 公元 & \SimHei 大事件 \tabularnewline
  % \midrule
  \endfirsthead
  \toprule
  \SimHei \normalsize 年数 & \SimHei \scriptsize 公元 & \SimHei 大事件 \tabularnewline
  \midrule
  \endhead
  \midrule
  元年 & 894 & \tabularnewline\hline
  二年 & 895 & \tabularnewline\hline
  三年 & 896 & \tabularnewline\hline
  四年 & 897 & \tabularnewline\hline
  五年 & 898 & \tabularnewline
  \bottomrule
\end{longtable}

\subsection{光化}

\begin{longtable}{|>{\centering\scriptsize}m{2em}|>{\centering\scriptsize}m{1.3em}|>{\centering}m{8.8em}|}
  % \caption{秦王政}\
  \toprule
  \SimHei \normalsize 年数 & \SimHei \scriptsize 公元 & \SimHei 大事件 \tabularnewline
  % \midrule
  \endfirsthead
  \toprule
  \SimHei \normalsize 年数 & \SimHei \scriptsize 公元 & \SimHei 大事件 \tabularnewline
  \midrule
  \endhead
  \midrule
  元年 & 898 & \tabularnewline\hline
  二年 & 899 & \tabularnewline\hline
  三年 & 900 & \tabularnewline\hline
  四年 & 901 & \tabularnewline
  \bottomrule
\end{longtable}

\subsection{天复}

\begin{longtable}{|>{\centering\scriptsize}m{2em}|>{\centering\scriptsize}m{1.3em}|>{\centering}m{8.8em}|}
  % \caption{秦王政}\
  \toprule
  \SimHei \normalsize 年数 & \SimHei \scriptsize 公元 & \SimHei 大事件 \tabularnewline
  % \midrule
  \endfirsthead
  \toprule
  \SimHei \normalsize 年数 & \SimHei \scriptsize 公元 & \SimHei 大事件 \tabularnewline
  \midrule
  \endhead
  \midrule
  元年 & 901 & \tabularnewline\hline
  二年 & 902 & \tabularnewline\hline
  三年 & 903 & \tabularnewline\hline
  四年 & 904 & \tabularnewline
  \bottomrule
\end{longtable}

\subsection{德王生平}

李{\fzk \xpinyin*{𥙿}}(880年代-905年3月17日)是唐昭宗的長子,母何皇后。本名李祐。

李祐生年不详,从其父生年推断(867年-904年),他可能生于880年代或892年之前。本被封為德王。

唐昭宗为躲避凤翔节度使李茂贞作乱,投奔镇国军节度使韩建。韩建趁机罢去领兵的宗室诸王的兵权,后又将他们诛杀。为了改善和昭宗的关系,他请求昭宗立太子。乾寧四年(897年)二月,李祐被立为皇太子,改名李{\fzk 𥙿},母何淑妃被立为皇后。五月,韩建上书请求置师傅教导太子、诸王。昭宗于是以太子宾客王牍为诸王侍读。韩建曾想请昭宗游幸南庄,趁机拥立李{\fzk 𥙿}为帝。其父韩叔丰对他说:“你只是陈、许间一介田夫,遭遇时乱,蒙天子厚恩到此,欲以两州百里之地行大事,我不忍见灭族之祸,不如先死!”于是哭了。李茂贞及宣武军节度使朱全忠都想发兵迎天子,韩建有些恐惧,于是作罢。

后来昭宗与李茂贞讲和,回京。光化三年(900年)四月,何皇后与太子拜谒太庙。十一月,唐昭宗醉后亲手杀了几个宦官、侍女,宦官神策军左中尉刘季述因而图谋废立。当天,昭宗打猎夜归,何皇后遣李{\fzk 𥙿}还邸,李{\fzk 𥙿}遇到刘季述,被留在紫廷院。第二天刘季述挟持李{\fzk 𥙿},带兵逼唐昭宗禅让帝位给李{\fzk 𥙿}。何皇后闻讯赶到,说服昭宗就范。昭宗退位,李{\fzk 𥙿}登基,改名李缜;昭宗、何皇后被尊为太上皇、太上皇后,被软禁于东宫少阳院,改称问安宫,每日只从窗中送饭。

光化四年(901年)正月,宰相崔胤及神策军军官孙德昭、董彦弼、周承诲发动政变诛杀刘季述等,昭宗復辟,恢复李缜原名李{\fzk 𥙿},降为德王。曾改封濮王,天复三年(903年)二月,昭宗为褒赏宣武军节度使朱全忠,欲任命一皇子为诸道兵马元帅,朱全忠为副。因李{\fzk 𥙿}年长,昭宗有意任他为诸道兵马元帅,但崔胤按朱全忠的意思,因李{\fzk 𥙿}胞弟辉王李祚年幼易利用,坚请任李祚,最后李祚被任为诸道兵马元帅。

朱全忠因李{\fzk 𥙿}俊秀且年长而厌恶他,常想除掉他,曾通过崔胤请求昭宗以篡位为由杀之。但昭宗一直予以保全,问朱全忠是否有此事,朱全忠否认。后来昭宗被朱全忠胁迫迁都洛阳,又对朱全忠心腹枢密使蒋玄晖哭诉“德王是朕的爱子,为什么全忠总要杀他”,朱全忠因而怀恨。天祐元年(904年),朱全忠弑昭宗,矫诏立李祚为皇太子,改名李柷,继位为帝。二年(905年)二月,李{\fzk 𥙿}等随哀帝、何太后于长乐门外祭昭宗完毕回宫。当月,李{\fzk 𥙿}及其弟八人被朱全忠命蒋玄晖借设宴之机缢杀于九曲池,尸体投入池中。

因不是正常登基、在位短且为刘季述傀儡,李{\fzk 𥙿}通常不被认为正统意义上的唐朝皇帝。

%%% Local Variables:
%%% mode: latex
%%% TeX-engine: xetex
%%% TeX-master: "../Main"
%%% End:

%% -*- coding: utf-8 -*-
%% Time-stamp: <Chen Wang: 2021-10-29 16:48:47>

\section{哀帝李柷\tiny(904-907)}

\subsection{生平}

唐哀帝李柷(「柷」,拼音:zhù,注音:ㄓㄨˋ;892年10月27日-908年3月26日),原名祚。唐昭宗第九子,生母何皇后。是唐朝第23位皇帝(除武曌以外),904年-907年在位,在位3年,被废。次年死,得年16岁,葬于温陵。

始封辉王。天复三年(903年)二月,唐昭宗为褒赏宣武军节度使朱全忠,想以皇子为诸道兵马元帅,朱全忠为副帅。宰相崔胤请让李祚为此职,昭宗则因长子濮王李𥙿年长而有意濮王。崔胤受朱全忠所托,认为李祚年幼可利用,坚持请求,于是李祚被任为诸道兵马元帅。

天祐元年(904年),崔胤图谋对抗朱全忠,被杀。朱全忠迫昭宗迁都洛阳,八月遣枢密使蒋玄晖等弑唐昭宗,以昭宗名义下诏立李祚为皇太子,改名李柷,监军国事,又由宰相柳璨、独孤损矫何皇后诏奉其继位。次月,哀帝尊母何皇后为皇太后。

唐哀帝即位时,不过是朱全忠手中的一个傀儡皇帝。二年(905年),掌握实际权力的朱全忠见废帝灭唐时机已到,便先让蒋玄晖借设宴之机,将包括哀帝同胞兄长李裕在内的兄弟九人杀害于九曲池,再大量杀害朝臣(见白马驿之祸);后又因蒋玄晖与太常卿张廷范认为天下未平,时机未到,不宜受九锡而不悦,十二月听信宣徽副使王殷、赵殷衡诬告,认为蒋玄晖、张廷范、柳璨与何太后图谋复兴唐室,遂遣使捕杀蒋玄晖,派王、赵将何太后缢死于其住所积善宫,并迫哀帝下诏,称何太后秽乱宫闱自杀谢罪,追回皇太后宝册,追废为庶人。哀帝因太后丧废朝三日,并以太后因宫闱丑闻自杀为由,取消新年郊礼。张廷范和柳璨也被外贬,未及赴任皆被处死。

接着朱全忠在天祐四年(907年),又逼李柷禅位,降为济阴王,自己做皇帝,改名朱晃,是为后梁太祖,建国号“大梁”,史称“后梁”,改元“开平”。至此,立国总计289年(618-907)、传21帝的唐王朝灭亡,中国进入自黃巾之亂或永嘉之乱以来又一次大分裂时期——五代十国。

遜位後,李柷被迫遷到曹州(在今山東省內),次年(908年)二月二十一日被朱晃毒死。朱晃上谥号为哀皇帝,以王禮葬於济阴县之定陶鄉。

后唐明宗李嗣源改哀帝的谥号为昭宣光烈孝皇帝,但哀帝因系篡位奸臣朱全忠所立的傀儡且本人及父母都被朱全忠杀害,被认为不够称“宗”,故由横海军节度使卢质提议的庙号景宗未被採用,故后世称李柷为唐哀帝或唐昭宣帝。

\subsection{天佑}

\begin{longtable}{|>{\centering\scriptsize}m{2em}|>{\centering\scriptsize}m{1.3em}|>{\centering}m{8.8em}|}
  % \caption{秦王政}\
  \toprule
  \SimHei \normalsize 年数 & \SimHei \scriptsize 公元 & \SimHei 大事件 \tabularnewline
  % \midrule
  \endfirsthead
  \toprule
  \SimHei \normalsize 年数 & \SimHei \scriptsize 公元 & \SimHei 大事件 \tabularnewline
  \midrule
  \endhead
  \midrule
  元年 & 904 & \tabularnewline\hline
  二年 & 905 & \tabularnewline\hline
  三年 & 906 & \tabularnewline\hline
  四年 & 907 & \tabularnewline
  \bottomrule
\end{longtable}


%%% Local Variables:
%%% mode: latex
%%% TeX-engine: xetex
%%% TeX-master: "../Main"
%%% End:

% %% -*- coding: utf-8 -*-
%% Time-stamp: <Chen Wang: 2021-10-29 17:15:43>

\section{殇帝李重茂\tiny(710)}

\subsection{生平}

唐殇帝李重茂(695年-714年),唐高宗和武则天孙子,唐中宗幼子,安乐公主同父异母弟弟,唐睿宗和太平公主侄子,唐玄宗堂弟。唐朝第七位皇帝,景龙四年(710年)六月在位,开元二年(714年)逝世,谥号殇皇帝。

李重茂生于武后延载元年(695年),圣历三年(700年)封为北海郡王,中宗神龙元年(705年)进封温王,授右卫大将军,兼遥领赠州大都督,没有到任。

景龙四年(710年)中宗病逝后,六月初四韦后临朝,改元唐隆。六月初七韦后矫诏立时年仅16岁的李重茂为帝,韦后临朝称制。李重茂即位不足一个月,六月二十日夜,睿宗三子临淄王李隆基、中宗妹妹太平公主等交结禁军将领,发兵入宫,将韦后与安乐公主等人杀死,是为唐隆之变。

六月二十二日,宫人、宦官请求中书舍人刘幽求草诏立太后,刘幽求意图复辟睿宗,拒绝了。六月二十四日,太平公主称李重茂有意让位睿宗,更亲自将其提下御座,睿宗复辟。李重茂仍封温王。

李重茂的庶兄谯王李重福随即发动政变,与其党羽郑愔等策划矫诏称得中宗传位,以李重茂为皇太弟,但很快被镇压。

睿宗景云二年改封襄王,李重茂离开长安,被迁到集州,睿宗令中郎将率武士五百人守备。

玄宗开元二年(714年),李重茂除房州刺史,不久死于房州。葬于武功西原,年仅19岁。

\subsection{唐隆}

\begin{longtable}{|>{\centering\scriptsize}m{2em}|>{\centering\scriptsize}m{1.3em}|>{\centering}m{8.8em}|}
  % \caption{秦王政}\
  \toprule
  \SimHei \normalsize 年数 & \SimHei \scriptsize 公元 & \SimHei 大事件 \tabularnewline
  % \midrule
  \endfirsthead
  \toprule
  \SimHei \normalsize 年数 & \SimHei \scriptsize 公元 & \SimHei 大事件 \tabularnewline
  \midrule
  \endhead
  \midrule
  元年 & 710 & \tabularnewline
  \bottomrule
\end{longtable}


%%% Local Variables:
%%% mode: latex
%%% TeX-engine: xetex
%%% TeX-master: "../Main"
%%% End:


%%% Local Variables:
%%% mode: latex
%%% TeX-engine: xetex
%%% TeX-master: "../Main"
%%% End:
