%% -*- coding: utf-8 -*-
%% Time-stamp: <Chen Wang: 2019-12-24 15:49:15>

\section{中宗复辟\tiny(705-710)}

\subsection{景龙}

\begin{longtable}{|>{\centering\scriptsize}m{2em}|>{\centering\scriptsize}m{1.3em}|>{\centering}m{8.8em}|}
  % \caption{秦王政}\
  \toprule
  \SimHei \normalsize 年数 & \SimHei \scriptsize 公元 & \SimHei 大事件 \tabularnewline
  % \midrule
  \endfirsthead
  \toprule
  \SimHei \normalsize 年数 & \SimHei \scriptsize 公元 & \SimHei 大事件 \tabularnewline
  \midrule
  \endhead
  \midrule
  元年 & 707 & \tabularnewline\hline
  二年 & 708 & \tabularnewline\hline
  三年 & 709 & \tabularnewline\hline
  四年 & 710 & \tabularnewline
  \bottomrule
\end{longtable}


\section{殇帝}

\subsection{生平}


唐恭宗李重茂(695年-714年),唐中宗幼子,唐朝第七位皇帝,景龍四年(710年)六月在位,開元二年(714年)逝世,諡曰殤皇帝。

李重茂生於武后延載元年(695年),聖曆三年(700年)封為北海郡王,中宗神龍元年(705年)進封温王,授右衛大將軍,兼遙領赠州大都督,沒有到任。

景龍四年(710年)中宗病逝后,六月初四韋后臨朝,改元唐隆。六月初七韋后矯詔立时年仅16岁的李重茂為帝,韋后臨朝稱制。李重茂即位不足一個月,六月二十日夜,睿宗三子臨淄王李隆基、中宗妹妹太平公主等交結禁軍將領,發兵入宮,將韦后與安樂公主等人杀死,是為唐隆之變。

六月二十二日,宫人、宦官请求中书舍人刘幽求草诏立太后,刘幽求意图复辟睿宗,拒绝了。六月二十四日,太平公主称李重茂有意让位睿宗,更亲自将其提下御座,睿宗復辟。李重茂仍封温王。

李重茂的庶兄谯王李重福随即发动政变,与其党羽郑愔等策划矫诏称得中宗传位,以李重茂为皇太弟,但很快被镇压。

睿宗景雲二年改封襄王,李重茂離開长安,被遷到集州,睿宗令中郎將率武士五百人守備。

玄宗開元二年(714年),李重茂除房州刺史,不久死於房州。葬於武功西原,得年二十。

\subsection{唐隆}

\begin{longtable}{|>{\centering\scriptsize}m{2em}|>{\centering\scriptsize}m{1.3em}|>{\centering}m{8.8em}|}
  % \caption{秦王政}\
  \toprule
  \SimHei \normalsize 年数 & \SimHei \scriptsize 公元 & \SimHei 大事件 \tabularnewline
  % \midrule
  \endfirsthead
  \toprule
  \SimHei \normalsize 年数 & \SimHei \scriptsize 公元 & \SimHei 大事件 \tabularnewline
  \midrule
  \endhead
  \midrule
  元年 & 710 & \tabularnewline
  \bottomrule
\end{longtable}


%%% Local Variables:
%%% mode: latex
%%% TeX-engine: xetex
%%% TeX-master: "../Main"
%%% End:
