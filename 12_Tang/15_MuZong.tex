%% -*- coding: utf-8 -*-
%% Time-stamp: <Chen Wang: 2019-12-24 15:30:09>

\section{穆宗\tiny(820-824)}

\subsection{生平}

唐穆宗李恒(795年7月26日-824年2月25日),原名宥。唐宪宗第三子,母懿安皇后郭氏。他是唐朝第15代皇帝(除去武则天以外),在位4年。

唐德宗貞元十一年七月六日(795年7月26日)生於大明宮之別殿,初名李宥。当时的皇帝是他的曾祖父唐德宗,父亲唐宪宗时为广陵郡王,母亲郭氏为广陵郡王妃。贞元二十一年(805年)初,祖父唐顺宗即位后,三月,父亲李纯被立为太子。夏四月庚戌,李宥与其他兄弟五人同封郡王,食邑三千户。

其父唐宪宗在805年登基后,母亲郭氏做为唐宪宗的嫡妻却未能立为皇后,只是被封为贵妃。其后,在元和元年(806年)八月,李宥进封遂王。元和四年(809年),他的异母长兄邓王李宁被册为太子。元和五年(810年)三月,李宥领彰义军节度大使。元和六年闰十二月廿一日(812年2月7日),太子李宁逝世。元和七年(812年)十月,李宥被立为皇太子,改名恒。

元和十五年正月庚子(820年2月14日),唐宪宗暴卒,疑似被宦官陳弘志、王守澄下毒謀害。宦官梁守謙等擁立李恒,丙午(2月20日)登基,是为唐穆宗。唐穆宗在位期间“宴樂過多,畋遊無度”,“不留意天下之務”。任用的宰相萧俛、段文昌又无远见,认为藩镇已平,应当消兵。于是令天下军镇有兵处每年在100人中限八人或逃或死,消其兵籍。被取消兵籍的军士无处可去,又無法從事他業,只好藏於山林。不久河朔三鎮復叛,躲藏的军士纷纷归附三鎮。

朝廷内宦官权势日盛,官僚朋党斗争剧烈。使唐宪宗時期的“元和中兴”局面完全丧失。好服金石之藥,長慶二年十一月,一次他与宦官内臣打马球时,穆宗突然中風,长庆四年陰曆正月二十二日,崩於寢殿。同年十一月,葬於光陵。死后谥号为睿圣文惠孝皇帝。

\subsection{永新}

\begin{longtable}{|>{\centering\scriptsize}m{2em}|>{\centering\scriptsize}m{1.3em}|>{\centering}m{8.8em}|}
  % \caption{秦王政}\
  \toprule
  \SimHei \normalsize 年数 & \SimHei \scriptsize 公元 & \SimHei 大事件 \tabularnewline
  % \midrule
  \endfirsthead
  \toprule
  \SimHei \normalsize 年数 & \SimHei \scriptsize 公元 & \SimHei 大事件 \tabularnewline
  \midrule
  \endhead
  \midrule
  元年 & 820 & \tabularnewline
  \bottomrule
\end{longtable}

\subsection{长庆}

\begin{longtable}{|>{\centering\scriptsize}m{2em}|>{\centering\scriptsize}m{1.3em}|>{\centering}m{8.8em}|}
  % \caption{秦王政}\
  \toprule
  \SimHei \normalsize 年数 & \SimHei \scriptsize 公元 & \SimHei 大事件 \tabularnewline
  % \midrule
  \endfirsthead
  \toprule
  \SimHei \normalsize 年数 & \SimHei \scriptsize 公元 & \SimHei 大事件 \tabularnewline
  \midrule
  \endhead
  \midrule
  元年 & 821 & \tabularnewline\hline
  二年 & 822 & \tabularnewline\hline
  三年 & 823 & \tabularnewline\hline
  四年 & 824 & \tabularnewline
  \bottomrule
\end{longtable}


%%% Local Variables:
%%% mode: latex
%%% TeX-engine: xetex
%%% TeX-master: "../Main"
%%% End:
