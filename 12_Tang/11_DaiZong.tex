%% -*- coding: utf-8 -*-
%% Time-stamp: <Chen Wang: 2019-12-24 15:23:33>

\section{代宗\tiny(762-779)}

\subsection{生平}

唐代宗李豫(726年11月11日-779年6月10日),唐肃宗李亨的嫡长子。初名俶,小名冬郎,原封广平郡王,後改封廣平王、楚王、成王,天可汗,唐朝第11代皇帝(不计武则天),在位17年(762年5月18日-779年6月10日)。

据《旧唐书》,代宗以开元十四年十二月十三日(727年1月9日)生于東都洛阳上阳宫。据《册府元龟》,代宗以开元十四年十月十三日(726年11月11日)生于东都上阳宫,且多年之中的十月份都有庆祝活动(上寿)。似以十月生日更可信。唐代宗十月十三日天兴节,见令狐绹文集。

开元十四年十月十三日(726年11月11日)生于东都洛阳上阳宫,初名李俶。当时,父亲唐肃宗李亨为藩王,母亲吴氏是他的妾室。开元二十六年(738年),父亲李亨被立为太子。数年后,十五岁的李俶封广平郡王。天宝元年(742年),长子李适出生。天宝五载(746年),娶崔氏为妃。

天宝十五载(756年),安禄山叛军攻占潼关,祖父玄宗逃至马嵬驿,当地民众揽留肃宗。于是李俶护送肃宗北上灵武即帝位。安史之乱中,以兵马元帅名义收复洛阳、长安两京。乾元元年(758年)三月改封成王,四月被立為皇太子。

宝应元年(762年),宦官李辅国杀张皇后,肃宗受惊吓而死,李俶于肃宗灵柩前依其遗诏即位,改名豫。

次年,安史之亂结束,大唐开始走向衰落。当时,东部有诸多藩镇割据,北方又有邻国回鶻不断勒索,西面有邻国吐蕃侵扰。吐蕃甚至在廣德元年(763年)佔領首都長安十五日,立李承宏為帝,河西走廊亦被吐蕃佔領。

代宗迷信佛教,“有寇至则令僧讲《仁王经》以禳之,寇去则厚加赏赐”,宰相元载、王缙、杜鸿渐三人都信佛,以王缙尤甚。寺院多占有田地,“造金閣寺於五台山,鑄銅塗金為瓦,所費巨億”,朝廷政治经济进一步恶化。

大历十四年(779年)五月初二,宫中传出代宗生病的消息,結果一病不起,不到十天就无法上朝。五月辛酉(6月10日),下达了令皇太子監國的制书,当晚即在紫宸殿驾崩,享年五十三歲。死後葬于元陵(今陕西省富平县西北三十里的檀山),谥号睿文孝武皇帝,廟號代宗。太子李适繼位,是為唐德宗。

由於李豫平定安史之亂,有不世之功,其廟號原議定為世宗,但為避太宗李世民諱,最終定為代宗意為世代。

\subsection{广德}

\begin{longtable}{|>{\centering\scriptsize}m{2em}|>{\centering\scriptsize}m{1.3em}|>{\centering}m{8.8em}|}
  % \caption{秦王政}\
  \toprule
  \SimHei \normalsize 年数 & \SimHei \scriptsize 公元 & \SimHei 大事件 \tabularnewline
  % \midrule
  \endfirsthead
  \toprule
  \SimHei \normalsize 年数 & \SimHei \scriptsize 公元 & \SimHei 大事件 \tabularnewline
  \midrule
  \endhead
  \midrule
  元年 & 763 & \tabularnewline\hline
  二年 & 764 & \tabularnewline
  \bottomrule
\end{longtable}

\subsection{永泰}

\begin{longtable}{|>{\centering\scriptsize}m{2em}|>{\centering\scriptsize}m{1.3em}|>{\centering}m{8.8em}|}
  % \caption{秦王政}\
  \toprule
  \SimHei \normalsize 年数 & \SimHei \scriptsize 公元 & \SimHei 大事件 \tabularnewline
  % \midrule
  \endfirsthead
  \toprule
  \SimHei \normalsize 年数 & \SimHei \scriptsize 公元 & \SimHei 大事件 \tabularnewline
  \midrule
  \endhead
  \midrule
  元年 & 765 & \tabularnewline\hline
  二年 & 766 & \tabularnewline
  \bottomrule
\end{longtable}

\subsection{大历}

\begin{longtable}{|>{\centering\scriptsize}m{2em}|>{\centering\scriptsize}m{1.3em}|>{\centering}m{8.8em}|}
  % \caption{秦王政}\
  \toprule
  \SimHei \normalsize 年数 & \SimHei \scriptsize 公元 & \SimHei 大事件 \tabularnewline
  % \midrule
  \endfirsthead
  \toprule
  \SimHei \normalsize 年数 & \SimHei \scriptsize 公元 & \SimHei 大事件 \tabularnewline
  \midrule
  \endhead
  \midrule
  元年 & 766 & \tabularnewline\hline
  二年 & 767 & \tabularnewline\hline
  三年 & 768 & \tabularnewline\hline
  四年 & 769 & \tabularnewline\hline
  五年 & 770 & \tabularnewline\hline
  六年 & 771 & \tabularnewline\hline
  七年 & 772 & \tabularnewline\hline
  八年 & 773 & \tabularnewline\hline
  九年 & 774 & \tabularnewline\hline
  十年 & 775 & \tabularnewline\hline
  十一年 & 776 & \tabularnewline\hline
  十二年 & 777 & \tabularnewline\hline
  十三年 & 778 & \tabularnewline\hline
  十四年 & 779 & \tabularnewline
  \bottomrule
\end{longtable}


%%% Local Variables:
%%% mode: latex
%%% TeX-engine: xetex
%%% TeX-master: "../Main"
%%% End:
