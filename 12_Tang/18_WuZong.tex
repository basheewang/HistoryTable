%% -*- coding: utf-8 -*-
%% Time-stamp: <Chen Wang: 2019-12-24 15:34:42>

\section{武宗\tiny(840-846)}

\subsection{生平}

唐武宗李瀍(814年7月2日-846年4月22日)(「瀍」,拼音:chán),临死前12天改名“炎”,唐穆宗的第五子和事实上的第九子,母韦贵妃。他是唐朝的第18代皇帝(除去武则天以外),在位时间是840年至846年,在位6年,享年31岁。

唐武宗本来是唐敬宗、唐文宗的弟弟,被封为颍王。在宦官仇士良的操纵下,趁文宗病,矫诏立他为皇太弟,废原来的太子敬宗子李成美为陈王,武宗由此得以登基,并赐死李成美、文宗杨贤妃和皇兄安王李溶。

唐武宗登基後,召李黨人物李德裕回朝,任為宰相,李德裕提倡“政歸中書”等政策。在李德裕執政下,國家漸漸回復元氣,被稱為會昌中興。而仇士良的權勢亦被壓抑,仇士良不得不退下政治舞台。

唐武宗外攘回纥,内平泽潞,威震中外;更严肃整顿吏治,裁汰冗官,制驭宦官,使朝政为之一新。

唐武宗信奉道教,从845年开始他大规模下令打击佛教,史称会昌灭法。除少数在長安的寺院外,全国所有寺院被拆毁,僧尼被迫还俗,寺院所有的田地被没收为国有。

这是中国历史上佛教受打击很激烈的一次。在唐朝历史上对于佛教势力不满的现象始终存在,武宗灭佛可能有多种原因。第一可能因为唐武宗本人更加信奉道教,因此打击佛教。此外当时佛教的势力非常强大,唐武宗在他的旨意中说,佛教寺院的规模比皇宫还要大,寺院不纳税,对国家财务稅收是一个重大损失。

最后有传说认为唐武宗继位后怕有人会另立他的叔叔光王李忱(即后来的唐宣宗)来威胁他的地位,李忱则逃入佛门,因此唐武宗灭佛是为了让李忱无处可藏。但这个说法可能只是传说,因为历史学家对于李忱是否真的做过和尚仍有争议。

唐武宗吃道士给他的长寿丹后中毒而死。死后葬于端陵,谥号为至道昭肃孝皇帝。

唐武宗虽有五子皆封王,但生前未确立继承人,宦官马元贽等遂矫诏立光王李忱为皇太叔并最终继位,即唐宣宗。武宗五子后事无载,一说皆被宣宗所害。


\subsection{会昌}

\begin{longtable}{|>{\centering\scriptsize}m{2em}|>{\centering\scriptsize}m{1.3em}|>{\centering}m{8.8em}|}
  % \caption{秦王政}\
  \toprule
  \SimHei \normalsize 年数 & \SimHei \scriptsize 公元 & \SimHei 大事件 \tabularnewline
  % \midrule
  \endfirsthead
  \toprule
  \SimHei \normalsize 年数 & \SimHei \scriptsize 公元 & \SimHei 大事件 \tabularnewline
  \midrule
  \endhead
  \midrule
  元年 & 841 & \tabularnewline\hline
  二年 & 842 & \tabularnewline\hline
  三年 & 843 & \tabularnewline\hline
  四年 & 844 & \tabularnewline\hline
  五年 & 845 & \tabularnewline\hline
  六年 & 846 & \tabularnewline
  \bottomrule
\end{longtable}


%%% Local Variables:
%%% mode: latex
%%% TeX-engine: xetex
%%% TeX-master: "../Main"
%%% End:
