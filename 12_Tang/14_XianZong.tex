%% -*- coding: utf-8 -*-
%% Time-stamp: <Chen Wang: 2019-12-24 15:27:40>

\section{宪宗\tiny(805-820)}

\subsection{生平}

唐宪宗李纯(778年3月17日-820年2月14日),原名李淳,唐朝第14代皇帝(除去武则天以外),805年—820年在位。唐憲宗透過宦官俱文珍等人的協助,迫使父親唐順宗讓位予自己,即“永貞內禪”。即位後,曾一度討平不服朝廷的藩鎮,短暫終結藩鎮割據,重新統一中國,史稱“元和中興”,也因此成為唐朝繼唐太宗、唐玄宗以後,歷史評價相當高的皇帝。

唐代宗大曆十三年(778年)陰曆二月十四日,李淳(唐憲宗)在長安之東內(大明宫)內出生。

當时的皇帝是曾祖父唐代宗,祖父唐德宗为皇太子。父唐顺宗为郡王,母為莊憲皇后王氏,为妾室。

貞元四年(788年),封為广陵郡王。

貞元九年(793年)十一月丁卯,纳妃郭氏。同年,长子李宁出生。

貞元二十一年(805年)初,父亲唐顺宗繼位,唐顺宗重用王叔文、韋執誼、柳宗元、劉禹錫等官員,王叔文集團试图进行永貞改革,抑制宦官勢力。

但當時唐順宗有中風、癱瘓的情況。不滿改革的宦官俱文珍、劉光琦等人聯合劍南節度使韋皋、荊南節度使裴均、河東節度使嚴綬等外藩,迫使唐順宗立李淳為太子,改名李純,七月,太子監國。八月,俱文珍又迫唐順宗退位为上皇,傳位予長子唐憲宗,史稱永貞內禪。王叔文等二王八司馬皆被貶謫。

元和元年正月,上皇駕崩,官方說法為病死。野史影射順宗是被宦官謀殺而死,可見於唐人傳奇《辛公平上仙》。

唐宪宗继位后,決心“以法度裁制藩鎮”,开始对割据的藩镇开展了一系列战争,他在继位的次年就开始对西川节度副使刘闢开战获胜,同年夏綏軍留後杨惠琳不肯交出他的兵权,宪宗也征討他,惠琳败死。

元和二年(807年),讨伐鎮海軍節度使李錡。

元和七年(812年),魏博节度使田兴歸服唐朝,同年他开始对抗拒唐朝的成德節度使王承宗作战,但没有能够获胜,从元和十年(815年)到元和十二年(817年)唐鄧節度使李愬平定了淮西吳元濟的叛乱。

元和十三年(818年),發五道兵討淄青節度使李師道。吴元济被平定後,全国所有藩镇至少名义上全部歸服唐朝。唐朝出現短暫統一,“至是盡遵朝廷約束。”史稱“元和中兴”。

元和十四年(819年)正月,憲宗遣使往鳳翔迎釋迦牟尼佛遺骨入宮供奉,刑部侍郎韩愈上「論佛骨表」勸諫,言語不敬,皇帝大怒,差點處死了韓愈,不過最後只將韓貶为潮州刺史。這次迎佛骨陣容浩大,《資治通鑒》載“中使迎佛骨至京師,上留禁中三日,乃歷送諸寺。王公士民瞻奉舍施,唯恐不及。有謁戶充施者,有燃香燒頂供養者。”

宪宗的帝位是由宦官擁立的,因此宪宗重用宦官,军队中许多将領與監軍由宦官担任,有些宦官拥有很高的军权,但宪宗對宦官亦不優待,其晚年好长生不老之術,多服金丹,“日加躁渴”,性情暴躁易怒,動輒責罰左右黃門,宦官們不堪鞭笞。

元和十五年(820年)陰曆正月二十七日,宪宗暴卒,据说是被宦官內常侍陈弘志和王守澄合謀毒死。享年42岁,在位15年,谥圣神章武孝皇帝。大中三年,加谥昭文章武大圣至神孝皇帝。

王夫之在其《讀通鑑論》中推理认为宪宗之暴毙实则是郭氏(穆宗生母)与穆宗纵逆之所为。

\subsection{元和}

\begin{longtable}{|>{\centering\scriptsize}m{2em}|>{\centering\scriptsize}m{1.3em}|>{\centering}m{8.8em}|}
  % \caption{秦王政}\
  \toprule
  \SimHei \normalsize 年数 & \SimHei \scriptsize 公元 & \SimHei 大事件 \tabularnewline
  % \midrule
  \endfirsthead
  \toprule
  \SimHei \normalsize 年数 & \SimHei \scriptsize 公元 & \SimHei 大事件 \tabularnewline
  \midrule
  \endhead
  \midrule
  元年 & 806 & \tabularnewline\hline
  二年 & 807 & \tabularnewline\hline
  三年 & 808 & \tabularnewline\hline
  四年 & 809 & \tabularnewline\hline
  五年 & 810 & \tabularnewline\hline
  六年 & 811 & \tabularnewline\hline
  七年 & 812 & \tabularnewline\hline
  八年 & 813 & \tabularnewline\hline
  九年 & 814 & \tabularnewline\hline
  十年 & 815 & \tabularnewline\hline
  十一年 & 816 & \tabularnewline\hline
  十二年 & 817 & \tabularnewline\hline
  十三年 & 818 & \tabularnewline\hline
  十四年 & 819 & \tabularnewline\hline
  十五年 & 820 & \tabularnewline
  \bottomrule
\end{longtable}


%%% Local Variables:
%%% mode: latex
%%% TeX-engine: xetex
%%% TeX-master: "../Main"
%%% End:
