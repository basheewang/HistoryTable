%% -*- coding: utf-8 -*-
%% Time-stamp: <Chen Wang: 2019-12-23 17:53:22>

\section{太宗\tiny(626-649)}

\subsection{生平}

唐太宗李世民(598年1月23日-649年7月10日),唐朝第二任皇帝。出自陇西成纪,626年至649年在位。父親是唐高祖李淵,母親是太穆皇后窦氏。竇皇后有四個兒子,一個女兒,按長幼顺序為:李建成、平陽昭公主、李世民、李玄霸、李元吉。吐火羅葉護稱唐太宗為「天可汗」。

李世民少年从军,曾于雁门关营救隋炀帝。唐朝建立后,李世民受封为秦公,后晋封为秦王,他是杰出的軍事家,率部平定了薛仁果、刘武周、竇建德、王世充等隋末群雄,在唐朝的建立与统一过程中立下赫赫战功,最終統一天下。

武德九年(626年)发动玄武門之變殺死自己的兄长太子李建成、四弟齐王李元吉二人及二人诸子,被立为太子,唐高祖李渊不久被迫退位,李世民即位,在位時間只使用一個年号贞观。

李世民登基後,積極聽取群臣的意見,以文治天下,并开疆拓土,成為中國史上著名的明君。他虛心納諫,在國內厲行節約,使百姓能够休養生息,終於使得社會出現了國泰民安的局面,開創了中国历史上著名的貞觀之治,為唐朝130年的盛世奠定重要基礎。李世民爱好文学与书法,其真迹今仅存晋祠之铭并序碑刻。649年7月10日(贞观二十三年五月己巳日),唐太宗李世民因病驾崩于含风殿,享年52岁,在位23年,庙号太宗 ,諡號「文皇帝」,葬于昭陵。

隋文帝開皇十七年十二月戊午日(598年1月23日),李世民出生於岐州武功縣,是當時擔任隋朝岐州刺史的漢人李淵與竇氏所生的嫡次子。

隋炀帝大业九年(613年),他的母親在涿郡(治今北京市西南)病逝,高士廉看中了李世民,把外甥女長孫氏(登基後稱長孫皇后)許配給李世民為妻。

隋大業十一年(615年),雲定興被授以左屯衛大將軍,奉命援救在雁門關被突厥始畢可汗所率大軍圍困的隋煬帝。雲定興向各地招募願意出征的軍士,李世民應募從軍,被劃歸雲定興的帳下,並建議使用疑兵之計攻突厥,雲定興採納,突厥軍遂撤退。

隋大業十二年(616年),父親李淵升任隋朝右驍衛將軍。

大業十三年(617年)正月,李淵又遷任太原郡留守。同年7月,殺郡丞王威、武牙郎將高君雅,李世民也跟隨到太原勸諫父親起兵反隋,接著打著“勤王定亂,迎回隋天子”的旗號,正式開始於晉陽縣起兵,並且得到李氏宗族、姻親的響應,是為晉陽起兵。李淵以李世民為敦煌公、右領軍都督、統右三軍。

李渊诛杀了隋炀帝派来监视他的王威、高君雅。派刘文静出使东突厥得到了始毕可汗的支持,派李建成、李世民夺取西河郡。六月,正式起兵。李渊自为大将军,以长子李建成、次子李世民为左右大都督,以四子李元吉留守太原,进兵大兴城(长安)。李渊在霍邑大破宋老生,从龙门渡黄河,开永丰仓赈济百姓。关中有其女李三娘等人起兵响应。

十一月,李渊攻克大興,以代王杨侑为皇帝,尊隋炀帝杨广为太上皇,李渊自为大丞相、唐王。次年三月,隋炀帝杨广在江都被宇文化及所杀,五月,李渊废黜杨侑,称帝,改国号为唐,定都大興,易名長安,唐朝建立。攻克隋大兴城後,李世民官拜京兆尹、受封秦國公。618年,遷趙國公。李淵稱帝后,李世民拜尚書令、晉為秦王。

唐朝建立後,疆土只限于关中和河东一带,尚未完全统治全国,因此,李世民經常出征,最终統一中國。自武德元年起,李世民親自参與四場大戰役。

其一,破薛舉,淺水原平定隴西薛仁果(薛舉之子),平定祖宗之地。其二,敗宋金剛、劉武周,收復並、汾失地,消灭北方地方军阀。其三,在虎牢之戰中,一舉殲滅中原兩大割據勢力—河南王世充和河北竇建德集團,消除河北、河南的地方势力。其四,重創竇建德餘部劉黑闥和山東的徐圓朗。

自此李世民威望日隆,尤其是在虎牢之戰後班師返京時,受到长安軍民的隆重歡迎。武德四年十月,封為天策上將,領司徒、陝東道大行臺尚書令,食邑增至二萬戶。李淵又下詔特許天策府自置官屬,李世民因此闢弘文館,收攬四方彥士入館備詢顧問,與秦王府相結合,儼然一個小內閣。

618年,李渊建立唐朝為唐高祖,并立世子李建成为太子。太原起兵是李世民的谋略,高祖曾答应他事成之后立他为太子,但天下平定后,李世民功名日盛,高祖却犹豫不决。太子李建成随即联合四弟齐王李元吉,共同排挤李世民。同时,高祖的优柔寡断,也使朝中政令相互冲突,加速了诸子的兵戎相见。

此後,長兄皇太子李建成知李世民終不肯屈為人臣,而李世民也認為是自己奠下唐朝開國的基業,遂與李建成、四弟齊王李元吉猜忌日深,兩派大臣之間互相傾軋。李世民曾在李建成东宫饮酒,吐血数升,怀疑李建成下毒。

其中宰相裴寂、謀士王珪、魏徵、東宮衛士將領薛萬徹等追隨李建成、李元吉。秦府謀士杜如晦、房玄齡,將領秦叔寶、尉遲敬德、段志玄、侯君集、王君廓等跟從李世民。宰相陳叔達、朝臣長孫無忌等暗中支援李世民。其餘將領李靖、李世勣,大臣宇文士及等保持中立。

武德九年,突厥侵犯唐边境,李建成向高祖建议由李元吉做统帅出征突厥。太子府率更丞王晊告诉了秦王:李建成想借此控制秦王的兵马,并准备在昆明池设伏兵杀李世民,于是李世民决定先发制人。武德九年六月初四庚申日(626年7月2日),李世民在首都長安城宮城的北门玄武门附近射殺皇太子李建成、齊王李元吉,史稱「玄武門之變」。

此後高祖讓出軍政大權予秦王,而建成、元吉则被宣布为作乱者,諸子则遭诛杀并从宗籍中除名。三天后(六月初七癸亥日,7月5日),李世民被立为皇太子,诏曰:“自今军国庶事,无大小悉委太子处决,然后闻奏”。八月初九甲子日(9月4日),高祖退位稱太上皇,禪位於李世民。李世民登基,是为唐太宗。当年十月,追封李建成为息隐王,李元吉为海陵剌王。次年改元貞觀。642年,追复李建成为隐太子,李元吉为巢剌王,并将皇子李福过继李建成为嗣(后来另一皇子李明也在唐高宗年间被出继李元吉为嗣)。

貞觀二年(628年),当时的人口已因隋末戰爭而銳減,此时唐朝只有290萬戶,經太宗君臣23年的努力,社會安定、經濟恢復並穩定發展,至唐高宗永徽三年(652年),人口達到380萬戶,奠下了高宗、武則天、玄宗年間大唐盛世的基礎,史稱貞觀之治。

貞觀二年四月二十六壬寅日(628年6月3日),朔方人梁洛仁杀夏州割据势力首领梁师都,归降唐朝,唐朝统一全国。貞觀四年(630年),太宗令李靖出師塞北,挑戰東突厥在東亞的霸主地位。唐軍在李靖的調遣下,滅亡東突厥,太宗因此被西域諸國尊為「天可汗」。在位期間,積極推行了府兵制、租庸調制和均田制,並加強科舉制等政策。

在民族政策方面,太宗一方面扩张疆土,另一方面又接受了拓跋魏、北齐、北周二百多年的历史现实,提出其蕃汉兼包、一视同仁的民族政策。李世民曾对他的左右大臣说:「自古帝王皆貴中華,賤夷狄,朕獨愛之如一,故其種落皆視朕如父母。」

太宗本身也是個既英武又善辯之人,但是有鑑於帝位得之不易,加上隋煬帝本人亦以雄健爾雅善辯聞名,隋卻因此鑄下滅亡的大錯,因此在位期間,太宗鼓勵群臣批評他的決策和風格。其中魏徵廷諫了200多次,在廷上直陳太宗的過失,在早朝時多次發生了使太宗尷尬、下不了臺的狀況。晚年的太宗因國富民強,納諫的器度不如初期,偶爾也發生誤殺大臣的遺憾,但是大致上仍克制、保有納言的風範。641年,唐室文成公主下嫁於吐蕃的松贊干布。

太宗即帝位不久,按秦王府文學館模式,新設弘文館,進一步儲備天下文才。另外,太宗精擅書法,以行書寫碑,稱「飛白」,聞名後世。著名作品有《溫泉銘》、《晉祠銘》等。晚年太宗著《帝範》一書以教戒太子李治,總結了他的施政經驗,同時自評一生功過。史家曾疑太宗生前,指定以東晉書法大家王羲之所作《蘭亭集序》為陪葬品。近年據考古學家和歷史學者研究,《蘭亭集序》應該不在太宗之昭陵,而在高宗、武則天所合葬的乾陵之中。

唐太宗與身邊大臣魏徵、王珪、房玄齡、杜如晦、虞世南、褚遂良等的對答也在開元十八(730年)、十九年間被吳兢輯為《貞觀政要》一書,以發揚太宗勵精圖治的治國精神。

武德九年(626年)八月,因唐朝發生玄武門之變,政局不穩,東突厥伺機入侵,攻至距首都長安僅40裡的涇陽(今陝西咸陽涇陽縣),京師震動。此時,長安兵力不過數萬,剛剛即位的太宗被迫設疑兵之計,親率高士廉、房玄齡等6騎在渭水隔河與頡利可汗對話,怒斥頡利、突利二可汗背約。《唐語林》記載太宗「空府庫」贈予頡利可汗金帛財物,以求突厥退軍,並與之結「渭水之盟」,突厥兵於是退去。之後,太宗勵精圖治,並且挑撥頡利、突利二可汗和突厥與鐵勒諸部的關係。627年,東突厥內部出現分裂。反對頡利可汗的薛延陀、回紇、拔也古、同羅諸部落對其變革國俗和推行的政令不滿,另立薛延陀為可汗。突利可汗也暗中與唐聯絡,並與頡利可汗決裂。同時東突厥又遇到大雪氣候,牲畜大多被凍死餓死,突厥勢力漸弱。太宗於629年八月任命李靖、李世勣、柴紹、李道宗等為行軍總管,出兵征討東突厥。630年三月頡利兵敗被俘,東突厥滅亡。唐朝在東突厥突利可汗故地設置順、祐、化、長四州都督府,頡利可汗故地置定襄都督府、雲中都督府。

東突厥滅亡後,薛延陀的真珠可汗乙失夷男接管了東突厥的故土。薛延陀表面臣服於唐朝,暗中却扩充自己的力量。639年,太宗試圖恢復東突厥,立俟力苾可汗阿史那思摩,以抗衡薛延陀的崛起,薛延陀为避免新恢复的东突厥站稳脚跟,與其進行了多次戰爭。為保住東突厥,李世勣在641年进攻薛延陀,并取得了胜利。但是644年,趁太宗征伐高句麗的機會,薛延陀部隊發起新一輪攻勢,擊敗東突厥,迫使阿史那思摩逃回云州。隨後,高句麗尋求薛延陀援助,但夷男希望避免與唐朝直接戰鬥。645年,夷男死後,他的兒子多彌可汗拔灼開始和唐軍作戰。646年,唐軍反擊並打敗拔灼後,薛延陀的附庸回紇、鐵勒等部落出兵,將他殺死。拔灼的堂兄伊特勿失可汗咄摩支向唐軍投降,薛延陀滅亡。太宗於鐵勒故地設六府七州:瀚海府(回紇)、金微府(僕骨)、燕然府(多濫葛)、盧山府(思結)、龜林府(同羅)、幽陵府(拔野古)。七州:皋蘭州(渾)、高闕州(斛薛)、雞鹿州(奚結)、雞田州(阿跌)、榆溪州(契苾)、蹛林州(思結別部)、窴顏州(白霫)。由燕然都護府管理,治所在陰山之麓(今內蒙古杭錦後旗),轄境東到大興安嶺、西到阿爾泰山、南到戈壁、北到貝加爾湖的整個蒙古高原。

634年,吐蕃贊普松贊干布遣使與唐朝修好,唐朝也派臣入蕃。636年,松贊干布派專使去長安請婚,唐朝不允許。638年,松贊干布遂借口唐朝屬國吐谷渾從中作梗,亲自指挥大约20万吐蕃军,开始攻击唐朝的松州(今四川阿坝藏族羌族自治州)。但同时松赞干布又派遣使者到唐朝国都长安再次请求,并宣称他们打算欢迎公主。唐太宗派侯君集为当弥道行军大总管指挥5万军队,执失思力、牛进达、刘简协助,援救松州。与此同时,吐蕃军正在围困松州的首县-嘉诚(今四川松潘),但唐军先遣部队在牛进达指挥下,打败了吐蕃军。唐軍在松州大勝吐蕃軍,但唐朝也見識到了吐蕃的力量。640年,松贊干布又派大臣祿東贊使唐求婚,唐太宗便以宗室之女文成公主許嫁於吐蕃贊普松贊干布,並派禮部尚書江夏王李道宗持節護送。641年文成公主入蕃,《新唐書》記載松贊干布親迎於柏海,文成公主進蕃時把各種漢地的生產技術轉入吐蕃。

唐太宗滅東突厥後,開始對西域(即現代新疆和中亞地區)的西突厥以及一些鬆散結盟綠洲國家的施加軍事實力,其主要針對西突厥,以恢復兩漢以來對西域的統治。高昌王麴文泰與西突厥欲谷設聯合,阻礙西域商路,進攻唐朝的伊州。639年冬,太宗以侯君集為交河道行軍大總管,率兵出擊高昌王麴文泰。640年,唐軍至磧口,麴文泰驚懼而病死。其子麴智盛即位後不久,侯君集圍城,麴智盛降唐軍。高昌國三州、五縣、二十二城,八千戶、三萬餘人歸屬唐朝,高昌國結束。唐朝在高昌設置西州。

吐谷渾可汗伏允聽信大臣天柱王的建議,屢次侵犯唐朝的西部邊境,634年,扣留唐朝使者趙德楷,六月,太宗以段志玄為行軍總管,討伐伏允,十二月,又以李靖、侯君集、李道宗等為行軍總管,大舉討吐谷渾。635年,伏允敗走,被部下所殺。伏允之子慕容順殺死天柱王,自立為可汗,投降唐朝,太宗冊封慕容順為吐谷渾可汗。慕容順死後,636年,太宗冊封慕容順之子諾曷缽為吐谷渾可汗。

640年,唐朝在交河城設安西都護府,用以針對西突厥和管理西域。644年,西突厥的盟友焉耆攻打西州,安西都護郭孝恪為西州道行軍總管,討伐依附西突厥的焉耆,佔領焉耆,俘虜國王龍突騎支,但後來焉耆再次脫離唐朝。648年,唐太宗派遣阿史那社爾、郭孝恪率軍討伐依附西突厥的焉耆和龜茲(今新疆阿克蘇),征服兩國。然後疏勒和于闐歸附唐朝,將安西都護府遷至龜茲,撫寧西域,統龜茲、焉耆、于闐、疏勒四國,史稱安西四鎮。

貞觀四年,西域諸國君主在長安尊稱太宗為「天可汗」,意為天下總皇帝或天下共主。「天可汗」除了是一種對唐朝皇帝的榮銜,還是一種有實質意義的國際組織體系,以維持當時各同盟國的集體安全。

642年,高句麗東部大人淵蓋蘇文殺死榮留王後立高藏為王並自封為「大莫離支」攝政。為征討淵蓋蘇文和保護唐朝的盟友新羅,唐太宗認為有必要對高句麗開戰。644年,太宗率領李世勣、李道宗、張亮和長孫無忌統軍10萬親征高句麗。645年,太宗衝破高句麗的防線準備攻打高句麗國都平壤,似乎大功在即。不料在安市(今遼寧鞍山)受阻,再也無法前行。在這之後,太宗對高句麗的進攻僅維持在一些小規模的突襲。646年,唐朝與回紇擊滅薛延陀後,647年,唐太宗令牛進達率兵從海上、李世勣率兵從陸路攻打遼東半島。648年,太宗再派薛萬徹率軍從海上攻打鴨綠江口。隨後,唐開始集結陸海部隊準備在649年再一次大規模攻高句麗。不過太宗於649年去世後,新皇帝唐高宗暫停東征的計劃。668年,高宗聯合新羅滅亡高句麗,載籍戶數69.7萬。並建立安東都護府等加以控制遼東。

在北方,貞觀四年(630年),唐军滅亡東突厥,漠南成為唐势力范围。貞觀二十年(646年),又一舉消滅了薛延陀汗國,至此大漠南北广大地区皆為唐的势力范围。唐朝廷在漠北設立安北都护府,在漠南設立单于都护府,建立了南至罗伏州(今越南河静)、北括玄阙州(後改名余吾州,今安加拉河地区)、西及安息州(今乌兹别克斯坦布哈拉)、东临哥勿州(今吉林通化)的辽阔疆域。

在西北,貞觀四年,唐朝廷在伊吾七城設立西伊州,開始經營西域。貞觀二十二年(648年),郭孝恪击败龟兹国,把安西都护府治所迁至龟兹。

在东北,644年唐太宗征讨高句丽未果,唐高宗在668年乃联合新罗灭高句丽,设立安东都护府。

在青藏高原上,吐蕃日漸興起,至六世紀末與吐谷浑、蘇毗為高原上三大勢力。七世纪初,贊普松贊干布即位,統一了高原,又征服了位於西藏西部的蘇毗、阿里地區的羊同和尼婆羅(今尼泊尔)。松贊干布於634年遣使與唐朝修好,唐朝也派臣入蕃。636年,松贊干布派專使去長安請婚,唐朝不允,638年,松贊干布遂借口唐朝屬國吐谷渾從中作梗,出兵入侵吐谷渾,唐軍在松州大勝吐蕃軍。640年,松贊干布又派大臣祿東贊使唐求婚,唐太宗便以宗室之女文成公主許嫁於吐蕃贊普松贊干布,並派禮部尚書江夏王李道宗持節護送,吐蕃赞普遂接受唐朝的册封。

面對自己空前的文治武功,太宗到晚年也出現一些過失。首先納諫不如貞觀早期積極,比如貞觀十年,魏徵發現他「漸惡直言」。其次奢侈之風日重。不過晚年他還是能反省自己過度奢靡的錯誤。司馬光說唐太宗:「好尚功名,不及禮樂,父子兄弟之間,慚德多矣」。同时,太宗晚年也由早年的清静转为奢纵,营建宫殿,计划封禅泰山等,并自辩“百姓无事则骄逸,劳役则易使”,魏征因此谏到“恐非兴邦之至言,岂安人之长算?”不过由于太宗晚年能够清醒认识自己的问题,所以也能进行调整,因此虽然太宗晚年存在这些过失,最终没有出现败亡的危机,“功大过微,故业不堕”,维持了贞观之治的局面。

《資治通鑑》記載,太宗貞觀十七年廢太子李承乾之後、改立李治為皇太子之前,李世民之三子一弟(長子李承乾、四子李泰、五子李祐、及七弟李元昌)俱謀取帝位,致太宗心灰意冷之曲折,史載:「承乾既廢,上御兩儀殿,群臣俱出,獨留長孫無忌、房玄齡、李世勣、褚遂良,謂曰:『我三子一弟,所為如是,我心誠無聊賴!』因自投於床,無忌等登前扶抱,上又抽佩刀欲自刎,遂良奪刀以授晉王治。」

貞觀二十二年(648年)正月,唐太宗撰寫《帝範》十二篇頒賜給太子李治。貞觀二十三年(649年),唐太宗得了痢疾(一種傳染病),醫治最終無效(一說是服用天竺长生藥无效),命李治到金掖門代理國事。貞觀二十三年五月二十六日(649年7月10日),唐太宗李世民崩逝于終南山翠微宮含風殿內,享年五十一岁,在位二十三年,初谥文皇帝,庙号太宗,唐高宗上元元年(674年)加谥文武圣皇帝,唐玄宗天寶八年(749年)加谥文武大圣皇帝,天寶十三年(754年)加谥文武大圣大广孝皇帝,安葬於唐昭陵(位于今中國陝西省禮泉縣東北50多里山峰上)。

《貞觀政要》贊貞觀之治:官吏多自清謹,王公妃主之家,大姓豪猾之伍,無敢侵欺細人。商旅野次,无复盗贼,囹圄常空,去年犯死者仅二十九人。又频致丰稔,米斗三钱,马牛布野,外户不闭,行旅自京师至于岭表,自山东至于沧海,皆不赍粮,取给于路。入山东村落,行客经过者,必厚加供待,或发时有赠遗。此皆古昔未有也。

后晋官修正史《旧唐书》刘昫等的評價是:“史臣曰:臣观文皇帝发迹多奇,聪明神武。拔人物则不私于党,负志业则咸尽其才。所以屈突、尉迟,由仇敌而愿倾心膂;马周、刘洎,自疏远而卒委钧衡。终平泰阶,谅由斯道。尝试论之:础润云兴,虫鸣螽跃。虽尧、舜之圣,不能用檮杌、穷奇而治平;伊、吕之贤,不能为夏桀、殷辛而昌盛。君臣之际,遭遇斯难,以至抉目剖心,虫流筋擢,良由遭值之异也。以房、魏之智,不逾于丘、轲,遂能尊主庇民者,遭时也。或曰:以太宗之贤,失爱于昆弟,失教于诸子,何也?曰:然,舜不能仁四罪,尧不能训丹朱,斯前志也。当神尧任谗之年,建成忌功之日,苟除畏逼,孰顾分崩,变故之兴,间不容发,方惧“毁巢”之祸,宁虞“尺布”之谣?承乾之愚,圣父不能移也。若文皇自定储于哲嗣,不骋志于高丽;用人如贞观之初,纳谏比魏徵之日。况周发、周成之世袭,我有遗妍;较汉文、汉武之恢弘,彼多惭德。迹其听断不惑,从善如流,千载可称,一人而已!赞曰:昌、发启国,一门三圣。文定高位,友于不令。管、蔡既诛,成、康道正。贞观之风,到今歌咏。”

北宋官修正史《新唐书》欧阳修、宋祁等的評價是:“甚矣,至治之君不世出也!禹有天下,传十有六王,而少康有中兴之业。汤有天下,传二十八王,而其甚盛者,号称三宗。武王有天下,传三十六王,而成、康之治与宣之功,其余无所称焉。虽《诗》、《书》所载,时有阙略,然三代千有七百余年,传七十余君,其卓然著见于后世者,此六七君而已。呜呼,可谓难得也!唐有天下,传世二十,其可称者三君,玄宗、宪宗皆不克其终,盛哉,太宗之烈也!其除隋之乱,比迹汤、武;致治之美,庶几成、康。自古功德兼隆,由汉以来未之有也。至其牵于多爱,复立浮图,好大喜功,勤兵于远,此中材庸主之所常为。然《春秋》之法,常责备于贤者,是以后世君子之欲成人之美者,莫不叹息于斯焉。”

《新唐书·北狄列传》:唐之德大矣!際天所覆,悉臣而屬之;薄海內外,無不州縣,遂尊天子曰“天可汗”。三王以來,未有以過之。至荒區君長,待唐璽纛乃能國;一為不賓,隨輒夷縛。故蠻琛夷寶,踵相逮於廷。

朱熹与陈亮书:“太宗之心,则吾恐其无一不出于人欲也。直以其能假仁假义,以行其私。而当时与之争者,才能知术既出其下,又不知有仁义之可饬。是以彼善于此,而得以成其功尔。”“论后世人,不当尽绳以古人礼法。毕竟高祖不当立建成。”“太宗功高,天下所系属,亦自无安顿处,只高祖不善处置了。”

文天祥《古代状元卷:文天祥殿试卷》:太宗全不知道、闺门之耻、将相之夸、末年辽东一行、终不能以克其血气之暴、其心也骄。

元朝戈直在《貞觀政要》集論中說:“夫太宗之於正心修身之道,齊家明倫之方,誠有愧於二帝三王之事矣。然其屈己而納諫,任賢而使能,恭儉而節用,寬厚而愛民,亦三代而下,絕無而僅有者也。後之人君,擇其善者而從之,其不善者而改之,豈不交有所益乎!”這裡所說,太宗在正心修身,齊家明倫方面,有愧于二帝三王之事,主要是指太宗與其兄李建成的皇位之爭。

明朝官修皇帝实录《明太祖实录》记载,明太祖朱元璋在洪武七年八月初一日(1374年9月7日),亲自前往南京历代帝王庙祭祀三皇、五帝、夏禹王、商汤王、周武王、汉高祖、汉光武帝、隋文帝,唐太宗、宋太祖、元世祖一共十七位帝王,其中对唐太宗李世民的祝文是:“惟唐太宗皇帝英姿盖世,武定四方,贞观之治,式昭文德。有君天下之德而安万世之功者也。元璋以菲德荷天佑人助,君临天下,继承中国帝王正统,伏念列圣去世已远,神灵在天,万古长存,崇报之礼,多未举行,故于祭祀有阙。是用肇新庙宇于京师,列序圣像及历代开基帝王,每岁祀以春、秋仲月,永为常典。今礼奠之初,谨奉牲醴、庶品致祭,伏惟神鉴。尚享!”

明憲宗在命儒臣訂正重刊《貞觀政要》時寫道:“太宗在唐為一代英明之君,其濟世康民,偉有成烈,卓乎不可及已。所可惜者,正心修身,有愧于二帝三王之道,而治未純也。”

毛泽东评价李世民说:“自古能军无出李世民之右者,其次则朱元璋耳。”

王仲荦《隋唐五代史》:“唐代的皇帝裡,唐太宗,早年的唐玄宗,唐宣宗,都是杰出的皇帝。”“我们认为旧日的封建歷史家对‘貞觀之治’是渲染得有點過分的。……固然,在唐太宗统治的二十多年间,人口有了较大的增长,但比之隋極盛时户数,还不到二分之一。”“魏徵疏文中也说到:‘今自伊洛以东,暨于海岱,灌莽巨泽,茫茫千里、人煙斷絕,鸡犬不闻。道路萧条,进退艰阻。’”“封建歷史家把貞觀時期當作理想的太平盛世,和實際情況是有很大距離的。”

吕思勉《隋唐五代史》:“唐太宗不过中材,论其恭俭之德,及忧深思远之资,实尚不如宋武帝,更无论梁武帝;其武略亦不如梁武帝,更无论宋武帝,陈武帝矣!”

据《贞观政要》李世民的生日是十二月癸丑,据《资治通鉴》李世民的生日是十二月癸未,据《旧唐书》李世民生于隋开皇十八年十二月戊午(599年1月23日),因此李世民的生日应为十二月份。据《旧唐书》李世民卒年五十二岁,其弟李玄霸无考;据《新唐书》李世民卒年五十三岁,其弟李玄霸年十六岁死于隋大业十年(614年),则李玄霸生卒年为公元599-614年,而李世民生卒年为公元597-649年;李世民以十二月出生,李世民生卒年月为598年1月-649年7月,与李玄霸(599-614)为同母兄弟。《新唐书》推翻了《旧唐书》关于李世民的生卒年月,增加了李玄霸的生卒年,使李世民与李玄霸的生卒更可信。胡如雷著《李世民传》即以《新唐书》为依据,考证李世民的出生年月为隋开皇十七年十二月戊午(598年1月28日)。

《新唐书》增加了李玄霸的生卒年岁,补正了李世民的生卒年岁,补充了《旧唐书》中没有的珍贵史料,《新唐书》与《旧唐书》同被列为《二十四史》之钦定官史。据胡如雷考证:“李世民生于开皇十八年十二月之说亦难成立,因窦氏在不到十三个月的时间里先后两次生子的可能性虽然不能完全排除,但就常情而言,这种可能性也不大”。根据李世民同母弟李玄霸十六岁时死于大业十年,而倒推出李玄霸生于开皇十九年,所以若李世民生于开皇十八年十二月,则李玄霸最迟生于开皇十九年十二月,两兄弟生辰过近,不太可能。

貞觀九年十月,即李淵死後五個月,李世民第一次要求觀覽《起居注》,未遂。

《貞觀政要·卷七·論文史第二十八》記載:貞觀十三年,褚遂良為諫議大夫,兼知太宗《起居註》。唐太宗欲查看起居註,褚遂良以「不聞帝王躬自觀史」為由拒絕了。唐太宗說:「朕有不善,卿必記耶?」褚遂良說:「臣職當載筆,何不書之?」黃門侍郎劉洎進言:「人君有過失,如日月之蝕,人皆見之。設令遂良不記,天下之人皆記之矣。」《舊唐書·褚遂良傳》和《資治通鑑·唐紀十二》也載有此事。

次年(640年),唐太宗再度要求看《起居注》,宰相房玄齡等人就刪減整理國史,撰寫成《高祖實錄》和《太宗實錄》各二十卷。當太宗見到「六月四日事,語多微文」——指史官對當年玄武門事變內容含糊其辭,多有隱諱文飾之語,太宗告訴房玄齡:不必替他遮遮掩掩,反正玄武門事件本來就是像「周公誅管、蔡,季友鴆叔牙」之義舉,目的是為了「安社稷、利萬民」,要求「削去浮詞,直書其事」。《資治通鑑·唐紀十三》亦有記載。

這一行為帶給史學考究極大困難,也遭到章太炎等學者指責:「太宗即立,懼於身後名,始以宰相監修國史,故兩朝《實錄》無信辭。」

《大唐新語·卷一》載,太宗繼位後曾在苑囿內狩獵,一群野豬從森林中衝出。太宗舉弓四箭射殺了四隻,但還是有一頭雄野豬向馬匹直衝而來。吏部尚書唐儉慌忙下馬,與之搏鬥。太宗拔劍砍死野豬,笑著對唐儉說,「天策長史,不見上將擊賊耶?何懼之甚!」唐儉當即回答道:「漢祖以馬上得之,不以馬上理之。陛下以神武定四方,豈復逞雄心於一獸!」太宗覺得唐儉說得有理,於是停止了狩獵。

由於唐太宗在即位前曾當過尚書令,故當太宗做皇帝後,大臣多不敢任其職,於是之後這個職務就幾乎不授人,尚書省的長官就只設左、右僕射,後用其他官員以「同中書門下三品」的頭銜參預朝政,執行宰相職務。至高宗時,又用低級官員以「同中書門下平章事」的頭銜參預朝政,執行宰相職務。左、右僕射成了聽令執行的官員,不能參加大政,唐中宗神龍革命復辟之後,僕射就非宰相職務。中書令、侍中在安史之亂後也不常設了。同中書門下平章事成了宰相最普遍的名稱。

《舊唐書·本紀第二:太宗上》记载,李世民四歲時,其父李淵任岐州刺史,有一書生自稱善相,拜訪李淵說:“公贵人也,且有贵子。”見到李世民時又說:“龙凤之姿,天日之表,年将二十,必能济世安民矣。”李淵害怕這話走漏,派人去追殺書生,書生卻忽然失蹤了。於是李淵就取“濟世安民”之意給李世民命名。

李世民酷愛書法,其書法以隸書見長,並且酷愛書法名品《蘭亭序》(即《蘭亭集序》,王羲之書法珍品,王羲之的字十分多變,就一「之」字就有十數種變化之多),相傳當年某大臣見太宗似有鬱結難紓,問之原因,知道其欲得《蘭亭序》,於是便與辯才和尚(王羲之當年墨寶輾轉傳至其七世孫智永,智永出家為僧,又將墨寶傳予其弟子辯才和尚)鬥智最後終於為李世民獲得。而王羲之本願並不想《蘭亭序》落入君王之手成為陪葬品。但最後結果事與願違,《蘭亭序》最終成為唐太宗的陪葬品。

據新舊唐書太宗本紀,李世民十六歲時參軍,跟隨隋將雲定興,一次隋煬帝楊廣被圍,雲定興軍負責救駕,李世民獻計,故佈疑陣,嚇退敵軍,救回天子。


\subsection{贞观}

\begin{longtable}{|>{\centering\scriptsize}m{2em}|>{\centering\scriptsize}m{1.3em}|>{\centering}m{8.8em}|}
  % \caption{秦王政}\
  \toprule
  \SimHei \normalsize 年数 & \SimHei \scriptsize 公元 & \SimHei 大事件 \tabularnewline
  % \midrule
  \endfirsthead
  \toprule
  \SimHei \normalsize 年数 & \SimHei \scriptsize 公元 & \SimHei 大事件 \tabularnewline
  \midrule
  \endhead
  \midrule
  元年 & 627 & \tabularnewline\hline
  二年 & 628 & \tabularnewline\hline
  三年 & 629 & \tabularnewline\hline
  四年 & 630 & \tabularnewline\hline
  五年 & 631 & \tabularnewline\hline
  六年 & 632 & \tabularnewline\hline
  七年 & 633 & \tabularnewline\hline
  八年 & 634 & \tabularnewline\hline
  九年 & 635 & \tabularnewline\hline
  十年 & 636 & \tabularnewline\hline
  十一年 & 637 & \tabularnewline\hline
  十二年 & 638 & \tabularnewline\hline
  十三年 & 639 & \tabularnewline\hline
  十四年 & 640 & \tabularnewline\hline
  十五年 & 641 & \tabularnewline\hline
  十六年 & 642 & \tabularnewline\hline
  十七年 & 643 & \tabularnewline\hline
  十八年 & 644 & \tabularnewline\hline
  十九年 & 645 & \tabularnewline\hline
  二十年 & 646 & \tabularnewline\hline
  二一年 & 647 & \tabularnewline\hline
  二二年 & 648 & \tabularnewline\hline
  二三年 & 649 & \tabularnewline
  \bottomrule
\end{longtable}


%%% Local Variables:
%%% mode: latex
%%% TeX-engine: xetex
%%% TeX-master: "../Main"
%%% End:
