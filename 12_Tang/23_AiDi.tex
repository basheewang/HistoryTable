%% -*- coding: utf-8 -*-
%% Time-stamp: <Chen Wang: 2019-12-24 15:45:17>

\section{哀帝\tiny(904-907)}

\subsection{生平}

唐哀帝李柷(「柷」,拼音:zhù,注音:ㄓㄨˋ;892年10月27日-908年3月26日),原名祚。唐昭宗第九子,生母何皇后。是唐朝第23位皇帝(除武曌以外),904年-907年在位,在位3年,被废。次年死,得年16岁,葬于温陵。

始封辉王。天复三年(903年)二月,唐昭宗为褒赏宣武军节度使朱全忠,想以皇子为诸道兵马元帅,朱全忠为副帅。宰相崔胤请让李祚为此职,昭宗则因长子濮王李𥙿年长而有意濮王。崔胤受朱全忠所托,认为李祚年幼可利用,坚持请求,于是李祚被任为诸道兵马元帅。

天祐元年(904年),崔胤图谋对抗朱全忠,被杀。朱全忠迫昭宗迁都洛阳,八月遣枢密使蒋玄晖等弑唐昭宗,以昭宗名义下诏立李祚为皇太子,改名李柷,监军国事,又由宰相柳璨、独孤损矫何皇后诏奉其继位。次月,哀帝尊母何皇后为皇太后。

唐哀帝即位时,不过是朱全忠手中的一个傀儡皇帝。二年(905年),掌握实际权力的朱全忠见废帝灭唐时机已到,便先让蒋玄晖借设宴之机,将包括哀帝同胞兄长李裕在内的兄弟九人杀害于九曲池,再大量杀害朝臣(见白马驿之祸);后又因蒋玄晖与太常卿张廷范认为天下未平,时机未到,不宜受九锡而不悦,十二月听信宣徽副使王殷、赵殷衡诬告,认为蒋玄晖、张廷范、柳璨与何太后图谋复兴唐室,遂遣使捕杀蒋玄晖,派王、赵将何太后缢死于其住所积善宫,并迫哀帝下诏,称何太后秽乱宫闱自杀谢罪,追回皇太后宝册,追废为庶人。哀帝因太后丧废朝三日,并以太后因宫闱丑闻自杀为由,取消新年郊礼。张廷范和柳璨也被外贬,未及赴任皆被处死。

接着朱全忠在天祐四年(907年),又逼李柷禅位,降为济阴王,自己做皇帝,改名朱晃,是为后梁太祖,建国号“大梁”,史称“后梁”,改元“开平”。至此,立国总计289年(618-907)、传21帝的唐王朝灭亡,中国进入自黃巾之亂或永嘉之乱以来又一次大分裂时期——五代十国。

遜位後,李柷被迫遷到曹州(在今山東省內),次年(908年)二月二十一日被朱晃毒死。朱晃上谥号为哀皇帝,以王禮葬於济阴县之定陶鄉。

后唐明宗李嗣源改哀帝的谥号为昭宣光烈孝皇帝,但哀帝因系篡位奸臣朱全忠所立的傀儡且本人及父母都被朱全忠杀害,被认为不够称“宗”,故由横海军节度使卢质提议的庙号景宗未被採用,故后世称李柷为唐哀帝或唐昭宣帝。

\subsection{天佑}

\begin{longtable}{|>{\centering\scriptsize}m{2em}|>{\centering\scriptsize}m{1.3em}|>{\centering}m{8.8em}|}
  % \caption{秦王政}\
  \toprule
  \SimHei \normalsize 年数 & \SimHei \scriptsize 公元 & \SimHei 大事件 \tabularnewline
  % \midrule
  \endfirsthead
  \toprule
  \SimHei \normalsize 年数 & \SimHei \scriptsize 公元 & \SimHei 大事件 \tabularnewline
  \midrule
  \endhead
  \midrule
  元年 & 904 & \tabularnewline\hline
  二年 & 905 & \tabularnewline\hline
  三年 & 906 & \tabularnewline\hline
  四年 & 907 & \tabularnewline
  \bottomrule
\end{longtable}


%%% Local Variables:
%%% mode: latex
%%% TeX-engine: xetex
%%% TeX-master: "../Main"
%%% End:
