%% -*- coding: utf-8 -*-
%% Time-stamp: <Chen Wang: 2021-10-29 15:27:43>

\section{高祖李渊\tiny(618-626)}

\subsection{生平}

唐高祖李渊(566年4月7日-635年6月25日),字叔德,陇西成纪人,唐朝开国皇帝及奠基者,618年6月18日-626年9月4日在位共8年 。玄武门之变后不久禅位于唐太宗,称号“太上皇”。

据《旧唐书》高祖以周天和元年生於長安,太宗在貞觀八年三月甲戌(初二)上壽,推其应为天和元年三月初二(566年4月7日)生。据《册府元龟》记载,李渊以北周天和元年十一月丁酉(566年12月21日)生於長安,似有誤。父親李昞,北周安州總管、柱國大將軍,襲封唐國公。李渊七岁,父亲去世,李渊世袭为唐国公。

581年,隋文帝逼迫北周静帝禪让,李渊任千牛備身(皇帝的禁衛武官),因與皇室的姻親關係,589年隨隋文帝滅陳,后累任谯、陇、岐三州刺史,荥阳郡太守。604年,隋文帝駕崩,遷樓煩太守,隋煬帝大業元年(605年)遷殿内少监;大业二年,除郑州刺史;大業九年(613年),遷衛尉少卿。是年隋煬帝征高句麗,李淵在怀远镇(今辽宁朝阳附近)負責督運。楊玄感之亂,煬帝詔李淵為弘化留守,知關右諸軍事。可見李淵與隋朝宗室關係密切,參與了朝廷的衆多大事,他也趁此機會招納人才,引起煬帝猜忌,李淵懼而以酗酒、受賄等行爲“自污”。

大業十一年(615年)李渊任山西河东郡慰抚大使。大業十二年(616年)升为右骁卫将军。大業十三年(617年)正月遷太原郡留守,7月殺郡丞王威、武牙郎將高君雅,打着勤王定乱,迎回隋天子的旗号正式开始於晉陽縣起兵。晋阳起兵即得到李氏宗族及姻親的響應。他一邊招降叛軍、流寇,一邊派親族迅速進兵,並且借助突厥始毕可汗的500骑兵进攻隋大兴城,于12月攻克。

他拥代王杨侑做傀儡皇帝,遥尊隋煬帝為太上皇,受假黃鉞、使持節、大都督內外諸軍事、大丞相,進封唐王,不久進位相國,加九錫。

义宁二年(公元618年6月18日),隋炀帝在四月被叛军所弑后,李渊逼迫隋恭帝禅让称帝,建立唐朝,隋朝灭亡。李渊开始着手消灭其他原隋朝领土上的諸侯、軍閥,開展唐朝統一戰爭,他的儿女李世民、李建成、李元吉和平陽昭公主的征讨下,用了七年时间,先后消灭薛仁果、薛举、李轨、宋金刚、刘武周、王世充、窦建德、萧铣、杜伏威和梁师都等割据势力。

最后一个梁师都是在贞观二年(628年)被平定的,此时他早已经将皇帝位让给儿子李世民了。同时他又利用东突厥和西突厥之间的分裂,维持了北部的边界,这是他有力量能够夺取中原的主要条件(参见唐与突厥的战争)。

在官制上李渊于武德七年(624年)颁布了唐的官僚制度,基本使用了隋的制度。在农业方面他于武德七年(624年)颁布均田制;对稅捐他也做了调节,减轻了受田农民的负担。在法律上他废弃了隋煬帝的许多苛政,颁布了武德律。李渊对唐朝的措施,为唐太宗“贞观之治”打下了非常重要的基础。

高祖在位期間,没有能尽早確立及处理好繼承人問題,雖然他一早立長子李建成為太子,他眼見皇太子李建成与各儿子明争暗斗,他卻一再纵容,图讓众子互相制衡並未加以控制,同时李世民拥护者众多,导致太子李建成、李元吉和李世民之间的矛盾激化。

最终李世民先下手为强发动政變,史稱玄武门之变,李建成、李元吉被李世民所殺,李世民的軍隊控制了長安,声称二人系作乱伏诛,加上群臣的支持和擁戴,李渊被迫将军国大事交由李世民处理,而李建成、李元吉不但被追废为庶人,诸子也遭诛杀,皆除宗籍。

三天后,高祖立已掌握实权的李世民为皇太子,三個月後更将帝位內禪给李世民,自己退位为太上皇,贞观三年,移居弘义宫。

貞觀九年五月初六(635年),李渊去世,享年六十九歲。死后谥号太武皇帝,廟號高祖,葬在献陵。唐高宗上元元年(674年)八月,改上諡號為神堯皇帝。唐玄宗天寶十三載(754年)二月,上尊號神堯大聖大光孝皇帝。

根據《舊唐書》和《新唐書》,兩者都聲稱李淵是受到兒子李世民的唆擺才起兵反隋。根据这两部史书的记载,李世民通过裴寂把李渊引进隋炀帝的晋阳行官,灌醉了李渊,使得李渊酒后与宫女发生关系,迫使李渊起兵。

反隋應該是李淵本人的意思。孟宪实认为无论从政治影响、军事经验、经济实力还是从社会地位来比较,李世民都无法与李渊相提并论。即便是有人愿意结交李世民,也是因为看重了李世民背后的李渊。李世民要结交那些非法的豪杰大侠,没有背后李渊的政治经济资源几乎是不可能的。晋阳起兵的历史真相是,以李渊为首的军事政治集团,看到隋朝大势已去,于是开始谋划夺取最高权力。这个集团的核心人物当然是李渊,作为李渊的次子,李世民不过是李渊手下的一员得力干将而已。因为父子关系,李渊信任李世民,李世民很早就参与了晋阳起兵的谋划,并且承担某些具体任务。但是,只有李渊才是主谋,这个地位任何人都无法取代。”。

李渊的祖父李虎曾為尚书左仆射,封隴西郡公,賜姓大野氏,與宇文泰等共八位柱国大将军并称八柱国。宇文泰的家族建立北周後,李虎已經去世,获封陇西郡公。父親李昞,北周安州總管、柱國大將軍,襲封陇西郡公,于550年加封唐国公,是为唐仁公,追封李虎为唐襄公。李渊七岁丧父,襲封唐国公。

李渊是隋炀帝杨广的姨表兄弟。北周明帝的明敬皇后、李淵生母元貞太后、隋文帝的文獻皇后分别是西魏八大柱国之一独孤信的長女、四女、七女。

唐朝皇室自称出自隴西李氏,為李暠第二子李歆的后裔,多称陇西狄道(陇西郡狄道县)人,亦可称陇西成纪(陇西郡成纪县)人。

由于唐朝皇室以老子后裔自居,崇尚道教,唐初武德九年太史令傅奕上疏抬道抑佛,引发佛道论争。和尚法琳作《破邪论》《辨证论》反对傅奕。法琳反对唐朝皇室为老子李耳后裔之说,亦与隴西李氏无关,而是拓跋氏后裔,法琳因而触怒唐太宗,被流放益州而死。劉盼遂與王桐齡考據認為李淵家族應為拓跋氏後裔。劉盼遂之後雖取消了自己的观点,但其學說仍引發學界討論。

陳寅恪依据唐祖陵在今河北省境,認為李氏出身赵郡李氏,原為漢人寒門。在崛起之後,分別宣稱自己是隴西李氏或拓跋氏的後代,以抬高身份,实际上不是赵郡李氏破落户,就是广阿庶姓李氏的假冒牌。

朱希祖经考据认为李熙与李买得不是同一个人,李熙曾作为强宗子弟镇戍武川,后卒于武川。其子李天锡为避六镇兵乱,携父遺骨南迁于赵郡广阿,因以为家,不久亦卒。其子李虎将父祖合葬,即所谓唐祖陵。李氏并非出身赵郡李氏,而确系为隴西李氏。

李淵善於騎射,與其妻竇皇后的成親曾經為一時佳話,竇氏未嫁之時為貴族,樣貌豔麗,明艷照人,故其父北周大將竇毅不肯輕易許嫁女兒。故而舉辦射箭之賽,比武招親,要求來求親的公子們,在一「雀屏」(繪有孔雀的屏風)上射箭,以射中孔雀為標準,李淵憑藉準確的目力與勁道,於數步外射箭,竟然成功射中「孔雀的眼睛」,而娶得竇氏,這段佳話流傳後世成為成語“雀屏中選”。今陕西省西安市碑林区,一街道名为“窦府巷”,以窦姓府第在此而得名,亦有传说此“窦府”即窦毅之府。


\subsection{武德}

\begin{longtable}{|>{\centering\scriptsize}m{2em}|>{\centering\scriptsize}m{1.3em}|>{\centering}m{8.8em}|}
  % \caption{秦王政}\
  \toprule
  \SimHei \normalsize 年数 & \SimHei \scriptsize 公元 & \SimHei 大事件 \tabularnewline
  % \midrule
  \endfirsthead
  \toprule
  \SimHei \normalsize 年数 & \SimHei \scriptsize 公元 & \SimHei 大事件 \tabularnewline
  \midrule
  \endhead
  \midrule
  元年 & 618 & \tabularnewline\hline
  二年 & 619 & \tabularnewline\hline
  三年 & 620 & \tabularnewline\hline
  四年 & 621 & \tabularnewline\hline
  五年 & 622 & \tabularnewline\hline
  六年 & 623 & \tabularnewline\hline
  七年 & 624 & \tabularnewline\hline
  八年 & 625 & \tabularnewline\hline
  九年 & 626 & \tabularnewline
  \bottomrule
\end{longtable}


%%% Local Variables:
%%% mode: latex
%%% TeX-engine: xetex
%%% TeX-master: "../Main"
%%% End:
