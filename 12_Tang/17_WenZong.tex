%% -*- coding: utf-8 -*-
%% Time-stamp: <Chen Wang: 2019-12-24 15:34:08>

\section{文宗\tiny(826-840)}

\subsection{生平}

唐文宗李昂(809年11月20日-840年2月10日),原名涵,唐穆宗第二子,母侍女萧氏。唐敬宗之弟。他是唐朝第17代皇帝(除去武则天以外),827年—840年在位,在位13年,享年30岁。

元和四年十月十日生,以其日為慶成節。長慶元年,封為江王。宝历二年(826年),唐敬宗荒淫無道,虐待宦官劉克明等人,劉克明於是刺殺敬宗,立絳王李悟為帝。宰相裴度、執金吾梁守謙、樞密使王守澄(宦官)等率神策軍攻入朝廷,殺李悟,改立江王李涵,李涵改名为李昂,是為唐文宗。

文宗在位期间勵精圖治,資遣宫女三千人,罷免官员一千二百餘人。朝臣朋党相互倾轧,官员调动频繁,牛李党争达到高潮。后起用李训、郑注等人,意欲铲除宦官。

太和五年(831年),唐文宗与宰相宋申锡暗中密谋除掉宦官,但是被宦官王守澄及其门客探听出来,诬告宋申锡谋立漳王李凑。唐文宗中计,宋申锡被赐死。太和九年(835年),文宗终于杀死王守澄。王守澄死后仅一个月,李训引诱仇士良等宦官往左金吾卫衙中取石榴树上的“甘露”,企图将其一举消灭,但事情败露,导致仇士良等宦官大肆屠杀朝臣一千餘人,史称“甘露之变”。事后,文宗更被宦官钳制,對當值學士周墀慨叹自己受制于家奴,境遇不如周赧王、汉献帝,不禁淒然淚下。周墀聽了也伏地流涕。

唐文宗时期,藩镇叛乱依旧频繁。

开成五年(840年)文宗抑郁成病,正月初四病死在大明宮中的太和殿,葬於章陵,死后谥号为元圣昭献孝皇帝。

太子李永死后,文宗曾立敬宗幼子陈王李成美为太子,但未行册礼就病重了,临终时托孤于宰相杨嗣复、李珏,但当权宦官仇士良、鱼弘志因太子不是自己力主所立,矫诏仍废太子为陈王,改立文宗弟颍王李瀍为皇太弟,文宗死后,二人说服李瀍逼令李成美自杀。李瀍继位,就是唐武宗。

唐文宗为庄恪太子李永选妃时,朝廷大臣的女儿们都进入了挑选名单之中,朝廷内外因此动荡不安。唐文宗得知后对宰相郑覃说:“我希望为太子求娶你们荥阳郑氏有礼数的女子为妻,听说在外的大臣们都不愿与我做亲戚,这是為甚麼呢?我家也是幾百年的世家大族,怎麼把神堯皇帝的後人當作佛家羅漢(不願締結親事)呢?”唐文宗于是放弃了选太子妃的计划。不久郑覃把孙女嫁给了一位「九品芝麻官」崔皋,唐文宗无可奈何地说:“民间缔结婚姻,不计较官品却崇尚门第。我家已做了二百年的天子,还比不上崔氏和卢氏吗?”

陈寅恪认为李唐数百年的天子门户还比不上山东旧族九品卫佐的崔皋,可以想见唐朝山东世族心目中两者社会价值的差距,李唐皇室出自关陇胡汉集团,与山东士族以礼法为门风的家法大有不同,李唐汉化程度较深后,与旧有的士族相比自觉相形见绌,越发仰慕,贵为天子也不能胜过山东世族九品卫佐的崔皋,说明山东旧族的自我高标准并非没有原因。


\subsection{大和}

\begin{longtable}{|>{\centering\scriptsize}m{2em}|>{\centering\scriptsize}m{1.3em}|>{\centering}m{8.8em}|}
  % \caption{秦王政}\
  \toprule
  \SimHei \normalsize 年数 & \SimHei \scriptsize 公元 & \SimHei 大事件 \tabularnewline
  % \midrule
  \endfirsthead
  \toprule
  \SimHei \normalsize 年数 & \SimHei \scriptsize 公元 & \SimHei 大事件 \tabularnewline
  \midrule
  \endhead
  \midrule
  元年 & 827 & \tabularnewline\hline
  二年 & 828 & \tabularnewline\hline
  三年 & 829 & \tabularnewline\hline
  四年 & 830 & \tabularnewline\hline
  五年 & 831 & \tabularnewline\hline
  六年 & 832 & \tabularnewline\hline
  七年 & 833 & \tabularnewline\hline
  八年 & 834 & \tabularnewline\hline
  九年 & 835 & \tabularnewline
  \bottomrule
\end{longtable}

\subsection{开成}

\begin{longtable}{|>{\centering\scriptsize}m{2em}|>{\centering\scriptsize}m{1.3em}|>{\centering}m{8.8em}|}
  % \caption{秦王政}\
  \toprule
  \SimHei \normalsize 年数 & \SimHei \scriptsize 公元 & \SimHei 大事件 \tabularnewline
  % \midrule
  \endfirsthead
  \toprule
  \SimHei \normalsize 年数 & \SimHei \scriptsize 公元 & \SimHei 大事件 \tabularnewline
  \midrule
  \endhead
  \midrule
  元年 & 836 & \tabularnewline\hline
  二年 & 837 & \tabularnewline\hline
  三年 & 838 & \tabularnewline\hline
  四年 & 839 & \tabularnewline\hline
  五年 & 840 & \tabularnewline
  \bottomrule
\end{longtable}


%%% Local Variables:
%%% mode: latex
%%% TeX-engine: xetex
%%% TeX-master: "../Main"
%%% End:
