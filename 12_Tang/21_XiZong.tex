%% -*- coding: utf-8 -*-
%% Time-stamp: <Chen Wang: 2021-10-29 16:48:10>

\section{僖宗李儇\tiny(873-888)}

\subsection{生平}

唐僖宗李儇(「儇」,拼音:xuān,注音:ㄒㄩㄢ,粤拼:hyun1;862年6月8日-888年4月20日),唐朝第21代皇帝(除去武则天以外)。唐懿宗第五子,初名俨。873年-888年在位,在位15年,得年27岁,死后谥号为惠圣恭定孝皇帝。唐朝皇帝中出逃时间最长的一位,在位年间有8年不在京师长安。

咸通三年(862年)五月八日,生於东内(大明宫),母王贵妃。初名李俨,六年(865年)七月封普王。十一年(870年)遥领魏博节度使。

在咸通十四年(873年)由宦官拥立,时年十二岁。僖宗一共有五個年號:乾符(6年)、廣明(1年)、中和(4年)、光啟(3年)、文德(1年)。在位期間政事全交给宦官田令孜掌握,自己却肆無忌憚地遊樂,喜歡鬥雞、賭鵝、騎射、劍槊、法算、音樂、圍棋。他對打馬球十分迷戀,對身邊的優伶石野豬說:“朕若參加擊球進士科考試,應該中個狀元。”当时灾害连年,人民生活困苦,官员盘剥沉重。

乾符元年(874年),濮州王仙芝发动起兵。次年,黄巢也起兵于冤句(今山東曹縣西北),隨後兩軍會合,「黃巢之亂」爆发。王仙芝失败后,叛军由黄巢率领,南下進攻浙東,開山路700里入福建,克廣州,回師北上,克潭州,下江陵,直進中原。

广明元年(880年)十一月,黃巢軍攻克洛陽。十二月,下潼關,占领长安,宰相盧攜自殺,田令孜率五百神策軍帶僖宗自長安西門的金光門逃亡入四川,召沙陀族人李克用入援。李克用擊敗黃巢軍於田陂,黃巢退出關中。中和二年(882年)朱溫降唐,賜名朱全忠。

中和四年(884年),黃巢在山東泰安的虎狼谷中自殺(一說為部下林言所殺)。次年三月,唐僖宗返回长安,唐朝已接近灭亡的尾声。此時地方軍閥割據,秦彥據宣、歙,劉漢宏據浙東,朱全忠據汴、滑,李克用據太原、上黨,李昌符據鳳翔,諸葛爽據河陽、洛陽,秦宗權據許、蔡,王敬武據淄、青,高駢據淮南八州,各擅兵賦,迭相吞噬,唐朝中央政府無法節制,能夠控制的地區不過河西、山南、劍南、嶺南西道數十州。

中和五年(885年)三月,田令孜與河中節度使王重榮交惡,王重榮求救於太原李克用,大敗和田令孜结盟的静难节度使朱玫和李昌符,進逼長安。田令孜再領僖宗於光啟元年十二月逃亡到鳳翔(今陝西寶雞),這時諸道兵馬進入長安,燒殺搶掠,宮室坊裏被縱火燒焚者大半,“宮闕蕭條,鞠為茂草”。朱玫立襄王李熅為帝,改元“建貞”。僖宗以正統為號召,把王重榮和李克用爭取過來反攻朱玫,密詔朱玫的愛將王行瑜攻朱,王行瑜將朱玫及其黨羽數百人斬殺,縱兵大掠,時值寒冬,凍死的百姓橫屍蔽地。王重榮殺死襄王煴,田令孜被貶斥。光啟三年(887年)三月,僖宗到達鳳翔,節度使李昌符強留車隊,六月,李昌符進攻僖宗行宮,兵敗出逃隴州,扈駕都將李茂貞追擊,李昌符被斬。

光啟四年(888年)二月,僖宗又回到長安,舉行大赦,改元“文德”。文德元年(888年)三月六日,去世,葬於靖陵(位於今陝西乾縣)。

\subsection{乾符}

\begin{longtable}{|>{\centering\scriptsize}m{2em}|>{\centering\scriptsize}m{1.3em}|>{\centering}m{8.8em}|}
  % \caption{秦王政}\
  \toprule
  \SimHei \normalsize 年数 & \SimHei \scriptsize 公元 & \SimHei 大事件 \tabularnewline
  % \midrule
  \endfirsthead
  \toprule
  \SimHei \normalsize 年数 & \SimHei \scriptsize 公元 & \SimHei 大事件 \tabularnewline
  \midrule
  \endhead
  \midrule
  元年 & 874 & \tabularnewline\hline
  二年 & 875 & \tabularnewline\hline
  三年 & 876 & \tabularnewline\hline
  四年 & 877 & \tabularnewline\hline
  五年 & 878 & \tabularnewline\hline
  六年 & 879 & \tabularnewline
  \bottomrule
\end{longtable}

\subsection{广明}

\begin{longtable}{|>{\centering\scriptsize}m{2em}|>{\centering\scriptsize}m{1.3em}|>{\centering}m{8.8em}|}
  % \caption{秦王政}\
  \toprule
  \SimHei \normalsize 年数 & \SimHei \scriptsize 公元 & \SimHei 大事件 \tabularnewline
  % \midrule
  \endfirsthead
  \toprule
  \SimHei \normalsize 年数 & \SimHei \scriptsize 公元 & \SimHei 大事件 \tabularnewline
  \midrule
  \endhead
  \midrule
  元年 & 880 & \tabularnewline\hline
  二年 & 881 & \tabularnewline
  \bottomrule
\end{longtable}

\subsection{中和}

\begin{longtable}{|>{\centering\scriptsize}m{2em}|>{\centering\scriptsize}m{1.3em}|>{\centering}m{8.8em}|}
  % \caption{秦王政}\
  \toprule
  \SimHei \normalsize 年数 & \SimHei \scriptsize 公元 & \SimHei 大事件 \tabularnewline
  % \midrule
  \endfirsthead
  \toprule
  \SimHei \normalsize 年数 & \SimHei \scriptsize 公元 & \SimHei 大事件 \tabularnewline
  \midrule
  \endhead
  \midrule
  元年 & 881 & \tabularnewline\hline
  二年 & 882 & \tabularnewline\hline
  三年 & 883 & \tabularnewline\hline
  四年 & 884 & \tabularnewline\hline
  五年 & 885 & \tabularnewline
  \bottomrule
\end{longtable}

\subsection{光启}

\begin{longtable}{|>{\centering\scriptsize}m{2em}|>{\centering\scriptsize}m{1.3em}|>{\centering}m{8.8em}|}
  % \caption{秦王政}\
  \toprule
  \SimHei \normalsize 年数 & \SimHei \scriptsize 公元 & \SimHei 大事件 \tabularnewline
  % \midrule
  \endfirsthead
  \toprule
  \SimHei \normalsize 年数 & \SimHei \scriptsize 公元 & \SimHei 大事件 \tabularnewline
  \midrule
  \endhead
  \midrule
  元年 & 885 & \tabularnewline\hline
  二年 & 886 & \tabularnewline\hline
  三年 & 887 & \tabularnewline\hline
  四年 & 888 & \tabularnewline
  \bottomrule
\end{longtable}

\subsection{文德}

\begin{longtable}{|>{\centering\scriptsize}m{2em}|>{\centering\scriptsize}m{1.3em}|>{\centering}m{8.8em}|}
  % \caption{秦王政}\
  \toprule
  \SimHei \normalsize 年数 & \SimHei \scriptsize 公元 & \SimHei 大事件 \tabularnewline
  % \midrule
  \endfirsthead
  \toprule
  \SimHei \normalsize 年数 & \SimHei \scriptsize 公元 & \SimHei 大事件 \tabularnewline
  \midrule
  \endhead
  \midrule
  元年 & 888 & \tabularnewline
  \bottomrule
\end{longtable}


%%% Local Variables:
%%% mode: latex
%%% TeX-engine: xetex
%%% TeX-master: "../Main"
%%% End:
