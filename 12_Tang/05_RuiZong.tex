%% -*- coding: utf-8 -*-
%% Time-stamp: <Chen Wang: 2019-12-24 11:49:23>

\section{睿宗\tiny(684-690)}

\subsection{生平}

唐睿宗李旦(662年6月22日-716年7月13日),唐高宗李治和武則天之子,唐朝的第五和第八任皇帝,曾用名李旭轮、李轮,他一生兩繼大統,兩度禪位。两次登基,第一次為天后武氏(登基前的武則天)廢唐中宗李顯而繼位,在位时间是文明元年至載初二年(684年2月27日-690年10月16日),後上表自行退位,禪讓予母親武則天;第二次是在唐隆之變誅除韋皇后及其黨羽後復辟,在位时间是景雲元年至延和元年(710年7月25日-712年9月8日),後退位,內禪予其子李隆基(唐玄宗)。李旦為唐高宗李治諸子之中排行第八,母为武则天,李弘、李贤、唐中宗李顯都是其同母兄长,太平公主則為其同母妹妹。

龍朔二年(662年)六月一日(6月22日)生於長安蓬萊宮含涼殿。当年,封殷王,遥领冀州大都督、单于大都护、右金吾卫大将军。睿宗“謙恭孝友,好學,工草隸,尤愛文字訓詁之書”。乾封元年(666年),徙封豫王。总章二年(669年),徙封冀王。上元二年(675年),徙封相王,拜右卫大将军。仪凤元年(676年),十四岁的李轮纳豆卢氏为孺人(妾室)。儀鳳三年(678年),迁洛牧;改名李旦,徙封豫王。在七月的一次宴会上,父亲唐高宗对叔祖父霍王李元轨说:因为他是最小的儿子,所以最为喜爱。679年,王妃刘氏生下他的长子李成器。

嗣聖元年(684年)二月七日,武则天废其兄中宗帝位,立他为帝,改元文明。不过,由於是武则天執政,“政事決于太后”,睿宗毫无实权,甚至連干预国家大政的权力都沒有,淪為傀儡。載初元年(690年),武则天废除睿宗後自登帝位,改國號周,睿宗被貶为「皇嗣」(候補性質的皇位繼承人。具儀一比皇太子,卻不給皇太子的名分),改名武轮,迁居东宫。

武则天聖曆元年(698年),武則天又改立中宗為儲君。睿宗則從「皇嗣」再被貶為親王,封號相王,他的五個兒子(李成器、李成義、李隆基、李隆範、李隆業)被封為郡王,唐睿宗從此重獲自由,擁有干预国家大政的权力。

神龍元年(705年),宰相張柬之等五人發動神龍革命,殺张易之、张昌宗兄弟,逼武則天退位,迎中宗復辟,不久武则天去世。此後中宗封其為安國相王,隨即辭去。景雲元年(710年),中宗駕崩(傳說是被韋皇后毒杀),(後在韋皇后矯詔下)由中宗幼子李重茂登位,改元唐隆,是為少帝。

睿宗的三子李隆基與太平公主等聯絡禁軍將領,擁兵入宮,將韦后誅殺,迫少帝李重茂遜位,史曰唐隆之變。六月二十四日睿宗復辟於承天門樓,大赦天下,与其子李隆基一起铲除了韋皇后一黨的势力。

延和元年(712年)七月二十五日,唐睿宗無法面對李隆基與太平公主的爭端,於是禪讓帝位於李隆基,是為唐玄宗,自称「太上皇」,每五天在太極殿接受群臣的朝賀,仍自称“朕”,三品已上除授及大刑狱仍然自决,命令称诰、令,而玄宗每日受朝于武德殿,自称“予”,决定三品已下除授及徒罪,命令称制、敕。后来玄宗发动先天政变消灭太平公主一党,睿宗才不得不彻底放权。

開元四年六月二十日(716年7月13日)李旦病逝,享年五十五歲。

\subsection{文明}

\begin{longtable}{|>{\centering\scriptsize}m{2em}|>{\centering\scriptsize}m{1.3em}|>{\centering}m{8.8em}|}
  % \caption{秦王政}\
  \toprule
  \SimHei \normalsize 年数 & \SimHei \scriptsize 公元 & \SimHei 大事件 \tabularnewline
  % \midrule
  \endfirsthead
  \toprule
  \SimHei \normalsize 年数 & \SimHei \scriptsize 公元 & \SimHei 大事件 \tabularnewline
  \midrule
  \endhead
  \midrule
  元年 & 684 & \tabularnewline
  \bottomrule
\end{longtable}

\subsection{光宅}

\begin{longtable}{|>{\centering\scriptsize}m{2em}|>{\centering\scriptsize}m{1.3em}|>{\centering}m{8.8em}|}
  % \caption{秦王政}\
  \toprule
  \SimHei \normalsize 年数 & \SimHei \scriptsize 公元 & \SimHei 大事件 \tabularnewline
  % \midrule
  \endfirsthead
  \toprule
  \SimHei \normalsize 年数 & \SimHei \scriptsize 公元 & \SimHei 大事件 \tabularnewline
  \midrule
  \endhead
  \midrule
  元年 & 684 & \tabularnewline
  \bottomrule
\end{longtable}

\subsection{垂拱}

\begin{longtable}{|>{\centering\scriptsize}m{2em}|>{\centering\scriptsize}m{1.3em}|>{\centering}m{8.8em}|}
  % \caption{秦王政}\
  \toprule
  \SimHei \normalsize 年数 & \SimHei \scriptsize 公元 & \SimHei 大事件 \tabularnewline
  % \midrule
  \endfirsthead
  \toprule
  \SimHei \normalsize 年数 & \SimHei \scriptsize 公元 & \SimHei 大事件 \tabularnewline
  \midrule
  \endhead
  \midrule
  元年 & 685 & \tabularnewline\hline
  二年 & 686 & \tabularnewline\hline
  三年 & 687 & \tabularnewline\hline
  四年 & 688 & \tabularnewline
  \bottomrule
\end{longtable}

\subsection{永昌}

\begin{longtable}{|>{\centering\scriptsize}m{2em}|>{\centering\scriptsize}m{1.3em}|>{\centering}m{8.8em}|}
  % \caption{秦王政}\
  \toprule
  \SimHei \normalsize 年数 & \SimHei \scriptsize 公元 & \SimHei 大事件 \tabularnewline
  % \midrule
  \endfirsthead
  \toprule
  \SimHei \normalsize 年数 & \SimHei \scriptsize 公元 & \SimHei 大事件 \tabularnewline
  \midrule
  \endhead
  \midrule
  元年 & 689 & \tabularnewline
  \bottomrule
\end{longtable}

\subsection{载初}

\begin{longtable}{|>{\centering\scriptsize}m{2em}|>{\centering\scriptsize}m{1.3em}|>{\centering}m{8.8em}|}
  % \caption{秦王政}\
  \toprule
  \SimHei \normalsize 年数 & \SimHei \scriptsize 公元 & \SimHei 大事件 \tabularnewline
  % \midrule
  \endfirsthead
  \toprule
  \SimHei \normalsize 年数 & \SimHei \scriptsize 公元 & \SimHei 大事件 \tabularnewline
  \midrule
  \endhead
  \midrule
  元年 & 689 & \tabularnewline\hline
  二年 & 690 & \tabularnewline
  \bottomrule
\end{longtable}



%%% Local Variables:
%%% mode: latex
%%% TeX-engine: xetex
%%% TeX-master: "../Main"
%%% End:
