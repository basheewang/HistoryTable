%% -*- coding: utf-8 -*-
%% Time-stamp: <Chen Wang: 2019-12-24 15:21:52>

\section{肃宗\tiny(756-762)}

\subsection{生平}

唐肃宗李亨(711年1月21日-762年5月16日),為唐玄宗李隆基第三子,母親為追尊元獻皇后楊貴嬪,唐朝第10代皇帝(不計武则天),756年8月12日—762年5月16日在位,在位6年。在位期間,是唐朝安史之亂時期。

唐睿宗景云二年(711年)九月三日(711年10月)出生在东宫之别殿。初名李嗣升,始封陝王,母親楊良媛懷孕時,父親唐玄宗身為太子,據說太平公主不滿,故玄宗一度想要楊氏墮胎,但玄宗炮製墮胎藥時睡著,天神託夢勸阻,幕僚張說得知,於是請求玄宗收回成命,後來玄宗登基,終於發動先天之變,賜死太平公主。

开元十三年(725年),一日早朝时,玄宗见李亨早衰,就在罢朝后驾临李亨府,见府中庭宇无人打扫,也无宫女使唤,就令高力士去京兆尹官衙,亟选民间女子入王府。高力士认为现在选秀過甚导致民间喧嚣,御史弹劾,不如在掖庭选取。如此选出三人,章敬皇后吴氏是其中一人。开元十四年(726年),吴氏生下他的长子李俶(即唐代宗)。开元十五年(727年),後徙封忠王,初改名為浚,後改名為璵。

开元二十六年(738年)皇太子李瑛為武惠妃所讒,貶庶人廢死;其被立為太子。他的妾室孺人韦氏被立为太子妃。当初唐玄宗欲立太子时,李林甫曾推举寿王李瑁。李亨成为太子后,李林甫惧怕,阴谋推翻太子。李林甫为构陷太子,诬陷太子妃兄韋堅,使得韦氏被迫与李亨离异,出家为尼。另一位妾室杜良娣的父亲杜有邻被赐死,杜良娣亦被废为庶人。

天寶三载(744年)改名為亨。安史之亂爆發,天寶十五载(756年)六月,鎮守潼關之大將哥舒翰受杨国忠逼迫出兵討叛,結果大敗,潼关陷落,长安震動,玄宗攜太子、寵妃倉皇逃往成都,行經馬嵬驛(今陕西省兴平市西),軍士譁變殺楊國忠,並逼迫玄宗賜死杨贵妃。馬嵬民眾攔阻玄宗請留,玄宗不從。太子李亨留下,隨即往朔方節度使所在地靈武(今寧夏靈武西南),同年农历七月十二日即位,尊玄宗為太上皇,改元至德,時年四十六歲,是為肅宗。

肅宗將郭子仪和李光弼部從河北召至靈武,並聯合回紇,開展大規模的反攻,並約定「克城之日,土地、士庶歸唐,金帛、子女皆歸回紇。」(縱容強姦搶奪)。至德二载(757年)正月,安祿山被其子安慶緒殺死。九月,郭子儀率唐軍和回紇騎兵收復長安,十二月太上皇玄宗回到长安。乾元元年(758年)九月,肅宗調動各路大軍進攻圍攻相州安慶緒,命宦官魚朝恩為觀軍容宣慰處置使,總攬全局。

玄宗回到长安后,厭惡張皇后與李輔國,常勸肅宗不要寵信他們。李輔國趁機構諂,說玄宗預謀復辟,故軟禁玄宗於西宮甘露殿,並流放高力士到巫州。上元元年(760年),玄宗被迫遷居西內太極宮,並於寶應元年農曆四月初五日(762年5月3日)病逝於西內的甘露殿。不久肅宗也一病不起,頒布詔令讓太子李豫監國。

唐肅宗在位僅六年,死於寶應元年四月十八日(762年5月16日),享年五十一歲。廟號肅宗,謚號文明武德大聖大宣孝皇帝,安葬於唐建陵(今陝西省咸陽市禮泉縣西北15公里處)。

安史之亂事直至唐代宗時,方完全平定,歷八年。

\subsection{至德}

\begin{longtable}{|>{\centering\scriptsize}m{2em}|>{\centering\scriptsize}m{1.3em}|>{\centering}m{8.8em}|}
  % \caption{秦王政}\
  \toprule
  \SimHei \normalsize 年数 & \SimHei \scriptsize 公元 & \SimHei 大事件 \tabularnewline
  % \midrule
  \endfirsthead
  \toprule
  \SimHei \normalsize 年数 & \SimHei \scriptsize 公元 & \SimHei 大事件 \tabularnewline
  \midrule
  \endhead
  \midrule
  元年 & 756 & \tabularnewline\hline
  二年 & 757 & \tabularnewline\hline
  三年 & 758 & \tabularnewline
  \bottomrule
\end{longtable}

\subsection{乾元}

\begin{longtable}{|>{\centering\scriptsize}m{2em}|>{\centering\scriptsize}m{1.3em}|>{\centering}m{8.8em}|}
  % \caption{秦王政}\
  \toprule
  \SimHei \normalsize 年数 & \SimHei \scriptsize 公元 & \SimHei 大事件 \tabularnewline
  % \midrule
  \endfirsthead
  \toprule
  \SimHei \normalsize 年数 & \SimHei \scriptsize 公元 & \SimHei 大事件 \tabularnewline
  \midrule
  \endhead
  \midrule
  元年 & 758 & \tabularnewline\hline
  二年 & 759 & \tabularnewline\hline
  三年 & 760 & \tabularnewline
  \bottomrule
\end{longtable}

\subsection{上元}

\begin{longtable}{|>{\centering\scriptsize}m{2em}|>{\centering\scriptsize}m{1.3em}|>{\centering}m{8.8em}|}
  % \caption{秦王政}\
  \toprule
  \SimHei \normalsize 年数 & \SimHei \scriptsize 公元 & \SimHei 大事件 \tabularnewline
  % \midrule
  \endfirsthead
  \toprule
  \SimHei \normalsize 年数 & \SimHei \scriptsize 公元 & \SimHei 大事件 \tabularnewline
  \midrule
  \endhead
  \midrule
  元年 & 760 & \tabularnewline\hline
  二年 & 761 & \tabularnewline
  \bottomrule
\end{longtable}

\subsection{宝应}

\begin{longtable}{|>{\centering\scriptsize}m{2em}|>{\centering\scriptsize}m{1.3em}|>{\centering}m{8.8em}|}
  % \caption{秦王政}\
  \toprule
  \SimHei \normalsize 年数 & \SimHei \scriptsize 公元 & \SimHei 大事件 \tabularnewline
  % \midrule
  \endfirsthead
  \toprule
  \SimHei \normalsize 年数 & \SimHei \scriptsize 公元 & \SimHei 大事件 \tabularnewline
  \midrule
  \endhead
  \midrule
  元年 & 762 & \tabularnewline\hline
  二年 & 763 & \tabularnewline
  \bottomrule
\end{longtable}


%%% Local Variables:
%%% mode: latex
%%% TeX-engine: xetex
%%% TeX-master: "../Main"
%%% End:
