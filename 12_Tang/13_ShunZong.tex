%% -*- coding: utf-8 -*-
%% Time-stamp: <Chen Wang: 2021-10-29 16:30:33>

\section{顺宗李诵\tiny(805)}

唐顺宗李诵(761年2月21日-806年2月11日),唐德宗长子,唐朝第13代皇帝(除武则天以外),805年在位。

唐顺宗也是唐朝唯一一位以先帝嫡长子身份成为皇太子并最终继位的皇帝

上元二年(761年)陰曆正月十二日(阳历761年2月21日),生於長安之東內。初封宣城郡王。大历十四年(779年)唐代宗崩,唐德宗继位,六月,李诵进封宣王,十二月乙卯,立为皇太子。贞元二十一年(805年)正月即位,改元永贞。任用東宮舊人王伾、王叔文为翰林学士,在宰相韋執誼、韩泰、韩晔、柳宗元、刘禹锡、陈谏、凌准、程异等人支持下,从事改革德宗以来的弊政,贬斥贪官,废除宫市,停止盐铁进钱和地方进奉,并试图收回宦官兵权,史称“永贞革新”。

顺宗即位時已患中风,喑啞不能言,詔令皆出牛昭容手。同年八月,宦官俱文珍等勾结部分官僚和藩镇,逼其退位,传位于太子李纯,贬王伾等人,史称“永貞內禪”(顺宗朝僅歷七月,舊制逾年改元,內禪暨革新失敗之後,方以太上皇詔,改元永貞)。又贬斥韩泰等八人,史称“二王八司马”。次年元和元年陰曆正月十九日(806年2月11日),崩於興慶宮之咸寧殿。官方說法為病死。野史認為順宗是被宦官謀殺而死,其事透過唐人傳奇《辛公平上仙》的影射以流傳後世。

死后谥号为至德弘道大圣大安孝皇帝。

\subsection{永贞}

\begin{longtable}{|>{\centering\scriptsize}m{2em}|>{\centering\scriptsize}m{1.3em}|>{\centering}m{8.8em}|}
  % \caption{秦王政}\
  \toprule
  \SimHei \normalsize 年数 & \SimHei \scriptsize 公元 & \SimHei 大事件 \tabularnewline
  % \midrule
  \endfirsthead
  \toprule
  \SimHei \normalsize 年数 & \SimHei \scriptsize 公元 & \SimHei 大事件 \tabularnewline
  \midrule
  \endhead
  \midrule
  元年 & 805 & \tabularnewline
  \bottomrule
\end{longtable}


%%% Local Variables:
%%% mode: latex
%%% TeX-engine: xetex
%%% TeX-master: "../Main"
%%% End:
