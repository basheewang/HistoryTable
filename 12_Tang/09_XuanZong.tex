%% -*- coding: utf-8 -*-
%% Time-stamp: <Chen Wang: 2021-10-29 15:50:47>

\section{玄宗李隆基\tiny(712-756)}

\subsection{生平}

唐玄宗李隆基(公元685年9月8日-762年5月3日),685年出生在神都洛阳,唐朝第九代皇帝(712年-756年在位);統治唐朝長達44年,是唐朝在位最久的皇帝,唐睿宗第三子,母窦德妃。庙号玄宗,谥号至道大圣大明孝皇帝。宋代避聖祖趙玄朗讳,清代避諱康熙帝「玄燁」,皆称为唐明皇,另有尊号“开元圣文神武皇帝”。性格英明果断,多才多艺,知晓音律,擅长书法,儀表雄伟俊丽。

唐隆元年(710年),李隆基与太平公主联手发动唐隆之变,诛韦皇后,擁立睿宗李旦,並掌握朝政與京師實際兵權。公元712年,李旦禅位于李隆基,是為玄宗,隨即發動先天之變,赐死可能爭奪大位的太平公主,取得了国家的最高统治权。玄宗在位44年間,前30年開元之治是唐朝的極盛之世,在位後期,由於其怠政加上政策失誤和重用安祿山等,導致了後來長達8年的安史之亂,逃往四川,為唐朝中衰埋下伏筆。756年李亨自立為帝,即肅宗,尊玄宗为太上皇。玄宗762年病逝,葬于泰陵。

武后垂拱元年秋八月五日戊寅(685年9月8日),李隆基生于东都(今河南省洛阳市),后以其日为千秋节、天长节。出生时其父李旦为帝,母窦氏为德妃。垂拱三年(687年)闰七月丁卯,封楚王。永昌元年(689年),祖母武則天命令李隆基過繼予孝敬帝為子,繼其香火。載初二年(690年),李隆基五岁时,父亲李旦被祖母武氏废除帝位,迁居东宫。

天授三年(692年)十月戊戌,李隆基出阁,开府置官属。李隆基英俊多藝,儀表堂堂,少年時代就顯出了極有膽識的性格。當他七歲時,正是武周时期,武懿宗自認為是武則天的姪子,趾高氣揚,根本不把李氏宗室放在眼裏。有一次,武氏諸王到朝堂參加每月朔望的兩次會見時,他看到李隆基的車騎儀仗威嚴而整齊,心中不悅,便利用自己金吾將軍糾察風紀的權力橫加阻撓。李隆基卻理直氣壯地責問:「我家的朝堂,干你甚麼事?竟敢挾迫我的車騎隨從!」祖母武則天知道此事後,不僅未加罪於他,反而更加宠爱他。虽然李隆基获得了祖母的宠爱,但在长寿二年(693年)正月,其母窦氏与嫡母皇嗣妃刘氏被武则天秘密杀害,尸骨无踪。根据史料可知,李旦的另一位妾室豆卢氏和李隆基的姨妈窦氏抚养、照料过年幼丧母的李隆基。同年腊月丁卯,李隆基由楚王改封临淄郡王。根据《唐会要》的记载,就在长寿二年,李隆基娶王氏为郡王妃。

圣历元年(698年),李隆基再度出阁,赐第于东都洛阳的积善坊。大足元年(701年),随祖母回到西京长安,赐宅于兴庆坊。长安中,历右卫郎将、尚辇奉御。神龙革命后,伯父唐中宗复位。景龙二年四月(708年),李隆基兼潞州別駕。

景龙四年(710年),李隆基從潞州(治所在今山西長治)回到長安。他暗中聚結才勇之士,在皇帝的親軍萬騎中發展勢力。太宗時,選官戶及蕃口中驍勇的武士穿虎紋衣,跨豹紋韉,從遊獵,於馬前射禽獸,謂之百騎。武則天時增加為千騎,中宗時發展為萬騎。李隆基非常重視萬騎的作用。

韋后想效法武則天自稱皇帝,但太平公主與上官婉兒密謀,以中宗遺制,立溫王李重茂(中宗少子)為皇太子,韋后知政事,父亲相王李旦參政。韋黨宗楚客、韋溫、紀處訥等人,極力反對相王參謀政事。相王不想捲入宮廷鬥爭,對事件採取迴避的態度,於是李隆基就主動地策劃了消滅韋黨的宮廷政變。

當時韋后想稱帝登基,對太平公主立李重茂為帝不滿;李隆基又借助太平公主的力量壯大自己。正當雙方劍拔弩張之際,原來親近韋氏的兵部侍郎崔日用改變態度,暗中向李隆基告密,勸其立即發動攻勢。於是,李隆基與太平公主的兒子薛崇簡,苑總監鍾紹京等,密謀策劃,欲先發制人。

有人建議,把發動政變的事先向相王報告,李隆基胸有成竹地說:「我是為了拯救社稷,為君主、父親救急,成功了福祉歸於宗廟與社稷,失敗了我因忠孝而死,不連累相王。怎可以報告,讓相王擔心呢!現在報告,相王若贊成,就是害他參與了危險的起事;若他不贊成,我計謀就失敗了。」於是,決定背著相王,立即行動。

唐隆元年(710年)六月庚子日申時,李隆基等人穿便服進入禁苑,到苑總監鍾紹京住處。這時,鍾紹京反悔,拒絕參加這次政變。但其妻許氏卻堅定地說:「忘身殉國,神必助之。既然參與同謀,即使不參加,勢難免罪。」鍾紹京明白,前往拜謁李隆基。入夜後,萬騎果毅李仙鳧、葛福順都先後來到,請李隆基發佈命令。

二更時分,葛福順拔劍直入羽林營,斬韋黨掌握軍隊的韋璿、韋播、高嵩,然後宣佈:「韋后毒死先帝,謀危社稷,今夕當共誅諸韋,身高有馬鞭之長者皆殺之,立相王為帝以安天下。敢有反對者,罪及三族。」羽林軍將士紛紛表示從命。李隆基率眾出禁苑南門,進攻宮城。葛福順率左萬騎攻玄德門,李仙鳧率右萬騎攻白獸門,相約在凌煙閣會見。李隆基率兵直入玄武門。韋后惶恐逃入飛騎營,被飛騎斬首獻於李隆基;安樂公主正在畫眉,也被斬首,其夫武延秀同時被殺(一作夫妇皆在内堂力战而死)。凡是諸韋及韋后親信均被逮捕斬首(但并未杀绝韦家人),史稱唐隆之變。

這時,李隆基才將唐隆之變的經過報告相王。相王抱著李隆基哭泣著說:「宗廟社稷的災禍是你平定的,神明與百姓也都仰賴你的力量了。」當日,隆基被改封為平王,兼殿中監,同中書門下三品、兼押左右萬騎。

李隆基與姑姑太平公主迫使李重茂禪讓於相王李旦。相王即位,是為睿宗。睿宗與大臣議立太子。按嫡長子繼承制度,兄长宋王李成器應為太子,但李成器堅決辭讓說:「國家安則先嫡長,國家危則先有功;平王有功於國,自己決不居平王之上。」參與消滅韋黨的功臣也多主張立李隆基為太子。睿宗順水推舟,遂在秋七月己巳,册立平王李隆基为皇太子,大赦天下,改元景云。九月庚戌,李隆基长子李嗣直封为许昌郡王,次子李嗣谦为真定郡王。

太平公主恃著擁立睿宗有功,經常干預政事。她又感到太子李隆基精明能幹,妨礙自己參政,總想另易太子。同时,太平公主在后宫中,包括李隆基的身边大量安插耳目。李隆基當然不願任人擺佈,亦想除掉太平公主。睿宗最初遇到困難先聽太平公主的意見,再徵求太子的意見。後來,愈來愈傾向太子。

景雲二年(711年)二月,睿宗命太子監國,六品以下除官及徒罪以下,由太子處分。九月,李隆基的一位妾室——楊良媛生下他的第三子李嗣升,即日后的唐肃宗。楊良媛怀孕时,东宫中依附于太平公主的耳目,“必阴伺察,事虽纤芥,皆闻于上”,李隆基心不自安,甚至因太平公主之故试图为楊良媛堕胎。在先天元年(712年)七月,睿宗禪讓於太子。太平公主雖力勸睿宗不要放棄處理大政的權力,但已無濟於事了。

李隆基於延和元年(712年)八月三日即位,是為唐玄宗,改元先天。當時,宰相多是太平公主之黨,文武大臣,也多依附她。於是,除掉太平公主就成了玄宗的當前要務。而太平公主的黨羽看到玄宗銳意親政,就想廢黜玄宗。

先天元年(713年)七月,玄宗與岐王李範、薛王李業、兵部尚書郭元振、龍武將軍王毛仲等決定起事。玄宗命王毛仲到閒廄取出御馬並調家兵三百餘人,親自率領太僕少卿李令向、王守一,內侍高力士,果毅李守德等親信十多人,先殺左、右羽林大將軍常元楷、李慈,又擒獲了太平公主的親信右散騎常侍賈膺福及中書舍人李猷,接著殺了宰相岑羲、蕭至忠;竇懷貞暫時走脫,最後自縊而死。太平公主驚恐萬狀,先逃入山寺,後被賜死於家,是為先天政變。自此以後,一切軍政大事玄宗完全可以自作主張了。

先天元年十月,玄宗到新豐(今陝西臨潼)閱兵於驪山下,調動二十萬人馬,旌旗連亙五十餘里,聲勢浩大。但由於軍容不整,欲斬兵部尚書郭元振,因宰相劉幽求、中書令張說求情,將其流於新州(今廣東新興)。接著,以制軍禮不肅罪殺了給事中、知禮儀事唐紹。本來,玄宗只是為了整頓軍紀,樹立自己的威信,並無意殺唐紹,但由於金吾將軍李邈倉促宣敕,無可挽回,故而玄宗罷了李邈的官。由於兩位大臣得罪,諸軍震動很大,秩序不穩,只有左軍節度薛訥、朔方道大總管解琬二軍穩定,玄宗讚嘆不已。

先天元年十二月,改元為開元。開元時期的三十年是唐朝的極盛時期。玄宗即位後,勵精圖治,重用姚崇,革新政治。姚崇建議:抑制權貴,重視爵賞,納諫諍,禁貢獻,他都採納。無關大局的具體問題,他都放手讓姚崇處理。有一次,姚崇奏請決定郎吏的任命問題,姚崇再三請求玄宗決定,玄宗只是仰視殿屋,置之不理。高力士提醒玄宗應置可否,他答曰:「朕委姚崇理政,大事應當與朕共議,郎吏小官的事,何須一一煩朕!」自此以後,群臣於是知道玄宗能尊重大臣的決定。

玄宗弟薛王李業母舅王仙童,凌辱百姓,被御史彈奏。薛王李業為其求情,玄宗命中書、門下復查。姚崇等奏曰:「王仙童罪狀明白,御史所言正確,不可縱容。」玄宗同意姚崇的意見。從此,所有貴族都不敢放肆。

為了糾正奢華的風氣,開元二年(714年)七月玄宗下令:「乘輿服御、金銀器玩,宜令有司銷毀,以供軍國之用;其珠玉、錦繡,焚於殿前;后妃以下,皆毋得服珠玉錦繡。」又下欶:「百官所服帶及酒器、馬銜、鐙,三品以上,聽飾以玉,四品以金,五品以銀,自餘皆禁之;婦人服飾從其夫、子。其舊成錦繡,聽染為皂。自今天下更毋得採珠玉,織錦繡等物,違者杖一百,工人減一等。」(《資治通鑑》卷二百二十一開元二年七月條)同時,還罷兩京織錦坊。他還反對厚葬,他認為厚葬無益於死者,有損於生者。於是,要求喪葬務遵簡儉,凡送終物品,均不得以金銀器為飾。如有違者,杖一百。州縣長官不能舉察者,一律貶官。

為了從歷史上總結經驗,汲取教訓,作為治理國家的借鑑,玄宗喜愛閱讀史書,讀到有關政事的問題,他特別留心。但常碰到不能解決的疑難問題,於是,他要宰相為他推薦侍讀,幫助他讀書。開元三年(715年)九月,馬懷素、褚無量被推薦為侍讀。玄宗對侍讀非常尊敬,親自迎送,待以師傅之禮。开元三年(715年)正月,玄宗次子李瑛被立为皇太子。

開元十三年(公元725年),唐朝在伯力(今俄羅斯哈巴羅夫斯克)設置黑水府,置黑水軍,對黑水靺鞨地區實施有效的行政管轄,並勘探了堪察加半島和千島群島。《新唐書·北狄傳》記載:“黑水西北又有思慕部,益北行十日得郡利部,東北行十日得窟說部,亦號屈設,稍東南行十日得莫曳皆部。”。

開元二十三年(735年)四月,玄宗與中書門下及禮官、學士宴於東都集仙殿。他說:「仙者憑虛之論,朕所不取。賢者能治理國家,朕與諸位合宴,宜更名曰:集賢殿。」「仙」、「賢」雖一字之差,卻反映了玄宗重視人才的態度。

隨著時間的流逝,玄宗自認為天下已經太平,逐漸喪失了積極進取的精神,以致生活奢華,減少過問政事。陳建平《中國通史一百講》:“開元二十三年的時候,他覺得國家太平,要表現國家的歡樂盛況,於是大宴五鳳樓,在五鳳樓的殿前,開了一個盛大的同樂會,各種音樂、舞蹈、戲劇,百劇雜陳,讓三百里之內的刺史縣令,都要帶領當地的樂舞伎人,集合到五鳳樓之下來表演,這種歡樂表演,熱鬧喧天,連續了五日之久。”

玄宗因所宠武惠妃谗言,将三个儿子太子李瑛、鄂王李瑶、光王李琚废为庶人并杀害,改立三子忠王李玙为太子;武惠妃不久也於開元二十五年(737年)去世,後宮雖多美人,但沒有一個能使他滿意。开元二十八年(740年)十月,玄宗以为逝世多年的母亲窦氏祈福的名义,敕书儿媳、第十四子寿王妃杨氏出家为女道士,道号“太真”。天寶四年(745年)八月,冊楊氏為貴妃。

楊貴妃不僅個人受寵,其三個姐姐也均賜府邸於京師,寵貴赫然;其遠堂兄楊國忠也因而飛黃騰達。楊貴妃每次乘馬,都有大宦官高力士親自執轡授鞭,貴妃院有織繡工七百人。嶺南經略史張九章、廣陵長史王翼,因所他們獻給楊貴妃的貢品精美,二人均被陞官。於是,官吏競相仿效。楊貴妃喜愛嶺南的荔枝,就有人千方百計急運新鮮荔枝到長安。在男尊女卑的社會裏,民間竟然流行歌謠日:「生男勿喜女勿悲,若今看女作門楣。」可見,玄宗寵愛楊貴妃的社會影響相當深遠。

生活的奢靡,隨之而來的是政治上的腐敗。天寶初年,口蜜腹劍的李林甫被重用為相。李林甫為了掌握大權,反對諫官有益的建議。他訓斥諸諫官道:「今明主在上,群臣將順之不暇,何須多言!」補闕杜璡上書言事,次日即被降為下邽(今陝西渭南東北)令。自此以後,沒有人敢再有諫諍之言了。

在用人方面,李林甫認為凡在德才方面超過自己者,他都設法將其除去。玄宗想重用兵部侍郎盧絢,他就把盧絢調任華州(治所在今陝西華縣)刺史,並欺騙玄宗說盧絢因病不能理事而棄而不用。玄宗又欲重用絳州(治所在今山西新絳)刺史嚴挺之,李林甫又欺騙玄宗說嚴挺之年老多病,宜授其散職,便於他養病。於是,嚴挺之又被送到東京(今河南洛陽)養病去了。李林甫雖然專權亂政,但其在位期間,政局尚穩。

李林甫欺上壓下並未引起玄宗注意,他反而仍然認為天下無事,把主要政事交由李林甫處理。高力士多次勸他不可使大權旁落以免失去君威,他還甚為不悅,致使高力士惶恐自責。天寶十一年(752年)李林甫死後,玄宗一方面重用擅權弄法的楊貴妃堂兄楊國忠為宰相,一方面信任居心叵測的邊將安祿山,以圖左右平衡。

楊國忠的專權亂政比李林甫更甚,重用親信,排斥異己。天寶十二年(753年),關中大饑,因京兆尹李峴不甚順從,遂以災氣歸罪於李峴,貶李峴為長沙(今湖南長沙)太守。後來霖雨成災,玄宗過問災情,楊國忠取最好的禾苗給玄宗看,掩蓋災情真象。扶風太守房琯反映了所管地區的災情,楊國忠就派御史去追究他的責任。因此,天寶十三年(754年)雖然關中災情嚴重,但無人敢如實上報。連玄宗身邊的宦官高力士也說,楊國忠大權在握,賞罰不公,連他也不敢說話了。

范陽(今北京附近)節度使安祿山為了和楊國忠在玄宗面前爭寵,二人互相詆譭。玄宗對此搖擺不定,認為主要政事交付宰相,邊防事務交付諸將,無可憂慮。這樣一來,蓄謀已久的安祿山終於發動了反唐的大叛亂。

唐玄宗雖然沒有發動過像唐太宗、唐高宗朝時那樣的大規模的開邊軍事行動,但是他在位期間中原周邊地區與鄰近少數族吐蕃、契丹、南詔等的戰事連綿不斷。在邊疆軍事勝利的刺激下,玄宗日益滋長了他好大喜功的思想,寵愛有戰功的邊將。邊將也因此不停對外族開戰,以邀功賞。特別是李林甫為遏制政敵而拉邊將牛仙客入相後,更開放了蕃將以邊功為手段,窺伺中央政權的機會。

唐玄宗晚年驕奢淫逸,終日只顧與楊貴妃遊樂。他罢免良相张九龄,任用奸相李林甫,朝政每况愈下。玄宗本不太相信鬼神之說,後來崇信方士張果,漸好神仙;並尊奉道教,企慕長生不老,以是朝野爭言符瑞。李林甫死后,又以楊貴妃之從兄楊國忠担任丞相,李林甫在位時尚可穩住朝政,楊國忠不仅没有李林甫的才幹,反而縱容贪污腐败,局面遂不可收拾。不久,楊國忠与手握兵權的范陽节度使安祿山发生冲突,安祿山决心先發制人,發動叛变。

天寶十四年(755年),安祿山趁唐朝內部空虛腐敗,發動兵變,於時承平日久,民不知戰,河北州縣,望風瓦解。史稱安史之乱。玄宗決定逃往四川,途中至馬嵬驛,士兵譁變,士兵砍殺楊國忠,又逼玄宗賜死楊貴妃,玄宗權衡輕重下後,為了保命及維持君威,不得已下令高力士把楊貴妃勒死。

對玄宗早有不滿的太子李亨與玄宗分道揚鑣;李亨率一部份禁軍北趨靈武(今寧夏靈武西南),七月即位,改元至德,是為唐肅宗。李隆基與陳玄禮率另一部份禁軍南逃成都,後被尊為太上皇,玄宗長達44年的統治告終。

至德二年(757年)十二月,隨著安祿山被殺,郭子仪收复长安,玄宗由成都返回長安,居興慶宮(南內),奉玄宗為太上皇。乾元三年(760年)七月,宦官李輔國奉承肅宗,離間玄宗與肅宗的關係,迫使玄宗軟禁於太極宮(西內)甘露殿。高力士、陈玄礼等人被贬谪,玄宗浸不自怿、憂鬱寡歡。

寶應元年农历四月初五日(762年5月3日),太上皇唐玄宗李隆基崩逝於長安城太極宮甘露殿內,享壽七十六歲,在位四十四年。同年四月十八日(762年5月16日),久病未癒的唐肅宗李亨亦駕崩于长生殿,享年五十一歲,在位僅短短的六年。广德元年(763年)三月,將唐玄宗李隆基安葬於唐泰陵(今陝西省渭南市蒲城縣東北15公里處)。廟號玄宗,諡號至道大聖大明孝皇帝

開元年間,玄宗勵精圖治,任用賢臣,革除弊害,鼓勵生產,經濟發展,史稱「開元之治」。开元十四年(726年)杜甫《憶昔》有詩證:「憶昔開元全盛日,小邑猶藏萬家室。稻米流脂粟米白,公私倉廩俱豐實。九州道路無豺狼,遠行不勞吉日出。齊紈魯縞車班班,男耕女桑不相失。」

雖然玄宗後期怠政,但直到他在位四十三年的天寶十三年(754年),仍是唐代的極盛之世,全國有三百二十一郡,一千五百三十八縣,一萬六千八百二十九鄉,九百零六萬九千一百五十四戶,五千二百八十八萬四百八十八口。史載:“戶口之盛,極於此”。

唐玄宗富有音樂才華,对唐朝音乐发展有重大影响,他愛好親自演奏琵琶、羯鼓,擅長作曲,作有《霓裳羽衣曲》,《小破陣樂》,《春光好》,《秋風高》等百餘首樂曲。他曾选乐工,宫女在禁院梨园中歌舞,这是后来称戏班为“梨园”的由来。


\subsection{先天}

\begin{longtable}{|>{\centering\scriptsize}m{2em}|>{\centering\scriptsize}m{1.3em}|>{\centering}m{8.8em}|}
  % \caption{秦王政}\
  \toprule
  \SimHei \normalsize 年数 & \SimHei \scriptsize 公元 & \SimHei 大事件 \tabularnewline
  % \midrule
  \endfirsthead
  \toprule
  \SimHei \normalsize 年数 & \SimHei \scriptsize 公元 & \SimHei 大事件 \tabularnewline
  \midrule
  \endhead
  \midrule
  元年 & 712 & \tabularnewline\hline
  二年 & 713 & \tabularnewline
  \bottomrule
\end{longtable}

\subsection{开元}

\begin{longtable}{|>{\centering\scriptsize}m{2em}|>{\centering\scriptsize}m{1.3em}|>{\centering}m{8.8em}|}
  % \caption{秦王政}\
  \toprule
  \SimHei \normalsize 年数 & \SimHei \scriptsize 公元 & \SimHei 大事件 \tabularnewline
  % \midrule
  \endfirsthead
  \toprule
  \SimHei \normalsize 年数 & \SimHei \scriptsize 公元 & \SimHei 大事件 \tabularnewline
  \midrule
  \endhead
  \midrule
  元年 & 713 & \tabularnewline\hline
  二年 & 714 & \tabularnewline\hline
  三年 & 715 & \tabularnewline\hline
  四年 & 716 & \tabularnewline\hline
  五年 & 717 & \tabularnewline\hline
  六年 & 718 & \tabularnewline\hline
  七年 & 719 & \tabularnewline\hline
  八年 & 720 & \tabularnewline\hline
  九年 & 721 & \tabularnewline\hline
  十年 & 722 & \tabularnewline\hline
  十一年 & 723 & \tabularnewline\hline
  十二年 & 724 & \tabularnewline\hline
  十三年 & 725 & \tabularnewline\hline
  十四年 & 726 & \tabularnewline\hline
  十五年 & 727 & \tabularnewline\hline
  十六年 & 728 & \tabularnewline\hline
  十七年 & 729 & \tabularnewline\hline
  十八年 & 730 & \tabularnewline\hline
  十九年 & 731 & \tabularnewline\hline
  二十年 & 732 & \tabularnewline\hline
  二一年 & 733 & \tabularnewline\hline
  二二年 & 734 & \tabularnewline\hline
  二三年 & 735 & \tabularnewline\hline
  二四年 & 736 & \tabularnewline\hline
  二五年 & 737 & \tabularnewline\hline
  二六年 & 738 & \tabularnewline\hline
  二七年 & 739 & \tabularnewline\hline
  二八年 & 740 & \tabularnewline\hline
  二九年 & 741 & \tabularnewline
  \bottomrule
\end{longtable}

\subsection{天宝}

\begin{longtable}{|>{\centering\scriptsize}m{2em}|>{\centering\scriptsize}m{1.3em}|>{\centering}m{8.8em}|}
  % \caption{秦王政}\
  \toprule
  \SimHei \normalsize 年数 & \SimHei \scriptsize 公元 & \SimHei 大事件 \tabularnewline
  % \midrule
  \endfirsthead
  \toprule
  \SimHei \normalsize 年数 & \SimHei \scriptsize 公元 & \SimHei 大事件 \tabularnewline
  \midrule
  \endhead
  \midrule
  元年 & 742 & \tabularnewline\hline
  二年 & 743 & \tabularnewline\hline
  三年 & 744 & \tabularnewline\hline
  四年 & 745 & \tabularnewline\hline
  五年 & 746 & \tabularnewline\hline
  六年 & 747 & \tabularnewline\hline
  七年 & 748 & \tabularnewline\hline
  八年 & 749 & \tabularnewline\hline
  九年 & 750 & \tabularnewline\hline
  十年 & 751 & \tabularnewline\hline
  十一年 & 752 & \tabularnewline\hline
  十二年 & 753 & \tabularnewline\hline
  十三年 & 754 & \tabularnewline\hline
  十四年 & 755 & \tabularnewline\hline
  十五年 & 756 & \tabularnewline
  \bottomrule
\end{longtable}


%%% Local Variables:
%%% mode: latex
%%% TeX-engine: xetex
%%% TeX-master: "../Main"
%%% End:
