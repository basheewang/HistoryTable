%% -*- coding: utf-8 -*-
%% Time-stamp: <Chen Wang: 2018-10-30 16:25:56>

\subsection{南北两宋}

\begin{longtable}{|>{\centering\namefont\heiti}m{2em}|>{\centering\tiny}m{3.0em}|>{\xzfont\kaiti}m{7em}|}
  % \caption{秦王政}\
  \toprule
  \SimHei \normalsize 姓名 & \SimHei \normalsize 异名 & \SimHei \normalsize \hspace{2.5em}小传 \tabularnewline
  % \midrule
  \endfirsthead
  \toprule
  \SimHei \normalsize 姓名 & \SimHei \normalsize 异名 & \SimHei \normalsize \hspace{2.5em}小传 \tabularnewline 
  \midrule
  \endhead
  \midrule
  王禹偁 & \begin{description}
  \item[字] 元之
  \item[号] 
  \item[谥] 
  \item[尊] 王黄州
  \item[生] 山东菏泽
  \end{description} & 王禹偁(954年-1001年),字元之,济州钜野(今山东菏泽市钜野县)人,北宋文学家。王禹偁出身清寒,家庭世代务农。从小发愤求学,五岁便能够写诗。宋太宗太平兴国八年(983年)中进士,最初担任成武县主簿。他对仕途充满抱负,曾在《吾志》诗中表白:“吾生非不辰,吾志复不卑,致君望尧舜,学业根孔姬”。端拱元年(988年),他被召见入京,担任右拾遗、直史馆。他旋即进谏,以《端拱箴》来批评皇宫的奢侈生活。后来历任左司谏、知制诰、翰林学士。为人刚直,敢直言进谏,誓言要“兼磨断佞剑,拟树直言旗”。曾三次被贬职:于淳化二年(991年),一贬商州,于至道元年,二贬滁州,于咸平元年(998年),三贬黄州。故有“王黄州”之称。谪居黄州期间,以骈散相间之"黄州新建小竹楼记"抒发虽遭贬谪却心地坦荡,具达观旷逸之胸怀。 \tabularnewline\hline
  寇准 & \begin{description}
  \item[字] 平仲
  \item[号] 
  \item[谥] 忠愍
  \item[尊] 寇莱公
  \item[生] 陕西渭南
  \end{description} & 寇准(961年-1023年10月24日),北宋名相。字平仲,华州下邽(今陕西渭南)人。寇凖系春秋司寇苏氏裔孙,其父寇湘,后晋开运中,很有学问,应辟为魏王府记室参军。寇准出生于山西太谷县,年少时期豪爽嗜酒,性格大方,喜欢在家里大摆筵席。[注 1] 。皇祐四年,诏翰林学士孙抃撰神道碑,帝为篆其首曰“旌忠”。寇凖善诗能文,七绝尤有韵味,今传《寇忠湣诗集》三卷。 \tabularnewline\hline
  丁谓 & \begin{description}
  \item[字] 谓之\\公言
  \item[号] 
  \item[谥] 
  \item[尊] 丁晋公
  \item[生] 江苏苏州
  \end{description} & 丁谓(966年-1037年),字谓之,后更字公言,北宋时期苏州长州(今江苏苏州)人。善言谈,喜欢作诗,于图书、博奕、音律无一不精。出自寇准门下,太宗淳化三年(992年)进士,授大理寺评事、通判饶州事。真宗咸平初除三司户部判官,大中祥符初,权三司使。咸平五年(1012年)任户部侍郎。官至参政知事。当政后极力排斥寇准,乾兴元年(1022年)二月,再贬寇准为雷州司户参军。丁谓同党雷允恭因先帝陵寝工程事故,坐“擅移皇堂”罪,丁谓受牵连,贬为太子太保。后以“丁谓前后欺罔”罪,被贬崖州(今海南省琼山县)司户参军。以秘书省监致仕归里。景祐四年(1037年)病卒。著有《丁谓集》8卷、《虎丘集》50卷、《刀笔集》2卷、《青衿集》3卷、《知命集》1卷,皆佚。所著《天香传》,则是中国最早系统性针对沉香尤其是海南沉香所写的专著,书中记载沉香自古便为人所用,最早用于祭天礼地的场合,焚沉香祝祷。丁谓实地考察海南沉香,提出了四名十二状的分类法,被后世多人藉鉴。 \tabularnewline\hline
  陈尧佐 & \begin{description}
  \item[字] 希元
  \item[号] 
  \item[谥] 文惠
  \item[尊] 
  \item[生] 四川南充
  \end{description} & 陈尧佐(963年—1044年),字希元,阆州阆中新井县(今四川省南充市南部县)人。北宋大臣,官至同中书门下平章事、集贤殿大学士,太子太师致仕,追赠司空兼侍中,谥号“文惠”。 陈尧佐,字希元,号知余子。阆州阆中人。北宋大臣、水利专家、书法家、诗人。左谏议大夫陈省华次子,枢密使陈尧叟之弟。陈尧佐与长兄陈尧叟、弟陈尧咨皆中状元。端拱元年(988年),陈尧佐进士及第,授魏县、中牟县尉。咸平初年,任潮州通判。历官翰林学士、枢密副使、参知政事。宋仁宗时官至宰相,景祐四年(1037年),拜同中书门下平章事。康定元年(1040年),以太子太师致仕。庆历四年(1044年),陈尧佐去世,年八十二,赠司空兼侍中,谥号“文惠”。陈尧佐明吏事,工书法,喜欢写特大的隶书字,著有《潮阳编》、《野庐编》、《遣兴集》、《愚邱集》等。 \tabularnewline\hline
  林逋 & \begin{description}
  \item[字] 君复
  \item[号] 
  \item[谥] 和靖先生
  \item[尊] 
  \item[生] 浙江杭州
  \end{description} & 林逋(967年或968年─1028年),汉族,北宋诗人。字君复,后人称为和靖先生,钱塘人(今浙江杭州)。出生于儒学世家,恬淡好古,早年曾游历于江淮等地,隐居于西湖孤山,终身不仕,未娶妻,与梅花、仙鹤作伴,称为“梅妻鹤子”。宋真宗闻其名,赐粟帛,诏长吏岁时劳问。性孤高自好,喜恬淡,不趋名利,自谓:“然吾志之所适,非室家也,非功名富贵也,只觉青山绿水与我情相宜。”林逋善为诗,其词澄浃峭特,多奇句。其诗大都反映隐居生活,描写梅花尤其入神,苏轼高度赞扬林逋之诗、书及人品,并诗跋其书:“诗如东野不言寒,书似留台差少肉。”宋仁宗天圣六年(1028年)去世,享寿六十二岁,仁宗赐谥“和靖先生”。留有《林和靖诗集》。宋代桑世昌著有《林逋传》。 \tabularnewline\hline
  杨亿 & \begin{description}
  \item[字] 大年
  \item[号] 
  \item[谥] 文
  \item[尊] 杨文公
  \item[生] 福建浦城
  \end{description} & 杨亿(974年-1020年),字大年,人称杨文公。建州浦城(今属福建浦城县)人。北宋文学家。自幼是个神童,博览强记,太宗雍熙元年(984年),十一岁受宋太宗召试,授秘书省正字(掌管图书秘籍的次长),淳化三年(992年)赐进士及第,迁光禄寺丞。淳化四年,直集贤院。至道二年(996年)迁著作佐郎。大中祥符六年(1013年)以太常少卿分司西京。天禧二年(1018年)拜工部侍郎。官至工部侍郎。以“秉清节”自许,“性特刚劲寡合”,为“忠清鲠亮之士”。又好谈禅。又好写诗,善于西昆体,朱熹评之为“巧中犹有混成底意思,便巧得来不觉”,与刘筠、钱惟演等诗歌唱和,其编著《西昆酬唱集》,收录十七位诗人作品,共250首,多言学李商隐,不喜杜工部诗,谓为村夫子。杨亿曾为翰林学士兼史馆修撰。长于典章制度,真宗即位初,曾参预修《太宗实录》,咸平元年(998年)书成,景德二年(1005年)与王钦若主修《册府元龟》。在政治上支持丞相寇准抵抗辽兵入侵。又反对宋真宗大兴土木。卒谥文,故称杨文公。著作多佚,今存《武夷新集》20卷。《宋史》卷三○五有传。清人全祖望有《杨文公论》。 \tabularnewline\hline
  张先 & \begin{description}
  \item[字] 子野
  \item[号] 
  \item[谥] 
  \item[尊] 张三影\\张郎中
  \item[生] 浙江湖州
  \end{description} & 张先(990年-1078年),字子野,湖州乌程(今浙江湖州吴兴)人,因张先曾在安陆郡(今湖北省安陆市)任职多年,人亦称张安陆,为北宋著名婉约派词人。父张维,好读书。张先是天圣八年进士,授汉阳军司理参军,调河南法曹参军,改著作佐郎知阆中县,代还,拜秘书丞,知亳州鹿邑县。与欧阳修友好。官至尚书都官郎中,著有《安陆集》一卷。清代侯文灿、葛鸣阳、鲍廷博等人根据宋人手抄本编纂《张子野词》。宝元二年卒,年八十八。 \tabularnewline\hline
  李冠 & \begin{description}
  \item[字] 世英
  \item[号] 
  \item[谥] 
  \item[尊] 
  \item[生] 山东济南
  \end{description} & 约公元1019年前后在世,字世英,齐州历城(今山东济南)人。生卒年均不详,约宋真宗天禧中前后在世。与王樵、贾同齐名;又与刘潜同时以文学称京东。举进士不第,得同三礼出身,调乾宁主。冠著有《东皋集》二十卷,不传。存词五首。《宋史本传》传于世。 沈谦《填词杂说》赞其《蝶恋花》“数点雨声风约住,朦胧淡月云来去”句,以为“‘红杏枝头春意闹’,‘云破月来花弄影’俱不及”。 \tabularnewline\hline
  晏殊 & \begin{description}
  \item[字] 同叔
  \item[号] 
  \item[谥] 元献
  \item[尊] 晏元献
  \item[生] 江西南昌
  \end{description} & 晏殊(991年-1055年2月27日),字同叔,抚州临川文港乡(今南昌进贤县)人。北宋著名词人晏几道父亲,世称晏殊为大晏,晏几道为小晏。为北宋前期著名婉约派词人,与欧阳修并称“晏欧”。晏殊自幼聪颖,七岁能文,十四岁时因宰相张知白推荐,以神童召试,被朝廷赐同进士出身,之后到秘书省做正字。宋仁宗康定初(1040年),官至同平章事兼枢密使,位同宰相,掌军政大权。仁宗至和二年(1055年)正月二十八日病逝,年六十五,封临淄公,谥号元献,世称晏元献。性刚简,自奉清俭,好燕饮。能荐拔人才,号称贤相,王安石、范仲淹、欧阳修均出其门下。 \tabularnewline\hline
  石延年 & \begin{description}
  \item[字] 曼卿
  \item[号] 
  \item[谥] 
  \item[尊] 
  \item[生] 河南商丘
  \end{description} & 石延年(994年-1041年),字曼卿,宋代的文学家和书法家。先世幽州(今河北省涿县)人,后迁宋州宋城(今河南省商丘市)。不拘礼法,不慕名利。屡试不中,宋真宗时,因为三举进士不中,最后补三班奉职(从九品下,俸钱七百文)。历官知金乡县,累迁大理寺丞。好饮酒,有时披头散发,双手要带着枷锁,称“囚饮”;有时爬到树上去饮,曰“巢饮”;有时用稻麦杆束身,伸出头来与人对饮,称作“鳖饮”;有时和朋友摸黑饮酒,称作“鬼饮”。在海州任通判时,与刘潜曾在王氏酒楼喝酒,从早饮到晚,不发一言,隔日,京城传出昨日王氏酒楼有二神仙来饮酒。和杜默、欧阳修合称“三豪”。 \tabularnewline\hline
  宋祁 & \begin{description}
  \item[字] 子京
  \item[号] 
  \item[谥] 景文
  \item[尊] 红杏尚书
  \item[生] 河南杞县
  \end{description} & 宋祁(998年-1061年),字子京,安陆(今属湖北)人,徙居开封雍丘(今河南杞县),中国北宋文学家、史学家。与其兄宋庠诗文齐名,时呼“小宋”、“大宋”,合称“二宋”。著有《宋景文公集》。宋祁处于北宋阶级矛盾的时期,宝元二年(1038年)时任同判礼院,上疏认为国用不足在于“三冗三费”,三冗即冗官、冗兵、冗僧,三费是道场斋醮、多建寺观、靡费公用。主张裁减官员,节省经费,宰相吕夷简指责他是朋党,并加以打击。 \tabularnewline\hline
  梅尧臣 & \begin{description}
  \item[字] 圣俞
  \item[号] 
  \item[谥] 
  \item[尊] 宛陵先生
  \item[生] 安徽宣城
  \end{description} & 梅尧臣(1002年-1060年),字圣俞,宣城(今安徽宣城)人,世称宛陵先生。北宋著名现实主义诗人。50岁后,始得宋仁宗召试,赐同进士出身,后任授国子监直讲,迁尚书屯田都官员外郎,故时称“梅直讲”、“梅都官”。梅尧臣少即能诗,与苏舜钦齐名,世人美称“苏梅”,同被誉为宋诗“开山祖师”。与欧阳修为挚友,同为宋诗革新推动者。晚年曾参与编撰《新唐书》。嘉祐五年(1060年)京师有大疫,四月以疾卒。 \tabularnewline\hline
  苏舜钦 & \begin{description}
  \item[字] 子美
  \item[号] 
  \item[谥] 
  \item[尊] 苏学士
  \item[生] 河南开封
  \end{description} & 苏舜钦(1009年-1049年),字子美,开封(今属河南)人,曾祖父苏协由梓州铜山(今四川中江)移家开封(今属河南)。父亲苏耆,母亲为王雍、王冲、王素之姐。宋仁宗景祐元年(1035年)进士。历任蒙城、长垣县令,庆历三年因母丧守制,后入大理评事、集贤校理、监进奏院等职。杜衍以女嫁之,进奏院祠神,售废纸公钱宴会。因参加范仲淹为首的革新集团,为人所弹劾,以“监守自盗罪”削职为民,闲居苏州沧浪亭。后再起用为湖州长史,庆历八年(1048年)十二月卒。 \tabularnewline\hline
  邵雍 & \begin{description}
  \item[字] 尧夫
  \item[号] 安乐先生
  \item[谥] 康节
  \item[尊] 百源先生
  \item[生] 河南辉县
  \end{description} & 邵雍(1011年1月21日-1077年7月27日[1]),字尧夫,自号安乐先生,人又称百源先生,谥康节,后世称邵康节,北宋五子之一,易学家、思想家、诗人。雍青年时期即有好学之名,《宋史》记载:“雍少时,自雄其才,慷慨欲树功名。于书无所不读,始为学,即坚苦刻厉,寒不炉,暑不扇,夜不就席者数年。已而叹曰:‘昔人尚友于古,而吾独未及四方。’于是逾河、汾,涉淮、汉,周流齐、鲁、宋、郑之墟,久之,幡然来归,曰:‘道在是矣。’遂不复出。”雍后居洛阳,与司马光、二程、吕公著等交游甚密。邵雍与二程、周敦颐、张载,合称为“北宋五子”。 \tabularnewline\hline
  曾巩 & \begin{description}
  \item[字] 子固
  \item[号] 
  \item[谥] 文定
  \item[尊] 曾南丰
  \item[生] 江西南丰
  \end{description} & 曾巩(1019年9月30日-1083年4月30日),字子固,建昌南丰(今江西南丰)人,汉族江右民系,北宋散文家,被誉为“唐宋八大家”之一。曾巩的文体风格为“古雅平正”,擅长引经据典;结构则平易理醇,章法开阖、承转、起伏、回环都有一定约束法度、严密、规矩。正因为其文章易于模仿和学习,他成为了唐宋文派和桐城派学习的首要对象。 \tabularnewline\hline
  晏几道 & \begin{description}
  \item[字] 叔原
  \item[号] 小山
  \item[谥] 
  \item[尊] 小晏
  \item[生] 南昌进贤
  \end{description} & 晏几道(1037年-1110年),字叔原,号小山,晏殊第七子。北宋婉约派词人。抚州临川文港乡(今属南昌进贤县)人。以父荫赐进士出身,历官开封府判官、颍昌府许田镇监、乾宁军通判等。一般讲到北宋词人时,称晏殊为大晏,称晏几道为小晏。《雪浪斋日记》云:“晏叔原工小词,不愧六朝宫掖体。”《鹧鸪天》中“舞低杨柳楼心月,歌尽桃花扇底风。”两句受人赞赏。晏几道的词经常都是多愁善感。可能与他晚年家道中落有关,他在《小山词自序》中回忆说:“追惟往昔过从饮酒之人,或垅木已长,或病不偶。考其篇中所记悲欢离合之事,如幻,如电,如昨梦前尘,但能掩卷怃然,感光阴之易迁,叹境缘之无实也!”《全宋词》存录有二百六十余首。 \tabularnewline\hline
  晁炯 & \begin{description}
  \item[字] 
  \item[号] 
  \item[谥] 文元
  \item[尊] 
  \item[生] 山东
  \end{description} & 真宗时晁炯声名显赫,此后,“晁氏自迥以来,家传文学,几于人人有集”。 \tabularnewline\hline
  刘攽 & \begin{description}
  \item[字] 贡父
  \item[号] 公非
  \item[谥] 
  \item[尊] 
  \item[生] 樟树市
  \end{description} & 刘攽(1022年-1088年),字贡父(一作戆父,或赣父),号公非。樟树市黄土岗镇荻斜刘家人。北宋史学家,著有《彭城集》。《资治通鉴》副主编之一。先世为彭城人,西晋末年,避胡兵乱,迁居江南,又迁庐陵。刘攽好谐谑,庆历六年(1046年)贾黯榜进士。历任汝州推官,至和二年乙未(1055年)调江阴县主簿,嘉祐二年丁酉(1057年)担任庐州推官等。历州县官二十年,嘉祐八年癸卯(1063年),入京为国子监直讲,迁馆阁校勘。宋神宗熙宁初年同知太常礼院,以反对新法出知曹州。博览群书,精于史学,助司马光修《资治通鉴》,专治汉史部分。元丰八年(1085年),由衡州盐仓起知襄州,元祐初年召拜中书舍人。四年卒,年六十七。 \tabularnewline\hline
  沈括 & \begin{description}
  \item[字] 存中
  \item[号] 梦溪丈人
  \item[谥] 
  \item[尊] 
  \item[生] 浙江杭州
  \end{description} & 沈括(1031年-1095年),字存中,号梦溪丈人,是中国北宋科学家、杭州钱塘县(今浙江省杭州市)人,随母寿昌县太君许氏入籍苏州吴县(今江苏省苏州市)。沈括在物理学、数学、天文学、地学、生物医学等方面都有重要的成就和贡献,在化学、工程技术等方面也有相当的成就。此外,沈括在文学、音乐、艺术、史学等方面都有一定的造诣。《宋史·沈括传》称他“博学善文,于天文、方志、律历、音乐、医药、卜算无所不通,皆有所论著”。沈括突出的成就主要集中在《梦溪笔谈》中。 \tabularnewline\hline
  文同 & \begin{description}
  \item[字] 与可
  \item[号] 笑笑先生
  \item[谥] 
  \item[尊] 石室先生\\文湖州
  \item[生] 四川盐亭
  \end{description} & 文同(1018年-1079年),字与可,自号笑笑先生或笑笑居士,人称石室先生,四川梓州永泰(今四川盐亭县东北面)人。文同历官邛州、洋州等知州,元丰初出知湖州,未到任而死,人称“文湖州”。曾参与校对《新唐书》。善画墨竹,他的表弟苏轼曾称赞他为诗、词、画、草书四绝,苏轼画竹受其影响,学他的人很多,有“湖州竹派”之称。成语“胸有成竹”正是从他画竹而来。 \tabularnewline\hline
  贺铸 & \begin{description}
  \item[字] 方回
  \item[号] 庆湖遗老
  \item[谥] 
  \item[尊] 贺鬼头\\贺梅子
  \item[生] 浙江绍兴
  \end{description} & 贺铸(1052年-1125年),字方回,号庆湖遗老。越州山阴(今浙江绍兴)人,生长于卫州(治今河南卫辉)。北宋词人。著有《东山词》2卷,《东山词补》1卷,今存词200余首。其词风格多样,字句锤炼,常借用古乐府、及唐人诗句入词,作品多写艳情及闺情离思,也描写世间沧桑,嗟叹功名不就,亦有个人闲愁、纵酒狂放之作。代表作为《青玉案》、《六州歌头》。其词风婉约而豪放。 \tabularnewline\hline
  潘大临 & \begin{description}
  \item[字] 君孚\\邠老
  \item[号] 
  \item[谥] 
  \item[尊] 
  \item[生] 湖北黄州
  \end{description} & 潘大临(生卒年不详),字君孚,一字邠老,原籍长乐三溪,黄州(今属湖北)吝安镇人。北宋著名诗人。大临与其弟潘大观皆有诗名。元丰三年(1080年)苏东坡贬黄州,二月与潘鲠、潘丙有交往;张耒谪黄州时,多有交往。 \tabularnewline\hline
  陈师道 & \begin{description}
  \item[字] 履常\\无己
  \item[号] 后山居士
  \item[谥] 
  \item[尊] 
  \item[生] 江苏徐州
  \end{description} & 陈师道(1053年-1101年),字履常,一字无己,别号后山居士,彭城(今江苏徐州)人,北宋诗人。师道一生淡薄名利,闭门苦吟,有“闭门觅句陈无己”之称。苏门六君子之一,常与苏轼、黄庭坚等唱和,见黄庭坚之诗,爱不释手,把自己的旧作全部烧掉,重学黄诗,后致力于学杜甫;方回的《瀛奎律髓》有“一祖三宗”之说,即以杜甫为祖,三宗便是黄庭坚、陈师道和陈与义。著有《后山集》、《后山谈丛》、《后山诗话》等。门人魏衍编有《彭城陈先生集》二十卷。 \tabularnewline\hline
  张耒 & \begin{description}
  \item[字] 文潜
  \item[号] 柯山
  \item[谥] 
  \item[尊] 
  \item[生] 江苏淮安
  \end{description} & 张耒(1054年-1114年),字文潜,号柯山,生于楚州淮阴(今江苏淮安市),祖籍亳州谯县(今安徽亳县)。北宋诗人。早年游学陈州,受到当时学官苏辙厚爱,从学于苏轼,苏轼说他的文章类似苏辙,“汪洋淡泊,有一唱三叹之声”。张耒嗜酒,晚年有疾。其诗学白居易、张籍,如:《田家》、《海州道中》、《输麦行》多反映下层人民的生活以及自己的生活感受,风格平易晓畅。他与黄庭坚、秦观、晁补之三人一同被时人誉为“苏门四学士”。编《苏门六君子文粹》,有四库全书版本。 \tabularnewline\hline
  蔡绦 & \begin{description}
  \item[字] 约之
  \item[号] 无为子\\百衲居士
  \item[谥] 
  \item[尊] 
  \item[生] 不详
  \end{description} & 蔡絛,字约之,别号无为子、百衲居士。蔡京之季子。徽宗宣和六年(1124年),蔡京担任太师,起领三省,因年老不能事事,奏判悉取决于蔡絛。宣和七年,赐进士出身,不久勒令停止,官至徽猷阁待制。靖康元年(1126年),蔡京垮台后,其子孙二十三人被流放,蔡絛亦遭到流放邵州,再改白州(广西博白),死于戌所。著有《国史后补》、《北征纪实》、《铁围山丛谈》、《西清诗话》及《蔡百衲诗评》等。 \tabularnewline\hline
  魏泰 & \begin{description}
  \item[字] 道辅
  \item[号] 汉上丈人
  \item[谥] 
  \item[尊] 
  \item[生] 湖北襄樊
  \end{description} & 魏泰,字道辅,号汉上丈人,晚号临汉隐居,北宋襄阳邓城(今湖北省襄樊市)人,生卒年不详,约活动于宋神宗、哲宗、徽宗时期。出生世族,为北宋著名女词人魏芷之弟。著有《东轩笔录》十一卷、《临汉隐居集》二十卷、《临汉隐居诗话》一卷、《东轩笔录》十五卷、《襄阳形胜赋》、《续录》一卷,是一本记载宋太祖至神宗六朝旧事的笔记,今存者唯笔录、诗话及诗四首。 \tabularnewline\hline
  范温 & \begin{description}
  \item[字] 元实
  \item[号] 
  \item[谥] 
  \item[尊] 
  \item[生] 不详
  \end{description} & 范温,字元实,范祖禹之子,秦少游之婿,吕居仁之表叔,曾学诗于黄庭坚。著有《潜溪诗眼》一卷。 \tabularnewline\hline
  唐庚 & \begin{description}
  \item[字] 子西
  \item[号] 
  \item[谥] 
  \item[尊] 
  \item[生] 四川眉山
  \end{description} & 唐庚(1071年-1121年),字子西,眉州丹棱(今属四川)人。唐庚是哲宗绍圣元年(1094年)进士,利州司法参军,为宰相张商英所赏识。绍圣四年(1110年),除京畿路提举常平。张商英罢相后,被贬惠州。政和七年(1117年)还京,提举上清太平宫。宣和三年(1121年)归返四川,卒于途中。著有《眉山诗集》、《眉山文集》,清四库全书集部存录本。 \tabularnewline\hline
  惠洪 & \begin{description}
  \item[字] 德洪
  \item[号] 觉范
  \item[谥] 
  \item[尊] 
  \item[生] 江西高安
  \end{description} & 惠洪(1071年-1128年),名德洪,号觉范,俗姓彭。北宋筠州(今江西高安)人。出生于今宜丰县桥西盐岭下竹园彭家,族叔彭几,官至协律郎。元丰七年(1084年)父母双亡,至县城北郊三峰山宝云寺为童子,元祐四年(1089年),参加东京天王寺佛经考试,冒惠洪名得剃度为僧。后依真净禅师,迁往洪州石门寺。后还俗。黄庭坚曾教他读书,与尚书右仆射张商英和节度使郭天信有往来。政和元年(1111年),因张商英一案牵连,流放朱崖(广东海口市)。三年后赦还,居筠州。建炎二年(1128年)逝世于新昌。惠洪长于诗文,“觉范斯须立就”,《彦周诗话》说:“颇似文章巨公所作,殊不类衲子。”被推为“宋僧之冠”,王安石女儿称其“浪子和尚”。 \tabularnewline\hline
  韩驹 & \begin{description}
  \item[字] 子苍
  \item[号] 牟阳
  \item[谥] 
  \item[尊] 陵阳先生
  \item[生] 四川井研
  \end{description} & 韩驹(1080年~1135年),字子苍,号牟阳,陵阳仙井(今四川井研)人。南宋初诗人,世称陵阳先生。少有文名,黄庭坚称其诗“超轶绝尘”。韩驹于元丰三年(1080年)出生,早年在许下从苏辙学,苏辙称读其诗“恍然重见储光羲”。韩驹是江西诗派人物,曾季狸《艇斋诗话》:“后山(陈师道)论诗说换骨,东湖(徐俯)论诗说中的,东莱(吕本中)论诗说活法,子苍论诗说饱参。”。晚年以为“学古人尚恐不至,况学今人哉!”。绍兴五年(1135年)卒于抚州(今江西临川),得年五十六岁。今存《陵阳先生诗》四卷。《宋史》卷四四五有传。 \tabularnewline\hline
  周紫芝 & \begin{description}
  \item[字] 小隐
  \item[号] 竹坡居士
  \item[谥] 
  \item[尊] 
  \item[生] 安徽宣城
  \end{description} & 周紫芝(1082年-1155年),字小隐,号竹坡居士。宣城(今属安徽)人, 南宋文学家、官员。少时家贫,勤学不辍,绍兴十二年(1142年)进士。历官枢密院编修, 绍兴十七年(1147年)为右迪功郎敕令所删定官。二十一年四月出京知兴国军(今湖北阳新县),为政简静,晚年隐居九江庐山。工于诗,不引典故,谀颂秦桧父子,为时论所嘲。约卒于绍兴末年。著有《太仓稊米集》、《竹坡诗话》、《竹坡词》。 \tabularnewline\hline
  吕本中 & \begin{description}
  \item[字] 居仁
  \item[号] 紫薇
  \item[谥] 
  \item[尊] 
  \item[生] 东莱先生
  \end{description} & 吕本中(1084年-1145年),初名大中,字居仁,号紫微、东莱,寿州(今安徽寿县)人。生于宋神宗元丰七年(1084年),是道学家,学者称之为“东莱先生”。著有《东莱先生诗集》、《江西诗社宗派图》、《紫微诗话》及《童蒙诗训》等。绍兴八年(1145年),卒于上饶。《宋史》卷376有传。 \tabularnewline\hline
  葛立方 & \begin{description}
  \item[字] 常之
  \item[号] 懒真子
  \item[谥] 
  \item[尊] 
  \item[生] 江苏江阴
  \end{description} & 葛立方(-1165年),字常之,号懒真子。南宋江阴人。葛密之孙,葛胜仲之子,母张濩。随父徙居吴兴。绍兴八年进士及第,历官左奉议郎、诸王宫大小学教授、太常博士。十七年,除秘书省正字。十九年,迁校书郎。二十一年,为尚书考功员外郎兼中书舍人。官至吏部侍郎。因得罪秦桧,被逼退出官场。绍兴二十六年归休于吴兴汛金溪上。他“博极群书,以文章名一世”,曾著有《韵语阳秋》二十卷、《西畴笔耕》五十卷、《方舆别志》二十卷、《归愚集》一卷,今存《韵语阳秋》与《归愚集》。《四库全书总目提要》评:“多平实铺叙,少清新宛转之思,然大致不失宋人规格。”隆兴二年卒。 \tabularnewline\hline
  周邦彦 & \begin{description}
  \item[字] 美成
  \item[号] 清真居士
  \item[谥] 
  \item[尊] 
  \item[生] 浙江杭州
  \end{description} & 周邦彦(1056年-1121年),中国北宋末期著名的词人,音乐家,字美成,号清真居士,钱塘(今浙江杭州)人。据记载他少年时期个性比较疏散,但相当喜欢读书,宋神宗时,他写了一篇《汴都赋》,赞扬新法,因此由诸生擢为太学正,任教太学。当上学正后,常有积极作为,但在仕途上并没有得意的成果,长期在州县间担任小官职。倒是词愈写愈受世人喜爱,加上精通音律,能自创新曲,词名愈来愈大。到宋徽宗时,周邦彦升为徽猷阁待制,并提举大晟府,任命周邦彦担任主管,从事审订古调,讨论古音,并创设许多音律,影响后世很大。徽宗时期是他作品最多的时期,大部分都带有他华美、轻狂的特质。长期被后人尊为“词家之冠”。 \tabularnewline\hline
  李清照 & \begin{description}
  \item[字] 
  \item[号] 易安居士
  \item[谥] 
  \item[尊] 
  \item[生] 山东济南
  \end{description} & 李清照(1084年3月13日-1155年5月12日),北宋齐州(今山东省济南市)人,为中国历史上最著名的女词人。自号易安居士,与辛幼安并称“济南二安”;又因其词有“新来瘦,非干病酒,不是悲秋”《凤凰台上忆吹箫》、“知否?知否?应是绿肥红瘦”《如梦令》、“莫道不销魂。帘卷西风,人比黄花瘦”《醉花阴》三句,故人称“李三瘦”。有《易安居士文集》七卷、《易安词》八卷,皆佚散。现有《漱玉词》的辑本,存其作约五十首。 \tabularnewline\hline
  徐俯 & \begin{description}
  \item[字] 师川
  \item[号] 东湖居士
  \item[谥] 
  \item[尊] 
  \item[生] 江西修水
  \end{description} & 徐俯(1075年-1141年),字师川,号东湖居士,洪州分宁(今江西修水)人。黄庭坚之甥,父徐禧死于宋夏战争。元丰末年,袭父爵授通直郎,后升司门郎,累官右谏议大夫。靖康元年(1126年),金人围汴京(今河南开封),次年攻陷东京,靖康二年(1127年)张邦昌僭位,徐俯辞归。入江西诗社,与董颖、韩驹等有往来。吕本中《江西诗社宗派图》列其名。建炎初年,内侍郑谌极赏识徐俯文才,向高宗荐举,胡直儒、汪藻等亦荐之,绍兴二年(1132年)赐进士出身。绍兴三年(1133年)升迁为翰林学士,再擢拔为端明殿学士。官至参知政事。因与赵鼎不合去职。绍兴九年,知信州,被劾不理郡事,又被罢免。晚年提举洞霄宫,绍兴十一年终老德兴天门村。著有《东湖诗集》六卷。 \tabularnewline\hline
  陈与义 & \begin{description}
  \item[字] 去非
  \item[号] 简斋
  \item[谥] 
  \item[尊] 
  \item[生] 河南洛阳
  \end{description} & 陈与义(1090年-1138年),字去非,号简斋,洛阳(今属河南)人。徽宗政和三年(1113年)甲科进士,授开德府教授。宣和四年(1122年)擢太学博士、著作佐郎。宋室南渡后,避乱于襄汉。高宗建炎四年(1130年),召为兵部员外郎。绍兴元年(1131年)迁中书舍人。绍兴五年(1135年),召为给事中。绍兴六年(1136年),拜翰林学士。绍兴八年(1138年),以资政殿学士知湖州,因病卒。有《简斋集》三十卷。 \tabularnewline\hline
  王铚 & \begin{description}
  \item[字] 性之
  \item[号] 汝阴老民
  \item[谥] 
  \item[尊] 
  \item[生] 安徽阜阳
  \end{description} & 王铚(?-1144年),字性之,自号汝阴老民。汝阴(今安徽阜阳)人。北宋学者王昭素五世孙,父王萃师事欧阳修。王铚约生于元祐初年,幼而博学,读书一目十行,尝从欧阳修学习。大观元年,王铚访曾布于京口,曾布以三子曾纡之女儿嫁之。南渡后寓居剡中,绍兴初年,官迪功郎,高宗建炎四年(1130年),权枢密院编修官。绍兴四年(1134年)撰成《枢庭备检》,为右承事郎。绍兴五年乙卯(1135年),以右承事郎主管江州庐山太平观。绍兴七年,遭秦桧排挤,避居剡溪山,以诗词自娱。世称雪溪先生,绍兴九年(1139)正月,献《元祐八年补录》及《七朝史》,由右承郎迁右宣义郎。绍兴十三年癸亥(1143年),献《太玄经解义》,绍兴十四年(1144年)卒。著有《默记》一卷、《杂纂续》一卷、《侍儿小名录》一卷、《国老谈苑》二卷等书。 \tabularnewline\hline
  吴沆 & \begin{description}
  \item[字] 德远
  \item[号] 环溪
  \item[谥] 
  \item[尊] 文通先生
  \item[生] 江西抚州
  \end{description} & 吴沆,字德远,抚州崇仁(今属江西)人。兄弟吴涛、吴澥皆有文名。吴沆少年学《易经》,绍兴十六年(1146年)与弟吴澥献著作《易璇玑》﹑《三坟训义》,入国子监,太学博士王之望驳其《三坟训义》之说。后以书法犯下庙讳罢归。隐居环溪,好读杜甫诗,认为杜诗最明显的特色是一句说多件事。卒后其弟子私谥文通先生。后人辑有《环溪诗话》一卷。 \tabularnewline\hline
  范成大 & \begin{description}
  \item[字] 致能
  \item[号] 石湖居士
  \item[谥] 文穆
  \item[尊] 
  \item[生] 江苏苏州
  \end{description} & 范成大(1126年-1193年),字致能,号石湖居士,谥文穆,吴郡(今江苏苏州)人。宋代绍兴二十四年(1154年)中进士,初授司户参军,历官监“和剂局”、检讨、编修、正字、校书郎、处州知州、礼部员外郎、祈请国信使、集英殿修撰、出知静江府、广西经略使、敷文阁待制、四川制置使、礼部尚书、资政殿学士等,官至参知政事,追赠少师、崇国公。范成大曾出使金国,在金国气节不屈,撼动了金世宗,有日记《揽辔录》。范成大与杨万里、尤袤、陆游号称“南宋四大诗人”。范成大的诗作在宋代即有显著影响,到清初则影响尤大,有“家剑南而户石湖”(“剑南”指陆游)之说,其诗风格轻巧,但好用僻典、佛经。范成大同时还是著名的词作家、旅游作家,另有《石湖诗集》、《石湖词》、《桂海虞衡志》、《骖鸾录》、《吴船录》、《吴郡志》等著作传世。 \tabularnewline\hline
  杨万里 & \begin{description}
  \item[字] 廷秀
  \item[号] 诚斋
  \item[谥] 
  \item[尊] 
  \item[生] 江西吉水
  \end{description} & 杨万里(1127年10月29日-1206年6月15日),字廷秀,号诚斋,吉水(今江西省吉水县)人,官至宝谟阁学士。一生力主抗金,与尤袤、范成大、陆游合称南宋“中兴四大诗人”。与欧阳修、杨邦乂、胡铨、周必大、文天祥,合称庐陵“五忠一节”。其诗起初模仿江西诗派,后尽焚少时千余首作品,而另辟蹊径。他在《荆溪集自序》自述:“余之诗,始学江西诸君子,既又学后山(陈师道)五字律,既又学半山老人(王安石)七字绝句,晚乃学绝句于唐人。……戊戌作诗,忽若有悟,于是辞谢唐人及王、陈、江西诸君子皆不敢学,而后欣如也。”终于自成一家,即严羽《沧浪诗话》所谓“诚斋体”。诚斋体的特色是富于幽默诙谐、活泼自然,一反“江西诗派”的生硬槎桠。此对当时诗坛风气之转变,颇起作用。 \tabularnewline\hline
  朱熹 & \begin{description}
  \item[字] 元晦\\仲晦
  \item[号] 晦庵\\考亭\\晦翁
  \item[谥] 文
  \item[尊] 朱子
  \item[生] 江西上饶
  \end{description} & 朱熹(1130年10月22日-1200年4月23日),字元晦,一字仲晦,斋号晦庵、考亭,晚称晦翁,又称紫阳先生、紫阳夫子、沧州病叟、云谷老人,行五十二,小名沋郎,小字季延,谥文,又称朱文公。南宋江南东路徽州婺源(今江西上饶市婺源县)人,生于福建路尤溪县(今福建三明市尤溪县)。南宋理学家,程朱理学集大成者,学者尊称朱子。朱熹家境穷困,但自幼聪颖,绍兴十八年(1148年)中进士,年仅十九岁,历高宗、孝宗、光宗、甯宗四朝。于建阳云谷结草堂名“晦庵”,在此讲学,宋理宗赐名“考亭书院”,故世称“考亭学派”,又因朱熹别号“紫阳”,故世称“紫阳学派”。朱熹是程颢、程颐的三传弟子李侗的学生,承北宋周敦颐与二程学说,创立宋代研究哲理的学风,称为理学。其著作甚多,辑定《大学》、《中庸》、《论语》、《孟子》为四书作为教本,也成为后代科举应试的科目,在中国,有专家认为他确立了完整的客观唯心主义体系。 \tabularnewline\hline
  张栻 & \begin{description}
  \item[字] 敬甫
  \item[号] 南轩
  \item[谥] 宣
  \item[尊] 
  \item[生] 四川绵竹
  \end{description} & 张栻(1133年-1180年) 南宋时理学学者。字敬甫,号南轩,汉州绵竹县(今属四川省)人,仕至右文殿修撰。张栻十三岁写“连州八景”诗,与吕祖谦和朱熹齐名,时称“东南三贤”。张栻曾师从胡宏,被誉为“圣门有人,吾道幸矣”。学成归长沙,先后主讲岳麓书院、城南书院。张栻为“湖湘学派”代表人物,与朱熹的“闽学”,吕祖谦的“婺学”鼎足而三。张栻政治上誓不与秦桧为伍,力主抗金,学术上虽承二程,但有别于二程。《宋史·道学传序》称:“张栻之学,亦出程氏,既见朱熹,相与博约,又大进焉!”主要著作有:《论语解》、《孟子说》、《洙泗言仁》、《诸葛忠武侯传》、《经世编年》等。 \tabularnewline\hline
  陈亮 & \begin{description}
  \item[字] 同甫
  \item[号] 龙川先生
  \item[谥] 
  \item[尊] 
  \item[生] 浙江金华
  \end{description} & 陈亮(1143年10月16日-1194年),南宋两浙东路婺州永康县(今浙江金华永康市)人,字同甫,号龙川先生,南宋政治家、哲学家、词人。反对以朱熹为代表的理学。著有《龙川先生集》。中又以上孝宗皇帝四书、《酌古论》最知名。龙川先生是朴素唯物主义思想的哲学家。创立永康学派,主“事功”。陈亮词风以豪迈雄健为主,有慷慨悲歌,“自负以经济之意具在。”。辛弃疾曾称赞陈亮,“同父之才,落笔千言,俊丽雄伟,珠明玉坚,文方窘步,我独沛然。”南宋著名词人,风格豪放激昂,是豪放派代表。词句中常抒发自己的政治抱负和爱国激情。有《龙川文集》、《龙川词》传世。 \tabularnewline\hline
  王楙 & \begin{description}
  \item[字] 勉夫
  \item[号] 分定居士
  \item[谥] 
  \item[尊] 
  \item[生] 平江吴县
  \end{description} & 王楙,宋福州福清人,徙居平江吴县,字勉夫,号分定居士。生于绍兴二十一年,少失父,事母以孝闻。宽厚诚实,刻苦嗜书。功名不偶,杜门著述,当时称为讲书君。客湖南仓使张頠门三十年,宾主相欢如一日。所著《野客丛书》三十卷,分门类聚,钩隐抉微,考证经史百家,下至骚人墨客,佚草佚事,细大不捐。另有《巢睫稿笔》。宋宁宗嘉定六年卒,年六十三。事见《野客丛书》附《宋王勉夫圹铭》。 \tabularnewline\hline
  刘过 & \begin{description}
  \item[字] 改之
  \item[号] 龙洲道人
  \item[谥] 
  \item[尊] 
  \item[生] 江西吉安
  \end{description} & 刘过(1154年-1206年),字改之,号龙洲道人,太和(今江西吉安市泰和县)人,一作庐陵(今江西吉安市)人。南宋词人。喜言兵事,早年流落江湖,重义气,力主恢复北土,与岳珂友好,与辛弃疾有唱和,词风亦相近,“赡逸有思致”。刘熙载说“刘改之词狂逸中自饶俊致”。与刘仙伦齐名,世称庐陵二布衣。有《龙洲集》、《龙洲词》。代表作有《唐多令》等。 \tabularnewline\hline
  赵蕃 & \begin{description}
  \item[字] 昌父
  \item[号] 章泉
  \item[谥] 文节
  \item[尊] 
  \item[生] 郑州
  \end{description} & 赵蕃(1143年-1229年),字昌父,号章泉,其先祖为郑州人。靖康之变后,居信州玉山(今属江西)。师从刘清之,以曾祖赵旸恩荫补州文学,调浮梁(今景德镇)尉、连江(今福建连江)主簿,皆不赴任,又曾为太和(今江西泰和县)主簿。后调辰州司理参军,因与知州争狱罢官。居家三十三年,五十岁问学于朱熹。能诗,宗黄庭坚,与韩淲(号涧泉)合称“二泉先生”。理宗绍定二年,以直秘阁致仕,不久卒,享寿八十七。 \tabularnewline\hline
  史达祖 & \begin{description}
  \item[字] 邦卿
  \item[号] 梅溪
  \item[谥] 
  \item[尊] 
  \item[生] 河南开封
  \end{description} & 史达祖字邦卿,号梅溪,汴京(河南开封)人。寓居杭州。早年师事张磁,但屡试不中,只好当韩侂胄的幕僚,任“省吏”,负责撰拟文稿,“奉行文字,拟帖撰旨,俱出其手”,颇得韩的倚重。开禧三年(1207年)韩侂胄因北伐事败被杀,达祖遭到牵连,被处以黥刑。流放到江汉。晚年困顿而死。达祖工于填词,姜夔称其词风“奇秀清逸”,善咏物,精于描写刻画,有《梅溪词》传世。王士祯在《花草蒙恰》中说:“仆每读史邦卿‘咏燕’词,以为咏物至此,人巧极天工矣。”。 \tabularnewline\hline
  姜夔 & \begin{description}
  \item[字] 尧章
  \item[号] 白石道人\\石帚
  \item[谥] 
  \item[尊] 
  \item[生] 江西鄱阳
  \end{description} & 姜夔(1155年-1209年),字尧章,号白石道人,饶州鄱阳(今江西省鄱阳)人。中国南宋词人。一生没有做过官,家贫,无立锥之地。精通音乐,会为诗,初学山谷之江西诗派,后被归类为江湖诗派。亦善填词,自度十七曲传世。范成大称其:“翰墨人品,皆似晋宋之雅士。”他的词对于南宋后期词坛的格律化有巨大的影响,姜夔和张炎并称为“姜张”。曾与杨万里、范成大、辛弃疾等交游。约卒于嘉定二年(1209年)。 \tabularnewline\hline
  韩淲 & \begin{description}
  \item[字] 仲止
  \item[号] 涧泉
  \item[谥] 
  \item[尊] 
  \item[生] 河南杞县
  \end{description} & 韩淲(biāo)(1159年-1224年),字仲止,一作子仲,号涧泉。开封雍丘(今河南杞县)人。吏部尚书韩元吉之子。生于绍兴二十九年(1159年),早年以父荫入仕,为平江府属官,嘉泰元年(1201年)曾入吴应试。不久被斥。后家居二十年。南渡后,落籍信州上饶(今属江西),与赵蕃(号章泉)合称“上饶二泉”。嘉定十七年(1224年),得疾而卒,得年六十六。著有《涧泉集》二十卷、《涧泉日记》三卷、《涧泉诗馀》一卷。 \tabularnewline\hline
  施岳 & \begin{description}
  \item[字] 仲山
  \item[号] 梅川
  \item[谥] 
  \item[尊] 
  \item[生] 江苏苏州
  \end{description} & 施岳,字仲山,号梅川。吴(今苏州)人。生卒年均不详,约宋理宗淳佑中前后在世。精于音律,死后由杨缵为树梅作亭,薛梦珪为作墓志,李彭老书,周密题,葬于西湖虎头岩下。生平事迹因无相关记载已经不可考。 \tabularnewline\hline
  严羽 & \begin{description}
  \item[字] 丹邱\\仪卿
  \item[号] 沧浪逋客
  \item[谥] 
  \item[尊] 
  \item[生] 福建邵武
  \end{description} & 严羽,字丹邱,号沧浪逋客。宋邵武(今属福建)人。生卒年不详。早年就学于邻县光泽县学教授包恢门下,包恢之父包扬曾受学于朱熹。他与严仁、严参并称“邵武三严”。并且受到司空图的影响,而有“妙悟说”。宋亡后隐居不仕,曾浪迹江、楚等地。严羽教人学诗,必先熟读《楚辞》,乃至于盛唐名家作品,并且反对苏轼、黄庭坚的诗风,称其为诗虽工,“盖于一唱三叹之音有所歉焉”,同时批评四灵派和江湖派。戴复古《祝二严》称:“羽也天资高,不肯事科举,风雅与骚些,历历在肺腑。持论伤太高,与世或龃龉。”。严氏事不见《宋史》,《福建通志》有载。著有《沧浪诗话》。 \tabularnewline\hline
  翁卷 & \begin{description}
  \item[字] 续古\\灵舒
  \item[号] 
  \item[谥] 
  \item[尊] 
  \item[生] 浙江乐清
  \end{description} & 翁卷,字续古,一字灵舒,乐清(今属浙江)人,南宋时期诗人。曾领乡荐,但一生布衣。工于诗,与徐照、徐玑、赵师秀合称“永嘉四灵”。中年以后迁居永嘉县城。有《四岩集》,《苇碧轩集》。 \tabularnewline\hline
  戴复古 & \begin{description}
  \item[字] 式之
  \item[号] 石屏
  \item[谥] 
  \item[尊] 
  \item[生] 浙江温岭
  \end{description} & 复古(1167年-1248年),字式之。天台黄岩南塘(今属浙江省温岭市新河镇)人。常居南塘石屏山,故自号石屏,南宋著名的江湖派诗人。南宋孝宗乾道三年(1167年)出生于天台道黄岩县南塘屏山,终身布衣,浪游江湖,“凡空迥奇特荒怪古僻之迹,靡不登历”。曾从陆游学诗,作品受晚唐诗风影响,兼具江西诗派风格。部分作品抒发爱国思想,反映人民疾苦,具有现实意义。其诗词格调高朗,诗笔俊爽,清健轻捷,工整自然。“往往作豪放语,锦丽是其本色。”(况周颐语)。他以诗鸣江湖间,楼钥称其“尤笃意古律……又登三山陆放翁之门,而诗益进”,真德秀《石屏词跋》云:“戴复古诗词,高处不减孟浩然。”。传世有《石屏集》六卷,《石屏长短句》一卷。 \tabularnewline\hline
  刘克庄 & \begin{description}
  \item[字] 潜夫
  \item[号] 后村居士
  \item[谥] 
  \item[尊] 
  \item[生] 福建莆田
  \end{description} & 刘克庄(1187年-1269年)初名灼,字潜夫,号后村居士,莆田城厢(今属福建)人。吏部侍郎刘弥正之子。宋朝爱国诗词家,为江湖诗派人物。理宗淳佑六年(1246年)以“文名久著,史学尤精”,赐进士,历任枢密院编修、中书舍人、兵部侍郎等,官至龙图阁直学士。其间因弹劾宰相史嵩之而先后五次贬官。惜晚节不保,晚年与奸臣贾似道交好,为人所讥。诗学晚唐,刻琢精丽,为江湖诗派中的领军人物。创作大量的爱国诗词,著作有《后村别调》和《后村先生大全集》,有诗5000多首,词200多首。 \tabularnewline\hline
  谢枋得 & \begin{description}
  \item[字] 君直
  \item[号] 叠山
  \item[谥] 
  \item[尊] 
  \item[生] 江西信州
  \end{description} & 谢枋得(1226年3月23日-1289年4月25日),字君直,号叠山,远祖居会稽,信州弋阳(今属江西)人,南宋移民、文学家,隐居福建建宁、泉州安溪、被元朝征调至燕京,不降,绝食而死,门人私谥文节。南宋灭亡后,枋得隐居于福建建宁县,又至泉州府安溪县唐石山,流寓槐植村,以卜卦、教书度日,不索钱财,惟取米、屦(白米和草鞋)而已。曾到武夷山拜访遗民熊禾。元朝先后五次征聘,坚辞不应,并写《却聘书》:“人莫不有一死,或重于泰山,或轻于鸿毛,若逼我降元,我必慷慨赴死,决不失志。”著《叠山集》16卷。他评点的《文章轨范》,是科举考试的范本,以文章类别编选文章,是南宋一部重要的评注选本,被誉为集合宋人评点学之大成。 《千家诗》原名《分门纂类唐宋时贤千家诗选》,刘克庄编辑。谢枋得对原有《千家诗》有所整理增删,成为谢枋得编辑《千家诗》。从此《千家诗》有两种版本并行与世。 \tabularnewline\hline
  吴文英 & \begin{description}
  \item[字] 君特
  \item[号] 梦窗\\觉翁
  \item[谥] 
  \item[尊] 
  \item[生] 浙江宁波
  \end{description} & 吴文英(1200年-1260年),字君特,号梦窗,晚年号觉翁,四明(今浙江宁波鄞县)人,南宋词人。吴文英本姓翁,后来过继给姓吴的人改姓吴。终生未仕。早年居苏州,后入杭州,与当朝的达官贵人交接甚密,比如丞相吴潜、史弥远等。其词多为恋情怀旧之作。当时由于他写词奉承贾似道,被当时的人所鄙视。晚年他寄居荣王赵与芮门下。有《梦窗词集》一部,存词三百四十余首,分四卷本与一卷本。其词作数量丰沃,风格雅致,多酬答、伤时与忆悼之作,号“词中李商隐”。而后世品评却甚有争论。 \tabularnewline\hline
  刘辰翁 & \begin{description}
  \item[字] 会孟
  \item[号] 须溪
  \item[谥] 
  \item[尊] 
  \item[生] 江西吉安
  \end{description} & 刘辰翁(1232年-1297年),南宋诗人,字会孟,号须溪。吉州庐陵(今江西吉安)人。生于绍定五年(1232年),早年入太学,理宗景定三年(1262年)进士,因对策忤权奸贾似道,被评丙等。曾任濂溪书院山长,咸淳元年(1265年),授临安府学教授、参江东转运幕,后荐入史馆,除太学博士。宣传《庄子》思想,与江万里友好。德佑元年(1275年),文天祥勤王,辰翁参与江西幕府。宋亡元后不仕,隐居以终。卒于元大德元年(1297年)。有《须溪集》10卷、《须溪词》3卷。 \tabularnewline\hline
  周密 & \begin{description}
  \item[字] 公谨
  \item[号] 草窗\\四水潜夫\\弁阳老人
  \item[谥] 
  \item[尊] 
  \item[生] 山东
  \end{description} & 周密(1232年-1298年),宋末元初人,字公谨,号草窗,又号四水潜夫、弁阳老人、弁阳啸翁。著有《齐东野语》等书。周密为南宋末年雅词词派领袖,有词集《萍洲渔笛谱》,词选《绝妙好词》流传于世。周密曾作有《三姝媚》送王沂孙,王沂孙也赋词相和。周密、张炎,和王沂孙、蒋捷并称宋末四大词家。他虽出身望族,却无意仕进,一生中大部分时间为平民,可谓一个“职业江湖雅人”,从其自号“草窗”便可见端倪,其词风格在姜夔、吴文英之间,与吴文英并称“二窗”。 \tabularnewline\hline
  仇远 & \begin{description}
  \item[字] 仁近
  \item[号] 
  \item[谥] 
  \item[尊] 
  \item[生] 浙江杭州
  \end{description} & 仇远(1247年~1326年),字仁近,一字仁父,钱塘(今浙江杭州)人。因居余杭溪上之仇山,自号山村、山村民,人称山村先生。元代文学家、书法家。元大德年间(1297~1307)五十八岁的他任溧阳儒学教授,不久罢归,遂在忧郁中游山河以终。著有《金渊集》六卷,皆官溧阳时所作,清人从《永乐大典》中辑出。另有《兴观集》、《山村遗集》,是清项梦昶所编,残缺不全。 \tabularnewline\hline
  唐珏 & \begin{description}
  \item[字] 玉潜
  \item[号] 菊山
  \item[谥] 
  \item[尊] 
  \item[生] 浙江绍兴
  \end{description} & 唐珏(1247-?),字玉潜,号菊山,南宋词人、义士。会稽山阴(今浙江绍兴)人。于《宋史翼》、《新元史》有传。亦记载于《宋人轶事汇编》。今存词四首,《全宋词》据《乐府补题》辑录。 \tabularnewline\hline
  文天祥 & \begin{description}
  \item[字] 宋瑞\\履善
  \item[号] 浮休道人\\文山
  \item[谥] 
  \item[尊] 
  \item[生] 江西吉安
  \end{description} & 文天祥(1236年6月6日-1283年1月9日),初名云孙,字宋瑞,一字履善。道号浮休道人、文山。江西吉州庐陵(今江西省吉安市青原区富田镇   )人,南宋末政治家、文学家,爱国诗人,抗元名臣、民族英雄,与陆秀夫、张世杰并称为“宋末三杰”。元至元十九年十二月初九(1283年1月9日),于大都就义。终年47岁。 著有《文山诗集》、《指南录》、《指南后录》、《正气歌》等。 \tabularnewline\hline
  王沂孙 & \begin{description}
  \item[字] 圣与
  \item[号] 碧山\\中仙
  \item[谥] 
  \item[尊] 竹笥山人
  \item[生] 浙江绍兴
  \end{description} & 王沂孙,生卒年不详,字圣与,又字咏道,号碧山,又号中仙,因家住玉笥山,故又号玉笥山人,南宋会稽(今浙江绍兴)人,大约生活在1230年至1291年之间,曾任庆元路(路治今宁波鄞州)学正。王沂孙工词,风格接近周邦彦,含蓄深婉,如《花犯·苔梅》之类。其清峭处,又颇似姜夔,张炎说他“琢语峭拔,有(姜)白石意度”。尤以咏物为工,如《齐天乐·蝉》、《水龙吟·白莲》等,皆善于体会物象以寄托感慨。其词章法缜密,在宋末格律派词人中是一位有显著艺术个性的词家,与周密、张炎、蒋捷并称“宋末词坛四大家”。词集《碧山乐府》,一称《花外集》,收词60余首。 \tabularnewline
  \bottomrule
\end{longtable}


%%% Local Variables:
%%% mode: latex
%%% TeX-engine: xetex
%%% TeX-master: "../../Main"
%%% End:
