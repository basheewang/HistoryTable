%% -*- coding: utf-8 -*-
%% Time-stamp: <Chen Wang: 2021-11-01 11:48:02>

\section{安帝司马德宗\tiny(397-418)}

\subsection{生平}

晋安帝司马德宗(382年-419年1月28日),字德宗,东晋的第十位皇帝。晋孝武帝司马曜的长子,母亲是陈归女。晉安帝由於痴愚而無能力掌握國政,在位廿二年間朝權都旁落在臣下之中,國內內亂頻仍,期間甚至發生了桓玄篡位的事件。最後東晉國祚及國力在北府將領劉裕的主持下獲得恢復,但亦為劉裕奠下篡位的基礎,安帝自己亦因劉裕欲篡而遇害。

司马德宗於太元十二年八月辛巳(387年9月16日)被立为皇太子。太元二十一年九月庚申(396年11月6日)孝武帝被張貴人所弒,次日安帝正式继位。

安帝本愚,從小到大连话都不太会说,就连冬夏的区别都认不出来,因此朝廷的权力实际上完全由当朝大臣掌握,沒有一道詔旨,一個行動是出自安帝自己的意願。安帝初期朝廷政策主要由以太傅攝政旳会稽王司马道子主持,王恭之亂後則由會稽世子司馬元顯掌握。元興元年(402年),司馬元顯讓安帝改元「元興」,預備出兵討伐桓玄,但其年為桓玄所敗,朝政亦從此轉歸桓玄掌握。桓玄先廢元興年號,改元「大亨」,後專制朝廷,準備篡位。翌年十二月壬辰(404年1月1日),桓玄篡位,建立桓楚政權,廢安帝為平固王,並在次日被送至尋陽。

桓玄篡位後僅兩月,北府將領劉裕等人於京口、廣陵成功舉兵,並合軍進攻建康,桓玄軍隊不敵並撤到尋陽,隨即又挾持晉安帝退至江陵。桓玄在江陵試圖重整旗鼓,於五月癸酉(404年6月10日)逼令安帝隨軍之下於崢嶸洲與劉毅等軍決戰。不過,桓玄再敗,只有率敗軍及安帝退還江陵,隨後在往蜀地的路上被殺。留在江陵的安帝在荊州別駕王康產及南郡太守王騰之的支援下於江陵復位。然而,不久桓振率領桓楚殘部進襲江陷,安帝再度被俘。義熙元年(405年)正月,晉軍收復江陵,安帝再度復位,改元「義熙」,隨後獲迎回建康。時朝廷就由劉裕為首,至義熙四年(408年)接替去世的王謐領揚州刺史、錄尚書事時完全掌握朝權。

安帝即位後,先有王恭兩度舉兵,荊州各地亦陷入割據狀態,朝廷影響力量在司馬道子父子主政時一度僅及三吳。內亂不休的晉廷因而無力守衞北方領土,至隆安三年(399年),後秦佔領洛陽及河南地區,南燕則佔領山東半島,建都廣固城。東晉再次喪失了淮水以北的大部分領土。接著發生的孫恩盧循之亂以及桓玄篡位事件,亦讓東晉未能有暇收復失地,蜀地甚至在討伐桓玄期間由譙縱建立的譙蜀控制。劉裕主政之下,東晉於義熙五年(409年)出兵進攻南燕,並翌年滅南燕,收復齊地;雖然盧循乘劉裕北伐的機會襲擊建康,但劉裕適時回軍,成功防禦京師,並於稍後擊潰盧循主力,盧循及其勢力於義熙七年(411年)完全肅清。劉裕又在義熙九年(413年)派將領朱齡石收復蜀地。另一方面,他又於義熙八年(412年)先後消滅了當日與他一同起兵討伐桓玄的劉毅及諸葛長民,又於義熙十一年(415年)消滅了宗室司馬休之,除去了政敵。義熙十二年(416年),劉裕奉琅邪王司馬德文的名義北伐後秦,於翌年成功滅掉後秦,不但收服了河南及洛陽失地,更重奪關中。雖然關中於義熙十四年(418年)劉裕班師後為夏國所佔領,但劉裕地位已經穩固,受封十郡宋公,亦圖篡位,因為「昌明之後有二帝」的預言,故晉安帝在劉裕授意下於十二月戊寅日(419年1月28日)被中書侍郎王韶之殺害,其弟司馬德文被擁立,以應預言。

\subsection{隆安}

\begin{longtable}{|>{\centering\scriptsize}m{2em}|>{\centering\scriptsize}m{1.3em}|>{\centering}m{8.8em}|}
  % \caption{秦王政}\
  \toprule
  \SimHei \normalsize 年数 & \SimHei \scriptsize 公元 & \SimHei 大事件 \tabularnewline
  % \midrule
  \endfirsthead
  \toprule
  \SimHei \normalsize 年数 & \SimHei \scriptsize 公元 & \SimHei 大事件 \tabularnewline
  \midrule
  \endhead
  \midrule
  元年 & 397 & \tabularnewline\hline
  二年 & 398 & \tabularnewline\hline
  三年 & 399 & \tabularnewline\hline
  四年 & 400 & \tabularnewline\hline
  五年 & 401 & \tabularnewline
  \bottomrule
\end{longtable}

\subsection{元兴}

\begin{longtable}{|>{\centering\scriptsize}m{2em}|>{\centering\scriptsize}m{1.3em}|>{\centering}m{8.8em}|}
  % \caption{秦王政}\
  \toprule
  \SimHei \normalsize 年数 & \SimHei \scriptsize 公元 & \SimHei 大事件 \tabularnewline
  % \midrule
  \endfirsthead
  \toprule
  \SimHei \normalsize 年数 & \SimHei \scriptsize 公元 & \SimHei 大事件 \tabularnewline
  \midrule
  \endhead
  \midrule
  元年 & 402 & \tabularnewline\hline
  二年 & 403 & \tabularnewline\hline
  三年 & 404 & \tabularnewline
  \bottomrule
\end{longtable}

\subsection{大亨}

\begin{longtable}{|>{\centering\scriptsize}m{2em}|>{\centering\scriptsize}m{1.3em}|>{\centering}m{8.8em}|}
  % \caption{秦王政}\
  \toprule
  \SimHei \normalsize 年数 & \SimHei \scriptsize 公元 & \SimHei 大事件 \tabularnewline
  % \midrule
  \endfirsthead
  \toprule
  \SimHei \normalsize 年数 & \SimHei \scriptsize 公元 & \SimHei 大事件 \tabularnewline
  \midrule
  \endhead
  \midrule
  元年 & 402 & \tabularnewline
  \bottomrule
\end{longtable}

\subsection{义熙}

\begin{longtable}{|>{\centering\scriptsize}m{2em}|>{\centering\scriptsize}m{1.3em}|>{\centering}m{8.8em}|}
  % \caption{秦王政}\
  \toprule
  \SimHei \normalsize 年数 & \SimHei \scriptsize 公元 & \SimHei 大事件 \tabularnewline
  % \midrule
  \endfirsthead
  \toprule
  \SimHei \normalsize 年数 & \SimHei \scriptsize 公元 & \SimHei 大事件 \tabularnewline
  \midrule
  \endhead
  \midrule
  元年 & 405 & \tabularnewline\hline
  二年 & 406 & \tabularnewline\hline
  三年 & 407 & \tabularnewline\hline
  四年 & 408 & \tabularnewline\hline
  五年 & 409 & \tabularnewline\hline
  六年 & 410 & \tabularnewline\hline
  七年 & 411 & \tabularnewline\hline
  八年 & 412 & \tabularnewline\hline
  九年 & 413 & \tabularnewline\hline
  十年 & 414 & \tabularnewline\hline
  十一年 & 415 & \tabularnewline\hline
  十二年 & 416 & \tabularnewline\hline
  十三年 & 417 & \tabularnewline\hline
  十四年 & 418 & \tabularnewline
  \bottomrule
\end{longtable}


%%% Local Variables:
%%% mode: latex
%%% TeX-engine: xetex
%%% TeX-master: "../Main"
%%% End:
