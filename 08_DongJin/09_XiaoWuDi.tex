%% -*- coding: utf-8 -*-
%% Time-stamp: <Chen Wang: 2019-12-18 13:38:12>

\section{孝武帝\tiny(372-396)}

\subsection{生平}

晋孝武帝司马曜(362年-396年11月6日),字昌明,东晋的第九个皇帝,在位时间是372年至396年。他是晋简文帝的第三个儿子,晋安帝和晋恭帝的父亲,母李陵容。

晋孝武帝四岁时被封为会稽王,372年9月12日被立为太子,同日晋简文帝逝,继位時年僅十一岁。次年年号为宁康,由太后摄政。

14岁时(376年)开始亲政,改年号为太元。当年他改革税收,放弃以田地多少来收税的方法,改为王公以下每人收米三斛,在役的人不交税。此外他在位期间大力加强皇帝的权力和地位,史載他“威權己出”,扭轉了東晉自晉明帝死後皇权旁落的局面。

383年前秦进攻东晋,试图消灭长年偏安的东晋,结果在淝水之战中,晋军大胜。

384年後,晉孝武帝趁著前秦崩解的契機北伐,陸續收復了黃河以南的所有領土(包含河南洛陽及山東半島),甚至劉牢之一度佔領河北鄴城。這使得390年代的東晉版圖,達到了自東晉開始以來的最大值。但是連年征戰,遽增的兵役賦稅使人民痛苦難當,既疲又怨。

晉孝武帝即位初期由於稅賦改革與謝安當國,被稱為東晉後期的復興;但是謝安死後司馬道子當國,以及晋孝武帝北伐成功后开始嗜酒,“醒日既少”,連帶導致“刑網峻急,風俗奢宕”的不良政風。

396年11月6日,晉孝武帝由於对他当时宠信的张贵人开玩笑说:“你已经快要三十歲了,按年龄应该要被废弃了”,導致当晚张贵人一怒之下在清暑殿杀了他,享年34歲。11月30日,葬于今江苏南京的隆平陵。

孝武帝自幼年聰穎,他十歲時父親簡文帝崩逝,但他到了下午仍不去父親遺體旁哭喪,侍從勸告他應按照禮節哭喪,他卻回答說:「哀痛時就是哭喪的最好時機,哪裡需要被常規禮節束縛呢?」宰相謝安對他的清談義理頗為讚嘆,認為他所掌握的精微義理,不下於其父簡文帝。孝武帝親政後將治國大權收歸己手,很有君主的才幹器量。但他年長後沉溺於酒色之中,將政務細節交給位居宰相的弟弟司馬道子,常與道子一同飲酒酣歌。他晚年更通宵飲酒而睡到大白天,因此少有白日清醒的時刻。周遭缺乏剛正的大臣規勸,因此沒法改正嗜酒缺失。

唐代房玄齡於《晉書》評論說:「太宗晏駕,寧康(按:以年號代稱晉孝武帝)纂業,天誘其衷,姦臣自隕,于時西踰劍岫而跨靈山,北振長河而臨清、洛;荊、吳戰旅,嘯吒成雲;名賢間出,舊德斯在:謝安可以鎮雅俗,彪之足以正紀綱,桓沖之夙夜王家,謝玄之善料軍事。于時上天乃眷,強氐自泯。五尺童子,振袂臨江,思所以挂旆天山,封泥函谷;而條綱弗垂,威恩罕樹,道子荒乎朝政,國寶彙以小人,拜授之榮,初非天旨,鬻刑之貨,自走權門,毒賦年滋,愁民歲廣。是以聞人、許榮馳書詣闕,烈宗知其抗直,而惡聞逆耳,肆一醉於崇朝,飛千觴於長夜。雖復『昌明』表夢,安聽神言?而金行穨弛,抑亦人事,語曰『大國之政未陵夷,小邦之亂已傾覆』也。屬苻堅百六之秋,棄肥水之眾,帝號為 『武』,不亦優哉!」

唐代某貴族「公子」與虞世南的對話:「公子曰:『(東晉)中興之政,咸歸大臣,唯孝武為君,威福自己,外摧疆寇,人安吏肅。比于明帝,功業何如?』先生(虞世南)曰:『孝武克夷外難,乃謝安之力也,非人主之功。至于委任會稽,棟梁已撓,殷、王作鎮,亂階斯起,昌明之讖,乃驗于茲。加以末年沉晏,卒致傾覆,比蹤前哲(按:前哲指晉明帝),其何遠乎?』」

\subsection{宁康}

\begin{longtable}{|>{\centering\scriptsize}m{2em}|>{\centering\scriptsize}m{1.3em}|>{\centering}m{8.8em}|}
  % \caption{秦王政}\
  \toprule
  \SimHei \normalsize 年数 & \SimHei \scriptsize 公元 & \SimHei 大事件 \tabularnewline
  % \midrule
  \endfirsthead
  \toprule
  \SimHei \normalsize 年数 & \SimHei \scriptsize 公元 & \SimHei 大事件 \tabularnewline
  \midrule
  \endhead
  \midrule
  元年 & 373 & \tabularnewline\hline
  二年 & 374 & \tabularnewline\hline
  三年 & 375 & \tabularnewline
  \bottomrule
\end{longtable}

\subsection{太元}

\begin{longtable}{|>{\centering\scriptsize}m{2em}|>{\centering\scriptsize}m{1.3em}|>{\centering}m{8.8em}|}
  % \caption{秦王政}\
  \toprule
  \SimHei \normalsize 年数 & \SimHei \scriptsize 公元 & \SimHei 大事件 \tabularnewline
  % \midrule
  \endfirsthead
  \toprule
  \SimHei \normalsize 年数 & \SimHei \scriptsize 公元 & \SimHei 大事件 \tabularnewline
  \midrule
  \endhead
  \midrule
  元年 & 376 & \tabularnewline\hline
  二年 & 377 & \tabularnewline\hline
  三年 & 378 & \tabularnewline\hline
  四年 & 379 & \tabularnewline\hline
  五年 & 380 & \tabularnewline\hline
  六年 & 381 & \tabularnewline\hline
  七年 & 382 & \tabularnewline\hline
  八年 & 383 & \tabularnewline\hline
  九年 & 384 & \tabularnewline\hline
  十年 & 385 & \tabularnewline\hline
  十一年 & 386 & \tabularnewline\hline
  十二年 & 387 & \tabularnewline\hline
  十三年 & 388 & \tabularnewline\hline
  十四年 & 389 & \tabularnewline\hline
  十五年 & 390 & \tabularnewline\hline
  十六年 & 391 & \tabularnewline\hline
  十七年 & 392 & \tabularnewline\hline
  十八年 & 393 & \tabularnewline\hline
  十九年 & 394 & \tabularnewline\hline
  二十年 & 395 & \tabularnewline\hline
  二一年 & 396 & \tabularnewline
  \bottomrule
\end{longtable}


%%% Local Variables:
%%% mode: latex
%%% TeX-engine: xetex
%%% TeX-master: "../Main"
%%% End:
