%% -*- coding: utf-8 -*-
%% Time-stamp: <Chen Wang: 2019-12-18 13:34:52>

\chapter{东晋\tiny(317-420)}

\section{简介}

东晋(317年4月6日-420年7月10日),中國朝代,乃西晉司馬氏政權的延續。因内迁的北方游牧民族造反,晉懷帝與晉愍帝先後被俘殺,琅琊王司馬睿在群臣擁戴下在建康(今南京)即位,即晉元帝,史稱東晉。東晉與先前三国时期的东吴以及其後的宋、齊、梁、陳,合稱為六朝。此外,東晉又仿蜀汉称東漢為中汉,称西晋为中朝;又东晋统治地区大部分在江东,古称江左,因此以江左代指东晋。因江東被晉人視為東方,因而司馬睿之後的晉朝被劉宋稱作東晉。同时北方有多个游牧民族建立政权並连年征战,史称五胡十六国时期。

東晉雖然是司馬氏政權的延續,但司馬氏在政治上威望不高,朝廷由世族大家把持,最先的一個乃出身琅琊王氏的王導,其後又有陳郡謝氏的謝安、謝玄等等。而世家大族中的代表者有南下的王、謝、袁、蕭等僑姓,和本身居於江南的朱、張、顧、陸的吳姓。最初東晉有賴權臣王導主持大局,一方面拉攏江南士族,一方面又安排予從中原南下的士族,並以司馬氏作為共同擁戴的對象,司馬氏實際上成為傀儡。世家大族本身並不真正忠於司馬氏,尤其是他們本身都擁有大量田地,以至擁有自家部隊(即所謂「部曲」),有足夠實力抗衡司馬氏政權。最初有王導主持大局,東晉政權得以穩定,故時人稱「王與馬,共天下」。但晉元帝以降則內亂頻生,如有早期王敦之亂、蘇峻之亂,後期又有孫盧之亂等。

东晋也曾多次试图北伐,但由于内部不团结,除了最后篡晋的刘裕取得一定成果外,其余都无建树。祖逖本有希望恢复旧土,但他被晉元帝及世家大族挾制,郁郁而终。桓温的北伐则被慕容垂击败。

376年,前秦苻坚滅掉代國,統一了北方,南北分立之势从此而成。其後苻坚開始率兵南侵。383年,苻坚率約八十七萬兵馬大幅南侵,东晋宰相谢安力主抗击,派谢石谢玄率军,在淝水之战大获全胜,苻坚逃回北方。之後苻坚力量衰弱,因此原本統一的北方再次分裂為多國。后有桓玄叛乱,废安帝,自立为天子,後为大将刘裕所平,拥恭帝,然大权落于刘裕。

420年,刘裕篡位建立劉宋,開啟南北朝時代,東晉亡。

317年,皇族司馬睿在建康城(今江苏省南京市)稱晋王(318年称帝),是為晉元帝,史稱東晉。東晉本身並沒有強大的實力,主要是憑著長江天險,偏安江南;及依靠丞相王導號召南遷避難的中原士族,並聯合南方大族,取得他們的支持。不過,南北大族之間時常發生衝突,內亂頻生,導致東晉政權並不穩定。

自西晉末年劉淵建汉赵以來,南匈奴、羯、白奴、丁零、铁弗、卢水胡、拓跋鲜卑、宇文鲜卑、段氏鲜卑、慕容鲜卑、秃发鲜卑、乞伏鲜卑、九大石胡、大月氐、小月氐和巴氐、姜、夫余、乌桓、高句丽,在中國北方的黃河流域一帶先後建立六十二个割據政權,連同漢族所建立的政權,較重要的有十六個國家,歷史上稱為「五胡十六國」。

從北方南遷的人民時常懷念家鄉,因此一些有志之士多次進行北伐,希望能夠收復北方的國土。祖逖是東晉率先北伐的將領,他曾經率軍收復黃河以南地區,但由於東晉內部出現糾紛,朝廷又擔心他北伐成功後威望太高,結果沒有給予支持,以致功敗垂成,祖逖於321年憂憤而死,曾收復的土地又被胡人重新佔領。

繼祖逖之後,又有桓溫於354年、356年及369年三次北伐,曾一度收復洛陽,他屢次請求朝廷把都城遷回洛陽,但遭到大族的反對,東晉君臣又怕他權勢太大,難以控制,因而無法實現。其後劉裕北伐亦曾收復洛陽和長安。

氐族所建立的前秦,在苻堅時,任用漢人王猛為相,大修政教,富國強兵。前秦強大起來,統一了五胡所據之華北大部分地區。383年,苻堅率軍南下,聲勢浩大,企圖一舉消滅東晉,於是發生了歷史上著名的「淝水之戰」。淝水之戰後,前秦瓦解,北方大亂,再次陷入長期分裂的狀態,胡人無暇南侵。东晋以弱勝強,局勢暫時穩定下來。

東晉的宗室和士族,經常爭權奪利,人民生活相當困苦,以致盜賊四起。淝水之戰後,南方人民暫獲安定,但政治混亂和貪污腐敗的情況,並沒有改善。東晉大臣桓溫死後,其子桓玄逼晋安帝禅位给他,改国号为楚,史称“桓楚”;劉裕起兵聲討,殺死桓玄,恢復東晉的統治。但劉裕自己有奪位的野心,終於在420年,廢晉恭帝自立,改國號為宋,史称“刘宋”。東晉至此正式滅亡。

東晉偏安江南,士族掌權,國君權力旁落,同時各士族之間常為了爭權而北伐,並無單一世族能將司馬氏取而代之,這是政治上特點。

但在另一方面,東晉在文學上卻有一定成就,各類詩文歌賦都大盛於西晉。著名的文學家,有谢灵运、陶渊明、王羲之等人,也流行了駢文。而繪畫、書法也有頗傑出的成就,如東晉人顧愷之的畫作,王羲之的書法,都有很高藝術價值。

著名的中国四大民间传说之一的梁山伯與祝英台的故事背景也發生在東晉時代。

東晉雖非中國史上強盛的時期,卻為文學、藝術極興盛的時代。首都建康成為文化中心,吸引許多東南亞、印度的佛教僧侶及商人前來。338年所鑄造模仿罽賓的佛教模型,為今日所知最早的鎏金銅佛像。中國史上最具影響力的書法家王羲之活躍於此時期。東晉的陶器形式較西晉時期創新。南京富貴山曾挖掘出此時期的墓穴,根據史料記載,此處為東晉皇室墓葬的地點。

東晉也是中國清談盛行的時代。

%% -*- coding: utf-8 -*-
%% Time-stamp: <Chen Wang: 2019-12-18 13:27:04>

\section{元帝\tiny(318-322)}

\subsection{生平}

晉元帝司馬睿(276年5月27日-323年1月3日),字景文,東晉時期第一位皇帝。司馬懿的曾孫、琅邪武王司馬伷之孫、琅邪恭王司馬覲之子,母為琅邪王妃夏侯光姬。《魏書》說司馬睿是牛金和夏侯光姬的私生子。

司馬睿於290年袭封琅邪王,曾經參與討伐成都王司馬穎的戰役;但是由於作戰失利,司馬睿便離開洛陽,回到封國;晉懷帝即位後,司馬睿被封為鎮東大將軍、都督揚州諸軍事,後來在王導的建議之下前往建康,並且極力結交江東大族。311年晉懷帝被俘遇害後,晉愍帝即位,晉愍帝封司馬睿為丞相、大都督中外諸軍事。晉愍帝被俘後,司馬睿在晉朝貴族與江東大族的支持下於317年三月辛卯(公历4月6日)称晋王,318年三月丙辰(公历4月26日)即帝位,為晉元帝。

即位之初曾嘗試北伐,其中祖逖本有希望恢复旧土,但他被晉元帝及世家大族挾制,郁郁而终。

晉元帝實際上為一個被扶持者,本身並無實際權力,大權掌握在王導與王敦之手。晉元帝聽從刁協與劉隗的言論並有意削弱琅邪王氏權力,導致王敦於322年反叛,攻入建康,並且殺害重臣戴淵、周顗等人。但是王敦無力消滅東晉,最後採取與晉元帝和睦的策略。晉元帝便在王敦之亂中因憂鬱過度而過世。

唐代房玄齡於《晉書》的「史臣曰」評論說:「晉氏不虞,自中流外,五胡扛鼎,七廟隳尊,滔天方駕,則民懷其舊德者矣。昔光武以數郡加名,元皇(案:晉元帝)以一州臨極,豈武、宣餘化,猶暢於琅邪,文、景垂仁,傳芳於南頓?所謂後乎天時,先諸人事者也。馳章獻號,高蓋成陰,星斗呈祥,金陵表慶。陶士行擁三州之旅,郢外以安;王茂弘爲分陝之計,江東可立。或高旌未拂,而遐心斯偃,回首朝陽,仰希乾棟,帝猶六讓不居,七辭而不免也。布帳綀帷,詳刑簡化,抑揚前軌,光啓中興。古首私家不蓄甲兵,大臣不爲威福,王之常制,以訓股肱。中宗失馭強臣,自亡齊斧,兩京胡羯,風埃相望。雖復《六月》之駕無聞,而《鴻雁》之歌方遠,享國無幾,哀哉!」

唐代某貴族「公子」與虞世南的對話:「公子曰:『中宗值天下崩離,創立江左,俱為中興之主,比於前代,功德云何?』先生曰:『元帝自居藩邸,少有令聞,及建策南渡,興亡繼絕,委任宏茂,撫綏新舊,故能嗣晉配天,良有以也。然仁恕為懷,剛毅情少,是以王敦縱暴,幾危社稷,蹙國舒禍,其周平之匹乎?』」

\subsection{建武}

\begin{longtable}{|>{\centering\scriptsize}m{2em}|>{\centering\scriptsize}m{1.3em}|>{\centering}m{8.8em}|}
  % \caption{秦王政}\
  \toprule
  \SimHei \normalsize 年数 & \SimHei \scriptsize 公元 & \SimHei 大事件 \tabularnewline
  % \midrule
  \endfirsthead
  \toprule
  \SimHei \normalsize 年数 & \SimHei \scriptsize 公元 & \SimHei 大事件 \tabularnewline
  \midrule
  \endhead
  \midrule
  元年 & 317 & \tabularnewline\hline
  二年 & 318 & \tabularnewline
  \bottomrule
\end{longtable}

\subsection{大兴}

\begin{longtable}{|>{\centering\scriptsize}m{2em}|>{\centering\scriptsize}m{1.3em}|>{\centering}m{8.8em}|}
  % \caption{秦王政}\
  \toprule
  \SimHei \normalsize 年数 & \SimHei \scriptsize 公元 & \SimHei 大事件 \tabularnewline
  % \midrule
  \endfirsthead
  \toprule
  \SimHei \normalsize 年数 & \SimHei \scriptsize 公元 & \SimHei 大事件 \tabularnewline
  \midrule
  \endhead
  \midrule
  元年 & 318 & \tabularnewline\hline
  二年 & 319 & \tabularnewline\hline
  三年 & 320 & \tabularnewline\hline
  四年 & 321 & \tabularnewline
  \bottomrule
\end{longtable}

\subsection{永昌}

\begin{longtable}{|>{\centering\scriptsize}m{2em}|>{\centering\scriptsize}m{1.3em}|>{\centering}m{8.8em}|}
  % \caption{秦王政}\
  \toprule
  \SimHei \normalsize 年数 & \SimHei \scriptsize 公元 & \SimHei 大事件 \tabularnewline
  % \midrule
  \endfirsthead
  \toprule
  \SimHei \normalsize 年数 & \SimHei \scriptsize 公元 & \SimHei 大事件 \tabularnewline
  \midrule
  \endhead
  \midrule
  元年 & 322 & \tabularnewline\hline
  二年 & 323 & \tabularnewline
  \bottomrule
\end{longtable}


%%% Local Variables:
%%% mode: latex
%%% TeX-engine: xetex
%%% TeX-master: "../Main"
%%% End:

%% -*- coding: utf-8 -*-
%% Time-stamp: <Chen Wang: 2019-12-18 13:28:48>

\section{明帝\tiny(322-325)}

\subsection{生平}

晉明帝司馬紹(299年-325年),字道畿,東晉的第二代皇帝,晉元帝司馬睿長子。母親是豫章郡君荀氏。在位不足三年,但在位期間平定了王敦之亂。

司馬紹自小聰慧,故此特別受父親司馬睿所寵愛。後於永嘉元年(307年)隨父親一同移鎮建業(後改建康,今江蘇南京市)。建興元年(313年),司馬睿升任左丞相,拜司馬紹為東中郎將,鎮守廣陵。316年,晉愍帝所在的長安被前趙攻陷,晉愍帝出降,西晋灭亡。有鉴于此,317年,司馬睿稱晉王,建元建武,並立司馬紹為晉王太子。318年,司馬睿即位称帝,改元太興,司馬紹被立為皇太子。

永昌元年(322年)發生王敦之亂,大將軍王敦領兵進攻建康並佔領石頭城,晉元帝派王導等人進攻石頭城但都被王敦擊敗,司馬紹於是打算率領將士與王敦決一死戰,即將出發時因遭太子中庶子溫嶠極力勸阻而沒有實行。隨後王敦自任丞相並掌握朝政,見司馬紹勇而有謀,而且朝野中亦有很高名望,於是打算誣陷他不孝而將他廢掉,但因溫嶠等大臣支持司馬紹,王敦終也不能廢掉司馬紹。

晉元帝因王敦之亂而憂憤成疾,於當年閏十一月己丑日(323年1月3日)病逝,司馬紹在次日繼位,为晉明帝,並由司空王導輔政。

王敦雖於永昌元年(322年)就回到武昌遙控朝廷,但因為圖謀篡位,於太寧元年(323年)暗示要朝廷徵召自己入朝,晉明帝於是以手詔徵召王敦。同年,王允之乘酒宴而知道王敦的圖謀,於是回京告訴其父王舒,王舒於是與王導一同報告晉明帝,得以早作防備。

次年,晉明帝既心知王敦意圖,於是騎馬微服去視察王敦於于湖的營地,但遭到軍人發現,並派五名騎兵追捕。晉明帝逃走時,用水浸濕所騎馬匹的粪便来使其降温,又拿出七寶鞭交給路旁賣食物的婆婆,並要她出示給追來的騎兵。晉明帝走後不久,追兵就來到,並詢問婆婆,婆婆於是取出七寶鞭,並稱那人已經走得很遠。騎兵們顧著傳玩七寶鞭而在那裏停留了很久,而且見馬糞已冷,以為追不及了,於是都沒有再追,晉明帝因此成功逃脫。

及後,晉明帝積極準備京師建康的防護,最終於當年成功擊敗王敦派來進攻的軍隊,平定了王敦之亂。王敦之亂後,晉明帝下令不再問罪於王敦一眾官屬,又分別以應詹為江州刺史、劉遐為徐州刺史、陶侃為荊州刺史、王舒為湘州刺史,重整各州形勢,消除王敦以王氏宗族各領諸州以凌弱帝室的失衡情形。

太寧三年閏八月戊子(325年10月18日),司馬紹病逝於東堂,年僅二十七歲。葬於武平陵,廟號肅祖。

司馬紹年少聰明,小時候便曾經與父親就「太陽與長安孰近」的問題作出不同答案的爭辯。長大後聰明有機斷,精於事理,於是能讓國家從王敦之亂的亂局回復平定。

司馬紹性至孝,有文武才略,敬重賢人,素好文辭,於是當時如王導、庾亮、溫嶠、桓彝、阮放等名臣都親待他。而因他習武藝,善於安撫將士,於是任太子時東宮聚集很多人,亦得遠近各人歸心。

王敦曾称呼晋明帝为:「黄鬚鲜卑奴」,這是因為其母建安郡君荀氏是燕代人,混雜了當地鮮卑人血統,故明帝可能也長得有一些像外族,鬚為黃色。

司馬紹任太子時,想修建池苑樓臺,但元帝不許。司馬紹於是命手下的武士在一晚之間修好太子西池。

司馬紹有寵妃宋褘,褘國色天香,善吹笛,乃石崇妾綠珠之女弟子,不久司馬紹病篤,群臣进谏,请出宋袆,最後宋褘被送給吏部尚书阮孚。

司馬紹在位時,曾問晉室得天下的事。王導於是告訴他司馬懿當日發動高平陵之變誅除曹爽,樹立蔣濟等與自己同心的大臣;又說道曹髦被司馬昭親信賈充所命的成濟弒殺一事。司馬紹聽後,將面龐伏在牀上,說:「若真的像你所說,晉室國祚又怎能夠長遠!」

唐代房玄齡於《晉書》的「史臣曰」評論說:「維揚作宇,憑帶洪流,楚江恆戰,方城對敵,不得不推誠將相,以總戎麾。樓船萬計,兵倍王室,處其利而無心者,周公其人也。威權外假,嫌隙內興,彼有順流之師,此無強籓之援。商逢九亂,堯止八音,明皇(案:晉明帝)負圖,屬在茲日。運龍韜於掌握,起天旆於江靡,燎其餘燼,有若秋原。去縗絰而踐戎場,斬鯨鯢而拜園闕。鎮削威權,州分江漢,覆車不踐,貽厥孫謀。其後七十餘年,終罹敬道之害。或曰:『興亡在運,非止上流。』豈創制不殊,而弘之者異也。」

唐代某貴族「公子」與虞世南的對話:「公子曰:『東晉自元帝已下,何為賢主?』先生曰:『晉自遷都江左,強臣擅命,(天子)垂拱南面,政非己出。王敦以磐石之宗,居上流之地,負才矜地,志懷沖問鼎,非明帝之雄斷,王導之忠誠,則晉祚其移於他族矣。若使降年永久,佐任群賢,因洛、澗之遺黎,乘劉、石之衰運,興復中原,不難圖也。』」


\subsection{太宁}

\begin{longtable}{|>{\centering\scriptsize}m{2em}|>{\centering\scriptsize}m{1.3em}|>{\centering}m{8.8em}|}
  % \caption{秦王政}\
  \toprule
  \SimHei \normalsize 年数 & \SimHei \scriptsize 公元 & \SimHei 大事件 \tabularnewline
  % \midrule
  \endfirsthead
  \toprule
  \SimHei \normalsize 年数 & \SimHei \scriptsize 公元 & \SimHei 大事件 \tabularnewline
  \midrule
  \endhead
  \midrule
  元年 & 323 & \tabularnewline\hline
  二年 & 324 & \tabularnewline\hline
  三年 & 325 & \tabularnewline\hline
  四年 & 326 & \tabularnewline
  \bottomrule
\end{longtable}


%%% Local Variables:
%%% mode: latex
%%% TeX-engine: xetex
%%% TeX-master: "../Main"
%%% End:

%% -*- coding: utf-8 -*-
%% Time-stamp: <Chen Wang: 2021-11-01 11:46:28>

\section{成帝司馬衍\tiny(325-342)}

\subsection{生平}

晉成帝司馬衍(321年12月或322年1月-342年7月26日),字世根,東晉的第三代皇帝,晉明帝之長子。晉成帝年幼即位,即位不久即遇上蘇峻之亂,成帝亦一度遭蘇峻叛軍劫持。成帝一朝軍政主要由外戚穎川庾氏把持,在庾亮的主導下還曾謀北伐,但因後趙強盛而遭到失敗。

太寧三年三月戊辰(325年4月1日),晉明帝立司馬衍為皇太子。同年閏八月戊子(325年10月18日),晉明帝去世,翌日五歲的晉成帝即位為帝。由於年幼,由母親皇太后庾文君臨朝稱制,由七位顧命大臣輔政,中書令庾亮以国舅身份主政。。

咸和二年(327年)年末,歷陽內史蘇峻與豫州刺史祖約叛亂,並在翌年率兵攻至建康,庾亮試圖抵抗但失敗,被逼出逃,晉成帝就與王導等眾官為蘇峻所挾持,宮中就遭到蘇峻軍搶掠和焚燒,太官也僅餘下數石米供成帝食用。咸和三年五月乙未,蘇峻強逼晉成帝遷居至石頭城一個倉庫中,成帝哭著登車出發,宮中人們亦都傷心痛哭。咸和四年(329年),以陶侃為首的軍隊平定蘇峻之亂,迎回成帝,因為宮殿遭戰火破壞,故修繕建平園作為宮室,至咸和七年(332年)新建的建康宮落城後才遷去新宮。

蘇峻之亂後,朝內就由王導專制,成帝對王導亦相當敬重,甚至屢幸王導宅第;庾亮則領豫州刺史出鎮蕪湖,主掌軍事,隨著陶侃去世,庾亮更兼荊江豫三州,轉鎮武昌,並著眼對後趙的北伐。咸康五年(339年),庾亮作出北伐部署,上奏移鎮襄陽石城,並且增兵長江、漢水流域以及淮泗壽陽地區要地,為一舉北伐作好準備,當時庾亮更派兵進攻巴郡,攻至江陽,俘獲後趙將領李閎及黃植。晉成帝下給群臣議論,上疏得王導支持,但郗鑒以資源不足為由反對。不過未等到允許,庾亮的行動就遭後趙以軍事行動作回應,派軍大舉南侵,庾亮所定的重鎮邾城更加被攻陷,庾亮北伐遂流產。

咸康二年(336年)晉成帝頒布壬辰詔書,禁止士族、官吏將私佔山川大澤;咸康七年(341年),又以土斷方式將自江北遷來的世族編入戶籍。

咸康八年(342年)7月23日,晉成帝患病,中書監庾冰為了留住穎川庾氏家族與皇帝的血緣親近,於是以國家外有強敵,宜立年長君主為由勸服成帝以弟弟琅琊王司馬岳為儲君。7月26日,晉成帝駕崩,年僅22歲,廟號顯宗。8月18日,葬於興平陵。

晉成帝年紀小小就很聰敏,有成年人的量度。蘇峻之亂前,庾亮以謀反罪誅殺了南頓王司馬宗,成帝一直不知,至亂事平定後才問及失蹤的司馬宗,庾亮答稱他因謀反而被誅,豈料成帝卻哭著說:「舅言人作賊,便殺之。人言舅作賊,復若何?」嚇得庾亮恐懼失色。至後來,庾懌送毒酒意圖毒殺江州刺史王允之,被揭發後成帝就怒道:「大舅已亂天下,小舅復欲爾邪?」庾懌被逼自殺。不過成帝年輕時被舅舅家族颖川庾氏勢力所限制,並不親政。至後來長大,卻留心事務,而且生活儉約,曾因射堂需耗用四十金而放棄建造。

在石頭城時右衞將軍劉超仍為成帝講授《孝經》及《論語》,但因劉超與鍾雅帶成帝逃出去的圖謀泄漏,二人遭蘇峻派任讓收捕殺害,期間晉成帝抱住任讓哭求:「還我侍中、右衞!」但任讓不聽小皇帝的命令,將二人殺了。蘇峻之亂被平定後,任讓原本因與陶侃有舊情而得免死,但成帝記恨他,任讓還是被誅殺。

\subsection{咸和}

\begin{longtable}{|>{\centering\scriptsize}m{2em}|>{\centering\scriptsize}m{1.3em}|>{\centering}m{8.8em}|}
  % \caption{秦王政}\
  \toprule
  \SimHei \normalsize 年数 & \SimHei \scriptsize 公元 & \SimHei 大事件 \tabularnewline
  % \midrule
  \endfirsthead
  \toprule
  \SimHei \normalsize 年数 & \SimHei \scriptsize 公元 & \SimHei 大事件 \tabularnewline
  \midrule
  \endhead
  \midrule
  元年 & 326 & \tabularnewline\hline
  二年 & 327 & \tabularnewline\hline
  三年 & 328 & \tabularnewline\hline
  四年 & 329 & \tabularnewline\hline
  五年 & 330 & \tabularnewline\hline
  六年 & 331 & \tabularnewline\hline
  七年 & 332 & \tabularnewline\hline
  八年 & 333 & \tabularnewline\hline
  九年 & 334 & \tabularnewline
  \bottomrule
\end{longtable}

\subsection{咸康}

\begin{longtable}{|>{\centering\scriptsize}m{2em}|>{\centering\scriptsize}m{1.3em}|>{\centering}m{8.8em}|}
  % \caption{秦王政}\
  \toprule
  \SimHei \normalsize 年数 & \SimHei \scriptsize 公元 & \SimHei 大事件 \tabularnewline
  % \midrule
  \endfirsthead
  \toprule
  \SimHei \normalsize 年数 & \SimHei \scriptsize 公元 & \SimHei 大事件 \tabularnewline
  \midrule
  \endhead
  \midrule
  元年 & 335 & \tabularnewline\hline
  二年 & 336 & \tabularnewline\hline
  三年 & 337 & \tabularnewline\hline
  四年 & 338 & \tabularnewline\hline
  五年 & 339 & \tabularnewline\hline
  六年 & 340 & \tabularnewline\hline
  七年 & 341 & \tabularnewline\hline
  八年 & 342 & \tabularnewline
  \bottomrule
\end{longtable}


%%% Local Variables:
%%% mode: latex
%%% TeX-engine: xetex
%%% TeX-master: "../Main"
%%% End:

%% -*- coding: utf-8 -*-
%% Time-stamp: <Chen Wang: 2019-12-18 13:33:01>

\section{康帝\tiny(342-344)}

\subsection{生平}

晉康帝司馬岳(322年-344年11月17日),字世同,東晉的第四代皇帝。晉康帝是晉明帝之子,母庾文君,是晉成帝的同母弟。

晉康帝於326年被封為吳王,後封琅琊王,342年晉成帝死後,由於權臣庾冰與庾翼力主之故,晉康帝才得以用兄終弟及的方式繼承帝位,但不久便於建元二年344年患病駕崩,時年二十三歲,葬於崇平陵,其子司馬聃繼位。

\subsection{建元}

\begin{longtable}{|>{\centering\scriptsize}m{2em}|>{\centering\scriptsize}m{1.3em}|>{\centering}m{8.8em}|}
  % \caption{秦王政}\
  \toprule
  \SimHei \normalsize 年数 & \SimHei \scriptsize 公元 & \SimHei 大事件 \tabularnewline
  % \midrule
  \endfirsthead
  \toprule
  \SimHei \normalsize 年数 & \SimHei \scriptsize 公元 & \SimHei 大事件 \tabularnewline
  \midrule
  \endhead
  \midrule
  元年 & 343 & \tabularnewline\hline
  二年 & 344 & \tabularnewline
  \bottomrule
\end{longtable}


%%% Local Variables:
%%% mode: latex
%%% TeX-engine: xetex
%%% TeX-master: "../Main"
%%% End:

%% -*- coding: utf-8 -*-
%% Time-stamp: <Chen Wang: 2021-11-01 11:46:39>

\section{穆帝司馬聃\tiny(344-361)}

\subsection{生平}

晉穆帝司馬聃(343年-361年7月10日),字彭子,東晉第五代皇帝,廟號孝宗。晉穆帝是晉康帝之子,母褚蒜子。

建元二年(344年),由於晉康帝僅22歲便駕崩,晉穆帝即位,時年兩歲;由於年幼而由褚太后掌政,並由何充輔政。何充過世後改由蔡謨與司馬昱輔政。晉穆帝在位期間東晉雖然北伐失敗,但是由於桓溫消滅了在四川立國的成漢,並且於永和十二年(356年)三月奪回洛陽,雖然不久就因為糧運不繼而撤退,東晉的版圖仍然有所擴大。昇平元年 ( 357年 ) 晉穆帝行冠禮後褚太后歸政,晉穆帝開始親政。

361年7月10日,晉穆帝過世,得年十八歲,9月9日,葬永平陵。由晉成帝長子琅邪王司馬丕繼位。


\subsection{永和}

\begin{longtable}{|>{\centering\scriptsize}m{2em}|>{\centering\scriptsize}m{1.3em}|>{\centering}m{8.8em}|}
  % \caption{秦王政}\
  \toprule
  \SimHei \normalsize 年数 & \SimHei \scriptsize 公元 & \SimHei 大事件 \tabularnewline
  % \midrule
  \endfirsthead
  \toprule
  \SimHei \normalsize 年数 & \SimHei \scriptsize 公元 & \SimHei 大事件 \tabularnewline
  \midrule
  \endhead
  \midrule
  元年 & 345 & \tabularnewline\hline
  二年 & 346 & \tabularnewline\hline
  三年 & 347 & \tabularnewline\hline
  四年 & 348 & \tabularnewline\hline
  五年 & 349 & \tabularnewline\hline
  六年 & 350 & \tabularnewline\hline
  七年 & 351 & \tabularnewline\hline
  八年 & 352 & \tabularnewline\hline
  九年 & 353 & \tabularnewline\hline
  十年 & 354 & \tabularnewline\hline
  十一年 & 355 & \tabularnewline\hline
  十二年 & 356 & \tabularnewline
  \bottomrule
\end{longtable}

\subsection{升平}

\begin{longtable}{|>{\centering\scriptsize}m{2em}|>{\centering\scriptsize}m{1.3em}|>{\centering}m{8.8em}|}
  % \caption{秦王政}\
  \toprule
  \SimHei \normalsize 年数 & \SimHei \scriptsize 公元 & \SimHei 大事件 \tabularnewline
  % \midrule
  \endfirsthead
  \toprule
  \SimHei \normalsize 年数 & \SimHei \scriptsize 公元 & \SimHei 大事件 \tabularnewline
  \midrule
  \endhead
  \midrule
  元年 & 357 & \tabularnewline\hline
  二年 & 358 & \tabularnewline\hline
  三年 & 359 & \tabularnewline\hline
  四年 & 360 & \tabularnewline\hline
  五年 & 361 & \tabularnewline
  \bottomrule
\end{longtable}


%%% Local Variables:
%%% mode: latex
%%% TeX-engine: xetex
%%% TeX-master: "../Main"
%%% End:

%% -*- coding: utf-8 -*-
%% Time-stamp: <Chen Wang: 2019-12-18 13:34:36>

\section{哀帝\tiny(361-365)}

\subsection{生平}

晉哀帝司馬丕(341年-365年3月30日),字千齡,為東晉的第六代皇帝,晉成帝之子,晉穆帝之堂兄。

342年封为琅琊王,345年拜散骑常侍。356年加中军将军,359年十二月除骠骑将军。晉哀帝本應繼晉成帝之位即位,但是由於權臣庾冰的意見而無法即位;司馬丕於361年在晉穆帝死後即位,改元隆和,但是大將桓溫當國,晉哀帝形同傀儡。

晉哀帝好重佛法,又好黄老道,即位不久就迷上了長生術,按照道士傳授的長生法,斷榖、服丹藥,結果服藥後藥性大發而不能聽政,遂由褚太后再次臨朝。興甯三年(365年),晉哀帝因藥物中毒死於太極殿,年僅二十五歲,葬於安平陵。

\subsection{隆和}

\begin{longtable}{|>{\centering\scriptsize}m{2em}|>{\centering\scriptsize}m{1.3em}|>{\centering}m{8.8em}|}
  % \caption{秦王政}\
  \toprule
  \SimHei \normalsize 年数 & \SimHei \scriptsize 公元 & \SimHei 大事件 \tabularnewline
  % \midrule
  \endfirsthead
  \toprule
  \SimHei \normalsize 年数 & \SimHei \scriptsize 公元 & \SimHei 大事件 \tabularnewline
  \midrule
  \endhead
  \midrule
  元年 & 362 & \tabularnewline\hline
  二年 & 363 & \tabularnewline
  \bottomrule
\end{longtable}

\subsection{兴宁}

\begin{longtable}{|>{\centering\scriptsize}m{2em}|>{\centering\scriptsize}m{1.3em}|>{\centering}m{8.8em}|}
  % \caption{秦王政}\
  \toprule
  \SimHei \normalsize 年数 & \SimHei \scriptsize 公元 & \SimHei 大事件 \tabularnewline
  % \midrule
  \endfirsthead
  \toprule
  \SimHei \normalsize 年数 & \SimHei \scriptsize 公元 & \SimHei 大事件 \tabularnewline
  \midrule
  \endhead
  \midrule
  元年 & 363 & \tabularnewline\hline
  二年 & 364 & \tabularnewline\hline
  三年 & 365 & \tabularnewline
  \bottomrule
\end{longtable}


%%% Local Variables:
%%% mode: latex
%%% TeX-engine: xetex
%%% TeX-master: "../Main"
%%% End:

%% -*- coding: utf-8 -*-
%% Time-stamp: <Chen Wang: 2019-12-18 13:35:41>

\section{废帝\tiny(365-371)}

\subsection{生平}

司馬奕(342年-386年),字延齡,東晉的第七代皇帝,晉成帝之子、晉哀帝之弟。晉哀帝死後於365年即帝位,史稱「廢帝」。

342年六月封为东海王,352年拜散骑常侍镇军将军。360年升车骑将军,361年改封琅琊王。362年七月為侍中骠骑大将军开府仪同三司。司馬奕即位之時,桓溫掌握朝政,桓的幕府參軍郗超建議桓溫效仿伊尹、霍光,廢除天子以立威信,但司馬奕本身並無過失可言,桓溫便指司馬奕陽痿不能人道,指田、孟二妃所生三皇子为司马奕的男宠相龙、计好及朱灵宝所生,於太和六年(371年)廢司馬奕為東海王,之後再貶為海西縣公,遷居吳縣西柴里,并将田、孟二妃及三皇子处死。

司馬奕遭廢位后心灰意冷,又怕再遭禍端,便苟且偷生。之後司馬奕更是沉迷於酒色,成日过着荒淫的生活,甚至生了孩子也不养,桓溫及之後继位的晋孝武帝也因此对他不再防范。

司马奕於386年過世,享年四十五歲,他亦是東晉較為長壽的皇帝。

\subsection{太和}

\begin{longtable}{|>{\centering\scriptsize}m{2em}|>{\centering\scriptsize}m{1.3em}|>{\centering}m{8.8em}|}
  % \caption{秦王政}\
  \toprule
  \SimHei \normalsize 年数 & \SimHei \scriptsize 公元 & \SimHei 大事件 \tabularnewline
  % \midrule
  \endfirsthead
  \toprule
  \SimHei \normalsize 年数 & \SimHei \scriptsize 公元 & \SimHei 大事件 \tabularnewline
  \midrule
  \endhead
  \midrule
  元年 & 366 & \tabularnewline\hline
  二年 & 367 & \tabularnewline\hline
  三年 & 368 & \tabularnewline\hline
  四年 & 369 & \tabularnewline\hline
  五年 & 370 & \tabularnewline\hline
  六年 & 371 & \tabularnewline
  \bottomrule
\end{longtable}



%%% Local Variables:
%%% mode: latex
%%% TeX-engine: xetex
%%% TeX-master: "../Main"
%%% End:

%% -*- coding: utf-8 -*-
%% Time-stamp: <Chen Wang: 2019-12-18 13:37:03>

\section{简文帝\tiny(371-372)}

\subsection{生平}

晉簡文帝司馬\xpinyin*{昱}(320年-372年9月12日),字道萬。東晉第八代皇帝。东晋开国皇帝晋元帝少子,母郑阿春。自永和元年(345年)開始一直以會稽王輔政,掌握朝廷的實權,但其時權臣桓溫的勢力亦一直增強。52歲時於太和六年十一月己酉(372年1月6日)被桓溫擁立为帝,改年号为咸安。次年七月己未(372年9月12日)病逝。在位期間只有250日,期間桓温擅权。

永昌元年(322年),晉元帝下詔封司馬昱為琅邪王,作為自己入繼大宗後父親爵位的繼嗣。咸和二年(327年)因其母喪,請求服重而改封會稽王,官拜散騎常侍。咸和九年(334年)轉任右將軍,加侍中。咸康六年(340年)進撫軍將軍,領祕書監。建元元年(343年)加領太常。

永和元年(345年),因著上一年晉康帝去世,年幼的晉穆帝登位,崇德太后褚蒜子抱晉穆帝臨朝。當時輔政的驃騎將軍何充希望由太后父親褚裒入朝輔政但對方辭讓,當時司馬昱聲望高,故升司馬昱為撫軍大將軍,錄尚書六條事,與何充輔政。同年,荊州刺史庾翼去世,死前請求以其子庾爰之接代其位,何充則屬意徐州刺史桓溫取代庾氏掌握荊州。司馬昱倚重的名士劉惔熟悉桓溫,指桓溫雖然有才幹但也極有野心,不能讓他居於荊州這個控制長江上游的「形勝之地」,建議由司馬昱親自外鎮荊州,或由劉惔自己任荊州刺史。司馬昱沒有聽從劉惔的建言,桓溫任荊州刺史,獲得了日後奪權掌政的資本。

永和二年(346年),何充去世,左光祿大夫蔡謨加領司徒兼錄尚書六條事,與司馬昱一同輔政,司馬昱總理萬機,其實是東晉朝廷的決策者。何充兼領的揚州刺史此時出缺,褚裒舉薦了名士殷浩,殷浩辭讓並寫信給司馬昱說明理由,但司馬昱勸他出仕,四個月後殷浩終於出仕。次年,桓溫平滅成漢,建立大功,威望和勢力都大為提升,同時也引來朝內對其的忌憚。司馬昱決定以殷浩抗衡桓溫,後來後趙國內大亂,授殷浩以北伐的重任。然而殷浩北伐失敗,桓溫在永和十年借朝野對殷浩北伐失敗的不滿廢掉殷浩,司馬昱亦無力抗衡桓溫高漲的力量,令得桓溫在朝中獨大。

永和八年(352年),詔升司馬昱為司徒,司馬昱辭讓。興寧三年(365年),琅邪王司馬奕即位為晉廢帝,琅邪國無嗣,晉廢帝封司馬昱爲琅邪王,改以司馬昱子司馬昌明為會稽王。司馬昱辭讓,所以仍以會稽王號封琅邪王。次年,詔進司馬昱為丞相、錄尚書事、入朝不趨,贊拜不名,劍履上殿、賜羽葆,鼓吹及班劍六十人,司馬昱又辭讓。

太和四年(369年),桓溫第三次北伐大敗於前燕和前秦聯軍,豫州刺史袁真不堪被桓溫所誣要負上北伐失敗的責任而叛變。司馬昱在涂中與桓溫會面,商討隨後的行動,以桓溫子桓熙為豫州刺史。

太和六年十一月己酉日(372年1月6日),大司馬桓溫廢黜晉廢帝為東海王,率百官到會稽王府奉迎司馬昱,司馬昱即日即位為帝,是為晉簡文帝,改元咸安。桓溫及後就寫了講辭,打算向司馬昱陳述自己廢立的本意,但司馬昱每接見他都不停流淚,如此令桓溫恐懼,居然不能說一句話。

司馬昱的哥哥武陵王司馬晞有軍事才幹,被桓溫所忌。廢立不久,桓溫就誣陷司馬晞謀反將其免官,及後更逼令新蔡王司馬晃自誣與司馬晞及庾倩等人謀反,以求翦滅陳郡殷氏和穎川庾氏在朝中的勢力。隨後桓溫指示御史中丞司馬恬奏請司馬昱依律法處死司馬晞,司馬昱不肯,下令再作詳細議論。桓溫再次上奏求誅司馬晞,言詞十分嚴厲急切,司馬昱於是手詔給桓溫,寫道:「若果晉室國祚長久,那麼你就應該依從早前的詔命從事;如晉室大勢已去,那你就讓我退位讓賢吧。」桓溫看後流汗色變,不敢再逼,只上奏廢掉司馬晞和他三名兒子,並流放其家屬。

桓溫既行廢立,亦誅滅了與司馬皇室親密的殷氏和庾氏,威勢達至高峯。不過,桓溫在當時仍受制於以王坦之為首的太原王氏及謝安為首的陳郡謝氏世族力量,有篡位心而不能得逞。而司馬昱雖天子,其實如同傀儡皇帝,未敢多言,更怕又被桓溫所廢。當時司馬昱見熒惑入太微垣,因晉廢帝被廢時亦有同樣天象,故此十分不安,甚至對桓溫親信也是自己昔日僚屬的郗超問桓溫會否再行廢立之事。郗超斷言桓溫不會這樣作,司馬昱仍十分感慨,並詠庾闡之詩:「志士痛朝危,忠臣哀主辱。」司馬昱憂憤而得病,在咸安二年七月甲寅日(372年9月7日)因病急召桓溫入朝輔政,桓溫數度辭讓,司馬昱於是在己未日(9月12日)立兒子司馬昌明為太子。同日在東堂去世,享年五十三歲。臨終前,司馬昱寫了遺詔,要桓溫依周公先例居攝,更寫:「少子可輔者輔之,如不可,君自取之。」面對桓溫的野心,此舉幾近讓國。王坦之在司馬昱面前親手撕毀遺詔。司馬昱說:「晉室天下,只是因好運而意外獲得,你又對這個決定有甚麼不滿呢!」王坦之卻說:「晉室天下,是晉宣帝和晉元帝建立的,怎由得陛下你獨斷獨行!」司馬昱於是命王坦之改寫遺詔,寫道:「家國事一稟大司馬,如諸葛武侯、王丞相故事。」桓溫其實亦希望司馬昱臨終禪讓帝位給自己,又或者讓他像周公般居攝行事,王坦之改寫後的遺詔令桓溫大失所望。

司馬昱葬高平陵,廟號為太宗,諡為簡文皇帝。

简文帝外表清秀俊朗,擅长玄學清谈。尽管有文人雅士的风度,恬靜豁達,但政治手腕可说非常平庸,無濟世大略。谢安曾尖刻地评论:“比晉惠帝(以「何不食肉糜」著名的白痴皇帝)惟有清谈差胜耳!”謝靈運亦以他比作周赧王及漢獻帝等亡國之君。

司馬昱崇尚清談,長期坐著的胡床上即使積了灰塵也不清理。一次司馬昱發現有老鼠走過胡床的痕跡,覺得是好事。參軍見到有老鼠在白天走了出來,以手板將老鼠殺掉,司馬昱很不高興。當時門下的部屬就檢舉殺鼠的人,以圖取悅司馬昱,司馬昱卻說:「老鼠被殺,到現在還不能忘記;而現在又因老鼠而影響到他人,豈不是更不應該嗎?」可見其在醉心玄學之餘亦聰明有仁心。

司馬昱輔政時,一些政事拖了整年才得批准,桓溫覺得太慢,常常勸告司馬昱。但司馬昱卻說:「一日萬機,怎能快呀。」

王濛昔日請求當東陽太守,司馬昱不答應。及至王濛病重臨終,司馬昱就悲哀地說:「我將有負於仲祖(王濛字)呀!」下令命其為東陽太守。王濛說:「人們說會稽王痴心,真是痴心呀。」

司馬昱看見稻田,不知是甚麼,於是問左右是甚麼草,左右於是答那是稻。司馬昱事後三日沒有出外,說:「哪有依賴其結果而不知其根本。」

《晉書》載一次桓溫與司馬晞及司馬昱同車遊板桥,桓溫特意命人吹響號角,令馬匹受驚狂奔,藉此看兩人的反應。司馬晞當時大驚而想下車,而司馬昱就處之泰然。《世說新語》亦有類似記載。不過有認為司馬晞既然受桓溫所忌,不應有如此反應,這是對日後成為皇帝的司馬昱的溢美之作。

\subsection{咸安}

\begin{longtable}{|>{\centering\scriptsize}m{2em}|>{\centering\scriptsize}m{1.3em}|>{\centering}m{8.8em}|}
  % \caption{秦王政}\
  \toprule
  \SimHei \normalsize 年数 & \SimHei \scriptsize 公元 & \SimHei 大事件 \tabularnewline
  % \midrule
  \endfirsthead
  \toprule
  \SimHei \normalsize 年数 & \SimHei \scriptsize 公元 & \SimHei 大事件 \tabularnewline
  \midrule
  \endhead
  \midrule
  元年 & 371 & \tabularnewline\hline
  二年 & 372 & \tabularnewline
  \bottomrule
\end{longtable}



%%% Local Variables:
%%% mode: latex
%%% TeX-engine: xetex
%%% TeX-master: "../Main"
%%% End:

%% -*- coding: utf-8 -*-
%% Time-stamp: <Chen Wang: 2021-11-01 11:47:57>

\section{孝武帝司马曜\tiny(372-396)}

\subsection{生平}

晋孝武帝司马曜(362年-396年11月6日),字昌明,东晋的第九个皇帝,在位时间是372年至396年。他是晋简文帝的第三个儿子,晋安帝和晋恭帝的父亲,母李陵容。

晋孝武帝四岁时被封为会稽王,372年9月12日被立为太子,同日晋简文帝逝,继位時年僅十一岁。次年年号为宁康,由太后摄政。

14岁时(376年)开始亲政,改年号为太元。当年他改革税收,放弃以田地多少来收税的方法,改为王公以下每人收米三斛,在役的人不交税。此外他在位期间大力加强皇帝的权力和地位,史載他“威權己出”,扭轉了東晉自晉明帝死後皇权旁落的局面。

383年前秦进攻东晋,试图消灭长年偏安的东晋,结果在淝水之战中,晋军大胜。

384年後,晉孝武帝趁著前秦崩解的契機北伐,陸續收復了黃河以南的所有領土(包含河南洛陽及山東半島),甚至劉牢之一度佔領河北鄴城。這使得390年代的東晉版圖,達到了自東晉開始以來的最大值。但是連年征戰,遽增的兵役賦稅使人民痛苦難當,既疲又怨。

晉孝武帝即位初期由於稅賦改革與謝安當國,被稱為東晉後期的復興;但是謝安死後司馬道子當國,以及晋孝武帝北伐成功后开始嗜酒,“醒日既少”,連帶導致“刑網峻急,風俗奢宕”的不良政風。

396年11月6日,晉孝武帝由於对他当时宠信的张贵人开玩笑说:“你已经快要三十歲了,按年龄应该要被废弃了”,導致当晚张贵人一怒之下在清暑殿杀了他,享年34歲。11月30日,葬于今江苏南京的隆平陵。

孝武帝自幼年聰穎,他十歲時父親簡文帝崩逝,但他到了下午仍不去父親遺體旁哭喪,侍從勸告他應按照禮節哭喪,他卻回答說:「哀痛時就是哭喪的最好時機,哪裡需要被常規禮節束縛呢?」宰相謝安對他的清談義理頗為讚嘆,認為他所掌握的精微義理,不下於其父簡文帝。孝武帝親政後將治國大權收歸己手,很有君主的才幹器量。但他年長後沉溺於酒色之中,將政務細節交給位居宰相的弟弟司馬道子,常與道子一同飲酒酣歌。他晚年更通宵飲酒而睡到大白天,因此少有白日清醒的時刻。周遭缺乏剛正的大臣規勸,因此沒法改正嗜酒缺失。

唐代房玄齡於《晉書》評論說:「太宗晏駕,寧康(按:以年號代稱晉孝武帝)纂業,天誘其衷,姦臣自隕,于時西踰劍岫而跨靈山,北振長河而臨清、洛;荊、吳戰旅,嘯吒成雲;名賢間出,舊德斯在:謝安可以鎮雅俗,彪之足以正紀綱,桓沖之夙夜王家,謝玄之善料軍事。于時上天乃眷,強氐自泯。五尺童子,振袂臨江,思所以挂旆天山,封泥函谷;而條綱弗垂,威恩罕樹,道子荒乎朝政,國寶彙以小人,拜授之榮,初非天旨,鬻刑之貨,自走權門,毒賦年滋,愁民歲廣。是以聞人、許榮馳書詣闕,烈宗知其抗直,而惡聞逆耳,肆一醉於崇朝,飛千觴於長夜。雖復『昌明』表夢,安聽神言?而金行穨弛,抑亦人事,語曰『大國之政未陵夷,小邦之亂已傾覆』也。屬苻堅百六之秋,棄肥水之眾,帝號為 『武』,不亦優哉!」

唐代某貴族「公子」與虞世南的對話:「公子曰:『(東晉)中興之政,咸歸大臣,唯孝武為君,威福自己,外摧疆寇,人安吏肅。比于明帝,功業何如?』先生(虞世南)曰:『孝武克夷外難,乃謝安之力也,非人主之功。至于委任會稽,棟梁已撓,殷、王作鎮,亂階斯起,昌明之讖,乃驗于茲。加以末年沉晏,卒致傾覆,比蹤前哲(按:前哲指晉明帝),其何遠乎?』」

\subsection{宁康}

\begin{longtable}{|>{\centering\scriptsize}m{2em}|>{\centering\scriptsize}m{1.3em}|>{\centering}m{8.8em}|}
  % \caption{秦王政}\
  \toprule
  \SimHei \normalsize 年数 & \SimHei \scriptsize 公元 & \SimHei 大事件 \tabularnewline
  % \midrule
  \endfirsthead
  \toprule
  \SimHei \normalsize 年数 & \SimHei \scriptsize 公元 & \SimHei 大事件 \tabularnewline
  \midrule
  \endhead
  \midrule
  元年 & 373 & \tabularnewline\hline
  二年 & 374 & \tabularnewline\hline
  三年 & 375 & \tabularnewline
  \bottomrule
\end{longtable}

\subsection{太元}

\begin{longtable}{|>{\centering\scriptsize}m{2em}|>{\centering\scriptsize}m{1.3em}|>{\centering}m{8.8em}|}
  % \caption{秦王政}\
  \toprule
  \SimHei \normalsize 年数 & \SimHei \scriptsize 公元 & \SimHei 大事件 \tabularnewline
  % \midrule
  \endfirsthead
  \toprule
  \SimHei \normalsize 年数 & \SimHei \scriptsize 公元 & \SimHei 大事件 \tabularnewline
  \midrule
  \endhead
  \midrule
  元年 & 376 & \tabularnewline\hline
  二年 & 377 & \tabularnewline\hline
  三年 & 378 & \tabularnewline\hline
  四年 & 379 & \tabularnewline\hline
  五年 & 380 & \tabularnewline\hline
  六年 & 381 & \tabularnewline\hline
  七年 & 382 & \tabularnewline\hline
  八年 & 383 & \tabularnewline\hline
  九年 & 384 & \tabularnewline\hline
  十年 & 385 & \tabularnewline\hline
  十一年 & 386 & \tabularnewline\hline
  十二年 & 387 & \tabularnewline\hline
  十三年 & 388 & \tabularnewline\hline
  十四年 & 389 & \tabularnewline\hline
  十五年 & 390 & \tabularnewline\hline
  十六年 & 391 & \tabularnewline\hline
  十七年 & 392 & \tabularnewline\hline
  十八年 & 393 & \tabularnewline\hline
  十九年 & 394 & \tabularnewline\hline
  二十年 & 395 & \tabularnewline\hline
  二一年 & 396 & \tabularnewline
  \bottomrule
\end{longtable}


%%% Local Variables:
%%% mode: latex
%%% TeX-engine: xetex
%%% TeX-master: "../Main"
%%% End:

%% -*- coding: utf-8 -*-
%% Time-stamp: <Chen Wang: 2021-11-01 11:48:02>

\section{安帝司马德宗\tiny(397-418)}

\subsection{生平}

晋安帝司马德宗(382年-419年1月28日),字德宗,东晋的第十位皇帝。晋孝武帝司马曜的长子,母亲是陈归女。晉安帝由於痴愚而無能力掌握國政,在位廿二年間朝權都旁落在臣下之中,國內內亂頻仍,期間甚至發生了桓玄篡位的事件。最後東晉國祚及國力在北府將領劉裕的主持下獲得恢復,但亦為劉裕奠下篡位的基礎,安帝自己亦因劉裕欲篡而遇害。

司马德宗於太元十二年八月辛巳(387年9月16日)被立为皇太子。太元二十一年九月庚申(396年11月6日)孝武帝被張貴人所弒,次日安帝正式继位。

安帝本愚,從小到大连话都不太会说,就连冬夏的区别都认不出来,因此朝廷的权力实际上完全由当朝大臣掌握,沒有一道詔旨,一個行動是出自安帝自己的意願。安帝初期朝廷政策主要由以太傅攝政旳会稽王司马道子主持,王恭之亂後則由會稽世子司馬元顯掌握。元興元年(402年),司馬元顯讓安帝改元「元興」,預備出兵討伐桓玄,但其年為桓玄所敗,朝政亦從此轉歸桓玄掌握。桓玄先廢元興年號,改元「大亨」,後專制朝廷,準備篡位。翌年十二月壬辰(404年1月1日),桓玄篡位,建立桓楚政權,廢安帝為平固王,並在次日被送至尋陽。

桓玄篡位後僅兩月,北府將領劉裕等人於京口、廣陵成功舉兵,並合軍進攻建康,桓玄軍隊不敵並撤到尋陽,隨即又挾持晉安帝退至江陵。桓玄在江陵試圖重整旗鼓,於五月癸酉(404年6月10日)逼令安帝隨軍之下於崢嶸洲與劉毅等軍決戰。不過,桓玄再敗,只有率敗軍及安帝退還江陵,隨後在往蜀地的路上被殺。留在江陵的安帝在荊州別駕王康產及南郡太守王騰之的支援下於江陵復位。然而,不久桓振率領桓楚殘部進襲江陷,安帝再度被俘。義熙元年(405年)正月,晉軍收復江陵,安帝再度復位,改元「義熙」,隨後獲迎回建康。時朝廷就由劉裕為首,至義熙四年(408年)接替去世的王謐領揚州刺史、錄尚書事時完全掌握朝權。

安帝即位後,先有王恭兩度舉兵,荊州各地亦陷入割據狀態,朝廷影響力量在司馬道子父子主政時一度僅及三吳。內亂不休的晉廷因而無力守衞北方領土,至隆安三年(399年),後秦佔領洛陽及河南地區,南燕則佔領山東半島,建都廣固城。東晉再次喪失了淮水以北的大部分領土。接著發生的孫恩盧循之亂以及桓玄篡位事件,亦讓東晉未能有暇收復失地,蜀地甚至在討伐桓玄期間由譙縱建立的譙蜀控制。劉裕主政之下,東晉於義熙五年(409年)出兵進攻南燕,並翌年滅南燕,收復齊地;雖然盧循乘劉裕北伐的機會襲擊建康,但劉裕適時回軍,成功防禦京師,並於稍後擊潰盧循主力,盧循及其勢力於義熙七年(411年)完全肅清。劉裕又在義熙九年(413年)派將領朱齡石收復蜀地。另一方面,他又於義熙八年(412年)先後消滅了當日與他一同起兵討伐桓玄的劉毅及諸葛長民,又於義熙十一年(415年)消滅了宗室司馬休之,除去了政敵。義熙十二年(416年),劉裕奉琅邪王司馬德文的名義北伐後秦,於翌年成功滅掉後秦,不但收服了河南及洛陽失地,更重奪關中。雖然關中於義熙十四年(418年)劉裕班師後為夏國所佔領,但劉裕地位已經穩固,受封十郡宋公,亦圖篡位,因為「昌明之後有二帝」的預言,故晉安帝在劉裕授意下於十二月戊寅日(419年1月28日)被中書侍郎王韶之殺害,其弟司馬德文被擁立,以應預言。

\subsection{隆安}

\begin{longtable}{|>{\centering\scriptsize}m{2em}|>{\centering\scriptsize}m{1.3em}|>{\centering}m{8.8em}|}
  % \caption{秦王政}\
  \toprule
  \SimHei \normalsize 年数 & \SimHei \scriptsize 公元 & \SimHei 大事件 \tabularnewline
  % \midrule
  \endfirsthead
  \toprule
  \SimHei \normalsize 年数 & \SimHei \scriptsize 公元 & \SimHei 大事件 \tabularnewline
  \midrule
  \endhead
  \midrule
  元年 & 397 & \tabularnewline\hline
  二年 & 398 & \tabularnewline\hline
  三年 & 399 & \tabularnewline\hline
  四年 & 400 & \tabularnewline\hline
  五年 & 401 & \tabularnewline
  \bottomrule
\end{longtable}

\subsection{元兴}

\begin{longtable}{|>{\centering\scriptsize}m{2em}|>{\centering\scriptsize}m{1.3em}|>{\centering}m{8.8em}|}
  % \caption{秦王政}\
  \toprule
  \SimHei \normalsize 年数 & \SimHei \scriptsize 公元 & \SimHei 大事件 \tabularnewline
  % \midrule
  \endfirsthead
  \toprule
  \SimHei \normalsize 年数 & \SimHei \scriptsize 公元 & \SimHei 大事件 \tabularnewline
  \midrule
  \endhead
  \midrule
  元年 & 402 & \tabularnewline\hline
  二年 & 403 & \tabularnewline\hline
  三年 & 404 & \tabularnewline
  \bottomrule
\end{longtable}

\subsection{大亨}

\begin{longtable}{|>{\centering\scriptsize}m{2em}|>{\centering\scriptsize}m{1.3em}|>{\centering}m{8.8em}|}
  % \caption{秦王政}\
  \toprule
  \SimHei \normalsize 年数 & \SimHei \scriptsize 公元 & \SimHei 大事件 \tabularnewline
  % \midrule
  \endfirsthead
  \toprule
  \SimHei \normalsize 年数 & \SimHei \scriptsize 公元 & \SimHei 大事件 \tabularnewline
  \midrule
  \endhead
  \midrule
  元年 & 402 & \tabularnewline
  \bottomrule
\end{longtable}

\subsection{义熙}

\begin{longtable}{|>{\centering\scriptsize}m{2em}|>{\centering\scriptsize}m{1.3em}|>{\centering}m{8.8em}|}
  % \caption{秦王政}\
  \toprule
  \SimHei \normalsize 年数 & \SimHei \scriptsize 公元 & \SimHei 大事件 \tabularnewline
  % \midrule
  \endfirsthead
  \toprule
  \SimHei \normalsize 年数 & \SimHei \scriptsize 公元 & \SimHei 大事件 \tabularnewline
  \midrule
  \endhead
  \midrule
  元年 & 405 & \tabularnewline\hline
  二年 & 406 & \tabularnewline\hline
  三年 & 407 & \tabularnewline\hline
  四年 & 408 & \tabularnewline\hline
  五年 & 409 & \tabularnewline\hline
  六年 & 410 & \tabularnewline\hline
  七年 & 411 & \tabularnewline\hline
  八年 & 412 & \tabularnewline\hline
  九年 & 413 & \tabularnewline\hline
  十年 & 414 & \tabularnewline\hline
  十一年 & 415 & \tabularnewline\hline
  十二年 & 416 & \tabularnewline\hline
  十三年 & 417 & \tabularnewline\hline
  十四年 & 418 & \tabularnewline
  \bottomrule
\end{longtable}


%%% Local Variables:
%%% mode: latex
%%% TeX-engine: xetex
%%% TeX-master: "../Main"
%%% End:

%% -*- coding: utf-8 -*-
%% Time-stamp: <Chen Wang: 2021-11-01 11:48:09>

\section{恭帝司馬德文\tiny(419-420)}

\subsection{生平}

晉恭帝司馬德文(386年-421年11月2日),字德文,河內溫縣(今河南溫縣)人。東晉的末代皇帝。為晉孝武帝之子,晉安帝之胞弟,母親是淑媛陳歸女。初封琅邪王,後在桓玄篡位後長期侍奉晉安帝左右。晉安帝死後被劉裕以遺詔立為皇帝,但其時劉裕已經完全掌握東晉朝政,司馬德文僅為傀儡而已。劉裕篡晉後為零陵王,次年遇害。

太元十七年十一月庚寅日(392年12月27日)受封為琅邪王,後又拜中軍將軍、散騎常侍。隆安二年(398年)轉衞將軍、開府儀同三司。隆安三年(399年)遷侍中,領司徒、錄尚書六條事。元興元年(402年),桓玄擊敗司馬道子父子,掌握朝政,改以司馬德文為太宰。

元興二年十二月壬辰日(404年1月1日),桓玄篡位稱帝,貶晉安帝為平固王,司馬德文亦因而降封「石陽縣公」。不久桓玄遷安帝至尋陽(今江西九江市),司馬德文亦跟隨。元興三年(404年),劉裕起兵討伐桓玄,桓玄兵敗逃到尋陽,得郭昶之給予器具及士兵後再逼晉安帝與其同至江陵(今湖北江陵);及至桓玄敗死於逃往益州途中,荊州別駕王康產及南郡太守王騰之迎晉安帝至南郡府舍時,司馬德文亦緊隨。然而,桓振等桓楚餘眾趁劉毅等軍未及趕至江陵,乘虛來襲,最終江陵城陷,王康產及王騰之遇害,桓振亦騎馬揮戈直入,問桓玄子桓昇下落,並在得知其死訊後大怒,指責他們屠殺桓氏。當時司馬德文辯護道:「這又豈會是我們兄弟的意思!」在桓謙苦勸下,桓振才沒有加害安帝。隨後桓振繼續控制江陵,並以司馬德文為徐州刺史,繼續對抗由劉毅所統領的討伐軍隊。

義熙元年(405年),劉毅等軍攻下江陵,司馬德文亦與安帝一同在何無忌護送下返回建康。回建康後,司馬德文遷大司馬,並於義熙四年(408年)加領司徒。

義熙十二年(416年),劉裕預備北伐後秦,時劉裕圖以晉室名聲安撫北方人民,故想奉司馬德文之名北伐,司馬德文因而上書出兵,以修謁晉室山陵,最終劉裕就與司馬德文一同率兵出發。義熙十三年(417年),劉裕成功滅亡後秦,同年年末班師東歸,司馬德文亦跟隨,至次年(418年)夏季,劉裕到達彭城(今江蘇徐州市),司馬德文先回建康。不久,劉裕受九錫,封宋王。

劉裕當時指派了中書侍郎王韶之圖謀殺害晉安帝,立司馬德文為帝,以應「昌明之後尚有二帝」的預言。不過因司馬德文無論飲食還是睡覺都和晉安帝在一起,王韶之無法下手。可是司馬德文卻於當年年末患病,離開了安帝,王韶之趁機會下手,將安帝殺死。劉裕則假稱遺詔,以司馬德文繼位。

元熙二年六月壬戌(420年7月5日),劉裕入朝,傅亮暗示司馬德文禪讓帝位給劉裕,並將禪讓詔書的草稿上呈,要他抄寫。司馬德文欣然接受,執筆抄寫,並說:「桓玄篡位那時,晉室經已失去天下了,又因劉公延長了國祚,至今已將近二十年了;今日作這種事,是心甘情願的。」兩日後,司馬德文退居琅邪王府,百官向晉帝告別,東晉至此滅亡。又三日後,劉裕正式登位,並奉司馬德文為零陵王,讓他遷至秣陵縣(今江蘇江寧縣)的舊縣治作為其府第,正朔、車駕、衣服等都依晉朝規格,正如昔日晉篡魏的先例,並命劉遵派兵守衞。

及後劉裕就有殺害司馬德文的意圖,最初就命前琅邪國郎中令張偉拿毒酒去殺司馬德文,但張偉就嘆道:「要毒殺主君去讓自己活下去,不如死了!」竟在路上喝下毒酒自盡。司馬德文自己也十分害怕會遭毒手,於是起居飲食都由王妃褚靈媛打點,食物也在自己面前烹煮,令加害者無從下手。不過,褚靈媛兄褚秀之及褚淡之都忠於劉裕,一直以來司馬德文生下的男嬰都被二人借故害死。至永初二年(421年)九月,劉裕即命褚淡之及褚叔度去見褚靈媛,乘機支開她到另一個房間。及後士兵就翻過牆進入府內,逼司馬德文服食毒藥。但司馬德文不肯,更說:「佛教所稱,自殺的人都不能輪迴再生為人。」士兵於是用被褥將其悶死,享年三十六歲。司馬德文以晉禮下葬於沖平陵,諡恭皇帝。

史載,晉安帝司馬德宗從小到大都不會說話,甚至連冬夏的氣侯轉變也不能分辨。而司馬德文一直侍奉左右,打理他的生活起居,以恭敬謹慎而聞名,亦得當時人們稱許。

司馬德文信奉佛教,曾下令打造了一個高一丈六寸的黃金佛像,並親身到瓦官寺迎其上位。其死前所言亦証其篤信佛教。

據說司馬德文年幼時頗為殘忍急躁,在琅邪國時更曾命擅長射箭的人射擊馬匹作為娛樂。當時有人說:「馬是國姓,而你自己就去殺牠,這是很不祥的事呀!」司馬德文明白此言,亦甚為後悔。

\subsection{元熙}

\begin{longtable}{|>{\centering\scriptsize}m{2em}|>{\centering\scriptsize}m{1.3em}|>{\centering}m{8.8em}|}
  % \caption{秦王政}\
  \toprule
  \SimHei \normalsize 年数 & \SimHei \scriptsize 公元 & \SimHei 大事件 \tabularnewline
  % \midrule
  \endfirsthead
  \toprule
  \SimHei \normalsize 年数 & \SimHei \scriptsize 公元 & \SimHei 大事件 \tabularnewline
  \midrule
  \endhead
  \midrule
  元年 & 419 & \tabularnewline\hline
  二年 & 429 & \tabularnewline
  \bottomrule
\end{longtable}


%%% Local Variables:
%%% mode: latex
%%% TeX-engine: xetex
%%% TeX-master: "../Main"
%%% End:

%% -*- coding: utf-8 -*-
%% Time-stamp: <Chen Wang: 2019-12-18 13:45:57>

\section{桓楚\tiny(403-405)}

\subsection{简介}

桓楚,中國東晉時期由將領桓玄所建立的一個短期政權,存續期間為403年至404年。

東晉晉安帝元興二年(403年),控制東晉中央政府的楚王桓玄篡奪政權。十一月二十一日(陽曆為403年12月20日),安帝獻上玉璽,禪位於桓玄。十二月三日(陽曆為404年1月1日),桓玄正式稱帝,國號楚,改元永始。為與其餘國號為楚的政權區分,故史家稱桓玄建立者為桓楚。

名義上,桓楚於建立後直接繼承東晉的領土,但實際上其勢力範圍僅及江陵(今湖北江陵)以東的長江中下游一帶。404年,以劉裕為首的數名將領,起兵勤王,楚軍不敵,桓玄退出建康(今江蘇南京),並挾持安帝西逃至江陵。同年稍後,桓玄敗死,其堂弟桓謙將國璽奉還安帝,桓楚亡。

桓楚亡後,桓氏家族仍不斷在長江中游一帶興兵與東晉政府軍對抗,直到數年後才被消滅。

\subsection{桓温生平}

桓溫(312年-373年),字元子,譙國龍亢(今安徽懷遠縣龍亢鎮)人。東晉重要將領及權臣、軍事家,譙國桓氏代表人物。官至大司馬、錄尚書事。宣城內史桓彝長子,因領兵消滅成漢而聲名大盛,又曾三次領導北伐,掌握朝政並曾操縱廢立,更有意奪取帝位,但終因最後一次北伐大敗而令聲望受損,受制於朝中王氏和謝氏勢力而未能如願。死前欲得九錫亦因謝安等人借故拖延,直至去世時也未能實現。因桓溫獲賜諡號宣武,故《世說新語》稱其為「桓宣武」。其子桓玄後來一度篡奪東晉帝位而建立桓楚,追尊桓溫為「楚宣武帝」。

桓溫出生後還未夠一歲,就被溫嶠稱許,父親桓彝於是以「溫」作為桓溫的名字。年少與殷浩齊名。咸和三年(328年),桓彝在蘇峻之亂中被蘇峻將領韓晃所殺,當時桓彝所駐涇縣的縣令江播亦有協助。桓溫當時極度痛心,且一直想著為父報仇。桓溫十八歲時,江播已死,江播的三名兒子則在守喪,但他們仍有防備桓溫,將刀刃藏在杖中。桓溫則以弔唁為名,得以進入三人守喪的廬屋內,立殺江彪,及後追殺其餘兩人。

後桓溫娶南康長公主,拜任駙馬都尉,並承襲父爵萬寧男。咸康元年(335年)任琅琊太守。後升輔國將軍。建元元年(343年),桓溫配合征西將軍庾翼的北伐行動,假節任前鋒小督,進據臨淮。三個月後,桓溫升為都督青徐兗三州諸軍事、徐州刺史。永和元年(345年),庾翼病死,臨終前上表求以兒子庾爰之接掌荊州,作為自己繼任者。但輔政的何充則推薦桓溫,桓溫於是於當年獲升任安西將軍,持節都督荊司雍益梁寧六州諸軍事、領護南蠻校尉、荊州刺史,代替庾氏鎮守荊州。

永和二年(346年),桓溫趁成漢內部不穩,汉主李勢荒淫無道令國家衰弱,決心征伐。當年十一月就上表朝廷,並立刻率領益州刺史周撫、南郡太守譙王司馬無忌和建武將軍袁喬等進攻成漢。當時朝廷內部多數都認為蜀地險要偏遠,而且桓溫兵少而深入蜀境,都為他擔憂。次年三月,桓溫進兵至彭模,並聽從袁喬全軍進擊,只帶三日糧食直攻成都的計謀,只留參軍孫盛和周楚以弱兵在彭模守輜重,桓溫則親自率兵直攻成都。

及後李勢所派抵抗桓溫的將領李福嘗試襲擊彭模,但在孫盛等人奮戰之下被擊退。而桓溫進兵時遇到守將李權,三戰三勝,並一直逼近成都,李勢於是在笮橋率所有兵力抵抗桓溫。桓溫前鋒初時陷於不利,參軍龔護戰死,箭矢更射到桓溫所騎馬匹以前,兵眾十分恐懼而要撤退。但當時戰鼓鼓手卻錯誤擂鼓命兵眾進攻,袁喬於是拔劍領兵與成漢軍激戰,終大敗對方,桓溫於是進攻至成都城下並燒了城門。成漢人見此,再無鬥志,李勢亦乘夜棄城逃至葭萌。不久,李勢決定投降,桓溫受降並遷李勢及成漢宗室到建康。

桓溫平蜀後留駐成都一個月,在當地舉任賢能,表彰美善。又以成漢舊臣譙獻之、常璩等人作為自己參佐,成功安撫當地人民。桓溫即將返回荊州時,隗文、鄧定等人在蜀地叛亂,桓溫與袁喬、周撫等各自領兵討伐,都將對方擊破。後桓溫領兵還鎮江陵(今湖北江陵縣)。

永和四年(348年),朝廷以桓溫平蜀的功勳,升桓溫為征西大將軍、開府儀同三司,封臨賀郡公。

桓溫雖然滅掉成漢,聲名大振,但亦因此令朝廷忌憚他功高不能控制,輔政的會稽王司馬昱於是擢升揚州刺史殷浩處理朝政,以抗衡桓溫日漸增長的勢力。永和五年(349年),後趙君主石虎死,北方因石虎諸子爭位而再度混亂。桓溫見此,即進據安陸,並上疏請求北伐,但久久都沒有回音。至永和六年(350年),朝廷以殷浩為中軍將軍、都督五州諸軍事,委以北伐重任,以此抗衡桓溫。桓溫亦知朝廷以殷浩抗衡自己,感到很不忿,但桓溫亦知殷浩為人,並不憂心。當時桓溫除本官所都督六州外亦加都督交、廣二州共八州,此八州士兵和資源調配都不由朝廷掌握。故此當時桓溫屢次上表請求北伐不果後,再次上表請求北伐並立刻自行率四、五萬兵沿長江東進武昌,便令當時人心驚駭,殷浩亦曾打算辭官迴避,而司馬昱亦要寫信勸止桓溫,終令桓溫退兵回荊州。朝廷及後讓桓溫進位太尉,但桓溫辭讓不拜。

隨後兩年,殷浩都有率兵進行北伐,但沒有成果,反倒屢次戰敗,軍需物資更被略奪殆盡,令朝野怨恨。永和十年(354年),桓溫趁機上奏列舉殷浩罪行,逼使朝廷廢殷浩為庶人。桓溫開始掌權。

桓溫分別於354年、356年及369年發動北伐北方十六國的戰役。但除了第二次北伐成功收復洛陽,其餘兩次皆被擊退,成效不大。

永和十年(354年)二月,桓溫奏免殷浩後不久便發動第一次北伐,親率步騎四萬餘人進攻武關,水軍直指南鄉(今河南淅川县滔河乡),命司馬勳從子午道(秦嶺棧道,通向漢中)進攻以關中地區為根據地的前秦。桓溫後率軍在藍田(今陝西藍田縣)擊破氐族苻健軍隊數萬人,進駐長安東面的霸上,逼使前秦君主苻健以數千人退守長安小城。當地民眾很多都以牛和酒款待桓溫軍,而老人亦感觸得哭泣著說:「沒想過今天還能看到官軍!」然而,桓溫未有聽從順陽太守薛珍所言追逼長安,反待敵自潰。六月,苻雄率所有軍力在白鹿原擊敗桓溫。九月,因桓溫本想收割作軍糧的麥子被秦軍搶先收割,並堅壁清野,令晉軍糧秣不繼,被迫徙關中三千多戶一同撤返江陵。撤軍時更遭前秦軍攻擊,死亡失蹤者數以萬計。

桓溫在北伐期間,王猛曾經前來拜見,並大談當世之事,並署任王猛為軍諮祭酒。桓溫撤退時曾請王猛一同南行,並任命他為高官督護,但王猛沒有跟隨。

永和十二年(356年)三月,桓溫打算發動第二次北伐,請求移都洛陽,修復園陵。雖然桓溫上奏十多次都不被允許,但朝廷卻拜桓溫為征討大都督,督司、冀二州諸軍事,主征討之事。七月,桓溫從江陵起兵發動第二次北伐。八月,桓溫進軍洛陽以南的伊水,當時羌人姚襄正在圍攻洛陽,見桓溫攻來,於是撤去洛陽的圍城軍隊去抵禦桓溫。桓溫終在伊水大破姚襄,姚襄逃走。及後,據有洛陽的周成獻城向桓溫投降,桓溫於是成功收復故都洛陽。桓溫及後拜謁各皇陵及修復其中已被毀壞者。桓溫留穎川太守毛穆之、河南太守戴施等守護洛陽,自己則領三千多家歸降的人民南遷至長江、漢水一帶,返回荊州。升平四年(360年),桓溫進封南郡公。

隆和元年(362年),前燕將領呂護進攻洛陽,桓溫派庾希及鄧遐助陳祐守城。桓溫亦上奏請求晉室遷都洛陽,又建議南遷的士族返鄉,朝廷畏懼桓溫,不敢有異議;但士族們卻已安於南方,根本不願北返。在此憂慮之時,揚州刺史王述認為桓溫只是以遷都之名威壓朝廷,並非真心想還都洛陽,只要表示順從便可,毋須實行。詔書下達後,晉室始終沒有還都洛陽。

興寧元年(363年),桓溫獲加授侍中、大司馬、都督中外諸軍事、錄尚書事、假黃鉞。次年,桓溫率水軍移守合肥,朝廷改以桓溫為揚州牧、錄尚書事,並兩度徵桓溫入朝。桓溫在第二次徵召時才入朝,行至赭圻時停止並留駐當地。當時前燕又再進攻洛陽,守將陳祐留兵出奔。司馬昱知道後,於是於興寧三年(365年)與桓溫商議征討之事,並讓桓溫移鎮姑孰。但同年因晉哀帝死,征伐之事就暫停。同年,前燕攻陷洛陽。

太和四年(369年),桓溫為了樹立更高的威望,發動第三次北伐,並請與徐、兗二州刺史郗愔、江州刺史桓沖及豫州刺史袁真等一同討伐前燕。而桓溫其實一直都希望控制郗愔在京口(今江蘇鎮江)所統領的精兵,郗愔子郗超時為桓溫參軍,便修改了父親寫給桓溫的書信,變成以老病辭任二州刺史職位,並勸桓溫接掌自己所領軍隊。桓溫看信後十分高興,桓溫亦因而得以自領徐、兗二州刺史。及後,桓溫正式起兵,率五萬人從姑孰出發北伐。

桓溫前進至金鄉,因大旱引水讓水軍舟船得以進入黃河。當時郗超認為如此難以運輸補給,建議直攻前燕都城鄴城,或者停駐黃河、濟水一帶管理漕運,積聚足夠的物資待次年夏天才進攻。但桓溫都沒有聽從。桓溫派軍先後攻敗湖陸守軍、在黃墟迎擊的慕容厲和林渚的傅顏,前燕於是向前秦求救,桓溫亦前進至枋頭。桓溫及後沒有再進逼前燕,反希望以持久戰坐取全勝。九月,因袁真無法開通石門以通水路運輸,而前燕亦斷了桓溫糧道,桓溫見戰事不利而糧食又已盡,更聽闻前秦援兵将至,於是燒船、弃輜重鎧甲,自陆道撤退。途中遭前燕騎兵追擊,損失三萬餘人;更被前秦軍在譙郡擊敗,於是這次北伐以大败告終。

桓溫北伐後,命人修築廣陵城池並移鎮當地。又因北伐失敗而感到十分羞恥,並將罪責推給未能開通石門水道的袁真。袁真不甘心被桓溫誣以罪責,而上奏桓溫罪狀又不果,於是以壽春(今安徽寿县)叛歸前燕。同年袁真逝世,太和六年(371年),桓溫率军擊敗前秦援軍,並攻陷寿春,俘斩袁真子袁瑾。

桓溫雖然自從363年獲錄尚書事開始就干預朝政,而且自負有才能,早就有異志,所以才發起北伐希望先建功勳,然後領受九錫並進圖篡位。但因第三次北伐遭前燕及前秦擊敗,聲名和實力都減弱,圖謀不成。壽春被桓溫攻下後,參軍郗超知道桓溫的心意,於是建議廢立之計而加強桓溫聲威。桓溫亦早有此謀,於是在當年便廢晉廢帝司馬奕為東海王,改立司馬昱為帝,即晉簡文帝,自己以大司馬專權。

桓溫隨後就因厭惡殷氏和庾氏強盛,又忌憚時任太宰的武陵王司馬晞的軍事才幹,於是先上奏彈劾司馬晞「聚納輕剽,苞藏亡命」,並誣司馬晞將成叛亂禍根,成功將司馬晞及其子司馬綜免官。及後又派弟弟桓祕逼迫新蔡王司馬晃誣稱自己與司馬晞、司馬綜、著作郎殷涓、太宰長史庾倩、太宰掾曹秀、散騎常侍庾柔等人謀反。桓溫下令將他們收付廷尉,晉簡文帝只有哭泣。後在桓溫意願下,廷尉上奏要賜死司馬晞,簡文帝不願,下詔要再作議論。桓溫於是上書請誅司馬晞,言辭十分嚴厲急切。簡文帝見此,只得寫書給桓溫:「若果晉室國祚長久,那麼你就應該依從早前的詔命從事;如晉室大勢已去,那你就讓我退位讓賢吧。」桓溫見後,流汗色變,而司馬晞亦只被廢為庶人,未被誅殺。但庾柔、殷涓等人都被族誅。桓溫此後,威勢極盛,連謝安見他亦對他遙拜,更以君臣稱作二人關係,足見當時桓溫權勢已經比皇室更高,如同君主。

次年,簡文帝死,死前遺詔由桓溫輔政,如諸葛亮、王導的先例。當時群臣都因桓溫權勢而不敢以皇太子司馬曜為帝,反等待桓溫的決定。尚書僕射王彪之則以太子即位之正當性釋除群臣疑慮,迎司馬曜繼位為晉孝武帝。桓溫原本寄望簡文帝會將帝位禪讓給自己,或讓自己倣效周公為君主主理朝政。如今兩者皆否,大失所望,因而十分怨憤,更懷疑這是王坦之、謝安做的。不久,朝廷下詔桓溫入朝輔政,並加前部羽葆鼓吹,武賁六十人,桓溫辭讓。寧康元年(373年),桓溫入朝拜山陵,朝廷詔謝安及王坦之到新亭迎接桓溫,百官拜於道側。三月,桓溫患病,停建康十四日後退還姑孰。當時桓溫表示想受九錫,多番催促,而王彪之及謝安見桓溫病重,則借修改袁宏所寫的錫文暗中拖延。七月己亥日(8月18日),桓溫逝世,享年六十二歲,至此錫文仍未完成。朝廷追贈丞相,諡號為宣武。喪禮依司馬孚、霍光的儀式,葬姑孰青山。

桓溫少時有豪邁風氣。姿貌甚偉,面有七星。

桓溫伐蜀時經過長江三峽,部隊中有人捉了一隻小猿,母猿則在岸邊哀號,一直跟了桓溫的船隊行了百多里。及更跳了上船,但隨即就死了。及後有人剖開其腹,見其腸臟都斷成很多小段。桓溫知道後大怒,貶黜了捉了小猿的人。又一次,桓溫與眾人一同吃飯,一名參軍用筷子夾蒸薤菜時,薤菜秥在一起夾不起,而其他同桌的人又不幫助,看見參軍夾著不放的模樣更笑起來。桓溫見此,說:「一同吃飯仍不相助,何況遇到危難時呢?」於是將他們免官。都見到桓溫雖然是極具野心的將領,但亦能在小事上顯出顧及人情的性格。

桓溫性格儉樸,每次宴會只吃七個奠柈茶果。

桓溫每逢大事無靜氣。既要行伊霍之事,又慮太后意異(褚蒜子時為崇德太后),等待時「悚動流汗,見於顏色」(《晉書·后妃傳》)。太后禮佛畢,從容作答,桓溫又大喜過望。

桓溫因晉成帝姊南康公主而大貴,及掌軍權,盡廢明帝後人。簡文帝說,找郎婿找得王敦、桓溫輩,稍得志便要廢立人。

桓溫待南康公主寡仁義。行軍司馬謝奕為人狂放,醉後窮追不捨,桓溫只得到處逃竄。南康公主見之,訴曰:“非是狂司馬,安得見郎君。”

桓溫與弟桓沖志趣不合,情誼頗深。溫死,沖為國家故,立其幼子玄。

桓溫年少家貧,與人玩摴蒱曾大敗,要找袁耽求取勝之法。

有人曾問桓溫有關謝安及王坦之的優劣,桓溫正想說,但就後悔說:「卿喜傳人語,不能復語卿。」

桓溫第三次北伐時,行軍至金城,看到自己任琅邪太守時所種的柳樹已經十分粗大,慨嘆:「木猶如此,人何以堪。」於是扶著枝幹,拿著枝條,流下眼淚。

桓溫曾躺臥著說:「作此寂寂,將為文、景所笑。」然後起坐,又說:「既不能流芳後世,亦不足復遺臭萬歲邪?」。又曾經在經過王敦墓,說:「可人,可人!」

桓溫自以雄姿風氣是司馬懿、劉琨之流,若有人將他比作王敦就會很不高興。第一次北伐後,在北方獲得了一個老婢,是昔日劉琨的女伎。老婢見桓溫後就掩面哭泣,桓溫追問,老婢則答:「你很像劉司空大人(劉琨)。」桓溫聽後十分高興,便去整理衣冠,及後又召來老婢來問。老婢則說:「脣很像,但可惜太薄;鬚很像,但可惜是赤色;體形很像,但可惜太矮;聲音很像,但可惜不雄壯。」桓溫聽後脫下冠帶去睡,不高興了數日。

桓溫一次乘下雪打獵,先見王濛、劉惔等人。劉惔見他一身戎裝,問:「老賊欲持此何作?」桓溫說:「我若不為此,卿輩那得坐談?」

桓温曾讀皇甫謐的《高士傳》,讀到於陵仲子時就擲去書本,說:「誰能這樣苛刻對待自己!」

《晉書》評:「桓溫挺雄豪之逸氣,韞文武之奇才,見賞通人,夙標令譽。時既豺狼孔熾,疆場多虞,受寄扞城,用恢威略,乃踰越險阻,戡定岷峨,獨克之功,有可稱矣。及觀兵洛汭,修復五陵,引斾秦郊,威懷三輔,雖未能梟除凶逆,亦足以宣暢王靈。既而總戎馬之權,居形勝之地,自謂英猷不世,勳績冠時。挾震主之威,蓄無君之志,企景文而概息,想處仲而思齊,睥睨漢廷,窺覦周鼎。復欲立奇功於趙魏,允歸望於天人;然後步驟前王,憲章虞夏。逮乎石門路阻,襄邑兵摧,懟謀略之乖違,恥師徒之撓敗,遷怒於朝廷,委罪於偏裨,廢主以立威,殺人以逞欲,曾弗知寶命不可以求得,神器不可以力征。豈不悖哉!豈不悖哉!斯實斧鉞之所宜加,人神之所同棄。然猶存極光寵,沒享哀榮,是知朝政之無章,主威之不立也。」

庾翼:「桓溫少有雄略,願陛下勿以常人遇之,常壻畜之,宜委以方召之任,託其弘濟艱難之勳。」

何充:「桓溫英略過人,有文武識度。」

孫綽:「高爽邁出。」

劉惔:「鬚如反猬毛,眼如紫石稜,自是孫仲謀、司馬宣王一流人。」

\subsection{桓玄\tiny(403-404)}

\subsubsection{生平}

桓玄(369年-404年6月19日),字敬道,一名靈寶,譙國龍亢(今安徽懷遠)人,譙國桓氏代表人物,東晉名將桓溫之子,東晉末期桓楚政權建立者。曾消滅殷仲堪和楊佺期佔據荊江廣大土地,後更消滅了掌握朝政的司馬道子父子,掌握朝權。次年桓玄就篡位建立桓楚,但三個月後劉裕就舉義兵反抗桓玄,桓玄不敵而逃奔江陵重整軍力,但後再遭西討的義軍擊敗。試圖入蜀途中遇上護送毛璠靈柩的費恬等人,遭益州督護馮遷殺害。因曾襲父親「南郡公」之爵,故世稱「桓南郡」。

桓玄自幼為桓溫所喜愛。寧康元年(373年),桓溫去世,遺命其弟桓沖統率其軍隊,並接替他任揚州刺史,並以時年五歲的桓玄承襲其封爵南郡公。兩年後,桓玄的服喪期滿,桓沖亦離任揚州刺史,揚州文武官員與桓沖告別,桓沖摸著桓玄的頭說:「這是你家的舊官屬呀。」桓玄聽後就掩面哭泣,眾人都對這反應感到詫異。

桓玄長大後,相貌奇偉,神態爽朗,博通藝術,亦善寫文章。他對自己的才能和門第頗為自負,總認為自己是英雄豪傑,然而由於其父桓溫晚年有篡位的跡象,所以朝廷一直對他深懷戒心而不敢任用。直至太元十六年(391年),二十三歲的桓玄才被任命為太子洗馬。幾年後出京任義興(今江蘇宜興)太守,但還是頗覺不得志,曾感歎:「父為九州伯,兒為五湖長!」於是就棄官回到其封國南郡。

桓玄住在南郡的治所,也就是荊州的治所江陵,優游無事,荊州刺史殷仲堪本來對他十分敬憚,而桓玄因著父叔長年治理荊州的威望而專橫荊州,士民畏懼他更過於殷仲堪,殷仲堪因而與其深交。桓玄也打算借助其軍力,故此取悅他。

隆安元年(397年),尚書僕射王國寶、建威將軍王緒倚仗當權的會稽王司馬道子,因畏懼青兗二州刺史王恭,圖謀削弱各方鎮,桓玄知道王恭面對王國寶亂政有憂國之言,故此勸說殷仲堪起兵討伐王國寶,並派人勸說王恭,推王恭為盟主。當時,殷仲堪個人擔憂沒有孝武帝的支持,自己被群眾認為能力未達一州方伯的情況下會被王國寶等人利用,終令他被調離荊州。桓玄亦利用這個擔憂勸說殷仲堪,但殷仲堪始終遲疑。不過,當時王恭原來已決定主動起兵,並聯結殷仲堪,殷仲堪此時得報,於是答應了響應王恭。不久朝廷畏懼,故殺王國寶、王緒以息事寧人,王恭亦罷兵。然而,始終殷仲堪與桓玄始終沒有進行實質的軍事行動。

王恭舉兵以後,司馬道子憂慮王恭和殷仲堪的威脅,於是引司馬尚之和司馬休之為心腹。隆安二年(398年),因著桓玄請求朝廷讓他任廣州刺史,而司馬道子亦忌憚他,不想他繼續長據荊州,於是下詔以他督交廣二州、建威將軍、平越中郎將、廣州刺史、假節。桓玄受命但沒有到廣州上任。同時司馬道子聽從司馬尚之多樹外藩的建議,不料卻因削奪了豫州刺史庾楷都督地區而令其勸王恭再度舉兵,王恭遂於當年聯結桓玄、殷仲堪等舉兵討伐司馬尚之兄弟,桓、殷亦奉其為盟主。殷仲堪認為王恭這次肯定成功,於是積極參戰,更分五千兵給桓玄,緊隨擔任前鋒的南郡相楊佺期順江南下。楊、桓二人到湓口時,亦為討伐對象的江州刺史王愉逃奔臨川,但被桓玄派兵追獲。及後雖然庾楷大敗給司馬尚之,前來投奔桓玄,但桓玄也於白石大敗朝廷軍隊。及後雖然王恭敗死,但桓玄和楊佺期進至石頭,令司馬元顯回防京師,並命丹楊尹王愷守石頭城。不過,因為剛剛背叛王恭的劉牢之率北府軍入援京師,桓玄和楊佺期因畏懼而撤回蔡州,與朝廷軍對峙。

當時司馬道子打算利誘桓玄和楊佺期,令二人倒伐攻擊殷仲堪,於是以桓玄為江州刺史,楊佺期為雍州刺史,而殷仲堪就被貶廣州刺史。此舉卻令殷仲堪大怒,並命桓玄和楊佺期率兵進攻建康。不過桓玄卻對任命十分高興,打算接受,卻猶豫不決。當時殷仲堪從堂弟殷遹口中又聽聞楊佺期也決定受命,於是開始撤軍。隨著殷仲堪撤退,楊佺期部將劉系亦先行撤退,桓玄等大懼,又狼狽西退,直至尋陽(今江西九江市)追上殷仲堪。殷仲堪既失荊州刺史,倚仗桓玄為援;而桓玄本身亦要借助殷仲堪的兵力,故此據勢相結,殷仲堪與楊佺期因著其家世聲望,共推桓玄為盟主,皆不受朝命。朝廷見此大加恐懼,唯有下詔安撫,並讓殷仲堪復任荊州刺史,請求和解。眾人於是受命返回駐地。

早在桓玄在江陵橫行時,殷仲堪親黨就已勸殷仲堪殺死桓玄,但沒得聽從。桓玄自被推為盟主後,就更加矜侉倨傲,而楊佺期就被桓玄以寒門相待,然而出身弘農楊氏的楊佺期卻自以其族是華夏貴冑,一直都認為江東其他士族根本比不上他家,於是對桓玄十分不滿,更打算襲殺桓玄,可是因殷仲堪顧忌桓玄死後無法控制楊佺期兄弟才阻止。當時桓玄亦知楊佺期想殺死自己,於是有了消滅楊佺期的意圖,更屯駐夏口,並以始安太守卞範之為謀主。

隆安三年(399年)請求擴大其轄區,而司馬元顯亦想以此離間桓玄與殷、楊二人的關係,故此加桓玄都督荊州長沙郡、衡陽郡、湘東郡及零陵郡四郡諸軍事,並改以桓玄兄桓偉代楊佺期兄楊廣為南蠻校尉。此舉觸怒了楊佺期兄弟,楊佺期更以支援後秦圍攻的洛陽為名起兵,但皆被殷仲堪阻止。當年荊州有大水,殷仲堪開倉賑濟災民,桓玄就乘此機會起兵,亦以救援洛陽為名。當時桓玄寫信給殷仲堪,稱他要消滅楊佺期,並命殷仲堪收殺楊廣,否則會進攻江陵。桓玄並襲取殷仲堪在巴陵的積糧,又向路經夏口的梁州刺史郭銓假稱收到朝廷下令命郭銓為自己前鋒以討楊佺期,故此授江夏兵予他,命他督諸軍前進。

當時桓玄密報桓偉作為內應,但桓偉遑恐,更向殷仲堪自首,於是被對方擄為人質,並命其寫信給桓玄,在信中苦勸桓玄罷兵,不過桓玄不為所動,自度桓偉必因殷仲堪優柔寡斷,常慮兒子的性格而無危險。殷仲堪亦派了殷遹率七千水軍至西江口,桓玄派郭銓和苻宏擊敗他;及後殷仲堪又派楊廣及殷道護進攻,桓玄再在楊口擊敗他們,直逼至離江陵二十里的零口,震動江陵。後楊佺期自襄陽來攻,桓玄一度退後避其鋒銳,但終大敗楊佺期,及後由部將馮該並追獲及殺掉他。殷仲堪出奔,又被馮該追獲,及後被桓玄逼令自殺。

桓玄年末消滅了楊佺期和殷仲堪,於是在次年(400年)向朝廷求領荊江二州刺史。朝廷下詔以桓玄都督荊司雍秦梁益寧七州諸軍事、後將軍、荊州刺史、假節;另以桓偉為江州刺史。但桓玄堅持要由自己領江州刺史,朝廷唯有讓桓玄加都督江州及揚州豫州共八郡諸軍事,領江州刺史;桓玄又以桓偉為雍州刺史,朝廷礙於當時孫恩叛亂惡化,不能違抗。桓玄於是趁機在荊州任用腹心,訓練兵馬,並屢次請求討伐孫恩,但都被朝廷阻止。

隆安五年(401年),孫恩循海道進攻京口,逼近建康,桓玄聲稱勤王起兵,實質想乘亂而入,司馬元顯於是在孫恩北走遠離京師後下詔命桓玄解嚴。不過,桓玄當時完全控制了其轄區,不但作出調桓偉為江州、鎮守夏口,又以司馬刁暢督八郡、鎮守襄陽,桓振、皇甫敷、馮該等駐湓口等軍事調動,更建立了武寧郡和綏安郡分別安置遷徙的蠻族以及招集的流民。朝廷曾下詔徵廣州刺史刁逵和豫章太守郭昶之,亦被桓玄所留。

元興元年(402年),司馬元顯下詔討伐桓玄,在京的堂兄桓石生密報桓玄。桓玄既封鎖長江漕運,令東土饑乏,又因孫恩之亂未平,故認為司馬元顯無力討伐,於是一直在荊州等待時機,蓄勢待發。然而收到桓石生的通報後,桓玄甚懼,打算堅守江陵。不過卞範之卻勸桓玄出兵東下,以桓玄的威名和軍力,令其土崩瓦解;反不應主動示弱於人。桓玄於是留桓偉守江陵,親自率兵東下。桓玄初仍憂抗拒朝命,手下士兵都不會為他所用,然而過了潯陽仍未見朝廷軍隊,於是十分高興,士氣亦上升,移檄上奏司馬元顯之罪。桓玄到姑孰時,派馮該等擊敗並俘獲豫州刺史司馬尚之,並奪取了歷陽(今安徽和縣)。當時司馬元顯因畏懼,登船而未敢出兵,而劉牢之因擔憂擊敗桓玄後會不容於司馬元顯,竟與其手下北府軍向桓玄投降。桓玄逼近建康,司馬元顯試圖守城但潰敗。桓玄入京後,稱詔解嚴,並以自己總掌國事,受命侍中、都督中外諸軍事、丞相、錄尚書事、揚州牧,領徐州刺史,加假黃鉞、羽葆鼓吹、班劍二十人。

桓玄又列會稽王司馬道子及司馬元顯的罪惡,流放司馬道子到安成郡,數月後桓玄更派人殺死司馬道子;又殺司馬元顯、庾楷、司馬尚之和司馬道子的太傅府中屬吏。桓玄又圖除去劉牢之,先命他為會稽太守,令其遠離京口。劉牢之意圖反叛但得不到北府軍將領支持,於是北逃廣陵投靠廣陵相高雅之,於途中自殺。司馬休之、高雅之和劉牢之子劉敬宣於是北逃南燕。

桓玄在三月攻入建康時就廢除了元興年號,恢復隆安年號,不久又改元大亨。及後,桓玄自讓丞相及荊江徐三州刺史,以桓偉出任荊州刺史、桓脩為徐、兗二州刺史、桓石生為江州刺史、卞範之為丹楊尹、桓謙為尚書左僕射,分派桓氏宗族和親信出任內外職位。自置為太尉、平西將軍、都督中外諸軍事、揚州牧、領豫州刺史。另外又加袞冕之服,綠綟綬,增班劍至六十人,劍履上殿,入朝不趨,讚奏不名的禮遇。

四月,桓玄出鎮姑孰,辭錄尚書事,但朝中大事仍要諮詢他,小事則由朝中桓謙和卞範之決定。自晉安帝繼位以來,東晉國內戰禍連年,人民都厭戰不已,而桓玄上台後就罷黜奸佞之徒,擢用俊賢之士,令建康城中都一片歡欣景象,希望能過安定日子。不過很快,桓玄凌侮朝廷,豪奢縱欲,政令無常,故令人民失望。當時三吳大飢荒,很多人死亡,即使是富有的也不過著金玉財寶活活餓死家中,桓玄雖曾下令賑災,但米糧不多,給予不足,縱然會稽內史王愉召還出外尋食的飢民回去領糧,也就有很多人在道旁餓死。

另一方面,桓玄亦先後殺害吳興太守高素、竺謙之、高平相竺朗之、劉襲、彭城內史劉季武、冠軍將軍孫無終等北府軍舊將,以圖消滅劉牢之領下北府軍勢力。另亦要朝廷追論平司馬元顯和殷仲堪、楊佺期的功勳,分別加封豫章公及桂陽公,並轉讓給兒子桓昇及侄兒桓濬。又下詔全國避其父桓溫名諱,同名同姓者皆要改名,又贈其生母氏為豫章公太夫人。

元興二年(403年),桓玄遷大將軍,又上請率軍北伐後秦,但隨後就暗示朝廷下詔不准。桓玄本身就無意北伐,就裝作出尊重詔命的姿態停止。同年,桓偉去世,桓玄因公簡約禮儀,脫下喪服後又作樂。而桓偉一直是桓玄親仗的人,桓偉死後桓玄孤危,桓玄不臣之心已露,同時全國對其有怨氣,於是打算加快篡位工作。而桓玄親信殷仲文及卞範之當時亦勸桓玄早日篡位,連朝廷加授桓玄九錫的詔命和冊命都暗中寫好。桓玄於是進升桓謙、王謐和桓脩等人,讓朝廷命自己為相國,更劃南郡、南平郡、天門郡、零陵郡、營陽郡、桂陽郡、衡陽郡、義陽郡和建平郡共十郡封自己為楚王,加九錫,並能置楚國國內官屬。及後桓玄自解平西將軍和豫州刺史,將官屬併入相國府。

當時桓玄的行動令原為殷仲堪黨眾的庾仄起兵七千人反抗,趁著接替桓偉的荊州刺史桓石康未到就襲取襄陽,震動江陵,不過不久就被桓石康等所平定。桓玄及後又假意上表歸藩,卻又自己代朝廷作詔挽留自己,然後再請歸藩,又要晉安帝下手詔挽留,只因桓玄喜歡炫耀這些詔文,故此常常做這些自篇自導的上表和下詔事件。另桓玄亦命人報告祥瑞出現,又想像歷代般有高士出現,不惜命皇甫謐六世孫皇甫希之假扮高士,最終竟被時人稱作「充隱」。而桓玄對政令執行亦無堅定意志,常改變主意,令政命不一,改變起來亂七八糟。

元興二年(403年)十一月,桓玄加自己的冠冕至皇帝規格的十二旒,又加車馬儀仗及樂器,以楚王妃為王后,楚國世子為太子。十一月丁丑日(12月17日),由卞範之寫好禪讓詔書並命臨川王司馬寶逼晉安帝抄寫。庚辰日(12月20日),由兼太保、司徒王謐奉璽綬,將晉安帝的帝位禪讓給桓玄,隨後遷晉安帝至永安宮,又遷太廟的晉朝諸帝神主至琅邪國。及後百官到姑孰勸進,桓玄又假意辭讓,官員又堅持勸請,桓玄於是築壇告天,於十二月壬辰日(404年1月1日)正式登位為帝,並改元「永始」,改封晉安帝為平固王,不久遷於尋陽。

桓玄即帝位後,好行小惠以籠絡人心,例如他親自審訊囚犯時,不管罪刑輕重,多予釋放;攔御駕喊冤者,通常也可以得到救濟;然而為政繁瑣苛刻,又喜歡炫耀自己,官員有將詔書中「春蒐」字誤繕為「春菟」,經辦人員即全被降級或免職。

桓玄篡位以後,驕奢荒侈,遊獵無度,夜以繼日地遊樂。即使是兄長桓偉下葬的日子,桓玄白天哭喪到晚上就去遊玩了,有時甚至一日之間多次出遊。又因桓玄性格急躁,呼召時都要快速,當值官員都在省前繫馬備用,令宮禁內煩雜,已經不像朝廷了;另桓玄又興修宮殿、建造可容納三十人的大乘輿。百姓更因而疲憊困苦,民心思變。北府舊將劉裕、何無忌與劉毅等人於是乘時舉義兵討伐桓玄。元興三年二月乙卯日(404年3月24日),劉裕等人正式舉兵,計劃在京口(今江蘇鎮江)、廣陵(今江蘇揚州市)、歷陽和建康四地一同舉兵。其中劉裕派了周安穆向建康的劉毅兄劉邁報告,通知他作內應,然而劉邁惶恐,後更以為圖謀被揭向桓玄報告,桓玄初封劉邁為重安侯,但後又以劉邁沒有及時收捕周安穆,於是殺害劉邁和其他劉裕於建康的內應。原於歷陽舉兵的諸葛長民亦被刁逵所捕,但劉裕等終也成功奪取了京口和廣陵,鎮守兩地的桓脩和桓弘皆被殺。

劉裕率義軍進軍至竹里,桓玄加桓謙為征討都督。桓謙請求桓玄派兵攻劉裕,但桓玄畏於劉裕兵銳,打算屯兵覆舟山等待劉裕,認為對方自京口到建康後見到大軍必然驚愕,且桓玄軍堅守不出,對方求戰不得,會自動散走。不過桓謙堅持,桓玄就派了頓丘太守吳甫之及右衞將軍皇甫敷迎擊。不過二人皆在與劉裕作戰中戰死,桓玄大懼,就召見一眾會道術的人作法試圖對抗劉裕。後桓玄又命桓謙、何澹之屯東陵,卞範之屯覆舟山西,共以二萬兵抵抗劉裕。不過劉裕進至覆舟山東時故設疑兵,令敵方以為劉裕兵力眾多,桓玄得報後更派庾賾之率兵增援諸軍。然而,因為劉裕的兵眾大多是北府軍出身,故桓謙軍隊都畏懼劉裕,未有戰意,而劉裕則領兵死戰,並乘風施以火攻,終擊潰桓謙等。

在桓玄派桓謙等抵抗劉裕時,其實已經萌生離去的念頭,並命殷仲文準備船隻。桓謙等敗後,桓玄就於三月己未日(3月28日)與一眾親信西走。桓玄當天沒有進食,隨行人員就進糙米飯給桓玄,但桓玄吞不下,年幼的桓昇抱著桓玄撫慰他,更令桓玄忍不住心中悲傷。

桓玄一直到尋陽,得江州刺史郭昶之供給其物資及軍隊。後挾持晉安帝至江陵,在江陵署置百官,並且大修水軍,不足一個月就已有兵二萬,樓船和兵器都顯得很強盛的樣子。不過桓玄西奔後就怕法令不能認真執行,就輕易處以死刑,故令人心離異。

及後何無忌擊敗桓玄所派何澹之等軍,攻陷湓口,進佔尋陽,然後與劉毅等一直西進。桓玄亦自江陵率軍迎擊,兩軍於五月癸酉日(6月10日)在崢嶸洲相遇,當時桓玄軍雖然有兵力優勢,但因桓玄經常在船側泛舟,預演敗走時的動作,於是士眾毫無鬥志,在劉毅的進攻下潰敗,焚毁輜重乘夜逃走,郭銓遂向劉毅投降。桓玄於是挾晉安帝繼續西走,留晉穆帝皇后何法倪及安帝皇后王神愛於巴陵。殷仲文當時以收集散卒為名移駐別船,並趁機叛變,迎二后回建康。

桓玄於五月己卯日(6月16日)再到江陵,馮該勸桓玄再戰,但桓玄不肯,更想投奔梁州刺史桓希。不過當時人心已離,桓玄的命令都沒有人執行了。次日,江陵城中大亂,桓玄與心腹數百人出發,到城門時隨行有人從暗處走出要斬殺桓玄,但不中,於是彼此廝殺,桓玄勉強登船,身邊人員因亂分散,只有卞範之跟隨在側。桓玄正打算到梁州治所漢中(今陝西漢中市)時,但屯騎校尉毛脩之誘使桓玄入蜀,桓玄聽從。而當時正值寧州刺史毛璠去世,益州刺史毛璩派了侄孫毛祐之及參軍費恬等領數百人送毛璠喪至江陵,並於五月壬午日(6月19日)在枚回洲與桓玄相遇,二人於是進攻桓玄,箭矢如雨,桓玄寵信的丁仙期、萬蓋等為桓玄擋箭而死,益州都護馮遷跳上桓玄坐船,抽刀向前,桓玄拔下頭上玉飾遞給馮遷,說:「你是什麼人,竟敢殺天子?」馮遷說:「我這只是在殺天子之叛賊而已!」桓玄遂被殺,享年三十六歲。桓玄死後,堂弟桓謙在沮中為桓玄舉哀,上諡為武悼皇帝。桓玄頭顱則被傳至建康,掛在大桁上,百姓看見後都十分欣喜。

桓玄擅寫文章,可從其事跡中看到。王恭死後,桓玄曾登江陵城南樓,說:「我現在想為王孝伯作悼詞。」吟嘯良久後就下筆,很快就寫好了。桓玄消滅殷仲堪、楊佺期後,荊州刺史府、江州刺史府、後將軍府、七州都督府、南郡公府皆來賀,五個版牘一同進入,桓玄見版至使即答,皆美而成章,並不揉雜。

桓玄小時,與一眾堂兄弟鬥鵝,但桓玄的鵝總是不及堂兄弟強,十分不忿。於是有一晚到鵝欄殺死了堂兄弟們的鵝。天亮後家人都驚駭不已,以為發生了怪事,向桓沖報告。但桓沖心知是桓玄作的,一問,果然如此。

桓玄喜好裝飾和書畫,在擊敗司馬元顯後,桓玄遷鎮姑孰,就大築城內官府,建築物和假山水池等都十分壯麗。另又曾以輕舟載著他的書畫、服飾和玩物,有人因而勸諫他,桓玄竟說這些東西應該隨身,而且稱當時兵凶戰危,若發生問題就可以很快運走。眾人聽後都笑他。《晉書》又載他性格貪鄙,極愛奇珍異寶,珠玉等寶物更時不離手。別人有好書畫或佳園田宅,桓玄都想得到手,逼不到就在賭桌上奪得。桓玄又曾派下屬四出遷移果樹美竹收歸己有,令數千里內好的果樹和竹子都被一掃而空。

桓玄尊崇其父桓溫,故在篡位稱帝後就追尊桓溫為「宣武皇帝」,太廟都只供奉他,卻沒有追尊祖父桓彝或以上的祖宗。故及至桓玄遭受劉裕義軍來勢洶洶的進攻時,曹靖之稱其令晉室神主流離飄泊以及追尊不及祖父觸怒神明,令桓玄很是恐懼忿怨。

桓玄因劉裕討伐而西走江陵時,就於道上作《起居注》,內容都是他抵抗劉裕義軍的事,自稱自己指揮各軍,算無遺策,只因諸將違反其節度才兵敗,是非戰之罪。由於桓玄專心寫《起居注》,所以都沒閑暇時間和群下商議對策。寫成後桓玄就將《起居注》宣示遠近。

據說桓玄出生時,有光照亮房間,占卜者都感到奇異,故得桓玄小名靈寶。

桓玄早年頗善騎馬,曾在荊州刺史殷仲堪的江陵公廳前駕馬使矟,耀武揚威,卻被殷的部下劉邁(北府兵將劉毅之兄)貶低為:「馬矟的才能很夠,清談的義理卻不足」,桓玄因此痛恨劉邁,派人刺殺他,幸虧劉邁在殷仲堪的主意下,早一步回到京師,才躲過殺身之禍。

桓玄稱帝之後,入宮,因為身材發福肥大,當他坐上御牀後,不堪重擔的御牀就被壓爛陷地,眾人見此皆失色,殷仲文奉承說:「將由聖德深厚,地不能載。」令桓玄十分高興。又因為桓玄喜歡到宮外出遊,但肥大的體型對他上馬下馬諸多不便,他因此設計了能夠四面轉動的迴轉車,自己坐在上面可以方便地轉向移動。

據說,元興年間衡陽有母雞變成雄雞,八十日後雞冠卻萎縮了。後來桓玄建立楚國,衡陽郡亦在十郡以內,而自桓玄即位至敗走建康,也大約是八十日。當時亦有童謠:「長干巷,巷長干,今年殺郎君,後年斬諸桓。」郎君即司馬元顯,司馬元顯於元興元年(402年)被殺,桓氏則於元興三年(404年)因桓玄敗死而遭誅殺。

唐代房玄齡於晉書的「史臣曰」評論說:「桓玄纂凶,父之餘基。挾姦回之本性,含怒於失職;苞藏其豕心,抗表以稱冤。登高以發憤,觀釁而動,竊圖非望。始則假寵於仲堪,俄而戮殷以逞欲,遂得據全楚之地,驅勁勇之兵,因晉政之陵遲,乘會稽之酗醟,縱其狙詐之計,扇其陵暴之心,敢率犬羊,稱兵內侮。天長喪亂,凶力實繁,踰年之間,奄傾晉祚,自謂法堯禪舜,改物君臨,鼎業方隆,卜年惟永。俄而義旗電發,忠勇雷奔,半辰而都邑廓清,踰月而凶渠即戮,更延墜曆,復振頹綱。是知神器不可以闇干,天祿不可以妄處者也。夫帝王者,功高宇內,道濟含靈,龍宮鳳曆表其祥,彤雲玄石呈其瑞,然後光臨大寶,克享鴻名,允徯后之心,副樂推之望。若桓玄之么麼,豈足數哉!適所以干紀亂常,傾宗絕嗣,肇金行之禍難,成宋氏之驅除者乎!」

唐代某貴族「公子」與士族虞世南的對話:「公子曰:『桓玄聰明夙智,有奇才遠略,亦一代之異人,而遂至滅亡,運祚不終,何也?』先生(虞世南)曰:『夫人君之量,必器度宏遠,虛己應物,覆載同於天地,信誓合於寒暄,然後萬姓樂推而不厭也。彼桓玄者,蓋有浮狡之小智,而無含宏之大德,值晉室衰亂,威不迨下,故能肆其爪牙,一時篡奪,安德治民無聞焉。以僥幸之才,逢神武之運,至於夷滅,固其宜也。』」

梁代史家裴子野評論:「桓敬道有文武奇才,志雪餘恥,校〔狡〕動離亂之中,掩天下而不血刃,既而嘯命六合,規模凌取,未及逾年,坐盜社稷。自以名高漢祖,事捷魏、晉,思專其侈,而莫己知。王謐以民望鎮領〔袖〕,王綏、謝混以後進〔相〕光輝,群從兄弟,方州連郡,民駭其速而服其強,無異望矣。(宋)高祖是時,殊〔朱〕方之一匹夫也,無千百之眾,糾合同盟,雷擊三州,曾未及旬,蕩清京邑,號令群后,長驅江、漢,推亡楚於已拔,拯衰晉於已顛,自羲、軒以來,用兵之速,未始有也。自非雄略蓋世,天命至止,焉能若此者乎!於是,民知攸暨而王跡興。」

\subsubsection{永始}


\begin{longtable}{|>{\centering\scriptsize}m{2em}|>{\centering\scriptsize}m{1.3em}|>{\centering}m{8.8em}|}
  % \caption{秦王政}\
  \toprule
  \SimHei \normalsize 年数 & \SimHei \scriptsize 公元 & \SimHei 大事件 \tabularnewline
  % \midrule
  \endfirsthead
  \toprule
  \SimHei \normalsize 年数 & \SimHei \scriptsize 公元 & \SimHei 大事件 \tabularnewline
  \midrule
  \endhead
  \midrule
  元年 & 403 & \tabularnewline\hline
  二年 & 404 & \tabularnewline
  \bottomrule
\end{longtable}

\subsection{桓谦\tiny(404-405)}

\subsubsection{生平}

桓謙(4世紀-410年),字敬祖,譙國龍亢(今安徽懷遠)人。東晉末期人物,車騎將軍桓沖次子。在晉官至西中郎將、荊州刺史;桓楚時官至侍中、衞將軍。桓玄死後,桓謙仍然抵抗東晉,並於失敗後出奔後秦。後又因支持西蜀王譙縱對抗東晉而入蜀,終在西蜀的軍事行動下而再度與東晉作戰,被劉道規擊敗,被殺。

他初以父親的功勞封宜陽縣開國侯,歷次升遷官拜輔國將軍、吳國內史。隆安三年(399年),孫恩率眾進攻下會稽,並殺太守王凝之,三吳諸郡都有人起兵響應孫恩,桓謙聞亂出奔無錫(今江蘇無錫)。後桓謙獲徵召入朝擔任尚書,不久又先後轉任驃騎大將軍司馬元顯的諮議參軍及司馬。

元興元年(402年),司馬元顯要討伐荊州刺史桓玄,司馬元顯心腹張法順認為桓謙是桓玄在朝中的的耳目,應該除去,又建議命令劉牢之去下手,以測試其忠心。但司馬元顯不聽從,反而想借助桓謙父桓沖在荊州的威望去安撫荊州人,於是調桓謙為都督荊益寧梁四州諸軍事、西中郎將、荊州刺史。

同年,桓玄消滅司馬道子、司馬元顯勢力,掌握朝政,就以桓謙為尚書左僕射,領吏部,加中軍將軍,甚得桓玄倚仗。後改封寧都侯,升任尚書令,加散騎常侍,不久再遷任侍中、衞將軍、開府、錄尚書事。元興二年(403年),桓玄篡位稱帝,桓謙加領揚州刺史,封新安郡王。

元興三年(404年),劉裕起兵討伐桓玄,並進攻建康,桓玄於是命桓謙與何澹之出屯東陵(今南京九華山東北),與屯於覆舟山西的卞範之一同抵禦劉裕。但因桓謙等軍主要也是北府軍出身,面對北府軍將領出身的劉裕並沒鬥志,於是桓謙等大敗。及後隨桓玄西奔江陵(今湖北荊州市荊州區)。

同年五月,桓玄敗死,江陵亦被晉軍收復,桓謙藏匿在沮中。不久桓振襲取江陵,桓謙亦召集部眾響應,至閏五月己丑日(6月26日)重奪江陵,並俘虏仍在江陵的晉安帝。當時桓振打算殺害晉安帝,在桓謙竭力勸止下,終保存了晉安帝的性命,又與江陵群臣奉還玉璽給晉安帝。桓謙於是復任侍中、衞將軍,加江、豫二州刺史。桓振奪江陵後縱情酒色,肆意誅殺,當時桓謙勸桓振率兵出戰,自己留守江陵,但因桓振向來輕視桓謙而沒有聽從。

義熙元年(405年),晉軍反攻江陵,桓振留桓謙及馮該守江陵,親自率兵進攻南陽太守魯宗之,但當時劉毅已於江陵城外二十里的豫章口擊敗馮該,桓謙於是棄城出逃,劉毅於是成功收復江陵,桓振見此亦自潰。桓謙與桓怡、桓蔚、何澹之及溫楷等人於是投奔後秦。

義熙三年(407年),西蜀君主譙縱向後秦稱藩,後更上表以討伐劉裕為名向後秦借兵,又求後秦派桓謙入蜀協助。當時後秦天王姚興就特別問桓謙意見,桓謙也同意入蜀,然而姚興卻說:「小水池容不下大魚,若果譙縱他憑自己力量可成事,也就不必請你去協助他了。你最好還是自求多福吧。」桓謙到成都後虛心招引蜀地士人,終惹來譙縱懷疑,安置他於龍格(今四川雙流),並命人監視他。

義熙六年(410年),當時東晉正在鎮壓盧循的叛亂,譙縱於是趁機向後秦請兵進伐東晉,桓謙於是獲譙縱任命為荊州刺史,與譙道福共率二萬進攻東晉荊州。桓謙在道上招集當地支持桓氏的民眾,又招得了二萬人,並屯駐於枝江(今湖北枝江西南),一度威脅江陵,江陵人民甚至向桓謙報告城內狀況。東晉荊州刺史劉道規決定水陸並進,進攻桓謙;桓謙亦以水軍配以步騎兵與劉道規決戰,但桓謙最終戰敗,想要投靠前來助攻的後秦前將軍苟林,但被追擊的劉道規所殺。

\subsubsection{天康}


\begin{longtable}{|>{\centering\scriptsize}m{2em}|>{\centering\scriptsize}m{1.3em}|>{\centering}m{8.8em}|}
  % \caption{秦王政}\
  \toprule
  \SimHei \normalsize 年数 & \SimHei \scriptsize 公元 & \SimHei 大事件 \tabularnewline
  % \midrule
  \endfirsthead
  \toprule
  \SimHei \normalsize 年数 & \SimHei \scriptsize 公元 & \SimHei 大事件 \tabularnewline
  \midrule
  \endhead
  \midrule
  元年 & 404 & \tabularnewline\hline
  二年 & 405 & \tabularnewline
  \bottomrule
\end{longtable}


%%% Local Variables:
%%% mode: latex
%%% TeX-engine: xetex
%%% TeX-master: "../Main"
%%% End:


%%% Local Variables:
%%% mode: latex
%%% TeX-engine: xetex
%%% TeX-master: "../Main"
%%% End:
