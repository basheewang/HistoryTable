%% -*- coding: utf-8 -*-
%% Time-stamp: <Chen Wang: 2019-12-18 13:27:04>

\section{元帝\tiny(318-322)}

\subsection{生平}

晉元帝司馬睿(276年5月27日-323年1月3日),字景文,東晉時期第一位皇帝。司馬懿的曾孫、琅邪武王司馬伷之孫、琅邪恭王司馬覲之子,母為琅邪王妃夏侯光姬。《魏書》說司馬睿是牛金和夏侯光姬的私生子。

司馬睿於290年袭封琅邪王,曾經參與討伐成都王司馬穎的戰役;但是由於作戰失利,司馬睿便離開洛陽,回到封國;晉懷帝即位後,司馬睿被封為鎮東大將軍、都督揚州諸軍事,後來在王導的建議之下前往建康,並且極力結交江東大族。311年晉懷帝被俘遇害後,晉愍帝即位,晉愍帝封司馬睿為丞相、大都督中外諸軍事。晉愍帝被俘後,司馬睿在晉朝貴族與江東大族的支持下於317年三月辛卯(公历4月6日)称晋王,318年三月丙辰(公历4月26日)即帝位,為晉元帝。

即位之初曾嘗試北伐,其中祖逖本有希望恢复旧土,但他被晉元帝及世家大族挾制,郁郁而终。

晉元帝實際上為一個被扶持者,本身並無實際權力,大權掌握在王導與王敦之手。晉元帝聽從刁協與劉隗的言論並有意削弱琅邪王氏權力,導致王敦於322年反叛,攻入建康,並且殺害重臣戴淵、周顗等人。但是王敦無力消滅東晉,最後採取與晉元帝和睦的策略。晉元帝便在王敦之亂中因憂鬱過度而過世。

唐代房玄齡於《晉書》的「史臣曰」評論說:「晉氏不虞,自中流外,五胡扛鼎,七廟隳尊,滔天方駕,則民懷其舊德者矣。昔光武以數郡加名,元皇(案:晉元帝)以一州臨極,豈武、宣餘化,猶暢於琅邪,文、景垂仁,傳芳於南頓?所謂後乎天時,先諸人事者也。馳章獻號,高蓋成陰,星斗呈祥,金陵表慶。陶士行擁三州之旅,郢外以安;王茂弘爲分陝之計,江東可立。或高旌未拂,而遐心斯偃,回首朝陽,仰希乾棟,帝猶六讓不居,七辭而不免也。布帳綀帷,詳刑簡化,抑揚前軌,光啓中興。古首私家不蓄甲兵,大臣不爲威福,王之常制,以訓股肱。中宗失馭強臣,自亡齊斧,兩京胡羯,風埃相望。雖復《六月》之駕無聞,而《鴻雁》之歌方遠,享國無幾,哀哉!」

唐代某貴族「公子」與虞世南的對話:「公子曰:『中宗值天下崩離,創立江左,俱為中興之主,比於前代,功德云何?』先生曰:『元帝自居藩邸,少有令聞,及建策南渡,興亡繼絕,委任宏茂,撫綏新舊,故能嗣晉配天,良有以也。然仁恕為懷,剛毅情少,是以王敦縱暴,幾危社稷,蹙國舒禍,其周平之匹乎?』」

\subsection{建武}

\begin{longtable}{|>{\centering\scriptsize}m{2em}|>{\centering\scriptsize}m{1.3em}|>{\centering}m{8.8em}|}
  % \caption{秦王政}\
  \toprule
  \SimHei \normalsize 年数 & \SimHei \scriptsize 公元 & \SimHei 大事件 \tabularnewline
  % \midrule
  \endfirsthead
  \toprule
  \SimHei \normalsize 年数 & \SimHei \scriptsize 公元 & \SimHei 大事件 \tabularnewline
  \midrule
  \endhead
  \midrule
  元年 & 317 & \tabularnewline\hline
  二年 & 318 & \tabularnewline
  \bottomrule
\end{longtable}

\subsection{大兴}

\begin{longtable}{|>{\centering\scriptsize}m{2em}|>{\centering\scriptsize}m{1.3em}|>{\centering}m{8.8em}|}
  % \caption{秦王政}\
  \toprule
  \SimHei \normalsize 年数 & \SimHei \scriptsize 公元 & \SimHei 大事件 \tabularnewline
  % \midrule
  \endfirsthead
  \toprule
  \SimHei \normalsize 年数 & \SimHei \scriptsize 公元 & \SimHei 大事件 \tabularnewline
  \midrule
  \endhead
  \midrule
  元年 & 318 & \tabularnewline\hline
  二年 & 319 & \tabularnewline\hline
  三年 & 320 & \tabularnewline\hline
  四年 & 321 & \tabularnewline
  \bottomrule
\end{longtable}

\subsection{永昌}

\begin{longtable}{|>{\centering\scriptsize}m{2em}|>{\centering\scriptsize}m{1.3em}|>{\centering}m{8.8em}|}
  % \caption{秦王政}\
  \toprule
  \SimHei \normalsize 年数 & \SimHei \scriptsize 公元 & \SimHei 大事件 \tabularnewline
  % \midrule
  \endfirsthead
  \toprule
  \SimHei \normalsize 年数 & \SimHei \scriptsize 公元 & \SimHei 大事件 \tabularnewline
  \midrule
  \endhead
  \midrule
  元年 & 322 & \tabularnewline\hline
  二年 & 323 & \tabularnewline
  \bottomrule
\end{longtable}


%%% Local Variables:
%%% mode: latex
%%% TeX-engine: xetex
%%% TeX-master: "../Main"
%%% End:
