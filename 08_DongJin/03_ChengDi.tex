%% -*- coding: utf-8 -*-
%% Time-stamp: <Chen Wang: 2019-12-18 13:32:07>

\section{成帝\tiny(325-342)}

\subsection{生平}

晉成帝司馬衍(321年12月或322年1月-342年7月26日),字世根,東晉的第三代皇帝,晉明帝之長子。晉成帝年幼即位,即位不久即遇上蘇峻之亂,成帝亦一度遭蘇峻叛軍劫持。成帝一朝軍政主要由外戚穎川庾氏把持,在庾亮的主導下還曾謀北伐,但因後趙強盛而遭到失敗。

太寧三年三月戊辰(325年4月1日),晉明帝立司馬衍為皇太子。同年閏八月戊子(325年10月18日),晉明帝去世,翌日五歲的晉成帝即位為帝。由於年幼,由母親皇太后庾文君臨朝稱制,由七位顧命大臣輔政,中書令庾亮以国舅身份主政。。

咸和二年(327年)年末,歷陽內史蘇峻與豫州刺史祖約叛亂,並在翌年率兵攻至建康,庾亮試圖抵抗但失敗,被逼出逃,晉成帝就與王導等眾官為蘇峻所挾持,宮中就遭到蘇峻軍搶掠和焚燒,太官也僅餘下數石米供成帝食用。咸和三年五月乙未,蘇峻強逼晉成帝遷居至石頭城一個倉庫中,成帝哭著登車出發,宮中人們亦都傷心痛哭。咸和四年(329年),以陶侃為首的軍隊平定蘇峻之亂,迎回成帝,因為宮殿遭戰火破壞,故修繕建平園作為宮室,至咸和七年(332年)新建的建康宮落城後才遷去新宮。

蘇峻之亂後,朝內就由王導專制,成帝對王導亦相當敬重,甚至屢幸王導宅第;庾亮則領豫州刺史出鎮蕪湖,主掌軍事,隨著陶侃去世,庾亮更兼荊江豫三州,轉鎮武昌,並著眼對後趙的北伐。咸康五年(339年),庾亮作出北伐部署,上奏移鎮襄陽石城,並且增兵長江、漢水流域以及淮泗壽陽地區要地,為一舉北伐作好準備,當時庾亮更派兵進攻巴郡,攻至江陽,俘獲後趙將領李閎及黃植。晉成帝下給群臣議論,上疏得王導支持,但郗鑒以資源不足為由反對。不過未等到允許,庾亮的行動就遭後趙以軍事行動作回應,派軍大舉南侵,庾亮所定的重鎮邾城更加被攻陷,庾亮北伐遂流產。

咸康二年(336年)晉成帝頒布壬辰詔書,禁止士族、官吏將私佔山川大澤;咸康七年(341年),又以土斷方式將自江北遷來的世族編入戶籍。

咸康八年(342年)7月23日,晉成帝患病,中書監庾冰為了留住穎川庾氏家族與皇帝的血緣親近,於是以國家外有強敵,宜立年長君主為由勸服成帝以弟弟琅琊王司馬岳為儲君。7月26日,晉成帝駕崩,年僅22歲,廟號顯宗。8月18日,葬於興平陵。

晉成帝年紀小小就很聰敏,有成年人的量度。蘇峻之亂前,庾亮以謀反罪誅殺了南頓王司馬宗,成帝一直不知,至亂事平定後才問及失蹤的司馬宗,庾亮答稱他因謀反而被誅,豈料成帝卻哭著說:「舅言人作賊,便殺之。人言舅作賊,復若何?」嚇得庾亮恐懼失色。至後來,庾懌送毒酒意圖毒殺江州刺史王允之,被揭發後成帝就怒道:「大舅已亂天下,小舅復欲爾邪?」庾懌被逼自殺。不過成帝年輕時被舅舅家族颖川庾氏勢力所限制,並不親政。至後來長大,卻留心事務,而且生活儉約,曾因射堂需耗用四十金而放棄建造。

在石頭城時右衞將軍劉超仍為成帝講授《孝經》及《論語》,但因劉超與鍾雅帶成帝逃出去的圖謀泄漏,二人遭蘇峻派任讓收捕殺害,期間晉成帝抱住任讓哭求:「還我侍中、右衞!」但任讓不聽小皇帝的命令,將二人殺了。蘇峻之亂被平定後,任讓原本因與陶侃有舊情而得免死,但成帝記恨他,任讓還是被誅殺。

\subsection{咸和}

\begin{longtable}{|>{\centering\scriptsize}m{2em}|>{\centering\scriptsize}m{1.3em}|>{\centering}m{8.8em}|}
  % \caption{秦王政}\
  \toprule
  \SimHei \normalsize 年数 & \SimHei \scriptsize 公元 & \SimHei 大事件 \tabularnewline
  % \midrule
  \endfirsthead
  \toprule
  \SimHei \normalsize 年数 & \SimHei \scriptsize 公元 & \SimHei 大事件 \tabularnewline
  \midrule
  \endhead
  \midrule
  元年 & 326 & \tabularnewline\hline
  二年 & 327 & \tabularnewline\hline
  三年 & 328 & \tabularnewline\hline
  四年 & 329 & \tabularnewline\hline
  五年 & 330 & \tabularnewline\hline
  六年 & 331 & \tabularnewline\hline
  七年 & 332 & \tabularnewline\hline
  八年 & 333 & \tabularnewline\hline
  九年 & 334 & \tabularnewline
  \bottomrule
\end{longtable}

\subsection{咸康}

\begin{longtable}{|>{\centering\scriptsize}m{2em}|>{\centering\scriptsize}m{1.3em}|>{\centering}m{8.8em}|}
  % \caption{秦王政}\
  \toprule
  \SimHei \normalsize 年数 & \SimHei \scriptsize 公元 & \SimHei 大事件 \tabularnewline
  % \midrule
  \endfirsthead
  \toprule
  \SimHei \normalsize 年数 & \SimHei \scriptsize 公元 & \SimHei 大事件 \tabularnewline
  \midrule
  \endhead
  \midrule
  元年 & 335 & \tabularnewline\hline
  二年 & 336 & \tabularnewline\hline
  三年 & 337 & \tabularnewline\hline
  四年 & 338 & \tabularnewline\hline
  五年 & 339 & \tabularnewline\hline
  六年 & 340 & \tabularnewline\hline
  七年 & 341 & \tabularnewline\hline
  八年 & 342 & \tabularnewline
  \bottomrule
\end{longtable}


%%% Local Variables:
%%% mode: latex
%%% TeX-engine: xetex
%%% TeX-master: "../Main"
%%% End:
