%% -*- coding: utf-8 -*-
%% Time-stamp: <Chen Wang: 2019-12-18 13:40:15>

\section{恭帝\tiny(419-420)}

\subsection{生平}

晉恭帝司馬德文(386年-421年11月2日),字德文,河內溫縣(今河南溫縣)人。東晉的末代皇帝。為晉孝武帝之子,晉安帝之胞弟,母親是淑媛陳歸女。初封琅邪王,後在桓玄篡位後長期侍奉晉安帝左右。晉安帝死後被劉裕以遺詔立為皇帝,但其時劉裕已經完全掌握東晉朝政,司馬德文僅為傀儡而已。劉裕篡晉後為零陵王,次年遇害。

太元十七年十一月庚寅日(392年12月27日)受封為琅邪王,後又拜中軍將軍、散騎常侍。隆安二年(398年)轉衞將軍、開府儀同三司。隆安三年(399年)遷侍中,領司徒、錄尚書六條事。元興元年(402年),桓玄擊敗司馬道子父子,掌握朝政,改以司馬德文為太宰。

元興二年十二月壬辰日(404年1月1日),桓玄篡位稱帝,貶晉安帝為平固王,司馬德文亦因而降封「石陽縣公」。不久桓玄遷安帝至尋陽(今江西九江市),司馬德文亦跟隨。元興三年(404年),劉裕起兵討伐桓玄,桓玄兵敗逃到尋陽,得郭昶之給予器具及士兵後再逼晉安帝與其同至江陵(今湖北江陵);及至桓玄敗死於逃往益州途中,荊州別駕王康產及南郡太守王騰之迎晉安帝至南郡府舍時,司馬德文亦緊隨。然而,桓振等桓楚餘眾趁劉毅等軍未及趕至江陵,乘虛來襲,最終江陵城陷,王康產及王騰之遇害,桓振亦騎馬揮戈直入,問桓玄子桓昇下落,並在得知其死訊後大怒,指責他們屠殺桓氏。當時司馬德文辯護道:「這又豈會是我們兄弟的意思!」在桓謙苦勸下,桓振才沒有加害安帝。隨後桓振繼續控制江陵,並以司馬德文為徐州刺史,繼續對抗由劉毅所統領的討伐軍隊。

義熙元年(405年),劉毅等軍攻下江陵,司馬德文亦與安帝一同在何無忌護送下返回建康。回建康後,司馬德文遷大司馬,並於義熙四年(408年)加領司徒。

義熙十二年(416年),劉裕預備北伐後秦,時劉裕圖以晉室名聲安撫北方人民,故想奉司馬德文之名北伐,司馬德文因而上書出兵,以修謁晉室山陵,最終劉裕就與司馬德文一同率兵出發。義熙十三年(417年),劉裕成功滅亡後秦,同年年末班師東歸,司馬德文亦跟隨,至次年(418年)夏季,劉裕到達彭城(今江蘇徐州市),司馬德文先回建康。不久,劉裕受九錫,封宋王。

劉裕當時指派了中書侍郎王韶之圖謀殺害晉安帝,立司馬德文為帝,以應「昌明之後尚有二帝」的預言。不過因司馬德文無論飲食還是睡覺都和晉安帝在一起,王韶之無法下手。可是司馬德文卻於當年年末患病,離開了安帝,王韶之趁機會下手,將安帝殺死。劉裕則假稱遺詔,以司馬德文繼位。

元熙二年六月壬戌(420年7月5日),劉裕入朝,傅亮暗示司馬德文禪讓帝位給劉裕,並將禪讓詔書的草稿上呈,要他抄寫。司馬德文欣然接受,執筆抄寫,並說:「桓玄篡位那時,晉室經已失去天下了,又因劉公延長了國祚,至今已將近二十年了;今日作這種事,是心甘情願的。」兩日後,司馬德文退居琅邪王府,百官向晉帝告別,東晉至此滅亡。又三日後,劉裕正式登位,並奉司馬德文為零陵王,讓他遷至秣陵縣(今江蘇江寧縣)的舊縣治作為其府第,正朔、車駕、衣服等都依晉朝規格,正如昔日晉篡魏的先例,並命劉遵派兵守衞。

及後劉裕就有殺害司馬德文的意圖,最初就命前琅邪國郎中令張偉拿毒酒去殺司馬德文,但張偉就嘆道:「要毒殺主君去讓自己活下去,不如死了!」竟在路上喝下毒酒自盡。司馬德文自己也十分害怕會遭毒手,於是起居飲食都由王妃褚靈媛打點,食物也在自己面前烹煮,令加害者無從下手。不過,褚靈媛兄褚秀之及褚淡之都忠於劉裕,一直以來司馬德文生下的男嬰都被二人借故害死。至永初二年(421年)九月,劉裕即命褚淡之及褚叔度去見褚靈媛,乘機支開她到另一個房間。及後士兵就翻過牆進入府內,逼司馬德文服食毒藥。但司馬德文不肯,更說:「佛教所稱,自殺的人都不能輪迴再生為人。」士兵於是用被褥將其悶死,享年三十六歲。司馬德文以晉禮下葬於沖平陵,諡恭皇帝。

史載,晉安帝司馬德宗從小到大都不會說話,甚至連冬夏的氣侯轉變也不能分辨。而司馬德文一直侍奉左右,打理他的生活起居,以恭敬謹慎而聞名,亦得當時人們稱許。

司馬德文信奉佛教,曾下令打造了一個高一丈六寸的黃金佛像,並親身到瓦官寺迎其上位。其死前所言亦証其篤信佛教。

據說司馬德文年幼時頗為殘忍急躁,在琅邪國時更曾命擅長射箭的人射擊馬匹作為娛樂。當時有人說:「馬是國姓,而你自己就去殺牠,這是很不祥的事呀!」司馬德文明白此言,亦甚為後悔。

\subsection{元熙}

\begin{longtable}{|>{\centering\scriptsize}m{2em}|>{\centering\scriptsize}m{1.3em}|>{\centering}m{8.8em}|}
  % \caption{秦王政}\
  \toprule
  \SimHei \normalsize 年数 & \SimHei \scriptsize 公元 & \SimHei 大事件 \tabularnewline
  % \midrule
  \endfirsthead
  \toprule
  \SimHei \normalsize 年数 & \SimHei \scriptsize 公元 & \SimHei 大事件 \tabularnewline
  \midrule
  \endhead
  \midrule
  元年 & 419 & \tabularnewline\hline
  二年 & 429 & \tabularnewline
  \bottomrule
\end{longtable}


%%% Local Variables:
%%% mode: latex
%%% TeX-engine: xetex
%%% TeX-master: "../Main"
%%% End:
