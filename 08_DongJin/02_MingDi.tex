%% -*- coding: utf-8 -*-
%% Time-stamp: <Chen Wang: 2019-12-18 13:28:48>

\section{明帝\tiny(322-325)}

\subsection{生平}

晉明帝司馬紹(299年-325年),字道畿,東晉的第二代皇帝,晉元帝司馬睿長子。母親是豫章郡君荀氏。在位不足三年,但在位期間平定了王敦之亂。

司馬紹自小聰慧,故此特別受父親司馬睿所寵愛。後於永嘉元年(307年)隨父親一同移鎮建業(後改建康,今江蘇南京市)。建興元年(313年),司馬睿升任左丞相,拜司馬紹為東中郎將,鎮守廣陵。316年,晉愍帝所在的長安被前趙攻陷,晉愍帝出降,西晋灭亡。有鉴于此,317年,司馬睿稱晉王,建元建武,並立司馬紹為晉王太子。318年,司馬睿即位称帝,改元太興,司馬紹被立為皇太子。

永昌元年(322年)發生王敦之亂,大將軍王敦領兵進攻建康並佔領石頭城,晉元帝派王導等人進攻石頭城但都被王敦擊敗,司馬紹於是打算率領將士與王敦決一死戰,即將出發時因遭太子中庶子溫嶠極力勸阻而沒有實行。隨後王敦自任丞相並掌握朝政,見司馬紹勇而有謀,而且朝野中亦有很高名望,於是打算誣陷他不孝而將他廢掉,但因溫嶠等大臣支持司馬紹,王敦終也不能廢掉司馬紹。

晉元帝因王敦之亂而憂憤成疾,於當年閏十一月己丑日(323年1月3日)病逝,司馬紹在次日繼位,为晉明帝,並由司空王導輔政。

王敦雖於永昌元年(322年)就回到武昌遙控朝廷,但因為圖謀篡位,於太寧元年(323年)暗示要朝廷徵召自己入朝,晉明帝於是以手詔徵召王敦。同年,王允之乘酒宴而知道王敦的圖謀,於是回京告訴其父王舒,王舒於是與王導一同報告晉明帝,得以早作防備。

次年,晉明帝既心知王敦意圖,於是騎馬微服去視察王敦於于湖的營地,但遭到軍人發現,並派五名騎兵追捕。晉明帝逃走時,用水浸濕所騎馬匹的粪便来使其降温,又拿出七寶鞭交給路旁賣食物的婆婆,並要她出示給追來的騎兵。晉明帝走後不久,追兵就來到,並詢問婆婆,婆婆於是取出七寶鞭,並稱那人已經走得很遠。騎兵們顧著傳玩七寶鞭而在那裏停留了很久,而且見馬糞已冷,以為追不及了,於是都沒有再追,晉明帝因此成功逃脫。

及後,晉明帝積極準備京師建康的防護,最終於當年成功擊敗王敦派來進攻的軍隊,平定了王敦之亂。王敦之亂後,晉明帝下令不再問罪於王敦一眾官屬,又分別以應詹為江州刺史、劉遐為徐州刺史、陶侃為荊州刺史、王舒為湘州刺史,重整各州形勢,消除王敦以王氏宗族各領諸州以凌弱帝室的失衡情形。

太寧三年閏八月戊子(325年10月18日),司馬紹病逝於東堂,年僅二十七歲。葬於武平陵,廟號肅祖。

司馬紹年少聰明,小時候便曾經與父親就「太陽與長安孰近」的問題作出不同答案的爭辯。長大後聰明有機斷,精於事理,於是能讓國家從王敦之亂的亂局回復平定。

司馬紹性至孝,有文武才略,敬重賢人,素好文辭,於是當時如王導、庾亮、溫嶠、桓彝、阮放等名臣都親待他。而因他習武藝,善於安撫將士,於是任太子時東宮聚集很多人,亦得遠近各人歸心。

王敦曾称呼晋明帝为:「黄鬚鲜卑奴」,這是因為其母建安郡君荀氏是燕代人,混雜了當地鮮卑人血統,故明帝可能也長得有一些像外族,鬚為黃色。

司馬紹任太子時,想修建池苑樓臺,但元帝不許。司馬紹於是命手下的武士在一晚之間修好太子西池。

司馬紹有寵妃宋褘,褘國色天香,善吹笛,乃石崇妾綠珠之女弟子,不久司馬紹病篤,群臣进谏,请出宋袆,最後宋褘被送給吏部尚书阮孚。

司馬紹在位時,曾問晉室得天下的事。王導於是告訴他司馬懿當日發動高平陵之變誅除曹爽,樹立蔣濟等與自己同心的大臣;又說道曹髦被司馬昭親信賈充所命的成濟弒殺一事。司馬紹聽後,將面龐伏在牀上,說:「若真的像你所說,晉室國祚又怎能夠長遠!」

唐代房玄齡於《晉書》的「史臣曰」評論說:「維揚作宇,憑帶洪流,楚江恆戰,方城對敵,不得不推誠將相,以總戎麾。樓船萬計,兵倍王室,處其利而無心者,周公其人也。威權外假,嫌隙內興,彼有順流之師,此無強籓之援。商逢九亂,堯止八音,明皇(案:晉明帝)負圖,屬在茲日。運龍韜於掌握,起天旆於江靡,燎其餘燼,有若秋原。去縗絰而踐戎場,斬鯨鯢而拜園闕。鎮削威權,州分江漢,覆車不踐,貽厥孫謀。其後七十餘年,終罹敬道之害。或曰:『興亡在運,非止上流。』豈創制不殊,而弘之者異也。」

唐代某貴族「公子」與虞世南的對話:「公子曰:『東晉自元帝已下,何為賢主?』先生曰:『晉自遷都江左,強臣擅命,(天子)垂拱南面,政非己出。王敦以磐石之宗,居上流之地,負才矜地,志懷沖問鼎,非明帝之雄斷,王導之忠誠,則晉祚其移於他族矣。若使降年永久,佐任群賢,因洛、澗之遺黎,乘劉、石之衰運,興復中原,不難圖也。』」


\subsection{太宁}

\begin{longtable}{|>{\centering\scriptsize}m{2em}|>{\centering\scriptsize}m{1.3em}|>{\centering}m{8.8em}|}
  % \caption{秦王政}\
  \toprule
  \SimHei \normalsize 年数 & \SimHei \scriptsize 公元 & \SimHei 大事件 \tabularnewline
  % \midrule
  \endfirsthead
  \toprule
  \SimHei \normalsize 年数 & \SimHei \scriptsize 公元 & \SimHei 大事件 \tabularnewline
  \midrule
  \endhead
  \midrule
  元年 & 323 & \tabularnewline\hline
  二年 & 324 & \tabularnewline\hline
  三年 & 325 & \tabularnewline\hline
  四年 & 326 & \tabularnewline
  \bottomrule
\end{longtable}


%%% Local Variables:
%%% mode: latex
%%% TeX-engine: xetex
%%% TeX-master: "../Main"
%%% End:
