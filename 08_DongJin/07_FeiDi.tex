%% -*- coding: utf-8 -*-
%% Time-stamp: <Chen Wang: 2021-11-01 11:47:04>

\section{废帝司馬奕\tiny(365-371)}

\subsection{生平}

司馬奕(342年-386年),字延齡,東晉的第七代皇帝,晉成帝之子、晉哀帝之弟。晉哀帝死後於365年即帝位,史稱「廢帝」。

342年六月封为东海王,352年拜散骑常侍镇军将军。360年升车骑将军,361年改封琅琊王。362年七月為侍中骠骑大将军开府仪同三司。司馬奕即位之時,桓溫掌握朝政,桓的幕府參軍郗超建議桓溫效仿伊尹、霍光,廢除天子以立威信,但司馬奕本身並無過失可言,桓溫便指司馬奕陽痿不能人道,指田、孟二妃所生三皇子为司马奕的男宠相龙、计好及朱灵宝所生,於太和六年(371年)廢司馬奕為東海王,之後再貶為海西縣公,遷居吳縣西柴里,并将田、孟二妃及三皇子处死。

司馬奕遭廢位后心灰意冷,又怕再遭禍端,便苟且偷生。之後司馬奕更是沉迷於酒色,成日过着荒淫的生活,甚至生了孩子也不养,桓溫及之後继位的晋孝武帝也因此对他不再防范。

司马奕於386年過世,享年四十五歲,他亦是東晉較為長壽的皇帝。

\subsection{太和}

\begin{longtable}{|>{\centering\scriptsize}m{2em}|>{\centering\scriptsize}m{1.3em}|>{\centering}m{8.8em}|}
  % \caption{秦王政}\
  \toprule
  \SimHei \normalsize 年数 & \SimHei \scriptsize 公元 & \SimHei 大事件 \tabularnewline
  % \midrule
  \endfirsthead
  \toprule
  \SimHei \normalsize 年数 & \SimHei \scriptsize 公元 & \SimHei 大事件 \tabularnewline
  \midrule
  \endhead
  \midrule
  元年 & 366 & \tabularnewline\hline
  二年 & 367 & \tabularnewline\hline
  三年 & 368 & \tabularnewline\hline
  四年 & 369 & \tabularnewline\hline
  五年 & 370 & \tabularnewline\hline
  六年 & 371 & \tabularnewline
  \bottomrule
\end{longtable}



%%% Local Variables:
%%% mode: latex
%%% TeX-engine: xetex
%%% TeX-master: "../Main"
%%% End:
