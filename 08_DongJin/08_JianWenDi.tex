%% -*- coding: utf-8 -*-
%% Time-stamp: <Chen Wang: 2021-11-01 11:47:47>

\section{简文帝司馬昱\tiny(371-372)}

\subsection{生平}

晉簡文帝司馬\xpinyin*{昱}(320年-372年9月12日),字道萬。東晉第八代皇帝。东晋开国皇帝晋元帝少子,母郑阿春。自永和元年(345年)開始一直以會稽王輔政,掌握朝廷的實權,但其時權臣桓溫的勢力亦一直增強。52歲時於太和六年十一月己酉(372年1月6日)被桓溫擁立为帝,改年号为咸安。次年七月己未(372年9月12日)病逝。在位期間只有250日,期間桓温擅权。

永昌元年(322年),晉元帝下詔封司馬昱為琅邪王,作為自己入繼大宗後父親爵位的繼嗣。咸和二年(327年)因其母喪,請求服重而改封會稽王,官拜散騎常侍。咸和九年(334年)轉任右將軍,加侍中。咸康六年(340年)進撫軍將軍,領祕書監。建元元年(343年)加領太常。

永和元年(345年),因著上一年晉康帝去世,年幼的晉穆帝登位,崇德太后褚蒜子抱晉穆帝臨朝。當時輔政的驃騎將軍何充希望由太后父親褚裒入朝輔政但對方辭讓,當時司馬昱聲望高,故升司馬昱為撫軍大將軍,錄尚書六條事,與何充輔政。同年,荊州刺史庾翼去世,死前請求以其子庾爰之接代其位,何充則屬意徐州刺史桓溫取代庾氏掌握荊州。司馬昱倚重的名士劉惔熟悉桓溫,指桓溫雖然有才幹但也極有野心,不能讓他居於荊州這個控制長江上游的「形勝之地」,建議由司馬昱親自外鎮荊州,或由劉惔自己任荊州刺史。司馬昱沒有聽從劉惔的建言,桓溫任荊州刺史,獲得了日後奪權掌政的資本。

永和二年(346年),何充去世,左光祿大夫蔡謨加領司徒兼錄尚書六條事,與司馬昱一同輔政,司馬昱總理萬機,其實是東晉朝廷的決策者。何充兼領的揚州刺史此時出缺,褚裒舉薦了名士殷浩,殷浩辭讓並寫信給司馬昱說明理由,但司馬昱勸他出仕,四個月後殷浩終於出仕。次年,桓溫平滅成漢,建立大功,威望和勢力都大為提升,同時也引來朝內對其的忌憚。司馬昱決定以殷浩抗衡桓溫,後來後趙國內大亂,授殷浩以北伐的重任。然而殷浩北伐失敗,桓溫在永和十年借朝野對殷浩北伐失敗的不滿廢掉殷浩,司馬昱亦無力抗衡桓溫高漲的力量,令得桓溫在朝中獨大。

永和八年(352年),詔升司馬昱為司徒,司馬昱辭讓。興寧三年(365年),琅邪王司馬奕即位為晉廢帝,琅邪國無嗣,晉廢帝封司馬昱爲琅邪王,改以司馬昱子司馬昌明為會稽王。司馬昱辭讓,所以仍以會稽王號封琅邪王。次年,詔進司馬昱為丞相、錄尚書事、入朝不趨,贊拜不名,劍履上殿、賜羽葆,鼓吹及班劍六十人,司馬昱又辭讓。

太和四年(369年),桓溫第三次北伐大敗於前燕和前秦聯軍,豫州刺史袁真不堪被桓溫所誣要負上北伐失敗的責任而叛變。司馬昱在涂中與桓溫會面,商討隨後的行動,以桓溫子桓熙為豫州刺史。

太和六年十一月己酉日(372年1月6日),大司馬桓溫廢黜晉廢帝為東海王,率百官到會稽王府奉迎司馬昱,司馬昱即日即位為帝,是為晉簡文帝,改元咸安。桓溫及後就寫了講辭,打算向司馬昱陳述自己廢立的本意,但司馬昱每接見他都不停流淚,如此令桓溫恐懼,居然不能說一句話。

司馬昱的哥哥武陵王司馬晞有軍事才幹,被桓溫所忌。廢立不久,桓溫就誣陷司馬晞謀反將其免官,及後更逼令新蔡王司馬晃自誣與司馬晞及庾倩等人謀反,以求翦滅陳郡殷氏和穎川庾氏在朝中的勢力。隨後桓溫指示御史中丞司馬恬奏請司馬昱依律法處死司馬晞,司馬昱不肯,下令再作詳細議論。桓溫再次上奏求誅司馬晞,言詞十分嚴厲急切,司馬昱於是手詔給桓溫,寫道:「若果晉室國祚長久,那麼你就應該依從早前的詔命從事;如晉室大勢已去,那你就讓我退位讓賢吧。」桓溫看後流汗色變,不敢再逼,只上奏廢掉司馬晞和他三名兒子,並流放其家屬。

桓溫既行廢立,亦誅滅了與司馬皇室親密的殷氏和庾氏,威勢達至高峯。不過,桓溫在當時仍受制於以王坦之為首的太原王氏及謝安為首的陳郡謝氏世族力量,有篡位心而不能得逞。而司馬昱雖天子,其實如同傀儡皇帝,未敢多言,更怕又被桓溫所廢。當時司馬昱見熒惑入太微垣,因晉廢帝被廢時亦有同樣天象,故此十分不安,甚至對桓溫親信也是自己昔日僚屬的郗超問桓溫會否再行廢立之事。郗超斷言桓溫不會這樣作,司馬昱仍十分感慨,並詠庾闡之詩:「志士痛朝危,忠臣哀主辱。」司馬昱憂憤而得病,在咸安二年七月甲寅日(372年9月7日)因病急召桓溫入朝輔政,桓溫數度辭讓,司馬昱於是在己未日(9月12日)立兒子司馬昌明為太子。同日在東堂去世,享年五十三歲。臨終前,司馬昱寫了遺詔,要桓溫依周公先例居攝,更寫:「少子可輔者輔之,如不可,君自取之。」面對桓溫的野心,此舉幾近讓國。王坦之在司馬昱面前親手撕毀遺詔。司馬昱說:「晉室天下,只是因好運而意外獲得,你又對這個決定有甚麼不滿呢!」王坦之卻說:「晉室天下,是晉宣帝和晉元帝建立的,怎由得陛下你獨斷獨行!」司馬昱於是命王坦之改寫遺詔,寫道:「家國事一稟大司馬,如諸葛武侯、王丞相故事。」桓溫其實亦希望司馬昱臨終禪讓帝位給自己,又或者讓他像周公般居攝行事,王坦之改寫後的遺詔令桓溫大失所望。

司馬昱葬高平陵,廟號為太宗,諡為簡文皇帝。

简文帝外表清秀俊朗,擅长玄學清谈。尽管有文人雅士的风度,恬靜豁達,但政治手腕可说非常平庸,無濟世大略。谢安曾尖刻地评论:“比晉惠帝(以「何不食肉糜」著名的白痴皇帝)惟有清谈差胜耳!”謝靈運亦以他比作周赧王及漢獻帝等亡國之君。

司馬昱崇尚清談,長期坐著的胡床上即使積了灰塵也不清理。一次司馬昱發現有老鼠走過胡床的痕跡,覺得是好事。參軍見到有老鼠在白天走了出來,以手板將老鼠殺掉,司馬昱很不高興。當時門下的部屬就檢舉殺鼠的人,以圖取悅司馬昱,司馬昱卻說:「老鼠被殺,到現在還不能忘記;而現在又因老鼠而影響到他人,豈不是更不應該嗎?」可見其在醉心玄學之餘亦聰明有仁心。

司馬昱輔政時,一些政事拖了整年才得批准,桓溫覺得太慢,常常勸告司馬昱。但司馬昱卻說:「一日萬機,怎能快呀。」

王濛昔日請求當東陽太守,司馬昱不答應。及至王濛病重臨終,司馬昱就悲哀地說:「我將有負於仲祖(王濛字)呀!」下令命其為東陽太守。王濛說:「人們說會稽王痴心,真是痴心呀。」

司馬昱看見稻田,不知是甚麼,於是問左右是甚麼草,左右於是答那是稻。司馬昱事後三日沒有出外,說:「哪有依賴其結果而不知其根本。」

《晉書》載一次桓溫與司馬晞及司馬昱同車遊板桥,桓溫特意命人吹響號角,令馬匹受驚狂奔,藉此看兩人的反應。司馬晞當時大驚而想下車,而司馬昱就處之泰然。《世說新語》亦有類似記載。不過有認為司馬晞既然受桓溫所忌,不應有如此反應,這是對日後成為皇帝的司馬昱的溢美之作。

\subsection{咸安}

\begin{longtable}{|>{\centering\scriptsize}m{2em}|>{\centering\scriptsize}m{1.3em}|>{\centering}m{8.8em}|}
  % \caption{秦王政}\
  \toprule
  \SimHei \normalsize 年数 & \SimHei \scriptsize 公元 & \SimHei 大事件 \tabularnewline
  % \midrule
  \endfirsthead
  \toprule
  \SimHei \normalsize 年数 & \SimHei \scriptsize 公元 & \SimHei 大事件 \tabularnewline
  \midrule
  \endhead
  \midrule
  元年 & 371 & \tabularnewline\hline
  二年 & 372 & \tabularnewline
  \bottomrule
\end{longtable}



%%% Local Variables:
%%% mode: latex
%%% TeX-engine: xetex
%%% TeX-master: "../Main"
%%% End:
