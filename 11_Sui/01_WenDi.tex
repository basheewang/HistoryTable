%% -*- coding: utf-8 -*-
%% Time-stamp: <Chen Wang: 2021-10-29 15:26:08>

\section{文帝杨坚\tiny(581-604)}

\subsection{生平}

隋文帝杨坚(541年7月21日-604年8月13日),隋朝开国皇帝,谥号文帝,廟号高祖,西元581年3月4日-西元604年8月13日在位,在位24年。杨坚小字为那羅延(梵语,意为金剛不壞),鮮卑赐姓為普六茹,普六茹氏为其父杨忠受西魏恭帝所赐。掌权之後,下令“以前赐姓,皆复其旧”,恢复汉姓“杨”,并让宇文泰鲜卑化政策中改姓的汉人恢复汉姓。杨坚建立的隋朝,统一了全國。

杨坚出身于河东杨氏,于西魏大统七年六月十三癸丑夜(541年7月21日)生于冯翊般若寺,其父是西魏随国公、北周柱国、大司空杨忠,生母吕氏。其妻独孤皇后为北周时八大柱国之一独孤信之女獨孤氏。

其长女嫁北周宣帝(宇文贇)为后,地位显赫。杨坚在北周时曾官拜骠骑大将军,又封为大兴郡公,後袭父爵柱国,北周武帝时任隋州刺史,参加过北周灭北齐之战。

楊堅壯年時曾隨北周武帝(宇文邕)伐滅北齊統一華北,不久後武帝于宣政元年(578年)病逝,北周宣帝(宇文贇)即位。北周宣帝行为乖戾,诛杀元老重臣,将国政交给东宫的旧僚郑译,引起朝野恐慌。大成元年(579年)二月十九日,宇文赟下诏传位于仅7岁的长子宇文阐(北周静帝),并改年号为大象,自称天元皇帝。

大象二年(580年)3月31日,北周静帝宇文阐任命随国公杨坚世子杨勇为洛州总管、东京小冢宰,统辖故北齐旧地。在先前的2月20日,罢左、右丞相之官,改杨坚为大丞相。

大象二年(580年)5月11乙未日(6月8日),北周宣帝病逝。近臣刘昉、郑译等和杨坚有旧,并且杨坚有“重名”,矫诏谋引杨坚辅政。据记载杨坚开始是:“固辞,不敢当”。但是刘昉警告他:“公不为,昉自为之”,杨坚遂接受此建议,之后郑译等对北周宣帝病逝的死讯秘不发丧,便以太上皇的名义假传圣旨,以杨坚为总知中外兵马事,杨坚于是集军政大权于一身。

宣帝胞弟宇文赞为右大丞相,貌似尊崇,其实无实权,宇文赞本人年未及二十,见识庸劣,在刘昉蛊惑下,回府不理政事。杨坚手握军权后,恐北周诸王在外叛变,假借护送千金公主出嫁突厥为由,召赵王宇文招、陈王宇文纯、越王宇文盛、代王宇文达、滕王宇文逌进京朝见,并除去军权。

杨坚等人秘不发丧和假传圣旨的主张最初被司书上士颜之仪反对,但刘昉自知不能使颜之仪屈服,就代他签字,各诸卫在接到假诏书的情况下,军权就完全被杨坚控制。其后,杨坚又向颜之仪索取皇帝符节玉玺,但遭拒绝,遂大怒,想斩了颜之仪,但顾虑到颜之仪在民间声望很高,便把他贬到西边当郡守。

杨坚专政后不久,相州总管尉迟迥发兵讨伐杨坚,关东诸州群起响应,益州总管王谦也起兵进攻杨坚。杨坚派出韦孝宽、王谊等迅速平定叛乱,后又尽杀周室诸王。尉迟迥起兵时奉赵王宇文招留在赵国的幼子号令,宇文泰侄孙荥州刺史邵国公宇文胄响应,为杨素所败被杀。尉迟迥兵败后,宇文招幼子下落不明。

大象二年(580年)六月,明帝子雍州牧毕王宇文贤与五王(即宇文泰尚在世的五个儿子)谋杀杨坚,事泄,杨坚杀宇文贤及其三子,但对五王不问。七月,赵王宇文招布下鸿门宴,企图密谋暗杀杨坚,于是把杨坚邀请到寝室饮宴,命儿子宇文员、宇文贯以及王妃弟弟鲁封等身带佩刀站在左右,帷席之间都暗藏了兵器,在室后埋伏了壮士打算乘势刺杀杨坚。不料在杨坚身边心腹拓跋胄的掩护下没能成功。12月31日,杨坚在得知赵王宇文招想谋杀他,便诬陷赵王宇文招及越王宇文盛谋反,赵王和越王的诸子也皆被杨坚一起诛杀。十二月,又杀宇文泰仅剩的两个儿子代王宇文达、滕王宇文逌及二人诸子。同年并杀宇文泰孙冀王宇文绚。

大象二年(580年)12月2日,北周静帝宇文阐晋升随国公、大丞相杨坚为相国,总管全国文武百官;撤销全国都督、大冢宰之号称号,晋封随王,以安陆等二十郡为随国采邑,启奏时不再称名,享有九锡之礼;随国公、大丞相杨坚虚心“谦让”仅接受王爵和十郡作为采邑。直到大象三年(581年)2月4日,随王杨坚才接受了北周静帝宇文阐所赏赐的相国、总管全国文武百官、九锡之命,立随国文武百官。

大象三年(581年)2月14日,北周相国、随王杨坚“顺应人心”逼迫北周静帝宇文阐让出皇位登基,于是北周静帝下诏禅让,移居其他宫殿。2月19日,杨坚大开杀戒,把北周宇文皇族诸子全部屠之,李德林一再规劝却遭杨坚一句「君书生,不足与议此!”所拒,于是北周太祖宇文泰的孙儿谯公宇文乾恽;北周闵帝宇文觉孙纪公宇文湜;北周明帝宇文毓诸子酆公宇文贞、宋公宇文寔;北周武帝宇文邕诸子汉公宇文赞、秦公宇文贽、曹公宇文允、道公宇文充、蔡公宇文兑、荆公宇文元;北周宣帝宇文贇诸子莱公宇文衍、郢公宇文术等13人及他们的儿子们皆被处死。而李德林也因与杨坚意见不合,官职爵位从此没能升过。大定元年二月十四甲子日(581年3月4日),杨坚便以“受禅”的名义篡位称帝,受册、玉玺,改戴纱帽、身穿黄袍;入御临光殿,改国号为「隋」,定都大兴,是为隋朝也,改元开皇,大赦天下。

杨坚废北周静帝,自立为帝,改元开皇,建立隋朝。从他专政到称帝,前后不过10个月时间,“得國之易,無有如楊堅者”,楊堅之所以能这么快称帝,与北周末期军政大权迁移于汉人、受到汉族官员欢迎、北周府兵强大有关。

5月9日,杨坚又暗中派人杀死介公宇文阐(年仅9岁),后表示大为震惊,发布死讯,隆重祭悼,葬于恭陵。同年,杨坚又杀宇文泰侄曾孙豳国公宇文洽、宇文洽叔父杞国公宇文椿及其诸子等。宇文椿弟天水郡公宇文众不聪明不大说话,杨坚一度视其为继任介国公人选,但最终仍杀宇文众及其诸子,而另找远支族人宇文洛继嗣。

隋文帝在北防突厥(隋與突厥之戰)成功後,603年,隋朝击败占据漠北的达头可汗,次年,启民可汗称隋朝皇帝为圣人可汗,南迁的突厥部众成为隋朝的属国。隋文帝为归附的启民可汗率领的突厥人筑金河、定襄二城于河套。

杨坚平定叛乱之后,因为位于江陵一隅之地的西梁弱小且长期依附北朝,其统一天下的对手只剩下南方的陈朝。开皇七年九月十九辛卯日(587年10月26日),廢西梁後主蕭琮,西梁亡。

陈朝的兵力非常薄弱,据估计,只有十万人。 杨坚即位后,派贺若弼镇广陵、韩擒虎镇庐江,密谋灭陈。又派兵在陈国农时骚扰对方,纵火烧毁他们积蓄的粮食物资。开皇八年(588年),以杨广出六合、杨俊出襄阳、杨素带领水军出永安,共五十一万八千大军,三路大军伐陈。八年十二月杨素沿长江击破陈的沿江守军,顺流而东。但因为施文卿、沈客卿等扣留告急文书,导致陈朝无法把大军从建康调出。

九月,贺若弼和韩擒虎攻下京口、姑苏。沿江守军望风而逃,在建康城外陈军主力与贺若弼、韩擒虎八千部队激战,由于陈军不能合力,被击破。

开皇九年正月二十甲申日(589年2月10日),陈将任忠引韩擒虎攻入建康城,捉住陈叔宝,陈朝亡。

不久,各地陳軍或受陳後主號令投降、或抵抗隋軍而被消滅,只有嶺南地區受冼夫人保境据守。开皇十年(590年)八月,隋派使臣韦洸等人安撫嶺南,冼夫人率眾迎接隋使,嶺南諸州悉為隋地。至此,天下一统。

自西晋永嘉之乱(公元316年)以来,中原和江南地區政權分裂长达273年之久,至此隋文帝再造统一之局,并開創長達167年的隋唐盛世,直至安史之亂。

政治方面的支持功不可沒。漢人如鄭譯、蘇威、高熲等名臣有助推動國策。楊堅亦因前朝酷刑甚多,影響民生,故命蘇威等人編纂《開皇律》,修訂刑律,訂立國家刑法,使人民有法可守,又減省刑罰,死刑只設絞、斬二等,以示隋朝對民之寬大。在澄清吏治方面,楊堅得國以來,勵精圖治,兼且天資刻薄,自不容貪污枉法之行為存在。楊堅命柳盛持節巡省河北五十二州,奏免長吏贓污不稱者二百餘人,州縣肅然。吏治之整肅,不僅上裕國庫,下紓民困,隋高祖在位时之隆盛,此亦為要因。

军事上,隋文帝改变府兵制初设时,兵农分离情况。转变为和平时期府兵耕地种田,并在折冲将军领导下进行日常训练;战争发生时,由朝廷另派将领聚集各地府兵出征的“兵农合一”的制度。

地方行政方面,文帝鑑於南北朝政區劃分繁杂随意,地方行政交错混乱,支出龐大,楊堅遂於開皇三年(583年),盡罷諸郡,實行州縣二級制,使國家地方行政漸上軌道。誠如學者錢穆所言:開皇之治的成功,簡化地方行政機構是一個基本因素。據統計隋文帝时期朝廷開支減省三分之二,地方官府之開支減省四分之三,全國於行政之經費,大约是南北朝时期開支三分1而已。故隋國庫之豐積,不無原因。

经济上鑑於南北分裂達二百七十年之久,民生困苦,國庫空虛,由以中原地區為盛。故自隋開皇九年(589年),隋軍統一天下後,即以富國為首要目標。故楊堅接納司馬蘇威建議,罷鹽、酒專賣及入市稅,其後多次減稅,減輕人民負擔,促進國家農業生產,穩定經濟發展。隋文帝在位时代之富饒既非重歛於民,究其原因,與全國推行均田制有關。此舉既可增加賦稅,又可穩定經濟發展,且南朝士族亦漸由衰弱至於消逝。均田制能順利推行,對隋前中期的經濟發展收益甚大。輕徭薄賦以解民困。在確保國家賦稅收入之同時,穩定民生。由於魏晉南北朝以來,戶籍不清,稅收不穩。於是于开皇五年(585年)下令實行大索貌閱。並接納尚書左僕射高熲之建議,推行輸籍法,作全國性戶口調查,结果查获没有户籍的百姓达165万余口,其中丁壮44.3万人, 以增加國家稅收,改善經濟,盡掃魏晉南北朝以來隱瞞戶籍之積弊。

另外,隋文帝废除九品中正制,改为五省六曹制,後改稱五省六部制,是為唐代三省六部制之藍圖。中书、门下两省负责诏令的起草和封驳,尚书省负责政务的管理。尚书省又下设吏、戶、礼、兵、刑、工六部。吏部,掌管全国官吏的任免、考核、升降和调动;戶部,掌管全国的土地、户籍以及赋税、财政收支;礼部,掌管祭祀、礼仪和对外交往;兵部,掌管全国武官的选拔,和兵籍、军械等;刑部,掌管全国的刑律、断狱;工部,掌管各种工程、工匠、水利、交通等。

楊堅開了科舉制度之先河,他即位後,廢除了以前選官用的九品中正制,選官不問門第。規定各州每年向中央選送三人,參加秀才、明經等科的考試,合格者錄用為官。

科舉制度順應了歷代庶族地主在政治上得到應有的地位的要求,緩和了他們和朝廷的矛盾,使他們忠心擁戴中央,有利於選拔人才,增強政治效率,對封建專制中央集權的鞏固起了積極的作用。

人口在隋朝中前期大為增長,隋文帝开皇元年(581年)全国户口462万户,到隋炀帝大业五年(609年)达到8,907,536户,46,019,956人。其中在开皇九年(589年)南下平陈增50.0万,此时的全国户口700多万,平均年增长226,708户。

隋文帝統治残暴及滥杀大臣、企图独裁天下,他慢慢被大臣疏远。文帝殘暴专制,苛刻刑法,百姓惶恐。让“开皇盛世”大为失色。具体来说,功臣虞庆则、史万岁等人先后被杀,刑罚也逐渐变得严苛无情,不复隋朝初建时的依法行事。

他也对几个儿子进行打压,如将三子杨俊软禁,将四子杨秀贬为庶人。文帝未有鞏固太子的地位,並废长立幼,立杨广而废杨勇,而之後建立唐朝的唐高祖李淵也同樣犯下相同的錯誤,故之後修隋唐史者认为是隋朝败亡的重大原因。

長子楊勇為人忠厚善良,但生活奢華,因而受楊堅厭惡,常告誡太子楊勇說:“自古帝王未有好奢侈而能長久者。汝爲儲後,當以儉約爲先,乃能奉承宗廟。”次子杨廣(即隋煬帝)為人文采極高負有抱負,但善於作偽,在楊堅面前裝得很樸素,所以受到楊堅喜愛而代其兄為太子。

604年8月13日,杨坚病逝于仁寿宫大宝殿(一说被太子杨广所謀殺),享壽64岁,葬于泰陵(位于今陕西省杨凌示范区五泉乡双庙坡村,为皇帝杨坚与皇后独孤氏的合葬墓)。

齐王宇文宪曾对周武帝宇文邕评价杨坚:“普六茹坚相貌非常,人颇狡诈,臣每见之不觉自失,请早除之。”

隋朝作家李德林在《天命论》中描述隋文帝说:“帝体貌多奇,其面有日月河海,赤龙自通,天角洪大,双上权骨,弯回抱目,口如四字,声若钏鼓,手内有王文,乃受九锡。昊天成命,于是乎在。顾盼闲雅,望之如神,气调精灵,括囊宇宙,威范也可敬,慈爱也可亲。”

《隋书·帝纪二》评:“虽未能臻于至治,亦足称近代之良主。然天性沉猜,素无学术,好为小数,不达大体,故忠臣义士莫得尽心竭辞。其草创元勋及有功诸将,诛夷罪退,罕有存者。”又说他“唯妇言是用”、“喜怒不常,过于杀戮”。

初唐李延壽《北史》中讚美隋文帝外相:「皇考美鬚髯,身長七尺八寸,狀貌瑰偉,武藝絕倫,識量深重,有將率之略。」

初唐李延壽《北史》評隋文帝君臣失和、晚年聽信讒言與廢嫡長子,種下隋朝禍根: 「而素無術業,不能盡下,無寬仁之度,有刻薄之資,暨乎暮年,此風愈扇。又雅好瑞符,暗於大道。建彼維城,權侔京室,皆同帝制,靡所適從。聽妒婦之言,惑邪臣之說,溺寵廢嫡,託付失所。滅父子之道,開昆弟之隙,縱其尋斧,翦伐本根。墳土未乾,子孫繼踵為戮,松檟纔列,天下已非隋有。惜哉!跡其衰怠之源,稽其亂亡之兆,起自文皇,成於煬帝,所由來遠矣,非一朝一夕,其不祀忽諸,未為不幸也。」

北宋司馬光《資治通鑑》評隋文帝一生的功與過:「高祖性嚴重,令行禁止,勤於政事。每旦聽朝,日昃忘倦。雖嗇於財,至於賞賜有功,即無所愛;將士戰沒,必加優賞,仍遣使者勞問其家。愛養百姓,勸課農桑,輕徭薄賦。其自奉養,務為儉素,乘輿御物,故弊者隨令補用;自非享宴,所食不過一肉;後宮皆服浣濯之衣。天下化之,開皇、仁壽之間,丈夫率衣絹布,不服綾綺,裝帶不過銅鐵骨角,無金玉之飾。故衣食滋殖,倉庫盈溢。受禪之初,民戶不滿四百萬,末年,逾八百九十萬,獨冀州已一百萬戶。然猜忌苛察,信受讒言,功臣故舊,無始終保全者;乃至子弟,皆如仇敵,此其所短也。」

明朝官修皇帝实录《明太祖实录》记载,明太祖朱元璋在洪武七年八月初一日(1374年9月7日),亲自前往南京历代帝王庙祭祀三皇、五帝、夏禹王、商汤王、周武王、汉高祖、汉光武帝、隋文帝、唐太宗、宋太祖、元世祖一共十七位帝王,其中对隋文帝杨坚的祝文是:“惟隋高祖皇帝勤政不怠,赏功弗吝,节用安民,时称平治。有君天下之德而安万世之功者也。元璋以菲德荷天佑人助,君临天下,继承中国帝王正统,伏念列圣去世已远,神灵在天,万古长存,崇报之礼,多未举行,故于祭祀有阙。是用肇新庙宇于京师,列序圣像及历代开基帝王,每岁祀以春、秋仲月,永为常典。今礼奠之初,谨奉牲醴、庶品致祭,伏惟神鉴。尚享!”

\subsection{开皇}

\begin{longtable}{|>{\centering\scriptsize}m{2em}|>{\centering\scriptsize}m{1.3em}|>{\centering}m{8.8em}|}
  % \caption{秦王政}\
  \toprule
  \SimHei \normalsize 年数 & \SimHei \scriptsize 公元 & \SimHei 大事件 \tabularnewline
  % \midrule
  \endfirsthead
  \toprule
  \SimHei \normalsize 年数 & \SimHei \scriptsize 公元 & \SimHei 大事件 \tabularnewline
  \midrule
  \endhead
  \midrule
  元年 & 581 & \tabularnewline\hline
  二年 & 582 & \tabularnewline\hline
  三年 & 583 & \tabularnewline\hline
  四年 & 584 & \tabularnewline\hline
  五年 & 585 & \tabularnewline\hline
  六年 & 586 & \tabularnewline\hline
  七年 & 587 & \tabularnewline\hline
  八年 & 588 & \tabularnewline\hline
  九年 & 589 & \tabularnewline\hline
  十年 & 560 & \tabularnewline\hline
  十一年 & 561 & \tabularnewline\hline
  十二年 & 562 & \tabularnewline\hline
  十三年 & 563 & \tabularnewline\hline
  十四年 & 564 & \tabularnewline\hline
  十五年 & 565 & \tabularnewline\hline
  十六年 & 566 & \tabularnewline\hline
  十七年 & 567 & \tabularnewline\hline
  十八年 & 568 & \tabularnewline\hline
  十九年 & 569 & \tabularnewline\hline
  二十年 & 600 & \tabularnewline
  \bottomrule
\end{longtable}

\subsection{仁寿}

\begin{longtable}{|>{\centering\scriptsize}m{2em}|>{\centering\scriptsize}m{1.3em}|>{\centering}m{8.8em}|}
  % \caption{秦王政}\
  \toprule
  \SimHei \normalsize 年数 & \SimHei \scriptsize 公元 & \SimHei 大事件 \tabularnewline
  % \midrule
  \endfirsthead
  \toprule
  \SimHei \normalsize 年数 & \SimHei \scriptsize 公元 & \SimHei 大事件 \tabularnewline
  \midrule
  \endhead
  \midrule
  元年 & 601 & \tabularnewline\hline
  二年 & 602 & \tabularnewline\hline
  三年 & 603 & \tabularnewline\hline
  四年 & 604 & \tabularnewline
  \bottomrule
\end{longtable}


%%% Local Variables:
%%% mode: latex
%%% TeX-engine: xetex
%%% TeX-master: "../Main"
%%% End:
