%% -*- coding: utf-8 -*-
%% Time-stamp: <Chen Wang: 2019-12-23 17:25:02>

\section{恭帝杨侑(617-618)}
\subsection{生平}

隋恭帝杨侑(605年-619年9月14日),隋煬帝長子楊昭的第三子,母韦妃。李渊所立傀儡皇帝,入唐後改封酅國公。

当生于大业元年(605年),同年,其父杨昭被立为太子。大业二年(606年),父亲杨昭逝世。大业三年(607年),初封陳王,後改封代王。後隋煬帝幸江都,留其为长安留守。於太原留守的李淵發動晉陽起兵攻克长安之后,被李渊立为傀儡皇帝,尊杨广为太上皇。618年禅位於李淵,封楊侑為酅國公。唐朝武德夏五月(619年6月)去世,終年15歲,死后諡號隋恭帝。葬於陝西省乾縣陽洪鄉乳台村南800米處。

无子,以族子杨行基為螟蛉子,嗣酅國公。

唐魏徵《隋書》評價楊侑年幼遭逢大難,一生受到擺佈,所謂禪讓也是不得之行為:「史臣曰:恭帝年在幼沖,遭家多難,一人失德,四海土崩。羣盜蜂起,豺狼塞路,南巢遂往,流彘不歸。既鍾百六之期,躬踐數終之運,謳歌有屬,笙鍾變響,雖欲不遵堯舜之迹,其庸可得乎!」

\subsection{义宁}

\begin{longtable}{|>{\centering\scriptsize}m{2em}|>{\centering\scriptsize}m{1.3em}|>{\centering}m{8.8em}|}
  % \caption{秦王政}\
  \toprule
  \SimHei \normalsize 年数 & \SimHei \scriptsize 公元 & \SimHei 大事件 \tabularnewline
  % \midrule
  \endfirsthead
  \toprule
  \SimHei \normalsize 年数 & \SimHei \scriptsize 公元 & \SimHei 大事件 \tabularnewline
  \midrule
  \endhead
  \midrule
  元年 & 617 & \tabularnewline\hline
  二年 & 618 & \tabularnewline
  \bottomrule
\end{longtable}

\section{秦王杨浩(618)}
\subsection{生平}

杨浩(6世纪-618年)是中国隋朝皇帝,隋文帝之孙,隋炀帝之姪,秦孝王楊俊之子,母亲是王妃崔氏。

父亲杨俊好声色,母亲崔氏很忿怒,下毒谋杀丈夫杨俊。杨俊在600年逝世,事后母亲崔氏被处死,杨浩也因此受连坐,被取消了世子资格,甚至不能为父亲主丧。至其伯父隋炀帝即位,才允许杨浩继承爵位为秦王,后又用为河阳都尉,左翊卫大将军宇文述讨伐作乱的杨玄感时到河阳和杨浩有来往,有司弹劾杨浩交通内臣,杨浩因而获罪被免。

618年三月,炀帝在江都被宇文化及所弑,宗室多遇害,杨浩因与宇文化及弟宇文智及交好得以保全。宇文化及自称大丞相,总百揆,以萧皇后名义擁立杨浩繼承大寶,他其实是一个傀儡皇帝,居于别宫,負責簽發詔令、敕书而已,宇文化及以兵监守之。四月,宇文化及将杨浩交给尚书省,令卫士十余人看守,遣令史取其画敕,百官不复朝参。九月,宇文化及决定称帝,遂毒死杨浩,自立为皇帝。

\subsection{大业}

\begin{longtable}{|>{\centering\scriptsize}m{2em}|>{\centering\scriptsize}m{1.3em}|>{\centering}m{8.8em}|}
  % \caption{秦王政}\
  \toprule
  \SimHei \normalsize 年数 & \SimHei \scriptsize 公元 & \SimHei 大事件 \tabularnewline
  % \midrule
  \endfirsthead
  \toprule
  \SimHei \normalsize 年数 & \SimHei \scriptsize 公元 & \SimHei 大事件 \tabularnewline
  \midrule
  \endhead
  \midrule
  十四年 & 618 & \tabularnewline
  \bottomrule
\end{longtable}

\section{皇泰主杨侗\tiny(618-619)}
\subsection{生平}

皇泰主(?-619年7月),名為杨侗,隋炀帝之孙,元德太子楊昭次子,母小劉良娣,618年6月22日至619年5月23日在位,年号皇泰,是隋朝最後一位君主。

杨侗原本封為越王,駐守东都洛陽。618年4月11日,隋煬帝被弑,消息传到东都后,皇泰元年五月二十四戊辰日(618年6月22日),王世充與元文都、盧楚等拥立杨侗为皇帝,史稱皇泰主。楊侗以世充為吏部尚書,封鄭國公,與陳國公段達、內史令元文都、內史侍郎郭文懿、黃門侍郎趙長文、內史令盧楚、兵部尚書皇甫無逸等六人共同輔政,時人號稱「七貴」。元文都欲暗殺王世充,段達暗中通知王世充,結果行刺失敗,元文都臨死前對楊侗說:“臣今朝死,陛下夕及矣”,楊侗亦哭。

皇泰二年四月初五癸卯日(619年5月23日),王世充罷黜楊侗,将楊侗囚於含涼殿,两天后,王世充自稱皇帝,建立鄭朝,改元開明,改封楊侗為潞國公,楊侗每日只能焚香禮佛以祈求平安。五月,王世充部將裴仁基、裴行儼父子策劃攻殺王世充,改立楊侗為君。事情敗露,王世充殺死裴仁基父子。

世充意圖毒殺楊侗,六月派侄子王仁則和家僕梁百年,攜鴆酒去楊侗處,楊侗自知難逃一死,向佛像祈禱:“願生生世世不要再生在帝王之家”,遂仰药;但一時半刻竟沒毒發,最後被縊死,谥号恭皇帝。


\subsection{皇泰}

\begin{longtable}{|>{\centering\scriptsize}m{2em}|>{\centering\scriptsize}m{1.3em}|>{\centering}m{8.8em}|}
  % \caption{秦王政}\
  \toprule
  \SimHei \normalsize 年数 & \SimHei \scriptsize 公元 & \SimHei 大事件 \tabularnewline
  % \midrule
  \endfirsthead
  \toprule
  \SimHei \normalsize 年数 & \SimHei \scriptsize 公元 & \SimHei 大事件 \tabularnewline
  \midrule
  \endhead
  \midrule
  元年 & 618 & \tabularnewline\hline
  二年 & 619 & \tabularnewline
  \bottomrule
\end{longtable}


%%% Local Variables:
%%% mode: latex
%%% TeX-engine: xetex
%%% TeX-master: "../Main"
%%% End:
