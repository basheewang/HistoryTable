%% -*- coding: utf-8 -*-
%% Time-stamp: <Chen Wang: 2019-12-23 17:16:53>

\section{炀帝\tiny(604-618)}

\subsection{生平}

隋炀帝杨广(569年-618年4月11日),又名英,小字阿𡡉。隋文帝杨坚和文献皇后獨孤伽羅的次子,是隋朝第二位皇帝。隋恭帝杨侑諡杨广为炀帝;夏王窦建德諡杨广为闵帝;皇泰主杨侗諡杨广为明帝,庙号世祖。炀帝十三岁被封为晋王,兼任并州主管。

隋炀帝於604年8月21日由楊素協助登基,在位期间加强了中央集权,扩大了统治的社会基础。但他好大喜功及經常南征北伐,据研究仅从公元604年至608年短短4年间就动用了近540万民力修建大运河(开凿通济渠、永济渠),长城和洛阳城。又西巡张掖、亲征吐谷浑、以厚利诱使西域商贾至洛阳,大业七年(公元611年)引发民众乃至贵族大规模的起义——隋末民变,更于大业八年(612年)征集三十万军队攻打高句丽(不包括后勤100多万民力),几乎动用了举国之力,618年杨广在江都被部下缢杀。

隋炀帝在位期間,将科举制度(科举制萌芽于魏晋南北朝)正式归为国家政策,对后世有重大影响,此后历代均以科举而選拔人才,隋唐大運河是世界史上最長的運河。但是隨着時間的推移,隋朝大運河部分河段失去通航功能,被元世祖忽必烈所修的京杭大运河代取代。其他功绩如讨伐吐谷浑(隋炀帝609年攻灭吐谷浑,但到了615年吐谷浑可汗伏允在西海、河源、鄯善、且末四郡复国),讨占城(隋炀帝605年攻占城,随着军队班师后,占城王商菩跋摩遂在比景、海阴、林邑三郡故地复国。此战隋军死者什四、五,指揮官刘方也病死于班师途中),讨高句丽(三战均以失败告终)则对后世影响较小。

隋炀帝即位后几乎每年征发重役。仁寿四年十一月,他为了开掘长堑拱卫洛阳,调发今山西、河南几十万农民;次年营建东都洛阳,每月役使丁男多达两百万人;自大业元年至六年,开发了各段运河,先后调发河南、淮北、淮南、河北、江南诸郡的农民和士兵三百多万人;大业三年和四年在榆林(今内蒙古托克托西南)以东修长城,两次调发男丁一百二十万,役死者过半。总计十余年间被征发扰动的农民不下一千万人次,平均每户就役者一人以上,造成“天下死于役”的惨象。隋炀帝年年远出巡游,曾三游江都,两巡塞北,一游河右,三至涿郡,还在长安、洛阳间频繁往还。

杨广,一名杨英,小字阿摐,华阴(今陕西华阴)人,隋文帝杨坚次子,母为文献皇后独孤伽罗,北周天和四年(569年)生于长安(今陕西省西安市),史书称其“美姿仪,少聪慧”,很得双亲喜爱。在北周时因父杨坚的功勋,被封为雁门郡公。开皇元年(581年)封晋王,拜柱国、并州总管,时年十三岁。后又授武卫大将军,进位上柱国、河北道行台尚书令等。

开皇六年(586年),轉淮南道行臺尚書令。其年,徵拜雍州牧、內史令。开皇八年(588年)冬天,隋朝兴兵平南朝陈,二十岁的杨广是领衔的统帅,真正在前线作战的是贺若弼和韩擒虎等名将。次年平陈之后,进驻建康,意欲纳陈后主宠妃张丽华为妾,元帅长史高颎斩杀张丽华故作罢。封存府库,将陈叔宝及其皇后等人带返隋京。班师后,杨广进封太尉。平陈之后因为隋朝政策有所偏差,江南各地叛乱。杨广从并州改镇扬州,在镇守江南期间为稳定叛变局势颇有成效、政绩突出。同时他表现得作风简朴、不好声色、礼贤下士、谦恭谨慎,由此赢得了朝野赞颂和隋文帝的欢心。

隋文帝和皇太子杨勇的矛盾逐渐爆发,杨广趁机而入争夺储位,博得了独孤文献皇后和右仆射杨素的支持。开皇二十年(600年),隋文帝下诏废黜长子杨勇,立次子晋王杨广为皇太子,杨广率军北上击破突厥的攻势。

仁寿四年(604年)七月,隋文帝驾崩,36岁的杨广即帝位,君临天下。即位后假造隋文帝遗诏缢杀兄长废太子杨勇。次年,其弟汉王杨谅以讨伐杨素为名,在并州起兵,炀帝派杨素镇压,杨谅降后被幽禁至死。不久,炀帝听从云定兴建议毒死杨勇诸子,翦除了对帝位的威胁。大业三年(607年),为了招待突厥来使,炀帝下令宇文恺制作下面可容3000人的帐蓬,立于城东,高颎与贺若弼见隋炀帝奢靡,甚为忧虑,有所议论,为人告发而被杀害。

隋炀帝以早年的军旅生涯为基础,对高句丽、吐谷浑和突厥发动了战争。大业五年(609年),他亲征平定吐谷浑,设置西海、河源、鄯善、且末四郡,阔疆五千里。高昌王麴伯雅也到张掖朝见炀帝。炀帝命在河源郡驻兵屯田。当时,全国有一百九十郡,一千二百五十五县,在籍户八百九十万余,人口四千六百余万,隋朝达于极盛。

高句丽地跨鸭绿江两岸,位于今中国辽宁东部、吉林中部和朝鲜北部,隋炀帝即位后,三次大举进攻高句丽。大业八年(612年),隋炀帝第一次进攻高句丽,征调士卒一百一十三万余,陆军集中于涿郡(今北京),水军集中于东莱(今山东莱州)。另调民伕二百万,以运送衣甲、粮食等。造海船的民工日夜站在水中,皮肤溃烂,腰以下生蛆,死者甚众。隋军虽然曾攻至平壤附近,最后却大败而还。大业九年(613年)第二次进攻高句丽,正当双方相持不下时,礼部尚书杨玄感起兵叛隋,隋炀帝仓皇撤军。大业十年(614年),第三次进攻高句丽。隋炀帝因国内民變已成燎原之势而不敢久战,高句丽也疲于战争而遣使请降,隋炀帝就此撤军。进攻高句丽的战争,先后动用人力数百万,征调财物无数,大量士兵、民伕死于战场和劳役,由于农村中极度缺乏劳力和耕畜,大量土地荒芜,社会经济受到严重破坏,人民难以生活下去,成为隋末民變的导火线。隋大业十一年(公元615年),隋炀帝依例北巡长城,始毕可汗率兵将其围困在雁门(今山西代县),炀帝派人向始毕之妻、隋义成公主求救,公主遣使告知始毕“北边有急”,加上隋朝援军相继抵达,始毕在九月撤围而去。

隋炀帝为了满足其骄奢淫逸的生活,在各地大修宫殿苑囿、离宫别馆。其中著名的有显仁宫、江都宫、临江宫、晋阳宫、西苑等。西苑在洛阳之西,周围二百余里,苑内有人工湖,周围十余里,湖内有山,堂殿楼观,布置奇巧,穷极华丽。隋炀帝常在月夜带宫女数千人骑马游西苑,令宫女在马上演奏《清夜游》曲,弦歌达旦。炀帝游江都时,率领诸王、百官、后妃、宫女等一、二十万人,船队长达二百余里,所经州县,五百里内都要贡献食物,挥霍浪费的情况十分严重。

隋炀帝为夸耀国家富强,每年正月当少数民族和外国首领、商人聚集洛阳时,命人在洛阳端门外大街上盛陈百戏散乐,戏场绵亘八里,动用歌伎近三万人,乐声传数十里外。西域商人要到市上交易,炀帝就下令盛饰市容,装璜店肆,房檐一律,珍货充积,连卖菜的都要垫以龙须席。当这些商人从酒店饭馆前经过时,都要请他们就坐用餐。并说:“中国丰饶,酒食例不取直(值)。”还将市上树木缠以丝织品做装饰。有些胡商说:“中国亦有贫者,衣不盖形,何如以此物与之!缠树何为?”。

隋炀帝即位后,造龙舟等各种船数万艘。他游江都时所乘龙舟高四十五尺,阔五十尺,长二百尺,上有四层楼,上层有正殿、内殿、东西朝堂,中间两层有房一百二十间,下层为内侍居处。

大业元年(605年),隋炀帝开始营建东都,历时十个月,每月征调民夫二百万人。东都在旧洛阳城之西,规模宏大,周长五十余里,分为宫城、皇城、外郭城等三部分。宫城是宫殿所在地,皇城是官衙所在地,外郭城是官吏私宅和百姓居处所在地。外郭城有居民区一百余坊,另有丰都市、大同市、通远市等三大市场。隋炀帝常住洛阳,将其作为东方的政治、军事、经济中心。

隋炀帝在营建东都的同时,又下令开凿大运河。隋朝大运河以洛阳为中心,分为三大段。中段包括通济渠与邗沟。通济渠北起洛阳,东南入淮水。邗沟北起淮水南岸之山阳(今江苏淮安),南达江都(今扬州)入长江。南段名江南河,北起长江南岸之京口(今镇江),南通余杭(今浙江杭州)。北段名永济渠,南起洛阳,北通涿郡(今北京城西南)。大运河分段开凿,前后历时五年,全长两千七百余公里,是世界著名的伟大工程之一,后经元朝取直疏浚,全长1794公里,成为现今的京杭大运河,京杭大运河利用了隋朝大运河不少河段,缩短了900多公里的航程。开凿大运河的目的是为了加强中央对东方和南方的统治,同时也是为了从南方漕运粮食和便利对东北用兵。大运河对中国南北的经济、文化交流和巩固国家的统一都起了巨大的作用。

在教育制度上,隋炀帝发展科举制度,增置进士科,使国务的操持由世族门阀政治而逐渐改向科举取士。科举制度一直延续到清德宗光绪三十一年(1905年)才被终止,为古代中国的育才政策有很大的贡献。政治上,他企图打破由关陇集团垄断仕途的局面,重用了虞世基、裴蕴等南方集团官员。军事上,在即位前曾参与与突厥、契丹之战事,皆有所获。但是隋炀帝又是一位急功近利的人。大业八年(612年),首度亲征高句丽失败后,隋炀帝为扳回其颜面,连续三年一再亲征。即位后为实现个人构想,劳动全国投入新宫殿的营造,皇宫用金玉装饰,金碧辉煌,致使国库亏空,劳民伤财。晚年为消除强烈的失落感和政务上的压力,逃避现实,三下江都,远离朝政。

大业七年(611年),王薄率领民众在长白山(今山东章丘东北)起义,隋末民变终于爆发。杨玄感起兵后,民變发展为全国规模。隋炀帝意图遏止起义力量发展,下令各地郡县、驿亭、村坞筑城堡,将民众迁往城堡中居住,于近处种田,以图控制。他不愿正视民變蓬勃发展的现实,身边的佞臣也不以实情相告,谎称造反的民眾“渐少”。大业十二年(616年)七月,隋炀帝从东都去江都。次年四月,魏公李密率领的瓦岗军逼围东都,并且向各郡县发布檄文,历数隋炀帝十大罪状。隋炀帝在江都却越发荒淫昏乱,命王世充挑选江淮民间美女充实后宫,每日酒色取乐,又引镜自照,预感末日将到,锐意尽失的隋炀帝晚年常引镜自照,对萧皇后和臣下说:“好头颈,谁当斫之!”

大业十四年三月十一丙辰日(618年4月11日),隋炀帝于江都被叛军宇文化及所弑,终年五十岁。隋炀帝死前,宇文化及煽动叛军将之包围,炀帝闻变,仓皇换装,逃入西阁。炀帝因与次子齐王杨暕彼此猜忌,此时竟然以为作乱者是杨暕,对萧皇后说“是不是阿孩?”炀帝被叛军裴虔通、元礼、马文举等逮获,说:“朕实负百姓,至于尔辈,荣禄兼极,何乃如是!今日之事,孰为首邪?”宇文化及命令封德彝宣布炀帝罪状。炀帝说:“卿乃士人,何为亦尔?”德彝一时惭愧退下。炀帝爱子赵王杲,才十二岁,在帝侧,号恸不已,叛军裴虔通将其斩杀,血溅御服。炀帝自知难逃一死,说天子自有天子的死法,欲饮毒酒自尽,叛军马文举等不许,遂命令令狐行达将其缢弑。宇文化及等并杀炀帝孙杨倓、杨暕及其二子、杨秀及其七子等。

唐朝谥杨广为炀皇帝,隋恭帝杨侗谥世祖明皇帝,夏王窦建德谥闵皇帝。隋江都太守陈棱找到炀帝灵柩,粗备天子仪卫,改葬于江都宫西吴公台下,当时牺牲的王公,皆埋葬在炀帝坟茔的两侧。唐平江南后,以帝礼改葬雷塘。唐代以后,煬帝陵所在不为人知。

清嘉庆时,原籍为今江苏省扬州市邗江区槐泗镇的大学士阮元考证槐泗镇槐二村一处大土墩为炀帝陵,并出资修复,扬州知府伊秉绶书写墓碑。1995年成为江苏省文物保护单位。2013年4月,有报道称在扬州市邗江区最近发掘的两座古墓中,一座的墓志铭显示墓主为隋炀帝杨广。经过半年时间的考古发掘论证,2013年11月16日,中国考古学会召开新闻发布会,确认扬州市邗江区西湖镇司徒村曹庄组隋唐墓葬为隋炀帝墓,是隋炀帝杨广与萧后最终的埋葬之地。

588年(开皇八年)以行军元帅身份参加平陈战争

589年(开皇九年)占领建康,灭陈

590年(开皇十年)出任杨州总管,镇守江都,

599年(开皇十九年)离开江都入朝,

600年(开皇二十年)出灵武道,抗击突厥达头可汗

605年(大业元年)营建东都洛阳,开修通济渠,八月坐船游江都(扬州)

606年(大业二年)四月驾返洛阳

607年 北巡榆林;

608年 第四年至五原,出长城,巡行至塞外;

609年 西行到张掖;

610年 再游江都;

611年 到614年,三次亲征高丽,均遭失败;

615年 北巡长城,被突厥始毕可汗围困于雁门;

616年 三游江都;

618年 被叛军所弑,同年李淵建立唐朝,隋朝滅亡。

隋炀帝在位的十四年间,起初为提升经济发展和民生便利的层次,曾推动各种建筑包括南北隋唐大运河等艰钜工程。一方面这些巨大的工程促进了经济,另一方面也给民生带来沉重的负担。但大运河的修建使百万计的中国劳工伤亡,甚至出現了“丁男不供,始役妇人”的情況,为隋朝带来不稳定因素。另外,在执行政策的同时,隋炀帝也搜罗江南、五岭以北的珍材异石来犒赏自己。又为求完美,令各地献上特产、奇禽异兽至京,动用了大量的劳役。隋炀帝还遣大军远征高句丽。连年的征战,使百万隋军丧命异国他乡并间接引发隋末民变。

大业元年(605年),杨广继位之初,征发河南、淮北一百多万人开通济渠(唐时称广济渠,宋称汴河),由洛阳通到淮水。同年,又遣淮南十几万人开邗沟,从山阳(今江苏淮安)到扬子(今江苏扬州南)入江,又称“山阳渎”。自大兴至江都(今扬州),全长四千多里。运河的两旁开辟了大道,为美化环境和鼓励人民亲近利用,皆种有榆树和柳树,可谓当代良策。大业四年(608年),征河北一百多万人开永济渠,引沁水南达黄河,北通涿郡(今北京)。大业六年(610年),开江南河,从京口通到余杭(今浙江杭州)。自大业元年(605年)起,以六年时间开凿邗沟、通济渠、永济渠和江南运河。

大业七年(611年),大运河建成后,隋炀帝随后于大业八年(612年),募集30万人的作战军队攻打高句丽。高句丽全国亦撼动,隋军都认为易如反掌,结果在辽东城和平壤城伤亡惨重,大败而归。次年再度发兵围攻辽东城,但国内杨玄感叛变,隋炀帝不得不中返平乱。大业十年(614年),第三次发兵进攻高句丽,高句丽王高元不敌,只有投降,隋炀帝便班师回朝,耗尽国力,民间烽火遍地,不久灭亡。

隋炀帝除了在位施政及功過饱受争议之外,他还是隋唐两代代表性的诗人之一。他的诗风广阔,既有千军万马出征时的雄伟,又能描写夕阳下长江宁静的江景;在他帝王生涯的最后,彷彿意识到自己帝王运尽,诗风转变为寂寥多感,主以抒情诗为主。

乐府春江花月夜二首其一“暮江平不动,春花满正开。流波将月去,潮水带星来。”

杨广继位后,假传文帝遗嘱,逼迫杨勇自尽,将杨勇处死。还有亲弟蜀王杨秀被他诬陷使用巫蛊诅咒隋文帝及幼弟汉王杨谅,被剥夺官爵贬为庶民软禁于内侍省,后与诸子一起被软禁,不得与妻子相见。杨广将起兵造反的幼弟汉王杨谅“除名为民,绝其属籍”后,于大业三年(607年)3月4日,诛杀侄兒长宁王杨俨、又把剩余的侄兒(杨勇诸子)平原王楊裕、安城王杨筠、安平王杨嶷、襄城王杨恪、高阳王杨该、建安王杨韶、颍川王杨煚、杨孝宝、杨孝范贬到岭南,在路途中全部被处死。

据《资治通鉴180》载,在仁寿四年(604年)7月,文帝卧病在床,杨广于是写信给杨素,请教如何处理文帝后事和自己登基事宜。不料送信人误将杨素的回信送至了文帝手上。文帝大怒,随即宣杨广入宫,要当面责问他。正在此时,宣华夫人陈氏也哭诉杨广在她来途中意圖非禮她,使文帝顿悟,拍床大骂:“畜生何足付大事!独孤误我!”急忙命人传大臣柳述(文帝女婿)、元岩草拟诏书,废黜杨广,重立杨勇为太子。杨广得知后将柳述、元岩抓入狱,并让右庶子张衡入文帝寝殿侍疾并将文帝周围的侍从打发走。據傳文帝就是他親手所殺。不久文帝便驾崩。

杨广弑父在《隋书列传第十三》杨素传、《隋书列传第十》杨勇传、《隋书列传第二十一》张衡传、《隋书·后妃列传》等《隋书》章节中也有多处记载。

唐代人马总在《通历》中记载隋文帝被张衡“血溅屏风”,而赵毅在《大业略记》中记载隋文帝被张衡毒死。

仁寿四年(604年),隋文帝病重,宣华夫人与容华夫人都侍立在侧。隋文帝命兩人更衣小憩。宣华夫人更衣时,遇上太子杨广,杨广见父皇病重,便色性大發,上前非礼庶母。宣华夫人挣脱了杨广的纠缠。她衣履不整地赶回仁寿宫,向隋文帝哭诉杨广的无礼。隋文帝大怒,大罵“畜生何足付大事,獨孤誤我!”便命內侍急召兵部尚書柳述、黃門侍郎元岩,商討廢太子楊廣,扶楊勇為太子,但楊廣命張衡入宮,不久隋文帝即崩。不過初唐趙毅筆記《大業略記》記載容華夫人蔡氏為仁壽宮變女主角。隋煬帝楊廣「因色弒父」這種說法也被諸多近代史學家質疑,仁壽宮變也成為疑案。

隋文帝驾崩,杨广送去同心結,宣华夫人害怕,但在宮人催促下,勉强收下同心结。当夜,杨广姦淫後母宣华夫人、容华夫人。萧皇后發現此事,在宣华夫人面前斥責杨广姦淫庶母的罪行,又威脅若不送走宣华夫人,便將此事公诸天下,最後杨广只好把宣华夫人送到仙都宫居住。但他对宣华夫人念念不忘,不久又把宣华夫人迎回宫中,但她回宫一年多便病逝,终年二十九岁。


\subsection{大业}

\begin{longtable}{|>{\centering\scriptsize}m{2em}|>{\centering\scriptsize}m{1.3em}|>{\centering}m{8.8em}|}
  % \caption{秦王政}\
  \toprule
  \SimHei \normalsize 年数 & \SimHei \scriptsize 公元 & \SimHei 大事件 \tabularnewline
  % \midrule
  \endfirsthead
  \toprule
  \SimHei \normalsize 年数 & \SimHei \scriptsize 公元 & \SimHei 大事件 \tabularnewline
  \midrule
  \endhead
  \midrule
  元年 & 605 & \tabularnewline\hline
  二年 & 606 & \tabularnewline\hline
  三年 & 607 & \tabularnewline\hline
  四年 & 608 & \tabularnewline\hline
  五年 & 609 & \tabularnewline\hline
  六年 & 610 & \tabularnewline\hline
  七年 & 611 & \tabularnewline\hline
  八年 & 612 & \tabularnewline\hline
  九年 & 613 & \tabularnewline\hline
  十年 & 614 & \tabularnewline\hline
  十一年 & 615 & \tabularnewline\hline
  十二年 & 616 & \tabularnewline\hline
  十三年 & 617 & \tabularnewline\hline
  十四年 & 618 & \tabularnewline
  \bottomrule
\end{longtable}


%%% Local Variables:
%%% mode: latex
%%% TeX-engine: xetex
%%% TeX-master: "../Main"
%%% End:
