%% -*- coding: utf-8 -*-
%% Time-stamp: <Chen Wang: 2019-12-17 16:24:28>

\section{明帝\tiny(57-75)}

\subsection{生平}

漢明帝劉莊(28年6月15日-75年9月5日),原名刘阳,字子丽,东汉第二位皇帝,在位十八年。其正式諡號為「孝明皇帝」,後世省略「孝」字稱「漢明帝」,庙号显宗。汉光武帝刘秀的第四子,母亲为光烈皇后阴丽华。

汉明帝生于建武四年五月甲申(28年6月15日)。他从小就聪明好学,十岁时能够通读《春秋》。

建武十五年(39年)封东海公,十七年(41年)进爵为东海王,十九年(43年)被立为皇太子。建武中元二年初五戊戌(57年3月29日),三十岁的刘庄即皇帝位。

明帝即位后,一切遵奉汉光武帝的制度。明帝热心提倡儒学,注重刑名文法,为政苛察,总揽权柄,权不借下。他严令后妃之家不得封侯与政,对贵戚功臣也多方防范。同时,基本上消除了因为王莽虐政而引起的周边蛮夷侵扰的威胁,使汉跟周边蛮夷的友好关系得到了恢复和发展。

明帝允北匈奴互市之请,但并未消弥北匈奴的寇掠,反而动摇了早已归附的南匈奴。只得改变光武时期息兵养民的策略,重新对匈奴开战。永平十六年(73年),命祭肜、窦固、耿秉、來苗征伐北匈奴,汉军进抵天山,击呼衍王,斩首千余级,追至蒲类海(今新疆巴里坤湖),取伊吾卢地。永平十七年(74年),命窦固、耿秉、劉張征白山虜於蒲類海,复置西域都护府,用来管辖西域地区。其后,窦固又以班超出使西域,由是西域诸国皆遣子入侍。自新朝地皇四年(23年)以来,西域与中原断绝关系50年后又恢复了正常交往。班超以三十六人征服鄯善、于寘诸国、耿恭守疏勒城力拒匈奴等故事都发生在这一时期。

此外,随着对外交往的正常发展,佛教已在西汉末年传入西域,永平十年(67年),明帝梦见金人,其名曰佛,于是派使者赴天竺求得其书及沙门,并于雒阳建立中国第一座佛教庙宇白马寺。

明帝之世,吏治非常清明,境内安定。加以多次下诏招抚流民,以郡国公田赐贫人、贷种食,并兴修水利。因此,史书记载当时民安其业,户口滋殖。据《后汉书》记载:光武帝建武中元二年(57年),人口为2100万,至汉明帝永平十八年(75年),在不到20年的时间里增加至3412万。明帝以及随后的章帝在位时期,史称“明章之治”。

永平十八年八月初六壬子(75年9月5日),汉明帝逝世于雒阳东宫前殿,终年四十八岁。八月壬戌(9月15日),葬于显节陵(今河南洛阳市东南)。庙号显宗,谥号孝明皇帝。

\subsection{永平}

\begin{longtable}{|>{\centering\scriptsize}m{2em}|>{\centering\scriptsize}m{1.3em}|>{\centering}m{8.8em}|}
  % \caption{秦王政}\
  \toprule
  \SimHei \normalsize 年数 & \SimHei \scriptsize 公元 & \SimHei 大事件 \tabularnewline
  % \midrule
  \endfirsthead
  \toprule
  \SimHei \normalsize 年数 & \SimHei \scriptsize 公元 & \SimHei 大事件 \tabularnewline
  \midrule
  \endhead
  \midrule
  元年 & 58 & \tabularnewline\hline
  二年 & 59 & \tabularnewline\hline
  三年 & 60 & \tabularnewline\hline
  四年 & 61 & \tabularnewline\hline
  五年 & 62 & \tabularnewline\hline
  六年 & 63 & \tabularnewline\hline
  七年 & 64 & \tabularnewline\hline
  八年 & 65 & \tabularnewline\hline
  九年 & 66 & \tabularnewline\hline
  十年 & 67 & \tabularnewline\hline
  十一年 & 68 & \tabularnewline\hline
  十二年 & 69 & \tabularnewline\hline
  十三年 & 70 & \tabularnewline\hline
  十四年 & 71 & \tabularnewline\hline
  十五年 & 72 & \tabularnewline\hline
  十六年 & 73 & \tabularnewline\hline
  十七年 & 74 & \tabularnewline\hline
  十八年 & 75 & \tabularnewline
  \bottomrule
\end{longtable}


%%% Local Variables:
%%% mode: latex
%%% TeX-engine: xetex
%%% TeX-master: "../Main"
%%% End:
