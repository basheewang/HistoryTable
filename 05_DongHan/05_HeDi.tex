%% -*- coding: utf-8 -*-
%% Time-stamp: <Chen Wang: 2019-12-17 17:22:02>

\section{和帝\tiny(88-105)}

\subsection{生平}

汉和帝刘肇(79年-106年2月13日),东汉第四位皇帝(88年4月9日—106年2月13日在位),在位17年,得年僅27岁,其正式諡號為「孝和皇帝」,後世省略「孝」字稱「漢和帝」,他是章帝第四子,母贵人梁氏,死後庙号穆宗,葬于慎陵。

建初四年(79年),梁贵人生刘肇。皇后窦氏将刘肇养为己子。建初七年(82年),汉章帝废太子刘庆,立刘肇为皇太子。

章和二年二月三十壬辰(88年4月9日),汉章帝逝世,刘肇即位,是为汉和帝。当时他只有十岁,由养母窦太后执政,窦太后排斥异己,让哥哥窦宪掌权,窦家人一犯法,窦太后就再三庇护,窦氏的专横跋扈,引起汉和帝的不满。永元四年壬辰年六月二十三日(92年8月14日),汉和帝联合宦官鄭眾将窦氏一网打尽,但也导致“于是中官始盛焉”。

在一举扫平了外戚窦氏集团的势力之后,汉和帝开始亲理政事,他每天早起临朝,深夜批阅奏章,从不荒怠政事,故有「劳谦有终」之称,但却因而积劳成疾,加上和帝本身已体弱多病,所以年仅二十七岁便英年早逝,从他亲政后的政绩,不失为一代贤君英主。和帝当政时期,曾多次下诏赈济灾民、减免赋税、安置流民、勿违农时,并多次下诏纳贤,在法制上也主张宽刑,并在西域复置西域都护。汉和帝十分体恤民众疾苦,多次诏令理冤狱,恤鳏寡,矜孤弱,薄赋敛,告诫上下官吏认真思考造成天灾人祸的自身原因。汉和帝亲政后使东汉国力达到极盛,时人称为「永元之隆」。

汉和帝在位时期,在科技、文化、军事、外交上也有不少建树,蔡伦改进了造纸术,班固修成《汉书》,窦宪击破北匈奴促使其西迁,班超平定西域,并派遣甘英出使大秦。元興元年乙巳年十二月廿二日辛未(106年2月13日),汉和帝病逝于京都洛阳的章德前殿,时年二十七岁。4月27日,葬於漢慎陵。

《后汉书》:“自中兴以后,逮于永元,虽颇有弛张,而俱存不扰,是以齐民岁增,辟土世广。偏师出塞,则漠北地空;都护西指,则通译四万。岂其道远三代,术长前世?将服叛去来,自有数也?”

《东观汉记》:“孝和皇帝,章帝中子也,上自歧嶷,至於总角,孝顺聪明,宽和仁孝,帝由是深珍之,以为宜承天位,年四岁,立为太子,初治尚书,遂兼览书传,好古乐道,无所不照,上以五经义异,书传意殊,亲幸东观,览书林,阅篇藉,朝无宠族,惠泽沾濡,外忧庶绩,内勤经艺,自左右近臣,皆诵诗书,德教在宽,仁恕并洽,是以黎元宁康,万国协和,符瑞八十馀品,帝让而不宣,故靡得而纪。”

《帝王世纪》:“孝和之嗣世,正身履道,以奉大业,宾礼耆艾,动式旧典,宫无嫔嫱郑卫之燕,囿无般乐游畋之豫,躬履至德,虚静自损,是以屡获丰年,远近承风。”

後汉苏顺和帝诔曰:“天王徂登,率土奄伤,如何昊穹,夺我圣皇,恩德累代,乃作铭章,其辞曰:恭惟大行,配天建德,陶元二化,风流万国,立我蒸民,宜此仪则,厥初生民,三五作刚,载藉之盛,著於虞唐,恭惟大行,爰同其光,自昔何为,钦明允塞,恭惟大行,天覆地载,无为而治,冠斯往代,往代崎岖,诸夏擅命,爰兹发号,民乐其政,奄有万国,民臣咸祑,大孝备矣,閟宫有侐,由昔姜嫄,祖妣之室,本枝百世,神契惟一,弥留不豫,道扬末命,劳谦有终,实惟其性,衣不制新,犀玉远屏,履和而行,威棱上古,洪泽滂流,茂化沾溥,不玦少留,民斯何怙,歔欷成云,泣涕成雨,昊天不吊,丧我慈父。”

後汉崔瑗和帝诔曰:“玄景寝曜,云物见徵,冯相考妖,遂当帝躬,三载四海,遏密八音,如丧考妣,擗踊号吟,大遂既启,乃徂玄宫,永背神器,升遐皇穹,长夜冥冥,曷云其穷。”

洪迈《容斋随笔‧卷三》:“汉昭帝年十四,能察霍光之忠,知燕王上书之诈,诛桑弘羊、上官桀,后世称其明。然和帝时,窦宪兄弟专权,太后临朝,共图杀害。帝阴知其谋,而与内外臣僚莫由亲接,独知中常侍郑众不事豪党,遂与定议诛宪,时亦年十四,其刚决不下昭帝,但范史发明不出,故后世无称焉。”

《续汉书》:“论曰:孝和年十四,能折外戚骄横之权,即昭帝毙上官之类矣。朝政遂一,民安职业,勤恤本务,苑囿希幸,远夷稽服,西域开泰,郡国言符瑞八十余品,咸惧虚妄,抑而不宣云尔。”

李贤注引《序例》曰:“凡瑞应,自和帝以上,政事多美,近於有实,故书见於某处。自安帝以下,王道衰缺,容或虚饰,故书某处上言也。”

李尤《辟雍赋》曰:“卓矣煌煌,永元之隆。含弘该要,周建大中。蓄纯和之优渥兮,化盛溢而兹丰。”

《通典》:“明章之后,天下无事,务在养民。至於孝和,人户滋殖。”

叶适《习学记言序目》:“东汉至孝和八十年间,上无败政,天下乂安。”

\subsection{永元}

\begin{longtable}{|>{\centering\scriptsize}m{2em}|>{\centering\scriptsize}m{1.3em}|>{\centering}m{8.8em}|}
  % \caption{秦王政}\
  \toprule
  \SimHei \normalsize 年数 & \SimHei \scriptsize 公元 & \SimHei 大事件 \tabularnewline
  % \midrule
  \endfirsthead
  \toprule
  \SimHei \normalsize 年数 & \SimHei \scriptsize 公元 & \SimHei 大事件 \tabularnewline
  \midrule
  \endhead
  \midrule
  元年 & 89 & \tabularnewline\hline
  二年 & 90 & \tabularnewline\hline
  三年 & 91 & \tabularnewline\hline
  四年 & 92 & \tabularnewline\hline
  五年 & 93 & \tabularnewline\hline
  六年 & 94 & \tabularnewline\hline
  七年 & 95 & \tabularnewline\hline
  八年 & 96 & \tabularnewline\hline
  九年 & 97 & \tabularnewline\hline
  十年 & 98 & \tabularnewline\hline
  十一年 & 99 & \tabularnewline\hline
  十二年 & 100 & \tabularnewline\hline
  十三年 & 101 & \tabularnewline\hline
  十四年 & 102 & \tabularnewline\hline
  十五年 & 103 & \tabularnewline\hline
  十六年 & 104 & \tabularnewline\hline
  十七年 & 105 & \tabularnewline
  \bottomrule
\end{longtable}

\subsection{元兴}

\begin{longtable}{|>{\centering\scriptsize}m{2em}|>{\centering\scriptsize}m{1.3em}|>{\centering}m{8.8em}|}
  % \caption{秦王政}\
  \toprule
  \SimHei \normalsize 年数 & \SimHei \scriptsize 公元 & \SimHei 大事件 \tabularnewline
  % \midrule
  \endfirsthead
  \toprule
  \SimHei \normalsize 年数 & \SimHei \scriptsize 公元 & \SimHei 大事件 \tabularnewline
  \midrule
  \endhead
  \midrule
  元年 & 105 & \tabularnewline
  \bottomrule
\end{longtable}


%%% Local Variables:
%%% mode: latex
%%% TeX-engine: xetex
%%% TeX-master: "../Main"
%%% End:
