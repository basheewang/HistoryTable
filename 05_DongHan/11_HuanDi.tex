%% -*- coding: utf-8 -*-
%% Time-stamp: <Chen Wang: 2019-12-17 21:11:08>

\section{桓帝\tiny(147-167)}

\subsection{生平}

汉桓帝刘志(132年-168年1月25日),东汉第十一位皇帝(146年8月1日-168年1月25日在位),其正式諡號為「孝桓皇帝」,後世省略「孝」字稱「漢桓帝」,他是汉章帝曾孙,河間孝王劉開之孫,蠡吾侯劉翼之子,在位21年。

146年,外戚梁冀毒死九岁的汉质帝,立十五岁的刘志即位,承汉顺帝嗣。

刘志从小就对梁氏不满,他即位后,就想方设法的诛灭梁氏。延熹二年(159年),桓帝联合宦官单超等5人一舉殲灭了梁氏,5人同日被封侯,称之为“五侯”。不過,五侯比外戚更加腐敗,他们对百姓们勒索抢劫,民不聊生,四处怨声载道,東汉政治更加衰頹,国势益弱。汉桓帝统治后期,一批太学士看到朝政败壞,便要求朝廷整肅宦官、改革政治。宦官气極败坏,在延熹九年(166年)与德揚天下的司隸校尉李膺发生大规模冲突。桓帝大怒,下令逮捕替李膺請願的太学生200余人,后来在太傅陈蕃、将军窦武的反对下才释放太学生,但是禁锢终身,不许再做官,史称“党锢之祸”,東漢朝政更加黑暗腐敗。汉桓帝在位期间沉迷女色,荒淫无度,后宫人数竟达五六千人。

汉桓帝於168年1月25日去世,死后谥号孝桓皇帝,庙号为威宗,168年3月9日葬於宣陵。後於漢獻帝初平元年因其無功德故除去廟號。

诸葛亮:“亲贤臣,远小人,此先汉所以兴隆也;亲小人,远贤臣,此后汉所以倾颓也。先帝在时,每与臣论此事,未尝不叹息痛恨于桓、灵也。”

范晔:“前史称桓帝好音乐,善琴笙。饰芳林而考濯龙之宫,设华盖以祠浮图、老子,斯将所谓“听于神”乎!及诛梁冀,奋威怒,天下犹企其休息。而五邪嗣虐,流衍四方。自非忠贤力争,屡折奸锋,虽愿依斟流彘,亦不可得已。”

虞世南:“桓帝赫然奋怒,诛灭梁冀,有刚断之节焉。然阉人擅命,党锢事起,非乎乱阶,始於桓帝。”

周昙:“能嫌跋扈斩梁王,宁便荣枯信段张。襄楷忠言谁佞惑,忍教奸祸起萧墙。”

\subsection{建和}

\begin{longtable}{|>{\centering\scriptsize}m{2em}|>{\centering\scriptsize}m{1.3em}|>{\centering}m{8.8em}|}
  % \caption{秦王政}\
  \toprule
  \SimHei \normalsize 年数 & \SimHei \scriptsize 公元 & \SimHei 大事件 \tabularnewline
  % \midrule
  \endfirsthead
  \toprule
  \SimHei \normalsize 年数 & \SimHei \scriptsize 公元 & \SimHei 大事件 \tabularnewline
  \midrule
  \endhead
  \midrule
  元年 & 147 & \tabularnewline\hline
  二年 & 148 & \tabularnewline\hline
  三年 & 149 & \tabularnewline
  \bottomrule
\end{longtable}

\subsection{和平}

\begin{longtable}{|>{\centering\scriptsize}m{2em}|>{\centering\scriptsize}m{1.3em}|>{\centering}m{8.8em}|}
  % \caption{秦王政}\
  \toprule
  \SimHei \normalsize 年数 & \SimHei \scriptsize 公元 & \SimHei 大事件 \tabularnewline
  % \midrule
  \endfirsthead
  \toprule
  \SimHei \normalsize 年数 & \SimHei \scriptsize 公元 & \SimHei 大事件 \tabularnewline
  \midrule
  \endhead
  \midrule
  元年 & 150 & \tabularnewline
  \bottomrule
\end{longtable}

\subsection{元嘉}

\begin{longtable}{|>{\centering\scriptsize}m{2em}|>{\centering\scriptsize}m{1.3em}|>{\centering}m{8.8em}|}
  % \caption{秦王政}\
  \toprule
  \SimHei \normalsize 年数 & \SimHei \scriptsize 公元 & \SimHei 大事件 \tabularnewline
  % \midrule
  \endfirsthead
  \toprule
  \SimHei \normalsize 年数 & \SimHei \scriptsize 公元 & \SimHei 大事件 \tabularnewline
  \midrule
  \endhead
  \midrule
  元年 & 151 & \tabularnewline\hline
  二年 & 152 & \tabularnewline\hline
  三年 & 153 & \tabularnewline
  \bottomrule
\end{longtable}

\subsection{永兴}

\begin{longtable}{|>{\centering\scriptsize}m{2em}|>{\centering\scriptsize}m{1.3em}|>{\centering}m{8.8em}|}
  % \caption{秦王政}\
  \toprule
  \SimHei \normalsize 年数 & \SimHei \scriptsize 公元 & \SimHei 大事件 \tabularnewline
  % \midrule
  \endfirsthead
  \toprule
  \SimHei \normalsize 年数 & \SimHei \scriptsize 公元 & \SimHei 大事件 \tabularnewline
  \midrule
  \endhead
  \midrule
  元年 & 153 & \tabularnewline\hline
  二年 & 154 & \tabularnewline
  \bottomrule
\end{longtable}

\subsection{永寿}

\begin{longtable}{|>{\centering\scriptsize}m{2em}|>{\centering\scriptsize}m{1.3em}|>{\centering}m{8.8em}|}
  % \caption{秦王政}\
  \toprule
  \SimHei \normalsize 年数 & \SimHei \scriptsize 公元 & \SimHei 大事件 \tabularnewline
  % \midrule
  \endfirsthead
  \toprule
  \SimHei \normalsize 年数 & \SimHei \scriptsize 公元 & \SimHei 大事件 \tabularnewline
  \midrule
  \endhead
  \midrule
  元年 & 155 & \tabularnewline\hline
  二年 & 156 & \tabularnewline\hline
  三年 & 157 & \tabularnewline\hline
  四年 & 158 & \tabularnewline
  \bottomrule
\end{longtable}

\subsection{延熹}

\begin{longtable}{|>{\centering\scriptsize}m{2em}|>{\centering\scriptsize}m{1.3em}|>{\centering}m{8.8em}|}
  % \caption{秦王政}\
  \toprule
  \SimHei \normalsize 年数 & \SimHei \scriptsize 公元 & \SimHei 大事件 \tabularnewline
  % \midrule
  \endfirsthead
  \toprule
  \SimHei \normalsize 年数 & \SimHei \scriptsize 公元 & \SimHei 大事件 \tabularnewline
  \midrule
  \endhead
  \midrule
  元年 & 158 & \tabularnewline\hline
  二年 & 159 & \tabularnewline\hline
  三年 & 160 & \tabularnewline\hline
  四年 & 161 & \tabularnewline\hline
  五年 & 162 & \tabularnewline\hline
  六年 & 163 & \tabularnewline\hline
  七年 & 164 & \tabularnewline\hline
  八年 & 165 & \tabularnewline\hline
  九年 & 166 & \tabularnewline\hline
  十年 & 167 & \tabularnewline
  \bottomrule
\end{longtable}


\subsection{永康}

\begin{longtable}{|>{\centering\scriptsize}m{2em}|>{\centering\scriptsize}m{1.3em}|>{\centering}m{8.8em}|}
  % \caption{秦王政}\
  \toprule
  \SimHei \normalsize 年数 & \SimHei \scriptsize 公元 & \SimHei 大事件 \tabularnewline
  % \midrule
  \endfirsthead
  \toprule
  \SimHei \normalsize 年数 & \SimHei \scriptsize 公元 & \SimHei 大事件 \tabularnewline
  \midrule
  \endhead
  \midrule
  元年 & 167 & \tabularnewline
  \bottomrule
\end{longtable}


%%% Local Variables:
%%% mode: latex
%%% TeX-engine: xetex
%%% TeX-master: "../Main"
%%% End:
