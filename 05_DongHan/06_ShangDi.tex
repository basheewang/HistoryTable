%% -*- coding: utf-8 -*-
%% Time-stamp: <Chen Wang: 2019-12-17 17:23:12>

\section{殇帝\tiny(106)}

\subsection{生平}

汉殇帝刘隆(105年10月或11月-106年9月21日),汉和帝幼子,养于民间,东汉第五位皇帝(106年在位),其正式諡號為「孝殤皇帝」,後世省略「孝」字稱「漢殤帝」。汉殇帝是即位年龄最小、寿命最短的中国皇帝。

和帝在世的时候,生了许多皇子,大都夭折。和帝以为宦官、外戚在谋害他的儿子,便将剩余的皇子留在民间扶养。元興元年乙巳年十二月廿二日辛未(106年2月13日),汉和帝死,邓皇后因长子刘胜有絕症,将刘隆迎回皇宫做皇帝,將刘胜封为平原王。刘隆登基时候才出生100余天,改元“延平”。朝政由外戚邓騭掌权。

仍在襁褓之中的汉殇帝,于延平元年八月辛亥(西元106年9月21日)得了场大病后驾崩,在位只有短短八個月。

刘隆年幼,邓太后以女主临政,期间政事多委以宦官。自汉明帝至延平年间,宦官的人数逐渐增多,中常侍达到十名,小黄门有二十名。

\subsection{延平}

\begin{longtable}{|>{\centering\scriptsize}m{2em}|>{\centering\scriptsize}m{1.3em}|>{\centering}m{8.8em}|}
  % \caption{秦王政}\
  \toprule
  \SimHei \normalsize 年数 & \SimHei \scriptsize 公元 & \SimHei 大事件 \tabularnewline
  % \midrule
  \endfirsthead
  \toprule
  \SimHei \normalsize 年数 & \SimHei \scriptsize 公元 & \SimHei 大事件 \tabularnewline
  \midrule
  \endhead
  \midrule
  元年 & 106 & \tabularnewline
  \bottomrule
\end{longtable}


%%% Local Variables:
%%% mode: latex
%%% TeX-engine: xetex
%%% TeX-master: "../Main"
%%% End:
