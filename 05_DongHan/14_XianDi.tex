%% -*- coding: utf-8 -*-
%% Time-stamp: <Chen Wang: 2021-11-01 11:31:35>

\section{献帝刘协\tiny(189-220)}

\subsection{生平}

刘协(181年4月2日-234年4月21日),字伯和,是东汉政權最后一位皇帝,189年至220年在位,曹魏給其諡號為「孝獻皇帝」,後世省略「孝」字稱「漢獻帝」。蜀漢給其諡號為「孝愍皇帝」。

漢獻帝刘协是汉灵帝刘宏的儿子,汉少帝刘辩的庶弟,母親是美人王榮(五官中郎將王苞的孫女)。因汉灵帝何皇后性强忌威慑后宫,王荣刚怀孕时害怕,服药欲堕胎,而胎安不动,又数次梦到肩负著太阳行走,是为吉兆。刘协出生后,母親王荣即被何皇后毒杀,他由祖母董太后撫養成人。中平六年(189年)四月,汉灵帝崩,刘协的异母兄刘辩即位,是为汉少帝,封刘协勃海王,又改封陈留王。

袁绍等人诛杀宦官时,他随刘辩被宦官張讓和段珪绑架,遇到董卓。董卓曾和少帝谈话,少帝语无伦次,再和刘协谈话,刘协则将事情经过完整交代。董卓以为刘协贤能,且为董太后所养,又自以为与董太后同族,再加上要显示自己的权力,遂有废立之意。

中平六年(189年),董卓為了立威,废少帝,於九月甲戌日(9月28日)立当时九岁的刘协为皇帝。關東諸侯起兵討伐董卓時,董卓杀少帝、何太后,火燒都城雒陽,挾劉協遷都長安。初平三年(192年),司徒王允成功诛杀董卓,但不久董卓的部下李傕、郭汜即攻陷长安,杀死大批大臣,再次挟持献帝。後來李郭二人內訌戰鬥,民不聊生,献帝与一批朝臣于兴平二年(195年)七月逃离长安,途中多次成为李傕、郭汜、张济、杨奉等军阀争夺挟制的目标。

汉献帝与朝臣历经一年才于建安元年(196年)七月到达旧都雒阳。不过雒阳早经董卓撤离时焚烧,宫室尽毁,百官披荆棘藏身断壁之间,更加粮草断绝。八月,朝臣曹操从雒阳挾刘协到許縣,稱許都,作为其“挟天子以令诸侯”战略的一部分,刘协成为一位毫無實權的傀儡皇帝。宮中的守衛和侍從其實都是曹操的人,對內外官員的殺戮亦很常見,例如議郎趙彥曾經向獻帝進言分析局勢和對策,因而被曹操殺害。

建安五年(200年),大臣董承連同與王子服、种輯、劉備等人谋划对付曹操,董承卻被曹操所刺。曹操欲杀董承之女董贵人,獻帝以董貴人懷有漢室的血脈拒絕,但最終董貴人仍被曹操弒殺。

建安十八年(213年)時,曹操將三個女兒曹憲、曹節和曹華送入漢獻帝後宮為夫人,年紀最小的曹華在國待年。次年,三女都被封為貴人。

建安十九年(214年),献帝的皇后伏壽之前写信给父亲伏完圖謀誅殺曹操之事洩露,曹操大怒,逼獻帝廢后,更先替獻帝寫好詔書,命郗慮持節接收皇后印綬,又命尚書令華歆領兵入宮捉拿伏壽。伏壽躲在牆壁之中但被華歆發現拉出,經過外殿時見獻帝,伏壽哭著說:「不能復相活邪?(不能再救救我嗎?)」獻帝答:「我亦不知命在何時。(我亦不知我何時會死。)」又對身旁的郗慮說:「郗公,天下寧有是邪?(郗公,天下間有這樣的事嗎?)」伏壽后被幽闭至死(《曹瞞傳》稱當場被弒),所生的兩位皇子亦以毒酒毒害,伏氏宗族有百多人亦被處死,伏壽母親盈等十九人都被流放到涿郡。

建安二十年正月(215年),因曹操和群臣的壓力下,改立貴人曹節為皇后。

建安二十五年(220年)正月,曹操去世。三月献帝改元延康。嗣魏王位的曹丕提出要求獻帝禪讓皇位,篡位自立。在曹氏政權威迫之下,十月乙卯日(11月25日)献帝將帝位在繁陽的受禪臺之上「禪讓」給曹丕,东汉結束。

曹丕篡漢後,封劉協為山陽公(屬司隸河內郡),「邑一萬戶,位在諸侯王上,奏事不稱臣,受詔不拜,以天子車服郊祀天地,宗廟、祖、臘皆如漢制,都山陽之濁鹿城」(《後漢書》本紀)。曹丕並對衛臻說「天下之珍,吾與山陽共之」(《三國志·衛臻傳》)。

曹魏青龍二年(234年)三月庚寅(4月21日),劉協駕崩,享年五十四歲。曹叡聞訊後,「素服發哀,遣使持節典護喪事……追謚山陽公曰孝獻皇帝,冊贈璽紱……車旗服章喪葬禮儀,一如漢氏故制」,並宣布大赦天下。八月壬申(9月30日),劉協被安葬于山陽國,陵曰禪陵,置園邑。諡號為孝獻皇帝。

袁术:「圣主聪叡,有周成之质。」(《三国志·卷六·魏书六·董二袁刘传第六》)

袁山松:「献帝崎岖危乱之间,飘薄万里之衢,萍流蓬转,险阻备经,自古帝王未之有也。观其天性慈爱,弱而神惠,若辅之以德,真守文令主也。曹氏始於勤王,终至陷天,遂力制群雄,负鼎而趋,然因其利器,假而不反,回山倒海,遂移天日。昔田常假汤、武而杀君,操因尧、舜而窃国,所乘不同,济其盗贼之身一也。善乎!庄生之言:窃钩者诛,窃国者为诸侯,诸侯之门仁义存焉。信矣。」

范晔:「传称鼎之为器,虽小而重,故神之所宝,不可夺移。至令负而趋者,此亦穷运之归乎!天厌汉德久矣,山阳其何诛焉!」「献生不辰,身播国屯。终我四百,永作虞宾。」


\subsection{永汉}

\begin{longtable}{|>{\centering\scriptsize}m{2em}|>{\centering\scriptsize}m{1.3em}|>{\centering}m{8.8em}|}
  % \caption{秦王政}\
  \toprule
  \SimHei \normalsize 年数 & \SimHei \scriptsize 公元 & \SimHei 大事件 \tabularnewline
  % \midrule
  \endfirsthead
  \toprule
  \SimHei \normalsize 年数 & \SimHei \scriptsize 公元 & \SimHei 大事件 \tabularnewline
  \midrule
  \endhead
  \midrule
  元年 & 189 & \tabularnewline
  \bottomrule
\end{longtable}

\subsection{中平}

\begin{longtable}{|>{\centering\scriptsize}m{2em}|>{\centering\scriptsize}m{1.3em}|>{\centering}m{8.8em}|}
  % \caption{秦王政}\
  \toprule
  \SimHei \normalsize 年数 & \SimHei \scriptsize 公元 & \SimHei 大事件 \tabularnewline
  % \midrule
  \endfirsthead
  \toprule
  \SimHei \normalsize 年数 & \SimHei \scriptsize 公元 & \SimHei 大事件 \tabularnewline
  \midrule
  \endhead
  \midrule
  元年 & 189 & \tabularnewline
  \bottomrule
\end{longtable}

\subsection{初平}

\begin{longtable}{|>{\centering\scriptsize}m{2em}|>{\centering\scriptsize}m{1.3em}|>{\centering}m{8.8em}|}
  % \caption{秦王政}\
  \toprule
  \SimHei \normalsize 年数 & \SimHei \scriptsize 公元 & \SimHei 大事件 \tabularnewline
  % \midrule
  \endfirsthead
  \toprule
  \SimHei \normalsize 年数 & \SimHei \scriptsize 公元 & \SimHei 大事件 \tabularnewline
  \midrule
  \endhead
  \midrule
  元年 & 190 & \tabularnewline\hline
  二年 & 191 & \tabularnewline\hline
  三年 & 192 & \tabularnewline\hline
  四年 & 193 & \tabularnewline
  \bottomrule
\end{longtable}


\subsection{兴平}

\begin{longtable}{|>{\centering\scriptsize}m{2em}|>{\centering\scriptsize}m{1.3em}|>{\centering}m{8.8em}|}
  % \caption{秦王政}\
  \toprule
  \SimHei \normalsize 年数 & \SimHei \scriptsize 公元 & \SimHei 大事件 \tabularnewline
  % \midrule
  \endfirsthead
  \toprule
  \SimHei \normalsize 年数 & \SimHei \scriptsize 公元 & \SimHei 大事件 \tabularnewline
  \midrule
  \endhead
  \midrule
  元年 & 194 & \tabularnewline\hline
  二年 & 195 & \tabularnewline
  \bottomrule
\end{longtable}

\subsection{建安}

\begin{longtable}{|>{\centering\scriptsize}m{2em}|>{\centering\scriptsize}m{1.3em}|>{\centering}m{8.8em}|}
  % \caption{秦王政}\
  \toprule
  \SimHei \normalsize 年数 & \SimHei \scriptsize 公元 & \SimHei 大事件 \tabularnewline
  % \midrule
  \endfirsthead
  \toprule
  \SimHei \normalsize 年数 & \SimHei \scriptsize 公元 & \SimHei 大事件 \tabularnewline
  \midrule
  \endhead
  \midrule
  元年 & 196 & \tabularnewline\hline
  二年 & 197 & \tabularnewline\hline
  三年 & 198 & \tabularnewline\hline
  四年 & 199 & \tabularnewline\hline
  五年 & 200 & \tabularnewline\hline
  六年 & 201 & \tabularnewline\hline
  七年 & 202 & \tabularnewline\hline
  八年 & 203 & \tabularnewline\hline
  九年 & 204 & \tabularnewline\hline
  十年 & 205 & \tabularnewline\hline
  十一年 & 206 & \tabularnewline\hline
  十二年 & 207 & \tabularnewline\hline
  十三年 & 208 & \tabularnewline\hline
  十四年 & 209 & \tabularnewline\hline
  十五年 & 210 & \tabularnewline\hline
  十六年 & 211 & \tabularnewline\hline
  十七年 & 212 & \tabularnewline\hline
  十八年 & 213 & \tabularnewline\hline
  十九年 & 214 & \tabularnewline\hline
  二十年 & 215 & \tabularnewline\hline
  二一年 & 216 & \tabularnewline\hline
  二二年 & 217 & \tabularnewline\hline
  二三年 & 218 & \tabularnewline\hline
  二四年 & 219 & \tabularnewline\hline
  二五年 & 220 & \tabularnewline
  \bottomrule
\end{longtable}

\subsection{延康}

\begin{longtable}{|>{\centering\scriptsize}m{2em}|>{\centering\scriptsize}m{1.3em}|>{\centering}m{8.8em}|}
  % \caption{秦王政}\
  \toprule
  \SimHei \normalsize 年数 & \SimHei \scriptsize 公元 & \SimHei 大事件 \tabularnewline
  % \midrule
  \endfirsthead
  \toprule
  \SimHei \normalsize 年数 & \SimHei \scriptsize 公元 & \SimHei 大事件 \tabularnewline
  \midrule
  \endhead
  \midrule
  元年 & 220 & \tabularnewline
  \bottomrule
\end{longtable}


%%% Local Variables:
%%% mode: latex
%%% TeX-engine: xetex
%%% TeX-master: "../Main"
%%% End:
