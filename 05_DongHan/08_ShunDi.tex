%% -*- coding: utf-8 -*-
%% Time-stamp: <Chen Wang: 2019-12-17 21:05:58>

\section{顺帝\tiny(125-144)}

\subsection{前少帝生平}

刘懿(?-125年12月10日),一名犊,东汉第七位皇帝(125年5月18日-12月10日在位)。济北惠王刘寿的兒子,即位前為北鄉侯,东汉前少帝,汉朝官方没有把他算作汉朝皇帝之一。

汉安帝病危期间,征济北、河间王子年十四以下、七岁以上前往洛阳。汉安帝去世后,阎皇后为了把持国政,在阎显支持下,迎立北乡侯刘懿为帝,承嗣汉安帝(虽说两者是堂兄弟关系) 。少帝在位时,阎显兄弟把持朝政,作威作福。但少帝即位數月后就因病去世,之后宦官孙程等人合谋诛杀阎显兄弟和江京,并迎立济阴王刘保为帝,是为汉顺帝。

刘懿去世后以诸侯王规格下葬。永和元年(136年),災異頻繁,漢順帝感到恐懼,認為是北鄉侯當過皇帝卻以諸侯王規格下葬導致的報應。漢順帝打算追謚北鄉侯,納入漢朝皇帝體系。周舉不讚成,認為北鄉侯是奸臣閻顯等所立,並非正統,且在位一年不到就去世,年號未改,加上北鄉侯沒有其他功德,用諸侯王規格下葬已經很好了,不值得給他加上謚號和追認為皇帝。漢順帝聽從。

\subsection{顺帝生平}

汉顺帝刘保(115年-144年9月20日),东汉第八位皇帝(125年12月16日—144年9月20日在位),其正式諡號為「孝順皇帝」,後世省略「孝」字稱「漢順帝」。汉安帝和宫人李氏之子。

刘保出生后,生母李氏就被皇后阎姬毒杀。

劉保從小學習孝經章句,很得鄧太后欣賞,認為他可以繼承大統。永宁元年(120年),身为汉安帝独子的刘保被立为皇太子。

延光三年(124年),刘保生病,来到汉安帝乳母王圣家居住。当时王圣宅邸刚完成不久,刘保乳母王男、厨监邴吉认为不祥,反对太子刘保前去居住,于是与王圣等人爆发激烈争吵。王圣等人大怒,于是联合大长秋江京、中常侍樊丰等诬陷太子劉保的乳母王男、厨监邴吉。两人被杀,太子数为叹息。王圣等人惧有后祸,遂与樊丰、江京、汉安帝皇后阎姬共同构陷太子劉保。汉安帝召集大臣议论,太常桓焉、太仆来历、廷尉张皓等反对,汉安帝派人威胁反对废太子的大臣,最后只有来历坚决阻止汉安帝废太子。汉安帝大怒,下令罢免来历的官位,并立即废太子劉保为济阴王。来历不服,纠集11位官员和百姓上书喊冤,汉安帝不为所动。

汉安帝死后,阎皇后无子,便找个幼儿刘懿为皇帝,自己垂帘听政,掌握朝政大权。漢安帝喪葬期間,阎皇后等不讓劉保上殿靠近棺材,劉保悲傷吐血,餐粥不食。刘懿做了7个月的皇帝就死了,阎显等認為先前不立劉保,現在如果立他為帝,劉保會怨恨我們。於是稟告閻太后,繼續讓諸侯王子來京師挑選繼承人。宦官王康、孙程等19人看不下去,便发动宫廷政变,赶走阎太后,将时年11岁的刘保拥立为帝,改元“永建”,那19位拥立刘保的宦官也全部封侯。同時閻太后黨羽也被罷黜。阎太后被幽禁离宫,但顺帝拒绝了陈禅等人以无母子之情为由废太后的提议,仍尊奉阎太后直至其去世。

汉顺帝雖本为太子,但他的皇位是靠宦官得来的,所以将大权交给宦官。順帝本人則溫和但是軟弱,無法阻止宦官与外戚专政的局面。

后来宦官与外戚梁氏勾結,开始了长达20多年的梁冀专权。宦官、外戚互相勾结,弄权专横,東漢政治更加腐败,阶级矛盾日益尖锐,百姓怨声载道。

建康元年(144年)9月20日,汉顺帝死,享年30岁,在位19年。10月26日,葬於漢憲陵。汉顺帝安葬当年,憲陵就被盗贼盗掘。

顺帝死后谥号孝顺皇帝,庙号敬宗,後於漢獻帝初平元年因其無功德故除去廟號。

\subsection{永建}

\begin{longtable}{|>{\centering\scriptsize}m{2em}|>{\centering\scriptsize}m{1.3em}|>{\centering}m{8.8em}|}
  % \caption{秦王政}\
  \toprule
  \SimHei \normalsize 年数 & \SimHei \scriptsize 公元 & \SimHei 大事件 \tabularnewline
  % \midrule
  \endfirsthead
  \toprule
  \SimHei \normalsize 年数 & \SimHei \scriptsize 公元 & \SimHei 大事件 \tabularnewline
  \midrule
  \endhead
  \midrule
  元年 & 126 & \tabularnewline\hline
  二年 & 127 & \tabularnewline\hline
  三年 & 128 & \tabularnewline\hline
  四年 & 129 & \tabularnewline\hline
  五年 & 130 & \tabularnewline\hline
  六年 & 131 & \tabularnewline\hline
  七年 & 132 & \tabularnewline
  \bottomrule
\end{longtable}

\subsection{阳嘉}

\begin{longtable}{|>{\centering\scriptsize}m{2em}|>{\centering\scriptsize}m{1.3em}|>{\centering}m{8.8em}|}
  % \caption{秦王政}\
  \toprule
  \SimHei \normalsize 年数 & \SimHei \scriptsize 公元 & \SimHei 大事件 \tabularnewline
  % \midrule
  \endfirsthead
  \toprule
  \SimHei \normalsize 年数 & \SimHei \scriptsize 公元 & \SimHei 大事件 \tabularnewline
  \midrule
  \endhead
  \midrule
  元年 & 132 & \tabularnewline\hline
  二年 & 133 & \tabularnewline\hline
  三年 & 134 & \tabularnewline\hline
  四年 & 135 & \tabularnewline
  \bottomrule
\end{longtable}

\subsection{永和}

\begin{longtable}{|>{\centering\scriptsize}m{2em}|>{\centering\scriptsize}m{1.3em}|>{\centering}m{8.8em}|}
  % \caption{秦王政}\
  \toprule
  \SimHei \normalsize 年数 & \SimHei \scriptsize 公元 & \SimHei 大事件 \tabularnewline
  % \midrule
  \endfirsthead
  \toprule
  \SimHei \normalsize 年数 & \SimHei \scriptsize 公元 & \SimHei 大事件 \tabularnewline
  \midrule
  \endhead
  \midrule
  元年 & 136 & \tabularnewline\hline
  二年 & 137 & \tabularnewline\hline
  三年 & 138 & \tabularnewline\hline
  四年 & 139 & \tabularnewline\hline
  五年 & 140 & \tabularnewline\hline
  六年 & 141 & \tabularnewline
  \bottomrule
\end{longtable}

\subsection{汉安}

\begin{longtable}{|>{\centering\scriptsize}m{2em}|>{\centering\scriptsize}m{1.3em}|>{\centering}m{8.8em}|}
  % \caption{秦王政}\
  \toprule
  \SimHei \normalsize 年数 & \SimHei \scriptsize 公元 & \SimHei 大事件 \tabularnewline
  % \midrule
  \endfirsthead
  \toprule
  \SimHei \normalsize 年数 & \SimHei \scriptsize 公元 & \SimHei 大事件 \tabularnewline
  \midrule
  \endhead
  \midrule
  元年 & 142 & \tabularnewline\hline
  二年 & 143 & \tabularnewline\hline
  三年 & 144 & \tabularnewline
  \bottomrule
\end{longtable}

\subsection{建康}

\begin{longtable}{|>{\centering\scriptsize}m{2em}|>{\centering\scriptsize}m{1.3em}|>{\centering}m{8.8em}|}
  % \caption{秦王政}\
  \toprule
  \SimHei \normalsize 年数 & \SimHei \scriptsize 公元 & \SimHei 大事件 \tabularnewline
  % \midrule
  \endfirsthead
  \toprule
  \SimHei \normalsize 年数 & \SimHei \scriptsize 公元 & \SimHei 大事件 \tabularnewline
  \midrule
  \endhead
  \midrule
  元年 & 144 & \tabularnewline
  \bottomrule
\end{longtable}


%%% Local Variables:
%%% mode: latex
%%% TeX-engine: xetex
%%% TeX-master: "../Main"
%%% End:
