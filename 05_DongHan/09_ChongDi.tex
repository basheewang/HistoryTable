%% -*- coding: utf-8 -*-
%% Time-stamp: <Chen Wang: 2021-11-01 11:29:46>

\section{冲帝刘炳\tiny(144-145)}

\subsection{生平}

汉冲帝刘炳(漢安二年至永憙元年正月初六戊戌日,即公元143年-145年2月15日),中国汉朝皇帝(建康元年八月初六庚午日至永憙元年正月初六戊戌日,即西元144年9月20日-145年2月15日在位),东汉第九位皇帝,在位僅148天,享年僅3岁,其正式諡號為「孝沖皇帝」,後世省略「孝」字稱「漢沖帝」。

刘炳是汉顺帝的独子,他的母亲是虞美人,建康元年(144年)被立为太子。同年顺帝死后,2岁的刘炳于八月庚午登基,是为汉冲帝,改元“永憙”。

顺帝的皇后梁氏被尊为皇太后,临朝问政。赵峻为太傅;李固为太尉。在位时期外戚梁氏当政,朝政腐败,民不聊生,九江发生暴乱,叛军在145年初一直攻到合肥。

145年正月戊戌(2月15日),冲帝病死在玉堂前殿。梁太后降旨立年僅八歲勃海王劉鴻的獨子劉纘入承大統,是為漢質帝,藉此再度以皇太后身份臨朝稱制。正月己未(3月8日),葬於漢懷陵。

\subsection{永嘉}

\begin{longtable}{|>{\centering\scriptsize}m{2em}|>{\centering\scriptsize}m{1.3em}|>{\centering}m{8.8em}|}
  % \caption{秦王政}\
  \toprule
  \SimHei \normalsize 年数 & \SimHei \scriptsize 公元 & \SimHei 大事件 \tabularnewline
  % \midrule
  \endfirsthead
  \toprule
  \SimHei \normalsize 年数 & \SimHei \scriptsize 公元 & \SimHei 大事件 \tabularnewline
  \midrule
  \endhead
  \midrule
  元年 & 145 & \tabularnewline
  \bottomrule
\end{longtable}

%%% Local Variables:
%%% mode: latex
%%% TeX-engine: xetex
%%% TeX-master: "../Main"
%%% End:
