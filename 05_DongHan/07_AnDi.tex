%% -*- coding: utf-8 -*-
%% Time-stamp: <Chen Wang: 2021-11-01 11:28:50>

\section{安帝刘祜\tiny(106-125)}

\subsection{生平}

汉安帝刘祜(94年-125年4月30日),东汉第六位皇帝(106年9月21日-125年4月30日在位),在位19年,其正式諡號為「孝安皇帝」,後世省略「孝」字稱「漢安帝」。

他是汉章帝的孙子、当年被废太子清河王刘庆的儿子,母左小娥。

延平元年八月辛亥(106年9月21日),是汉殇帝崩,他被外戚邓氏拥立为帝,承嗣汉和帝刘肇,改元永初。

汉安帝即位后,仍由邓太后聽政。外戚邓氏吸取窦氏灭亡的教训,联合宦官,袒护族人。永宁元年(120年),立李氏之子刘保为皇太子。

永宁二年(121年),邓太后去世,安帝才亲政。當初漢安帝號稱聰明,鄧太后才立他為帝,後來對漢安帝不滿意,漢安帝乳母王聖得知內情,又看到鄧太后遲遲不歸政,擔心鄧太后會廢除漢安帝。鄧太后去世后,向漢安帝告發鄧太后兄弟邓悝等曾經想立平原王劉勝。安帝大怒,下令灭了邓氏一族。

安帝虽灭邓氏,但未制止外戚干政的局面。再加上安帝不理朝政,沉湎于酒色,昏庸不堪,且在掖庭挑選了一位美女,封為貴人,非常寵愛,未滿一年,便立即封她為皇后,而這個皇后即為閻姬,导致当时东汉朝政腐败,社会黑暗,奸佞当道,社会矛盾日益尖锐,边患也十分严重。全国多地震,水旱蝗灾频繁不断,外有西羌等入侵边境,内有杜琦等领导的长达十多年民變,社会危机日益加深。东汉王朝衰落。

延光三年(124年),安帝乳母王圣与樊豐、江京共同构陷太子,太子刘保被废为济阴王。

延光四年三月庚申(125年4月23日),漢安帝在外巡遊途中在宛出現身體不適。三月丁卯(125年4月30日),汉安帝在葉死在乘舆上,享年32岁。當時閻姬、閻顯等人隨同漢安帝出遊,而前太子在洛陽,閻姬等害怕大臣立前太子為帝,於是詐稱安帝重病,秘不發喪,一路星夜兼程。庚午(5月3日),巡遊車隊回到皇宮。辛未(5月4日)晚上安帝駕崩消息才被公佈。四月己酉(6月11日),安帝葬于恭陵。谥号孝安皇帝,庙号恭宗(後於漢獻帝初平元年因其無功德故除去廟號)。后阎皇后迎立刘寿子刘懿为帝。刘懿死后,漢安帝独子刘保(漢順帝)才在宦官的拥戴下登基。

\subsection{永初}

\begin{longtable}{|>{\centering\scriptsize}m{2em}|>{\centering\scriptsize}m{1.3em}|>{\centering}m{8.8em}|}
  % \caption{秦王政}\
  \toprule
  \SimHei \normalsize 年数 & \SimHei \scriptsize 公元 & \SimHei 大事件 \tabularnewline
  % \midrule
  \endfirsthead
  \toprule
  \SimHei \normalsize 年数 & \SimHei \scriptsize 公元 & \SimHei 大事件 \tabularnewline
  \midrule
  \endhead
  \midrule
  元年 & 107 & \tabularnewline\hline
  二年 & 108 & \tabularnewline\hline
  三年 & 109 & \tabularnewline\hline
  四年 & 110 & \tabularnewline\hline
  五年 & 111 & \tabularnewline\hline
  六年 & 112 & \tabularnewline\hline
  七年 & 113 & \tabularnewline
  \bottomrule
\end{longtable}

\subsection{元初}

\begin{longtable}{|>{\centering\scriptsize}m{2em}|>{\centering\scriptsize}m{1.3em}|>{\centering}m{8.8em}|}
  % \caption{秦王政}\
  \toprule
  \SimHei \normalsize 年数 & \SimHei \scriptsize 公元 & \SimHei 大事件 \tabularnewline
  % \midrule
  \endfirsthead
  \toprule
  \SimHei \normalsize 年数 & \SimHei \scriptsize 公元 & \SimHei 大事件 \tabularnewline
  \midrule
  \endhead
  \midrule
  元年 & 114 & \tabularnewline\hline
  二年 & 115 & \tabularnewline\hline
  三年 & 116 & \tabularnewline\hline
  四年 & 117 & \tabularnewline\hline
  五年 & 118 & \tabularnewline\hline
  六年 & 119 & \tabularnewline\hline
  七年 & 120 & \tabularnewline
  \bottomrule
\end{longtable}

\subsection{永宁}

\begin{longtable}{|>{\centering\scriptsize}m{2em}|>{\centering\scriptsize}m{1.3em}|>{\centering}m{8.8em}|}
  % \caption{秦王政}\
  \toprule
  \SimHei \normalsize 年数 & \SimHei \scriptsize 公元 & \SimHei 大事件 \tabularnewline
  % \midrule
  \endfirsthead
  \toprule
  \SimHei \normalsize 年数 & \SimHei \scriptsize 公元 & \SimHei 大事件 \tabularnewline
  \midrule
  \endhead
  \midrule
  元年 & 120 & \tabularnewline\hline
  二年 & 121 & \tabularnewline
  \bottomrule
\end{longtable}

\subsection{建光}

\begin{longtable}{|>{\centering\scriptsize}m{2em}|>{\centering\scriptsize}m{1.3em}|>{\centering}m{8.8em}|}
  % \caption{秦王政}\
  \toprule
  \SimHei \normalsize 年数 & \SimHei \scriptsize 公元 & \SimHei 大事件 \tabularnewline
  % \midrule
  \endfirsthead
  \toprule
  \SimHei \normalsize 年数 & \SimHei \scriptsize 公元 & \SimHei 大事件 \tabularnewline
  \midrule
  \endhead
  \midrule
  元年 & 121 & \tabularnewline\hline
  二年 & 122 & \tabularnewline
  \bottomrule
\end{longtable}

\subsection{延光}

\begin{longtable}{|>{\centering\scriptsize}m{2em}|>{\centering\scriptsize}m{1.3em}|>{\centering}m{8.8em}|}
  % \caption{秦王政}\
  \toprule
  \SimHei \normalsize 年数 & \SimHei \scriptsize 公元 & \SimHei 大事件 \tabularnewline
  % \midrule
  \endfirsthead
  \toprule
  \SimHei \normalsize 年数 & \SimHei \scriptsize 公元 & \SimHei 大事件 \tabularnewline
  \midrule
  \endhead
  \midrule
  元年 & 122 & \tabularnewline\hline
  二年 & 123 & \tabularnewline\hline
  三年 & 124 & \tabularnewline\hline
  四年 & 125 & \tabularnewline
  \bottomrule
\end{longtable}


%%% Local Variables:
%%% mode: latex
%%% TeX-engine: xetex
%%% TeX-master: "../Main"
%%% End:
