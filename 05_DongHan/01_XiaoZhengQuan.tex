%% -*- coding: utf-8 -*-
%% Time-stamp: <Chen Wang: 2019-12-17 16:12:36>

\section{小政权}

\subsection{汉夏简介}

汉复(一作复汉、朔宁;元年:23年七月 - 末年:34年十月)是西汉末年隗嚣自立的年号,共计12年。

地皇四年八月(更始元年七月),隗嚣反新起事,檄文署“汉复元年七月”。

汉光武帝建武 九年(33年)春,隗嚣死,子隗纯立,仍然沿用汉复年号,直到34年十月隗纯降于东汉政权。

\subsection{隗囂生平}

隗\xpinyin*{囂}(前1世纪?-33年春),字季孟,天水成紀(今甘肃省静宁县)人。

隗囂出身隴右大族,在州郡為官,以知书通经闻名,陇上國師劉歆舉為國士。劉歆自杀後返乡。更始元年七月(新地皇四年八月,漢复元年七月)王莽兵連敗。隗囂与兄隗义及上邽杨广、冀县周宗攻下平襄,殺镇戎郡(治平襄)大尹起兵。以平陵方望爲軍師,隗囂趁勢攻克雍州,殺州牧陈庆;攻克安定,殺大尹王莽堂弟平阿侯王谭之子王向。

漢更始元年十月(新地皇四年十一月,漢复元年十月)王莽被殺。隗囂分兵攻占陇西、武都、金城、武威、张掖、酒泉和敦煌七郡。漢更始二年(24年)隗囂歸順更始帝,被封為右將軍,至冬,隗崔與隗义叛更始帝,隗囂遣將平之,因功封御史大夫。次年夏劉秀稱帝,隗囂勸更始帝東歸光武帝。因更始帝不允而逃回天水郡,自称西州大将军。其人谦爱士卒,倾身引接为布衣。交聘馬援為綏德將軍,得到光武帝的器重。建武二年(26年),大司徒邓禹屯云阳,西击赤眉军。裨将冯愔叛攻天水,隗囂迎击之,大敗馮愔于高平,缴獲冯愔的辎重。於是隗囂投奔光武帝,被封爲西州大将军知涼州朔方諸軍事;遣楊廣击败赤眉军攻陇之軍,追击,败赤眉军于乌氏泾阳间。陈仓吕鲔率数万軍結公孙述犯三輔,派兵助征西大将军冯异擊退之。建武六年(30年)公孙述犯南郡,光武帝下诏隗囂自天水伐蜀,隗囂拒絕。光武帝派建威大将军耿弇伐蜀意在滅隗。隗囂謝罪,然而仍與公孙述往來。

建武八年(32年)春天,光武帝派来歙袭取了略阳(今甘肃秦安陇城镇),隗嚣派王元擋之,為漢光武帝聯合河西的竇融所攻滅,隗嚣携家奔西城(今天水市西南),光武帝杀死作為政治人质的隗嚣之子隗恂,派吴汉、岑彭包围西城,隗嚣為公孙述援军救出。

建武九年(33年),隗囂憂憤而死,王元等立隗嚣少子隗纯为王,汉军来歙攻破洛门,隗纯投降,史稱東漢平隴西之戰。

\subsection{汉复\tiny(23-34)}

\begin{longtable}{|>{\centering\scriptsize}m{2em}|>{\centering\scriptsize}m{1.3em}|>{\centering}m{8.8em}|}
  % \caption{秦王政}\
  \toprule
  \SimHei \normalsize 年数 & \SimHei \scriptsize 公元 & \SimHei 大事件 \tabularnewline
  % \midrule
  \endfirsthead
  \toprule
  \SimHei \normalsize 年数 & \SimHei \scriptsize 公元 & \SimHei 大事件 \tabularnewline
  \midrule
  \endhead
  \midrule
  元年 & 23 & \tabularnewline\hline
  二年 & 24 & \tabularnewline\hline
  三年 & 25 & \tabularnewline\hline
  四年 & 26 & \tabularnewline\hline
  五年 & 27 & \tabularnewline\hline
  六年 & 28 & \tabularnewline\hline
  七年 & 29 & \tabularnewline\hline
  八年 & 30 & \tabularnewline\hline
  九年 & 31 & \tabularnewline\hline
  十年 & 32 & \tabularnewline\hline
  十一年 & 33 & \tabularnewline\hline
  十二年 & 34 & \tabularnewline
  \bottomrule
\end{longtable}

\subsection{成家简介}

成家(25年-36年,又称“大成”或“成”)是两汉之交在中国四川地区存在的一个独立政权,定都成都。成家由公孙述创立,鼎盛时期据有西汉所置益州大部分地区,即蜀郡、巴郡、广汉郡、犍为郡、越嶲郡、汉中郡全境和武都郡、南郡部分地区。成家成立之初推行了一些促进经济、文化发展的措施,在军阀割据混战的局势下,力保巴蜀太平,受到了蜀地人士拥护。但由于成、汉实力悬殊,公孙述治国失误等原因,成家最终为东汉所灭,前后历时12年。

成家是秦灭巴蜀后,巴蜀地区出现的第一个独立政权,也是四川历史上第一个完整占据巴蜀地区的政权。之后蜀地政权多纳入汉中盆地,北抵秦岭、东至三峡的格局也由此形成,其在四川历史上具有重要的地位。

西汉外戚王莽称帝建立新朝后,于天凤年间(公元14年至19年)提拔中散大夫公孙述为导江卒正(王莽改蜀郡为导江,改郡守为卒正),治临邛(今四川邛崃)。公孙述任导江卒正期间以才能出众而闻名于蜀中,以至于蜀地百姓将临邛城称为公孙述城。

公元23年,新朝覆灭,更始帝刘玄登基,将国号恢复为“汉”。在大乱初期尚很平静的巴蜀地区,也开始动荡起来。宗成被刘玄拜为“虎牙将军”,自汉中起兵,领兵入蜀。与此同时,王岑以拥立汉宗室刘辟为名,在雒城(今四川广汉)起兵,自称“定汉将军”,并攻陷成都。公孙述借宗成之力消灭掉王岑、刘辟后,见宗成掳掠暴横,恐对己不利,遂于临邛起兵,同时又派人诈称汉朝使者,授其辅汉将军、蜀郡太守兼益州牧印。公孙述摇身一变成为汉臣,加之其在蜀地名望甚高,其军队迅速击败宗成、攻占成都,并占领巴郡、广汉郡。次年,公孙述之弟公孙恢又在绵竹(今四川德阳)大败刘玄军队,加强了公孙述在益州的威望。随后,公孙述自立为蜀王,定都成都,并得到了多数巴蜀民众的拥护。此时蜀地民富兵强,吸引了众多臣子前来投靠,邛人、笮人首领也前来相贺。

由于自己并不是刘氏正宗,公孙述对称天子一直有所顾忌。其功曹李熊力谏,认为在蜀地称天子条件成熟,并称“天命无常,百姓与能。能者当之,王何疑焉?”,打消了公孙述的顾虑。于是公孙述放出风声,称有龙在公孙述宫殿中出现,其掌上还出现了“公孙帝”三字。在完成了称天子的舆论准备后,建武元年四月(公元25年),公孙述正式在成都称天子,以成都起事而定国号为“成家”,并改年号为“龙兴”。

在公孙述称天子之初,其实际控制的仅有蜀、巴和广汉三郡。虽然巴蜀以西的邛人、笮人在其称天子之前便已归附,随后公孙述又相继占领了犍为、越嶲两郡,但其却一直未能占领南部原属益州的益州、牂柯两郡。在后方尚未安定的情况下,公孙述却又不满足于偏安一隅,积极准备北出东进。

龙兴二年(公元26年),适逢更始帝出击占据汉中郡的严岑,在双方两败俱伤之时,公孙述趁机出兵北伐占领汉中。同年底亲率数十万大军进驻汉中,准备御驾亲征,直取关中。而此时,忙于关东战事无暇西顾的刘秀决意联合占据陇地的隗嚣来制衡成家。龙兴三年(公元27年),公孙述多次自汉中遣兵数万,攻打关中,却皆败于刘秀大将冯异与隗嚣的联军。之后隗嚣又出兵攻蜀,连破蜀军,迫使公孙述放弃了北伐关中的想法。

龙兴六年(公元30年),刘秀成功平定关东,欲讨伐隗嚣,但因其拒蜀有功,苦于没有借口,于是命隗嚣出兵讨伐成家。若隗嚣伐蜀,必定两败俱伤;若不从命,也有了讨伐其的理由。隗嚣察觉了刘秀的意图,婉言拒绝。于是刘秀亲率大军西征讨伐隗嚣。隗嚣见形势不利,经过反复考虑后,决定遣使者入蜀,称臣于公孙述。龙兴七年(公元31年),公孙述封隗嚣为朔宁王,并多次派兵相助,一时双方相持不下。龙兴九年(公元33年)春,隗嚣病逝,其次子隗纯被立为朔宁王。汉军再度发起攻势,在成家军队的协助下,双方相持近一年。龙兴十年(公元34年)七月,隗纯降汉。

龙兴九年隗嚣病逝后,公孙述派兵出击南郡,并成功夺取夷道(今湖北枝城)、夷陵(今湖北宜昌)等地。之后,汉大将岑彭曾多次出兵攻击蜀军,均以失败告终。

龙兴十年,刘秀成功夺取陇地后,便积极准备进攻成家。同年冬,刘秀命岑彭率军从东面水路进攻成家。龙兴十一年(公元35年)三月,南郡失守,蜀军守将田戎退守江州(今重庆)。见江州难以攻克,汉军绕开江州,直取垫江(今重庆合川)、平曲(今合川南)。南郡失守后,汉军又在北面从陆路展开攻势。成家军队在河池(今甘肃徽县)、下辨(今甘肃成县北)失守后接连败退。随后,岑彭派臧宫在涪江与蜀将延岑对峙,自己则率主力绕道岷江,准备攻打成都。

涪江一战,成家军队惨败,汉军乘胜追击,接连攻下涪县(今四川绵阳)、绵竹、繁县(今四川彭州东南)、郫县(今四川郫县北)、武阳(今四川彭山)、广都(今四川双流)等地,直逼成都。同时,在坚守十七个月后,江州也于龙兴十二年(公元36年)七月落入汉军之手。虽然公孙述派遣刺客成功将汉军主帅岑彭刺杀,并趁机夺回武阳等地,但失败已成定局。之后,成、汉军队在成都附近又激战数月。龙兴十二年十一月,公孙述在率军作战时身受重伤,不治身亡,次日延岑便开城投降。在经历了23个月的顽强抵抗后,成家被东汉所灭,历时12年。

成家立国之初(公元24年)仅占有西汉所置益州的蜀、巴、广汉三郡,后来成家积极向外扩张。公元25年成家占领犍为、越嶲(gui)两郡,公元26年又占领汉中、武都两郡。公元30年,占有陇西、金城、武威、张掖、酒泉、敦煌等郡的隗嚣归附成家。公元33年,成家又占领南郡,使成家疆域达到顶峰。

顶峰时期的成家疆域相当于现今中国四川省、重庆市的大部分地区;陕西省、甘肃省南部,贵州省北部和湖北省西部部分地区。若记入藩属于成家的隗嚣占据的地区,则还包括现今甘肃省的东部地区。

成家立国后,由于地域有限,所辖郡县数目不多,便改益州为虚设的司隶,没有实权,由朝廷直接管辖各郡。地方行政由原本的州郡县三级制变为郡县两级制。成家朝廷之下总共辖8郡,其郡县设置基本沿袭西汉旧制,变化不大。

成家的创立者公孙述原为西汉文官,熟知西汉典章。成家建立后,也基本沿袭西汉旧制,实行三公九卿制,但未设丞相(或类似官职)。大司徒、大司马、太尉称“三公”。大司徒掌教化礼仪,大司马掌监察百官,太尉掌管军事,是武官首长。成家还设有大司空,掌管某支军队,也被列入“三公”之中。九卿则是太常(掌祭祀鬼神)、光禄勋(掌门房)、卫尉(掌卫兵)、太仆(掌车马)、廷尉(掌法律)、大鸿胪(掌礼宾)、宗正(掌皇帝族谱)、大司农(掌全国经济)、少府(掌皇室财政)。

成家仅有封王,从未封侯,但公孙述曾经以封侯为诱,劝说益州郡守文齐降伏于成家,只是未能成功。公孙述称天子之后不久便将自己的两个儿子封王,以梓潼、犍为两郡数县为封地。之后,公孙述又相继将一些原本各自割据一方而后投奔成家的首领为王。公元29年,封田戎为翼江王,封延岑为汝宁王;公元31年,封隗嚣为朔宁王,隗嚣病逝后,又封其次子隗纯为朔宁王。另外,邛人长贵在公孙述称天子之前杀越嶲郡守,自立为邛谷王,后归顺成家,公孙述可能仍将其封为邛谷王。

\subsection{公孙述生平}

公孙述(?-36年),字子阳,右扶风茂陵(今陕西兴平县)人。两汉间政治人物。曾经割据蜀郡,並以「白帝」自比。

西汉末年,以父荫为郎,补清水(今属甘肃)县长。他为官有方,一方太平,因而闻名。王莽篡汉后,任导江卒正(即蜀郡太守)。新朝末年,自称辅汉将军兼任益州牧,势力大增,自稱為蜀王。

东汉光武帝建武元年(25年)四月,與劉秀同年自立为天子,国号“成家”,建元龙兴。公孙述迷信讳谶符命之说,废止铜钱,设官铸铁钱,一時間難以流通。好事者竊言“黃牛白腹,五銖當復。”建武五年(30年),光武帝派耿弇等由隴道伐公孫述,隗囂稱臣於公孫述,公孫述封其為“朔寧王”。建武十一年(35年),光武帝派兵征讨,不克。

建武十二年冬十一月戊寅(36年12月24日),东汉大司马吴汉、臧宫于成都打败了公孙述,公孙述受伤当夜死亡。吴汉破屠成都,纵兵大掠,白帝城化為灰燼,漢軍尽诛公孙氏及延岑,成家亡,存世凡十二年。應驗了公孫述稱帝前的夢:「八厶子系,十二為期。」


\subsection{龙兴\tiny(25-36)}

\begin{longtable}{|>{\centering\scriptsize}m{2em}|>{\centering\scriptsize}m{1.3em}|>{\centering}m{8.8em}|}
  % \caption{秦王政}\
  \toprule
  \SimHei \normalsize 年数 & \SimHei \scriptsize 公元 & \SimHei 大事件 \tabularnewline
  % \midrule
  \endfirsthead
  \toprule
  \SimHei \normalsize 年数 & \SimHei \scriptsize 公元 & \SimHei 大事件 \tabularnewline
  \midrule
  \endhead
  \midrule
  元年 & 25 & \tabularnewline\hline
  二年 & 26 & \tabularnewline\hline
  三年 & 27 & \tabularnewline\hline
  四年 & 28 & \tabularnewline\hline
  五年 & 29 & \tabularnewline\hline
  六年 & 30 & \tabularnewline\hline
  七年 & 31 & \tabularnewline\hline
  八年 & 32 & \tabularnewline\hline
  九年 & 33 & \tabularnewline\hline
  十年 & 34 & \tabularnewline\hline
  十一年 & 35 & \tabularnewline\hline
  十二年 & 36 & \tabularnewline
  \bottomrule
\end{longtable}

\subsection{刘盆子生平}

刘盆子(10年-?),汉高祖刘邦之孙城阳景王刘章之后。曾祖父为城阳荒王刘顺,祖父式侯刘宪,父式侯刘萌,新莽篡位,国除,为庶人,一度被赤眉軍立為皇帝,劉秀建立東漢後,封劉盆子為趙王劉良的郎中。

新莽天凤四年(17年),绿林赤眉起义爆发,赤眉军过式,刘盆子及其兄刘恭、刘茂为其掳略至军中。更始帝立,赤眉军奉更始帝为帝,刘盆子之兄刘恭入长安,以其通《尚书》,复封为式侯,以明经数言事,拜侍中。刘盆子及其兄刘茂仍留赤眉军中牧牛。

更始初年,樊崇等闻汉光复,遂降更始,率二十余人入洛阳。然更始帝安抚不力,赤眉军发生骚乱。樊崇等逃回赤眉军中,分其军为两部,开始叛乱。更始二年冬,已攻至弘农郡,赤眉军大盛。虽然连胜更始帝部队,但是赤眉军内部非常悲观。赤眉军多为齐人,盛行巫祝。赤眉诸将求军内巫祝求神降示,巫师言他们之所以不成功,人心涣散,是因为刘章发怒之故。城阳景王刘章,是西汉中期至汉末为齐地人士廣泛信仰之鄉土神,直至曹操除淫祀,其信仰才稍稍断绝。于是赤眉军于营中寻刘章后裔,得七十余人,唯刘茂、刘盆子兄弟与前西安侯刘孝与刘章的血缘最近。后赤眉军采用抽签方式,选定刘盆子为帝,年号建世。

刘盆子虽立,仍与其他牧儿游。赤眉军诸将多不识书术,唯徐宣曾为县吏,略知书,推为丞相,樊崇为御使大夫,逄安为左大司马,谢禄为右大司马,自杨音以下皆为列卿。赤眉军进至高陵,与更始叛将张卬等联合,攻下长安。不久更始帝向赤眉军投降。在这之前的六月,汉光武帝刘秀已于河北称帝,改更始三年为建武元年。

刘盆子居长乐宫,诸将日日争功,声言欢呼,拔剑击柱,不能相一。三辅郡县输入京师物资,辄为兵士剽掠。腊月,宫中举行宴会,结果席间大乱,席上酒肉为冲进来的军士哄抢而光。多人互鬥受伤。卫尉诸葛释闻讯后,率军前来维持秩序,格杀百余人后方才镇压下去。刘盆子对此十分惶恐,不敢独居。时掖庭仍有宫女数百千人,自更始败亡后皆闭门不出,在宫内掘食草根、池鱼,许多人因此饿死。后宫人见刘盆子,对其言饥,刘盆子怜之,每人赐米数斗。后刘盆子离去,宫人皆饿死。

时刘盆子之兄式侯刘恭亦在朝中,见赤眉乱,知其必不长久,为求自保,密教刘盆子封玺绶,习辞让之礼,以使兄弟免祸,但赤眉首领拒绝刘盆子逊位。赤眉军乏食,于长安周边大肆劫掠。刘秀大将邓禹攻赤眉,亦为所败。后赤眉与汉中贼延岑大战,死者万人。延岑战败,其部将李宝投降。后李宝为内应,延岑再次挑战,与赤眉军大战,李宝趁机拔去赤眉军旗帜,赤眉军以为已败,大为逃亡,自投山谷,死者十余万。时三辅地区发生饥荒,赤眉军无法劫掠到东西,于是引兵东归。

赤眉军东归,刘秀遣兵于其必经之地将其包围。赤眉军士众涣散,已无战斗力,只好向刘秀投降。

29年以後,刘秀任命刘盆子为刘秀叔父赵王刘良的郎中。后来刘盆子因病双目失明,刘秀又下令用荥阳的官田租税,来奉养刘盆子终身。

刘盆子墓在河北省深泽县西北留村,1970年前,古垄尚存,现墓迹已废。

范晔所著的《后汉书》中评价道:“圣公靡闻,假我风云,始顺归历,终然崩分。赤眉阻乱,盆子探符。虽盗皇器,乃食均输。”

\subsection{建世\tiny(25-27)}

\begin{longtable}{|>{\centering\scriptsize}m{2em}|>{\centering\scriptsize}m{1.3em}|>{\centering}m{8.8em}|}
  % \caption{秦王政}\
  \toprule
  \SimHei \normalsize 年数 & \SimHei \scriptsize 公元 & \SimHei 大事件 \tabularnewline
  % \midrule
  \endfirsthead
  \toprule
  \SimHei \normalsize 年数 & \SimHei \scriptsize 公元 & \SimHei 大事件 \tabularnewline
  \midrule
  \endhead
  \midrule
  元年 & 25 & \tabularnewline\hline
  二年 & 26 & \tabularnewline\hline
  三年 & 27 & \tabularnewline
  \bottomrule
\end{longtable}


%%% Local Variables:
%%% mode: latex
%%% TeX-engine: xetex
%%% TeX-master: "../Main"
%%% End:
