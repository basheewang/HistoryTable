%% -*- coding: utf-8 -*-
%% Time-stamp: <Chen Wang: 2019-12-17 21:09:32>

\section{质帝\tiny(145-146)}

\subsection{生平}

汉质帝刘缵(138年-146年7月26日),一名续,东汉第十位皇帝。145年3月6日即位,在位时间1年余,其正式諡號為「孝質皇帝」,後世省略「孝」字稱「漢質帝」。

前任皇帝汉冲帝駕崩时只有3岁,當時尊爲梁太后(漢順帝皇后)之弟梁冀拥立汉章帝玄孙刘缵为帝,承汉顺帝嗣,改元本初,是为汉质帝。

當時梁冀一家专权,朝政腐败,吏治不修。梁冀當時權勢極盛,威勢橫行朝廷和宮外;大臣們害怕梁冀的威勢,不敢抗命。质帝虽年幼,但他聪明伶俐,不堪梁冀的专横跋扈。质帝曾在朝見大臣時當面對梁冀说:「此跋扈将军也!」。

梁冀听罢,大为反感,便命手下在质帝的饼裏下毒弒君,146年7月26日,9岁的质帝食用毒餅後死亡。8月26日,葬於漢靜陵。

质帝崩后,继任的汉桓帝终于诛灭了梁氏。

\subsection{本初}

\begin{longtable}{|>{\centering\scriptsize}m{2em}|>{\centering\scriptsize}m{1.3em}|>{\centering}m{8.8em}|}
  % \caption{秦王政}\
  \toprule
  \SimHei \normalsize 年数 & \SimHei \scriptsize 公元 & \SimHei 大事件 \tabularnewline
  % \midrule
  \endfirsthead
  \toprule
  \SimHei \normalsize 年数 & \SimHei \scriptsize 公元 & \SimHei 大事件 \tabularnewline
  \midrule
  \endhead
  \midrule
  元年 & 146 & \tabularnewline
  \bottomrule
\end{longtable}

%%% Local Variables:
%%% mode: latex
%%% TeX-engine: xetex
%%% TeX-master: "../Main"
%%% End:
