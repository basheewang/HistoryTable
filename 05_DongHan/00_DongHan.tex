%% -*- coding: utf-8 -*-
%% Time-stamp: <Chen Wang: 2019-12-17 15:41:16>

\chapter{东汉\tiny(25-220)}

\section{简介}

东汉(25年-220年)是中国歷史上的一个朝代,是汉朝的一部分,与西汉合称兩漢。東漢与西漢之間为新朝,亡於綠林軍。东汉时期第一位皇帝是刘秀。西汉建都长安,東漢建都雒陽,故而得名。同時東漢又稱后汉和中汉。東漢是當時世界上的強大國家,而前期六十多年的光武中興和明章之治,亦是中國史上的盛世之一。

由於東漢中後期的帝王普遍壽命不長,而且不少是幼年即位,導致汉和帝以后至汉末近百年间,外戚及宦官轮流执政,成為固定的惡性循環,兩派互相残杀,把东汉朝廷弄得十分腐败。東漢中平六年(189年),外戚大將軍何进遭宦官十常侍所殺,而十常侍也被袁绍、袁术等人杀死,後并州牧董卓引兵到雒陽,废少帝刘辩,杀何太后,立献帝劉協。长期左右东汉皇室的外戚、宦官一起被消灭,但卻引起了各地州郡长官借此反对董卓并演化成群雄割據的局面,漢廷无法掌控全国,漢献帝也成為傀儡,其后被军阀曹操控制,最後汉室被曹魏取代。

東漢在文化、军事等方面亦有显著成就。涌现了蔡伦、张衡、张仲景、华佗等卓越人才;班超出使西域,在西域長駐了三十多年,先後擊破了被匈奴控制的西域諸國,不但令西域諸國一一歸順漢朝,並開拓了東西文化的交流。期間他還派出甘英出使西域的大秦國,雖未有成功,但足跡已達今日波斯灣諸國。

另外,东汉在91年灭北匈奴。南匈奴内附漢朝。216年,南匈奴最后一個呼厨泉单于去邺城拜见曹操,曹操分南匈奴为五部,匈奴国不复存在,困扰汉朝数百年的北方外患終告一段落。

同時佛教也在這段期間傳入中國。根據記載,汉哀帝元壽元年(前2年)博士弟子景盧出使大月氏,其王使人口授《浮屠經》。到了東漢永平十年(67年),漢明帝派人去西域,迎來兩位高僧,並且帶來了許多佛像和佛经,用白馬駝迴首都雒陽,皇帝命人修建房屋供其居住,翻譯佛經。也就是現在的白马寺。

光武中兴(25年-57年):新朝末年王莽改制失敗,並引發內戰,其時身為漢朝宗室的漢景帝後裔劉秀乘勢而起,在綠林軍的協助下推翻新莽而即位,是為光武帝。復國號漢,史稱東漢。同时因洛陽為其軍事根據地,而西汉旧都長安亦逢多次戰亂而日殘,所以定都于洛阳,并復名雒阳。建武二年(26年),光武帝下令整顿吏治,设尚书六人分掌国家大事,进一步削弱三公(太尉、司徒、司空)的权力;同时清查土地,新定稅金,振興農業,使人民生活逐步稳定下来,史稱光武中興。

明章之治(57年-88年):漢光武帝死後,明帝即位,命窦固、耿忠征伐北匈奴。汉军进抵天山,击呼衍王,斩首千余级,追至蒲类海(今新疆巴里坤湖),取伊吾卢地。其后,窦固又以班超出使西域,恢复了西域与汉朝的联系。明帝及其子章帝在位期間,為東漢的黃金時代,史稱明章之治。

外戚政治的勃兴(88年-159年):汉章帝是一个贤明的皇帝,但他却开东汉大力任用外戚之先河。在他死后,刚登基的汉和帝刘肇只是一个10岁的孩子,由他的养母窦太后执政。窦太后仰仗他的兄长窦宪,窦氏戚族开始掌权。尽管汉和帝后来联合宦官力量消灭了窦氏,但是东汉政治的格局已经无法扭转。

和帝去世后,汉殇帝刘隆年龄更小,只是一个刚满三个月的孩子。政权当然又到了外戚的手中。这一次由邓太后的兄长鄧騭为代表的邓氏戚族掌握实际权力。汉殇帝只当了約八個月的皇帝就去世。由他的堂兄刘祜即位,也就是汉安帝。汉安帝本身就是由邓氏戚族拥立的,所以自然也成了傀儡。邓太后死后,安帝才亲政,他消灭了邓氏。然而他却未能阻止其他外戚集团掌握权力,东汉王朝开始走向下坡路。

汉安帝死后,刘懿在阎氏戚族的支持下登基,即位二百余日后就因病去世。不久之后阎氏戚族就被宦官消灭。宦官拥立汉顺帝。但是汉顺帝对外戚继续放任自流,结果导致梁氏戚族长达20多年的专政。梁冀更是达到了外戚权力的巅峰,汉冲帝、汉质帝都被他牢牢控制。汉质帝仅仅因为一句怨言就被他毒死,汉桓帝即位。

政治局面的恶化(159年-189年):159年,汉桓帝联合宦官一起诛灭了梁氏。汉桓帝将与他同谋的十三个宦官封侯,宦官开始成为东汉政权的主导力量。问题是,宦官的腐败比外戚更甚。这引起了很多士大夫的不满,他们与外戚联合,一同对抗宦官。宦官当然不愿意放弃权力,双方斗争激烈。最终导致了两次黨錮之禍,正直的士大夫全被排斥出政府。汉灵帝即位后,昏庸无道,成天沉迷于女色。汉灵帝比桓帝更信用宦官,他曾指着两个恶名昭著的宦官说:“张让是我父,赵忠是我母。”汉灵帝把朝政全交给宦官,使政局更为恶化。184年,黄巾之乱爆发,东汉政府陷入混乱。黨錮终于被解除,但已经太迟了。

名存实亡的朝廷(189年-220年):189年,汉灵帝去世,汉少帝刘辩即位。外戚何进官拜大將軍,掌控朝廷,他打算铲除宦官势力。但是少帝的母亲何太后反对。此时,士大夫领袖的袁绍提出建议,让擁兵自重的西北軍董卓进京,逼迫何太后答应。何进同意了袁绍建议,一场铲除宦官的计划开始了。

然而,事情不幸泄漏。宦官当然不愿意坐以待毙,他们先下手为强,殺掉了何进。時在西園軍的袁绍闻讯,立即率军攻入皇宫,对宦官进行屠杀。大宦官张让挟持汉少帝逃走,追兵赶到,張讓自杀身亡。可此时董卓的武力已经到达帝都雒邑,外戚和宦官的势力同归于尽,董卓控制了朝廷。

董卓为了树立威望,他首先废掉了汉少帝,立他的弟弟刘协为皇帝,即汉献帝。190年,他又把汉少帝和何太后一起杀掉。这种倒行逆施的行径引起了地方官員的不满。他们推举歷代公卿的世族人士袁绍为代表,组成关东联军讨伐董卓。虽然这次战争虎头蛇尾,未能达到目的,但仍使董卓感到不安。董卓於是挟持漢獻帝迁都到长安,并且焚烧了雒陽,经营多年的京師雒陽城毁于一旦。与此同时,各地的地方官員拒絕聽命於董卓控制的朝廷,並演變為军阀积极扩充自己的势力,朝廷的威望荡然无存。

192年,大臣王允唆使董卓的部将吕布,合作謀杀了董卓,下令大赦。朝廷的权威一度恢复。然而不久之后,董卓幕府的部将李傕、郭汜卷土重来,王允被杀,呂布出逃,朝廷再度陷入混乱。195年,李傕、郭汜发生内斗,汉献帝刘协和群臣逃出长安,回到雒陽。但雒陽已是一片废墟,汉献帝陷入窘迫的处境;而地方早已是群雄割據,之後形成以袁紹、袁術為首的兩大陣營群雄相互爭戰局面。196年,曹操挾漢獻帝到許都(今許昌)。之後,曹操逐漸掌握朝廷權力,漢獻帝只能受制於曹操。自200年起,曹操在官渡、倉亭與袁紹決戰取勝,中原、河北之地盡歸曹操。之後更趁戰勝袁氏勢力的餘威,併吞荊州;然而208年,曹操在赤壁之戰中敗於孫權、劉備聯軍,曹操自此失去以武力征服南方的機會,使孫權、劉備二人在日後趁勢逐步發展勢力,形成日後三國鼎立之局。在212年及216年,曹操先後進位為魏公、魏王,並立嫡子曹丕為世子,打破漢初劉邦定下白馬之盟的規矩。220年曹操病逝,曹丕繼任丞相、魏王,同年曹丕逼迫漢獻帝禪讓而篡位,東漢結束。

王莽篡汉时期已经仅剩秦朝时的疆域,西域各国因为汉成帝的亂政而逐渐脱离管制。王莽末年中原战乱不断,遂放弃定襄、云中、五原、朔方、上郡、北地六郡。河套、陕北、晋西北、河北北部地方先后放弃。高句丽与林邑两国蚕食东北及南方国土。只有西南地区扩展至大盈江一带。

东汉汉顺帝永和五年(140年)的疆域政区包括司隶校尉部、十二州刺史部所察各郡、国(王国)、属国和西域长史府辖区,以及当时中国边区各族的分布地。

州作为行政区划,在西汉时期逐步发展,到东汉时期宣告形成。前106年,始在郡之上又设了十三行部,每部派一刺史,每个行部管辖若干郡(国)。但此时的行部是监察区,还不是真正意义上的行政区。东汉末年,地方多事。中平五年(188年),朝廷选重臣出任刺史,称州牧,掌一州军民。州从监察区变为行政区。至此,中国地方行政由原本的郡县两级制度变为州郡县三级制。十三个州为:司隶(治雒阳)、徐州(治剡县)、青州(治临淄)、豫州(治谯县)、冀州(治高邑)、并州(治晋阳)、幽州(治蓟县)、兖州(治昌邑)、凉州(治陇县)、益州(治雒县)、荆州(治汉寿)、扬州(治历阳)和交州(治龙编)。兴平元年(194年),又分雍州。则至东汉灭亡,全国有十四州。州从监察区变为行政区。

刘秀为了进一步加强对地方的控制,把刺史固定为州一级的地方长官。刺史处理地方政务,不通过三公,可直接上奏给皇帝,使地方郡县也直接置于皇帝的控制之下。

東漢政權和士族、豪族關係密切,代表了士族、豪族的利益,雲台二十八將幾乎全部都是豪族出身。士族在東漢社會起著主導作用,光武帝跟士族、豪族取得協調,成為奪取天下的主因。

东汉王朝在政治上防范功臣、宗室诸王及外戚专权,通过各种办法加以控制。让他们享受优厚的待遇,而不参与政治。对于朝中诸臣,督责尤严,史称“自是大臣难居相任”。

加强中央集权,一方面削弱三公权力,另一方面则扩大尚书台的权力。加强尚书台的职权。一切政务不再经三公管理。尚书台成为皇帝发号施令的执行机构,所有权力集中于皇帝一身。

东汉建立不久,就废除了执掌地方兵权的郡国都尉,以后又罢轻车、骑士、材官、楼船士及军假吏,实际上取消了地方军队。在和平时期,少量维持地方治安的郡县兵,皆由太守令长兼领,但在某些沿边及民族斗争紧张的地区,则设都尉或属国都尉别领。

随着豪强地主势力的膨胀,在田庄内部发展了一种部曲家兵制。这种部曲家兵,承担着镇压农民、维持地方治安的某种职能。正因为有这样一支武装队伍,东汉政府才能裁减、甚至在某些地区取消地方军队。

与削弱地方兵权的同时,还逐步扩大中央军队,在重要的沿边地区,设有边防军,为中央军队的一部分。东汉政府还经常用赦免和减罪的办法,募集犯罪的人戍守边疆。

两伍一什,五什一队,两队一屯,两屯一曲,曲由军候率领。部下设曲,部由部校尉和军司马率领。部队分辖若干部,将军下有长史、司马辅助。

農業技術得到較大進步,牛耕技術在東漢已普遍採用。西漢時僅限於少數地方。

在东汉时期灌溉工具、水利事业获得改进和发展。在中国各地已知发现的东汉墓葬裏,可以看到水田和池塘组合的模型,有从池塘通向水田的自流水渠,有的还在出口处安置闸门。汉灵帝时,宦官毕岚总结劳动人民的实践经验,创作翻车和渴乌,大大提高了灌溉水平。

许多已堙废的陂塘在东汉时期不仅得到了修复还扩建了,而且又新修了一批水利灌溉工程。比如在中国汝南地区的鸿隙陂,西汉时被堙废。东汉初,邓晨当地太守时,进行了修复,可以灌溉几千顷良田。后来又不断加以扩建。汉和帝时,太守何敞又在那里修治渠道,开垦良田三万多顷。又如下邳徐县北的蒲阳陂、庐江的芍陂、会稽的镜湖等,都是当时著名的灌溉工程。

东汉前期,中国各地还开凿了许多灌溉渠道,三辅、河内、山阳、河东、上党、太原、赵、魏及河西、江南地区,也都「穿渠灌溉」,有的地区还开辟了很多稻田。东汉时还有一项巨大的水利工程,那就是对黄河的治理。公元1世纪初,黄河在今河南、河北交界地区决堤,河道南移,改从千乘(山东高宛以北)入海。河水泛滥成灾,淹没了几十个县。汉明帝时,在著名水利专家王景、王吴的主持下,用「堰流法」修了浚仪渠,并从荥阳至千乘海口千余里间修渠筑堤,从而使河、汴分流。黄河由于收到了两堤的约束,水势足以冲刷沙土,通流入海。经过大规模的治理,终于战胜了黄河水患。此后八百年间,黄河没有改道,水灾也减少了。

在东汉时生产工具也有改进,出现了短辕一牛挽犁,这种工具操作灵活,便于在小块农田上耕作。这种短辕一牛挽犁的出现,是跟犁铧的改进结合在一起的。东汉时期,已经大量使用全铁制犁铧,它比以往的V形犁,刃端角度已逐渐缩小,不但起土省力,还可以深耕。此外,新型的全铁制的耕作工具也逐渐增多。在四川乐山崖墓石刻画像中见到的曲柄锄,是便于铲除杂草的中耕工具;四川绵阳发现的铁制钩镰,全长35厘米,是专用于收割的小型农具,操作起来很方便。

汉献帝末年,雍州刺史张既曾令陇西、天水、南安三郡富人造屋宅水碓。水碓在当时已经普遍采用了。水碓是用水力带动石碓的舂米工具,它比以前用柱臼或脚踏石碓舂米,不但省力,而且效率要高得多。考古发掘还不断有陶风车、陶磨盘模型出土,都说明农产品加工工具有了显著进步。

在东汉时牛耕技术也受到了重视。一些地方官吏会注意推广牛耕技术,铁犁牛耕技术已从中原向北方高原和江南一带推广。在中国陕西绥德县东汉画像石上的牛耕图,和米脂县东汉牛耕图,证明了在陕北高原的牛耕技术和中原地区已没有什么不同。任延做九真太守,在当地推广牛耕,田亩年年增辟。在西汉后期发明的精耕细作的区种法,到东汉时期得到了推广。

崔实所写的《四民月令》中,记载了地主田庄内精耕细作经营农业的一些情况。这样的田庄的农业经营,注意时令节气,重视杀草施肥,还通过不同的土壤性质,种植不同的作物,采用不同的种植密度。并能及时翻土晒田,双季轮作,提高土地的利用率。

由于农田水利工程的兴建,农耕工具的改进,农业耕作技术的提高以及精耕细作方法的推广,大大提高了社会生产力,使东汉时期的农业生产有了较大的发展。和帝永兴元年(公元105年)的垦田数字达到732万多顷,人口达到5325万多人。这个数字略低于西汉,但如果把东汉豪强地主隐瞒的田亩和人口包括在内,实际的垦田面积和人口数字肯定要超过西汉。

在东汉时期由于铁制农具的普及,钢铁需要量也增加,同时也推动了冶铁技术的改进。东汉初,在南阳地区的冶铁工人发明了鼓风炉(即水排),利用水力转动机械,使鼓风皮囊张缩,不断给高炉加氧。水力鼓风炉的发明,是冶炼技术史上的一大进步。

东汉时在铁器铸造方面已熟练地掌握了层叠铸造这一先进技术。在中国河南温县发现的一座烘范窑,出土了五百多套铸造车马器零件的叠铸泥范。把若干个泥范叠合起来,装配成套,一次就能铸造几个或几十个铸件。同时,叠铸技术有重大改进,由原来的双孔浇铸,改为单孔浇铸。叠铸技术的改进,进一步提高了生产效率,节省了原料。根据已有的考古发掘资料证明,和铁钉、铁锅、铁刀、铁剪、铁灯等的大量出土,证明东汉时铁制用具已普遍应用到生活的各方面。

在冶铁手工业中已经使用煤(石炭)做燃料。在中国河南巩县的冶铁遗址中曾发现混杂了泥土、草茎制成的煤饼,说明煤已被用来炼铁。巴蜀地区还利用天然煤气煮盐。

在纺织业方面,东汉初年已能用织花机织成色彩缤纷、花纹复杂的织锦。当时,蜀锦已驰名全国,襄邑(河南睢县)和齐(山东临淄)的丝织业特别发达。考古材料还证明,在边疆地区,丝织业也有很大的发展。新疆不少地方汉墓出土的红色杯纹罗,织造匀细,花纹规整,反映丝织工艺水平相当高。在同一地区出土了组织细密的织花毛织品,颜色鲜丽,显示出当时西北高度发展的毛纺织工艺技术。

东汉时期的书法、绘画艺术地位逐渐显露出来。大量產製畫像磚及雕刻墓碑,典型的圖案為製酒、收割、宴會等。

汉光武帝刘秀十分尊崇儒学,他的功臣集团中儒生也发挥着重要作用,甚至军事领袖也「皆有儒者气象」。「诸将之应运而兴者,亦皆多近于儒」,「东汉功臣多近儒」的情形。不仅东汉的帝王亲自倡导儒家经典的认真研读,太学和郡国官学都得到空前优越的发展条件,东汉私学也繁盛一时。社会上出现了一些累世专攻一经的士大夫家族。他们世代相继,广收门徒。许多名师教授的弟子,往往多至数百人乃至数千人 。

汉代童蒙教育的进步,是当时文化成就的突出内容之一。汉代童蒙教育在中国古代教育史上也有特别值得重视的地位。学习成绩优异的孩子,得到「圣童」、「奇童」、「神童」的称号。「神」童称谓,最早就是从东汉开始使用的。

東漢報仇風氣極為盛行,只要有人侮辱父母或師長便可殺之,雖為法律所不許但會被鄉里視為維護尊嚴的義行,故東漢有許多為了報仇殺人跑路但仍受歌頌讚許的記載。

105年,蔡伦在前人的基础上改进造纸术,使中国的文字记录方式开始逐步脱离使用竹简的时代。製陶业得到了发展,一些以前为豪门贵族专有的用品开始进入了寻常百姓家。黃河流域及長江流域均可見生產褐綠釉陶器。天文学家张衡以高超的工艺制造了浑天仪、地动仪等科学仪器,制造这些仪器的原理至今仍被广泛使用。東漢末年長沙太守張機(字仲景)亦為醫家,著有漢醫內科學經典——《傷寒雜病論》。东汉末年名医华佗是有记载以来第一位利用麻醉技术对病人进行手术治疗的外科医生。

在东汉时期,周边还有不少少数民族分布在中国边疆地区,比如南、北匈奴,西域、羌族、乌桓、鲜卑、蛮族等各个民族。

西汉以来中央与各少数民族之间保持的臣属关系到了王莽时期也陷于瓦解,匈奴乘机控制了整个西域及东北各族。东汉初期,当光武帝进行国内统一战争时,匈奴的势力有所发展。匈奴单于勾结河北、山西的割据势力,经常深入长城以南,进行骚扰掠夺。当时的东汉政权因无力对付匈奴势力,一直采取以防守为主的策略  。

公元46年(建武二十二年),匈奴贵族为了争夺王位,互相猜忌,出现裂痕。加上匈奴所在的蒙古草原上连年旱蝗灾害不断,「人畜饥疫,死耗太半」,匈奴遂分裂为南北两部。刘秀接受了南匈奴的归附,令其入居云中,东汉政府每年供给南匈奴一定数量的粮食、牛马及丝帛等物资。南匈奴的依附,极大地加强了东汉王朝的北方边防,对北方经济文化的发展,也具有积极作用。与此同时,南匈奴入居塞内,有更多机会接触中原地区的先进文化,也促进了其本身的发展。南匈奴长期和汉族杂居在一起,逐渐改变了游牧生活,学会了农业生产,文化也深受汉族影响。

南匈奴归附东汉后,北匈奴的势力大大削弱,在交战中,数次被南匈奴击败。北单于无奈,从公元51年后,不断向东汉王朝遣使朝贡,要求和亲。东汉政府内部,在对待北匈奴的问题上,有过多次争论。最终采取了礼尚往来,报答使者的办法,以求和好安边 。

由于匈奴力量的削弱,原来受匈奴奴役的乌桓、鲜卑,也先后摆脱了北匈奴的控制。南匈奴、乌桓、鲜卑归附东汉后,切断了北匈奴与中原的经济联系,使其最必需的粮食、布匹、食盐等日常生活品出现匮乏。北匈奴不断寇扰东汉的北部的缘边郡县。公元73年(永平十六年),东汉王朝大举反击,窦固等分兵四路,取得很大的军事胜利,往北追至蒲类海(新疆巴里坤湖),并留屯于伊吾卢城(新疆哈密)。到汉章帝时,北匈奴日益衰弱,先后有数十万口入塞投降。公元89年(永元元年),窦宪、耿秉等率领汉军会合南匈奴大举北进,与北单于交战,连战皆捷,降者前后二十余万人。在以后的二年内,北匈奴不断失败,向西迁移。以后,北匈奴的一部分越过中亚、西亚迁往欧洲。从这时起,匈奴东面的鲜卑族逐步西进,占据了匈奴的故地。

%% -*- coding: utf-8 -*-
%% Time-stamp: <Chen Wang: 2019-12-17 16:12:36>

\section{小政权}

\subsection{汉夏简介}

汉复(一作复汉、朔宁;元年:23年七月 - 末年:34年十月)是西汉末年隗嚣自立的年号,共计12年。

地皇四年八月(更始元年七月),隗嚣反新起事,檄文署“汉复元年七月”。

汉光武帝建武 九年(33年)春,隗嚣死,子隗纯立,仍然沿用汉复年号,直到34年十月隗纯降于东汉政权。

\subsection{隗囂生平}

隗\xpinyin*{囂}(前1世纪?-33年春),字季孟,天水成紀(今甘肃省静宁县)人。

隗囂出身隴右大族,在州郡為官,以知书通经闻名,陇上國師劉歆舉為國士。劉歆自杀後返乡。更始元年七月(新地皇四年八月,漢复元年七月)王莽兵連敗。隗囂与兄隗义及上邽杨广、冀县周宗攻下平襄,殺镇戎郡(治平襄)大尹起兵。以平陵方望爲軍師,隗囂趁勢攻克雍州,殺州牧陈庆;攻克安定,殺大尹王莽堂弟平阿侯王谭之子王向。

漢更始元年十月(新地皇四年十一月,漢复元年十月)王莽被殺。隗囂分兵攻占陇西、武都、金城、武威、张掖、酒泉和敦煌七郡。漢更始二年(24年)隗囂歸順更始帝,被封為右將軍,至冬,隗崔與隗义叛更始帝,隗囂遣將平之,因功封御史大夫。次年夏劉秀稱帝,隗囂勸更始帝東歸光武帝。因更始帝不允而逃回天水郡,自称西州大将军。其人谦爱士卒,倾身引接为布衣。交聘馬援為綏德將軍,得到光武帝的器重。建武二年(26年),大司徒邓禹屯云阳,西击赤眉军。裨将冯愔叛攻天水,隗囂迎击之,大敗馮愔于高平,缴獲冯愔的辎重。於是隗囂投奔光武帝,被封爲西州大将军知涼州朔方諸軍事;遣楊廣击败赤眉军攻陇之軍,追击,败赤眉军于乌氏泾阳间。陈仓吕鲔率数万軍結公孙述犯三輔,派兵助征西大将军冯异擊退之。建武六年(30年)公孙述犯南郡,光武帝下诏隗囂自天水伐蜀,隗囂拒絕。光武帝派建威大将军耿弇伐蜀意在滅隗。隗囂謝罪,然而仍與公孙述往來。

建武八年(32年)春天,光武帝派来歙袭取了略阳(今甘肃秦安陇城镇),隗嚣派王元擋之,為漢光武帝聯合河西的竇融所攻滅,隗嚣携家奔西城(今天水市西南),光武帝杀死作為政治人质的隗嚣之子隗恂,派吴汉、岑彭包围西城,隗嚣為公孙述援军救出。

建武九年(33年),隗囂憂憤而死,王元等立隗嚣少子隗纯为王,汉军来歙攻破洛门,隗纯投降,史稱東漢平隴西之戰。

\subsection{汉复\tiny(23-34)}

\begin{longtable}{|>{\centering\scriptsize}m{2em}|>{\centering\scriptsize}m{1.3em}|>{\centering}m{8.8em}|}
  % \caption{秦王政}\
  \toprule
  \SimHei \normalsize 年数 & \SimHei \scriptsize 公元 & \SimHei 大事件 \tabularnewline
  % \midrule
  \endfirsthead
  \toprule
  \SimHei \normalsize 年数 & \SimHei \scriptsize 公元 & \SimHei 大事件 \tabularnewline
  \midrule
  \endhead
  \midrule
  元年 & 23 & \tabularnewline\hline
  二年 & 24 & \tabularnewline\hline
  三年 & 25 & \tabularnewline\hline
  四年 & 26 & \tabularnewline\hline
  五年 & 27 & \tabularnewline\hline
  六年 & 28 & \tabularnewline\hline
  七年 & 29 & \tabularnewline\hline
  八年 & 30 & \tabularnewline\hline
  九年 & 31 & \tabularnewline\hline
  十年 & 32 & \tabularnewline\hline
  十一年 & 33 & \tabularnewline\hline
  十二年 & 34 & \tabularnewline
  \bottomrule
\end{longtable}

\subsection{成家简介}

成家(25年-36年,又称“大成”或“成”)是两汉之交在中国四川地区存在的一个独立政权,定都成都。成家由公孙述创立,鼎盛时期据有西汉所置益州大部分地区,即蜀郡、巴郡、广汉郡、犍为郡、越嶲郡、汉中郡全境和武都郡、南郡部分地区。成家成立之初推行了一些促进经济、文化发展的措施,在军阀割据混战的局势下,力保巴蜀太平,受到了蜀地人士拥护。但由于成、汉实力悬殊,公孙述治国失误等原因,成家最终为东汉所灭,前后历时12年。

成家是秦灭巴蜀后,巴蜀地区出现的第一个独立政权,也是四川历史上第一个完整占据巴蜀地区的政权。之后蜀地政权多纳入汉中盆地,北抵秦岭、东至三峡的格局也由此形成,其在四川历史上具有重要的地位。

西汉外戚王莽称帝建立新朝后,于天凤年间(公元14年至19年)提拔中散大夫公孙述为导江卒正(王莽改蜀郡为导江,改郡守为卒正),治临邛(今四川邛崃)。公孙述任导江卒正期间以才能出众而闻名于蜀中,以至于蜀地百姓将临邛城称为公孙述城。

公元23年,新朝覆灭,更始帝刘玄登基,将国号恢复为“汉”。在大乱初期尚很平静的巴蜀地区,也开始动荡起来。宗成被刘玄拜为“虎牙将军”,自汉中起兵,领兵入蜀。与此同时,王岑以拥立汉宗室刘辟为名,在雒城(今四川广汉)起兵,自称“定汉将军”,并攻陷成都。公孙述借宗成之力消灭掉王岑、刘辟后,见宗成掳掠暴横,恐对己不利,遂于临邛起兵,同时又派人诈称汉朝使者,授其辅汉将军、蜀郡太守兼益州牧印。公孙述摇身一变成为汉臣,加之其在蜀地名望甚高,其军队迅速击败宗成、攻占成都,并占领巴郡、广汉郡。次年,公孙述之弟公孙恢又在绵竹(今四川德阳)大败刘玄军队,加强了公孙述在益州的威望。随后,公孙述自立为蜀王,定都成都,并得到了多数巴蜀民众的拥护。此时蜀地民富兵强,吸引了众多臣子前来投靠,邛人、笮人首领也前来相贺。

由于自己并不是刘氏正宗,公孙述对称天子一直有所顾忌。其功曹李熊力谏,认为在蜀地称天子条件成熟,并称“天命无常,百姓与能。能者当之,王何疑焉?”,打消了公孙述的顾虑。于是公孙述放出风声,称有龙在公孙述宫殿中出现,其掌上还出现了“公孙帝”三字。在完成了称天子的舆论准备后,建武元年四月(公元25年),公孙述正式在成都称天子,以成都起事而定国号为“成家”,并改年号为“龙兴”。

在公孙述称天子之初,其实际控制的仅有蜀、巴和广汉三郡。虽然巴蜀以西的邛人、笮人在其称天子之前便已归附,随后公孙述又相继占领了犍为、越嶲两郡,但其却一直未能占领南部原属益州的益州、牂柯两郡。在后方尚未安定的情况下,公孙述却又不满足于偏安一隅,积极准备北出东进。

龙兴二年(公元26年),适逢更始帝出击占据汉中郡的严岑,在双方两败俱伤之时,公孙述趁机出兵北伐占领汉中。同年底亲率数十万大军进驻汉中,准备御驾亲征,直取关中。而此时,忙于关东战事无暇西顾的刘秀决意联合占据陇地的隗嚣来制衡成家。龙兴三年(公元27年),公孙述多次自汉中遣兵数万,攻打关中,却皆败于刘秀大将冯异与隗嚣的联军。之后隗嚣又出兵攻蜀,连破蜀军,迫使公孙述放弃了北伐关中的想法。

龙兴六年(公元30年),刘秀成功平定关东,欲讨伐隗嚣,但因其拒蜀有功,苦于没有借口,于是命隗嚣出兵讨伐成家。若隗嚣伐蜀,必定两败俱伤;若不从命,也有了讨伐其的理由。隗嚣察觉了刘秀的意图,婉言拒绝。于是刘秀亲率大军西征讨伐隗嚣。隗嚣见形势不利,经过反复考虑后,决定遣使者入蜀,称臣于公孙述。龙兴七年(公元31年),公孙述封隗嚣为朔宁王,并多次派兵相助,一时双方相持不下。龙兴九年(公元33年)春,隗嚣病逝,其次子隗纯被立为朔宁王。汉军再度发起攻势,在成家军队的协助下,双方相持近一年。龙兴十年(公元34年)七月,隗纯降汉。

龙兴九年隗嚣病逝后,公孙述派兵出击南郡,并成功夺取夷道(今湖北枝城)、夷陵(今湖北宜昌)等地。之后,汉大将岑彭曾多次出兵攻击蜀军,均以失败告终。

龙兴十年,刘秀成功夺取陇地后,便积极准备进攻成家。同年冬,刘秀命岑彭率军从东面水路进攻成家。龙兴十一年(公元35年)三月,南郡失守,蜀军守将田戎退守江州(今重庆)。见江州难以攻克,汉军绕开江州,直取垫江(今重庆合川)、平曲(今合川南)。南郡失守后,汉军又在北面从陆路展开攻势。成家军队在河池(今甘肃徽县)、下辨(今甘肃成县北)失守后接连败退。随后,岑彭派臧宫在涪江与蜀将延岑对峙,自己则率主力绕道岷江,准备攻打成都。

涪江一战,成家军队惨败,汉军乘胜追击,接连攻下涪县(今四川绵阳)、绵竹、繁县(今四川彭州东南)、郫县(今四川郫县北)、武阳(今四川彭山)、广都(今四川双流)等地,直逼成都。同时,在坚守十七个月后,江州也于龙兴十二年(公元36年)七月落入汉军之手。虽然公孙述派遣刺客成功将汉军主帅岑彭刺杀,并趁机夺回武阳等地,但失败已成定局。之后,成、汉军队在成都附近又激战数月。龙兴十二年十一月,公孙述在率军作战时身受重伤,不治身亡,次日延岑便开城投降。在经历了23个月的顽强抵抗后,成家被东汉所灭,历时12年。

成家立国之初(公元24年)仅占有西汉所置益州的蜀、巴、广汉三郡,后来成家积极向外扩张。公元25年成家占领犍为、越嶲(gui)两郡,公元26年又占领汉中、武都两郡。公元30年,占有陇西、金城、武威、张掖、酒泉、敦煌等郡的隗嚣归附成家。公元33年,成家又占领南郡,使成家疆域达到顶峰。

顶峰时期的成家疆域相当于现今中国四川省、重庆市的大部分地区;陕西省、甘肃省南部,贵州省北部和湖北省西部部分地区。若记入藩属于成家的隗嚣占据的地区,则还包括现今甘肃省的东部地区。

成家立国后,由于地域有限,所辖郡县数目不多,便改益州为虚设的司隶,没有实权,由朝廷直接管辖各郡。地方行政由原本的州郡县三级制变为郡县两级制。成家朝廷之下总共辖8郡,其郡县设置基本沿袭西汉旧制,变化不大。

成家的创立者公孙述原为西汉文官,熟知西汉典章。成家建立后,也基本沿袭西汉旧制,实行三公九卿制,但未设丞相(或类似官职)。大司徒、大司马、太尉称“三公”。大司徒掌教化礼仪,大司马掌监察百官,太尉掌管军事,是武官首长。成家还设有大司空,掌管某支军队,也被列入“三公”之中。九卿则是太常(掌祭祀鬼神)、光禄勋(掌门房)、卫尉(掌卫兵)、太仆(掌车马)、廷尉(掌法律)、大鸿胪(掌礼宾)、宗正(掌皇帝族谱)、大司农(掌全国经济)、少府(掌皇室财政)。

成家仅有封王,从未封侯,但公孙述曾经以封侯为诱,劝说益州郡守文齐降伏于成家,只是未能成功。公孙述称天子之后不久便将自己的两个儿子封王,以梓潼、犍为两郡数县为封地。之后,公孙述又相继将一些原本各自割据一方而后投奔成家的首领为王。公元29年,封田戎为翼江王,封延岑为汝宁王;公元31年,封隗嚣为朔宁王,隗嚣病逝后,又封其次子隗纯为朔宁王。另外,邛人长贵在公孙述称天子之前杀越嶲郡守,自立为邛谷王,后归顺成家,公孙述可能仍将其封为邛谷王。

\subsection{公孙述生平}

公孙述(?-36年),字子阳,右扶风茂陵(今陕西兴平县)人。两汉间政治人物。曾经割据蜀郡,並以「白帝」自比。

西汉末年,以父荫为郎,补清水(今属甘肃)县长。他为官有方,一方太平,因而闻名。王莽篡汉后,任导江卒正(即蜀郡太守)。新朝末年,自称辅汉将军兼任益州牧,势力大增,自稱為蜀王。

东汉光武帝建武元年(25年)四月,與劉秀同年自立为天子,国号“成家”,建元龙兴。公孙述迷信讳谶符命之说,废止铜钱,设官铸铁钱,一時間難以流通。好事者竊言“黃牛白腹,五銖當復。”建武五年(30年),光武帝派耿弇等由隴道伐公孫述,隗囂稱臣於公孫述,公孫述封其為“朔寧王”。建武十一年(35年),光武帝派兵征讨,不克。

建武十二年冬十一月戊寅(36年12月24日),东汉大司马吴汉、臧宫于成都打败了公孙述,公孙述受伤当夜死亡。吴汉破屠成都,纵兵大掠,白帝城化為灰燼,漢軍尽诛公孙氏及延岑,成家亡,存世凡十二年。應驗了公孫述稱帝前的夢:「八厶子系,十二為期。」


\subsection{龙兴\tiny(25-36)}

\begin{longtable}{|>{\centering\scriptsize}m{2em}|>{\centering\scriptsize}m{1.3em}|>{\centering}m{8.8em}|}
  % \caption{秦王政}\
  \toprule
  \SimHei \normalsize 年数 & \SimHei \scriptsize 公元 & \SimHei 大事件 \tabularnewline
  % \midrule
  \endfirsthead
  \toprule
  \SimHei \normalsize 年数 & \SimHei \scriptsize 公元 & \SimHei 大事件 \tabularnewline
  \midrule
  \endhead
  \midrule
  元年 & 25 & \tabularnewline\hline
  二年 & 26 & \tabularnewline\hline
  三年 & 27 & \tabularnewline\hline
  四年 & 28 & \tabularnewline\hline
  五年 & 29 & \tabularnewline\hline
  六年 & 30 & \tabularnewline\hline
  七年 & 31 & \tabularnewline\hline
  八年 & 32 & \tabularnewline\hline
  九年 & 33 & \tabularnewline\hline
  十年 & 34 & \tabularnewline\hline
  十一年 & 35 & \tabularnewline\hline
  十二年 & 36 & \tabularnewline
  \bottomrule
\end{longtable}

\subsection{刘盆子生平}

刘盆子(10年-?),汉高祖刘邦之孙城阳景王刘章之后。曾祖父为城阳荒王刘顺,祖父式侯刘宪,父式侯刘萌,新莽篡位,国除,为庶人,一度被赤眉軍立為皇帝,劉秀建立東漢後,封劉盆子為趙王劉良的郎中。

新莽天凤四年(17年),绿林赤眉起义爆发,赤眉军过式,刘盆子及其兄刘恭、刘茂为其掳略至军中。更始帝立,赤眉军奉更始帝为帝,刘盆子之兄刘恭入长安,以其通《尚书》,复封为式侯,以明经数言事,拜侍中。刘盆子及其兄刘茂仍留赤眉军中牧牛。

更始初年,樊崇等闻汉光复,遂降更始,率二十余人入洛阳。然更始帝安抚不力,赤眉军发生骚乱。樊崇等逃回赤眉军中,分其军为两部,开始叛乱。更始二年冬,已攻至弘农郡,赤眉军大盛。虽然连胜更始帝部队,但是赤眉军内部非常悲观。赤眉军多为齐人,盛行巫祝。赤眉诸将求军内巫祝求神降示,巫师言他们之所以不成功,人心涣散,是因为刘章发怒之故。城阳景王刘章,是西汉中期至汉末为齐地人士廣泛信仰之鄉土神,直至曹操除淫祀,其信仰才稍稍断绝。于是赤眉军于营中寻刘章后裔,得七十余人,唯刘茂、刘盆子兄弟与前西安侯刘孝与刘章的血缘最近。后赤眉军采用抽签方式,选定刘盆子为帝,年号建世。

刘盆子虽立,仍与其他牧儿游。赤眉军诸将多不识书术,唯徐宣曾为县吏,略知书,推为丞相,樊崇为御使大夫,逄安为左大司马,谢禄为右大司马,自杨音以下皆为列卿。赤眉军进至高陵,与更始叛将张卬等联合,攻下长安。不久更始帝向赤眉军投降。在这之前的六月,汉光武帝刘秀已于河北称帝,改更始三年为建武元年。

刘盆子居长乐宫,诸将日日争功,声言欢呼,拔剑击柱,不能相一。三辅郡县输入京师物资,辄为兵士剽掠。腊月,宫中举行宴会,结果席间大乱,席上酒肉为冲进来的军士哄抢而光。多人互鬥受伤。卫尉诸葛释闻讯后,率军前来维持秩序,格杀百余人后方才镇压下去。刘盆子对此十分惶恐,不敢独居。时掖庭仍有宫女数百千人,自更始败亡后皆闭门不出,在宫内掘食草根、池鱼,许多人因此饿死。后宫人见刘盆子,对其言饥,刘盆子怜之,每人赐米数斗。后刘盆子离去,宫人皆饿死。

时刘盆子之兄式侯刘恭亦在朝中,见赤眉乱,知其必不长久,为求自保,密教刘盆子封玺绶,习辞让之礼,以使兄弟免祸,但赤眉首领拒绝刘盆子逊位。赤眉军乏食,于长安周边大肆劫掠。刘秀大将邓禹攻赤眉,亦为所败。后赤眉与汉中贼延岑大战,死者万人。延岑战败,其部将李宝投降。后李宝为内应,延岑再次挑战,与赤眉军大战,李宝趁机拔去赤眉军旗帜,赤眉军以为已败,大为逃亡,自投山谷,死者十余万。时三辅地区发生饥荒,赤眉军无法劫掠到东西,于是引兵东归。

赤眉军东归,刘秀遣兵于其必经之地将其包围。赤眉军士众涣散,已无战斗力,只好向刘秀投降。

29年以後,刘秀任命刘盆子为刘秀叔父赵王刘良的郎中。后来刘盆子因病双目失明,刘秀又下令用荥阳的官田租税,来奉养刘盆子终身。

刘盆子墓在河北省深泽县西北留村,1970年前,古垄尚存,现墓迹已废。

范晔所著的《后汉书》中评价道:“圣公靡闻,假我风云,始顺归历,终然崩分。赤眉阻乱,盆子探符。虽盗皇器,乃食均输。”

\subsection{建世\tiny(25-27)}

\begin{longtable}{|>{\centering\scriptsize}m{2em}|>{\centering\scriptsize}m{1.3em}|>{\centering}m{8.8em}|}
  % \caption{秦王政}\
  \toprule
  \SimHei \normalsize 年数 & \SimHei \scriptsize 公元 & \SimHei 大事件 \tabularnewline
  % \midrule
  \endfirsthead
  \toprule
  \SimHei \normalsize 年数 & \SimHei \scriptsize 公元 & \SimHei 大事件 \tabularnewline
  \midrule
  \endhead
  \midrule
  元年 & 25 & \tabularnewline\hline
  二年 & 26 & \tabularnewline\hline
  三年 & 27 & \tabularnewline
  \bottomrule
\end{longtable}


%%% Local Variables:
%%% mode: latex
%%% TeX-engine: xetex
%%% TeX-master: "../Main"
%%% End:

%% -*- coding: utf-8 -*-
%% Time-stamp: <Chen Wang: 2019-12-17 16:21:25>

\section{光武帝\tiny(25-57)}

\subsection{生平}

漢光武帝劉秀(前5年1月15日-57年3月29日),字文叔,小名呼 南陽郡蔡陽縣人[a](今湖北省襄阳枣阳市),祖籍徐州沛縣,東漢第一位皇帝,25年8月5日-57年3月29日在位。廟號世祖,諡號光武皇帝。

劉秀為漢高帝九世孫,漢景帝七世孫,长沙定王刘发之后,出身於南陽郡的地方豪族。新朝末年國家動蕩,各地寇盜蜂起。地皇三年(22年),劉秀與其兄長劉縯在宛(今河南省南陽市)起兵。25年,在鄗縣(今河北省石家莊市高邑縣)登基稱帝,改元建武,國號為「漢」,史稱東漢。此後,劉秀逐步掃平各方勢力,最終統一中國。劉秀在位三十二年,社會逐漸從新朝末年的動蕩中恢復,故稱「光武中興」。建武中元二年(57年),劉秀逝世於雒陽。

劉秀的軍事才能很高。稱帝之後遣眾將攻伐四方,往往能從前方上報的排兵布陣形勢中發現問題,有時因前方不能及時得到糾正,便為敵人所敗。此外,劉秀待人誠懇簡約,寬厚有信,竇融、馬援等均由此歸心。對外政策方面,引南匈奴內遷入塞,分置諸部於北地、朔方、五原、雲中、定襄、雁門、代郡、西河緣邊八郡,詔單于徙居西河美稷。但此舉也成為東漢朝廷和民眾沈重的經濟負擔,在東漢與北匈奴的戰爭中南匈奴僅起到出兵助攻的作用,談不上替東漢守衛北邊。到了東漢中期由於羌患,使得南匈奴在北邊不斷發起暴亂,對東漢北邊邊防乃至北方內地的安全構成了嚴重威脅,從而成為東漢北邊的一大邊患。

漢哀帝建平元年十二月甲子(前5年1月15日)夜於陳留郡濟陽縣出生。刘秀出生的时候,有赤光照耀整個房間,當年稻禾(嘉禾)一茎九穗,因此得名秀。

刘秀是汉高帝刘邦九世孙,西汉景帝子长沙定王刘发之子舂陵節侯劉買的玄孙,與更始帝有同一位高祖父劉買。其父为南顿令劉欽,母樊娴都。世代居住在南阳郡蔡陽(今湖北省枣阳市西南),屬地方豪族。刘秀九岁时,父亲逝世,便由叔父劉良抚养。由于刘秀勤于农事,而兄劉縯好侠养士,经常取笑刘秀,将他比做刘邦的兄弟刘喜。新朝天凤年间(14年—19年),刘秀至长安,学习《尚书》,略通大义。成年後劉秀身高七尺三寸(身高175厘米以上)。

刘秀在新野县时,听闻陰麗華的美貌,心悦之。後至長安,見執金吾車騎甚盛,因歎曰:“仕宦當作執金吾,娶妻當得陰麗華。”

時值新莽天鳳五年(17年),天下大亂,赤眉軍與綠林軍各自起兵反王莽。地皇三年(22年),劉秀避吏於新野,因賣穀而至宛(今河南省南陽市),經李通勸說在宛起兵。地皇四年(23年)二月,劉縯、劉秀兄弟與綠林兵共同擁護劉玄稱帝,國號仍為漢,改元更始,史稱更始帝。同年劉秀率綠林軍1萬以少勝多於昆陽滅王莽軍42萬,殺其主帥王尋,史稱昆陽之戰。

此後劉縯、劉秀兄弟威望大盛,遭到劉玄的猜忌。劉秀有所察覺,但劉縯不以為意,終被劉玄藉故殺死,同被殺死的還有同宗劉稷。此時劉秀也處於危險之中,只得向劉玄謝罪,並不敢為哥哥服喪,飲食言笑如常。劉玄心有所慚,故而拜劉秀為破虜大將軍、武信侯。

後更始帝劉玄攻占长安,新莽灭亡。时河北王郎起兵,于是更始帝派劉秀巡視黃河以北,劉秀始得脫離險境。劉秀遂在河北積蓄力量,日益壯大,被更始帝封為「蕭王」。劉秀率吳漢、鄧禹等手下大將,繼續在北方大破銅馬等割據勢力,被關西人號為「銅馬帝」。由於劉秀與更始帝心生二意,自此劉秀手下便不斷勸進。

更始三年(25年)六月,赤眉军立刘盆子为傀儡皇帝。同月二十二己未日(25年8月5日),劉秀于鄗城即皇帝位,改元建武,国号仍为汉,史称东汉。九月,赤眉军击败刘玄,漢更始帝投降,同年十二月被杀。

劉秀因为汉朝是火德的缘故迁都洛阳,改洛阳为「雒阳」。刘秀先后荡平赤眉、张步、隗嚣等割据势力,然割据一方的卢芳,刘秀屡次遣吴汉、杜茂往击,均不克。建武十二年(36年),卢芳进攻云中郡,留守九原的部将随育胁迫卢芳降伏刘秀。卢芳放弃军队,逃往匈奴。同年十一月十九己卯日(36年12月25日),吴汉攻克成都,割据四川的公孙述成家政权灭亡,东汉统一中国。

公元前108年,漢武帝滅朝鮮。 建武中元二年(公元57年),第一次有明文記載「倭人國家」與中國往來。九州北部(博多灣沿岸)的倭奴國接受漢王朝的策封,光武帝封其為「倭奴國王」,並授予金印。

1784年,在日本北九州地區博多灣志賀島,出土一枚刻有「漢委奴國王」五個字的金印。這一枚金印也為中日兩國最早交往的證明。

刘秀勤于政事,“每旦视朝,日仄乃罢,数引公卿郎将议论经理,夜分乃寐”。在位期间,多次发布释放奴婢和禁止残害奴婢的诏书。为减少贫民卖身为奴婢,经常发救济粮,减少租徭役,兴修水利,发展农业生产。裁并郡县,精简官员。结果,裁并四百余县,官员十置其一。历史上称其统治时期为光武中兴。其间国势昌隆,号称“建武盛世”。 刘秀统一中国后,厌武事,不言军旅,建武二十七年(51年),朗陵侯臧宫、扬虚侯马武上书:请乘匈奴分裂、北匈奴衰弱之际发兵击之,立“万世刻石之功”。光武却下诏:“今国无善政,灾变不息,人不自保,而复欲远事边外乎!……不如息民。”刘秀不同于明太祖朱元璋得天下后诛杀大批功臣只留下汤和徐達的无情,刘秀分封三百六十多位功臣为列侯,给予他们尊崇的地位,只解其兵权,刘秀诛杀功臣一说源于戏剧,令刘秀蒙受「不白之冤」。其实,在统一中国之前,他就开始削弱国防建设,废郡国兵制,罢郡国都尉。削弱地方兵权的同时,导致后来无力抵御外患,而豪强地主的部曲家兵则迅速发展,像东汉末年的董卓就是一例。刘秀以后不设丞相,而是“虽置三公”但“事归台阁”;一方面削弱三公权力,使三公成为虚位,另一方面又扩大尚书台的职权,成为皇帝发号施令的执行机构,所有权力集中于皇帝一身。”《后汉书·申屠刚传》说:“时内外群官,多帝自选举,加以法理严察,职事过苦,尚书近臣,至乃捶扑牵曳于前,群臣莫敢正言。”“自是大臣难居相任”。建武二十八年(52年)他借故搜捕王侯宾客,“坐死者数千人”,严禁结党营私。

建武中元二年二月初五日戊戌(57年3月29日),崩于雒阳南宫前殿,享壽六十二岁,在位三十二年。三月丁卯(4月27日),安葬于漢原陵(今河南孟津县铁谢村附近),庙号世祖,谥光武皇帝。刘秀駕崩后,其子汉明帝刘庄将统一战争中功劳最大的二十八人的影像画在云台阁,称云台二十八将。

《资治通鉴》称刘秀是个宽厚简易的人。在统一过程中,刘玄的一些手下曾参与谋害他的哥哥,他能够不计前嫌地招降并厚待;分封功臣时,不顾他人劝说,将最大的封地劃到了四县之广;战争尚未结束,就将原来十分之一的税率减到三十分之一;马援为隗嚣所使,分别访问公孙述和刘秀,独为刘秀的人格魅力折服;耿弇、窦融曾专制一方,以兵多权大心不自安,而刘秀对他们未有半点疑虑。凡此种种,都成为他成功的决定性因素。甚至在统一之后,他废郭皇后及太子劉彊,立阴皇后及次子刘阳(后改名莊),犹能令郭皇后到其子中山王的封国安享餘年,两子之间不生嫌隙,也没有受到臣下及后人的议论。

范曄:「雖身濟大業,競競如不及,故能明慎政體,總欖權綱,量時度力,舉無過事,退功臣而進文吏,戢弓矢而散馬牛,雖道未方古,斯亦止戈之武焉。」

诸葛亮:“光武神略计较,生于天心,故帷幄无他所思,六奇无他所出,于是以谋合议同,共成王业而已。”

明朝官修皇帝实录《明太祖实录》记载,明太祖朱元璋在洪武七年八月初一日(1374年9月7日),亲自前往南京历代帝王庙祭祀三皇、五帝、夏禹王、商汤王、周武王、汉高祖、汉光武帝、隋文帝、唐太宗、宋太祖、元世祖一共十七位帝王,其中对汉光武帝刘秀的祝文是:“惟汉光武皇帝延揽英雄,励精图治,载兴炎运,四海咸安。有君天下之德而安万世之功者也。元璋以菲德荷天佑人助,君临天下,继承中国帝王正统,伏念列圣去世已远,神灵在天,万古长存,崇报之礼,多未举行,故于祭祀有阙。是用肇新庙宇于京师,列序圣像及历代开基帝王,每岁祀以春、秋仲月,永为常典。今礼奠之初,谨奉牲醴、庶品致祭,伏惟神鉴。尚享!”

王夫之《读通鉴论》:“光武之得天下,较高帝而尤难矣。光武之神武不可测也!三代而下,取天下者,唯光武独焉!”“自三代而下,唯光武允冠百王矣”。

光武帝承袭西汉后期法律宽松的弊病,又过于剝奪三公的職權,明章之治以后皇帝幼小,陷入了长期的戚宦之爭之黑暗和混乱。

刘秀迷信图谶,與不信谶的大臣發生衝突,且时而感情用事,处事不公,韩歆因直谏被逼死,劉秀又包庇湖阳公主险些杀死董宣。

刘秀縱容部下吴汉军对邓奉的家乡进行劫掠,导致邓奉反叛,后来邓奉兵败投降被杀。平狄将军庞萌与盖延共击董宪,而诏书却只下达给盖延、不给庞萌,庞萌以为盖延说自己坏话,起疑,反叛,后来庞萌兵败被杀。

\subsection{建武}

\begin{longtable}{|>{\centering\scriptsize}m{2em}|>{\centering\scriptsize}m{1.3em}|>{\centering}m{8.8em}|}
  % \caption{秦王政}\
  \toprule
  \SimHei \normalsize 年数 & \SimHei \scriptsize 公元 & \SimHei 大事件 \tabularnewline
  % \midrule
  \endfirsthead
  \toprule
  \SimHei \normalsize 年数 & \SimHei \scriptsize 公元 & \SimHei 大事件 \tabularnewline
  \midrule
  \endhead
  \midrule
  元年 & 25 & \tabularnewline\hline
  二年 & 26 & \tabularnewline\hline
  三年 & 27 & \tabularnewline\hline
  四年 & 28 & \tabularnewline\hline
  五年 & 29 & \tabularnewline\hline
  六年 & 30 & \tabularnewline\hline
  七年 & 31 & \tabularnewline\hline
  八年 & 32 & \tabularnewline\hline
  九年 & 33 & \tabularnewline\hline
  十年 & 34 & \tabularnewline\hline
  十一年 & 35 & \tabularnewline\hline
  十二年 & 36 & \tabularnewline\hline
  十三年 & 37 & \tabularnewline\hline
  十四年 & 38 & \tabularnewline\hline
  十五年 & 39 & \tabularnewline\hline
  十六年 & 40 & \tabularnewline\hline
  十七年 & 41 & \tabularnewline\hline
  十八年 & 42 & \tabularnewline\hline
  十九年 & 43 & \tabularnewline\hline
  二十年 & 44 & \tabularnewline\hline
  二一年 & 45 & \tabularnewline\hline
  二二年 & 46 & \tabularnewline\hline
  二三年 & 47 & \tabularnewline\hline
  二四年 & 48 & \tabularnewline\hline
  二五年 & 49 & \tabularnewline\hline
  二六年 & 50 & \tabularnewline\hline
  二七年 & 51 & \tabularnewline\hline
  二八年 & 52 & \tabularnewline\hline
  二九年 & 53 & \tabularnewline\hline
  三十年 & 54 & \tabularnewline\hline
  三一年 & 55 & \tabularnewline\hline
  三二年 & 56 & \tabularnewline
  \bottomrule
\end{longtable}

\subsection{建武中元}

\begin{longtable}{|>{\centering\scriptsize}m{2em}|>{\centering\scriptsize}m{1.3em}|>{\centering}m{8.8em}|}
  % \caption{秦王政}\
  \toprule
  \SimHei \normalsize 年数 & \SimHei \scriptsize 公元 & \SimHei 大事件 \tabularnewline
  % \midrule
  \endfirsthead
  \toprule
  \SimHei \normalsize 年数 & \SimHei \scriptsize 公元 & \SimHei 大事件 \tabularnewline
  \midrule
  \endhead
  \midrule
  元年 & 56 & \tabularnewline\hline
  二年 & 57 & \tabularnewline
  \bottomrule
\end{longtable}


%%% Local Variables:
%%% mode: latex
%%% TeX-engine: xetex
%%% TeX-master: "../Main"
%%% End:

%% -*- coding: utf-8 -*-
%% Time-stamp: <Chen Wang: 2019-12-17 16:24:28>

\section{明帝\tiny(57-75)}

\subsection{生平}

漢明帝劉莊(28年6月15日-75年9月5日),原名刘阳,字子丽,东汉第二位皇帝,在位十八年。其正式諡號為「孝明皇帝」,後世省略「孝」字稱「漢明帝」,庙号显宗。汉光武帝刘秀的第四子,母亲为光烈皇后阴丽华。

汉明帝生于建武四年五月甲申(28年6月15日)。他从小就聪明好学,十岁时能够通读《春秋》。

建武十五年(39年)封东海公,十七年(41年)进爵为东海王,十九年(43年)被立为皇太子。建武中元二年初五戊戌(57年3月29日),三十岁的刘庄即皇帝位。

明帝即位后,一切遵奉汉光武帝的制度。明帝热心提倡儒学,注重刑名文法,为政苛察,总揽权柄,权不借下。他严令后妃之家不得封侯与政,对贵戚功臣也多方防范。同时,基本上消除了因为王莽虐政而引起的周边蛮夷侵扰的威胁,使汉跟周边蛮夷的友好关系得到了恢复和发展。

明帝允北匈奴互市之请,但并未消弥北匈奴的寇掠,反而动摇了早已归附的南匈奴。只得改变光武时期息兵养民的策略,重新对匈奴开战。永平十六年(73年),命祭肜、窦固、耿秉、來苗征伐北匈奴,汉军进抵天山,击呼衍王,斩首千余级,追至蒲类海(今新疆巴里坤湖),取伊吾卢地。永平十七年(74年),命窦固、耿秉、劉張征白山虜於蒲類海,复置西域都护府,用来管辖西域地区。其后,窦固又以班超出使西域,由是西域诸国皆遣子入侍。自新朝地皇四年(23年)以来,西域与中原断绝关系50年后又恢复了正常交往。班超以三十六人征服鄯善、于寘诸国、耿恭守疏勒城力拒匈奴等故事都发生在这一时期。

此外,随着对外交往的正常发展,佛教已在西汉末年传入西域,永平十年(67年),明帝梦见金人,其名曰佛,于是派使者赴天竺求得其书及沙门,并于雒阳建立中国第一座佛教庙宇白马寺。

明帝之世,吏治非常清明,境内安定。加以多次下诏招抚流民,以郡国公田赐贫人、贷种食,并兴修水利。因此,史书记载当时民安其业,户口滋殖。据《后汉书》记载:光武帝建武中元二年(57年),人口为2100万,至汉明帝永平十八年(75年),在不到20年的时间里增加至3412万。明帝以及随后的章帝在位时期,史称“明章之治”。

永平十八年八月初六壬子(75年9月5日),汉明帝逝世于雒阳东宫前殿,终年四十八岁。八月壬戌(9月15日),葬于显节陵(今河南洛阳市东南)。庙号显宗,谥号孝明皇帝。

\subsection{永平}

\begin{longtable}{|>{\centering\scriptsize}m{2em}|>{\centering\scriptsize}m{1.3em}|>{\centering}m{8.8em}|}
  % \caption{秦王政}\
  \toprule
  \SimHei \normalsize 年数 & \SimHei \scriptsize 公元 & \SimHei 大事件 \tabularnewline
  % \midrule
  \endfirsthead
  \toprule
  \SimHei \normalsize 年数 & \SimHei \scriptsize 公元 & \SimHei 大事件 \tabularnewline
  \midrule
  \endhead
  \midrule
  元年 & 58 & \tabularnewline\hline
  二年 & 59 & \tabularnewline\hline
  三年 & 60 & \tabularnewline\hline
  四年 & 61 & \tabularnewline\hline
  五年 & 62 & \tabularnewline\hline
  六年 & 63 & \tabularnewline\hline
  七年 & 64 & \tabularnewline\hline
  八年 & 65 & \tabularnewline\hline
  九年 & 66 & \tabularnewline\hline
  十年 & 67 & \tabularnewline\hline
  十一年 & 68 & \tabularnewline\hline
  十二年 & 69 & \tabularnewline\hline
  十三年 & 70 & \tabularnewline\hline
  十四年 & 71 & \tabularnewline\hline
  十五年 & 72 & \tabularnewline\hline
  十六年 & 73 & \tabularnewline\hline
  十七年 & 74 & \tabularnewline\hline
  十八年 & 75 & \tabularnewline
  \bottomrule
\end{longtable}


%%% Local Variables:
%%% mode: latex
%%% TeX-engine: xetex
%%% TeX-master: "../Main"
%%% End:

%% -*- coding: utf-8 -*-
%% Time-stamp: <Chen Wang: 2021-11-01 11:25:35>

\section{章帝劉炟\tiny(75-88)}

\subsection{生平}

汉章帝劉\xpinyin*{炟}(57年-88年4月9日),汉明帝刘庄第五子,东汉第三位皇帝(75年9月5日—88年4月9日在位),其正式諡號為「孝章皇帝」,後世省略「孝」字稱「漢章帝」,庙号肃宗。在位13年,享年僅32岁。

建武中元二年(57年)二月,其父刘庄即位。同年,刘炟出生,为汉明帝刘庄第五子。生母贾氏,被册封为贵人。当时,父亲明帝宠爱的马贵人无子,于是明帝把刘炟交给马贵人抚养。同时,刘炟生母贾贵人亦是马贵人同父异母姐姐的女儿。马贵人尽心抚育刘炟,操劳超过其亲生。刘炟亦“孝性淳笃,恩性天至”。养母养子之间关系容融洽,始终没有嫌隙。永平三年(60年)春,马贵人被立为皇后,同年,刘炟被立为太子。年少时,刘炟为人宽容,喜好儒家学说,明帝亦器重他。

永平十八年八月初六壬子(75年9月5日),刘炟即位,是为汉章帝。他即位后,励精图治,注重农桑,兴修水利,减轻徭役,衣食朴素,实行“与民休息”,并且“好儒术”,使得东汉经济、文化在此时得到很大的发展。这时思想也比较活跃,如王充等。此时政治清明,经济繁荣。章帝还两度派班超出使西域,使得西域地区重新称藩于汉,与汉明帝共称「明章之治」的東漢盛世。

建初四年(79年),章帝召开并参与了在洛阳的白虎观会议。

章帝為明主是毫無疑問的,只不过他卻因在位期間过于放纵外戚竇氏,导致汉和帝时期外戚专权,种下了東漢晚期外戚专权和宦官专政的远因。

建初八年(83年),使人鑄一把金劍,令投於伊水之中。

章和二年二月三十壬辰(88年4月9日),章帝崩於章德前殿,三月癸卯(4月20日),葬于敬陵(今河南洛阳东南)。

\subsection{建初}

\begin{longtable}{|>{\centering\scriptsize}m{2em}|>{\centering\scriptsize}m{1.3em}|>{\centering}m{8.8em}|}
  % \caption{秦王政}\
  \toprule
  \SimHei \normalsize 年数 & \SimHei \scriptsize 公元 & \SimHei 大事件 \tabularnewline
  % \midrule
  \endfirsthead
  \toprule
  \SimHei \normalsize 年数 & \SimHei \scriptsize 公元 & \SimHei 大事件 \tabularnewline
  \midrule
  \endhead
  \midrule
  元年 & 76 & \tabularnewline\hline
  二年 & 77 & \tabularnewline\hline
  三年 & 78 & \tabularnewline\hline
  四年 & 79 & \tabularnewline\hline
  五年 & 80 & \tabularnewline\hline
  六年 & 81 & \tabularnewline\hline
  七年 & 82 & \tabularnewline\hline
  八年 & 83 & \tabularnewline\hline
  九年 & 84 & \tabularnewline
  \bottomrule
\end{longtable}

\subsection{元和}

\begin{longtable}{|>{\centering\scriptsize}m{2em}|>{\centering\scriptsize}m{1.3em}|>{\centering}m{8.8em}|}
  % \caption{秦王政}\
  \toprule
  \SimHei \normalsize 年数 & \SimHei \scriptsize 公元 & \SimHei 大事件 \tabularnewline
  % \midrule
  \endfirsthead
  \toprule
  \SimHei \normalsize 年数 & \SimHei \scriptsize 公元 & \SimHei 大事件 \tabularnewline
  \midrule
  \endhead
  \midrule
  元年 & 84 & \tabularnewline\hline
  二年 & 85 & \tabularnewline\hline
  三年 & 86 & \tabularnewline\hline
  四年 & 87 & \tabularnewline
  \bottomrule
\end{longtable}

\subsection{章和}

\begin{longtable}{|>{\centering\scriptsize}m{2em}|>{\centering\scriptsize}m{1.3em}|>{\centering}m{8.8em}|}
  % \caption{秦王政}\
  \toprule
  \SimHei \normalsize 年数 & \SimHei \scriptsize 公元 & \SimHei 大事件 \tabularnewline
  % \midrule
  \endfirsthead
  \toprule
  \SimHei \normalsize 年数 & \SimHei \scriptsize 公元 & \SimHei 大事件 \tabularnewline
  \midrule
  \endhead
  \midrule
  元年 & 87 & \tabularnewline\hline
  二年 & 88 & \tabularnewline
  \bottomrule
\end{longtable}


%%% Local Variables:
%%% mode: latex
%%% TeX-engine: xetex
%%% TeX-master: "../Main"
%%% End:

%% -*- coding: utf-8 -*-
%% Time-stamp: <Chen Wang: 2019-12-17 17:22:02>

\section{和帝\tiny(88-105)}

\subsection{生平}

汉和帝刘肇(79年-106年2月13日),东汉第四位皇帝(88年4月9日—106年2月13日在位),在位17年,得年僅27岁,其正式諡號為「孝和皇帝」,後世省略「孝」字稱「漢和帝」,他是章帝第四子,母贵人梁氏,死後庙号穆宗,葬于慎陵。

建初四年(79年),梁贵人生刘肇。皇后窦氏将刘肇养为己子。建初七年(82年),汉章帝废太子刘庆,立刘肇为皇太子。

章和二年二月三十壬辰(88年4月9日),汉章帝逝世,刘肇即位,是为汉和帝。当时他只有十岁,由养母窦太后执政,窦太后排斥异己,让哥哥窦宪掌权,窦家人一犯法,窦太后就再三庇护,窦氏的专横跋扈,引起汉和帝的不满。永元四年壬辰年六月二十三日(92年8月14日),汉和帝联合宦官鄭眾将窦氏一网打尽,但也导致“于是中官始盛焉”。

在一举扫平了外戚窦氏集团的势力之后,汉和帝开始亲理政事,他每天早起临朝,深夜批阅奏章,从不荒怠政事,故有「劳谦有终」之称,但却因而积劳成疾,加上和帝本身已体弱多病,所以年仅二十七岁便英年早逝,从他亲政后的政绩,不失为一代贤君英主。和帝当政时期,曾多次下诏赈济灾民、减免赋税、安置流民、勿违农时,并多次下诏纳贤,在法制上也主张宽刑,并在西域复置西域都护。汉和帝十分体恤民众疾苦,多次诏令理冤狱,恤鳏寡,矜孤弱,薄赋敛,告诫上下官吏认真思考造成天灾人祸的自身原因。汉和帝亲政后使东汉国力达到极盛,时人称为「永元之隆」。

汉和帝在位时期,在科技、文化、军事、外交上也有不少建树,蔡伦改进了造纸术,班固修成《汉书》,窦宪击破北匈奴促使其西迁,班超平定西域,并派遣甘英出使大秦。元興元年乙巳年十二月廿二日辛未(106年2月13日),汉和帝病逝于京都洛阳的章德前殿,时年二十七岁。4月27日,葬於漢慎陵。

《后汉书》:“自中兴以后,逮于永元,虽颇有弛张,而俱存不扰,是以齐民岁增,辟土世广。偏师出塞,则漠北地空;都护西指,则通译四万。岂其道远三代,术长前世?将服叛去来,自有数也?”

《东观汉记》:“孝和皇帝,章帝中子也,上自歧嶷,至於总角,孝顺聪明,宽和仁孝,帝由是深珍之,以为宜承天位,年四岁,立为太子,初治尚书,遂兼览书传,好古乐道,无所不照,上以五经义异,书传意殊,亲幸东观,览书林,阅篇藉,朝无宠族,惠泽沾濡,外忧庶绩,内勤经艺,自左右近臣,皆诵诗书,德教在宽,仁恕并洽,是以黎元宁康,万国协和,符瑞八十馀品,帝让而不宣,故靡得而纪。”

《帝王世纪》:“孝和之嗣世,正身履道,以奉大业,宾礼耆艾,动式旧典,宫无嫔嫱郑卫之燕,囿无般乐游畋之豫,躬履至德,虚静自损,是以屡获丰年,远近承风。”

後汉苏顺和帝诔曰:“天王徂登,率土奄伤,如何昊穹,夺我圣皇,恩德累代,乃作铭章,其辞曰:恭惟大行,配天建德,陶元二化,风流万国,立我蒸民,宜此仪则,厥初生民,三五作刚,载藉之盛,著於虞唐,恭惟大行,爰同其光,自昔何为,钦明允塞,恭惟大行,天覆地载,无为而治,冠斯往代,往代崎岖,诸夏擅命,爰兹发号,民乐其政,奄有万国,民臣咸祑,大孝备矣,閟宫有侐,由昔姜嫄,祖妣之室,本枝百世,神契惟一,弥留不豫,道扬末命,劳谦有终,实惟其性,衣不制新,犀玉远屏,履和而行,威棱上古,洪泽滂流,茂化沾溥,不玦少留,民斯何怙,歔欷成云,泣涕成雨,昊天不吊,丧我慈父。”

後汉崔瑗和帝诔曰:“玄景寝曜,云物见徵,冯相考妖,遂当帝躬,三载四海,遏密八音,如丧考妣,擗踊号吟,大遂既启,乃徂玄宫,永背神器,升遐皇穹,长夜冥冥,曷云其穷。”

洪迈《容斋随笔‧卷三》:“汉昭帝年十四,能察霍光之忠,知燕王上书之诈,诛桑弘羊、上官桀,后世称其明。然和帝时,窦宪兄弟专权,太后临朝,共图杀害。帝阴知其谋,而与内外臣僚莫由亲接,独知中常侍郑众不事豪党,遂与定议诛宪,时亦年十四,其刚决不下昭帝,但范史发明不出,故后世无称焉。”

《续汉书》:“论曰:孝和年十四,能折外戚骄横之权,即昭帝毙上官之类矣。朝政遂一,民安职业,勤恤本务,苑囿希幸,远夷稽服,西域开泰,郡国言符瑞八十余品,咸惧虚妄,抑而不宣云尔。”

李贤注引《序例》曰:“凡瑞应,自和帝以上,政事多美,近於有实,故书见於某处。自安帝以下,王道衰缺,容或虚饰,故书某处上言也。”

李尤《辟雍赋》曰:“卓矣煌煌,永元之隆。含弘该要,周建大中。蓄纯和之优渥兮,化盛溢而兹丰。”

《通典》:“明章之后,天下无事,务在养民。至於孝和,人户滋殖。”

叶适《习学记言序目》:“东汉至孝和八十年间,上无败政,天下乂安。”

\subsection{永元}

\begin{longtable}{|>{\centering\scriptsize}m{2em}|>{\centering\scriptsize}m{1.3em}|>{\centering}m{8.8em}|}
  % \caption{秦王政}\
  \toprule
  \SimHei \normalsize 年数 & \SimHei \scriptsize 公元 & \SimHei 大事件 \tabularnewline
  % \midrule
  \endfirsthead
  \toprule
  \SimHei \normalsize 年数 & \SimHei \scriptsize 公元 & \SimHei 大事件 \tabularnewline
  \midrule
  \endhead
  \midrule
  元年 & 89 & \tabularnewline\hline
  二年 & 90 & \tabularnewline\hline
  三年 & 91 & \tabularnewline\hline
  四年 & 92 & \tabularnewline\hline
  五年 & 93 & \tabularnewline\hline
  六年 & 94 & \tabularnewline\hline
  七年 & 95 & \tabularnewline\hline
  八年 & 96 & \tabularnewline\hline
  九年 & 97 & \tabularnewline\hline
  十年 & 98 & \tabularnewline\hline
  十一年 & 99 & \tabularnewline\hline
  十二年 & 100 & \tabularnewline\hline
  十三年 & 101 & \tabularnewline\hline
  十四年 & 102 & \tabularnewline\hline
  十五年 & 103 & \tabularnewline\hline
  十六年 & 104 & \tabularnewline\hline
  十七年 & 105 & \tabularnewline
  \bottomrule
\end{longtable}

\subsection{元兴}

\begin{longtable}{|>{\centering\scriptsize}m{2em}|>{\centering\scriptsize}m{1.3em}|>{\centering}m{8.8em}|}
  % \caption{秦王政}\
  \toprule
  \SimHei \normalsize 年数 & \SimHei \scriptsize 公元 & \SimHei 大事件 \tabularnewline
  % \midrule
  \endfirsthead
  \toprule
  \SimHei \normalsize 年数 & \SimHei \scriptsize 公元 & \SimHei 大事件 \tabularnewline
  \midrule
  \endhead
  \midrule
  元年 & 105 & \tabularnewline
  \bottomrule
\end{longtable}


%%% Local Variables:
%%% mode: latex
%%% TeX-engine: xetex
%%% TeX-master: "../Main"
%%% End:

%% -*- coding: utf-8 -*-
%% Time-stamp: <Chen Wang: 2019-12-17 17:23:12>

\section{殇帝\tiny(106)}

\subsection{生平}

汉殇帝刘隆(105年10月或11月-106年9月21日),汉和帝幼子,养于民间,东汉第五位皇帝(106年在位),其正式諡號為「孝殤皇帝」,後世省略「孝」字稱「漢殤帝」。汉殇帝是即位年龄最小、寿命最短的中国皇帝。

和帝在世的时候,生了许多皇子,大都夭折。和帝以为宦官、外戚在谋害他的儿子,便将剩余的皇子留在民间扶养。元興元年乙巳年十二月廿二日辛未(106年2月13日),汉和帝死,邓皇后因长子刘胜有絕症,将刘隆迎回皇宫做皇帝,將刘胜封为平原王。刘隆登基时候才出生100余天,改元“延平”。朝政由外戚邓騭掌权。

仍在襁褓之中的汉殇帝,于延平元年八月辛亥(西元106年9月21日)得了场大病后驾崩,在位只有短短八個月。

刘隆年幼,邓太后以女主临政,期间政事多委以宦官。自汉明帝至延平年间,宦官的人数逐渐增多,中常侍达到十名,小黄门有二十名。

\subsection{延平}

\begin{longtable}{|>{\centering\scriptsize}m{2em}|>{\centering\scriptsize}m{1.3em}|>{\centering}m{8.8em}|}
  % \caption{秦王政}\
  \toprule
  \SimHei \normalsize 年数 & \SimHei \scriptsize 公元 & \SimHei 大事件 \tabularnewline
  % \midrule
  \endfirsthead
  \toprule
  \SimHei \normalsize 年数 & \SimHei \scriptsize 公元 & \SimHei 大事件 \tabularnewline
  \midrule
  \endhead
  \midrule
  元年 & 106 & \tabularnewline
  \bottomrule
\end{longtable}


%%% Local Variables:
%%% mode: latex
%%% TeX-engine: xetex
%%% TeX-master: "../Main"
%%% End:

%% -*- coding: utf-8 -*-
%% Time-stamp: <Chen Wang: 2021-11-01 11:28:50>

\section{安帝刘祜\tiny(106-125)}

\subsection{生平}

汉安帝刘祜(94年-125年4月30日),东汉第六位皇帝(106年9月21日-125年4月30日在位),在位19年,其正式諡號為「孝安皇帝」,後世省略「孝」字稱「漢安帝」。

他是汉章帝的孙子、当年被废太子清河王刘庆的儿子,母左小娥。

延平元年八月辛亥(106年9月21日),是汉殇帝崩,他被外戚邓氏拥立为帝,承嗣汉和帝刘肇,改元永初。

汉安帝即位后,仍由邓太后聽政。外戚邓氏吸取窦氏灭亡的教训,联合宦官,袒护族人。永宁元年(120年),立李氏之子刘保为皇太子。

永宁二年(121年),邓太后去世,安帝才亲政。當初漢安帝號稱聰明,鄧太后才立他為帝,後來對漢安帝不滿意,漢安帝乳母王聖得知內情,又看到鄧太后遲遲不歸政,擔心鄧太后會廢除漢安帝。鄧太后去世后,向漢安帝告發鄧太后兄弟邓悝等曾經想立平原王劉勝。安帝大怒,下令灭了邓氏一族。

安帝虽灭邓氏,但未制止外戚干政的局面。再加上安帝不理朝政,沉湎于酒色,昏庸不堪,且在掖庭挑選了一位美女,封為貴人,非常寵愛,未滿一年,便立即封她為皇后,而這個皇后即為閻姬,导致当时东汉朝政腐败,社会黑暗,奸佞当道,社会矛盾日益尖锐,边患也十分严重。全国多地震,水旱蝗灾频繁不断,外有西羌等入侵边境,内有杜琦等领导的长达十多年民變,社会危机日益加深。东汉王朝衰落。

延光三年(124年),安帝乳母王圣与樊豐、江京共同构陷太子,太子刘保被废为济阴王。

延光四年三月庚申(125年4月23日),漢安帝在外巡遊途中在宛出現身體不適。三月丁卯(125年4月30日),汉安帝在葉死在乘舆上,享年32岁。當時閻姬、閻顯等人隨同漢安帝出遊,而前太子在洛陽,閻姬等害怕大臣立前太子為帝,於是詐稱安帝重病,秘不發喪,一路星夜兼程。庚午(5月3日),巡遊車隊回到皇宮。辛未(5月4日)晚上安帝駕崩消息才被公佈。四月己酉(6月11日),安帝葬于恭陵。谥号孝安皇帝,庙号恭宗(後於漢獻帝初平元年因其無功德故除去廟號)。后阎皇后迎立刘寿子刘懿为帝。刘懿死后,漢安帝独子刘保(漢順帝)才在宦官的拥戴下登基。

\subsection{永初}

\begin{longtable}{|>{\centering\scriptsize}m{2em}|>{\centering\scriptsize}m{1.3em}|>{\centering}m{8.8em}|}
  % \caption{秦王政}\
  \toprule
  \SimHei \normalsize 年数 & \SimHei \scriptsize 公元 & \SimHei 大事件 \tabularnewline
  % \midrule
  \endfirsthead
  \toprule
  \SimHei \normalsize 年数 & \SimHei \scriptsize 公元 & \SimHei 大事件 \tabularnewline
  \midrule
  \endhead
  \midrule
  元年 & 107 & \tabularnewline\hline
  二年 & 108 & \tabularnewline\hline
  三年 & 109 & \tabularnewline\hline
  四年 & 110 & \tabularnewline\hline
  五年 & 111 & \tabularnewline\hline
  六年 & 112 & \tabularnewline\hline
  七年 & 113 & \tabularnewline
  \bottomrule
\end{longtable}

\subsection{元初}

\begin{longtable}{|>{\centering\scriptsize}m{2em}|>{\centering\scriptsize}m{1.3em}|>{\centering}m{8.8em}|}
  % \caption{秦王政}\
  \toprule
  \SimHei \normalsize 年数 & \SimHei \scriptsize 公元 & \SimHei 大事件 \tabularnewline
  % \midrule
  \endfirsthead
  \toprule
  \SimHei \normalsize 年数 & \SimHei \scriptsize 公元 & \SimHei 大事件 \tabularnewline
  \midrule
  \endhead
  \midrule
  元年 & 114 & \tabularnewline\hline
  二年 & 115 & \tabularnewline\hline
  三年 & 116 & \tabularnewline\hline
  四年 & 117 & \tabularnewline\hline
  五年 & 118 & \tabularnewline\hline
  六年 & 119 & \tabularnewline\hline
  七年 & 120 & \tabularnewline
  \bottomrule
\end{longtable}

\subsection{永宁}

\begin{longtable}{|>{\centering\scriptsize}m{2em}|>{\centering\scriptsize}m{1.3em}|>{\centering}m{8.8em}|}
  % \caption{秦王政}\
  \toprule
  \SimHei \normalsize 年数 & \SimHei \scriptsize 公元 & \SimHei 大事件 \tabularnewline
  % \midrule
  \endfirsthead
  \toprule
  \SimHei \normalsize 年数 & \SimHei \scriptsize 公元 & \SimHei 大事件 \tabularnewline
  \midrule
  \endhead
  \midrule
  元年 & 120 & \tabularnewline\hline
  二年 & 121 & \tabularnewline
  \bottomrule
\end{longtable}

\subsection{建光}

\begin{longtable}{|>{\centering\scriptsize}m{2em}|>{\centering\scriptsize}m{1.3em}|>{\centering}m{8.8em}|}
  % \caption{秦王政}\
  \toprule
  \SimHei \normalsize 年数 & \SimHei \scriptsize 公元 & \SimHei 大事件 \tabularnewline
  % \midrule
  \endfirsthead
  \toprule
  \SimHei \normalsize 年数 & \SimHei \scriptsize 公元 & \SimHei 大事件 \tabularnewline
  \midrule
  \endhead
  \midrule
  元年 & 121 & \tabularnewline\hline
  二年 & 122 & \tabularnewline
  \bottomrule
\end{longtable}

\subsection{延光}

\begin{longtable}{|>{\centering\scriptsize}m{2em}|>{\centering\scriptsize}m{1.3em}|>{\centering}m{8.8em}|}
  % \caption{秦王政}\
  \toprule
  \SimHei \normalsize 年数 & \SimHei \scriptsize 公元 & \SimHei 大事件 \tabularnewline
  % \midrule
  \endfirsthead
  \toprule
  \SimHei \normalsize 年数 & \SimHei \scriptsize 公元 & \SimHei 大事件 \tabularnewline
  \midrule
  \endhead
  \midrule
  元年 & 122 & \tabularnewline\hline
  二年 & 123 & \tabularnewline\hline
  三年 & 124 & \tabularnewline\hline
  四年 & 125 & \tabularnewline
  \bottomrule
\end{longtable}


%%% Local Variables:
%%% mode: latex
%%% TeX-engine: xetex
%%% TeX-master: "../Main"
%%% End:

%% -*- coding: utf-8 -*-
%% Time-stamp: <Chen Wang: 2019-12-17 21:05:58>

\section{顺帝\tiny(125-144)}

\subsection{前少帝生平}

刘懿(?-125年12月10日),一名犊,东汉第七位皇帝(125年5月18日-12月10日在位)。济北惠王刘寿的兒子,即位前為北鄉侯,东汉前少帝,汉朝官方没有把他算作汉朝皇帝之一。

汉安帝病危期间,征济北、河间王子年十四以下、七岁以上前往洛阳。汉安帝去世后,阎皇后为了把持国政,在阎显支持下,迎立北乡侯刘懿为帝,承嗣汉安帝(虽说两者是堂兄弟关系) 。少帝在位时,阎显兄弟把持朝政,作威作福。但少帝即位數月后就因病去世,之后宦官孙程等人合谋诛杀阎显兄弟和江京,并迎立济阴王刘保为帝,是为汉顺帝。

刘懿去世后以诸侯王规格下葬。永和元年(136年),災異頻繁,漢順帝感到恐懼,認為是北鄉侯當過皇帝卻以諸侯王規格下葬導致的報應。漢順帝打算追謚北鄉侯,納入漢朝皇帝體系。周舉不讚成,認為北鄉侯是奸臣閻顯等所立,並非正統,且在位一年不到就去世,年號未改,加上北鄉侯沒有其他功德,用諸侯王規格下葬已經很好了,不值得給他加上謚號和追認為皇帝。漢順帝聽從。

\subsection{顺帝生平}

汉顺帝刘保(115年-144年9月20日),东汉第八位皇帝(125年12月16日—144年9月20日在位),其正式諡號為「孝順皇帝」,後世省略「孝」字稱「漢順帝」。汉安帝和宫人李氏之子。

刘保出生后,生母李氏就被皇后阎姬毒杀。

劉保從小學習孝經章句,很得鄧太后欣賞,認為他可以繼承大統。永宁元年(120年),身为汉安帝独子的刘保被立为皇太子。

延光三年(124年),刘保生病,来到汉安帝乳母王圣家居住。当时王圣宅邸刚完成不久,刘保乳母王男、厨监邴吉认为不祥,反对太子刘保前去居住,于是与王圣等人爆发激烈争吵。王圣等人大怒,于是联合大长秋江京、中常侍樊丰等诬陷太子劉保的乳母王男、厨监邴吉。两人被杀,太子数为叹息。王圣等人惧有后祸,遂与樊丰、江京、汉安帝皇后阎姬共同构陷太子劉保。汉安帝召集大臣议论,太常桓焉、太仆来历、廷尉张皓等反对,汉安帝派人威胁反对废太子的大臣,最后只有来历坚决阻止汉安帝废太子。汉安帝大怒,下令罢免来历的官位,并立即废太子劉保为济阴王。来历不服,纠集11位官员和百姓上书喊冤,汉安帝不为所动。

汉安帝死后,阎皇后无子,便找个幼儿刘懿为皇帝,自己垂帘听政,掌握朝政大权。漢安帝喪葬期間,阎皇后等不讓劉保上殿靠近棺材,劉保悲傷吐血,餐粥不食。刘懿做了7个月的皇帝就死了,阎显等認為先前不立劉保,現在如果立他為帝,劉保會怨恨我們。於是稟告閻太后,繼續讓諸侯王子來京師挑選繼承人。宦官王康、孙程等19人看不下去,便发动宫廷政变,赶走阎太后,将时年11岁的刘保拥立为帝,改元“永建”,那19位拥立刘保的宦官也全部封侯。同時閻太后黨羽也被罷黜。阎太后被幽禁离宫,但顺帝拒绝了陈禅等人以无母子之情为由废太后的提议,仍尊奉阎太后直至其去世。

汉顺帝雖本为太子,但他的皇位是靠宦官得来的,所以将大权交给宦官。順帝本人則溫和但是軟弱,無法阻止宦官与外戚专政的局面。

后来宦官与外戚梁氏勾結,开始了长达20多年的梁冀专权。宦官、外戚互相勾结,弄权专横,東漢政治更加腐败,阶级矛盾日益尖锐,百姓怨声载道。

建康元年(144年)9月20日,汉顺帝死,享年30岁,在位19年。10月26日,葬於漢憲陵。汉顺帝安葬当年,憲陵就被盗贼盗掘。

顺帝死后谥号孝顺皇帝,庙号敬宗,後於漢獻帝初平元年因其無功德故除去廟號。

\subsection{永建}

\begin{longtable}{|>{\centering\scriptsize}m{2em}|>{\centering\scriptsize}m{1.3em}|>{\centering}m{8.8em}|}
  % \caption{秦王政}\
  \toprule
  \SimHei \normalsize 年数 & \SimHei \scriptsize 公元 & \SimHei 大事件 \tabularnewline
  % \midrule
  \endfirsthead
  \toprule
  \SimHei \normalsize 年数 & \SimHei \scriptsize 公元 & \SimHei 大事件 \tabularnewline
  \midrule
  \endhead
  \midrule
  元年 & 126 & \tabularnewline\hline
  二年 & 127 & \tabularnewline\hline
  三年 & 128 & \tabularnewline\hline
  四年 & 129 & \tabularnewline\hline
  五年 & 130 & \tabularnewline\hline
  六年 & 131 & \tabularnewline\hline
  七年 & 132 & \tabularnewline
  \bottomrule
\end{longtable}

\subsection{阳嘉}

\begin{longtable}{|>{\centering\scriptsize}m{2em}|>{\centering\scriptsize}m{1.3em}|>{\centering}m{8.8em}|}
  % \caption{秦王政}\
  \toprule
  \SimHei \normalsize 年数 & \SimHei \scriptsize 公元 & \SimHei 大事件 \tabularnewline
  % \midrule
  \endfirsthead
  \toprule
  \SimHei \normalsize 年数 & \SimHei \scriptsize 公元 & \SimHei 大事件 \tabularnewline
  \midrule
  \endhead
  \midrule
  元年 & 132 & \tabularnewline\hline
  二年 & 133 & \tabularnewline\hline
  三年 & 134 & \tabularnewline\hline
  四年 & 135 & \tabularnewline
  \bottomrule
\end{longtable}

\subsection{永和}

\begin{longtable}{|>{\centering\scriptsize}m{2em}|>{\centering\scriptsize}m{1.3em}|>{\centering}m{8.8em}|}
  % \caption{秦王政}\
  \toprule
  \SimHei \normalsize 年数 & \SimHei \scriptsize 公元 & \SimHei 大事件 \tabularnewline
  % \midrule
  \endfirsthead
  \toprule
  \SimHei \normalsize 年数 & \SimHei \scriptsize 公元 & \SimHei 大事件 \tabularnewline
  \midrule
  \endhead
  \midrule
  元年 & 136 & \tabularnewline\hline
  二年 & 137 & \tabularnewline\hline
  三年 & 138 & \tabularnewline\hline
  四年 & 139 & \tabularnewline\hline
  五年 & 140 & \tabularnewline\hline
  六年 & 141 & \tabularnewline
  \bottomrule
\end{longtable}

\subsection{汉安}

\begin{longtable}{|>{\centering\scriptsize}m{2em}|>{\centering\scriptsize}m{1.3em}|>{\centering}m{8.8em}|}
  % \caption{秦王政}\
  \toprule
  \SimHei \normalsize 年数 & \SimHei \scriptsize 公元 & \SimHei 大事件 \tabularnewline
  % \midrule
  \endfirsthead
  \toprule
  \SimHei \normalsize 年数 & \SimHei \scriptsize 公元 & \SimHei 大事件 \tabularnewline
  \midrule
  \endhead
  \midrule
  元年 & 142 & \tabularnewline\hline
  二年 & 143 & \tabularnewline\hline
  三年 & 144 & \tabularnewline
  \bottomrule
\end{longtable}

\subsection{建康}

\begin{longtable}{|>{\centering\scriptsize}m{2em}|>{\centering\scriptsize}m{1.3em}|>{\centering}m{8.8em}|}
  % \caption{秦王政}\
  \toprule
  \SimHei \normalsize 年数 & \SimHei \scriptsize 公元 & \SimHei 大事件 \tabularnewline
  % \midrule
  \endfirsthead
  \toprule
  \SimHei \normalsize 年数 & \SimHei \scriptsize 公元 & \SimHei 大事件 \tabularnewline
  \midrule
  \endhead
  \midrule
  元年 & 144 & \tabularnewline
  \bottomrule
\end{longtable}


%%% Local Variables:
%%% mode: latex
%%% TeX-engine: xetex
%%% TeX-master: "../Main"
%%% End:

%% -*- coding: utf-8 -*-
%% Time-stamp: <Chen Wang: 2021-11-01 11:29:46>

\section{冲帝刘炳\tiny(144-145)}

\subsection{生平}

汉冲帝刘炳(漢安二年至永憙元年正月初六戊戌日,即公元143年-145年2月15日),中国汉朝皇帝(建康元年八月初六庚午日至永憙元年正月初六戊戌日,即西元144年9月20日-145年2月15日在位),东汉第九位皇帝,在位僅148天,享年僅3岁,其正式諡號為「孝沖皇帝」,後世省略「孝」字稱「漢沖帝」。

刘炳是汉顺帝的独子,他的母亲是虞美人,建康元年(144年)被立为太子。同年顺帝死后,2岁的刘炳于八月庚午登基,是为汉冲帝,改元“永憙”。

顺帝的皇后梁氏被尊为皇太后,临朝问政。赵峻为太傅;李固为太尉。在位时期外戚梁氏当政,朝政腐败,民不聊生,九江发生暴乱,叛军在145年初一直攻到合肥。

145年正月戊戌(2月15日),冲帝病死在玉堂前殿。梁太后降旨立年僅八歲勃海王劉鴻的獨子劉纘入承大統,是為漢質帝,藉此再度以皇太后身份臨朝稱制。正月己未(3月8日),葬於漢懷陵。

\subsection{永嘉}

\begin{longtable}{|>{\centering\scriptsize}m{2em}|>{\centering\scriptsize}m{1.3em}|>{\centering}m{8.8em}|}
  % \caption{秦王政}\
  \toprule
  \SimHei \normalsize 年数 & \SimHei \scriptsize 公元 & \SimHei 大事件 \tabularnewline
  % \midrule
  \endfirsthead
  \toprule
  \SimHei \normalsize 年数 & \SimHei \scriptsize 公元 & \SimHei 大事件 \tabularnewline
  \midrule
  \endhead
  \midrule
  元年 & 145 & \tabularnewline
  \bottomrule
\end{longtable}

%%% Local Variables:
%%% mode: latex
%%% TeX-engine: xetex
%%% TeX-master: "../Main"
%%% End:

%% -*- coding: utf-8 -*-
%% Time-stamp: <Chen Wang: 2021-11-01 11:29:52>

\section{质帝刘缵\tiny(145-146)}

\subsection{生平}

汉质帝刘缵(138年-146年7月26日),一名续,东汉第十位皇帝。145年3月6日即位,在位时间1年余,其正式諡號為「孝質皇帝」,後世省略「孝」字稱「漢質帝」。

前任皇帝汉冲帝駕崩时只有3岁,當時尊爲梁太后(漢順帝皇后)之弟梁冀拥立汉章帝玄孙刘缵为帝,承汉顺帝嗣,改元本初,是为汉质帝。

當時梁冀一家专权,朝政腐败,吏治不修。梁冀當時權勢極盛,威勢橫行朝廷和宮外;大臣們害怕梁冀的威勢,不敢抗命。质帝虽年幼,但他聪明伶俐,不堪梁冀的专横跋扈。质帝曾在朝見大臣時當面對梁冀说:「此跋扈将军也!」。

梁冀听罢,大为反感,便命手下在质帝的饼裏下毒弒君,146年7月26日,9岁的质帝食用毒餅後死亡。8月26日,葬於漢靜陵。

质帝崩后,继任的汉桓帝终于诛灭了梁氏。

\subsection{本初}

\begin{longtable}{|>{\centering\scriptsize}m{2em}|>{\centering\scriptsize}m{1.3em}|>{\centering}m{8.8em}|}
  % \caption{秦王政}\
  \toprule
  \SimHei \normalsize 年数 & \SimHei \scriptsize 公元 & \SimHei 大事件 \tabularnewline
  % \midrule
  \endfirsthead
  \toprule
  \SimHei \normalsize 年数 & \SimHei \scriptsize 公元 & \SimHei 大事件 \tabularnewline
  \midrule
  \endhead
  \midrule
  元年 & 146 & \tabularnewline
  \bottomrule
\end{longtable}

%%% Local Variables:
%%% mode: latex
%%% TeX-engine: xetex
%%% TeX-master: "../Main"
%%% End:

%% -*- coding: utf-8 -*-
%% Time-stamp: <Chen Wang: 2019-12-17 21:11:08>

\section{桓帝\tiny(147-167)}

\subsection{生平}

汉桓帝刘志(132年-168年1月25日),东汉第十一位皇帝(146年8月1日-168年1月25日在位),其正式諡號為「孝桓皇帝」,後世省略「孝」字稱「漢桓帝」,他是汉章帝曾孙,河間孝王劉開之孫,蠡吾侯劉翼之子,在位21年。

146年,外戚梁冀毒死九岁的汉质帝,立十五岁的刘志即位,承汉顺帝嗣。

刘志从小就对梁氏不满,他即位后,就想方设法的诛灭梁氏。延熹二年(159年),桓帝联合宦官单超等5人一舉殲灭了梁氏,5人同日被封侯,称之为“五侯”。不過,五侯比外戚更加腐敗,他们对百姓们勒索抢劫,民不聊生,四处怨声载道,東汉政治更加衰頹,国势益弱。汉桓帝统治后期,一批太学士看到朝政败壞,便要求朝廷整肅宦官、改革政治。宦官气極败坏,在延熹九年(166年)与德揚天下的司隸校尉李膺发生大规模冲突。桓帝大怒,下令逮捕替李膺請願的太学生200余人,后来在太傅陈蕃、将军窦武的反对下才释放太学生,但是禁锢终身,不许再做官,史称“党锢之祸”,東漢朝政更加黑暗腐敗。汉桓帝在位期间沉迷女色,荒淫无度,后宫人数竟达五六千人。

汉桓帝於168年1月25日去世,死后谥号孝桓皇帝,庙号为威宗,168年3月9日葬於宣陵。後於漢獻帝初平元年因其無功德故除去廟號。

诸葛亮:“亲贤臣,远小人,此先汉所以兴隆也;亲小人,远贤臣,此后汉所以倾颓也。先帝在时,每与臣论此事,未尝不叹息痛恨于桓、灵也。”

范晔:“前史称桓帝好音乐,善琴笙。饰芳林而考濯龙之宫,设华盖以祠浮图、老子,斯将所谓“听于神”乎!及诛梁冀,奋威怒,天下犹企其休息。而五邪嗣虐,流衍四方。自非忠贤力争,屡折奸锋,虽愿依斟流彘,亦不可得已。”

虞世南:“桓帝赫然奋怒,诛灭梁冀,有刚断之节焉。然阉人擅命,党锢事起,非乎乱阶,始於桓帝。”

周昙:“能嫌跋扈斩梁王,宁便荣枯信段张。襄楷忠言谁佞惑,忍教奸祸起萧墙。”

\subsection{建和}

\begin{longtable}{|>{\centering\scriptsize}m{2em}|>{\centering\scriptsize}m{1.3em}|>{\centering}m{8.8em}|}
  % \caption{秦王政}\
  \toprule
  \SimHei \normalsize 年数 & \SimHei \scriptsize 公元 & \SimHei 大事件 \tabularnewline
  % \midrule
  \endfirsthead
  \toprule
  \SimHei \normalsize 年数 & \SimHei \scriptsize 公元 & \SimHei 大事件 \tabularnewline
  \midrule
  \endhead
  \midrule
  元年 & 147 & \tabularnewline\hline
  二年 & 148 & \tabularnewline\hline
  三年 & 149 & \tabularnewline
  \bottomrule
\end{longtable}

\subsection{和平}

\begin{longtable}{|>{\centering\scriptsize}m{2em}|>{\centering\scriptsize}m{1.3em}|>{\centering}m{8.8em}|}
  % \caption{秦王政}\
  \toprule
  \SimHei \normalsize 年数 & \SimHei \scriptsize 公元 & \SimHei 大事件 \tabularnewline
  % \midrule
  \endfirsthead
  \toprule
  \SimHei \normalsize 年数 & \SimHei \scriptsize 公元 & \SimHei 大事件 \tabularnewline
  \midrule
  \endhead
  \midrule
  元年 & 150 & \tabularnewline
  \bottomrule
\end{longtable}

\subsection{元嘉}

\begin{longtable}{|>{\centering\scriptsize}m{2em}|>{\centering\scriptsize}m{1.3em}|>{\centering}m{8.8em}|}
  % \caption{秦王政}\
  \toprule
  \SimHei \normalsize 年数 & \SimHei \scriptsize 公元 & \SimHei 大事件 \tabularnewline
  % \midrule
  \endfirsthead
  \toprule
  \SimHei \normalsize 年数 & \SimHei \scriptsize 公元 & \SimHei 大事件 \tabularnewline
  \midrule
  \endhead
  \midrule
  元年 & 151 & \tabularnewline\hline
  二年 & 152 & \tabularnewline\hline
  三年 & 153 & \tabularnewline
  \bottomrule
\end{longtable}

\subsection{永兴}

\begin{longtable}{|>{\centering\scriptsize}m{2em}|>{\centering\scriptsize}m{1.3em}|>{\centering}m{8.8em}|}
  % \caption{秦王政}\
  \toprule
  \SimHei \normalsize 年数 & \SimHei \scriptsize 公元 & \SimHei 大事件 \tabularnewline
  % \midrule
  \endfirsthead
  \toprule
  \SimHei \normalsize 年数 & \SimHei \scriptsize 公元 & \SimHei 大事件 \tabularnewline
  \midrule
  \endhead
  \midrule
  元年 & 153 & \tabularnewline\hline
  二年 & 154 & \tabularnewline
  \bottomrule
\end{longtable}

\subsection{永寿}

\begin{longtable}{|>{\centering\scriptsize}m{2em}|>{\centering\scriptsize}m{1.3em}|>{\centering}m{8.8em}|}
  % \caption{秦王政}\
  \toprule
  \SimHei \normalsize 年数 & \SimHei \scriptsize 公元 & \SimHei 大事件 \tabularnewline
  % \midrule
  \endfirsthead
  \toprule
  \SimHei \normalsize 年数 & \SimHei \scriptsize 公元 & \SimHei 大事件 \tabularnewline
  \midrule
  \endhead
  \midrule
  元年 & 155 & \tabularnewline\hline
  二年 & 156 & \tabularnewline\hline
  三年 & 157 & \tabularnewline\hline
  四年 & 158 & \tabularnewline
  \bottomrule
\end{longtable}

\subsection{延熹}

\begin{longtable}{|>{\centering\scriptsize}m{2em}|>{\centering\scriptsize}m{1.3em}|>{\centering}m{8.8em}|}
  % \caption{秦王政}\
  \toprule
  \SimHei \normalsize 年数 & \SimHei \scriptsize 公元 & \SimHei 大事件 \tabularnewline
  % \midrule
  \endfirsthead
  \toprule
  \SimHei \normalsize 年数 & \SimHei \scriptsize 公元 & \SimHei 大事件 \tabularnewline
  \midrule
  \endhead
  \midrule
  元年 & 158 & \tabularnewline\hline
  二年 & 159 & \tabularnewline\hline
  三年 & 160 & \tabularnewline\hline
  四年 & 161 & \tabularnewline\hline
  五年 & 162 & \tabularnewline\hline
  六年 & 163 & \tabularnewline\hline
  七年 & 164 & \tabularnewline\hline
  八年 & 165 & \tabularnewline\hline
  九年 & 166 & \tabularnewline\hline
  十年 & 167 & \tabularnewline
  \bottomrule
\end{longtable}


\subsection{永康}

\begin{longtable}{|>{\centering\scriptsize}m{2em}|>{\centering\scriptsize}m{1.3em}|>{\centering}m{8.8em}|}
  % \caption{秦王政}\
  \toprule
  \SimHei \normalsize 年数 & \SimHei \scriptsize 公元 & \SimHei 大事件 \tabularnewline
  % \midrule
  \endfirsthead
  \toprule
  \SimHei \normalsize 年数 & \SimHei \scriptsize 公元 & \SimHei 大事件 \tabularnewline
  \midrule
  \endhead
  \midrule
  元年 & 167 & \tabularnewline
  \bottomrule
\end{longtable}


%%% Local Variables:
%%% mode: latex
%%% TeX-engine: xetex
%%% TeX-master: "../Main"
%%% End:

%% -*- coding: utf-8 -*-
%% Time-stamp: <Chen Wang: 2021-11-01 11:30:41>

\section{灵帝刘宏\tiny(168-189)}

\subsection{生平}

汉灵帝刘宏(157年-189年5月13日),东汉第十二位皇帝(168年2月17日—189年5月13日在位),在位22年,葬于汉文陵,其正式諡號為「孝靈皇帝」,後世省略「孝」字稱「漢灵帝」。灵帝是东汉最后一个握有实权的皇帝。自從靈帝崩後,外戚何太后、何進掌權,漢帝自此淪爲傀儡。再後董卓、李傕、曹操相繼把持朝政,東漢大權完全落入董卓、李傕、曹操手中。

刘宏本封解渎亭侯,为承袭其父刘苌的爵位。母董夫人。他是漢章帝的玄孫,漢桓帝的堂侄。

永康元年(168年1月25日)桓帝崩,刘儵以光禄大夫身份与中常侍曹节带领中黄门、虎贲、羽林军一千多人,前往河间迎接刘宏。建宁元年正月二十日(168年2月16日),刘宏来到夏门亭,窦武亲自持节用青盖车把他迎入殿内。第二天,登基称帝,改元为“建宁”。由桓帝的皇后竇妙立為皇帝,承嗣汉桓帝,是为汉灵帝。 

汉灵帝即位后,东汉政治已经病入膏肓,天下水灾、旱灾、蝗灾、瘟疫等灾祸频繁,四处怨声载道,百姓民不聊生,国势进一步衰落。再加上宦官与外戚争权夺利,最后宦官曹节、王甫等推翻外戚窦氏並軟禁竇太后,夺得了大权,又杀死正义的太学生李膺、范滂等100余人,流放、关押800多人,多惨死于狱中,造成第二次党锢之祸。灵帝一方面保留窦太后的尊号,一方面将生母董氏迎入宫中尊为太后。将军张奂、郎中谢弼、黄门令董萌都为窦太后求情,灵帝感念窦太后拥立之恩,也一度被打动,率群臣为其上寿及增加供奉,但始终没有解除其幽禁,谢弼、董萌反而被宦官报复而死。窦太后忧死后,曹节、王甫因深恨窦氏,提出追废她及改以冯贵人配享桓帝,在廷尉陈球、太尉李咸的据理力争及灵帝本人坚持下,未果。

熹平四年(175年),议郎蔡邕认为儒家经典流传过程中出现许多错误,于是联合中常侍李巡、五官中郎将堂谿典、光禄大夫杨赐、谏议大夫马日磾、议郎张驯、韩说、太史令单飏等人共同上书要求校勘儒家经典。于是汉灵帝设立熹平石经,将校勘后的儒家经典分别刻在四十六块石碑之上,并安置在太学门外,作为经典标准,供人学习。

熹平六年(177年),鑒於鮮卑多次侵擾漢朝邊境。夏育建議討伐鮮卑,在朝廷多次商討后,派夏育、田晏、臧旻三路大軍討伐鮮卑。結果大敗而歸,夏育、田晏、臧旻被廢為庶人。

光和元年(178年),靈帝建立鴻都門學,最初號稱以研究儒术经义为名,后招集众多文士从事辞赋及书法等文艺创作活动。因鴻都門學专重文艺而轻儒家經典,引起不少大臣反對。

光和四年(181年),靈帝在皇宫之中扩建西园,修建集市供自己享乐。靈帝和宮女模仿民间市集里的商人、窃贼、地痞,并驾着白驴在西园中来回穿梭。汉灵帝同時长期沉迷于女色,灵帝特别喜欢一些冰肌玉洁的少女,还为此修建水池园林,是为裸游馆,和一群美女嬉戏于其中,并命令宫女只能穿开档裤,原因竟是为了方便自己临幸宫女。

昏庸荒淫的灵帝除了沉湎酒色以外,还一味宠信宦官,尊张让等人为“十常侍”,并说“张常侍乃我父、赵常侍乃我母”,宦官杖着皇帝的宠幸,胡作非为,对百姓勒索钱财,大肆搜刮民脂民膏,可谓腐败到极点。靈帝還多次賣官,先后有段颎、张温、崔烈、樊陵、曹嵩等人花钱买到三公之位。

在朝政腐败和天灾的双重压迫之下,叛乱有了广大的市场,巨鹿(今河北省平乡县)人张角煽动百姓,聚众造反。光和七年(184年)张角兄弟三人以“苍天已死、黄天当立、岁在甲子、天下大吉”为口号举事,史称“黃巾之亂”,这次暴乱所向披靡,给病入膏肓的东汉王朝以沉重打击。同時涼州爆發北宮伯玉之亂,國家一片衰敗。但靈帝不思悔改,繼續大幅增修宮殿,為此靈帝不惜增加民眾賦稅。

中平五年(188年),張純、張舉等人勾結烏桓叛亂,而冀州刺史王芬看見局勢混亂,圖謀廢除靈帝,但最終失敗。

鑒於漢室朝綱廢弛民變頻繁,靈帝以宦官蹇硕為統帥組建西園軍,自號無上將軍,令西園軍一度權勢高於大將軍何進。

公元189年5月13日,汉灵帝去世,终年32岁。7月17日,葬於漢文陵。漢靈帝死後引發漢朝最後一次戚宦相爭之宮變。

漢靈帝荒淫昏庸,曾于西園起裸游館千間,灵帝特别喜欢娇嫩纯洁的幼女,選十四歲以上十八歲以下的宮女于池中裸游,又曾于西園弄狗與人獸交。其人貪財,公開賣官鬻爵,致使朝政更加黑暗。但早年又有辞赋、書法和音樂爱好。

范晔《后汉书·孝灵帝纪》:“《秦本纪》说赵高谲二世,指鹿为马,而赵忠、张让亦绐灵帝不得登高临观,故知亡敝者同其致矣。然则灵帝之为灵也优哉!”、“灵帝负乘,委体宦孽。征亡备兆,《小雅》尽缺。麋鹿霜露,遂栖宫卫。”

董卓:“天下之主,宜得贤明,每念灵帝,令人愤毒!”《后汉书·卷七十四上·袁绍刘表列传第六十四上》

盖勋:“吾仍见上,上甚聪明,但拥蔽于左右耳。”《后汉书·虞傅盖臧列传第四十八》

张超《靈帝河閒舊廬碑》:赫赫在上.陶唐是承.繼德二祖.四宗是憑.上納鑒乎羲農.中結軌乎夏商.元首既明.股肱惟良.乃因舊宇.福德所基.修飾經構.農隙得時.樹中天之雙闕.崇冠山之華堂.通樓閑道.丹階紫房.金窗鬱律.玉璧內璫.青蒲充庖.朱草栖箱.川魚踊躍.雲鳥舞翔.煌煌大漢.含德乾綱.體效日月.驗化陰陽.格于上下.震暢八荒.三光宣曜.四靈效祥.天其嘉享.豐年穰穰.騶虞奏樂.鹿鳴薦觴.二祝致告.福祿來將.永保萬國.南山無量.(《艺文类聚 卷六十四》)

汉灵帝与其前任皇帝汉桓帝的统治时期是东汉最黑暗的时期,诸葛亮的《出师表》中就有蜀汉开国皇帝刘备每次“叹息痛恨于桓灵”的陈述:“亲贤臣,远小人,此先汉所以兴隆也;亲小人,远贤臣,此后汉所以倾颓也。先帝在时,每与臣论此事,未尝不叹息痛恨于桓、灵也。”

薛莹:“汉氏中兴,至于延平而世业损矣。冲质短祚,孝桓无嗣,母后称制,奸臣执政。孝灵以支庶而登至尊,由蕃侯而绍皇统,不恤宗绪,不祗天命;上亏三光之明,下伤亿兆之望。于时爵服横流,官以贿成。自公侯卿士降于皂隶,迁官袭级无不以货,刑戮无辜,摧扑忠良;佞谀在侧,直言不闻。是以贤智退而穷处,忠良摈于下位;遂至奸雄蜂起,当防隳坏,夷狄并侵,盗贼糜沸。小者带城邑,大者连州郡。编户骚动,人人思乱。当此之时,已无天子矣。会灵帝即世,盗贼相寻,其後宫室。焚灭,郊社无主,危自上起,覃及华夏。使京室为墟,海内萧条,岂不痛哉!”(《全晋文·卷八十一》)

王嘉《拾遗记》:“安、灵二帝,同为败德。夫悦目快心,罕不沦乎情欲,自非远鉴兴亡,孰能移隔下俗。佣才缘心,缅乎嗜欲,塞谏任邪,没情于淫靡。至如列代亡主,莫不凭威猛以丧家国,肆奢丽以覆宗祀。询考先坟,往往而载,佥求历古,所记非一。贩爵鬻官,乖分职之本;露宿郊居,违省方之义。”

虞世南:“灵帝承疲民之后,易为善政,黎庶倾耳。咸冀中兴,而帝袭彼覆车,毒逾前辈,倾覆宗社,职帝之由。天年厌世,为幸多矣。”(《唐文拾遗·卷十三》)

杜牧:“桓、灵四十年间杀千百比干,毒流其社稷,可以血食乎?可以坛?单父天拜郊乎?”(《樊川文集》)

周昙:“榜悬金价鬻官荣,千万为公五百卿。公瑾孔明穷退者,安知高卧遇雄英。”(《全唐诗·卷七百二十九》)

胡三省:“观灵帝以尚但之言不敢复升台榭,诚恐百姓虚散也,谓无爱民之心可乎!使其以信尚但者信诸君子之言,则汉之为汉,未可知也。”(《资治通鉴·卷第五十八·汉纪五十》)

蔡东藩《后汉演义》:“汉季之中常侍,谁不曰可杀?惟庸主如桓灵,方信而用之。”「国家赏罚有明经,宵小谗言怎可听?功罪不分昏愦甚,从知灵帝本无灵!」“若平乐观中之讲武,设坛张盖,夸示威风,灵帝自以为耀武,而盖勋乃以黩武为对,犹非知本之谈。黩武二字,惟汉武足以当之,灵帝岂足语此?彼之所信任者,妇寺而已,如皇甫嵩、朱儁诸才,皆不知重用;甚至一病不起,犹视赛硕为忠贞,托孤寄命,《范史》谓灵帝负扆,委体宦孽,征亡备兆,小雅尽缺,其亦所谓月旦之定评也乎?”


\subsection{建宁}

\begin{longtable}{|>{\centering\scriptsize}m{2em}|>{\centering\scriptsize}m{1.3em}|>{\centering}m{8.8em}|}
  % \caption{秦王政}\
  \toprule
  \SimHei \normalsize 年数 & \SimHei \scriptsize 公元 & \SimHei 大事件 \tabularnewline
  % \midrule
  \endfirsthead
  \toprule
  \SimHei \normalsize 年数 & \SimHei \scriptsize 公元 & \SimHei 大事件 \tabularnewline
  \midrule
  \endhead
  \midrule
  元年 & 168 & \tabularnewline\hline
  二年 & 169 & \tabularnewline\hline
  三年 & 170 & \tabularnewline\hline
  四年 & 171 & \tabularnewline\hline
  五年 & 172 & \tabularnewline
  \bottomrule
\end{longtable}

\subsection{熹平}

\begin{longtable}{|>{\centering\scriptsize}m{2em}|>{\centering\scriptsize}m{1.3em}|>{\centering}m{8.8em}|}
  % \caption{秦王政}\
  \toprule
  \SimHei \normalsize 年数 & \SimHei \scriptsize 公元 & \SimHei 大事件 \tabularnewline
  % \midrule
  \endfirsthead
  \toprule
  \SimHei \normalsize 年数 & \SimHei \scriptsize 公元 & \SimHei 大事件 \tabularnewline
  \midrule
  \endhead
  \midrule
  元年 & 172 & \tabularnewline\hline
  二年 & 173 & \tabularnewline\hline
  三年 & 174 & \tabularnewline\hline
  四年 & 175 & \tabularnewline\hline
  五年 & 176 & \tabularnewline\hline
  六年 & 177 & \tabularnewline\hline
  七年 & 178 & \tabularnewline
  \bottomrule
\end{longtable}

\subsection{光和}

\begin{longtable}{|>{\centering\scriptsize}m{2em}|>{\centering\scriptsize}m{1.3em}|>{\centering}m{8.8em}|}
  % \caption{秦王政}\
  \toprule
  \SimHei \normalsize 年数 & \SimHei \scriptsize 公元 & \SimHei 大事件 \tabularnewline
  % \midrule
  \endfirsthead
  \toprule
  \SimHei \normalsize 年数 & \SimHei \scriptsize 公元 & \SimHei 大事件 \tabularnewline
  \midrule
  \endhead
  \midrule
  元年 & 178 & \tabularnewline\hline
  二年 & 179 & \tabularnewline\hline
  三年 & 180 & \tabularnewline\hline
  四年 & 181 & \tabularnewline\hline
  五年 & 182 & \tabularnewline\hline
  六年 & 183 & \tabularnewline\hline
  七年 & 184 & \tabularnewline
  \bottomrule
\end{longtable}

\subsection{中平}

\begin{longtable}{|>{\centering\scriptsize}m{2em}|>{\centering\scriptsize}m{1.3em}|>{\centering}m{8.8em}|}
  % \caption{秦王政}\
  \toprule
  \SimHei \normalsize 年数 & \SimHei \scriptsize 公元 & \SimHei 大事件 \tabularnewline
  % \midrule
  \endfirsthead
  \toprule
  \SimHei \normalsize 年数 & \SimHei \scriptsize 公元 & \SimHei 大事件 \tabularnewline
  \midrule
  \endhead
  \midrule
  元年 & 184 & \tabularnewline\hline
  二年 & 185 & \tabularnewline\hline
  三年 & 186 & \tabularnewline\hline
  四年 & 187 & \tabularnewline\hline
  五年 & 188 & \tabularnewline\hline
  六年 & 189 & \tabularnewline
  \bottomrule
\end{longtable}


%%% Local Variables:
%%% mode: latex
%%% TeX-engine: xetex
%%% TeX-master: "../Main"
%%% End:

%% -*- coding: utf-8 -*-
%% Time-stamp: <Chen Wang: 2018-07-10 20:22:57>

\section{刘辩\tiny(189)}

\subsection{光熹}

\begin{longtable}{|>{\centering\scriptsize}m{2em}|>{\centering\scriptsize}m{1.3em}|>{\centering}m{8.8em}|}
  % \caption{秦王政}\
  \toprule
  \SimHei \normalsize 年数 & \SimHei \scriptsize 公元 & \SimHei 大事件 \tabularnewline
  % \midrule
  \endfirsthead
  \toprule
  \SimHei \normalsize 年数 & \SimHei \scriptsize 公元 & \SimHei 大事件 \tabularnewline
  \midrule
  \endhead
  \midrule
  元年 & 189 & \tabularnewline
  \bottomrule
\end{longtable}

\subsection{昭宁}

\begin{longtable}{|>{\centering\scriptsize}m{2em}|>{\centering\scriptsize}m{1.3em}|>{\centering}m{8.8em}|}
  % \caption{秦王政}\
  \toprule
  \SimHei \normalsize 年数 & \SimHei \scriptsize 公元 & \SimHei 大事件 \tabularnewline
  % \midrule
  \endfirsthead
  \toprule
  \SimHei \normalsize 年数 & \SimHei \scriptsize 公元 & \SimHei 大事件 \tabularnewline
  \midrule
  \endhead
  \midrule
  元年 & 189 & \tabularnewline
  \bottomrule
\end{longtable}


%%% Local Variables:
%%% mode: latex
%%% TeX-engine: xetex
%%% TeX-master: "../Main"
%%% End:

%% -*- coding: utf-8 -*-
%% Time-stamp: <Chen Wang: 2021-11-01 11:31:35>

\section{献帝刘协\tiny(189-220)}

\subsection{生平}

刘协(181年4月2日-234年4月21日),字伯和,是东汉政權最后一位皇帝,189年至220年在位,曹魏給其諡號為「孝獻皇帝」,後世省略「孝」字稱「漢獻帝」。蜀漢給其諡號為「孝愍皇帝」。

漢獻帝刘协是汉灵帝刘宏的儿子,汉少帝刘辩的庶弟,母親是美人王榮(五官中郎將王苞的孫女)。因汉灵帝何皇后性强忌威慑后宫,王荣刚怀孕时害怕,服药欲堕胎,而胎安不动,又数次梦到肩负著太阳行走,是为吉兆。刘协出生后,母親王荣即被何皇后毒杀,他由祖母董太后撫養成人。中平六年(189年)四月,汉灵帝崩,刘协的异母兄刘辩即位,是为汉少帝,封刘协勃海王,又改封陈留王。

袁绍等人诛杀宦官时,他随刘辩被宦官張讓和段珪绑架,遇到董卓。董卓曾和少帝谈话,少帝语无伦次,再和刘协谈话,刘协则将事情经过完整交代。董卓以为刘协贤能,且为董太后所养,又自以为与董太后同族,再加上要显示自己的权力,遂有废立之意。

中平六年(189年),董卓為了立威,废少帝,於九月甲戌日(9月28日)立当时九岁的刘协为皇帝。關東諸侯起兵討伐董卓時,董卓杀少帝、何太后,火燒都城雒陽,挾劉協遷都長安。初平三年(192年),司徒王允成功诛杀董卓,但不久董卓的部下李傕、郭汜即攻陷长安,杀死大批大臣,再次挟持献帝。後來李郭二人內訌戰鬥,民不聊生,献帝与一批朝臣于兴平二年(195年)七月逃离长安,途中多次成为李傕、郭汜、张济、杨奉等军阀争夺挟制的目标。

汉献帝与朝臣历经一年才于建安元年(196年)七月到达旧都雒阳。不过雒阳早经董卓撤离时焚烧,宫室尽毁,百官披荆棘藏身断壁之间,更加粮草断绝。八月,朝臣曹操从雒阳挾刘协到許縣,稱許都,作为其“挟天子以令诸侯”战略的一部分,刘协成为一位毫無實權的傀儡皇帝。宮中的守衛和侍從其實都是曹操的人,對內外官員的殺戮亦很常見,例如議郎趙彥曾經向獻帝進言分析局勢和對策,因而被曹操殺害。

建安五年(200年),大臣董承連同與王子服、种輯、劉備等人谋划对付曹操,董承卻被曹操所刺。曹操欲杀董承之女董贵人,獻帝以董貴人懷有漢室的血脈拒絕,但最終董貴人仍被曹操弒殺。

建安十八年(213年)時,曹操將三個女兒曹憲、曹節和曹華送入漢獻帝後宮為夫人,年紀最小的曹華在國待年。次年,三女都被封為貴人。

建安十九年(214年),献帝的皇后伏壽之前写信给父亲伏完圖謀誅殺曹操之事洩露,曹操大怒,逼獻帝廢后,更先替獻帝寫好詔書,命郗慮持節接收皇后印綬,又命尚書令華歆領兵入宮捉拿伏壽。伏壽躲在牆壁之中但被華歆發現拉出,經過外殿時見獻帝,伏壽哭著說:「不能復相活邪?(不能再救救我嗎?)」獻帝答:「我亦不知命在何時。(我亦不知我何時會死。)」又對身旁的郗慮說:「郗公,天下寧有是邪?(郗公,天下間有這樣的事嗎?)」伏壽后被幽闭至死(《曹瞞傳》稱當場被弒),所生的兩位皇子亦以毒酒毒害,伏氏宗族有百多人亦被處死,伏壽母親盈等十九人都被流放到涿郡。

建安二十年正月(215年),因曹操和群臣的壓力下,改立貴人曹節為皇后。

建安二十五年(220年)正月,曹操去世。三月献帝改元延康。嗣魏王位的曹丕提出要求獻帝禪讓皇位,篡位自立。在曹氏政權威迫之下,十月乙卯日(11月25日)献帝將帝位在繁陽的受禪臺之上「禪讓」給曹丕,东汉結束。

曹丕篡漢後,封劉協為山陽公(屬司隸河內郡),「邑一萬戶,位在諸侯王上,奏事不稱臣,受詔不拜,以天子車服郊祀天地,宗廟、祖、臘皆如漢制,都山陽之濁鹿城」(《後漢書》本紀)。曹丕並對衛臻說「天下之珍,吾與山陽共之」(《三國志·衛臻傳》)。

曹魏青龍二年(234年)三月庚寅(4月21日),劉協駕崩,享年五十四歲。曹叡聞訊後,「素服發哀,遣使持節典護喪事……追謚山陽公曰孝獻皇帝,冊贈璽紱……車旗服章喪葬禮儀,一如漢氏故制」,並宣布大赦天下。八月壬申(9月30日),劉協被安葬于山陽國,陵曰禪陵,置園邑。諡號為孝獻皇帝。

袁术:「圣主聪叡,有周成之质。」(《三国志·卷六·魏书六·董二袁刘传第六》)

袁山松:「献帝崎岖危乱之间,飘薄万里之衢,萍流蓬转,险阻备经,自古帝王未之有也。观其天性慈爱,弱而神惠,若辅之以德,真守文令主也。曹氏始於勤王,终至陷天,遂力制群雄,负鼎而趋,然因其利器,假而不反,回山倒海,遂移天日。昔田常假汤、武而杀君,操因尧、舜而窃国,所乘不同,济其盗贼之身一也。善乎!庄生之言:窃钩者诛,窃国者为诸侯,诸侯之门仁义存焉。信矣。」

范晔:「传称鼎之为器,虽小而重,故神之所宝,不可夺移。至令负而趋者,此亦穷运之归乎!天厌汉德久矣,山阳其何诛焉!」「献生不辰,身播国屯。终我四百,永作虞宾。」


\subsection{永汉}

\begin{longtable}{|>{\centering\scriptsize}m{2em}|>{\centering\scriptsize}m{1.3em}|>{\centering}m{8.8em}|}
  % \caption{秦王政}\
  \toprule
  \SimHei \normalsize 年数 & \SimHei \scriptsize 公元 & \SimHei 大事件 \tabularnewline
  % \midrule
  \endfirsthead
  \toprule
  \SimHei \normalsize 年数 & \SimHei \scriptsize 公元 & \SimHei 大事件 \tabularnewline
  \midrule
  \endhead
  \midrule
  元年 & 189 & \tabularnewline
  \bottomrule
\end{longtable}

\subsection{中平}

\begin{longtable}{|>{\centering\scriptsize}m{2em}|>{\centering\scriptsize}m{1.3em}|>{\centering}m{8.8em}|}
  % \caption{秦王政}\
  \toprule
  \SimHei \normalsize 年数 & \SimHei \scriptsize 公元 & \SimHei 大事件 \tabularnewline
  % \midrule
  \endfirsthead
  \toprule
  \SimHei \normalsize 年数 & \SimHei \scriptsize 公元 & \SimHei 大事件 \tabularnewline
  \midrule
  \endhead
  \midrule
  元年 & 189 & \tabularnewline
  \bottomrule
\end{longtable}

\subsection{初平}

\begin{longtable}{|>{\centering\scriptsize}m{2em}|>{\centering\scriptsize}m{1.3em}|>{\centering}m{8.8em}|}
  % \caption{秦王政}\
  \toprule
  \SimHei \normalsize 年数 & \SimHei \scriptsize 公元 & \SimHei 大事件 \tabularnewline
  % \midrule
  \endfirsthead
  \toprule
  \SimHei \normalsize 年数 & \SimHei \scriptsize 公元 & \SimHei 大事件 \tabularnewline
  \midrule
  \endhead
  \midrule
  元年 & 190 & \tabularnewline\hline
  二年 & 191 & \tabularnewline\hline
  三年 & 192 & \tabularnewline\hline
  四年 & 193 & \tabularnewline
  \bottomrule
\end{longtable}


\subsection{兴平}

\begin{longtable}{|>{\centering\scriptsize}m{2em}|>{\centering\scriptsize}m{1.3em}|>{\centering}m{8.8em}|}
  % \caption{秦王政}\
  \toprule
  \SimHei \normalsize 年数 & \SimHei \scriptsize 公元 & \SimHei 大事件 \tabularnewline
  % \midrule
  \endfirsthead
  \toprule
  \SimHei \normalsize 年数 & \SimHei \scriptsize 公元 & \SimHei 大事件 \tabularnewline
  \midrule
  \endhead
  \midrule
  元年 & 194 & \tabularnewline\hline
  二年 & 195 & \tabularnewline
  \bottomrule
\end{longtable}

\subsection{建安}

\begin{longtable}{|>{\centering\scriptsize}m{2em}|>{\centering\scriptsize}m{1.3em}|>{\centering}m{8.8em}|}
  % \caption{秦王政}\
  \toprule
  \SimHei \normalsize 年数 & \SimHei \scriptsize 公元 & \SimHei 大事件 \tabularnewline
  % \midrule
  \endfirsthead
  \toprule
  \SimHei \normalsize 年数 & \SimHei \scriptsize 公元 & \SimHei 大事件 \tabularnewline
  \midrule
  \endhead
  \midrule
  元年 & 196 & \tabularnewline\hline
  二年 & 197 & \tabularnewline\hline
  三年 & 198 & \tabularnewline\hline
  四年 & 199 & \tabularnewline\hline
  五年 & 200 & \tabularnewline\hline
  六年 & 201 & \tabularnewline\hline
  七年 & 202 & \tabularnewline\hline
  八年 & 203 & \tabularnewline\hline
  九年 & 204 & \tabularnewline\hline
  十年 & 205 & \tabularnewline\hline
  十一年 & 206 & \tabularnewline\hline
  十二年 & 207 & \tabularnewline\hline
  十三年 & 208 & \tabularnewline\hline
  十四年 & 209 & \tabularnewline\hline
  十五年 & 210 & \tabularnewline\hline
  十六年 & 211 & \tabularnewline\hline
  十七年 & 212 & \tabularnewline\hline
  十八年 & 213 & \tabularnewline\hline
  十九年 & 214 & \tabularnewline\hline
  二十年 & 215 & \tabularnewline\hline
  二一年 & 216 & \tabularnewline\hline
  二二年 & 217 & \tabularnewline\hline
  二三年 & 218 & \tabularnewline\hline
  二四年 & 219 & \tabularnewline\hline
  二五年 & 220 & \tabularnewline
  \bottomrule
\end{longtable}

\subsection{延康}

\begin{longtable}{|>{\centering\scriptsize}m{2em}|>{\centering\scriptsize}m{1.3em}|>{\centering}m{8.8em}|}
  % \caption{秦王政}\
  \toprule
  \SimHei \normalsize 年数 & \SimHei \scriptsize 公元 & \SimHei 大事件 \tabularnewline
  % \midrule
  \endfirsthead
  \toprule
  \SimHei \normalsize 年数 & \SimHei \scriptsize 公元 & \SimHei 大事件 \tabularnewline
  \midrule
  \endhead
  \midrule
  元年 & 220 & \tabularnewline
  \bottomrule
\end{longtable}


%%% Local Variables:
%%% mode: latex
%%% TeX-engine: xetex
%%% TeX-master: "../Main"
%%% End:


%%% Local Variables:
%%% mode: latex
%%% TeX-engine: xetex
%%% TeX-master: "../Main"
%%% End:
