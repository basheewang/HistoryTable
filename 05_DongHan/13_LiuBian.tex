%% -*- coding: utf-8 -*-
%% Time-stamp: <Chen Wang: 2021-11-01 11:31:21>

\section{少帝刘辩\tiny(189)}

\subsection{生平}

劉辯(176年-190年3月6日,熹平五年-初平元年正月十二癸丑日),中國汉朝皇帝(光熹元年四月十三戊午日至昭宁元年九月初一甲戌日,即公元189年5月15日-189年9月28日在位)。他是东汉第十三位、亦即倒数第二位皇帝,是汉灵帝劉宏与皇后何氏的独生兒子,即是嫡长子。

刘辩在灵帝驾崩后继位为帝,由于年幼,实权掌握在临朝称制的母亲何太后和母舅大将军何进手中。少帝在位时期,东汉政权已经名存实亡,他即位后不久即遭遇以何进为首的外戚集团和以十常侍为首的内廷宦官集团这两大敌对政治集团的火并,被迫出宫,回宫后又受制于以“勤王”为名进京的西北军阀董卓,之後被废为弘农王,成为东汉唯一被废黜的皇帝,其同父异母弟陈留王刘协继位为帝,是为汉献帝。被废黜一年之后,刘辩在董卓胁迫下自尽,时年仅十五岁,其弟献帝追谥他为怀王。

中国古代的史书中称刘辩为皇子辩、少帝和弘农王等,但因为刘辩在位不逾年,一般不把他看作是汉朝正统的皇帝,不单独为他撰写专属于帝王的傳記(即本紀),不过现代史学界也有观点承认他是汉朝皇帝。

劉辯出生于汉灵帝熹平五年(176年)是西漢景帝的第12代孫,父親是当朝皇帝漢靈帝刘宏,母親是来自南陽郡宛县(今河南省南阳市)的宫女何氏,所以刘辩是庶出。皇子辩出生后,何氏母以子贵,被封为贵人,宋皇后被废两年后又被晋封为皇后。

在皇子辩出生之前,灵帝的皇子们都已夭折,所以皇子辩出生后没有养在皇宫中,而养在道人史子眇的家里,不敢叫他的正名,称他为“史侯”。因为史道人有道术,何氏想凭借他的道术保护皇子辩。

灵帝不喜欢皇子辩,而喜欢王美人所生的皇子协。群臣奏请灵帝立皇太子时,灵帝认为皇子辩行为轻佻,没有君主的威仪,不适合做皇帝,想立皇子协为太子,但因何皇后在宫中受宠,而且何皇后的兄长何进任大将军,控制軍隊,并在朝中位高权重,故立太子之事久拖不决,一直到灵帝驾崩时都没有立太子。但《后汉书·盖勋传》《三国志·武帝纪》提到刘辩时都称之为“太子”,未详孰是。

小黄门高望为尚药监,为刘辩所宠信,刘辩通过灵帝信任的宦官上军校尉蹇硕意图让高望之子进为孝廉,京兆尹盖勋不肯。有人对盖勋说:“太子是未来的皇帝,蹇硕是陛下的宠臣,你都违背了他们的意愿,恐怕三怨成府。”盖勋回答:“选举贤良的人为官是为了报效国家,不是贤良我不会举荐,死又何悔!”

中平六年(189年),靈帝病重,弥留之际将心目中的继承人皇子协托付给蹇硕。同年夏四月十一丙辰日(189年5月13日),灵帝驾崩,蹇硕想先杀何进再立皇子协为帝,于是请何进入后宫。但是何进刚从外朝进入后宫,蹇硕的司马潘隱是何进的旧识,向何进迎面走去,并用眼神暗示何进。何进大惊,先退出,从便道回到军营,然后谎称自己生病,不能入宫,蹇硕的计划因此失败,而皇子辯也得以顺利繼承帝位。

灵帝驾崩两天后(同年夏四月十三戊午日,即189年5月15日),皇子辩即位,是為漢少帝,时年十四岁。少帝辩尊母亲何皇后为皇太后,由于少帝年少,何太后临朝称制。宣布大赦天下,改元为光熹。皇弟协时年九岁,封为渤海王。又封后将军袁隗为太傅,与大将军何进同录尚书事,诏书如下:

朕以眇身,君主海内,夙夜忧惧,靡知所济。夫天地人道,其用在三,必须辅佐,以昭其功。后将军袁隗德量宽重,奕世忠恪。今以隗为太傅录尚书事。朕且谅闇,委成群后,各率其职,称朕意焉。

蹇硕在少帝即位后仍然想改立皇子协,又害怕何进掌权后誅殺自己,乃求助于赵忠等其它宦官,但他们或与何进亲近,或急于自保,反而向何进出卖了蹇硕。不久,何进命令黄门令逮捕蹇硕并诛杀之。

灵帝末年,东汉王朝已经摇摇欲坠:朝廷外部爆发了黃巾之亂,东汉中央政府在镇压民变的过程中将权力下放给地方,形成各地事实上的军阀割据的局面,东汉政权已经名存实亡;朝廷内部以大将军何进为首的外戚集团和以十常侍为首的宦官集团在镇压民变期间短暂联合后再度敌对。刘辩正是在这样的情况下即位的。少帝辩继位后,由于年少,一切政事取决于临朝称制的母后和手握兵权的国舅大将军何进。何进外戚集团和宦官集团的矛盾愈演愈烈,终于导致宫变的发生。

何进在诛杀蹇硕后,更进一步欲将宦官们全部诛杀,便召并州牧董卓带兵入京助他一臂之力,并要求妹妹何太后同意杀宦官,但何太后不准。光熹元年八月廿五戊辰日(189年9月22日),何进再次进入何太后居住的长乐宫,要求何太后同意他诛杀全体中常侍。张让、段珪等宦官们听到风声,乃先下手为强,在宫中的嘉德殿前将他杀死。何进的属下吳匡、张璋、袁术等得知何进被殺害,乃带兵欲入后宫杀尽宦官。

第二天(八月廿六己巳日,即189年9月23日)天亮后,宦官们在后宫中坚持不住,乃入长乐宫奏报何太后,谎称大将军的部下谋反,乘机裹胁何太后、少帝、陈留王协和省内官属,劫持宫内其他官员从天桥阁道逃向北宫德阳殿。何太后中途被尚书卢植所救。

八月廿七庚午日(189年9月24日),张让、段珪等迫于追兵,被困北宫中,无计可施,只好带着少帝、陈留王等数十人步行出谷门。入夜后到达小平津。皇帝所用的六颗玉玺未随身攜带,无公卿跟随,只有尚书卢植、河南中部掾闵贡夜里到达黄河岸边,碰到了少帝一行。闵贡率骑兵于快破晓时追上。少帝又饿又渴,闵贡乃杀羊进上,又厉声斥责张让等人祸国乱政,并持剑砍死数名宦官。张让等人惶恐不安,知道死期已到,乃向少帝拱手再拜,并叩头辞别,随即投河自尽。

闵贡扶着少帝与陈留王,乘着夜色追着萤火虫发出的微光徒步往南行,欲回皇宫。走了几里地,得到百姓家一辆板车,三人乘车到洛舍後,下车休息。天亮后(八月廿八辛未日,即189年9月25日),找到两匹马,少帝独骑一匹,陈留王与闵贡合骑一匹,从雒舍往南行,这时才渐渐地有公卿赶来会合。

奉何进之令入京“勤王”的并州牧董卓率军来到显阳苑,远远望见宫中起火,知有变故发生,便统兵急速前进。天还没亮,来到京城西,听说少帝一行在北边即將回宫,就率军与大臣们一起到雒阳城北的北芒阪(今北邙山)下迎接少帝。少帝望见董卓突然率大军前来,吓得哭泣流泪。董卓上前与少帝叙话,少帝语无伦次,一旁的陈留王则对答如流。董卓十分高兴,觉得陈留王比少帝贤能,而且认为自己与抚养陈留王的董太后是同族,于是有废黜少帝,改立陈留王之念。

据说汉灵帝末年,民间盛传童谣:“侯非侯,王非王,千乘万骑上北芒。”宫变发生时,刘辩已经登基为帝,已非昔日的“史侯”,而当时的陈留王刘协不久就成为皇帝(献帝),所以叫“侯非侯,王非王”,而“千乘万骑走北芒”指的是百官公卿乘车骑马保护少帝和陈留王上了北邙山,至此童谣灵验,形容当时的政局混乱。然而,现代研究認為這是献帝即位后才被编造出来的,为的是愚民,以巩固统治地位。

当日(八月廿八辛未日,189年9月25日),少帝返回宫中,大赦天下,改元为昭宁。

董卓入京後,取代司空刘弘之位,权倾朝野,但他知道并不是所有人都服他,乃决定废少帝,另立陈留王为帝以提高自己的威望。由于董卓手握重兵,朝中除卢植和袁绍外无人敢反对,遂定下废立大计。

昭宁元年九月初一甲戌日(189年9月28日),董卓在崇德前殿召集百官,逼何太后下诏书废黜少帝,诏书写道:“孝灵皇帝不究高宗眉寿之祚,早弃臣子。皇帝承绍,海内侧望,而帝天姿轻佻,威仪不恪,在丧慢惰,衰如故焉;凶德既彰,淫秽发闻,损辱神器,忝污宗庙。皇太后教无母仪,统政荒乱。永乐太后暴崩,众论惑焉。三纲之道,天地之纪,而乃有阙,罪之大者。陈留王协,圣德伟茂,规矩邈然,丰下兑上,有尧图之表;居丧哀戚,言不及邪,岐嶷之性,有周成之懿。休声美称,天下所闻,宜承洪业,为万世统,可以承宗庙。废皇帝为弘农王。皇太后还政。”

诏书大意是少帝天生举止轻佻,仪表缺乏君王应有的威严,在为先帝(汉灵帝)守丧期间,没有尽到作儿子的孝心,懒散怠慢,和平日不守丧时没什么两样,甚至做出淫乱的行为,丑闻被天下人所知,有辱社稷和祖宗;何太后教无母仪,政事混乱,永乐宫(董太皇太后)死得不明不白。因此将少帝废黜,而陈留王贤明,故另立陈留王为帝,何太后还政。

诏书颁布后,太傅袁隗把废帝弘农王身上佩带的玺绶解下来,进奉给陈留王。陈留王遂即位,改元永汉,是为汉献帝。然后袁隗扶弘农王下殿,向坐在北面的新皇帝称臣。见废帝此状,何太后哽咽流涕,群臣心中悲痛,但都敢怒不敢言。

同年,董卓以“太后踧迫永乐宫,至令忧死,逆妇姑之礼”为由迁何太后于永安宫,后又以鸩酒毒杀,让汉献帝出奉常亭举哀,以太后被废为由,公卿皆穿白衣与会,不成丧。以皇后而非太后礼合葬何太后于灵帝文昭陵。

董卓废少立献的第二年正月(初平元年,190年),山东各地的諸侯、官吏等起兵讨伐董卓。董卓怕這些刺史、州牧、太守以迎废帝弘农王复辟为名讨伐自己,干脆将弘农王害死。董卓将弘农王置于阁上,于初平元年正月十二癸丑日(190年3月6日)派郎中令李儒进献毒酒给弘农王,说道:“服此药,可以辟恶。”弘农王说:“我没有病,这是想杀我罢了!”不肯喝。李儒强迫他喝,不得已,乃与妻唐姬及随从宫人饮宴而别。饮酒过程中,弘农王悲歌道:“天道易兮我何艰!弃万乘兮退守蕃。逆臣见迫兮命不延,逝将去汝兮适幽玄!”乃令唐姬起舞,唐姬举袖而歌。歌毕,弘农王对唐姬说:“爱卿是本王的妃子,势将不复为吏民之妻。自己保重,从此长辞!”遂喝毒酒而死,时年十五岁。

初平元年二月,献帝下诏将哥哥弘农王葬于已故中常侍赵忠的墓穴中,谥曰怀王。李傕之乱后,献帝得知嫂子唐姬尚在人世,乃颁下诏书迎接她,封她为弘农王妃,将她安置在已故弘农王的墓园中。

刘辩的父亲汉灵帝刘宏在群臣请立太子时对刘辩的评语:“刘辩行为轻佻,不具备帝王应有的威仪,不可以做人主(皇帝)。”

蹇硕在刘辩继位后仍然认为“皇帝轻佻无品行”,依然准备改立渤海王刘协。

董卓对刘辩评价很差,但他只是以此为借口想废黜刘辩的帝位。

董卓对袁绍说他想要废掉刘辩帝位的理由时对刘辩的评价:“皇帝年幼愚昧,不配做万乘之主(皇帝)。”袁绍则马上回击他:“当今圣上还很年轻,天下人没有听说他有什么不好的行为。”

董卓召集群臣商讨废立大事时对刘辩的评价:“如今皇帝愚昧软弱,不可以做天下之主(皇帝),他的牌位不配放在宗庙内让后人供奉。”群臣不敢多说,唯独尚书卢植不同意,他评价道:“当今圣上还很年轻,行为并没有失当之处,和以前的事没有可比性。”

董卓在废少帝当天对群臣说到他对刘辩的评价:“天子年幼、软弱而不配做一国之君。……皇帝应该像昌邑王那样被废黜”随后董卓逼迫何太后颁布废少帝立献帝的诏书,诏书对刘辩评价道:“皇帝继承皇统,海内之人满怀希望,但皇帝天生行为轻佻,不具备帝王应有的威仪,在为先帝守丧期间懒惰怠慢,和平日不守丧时没什么两样;甚至做出淫乱的行为,丑闻被天下人所知,有辱神器和宗庙。”

卢植:「今上富于春秋,行未有失,非前事之比也。」(《三国志·魏书·董二袁刘传第六》)

毛宗岗:「甚矣,帝之多文矣。既作感怀诗于前,复作绝命词于后。文章无救于祸患,我为天子一哭,更为文章一哭。」(汇评三国志演义)

中国现代文学家鲁迅评价刘辩临终前的诀别为“汉宫之楚声”。

虽然刘辩在位时间很短,且实权在他人手中,自己很难有所作为,但作为曾经名义上的帝国最高首脑,与他相关的事情,尤其是被废和被害两件事对历史的进程仍有一定的影响。

少帝的被废,导致董卓专权成为必然的趋势,东汉政权有覆亡的危险,因为废立之事仅仅是董卓政治图谋的第一步,下一步他可能就会取而代之,实现政权更迭。

而少帝的被废,也使得山东的诸位州牧、刺史、太守等有了起兵讨伐董卓的理由,即扶持少帝复位,随着他们控制的地方政府与董卓控制的中央政府的决裂,标志着东汉帝国开始解体,为汉末群雄割据和三国的形成埋下伏笔。

著名中国历史学家吕思勉认为,董卓“不懂得政治”,废少帝而立献帝,是一件“给人家藉口的事”,“無故廢立,那是怎樣容易受人攻击的事啊!”,可见得“他是一个草包”,种下“失败之由”。

弘农王的被害,政治上点中山东反董卓盟军的“死穴”,因为他们本来以扶持弘农王复位为由而起兵,而弘农王既然已死,他们的理由就不成立了。

但弘农王的被害又为反董卓盟军提供了新的理由,即董卓“杀害了皇帝(指逼死弘农王,犯下了弑君大罪),危及了江山社稷”,后来盟军的誓词中也说到了类似的理由:“汉室不幸,皇纲失统,贼臣董卓乘衅纵害,祸加至尊,虐流百姓,大惧沦丧社稷,翦覆四海。兖州刺史岱、豫州刺史伷、陈留太守邈、东郡太守瑁、广陵太守超等,纠合义兵,并赴国难。凡我同盟,齐心戮力,以致臣节,殒首丧元,必无二志。有渝此盟,俾坠其命,无克遗育。皇天后土,祖宗明灵,实皆鉴之!”大意也是如此。之后山東盟軍正式向董卓宣战,董卓因为山東盟軍人多势众,终于放弃都城雒阳(今河南省洛阳市),挟持献帝和百官迁都长安(今陕西省西安市汉长安城遗址)以避盟军的锋芒。

\subsection{光熹}

\begin{longtable}{|>{\centering\scriptsize}m{2em}|>{\centering\scriptsize}m{1.3em}|>{\centering}m{8.8em}|}
  % \caption{秦王政}\
  \toprule
  \SimHei \normalsize 年数 & \SimHei \scriptsize 公元 & \SimHei 大事件 \tabularnewline
  % \midrule
  \endfirsthead
  \toprule
  \SimHei \normalsize 年数 & \SimHei \scriptsize 公元 & \SimHei 大事件 \tabularnewline
  \midrule
  \endhead
  \midrule
  元年 & 189 & \tabularnewline
  \bottomrule
\end{longtable}

\subsection{昭宁}

\begin{longtable}{|>{\centering\scriptsize}m{2em}|>{\centering\scriptsize}m{1.3em}|>{\centering}m{8.8em}|}
  % \caption{秦王政}\
  \toprule
  \SimHei \normalsize 年数 & \SimHei \scriptsize 公元 & \SimHei 大事件 \tabularnewline
  % \midrule
  \endfirsthead
  \toprule
  \SimHei \normalsize 年数 & \SimHei \scriptsize 公元 & \SimHei 大事件 \tabularnewline
  \midrule
  \endhead
  \midrule
  元年 & 189 & \tabularnewline
  \bottomrule
\end{longtable}


%%% Local Variables:
%%% mode: latex
%%% TeX-engine: xetex
%%% TeX-master: "../Main"
%%% End:
