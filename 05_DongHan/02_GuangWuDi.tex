%% -*- coding: utf-8 -*-
%% Time-stamp: <Chen Wang: 2019-12-17 16:21:25>

\section{光武帝\tiny(25-57)}

\subsection{生平}

漢光武帝劉秀(前5年1月15日-57年3月29日),字文叔,小名呼 南陽郡蔡陽縣人[a](今湖北省襄阳枣阳市),祖籍徐州沛縣,東漢第一位皇帝,25年8月5日-57年3月29日在位。廟號世祖,諡號光武皇帝。

劉秀為漢高帝九世孫,漢景帝七世孫,长沙定王刘发之后,出身於南陽郡的地方豪族。新朝末年國家動蕩,各地寇盜蜂起。地皇三年(22年),劉秀與其兄長劉縯在宛(今河南省南陽市)起兵。25年,在鄗縣(今河北省石家莊市高邑縣)登基稱帝,改元建武,國號為「漢」,史稱東漢。此後,劉秀逐步掃平各方勢力,最終統一中國。劉秀在位三十二年,社會逐漸從新朝末年的動蕩中恢復,故稱「光武中興」。建武中元二年(57年),劉秀逝世於雒陽。

劉秀的軍事才能很高。稱帝之後遣眾將攻伐四方,往往能從前方上報的排兵布陣形勢中發現問題,有時因前方不能及時得到糾正,便為敵人所敗。此外,劉秀待人誠懇簡約,寬厚有信,竇融、馬援等均由此歸心。對外政策方面,引南匈奴內遷入塞,分置諸部於北地、朔方、五原、雲中、定襄、雁門、代郡、西河緣邊八郡,詔單于徙居西河美稷。但此舉也成為東漢朝廷和民眾沈重的經濟負擔,在東漢與北匈奴的戰爭中南匈奴僅起到出兵助攻的作用,談不上替東漢守衛北邊。到了東漢中期由於羌患,使得南匈奴在北邊不斷發起暴亂,對東漢北邊邊防乃至北方內地的安全構成了嚴重威脅,從而成為東漢北邊的一大邊患。

漢哀帝建平元年十二月甲子(前5年1月15日)夜於陳留郡濟陽縣出生。刘秀出生的时候,有赤光照耀整個房間,當年稻禾(嘉禾)一茎九穗,因此得名秀。

刘秀是汉高帝刘邦九世孙,西汉景帝子长沙定王刘发之子舂陵節侯劉買的玄孙,與更始帝有同一位高祖父劉買。其父为南顿令劉欽,母樊娴都。世代居住在南阳郡蔡陽(今湖北省枣阳市西南),屬地方豪族。刘秀九岁时,父亲逝世,便由叔父劉良抚养。由于刘秀勤于农事,而兄劉縯好侠养士,经常取笑刘秀,将他比做刘邦的兄弟刘喜。新朝天凤年间(14年—19年),刘秀至长安,学习《尚书》,略通大义。成年後劉秀身高七尺三寸(身高175厘米以上)。

刘秀在新野县时,听闻陰麗華的美貌,心悦之。後至長安,見執金吾車騎甚盛,因歎曰:“仕宦當作執金吾,娶妻當得陰麗華。”

時值新莽天鳳五年(17年),天下大亂,赤眉軍與綠林軍各自起兵反王莽。地皇三年(22年),劉秀避吏於新野,因賣穀而至宛(今河南省南陽市),經李通勸說在宛起兵。地皇四年(23年)二月,劉縯、劉秀兄弟與綠林兵共同擁護劉玄稱帝,國號仍為漢,改元更始,史稱更始帝。同年劉秀率綠林軍1萬以少勝多於昆陽滅王莽軍42萬,殺其主帥王尋,史稱昆陽之戰。

此後劉縯、劉秀兄弟威望大盛,遭到劉玄的猜忌。劉秀有所察覺,但劉縯不以為意,終被劉玄藉故殺死,同被殺死的還有同宗劉稷。此時劉秀也處於危險之中,只得向劉玄謝罪,並不敢為哥哥服喪,飲食言笑如常。劉玄心有所慚,故而拜劉秀為破虜大將軍、武信侯。

後更始帝劉玄攻占长安,新莽灭亡。时河北王郎起兵,于是更始帝派劉秀巡視黃河以北,劉秀始得脫離險境。劉秀遂在河北積蓄力量,日益壯大,被更始帝封為「蕭王」。劉秀率吳漢、鄧禹等手下大將,繼續在北方大破銅馬等割據勢力,被關西人號為「銅馬帝」。由於劉秀與更始帝心生二意,自此劉秀手下便不斷勸進。

更始三年(25年)六月,赤眉军立刘盆子为傀儡皇帝。同月二十二己未日(25年8月5日),劉秀于鄗城即皇帝位,改元建武,国号仍为汉,史称东汉。九月,赤眉军击败刘玄,漢更始帝投降,同年十二月被杀。

劉秀因为汉朝是火德的缘故迁都洛阳,改洛阳为「雒阳」。刘秀先后荡平赤眉、张步、隗嚣等割据势力,然割据一方的卢芳,刘秀屡次遣吴汉、杜茂往击,均不克。建武十二年(36年),卢芳进攻云中郡,留守九原的部将随育胁迫卢芳降伏刘秀。卢芳放弃军队,逃往匈奴。同年十一月十九己卯日(36年12月25日),吴汉攻克成都,割据四川的公孙述成家政权灭亡,东汉统一中国。

公元前108年,漢武帝滅朝鮮。 建武中元二年(公元57年),第一次有明文記載「倭人國家」與中國往來。九州北部(博多灣沿岸)的倭奴國接受漢王朝的策封,光武帝封其為「倭奴國王」,並授予金印。

1784年,在日本北九州地區博多灣志賀島,出土一枚刻有「漢委奴國王」五個字的金印。這一枚金印也為中日兩國最早交往的證明。

刘秀勤于政事,“每旦视朝,日仄乃罢,数引公卿郎将议论经理,夜分乃寐”。在位期间,多次发布释放奴婢和禁止残害奴婢的诏书。为减少贫民卖身为奴婢,经常发救济粮,减少租徭役,兴修水利,发展农业生产。裁并郡县,精简官员。结果,裁并四百余县,官员十置其一。历史上称其统治时期为光武中兴。其间国势昌隆,号称“建武盛世”。 刘秀统一中国后,厌武事,不言军旅,建武二十七年(51年),朗陵侯臧宫、扬虚侯马武上书:请乘匈奴分裂、北匈奴衰弱之际发兵击之,立“万世刻石之功”。光武却下诏:“今国无善政,灾变不息,人不自保,而复欲远事边外乎!……不如息民。”刘秀不同于明太祖朱元璋得天下后诛杀大批功臣只留下汤和徐達的无情,刘秀分封三百六十多位功臣为列侯,给予他们尊崇的地位,只解其兵权,刘秀诛杀功臣一说源于戏剧,令刘秀蒙受「不白之冤」。其实,在统一中国之前,他就开始削弱国防建设,废郡国兵制,罢郡国都尉。削弱地方兵权的同时,导致后来无力抵御外患,而豪强地主的部曲家兵则迅速发展,像东汉末年的董卓就是一例。刘秀以后不设丞相,而是“虽置三公”但“事归台阁”;一方面削弱三公权力,使三公成为虚位,另一方面又扩大尚书台的职权,成为皇帝发号施令的执行机构,所有权力集中于皇帝一身。”《后汉书·申屠刚传》说:“时内外群官,多帝自选举,加以法理严察,职事过苦,尚书近臣,至乃捶扑牵曳于前,群臣莫敢正言。”“自是大臣难居相任”。建武二十八年(52年)他借故搜捕王侯宾客,“坐死者数千人”,严禁结党营私。

建武中元二年二月初五日戊戌(57年3月29日),崩于雒阳南宫前殿,享壽六十二岁,在位三十二年。三月丁卯(4月27日),安葬于漢原陵(今河南孟津县铁谢村附近),庙号世祖,谥光武皇帝。刘秀駕崩后,其子汉明帝刘庄将统一战争中功劳最大的二十八人的影像画在云台阁,称云台二十八将。

《资治通鉴》称刘秀是个宽厚简易的人。在统一过程中,刘玄的一些手下曾参与谋害他的哥哥,他能够不计前嫌地招降并厚待;分封功臣时,不顾他人劝说,将最大的封地劃到了四县之广;战争尚未结束,就将原来十分之一的税率减到三十分之一;马援为隗嚣所使,分别访问公孙述和刘秀,独为刘秀的人格魅力折服;耿弇、窦融曾专制一方,以兵多权大心不自安,而刘秀对他们未有半点疑虑。凡此种种,都成为他成功的决定性因素。甚至在统一之后,他废郭皇后及太子劉彊,立阴皇后及次子刘阳(后改名莊),犹能令郭皇后到其子中山王的封国安享餘年,两子之间不生嫌隙,也没有受到臣下及后人的议论。

范曄:「雖身濟大業,競競如不及,故能明慎政體,總欖權綱,量時度力,舉無過事,退功臣而進文吏,戢弓矢而散馬牛,雖道未方古,斯亦止戈之武焉。」

诸葛亮:“光武神略计较,生于天心,故帷幄无他所思,六奇无他所出,于是以谋合议同,共成王业而已。”

明朝官修皇帝实录《明太祖实录》记载,明太祖朱元璋在洪武七年八月初一日(1374年9月7日),亲自前往南京历代帝王庙祭祀三皇、五帝、夏禹王、商汤王、周武王、汉高祖、汉光武帝、隋文帝、唐太宗、宋太祖、元世祖一共十七位帝王,其中对汉光武帝刘秀的祝文是:“惟汉光武皇帝延揽英雄,励精图治,载兴炎运,四海咸安。有君天下之德而安万世之功者也。元璋以菲德荷天佑人助,君临天下,继承中国帝王正统,伏念列圣去世已远,神灵在天,万古长存,崇报之礼,多未举行,故于祭祀有阙。是用肇新庙宇于京师,列序圣像及历代开基帝王,每岁祀以春、秋仲月,永为常典。今礼奠之初,谨奉牲醴、庶品致祭,伏惟神鉴。尚享!”

王夫之《读通鉴论》:“光武之得天下,较高帝而尤难矣。光武之神武不可测也!三代而下,取天下者,唯光武独焉!”“自三代而下,唯光武允冠百王矣”。

光武帝承袭西汉后期法律宽松的弊病,又过于剝奪三公的職權,明章之治以后皇帝幼小,陷入了长期的戚宦之爭之黑暗和混乱。

刘秀迷信图谶,與不信谶的大臣發生衝突,且时而感情用事,处事不公,韩歆因直谏被逼死,劉秀又包庇湖阳公主险些杀死董宣。

刘秀縱容部下吴汉军对邓奉的家乡进行劫掠,导致邓奉反叛,后来邓奉兵败投降被杀。平狄将军庞萌与盖延共击董宪,而诏书却只下达给盖延、不给庞萌,庞萌以为盖延说自己坏话,起疑,反叛,后来庞萌兵败被杀。

\subsection{建武}

\begin{longtable}{|>{\centering\scriptsize}m{2em}|>{\centering\scriptsize}m{1.3em}|>{\centering}m{8.8em}|}
  % \caption{秦王政}\
  \toprule
  \SimHei \normalsize 年数 & \SimHei \scriptsize 公元 & \SimHei 大事件 \tabularnewline
  % \midrule
  \endfirsthead
  \toprule
  \SimHei \normalsize 年数 & \SimHei \scriptsize 公元 & \SimHei 大事件 \tabularnewline
  \midrule
  \endhead
  \midrule
  元年 & 25 & \tabularnewline\hline
  二年 & 26 & \tabularnewline\hline
  三年 & 27 & \tabularnewline\hline
  四年 & 28 & \tabularnewline\hline
  五年 & 29 & \tabularnewline\hline
  六年 & 30 & \tabularnewline\hline
  七年 & 31 & \tabularnewline\hline
  八年 & 32 & \tabularnewline\hline
  九年 & 33 & \tabularnewline\hline
  十年 & 34 & \tabularnewline\hline
  十一年 & 35 & \tabularnewline\hline
  十二年 & 36 & \tabularnewline\hline
  十三年 & 37 & \tabularnewline\hline
  十四年 & 38 & \tabularnewline\hline
  十五年 & 39 & \tabularnewline\hline
  十六年 & 40 & \tabularnewline\hline
  十七年 & 41 & \tabularnewline\hline
  十八年 & 42 & \tabularnewline\hline
  十九年 & 43 & \tabularnewline\hline
  二十年 & 44 & \tabularnewline\hline
  二一年 & 45 & \tabularnewline\hline
  二二年 & 46 & \tabularnewline\hline
  二三年 & 47 & \tabularnewline\hline
  二四年 & 48 & \tabularnewline\hline
  二五年 & 49 & \tabularnewline\hline
  二六年 & 50 & \tabularnewline\hline
  二七年 & 51 & \tabularnewline\hline
  二八年 & 52 & \tabularnewline\hline
  二九年 & 53 & \tabularnewline\hline
  三十年 & 54 & \tabularnewline\hline
  三一年 & 55 & \tabularnewline\hline
  三二年 & 56 & \tabularnewline
  \bottomrule
\end{longtable}

\subsection{建武中元}

\begin{longtable}{|>{\centering\scriptsize}m{2em}|>{\centering\scriptsize}m{1.3em}|>{\centering}m{8.8em}|}
  % \caption{秦王政}\
  \toprule
  \SimHei \normalsize 年数 & \SimHei \scriptsize 公元 & \SimHei 大事件 \tabularnewline
  % \midrule
  \endfirsthead
  \toprule
  \SimHei \normalsize 年数 & \SimHei \scriptsize 公元 & \SimHei 大事件 \tabularnewline
  \midrule
  \endhead
  \midrule
  元年 & 56 & \tabularnewline\hline
  二年 & 57 & \tabularnewline
  \bottomrule
\end{longtable}


%%% Local Variables:
%%% mode: latex
%%% TeX-engine: xetex
%%% TeX-master: "../Main"
%%% End:
