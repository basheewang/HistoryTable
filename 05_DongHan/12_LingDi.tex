%% -*- coding: utf-8 -*-
%% Time-stamp: <Chen Wang: 2021-11-01 11:30:41>

\section{灵帝刘宏\tiny(168-189)}

\subsection{生平}

汉灵帝刘宏(157年-189年5月13日),东汉第十二位皇帝(168年2月17日—189年5月13日在位),在位22年,葬于汉文陵,其正式諡號為「孝靈皇帝」,後世省略「孝」字稱「漢灵帝」。灵帝是东汉最后一个握有实权的皇帝。自從靈帝崩後,外戚何太后、何進掌權,漢帝自此淪爲傀儡。再後董卓、李傕、曹操相繼把持朝政,東漢大權完全落入董卓、李傕、曹操手中。

刘宏本封解渎亭侯,为承袭其父刘苌的爵位。母董夫人。他是漢章帝的玄孫,漢桓帝的堂侄。

永康元年(168年1月25日)桓帝崩,刘儵以光禄大夫身份与中常侍曹节带领中黄门、虎贲、羽林军一千多人,前往河间迎接刘宏。建宁元年正月二十日(168年2月16日),刘宏来到夏门亭,窦武亲自持节用青盖车把他迎入殿内。第二天,登基称帝,改元为“建宁”。由桓帝的皇后竇妙立為皇帝,承嗣汉桓帝,是为汉灵帝。 

汉灵帝即位后,东汉政治已经病入膏肓,天下水灾、旱灾、蝗灾、瘟疫等灾祸频繁,四处怨声载道,百姓民不聊生,国势进一步衰落。再加上宦官与外戚争权夺利,最后宦官曹节、王甫等推翻外戚窦氏並軟禁竇太后,夺得了大权,又杀死正义的太学生李膺、范滂等100余人,流放、关押800多人,多惨死于狱中,造成第二次党锢之祸。灵帝一方面保留窦太后的尊号,一方面将生母董氏迎入宫中尊为太后。将军张奂、郎中谢弼、黄门令董萌都为窦太后求情,灵帝感念窦太后拥立之恩,也一度被打动,率群臣为其上寿及增加供奉,但始终没有解除其幽禁,谢弼、董萌反而被宦官报复而死。窦太后忧死后,曹节、王甫因深恨窦氏,提出追废她及改以冯贵人配享桓帝,在廷尉陈球、太尉李咸的据理力争及灵帝本人坚持下,未果。

熹平四年(175年),议郎蔡邕认为儒家经典流传过程中出现许多错误,于是联合中常侍李巡、五官中郎将堂谿典、光禄大夫杨赐、谏议大夫马日磾、议郎张驯、韩说、太史令单飏等人共同上书要求校勘儒家经典。于是汉灵帝设立熹平石经,将校勘后的儒家经典分别刻在四十六块石碑之上,并安置在太学门外,作为经典标准,供人学习。

熹平六年(177年),鑒於鮮卑多次侵擾漢朝邊境。夏育建議討伐鮮卑,在朝廷多次商討后,派夏育、田晏、臧旻三路大軍討伐鮮卑。結果大敗而歸,夏育、田晏、臧旻被廢為庶人。

光和元年(178年),靈帝建立鴻都門學,最初號稱以研究儒术经义为名,后招集众多文士从事辞赋及书法等文艺创作活动。因鴻都門學专重文艺而轻儒家經典,引起不少大臣反對。

光和四年(181年),靈帝在皇宫之中扩建西园,修建集市供自己享乐。靈帝和宮女模仿民间市集里的商人、窃贼、地痞,并驾着白驴在西园中来回穿梭。汉灵帝同時长期沉迷于女色,灵帝特别喜欢一些冰肌玉洁的少女,还为此修建水池园林,是为裸游馆,和一群美女嬉戏于其中,并命令宫女只能穿开档裤,原因竟是为了方便自己临幸宫女。

昏庸荒淫的灵帝除了沉湎酒色以外,还一味宠信宦官,尊张让等人为“十常侍”,并说“张常侍乃我父、赵常侍乃我母”,宦官杖着皇帝的宠幸,胡作非为,对百姓勒索钱财,大肆搜刮民脂民膏,可谓腐败到极点。靈帝還多次賣官,先后有段颎、张温、崔烈、樊陵、曹嵩等人花钱买到三公之位。

在朝政腐败和天灾的双重压迫之下,叛乱有了广大的市场,巨鹿(今河北省平乡县)人张角煽动百姓,聚众造反。光和七年(184年)张角兄弟三人以“苍天已死、黄天当立、岁在甲子、天下大吉”为口号举事,史称“黃巾之亂”,这次暴乱所向披靡,给病入膏肓的东汉王朝以沉重打击。同時涼州爆發北宮伯玉之亂,國家一片衰敗。但靈帝不思悔改,繼續大幅增修宮殿,為此靈帝不惜增加民眾賦稅。

中平五年(188年),張純、張舉等人勾結烏桓叛亂,而冀州刺史王芬看見局勢混亂,圖謀廢除靈帝,但最終失敗。

鑒於漢室朝綱廢弛民變頻繁,靈帝以宦官蹇硕為統帥組建西園軍,自號無上將軍,令西園軍一度權勢高於大將軍何進。

公元189年5月13日,汉灵帝去世,终年32岁。7月17日,葬於漢文陵。漢靈帝死後引發漢朝最後一次戚宦相爭之宮變。

漢靈帝荒淫昏庸,曾于西園起裸游館千間,灵帝特别喜欢娇嫩纯洁的幼女,選十四歲以上十八歲以下的宮女于池中裸游,又曾于西園弄狗與人獸交。其人貪財,公開賣官鬻爵,致使朝政更加黑暗。但早年又有辞赋、書法和音樂爱好。

范晔《后汉书·孝灵帝纪》:“《秦本纪》说赵高谲二世,指鹿为马,而赵忠、张让亦绐灵帝不得登高临观,故知亡敝者同其致矣。然则灵帝之为灵也优哉!”、“灵帝负乘,委体宦孽。征亡备兆,《小雅》尽缺。麋鹿霜露,遂栖宫卫。”

董卓:“天下之主,宜得贤明,每念灵帝,令人愤毒!”《后汉书·卷七十四上·袁绍刘表列传第六十四上》

盖勋:“吾仍见上,上甚聪明,但拥蔽于左右耳。”《后汉书·虞傅盖臧列传第四十八》

张超《靈帝河閒舊廬碑》:赫赫在上.陶唐是承.繼德二祖.四宗是憑.上納鑒乎羲農.中結軌乎夏商.元首既明.股肱惟良.乃因舊宇.福德所基.修飾經構.農隙得時.樹中天之雙闕.崇冠山之華堂.通樓閑道.丹階紫房.金窗鬱律.玉璧內璫.青蒲充庖.朱草栖箱.川魚踊躍.雲鳥舞翔.煌煌大漢.含德乾綱.體效日月.驗化陰陽.格于上下.震暢八荒.三光宣曜.四靈效祥.天其嘉享.豐年穰穰.騶虞奏樂.鹿鳴薦觴.二祝致告.福祿來將.永保萬國.南山無量.(《艺文类聚 卷六十四》)

汉灵帝与其前任皇帝汉桓帝的统治时期是东汉最黑暗的时期,诸葛亮的《出师表》中就有蜀汉开国皇帝刘备每次“叹息痛恨于桓灵”的陈述:“亲贤臣,远小人,此先汉所以兴隆也;亲小人,远贤臣,此后汉所以倾颓也。先帝在时,每与臣论此事,未尝不叹息痛恨于桓、灵也。”

薛莹:“汉氏中兴,至于延平而世业损矣。冲质短祚,孝桓无嗣,母后称制,奸臣执政。孝灵以支庶而登至尊,由蕃侯而绍皇统,不恤宗绪,不祗天命;上亏三光之明,下伤亿兆之望。于时爵服横流,官以贿成。自公侯卿士降于皂隶,迁官袭级无不以货,刑戮无辜,摧扑忠良;佞谀在侧,直言不闻。是以贤智退而穷处,忠良摈于下位;遂至奸雄蜂起,当防隳坏,夷狄并侵,盗贼糜沸。小者带城邑,大者连州郡。编户骚动,人人思乱。当此之时,已无天子矣。会灵帝即世,盗贼相寻,其後宫室。焚灭,郊社无主,危自上起,覃及华夏。使京室为墟,海内萧条,岂不痛哉!”(《全晋文·卷八十一》)

王嘉《拾遗记》:“安、灵二帝,同为败德。夫悦目快心,罕不沦乎情欲,自非远鉴兴亡,孰能移隔下俗。佣才缘心,缅乎嗜欲,塞谏任邪,没情于淫靡。至如列代亡主,莫不凭威猛以丧家国,肆奢丽以覆宗祀。询考先坟,往往而载,佥求历古,所记非一。贩爵鬻官,乖分职之本;露宿郊居,违省方之义。”

虞世南:“灵帝承疲民之后,易为善政,黎庶倾耳。咸冀中兴,而帝袭彼覆车,毒逾前辈,倾覆宗社,职帝之由。天年厌世,为幸多矣。”(《唐文拾遗·卷十三》)

杜牧:“桓、灵四十年间杀千百比干,毒流其社稷,可以血食乎?可以坛?单父天拜郊乎?”(《樊川文集》)

周昙:“榜悬金价鬻官荣,千万为公五百卿。公瑾孔明穷退者,安知高卧遇雄英。”(《全唐诗·卷七百二十九》)

胡三省:“观灵帝以尚但之言不敢复升台榭,诚恐百姓虚散也,谓无爱民之心可乎!使其以信尚但者信诸君子之言,则汉之为汉,未可知也。”(《资治通鉴·卷第五十八·汉纪五十》)

蔡东藩《后汉演义》:“汉季之中常侍,谁不曰可杀?惟庸主如桓灵,方信而用之。”「国家赏罚有明经,宵小谗言怎可听?功罪不分昏愦甚,从知灵帝本无灵!」“若平乐观中之讲武,设坛张盖,夸示威风,灵帝自以为耀武,而盖勋乃以黩武为对,犹非知本之谈。黩武二字,惟汉武足以当之,灵帝岂足语此?彼之所信任者,妇寺而已,如皇甫嵩、朱儁诸才,皆不知重用;甚至一病不起,犹视赛硕为忠贞,托孤寄命,《范史》谓灵帝负扆,委体宦孽,征亡备兆,小雅尽缺,其亦所谓月旦之定评也乎?”


\subsection{建宁}

\begin{longtable}{|>{\centering\scriptsize}m{2em}|>{\centering\scriptsize}m{1.3em}|>{\centering}m{8.8em}|}
  % \caption{秦王政}\
  \toprule
  \SimHei \normalsize 年数 & \SimHei \scriptsize 公元 & \SimHei 大事件 \tabularnewline
  % \midrule
  \endfirsthead
  \toprule
  \SimHei \normalsize 年数 & \SimHei \scriptsize 公元 & \SimHei 大事件 \tabularnewline
  \midrule
  \endhead
  \midrule
  元年 & 168 & \tabularnewline\hline
  二年 & 169 & \tabularnewline\hline
  三年 & 170 & \tabularnewline\hline
  四年 & 171 & \tabularnewline\hline
  五年 & 172 & \tabularnewline
  \bottomrule
\end{longtable}

\subsection{熹平}

\begin{longtable}{|>{\centering\scriptsize}m{2em}|>{\centering\scriptsize}m{1.3em}|>{\centering}m{8.8em}|}
  % \caption{秦王政}\
  \toprule
  \SimHei \normalsize 年数 & \SimHei \scriptsize 公元 & \SimHei 大事件 \tabularnewline
  % \midrule
  \endfirsthead
  \toprule
  \SimHei \normalsize 年数 & \SimHei \scriptsize 公元 & \SimHei 大事件 \tabularnewline
  \midrule
  \endhead
  \midrule
  元年 & 172 & \tabularnewline\hline
  二年 & 173 & \tabularnewline\hline
  三年 & 174 & \tabularnewline\hline
  四年 & 175 & \tabularnewline\hline
  五年 & 176 & \tabularnewline\hline
  六年 & 177 & \tabularnewline\hline
  七年 & 178 & \tabularnewline
  \bottomrule
\end{longtable}

\subsection{光和}

\begin{longtable}{|>{\centering\scriptsize}m{2em}|>{\centering\scriptsize}m{1.3em}|>{\centering}m{8.8em}|}
  % \caption{秦王政}\
  \toprule
  \SimHei \normalsize 年数 & \SimHei \scriptsize 公元 & \SimHei 大事件 \tabularnewline
  % \midrule
  \endfirsthead
  \toprule
  \SimHei \normalsize 年数 & \SimHei \scriptsize 公元 & \SimHei 大事件 \tabularnewline
  \midrule
  \endhead
  \midrule
  元年 & 178 & \tabularnewline\hline
  二年 & 179 & \tabularnewline\hline
  三年 & 180 & \tabularnewline\hline
  四年 & 181 & \tabularnewline\hline
  五年 & 182 & \tabularnewline\hline
  六年 & 183 & \tabularnewline\hline
  七年 & 184 & \tabularnewline
  \bottomrule
\end{longtable}

\subsection{中平}

\begin{longtable}{|>{\centering\scriptsize}m{2em}|>{\centering\scriptsize}m{1.3em}|>{\centering}m{8.8em}|}
  % \caption{秦王政}\
  \toprule
  \SimHei \normalsize 年数 & \SimHei \scriptsize 公元 & \SimHei 大事件 \tabularnewline
  % \midrule
  \endfirsthead
  \toprule
  \SimHei \normalsize 年数 & \SimHei \scriptsize 公元 & \SimHei 大事件 \tabularnewline
  \midrule
  \endhead
  \midrule
  元年 & 184 & \tabularnewline\hline
  二年 & 185 & \tabularnewline\hline
  三年 & 186 & \tabularnewline\hline
  四年 & 187 & \tabularnewline\hline
  五年 & 188 & \tabularnewline\hline
  六年 & 189 & \tabularnewline
  \bottomrule
\end{longtable}


%%% Local Variables:
%%% mode: latex
%%% TeX-engine: xetex
%%% TeX-master: "../Main"
%%% End:
