%% -*- coding: utf-8 -*-
%% Time-stamp: <Chen Wang: 2019-12-26 10:22:23>

\section{太祖\tiny(960-976)}

\subsection{生平}

宋太祖趙匡胤(927年3月21日-976年11月14日),宋朝開國皇帝,祖籍涿郡保州保塞縣(今河北省保定市清苑区),父親趙弘殷,母親杜氏。後唐明宗天成年間(927年3月21日)生於後唐洛陽夾馬營(今河南省洛陽市瀍河回族區東關)。後漢時,趙匡胤於後漢隱帝在位期間投奔郭威,之後郭威篡漢建立後周,是為周太祖;而趙匡胤則得任東西班行首,始入宦途。959年,後周世宗於北征回京後不久駕崩,逝世前任命趙匡胤為殿前都點檢,執掌殿前司諸軍。隔年(960年)元月初一,北漢及契丹聯兵犯邊,趙匡胤受命防禦。初三夜晚,大軍於京城開封府(今河南省開封市)東北二十公里的陳橋驛(今河南省封丘縣陳橋鎮)發生政變,將士於隔日清晨擁立趙匡胤為帝,史稱「陳橋兵變」。大軍隨即回師京城,後周恭帝禪位,趙匡胤登基,建國號「宋」,是為「宋太祖」,年號為建隆,建立北宋。北宋和之后宋高宗建立的南宋國祚合计長達319年,在中國古代歷代王朝中延續時間僅次於兩漢。

太祖在位期間,致力於統一全國。依據宰相趙普的「先南後北」策略,先後滅荊南、湖南、後蜀、南漢及南唐等南方割據政權,至宋太宗在位期間,迫使吳越、清源軍納土歸降,滅北漢,方才完成一統;太祖於961年及969年先後兩次「杯酒釋兵權」,解除禁軍將領及地方藩鎮的兵權,解決自唐朝中葉以來藩鎮割據的局面;設立「封樁庫」貯藏錢帛布匹,期能贖回被後晉高祖石敬瑭獻給契丹的燕雲十六州,但事未成而逝世。

976年11月14日,太祖逝世,葬于永昌陵,得年50歲,在位17年。由其胞弟趙光義繼承帝位,是為宋太宗,並於同年隨即改元。由於北宋中期的筆記《續湘山野錄》記載了「燭影斧聲」事件,暗示趙匡胤是由太宗所加害。加上太祖死後,帝位非由其子繼承,而是由弟登基垂統,違反中國長子繼承的傳統,雖然太宗即位後延续了太祖許多的執政措施,但欲蓋彌彰,与太祖有关的皇室成員亦相繼離奇亡故。此后的皇帝亦由太宗的子孫继承,直至宋孝宗才回归太祖一脈,使太祖的死因並不單純,成為千古之謎。

唐明宗天成二年二月十六日(927年3月21日),趙匡胤誕生於洛陽夾馬營(今河南省洛陽市瀍河回族區東關爽明街北段),是趙弘殷次子,母親杜氏。依《宋史·太祖本紀》記載,趙匡胤為“涿郡人”,然而此處應是依涿郡趙氏的郡望所指,實則屬籍保州(今河北省保定市清苑区)。年長後離家遊歷,投奔復州防禦使王彥超,不被接納。繼而往依隨州刺史董宗本,卻遭其子董遵誨憑勢欺侮,趙匡胤於是辭別北上。948年,趙匡胤投奔後漢樞密使郭威帳下,隨軍征討李守貞。951年,郭威稱帝,國號「周」,是為「周太祖,趙匡胤得補任東西班行首,加拜滑州(今河南省安陽市)副指揮使。953年,郭威養子柴榮擔任開封府尹,調趙匡胤至京師(今河南省開封市)任開封府馬直軍使。

954年初,周太祖逝世,外戚柴榮繼位為周世宗,趙匡胤調任中央禁軍。北漢及契丹隨即趁喪入侵,邊報頻仍,世宗決定親征,趙匡胤任侍衛將領隨行護駕。4月,周軍與北漢軍大戰於高平(今山西省晉城市高平市),趙匡胤率二千人參與戰鬥,並身先士卒,親自衝鋒,北漢軍披靡,七千餘人投降。時正颳南風,周軍因風奮擊,北漢軍大敗,周軍趁勝圍攻河東城,焚燒城門,趙匡胤於戰鬥中遭流矢射中左臂。高平之戰後,世宗拔擢趙匡胤為殿前司都虞候,領嚴州(今浙江省湖州市德清縣)刺史,並命趙匡胤整頓禁軍,汰羸除弱,更招募天下壯士至京師,設立殿前諸班,由趙匡胤選擇精銳將士充之,後周軍隊自始獨霸。

956年初,周世宗親征南唐,大軍圍攻壽城(今安徽省六安市壽縣),南唐駐紮一萬兵馬於塗山(今安徽省蚌埠市)下,停泊舟艦於淮河,世宗命趙匡胤往攻。3月,趙匡胤伏軍渦口(今安徽省蚌埠市懷遠縣東北),派遣一百餘名騎兵襲擊南唐軍營,交戰後佯敗逃遁,南唐軍追擊,至渦口而後周伏兵四起,南唐軍大敗,斬唐將何延錫,奪得五十餘艘戰艦。南唐另有十五萬援軍駐紮清流關(今安徽省滁州市),趙匡胤受命征討,攻克清流關、滁州城,並生擒唐將皇甫暉。5月,南唐軍攻揚州,趙匡胤屯兵六合支援。揚州守將韓令坤懼而欲退,趙匡胤下令軍中:「揚州士兵過六合者,砍斷雙腿!」韓令坤才決心固守。南唐又遣齊王李景達率兵支援,距離六合二十里處駐紮。諸將請主動攻擊,趙匡胤卻以逸待勞,俟南唐軍發動攻擊,方才領兵迎擊,南唐軍大敗,斬殺五千餘人,投江溺斃的不計其數。戰後趙匡胤因功受封殿前都指揮使,不久又加封定國軍節度使。

957年初,世宗再度親征南唐,4月,駐紮紫金山,命趙匡胤率領禁軍殲滅壽州外圍南唐援軍,趙匡胤連拔數寨,斬獲三千餘人,並切斷南唐援軍通道,壽州因此無援,於是投降。還京後,趙匡胤因功加拜義成軍節度、檢校太保,仍擔任殿前司都指揮使。年底,柴榮三征南唐,進攻濠州,城外有水灘環繞,南唐軍在其上設立柵欄防守,世宗遂命將士騎駱駝渡河,趙匡胤則領兵騎馬而渡,率先攻破南唐水寨,焚燒南唐戰艦七十餘艘,斬殺兩千餘人,攻陷濠州。其後陸續攻陷泗州(今安徽省宿州市泗縣)、楚州(今江蘇省淮安市淮陰區),南唐江北之地盡為周有。隔年5月,趙匡胤改封忠武軍節度使。

959年3月,世宗北征契丹,趙匡胤任水陸都部署,先期抵達瓦橋關(今河北省保定市雄縣),守將姚內斌投降。6月,世宗染病不適,命車駕回京。7月23日,世宗任命趙匡胤為殿前司都點檢,為殿前禁軍最高統帥。27日,世宗因病駕崩,年僅七歲的梁王柴宗訓繼位為周恭帝,改歸德軍節度使,檢校太尉。

960年正月初一(1月31日),北方邊境鎮州(今河北省石家莊市正定縣)、定州(今河北省定州市)守將急報,稱北漢與契丹合勢,聯兵入侵,周恭帝命趙匡胤率宿衛禁軍北上迎敵。

初二(2月1日),京城流言四起,謠傳大軍出征之日將策立都點檢做天子,士民不安,準備逃匿。

初三(2月2日),趙匡胤率大軍自京城東北邊的愛景門出城,士卒紀律嚴明,民心稍安。大軍當晚駐紮於京城東北二十公里的陳橋驛(今河南省封丘縣陳橋鎮),將士們謀劃擁立趙匡胤為天子,由都押衙李處耘轉告供奉官都知趙光義,兩人隨即與歸德軍節度掌書記趙普商討。談論中,諸將突然闖入,眾說紛紜,趙普及趙光義以理勸退,諸將稍稍引去,不久又復趨集,露白刃要脅,趙普於是同意,派人前往京城安排內應,將士們則環列於趙匡胤的大帳,等待天明。

初四(2月3日)清晨,趙匡胤因酒醉尚未清醒,將士於大營四周鼓譟喧嘩,趙普及趙光義入帳稟告趙匡胤,趙匡胤驚起,披衣而出,將領們持兵器羅立於庭。趙匡胤不及說話,諸將士即將黃袍披在趙匡胤身上,紛紛下拜高呼萬歲。趙匡胤堅拒,眾將士不聽,迫其上馬南行回京。趙匡胤見勢不可免,便攬轡對諸將說:「你們貪圖富貴,立我為天子,就必須聽從我的命令,不然,我不當這個皇帝。」眾將皆下馬跪地說:「唯命是從!」趙匡胤便下令不得侵擾後周皇帝、太后及群臣,也不得擅自擄掠及搜刮府庫,違者族誅,眾將應諾。趙匡胤率大軍自仁和門入京城,下令將士解甲歸營,趙匡胤則回殿前都點檢公署,脫下黃袍。不久,眾將逼擁司徒,同中書門下平章事,參知樞密院事范質、參知樞密院事,尚書省右僕射王溥、同中書門下平章事,刑部尚書魏仁浦至公署,迫其表態。范質等無可奈何,只得下拜,高呼萬歲。趙匡胤於是登崇元殿行禪讓禮,登基稱帝,建國號「大宋」,改元建隆,大赦天下。

上述記載見於《宋史》、《續資治通鑑長編》、《涑水記聞》等史書及筆記中,皆言趙匡胤是為眾所逼,被迫稱帝,事前並不知情。然而現代史家依史料記載之疑點及矛盾推論,普遍認為「陳橋兵變」是趙匡胤及其親信幕僚所預謀策劃的軍事政變。[需要更好来源]

960年5月,昭義節度使李筠據潞州(今山西省長治市)叛變,攻陷澤州(今山西省晉城市),並與北漢合兵,率眾南下。太祖遣侍衛親軍司馬步軍副都指揮使石守信、殿前司副都點檢高懷德往討,又派昭化軍節度使慕容延釗、彰德軍節度使王全斌前往合兵,於長平之戰大敗李筠,斬殺三千餘人。6月,太祖決定親征,與石守信等將領會合,於澤州之南大敗李筠及北漢的三萬聯軍,三千餘名北漢援軍投降,宋軍將其全數坑殺。李筠退守澤州城,宋軍隨即攻破澤州,李筠自焚而死。7月,宋軍兵圍潞州,李筠長子李守節舉城出降;同年(960年)10月,淮南節度使李重進據揚州叛變,太祖遣石守信、義成軍節度使王審琦、宣徽北院使李處耘、保信節度使宋偓率軍往討。11月,太祖再度親征,兵至大儀鎮(今江蘇省揚州市儀徵市),石守信遣使奏請太祖親臨觀看揚州城破,太祖隨即趕赴揚州城下,頃刻城破,李重進自焚而死,黨羽親信數百人皆搜捕斬殺。

964年,太祖下詔令州、郡所收稅賦,除地方日常行政經費外,其餘上繳中央,不得私留;置轉運使,掌管地方財政權,並檢查賦稅情形,以供上繳朝廷及地方支用。轉運使設通判官,到任時核對帳簿,並得查考民情、官吏違法情事上報朝廷,有審計、監察之權,以此削弱地方財政權;太祖並下詔全國之茶、酒、鹽由國家專賣,官吏與百姓不得私自販售,最重處死,國家因此收入大增;太祖派兵滅亡後蜀後,為儲備錢財以應急之用,於宮中設置「封樁庫」,中央政府年度財政盈餘全數納入,並打算儲積至三、五十萬後,以這筆錢贖回遭后晉高祖石敬瑭獻給契丹的燕雲十六州,但不久即逝世,贖地一事無疾而終。

太祖於平定湖南後,下令於其地取消「支郡」,使原屬藩鎮節度使管轄之州、郡獨立,直屬中央,至宋太宗在位時,全國「支郡」全部廢除;有鑑於唐至五代的藩鎮之患,太祖以朝廷文臣出守地方,稱「權知軍州事」,執行州、郡之軍事權及行政權,並置「通判」為其副官,地方的民政、財政、司法等事務由知州及通判共同簽署始得施行,通判並得監察主官的不法及瀆職情事,上報朝廷,以此分割守臣之權。

962年4月,太祖下詔各地死刑案件須上報中央,由刑部複審,以杜絕藩鎮枉法殺人的惡習;963年1月,太祖下令每縣設置一「縣尉」,負責地方治安,剝奪原由鎮將任命親信任職之權,以此制衡鎮將,使其勢力僅限所駐城郭而已;973年8月,太祖下詔改各州「馬步院」為「司寇院」,設司寇參軍,選派新科及第進士與選人資序相當者擔任,剝奪藩鎮對地方一般案件的審判權,解決藩鎮武將審理案件時有法不遵的現象。

961年7月,太祖與石守信、王審琦等禁軍將領宴飲,酒過數巡後,對他們說:「我如果沒有你們,就沒有今日的地位,所以對你們的恩德無日或忘。但當天子太過艱難,還不如做節度使來得快樂,因此我每夜都睡不安穩。」石守信等人問其故,太祖答說:「道理很簡單,皇帝這個位置,誰不願意坐呢?」眾將聽後十分惶恐,皆跪地磕頭說:「陛下何出此言?如今天命已定,誰還敢有二心?」太祖說:「這話不對!你們雖然沒有異心,但如果你們屬下貪圖富貴呢?一旦將黃袍加在你們身上,你們即使不想當皇帝,也不行了。」眾將聽後皆涕泣磕頭說:「臣等愚昧無知,沒想到這些,請陛下可憐我們,指示一條生路。」太祖說:「人生如同白駒過隙,晃眼即逝,所謂追求富貴之人,不過想多累積些金銀財寶,盡情享樂,使子孫不再貧乏而已。你們何不放棄兵權,出守藩鎮,買幾塊好地、幾間好房,為子孫留下永久的產業;多收些歌兒舞女,每日與她們飲酒取樂,以終天年。我再與你們約定聯姻,君臣之間,不相互猜疑,上下相安無事,這不是件好事嗎!」眾將皆下拜說:「陛下為臣等設想周到,是我們的再生父母。」隔日,石守信、高懷德、王審琦、張令鐸等將領皆稱病請辭禁軍官職,太祖隨即批准,使其出鎮地方為節度使,所遺職缺不再補實。

969年12月,太祖在御花園與進京述職的地方藩臣宴飲,酒酣之際,從容說道:「你們都是國家的元勳宿將,長久在藩鎮做官,公務繁忙,這不是朕優禮賢士的本意。」鳳翔軍節度使王彥超上前奏道:「臣本來就無功勞勳績,卻久受皇恩榮寵,十分慚愧。如今臣已衰老,乞求陛下賜臣退休,歸老園田,這是臣最後的願望。」安遠軍節度使武行德、護國軍節度使郭從義、定國軍節度使白重贊、保大軍節度使楊廷璋等將領卻仍競相自陳往昔攻戰之功勞及經歷之艱辛,太祖便說:「這是前代的事了,還有什麼好說的。」隔日,太祖下詔,免去其節度使職,授以「環衛官」虛銜,留任京師,改以朝臣出守諸郡,徹底避免自唐末、五代以來的藩鎮割據問題。

太祖鑑於五代時期藩鎮武將權力過重,以致國家混亂,建國後採取「重文抑武」政策,凡國家高階實權職位均由文官擔任,貶抑武官,以防籓鎮專權。但有史家認為此政策使宋朝積貧積弱、軍力不振。

太祖初即帝位,便與宰相趙普「雪夜定策」,決定「先南後北」統一全國的順序。

荊南:962年10月,武平(湖南)節度使周行逢病逝,傳位予十一歲的兒子周保權,衡州刺史張文表不服叛變,攻陷潭州(今湖南省長沙市),周保權派楊師璠往討,並求援於荊南及宋。963年2月3日,太祖趁機派遣山南東道節度使慕容延釗、樞密副使李處耘出兵湖南,討伐張文表,同時借道荊南。3月,楊師璠擊敗張文表,將其斬首。而荊南節度使高繼沖質疑宋軍借道意圖,便派人以犒師為名前往宋軍大營探查虛實。李處耘對使人熱情款待,卻暗中派數千騎兵急馳江陵,趁高繼沖出迎時佔據江陵城,高繼沖懼而投降,荊南割據政權滅亡。

湖南:張文表之亂雖平,宋軍仍繼續南下,武平節度使周保權派兵防禦。宋軍隨即於三江口(今湖南省岳陽市)大敗周保權軍,攻陷岳州,獲戰船七百餘艘,斬殺四千餘人。4月,於澧州(今湖南省常德市澧縣)以南擊潰周保權部屬張從富,都城朗州(今湖南省常德市)大懼,焚燒城池,居民逃亡山谷。4月6日,宋軍攻入朗州,擒斬張從富,於寺院中俘獲周保權,湖南割據政權滅亡。

後蜀:964年12月8日,太祖下詔兵分兩路共六萬大軍討伐後蜀,北路軍命忠武節度使王全斌為主帥、武信節度使崔彥進為副官、樞密副使王仁贍為監官,東路軍以寧江節度使劉光義為副官、樞密承旨曹彬為監官。後蜀皇帝孟昶則遣知樞密院事王昭遠禦敵。965年2月,劉光義於東路擊殺蜀將南光海,兵臨夔州(今重慶市奉節縣東)。後蜀夔州守將高彥儔的部將武守謙違令出戰,大敗而逃,宋軍趁亂入城,高彥儔力戰不敵,自焚而死,夔州淪陷,萬、施、開、忠、遂等五州相繼投降;後蜀軍主帥王昭遠率兵於北路迎戰王全斌,三戰三敗,退守劍門關(今四川省廣元市劍閣縣)。2月4日,王全斌攻入利州(今四川省廣元市利州區),獲軍糧八十萬斛,不久轉往劍門關,大敗後蜀軍,俘虜蜀將王昭遠、趙崇韜,攻陷劍州。孟昶聞之大懼,決定遣使請降。2月11日,孟昶派使者至宋軍營前遞降書,後蜀滅亡。

然而蜀地雖已收復,征蜀大軍卻在主帥王全斌等人的縱容下,任意燒殺劫掠、為非作歹。而王全斌與崔彥進、王仁贍等將領只知日夜飲酒作樂,不理軍務,以至軍紀弛廢,境內盜賊蜂起,蜀民苦不堪言。王全斌甚至剋扣太祖下令給投降蜀軍前往京城的路費,並多方騷擾,遂激成叛變,十萬叛軍推舉文州刺史全師雄為帥,攻陷彭州(今四川省成都市彭州市),成都十縣及邛、蜀、眉、陵等十七州響應叛亂,四川大亂,成都與汴梁斷絕聞訊。直至967年初,蜀地之亂經歷兩年鎮壓後方才平定。

北漢:968年8月23日,北漢皇帝劉鈞逝世,養子劉繼恩繼位。9月10日,太祖命客省使盧懷忠等二十二人領兵屯駐潞州,準備趁喪攻伐北漢,兩天後,命義軍節度使李繼勳領兵進入漢境,於洞過河擊敗北漢軍,斬殺二千餘人,獲戰馬五百匹,進圍北漢都城太原(今山西省太原市)。同時,北漢皇帝劉繼恩因與權臣郭無為政爭失敗而遭弒殺,劉繼恩胞弟劉繼元繼立皇帝,立即上表請求契丹出兵援救。11月,契丹援軍抵達,宋軍撤退,北漢軍趁機侵入宋境,擄掠居民而回。首次討伐北漢失利。

969年2月26日,太祖命宣徽南院使曹彬、侍衛步軍都指揮使党進等將領兵征伐北漢。3月1日,太祖下詔親征。3月7日,御駕自京城出發,大軍於十一天後到達潞州,因雨停駐。漢將劉繼業、馮進珂屯兵於團柏谷,遣偵騎往來巡邏,遭宋軍前鋒部隊擊敗,劉繼業等退回太原,宋軍遂包圍太原城。4月4日,潞州雨停,太祖率軍出發,六天後,抵達太原城下,下令於太原城外築長城牆,藉以圍困城池,絕其外援;又下令堵塞汾水,使之壅積,並於隔日決堤,水灌太原城,洪水從城門灌入城中,北漢派人緊急設置障礙填補,宋軍則頻射弓箭阻撓,使其無法施工,但不久即有成堆的草隨洪水漂流至決口處,使宋軍箭矢無法穿透,北漢便趁此堵住決口。宋軍久攻太原不下,將領多有死傷,加上部隊駐紮於甘草地上,正值盛暑大雨,疫疾橫生,將士多染病腹瀉,太常博士李光贊上奏建議退兵,趙普贊同,太祖便下令退兵。第二次討伐北漢亦失利。

南漢:征北漢失利後,太祖重拾「先南後北」策略。970年10月3日,詔令潭州防禦使潘美、朗州團練使尹崇珂、道州刺史王繼勳等將率兵討伐南漢,圍攻賀州(今廣西壯族自治區賀州市)。南漢皇帝劉鋹遣伍彥柔往援,遭宋軍擊潰,兵敗身死,賀州投降。隔年(971年)1月,宋軍進攻韶州(今廣東省韶關市),擊破漢將李承渥的大象陣,攻陷韶州,並相繼攻克英州(今廣東省清遠市英德市)、雄州(今廣東省韶關市南雄市),南漢韶州刺史辛延渥派人勸劉鋹投降,劉鋹不從,下令準備十多艘船裝載金銀財寶及妃嬪宮女,將出海逃亡,卻遭宦官及衛士盜其船而開走,劉鋹大懼,遣使請降又不獲准,只得堅守。3月3日,宋軍擊斬漢將植廷曉,並火燒位於馬徑(今廣東省佛山市南海區西北)的南漢軍營柵,南漢軍大敗,主帥郭崇岳戰死。劉鋹聽聞兵敗,便縱火焚燒宮殿府庫,成為灰燼。隔日,劉鋹素服出降,南漢滅亡。

南唐:974年10月9日,太祖命宣徽南院使曹彬等將率兵赴荊南,隔日又遣山南東道節度使潘美等將也赴荊南屯駐。11月4日,曹彬等將領兵出發荊南,直往南唐國都金陵(今江蘇省南京市)。太祖同時命吳越王錢俶合擊南唐,策應宋軍。11月21日,曹彬攻陷池州(今安徽省池州市),並陸續攻下銅陵、當塗、蕪湖,進逼采石磯(今安徽省馬鞍山市西)。12月9日,宋軍於采石磯擊敗南唐二萬大軍,俘獲一千餘人、戰馬三百餘匹。太祖隨即下令將先前已製成的浮橋自石牌(今安徽省安慶市懷寧縣石牌鎮)移至采石磯裝纜,三日而成,宋軍因此渡過長江。隔年(975年)3月2日,曹彬率軍圍攻金陵。南唐皇帝李煜下令戒嚴,並數次派遣使者徐鉉、周惟簡前往宋都汴梁請求暫緩進攻,太祖不許,徐鉉便陳述南唐國主無罪,與太祖反覆辯論,太祖大怒說:「不用再說了,我也知道南唐無罪,但天下本歸一家,臥榻之側,怎能容許其他人鼾睡呢!」976年元旦,宋軍攻陷金陵,李煜奉表請降,南唐滅亡。

太祖逝世後,太宗逼迫吳越王錢俶、清源軍節度使(閩南)陳洪進於978年納土歸降,並於隔年(979年)發兵滅亡北漢,宋朝至此統一除燕雲十六州外中國本部。

由元朝宰相脫脫所監修的《宋史‧太祖本紀》對宋太祖趙匡胤有極高評價:「讚曰:昔者堯、舜以禪代,湯、武以征伐,皆南面而有天下。四聖人者往,世道升降,否泰推移。當斯民塗炭之秋,皇天眷求民主,亦惟責其濟斯世而已。使其必得四聖人之才,而後以其行事畀之,則生民平治之期,殆無日也。五季亂極,宋太祖起介胄之中,踐九五之位,原其得國,視晉、漢、周亦豈甚相絕哉?及其發號施令,名藩大將,俯首聽命,四方列國,次第削平,此非人力所易致也。建隆以來,釋藩鎮兵權,繩贓吏重法,以塞濁亂之源。州郡司牧,下至令錄、幕職,躬自引對;務農興學,慎罰薄斂,與世休息,迄於丕平;治定功成,制禮作樂。在位十有七年之間,而三百餘載之基,傳之子孫,世有典則。遂使三代而降,考論聲明文物之治,道德仁義之風,宋於漢、唐,蓋無讓焉。嗚呼,創業垂統之君,規模若是,亦可謂遠也已矣!」——《宋史·本紀第三·太祖本紀三》

明太祖朱元璋於1374年9月親至南京歷代帝王廟祭祀自三皇至元世祖等十七位歷代帝王,並對其各有祝文,其中對宋太祖的祝文云:「惟宋太祖皇帝順天應人,統一海宇,祚延三百,天下文明。有君天下之德而安萬世之功者也。」——《明太祖高皇帝實錄·卷九十二》

北宋開寶九年十月二十日(癸丑,976年11月14日),趙匡胤於皇宮萬歲殿逝世,享年五十歲,在位十六年,予諡「英武聖文神德皇帝」,廟號「太祖」,三弟趙光義繼位,即宋太宗。

977年5月15日,靈柩奉葬永昌陵。1008年9月3日,宋真宗趙恆加諡為「啟運立極英武睿文神德聖功至明大孝皇帝」。

燭影斧聲:依據北宋中期由文瑩和尚所著《續湘山野錄》的記載,趙匡胤發跡前曾與一名道士來往,常相約飲酒至醉。一次醉酒後,道士以吟唱預言趙匡胤將當皇帝,醒後卻推說酒醉胡言。趙匡胤稱帝後兩人再也沒相見。十六年後的開寶九年(976年)上巳節,趙匡胤至西沼行祓禊禮,道士坐於岸邊樹蔭下,對趙匡胤說:「別來無恙。」趙匡胤大喜,即請至後宮飲酒歡續。趙匡胤說:「我想請你預測一事以久,無他事,我還有幾年壽命?」道士說:「只要今年十月二十日夜晚天氣晴朗,就可延續十二年;否則,即當從速安排後事。」當日夜,趙匡胤登太清閣觀象,天氣先晴朗而後轉惡,驟下大雪。趙匡胤急忙下閣傳令開皇宮端門,召晉王趙光義入宮,兄弟二人於內寢對坐飲酒,並屏去所有宦官、宮女。內侍們遙見寢室燭影下,趙光義時而起座離席,露出不可勝之情狀。兩人喝完已是午夜時分,室外積雪已達數寸,趙匡胤拿柱斧戳雪,一邊回頭對趙光義說:「好做,好做!」接著就寬衣就寢,鼾聲如雷。當晚,趙光義留宿宮中。天將五更,寢室周圍寂靜無聲,趙匡胤駕崩,享年四十九歲。趙光義受遺詔於柩前即位,是為宋太宗。

北宋史家司馬光所著《涑水記聞》則記載趙匡胤逝世後,宋皇后急派宦官王繼恩傳召趙匡胤第四子、秦王趙德芳進宮,王繼恩卻逕至趙光義府邸通報趙匡胤死訊,並催其盡速進宮即位。趙光義猶豫不定,王繼恩則說:「事情拖久就被他人搶先了。」於是趙光義趁夜踏雪入宮,進入寢殿。宋皇后聽說王繼恩已歸,便問:「德芳來了嗎?」王繼恩卻說:「晉王來了。」宋皇后見到趙光義,先是驚愕,隨即說:「我們母子的性命,都托付給官家了。」趙光義則涕泣說:「共保富貴,不需擔憂。」

南宋史家李燾所著《續資治通鑑長編》採信上述二說,只將趙匡胤語「好做,好做」改為「好為之」,道士則有姓名曰「張守真」,且言趙光義當晚並無於宮中留宿。李燾於書中引北宋史家蔡惇直的筆記,也有與《續湘山野錄》相似的記載。

依據上述疑點,加上史書中其他記載(如趙匡胤之死已有人先行預料),便有趙匡胤是被趙光義謀殺之說。

金匱之盟:趙光義表示:961年母親杜太后病危,召趙普入宮接受遺命。杜太后問趙匡胤:「你知道你為何能取得天下嗎?」趙匡胤泣不能答。杜太后說:「我是老死,哭也沒用,我正要跟你說大事,怎麼只是一直哭呢?」於是再問一遍。趙匡胤說:「都是祖上和太后積德所致。」杜太后說:「不對。是因為柴家讓孩童當皇帝,人心不服所致。如果後周有年長的君主,你能得到江山嗎?你和趙光義都是我親生的,你死後應將皇位傳給弟弟。天下之大、事務繁重,能立一個年長的君主來治理,這是社稷之福啊。」趙匡胤叩頭涕泣說:「一定遵照母后的教誨。」杜太后便對趙普說:「你將我剛才講的話記下來,不可違背。」趙普即於太后床前寫成誓書,並於末尾寫上「臣普記」三字。趙匡胤將誓書藏於金匱之中,命謹慎的宮人保管。

上述記載如屬實,則趙光義繼任皇位即有合法性及正當性。然而有學者指出諸多疑點:如此誓為真,則何以不在趙光義即位之初公布,而是等到981年10月才由趙普以「密奏」的方式啟奏趙光義,乃開金匱查驗屬實,前後竟隱瞞「太后遺詔」五年之久;記載內容所言皇位須「兄終弟及」是因杜太后擔心趙匡胤諸子皆幼,不足以坐穩江山,但趙匡胤死時,其次子趙德昭二十六歲,四子趙德芳十八歲,皆非幼弱,遺詔前提不復存在;記載來源《太祖實錄》經趙光義多次修改,已非原貌,而修改前的舊版則未有此記載,顯有杜撰之嫌;據《涑水記聞》、《續資治通鑑長編》、《宋史》等書記載,杜太后本意為趙匡胤死後,皇位傳給三弟趙光義,再傳給四弟趙廷美,最終傳回給趙匡胤次子趙德昭。然而修改後的新版《太祖實錄》只有傳位給趙光義的記載,且相關當事人竟於短時間內逐一逝世(979年,趙德昭自殺;981年,趙德芳猝死;984年,趙廷美遭貶,憂悸而死)。史學界即因前述疑點而有「金匱之盟」是趙普為取得趙光義信任、重得相位而杜撰「太后遺詔」的說法。加上趙光義不逾年而改元(開寶九年十二月甲寅,即977年1月14日改「開寶九年」為「太平興國元年」),也有學者據此認為趙光義因弒兄奪位而心虛,不等過年即倉促改元,欲使其繼位成為既定事實。

史載趙匡胤生時,紅光滿室,有香氣整夜不散,嬰兒體呈金色,長達三日。年輕時學騎射,嘗試馴服烈馬不加鞍繩,馬衝上城門斜坡道,致趙匡胤額頭撞擊城門門楣,目擊者皆認為其頭骨必定粉碎,趙匡胤卻從容站起,徒步追上烈馬騰騎而上,毫髮無傷;趙匡胤兒時曾與玩伴韓令坤在土屋裡玩耍,有群麻雀在室外聒噪互鬥,趙匡胤遂與韓令坤出土屋欲捕麻雀,剛出房而土屋隨即崩塌,二人幸免於難。

趙匡胤與胞弟趙光義幼時隨母親杜氏躲避戰亂,杜氏便將兄弟二人放至籮筐擔挑而走,為道士陳摶撞見,便嘆道:「別說當今世上沒有天子,都將天子用擔挑著走。」;多年後趙匡胤稱帝,陳摶聞之大笑,說:「天下從此安定了。」

959年,後周世宗柴榮親征契丹,於征途中批閱文書,發現其中有一囊袋,內有三尺長的木牌,上有字:「點檢作天子」。柴榮見後不悅,漸感身體不適,便命車駕返京。抵京後下詔撤殿前都點檢張永德之職,改任趙匡胤。隔年初,趙匡胤登基為帝,遂應此讖。

宋初,宰相范質等人仍循前代慣例,上朝時設有座椅,坐著奏事。一日早朝,范質猶坐著,趙匡胤便說:「我眼睛昏花,看不清楚,你把文書拿給我看。」范質於是起身持文書進呈,趙匡胤卻已密囑侍者趁此將其座撤去,待范質欲返座而座椅已撤,只得站立。自此宰相與群臣般站著上朝,成為慣例。

趙匡胤稱帝後第三年,秘密遣人鐫刻一通石碑,藏於太廟的夾室內,稱為「誓碑」,用黃金絲所鑲嵌成的布幔遮蓋,門禁森嚴。趙匡胤下令此後四時祭祀及新皇帝即位時,待拜完太廟,便須恭讀誓詞,由一個不識字的小太監持鑰匙開夾室,然後焚香、點亮燭火並將幔揭開,其餘隨臣須於遠方庭中佇候。當朝皇帝於碑前跪拜並默誦誓詞,再拜而出,群臣及近侍們皆不知誓詞為何。北宋历代皇帝皆承襲故例,按時恭讀,不敢洩漏。直到靖康之變爆發,皇宮大亂,太廟夾室門戶洞開,人們才發現內裡有一高約七、八尺,寬四尺餘的石碑,上面有三行誓詞:第一、柴氏子孫有罪,不得加刑,縱犯謀逆,止於獄中賜盡,不得市曹刑戮,亦不得連坐支屬;第二、不得殺士大夫及上書言事人;第三、子孫有渝此誓者,天必殛之。上述三誓史稱「太祖誓約」。南宋皇帝則是由曹勛自金國南歸時,向宋高宗轉達宋徽宗之語方才得知誓詞。


\subsection{建隆}


\begin{longtable}{|>{\centering\scriptsize}m{2em}|>{\centering\scriptsize}m{1.3em}|>{\centering}m{8.8em}|}
  % \caption{秦王政}\
  \toprule
  \SimHei \normalsize 年数 & \SimHei \scriptsize 公元 & \SimHei 大事件 \tabularnewline
  % \midrule
  \endfirsthead
  \toprule
  \SimHei \normalsize 年数 & \SimHei \scriptsize 公元 & \SimHei 大事件 \tabularnewline
  \midrule
  \endhead
  \midrule
  元年 & 960 & \tabularnewline\hline
  二年 & 961 & \tabularnewline\hline
  三年 & 962 & \tabularnewline\hline
  四年 & 963 & \tabularnewline
  \bottomrule
\end{longtable}

\subsection{乾德}

\begin{longtable}{|>{\centering\scriptsize}m{2em}|>{\centering\scriptsize}m{1.3em}|>{\centering}m{8.8em}|}
  % \caption{秦王政}\
  \toprule
  \SimHei \normalsize 年数 & \SimHei \scriptsize 公元 & \SimHei 大事件 \tabularnewline
  % \midrule
  \endfirsthead
  \toprule
  \SimHei \normalsize 年数 & \SimHei \scriptsize 公元 & \SimHei 大事件 \tabularnewline
  \midrule
  \endhead
  \midrule
  元年 & 963 & \tabularnewline\hline
  二年 & 964 & \tabularnewline\hline
  三年 & 965 & \tabularnewline\hline
  四年 & 966 & \tabularnewline\hline
  五年 & 967 & \tabularnewline\hline
  六年 & 968 & \tabularnewline
  \bottomrule
\end{longtable}

\subsection{开宝}

\begin{longtable}{|>{\centering\scriptsize}m{2em}|>{\centering\scriptsize}m{1.3em}|>{\centering}m{8.8em}|}
  % \caption{秦王政}\
  \toprule
  \SimHei \normalsize 年数 & \SimHei \scriptsize 公元 & \SimHei 大事件 \tabularnewline
  % \midrule
  \endfirsthead
  \toprule
  \SimHei \normalsize 年数 & \SimHei \scriptsize 公元 & \SimHei 大事件 \tabularnewline
  \midrule
  \endhead
  \midrule
  元年 & 968 & \tabularnewline\hline
  二年 & 969 & \tabularnewline\hline
  三年 & 970 & \tabularnewline\hline
  四年 & 971 & \tabularnewline\hline
  五年 & 972 & \tabularnewline\hline
  六年 & 973 & \tabularnewline\hline
  七年 & 974 & \tabularnewline\hline
  八年 & 975 & \tabularnewline\hline
  九年 & 976 & \tabularnewline
  \bottomrule
\end{longtable}


%%% Local Variables:
%%% mode: latex
%%% TeX-engine: xetex
%%% TeX-master: "../Main"
%%% End:
