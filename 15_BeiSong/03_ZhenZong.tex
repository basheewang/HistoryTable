%% -*- coding: utf-8 -*-
%% Time-stamp: <Chen Wang: 2019-12-26 10:27:44>

\section{真宗\tiny(997-1022)}

\subsection{生平}

宋真宗趙恒(968年12月23日-1022年3月23日),原名趙德昌,又曾名趙元休、趙元侃,北宋的第三位皇帝。他是宋太宗的第三个儿子,登基前曾被封为韩王、襄王和寿王,淳化五年(994年)九月,加檢校太傅行開封府尹,至道三年(997年)四月登基後離任,以太子身份继位,在位25年。

宋真宗是著名諺語「書中自有黃金屋,書中自有顏如玉」的作者。

宋太祖乾德六年十二月初二日(968年12月23日)趙恒生於東京開封府第,是趙光義第三子,與長兄楚王趙元佐同母,初名趙德昌。幼時英睿,姿表特異。與諸王嬉戲時,喜歡作戰陣之狀,自稱元帥。宋太祖喜愛他,將他養在宮中。

趙恆最初並非皇位繼承人,宋太宗最初屬意長子趙元佐為太子,但因趙元佐自秦王廷美亡后精神失常,而且因病傷人及在宮內縱火,最後被廢。太宗本計劃立次子趙元僖為太子,但趙元僖又早逝。趙元僖死後,太宗立三子趙元侃為太子,赐名恆,至道三年(997年)宋太宗箭傷復發而駕崩,趙恆繼位為帝,是為宋真宗。

景德元年(1004年)辽国入侵宋,宋朝大多数大臣建议不抵抗,以宰相寇準为首的少数人极力主张抵抗,最后宋真宗御驾亲征,双方在澶州(今河南濮阳附近)相交,宋胜後,真宗决定就此罢兵,以每年向辽纳白银十万両、绢二十万匹来換取与辽之間的和平,定澶渊之盟。这是宋朝以岁币换取和平的开始。

真宗時,鐵製工具製作進步,土地耕作面積增至5.2億畝(太宗至道二年,耕地有3億多畝),又引入暹罗良种水稻,農作物產量倍增,紡織、染色、造紙、製瓷等手工業、商業蓬勃發展,景德年间,專門製作瓷器(原名白崖场)的昌南镇遂改名为景德镇,貿易盛況空前,史称咸平之治。

宋真宗统治后期,以王钦若和丁谓为宰相,信奉道教和佛教,大中祥符元年(1008年)称受天书,封泰山、祀汾阳,詔令丁谓修建了玉清昭应宫,极侈土木,七年始成,有房屋近三千间,“小不中程,虽金碧已具,必毁而更造,有司不敢计其费。”,封禪给民众造成极大的负担。但亦有认為封禪其實只是為了要震懾強鄰遼國,在軍事屢居下風之際進行一種天命正當性的競爭,但之後發生了北宋帽妖案引發社會動盪,其內在原因史家至今各有主張。

宋真宗好文學,也是一名诗人,他比较著名的诗有《励学篇》、《勸學詩》等。一般人常說的「書中自有黃金屋、書中有女顏如玉」就是出自他的勸學詩。

乾興元年二月十九日(1022年3月23日)于汴京延慶殿駕崩,享年五十五歲,在位共二十五年。群臣為其上諡號為文明章聖元孝皇帝,廟號真宗。十月十三日,葬於永定陵;二十三日附祭太廟。太子宋仁宗继位,史称“仁宗以天书殉葬山陵,呜呼贤哉!”七年後,昭应宫遭雷击,被大火焚為灰烬。《宋史》稱真宗“及澶洲既盟,封禅事作,祥瑞踏臻,天書屢降,導迎奠安,一國君臣如病狂然,吁,可怪也。

元朝官修正史《宋史》脱脱等的評價是:“真宗英悟之主。其初践位,相臣李沆虑其聪明,必多作为,数奏灾异以杜其侈心,盖有所见也。及澶洲既盟,封禅事作,祥瑞沓臻,天书屡降,导迎奠安,一国君臣如病狂然,吁,可怪也。他日修《辽史》,见契丹故俗而后推求宋史之微言焉。宋自太宗幽州之败,恶言兵矣。契丹其主称天,其后称地,一岁祭天不知其几,猎而手接飞雁,鸨自投地,皆称为天赐,祭告而夸耀之。意者宋之诸臣,因知契丹之习,又见其君有厌兵之意,遂进神道设教之言,欲假是以动敌人之听闻,庶几足以潜消其窥觎之志欤?然不思修本以制敌,又效尤焉,计亦末矣。仁宗以天书殉葬山陵,呜呼贤哉!”

宋真宗派曹利用去辽国签订澶渊之盟之际,告诉曹“迫不得已,虽百万亦可!”。寇准知道后,指着曹怒道“超过30万两,提人头来见”。最后,经过曹利用再三讨价还价,以每年白银10万两、绢帛20万匹,订立澶渊之盟。曹利用回到宋朝之后,真宗急问金额多少,曹利用不敢直说,只竖起3根指头,真宗以为是300万两,大惊失声脱口而说,“太多了”,过了一会又安慰道:“金额是太多了,但就此把事情了结也好”,待知道是30万时,如释重负,转忧为喜,对曹利用大加赏赐。

另外,宋真宗亦是著名戲曲《貍貓換太子》第三主角。(另兩位為宋仁宗和李宸妃,相關人物為太監陳琳、郭槐、劉太后、王親貴族八賢王、狄太后及婢女寇珠。)

\subsection{咸平}


\begin{longtable}{|>{\centering\scriptsize}m{2em}|>{\centering\scriptsize}m{1.3em}|>{\centering}m{8.8em}|}
  % \caption{秦王政}\
  \toprule
  \SimHei \normalsize 年数 & \SimHei \scriptsize 公元 & \SimHei 大事件 \tabularnewline
  % \midrule
  \endfirsthead
  \toprule
  \SimHei \normalsize 年数 & \SimHei \scriptsize 公元 & \SimHei 大事件 \tabularnewline
  \midrule
  \endhead
  \midrule
  元年 & 998 & \tabularnewline\hline
  二年 & 999 & \tabularnewline\hline
  三年 & 1000 & \tabularnewline\hline
  四年 & 1001 & \tabularnewline\hline
  五年 & 1002 & \tabularnewline\hline
  六年 & 1003 & \tabularnewline
  \bottomrule
\end{longtable}

\subsection{景德}

\begin{longtable}{|>{\centering\scriptsize}m{2em}|>{\centering\scriptsize}m{1.3em}|>{\centering}m{8.8em}|}
  % \caption{秦王政}\
  \toprule
  \SimHei \normalsize 年数 & \SimHei \scriptsize 公元 & \SimHei 大事件 \tabularnewline
  % \midrule
  \endfirsthead
  \toprule
  \SimHei \normalsize 年数 & \SimHei \scriptsize 公元 & \SimHei 大事件 \tabularnewline
  \midrule
  \endhead
  \midrule
  元年 & 1004 & \tabularnewline\hline
  二年 & 1005 & \tabularnewline\hline
  三年 & 1006 & \tabularnewline\hline
  四年 & 1007 & \tabularnewline
  \bottomrule
\end{longtable}

\subsection{大中祥符}

\begin{longtable}{|>{\centering\scriptsize}m{2em}|>{\centering\scriptsize}m{1.3em}|>{\centering}m{8.8em}|}
  % \caption{秦王政}\
  \toprule
  \SimHei \normalsize 年数 & \SimHei \scriptsize 公元 & \SimHei 大事件 \tabularnewline
  % \midrule
  \endfirsthead
  \toprule
  \SimHei \normalsize 年数 & \SimHei \scriptsize 公元 & \SimHei 大事件 \tabularnewline
  \midrule
  \endhead
  \midrule
  元年 & 1008 & \tabularnewline\hline
  二年 & 1009 & \tabularnewline\hline
  三年 & 1010 & \tabularnewline\hline
  四年 & 1011 & \tabularnewline\hline
  五年 & 1012 & \tabularnewline\hline
  六年 & 1013 & \tabularnewline\hline
  七年 & 1014 & \tabularnewline\hline
  八年 & 1015 & \tabularnewline\hline
  九年 & 1016 & \tabularnewline
  \bottomrule
\end{longtable}

\subsection{天禧}

\begin{longtable}{|>{\centering\scriptsize}m{2em}|>{\centering\scriptsize}m{1.3em}|>{\centering}m{8.8em}|}
  % \caption{秦王政}\
  \toprule
  \SimHei \normalsize 年数 & \SimHei \scriptsize 公元 & \SimHei 大事件 \tabularnewline
  % \midrule
  \endfirsthead
  \toprule
  \SimHei \normalsize 年数 & \SimHei \scriptsize 公元 & \SimHei 大事件 \tabularnewline
  \midrule
  \endhead
  \midrule
  元年 & 1017 & \tabularnewline\hline
  二年 & 1018 & \tabularnewline\hline
  三年 & 1019 & \tabularnewline\hline
  四年 & 1020 & \tabularnewline\hline
  五年 & 1021 & \tabularnewline
  \bottomrule
\end{longtable}

\subsection{乾兴}

\begin{longtable}{|>{\centering\scriptsize}m{2em}|>{\centering\scriptsize}m{1.3em}|>{\centering}m{8.8em}|}
  % \caption{秦王政}\
  \toprule
  \SimHei \normalsize 年数 & \SimHei \scriptsize 公元 & \SimHei 大事件 \tabularnewline
  % \midrule
  \endfirsthead
  \toprule
  \SimHei \normalsize 年数 & \SimHei \scriptsize 公元 & \SimHei 大事件 \tabularnewline
  \midrule
  \endhead
  \midrule
  元年 & 1022 & \tabularnewline
  \bottomrule
\end{longtable}


%%% Local Variables:
%%% mode: latex
%%% TeX-engine: xetex
%%% TeX-master: "../Main"
%%% End:
