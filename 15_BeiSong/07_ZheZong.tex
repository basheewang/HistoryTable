%% -*- coding: utf-8 -*-
%% Time-stamp: <Chen Wang: 2021-11-01 15:54:56>

\section{哲宗赵煦\tiny(1085-1100)}

\subsection{生平}

宋哲宗赵煦(1077年1月4日-1100年2月23日),北宋第七位皇帝(1085年4月1日—1100年2月23日在位),為宋神宗第六子,母亲为钦成皇后朱氏。原名傭,曾封为延安郡王。生于熙宁九年十二月七日(1077年1月4日),神宗病危时立他为太子。元丰八年,神宗駕崩,赵煦登基为皇帝,是为宋哲宗,改元“元祐”。在位15年,得年二十三岁,葬于今天河南巩县的永泰陵。

宋哲宗生于熙宁九年十二月七日。最初名佣,授檢校太尉、天平軍節度使,并封均國公。元豐五年(1081年),遷開府儀同三司、彰武軍節度使,并進封延安郡王。元豐七年三月,逢神宗皇帝於集英殿宴請群臣,延安郡王陪同,因舉止有度,宰相於是恭賀神宗。元豐八年二月,神宗病重,宰相王珪請建儲君,并建議由皇太后暫時垂簾聽政。神宗同意后,當年三月,皇太后在福寧殿垂簾,將延安郡王為神宗祈福的手抄佛經出示給宰相等人。於是,奉制立延安郡王為皇太子。最初,太子宮中常有紅光。此後,紅光更如火光一般。

三月初五日,神宗駕崩,皇太子即皇帝位。次日,大赦天下,并遣使告丧於遼國。初七,尊高太后為太皇太后,向皇后为皇太后,生母德妃朱氏为皇太妃。并由宰相王珪为山陵使,营建神宗陵寢。三月二十一日,由于群臣再三请求,哲宗才正式与太皇太后一同听政。

哲宗登基时,只有9岁,由高太皇太后执政。高太皇太后执政后,任用保守派大臣司马光为宰相,“凡熙宁以来政事弗便者,次第罢之”。司马光上台後,不顧一切盡罷新法(熙宁变法),“举而仰听于太皇太后”。宋哲宗對此感到不满。

元祐八年(1093年),高太皇太后去世,哲宗亲政。哲宗亲政后表明紹述,追贬司马光,並贬谪苏轼、苏辙等舊黨黨人于岭南(今广西、廣東、海南),接着重用革新派如章惇、曾布等,恢复王安石变法中的保甲法、免役法、青苗法等,减轻农民负担,使国势有所起色。次年改元“绍圣”,并停止与西夏谈判,多次出兵讨伐西夏,迫使西夏向宋朝乞和。元符三年正月十二(1100年2月23日)病逝于汴梁(今河南开封)。

以章惇為相,主持恢復熙豐新法,史稱"紹述",北宋國力因而得以恢復發展,更取得對西夏的多次戰略性軍事勝利。

元朝官修正史《宋史》脱脱等的評價是:“哲宗以冲幼践阼,宣仁同政。初年召用马、吕诸贤,罢青苗,复常平,登俊良,辟言路,天下人心,翕然向治。而元祐之政,庶几仁宗。奈何熙、丰旧奸枿去未尽,已而媒蘖复用,卒假绍述之言,务反前政,报复善良,驯致党籍祸兴,君子尽斥,而宋政益敝矣。吁,可惜哉!”

哲宗是北宋较有作为的皇帝。不過在新黨與舊黨之間的黨爭始終未能獲得解決,反而在宋哲宗當政期間激化,多少造成朝廷的動盪。

哲宗親政後,絕大部分支持司馬光的舊黨黨人都被放逐,甚至於貶到嶺南等蠻荒地區;宰相章惇也進行言論控制,設立元祐提制局等單位對於反對新法的言論加以控制,甚至於在宮廷內部興獄。紹聖三年(1096年)章惇以巫蠱詛咒的罪名,要求宋哲宗廢宣仁太后所立的孟皇后,改立劉皇后,連宋哲宗都大嘆:「章惇壞我名節!」

1100年正月,宋哲宗患病,不数日死去,二月十日开工建陵,限五月十日完工。宋哲宗停丧七个月,于八月下葬永泰陵。陵台今日尚有17米高,底边每边长约50米,陵台正北有一段神墙残存,高约4米,是现今宋陵仅存的神墙遗迹。陵前石刻雕像还有11件保存完好,仅缺“象奴”一件(完整的为12件),以“象”的刻制最为生动完美,有“东陵狮子、西陵象”的说法。“西陵”就是指的永泰陵与东边的永裕陵相对。皇后刘氏陪葬在陵台西北,相距不足20米。永泰陵正西四五十米处,有哲宗第四女杨国公主墓。

仅石材一项,石匠就有4600多人,采用的各种石材达3万多块,调用民工、役兵超过1万人,山陵使章淳等官员督工急,民工们不堪虐待,纷纷逃亡,工地上饥饿、病、累而死的日日不断,死者多被弃尸荒野乱石之中。《采石场碑记》记载说:“居山土人皆云,至久积阴晦,常闻山中有若声役之歌者,意其不幸横夭者,沉魂未得解脱逍遥而然乎”?

1130年,刘齐政府盗掘北宋陵寝,陵上建筑被破坏殆尽,陵内洗劫一空。永泰陵的水晶注子卖到杭州,1148年南宋太常寺少卿方庭硕出使金朝,到宋陵,见各陵均被掘开,宋哲宗尸骨掷在永泰陵外,方庭硕脱下身上的袍服,将赵煦的尸骨包裹起来,重新置放陵中。后人曾有诗记述此事道:“先帝侍臣空洒泪,泰陵春望已模糊”。元朝初年,此陵再次遭劫。

\subsection{元祐}


\begin{longtable}{|>{\centering\scriptsize}m{2em}|>{\centering\scriptsize}m{1.3em}|>{\centering}m{8.8em}|}
  % \caption{秦王政}\
  \toprule
  \SimHei \normalsize 年数 & \SimHei \scriptsize 公元 & \SimHei 大事件 \tabularnewline
  % \midrule
  \endfirsthead
  \toprule
  \SimHei \normalsize 年数 & \SimHei \scriptsize 公元 & \SimHei 大事件 \tabularnewline
  \midrule
  \endhead
  \midrule
  元年 & 1086 & \tabularnewline\hline
  二年 & 1087 & \tabularnewline\hline
  三年 & 1088 & \tabularnewline\hline
  四年 & 1089 & \tabularnewline\hline
  五年 & 1090 & \tabularnewline\hline
  六年 & 1091 & \tabularnewline\hline
  七年 & 1092 & \tabularnewline\hline
  八年 & 1093 & \tabularnewline\hline
  九年 & 1094 & \tabularnewline
  \bottomrule
\end{longtable}

\subsection{绍圣}

\begin{longtable}{|>{\centering\scriptsize}m{2em}|>{\centering\scriptsize}m{1.3em}|>{\centering}m{8.8em}|}
  % \caption{秦王政}\
  \toprule
  \SimHei \normalsize 年数 & \SimHei \scriptsize 公元 & \SimHei 大事件 \tabularnewline
  % \midrule
  \endfirsthead
  \toprule
  \SimHei \normalsize 年数 & \SimHei \scriptsize 公元 & \SimHei 大事件 \tabularnewline
  \midrule
  \endhead
  \midrule
  元年 & 1094 & \tabularnewline\hline
  二年 & 1095 & \tabularnewline\hline
  三年 & 1096 & \tabularnewline\hline
  四年 & 1097 & \tabularnewline\hline
  五年 & 1098 & \tabularnewline
  \bottomrule
\end{longtable}

\subsection{元符}

\begin{longtable}{|>{\centering\scriptsize}m{2em}|>{\centering\scriptsize}m{1.3em}|>{\centering}m{8.8em}|}
  % \caption{秦王政}\
  \toprule
  \SimHei \normalsize 年数 & \SimHei \scriptsize 公元 & \SimHei 大事件 \tabularnewline
  % \midrule
  \endfirsthead
  \toprule
  \SimHei \normalsize 年数 & \SimHei \scriptsize 公元 & \SimHei 大事件 \tabularnewline
  \midrule
  \endhead
  \midrule
  元年 & 1098 & \tabularnewline\hline
  二年 & 1099 & \tabularnewline\hline
  三年 & 1100 & \tabularnewline
  \bottomrule
\end{longtable}



%%% Local Variables:
%%% mode: latex
%%% TeX-engine: xetex
%%% TeX-master: "../Main"
%%% End:
