%% -*- coding: utf-8 -*-
%% Time-stamp: <Chen Wang: 2019-12-26 10:31:12>

\section{神宗\tiny(1067-1085)}

\subsection{生平}

宋神宗赵顼(1048年5月25日-1085年4月1日),本名趙仲鍼,宋英宗的长子,北宋第六代皇帝,1067年1月25日-1085年4月1日在位。

宋仁宗慶曆八年(1048年)四月十日,出生在濮安懿王宮邸睦親宅。宋英宗趙曙和宣仁聖烈皇后高氏所生長子。八月仁宗賜名為仲針,後被授為率府副率,後三次升遷至右千牛衛將軍。

嘉佑八年(1063年),趙頊與其父趙曙一起入居慶甯宮。三月二九日仁宗趙禎逝世,趙頊父趙曙即位,授趙頊為安州觀察使,封光國公。五月趙頊受經於東宮。宋英宗趙曙見了他好學到廢寢忘食,常派遣內侍去勸他休息。侍講王陶進入宮內,趙頊率弟弟趙顥向他參拜,可見對師傅的尊重。九月加封忠武軍節度使、同中書門下平章事,封淮陽郡王改名為趙頊。

治平元年(1064年),進封潁王。治平三年(1066年)三月納故相向敏中的孫女為夫人。十月英宗病重,趙頊按照宋仁宗時舊制,請求兩日一到邇英閣講讀,以安朝廷百官之心。十二月趙頊被立為皇太子。趙頊太子時喜讀《韓非子》,對法家「富國強兵」之術感興趣;還讀過王安石的《上仁宗皇帝言事書》,讚賞王安石的理財治國思想。

熙寧變法與新舊黨爭:神宗即位後,對北宋積貧積弱深感憂心,而他素來都欣賞王安石的才幹,故即位後命王安石推行變法,振興北宋王朝,是為王安石变法,又稱熙寧變法。

在王安石的主持下,均输、青苗、农田水利、免役、市易、保甲、方田均税、保马等新法相继出籠。新法几乎涵盖社会的各个方面,惟操之過急,利弊互見。北宋学者陆佃说:“造元丰间,积票塞上,盖数千万石,而四方常平之钱,不可胜计。”當時垦田面积大幅度增加,全国高达7億畝,城镇商品经济取得了空前发展。但是变法受到守旧派激烈的反對,朝中的司马光、范镇、赵瞻纷纷上书陈述对新法的不滿,司马光与吕惠卿为了青苗法在皇帝面前争辩,新法維持了将近二十年,直到司馬光盡罷新法為止。此一時期,面临朝廷和后宫的双重阻力,神宗受到的打擊可想而知,高太后更是对神宗说:“王安石是在变乱天下呀!”岐王赵颢也从旁劝说神宗应该遵从皇太后的懿旨,神宗心烦意乱,怒斥歧王說:「那你來當皇帝好了。」岐王诚惶诚恐,失声痛哭。

烏臺詩案:發生於宋神宗元豐二年(1079年),蘇軾於當年移知湖州,到任後上表謝恩,朝臣以其上表中用語,暗藏譏刺。御史何正臣上表彈劾蘇軾,指其「愚弄朝廷,妄自尊大」,又以蘇軾動輒歸咎新法,要求朝廷明正刑賞。御史李定曾因不服母孝,受蘇軾譏諷,於此案中也指蘇軾有「悛終不悔,其惡已著」、「傲悖之語,日聞中外」、「言偽而辯,行偽而堅」、「怨己不用」等四大可廢之罪。

御史舒亶尋摘蘇軾詩句,指其心懷不軌,譏諷神宗青苗法、助役法、明法科、興水利、鹽禁等政策。神宗下令拘捕,太常博士皇甫遵奉令前往逮人。蘇轍時在商丘已預知消息,託王適協助安置蘇軾家屬,並上書神宗陳情,願以官職贖兄長之罪。 蘇軾在9月被捕後,寫信給蘇轍交代身後之事,長子蘇邁則隨途照顧。押解至太湖,蘇軾曾意圖自盡,幾經掙扎,終未成舉。捕至御史臺獄下,御史臺依平日書信詩文往來,構陷牽連七十餘人。

後因太皇太后曹氏、王安禮等人出面力挽,前宰相王安石也說:「豈有聖世而殺才士者乎?」蘇軾終免一死,貶謫為「檢校尚書水部員外郎黃州團練副使本州安置」,前往黃州。蘇轍被貶江西筠州任酒監,平日與蘇軾往來者,如曾鞏、李清臣、張方平、黃庭堅、范鎮、司馬光等29人亦遭處分。張方平、司馬光和范鎮罰紅銅三十斤,其餘各罰紅銅二十斤。烏臺詩案於十二月結束。

時值夏惠宗在位,母黨梁氏專權,西夏國勢日非,圖一舉殲滅羌夏。王韶在庆州(今甘肃庆阳)大破夏军,占领西夏二千里土地。不过后来在永乐城之战中惨败,“厥后兵不敢用于北,而稍试于西,灵武之役,丧师覆将,涂炭百万。帝中夜得报,起,环榻行,彻旦不寐。”灭夏之举未能实现。事後,宋神宗在朝中當眾痛哭。他有抱負,勵精圖治,想滅西羌,惜壯志未酬,抱憾而歿。其子宋哲宗親政後,竭盡所能完成父親遺志。

元丰八年三月初五日(1085年4月1日),宋神宗在福宁殿去世,享年38岁,殡于殿西阶,庙号神宗,群臣上谥号为「英文烈武圣孝皇帝」,十月二十四日,葬于永裕陵。绍圣二年(1095年)九月,加谥为「绍天法古运德建功英文烈武钦仁圣孝皇帝」。崇宁三年(1104年)十一月,改谥为「体天显道帝德王功英文烈武钦仁圣孝皇帝」。政和三年(1113年)十一月,加谥为「体元显道法古立宪帝德王功英文烈武钦仁圣孝皇帝」。

元朝官修正史《宋史》脱脱等的評價是:“帝天性孝友,其入事两宫,必侍立终日,虽寒暑不变。尝与岐、嘉二王读书东宫,侍讲王陶讲谕经史,辄相率拜之,由是中外翕然称贤。其即位也,小心谦抑,敬畏辅相,求直言,察民隐,恤孤独,养耆老,振匮乏。不治宫室,不事游幸,历精图治,将大有为。未几,王安石入相。安石为人,悻悻自信,知祖宗志吞幽蓟、灵武,而数败兵,帝奋然将雪数世之耻,未有所当,遂以偏见曲学起而乘之。青苗、保甲、均输、市易、水利之法既立,而天下汹汹骚动,恸哭流涕者接踵而至。帝终不觉悟,方断然废逐元老,摈斥谏士,行之不疑。卒致祖宗之良法美意,变坏几尽。自是邪佞日进,人心日离,祸乱日起。惜哉!”

重用名將王韶,發動『熙河之役』,率軍擊潰羌人和西夏的軍隊,置熙州(今甘肅臨洮),收復河、洮、岷、宕、亹五州,對西夏形成包圍的之勢。熙宁五年(1072年)王韶收復今臨洮與臨夏,升镇洮军为熙州,設熙河路。王韶为龙图阁待制、熙河路馬步軍都总管、经略安抚使兼知熙州。熙宁六年(1073年)、夏天率兵攻占武胜城(今甘肃临洮),乘胜追擊,进攻河州(今甘肃东乡西南),直捣定羌城 (今甘肃广河)。熙寧七年,收回被吐蕃侵略的二十萬平方公里故土,以功迁礼部司郎中、枢密院直学士,史稱:「宋幾振矣!」。熙宁八年(1075年)宋廷在熙州(今甘肃临洮)、河州(今甘肃临夏)、洮州(今甘肃临潭)、岷州(治所今甘肃西和)、永宁寨(今甘肃甘谷)等地设州、买马,进行民族贸易,此舉受到了邊境各族的热烈欢迎,史載“熙河人情甚喜”。朝廷以王韶为经略安抚使,“通远军自置市易司以来,收本息钱五十七万余缗”,吐蕃政权逐渐瓦解,被稱為『熙河開邊』。

明末清初思想家王夫之《宋論》稱“宋政之乱,自神宗始”。

烏臺詩案是宋神宗執政生涯中,最大的政治汙點。因蘇軾當年移知湖州,到任後上表謝恩,卻被宰相王珪等朝臣認為有貶新政與朝廷之意,御史李定和舒亶尋摘蘇軾詩句,指其心懷不軌,譏諷神宗青苗法、助役法、明法科、興水利、鹽禁等政策。神宗下令拘捕,太常博士皇甫遵奉令前往逮人。捕至御史臺獄下,御史臺依平日書信詩文往來,構陷牽連七十餘人,引起朝野不小震動。在當時蘇軾已成為繼歐陽脩之後的宋朝文壇泰斗,卻因為一句詩下獄,讓時人議論紛紛,而宰相王珪等朝臣又想置蘇軾於死,讓時人對朝局頗為不滿,其中神宗祖母太皇太后曹氏對此事有所發言,認為蘇軾是先帝宋仁宗所選定的太平宰相。而曾任神宗宰相的大臣王安石則向神宗上書,認為:「豈有聖世而殺才士者乎?」神宗迫於壓力,在查清蘇軾並未有詆毀朝廷之意,將其貶至黃州,但已為神宗的執政留下汙點。

永樂城之戰是宋神宗執政以來,最大規模的征伐行動,也是神宗執政以來,最大的政治挫折。在當時因西夏外戚梁太后與其弟梁乙埋姐弟當權,國勢衰落,政治腐敗,西夏舉國上下怨聲載道,民不聊生。梁太后多次出兵攻宋,想提高國內政治威望,卻都慘敗而歸。神宗認為西夏無理,下令攻滅西夏。宋軍於元豐四年(1081年)11月在慶州(今甘肅慶陽)擊潰夏軍,占領西夏兩千多里土地。神宗大喜,命給事中徐禧、鄜延道總管种諤於元豐五年(1082年)9月帶兵攻夏,準備一舉滅夏。徐禧為了建戰功,拒聽种諤的建議(种諤反對興建永樂城),築永樂城,與种諤發生內鬨。由於徐禧好大喜功,聽不進諫言,導致宋軍內鬨,永樂城之戰大敗,宋軍損失二十萬軍士,徐禧、李舜舉、李稷、高永能等人死難,被宋人批評為「永樂之恥」。經過永樂城之戰的戰敗,宋神宗開始悔悟,而不再輕言開戰。

宋神宗赵顼被《时代》杂志认为是有史以来第三富有的人,也是最有权势的人物之一。尽管他在位仅18年,去世时才30多岁,但积累的财富不容小觑。在很强的科技创新和税收能力的帮助下,宋神宗在位期间的北宋国内生产总值占到全世界的25\%至30\%。

\subsection{熙宁}


\begin{longtable}{|>{\centering\scriptsize}m{2em}|>{\centering\scriptsize}m{1.3em}|>{\centering}m{8.8em}|}
  % \caption{秦王政}\
  \toprule
  \SimHei \normalsize 年数 & \SimHei \scriptsize 公元 & \SimHei 大事件 \tabularnewline
  % \midrule
  \endfirsthead
  \toprule
  \SimHei \normalsize 年数 & \SimHei \scriptsize 公元 & \SimHei 大事件 \tabularnewline
  \midrule
  \endhead
  \midrule
  元年 & 1068 & \tabularnewline\hline
  二年 & 1069 & \tabularnewline\hline
  三年 & 1070 & \tabularnewline\hline
  四年 & 1071 & \tabularnewline\hline
  五年 & 1072 & \tabularnewline\hline
  六年 & 1073 & \tabularnewline\hline
  七年 & 1074 & \tabularnewline\hline
  八年 & 1075 & \tabularnewline\hline
  九年 & 1076 & \tabularnewline\hline
  十年 & 1077 & \tabularnewline
  \bottomrule
\end{longtable}

\subsection{元丰}

\begin{longtable}{|>{\centering\scriptsize}m{2em}|>{\centering\scriptsize}m{1.3em}|>{\centering}m{8.8em}|}
  % \caption{秦王政}\
  \toprule
  \SimHei \normalsize 年数 & \SimHei \scriptsize 公元 & \SimHei 大事件 \tabularnewline
  % \midrule
  \endfirsthead
  \toprule
  \SimHei \normalsize 年数 & \SimHei \scriptsize 公元 & \SimHei 大事件 \tabularnewline
  \midrule
  \endhead
  \midrule
  元年 & 1078 & \tabularnewline\hline
  二年 & 1079 & \tabularnewline\hline
  三年 & 1080 & \tabularnewline\hline
  四年 & 1081 & \tabularnewline\hline
  五年 & 1082 & \tabularnewline\hline
  六年 & 1083 & \tabularnewline\hline
  七年 & 1084 & \tabularnewline\hline
  八年 & 1085 & \tabularnewline
  \bottomrule
\end{longtable}



%%% Local Variables:
%%% mode: latex
%%% TeX-engine: xetex
%%% TeX-master: "../Main"
%%% End:
