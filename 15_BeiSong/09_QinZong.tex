%% -*- coding: utf-8 -*-
%% Time-stamp: <Chen Wang: 2021-11-01 15:55:04>

\section{钦宗赵桓\tiny(1126-1127)}

\subsection{生平}

宋钦宗赵桓(1100年5月23日-1161年6月14日),北宋第九位皇帝(1126年1月19日-1127年3月20日在位)。宋徽宗赵佶长子,谥号「恭文顺德仁孝皇帝」。

靖康之耻:赵桓1100年5月23日生于坤宁殿。初名赵亶,封韩国公,明年六月进封京兆郡王。崇宁元年二月甲午,更名赵烜,十一月丁亥,改名赵桓。大观二年正月,进封定王。政和元年三月,讲学于资善堂。三年正月,加太保。四年二月癸酉,在文德殿加冠。五年二月乙巳,立为皇太子,大赦天下。丁巳,谒太庙。诏乘金辂,设卤簿,如至道、天禧故事,宫僚参谒、称臣,都辞让。六年六月癸未,纳朱氏为太子妃。

北宋末年,金朝南侵。赵桓受父宋徽宗禪讓后即位为皇帝,是为宋钦宗,改元靖康。徽宗称太上皇,率亲信蔡京、童贯等人逃奔江南。钦宗即位後立刻貶被称为奸臣的蔡京、童贯等人,然后重用李纲抗金,一度令金军有北返之意。后来钦宗听从奸臣谗言,罢免了李纲,向金朝求和,加上在背後受徽宗牽制,作用有限。钦宗装作恭顺迎接太上皇回京后即贬去太上皇左右,软禁太上皇于龙德宫,因为对太上皇怀有戒心,甚至在太上皇过寿时连太上皇敬的酒都不喝,气得太上皇哭着回宫。钦宗亦将蔡京、童贯等“六贼”皆贬杀。

但钦宗在抗金时战和不定,又写密信交给金使萧仲恭意图策反降金的原辽将耶律余睹,给了金朝再度发难的理由。金朝趁此机会再次南下渡黄河再次包围宋京东京(今开封)。宋钦宗误信宰相何栗、枢密使孙傅,以为无赖郭京可以用神兵退敌,提拔郭京,并撤去东京外城守备让郭京用神兵退敌,郭京大败,金军趁机攻破东京外城,同时又放言和谈让钦宗心存幻想。钦宗率大臣亲赴金营和谈,金军趁机派萧庆入住北宋尚书省。钦宗在上表投降称藩及同意召在外的皇弟康王赵构回京并废除其大元帅头衔等条件后被放回,萧庆则以北宋朝廷名义搜刮钱物满足金军所需。靖康二年(1127年)正月,钦宗再赴金营求和,被扣押,金军以放回钦宗为条件继续向北宋索要财物,宋将范琼等将太上皇、钦宗太子赵谌及后妃、亲王、公主等交给金军作为抵债,史称靖康之变。

北宋靖康二年(金朝天會五年)二月初六(1127年3月20日),金太宗下詔廢徽、欽二帝,貶為庶人,強行脱去二帝龙袍,随行的李若水抱著欽宗身體,嚴斥金人为狗賊之辈,金人將其割喉殘忍殺害,並册封张邦昌为帝,国号大楚,他成為亡國之君,立國一百六十八年的北宋滅亡。

金国拘禁:七月,二帝被俘北上,遷到燕京,天會六年(1128年)八月二十一日抵達金上京會寧府。二十四日,二帝著素服跪拜金太祖廟,行「牽羊禮」,在乾元殿拜謁金太宗完顏吳乞買。金太宗封宋徽宗為昏德公,欽宗為重昏侯,十月,二帝遷往韓州(吉林省梨樹縣北偏臉城)。天會八年(1130年)七月,又將二帝遷到五國城(今黑龍江省依蘭縣城北舊古城)軟禁。

八年後,天會十三年(紹興五年,1135年)四月,徽宗病死於五國城。

天眷三年(1140年),金主战派完颜宗弼率领金国军队南侵,先在开封正南的顺昌败于刘錡所部的「八字军」,再于开封西南的郾城和颍昌,在金国女真精锐部队所拿手的骑兵对阵中两次败于岳飞的岳家军,只在开封东南面的淮西亳州、宿州一带战胜了宋军中最弱的张俊一军,在宋高宗以「十二道金牌」敕召回岳家军前,金军已被压缩到汴梁东部和北部。完颜宗弼开始转向接受與宋议和。

皇统元年(1141年) 二月,金熙宗为与南宋達成完颜宗弼期望中的和议,将死去的徽宗追封为天水郡王,将钦宗封为天水郡公。第一提高了级别,原来封徽宗为二品昏德公,追封为王升为一品,原封钦宗为三品重昏侯,现封为公升为二品。第二是去掉了原封号中的侮辱含义。第三是以赵姓天水郡望作为封号,以示尊重。同时南宋朝廷解除了岳飛、韓世忠、刘錡、杨沂中等將的兵權,为《绍兴和议》做好了准备。十一月间,宋、金为《绍兴和议》达成书面协议。十二月末除夕夜(1142年1月27日),宋高宗在餘杭風波亭賜死了岳飞,据《宋史》记载是为了满足议和所设前提。

紹興十二年(1142年)三月,宋金《绍兴和议》彻底完成所有手续。夏四月丁卯(1142年5月1日),宋高宗生母韋賢妃同徽宗棺槨歸宋。离行时,钦宗挽住她的车轮,请她转告高宗,若能回去,他絕對不爭權,不當皇帝,只要當個「太乙宫」之主就滿足了。韋賢妃哭著說,如果我看不到你回來,我寧願眼睛瞎掉算了。宋高宗曾提出钦宗南归,金朝也答应了,高宗令临安府为钦宗修宫殿;但金朝出现政局变动,执政者希望保留在汴京拥立钦宗复辟以挑战宋高宗合法性的选择,钦宗南归一事遂被搁置。韋賢妃晚年果真因眼疾而雙目俱盲,後遇道人治癒一眼,另一眼則因道人曰:「后以一目視,足矣。以一目存誓,可也。」而未獲醫治。

南宋紹興二十六年(金朝正隆元年,1156年)六月,宋钦宗去世。关于死因众说纷纭,有指钦宗是病死。另据《大宋宣和遺事》记载,金朝皇帝完顏亮叫57歲的钦宗和81岁的辽天祚帝耶律延禧去比賽馬球,宋钦宗從馬上跌下來,被馬亂踐而死。

宋绍兴三十一年(金正隆六年,1161年),钦宗的死讯才传到南宋,宋高宗发喪。欽宗死后,高宗再无后顧之憂,因而翌年传位于宋孝宗。

元朝官修正史《宋史》脱脱等的評價是:“帝在东宫,不见失德。及其践阼,声技音乐一无所好。靖康初政,能正王黼、朱勔等罪而窜殛之,故金人闻帝内禅,将有卷甲北旆之意矣。惜其乱势已成,不可救药,君臣相视,又不能同力协谋,以济斯难,惴惴然讲和之不暇。卒致父子沦胥,社稷芜茀。帝至于是,盖亦巽懦而不知义者欤!享国日浅,而受祸至深,考其所自,真可悼也夫!真可悼也夫!”

《靖康稗史》金国方面所记录的俘虏供词:二王令成棣译询宫中事:……少帝贤,务读书。不迩声色。受禅半年,无以备执事,乃立一妃、十夫人,仅三人得幸。自余俭德,不可举数。

\subsection{靖康}


\begin{longtable}{|>{\centering\scriptsize}m{2em}|>{\centering\scriptsize}m{1.3em}|>{\centering}m{8.8em}|}
  % \caption{秦王政}\
  \toprule
  \SimHei \normalsize 年数 & \SimHei \scriptsize 公元 & \SimHei 大事件 \tabularnewline
  % \midrule
  \endfirsthead
  \toprule
  \SimHei \normalsize 年数 & \SimHei \scriptsize 公元 & \SimHei 大事件 \tabularnewline
  \midrule
  \endhead
  \midrule
  元年 & 1126 & \tabularnewline\hline
  二年 & 1127 & \tabularnewline
  \bottomrule
\end{longtable}



%%% Local Variables:
%%% mode: latex
%%% TeX-engine: xetex
%%% TeX-master: "../Main"
%%% End:
