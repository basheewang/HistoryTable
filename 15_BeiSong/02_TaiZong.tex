%% -*- coding: utf-8 -*-
%% Time-stamp: <Chen Wang: 2021-11-01 15:53:06>

\section{太宗趙炅\tiny(976-997)}

\subsection{生平}

宋太宗趙\xpinyin*{炅}(939年11月20日-997年5月8日),北宋第二位皇帝(976年11月15日-997年5月8日在位),在位21年,享年58岁。趙弘殷第三子,是北宋開國君主宋太祖趙匡胤的胞弟。本名趙匡義,字廷宜,其兄长赵匡胤登基後為避諱,改名趙光義,趙匡胤去世後,勾結兄長身邊宦官篡位,史說為燭光斧影,後流放和逼迫兄長趙匡胤的兩個兒子自殺。

宋太宗文治有為,但不善武功。于太平兴国三年(978年)迫使吴越纳土归降;之后又灭亡五代十国最后一个割据政权北汉,结束北宋统一战爭。次年(979年)宋太宗趙光義移师幽州,试图一举收复燕云十六州,在高粱河(今北京西直门外)展开激战,宋军大败,宋太宗被耶律休哥射伤,乘驴车逃走。他两度伐辽失败。980年又试图兼併交趾,但惨败,使交趾(越南)最终得以保持独立地位。

后晋天福四年十月七日(939年11月20日)生于东京(河南开封)护圣营,为乾明节,后改为寿宁节。

趙光義於建隆二年(961年)七月任開封府尹,至開寶九年(976年)十月登基後離任,主政天府十五年,治績斐然。是任職最長的开封府尹,这为他取得帝位打下基礎。

有野史記載認為宋太宗即位之事甚為蹊蹺,有弒兄謀朝篡位之嫌,即「燭影斧聲」之疑案。

一般來說,新皇帝在先帝駕崩后翌年改元。宋太宗的改元時間是繼位當年的農曆十二月,當時距離農曆新的一年已經只剩下幾天而已。這一點為後人所詬病,也為即位爭議留下了另一個頗具爭議性的口實。

為確保政權合法性,太宗拋出其母杜太后遺命之說,即「金匱之盟」。在宋建隆二年(西元961年),即太祖即位的第二年,皇太后杜氏臨終前,告誡太祖前朝後周之所以滅亡,是因為繼位的君主過於年幼。若要常保大宋江山,必須要兄終弟及,傳位給年長的皇室成員當天子。太祖死後先傳其弟光義,再傳弟光美(後改名為廷美),等到皇兄的兒子成人,再由皇弟傳回給皇兄的兒子,即太祖長子趙德昭。趙普入宮記錄遺命,這份遺書藏於金匱中,因此名為金匱之盟。

然而,太宗在位期間逼死太祖之子德昭,趙德芳暴斃,太宗又貶黜其弟廷美到房州,兩年後趙廷美死於谪所。1940年代鄧廣銘、張蔭麟等論證金匱之盟為虛構,影響至今:皇太后杜氏去世之時,太祖年僅三十四歲,正值壯年,根本不會聯想到死亡;退一萬步言,就算太祖立刻死亡,而其長子趙德昭當時已十一歲,離成年已不遠,根本不會出現如周世宗遺下七歲孤兒的局面,而且後來太祖五十歲駕崩時,趙德昭已經二十六歲,過了二十歲成年已經又六年了,故偽造說成為最普遍的說法。但近年也有學者質疑偽造說,如施秀娥、王育濟、何冠環。

太宗登基以后,“太祖之后当再有天下”之说一直不断,至靖康之變後,宋欽宗之弟康王赵构自立於江南,是為宋高宗。当时普遍有种传说,说因为太宗登基不明不白,所以才会让后代失去半壁江山,后又有孟太后之宋太祖托梦一说,加上高宗無子,最终1162年传位给趙德芳之后代宋孝宗,宋朝在太宗一脈統治186年后,再回到太祖一脈。

太宗穩固帝位後,繼續統一事業。其後割據福建漳泉兩府的陳洪進,割據吳越錢氏相继歸降。太宗遣大将潘美揮师北上围攻北漢都城太原,击退辽国援兵,滅亡北漢,終於结束了安史之亂后近二百年藩镇割据的局面。

太平兴国四年(979年)五月,太宗不顧眾臣反對,趁伐取北漢之勢,從太原出發展開北伐。北伐初期一度收復河北易州和涿州。太宗志得意滿,下令圍攻燕京,宋軍與遼人在高粱河畔展开激战。太宗亲临战场,結果受傷中箭,乘驴车仓惶撤离,北伐未果。

980年宋朝知邕州太常博士侯仁宝上奏宋太宗,请求趁交趾(越南)丁朝内乱之机南下讨伐,恢复汉唐故疆,统一交趾(越南)。宋太宗任命侯仁宝为交州陆路水路转运使;任命兰陵团练使孙全兴、漆作使郝守俊、鞍辔库使陈钦祚、左监门将军崔亮为兵马都部署;宁州刺史刘澄、军器库副使贾湜、供奉官阁门祗候王僎为兵马都部署,伺机进攻丁朝。但在981年白藤江之战中先胜后败,统一交趾(越南)的计划最终成为泡影,交趾(越南)得以保持独立地位。

雍熙三年(986年),太宗遣潘美、杨业、田重进、曹彬、崔彦五位大将分东中西三路,以東路為主再行北伐。西路、中路军進軍順利,而主力东路軍屡遭辽军挫敗,粮道被切断,终未能与中西二路汇合,于岐沟关大败而潰。中、西二路亦只得南撤。西路主將杨业因掩护军民南撤被辽军俘虏,在狱中绝食三日而死。之後,北宋在對西夏党項族的三川口、好水川、定川寨等戰役中屢次失敗,但因其厌战,与宋廷议和。

太宗朝以亲信傅潜、王超、柴禹锡、赵镕、张逊、杨守一及弭德超等为禁军统帅,多庸碌之徒,临阵惧战。元人修《宋史》时称:“自柴禹锡而下,率因给事藩邸,以攀附致通显……故莫逃于龊龊之讥。”

幾次边陲防線的失利、後方起義的爆發遏制了北宋進一步開闢疆土,太宗的施政也不得不轉為重內虛外。太宗本人附庸風雅喜好詩賦,政府也因此特別重視文化事業,宋朝重教之風因而展開。太宗喜好書法,善草、隸(八分)、行、篆、飞白等數種字體,尤其善書飛白體,宋朝的貨幣淳化元寶也是太宗親自题寫的。

太宗長子元佐因為同情廷美被廢,另一子元僖暴死,最後襄王元侃被立為太子,改名恒。至道三年(997年),太宗崩,李皇后和宦官王繼恩等企圖立元佐為帝。時宰相呂端處置得當,赵恒順利即位,庙号真宗。宋朝始步入安穩守成時期。

至道三年(997年),三月二十九日帝因箭瘡舊疾崩于東京宮中之萬歲殿,享年五十八歲,在位二十二年。皇太子趙恒登基為帝,是為宋真宗。群臣上尊諡曰神功聖德文武皇帝,廟號太宗。同年十月葬在永熙陵。死后谥号「至仁應道神功聖德文武睿烈大明廣孝皇帝」。宋太宗的後代為宋真宗至宋高宗所有宋朝皇帝及宋寧宗初的宗室大臣趙汝愚。

宋太宗好讀書,「開卷有益」典故即來自他。王闢之的《澠水燕談錄》卷六:“太宗日閱《御覽》三卷,因事有缺,暇日追補之,嘗曰:開卷有益,朕不以為勞也。” 宋太宗曾對日本萬世一系的感慨

元朝官修正史《宋史》脱脱等的評價是:“帝沈谋英断,慨然有削平天下之志。既即大位,陈洪进、钱俶相继纳土。未几,取太原,伐契丹,继有交州、西夏之役。干戈不息,天灾方行,俘馘日至,而民不知兵;水旱螟蝗,殆遍天下,而民不思乱。其故何也?帝以慈俭为宝,服浣濯之衣,毁奇巧之器,却女乐之献,悟畋游之非。绝远物,抑符瑞,闵农事,考治功。讲学以求多闻,不罪狂悖以劝谏士,哀矜恻怛,勤以自励,日晏忘食。”毛澤東閱讀《宋史·太宗本紀》時对此批曰“但無能”。

趙匡義對軍事理論見解獨到,他建立了參謀本部制度、陣圖制度和軍事學院體系,是現代化軍隊的淵源。同時確立了文官掌兵的軍隊國有化體制,宋朝開始逐漸形成現代化集團軍編制,也和宋太宗有重大關係。但他指揮軍隊的能力十分薄弱,加上遼國立國早宋朝五十年,底蘊深厚,又有耶律斜軫等名將輔佐;宋遼數度交鋒皆以慘敗收場,毛澤東對此表示“此人不知兵,非契丹对手。爾後屢敗,契丹均以誘敵深入、聚而殲之的辦法,宋人終不省。”

《宋史·太宗本纪》:“欲自焚以答天谴,欲尽除天下之赋以纾民力”,老百姓比肩接踵而至,请其“登禅”即位。“故帝之功德,炳焕史牒,号称贤君。若夫太祖之崩不逾年而改元,涪陵县公之贬死,武功王之自杀,宋后之不成丧,则后世不能无议论焉。”毛泽东批道:“不择手段,急于登台。”《宋史纪事本末》记载宋太宗诏立太子后回宫途中,百姓都欢呼雀跃,欢呼“少年天子”,宋太宗听了很不高兴,召见宰相寇准说:“人心遽属太子,欲置我何地?”寇准向他道贺说:“此社稷之福也。”他转怒为喜请寇准喝酒,“极醉而罢”,毛泽东批道:“赵匡义小人之言。”

\subsection{太平兴国}


\begin{longtable}{|>{\centering\scriptsize}m{2em}|>{\centering\scriptsize}m{1.3em}|>{\centering}m{8.8em}|}
  % \caption{秦王政}\
  \toprule
  \SimHei \normalsize 年数 & \SimHei \scriptsize 公元 & \SimHei 大事件 \tabularnewline
  % \midrule
  \endfirsthead
  \toprule
  \SimHei \normalsize 年数 & \SimHei \scriptsize 公元 & \SimHei 大事件 \tabularnewline
  \midrule
  \endhead
  \midrule
  元年 & 976 & \tabularnewline\hline
  二年 & 977 & \tabularnewline\hline
  三年 & 978 & \tabularnewline\hline
  四年 & 979 & \tabularnewline\hline
  五年 & 980 & \tabularnewline\hline
  六年 & 981 & \tabularnewline\hline
  七年 & 982 & \tabularnewline\hline
  八年 & 983 & \tabularnewline\hline
  九年 & 984 & \tabularnewline
  \bottomrule
\end{longtable}

\subsection{雍熙}

\begin{longtable}{|>{\centering\scriptsize}m{2em}|>{\centering\scriptsize}m{1.3em}|>{\centering}m{8.8em}|}
  % \caption{秦王政}\
  \toprule
  \SimHei \normalsize 年数 & \SimHei \scriptsize 公元 & \SimHei 大事件 \tabularnewline
  % \midrule
  \endfirsthead
  \toprule
  \SimHei \normalsize 年数 & \SimHei \scriptsize 公元 & \SimHei 大事件 \tabularnewline
  \midrule
  \endhead
  \midrule
  元年 & 984 & \tabularnewline\hline
  二年 & 985 & \tabularnewline\hline
  三年 & 986 & \tabularnewline\hline
  四年 & 987 & \tabularnewline
  \bottomrule
\end{longtable}

\subsection{端拱}

\begin{longtable}{|>{\centering\scriptsize}m{2em}|>{\centering\scriptsize}m{1.3em}|>{\centering}m{8.8em}|}
  % \caption{秦王政}\
  \toprule
  \SimHei \normalsize 年数 & \SimHei \scriptsize 公元 & \SimHei 大事件 \tabularnewline
  % \midrule
  \endfirsthead
  \toprule
  \SimHei \normalsize 年数 & \SimHei \scriptsize 公元 & \SimHei 大事件 \tabularnewline
  \midrule
  \endhead
  \midrule
  元年 & 988 & \tabularnewline\hline
  二年 & 989 & \tabularnewline
  \bottomrule
\end{longtable}

\subsection{淳化}

\begin{longtable}{|>{\centering\scriptsize}m{2em}|>{\centering\scriptsize}m{1.3em}|>{\centering}m{8.8em}|}
  % \caption{秦王政}\
  \toprule
  \SimHei \normalsize 年数 & \SimHei \scriptsize 公元 & \SimHei 大事件 \tabularnewline
  % \midrule
  \endfirsthead
  \toprule
  \SimHei \normalsize 年数 & \SimHei \scriptsize 公元 & \SimHei 大事件 \tabularnewline
  \midrule
  \endhead
  \midrule
  元年 & 990 & \tabularnewline\hline
  二年 & 991 & \tabularnewline\hline
  三年 & 992 & \tabularnewline\hline
  四年 & 993 & \tabularnewline\hline
  五年 & 994 & \tabularnewline
  \bottomrule
\end{longtable}

\subsection{至道}

\begin{longtable}{|>{\centering\scriptsize}m{2em}|>{\centering\scriptsize}m{1.3em}|>{\centering}m{8.8em}|}
  % \caption{秦王政}\
  \toprule
  \SimHei \normalsize 年数 & \SimHei \scriptsize 公元 & \SimHei 大事件 \tabularnewline
  % \midrule
  \endfirsthead
  \toprule
  \SimHei \normalsize 年数 & \SimHei \scriptsize 公元 & \SimHei 大事件 \tabularnewline
  \midrule
  \endhead
  \midrule
  元年 & 995 & \tabularnewline\hline
  二年 & 996 & \tabularnewline\hline
  三年 & 997 & \tabularnewline
  \bottomrule
\end{longtable}


%%% Local Variables:
%%% mode: latex
%%% TeX-engine: xetex
%%% TeX-master: "../Main"
%%% End:
