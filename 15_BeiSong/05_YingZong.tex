%% -*- coding: utf-8 -*-
%% Time-stamp: <Chen Wang: 2021-11-01 15:54:41>

\section{英宗趙曙\tiny(1063-1067)}

\subsection{生平}

宋英宗趙曙(1032年2月16日-1067年1月25日),原名趙宗實,是濮王赵允让之子,过继给宋仁宗为嗣,是北宋第五代皇帝,1063年5月1日—1067年1月25日在位。

趙曙是前任第四代皇帝宋仁宗的堂兄趙允讓的第十三子,是宋太宗第四子趙元份的後裔,生母為仙遊縣君任氏。天聖十年(明道元年)壬申年正月三日甲戌(1032年2月15日)生於宣平坊宅第〔嘉祐八年(1063年),英宗把這天定為“壽聖節”〕,最初,濮王夢兩龍與太陽一起掉落下來,用衣服裝住了它們,到英宗出生時,赤光滿室,有黃龍在赤光中游走。英宗幼年被仁宗接入皇宮撫養,賜名為宗實。1050年為岳州團練使,後為秦州防禦使。1055年立趙曙為嗣。

宋仁宗所生的三名兒子皆幼年夭折,而仁宗的兄弟也早逝(早於仁宗登基時逝世),故在嘉祐七年(1062年)立趙曙為皇太子,封鉅鹿郡公。嘉祐八年即帝位。

宋英宗時代對生父尊禮濮安懿王趙允讓的討論,引起了一系列政治事件。

宋英宗趙曙原是濮王趙允讓的兒子,過繼給宋仁宗為皇子。宋英宗即位後,討論崇奉濮王的典禮。治平元年(1064年),韓琦、歐陽修等奏請尊禮濮安懿王為皇考。尊禮之事引起與王珪、司馬光、呂誨、范純仁、呂大防等台諫大臣的不滿,主張稱濮王為皇伯。史稱濮議。呂誨、范純仁、呂大防等人被貶黜,治平三年(1066年),由於宋英宗強烈意願,使曹太后認可尊濮安懿王為皇考濮安懿皇。但是,趙允讓始終沒有獲得明確的皇帝尊號,隨著英宗的去世,事情不了了之。

英宗即位不久即病,無法處理朝政,由曹太后於內東門小殿垂簾聽政,待英宗病情好轉後,曹太后即撤簾歸政。

英宗雖然多病,行事甚至有些荒唐,但剛即位時,還是表現出了一個有為之君的風範。仁宗暴亡,醫官應當負有責任,主要的兩名醫官便被英宗逐出皇宮,送邊遠州縣編管。其他一些醫官,唯恐也遭貶謫。顯然,英宗行事很有些雷厲風行的風格,與濫施仁政的仁宗有著很大的不同。不僅如此,英宗也是一個很勤勉的皇帝。當時,輔臣奏事,英宗每每詳細詢問事情始末,方才裁決,處理政務非常認真。

英宗繼續任用仁宗時的改革派重臣韓琦、歐陽修、富弼等人,面對積弱積貧的國勢,力圖進行一些改革。

宋英宗治平三年任命司馬光設局專修《資治通鑑》,經費由政府資助,更准借閱秘閣藏書,並自選助手(劉恕、范祖禹、劉攽、司馬康),並提供筆硯文具、撥款、水果、糕點等,讓司馬光無後顧之憂的從事史書撰述。於神宗元豐七年成書,神宗並親自為此書作序。司馬光為了報答英宗的知遇之恩,在此後漫長的19年裡,將全部精力都耗在《資治通鑑》這部巨著的編纂上。應該說,史學巨著《資治通鑑》的最後編成也有英宗的一份功勞。

治平四年正月八日丁巳(1067年1月25日)英宗崩,享年36歲,殯於殿西階,廟號英宗,群臣上諡憲文肅武宣孝皇帝,八月二十七日癸酉,葬英宗於永厚陵(今河南鞏義孝義堡)。

英宗本人對於北宋中興抱有極大期望,相對其子神宗,政治手段也更為成熟。無奈壽短,使得北宋過早進入神宗朝,從而失掉了可能的中興計劃,為神宗朝王安石的變法提供了機會。

元朝官修正史《宋史》脱脱等的評價是:“昔人有言,天之所命,人不能违。信哉!英宗以明哲之资,膺继统之命,执心固让,若将终身,而卒践帝位,岂非天命乎?及其临政,臣下有奏,必问朝廷故事与古治所宜,每有裁决,皆出群臣意表。虽以疾疹不克大有所为,然使百世之下,钦仰高风,咏叹至德,何其盛也!彼隋晋王广、唐魏王泰窥觎神器,矫揉夺嫡,遂启祸原,诚何心哉!诚何心哉!”


\subsection{治平}


\begin{longtable}{|>{\centering\scriptsize}m{2em}|>{\centering\scriptsize}m{1.3em}|>{\centering}m{8.8em}|}
  % \caption{秦王政}\
  \toprule
  \SimHei \normalsize 年数 & \SimHei \scriptsize 公元 & \SimHei 大事件 \tabularnewline
  % \midrule
  \endfirsthead
  \toprule
  \SimHei \normalsize 年数 & \SimHei \scriptsize 公元 & \SimHei 大事件 \tabularnewline
  \midrule
  \endhead
  \midrule
  元年 & 1064 & \tabularnewline\hline
  二年 & 1065 & \tabularnewline\hline
  三年 & 1066 & \tabularnewline\hline
  四年 & 1067 & \tabularnewline
  \bottomrule
\end{longtable}



%%% Local Variables:
%%% mode: latex
%%% TeX-engine: xetex
%%% TeX-master: "../Main"
%%% End:
