%% -*- coding: utf-8 -*-
%% Time-stamp: <Chen Wang: 2021-11-01 15:55:00>

\section{徽宗赵佶\tiny(1100-1125)}

\subsection{生平}

宋徽宗赵佶(1082年6月7日-1135年6月4日),北宋第八位皇帝,自稱教主道君皇帝,同時也是畫家、書法家、詩人、詞人和收藏家,且擅長古琴、蹴鞠、擊鞠、打獵,自創“瘦金書”字體。徽宗在書畫上的花押大概是中國歷史上最出名的花押。

徽宗為宋神宗十一子,宋哲宗之弟,先後被封为遂宁王、端王。其兄长宋哲宗于公元1100年正月病逝时无子,向太后于同月立他为帝,並垂簾聽政一年,第二年改年號为“建中靖國”。在位26年(1100年2月23日—1126年1月18日),国亡被俘因病而死,终年54岁,葬于都城绍兴永祐陵(今浙江省绍兴市柯桥区东南35里处)。

被後世評為「宋徽宗诸事皆能,獨不能為君耳!」编写《宋史》的史官,也感慨地说,如果當初章惇「端王轻佻,不可君天下」的意见被采纳,北宋也許是另一種结局,“宋不立徽宗,金雖强,何衅以伐宋哉。”

徽宗為宋神宗第十一子,宋哲宗之弟,本未必有機會繼承大統。惟宋哲宗23歲英年早逝,無子,故宋室由哲宗眾弟中尋找繼承人。本來哲宗眾弟中以申王趙佖最長,惜因患有眼疾而不被選為繼位者,故以當時封為端王的徽宗繼承大統。宰相章惇曾反對端王繼位,反而建議立哲宗同母弟簡王趙似,但向太后支持端王繼位,故徽宗順利成為大宋皇帝。

建中靖國元年(1101年),向太后去世,徽宗親政。宋徽宗親政後,“妄耗百出,不可勝數”,过分追求奢侈生活,在南方采办“花石纲”,搜集奇花异石运到汴京開封府,修建艮岳等工程浩大的园林宫殿,崇信道教,尊號「教主道君皇帝」,任用贪官宦官横徵暴敛,激起各地民变。其中以新黨蔡京任丞相與宦官童貫為將軍所引致的問題最嚴重。

徽宗好大喜功,不顾宋辽已百年和平相处,于宣和二年(1120年),与金国结成“海上之盟”,联合灭辽。1122年,金军攻克辽南京(今北京市)。

宣和七年(1125年)十月,金太宗遣諳班勃極烈完颜斜也、完颜宗望、乙室勃極烈完颜宗翰分兩路南下入侵北宋。宣和七年十二月二十三日(1126年1月18日),徽宗无法应付时,急忙禪讓天子的寶座给他儿子宋钦宗去对付,自己则当“太上皇”并出逃,但终究无法挽回局势。金军暂退后,徽宗回京,居龙德宫,实际上被钦宗软禁,甚至连在过寿时给钦宗敬的酒钦宗也不喝,气得哭着回宫。

靖康元年(1126年)八月,金太宗再次命東、西兩路軍大舉南下,《三朝北盟会编卷六十九》等史书记载宋兵部尚書孫傅把希望放在禁军老兵郭京身上,郭京伪称精通佛道二教之法术,能施道门“六甲法”,用七千七百七十七人布阵,并会佛教“毘沙門天王像法”,布阵画像,但神兵大敗,金兵分四路乘機攻入城內,金軍攻佔了帝都汴京。宋欽宗遣使臣何㮚到金營請和,宗翰、宗望二帥不允。金军提出见徽宗,钦宗不肯。北宋靖康二年(金朝天會五年)正月,钦宗亲自请和被扣押,宋将范琼变节将徽宗、宗室、后妃公主等交给金军。二月初六(1127年3月20日),金太宗下詔廢徽、欽二帝,貶為庶人,北宋滅亡,二帝被俘北上。七月二十日,二帝遷到中京(今北京市),父子抱头痛哭。

天會六年(1128年)八月二十一日抵達金上京會寧府。二十四日,二帝及男女宋俘均坦胸赤背,身披羊皮,跪拜金太祖廟,行「牽羊禮」,在乾元殿拜謁金太宗完颜吳乞買。金太宗封宋徽宗為昏德公,欽宗為重昏侯,十月二十六日,二帝遷往韓州(辽宁省昌图八面城)。在韩州,金人将城内女真住户全部迁出,只供二帝等二千余宋俘居住。据《宋俘记》载:「给田四十五顷,种莳自给。」据《南征錄彙》说这还是金国二太子完颜宗望(劫宋徽宗之女茂德帝姬为妻)格外开恩,要求性格兇狠的完颜宗翰等不可像虐待辽天祚帝那样对待宋朝的徽、欽兩帝。

天會八年(1130年)七月,又將二帝遷到五國城(今黑龍江省依蘭縣城北舊古城)軟禁。到达五国城时,隨行男女仅140余人。流放期间徽宗仍雅好寫詩,读唐代李泌傳,感触颇深。五年后,天會十三年(绍兴五年,1135年)四月,病死於五國城。照當地習俗火葬。

皇统元年(1141年)二月,金熙宗为改善与南宋的关系,将死去的徽宗追封为天水郡王,将钦宗封为天水郡公。第一提高了级别,原来封徽宗为二品昏德公,追封为王升为一品,原封钦宗为三品重昏侯,现封为公升为二品。第二是去掉了原封号中的侮辱含义。第三是以赵姓天水郡望之为封号,以示尊重。同时南宋朝廷解除了岳飛、韓世忠、刘錡、杨沂中等將領的兵權,为《绍兴和议》做好了准备。十一月间,宋、金为《绍兴和议》达成书面协议。十二月末除夕夜(1142年1月27日),宋高宗赐死岳飞,据《宋史》载是为了满足完颜宗弼议和所设前提。紹興十二年(1142年)三月,宋金《绍兴和议》彻底完成所有手续。夏四月丁卯(1142年5月1日),高宗生母韋賢妃同徽宗棺槨歸宋。同年八月十餘辆牛车到達临安,十月,宋高宗将徽宗暂葬于会稽(今浙江省绍兴市),名曰永固陵(後改名永祐陵)。

三国时期曹魏李康《运命论》:“夫黄河清而圣人生。”宋徽宗在位時,曾出現過三次“河清” - 黃河變得清澈的奇觀,使得當時百官弹冠相庆,大肆歌功颂德。在黄河中下游,河水偶爾也有暫時变得清澈的时候,即史书中作为祥瑞记下的“河清”,并不是五百年乃至千年一遇。据地质学史专家李鄂荣先生考证,中国历史上的“河清”,有记载可查的便有43次,首见于汉桓帝延熹八年(165年),如从此时起算,平均不到40年就有一次。

根据《宋史》,宋徽宗在位年间的三次“河清”,分别为:第一次,大观元年(1107年),“乾宁军、同州黄河清。”第二次,大观二年(1108年),“同州黄河清。”,第三次,大观三年(1109年),“陕州、同州黄河清。”

大观元年(1107年)“乾宁军言黄河清,逾八百里,凡七昼夜,诏以乾宁军为清州”。“黄河清”被谱写成新曲流传,还在韩城建立记载这些祥瑞的“河渎碑”。此碑至今尚在。

可是僅僅到了立碑15年后的1127年,宋徽宗便和他的儿子宋钦宗一起被金兵俘虏,押到了金朝统治下的东北地区,北宋至此滅亡。被民眾譏為“聖人豈女真人乎?”

在宋代的社会风气中,文人雅士和歌伎、妓女交往属于正常现象。民间流传宋徽宗十分喜欢青楼女子李师师。李师师是北宋东京有名的艺伎,色艺双绝,诗词歌赋、笙管笛箫样样精通,宋徽宗得知后不顾九五之尊,数次前去青楼与李师师见面。后来在皇宫和妓院之间挖了一条地道,方便和李师师见面,现在在开封的宋城遗址当中,还能看到这个神秘地道的一点痕迹。宋徽宗还和李师师旧时相好的著名的词人周邦彦,争风吃醋,不久便在整个东京城传得沸沸扬扬。

宋徽宗酷爱艺术,在位时将藝術的地位提到在中国历史上最高的位置,成立翰林书画院,即当时的宫廷画院。

更特別的是以画作为科举升官的一种考试方法,每年以诗词做题目曾刺激出许多新的创意佳话。如题目为“山中藏古寺”,许多人画深山寺院飞檐,但得第一名的没有画任何房屋,只画了一个和尚在山溪挑水;另题为“踏花归去马蹄香”,得第一名的没有画任何花卉,只画了一人骑马,有蝴蝶飞绕马蹄间,凡此等等。这些都极大地刺激了中国画意境的发展。

他对自然观察入微,曾写到:“孔雀登高,必先举左腿”等有关绘画的理论文章。广泛搜集历代文物,令下属编辑《宣和书谱》、《宣和画谱》、《宣和博古录》等著名美术史书籍。对研究美术史有相当大的贡献。

赵佶还喜爱在自己喜欢的书画上题诗作跋,后人把这种画叫“御题画”。由于许多画上并没有留下作者的名字,他本人又擅长繪画。对鉴别这些画是否是赵佶的作品有不小的难度。一般認為《詩帖》在內的書法,以及粗筆的《柳鸭图》和《池塘秋晚图》為其繪畫風格,而細筆的《芙蓉锦鸡图》、《腊梅山禽图》、《竹禽图》、《四禽图》等御题画則尚無定論。

宋徽宗在其創作的書畫上使用一個類似拉長了的「天」字的花押,據說象徵「天下一人」。這也是中國歷史上最出名的花押。

章惇反對宋徽宗即位時曾說:「端王輕佻,不可以君天下。」

元朝官修正史《宋史》脱脱等的評價是:“宋中叶之祸,章、蔡首恶,赵良嗣厉阶。然哲宗之崩,徽宗未立,惇谓其轻佻不可以君于下。辽天祚之亡,张觉举平州来归,良嗣以为纳之失信于金,必启外侮。使二人之计行,宋不立徽宗,不纳张觉,金虽强,何衅以伐宋哉?以是知事变之来,虽小人亦能知之,而君子有所不能制也。迹徽宗失国之由,非若晋惠之愚、孙皓之暴,亦非有曹、马之篡夺,特恃其私智小慧,用心一偏,疏斥正士,狎近奸谀。于是蔡京以獧薄巧佞之资,济其骄奢淫佚之志。溺信虚无,崇饰游观,困竭民力。君臣逸豫,相为诞谩,怠弃国政,日行无稽。及童贯用事,又佳兵勤远,稔祸速乱。他日国破身辱,遂与石晋重贵同科,岂得诿诸数哉?昔西周新造之邦,召公犹告武王以不作无益害有益,不贵异物贱用物,况宣、政之为宋,承熙、丰、绍圣椓丧之余,而徽宗又躬蹈二事之弊乎?自古人君玩物而丧志,纵欲而败度,鲜不亡者,徽宗甚焉,故特著以为戒。”

《靖康稗史箋證》卷5《呻吟語》記:「二王令成棣譯詢宮中事:道宗五七日必御一處女,得御一次即畀位號,續幸一次進一階。退位後,出宮女六千人,宜其亡國。」

元朝脱脱撰《宋史》的《徽宗记》,不由掷笔叹曰:“宋徽宗诸事皆能,独不能为君耳!”

1964年3月24日,毛泽东在一次谈话评点知识分子型皇帝说:“可不要看不起老粗。知识分子是比较最没有出息的。历史上当皇帝,有许多是知识分子,是没有出息的,隋炀帝就是一个会做文章、诗词的人。陈后主、李后主都是能诗能赋的人。宋徽宗既能写诗,又能绘画。一些老粗能办大事情,成吉思汗、刘邦、朱元璋。”


\subsection{建中靖国}


\begin{longtable}{|>{\centering\scriptsize}m{2em}|>{\centering\scriptsize}m{1.3em}|>{\centering}m{8.8em}|}
  % \caption{秦王政}\
  \toprule
  \SimHei \normalsize 年数 & \SimHei \scriptsize 公元 & \SimHei 大事件 \tabularnewline
  % \midrule
  \endfirsthead
  \toprule
  \SimHei \normalsize 年数 & \SimHei \scriptsize 公元 & \SimHei 大事件 \tabularnewline
  \midrule
  \endhead
  \midrule
  元年 & 1101 & \tabularnewline
  \bottomrule
\end{longtable}

\subsection{崇宁}

\begin{longtable}{|>{\centering\scriptsize}m{2em}|>{\centering\scriptsize}m{1.3em}|>{\centering}m{8.8em}|}
  % \caption{秦王政}\
  \toprule
  \SimHei \normalsize 年数 & \SimHei \scriptsize 公元 & \SimHei 大事件 \tabularnewline
  % \midrule
  \endfirsthead
  \toprule
  \SimHei \normalsize 年数 & \SimHei \scriptsize 公元 & \SimHei 大事件 \tabularnewline
  \midrule
  \endhead
  \midrule
  元年 & 1102 & \tabularnewline\hline
  二年 & 1103 & \tabularnewline\hline
  三年 & 1104 & \tabularnewline\hline
  四年 & 1105 & \tabularnewline\hline
  五年 & 1106 & \tabularnewline
  \bottomrule
\end{longtable}

\subsection{大观}

\begin{longtable}{|>{\centering\scriptsize}m{2em}|>{\centering\scriptsize}m{1.3em}|>{\centering}m{8.8em}|}
  % \caption{秦王政}\
  \toprule
  \SimHei \normalsize 年数 & \SimHei \scriptsize 公元 & \SimHei 大事件 \tabularnewline
  % \midrule
  \endfirsthead
  \toprule
  \SimHei \normalsize 年数 & \SimHei \scriptsize 公元 & \SimHei 大事件 \tabularnewline
  \midrule
  \endhead
  \midrule
  元年 & 1107 & \tabularnewline\hline
  二年 & 1108 & \tabularnewline\hline
  三年 & 1109 & \tabularnewline\hline
  四年 & 1110 & \tabularnewline
  \bottomrule
\end{longtable}

\subsection{政和}

\begin{longtable}{|>{\centering\scriptsize}m{2em}|>{\centering\scriptsize}m{1.3em}|>{\centering}m{8.8em}|}
  % \caption{秦王政}\
  \toprule
  \SimHei \normalsize 年数 & \SimHei \scriptsize 公元 & \SimHei 大事件 \tabularnewline
  % \midrule
  \endfirsthead
  \toprule
  \SimHei \normalsize 年数 & \SimHei \scriptsize 公元 & \SimHei 大事件 \tabularnewline
  \midrule
  \endhead
  \midrule
  元年 & 1111 & \tabularnewline\hline
  二年 & 1112 & \tabularnewline\hline
  三年 & 1113 & \tabularnewline\hline
  四年 & 1114 & \tabularnewline\hline
  五年 & 1115 & \tabularnewline\hline
  六年 & 1116 & \tabularnewline\hline
  七年 & 1117 & \tabularnewline\hline
  八年 & 1118 & \tabularnewline
  \bottomrule
\end{longtable}

\subsection{重和}

\begin{longtable}{|>{\centering\scriptsize}m{2em}|>{\centering\scriptsize}m{1.3em}|>{\centering}m{8.8em}|}
  % \caption{秦王政}\
  \toprule
  \SimHei \normalsize 年数 & \SimHei \scriptsize 公元 & \SimHei 大事件 \tabularnewline
  % \midrule
  \endfirsthead
  \toprule
  \SimHei \normalsize 年数 & \SimHei \scriptsize 公元 & \SimHei 大事件 \tabularnewline
  \midrule
  \endhead
  \midrule
  元年 & 1118 & \tabularnewline\hline
  二年 & 1119 & \tabularnewline
  \bottomrule
\end{longtable}

\subsection{宣和}

\begin{longtable}{|>{\centering\scriptsize}m{2em}|>{\centering\scriptsize}m{1.3em}|>{\centering}m{8.8em}|}
  % \caption{秦王政}\
  \toprule
  \SimHei \normalsize 年数 & \SimHei \scriptsize 公元 & \SimHei 大事件 \tabularnewline
  % \midrule
  \endfirsthead
  \toprule
  \SimHei \normalsize 年数 & \SimHei \scriptsize 公元 & \SimHei 大事件 \tabularnewline
  \midrule
  \endhead
  \midrule
  元年 & 1119 & \tabularnewline\hline
  二年 & 1120 & \tabularnewline\hline
  三年 & 1121 & \tabularnewline\hline
  四年 & 1122 & \tabularnewline\hline
  五年 & 1123 & \tabularnewline\hline
  六年 & 1124 & \tabularnewline\hline
  七年 & 1125 & \tabularnewline
  \bottomrule
\end{longtable}



%%% Local Variables:
%%% mode: latex
%%% TeX-engine: xetex
%%% TeX-master: "../Main"
%%% End:
