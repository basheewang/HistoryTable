%% -*- coding: utf-8 -*-
%% Time-stamp: <Chen Wang: 2019-12-26 10:28:37>

\section{仁宗\tiny(1022-1063)}

\subsection{生平}

宋仁宗趙禎(1010年5月30日-1063年4月30日),北宋第四代皇帝(1022年3月23日-1063年4月30日在位)。初名受益,宋真宗的第六子,生母李宸妃。其生于大中祥符三年四月十四日,大中祥符七年(1014年)受封為庆国公,八年(1015年)受封為寿春郡王;天禧元年(1017年)进中书令,二年(1018年)进封昇王,同年九月立为皇太子,赐名趙禎;乾兴元年(1022年)二月,真宗崩,仁宗即帝位,时年13岁;1023年改年號為天圣;1063年駕崩於汴梁皇宮中,享年53岁,在位41年。民間流傳“狸猫换太子”中的太子就是影射宋仁宗。

宋大中祥符三年(1010年),四月十四日(1010年5月30日)趙受益出生(後改名趙禎),是宋真宗趙恒第六子,母為李宸妃。趙恒所寵信的美人劉氏(章獻明肅皇后)無子,趙恒便對外聲稱趙受益為劉氏所生。

乾興元年(1022年),二月十九日,真宗趙恒逝世,趙禎即位,十二歲時由養母刘太后垂帘听政。在刘太后的主导下,他放弃了自己宠爱的后宫张氏,于天圣二年(1024年)立郭氏为皇后。

明道二年(1033年)太后听政十一年後病卒,23歲的仁宗始亲政。

在位期间最主要的军事冲突在於西夏,夏景宗李元昊即位后改变其父夏太宗李德明国策,展开宋夏战争,延州、好水川、定川三战宋军皆有失利之处,韓琦、范仲淹更在好水川之战後被贬。到定川寨之戰,西夏分兵欲直捣关中的西夏军遭宋朝原州(今甘肃镇原)知州景泰的顽强抵抗,全军覆灭,西夏攻占关中的战略目标就此破灭。西夏因连年征战国力难支,最后两国和谈:夏向宋称臣,宋每年赐西夏绢十三万匹、银五万两、茶二万斤,史称“庆历和议”,取得了近半世纪的和平。

辽兴宗時以萧惠陈兵宋境。接著,宋朝與遼朝协议,以增加岁币为条件,维持澶渊之盟的和平协议,史稱重熙增幣。

岁币支出并非沉重负担,比起选择战争的军费,岁币开支无足轻重。宝元元年,陕西出支為1551万;宝元二年宋夏战争后,庆历二年陕西出支為3363万,几近赤字。辽国失去南下劫掠的经济诱因,也是辽宋能维持百年和平的重要因素之一。

皇佑四年(1052年),侬智高反宋,军队席卷广西、广东各地。仁宗任用狄青、余靖率兵南征。皇佑五年,狄青夜袭昆仑关,大败侬智高于归仁铺之戰。次年,侬智高死於大理国,乱平。

仁宗時期,承平日久,經濟快速發展,並出現交子。而宮內治理略有缺失,朝中也有些許小人奸臣,慶曆8年發生疑似後宮爭鬥的坤寧宮事變,事後在眾多忠臣的輔佐之下奸臣的計策未能成功。仁宗時冗官與冗兵特別嚴重,皇祐元年(1049),户部副使包拯即已指出冗官問題:“今内外官属总一万七千三百余员,其未受差遣京官、使臣及守选人不在数内,较之先朝,才四十余年,已逾一倍多矣”,而州郡县的地方官,则更是“三倍其多。”全国军队总计125万9千人,佔赋税十分之七。真宗與仁宗兩朝土地兼并更嚴重,公卿大臣大都佔地千顷以上。仁宗晚年,“势官富姓占田无限,兼并冒伪习以为俗,重禁莫能止焉”,最後“富者有弥望之田,贫者无卓锥之地。”國家財政出现危机,“当仁宗四十二年,号为本朝至平极盛之世,而财用始大乏。”

慶曆新政由范仲淹十大政策揭開序目——明黜陟、抑侥幸、精贡举、择官长、均公田、厚农桑、修武备、减徭役、覃恩信、重命令。但反對勢力龐大,難以推動,一年四個月後便宣布中止。

仁宗一朝對外雖無重大戰爭,對內亦無重大革新,對外需要應對遼夏的軍事威脅。

嘉祐八年(1063年),三月二十九日,於汴梁皇宮駕崩,年五十四歲,死后葬于永昭陵。《宋史》記載,「趙禎駕崩的消息傳出後,京師(汴梁)罷市巷哭,數日不絕,雖乞丐與小兒,皆焚紙錢哭於大內之前」。

仁宗在位41年,是宋朝皇帝中執政最長的一位,生性恭俭仁恕,百司曾奏请扩大苑林,宋仁宗说:「我繼承先帝的園林,尚且覺得十分大,為什麼要這樣做(擴建)呢?」宋仁宗去世后,就連讣告送到辽国時,竟“燕境之人无远近皆哭”,辽道宗耶律洪基痛哭道:“四十二年不识兵革矣”,史載其“惊肃再拜,谓左右曰:‘我若生中国,不过与之执鞭持盖一都虞侯耳!’”

在宋仁宗出生的那天,皇帝賞賜群臣包子。

宋代王明清《揮麈後錄》卷一載:“仁宗母李后,曾夢一羽衣之士,跣足從空而下云:來為汝子。後召幸有娠而生仁宗。仁宗幼年,每穿履襪,即亟令脫去,常徒步禁掖,宮中皆呼為赤腳仙人,蓋古之得道者李君也。”

元朝脫脫《宋史》評價宋仁宗個性仁愛、勤儉,一時朝野上下充滿惻隱善心、行忠義仁厚之政,要不是後代子孫的作為,仁宗之政是可為宋朝三百年的未來奠基:『贊曰:仁宗恭儉仁恕,出於天性,一遇水旱,或密禱禁庭,或跣立殿下。有司請以玉清舊地為御苑,帝曰:「吾奉先帝苑囿,猶以為廣,何以是為?」燕私常服浣濯,帷帟衾裯,多用繒絁。宮中夜饑,思膳燒羊,戒勿宣索,恐膳夫自此戕賊物命,以備不時之須。大辟疑者,皆令上讞,歲常活千餘。吏部選人,一坐失入死罪,皆終身不遷。每諭輔臣曰:「朕未嘗詈人以死,況敢濫用辟乎!」至於夏人犯邊,御之出境;契丹渝盟,增以歲幣。在位四十二年之間,吏治若偷惰,而任事蔑殘刻之人;刑法似縱弛,而決獄多平允之士。國未嘗無弊幸,而不足以累治世之體;朝未嘗無小人,而不足以勝善類之氣。君臣上下惻怛之心,忠厚之政,有以培壅宋三百餘年之基。子孫一矯其所為,馴致於亂。《傳》曰:「為人君,止於仁。」帝誠無愧焉。』

王夫之评论宋仁宗“无定志”,指出仁宗親政至去世的三十年間,兩府大臣更迭頻繁,計有的四十多名員都曾多次上任,但也多次被仁宗因小故而撤換,因此官員們的政策都因在位時間不長而無法貫徹實行,因人事改易而引起的頻密政策轉變亦令在下面的官吏和平民無所適從。當時官員亦清楚仁宗這一點,故蔡襄曾在庆历改革之初,仁宗起用歐陽修、余靖及王素為諫官時,就曾提醒仁宗:“朝廷增用谏臣,修、靖、素一日并命,朝野相庆,然任谏非难,听谏为难,听谏非难,用谏为难。三人忠诚则正,必能尽言。臣恐邪人不利,必造为御之说。……愿陛下察之,毋使有好谏之名而无其实。」又曾指出仁宗“宽仁少断”、「不顓聽斷,不攬威權」。


\subsection{天圣}


\begin{longtable}{|>{\centering\scriptsize}m{2em}|>{\centering\scriptsize}m{1.3em}|>{\centering}m{8.8em}|}
  % \caption{秦王政}\
  \toprule
  \SimHei \normalsize 年数 & \SimHei \scriptsize 公元 & \SimHei 大事件 \tabularnewline
  % \midrule
  \endfirsthead
  \toprule
  \SimHei \normalsize 年数 & \SimHei \scriptsize 公元 & \SimHei 大事件 \tabularnewline
  \midrule
  \endhead
  \midrule
  元年 & 1023 & \tabularnewline\hline
  二年 & 1024 & \tabularnewline\hline
  三年 & 1025 & \tabularnewline\hline
  四年 & 1026 & \tabularnewline\hline
  五年 & 1027 & \tabularnewline\hline
  六年 & 1028 & \tabularnewline\hline
  七年 & 1029 & \tabularnewline\hline
  八年 & 1030 & \tabularnewline\hline
  九年 & 1031 & \tabularnewline\hline
  十年 & 1032 & \tabularnewline
  \bottomrule
\end{longtable}

\subsection{明道}

\begin{longtable}{|>{\centering\scriptsize}m{2em}|>{\centering\scriptsize}m{1.3em}|>{\centering}m{8.8em}|}
  % \caption{秦王政}\
  \toprule
  \SimHei \normalsize 年数 & \SimHei \scriptsize 公元 & \SimHei 大事件 \tabularnewline
  % \midrule
  \endfirsthead
  \toprule
  \SimHei \normalsize 年数 & \SimHei \scriptsize 公元 & \SimHei 大事件 \tabularnewline
  \midrule
  \endhead
  \midrule
  元年 & 1032 & \tabularnewline\hline
  二年 & 1033 & \tabularnewline
  \bottomrule
\end{longtable}

\subsection{景祐}

\begin{longtable}{|>{\centering\scriptsize}m{2em}|>{\centering\scriptsize}m{1.3em}|>{\centering}m{8.8em}|}
  % \caption{秦王政}\
  \toprule
  \SimHei \normalsize 年数 & \SimHei \scriptsize 公元 & \SimHei 大事件 \tabularnewline
  % \midrule
  \endfirsthead
  \toprule
  \SimHei \normalsize 年数 & \SimHei \scriptsize 公元 & \SimHei 大事件 \tabularnewline
  \midrule
  \endhead
  \midrule
  元年 & 1034 & \tabularnewline\hline
  二年 & 1035 & \tabularnewline\hline
  三年 & 1036 & \tabularnewline\hline
  四年 & 1037 & \tabularnewline\hline
  五年 & 1038 & \tabularnewline
  \bottomrule
\end{longtable}

\subsection{宝元}

\begin{longtable}{|>{\centering\scriptsize}m{2em}|>{\centering\scriptsize}m{1.3em}|>{\centering}m{8.8em}|}
  % \caption{秦王政}\
  \toprule
  \SimHei \normalsize 年数 & \SimHei \scriptsize 公元 & \SimHei 大事件 \tabularnewline
  % \midrule
  \endfirsthead
  \toprule
  \SimHei \normalsize 年数 & \SimHei \scriptsize 公元 & \SimHei 大事件 \tabularnewline
  \midrule
  \endhead
  \midrule
  元年 & 1038 & \tabularnewline\hline
  二年 & 1039 & \tabularnewline\hline
  三年 & 1040 & \tabularnewline
  \bottomrule
\end{longtable}

\subsection{康定}

\begin{longtable}{|>{\centering\scriptsize}m{2em}|>{\centering\scriptsize}m{1.3em}|>{\centering}m{8.8em}|}
  % \caption{秦王政}\
  \toprule
  \SimHei \normalsize 年数 & \SimHei \scriptsize 公元 & \SimHei 大事件 \tabularnewline
  % \midrule
  \endfirsthead
  \toprule
  \SimHei \normalsize 年数 & \SimHei \scriptsize 公元 & \SimHei 大事件 \tabularnewline
  \midrule
  \endhead
  \midrule
  元年 & 1040 & \tabularnewline\hline
  二年 & 1041 & \tabularnewline
  \bottomrule
\end{longtable}

\subsection{庆历}

\begin{longtable}{|>{\centering\scriptsize}m{2em}|>{\centering\scriptsize}m{1.3em}|>{\centering}m{8.8em}|}
  % \caption{秦王政}\
  \toprule
  \SimHei \normalsize 年数 & \SimHei \scriptsize 公元 & \SimHei 大事件 \tabularnewline
  % \midrule
  \endfirsthead
  \toprule
  \SimHei \normalsize 年数 & \SimHei \scriptsize 公元 & \SimHei 大事件 \tabularnewline
  \midrule
  \endhead
  \midrule
  元年 & 1041 & \tabularnewline\hline
  二年 & 1042 & \tabularnewline\hline
  三年 & 1043 & \tabularnewline\hline
  四年 & 1044 & \tabularnewline\hline
  五年 & 1045 & \tabularnewline\hline
  六年 & 1046 & \tabularnewline\hline
  七年 & 1047 & \tabularnewline\hline
  八年 & 1048 & \tabularnewline
  \bottomrule
\end{longtable}

\subsection{皇祐}

\begin{longtable}{|>{\centering\scriptsize}m{2em}|>{\centering\scriptsize}m{1.3em}|>{\centering}m{8.8em}|}
  % \caption{秦王政}\
  \toprule
  \SimHei \normalsize 年数 & \SimHei \scriptsize 公元 & \SimHei 大事件 \tabularnewline
  % \midrule
  \endfirsthead
  \toprule
  \SimHei \normalsize 年数 & \SimHei \scriptsize 公元 & \SimHei 大事件 \tabularnewline
  \midrule
  \endhead
  \midrule
  元年 & 1049 & \tabularnewline\hline
  二年 & 1050 & \tabularnewline\hline
  三年 & 1051 & \tabularnewline\hline
  四年 & 1052 & \tabularnewline\hline
  五年 & 1053 & \tabularnewline\hline
  六年 & 1054 & \tabularnewline
  \bottomrule
\end{longtable}

\subsection{至和}

\begin{longtable}{|>{\centering\scriptsize}m{2em}|>{\centering\scriptsize}m{1.3em}|>{\centering}m{8.8em}|}
  % \caption{秦王政}\
  \toprule
  \SimHei \normalsize 年数 & \SimHei \scriptsize 公元 & \SimHei 大事件 \tabularnewline
  % \midrule
  \endfirsthead
  \toprule
  \SimHei \normalsize 年数 & \SimHei \scriptsize 公元 & \SimHei 大事件 \tabularnewline
  \midrule
  \endhead
  \midrule
  元年 & 1054 & \tabularnewline\hline
  二年 & 1055 & \tabularnewline\hline
  三年 & 1056 & \tabularnewline
  \bottomrule
\end{longtable}

\subsection{嘉佑}

\begin{longtable}{|>{\centering\scriptsize}m{2em}|>{\centering\scriptsize}m{1.3em}|>{\centering}m{8.8em}|}
  % \caption{秦王政}\
  \toprule
  \SimHei \normalsize 年数 & \SimHei \scriptsize 公元 & \SimHei 大事件 \tabularnewline
  % \midrule
  \endfirsthead
  \toprule
  \SimHei \normalsize 年数 & \SimHei \scriptsize 公元 & \SimHei 大事件 \tabularnewline
  \midrule
  \endhead
  \midrule
  元年 & 1056 & \tabularnewline\hline
  二年 & 1057 & \tabularnewline\hline
  三年 & 1058 & \tabularnewline\hline
  四年 & 1059 & \tabularnewline\hline
  五年 & 1060 & \tabularnewline\hline
  六年 & 1061 & \tabularnewline\hline
  七年 & 1062 & \tabularnewline\hline
  八年 & 1063 & \tabularnewline
  \bottomrule
\end{longtable}


%%% Local Variables:
%%% mode: latex
%%% TeX-engine: xetex
%%% TeX-master: "../Main"
%%% End:
