%% -*- coding: utf-8 -*-
%% Time-stamp: <Chen Wang: 2019-12-26 10:19:35>

\chapter{北宋\tiny(960-1127)}

\section{简介}

北宋(960年2月4日—1127年3月20日)是中国歷史上宋朝的一个時期,自趙匡胤發動陈桥兵變強迫后周末帝禅让(960年)始,靖康二年(1127年)金兵攻入开封,俘虜宋徽宗、宋欽宗,標誌著北宋結束。北宋建都汴梁(今河南开封市),共历9帝,167年。后宋高宗在应天府稱帝,南宋開始,南宋与北宋合称“两宋”。依據五行相生的順序,後周的「木」德之後為「火」德,因此宋朝以「火」為五行德運,並取紅色為王朝正色。

北宋的最大统治区域包括东、南到海,北以今海河、河北霸州、山西雁门关为界与辽相交;西北以陕西横山、甘肃东部、青海湟水与西夏交界;西南以岷山、大渡河与青藏高原、大理国交界,以广西与越南交界。北宋是面積最少的中原统一皇朝,亦無法統治河西走廊及燕雲十六州。据《太平寰宇记》所载,北宋人口从太平兴国五年(980年)的三千二百五十万增至大观四年(1110年)的一亿一千二百七十五万。

宋朝的開國皇帝是趙匡胤,庙号太祖。他本來是后周的殿前都點檢,掌管禁軍,乃職業軍人。趙匡胤由於戰功卓著,成為了後周世宗的左膀右臂。世宗死後,繼位的恭帝年幼,后通過陈橋兵變奪取後周政權,建立北宋王朝。北宋建立后,赵匡胤加强专制主义中央集权制度,使得中唐以來重臣专权,武将拥兵自重和藩镇割据的基础得以铲除,从而维护专制主义中央集权的统一,有利於社会经济的稳定和发展。但是,北宋初期在赵匡胤的带领下,增设大量的官僚机构,实行一官多职制度,导致机构重叠,官员数量庞杂,财政开支极大,形成“冗官”局面,因此北宋中期机构臃肿,人浮于事,办事效率低下,财政困难。另外,北宋大量地扩充军队,大力削弱武将的兵权,统兵权和调兵权相互牵制,导致兵将分离,军队战斗力下降,对大辽,西夏等少数民族的侵略屡战屡败,形成“冗兵”局面,边疆防备空虚,不利于国防安全。综上所述,加强专制主义中央集权,雖有利于封建国家的统一和社会的安定,能够有效地防止地方分裂和农民起义的发生。但是,过度的中央集权会导致机构臃肿,边疆防备空虚,不利于国防安全,这也为北宋中期的积弱埋下了伏笔。

顯德七年(960年)春節,趙匡胤的黨羽製造遼國進攻的假情報,當時的宰相范質急令趙匡胤率軍北上禦敵。正月初三,趙匡胤抵達陳橋驛,當日夜裏他酣睡之時,被手下軍隊黃袍加身,三呼萬歲,擁戴為天子。後周官僚得知後已知無力回天,只得面對現實。周恭帝被迫遜位。趙匡胤登基成為宋太祖。

建隆二年(961年)七月與開寶二年(969年)十月,宋太祖前後兩次「杯酒釋兵權」,將手握重兵的將軍與地方官吏的武將軍權予以剝奪,委以虛職,並改以文官帶軍,将軍權與財政大權全部集中到中央。宋朝因此得以免於出現唐朝藩鎮割據的局面。但是這也導致地方資源狹少,最終讓宋朝在對外戰爭中屢屢失敗。

北宋的基本国策是「重文轻武」,这個政策對宋朝有利有弊,好處乃在於使北宋初期政治、經濟等各方面都比较安定,尤其是没有宦官专权、地方割据等祸事。即使帝王中多表現平平,但也無損國家的繁榮安定。而壞處則是令北宋在军事上接連挫敗,連同南宋共三百多年,整個宋朝的歷史重心,都是戰事的挫敗和退卻。

宋太祖所面臨的另外一項事業就是統一全國。趙匡胤在與趙普雪夜商討後,決定以先南後北為統一全國之步驟。趙匡胤首先行假途滅虢之計,滅亡了南平和武平。之後又滅亡後蜀、南漢、南唐三國。太祖一心希望統一全國,還設立封樁庫來儲蓄錢財布匹,希望日後能夠從遼朝手中贖買燕雲十六州。開寶九年(976年)八月,太祖再次進行北伐。但十月十九日太祖忽然去世,其弟趙光義立刻即位,傳光義謀殺兄長而篡位,是為燭影斧聲之案,北宋統一事業暫告停止。赵光义即位,是为宋太宗。

太宗穩固統治地位後,繼續國家統一事業,先是割據福建漳、泉兩府的陳洪進及吳越王錢弘俶於978年歸降,其後再於979年滅亡北漢。太平興國四年(979年)五月,太宗不顧大臣反對,趁滅亡北漢的餘威,從太原出發進行北伐。起初一度收復易州和涿州。太宗志得意滿,下令進攻燕京。結果在幽州外高粱河(今北京西直门外)遭遇慘敗。此役之後,宋朝的戰略便轉為被動。雍熙三年(986年),太宗再次北伐,結果又敗,著名的大將楊業也在此役中陣亡。之後四川又爆發王小波、李顺民變。太宗的施政不得不轉為重內虛外。

太宗本人附庸風雅,喜好詩賦,政府也因此特別重視文化事業,宋朝重教之風因此而開。太宗還喜好書法,善草、隸、行、篆、八分、飛白六種字體,尤其是飛白體。連宋朝的貨幣淳化元寶上的字也是太宗親题。

宋太祖有年長兒子,卻由胞弟即位之事頗有疑雲,是為「燭影斧聲」事件,民間也一直傳說趙匡胤是被趙光義謀殺的。為確保政權的合法性,趙光義拋出其母杜太后遺言之說,即「金櫃之盟」(或作「匱」)。金櫃之盟起源于杜太后臨終時召趙普入宮記錄遺命,杜太后稱要趙匡胤死後先傳光義,再傳光美(後改名為廷美),再傳德昭(趙匡胤長子)。這份遺書藏于金櫃之中,因此名為金櫃之盟。雖然有金櫃之盟的说法,但是太宗却先後逼死太祖之子德昭和德芳,又貶黜廷美到房州,兩年後廷美就死于谪所。太宗長子太子元佐也因為同情廷美而被罷為庶人,另一子元僖暴死,最後襄王元侃被立為太子,改名恒。至道三年(997年),太宗駕崩,李皇后和宦官王繼恩等企圖立元佐為帝。幸虧宰相呂端處置得當,趙恒才順利即位,是为真宗。宋朝也開始進入全盛時期。

太宗死後,真宗趙恒接替即位。真宗奉行太宗的黃老政治,無為而治。自從雍熙北伐之後,遼朝就經常在宋遼交界處搶劫殺掠,到景德元年終於演變成大規模侵宋戰爭。宰相寇準力主抗戰,結果真宗親征,宋軍士氣大振,與遼軍相持在澶州城下,剛巧遼大將蕭撻凜中了宋軍的床子弩而亡, 遼軍士氣大挫, 遼見勝利無望, 便與北宋求和。經過幾番交涉,兩國議和成功。和約主要內容是:宋每年給遼絹廿萬匹,銀十萬兩,雙方為兄弟之國。史稱該和約為「澶淵之盟」。

後來,寇準漸漸失寵,最終被罷相。真宗開始信用一佞臣王钦若。王欽若擅長逢迎,深知真宗希望營造天下太平的氛圍,於是極力鼓吹真宗封禪。王欽若本人也與另外一位宰相王旦聯手,在各地製造很多「祥瑞」之象,深得真宗之意。結果真宗在大中祥符元年先後三次封禪。

真宗與皇后劉氏無子。一次真宗偶爾巡幸劉氏的一名侍女李氏,結果李氏與于大中祥符三年產下一子(趙受益),也就是後來的仁宗。後來劉氏與另外一名嬪妃楊氏共同撫養這名孩子。天禧二年中秋,真宗正式封趙受益為太子,並改名為趙禎。乾興元年二月廿日,真宗駕崩。太子趙禎即位,劉皇后被尊為皇太后,在仁宗成年前代理軍國大事。從此開始了劉太后十六年的垂簾聽政時代。

仁宗執政早期一直處在劉氏的陰影之下,直到劉氏死後他才得以施展抱負。仁宗皇后雖是曹氏,但他一直特別寵愛一名張貴妃。但張氏出身低賤,一直未能成為皇后。皇祐六年正月初八,張氏去世。仁宗竟以皇后之禮處理喪事,並追封為溫成皇后,結果出現的一生一死兩皇后,可謂曠古未見。

北宋在仁宗时比较强盛,经济非常繁荣,开创了北宋的最顶峰,不过當時卻要面對兩大難題:朝廷架構膨脹和軍人數目龐大,形成財政上非常拮据,同時因以經濟手段解決邊患,常要向外族贈送,消耗了大量財富。

党項人李元昊於大慶三年(1038年)独立稱帝,建立西夏後,宋夏之間爆發了數年的戰爭,宋軍屢戰屢敗,導致了重熙增幣。爾後仁宗任用范仲淹、呂夷簡、富弼、包拯、韓琦等能臣推行慶曆新政,取得非常好的效果。國家進入建國以來最繁榮的階段。但是一些守舊派人物指稱這些改革派官吏拉幫結夥,互相吹捧,是為朋黨。由於仁宗一向最厭惡結黨營私,這些改革派官僚後來多被貶為地方官。短暫的慶曆新政就此結束。在邊疆上,仁宗任用大將狄青,先後弭平了南蠻儂智高的叛亂和西夏的挑釁。

仁宗死後,接替即位的是英宗趙曙。他是真宗之弟商王趙元份之孫。嘉祐七年被立為皇太子。英宗多病,最初朝政都由曹太后掌管。治平元年五月後,英宗才開始親政。但是英宗親政半個月後就爆發了濮議事件,這場爭論長達十八個月。事件起因是宰相韓琦提請討論關於英宗生父的名分問題。朝中因此分成兩個派別,一派認為應稱英宗生父濮王為皇伯,另外一派則認為應該稱為皇考。最終曹太后下旨,稱英宗之父為皇考。才平息了這場爭論。但總體來說,英宗還是一位有為的君主。他繼續任用前朝能臣,也大膽挖掘新人。英宗也非常重視書籍的編修,《資治通鑑》的寫作就是英宗所提出發起的。

英宗死後,他的長子神宗即位。神宗在位期間,宋朝初期制訂的制度已經產生諸多流弊,民生狀況開始倒退,而邊境上遼和夏又虎視眈眈。神宗因此銳意改革。神宗啟用著名改革派名臣王安石進行朝政改革,將其任命為參知政事。王安石推行的新法包括均輸、青苗、免役、市易、保甲、保馬、方田均稅等。但是,新法的实行遭到了以司馬光為首的保守派對新法強烈反彈。加上全國天災不斷,神宗的新法實行的決心也有所動搖。熙寧七年,华北大旱,一名名叫鄭俠的官員向神宗上呈一幅流民圖,圖中景象慘不忍睹,神宗因此受到極大震撼。第二天神宗就下令暫罷青苗、方田、免役等十八項法令。儘管這些法令不久之後得到恢復,但神宗與王安石之間已經開始不信任。熙寧七年四月,王安石第一次被罷相,出知江寧府。後來變法派中的官員呂惠卿肆意妄為。王安石因此回京複職,但是他依然受到保守派的堅決阻撓。熙寧九年六月,王安石長子去世,王安石借機堅決求退,神宗於十月再次罷免王安石的相位,此後王安石便不問世事。

儘管後人對熙寧新法的評價非常兩極,但無庸質疑,新法的推行效果遠不如王安石預想。新法的實行雖然大大增加了國家的財政收入和耕地面積,但是却增加了平民的負擔。熙寧新法在軍事上的改革也只是頭痛醫頭,腳痛醫腳,軍隊戰鬥力無明顯改善。加上王安石操之過急,將需要很長時間社會演進才能完成的十余項改革在短短數年內全盤推行,使變法陷入了欲速則不達的困境。而且,新法實施到了後期,條文與執行效果相差越來越大,一些措施從利民變成擾民。新法執行過程中用人不當也是最後失去民心的原因,變法派中如呂惠卿、曾布、李定和蔡京等都是人品相當有爭議的人物。有的更被視為小人。對於這次改革,以「大歷史」著稱的史學家黃仁宇評論這次變法:「早我們之前九百年,中國即企圖以金融管制的辦法操縱國事,其範圍與深度不曾在世界裏其他地方提出。但現代金融是一種無所不至的全能性組織力量,它之統治所及概要全部包含,又要不容與它類似的其他因素分庭抗禮。」

王安石被罷相後,神宗繼續改革事業,號為「元豐改制」。元豐改制雖與熙寧變法並稱為「熙豐新法」,但改革力度無法同熙寧變法相提並論。伴隨著國力的增強,神宗將焦點轉移到外患上。他決心消滅西夏。熙寧五年五月,神宗開始西征西夏,取得了很大勝利,也大大鼓舞了神宗的信心。元豐四年四月,西夏發生政變,神宗借此再次征討西夏。結果卻遭到慘敗。神宗因此一病不起。元豐八年正月初,神宗立六子趙傭為太子。而神宗頒佈的新法雖然曾短暫被其母高太后廢,但不久又陸續恢復,不少甚至沿用到南宋時期。

神宗駕崩後,太后高氏垂簾聽政,對剛即位的哲宗趙煦嚴加鉗制。高太后信用以司馬光為首的舊黨,並冷落哲宗,結果引發嚴重的新舊黨爭,是為元祐黨爭。哲宗親政後,貶斥舊党,信用新黨,變法事業因此得到了持續。

哲宗沒有留下子嗣,死後由他弟弟趙佶即位,是為宋徽宗。徽宗專好享樂,對朝政毫無興趣,即位初由向太后垂廉聽政,向太后啟用新舊兩黨人士試圖重整融合朝中和諧,但隨著向太后駕崩而告終。徽宗自幼愛好筆墨、丹青、騎馬等。親政後的徽宗的生活糜爛,喜好逛青樓。還大興土木,聽信道士所言,在開封東北角修建萬歲山,後改名為艮岳。艮岳方圓十餘里,其中有芙蓉池、慈溪等勝地。裏面亭臺樓閣、飛禽走獸應有盡有。徽宗還在蘇州設立應奉局,專門在東南搜刮奇石,是為花石綱,引得民怨沸騰。

徽宗宣和元年(1119年)宋江聚众三十六人起事,活动于山东、河北一带,后受招安。宣和二年(1120年)摩尼教教主方腊率众在歙县七贤村起事(一说在淳安万年乡帮源峒起事),其后攻占杭州,建立了横跨今日江苏、浙江、安徽、江西四省六州五十二县在内的农民政权,次年被宋将韩世忠镇压。

宋徽宗不理朝政,政務都交給以蔡京為首的六賊。蔡京以恢復新法為名大興黨禁,排斥異己。蔡京主政次日,就下達了一個禁止元祐法的詔書。此即謂元祐奸黨案。正直的大臣因此全被排斥出政治中心。徽宗本人好大喜功,當他看到遼國被金國進攻後,便於重和元年春,派遣使節馬政自登州渡海至金。雙方商議兩國共同攻遼,北宋負責攻打遼的南京和西京。滅遼後,燕雲之地歸宋,過去宋朝給遼國的歲幣改繳金國。此即為海上之盟。但宋朝軍隊卻被打得大敗。最後金兵掠去燕京的人口,並克扣營、平、灤三州。虽然宋朝策反三州守将张觉甚至一度收回幽云十六州中的十一州,但宣和七年,金兵分兩路南下攻宋,很快占领三州及幽云各州,直指宋都。趙佶嚇得立刻傳位其子钦宗,逃到江南。欽宗患得患失,在戰和之間舉棋不定。後來在萬般無奈的情況下啟用李綱來保衛東京。雖然一度取得了勝利,但是金朝並未死心,二度南下。靖康元年九月,太原淪陷。十一月,開封外城淪陷,金軍逼迫欽宗前去議和。閏十一月卅日,欽宗被迫前去金營議和,三日後返回。金人要求索要大量金銀。欽宗因此大肆搜刮開封城內財物。開封城被金軍圍困,城內疫病流行,餓死病死者不在少數。金人又迫使钦宗亲往议和,将其扣押,再以放还为条件勒索财物,但宋朝已拿不出其所需的财物。金人不曾攻克开封内城,但宋将范琼将徽宗及皇室成员押往金营。靖康二年二月六日,徽、欽二帝被金太宗所廢,貶為庶人。金朝掠走北宋宫厅几乎所有皇室成员和财宝后,建立了一個名為「大楚」的傀儡政權,另立張邦昌为帝,稍后于1130年建立一个大齐政权,立刘豫为帝,史称“刘齐”。徽欽二宗及其兄弟全家被金人掠到五國城,史稱「靖康之難」,仅钦宗弟康王赵构在外募兵勤王得以幸免。徽宗被封為昏德公,欽宗被封為重昏候。最後兩人皆客死異鄉五国城。

儘管徽宗在朝政上毫無建樹,但毋庸置疑,他在書畫上的造詣無與倫比。徽宗的書法和繪畫都在中國藝術史上有重要地位。徽宗獨創瘦金體,並重視書畫事業。翰林書畫院的地位大幅提高,著名畫家,清明上河圖的作者張擇端就是其提拔。就連其子趙構也受到薰陶,成為傑出書法家。

赵构后来称帝建立南宋,但与北宋相比失去了河北、河南、山东,疆域大减。

宋朝集權中央,強榦弱枝,地方官員都是由中央派遣,不得常駐。地方全部財富轉運到中央去,地方更無存儲。平常就很艱苦,臨時地方有事,更是不可想象。所謂宋代的中央集權,是軍權集中,財權集中,而地方則日趨貧弱。地方貧弱,所以金兵南下,只要首都汴京(開封)攻佔,全國瓦解,難以抵抗。唐朝的安史之亂,其軍力並不比金人弱,唐朝兩都(首都長安,東都洛陽)被攻破,可是州郡財富厚,每一座城池,都存有幾年的米,軍裝武器都有儲積,所以到處可以到處作戰。宋朝則把財富兵力都集中到中央,不留一點在地方上,所以首都一被攻陷,全國隨即瓦解。。

宋代考試制度,大體也沿襲唐代,但宋代科舉所獲影響,卻與唐代不同。第一是唐代門第勢力正盛,在那時推行考試,應考的還是有許多是門第子弟。門第子弟在家庭中有家教熏染,並已早懂得許多政治掌故,一旦從政,比較有辦法。如是積漸到晚唐,大門第逐步墮落,應考的多數是寒窗苦讀的窮書生。他們除卻留心應考的科目,專心在文選詩賦,或是經籍記誦外,國家並未對他們有所謂教育。門第教訓也沒有了,政治傳統更是茫然無知。於是進士輕薄,成為晚唐一句流行語。因循而至宋代,除卻呂家韓家少數幾個家庭外,門第傳統全消失了。農村子弟,白屋書生,偏遠的考童,驟然中試,進入仕途,對實際政治自不免生疏格,至於私人學養,也一切談不上。

其次,唐代考試,有公卷通榜之制。所謂公卷,是由考生把平日詩文成績,到中央時,遍送政府中能文章有學問的先進大僚閱看。此輩先進,看了考生平日作品,先為之揄揚品第,在未考以前,早已有許多知名之士,獲得了客觀的地位。通榜是考後出榜,即據社會及政府先輩輿論,來拔取知名之士,卻不專憑考試之一日短長。甚至主考官謙遜,因其不了解這一次考場中的學術公評,不自定榜,而倩人代定榜次,並有倩及應考人代定,而應考人又自定為榜首狀元的。但此等事在當時反成嘉話,不算舞弊。本來考試是為國家選拔真才,明白的此項制度之主要精神與本原意義,又何必在細節上一一計較。但有些人便要借此制度之寬大處作弊,於是政府不免為要防弊而把制度嚴密化。這是一切制度皆然的。但制度逐步嚴密化,有時反而失卻本義,而專在防弊上著想。宋代考試制度,是遠比唐代嚴格了,那時則有糊名之制,所憑則真是考試成績。其實考試成績,只是一日之短長,故有主考官存心要錄取他平日最得意的門生從學,而因是糊名,尋覓不出該人之卷,而該人終於落第的。如是則考試防制嚴了,有時反得不到真才。

考試只能選拔人才,卻未能培養人才。在兩漢有太學,在唐代有門第,這些都是培養人才的。社會培養出人才,政府考試始有選擇。宋人頗想積極興辦教育,這是不錯的。但此非咄嗟可望。第二是想把考試內容改變,不考詩賦,改考經義。這一層用意亦甚是。人人學詩賦,風花雪月,用此標準來為政府物色人才,終不是妥當辦法。但改革後卻所得不償所失,考經義反而不如考詩賦。王荊公因此嘆息,說本欲變學究為秀才,不料轉變秀才為學究。

而且,恩蔭補官、任子太濫,是宋代一大弊政;科舉出身輕視恩蔭出身,補官當中並不一定沒有人才,但也使的宋代冗官過多,經濟壓力沉重。

宋代軍隊分兩種,一稱禁軍,一稱廂軍。在唐末五代時,藩鎮驕橫,兵亂頻仍,當時社會幾乎大家都當兵,讀書人並不多見。開頭軍隊還像樣,以後都變成了老弱殘兵。軍隊不能上陣打仗,便把來像罪犯般當勞役用。其時凡當兵的,都要面上刺花字,稱為配軍,防他逃跑。如《水滸傳》裡的宋江、武松一類人,臉上刺了字,送到某地方軍營中當兵做苦工,人家罵他賊配軍,這是遠從五代起,直到宋朝,亦沒有能徹底改。這樣的軍隊,实际战斗力有限。其實這些軍隊,在漢是更役,在唐則是庸。而宋代之所謂役,在漢代卻是地方自治之代表。這些兵,並不要他們上陣打仗,只在地方當雜差。地方政府有什麼力役,就叫他們做。照理,宋代開國第一件該做的事,便是裁兵復員,而宋代卻只照上面所說的這樣裁,至於復員則始終復不了。這也因宋代得天下,並未能真的統一了全國,他們的大敵遼國,已經先宋立國有了五十多年的歷史。

所謂燕雲十六州,早被石敬瑭割讓予遼人。當時遼寧乃及山西、河北的一部分疆土,都在遼人手裡。北方藩籬盡撤,而宋代又建都開封,開封是一片平地,豁露在黃河邊。太行山以東盡是個大平原,騎兵從北南下,三幾天就可到黃河邊。一渡黃河,即達開封城門下。所以宋代立國時沒有國防的。倘使能建都洛陽,敵人從北平下來,渡了河,由現在的隴海線向西,還需越過鄭州一帶所謂京索之山,勉強還有險可守。若從山西邊塞南下,五台山雁門關是那裡的內險,可算得第二道國防線。要一氣沖到黃河邊,還不容易。所以建都洛陽還比較好。若能恢復漢唐規模,更向西建都長安,那當然更好。但宋太祖為何不建都洛陽,長安,二偏要建都開封呢?這也有他的苦衷。因為當時國防線早經殘破,燕雲失地未復,他不得不養兵。養兵要糧食,而當時的軍糧,也已經要全靠長江流域給養。古代所謂大河中原地帶,早在唐末五代殘破不堪,經濟全賴南方支持。由揚州往北有一條運河,這不是元以後的運河,而是從揚州往北沿今隴海線西達開封的,這是隋煬帝以來的所謂通濟渠。米糧到了開封,若要再往洛陽運,那時汴渠已壞。若靠陸路運輸,更艱難,要浪費許多人力物力。宋代開國,承接五代一般長期混亂黑暗殘破的局面,沒有力量把軍糧再運洛陽去,長安一片荒涼,更不用提。為要節省一點糧運費用,所以遷就建都在開封。宋太祖當時也講過,將來國家太平,國都還是要西遷的。

在當時本有兩個國策,一是先打黃河北岸,把北漢及遼打平了,長江流域就可不打自下。這個政策是積極進取的,不過也很危險。假使打了敗仗,連退路都沒有。一個是先平長江流域,統一了南方,再打北方,這個政策比較持重穩健。宋太祖採了第二策,先平南方,卻留著艱難的事給後人做。太宗即位,曾兩次對遼親征,但都打了敗仗。一次是在今北平西直門外直去西山頤和園的那條高粱河邊上交戰,這一仗打敗,他自己中了箭,回來因創傷死了。在歷史上,這種事是隱諱不講的。只因宋代開國形勢如此,以後就不能裁兵,不能復員,而同時也不敢和遼國再打仗。因為要打就只能勝,不能敗。敗了一退就到黃河邊,國本就動搖。在這種情形下,宋代就變成養兵而不能打仗,明知不能打仗而又不得不養兵。更奇怪的,養了兵又不看重他們,卻來竭力提倡文治。想把這些兵隊來抵御外患,一面提倡文治,重文輕武,好漸漸裁抑軍人跋扈,不再蹈唐末五代覆轍。因此上養兵而癒不得兵之用,以後就癒養癒多。《水滸傳》說林沖是八十三萬禁軍教頭,實際上太祖開國時只有二十萬軍隊,太宗時有六十六萬,到仁宗時已經有了一百二十五萬。所以王荊公變法行新政,便要著手裁兵。裁兵的步驟,是想恢復古代民兵制度,來代替當時的傭兵。但民兵制度,急切未易推行到全國,遂有所謂保甲制,先在黃河流域一帶試行。保甲就是把農民就地訓練,希望臨時需要,可以編成軍隊,而又可免除養兵之費。

宋代的國防精神是防御性的,不敢主動攻擊,用意始終在防守。把募兵制度與長期的防守政策相配合,這卻差誤了。宋人最怕唐末五代以來的驕兵悍卒,但宋代依然是兵驕卒悍。國家不能不給他們待遇,而且須時時加優,否則就要叛變。政府無奈何,加意崇獎文人,把文官地位提高,武官地位抑低。節度使閑來沒事做,困住在京城,每年冬天送幾百斤薪炭,如是種種,把他們養著就算。養了武的又要養文的,文官數目也就逐漸增多,待遇亦逐漸提高。弄得一方面是冗兵,一方面是冗吏,國家負擔一年重過一年,弱了轉貧,貧了更轉弱,宋代政府再也扭不轉這形勢來。因養了許多無用兵,使宋代成為一個因養兵而亡國的朝代。

關於國防資源問題,這也是宋代一個最大的缺憾。在中國只有兩個地方出產馬。一在東北,一在西北。一是所謂薊北之野,即今熱察一帶。一是甘涼河套一帶。一定要高寒之地,才能養好馬。養馬又不能一匹一匹分散養,要在長山大谷,有美草,有甘泉,有曠地,才能成群養,才能為騎兵出塞長途追擊之用。而這兩個出馬地方,在宋初開國時,正好一個被遼拿去,一個被西夏拿去,都不在宋朝手裡。與馬相關聯的尚有鐵,精良的鐵礦,亦都在東北塞外,這也是宋代弱征之一。可馬在溫濕地帶飼養不易,很容易生病死亡,因此馬匹也成了宋代國防上所遭遇的大難題。

自宋遼兩國講和以後,宋朝的國防形勢是很可憐的。兩國既不正式開戰,也不好正式布置邊防。只獎勵民間種水田,多開渠道,於渠旁多植榆楊。萬一打仗,可以做障礙,稍稍抵御遼人之大隊騎兵。

宋真宗大中祥符五年(1012年)引进占城稻后,亲自推广占城稻,从福建取占城稻三万斛,分给江淮、两浙种植,这是中国历史上水稻的一次大规模引种。宋神宗熙宁年间,进行大规模引浊放淤、改良农田,全国兴修水利10793处,溉田3600多万亩。长江下游出现稻麦一年二熟制,提高农田单位产量。农具也有很大改进,出现了拔秧工具—秧马。发现石灰、硫黄、钟乳粉等矿物可以作为农田肥料施用。油菜已成为江南地区的主要油料作物。南方各地普遍栽种茶树。川蜀、两广、两浙、福建是著名的甘蔗种植区。秦观作《蚕书》,是为中国现存最早的蚕业著作。

北宋全盛时,麻布产量比盛唐时期多二倍。棉织品在全部纺织品中的比重有所上升。两浙、川蜀地区成为丝织业中心。宋代制瓷窑户几乎遍布全国各地。定州(今属河北)定窑、汝州(今河南临汝)汝窑、颍昌府阳翟(今河南禹州)的钧瓷、饶州(今江西波阳)景德镇窑各有特色。随着雕版印刷业的兴盛,纸张的需要量激增,促使民间造纸业迅速发展。宋代造纸技术比前代大幅度提高。北宋已大量开采石炭(煤),用于冶金和民间日用燃料。在军事和医药上都已利用石油。

北宋工商业发达,北宋后期出现了总计50座户口在10万户以上的大城市。宋真宗天禧五年(1021年)川陕四路地区出现了世界上最早的纸币─—交子。内河航运和造船也极其发达。遍布各地的驿站网,除官府文书外,还可邮寄私人信件。北宋对外贸易也盛况空前。在四川出现最早的茶马互市。在海上,北自登州、密州,南到泉州、广州,总共开放了12个官方对外通商港口,设立市舶司,管理对外贸易。从泉州;广州等港口出发的船只,远达非洲的埃及和当时的东非诸国,将宋国瓷器和丝绸等工业产品出口到西洋,换回香料、药材、象牙及宝石等奢侈品。海上交通业的发达,也拉动了造船业的发展。沈括写道:“国初,两浙献龙船,长二十余丈,上为宫室层楼,设御塌以备游幸。”

宋代賦稅制度,大體也是由唐代兩稅制沿下。只講一點較重要的。本來兩稅制度,把一切賦稅項目,都歸並了,成為單一的兩稅。宋代之差役,也如秦代之戍邊,都是由前面歷史沿襲下來,政府沒有仔細注意,而遂為社會之大害。

北宋时代的儒学复兴和重视文化正常,因而大力提拔文人,使文人得到自由发展的空间,其中较著名的文人有王安石、范仲淹、苏洵、苏轼、苏辙、司马光、欧阳修等人,他們擅于写词,並达到极高水平,与唐诗成为中国古典文学艺术的瑰宝。書法也是突出,國立故宮博物院典藏之黃庭堅、蘇軾、米芾、蔡襄並列為北宋四大書家。张择端的《清明上河图》,这幅长卷通过描绘汴京的风物,使近六百人跃然纸上,成为中国绘画史上不朽的佳作。图画中显現出北宋宣和年间汴梁城经济和商业发达的景象,故有人认为宋朝時資本主義開始萌芽。

中国四大发明之一的活字印刷术正式被发明,改良了書籍印刷的技術,對後世文化有很大推動作用。黑色火药在宋初首次应用于军事。指南针已经在北宋军队中用于辨明方向。北宋文人沈括撰寫《夢溪筆談》,乃中國歷史上著名的科學著作之一。


%% -*- coding: utf-8 -*-
%% Time-stamp: <Chen Wang: 2019-12-26 10:22:23>

\section{太祖\tiny(960-976)}

\subsection{生平}

宋太祖趙匡胤(927年3月21日-976年11月14日),宋朝開國皇帝,祖籍涿郡保州保塞縣(今河北省保定市清苑区),父親趙弘殷,母親杜氏。後唐明宗天成年間(927年3月21日)生於後唐洛陽夾馬營(今河南省洛陽市瀍河回族區東關)。後漢時,趙匡胤於後漢隱帝在位期間投奔郭威,之後郭威篡漢建立後周,是為周太祖;而趙匡胤則得任東西班行首,始入宦途。959年,後周世宗於北征回京後不久駕崩,逝世前任命趙匡胤為殿前都點檢,執掌殿前司諸軍。隔年(960年)元月初一,北漢及契丹聯兵犯邊,趙匡胤受命防禦。初三夜晚,大軍於京城開封府(今河南省開封市)東北二十公里的陳橋驛(今河南省封丘縣陳橋鎮)發生政變,將士於隔日清晨擁立趙匡胤為帝,史稱「陳橋兵變」。大軍隨即回師京城,後周恭帝禪位,趙匡胤登基,建國號「宋」,是為「宋太祖」,年號為建隆,建立北宋。北宋和之后宋高宗建立的南宋國祚合计長達319年,在中國古代歷代王朝中延續時間僅次於兩漢。

太祖在位期間,致力於統一全國。依據宰相趙普的「先南後北」策略,先後滅荊南、湖南、後蜀、南漢及南唐等南方割據政權,至宋太宗在位期間,迫使吳越、清源軍納土歸降,滅北漢,方才完成一統;太祖於961年及969年先後兩次「杯酒釋兵權」,解除禁軍將領及地方藩鎮的兵權,解決自唐朝中葉以來藩鎮割據的局面;設立「封樁庫」貯藏錢帛布匹,期能贖回被後晉高祖石敬瑭獻給契丹的燕雲十六州,但事未成而逝世。

976年11月14日,太祖逝世,葬于永昌陵,得年50歲,在位17年。由其胞弟趙光義繼承帝位,是為宋太宗,並於同年隨即改元。由於北宋中期的筆記《續湘山野錄》記載了「燭影斧聲」事件,暗示趙匡胤是由太宗所加害。加上太祖死後,帝位非由其子繼承,而是由弟登基垂統,違反中國長子繼承的傳統,雖然太宗即位後延续了太祖許多的執政措施,但欲蓋彌彰,与太祖有关的皇室成員亦相繼離奇亡故。此后的皇帝亦由太宗的子孫继承,直至宋孝宗才回归太祖一脈,使太祖的死因並不單純,成為千古之謎。

唐明宗天成二年二月十六日(927年3月21日),趙匡胤誕生於洛陽夾馬營(今河南省洛陽市瀍河回族區東關爽明街北段),是趙弘殷次子,母親杜氏。依《宋史·太祖本紀》記載,趙匡胤為“涿郡人”,然而此處應是依涿郡趙氏的郡望所指,實則屬籍保州(今河北省保定市清苑区)。年長後離家遊歷,投奔復州防禦使王彥超,不被接納。繼而往依隨州刺史董宗本,卻遭其子董遵誨憑勢欺侮,趙匡胤於是辭別北上。948年,趙匡胤投奔後漢樞密使郭威帳下,隨軍征討李守貞。951年,郭威稱帝,國號「周」,是為「周太祖,趙匡胤得補任東西班行首,加拜滑州(今河南省安陽市)副指揮使。953年,郭威養子柴榮擔任開封府尹,調趙匡胤至京師(今河南省開封市)任開封府馬直軍使。

954年初,周太祖逝世,外戚柴榮繼位為周世宗,趙匡胤調任中央禁軍。北漢及契丹隨即趁喪入侵,邊報頻仍,世宗決定親征,趙匡胤任侍衛將領隨行護駕。4月,周軍與北漢軍大戰於高平(今山西省晉城市高平市),趙匡胤率二千人參與戰鬥,並身先士卒,親自衝鋒,北漢軍披靡,七千餘人投降。時正颳南風,周軍因風奮擊,北漢軍大敗,周軍趁勝圍攻河東城,焚燒城門,趙匡胤於戰鬥中遭流矢射中左臂。高平之戰後,世宗拔擢趙匡胤為殿前司都虞候,領嚴州(今浙江省湖州市德清縣)刺史,並命趙匡胤整頓禁軍,汰羸除弱,更招募天下壯士至京師,設立殿前諸班,由趙匡胤選擇精銳將士充之,後周軍隊自始獨霸。

956年初,周世宗親征南唐,大軍圍攻壽城(今安徽省六安市壽縣),南唐駐紮一萬兵馬於塗山(今安徽省蚌埠市)下,停泊舟艦於淮河,世宗命趙匡胤往攻。3月,趙匡胤伏軍渦口(今安徽省蚌埠市懷遠縣東北),派遣一百餘名騎兵襲擊南唐軍營,交戰後佯敗逃遁,南唐軍追擊,至渦口而後周伏兵四起,南唐軍大敗,斬唐將何延錫,奪得五十餘艘戰艦。南唐另有十五萬援軍駐紮清流關(今安徽省滁州市),趙匡胤受命征討,攻克清流關、滁州城,並生擒唐將皇甫暉。5月,南唐軍攻揚州,趙匡胤屯兵六合支援。揚州守將韓令坤懼而欲退,趙匡胤下令軍中:「揚州士兵過六合者,砍斷雙腿!」韓令坤才決心固守。南唐又遣齊王李景達率兵支援,距離六合二十里處駐紮。諸將請主動攻擊,趙匡胤卻以逸待勞,俟南唐軍發動攻擊,方才領兵迎擊,南唐軍大敗,斬殺五千餘人,投江溺斃的不計其數。戰後趙匡胤因功受封殿前都指揮使,不久又加封定國軍節度使。

957年初,世宗再度親征南唐,4月,駐紮紫金山,命趙匡胤率領禁軍殲滅壽州外圍南唐援軍,趙匡胤連拔數寨,斬獲三千餘人,並切斷南唐援軍通道,壽州因此無援,於是投降。還京後,趙匡胤因功加拜義成軍節度、檢校太保,仍擔任殿前司都指揮使。年底,柴榮三征南唐,進攻濠州,城外有水灘環繞,南唐軍在其上設立柵欄防守,世宗遂命將士騎駱駝渡河,趙匡胤則領兵騎馬而渡,率先攻破南唐水寨,焚燒南唐戰艦七十餘艘,斬殺兩千餘人,攻陷濠州。其後陸續攻陷泗州(今安徽省宿州市泗縣)、楚州(今江蘇省淮安市淮陰區),南唐江北之地盡為周有。隔年5月,趙匡胤改封忠武軍節度使。

959年3月,世宗北征契丹,趙匡胤任水陸都部署,先期抵達瓦橋關(今河北省保定市雄縣),守將姚內斌投降。6月,世宗染病不適,命車駕回京。7月23日,世宗任命趙匡胤為殿前司都點檢,為殿前禁軍最高統帥。27日,世宗因病駕崩,年僅七歲的梁王柴宗訓繼位為周恭帝,改歸德軍節度使,檢校太尉。

960年正月初一(1月31日),北方邊境鎮州(今河北省石家莊市正定縣)、定州(今河北省定州市)守將急報,稱北漢與契丹合勢,聯兵入侵,周恭帝命趙匡胤率宿衛禁軍北上迎敵。

初二(2月1日),京城流言四起,謠傳大軍出征之日將策立都點檢做天子,士民不安,準備逃匿。

初三(2月2日),趙匡胤率大軍自京城東北邊的愛景門出城,士卒紀律嚴明,民心稍安。大軍當晚駐紮於京城東北二十公里的陳橋驛(今河南省封丘縣陳橋鎮),將士們謀劃擁立趙匡胤為天子,由都押衙李處耘轉告供奉官都知趙光義,兩人隨即與歸德軍節度掌書記趙普商討。談論中,諸將突然闖入,眾說紛紜,趙普及趙光義以理勸退,諸將稍稍引去,不久又復趨集,露白刃要脅,趙普於是同意,派人前往京城安排內應,將士們則環列於趙匡胤的大帳,等待天明。

初四(2月3日)清晨,趙匡胤因酒醉尚未清醒,將士於大營四周鼓譟喧嘩,趙普及趙光義入帳稟告趙匡胤,趙匡胤驚起,披衣而出,將領們持兵器羅立於庭。趙匡胤不及說話,諸將士即將黃袍披在趙匡胤身上,紛紛下拜高呼萬歲。趙匡胤堅拒,眾將士不聽,迫其上馬南行回京。趙匡胤見勢不可免,便攬轡對諸將說:「你們貪圖富貴,立我為天子,就必須聽從我的命令,不然,我不當這個皇帝。」眾將皆下馬跪地說:「唯命是從!」趙匡胤便下令不得侵擾後周皇帝、太后及群臣,也不得擅自擄掠及搜刮府庫,違者族誅,眾將應諾。趙匡胤率大軍自仁和門入京城,下令將士解甲歸營,趙匡胤則回殿前都點檢公署,脫下黃袍。不久,眾將逼擁司徒,同中書門下平章事,參知樞密院事范質、參知樞密院事,尚書省右僕射王溥、同中書門下平章事,刑部尚書魏仁浦至公署,迫其表態。范質等無可奈何,只得下拜,高呼萬歲。趙匡胤於是登崇元殿行禪讓禮,登基稱帝,建國號「大宋」,改元建隆,大赦天下。

上述記載見於《宋史》、《續資治通鑑長編》、《涑水記聞》等史書及筆記中,皆言趙匡胤是為眾所逼,被迫稱帝,事前並不知情。然而現代史家依史料記載之疑點及矛盾推論,普遍認為「陳橋兵變」是趙匡胤及其親信幕僚所預謀策劃的軍事政變。[需要更好来源]

960年5月,昭義節度使李筠據潞州(今山西省長治市)叛變,攻陷澤州(今山西省晉城市),並與北漢合兵,率眾南下。太祖遣侍衛親軍司馬步軍副都指揮使石守信、殿前司副都點檢高懷德往討,又派昭化軍節度使慕容延釗、彰德軍節度使王全斌前往合兵,於長平之戰大敗李筠,斬殺三千餘人。6月,太祖決定親征,與石守信等將領會合,於澤州之南大敗李筠及北漢的三萬聯軍,三千餘名北漢援軍投降,宋軍將其全數坑殺。李筠退守澤州城,宋軍隨即攻破澤州,李筠自焚而死。7月,宋軍兵圍潞州,李筠長子李守節舉城出降;同年(960年)10月,淮南節度使李重進據揚州叛變,太祖遣石守信、義成軍節度使王審琦、宣徽北院使李處耘、保信節度使宋偓率軍往討。11月,太祖再度親征,兵至大儀鎮(今江蘇省揚州市儀徵市),石守信遣使奏請太祖親臨觀看揚州城破,太祖隨即趕赴揚州城下,頃刻城破,李重進自焚而死,黨羽親信數百人皆搜捕斬殺。

964年,太祖下詔令州、郡所收稅賦,除地方日常行政經費外,其餘上繳中央,不得私留;置轉運使,掌管地方財政權,並檢查賦稅情形,以供上繳朝廷及地方支用。轉運使設通判官,到任時核對帳簿,並得查考民情、官吏違法情事上報朝廷,有審計、監察之權,以此削弱地方財政權;太祖並下詔全國之茶、酒、鹽由國家專賣,官吏與百姓不得私自販售,最重處死,國家因此收入大增;太祖派兵滅亡後蜀後,為儲備錢財以應急之用,於宮中設置「封樁庫」,中央政府年度財政盈餘全數納入,並打算儲積至三、五十萬後,以這筆錢贖回遭后晉高祖石敬瑭獻給契丹的燕雲十六州,但不久即逝世,贖地一事無疾而終。

太祖於平定湖南後,下令於其地取消「支郡」,使原屬藩鎮節度使管轄之州、郡獨立,直屬中央,至宋太宗在位時,全國「支郡」全部廢除;有鑑於唐至五代的藩鎮之患,太祖以朝廷文臣出守地方,稱「權知軍州事」,執行州、郡之軍事權及行政權,並置「通判」為其副官,地方的民政、財政、司法等事務由知州及通判共同簽署始得施行,通判並得監察主官的不法及瀆職情事,上報朝廷,以此分割守臣之權。

962年4月,太祖下詔各地死刑案件須上報中央,由刑部複審,以杜絕藩鎮枉法殺人的惡習;963年1月,太祖下令每縣設置一「縣尉」,負責地方治安,剝奪原由鎮將任命親信任職之權,以此制衡鎮將,使其勢力僅限所駐城郭而已;973年8月,太祖下詔改各州「馬步院」為「司寇院」,設司寇參軍,選派新科及第進士與選人資序相當者擔任,剝奪藩鎮對地方一般案件的審判權,解決藩鎮武將審理案件時有法不遵的現象。

961年7月,太祖與石守信、王審琦等禁軍將領宴飲,酒過數巡後,對他們說:「我如果沒有你們,就沒有今日的地位,所以對你們的恩德無日或忘。但當天子太過艱難,還不如做節度使來得快樂,因此我每夜都睡不安穩。」石守信等人問其故,太祖答說:「道理很簡單,皇帝這個位置,誰不願意坐呢?」眾將聽後十分惶恐,皆跪地磕頭說:「陛下何出此言?如今天命已定,誰還敢有二心?」太祖說:「這話不對!你們雖然沒有異心,但如果你們屬下貪圖富貴呢?一旦將黃袍加在你們身上,你們即使不想當皇帝,也不行了。」眾將聽後皆涕泣磕頭說:「臣等愚昧無知,沒想到這些,請陛下可憐我們,指示一條生路。」太祖說:「人生如同白駒過隙,晃眼即逝,所謂追求富貴之人,不過想多累積些金銀財寶,盡情享樂,使子孫不再貧乏而已。你們何不放棄兵權,出守藩鎮,買幾塊好地、幾間好房,為子孫留下永久的產業;多收些歌兒舞女,每日與她們飲酒取樂,以終天年。我再與你們約定聯姻,君臣之間,不相互猜疑,上下相安無事,這不是件好事嗎!」眾將皆下拜說:「陛下為臣等設想周到,是我們的再生父母。」隔日,石守信、高懷德、王審琦、張令鐸等將領皆稱病請辭禁軍官職,太祖隨即批准,使其出鎮地方為節度使,所遺職缺不再補實。

969年12月,太祖在御花園與進京述職的地方藩臣宴飲,酒酣之際,從容說道:「你們都是國家的元勳宿將,長久在藩鎮做官,公務繁忙,這不是朕優禮賢士的本意。」鳳翔軍節度使王彥超上前奏道:「臣本來就無功勞勳績,卻久受皇恩榮寵,十分慚愧。如今臣已衰老,乞求陛下賜臣退休,歸老園田,這是臣最後的願望。」安遠軍節度使武行德、護國軍節度使郭從義、定國軍節度使白重贊、保大軍節度使楊廷璋等將領卻仍競相自陳往昔攻戰之功勞及經歷之艱辛,太祖便說:「這是前代的事了,還有什麼好說的。」隔日,太祖下詔,免去其節度使職,授以「環衛官」虛銜,留任京師,改以朝臣出守諸郡,徹底避免自唐末、五代以來的藩鎮割據問題。

太祖鑑於五代時期藩鎮武將權力過重,以致國家混亂,建國後採取「重文抑武」政策,凡國家高階實權職位均由文官擔任,貶抑武官,以防籓鎮專權。但有史家認為此政策使宋朝積貧積弱、軍力不振。

太祖初即帝位,便與宰相趙普「雪夜定策」,決定「先南後北」統一全國的順序。

荊南:962年10月,武平(湖南)節度使周行逢病逝,傳位予十一歲的兒子周保權,衡州刺史張文表不服叛變,攻陷潭州(今湖南省長沙市),周保權派楊師璠往討,並求援於荊南及宋。963年2月3日,太祖趁機派遣山南東道節度使慕容延釗、樞密副使李處耘出兵湖南,討伐張文表,同時借道荊南。3月,楊師璠擊敗張文表,將其斬首。而荊南節度使高繼沖質疑宋軍借道意圖,便派人以犒師為名前往宋軍大營探查虛實。李處耘對使人熱情款待,卻暗中派數千騎兵急馳江陵,趁高繼沖出迎時佔據江陵城,高繼沖懼而投降,荊南割據政權滅亡。

湖南:張文表之亂雖平,宋軍仍繼續南下,武平節度使周保權派兵防禦。宋軍隨即於三江口(今湖南省岳陽市)大敗周保權軍,攻陷岳州,獲戰船七百餘艘,斬殺四千餘人。4月,於澧州(今湖南省常德市澧縣)以南擊潰周保權部屬張從富,都城朗州(今湖南省常德市)大懼,焚燒城池,居民逃亡山谷。4月6日,宋軍攻入朗州,擒斬張從富,於寺院中俘獲周保權,湖南割據政權滅亡。

後蜀:964年12月8日,太祖下詔兵分兩路共六萬大軍討伐後蜀,北路軍命忠武節度使王全斌為主帥、武信節度使崔彥進為副官、樞密副使王仁贍為監官,東路軍以寧江節度使劉光義為副官、樞密承旨曹彬為監官。後蜀皇帝孟昶則遣知樞密院事王昭遠禦敵。965年2月,劉光義於東路擊殺蜀將南光海,兵臨夔州(今重慶市奉節縣東)。後蜀夔州守將高彥儔的部將武守謙違令出戰,大敗而逃,宋軍趁亂入城,高彥儔力戰不敵,自焚而死,夔州淪陷,萬、施、開、忠、遂等五州相繼投降;後蜀軍主帥王昭遠率兵於北路迎戰王全斌,三戰三敗,退守劍門關(今四川省廣元市劍閣縣)。2月4日,王全斌攻入利州(今四川省廣元市利州區),獲軍糧八十萬斛,不久轉往劍門關,大敗後蜀軍,俘虜蜀將王昭遠、趙崇韜,攻陷劍州。孟昶聞之大懼,決定遣使請降。2月11日,孟昶派使者至宋軍營前遞降書,後蜀滅亡。

然而蜀地雖已收復,征蜀大軍卻在主帥王全斌等人的縱容下,任意燒殺劫掠、為非作歹。而王全斌與崔彥進、王仁贍等將領只知日夜飲酒作樂,不理軍務,以至軍紀弛廢,境內盜賊蜂起,蜀民苦不堪言。王全斌甚至剋扣太祖下令給投降蜀軍前往京城的路費,並多方騷擾,遂激成叛變,十萬叛軍推舉文州刺史全師雄為帥,攻陷彭州(今四川省成都市彭州市),成都十縣及邛、蜀、眉、陵等十七州響應叛亂,四川大亂,成都與汴梁斷絕聞訊。直至967年初,蜀地之亂經歷兩年鎮壓後方才平定。

北漢:968年8月23日,北漢皇帝劉鈞逝世,養子劉繼恩繼位。9月10日,太祖命客省使盧懷忠等二十二人領兵屯駐潞州,準備趁喪攻伐北漢,兩天後,命義軍節度使李繼勳領兵進入漢境,於洞過河擊敗北漢軍,斬殺二千餘人,獲戰馬五百匹,進圍北漢都城太原(今山西省太原市)。同時,北漢皇帝劉繼恩因與權臣郭無為政爭失敗而遭弒殺,劉繼恩胞弟劉繼元繼立皇帝,立即上表請求契丹出兵援救。11月,契丹援軍抵達,宋軍撤退,北漢軍趁機侵入宋境,擄掠居民而回。首次討伐北漢失利。

969年2月26日,太祖命宣徽南院使曹彬、侍衛步軍都指揮使党進等將領兵征伐北漢。3月1日,太祖下詔親征。3月7日,御駕自京城出發,大軍於十一天後到達潞州,因雨停駐。漢將劉繼業、馮進珂屯兵於團柏谷,遣偵騎往來巡邏,遭宋軍前鋒部隊擊敗,劉繼業等退回太原,宋軍遂包圍太原城。4月4日,潞州雨停,太祖率軍出發,六天後,抵達太原城下,下令於太原城外築長城牆,藉以圍困城池,絕其外援;又下令堵塞汾水,使之壅積,並於隔日決堤,水灌太原城,洪水從城門灌入城中,北漢派人緊急設置障礙填補,宋軍則頻射弓箭阻撓,使其無法施工,但不久即有成堆的草隨洪水漂流至決口處,使宋軍箭矢無法穿透,北漢便趁此堵住決口。宋軍久攻太原不下,將領多有死傷,加上部隊駐紮於甘草地上,正值盛暑大雨,疫疾橫生,將士多染病腹瀉,太常博士李光贊上奏建議退兵,趙普贊同,太祖便下令退兵。第二次討伐北漢亦失利。

南漢:征北漢失利後,太祖重拾「先南後北」策略。970年10月3日,詔令潭州防禦使潘美、朗州團練使尹崇珂、道州刺史王繼勳等將率兵討伐南漢,圍攻賀州(今廣西壯族自治區賀州市)。南漢皇帝劉鋹遣伍彥柔往援,遭宋軍擊潰,兵敗身死,賀州投降。隔年(971年)1月,宋軍進攻韶州(今廣東省韶關市),擊破漢將李承渥的大象陣,攻陷韶州,並相繼攻克英州(今廣東省清遠市英德市)、雄州(今廣東省韶關市南雄市),南漢韶州刺史辛延渥派人勸劉鋹投降,劉鋹不從,下令準備十多艘船裝載金銀財寶及妃嬪宮女,將出海逃亡,卻遭宦官及衛士盜其船而開走,劉鋹大懼,遣使請降又不獲准,只得堅守。3月3日,宋軍擊斬漢將植廷曉,並火燒位於馬徑(今廣東省佛山市南海區西北)的南漢軍營柵,南漢軍大敗,主帥郭崇岳戰死。劉鋹聽聞兵敗,便縱火焚燒宮殿府庫,成為灰燼。隔日,劉鋹素服出降,南漢滅亡。

南唐:974年10月9日,太祖命宣徽南院使曹彬等將率兵赴荊南,隔日又遣山南東道節度使潘美等將也赴荊南屯駐。11月4日,曹彬等將領兵出發荊南,直往南唐國都金陵(今江蘇省南京市)。太祖同時命吳越王錢俶合擊南唐,策應宋軍。11月21日,曹彬攻陷池州(今安徽省池州市),並陸續攻下銅陵、當塗、蕪湖,進逼采石磯(今安徽省馬鞍山市西)。12月9日,宋軍於采石磯擊敗南唐二萬大軍,俘獲一千餘人、戰馬三百餘匹。太祖隨即下令將先前已製成的浮橋自石牌(今安徽省安慶市懷寧縣石牌鎮)移至采石磯裝纜,三日而成,宋軍因此渡過長江。隔年(975年)3月2日,曹彬率軍圍攻金陵。南唐皇帝李煜下令戒嚴,並數次派遣使者徐鉉、周惟簡前往宋都汴梁請求暫緩進攻,太祖不許,徐鉉便陳述南唐國主無罪,與太祖反覆辯論,太祖大怒說:「不用再說了,我也知道南唐無罪,但天下本歸一家,臥榻之側,怎能容許其他人鼾睡呢!」976年元旦,宋軍攻陷金陵,李煜奉表請降,南唐滅亡。

太祖逝世後,太宗逼迫吳越王錢俶、清源軍節度使(閩南)陳洪進於978年納土歸降,並於隔年(979年)發兵滅亡北漢,宋朝至此統一除燕雲十六州外中國本部。

由元朝宰相脫脫所監修的《宋史‧太祖本紀》對宋太祖趙匡胤有極高評價:「讚曰:昔者堯、舜以禪代,湯、武以征伐,皆南面而有天下。四聖人者往,世道升降,否泰推移。當斯民塗炭之秋,皇天眷求民主,亦惟責其濟斯世而已。使其必得四聖人之才,而後以其行事畀之,則生民平治之期,殆無日也。五季亂極,宋太祖起介胄之中,踐九五之位,原其得國,視晉、漢、周亦豈甚相絕哉?及其發號施令,名藩大將,俯首聽命,四方列國,次第削平,此非人力所易致也。建隆以來,釋藩鎮兵權,繩贓吏重法,以塞濁亂之源。州郡司牧,下至令錄、幕職,躬自引對;務農興學,慎罰薄斂,與世休息,迄於丕平;治定功成,制禮作樂。在位十有七年之間,而三百餘載之基,傳之子孫,世有典則。遂使三代而降,考論聲明文物之治,道德仁義之風,宋於漢、唐,蓋無讓焉。嗚呼,創業垂統之君,規模若是,亦可謂遠也已矣!」——《宋史·本紀第三·太祖本紀三》

明太祖朱元璋於1374年9月親至南京歷代帝王廟祭祀自三皇至元世祖等十七位歷代帝王,並對其各有祝文,其中對宋太祖的祝文云:「惟宋太祖皇帝順天應人,統一海宇,祚延三百,天下文明。有君天下之德而安萬世之功者也。」——《明太祖高皇帝實錄·卷九十二》

北宋開寶九年十月二十日(癸丑,976年11月14日),趙匡胤於皇宮萬歲殿逝世,享年五十歲,在位十六年,予諡「英武聖文神德皇帝」,廟號「太祖」,三弟趙光義繼位,即宋太宗。

977年5月15日,靈柩奉葬永昌陵。1008年9月3日,宋真宗趙恆加諡為「啟運立極英武睿文神德聖功至明大孝皇帝」。

燭影斧聲:依據北宋中期由文瑩和尚所著《續湘山野錄》的記載,趙匡胤發跡前曾與一名道士來往,常相約飲酒至醉。一次醉酒後,道士以吟唱預言趙匡胤將當皇帝,醒後卻推說酒醉胡言。趙匡胤稱帝後兩人再也沒相見。十六年後的開寶九年(976年)上巳節,趙匡胤至西沼行祓禊禮,道士坐於岸邊樹蔭下,對趙匡胤說:「別來無恙。」趙匡胤大喜,即請至後宮飲酒歡續。趙匡胤說:「我想請你預測一事以久,無他事,我還有幾年壽命?」道士說:「只要今年十月二十日夜晚天氣晴朗,就可延續十二年;否則,即當從速安排後事。」當日夜,趙匡胤登太清閣觀象,天氣先晴朗而後轉惡,驟下大雪。趙匡胤急忙下閣傳令開皇宮端門,召晉王趙光義入宮,兄弟二人於內寢對坐飲酒,並屏去所有宦官、宮女。內侍們遙見寢室燭影下,趙光義時而起座離席,露出不可勝之情狀。兩人喝完已是午夜時分,室外積雪已達數寸,趙匡胤拿柱斧戳雪,一邊回頭對趙光義說:「好做,好做!」接著就寬衣就寢,鼾聲如雷。當晚,趙光義留宿宮中。天將五更,寢室周圍寂靜無聲,趙匡胤駕崩,享年四十九歲。趙光義受遺詔於柩前即位,是為宋太宗。

北宋史家司馬光所著《涑水記聞》則記載趙匡胤逝世後,宋皇后急派宦官王繼恩傳召趙匡胤第四子、秦王趙德芳進宮,王繼恩卻逕至趙光義府邸通報趙匡胤死訊,並催其盡速進宮即位。趙光義猶豫不定,王繼恩則說:「事情拖久就被他人搶先了。」於是趙光義趁夜踏雪入宮,進入寢殿。宋皇后聽說王繼恩已歸,便問:「德芳來了嗎?」王繼恩卻說:「晉王來了。」宋皇后見到趙光義,先是驚愕,隨即說:「我們母子的性命,都托付給官家了。」趙光義則涕泣說:「共保富貴,不需擔憂。」

南宋史家李燾所著《續資治通鑑長編》採信上述二說,只將趙匡胤語「好做,好做」改為「好為之」,道士則有姓名曰「張守真」,且言趙光義當晚並無於宮中留宿。李燾於書中引北宋史家蔡惇直的筆記,也有與《續湘山野錄》相似的記載。

依據上述疑點,加上史書中其他記載(如趙匡胤之死已有人先行預料),便有趙匡胤是被趙光義謀殺之說。

金匱之盟:趙光義表示:961年母親杜太后病危,召趙普入宮接受遺命。杜太后問趙匡胤:「你知道你為何能取得天下嗎?」趙匡胤泣不能答。杜太后說:「我是老死,哭也沒用,我正要跟你說大事,怎麼只是一直哭呢?」於是再問一遍。趙匡胤說:「都是祖上和太后積德所致。」杜太后說:「不對。是因為柴家讓孩童當皇帝,人心不服所致。如果後周有年長的君主,你能得到江山嗎?你和趙光義都是我親生的,你死後應將皇位傳給弟弟。天下之大、事務繁重,能立一個年長的君主來治理,這是社稷之福啊。」趙匡胤叩頭涕泣說:「一定遵照母后的教誨。」杜太后便對趙普說:「你將我剛才講的話記下來,不可違背。」趙普即於太后床前寫成誓書,並於末尾寫上「臣普記」三字。趙匡胤將誓書藏於金匱之中,命謹慎的宮人保管。

上述記載如屬實,則趙光義繼任皇位即有合法性及正當性。然而有學者指出諸多疑點:如此誓為真,則何以不在趙光義即位之初公布,而是等到981年10月才由趙普以「密奏」的方式啟奏趙光義,乃開金匱查驗屬實,前後竟隱瞞「太后遺詔」五年之久;記載內容所言皇位須「兄終弟及」是因杜太后擔心趙匡胤諸子皆幼,不足以坐穩江山,但趙匡胤死時,其次子趙德昭二十六歲,四子趙德芳十八歲,皆非幼弱,遺詔前提不復存在;記載來源《太祖實錄》經趙光義多次修改,已非原貌,而修改前的舊版則未有此記載,顯有杜撰之嫌;據《涑水記聞》、《續資治通鑑長編》、《宋史》等書記載,杜太后本意為趙匡胤死後,皇位傳給三弟趙光義,再傳給四弟趙廷美,最終傳回給趙匡胤次子趙德昭。然而修改後的新版《太祖實錄》只有傳位給趙光義的記載,且相關當事人竟於短時間內逐一逝世(979年,趙德昭自殺;981年,趙德芳猝死;984年,趙廷美遭貶,憂悸而死)。史學界即因前述疑點而有「金匱之盟」是趙普為取得趙光義信任、重得相位而杜撰「太后遺詔」的說法。加上趙光義不逾年而改元(開寶九年十二月甲寅,即977年1月14日改「開寶九年」為「太平興國元年」),也有學者據此認為趙光義因弒兄奪位而心虛,不等過年即倉促改元,欲使其繼位成為既定事實。

史載趙匡胤生時,紅光滿室,有香氣整夜不散,嬰兒體呈金色,長達三日。年輕時學騎射,嘗試馴服烈馬不加鞍繩,馬衝上城門斜坡道,致趙匡胤額頭撞擊城門門楣,目擊者皆認為其頭骨必定粉碎,趙匡胤卻從容站起,徒步追上烈馬騰騎而上,毫髮無傷;趙匡胤兒時曾與玩伴韓令坤在土屋裡玩耍,有群麻雀在室外聒噪互鬥,趙匡胤遂與韓令坤出土屋欲捕麻雀,剛出房而土屋隨即崩塌,二人幸免於難。

趙匡胤與胞弟趙光義幼時隨母親杜氏躲避戰亂,杜氏便將兄弟二人放至籮筐擔挑而走,為道士陳摶撞見,便嘆道:「別說當今世上沒有天子,都將天子用擔挑著走。」;多年後趙匡胤稱帝,陳摶聞之大笑,說:「天下從此安定了。」

959年,後周世宗柴榮親征契丹,於征途中批閱文書,發現其中有一囊袋,內有三尺長的木牌,上有字:「點檢作天子」。柴榮見後不悅,漸感身體不適,便命車駕返京。抵京後下詔撤殿前都點檢張永德之職,改任趙匡胤。隔年初,趙匡胤登基為帝,遂應此讖。

宋初,宰相范質等人仍循前代慣例,上朝時設有座椅,坐著奏事。一日早朝,范質猶坐著,趙匡胤便說:「我眼睛昏花,看不清楚,你把文書拿給我看。」范質於是起身持文書進呈,趙匡胤卻已密囑侍者趁此將其座撤去,待范質欲返座而座椅已撤,只得站立。自此宰相與群臣般站著上朝,成為慣例。

趙匡胤稱帝後第三年,秘密遣人鐫刻一通石碑,藏於太廟的夾室內,稱為「誓碑」,用黃金絲所鑲嵌成的布幔遮蓋,門禁森嚴。趙匡胤下令此後四時祭祀及新皇帝即位時,待拜完太廟,便須恭讀誓詞,由一個不識字的小太監持鑰匙開夾室,然後焚香、點亮燭火並將幔揭開,其餘隨臣須於遠方庭中佇候。當朝皇帝於碑前跪拜並默誦誓詞,再拜而出,群臣及近侍們皆不知誓詞為何。北宋历代皇帝皆承襲故例,按時恭讀,不敢洩漏。直到靖康之變爆發,皇宮大亂,太廟夾室門戶洞開,人們才發現內裡有一高約七、八尺,寬四尺餘的石碑,上面有三行誓詞:第一、柴氏子孫有罪,不得加刑,縱犯謀逆,止於獄中賜盡,不得市曹刑戮,亦不得連坐支屬;第二、不得殺士大夫及上書言事人;第三、子孫有渝此誓者,天必殛之。上述三誓史稱「太祖誓約」。南宋皇帝則是由曹勛自金國南歸時,向宋高宗轉達宋徽宗之語方才得知誓詞。


\subsection{建隆}


\begin{longtable}{|>{\centering\scriptsize}m{2em}|>{\centering\scriptsize}m{1.3em}|>{\centering}m{8.8em}|}
  % \caption{秦王政}\
  \toprule
  \SimHei \normalsize 年数 & \SimHei \scriptsize 公元 & \SimHei 大事件 \tabularnewline
  % \midrule
  \endfirsthead
  \toprule
  \SimHei \normalsize 年数 & \SimHei \scriptsize 公元 & \SimHei 大事件 \tabularnewline
  \midrule
  \endhead
  \midrule
  元年 & 960 & \tabularnewline\hline
  二年 & 961 & \tabularnewline\hline
  三年 & 962 & \tabularnewline\hline
  四年 & 963 & \tabularnewline
  \bottomrule
\end{longtable}

\subsection{乾德}

\begin{longtable}{|>{\centering\scriptsize}m{2em}|>{\centering\scriptsize}m{1.3em}|>{\centering}m{8.8em}|}
  % \caption{秦王政}\
  \toprule
  \SimHei \normalsize 年数 & \SimHei \scriptsize 公元 & \SimHei 大事件 \tabularnewline
  % \midrule
  \endfirsthead
  \toprule
  \SimHei \normalsize 年数 & \SimHei \scriptsize 公元 & \SimHei 大事件 \tabularnewline
  \midrule
  \endhead
  \midrule
  元年 & 963 & \tabularnewline\hline
  二年 & 964 & \tabularnewline\hline
  三年 & 965 & \tabularnewline\hline
  四年 & 966 & \tabularnewline\hline
  五年 & 967 & \tabularnewline\hline
  六年 & 968 & \tabularnewline
  \bottomrule
\end{longtable}

\subsection{开宝}

\begin{longtable}{|>{\centering\scriptsize}m{2em}|>{\centering\scriptsize}m{1.3em}|>{\centering}m{8.8em}|}
  % \caption{秦王政}\
  \toprule
  \SimHei \normalsize 年数 & \SimHei \scriptsize 公元 & \SimHei 大事件 \tabularnewline
  % \midrule
  \endfirsthead
  \toprule
  \SimHei \normalsize 年数 & \SimHei \scriptsize 公元 & \SimHei 大事件 \tabularnewline
  \midrule
  \endhead
  \midrule
  元年 & 968 & \tabularnewline\hline
  二年 & 969 & \tabularnewline\hline
  三年 & 970 & \tabularnewline\hline
  四年 & 971 & \tabularnewline\hline
  五年 & 972 & \tabularnewline\hline
  六年 & 973 & \tabularnewline\hline
  七年 & 974 & \tabularnewline\hline
  八年 & 975 & \tabularnewline\hline
  九年 & 976 & \tabularnewline
  \bottomrule
\end{longtable}


%%% Local Variables:
%%% mode: latex
%%% TeX-engine: xetex
%%% TeX-master: "../Main"
%%% End:

%% -*- coding: utf-8 -*-
%% Time-stamp: <Chen Wang: 2021-11-01 15:53:06>

\section{太宗趙炅\tiny(976-997)}

\subsection{生平}

宋太宗趙\xpinyin*{炅}(939年11月20日-997年5月8日),北宋第二位皇帝(976年11月15日-997年5月8日在位),在位21年,享年58岁。趙弘殷第三子,是北宋開國君主宋太祖趙匡胤的胞弟。本名趙匡義,字廷宜,其兄长赵匡胤登基後為避諱,改名趙光義,趙匡胤去世後,勾結兄長身邊宦官篡位,史說為燭光斧影,後流放和逼迫兄長趙匡胤的兩個兒子自殺。

宋太宗文治有為,但不善武功。于太平兴国三年(978年)迫使吴越纳土归降;之后又灭亡五代十国最后一个割据政权北汉,结束北宋统一战爭。次年(979年)宋太宗趙光義移师幽州,试图一举收复燕云十六州,在高粱河(今北京西直门外)展开激战,宋军大败,宋太宗被耶律休哥射伤,乘驴车逃走。他两度伐辽失败。980年又试图兼併交趾,但惨败,使交趾(越南)最终得以保持独立地位。

后晋天福四年十月七日(939年11月20日)生于东京(河南开封)护圣营,为乾明节,后改为寿宁节。

趙光義於建隆二年(961年)七月任開封府尹,至開寶九年(976年)十月登基後離任,主政天府十五年,治績斐然。是任職最長的开封府尹,这为他取得帝位打下基礎。

有野史記載認為宋太宗即位之事甚為蹊蹺,有弒兄謀朝篡位之嫌,即「燭影斧聲」之疑案。

一般來說,新皇帝在先帝駕崩后翌年改元。宋太宗的改元時間是繼位當年的農曆十二月,當時距離農曆新的一年已經只剩下幾天而已。這一點為後人所詬病,也為即位爭議留下了另一個頗具爭議性的口實。

為確保政權合法性,太宗拋出其母杜太后遺命之說,即「金匱之盟」。在宋建隆二年(西元961年),即太祖即位的第二年,皇太后杜氏臨終前,告誡太祖前朝後周之所以滅亡,是因為繼位的君主過於年幼。若要常保大宋江山,必須要兄終弟及,傳位給年長的皇室成員當天子。太祖死後先傳其弟光義,再傳弟光美(後改名為廷美),等到皇兄的兒子成人,再由皇弟傳回給皇兄的兒子,即太祖長子趙德昭。趙普入宮記錄遺命,這份遺書藏於金匱中,因此名為金匱之盟。

然而,太宗在位期間逼死太祖之子德昭,趙德芳暴斃,太宗又貶黜其弟廷美到房州,兩年後趙廷美死於谪所。1940年代鄧廣銘、張蔭麟等論證金匱之盟為虛構,影響至今:皇太后杜氏去世之時,太祖年僅三十四歲,正值壯年,根本不會聯想到死亡;退一萬步言,就算太祖立刻死亡,而其長子趙德昭當時已十一歲,離成年已不遠,根本不會出現如周世宗遺下七歲孤兒的局面,而且後來太祖五十歲駕崩時,趙德昭已經二十六歲,過了二十歲成年已經又六年了,故偽造說成為最普遍的說法。但近年也有學者質疑偽造說,如施秀娥、王育濟、何冠環。

太宗登基以后,“太祖之后当再有天下”之说一直不断,至靖康之變後,宋欽宗之弟康王赵构自立於江南,是為宋高宗。当时普遍有种传说,说因为太宗登基不明不白,所以才会让后代失去半壁江山,后又有孟太后之宋太祖托梦一说,加上高宗無子,最终1162年传位给趙德芳之后代宋孝宗,宋朝在太宗一脈統治186年后,再回到太祖一脈。

太宗穩固帝位後,繼續統一事業。其後割據福建漳泉兩府的陳洪進,割據吳越錢氏相继歸降。太宗遣大将潘美揮师北上围攻北漢都城太原,击退辽国援兵,滅亡北漢,終於结束了安史之亂后近二百年藩镇割据的局面。

太平兴国四年(979年)五月,太宗不顧眾臣反對,趁伐取北漢之勢,從太原出發展開北伐。北伐初期一度收復河北易州和涿州。太宗志得意滿,下令圍攻燕京,宋軍與遼人在高粱河畔展开激战。太宗亲临战场,結果受傷中箭,乘驴车仓惶撤离,北伐未果。

980年宋朝知邕州太常博士侯仁宝上奏宋太宗,请求趁交趾(越南)丁朝内乱之机南下讨伐,恢复汉唐故疆,统一交趾(越南)。宋太宗任命侯仁宝为交州陆路水路转运使;任命兰陵团练使孙全兴、漆作使郝守俊、鞍辔库使陈钦祚、左监门将军崔亮为兵马都部署;宁州刺史刘澄、军器库副使贾湜、供奉官阁门祗候王僎为兵马都部署,伺机进攻丁朝。但在981年白藤江之战中先胜后败,统一交趾(越南)的计划最终成为泡影,交趾(越南)得以保持独立地位。

雍熙三年(986年),太宗遣潘美、杨业、田重进、曹彬、崔彦五位大将分东中西三路,以東路為主再行北伐。西路、中路军進軍順利,而主力东路軍屡遭辽军挫敗,粮道被切断,终未能与中西二路汇合,于岐沟关大败而潰。中、西二路亦只得南撤。西路主將杨业因掩护军民南撤被辽军俘虏,在狱中绝食三日而死。之後,北宋在對西夏党項族的三川口、好水川、定川寨等戰役中屢次失敗,但因其厌战,与宋廷议和。

太宗朝以亲信傅潜、王超、柴禹锡、赵镕、张逊、杨守一及弭德超等为禁军统帅,多庸碌之徒,临阵惧战。元人修《宋史》时称:“自柴禹锡而下,率因给事藩邸,以攀附致通显……故莫逃于龊龊之讥。”

幾次边陲防線的失利、後方起義的爆發遏制了北宋進一步開闢疆土,太宗的施政也不得不轉為重內虛外。太宗本人附庸風雅喜好詩賦,政府也因此特別重視文化事業,宋朝重教之風因而展開。太宗喜好書法,善草、隸(八分)、行、篆、飞白等數種字體,尤其善書飛白體,宋朝的貨幣淳化元寶也是太宗親自题寫的。

太宗長子元佐因為同情廷美被廢,另一子元僖暴死,最後襄王元侃被立為太子,改名恒。至道三年(997年),太宗崩,李皇后和宦官王繼恩等企圖立元佐為帝。時宰相呂端處置得當,赵恒順利即位,庙号真宗。宋朝始步入安穩守成時期。

至道三年(997年),三月二十九日帝因箭瘡舊疾崩于東京宮中之萬歲殿,享年五十八歲,在位二十二年。皇太子趙恒登基為帝,是為宋真宗。群臣上尊諡曰神功聖德文武皇帝,廟號太宗。同年十月葬在永熙陵。死后谥号「至仁應道神功聖德文武睿烈大明廣孝皇帝」。宋太宗的後代為宋真宗至宋高宗所有宋朝皇帝及宋寧宗初的宗室大臣趙汝愚。

宋太宗好讀書,「開卷有益」典故即來自他。王闢之的《澠水燕談錄》卷六:“太宗日閱《御覽》三卷,因事有缺,暇日追補之,嘗曰:開卷有益,朕不以為勞也。” 宋太宗曾對日本萬世一系的感慨

元朝官修正史《宋史》脱脱等的評價是:“帝沈谋英断,慨然有削平天下之志。既即大位,陈洪进、钱俶相继纳土。未几,取太原,伐契丹,继有交州、西夏之役。干戈不息,天灾方行,俘馘日至,而民不知兵;水旱螟蝗,殆遍天下,而民不思乱。其故何也?帝以慈俭为宝,服浣濯之衣,毁奇巧之器,却女乐之献,悟畋游之非。绝远物,抑符瑞,闵农事,考治功。讲学以求多闻,不罪狂悖以劝谏士,哀矜恻怛,勤以自励,日晏忘食。”毛澤東閱讀《宋史·太宗本紀》時对此批曰“但無能”。

趙匡義對軍事理論見解獨到,他建立了參謀本部制度、陣圖制度和軍事學院體系,是現代化軍隊的淵源。同時確立了文官掌兵的軍隊國有化體制,宋朝開始逐漸形成現代化集團軍編制,也和宋太宗有重大關係。但他指揮軍隊的能力十分薄弱,加上遼國立國早宋朝五十年,底蘊深厚,又有耶律斜軫等名將輔佐;宋遼數度交鋒皆以慘敗收場,毛澤東對此表示“此人不知兵,非契丹对手。爾後屢敗,契丹均以誘敵深入、聚而殲之的辦法,宋人終不省。”

《宋史·太宗本纪》:“欲自焚以答天谴,欲尽除天下之赋以纾民力”,老百姓比肩接踵而至,请其“登禅”即位。“故帝之功德,炳焕史牒,号称贤君。若夫太祖之崩不逾年而改元,涪陵县公之贬死,武功王之自杀,宋后之不成丧,则后世不能无议论焉。”毛泽东批道:“不择手段,急于登台。”《宋史纪事本末》记载宋太宗诏立太子后回宫途中,百姓都欢呼雀跃,欢呼“少年天子”,宋太宗听了很不高兴,召见宰相寇准说:“人心遽属太子,欲置我何地?”寇准向他道贺说:“此社稷之福也。”他转怒为喜请寇准喝酒,“极醉而罢”,毛泽东批道:“赵匡义小人之言。”

\subsection{太平兴国}


\begin{longtable}{|>{\centering\scriptsize}m{2em}|>{\centering\scriptsize}m{1.3em}|>{\centering}m{8.8em}|}
  % \caption{秦王政}\
  \toprule
  \SimHei \normalsize 年数 & \SimHei \scriptsize 公元 & \SimHei 大事件 \tabularnewline
  % \midrule
  \endfirsthead
  \toprule
  \SimHei \normalsize 年数 & \SimHei \scriptsize 公元 & \SimHei 大事件 \tabularnewline
  \midrule
  \endhead
  \midrule
  元年 & 976 & \tabularnewline\hline
  二年 & 977 & \tabularnewline\hline
  三年 & 978 & \tabularnewline\hline
  四年 & 979 & \tabularnewline\hline
  五年 & 980 & \tabularnewline\hline
  六年 & 981 & \tabularnewline\hline
  七年 & 982 & \tabularnewline\hline
  八年 & 983 & \tabularnewline\hline
  九年 & 984 & \tabularnewline
  \bottomrule
\end{longtable}

\subsection{雍熙}

\begin{longtable}{|>{\centering\scriptsize}m{2em}|>{\centering\scriptsize}m{1.3em}|>{\centering}m{8.8em}|}
  % \caption{秦王政}\
  \toprule
  \SimHei \normalsize 年数 & \SimHei \scriptsize 公元 & \SimHei 大事件 \tabularnewline
  % \midrule
  \endfirsthead
  \toprule
  \SimHei \normalsize 年数 & \SimHei \scriptsize 公元 & \SimHei 大事件 \tabularnewline
  \midrule
  \endhead
  \midrule
  元年 & 984 & \tabularnewline\hline
  二年 & 985 & \tabularnewline\hline
  三年 & 986 & \tabularnewline\hline
  四年 & 987 & \tabularnewline
  \bottomrule
\end{longtable}

\subsection{端拱}

\begin{longtable}{|>{\centering\scriptsize}m{2em}|>{\centering\scriptsize}m{1.3em}|>{\centering}m{8.8em}|}
  % \caption{秦王政}\
  \toprule
  \SimHei \normalsize 年数 & \SimHei \scriptsize 公元 & \SimHei 大事件 \tabularnewline
  % \midrule
  \endfirsthead
  \toprule
  \SimHei \normalsize 年数 & \SimHei \scriptsize 公元 & \SimHei 大事件 \tabularnewline
  \midrule
  \endhead
  \midrule
  元年 & 988 & \tabularnewline\hline
  二年 & 989 & \tabularnewline
  \bottomrule
\end{longtable}

\subsection{淳化}

\begin{longtable}{|>{\centering\scriptsize}m{2em}|>{\centering\scriptsize}m{1.3em}|>{\centering}m{8.8em}|}
  % \caption{秦王政}\
  \toprule
  \SimHei \normalsize 年数 & \SimHei \scriptsize 公元 & \SimHei 大事件 \tabularnewline
  % \midrule
  \endfirsthead
  \toprule
  \SimHei \normalsize 年数 & \SimHei \scriptsize 公元 & \SimHei 大事件 \tabularnewline
  \midrule
  \endhead
  \midrule
  元年 & 990 & \tabularnewline\hline
  二年 & 991 & \tabularnewline\hline
  三年 & 992 & \tabularnewline\hline
  四年 & 993 & \tabularnewline\hline
  五年 & 994 & \tabularnewline
  \bottomrule
\end{longtable}

\subsection{至道}

\begin{longtable}{|>{\centering\scriptsize}m{2em}|>{\centering\scriptsize}m{1.3em}|>{\centering}m{8.8em}|}
  % \caption{秦王政}\
  \toprule
  \SimHei \normalsize 年数 & \SimHei \scriptsize 公元 & \SimHei 大事件 \tabularnewline
  % \midrule
  \endfirsthead
  \toprule
  \SimHei \normalsize 年数 & \SimHei \scriptsize 公元 & \SimHei 大事件 \tabularnewline
  \midrule
  \endhead
  \midrule
  元年 & 995 & \tabularnewline\hline
  二年 & 996 & \tabularnewline\hline
  三年 & 997 & \tabularnewline
  \bottomrule
\end{longtable}


%%% Local Variables:
%%% mode: latex
%%% TeX-engine: xetex
%%% TeX-master: "../Main"
%%% End:

%% -*- coding: utf-8 -*-
%% Time-stamp: <Chen Wang: 2019-12-26 10:27:44>

\section{真宗\tiny(997-1022)}

\subsection{生平}

宋真宗趙恒(968年12月23日-1022年3月23日),原名趙德昌,又曾名趙元休、趙元侃,北宋的第三位皇帝。他是宋太宗的第三个儿子,登基前曾被封为韩王、襄王和寿王,淳化五年(994年)九月,加檢校太傅行開封府尹,至道三年(997年)四月登基後離任,以太子身份继位,在位25年。

宋真宗是著名諺語「書中自有黃金屋,書中自有顏如玉」的作者。

宋太祖乾德六年十二月初二日(968年12月23日)趙恒生於東京開封府第,是趙光義第三子,與長兄楚王趙元佐同母,初名趙德昌。幼時英睿,姿表特異。與諸王嬉戲時,喜歡作戰陣之狀,自稱元帥。宋太祖喜愛他,將他養在宮中。

趙恆最初並非皇位繼承人,宋太宗最初屬意長子趙元佐為太子,但因趙元佐自秦王廷美亡后精神失常,而且因病傷人及在宮內縱火,最後被廢。太宗本計劃立次子趙元僖為太子,但趙元僖又早逝。趙元僖死後,太宗立三子趙元侃為太子,赐名恆,至道三年(997年)宋太宗箭傷復發而駕崩,趙恆繼位為帝,是為宋真宗。

景德元年(1004年)辽国入侵宋,宋朝大多数大臣建议不抵抗,以宰相寇準为首的少数人极力主张抵抗,最后宋真宗御驾亲征,双方在澶州(今河南濮阳附近)相交,宋胜後,真宗决定就此罢兵,以每年向辽纳白银十万両、绢二十万匹来換取与辽之間的和平,定澶渊之盟。这是宋朝以岁币换取和平的开始。

真宗時,鐵製工具製作進步,土地耕作面積增至5.2億畝(太宗至道二年,耕地有3億多畝),又引入暹罗良种水稻,農作物產量倍增,紡織、染色、造紙、製瓷等手工業、商業蓬勃發展,景德年间,專門製作瓷器(原名白崖场)的昌南镇遂改名为景德镇,貿易盛況空前,史称咸平之治。

宋真宗统治后期,以王钦若和丁谓为宰相,信奉道教和佛教,大中祥符元年(1008年)称受天书,封泰山、祀汾阳,詔令丁谓修建了玉清昭应宫,极侈土木,七年始成,有房屋近三千间,“小不中程,虽金碧已具,必毁而更造,有司不敢计其费。”,封禪给民众造成极大的负担。但亦有认為封禪其實只是為了要震懾強鄰遼國,在軍事屢居下風之際進行一種天命正當性的競爭,但之後發生了北宋帽妖案引發社會動盪,其內在原因史家至今各有主張。

宋真宗好文學,也是一名诗人,他比较著名的诗有《励学篇》、《勸學詩》等。一般人常說的「書中自有黃金屋、書中有女顏如玉」就是出自他的勸學詩。

乾興元年二月十九日(1022年3月23日)于汴京延慶殿駕崩,享年五十五歲,在位共二十五年。群臣為其上諡號為文明章聖元孝皇帝,廟號真宗。十月十三日,葬於永定陵;二十三日附祭太廟。太子宋仁宗继位,史称“仁宗以天书殉葬山陵,呜呼贤哉!”七年後,昭应宫遭雷击,被大火焚為灰烬。《宋史》稱真宗“及澶洲既盟,封禅事作,祥瑞踏臻,天書屢降,導迎奠安,一國君臣如病狂然,吁,可怪也。

元朝官修正史《宋史》脱脱等的評價是:“真宗英悟之主。其初践位,相臣李沆虑其聪明,必多作为,数奏灾异以杜其侈心,盖有所见也。及澶洲既盟,封禅事作,祥瑞沓臻,天书屡降,导迎奠安,一国君臣如病狂然,吁,可怪也。他日修《辽史》,见契丹故俗而后推求宋史之微言焉。宋自太宗幽州之败,恶言兵矣。契丹其主称天,其后称地,一岁祭天不知其几,猎而手接飞雁,鸨自投地,皆称为天赐,祭告而夸耀之。意者宋之诸臣,因知契丹之习,又见其君有厌兵之意,遂进神道设教之言,欲假是以动敌人之听闻,庶几足以潜消其窥觎之志欤?然不思修本以制敌,又效尤焉,计亦末矣。仁宗以天书殉葬山陵,呜呼贤哉!”

宋真宗派曹利用去辽国签订澶渊之盟之际,告诉曹“迫不得已,虽百万亦可!”。寇准知道后,指着曹怒道“超过30万两,提人头来见”。最后,经过曹利用再三讨价还价,以每年白银10万两、绢帛20万匹,订立澶渊之盟。曹利用回到宋朝之后,真宗急问金额多少,曹利用不敢直说,只竖起3根指头,真宗以为是300万两,大惊失声脱口而说,“太多了”,过了一会又安慰道:“金额是太多了,但就此把事情了结也好”,待知道是30万时,如释重负,转忧为喜,对曹利用大加赏赐。

另外,宋真宗亦是著名戲曲《貍貓換太子》第三主角。(另兩位為宋仁宗和李宸妃,相關人物為太監陳琳、郭槐、劉太后、王親貴族八賢王、狄太后及婢女寇珠。)

\subsection{咸平}


\begin{longtable}{|>{\centering\scriptsize}m{2em}|>{\centering\scriptsize}m{1.3em}|>{\centering}m{8.8em}|}
  % \caption{秦王政}\
  \toprule
  \SimHei \normalsize 年数 & \SimHei \scriptsize 公元 & \SimHei 大事件 \tabularnewline
  % \midrule
  \endfirsthead
  \toprule
  \SimHei \normalsize 年数 & \SimHei \scriptsize 公元 & \SimHei 大事件 \tabularnewline
  \midrule
  \endhead
  \midrule
  元年 & 998 & \tabularnewline\hline
  二年 & 999 & \tabularnewline\hline
  三年 & 1000 & \tabularnewline\hline
  四年 & 1001 & \tabularnewline\hline
  五年 & 1002 & \tabularnewline\hline
  六年 & 1003 & \tabularnewline
  \bottomrule
\end{longtable}

\subsection{景德}

\begin{longtable}{|>{\centering\scriptsize}m{2em}|>{\centering\scriptsize}m{1.3em}|>{\centering}m{8.8em}|}
  % \caption{秦王政}\
  \toprule
  \SimHei \normalsize 年数 & \SimHei \scriptsize 公元 & \SimHei 大事件 \tabularnewline
  % \midrule
  \endfirsthead
  \toprule
  \SimHei \normalsize 年数 & \SimHei \scriptsize 公元 & \SimHei 大事件 \tabularnewline
  \midrule
  \endhead
  \midrule
  元年 & 1004 & \tabularnewline\hline
  二年 & 1005 & \tabularnewline\hline
  三年 & 1006 & \tabularnewline\hline
  四年 & 1007 & \tabularnewline
  \bottomrule
\end{longtable}

\subsection{大中祥符}

\begin{longtable}{|>{\centering\scriptsize}m{2em}|>{\centering\scriptsize}m{1.3em}|>{\centering}m{8.8em}|}
  % \caption{秦王政}\
  \toprule
  \SimHei \normalsize 年数 & \SimHei \scriptsize 公元 & \SimHei 大事件 \tabularnewline
  % \midrule
  \endfirsthead
  \toprule
  \SimHei \normalsize 年数 & \SimHei \scriptsize 公元 & \SimHei 大事件 \tabularnewline
  \midrule
  \endhead
  \midrule
  元年 & 1008 & \tabularnewline\hline
  二年 & 1009 & \tabularnewline\hline
  三年 & 1010 & \tabularnewline\hline
  四年 & 1011 & \tabularnewline\hline
  五年 & 1012 & \tabularnewline\hline
  六年 & 1013 & \tabularnewline\hline
  七年 & 1014 & \tabularnewline\hline
  八年 & 1015 & \tabularnewline\hline
  九年 & 1016 & \tabularnewline
  \bottomrule
\end{longtable}

\subsection{天禧}

\begin{longtable}{|>{\centering\scriptsize}m{2em}|>{\centering\scriptsize}m{1.3em}|>{\centering}m{8.8em}|}
  % \caption{秦王政}\
  \toprule
  \SimHei \normalsize 年数 & \SimHei \scriptsize 公元 & \SimHei 大事件 \tabularnewline
  % \midrule
  \endfirsthead
  \toprule
  \SimHei \normalsize 年数 & \SimHei \scriptsize 公元 & \SimHei 大事件 \tabularnewline
  \midrule
  \endhead
  \midrule
  元年 & 1017 & \tabularnewline\hline
  二年 & 1018 & \tabularnewline\hline
  三年 & 1019 & \tabularnewline\hline
  四年 & 1020 & \tabularnewline\hline
  五年 & 1021 & \tabularnewline
  \bottomrule
\end{longtable}

\subsection{乾兴}

\begin{longtable}{|>{\centering\scriptsize}m{2em}|>{\centering\scriptsize}m{1.3em}|>{\centering}m{8.8em}|}
  % \caption{秦王政}\
  \toprule
  \SimHei \normalsize 年数 & \SimHei \scriptsize 公元 & \SimHei 大事件 \tabularnewline
  % \midrule
  \endfirsthead
  \toprule
  \SimHei \normalsize 年数 & \SimHei \scriptsize 公元 & \SimHei 大事件 \tabularnewline
  \midrule
  \endhead
  \midrule
  元年 & 1022 & \tabularnewline
  \bottomrule
\end{longtable}


%%% Local Variables:
%%% mode: latex
%%% TeX-engine: xetex
%%% TeX-master: "../Main"
%%% End:

%% -*- coding: utf-8 -*-
%% Time-stamp: <Chen Wang: 2019-12-26 10:28:37>

\section{仁宗\tiny(1022-1063)}

\subsection{生平}

宋仁宗趙禎(1010年5月30日-1063年4月30日),北宋第四代皇帝(1022年3月23日-1063年4月30日在位)。初名受益,宋真宗的第六子,生母李宸妃。其生于大中祥符三年四月十四日,大中祥符七年(1014年)受封為庆国公,八年(1015年)受封為寿春郡王;天禧元年(1017年)进中书令,二年(1018年)进封昇王,同年九月立为皇太子,赐名趙禎;乾兴元年(1022年)二月,真宗崩,仁宗即帝位,时年13岁;1023年改年號為天圣;1063年駕崩於汴梁皇宮中,享年53岁,在位41年。民間流傳“狸猫换太子”中的太子就是影射宋仁宗。

宋大中祥符三年(1010年),四月十四日(1010年5月30日)趙受益出生(後改名趙禎),是宋真宗趙恒第六子,母為李宸妃。趙恒所寵信的美人劉氏(章獻明肅皇后)無子,趙恒便對外聲稱趙受益為劉氏所生。

乾興元年(1022年),二月十九日,真宗趙恒逝世,趙禎即位,十二歲時由養母刘太后垂帘听政。在刘太后的主导下,他放弃了自己宠爱的后宫张氏,于天圣二年(1024年)立郭氏为皇后。

明道二年(1033年)太后听政十一年後病卒,23歲的仁宗始亲政。

在位期间最主要的军事冲突在於西夏,夏景宗李元昊即位后改变其父夏太宗李德明国策,展开宋夏战争,延州、好水川、定川三战宋军皆有失利之处,韓琦、范仲淹更在好水川之战後被贬。到定川寨之戰,西夏分兵欲直捣关中的西夏军遭宋朝原州(今甘肃镇原)知州景泰的顽强抵抗,全军覆灭,西夏攻占关中的战略目标就此破灭。西夏因连年征战国力难支,最后两国和谈:夏向宋称臣,宋每年赐西夏绢十三万匹、银五万两、茶二万斤,史称“庆历和议”,取得了近半世纪的和平。

辽兴宗時以萧惠陈兵宋境。接著,宋朝與遼朝协议,以增加岁币为条件,维持澶渊之盟的和平协议,史稱重熙增幣。

岁币支出并非沉重负担,比起选择战争的军费,岁币开支无足轻重。宝元元年,陕西出支為1551万;宝元二年宋夏战争后,庆历二年陕西出支為3363万,几近赤字。辽国失去南下劫掠的经济诱因,也是辽宋能维持百年和平的重要因素之一。

皇佑四年(1052年),侬智高反宋,军队席卷广西、广东各地。仁宗任用狄青、余靖率兵南征。皇佑五年,狄青夜袭昆仑关,大败侬智高于归仁铺之戰。次年,侬智高死於大理国,乱平。

仁宗時期,承平日久,經濟快速發展,並出現交子。而宮內治理略有缺失,朝中也有些許小人奸臣,慶曆8年發生疑似後宮爭鬥的坤寧宮事變,事後在眾多忠臣的輔佐之下奸臣的計策未能成功。仁宗時冗官與冗兵特別嚴重,皇祐元年(1049),户部副使包拯即已指出冗官問題:“今内外官属总一万七千三百余员,其未受差遣京官、使臣及守选人不在数内,较之先朝,才四十余年,已逾一倍多矣”,而州郡县的地方官,则更是“三倍其多。”全国军队总计125万9千人,佔赋税十分之七。真宗與仁宗兩朝土地兼并更嚴重,公卿大臣大都佔地千顷以上。仁宗晚年,“势官富姓占田无限,兼并冒伪习以为俗,重禁莫能止焉”,最後“富者有弥望之田,贫者无卓锥之地。”國家財政出现危机,“当仁宗四十二年,号为本朝至平极盛之世,而财用始大乏。”

慶曆新政由范仲淹十大政策揭開序目——明黜陟、抑侥幸、精贡举、择官长、均公田、厚农桑、修武备、减徭役、覃恩信、重命令。但反對勢力龐大,難以推動,一年四個月後便宣布中止。

仁宗一朝對外雖無重大戰爭,對內亦無重大革新,對外需要應對遼夏的軍事威脅。

嘉祐八年(1063年),三月二十九日,於汴梁皇宮駕崩,年五十四歲,死后葬于永昭陵。《宋史》記載,「趙禎駕崩的消息傳出後,京師(汴梁)罷市巷哭,數日不絕,雖乞丐與小兒,皆焚紙錢哭於大內之前」。

仁宗在位41年,是宋朝皇帝中執政最長的一位,生性恭俭仁恕,百司曾奏请扩大苑林,宋仁宗说:「我繼承先帝的園林,尚且覺得十分大,為什麼要這樣做(擴建)呢?」宋仁宗去世后,就連讣告送到辽国時,竟“燕境之人无远近皆哭”,辽道宗耶律洪基痛哭道:“四十二年不识兵革矣”,史載其“惊肃再拜,谓左右曰:‘我若生中国,不过与之执鞭持盖一都虞侯耳!’”

在宋仁宗出生的那天,皇帝賞賜群臣包子。

宋代王明清《揮麈後錄》卷一載:“仁宗母李后,曾夢一羽衣之士,跣足從空而下云:來為汝子。後召幸有娠而生仁宗。仁宗幼年,每穿履襪,即亟令脫去,常徒步禁掖,宮中皆呼為赤腳仙人,蓋古之得道者李君也。”

元朝脫脫《宋史》評價宋仁宗個性仁愛、勤儉,一時朝野上下充滿惻隱善心、行忠義仁厚之政,要不是後代子孫的作為,仁宗之政是可為宋朝三百年的未來奠基:『贊曰:仁宗恭儉仁恕,出於天性,一遇水旱,或密禱禁庭,或跣立殿下。有司請以玉清舊地為御苑,帝曰:「吾奉先帝苑囿,猶以為廣,何以是為?」燕私常服浣濯,帷帟衾裯,多用繒絁。宮中夜饑,思膳燒羊,戒勿宣索,恐膳夫自此戕賊物命,以備不時之須。大辟疑者,皆令上讞,歲常活千餘。吏部選人,一坐失入死罪,皆終身不遷。每諭輔臣曰:「朕未嘗詈人以死,況敢濫用辟乎!」至於夏人犯邊,御之出境;契丹渝盟,增以歲幣。在位四十二年之間,吏治若偷惰,而任事蔑殘刻之人;刑法似縱弛,而決獄多平允之士。國未嘗無弊幸,而不足以累治世之體;朝未嘗無小人,而不足以勝善類之氣。君臣上下惻怛之心,忠厚之政,有以培壅宋三百餘年之基。子孫一矯其所為,馴致於亂。《傳》曰:「為人君,止於仁。」帝誠無愧焉。』

王夫之评论宋仁宗“无定志”,指出仁宗親政至去世的三十年間,兩府大臣更迭頻繁,計有的四十多名員都曾多次上任,但也多次被仁宗因小故而撤換,因此官員們的政策都因在位時間不長而無法貫徹實行,因人事改易而引起的頻密政策轉變亦令在下面的官吏和平民無所適從。當時官員亦清楚仁宗這一點,故蔡襄曾在庆历改革之初,仁宗起用歐陽修、余靖及王素為諫官時,就曾提醒仁宗:“朝廷增用谏臣,修、靖、素一日并命,朝野相庆,然任谏非难,听谏为难,听谏非难,用谏为难。三人忠诚则正,必能尽言。臣恐邪人不利,必造为御之说。……愿陛下察之,毋使有好谏之名而无其实。」又曾指出仁宗“宽仁少断”、「不顓聽斷,不攬威權」。


\subsection{天圣}


\begin{longtable}{|>{\centering\scriptsize}m{2em}|>{\centering\scriptsize}m{1.3em}|>{\centering}m{8.8em}|}
  % \caption{秦王政}\
  \toprule
  \SimHei \normalsize 年数 & \SimHei \scriptsize 公元 & \SimHei 大事件 \tabularnewline
  % \midrule
  \endfirsthead
  \toprule
  \SimHei \normalsize 年数 & \SimHei \scriptsize 公元 & \SimHei 大事件 \tabularnewline
  \midrule
  \endhead
  \midrule
  元年 & 1023 & \tabularnewline\hline
  二年 & 1024 & \tabularnewline\hline
  三年 & 1025 & \tabularnewline\hline
  四年 & 1026 & \tabularnewline\hline
  五年 & 1027 & \tabularnewline\hline
  六年 & 1028 & \tabularnewline\hline
  七年 & 1029 & \tabularnewline\hline
  八年 & 1030 & \tabularnewline\hline
  九年 & 1031 & \tabularnewline\hline
  十年 & 1032 & \tabularnewline
  \bottomrule
\end{longtable}

\subsection{明道}

\begin{longtable}{|>{\centering\scriptsize}m{2em}|>{\centering\scriptsize}m{1.3em}|>{\centering}m{8.8em}|}
  % \caption{秦王政}\
  \toprule
  \SimHei \normalsize 年数 & \SimHei \scriptsize 公元 & \SimHei 大事件 \tabularnewline
  % \midrule
  \endfirsthead
  \toprule
  \SimHei \normalsize 年数 & \SimHei \scriptsize 公元 & \SimHei 大事件 \tabularnewline
  \midrule
  \endhead
  \midrule
  元年 & 1032 & \tabularnewline\hline
  二年 & 1033 & \tabularnewline
  \bottomrule
\end{longtable}

\subsection{景祐}

\begin{longtable}{|>{\centering\scriptsize}m{2em}|>{\centering\scriptsize}m{1.3em}|>{\centering}m{8.8em}|}
  % \caption{秦王政}\
  \toprule
  \SimHei \normalsize 年数 & \SimHei \scriptsize 公元 & \SimHei 大事件 \tabularnewline
  % \midrule
  \endfirsthead
  \toprule
  \SimHei \normalsize 年数 & \SimHei \scriptsize 公元 & \SimHei 大事件 \tabularnewline
  \midrule
  \endhead
  \midrule
  元年 & 1034 & \tabularnewline\hline
  二年 & 1035 & \tabularnewline\hline
  三年 & 1036 & \tabularnewline\hline
  四年 & 1037 & \tabularnewline\hline
  五年 & 1038 & \tabularnewline
  \bottomrule
\end{longtable}

\subsection{宝元}

\begin{longtable}{|>{\centering\scriptsize}m{2em}|>{\centering\scriptsize}m{1.3em}|>{\centering}m{8.8em}|}
  % \caption{秦王政}\
  \toprule
  \SimHei \normalsize 年数 & \SimHei \scriptsize 公元 & \SimHei 大事件 \tabularnewline
  % \midrule
  \endfirsthead
  \toprule
  \SimHei \normalsize 年数 & \SimHei \scriptsize 公元 & \SimHei 大事件 \tabularnewline
  \midrule
  \endhead
  \midrule
  元年 & 1038 & \tabularnewline\hline
  二年 & 1039 & \tabularnewline\hline
  三年 & 1040 & \tabularnewline
  \bottomrule
\end{longtable}

\subsection{康定}

\begin{longtable}{|>{\centering\scriptsize}m{2em}|>{\centering\scriptsize}m{1.3em}|>{\centering}m{8.8em}|}
  % \caption{秦王政}\
  \toprule
  \SimHei \normalsize 年数 & \SimHei \scriptsize 公元 & \SimHei 大事件 \tabularnewline
  % \midrule
  \endfirsthead
  \toprule
  \SimHei \normalsize 年数 & \SimHei \scriptsize 公元 & \SimHei 大事件 \tabularnewline
  \midrule
  \endhead
  \midrule
  元年 & 1040 & \tabularnewline\hline
  二年 & 1041 & \tabularnewline
  \bottomrule
\end{longtable}

\subsection{庆历}

\begin{longtable}{|>{\centering\scriptsize}m{2em}|>{\centering\scriptsize}m{1.3em}|>{\centering}m{8.8em}|}
  % \caption{秦王政}\
  \toprule
  \SimHei \normalsize 年数 & \SimHei \scriptsize 公元 & \SimHei 大事件 \tabularnewline
  % \midrule
  \endfirsthead
  \toprule
  \SimHei \normalsize 年数 & \SimHei \scriptsize 公元 & \SimHei 大事件 \tabularnewline
  \midrule
  \endhead
  \midrule
  元年 & 1041 & \tabularnewline\hline
  二年 & 1042 & \tabularnewline\hline
  三年 & 1043 & \tabularnewline\hline
  四年 & 1044 & \tabularnewline\hline
  五年 & 1045 & \tabularnewline\hline
  六年 & 1046 & \tabularnewline\hline
  七年 & 1047 & \tabularnewline\hline
  八年 & 1048 & \tabularnewline
  \bottomrule
\end{longtable}

\subsection{皇祐}

\begin{longtable}{|>{\centering\scriptsize}m{2em}|>{\centering\scriptsize}m{1.3em}|>{\centering}m{8.8em}|}
  % \caption{秦王政}\
  \toprule
  \SimHei \normalsize 年数 & \SimHei \scriptsize 公元 & \SimHei 大事件 \tabularnewline
  % \midrule
  \endfirsthead
  \toprule
  \SimHei \normalsize 年数 & \SimHei \scriptsize 公元 & \SimHei 大事件 \tabularnewline
  \midrule
  \endhead
  \midrule
  元年 & 1049 & \tabularnewline\hline
  二年 & 1050 & \tabularnewline\hline
  三年 & 1051 & \tabularnewline\hline
  四年 & 1052 & \tabularnewline\hline
  五年 & 1053 & \tabularnewline\hline
  六年 & 1054 & \tabularnewline
  \bottomrule
\end{longtable}

\subsection{至和}

\begin{longtable}{|>{\centering\scriptsize}m{2em}|>{\centering\scriptsize}m{1.3em}|>{\centering}m{8.8em}|}
  % \caption{秦王政}\
  \toprule
  \SimHei \normalsize 年数 & \SimHei \scriptsize 公元 & \SimHei 大事件 \tabularnewline
  % \midrule
  \endfirsthead
  \toprule
  \SimHei \normalsize 年数 & \SimHei \scriptsize 公元 & \SimHei 大事件 \tabularnewline
  \midrule
  \endhead
  \midrule
  元年 & 1054 & \tabularnewline\hline
  二年 & 1055 & \tabularnewline\hline
  三年 & 1056 & \tabularnewline
  \bottomrule
\end{longtable}

\subsection{嘉佑}

\begin{longtable}{|>{\centering\scriptsize}m{2em}|>{\centering\scriptsize}m{1.3em}|>{\centering}m{8.8em}|}
  % \caption{秦王政}\
  \toprule
  \SimHei \normalsize 年数 & \SimHei \scriptsize 公元 & \SimHei 大事件 \tabularnewline
  % \midrule
  \endfirsthead
  \toprule
  \SimHei \normalsize 年数 & \SimHei \scriptsize 公元 & \SimHei 大事件 \tabularnewline
  \midrule
  \endhead
  \midrule
  元年 & 1056 & \tabularnewline\hline
  二年 & 1057 & \tabularnewline\hline
  三年 & 1058 & \tabularnewline\hline
  四年 & 1059 & \tabularnewline\hline
  五年 & 1060 & \tabularnewline\hline
  六年 & 1061 & \tabularnewline\hline
  七年 & 1062 & \tabularnewline\hline
  八年 & 1063 & \tabularnewline
  \bottomrule
\end{longtable}


%%% Local Variables:
%%% mode: latex
%%% TeX-engine: xetex
%%% TeX-master: "../Main"
%%% End:

%% -*- coding: utf-8 -*-
%% Time-stamp: <Chen Wang: 2021-11-01 15:54:41>

\section{英宗趙曙\tiny(1063-1067)}

\subsection{生平}

宋英宗趙曙(1032年2月16日-1067年1月25日),原名趙宗實,是濮王赵允让之子,过继给宋仁宗为嗣,是北宋第五代皇帝,1063年5月1日—1067年1月25日在位。

趙曙是前任第四代皇帝宋仁宗的堂兄趙允讓的第十三子,是宋太宗第四子趙元份的後裔,生母為仙遊縣君任氏。天聖十年(明道元年)壬申年正月三日甲戌(1032年2月15日)生於宣平坊宅第〔嘉祐八年(1063年),英宗把這天定為“壽聖節”〕,最初,濮王夢兩龍與太陽一起掉落下來,用衣服裝住了它們,到英宗出生時,赤光滿室,有黃龍在赤光中游走。英宗幼年被仁宗接入皇宮撫養,賜名為宗實。1050年為岳州團練使,後為秦州防禦使。1055年立趙曙為嗣。

宋仁宗所生的三名兒子皆幼年夭折,而仁宗的兄弟也早逝(早於仁宗登基時逝世),故在嘉祐七年(1062年)立趙曙為皇太子,封鉅鹿郡公。嘉祐八年即帝位。

宋英宗時代對生父尊禮濮安懿王趙允讓的討論,引起了一系列政治事件。

宋英宗趙曙原是濮王趙允讓的兒子,過繼給宋仁宗為皇子。宋英宗即位後,討論崇奉濮王的典禮。治平元年(1064年),韓琦、歐陽修等奏請尊禮濮安懿王為皇考。尊禮之事引起與王珪、司馬光、呂誨、范純仁、呂大防等台諫大臣的不滿,主張稱濮王為皇伯。史稱濮議。呂誨、范純仁、呂大防等人被貶黜,治平三年(1066年),由於宋英宗強烈意願,使曹太后認可尊濮安懿王為皇考濮安懿皇。但是,趙允讓始終沒有獲得明確的皇帝尊號,隨著英宗的去世,事情不了了之。

英宗即位不久即病,無法處理朝政,由曹太后於內東門小殿垂簾聽政,待英宗病情好轉後,曹太后即撤簾歸政。

英宗雖然多病,行事甚至有些荒唐,但剛即位時,還是表現出了一個有為之君的風範。仁宗暴亡,醫官應當負有責任,主要的兩名醫官便被英宗逐出皇宮,送邊遠州縣編管。其他一些醫官,唯恐也遭貶謫。顯然,英宗行事很有些雷厲風行的風格,與濫施仁政的仁宗有著很大的不同。不僅如此,英宗也是一個很勤勉的皇帝。當時,輔臣奏事,英宗每每詳細詢問事情始末,方才裁決,處理政務非常認真。

英宗繼續任用仁宗時的改革派重臣韓琦、歐陽修、富弼等人,面對積弱積貧的國勢,力圖進行一些改革。

宋英宗治平三年任命司馬光設局專修《資治通鑑》,經費由政府資助,更准借閱秘閣藏書,並自選助手(劉恕、范祖禹、劉攽、司馬康),並提供筆硯文具、撥款、水果、糕點等,讓司馬光無後顧之憂的從事史書撰述。於神宗元豐七年成書,神宗並親自為此書作序。司馬光為了報答英宗的知遇之恩,在此後漫長的19年裡,將全部精力都耗在《資治通鑑》這部巨著的編纂上。應該說,史學巨著《資治通鑑》的最後編成也有英宗的一份功勞。

治平四年正月八日丁巳(1067年1月25日)英宗崩,享年36歲,殯於殿西階,廟號英宗,群臣上諡憲文肅武宣孝皇帝,八月二十七日癸酉,葬英宗於永厚陵(今河南鞏義孝義堡)。

英宗本人對於北宋中興抱有極大期望,相對其子神宗,政治手段也更為成熟。無奈壽短,使得北宋過早進入神宗朝,從而失掉了可能的中興計劃,為神宗朝王安石的變法提供了機會。

元朝官修正史《宋史》脱脱等的評價是:“昔人有言,天之所命,人不能违。信哉!英宗以明哲之资,膺继统之命,执心固让,若将终身,而卒践帝位,岂非天命乎?及其临政,臣下有奏,必问朝廷故事与古治所宜,每有裁决,皆出群臣意表。虽以疾疹不克大有所为,然使百世之下,钦仰高风,咏叹至德,何其盛也!彼隋晋王广、唐魏王泰窥觎神器,矫揉夺嫡,遂启祸原,诚何心哉!诚何心哉!”


\subsection{治平}


\begin{longtable}{|>{\centering\scriptsize}m{2em}|>{\centering\scriptsize}m{1.3em}|>{\centering}m{8.8em}|}
  % \caption{秦王政}\
  \toprule
  \SimHei \normalsize 年数 & \SimHei \scriptsize 公元 & \SimHei 大事件 \tabularnewline
  % \midrule
  \endfirsthead
  \toprule
  \SimHei \normalsize 年数 & \SimHei \scriptsize 公元 & \SimHei 大事件 \tabularnewline
  \midrule
  \endhead
  \midrule
  元年 & 1064 & \tabularnewline\hline
  二年 & 1065 & \tabularnewline\hline
  三年 & 1066 & \tabularnewline\hline
  四年 & 1067 & \tabularnewline
  \bottomrule
\end{longtable}



%%% Local Variables:
%%% mode: latex
%%% TeX-engine: xetex
%%% TeX-master: "../Main"
%%% End:

%% -*- coding: utf-8 -*-
%% Time-stamp: <Chen Wang: 2021-11-01 15:54:48>

\section{神宗赵顼\tiny(1067-1085)}

\subsection{生平}

宋神宗赵顼(1048年5月25日-1085年4月1日),本名趙仲鍼,宋英宗的长子,北宋第六代皇帝,1067年1月25日-1085年4月1日在位。

宋仁宗慶曆八年(1048年)四月十日,出生在濮安懿王宮邸睦親宅。宋英宗趙曙和宣仁聖烈皇后高氏所生長子。八月仁宗賜名為仲針,後被授為率府副率,後三次升遷至右千牛衛將軍。

嘉佑八年(1063年),趙頊與其父趙曙一起入居慶甯宮。三月二九日仁宗趙禎逝世,趙頊父趙曙即位,授趙頊為安州觀察使,封光國公。五月趙頊受經於東宮。宋英宗趙曙見了他好學到廢寢忘食,常派遣內侍去勸他休息。侍講王陶進入宮內,趙頊率弟弟趙顥向他參拜,可見對師傅的尊重。九月加封忠武軍節度使、同中書門下平章事,封淮陽郡王改名為趙頊。

治平元年(1064年),進封潁王。治平三年(1066年)三月納故相向敏中的孫女為夫人。十月英宗病重,趙頊按照宋仁宗時舊制,請求兩日一到邇英閣講讀,以安朝廷百官之心。十二月趙頊被立為皇太子。趙頊太子時喜讀《韓非子》,對法家「富國強兵」之術感興趣;還讀過王安石的《上仁宗皇帝言事書》,讚賞王安石的理財治國思想。

熙寧變法與新舊黨爭:神宗即位後,對北宋積貧積弱深感憂心,而他素來都欣賞王安石的才幹,故即位後命王安石推行變法,振興北宋王朝,是為王安石变法,又稱熙寧變法。

在王安石的主持下,均输、青苗、农田水利、免役、市易、保甲、方田均税、保马等新法相继出籠。新法几乎涵盖社会的各个方面,惟操之過急,利弊互見。北宋学者陆佃说:“造元丰间,积票塞上,盖数千万石,而四方常平之钱,不可胜计。”當時垦田面积大幅度增加,全国高达7億畝,城镇商品经济取得了空前发展。但是变法受到守旧派激烈的反對,朝中的司马光、范镇、赵瞻纷纷上书陈述对新法的不滿,司马光与吕惠卿为了青苗法在皇帝面前争辩,新法維持了将近二十年,直到司馬光盡罷新法為止。此一時期,面临朝廷和后宫的双重阻力,神宗受到的打擊可想而知,高太后更是对神宗说:“王安石是在变乱天下呀!”岐王赵颢也从旁劝说神宗应该遵从皇太后的懿旨,神宗心烦意乱,怒斥歧王說:「那你來當皇帝好了。」岐王诚惶诚恐,失声痛哭。

烏臺詩案:發生於宋神宗元豐二年(1079年),蘇軾於當年移知湖州,到任後上表謝恩,朝臣以其上表中用語,暗藏譏刺。御史何正臣上表彈劾蘇軾,指其「愚弄朝廷,妄自尊大」,又以蘇軾動輒歸咎新法,要求朝廷明正刑賞。御史李定曾因不服母孝,受蘇軾譏諷,於此案中也指蘇軾有「悛終不悔,其惡已著」、「傲悖之語,日聞中外」、「言偽而辯,行偽而堅」、「怨己不用」等四大可廢之罪。

御史舒亶尋摘蘇軾詩句,指其心懷不軌,譏諷神宗青苗法、助役法、明法科、興水利、鹽禁等政策。神宗下令拘捕,太常博士皇甫遵奉令前往逮人。蘇轍時在商丘已預知消息,託王適協助安置蘇軾家屬,並上書神宗陳情,願以官職贖兄長之罪。 蘇軾在9月被捕後,寫信給蘇轍交代身後之事,長子蘇邁則隨途照顧。押解至太湖,蘇軾曾意圖自盡,幾經掙扎,終未成舉。捕至御史臺獄下,御史臺依平日書信詩文往來,構陷牽連七十餘人。

後因太皇太后曹氏、王安禮等人出面力挽,前宰相王安石也說:「豈有聖世而殺才士者乎?」蘇軾終免一死,貶謫為「檢校尚書水部員外郎黃州團練副使本州安置」,前往黃州。蘇轍被貶江西筠州任酒監,平日與蘇軾往來者,如曾鞏、李清臣、張方平、黃庭堅、范鎮、司馬光等29人亦遭處分。張方平、司馬光和范鎮罰紅銅三十斤,其餘各罰紅銅二十斤。烏臺詩案於十二月結束。

時值夏惠宗在位,母黨梁氏專權,西夏國勢日非,圖一舉殲滅羌夏。王韶在庆州(今甘肃庆阳)大破夏军,占领西夏二千里土地。不过后来在永乐城之战中惨败,“厥后兵不敢用于北,而稍试于西,灵武之役,丧师覆将,涂炭百万。帝中夜得报,起,环榻行,彻旦不寐。”灭夏之举未能实现。事後,宋神宗在朝中當眾痛哭。他有抱負,勵精圖治,想滅西羌,惜壯志未酬,抱憾而歿。其子宋哲宗親政後,竭盡所能完成父親遺志。

元丰八年三月初五日(1085年4月1日),宋神宗在福宁殿去世,享年38岁,殡于殿西阶,庙号神宗,群臣上谥号为「英文烈武圣孝皇帝」,十月二十四日,葬于永裕陵。绍圣二年(1095年)九月,加谥为「绍天法古运德建功英文烈武钦仁圣孝皇帝」。崇宁三年(1104年)十一月,改谥为「体天显道帝德王功英文烈武钦仁圣孝皇帝」。政和三年(1113年)十一月,加谥为「体元显道法古立宪帝德王功英文烈武钦仁圣孝皇帝」。

元朝官修正史《宋史》脱脱等的評價是:“帝天性孝友,其入事两宫,必侍立终日,虽寒暑不变。尝与岐、嘉二王读书东宫,侍讲王陶讲谕经史,辄相率拜之,由是中外翕然称贤。其即位也,小心谦抑,敬畏辅相,求直言,察民隐,恤孤独,养耆老,振匮乏。不治宫室,不事游幸,历精图治,将大有为。未几,王安石入相。安石为人,悻悻自信,知祖宗志吞幽蓟、灵武,而数败兵,帝奋然将雪数世之耻,未有所当,遂以偏见曲学起而乘之。青苗、保甲、均输、市易、水利之法既立,而天下汹汹骚动,恸哭流涕者接踵而至。帝终不觉悟,方断然废逐元老,摈斥谏士,行之不疑。卒致祖宗之良法美意,变坏几尽。自是邪佞日进,人心日离,祸乱日起。惜哉!”

重用名將王韶,發動『熙河之役』,率軍擊潰羌人和西夏的軍隊,置熙州(今甘肅臨洮),收復河、洮、岷、宕、亹五州,對西夏形成包圍的之勢。熙宁五年(1072年)王韶收復今臨洮與臨夏,升镇洮军为熙州,設熙河路。王韶为龙图阁待制、熙河路馬步軍都总管、经略安抚使兼知熙州。熙宁六年(1073年)、夏天率兵攻占武胜城(今甘肃临洮),乘胜追擊,进攻河州(今甘肃东乡西南),直捣定羌城 (今甘肃广河)。熙寧七年,收回被吐蕃侵略的二十萬平方公里故土,以功迁礼部司郎中、枢密院直学士,史稱:「宋幾振矣!」。熙宁八年(1075年)宋廷在熙州(今甘肃临洮)、河州(今甘肃临夏)、洮州(今甘肃临潭)、岷州(治所今甘肃西和)、永宁寨(今甘肃甘谷)等地设州、买马,进行民族贸易,此舉受到了邊境各族的热烈欢迎,史載“熙河人情甚喜”。朝廷以王韶为经略安抚使,“通远军自置市易司以来,收本息钱五十七万余缗”,吐蕃政权逐渐瓦解,被稱為『熙河開邊』。

明末清初思想家王夫之《宋論》稱“宋政之乱,自神宗始”。

烏臺詩案是宋神宗執政生涯中,最大的政治汙點。因蘇軾當年移知湖州,到任後上表謝恩,卻被宰相王珪等朝臣認為有貶新政與朝廷之意,御史李定和舒亶尋摘蘇軾詩句,指其心懷不軌,譏諷神宗青苗法、助役法、明法科、興水利、鹽禁等政策。神宗下令拘捕,太常博士皇甫遵奉令前往逮人。捕至御史臺獄下,御史臺依平日書信詩文往來,構陷牽連七十餘人,引起朝野不小震動。在當時蘇軾已成為繼歐陽脩之後的宋朝文壇泰斗,卻因為一句詩下獄,讓時人議論紛紛,而宰相王珪等朝臣又想置蘇軾於死,讓時人對朝局頗為不滿,其中神宗祖母太皇太后曹氏對此事有所發言,認為蘇軾是先帝宋仁宗所選定的太平宰相。而曾任神宗宰相的大臣王安石則向神宗上書,認為:「豈有聖世而殺才士者乎?」神宗迫於壓力,在查清蘇軾並未有詆毀朝廷之意,將其貶至黃州,但已為神宗的執政留下汙點。

永樂城之戰是宋神宗執政以來,最大規模的征伐行動,也是神宗執政以來,最大的政治挫折。在當時因西夏外戚梁太后與其弟梁乙埋姐弟當權,國勢衰落,政治腐敗,西夏舉國上下怨聲載道,民不聊生。梁太后多次出兵攻宋,想提高國內政治威望,卻都慘敗而歸。神宗認為西夏無理,下令攻滅西夏。宋軍於元豐四年(1081年)11月在慶州(今甘肅慶陽)擊潰夏軍,占領西夏兩千多里土地。神宗大喜,命給事中徐禧、鄜延道總管种諤於元豐五年(1082年)9月帶兵攻夏,準備一舉滅夏。徐禧為了建戰功,拒聽种諤的建議(种諤反對興建永樂城),築永樂城,與种諤發生內鬨。由於徐禧好大喜功,聽不進諫言,導致宋軍內鬨,永樂城之戰大敗,宋軍損失二十萬軍士,徐禧、李舜舉、李稷、高永能等人死難,被宋人批評為「永樂之恥」。經過永樂城之戰的戰敗,宋神宗開始悔悟,而不再輕言開戰。

宋神宗赵顼被《时代》杂志认为是有史以来第三富有的人,也是最有权势的人物之一。尽管他在位仅18年,去世时才30多岁,但积累的财富不容小觑。在很强的科技创新和税收能力的帮助下,宋神宗在位期间的北宋国内生产总值占到全世界的25\%至30\%。

\subsection{熙宁}


\begin{longtable}{|>{\centering\scriptsize}m{2em}|>{\centering\scriptsize}m{1.3em}|>{\centering}m{8.8em}|}
  % \caption{秦王政}\
  \toprule
  \SimHei \normalsize 年数 & \SimHei \scriptsize 公元 & \SimHei 大事件 \tabularnewline
  % \midrule
  \endfirsthead
  \toprule
  \SimHei \normalsize 年数 & \SimHei \scriptsize 公元 & \SimHei 大事件 \tabularnewline
  \midrule
  \endhead
  \midrule
  元年 & 1068 & \tabularnewline\hline
  二年 & 1069 & \tabularnewline\hline
  三年 & 1070 & \tabularnewline\hline
  四年 & 1071 & \tabularnewline\hline
  五年 & 1072 & \tabularnewline\hline
  六年 & 1073 & \tabularnewline\hline
  七年 & 1074 & \tabularnewline\hline
  八年 & 1075 & \tabularnewline\hline
  九年 & 1076 & \tabularnewline\hline
  十年 & 1077 & \tabularnewline
  \bottomrule
\end{longtable}

\subsection{元丰}

\begin{longtable}{|>{\centering\scriptsize}m{2em}|>{\centering\scriptsize}m{1.3em}|>{\centering}m{8.8em}|}
  % \caption{秦王政}\
  \toprule
  \SimHei \normalsize 年数 & \SimHei \scriptsize 公元 & \SimHei 大事件 \tabularnewline
  % \midrule
  \endfirsthead
  \toprule
  \SimHei \normalsize 年数 & \SimHei \scriptsize 公元 & \SimHei 大事件 \tabularnewline
  \midrule
  \endhead
  \midrule
  元年 & 1078 & \tabularnewline\hline
  二年 & 1079 & \tabularnewline\hline
  三年 & 1080 & \tabularnewline\hline
  四年 & 1081 & \tabularnewline\hline
  五年 & 1082 & \tabularnewline\hline
  六年 & 1083 & \tabularnewline\hline
  七年 & 1084 & \tabularnewline\hline
  八年 & 1085 & \tabularnewline
  \bottomrule
\end{longtable}



%%% Local Variables:
%%% mode: latex
%%% TeX-engine: xetex
%%% TeX-master: "../Main"
%%% End:

%% -*- coding: utf-8 -*-
%% Time-stamp: <Chen Wang: 2019-12-26 10:31:55>

\section{哲宗\tiny(1085-1100)}

\subsection{生平}

宋哲宗赵煦(1077年1月4日-1100年2月23日),北宋第七位皇帝(1085年4月1日—1100年2月23日在位),為宋神宗第六子,母亲为钦成皇后朱氏。原名傭,曾封为延安郡王。生于熙宁九年十二月七日(1077年1月4日),神宗病危时立他为太子。元丰八年,神宗駕崩,赵煦登基为皇帝,是为宋哲宗,改元“元祐”。在位15年,得年二十三岁,葬于今天河南巩县的永泰陵。

宋哲宗生于熙宁九年十二月七日。最初名佣,授檢校太尉、天平軍節度使,并封均國公。元豐五年(1081年),遷開府儀同三司、彰武軍節度使,并進封延安郡王。元豐七年三月,逢神宗皇帝於集英殿宴請群臣,延安郡王陪同,因舉止有度,宰相於是恭賀神宗。元豐八年二月,神宗病重,宰相王珪請建儲君,并建議由皇太后暫時垂簾聽政。神宗同意后,當年三月,皇太后在福寧殿垂簾,將延安郡王為神宗祈福的手抄佛經出示給宰相等人。於是,奉制立延安郡王為皇太子。最初,太子宮中常有紅光。此後,紅光更如火光一般。

三月初五日,神宗駕崩,皇太子即皇帝位。次日,大赦天下,并遣使告丧於遼國。初七,尊高太后為太皇太后,向皇后为皇太后,生母德妃朱氏为皇太妃。并由宰相王珪为山陵使,营建神宗陵寢。三月二十一日,由于群臣再三请求,哲宗才正式与太皇太后一同听政。

哲宗登基时,只有9岁,由高太皇太后执政。高太皇太后执政后,任用保守派大臣司马光为宰相,“凡熙宁以来政事弗便者,次第罢之”。司马光上台後,不顧一切盡罷新法(熙宁变法),“举而仰听于太皇太后”。宋哲宗對此感到不满。

元祐八年(1093年),高太皇太后去世,哲宗亲政。哲宗亲政后表明紹述,追贬司马光,並贬谪苏轼、苏辙等舊黨黨人于岭南(今广西、廣東、海南),接着重用革新派如章惇、曾布等,恢复王安石变法中的保甲法、免役法、青苗法等,减轻农民负担,使国势有所起色。次年改元“绍圣”,并停止与西夏谈判,多次出兵讨伐西夏,迫使西夏向宋朝乞和。元符三年正月十二(1100年2月23日)病逝于汴梁(今河南开封)。

以章惇為相,主持恢復熙豐新法,史稱"紹述",北宋國力因而得以恢復發展,更取得對西夏的多次戰略性軍事勝利。

元朝官修正史《宋史》脱脱等的評價是:“哲宗以冲幼践阼,宣仁同政。初年召用马、吕诸贤,罢青苗,复常平,登俊良,辟言路,天下人心,翕然向治。而元祐之政,庶几仁宗。奈何熙、丰旧奸枿去未尽,已而媒蘖复用,卒假绍述之言,务反前政,报复善良,驯致党籍祸兴,君子尽斥,而宋政益敝矣。吁,可惜哉!”

哲宗是北宋较有作为的皇帝。不過在新黨與舊黨之間的黨爭始終未能獲得解決,反而在宋哲宗當政期間激化,多少造成朝廷的動盪。

哲宗親政後,絕大部分支持司馬光的舊黨黨人都被放逐,甚至於貶到嶺南等蠻荒地區;宰相章惇也進行言論控制,設立元祐提制局等單位對於反對新法的言論加以控制,甚至於在宮廷內部興獄。紹聖三年(1096年)章惇以巫蠱詛咒的罪名,要求宋哲宗廢宣仁太后所立的孟皇后,改立劉皇后,連宋哲宗都大嘆:「章惇壞我名節!」

1100年正月,宋哲宗患病,不数日死去,二月十日开工建陵,限五月十日完工。宋哲宗停丧七个月,于八月下葬永泰陵。陵台今日尚有17米高,底边每边长约50米,陵台正北有一段神墙残存,高约4米,是现今宋陵仅存的神墙遗迹。陵前石刻雕像还有11件保存完好,仅缺“象奴”一件(完整的为12件),以“象”的刻制最为生动完美,有“东陵狮子、西陵象”的说法。“西陵”就是指的永泰陵与东边的永裕陵相对。皇后刘氏陪葬在陵台西北,相距不足20米。永泰陵正西四五十米处,有哲宗第四女杨国公主墓。

仅石材一项,石匠就有4600多人,采用的各种石材达3万多块,调用民工、役兵超过1万人,山陵使章淳等官员督工急,民工们不堪虐待,纷纷逃亡,工地上饥饿、病、累而死的日日不断,死者多被弃尸荒野乱石之中。《采石场碑记》记载说:“居山土人皆云,至久积阴晦,常闻山中有若声役之歌者,意其不幸横夭者,沉魂未得解脱逍遥而然乎”?

1130年,刘齐政府盗掘北宋陵寝,陵上建筑被破坏殆尽,陵内洗劫一空。永泰陵的水晶注子卖到杭州,1148年南宋太常寺少卿方庭硕出使金朝,到宋陵,见各陵均被掘开,宋哲宗尸骨掷在永泰陵外,方庭硕脱下身上的袍服,将赵煦的尸骨包裹起来,重新置放陵中。后人曾有诗记述此事道:“先帝侍臣空洒泪,泰陵春望已模糊”。元朝初年,此陵再次遭劫。

\subsection{元祐}


\begin{longtable}{|>{\centering\scriptsize}m{2em}|>{\centering\scriptsize}m{1.3em}|>{\centering}m{8.8em}|}
  % \caption{秦王政}\
  \toprule
  \SimHei \normalsize 年数 & \SimHei \scriptsize 公元 & \SimHei 大事件 \tabularnewline
  % \midrule
  \endfirsthead
  \toprule
  \SimHei \normalsize 年数 & \SimHei \scriptsize 公元 & \SimHei 大事件 \tabularnewline
  \midrule
  \endhead
  \midrule
  元年 & 1086 & \tabularnewline\hline
  二年 & 1087 & \tabularnewline\hline
  三年 & 1088 & \tabularnewline\hline
  四年 & 1089 & \tabularnewline\hline
  五年 & 1090 & \tabularnewline\hline
  六年 & 1091 & \tabularnewline\hline
  七年 & 1092 & \tabularnewline\hline
  八年 & 1093 & \tabularnewline\hline
  九年 & 1094 & \tabularnewline
  \bottomrule
\end{longtable}

\subsection{绍圣}

\begin{longtable}{|>{\centering\scriptsize}m{2em}|>{\centering\scriptsize}m{1.3em}|>{\centering}m{8.8em}|}
  % \caption{秦王政}\
  \toprule
  \SimHei \normalsize 年数 & \SimHei \scriptsize 公元 & \SimHei 大事件 \tabularnewline
  % \midrule
  \endfirsthead
  \toprule
  \SimHei \normalsize 年数 & \SimHei \scriptsize 公元 & \SimHei 大事件 \tabularnewline
  \midrule
  \endhead
  \midrule
  元年 & 1094 & \tabularnewline\hline
  二年 & 1095 & \tabularnewline\hline
  三年 & 1096 & \tabularnewline\hline
  四年 & 1097 & \tabularnewline\hline
  五年 & 1098 & \tabularnewline
  \bottomrule
\end{longtable}

\subsection{元符}

\begin{longtable}{|>{\centering\scriptsize}m{2em}|>{\centering\scriptsize}m{1.3em}|>{\centering}m{8.8em}|}
  % \caption{秦王政}\
  \toprule
  \SimHei \normalsize 年数 & \SimHei \scriptsize 公元 & \SimHei 大事件 \tabularnewline
  % \midrule
  \endfirsthead
  \toprule
  \SimHei \normalsize 年数 & \SimHei \scriptsize 公元 & \SimHei 大事件 \tabularnewline
  \midrule
  \endhead
  \midrule
  元年 & 1098 & \tabularnewline\hline
  二年 & 1099 & \tabularnewline\hline
  三年 & 1100 & \tabularnewline
  \bottomrule
\end{longtable}



%%% Local Variables:
%%% mode: latex
%%% TeX-engine: xetex
%%% TeX-master: "../Main"
%%% End:

%% -*- coding: utf-8 -*-
%% Time-stamp: <Chen Wang: 2021-11-01 15:55:00>

\section{徽宗赵佶\tiny(1100-1125)}

\subsection{生平}

宋徽宗赵佶(1082年6月7日-1135年6月4日),北宋第八位皇帝,自稱教主道君皇帝,同時也是畫家、書法家、詩人、詞人和收藏家,且擅長古琴、蹴鞠、擊鞠、打獵,自創“瘦金書”字體。徽宗在書畫上的花押大概是中國歷史上最出名的花押。

徽宗為宋神宗十一子,宋哲宗之弟,先後被封为遂宁王、端王。其兄长宋哲宗于公元1100年正月病逝时无子,向太后于同月立他为帝,並垂簾聽政一年,第二年改年號为“建中靖國”。在位26年(1100年2月23日—1126年1月18日),国亡被俘因病而死,终年54岁,葬于都城绍兴永祐陵(今浙江省绍兴市柯桥区东南35里处)。

被後世評為「宋徽宗诸事皆能,獨不能為君耳!」编写《宋史》的史官,也感慨地说,如果當初章惇「端王轻佻,不可君天下」的意见被采纳,北宋也許是另一種结局,“宋不立徽宗,金雖强,何衅以伐宋哉。”

徽宗為宋神宗第十一子,宋哲宗之弟,本未必有機會繼承大統。惟宋哲宗23歲英年早逝,無子,故宋室由哲宗眾弟中尋找繼承人。本來哲宗眾弟中以申王趙佖最長,惜因患有眼疾而不被選為繼位者,故以當時封為端王的徽宗繼承大統。宰相章惇曾反對端王繼位,反而建議立哲宗同母弟簡王趙似,但向太后支持端王繼位,故徽宗順利成為大宋皇帝。

建中靖國元年(1101年),向太后去世,徽宗親政。宋徽宗親政後,“妄耗百出,不可勝數”,过分追求奢侈生活,在南方采办“花石纲”,搜集奇花异石运到汴京開封府,修建艮岳等工程浩大的园林宫殿,崇信道教,尊號「教主道君皇帝」,任用贪官宦官横徵暴敛,激起各地民变。其中以新黨蔡京任丞相與宦官童貫為將軍所引致的問題最嚴重。

徽宗好大喜功,不顾宋辽已百年和平相处,于宣和二年(1120年),与金国结成“海上之盟”,联合灭辽。1122年,金军攻克辽南京(今北京市)。

宣和七年(1125年)十月,金太宗遣諳班勃極烈完颜斜也、完颜宗望、乙室勃極烈完颜宗翰分兩路南下入侵北宋。宣和七年十二月二十三日(1126年1月18日),徽宗无法应付时,急忙禪讓天子的寶座给他儿子宋钦宗去对付,自己则当“太上皇”并出逃,但终究无法挽回局势。金军暂退后,徽宗回京,居龙德宫,实际上被钦宗软禁,甚至连在过寿时给钦宗敬的酒钦宗也不喝,气得哭着回宫。

靖康元年(1126年)八月,金太宗再次命東、西兩路軍大舉南下,《三朝北盟会编卷六十九》等史书记载宋兵部尚書孫傅把希望放在禁军老兵郭京身上,郭京伪称精通佛道二教之法术,能施道门“六甲法”,用七千七百七十七人布阵,并会佛教“毘沙門天王像法”,布阵画像,但神兵大敗,金兵分四路乘機攻入城內,金軍攻佔了帝都汴京。宋欽宗遣使臣何㮚到金營請和,宗翰、宗望二帥不允。金军提出见徽宗,钦宗不肯。北宋靖康二年(金朝天會五年)正月,钦宗亲自请和被扣押,宋将范琼变节将徽宗、宗室、后妃公主等交给金军。二月初六(1127年3月20日),金太宗下詔廢徽、欽二帝,貶為庶人,北宋滅亡,二帝被俘北上。七月二十日,二帝遷到中京(今北京市),父子抱头痛哭。

天會六年(1128年)八月二十一日抵達金上京會寧府。二十四日,二帝及男女宋俘均坦胸赤背,身披羊皮,跪拜金太祖廟,行「牽羊禮」,在乾元殿拜謁金太宗完颜吳乞買。金太宗封宋徽宗為昏德公,欽宗為重昏侯,十月二十六日,二帝遷往韓州(辽宁省昌图八面城)。在韩州,金人将城内女真住户全部迁出,只供二帝等二千余宋俘居住。据《宋俘记》载:「给田四十五顷,种莳自给。」据《南征錄彙》说这还是金国二太子完颜宗望(劫宋徽宗之女茂德帝姬为妻)格外开恩,要求性格兇狠的完颜宗翰等不可像虐待辽天祚帝那样对待宋朝的徽、欽兩帝。

天會八年(1130年)七月,又將二帝遷到五國城(今黑龍江省依蘭縣城北舊古城)軟禁。到达五国城时,隨行男女仅140余人。流放期间徽宗仍雅好寫詩,读唐代李泌傳,感触颇深。五年后,天會十三年(绍兴五年,1135年)四月,病死於五國城。照當地習俗火葬。

皇统元年(1141年)二月,金熙宗为改善与南宋的关系,将死去的徽宗追封为天水郡王,将钦宗封为天水郡公。第一提高了级别,原来封徽宗为二品昏德公,追封为王升为一品,原封钦宗为三品重昏侯,现封为公升为二品。第二是去掉了原封号中的侮辱含义。第三是以赵姓天水郡望之为封号,以示尊重。同时南宋朝廷解除了岳飛、韓世忠、刘錡、杨沂中等將領的兵權,为《绍兴和议》做好了准备。十一月间,宋、金为《绍兴和议》达成书面协议。十二月末除夕夜(1142年1月27日),宋高宗赐死岳飞,据《宋史》载是为了满足完颜宗弼议和所设前提。紹興十二年(1142年)三月,宋金《绍兴和议》彻底完成所有手续。夏四月丁卯(1142年5月1日),高宗生母韋賢妃同徽宗棺槨歸宋。同年八月十餘辆牛车到達临安,十月,宋高宗将徽宗暂葬于会稽(今浙江省绍兴市),名曰永固陵(後改名永祐陵)。

三国时期曹魏李康《运命论》:“夫黄河清而圣人生。”宋徽宗在位時,曾出現過三次“河清” - 黃河變得清澈的奇觀,使得當時百官弹冠相庆,大肆歌功颂德。在黄河中下游,河水偶爾也有暫時变得清澈的时候,即史书中作为祥瑞记下的“河清”,并不是五百年乃至千年一遇。据地质学史专家李鄂荣先生考证,中国历史上的“河清”,有记载可查的便有43次,首见于汉桓帝延熹八年(165年),如从此时起算,平均不到40年就有一次。

根据《宋史》,宋徽宗在位年间的三次“河清”,分别为:第一次,大观元年(1107年),“乾宁军、同州黄河清。”第二次,大观二年(1108年),“同州黄河清。”,第三次,大观三年(1109年),“陕州、同州黄河清。”

大观元年(1107年)“乾宁军言黄河清,逾八百里,凡七昼夜,诏以乾宁军为清州”。“黄河清”被谱写成新曲流传,还在韩城建立记载这些祥瑞的“河渎碑”。此碑至今尚在。

可是僅僅到了立碑15年后的1127年,宋徽宗便和他的儿子宋钦宗一起被金兵俘虏,押到了金朝统治下的东北地区,北宋至此滅亡。被民眾譏為“聖人豈女真人乎?”

在宋代的社会风气中,文人雅士和歌伎、妓女交往属于正常现象。民间流传宋徽宗十分喜欢青楼女子李师师。李师师是北宋东京有名的艺伎,色艺双绝,诗词歌赋、笙管笛箫样样精通,宋徽宗得知后不顾九五之尊,数次前去青楼与李师师见面。后来在皇宫和妓院之间挖了一条地道,方便和李师师见面,现在在开封的宋城遗址当中,还能看到这个神秘地道的一点痕迹。宋徽宗还和李师师旧时相好的著名的词人周邦彦,争风吃醋,不久便在整个东京城传得沸沸扬扬。

宋徽宗酷爱艺术,在位时将藝術的地位提到在中国历史上最高的位置,成立翰林书画院,即当时的宫廷画院。

更特別的是以画作为科举升官的一种考试方法,每年以诗词做题目曾刺激出许多新的创意佳话。如题目为“山中藏古寺”,许多人画深山寺院飞檐,但得第一名的没有画任何房屋,只画了一个和尚在山溪挑水;另题为“踏花归去马蹄香”,得第一名的没有画任何花卉,只画了一人骑马,有蝴蝶飞绕马蹄间,凡此等等。这些都极大地刺激了中国画意境的发展。

他对自然观察入微,曾写到:“孔雀登高,必先举左腿”等有关绘画的理论文章。广泛搜集历代文物,令下属编辑《宣和书谱》、《宣和画谱》、《宣和博古录》等著名美术史书籍。对研究美术史有相当大的贡献。

赵佶还喜爱在自己喜欢的书画上题诗作跋,后人把这种画叫“御题画”。由于许多画上并没有留下作者的名字,他本人又擅长繪画。对鉴别这些画是否是赵佶的作品有不小的难度。一般認為《詩帖》在內的書法,以及粗筆的《柳鸭图》和《池塘秋晚图》為其繪畫風格,而細筆的《芙蓉锦鸡图》、《腊梅山禽图》、《竹禽图》、《四禽图》等御题画則尚無定論。

宋徽宗在其創作的書畫上使用一個類似拉長了的「天」字的花押,據說象徵「天下一人」。這也是中國歷史上最出名的花押。

章惇反對宋徽宗即位時曾說:「端王輕佻,不可以君天下。」

元朝官修正史《宋史》脱脱等的評價是:“宋中叶之祸,章、蔡首恶,赵良嗣厉阶。然哲宗之崩,徽宗未立,惇谓其轻佻不可以君于下。辽天祚之亡,张觉举平州来归,良嗣以为纳之失信于金,必启外侮。使二人之计行,宋不立徽宗,不纳张觉,金虽强,何衅以伐宋哉?以是知事变之来,虽小人亦能知之,而君子有所不能制也。迹徽宗失国之由,非若晋惠之愚、孙皓之暴,亦非有曹、马之篡夺,特恃其私智小慧,用心一偏,疏斥正士,狎近奸谀。于是蔡京以獧薄巧佞之资,济其骄奢淫佚之志。溺信虚无,崇饰游观,困竭民力。君臣逸豫,相为诞谩,怠弃国政,日行无稽。及童贯用事,又佳兵勤远,稔祸速乱。他日国破身辱,遂与石晋重贵同科,岂得诿诸数哉?昔西周新造之邦,召公犹告武王以不作无益害有益,不贵异物贱用物,况宣、政之为宋,承熙、丰、绍圣椓丧之余,而徽宗又躬蹈二事之弊乎?自古人君玩物而丧志,纵欲而败度,鲜不亡者,徽宗甚焉,故特著以为戒。”

《靖康稗史箋證》卷5《呻吟語》記:「二王令成棣譯詢宮中事:道宗五七日必御一處女,得御一次即畀位號,續幸一次進一階。退位後,出宮女六千人,宜其亡國。」

元朝脱脱撰《宋史》的《徽宗记》,不由掷笔叹曰:“宋徽宗诸事皆能,独不能为君耳!”

1964年3月24日,毛泽东在一次谈话评点知识分子型皇帝说:“可不要看不起老粗。知识分子是比较最没有出息的。历史上当皇帝,有许多是知识分子,是没有出息的,隋炀帝就是一个会做文章、诗词的人。陈后主、李后主都是能诗能赋的人。宋徽宗既能写诗,又能绘画。一些老粗能办大事情,成吉思汗、刘邦、朱元璋。”


\subsection{建中靖国}


\begin{longtable}{|>{\centering\scriptsize}m{2em}|>{\centering\scriptsize}m{1.3em}|>{\centering}m{8.8em}|}
  % \caption{秦王政}\
  \toprule
  \SimHei \normalsize 年数 & \SimHei \scriptsize 公元 & \SimHei 大事件 \tabularnewline
  % \midrule
  \endfirsthead
  \toprule
  \SimHei \normalsize 年数 & \SimHei \scriptsize 公元 & \SimHei 大事件 \tabularnewline
  \midrule
  \endhead
  \midrule
  元年 & 1101 & \tabularnewline
  \bottomrule
\end{longtable}

\subsection{崇宁}

\begin{longtable}{|>{\centering\scriptsize}m{2em}|>{\centering\scriptsize}m{1.3em}|>{\centering}m{8.8em}|}
  % \caption{秦王政}\
  \toprule
  \SimHei \normalsize 年数 & \SimHei \scriptsize 公元 & \SimHei 大事件 \tabularnewline
  % \midrule
  \endfirsthead
  \toprule
  \SimHei \normalsize 年数 & \SimHei \scriptsize 公元 & \SimHei 大事件 \tabularnewline
  \midrule
  \endhead
  \midrule
  元年 & 1102 & \tabularnewline\hline
  二年 & 1103 & \tabularnewline\hline
  三年 & 1104 & \tabularnewline\hline
  四年 & 1105 & \tabularnewline\hline
  五年 & 1106 & \tabularnewline
  \bottomrule
\end{longtable}

\subsection{大观}

\begin{longtable}{|>{\centering\scriptsize}m{2em}|>{\centering\scriptsize}m{1.3em}|>{\centering}m{8.8em}|}
  % \caption{秦王政}\
  \toprule
  \SimHei \normalsize 年数 & \SimHei \scriptsize 公元 & \SimHei 大事件 \tabularnewline
  % \midrule
  \endfirsthead
  \toprule
  \SimHei \normalsize 年数 & \SimHei \scriptsize 公元 & \SimHei 大事件 \tabularnewline
  \midrule
  \endhead
  \midrule
  元年 & 1107 & \tabularnewline\hline
  二年 & 1108 & \tabularnewline\hline
  三年 & 1109 & \tabularnewline\hline
  四年 & 1110 & \tabularnewline
  \bottomrule
\end{longtable}

\subsection{政和}

\begin{longtable}{|>{\centering\scriptsize}m{2em}|>{\centering\scriptsize}m{1.3em}|>{\centering}m{8.8em}|}
  % \caption{秦王政}\
  \toprule
  \SimHei \normalsize 年数 & \SimHei \scriptsize 公元 & \SimHei 大事件 \tabularnewline
  % \midrule
  \endfirsthead
  \toprule
  \SimHei \normalsize 年数 & \SimHei \scriptsize 公元 & \SimHei 大事件 \tabularnewline
  \midrule
  \endhead
  \midrule
  元年 & 1111 & \tabularnewline\hline
  二年 & 1112 & \tabularnewline\hline
  三年 & 1113 & \tabularnewline\hline
  四年 & 1114 & \tabularnewline\hline
  五年 & 1115 & \tabularnewline\hline
  六年 & 1116 & \tabularnewline\hline
  七年 & 1117 & \tabularnewline\hline
  八年 & 1118 & \tabularnewline
  \bottomrule
\end{longtable}

\subsection{重和}

\begin{longtable}{|>{\centering\scriptsize}m{2em}|>{\centering\scriptsize}m{1.3em}|>{\centering}m{8.8em}|}
  % \caption{秦王政}\
  \toprule
  \SimHei \normalsize 年数 & \SimHei \scriptsize 公元 & \SimHei 大事件 \tabularnewline
  % \midrule
  \endfirsthead
  \toprule
  \SimHei \normalsize 年数 & \SimHei \scriptsize 公元 & \SimHei 大事件 \tabularnewline
  \midrule
  \endhead
  \midrule
  元年 & 1118 & \tabularnewline\hline
  二年 & 1119 & \tabularnewline
  \bottomrule
\end{longtable}

\subsection{宣和}

\begin{longtable}{|>{\centering\scriptsize}m{2em}|>{\centering\scriptsize}m{1.3em}|>{\centering}m{8.8em}|}
  % \caption{秦王政}\
  \toprule
  \SimHei \normalsize 年数 & \SimHei \scriptsize 公元 & \SimHei 大事件 \tabularnewline
  % \midrule
  \endfirsthead
  \toprule
  \SimHei \normalsize 年数 & \SimHei \scriptsize 公元 & \SimHei 大事件 \tabularnewline
  \midrule
  \endhead
  \midrule
  元年 & 1119 & \tabularnewline\hline
  二年 & 1120 & \tabularnewline\hline
  三年 & 1121 & \tabularnewline\hline
  四年 & 1122 & \tabularnewline\hline
  五年 & 1123 & \tabularnewline\hline
  六年 & 1124 & \tabularnewline\hline
  七年 & 1125 & \tabularnewline
  \bottomrule
\end{longtable}



%%% Local Variables:
%%% mode: latex
%%% TeX-engine: xetex
%%% TeX-master: "../Main"
%%% End:

%% -*- coding: utf-8 -*-
%% Time-stamp: <Chen Wang: 2021-11-01 15:55:04>

\section{钦宗赵桓\tiny(1126-1127)}

\subsection{生平}

宋钦宗赵桓(1100年5月23日-1161年6月14日),北宋第九位皇帝(1126年1月19日-1127年3月20日在位)。宋徽宗赵佶长子,谥号「恭文顺德仁孝皇帝」。

靖康之耻:赵桓1100年5月23日生于坤宁殿。初名赵亶,封韩国公,明年六月进封京兆郡王。崇宁元年二月甲午,更名赵烜,十一月丁亥,改名赵桓。大观二年正月,进封定王。政和元年三月,讲学于资善堂。三年正月,加太保。四年二月癸酉,在文德殿加冠。五年二月乙巳,立为皇太子,大赦天下。丁巳,谒太庙。诏乘金辂,设卤簿,如至道、天禧故事,宫僚参谒、称臣,都辞让。六年六月癸未,纳朱氏为太子妃。

北宋末年,金朝南侵。赵桓受父宋徽宗禪讓后即位为皇帝,是为宋钦宗,改元靖康。徽宗称太上皇,率亲信蔡京、童贯等人逃奔江南。钦宗即位後立刻貶被称为奸臣的蔡京、童贯等人,然后重用李纲抗金,一度令金军有北返之意。后来钦宗听从奸臣谗言,罢免了李纲,向金朝求和,加上在背後受徽宗牽制,作用有限。钦宗装作恭顺迎接太上皇回京后即贬去太上皇左右,软禁太上皇于龙德宫,因为对太上皇怀有戒心,甚至在太上皇过寿时连太上皇敬的酒都不喝,气得太上皇哭着回宫。钦宗亦将蔡京、童贯等“六贼”皆贬杀。

但钦宗在抗金时战和不定,又写密信交给金使萧仲恭意图策反降金的原辽将耶律余睹,给了金朝再度发难的理由。金朝趁此机会再次南下渡黄河再次包围宋京东京(今开封)。宋钦宗误信宰相何栗、枢密使孙傅,以为无赖郭京可以用神兵退敌,提拔郭京,并撤去东京外城守备让郭京用神兵退敌,郭京大败,金军趁机攻破东京外城,同时又放言和谈让钦宗心存幻想。钦宗率大臣亲赴金营和谈,金军趁机派萧庆入住北宋尚书省。钦宗在上表投降称藩及同意召在外的皇弟康王赵构回京并废除其大元帅头衔等条件后被放回,萧庆则以北宋朝廷名义搜刮钱物满足金军所需。靖康二年(1127年)正月,钦宗再赴金营求和,被扣押,金军以放回钦宗为条件继续向北宋索要财物,宋将范琼等将太上皇、钦宗太子赵谌及后妃、亲王、公主等交给金军作为抵债,史称靖康之变。

北宋靖康二年(金朝天會五年)二月初六(1127年3月20日),金太宗下詔廢徽、欽二帝,貶為庶人,強行脱去二帝龙袍,随行的李若水抱著欽宗身體,嚴斥金人为狗賊之辈,金人將其割喉殘忍殺害,並册封张邦昌为帝,国号大楚,他成為亡國之君,立國一百六十八年的北宋滅亡。

金国拘禁:七月,二帝被俘北上,遷到燕京,天會六年(1128年)八月二十一日抵達金上京會寧府。二十四日,二帝著素服跪拜金太祖廟,行「牽羊禮」,在乾元殿拜謁金太宗完顏吳乞買。金太宗封宋徽宗為昏德公,欽宗為重昏侯,十月,二帝遷往韓州(吉林省梨樹縣北偏臉城)。天會八年(1130年)七月,又將二帝遷到五國城(今黑龍江省依蘭縣城北舊古城)軟禁。

八年後,天會十三年(紹興五年,1135年)四月,徽宗病死於五國城。

天眷三年(1140年),金主战派完颜宗弼率领金国军队南侵,先在开封正南的顺昌败于刘錡所部的「八字军」,再于开封西南的郾城和颍昌,在金国女真精锐部队所拿手的骑兵对阵中两次败于岳飞的岳家军,只在开封东南面的淮西亳州、宿州一带战胜了宋军中最弱的张俊一军,在宋高宗以「十二道金牌」敕召回岳家军前,金军已被压缩到汴梁东部和北部。完颜宗弼开始转向接受與宋议和。

皇统元年(1141年) 二月,金熙宗为与南宋達成完颜宗弼期望中的和议,将死去的徽宗追封为天水郡王,将钦宗封为天水郡公。第一提高了级别,原来封徽宗为二品昏德公,追封为王升为一品,原封钦宗为三品重昏侯,现封为公升为二品。第二是去掉了原封号中的侮辱含义。第三是以赵姓天水郡望作为封号,以示尊重。同时南宋朝廷解除了岳飛、韓世忠、刘錡、杨沂中等將的兵權,为《绍兴和议》做好了准备。十一月间,宋、金为《绍兴和议》达成书面协议。十二月末除夕夜(1142年1月27日),宋高宗在餘杭風波亭賜死了岳飞,据《宋史》记载是为了满足议和所设前提。

紹興十二年(1142年)三月,宋金《绍兴和议》彻底完成所有手续。夏四月丁卯(1142年5月1日),宋高宗生母韋賢妃同徽宗棺槨歸宋。离行时,钦宗挽住她的车轮,请她转告高宗,若能回去,他絕對不爭權,不當皇帝,只要當個「太乙宫」之主就滿足了。韋賢妃哭著說,如果我看不到你回來,我寧願眼睛瞎掉算了。宋高宗曾提出钦宗南归,金朝也答应了,高宗令临安府为钦宗修宫殿;但金朝出现政局变动,执政者希望保留在汴京拥立钦宗复辟以挑战宋高宗合法性的选择,钦宗南归一事遂被搁置。韋賢妃晚年果真因眼疾而雙目俱盲,後遇道人治癒一眼,另一眼則因道人曰:「后以一目視,足矣。以一目存誓,可也。」而未獲醫治。

南宋紹興二十六年(金朝正隆元年,1156年)六月,宋钦宗去世。关于死因众说纷纭,有指钦宗是病死。另据《大宋宣和遺事》记载,金朝皇帝完顏亮叫57歲的钦宗和81岁的辽天祚帝耶律延禧去比賽馬球,宋钦宗從馬上跌下來,被馬亂踐而死。

宋绍兴三十一年(金正隆六年,1161年),钦宗的死讯才传到南宋,宋高宗发喪。欽宗死后,高宗再无后顧之憂,因而翌年传位于宋孝宗。

元朝官修正史《宋史》脱脱等的評價是:“帝在东宫,不见失德。及其践阼,声技音乐一无所好。靖康初政,能正王黼、朱勔等罪而窜殛之,故金人闻帝内禅,将有卷甲北旆之意矣。惜其乱势已成,不可救药,君臣相视,又不能同力协谋,以济斯难,惴惴然讲和之不暇。卒致父子沦胥,社稷芜茀。帝至于是,盖亦巽懦而不知义者欤!享国日浅,而受祸至深,考其所自,真可悼也夫!真可悼也夫!”

《靖康稗史》金国方面所记录的俘虏供词:二王令成棣译询宫中事:……少帝贤,务读书。不迩声色。受禅半年,无以备执事,乃立一妃、十夫人,仅三人得幸。自余俭德,不可举数。

\subsection{靖康}


\begin{longtable}{|>{\centering\scriptsize}m{2em}|>{\centering\scriptsize}m{1.3em}|>{\centering}m{8.8em}|}
  % \caption{秦王政}\
  \toprule
  \SimHei \normalsize 年数 & \SimHei \scriptsize 公元 & \SimHei 大事件 \tabularnewline
  % \midrule
  \endfirsthead
  \toprule
  \SimHei \normalsize 年数 & \SimHei \scriptsize 公元 & \SimHei 大事件 \tabularnewline
  \midrule
  \endhead
  \midrule
  元年 & 1126 & \tabularnewline\hline
  二年 & 1127 & \tabularnewline
  \bottomrule
\end{longtable}



%%% Local Variables:
%%% mode: latex
%%% TeX-engine: xetex
%%% TeX-master: "../Main"
%%% End:


%%% Local Variables:
%%% mode: latex
%%% TeX-engine: xetex
%%% TeX-master: "../Main"
%%% End:
