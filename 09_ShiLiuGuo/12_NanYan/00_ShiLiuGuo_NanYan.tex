%% -*- coding: utf-8 -*-
%% Time-stamp: <Chen Wang: 2019-12-19 16:23:37>


\section{南燕\tiny(398-405)}

\subsection{简介}

南燕(398年-410年)是中國南北朝時五胡十六国中,由鮮卑慕容部的慕容德所建立的國家,是慕容氏諸燕之一。国号燕。

慕容德原是後燕宗室范陽王。397年,當後燕君主慕容宝於參合陂之戰為北魏所敗之後,后燕被截成南北两部分。次年慕容德於滑台(今河南滑县)自稱燕王,拒绝接纳逃难的慕容宝,甚至险些将其杀害。公元400年迁广固(今山东青州西北)稱帝。南燕的国土,东到大海,南达泗上,西至巨野泽,北临黄河,共有十五个郡、八十二个县,约三十三万户,基本上就是原西晋的青州。统治范围包括今山东、河南、江苏各一部分。慕容德将南燕国土一分为五:青州,治所设在东莱(今山东莱州);幽州,治所设在发干(今山东沂水县西北);徐州,治所设在东莞(今山东莒县);兖州,治所设在梁父(今山东泰安南);并州,治所设在阴平(今江苏沭阳)。所以南燕官方在提到本国疆域时,常自称“五州之地”。

后主慕容超在位時,被東晋的劉裕擊敗,經歷兩代後滅國。

“南燕”之别称,始于当时人张诠所写《南燕书》(已佚),因相对于北燕位于南方故名。

%% -*- coding: utf-8 -*-
%% Time-stamp: <Chen Wang: 2019-12-19 16:24:33>

\subsection{献武帝\tiny(398-405)}

\subsubsection{生平}

燕献武帝慕容德(336年-405年11月17日),後改名慕容備德,字玄明,十六國時期南燕皇帝,鮮卑人,是前燕文明帝慕容皝之幼子,前燕景昭帝慕容儁、後燕成武帝慕容垂皆為其兄。《晉書》載其「年未弱冠,身長八尺二寸,姿貌雄偉」。又「博觀群書,性清慎,多才藝」。

前燕時期慕容儁在位時,慕容德被封為梁公。後來慕容儁之子慕容暐繼帝位,再被改封為范陽王。369年,曾與慕容垂一同大敗東晉桓溫的進攻。370年,前燕為前秦所滅後,一度被前秦帝苻堅任命為張掖(今中國甘肅省張掖市)太守,數年後被免職。

後來苻堅欲南征東晉,慕容德被任命為奮威將軍隨軍,留下金刀拜別母親公孫氏及胞兄原北海王慕容納而去。383年,前秦於淝水之战敗北,苻坚宠妃张夫人走失投靠慕容暐,慕容暐送她回京,慕容德阻止并劝他趁乱复国,未果。

后来慕容垂趁機起兵建後燕,慕容德嚮應之,被命為車騎大將軍,重新受封為范陽王,然其諸子及慕容纳皆因留在後方而被前秦所殺。

396年,慕容垂臨終,遺命太子慕容寶將鄴城(今中國河南省臨漳縣)委由慕容德鎮守。慕容寶繼位後,即以慕容德為使持節、都督冀、兗、青、徐、荊、豫六州諸軍事、特進、車騎大將軍、冀州牧,領南蠻校尉,鎮守鄴城。

397年,北魏攻擊後燕,後燕兵敗如山倒,皇帝慕容寶向北方故地逃亡,後燕國土被截為南北二部,位在南方的慕容德被慕容寶任命為丞相,領冀州牧。不久,慕容垂另一子趙王慕容麟來逃至鄴城,以鄴城難守,建議慕容德南遷滑台(今中國河南省滑縣)。398年正月,慕容德又受慕容麟建議先稱燕王,稱燕王元年,史稱此一政權為南燕。次年(399年),再遷廣固(今中國山東省青州市),以為都城。

400年,慕容德正式稱帝,改元建平,並在此時把自己名字改為慕容備德,以便臣民避諱。慕容德正式登基時年已65歲,在中國歷史上僅次於唐朝的武則天,武則天登基時已經67歲了。

慕容备德不知道母亲公孙夫人和胞兄慕容纳都已经不在人世,曾于建平二年(401年)十月派平原人杜弘去长安寻访。杜弘说:“臣至长安,若不能得知太后动止,当西往张掖,以死效力。”并为自己年逾六十的父亲杜雄乞求本县县令之职。慕容备德不顾中书令张华反对,认为杜弘“为君迎母,为父求禄,忠孝备矣,何罪之有!”以杜雄为平原令。杜弘到张掖为盗贼所杀。四年(403年),慕容备德旧部赵融从长安前来,告知公孙夫人和慕容纳的死讯,慕容备德放声痛哭以至于吐血,因而卧病不起,从此健康恶化。

慕容备德有女兒無兒子,為繼承人心焦,慕容納之子慕容超持當年慕容德拜別母親的金刀來歸,慕容备德遂以慕容超袭封北海王,后立為太子。

建平六年九月戊午(405年11月17日),慕容备德去世,慕容超繼位。去世當晚從四方城門抬出十餘口棺木,秘密埋葬在山谷之中,因此實際上他並未葬於其陵寢「東陽陵」,後人遂不知其安葬之處。慕容德後來被諡為獻武皇帝,廟號世宗。

\subsubsection{燕平}

\begin{longtable}{|>{\centering\scriptsize}m{2em}|>{\centering\scriptsize}m{1.3em}|>{\centering}m{8.8em}|}
  % \caption{秦王政}\
  \toprule
  \SimHei \normalsize 年数 & \SimHei \scriptsize 公元 & \SimHei 大事件 \tabularnewline
  % \midrule
  \endfirsthead
  \toprule
  \SimHei \normalsize 年数 & \SimHei \scriptsize 公元 & \SimHei 大事件 \tabularnewline
  \midrule
  \endhead
  \midrule
  元年 & 398 & \tabularnewline\hline
  二年 & 399 & \tabularnewline
  \bottomrule
\end{longtable}

\subsubsection{建平}

\begin{longtable}{|>{\centering\scriptsize}m{2em}|>{\centering\scriptsize}m{1.3em}|>{\centering}m{8.8em}|}
  % \caption{秦王政}\
  \toprule
  \SimHei \normalsize 年数 & \SimHei \scriptsize 公元 & \SimHei 大事件 \tabularnewline
  % \midrule
  \endfirsthead
  \toprule
  \SimHei \normalsize 年数 & \SimHei \scriptsize 公元 & \SimHei 大事件 \tabularnewline
  \midrule
  \endhead
  \midrule
  元年 & 400 & \tabularnewline\hline
  二年 & 401 & \tabularnewline\hline
  三年 & 402 & \tabularnewline\hline
  四年 & 403 & \tabularnewline\hline
  五年 & 404 & \tabularnewline\hline
  六年 & 405 & \tabularnewline
  \bottomrule
\end{longtable}


%%% Local Variables:
%%% mode: latex
%%% TeX-engine: xetex
%%% TeX-master: "../../Main"
%%% End:

%% -*- coding: utf-8 -*-
%% Time-stamp: <Chen Wang: 2019-12-19 16:25:10>

\subsection{慕容超\tiny(405-410)}

\subsubsection{生平}

燕末主慕容超(385年-410年),字祖明,十六國南燕末代皇帝,鮮卑人。

前秦建元六年(370年),前燕為前秦所滅後,慕容超之父慕容納一度仕於前秦,後來遷居於張掖(今中國甘肅省張掖市)。慕容納之弟慕容德受前秦帝苻堅之命隨軍南征東晉,留下金刀拜別母親公孫氏而去。

前秦建元十九年(383年),前秦於肥水之戰敗北,慕容納、德之兄慕容垂趁機起兵建後燕,前秦遂殺慕容納本人及慕容德諸子。公孫氏因年老而免死,慕容納之妻段氏正好懷孕,暫不執行死刑,羈押於獄中。有個叫呼延平的獄卒,是慕容德以前的下屬,慕容德曾免其死罪,對其有恩。因此,他幫助公孫氏及段氏逃至羌地,慕容超即於該處誕生。

慕容超十歲時,祖母公孫氏去世,臨終前把金刀給慕容超,說:「如果天下太平,你能夠向東回到故土,可以將這把刀還給你叔叔(慕容德)。」呼延平後來又讓慕容超母子逃亡到呂光在位時的後涼。後來的後涼王呂隆向後秦姚興投降,慕容超母子又被遷往長安(今中國陝西省西安市)。呼延平去世後,慕容超之母段氏讓慕容超娶呼延平之女。

慕容超認為幾位伯叔父先後在東方稱帝,恐怕被後秦知道身分,所以就裝成神智失常之人,並以行乞維生。後秦人都看不起他,遂對他不起疑,所以行動自由不受限制。當時已登上南燕帝位的慕容德聽說這件事,立即派使者迎接他,慕容超不告別母親、妻子即東行。後來到達南燕,呈獻金刀給慕容德,並告以其祖母也就是慕容德之母臨終的遺言,慕容德聽了之後哀傷不已。將慕容超封為北海王(即慕容纳在前燕的王爵),任命為侍中、驃騎大將軍、司隸校尉,開王府置僚佐。

史載「慕容超身高八尺,腰帶九圍,姿器魁傑」,和慕容德頗為相似,而且「精彩秀髮,容止可觀」,《晉書》和《十六國春秋》皆載此時他才被取名為慕容超。慕容德由於年輕時生的兒子已經在前秦被殺害,晚年只有女兒沒有兒子,所以動了讓慕容超繼承之心。而慕容超亦深知慕容德的意思,因此「入則盡歡承奉,出則傾身下士」,於是輿論一致稱讚,不久即被立為太子。

南燕建平六年九月戊午(405年11月17日),慕容德去世,九月己未(11月18日),慕容超即皇帝位,改元太上。慕容超登位後,寵信舊部公孫五樓,聽信其言,大殺功臣,時稱“欲得侯,事五樓”。又喜好遊獵,使得人民苦不堪言。他的嬸母、皇太后段季妃等密謀廢掉他立慕容鐘,事發,慕容超殺了相關諸臣,廢黜了段季妃。

太上三年(407年),因母段氏、妻呼延氏尚留在後秦,遂向後秦稱藩,後秦就將其母、妻送還。慕容超追尊其父慕容納為穆皇帝,立其母為皇太后,妻為皇后。

南燕向後秦稱藩後,慕容超即計畫南下攻擊淮北,使得東晉不堪其擾。太上五年(409年),東晉將領劉裕率軍進攻南燕反擊。次年二月丁亥日(410年3月25日),南燕都城廣固(今中國山東省青州市)陷落,慕容超被俘,被送往東晉都城建康(今中國江蘇省南京市)斬首。死後無諡號及廟號,有史家稱他為南燕末主。

慕容超同時也是除了系出同源的吐谷渾外,五胡十六國時期源自鲜卑慕容部的最後一位帝王。

\subsubsection{太上}

\begin{longtable}{|>{\centering\scriptsize}m{2em}|>{\centering\scriptsize}m{1.3em}|>{\centering}m{8.8em}|}
  % \caption{秦王政}\
  \toprule
  \SimHei \normalsize 年数 & \SimHei \scriptsize 公元 & \SimHei 大事件 \tabularnewline
  % \midrule
  \endfirsthead
  \toprule
  \SimHei \normalsize 年数 & \SimHei \scriptsize 公元 & \SimHei 大事件 \tabularnewline
  \midrule
  \endhead
  \midrule
  元年 & 405 & \tabularnewline\hline
  二年 & 406 & \tabularnewline\hline
  三年 & 407 & \tabularnewline\hline
  四年 & 408 & \tabularnewline\hline
  五年 & 409 & \tabularnewline\hline
  六年 & 410 & \tabularnewline
  \bottomrule
\end{longtable}


%%% Local Variables:
%%% mode: latex
%%% TeX-engine: xetex
%%% TeX-master: "../../Main"
%%% End:



%%% Local Variables:
%%% mode: latex
%%% TeX-engine: xetex
%%% TeX-master: "../../Main"
%%% End:
