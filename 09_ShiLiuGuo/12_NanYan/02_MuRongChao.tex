%% -*- coding: utf-8 -*-
%% Time-stamp: <Chen Wang: 2021-11-01 14:54:03>

\subsection{末主慕容超\tiny(405-410)}

\subsubsection{生平}

燕末主慕容超(385年-410年),字祖明,十六國南燕末代皇帝,鮮卑人。

前秦建元六年(370年),前燕為前秦所滅後,慕容超之父慕容納一度仕於前秦,後來遷居於張掖(今中國甘肅省張掖市)。慕容納之弟慕容德受前秦帝苻堅之命隨軍南征東晉,留下金刀拜別母親公孫氏而去。

前秦建元十九年(383年),前秦於肥水之戰敗北,慕容納、德之兄慕容垂趁機起兵建後燕,前秦遂殺慕容納本人及慕容德諸子。公孫氏因年老而免死,慕容納之妻段氏正好懷孕,暫不執行死刑,羈押於獄中。有個叫呼延平的獄卒,是慕容德以前的下屬,慕容德曾免其死罪,對其有恩。因此,他幫助公孫氏及段氏逃至羌地,慕容超即於該處誕生。

慕容超十歲時,祖母公孫氏去世,臨終前把金刀給慕容超,說:「如果天下太平,你能夠向東回到故土,可以將這把刀還給你叔叔(慕容德)。」呼延平後來又讓慕容超母子逃亡到呂光在位時的後涼。後來的後涼王呂隆向後秦姚興投降,慕容超母子又被遷往長安(今中國陝西省西安市)。呼延平去世後,慕容超之母段氏讓慕容超娶呼延平之女。

慕容超認為幾位伯叔父先後在東方稱帝,恐怕被後秦知道身分,所以就裝成神智失常之人,並以行乞維生。後秦人都看不起他,遂對他不起疑,所以行動自由不受限制。當時已登上南燕帝位的慕容德聽說這件事,立即派使者迎接他,慕容超不告別母親、妻子即東行。後來到達南燕,呈獻金刀給慕容德,並告以其祖母也就是慕容德之母臨終的遺言,慕容德聽了之後哀傷不已。將慕容超封為北海王(即慕容纳在前燕的王爵),任命為侍中、驃騎大將軍、司隸校尉,開王府置僚佐。

史載「慕容超身高八尺,腰帶九圍,姿器魁傑」,和慕容德頗為相似,而且「精彩秀髮,容止可觀」,《晉書》和《十六國春秋》皆載此時他才被取名為慕容超。慕容德由於年輕時生的兒子已經在前秦被殺害,晚年只有女兒沒有兒子,所以動了讓慕容超繼承之心。而慕容超亦深知慕容德的意思,因此「入則盡歡承奉,出則傾身下士」,於是輿論一致稱讚,不久即被立為太子。

南燕建平六年九月戊午(405年11月17日),慕容德去世,九月己未(11月18日),慕容超即皇帝位,改元太上。慕容超登位後,寵信舊部公孫五樓,聽信其言,大殺功臣,時稱“欲得侯,事五樓”。又喜好遊獵,使得人民苦不堪言。他的嬸母、皇太后段季妃等密謀廢掉他立慕容鐘,事發,慕容超殺了相關諸臣,廢黜了段季妃。

太上三年(407年),因母段氏、妻呼延氏尚留在後秦,遂向後秦稱藩,後秦就將其母、妻送還。慕容超追尊其父慕容納為穆皇帝,立其母為皇太后,妻為皇后。

南燕向後秦稱藩後,慕容超即計畫南下攻擊淮北,使得東晉不堪其擾。太上五年(409年),東晉將領劉裕率軍進攻南燕反擊。次年二月丁亥日(410年3月25日),南燕都城廣固(今中國山東省青州市)陷落,慕容超被俘,被送往東晉都城建康(今中國江蘇省南京市)斬首。死後無諡號及廟號,有史家稱他為南燕末主。

慕容超同時也是除了系出同源的吐谷渾外,五胡十六國時期源自鲜卑慕容部的最後一位帝王。

\subsubsection{太上}

\begin{longtable}{|>{\centering\scriptsize}m{2em}|>{\centering\scriptsize}m{1.3em}|>{\centering}m{8.8em}|}
  % \caption{秦王政}\
  \toprule
  \SimHei \normalsize 年数 & \SimHei \scriptsize 公元 & \SimHei 大事件 \tabularnewline
  % \midrule
  \endfirsthead
  \toprule
  \SimHei \normalsize 年数 & \SimHei \scriptsize 公元 & \SimHei 大事件 \tabularnewline
  \midrule
  \endhead
  \midrule
  元年 & 405 & \tabularnewline\hline
  二年 & 406 & \tabularnewline\hline
  三年 & 407 & \tabularnewline\hline
  四年 & 408 & \tabularnewline\hline
  五年 & 409 & \tabularnewline\hline
  六年 & 410 & \tabularnewline
  \bottomrule
\end{longtable}


%%% Local Variables:
%%% mode: latex
%%% TeX-engine: xetex
%%% TeX-master: "../../Main"
%%% End:
