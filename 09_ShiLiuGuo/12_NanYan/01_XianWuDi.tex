%% -*- coding: utf-8 -*-
%% Time-stamp: <Chen Wang: 2019-12-19 16:24:33>

\subsection{献武帝\tiny(398-405)}

\subsubsection{生平}

燕献武帝慕容德(336年-405年11月17日),後改名慕容備德,字玄明,十六國時期南燕皇帝,鮮卑人,是前燕文明帝慕容皝之幼子,前燕景昭帝慕容儁、後燕成武帝慕容垂皆為其兄。《晉書》載其「年未弱冠,身長八尺二寸,姿貌雄偉」。又「博觀群書,性清慎,多才藝」。

前燕時期慕容儁在位時,慕容德被封為梁公。後來慕容儁之子慕容暐繼帝位,再被改封為范陽王。369年,曾與慕容垂一同大敗東晉桓溫的進攻。370年,前燕為前秦所滅後,一度被前秦帝苻堅任命為張掖(今中國甘肅省張掖市)太守,數年後被免職。

後來苻堅欲南征東晉,慕容德被任命為奮威將軍隨軍,留下金刀拜別母親公孫氏及胞兄原北海王慕容納而去。383年,前秦於淝水之战敗北,苻坚宠妃张夫人走失投靠慕容暐,慕容暐送她回京,慕容德阻止并劝他趁乱复国,未果。

后来慕容垂趁機起兵建後燕,慕容德嚮應之,被命為車騎大將軍,重新受封為范陽王,然其諸子及慕容纳皆因留在後方而被前秦所殺。

396年,慕容垂臨終,遺命太子慕容寶將鄴城(今中國河南省臨漳縣)委由慕容德鎮守。慕容寶繼位後,即以慕容德為使持節、都督冀、兗、青、徐、荊、豫六州諸軍事、特進、車騎大將軍、冀州牧,領南蠻校尉,鎮守鄴城。

397年,北魏攻擊後燕,後燕兵敗如山倒,皇帝慕容寶向北方故地逃亡,後燕國土被截為南北二部,位在南方的慕容德被慕容寶任命為丞相,領冀州牧。不久,慕容垂另一子趙王慕容麟來逃至鄴城,以鄴城難守,建議慕容德南遷滑台(今中國河南省滑縣)。398年正月,慕容德又受慕容麟建議先稱燕王,稱燕王元年,史稱此一政權為南燕。次年(399年),再遷廣固(今中國山東省青州市),以為都城。

400年,慕容德正式稱帝,改元建平,並在此時把自己名字改為慕容備德,以便臣民避諱。慕容德正式登基時年已65歲,在中國歷史上僅次於唐朝的武則天,武則天登基時已經67歲了。

慕容备德不知道母亲公孙夫人和胞兄慕容纳都已经不在人世,曾于建平二年(401年)十月派平原人杜弘去长安寻访。杜弘说:“臣至长安,若不能得知太后动止,当西往张掖,以死效力。”并为自己年逾六十的父亲杜雄乞求本县县令之职。慕容备德不顾中书令张华反对,认为杜弘“为君迎母,为父求禄,忠孝备矣,何罪之有!”以杜雄为平原令。杜弘到张掖为盗贼所杀。四年(403年),慕容备德旧部赵融从长安前来,告知公孙夫人和慕容纳的死讯,慕容备德放声痛哭以至于吐血,因而卧病不起,从此健康恶化。

慕容备德有女兒無兒子,為繼承人心焦,慕容納之子慕容超持當年慕容德拜別母親的金刀來歸,慕容备德遂以慕容超袭封北海王,后立為太子。

建平六年九月戊午(405年11月17日),慕容备德去世,慕容超繼位。去世當晚從四方城門抬出十餘口棺木,秘密埋葬在山谷之中,因此實際上他並未葬於其陵寢「東陽陵」,後人遂不知其安葬之處。慕容德後來被諡為獻武皇帝,廟號世宗。

\subsubsection{燕平}

\begin{longtable}{|>{\centering\scriptsize}m{2em}|>{\centering\scriptsize}m{1.3em}|>{\centering}m{8.8em}|}
  % \caption{秦王政}\
  \toprule
  \SimHei \normalsize 年数 & \SimHei \scriptsize 公元 & \SimHei 大事件 \tabularnewline
  % \midrule
  \endfirsthead
  \toprule
  \SimHei \normalsize 年数 & \SimHei \scriptsize 公元 & \SimHei 大事件 \tabularnewline
  \midrule
  \endhead
  \midrule
  元年 & 398 & \tabularnewline\hline
  二年 & 399 & \tabularnewline
  \bottomrule
\end{longtable}

\subsubsection{建平}

\begin{longtable}{|>{\centering\scriptsize}m{2em}|>{\centering\scriptsize}m{1.3em}|>{\centering}m{8.8em}|}
  % \caption{秦王政}\
  \toprule
  \SimHei \normalsize 年数 & \SimHei \scriptsize 公元 & \SimHei 大事件 \tabularnewline
  % \midrule
  \endfirsthead
  \toprule
  \SimHei \normalsize 年数 & \SimHei \scriptsize 公元 & \SimHei 大事件 \tabularnewline
  \midrule
  \endhead
  \midrule
  元年 & 400 & \tabularnewline\hline
  二年 & 401 & \tabularnewline\hline
  三年 & 402 & \tabularnewline\hline
  四年 & 403 & \tabularnewline\hline
  五年 & 404 & \tabularnewline\hline
  六年 & 405 & \tabularnewline
  \bottomrule
\end{longtable}


%%% Local Variables:
%%% mode: latex
%%% TeX-engine: xetex
%%% TeX-master: "../../Main"
%%% End:
