%% -*- coding: utf-8 -*-
%% Time-stamp: <Chen Wang: 2021-11-01 14:59:16>

\subsection{赫连昌\tiny(425-428)}

\subsubsection{生平}

赫連昌(?-434年),一名折,字還國,十六國時期夏國皇帝,匈奴鐵弗部人,赫連勃勃三子,赫連勃勃在位時被封太原公。

夏真興六年(424年),赫連勃勃欲廢太子赫連璝,改立酒泉公赫連倫,赫連璝發兵攻殺赫連倫,後赫連昌再襲殺赫連璝平亂,赫連勃勃遂以赫連昌為太子。真興七年(425年)赫連勃勃去世,赫連昌繼位,改元承光。

承光二年(426年),北魏大舉攻夏,克長安。次年(427年),占領夏國都城統萬(今內蒙古烏審旗南白城子),赫連昌逃往上邽(今甘肅天水)。承光四年(428年),北魏攻上邽,會戰中赫連昌因馬失前蹄墜地而被生擒。

赫連昌被俘後,北魏太武帝拓跋燾十分禮遇他,不僅使其住在西宮,更把皇妹嫁給他,並封會稽公。拓跋燾亦常命赫連昌隨待在側,打獵時二人有時亦單獨並騎,赫連昌素有勇名,因此拓跋燾可說對赫連昌十分信任。北魏神䴥三年(430年)三月又被封為秦王。

北魏延和三年(434年)閏三月,赫連昌叛魏西逃,途中被抓獲格殺。

\subsubsection{承光}

\begin{longtable}{|>{\centering\scriptsize}m{2em}|>{\centering\scriptsize}m{1.3em}|>{\centering}m{8.8em}|}
  % \caption{秦王政}\
  \toprule
  \SimHei \normalsize 年数 & \SimHei \scriptsize 公元 & \SimHei 大事件 \tabularnewline
  % \midrule
  \endfirsthead
  \toprule
  \SimHei \normalsize 年数 & \SimHei \scriptsize 公元 & \SimHei 大事件 \tabularnewline
  \midrule
  \endhead
  \midrule
  元年 & 425 & \tabularnewline\hline
  二年 & 426 & \tabularnewline\hline
  三年 & 427 & \tabularnewline\hline
  四年 & 428 & \tabularnewline
  \bottomrule
\end{longtable}


%%% Local Variables:
%%% mode: latex
%%% TeX-engine: xetex
%%% TeX-master: "../../Main"
%%% End:
