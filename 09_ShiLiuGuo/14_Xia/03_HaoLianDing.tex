%% -*- coding: utf-8 -*-
%% Time-stamp: <Chen Wang: 2019-12-19 16:32:20>

\subsection{郝连定\tiny(428-437)}

\subsubsection{生平}

赫連定(?-432年),小字直獖,十六國時期夏國君主,匈奴鐵弗部人,赫連勃勃五子,赫連昌之弟,赫連勃勃在位時被封平原公,鎮守長安。

夏國皇帝赫連昌承光二年(426年),北魏大舉攻夏,赫連定與北魏軍對峙於長安一帶。次年(427年),夏國都城統萬(今內蒙古烏審旗南白城子)陷落,赫連定逃往上邽(今甘肅天水)與赫連昌會合,被進封平原王。承光四年(428年),北魏攻上邽,赫連昌被擒,赫連定逃奔平涼(今甘肅平涼),即皇帝位,改年號勝光。

赫連定繼位時夏國已侷促一隅,情勢窘迫,不復當年,因此欲與正在北伐的南朝宋結盟,北魏得到消息後決定一舉滅夏國。勝光四年(431年),一路敗退的赫連定無路可退,遂向西攻滅為北涼所逼情勢更加窘迫的西秦。數月後欲再攻北涼,於半渡黃河時,被吐谷渾首領慕容慕璝派軍襲擊,赫連定被俘。次年(432年),赫連定被吐谷渾送往北魏,北魏將其處死。

\subsubsection{胜光}

\begin{longtable}{|>{\centering\scriptsize}m{2em}|>{\centering\scriptsize}m{1.3em}|>{\centering}m{8.8em}|}
  % \caption{秦王政}\
  \toprule
  \SimHei \normalsize 年数 & \SimHei \scriptsize 公元 & \SimHei 大事件 \tabularnewline
  % \midrule
  \endfirsthead
  \toprule
  \SimHei \normalsize 年数 & \SimHei \scriptsize 公元 & \SimHei 大事件 \tabularnewline
  \midrule
  \endhead
  \midrule
  元年 & 428 & \tabularnewline\hline
  二年 & 429 & \tabularnewline\hline
  三年 & 430 & \tabularnewline\hline
  四年 & 431 & \tabularnewline
  \bottomrule
\end{longtable}


%%% Local Variables:
%%% mode: latex
%%% TeX-engine: xetex
%%% TeX-master: "../../Main"
%%% End:
