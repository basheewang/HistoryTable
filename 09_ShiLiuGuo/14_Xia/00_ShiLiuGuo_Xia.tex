%% -*- coding: utf-8 -*-
%% Time-stamp: <Chen Wang: 2019-12-19 16:30:29>


\section{夏\tiny(407-431)}

\subsection{简介}

夏(407年-431年)又称为大夏或北夏,因为建立者赫连勃勃是匈奴铁弗部人,故又称胡夏或赫连夏,是407年到431年存在于关中与河套地区的一个国家,国都在大部分时间都位于统万城(今陕西省靖边县红墩界乡白城子村无定河北岸)。夏是匈奴铁弗部首领赫连勃勃在407年自称大单于后所建。418年,赫连勃勃在攻陷长安后称帝。427年,北魏太武帝出兵攻陷统万城并于428年俘虏夏国第二代君主赫连昌。赫连昌之弟赫连定随后被拥立为君主并在430年灭亡西秦。但431年赫连定被吐谷浑俘虏,之后在432年被送往北魏处死。夏国共存在25年,历经三代君主。

夏国是五胡十六国中最晚建立的政权,其都城统万城遗址是至今唯一保存基本完好的早期北方王国都城遗址,也是匈奴人历史上留下的唯一都城遗址。

夏国最初占有大城(今内蒙古杭锦旗东南),之后于413年又修建统万城作为首都。至417年后秦灭亡前,夏国占据河套至陇东与陕西北部,

夏国的地方行政主要分为州、城(县)两级。有史可考的有幽、朔、秦、北秦、雍、并、梁、豫、荆九州。文献记载最多的只有幽州。

%% -*- coding: utf-8 -*-
%% Time-stamp: <Chen Wang: 2019-12-19 16:31:30>

\subsection{武烈帝\tiny(407-425)}

\subsubsection{生平}

夏武烈帝赫连勃勃(381年-425年),字屈孑,匈奴铁弗部人,原名劉勃勃,中國十六国时期夏國建立者。勃勃是南匈奴單于的後裔,其父劉衞辰死於北魏進攻後,勃勃依靠後秦高平公沒弈干,又得後秦君主姚興賞識。及後就以後秦與魏通好而叛秦,殺害沒奕干並自立,建北夏國,屢度進攻後秦。隨後更乘東晉滅後秦後班師的機會佔領關中。

赫連勃勃曾祖父劉虎領導鐵弗部,並曾與代國發生戰鬥,為代國所敗。祖父劉務桓重整部眾,重新壯大鐵弗部,並受後趙封為平北將軍、左賢王。父刘卫辰繼位後搖擺於前秦及代國之間,前秦皇帝苻坚滅代國後更任命劉衞辰为西单于,屯駐代來城(今內蒙古伊克昭盟東勝區西),督摄河西诸部族。前秦瓦解後,劉衞辰一度據有朔方一帶,但在391年受到北魏的攻擊,代來城被攻陷,劉衛辰被殺。年幼的劉勃勃逃奔薛干部,薛干酋長把劉勃勃送給後秦高平公沒弈干,沒弈干就把女兒嫁給劉勃勃。勃勃受到後秦姚興的寵遇,任為安北將軍、五原公,鎮朔方。此後一直从属后秦。

後秦弘始九年(407年),赫連勃勃因怨恨後秦與北魏通訊,決意背叛後秦。於是先扣押起柔然可汗送給後秦的八千匹馬,然後假裝在高平川(今寧夏南清水河)狩獵,襲殺沒弈干,併吞其部眾。勃勃自以是匈奴夏后氏後裔,建國號「大夏」,自立为天王,大单于,国号夏,改年號龙升。

勃勃自立後不久就出兵薛干部等三部,收降數萬人後轉攻後秦三城(今陝西綏德縣)以北諸戍。當時諸將都反對出兵後秦,建議勃勃先固守高平,穩固根本,然後才圖長安。但勃勃認為夏國初建,實力仍弱,關中仍因後秦強大而未能攻取,若果自守一城,必會引來後秦各鎮的聯手攻伐,終兵敗亡國,故此特意不長居一處,以游擊戰術,出其不意,讓對方疲於奔命,以取嶺兵、河東之地,再待後秦君主姚興死後才攻取長安。最終勃勃的進襲令到嶺兵各城日間也要緊閉城門。勃勃稱天王後曾向南涼王禿髮傉檀請婚但遭拒,於是怒而率兵進攻南涼,殺傷一萬多人並掠奪二萬七千人及數十萬頭牲畜。後更在陽武(今甘肅靖遠縣境)大敗來攻的禿髮傉檀,殺傷甚眾,很多南涼的名將都戰死。

大破南涼後,勃勃與後秦屢有戰事。勃勃先於青石原(今甘肅涇川縣境)擊敗秦將張佛生,次年(408年)後秦將齊難來攻,勃勃先退守河曲,待齊難縱兵掠奪時就進軍,並追擊至木城(今陝西榆林市榆陽區),生擒齊難,俘獲大量兵眾及戰馬。戰後,歸降夏國的嶺北胡漢數以萬計,勃勃更置地方官員安撫他們。龍升三年(409年),勃勃又率兵攻秦,掠奪平涼雜胡共七千多戶配給後援軍隊,進據依力川。同年姚興親征勃勃,但勃勃乘秦軍未集,率軍進攻姚興所駐的貮城,秦軍兵敗,姚興只得退回長安。接著勃勃又攻下了敕奇堡、黃石固及我羅城。龍升四年(410年),勃勃派兵攻平涼,為姚興所敗,但進攻定陽(今陝西宜川縣西北)的一軍卻取勝,接著勃勃親自率軍進攻隴右,攻破白崖堡,並進逼清水城,令後秦略陽太守姚壽都棄城逃走。龍升五年(411年),勃勃攻安定,在青石以北平原擊破楊佛嵩,又攻下東鄉。

鳳翔元年(413年),勃勃下令修築夏國都城统万城(今陕西靖边北白城子),並任用了殘忍的叱干阿利為將作大匠。工人以蒸土築城,而巡工发现墙面能用铁锥子刺入一寸,便把修築那處的人处死,尸体也被筑入墙内,因此,统万城的城墙坚硬如铁。其時又用銅鑄造了大鼓、飛廉、翁仲、銅駝、龍獸等裝飾物,並用黃金裝飾,排在宮殿前,但製作這些東西又殺了數千個工匠。同時,又以其祖輩跟隨母系姓劉不合禮,於是改姓赫連,表示「徽赫實與天連」;又將非皇族的其他鐵弗部眾改姓「鐵伐」,以示「剛銳如鐵,皆堪伐人。」

此後,勃勃仍然繼續侵襲後秦,先於鳳翔三年(415年)攻下杏城(今陝西黃陵縣西南);次年又乘後秦與仇池楊盛爭戰的時機先後攻下上邽(仱甘肅天水市)及陰密(今陝西靈臺縣西),更令駐守安定的姚恢棄城出走,當地人胡儼等於是獻城夏國。勃勃隨後又進攻雍城(今陝西鳳翔縣南)及郿城(今陝西郿縣),但在郿城遭姚紹抵抗而未能攻下,於是退回安定。胡儼等人此時卻殺勃勃所命留守安定的羊苟兒,轉而降秦。勃勃唯有退返杏城。不過,其時勃勃知東晉劉裕要進攻後秦,他估計劉裕必定能攻滅後秦,但肯定很快班師,留下子弟及將領守關中。勃勃認定這就是他奪取關中的好時機,並且十分輕易,故此不必耗費兵力與後秦作戰。故此秣馬厲兵,休養士卒。不久勃勃再引兵佔據安定,後秦在嶺北的各戍及郡縣都向夏國投降,嶺北全境盡入夏國。另一方面,勃勃又先後與北燕及北涼結盟。

凤翔五年(417年),東晉大将刘裕滅後秦,同年年末班師,留兒子劉義真及王鎮惡、沈田子、傅弘之等諸將守關中。勃勃聞訊十分高興,就派兒子赫連璝督前鋒攻長安、赫連昌出兵堵塞潼關,又派王買德阻斷青泥,然後自率大軍在後。次年,赫連璝行軍至渭陽時已經有很多人在路邊請降,其時晉軍沈田子作戰失利,更因與王鎮惡不和而殺了他,沈田子隨後亦被劉義真長史王脩處死。劉義真於是召集外軍入城並閉門拒守,關中各郡縣於是都降夏。勃勃隨後進據咸陽,令長安城無法獲得物資補給。劉裕見此唯有派朱齡石接替劉義真,並命劉義真東歸。當時劉義真部眾大肆掠奪物資才離開,令關中人民驅逐朱齡石,迎勃勃入主長安。勃勃入長安後大宴將士,不久就在灞上(今陝西蓝田县)称帝,改元昌武。及後群臣都勸勃勃遷都長安,但勃勃慮及全國中心南遷長安後,北魏會易於攻擊距邊界才百里的統萬,認為定都統萬才能阻遏北魏侵襲北境。於是在次年(419年)於長安置南臺,留太子赫連璝留守。不久回師,因統萬宮殿完工而刻石於城南,歌功頌德。

真兴六年(424年),勃勃想要廢黜太子赫連璝,改立幼子赫連倫。赫連璝知道後率兵七萬自長安攻伐赫連倫,終在高平一戰中擊敗並殺死對方。赫連倫兄赫連昌則率軍襲擊赫連璝,將其殺死,勃勃於是立赫連昌為太子。勃勃於真興七年(425年)死于帝位,諡号武烈皇帝,廟號世祖。

勃勃身材魁武,高八尺五寸,且聰慧有儀態,有辯才的機悟。

勃勃頗有權謀智術,當劉裕滅後秦、占關中之時,勃勃一度震攝於劉裕的兵鋒威勢,答應劉裕「約為兄弟」的和平要求,但勃勃為了在氣勢上勝過劉裕,在接見劉裕的使者之前,先讓文才優異的部下皇甫徽寫好給劉裕的回書,再將文字背的爛熟,然後才接見使者,命令部下將自己當場背出的回書寫下並交付給使者。結果使者就以為勃勃真的即席創作出文采斐然的回書,將此事回報給劉裕,果然讓老粗一名的劉裕敬佩勃勃的文思敏捷,邊讀回書邊感嘆說:「吾不如也!」

勃勃凶暴好殺,在陽武大敗南涼軍及關中大敗東晉軍時曾將屍體或人頭堆積起來,建起「髑髏臺」,當作景觀觀賞。也常常在城上,身旁準備好弓箭刀劍,一旦對人有所不滿就會動手殺人。而群臣若敢直接與其對視就會被弄瞎,敢笑就割下其嘴唇,敢進諫就先割下其舌頭再斬殺。這令當時人們十分不安。

勃勃十分自大,建的統萬城四個城門,東門叫招魏門,南門叫朝宋門,西門叫服涼門,北門叫平朔門。

後秦君臣姚興、姚邕兩方的意見:「姚邕說:『勃勃不可近也。』姚興說:『勃勃有濟世之才,吾方與之平天下,柰何逆忌之?』姚邕說:『勃勃奉上慢,御眾殘,貪猾不仁,輕為去就;寵之踰分,恐終為邊患。』」後來勃勃反叛後秦並成為大患,姚興因此感嘆說:「吾不用黃兒之言,以至於此!」(按:姚邕小字黃兒)

南涼大臣焦朗評論:「勃勃天姿雄健,御軍嚴整,未可輕也。」

\subsubsection{龙昇}

\begin{longtable}{|>{\centering\scriptsize}m{2em}|>{\centering\scriptsize}m{1.3em}|>{\centering}m{8.8em}|}
  % \caption{秦王政}\
  \toprule
  \SimHei \normalsize 年数 & \SimHei \scriptsize 公元 & \SimHei 大事件 \tabularnewline
  % \midrule
  \endfirsthead
  \toprule
  \SimHei \normalsize 年数 & \SimHei \scriptsize 公元 & \SimHei 大事件 \tabularnewline
  \midrule
  \endhead
  \midrule
  元年 & 407 & \tabularnewline\hline
  二年 & 408 & \tabularnewline\hline
  三年 & 409 & \tabularnewline\hline
  四年 & 410 & \tabularnewline\hline
  五年 & 411 & \tabularnewline\hline
  六年 & 412 & \tabularnewline\hline
  七年 & 413 & \tabularnewline
  \bottomrule
\end{longtable}

\subsubsection{凤翔}

\begin{longtable}{|>{\centering\scriptsize}m{2em}|>{\centering\scriptsize}m{1.3em}|>{\centering}m{8.8em}|}
  % \caption{秦王政}\
  \toprule
  \SimHei \normalsize 年数 & \SimHei \scriptsize 公元 & \SimHei 大事件 \tabularnewline
  % \midrule
  \endfirsthead
  \toprule
  \SimHei \normalsize 年数 & \SimHei \scriptsize 公元 & \SimHei 大事件 \tabularnewline
  \midrule
  \endhead
  \midrule
  元年 & 413 & \tabularnewline\hline
  二年 & 414 & \tabularnewline\hline
  三年 & 415 & \tabularnewline\hline
  四年 & 416 & \tabularnewline\hline
  五年 & 417 & \tabularnewline\hline
  六年 & 418 & \tabularnewline
  \bottomrule
\end{longtable}

\subsubsection{昌武}

\begin{longtable}{|>{\centering\scriptsize}m{2em}|>{\centering\scriptsize}m{1.3em}|>{\centering}m{8.8em}|}
  % \caption{秦王政}\
  \toprule
  \SimHei \normalsize 年数 & \SimHei \scriptsize 公元 & \SimHei 大事件 \tabularnewline
  % \midrule
  \endfirsthead
  \toprule
  \SimHei \normalsize 年数 & \SimHei \scriptsize 公元 & \SimHei 大事件 \tabularnewline
  \midrule
  \endhead
  \midrule
  元年 & 418 & \tabularnewline\hline
  二年 & 419 & \tabularnewline
  \bottomrule
\end{longtable}

\subsubsection{真兴}

\begin{longtable}{|>{\centering\scriptsize}m{2em}|>{\centering\scriptsize}m{1.3em}|>{\centering}m{8.8em}|}
  % \caption{秦王政}\
  \toprule
  \SimHei \normalsize 年数 & \SimHei \scriptsize 公元 & \SimHei 大事件 \tabularnewline
  % \midrule
  \endfirsthead
  \toprule
  \SimHei \normalsize 年数 & \SimHei \scriptsize 公元 & \SimHei 大事件 \tabularnewline
  \midrule
  \endhead
  \midrule
  元年 & 419 & \tabularnewline\hline
  二年 & 420 & \tabularnewline\hline
  三年 & 421 & \tabularnewline\hline
  四年 & 422 & \tabularnewline\hline
  五年 & 423 & \tabularnewline\hline
  六年 & 424 & \tabularnewline\hline
  七年 & 425 & \tabularnewline
  \bottomrule
\end{longtable}


%%% Local Variables:
%%% mode: latex
%%% TeX-engine: xetex
%%% TeX-master: "../../Main"
%%% End:

%% -*- coding: utf-8 -*-
%% Time-stamp: <Chen Wang: 2021-11-01 14:59:16>

\subsection{赫连昌\tiny(425-428)}

\subsubsection{生平}

赫連昌(?-434年),一名折,字還國,十六國時期夏國皇帝,匈奴鐵弗部人,赫連勃勃三子,赫連勃勃在位時被封太原公。

夏真興六年(424年),赫連勃勃欲廢太子赫連璝,改立酒泉公赫連倫,赫連璝發兵攻殺赫連倫,後赫連昌再襲殺赫連璝平亂,赫連勃勃遂以赫連昌為太子。真興七年(425年)赫連勃勃去世,赫連昌繼位,改元承光。

承光二年(426年),北魏大舉攻夏,克長安。次年(427年),占領夏國都城統萬(今內蒙古烏審旗南白城子),赫連昌逃往上邽(今甘肅天水)。承光四年(428年),北魏攻上邽,會戰中赫連昌因馬失前蹄墜地而被生擒。

赫連昌被俘後,北魏太武帝拓跋燾十分禮遇他,不僅使其住在西宮,更把皇妹嫁給他,並封會稽公。拓跋燾亦常命赫連昌隨待在側,打獵時二人有時亦單獨並騎,赫連昌素有勇名,因此拓跋燾可說對赫連昌十分信任。北魏神䴥三年(430年)三月又被封為秦王。

北魏延和三年(434年)閏三月,赫連昌叛魏西逃,途中被抓獲格殺。

\subsubsection{承光}

\begin{longtable}{|>{\centering\scriptsize}m{2em}|>{\centering\scriptsize}m{1.3em}|>{\centering}m{8.8em}|}
  % \caption{秦王政}\
  \toprule
  \SimHei \normalsize 年数 & \SimHei \scriptsize 公元 & \SimHei 大事件 \tabularnewline
  % \midrule
  \endfirsthead
  \toprule
  \SimHei \normalsize 年数 & \SimHei \scriptsize 公元 & \SimHei 大事件 \tabularnewline
  \midrule
  \endhead
  \midrule
  元年 & 425 & \tabularnewline\hline
  二年 & 426 & \tabularnewline\hline
  三年 & 427 & \tabularnewline\hline
  四年 & 428 & \tabularnewline
  \bottomrule
\end{longtable}


%%% Local Variables:
%%% mode: latex
%%% TeX-engine: xetex
%%% TeX-master: "../../Main"
%%% End:

%% -*- coding: utf-8 -*-
%% Time-stamp: <Chen Wang: 2019-12-19 16:32:20>

\subsection{郝连定\tiny(428-437)}

\subsubsection{生平}

赫連定(?-432年),小字直獖,十六國時期夏國君主,匈奴鐵弗部人,赫連勃勃五子,赫連昌之弟,赫連勃勃在位時被封平原公,鎮守長安。

夏國皇帝赫連昌承光二年(426年),北魏大舉攻夏,赫連定與北魏軍對峙於長安一帶。次年(427年),夏國都城統萬(今內蒙古烏審旗南白城子)陷落,赫連定逃往上邽(今甘肅天水)與赫連昌會合,被進封平原王。承光四年(428年),北魏攻上邽,赫連昌被擒,赫連定逃奔平涼(今甘肅平涼),即皇帝位,改年號勝光。

赫連定繼位時夏國已侷促一隅,情勢窘迫,不復當年,因此欲與正在北伐的南朝宋結盟,北魏得到消息後決定一舉滅夏國。勝光四年(431年),一路敗退的赫連定無路可退,遂向西攻滅為北涼所逼情勢更加窘迫的西秦。數月後欲再攻北涼,於半渡黃河時,被吐谷渾首領慕容慕璝派軍襲擊,赫連定被俘。次年(432年),赫連定被吐谷渾送往北魏,北魏將其處死。

\subsubsection{胜光}

\begin{longtable}{|>{\centering\scriptsize}m{2em}|>{\centering\scriptsize}m{1.3em}|>{\centering}m{8.8em}|}
  % \caption{秦王政}\
  \toprule
  \SimHei \normalsize 年数 & \SimHei \scriptsize 公元 & \SimHei 大事件 \tabularnewline
  % \midrule
  \endfirsthead
  \toprule
  \SimHei \normalsize 年数 & \SimHei \scriptsize 公元 & \SimHei 大事件 \tabularnewline
  \midrule
  \endhead
  \midrule
  元年 & 428 & \tabularnewline\hline
  二年 & 429 & \tabularnewline\hline
  三年 & 430 & \tabularnewline\hline
  四年 & 431 & \tabularnewline
  \bottomrule
\end{longtable}


%%% Local Variables:
%%% mode: latex
%%% TeX-engine: xetex
%%% TeX-master: "../../Main"
%%% End:



%%% Local Variables:
%%% mode: latex
%%% TeX-engine: xetex
%%% TeX-master: "../../Main"
%%% End:
