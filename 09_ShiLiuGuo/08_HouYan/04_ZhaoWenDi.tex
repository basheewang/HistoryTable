%% -*- coding: utf-8 -*-
%% Time-stamp: <Chen Wang: 2019-12-19 15:31:28>

\subsection{昭文帝\tiny(401-407)}

\subsubsection{生平}

燕昭文帝慕容熙(385年-407年9月14日),字道文,一字長生,十六國時期後燕國君主,鮮卑人,成武帝慕容垂的幼子,惠愍帝慕容寶之弟,母親是貴嬪段氏。原封河間王,蘭汗之亂時曾被封為遼東公,慕容盛即位後,封河間公。

後燕長樂三年(401年),慕容盛被變軍殺害。慕容盛有兒子慕容定,年紀幼小。群臣希望慕容盛之弟慕容元繼位。但慕容熙因与慕容盛之母丁太后有私情,備受她寵愛,八月癸巳日(9月14日),遂被密迎入宮即天王位,慕容元被賜死,不久慕容熙改元光始。次年(402年),慕容熙又害死了慕容定,且娶了苻秦中山尹苻謨的兩個女兒苻娀娥為貴人、苻訓英為貴嬪,苻訓英極其受寵。丁太后怨恨,遂謀廢慕容熙,事洩,丁太后被殺。

慕容熙立苻訓英為皇后,苻娀娥為貴人,非常寵愛苻氏姐妹,因此興築宮殿、遊玩打獵,導致軍民死亡的數以萬計。苻娀娥生病,有人自稱能醫,結果醫死了,慕容熙遂將醫生支解後焚燒,追封苻娀娥為愍皇后。慕容熙與苻訓英更是玩樂不知節制。元始五年(405年)攻高句麗遼東城,原本城將攻陷,慕容熙只為了要與苻后一同坐輦車進城,因而命軍暫緩登城,以致延誤戰機,不能攻下遼東。次年(406年),後燕攻契丹未果而回師,又為了苻后想要觀戰而臨時拋棄輜重轉而偷襲高句麗,致士卒馬匹,疲累寒冷,沿路死亡不可勝數。又如苻后夏天想要吃凍魚,冬天要吃生地黃,官員也因不能取得而被斬首。

建始元年(407年),苻后去世,慕容熙痛不欲生,喪禮上命檢查百官有無哭泣,規定未哭者給予處罰,群臣只好口含辣物以刺激流淚。又賜死高陽王慕容隆的王妃張氏以殉葬,右仆射韦璆等人都害怕自己去殉葬,每天都洗澡换衣等候命令。此外規定家家戶戶都要參與建造苻后陵墓的工程,更使得國家財政揮霍一空。臨葬,慕容熙竟打開棺材,与苻訓英的屍體親熱一番,才准下葬。

由於早先中衛將軍馮跋與其弟馮素弗曾因事獲罪於後燕帝慕容熙,因此慕容熙一直有殺馮跋兄弟之意。七月甲子日(407年9月14日),馮跋兄弟於是趁慕容熙送葬苻后時起事,推高雲(慕容雲)為燕王,慕容熙被生擒後斬首,和苻訓英合葬。後來被諡昭文皇帝。

\subsubsection{光始}

\begin{longtable}{|>{\centering\scriptsize}m{2em}|>{\centering\scriptsize}m{1.3em}|>{\centering}m{8.8em}|}
  % \caption{秦王政}\
  \toprule
  \SimHei \normalsize 年数 & \SimHei \scriptsize 公元 & \SimHei 大事件 \tabularnewline
  % \midrule
  \endfirsthead
  \toprule
  \SimHei \normalsize 年数 & \SimHei \scriptsize 公元 & \SimHei 大事件 \tabularnewline
  \midrule
  \endhead
  \midrule
  元年 & 401 & \tabularnewline\hline
  二年 & 402 & \tabularnewline\hline
  三年 & 403 & \tabularnewline\hline
  四年 & 404 & \tabularnewline\hline
  五年 & 405 & \tabularnewline\hline
  六年 & 406 & \tabularnewline
  \bottomrule
\end{longtable}

\subsubsection{建始}

\begin{longtable}{|>{\centering\scriptsize}m{2em}|>{\centering\scriptsize}m{1.3em}|>{\centering}m{8.8em}|}
  % \caption{秦王政}\
  \toprule
  \SimHei \normalsize 年数 & \SimHei \scriptsize 公元 & \SimHei 大事件 \tabularnewline
  % \midrule
  \endfirsthead
  \toprule
  \SimHei \normalsize 年数 & \SimHei \scriptsize 公元 & \SimHei 大事件 \tabularnewline
  \midrule
  \endhead
  \midrule
  元年 & 407 & \tabularnewline
  \bottomrule
\end{longtable}


%%% Local Variables:
%%% mode: latex
%%% TeX-engine: xetex
%%% TeX-master: "../../Main"
%%% End:
