%% -*- coding: utf-8 -*-
%% Time-stamp: <Chen Wang: 2019-12-19 15:30:59>


\section{后燕\tiny(384-407)}

\subsection{简介}

後燕(384年-407年或409年)是中国五胡十六国時慕容氏諸燕之一,由鮮卑人前燕文明帝慕容皝第五子慕容垂所建立的政權。

後燕建國之初定都中山(今河北省定州市),後遷往龍城(今遼寧省朝陽市)。全盛時統治範圍「南至琅琊,東訖遼海,西屆河汾,北暨燕代」(《讀史方輿紀要》),即今河北、山東、山西和河南、遼寧的一部分。自384年慕容垂稱燕王到407年慕容熙被殺(或到409年慕容雲被杀),立國凡24年(一说26年)。

《十六国春秋》始称后燕,以别于慕容氏諸燕,后世袭用之。

重建燕國(383年─385年):前秦在383年淝水之戰大敗後,投降前秦的前燕貴族慕容垂在苻堅同意下回到鄴城。時丁零族翟斌在洛陽新安一帶起兵反秦,鎮守鄴城苻堅庶長子苻丕撥兵2千給慕容垂,派宗室苻飛龍領兵1千為慕容垂的副手,前去對付翟斌,但慕容垂於行軍中襲殺苻飛龍,與前秦正式決裂。

384年正月,慕容垂渡過黃河移至洛陽附近,與翟斌聯兵攻洛陽。後引兵東下,在滎陽自稱大將軍、大都督、燕王,建元燕元元年。後自石門渡黃河,向鄴前進,時有眾20餘萬。因苻丕堅守鄴城,慕容垂久攻不下,因战争,河北經濟受到很大的破壞。到了385年八月,苻丕撤出鄴城,退往晉陽,整個河北,皆落入慕容垂手中。386年正月,慕容垂稱帝,定都中山(今河北省定州),改元建興,史稱後燕。

攻滅西燕(386年─394年):386年十月,西燕慕容永進至長子(今山西省長子縣西),稱帝,改元中興,佔有今山西省一帶。由於後燕不容許作為宗室一方的西燕「僭舉位號,惑民視聽」,與後燕爭奪燕國領導權。在392年消滅翟魏後,出兵攻伐西燕。

393年冬,慕容垂徵發步騎兵7萬,命丹陽王慕容瓚出井陘關(今河北省井陘縣井陘山),攻晉陽,西燕守將慕容友領兵5萬防守潞川。明年春,慕容垂增調司、冀、青、兗四州兵,分兵三路出滏口(今河北省磁縣西北石鼓山)、壺關、沙亭,西燕分兵拒守。後慕容垂在鄴城西南屯兵月餘,慕容永懷疑後燕欲從太行山南口進兵,將大部兵力調往軹關。夏,慕容垂率大軍出滏口,由天井關向南直趨臺壁(今山西省黎城縣西南),慕容永倉卒集結5萬精兵,與後燕軍大戰於臺壁南,西燕軍中伏大敗,慕容永逃回長子。後燕攻下晉陽,進圍長子,八月間滅西燕。

慕容垂滅西燕後,趁東晉衰亂之際,略地青、兗,把疆域向南擴展到今山東的臨沂、棗莊一帶。

燕魏對峙(394年─396年):386年,拓跋珪建立北魏。起初後燕與北魏的關係本來是友好的,因後燕戰馬缺乏,屢求於魏,甚至發生扣留北魏使者以求名馬的事,兩國關係告結。而北魏採取聯西燕拒後燕的政策,對付後燕。394年西燕危急時,北魏派兵5萬為西燕聲援。次年五月,慕容垂命太子慕容寶、趙王慕容麟率兵8萬伐魏,遣范陽王慕容德率步騎1.8萬為後繼。北魏聽說燕軍北上,把部落、畜產及大軍轉移至黃河以南(今內蒙古伊克昭盟),避開燕軍。到十月,由於塞外嚴寒、士氣低落,後燕不得不撤退。這時北魏派拓跋遵領騎兵7萬,堵塞燕軍南歸之路。拓跋珪自領2萬,進擊後燕軍,後燕軍大敗,亂不成軍,四、五萬兵投降,北魏俘虜了後燕文武將吏數千人,繳獲了兵器、衣甲、糧食無數,拓跋珪將後燕降兵全部坑殺於參合陂,慕容寶等單騎逃回。史稱參合陂之役。

慕容寶等逃回中山後,屢請求再次伐魏,慕容德也勸說慕容垂趁自己尚健在時親征,以免遺留後患。慕容垂接受了他們的意見。396年三月,慕容垂率大軍再次伐魏,敗北魏陳留公拓跋虔,其後慕容垂病情加重,急忙退兵。四月,慕容垂病死。慕容垂此次的北伐,並沒有能夠挽回後燕軍事上的頹勢。此後,拓跋珪就挾其三、四十萬騎兵,長驅進入中原。

衰落滅亡(396年─409年):396年四月,慕容垂病死,子慕容寶繼承帝位。後燕在全國重要的戰略及政治中心有五處,即中山、龍城、鄴、晉陽、薊。八、九月間,北魏拓跋珪率領40餘萬大軍,攻取晉陽。十一月,攻下常山、信都,河北許多郡縣的官員,不是逃亡就是投降。這時慕容寶在中山有步兵12萬、騎兵3.7萬,悉數出抗拒魏軍,大敗而還。魏軍進軍包圍了中山,397年三月,慕容寶率軍突圍,退往中山。十月,魏軍攻下中山,後燕官吏兵投降兩萬餘人,後燕的疆域被切斷為南、北二部。

398年,慕容德在滑臺稱燕王,建立南燕。蘭汗殺死慕容寶,自稱大將軍、大單于、昌黎王。慕容盛殺蘭汗自立,後來討伐高麗及庫莫奚有功,然因治下太嚴,刑罰殘忍,在401年為大臣段璣所暗殺。鮮卑貴族立慕容垂少子慕容熙為帝,他採行了胡漢分治的政策來統治國家。這時的後燕疆域,僅有遼西一帶,疆域狹小,民戶不多,但他卻大興土木,營建宮苑殿閣,給人民帶來無窮的災難。407年,馮跋兄弟趁慕容熙送葬苻后時起事,推高雲(慕容雲)為燕王,殺死慕容熙。409年高雲的禁衛離班、桃仁殺死高雲,馮跋稱燕天王,後燕滅亡。

384年 建国。慕容垂包圍鄴。

385年 打敗高句麗,進入遼東。

394年 攻滅西燕。

395年 燕軍與北魏軍在参合陂大戦,燕軍大敗。

396年 慕容垂親率大軍進攻北魏,途中病故。太子慕容宝即位。

397年 魏拓跋珪攻擊燕都中山。燕王慕容宝逃往龍城。

398年 慕容宝被殺,慕容盛推翻弑君的蘭汗,奪取皇位。

401年 慕容盛被暗殺。太后丁氏擁立慕容熙。

407年 漢人將軍馮跋擁立高雲為燕天王,殺死慕容熙。有些史學家把這年認定為後燕滅亡,北燕建立之年。

409年 高雲被殺,馮跋繼位。有些史學家把這年認定為後燕滅亡,北燕建立之年。


%% -*- coding: utf-8 -*-
%% Time-stamp: <Chen Wang: 2019-12-19 15:20:37>

\subsection{成武帝\tiny(384-396)}

\subsubsection{生平}

燕成武帝慕容垂(326年-396年6月2日),字道明,原名霸,字道業,一說字叔仁,鮮卑名阿六敦,昌黎棘城(今遼寧義縣)鮮卑族人。十六國後燕開國君主。前燕文明帝慕容皝的第五子。在前燕時屢有戰功,更加曾擊退東晉桓溫的北伐軍。然而因為受到當政的慕容評排擠而被逼出走前秦,但很受前秦君主苻堅的寵信。淝水之戰後慕容垂乘時而起,復建燕國,建立後燕,後又滅了同為慕容氏所建的西燕。參合陂之戰戰敗後率軍再攻北魏,在期間發病病重,並在退軍時去世。

原名慕容霸的慕容垂甚得父親慕容皝寵愛,甚至比起身為世子的哥哥慕容儁更多,故此慕容儁忿忿不平。咸康八年(342年),慕容皝進攻高句麗,慕容霸與慕容翰作前鋒,終攻陷高句麗都城丸都(今吉林集安西)。建元二年(344年),慕容皝攻伐宇文逸豆歸,慕容翰為前鋒都督,慕容霸與慕容軍、慕容恪及慕輿根則受命兵分三道進攻。當時逸豆歸派遣涉奕于率領精兵抵禦,慕容翰決意以擊敗涉奕于以摧毀宇文部士氣,令宇文部自潰,於是主動進攻,涉奕于親自迎戰,慕容霸於是在側邀擊,與慕容翰擊敗涉奕于。宇文部士兵於戰後果然自潰,宇文逸豆歸出逃敗死漠北,成功消滅了宇文部。慕容霸則以此功封都鄉侯。永和元年(345年),後趙將領鄧恆領兵數萬駐屯樂安(今河北樂亭縣東北),意圖併吞前燕。慕容皝以慕容霸為平狄將軍,駐軍徒河(今遼寧錦州西北),鄧恆因為畏懼慕容霸而不敢進犯。

永和四年(348年),慕容皝去世,慕容儁繼位燕王,就以慕容霸曾經墮馬而撞斷了牙齒為由改其名為「慕容𡙇」,後更去「夬」而改名慕容垂。次年後趙皇帝石虎去世,國內因諸子爭位而大亂,慕容垂於是上書慕容儁建議出兵後趙。慕容儁初以慕容皝新死而不允,但慕容垂親往都城龍城(今遼寧朝陽市)勸說慕容儁,更自請為前驅領兵威逼鄧恆。在封奕等人的支持下,慕容儁以慕容垂為前鋒都督、建鋒將軍,選二十多萬精兵準備伐趙。

永和六年(350年)二月,慕容儁命慕容垂領二萬兵經循東路經徒河伐趙,另遣慕輿于出西道,自率中軍,兵分三路伐趙。慕容垂到三陘(今河北撫寧縣矛石山),鄧恆驚懼而燒倉庫出逃,退保薊城(今北京)。慕容垂到後盡收樂安、北平兩郡兵糧,與慕容儁會合共攻薊城。三月,燕軍攻下薊城,慕容垂勸止了慕容儁阬殺後趙士卒的決定。不久慕容儁又親率軍隊進攻鄧恆,至清梁(今河北清苑縣西南)時趙將鹿勃早率數千人夜襲燕軍,突入慕容垂幕下,慕容垂於是奮力反擊,手刃了十多人,遏制了鹿勃早的攻擊,及後慕輿根等人領兵擊敗鹿勃早,成功擊退了來襲。

元璽元年(352年),慕容儁稱帝,任黃門侍郎,又遷安東將軍、冀州刺史,鎮常山。至元璽三年(354年)封慕容垂為吳王,並移鎮信都(今河北冀縣)。後召為侍中、右禁將軍、錄留臺事,轉鎮龍城,但因慕容垂在當地很得人心,故被慕容儁召還。後又轉撫軍將軍,並於光壽元年(357年)與中軍將軍慕容虔等率軍大敗敕勒。

光壽二年(358年),中常侍涅皓知慕容儁不喜歡慕容垂,又因可足渾皇后不滿慕容垂妻段氏,於是誣稱段氏與吳國典書令高弼行巫蠱之術,意圖以此牽連慕容垂。段氏寧死不屈,雖然最終死在獄中,但都沒有將慕容垂牽連到事件中,後慕容垂遷鎮東將軍、平州刺史,外鎮遼東。

建熙元年(360年),慕容儁去世,太子慕容暐繼位,以慕容垂為河南大都督、征南將軍、兗州牧、荊州刺史,領護南蠻校尉,鎮梁國。建熙六年(365年),慕容垂與慕容恪共攻東晉控制的洛陽(今河南洛陽市),擊敗並俘虜晉將沈勁,攻下了洛陽,隨後遷都督荊揚洛徐兗豫雍益涼秦十州諸軍事、征南大將軍、荊州牧,鎮魯陽。

太宰慕容恪深知慕容垂的才能,故此在建熙八年(367年)病死前向樂安王慕容臧指出應以慕容垂擔任大司馬一職,又向慕容暐推薦慕容垂在其死後接替自己,將政事都交給慕容垂處理。慕容臧雖將慕容恪的話告訴主政的太傅慕容評,但慕容評沒有按慕容恪的意思做,以慕容沖為大司馬,又調慕容垂為侍中、車騎大將軍、儀同三司。

建熙十年(369年)四月,東晉大司馬桓溫北伐前燕,諸將都無法抵抗晉軍,讓晉軍於七月進駐枋頭(今河南浚縣)。當時慕容暐及慕容評皆大驚,想逃回故都龍城避難。慕容垂於是請求讓他出戰。慕容暐就任命他接替慕容臧擔任南討大都督,率慕容德等五萬兵出戰。慕容垂又請了黃門侍郎封孚、司徒左長史申胤及尚書郎悉羅騰從軍。桓溫當時以降人段思為響導,悉羅騰與晉軍接戰,生擒了段思;接著桓溫派李述進攻,又被悉羅騰所敗,李述更戰死,晉軍士氣於是下降。同時慕容德等又至石門阻止晉軍開通漕運,豫州刺史李邽又斷晉軍糧道,桓溫屢戰不利,糧食又不足,終於九月循陸路撤軍。當時諸將打算立刻追擊,但慕容垂以晉軍初退,必定嚴加戒備,以精銳軍隊斷後,於是打算遲點才追擊,待晉軍乘追兵未至而加速行軍,令兵士筋疲力盡時才進攻。慕容垂因而率領八千騎兵緩緩尾隨晉軍,發現桓溫果然在看不見追兵後加速。數日後慕容垂下令進攻,騎兵於是加速,於襄邑(今河南睢縣西)趕上晉軍,配合慕容德所率埋伏於襄邑的伏兵夾擊桓溫,於是大敗晉軍,殺三萬人。桓溫只有收拾殘軍南退。

枋頭之戰大勝後,慕容垂威名大振,卻令慕容評更加嫌忌他,慕容垂上請有戰功的將領獲得封賞都沒得批准,兩人就因此事在廷上互相爭論,更加深化了兩人的嫌隙。時為太后的可足渾皇后亦厭惡慕容垂,於是與慕容評密謀誅除他。慕容恪子慕容楷及慕容垂舅舅蘭建得悉陰謀,於是建議慕容垂先發制人,除去慕容臧及慕容評。然而慕容垂卻表示寧願出奔國外亦不想骨肉相殘。世子慕容令得知後建議慕容垂北奔龍城,並向慕容暐謝罪,盼望慕容暐感悟召還;即使不然,仍可以固守當地以求自保。慕容垂聽從,於同年十一月就上請到大陸澤狩獵,微服潛歸龍城。然而到邯鄲(今河北邯鄲)時,向來不得寵的兒子慕容麟卻逃還鄴城(今河北臨漳西)告發父親的意圖,於是跟隨慕容垂的人大多都逃走,慕容強亦奉命追捕慕容垂。至范陽(今河北涿縣)時慕容強追上慕容垂,但因慕容令親自斷後,慕容強也不敢進逼。日落後,慕容令表示原本的計劃已不再可行,又建議投奔前秦,慕容垂計窮,亦得接受,於是棄用馬匹以免留下蹤跡,悄悄回鄴城並躲於顯原陵。不久竟有數百個獵人從四方向他們所在聚集,慕容垂等人敵不過他們,卻又無處可逃,甚麼也做不了。就在此時,獵人的獵鷹卻同時飛起,獵人於是散去,慕容垂因而殺白馬祭天,與隨行者誓盟。慕容令在那時又建議讓他回鄴城襲殺慕容評,並以慕容垂的名望取而代之,入輔朝廷。但慕容垂以此危險而否決,於是與妻段氏、慕容令、慕容寶、慕容農、慕容楷及蘭建、高弼等西奔前秦。前秦天王苻堅得知慕容垂來奔,十分高興並親自迎接,以慕容垂為冠軍將軍,封賓徒侯。

慕容垂奔秦次年,前秦就滅了前燕,而慕容垂在前秦官至京兆尹,進封為泉州侯。建元十八年(382年),苻堅執意要攻伐東晉,苻融、石越、苻宏等人都反對,而慕容垂卻說:「弱者被強者所吞,小的被大的兼併,這是合乎自然的,並不難理解。以陛下神武,順應天期,聲威布於海外,百萬衞士,滿朝韓信、白起那樣的良將,晉這個於江南的小國獨獨違抗王命,怎可以再留她給子孫。《詩經》說:「谋夫孔多,是用不集」陛下自己決定就夠了,又何必詢問一眾朝臣!晉武帝平滅東吳,也不過只有張華、杜預幾個臣子支持而已,若果他順從朝臣主流意見,又怎能成就統一大業!」苻堅聽後大喜,更說:「和我一起平定天下的人,就只有你呀。」建元十九年(383年)五月,東晉荊州刺史桓沖北伐,親率主力進攻襄陽(今湖北襄陽市),慕容垂就與苻叡率兵救援。苻叡以慕容垂為前鋒進至沔水,慕容垂在夜間命士兵每人拿十個火把,將它們縛在樹枝上,讓桓沖以為援軍兵力很強,成功逼使他撤還。同年八月,苻堅正式出兵伐晉,並命苻融及慕容垂率二十五萬兵作為前鋒。苻融攻下了壽春(今安徽壽縣),而慕容垂就率別軍攻下了鄖城(今湖北鄖縣)。

十一月,苻堅於淝水大敗給晉軍,前線的前秦軍隊潰敗,就只有沒有參加淝水之戰的慕容垂一軍是完整的,故此苻堅就率殘軍投靠他。當時慕容寶等人就勸慕容垂殺了苻堅,但慕容垂不肯,更分兵給苻堅。苻堅到了洛陽後已經又招聚了十多萬人,一直到了澠池(今河南澠池縣西),慕容垂表示想去安撫河北,並想去拜謁宗廟。苻堅不顧權翼反對而准許慕容垂所請。

當時駐守鄴城的苻丕知道慕容垂要來,懷疑他意圖作亂,更想襲擊他,只是姜讓以慕容垂未有謀反舉動,勸苻丕先嚴兵守衞,注意其舉動,苻丕才安置慕容垂住在鄴城西部。慕容垂當時雖然不肯乘機殺死苻丕,但仍暗中聯結前燕舊臣,密謀復國。此時,丁零人翟斌起兵,苻堅命慕容垂討伐,苻丕一直怕慕容垂於鄴城作亂,正就打算借此機會送走他,更期望他與翟斌打得兩敗俱傷,好讓自己消滅兩股勢力。於是給了慕容垂二千弱兵及差劣的兵器鎧甲,更派了苻飛龍為副手,意圖以他解決慕容垂。

慕容垂留了慕容農、慕容楷及慕容紹於鄴,在行軍途中閔亮和李毗就從鄴來到,並告知苻丕與苻飛龍的圖謀。慕容垂於是以此激怒士眾,又以兵少為由留於河內郡募兵,十日間就令部眾增至八千人。及後正受翟斌攻擊的豫州刺史苻暉請慕容垂快點進兵,慕容垂向苻飛龍說要改在夜裏行軍,出其不意,然而其實就已與諸子計劃襲殺苻飛龍,終在晚上襲殺了苻飛龍及他手下的一千氐兵。第二日,慕容垂命田山回鄴告知留於鄴城的慕容農等起兵響應自己,三人於是與數十騎微服出走,在列人(今河北肥鄉縣東北)起兵。

燕元元年(384年),慕容垂圖攻洛陽,當時翟斌帳下有前燕宗室慕容鳳及前燕舊臣之子段延等,都勸翟斌奉慕容垂為盟主,慕容垂原本不知翟斌究竟是否真心歸附,並沒答允,但到洛陽後苻暉因知苻飛龍遇害而拒絕以營救苻暉為名的慕容垂進城,至此慕容垂才接受了翟斌。不久慕容垂以洛陽是四戰之地,於是改攻鄴城,至滎陽(今河南滎陽)時,群下請慕容垂稱帝。正月丙戌(384年2月9日),慕容垂則以晉元帝的先例,先稱大將軍、大都督,燕王,承制行事。接著率二十多萬大軍直攻鄴城。慕容垂至鄴後改元「燕元」。

慕容垂接著引兵攻鄴,苻丕派了姜讓去責備慕容垂,又勸他放棄叛變。然而慕容垂卻表示只想苻丕和平離開,獻出鄴城,並允諾與前秦世代友好;又恐嚇若果苻丕不從,將要以兵力強攻,怕苻丕到時即使想全身而退也不能。姜讓聽後指責慕容垂背叛王室,不顧昔日前秦收留自己的恩德,現在要做叛逆的鬼。慕容垂聽後沉默,但沒有聽從旁人所說將姜讓殺害,反表示尊敬,讓他回去。然而最終仍然陳述利害,勸苻丕棄城出走,激得苻堅及苻丕再寫書指責。游說不果後,燕軍開始進攻鄴城,並攻下其外城,苻丕退守中城。接著慕容垂又用二十多萬丁零及烏桓人用梯及地道戰術攻城,但都不成功,於是下令修築長圍作防守,築新興城放置輜重,作長期戰。不久又以漳水灌城,仍不能攻下,於是改為圍困鄴城,只留西邊缺口讓秦軍西走。

燕元二年(385年)四月,東晉將領劉牢之入援鄴城,慕容垂詐敗誘敵,於是撤圍退屯新城,不久再北撤,劉牢之於是追擊,苻丕聞訊亦率軍後繼,劉牢之一路追擊至五橋澤,因為軍隊忙於搶奪燕軍輜重而遭慕容垂擊敗。至八月,苻丕棄守鄴城,燕軍終成功佔領鄴城。十二月,慕容垂正式定都中山(今河北定州市)。燕元三年(386年)正月,慕容垂稱帝,二月改元「建興」,始置百官。八月,慕容垂率兵南征以擴疆土,並於次年正月襲河東地區,擊敗晉濟北太守溫詳。

慕容柔、慕容盛及慕容會於建興三年(387年)從西燕都城長子(今山西長子縣西)到達中山,投奔後燕,當時慕容垂就問當地情況,意圖攻取。不久,慕容永将治下慕容儁、慕容垂子孙不问男女全部杀死。建興八年(392年),慕容垂率軍擊潰了丁零人翟釗,吞併了其部眾。次年十一月,慕容垂就親率七萬兵西征西燕;次年二月慕容垂大發司、冀、青、兗四州兵,分置各兵準備進攻。至五月,燕軍經天井關進攻臺壁,先後擊敗大逸豆歸及小逸豆歸,圍困了臺壁。慕容永自太行回軍臺壁,慕容垂亦率軍到臺壁,兩軍於是交戰。事前慕容垂派了驍騎將軍慕容國在澗下設伏,於是假裝撤退引慕容永追擊,數里後慕容國伏兵出現斷慕容永後路,燕軍於是四面進攻,大敗慕容永。慕容永敗後逃回長子,慕容垂就於六月追至,並圍困城池。至八月,被圍的慕容永困急,先後向東晉及北魏求援,但在援軍到來前大逸豆歸部將伐勤就開城門迎燕軍,慕容垂於是俘虜慕容永並將其殺害,吞併了西燕。

建興二年(386年),拓跋珪復代國,不久改稱魏王,建立了北魏。同年因國內不穩而請後燕援軍,慕容垂派慕容麟救援,終助拓跋珪解決事件。事後雖然拓跋珪不接受後燕封爵,但燕魏兩國每年都有使臣往來。建興七年(391年),拓跋珪派弟弟拓跋觚出使後燕,但當時主事的慕容垂諸子為求良馬,竟扣留了拓跋觚,如此令拓跋珪中斷兩國交往。至建興十一年(395年)五月,慕容垂因北魏侵擾邊塞諸郡而命太子慕容寶等人率兵伐魏。當時魏軍率眾迴避,燕軍於七月到了五原(今內蒙古包頭西北),收降三萬多家及大量糧食,但未與魏軍決戰。而拓跋珪乘當時慕容垂患病,故意阻截燕軍通往中山的道通,捕捉後燕使者,令燕軍與其國內通訊斷絕,從而以慕容垂已死的假消息擾動燕軍軍心。兩軍自九月臨五原河相持至十月,慕容寶及慕容麟因為慕容麟部將慕輿嵩相信慕容垂死訊而圖謀作亂的事件而互相猜疑,終於燒船乘夜撤退。當時河面尚未結冰,慕容寶認為魏軍不能渡河追擊,於是不設斥候監視魏軍。至十一月,魏軍因暴風令河面結冰而追擊,在參合陂追上燕軍,並發動突襲大敗燕軍,大量文武官員及四五萬人的燕軍士兵都被俘,後北魏更阬殺全數燕軍士兵。

慕容寶敗逃回中山,並以參合陂之戰為恥,再請進攻北魏。當時司徒慕容德建言說慕容寶大敗後已被北魏輕視,想要慕容垂親自率兵征服他們,以免留為後患。慕容垂於是命幽州牧慕容隆及留守薊城行臺的慕容盛率手下精兵到中山,決定次年再度伐魏。

三月,慕容垂秘密出兵,跨越青嶺(今河北易縣西南五廻山),經天門(今河北淶源縣)鑿山開路,出魏軍不意直攻雲中郡。慕容垂率軍至獵嶺(今山西代縣夏屋山)時就命慕容隆及慕容農為前鋒,進襲平城(今山西大同市)。當時燕國軍隊都因參合陂之戰大敗而畏懼魏軍,就只有慕容隆這批來自龍城的士兵仍然奮勇進攻;而留守平城的魏將拓跋虔亦沒作防備,故此在閏三月慕容隆兵臨平城時才發現燕軍,率眾抵抗,最終敗死,部眾都被燕軍接收。拓跋虔戰死的消息令身處盛樂(今內蒙古和林格爾北)的拓跋珪感到恐懼,打算出走迴避,但各諸知拓跋虔死訊亦各懷二心,令拓跋珪不知何去何從。

慕容垂經過參合陂戰場時看見被阬殺的士兵骸骨堆積如山,就為他們置祭,士兵們見此皆傷心痛哭,這令慕容垂既慚愧又憤恨,終因而嘔血病發,要坐馬車前進,到平城西北三十里處停駐。當時慕容寶已領兵至雲中,聞訊亦退兵。有叛燕軍人就因而向北魏報告慕容垂已死的消息,拓跋珪想去追擊,但知平城陷落後就打消念頭。慕容垂在平城停留了十日後病情加重,於是修築燕昌城而南歸,至四月癸未日(6月2日)於沮陽(今河北懷來縣)去世,享年七十一歲。諡號為成武皇帝,廟號世祖。

崔浩:「垂藉父兄之資,修復舊業,國人歸之,若夜蟲之就火,少加倚仗,易以立功。」(《資治通鑑·卷一百一十八·晉紀四十》)

\subsubsection{燕元}

\begin{longtable}{|>{\centering\scriptsize}m{2em}|>{\centering\scriptsize}m{1.3em}|>{\centering}m{8.8em}|}
  % \caption{秦王政}\
  \toprule
  \SimHei \normalsize 年数 & \SimHei \scriptsize 公元 & \SimHei 大事件 \tabularnewline
  % \midrule
  \endfirsthead
  \toprule
  \SimHei \normalsize 年数 & \SimHei \scriptsize 公元 & \SimHei 大事件 \tabularnewline
  \midrule
  \endhead
  \midrule
  元年 & 384 & \tabularnewline\hline
  二年 & 385 & \tabularnewline\hline
  三年 & 386 & \tabularnewline
  \bottomrule
\end{longtable}

\subsubsection{建兴}

\begin{longtable}{|>{\centering\scriptsize}m{2em}|>{\centering\scriptsize}m{1.3em}|>{\centering}m{8.8em}|}
  % \caption{秦王政}\
  \toprule
  \SimHei \normalsize 年数 & \SimHei \scriptsize 公元 & \SimHei 大事件 \tabularnewline
  % \midrule
  \endfirsthead
  \toprule
  \SimHei \normalsize 年数 & \SimHei \scriptsize 公元 & \SimHei 大事件 \tabularnewline
  \midrule
  \endhead
  \midrule
  元年 & 386 & \tabularnewline\hline
  二年 & 387 & \tabularnewline\hline
  三年 & 388 & \tabularnewline\hline
  四年 & 389 & \tabularnewline\hline
  五年 & 390 & \tabularnewline\hline
  六年 & 391 & \tabularnewline\hline
  七年 & 392 & \tabularnewline\hline
  八年 & 393 & \tabularnewline\hline
  九年 & 394 & \tabularnewline\hline
  十年 & 395 & \tabularnewline\hline
  十一年 & 396 & \tabularnewline
  \bottomrule
\end{longtable}


%%% Local Variables:
%%% mode: latex
%%% TeX-engine: xetex
%%% TeX-master: "../../Main"
%%% End:

%% -*- coding: utf-8 -*-
%% Time-stamp: <Chen Wang: 2021-11-01 11:59:08>

\subsection{惠愍帝慕容寶\tiny(396-398)}

\subsubsection{生平}

燕惠愍帝慕容寶(355年-398年5月27日),字道祐,昌黎郡棘城县(今辽宁省锦州市义县西北)人,後燕第二任君主,慕容垂的第四子,母親是先段后。慕容垂建後燕後,立慕容寶為太子,曾領燕軍攻伐北魏,但在參合陂之戰慘敗。慕容垂死後慕容寶繼位為帝,但就面對北魏南侵,最終慕容寶沒能保住後燕在中原的土地,率眾北走龍城(今遼寧朝陽市),但先後遇上兒子慕容會及大臣段速骨的叛亂。慕容寶出走後為蘭汗所誘而歸龍城,最終被其殺害。

369年,慕容寶隨父親慕容垂等人自前燕逃亡至前秦,在前秦曾任太子洗馬及萬年令。

《太平御覽》載慕容寶玩樗蒲時向神祈禱富貴,擲出機率只有1/32768的三次「盧」采,讓他決心復國。

383年,前秦天王苻堅南伐東晉,慕容寶任陵江將軍。同年苻堅於淝水之戰大敗,軍隊潰散,只有未參與戰事的慕容垂軍隊仍然完整,於是前往投奔。慕容寶於是向父親建議趁機殺掉苻堅,復興燕國,不過慕容垂不肯。慕容垂終於384年稱燕王,立慕容寶為太子,建後燕。

慕容寶隨後經常留守後燕首都中山(今河北定州市),並在慕容垂在外時留守。387年,時慕容垂南征翟遼,井陘人賈鮑招引北山丁零翟瑤等夜襲中山,並攻下外城。章武王慕容宙率奇兵出外,而慕容寶在內鳴鼓抗敵,兩人夾擊之下大敗賈鮑等人,盡俘其眾,賈鮑及翟瑤隻身逃走。

395年五月,因北魏侵略後燕附塞諸部,慕容垂派慕容寶與慕容農、慕容麟等率八萬進攻北魏。拓跋珪率眾西渡黃河作迴避,並在河南治軍。慕容寶率眾到黃河邊就建造船隻打算渡河進攻,不過就在九月要列兵渡河時就遇上大風,船隻都被吹到南岸去。拓跋珪又派人從後阻截慕容寶與後燕國內的通訊,更派抓來的後燕使者訛稱慕容垂已死,令得軍心不穩,慕容寶亦都相當恐懼。十月辛末(11月23日),慕容寶燒船乘夜逃走,當時黃河尚未結冰,慕容寶以為北魏軍隊不能即時渡河追擊,故此不設斥候監察。不過八日後黃河面就因大風而結了冰,拓跋珪率眾渡河,並派二萬騎兵追擊。燕軍至參合陂時有遇上大風,更有一大片黑色塵土從後而來。僧人支曇猛認為這些都預示魏軍將來,建議慕容寶派兵防禦,但慕容寶以為已經走得很遠,笑而不答。慕容麟更奉承地說:「以殿下神武及強盛的兵眾,足以橫行沙漠了,索虜怎敢遠來呀!曇猛亂說話動搖眾心,應該處死呀!」支曇猛堅持,慕容德亦勸慕容寶聽從,慕容寶於是就派了慕容麟率三萬兵在後防備。不過慕容麟根本沒有防備的心,只顧著打獵。最終魏軍於參合陂突襲燕軍,大量兵眾不是在驚慌下互相踐踏或在河中遇溺而死就是束手就擒。慕容寶等人就帶著數千騎兵一同逃返後燕。戰後魏軍更盡坑俘獲的燕軍。

慕容寶回國後以此敗為恥,屢請慕容垂再次攻魏,慕容垂於是於次年(396年)大舉伐魏,並攻下平城(今山西大同市)。不過慕容垂在經過參合陂時看到被坑殺的燕兵骸骨堆積如山,士兵的痛哭聲又遍布山谷,令慕容垂在愧疚及憤恨下患病,被逼終止北伐。時慕容寶等人正率軍至雲中,追擊迴避的拓跋珪,但知慕容垂患病亦只好撤還。

慕容垂在回軍途中去世,慕容寶待回到中山時才為父發喪,並即位為帝,改元為永康。慕容寶年少無大志,喜歡別人奉承。但當太子後則磨煉自己,崇尚儒學,變得善談論,能作文,又卑委地討好慕容垂身邊小臣,以求得美譽。當時朝野都稱許慕容寶,而慕容垂亦認為他能夠保住家業,相當敬重他。後慕容垂為其建承華觀,又於388年以他錄尚書事,授予處理政務的權力,自己只處理一些重要的事務;又以其領大單于職位。不過慕容垂皇后段氏就曾指出太子才能不足,建議慕容垂立遼西王慕容農或高陽王慕容隆。又指出慕容麟為人奸詐而不肯屈於人下,有輕視太子之心,建議慕容垂早日除去他。不過慕容垂並不接納。慕容寶及慕容麟聽聞段皇后有這番話更是十分痛恨。慕容寶即位後,便派了慕容麟去逼令段后自殺。段后憤怒地說:「你們兄弟連逼殺嫡母的事也做,怎能保護國家!我怎會怕死,就可惜國家快滅亡了。」隨後便自殺。段后死後,慕容寶更因痛恨段后,以其無母后之道而打算不為其行居喪之禮,不過計劃最終在中書令眭邃反對之下擱置。

慕容寶繼位不久,北魏就出兵進攻後燕,並進攻中山,但被慕容隆擊退。及後北魏大人沒根因被拓跋珪厭惡而投降後燕,並請還攻北魏。慕容寶不敢給他重兵,只分了數百騎兵給他。沒根接著夜襲魏營,拓跋珪發覺有變而狼狽逃走,但沒根礙於兵少,無法對魏軍造成大傷害。永康二年(397年),任北魏并州監軍的沒根侄兒醜提因沒根降燕而害怕被株連,於是率部眾回國預備作亂。拓跋珪聞訊就想北返,派使者向後燕求和,但其時慕容寶知北魏有內亂,故此不肯答允,並率步兵十二萬及騎兵三萬七千的大軍到柏肆預備截擊返兵的魏軍。不久,魏軍到了滹沱水南岸紥營,慕容寶就率兵在夜間渡河,並招募了勇士一萬多人夜襲魏營,而慕容寶就在營北列陣作支援。夜襲部隊乘風縱火並迅速發動進攻,魏軍大亂,拓跋珪亦在驚惶中棄營出逃,燕軍到來帳中只得其衣物。不過接著燕軍竟然自亂,互相攻擊。拓跋珪於營外看見這情況就鳴鼓收整部眾,終大敗夜襲軍,更轉攻慕容寶軍,慕容寶只得回到北岸。次日,魏軍已經重整並與燕軍對峙,相反燕軍就士氣盡失。慕容寶最終只得退還中山,北魏軍跟著追擊,屢敗燕軍。慕容寶因屢敗而恐懼,竟拋棄大軍,自率二萬騎兵速速退回中山,又命士兵拋棄戰袍武器,以加快速度,丟失了大量軍需品,而且其時正遇大風雪,大量士兵凍死道上。拓跋珪及後再派兵進圍中山,駐屯在芳林園。當時中山城中將士都想出戰擊退圍城魏軍,慕容隆亦向慕容寶建議乘城中將士的鬥志進攻。慕容寶原本同意,但慕容麟卻多次反對,令慕容寶反悔,慕容隆於是多次列兵備戰都被逼罷兵。後慕容寶又意圖求和,以交還拓跋觚及割常山以西土地為條件,但不久即反悔,氣得拓跋珪親自率軍圍攻中山。當時有數千將士都自願請戰,但慕容隆披甲上馬,正待命令與魏軍決戰時,慕容麟再次勸止慕容寶,令兵眾忿恨,慕容隆亦痛心哭泣。

早前,慕輿皓謀弒慕容寶而改立慕容麟,失敗出逃,但令慕容麟內心不安。就在慕容麟勸止慕容寶派兵出戰當晚,以兵劫逼左衞將軍北地王慕容精,要他率禁軍弒慕容寶。慕容精拒絕,慕容麟就殺害慕容精,出奔西山依附丁零餘眾。其時慕容寶知慕容會正領兵前來,怕慕容麟劫奪慕容會的軍隊,先一步據有龍城,於是召見慕容隆及慕容農,想放棄中山,退保龍城,最終就與太子慕容策、慕容農、慕容隆、慕容盛等人率萬餘騎出城與慕容會軍會合。慕容寶到薊城時身邊的近衞已經散盡,只餘慕容隆的數百騎守。慕容會率眾於薊南迎接後,慕容寶削減慕容會的軍隊而分給慕容農及慕容隆,不久使率眾北歸龍城。當時慕容會整兵與慕容隆及慕容農的騎兵擊敗前來追擊的魏將石河頭,而其時慕容會的兵眾都不想歸屬於慕容農及慕容隆,於是向慕容寶提議讓慕容會率兵解中山之圍,然後還都中山。不過慕容寶拒絕,而慕容寶身邊的人則勸慕容寶殺掉慕容會,慕容寶亦感到慕容會謀反之心,意圖除去他,只因慕容農及慕容隆反對而作罷。慕容會恐懼,就派了仇尼歸襲擊慕容隆及慕容農,殺了慕容隆並重創慕容農。慕容會自宣稱二人謀逆,已經被殺,慕容寶一心要殺慕容會,於是出言讓他安心,接著就暗中命慕輿騰斬殺慕容會,但失敗。慕容會回到其軍中,接著進攻慕容寶,慕容寶就率數百騎直奔龍城。慕容寶及後拒絕慕容會誅除左右,立其為皇太子的要求,於是引來慕容會進攻龍城。慕容寶更在西門特意責罵慕容會,令慕容會下令士兵向慕容寶鼓譟揚威,藉此激起城中士兵憤怒。慕容寶軍於是在黃昏大敗慕容會,接著又派了高雲率敢死隊夜襲慕容會,再敗慕容會,令其逃奔中山。

永康三年(398年),慕容德派李延北上告知拓跋珪北歸的消息,慕容寶於是決意南征。慕容寶率兵至乙連時,長上段速骨、宋赤眉等人因為兵眾恐懼出征作亂,先逼高陽王慕容崇為主,殺害慕容宙及段誼等人。慕容寶與慕容農及慕輿騰會合,試圖討伐段速骨,但因為士兵厭戰,兵眾都潰散,慕容寶等人唯有奔還龍城。其時蘭汗暗中與段速骨勾結,將龍城軍隊帶到龍城以東,大大削弱了龍城的防禦,而慕容盛則內徙附近的人民,選取了一萬多個男丁守城。段速骨攻城時,慕容農因受蘭汗所誘,竟然叛歸段速骨。原本龍城守軍戰鬥力尚足以抵禦段速骨,令段速骨軍死傷甚大,但段速骨讓守軍看見慕容農後就瓦解了軍心,最終令龍城失守,慕容寶等人出走。

慕容寶到薊城後,在慕容盛等人反對下沒有回龍城,轉而想南投慕容德,可是在知道慕容德已稱燕王後就不敢繼續前進。當時慕容盛等人在冀州成功招集了一些支持慕容寶的力量,但其時蘭汗又派人來迎慕容寶,慕容寶以為蘭汗是忠臣,又想到蘭汗是父親慕容垂的舅舅,於是都不再懷疑,決意回龍城。慕容寶快到龍城時,蘭汗就派了弟蘭加難去迎接,但同時又命兄蘭堤封閉城門,最終蘭加難引慕容寶到龍城外邸並將之殺害,享年四十四歲。蘭汗殺太子慕容策及王公大臣,自稱大都督、大將軍、大單于,昌黎王。不久慕容盛殺蘭汗,改慕容寶諡號為惠愍皇帝,上廟號烈宗。

\subsubsection{永康}

\begin{longtable}{|>{\centering\scriptsize}m{2em}|>{\centering\scriptsize}m{1.3em}|>{\centering}m{8.8em}|}
  % \caption{秦王政}\
  \toprule
  \SimHei \normalsize 年数 & \SimHei \scriptsize 公元 & \SimHei 大事件 \tabularnewline
  % \midrule
  \endfirsthead
  \toprule
  \SimHei \normalsize 年数 & \SimHei \scriptsize 公元 & \SimHei 大事件 \tabularnewline
  \midrule
  \endhead
  \midrule
  元年 & 396 & \tabularnewline\hline
  二年 & 397 & \tabularnewline\hline
  三年 & 398 & \tabularnewline
  \bottomrule
\end{longtable}


%%% Local Variables:
%%% mode: latex
%%% TeX-engine: xetex
%%% TeX-master: "../../Main"
%%% End:

%% -*- coding: utf-8 -*-
%% Time-stamp: <Chen Wang: 2021-11-01 11:59:56>

\subsection{昭武帝慕容盛\tiny(398-401)}

\subsubsection{开封公慕容详生平}

慕容詳(?-397年),昌黎郡棘城县人(今辽宁省锦州市义县)人,追尊燕文明帝慕容皝的曾孙,后燕宗室,封开封公,後一度稱燕帝。

北魏君主、魏王拓跋珪率領魏軍圍攻后燕首都中山,後燕永康二年(397年),後燕不敵北魏的進攻,皇帝慕容寶等撤出都城中山(今中國河北省定州市),出逃龍城(今辽宁省朝阳市)。依然使用永康年號至398年。城內大亂,慕容詳當時不及跟隨撤退,因此被推為盟主以抵禦北魏的攻擊。然而,慕容詳為鞏固自己的地位,不斷翦除城內其他勢力。同年稍後不久,魏軍退卻后,慕容詳即皇帝位,改元建始。

由於慕容詳嗜酒好殺,不恤士民。七月,中山城民遂迎趙王慕容麟入城,慕容麟入城後,慕容詳被逮捕後處死。慕容麟自立,改元延平。

\subsubsection{趙王慕容麟生平}

慕容麟(4世纪-398年),昌黎郡棘城县(今辽宁省锦州市义县)人,後燕成武帝慕容垂之子,婢妾所生。惠愍帝慕容寶庶弟。原為後燕的趙王,後來一度稱燕帝。

早年慕容垂於前燕時期,叛前燕奔前秦時,慕容麟曾逃回前燕告發(369年)。其嫡長兄慕容令被前燕放逐後,欲偷襲龍城(今中國遼寧省遼陽縣),亦是被慕容麟告發,事敗身死(370年)。雖然慕容麟屢次出賣父兄,但後來前秦統一華北,慕容垂回到前燕故地時,還是不忍心殺掉慕容麟,最後是殺了慕容麟的母親頂罪,而把他放逐在外,很少見面。

383年,慕容垂於前秦淝水之戰敗後,陰謀背叛,慕容麟從中貢獻不少計策,慕容垂大為讚賞,待慕容麟開始與其他兒子相同。384年,慕容垂建後燕,慕容麟被任命為撫軍大將軍。同年,率軍攻陷中山(今中國河北省定州市),聲威大振,遂留守中山。386年,慕容垂稱帝後,慕容麟被封為趙王。其後數年,帶領燕軍南征北討,立下不少戰功。396年,慕容垂去世,太子慕容寶繼位,慕容麟被任命為尚書左僕射。

395年的參合陂之战及397年的柏肆之战,後燕二度慘敗給北魏,國力大衰。397年,北魏君主魏王拓跋珪率領魏軍進圍後燕都城中山,慕容麟謀叛,遂以武力威脅北地王慕容精,命其率領禁軍謀殺慕容寶,慕容精拒絕,慕容麟於是殺慕容精,逃出中山,依附丁零遺眾。不久,慕容寶等率領部下撤出中山,出逃龍城,依然使用永康年號至398年。城內大亂,開封公慕容詳被推為盟主以抵禦北魏的攻擊,後來魏軍退卻后,慕容詳即皇帝位,改元建始。但由於慕容詳嗜酒好殺,不恤士民,中山城民遂迎慕容麟入城,慕容麟入城後,殺慕容詳,亦稱帝,改元延平。但隨後北魏再攻中山,又被北魏擊敗,慕容麟自去年號,南奔鄴城(今中國河南省臨漳縣)投靠范陽王慕容德,並不再稱帝。

398年,慕容麟向慕容德上尊號,慕容德於是稱燕王,建立南燕,但不久慕容麟又陰謀推翻慕容德,因此被慕容德所殺。

\subsubsection{兰汗生平}

蘭汗(?-398年8月15日),昌黎郡棘城县(今辽宁省锦州市义县)人,慕容垂堂舅。

昌黎王蘭汗與段速骨密謀叛亂,后又杀段速骨,派兰加难诱杀慕容寶,改元青龍。夺位当年即为慕容盛所杀。蘭穆、蘭堤、蘭加難、蘭和、蘭揚也都被杀。慕容盛继位。

\subsubsection{昭武帝慕容盛生平}

燕昭武帝慕容盛(373年-401年9月13日),字道運,十六国后燕国主,慕容寶之庶長子。

年少時沈實敏銳,富謀略。當前秦天王苻堅誅殺慕容氏時,与叔父慕容柔潛逃投靠慕容沖。385年,慕容冲在阿房宫即皇帝位。慕容盛对慕容柔说:“夫十人之长,亦须才过九人,然后得安。今中山王才不逮人,功未有成,而骄汰已甚,殆难济乎!”慕容冲果然很快被杀,慕容柔、慕容盛以及慕容盛的弟弟慕容会又投靠慕容永。慕容盛指出三人是慕容垂子孙,正被世系疏远的慕容永猜疑,不如投奔祖父慕容垂。387年,他们从长子县一起逃回了后燕。不久慕容永果然尽杀治下的慕容儁、慕容垂子孙。

其父慕容宝登基后,慕容盛反对立祖父慕容垂所爱的庶弟慕容会为储,而支持嫡出的三弟慕容策。慕容会后来谋反被诛。

段速骨叛变后,慕容寶想南奔投靠叔叔慕容德,被慕容盛劝阻。慕容德已自称燕王,不但无意迎接慕容宝还意图谋害,慕容宝又回到龙城,受兰汗诓骗而遭殺害,慕容策亦遇害。慕容盛因為是蘭汗的女婿,与妻子關係甚篤,非但得以不死,還被蘭汗封做侍中。慕容盛乘机离间蘭汗、兰堤和兰加难三兄弟,派遣太原王慕容奇(兰汗外孙)在建安聚眾討伐蘭汗,当兰汗派出其兄长太尉兰堤讨伐时,慕容盛又反間蘭汗称慕容奇实力不足,兰堤才是慕容奇的幕后主谋。于是,兰汗将兰堤之职务转予抚军将军仇尼慕。如此种种,导致兰堤和兰加难兄弟生惧而背叛兰汗。太子兰穆提醒兰汗,慕容盛是仇家,必与慕容奇勾结,兰汗因而召见慕容盛,但慕容盛在妻兰王妃告密下佯病不出,躲过一劫。蘭汗派遣兄子蘭全反擊慕容奇卻反被滅。兰穆出兵讨伐兰堤、兰加难前大宴将士,兰汗父子喝得酩酊大醉,慕容盛与李旱等人趁機杀死兰穆,又引兵將兰汗乱刀砍死。又遣李旱及张真袭杀兰汗子鲁公兰和于令支及陈公兰扬于白狼,並捕杀兰堤和兰加难。

為父復仇之後,慕容盛一度想斬草除根殺死妻子蘭氏,母后丁氏不忍,進行勸阻,因此只廢黜蘭氏,终身未曾立后。慕容盛於398年8月19日(七月廿一辛亥)只改元建平,仍以长乐王称制,诸王皆降为公。又命慕容奇停止用兵,慕容奇抗命且带兵来攻,被慕容盛打败並赐死。11月12日(十月十七丙子)即皇帝位,並誅殺幽州刺史慕容豪、尚書左僕射張通及昌黎尹張順等人。399年改年號為長樂。400年2月11日(正月初一壬子)慕容盛自贬号为庶人天王。慕容盛後來討伐高丽及庫莫奚有功,然治法太嚴,刑罰殘忍,401年9月13日(八月二十壬辰),左將軍慕容國與殿中將軍秦輿、段贊等人密謀暗殺慕容盛,卻東窗事發,眾人皆被誅,軍中大亂。最終平亂時,慕容盛本人卻身中暗器,傷重不治,享年29歲,在位僅三年,慕容熙繼其位。

父親慕容寶。妻子蘭汗的女兒蘭氏。兒子慕容定在慕容盛被殺害後的時候還是年紀幼小。

慕容盛于401年闰八月十九葬于兴平陵(具体方位不详),庙号中宗,谥号昭武皇帝。

\subsubsection{建平}

\begin{longtable}{|>{\centering\scriptsize}m{2em}|>{\centering\scriptsize}m{1.3em}|>{\centering}m{8.8em}|}
  % \caption{秦王政}\
  \toprule
  \SimHei \normalsize 年数 & \SimHei \scriptsize 公元 & \SimHei 大事件 \tabularnewline
  % \midrule
  \endfirsthead
  \toprule
  \SimHei \normalsize 年数 & \SimHei \scriptsize 公元 & \SimHei 大事件 \tabularnewline
  \midrule
  \endhead
  \midrule
  元年 & 396 & \tabularnewline
  \bottomrule
\end{longtable}

\subsubsection{长乐}

\begin{longtable}{|>{\centering\scriptsize}m{2em}|>{\centering\scriptsize}m{1.3em}|>{\centering}m{8.8em}|}
  % \caption{秦王政}\
  \toprule
  \SimHei \normalsize 年数 & \SimHei \scriptsize 公元 & \SimHei 大事件 \tabularnewline
  % \midrule
  \endfirsthead
  \toprule
  \SimHei \normalsize 年数 & \SimHei \scriptsize 公元 & \SimHei 大事件 \tabularnewline
  \midrule
  \endhead
  \midrule
  元年 & 399 & \tabularnewline\hline
  二年 & 400 & \tabularnewline\hline
  三年 & 401 & \tabularnewline
  \bottomrule
\end{longtable}


%%% Local Variables:
%%% mode: latex
%%% TeX-engine: xetex
%%% TeX-master: "../../Main"
%%% End:

%% -*- coding: utf-8 -*-
%% Time-stamp: <Chen Wang: 2021-11-01 12:00:09>

\subsection{昭文帝慕容熙\tiny(401-407)}

\subsubsection{生平}

燕昭文帝慕容熙(385年-407年9月14日),字道文,一字長生,十六國時期後燕國君主,鮮卑人,成武帝慕容垂的幼子,惠愍帝慕容寶之弟,母親是貴嬪段氏。原封河間王,蘭汗之亂時曾被封為遼東公,慕容盛即位後,封河間公。

後燕長樂三年(401年),慕容盛被變軍殺害。慕容盛有兒子慕容定,年紀幼小。群臣希望慕容盛之弟慕容元繼位。但慕容熙因与慕容盛之母丁太后有私情,備受她寵愛,八月癸巳日(9月14日),遂被密迎入宮即天王位,慕容元被賜死,不久慕容熙改元光始。次年(402年),慕容熙又害死了慕容定,且娶了苻秦中山尹苻謨的兩個女兒苻娀娥為貴人、苻訓英為貴嬪,苻訓英極其受寵。丁太后怨恨,遂謀廢慕容熙,事洩,丁太后被殺。

慕容熙立苻訓英為皇后,苻娀娥為貴人,非常寵愛苻氏姐妹,因此興築宮殿、遊玩打獵,導致軍民死亡的數以萬計。苻娀娥生病,有人自稱能醫,結果醫死了,慕容熙遂將醫生支解後焚燒,追封苻娀娥為愍皇后。慕容熙與苻訓英更是玩樂不知節制。元始五年(405年)攻高句麗遼東城,原本城將攻陷,慕容熙只為了要與苻后一同坐輦車進城,因而命軍暫緩登城,以致延誤戰機,不能攻下遼東。次年(406年),後燕攻契丹未果而回師,又為了苻后想要觀戰而臨時拋棄輜重轉而偷襲高句麗,致士卒馬匹,疲累寒冷,沿路死亡不可勝數。又如苻后夏天想要吃凍魚,冬天要吃生地黃,官員也因不能取得而被斬首。

建始元年(407年),苻后去世,慕容熙痛不欲生,喪禮上命檢查百官有無哭泣,規定未哭者給予處罰,群臣只好口含辣物以刺激流淚。又賜死高陽王慕容隆的王妃張氏以殉葬,右仆射韦璆等人都害怕自己去殉葬,每天都洗澡换衣等候命令。此外規定家家戶戶都要參與建造苻后陵墓的工程,更使得國家財政揮霍一空。臨葬,慕容熙竟打開棺材,与苻訓英的屍體親熱一番,才准下葬。

由於早先中衛將軍馮跋與其弟馮素弗曾因事獲罪於後燕帝慕容熙,因此慕容熙一直有殺馮跋兄弟之意。七月甲子日(407年9月14日),馮跋兄弟於是趁慕容熙送葬苻后時起事,推高雲(慕容雲)為燕王,慕容熙被生擒後斬首,和苻訓英合葬。後來被諡昭文皇帝。

\subsubsection{光始}

\begin{longtable}{|>{\centering\scriptsize}m{2em}|>{\centering\scriptsize}m{1.3em}|>{\centering}m{8.8em}|}
  % \caption{秦王政}\
  \toprule
  \SimHei \normalsize 年数 & \SimHei \scriptsize 公元 & \SimHei 大事件 \tabularnewline
  % \midrule
  \endfirsthead
  \toprule
  \SimHei \normalsize 年数 & \SimHei \scriptsize 公元 & \SimHei 大事件 \tabularnewline
  \midrule
  \endhead
  \midrule
  元年 & 401 & \tabularnewline\hline
  二年 & 402 & \tabularnewline\hline
  三年 & 403 & \tabularnewline\hline
  四年 & 404 & \tabularnewline\hline
  五年 & 405 & \tabularnewline\hline
  六年 & 406 & \tabularnewline
  \bottomrule
\end{longtable}

\subsubsection{建始}

\begin{longtable}{|>{\centering\scriptsize}m{2em}|>{\centering\scriptsize}m{1.3em}|>{\centering}m{8.8em}|}
  % \caption{秦王政}\
  \toprule
  \SimHei \normalsize 年数 & \SimHei \scriptsize 公元 & \SimHei 大事件 \tabularnewline
  % \midrule
  \endfirsthead
  \toprule
  \SimHei \normalsize 年数 & \SimHei \scriptsize 公元 & \SimHei 大事件 \tabularnewline
  \midrule
  \endhead
  \midrule
  元年 & 407 & \tabularnewline
  \bottomrule
\end{longtable}


%%% Local Variables:
%%% mode: latex
%%% TeX-engine: xetex
%%% TeX-master: "../../Main"
%%% End:

%% -*- coding: utf-8 -*-
%% Time-stamp: <Chen Wang: 2021-11-01 12:00:17>

\subsection{惠懿帝高雲\tiny(401-407)}

\subsubsection{生平}

燕惠懿帝高雲(4世纪-409年11月6日),曾改名慕容雲,字子雨,高句驪人。十六国時期後燕末代君主,一說為北燕开国国主,称号天王。

早期的高雲於後燕時沉默寡言,並沒有什麼名氣,只有中衛將軍馮跋看出他的氣度與他結交。

後燕永康二年(397年),高雲因率軍擊敗慕容寶之子慕容會的叛軍,被慕容寶收養,賜姓慕容氏,封夕陽公。

後燕建初元年(407年)馮跋反,殺皇帝慕容熙,在馮跋支持之下,慕容雲即天王位,改元曰正始,國號大燕,恢復原本的高姓。高雲自知無功而登大位,因此培養一批禁衛保護自己,但後來反被禁衛離班和桃仁所殺,高雲死後被諡惠懿皇帝。

由於對高雲是否屬後燕慕容氏一族成員的看法不同,因此有人認為高雲是後燕末任君主,也有人把他視為北燕立國君主。

\subsubsection{正始}

\begin{longtable}{|>{\centering\scriptsize}m{2em}|>{\centering\scriptsize}m{1.3em}|>{\centering}m{8.8em}|}
  % \caption{秦王政}\
  \toprule
  \SimHei \normalsize 年数 & \SimHei \scriptsize 公元 & \SimHei 大事件 \tabularnewline
  % \midrule
  \endfirsthead
  \toprule
  \SimHei \normalsize 年数 & \SimHei \scriptsize 公元 & \SimHei 大事件 \tabularnewline
  \midrule
  \endhead
  \midrule
  元年 & 407 & \tabularnewline\hline
  二年 & 409 & \tabularnewline
  \bottomrule
\end{longtable}


%%% Local Variables:
%%% mode: latex
%%% TeX-engine: xetex
%%% TeX-master: "../../Main"
%%% End:



%%% Local Variables:
%%% mode: latex
%%% TeX-engine: xetex
%%% TeX-master: "../../Main"
%%% End:
