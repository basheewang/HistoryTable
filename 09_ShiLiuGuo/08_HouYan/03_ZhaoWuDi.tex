%% -*- coding: utf-8 -*-
%% Time-stamp: <Chen Wang: 2021-11-01 11:59:56>

\subsection{昭武帝慕容盛\tiny(398-401)}

\subsubsection{开封公慕容详生平}

慕容詳(?-397年),昌黎郡棘城县人(今辽宁省锦州市义县)人,追尊燕文明帝慕容皝的曾孙,后燕宗室,封开封公,後一度稱燕帝。

北魏君主、魏王拓跋珪率領魏軍圍攻后燕首都中山,後燕永康二年(397年),後燕不敵北魏的進攻,皇帝慕容寶等撤出都城中山(今中國河北省定州市),出逃龍城(今辽宁省朝阳市)。依然使用永康年號至398年。城內大亂,慕容詳當時不及跟隨撤退,因此被推為盟主以抵禦北魏的攻擊。然而,慕容詳為鞏固自己的地位,不斷翦除城內其他勢力。同年稍後不久,魏軍退卻后,慕容詳即皇帝位,改元建始。

由於慕容詳嗜酒好殺,不恤士民。七月,中山城民遂迎趙王慕容麟入城,慕容麟入城後,慕容詳被逮捕後處死。慕容麟自立,改元延平。

\subsubsection{趙王慕容麟生平}

慕容麟(4世纪-398年),昌黎郡棘城县(今辽宁省锦州市义县)人,後燕成武帝慕容垂之子,婢妾所生。惠愍帝慕容寶庶弟。原為後燕的趙王,後來一度稱燕帝。

早年慕容垂於前燕時期,叛前燕奔前秦時,慕容麟曾逃回前燕告發(369年)。其嫡長兄慕容令被前燕放逐後,欲偷襲龍城(今中國遼寧省遼陽縣),亦是被慕容麟告發,事敗身死(370年)。雖然慕容麟屢次出賣父兄,但後來前秦統一華北,慕容垂回到前燕故地時,還是不忍心殺掉慕容麟,最後是殺了慕容麟的母親頂罪,而把他放逐在外,很少見面。

383年,慕容垂於前秦淝水之戰敗後,陰謀背叛,慕容麟從中貢獻不少計策,慕容垂大為讚賞,待慕容麟開始與其他兒子相同。384年,慕容垂建後燕,慕容麟被任命為撫軍大將軍。同年,率軍攻陷中山(今中國河北省定州市),聲威大振,遂留守中山。386年,慕容垂稱帝後,慕容麟被封為趙王。其後數年,帶領燕軍南征北討,立下不少戰功。396年,慕容垂去世,太子慕容寶繼位,慕容麟被任命為尚書左僕射。

395年的參合陂之战及397年的柏肆之战,後燕二度慘敗給北魏,國力大衰。397年,北魏君主魏王拓跋珪率領魏軍進圍後燕都城中山,慕容麟謀叛,遂以武力威脅北地王慕容精,命其率領禁軍謀殺慕容寶,慕容精拒絕,慕容麟於是殺慕容精,逃出中山,依附丁零遺眾。不久,慕容寶等率領部下撤出中山,出逃龍城,依然使用永康年號至398年。城內大亂,開封公慕容詳被推為盟主以抵禦北魏的攻擊,後來魏軍退卻后,慕容詳即皇帝位,改元建始。但由於慕容詳嗜酒好殺,不恤士民,中山城民遂迎慕容麟入城,慕容麟入城後,殺慕容詳,亦稱帝,改元延平。但隨後北魏再攻中山,又被北魏擊敗,慕容麟自去年號,南奔鄴城(今中國河南省臨漳縣)投靠范陽王慕容德,並不再稱帝。

398年,慕容麟向慕容德上尊號,慕容德於是稱燕王,建立南燕,但不久慕容麟又陰謀推翻慕容德,因此被慕容德所殺。

\subsubsection{兰汗生平}

蘭汗(?-398年8月15日),昌黎郡棘城县(今辽宁省锦州市义县)人,慕容垂堂舅。

昌黎王蘭汗與段速骨密謀叛亂,后又杀段速骨,派兰加难诱杀慕容寶,改元青龍。夺位当年即为慕容盛所杀。蘭穆、蘭堤、蘭加難、蘭和、蘭揚也都被杀。慕容盛继位。

\subsubsection{昭武帝慕容盛生平}

燕昭武帝慕容盛(373年-401年9月13日),字道運,十六国后燕国主,慕容寶之庶長子。

年少時沈實敏銳,富謀略。當前秦天王苻堅誅殺慕容氏時,与叔父慕容柔潛逃投靠慕容沖。385年,慕容冲在阿房宫即皇帝位。慕容盛对慕容柔说:“夫十人之长,亦须才过九人,然后得安。今中山王才不逮人,功未有成,而骄汰已甚,殆难济乎!”慕容冲果然很快被杀,慕容柔、慕容盛以及慕容盛的弟弟慕容会又投靠慕容永。慕容盛指出三人是慕容垂子孙,正被世系疏远的慕容永猜疑,不如投奔祖父慕容垂。387年,他们从长子县一起逃回了后燕。不久慕容永果然尽杀治下的慕容儁、慕容垂子孙。

其父慕容宝登基后,慕容盛反对立祖父慕容垂所爱的庶弟慕容会为储,而支持嫡出的三弟慕容策。慕容会后来谋反被诛。

段速骨叛变后,慕容寶想南奔投靠叔叔慕容德,被慕容盛劝阻。慕容德已自称燕王,不但无意迎接慕容宝还意图谋害,慕容宝又回到龙城,受兰汗诓骗而遭殺害,慕容策亦遇害。慕容盛因為是蘭汗的女婿,与妻子關係甚篤,非但得以不死,還被蘭汗封做侍中。慕容盛乘机离间蘭汗、兰堤和兰加难三兄弟,派遣太原王慕容奇(兰汗外孙)在建安聚眾討伐蘭汗,当兰汗派出其兄长太尉兰堤讨伐时,慕容盛又反間蘭汗称慕容奇实力不足,兰堤才是慕容奇的幕后主谋。于是,兰汗将兰堤之职务转予抚军将军仇尼慕。如此种种,导致兰堤和兰加难兄弟生惧而背叛兰汗。太子兰穆提醒兰汗,慕容盛是仇家,必与慕容奇勾结,兰汗因而召见慕容盛,但慕容盛在妻兰王妃告密下佯病不出,躲过一劫。蘭汗派遣兄子蘭全反擊慕容奇卻反被滅。兰穆出兵讨伐兰堤、兰加难前大宴将士,兰汗父子喝得酩酊大醉,慕容盛与李旱等人趁機杀死兰穆,又引兵將兰汗乱刀砍死。又遣李旱及张真袭杀兰汗子鲁公兰和于令支及陈公兰扬于白狼,並捕杀兰堤和兰加难。

為父復仇之後,慕容盛一度想斬草除根殺死妻子蘭氏,母后丁氏不忍,進行勸阻,因此只廢黜蘭氏,终身未曾立后。慕容盛於398年8月19日(七月廿一辛亥)只改元建平,仍以长乐王称制,诸王皆降为公。又命慕容奇停止用兵,慕容奇抗命且带兵来攻,被慕容盛打败並赐死。11月12日(十月十七丙子)即皇帝位,並誅殺幽州刺史慕容豪、尚書左僕射張通及昌黎尹張順等人。399年改年號為長樂。400年2月11日(正月初一壬子)慕容盛自贬号为庶人天王。慕容盛後來討伐高丽及庫莫奚有功,然治法太嚴,刑罰殘忍,401年9月13日(八月二十壬辰),左將軍慕容國與殿中將軍秦輿、段贊等人密謀暗殺慕容盛,卻東窗事發,眾人皆被誅,軍中大亂。最終平亂時,慕容盛本人卻身中暗器,傷重不治,享年29歲,在位僅三年,慕容熙繼其位。

父親慕容寶。妻子蘭汗的女兒蘭氏。兒子慕容定在慕容盛被殺害後的時候還是年紀幼小。

慕容盛于401年闰八月十九葬于兴平陵(具体方位不详),庙号中宗,谥号昭武皇帝。

\subsubsection{建平}

\begin{longtable}{|>{\centering\scriptsize}m{2em}|>{\centering\scriptsize}m{1.3em}|>{\centering}m{8.8em}|}
  % \caption{秦王政}\
  \toprule
  \SimHei \normalsize 年数 & \SimHei \scriptsize 公元 & \SimHei 大事件 \tabularnewline
  % \midrule
  \endfirsthead
  \toprule
  \SimHei \normalsize 年数 & \SimHei \scriptsize 公元 & \SimHei 大事件 \tabularnewline
  \midrule
  \endhead
  \midrule
  元年 & 396 & \tabularnewline
  \bottomrule
\end{longtable}

\subsubsection{长乐}

\begin{longtable}{|>{\centering\scriptsize}m{2em}|>{\centering\scriptsize}m{1.3em}|>{\centering}m{8.8em}|}
  % \caption{秦王政}\
  \toprule
  \SimHei \normalsize 年数 & \SimHei \scriptsize 公元 & \SimHei 大事件 \tabularnewline
  % \midrule
  \endfirsthead
  \toprule
  \SimHei \normalsize 年数 & \SimHei \scriptsize 公元 & \SimHei 大事件 \tabularnewline
  \midrule
  \endhead
  \midrule
  元年 & 399 & \tabularnewline\hline
  二年 & 400 & \tabularnewline\hline
  三年 & 401 & \tabularnewline
  \bottomrule
\end{longtable}


%%% Local Variables:
%%% mode: latex
%%% TeX-engine: xetex
%%% TeX-master: "../../Main"
%%% End:
