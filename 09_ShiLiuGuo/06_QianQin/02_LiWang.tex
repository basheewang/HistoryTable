%% -*- coding: utf-8 -*-
%% Time-stamp: <Chen Wang: 2019-12-19 10:09:16>

\subsection{苻生\tiny(355-357)}

\subsubsection{生平}

秦越厉王苻生(335年-357年),字長生,略陽臨渭(今甘肅秦安)氐族人。十六國時期前秦景明帝苻健的第三子。史載苻生「荒耽淫虐,殺戮無道,常彎弓露刃以見朝臣,錘鉗鋸鑿備置左右」在位兩年期間殺害了多位大臣,以及做了多項殘忍變態的事。最終苻生被苻堅發動政變推翻,降封越王,不久被殺。不過後世亦有人認為苻生的暴政其實是史家誣捏渲染的結果。

苻生天生就只有一隻眼,年幼而無賴,爺爺苻洪因而十分討厭他。一次苻洪特地戲弄他,問侍者:「我聽說瞎子都只有一行眼淚,這是真的嗎?」侍者回答:「是呀。」在場的苻生聽後大怒,取出佩刀自殘,流出一行血,說:「這也是一行眼淚呀。」苻洪見狀大驚,鞭打他。苻生說:「我耐得下兵器,受不住鞭打!」。苻洪骂他:「你再是這樣,我就要把你送去当奴隶!」苻生竟答:「那可不就像石勒那樣嗎?」當時苻洪正歸屬後趙,而當時後趙皇帝石虎心中其實十分忌憚苻氏的勢力,故苻洪聽後震惊,光着脚就跑来遮住他的嘴巴。苻洪隨後勸苻健把苻生殺掉,但苻健要動手時就被其弟苻雄制止,說:「男孩子長大後就會改過的了,為何要這樣做呢!」

苻生长大后,力大無比,能徒手格击猛兽,奔跑速度飛快,而擊、刺、騎射的能力亦勇冠一時。皇始元年(351年),苻健稱天王,建立前秦,苻生獲封為淮南公,次年苻健稱帝,苻生進封淮南王。皇始四年(354年),桓温北伐前秦,苻生與太子苻萇、丞相苻雄等出兵迎擊,他就曾經十多次单马突擊晉軍,令晉軍傷亡甚大。

太子苻萇在追擊撤退的桓溫軍隊時受了傷,不久死去,苻健以讖言「三羊五眼」應符,於皇始五年(晉永和十一年,355年)立苻生為太子。同年,苻健患病,太尉苻菁乘時想殺苻生奪位但失敗被殺。隨後苻健以太師魚遵、丞相雷弱兒、太傅毛貴、司空王墮、尚書令梁楞、尚書左僕射梁安、尚書右僕射段純及吏部尚書辛牢八人為顧命大臣,輔助苻生。然而,苻健慮及苻生凶暴嗜酒,擔心他不能保全家業,被大臣有機可乘,於是對苻生說:「六夷酋帥及掌權的大臣,若果不遵從你的命令,那就立即除去他們。」

同年六月乙酉日(355年7月10日),苻健死,次日苻生即位為帝,改元壽光。不過,苻生本身酗酒,在登位後就常常酒醉,群臣上朝都很少見到苻生,連群臣的上奏都因苻生長醉而被擱在一邊。即使上朝,苻生每當發怒都只會殺人,即位後就多次出現殺戮大臣以至殘害生命的凶殘事件,苻健設的八名輔政大臣全都被苻生所殺。最終造成「宗室、勳舊、親戚、忠良殺害殆盡,王公在位者悉以告歸,人情危駭,道路以目」的狀況。

苻生即位後,立刻就改了年號,當時群臣上奏:「先帝死後未逾年而改元,不合禮法呀。」苻生卻大怒,要找出最初提出這上奏的大臣,最終找出了顧命大臣之一的段純,就將他殺害。後來,中書監胡文及中書令王魚向苻生報告天象:「最近有客星(彗星)在大角,熒惑(火星)入東井。大角,是皇帝之坐;東井,表示秦地;按占卜,不出三年,國內將有大喪,大臣會被殺戮,希望陛下自脩德行以避禍。」苻生卻說:「皇后和朕位置相應,可以應了國喪之劫。毛太傅、梁車騎、梁僕射受遺詔輔政,就應了大臣被戮的劫。」於是就殺了梁皇后、毛貴、梁楞及梁安四人;太師魚遵亦於壽光三年(357年)因民謠「東海大魚化為龍,男皆為王女為公」而被殺。又一次苻生與大臣飲宴,更在奏樂時唱起歌來,命尚書令辛牢勸酒。然而,就因為大臣們都沒有全都醉倒,於是就拿起弓將辛牢射殺。更有一次在咸陽故城設宴,將遲到的大臣殺害。

另外,苻生亦寵信趙韶、董榮等人,丞相雷弱兒以他們亂政,經常公開在朝堂批評他們,他們於是在苻生面前中傷雷弱兒。苻生於是誅殺雷弱兒及其家人,最終因為雷弱兒南安羌族酋長的身分,各羌族部落都有離心。司空王墮亦痛恨董榮等人,不肯親附,在董榮的唆使下,苻生又殺王墮以應日蝕之變。

亦因苻生天生殘疾,「不足、不具、少、無、缺、傷、殘、毀、偏、隻」等字都是要避諱的,絕不能說。但就有不少大臣和侍從因此而死。其中太醫令程延在研安胎藥時向苻生解釋人參,就說了「雖小小不具,自可堪用」而被苻生下令鑿出雙眼,然後斬首。

苻生賞罰沒有準則,大臣不論稱頌他還是批評他稍有不當,都可能被殺,但寵臣的姦佞之言卻都接納。而苻生的姬妾只要表現得稍不合其意,都會被殺,並棄屍渭水。苻生又愛虐待動物,活活的剝下牛、羊、驢、馬的皮毛,或者用熱水燙雞、豬、鵝,將三、五十隻這樣的動物一起放在殿上觀賞。苻生甚至還將死囚的臉皮活活剝掉,命其在群臣面前跳舞。苻生更曾命宮女與男子裸體在其面前性交,甚至曾在路上看見一對同行的兄妹,就命他們亂倫,兄妹最終因不肯聽從而被殺。受斬腳、刳胎、拉脅、鋸頸等其他酷刑的人亦數以千計。

苻生聽到有對自己的怨言,更下詔書稱自己並沒有不善,自己所作的根本不算濫刑暴虐。據說當時還有食人野獸橫行,平民為了避開猛獸自保就聚居而且荒廢農業。苻生則認為野獸吃飽了人就會走,不會長久的,且認為天降災劫其實正是對應平民一直犯罪,協助天子以刑罰教導平民而已,只要不犯罪就不必怨天尤人。

苻生曾經命三輔居民興建渭橋,金紫光祿大夫程肱以妨礙農業為由勸諫,反觸怒苻生,被殺。又一次長安突然颳起大風,極之影響人們活動,苻生舅舅左光祿大夫強平於是借天變而勸諫苻生愛護禮待公卿,致敬宗社,去如秋霜的威嚴而立三春般的恩澤等。但苻生則認為強平是妖言,不顧臣下以至太后的懇求,堅持殺死強平。

壽光三年(晉升平元年,357年),姚襄進圖關中,更派人招納因雷弱兒被誅而產生離心的關內羌胡。苻生於是派了衞大將軍苻黃眉等率兵抵抗,最終大敗敵軍,更殺姚襄,令姚襄弟姚萇率眾歸降。苻黃眉立了大功,但凱旋後卻沒有獲得苻生褒賞,反而被多次當眾侮辱。苻黃眉因而憤怒,圖謀殺死苻生,但風聲洩露,反被殺,更株連不少王公親戚。

而當時御史中丞梁平老等人都勸有時譽的苻生堂弟、東海王苻堅殺苻生以救國,苻堅同意但不敢發難。但六月有一晚,苻生對侍婢表示翌日就要殺苻法、苻堅兩兄弟,侍婢於是立刻告訴二人,於是二人與強汪、梁平老和呂婆樓等都率兵衝入宮,宮中宿衞將士知道苻堅奪位都向其投降。苻生當時仍然在酒醉中,知有人攻來,就大驚,問侍從:「那是甚麼人?」侍從答:「是賊!」苻生就說:「為甚麼不下拜!」苻堅兵眾聽後大笑,苻生更說:「還不快快下拜,不拜的我就斬了他!」苻堅於是廢苻生為越王,自己繼承帝位,並降稱天王。不久,苻生被苻堅杀害,享年二十三歲,諡為厲王。儿子苻馗被封为越侯。

苻生无后。苻坚后来平定苻生弟苻廋等人叛乱,赐苻廋死,赦免苻廋诸子,并安排苻廋的儿子过继苻生为后。

苻洪:「此兒狂悖勃,宜早除之,不然,長大必破人家。」

薛讚、權翼:「主上猜忍暴虐,中外離心。」

《晉書》史臣曰:「長生慘虐,稟自率由。覩辰象之災,謂法星之夜飲;忍生靈之命,疑猛獸之朝飢。但肆毒於刑殘,曾無心於戒懼。招亂速禍,不亦宜乎!」

《晉書》贊曰:「長生昏虐,敗不旋踵。」

但有些記述表示所謂苻生暴虐也可能是史臣渲染的結果。楊衒之《洛陽伽藍記》卷二記載隱士趙逸之言,云:「國滅之後,觀其史書,皆非實錄,莫不推過於人,引善自向」,如「苻生雖好勇嗜酒,亦仁而不殺。觀其治典,未為凶暴,及詳其史,天下之惡皆歸焉。苻堅自是賢主,然賊君取位,妄書君惡,凡諸史官,皆是類也。」劉知幾《史通》曲筆篇云:「昔秦人不死,驗苻生之厚誣」,即是據此。

呂思勉亦懷疑苻生的一系列殘忍殺人、文詞避諱、以刀刃錘斧威懾群臣等都是史官誣陷、醜化苻生的結果。誅殺梁安、雷弱兒等人亦因他們其實是有通晉的嫌疑,是不得已,卻招來謗毀之聲。稱「他如怠荒、淫穢,自更易誣。《金史·海陵本紀》述其不德之亂,連章累牘,而篇末著論,即明言其不足信,正同一律。」

\subsubsection{寿光}

\begin{longtable}{|>{\centering\scriptsize}m{2em}|>{\centering\scriptsize}m{1.3em}|>{\centering}m{8.8em}|}
  % \caption{秦王政}\
  \toprule
  \SimHei \normalsize 年数 & \SimHei \scriptsize 公元 & \SimHei 大事件 \tabularnewline
  % \midrule
  \endfirsthead
  \toprule
  \SimHei \normalsize 年数 & \SimHei \scriptsize 公元 & \SimHei 大事件 \tabularnewline
  \midrule
  \endhead
  \midrule
  元年 & 355 & \tabularnewline\hline
  二年 & 356 & \tabularnewline\hline
  三年 & 357 & \tabularnewline
  \bottomrule
\end{longtable}


%%% Local Variables:
%%% mode: latex
%%% TeX-engine: xetex
%%% TeX-master: "../../Main"
%%% End:
