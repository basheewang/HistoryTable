%% -*- coding: utf-8 -*-
%% Time-stamp: <Chen Wang: 2019-12-19 10:03:55>


\section{前秦\tiny(351-394)}

\subsection{简介}

前秦(350年—394年)是十六国之一。350年氐族人苻洪占据关中,称三秦王。352年苻健称帝,定都长安,国号“秦”。370年起,先後灭前燕、前凉及代国,统一北方。394年被西秦和後秦所灭。當時朝鮮半島由高句麗、百濟、新羅割據,接受前秦册封。北方外族有柔然、庫莫奚、契丹及高車。西有吐谷渾及白蘭。

因其所据为战国时秦国故地,故以此立国号。前秦之称最早见于《十六国春秋》,后为别于其他以“秦”为国号政权,而袭用之。又以其王室姓苻,故又称为苻秦。

西晉末年,西晉政權顛覆之際,略陽氐族推出貴族苻洪為首領。前趙劉曜在長安稱帝,以苻洪為氐王。後石勒滅前趙,苻洪降於石勒。333年,石虎徙關中豪傑及羌戎至關東,以苻洪為流民都督,居於枋頭。苻洪自稱大都督、大將軍、大單于、三秦王,不久為石虎舊將麻秋所毒死,其第三子苻健代統其眾。

苻健自枋頭而西,關中氐人紛起響應,苻健乘机進占关中,據有关陇。351年建都長安,苻健自稱大秦天王、大單于。352年,改稱皇帝,国号秦,史稱前秦。

起初苻健知道中原「民心思晉」,在枋頭時,打著晉征西大將軍、都督關中諸軍事、雍州刺史來作號召;抵達關中之後,遣使向東晉稱臣,以緩和關中地區的矛盾,直到他稱帝後,才和東晉斷絕關係。354年,東晉大將桓溫親率大軍四萬攻秦,因苻健採清野政策,晉軍在給養問題遇到困難,只好撤退。355年苻健死,子苻生繼位,因淫殺過度,357年,苻健弟苻雄之子苻堅殺死苻生自立。

苻堅在登位以前,就聽見王猛的名聲,並約見王猛,談得十分投契。即帝位,任王猛以政。王猛採取政治改革,加強中央集權,抑制貴族勢力發展來強化中央力量,並興修關中水利,前秦國力逐漸增強。370年,前秦滅前燕,擒慕容暐;371年,滅仇池氐楊氏;373年,攻取東晉梁、益二州,西南夷邛、筰、夜郎皆歸附於秦;376年,滅前涼張氏;同年,乘鮮卑拓跋氏衰亂之際,進兵滅代;382年,命呂光駐西域。中原地區盡為前秦版土之下,史稱「東極滄海,西併龜茲,南包襄陽,北盡沙漠」。東北、西域各國都遣使和前秦建立關係,只有東南一隅的東晉與他對峙。

378年,前秦征南大將軍苻丕等率領步騎兵七萬人,攻擊東晉所屬的襄陽,東晉梁州刺史朱序死守近一年,城池陷落被俘。

379年,前秦右將軍毛當、強弩將軍王顯,率二萬人自襄陽出發,跟後將軍俱難、兗州刺史彭超會師,攻擊東晉淮河以南各城池,攻陷盱眙,包圍三阿。東晉兗州刺史謝玄出兵救援,四次擊敗秦軍,俱難、彭超向北逃走,僅保住一命。

建元十八年(382年)苻堅之大将吕光率兵七万伐龟兹,龟兹王白纯不降,吕光进军讨平龟兹。

383年,前秦天王苻堅親率騎兵二十七萬、步兵六十萬南下,對東晉發動總攻擊。弟弟陽平公苻融擔任前鋒,十月,攻陷壽陽。苻堅派東晉降將朱序向東晉征討大都督謝石勸降,但朱序反而將秦軍狀況密告謝石,建議晉軍乘秦軍未全部集結時發動攻擊。十一月,東晉前鋒都督謝玄的部將劉牢之率兵五千突襲洛澗,秦軍大敗,死一萬五千人。

晉軍乘勝西進,秦軍在淝水西岸佈陣對峙。謝玄派人要求秦軍略向後撤,讓晉軍渡水決戰。苻堅企圖乘晉軍半渡淝水時予以截擊,同意後退,但秦軍軍心不穩,一退陣腳大亂,不能停止。晉軍乘勢渡水猛烈攻擊,混亂中苻融墮馬被晉軍所殺,朱序又在陣後大呼「秦兵敗矣!」於是秦軍崩潰,四散逃亡,前鋒三十萬人中,死亡的佔十分之七八。

淝水之戰後,原先歸附前秦的其他民族,紛紛乘機獨立,黃河以北又再陷入分裂的狀態。

383年,前燕降將、鮮卑族的冠軍將軍慕容垂,奉命攻擊在新安起兵的丁零部落首領翟斌,途中屠殺副將苻飛龍及一千人的氐人部隊。384年,慕容垂自稱「燕王」,廢除前秦年號,建立後燕,並進攻駐守鄴城的前秦長樂公苻丕。

前秦北地長史慕容泓(前燕帝慕容暐的弟弟),聽到叔父慕容垂攻鄴的消息,投奔關東集結數千鮮卑人,自稱大將軍、濟北王,建立西燕。苻堅派兒子鉅鹿公苻叡當統帥,羌人將領姚萇任參謀,出兵討伐,在華澤大敗,苻叡被斬殺。苻堅大怒,姚萇畏罪逃到渭北,被族人推為盟主。姚萇遂自稱大將軍、大單于、萬年秦王,建立後秦。

385年,後秦包圍新平郡,苻堅投奔五將山,被後秦將領吳忠俘擄,送回新平郡單獨囚禁。八月,姚萇派人向苻堅索取傳國玉璽,又遊說苻堅禪讓帝位,苻堅大怒拒絕,痛罵姚萇,只求一死,又先殺女兒苻寶、苻錦。八月二十六日,姚萇派人闖入囚禁苻堅的佛寺,縊殺苻堅,時年四十八歲。姚萇為掩飾自己的弒逆惡名,追尊苻堅為「壯烈天王」。

394年,七月,前秦帝苻登在馬毛山以南跟後秦帝姚興交戰,被生擒後斬首,太子苻崇投奔湟中繼承帝位。十月,苻崇被西秦首領乞伏乾歸驅逐,投奔隴西王楊定,兩人於攻擊西秦時被西秦涼州刺史乞伏軻彈斬殺,前秦到此滅亡。

\subsection{惠武帝生平}

苻洪(285年-350年),字廣世,略陽臨渭氐人,是前秦政權奠基者。苻洪原名蒲洪,後以讖文有「艸付應王」,遂改苻姓。氐族部落小帥蒲怀归之子,亦是前秦開國君主苻健之父。苻洪先後歸附前趙和後趙兩個政權,後在後趙內大亂時試圖謀取中原。最終雖然遭毒殺,但他所累積的力量令其子苻健在關中成功建立前秦。

蒲洪原是氐族酋長,因為驍勇而多謀略而得氐人畏服,永嘉四年(310年)時曾獲前趙(當時國號為「漢」)皇帝劉聰任命為平遠將軍,但蒲洪不受,反自稱護氐權尉、秦州刺史、「略陽公」。後因在永嘉之亂時大散錢財以向英傑之士訪尋轉危為安的方法,於是被宗人蒲光及蒲突推舉為盟主。太興二年(319年),前趙皇帝劉曜遷都長安,蒲光等逼蒲洪附前趙,於是獲授為率義侯。咸和三年(328年),劉曜在與石勒的決戰中被俘,蒲洪於是西保隴山自守。次年,前趙退保上邽的殘餘力量遭後趙將領消滅,蒲洪於是向石虎歸降。石虎於是以蒲洪為冠軍將軍,監六夷軍事,委以西方之事。

咸和八年(333年),後趙鎮守關中的河東王石生聯同鎮洛陽的石朗反抗丞相石虎,蒲洪於是自稱雍州刺史,歸附前涼。同年石生及石朗敗死,石虎命麻秋討伐蒲洪,蒲洪於是率二萬戶向石虎歸降。石虎則以蒲洪為光烈將軍、護氐校尉。蒲洪到長安後游說石虎遷關中豪傑及氐、羌人到東方,充實京師襄國。石虎於是遷秦、雍二州人民及氐、羌族人到關東,並以蒲洪為龍驤將軍、流民都督,命其居於枋頭。

蒲洪後多有征戰,累有戰功,於咸康四年(338年)被石虎拜為使持節、都督六夷諸軍事、冠軍大將軍,西平郡公。當時石虎養子石閔就因蒲洪實力強大,諸子有才且接近京畿,勸石虎誅除蒲洪,但石虎不單沒聽從,對蒲洪的待遇反更為優厚。永和五年(349年)蒲洪遷車騎大將軍、開府儀同三司、都督雍、秦二州諸軍事、雍州刺史,進封為略陽郡公。

同年,石虎去世,石遵繼位,石閔再提出對付蒲洪的建議,石遵於是削去蒲洪都督一職。此舉觸怒了蒲洪,於是憤而向東晉投降。同時,因為當時後趙因諸子爭位內亂,原先被遷到關東的秦雍流民都西歸故土,經過枋頭時就以蒲洪為主,於是令蒲洪的部眾增至十多萬人,當時在首都鄴城的蒲洪子蒲健亦出奔枋頭。新登位的後趙皇帝石鑒為怕蒲洪以其力量威脅中央,於是以蒲洪都督關中諸軍事、征西大將軍、雍州牧、領秦州刺史,讓他率眾返回關中。不過,當時蒲洪其實已經有心稱帝得天下。次年,東晉朝廷以蒲洪歸降,以其為氐王、使持節、征北大將軍、都督河北諸軍事、冀州刺史、廣川郡公。

當時,有人勸蒲洪稱王,蒲洪於是以「艸付應王」的讖文而改姓「苻」,自稱大將軍、大單于、三秦王。早前,後趙將領麻秋自長安率眾返鄴,途中被苻洪派兵俘獲,以其為軍師將軍。麻秋及後勸苻洪先放棄中原,先取關中作為基地,然後才再圖中原。苻洪十分同意,但不久麻秋就在宴會中以毒酒毒殺苻洪,意圖併吞其部眾;苻健於是斬殺麻秋。中毒的苻洪在死前囑咐苻健在其死後要速速入關,享年六十六歲。

苻健於永和七年(351年)即天王位,建立前秦,追諡苻洪為惠武皇帝,廟號太祖。

%% -*- coding: utf-8 -*-
%% Time-stamp: <Chen Wang: 2021-11-01 11:56:49>

\subsection{景明帝苻健\tiny(351-355)}

\subsubsection{生平}

秦景明帝苻健(317年-355年7月10日),字建業,略阳临渭(今甘肃秦安)人,氐族,苻洪第三子,十六国前秦開國皇帝。苻健繼父親苻洪統領部眾並成功入關,定都長安(今陝西西安),建立前秦。後屢次作戰征服其他反抗前秦的關內勢力,更擊敗北伐的晉軍。

苻健弓馬嫻熟,驍勇果敢,好施予亦善於事奉人,故此深得後趙皇帝石虎父子寵愛,當時石虎心中仍提防苻氏,暗殺了苻健的兩個兄長,但就沒有加害苻健。永和六年(350年),因應苻洪歸降東晉,苻健獲授假節、右將軍、監河北征討前鋒諸軍事、襄國縣公。同年苻洪軍師將軍麻秋毒殺苻洪,意圖併吞苻洪部眾,苻健於是收殺麻秋。苻洪臨死時向苻健說:「我之所以一直未入關中,就是以為能夠奪得中原;今天卻不幸被麻秋那小子加害。中原不是你們兄弟能夠爭奪到的,我死了後,你就快快入關中呀!」苻健於是接領父親的部眾,去掉父親自稱的大都督、大將軍、「三秦王」的稱號,稱東晉所授的官爵,並派叔父苻安到東晉報喪,請示朝命。

同年,後趙新興王石祗在襄國(今河北邢台)即位為帝,又以苻健為都督河南諸軍事、鎮南大將軍、開府儀同三司、兗州牧、「略陽郡公」。不過,苻健當時並沒有助石祗對付冉閔,反將目標對準關中,只為麻痺當時據有關中的杜洪才接受後趙的任命。苻健又在駐地枋頭(今河南浚縣西)興治宮室,教人種麥,顯得根本沒有心思佔領關中。但及後苻健就自稱晉征西大將軍、都督關中諸軍事、雍州刺史,率眾西進,並在盟津渡過黃河。渡河前,苻健命苻雄和苻菁分別領兵從潼關(今陝西渭南市潼關縣北)和軹關(今河南濟源東北)進攻,自己則跟隨苻雄渡河,並在渡河後燒掉浮橋,意在死戰。杜洪部將張先在潼關抵抗苻健軍,但被擊敗。及後苻健派苻雄兵行渭北,附近的氐、羌酋長都斬杜洪使而向苻健投降,苻菁、魚遵經過的城邑亦都投降,更在渭北生擒張先,令三輔地區大致都落在苻健之手。杜洪見局勢如此,唯有退守長安,但苻健隨即進攻長安,杜洪被逼棄長安而逃奔司竹(今陝西司竹鄉),苻健於是進據長安。苻健見長安人心思晉,於是向東晉獻捷報,並與東晉征西大將軍桓溫修好。於是令秦雍二州的少數民族和漢人都向苻健歸附,苻健亦攻滅佔領上邽(今甘肅天水市),不肯歸降的後趙涼州刺史石寧。

永和七年(351年),左長史賈玄碩請苻健依劉備稱「漢中王」事,表苻健為都督關中諸軍事、大將軍、大單于、「秦王」。但苻健則假裝憤怒的說:「我豈有能力當秦王呀!而且出使東晉的使者還未回來,你們又怎知我的官爵呀。」然而,不久就又暗示賈玄碩等為他上尊號,最終在再三推讓後,于正月丙辰日(351年3月4日)即天王、大單于位,大封宗室及諸子為公爵,建國號大秦,年號「皇始」,正式建立前秦政權。次年正月辛卯日(352年2月2日),苻健稱帝,進諸公爵為王爵,並授大單于位予太子苻萇。

皇始元年(351年),被苻氏驅逐的杜洪引東晉梁州刺史司馬勳伐前秦,苻健於是率兵在五丈原擊退他。司馬勳敗歸漢中(今陝西漢中)後,杜洪被其部將張琚所殺,不久苻健領二萬兵攻滅張琚,更派兵擄掠關東,助後趙豫州刺史張遇擊敗東晉將領謝尚,及後擄張遇及其部眾回長安,並對張遇授官。後張遇謀反事敗,引發雍州孔特等人舉兵反抗前秦。最終苻健亦派兵成功平定。

皇始四年(354年),桓溫北伐,自率主力軍自武關(今陝西丹鳳縣東)直取長安,另命司馬勳在進攻隴西。前秦初戰不利,被桓溫進攻至長安東南防近的灞上,逼得苻健要盡出三萬精兵出城抵禦桓溫。然而因桓溫並不急於進攻,而且苻健先晉兵一步收取熟麥,故此最終逼得桓溫退兵,苻健更乘勢追擊晉軍,大敗對方。

苻健勤於政事,多次召見公卿談論治國之道,而且一改後趙時苛刻奢侈之風,改以薄賦節儉,更專崇儒學,禮待長者,故此得到人們稱許。

皇始四年(354年),皇太子苻萇在追擊桓溫時受傷,同年傷重而死。次年(晉永和十一年,355年),苻健因讖文中有「三羊五眼」字句,遂以淮南王苻生當太子。至當年六月,苻健患病,苻生在苻健宮室侍疾,而當時任太尉的平昌公苻菁則以為苻健已死,直接領兵入宮,打算殺死苻生自立。但到東掖門時,苻健知道宮中發生事變,自登端門,陳兵自衞。當時苻菁部眾見苻健未死,於是驚懼潰散,苻健於是拿下苻菁,將他殺死。不久,苻健以太師魚遵、丞相雷弱兒、太傅毛貴、司空王墮、尚書令梁楞、尚書左僕射梁安、尚書右僕射段純及吏部尚書辛牢等為輔政大臣。但又告訴太子:「六夷酋帥及掌權的大臣,若果不遵從你的命令,那就立即除去他們。」。六月乙酉日(7月10日),苻健病逝,享年三十九歲。苻健死後諡為明皇帝,廟號稱世宗,後改諡為景明皇帝,廟稱高祖。

\subsubsection{皇始}

\begin{longtable}{|>{\centering\scriptsize}m{2em}|>{\centering\scriptsize}m{1.3em}|>{\centering}m{8.8em}|}
  % \caption{秦王政}\
  \toprule
  \SimHei \normalsize 年数 & \SimHei \scriptsize 公元 & \SimHei 大事件 \tabularnewline
  % \midrule
  \endfirsthead
  \toprule
  \SimHei \normalsize 年数 & \SimHei \scriptsize 公元 & \SimHei 大事件 \tabularnewline
  \midrule
  \endhead
  \midrule
  元年 & 351 & \tabularnewline\hline
  二年 & 352 & \tabularnewline\hline
  三年 & 353 & \tabularnewline\hline
  四年 & 354 & \tabularnewline\hline
  五年 & 355 & \tabularnewline
  \bottomrule
\end{longtable}


%%% Local Variables:
%%% mode: latex
%%% TeX-engine: xetex
%%% TeX-master: "../../Main"
%%% End:

%% -*- coding: utf-8 -*-
%% Time-stamp: <Chen Wang: 2019-12-19 10:09:16>

\subsection{苻生\tiny(355-357)}

\subsubsection{生平}

秦越厉王苻生(335年-357年),字長生,略陽臨渭(今甘肅秦安)氐族人。十六國時期前秦景明帝苻健的第三子。史載苻生「荒耽淫虐,殺戮無道,常彎弓露刃以見朝臣,錘鉗鋸鑿備置左右」在位兩年期間殺害了多位大臣,以及做了多項殘忍變態的事。最終苻生被苻堅發動政變推翻,降封越王,不久被殺。不過後世亦有人認為苻生的暴政其實是史家誣捏渲染的結果。

苻生天生就只有一隻眼,年幼而無賴,爺爺苻洪因而十分討厭他。一次苻洪特地戲弄他,問侍者:「我聽說瞎子都只有一行眼淚,這是真的嗎?」侍者回答:「是呀。」在場的苻生聽後大怒,取出佩刀自殘,流出一行血,說:「這也是一行眼淚呀。」苻洪見狀大驚,鞭打他。苻生說:「我耐得下兵器,受不住鞭打!」。苻洪骂他:「你再是這樣,我就要把你送去当奴隶!」苻生竟答:「那可不就像石勒那樣嗎?」當時苻洪正歸屬後趙,而當時後趙皇帝石虎心中其實十分忌憚苻氏的勢力,故苻洪聽後震惊,光着脚就跑来遮住他的嘴巴。苻洪隨後勸苻健把苻生殺掉,但苻健要動手時就被其弟苻雄制止,說:「男孩子長大後就會改過的了,為何要這樣做呢!」

苻生长大后,力大無比,能徒手格击猛兽,奔跑速度飛快,而擊、刺、騎射的能力亦勇冠一時。皇始元年(351年),苻健稱天王,建立前秦,苻生獲封為淮南公,次年苻健稱帝,苻生進封淮南王。皇始四年(354年),桓温北伐前秦,苻生與太子苻萇、丞相苻雄等出兵迎擊,他就曾經十多次单马突擊晉軍,令晉軍傷亡甚大。

太子苻萇在追擊撤退的桓溫軍隊時受了傷,不久死去,苻健以讖言「三羊五眼」應符,於皇始五年(晉永和十一年,355年)立苻生為太子。同年,苻健患病,太尉苻菁乘時想殺苻生奪位但失敗被殺。隨後苻健以太師魚遵、丞相雷弱兒、太傅毛貴、司空王墮、尚書令梁楞、尚書左僕射梁安、尚書右僕射段純及吏部尚書辛牢八人為顧命大臣,輔助苻生。然而,苻健慮及苻生凶暴嗜酒,擔心他不能保全家業,被大臣有機可乘,於是對苻生說:「六夷酋帥及掌權的大臣,若果不遵從你的命令,那就立即除去他們。」

同年六月乙酉日(355年7月10日),苻健死,次日苻生即位為帝,改元壽光。不過,苻生本身酗酒,在登位後就常常酒醉,群臣上朝都很少見到苻生,連群臣的上奏都因苻生長醉而被擱在一邊。即使上朝,苻生每當發怒都只會殺人,即位後就多次出現殺戮大臣以至殘害生命的凶殘事件,苻健設的八名輔政大臣全都被苻生所殺。最終造成「宗室、勳舊、親戚、忠良殺害殆盡,王公在位者悉以告歸,人情危駭,道路以目」的狀況。

苻生即位後,立刻就改了年號,當時群臣上奏:「先帝死後未逾年而改元,不合禮法呀。」苻生卻大怒,要找出最初提出這上奏的大臣,最終找出了顧命大臣之一的段純,就將他殺害。後來,中書監胡文及中書令王魚向苻生報告天象:「最近有客星(彗星)在大角,熒惑(火星)入東井。大角,是皇帝之坐;東井,表示秦地;按占卜,不出三年,國內將有大喪,大臣會被殺戮,希望陛下自脩德行以避禍。」苻生卻說:「皇后和朕位置相應,可以應了國喪之劫。毛太傅、梁車騎、梁僕射受遺詔輔政,就應了大臣被戮的劫。」於是就殺了梁皇后、毛貴、梁楞及梁安四人;太師魚遵亦於壽光三年(357年)因民謠「東海大魚化為龍,男皆為王女為公」而被殺。又一次苻生與大臣飲宴,更在奏樂時唱起歌來,命尚書令辛牢勸酒。然而,就因為大臣們都沒有全都醉倒,於是就拿起弓將辛牢射殺。更有一次在咸陽故城設宴,將遲到的大臣殺害。

另外,苻生亦寵信趙韶、董榮等人,丞相雷弱兒以他們亂政,經常公開在朝堂批評他們,他們於是在苻生面前中傷雷弱兒。苻生於是誅殺雷弱兒及其家人,最終因為雷弱兒南安羌族酋長的身分,各羌族部落都有離心。司空王墮亦痛恨董榮等人,不肯親附,在董榮的唆使下,苻生又殺王墮以應日蝕之變。

亦因苻生天生殘疾,「不足、不具、少、無、缺、傷、殘、毀、偏、隻」等字都是要避諱的,絕不能說。但就有不少大臣和侍從因此而死。其中太醫令程延在研安胎藥時向苻生解釋人參,就說了「雖小小不具,自可堪用」而被苻生下令鑿出雙眼,然後斬首。

苻生賞罰沒有準則,大臣不論稱頌他還是批評他稍有不當,都可能被殺,但寵臣的姦佞之言卻都接納。而苻生的姬妾只要表現得稍不合其意,都會被殺,並棄屍渭水。苻生又愛虐待動物,活活的剝下牛、羊、驢、馬的皮毛,或者用熱水燙雞、豬、鵝,將三、五十隻這樣的動物一起放在殿上觀賞。苻生甚至還將死囚的臉皮活活剝掉,命其在群臣面前跳舞。苻生更曾命宮女與男子裸體在其面前性交,甚至曾在路上看見一對同行的兄妹,就命他們亂倫,兄妹最終因不肯聽從而被殺。受斬腳、刳胎、拉脅、鋸頸等其他酷刑的人亦數以千計。

苻生聽到有對自己的怨言,更下詔書稱自己並沒有不善,自己所作的根本不算濫刑暴虐。據說當時還有食人野獸橫行,平民為了避開猛獸自保就聚居而且荒廢農業。苻生則認為野獸吃飽了人就會走,不會長久的,且認為天降災劫其實正是對應平民一直犯罪,協助天子以刑罰教導平民而已,只要不犯罪就不必怨天尤人。

苻生曾經命三輔居民興建渭橋,金紫光祿大夫程肱以妨礙農業為由勸諫,反觸怒苻生,被殺。又一次長安突然颳起大風,極之影響人們活動,苻生舅舅左光祿大夫強平於是借天變而勸諫苻生愛護禮待公卿,致敬宗社,去如秋霜的威嚴而立三春般的恩澤等。但苻生則認為強平是妖言,不顧臣下以至太后的懇求,堅持殺死強平。

壽光三年(晉升平元年,357年),姚襄進圖關中,更派人招納因雷弱兒被誅而產生離心的關內羌胡。苻生於是派了衞大將軍苻黃眉等率兵抵抗,最終大敗敵軍,更殺姚襄,令姚襄弟姚萇率眾歸降。苻黃眉立了大功,但凱旋後卻沒有獲得苻生褒賞,反而被多次當眾侮辱。苻黃眉因而憤怒,圖謀殺死苻生,但風聲洩露,反被殺,更株連不少王公親戚。

而當時御史中丞梁平老等人都勸有時譽的苻生堂弟、東海王苻堅殺苻生以救國,苻堅同意但不敢發難。但六月有一晚,苻生對侍婢表示翌日就要殺苻法、苻堅兩兄弟,侍婢於是立刻告訴二人,於是二人與強汪、梁平老和呂婆樓等都率兵衝入宮,宮中宿衞將士知道苻堅奪位都向其投降。苻生當時仍然在酒醉中,知有人攻來,就大驚,問侍從:「那是甚麼人?」侍從答:「是賊!」苻生就說:「為甚麼不下拜!」苻堅兵眾聽後大笑,苻生更說:「還不快快下拜,不拜的我就斬了他!」苻堅於是廢苻生為越王,自己繼承帝位,並降稱天王。不久,苻生被苻堅杀害,享年二十三歲,諡為厲王。儿子苻馗被封为越侯。

苻生无后。苻坚后来平定苻生弟苻廋等人叛乱,赐苻廋死,赦免苻廋诸子,并安排苻廋的儿子过继苻生为后。

苻洪:「此兒狂悖勃,宜早除之,不然,長大必破人家。」

薛讚、權翼:「主上猜忍暴虐,中外離心。」

《晉書》史臣曰:「長生慘虐,稟自率由。覩辰象之災,謂法星之夜飲;忍生靈之命,疑猛獸之朝飢。但肆毒於刑殘,曾無心於戒懼。招亂速禍,不亦宜乎!」

《晉書》贊曰:「長生昏虐,敗不旋踵。」

但有些記述表示所謂苻生暴虐也可能是史臣渲染的結果。楊衒之《洛陽伽藍記》卷二記載隱士趙逸之言,云:「國滅之後,觀其史書,皆非實錄,莫不推過於人,引善自向」,如「苻生雖好勇嗜酒,亦仁而不殺。觀其治典,未為凶暴,及詳其史,天下之惡皆歸焉。苻堅自是賢主,然賊君取位,妄書君惡,凡諸史官,皆是類也。」劉知幾《史通》曲筆篇云:「昔秦人不死,驗苻生之厚誣」,即是據此。

呂思勉亦懷疑苻生的一系列殘忍殺人、文詞避諱、以刀刃錘斧威懾群臣等都是史官誣陷、醜化苻生的結果。誅殺梁安、雷弱兒等人亦因他們其實是有通晉的嫌疑,是不得已,卻招來謗毀之聲。稱「他如怠荒、淫穢,自更易誣。《金史·海陵本紀》述其不德之亂,連章累牘,而篇末著論,即明言其不足信,正同一律。」

\subsubsection{寿光}

\begin{longtable}{|>{\centering\scriptsize}m{2em}|>{\centering\scriptsize}m{1.3em}|>{\centering}m{8.8em}|}
  % \caption{秦王政}\
  \toprule
  \SimHei \normalsize 年数 & \SimHei \scriptsize 公元 & \SimHei 大事件 \tabularnewline
  % \midrule
  \endfirsthead
  \toprule
  \SimHei \normalsize 年数 & \SimHei \scriptsize 公元 & \SimHei 大事件 \tabularnewline
  \midrule
  \endhead
  \midrule
  元年 & 355 & \tabularnewline\hline
  二年 & 356 & \tabularnewline\hline
  三年 & 357 & \tabularnewline
  \bottomrule
\end{longtable}


%%% Local Variables:
%%% mode: latex
%%% TeX-engine: xetex
%%% TeX-master: "../../Main"
%%% End:

%% -*- coding: utf-8 -*-
%% Time-stamp: <Chen Wang: 2019-12-19 10:13:46>

\subsection{宣昭帝\tiny(357-385)}

\subsubsection{文桓帝生平}

苻雄(4世纪?-354年7月26日),字元才,略陽臨渭(今甘肅秦安)氐族人。十六国時前秦的开国元勋、宗室。苻洪之幼子,景明帝苻健之弟,屡建军功,官至丞相,曾參與抵抗東晉將領桓溫發動的北伐戰爭。

苻雄年輕就已熟讀兵書、富有謀略,亦擅長射馬射箭;此外亦有為政治術,慷慨施予地位不高但有才德的人。因著父親苻洪在後趙滅前趙後歸附後趙並接受其官職,苻雄亦仕於後趙,更因戰功而獲後趙君主石虎授予龍驤將軍。

永和五年(349年),石虎去世,諸子爭位令國內漸亂,而苻洪亦因遭當時後趙皇帝石遵削職而叛投東晉。次年,後趙將領麻秋東歸鄴城,苻雄受父命領兵迎擊,成功俘獲麻秋,並以其為軍師將軍。但不久麻秋就藉宴會而以毒酒毒殺苻洪,意圖盡收苻氏部眾。苻健殺麻秋後接掌苻洪部眾,並順從父親遺言,進據關中。

當時關中為自稱晉臣的杜洪所控制,苻健因應人心思晉,於是稱東晉早前加予的官爵,並以苻雄為輔國將軍。不久苻健正式出兵關中,苻雄在大軍渡過黃河後就受命率五千兵取道潼關進攻長安,作為苻健的前驅。苻雄在潼關以北擊敗杜洪派去抵抗的張先,及後苻雄北巡渭北,所過的城邑都向其歸降。苻健進據長安後向東晉獻捷,更得秦、雍二州的胡族及漢人歸附,苻雄亦攻陷後趙涼州刺史石寧據守的上邽,斬殺石寧,鞏固苻氏在關中的統治。

皇始元年(351年),苻健稱天王、大單于,正式建立前秦,並封苻雄為東海公,以其為都督中外諸軍事、丞相、領車騎大將軍、雍州牧。次年,以苻雄為首的百官上請苻健稱帝,苻雄於是進封東海王。

同年,東晉安西將軍謝尚因不能安撫歸附的後趙豫州刺史張遇而令其叛變,謝尚於是與姚襄攻伐張遇,而苻雄就奉命與苻菁出兵略地關東,並救援張遇。最終苻雄在潁水的誡橋擊敗晉軍,逼其退還淮南,並略陳郡、穎川、許昌及洛陽附近共五萬多戶及張遇回軍關中。

同年,苻雄又在隴西擊敗後趙將領王擢,令其逃奔前涼。皇始三年(353年)二月,王擢會同前涼軍隊伐秦,苻雄又率兵擊敗王擢等。但隨著秦州刺史苻願及將軍苻飛分別敗於王擢及楊初,苻雄等於是回屯隴東,而不久張遇更在聯結關中豪族反叛,雖然張還事敗被殺,但孔特、、劉珍、夏侯顯、喬秉、胡陽赤及呼延毒就各自據城起兵叛秦。苻雄於是回軍長安並與苻法、苻飛等分兵平定孔特等人。

皇始四年(354年),就在苻雄攻下胡陽赤據守的司竹,張遇謀反引發的數個主要叛亂勢力僅剩下呼延毒及喬秉時,東晉征西大將軍桓溫發動北伐,自荊州進攻前秦,另由司馬勳率偏師由梁州北上關中。苻健於是派苻雄、太子苻萇等人共率五萬軍抵抗,但大軍在藍田縣被桓溫率領的主力擊敗,苻雄亦在白鹿原敗於桓沖,於是與苻萇等退守長安城南,與雷弱兒所率的三萬精兵共同抵抗。不過桓溫當時只駐屯長安東南的灞上,未有進逼長安,苻雄此時就率領七千騎兵突襲時正經子午谷入關中的司馬勳軍,令其敗退至女媧堡。苻雄及後返回長安,於白鹿原擊敗桓溫,而桓溫亦因乏糧而在六月被逼退兵,呼延毒亦跟隨桓溫南走。苻雄接著討伐佔領了陳倉的王擢及司馬勳,令兩人分別敗退回漢中及略陽,成功解除了桓溫這次北伐帶來的危機。

六月丙申日(7月26日),苻雄在進攻喬秉據守的雍城時去世,苻健聞訊悲傷得嘔血,說:「上天不讓我平定四海麼!為何這麼快就奪去我的元才呀?」追贈魏王,賜諡號敬武,葬禮依西晉安平獻王司馬孚的先例。

其子苻堅後來即位稱天王,追尊苻雄為文桓皇帝。

有載苻雄「醜形貌,頭大而足短」,並沒有雄偉的形象,在後趙任龍驤將軍時更被人稱為「大頭龍驤」。
苻雄作為前秦開國元勳,亦是君主的親弟弟,位至丞相,位高權重,但為人謙虛恭順,遵奉法度,加上在政事和軍事都有才能,故深受苻健倚重,更稱:「元才,是我的姬旦。」

\subsubsection{宣昭帝生平}

秦宣昭帝苻坚(338年-385年10月16日),字永固,一名文玉,略阳临渭(今甘肃秦安)人,氐族,苻雄之子,苻洪之孫,苻健之侄,是十六国时期前秦的君主,称大秦天王。

初封東海王,後發動政變推翻堂兄苻生而即位,在位期間重用漢人王猛,亦推行一系列政策與民休息,加強生產,終令國家強盛,接著以軍事力量消滅北方多個獨立政權,成功統一北方,並攻佔了東晉領有的蜀地,與東晉南北對峙。苻堅於383年發動戰爭意圖消滅東晉,史稱淝水之戰,但最終敗給東晉謝安、謝玄領導的北府兵,國家亦陷入混亂,各民族紛紛叛變獨立,苻堅最終亦遭羌人姚萇殺害,谥号宣昭,庙号世祖。

苻堅的母親曾夢見受到神靈寵幸而懷孕,足足懷了十二個月才生下苻堅,苻堅生於正月初二日,年僅七歲就已顯現其聰敏善良的特質,且舉止都循規蹈矩。又因妥貼地侍奉祖父苻洪,不需詢問就能猜到祖父的想要取甚麼,並及時為祖父拿來,故此深得苻洪疼愛。八歲時,苻堅就請求苻洪請老師到家教他學習,苻洪見他熱心求學十分高興,欣然同意。

皇始四年(354年),苻雄去世,苻堅承襲父爵東海王。另苻堅亦獲授龍驤將軍,苻健更以苻洪曾經在後趙獲授此號勉勵苻堅, 苻堅當時亦「揮劍捶馬」,被苻健的話所感動和激勵,士卒見此,亦心服苻堅。苻堅當時亦博學多才藝,更有經略大志,廣交豪傑,結交了呂婆樓、強汪、梁平老及王猛等人,都成為其左右手。

壽光三年(357年),姚襄謀圖關中,並聯結前秦境內的羌人,苻堅與苻黃眉、鄧羌等人率兵抵抗,終在鄧羌成功誘使姚襄出擊而由苻黃眉率主力將姚襄擊敗,並擒殺姚襄,逼令姚襄弟姚萇率其部眾歸降前秦。然而,當時前秦皇帝苻生賞罰失當,兇殘好殺,苻黃眉因立大功後未受褒賞,反受侮辱而謀反。雖然最終苻黃眉謀反失敗,但苻堅當時很有聲譽,姚襄舊將薛讚和權翼亦欣賞苻堅的才能,並勸苻堅學湯武伐昏君奪帝位 ;梁平老等人亦勸苻堅謀反。同年,苻堅與其兄苻法得知苻生有意加害,於是先發制人,入宮罷黜苻生,不久更殺死苻生。苻堅將帝位讓給苻法,但苻法自以庶出不敢受。苻坚在群臣的勸進下即位,並降號天王,稱大秦天王。即位後苻堅先誅除苻生寵信的董榮等人,隨後擢用李威、呂婆樓、王猛、權翼、薛讚等人。又追復被苻生所誅殺的八個顧命大臣的官位,隨才選用其子孫為官。

苻堅即位後,亦修整一些名實不符的官職,恢復已絕的宗祀,上禮神祇,鼓勵農業,設立學校,扶持鰥寡孤獨和年老無依者。另亦褒揚稱頌一些有特殊才行、孝友忠義、有德業的人。後苻堅下令各地方官員都上舉孝悌、廉直、文學、政事四項才德的人才,若真的是人才就得賞賜,否則就被降罪;另苻堅亦不優待宗室,即使是宗室中人,若無才幹都會被棄用,於是當時國內官員都十分稱職。而通過開墾耕地,令前秦倉庫充實,人民溫飽而令盜賊也少了。

苻堅亦下令與民休息,在即位次年(358年)討平於并州叛變的張平後就下令偃甲息兵,直至365年出兵平定劉衞辰及曹轂的叛亂前都沒有大型的軍事行動。苻堅又應當時秋旱而下令減省膳食和暫停奏樂,將金玉錦繡等貴重物品散發給軍士,並命後宮省儉服飾。苻堅更開發山澤,且得出的資源不限於官府,連平民也可用。

至甘露六年(364年),苻堅下令各公國自置中尉、大農及其他官屬,然而眾人卻以當時富商趙掇、丁妃等人車服盛如王侯,紛紛延攬這些富商為二卿。苻堅於是下令延攬富商為卿者降爵位到侯爵,並下令沒爵位或官職的人都不能在都城百里以內乘車馬;工商、奴隸及婦人亦不得穿戴金銀錦繡,違者處死。

另一方面,苻堅重用漢人王猛,機要之事王猛幾乎無不知道,這令一眾氐族豪族及元勳十分不滿。其中特進樊世自恃是氐族豪族,且有大功勳,當眾直斥王猛竊取為前秦立下赫赫功勳的功臣之成果。苻堅知道後,決意殺樊世以威懾所有氐族豪族。樊世死後,各氐人都爭相批評王猛,苻堅更為王猛而謾罵和鞭撻大臣,終令氐人都畏懼王猛,壓制了氐族豪強對王猛新政的反抗力量。而王猛於359年捕殺酗酒橫行,掠貨擄人的強太后弟強德,苻堅想下令赦免亦趕不及,後來不但沒有問罪王猛,更讓王猛在數十日內處罰了二十多個權豪貴戚,其嚴正執法亦為苻堅所允許,亦為苻堅所認同。

甘露六年(364年),汝南公苻騰謀反被殺,當時王猛以苻生諸弟尚有五人,建議苻堅除去五人,否則終會為患,然而苻堅不聽。至次年,因著劉衞辰及曹轂的叛亂,苻堅親自率軍出征平定,並北巡朔方以撫諸胡。時為征北將軍的苻幼趁機領兵進攻當時由太子苻宏、王猛及李威留守的首都長安,只因李威領兵擊斬苻幼而平定亂事。

苻幼起事時其實還暗中聯結了征東大將軍、并州牧、晉公苻柳以及征西大將軍、秦州刺史、趙公苻雙,但苻堅以二人分別為伯父苻健愛子及同母弟弟而不問罪,亦不將此事公布。然而,二人卻與時為鎮東將軍、洛州刺史的魏公苻廋及安西將軍、雍州刺史的燕公苻武共謀作亂。苻堅得知,於是召眾人到長安,但四人就在建元三年(367年)十月各據州治起兵反叛,苻堅試圖勸其罷兵,答應一切如故,不作追究,並以齧棃 為信物,但四人都沒有任何動搖。次年正月,苻堅正式派軍鎮壓叛亂,派楊成世、毛嵩、鄧羌、王猛、張蚝等人分途出兵,分別進攻四地。但當時楊成世及毛嵩都分別敗於苻雙和苻武的叛軍,逼使苻堅再將王鑒、呂光等人率兵再攻。最終王鑒、呂光及王猛等先後擊敗並斬殺四公,才令亂事成功於當年平定。而在苻堅進攻苻廋時,苻廋主動獻州治陝城(今河南陝縣)歸降前燕,並請兵接應。此舉震動前秦,更逼使苻堅派大軍至華陰(今陝西華陰)防備,只因前燕太傅慕容評拒絕迎降,才避免了更大的危機。

建元五年(369年),前燕吳王慕容垂在擊退東晉桓溫的北伐軍後因受到慕容評排擠,於是出奔降秦。苻堅早於兩年前知道慕容恪去世的消息時就已經有吞併前燕的計劃,還特地派了使者出使前燕以探虛實,然而苻堅因為慕容垂的威名而不敢出兵。現在慕容垂自來,苻堅十分高興,並親自出郊迎接,對其極為禮待,更以其為冠軍將軍,不顧王猛要他提防慕容垂的諫言。

同年十二月,苻堅以前燕違背當日請兵的諾言,不割讓虎牢(今河南滎陽汜水縣西北)以西土地予前秦為藉口出兵前燕,以王猛、梁成和鄧羌率軍,進攻洛陽(今河南洛陽市),並於次年年初攻下。六月,苻堅再命王猛等出兵前燕,自己更親自送行。王猛終在潞川擊潰率領三十多萬 大軍的前燕太傅慕容評,並乘勝直取前燕首都鄴城(今河北臨漳縣西南),苻堅更在王猛圍攻鄴城時親自率軍前往鄴城助戰。拿下鄴城後,正出奔遼東的前燕皇帝慕容暐被前秦追兵生擒,前燕在遼東的殘餘反抗力量亦遭消滅,前秦正式吞併前燕。

另一方面,369年,東晉將領袁真在桓溫北伐失敗後因被桓溫委以戰爭失利的罪責,憤而據壽春(今安徽壽縣)叛變,聯結前燕及前秦。 袁真不久去世,但其子袁瑾仍然堅守壽春,並在前燕亡後繼續向苻堅求救。苻堅於是於371年派王鑒及張蚝救援,但圍城的桓溫派將領桓伊等擊敗王鑒等,逼其退屯慎城(今安徽穎上縣),不久壽春被晉軍攻陷。

在前秦吞併前燕,收撫前燕領土的同一年,名義上臣服於前秦的仇池公楊世死,其子楊纂襲位後只受東晉朝命,斷絕與前秦的臣屬關係,苻堅遂在次年(371年)派兵進攻仇池。當時楊纂叔父楊統正與楊纂兵戎相見,東晉梁州刺史楊亮知道前秦進攻後亦派了郭寶等領兵協助楊纂,然而最終楊纂軍大敗,在仇池兵臨城下及楊統率眾降秦之下,楊纂只得出降。此後前秦命參與進攻仇池的將領楊安鎮守仇池。前仇池至此滅亡。當時苻堅有意在河西樹立威信,以德懷民,於是盡釋早前俘獲的前涼將領陰據及其所統五千兵士,前涼君主張天錫在佑道前仇池被前秦攻滅後甚為畏懼,至此就被逼向前秦稱藩。吐谷渾君主碎奚亦因前仇池滅亡而遺使向前秦進貢,苻堅亦授予其官職爵位。另外,苻堅又出兵攻伐隴西鮮卑首領乞伏司繁,盡降其眾,苻堅留乞伏司繁在長安,只由其堂叔乞伏吐雷統眾。

建元九年(373年),東晉梁州刺史楊亮派其子楊廣進攻仇池。但楊廣敗於仇池守將楊安,原先駐守沮水防備前秦的各軍戍更因而棄守潰逃,逼使楊亮退守磬險。而楊安亦趁機進攻東晉,進攻漢川。不久,苻堅更命益州刺史王統領攻漢川,毛當等攻劍門(今四川劍閣東北),大舉進攻東晉梁、益二州。楊亮在青谷率巴獠抵抗但失敗,只得退保西城(今陝西安康西北),結果漢中(今陝西漢中)、劍閣(今四川劍閣)、梓潼(今四川梓潼縣)等地先後失陷。東晉益州刺史周仲孫在緜竹(今四川綿竹縣)要抵抗來侵的朱肜部時,另一邊的毛當已經快攻到益州治所成都(今四川成都),周仲孫唯有逃到南中,於是前秦攻下了益、梁二州。

次年,益州發生叛亂,蜀人張育、楊光起兵反抗前秦,並向東晉稱藩,而巴獠酋帥張重、尹萬等亦參與,苻堅於是命鄧羌入蜀鎮壓;同一時間,東晉益州刺史竺瑤及威遠將軍桓石虔則受命入蜀,進攻墊江(今重慶墊江縣)。當時張育等人圍攻成都,但期間他們內訌爭權,終被鄧羌等人擊敗,叛亂被平定。竺瑤和桓石虔雖於墊江擊敗寧州刺史姚萇,但不能擴大戰事,只得退還巴東,前秦始終固守了蜀地。

建元十二年(376年),苻堅以張天錫「雖稱藩受位,然臣道未純」為由出兵十三萬進攻前涼。當時苻堅亦派閻負和梁殊出使前涼,徵召張天錫到長安,然而張天錫不願投降,決意與前秦決一死戰,下令斬殺二人,並派馬建抵抗前秦。隨著前秦軍西渡黃河,攻下纏縮城(今甘肅永登縣南),張天錫更派掌據到洪池(今甘肅天祝縣西北烏鞘嶺)協同馬建作戰,自己更親自率兵到金昌助戰。然而,前秦軍進攻二人時,馬建竟向前秦投降而掌據戰死,張天錫驚懼而退還都城姑臧(今甘肅武威)。前秦軍接著直攻姑臧,張天錫被逼出降,前涼至此滅亡。

隨著先後攻滅前燕、前仇池和前涼三個割據政權,北方唯一的割據政權就是拓跋氏建立的代國。在滅前涼的同一年,苻堅以應劉衞辰求救為由,命幽州刺史苻洛率兵十萬,另派鄧羌等率兵二十萬,一起北征代國。當時代王拓跋什翼犍先後命白部、獨狐部及南部大人劉庫仁抵禦,但都失敗,而什翼犍因病而不能率兵,被逼北走陰山,但高車部族此時卻叛變,什翼犍只得回到漠南,並看準前秦軍稍退,於是返回雲中郡盛樂(今內蒙古和林格爾北)的都城。此時,拓跋斤挑撥什翼犍子拓跋寔君,令其起兵殺死父親及其他弟弟;前秦軍聞訊亦立刻出兵雲中,代國於是崩潰,為前秦所滅。

苻堅隨後殺死拓跋斤及拓跋寔君,拓跋窟咄被強遷至長安,而什翼犍諸子亦被殺,什翼犍孫拓跋珪尚幼,再無於當地有效控制代國統下諸部的人。苻堅因而聽從燕鳳的話,分別以劉庫仁及劉衞辰分統代國諸部,借兩人之間的矛盾互相制衡。至此,前秦成功統一北方,只剩下據有江南地區的東晉。

建元十四年(378年),苻堅派苻丕等人進攻襄陽(今湖北襄陽市),另分一路由慕容垂、姚萇率領的軍隊經武當,配合苻丕進攻襄陽。數月後,兗州刺史彭超請求進攻彭城(今江蘇徐州市),並上言請派重將出兵淮南,與進攻襄陽的苻丕配合,形成東西並進之勢,最終消滅東晉。苻堅同意並派了俱難、毛盛等人進攻淮陰(今江蘇淮陰)、盱眙(今江蘇盱眙縣東北),由彭超都督東討諸軍事。

進攻襄陽的軍隊因著守將朱序堅守以及苟萇意圖孤立襄陽而逼其自降的戰略,一直與晉軍相持至年末。此事令苻丕等遭到彈劾,苻堅亦下令要求苻丕最遲在明年春季就要取勝。苻丕於是轉而急攻,終於在次年正月攻下襄陽。另一方面,晉兗州刺史謝玄於建元十五年(379年)奉命救援彭城,最終雖然護送城內的晉軍和沛郡太守戴逯離開,但彭城仍被前秦攻下,及後秦軍亦先後攻下盱眙和淮陰,並在三阿(今江蘇寶應)圍困晉幽州刺史田洛,威脅東晉江北重鎮廣陵(今江蘇揚州市)。此時,晉軍發動反擊,成功擊敗圍攻三阿的俱難、彭超等,逼他們退屯盱眙;次月二人再失盱眙,退保淮陰,但晉軍水軍當時乘潮北上,焚毀秦軍建在淮河上的橋,並擊敗俱難等,逼其退還淮北。而面對謝玄等的追擊,二人終在君川(今江蘇盱眙縣北)大敗給晉軍。面對東線的大敗,苻堅大怒並收捕彭超,嚇得彭超自殺,又將俱難貶為庶民。

就在建元十四年(378年)東西二線南攻東晉之時,鎮守洛陽的北海公苻重謀反,不過很快就因苻重長史呂光忠於苻堅而被平定,苻重獲赦而返回府第。至建元十六年(380年),苻堅卻再度命苻重為鎮北大將軍,駐鎮薊(今北京)。同年,苻堅亦命行唐公苻洛為征南大將軍,鎮守成都,並命其由襄陽循漢水西上上任。但其實苻洛在立下滅亡代國的大功後因為沒有獲苻堅封為將相重臣,反倒仍以其作為邊境州牧深感不滿,更懷疑命他到襄陽其實是苻堅殺他的陰謀,於是決定叛變。當時雖然只有苻重支持苻洛,但苻洛仍自和龍(今遼寧錦州)率兵七萬直指長安。關中人民恐懼戰亂,人心騷動,盜賊興起,苻堅試圖勸降,於是以永封幽州請苻洛罷兵。然而苻洛拒絕,並聲言要「還王咸陽,以承高祖之業」,更反說若苻堅在潼關候駕,他會以他為上公,還爵東海。苻堅於是大怒,出兵討伐,並在中山與苻重及苻洛的十萬聯軍會戰,終生擒苻洛並斬殺苻重,平定亂事。

事後,苻堅認為關東地區地廣人多,於是決定從原居於三原(今陝西三原縣)、九嵕(今陝西乾縣東北)、武都(今甘肅成縣西)、汧(今陝西鳳翔縣南)、雍(今陝西鳳翔縣南)的氐族人中分出十五萬戶,由各宗室統領分布於各方鎮,如古時諸侯一般。不過,被遷移居方鎮的人們因為要與家人分別,都哀傷號哭,路人看見都感到傷心。

王猛於建元十一年(375年)去世,臨死時說:「晉室現在雖然立於偏遠的江南地區,但承繼正統。現在國家最寶貴的就是親近仁德之人以及與鄰國友好。臣死以後,希望不要對東晉有所圖謀。鮮卑、羌虜都是我們的仇敵,終會成為禍患,應該將他們除去,以利社稷。」 希望苻堅先解決國內鮮卑和羌族等其他少數民族對前秦政權的暗藏問題。不過,苻堅在統一北方後仍未聽從王猛之言,著力解決國內民族問題。

苻堅從車師前部王彌窴及鄯善國王休密馱等處聽說西域有高僧鳩摩羅什,苻堅視為國寶,請求西域派羅什入秦遭到拒絕。建元十八年(382年),苻堅派呂光領七萬大軍征伐西域不服前秦要求的,並於次年正月出發。呂光征伐西域龜茲等國大獲全勝,西域諸國歸附前秦。鳩摩羅什也被呂光攜帶身邊。中國境內只剩東晉一地不是前秦版圖,苻堅急於統一中國,開始謀劃出兵東晉。

建元十八年(382年),苻堅大會群臣,自以能得九十七萬兵力,提出親征東晉,統一全國的計劃。當時秘書監朱肜表示支持,尚書左僕射權翼及太子左衞率石越卻都以東晉君臣和睦,且當時為重臣的謝安及桓沖都是人才,皆予以反對。而當時群臣亦各有意見,未有共識。苻堅見此,就說:「像在道旁建房子去問意見,就因聽太多不同的議論而一事無成,我心中自有決斷。」群臣退下後,苻堅留下其弟苻融繼續和他討論,然而苻融亦以天象不利、晉室上下和睦以及兵疲將倦三點為由反對。苻堅因而大怒,苻融後哭著勸諫,並重提王猛死前的話也未能說動苻堅。後名僧釋道安、太子苻宏、幼子中山公苻詵以至寵妃張夫人皆反對伐晉,苻融等人亦屢次上書表示反對,苻堅仍然不肯放棄出兵東晉的計劃,可見苻堅當時其實下了決心。相反,慕容垂向苻堅表示支持出兵東晉,苻堅聽後十分高興,於是向慕容垂說:「與我平定天下的人,就只有你一個呀。」更賜其五百匹布帛。

建元十九年(383年)五月,東晉荊州刺史桓沖出兵襄陽、沔北及蜀地。桓沖於七月退軍後,苻堅便下令大舉出兵東晉,每十丁就遣一人為兵;二十歲以下的良家子但凡有武藝、驍勇、富有、有雄材都拜為羽林郎,最終召得三萬多人。八月,苻堅命苻融率張蚝、梁成和慕容垂等以二十五萬步騎兵作為前鋒,自己則隨後自長安發兵,率領六十餘萬戎卒及二十七萬騎兵的主力,大軍旗鼓相望,前後千里。十月,苻融攻陷壽陽(今安徽壽縣),並以梁成率五萬兵駐守洛澗,阻止率領晉軍主力的謝石和謝玄等人的進攻。當時正進攻晉將胡彬的苻融捕獲胡彬的所派去聯絡謝石的使者,得知胡彬糧盡乏援的困境,於是派使者向正率大軍在項城的苻堅聯絡:「晉軍兵少易擒,但就怕他們會逃走,應該快快進攻他們。」苻堅於是留下大軍,秘密自率八千輕騎直抵壽陽 。然而,晉將劉牢之及後率軍進攻洛澗,擊殺梁成,前秦軍隊潰敗,謝石等於是率領大軍水陸並進,與前秦軍隔淝水對峙。苻堅和苻融此時從壽陽城觀察晉軍,見其軍容整齊,連八公山上的草木都以為是晉軍,於是說:「這也是勁敵,怎能說他們弱呀!」由此悵然失意並有懼色。苻堅及後答允晉軍要他們稍為後撤,讓晉軍渡過淝水作戰的要求,並認為能待晉軍半渡、陷於河中之時出擊,便能將其一舉擊潰。但當前秦大軍開始後退時,先前於襄陽被擒投誠的降將朱序大叫「秦兵敗矣」,秦軍頓時軍心大亂而潰散。苻融親自騎馬入陣中試圖重整亂軍,但反而墮馬被踩死,晉軍於是追擊潰敗的前秦軍,令前秦軍傷亡慘重,連苻堅本人亦中流矢受傷,單騎逃到淮北。

苻堅敗退到淮北時十分飢餓,有平民送他飯菜,苻堅於是給予賞賜,然而該平民卻拒絕,更稱苻堅自取厄困,自己身為其子民即為其子,不圖回報。苻堅因而大感慚愧。及後苻堅與慕容垂的三萬軍隊會合,隨後一直沿途收集逃散的敗兵,到洛陽時聚集了十餘萬人,百官、儀物和軍容都大致齊備了。後苻堅返回長安,哭悼苻融並告罪宗廟後下令大赦,下令鍛煉兵器並監督農務,撫順孤老及陣亡士兵的家屬,試圖重建國家秩序。

隴西鮮卑的乞伏步頹在苻堅出兵東晉時乘機反叛,苻堅派乞伏步頹的侄兒、原降於前秦的乞伏司繁子乞伏國仁出兵討伐,但二人卻相結。淝水戰敗後,乞伏國仁於是裹脅隴西鮮卑諸部叛變,後建立起西秦。而苻堅在洛陽時,不顧權翼的反對,答允讓慕容垂到河北地區安撫民眾及拜謁慕容氏宗廟陵墓。然而慕容垂後來則乘被當時駐鎮鄴城的長樂公苻丕派往鎮壓丁零人翟斌叛亂的機會,聯結丁零人叛秦,並於建元二十年(384年)反與丁零人圍攻鄴城,建立後燕。在圍攻鄴城的同年,慕容泓知道慕容垂的行動亦在關東收集部眾自立,甚為強盛;慕容沖亦在平陽叛變,後投奔慕容泓,慕容泓於是建立西燕,並率眾進攻長安。

為征討大舉叛變的慕容鮮卑,苻堅徵召鉅鹿公苻叡,令其與竇衝及姚萇同討慕容泓,但最終苻叡兵敗戰死,姚萇遣使謝罪卻因苻堅殺其使者而逃到渭北牧馬場,乘機煽動羌族豪帥共五萬餘家歸附,自稱秦王,建立後秦。苻堅自率二萬步騎討伐後秦軍,屢敗後秦軍,更逼得後秦軍中缺水,更有人渴死,但此時天降大雨,後秦軍隊再起,隨後更反敗前秦軍隊。苻堅見慕容沖等已逼近長安,於是回軍長安並組織抵抗,但所派的苻琳、姜宇都兵敗,慕容沖成功佔領阿房城(今陝西西安市西),長安遭圍困。建元二十一年(385年),苻堅在長安宴請群臣,但當時長安已鬧饑荒,發生人食人的事,諸將回家後都吐出宴中吃下的肉來餵饑餓的妻兒。隨後前秦與西燕軍互相攻伐,互有勝負,但在衞將軍楊定被西燕所俘後,苻堅大懼,竟相信他曾經下令禁止的讖諱之言,留太子苻宏留守長安,自己率數百騎及張夫人、苻詵和苻寶、苻錦兩名女兒一同出奔五將山。然而苻堅到五將山後,後秦將領吳忠就來圍攻。苻堅雖見身邊的前秦軍都潰散,但亦神色自若,坐著安然等待吳忠。吳忠及後將苻堅送至新平幽禁。

姚萇及後向苻堅索要傳國玉璽,苻堅張目喝道:「小小羌胡竟敢逼迫天子,五胡的曆數次序,沒有你這個羌人的名字。玉璽已送到晉朝那裏,你得不到的了!」姚萇於是又派人提出苻堅禪讓給他,苻堅亦說:「禪代,是聖賢的事,姚萇是叛賊,有甚麼資格做這事!」苻堅自以平生都待姚萇不薄,甚至在淝水之戰前將「龍驤將軍」這個祖父曾受以及自己殺苻生奪位時有的將軍號珍而重之地封予姚萇,現在姚萇反叛並逼迫他,於是屢次責罵姚萇以求死,並為免姚萇凌辱兩名女兒,於是先殺苻寶和苻錦。八月辛丑日(10月16日),姚萇命人將苻堅絞死於新平佛寺(今彬縣南靜光寺)內,享年四十八歲。張夫人及苻詵亦跟著自殺。

姚萇為掩飾他殺死苻堅的事,故意諡苻堅為壯烈天王。而苻堅去世同年,苻丕得知其死訊,便即位為帝,諡苻堅為宣昭皇帝,上廟號世祖。征西域後回到涼州的呂光得知苻堅去世,亦諡其為文昭皇帝。

苻坚死後就地埋葬,當地人稱“長角塚”。許多人民尊其為苻王爺奉祀之,謂能避免疫病、兵亂。根據《晉書》記載,姚萇被苻堅冤魂作祟,終至發狂,武士欲去救援,竟然打傷其陰部,大出血而死。萇死前還一直跪地叩首,請求苻堅原諒他。

苻堅除了一系列減省奢侈品、鼓勵農業、停止征戰外,更建立學校,重視文教,尤其留心儒學。苻堅曾下令廣收學官,重視經學,郡國弟子員只要通曉一經或以上就獲授官,亦表彰有才德和努力營田之人,令人們都望得朝廷勸勵,崇尚清廉正直,物資亦豐盛。苻堅更每月親臨太學考拔學生,消滅前燕後更在長安祭祀孔子。而王猛亦助苻堅整順風俗,令全國學校漸興。在苻堅治下的關隴地區豐盛安定,地區回復秩序,工商業興盛,一片繁華景象。及至後來王猛去世後,苻堅仍然尊崇儒學,不但命太子、公侯和官員之子以及中外四禁 、二衞、四軍長上 的將士都要受學,連帶後宮亦設有典學,教宮內宦官及宮婢經學。另亦嚴厲禁止老莊以及圖讖學說。後來西域大宛獻馬,苻堅效法西漢漢文帝送還進貢的千里馬,更加命群臣作《止馬詩》送到西域,以示沒有取千里馬的欲望。最終共有四百多人獻詩。

苻堅亦重視生產,遇上天旱不但曾下令節儉及開山澤資源與民共享,亦督導百姓耕種,自己更親身躬耕藉田,讓苟皇后親身養蠶,以示對農業的重視。後又徵集王侯以下及豪門富戶的家僮奴僕共三萬人開通涇水上流,引水灌溉解決關中水旱問題。

苻堅對於前秦這個多民族組成的國家其實沒有作出民族融合的措施。如隴西鮮卑首領乞伏司繁投降後,只遷乞伏司繁到長安,仍留其部眾在隴西地區;前燕鮮卑族人除了慕容氏皇族及部分關東豪族被遷至關中地區外,尚有大部分留在前燕故地,另亦遷原居中山的丁零族人到新安(今河南新安縣);消滅代國後,苻堅雖然由北方匈奴人代領代國遺眾,但仍居北方。在苻洛叛亂被平定後,苻堅則為更好的管理關東以至各地民族,於是從原集中於關中的氐族人分出十五萬戶,各由宗親率領出鎮,如古分封諸侯般管治地方 。然而此舉卻分散了氐族的民族力量,影響對各地的軍事影響力,而移居關中的各少數民族更成前秦的心腹大患 。

史載苻堅「臂垂過膝,目有紫光」。

苻堅與苻法兄弟友好,然而在苻堅即位之初其母苟氏以苻法年長、賢能以及得人心而殺害苻法,苻堅無奈下只有哭著與他訣別,傷心得吐血。後來其子苻陽因憤恨父親無罪遭戮,而謀反,苻堅亦不誅殺。

苻堅寛貸容人,如後趙舊將張平在秦、燕之間搖擺,維持半獨立狀態 ,357年更以并州叛秦,但仍然加以寛貸,署為右將軍。後苻重在洛陽叛變,苻堅也赦而不誅,後更再派他出鎮,終招來苻重聯同苻洛再叛;而苻洛敗後苻堅仍不殺,只流放他到西海郡。另苻堅亦善待亡國貴族,如前涼張氏、前燕慕容氏等都沒有進行屠殺,甚至頗見親待。

苻堅執政前期大推善政,崇尚節儉,然而在王猛死後,苻堅卻因聽後趙前將作功曹熊邈講述後趙宮室器具的規模,下令以其為將作長史,大修舟艦、兵器,並以金銀裝飾,講求精巧,一改之前節儉之風。慕容農亦因而說:「自從王猛死後,秦的法制日漸頹靡,今日又著重奢侈,大禍將來了。」

苻堅初年虛心接納臣下的諫言,如即位初期曾經登龍門,向群臣展現他甚為滿足於關中的穩固。而權翼、薛讚當時則以夏、商、周、秦四個朝代由興盛的基礎而到最終遭他人所滅,表達出修備德行的重要,穩固的地勢並不足以固國。苻堅聽後大喜,隨後就施行一系列新政與民休息。後苻堅在鄴附近狩獵十多日,樂而忘返,亦聽從伶人王洛的勸言,不再出獵。但後來苻堅卻在出兵東晉等事上聽不下諫言,只想聽到支持自己的論調。

苻堅生母因為年輕守寡,於是寵幸將軍李威,當時史官亦記載此事。但苻堅後來看起居注和史官所著的著作發現載有這種事,於是發怒燒書並大檢史官,要加罪於史官,因著作郎趙淵、車敬等已死才了事。

《晉書》史臣曰:「永固雅量瓌姿,變夷從夏,叶魚龍之遙詠,挺莫苻之休徵,克翦姦回,纂承偽曆,遵明王之德教,闡先聖之儒風,撫育黎元,憂勤庶政。……乃平燕定蜀,擒代吞涼,跨三分之二,居九州之七,遐荒慕義,幽險宅心,因止馬而獻歌,託棲以成頌,因以功侔曩烈,豈直化洽當年!雖五胡之盛,莫之比也。既而足己夸世,複諫違謀,輕敵怒鄰,窮兵黷武。懟三正之未叶,恥五運之猶乖,傾率土之師,起滔天之寇,負其犬羊之力,肆其吞噬之能。自謂戰必勝,攻必取,便欲鳴鷥禹穴,駐蹕疑山,疏爵以侯楚材,築館以須歸命。曾鬥知人道助順,神理害盈,雖矜涿野之強,終致昆陽之敗。道使文渠候隙,狡寇伺間,步搖啟其禍先,燒當乘其亂極,宗社遷於他族,身首罄於賊臣,賊戒將來,取笑天下,豈不哀哉!豈不謬哉!」

《晉書》贊曰:「永固禎祥,肇自龍驤。垂旒負扆,竊帝圖王。患生縱敵,難起矜強。」

苻洪:「此兒姿貌瓖偉,質性過人,非常相也。」

徐統:「此兒有霸王之相。」又曰:「苻郎骨相不恒,後當大貴,但僕不見。」

薛禮、權翼:「非常人也!」

苻廋:「苻堅、王猛,皆人傑也。」

司馬光:「夫有功不賞,有罪不誅,雖堯、舜不能為治,況他人乎!秦王堅每得反者輒宥之,使其臣狃於為逆,行險徼幸,力屈被擒,猶不憂死,亂何自而息哉!《書》曰:『威克厥愛,允濟;愛克厥威,允罔功。』《詩》云:『毋縱詭隨,以謹罔極;式遏寇虐,無俾作慝。』今堅違之,能無亡乎!」又言:「論者皆以為秦王堅之亡,由不殺慕容垂、姚萇故也。臣獨以為不然。許劭謂魏武帝治世之能臣,亂世之姦雄。使堅治國無失其道,則垂、萇皆秦之能臣也,烏能為亂哉!堅之所以亡,由驟勝而驕故也。魏文侯問李克,吳之所以亡,對曰:『數戰數勝。』文侯曰:『數戰數勝,國之福也,何故亡?』對曰:『數戰則民疲,數勝則主驕,以驕主御疲民,未有不亡者也。』秦王堅似之矣。」

歷史學家陳登原認為苻堅有四大善事——文學優良,內政修明,大度容人,武功赫赫。后人对待亡国贵族往往以苻坚之仁为戒,选择屠杀殆尽。

呂思勉:「苻堅在諸胡中,尚為稍知治體者,然究非大器。嘗縣珠簾於正殿,以朝群臣。宮宇、車乘、器物、服御、乘以珠璣、琅玕、奇寶、珍怪飾之。雖以尚書裴元略之諫,命去珠簾,且以元略為諫議大夫,然此特好名之為,其諸事不免淫侈,則可想見矣。」後又以苻堅以慕容沖及前燕清河公主姐弟皆有美色而皆寵幸,直斥其「荒淫」。又指其命呂光征西域是「蓋一欲誇耀武功,一亦貪其珍寶也。」又曰:「堅知晉終為秦患,命將出師之不足以晉,而未知躬自入犯之更招大禍,仍是失之於疏;而其疏,亦仍是失之於驕耳。」

著名作家柏楊於柏楊版資治通鑑第25冊的序言中寫到:在大分裂時代中,苻堅大帝以超時代的睿智之姿,出現舞台,為苦難的北中國人民,帶來一個太平盛世。

據說苻堅生母苟氏曾在漳水遊玩,並在西門豹寺祈子,在當晚夢與神交,於是懷有苻堅。十二個月後苻堅才出生,當時天上有神光照耀門庭,苻堅背上亦有紅色字,寫著「草苻臣又土王咸陽」。後苻洪以此及「艸付應王」的讖言改姓苻氏。

姚苌曾把苻坚的屍體挖出来鞭尸,脫掉衣服用荆棘裹起来,再以土坑埋掉。苻坚的冤魂作祟非常顯著,姚萇後來諸事不順,屢屢敗陣,認為是苻堅顯靈,於是也在軍中樹立苻堅像祈求道:「新平之禍,不是臣姚萇的錯啊,臣的兄長姚襄從陝州北渡,順著道路要往西邊去,像狐狸死時把頭朝向原本洞穴一樣,只是想要見一見鄉里啊。陛下與苻眉攔阻於路上攻擊他,害他不能成功就死了,姚襄遺命臣一定要報仇。苻登是陛下的遠親亦想復仇,臣為自己的兄長報仇,又怎麼說是辜負了義理呢?當年陛下封我為龍驤將軍,跟我說:『朕從龍驤將軍當上了皇帝,卿也好好努力罷!』這明明白白的詔諭非常顯然,好像還在耳邊一樣。陛下已經過世成為神明了,怎麼會透過苻登而謀害臣,忘卻當年說的話呢!現在為陛下立神像,請陛下的靈魂進入這裏,不要計較臣的過失了,聽臣至誠的禱告。」 不過姚萇戰況仍未有改善,反而睡不安穩,並招來苻登批評「古今以來,豈有人殺了主公卻反而為主公立神像請求賜福?他期望會有好處嗎?」姚萇終毀了苻堅神像。據說姚萇死前曾夢見過苻堅率天官、鬼兵去襲擊他,期間他被救援自己的士兵誤傷陰部至大量出血。醒後就發現陰部腫脹,醫者刺腫處則如夢中一樣大量出血,一石有餘。,如此嚇得姚萇發狂胡言,又求苻堅原諒,姚萇不久傷重身亡,臨終前跪伏床頭,叩首不已。

據《湧幢小品》言:傳聞死於新平寺之苻堅託夢該寺寺主摩訶,望該寺改為祭祀苻堅及侍衛十餘人的廟宇。住持不從,該寺所在縣鎮,果然死疫相繼,後不得已,即尊其靈示,改廟後,果真無疾。

道教信徒衍其義,逢瘟疫競建祠避禍,稱為苻王爺、苻家神,並於每年正月初二以太牢奉之,稱為祭苻家神。祭苻家神為台灣道教現有祭典之一,祭典日為每年農曆正月初二。

\subsubsection{永光}

\begin{longtable}{|>{\centering\scriptsize}m{2em}|>{\centering\scriptsize}m{1.3em}|>{\centering}m{8.8em}|}
  % \caption{秦王政}\
  \toprule
  \SimHei \normalsize 年数 & \SimHei \scriptsize 公元 & \SimHei 大事件 \tabularnewline
  % \midrule
  \endfirsthead
  \toprule
  \SimHei \normalsize 年数 & \SimHei \scriptsize 公元 & \SimHei 大事件 \tabularnewline
  \midrule
  \endhead
  \midrule
  元年 & 357 & \tabularnewline\hline
  二年 & 358 & \tabularnewline\hline
  三年 & 359 & \tabularnewline
  \bottomrule
\end{longtable}

\subsubsection{甘露}

\begin{longtable}{|>{\centering\scriptsize}m{2em}|>{\centering\scriptsize}m{1.3em}|>{\centering}m{8.8em}|}
  % \caption{秦王政}\
  \toprule
  \SimHei \normalsize 年数 & \SimHei \scriptsize 公元 & \SimHei 大事件 \tabularnewline
  % \midrule
  \endfirsthead
  \toprule
  \SimHei \normalsize 年数 & \SimHei \scriptsize 公元 & \SimHei 大事件 \tabularnewline
  \midrule
  \endhead
  \midrule
  元年 & 359 & \tabularnewline\hline
  二年 & 360 & \tabularnewline\hline
  三年 & 361 & \tabularnewline\hline
  四年 & 362 & \tabularnewline\hline
  五年 & 363 & \tabularnewline\hline
  六年 & 364 & \tabularnewline
  \bottomrule
\end{longtable}

\subsubsection{建元}

\begin{longtable}{|>{\centering\scriptsize}m{2em}|>{\centering\scriptsize}m{1.3em}|>{\centering}m{8.8em}|}
  % \caption{秦王政}\
  \toprule
  \SimHei \normalsize 年数 & \SimHei \scriptsize 公元 & \SimHei 大事件 \tabularnewline
  % \midrule
  \endfirsthead
  \toprule
  \SimHei \normalsize 年数 & \SimHei \scriptsize 公元 & \SimHei 大事件 \tabularnewline
  \midrule
  \endhead
  \midrule
  元年 & 365 & \tabularnewline\hline
  二年 & 366 & \tabularnewline\hline
  三年 & 367 & \tabularnewline\hline
  四年 & 368 & \tabularnewline\hline
  五年 & 369 & \tabularnewline\hline
  六年 & 370 & \tabularnewline\hline
  七年 & 371 & \tabularnewline\hline
  八年 & 372 & \tabularnewline\hline
  九年 & 373 & \tabularnewline\hline
  十年 & 374 & \tabularnewline\hline
  十一年 & 375 & \tabularnewline\hline
  十二年 & 376 & \tabularnewline\hline
  十三年 & 377 & \tabularnewline\hline
  十四年 & 378 & \tabularnewline\hline
  十五年 & 379 & \tabularnewline\hline
  十六年 & 380 & \tabularnewline\hline
  十七年 & 381 & \tabularnewline\hline
  十八年 & 382 & \tabularnewline\hline
  十九年 & 383 & \tabularnewline\hline
  二十年 & 384 & \tabularnewline\hline
  二一年 & 385 & \tabularnewline
  \bottomrule
\end{longtable}


%%% Local Variables:
%%% mode: latex
%%% TeX-engine: xetex
%%% TeX-master: "../../Main"
%%% End:

%% -*- coding: utf-8 -*-
%% Time-stamp: <Chen Wang: 2019-12-19 10:15:33>

\subsection{哀平帝\tiny(385-386)}

\subsubsection{生平}

秦哀平帝苻丕(4世紀-386年),字永叔(或作永敘),略陽臨渭(今甘肅秦安)氐族人。前秦皇帝,宣昭帝苻堅的庶長子,淝水之戰後與後燕君主慕容垂一度相持於鄴城。並在苻堅死後繼承帝位,繼續與後秦、西燕及後燕勢力對抗。最終在進攻洛陽時遭晉軍所殺,死後獲諡為哀平皇帝。

苻丕少時聰慧好學,博通經史。苻堅曾經與他談將略,嘉許了他並命鄧羌教他兵法。苻丕的文武才幹不及叔父苻融,不過他當將領時善於籠絡士卒之心。永興元年(357年)苻堅稱天王時受封為長樂公。

建元四年(368年),苻堅在攻滅叛亂的雍州刺史苻武後,以苻丕為雍州刺史。建元六年(370年)因取消雍州而離任,但次年苻丕就因雍州復置而任使持節、征東大將軍、雍州刺史。後遷征南大將軍,都督征討諸軍事,守尚書令。建元十四年(378年)二月,奉命與苟萇等進攻東晉襄陽。當時前秦軍很快就攻下了襄陽外城,守將朱序只得固守內城,苻丕於是打算急攻內城。然而最終卻聽從了苟萇長期圍困,待其自降的策略,雖然及得慕容垂攻陷南陽郡後與苻丕會合,秦軍仍只一直圍困襄陽;而當時的荊州刺史桓沖以及在次年受命領兵救援襄陽的劉波皆因畏懼秦軍而未敢前進,都沒有起到實質作用。不過,朱序仍一直堅持到年末,前秦御史中丞李柔因而彈劾苻丕等人圍攻襄陽近一年仍未能攻陷,耗費日深而無收效。苻堅亦下詔苻丕要以攻取襄陽贖罪,並命人賜劍苻丕,明言若果不能在下一年春天攻下襄陽就要苻丕以劍自殺。苻丕得詔後惶恐,並下令各軍加緊進攻,終於在次年二月攻下襄陽。

建元十六年(380年),苻堅為加強管理關東領土,於是決定分十五萬戶關中氐族人並分配給宗親重臣,在他們帶領下分駐各重鎮,如同古代諸侯。苻丕則為都督關東諸軍事、征東大將軍、冀州牧,派遣他到鄴鎮守。

苻堅在建元十九年(383年)的淝水之戰大敗給晉軍後返率敗軍回長安,並在洛陽答允讓冠軍將軍慕容垂出撫河北地區。慕容垂到鄴城西南的安陽時修書苻丕,而苻丕知慕容垂北來就已思疑他圖謀作亂,但仍親身迎接,又聽從侍郎姜讓的諫言,放棄襲殺慕容垂的計劃。不久,在新安的丁零人翟斌起兵叛變,苻堅命慕容垂討伐。苻丕當時自覺慕容垂長在鄴城令自己終日都提防他,於是想趁此機會將慕容垂調離鄴城,更希望翟斌和慕容垂打得兩敗俱傷,令自己能從容控制他們,於是給了慕容垂兩千弱兵以及差劣的兵器,並以苻飛龍領一千氐族騎兵作為其副手,作提防監視之用。不久慕容垂請求拜謁前燕在鄴城宗廟遭苻丕拒絕,微服而入亦被亭吏阻止,令其殺掉亭吏,燒亭而去;慕容垂出發後又因知道苻丕想用苻飛龍除掉自己,所以就借機殺了苻飛龍,並開始招集兵士,更密召留鄴的慕容農、慕容楷等出城起兵響應自己。

建元二十年(384年)春,苻丕大宴賓客卻請不來慕容農等,調查三天才知他們已在列人起兵了,而慕容垂及後亦自稱燕王,率兵進攻鄴城。苻丕派了重將石越討伐慕容農等但石越卻兵敗被殺,石越之死更令當地人心騷動。隨著前燕舊臣想相繼響應慕容垂並到鄴城會同慕容垂進攻,慕容垂更寫信給苻丕及苻堅,向其陳述利害,想苻堅放棄鄴城,送苻丕回長安,但遭二人憤怒地拒絕,並回信嚴厲指責慕容垂叛秦。二月,慕容垂就開始進攻鄴城,直至八月仍未能攻下鄴城,但城內糧草已盡,要以松木餵飼戰馬。苻丕向張蚝及并州刺史王騰請兵不得,亦不想向東晉求援;此時謝玄率兵北伐,苻丕派兵抵抗但失敗,終令苻丕屈服,寫信給謝玄說:「我想向你求糧,以西赴國難,當我與援軍相接時就會交鄴城給你。若果不能西進而長安失陷,請你領兵助我保護鄴城。」不過姜讓、最早請苻丕南附東晉的司馬楊膺以及擔任使者的焦逵皆認為苻丕至此仍不肯放下身段,認定事必無成,反而自己修改苻丕的信,改成願意在晉軍來後向東晉歸降,更決定若苻丕屆時不肯就想辦法逼他就範。而當時慕容垂亦派兵圍困鄴城,只留西走長安的路,仍願苻丕自願棄城;而謝玄亦答應出兵救鄴,不但派劉牢之等領二萬兵作援,亦運二千斛米以解城中糧荒。

就在次年(385年)劉牢之北行至枋頭時,楊膺等人改寫苻丕書信並想逼苻丕就範的事被揭發,苻丕於是殺害他們,而因焦逵亦向謝玄等提及此事,令劉牢之聞訊後盤桓不進。此時慕容垂亦因鄴城久久未下而想先取冀州,於是調了慕容農到鄴城。及後因應劉牢之進攻黎陽,苻丕趁慕容垂出兵,留慕容農守鄴圍的機會試圖突圍但失敗,慕容垂亦在擊退劉牢之後回軍鄴城。四月,劉牢之在鄴擊敗慕容垂,終解了鄴城之圍,令慕容垂北走。雖然劉牢之追擊燕軍失敗還須苻丕救援,但苻丕最終都能夠率眾到枋頭獲得晉軍糧食,解決部眾缺糧問題。然而苻丕並非真心與晉合作,亦不曾想放棄鄴城,於是在重返鄴城時就與晉將檀玄發生了戰鬥,終由苻丕取勝並重奪鄴城。

燕、秦兩軍至此時已經相持一整年了,弄得幽、冀地區發生饑荒,人食人且城池都蕭條;而且當時長安亦受到西燕軍隊的攻擊,苻丕於是在當地收兵並要西赴長安。幽州刺史王永因為抵抗不了燕軍進攻而率兵退至壺關,並派使者招請苻丕,苻丕於是率鄴城中六萬多人西赴潞川,並獲張蚝和王騰迎至晉陽。苻丕到了晉陽才知苻堅已經被姚萇所殺,於是發喪並於晉陽南即位為帝,改年號為太安。

在王永等人的協助下,關中及隴右的前秦遺眾都相繼起兵響應苻丕,以對抗慕容氏及姚氏的勢力。太安二年(386年),苻丕留戍晉陽及壺關,自率四萬兵進屯平陽。西燕君主慕容永見此擔憂抵抗不了秦軍,於是請求苻丕讓他取道東歸河北會合慕容垂。但苻丕拒絕並命左丞相王永、俱石子等進攻慕容永。西燕軍於是在襄陵與王永所率的秦軍發生戰鬥,王永及俱石子皆兵敗被殺,苻丕以兵敗,更怕他一直猜忌的苻纂趁他新敗而對其不利,於是率數千南奔東垣,更圖進攻當時受東晉控制的洛陽。晉將馮該就從陝城出兵邀擊苻丕,最終苻丕被殺,除了苻纂等率數萬兵出走杏城外,苻丕統下的官員皆為西燕所得。苻丕死後,族子苻登繼位,諡苻丕為哀平皇帝。

\subsubsection{太安}

\begin{longtable}{|>{\centering\scriptsize}m{2em}|>{\centering\scriptsize}m{1.3em}|>{\centering}m{8.8em}|}
  % \caption{秦王政}\
  \toprule
  \SimHei \normalsize 年数 & \SimHei \scriptsize 公元 & \SimHei 大事件 \tabularnewline
  % \midrule
  \endfirsthead
  \toprule
  \SimHei \normalsize 年数 & \SimHei \scriptsize 公元 & \SimHei 大事件 \tabularnewline
  \midrule
  \endhead
  \midrule
  元年 & 385 & \tabularnewline\hline
  二年 & 386 & \tabularnewline
  \bottomrule
\end{longtable}


%%% Local Variables:
%%% mode: latex
%%% TeX-engine: xetex
%%% TeX-master: "../../Main"
%%% End:

%% -*- coding: utf-8 -*-
%% Time-stamp: <Chen Wang: 2019-12-19 10:16:57>

\subsection{高帝\tiny(386-394)}

\subsubsection{生平}

秦高帝苻登(343年-394年),字文高,略陽臨渭(今甘肅秦安)氐族人,十六国前秦皇帝,苻堅族孫,建節將軍苻敞之子。在苻堅遭後秦君主姚萇殺害後,苻登曾領率氐族殘餘力量於關隴地區對抗後秦,後更被擁立為前秦皇帝。苻登起初屢次獲勝,但終敗給後秦,更遭俘殺。死後獲上廟號太宗,諡高皇帝。

苻登年輕時就勇猛威武,有雄壯的氣慨,但為人粗豪好險而不修小節,並不受苻堅重視。苻登長大成人後卻一改舊習,謹慎厚道,亦看典籍。苻堅曾以其為殿上將軍、羽林監、揚武將軍、長安令,後來因過失被降為狄道長。

淝水之戰後,關中地區大亂,苻登逃到河州牧毛興駐守的枹罕(今甘肅臨夏市)。苻登兄苻同成是毛興的長史,於是請毛興以苻登任其司馬。苻登當時表現得器量不凡,喜歡設奇謀,而他對事物的分析連毛興也十分佩服,然而卻因受毛興所憚而沒有獲重用。

前秦太安二年(386年),時與後秦姚碩德對抗的毛興亦同時與同屬前秦的益州牧王廣及秦州牧王統作戰,頻繁的戰事令厭戰的氐人將其殺害。毛興臨死時就表示苻登能夠消滅姚碩德[1]。同年七月,苻登獲眾人推舉,取代被指年老的氐豪衞平統領原毛興部眾,自稱使持節、都督隴右諸軍事、撫軍大將軍、雍河二州牧、「略陽公」,並即率兵五萬攻佔南安,又派使者向當時前秦皇帝苻丕請求任命。苻丕亦應其自稱授官,並以其為征西大將軍、開府儀同三司、南安王。苻登佔據南安後獲當地胡、漢共三萬多戶歸附,聲勢漸盛,於是進攻姚碩德,並在胡奴阜大敗前往救援的姚萇,更令其身受重傷。

十月,苻丕進攻洛陽時被東晉將領所殺,當時苻丕子苻懿及苻昶都被帶到南安,苻登於是打算立苻懿為帝。但部眾都力勸苻登立長君,並指出非苻登一人不可。苻登於是即位為帝,改元「太初」,立了苻懿為太弟。

當時前秦宗室苻纂為另一軍事力量,他支持苻登令前秦聲勢大盛,並曾與楊定於涇陽(今陝西涇陽縣)大敗姚碩德,更圖謀攻取後秦都城長安。不過苻纂不久卻因不肯自立為帝而遭其弟苻師奴殺害,苻師奴亦遭姚萇擊敗,部眾遭後秦吸納,進攻長安行動亦告吹。

太初三年(388年)二月,苻登與姚萇各據朝那(今寧夏彭陽縣)及武都相持不下,互有勝負。當時關西豪傑見後秦久久不能消滅前秦勢力,很多都轉歸前秦。姚萇終於十月退還根據地安定,苻登亦到新平取軍糧以解軍中饑饉的狀況,並自率萬餘人兵圍姚萇軍營,四面以哭聲震動其軍心;不過姚萇亦命軍人以哭聲回應,苻登見不成功就退兵。

太初四年(389年),苻登在大界留下輜重,自率萬多名輕騎兵進攻安定,先後擊敗安定羌密造保及後秦將吳忠等,並於八月進逼安定。但姚萇卻奇兵夜襲大界,殺害留守的毛氏並擒獲數十名名將,擄掠五萬多人。苻登見此唯有退守根據地胡空堡(今陝西省彬縣西南)。太初六年(391年),苻登因苟曜作為內應而進攻後秦,並擊敗姚萇,殺後秦將吳忠,但姚萇立刻重整軍勢再戰,苻登這次大敗,退兵至郿縣(今陝西眉縣)。同年苻登先後進攻新平及安定,但都遭姚萇擊敗。而當時氐族人強金槌叛歸後秦,兩年前勇略過人的羌人雷惡地亦因遭苻登所忌憚而出奔後秦,次年驃騎將軍沒弈干亦叛降後秦,這些事件都削弱了苻登的力量。

太初七年(392年),苻登以姚萇患病而出兵安定,但在城外九十多里就遭姚萇所派的軍隊攻擊,被逼退還;而姚萇更特意在夜裏命軍隊旁出跟隨苻登軍,苻登聽聞姚萇軍營空無一人,驚懼得說:「他究竟是甚麼人,離開時我不知道,來到時我亦不察覺,人說他快死了,突然又來了。朕和這個羌人活在同一年代,根本是不幸。」

太初九年(394年),苻登知姚萇已死,於是十分高興,並盡率大軍進攻後秦。至夏季,苻登進攻廢橋以得水源,但為後秦將尹緯所阻,部分士兵更渴死。苻登因而急攻尹緯,而尹緯卻大敗苻登,兵眾潰散,苻登單騎逃返原由其弟苻廣留守的雍城(今陝西鳳翔縣南)卻發現苻廣已棄城,另一根據地胡空堡亦遭留守的太子苻崇所棄,苻登無處容身,只有逃到平涼(今甘肅平涼市),收集部眾據守馬髦山。苻登及後向乞伏乾歸求救,得其命乞伏益州領兵救援,卻就在七月苻登率兵迎接乞伏益州時就遇上後秦軍,苻登被生擒並處決,享年五十二歲。

其子苻崇在湟中稱帝,追諡苻登為高皇帝,上廟號太宗。

苻登與後秦軍作戰多年,其為史傳所載的生平事跡多為他在軍旅中的事跡。

苻登取代衞平後,銳意出兵,但當年天旱,兵眾都吃不飽,苻登於是都將戰爭中殺死的敵軍都叫做「熟食」,更向軍人說:「你們早上作戰,黃昏就能吃飽肉了,還怕飢餓麼!」士兵於是就吃屍肉為生,吃飽後都有氣力戰鬥,逼得姚萇急召姚碩德:「你再不來,我們就要被苻登吃光了。」

苻登曾在軍中設苻堅神主,每次作戰或有所決定都會向其稟告。而苻登即皇帝位後要出兵後秦,亦向苻堅稟告,發言後欷歔流涕,更感染了將士們,令他們都在鎧甲和矛上都刻上「死休」二字,以作至死方休之志。立神主一事甚至令時屢敗的姚萇認為這真是苻堅神助,也一度在軍中設了苻堅像。

\subsubsection{太初}

\begin{longtable}{|>{\centering\scriptsize}m{2em}|>{\centering\scriptsize}m{1.3em}|>{\centering}m{8.8em}|}
  % \caption{秦王政}\
  \toprule
  \SimHei \normalsize 年数 & \SimHei \scriptsize 公元 & \SimHei 大事件 \tabularnewline
  % \midrule
  \endfirsthead
  \toprule
  \SimHei \normalsize 年数 & \SimHei \scriptsize 公元 & \SimHei 大事件 \tabularnewline
  \midrule
  \endhead
  \midrule
  元年 & 386 & \tabularnewline\hline
  二年 & 387 & \tabularnewline\hline
  三年 & 388 & \tabularnewline\hline
  四年 & 389 & \tabularnewline\hline
  五年 & 390 & \tabularnewline\hline
  六年 & 391 & \tabularnewline\hline
  七年 & 392 & \tabularnewline\hline
  八年 & 393 & \tabularnewline\hline
  九年 & 394 & \tabularnewline
  \bottomrule
\end{longtable}


%%% Local Variables:
%%% mode: latex
%%% TeX-engine: xetex
%%% TeX-master: "../../Main"
%%% End:

%% -*- coding: utf-8 -*-
%% Time-stamp: <Chen Wang: 2019-12-19 10:17:55>

\subsection{苻崇\tiny(394)}

\subsubsection{生平}

苻崇(4世紀-394年),十六国前秦末主。

苻崇是高帝苻登之子,太初三年(388年)八月立為太子。

太初九年(394年)七月,苻登兵敗,被後秦姚興殺死,崇逃到湟中即帝位,改元延初。十月,苻崇被隴西鮮卑的梁王乞伏乾歸(後來的西秦王)驅逐,逃到隴西王楊定那裡。楊定率領二萬人與苻崇共攻乾歸,先勝後大敗,定及崇俱被殺,乾歸盡有隴西之地。

前秦太子苻宣投靠仇池楊盛,不再設置郡縣,前秦亡。

\subsubsection{延初}

\begin{longtable}{|>{\centering\scriptsize}m{2em}|>{\centering\scriptsize}m{1.3em}|>{\centering}m{8.8em}|}
  % \caption{秦王政}\
  \toprule
  \SimHei \normalsize 年数 & \SimHei \scriptsize 公元 & \SimHei 大事件 \tabularnewline
  % \midrule
  \endfirsthead
  \toprule
  \SimHei \normalsize 年数 & \SimHei \scriptsize 公元 & \SimHei 大事件 \tabularnewline
  \midrule
  \endhead
  \midrule
  元年 & 394 & \tabularnewline
  \bottomrule
\end{longtable}


%%% Local Variables:
%%% mode: latex
%%% TeX-engine: xetex
%%% TeX-master: "../../Main"
%%% End:


%%% Local Variables:
%%% mode: latex
%%% TeX-engine: xetex
%%% TeX-master: "../../Main"
%%% End:
  