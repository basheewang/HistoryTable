%% -*- coding: utf-8 -*-
%% Time-stamp: <Chen Wang: 2021-11-01 11:56:49>

\subsection{景明帝苻健\tiny(351-355)}

\subsubsection{生平}

秦景明帝苻健(317年-355年7月10日),字建業,略阳临渭(今甘肃秦安)人,氐族,苻洪第三子,十六国前秦開國皇帝。苻健繼父親苻洪統領部眾並成功入關,定都長安(今陝西西安),建立前秦。後屢次作戰征服其他反抗前秦的關內勢力,更擊敗北伐的晉軍。

苻健弓馬嫻熟,驍勇果敢,好施予亦善於事奉人,故此深得後趙皇帝石虎父子寵愛,當時石虎心中仍提防苻氏,暗殺了苻健的兩個兄長,但就沒有加害苻健。永和六年(350年),因應苻洪歸降東晉,苻健獲授假節、右將軍、監河北征討前鋒諸軍事、襄國縣公。同年苻洪軍師將軍麻秋毒殺苻洪,意圖併吞苻洪部眾,苻健於是收殺麻秋。苻洪臨死時向苻健說:「我之所以一直未入關中,就是以為能夠奪得中原;今天卻不幸被麻秋那小子加害。中原不是你們兄弟能夠爭奪到的,我死了後,你就快快入關中呀!」苻健於是接領父親的部眾,去掉父親自稱的大都督、大將軍、「三秦王」的稱號,稱東晉所授的官爵,並派叔父苻安到東晉報喪,請示朝命。

同年,後趙新興王石祗在襄國(今河北邢台)即位為帝,又以苻健為都督河南諸軍事、鎮南大將軍、開府儀同三司、兗州牧、「略陽郡公」。不過,苻健當時並沒有助石祗對付冉閔,反將目標對準關中,只為麻痺當時據有關中的杜洪才接受後趙的任命。苻健又在駐地枋頭(今河南浚縣西)興治宮室,教人種麥,顯得根本沒有心思佔領關中。但及後苻健就自稱晉征西大將軍、都督關中諸軍事、雍州刺史,率眾西進,並在盟津渡過黃河。渡河前,苻健命苻雄和苻菁分別領兵從潼關(今陝西渭南市潼關縣北)和軹關(今河南濟源東北)進攻,自己則跟隨苻雄渡河,並在渡河後燒掉浮橋,意在死戰。杜洪部將張先在潼關抵抗苻健軍,但被擊敗。及後苻健派苻雄兵行渭北,附近的氐、羌酋長都斬杜洪使而向苻健投降,苻菁、魚遵經過的城邑亦都投降,更在渭北生擒張先,令三輔地區大致都落在苻健之手。杜洪見局勢如此,唯有退守長安,但苻健隨即進攻長安,杜洪被逼棄長安而逃奔司竹(今陝西司竹鄉),苻健於是進據長安。苻健見長安人心思晉,於是向東晉獻捷報,並與東晉征西大將軍桓溫修好。於是令秦雍二州的少數民族和漢人都向苻健歸附,苻健亦攻滅佔領上邽(今甘肅天水市),不肯歸降的後趙涼州刺史石寧。

永和七年(351年),左長史賈玄碩請苻健依劉備稱「漢中王」事,表苻健為都督關中諸軍事、大將軍、大單于、「秦王」。但苻健則假裝憤怒的說:「我豈有能力當秦王呀!而且出使東晉的使者還未回來,你們又怎知我的官爵呀。」然而,不久就又暗示賈玄碩等為他上尊號,最終在再三推讓後,于正月丙辰日(351年3月4日)即天王、大單于位,大封宗室及諸子為公爵,建國號大秦,年號「皇始」,正式建立前秦政權。次年正月辛卯日(352年2月2日),苻健稱帝,進諸公爵為王爵,並授大單于位予太子苻萇。

皇始元年(351年),被苻氏驅逐的杜洪引東晉梁州刺史司馬勳伐前秦,苻健於是率兵在五丈原擊退他。司馬勳敗歸漢中(今陝西漢中)後,杜洪被其部將張琚所殺,不久苻健領二萬兵攻滅張琚,更派兵擄掠關東,助後趙豫州刺史張遇擊敗東晉將領謝尚,及後擄張遇及其部眾回長安,並對張遇授官。後張遇謀反事敗,引發雍州孔特等人舉兵反抗前秦。最終苻健亦派兵成功平定。

皇始四年(354年),桓溫北伐,自率主力軍自武關(今陝西丹鳳縣東)直取長安,另命司馬勳在進攻隴西。前秦初戰不利,被桓溫進攻至長安東南防近的灞上,逼得苻健要盡出三萬精兵出城抵禦桓溫。然而因桓溫並不急於進攻,而且苻健先晉兵一步收取熟麥,故此最終逼得桓溫退兵,苻健更乘勢追擊晉軍,大敗對方。

苻健勤於政事,多次召見公卿談論治國之道,而且一改後趙時苛刻奢侈之風,改以薄賦節儉,更專崇儒學,禮待長者,故此得到人們稱許。

皇始四年(354年),皇太子苻萇在追擊桓溫時受傷,同年傷重而死。次年(晉永和十一年,355年),苻健因讖文中有「三羊五眼」字句,遂以淮南王苻生當太子。至當年六月,苻健患病,苻生在苻健宮室侍疾,而當時任太尉的平昌公苻菁則以為苻健已死,直接領兵入宮,打算殺死苻生自立。但到東掖門時,苻健知道宮中發生事變,自登端門,陳兵自衞。當時苻菁部眾見苻健未死,於是驚懼潰散,苻健於是拿下苻菁,將他殺死。不久,苻健以太師魚遵、丞相雷弱兒、太傅毛貴、司空王墮、尚書令梁楞、尚書左僕射梁安、尚書右僕射段純及吏部尚書辛牢等為輔政大臣。但又告訴太子:「六夷酋帥及掌權的大臣,若果不遵從你的命令,那就立即除去他們。」。六月乙酉日(7月10日),苻健病逝,享年三十九歲。苻健死後諡為明皇帝,廟號稱世宗,後改諡為景明皇帝,廟稱高祖。

\subsubsection{皇始}

\begin{longtable}{|>{\centering\scriptsize}m{2em}|>{\centering\scriptsize}m{1.3em}|>{\centering}m{8.8em}|}
  % \caption{秦王政}\
  \toprule
  \SimHei \normalsize 年数 & \SimHei \scriptsize 公元 & \SimHei 大事件 \tabularnewline
  % \midrule
  \endfirsthead
  \toprule
  \SimHei \normalsize 年数 & \SimHei \scriptsize 公元 & \SimHei 大事件 \tabularnewline
  \midrule
  \endhead
  \midrule
  元年 & 351 & \tabularnewline\hline
  二年 & 352 & \tabularnewline\hline
  三年 & 353 & \tabularnewline\hline
  四年 & 354 & \tabularnewline\hline
  五年 & 355 & \tabularnewline
  \bottomrule
\end{longtable}


%%% Local Variables:
%%% mode: latex
%%% TeX-engine: xetex
%%% TeX-master: "../../Main"
%%% End:
