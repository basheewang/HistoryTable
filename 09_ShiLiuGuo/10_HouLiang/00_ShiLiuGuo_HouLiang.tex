%% -*- coding: utf-8 -*-
%% Time-stamp: <Chen Wang: 2019-12-19 15:47:32>


\section{后凉\tiny(389-403)}

\subsection{简介}

後凉(386年-403年)是十六国时期氐人贵族吕光建立的政权。

其国号以地处凉州为名。《十六国春秋》始称“后凉”,以别于其他以“凉”为国号的政权,后世袭用之。

東晋太元八年(383年)前秦将军吕光受命率7万餘众讨平西域。苻坚淝水兵败後前秦瓦解,吕光据有姑臧(今甘肃武威)于太元十一年(386年)称大将军、凉州牧。太元十四年(389年)吕光称三河王,後改称天王,史称後凉。

统治范围包括甘肃西部和宁夏、青海、新疆一部分。

後凉以氐人军事力量为基础,势力孤弱,刑法峻重,社会局势不稳,叛者连城。

後凉龙飞四年(399年)吕光卒,子吕绍继位,庶长子吕纂又杀吕绍自立。後凉咸宁三年(401年)吕隆(吕光弟吕宝之子)又杀吕纂自立,国势益衰。连年战争,经济凋敝,太元十二年(403年),穀价昂贵,人相食。

神鼎三年(403年),吕隆因后秦、南凉、北凉交相攻逼,降于後秦,後凉亡。

%% -*- coding: utf-8 -*-
%% Time-stamp: <Chen Wang: 2019-12-19 15:49:27>

\subsection{懿武帝\tiny(386-399)}

\subsubsection{生平}

涼懿武帝呂光(338年-399年),字世明,略陽(今甘肅天水)氐人,前秦太尉呂婆樓之子。十六國時期後涼建立者。呂光初為前秦將領,屢立戰功,前秦天王苻堅就派了他出兵西域。呂光降服西域,但當時前秦因淝水之戰戰敗而國亂,回軍時為涼州刺史梁熙所阻,呂光消滅了梁熙而入主涼州,遂在當地建立政權。

呂光得王猛看重,並將他推薦給苻堅,苻堅於是以呂光為美陽令,任內呂光得當地人民愛戴信服。呂光後遷鷹揚將軍,以功封關內侯,並於永興二年(358年)隨苻堅等討伐張平。苻堅與張平於銅壁決戰,張平驍勇大力的養子張蚝單騎屢次進出前秦軍陣中,呂光於是去襲擊張蚝並成功擊傷他。張蚝受傷被擒,張平潰敗,呂光亦因而聲名大噪。

建元四年(368年),呂光與王鑒等因應楊成世討伐上邽叛變的苻雙失敗而率軍再行討伐,王鑒到後打算與苻雙前鋒苟興速戰速決,但呂光慮及對方因剛獲勝而士氣高漲,建議謹慎待敵,讓其糧盡退兵時就是進攻的時機。二十日後苟興退兵,王鑒追擊並擊敗苟興,隨後又大敗苻雙,終攻下上邽,斬殺苻雙。建元六年(370年),呂光隨軍攻滅前燕,獲封都亭侯。後苻重出鎮洛陽,呂光擔任其長史。苻重於建元十四年(378年)謀反,苻堅以呂光忠誠正直,不會與苻重連謀,於是下令呂光收捕苻重,呂光聽命並以檻車押送苻重回長安。後呂光遷太子右率,頗受敬重。次年呂光又以破虜將軍身份率兵擊敗進攻成都的李烏,遷步兵校尉。建元十六年(380年)呂光又奉命與左將軍竇衝共領四萬兵討伐叛亂的苻重,又將其生擒,戰後獲授驍騎將軍。

前秦十八年(382年),呂光受命征討西域,以使持節都督西討諸軍事身份率領姜飛等將領、七萬兵及五千鐵騎出發。呂光越過三百多里長的沙漠到達西域,降服焉耆等西域各國,又擊破唯一拒守的龜茲,威震西域。苻堅知呂光征服西域,即任命其為使持節、散騎常侍、都督玉門以西諸軍事、安西將軍、西域校尉,封順鄉侯,但因前秦於淝水之戰後國內大亂而道路不通,未能傳達。呂光本來想要留在龜茲,但是受到名僧鳩摩羅什勸阻,而且部眾們也想回到中原,遂回師。

太安元年(385年),呂光軍抵宜禾(今新疆安西南),高昌太守楊翰告訴涼州刺史梁熙,稱呂光還軍必定別有所圖,建議關閉天險要道,拒之於外,但梁熙沒有聽從。呂光最初知道楊翰的計劃時曾打算不再前進,但在杜進勸告下還是繼續,楊翰即在呂光到達高昌時向呂光請降。梁熙在呂光到遠玉門時傳檄指責呂光擅自班師,又派其子梁胤等率軍五萬往酒泉阻擊呂光。呂光也傳檄指責梁熙沒有為前秦赴國難的忠誠,還阻攔歸國軍隊,並派了姜飛等為前鋒進攻梁胤。姜飛等在安彌大破梁胤並生擒他,於是周邊外族都紛紛依附呂光,武威太守彭濟更將梁熙抓起來叛歸呂光。呂光殺死梁熙,入主姑臧,自領涼州刺史、護羌校尉。

386年,呂光收到苻堅死訊,改元太安,並自稱使持節、侍中、中外大都督、督隴右河西諸軍事、大將軍、涼州牧、酒泉公。呂光入主涼州時,因尉祐與彭濟共謀抓住梁熙的功勞而寵任他,但呂光卻在尉祐中傷下殺了姚皓、尹景等十多個名士,人心見離。當時國內米價也高漲至一斗五百,饑荒中更發生人吃人事件,死了很多人。呂光與群僚在飲宴中談及為政時用嚴峻刑法的問題,在參軍段業勸言下終下令自省並行寬簡之政。

呂光於太安二年(387年)殺了進逼姑臧的張大豫,但王穆尚據酒泉;西平太守康寧也叛變,阻兵據守,呂光試圖討伐但都不果。及後連呂光部將徐炅及張掖太守彭晃都謀叛,並聯結了王穆及康寧。呂光力排眾議親率三萬兵速攻彭晃,二十日後攻破張掖,殺了彭晃。不久,呂光乘王穆進攻其將索嘏的機會率二萬兵襲破酒泉,王穆率兵東返但部眾在途中就潰散,王穆隻身逃走但為騂馬令郭文所殺。

389年,呂光稱三河王,改元麟嘉。396年六月又改稱天王,國號大涼,改元龍飛。呂光曾先後多次進攻西秦,其中呂光弟呂延於龍飛二年(397年)的進攻中兵敗被殺。呂光聽信讒言,怪罪從軍的尚書沮渠羅仇及三河太守沮渠麴粥,並殺二人。二人歸葬時,因諸部聯姻而共計有萬多人參與葬禮,羅仇之侄沮渠蒙遜遂反,蒙遜堂兄沮渠男成舉兵響應,並推建康太守段業為主,建北涼與後涼對抗,呂光曾派呂纂討伐,但最終無法消滅北涼。

同年,善於天文術數的太常郭黁與僕射王詳認為呂光年老、太子闇弱而呂纂等凶悍,料定呂光死後必會有禍亂,並禍及自己,故圖謀攻奪姑臧東西苑城,推王乞基為主。不過王詳因事泄而被殺,郭黁遂據東苑叛變,當時民間還有很多人支持郭黁。呂光召呂纂回兵討伐郭黁,呂纂遂屢破郭黁,令其於龍飛三年(398年)出走西秦,平定亂事。

龍飛四年(399年),呂光病重,立太子呂紹為天王,自號太上皇帝(太上天王)。呂光又讓呂纂及呂弘分任太尉及司徒,告誡呂紹要倚重二人,放權讓他們處理軍政大事才能保國家安穩;另也對呂纂及呂弘說二人要與天王呂紹同心合力才能保全國家,否則禍亂必會來。呂光於不久去世,享年六十三歲,諡懿武皇帝,廟號太祖。

呂光年輕時已展現其軍事能力,十歲時與其他小童一起玩耍時就創制戰爭陣法,於是同年的人都推其為主,而呂光處事平允,更令眾小童佩服。呂光也不喜歡讀書,只好打獵。

呂光高八尺四寸,雙目重瞳,為人沈著堅毅,凝重且寛大有度量,喜怒不形於色,故王猛賞識他,稱:「此非常人。」

呂光出生於枋頭(今河南浚縣西南),當夜有神光,全家覺得奇怪,遂以光为名。

呂光左肘有一肉印,據說在一次戰爭中肉印隱約顯出「巨霸」兩字。

\subsubsection{太安}

\begin{longtable}{|>{\centering\scriptsize}m{2em}|>{\centering\scriptsize}m{1.3em}|>{\centering}m{8.8em}|}
  % \caption{秦王政}\
  \toprule
  \SimHei \normalsize 年数 & \SimHei \scriptsize 公元 & \SimHei 大事件 \tabularnewline
  % \midrule
  \endfirsthead
  \toprule
  \SimHei \normalsize 年数 & \SimHei \scriptsize 公元 & \SimHei 大事件 \tabularnewline
  \midrule
  \endhead
  \midrule
  元年 & 386 & \tabularnewline\hline
  二年 & 387 & \tabularnewline\hline
  三年 & 388 & \tabularnewline\hline
  四年 & 389 & \tabularnewline
  \bottomrule
\end{longtable}

\subsubsection{麟嘉}

\begin{longtable}{|>{\centering\scriptsize}m{2em}|>{\centering\scriptsize}m{1.3em}|>{\centering}m{8.8em}|}
  % \caption{秦王政}\
  \toprule
  \SimHei \normalsize 年数 & \SimHei \scriptsize 公元 & \SimHei 大事件 \tabularnewline
  % \midrule
  \endfirsthead
  \toprule
  \SimHei \normalsize 年数 & \SimHei \scriptsize 公元 & \SimHei 大事件 \tabularnewline
  \midrule
  \endhead
  \midrule
  元年 & 389 & \tabularnewline\hline
  二年 & 390 & \tabularnewline\hline
  三年 & 391 & \tabularnewline\hline
  四年 & 392 & \tabularnewline\hline
  五年 & 393 & \tabularnewline\hline
  六年 & 394 & \tabularnewline\hline
  七年 & 395 & \tabularnewline\hline
  八年 & 396 & \tabularnewline
  \bottomrule
\end{longtable}

\subsubsection{龙飞}

\begin{longtable}{|>{\centering\scriptsize}m{2em}|>{\centering\scriptsize}m{1.3em}|>{\centering}m{8.8em}|}
  % \caption{秦王政}\
  \toprule
  \SimHei \normalsize 年数 & \SimHei \scriptsize 公元 & \SimHei 大事件 \tabularnewline
  % \midrule
  \endfirsthead
  \toprule
  \SimHei \normalsize 年数 & \SimHei \scriptsize 公元 & \SimHei 大事件 \tabularnewline
  \midrule
  \endhead
  \midrule
  元年 & 396 & \tabularnewline\hline
  二年 & 397 & \tabularnewline\hline
  三年 & 398 & \tabularnewline\hline
  四年 & 399 & \tabularnewline
  \bottomrule
\end{longtable}

%%% Local Variables:
%%% mode: latex
%%% TeX-engine: xetex
%%% TeX-master: "../../Main"
%%% End:

%% -*- coding: utf-8 -*-
%% Time-stamp: <Chen Wang: 2021-11-01 14:51:50>

\subsection{灵帝呂紹\tiny(399-401)}

\subsubsection{隐王生平}

涼隱王呂紹(380年代-399年),字永業,略陽(今甘肅天水)氐人。十六國時期後涼國第二任君主,後涼懿武帝呂光嫡子。呂紹登位不久即被呂纂及呂弘兩位兄長發動政變所推翻,呂紹自殺。

呂光出征西域時,呂紹與石氏等人留在前秦。淝水之戰後,前秦因戰敗而國亂,長安亦構亂,呂紹等人於是出奔仇池,直至麟嘉元年(389年)才到後涼,甫稱三河王的呂光遂立吕紹為世子。龍飛元年(396年),呂光立其為太子。

吕绍唯一一次有记载的亲自指挥的军事行动在龍飛四年(399年),当时他与庶兄吕纂攻打北凉天王段业,段业求助于南凉天王秃发乌孤。秃发乌孤的弟弟秃发利鹿孤率援军赶到,吕绍和吕纂只得撤退。

同年年末,呂光病重,立呂紹為天王,以吕纂为太尉,吕弘為司徒,臨終前叮囑呂紹說:「如今三寇(乞伏乾歸、段業和禿髮烏孤)未平,我死之後,呂纂帶領軍隊,呂弘治理朝政,你自己無為而治,把重任交給兩個哥哥」。也对两位长子有所嘱咐:“永业并非治理乱世的人才,只不过因嫡长的规举才让其处元首之位。现在外有强寇,人心不定,你们兄弟和睦则会让国家流传万世;若果自己内斗,则祸乱立即就会来了。”还对吕纂说:“你本性粗豪勇武,很令我担心。开展基业本来就艰难,守成也不容易。好好辅助永业,不要听谗言呀。”不久呂光去世,呂紹秘不發喪,呂纂推門入殿哭喪,竭盡哀思才出來。呂紹害怕被殺害,想讓位給他,但呂纂以呂紹是嫡子身份推辭,呂紹固請也不獲呂纂答允,於是即位。呂光侄子呂超勸呂紹及早除去既有兵權,又有極高威名的呂纂,但呂紹雖也憂心呂纂,但仍以父親遺命及袁尚兄弟相爭之事一再拒絕對付呂纂,令呂超很失望。呂紹在湛露堂面見呂纂時,呂超持刀在側侍候,用眼神請求呂紹收捕呂纂,但呂紹都不肯。

在呂紹到後涼前,呂光曾經想立呂弘為世子,不過因為知道呂紹在仇池而打消念頭。可是呂弘一直記恨在心,不久即派尚書姜紀唆使呂纂和他一起叛變。呂纂順從,於是在一夜率軍攻入宮廷,呂紹試圖出兵抵抗,但兵眾都因為忌憚呂纂威名而潰散。呂紹見此便在紫閣自殺。呂纂即位後諡呂紹為隱王。

\subsubsection{灵帝生平}

涼靈帝呂纂(4世紀?-401年),字永緒,略陽(今甘肅天水)氐人。十六國時期後涼國君主,後涼開國君主呂光庶長子,母親是趙淑媛,隱王呂紹兄。呂纂在呂光死後不久即以政變逼死呂紹登位,但在位一年多就在呂超等人的變亂被殺。

呂纂年少時已熟練弓馬,雖然入了太學,但不愛讀書,只會交結公侯。淝水之戰後前秦國亂,呂纂逃到上邽(今甘肅天水市),至太安元年(386年)才到達後涼都城姑臧(今甘肅武威市),拜虎賁中郎將。麟嘉四年(392年),呂光派了呂纂進攻南羌彭奚念,但在盤夷大敗而還。呂光遂親率大軍再攻,讓呂纂及楊軌、沮渠羅仇進軍左南(今青海西寧市東),逼得彭奚念憑湟河自守,然呂光還是派兵渡過湟河,攻下枹罕(今甘肅臨夏市),令彭奚念敗走甘松(今甘肅叠部縣東南)。

龍飛元年(396年),呂光稱天王,以呂纂為太原公。次年,呂光攻西秦,派呂纂、楊軌及竇苟等率三萬兵攻金城(今甘肅蘭州市),攻陷了金城。同年,呂光殺沮渠羅仇及沮渠麴粥,令得羅仇侄沮渠蒙遜反叛。蒙遜堂兄沮渠男成也推了建康太守段業為主,呂纂奉命討伐段業,然而因為沮渠蒙遜率眾到臨洮為聲援段業,呂纂在合離大敗給段業。同時,太常郭黁在姑臧作亂,呂光立即召回呂纂,當時諸將顧慮段業會乘大軍撤退而從後跟隨,建議乘夜暗中撤走,不過呂纂看准段業無謀略,乘夜退走只會助長敵人,於是在退兵時前派了使者向段業說:「郭黁作亂,吾今還都。卿能決者,可出戰。」段業果然不敢追擊。郭黁派軍於白石邀擊呂纂,呂纂大敗,但不久因西安太守石元良率兵援救才得以擊敗郭黁,攻入姑臧。呂纂隨後在城西擊破郭黁將王斐,令郭黁勢力開始衰敗。不過郭黁卻推了楊軌為盟主,讓楊軌前赴姑臧支援自己。時呂弘為段業所逼,呂纂就前去迎接呂弘,楊軌認為這是機會,於是率兵邀擊,但卻為呂纂所敗,郭黁於是出奔西秦,楊軌隨後亦奔廉川,亂事終告平定。

龍飛四年(399年),吕纂与吕绍一同统兵攻打北凉天王段业,段业求救于南凉天王秃发乌孤,秃发乌孤之弟秃发利鹿孤率援军赶到,段业坚守不战,吕纂、吕绍于是退兵。

同年,呂光病重,立呂紹為天王,以呂纂為太尉,掌握軍權。呂光死前曾向呂纂及呂弘說:「永業並非治理亂世的人才,只不過因嫡長的規舉才讓其處元首之位。現在外有強寇,人心不定,你們兄弟和睦則會讓國家流傳萬世;若果自己內鬥,則禍亂立即就會來了。」另也特別對呂纂說:「你本性粗豪勇武,很令我擔心。開展基業本來就艱難,守成也不容易。好好輔助永業,不要聽讒言呀。」不久呂光去世,呂紹懼怕呂纂,曾經想要讓位給呂纂,然而呂纂以嫡庶之別拒絕;另呂光侄呂超又勸呂紹殺了呂纂,但呂紹不肯。可是不久吕纂就在呂弘的煽動下反叛,夜裏率壯士數百進攻廣夏門,守融明觀的齊從抽劍攻擊呂纂,擊中其額,但為呂纂部眾制服。呂紹所派部隊因懼怕呂纂而潰散,吕紹被逼自殺。呂纂遂即天王位,改年號咸寧。

咸寧二年(400年),呂弘舉兵反叛,但為呂纂將焦辨擊敗,出奔廣武(今甘肅永登縣),不久為呂方所捕,被殺。呂纂隨後縱兵大掠,以原屬呂弘的東苑中之婦女賞給軍士,呂弘的妻兒都被士兵侵辱。呂纂笑著對群臣說:「今日一戰怎樣呀?」侍中房晷卻答:「天要降禍給涼室,故藩王起兵釁。先帝駕崩不久,隱王幽逼而死,山陵才剛建好,大司馬就因驚懼疑惑而反叛肆逆,京邑成了兄弟交戰的戰場。雖然呂弘自取滅亡,亦是因為陛下沒有棠棣所說的兄弟之義。現在應該反思自省,以為向百姓謝過,卻反而縱容士兵大肆掠奪,侮辱士女。兵釁因呂弘而起,百姓有甚麼錯!而且呂弘的妻子是陛下的弟婦,女兒也是陛下的姪女,怎能讓她們成為無賴小人的婢妾。天地神明怎會忍心見到這樣!」呂纂聽後向房晷道歉,又接回呂弘的妻兒到東宮。

隨後,呂纂不顧中書令楊穎反對堅決攻伐南涼,卻為南涼將禿髮傉檀所敗。呂纂不久又不聽姜紀諫言而攻北涼,圍攻張掖(今甘肅張掖)並攻略建康郡地,然而禿髮傉檀果如姜紀所言進攻姑臧,呂纂亦被逼退兵。呂纂在位時沉溺於酒色,又常常出獵,諸大臣皆曾勸阻,然而呂纂皆不能聽從。

咸寧三年(401年)呂纂因番禾太守呂超擅攻鮮卑思盤一事召呂超及思盤入朝,呂超因恐懼而事先結交了殿中監杜尚。呂纂憤怒地斥責呂超,更聲言「要當斬卿,然後天下可定」,嚇得呂超叩頭稱不敢。不過呂纂及後就和呂超及眾大臣宴會,呂超兄呂隆於是頻頻向呂纂勸酒要灌醉他。呂纂飲至昏醉便乘坐步輓車與呂超等人在宮內遊走,在到琨華殿東閤時步輓車過不了去,呂纂親將竇川及駱騰於是放下配劍推車。呂超乘此機會拿起二人配劍襲擊呂纂,呂纂試圖下車抓住呂超但被對方刺穿胸部;呂超又殺了竇川和駱騰。呂纂后楊氏下令禁軍討伐呂超,但杜尚卻命禁軍放下武器。將軍魏益多遂斬下呂纂的頭,聲言:「呂纂違反先帝遺命,殺害太子、沉溺飲酒和田獵、親近小人、輕易殺害忠良、視百姓為草芥。番禾太守呂超以骨肉之親,恐懼國家傾覆,已經除去他了。上可以安寧宗廟,下可為太子報仇。但凡國人都應歡慶。」

呂隆不久繼位,諡呂纂為靈皇帝,葬白石陵。

即序胡安據曾盜張駿的墓,獲得大量珍寶,呂纂誅殺安據和其親黨五十多家人,派使者弔祭張駿,並復修其陵墓。

咸寧二年,有母豬生下小豬,一身三頭,又有飛龍夜裡從東廂的井中出現,名僧鳩摩羅什以為不祥,勸纂廣施仁德。一日羅什與呂纂玩博戲,呂纂吃多子,玩笑道:“砍胡奴頭!”羅什糾正說:“不斫胡奴頭,胡奴斫人頭。”預言了呂纂因小字「胡奴」的呂超而被殺的命運。

\subsubsection{咸宁}

\begin{longtable}{|>{\centering\scriptsize}m{2em}|>{\centering\scriptsize}m{1.3em}|>{\centering}m{8.8em}|}
  % \caption{秦王政}\
  \toprule
  \SimHei \normalsize 年数 & \SimHei \scriptsize 公元 & \SimHei 大事件 \tabularnewline
  % \midrule
  \endfirsthead
  \toprule
  \SimHei \normalsize 年数 & \SimHei \scriptsize 公元 & \SimHei 大事件 \tabularnewline
  \midrule
  \endhead
  \midrule
  元年 & 399 & \tabularnewline\hline
  二年 & 400 & \tabularnewline\hline
  三年 & 401 & \tabularnewline
  \bottomrule
\end{longtable}


%%% Local Variables:
%%% mode: latex
%%% TeX-engine: xetex
%%% TeX-master: "../../Main"
%%% End:

%% -*- coding: utf-8 -*-
%% Time-stamp: <Chen Wang: 2019-12-19 15:51:37>

\subsection{吕隆\tiny(401-403)}

\subsubsection{生平}

呂隆(4世紀?-416年),字永基,略陽(今甘肅天水)氐人。十六國時期後涼最後一位君主,後涼開國君主呂光之弟呂寶子。呂隆即位不久即遭後秦攻擊,被逼向後秦請降,其在位時間亦不斷遭南涼及北涼二國攻擊,國力大衰,最終呂隆向後秦請求迎其東遷,後涼遂為後秦所併。

呂隆長得俊美,擅長騎射。呂光時曾任北部護軍。咸寧三年(401年),呂隆弟呂超以兵變弒殺天王呂纂,隨後就擁立呂隆。呂隆面有難色,但呂超說:「現在就好像騎著龍飛在天上,豈可以中途下來!」呂隆於是登位,改元神鼎。

呂隆登位後多殺豪望以圖立威,反不得人心,令人人自危。魏安人焦朗遂招請後秦將領姚碩德攻涼,姚碩德聽從並率軍進攻,兵臨姑臧。呂隆派了呂超及呂邈抵抗但大敗而還,呂邈更戰死,呂隆只得嬰城固守。不過,後秦軍接著數月的圍困令城中原來自東面的人圖謀叛變,將軍魏益多更煽動人們殺呂隆及呂超,呂隆遂在事件被揭發後誅殺共三百多家人。當時後涼群臣勸呂隆和後秦請和,呂隆原本不肯,但在呂超勸諫下向後秦請降。姚碩德於是表呂隆為鎮西大將軍、涼州刺史、建康公。

神鼎二年(402年),北涼沮渠蒙遜率兵進攻姑臧,呂隆請得南涼將禿髮傉檀援救,但傉檀未到呂隆就擊敗蒙遜。蒙遜於是與呂隆結盟,並留下萬多斛穀。但其時姑臧穀價已經高達五千文一斗,發生人吃人事件,死了十多萬人。百姓因為姑臧整天關上城門而無法出城找食物,於是每日都有數百人請求出城當別人奴婢以求生,呂隆怕他們會動搖人心,遂將這些人都盡數殺害,屍體堆滿路上。然而,接著後涼仍不斷受到北涼及南涼攻擊,呂隆被逼於神鼎三年(403年)借後秦徵呂超入侍的機會命其帶著珍寶,請後秦派兵迎其離開。秦將齊難等於該年八月到達姑臧,呂隆率眾隨之東遷長安,呂隆獲後秦授散騎常侍,後涼至此滅亡。後秦弘始十八年(416年),受後秦皇帝姚興子廣平公姚弼謀反案牽連,被殺。

\subsubsection{神鼎}

\begin{longtable}{|>{\centering\scriptsize}m{2em}|>{\centering\scriptsize}m{1.3em}|>{\centering}m{8.8em}|}
  % \caption{秦王政}\
  \toprule
  \SimHei \normalsize 年数 & \SimHei \scriptsize 公元 & \SimHei 大事件 \tabularnewline
  % \midrule
  \endfirsthead
  \toprule
  \SimHei \normalsize 年数 & \SimHei \scriptsize 公元 & \SimHei 大事件 \tabularnewline
  \midrule
  \endhead
  \midrule
  元年 & 401 & \tabularnewline\hline
  二年 & 402 & \tabularnewline\hline
  三年 & 403 & \tabularnewline
  \bottomrule
\end{longtable}


%%% Local Variables:
%%% mode: latex
%%% TeX-engine: xetex
%%% TeX-master: "../../Main"
%%% End:


%%% Local Variables:
%%% mode: latex
%%% TeX-engine: xetex
%%% TeX-master: "../../Main"
%%% End:
