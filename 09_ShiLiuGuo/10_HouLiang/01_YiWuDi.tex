%% -*- coding: utf-8 -*-
%% Time-stamp: <Chen Wang: 2021-11-01 12:02:38>

\subsection{懿武帝呂光\tiny(386-399)}

\subsubsection{生平}

涼懿武帝呂光(338年-399年),字世明,略陽(今甘肅天水)氐人,前秦太尉呂婆樓之子。十六國時期後涼建立者。呂光初為前秦將領,屢立戰功,前秦天王苻堅就派了他出兵西域。呂光降服西域,但當時前秦因淝水之戰戰敗而國亂,回軍時為涼州刺史梁熙所阻,呂光消滅了梁熙而入主涼州,遂在當地建立政權。

呂光得王猛看重,並將他推薦給苻堅,苻堅於是以呂光為美陽令,任內呂光得當地人民愛戴信服。呂光後遷鷹揚將軍,以功封關內侯,並於永興二年(358年)隨苻堅等討伐張平。苻堅與張平於銅壁決戰,張平驍勇大力的養子張蚝單騎屢次進出前秦軍陣中,呂光於是去襲擊張蚝並成功擊傷他。張蚝受傷被擒,張平潰敗,呂光亦因而聲名大噪。

建元四年(368年),呂光與王鑒等因應楊成世討伐上邽叛變的苻雙失敗而率軍再行討伐,王鑒到後打算與苻雙前鋒苟興速戰速決,但呂光慮及對方因剛獲勝而士氣高漲,建議謹慎待敵,讓其糧盡退兵時就是進攻的時機。二十日後苟興退兵,王鑒追擊並擊敗苟興,隨後又大敗苻雙,終攻下上邽,斬殺苻雙。建元六年(370年),呂光隨軍攻滅前燕,獲封都亭侯。後苻重出鎮洛陽,呂光擔任其長史。苻重於建元十四年(378年)謀反,苻堅以呂光忠誠正直,不會與苻重連謀,於是下令呂光收捕苻重,呂光聽命並以檻車押送苻重回長安。後呂光遷太子右率,頗受敬重。次年呂光又以破虜將軍身份率兵擊敗進攻成都的李烏,遷步兵校尉。建元十六年(380年)呂光又奉命與左將軍竇衝共領四萬兵討伐叛亂的苻重,又將其生擒,戰後獲授驍騎將軍。

前秦十八年(382年),呂光受命征討西域,以使持節都督西討諸軍事身份率領姜飛等將領、七萬兵及五千鐵騎出發。呂光越過三百多里長的沙漠到達西域,降服焉耆等西域各國,又擊破唯一拒守的龜茲,威震西域。苻堅知呂光征服西域,即任命其為使持節、散騎常侍、都督玉門以西諸軍事、安西將軍、西域校尉,封順鄉侯,但因前秦於淝水之戰後國內大亂而道路不通,未能傳達。呂光本來想要留在龜茲,但是受到名僧鳩摩羅什勸阻,而且部眾們也想回到中原,遂回師。

太安元年(385年),呂光軍抵宜禾(今新疆安西南),高昌太守楊翰告訴涼州刺史梁熙,稱呂光還軍必定別有所圖,建議關閉天險要道,拒之於外,但梁熙沒有聽從。呂光最初知道楊翰的計劃時曾打算不再前進,但在杜進勸告下還是繼續,楊翰即在呂光到達高昌時向呂光請降。梁熙在呂光到遠玉門時傳檄指責呂光擅自班師,又派其子梁胤等率軍五萬往酒泉阻擊呂光。呂光也傳檄指責梁熙沒有為前秦赴國難的忠誠,還阻攔歸國軍隊,並派了姜飛等為前鋒進攻梁胤。姜飛等在安彌大破梁胤並生擒他,於是周邊外族都紛紛依附呂光,武威太守彭濟更將梁熙抓起來叛歸呂光。呂光殺死梁熙,入主姑臧,自領涼州刺史、護羌校尉。

386年,呂光收到苻堅死訊,改元太安,並自稱使持節、侍中、中外大都督、督隴右河西諸軍事、大將軍、涼州牧、酒泉公。呂光入主涼州時,因尉祐與彭濟共謀抓住梁熙的功勞而寵任他,但呂光卻在尉祐中傷下殺了姚皓、尹景等十多個名士,人心見離。當時國內米價也高漲至一斗五百,饑荒中更發生人吃人事件,死了很多人。呂光與群僚在飲宴中談及為政時用嚴峻刑法的問題,在參軍段業勸言下終下令自省並行寬簡之政。

呂光於太安二年(387年)殺了進逼姑臧的張大豫,但王穆尚據酒泉;西平太守康寧也叛變,阻兵據守,呂光試圖討伐但都不果。及後連呂光部將徐炅及張掖太守彭晃都謀叛,並聯結了王穆及康寧。呂光力排眾議親率三萬兵速攻彭晃,二十日後攻破張掖,殺了彭晃。不久,呂光乘王穆進攻其將索嘏的機會率二萬兵襲破酒泉,王穆率兵東返但部眾在途中就潰散,王穆隻身逃走但為騂馬令郭文所殺。

389年,呂光稱三河王,改元麟嘉。396年六月又改稱天王,國號大涼,改元龍飛。呂光曾先後多次進攻西秦,其中呂光弟呂延於龍飛二年(397年)的進攻中兵敗被殺。呂光聽信讒言,怪罪從軍的尚書沮渠羅仇及三河太守沮渠麴粥,並殺二人。二人歸葬時,因諸部聯姻而共計有萬多人參與葬禮,羅仇之侄沮渠蒙遜遂反,蒙遜堂兄沮渠男成舉兵響應,並推建康太守段業為主,建北涼與後涼對抗,呂光曾派呂纂討伐,但最終無法消滅北涼。

同年,善於天文術數的太常郭黁與僕射王詳認為呂光年老、太子闇弱而呂纂等凶悍,料定呂光死後必會有禍亂,並禍及自己,故圖謀攻奪姑臧東西苑城,推王乞基為主。不過王詳因事泄而被殺,郭黁遂據東苑叛變,當時民間還有很多人支持郭黁。呂光召呂纂回兵討伐郭黁,呂纂遂屢破郭黁,令其於龍飛三年(398年)出走西秦,平定亂事。

龍飛四年(399年),呂光病重,立太子呂紹為天王,自號太上皇帝(太上天王)。呂光又讓呂纂及呂弘分任太尉及司徒,告誡呂紹要倚重二人,放權讓他們處理軍政大事才能保國家安穩;另也對呂纂及呂弘說二人要與天王呂紹同心合力才能保全國家,否則禍亂必會來。呂光於不久去世,享年六十三歲,諡懿武皇帝,廟號太祖。

呂光年輕時已展現其軍事能力,十歲時與其他小童一起玩耍時就創制戰爭陣法,於是同年的人都推其為主,而呂光處事平允,更令眾小童佩服。呂光也不喜歡讀書,只好打獵。

呂光高八尺四寸,雙目重瞳,為人沈著堅毅,凝重且寛大有度量,喜怒不形於色,故王猛賞識他,稱:「此非常人。」

呂光出生於枋頭(今河南浚縣西南),當夜有神光,全家覺得奇怪,遂以光为名。

呂光左肘有一肉印,據說在一次戰爭中肉印隱約顯出「巨霸」兩字。

\subsubsection{太安}

\begin{longtable}{|>{\centering\scriptsize}m{2em}|>{\centering\scriptsize}m{1.3em}|>{\centering}m{8.8em}|}
  % \caption{秦王政}\
  \toprule
  \SimHei \normalsize 年数 & \SimHei \scriptsize 公元 & \SimHei 大事件 \tabularnewline
  % \midrule
  \endfirsthead
  \toprule
  \SimHei \normalsize 年数 & \SimHei \scriptsize 公元 & \SimHei 大事件 \tabularnewline
  \midrule
  \endhead
  \midrule
  元年 & 386 & \tabularnewline\hline
  二年 & 387 & \tabularnewline\hline
  三年 & 388 & \tabularnewline\hline
  四年 & 389 & \tabularnewline
  \bottomrule
\end{longtable}

\subsubsection{麟嘉}

\begin{longtable}{|>{\centering\scriptsize}m{2em}|>{\centering\scriptsize}m{1.3em}|>{\centering}m{8.8em}|}
  % \caption{秦王政}\
  \toprule
  \SimHei \normalsize 年数 & \SimHei \scriptsize 公元 & \SimHei 大事件 \tabularnewline
  % \midrule
  \endfirsthead
  \toprule
  \SimHei \normalsize 年数 & \SimHei \scriptsize 公元 & \SimHei 大事件 \tabularnewline
  \midrule
  \endhead
  \midrule
  元年 & 389 & \tabularnewline\hline
  二年 & 390 & \tabularnewline\hline
  三年 & 391 & \tabularnewline\hline
  四年 & 392 & \tabularnewline\hline
  五年 & 393 & \tabularnewline\hline
  六年 & 394 & \tabularnewline\hline
  七年 & 395 & \tabularnewline\hline
  八年 & 396 & \tabularnewline
  \bottomrule
\end{longtable}

\subsubsection{龙飞}

\begin{longtable}{|>{\centering\scriptsize}m{2em}|>{\centering\scriptsize}m{1.3em}|>{\centering}m{8.8em}|}
  % \caption{秦王政}\
  \toprule
  \SimHei \normalsize 年数 & \SimHei \scriptsize 公元 & \SimHei 大事件 \tabularnewline
  % \midrule
  \endfirsthead
  \toprule
  \SimHei \normalsize 年数 & \SimHei \scriptsize 公元 & \SimHei 大事件 \tabularnewline
  \midrule
  \endhead
  \midrule
  元年 & 396 & \tabularnewline\hline
  二年 & 397 & \tabularnewline\hline
  三年 & 398 & \tabularnewline\hline
  四年 & 399 & \tabularnewline
  \bottomrule
\end{longtable}

%%% Local Variables:
%%% mode: latex
%%% TeX-engine: xetex
%%% TeX-master: "../../Main"
%%% End:
