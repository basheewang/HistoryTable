%% -*- coding: utf-8 -*-
%% Time-stamp: <Chen Wang: 2019-12-19 15:51:37>

\subsection{吕隆\tiny(401-403)}

\subsubsection{生平}

呂隆(4世紀?-416年),字永基,略陽(今甘肅天水)氐人。十六國時期後涼最後一位君主,後涼開國君主呂光之弟呂寶子。呂隆即位不久即遭後秦攻擊,被逼向後秦請降,其在位時間亦不斷遭南涼及北涼二國攻擊,國力大衰,最終呂隆向後秦請求迎其東遷,後涼遂為後秦所併。

呂隆長得俊美,擅長騎射。呂光時曾任北部護軍。咸寧三年(401年),呂隆弟呂超以兵變弒殺天王呂纂,隨後就擁立呂隆。呂隆面有難色,但呂超說:「現在就好像騎著龍飛在天上,豈可以中途下來!」呂隆於是登位,改元神鼎。

呂隆登位後多殺豪望以圖立威,反不得人心,令人人自危。魏安人焦朗遂招請後秦將領姚碩德攻涼,姚碩德聽從並率軍進攻,兵臨姑臧。呂隆派了呂超及呂邈抵抗但大敗而還,呂邈更戰死,呂隆只得嬰城固守。不過,後秦軍接著數月的圍困令城中原來自東面的人圖謀叛變,將軍魏益多更煽動人們殺呂隆及呂超,呂隆遂在事件被揭發後誅殺共三百多家人。當時後涼群臣勸呂隆和後秦請和,呂隆原本不肯,但在呂超勸諫下向後秦請降。姚碩德於是表呂隆為鎮西大將軍、涼州刺史、建康公。

神鼎二年(402年),北涼沮渠蒙遜率兵進攻姑臧,呂隆請得南涼將禿髮傉檀援救,但傉檀未到呂隆就擊敗蒙遜。蒙遜於是與呂隆結盟,並留下萬多斛穀。但其時姑臧穀價已經高達五千文一斗,發生人吃人事件,死了十多萬人。百姓因為姑臧整天關上城門而無法出城找食物,於是每日都有數百人請求出城當別人奴婢以求生,呂隆怕他們會動搖人心,遂將這些人都盡數殺害,屍體堆滿路上。然而,接著後涼仍不斷受到北涼及南涼攻擊,呂隆被逼於神鼎三年(403年)借後秦徵呂超入侍的機會命其帶著珍寶,請後秦派兵迎其離開。秦將齊難等於該年八月到達姑臧,呂隆率眾隨之東遷長安,呂隆獲後秦授散騎常侍,後涼至此滅亡。後秦弘始十八年(416年),受後秦皇帝姚興子廣平公姚弼謀反案牽連,被殺。

\subsubsection{神鼎}

\begin{longtable}{|>{\centering\scriptsize}m{2em}|>{\centering\scriptsize}m{1.3em}|>{\centering}m{8.8em}|}
  % \caption{秦王政}\
  \toprule
  \SimHei \normalsize 年数 & \SimHei \scriptsize 公元 & \SimHei 大事件 \tabularnewline
  % \midrule
  \endfirsthead
  \toprule
  \SimHei \normalsize 年数 & \SimHei \scriptsize 公元 & \SimHei 大事件 \tabularnewline
  \midrule
  \endhead
  \midrule
  元年 & 401 & \tabularnewline\hline
  二年 & 402 & \tabularnewline\hline
  三年 & 403 & \tabularnewline
  \bottomrule
\end{longtable}


%%% Local Variables:
%%% mode: latex
%%% TeX-engine: xetex
%%% TeX-master: "../../Main"
%%% End:
