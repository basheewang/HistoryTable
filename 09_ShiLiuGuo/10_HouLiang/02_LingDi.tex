%% -*- coding: utf-8 -*-
%% Time-stamp: <Chen Wang: 2021-11-01 14:51:50>

\subsection{灵帝呂紹\tiny(399-401)}

\subsubsection{隐王生平}

涼隱王呂紹(380年代-399年),字永業,略陽(今甘肅天水)氐人。十六國時期後涼國第二任君主,後涼懿武帝呂光嫡子。呂紹登位不久即被呂纂及呂弘兩位兄長發動政變所推翻,呂紹自殺。

呂光出征西域時,呂紹與石氏等人留在前秦。淝水之戰後,前秦因戰敗而國亂,長安亦構亂,呂紹等人於是出奔仇池,直至麟嘉元年(389年)才到後涼,甫稱三河王的呂光遂立吕紹為世子。龍飛元年(396年),呂光立其為太子。

吕绍唯一一次有记载的亲自指挥的军事行动在龍飛四年(399年),当时他与庶兄吕纂攻打北凉天王段业,段业求助于南凉天王秃发乌孤。秃发乌孤的弟弟秃发利鹿孤率援军赶到,吕绍和吕纂只得撤退。

同年年末,呂光病重,立呂紹為天王,以吕纂为太尉,吕弘為司徒,臨終前叮囑呂紹說:「如今三寇(乞伏乾歸、段業和禿髮烏孤)未平,我死之後,呂纂帶領軍隊,呂弘治理朝政,你自己無為而治,把重任交給兩個哥哥」。也对两位长子有所嘱咐:“永业并非治理乱世的人才,只不过因嫡长的规举才让其处元首之位。现在外有强寇,人心不定,你们兄弟和睦则会让国家流传万世;若果自己内斗,则祸乱立即就会来了。”还对吕纂说:“你本性粗豪勇武,很令我担心。开展基业本来就艰难,守成也不容易。好好辅助永业,不要听谗言呀。”不久呂光去世,呂紹秘不發喪,呂纂推門入殿哭喪,竭盡哀思才出來。呂紹害怕被殺害,想讓位給他,但呂纂以呂紹是嫡子身份推辭,呂紹固請也不獲呂纂答允,於是即位。呂光侄子呂超勸呂紹及早除去既有兵權,又有極高威名的呂纂,但呂紹雖也憂心呂纂,但仍以父親遺命及袁尚兄弟相爭之事一再拒絕對付呂纂,令呂超很失望。呂紹在湛露堂面見呂纂時,呂超持刀在側侍候,用眼神請求呂紹收捕呂纂,但呂紹都不肯。

在呂紹到後涼前,呂光曾經想立呂弘為世子,不過因為知道呂紹在仇池而打消念頭。可是呂弘一直記恨在心,不久即派尚書姜紀唆使呂纂和他一起叛變。呂纂順從,於是在一夜率軍攻入宮廷,呂紹試圖出兵抵抗,但兵眾都因為忌憚呂纂威名而潰散。呂紹見此便在紫閣自殺。呂纂即位後諡呂紹為隱王。

\subsubsection{灵帝生平}

涼靈帝呂纂(4世紀?-401年),字永緒,略陽(今甘肅天水)氐人。十六國時期後涼國君主,後涼開國君主呂光庶長子,母親是趙淑媛,隱王呂紹兄。呂纂在呂光死後不久即以政變逼死呂紹登位,但在位一年多就在呂超等人的變亂被殺。

呂纂年少時已熟練弓馬,雖然入了太學,但不愛讀書,只會交結公侯。淝水之戰後前秦國亂,呂纂逃到上邽(今甘肅天水市),至太安元年(386年)才到達後涼都城姑臧(今甘肅武威市),拜虎賁中郎將。麟嘉四年(392年),呂光派了呂纂進攻南羌彭奚念,但在盤夷大敗而還。呂光遂親率大軍再攻,讓呂纂及楊軌、沮渠羅仇進軍左南(今青海西寧市東),逼得彭奚念憑湟河自守,然呂光還是派兵渡過湟河,攻下枹罕(今甘肅臨夏市),令彭奚念敗走甘松(今甘肅叠部縣東南)。

龍飛元年(396年),呂光稱天王,以呂纂為太原公。次年,呂光攻西秦,派呂纂、楊軌及竇苟等率三萬兵攻金城(今甘肅蘭州市),攻陷了金城。同年,呂光殺沮渠羅仇及沮渠麴粥,令得羅仇侄沮渠蒙遜反叛。蒙遜堂兄沮渠男成也推了建康太守段業為主,呂纂奉命討伐段業,然而因為沮渠蒙遜率眾到臨洮為聲援段業,呂纂在合離大敗給段業。同時,太常郭黁在姑臧作亂,呂光立即召回呂纂,當時諸將顧慮段業會乘大軍撤退而從後跟隨,建議乘夜暗中撤走,不過呂纂看准段業無謀略,乘夜退走只會助長敵人,於是在退兵時前派了使者向段業說:「郭黁作亂,吾今還都。卿能決者,可出戰。」段業果然不敢追擊。郭黁派軍於白石邀擊呂纂,呂纂大敗,但不久因西安太守石元良率兵援救才得以擊敗郭黁,攻入姑臧。呂纂隨後在城西擊破郭黁將王斐,令郭黁勢力開始衰敗。不過郭黁卻推了楊軌為盟主,讓楊軌前赴姑臧支援自己。時呂弘為段業所逼,呂纂就前去迎接呂弘,楊軌認為這是機會,於是率兵邀擊,但卻為呂纂所敗,郭黁於是出奔西秦,楊軌隨後亦奔廉川,亂事終告平定。

龍飛四年(399年),吕纂与吕绍一同统兵攻打北凉天王段业,段业求救于南凉天王秃发乌孤,秃发乌孤之弟秃发利鹿孤率援军赶到,段业坚守不战,吕纂、吕绍于是退兵。

同年,呂光病重,立呂紹為天王,以呂纂為太尉,掌握軍權。呂光死前曾向呂纂及呂弘說:「永業並非治理亂世的人才,只不過因嫡長的規舉才讓其處元首之位。現在外有強寇,人心不定,你們兄弟和睦則會讓國家流傳萬世;若果自己內鬥,則禍亂立即就會來了。」另也特別對呂纂說:「你本性粗豪勇武,很令我擔心。開展基業本來就艱難,守成也不容易。好好輔助永業,不要聽讒言呀。」不久呂光去世,呂紹懼怕呂纂,曾經想要讓位給呂纂,然而呂纂以嫡庶之別拒絕;另呂光侄呂超又勸呂紹殺了呂纂,但呂紹不肯。可是不久吕纂就在呂弘的煽動下反叛,夜裏率壯士數百進攻廣夏門,守融明觀的齊從抽劍攻擊呂纂,擊中其額,但為呂纂部眾制服。呂紹所派部隊因懼怕呂纂而潰散,吕紹被逼自殺。呂纂遂即天王位,改年號咸寧。

咸寧二年(400年),呂弘舉兵反叛,但為呂纂將焦辨擊敗,出奔廣武(今甘肅永登縣),不久為呂方所捕,被殺。呂纂隨後縱兵大掠,以原屬呂弘的東苑中之婦女賞給軍士,呂弘的妻兒都被士兵侵辱。呂纂笑著對群臣說:「今日一戰怎樣呀?」侍中房晷卻答:「天要降禍給涼室,故藩王起兵釁。先帝駕崩不久,隱王幽逼而死,山陵才剛建好,大司馬就因驚懼疑惑而反叛肆逆,京邑成了兄弟交戰的戰場。雖然呂弘自取滅亡,亦是因為陛下沒有棠棣所說的兄弟之義。現在應該反思自省,以為向百姓謝過,卻反而縱容士兵大肆掠奪,侮辱士女。兵釁因呂弘而起,百姓有甚麼錯!而且呂弘的妻子是陛下的弟婦,女兒也是陛下的姪女,怎能讓她們成為無賴小人的婢妾。天地神明怎會忍心見到這樣!」呂纂聽後向房晷道歉,又接回呂弘的妻兒到東宮。

隨後,呂纂不顧中書令楊穎反對堅決攻伐南涼,卻為南涼將禿髮傉檀所敗。呂纂不久又不聽姜紀諫言而攻北涼,圍攻張掖(今甘肅張掖)並攻略建康郡地,然而禿髮傉檀果如姜紀所言進攻姑臧,呂纂亦被逼退兵。呂纂在位時沉溺於酒色,又常常出獵,諸大臣皆曾勸阻,然而呂纂皆不能聽從。

咸寧三年(401年)呂纂因番禾太守呂超擅攻鮮卑思盤一事召呂超及思盤入朝,呂超因恐懼而事先結交了殿中監杜尚。呂纂憤怒地斥責呂超,更聲言「要當斬卿,然後天下可定」,嚇得呂超叩頭稱不敢。不過呂纂及後就和呂超及眾大臣宴會,呂超兄呂隆於是頻頻向呂纂勸酒要灌醉他。呂纂飲至昏醉便乘坐步輓車與呂超等人在宮內遊走,在到琨華殿東閤時步輓車過不了去,呂纂親將竇川及駱騰於是放下配劍推車。呂超乘此機會拿起二人配劍襲擊呂纂,呂纂試圖下車抓住呂超但被對方刺穿胸部;呂超又殺了竇川和駱騰。呂纂后楊氏下令禁軍討伐呂超,但杜尚卻命禁軍放下武器。將軍魏益多遂斬下呂纂的頭,聲言:「呂纂違反先帝遺命,殺害太子、沉溺飲酒和田獵、親近小人、輕易殺害忠良、視百姓為草芥。番禾太守呂超以骨肉之親,恐懼國家傾覆,已經除去他了。上可以安寧宗廟,下可為太子報仇。但凡國人都應歡慶。」

呂隆不久繼位,諡呂纂為靈皇帝,葬白石陵。

即序胡安據曾盜張駿的墓,獲得大量珍寶,呂纂誅殺安據和其親黨五十多家人,派使者弔祭張駿,並復修其陵墓。

咸寧二年,有母豬生下小豬,一身三頭,又有飛龍夜裡從東廂的井中出現,名僧鳩摩羅什以為不祥,勸纂廣施仁德。一日羅什與呂纂玩博戲,呂纂吃多子,玩笑道:“砍胡奴頭!”羅什糾正說:“不斫胡奴頭,胡奴斫人頭。”預言了呂纂因小字「胡奴」的呂超而被殺的命運。

\subsubsection{咸宁}

\begin{longtable}{|>{\centering\scriptsize}m{2em}|>{\centering\scriptsize}m{1.3em}|>{\centering}m{8.8em}|}
  % \caption{秦王政}\
  \toprule
  \SimHei \normalsize 年数 & \SimHei \scriptsize 公元 & \SimHei 大事件 \tabularnewline
  % \midrule
  \endfirsthead
  \toprule
  \SimHei \normalsize 年数 & \SimHei \scriptsize 公元 & \SimHei 大事件 \tabularnewline
  \midrule
  \endhead
  \midrule
  元年 & 399 & \tabularnewline\hline
  二年 & 400 & \tabularnewline\hline
  三年 & 401 & \tabularnewline
  \bottomrule
\end{longtable}


%%% Local Variables:
%%% mode: latex
%%% TeX-engine: xetex
%%% TeX-master: "../../Main"
%%% End:
