%% -*- coding: utf-8 -*-
%% Time-stamp: <Chen Wang: 2019-12-19 16:28:26>

\subsection{李歆\tiny(417-420)}

\subsubsection{生平}

李\xpinyin*{歆}(?-420年),字士業,小字桐椎,陇西狄道(今甘肃临洮县)人,十六國西涼公,為李暠世子。其第三子李重耳是李唐王朝皇室的直系祖先。

西涼建初十三年(417年),李暠過世,李歆被部下擁護為大都督、大將軍、涼公、涼州牧,改元嘉興。李歆在位時,繼承其父稱臣於東晉的政策,因此東晉封其為酒泉公。

李歆用刑頗嚴,又喜歡建築宮殿,臣屬多有勸諫,然而李歆並不能接納。嘉興四年(420年),北涼佯攻西秦以誘西涼,李歆因此出兵攻擊,戰敗被殺。

\subsubsection{建兴}

\begin{longtable}{|>{\centering\scriptsize}m{2em}|>{\centering\scriptsize}m{1.3em}|>{\centering}m{8.8em}|}
  % \caption{秦王政}\
  \toprule
  \SimHei \normalsize 年数 & \SimHei \scriptsize 公元 & \SimHei 大事件 \tabularnewline
  % \midrule
  \endfirsthead
  \toprule
  \SimHei \normalsize 年数 & \SimHei \scriptsize 公元 & \SimHei 大事件 \tabularnewline
  \midrule
  \endhead
  \midrule
  元年 & 417 & \tabularnewline\hline
  二年 & 418 & \tabularnewline\hline
  三年 & 419 & \tabularnewline\hline
  四年 & 420 & \tabularnewline
  \bottomrule
\end{longtable}


%%% Local Variables:
%%% mode: latex
%%% TeX-engine: xetex
%%% TeX-master: "../../Main"
%%% End:
