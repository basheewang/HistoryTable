%% -*- coding: utf-8 -*-
%% Time-stamp: <Chen Wang: 2019-12-19 16:28:45>

\subsection{李恂\tiny(420-421)}

\subsubsection{生平}

李恂(?-421年),字士如,陇西狄道(今甘肃临洮县)人,十六國時期西涼的君主,凉武昭王李暠第五子,李歆之弟,李歆在位時任敦煌太守。

西涼嘉興四年(420年),北涼敗西涼軍殺李歆,隨即攻佔西涼都城酒泉,李恂及其他諸弟逃往北山。數月後,因北涼王沮渠蒙遜所派敦煌太守索元緒凶險好殺,大失人心,而李恂在敦煌施政名聲卓著,敦煌人民遂密迎李恂,李恂率數十騎入敦煌,索元緒逃走,李恂被推為冠軍將軍、涼州刺史,改元永建。不久,沮渠蒙遜派軍討伐。隔年(421年),北涼軍引水灌敦煌,李恂乞降不成,部下投降,李恂遂自殺,西涼亦亡。

\subsubsection{永建}

\begin{longtable}{|>{\centering\scriptsize}m{2em}|>{\centering\scriptsize}m{1.3em}|>{\centering}m{8.8em}|}
  % \caption{秦王政}\
  \toprule
  \SimHei \normalsize 年数 & \SimHei \scriptsize 公元 & \SimHei 大事件 \tabularnewline
  % \midrule
  \endfirsthead
  \toprule
  \SimHei \normalsize 年数 & \SimHei \scriptsize 公元 & \SimHei 大事件 \tabularnewline
  \midrule
  \endhead
  \midrule
  元年 & 420 & \tabularnewline\hline
  二年 & 421 & \tabularnewline
  \bottomrule
\end{longtable}


%%% Local Variables:
%%% mode: latex
%%% TeX-engine: xetex
%%% TeX-master: "../../Main"
%%% End:
