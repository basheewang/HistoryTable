%% -*- coding: utf-8 -*-
%% Time-stamp: <Chen Wang: 2019-12-19 16:26:47>


\section{西凉\tiny(400-417)}

\subsection{简介}

西涼(400年—421年)是十六國之一。

400年李暠在敦煌郡称“凉公”。405年遷都酒泉郡,逼近北涼。疆域在今中國甘肅西部及新疆部分。417年,李暠卒,子李歆嗣位。420年,李歆與北涼交戰被殺,其弟敦煌太守李恂在敦煌嗣位。但次年,北涼軍圍敦煌,李恂戰敗,乞降不成後自殺。西涼因此亡於北涼。二十余年之后,李恂的侄子李宝趁北魏攻灭北凉之际,一度恢复先人的基业,同年向北魏投诚,该政权被称为后西凉。

因其统治地区古为凉州,故国号为“凉”,又位于凉州西部,故名“西凉”。

%% -*- coding: utf-8 -*-
%% Time-stamp: <Chen Wang: 2021-11-01 14:54:20>

\subsection{武昭王李暠\tiny(400-417)}

\subsubsection{生平}

涼武昭王李\xpinyin*{暠}(351年-417年),字玄盛,小字長生,陇西郡狄道县(今甘肃省定西市临洮县)人,是李昶的遺腹子,十六國時期西涼的建立者。自稱是西漢將領李廣之十六世孫。李暠的后代形成了陇西李氏的镇远将军房、平凉房、武阳房、姑臧房、敦煌房、仆射房和绛郡房,唐朝皇室和诗人李白亦稱李暠為其先祖。天宝二年(743),唐玄宗追尊李暠为兴圣皇帝。

李暠年少好學,性格寬和,讀遍經史,尤能理解文章的義理。李暠長大後也學習武藝,讀孫吳兵法。北涼神璽元年(397年),後涼建康太守段業自立,次年孟敏降北涼獲授沙州刺史,以李暠為效穀縣令。不久孟敏去世,敦煌护军郭谦及沙洲治中索仙認為李暠在任縣令期間治績頗可取,故推舉李暠为敦煌太守及甯朔將軍,李暠遂向段業請命,獲授安西將軍、敦煌太守,領護西胡校尉。後來北涼右衞將軍索嗣向段業中傷李暠,讓其改以自己擔當敦煌太守。索嗣率五百騎到敦煌外二十里時才通告李暠去迎接自己,李暠聞訊驚訝疑惑,一度想順從出迎,但在張邈及宋繇勸阻下改為派兵抵抗,索嗣兵敗退還張掖。李暠昔日與索嗣十分友好,但知道他在段業面前中傷他並奪去其官位後就相當痛恨他,遂上陳索嗣罪狀。在沮渠男成的勸說下,段業就將索嗣殺了,並派使者向李暠陳謝,又分劃出涼興郡,進李暠為持節,都督涼興以西諸軍事、鎮西將軍,領西夷校尉。

北涼天璽二年(400年),晉昌太守唐瑤移檄六郡,推李暠為大都督、大將軍、護羌校尉、領秦涼二州牧、涼公,改元庚子,以敦煌為都城,建立西涼。李暠又派兵東伐涼興,又西攻玉門西諸城,令疆域廣及西域。次年,北涼將沮渠蒙遜殺段業,自立為北涼君主,李暠又派唐瑤攻酒泉,擒北涼酒泉太守沮渠益生。建初元年(405年),李暠改元並遣使奉表於晉,又遷都酒泉,與北涼長期爭戰。

李暠立國以後鼓勵農事,為對抗北涼積聚軍資,亦令百姓安居樂業。他亦喜好讀書,因此在位時注重文化教育,境內文風頗盛。

建初十三年(417年),李暠過世,享年六十七歲,臨終遺命宋繇輔助諸子。西涼諡武昭王,廟號太祖,次子李歆繼位。

\subsubsection{庚子}

\begin{longtable}{|>{\centering\scriptsize}m{2em}|>{\centering\scriptsize}m{1.3em}|>{\centering}m{8.8em}|}
  % \caption{秦王政}\
  \toprule
  \SimHei \normalsize 年数 & \SimHei \scriptsize 公元 & \SimHei 大事件 \tabularnewline
  % \midrule
  \endfirsthead
  \toprule
  \SimHei \normalsize 年数 & \SimHei \scriptsize 公元 & \SimHei 大事件 \tabularnewline
  \midrule
  \endhead
  \midrule
  元年 & 400 & \tabularnewline\hline
  二年 & 401 & \tabularnewline\hline
  三年 & 402 & \tabularnewline\hline
  四年 & 403 & \tabularnewline\hline
  五年 & 404 & \tabularnewline
  \bottomrule
\end{longtable}

\subsubsection{建初}

\begin{longtable}{|>{\centering\scriptsize}m{2em}|>{\centering\scriptsize}m{1.3em}|>{\centering}m{8.8em}|}
  % \caption{秦王政}\
  \toprule
  \SimHei \normalsize 年数 & \SimHei \scriptsize 公元 & \SimHei 大事件 \tabularnewline
  % \midrule
  \endfirsthead
  \toprule
  \SimHei \normalsize 年数 & \SimHei \scriptsize 公元 & \SimHei 大事件 \tabularnewline
  \midrule
  \endhead
  \midrule
  元年 & 405 & \tabularnewline\hline
  二年 & 406 & \tabularnewline\hline
  三年 & 407 & \tabularnewline\hline
  四年 & 408 & \tabularnewline\hline
  五年 & 409 & \tabularnewline\hline
  六年 & 410 & \tabularnewline\hline
  七年 & 411 & \tabularnewline\hline
  八年 & 412 & \tabularnewline\hline
  九年 & 413 & \tabularnewline\hline
  十年 & 414 & \tabularnewline\hline
  十一年 & 415 & \tabularnewline\hline
  十二年 & 416 & \tabularnewline\hline
  十三年 & 417 & \tabularnewline
  \bottomrule
\end{longtable}


%%% Local Variables:
%%% mode: latex
%%% TeX-engine: xetex
%%% TeX-master: "../../Main"
%%% End:

%% -*- coding: utf-8 -*-
%% Time-stamp: <Chen Wang: 2021-11-01 14:54:43>

\subsection{酒泉公李歆\tiny(417-420)}

\subsubsection{生平}

李\xpinyin*{歆}(?-420年),字士業,小字桐椎,陇西狄道(今甘肃临洮县)人,十六國西涼公,為李暠世子。其第三子李重耳是李唐王朝皇室的直系祖先。

西涼建初十三年(417年),李暠過世,李歆被部下擁護為大都督、大將軍、涼公、涼州牧,改元嘉興。李歆在位時,繼承其父稱臣於東晉的政策,因此東晉封其為酒泉公。

李歆用刑頗嚴,又喜歡建築宮殿,臣屬多有勸諫,然而李歆並不能接納。嘉興四年(420年),北涼佯攻西秦以誘西涼,李歆因此出兵攻擊,戰敗被殺。

\subsubsection{建兴}

\begin{longtable}{|>{\centering\scriptsize}m{2em}|>{\centering\scriptsize}m{1.3em}|>{\centering}m{8.8em}|}
  % \caption{秦王政}\
  \toprule
  \SimHei \normalsize 年数 & \SimHei \scriptsize 公元 & \SimHei 大事件 \tabularnewline
  % \midrule
  \endfirsthead
  \toprule
  \SimHei \normalsize 年数 & \SimHei \scriptsize 公元 & \SimHei 大事件 \tabularnewline
  \midrule
  \endhead
  \midrule
  元年 & 417 & \tabularnewline\hline
  二年 & 418 & \tabularnewline\hline
  三年 & 419 & \tabularnewline\hline
  四年 & 420 & \tabularnewline
  \bottomrule
\end{longtable}


%%% Local Variables:
%%% mode: latex
%%% TeX-engine: xetex
%%% TeX-master: "../../Main"
%%% End:

%% -*- coding: utf-8 -*-
%% Time-stamp: <Chen Wang: 2019-12-19 16:28:45>

\subsection{李恂\tiny(420-421)}

\subsubsection{生平}

李恂(?-421年),字士如,陇西狄道(今甘肃临洮县)人,十六國時期西涼的君主,凉武昭王李暠第五子,李歆之弟,李歆在位時任敦煌太守。

西涼嘉興四年(420年),北涼敗西涼軍殺李歆,隨即攻佔西涼都城酒泉,李恂及其他諸弟逃往北山。數月後,因北涼王沮渠蒙遜所派敦煌太守索元緒凶險好殺,大失人心,而李恂在敦煌施政名聲卓著,敦煌人民遂密迎李恂,李恂率數十騎入敦煌,索元緒逃走,李恂被推為冠軍將軍、涼州刺史,改元永建。不久,沮渠蒙遜派軍討伐。隔年(421年),北涼軍引水灌敦煌,李恂乞降不成,部下投降,李恂遂自殺,西涼亦亡。

\subsubsection{永建}

\begin{longtable}{|>{\centering\scriptsize}m{2em}|>{\centering\scriptsize}m{1.3em}|>{\centering}m{8.8em}|}
  % \caption{秦王政}\
  \toprule
  \SimHei \normalsize 年数 & \SimHei \scriptsize 公元 & \SimHei 大事件 \tabularnewline
  % \midrule
  \endfirsthead
  \toprule
  \SimHei \normalsize 年数 & \SimHei \scriptsize 公元 & \SimHei 大事件 \tabularnewline
  \midrule
  \endhead
  \midrule
  元年 & 420 & \tabularnewline\hline
  二年 & 421 & \tabularnewline
  \bottomrule
\end{longtable}


%%% Local Variables:
%%% mode: latex
%%% TeX-engine: xetex
%%% TeX-master: "../../Main"
%%% End:


%%% Local Variables:
%%% mode: latex
%%% TeX-engine: xetex
%%% TeX-master: "../../Main"
%%% End:
