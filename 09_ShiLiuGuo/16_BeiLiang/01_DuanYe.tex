%% -*- coding: utf-8 -*-
%% Time-stamp: <Chen Wang: 2019-12-19 16:39:00>

\subsection{段业\tiny(397-401)}

\subsubsection{生平}

段業(?-401年),京兆郡(治今陝西西安)漢人。十六国时期北涼国開國君主,但其本身只是為盧水胡沮渠蒙遜及沮渠男成所推,他也很忌憚沮渠蒙遜,蒙遜亦十分不安,最終沮渠蒙遜發動兵變推翻並殺害段業。

段業博覽史傳,有文辭才學,原是前秦將領吕光部將杜進僚屬,從征西域。後呂光建後涼,出任建康太守。後涼龍飛二年(397年),沮渠蒙遜叛後涼,其堂兄沮渠男成亦叛,並進攻段業所守的建康。男成派使者勸段業支持自己,段業最初不肯,但在男成圍困二十日後還是答應了,遂被推為大都督、龍驤大將軍、涼州牧、建康公,改年號神璽。

神璽元年(397年)以沮渠男成為輔國將軍、沮渠男成的堂弟沮渠蒙遜為張掖太守,委以軍國重任。

神璽二年(398年),段業在沮渠蒙遜的支持下,力排眾議命蒙遜進攻西郡,終擒太守呂純,隨後晉昌太守王德及敦煌太守孟敏都向段業投降。不久段業又攻呂弘鎮守的張掖,呂弘率兵棄城東歸,段業不聽沮渠蒙遜歸師勿遏、窮寇勿追的諫言,執意追擊,終大敗而還。神璽三年(399年),稱涼王,改元天璽。同年後涼太子呂紹及呂纂來攻,段業請得禿髮烏孤派楊軌等協助,就打算進攻結陣迎戰的後涼軍。蒙遜卻認為楊軌伺機圖謀北涼,而且後涼軍兵處死地,肯定會奮戰求生,故段業不要出戰,免陷入危機。段業同意,最終按兵不戰,後涼軍也退兵。

段業本來只是一個有德望的儒者,因緣際會被推上王位,其實本人並沒有權謀,無法約束下屬,只信任卜卦、巫術。而一直以來,段業對於沮渠蒙遜的勇略就頗為忌憚,最初就讓沮渠蒙遜由尚書左丞外調到臨池郡任太守,想疏遠他。段業又親近信任門下侍郎馬權,以其代替蒙遜張掖太守之位,但蒙遜怨恨馬權常輕侮自己,於是向段業中傷馬權,段業遂殺馬權。另段業因索嗣認為李暠不能留在敦煌,任由其發展其勢力的建議而派索嗣接替李暠任敦煌太守,然李暠擊敗了索嗣,並上請段業誅殺索嗣,段業在沮渠男成勸告下就將索嗣殺了。此時,蒙遜有除掉段業之意,遂和男成表示既馬權、索嗣二人已死,應當殺死段業,改奉男成為主,但為男成拒絕。蒙遜因段業忌憚自己而愈見不安,遂自請任西安太守,段業亦怕蒙遜很快會反叛,答應了其請求。

天璽三年(401年),沮渠蒙遜誣沮渠男成謀反,段業收捕沮渠男成並命其自殺,男成死前對段業說:「蒙遜早就和臣說過他要叛亂了,只是臣以兄弟緣故才不說出來。蒙遜以臣還在,怕部眾不聽從他,於是約臣與其祭山,反派人誣告臣。臣若果死了,蒙遜肯定很快就起兵了。請假稱臣死了,宣告臣的罪行,蒙遜肯定會起兵叛亂,而臣立即就會討伐他,必會成功。」可是,段業沒有聽信。男成死後,沮渠蒙遜以此為藉口激怒將士,並率領他們攻擊段業,連羌胡都起兵響應。段業見此就讓田昂及梁中庸率兵攻蒙遜,當時將領王豐孫警告稱西平田氏世代都有反叛者,而田昂「貌恭而心狠,志大而情險」,並不可信;但段業自以只能倚仗他對抗蒙遜,還是沒有聽從。最終田昂果然臨陣降於蒙遜,梁中庸亦被逼投降。接著田昂侄田承愛在蒙遜兵臨張掖時讓蒙遜入城,段業左右潰散,段業請求蒙遜饒他一命,讓其東歸與妻兒見面,但蒙遜還是殺了他。沮渠蒙遜隨後獲推為張掖公,繼立為北涼君主。

\subsubsection{神玺}

\begin{longtable}{|>{\centering\scriptsize}m{2em}|>{\centering\scriptsize}m{1.3em}|>{\centering}m{8.8em}|}
  % \caption{秦王政}\
  \toprule
  \SimHei \normalsize 年数 & \SimHei \scriptsize 公元 & \SimHei 大事件 \tabularnewline
  % \midrule
  \endfirsthead
  \toprule
  \SimHei \normalsize 年数 & \SimHei \scriptsize 公元 & \SimHei 大事件 \tabularnewline
  \midrule
  \endhead
  \midrule
  元年 & 397 & \tabularnewline\hline
  二年 & 398 & \tabularnewline\hline
  三年 & 399 & \tabularnewline
  \bottomrule
\end{longtable}


\subsubsection{天玺}

\begin{longtable}{|>{\centering\scriptsize}m{2em}|>{\centering\scriptsize}m{1.3em}|>{\centering}m{8.8em}|}
  % \caption{秦王政}\
  \toprule
  \SimHei \normalsize 年数 & \SimHei \scriptsize 公元 & \SimHei 大事件 \tabularnewline
  % \midrule
  \endfirsthead
  \toprule
  \SimHei \normalsize 年数 & \SimHei \scriptsize 公元 & \SimHei 大事件 \tabularnewline
  \midrule
  \endhead
  \midrule
  元年 & 399 & \tabularnewline\hline
  二年 & 400 & \tabularnewline\hline
  三年 & 401 & \tabularnewline
  \bottomrule
\end{longtable}


%%% Local Variables:
%%% mode: latex
%%% TeX-engine: xetex
%%% TeX-master: "../../Main"
%%% End:
