%% -*- coding: utf-8 -*-
%% Time-stamp: <Chen Wang: 2019-12-18 13:59:00>

\chapter{十六国\tiny(304-439)}

\section{简介}

五胡十六国(304年-439年),是中国历史上的一段时期。該時期自304年劉淵及李雄分別建立漢趙及成汉起至439年北魏拓跋燾(太武帝)灭北凉為止。範圍大致上涵蓋華北、蜀地、遼東,最遠可達漠北、江淮及西域。在入主中原眾多民族中,以匈奴、羯、鮮卑、羌及氐為主,統稱五胡。他們在這個範圍內相继建立許多國家,而北魏史學家崔鴻以其中十六個國家撰写《十六国春秋》(五凉、四燕、三秦、二赵,成漢、胡夏为十六國),於是後世史學家稱這時期为「五胡十六国」。

在西晉時期,五胡居於西晉北方、西方的邊陲地區,對晉王朝呈現半包圍局面。由於晉廷的腐敗和漢官的貪污殘暴,五胡在八王之亂後紛紛舉兵,史稱五胡乱华。在西晉滅亡後,華北地區战火纷飞,掠奪與屠殺不斷。经济受到嚴重摧毀,影响中華的民族、文化、政治、军事等发展走向。永嘉之亂帶給人民巨大痛苦,大多逃難到涼州、遼東以及江南地區,使這些地區的經濟文化漸漸繁榮。在諸國混戰期間,前秦宣昭帝苻堅一度統一華北,但在南征東晉時,於淝水之戰慘敗。其後各族於關東及空虛的關中叛變,加上東晉北伐,前秦全面崩潰,北方再度混亂。北魏立國後,經過道武帝拓跋珪、明元帝拓跋嗣及太武帝拓跋燾的經營,最後於439年統一華北,進入南北朝時期。

北方各族的内徙促成民族大融合,在中国作为多民族国家的发展过程中具有重要意义。同时,各國的君主为了增强实力,也在各自的根據地上实行一些发展生产的政策,使得各地区在華北动荡的背景下,形成局部稳定的局面。該時期的民族大融合持續到隋朝時期才大致上完成。

五胡十六國時期是西晉滅亡到北魏統一華北期間,涵蓋華北、華中北部和四川等地區的北方諸國的概稱,相對於南方的東晉時期。「五胡」即匈奴、羯、鮮卑、氐、羌五个民族,代表統治北方諸國的民族。實際上,北方諸國的統治者還包含漢人(前涼、西涼等等)、丁零人(翟魏)、盧水胡(北涼)與匈奴人鐵弗(胡夏)等民族。而地方人民也遺留不少漢人,與統治民族形成合作關係。「十六國」則是源自北魏末年的史官崔鴻私下撰寫的《十六國春秋》而得名。他自北方諸國中選出國祚較長、影響力大、較具代表性的十六國(“五凉、四燕、三秦、二趙,成漢、胡夏为十六”),共有:成汉、前赵、后赵、前涼、前燕、前秦、后燕、后秦、西秦、後涼、南涼、西涼、北涼、南燕、北燕及胡夏等國;實際上,北方諸國還包括冉魏、翟魏、西燕等等國家。總之,五胡十六國只是北方諸國的概稱,並不是北方諸國只有五個民族統治,以及只存在十六個國家。

關於「五胡十六國」稱呼的出處。在文獻上,五胡之名最早出自苻堅之口,但沒有明確定義五胡是哪五胡。而定義五胡內容的來源,學界有所爭議。歷史學家王樹民、孫仲匯、雷家驥等人認為五胡即五部胡人,源自劉淵領導的五部匈奴,但在這個時期的史書中,五胡常被當成所有胡人的泛稱,未特定指某個種族,在談到匈奴時,通常直接稱其匈奴,因此這個說法未得到學界一致認同。而陳寅恪認為五胡之名起自於五德終始說,是圖緯符命思想下的產物,周一良也支持這個說法。

川本芳昭認為,在《十六國春秋》成書之後,中國傳統史家依此思路整理史料,才開始將五胡的具體內涵確定下來,日本學者礪波護在《隋唐帝國與古代朝鮮》一書中認為五胡十六國這個概念是在唐朝初期編定正史時才形成的,南宋洪邁在《容齋隨筆》〈五胡亂華〉條中列舉七個人︰劉聰、劉曜、石勒、石虎、慕容皝、苻堅、慕容垂,這七個人分屬四個民族:匈奴、鮮卑、羯、氐,因此五胡的內涵在南宋時可能仍未完全確定,王應麟將五胡解釋為「劉淵匈奴,石勒羯,慕容鮮卑,苻洪氐,姚萇羌。」元胡三省註《資治通鑑》時,將五胡定義為「匈奴、羯、鮮卑、氐、羌」,這個定義可能來自劉曜。在胡三省之後,五胡即「匈奴、羯、鮮卑、氐、羌」這個定義開始被廣泛接受。陳寅恪曾認為五胡與十六國是兩個不同概念,不可混合

秦末漢初,漠北的匈奴成為一個強大帝國,並多次南下劫掠,在被漢朝打敗後,一部分受到漢王朝控制。公元46年之後,東漢朝廷常以招引的方式,將邊疆的草原各族內遷,以便監控或是增加兵源和勞動力。朝廷有意識的削弱邊疆民族的勢力,降低其地位,以方便監控。

到了西晉時,中國漢地北部、東部和西部,尤其是并州和關中一帶,大量胡族與漢族雜住。史書記載「西北諸郡皆爲戎居」,關中百萬餘口「戎狄居半」,對晉帝國呈現半包圍形勢。除了辽河流域的鲜卑和青海、甘肃的氐、羌外,大都由其原住地遷來。這些胡族逐漸成為漢人管轄下的編戶,由於他們需要納稅,且時時受漢官欺壓或歧視,因此心生不滿,時有舉兵之事。270年晉武帝時,河西鮮卑禿髮樹機能與匈奴劉猛率眾內侵,直至九年後始平。294年晉惠帝時,匈奴郝散叛,不久平定。兩年後其弟劉度元以齊萬年為首,聯合西北馬蘭羌、盧水胡舉兵,晉將周處陣亡,此事至299年方平。而後郭欽與江統相繼建議強制遷離胡族,江統更著有《徙戎論》,但晉室不予採納。由於胡漢摩擦的狀況沒有改善,當朝廷元氣大傷後,周邊胡族便趁機舉兵。

八王之亂的爆發,使晉廷失去在地方的影響力,胡族陸續叛變。晉惠帝時,益州內亂,巴氐勢力擴大。之後益州刺史羅尚擊殺巴氐領袖李特。304年,李特子李雄繼立後擊敗羅尚,攻入成都,自稱“成都王”(此时晋廷所封成都王为驻邺城的司馬穎),又於306年稱帝,國號「大成」,338年改国号为“汉”,史稱成漢。匈奴劉淵統領五部匈奴,成都王司馬穎結其為外援。304年司馬穎遭王浚圍攻,遣劉淵回并州發兵支援。劉淵回并州后乘機宣布獨立,稱漢王,自稱繼承漢朝正統。308年劉淵稱帝並遷都至平陽,國號「漢」,后稱「趙」。304年成漢與汉趙的建立,開啟了「五胡十六國」時期。

八王之亂結束後,劉淵為了擴充版圖,遣子劉聰掠奪洛陽,大將石勒及王彌掠奪關東各州。310年劉淵去世,劉聰殺新帝劉和自立為帝。同年,石勒經宛城、襄陽,掠奪江漢一帶,隔年北返。當時關東發生蝗災,洛陽缺糧,司馬越棄晉懷帝於洛陽,率朝中重臣及諸將東行。而後懷帝動員諸將討伐,司馬越病逝,王衍率軍歸葬封國(在東海)。石勒趁王衍東行至苦縣(今河南鹿邑縣)時率軍襲擊,晉軍精銳受屠盡亡,重臣降後被殺。劉聰、王彌及石勒趁洛陽空虛之際合兵攻破,殺害官員百姓三萬餘人,擄走晉懷帝,史稱「永嘉之亂」。313年晉懷帝被殺,晉愍帝於長安繼立帝位,劉聰派劉曜持續攻打。316年晉愍帝投降,最後受辱被殺,至此西晉灭亡。北方諸國紛紛成立。313年張軌控制涼州,封西平公,史稱前涼。315年拓跋猗盧建立代國。334年慕容鮮卑據遼東立國。

劉聰滅西晉後安逸豪奢,疏忽政事,當時曹嶷、石勒等將領分別佔據山東及關東。實際範圍只有山西和劉曜鎮守的關中。318年劉粲繼立,但遭靳準殺害奪權。劉曜與石勒得知後共同平亂,期間劉曜稱帝,改國號為「趙」,史称前赵。石勒得知後也於襄國稱趙王,史稱後趙,雙方決裂。劉曜平定上郡羌、仇池氐等關隴羌氐,威服前涼,雄踞關中。石勒則派石虎擊敗晉將段匹磾奪幽州,擊敗曹嶷奪青州。石勒雄踞關東後,於328年西征劉曜,329年攻滅前趙。330年石勒稱帝,國號亦為「趙」。前涼方面,由於戰亂較少,難民紛紛前往安居,保存了晉代典章制度,久之形成「河西文化」。

石勒為一時雄才,他得漢人張賓相助,安撫世族,重建經濟。當時胡漢關係欠佳,石勒採胡漢分治,於皇帝外另設大單于。稱胡人為國人,漢人為趙人。但這樣未能緩和雙方關係,仍然有衝突發生。石虎於石勒去世後殺石弘自立為天王。他奢侈極淫,任意濫殺,又聽信讒言,奴役非國人的漢人及「六夷」,後趙國勢漸衰。因帝位等因素,石虎與其子石邃(太子)、石宣、石韜發生骨肉相殘,宗室關係降至冰點。349年石虎稱帝後,舊太子黨人梁犢於關中叛變,石虎遣羌將姚弋仲及氐將苻洪平定,羌氐二族坐大。石虎去世後,諸子爭位,殘殺甚烈,後為養子漢人冉閔(石閔)奪得,於350年改國號為魏,是為冉魏。冉闵重用漢人,並鼓勵誅殺羯人,造成對胡人的大屠殺。之後石祗於襄國稱王,號召鮮卑、氐、羌等族抵抗冉閔。冉閔欲聯合東晉驅除胡族,但晉廷因為他稱帝而不理,反而支持向東晉稱臣的鮮卑慕容儁。352年慕容儁攻破邺都,杀冉闵,冉魏灭亡。另外,346年東晉將領桓溫攻擊成漢(成漢於338年為李壽篡位,改國號為「漢」),次年攻入蜀地,成漢亡。

慕容鲜卑於晉室南渡後佔據遼東。337年慕容皝稱燕王,他擊潰來犯的石虎,攻滅遼西段氏鮮卑,繼而重創高句麗,其勢壯盛。慕容儁繼位後,乘後趙内讧之際發兵南侵。352年攻滅冉魏,冉閔兵敗被殺,慕容儁稱帝,建國前燕。先前前燕向東晉稱臣,等冉魏滅後,慕容儁對東晉使者言道:「汝還白汝天子,我承人之乏,為中國所推,已為帝矣」。此時前燕據有關東,關中則為前秦據之。之後慕容儁又派慕容垂、慕容虔與平熙等北伐大破丁零(敕勒)。356年桓溫北伐前燕,攻陷洛陽以及司、兗、青、豫四州,之後桓溫返國,前燕復奪回四州。358年慕容儁下令全國州郡整頓戶口,準備組織150萬大軍以滅東晉,但於隔年閱軍時逝世。慕容暐繼立後,以名將慕容恪輔政,期間慕容恪將東晉收復的洛陽攻下。但慕容暐窮奢極慾,國庫逐漸掏空。慕容恪去世後由慕容評執政,他貪墨昏庸,國政更亂。369年東晉桓溫率軍北伐,進駐枋頭(今河南浚縣附近)。慕容垂率軍嚴防,最後追擊晉軍,晉軍大潰。戰後慕容垂聲名日盛,但遭慕容評排擠而投奔前秦。

氐將苻洪在石虎去世後投降東晉,在後趙内讧時意圖奪下關中,但遭人毒死。350年其子苻健成功奪下關中,建國前秦,與東晉斷絕。之後東晉履次派褚裒、殷浩、桓溫等率軍伐之,苻健皆成功抵禦,國勢漸固。之後苻健之子苻生繼立,他淫殺無度,苻健之侄苻堅殺而代之。苻堅崇尚儒學,獎勵文教。他得王猛輔政,得以集權中央,經濟提升,國勢大盛,史稱「關隴清晏,百姓豐樂」。前秦強盛後,苻堅有意一統天下。當時前燕混亂,369年慕容垂投奔前秦。苻堅趁勢派王猛、慕容垂率軍於隔年成功滅燕,取得關東地區。隨後於373年滅前仇池,376年滅代國(拓拔鮮卑)及前涼,前秦統一北方。

在統一北方前,苻堅也開始入侵東晉,於373年攻下東晉梁益二州。五年後派苻丕攻下襄陽,俘虜朱序;派彭超圍攻彭城,但被謝玄擊敗。383年派吕光西定西域,這是自東漢之後再度佔據西域。前秦統一北方後,四周諸國遣使通好,此時只剩東晉,苻堅有意伐之。鮮卑慕容垂與羌將姚萇皆盡力支持苻堅,但王猛與苻融等氐族大臣則強烈反對。這時因為苻堅將諸胡遷入關中以便控制,又將氐族勢力置於國內要衝,以鞏固勢力,此法卻使京師空虛。而且他為人寬弘,亡國君臣皆授官位,但任其率領舊部,造成隱憂。

王猛去世前告誡苻堅應該先整合好國內異族再南征,但苻堅仍一意孤行。383年5月桓沖率10萬兵攻襄陽,苻堅派苻睿、慕容垂等人防禦。苻堅認為時機已到,於8月率舉國之師南征東晉,兵分三路,聲勢浩大。他親率步兵60萬抵達項城,派苻融為先鋒率27萬兵攻打壽陽,梁成等人屯洛澗以控淮河。東晉謝安則命謝石、謝玄等人率8萬北府兵北上救援。10月秦軍前鋒攻陷壽陽後,苻堅趕往指揮,並派朱序向謝石諸將勸降,但朱序盡洩秦軍虛實。11月晉將謝玄派劉牢之率五千精兵攻破洛澗並率軍西行,與秦軍對峙淝水。12月謝玄向苻堅建議後退決戰。諸秦將認為阻敵淝水畔比較安全,但苻堅認為半渡而擊可主動對決。當秦軍後移時,晉軍渡水突擊,朱序於後軍大喊秦軍已敗。此時秦軍大亂,謝玄等人乘勝追擊,秦軍全面崩潰,苻融戰死,苻堅中箭,孤身北返,後由慕容垂護送,史稱淝水之戰。

由於前秦的主力在前方,京師兵力不足,關中的鮮卑、羌、羯等族在得知前秦大敗後紛紛獨立。隔年東晉發動北伐,攻下山東河南一帶。至此前秦崩潰,北方再度回到諸雄混戰的局面。淝水之戰隔年(384年),各胡族紛紛獨立。鮮卑慕容垂於河北復國,史稱後燕;前燕皇族慕容泓與慕容沖於山西建國西燕;前秦羌將姚萇自立,建國後秦。第二年(385年)西燕軍攻陷長安,苻堅最後被姚萇所殺。由於前秦鄴城被後燕攻下,苻丕於晉陽繼立。苻堅被殺後,鎮守前秦勇士川(今甘肅榆中)的鮮卑將乞伏國仁自立,建國西秦。仇池氐楊定也宣佈復國,並稱藩於東晉,史稱後仇池。

第三年(386年)拓跋鮮卑拓跋珪於代地復國,國號「魏」,臣服於後燕,史稱北魏。西定西域的前秦氐將呂光返國並佔據涼州,在得知苻堅被殺後於姑臧(今甘肅武威)建國後涼。西燕的人民(鮮卑族)欲東歸故鄉而發生內亂,最後由慕容永率眾東征佔據并州(今山西省範圍),建都長子。而前秦苻丕欲西行關中但被西燕帝慕容永所阻,南下東桓被東晉守將馮該殺死。前秦苻登於南安繼立,據有隴西。三年內,北方八國並立,關隴地區有前秦、後秦、西秦、後涼、後仇池,關東地區則有後燕、西燕及北魏,維持了九年。

關中方面,後秦帝姚興於394年連同西秦帝乞伏乾歸滅前秦。六年後後秦攻滅西秦,乞伏乾歸投降,受姚興重用。而匈奴鐵弗部族長劉衛辰因攻北魏戰敗而亡,其子劉勃勃(後改姓赫連)投奔後秦。在後涼投降後秦後,關中暫時為後秦盤據。407年赫連勃勃叛秦,於統萬建國胡夏,並屢次攻擊後秦。後秦國勢大衰,乞伏乾歸趁機光復西秦。其子乞伏熾磐繼立後攻滅南涼,據有隴西。416年12月後秦幼主姚泓初立,東晉劉裕發動第二次北伐,率王鎮惡等將伐後秦。晉軍連克許昌、洛陽。隔年攻破長安,後秦亡。之後劉裕因故返國,留守將領發生內鬨。夏帝赫連勃勃趁機率軍攻下長安,據有關中。另外,於405年建國譙蜀的譙縱,早在劉裕第一次北伐後就派朱齡石攻陷成都,譙蜀亡。

河西方面,後涼分裂出南涼及北涼,由於四周強敵漸漸威脅,最後向後秦投降。397年禿髮烏孤脫離後涼,建國南涼,最後南涼敗於北涼和夏,為西秦所滅。同年匈奴別部盧水胡沮渠蒙遜擁漢人段業於張掖獨立,401年沮渠蒙遜殺段業取代,史稱北涼。405年敦煌太守李暠(漢族)叛北涼,建國西涼,後亡於北涼。此時關隴地區有胡夏、西秦、北涼及後仇池四國。

關東方面,西燕在并州(今山西省範圍)建國後,於394年被後燕帝慕容垂所滅。由於北魏帝拓跋珪派兵幫助西燕,所以隔年慕容垂派太子慕容寶北伐北魏。慕容寶於參合陂之戰慘敗給拓跋珪後,請求其父慕容垂為他雪恥。於是慕容垂於隔年親率大軍伐魏,攻陷平城,拓跋珪則率眾北遁以迴避之。但慕容垂於返途中去世,之後後燕逐漸衰弱。396年拓跋珪攻下并州,隔年慕容寶企圖反擊并州,最後被拓跋珪擊敗。而後拓跋珪大舉入侵,圍陷後燕首都中山,並遷都到平城。慕容寶則撤至根本之地龍城,後燕分裂為兩地。此時慕容德不願撤往北方,南下滑台,建國南燕,之後遷都至廣固。後燕在慕容熙稱帝後,君主昏庸,百姓勞苦,國家衰敗。409年馮跋舉兵殺慕容熙,擁高雲為帝,建都龍城,之後馮跋繼立,史稱北燕。而南燕在慕容超繼任後屢次攻伐東晉,最後於隔年被東晉的劉裕討伐而亡。此時關東僅北魏、北燕兩國。

北魏拓跋嗣繼立後,時常攻掠劉宋(劉裕篡東晉後所建之國)的河南地。423年北魏拓跋燾繼立,他勵精圖治,國力大盛。拓跋燾在解除北方柔然的威脅後,開始統一華北。北魏對各民族的文化與制度採取包容態度,這減少北魏進軍的阻礙,但也使北方民戶複雜化。三年後拓跋燾大舉伐夏,攻下關中,胡夏遷至平涼。430年西秦為北涼所逼,意圖投降北魏,但隔年為夏帝赫連定所滅。赫連定意圖再滅北涼以維持胡夏,但卻被吐谷渾君主慕容慕璝襲擊而俘虜,最後斬於北魏,胡夏亡。436年拓跋燾率軍遠征北燕,馮弘逃至高句麗,北燕亡,馮弘最後被殺。439年北魏大軍圍攻姑臧,沮渠牧犍出降,北涼亡。至此,北魏統一華北,進入「南北朝時期」。然而,還有後仇池未滅,直至443年方亡於北魏。

西晉末年,全國共有21州。十六國時期,北方諸國的範圍大約是華北地區及四川地區。疆域的變更可分成五期,分別是:前趙、後趙、成漢及東晉時期;前燕、前秦及東晉時期;前秦東晉對峙時期;諸國混戰與東晉時期,此時北方以後燕及後秦最盛;北魏、胡夏、北涼及東晉時期。在諸國分立的時期,只有局部地區短暫的統一,例如前趙、後趙、前燕、後燕先後統一中原。只有前秦一度統一華北、華中北部與四川等地,為五胡十六國單一國家的最大範圍。

北方諸國的行政區劃大多繼承西晉,為州、郡、縣三級制。雖然各國佔地不大,但往往分置許多州,以致州境縮小。並且將自己沒有統治的州郡也常常在境內設置,例如前趙將幽州設在北地郡,後秦將冀州設在蒲坂,南燕將徐州設在莒縣等。由於一些國家採胡漢分治的制度,所以設置各種族專屬的行政區。例如前趙劉聰置左、右司隸及內史,用來統治漢人。單于左、右輔及都尉,則用來統治胡人。為求虛名,以表示境域廣大,常將境外鄰境的州增設於本國內。例如後燕設置雍州於長子(原屬并州),成漢設置荊州於巴郡(原屬梁州),南燕置并州於陰平(今江蘇沭陽北)。所以往往多個國家同時擁有同名異地的州。北魏統一華北後即整合政區。由於州境縮小,郡失去意義而逐漸廢除。

此外,北方諸國會成立僑州郡縣以安置流民,通常會依據流民原籍來定新州郡名。如前燕慕容廆立國於遼東時,他將投奔來的冀州人設冀陽郡、豫州人設成周郡、青州人設營丘郡、并州人設唐國郡。河西在西晉末已有為流民設置的郡縣,在張軌為涼州刺史時,就為秦、雍流民設置武興郡。405年,西涼李暠即為南人置會稽郡、中州人置廣夏郡。這些郡縣略同於東晉南朝的僑州郡縣,只是使用大略的地名而非流民原籍。

十六國也如同西晉一樣,設有行台制,但性質較為不同。行台是魏晉時期的機構,在戰爭發生時,中央機構尚書台派出機動的行政單位(可能只是一部分尚書台官員或只有使者之類的),代表朝廷隨軍都督。但是十六國的行台是尚書台派出單位,設置於戰略位置的地方最高軍政機構。例如後趙石勒建都於襄國,設行台於洛陽。後燕慕容垂建都中山,設行台於薊,以慕容盛為尚書左僕射錄行尚書事。北魏初期設有鄴、中山行台,皆為軍事重鎮。到南北朝時期,行台逐漸成為最高一級地方行政機構而凌駕於州郡之上。由於行台掌握地方軍政大權,減弱中央對地方的控制力,外重內輕的狀況更嚴重。

十六國時期的政治比較混亂,皇權不穩固,諸侯想要獨立,常常是強者奪位,弱者被殺。例如前趙靳準殺害皇帝劉粲奪權、後趙石虎殺皇帝石弘自立為帝等等。或是各地豪強舉兵叛變,例如前秦在淝水之戰敗給東晉後,諸族分裂獨立,前秦帝苻堅、苻丕也先後被各地叛軍殺害;後涼分裂出南涼及北涼,而北涼又分裂出西涼;胡夏叛變,屢次襲擊後秦,使得西秦得以脫離後秦復國;北魏叛後燕,襲擊攻入後燕首都,使得後燕分裂成北燕與南燕等。皇帝為了鞏固政權,往往採取卑劣和殘暴的手段來消除妨礙皇權的不安因素,甚至骨肉相殘,手足相鬥。例如石虎與其子石邃(太子)、石宣、石韜發生骨肉相殘。慕容垂被慕容評與前燕帝慕容暐排擠,最後投奔敵國前秦。直到北魏一統華北,皇權不穩的問題才告一段落。

十六國時期政治的一個特色是胡汉分治,將漢人與胡人以不同的制度作統治。以漢國(即前趙)為例,劉聰同時居皇帝(漢人的君主)和單于(胡人的首領),漢人以戶為單位設官統治,而胡人以落(指以帳篷營生的單位)為單位,設不同系統的官員來統治。另一個統治特色是,以種族、部族為中心的政軍結構。許多國家延續原本遊牧社會中,以部族和血緣為中心的體制,國家僅是各部族之間的聯盟,因此各部族領袖在軍政上有較高的權力,皇帝的君權較不能如其他朝代那樣直接透過官僚機構達成,也容易造成因宗室、部族領袖之間發生內訌而造成內戰。。前秦的苻堅和王猛即希望針對加以改革但尚未完全成功,後來北魏的拓跋珪將部落解散,設立新的公家統治機構,才逐漸減弱這種統治特色。

許多五胡的君主如劉淵、苻堅等等皆深染中國文化,所以皆採用其文化如提倡儒術、禁止烝妻報嫂等等。九品中正制也繼續使用,用來拔選世族人才,使為己用。當時世族之所以和胡族君主合作,主要為了苟全性命,許多世族輕視胡族君主文化低落。甚至有些世族,告誡子孫不可將出仕胡族的經過寫在墓碑上。石勒曾典定士族九法、慕容寶定士族舊籍貫、苻堅復魏晉士籍,皆用來承認世族權利石勒每破一州,必集中世族於「君子城」或「君子營」,下令不可欺辱之。華北動亂時,眾多人民逃往遼東,慕容皝設僑郡收留,並辨別世族清濁,後來這些世族成為前燕的基石。直到慕容氏諸燕後燕、西燕及南燕仍然繼續執行。前秦君主苻堅受謀士王猛影響,十分熱愛漢文化。他在攻滅前燕後,即聽王猛建議,重用關東世族。後來在王猛與眾士大夫經營之下,前秦國力提昇。苻堅也接受「大一統」的思想,發兵南征,但大敗。鮮卑北魏拓跋自開國之初即重用清河崔氏,大約亦採用九品官人法,至拓跋燾時期已出現了「中正官」的記載。這些都助長北方世族的發展。

東北以高句麗和慕容鮮卑最強。高句麗原受慕容鮮卑多次打擊,342年前燕慕容皝攻陷其都城丸都。在諸王勵精圖治下,於廣開土王高談德即位後,入侵新羅、百濟及夫餘等國,並與後燕的戰爭中奪得遼河流域及遼東半島。後燕帝慕容熙兩次出兵反擊,力圖奪回遼東地區,均未達到目的。436年北燕被北魏攻滅後,燕帝馮弘投奔高句麗。然而馮弘在高句麗號令如在本國,引起長壽王高璉嫌惡,最後殺之。位於幽州北方,宇文鮮卑的別支庫莫奚與契丹也開始崛起。414年庫莫奚虞出庫真率部落與北燕在營丘互市,隨後與契丹歸附北燕。北魏滅北燕後,契丹與庫莫奚也先後歸附北魏。

蒙古高原則為拓跋鮮卑(即後來的北魏)的勢力範圍,於南北朝時期為鮮卑別支柔然佔領。柔然始祖木骨閭是鮮卑拓跋部奴隸。鮮卑拓跋一部份南遷中原後,留下的部份進居陰山一帶。402年首領社侖自號「豆伐可汗」,建庭於鹿渾(今蒙古國哈爾和林西北),合併附近部落建立柔然汗國。柔然稱霸漠南漠北,在土拉河一帶打敗敕勒,也多次與拓跋部建國的北魏對戰。敕勒最早生活在勒拿河至貝加爾湖附近,又被稱為丁零、鐵勒與高車。五胡亂華後,在中原的丁零人曾建立翟魏國。487年漠北的阿伏至羅擺脫柔然統治,率10萬多人西遷,在車師地區建立高車國。著名的《敕勒歌》,是北齊時敕勒人的鮮卑語牧歌,後被翻譯成漢語。

西域方面有鄯善、龜兹、于闐、車師及疏勒等國,屬於涼州各國的勢力範圍。前涼、後涼、西涼及北涼都先後擁有部份西域地區。前涼在西域設置高昌郡和闐地縣,歸沙州刺史屬下的高昌太守管轄。還設置西域長史營、戊己校尉營及玉門大護軍營等管理西域日常事務。其他涼州國家一直延續此制度。382年前秦苻堅應車師前部王彌闐等人要求,派呂光遠征西域大宛諸國,並於西域設置都護。呂光後來建立後涼,派其子呂覆為西域大都護。420年西域歸北涼時,鄯善王比龍入朝北涼,西域各國也紛紛向北涼稱臣納貢。

吐谷渾原為鮮卑慕容部的一支,283年鮮卑單于慕容涉歸的庶長子慕容吐谷渾,因與慕容廆雙方不和,率所部西遷。313年時至隴西枹罕立國,統治今青海省、甘肅省南部、四川省西北等地的氐、羌民族。碎奚繼位時,於371年隨仇池氐王向前秦稱臣,被封為安遠將軍。後來,繼位的視連、視羆都臣服於西秦,被封為沙州牧、白蘭王。慕璝繼位時,率軍襲擊胡夏末帝赫連定,使胡夏滅亡。

十六國時期的北方諸國多實行異族分治制度,或稱為胡漢分治制度,在一國之中,實行兩種不同的軍政體制。對漢族人民,仍按漢族的傳統方式進行統治。對少數民族,則按各自的部落傳統進行統治。這使得軍事統帥被分為單于台與都督中外諸軍事並立,後來隨形勢發展漸漸合併。在軍隊形式上大致同西晉兵制,具有中軍、外軍組織及都督、將領等職務。中軍直屬中央,編為軍、營,主要保衛京師;外軍為中央直轄的各州都督所統率的軍隊。各國兵權,大多掌握在宗室手裡,任都督中外諸軍事除了前秦王猛(非宗室)外,有前趙劉宣、劉曜等人,後趙石弘、石斌,前秦苻雄、苻法等人。這本來是加強朝廷的措施。但往往變成皇位之爭而與太子自相殘殺,最後導致亡國。

各國軍隊以騎兵為主,步兵其次。各國本民族的部落兵多為騎兵。隨著攻城戰的出現以及讓漢人編列為軍隊,步兵數量也逐漸增加。如前秦南征東晉之際,即以步兵六十萬,騎兵為二十七萬,不過各國並非都改任步兵為主力。在兵役制度方面,則是實行本族全民皆兵的部落兵制,並兼有魏晉世兵制的特點。只要是凡識於戰鬥的本族人民,皆作為軍隊基本兵力。基本上中軍為終身制,其家屬通常隨營聚居,稱營戶,負責供應軍糧。鎮守各地的外軍,其隨營聚居的家屬則稱鎮戶。營戶與鎮戶都是其兵力來源。其他人民方面皆實行徵兵制,徵發各郡、縣的各族人民補充軍隊。其中漢族兵的來源,還包括來自投降的塢堡和招募的農民,一般都是終身為兵。

十六國時期各國騎兵均已強化。當時馬蹬已經十分普遍,其最大功能是可以解放雙手,騎兵開始可以靠雙腳控制平衡在馬上衝、刺、劈、擊,這大大提升騎兵的戰鬥力。馬鎧也成為騎兵較普遍的裝備,來保護戰馬免受遠射兵器攻擊。

五胡十六国前期,西晉人民为了躲避战乱,大量人口南迁,其规模之大、持续时间之长可谓史无前例。成汉的益州(四川)、前凉的河西走廊、前燕的辽河流域吸引了大量難民,成為立國的基石。河西姑臧還成為丝绸之路上的经贸外来重镇。至於留在中原地區的人民則庇護在塢堡或是部落貴族。塢堡大多由世族豪強建立,主要作為軍事防衛。世族豪強所擁有蔭戶不承擔國家賦役,僅對塢主負有義務。為了保證國庫收入和勞役來源,各族君主往往進行戶口檢查,將蔭戶復歸於編戶。

當時在中原活躍的北方民族有鮮卑、烏桓、高句麗、丁零、羯、南匈奴、匈奴別支铁弗及盧水胡、以及西部的羌、氐、巴等人。卡尔·魏特夫認為這些民族所建立國家屬於滲透王朝。這些遷入的民族與滯留北方的漢人產生「文化採借」,雙方逐漸進行文化交流與民族的融合,其中北方諸國的典章制度與禮儀法律幾乎交由漢人制定。杉山正明認為,這些游牧民族原本就存在於中原,在農耕為主的漢族居地之間活動,並逐漸定居,改成以農業生活。認為中原是漢族固有土地,這些民族是外來滲透,是中國傳統上以漢族為中心的史觀所造成。

在交流中,因為思想衝突、種族糾紛及政治鬥爭等因素,時常發生破壞、屠殺等衝突。在前秦之前,由于相互攻杀,导致游牧民族和汉族的人口大幅减少,以冉魏为例,冉闵曾经下令漢人屠杀胡羯,為殺胡令,造成二十余万胡人死亡,羯人滅族;石赵被灭时,各路百姓和游牧民族各自返回原住地,来往途中相互攻杀,加上粮食短缺,最终能够返回家乡的只有出发时的十分之三不到。

在五胡十六國中期,北方各族與漢民族彼此間展開民族與文化的融合,社会环境趋于稳定,人口开始逐渐回升。早東漢至魏晉時期,北方各族陸續內遷至中原,與漢族一同居住,但是時常受到漢官欺壓或受漢人歧視。當時北方各族即受漢文化影響。如匈奴、氐族改用漢姓並學漢語及經書。中原也流行胡族文化,對於北方民族的生活用具、服裝及音樂均感興趣,並普遍食用牛羊酪漿。當北方諸國一一滅亡之後,由於草原故鄉被柔然等新興民族佔據,而且已經適應中原文化與生活。所以這些民族絕大部分沒有退返草原,而是留在中原與漢族合為一體。民族的融合直到北周、隋朝方完成。在東晉南朝方面,中原漢人在衣冠南渡後,也和當地漢人、山越等百越诸族、及南方其他各民族發生衝突及融合。在隋朝統一中國後,南北漢人的界線逐漸模糊,融為一體。

當時黃河南北與關中地區是遭受戰禍最劇,經濟破壞最為嚴重。當時人民不是依附塢堡,成為塢主的部曲。就是遷移至各國首都附近,提供生產或兵役用。各國也會互相掠奪人民、財富以充實國力或是補給軍隊。由於人民頻繁的遷移,使得在初期難有經濟發展。

有些國家在穩定之後,開始發展經濟。例如後趙石勒在崛起過程中,大廝殺掠。但在立國後開始發展經濟,勸課農桑,頒布的稅收卻比西晉還輕,經濟逐漸復甦。但在石虎統治之後,勞役漢人,揮霍無度,經濟下滑。另外,有些國家早在開創時期就已經打下基礎,做好內政,吸引不少流民投靠。早在成漢成立之前,已有大批流民投靠巴氐李氏。李雄建立成漢後,在他統治之下「事少役稀,百姓富實」,成為最安定的地區。前燕慕容皝在統治遼東時即仿照曹魏,開放荒地讓流民種植。前涼統治的河西地區,由於相對中原較少戰亂,大量流民投奔。農業、畜牧業都有所發展。絲路也能保持暢通,使得首都姑臧成為商旅往來的樞紐,漸漸發展出「河西文化」。

前秦苻堅崇尚儒學,獎勵文教。他任漢人王猛輔政,王猛發展經濟,關中的農業、手工業和商業獲得恢復和發展。使得前秦國勢大盛,史稱「關隴清晏,百姓豐樂」,打下統一華北的基礎。前秦崩潰之後,後秦姚興注重刑罰,懲治貪污,關中經濟稍微恢復。之後西涼李暠在玉門關、陽關開墾荒地,史籍記載「年穀頻登,百姓樂業」。北燕馮跋減輕賦役,南涼禿髮烏孤注重農業,皆重視根據地的經濟發展。

邊疆各族在華北地區立國後,互相混戰。在這些國家中,以前秦(氐族)和後秦(羌族)的文化最為興盛,其次則是鮮卑慕容氏建立的前燕及後燕。此外,漢族張軌、李暠所建立的前涼和西涼,更是當時的文化中心,史稱「河西文化」。各国的统治者为了维护政权的稳定也发展教育。前赵刘曜设置太学、小学,选拔人才。前燕慕容皝设置官学,并著教材《太上章》和《典诫》。后秦、南凉设置律学,召集地方散吏入学。這促使北方各族接受漢文化,對於民族融合具有積極意義。

當時流傳下來的詩及賦不多,可能因為藝術價值不高,所以流傳不廣。至於章奏符檄,《周書‧王褒庾信傳論》認為有可觀之作,文風上接近西晉末年的風格。民歌方面,著名的大抵保存於《樂府詩集》的《梁鼓角橫吹曲》。其中有出於氐族的《企喻歌》、出於羌族的《瑯琊王歌辭》、出於鮮卑族的《慕容垂歌辭》。《晉書》的「載記」還保存一些當時的諺語,如流傳於前秦的「長鞘馬鞭擊左股,太歲南行當復虜」、「河水清復清,苻詔死新城」等。

該時期的作品以前涼和前秦的文人居多。前涼張駿著有樂府詩《薤露》、《東門行》兩首,收錄於《樂府詩集》。前涼大臣謝艾的奏疏曾被《文心雕龍》提到,他的文集可在《隋書‧經籍誌》看到。西涼李暠所著的《述志賦》載於《晉書》本傳,這篇賦表現出他建功立業的志趣和對西涼局勢的憂慮,內容頗有文采。前秦趙整著有兩首五言四句詩,用比興的手法諷諫苻堅。他還有一首琴歌《阿得脂》是雜言體,有些字句難解,大約雜用氐語。苻堅的侄子苻朗為散文家,作有《苻子》,其中有不少片斷頗具文學意味。女詩人蘇蕙的迴文詩《璇璣圖》雖然有文字遊戲的意味,但仍表現出遣詞用語的功力,成為流傳不絕的佳話。另外,後秦宗敞為王尚申辯的奏章,被呂超認為可與曹魏的陳琳、徐幹,以及西晉的潘岳、陸機相比。後秦胡義周(作者存疑。)為赫連勃勃作《統萬城銘》,獲《周書‧王褒庾信傳論》讚揚為典雅莊重。

佛教早就在東漢時期傳入中國,當時由於儒教興盛,所以沒有廣泛發展。等到十六國時期,北方動盪不安,以致人人厭苦、家家思亂。時而感到人生無常,精神缺乏寄託。此時五胡君主希望利用佛教教理的戒惡修善、六道輪迴來安撫各族百姓,並藉由屬於外來宗教的佛教來支持其政權。最後佛教得以在北方流行,並與南方佛教互相交流。至於道教,雖然在西晉就有五斗米道(天師道)的出現,但在十六國時期衰弱下來。一直到十六國末期北魏的寇謙之改革道教,才有能力與佛教抗衡。

當時從西域進入中土的僧侶,為數眾多,或譯經論,或弘教理。在佛圖澄、道安及鳩摩羅什的推廣下,為佛教奠定發揚的基石。五胡君主中,石勒、石虎、苻堅與姚興等極力支持佛教發展。苻坚的從兄之子苻朗著有佛学論書《苻子》。佛圖澄為西域僧人,他精通經文並擅長幻術。後趙的石勒、石虎奉他為「大和尚」,讓他參與軍政機要。並支持佛教發展,甚至下令不論華夷貴賤,都可以出家,開啟漢人出家之端。一时人民多营寺庙,争先出家。和佛图澄同时在后赵的,还有敦煌人单道开,襄阳羊叔子寺竺法慧和中山帛法桥等。道安為佛圖澄的弟子,在晚年備受前秦苻堅的崇敬。他致力整理和翻譯佛經,將長安經營成北方佛教的譯經中心。他於襄陽編定《綜理眾經目錄》,還為僧團制定法規,為寺院制度奠定基礎。中國出家僧人改姓「釋」,即是從道安開始。道安的弟子後來分佈各地,成為傳教的主要力量。

鳩摩羅什為西域龜茲人。382年,前秦苻堅聽從道安之建議,命大將呂光西征龜茲、迎接鳩摩羅什到長安。但後來前秦大亂,呂光隨即割據涼州,鳩摩羅什留居涼州共十七年。直到401年,後秦姚興得以迎至長安。鳩摩羅什備受姚興尊敬,待以國師之禮,入长安西明阁和逍遥园从事翻译。405年姚兴以罗什的弟子僧略为“僧正”,僧迁为“悦众”,法钦、慧斌为“僧录”,令管理僧尼的事务。鳩摩羅什主持下譯出《般若經》和大乘中觀學派的論書《中論》、《十二門論》、《百論》及《大智度論》、《法華經》等三十五部兩百多卷經典。這些皆成為後來佛學教派和宗派所依據的主要法典。其时四方的义学沙门群集长安,次第增加到三千人。

当时北方凿窟造像之风兴起,366年后秦沙门乐僔在敦煌东南鸣沙山麓,开凿石窟,镌造佛像,这就是著名的莫高窟。麥積山石窟始建於十六國的後秦,大約384年前後,當時佛教在中國開始興盛。麥積山石窟同是中國唯一保存北朝造像體系最完整的石窟,也是唯一能比較全面反映北魏至明清時期中國泥塑藝術演變歷史的石窟。后期北魏太武帝、北周武帝进行大规模的灭佛活动,对佛教的发展造成严重破坏。

民族的大融合帶來藝術文化的交流與整合,由於多元民族文化的淵源,不僅增補了固有文化停滯的不足,更可以強化文化新生發展的生機。由於佛教的興盛,帶動石窟雕像的發展。這個時期最突出的建築類型是佛寺、佛塔和石窟。

佛教的興盛帶來高層佛塔的建築以及印度、中亞一帶的雕刻、繪畫藝術。使當時的石窟、佛像、壁畫等有了巨大發展,將漢代比較樸直的風格,變得更為成熟、圓淳。位居中國四大石窟的敦煌莫高窟。和麦积山石窟,都是在十六國時期建造。

麥積山石窟始建於後秦時期(約384年前後),素有「東方雕塑陳列館」美譽。敦煌莫高窟則建於前秦時期,是世界上現存規模最大、內容最豐富的佛教藝術地,以精美的壁畫和塑像聞名於世。由於當時敦煌與西域各國交流頻繁,使得早期的莫高窟包含河西文化及西域藝術的風格。其中屬於十六國時期的275窟,繪有本生、佛傳等故事畫。這些繪畫以圈圈暈染的方式凸顯出人體特徵,並以細線勾勒,畫風豪放生動,是當時壁畫的典型風格。

書法方面,著名的作品有前涼的《李柏文書》。、前秦的《譬喻經》、西涼的《十誦比丘戒本經》和《妙法蓮華經》等。其中《李柏文書》與東晉王羲之的《姨母帖》皆保存行、楷書變遷過程,對書寫考究與風格變化有很高的參考價值。其他作品則介於書、楷之間。至於碑刻方面,著名作品有前秦的《廣武將軍碑》及《鄭太尉祠碑》、北涼的《沮渠安周造像碑》等。其字體大多在隸、楷之間,風格墣茂古拙。《沮渠安周造像碑》為沮渠安周在高昌所立,原石在新疆吐魯番高昌故城出土。《廣武將軍碑》則於前秦建元四年(368)刻。筆劃渾樸,結構拙厚,天趣渾成。書法家于右任曾作《廣武將軍歌》以推崇之。由於前秦碑文稀少,所以此碑與《鄧太尉祠碑》皆備受珍惜。

经过八王之乱和永嘉之乱後,中原殘破不堪,人民四處逃難,形成流民潮。諸國君主亦掠奪人口,以充實國力,深深破壞北方的社會結構。殘留在北方的世族,在面對險惡的環境下,有些聚集鄉民和自家的附屬人口,建立塢堡以便自守。而流民也紛紛投靠,形成人數眾多的部曲。有些則與諸國君主合作,以保本族安全。五胡君主在建國後,為了能夠統治中原地區,也需要熟悉典章制度的士大夫(世族)的協助。由於處境艱困,北方世族對同族常存抱恤的溫情,家族組織趨向大家庭制,有遠來相投的親戚,莫不極力相助。在團結力量及參與政事後,北方世族並沒有因戰亂而衰落,反而經過長期相處,使胡人融入漢人文化中。

塢堡是一個自給自足的社會組織,投奔的流民可以受塢堡保護。人民必須服從塢主命令,平時接受軍事訓練及農業生產,戰時成為保衛塢堡的戰士。人民的生產所得也須課稅給塢主。塢主除負責生產與作戰外,也要提倡教育及制定法律。由於塢堡眾多又難攻破,往往會左右戰局,使得五胡君主十分忌諱。例如祖逖北伐時,由於與當地塢堡合作,最後成功收復黃河以南領土,與石勒隔河相持。胡人君主為了解決塢堡問題,往往會與其妥協以籠絡之。到北魏宗主督護出現,塢堡的時代漸漸過去。

%% -*- coding: utf-8 -*-
%% Time-stamp: <Chen Wang: 2019-12-18 14:02:10>


\section{汉赵\tiny(304-329)}

\subsection{简介}

漢趙(304年-329年),又称前趙,是匈奴人劉渊所建的君主制割据政权,都平阳郡(今山西临汾西北),這是十六国時期建立的第一個政權。

304年,劉淵起兵,称漢王。308年称帝,国号“汉”。310年劉聰即位,311年和316年兩次攻破西晋都城洛陽、長安。318年劉曜即位,殺死靳準,次年改国号為「趙」。329年被後趙所滅,立國凡26年。其统治地区包含并州刺史部、雍州刺史部、秦州刺史部、豫州刺史部、司隶校尉部、冀州刺史部部分地区。

劉淵以自己祖先與漢朝宗室劉氏約為兄弟而自稱漢王,并自称继承汉朝,故以“汉”为国号,史稱「前汉」;以多为匈奴人,又称「胡漢」或「匈奴汉」;又统治地区位于中原北方,故称「北汉」,但此稱呼因易于與五代十国时期的北汉混淆而很少使用。劉曜以其发迹之地为战国时赵国之地,改国号为赵,为别于石勒的后赵,而史称「前趙」,或合稱之為「漢趙」。

劉淵為南匈奴單于的後裔,其父劉豹為匈奴左部帥,在五部中勢力最強。劉豹卒后,代父為左部帥。西晉有意削弱他與部落的關係,後二遷為離石將兵都尉,劉淵則利用此職位的權限,暗中擴展勢力。楊駿輔政時,為了拉攏劉淵,命他為建威將軍、五部大都督,封漢光乡侯,給予統率匈奴五部軍事的大權。到元康末年,成都王司馬穎為了擴大自己的勢力,極力拉攏劉淵,表其為「行寧朔將軍,監五部軍事」,加強劉淵在匈奴五部中的地位,並命劉淵居鄴城,以便控制。

到晉惠帝太安中(302年─303年),因河間王司馬顒、成都王司馬穎、齊王司馬冏、長沙王司馬乂等諸王相互殘殺,益州刺史部流民起義爆發,各地局勢不穩,在并州刺史部的匈奴五部右賢王劉宣等人也醞釀著反晋兴匈奴。右賢王劉宣與各部貴族商議共推劉淵為大單于,并派呼延攸告诉在邺城的刘渊,劉淵让呼延攸先回去告诉劉宣等召集各部,聲言聚集五部協助司馬穎,實際是為反晉作準備。

晋惠帝永興元年(304年)三月,司馬穎等攻占洛陽,司馬越挾持晉惠帝攻鄴,司马颖打敗司馬越,並虜獲晉惠帝。八月,司马越势力王浚、司馬騰攻鄴城,刘渊请求带领匈奴五部帮助司马颖抵御,司馬穎同意,并拜劉淵為北單于,派遣回并州刺史部的平阳郡調發匈奴五部為援。劉淵返回并州離石,眾人共推劉淵為大單于,并聚集五萬之眾。刘渊得知王浚军队已攻破邺城,司马颖南逃洛阳。刘渊还想遵守先前承诺帮助司马颖,劉宣等劝说刘渊起兵反晋。十月,劉淵從離石遷于左國城,稱漢王,改年號為元熙,置百官,大赦境內,並以復漢為名義,正式建立政權。

漢元熙元年(304年)十二月,晉并州刺史司馬騰遣兵攻漢,雙方大戰于大陵(今山西省文水北),劉淵大勝,並遣劉曜等攻取上黨、太原、西河各郡縣。當時在青、徐二州的王彌,魏郡的汲桑、石勒,上郡四部鮮卑陸逐延,氐族酋長單徵等人均擁立劉淵為共主。劉淵命王彌、石勒等人攻取河北各郡縣,並一度攻入西晉的重鎮許昌,其兵鋒進抵至西晉的首都洛陽城下。308年十月,劉淵正式稱帝,改年號為永鳳。309年,劉淵遣將攻占黎陽(今河南省浚縣東北),擊敗晉將王湛於延津(今河南省延津縣北),沉殺男女三萬人,又派遣四子劉聰進攻包圍洛陽。

310年,劉淵病重,命劉聰輔佐太子劉和。劉淵病死,劉和繼位,不久劉聰殺死劉和自立為帝。

劉聰繼位後,派遣族弟劉曜、大將王彌等率領四萬大軍攻取洛陽周邊的郡縣,以孤立斷絕洛陽。311年,石勒在苦縣(今河南鹿邑)消滅西晉主力部隊十多萬人。同年夏季,劉曜、王彌攻破洛陽,虜走晉懷帝,殺害官員百姓三萬餘人,史稱永嘉之亂。晉懷帝於次年被殺後,晉愍帝於長安即位。316年,劉聰派遣劉曜攻破長安,俘晉愍帝,西晉滅亡。隨著西晉的滅亡,中原廣大的地區,皆成為漢政權的統治範圍。

雖然劉聰名義上是中原的共主,但隨著领域的擴大,地方的割據迅速形成,漢國統治的地區實際上只有一小部分。

318年,劉聰病死,太子劉粲繼位。匈奴貴族靳準殺死劉粲奪權,在平陽的劉氏男女不分老少全部被殺,靳準自立為漢天王。鎮守長安的劉聰族弟劉曜得知平陽有變,自立為皇帝,派遣軍隊至平陽,族滅靳氏。與此同時,石勒亦以討伐靳準為名,率軍至漢都平陽,于是,平陽、洛陽以東的地區,皆落入石勒勢力之中。漢國於是遷都到長安。

319年,劉曜改國號「漢」為「趙」,史稱「前趙」或「漢趙」。同年,石勒在襄國自稱趙王,從前趙中分離出來,史稱「後趙」,双方決裂。後數年,關中地區連年叛亂及大疫,百姓死者眾多,劉曜撲滅了關中各地氐、羌人的反抗,於是遷徙上郡氐、羌二十萬人及隴西大姓楊、姜等一萬多戶到關中以充實人口。

前趙政權初步鞏固後,即向外擴張,平定隴右一帶的陳安,並向西進擊前涼。雙方在黃河沿岸僵持,張茂稱藩,並獻貢。前趙全盛时,擁兵二十八万五千餘人,據有司隶州、雍州、并州、豫州、秦州各一部,時關隴氐、羌,莫不降附。

324年,前赵军队開始向東挺进,意图夺取石勒所占的河南。325年,劉曜命劉岳率兵一萬五千人圍攻後趙石生於洛陽金墉城,石勒命從子石虎率軍救援,與劉岳在洛水西岸交戰,劉岳兵敗,退守石梁戌,石虎包圍石梁戌。劉曜率軍救援,屯兵于金谷(今河南省洛陽市西北),夜中前赵軍中哗变,士卒潰散,劉曜退歸長安。不久,石虎攻下石梁戌,生擒劉岳等人。

328年,石勒命石虎率大軍四萬從軹關(今河南省濟源市西北十五里)西攻蒲坂(今山西省永濟市蒲州鎮),劉曜親自率領水陸大軍從潼關渡河救援,石虎引兵撤退,劉曜追及並大破,石虎逃奔朝歌。劉曜取得這次大勝之後,從大陽關(今山西省平陸縣茅津渡)南渡,在洛陽金墉城圍攻石生。後趙的滎陽郡太守尹矩、野王郡太守張進等人相繼投降。这次战败震動了後趙。石勒認為洛陽一失守,劉曜必定會進攻河北,於是集結步兵六萬,騎兵二萬七千,從鞏縣渡洛水,進抵洛陽城下。

劉曜得知石勒親率大軍增援,撤走包圍金墉城的軍隊,在洛陽之西列陣十多萬軍隊,南北距離十多里。石勒率軍進入洛陽。到了決戰當天,由石虎率步兵三萬,從洛陽北方向西移動,攻擊劉曜的中軍;石堪、石聰各率騎兵八千,從洛陽西方向北移動,攻擊劉曜的前鋒。雙方大戰於洛陽西面的宣陽門外,交戰之後,石勒親自帶領主力,從西北大門出城,夾擊前趙軍,前趙軍大潰。劉曜飲酒過量,在昏醉中退走,為石堪所擒,這一仗前趙軍被斬首五萬人,主力部隊損失殆盡。

劉曜戰敗被擒,不久被殺。石勒軍乘勝西進,劉曜子劉熙、劉胤等人放棄長安,逃奔上邽(今甘肅省天水市)。329年九月,後趙出兵攻占上邽,殺趙太子劉熙及諸王公侯、將相卿校以下三千餘人,又在洛陽坑殺其王公及五郡屠各五千多人,並遷徙其百官、關東流民、秦雍大族九千多人到襄國,前趙滅亡。

在劉淵、劉聰時期,其範圍控有冀州刺史部、兖州刺史部、青州刺史部、徐州刺史部、豫州刺史部、并州刺史部、雍州刺史部、司隶校尉部、秦州刺史部一帶,然而實際控制範圍不大,劉聰時期,只局限在并州的一角(其餘部分在劉琨手中)和由劉曜坐鎮的關中一部分地區。黄河以北地区一帶由石勒所有,王彌的部將曹嶷控有青州、兗州、徐州一帶,慕容鲜卑更是趁机向南统治到幽州的一帶。

劉曜時期,史稱「東不踰太行,南不越嵩、洛,西不踰隴坻,北不出汾、晉」(引顧祖禹《讀史方輿紀要》),疆域範圍包括雍州、司隶州的渭水流域以及并州、豫州、秦州黃河以東一帶。

基本上,前趙的政治制度承襲漢魏以來的制度而又雜以舊俗。漢國的官制,自304年劉淵稱漢王建立割据的君主制政權後,即採取漢朝的官制,設丞相、御史大夫、太尉及六卿等中樞之官。軍事之官有大司馬、太尉、大將軍等高級將軍以及雜號將軍。而地方之官則沿習魏晉以來的州郡制,採用漢胡分治的政策來進行統治。大單于的權力極大,僅次於皇帝。到劉聰嘉平四年(314年),達到了較為完善的階段。而劉曜的前趙,繼承漢國之制度,小有改革。劉曜繼承君主制的前赵政權胡、汉分治的政策。以子劉胤為大司馬、大單于,置單于台于渭城(今陝西咸陽),自左、右賢王以下皆用少數族豪酋充當。另方面又大体沿用魏晉九品官人法(見九品中正制),設立學校,肯定士族特權,笼络漢人的世家大族、士族,以巩固其統治。

劉淵時,設單于台,最高長官為大單于,統率六夷部落,單于台的設置,是沿匈奴舊制而來。劉聰時,在統治區內設置左、右司隸,各領戶20多萬,每1萬戶設置一名內史,內史共有43人。在大單于下設置單于左、右輔,各主六夷十萬落,萬落叟置一名都尉。

前趙的社會經濟主要是農業,其次是畜牧業,其生產方式,沿襲漢魏以來的生產方式。

在前趙社會中,從事農業、手工業、牧業生產的還有奴隸。奴隸的來源主要是戰俘,其次是犯罪的官吏。國內還有大量從事遊牧及畜牧業的「六夷」部落,因歸降及征服的部落日益增多,故設單于台進行管理。

漢在劉聰時(310—318年),杂夷戶口大約有六十三萬戶,人口大約有三四百萬人以上;汉户未详。

在劉曜全盛時期,有兵力二十八萬五千人,在他出兵時,史稱「臨河列陣,百餘里中,鍾鼓之聲沸河動地,自古軍旅之盛未有斯比」(《晉書》.劉曜載記》)。

%% -*- coding: utf-8 -*-
%% Time-stamp: <Chen Wang: 2019-12-18 14:03:32>

\subsection{光文帝\tiny(304-310)}

\subsubsection{生平}

漢趙光文帝劉淵(249年至254年間-310年8月19日),字元海,新興匈奴人(今山西忻州市北),出身匈奴屠各部。為五胡十六國時代中,汉赵的開國君王。西晉末年八王之亂時諸王互相攻伐,南匈奴族人擁立其為大單于。304年,劉淵乘朝廷內亂而在并州自立,稱漢王,国号为漢(后改为趙,史称前漢、前趙或漢趙),5年後稱帝,改元永鳳。310年,劉淵在位六年病死,諡光文皇帝。

劉淵出身屠各族(南匈奴),是西漢冒頓單于的後代挛鞮家族的人,該家族因西漢劉邦以來,長期與漢朝王室通婚,同時兼具漢朝王室與匈奴貴族的血脈,故漢名多採取漢朝王族的劉姓為姓氏。

東漢獻帝年間,曹操統一華北地區後,重整匈奴五部,劉淵父親劉豹原是匈奴王族的左賢王,在此一時期被曹操任命為「左部元帥」;而劉淵的母親呼延氏,亦是《史記》紀載下的三大匈奴貴族姓氏之一,足見劉淵身份之高貴。

劉淵童稚時已十分聰明,七歲時母親呼延氏逝世,劉淵傷心得捶胸頓足地號叫,旁人都被其哀傷所感染,宗族部落的人都因其表現而對他十分欣賞。連當時曹魏司空王昶聽聞其行為後都讚賞他,又派人弔唁和送禮物。劉淵亦十分好學,拜崔游為師,學習《毛詩》、《京氏易》和《馬氏尚書》,劉淵尤其喜歡《春秋左氏傳》及《孫吳兵法》,《史記》、《漢書》等歷史典籍亦一一看過。同時,劉淵自以書傳中都因隨何、陸賈無武跡;周勃、灌嬰沒文才而都遭後人看不起,認為文武兼備才能獲世人欣賞,因而習武。劉淵臂力過人,善於射擊,可謂文武雙全。崔懿之、公師彧、王渾等都與他結交。

咸熙年間,劉淵到洛陽作任子,受到當時曹魏權臣司馬昭厚待。司馬炎篡魏建立西晉後,王渾向晉武帝司馬炎推薦劉淵,武帝接見劉淵後亦對他十分欣賞,更打算任命他參與平滅東吳的事,但因孔恂和楊珧以「非我族類,其心必異」為由,擔心一旦向劉淵委以重任並平滅東吳,他會在當地叛晉自立。武帝聽後才將擱置這打算。及後禿髮樹機能先後擊敗秦州刺史胡烈及涼州刺史楊欣,李熹建議任用劉淵討伐,但孔恂仍指劉淵可能會作亂涼州,武帝因而又否決了建議。當時在洛陽流浪的王彌正要回故鄉東萊,與劉淵餞別時,劉淵泣訴被人屢進讒言中傷,恐怕將會在洛陽遇害而不能再見到他。劉淵於是縱酒長嘯,同坐的都因他流淚。齊王司馬攸見劉淵後,更建議武帝殺劉淵,以免日後回匈奴五部所在的并州後會禍亂當地,但王渾反對。武帝同意王渾所言,最終沒有殺劉淵。

正巧任匈奴左部帥的父親劉豹於當時逝世,劉淵於是回到并州接替父親左部帥之位。太康末年劉淵官拜北部都尉。劉淵在當地申明刑法,禁止奸邪惡行,而且誠心與人交往,於是匈奴五部中的俊才都投歸劉淵,連幽州和冀州的名儒和寒門秀士都前來與他結交。永熙元年(290年),晉惠帝司馬衷繼位,由外戚楊駿輔政。楊駿為了拉攏遠人,樹立私恩,便任命劉淵為建威將軍、五部大都督,封漢光鄉侯。但至元康末年劉淵便因部下族人叛變出塞而免官。不久成都王司馬穎出鎮鄴城(今河北臨漳縣西南),為拉攏劉淵而表他行寧朔將軍、監五部軍事,並召他至鄴城。

當時八王之亂戰火再起,趙王司馬倫、齊王司馬冏及長沙王司馬乂先後以軍事力量上台掌權,司馬倫更曾篡位稱帝,天下大亂,盜賊蜂起。劉淵叔祖父劉宣見此,決心乘著西晉朝政混亂振興匈奴,於是秘密與族人推舉劉淵為大單于,又派遣呼延攸到鄴城通知劉淵。劉淵向司馬穎請歸不果,於是派呼延攸先回并州,命劉宣召集五部匈奴和在宜陽的一眾胡人,名為支持司馬穎,實質上卻圖謀叛變。

永安元年(304年)司馬穎擊敗司馬乂,成為皇太弟,任命劉淵為屯騎校尉。不久東海王司馬越和陳昣等與惠帝征討司馬穎,司馬穎又任命劉淵為輔國將軍、督北城守事。及至惠帝兵敗蕩陰(今河南湯陰縣)被俘至鄴城,司馬穎再任命劉淵為冠軍將軍,封"盧奴伯"。但在蕩陰之戰後不久,東嬴公司馬騰和安北將軍王淩等就起兵討伐司馬穎,劉淵趁機向司馬穎建議讓他回匈奴五部領部眾支援司馬穎,共同抵抗司馬騰和王淩的討伐部隊。司馬穎同意並拜劉淵為北單于、參丞相軍事。

劉淵回左國城(今山西吕梁市离石区)後,劉宣便為劉淵上大單于稱號,二十日之間就聚眾五萬,定都離石。及後劉淵被司馬騰盟友拓跋猗㐌和拓跋猗盧擊敗,同時司馬穎亦因受不住王淩大軍的進逼而棄守鄴城,帶惠帝逃回洛陽。劉淵在劉宣的反對下,最終決定不援救司馬穎,遷至左國城(今山西吕梁市离石区東北),又吸引數萬人歸附。

永興元年(304年)十一月,劉淵以自己祖先與漢朝宗室劉氏約為兄弟而自稱“漢王”,建國號漢,改元元熙,並追尊蜀漢後主劉禪為孝懷皇帝,又設漢高祖劉邦、漢世祖劉秀、漢昭烈帝劉備、漢文帝劉恆、漢武帝劉徹、漢宣帝劉詢、漢明帝劉莊和漢章帝劉炟等八位西漢、東漢和蜀漢皇帝的牌位;前三者為三祖,後五者為五宗,以漢室繼承者自居。同時自置百官,正式建立一個脫離西晉朝廷的獨立政權。

劉淵稱王後,身為并州刺史的司馬騰便派將軍聶玄討伐,但遭劉淵於大陵(今山西文水縣)擊敗。司馬騰知道聶玄兵敗後十分恐懼,率并州二萬多戶人南下山東地區。劉淵亦派劉曜先後攻陷太原、泫氏、屯留、長子、中都等地方,擴闊領土。次年(305年),劉淵所派將領劉欽再度擊敗司馬騰所派的討伐軍。同年并州爆發大饑荒,離石亦受影響,劉淵於是遷都黎亭。永嘉元年(307年),劉淵已攻陷并州大部份郡縣,並派兵進攻新任并州刺史劉琨。但劉琨擊敗漢軍,成功保著治所晉陽(今山西太原市)。戰後劉琨努力經營并州,更離間收降劉淵部下雜虜,漢軍向并州北部擴張的計劃因而受阻。劉淵於是聽從侍中劉殷和王育派兵進攻其他州郡,南侵進據長安(今陝西西安市未央區)和洛陽(今河南洛陽市)的建議;同時,汲桑、石勒、王彌、鮮卑陸逐延和氐酋大單于單徵數個在其他地方的軍事力量都相繼歸降劉淵,劉淵亦一一任官封爵,令漢國力量更為壯大;亦因這些加入者起事和影響的地方在冀州、徐州、青州等地,西晉受漢國侵襲的地區大大增加。永嘉二年(308年),劉淵攻破司州河東郡的蒲阪和平陽郡的平陽城(今山西臨汾市),更遷都蒲子(今山西交口縣),令兩郡屬下各縣抵抗劉淵的營壘都全部投降。同時亦派劉聰、石勒等南攻太行山、趙、魏地區。

十月甲戌日(308年11月2日),劉淵稱帝,改元永鳳。永嘉三年(309年),太史令宣于脩之認為都城蒲子所處崎嶇難以久安,建議遷都平陽。劉淵聽從並立刻遷都至平陽,改元河瑞。劉淵及後派劉聰、王彌等進攻壺關,先破劉琨所派援軍,後於長平擊敗晉東海王司馬越所派的援軍,成功攻陷壺關。劉淵於是先後於當年八月和十月派劉聰等領兵進攻洛陽,但都被晉軍擊敗,劉淵唯有撤軍。

次年劉淵病重,命太宰劉歡樂、太傅劉洋等宗室重臣入宮接受遺詔輔政。七月己卯日(8月19日),劉淵逝世,由太子劉和繼位。九月辛未日(10月20日)下葬永光陵,諡光文皇帝,廟號高祖,後改太祖。

劉淵對部眾的暴行顯得不能容忍,如一次派遣喬晞進攻西河郡,喬晞先殺不肯投降的介休縣令賈渾,後殺哭罵他的賈渾妻宗氏。劉淵知道後大怒,將喬晞追回並降秩四等,又為賈渾收葬。又將領劉景一次進攻黎陽,在延津擊敗晉將王堪後在黃河將三萬多人溺死,劉淵知道後大怒,更說:「劉景還有何顏面見朕!天道又怎能接受這種事!朕想消滅的只是司馬氏,平民有何罪!」於是貶劉景的官位。

根據《晉書》所載,劉淵膂力過人,姿儀魁偉奇特,身長超過兩米,鬍鬚長三尺有餘,其中雜有少量赤色毛髮。

\subsubsection{元熙}

\begin{longtable}{|>{\centering\scriptsize}m{2em}|>{\centering\scriptsize}m{1.3em}|>{\centering}m{8.8em}|}
  % \caption{秦王政}\
  \toprule
  \SimHei \normalsize 年数 & \SimHei \scriptsize 公元 & \SimHei 大事件 \tabularnewline
  % \midrule
  \endfirsthead
  \toprule
  \SimHei \normalsize 年数 & \SimHei \scriptsize 公元 & \SimHei 大事件 \tabularnewline
  \midrule
  \endhead
  \midrule
  元年 & 304 & \tabularnewline\hline
  二年 & 305 & \tabularnewline\hline
  三年 & 306 & \tabularnewline\hline
  四年 & 307 & \tabularnewline\hline
  五年 & 308 & \tabularnewline
  \bottomrule
\end{longtable}

\subsubsection{永凤}

\begin{longtable}{|>{\centering\scriptsize}m{2em}|>{\centering\scriptsize}m{1.3em}|>{\centering}m{8.8em}|}
  % \caption{秦王政}\
  \toprule
  \SimHei \normalsize 年数 & \SimHei \scriptsize 公元 & \SimHei 大事件 \tabularnewline
  % \midrule
  \endfirsthead
  \toprule
  \SimHei \normalsize 年数 & \SimHei \scriptsize 公元 & \SimHei 大事件 \tabularnewline
  \midrule
  \endhead
  \midrule
  元年 & 308 & \tabularnewline\hline
  二年 & 309 & \tabularnewline
  \bottomrule
\end{longtable}

\subsubsection{河瑞}

\begin{longtable}{|>{\centering\scriptsize}m{2em}|>{\centering\scriptsize}m{1.3em}|>{\centering}m{8.8em}|}
  % \caption{秦王政}\
  \toprule
  \SimHei \normalsize 年数 & \SimHei \scriptsize 公元 & \SimHei 大事件 \tabularnewline
  % \midrule
  \endfirsthead
  \toprule
  \SimHei \normalsize 年数 & \SimHei \scriptsize 公元 & \SimHei 大事件 \tabularnewline
  \midrule
  \endhead
  \midrule
  元年 & 309 & \tabularnewline\hline
  二年 & 310 & \tabularnewline
  \bottomrule
\end{longtable}


%%% Local Variables:
%%% mode: latex
%%% TeX-engine: xetex
%%% TeX-master: "../../Main"
%%% End:

%% -*- coding: utf-8 -*-
%% Time-stamp: <Chen Wang: 2019-12-18 14:07:20>

\subsection{昭武帝\tiny(310-318)}

\subsubsection{戾太子生平}

刘和(?-310年),字玄泰,新兴(今山西忻州市)匈奴人。十六國時汉赵國君,光文帝劉淵長子,呼延皇后所生。劉淵死後以太子身份繼位,但即位後即試圖剷除劉聰等勢力,反被劉聰所殺。

劉和身长八尺,雄毅美姿仪,好學,從小开始学习《毛诗》、《左氏春秋》、《郑氏易》。但性格多作猜忌,對屬下無恩德。

永鳳元年(晉永嘉二年,308年),劉淵稱帝,任命劉和為大將軍。兩個月後遷大司馬,封梁王。河瑞二年(晉永嘉四年,310年)被刘渊立为太子。同年刘渊病死,刘和即位,由太宰劉歡樂、太傅劉洋等人輔政。

即位后,卫尉刘锐和劉和舅父宗正呼延攸怨恨自己不被任命為輔政大臣;侍中刘乘則厭惡握有重兵的楚王劉聰,於是共同合謀,向劉和進讒,稱諸王擁兵於都城平陽內外,其中劉聰更加擁兵十萬,嚴重影響劉和的皇權,要劉和有所行動。劉和聽信,及後召其領軍安昌王劉盛和安邑王劉欽將意圖告知。劉盛聽後勸諫劉和不要懷疑兄弟們,但遭呼延攸和劉銳命左右殺死;劉欽見此畏懼,只好對劉和唯命是從。

翌日,劉銳率馬景領兵攻劉聰,呼延攸率永安王劉安國攻齊王劉裕,劉乘則率劉欽攻魯王劉隆,劉和又派尚書田密、武衞將軍劉璿攻北海王劉乂。但田密和劉璿命人攻破城門,並歸降劉聰;而劉銳知劉聰早作準備於是聯合呼延攸等攻擊並殺害劉隆及劉裕,又因害怕劉安國和劉欽有異心而將二人殺害。此時劉聰率軍攻克西明門入宮,劉銳等在劉聰軍前鋒緊追下逃到南宮。劉和則在光極殿西室被殺,劉銳等皆被收捕並被斬首示眾。

劉聰及後自立为帝,改元“光兴”,即昭武皇帝。

\subsubsection{武帝生平}

漢昭武帝刘聪(?-318年8月31日),字玄明,新兴(今山西忻州市)匈奴人。十六国时汉赵国君。汉光文帝劉淵第四子,母张夫人。劉聰學習漢人典籍,深受漢化。執政時期先後派兵攻破洛陽和長安,俘虜並殺害晉懷帝及晉愍帝,覆滅西晉政權並拓展大片疆土。政治上创建了一套胡、汉分治的政治体制。但同時大行殺戮,又寵信宦官和靳準等人,甚至在在位晚期疏於朝政,只顧情色享樂。其執政末期甚至出現「三后並立」的情況。

劉聰年幼時就已經很聰明和好學,令到博士朱紀都覺得十分驚奇。劉聰非但通曉經史和百家之學,更熟讀《孫吳兵法》,而且善寫文章,又習書法,擅長草書和隸書;另外,劉聰亦學習武藝,擅長射箭,能張開三百斤的弓,勇猛矯捷,冠絕一時。可謂文武皆能。

劉聰二十歲後到洛陽遊歷,得到大量名士結交。後擔任新興太守郭頤的主簿。及後遷任右部都尉,因安撫接納得宜而得到匈奴五部豪族的歸心。河間王司馬顒表劉聰為赤沙中郎將,但當時劉淵在鄴城任官,因害怕駐守鄴城的成都王司馬穎加害父親,於是投奔司馬穎,任右積弩將軍,參前鋒戰事。

永安元年(304年),司馬穎任命劉淵為北單于,劉聰於是被立為右賢王,並與父親應命回到匈奴五部為司馬穎帶來匈奴援軍。但劉淵回到五部後就稱大單于,劉聰亦改拜鹿蠡王。劉淵聚眾自立,同年即稱漢王,建立漢國。後來任命劉聰為撫軍將軍。

元熙五年(晉永嘉二年,308年),劉聰被派遣南據太行山。同年年末淵稱帝,劉聰升任車騎大將軍。不久封楚王。次年與王彌和石勒等進攻壺關,擊敗司馬越派去抵抗的施融和曹超,攻破屯留和長子,令上黨太守龐淳獻壺關投降。數月後又領兵攻洛陽,擊敗平北將軍曹武,長驅直進至宜陽。但劉聰因連番勝利而輕敵,被詐降的弘農太守垣延率兵乘夜偷襲劉聰,最終劉聰大敗而還。兩月後劉聰再與王彌、劉曜、呼延翼等進攻洛陽。晉室以為漢國剛遭大敗,短時間不會再南侵,於是疏於防備,知道劉聰等來攻十分畏懼,劉聰更一度進兵至洛陽附近的洛水。當時晉將北宮純率兵夜襲漢國軍壁壘,斬殺將領呼延顥;及後呼延翼更被部下所殺,所率部隊因喪失主帥而潰退,劉淵於是下令撤兵。劉聰則上表稱晉朝軍隊又少又弱,不能因呼延翼等人之死而放棄進攻,堅持要留下來。劉淵允許。而面對漢軍,防守洛陽的司馬越唯有嬰城固守。但及後司馬越乘劉聰到嵩山祭祀的機會派兵進攻留守的漢軍,斬殺呼延朗。安陽王劉厲見此,害怕劉聰怪罪自己而跳進洛水自殺。王彌此時以洛陽守備仍堅固和糧食不繼勸劉聰撤軍,但劉聰因為是自己請求留下,不敢自行撤軍。劉淵及後聽從宣于脩之之言,命劉聰領軍撤退,劉聰見此才撤軍。

劉淵回到平陽後,任命劉聰為大司徒。河瑞二年(晉永嘉四年,310年),劉淵患病,任命劉聰為大司馬、大單于,與太宰劉歡樂和太傅劉洋共錄尚書事,並在都城平陽西置單于臺。不久劉淵逝世,由太子刘和即位。

劉和即位後,受宗正呼延攸、衞尉劉銳及素來厭惡劉聰的侍中劉乘進言唆擺,決意要消除諸王勢力,尤其當時擁兵十萬的劉聰。劉和不久就採取行動,但因劉聰有備而戰,最終劉聰率軍從西明門攻進皇宮,並於光極殿西室殺害劉和,又收捕逃到南宮的呼延攸等人,並將他們斬首示眾。

劉和死後,群臣請劉聰繼位,劉聰以其弟北海王劉乂是單皇后之子而讓位給他,但劉乂仍堅持由劉聰繼位。劉聰最終答應,並說要在劉乂長大後將皇位讓給他,登位後即立劉乂為皇太弟。

刘聪为了稳固地位,又杀死嫡兄刘恭。

劉聰即位後三個月,即派劉曜、王彌和其子河內王劉粲領兵進攻洛陽,因與石勒於大陽會師並在澠池擊敗晉將裴邈,因此直入洛川,擄掠梁、陳、汝南、潁川之間大片土地,並攻陷百多個壁壘。次年,又派前軍大將軍呼延晏領二萬七千人進攻洛陽,行軍至河南時就已十二度擊敗抵抗的晉軍,殺三萬多人。後劉曜、王彌和石勒都奉命與呼延晏會合。呼延晏在劉曜等人未到時就先行進攻洛陽城,攻陷平昌門並大肆搶掠,更於洛水焚毀晉懷帝打算出逃用的船隻。劉曜等人到達後就一起攻進洛陽城,並攻進皇宮縱兵搶掠,盡收皇宮中的宮人和珍寶,又大殺官員和宗室。另外更俘擄晉怀帝和羊皇后,將他們移送到平陽。

永嘉之亂後,劉聰又因晉牙門趙染叛晉歸降而命劉曜和劉粲攻打關中,最終攻陷長安並殺晉南陽王司馬模,並讓劉曜據守長安。但不久就被晉馮翊太守索綝、安定太守賈疋和雍州刺史麴特等反擊,劉曜等兵敗,劉曜更被圍困於長安。終於嘉平二年(永嘉六年,312年)被逼退出長安,撤回平陽。

嘉平二年(312年)年初,劉聰曾派靳沖和卜翊圍困晉并州治所晉陽,但因拓跋猗盧率兵營救而失敗。不久,令狐泥因其父令狐盛被晉并州刺史劉琨殺害而投奔漢國,並說出晉陽虛實。劉聰十分高興,便派劉粲和劉曜攻晉陽,由令狐泥作嚮導。劉琨知道漢國來攻後就到中山郡和常山郡招兵,並向拓跋猗盧求救;同時由張喬和郝詵領兵擋住漢軍。但張、郝皆敗死,劉粲於是乘劉琨未及救援而攻陷並佔領晉陽。但不久拓跋猗盧則親率大軍與劉琨反攻晉陽,劉曜兵敗,唯有棄守晉陽,撤走時遭拓跋猗盧追及,在藍谷交戰但大敗。晉陽得而復失。

晉懷帝被擄至平陽後,就被劉聰任命為特進、左光祿大夫、平阿公。後來改封會稽郡公。劉聰曾與懷帝回憶昔日與王濟造訪他的往事,亦談到西晉八王之亂,宗室相殘之事。劉聰談得十分高興,更賜小劉貴人給懷帝。但於嘉平三年(313年)正月,劉聰在與群臣的宴會中命懷帝以青衣行酒,晉朝舊臣庾珉和王儁見此忍不住心中悲憤而號哭,令劉聰十分厭惡。當時又有人流傳庾珉等會作劉琨的內應以助他攻取平陽,於是殺害懷帝和庾珉等十多名晉朝舊臣。

晉懷帝被殺的消息於四月傳至長安後,在長安的皇太子司馬鄴便即位為晉愍帝。劉聰則派趙染與劉曜和司隸校尉喬智明等進攻長安,多次擊敗抵抗的麴允。趙染後更乘夜攻進長安外城縱火搶掠,至天亮才因麴鑒救援長安而撤出長安,但麴鑒追擊時又遭劉曜擊敗。後因劉曜輕敵而被麴允偷襲,喬智明被殺,劉曜唯有撤兵回平陽。

次年,再派劉曜與趙染出兵長安,索綝領兵抵抗,但趙染初戰於新豐城西因輕敵而敗北。不久二人與將軍殷凱再攻長安,在馮翊擊敗麴允,但當晚又被麴允夜襲殷凱軍營,殷凱戰死。隨後劉曜到懷縣轉攻晉河內太守郭默,但郭默固守不降。在新鄭的李矩此時還到劉琨所派的鮮卑騎兵,說服帶領他們的張肇進攻劉曜。漢國士兵看見鮮卑騎兵就不戰而走,劉聰見進攻不成,打算先消滅劉琨,故命令劉曜撤軍。

建元元年(晉建興三年,315年),劉曜在襄垣擊敗劉琨所派軍隊,並打算進攻陽曲。但此時劉聰又認為要先攻取長安,於是命劉曜撤軍回蒲阪。

劉聰命劉曜撤回蒲阪後數月即派劉曜進攻北地,劉曜先攻破馮翊,後攻上郡,麴允雖然領兵在靈武抵抗但因兵少而不敢進攻。建元二年(316年),劉曜攻取北地,後即進逼長安。雖然有多批援兵救援長安,但都因畏懼漢國軍隊而不敢進擊。而司馬保所派將領胡崧雖然在靈臺擊敗劉曜,但卻因不願見擊退劉曜後麴允和索綝勢力變得強大,竟然勒兵退還槐里。劉曜因而得以攻佔長安外城並圍困愍帝所在小城。在爆發飢荒的小城內死守兩個月後,愍帝決定出降,被送至平陽。西晉正式滅亡。次年出獵時命愍帝穿戎服執戟作前導,被認出後有老人哭泣。劉粲勸劉聰殺愍帝但劉聰想再作觀望。及後又命愍帝行酒、洗爵和執蓋等僕役工作,令晉朝舊臣流淚哭泣,辛賓更抱著愍帝大哭。劉聰終也殺害愍帝。

劉聰自嘉平三年(晉建興二年,314年)十一月立劉粲為相國、大單于,總管各事務後,就將國事委託給他。自己則開始貪圖享樂,次年更設上皇后、左皇后和右皇后以封妃嬪,造成「三后並立」。後來更立中皇后。在委託政務給劉粲的同時,劉聰亦寵信中常侍王沈、宣懷、俞容等人,劉聰因於後宮享樂而長時間不去朝會,群臣有事都會向王沈等人報告而不是上表送呈劉聰。而王沈亦大多不報告劉聰,只以自己喜惡去議決事項。王沈等人又貶抑朝中賢良,任命奸佞小人任官。劉聰又聽信王沈等人的讒言,於建元二年(316年)二月殺特進綦毋達、太中大夫公師彧、尚書王琰等七名宦官厭惡的官員,侍中卜幹哭著勸諫但就遭劉聰免為庶人。

太宰劉易、御史大夫陳元達、金紫光祿大夫劉延和劉聰子大將軍劉敷都曾上表勸諫劉聰不要寵信宦官。但劉聰完全相信王沈等,都不聽從。劉粲與王沈等人勾結,因此向劉聰大讚王沈等人,劉聰聽後即將王沈等人封列侯。劉易見此又上表進諫,終令劉聰發怒,更親手毀壞劉易的諫書,劉易於是怨憤而死;陳元達見劉易之死,亦對劉聰失望,憤而自殺。朝廷在王沈和劉粲等人把持之下綱紀全無,而且貪污盛行,臣下只會奉承上級;對後宮妃嬪宮人的賞賜豐盛,反而在外軍隊卻資源不足。劉敷見此就曾多次勸諫,劉聰卻責罵劉敷常常在他面前哭諫,令劉敷憂憤得病,不久逝世。

因為劉聰的完全信任,王沈和劉粲等人又與靳準聯手誣稱皇太弟劉乂叛變,不但廢掉並殺害劉乂,更趁機誅除一些自己討厭的官員,又坑殺平陽城中一萬五千多名士兵。劉粲在劉乂死後被立為皇太子。

麟嘉三年(318年),刘聪患病,以太宰劉景、大司馬劉驥、太師劉顗、太傅朱紀和太保呼延晏並錄尚書事,又命范隆為守尚書令、儀同三司,靳準為大司空,二人皆決尚書奏事,以作輔政。七月癸亥日(8月31日)逝世,在位九年。諡為昭武皇帝,庙号烈宗。

據說劉聰出生時形體非常,左耳有一白毛,長逾二尺,有光澤。

劉聰雖然因眾意而登位,但仍認為自己是不依長幼次序而被擁立,於是忌憚兄長劉恭,並乘他睡覺時將他刺殺。

劉聰因單太后的絕美姿貌而與她亂倫,單太后子皇太弟劉乂曾多次規勸母親,單太后因而慚愧憤恨而死。雖然及後知道劉乂曾作規勸間接令單太后逝世,但因懷念單太后而沒有廢去其皇太弟身份。

劉聰曾濫殺大臣,如左都水使者王攄就曾因魚蟹供應不足而被劉聰殺害;將作大匠靳陵就因未能如期建成「溫明」、「徽光」二殿而被殺。王彰曾勸諫劉聰不要游獵過度,要劉聰念及劉淵建國艱難,應專心朝政。但劉聰聽後大怒,又要殺王彰,只因太后張夫人絕食以及劉乂和劉粲死諫才赦免王彰。後來設立中皇后時,尚書令王鑒和中書監崔懿之等又諫止劉聰濫封皇后,亦被劉聰所殺。

\subsubsection{光兴}

\begin{longtable}{|>{\centering\scriptsize}m{2em}|>{\centering\scriptsize}m{1.3em}|>{\centering}m{8.8em}|}
  % \caption{秦王政}\
  \toprule
  \SimHei \normalsize 年数 & \SimHei \scriptsize 公元 & \SimHei 大事件 \tabularnewline
  % \midrule
  \endfirsthead
  \toprule
  \SimHei \normalsize 年数 & \SimHei \scriptsize 公元 & \SimHei 大事件 \tabularnewline
  \midrule
  \endhead
  \midrule
  元年 & 310 & \tabularnewline\hline
  二年 & 311 & \tabularnewline
  \bottomrule
\end{longtable}

\subsubsection{嘉平}

\begin{longtable}{|>{\centering\scriptsize}m{2em}|>{\centering\scriptsize}m{1.3em}|>{\centering}m{8.8em}|}
  % \caption{秦王政}\
  \toprule
  \SimHei \normalsize 年数 & \SimHei \scriptsize 公元 & \SimHei 大事件 \tabularnewline
  % \midrule
  \endfirsthead
  \toprule
  \SimHei \normalsize 年数 & \SimHei \scriptsize 公元 & \SimHei 大事件 \tabularnewline
  \midrule
  \endhead
  \midrule
  元年 & 311 & \tabularnewline\hline
  二年 & 312 & \tabularnewline\hline
  三年 & 313 & \tabularnewline\hline
  四年 & 314 & \tabularnewline\hline
  五年 & 315 & \tabularnewline
  \bottomrule
\end{longtable}

\subsubsection{建元}

\begin{longtable}{|>{\centering\scriptsize}m{2em}|>{\centering\scriptsize}m{1.3em}|>{\centering}m{8.8em}|}
  % \caption{秦王政}\
  \toprule
  \SimHei \normalsize 年数 & \SimHei \scriptsize 公元 & \SimHei 大事件 \tabularnewline
  % \midrule
  \endfirsthead
  \toprule
  \SimHei \normalsize 年数 & \SimHei \scriptsize 公元 & \SimHei 大事件 \tabularnewline
  \midrule
  \endhead
  \midrule
  元年 & 315 & \tabularnewline\hline
  二年 & 316 & \tabularnewline
  \bottomrule
\end{longtable}

\subsubsection{麟嘉}

\begin{longtable}{|>{\centering\scriptsize}m{2em}|>{\centering\scriptsize}m{1.3em}|>{\centering}m{8.8em}|}
  % \caption{秦王政}\
  \toprule
  \SimHei \normalsize 年数 & \SimHei \scriptsize 公元 & \SimHei 大事件 \tabularnewline
  % \midrule
  \endfirsthead
  \toprule
  \SimHei \normalsize 年数 & \SimHei \scriptsize 公元 & \SimHei 大事件 \tabularnewline
  \midrule
  \endhead
  \midrule
  元年 & 316 & \tabularnewline\hline
  二年 & 317 & \tabularnewline\hline
  三年 & 318 & \tabularnewline
  \bottomrule
\end{longtable}


%%% Local Variables:
%%% mode: latex
%%% TeX-engine: xetex
%%% TeX-master: "../../Main"
%%% End:

%% -*- coding: utf-8 -*-
%% Time-stamp: <Chen Wang: 2019-12-18 15:50:29>

\subsection{隐帝\tiny(318)}

\subsubsection{生平}

汉隐帝刘粲(?-318年),字士光,新興(今山西忻州市)匈奴人,是十六国时汉赵国君。汉昭武帝刘聪子。劉粲即位後便沉醉於酒色,更與其父的四位皇后亂倫,又大殺輔政大臣,將軍國大事全交給靳準。最終令靳準成功在平陽叛亂,劉粲亦在其中被殺。

劉粲才兼文武,年輕時即為當時俊傑。光興元年(晉永嘉四年,310年)劉聰即位為帝後,封劉粲為河內王,任命為撫軍大將軍,都督中外諸軍事。

永嘉之亂後,因晉牙門趙染叛晉歸降,劉聰命趙染等進攻鎮守長安的南陽王司馬模,劉粲與劉曜則領大軍作趙染後繼。同年攻陷長安,司馬模投降並被劉粲所殺。劉粲於是與劉曜等留守關中地區。但不久,司馬模從事中郎索綝等人圖謀復興晉室,聯合一些不肯投降的郡守起兵進攻長安,並於新豐擊敗劉粲,劉粲被逼撤還首都平陽。

次年,劉聰命劉粲與劉曜領兵進攻晉并州刺史劉琨所在的并州,並成功攻陷治所晉陽。不久劉琨與拓跋猗盧領大軍反攻晉陽,於汾河以東擊敗劉曜。劉曜回晉陽後,與劉粲等擄晉陽城中平民撤退,但被拓跋猗盧追及並於藍谷大戰,最終漢軍大敗,屍橫遍野,但劉粲等人成功撤退。

嘉平四年(建興二年,314年),劉聰升劉粲為丞相、領大將軍、錄尚書事,並進封晉王。年末再升相國、大單于,總管百事。劉聰於是將朝事都交給劉粲等人,漸漸不理朝政。劉粲亦專橫放肆,親近中護軍靳準和中常侍王沈等人而疏遠朝中如陳元禮等忠良官員。性格刻薄無恩,又不聽勸諫。而且又喜好營造宮室,將相國府建得像皇宮一般華麗,國民都開始厭惡他。

及後,因宦官郭猗和靳準都與皇太弟劉乂有積怨,於是建議劉粲誣陷劉乂謀反,以讓劉粲奪去儲君的地位。劉粲聽從,於是命卜抽領兵到東宮監視劉乂。麟嘉二年(建武元年,317年),劉粲命黨羽王平向劉乂說有詔稱京師平陽將有事變,要劉乂要穿護甲在衣內以作防備。劉乂信以為真,更命東宮臣下都穿護甲衣在衣服內。消息被劉粲知道後就派人報告王沈和靳準,靳準於是向劉聰稱劉乂將謀反作亂。劉聰初時不信,但王沈等都說:「臣等早就聽聞了,但怕說出來陛下不相信而已。」劉聰於是派劉粲領兵包圍東宮。同時劉粲又派王沈和靳準收捕十多個氐族和羌族酋長並對他們審問,更加將他們吊起在高處,並用燒熱的鐵灼他們的眼,逼他們誣陷自己與劉乂串通作亂。劉聰於是認定劉乂謀反而靳準等盡忠於他,於是廢劉乂為北部王。劉粲及後就派靳準殺死劉乂。事後劉粲被立為皇太子。

麒嘉三年(318年),劉聰病逝,死前遺命太宰劉景、大司馬劉驥、太師劉顗、太傅朱紀、太保呼延晏、守尚書令范隆和大司空靳準輔政。劉粲隨後繼位。靳準心有異志,於是先打算剷除朝中劉氏勢力,於是向劉粲誣稱一眾王公大臣想行廢立之事,謀圖誅殺皇太后靳月華及自己,改以劉粲弟劉驥掌權,勸劉粲盡早行動。但劉粲不接納。靳準為了令劉粲聽從自己,於是恐嚇靳月華和皇后靳氏,稱一旦劉粲被廢,靳氏一族就會遭到誅殺。二人於是趁劉粲寵幸之機勸說劉粲,終令劉粲聽從,並殺害劉景、劉顗、劉驥、齊王劉勱和大將軍劉逞等人,朱紀和范隆則被逼出奔長安投靠劉曜。八月,劉粲於上林苑閱兵,謀圖進攻擁兵在外的石勒,又以靳準為大將軍,錄尚書事。而劉粲又繼續貪圖酒色歡樂,將軍政大權都交給靳準。而靳準亦扶植宗族勢力,命堂弟靳明為車騎將軍,靳康為衞將軍。

後來,靳準即將作亂,於是招攬年長有德而且有聲望的金紫光祿大夫王延。但王廷不肯與他一同叛亂,並立刻趕去向劉粲報告,但途中遇到靳康並被對方抓去。靳準及後便領兵入宮,在光極前殿命士兵去將劉粲抓來,盡數其罪後將他殺害。諡劉粲為隱皇帝。


\subsubsection{汉昌}

\begin{longtable}{|>{\centering\scriptsize}m{2em}|>{\centering\scriptsize}m{1.3em}|>{\centering}m{8.8em}|}
  % \caption{秦王政}\
  \toprule
  \SimHei \normalsize 年数 & \SimHei \scriptsize 公元 & \SimHei 大事件 \tabularnewline
  % \midrule
  \endfirsthead
  \toprule
  \SimHei \normalsize 年数 & \SimHei \scriptsize 公元 & \SimHei 大事件 \tabularnewline
  \midrule
  \endhead
  \midrule
  元年 & 318 & \tabularnewline
  \bottomrule
\end{longtable}


%%% Local Variables:
%%% mode: latex
%%% TeX-engine: xetex
%%% TeX-master: "../../Main"
%%% End:

%% -*- coding: utf-8 -*-
%% Time-stamp: <Chen Wang: 2019-12-18 15:53:14>

\subsection{刘曜\tiny(318-328)}

\subsubsection{生平}

刘曜(?-329年),字永明,新興(今山西忻州市)匈奴人。是十六国时汉赵(又称前趙)国君。漢趙光文帝劉淵族子。劉曜由漢趙建國開始就經已為國征戰,參與覆滅西晉的戰爭,並於西晉亡後駐鎮長安(今陝西西安市)。後於靳準之亂中登上帝位,後遷都長安。但登位後不久,將領石勒就自立後趙,國家分裂。劉曜在其在位期間多番出兵平定和招降西戎和西方的割據勢力如仇池和前涼等。在國內亦提倡漢學,設立學校。及後與後趙交戰,一度大敗後趙軍並圍攻洛陽(今河南洛陽市),但終被石勒擊敗並被俘。劉曜及後被殺,死後不久前趙亦被後趙所滅。

劉曜年幼喪父,於是由劉淵撫養。年幼聰慧,有非凡氣度。八歲時隨劉淵到西山狩獵,其間因天雨而在一棵樹下避雨,突然一下雷電令該樹震動,旁邊的人都嚇得跌倒,但劉曜卻神色自若,因而得到劉淵欣賞。劉曜喜歡看書,但志在廣泛涉獵而非精讀文句,尤其喜愛兵書,大致都熟讀。劉曜亦擅长写作和書法,習草書和隸書。另一方面劉曜亦雄健威武,箭术娴熟,能一箭射穿寸余厚的铁板,號稱神射。劉曜亦时常自比乐毅、蕭何和曹參,当时人們都不認同,唯刘聪知道其才能。

二十歲時到洛陽遊歷,但期間就被定罪而要被誅殺,於是逃亡到朝鮮,後來遇到朝廷大赦才敢回來。劉曜亦覺得自己外表異於常人,怕不被世人所接納,於是在管涔山隱居。

晉永興元年(304年),劉淵自稱漢王,國號漢,改元元熙任命劉曜為建武將軍。劉曜當年就被派往進攻并州郡縣以開拓疆土。漢永鳳元年(晉永嘉二年,308年),劉淵稱帝,拜劉曜為龍驤大將軍。後封為始安王。漢河瑞元年(晉永嘉三年,309年),劉曜與劉聰等進攻洛陽,但被晉軍乘虛擊敗。河瑞二年(310年),劉淵患病,命劉曜為征討大都督、領單于左輔。不久劉淵逝世,太子劉和繼位。劉和後又被劉聰所殺,劉聰及後登位為帝。

劉聰登位後,不久就命劉曜與河內王劉粲等進攻洛陽,並擊敗晉將裴邈,在梁、陳、汝南、潁川之間大肆搶掠。次年,劉聰命呼延晏領兵攻洛陽,劉曜奉命領兵與其會合,並於六月壬辰日(7月8日)抵達西明門。五日後,劉曜等便攻入洛陽大肆搶掠和殺害大臣,並擄晉懷帝等人,將他們送到平陽(今山西臨汾市)。史稱「永嘉之亂」。當時王彌認為洛陽城池和宮室都完好,建議劉曜向劉聰建議遷都洛陽,但劉曜認為天下未定而洛陽四面受敵,並不可守,於是焚毀洛陽宮殿。

永嘉之亂後,鎮守長安的南陽王司馬模命牙門趙染領兵在蒲阪(今山西省永濟市)守備,但趙染因請求馮翊太守一職被拒絕而投降漢國,劉聰於是在八月命趙染攻取長安,又命劉曜和劉粲領兵跟隨。司馬模兵敗投降,並於九月被劉粲所殺,劉曜則獲任命為車騎大將軍、雍州牧,並改封中山王,鎮守長安。

劉曜取得長安後,司馬模的從事中郎索綝投靠安定太守賈疋,並與賈疋等人圖謀復興晉室,於是推舉賈疋為平西將軍,率五萬兵攻向長安。當時拒降漢國的晉雍州刺史麴特等人亦領兵與賈疋會合。劉曜於是領兵在黃丘與賈疋大戰,但被擊敗。梁州刺史彭蕩仲和駐守新豐(今陝西西安市臨潼區)的劉粲都先後被賈疋等人所擊敗,彭蕩仲死而劉粲北歸平陽,賈疋等人於是聲勢大振,關西胡人和漢人都響應。劉曜只得據守長安。嘉平二年(晉永嘉六年,312年),劉曜因賈疋圍困長安經已數月,且連續戰敗,於是掠長安八萬多名平民棄守長安,逃奔平陽。劉曜及後因長安失守而被貶為龍驤大將軍,行大司馬。

同年,晉并州刺史劉琨部下令狐泥叛歸漢國,劉聰於是命令狐泥作嚮導,以劉粲和劉曜領兵進攻并州治所晉陽(今山西太原市)。二人最終乘虛攻陷晉陽,奪取劉琨的根據地。因此功績,劉曜復任車騎大將軍。但兩個月後,劉琨即與拓跋猗盧聯手反攻晉陽,劉曜在汾河以東與拓跋六脩交戰,但兵敗墜馬並受重傷,因討虜將軍傅虎協助才得以逃回晉陽。劉曜及後掠晉陽城中人民逃歸平陽,但遭拓跋猗盧追及,在藍谷交戰但慘敗。但仍成功回到平陽。

嘉平三年(晉建興元年,313年),劉曜與司隸校尉喬智明等進攻長安,但遭麴允擊敗。次年又與趙染和殷凱進攻長安,但殷凱被麴允擊殺。劉曜於是轉攻河內太守郭默但不能攻破,後更被鮮卑騎兵所嚇退。建元元年(晉建興三年,315年),劉曜一度轉戰并州,雖曾獲勝,但不久又再回到蒲坂準備再次進攻長安。及後劉曜即被派往進攻北地,先攻馮翊而再攻上郡,前去抵抗的麴允不敢進擊。

建元二年(316年),劉曜圍困並攻陷北地,並逼近長安。九月,劉曜雖被司馬保將領胡崧所敗,但胡崧並沒有進一步攻擊,反而退守槐里,而其他援軍亦因懼怕漢國軍而不敢進逼,劉曜於是成功攻陷長安外城,逼得麴允和索綝只好據守城內小城。終於在十一月,因為城內被圍困三個月而食糧嚴重困乏,晉愍帝被逼向劉曜投降。劉曜受降並於隨後遷晉愍帝和眾官員到平陽,西晉正式滅亡。劉曜因此功而獲任命以假黃鉞、大都督、督陝西諸軍事、太宰。並被改封為秦王,再度鎮守長安。

麒嘉三年(晉太興元年,318年),劉聰患病,徵召劉曜為丞相,錄尚書事,與石勒一同受遺詔輔政。但劉曜和石勒都辭讓。劉聰於是任命劉曜為丞相、領雍州牧。同年劉聰死,太子劉粲登位。八月升劉曜為相國、都督中外諸軍事,仍舊鎮守長安。但當月大將軍靳準就叛變,殺害劉粲和大殺劉氏,並自稱漢天王,向東晉稱藩。劉曜知道靳準作亂,於是進兵平陽。

十月,劉曜進佔赤壁(今山西河津縣西北赤石川),太保呼延晏等人從平陽前來歸附,並興早前因靳準誅殺王公而逃至長安的太傅朱紀等共推劉曜為帝。劉曜稱帝後,派征北將軍劉雅和鎮北將軍劉策進屯汾陰(今山西萬榮),與石勒有掎角之勢,共同討伐靳準。

靳準先前已敗於石勒,見劉曜和石勒現在共同討伐自己,於是在十一月派侍中卜泰向石勒請和,但石勒將卜泰囚禁被送交劉曜。劉曜於是向卜泰說:「先帝劉粲在位時確實亂了倫常,司空靳準你執行伊尹和霍光廢立之權,令我得以登位,實在是很大的功勳。若你早早迎接我入平陽,我就要將朝政大事都全部委託給你了,何止免死?你就為我人入城傳話吧。」於是將卜泰送返平陽。靳準聽到卜泰的傳話後,因為自知當日奪權時殺了劉曜母親胡氏和劉曜兄長,於是猶豫不決。十二月,靳康聯結喬泰和王騰等人殺死靳準,共推尚書令靳明為主,又命卜泰帶六顆傳國璽向劉曜投降。此舉令石勒十分憤怒,領兵進攻靳明,靳明大敗而只得退入平陽,嬰城固守。隨後石勒與石虎一同進攻平陽,靳明於是向劉曜求救,劉曜於是派劉雅和劉策迎接,靳明於是帶著一萬五千名平陽人民逃出平陽。劉曜及後卻大殺靳氏,一如靳準殺劉氏一樣。在其欲纳靳康女为妾时,靳女说及家族被灭,号泣请死,刘曜出于哀怜才放过了靳康的一个儿子。

石勒在靳明逃離後亦攻入平陽,留兵戍守後東歸,並於光初元年(晉太興二年,319年)年初命左長史王脩獻捷報給劉曜。劉曜於是派司徒郭汜授予他趙王和太宰、領大將軍的職位,並加如同曹操輔東漢時的特殊禮待。但留仕劉曜的王脩舍人曹平樂卻向劉曜稱王脩此行其實是要來探聽劉曜虛實,以讓石勒趁機襲擊劉曜。劉曜眼見其軍隊疲憊不堪,於是聽信曹平樂之言,追還郭汜並處斬王脩。石勒及後從逃亡回來的王脩副手劉茂口中得知王脩被殺,因此大怒,開始與劉曜交惡。

劉曜回到長安後,即遷都長安,並設立宗廟、社稷壇和祭天地的南北郊。又改國號為「趙」,史稱「前趙」。同年,石勒自稱趙王,正式建立「後趙」。漢國就此一分為二。

及後,黃石屠各人路松多在新平郡和扶風郡起兵,依附南陽王司馬保。司馬保又讓雍州刺史楊曼及扶風太守王連據守陳倉(今陝西寶雞市東),路松多據守草壁。劉曜派劉雅等人進攻但不能攻下。光初二年(320年),劉曜親自率軍進攻陳倉,擊殺王連並逼楊曼投奔氐族。接著接連攻下草壁和安定,令司馬保恐懼而遷守桑城(今甘肃临洮县东)。不久司馬保被部下張春所殺,已向劉曜投降的司馬保部將陳安則請求進攻張春等,劉曜於是任命陳安為大將軍,進攻張春。陳安最終令張春逃至枹罕(今甘肅臨夏),並殺死張春同黨楊次,消滅司馬保殘餘勢力。

不久,前趙將領解虎和長水校尉尹車與巴氐酋長句徐和庫彭等聯結,意圖謀反。但事敗露,劉曜於是誅殺解虎和尹車,並囚禁句徐和庫彭等五十多人,打算誅殺。光祿大夫游子遠極力勸阻,但劉曜都不聽,游子遠一直叩頭至流血,更惹怒劉曜而將他囚禁;劉曜後盡殺句徐等人,更在長安市內將曝屍十日,然後丟進河中。此舉終令巴氐悉數反叛,自稱大秦,並得其他少數民族共三十多萬人響應,於是關中大亂,城門都日夜緊閉。在獄中的游子遠再度上書勸諫劉曜,劉曜看後大怒,命人要立刻殺害游子遠,幸得劉雅等人勸止劉曜,游子遠才得被赦免。劉曜下令內外戒嚴,打算親自討伐叛亂首領句渠知。但此時游子遠向劉曜進言獻策,認為出兵強行鎮壓會耗費太多時間和資源,建議劉曜大赦叛民,讓他們重回正常生活,讓他們自動歸降。又請給兵五千人讓他討伐可能不肯歸降的句渠知。劉曜聽從。隨著大赦令下達,游子遠所到之都有大批人歸降,游子遠又於陰密平定不肯投降的句氏宗族黨眾。及後游子遠更進兵隴右,擊敗自號秦王的虛除權渠,並令他歸降。由於虛除權渠一部是西戎中力量最強的,故此其他西戎部族都相繼歸降前趙。

光初五年(322年),劉曜親征仇池,仇池首領楊難敵率兵迎擊但被擊敗,被逼退保仇池城。此時仇池轄下的氐羌部落大多都向前趙投降。及後劉曜轉攻楊韜,楊韜因畏懼而與隴西太守梁勛等人投降。劉曜於是再攻仇池,但此時劉曜患病,而且軍中有疫症,被逼退兵。劉曜因怕楊難敵乘機追擊,於是派光國中郎將王獷游說楊難敵,最終令楊難敵投降。劉曜於是臣服仇池,並領兵撤回長安。

此時,秦州刺史陳安請求朝見劉曜,但劉曜以患病為由推辭,陳安於是大怒,以為劉曜已死,於是決心反叛。劉曜此時病情卻愈來愈嚴重,改乘馬輿先回長安,而命呼延寔在後守護輜重。但陳安卻領騎兵邀截,俘獲呼延寔並奪取輜重,後更將呼延寔殺害。陳安又派其弟陳集等領騎兵三萬追劉曜車駕,劉曜則派呼延瑜擊殺陳集並盡俘部眾。陳安見此感到恐懼,退還上邽(今甘肃天水市),但隨後又佔領汧城,並得到隴上少數民族的歸附,於是自稱涼王。次年,陳安圍攻前趙征西將軍劉貢,但被歸附前趙的休屠王石武與劉貢的聯軍擊敗,只得收拾兵眾退保隴城(今秦安縣東北)。不久劉曜親自率軍圍困隴城,並派別軍進攻陳安根據地上邽和平襄。陳安於是出城,試圖領上邽和平襄的軍隊解圍,當知道上邽被圍而平襄被攻破後,改為南逃陝中,最終被前趙將領呼延清追及並殺害。上邽和隴城都先後投降,原本歸附陳安的隴上部落都歸降前趙。

平定陳安後,劉曜於當年即進攻前涼,親自率兵臨西河並命二十八萬兵眾沿黃河立營,延綿百多里,軍中鐘鼓之聲震動河水和大地,嚇得前涼沿河的軍旅都望風奔退。劉曜又聲言讓軍隊分百道一同渡河進攻前涼都城姑臧,令前涼震動。前涼君主張茂於是向前趙稱藩。劉曜亦達成目的,領兵退還。

光初七年(324年),後趙司州刺史石生在新安擊斬前趙河南太守尹平,並掠五千多戶東歸。自此前趙和後趙在河東、弘農之間就常有戰事。光初八年(325年),後趙將領石佗攻前趙北羌王盆句徐,大掠而歸。劉曜因而大怒,派中山王劉岳追擊,自己更移屯富平作為聲援,終大敗後趙軍並斬殺石佗。不久後趙西夷中郎將王騰以并州投降前趙。

五月,晉司州刺史李矩等因多次被後趙石生所攻,投靠前趙。劉曜於是派劉岳和呼延謨領兵與李矩等人共同進攻石生。但劉岳圍困石生於金鏞城時,被救援石生的石虎擊敗,退保石梁,更反被石虎所圍;呼延謨亦被石虎所殺。劉曜於是親自率兵救援劉岳,但及後卻因軍中夜驚而被逼退回長安。劉岳因無援而且物資缺乏,終被石虎所俘並送往後趙都城襄國(今河北邢台)。王騰亦為石虎擊敗並殺害,郭默和李矩亦被逼南歸東晉,李矩長史崔宣則向後趙投降。此戰令後趙盡得司州。

光初十一年(328年),石虎領四萬人進攻河東,獲五十多縣反叛響應,於是進攻蒲阪。因楊難敵先於光初八年(325年)反攻前趙於光初六年(323年)所佔領的仇池;又成功抵抗前趙於光初十年(327年)的攻擊。另一方面前涼於光初十年知道前趙光初八年被後趙擊敗後,即恢復其晉朝的官爵,並侵略前趙。劉曜於是派河間王劉述領氐族和羌族兵眾守備秦州以防仇池和前涼從後偷襲,自己則親率全國精銳救援蒲阪。石虎恐懼退軍,劉曜追擊並在高候大敗石虎,斬殺石曕。後劉曜又進攻石生所駐的金鏞城,以千金堨之水灌城,又派兵攻汲郡和河內,令後趙滎陽太守尹矩和野王太守張進等投降。這次大敗震動後趙人心。而劉曜此時卻不安撫士眾,只與寵臣飲酒博戲。

三個月後,石勒親率大軍救援石生,並命石堪等人在滎陽與石勒會師。劉曜在得悉石勒已渡黃河,才建議增加滎陽守戍和封鎖黃馬關以阻後趙軍。不久洛水斥候與石勒前鋒交戰,劉曜從俘獲的羯人口中得知石勒來攻的軍隊強盛才感懼怕,於是解金鏞之圍,在洛水以西佈陣。石勒則領兵進入洛陽城。

後前趙前鋒在西陽門與後趙軍大戰,劉曜親自出戰,但未出戰就已飲酒數斗;出戰後再飲酒一斗多。後趙將石堪乘其酒醉大敗趙軍,劉曜在昏醉中退走,期間墮馬重傷,被石堪俘獲。

劉曜被俘後被送往襄國,途中石勒派李永醫治劉曜。到襄國後,石勒让他住在永豐小城,給予侍姬,更命令劉岳等人去探望劉曜。石勒後來命劉曜寫信勸留守長安的太子劉熙儘快投降,但劉曜卻在信中命令劉熙和大臣們匡正和維護國家,不要因為自己而放棄。石勒看見後感到厭惡,後來劉曜還是被石勒所殺。

刘曜在霸陵西南建寿陵,侍中乔豫、和苞上疏进谏,刘曜对规谏还听得进去。但是刘曜身死国灭,他的实际墓葬地不详。

劉曜高九尺三寸(2.2米以上),垂手過膝,目有赤光,眉色發白,鬚髯雖長卻相當稀疏。劉曜自少就酗酒,及至後來就更加嚴重。在其在洛陽兵敗被俘一戰中臨陣昏醉,可謂其戰敗的其中一個原因。劉曜亦好殺,如靳準之亂中報復性盡誅靳氏和誅殺句徐等人等都可見。對大臣亦時見殺戮,差點殺了游子遠;又以毒酒殺害進言勸諫的大臣郝述和支當。

劉淵:「此吾家千里駒也,從兄為不亡矣。」

劉聰:「永明,世祖、魏武之流,何數公足道哉!」

晉書評:「曜則天資虓勇,運偶時艱,用兵則王翦之倫,好殺亦董公之亞。而承基醜類,或有可稱。」

张茂:“曜可方吕布、关羽,而云孟德不及,岂不过哉。”(《十六国春秋》)

\subsubsection{光初}

\begin{longtable}{|>{\centering\scriptsize}m{2em}|>{\centering\scriptsize}m{1.3em}|>{\centering}m{8.8em}|}
  % \caption{秦王政}\
  \toprule
  \SimHei \normalsize 年数 & \SimHei \scriptsize 公元 & \SimHei 大事件 \tabularnewline
  % \midrule
  \endfirsthead
  \toprule
  \SimHei \normalsize 年数 & \SimHei \scriptsize 公元 & \SimHei 大事件 \tabularnewline
  \midrule
  \endhead
  \midrule
  元年 & 318 & \tabularnewline\hline
  二年 & 319 & \tabularnewline\hline
  三年 & 320 & \tabularnewline\hline
  四年 & 321 & \tabularnewline\hline
  五年 & 322 & \tabularnewline\hline
  六年 & 323 & \tabularnewline\hline
  七年 & 324 & \tabularnewline\hline
  八年 & 325 & \tabularnewline\hline
  九年 & 326 & \tabularnewline\hline
  十年 & 327 & \tabularnewline\hline
  十一年 & 328 & \tabularnewline\hline
  十二年 & 329 & \tabularnewline
  \bottomrule
\end{longtable}

%%% Local Variables:
%%% mode: latex
%%% TeX-engine: xetex
%%% TeX-master: "../../Main"
%%% End:


%%% Local Variables:
%%% mode: latex
%%% TeX-engine: xetex
%%% TeX-master: "../../Main"
%%% End:

%% -*- coding: utf-8 -*-
%% Time-stamp: <Chen Wang: 2019-12-18 16:28:29>


\section{成汉\tiny(306-347)}

\subsection{简介}

成汉(304年-347年)也称成、后蜀,是中国历史上五胡十六国时期之割据政權之一。

301年益州的蜀郡的巴氐族领袖李特在蜀郡地领导西北難民反抗西晋的統治,304年其子李雄称成都王,306年李雄称帝,建国号“成”,建都蜀郡的治所成都。338年李寿改国号为“汉”。其领土疆域为益州全部。347年为东晋桓温攻破成都。

在成漢建國之前,李雄之父李特就在益州發展勢力。297年,李特率領關中流民團南下漢中。302年,李特招集流民團起兵,自稱為使持節、大都督、鎮北大將軍,第二年定年號建初元年(因此有人认为303年也可作为建国年)。率軍攻打成都,益州刺史羅尚拒守,李特敗亡,其弟李流繼續統領流民作戰,然不久後病死。李特之子李雄繼位,並於304年攻下成都,開始稱王,國號「大成」,年號建興。:159

334年,李雄病死,其兄之子李班繼位,不久之後李雄之子李期即殺李班自立。338年,李驤之子李壽又殺了李期自立為帝,將國號改為「漢」。大修宮殿,生活奢侈荒淫,人民受到嚴酷的徭役壓迫。李壽死後,其子李勢繼位,大肆殺伐,國勢更加衰弱。347年,東晉桓溫率兵入蜀,李勢投降,成漢滅亡,立國共44年,两年后残余力量也被东晋消灭。:160

其国号先为“成”,史书也有称“大成”,或以为“大”是尊称。国号“成”来自于成都这个地名,也有说是袭用公孙述的旧称(成家)。后李特弟李骧之子汉王李寿发动兵变夺取政权,改国号为“汉”,史书上又合称为“成汉”,以区别于其他称为“汉”的政权。又其统治地区主要为益州的蜀地,故又被一些史书称为“蜀”(例如《十六国春秋·蜀录》)。《晋书》又称之为“后蜀”,以别于三国时期刘备的前蜀,唐代以后,已基本不使用“后蜀”來指称,而专用作五代十国时期后蜀政权的专称。


%% -*- coding: utf-8 -*-
%% Time-stamp: <Chen Wang: 2019-12-18 15:58:30>

\subsection{李特\tiny(303)}

\subsubsection{生平}

李特(3世紀?-303年),字玄休,西晉末年巴氐人(一說為賨人),其父為李慕;十六國時期成漢國建立者李雄之父,是成漢政權的奠基者。後來李特之子李雄稱王時,追諡李特為成都景王,等到稱帝時,再追諡為景皇帝,廟號為始祖。

李特祖籍為巴西郡宕渠縣(今中國四川省渠縣),其先祖後於曹魏時被遷至略陽(今中國甘肅省秦安縣)。李特身長八尺,在兄弟間排行第二,並與其兄弟都精於騎射,以武略聞名,鄉里紛紛歸附李氏兄弟。西晉元康八年(298年),因齊萬年叛亂使得關中混亂,加上多年饑荒,李特兄弟於是與關中人民一同入蜀。原本朝廷不容許他們進入蜀地,僅讓他們留駐漢中等地,並派侍御史李苾前往慰勞並監察,不容許他們經劍閣入蜀。但因李苾受賄並上奏朝廷,故此李特和一眾中流民都得以在益州和梁州一帶居住。

永康元年(300年),益州刺史趙廞被朝廷徵召為大長秋,原職由成都內史耿滕接任。趙廞身為皇后賈南風姻親,但當年趙王司馬倫就廢黜賈南風並執掌朝政,趙廞因此害怕會因為自己與賈南風的關係而受逼害;而且趙廞亦見晉室宗室相殘,暗有割據巴蜀之意,於是決心叛晉,不旦開倉賑擠流民以收買人心,亦因李特兄弟和其黨眾都強壯勇猛,趙廞於是厚待他們並作為自己爪牙。李特等人亦恃仗趙廞的勢力,聚眾為盜,蜀人視為大患。及後趙廞擊殺耿滕,自稱大都督、大將軍、益州牧。當時李特三弟李庠率親族、黨眾及四千騎兵歸附趙廞,但趙廞因李庠通曉兵法,軍容齊整而感到不快,最終於次年(301年)殺害李庠。

趙廞雖然歸還李庠屍體給李特,並任用李特兄弟為督護以作安撫,但李特兄弟都怨恨趙廞,引兵北歸緜竹。李特後秘密地招收到七千多名兵眾,夜襲並大破趙廞所派北防晉兵的軍隊,並進攻成都(今四川省成都市)。趙廞猝不及防,逃亡被殺,李特則攻陷成都,縱兵大掠,殺趙廞屬官及任命的官員,並派牙門王角及李基向西晉朝廷陳述趙廞罪狀。

在趙廞叛變之時,朝廷另派梁州刺史羅尚入蜀任益州刺史。李特知道羅尚入蜀的消息後十分畏懼,特意派其弟李驤帶著寶物迎接,令羅尚十分高興。李流及後在緜竹為羅尚勞軍,但廣漢太守辛冉和羅尚牙門將王敦卻勸羅尚殺李流。羅尚雖未接納,但李流已經十分畏懼。

及後,朝廷命秦、雍二州召還入蜀的流民。但李特在後來才入蜀的兄長李輔口中得知中國已亂,因此不欲回到關中,於是派閻式請求羅尚,又賄賂羅尚及監督流民回州的御史馮該等,成功讓他們延遲到秋天才起行。同時,朝廷以平定趙廞之功封賞李特,拜李特為宣威將軍,封長樂鄉侯。同時下詔命州府列出當地與李特平定趙廞的流人以作封賞,但辛冉卻沒有如實上報,意圖將平定趙廞作為自己的功勳,於是招來眾人的怨恨。

七月,羅尚再催逼流民起程,然而流民都不願歸去,而且未收割穀物,未有旅費,於是深感憂慮。李特於是再派閻式請求再延遲至冬季才起行,但羅尚聽從辛冉和李苾之言,不再答允。辛冉當時又打算殺害流民首領以獲得他們的物資,於是以當日趙廞敗死時流民大掠成都為由,要在關口搜奪經過的流民的物資財寶。閻式看到這些情形,於是回到李特所駐的緜竹,並勸李特防備可能進襲的辛冉。而當時李特亦因多次為流民發聲,於是獲流民歸心和歸附。而常時李特又將辛冉懸紅捕殺李特兄弟的文告全部收下並改為求取當地李氏、任氏、閻氏等豪族和氐、叟侯王首級,於是令流民大懼,短時間內就有超過二萬人在李特麾下。李特於是特定將部眾分為兩營,分別由自己和李流統率。

不久,辛冉就派广漢都尉曾元、牙門張顯等領兵三萬進攻李特,羅尚亦派督護田佐助戰,而李特因早有準備,下令戒嚴等待曾元等到來。曾元等人到後,李特仍安然躺臥著,沒有任何動作,但當約半數軍隊進入營壘時,李特就命伏兵突擊曾元,大敗敵軍,並殺死曾元、張顯和田佐,並送首給羅尚。李特至此反叛。

當地流民於是共推李特為主,並上書請行鎮北大將軍,承制封拜。隨後便領兵進攻辛冉所在的廣漢,辛冉不敵而退奔德陽。李特在攻佔廣漢後便進攻成都。因羅尚貪婪殘暴,對比李特與蜀人約法三章,並且施捨人民,賑濟借貸,禮賢下士,拔擢人才,軍紀及施政肅然,人民都支持李特。李特屢次擊敗羅尚,羅尚唯有死守成都,並向梁州及南夷都尉李毅求救。

永寧二年(302年),平西將軍、河間王司馬顒派衙博及張微討伐李特,李毅亦派兵支援羅尚,羅尚亦派張龜進攻李特。但李特自領兵擊潰張龜,並命李蕩和李雄攻衙博,不但擊退對方,並收降了巴西郡和葭萌。同年,李特自稱為大將軍、益州牧,都督梁、益二州諸軍事。及後李特就進攻張微,但張微居高據險防守,並趁李特營壘空虛時派兵進攻李特。當時李特處於劣勢,幸李蕩援軍趕到並拼死一戰擊潰張微,才令李特脫險;及後更進攻並斬殺張微。當時羅尚繼續進攻城外李特等軍,但多次交戰皆戰敗,更令李特軍獲得大量兵器和盔甲。及後又多次擊敗梁州刺史許雄所派的軍隊。

太安二年(303年),李特擊潰羅尚駐紥在郫水的水軍,並再進攻成都,蜀郡太守徐儉於是以成都少城投降。但李特進城後僅取用馬匹作軍隊使用,並沒有進行搶掠,並且改元建初。當時蜀人聚居成各個塢自守,都款待李特,李特亦派人安撫並讓流民到各個塢內取食以節省軍糧開支。當時李流和上官惇都勸李特小心各塢都不是誠心支持自己,提防他們反叛,但李特決意安民,不去提防他們。

及後荊州刺史宗岱和建平太守孫阜率水軍救援據守成都太城的羅尚,李特派李蕩等與任臧合兵抵禦。其時宗岱等軍軍勢強盛,令各塢生有二心;同時羅尚又派益州從事任叡又詐降李特,暗中聯結塢主與羅尚一同舉兵,並假稱成都太城內糧食將盡。二月,羅尚率兵乘虛襲擊李特,各塢都響應,於是李特大敗,收兵駐守新繁。後李特見羅尚退兵,於是追擊,最終被羅尚出大軍反擊,李特及李輔和李遠都戰死,屍體被焚毀並送首級到首都洛陽。其弟李流接管其部眾。

\subsubsection{建初}

\begin{longtable}{|>{\centering\scriptsize}m{2em}|>{\centering\scriptsize}m{1.3em}|>{\centering}m{8.8em}|}
  % \caption{秦王政}\
  \toprule
  \SimHei \normalsize 年数 & \SimHei \scriptsize 公元 & \SimHei 大事件 \tabularnewline
  % \midrule
  \endfirsthead
  \toprule
  \SimHei \normalsize 年数 & \SimHei \scriptsize 公元 & \SimHei 大事件 \tabularnewline
  \midrule
  \endhead
  \midrule
  元年 & 303 & \tabularnewline\hline
  二年 & 304 & \tabularnewline
  \bottomrule
\end{longtable}


%%% Local Variables:
%%% mode: latex
%%% TeX-engine: xetex
%%% TeX-master: "../../Main"
%%% End:

%% -*- coding: utf-8 -*-
%% Time-stamp: <Chen Wang: 2019-12-18 16:24:45>

\subsection{武帝\tiny(304-334)}

\subsubsection{生平}

成武帝李雄(274年-334年),字仲儁,氐人,十六国时期成漢開國皇帝(304年至334年在位)。李特第三子,母羅氏。304年李雄自稱成都王,建年號建興。306年正式稱帝,國號大成,史稱成漢。

李雄身高八尺三寸,容貌俊美。少年時以剛烈氣概聞名,常常在鄉里間周旋,有見識的人士都很器重他。有個叫劉化的人,是道家術士,常對人說:“關、隴一帶的士人都將往南去,李家兒子中只有仲俊有非凡的儀表,終歸會成為人主的。 ”

李特在蜀地率流民起義,承皇帝旨意,任命李雄為前將軍。西晉太安二年(303年),李特被益州刺史羅尚擊殺。繼任者李流旋亦病故,李雄自稱大都督、大將軍、益州牧,住在郫城。羅尚派部將攻打李雄,李雄將其擊跑。叔父李驤攻打犍為,切斷羅尚運糧路錢,羅尚的軍隊非常缺糧,攻打得又很急,於是留下牙門羅特固守,羅尚棄城在夜晚逃走。羅特打開城門迎李雄進城,接著攻克成都。在當時李雄的軍隊非常飢餓,於是就率部眾到郪地去就食,挖掘野芋頭來吃。蜀人流亡逃散,往東下到江陽,往南進入七郡。李雄因為西山的范長生居住在山崖洞穴裡,求道養志,想要迎他來立為君而自己做他的臣子。范長生執意推辭。李雄於是盡量避讓,不敢稱制,無論大小事情,都由李國、李離兄弟決斷。李國等人事奉李雄更加恭謹。

永興元年(304年),將領們執意請李雄即尊位,於是李雄自稱成都王,赦免境內罪犯,建年號建興,廢除晉朝法律,約法七章。任命叔父李驤為太傅,兄長李始為太保,折衝將軍李離為太尉,建威將軍李雲為司徒,翊軍將軍李璜為司空,材官李國為太宰,其餘的人委任各自不同。追尊他的曾祖父李虎(即李武)為巴郡桓公,祖父李慕為隴西襄王,父親李特為成都景王,母親羅氏為王太后。范長生從西山乘坐素車來到成都,李雄在門前迎接,執版讓坐,拜為丞相,尊稱為範賢。建興三年(306年),范長生勸李雄稱帝,李雄於是即皇帝位,赦免境內罪犯,改年號為晏平,國號大成,史称成漢。追尊父親李特為景皇帝,廟號始祖,母親羅氏為太后。加授范長生為天地太師,封為西山侯,允許他的部下不參與軍事征伐,租稅全部歸入​​他的家裡。李雄當時建國初始,本來沒有法紀禮儀,將軍們仗著恩情,各自爭奪班次位置。他的尚書令閻式上疏說:“凡是治理國家製定法紀,總是以遵循舊制度為好。漢、晉舊例,只有太尉、大司馬執掌兵權,太傅、太保是父兄一樣的官,講論道義的職位,司徒、司空掌管五教九土的事情。秦代設置丞相,統掌各類政務。漢武末期,破例讓大將軍統掌政務。如今國家的基業剛剛建立,百事還沒有周全,諸公大將們的班列位次有不同,隨之競相請求設置官職,和典章舊制不相符合,應該建立制度來作為楷模法式。”李雄聽從了他的建議。

晏平二年(307年),李雄派李離、李國、李雲等率領二萬徒眾攻入漢中,梁州刺史張殷逃奔到長安。李國等人攻陷南鄭,將漢中人全部遷到蜀地。晏平四年(309年),當時李離駐鎮梓潼,他的部將訇琦、羅羕、張金苟等殺了李離和閻式,以梓潼歸降晉益州刺史羅尚。羅尚派他的部將向奮屯兵在安漢的宜福來威逼李雄,李雄率兵攻打向奮,但是不能克敵。晏平五年(310年),鎮守巴西的李國也被他帳下的文碩殺死,並以巴西投降羅尚。面對如此情況,李雄於是率眾退回,但派他的部將張寶以殺了人逃亡的名義進入了梓潼,並取得訇琦等人的信任。不久,張寶趁訇琦等人出迎羅尚使者的機會關了城門,成功重奪梓潼。正逢羅尚去世,巴郡混亂,李驤攻打涪城。玉衡元年(311年)正月,李驤攻陷涪城,擒獲梓潼太守譙登,接著乘勝進軍討伐文碩,將文碩殺死。李雄很高興,赦免境內罪犯,改年號為玉衡。

玉衡四年(314年),成漢南得漢嘉、涪陵二城,遠方的人相繼歸附,李雄於是下了有關寬大的命令,對投降依附的人都寬免他們的徭役賦稅。虛心而愛惜人才,授職任用都符合接受者的才能,益州於是安定下來。玉衡五年(315年),李雄立其妻任氏為皇后。當時氐王楊難敵兄弟被前趙劉曜打敗,逃奔葭萌,派兒子來成漢作人質。隴西賊人的統帥陳安又依附了李雄。

王衡九年(319年),李雄派李驤征伐越巂郡,於次年逼降越巂太守李釗。李驤進兵從小會攻打寧州刺史王遜,王遜讓他的部將姚岳率全部兵眾迎戰。李驤的軍隊失利,又遇上連日大雨,李驤領軍隊撤回,爭著渡過瀘水,士卒死了很多。李釗到了成都,李雄對待他非常優厚,朝廷的儀式,喪期的禮節,都由李釗決定。

楊難敵、楊堅頭兄弟因敗予前趙而逃奔葭萌時,李雄的侄兒安北將軍李稚優厚地撫慰他們,沒有送其到成都,反待前趙退兵時放他們兄弟回武都,楊難敵於是仗著天險幹了很多不守法紀的事,李稚請求討伐他。李雄不聽群臣諫言,派李稚的長兄中領軍李琀和將軍樂次、費他、李乾等從白水橋進攻下辯,征東將軍李壽督統李琀的弟弟李玝攻打陰平。楊難敵派軍隊抵禦他們,李壽不能推進,可是李琀、李稚長驅直入到達武街。楊難敵派兵切斷他們的後路,四面圍攻,俘虜李琀、李稚,死了數千人。李琀和李稚都是李雄的兄長李蕩的兒子。李雄深深痛悼他們,幾天不吃飯,說起來就流淚,深深地責備自己。

玉衡十四年(324年),李雄打算立兄李蕩之子李班為太子。李雄有十多個兒子,群臣都想立李雄親生的。李雄說:“當初起兵,好比常人舉手保護腦袋一樣,本來不希求帝王的基業。適逢天下喪亂,西晉皇室流離,群情舉兵起義,志在拯救塗炭的生靈,而各位於是推舉我,處在王公的地位之上。這一份基業的建立,功勞本來是先帝的。我兄長是嫡親血統,大柞應歸他繼承,恢弘懿美明智聰睿,就像是上天賦予了他這一使命,大事垂成,死於戰場。李班姿質性情仁厚孝順,好學素有所成,必定會成為大器。”李驤和司徒王達諫阻說:“先王樹立太子的原因,是用來防止篡位奪權的萌芽產生,不能不慎重。吳子捨棄他的兒子而立他的弟弟,所以會有專諸行刺的大禍;宋宣公不立與夷而立宋穆公,終於導致宋督的事變。說到像兒子的話,哪裡比得上真兒子呢?懇請陛下深思。”李雄不聽從,終於立了李班。李驤退下後流著淚說:“禍亂從此開始了!”

前涼文王張駿派遣使者給李雄一封信,勸他去掉皇帝尊號,向晉朝稱藩做屬臣。李雄回信說: “我以前被士大夫們推舉,卻原本無心做帝王,進一步說想成為晉室有大功的臣子,退一步說想和你一樣同為守禦邊藩的將領,掃除亂氛塵埃,以使皇帝的天下安康太平。可是晉室衰微頹敗,恩德聲譽都沒有,我引領東望,有些年月了。正好收到你的來信,在暗室獨處時體會你的真情,感慨無限。知道你想要按照古時候楚漢的舊事,尊奉楚義帝,《春秋》的大義,在這方面沒有人比得上你。”張駿很重視他的話,不斷派使者來往。巴郡曾告急,說有東面來的軍隊。李雄說:“我曾憂慮石勒飛揚跋扈,侵犯威逼琅邪,為這點耿耿於懷。沒想到竟然能夠舉兵,使人感到欣然。”李雄平時清談,有很多類似這樣的話。

李雄因為中原地區喪亡禍亂,就頻繁派遣使者朝貢,和晉穆帝分割天下。張駿統領秦梁二州,在這之前,派傅穎向成漢借道,以便向京師報送表章,李雄不答應。張駿又派治中從事張淳向成漢自稱藩屬,以此來借道。李雄很高興,對張淳說:“貴主英名蓋世,地形險要兵馬強盛,為什麼不自己在一方稱帝?”張淳說:“寡君因為先祖世代是忠良,沒能夠為天下雪恥,解眾人於倒懸,因而日頭偏西還想不起吃飯,枕戈待旦。想憑藉琅邪來中興江東,所以遠隔萬里仍然翼戴朝廷,打算成就齊桓公、晉文公一樣的事業,說什麼自取天下呢!”李雄表情慚愧,說:“我的先祖先父也是晉朝臣民,從前和六郡人避難到此,被同盟的人推舉,才有今天。琅邪如果能在中原使大晉中興,我也會率眾人助他一臂之力。”張淳回去後,向京師報送了表章,天子讚揚了他們。

當時李驤去世,李雄任命李驤的兒子李壽為大將軍、西夷校尉。玉衡二十年(330年)十月,李壽督率征南將軍費黑、征東將軍任巳攻陷巴東,太守楊謙退守建平。李壽另派費黑侵擾建平,東晉巴東監軍毌丘奧退守宜都。

玉衡二十一年(331年)七月,李壽進攻陰平、武都,氐王楊難敵投降。。玉衡二十三年(333年),李雄再派李壽進攻朱提,任命費黑、仰攀為先鋒,又派鎮南將軍任回征伐木落,分散寧州的援兵。寧州刺史尹奉投降,於是佔有南中地區。李雄在這種情況下赦免境內罪犯,派李班討伐平定寧州的夷人,任命李班為撫軍。

玉衡二十四年(334年),李雄頭上生毒瘡。六月二十五日,李雄去世,時年六十一歲,在位三十一年。諡號武皇帝,廟號太宗。葬於安都陵。

李雄的母親羅氏,夢見兩道彩虹從門口升向天空,其中一道虹中間斷開,而後生下李蕩。後來羅氏因為去打水,忽然間像是睡著了,又夢見大蛇繞在她的身上,於是有了身孕,十四個月之後才生下李雄。羅氏常常說:「我的兩個兒子如果有先死的,活著的必定有大富貴。」最終李盪死在李雄前面。

李雄的母親羅氏去世時,李雄相信巫師的話,有很多忌諱,以至於想不入葬。他的司空趙肅諫阻他,李雄才聽從了。李雄想行三年守喪之禮,群臣執意諫阻,李雄不聽。李驤對司空上官惇說:“如今正有急難還沒有消解,我想堅持諫阻,不讓主上最終守居喪之禮,你認為怎麼樣?”上官惇說:“三年的喪制,從天子直到庶人,所以孔子說:'不一定是高宗,古時候的人都是這樣。'但是漢魏以後,天下多難,宗廟是最重要的,不能長時間無人管理,所以不行衰絰一類的禮,盡哀就罷了。”李驤說:“任回將要到來,這個人在處事方面很有決斷,而且主上常常很難不聽他的話,等他到了,就和他一起去請求。”任回抵達後,李驤和任回一同去見李雄。李驤脫去冠流著淚,一再請求因公除去喪服。李雄大哭不答應。任回跪著上前說:“如今王業剛剛開始建立,各種事情都在草創階段,一天沒有主上,天下人心惶惶。從前周武王披著素甲檢閱軍隊,晉襄公繫著墨絰出征,難道是他們希望做的嗎?是為了天下人而委屈自己的原故呀!希望陛下割捨親情順從權宜的方法,以使國運永遠興隆。”於是強行扶李雄起來,脫去喪服親理政事。


\subsubsection{建兴}

\begin{longtable}{|>{\centering\scriptsize}m{2em}|>{\centering\scriptsize}m{1.3em}|>{\centering}m{8.8em}|}
  % \caption{秦王政}\
  \toprule
  \SimHei \normalsize 年数 & \SimHei \scriptsize 公元 & \SimHei 大事件 \tabularnewline
  % \midrule
  \endfirsthead
  \toprule
  \SimHei \normalsize 年数 & \SimHei \scriptsize 公元 & \SimHei 大事件 \tabularnewline
  \midrule
  \endhead
  \midrule
  元年 & 304 & \tabularnewline\hline
  二年 & 305 & \tabularnewline\hline
  三年 & 306 & \tabularnewline
  \bottomrule
\end{longtable}

\subsubsection{晏平}

\begin{longtable}{|>{\centering\scriptsize}m{2em}|>{\centering\scriptsize}m{1.3em}|>{\centering}m{8.8em}|}
  % \caption{秦王政}\
  \toprule
  \SimHei \normalsize 年数 & \SimHei \scriptsize 公元 & \SimHei 大事件 \tabularnewline
  % \midrule
  \endfirsthead
  \toprule
  \SimHei \normalsize 年数 & \SimHei \scriptsize 公元 & \SimHei 大事件 \tabularnewline
  \midrule
  \endhead
  \midrule
  元年 & 306 & \tabularnewline\hline
  二年 & 307 & \tabularnewline\hline
  三年 & 308 & \tabularnewline\hline
  四年 & 309 & \tabularnewline\hline
  五年 & 310 & \tabularnewline
  \bottomrule
\end{longtable}

\subsubsection{玉衡}

\begin{longtable}{|>{\centering\scriptsize}m{2em}|>{\centering\scriptsize}m{1.3em}|>{\centering}m{8.8em}|}
  % \caption{秦王政}\
  \toprule
  \SimHei \normalsize 年数 & \SimHei \scriptsize 公元 & \SimHei 大事件 \tabularnewline
  % \midrule
  \endfirsthead
  \toprule
  \SimHei \normalsize 年数 & \SimHei \scriptsize 公元 & \SimHei 大事件 \tabularnewline
  \midrule
  \endhead
  \midrule
  元年 & 311 & \tabularnewline\hline
  二年 & 312 & \tabularnewline\hline
  三年 & 313 & \tabularnewline\hline
  四年 & 314 & \tabularnewline\hline
  五年 & 315 & \tabularnewline\hline
  六年 & 316 & \tabularnewline\hline
  七年 & 317 & \tabularnewline\hline
  八年 & 318 & \tabularnewline\hline
  九年 & 319 & \tabularnewline\hline
  十年 & 320 & \tabularnewline\hline
  十一年 & 321 & \tabularnewline\hline
  十二年 & 322 & \tabularnewline\hline
  十三年 & 323 & \tabularnewline\hline
  十四年 & 324 & \tabularnewline\hline
  十五年 & 325 & \tabularnewline\hline
  十六年 & 326 & \tabularnewline\hline
  十七年 & 327 & \tabularnewline\hline
  十八年 & 328 & \tabularnewline\hline
  十九年 & 329 & \tabularnewline\hline
  二十年 & 330 & \tabularnewline\hline
  二一年 & 331 & \tabularnewline\hline
  二二年 & 332 & \tabularnewline\hline
  二三年 & 333 & \tabularnewline\hline
  二四年 & 334 & \tabularnewline
  \bottomrule
\end{longtable}


%%% Local Variables:
%%% mode: latex
%%% TeX-engine: xetex
%%% TeX-master: "../../Main"
%%% End:

%% -*- coding: utf-8 -*-
%% Time-stamp: <Chen Wang: 2021-11-01 11:52:53>

\subsection{幽公\tiny(334-338)}

\subsubsection{哀帝李班生平}

成哀帝李班(288年-334年),字世文。十六国时期成汉政权的皇帝。为李雄之兄李荡之子。

李班初任平南將軍。李班的叔父李雄雖然有十個兒子,但都不成氣候,所以李雄捨棄自己的兒子而立李班為太子。

李班為人謙虛能廣泛採納意見,尊敬愛護儒士賢人,從何點、李釗以下,李班皆以老師的禮節對待他們,又接納名士王嘏和隴西人董融、天水人文夔等作為賓客朋友。常常對董融等人說:“看到周景王的太子晉、曹魏的太子曹丕、東吳的太子孫登,文章審察辨識的能力,超然出群,自己總是感到慚愧。怎麼古代的賢人那樣高明,而後人就是望塵莫及呀!”李班為人性情博愛,行為符合軌範法度。當時李氏的子弟都崇尚奢侈靡費,可是李班常常自省自勉。每當朝廷上有重大問題要討論,叔父李雄總是讓他參與。李班認為:“古時候開墾的田地平均分配,不論貧富可以一樣獲得土地,如今顯貴人物佔有大面積的荒田,貧苦人想耕種卻沒有土地,佔地多的人將自己多餘的土地出售給他們,這哪裡是王者使天下均等的大義呀!”李雄採納了他的意見。

玉衡二十四年(334年),李雄臥病不起,李班日夜侍奉在身邊。李雄年輕時頻頻作戰,受了很多傷,到這時病重,疤痕全部化膿潰爛,李雄的兒子李越等人都因厭惡而遠遠躲開。李班替他吸吮膿汁。毫無為難的表情,往往在嘗藥時流淚,不脫衣冠地服侍,他的孝心誠意大多如此。

同年六月二十五日,李雄去世,李班即位。任命堂叔建寧王李壽為錄尚書事,來輔佐朝政。李班在宮中依禮服喪,政事都委託給李壽和司徒何點、尚書令王瑰等人。當時李越鎮守江陽,因為李班不是父親李雄的兒子,心中很是不滿。同年九月,李越回到成都奔喪,和他的弟弟安東將軍李期密謀除掉李班。李班的兄弟李玝勸李班遣送李越回江陽,任命李期為梁州刺史,鎮守葭萌。李班認為李雄還未下葬,不忍心讓他們走,推誠待人而心地仁厚,沒有一點嫌隙。當時有兩道白氣出現在天空中,太史令韓豹奏道:“宮中有秘密陰謀的殺氣,要對親戚加以戒備。”李班沒有明白。十月,李班因為夜晚去哭靈,李越在殯宮殺了李班,時年四十七歲,李班共在位一年,於是群臣立李雄的兒子李期繼位。

\subsubsection{幽公李期生平}

李期(314年-338年),字世運,是十六国時期成汉政权的皇帝。为李雄第四子。

李期聰慧好學,二十歲時就能作文章,輕財物而好施捨,虛心招納人才。初任建威將軍,其父李雄讓兒子們和宗室的子弟們各自憑恩德信義聚集徒眾,多的不到數百人,可是李期單單招到了上千人。他推薦的人,李雄多半任用,所以長史、列署有不少出自他的門下。

玉衡二十四年(334年),李雄死,太子李班繼位。李雄之子李越回成都奔喪時與李期殺掉李班。

因李期多才多兿、并由皇后任氏(李雄正妻)養大而被擁立,即位改元玉恆。李期即位後,首先誅殺李班的弟弟李都。派堂叔李壽到涪城討伐李都之弟李玝,李玝棄城投降東晉。李期封李壽為漢王,任命他為梁州刺史、東羌校尉、中護軍、錄尚書事;封兄長李越為建寧王,任命為相國、大將軍、錄尚書事。立妻子閻氏為皇后。任命衛將軍尹奉為右丞相、驃騎將軍、尚書令,王瑰為司徒。李期自認為圖謀大事已經成功,不重視各位舊臣,在外則信任尚書令景騫、尚書姚華、田褒。田褒沒有別的才能,李雄在位時期,曾勸其立李期為太子,所以李期非常寵幸厚待他。對內則相信宦官許涪等人。國家的刑獄政事,很少讓卿相過問,獎賞和刑罰,都由這幾個人決定,於是國家的法紀紊亂。竟然誣陷尚書僕射、武陵公李載謀反,致使李載被下獄而死。在此之前,東晉建威將軍司馬勛屯兵漢中,李期派李壽攻陷漢中,於是設置守官,設防於南鄭。

李雄的兒子李霸、李保都無病而死,都說是李期毒死了他們,於是大臣們心懷恐懼,人人不能心安。李期誅殺夷滅了很多人家,抄沒他們的婦女和財物來充實自己的後庭,宮內宮外人心惶惶,路上相見也只敢用目光打招呼,勸諫的人都定了罪,人人只想苟且免禍。李期又毒死李壽的養弟安北將軍李攸,和李越、景騫、田褒、姚華商議襲擊李壽等人,打算燒毀市橋而發兵。李期又多次派中常侍許涪到李壽那裡去,察看他的動靜。

李攸死後,李壽非常害怕,又疑心許涪往來頻繁的情況。於玉恒四年(338年),率領一萬步兵、騎兵,從涪城出發前往成都,聲稱景騫、田褒擾亂朝政,所以發動晉陽兵士,以清除李期身邊的惡人。李壽到達成都,李期、李越沒料到他會來,一向不加防備,李壽於是佔領成都,駐兵到宮門前。李期派侍中慰勞李壽,李壽上奏章說李越、景騫,田褒、姚華、許涪、征西將軍李遐、將軍李西等人都心懷奸詐擾亂朝政,圖謀傾覆社稷,大逆不道,罪該誅殺。李期順從了李壽的意見,於是殺死李越、景騫等人。李壽假託太后任氏的名義下令,將李期廢為“邛都縣公”,幽禁在別宮內。李期嘆息說天下的君主竟然成了一個小小的縣公,真是生不如死。同年(338年),李期自缢而死,時年25歲,諡號幽公。

\subsubsection{玉恒}

\begin{longtable}{|>{\centering\scriptsize}m{2em}|>{\centering\scriptsize}m{1.3em}|>{\centering}m{8.8em}|}
  % \caption{秦王政}\
  \toprule
  \SimHei \normalsize 年数 & \SimHei \scriptsize 公元 & \SimHei 大事件 \tabularnewline
  % \midrule
  \endfirsthead
  \toprule
  \SimHei \normalsize 年数 & \SimHei \scriptsize 公元 & \SimHei 大事件 \tabularnewline
  \midrule
  \endhead
  \midrule
  元年 & 335 & \tabularnewline\hline
  二年 & 336 & \tabularnewline\hline
  三年 & 337 & \tabularnewline\hline
  四年 & 338 & \tabularnewline
  \bottomrule
\end{longtable}


%%% Local Variables:
%%% mode: latex
%%% TeX-engine: xetex
%%% TeX-master: "../../Main"
%%% End:

%% -*- coding: utf-8 -*-
%% Time-stamp: <Chen Wang: 2021-11-01 11:53:07>

\subsection{昭文帝李寿\tiny(338-343)}

\subsubsection{生平}

汉昭文帝李寿(300年-343年),字武考,十六国时期成汉政权的皇帝。为李特之弟李骧少子。

338年即位后改国号为“汉”。343年病死。

李壽天生聰敏好學,少尚禮容,在李氏諸子中相當突出,受到李雄欣賞,認為他足以擔當大任,乃授以前將軍,統領巴蜀軍事,彼遷征東將軍,當時年僅十九歲。在任期間以處士譙秀為謀主,對其言聽計從,令他在巴蜀威德日隆。

李驤卒,李壽再先後升遷為大將軍、大都督、侍中、並封扶風公、錄尚書事。在出征寧州時,圍攻百餘日,最終悉數平定諸郡,李雄大悅,再加封建寧王。

李雄卒,受遺命輔政,李期繼位,改封漢王,兼任梁州刺史,獲賜封梁州五郡。

自此,李壽威名遠播,卻同時深為李越、景騫等所忌憚,令李壽深為擔憂。在暫代李玝屯田涪水期間,每次朝覲日期到來,往趁以邊景賊寇橫行,不可放鬆戒備離開而推卻。同時李壽又因為李期、李越兄弟等十餘人年紀漸長,又擁有精兵,擔心不能自全,便數次欲聘得龔壯為其效命。龔壯雖然不答應,不過仍多次與李壽見面。時值岷山崩塌,江水因此枯竭,李壽認為此乃上天預示災劫,因而非常厭惡,便問龔壯自安之法。龔壯的父親及叔父,被李特殺害,為了假借李壽之手報仇,便向李壽提出起兵自立以自保的建議,最終獲得李壽採納,之後便暗中與長史略陽羅恆、巴西解思明共同謀奪首都成都,並得數千人加入。李壽軍起兵突襲成都,將其攻克,縱兵擄掠,甚至姦污李雄女兒及李氏諸婦,並將之殘殺。羅恆、解思明、李奕、王利等人乃勸李壽自稱鎮西將軍、益州牧、成都王,並向晉朝稱藩。

成玉恆四年(338年),大臣任調、司馬蔡興、侍中李艷以及張烈等勸李壽自立。李壽命巫師卜卦,得出「可當數年天子」的預示,任調大喜,進言「一日尚且滿足,何況數年!」(一日尚為足,而況數年乎!)李壽以「有道是「早上聽到警世的道理,就算當晚要死亦無悔無憾」(朝聞道,夕死可矣),任調的進言,實在是上乘之策!」乃自稱為帝,舉國大赦,並改元漢興,以董皎為相國、羅恆、馬當為股肱,李奕、任調、李閎為親信,解思明為謀主。李壽本想向龔壯,授安車束帛以命為大師,然而龔壯拒絕,僅接收縞巾素帶,以師友之位自居。同時拔擢幽滯,授以顯位。並追尊李驤為獻帝、母昝氏為太后、妻阎氏為皇后、世子李勢為太子。

李壽稱帝後,有人狀告廣漢太守李乾與大臣串通,密謀廢帝。李壽命兒子李廣與大臣齊集殿前,將李乾徙為漢嘉太守。一次遇上狂風暴雨,震動大履門柱,李壽為此深自悔責,下命郡臣要盡忠進言,切切拘泥忌諱。

後來後趙石虎向李壽提議結盟出兵晉朝,事成後兩人並分天下,李壽大悅,先是大修船艦,嚴兵繕甲,又令吏卒準備充足糧草。繼而以尚書令馬當為六軍都督,準備以七萬人兵力,乘舟溯江而上。當船隊經過成都時,鼓聲震天,李壽登城檢閱,群臣趁機以國小眾寡,吳越、會稽路遠,不易成功為由出言阻止,尤其解思明更是切誎懇至,於是李壽便讓群臣力陳利害。龔壯誎曰﹕「陞下與胡人互通,是否會比與晉朝更好呢? 胡人素來是豺狼一樣的國家。晉朝被滅,才不得不北面事之。如果與他們爭奪天下,結果只會令強弱更加懸殊,昔日虞國、虢國(成語「假途滅虢」的典故)的教訓在前,希望陛下可以深思熟慮。」群臣都同意龔壯的進言,更叩頭泣誎,終使李壽放棄,士眾大喜,更連聲萬歲。

之後李壽派遣鎮東將軍李奕征討牂柯,太守謝恕據城堅守多日未能攻克,適逢李奕糧盡,因而撤兵。同時以太子李勢為大將軍、錄尚書事。

李壽繼承李雄,同樣為政寬儉,在篡位之初,亦未表現其欲望。某次李閎、王嘏從鄴城歸來,盛讚石季龍的宮殿華麗,鄴中戶口殷實。卻同時聽聞石季龍濫用刑法,王侯表現不遜,亦以殺罰懲戒,反而能夠控制各地邦域,令李壽相當羨慕,決心效法他。下臣每有小過,動輒處死以立威。又以都城空虛、鄉效戶口未至充實、工匠器械仍未滿盈為由,遷徙鄰郡戶有三名男丁以上的家戶到成都,又建造尚方御府,派遣各州郡能工巧匠以充實之,並廣修宮室、引水入城,極盡奢華,又擴充太學、建立宴殿等,令百姓疲於奔命,悲呼嗟嘆怨聲載道,以致人心思亂者,竟有十之八九。左僕射蔡興進誎阻止,李壽以其散播謗言為由,將他處死。右僕射李嶷因為經常直言忤逆意旨,李壽對他素有積怨,便假以他罪將他收監然後處死。

後來李壽患有重病,經常夢見李期、蔡興索命。解思明等復議再次尊奉皇室,李壽不從。李演亦由越雟上書,勸他歸正返本,放棄稱帝,復稱為王,李壽大怒殺之,以警告龔壯、解思明等。龔壯於是作詩七篇,假借應璩之口諷刺李壽,李壽便回應道﹕「有道是「反省詩詞便可知其意思」,如果這篇是今人所作,就是賢哲之話語; 假若是古人所作,便只是死去鬼魂的平常辭令!」

漢興六年(343年),最終在憂患之中病死,享年四十四歲。李壽在位五年,谥昭文皇帝,廟號中宗,葬安昌陵。

李壽為帝之初,好學愛士,即使庶民小兒也對他稱道不已。每次閱到良將賢相建功立業的事蹟時,沒有一次不會反覆誦讀,故能征伐四克,開闢千里疆土。未稱帝之前,相對李雄一心求上,李壽亦能盡誠於下,因此被稱為賢相。到即位之後,改立宗廟,以父李驤為漢始祖廟、李特、李雄為大成廟,又下旨强調與李期、李越並非同族,大凡期、越時定制,都有所改動。公卿之下,悉數任用自己的幕僚輔佐。李雄時的舊臣以及六郡士人,全部罷黜。而李壽相當仰慕漢武帝、魏明帝的所為,同時恥於聞說父兄之事,禁止上書者妄言前任教化功績,而只能提及李壽在位時的當世事功。

\subsubsection{汉兴}

\begin{longtable}{|>{\centering\scriptsize}m{2em}|>{\centering\scriptsize}m{1.3em}|>{\centering}m{8.8em}|}
  % \caption{秦王政}\
  \toprule
  \SimHei \normalsize 年数 & \SimHei \scriptsize 公元 & \SimHei 大事件 \tabularnewline
  % \midrule
  \endfirsthead
  \toprule
  \SimHei \normalsize 年数 & \SimHei \scriptsize 公元 & \SimHei 大事件 \tabularnewline
  \midrule
  \endhead
  \midrule
  元年 & 338 & \tabularnewline\hline
  二年 & 339 & \tabularnewline\hline
  三年 & 340 & \tabularnewline\hline
  四年 & 341 & \tabularnewline\hline
  五年 & 342 & \tabularnewline\hline
  六年 & 343 & \tabularnewline
  \bottomrule
\end{longtable}


%%% Local Variables:
%%% mode: latex
%%% TeX-engine: xetex
%%% TeX-master: "../../Main"
%%% End:

%% -*- coding: utf-8 -*-
%% Time-stamp: <Chen Wang: 2019-12-18 16:34:50>

\subsection{李势\tiny(343-347)}

\subsubsection{生平}

李势(4世紀-361年),字子仁,十六国成汉末主。李寿长子,母李氏。降晉後,封歸義侯,卒於建康,後世稱「後主」。無子。

初,李壽妻閻氏無子,李驤殺李鳳,為李壽納李鳳之女為妻,生李勢。李期愛李勢姿貌,拜他為翊軍將軍、漢王世子。李勢身長七尺九寸,腰帶十四圍,善於俯仰,時人異之。成漢漢興六年(343年),李壽死,李勢嗣偽位,赦其境內,改元曰太和。尊母閻氏為太后,妻李氏為皇后。

太史令韓皓奏熒惑守心,以過廟禮廢,勢命群臣議之。其相國董皎、侍中王嘏等以為景武昌業,獻文承基,至親不遠,無宜疏絕。勢更令祭特、雄,同號曰漢王。

李勢弟大將軍、漢王李廣以李勢無子,求為太弟,李勢不許。馬當、解思明以李勢兄弟不多,若有所廢,則益孤危,固勸李勢准許。李勢疑當等與李廣有叛謀,遣其太保李奕襲廣於涪城,命董皎收馬當、解思明二人斬殺,夷其三族。貶李廣為臨邛侯,李廣自殺。解思明有計謀,強作諫諍,馬當甚得人心。自此之後,無複紀綱及諫諍者。

李奕自晉壽舉兵反之,蜀人多有從奕者,眾至數萬。勢登城距戰。奕單騎突門,門者射而殺之,眾乃潰散。勢既誅奕,大赦境內,改年嘉寧。

初,蜀土無獠,至此,始從山而出,北至犍為,梓潼,布在山谷,十余萬落,不可禁制,大為百姓之患。勢既驕吝,而性愛財色,常殺人而取其妻,荒淫不恤國事。夷獠叛亂,軍守離缺,境宇日蹙。加之荒儉,性多忌害,誅殘大臣,刑獄濫加,人懷危懼。斥外父祖臣佐,親任左右小人,群小因行威福。又常居內,少見公卿。史官屢陳災譴,乃加董皎太師,以名位優之,實欲與分災眚。

晉永和二年(346年)末,晉大司馬桓溫率水軍伐勢。桓溫次青衣,李勢大發軍距守,又遣李福與昝堅等數千人從山陽趣合水距溫。謂溫從步道而上,諸將皆欲設伏於江南以待王師,昝堅不從,率諸軍從江北鴛鴦碕渡向犍為,而桓溫從山陽出江南,昝堅到犍為,方知與溫異道,乃回從沙頭津北渡。及昝堅至,溫已造成都之十裏陌,昝堅之兵眾自潰。桓溫至城下,縱火燒其大城諸門。李勢的兵眾惶懼,無複固志,其中書監王嘏、散騎常侍常璩等勸李勢投降。

李勢以問侍中馮孚,馮孚言:「昔吳漢征蜀,盡誅公孫氏。今晉下書,不赦諸李,雖降,恐無全理。」勢乃夜出東門,與昝堅走至晉壽(今四川广元),然後送降文于溫曰:「偽嘉寧二年三月十七日,略陽李勢叩頭死罪。伏惟大將軍節下,先人播流,恃險因釁,竊自汶、蜀。勢以暗弱,複統未緒,偷安荏苒,未能改圖。猥煩硃軒,踐冒險阻。將士狂愚,干犯天威。仰慚俯愧,精魂飛散,甘受斧鑕,以釁軍鼓。伏惟大晉,天網恢弘,澤及四海,恩過陽日。逼迫倉卒,自投草野。即日到白水城,謹遣私署散騎常侍王幼奉箋以聞,並敕州郡投戈釋杖。窮池之魚,待命漏刻。」勢尋輿櫬面縛軍門,溫解其縛,焚其櫬,遷勢及弟福、從兄權親族十余人于建康,封勢歸義侯。升平五年(361年),卒於建康。在位五年而敗。

《妒记》記載晉時宣武候桓溫平蜀後,娶了李勢的妹妹為妾。桓溫妻南康公主知道後大為妒忌,乃拔刃往李氏居所,準備砍人。結果看見李氏正在窗前梳頭,「姿貌端丽,徐徐结发,敛手向主,神色闲正,辞甚凄惋。」見美而生憐生愛,於是擲刀前抱之曰:「阿子,我見汝亦憐,何況老奴。」意思是說連我看了都會心動了,更何況是那老傢伙。成語「我見猶憐」因此而來。

\subsubsection{太和}

\begin{longtable}{|>{\centering\scriptsize}m{2em}|>{\centering\scriptsize}m{1.3em}|>{\centering}m{8.8em}|}
  % \caption{秦王政}\
  \toprule
  \SimHei \normalsize 年数 & \SimHei \scriptsize 公元 & \SimHei 大事件 \tabularnewline
  % \midrule
  \endfirsthead
  \toprule
  \SimHei \normalsize 年数 & \SimHei \scriptsize 公元 & \SimHei 大事件 \tabularnewline
  \midrule
  \endhead
  \midrule
  元年 & 344 & \tabularnewline\hline
  二年 & 345 & \tabularnewline\hline
  三年 & 346 & \tabularnewline
  \bottomrule
\end{longtable}

\subsubsection{嘉宁}

\begin{longtable}{|>{\centering\scriptsize}m{2em}|>{\centering\scriptsize}m{1.3em}|>{\centering}m{8.8em}|}
  % \caption{秦王政}\
  \toprule
  \SimHei \normalsize 年数 & \SimHei \scriptsize 公元 & \SimHei 大事件 \tabularnewline
  % \midrule
  \endfirsthead
  \toprule
  \SimHei \normalsize 年数 & \SimHei \scriptsize 公元 & \SimHei 大事件 \tabularnewline
  \midrule
  \endhead
  \midrule
  元年 & 346 & \tabularnewline\hline
  二年 & 347 & \tabularnewline
  \bottomrule
\end{longtable}

\subsubsection{范賁生平}

范賁(3世紀?-349年),中國十六國初期東晉境內蜀地(今中國四川省)的民變領袖之一,是成漢丞相范長生之子。曾任成漢的侍中一職,318年范長生去世後,接任丞相。

范長生博學多聞,年近百歲才去世,而被蜀地之人敬若神明。347年,成漢被東晉所滅,成漢將領因此推范賁為帝,根據史書記載,范賁「以妖異惑眾」,因此蜀地很多人歸附。

349年,東晉益州刺史周撫、龍驤將軍朱燾攻擊范賁,范賁被殺,遂平定益州。


%%% Local Variables:
%%% mode: latex
%%% TeX-engine: xetex
%%% TeX-master: "../../Main"
%%% End:


%%% Local Variables:
%%% mode: latex
%%% TeX-engine: xetex
%%% TeX-master: "../../Main"
%%% End:

%% -*- coding: utf-8 -*-
%% Time-stamp: <Chen Wang: 2019-12-18 17:03:15>


\section{前凉\tiny(301-376)}

\subsection{简介}

前凉(320年-376年)是十六国政权之一。都姑臧(今甘肃武威)。 301年,凉州大姓汉人张轨被晋朝封为凉州刺史,313年封西平公,課農桑、立學校,多所建樹。又鑄五銖錢,全境通行。314年张轨病死,其子张\xpinyin*{寔}袭位。西晋灭亡后,仍然据守凉州,使用司马邺(晉愍帝)的建興年號,成为割据政权。

320年,张茂改元永元,前凉遂彻底成为独立政权。

345年,张寔子张骏称凉王,都姑臧,以所在地凉州为国号“凉”,史称“前凉”,以别于其他以“凉”为国号的政权。張駿、張重華父子統治時期,前涼極盛。353年張重華病死,宗室內亂不止,國勢大衰。

前涼極盛之時,统治范围包括甘肃、宁夏西部以及新疆大部。史載“南逾河、湟,東至秦、隴,西包蔥嶺,北暨居延”。張天錫時已失去甘肅南部。

376年,前秦天王苻堅以十三萬步騎大舉進攻,張天錫投降,前涼滅亡。

\subsection{武王生平}

張軌(255年-314年),字士彥,安定郡烏氏縣(今甘肅平涼市西北)人。西漢常山王張耳的十七世孫。晉朝時任涼州牧,是前涼政權奠定者,張寔、張茂皆為其子。314年去世,晉諡曰武公。至其曾孫張祚時,被追諡為武王,廟號太祖。

張軌其家世孝廉,以儒學著稱。張軌年少時已聰明好學,甚有名望,曾隱居於宜陽郡的女几山上。西晉建立後入朝任官,因與中書監張華議論經籍意義和政事而深得對方的器重。張軌歷任太子舍人、尚書郎、太子洗馬、太子中庶子、散騎常侍,征西將軍司馬。

晉惠帝元康元年(291年),「八王之亂」開始,天下大亂,張軌於是想佔據河西之地(今甘肅西部、新疆東部一帶),於是就要求調任涼州。在朝中官員的支持之下,張軌於永寧元年(301年)被任命為護羌校尉、涼州刺史。張軌到任後,使立刻領兵擊敗當時在涼州叛亂的鮮卑族,又消滅橫行當地的盜賊,斬首萬多人,從此威震西土,亦安定了涼州。張軌任用有才幹的涼州大姓如宋配、陰充、氾瑗和陰澹為股肱謀主,共同治理涼州。他又勸農桑,立學校,又設與州別駕同等的崇文祭酒、春秋行鄉射之禮,在涼州大行教化。

永興二年(305年),鮮卑若羅拔能侵襲涼州,張軌派司馬宋配討伐,最終斬殺若羅拔能,並俘據十多萬人,因而聲名大振。晉惠帝亦因此加張軌安西將軍,封安樂鄉侯,邑千戶。同時又大修涼州治所姑臧(今甘肅武威市)。此時,東羌校尉韓稚殺害秦州刺史張輔,張軌少府司馬楊胤主張討伐韓稚,亦勸張軌效法齊桓公主持地方,對韓稚擅殺刺史的行為予以嚴懲。張軌於是命中督護領二萬兵討伐,並先寫信給韓稚勸降。韓稚拉到書信後就向張軌投降。張軌報告南陽王司馬模後,司馬模十分高興,並將皇帝賜的劍送給張軌,並將隴西地區交給張軌管理。

張軌始終對西晉表示忠誠,以維繫民心。如太安三年(304年)河間王司馬顒和成都王司馬穎到洛陽討伐掌權的司馬乂,張軌亦曾派三千兵支援朝廷。永嘉二年(308年),劉淵部將王彌進攻洛陽,張軌派北宮純、張纂、馬魴和陰濬等領兵入衛洛陽,北宮純及後派百多名勇士突擊王彌軍,協助朝廷擊退王彌。不久北宮純在河東擊敗劉淵兒子劉聰,晉懷帝於是詔封張軌為西平郡公,但張辭讓。西晉自八王之亂起,天下大亂,各州都不再向西晉朝廷賦貢,亦惟有張軌貢獻不絕。

永嘉二年(308年),張軌因患風搐而不能說話,命兒子張茂代管涼州。張越是涼州大族,聽說有預言說張氏會興盛涼州,自以為自己就是預言中的張氏,於是不惜放下梁州刺史的職務告病回涼州,更與兄長酒泉太守張鎮等人合謀要除去張軌。張越兄弟意圖以秦州刺史賈龕取代張軌,於是派密使到洛陽請尚書侍郎曹祛任西平太守,作為援助。張軌別駕麴晁亦意圖借機弄權,派使者到長安告訴司馬模,說張軌已病得不再能繼續行使刺史職權,要求以賈龕代替張軌。賈龕原打算應命,但被兄長勸止。

張鎮和曹祛知道賈龕拒絕應命後,再上表請求新派刺史,但未上呈就已率先以軍司杜耽代領州事,讓杜耽支持並表張越為新任刺史。張軌見此,打算退避,想要回到曾經隱居的宜陽,但長史王融和參軍孟暢接到張鎮等人以杜耽代理涼州的檄命後並不服氣,決意支持張軌,於是領兵戒嚴,又命剛從洛陽回來的張軌長子張寔為中督護,領兵討伐張鎮。同時又派張鎮甥子令狐亞遊說張鎮。最終張鎮聽從,哭著說受了誤導,將事情都推給功曹魯連,更將魯連殺死向張寔請罪。張寔及後攻打曹祛,曹祛逃走。在王融舉兵同時,武威太守張琠亦派兒子張坦到洛陽上表支持張軌;而治中楊澹亦到長安向司馬模控訴張軌被誣,令司馬模上表停止選調新任刺史。張坦到洛陽後,晉懷帝慰勞張軌,又下令誅殺曹祛。張軌知道後十分高興,又命張寔領兵三萬討伐曹祛,最終將曹祛擊敗並殺死。

張軌及後命治中張閬送五千義兵和大量物資到洛陽。永嘉五年(311年)光祿大夫傅祗和太常摯虞及後寫信給張軌說洛陽物資缺乏,張軌又立刻派參軍杜勵進獻五百匹馬和氈布三萬匹。晉懷帝於是進拜張軌為鎮西將軍、都督隴右諸軍事,封霸城侯,並進車騎將軍、開府儀同三司。但使者還未到,王彌就再次進逼洛陽,張軌派將軍張斐、北宮純和郭敷等率五千名精銳騎兵保衛洛陽,但洛陽最終都被漢國大將劉曜攻克。

永嘉之亂後,洛陽和長安兩大重鎮都先後被漢國軍隊攻陷,中原和關中地區的很多百姓流入涼州避難,張軌在姑臧西北置武興郡;又分西平郡(今青海西寧市)界置晉興郡以收容流民。同時,張軌亦繼續支持西晉,晉懷帝被擄到平陽後,張軌曾打算傾一州之力進攻平陽。不久秦王司馬鄴入關,張軌又派兵支持。次年司馬鄴被擁立為皇太子,張軌獲拜驃騎大將軍、儀同三司,張軌辭讓。同時張軌又協助消滅在附近地區叛亂的勢力,如秦州刺史裴苞、西平郡的麴恪、鞠儒等。司馬鄴及後再度任命,但張軌亦再次辭讓。

永嘉七年(313年),晉懷帝被殺,司馬鄴繼位為晉愍帝,並升張軌為司空,張軌再辭讓。同時又聽從索輔的建議,復鑄五銖錢,恢復境內的錢幣流通,大大便利了當地人的生活,不必再以布匹作貨幣。同時,劉曜進逼長安,張軌又派參軍麴陶領三千兵入衛長安。

建興二年(314年),晉愍帝任命張軌為侍中、太尉、涼州牧,封西平公,但張軌仍然辭讓。五月己丑日,張軌病死,享年六十歲,諡曰武公。張軌的親信部下及後擁立張軌長子張寔繼任了涼州牧之職。

张轨墓在今凉州区境内,史称“建陵”,前凉国主的陵墓位置,学术界有三种推测:陵墓上方筑台,可能在今灵钧台、雷台等古台下;或依照汉制,重臣死后多陪葬君主墓旁,根据已出土的“梁舒墓”的方位,可能在今武威城西北太平滩一带;或因山为陵,可能在武威城南祁连山山坡地带。

\subsection{昭王生平}

张寔(271年-320年),字安遜,安定烏氏人。十六国时期前凉政权的君主。为张轨长子。張寔任內保持與晉廷關係,也支持晉元帝即位為帝,但仍一直在割據狀態,即使西晉亡後仍然用晉愍帝「建興」年號。

張寔高學識,觀察入微,而且敬重並愛惜有才德的人,獲舉為秀才,授命為郎中。永嘉初年,張寔辭讓驍騎將軍,並請求回當時由父親任刺史的涼州。朝廷准許張寔所求,遂改授張寔議郎。不過,張寔回到涼州治所姑臧(今甘肅武威市)時正遇上涼州大姓張鎮、張越兄弟與曹祛等人圖謀逼走父親,奪取涼州控制權的行動。張軌長史王融及參軍孟暢支持張軌,決意作出反擊,於是讓張寔為中督護,率兵討伐張鎮。張鎮恐懼並委罪予功曹魯連,將其處死後便向張寔請罪。張寔隨後攻伐曹祛,曹祛逃走。時前往洛陽為張軌陳情的張坦帶著晉懷帝慰問張軌和誅殺曹祛的詔命回來,張軌於是命張寔率尹員、宋配等領三萬步騎兵攻曹祛。曹祛派麴晃到黃阪抵禦,張寔就用計騙了麴晃,令自己得以進至浩亹(今青海海東市樂都區東),並戰於破羌(今青海海東市樂都區西)。曹祛等終為張軌所殺。戰後張寔獲封建武亭侯。

不久,張寔遷西中郎將,封福祿縣侯。晉愍帝即位後,又以西中郎將領護羌校尉。建興二年(314年),張軌去世,長史張璽等人表張寔代行張軌官位。晉愍帝及後下詔授予張寔持節都督涼州諸軍事、西中郎將、涼州刺史、領護羌校尉,封西平公。張寔接掌涼州後鼓勵諫言,當面進諫的賞布帛;書面進諫的賞竹器;在坊間論政的賞羊和米。另又聽從賊曹佐隗瑾所言設立諫官,處理大小事務時都與部屬們討論,廣納眾言,從而鼓勵吏民進言。

建興四年(316年),前趙將領劉曜率軍逼近長安(今陝西西安),張寔派王該率兵救援,晉愍帝於是加授張寔都督陝西諸軍事。同年晉愍帝被圍困被逼投降,降前下詔進張寔為大都督、涼州牧、侍中、司空,承制行事。張寔受詔後以愍帝被俘為由辭讓。及後晉愍帝遇害的消息傳至涼州,南陽王司馬保卻圖謀稱帝,張寔則支持在江東的司馬睿,並在建興六年(318年)派牙門蔡忠上表勸進。同年,司馬睿即位為帝,即晉元帝。

建興七年(319年),司馬保自稱晉王,置百官並改年號,又以張寔為征西大將軍、儀同三司,增食邑三千戶。但不久司馬保就因部將陳安叛變而陷險境,張寔先後派兵協助。次年(320年),司馬保因劉曜逼近而遷至桑城(今甘中肅臨洮县附近),並意圖到涼州避難,但張寔考慮到司馬保宗室的身份,若果到涼州肯定會對當地人心有所影響,於是派將領陰監派兵迎接司馬保,聲稱是保衞,其實是想阻止他前來。不久,司馬保被其部將張春所殺,餘眾離散並有萬多人逃至涼州,張寔至此自恃涼州險遠,頗為驕傲放縱。

當時天梯山上有一個叫劉弘的人,因為法術而有上千信眾,連張寔身邊的人都有其信眾。當時帳下閻涉及牙門趙卬都是劉弘同鄉,而劉弘向閻涉說:「上天賜我神璽,要我治理涼州。」二人深信不疑,於是秘密聯結張寔身邊十多人,意圖行刺張寔,奉劉弘為主。張寔已經從張茂口中得知這圖謀,於是派了牙門將史初收捕劉弘,但閻涉等人不知道,依計眾人懷刀而入,在外寢殺死張寔。劉弘見史初來,還說:「使君已經死了,還殺我做甚麼!」史初憤怒,割了他的舌頭然後囚禁,及後張茂更施車裂之刑,閻涉及其黨羽數百人亦被誅殺。享年五十歲。私諡為昭公,晉元帝則赐諡號元公。張祚稱帝時以張寔為昭王。

張寔死後,因兒子張駿年幼,由弟弟張茂繼位。墓葬称“宁陵”,亦在今凉州区境内。


\subsection{成王生平}

张茂(277年-324年),字成遜,安定烏氏人。中國十六国时期前凉政权的君主,为昭王张寔之同母弟。張茂任內前涼遭前趙出兵威壓,被逼向前趙稱藩,並接受其官爵。

永嘉二年(308年)張軌患病不能說話,時兄長張寔仍在朝中,故張茂就代父管理涼州事務。建興元年(312年),南陽王司馬保曾經請張茂作自己的從事中郎,後又推薦他任散騎侍郎、中壘將軍,但張茂都不應命。次年,張茂被徵召為侍中,但張茂以父親年老為由推辭。不久,改拜平西將軍、秦州刺史。

建興八年(320年),張寔被部下所殺,因其子駿年幼,张茂就代攝其位,殺劉弘數百名同黨。時涼州府推舉張茂為大都督、太尉、涼州牧,但張茂不肯受,只以使持節平西將軍、涼州牧職位,又以張寔子張駿為撫軍將軍、武威太守、西平公。。

建興十年(322年),張茂派韓璞率兵佔領隴西南安郡境,並在當地置秦州。建興十一年(323年),前趙劉曜派部將劉咸攻冀城,呼延晏攻桑壁,時臨洮人翟楷及石琮等又驅逐其地方官員響應劉曜,劉曜本人更发兵二十八万五千人沿黃河列陣百多里,張茂設在黃河沿岸守戍的軍隊望風奔逃,劉曜更聲言面率大軍渡河,直攻姑臧,遂震動河西。張茂聽從參軍馬岌所言出屯姑臧城東的石頭,在聽參軍陳珍分析劉曜其實不會盡力攻涼後,便派陳珍出兵救援在冀城的韓璞。劉曜亦自知其強大的兵力有三分之二是因為人們怯於其聲威而來,主力軍隊已經相當疲累,難以渡河進攻,於是一直按兵不動,想用聲威脅服張茂。張茂最終派使者向前趙稱藩,進獻大量物品,劉則授張茂使持節、假黃鉞、侍中、都督涼南北秦梁益巴漢隴右西域雜夷匈奴諸軍事、太師、領大司馬、涼州牧、領西域大都護、護氐羌校尉。封涼王。

建興十二年(324年)張茂病死,享年四十八歲。前涼私下為張茂上諡號成公,刘曜則遣使赠太宰,諡成烈王。後來张祚稱帝,追尊張茂為成王,庙号太宗。张茂临终时交代张骏“谨守人臣之节,无或失坠”。因張茂無子,張駿就被推舉繼位。

張茂為人謙虛恭敬,且又好學,不因為利祿而動心,又有志向及節操,有決定大事的能力。當時有一個人叫賈摹,不但出身涼州大姓,他也是張寔妻子的弟弟,勢力很大。曾經有一首童謠這樣說:「手莫頭,圖涼州。」張茂以此誘殺賈摹,終令涼州豪門大族不敢橫行,更有助前涼張氏對涼州的管治。


\subsection{文王生平}

张骏(307年-346年),字公庭,安定烏氏人。中國十六国时期前凉政权的君主,在位二十二年。为前凉明王张寔之子,前凉成王张茂之侄。張駿任內前涼國力提升,也乘前趙滅亡而盡得河南(甘肃地区黄河以南)隴西之地,又進攻西域。張駿亦先後接受後趙及東晉的官位,在位晚期亦建設起天子規格器物、儀式及官職架構。

建興四年(316年),張駿受封霸城侯。建興八年(320年),張寔去世,涼州州府推舉其叔張茂繼位,張茂於是以張駿為撫軍將軍、武威太守,襲爵西平公。建興十二年(324年),張駿在張茂死後繼位,並暗示時滯留在姑臧的晉愍帝使者史淑以晉廷名義授予自己使持節大都督、大將軍、涼州牧、領護羌校尉,封西平公。時前趙皇帝劉曜也授張駿大將軍、涼州牧,封涼王。

張駿繼位時,守枹罕的涼州將領辛晏據城反對張駿,不服其統治。張駿打算討伐但為從事劉慶勸止,而翌年辛晏也向張駿投降,收服了河南之地。建興十五年(327年),張駿聽聞前趙軍隊敗於後趙,於是除去前趙所授的官爵,用回晉朝的官爵,並派兵進攻前趙秦州。可時前涼軍敗於前趙南陽王劉胤所率的軍隊,劉胤更乘勝渡過黃河,攻陷令居並殺二萬多人,又進佔振武,震動河西,張駿派皇甫該前往防禦。金城太守張閬及枹罕護軍辛晏都向前趙投降,河南之地復失。建興十七年(329年),前趙亡於後趙,張駿於次年就乘機重奪河南地,進軍至狄道,置武街、石門、侯和、漒川及甘松五屯護軍。不久,後趙派了鴻臚孟毅授予張駿征西大將軍、涼州牧,但張駿恥於為後趙之臣,不接受並留下孟毅。但不久就因畏懼後趙強大而向其稱臣,送還孟毅。東晉朝廷也進張駿鎮西大將軍,仍授涼州刺史、領護羌校尉並封西平公,詔命於建興二十一年(333年)到達前涼,張駿接受任命,派王豐等人陳謝並上疏稱臣,但仍然用建興年號,不用東晉年號。次年東晉又進張駿為大將軍,此後晉涼每年都有使者往來。

張駿又曾派將領楊宣率兵進攻龜茲及鄯善,終令西域諸國都歸附前涼,焉耆前部王及于寘王都派人進貢。也曾上疏晉廷請求配合司空郗鑒及征西將軍庾亮進行北伐。

西域長史李柏敗於不肯服從張駿的戊己校尉趙貞,有人認為是李柏自設計謀導致失敗,請張駿誅殺他,然而張駿終免李柏一死,更得眾人歡心。張駿也改易原本犯下死罪者的親屬不得留在朝中的律令,只是限制他們不能參與宿衞,於是令前涼刑法清明,國家富強,群僚更於建興二十年(332年)勸張駿稱「涼王」,自領秦涼二州牧,置公卿百官。雖然張駿嚴詞拒絕,但其實前涼境內都用涼王去稱呼張駿。後張駿更努力改變自己,勤於庶政,統掌涼州文武事務,治績不錯,得四方稱頌,叫他做「積賢君」。而涼州自晉末以來連年都有戰事,至張駿在位時漸見平穩安定。建興二十七年(339年),張駿又設辟雍、明堂以行禮教。

張駿攻西域後,在涼州西界劃出設沙州,又將涼州東界劃出設河州,時屬官們都稱臣。張駿亦在姑臧附近增築新城,又修建用金玉和五色畫裝飾的謙光殿,極盡珍貴精巧,其四面都各建一殿,四季各居一殿。他又自稱大都督、大將軍、假涼王,督攝涼沙河三州,設六佾之舞,設天子的豹尾車,所設祭酒、郎中、大夫、舍人、謁者等官職官號都模仿晉朝體系,只是稍稍改了名字。

建興三十四年(346年),張駿去世,享年四十歲,前涼私諡為文公,晉廷則賜諡號忠成,贈大司馬,歸葬大陵。其子張祚稱帝時,追尊為文王,廟號世祖。

后凉年间,有名叫安据的即序胡人盗张骏墓,见张骏貌如生前,并盗得真珠簏、琉璃榼、白玉樽、赤玉箫、紫玉笛、珊瑚鞭、马脑钟、水陆奇珍不可胜数。后凉皇帝吕纂诛安据党徒五十余家,遣使吊祭张骏,并缮修其墓。

張駿年輕就已顯得奇特雄偉,十歲時就能寫文章,為人卓越不羈,但曾經縱情淫慾,常夜遊城邑里巷。《魏書·張寔傳》記載張駿為人貪婪,為求進圖秦隴而給予人民穀物布帛,一年後收一倍稅收,不夠的都要用田地屋宅抵償。然《晉書·張軌傳》則載是譚詳建議將倉庫的穀贈予百姓,然後在秋季收三倍稅收,為陰據所諫而放棄實行。《魏書》又寫其因畏懼大姓陰氏勢力大而逼陰澹弟陰鑒自殺,大失人心。此事《晉書》亦無載。


\subsection{桓王生平}

张重华(327年-353年),字泰臨,是十六国时期前凉政权的君主。为前凉文王张骏次子,353年病死。張重華統治時期,前涼國勢達於極盛,多次擊敗後趙石虎的進攻,後更乘後趙末年國亂而進取秦州。在位七年病死,年僅二十七歲。

建興二十年(332年),群僚請張駿立世子,張駿最初不肯,但在中堅將軍宋輯的勸說下,張駿還是立了張重華為世子。建興三十三年(345年),張駿從涼州分劃出沙州及河州,以武威、武興、西平、張掖、酒泉、建康、西郡、湟河、晉興、須武及安故十一郡仍為涼州,由張重華任五官中郎將、涼州刺史。

建興三十四年(346年),張駿去世,涼州官屬推張重華為使持節大都督、太尉、護羌校尉、涼州牧,襲爵西平公,假涼王。張重華即位後減輕賦斂,免除關稅,減省園囿,以撫恤貧窮者。同年後趙派麻秋、王擢等侵涼,金城太守張沖降趙,涼州震動。張重華任用謝艾抵抗,終大破趙軍,殺五千人。翌年,後趙再派石寧領二萬兵作為麻秋後援,前涼將領宋秦更加率二萬戶人投降後趙。張重華再度起用謝艾,命其率三萬步騎進軍臨河,又破趙軍,斬殺趙將杜勳、汲魚,一萬三千人被俘或陣亡。不久,石寧聯合麻秋等率十二萬兵進屯河南,再度進攻,張重華想要親自出擊,但為謝艾及索遐所勸止,遂派二人率兵二萬抵抗。時後趙將孫伏都、劉渾率步騎二萬增援麻秋,眾人渡過黃河並屯於長最。謝艾等進軍至神鳥,擊敗王擢前鋒,令其退回黃河以南,接著就進攻長最,再敗趙軍,麻秋等退還金城。石虎聞麻秋戰敗,也嘆息道:「我以偏師就平定了九個州,現在用九個州的力量卻在枹罕寸步難行,真是對方有能人,還不可以謀取呀。」不過,麻秋隨後擊敗了張瑁,枹罕護軍李逵降趙,於是河南地區羌、氐族人都附趙。

建興三十五年(347年),東晉派侍御史俞歸到涼州,授予張重華假節、侍中、大都督、督隴右關中諸軍事、護羌校尉、大將軍、涼州刺史,封西平公。俞歸到涼時,時張重華想稱涼王,故未受詔,更命親信沈猛向俞歸表示,但為俞歸拒絕,並言:「今天你的主公剛剛繼位就要稱王,若果率領河右部眾平定東方的胡、羯,修復晉朝帝陵及宗廟,迎天子還都洛陽,還有甚麼可以嘉獎呀?」張重華於是不圖稱王。

建興四十年(352年),因後趙國亂,苻健乘時於關中建立前秦,時任後趙西中郎將的王擢向東晉請降,獲授征西將軍、秦州刺史,但同年就被前秦將領苻雄擊破,於是出奔涼州,向前涼歸降。張重華厚待他,任命他為征虜將軍、秦州刺史。張重華更派了將軍張弘及宋修率一萬五千兵與王擢會合,讓他進攻前秦。次年(353年)兩軍交戰,王擢大敗逃奔姑臧,張弘及宋修都戰死。張重華素服為陣亡將士舉哀,也安慰其家人,更加再命王擢進攻前秦秦州,最終取勝,奪取秦州。張重華因而上疏東晉請求伐秦,東晉則進張重華涼州牧。

建興四十一年(353年),張重華因病去世,享年二十七,葬顯陵。私諡為昭公,後改桓公,東晉則賜諡號敬烈公。重華病重時曾下手令徵召謝艾為衞將軍、監中外諸軍事以輔政,但最終為重華兄張祚等人壓下,終由張祚輔政,不久更廢掉張重華的世子張曜靈,自己登位。張祚稱帝,追諡張重華為桓王,上廟號世宗。

張重華寬厚平和,深沉穩重,又寡言。不過在擊退後趙連番進攻後表現怠惰,疏於政事且很少親身接見賓客,司直索遐曾經進言勸諫,張重華雖然大感高興,但沒有改變。他又喜和身邊小人玩樂,更多次向左右近臣賞賜金錢。


\subsection{哀公生平}

張曜靈(344年-355年),字元舒,是十六国時期前凉政權的君主,前凉桓王張重華子。張曜靈即位不久就被伯父張祚奪位,及後更被殺害。

建興四十一年(353年),張重華患病,遂立張曜靈為世子。同年張重華去世,實歲仅九岁的張曜靈继位,稱大司馬、護羌校尉、涼州刺史、西平公。張重華原本想以謝艾輔政,但遭其兄張祚與寵臣趙長、尉緝等勾結而壓下張重華的命令,於是張曜靈繼位後,張祚就矯令擔任輔政工作。不久,趙長等以張曜靈太過年輕,建議立年長君主,其祖母马氏與張祚私通,遂废曜靈爲涼寧侯,由張祚繼位。

和平二年八月(355年),张瓘等大臣试图废黜张祚、迎張曜靈复位,未成,張曜靈被張祚派遣杨秋胡暗杀,匿尸沙坑。同年张祚被杀,私谥為哀公。

%% -*- coding: utf-8 -*-
%% Time-stamp: <Chen Wang: 2021-11-01 11:53:31>

\subsection{威王张祚\tiny(353-355)}

\subsubsection{生平}

涼威王张祚(327年前-355年),字太伯,安定烏氏人。十六国时期前凉皇帝,前凉文王张骏庶长子,前凉桓王张重华异母兄。張祚與張重華寵臣勾結,又與太后通姦,得以在張曜靈繼位不久即廢其自立,更曾經稱帝。然而在稱帝翌年就被政變推翻及被殺。

張祚受封長寧侯,他博學且強壯勇武,又有政治才能,可是為人狡詐善於奉承,與張重華寵臣趙長、尉緝等人勾結並結為異姓兄弟。建興四十一年(353年)張重華病重時,曾下手令召酒泉太守謝艾入朝輔政,但為趙長等壓下。同年張重華死,由其年幼的长子張曜靈繼位,趙長等就假稱張重華遺令,以張祚為持節、都督中外諸軍事、撫軍將軍身份輔政。時趙長等以張曜靈年幼,稱國家需要年長君主,张祚因与张重华之母馬太后通奸,遂煽动马太后废黜了張曜靈,立张祚為主。張祚於是自稱大都督、大將軍、涼州牧、涼公。張祚位後即淫亂張重華的妻妾及其未嫁女兒。

和平元年(354年),张祚称帝,改元「和平」,設宗廟、八佾舞,並置百官,尚書馬岌因切諫被免官,郎中丁琪進諫更被殺,又殺謝艾。張祚又曾進攻驪靬,但大敗而還。同年東晉桓溫北伐,也有配合北伐的秦州刺史王擢派人報告張祚稱桓溫善於用兵,軍勢難測。張祚聞訊恐懼,但還擔憂王擢會倒過來進攻自己,於是派人暗殺他,但因被王擢發現而不成。張祚在暗殺失敗後更加恐懼,於是出兵聲稱要東征,實則是想西退至敦煌自保,只是遇上桓溫退兵才取消行動。不過,張祚仍繼續打擊王擢,派了牛霸率三千兵打敗王擢,逼使王擢投降前秦。

张祚治国不道,曾置五都尉去專抓別人過失,又限定四品以下官員不得送贈衣布,庶人不能畜養奴婢及乘坐車馬。张祚为人荒淫暴虐,国人无不侧目,都作諷刺其淫亂的詩。和平二年(355年),張祚因不欲河州刺史張瓘強大,於是命令他去討伐叛胡,其實已派易揣及張玲率三千兵襲擊張瓘。王鸞識術數,向張祚說:「這支軍隊出去,肯定不會回來,涼國會陷於危險。」更上陳張祚三不道。張祚聞言大怒,認定王鸞所說是妖言,將他處斬。王鸞臨死前就說:「我死後,軍隊在外面戰敗,大王在內死亡,肯定會發生的!」張祚更誅殺王鸞一族人。不過,張瓘就殺了張祚派去代其守枹罕的索孚,易揣等渡過黃河途中就被張瓘攻擊,張瓘更出兵跟隨單騎逃還的易揣,兵向姑臧。張瓘軍前來的消息震動姑臧人心,時宋混、宋澄兄弟因其兄宋修與張祚有前嫌,就出城聚眾響應張瓘,並反攻姑臧。時張瓘傳檄州郡,要復立張曜靈,故張祚就派楊秋胡殺害張曜靈;另又收捕並處死張瓘的兩個弟弟張琚及張嵩。二人知要被捕時卻在市招募數百,大叫張瓘大軍已經到達姑臧城東,恐嚇敢動手的人要被誅三族。收捕的人果被嚇退,然後二人西城門迎宋混等入城。趙長等人懼怕因擁立張祚獲罪,於是請馬太后出殿,改立張玄靚為主,不過易揣等人卻引兵入殿,收殺趙長等人。宋混等入城後,張祚按劍命令部眾死戰,但因為他失眾心,將士根本毫無鬥志,張祚於是為宋混等殺死,頭被斬下宣示內外,更遭曝屍在大道左邊,城內人民都大呼萬歲。

事後張祚以庶人的禮儀下葬,直至其弟張天錫即位時,才改葬到愍陵,追諡為威王。

\subsubsection{和平}

\begin{longtable}{|>{\centering\scriptsize}m{2em}|>{\centering\scriptsize}m{1.3em}|>{\centering}m{8.8em}|}
  % \caption{秦王政}\
  \toprule
  \SimHei \normalsize 年数 & \SimHei \scriptsize 公元 & \SimHei 大事件 \tabularnewline
  % \midrule
  \endfirsthead
  \toprule
  \SimHei \normalsize 年数 & \SimHei \scriptsize 公元 & \SimHei 大事件 \tabularnewline
  \midrule
  \endhead
  \midrule
  元年 & 354 & \tabularnewline\hline
  二年 & 355 & \tabularnewline
  \bottomrule
\end{longtable}


%%% Local Variables:
%%% mode: latex
%%% TeX-engine: xetex
%%% TeX-master: "../../Main"
%%% End:

%% -*- coding: utf-8 -*-
%% Time-stamp: <Chen Wang: 2019-12-18 17:31:16>

\subsection{冲王\tiny(355-363)}

\subsubsection{生平}

涼沖王張玄靚(350年-363年),字元安,十六國時期前涼國君主,為張重華之子,張曜靈之弟。張玄靚年幼繼位,前涼國政先後在張瓘、宋混、宋澄、張邕及張天錫手中掌握,期間政變頻仍,張玄靚最終亦因張天錫政變而被殺。

張玄靚於和平元年(354年)獲張祚封為涼武侯。和平二年(355年),張祚被殺,張玄靚被宋混、張琚推為大將軍、涼州牧、護羌校尉、西平公,恢復年號為建興四十三年。不久,河州刺史張瓘返都城姑臧(今甘肅武威),張玄靚再被推為涼王,政事決於張瓘。次年(356年)前秦派使者閻負、梁殊前來,要勸說前涼臣服於前秦,張瓘恐懼,於是勸導張玄靚向前秦稱藩,而前秦亦以張玄靚所稱的官爵授命。

張玄靚繼位後,前涼國內先後有李儼、衞綝和馬基等人反叛,張瓘擊敗了衞綝並討平馬基。其時張瓘、張琚兄弟賞罰都依從自己愛惡,無視綱紀,又不聽諫言,故並不得人心。可是他們自以勢力強大,且有功勳,所以有篡位的意圖,然而就忌憚忠心剛直的宋混。建興四十七年(359年),張瓘徵集了數萬兵並會聚於姑臧,想要消滅宋混兄弟,宋混及宋澄知道後就率領壯士楊和等四十多騎到南城,並向各個兵營宣稱張瓘謀反,太后下令誅除他,很快就召集到二千多人。隨後宋混率眾與張瓘決戰,張瓘兵敗,其部眾都背棄張瓘,向宋混投降,張瓘兄弟於是自殺。事後宋混代替張瓘掌政,張玄靚為宋混所建議去涼王稱號,改稱涼州牧。建興四十九年(361年),宋混去世,張玄靚順從宋混遺言而讓宋澄掌政,不過右司馬張邕不滿宋澄專政,同年即起兵攻滅宋澄,並誅殺宋氏一族。張玄靚隨後又改讓張邕與叔父張天錫共同掌政。可是,張邕自恃功勳大而行事驕縱,濫用刑法,更與馬太后私通,樹立黨羽,很不得人心,張天錫就是再次發動政變,張邕兵敗自殺,其黨眾皆被張天錫誅殺。張玄靚遂以張天錫一人掌政。十二月,張天錫讓張玄靚改奉當時東晉的升平年號,稱升平五年。晉廷則授張玄靚大都督隴右諸軍事、護羌校尉、涼州刺史,西平公。

升平七年(363年),馬太后去世,張玄靚以其母郭夫人為太妃,而郭夫人因不滿張天錫專政而與張欽圖謀發動政變,可是圖謀外泄,張欽等人都被張天錫殺害。張天錫隨後便發動政變,派兵入宮殺死張玄靚,向外宣稱張玄靚暴斃,享年十四歲。

張玄靚被下葬平陵。張天錫私諡為沖公。東晉孝武帝司馬曜賜諡號敬悼。

\subsubsection{建兴}

\begin{longtable}{|>{\centering\scriptsize}m{2em}|>{\centering\scriptsize}m{1.3em}|>{\centering}m{8.8em}|}
  % \caption{秦王政}\
  \toprule
  \SimHei \normalsize 年数 & \SimHei \scriptsize 公元 & \SimHei 大事件 \tabularnewline
  % \midrule
  \endfirsthead
  \toprule
  \SimHei \normalsize 年数 & \SimHei \scriptsize 公元 & \SimHei 大事件 \tabularnewline
  \midrule
  \endhead
  \midrule
  四三年 & 355 & \tabularnewline\hline
  四四年 & 356 & \tabularnewline\hline
  四五年 & 357 & \tabularnewline\hline
  四六年 & 358 & \tabularnewline\hline
  四七年 & 359 & \tabularnewline\hline
  四八年 & 360 & \tabularnewline\hline
  四九年 & 361 & \tabularnewline
  \bottomrule
\end{longtable}

\subsubsection{升平}

\begin{longtable}{|>{\centering\scriptsize}m{2em}|>{\centering\scriptsize}m{1.3em}|>{\centering}m{8.8em}|}
  % \caption{秦王政}\
  \toprule
  \SimHei \normalsize 年数 & \SimHei \scriptsize 公元 & \SimHei 大事件 \tabularnewline
  % \midrule
  \endfirsthead
  \toprule
  \SimHei \normalsize 年数 & \SimHei \scriptsize 公元 & \SimHei 大事件 \tabularnewline
  \midrule
  \endhead
  \midrule
  五年 & 361 & \tabularnewline\hline
  六年 & 362 & \tabularnewline\hline
  七年 & 363 & \tabularnewline
  \bottomrule
\end{longtable}


%%% Local Variables:
%%% mode: latex
%%% TeX-engine: xetex
%%% TeX-master: "../../Main"
%%% End:

%% -*- coding: utf-8 -*-
%% Time-stamp: <Chen Wang: 2021-11-01 11:53:51>

\subsection{悼公張天錫\tiny(363-376)}

\subsubsection{生平}

張天錫(346年-406年),字純嘏,本字公純嘏,因被人嘲笑是三字而自行改字,小名獨活,安定烏氏人。中國十六國時期前涼政權的最後一位君主。張天錫為前涼文王張駿少子,前涼桓王張重華之弟。張天錫在位時前秦國力強盛,雖曾主動斷絕與前秦關係,但最終仍逼於軍事力量而再度稱藩。及後張天錫反抗前秦徵召入朝的命令並射殺使者,前秦大軍遂攻伐前涼,張天錫不敵投降,前涼國於是滅亡。淝水之戰後張天錫南歸東晉,並在東晉終老。

和平元年(354年),張祚封張天錫為長寧王。建興四十九年(361年),張邕殺死當政的宋澄,當時的前涼君主張玄靚就以張天錫為中領軍,與張邕共輔朝政。不過,張邕因樹立黨羽專權,經常濫用刑法殺人,很不得人心,張天錫親信劉肅則與其共謀除去他。十一月,張天錫與張邕一同入朝,劉肅就與趙白駒跟著張天錫行動,二人先後襲擊張邕但都失敗,於是與張天錫一同走入宮中。逃走的張邕率三百軍人進攻宮門,張天錫登門樓指責張邕凶惡悖逆,聲言自己是在冒死保衞國家社稷,並只會針對張邕而已。張邕兵眾聞言都逃散,張邕自殺,張天錫又誅殺了張邕黨羽,專掌朝政。

升平七年(363年),郭太妃以張天錫專政,與張欽密謀誅殺張天錫,事洩,欽等皆死;右將軍劉肅於是勸張天錫自立,天錫遂會劉肅夜襲皇宮,殺張玄靚。張天錫自稱使持節、大都督、大將軍、護羌校尉、涼州牧、西平公,並派使者出使建康請命,東晉於是在366年授張天錫為大將軍、大都督、督隴右關中諸軍事、護羌校尉、涼州刺史,封西平公。前秦亦派大鴻臚授張天錫大將軍、涼州牧、西平公。

張天錫登位後多次在園池設宴,又沉迷於歌舞和女色,荒廢政事。張天錫更將兩個親信劉肅及梁景收為養子,讓二人參與朝政,令人們有怨言和恐懼,索商及天錫堂弟張憲曾經勸諫他但不獲授納。張天錫於升平十年(366年)與前秦斷交,並在進攻李儼時與前秦發生軍事衝突,並俘獲了陰據和他率領的五千兵。升平十五年(371年),前秦攻滅仇池,送還陰據及其士兵回國,並派梁殊及閻負隨行,順道送達前秦丞相王猛的書信,暗示要張天錫別和前秦作對。張天錫看後十分恐懼,於是派使者向前秦謝罪,向前秦稱藩,前秦天王苻堅任命其為使持節、散騎常侍、都督河右諸軍事、驃騎大將軍、開府儀同三司,涼州刺史、西域都護、西平公。然而因張天錫因為懼怕前秦吞併,於同年在姑臧設壇,遙與晉三公盟誓,又派從事中郎韓博出使東晉,並寫信給東晉大司馬桓溫,約定大舉出兵北伐,會師上邽。

升平二十年(376年),苻堅徵召張天錫入朝任武衞將軍,同時派了苟萇、毛盛、梁熙及姚萇等率十三萬步騎至西河郡,預備一旦張天錫拒絕應命就進攻前涼。張天錫接到梁殊、閻負送來的詔命後問及眾僚意見,除席仂建議送貨款和質子,徐圖後計外,大部份人都認為涼州有精兵及天險,可以取勝。張天錫於是決定反抗,派馬建率兵二萬抵抗秦軍,並命人射殺兩名前秦使節。面對秦軍進攻,馬建懼而退守清塞,張天錫又派掌據率兵三萬與馬建屯於洪池,自率五萬屯金昌城。可是,苟萇隨後進攻掌據時馬建就投降前秦,掌據戰死,張天錫又派趙充哲為前鋒,率五萬兵與苟萇等作戰,但又在赤岸大敗,張天錫出城意圖再戰,但因金昌城中反叛而被逼逃回姑臧並請降。苟萇等到姑臧後受降,並送張天錫到長安,其他郡縣都降秦,前涼滅亡。苻堅在長安為張天錫建了府邸,任命他為侍中、北部尚書,封歸義侯。

晉太元八年(383年),晉軍於淝水之戰擊潰來攻的前秦軍,當時張天錫任征南大將軍苻融的司馬隨軍,趁機南奔東晉,東晉朝廷下詔以張天錫為散騎常侍左員外,復封為西平郡公。後轉拜金紫光祿大夫。後曾加授廬江太守,桓玄掌政時為了招撫四方而任命張天錫為護羌校尉、涼州刺史。義熙二年(406年),張天錫去世,享年六十一歲,追贈為鎮西將軍,諡號悼公。

張天錫因文才而聲名遠著,回歸晉廷後亦甚得晉孝武帝知遇,可是朝中官員卻以其曾經亡國被俘而中傷他。會稽王司馬道子曾經問及涼州有甚出產,張天錫立即就答道:「桑葚甘甜、鴟鴞會變聲音、乳酪養生、人沒有嫉妒之心。」不過,後來張天錫表現得昏亂喪志,雖然有公爵爵位也得不到別人禮遇。至晉安帝隆安年間,當政的會稽王世子司馬元顯更常常請他來戲弄他。擔任廬江太守亦因為其家貧而獲授。

\subsubsection{升平}

\begin{longtable}{|>{\centering\scriptsize}m{2em}|>{\centering\scriptsize}m{1.3em}|>{\centering}m{8.8em}|}
  % \caption{秦王政}\
  \toprule
  \SimHei \normalsize 年数 & \SimHei \scriptsize 公元 & \SimHei 大事件 \tabularnewline
  % \midrule
  \endfirsthead
  \toprule
  \SimHei \normalsize 年数 & \SimHei \scriptsize 公元 & \SimHei 大事件 \tabularnewline
  \midrule
  \endhead
  \midrule
  七年 & 363 & \tabularnewline\hline
  八年 & 364 & \tabularnewline\hline
  九年 & 365 & \tabularnewline\hline
  十年 & 366 & \tabularnewline\hline
  十一年 & 367 & \tabularnewline\hline
  十二年 & 368 & \tabularnewline\hline
  十三年 & 369 & \tabularnewline\hline
  十四年 & 370 & \tabularnewline\hline
  十五年 & 370 & \tabularnewline\hline
  十六年 & 372 & \tabularnewline\hline
  十七年 & 373 & \tabularnewline\hline
  十八年 & 374 & \tabularnewline\hline
  十九年 & 375 & \tabularnewline\hline
  二十年 & 376 & \tabularnewline
  \bottomrule
\end{longtable}


%%% Local Variables:
%%% mode: latex
%%% TeX-engine: xetex
%%% TeX-master: "../../Main"
%%% End:



%%% Local Variables:
%%% mode: latex
%%% TeX-engine: xetex
%%% TeX-master: "../../Main"
%%% End:

%% -*- coding: utf-8 -*-
%% Time-stamp: <Chen Wang: 2019-12-18 17:35:29>


\section{后赵\tiny(319-351)}

\subsection{简介}

后赵(319年-351年)是十六国时期羯族首领石勒建立的政权。

因石勒统治地区为战国时赵国故地,因此刘曜封其为赵王,立国即以此为号。为别于先建国的前赵,故史称“后赵”,又以其王室姓石,又称“石赵”。

在晋怀帝末年反晋浪潮中,石勒投附在并州刺史部的南匈奴贵族刘渊为部将,屡立战功,势力强盛。308年10月,刘渊正式称帝,建国号“汉”,(刘曜后改为赵),建都平阳(今山西临汾)年号为永凤。318年,国丈靳准杀死隐帝刘粲夺权,自立为汉天王。镇守长安的刘粲叔父刘曜得知平阳有变,自立为皇帝,派遣军队至平阳,族灭靳氏,迁都到长安。与此同时,石勒亦参与讨伐靳准,后来试图挑起城中变乱促其投降的计划失败,导致靳明掌权并倒向刘曜,石勒大怒,攻破平阳城。319年,刘曜在长安改国号“汉”为“赵”,史称前赵。同年,石勒在襄国(今河北邢台)自称大单于、赵王,与前赵决裂,史称后赵。329年石勒灭前赵,次年称帝。

石勒开拓疆土,灭前赵,占有除辽东、河西以外的北方地区。后赵前期仍采取胡汉分治政策,但注意笼络汉族士族,减轻租赋,发展农业生产,推行儒家教育,社会呈现丰裕景象。统治地区包括冀州、并州、豫州、兖州、青州、司州、雍州、秦州、徐州、凉州、荆州部分地区、幽州部分地区。

后赵建平四年(333年)石勒卒。次年其从子石虎篡位,335年迁都邺城(今河北临漳境内)。石虎非常残暴,征役无时,大兴土木,荒淫无度,社会矛盾十分尖锐。太宁元年(349年)后赵爆发梁犊领导的雍凉戍卒舉兵,一度攻克长安,有众40余万。同年石虎卒,其子为争帝位互相残杀。石虎养孙冉闵大杀石氏子孙及羯胡,次年(350年)自立为帝,改国号魏,史称冉魏。石虎子新兴王石祗在襄国称帝,与冉魏对抗。后石祗为得前燕相助,降称赵王。351年,石祗被手下刘显所杀,后赵亡。次年,其他幸存的石氏子孙投降东晋,也被杀及诛滅。

%% -*- coding: utf-8 -*-
%% Time-stamp: <Chen Wang: 2019-12-18 17:38:46>

\subsection{明帝\tiny(319-333)}

\subsubsection{生平}

趙明帝石勒(274年-333年8月17日),字世龍,原名㔨,小字匐勒,上黨武鄉(今山西榆社)羯族人,是五胡十六國時代後趙的開國君主。

石勒初期因公師藩而起兵,後投靠漢趙君主劉淵,之後卻與漢國決裂,由漢國分裂出去。石勒在他的謀臣,漢人張賓輔助之下以襄國(今河北邢台)為根據地,並陸續消滅了王浚、邵續、段匹磾等西晉於北方的勢力,繼而又消滅曹嶷,進侵東晉以及消滅劉曜領導的前趙,又北征代國,率領後趙成為當時北方最強盛的國家。石勒又實行多項措施,推動文教和經濟發展。另外他厚待來自西域的佛教僧侶佛圖澄,對當時佛教的傳播有一定貢獻。

石勒出身羯胡,為南匈奴羌渠人。其祖先為匈奴分支部落的贵族。石勒原沒有漢文姓名,其姓與名皆是由牧人汲桑所起。

羯人的起源不詳,可能起源自小月氏,而歷史學家陳寅恪認為可能起源於中亞康居。

石勒壯健有膽量和魄力,雄健威武,更喜愛騎射。父親周曷朱為部落小帥,因性格粗暴凶惡而不被一眾胡人心服,常命石勒代他領導部眾,卻得眾人信賴。當時相士和父老都稱石勒相貌奇特,氣度非常,前途無可限量,勸邑中人厚待他。但大部份人對這說法都嗤之以鼻,唯獨郭敬和甯驅相信,更加借資源給他,石勒亦感恩,盡心為他耕作。

太安二年(303年),并州發生大饑荒,石勒與一眾胡人逃散,於是去依靠甯驅。當時北澤都尉劉監打算將他賣掉,幸得甯驅協助才沒有成事。之後石勒暗中改投都尉李川,路上遇見郭敬,於是向他哭訴飢寒之苦。郭敬聽後傷心流涕,送他衣服和食物。當時石勒向郭敬建議誘一眾胡人到冀州吃糧,借故賣掉他們換取金錢,既可解諸胡饑困,亦能獲利。而同時建威將軍閻粹說服并州刺史司馬騰遷諸胡到太行山以東地區販賣,以獲得軍事資本,於是司馬騰就派人到冀州捕捉一眾胡人,連石勒都被抓著。當時負責捕捉胡人的張隆多次毆打石勒,而且路上常有人飢餓或病倒,石勒全靠郭敬親族郭陽和郭時的资助才成功到冀州。到冀州後石勒被賣給師懽為奴,師懽却因其儀表堂堂,氣质出眾,讓他做了自己的佃客。

當時師懽家在牧苑側,石勒於是與牧帥汲桑往來,更以自己有相馬的能力而自薦給汲桑。後結集王陽、夔安、支雄、冀保、吳豫、劉膺、桃豹、逯明、郭敖、劉徵、張曀僕、呼延莫、郭黑略、張越、孔豚、趙鹿、支屈六十八個壯士一同號稱為「十八騎」,並與他們搶掠園林,以財寶巴結汲桑。

永興二年(305年),成都王司馬穎被河間王司馬顒廢去官位和皇太弟身份,因司馬穎曾鎮鄴城,很多河北人都可憐司馬穎的遭遇。司馬穎舊將公師藩於是自稱將軍,以司馬穎之名在趙、魏之間舉兵,聚眾數萬,汲桑與石勒亦率數百騎師附公師藩。此時,汲桑才命石勒以石為姓,以勒為名。公師藩則拜石勒為前隊督,並與他進攻守鄴城的平昌公司馬模,卻被苟晞、丁紹和司馬模部將馮嵩擊敗。次年,公師藩在白馬縣打算南渡黃河,被苟晞擊殺。

公師藩死後,石勒與汲桑逃回茌平牧苑,石勒被汲桑命為伏夜牙門,率領牧人劫掠郡縣的囚犯,又招納潛居山間的亡命之徙。汲桑於是在永嘉元年(307年)自稱大將軍,聲稱要為上一年被殺的司馬穎報仇。汲桑以石勒為前驅,屢次取勝,於是署石勒為討虜將軍、忠明亭侯。石勒即隨汲桑進攻鄴城,擔任前鋒都督,大破馮嵩,並且長驅直進,於五月攻陷鄴城。汲桑在鄴城殺司馬騰和萬多個兵民,焚毀鄴城宮室和搶掠城中婦女珍寶後才離開。

石勒及後又跟汲桑進攻幽州刺史石尟。石勒在樂陵擊殺石尟後又擊敗率五萬兵營救石尟的乞活軍將領田禋,並與苟晞相持於平原、陽平之間數月,期間發生三十多場戰事,互有勝負,迫使太傅司馬越率兵在官渡為苟晞聲援。石勒和汲桑於九月大敗給苟晞,於是收拾餘眾,打算投奔劉淵建立的漢國,但又於赤橋敗於冀州刺史丁紹,石勒於是逃到樂平。後汲桑更在樂陵被晉兵所殺。

石勒投漢國後,於十月就成功讓據守上黨的㔨督和馮莫突歸降漢國,劉淵於是封石勒為輔漢將軍、平晉王。後又因據守樂平的烏桓人張伏利度不肯加盟漢國,石勒於是假稱得罪劉淵而投奔張伏利度,並與他結為兄弟,與其胡人部眾一同搶掠郡縣,所向無敵,於是眾人畏服。石勒在眾人心附自己後乘宴會抓著張伏利度,讓部眾推舉自己為主。石勒後釋放張伏利度而率領其部眾歸附漢國。劉淵於是加石勒為督山東征討諸軍事,並讓這些胡人部眾跟隨他。

劉淵派兵向外擴張,於永嘉二年(308年),派石勒領兵東侵。石勒於九月攻陷鄴城,征北將軍和郁逃走。十月劉淵稱帝,授予使持節,平東大將軍。不久石勒又率三萬進攻魏郡、汲郡和頓丘,五十多個由當地人集結的壁壘望風歸附,於是獲假壘主將軍、都尉印綬。後更殺魏郡太守王粹和冀州西部都尉馮沖,並擊敗殺害乞活軍將領赦亭和田禋。劉淵於是授予石勒安東大將軍、開府。石勒於永嘉三年(309年)進攻鉅鹿和常山,部眾增加至十多萬人,更有文士加入,以他們成立「君子營」,石勒以漢人張賓為謀主,刁膺、張敬為股肱。因軍事力量強大,在石勒派張斯游說之下,并州的胡羯大多亦跟從石勒。

劉淵之後派兵進攻壺關,石勒後被任命為前鋒都督,擊破劉琨派來救援壺關的軍隊,助漢國攻陷壺關。九月,晉司空王浚派祁弘與段務勿塵在飛龍山進攻石勒,石勒大敗,退屯黎陽,但仍能分派諸將攻打未及叛變的部眾,收降三十多個壁壘,並置守宰安撫。十一月,石勒進攻信都,殺害冀州刺史王斌。當時,王浚命裴整和王堪領兵討伐石勒,石勒於是立刻回軍抵禦。石勒到黎陽後,裴憲拋棄軍隊逃到淮南,王堪則退守倉垣。劉淵於是授命石勒為鎮東大將軍,封汲郡公,石勒辭讓封爵。

永嘉四年(310年),石勒南渡黃河,攻陷白馬後與王彌一同進攻徐、豫、兗三州。不久更攻下鄄城和倉垣,並北渡黃河進攻冀州諸軍,投降他的平民多達九萬多人。及後又協助劉聰等人進攻河內,並進攻冠軍將軍梁巨,晉懷帝派兵援救。梁巨因兵敗請降,石勒不許,最終坑殺萬多名降卒並殺死梁巨,援兵亦退還。此戰戰果使得河北各個自守的堡壘都震驚,紛紛送人質到石勒處。同年劉淵逝世,劉聰殺兄劉和繼位,任命石勒為征東大將軍、并州刺史、汲郡公。石勒這次辭讓征東大將軍。隨後便會合劉粲、劉曜、王彌大軍進攻洛陽,直入洛川。石勒又進攻倉垣,但被守將王讚擊敗。

石勒後來改攻南陽,早前在荊州叛變的雍州流民王如、侯脫和嚴嶷等都感到恐懼,於是派了一萬兵屯守襄城以作抵抗。但石勒到後擊敗守軍並將部眾全數俘虜,進駐宛城以北。當時侯脫據有宛城而王如守穰縣,王如怕石勒進攻,於是以珍寶賄賂石勒,與他結為兄弟;同時又因王如與侯脫不睦,於是勸石勒進攻侯脫。嚴嶷知道石勒攻宛後領兵救援,但石勒十二日便攻陷宛城,嚴嶷趕不及而直接向石勒投降。石勒誅殺侯脫,囚禁嚴嶷,呑併了二人部眾,軍力愈為強盛。

石勒於是進一步南侵,進攻襄陽並循漢水攻陷三十多個處於江西的壁壘。石勒留刁膺守襄陽後就率三萬精銳騎兵還攻王如,但因怕王如強盛,於是改攻襄城。王如知道後就命弟弟王璃率兵,假稱犒軍而襲擊石勒,但遭石勒擊滅。石勒至此有雄據長江、漢水一帶的意願,張賓雖然反對並勸他北歸但都不聽。

永嘉五年(311年),駐鎮建業的琅琊王司馬睿見石勒南侵荊州,於是派王導率兵討伐。而石勒軍糧不繼,更加因疫症損失大半士兵。石勒於是接納張賓建議,焚毀輜重,收好糧食和卷起盔甲,輕兵渡過沔水並進攻江夏,然後北歸,先攻陷新蔡,殺新蔡王司馬確,後再攻陷許昌。

永嘉五年(311年)三月,率領行臺和二十多萬晉兵討伐石勒的司馬越死在項縣,大軍於是在王衍及襄陽王司馬範帶領下護送司馬越靈柩回東海國。四月,石勒率輕騎追擊晉軍,終在苦縣寧平城追上大軍,並殺敗王衍所派的將軍錢端。晉兵在錢端敗死後潰敗,被石勒包圍並射殺,士兵在混亂中互相踐踏,全軍覆沒。石勒誅殺包括王衍以內隨行的官員和西晉宗室。不久石勒在洧倉追上司馬越世子司馬毗由洛陽東歸的部眾,又將司馬毗及宗室王等人殺害。

隨後,劉聰派呼延晏率大軍進攻洛陽,石勒領三萬騎兵到洛陽與大軍會合,攻陷洛陽,俘虜晉懷帝。戰後石勒將戰功歸於王彌和劉曜,於出屯許昌。七月,石勒領兵攻晉大將軍苟晞所駐蒙城,生擒苟晞並任用為左司馬。劉聰於是以石勒為幽州牧。

苟晞被擒後,王彌寫了一封言辭卑屈的書信祝賀石勒,同時又知道王彌忌憚自己,打算引自己到青州然後殺害。石勒於是聽從張賓的建議:乘王彌當時兵力減弱而消滅他。不久石勒就聽從張賓的建議,率兵救援與乞活軍相持不下的王彌以換取王彌的信任,隨後就借宴會的機會襲殺王彌,吞併了他的部眾,並假稱王彌謀反。劉聰知道石勒殺王彌後大怒,但又因怕他生了異心而不敢處罰,反而加授鎮東大將軍、督并、幽二州諸軍事、領并州刺史。

後來晉并州刺史劉琨將早年與石勒失散的石勒生母以及侄兒石虎送返,並授予侍中、車騎大將軍、領護匈奴中郎將、襄城郡公給石勒以作招降。但石勒拒絕,僅厚待劉琨使者和送名馬及珍寶給劉琨以作謝禮。

永嘉六年(312年),石勒在葛陂建屋宇,推廣耕作,營造船隻,打算攻略建業。但當年正遇上連綿三個月的大雨,司馬睿知道石勒的行動後更招集江南的兵眾會聚壽春以作抵禦。石勒軍中缺糧和有疫症,大量士兵死亡,而且多次收到來自司馬睿的討伐文告,似乎即將攻來,於是召集眾人討論。最後石勒接納張賓的建議,放棄留駐南方而北據鄴城三臺,經營河北,並以該處作根據地發展勢力。

石勒於是先將輜重北歸,又派石虎領兵攻壽春以防晉軍追擊輜重,最終晉兵雖然擊敗石虎,但仍因怕石勒有伏兵而只駐守壽春。然而石勒北歸時經過地方都堅壁清野,石勒試圖掠取物資都一無所獲,於是軍中有大飢荒,士兵相食。到東燕郡時因引誘當地建壁壘自守的向冰並成功在棘津擊敗向冰的軍隊,從而獲得軍需品,重振軍力,得以長驅直進,向鄴城進發。

守鄴城的晉北中郎將劉演知道石勒將來攻擊就加緊守城,然而其部將臨深和牟穆率部眾向石勒投降。石勒諸將當時打算強攻鄴城,但是張賓認為劉演仍能倚仗鄴城三臺而負隅頑抗,強攻未必能輕易奪取,反而暫時放棄攻取能讓劉演自己潰敗。於是建議石勒先消滅大司馬、領幽州刺史王浚和并州刺史劉琨這兩個大敵,並提出邯鄲和襄國兩處作為取鄴城前的臨時根據地。石勒聽從,率軍進據襄國。

石勒駐鎮襄國後,就上表劉聰陳述駐鎮當地的意圖,又分遣諸將進攻冀州各郡縣的壘壁,使他們大多都歸附,並運糧給石勒。劉聰收到上表後署石勒為使持節、散騎常侍、都督冀幽并營四州雜夷、征討諸軍事、冀州牧,進封上黨郡公,開府、幽州牧、東夷校尉如故。

石勒後進攻王浚將領游綸、張豺所駐的苑鄉,遭王浚派兵聯同段部鮮卑的段疾陸眷、段末柸和段匹磾所率部眾共五萬多人前來討伐。石勒屢次敗於段疾陸眷,更發現對方打算攻城,在張賓及孔萇的建言下,石勒在北城城內設立二十多道突門,並在門內藏伏兵;期間不出戰以示弱,待對方鬆懈來攻時,突門中的伏兵出擊,出其不意。石勒最終因而成功生擒段部鮮卑中最勇悍的段末柸,逼得段疾陸眷退兵。石勒之後派使者向段疾陵眷求和,並與其結為兄弟。隨著段疾陸眷退兵,王浚軍不能獨留,石勒於是解除了危機。同時,石勒厚待並送還段末柸的行動令他歸心於石勒,削弱了一直支持著王浚的鮮卑力量。游綸、張豺在戰後也向石勒請藩。

建興元年(313年),石勒派石虎攻陷鄴城,當地流人都向石勒歸降。石勒後又派孔萇攻定陵,殺兗州刺史田徽,王浚所任的青州刺史薄盛歸降石勒,山東地區各個郡縣相繼被石勒奪取,劉聰於是升石勒為侍中、征東大將軍。一直支持王浚的烏桓也背叛王浚,暗中歸附石勒,使得王浚勢力更弱。

永嘉之亂後,王浚就假立太子,設立行臺,自置百官,更打算自立為帝,驕奢淫虐。石勒打算消滅王浚,吞併其勢力,張賓建議石勒假意投降王浚。石勒於是卑屈的向王浚請降歸附,在王浚使者來時特意讓弱兵示人,並且故作卑下,接受王浚的書信時朝北向使者下拜和朝夕下拜王浚送來的塵尾,更假稱見塵尾如見王浚;又派人向王浚聲稱想親至幽州支持王浚稱帝。王浚於是完全相信石勒的忠誠。然而,石勒一直派去作為使者的王子春卻為石勒刺探了王浚的虛實,讓石勒做好充足準備。

建興二年(314年),石勒正式進兵攻打王浚,乘夜行軍至柏人縣,接受張賓的建議,利用王浚和劉琨的積怨,寫信並送人質向劉琨請和,聲稱要為他消滅王浚。因此劉琨最終都沒有救援王浚,樂見王浚被石勒所滅。石勒一直進軍至幽州治所薊縣,先以送王浚禮物為由驅趕數千頭牛羊入城,阻塞道路,之後更縱容士兵入城搶掠,並捕捉王浚,數落王浚不忠於晉室,殘害忠良的罪行。石勒命將領王洛生押解王浚到襄國處斬,又盡殺王浚手下精兵萬人,擢用裴憲和荀綽為官屬。石勒留薊兩日後就焚毀王浚宮殿,留劉翰守城而返。

石勒回到襄國後將王浚首級送給劉聰,劉聰於是任命石勒為大都督、督陝東諸軍事、驃騎大將軍、東單于,並增封二郡。劉聰更與建興三年(315年)賜石勒弓矢,加崇為陝東伯,專掌征伐,他所拜授的刺史、將軍、守宰、列侯每年將名字及官職上呈就可,又以石勒長子石興為上黨國世子。

建興四年(316年),石勒率兵在玷城圍困晉樂平太守韓據,韓據向劉琨求援。劉琨因不久以前代國內亂而獲得拓跋猗廬舊部箕澹及衞雄率代國晉人和烏桓人加入而大大強化了軍力,於是打算借此討伐石勒,因此不顧箕澹和衞雄的勸阻,動用所有軍力,派箕澹率二萬作前鋒,自己則進屯廣牧為箕澹聲援。石勒以箕澹部眾遠道而來而筋疲力竭,而且烏合之眾,號令不齊,不難應付,決意迎擊。石勒於是在山中設下伏兵,自己率兵與箕澹作戰,然後向北退兵引箕澹深入,與伏兵夾擊箕澹而大敗對方,箕澹北逃到代郡而韓據則棄城奔劉琨。此戰震動并州,守著治所陽曲的劉琨長史李弘竟以并州投降石勒,使得劉琨進退失據,唯有投奔幽州刺史段匹磾。

太興元年(318年),劉聰患病,徵石勒為大將軍、錄尚書事,受遺詔輔政,但石勒不受。劉聰於是又命石勒為大將軍、持節鉞,都督等如故,並增封十郡,又不受。不久劉聰死,太子劉粲繼位後不久便被靳準所殺,自稱漢天王。石勒於是命張敬率五千兵作前鋒,自己親率五萬兵討伐靳準。石勒進據襄陵北原,羌羯四萬多個部落向石勒投降,靳準數度挑戰都不能攻破石勒的防禦。十月劉曜北上討伐靳準,並於赤壁(今山西河津縣西北赤石川)即位為帝,任命石勒為大司馬、大將軍,加九錫,增封十郡,進爵為趙公。

隨後石勒進攻首都平陽,各族共十多萬部落都向石勒投降。十一月,靳準派卜泰向石勒請和,石勒將使者囚禁後送交劉曜,以示城內並無歸附劉曜之意。但劉曜卻由卜泰為他傳話,勸靳準迎接他到平陽。靳準考慮未決,於十二月被靳康等人所殺,推靳明為主,向劉曜請降。石勒見靳氏不向自己歸降,大怒,率軍進攻靳明,靳明出戰但被擊敗,於是閉門自守。不久石虎與石勒會合,共攻平陽,靳明向劉曜求救,劉曜派兵迎靳明出城。石勒則進平陽城,焚毀平陽宮室,遷城內渾儀、樂器到襄國,留兵戍守後返回襄國。

太興二年(319年)二月,石勒派左長史王脩獻捷報給劉曜,劉曜於是授予石勒太宰、領大將軍,進爵趙王,並加一系列特殊禮待,如同昔日曹操輔東漢的先例。劉曜讓王脩返回襄國後,石勒舍人曹平樂卻對劉曜說王脩前來的的目的是要探劉曜的虛實,王脩返回報告後,石勒就會進襲劉曜。當時劉曜實力的確大為損耗,聽到曹平樂的話後十分害怕王脩會向石勒報告他的虛實,於是追還王脩並殺害王脩,原本授予石勒的官位、封爵及禮遇亦擱置。王脩副手劉茂卻成功逃脫,到石勒於三月回到襄國時就報告王脩之死,石勒大怒:「我事奉劉氏,盡心做得比起人臣的本份更有餘了。他們的基業都是我打下來的,今日得志了竟想來謀算我。趙王、趙帝,我自己也能給自己,哪用得著由他們賜予!」自此與前趙結了仇怨。

當年十一月,石勒稱大將軍、大單于、領冀州牧、趙王,於襄國即趙王位,正式建立後趙,稱趙王元年。

雖然石勒於建興二年(314年)殺害王浚,取得薊縣,但不久石勒所命駐守薊縣的劉翰背叛石勒而歸附段匹磾,段匹磾於是進據薊縣。然而,因段匹磾多次與段末柸相攻,又於太興元年(318年)殺死劉琨,使得大批胡人和漢人投奔邵續、段末柸或石勒,導致實力大減。段匹磾於次年因石勒將領孔萇進攻幽州,不能自立,因而投奔晉冀州刺史邵續還據有的厭次。至太興三年(320年)段末柸再擊敗段匹磾,段匹磾與邵續聯手追擊段末柸並擊敗他,隨後就與弟弟段文鴦北攻段末柸弟弟駐守的薊城。此時,石勒知道邵續勢孤,於是派石虎進攻厭次,最終生擒出城迎擊的邵續,但厭次城尚由邵續子邵緝等人據守。段匹磾此時回軍,尚離厭次城八十里時就聽聞邵續被擒的消息,于是部眾潰散,石虎也前來襲擊,只因段文鴦奮戰才得以進入厭次城。

太興四年(321年),石勒又派石虎和孔萇進攻厭次,段文鴦力戰被擒,段匹磾無力抵抗,試圖南奔東晉又不行,亦被石虎所捕。至此,晉朝於河北的各個藩鎮皆被攻陷。

建興元年(313年),司馬睿以祖逖為奮威將軍、豫州刺史,祖逖由此開始收復中原的行動,並進據譙城。太興二年(319年)豫州一塢主陳川與祖逖相爭但不敵,於是向石勒投降,祖逖因此討伐陳川,石勒則派石虎率兵救援,將祖逖擊敗,祖逖敗退至淮南。但祖逖於下一年就發動反擊,擊敗守著陳川故城的將領桃豹,並多次邀擊當地的後趙軍隊,當地留戍的後趙兵鎮深為困擾,很多都歸附祖逖。

因為祖逖擅於安撫,不但黃河以南地區的人民歸附祖逖,連石勒根據地河北的塢主也向祖逖報告後趙的情況,以至於石勒不敢以軍事力量強攻豫州,因而決定與祖逖修好,又允許兩地通商。當時祖逖牙門童建殺新蔡內史周密歸降石勒,石勒卻殺死童建並將首級送交祖逖。而祖逖也不接納背叛後趙而歸降的人,因此兩國邊境安定,兗、豫二州人民得以休息,但不少人其實都有雙重身份,同時歸屬東晉與後趙。

實際上,祖逖一直未忘北伐,他將通商獲得的利錢用來準備軍需物資,而且又修繕虎牢城,瞭望四方,並建立壁壘,作為守護豫州土地的堡壘。但壁壘未建成祖逖就死去。永昌元年(322年),石勒因祖逖已死而再度南侵,接替祖逖的祖約不能抵抗,南退至壽春,石勒於是留兵駐屯豫州,豫州再次混亂,再次進入後趙的勢力範圍。同時石勒派兵侵擾徐、兗二州,東晉駐守當地的部隊都只有南退,很多當地塢主都向石勒歸降。

太寧元年(323年),石勒派石虎攻滅一直割據青州的曹嶷,盡有青州。

太宁二年(324年),後趙司州刺史石生進攻前趙河南太守尹平並殺害他,而且掠奪了新安縣五千多戶人。自此開始兩國之間的戰事,作為兩國邊界的河東和弘農兩郡之間淪為戰場。次年西夷中郎將王騰殺并州刺史崔琨並以并州歸降前趙,屢敗於石生的晉司州刺史李矩、穎川太守郭默等也遣使依附前趙,於是前趙大舉進攻後趙。但前趙所派的劉岳被石虎擊敗,遭生擒和坑殺九千餘人,王騰也被石虎攻滅,李矩等被擊敗而南奔東晉,大量部眾歸降後趙。戰後後趙盡有司、豫、徐、兗四州之地。

太和元年(328年),石虎攻蒲阪,前趙帝劉曜親率全國精兵救援蒲阪,大敗石虎,於是乘勢進攻石生鎮守的洛陽,以水灌城,同時又派諸將攻打汲郡和河內郡,後趙舉國震驚。石勒見此,不顧程遐的勸阻執意親自救援洛陽,於是命桃豹、石聰、石堪等到滎陽會合,自己領兵直攻洛陽金鏞城。及至十二月,石勒與後趙諸軍於成皋集合,發現劉曜竟不設守軍,於是輕兵潛行。劉曜直至石勒渡過黃河後才開始準備防禦,從前線捕獲的羯人口中知道石勒親率大軍前來進攻後更為害怕,於是解圍而於洛西列陣。石勒在開始進攻之時曾說:「劉曜設大軍於成皋關防禦,是他的上策;列兵於洛水阻截則次之;坐守洛陽,就會讓我生擒了。」見劉曜列陣於洛西,石勒十分高興,認為必勝無疑,隨後就與石虎及石堪、石聰分三道夾擊劉曜,最終大敗前趙,更生擒劉曜,押送到襄國。

次年,留守長安的前趙太子劉熙知道劉曜被擒後大驚,於是放棄長安而西奔上邽,各征鎮都棄守防地跟隨,導致關中大亂,前趙將領以長安城歸降後趙,石勒又派石虎進攻關中的前趙殘餘力量。終於當年八月,前趙劉胤率大軍反攻長安時被石虎擊敗,前趙一眾王公大臣都被石虎所捕,同年石勒亦殺劉曜,前趙亡。石勒又於咸和二年(327年)派石虎擊敗代王拓跋紇那,逼得對方徙居大寧迴避其軍事威脅。至此後趙除前涼、段部鮮卑的遼西國及慕容鮮卑的遼東國三個政權外幾乎佔領整個中國北方。

太和三年(330年)二月,石勒稱大趙天王,行皇帝事,並設立百官,分封一眾宗室。至九月,石勒正式稱帝。

石勒稱帝後,於次年四月到鄴城,打算營建鄴城新宮,如張賓昔日所言,以其作為新的都城。當時廷尉續咸大力反對,石勒堅決不納;後中山郡有洪水災害,有百多萬根大木頭隨水沖到堂陽,石勒視此為上天協助自己營建鄴都,於是正式施行,自己親自視察工程。

石勒在稱帝時立了兒子石弘為皇太子,石弘愛好文章,對儒士親敬,並沒有石勒的強悍。然而當時任太尉、尚書令石虎因為戰功顯赫,掌有重兵和實權,徐光和程遐都認為一旦石勒去世,石弘不能駕馭石虎;同時又因石虎怨恨二人,二人擔心一旦石虎奪權會誅滅二人及其宗族,於是多次向石勒進言,要求強化太子權力,讓太子親近朝政,並削弱石虎權力。石勒最終命太子省批核上書奏事,並由中常侍嚴震協助判斷,只有征伐殺人的大事才送交石勒裁決。於是嚴震權力高漲,石虎則失勢,心有不滿。但石勒始終沒有聽從二人除去石虎的建議。

建平三年(332年),石勒到鄴城,到石虎的府第中,石勒知道石虎的不滿,於是允諾皇宮建成後會為他建設新府第,以此作安撫。但其實石虎自太和三年(330年)石勒稱天王時將大單于位封給石宏就十分不滿;對於咸和元年(326年)石勒讓石弘駐鎮鄴城和修建鄴城三臺時逼遷其家室的事也懷恨在心。

石勒於建平四年(333年)患病,石虎入侍並詔不許親戚大臣見石勒,因此無人知道石勒的病況。後又矯詔召命石勒用以防備石虎而出為外藩的秦王石宏及彭城王石堪到襄國,將他們留在襄國,即使石勒知道後立刻命二人回到駐地,石虎仍然不讓他們回去,更騙石勒說二人已在歸途上。七月戊辰日(8月17日),石勒逝世,享年六十歲。廟號高祖,諡號明皇帝,葬於高平陵。

石勒重視教育,在段部鮮卑和烏桓都相繼歸附支持自己,王浚勢弱,領下司州、冀州等地安定,人民開始繳納租稅時,在當地設立太學,以明經善書的官吏作文學掾,選了部下子弟三百人接受教育。後來,石勒又在襄國增置宣文、宣教、崇儒、崇訓等十多間小學,選了部下和豪族子弟入學。石勒更曾親臨學校,考核學生對經典意義的理解,成績好的就獲獎賞。

石勒稱趙王後,命支雄和王陽為門臣祭酒,專掌胡人訴訟,命張離、劉謨等人為門生主書,專掌胡人出入,且禁制胡人欺侮衣冠華族,以胡人為國人。另又遷徙三百家士族到襄國,置崇仁里讓他們聚居,又置公族大夫統領,實行胡漢分治。

石勒亦重視修史工作,命任播、崔濬為史學祭酒,又命記室佐明楷、程陰、徐機撰寫《上黨國記》,中太夫傅彪、賈蒲、江軌撰寫《大將軍起居注》,參軍石泰、石同、石謙、孔隆撰寫《大單于志》。稱帝後又擢升五個太學生為佐著作郎,記錄時事。

石勒實行考試機制,初建五品,由張賓領選舉事。後又定九品,命左右執法郎典定士族,並且副任選舉職能。又令僚佐及州郡每年都舉秀才、至孝、廉清、賢良、直言、武勇之士各一人。後來更以王波為記室參軍,典定九流,始立秀、孝試經的制度。又於稱帝後命各郡國設立學官,每郡都置博士祭酒二人,學生一百五十人,經三次考試後才畢業入仕。

石勒見百姓久經戰亂,社會秩序剛剛恢復,資源不足,於是下令禁止釀酒,祭祀時都只用發酵一晚的甜酒。數年以後就再沒人釀酒了。

石勒又命人重訂度量衡。

石勒在北方推度耕作,以右常侍霍皓為勸課大夫,與典農使者朱表及典農都尉陸充等巡核各州郡,核實戶籍,鼓勵農桑。讓收獲最多的人爵五大夫。

石勒感恩,並會作出報答。例如郭敬在早年曾經對他有恩,接濟過他。後來石勒在上白攻滅乞活軍將領李惲時重遇郭敬,竟立刻下馬抓著他的手,說:「今日相遇,是天意呀!」於是賜他衣服車馬,署他為上將軍,更將原本打算坑殺的李惲餘眾賜給他作為部眾。劉琨曾送還石勒母親以圖招降石勒,雖然石勒拒絕,但仍以厚禮作回報;後來劉琨及石勒雖然互相敵對,但在石勒攻打北中郎將劉演時擒獲其弟劉啓,而劉演和劉啓都是劉琨的侄兒,石勒此時仍然感謝劉琨讓他母子重聚的恩德,不但沒有殺死劉啓,還賜他田宅,命儒官教授他經典。

石勒下令禁止說「胡」字,更是所有忌諱字中懲罰最重者。但一次有胡人喝醉了,騎馬突入止車門,違反門禁,於是石勒在憤怒之下召責宮門小執法馮翥。馮翥見石勒十分恐懼,只顧申辯而忘了忌諱,說:「剛才有個醉了的胡人,騎馬進了門,我已經大聲喝止並攔住他,但都不能和他對話。」石勒聽後,沒有憤怒,反而笑說:「胡人正就是難以與之對話的了。」並寬恕了他的罪。

石勒雖不識字,但喜好文史,即使行在軍旅仍常聽漢儒講讀中國歷史,隨時發表自己的見解。一次聽到酈食其勸劉邦得天下後分封六國諸王,大喊糟糕,懷疑劉邦怎能平定天下。後來知道張良勸阻,才連忙說「賴有此耳。」可見他天資之高,英明賢達。

石勒曾在夜間微服出行,到營衞時曾以錢財賄賂守門者讓他出去,但永昌門門候王假卻不受金錢,更打算收捕他,只因隨從及時來到才未被捕。下一日石勒就召王假為振中都尉,賜爵關內侯。

石勒曾問大臣徐光他能比作昔日哪位君主,徐光說石勒神謀武略,比漢朝開國君主劉邦更高,而劉邦以後再沒有人能和石勒比較。石勒笑言徐光說得太誇張,自我評價道:「我若果與劉邦同時,就當作他的臣下,與韓信、彭越皆為其將;若果與漢光武帝劉秀同時,就會與他爭奪中原,不知鹿死誰手。大丈夫行事,應該磊磊落落,如日月皎潔,絕不可以像曹操、司馬懿那樣欺負孤兒寡婦,用奸計奪取天下。」足見石勒尊崇劉邦、劉秀白手興家而貶抑曹操和司馬懿的奪權行為。

\subsubsection{太和}

\begin{longtable}{|>{\centering\scriptsize}m{2em}|>{\centering\scriptsize}m{1.3em}|>{\centering}m{8.8em}|}
  % \caption{秦王政}\
  \toprule
  \SimHei \normalsize 年数 & \SimHei \scriptsize 公元 & \SimHei 大事件 \tabularnewline
  % \midrule
  \endfirsthead
  \toprule
  \SimHei \normalsize 年数 & \SimHei \scriptsize 公元 & \SimHei 大事件 \tabularnewline
  \midrule
  \endhead
  \midrule
  元年 & 328 & \tabularnewline\hline
  二年 & 329 & \tabularnewline\hline
  三年 & 330 & \tabularnewline
  \bottomrule
\end{longtable}

\subsubsection{建平}

\begin{longtable}{|>{\centering\scriptsize}m{2em}|>{\centering\scriptsize}m{1.3em}|>{\centering}m{8.8em}|}
  % \caption{秦王政}\
  \toprule
  \SimHei \normalsize 年数 & \SimHei \scriptsize 公元 & \SimHei 大事件 \tabularnewline
  % \midrule
  \endfirsthead
  \toprule
  \SimHei \normalsize 年数 & \SimHei \scriptsize 公元 & \SimHei 大事件 \tabularnewline
  \midrule
  \endhead
  \midrule
  元年 & 330 & \tabularnewline\hline
  二年 & 331 & \tabularnewline\hline
  三年 & 332 & \tabularnewline\hline
  四年 & 333 & \tabularnewline
  \bottomrule
\end{longtable}


%%% Local Variables:
%%% mode: latex
%%% TeX-engine: xetex
%%% TeX-master: "../../Main"
%%% End:

%% -*- coding: utf-8 -*-
%% Time-stamp: <Chen Wang: 2019-12-18 17:41:04>

\subsection{石弘\tiny(333-334)}

\subsubsection{生平}

石弘(314年-335年),字大雅,是中國五胡十六國時代後趙的君王。上黨武鄉(今山西榆社)人,后赵明帝石勒二子,母程氏。

史載石弘「幼有孝行,以恭謹自守」,受经于杜嘏,诵律于续咸。石勒觉得他不似将门之子,派刘征、任播授以兵书,王阳教之击刺。石勒病重时,中山王石虎与石弘、中常侍严震在宫中侍候,石虎矫诏断绝内外消息。建平四年(333年)九月,石勒一死,石弘繼位,立嫡母劉氏為皇太后。石虎下達第一個“詔令”,將石弘舅父右光祿大夫程遐、中書令徐光論罪誅斬,拜石虎為丞相、魏王、大單于,加九錫,以魏郡等十三郡為邑。石弘恐懼丞相石虎,欲讓位於石虎。石虎拒絕:“君薨而世子立,臣安敢亂之!”遂即位,拜石虎为丞相。

刘太后与石勒养子彭城王石堪谋除石虎,擁皇弟南陽王石恢為盟主。石堪單騎出逃,直奔兗州。到達廩丘時,因事機不密,逮送至襄國,被活活烤死。劉太后被石虎发现参与其中,遭废黜弒害,石虎改尊石弘生母程氏为皇太后。河東王石生在關中起兵,石朗在洛陽起兵,聲言滅石虎。石虎擒下石朗,他先砍掉石朗的雙腳,再斬首。長安一戰,石虎大敗,“枕尸三百余里”,此時石生同盟的鮮卑人竟然反叛,石虎重振軍勢,石生被部下斬首,獻給石虎。延熙元年(334年)十月石弘持玺绶向石虎表明願意禅位。石虎说:“天下人自当有议,何为自论此也!”意思是只能自己逼石弘退位,而不能接受石弘禅位。石弘哭着回宫对程太后说:“先帝真要灭种了!”不久石虎称石弘居丧不孝,废为海阳王,与程太后及弟秦王石宏、石恢一同幽禁崇训宫,不久皆殺之。

\subsubsection{延熙}

\begin{longtable}{|>{\centering\scriptsize}m{2em}|>{\centering\scriptsize}m{1.3em}|>{\centering}m{8.8em}|}
  % \caption{秦王政}\
  \toprule
  \SimHei \normalsize 年数 & \SimHei \scriptsize 公元 & \SimHei 大事件 \tabularnewline
  % \midrule
  \endfirsthead
  \toprule
  \SimHei \normalsize 年数 & \SimHei \scriptsize 公元 & \SimHei 大事件 \tabularnewline
  \midrule
  \endhead
  \midrule
  元年 & 334 & \tabularnewline
  \bottomrule
\end{longtable}


%%% Local Variables:
%%% mode: latex
%%% TeX-engine: xetex
%%% TeX-master: "../../Main"
%%% End:

%% -*- coding: utf-8 -*-
%% Time-stamp: <Chen Wang: 2021-11-01 11:55:11>

\subsection{武帝石虎\tiny(334-349)}

\subsubsection{孝帝石寇覓生平}

石寇覓(3世紀-?),是後趙武帝石虎的父親。他早逝,因此石虎被石勒的父親石周曷朱收养,所以又有人稱石虎是石勒的弟弟。

石虎稱帝後,追封他為皇帝,諡號孝皇帝,廟號太宗。

\subsubsection{武帝石虎生平}

趙武帝石虎(295年-349年5月26日),字季龍,上黨武鄉(今山西榆社)人。中國五胡十六國時代中,後趙的第三位皇帝。廟號太祖,諡號武帝。石虎是後趙開國君主石勒的侄兒。石虎生性殘忍,發家前,不僅用殘酷的手段先後殺死兩位妻子,即使在軍隊中如果遇到與他一樣強健的戰士,他會以打獵戲鬥為由,借機將對手殺死,以解心頭之快;戰鬥中,對俘獲的俘虜,不分好壞,不分男女一律坑殺,很少有俘虜生還。

333年,石勒駕崩,其皇位由兒子石弘繼承。因石虎掌握兵權勢大,石勒妻刘太后與養子彭城王石堪擁立石勒子南陽王石恢欲舉兵反對石虎,不幸事洩,劉太后被殺,石堪被捕活活烤死,石恢被召回,咸康元年(334年)十月石弘持璽綬向石虎表明願意禪位,石虎拒绝。十一月,石虎称居摄赵天王,石弘被廢為海陽王,同年石虎殺海陽王石弘、弘母程氏、石弘弟秦王石宏、南陽王石恢。至335年,其首都由襄國(今中國河北邢台)遷至鄴(今河北邯郸市臨漳县城西南20公里邺城遗址),並特地派人到洛陽將九龍、翁仲、銅駝、飛廉轉運到鄴裝點宮殿。337年4月11日(二月辛巳),石虎称大赵天王,349年2月4日(正月初一辛未朔)正式即皇帝位。同年5月26日(四月己巳),患病而死,随后,他的儿子争夺皇位,后赵很快灭亡。石虎在位期間,表現了其殘暴好色的一面,如史書載石虎曾經下達過一條命令:全國二十歲以下、十三歲以上的女子,不論是否嫁人,都要做好準備隨時成為他後宮佳麗中的一員,「百姓妻有美色,豪勢因而脅之,率多自殺」,因此被評為五胡十六國中的暴君。

生性殘暴的石虎,少年時喜歡用彈弓打人為樂。十八歲時,由於其武藝超凡且勇猛過人,因此受到石勒的寵信,被封為征虜將軍。石勒其後又為石虎納聘將軍郭榮的妹妹為妻,但石虎心儀的是當時的雜技名角鄭櫻桃。於是便把郭氏殺死,而後迎娶鄭氏。之後,石虎又娶了崔氏,但崔氏最後因鄭氏的挑撥而死於石虎手中。

在軍中,凡是比石虎有才藝或有武藝的,石虎就會設法把他們殺死,死於他手上的人不可計數。石虎是好殺的人,每次攻下一座城後,不論男女都一律殺死。一次,石虎攻下青州後又下令屠城。此次血腥屠城,僅餘七百多人保全性命。

太和三年(330年)二月,石勒称大赵天王,行皇帝事;以妃刘氏为王后,世子石弘为皇太子,程遐为右仆射、领吏部尚书。中山王石虎怒,秘密对长子齐王石邃说:“我亲冒矢石随主上征战二十余年,是成大赵之业者,应该做大单于,主上却授予‘黄吻婢儿’,想起来就令人气塞,不能寝食!待主上晏驾之后,我不会给他留种。”

石勒臨終前,石虎威迫太子石弘把曾勸石勒除掉自己的大臣程遐和徐光逮捕入獄并杀死。又命兒子石邃率兵入宿衛,文武百官害怕不已,太子石弘也嚇得連忙對石虎說道自己不是治天下的人材,石虎才是真正的天子。但石虎明白石勒屍骨未寒,就這樣強登上皇帝只會眾叛親離,並受後世人的唾罵。因此寧願有點耐性,演齣曹操的「挾天子以令諸侯」的戲,由這位太子登位。

石弘坐上寶座後,成為了傀儡皇帝。石弘登基後便被石虎所逼,将程遐、徐光论罪诛斩,封石虎為丞相、魏王、大單于,再封土地,封邦建土。而他的三名兒子都被封為擁有軍權的職位,至於他的親人和親信都放排在有大權的職位上,而之前石勒的文武百官就放置在毫無權力的閑職上。這時後趙已真正的形成「挾天子以令諸侯」的局面。刘太后与石勒养子石堪合谋起兵拥戴石弘的弟弟石恢为盟主,石堪兵败被杀,石恢被征召回京,刘太后被石虎废黜杀害。石弘生母程氏被尊为太后,也没有实权。延熙元年(334年)十月石弘持玺绶向石虎表明愿意禅位。石虎说:“天下人自当有议,何为自论此也!”意思是只能自己逼石弘退位,而不能接受石弘禅位。石弘哭着回宫对程太后说:“先帝真要灭种了!”不久石虎称石弘居丧不孝,废为海阳王,自称天王,並把石弘、程太后和石弘的弟弟石宏、石恢都幽禁于崇训宫,旋即殺死他們。

石虎稱天王後,石邃為太子(之前为魏太子),並開始他極為奢侈的統治。石虎不顧人民負擔到處征殺,使人民的兵役和力役負擔相當重大,他又下令凡是有免兵役特權的家族,五丁取二,四丁取其二,而沒有特權的家族則所有丁壯都需服役。為了攻打東晉,在全國征調士兵的物品:每五人出車一乘、牛兩頭、米穀五十斛、絹十份,不交者格殺勿論。無數的百姓為了安全,不得不把自己的子女賣掉。

後趙建武二年(336年),石虎為了裝飾鄴城,令牙門將張彌把洛陽的鐘虞、九龍、翁仲、銅駝、飛廉等相生物運到去鄴城。在運送途中,一隻鐘虞沒入了黃河,於是張彌便下令三百多名人潛到水中,把鐘虞繫上繩,再利用百多頭牛和許多架轆轤把鐘虞拉上來,之後就地造了可裝萬斛的大船,把這些相生運過黃河。其後又製造了特大的車子以把相生運送到鄴城,這次的行動單是運送就足足用了人民千千萬萬的勞力和血汗了。

在鄴城以西三里,有石虎所建的桑梓苑,苑內臨漳水修建了很多座豪華的宮殿,下令从民间强行掠夺十三岁至二十岁的女子三万余人。仅在345年一年间,各郡县官吏为搜罗美女上交差事,公然抢掠貌美的有夫之妇九千余人,不忍受夺妻之辱而反抗的男人均遭残杀,被夺女子为避免受辱也大多自杀,一大批家庭夫妻离散,家破人亡。但石虎征集女人倒不完全是好色,石虎内置女官十有八等,教宫人星占及马步射。置女太史于灵台,仰观灾祥,以考外太史之虚实(《晋书·石季龙载记》)。石虎还鉴于东汉太监专权的危害,不信任太监,因此宫中没有太监,相关职务只能由女人充当。苑內養有奇珍異獸,石虎經常在此遊玩設宴。從襄國至鄴城的二百里內,每隔四十里使建一行宮,每宮都有一位夫人,數十位的侍婢居住,由黃門官守門。

而在浴室上,更是別出心裁:在皇后浴室中,門窗都是由木刻成的鏤孔圖案,石虎就是在這兒和皇后梳洗。而每年的4月8日,在這裏精工製造的九龍吐水浴太子之像。在太武殿前,溝的中間有多層以紗等的「過濾器」。

「鳳詔」也是石虎的發明之一,石虎處理政事時會和皇后一起坐在高高在上的樓觀上,並用五色紙上寫下詔書,把詔書放在一只由木雕刻成、外施漆畫、金腿的「鳳凰」口中。金鳳凰繫在轆轤牽引的繩上。當下詔時,待人把轆轤搖動,「鳳凰」就像從天空飛下來般,大臣們都要跪下接詔。

每隔不久,石虎便會大會群臣,每次都頭戴通天冠、身佩玉璽、循周禮的規定禮樂一番,然後觀賞雜技表演,群臣大會幾乎都有美酒佳釀給自己和群臣所飲用。殿上掛著了大鐵燈一百二十支。在燈下有數千戴金銀佩飾的宮女和石虎觀看表演。在殿外,三十部鼓吹同時演奏,鼓樂震天,場面極為震撼。

石虎好射獵,但因體胖而無法騎馬,因而改為用獵輦。而他的獵輦裝有豪華的華蓋羽葆,由二十人推行,座下有轉軸裝置,可以根據獵物的所在地轉動。在出獵時,石虎會戴上由金鏤織成的合歡帽、穿上合歡褲,手拿著弓箭。而石虎為了方便行獵,於是把黃河以北的大片良田為獵區,派御史監督,规定除自己外有敢在獵區獵獸者处死。而这“犯兽”的刑法,又被各官员用来欺压百姓,若百姓家有美女或好的牛马等家畜,官员要求不给,就诬陷其“犯獸”,因此被判死刑者甚多。

石虎像他伯父石勒一样崇拜大和尚佛圖澄,石勒因信佛圖澄之言而減少了很多殺虐。有次石虎向佛圖澄問甚麼是佛法,佛圖澄只說了四字:「佛法不殺」.石虎沒有聽取佛圖澄的勸告,後來倒是聽了一個叫吳進的假和尚說胡人的氣數已衰,而晉人的氣數開始恢復,一定要苦役晉人才能壓著他們的氣數。結果石虎下令強徵鄴城附近各郡的男女百姓十六萬多人、車十萬乘在鄴城東修華林苑,並圍苑建數十里的長牆。

在中國歷史上還記載著石虎父子的相互殘殺。

事緣石虎兒子石邃不滿父親寵愛其餘的兩個兒子石宣和石韜,漸漸地,這種不滿轉化為仇恨,對父亲石虎恨之入骨,恨不得弒父奪位。石虎得知後,把石邃的手下李顏捉來審問,李顏嚇得不知如何是好,便一五一十地都事情告訴石虎:石邃密謀殺石宣和弒石虎奪位。石虎得知後把李顏及其家人三十多人斬首處死,再把石邃幽禁於東宮。石邃被幽禁後仍然目中無人,石虎一怒之下,下令把石邃和他的妻子、家人殺死,再塞進同一口棺材內,同一時間又把石邃的黨羽二百多人殺死。

石邃死後,石宣為皇太子,石宣之母杜昭儀為天王皇后,鄭櫻桃廢為東海太妃。同时又让石韬掌握军政大权,打算让石宣和石韬之间达成一定的平衡。结果却引发新一轮内讧。

到了其後,石宣因不滿其父石虎較寵愛石韜而要除掉石韜。不久之後,兩兄弟經常發生衝突,石宣於是把石韜砍掉手足、雙眼刺爛、破肚慘死。石宣並計劃在石韜的喪禮上弒父,以奪皇位。

石虎得知愛兒石韜死了,昏迷了好一段時間,他本想出席兒子的喪禮,幸而大臣提醒,沒有出席喪禮。後來,石虎得到知情人的報告,得知皇太子石宣殺了石韜。憤怒到極點的石虎在设计控制石宣后,下令用鐵環穿透石宣下巴鎖著,又將他的飯菜倒入大木槽,使石宣進食時像豬、狗般。石虎逼石宣用舌頭舐著殺石韜的劍上的血,石宣發出了震動宮殿的哀聲。石虎下令在鄴城城北埋起柴堆,上面設置了木竿、竿上安裝了轆轤。並讓石韜生前最寵的宦官,郝稚和劉霸二人拽著石宣的舌头和頭髮,沿著梯子拉上柴堆,之後用轆轤把他絞起來,再用一模一樣的方法向石宣施刑。當石宣已奄奄一息時在柴堆四處點火,石宣被燒成了灰燼。這還未能平熄石虎的怒火,再下令把灰燼分散到名門道中,任人、馬、馬車的輾踏,又將石宣的妻、子九人殺死,又把石宣的衛士、宦官等數百人車裂,將屍體投進漳河。石宣的一个年幼的儿子抱着石虎的大腿求饶,石虎心生怜悯想赦免但大臣们却将其夺走处死,石虎的腰带都被孙子扯断。东宫卫兵十余万被流放边疆,途中举行暴动,石虎急忙调集重兵镇压了下去。但后赵统治基础动摇了。

连杀两位太子后,太尉张举认为燕公石斌、彭城公石遵都有武艺文德,建议从二人中选择储君。但戎昭将军张豺先前曾将刘曜的女儿献给石虎,生有齐公石世,于是他说服石虎立石世,这样刘氏成为太后,他可以辅政。石虎说:“太子二十多岁就想弑父,石世才十岁,等他二十岁了,我已经老了。”于是与张举、李农定议,敕令公卿上书请立石世为太子,于是立石世为太子,其母刘氏为皇后。

石虎病重时,以石遵为大将军,镇关右,石斌为丞相、录尚书事,张豺为镇卫大将军、领军将军、吏部尚书,同受遗诏辅政。刘皇后怕石斌辅政不利于石世,就与张豺合谋,派使者诈称石虎病愈,石斌性好酒猎,于是又恣意而为。刘皇后便矫命称石斌无忠孝之心,免其官,以王归第,派张豺弟张雄率龙腾五百人看守。石遵从幽州来朝,被打发走,石虎知道后说“恨不见之”。一次石虎驾临西阁,龙腾将军、中郎二百余人列拜于前,说宜令燕王石斌入宿卫,典兵马,也有请求以石斌为皇太子。石虎不知石斌已被罢官囚禁,命召石斌来,左右说石斌饮酒得病不能入。石虎又命以辇迎之,要将其玺绶交给他,最后也没人前去。不久石虎昏眩入内。张豺让张雄等矫石虎命杀石斌,刘皇后又矫命以张豺为太保、都督中外诸军、录尚书事,加千兵百骑,一依霍光辅汉故事。

石虎死后,石世继位,不久就被推翻,石虎诸子石遵、石鉴、石祗相继登基,又相继被杀。石虎死后三年,后赵就灭亡了。

\subsubsection{建武}

\begin{longtable}{|>{\centering\scriptsize}m{2em}|>{\centering\scriptsize}m{1.3em}|>{\centering}m{8.8em}|}
  % \caption{秦王政}\
  \toprule
  \SimHei \normalsize 年数 & \SimHei \scriptsize 公元 & \SimHei 大事件 \tabularnewline
  % \midrule
  \endfirsthead
  \toprule
  \SimHei \normalsize 年数 & \SimHei \scriptsize 公元 & \SimHei 大事件 \tabularnewline
  \midrule
  \endhead
  \midrule
  元年 & 335 & \tabularnewline\hline
  二年 & 336 & \tabularnewline\hline
  三年 & 337 & \tabularnewline\hline
  四年 & 338 & \tabularnewline\hline
  五年 & 339 & \tabularnewline\hline
  六年 & 340 & \tabularnewline\hline
  七年 & 341 & \tabularnewline\hline
  八年 & 342 & \tabularnewline\hline
  九年 & 343 & \tabularnewline\hline
  十年 & 344 & \tabularnewline\hline
  十一年 & 345 & \tabularnewline\hline
  十二年 & 346 & \tabularnewline\hline
  十三年 & 347 & \tabularnewline\hline
  十四年 & 348 & \tabularnewline
  \bottomrule
\end{longtable}

\subsubsection{太宁}

\begin{longtable}{|>{\centering\scriptsize}m{2em}|>{\centering\scriptsize}m{1.3em}|>{\centering}m{8.8em}|}
  % \caption{秦王政}\
  \toprule
  \SimHei \normalsize 年数 & \SimHei \scriptsize 公元 & \SimHei 大事件 \tabularnewline
  % \midrule
  \endfirsthead
  \toprule
  \SimHei \normalsize 年数 & \SimHei \scriptsize 公元 & \SimHei 大事件 \tabularnewline
  \midrule
  \endhead
  \midrule
  元年 & 349 & \tabularnewline
  \bottomrule
\end{longtable}


%%% Local Variables:
%%% mode: latex
%%% TeX-engine: xetex
%%% TeX-master: "../../Main"
%%% End:

%% -*- coding: utf-8 -*-
%% Time-stamp: <Chen Wang: 2021-11-01 11:55:43>

\subsection{义阳王石鑒\tiny(349-350)}

\subsubsection{少帝石世生平}

石世(339年-349年),字元安,十六國時期後趙國君主,後世稱「少帝」,為石虎之子。母為前趙帝劉曜幼女安定公主,後趙太和二年(329年),前趙被後趙所滅,石虎將年僅12歲的安定公主強占為妾,十年後安定公主生了石世。石虎在天王位時,石世被封為齊公,安定公主封為昭儀。

後趙建武十三年(348年),石虎廢殺了太子石宣之後,受石世之母昭儀劉氏及她的死黨將軍張豺的教唆鼓動,將劉氏立為皇后,年僅10歲的石世立為太子。次年(349年),石虎正式稱帝,並改元太寧。不久,石虎去世,石世遂即帝位,然而大權皆握在劉太后及張豺之手。

彭城王石遵得知石虎去世後,立即率軍攻回都城鄴城(今河北臨漳縣),殺張豺。數日後,石遵自即帝位,石世被改封為譙王,劉太后被廢為太妃,石世在位僅33日。不久,石世與劉太妃皆被殺。

\subsubsection{彭城王石遵生平}

石遵(?-349年),字大祗,十六國時期後趙皇帝,為石虎第九子,石世之兄,母為鄭櫻桃。後趙建平三年(333年),後趙帝石勒去世,石虎掌控大權,石遵當時被封為齊王。建武三年(337年),石虎改稱天王後,被降封為彭城公。太寧元年(349年),石虎稱帝後,再被進封為彭城王。

太尉张举曾建议石虎立石遵或燕公石斌為太子,然而因昭儀劉氏及戎昭将军張豺從中作梗,石虎遂立劉氏之子石世為太子。太寧元年(349年),石虎病重,石遵被任命為大將軍,鎮守關右。石遵从幽州来朝,被打发走,石虎知道后说“恨不见之”。

不久,石虎去世,石世即位,大權握於劉太后及張豺之手。石遵與姚弋仲、蒲洪、石閔等人商量後決定反擊,遂以石閔為前鋒,攻打都城鄴(今河北臨漳縣),不久,鄴城陷,劉太后不得已只好任命石遵為丞相、领大司马、大都督中外诸军、录尚书事,加黄钺、九锡,增封十郡。數日後,石遵假刘太后令廢石世、立石遵为帝,假装再三辞让后在群臣劝进下自登帝位于太武前殿。封石世为谯王,邑万户,待以不臣之礼,废刘太后为太妃,不久皆杀之。石遵兄沛王石冲讨伐石遵,石遵派将军王擢骑马以书信说和不成,派石闵、司空李农击败石冲于平棘,在元氏俘获石冲并赐死。

石遵可能没有儿子,當初在謀反前,曾答應事成後以石閔為太子,可是等到石遵登帝位後,太子卻是石遵之姪石衍,因此石閔頗為不滿,有反叛之意。經過旁人提醒,石遵遂召其兄石鑒、弟石苞與母親鄭櫻桃等人商議,不料會後卻被石鑒出賣,將此事告知石閔。不久,石閔即率軍入宮,派将军苏彦、周成率领披甲士兵三千人去南台的如意观抓石遵。石遵正在和女人弹棋,问周成:“造反的是谁?”周成说:“义阳王石鉴当立。”石遵说:“我尚且如此,石鉴又能支撑多长时间!”被殺,在位僅183日。

\subsubsection{义阳王石鑒生平}

石鑒(?-350年),字大郎,一作大朗,十六國時期後趙國君主,為石虎第三子,石遵、石世之兄。後趙建平三年(333年),後趙帝石勒去世,石虎掌控大權,石鑒當時被封為代王。建武三年(337年),石虎改稱天王後,被降封為義陽公。

建武五年(339年)九月,东晋征西将军庾亮镇武昌,让豫州刺史毛宝、西阳太守樊峻以一万精兵戍守邾城。石虎厌恶晋军如此动向,以夔安为大都督,率石鉴、养孙石闵、李农、张贺度、李菟五将军及兵五万人攻打荆、扬北境,以二万骑攻邾城。张贺度攻陷邾城,杀死六千人,又败毛宝于邾西,杀死万余人。赵军进犯江夏、义阳,毛宝、樊峻及东晋义阳太守郑进皆死。夔安等进围石城,被竟陵太守李阳所破才退兵。

太寧元年(349年),石虎稱帝後,再被進封為義陽王。

石鑒在鎮守關中的時候,賦役繁重,文武官員只要頭髮長得比較長,就會被拔下來做帽帶,有剩下的會給宮女,曾因為這種荒唐的行徑,被石虎召回都城鄴城(今河北臨漳縣)。

太寧元年(349年),石遵廢皇帝石世,自登帝位,石鑒被命為侍中、太傅。石遵因石閔有叛變之意,召两位兄弟石鑒、乐平王石苞與太后鄭櫻桃等人商議,不料會後石鑒出賣其他人,將此事告知石閔。不久,石閔即率軍入宮,殺石遵,石鑒因此被擁立為帝。石遵被杀时说:“我尚且如此,石鉴能长久吗?”

然而石鑒登位後,處處受制於大將軍石閔,於是派石苞和将军李松、张才暗殺之,然而卻事敗,他装作自己不知情,杀死石苞三人;后又鼓励将军孙伏都攻打石闵,不果,又对石闵说孙伏都谋反,命石闵讨灭。石閔知道石鑒有殺己之意,遂頒殺胡令,被殺的人共有20餘萬;并软禁石鉴于御龙观,派尚书王简、少府王郁率数千人看守,用绳子把食物吊给他。

次年(350年),完全控制國政的石閔將後趙國號改為魏(衛),石閔也将包括自己在内的后赵皇族改姓为李,並改年號為青龍。不久,石鑒為求擺脫控制,遂趁李閔外出作戰,秘密派宦官告知在外的將軍抚军将军张沈等,命他们趁虛攻都城鄴城,但宦官告知李閔此事,李閔因而回軍,石鑒遂被誅殺,在位僅103日。

\subsubsection{青龙}

\begin{longtable}{|>{\centering\scriptsize}m{2em}|>{\centering\scriptsize}m{1.3em}|>{\centering}m{8.8em}|}
  % \caption{秦王政}\
  \toprule
  \SimHei \normalsize 年数 & \SimHei \scriptsize 公元 & \SimHei 大事件 \tabularnewline
  % \midrule
  \endfirsthead
  \toprule
  \SimHei \normalsize 年数 & \SimHei \scriptsize 公元 & \SimHei 大事件 \tabularnewline
  \midrule
  \endhead
  \midrule
  元年 & 350 & \tabularnewline
  \bottomrule
\end{longtable}


%%% Local Variables:
%%% mode: latex
%%% TeX-engine: xetex
%%% TeX-master: "../../Main"
%%% End:

%% -*- coding: utf-8 -*-
%% Time-stamp: <Chen Wang: 2021-11-01 11:55:58>

\subsection{新兴王石祗\tiny(350-351)}

\subsubsection{生平}

石祗(?-351年),中國五胡十六國時代中,後趙的皇帝。为石虎子。

石祗早年经历不详。初封新兴王。大将军武德王石闵掌权后,开始杀戮羯族,羯族人纷纷出逃投奔石祗。350年石祗听说其兄皇帝石鉴被冉闵(即石闵,恢复本姓)杀死,于是在襄国(今河北省邢台市)自立为帝,并起兵讨伐冉闵。351年二月自去帝号,称赵王,以求获得前燕支持助讨冉闵。四月,战败被部将刘显杀死,后赵灭亡。

\subsubsection{永宁}

\begin{longtable}{|>{\centering\scriptsize}m{2em}|>{\centering\scriptsize}m{1.3em}|>{\centering}m{8.8em}|}
  % \caption{秦王政}\
  \toprule
  \SimHei \normalsize 年数 & \SimHei \scriptsize 公元 & \SimHei 大事件 \tabularnewline
  % \midrule
  \endfirsthead
  \toprule
  \SimHei \normalsize 年数 & \SimHei \scriptsize 公元 & \SimHei 大事件 \tabularnewline
  \midrule
  \endhead
  \midrule
  元年 & 350 & \tabularnewline\hline
  一年 & 351 & \tabularnewline
  \bottomrule
\end{longtable}


%%% Local Variables:
%%% mode: latex
%%% TeX-engine: xetex
%%% TeX-master: "../../Main"
%%% End:



%%% Local Variables:
%%% mode: latex
%%% TeX-engine: xetex
%%% TeX-master: "../../Main"
%%% End:

%% -*- coding: utf-8 -*-
%% Time-stamp: <Chen Wang: 2019-12-19 09:47:53>


\section{前燕\tiny(337-370)}

\subsection{简介}

前燕(337年 - 370年)是十六國时代由鮮卑人首領慕容皝所建立的政權,至慕容儁正式稱帝建國,其國號為「燕」。其全盛时的统治地区包括冀州、兖州、青州、并州、豫州、徐州、幽州等部分。

以其所在地为战国时燕国旧地,故国号为“燕”。《十六国春秋》始用“前燕”之名,為區別同期的慕容氏諸燕,歷史學家遂袭用之。又以其王室姓慕容,又称为“慕容燕”,而其他慕容氏諸燕都不用这个称呼,「慕容燕」成为前燕的专称。

西晉時,慕容廆為鮮卑族慕容氏的首領,曾效忠西晉,與鮮卑族以外的民族作戰。後來其兒子慕容皝於337年自稱為燕王。342年擊敗了後趙的二十萬大軍,解除了來自中原的壓力,建都龍城(今遼寧省朝陽市)。東破夫餘及高句麗,攻滅鮮卑宇文部,成為遼西唯一的武裝勢力,為慕容儁攻占中原奠定了坚实的基礎。

352年,皝子慕容儁滅冉魏稱帝,遷都薊,并隨後的幾年平定了北方的局勢,於357年遷都鄴。其地「南至汝颍,東盡青齊,西抵崤黽,北守雲中」,與關中的前秦平分黃河流域。358年,慕容儁下令全國州郡檢查戶口,每戶僅留一丁,此外全部徵發當兵,擬拼集150萬大軍以滅東晉、前秦以統一天下。

360年正月,慕容儁在鄴檢閱軍隊,但隨即逝世。其子慕容暐即位,改元「建熙」,此後宮廷裡發生了一次內訌。

慕容儁死時命弟慕容恪輔政,後慕容恪阻止了宮廷的內訌。從360年慕容恪輔政到367年病死其間,為前燕政治較為穩定的時期。自建熙二年、東晉升平五年(361年),以前燕河內太守呂護倒戈反覆为导火索,前燕与东晋在中原展开了连绵的战事。

前燕建熙四年、東晉興寧元年(公元363年)前燕全面的攻势开始发动,四月,慕容忠攻滎陽(今河南滎陽東北),東晉滎陽太守逃到魯陽(今河南魯山)。建熙五年、東晉興寧二年(公元364年)二月,前燕李洪開始略地河南,四月,前燕攻許昌、汝南、陳郡,徙上述三地萬餘戶於幽州,遣鎮南將軍慕容塵屯許昌。七月,太宰慕容恪親自領兵攻打洛陽,東晉洛陽守軍战败逃离。建熙六年、東晉興寧三年(公元365年)三月,前燕攻克洛陽。

这一系列的战役后,前燕从东晋手中获得了中原的控制权。但东晋仍然不放弃收复失地的计划。东晋太和四年(公元369年)东晋大将桓温北伐,前燕一度陷入危机。不久,东晋军中绝粮,桓温被迫撤退,途中遭前燕军队伏击,损失三万余人,大败而归。

太宰慕容恪死後,太輔慕容评掌握大權。慕容评為人貪婪,並得太后所信任,得以掌握大權。

東晉乘慕容恪死時,由大司馬桓溫領兵北上,被慕容皝的兒子、少帝慕容暐之叔吳王慕容垂击败,慕容垂却被掌权者慕容评所猜忌。慕容垂被逼无奈,出走前秦,被苻坚收留。苻坚早就想消灭前燕,一直忌惮慕容垂,如今最大劲敌已经投誠,苻坚遂开始讨伐前燕的计划。前燕军团起初并未处于下风,但由于當權的慕容评为人贪鄙,致使军心离散,结果前燕15万主力部队被王猛所率领的前秦军歼灭。苻坚趁势率10万军队包围前燕的首都邺城。“散骑侍郎徐蔚等率扶余、高句丽及上党质子五百余人,夜开城门以纳坚军。”公元370年十一月,慕容暐逃出邺城,试图返回辽东的根据地龙城,中途被前秦军抓获,前燕灭亡。

\subsection{武宣帝生平}

慕容\xpinyin*{廆}(269年-333年6月4日),字弈洛瓌,昌黎棘城(今遼寧義縣)人。晉朝時鮮卑人,慕容部首領慕容涉歸之子,前燕建立者慕容皝之父,吐谷渾第一代首領慕容吐谷渾是其庶兄。

慕容廆年紀輕輕就已經長得魁梧高大,才能出眾且有雄大抱負。張華出鎮北方時慕容廆曾去拜訪他,雖然當時慕容廆仍是兒童,但張華卻十分欣賞他,更與他結交。

西晉武帝太康四年(283年)慕容涉歸死,其弟慕容删篡奪政權,更意圖殺害慕容廆,慕容廆於是投奔躲藏於遼東郡人徐郁家。至太康六年(285年),慕容删被其部下所殺,其部眾於是迎慕容廆繼位。

由于宇文部鲜卑和慕容涉歸有仇,慕容廆繼位後就请求晋朝政府允许其出兵討伐,但遭到拒绝。慕容廆于是反叛晋朝,出兵劫掠遼西郡,令當地傷亡和財物損失都十分嚴重。不久雖然為晋軍所敗,但仍常常侵掠昌黎郡,又進攻東邊的扶餘國,逼死其王依虑,更毀滅了扶餘國都。晋东夷校尉派部將贾沈援助扶餘,助扶餘王子依羅復國,當時慕容廆派兵截擊但被擊敗,扶餘亦成功復國。

慕容廆後自以先世世代臣服於中原王朝,而且力量懸殊,不能與當時是統一王朝的晉朝爭鋒,又稱不能因與晉朝不和而令當地百姓受戰禍之苦,遂於太康十年(289年)重新归顺晋朝。晉廷受降並封慕容廆為鮮卑都督。慕容廆當時就去東夷校尉府拜見東夷校尉何龕,以士大夫禮,穿著巾衣前去;但到後見何龕嚴兵以待,慕容廆於是改穿戎服,又稱主人不以禮待客,客亦不以禮相待。何龕聽聞慕容廆這樣說,慚愧之餘亦敬重他。

慕容廆又因當時勢弱和聲威日上而受宇文部鮮卑與段部鮮卑不斷侵擾,採取忍讓政策,以卑下的言辭和大量金錢去討好對方,段部鮮卑酋長段階於是將女兒下嫁慕容廆。慕容廆認為遼東郡過於僻遠,遂向西遷徙至徒河縣(今遼寧省錦州市)境的青山(今遼寧省義縣東)。元康四年(西元294年),慕容廆遷居大棘城(今遼寧省義縣西)。慕容廆又於勢力範圍內推廣農桑,並且施行與晉朝一樣的法制。至永寧二年(302年)兗、豫、徐、冀四州發生水災,鄰近冀州的幽州亦受影響,慕容廆則開倉賑災,助幽州人民渡過困境。

太安元年(西元302年),宇文部鮮卑酋長宇文莫圭命其弟宇文屈雲率軍進攻慕容廆,慕容廆避其主力反擊重創其別部將領宇文素怒延;宇文素怒延因羞憤而動員十萬人包圍慕容廆所在的大棘城,當時城內部眾都十分恐懼,沒有抵抗的意志,然而慕容廆稱這是在其計劃之中,勉勵部眾作戰,並親自領軍出擊,再度重創宇文素怒延兵團,追擊一百華里並俘虜及斬殺近萬人。原在宇文部下的遼東郡人孟暉率眾數千家歸降慕容廆,慕容廆任命孟暉當建威將軍。

永嘉元年(307年),慕容廆自稱鮮卑大單于。永嘉三年(309年)遼東郡太守龐本因私怨而殺害東夷校尉李臻。當地的附塞鮮卑素喜連和木丸津以為李臻報仇為名起兵,但卻沒有因新任東夷校尉封釋設計殺死龐本而罷兵,竟乘機攻略遼東郡中諸縣。當地晉兵更屢次兵敗。亂事持續了兩年,封釋已經無力再戰,請和但不果。而遼東百姓期間大多因戰火而投靠慕容廆。永嘉五年(311年),慕容廆面對這個情況,接納兒子慕容翰的建議,起兵討伐素喜連等人,將兩人殺害並吞併其部眾,將他們所掠的三千多家人及早前歸附自己的遼東郡人送還本郡,保全了遼東郡。

永嘉之亂後,大司馬王浚承制假立太子,並以慕容廆為散騎常侍、冠軍將軍、前鋒大都督、大單于,但慕容廆以不是王命所授而拒絕。及至建武元年(317年),時為晉王的晉元帝司馬睿承制拜慕容廆為假節、散騎常侍、都督遼左雜夷流人諸軍事、龍驤將軍、大單于、昌黎公。但慕容廆辭讓。當時魯昌勸說慕容廆支持司馬睿為帝,並以司馬睿晉朝正統之名討伐其他擁兵的鮮卑部落。慕容廆接納並命人循海路到建康勸進。至次年司馬睿即位為帝,再次要授予上一年慕容廆拒絕的官位,慕容廆這次就接受昌黎公以外的職位。

當時慕容廆政事修明,愛護人才,在北方紛亂的環境下,士大夫和民眾多歸附之,好像永嘉五年(311年)東夷校尉封釋病死前就托付孫兒封奕給慕容廆,其子封悛和封抽前來奔喪後因道路不通而不能返回,亦願留在當地,被慕容廆任命為長史和參軍。建興元年(313年),据有乐浪、带方二郡的张统因不堪长期孤军与高句丽作战而率千余家投靠慕容廆,慕容廆为其在侨置乐浪郡。為著管理大批的流人,他為冀州人設冀陽郡、豫州人設成周郡、青州人設營丘郡、并州人設唐國郡。同時又任用大批漢人賢才去處理政事和作自己的參謀。同時又推行儒學,除了讓世子慕容皝受學以外,自己在有餘暇時也會去聽講,故此令他統領的地方到處都有頌讚之聲,守禮謙讓之風亦流行。

但慕容廆如此受流徙當地的漢人支持,受到出身清河崔氏的平州刺史、东夷校尉崔毖的妒忌。崔毖曾數度遣使招請慕容廆前去但都不果,於是打算以武力拘禁他。崔毖於太興二年(319年)成功游说宇文部鮮卑、段部鲜卑和高句丽联合攻伐慕容廆,並許約戰後瓜分其領地。

當時面對三國聯軍來攻,慕容廆拒絕諸將出擊,認為他們新聚而銳不可擋,反而應該固守去令他們漸漸互相猜忌,待人心離異後才一舉擊破。及後三國聯軍包圍棘城,慕容廆閉門自守,卻特意送牛酒去宇文部那裏勞軍,以離間計挑起其餘兩國對宇文部的懷疑,最終令兩國各自率軍離去。但當時宇文部大人宇文悉獨官自以兵強,仍然留下進攻棘城。面對當時宇文部數十萬兵力,連營四十里的軍勢,慕容廆打算召留守徒河的兒子慕容翰入援,然而慕容翰卻認為棘城守軍足以守城,派使者向父親表示自己應該作為奇兵伺機突襲,配合城中守軍出擊就能夠大破對手;若自己也進去守城,那宇文部就能專心攻城,而且更示以部眾勢弱,將會削弱士氣。慕容廆在韓壽的進言下接受慕容翰的建言,不再召他回防。而此時宇文悉獨官亦聽聞慕容翰沒有入援棘城,擔憂不久成為後方大患,於是分兵先行消滅慕容翰。但慕容翰則設計打敗來攻的軍隊,更乘勝進攻宇文部大軍,慕容廆在接到慕容翰的消息後亦從城內出兵,成功大敗宇文部。

戰後,三國都遣使請和,崔毖亦因畏懼而派侄兒崔燾前來假意祝賀,以消對方對自己的怨恨。慕容廆卻借由崔燾傳話,要崔毖投降或出走,崔毖最終出奔高句麗,慕容廆就併吞其部眾。同時,主簿宋該亦勸慕容廆向東晉獻捷報,慕容廆於是命其作表,由長史裴嶷出使,同時將大敗宇文部時獲得的三顆印璽送呈建康。明年,裴嶷到建康時盛讚慕容廆,晉元帝於是拜慕容廆為安北將軍、平州刺史。太興四年(321年)再升慕容廆為都督幽、平二州及東夷諸軍事、車騎將軍、平州牧,封遼東郡公,賜丹書鐵券,允許他承制選置平州官員。

太寧元年(323年),後趙王石勒派使者來與慕容廆結好,但慕容廆卻收捕使者並押送到建康。石勒知道後大怒,於太寧三年(325年)命宇文乞得歸進攻慕容廆,卻被慕容廆所派去抵抗的軍隊擊敗,慕容仁等更乘勝攻破宇文部國都並掠奪其大量物資和人馬。

後來,慕容廆與太尉陶侃通信,稱讚王導和庾亮,並稱陶侃是「海內之望中唯足為楚漢輕重者」,表示願意為復興晉朝作出努力,只是礙於自己孤軍進攻難有成果,期待東晉大舉北伐時響應。同時還附著封抽、韓矯等建議封慕容廆為燕王、行大將軍事的上疏。陶侃將封抽的上疏報告朝廷,讓朝議定奪。咸和八年五月甲寅日(333年6月4日),慕容廆去世,享年六十五歲,當時朝議仍未有定論,知道慕容廆去世後就停止了。東晉遣使贈慕容廆大將軍、開府儀同三司,諡號為襄。咸康三年(337年)慕容皝自稱燕王時追諡為武宣王。至永和八年(352年)慕容廆孫慕容儁稱帝時,追諡為武宣皇帝。

\subsection{文明帝生平}

燕文明帝慕容\xpinyin*{皝}(297年-348年10月25日),字元真,小字万年,昌黎棘城(今遼寧義縣)鲜卑族人。中國五胡十六國時代前燕的開國君主,不過當時仍名義上臣屬於東晉,直至其子慕容儁正式稱帝後,才追尊廟號太祖,諡號為文明皇帝。其父為慕容部落的首領、遼東公慕容廆,其母段夫人。其庶長兄為建威将军慕容翰。

慕容皝勇武剛毅且多有謀略,崇尚經學,熟悉天文。建武初年拜冠軍將軍、左賢王、封望平侯。太兴四年(321年)十二月,慕容廆封遼東郡公,立身為嫡子的慕容皝为世子。其曾率眾出征,累有戰功,如於永昌元年(322年)率眾入侵段末柸的都城令支(今河北遷安縣西)。太寧末年,慕容皝拜平北將軍,封朝鮮公。咸和八年五月甲寅(333年6月4日),慕容廆去世。六月,慕容皝嗣辽东郡公,以平北将军行平州刺史,督摄部内,统治辽东。

同年,宇文乞得歸被宇文逸豆歸逼逐而在外去世,慕容皝出兵討伐,令宇文逸豆歸畏懼請和,慕容皝於是修築了榆陰和安晉二城後回軍。慕容皝弟征虜將軍慕容仁和廣武將軍慕容昭很得慕容廆寵愛,惹來慕容皝不滿,而二人在慕容皝登位後怕慕容皝不能接納自己,於是在慕容仁在平郭(今遼寧熊岳城)舉兵西行至棘城以攻慕容皝,並以慕容昭為內應。不過,慕容仁尚在途中,其計劃就被揭發,慕容昭被慕容皝賜死,慕容仁唯有回軍據守平郭。慕容皝於是派兵讨伐,却大败于汶城以北。及後孫機更以遼東郡向慕容仁投降,令其盡得遼東之地,而且獲得段部鮮卑首領段遼和鮮卑諸部的支持,遙遙相援。

咸和九年(334年),慕容皝接連派兵攻殺鮮卑木堤和烏丸悉羅侯。段遼亦攻徒何,不能攻破後更派段蘭和在上一年因懼慕容皝猜忌而出奔段部的慕容翰進攻柳城(今遼寧朝陽市),守將石琮死守,終於退軍。同年,派往東晉報喪的隊伍回遼東,慕容皝獲東晉授予鎮軍大將軍、平州刺史、大單于、遼東公,持節,並因以往慕容廆之事,都督幽、平二州及東夷諸軍事並承制置百官,但皆被慕容仁所留,慕容皝一直至次年隊伍被放回棘城才獲受命。慕容皝亦於同年率军讨辽东,成功奪取襄平(今遼寧遼陽市),居就、新昌兩縣亦歸降,慕容皝置和陽、武次和西樂三縣就撤軍,又將遼東大姓分徒於棘城。

咸康二年(336年)正月,慕容皝堅持趁海面結冰而從海路進攻慕容仁,於是在壬午日(2月17日)自昌黎東出發,經結冰海面走三百多里,至歷林口就放下輜重輕兵直取平郭。慕容皝軍行至平郭七里以外時,慕容仁斥候才向慕容仁報告,令慕容仁狼狽到城西北迎戰。當時慕容軍率部向慕容皝投降,震動慕容仁軍心,慕容皝於是趁機進攻,大敗對方,慕容仁亦被擒和被賜死。

平定慕容仁後,至六月,段遼又派兵進攻慕容皝,分別攻擊武興以及柳城,當時宇文逸豆歸亦攻進安晉以作聲援,慕容皝別將擊破攻武興之軍,而自己率兵增援柳城,逼走屯於城西的段蘭後轉攻安晉,並派封奕大敗逃走的宇文逸豆歸部眾。及後慕容皝預料二部會再來,於是命封奕在馬兜山設伏,成攻大敗下月來攻的段遼。其後又命世子慕容儁和封奕分別進攻段部和宇文部,皆大勝。慕容又下令在乙連東築好城並置戍,又建曲水城作好城之援,以威逼乙連。當時乙連大饑,段遼命人輸送糧食,但就被戍守好城的蘭勃所敗。後段遼部將段屈雲進攻興國,又被慕容皝將慕容遵擊敗並盡俘其部眾。

咸康三年十月丁卯(337年11月23日),慕容皝聽從封奕的勸告,自称燕王,建前燕,追慕容廆为武宣王,夫人段氏为武宣后,立世子慕容儁为王太子。當年又因段部鮮卑多番入侵,於是派宋回向後趙稱藩,以其弟慕容汗為人質,請求後趙與其聯兵進攻段部鮮卑。後趙天王石虎大悅,答允並送還慕容汗,約定明年進攻。随后在咸康四年(338年),石虎率眾進攻段部鮮卑,慕容皝則出兵進掠令支以北諸城,並大敗追擊的段蘭,大掠而還。而因石虎一直進攻,四十多座城被石虎所得,段遼於是棄守令支而逃至密雲山。石虎入令支後,不滿慕容皝自掠人民牲畜後回軍,不與其會師,於是下令進攻慕容皝。後趙軍一直進攻,至五月戊子日(6月12日)攻至棘城時,慕容皝打算逃亡,但被慕輿根勸阻,當時玄菟太守劉佩更率敢死隊數百騎出城衝擊後趙軍,所向披靡,令城中士氣大增;封奕亦勸慕容皝堅守,終令慕容皝安心不降。兩軍相持十多日後,後趙軍引兵退還,慕容皝派慕容恪率二千騎進攻後趙軍,驚擾敵軍而令其棄甲潰散,殺三萬餘人。及後慕容皝分兵收復原本叛歸後趙的各個郡縣,並擴境至凡城,置戍而還。十二月,段遼降後趙,不久又悔而轉投慕容皝,而後趙已派麻秋支援段遼,慕容皝於是命慕容恪設伏於密雲山,大敗麻秋,並帶著段遼和其部眾撤還。

咸康五年(339年),慕容皝守將擊退來攻凡城的後趙軍隊,又因自稱燕王未受東晉朝命,於是命長史劉翔向建康獻捷,兼求假燕王璽綬,又請大舉出兵平定中原。不過當時朝廷議論未肯容讓慕容皝稱王。此時慕容皝得知庾亮去世,其弟庾冰及庾翼分掌朝廷中樞及荊州要地,於是上書要晉成帝以史為鑑,親近賢達,不要親信外戚。又寫信給庾冰,指責他掌握朝權,卻未能為國雪恥,只「安枕逍遙,雅談卒歲」。庾冰知道慕容皝的上表和書信後十分恐懼,自以道遠而不能控制他,於是奏請順應慕容皝的請求。咸康七年(341年),慕容皝獲東晉任命為使持節、大將軍、都督河北諸軍事、幽州牧、大單于,封燕王。

在受封燕王的同一年,慕容皝下令在柳城以北,龍山以西修建龍城,並改柳城為龍城。至次年(342年)正式遷入龍城。遷都後,慕容皝聽從早前歸國的慕容翰建議,先襲破高句麗,後才再攻取宇文鮮卑,以解後顧之憂,專心圖取中原土地。慕容皝並自率精兵四萬從險狹的南道進攻高句麗,以慕容翰及慕容垂為前鋒,以王寓領偏師五千走平廣開闊的北道引誘敵軍,終出其不意,成功攻陷高句麗都城丸都(今吉林集安),高句麗王高釗出逃,慕容皝招引不出,且因王寓敗沒而沒有追擊,於是挖出高釗父高乙弗利的屍體,連同丸都城中府庫收藏的珍寶、高釗母親和妻子及擄掠的五萬多人一同西還,更毀丸都。高句麗因而於翌年(343年)向慕容皝稱臣,慕容皝於是送還其父親屍體,留其母為人質。

高句麗稱臣於慕容皝後,慕容皝又擊敗了宇文逸豆歸派來進攻的國相莫淺渾。建元二年(344年),慕容皝親自率二萬騎兵討伐宇文鮮卑,又派慕容翰為前鋒,慕容軍、慕容恪、慕容垂及慕輿根兵分三路一同進攻。慕容翰與宇文逸豆歸大將涉奕于大戰,涉奕于戰死,宇文部軍心瓦解,被慕容皝所敗,都城紫蒙川陷落,宇文逸豆歸敗死漠北,宇文鮮卑至此被慕容皝所併。

此战后慕容皝终究不能对慕容翰放心,将其赐死。

永和元年(345年),慕容皝又派慕容恪攻高句麗,攻克南蘇並置戍而還。永和二年(346年)又命慕容儁與慕容軍、慕容恪及慕輿根率一萬七千兵東襲夫餘,成功俘虜夫餘王餘玄等五萬多人回國。

而早在咸康六年(340年),後趙已大舉徵兵,大行屯田,並收集戰馬,準備進攻慕容皝。當時慕容皝認為薊城因樂安得重兵駐守而防禦空虛,突襲薊城,守城的石光驚懼而不出擊,慕容皝攻陷高陽並焚毀積聚的軍糧,更掠奪了三萬餘戶。此舉打亂了後趙進攻計劃,而慕容皝平定高句麗和宇文部等主要對手後,前燕就能更集中對抗後趙,終令前燕得以專心在永和六年(350年)乘後趙內亂出兵中原。

永和四年九月丙申日(348年10月25日),慕容皝去世,时年五十二,諡為文明王。

永和元年(345年),慕容皝自以古時諸侯即位皆稱元年,故此不再用晉朝年號,追咸和八年(333年)登位起計,改稱十二年。

慕容皝汉化较深,崇尚儒学,设东庠(学校),以大臣子弟为官学生,号高门生。亲临讲授,每月考试优劣。

慕容皝鼓勵農耕,例如就曾在朝陽門東設籍田,置官主理。後又親自巡行各郡縣,鼓勵和督察農業活動。更加罷園林供沒有土地的農民耕種,更贈送一頭牧牛給沒有牛的農民。

慕容皝曾樹立納諫之木,以示他願意接受正直諫言。

史載慕容皝身長七尺八寸(約191厘米)。

慕容皝好文學典籍,故他勸於到東庠講授,學生多達千餘人。慕容皝更親作《太上章》以取代《急就篇》作學生識字的書籍,又寫了《典誡》共十五篇,皆用來教授學生。

《晉書》載慕容皝一次在國境西邊畋獵,將渡河時見一個騎白馬,穿紅衣的老人,舉手指揮著慕容皝,說該處不是狩獵場,要慕容皝離開。不過慕容皝沒有將事件說出來,更渡河狩獵,接連幾日大有收獲。慕容皝及後見到一隻白兔,於是策馬追射,但馬匹卻跌倒,慕容皝亦墮馬受傷,這時才說出他看見老人一事。慕容皝回龍城後將後事託付給世子慕容儁,後就死去了。王隱《晉書》亦有相近記載,不過是老人說話後就不見了,而後追獵白兔時墮馬撼石,當場死亡。


%% -*- coding: utf-8 -*-
%% Time-stamp: <Chen Wang: 2021-11-01 11:56:21>

\subsection{景昭帝慕容儁\tiny(348-359)}

\subsubsection{生平}

燕景昭帝慕容\xpinyin*{儁}(319年-360年2月23日),字宣英,鮮卑名賀賴跋,昌黎棘城(今遼寧義縣)鲜卑人,五胡十六國時代前燕的君主。前燕文明帝慕容皝次子。慕容儁即位時仍名義上為東晉的燕王,然而於永和八年(352年)正式稱帝獨立。慕容儁在位期間消滅了冉魏,入據原本由後趙所佔領的中原地區,勢力大增,並移都鄴城,終與南方的東晉和關中的前秦政權三足鼎立。

慕容儁博覽群書,有文武才幹,曾領兵攻略段部鮮卑並大勝而還。咸康七年(341年),東晉封慕容皝為燕王,亦以慕容儁為燕王世子,假節、安北將軍、東夷校尉、左賢王。

永和四年九月丙申日(348年10月25日),慕容皝去世。十一月甲辰日(349年1月1日),太子慕容儁繼襲燕王爵位。派使臣到建康向東晉報告了喪事。他還任命弟弟慕容友為左賢王,任命左長史陽鶩為郎中令。次年(349年)稱元年,仍不用東晉年號。同年後趙皇帝石虎去世,諸子爭位令國內大亂,慕容儁圖謀奪取中原土地,於是以慕容垂為前鋒都督、建鋒將軍,另外任命慕容恪為輔國將軍、慕容評為輔弼將軍和陽騖為輔義將軍,人稱三輔。挑選了二十多萬精兵等待時機。而同年東晉朝廷亦任命慕容儁為使持節、侍中、大都督、都督河北諸軍事、幽冀并平四州牧、大將軍、大單于、燕王,並依慕容廆和慕容皝的先例能承制封拜官員,在東晉授命下正式繼承了對遼東的管治。

永和六年(350年),後趙大將軍冉閔在鄴城稱帝,慕容儁亦乘機兵分三路南攻,自己親自率中軍出兵盧龍,攻下了薊城,並遷都至薊。因慕容儁聽從慕容垂不要坑殺薊城士卒的勸言,故得中原士民歸附。其他幽州郡縣多亦奪取,慕容儁於是設置幽州諸郡縣的官員。後慕容儁意圖進攻後趙幽州刺史王午和征東將軍鄧恆所守的魯口,不過被其將鹿勃早夜襲,雖然最終成功擊退對方,不過軍隊鋒銳已因這次突襲而受挫,只得暫緩戰事,返回薊城。不久代郡人趙榼率三百餘家叛歸後趙,慕容儁於是遷廣寧、上谷二郡人到徐無,代郡人到凡城,以防其再次叛歸後趙。不過,慕容儁亦南攻冀州,攻下了章武、河間二郡。

另一方面,守襄國的後趙皇帝石祗自永和六年起就被冉閔所圍攻。圍困百多日後,石祗被逼於永和七年(351年)向前燕求援,並許以傳國璽作交換。慕容儁欲得傳國璽,於是相信了後趙並派了悅綰救援襄國。同年冉閔被擊敗,襄國之圍解除,但悅綰沒有獲得傳國璽,慕容儁於是殺掉當日前來求援的後趙太尉張舉。慕容儁又派兵奪取中山和趙郡,又進攻魯口,擊敗王午派來迎擊的軍隊。

永和八年(352年),前燕王慕容儁派廣威將軍慕容軍、殿中將軍慕輿根、右司馬皇甫真等人率二萬人步、騎兵協助慕容評攻打冉魏鄴城。 永和八年(352年),冉閔攻陷襄國,將殺後趙皇帝石祗的將領劉顯勢力消滅。同年四月甲子日(5月5日),慕容儁命慕容恪等攻伐冉魏,最終擊敗冉閔並將其俘虜。己卯日(5月20日),冉閔被押送到薊城,慕容儁指責冉閔:「你只是配當奴僕的低下才幹,憑甚麼去稱帝?」冉閔卻說:「天下大亂,你這些夷狄禽獸都能稱帝,那我這種中土英雄,怎能不稱帝呀!」慕容儁聽後大怒,鞭打他三百下並送到龍城處死。同時,先前叛燕的段勤既受慕容垂進攻據地繹幕,看見慕容恪進據常山後就因畏懼而請降。

甲申日(5月25日),慕容儁命慕容評等進攻鄴城,冉魏太子冉智與將領蔣幹閉城門自守,得晉將戴施率百餘人入鄴助守,並以傳國璽向東晉請糧。不過,慕容評終於八月庚午日(9月8日)攻下鄴城,俘冉智等人至中山。冉魏亡後,當時擁兵據守州郡的後趙官員都派使者向前燕請降。

攻下鄴城後,慕容儁假稱冉閔皇后董氏獻傳國璽予他,賜董氏號「奉璽君」。十一月丁卯日(353年1月3日),慕容儁置百官,次日即位為皇帝,改年號為「元璽」,追尊慕容廆和慕容皝為皇帝並上廟號。當時東晉使者到了前燕,慕容儁就對他說:「你回去告訴你的天子,中原無主,我被士民推舉為主,已經做了皇帝了!」

前燕南侵幽州時據守魯口的王午在永和八年(352年)自稱安國王,同年被殺,由呂護承襲稱號並繼續據守魯口。永和九年(353年),衛將軍慕容恪、撫軍將軍慕容軍、左將軍慕容彪等人屢次薦舉給事黃門侍郎慕容霸,說他有顯赫於世之才,應總攬重任。前燕皇帝慕容儁任命慕容霸為使持節、安東將軍、北冀州刺史、鎮守常山。永和九年(353年),慕容儁派慕容恪進兵討伐,終令呂護於永和十年(354年)歸降。後慕容儁又命慕容恪鎮守洛水,以慕容強為前鋒都督,進據黃河以南地方。永和十年(354年),慕容儁封弟弟慕容恪為太原王,慕容評為上庸王,封左將軍慕容彭為武昌王,封撫軍將軍慕容軍為襄陽王,封安東將軍慕容霸為吳王,左賢王慕容友為范陽王,散騎常侍慕容厲為下邳王,散騎常侍慕容宜為廬江王,寧北將軍慕容度為樂浪王。慕容桓為宜都王,慕容遵為臨賀王,慕容徽為河間王,慕容龍為歷陽王,慕容納為北海王,慕容秀為蘭陵王,慕容岳為安豐王,慕容德為梁公,慕容默為始安公,慕容僂為南康公。兒子慕容臧為樂安王,慕容亮為勃海王,慕容溫為帶方王,慕容涉為漁陽王,慕容暐為中山王。

永和十一年(355年),東晉蘭陵太守孫黑、濟北太守高柱、建興太守高甕及前秦河內太守王會、黎陽太守韓高都以所在郡投降前燕。而先前屯據蕕城,歸降前秦的前車騎將軍劉寧亦率二千戶人到薊城歸降請罪,慕容儁亦任命劉寧為後將軍。高句麗王高釗亦向前燕進貢。同年,據守廣固並向東晉稱藩的段龕寫信非議慕容儁稱帝之事,觸怒了慕容儁並令他派了慕容恪進討段龕,終於在次年攻陷廣固,俘虜了段龕。升平元年(357年),慕容儁又命慕容垂等率八萬兵到塞北進攻丁零敕勒,大敗對方並俘殺十多萬人,奪去十三萬匹馬和億萬頭牛羊。及後匈奴單于賀賴頭率部歸降前燕。

升平元年十一月癸酉日(357年12月14日),慕容儁遷都鄴城。升平二年(358年),東晉泰山太守諸葛攸進攻東郡,被慕容恪等擊敗,慕容恪更乘機掠奪河南土地。不久東晉北中郎將荀羨攻陷山茌,處死太守賈堅,亦被前燕軍隊擊敗並收復失地。升平三年(359年)諸葛攸再攻前燕,在東阿被慕容評等人擊敗。同年十月,東晉西中郎將謝萬與北中郎將郗曇北伐,但因郗曇因病退兵以及謝萬統率失誤而令軍隊驚潰敗退,前燕得以乘機奪取許昌、穎川、譙及沛諸郡各城。

另一方面,前秦平州刺史劉特率眾向前燕投降。慕容儁又於升平二年(358年)派了司徒慕容評等人進攻盤據并州自立的將領張平、李歷等,令張平的部下諸葛驤、蘇象等率當地一百三十八個壁壘歸降前燕。及後張平等先後出奔,前燕於是收降了其部眾。

此時前燕正与东晋、前秦形成三足鼎立之势,并且在当时是国力最强的。

升平二年(358年),慕容儁因於擴張領土的戰爭中屢次獲勝,於是更圖謀消滅東晉以及前秦。為此下令州郡核實男丁數目,每戶只留下一個男丁,其餘都被徴為士兵,務求令全國步兵達至一百五十萬人。慕容儁更命士兵於明年就要集合,並攻取洛陽。在劉貴的諫止下,慕容儁才與官員議論,最終改為「三五占兵」,並將集合期限寬貸至一年後,定於下一年冬季於鄴城集合。

不過慕容儁於升平三年(359年)就患病,他向弟弟慕容恪表示他擔心自己一病不起,而前秦和東晉尚未滅亡,憂心皇太子慕容暐未有足夠能力治理國家,於是打算仿效宋宣公,以慕容恪繼位。不過慕容恪堅決拒絕,更矢言會輔助慕容暐。升平四年(360年)正月,慕容儁於鄴城閱兵後不久就於當月甲午日(2月23日)病死,臨終遺命大司馬太原王慕容恪、司徒上庸王慕容評、司空陽騖、領軍將軍慕輿根為輔政大臣,虚龄四十二,諡為景昭皇帝,廟號烈祖。

慕容儁于建熙元年三月葬于龙城(今辽宁省朝阳市)的龙陵(具体方位不详)。

慕容儁長子,獻懷太子慕容曄於永和十二年(356年)去世,慕容儁對此十分傷心。一次慕容儁在蒲池與群臣飲宴,因為談及東周時周靈王的太子晉,竟流下淚來,更表示自己在慕容曄死後「鬚髮中白」,更明白為何曹操和孫權昔日要分別為兒子曹沖和孫登早逝而痛惜不已。

慕容儁喜好文學典籍,即位以來都講論不斷,處理政務以外都是和侍臣交流典籍的義理,更有四十多篇著述。慕容儁亦於顯賢里設小學教育冑子。

慕容儁曾夢見石虎咬他的手臂,令慕容儁十分厭惡,於是下令挖開石虎的墓穴,罵道:「死胡竟然敢夢中嚇天子!」於是命御史中尉陽約數其殘酷之罪,鞭屍後丟到漳水去。《資治通鑑》更謂慕容儁在石虎墓找不到石虎屍首,於是懸賞百金求屍;後因鄴城女子李菟報告,在東明觀找到石虎屍首,發現他竟僵硬不腐;石虎屍首被投進漳水後,更靠在柱邊不流走。

\subsubsection{元玺}

\begin{longtable}{|>{\centering\scriptsize}m{2em}|>{\centering\scriptsize}m{1.3em}|>{\centering}m{8.8em}|}
  % \caption{秦王政}\
  \toprule
  \SimHei \normalsize 年数 & \SimHei \scriptsize 公元 & \SimHei 大事件 \tabularnewline
  % \midrule
  \endfirsthead
  \toprule
  \SimHei \normalsize 年数 & \SimHei \scriptsize 公元 & \SimHei 大事件 \tabularnewline
  \midrule
  \endhead
  \midrule
  元年 & 352 & \tabularnewline\hline
  二年 & 353 & \tabularnewline\hline
  三年 & 354 & \tabularnewline\hline
  四年 & 355 & \tabularnewline\hline
  五年 & 356 & \tabularnewline\hline
  六年 & 357 & \tabularnewline
  \bottomrule
\end{longtable}

\subsubsection{光寿}

\begin{longtable}{|>{\centering\scriptsize}m{2em}|>{\centering\scriptsize}m{1.3em}|>{\centering}m{8.8em}|}
  % \caption{秦王政}\
  \toprule
  \SimHei \normalsize 年数 & \SimHei \scriptsize 公元 & \SimHei 大事件 \tabularnewline
  % \midrule
  \endfirsthead
  \toprule
  \SimHei \normalsize 年数 & \SimHei \scriptsize 公元 & \SimHei 大事件 \tabularnewline
  \midrule
  \endhead
  \midrule
  元年 & 357 & \tabularnewline\hline
  二年 & 358 & \tabularnewline\hline
  三年 & 359 & \tabularnewline
  \bottomrule
\end{longtable}


%%% Local Variables:
%%% mode: latex
%%% TeX-engine: xetex
%%% TeX-master: "../../Main"
%%% End:

%% -*- coding: utf-8 -*-
%% Time-stamp: <Chen Wang: 2019-12-19 09:51:40>

\subsection{幽帝\tiny(360-370)}

\subsubsection{生平}

燕幽帝慕容\xpinyin*{暐}(350年-384年),字景茂,昌黎棘城(今遼寧義縣)鮮卑人。五胡十六國時代前燕的最後一位君主,前燕景昭帝慕容儁第三子。前期在慕容恪攝政之下仍能保持國家穩定,但後期在慕容評主政之下就漸漸衰落,最終被前秦所滅。慕容暐在前燕亡後成為前秦的臣下,獲封為新興侯。前秦於淝水之戰後崩潰,慕容垂、慕容泓先後舉兵建立「後燕」和「西燕」,慕容暐亦在西燕進攻前秦都城長安(今陝西西安)時意圖殺死苻堅並令城內混亂,響應外軍,但失敗被殺。

慕容暐最初獲封中山王。永和十二年(356年),皇太子慕容曄去世,慕容儁於是在次年立八歲的慕容暐為皇太子。升平四年(360年),慕容儁去世,臨終時遺命大司馬慕容恪、司空陽騖、司徒慕容評及領軍將軍慕輿根輔政。當時群臣打算立作為慕容儁弟弟的慕容恪繼位,但被慕容恪拒絕,而支持作為儲君的慕容暐即位。

慕容暐即位後便以慕容恪為太宰,讓他專攝朝政,而慕容評、陽騖和慕輿根則分別獲授太傅、太保及太師,參輔朝政。不過,當時慕輿根就自恃自己屢有戰功,顯得高傲自大,心中不服慕容恪。當時慕輿根打算作亂,初以可足渾太后干政煽動慕容恪謀反失敗,於是改向可足渾太后及慕容暐中傷慕容恪,想要他們誅殺慕容恪及慕容評。不過此時慕容暐卻信任慕容恪,勸止打算聽從的可足渾太后。及後慕容恪及慕容評密奏慕輿根罪狀,慕容暐於是命侍中皇甫真、右衞將軍傅顏等收捕慕輿根,並將其家人黨羽一併誅殺。

此時前燕國內正因慕容儁之死而混亂,原本徵集在鄴城的大軍都常常私下逃散,但在慕容恪的輔助下,最終都成功穩定了國家。在慕容恪攝政之下,先擊敗據守野王叛變的寧南將軍呂護,後更進侵當時為東晉所控的洛陽,終於興寧三年(365年)攻下洛陽。後又攻取了東晉的兗州諸郡。

不過,慕容恪於太和二年(367年)去世,死前想以吳王慕容垂代替自己為大司馬,但最終慕容評改以慕容暐弟慕容沖接替慕容恪。慕容恪死後,陽騖在同年亦死,唯一仍在世的輔政大臣慕容評就以太傅主政。當時僕射悅綰上奏盡罷軍封蔭戶,以釋放人口以充實國家地方,防止人口隱匿。慕容暐同意之下,最終在悅綰的規劃下釋放了二十多萬戶人,政令亦令朝野震驚,慕容評更是十分不滿,派人暗殺了悅綰。

太和四年(369年),東晉桓溫發動北伐戰爭,主動進攻前燕,慕容暐所派的慕容厲、傅顏及慕容臧皆不能抵抗桓溫進攻,於是令慕容暐及慕容評十分恐懼,向前秦求援以外還打算逃回和龍(今遼寧錦州)。這時慕容垂自請進攻,最終成功扭轉局勢,更在逼桓溫撤軍時大敗晉兵。然而慕容評在後十分忌憚剛取得大功的慕容垂,二人更因將領孫蓋軍功問題發生爭論。因可足渾太后亦討厭慕容垂,於是就與慕容評謀殺慕容垂,慕容垂只得與家人逃奔前秦。

不久,出使前秦的黃門侍郎梁琛歸國,報告前秦國內揚兵講武,而且運糧至陝東,更逢慕容垂出奔前秦,表示擔憂前秦和和前燕開戰,建議朝廷早作防備。然而慕容評不認為前秦會破壞和前燕的和平,慕容暐於是和慕容評都沒重視梁琛的話。及後皇甫真又上言表示擔憂前秦對前燕有所圖謀,建議增強洛陽、并州和壺關(今山西長治東南)各城的軍力。慕容暐於是召慕容評討論,但因慕容評認為前秦「國小力弱」,要倚靠前燕為援,前秦天王苻堅也不會因慕容垂而攻燕,勸慕容暐不要自亂陣腳。慕容暐於是亦沒有聽從皇甫真的話。

當日前燕向前秦求援時,允諾割讓虎牢(今河南滎陽西北汜水鎮)以西的土地給前秦,但戰後反悔。苻堅於是以此派王猛等進攻前燕,進攻洛陽。慕容暐於是派了慕容臧救援洛陽,然而卻在滎陽大敗給前秦軍,無法有效營救洛陽,洛陽守將慕容筑唯有向前秦投降,洛陽陷落。慕容臧只得築新樂城而退。面對當時的軍事形勢,而且太后干政、慕容評貪污,尚書左丞申紹上疏要改革,提出令將士用命,對士兵「習兵教戰」、「從戎之外,足營私業」等。又要君臣「罷浮華,禁絕奢,峻明婚姻喪葬之條」以及增加重地守備軍隊等措施,但慕容暐都沒聽從。

洛陽陷落的同年(370年),前秦再攻前燕,王猛攻壺關而楊安攻晉陽(今山西太原)。慕容暐命慕容評等率中外精兵三十多萬抵禦。不過,慕容評竟禁止士兵取水和柴,而自據水源和山,向士兵販賣柴水以斂財,導致軍心全無。最終被前秦軍夜燒輜重,火光連鄴城都看得見。慕容暐見狀十分恐懼,下令慕容評將金錢財帛都分給士兵,命他們作戰,慕容評因恐懼而向前秦請戰。最終前秦軍大敗前燕軍,俘殺超過十五萬人,慕容評單騎奔鄴城。

王猛在戰後追擊至鄴,苻堅亦派大軍後繼。面對前秦軍兵臨城下,慕容暐只得與慕容評等人逃奔龍城(今遼寧朝陽),但隨行衞士一出城就散走,只餘十多名仍然隨行。當時前奏將領郭慶亦在後追擊慕容暐,途中道路艱險難行而且時有盜賊,保衞慕容暐的左衞將軍孟高、殿中將軍艾朗皆戰死,慕容暐更因失去馬匹而只得徒步逃亡,最終在高陽被郭慶所俘。慕容暐隨後被押見苻堅,苻堅質問慕容暐為何不降而逃走,慕容暐答:「狐狸快要死時,也會將頭朝向自己出生的山丘,我都是想死在先人墳墓那裏而已。」苻堅憐憫慕容暐而將他釋放,命他回去率文武百官出降。另外逃奔遼東的前燕殘餘勢力不久亦被消滅,前燕正式滅亡。

同年十二月,慕容暐與慕容皇族及鮮卑族四萬戶一同被苻堅遷往長安安置,並受封為新興侯,署為尚書。

太元八年(383年),前秦大舉南侵東晉,即淝水之戰,慕容暐亦以平南將軍、別部都督隨軍。前秦於淝水之戰大敗後,時駐鄖城的慕容暐隨前秦軍北撤,並護送苻堅的張夫人。至滎陽時,叔父慕容德勸慕容暐乘前秦軍力大損而復國,但慕容暐又不聽從。慕容暐終與苻堅回到長安,但當時前秦對全國的控制已不如前。太元九年(384年),慕容垂在河北叛變建立後燕,不久,慕容暐之弟慕容泓也在關中叛變建立西燕。當時慕容泓向苻堅要求送還慕容暐以換取燕秦兩國和平,但為苻堅所拒絕。苻堅亦因此召慕容暐來斥責,終在慕容暐叩頭陳謝之下原諒他,並命他寫信招撫慕容垂、慕容泓和慕容沖。不過慕容暐就暗中派密使向慕容泓說:「我是鐵籠裡的人,肯定無法回去了;而且,我也是帝國的罪人,無必要顧慮我了。你就建立大業,以吳王慕容垂為相國,中山王慕容沖為太宰、領大司馬,你可以做大將軍、領司徒,承制封拜,收到我去世的消息後,你就自己稱帝吧。」及後西燕與前秦在長安多有戰事,慕容暐與慕容肅共謀聯同長安城中數千鮮卑人作亂,以應進攻長安的西燕軍,於是借兒子新婚為由設計在其家殺害苻堅。然而苻堅因大雨而沒有去,事情洩露,苻堅召慕容暐和慕容肅並殺害二人,更誅連城中的鮮卑人。慕容暐死時三十五歲。

西燕、后燕均没有追谥慕容暐。慕容德建立南燕時,諡慕容暐為幽皇帝。

慕容暐很在意他人對其的批評,如李績曾經向慕容儁表示慕容暐的缺點是「雅好遊田,娛心絲竹」,慕容儁亦因而要慕容暐好好記著李績的話,好作改善。但慕容暐登位後,慕容恪雖然屢請以李績為尚書右僕射,但慕容暐都不同意,更說:「萬機之事都交由叔父處理,但李績一人,我想自己決定。」最終李績憂死。

\subsubsection{建熙}

\begin{longtable}{|>{\centering\scriptsize}m{2em}|>{\centering\scriptsize}m{1.3em}|>{\centering}m{8.8em}|}
  % \caption{秦王政}\
  \toprule
  \SimHei \normalsize 年数 & \SimHei \scriptsize 公元 & \SimHei 大事件 \tabularnewline
  % \midrule
  \endfirsthead
  \toprule
  \SimHei \normalsize 年数 & \SimHei \scriptsize 公元 & \SimHei 大事件 \tabularnewline
  \midrule
  \endhead
  \midrule
  元年 & 360 & \tabularnewline\hline
  二年 & 361 & \tabularnewline\hline
  三年 & 362 & \tabularnewline\hline
  四年 & 363 & \tabularnewline\hline
  五年 & 364 & \tabularnewline\hline
  六年 & 365 & \tabularnewline\hline
  七年 & 366 & \tabularnewline\hline
  八年 & 367 & \tabularnewline\hline
  九年 & 368 & \tabularnewline\hline
  十年 & 369 & \tabularnewline\hline
  十一年 & 370 & \tabularnewline
  \bottomrule
\end{longtable}


%%% Local Variables:
%%% mode: latex
%%% TeX-engine: xetex
%%% TeX-master: "../../Main"
%%% End:


%%% Local Variables:
%%% mode: latex
%%% TeX-engine: xetex
%%% TeX-master: "../../Main"
%%% End:

%% -*- coding: utf-8 -*-
%% Time-stamp: <Chen Wang: 2019-12-19 10:03:55>


\section{前秦\tiny(351-394)}

\subsection{简介}

前秦(350年—394年)是十六国之一。350年氐族人苻洪占据关中,称三秦王。352年苻健称帝,定都长安,国号“秦”。370年起,先後灭前燕、前凉及代国,统一北方。394年被西秦和後秦所灭。當時朝鮮半島由高句麗、百濟、新羅割據,接受前秦册封。北方外族有柔然、庫莫奚、契丹及高車。西有吐谷渾及白蘭。

因其所据为战国时秦国故地,故以此立国号。前秦之称最早见于《十六国春秋》,后为别于其他以“秦”为国号政权,而袭用之。又以其王室姓苻,故又称为苻秦。

西晉末年,西晉政權顛覆之際,略陽氐族推出貴族苻洪為首領。前趙劉曜在長安稱帝,以苻洪為氐王。後石勒滅前趙,苻洪降於石勒。333年,石虎徙關中豪傑及羌戎至關東,以苻洪為流民都督,居於枋頭。苻洪自稱大都督、大將軍、大單于、三秦王,不久為石虎舊將麻秋所毒死,其第三子苻健代統其眾。

苻健自枋頭而西,關中氐人紛起響應,苻健乘机進占关中,據有关陇。351年建都長安,苻健自稱大秦天王、大單于。352年,改稱皇帝,国号秦,史稱前秦。

起初苻健知道中原「民心思晉」,在枋頭時,打著晉征西大將軍、都督關中諸軍事、雍州刺史來作號召;抵達關中之後,遣使向東晉稱臣,以緩和關中地區的矛盾,直到他稱帝後,才和東晉斷絕關係。354年,東晉大將桓溫親率大軍四萬攻秦,因苻健採清野政策,晉軍在給養問題遇到困難,只好撤退。355年苻健死,子苻生繼位,因淫殺過度,357年,苻健弟苻雄之子苻堅殺死苻生自立。

苻堅在登位以前,就聽見王猛的名聲,並約見王猛,談得十分投契。即帝位,任王猛以政。王猛採取政治改革,加強中央集權,抑制貴族勢力發展來強化中央力量,並興修關中水利,前秦國力逐漸增強。370年,前秦滅前燕,擒慕容暐;371年,滅仇池氐楊氏;373年,攻取東晉梁、益二州,西南夷邛、筰、夜郎皆歸附於秦;376年,滅前涼張氏;同年,乘鮮卑拓跋氏衰亂之際,進兵滅代;382年,命呂光駐西域。中原地區盡為前秦版土之下,史稱「東極滄海,西併龜茲,南包襄陽,北盡沙漠」。東北、西域各國都遣使和前秦建立關係,只有東南一隅的東晉與他對峙。

378年,前秦征南大將軍苻丕等率領步騎兵七萬人,攻擊東晉所屬的襄陽,東晉梁州刺史朱序死守近一年,城池陷落被俘。

379年,前秦右將軍毛當、強弩將軍王顯,率二萬人自襄陽出發,跟後將軍俱難、兗州刺史彭超會師,攻擊東晉淮河以南各城池,攻陷盱眙,包圍三阿。東晉兗州刺史謝玄出兵救援,四次擊敗秦軍,俱難、彭超向北逃走,僅保住一命。

建元十八年(382年)苻堅之大将吕光率兵七万伐龟兹,龟兹王白纯不降,吕光进军讨平龟兹。

383年,前秦天王苻堅親率騎兵二十七萬、步兵六十萬南下,對東晉發動總攻擊。弟弟陽平公苻融擔任前鋒,十月,攻陷壽陽。苻堅派東晉降將朱序向東晉征討大都督謝石勸降,但朱序反而將秦軍狀況密告謝石,建議晉軍乘秦軍未全部集結時發動攻擊。十一月,東晉前鋒都督謝玄的部將劉牢之率兵五千突襲洛澗,秦軍大敗,死一萬五千人。

晉軍乘勝西進,秦軍在淝水西岸佈陣對峙。謝玄派人要求秦軍略向後撤,讓晉軍渡水決戰。苻堅企圖乘晉軍半渡淝水時予以截擊,同意後退,但秦軍軍心不穩,一退陣腳大亂,不能停止。晉軍乘勢渡水猛烈攻擊,混亂中苻融墮馬被晉軍所殺,朱序又在陣後大呼「秦兵敗矣!」於是秦軍崩潰,四散逃亡,前鋒三十萬人中,死亡的佔十分之七八。

淝水之戰後,原先歸附前秦的其他民族,紛紛乘機獨立,黃河以北又再陷入分裂的狀態。

383年,前燕降將、鮮卑族的冠軍將軍慕容垂,奉命攻擊在新安起兵的丁零部落首領翟斌,途中屠殺副將苻飛龍及一千人的氐人部隊。384年,慕容垂自稱「燕王」,廢除前秦年號,建立後燕,並進攻駐守鄴城的前秦長樂公苻丕。

前秦北地長史慕容泓(前燕帝慕容暐的弟弟),聽到叔父慕容垂攻鄴的消息,投奔關東集結數千鮮卑人,自稱大將軍、濟北王,建立西燕。苻堅派兒子鉅鹿公苻叡當統帥,羌人將領姚萇任參謀,出兵討伐,在華澤大敗,苻叡被斬殺。苻堅大怒,姚萇畏罪逃到渭北,被族人推為盟主。姚萇遂自稱大將軍、大單于、萬年秦王,建立後秦。

385年,後秦包圍新平郡,苻堅投奔五將山,被後秦將領吳忠俘擄,送回新平郡單獨囚禁。八月,姚萇派人向苻堅索取傳國玉璽,又遊說苻堅禪讓帝位,苻堅大怒拒絕,痛罵姚萇,只求一死,又先殺女兒苻寶、苻錦。八月二十六日,姚萇派人闖入囚禁苻堅的佛寺,縊殺苻堅,時年四十八歲。姚萇為掩飾自己的弒逆惡名,追尊苻堅為「壯烈天王」。

394年,七月,前秦帝苻登在馬毛山以南跟後秦帝姚興交戰,被生擒後斬首,太子苻崇投奔湟中繼承帝位。十月,苻崇被西秦首領乞伏乾歸驅逐,投奔隴西王楊定,兩人於攻擊西秦時被西秦涼州刺史乞伏軻彈斬殺,前秦到此滅亡。

\subsection{惠武帝生平}

苻洪(285年-350年),字廣世,略陽臨渭氐人,是前秦政權奠基者。苻洪原名蒲洪,後以讖文有「艸付應王」,遂改苻姓。氐族部落小帥蒲怀归之子,亦是前秦開國君主苻健之父。苻洪先後歸附前趙和後趙兩個政權,後在後趙內大亂時試圖謀取中原。最終雖然遭毒殺,但他所累積的力量令其子苻健在關中成功建立前秦。

蒲洪原是氐族酋長,因為驍勇而多謀略而得氐人畏服,永嘉四年(310年)時曾獲前趙(當時國號為「漢」)皇帝劉聰任命為平遠將軍,但蒲洪不受,反自稱護氐權尉、秦州刺史、「略陽公」。後因在永嘉之亂時大散錢財以向英傑之士訪尋轉危為安的方法,於是被宗人蒲光及蒲突推舉為盟主。太興二年(319年),前趙皇帝劉曜遷都長安,蒲光等逼蒲洪附前趙,於是獲授為率義侯。咸和三年(328年),劉曜在與石勒的決戰中被俘,蒲洪於是西保隴山自守。次年,前趙退保上邽的殘餘力量遭後趙將領消滅,蒲洪於是向石虎歸降。石虎於是以蒲洪為冠軍將軍,監六夷軍事,委以西方之事。

咸和八年(333年),後趙鎮守關中的河東王石生聯同鎮洛陽的石朗反抗丞相石虎,蒲洪於是自稱雍州刺史,歸附前涼。同年石生及石朗敗死,石虎命麻秋討伐蒲洪,蒲洪於是率二萬戶向石虎歸降。石虎則以蒲洪為光烈將軍、護氐校尉。蒲洪到長安後游說石虎遷關中豪傑及氐、羌人到東方,充實京師襄國。石虎於是遷秦、雍二州人民及氐、羌族人到關東,並以蒲洪為龍驤將軍、流民都督,命其居於枋頭。

蒲洪後多有征戰,累有戰功,於咸康四年(338年)被石虎拜為使持節、都督六夷諸軍事、冠軍大將軍,西平郡公。當時石虎養子石閔就因蒲洪實力強大,諸子有才且接近京畿,勸石虎誅除蒲洪,但石虎不單沒聽從,對蒲洪的待遇反更為優厚。永和五年(349年)蒲洪遷車騎大將軍、開府儀同三司、都督雍、秦二州諸軍事、雍州刺史,進封為略陽郡公。

同年,石虎去世,石遵繼位,石閔再提出對付蒲洪的建議,石遵於是削去蒲洪都督一職。此舉觸怒了蒲洪,於是憤而向東晉投降。同時,因為當時後趙因諸子爭位內亂,原先被遷到關東的秦雍流民都西歸故土,經過枋頭時就以蒲洪為主,於是令蒲洪的部眾增至十多萬人,當時在首都鄴城的蒲洪子蒲健亦出奔枋頭。新登位的後趙皇帝石鑒為怕蒲洪以其力量威脅中央,於是以蒲洪都督關中諸軍事、征西大將軍、雍州牧、領秦州刺史,讓他率眾返回關中。不過,當時蒲洪其實已經有心稱帝得天下。次年,東晉朝廷以蒲洪歸降,以其為氐王、使持節、征北大將軍、都督河北諸軍事、冀州刺史、廣川郡公。

當時,有人勸蒲洪稱王,蒲洪於是以「艸付應王」的讖文而改姓「苻」,自稱大將軍、大單于、三秦王。早前,後趙將領麻秋自長安率眾返鄴,途中被苻洪派兵俘獲,以其為軍師將軍。麻秋及後勸苻洪先放棄中原,先取關中作為基地,然後才再圖中原。苻洪十分同意,但不久麻秋就在宴會中以毒酒毒殺苻洪,意圖併吞其部眾;苻健於是斬殺麻秋。中毒的苻洪在死前囑咐苻健在其死後要速速入關,享年六十六歲。

苻健於永和七年(351年)即天王位,建立前秦,追諡苻洪為惠武皇帝,廟號太祖。

%% -*- coding: utf-8 -*-
%% Time-stamp: <Chen Wang: 2021-11-01 11:56:49>

\subsection{景明帝苻健\tiny(351-355)}

\subsubsection{生平}

秦景明帝苻健(317年-355年7月10日),字建業,略阳临渭(今甘肃秦安)人,氐族,苻洪第三子,十六国前秦開國皇帝。苻健繼父親苻洪統領部眾並成功入關,定都長安(今陝西西安),建立前秦。後屢次作戰征服其他反抗前秦的關內勢力,更擊敗北伐的晉軍。

苻健弓馬嫻熟,驍勇果敢,好施予亦善於事奉人,故此深得後趙皇帝石虎父子寵愛,當時石虎心中仍提防苻氏,暗殺了苻健的兩個兄長,但就沒有加害苻健。永和六年(350年),因應苻洪歸降東晉,苻健獲授假節、右將軍、監河北征討前鋒諸軍事、襄國縣公。同年苻洪軍師將軍麻秋毒殺苻洪,意圖併吞苻洪部眾,苻健於是收殺麻秋。苻洪臨死時向苻健說:「我之所以一直未入關中,就是以為能夠奪得中原;今天卻不幸被麻秋那小子加害。中原不是你們兄弟能夠爭奪到的,我死了後,你就快快入關中呀!」苻健於是接領父親的部眾,去掉父親自稱的大都督、大將軍、「三秦王」的稱號,稱東晉所授的官爵,並派叔父苻安到東晉報喪,請示朝命。

同年,後趙新興王石祗在襄國(今河北邢台)即位為帝,又以苻健為都督河南諸軍事、鎮南大將軍、開府儀同三司、兗州牧、「略陽郡公」。不過,苻健當時並沒有助石祗對付冉閔,反將目標對準關中,只為麻痺當時據有關中的杜洪才接受後趙的任命。苻健又在駐地枋頭(今河南浚縣西)興治宮室,教人種麥,顯得根本沒有心思佔領關中。但及後苻健就自稱晉征西大將軍、都督關中諸軍事、雍州刺史,率眾西進,並在盟津渡過黃河。渡河前,苻健命苻雄和苻菁分別領兵從潼關(今陝西渭南市潼關縣北)和軹關(今河南濟源東北)進攻,自己則跟隨苻雄渡河,並在渡河後燒掉浮橋,意在死戰。杜洪部將張先在潼關抵抗苻健軍,但被擊敗。及後苻健派苻雄兵行渭北,附近的氐、羌酋長都斬杜洪使而向苻健投降,苻菁、魚遵經過的城邑亦都投降,更在渭北生擒張先,令三輔地區大致都落在苻健之手。杜洪見局勢如此,唯有退守長安,但苻健隨即進攻長安,杜洪被逼棄長安而逃奔司竹(今陝西司竹鄉),苻健於是進據長安。苻健見長安人心思晉,於是向東晉獻捷報,並與東晉征西大將軍桓溫修好。於是令秦雍二州的少數民族和漢人都向苻健歸附,苻健亦攻滅佔領上邽(今甘肅天水市),不肯歸降的後趙涼州刺史石寧。

永和七年(351年),左長史賈玄碩請苻健依劉備稱「漢中王」事,表苻健為都督關中諸軍事、大將軍、大單于、「秦王」。但苻健則假裝憤怒的說:「我豈有能力當秦王呀!而且出使東晉的使者還未回來,你們又怎知我的官爵呀。」然而,不久就又暗示賈玄碩等為他上尊號,最終在再三推讓後,于正月丙辰日(351年3月4日)即天王、大單于位,大封宗室及諸子為公爵,建國號大秦,年號「皇始」,正式建立前秦政權。次年正月辛卯日(352年2月2日),苻健稱帝,進諸公爵為王爵,並授大單于位予太子苻萇。

皇始元年(351年),被苻氏驅逐的杜洪引東晉梁州刺史司馬勳伐前秦,苻健於是率兵在五丈原擊退他。司馬勳敗歸漢中(今陝西漢中)後,杜洪被其部將張琚所殺,不久苻健領二萬兵攻滅張琚,更派兵擄掠關東,助後趙豫州刺史張遇擊敗東晉將領謝尚,及後擄張遇及其部眾回長安,並對張遇授官。後張遇謀反事敗,引發雍州孔特等人舉兵反抗前秦。最終苻健亦派兵成功平定。

皇始四年(354年),桓溫北伐,自率主力軍自武關(今陝西丹鳳縣東)直取長安,另命司馬勳在進攻隴西。前秦初戰不利,被桓溫進攻至長安東南防近的灞上,逼得苻健要盡出三萬精兵出城抵禦桓溫。然而因桓溫並不急於進攻,而且苻健先晉兵一步收取熟麥,故此最終逼得桓溫退兵,苻健更乘勢追擊晉軍,大敗對方。

苻健勤於政事,多次召見公卿談論治國之道,而且一改後趙時苛刻奢侈之風,改以薄賦節儉,更專崇儒學,禮待長者,故此得到人們稱許。

皇始四年(354年),皇太子苻萇在追擊桓溫時受傷,同年傷重而死。次年(晉永和十一年,355年),苻健因讖文中有「三羊五眼」字句,遂以淮南王苻生當太子。至當年六月,苻健患病,苻生在苻健宮室侍疾,而當時任太尉的平昌公苻菁則以為苻健已死,直接領兵入宮,打算殺死苻生自立。但到東掖門時,苻健知道宮中發生事變,自登端門,陳兵自衞。當時苻菁部眾見苻健未死,於是驚懼潰散,苻健於是拿下苻菁,將他殺死。不久,苻健以太師魚遵、丞相雷弱兒、太傅毛貴、司空王墮、尚書令梁楞、尚書左僕射梁安、尚書右僕射段純及吏部尚書辛牢等為輔政大臣。但又告訴太子:「六夷酋帥及掌權的大臣,若果不遵從你的命令,那就立即除去他們。」。六月乙酉日(7月10日),苻健病逝,享年三十九歲。苻健死後諡為明皇帝,廟號稱世宗,後改諡為景明皇帝,廟稱高祖。

\subsubsection{皇始}

\begin{longtable}{|>{\centering\scriptsize}m{2em}|>{\centering\scriptsize}m{1.3em}|>{\centering}m{8.8em}|}
  % \caption{秦王政}\
  \toprule
  \SimHei \normalsize 年数 & \SimHei \scriptsize 公元 & \SimHei 大事件 \tabularnewline
  % \midrule
  \endfirsthead
  \toprule
  \SimHei \normalsize 年数 & \SimHei \scriptsize 公元 & \SimHei 大事件 \tabularnewline
  \midrule
  \endhead
  \midrule
  元年 & 351 & \tabularnewline\hline
  二年 & 352 & \tabularnewline\hline
  三年 & 353 & \tabularnewline\hline
  四年 & 354 & \tabularnewline\hline
  五年 & 355 & \tabularnewline
  \bottomrule
\end{longtable}


%%% Local Variables:
%%% mode: latex
%%% TeX-engine: xetex
%%% TeX-master: "../../Main"
%%% End:

%% -*- coding: utf-8 -*-
%% Time-stamp: <Chen Wang: 2021-11-01 11:57:05>

\subsection{越厉王苻生\tiny(355-357)}

\subsubsection{生平}

秦越厉王苻生(335年-357年),字長生,略陽臨渭(今甘肅秦安)氐族人。十六國時期前秦景明帝苻健的第三子。史載苻生「荒耽淫虐,殺戮無道,常彎弓露刃以見朝臣,錘鉗鋸鑿備置左右」在位兩年期間殺害了多位大臣,以及做了多項殘忍變態的事。最終苻生被苻堅發動政變推翻,降封越王,不久被殺。不過後世亦有人認為苻生的暴政其實是史家誣捏渲染的結果。

苻生天生就只有一隻眼,年幼而無賴,爺爺苻洪因而十分討厭他。一次苻洪特地戲弄他,問侍者:「我聽說瞎子都只有一行眼淚,這是真的嗎?」侍者回答:「是呀。」在場的苻生聽後大怒,取出佩刀自殘,流出一行血,說:「這也是一行眼淚呀。」苻洪見狀大驚,鞭打他。苻生說:「我耐得下兵器,受不住鞭打!」。苻洪骂他:「你再是這樣,我就要把你送去当奴隶!」苻生竟答:「那可不就像石勒那樣嗎?」當時苻洪正歸屬後趙,而當時後趙皇帝石虎心中其實十分忌憚苻氏的勢力,故苻洪聽後震惊,光着脚就跑来遮住他的嘴巴。苻洪隨後勸苻健把苻生殺掉,但苻健要動手時就被其弟苻雄制止,說:「男孩子長大後就會改過的了,為何要這樣做呢!」

苻生长大后,力大無比,能徒手格击猛兽,奔跑速度飛快,而擊、刺、騎射的能力亦勇冠一時。皇始元年(351年),苻健稱天王,建立前秦,苻生獲封為淮南公,次年苻健稱帝,苻生進封淮南王。皇始四年(354年),桓温北伐前秦,苻生與太子苻萇、丞相苻雄等出兵迎擊,他就曾經十多次单马突擊晉軍,令晉軍傷亡甚大。

太子苻萇在追擊撤退的桓溫軍隊時受了傷,不久死去,苻健以讖言「三羊五眼」應符,於皇始五年(晉永和十一年,355年)立苻生為太子。同年,苻健患病,太尉苻菁乘時想殺苻生奪位但失敗被殺。隨後苻健以太師魚遵、丞相雷弱兒、太傅毛貴、司空王墮、尚書令梁楞、尚書左僕射梁安、尚書右僕射段純及吏部尚書辛牢八人為顧命大臣,輔助苻生。然而,苻健慮及苻生凶暴嗜酒,擔心他不能保全家業,被大臣有機可乘,於是對苻生說:「六夷酋帥及掌權的大臣,若果不遵從你的命令,那就立即除去他們。」

同年六月乙酉日(355年7月10日),苻健死,次日苻生即位為帝,改元壽光。不過,苻生本身酗酒,在登位後就常常酒醉,群臣上朝都很少見到苻生,連群臣的上奏都因苻生長醉而被擱在一邊。即使上朝,苻生每當發怒都只會殺人,即位後就多次出現殺戮大臣以至殘害生命的凶殘事件,苻健設的八名輔政大臣全都被苻生所殺。最終造成「宗室、勳舊、親戚、忠良殺害殆盡,王公在位者悉以告歸,人情危駭,道路以目」的狀況。

苻生即位後,立刻就改了年號,當時群臣上奏:「先帝死後未逾年而改元,不合禮法呀。」苻生卻大怒,要找出最初提出這上奏的大臣,最終找出了顧命大臣之一的段純,就將他殺害。後來,中書監胡文及中書令王魚向苻生報告天象:「最近有客星(彗星)在大角,熒惑(火星)入東井。大角,是皇帝之坐;東井,表示秦地;按占卜,不出三年,國內將有大喪,大臣會被殺戮,希望陛下自脩德行以避禍。」苻生卻說:「皇后和朕位置相應,可以應了國喪之劫。毛太傅、梁車騎、梁僕射受遺詔輔政,就應了大臣被戮的劫。」於是就殺了梁皇后、毛貴、梁楞及梁安四人;太師魚遵亦於壽光三年(357年)因民謠「東海大魚化為龍,男皆為王女為公」而被殺。又一次苻生與大臣飲宴,更在奏樂時唱起歌來,命尚書令辛牢勸酒。然而,就因為大臣們都沒有全都醉倒,於是就拿起弓將辛牢射殺。更有一次在咸陽故城設宴,將遲到的大臣殺害。

另外,苻生亦寵信趙韶、董榮等人,丞相雷弱兒以他們亂政,經常公開在朝堂批評他們,他們於是在苻生面前中傷雷弱兒。苻生於是誅殺雷弱兒及其家人,最終因為雷弱兒南安羌族酋長的身分,各羌族部落都有離心。司空王墮亦痛恨董榮等人,不肯親附,在董榮的唆使下,苻生又殺王墮以應日蝕之變。

亦因苻生天生殘疾,「不足、不具、少、無、缺、傷、殘、毀、偏、隻」等字都是要避諱的,絕不能說。但就有不少大臣和侍從因此而死。其中太醫令程延在研安胎藥時向苻生解釋人參,就說了「雖小小不具,自可堪用」而被苻生下令鑿出雙眼,然後斬首。

苻生賞罰沒有準則,大臣不論稱頌他還是批評他稍有不當,都可能被殺,但寵臣的姦佞之言卻都接納。而苻生的姬妾只要表現得稍不合其意,都會被殺,並棄屍渭水。苻生又愛虐待動物,活活的剝下牛、羊、驢、馬的皮毛,或者用熱水燙雞、豬、鵝,將三、五十隻這樣的動物一起放在殿上觀賞。苻生甚至還將死囚的臉皮活活剝掉,命其在群臣面前跳舞。苻生更曾命宮女與男子裸體在其面前性交,甚至曾在路上看見一對同行的兄妹,就命他們亂倫,兄妹最終因不肯聽從而被殺。受斬腳、刳胎、拉脅、鋸頸等其他酷刑的人亦數以千計。

苻生聽到有對自己的怨言,更下詔書稱自己並沒有不善,自己所作的根本不算濫刑暴虐。據說當時還有食人野獸橫行,平民為了避開猛獸自保就聚居而且荒廢農業。苻生則認為野獸吃飽了人就會走,不會長久的,且認為天降災劫其實正是對應平民一直犯罪,協助天子以刑罰教導平民而已,只要不犯罪就不必怨天尤人。

苻生曾經命三輔居民興建渭橋,金紫光祿大夫程肱以妨礙農業為由勸諫,反觸怒苻生,被殺。又一次長安突然颳起大風,極之影響人們活動,苻生舅舅左光祿大夫強平於是借天變而勸諫苻生愛護禮待公卿,致敬宗社,去如秋霜的威嚴而立三春般的恩澤等。但苻生則認為強平是妖言,不顧臣下以至太后的懇求,堅持殺死強平。

壽光三年(晉升平元年,357年),姚襄進圖關中,更派人招納因雷弱兒被誅而產生離心的關內羌胡。苻生於是派了衞大將軍苻黃眉等率兵抵抗,最終大敗敵軍,更殺姚襄,令姚襄弟姚萇率眾歸降。苻黃眉立了大功,但凱旋後卻沒有獲得苻生褒賞,反而被多次當眾侮辱。苻黃眉因而憤怒,圖謀殺死苻生,但風聲洩露,反被殺,更株連不少王公親戚。

而當時御史中丞梁平老等人都勸有時譽的苻生堂弟、東海王苻堅殺苻生以救國,苻堅同意但不敢發難。但六月有一晚,苻生對侍婢表示翌日就要殺苻法、苻堅兩兄弟,侍婢於是立刻告訴二人,於是二人與強汪、梁平老和呂婆樓等都率兵衝入宮,宮中宿衞將士知道苻堅奪位都向其投降。苻生當時仍然在酒醉中,知有人攻來,就大驚,問侍從:「那是甚麼人?」侍從答:「是賊!」苻生就說:「為甚麼不下拜!」苻堅兵眾聽後大笑,苻生更說:「還不快快下拜,不拜的我就斬了他!」苻堅於是廢苻生為越王,自己繼承帝位,並降稱天王。不久,苻生被苻堅杀害,享年二十三歲,諡為厲王。儿子苻馗被封为越侯。

苻生无后。苻坚后来平定苻生弟苻廋等人叛乱,赐苻廋死,赦免苻廋诸子,并安排苻廋的儿子过继苻生为后。

苻洪:「此兒狂悖勃,宜早除之,不然,長大必破人家。」

薛讚、權翼:「主上猜忍暴虐,中外離心。」

《晉書》史臣曰:「長生慘虐,稟自率由。覩辰象之災,謂法星之夜飲;忍生靈之命,疑猛獸之朝飢。但肆毒於刑殘,曾無心於戒懼。招亂速禍,不亦宜乎!」

《晉書》贊曰:「長生昏虐,敗不旋踵。」

但有些記述表示所謂苻生暴虐也可能是史臣渲染的結果。楊衒之《洛陽伽藍記》卷二記載隱士趙逸之言,云:「國滅之後,觀其史書,皆非實錄,莫不推過於人,引善自向」,如「苻生雖好勇嗜酒,亦仁而不殺。觀其治典,未為凶暴,及詳其史,天下之惡皆歸焉。苻堅自是賢主,然賊君取位,妄書君惡,凡諸史官,皆是類也。」劉知幾《史通》曲筆篇云:「昔秦人不死,驗苻生之厚誣」,即是據此。

呂思勉亦懷疑苻生的一系列殘忍殺人、文詞避諱、以刀刃錘斧威懾群臣等都是史官誣陷、醜化苻生的結果。誅殺梁安、雷弱兒等人亦因他們其實是有通晉的嫌疑,是不得已,卻招來謗毀之聲。稱「他如怠荒、淫穢,自更易誣。《金史·海陵本紀》述其不德之亂,連章累牘,而篇末著論,即明言其不足信,正同一律。」

\subsubsection{寿光}

\begin{longtable}{|>{\centering\scriptsize}m{2em}|>{\centering\scriptsize}m{1.3em}|>{\centering}m{8.8em}|}
  % \caption{秦王政}\
  \toprule
  \SimHei \normalsize 年数 & \SimHei \scriptsize 公元 & \SimHei 大事件 \tabularnewline
  % \midrule
  \endfirsthead
  \toprule
  \SimHei \normalsize 年数 & \SimHei \scriptsize 公元 & \SimHei 大事件 \tabularnewline
  \midrule
  \endhead
  \midrule
  元年 & 355 & \tabularnewline\hline
  二年 & 356 & \tabularnewline\hline
  三年 & 357 & \tabularnewline
  \bottomrule
\end{longtable}


%%% Local Variables:
%%% mode: latex
%%% TeX-engine: xetex
%%% TeX-master: "../../Main"
%%% End:

%% -*- coding: utf-8 -*-
%% Time-stamp: <Chen Wang: 2019-12-19 10:13:46>

\subsection{宣昭帝\tiny(357-385)}

\subsubsection{文桓帝生平}

苻雄(4世纪?-354年7月26日),字元才,略陽臨渭(今甘肅秦安)氐族人。十六国時前秦的开国元勋、宗室。苻洪之幼子,景明帝苻健之弟,屡建军功,官至丞相,曾參與抵抗東晉將領桓溫發動的北伐戰爭。

苻雄年輕就已熟讀兵書、富有謀略,亦擅長射馬射箭;此外亦有為政治術,慷慨施予地位不高但有才德的人。因著父親苻洪在後趙滅前趙後歸附後趙並接受其官職,苻雄亦仕於後趙,更因戰功而獲後趙君主石虎授予龍驤將軍。

永和五年(349年),石虎去世,諸子爭位令國內漸亂,而苻洪亦因遭當時後趙皇帝石遵削職而叛投東晉。次年,後趙將領麻秋東歸鄴城,苻雄受父命領兵迎擊,成功俘獲麻秋,並以其為軍師將軍。但不久麻秋就藉宴會而以毒酒毒殺苻洪,意圖盡收苻氏部眾。苻健殺麻秋後接掌苻洪部眾,並順從父親遺言,進據關中。

當時關中為自稱晉臣的杜洪所控制,苻健因應人心思晉,於是稱東晉早前加予的官爵,並以苻雄為輔國將軍。不久苻健正式出兵關中,苻雄在大軍渡過黃河後就受命率五千兵取道潼關進攻長安,作為苻健的前驅。苻雄在潼關以北擊敗杜洪派去抵抗的張先,及後苻雄北巡渭北,所過的城邑都向其歸降。苻健進據長安後向東晉獻捷,更得秦、雍二州的胡族及漢人歸附,苻雄亦攻陷後趙涼州刺史石寧據守的上邽,斬殺石寧,鞏固苻氏在關中的統治。

皇始元年(351年),苻健稱天王、大單于,正式建立前秦,並封苻雄為東海公,以其為都督中外諸軍事、丞相、領車騎大將軍、雍州牧。次年,以苻雄為首的百官上請苻健稱帝,苻雄於是進封東海王。

同年,東晉安西將軍謝尚因不能安撫歸附的後趙豫州刺史張遇而令其叛變,謝尚於是與姚襄攻伐張遇,而苻雄就奉命與苻菁出兵略地關東,並救援張遇。最終苻雄在潁水的誡橋擊敗晉軍,逼其退還淮南,並略陳郡、穎川、許昌及洛陽附近共五萬多戶及張遇回軍關中。

同年,苻雄又在隴西擊敗後趙將領王擢,令其逃奔前涼。皇始三年(353年)二月,王擢會同前涼軍隊伐秦,苻雄又率兵擊敗王擢等。但隨著秦州刺史苻願及將軍苻飛分別敗於王擢及楊初,苻雄等於是回屯隴東,而不久張遇更在聯結關中豪族反叛,雖然張還事敗被殺,但孔特、、劉珍、夏侯顯、喬秉、胡陽赤及呼延毒就各自據城起兵叛秦。苻雄於是回軍長安並與苻法、苻飛等分兵平定孔特等人。

皇始四年(354年),就在苻雄攻下胡陽赤據守的司竹,張遇謀反引發的數個主要叛亂勢力僅剩下呼延毒及喬秉時,東晉征西大將軍桓溫發動北伐,自荊州進攻前秦,另由司馬勳率偏師由梁州北上關中。苻健於是派苻雄、太子苻萇等人共率五萬軍抵抗,但大軍在藍田縣被桓溫率領的主力擊敗,苻雄亦在白鹿原敗於桓沖,於是與苻萇等退守長安城南,與雷弱兒所率的三萬精兵共同抵抗。不過桓溫當時只駐屯長安東南的灞上,未有進逼長安,苻雄此時就率領七千騎兵突襲時正經子午谷入關中的司馬勳軍,令其敗退至女媧堡。苻雄及後返回長安,於白鹿原擊敗桓溫,而桓溫亦因乏糧而在六月被逼退兵,呼延毒亦跟隨桓溫南走。苻雄接著討伐佔領了陳倉的王擢及司馬勳,令兩人分別敗退回漢中及略陽,成功解除了桓溫這次北伐帶來的危機。

六月丙申日(7月26日),苻雄在進攻喬秉據守的雍城時去世,苻健聞訊悲傷得嘔血,說:「上天不讓我平定四海麼!為何這麼快就奪去我的元才呀?」追贈魏王,賜諡號敬武,葬禮依西晉安平獻王司馬孚的先例。

其子苻堅後來即位稱天王,追尊苻雄為文桓皇帝。

有載苻雄「醜形貌,頭大而足短」,並沒有雄偉的形象,在後趙任龍驤將軍時更被人稱為「大頭龍驤」。
苻雄作為前秦開國元勳,亦是君主的親弟弟,位至丞相,位高權重,但為人謙虛恭順,遵奉法度,加上在政事和軍事都有才能,故深受苻健倚重,更稱:「元才,是我的姬旦。」

\subsubsection{宣昭帝生平}

秦宣昭帝苻坚(338年-385年10月16日),字永固,一名文玉,略阳临渭(今甘肃秦安)人,氐族,苻雄之子,苻洪之孫,苻健之侄,是十六国时期前秦的君主,称大秦天王。

初封東海王,後發動政變推翻堂兄苻生而即位,在位期間重用漢人王猛,亦推行一系列政策與民休息,加強生產,終令國家強盛,接著以軍事力量消滅北方多個獨立政權,成功統一北方,並攻佔了東晉領有的蜀地,與東晉南北對峙。苻堅於383年發動戰爭意圖消滅東晉,史稱淝水之戰,但最終敗給東晉謝安、謝玄領導的北府兵,國家亦陷入混亂,各民族紛紛叛變獨立,苻堅最終亦遭羌人姚萇殺害,谥号宣昭,庙号世祖。

苻堅的母親曾夢見受到神靈寵幸而懷孕,足足懷了十二個月才生下苻堅,苻堅生於正月初二日,年僅七歲就已顯現其聰敏善良的特質,且舉止都循規蹈矩。又因妥貼地侍奉祖父苻洪,不需詢問就能猜到祖父的想要取甚麼,並及時為祖父拿來,故此深得苻洪疼愛。八歲時,苻堅就請求苻洪請老師到家教他學習,苻洪見他熱心求學十分高興,欣然同意。

皇始四年(354年),苻雄去世,苻堅承襲父爵東海王。另苻堅亦獲授龍驤將軍,苻健更以苻洪曾經在後趙獲授此號勉勵苻堅, 苻堅當時亦「揮劍捶馬」,被苻健的話所感動和激勵,士卒見此,亦心服苻堅。苻堅當時亦博學多才藝,更有經略大志,廣交豪傑,結交了呂婆樓、強汪、梁平老及王猛等人,都成為其左右手。

壽光三年(357年),姚襄謀圖關中,並聯結前秦境內的羌人,苻堅與苻黃眉、鄧羌等人率兵抵抗,終在鄧羌成功誘使姚襄出擊而由苻黃眉率主力將姚襄擊敗,並擒殺姚襄,逼令姚襄弟姚萇率其部眾歸降前秦。然而,當時前秦皇帝苻生賞罰失當,兇殘好殺,苻黃眉因立大功後未受褒賞,反受侮辱而謀反。雖然最終苻黃眉謀反失敗,但苻堅當時很有聲譽,姚襄舊將薛讚和權翼亦欣賞苻堅的才能,並勸苻堅學湯武伐昏君奪帝位 ;梁平老等人亦勸苻堅謀反。同年,苻堅與其兄苻法得知苻生有意加害,於是先發制人,入宮罷黜苻生,不久更殺死苻生。苻堅將帝位讓給苻法,但苻法自以庶出不敢受。苻坚在群臣的勸進下即位,並降號天王,稱大秦天王。即位後苻堅先誅除苻生寵信的董榮等人,隨後擢用李威、呂婆樓、王猛、權翼、薛讚等人。又追復被苻生所誅殺的八個顧命大臣的官位,隨才選用其子孫為官。

苻堅即位後,亦修整一些名實不符的官職,恢復已絕的宗祀,上禮神祇,鼓勵農業,設立學校,扶持鰥寡孤獨和年老無依者。另亦褒揚稱頌一些有特殊才行、孝友忠義、有德業的人。後苻堅下令各地方官員都上舉孝悌、廉直、文學、政事四項才德的人才,若真的是人才就得賞賜,否則就被降罪;另苻堅亦不優待宗室,即使是宗室中人,若無才幹都會被棄用,於是當時國內官員都十分稱職。而通過開墾耕地,令前秦倉庫充實,人民溫飽而令盜賊也少了。

苻堅亦下令與民休息,在即位次年(358年)討平於并州叛變的張平後就下令偃甲息兵,直至365年出兵平定劉衞辰及曹轂的叛亂前都沒有大型的軍事行動。苻堅又應當時秋旱而下令減省膳食和暫停奏樂,將金玉錦繡等貴重物品散發給軍士,並命後宮省儉服飾。苻堅更開發山澤,且得出的資源不限於官府,連平民也可用。

至甘露六年(364年),苻堅下令各公國自置中尉、大農及其他官屬,然而眾人卻以當時富商趙掇、丁妃等人車服盛如王侯,紛紛延攬這些富商為二卿。苻堅於是下令延攬富商為卿者降爵位到侯爵,並下令沒爵位或官職的人都不能在都城百里以內乘車馬;工商、奴隸及婦人亦不得穿戴金銀錦繡,違者處死。

另一方面,苻堅重用漢人王猛,機要之事王猛幾乎無不知道,這令一眾氐族豪族及元勳十分不滿。其中特進樊世自恃是氐族豪族,且有大功勳,當眾直斥王猛竊取為前秦立下赫赫功勳的功臣之成果。苻堅知道後,決意殺樊世以威懾所有氐族豪族。樊世死後,各氐人都爭相批評王猛,苻堅更為王猛而謾罵和鞭撻大臣,終令氐人都畏懼王猛,壓制了氐族豪強對王猛新政的反抗力量。而王猛於359年捕殺酗酒橫行,掠貨擄人的強太后弟強德,苻堅想下令赦免亦趕不及,後來不但沒有問罪王猛,更讓王猛在數十日內處罰了二十多個權豪貴戚,其嚴正執法亦為苻堅所允許,亦為苻堅所認同。

甘露六年(364年),汝南公苻騰謀反被殺,當時王猛以苻生諸弟尚有五人,建議苻堅除去五人,否則終會為患,然而苻堅不聽。至次年,因著劉衞辰及曹轂的叛亂,苻堅親自率軍出征平定,並北巡朔方以撫諸胡。時為征北將軍的苻幼趁機領兵進攻當時由太子苻宏、王猛及李威留守的首都長安,只因李威領兵擊斬苻幼而平定亂事。

苻幼起事時其實還暗中聯結了征東大將軍、并州牧、晉公苻柳以及征西大將軍、秦州刺史、趙公苻雙,但苻堅以二人分別為伯父苻健愛子及同母弟弟而不問罪,亦不將此事公布。然而,二人卻與時為鎮東將軍、洛州刺史的魏公苻廋及安西將軍、雍州刺史的燕公苻武共謀作亂。苻堅得知,於是召眾人到長安,但四人就在建元三年(367年)十月各據州治起兵反叛,苻堅試圖勸其罷兵,答應一切如故,不作追究,並以齧棃 為信物,但四人都沒有任何動搖。次年正月,苻堅正式派軍鎮壓叛亂,派楊成世、毛嵩、鄧羌、王猛、張蚝等人分途出兵,分別進攻四地。但當時楊成世及毛嵩都分別敗於苻雙和苻武的叛軍,逼使苻堅再將王鑒、呂光等人率兵再攻。最終王鑒、呂光及王猛等先後擊敗並斬殺四公,才令亂事成功於當年平定。而在苻堅進攻苻廋時,苻廋主動獻州治陝城(今河南陝縣)歸降前燕,並請兵接應。此舉震動前秦,更逼使苻堅派大軍至華陰(今陝西華陰)防備,只因前燕太傅慕容評拒絕迎降,才避免了更大的危機。

建元五年(369年),前燕吳王慕容垂在擊退東晉桓溫的北伐軍後因受到慕容評排擠,於是出奔降秦。苻堅早於兩年前知道慕容恪去世的消息時就已經有吞併前燕的計劃,還特地派了使者出使前燕以探虛實,然而苻堅因為慕容垂的威名而不敢出兵。現在慕容垂自來,苻堅十分高興,並親自出郊迎接,對其極為禮待,更以其為冠軍將軍,不顧王猛要他提防慕容垂的諫言。

同年十二月,苻堅以前燕違背當日請兵的諾言,不割讓虎牢(今河南滎陽汜水縣西北)以西土地予前秦為藉口出兵前燕,以王猛、梁成和鄧羌率軍,進攻洛陽(今河南洛陽市),並於次年年初攻下。六月,苻堅再命王猛等出兵前燕,自己更親自送行。王猛終在潞川擊潰率領三十多萬 大軍的前燕太傅慕容評,並乘勝直取前燕首都鄴城(今河北臨漳縣西南),苻堅更在王猛圍攻鄴城時親自率軍前往鄴城助戰。拿下鄴城後,正出奔遼東的前燕皇帝慕容暐被前秦追兵生擒,前燕在遼東的殘餘反抗力量亦遭消滅,前秦正式吞併前燕。

另一方面,369年,東晉將領袁真在桓溫北伐失敗後因被桓溫委以戰爭失利的罪責,憤而據壽春(今安徽壽縣)叛變,聯結前燕及前秦。 袁真不久去世,但其子袁瑾仍然堅守壽春,並在前燕亡後繼續向苻堅求救。苻堅於是於371年派王鑒及張蚝救援,但圍城的桓溫派將領桓伊等擊敗王鑒等,逼其退屯慎城(今安徽穎上縣),不久壽春被晉軍攻陷。

在前秦吞併前燕,收撫前燕領土的同一年,名義上臣服於前秦的仇池公楊世死,其子楊纂襲位後只受東晉朝命,斷絕與前秦的臣屬關係,苻堅遂在次年(371年)派兵進攻仇池。當時楊纂叔父楊統正與楊纂兵戎相見,東晉梁州刺史楊亮知道前秦進攻後亦派了郭寶等領兵協助楊纂,然而最終楊纂軍大敗,在仇池兵臨城下及楊統率眾降秦之下,楊纂只得出降。此後前秦命參與進攻仇池的將領楊安鎮守仇池。前仇池至此滅亡。當時苻堅有意在河西樹立威信,以德懷民,於是盡釋早前俘獲的前涼將領陰據及其所統五千兵士,前涼君主張天錫在佑道前仇池被前秦攻滅後甚為畏懼,至此就被逼向前秦稱藩。吐谷渾君主碎奚亦因前仇池滅亡而遺使向前秦進貢,苻堅亦授予其官職爵位。另外,苻堅又出兵攻伐隴西鮮卑首領乞伏司繁,盡降其眾,苻堅留乞伏司繁在長安,只由其堂叔乞伏吐雷統眾。

建元九年(373年),東晉梁州刺史楊亮派其子楊廣進攻仇池。但楊廣敗於仇池守將楊安,原先駐守沮水防備前秦的各軍戍更因而棄守潰逃,逼使楊亮退守磬險。而楊安亦趁機進攻東晉,進攻漢川。不久,苻堅更命益州刺史王統領攻漢川,毛當等攻劍門(今四川劍閣東北),大舉進攻東晉梁、益二州。楊亮在青谷率巴獠抵抗但失敗,只得退保西城(今陝西安康西北),結果漢中(今陝西漢中)、劍閣(今四川劍閣)、梓潼(今四川梓潼縣)等地先後失陷。東晉益州刺史周仲孫在緜竹(今四川綿竹縣)要抵抗來侵的朱肜部時,另一邊的毛當已經快攻到益州治所成都(今四川成都),周仲孫唯有逃到南中,於是前秦攻下了益、梁二州。

次年,益州發生叛亂,蜀人張育、楊光起兵反抗前秦,並向東晉稱藩,而巴獠酋帥張重、尹萬等亦參與,苻堅於是命鄧羌入蜀鎮壓;同一時間,東晉益州刺史竺瑤及威遠將軍桓石虔則受命入蜀,進攻墊江(今重慶墊江縣)。當時張育等人圍攻成都,但期間他們內訌爭權,終被鄧羌等人擊敗,叛亂被平定。竺瑤和桓石虔雖於墊江擊敗寧州刺史姚萇,但不能擴大戰事,只得退還巴東,前秦始終固守了蜀地。

建元十二年(376年),苻堅以張天錫「雖稱藩受位,然臣道未純」為由出兵十三萬進攻前涼。當時苻堅亦派閻負和梁殊出使前涼,徵召張天錫到長安,然而張天錫不願投降,決意與前秦決一死戰,下令斬殺二人,並派馬建抵抗前秦。隨著前秦軍西渡黃河,攻下纏縮城(今甘肅永登縣南),張天錫更派掌據到洪池(今甘肅天祝縣西北烏鞘嶺)協同馬建作戰,自己更親自率兵到金昌助戰。然而,前秦軍進攻二人時,馬建竟向前秦投降而掌據戰死,張天錫驚懼而退還都城姑臧(今甘肅武威)。前秦軍接著直攻姑臧,張天錫被逼出降,前涼至此滅亡。

隨著先後攻滅前燕、前仇池和前涼三個割據政權,北方唯一的割據政權就是拓跋氏建立的代國。在滅前涼的同一年,苻堅以應劉衞辰求救為由,命幽州刺史苻洛率兵十萬,另派鄧羌等率兵二十萬,一起北征代國。當時代王拓跋什翼犍先後命白部、獨狐部及南部大人劉庫仁抵禦,但都失敗,而什翼犍因病而不能率兵,被逼北走陰山,但高車部族此時卻叛變,什翼犍只得回到漠南,並看準前秦軍稍退,於是返回雲中郡盛樂(今內蒙古和林格爾北)的都城。此時,拓跋斤挑撥什翼犍子拓跋寔君,令其起兵殺死父親及其他弟弟;前秦軍聞訊亦立刻出兵雲中,代國於是崩潰,為前秦所滅。

苻堅隨後殺死拓跋斤及拓跋寔君,拓跋窟咄被強遷至長安,而什翼犍諸子亦被殺,什翼犍孫拓跋珪尚幼,再無於當地有效控制代國統下諸部的人。苻堅因而聽從燕鳳的話,分別以劉庫仁及劉衞辰分統代國諸部,借兩人之間的矛盾互相制衡。至此,前秦成功統一北方,只剩下據有江南地區的東晉。

建元十四年(378年),苻堅派苻丕等人進攻襄陽(今湖北襄陽市),另分一路由慕容垂、姚萇率領的軍隊經武當,配合苻丕進攻襄陽。數月後,兗州刺史彭超請求進攻彭城(今江蘇徐州市),並上言請派重將出兵淮南,與進攻襄陽的苻丕配合,形成東西並進之勢,最終消滅東晉。苻堅同意並派了俱難、毛盛等人進攻淮陰(今江蘇淮陰)、盱眙(今江蘇盱眙縣東北),由彭超都督東討諸軍事。

進攻襄陽的軍隊因著守將朱序堅守以及苟萇意圖孤立襄陽而逼其自降的戰略,一直與晉軍相持至年末。此事令苻丕等遭到彈劾,苻堅亦下令要求苻丕最遲在明年春季就要取勝。苻丕於是轉而急攻,終於在次年正月攻下襄陽。另一方面,晉兗州刺史謝玄於建元十五年(379年)奉命救援彭城,最終雖然護送城內的晉軍和沛郡太守戴逯離開,但彭城仍被前秦攻下,及後秦軍亦先後攻下盱眙和淮陰,並在三阿(今江蘇寶應)圍困晉幽州刺史田洛,威脅東晉江北重鎮廣陵(今江蘇揚州市)。此時,晉軍發動反擊,成功擊敗圍攻三阿的俱難、彭超等,逼他們退屯盱眙;次月二人再失盱眙,退保淮陰,但晉軍水軍當時乘潮北上,焚毀秦軍建在淮河上的橋,並擊敗俱難等,逼其退還淮北。而面對謝玄等的追擊,二人終在君川(今江蘇盱眙縣北)大敗給晉軍。面對東線的大敗,苻堅大怒並收捕彭超,嚇得彭超自殺,又將俱難貶為庶民。

就在建元十四年(378年)東西二線南攻東晉之時,鎮守洛陽的北海公苻重謀反,不過很快就因苻重長史呂光忠於苻堅而被平定,苻重獲赦而返回府第。至建元十六年(380年),苻堅卻再度命苻重為鎮北大將軍,駐鎮薊(今北京)。同年,苻堅亦命行唐公苻洛為征南大將軍,鎮守成都,並命其由襄陽循漢水西上上任。但其實苻洛在立下滅亡代國的大功後因為沒有獲苻堅封為將相重臣,反倒仍以其作為邊境州牧深感不滿,更懷疑命他到襄陽其實是苻堅殺他的陰謀,於是決定叛變。當時雖然只有苻重支持苻洛,但苻洛仍自和龍(今遼寧錦州)率兵七萬直指長安。關中人民恐懼戰亂,人心騷動,盜賊興起,苻堅試圖勸降,於是以永封幽州請苻洛罷兵。然而苻洛拒絕,並聲言要「還王咸陽,以承高祖之業」,更反說若苻堅在潼關候駕,他會以他為上公,還爵東海。苻堅於是大怒,出兵討伐,並在中山與苻重及苻洛的十萬聯軍會戰,終生擒苻洛並斬殺苻重,平定亂事。

事後,苻堅認為關東地區地廣人多,於是決定從原居於三原(今陝西三原縣)、九嵕(今陝西乾縣東北)、武都(今甘肅成縣西)、汧(今陝西鳳翔縣南)、雍(今陝西鳳翔縣南)的氐族人中分出十五萬戶,由各宗室統領分布於各方鎮,如古時諸侯一般。不過,被遷移居方鎮的人們因為要與家人分別,都哀傷號哭,路人看見都感到傷心。

王猛於建元十一年(375年)去世,臨死時說:「晉室現在雖然立於偏遠的江南地區,但承繼正統。現在國家最寶貴的就是親近仁德之人以及與鄰國友好。臣死以後,希望不要對東晉有所圖謀。鮮卑、羌虜都是我們的仇敵,終會成為禍患,應該將他們除去,以利社稷。」 希望苻堅先解決國內鮮卑和羌族等其他少數民族對前秦政權的暗藏問題。不過,苻堅在統一北方後仍未聽從王猛之言,著力解決國內民族問題。

苻堅從車師前部王彌窴及鄯善國王休密馱等處聽說西域有高僧鳩摩羅什,苻堅視為國寶,請求西域派羅什入秦遭到拒絕。建元十八年(382年),苻堅派呂光領七萬大軍征伐西域不服前秦要求的,並於次年正月出發。呂光征伐西域龜茲等國大獲全勝,西域諸國歸附前秦。鳩摩羅什也被呂光攜帶身邊。中國境內只剩東晉一地不是前秦版圖,苻堅急於統一中國,開始謀劃出兵東晉。

建元十八年(382年),苻堅大會群臣,自以能得九十七萬兵力,提出親征東晉,統一全國的計劃。當時秘書監朱肜表示支持,尚書左僕射權翼及太子左衞率石越卻都以東晉君臣和睦,且當時為重臣的謝安及桓沖都是人才,皆予以反對。而當時群臣亦各有意見,未有共識。苻堅見此,就說:「像在道旁建房子去問意見,就因聽太多不同的議論而一事無成,我心中自有決斷。」群臣退下後,苻堅留下其弟苻融繼續和他討論,然而苻融亦以天象不利、晉室上下和睦以及兵疲將倦三點為由反對。苻堅因而大怒,苻融後哭著勸諫,並重提王猛死前的話也未能說動苻堅。後名僧釋道安、太子苻宏、幼子中山公苻詵以至寵妃張夫人皆反對伐晉,苻融等人亦屢次上書表示反對,苻堅仍然不肯放棄出兵東晉的計劃,可見苻堅當時其實下了決心。相反,慕容垂向苻堅表示支持出兵東晉,苻堅聽後十分高興,於是向慕容垂說:「與我平定天下的人,就只有你一個呀。」更賜其五百匹布帛。

建元十九年(383年)五月,東晉荊州刺史桓沖出兵襄陽、沔北及蜀地。桓沖於七月退軍後,苻堅便下令大舉出兵東晉,每十丁就遣一人為兵;二十歲以下的良家子但凡有武藝、驍勇、富有、有雄材都拜為羽林郎,最終召得三萬多人。八月,苻堅命苻融率張蚝、梁成和慕容垂等以二十五萬步騎兵作為前鋒,自己則隨後自長安發兵,率領六十餘萬戎卒及二十七萬騎兵的主力,大軍旗鼓相望,前後千里。十月,苻融攻陷壽陽(今安徽壽縣),並以梁成率五萬兵駐守洛澗,阻止率領晉軍主力的謝石和謝玄等人的進攻。當時正進攻晉將胡彬的苻融捕獲胡彬的所派去聯絡謝石的使者,得知胡彬糧盡乏援的困境,於是派使者向正率大軍在項城的苻堅聯絡:「晉軍兵少易擒,但就怕他們會逃走,應該快快進攻他們。」苻堅於是留下大軍,秘密自率八千輕騎直抵壽陽 。然而,晉將劉牢之及後率軍進攻洛澗,擊殺梁成,前秦軍隊潰敗,謝石等於是率領大軍水陸並進,與前秦軍隔淝水對峙。苻堅和苻融此時從壽陽城觀察晉軍,見其軍容整齊,連八公山上的草木都以為是晉軍,於是說:「這也是勁敵,怎能說他們弱呀!」由此悵然失意並有懼色。苻堅及後答允晉軍要他們稍為後撤,讓晉軍渡過淝水作戰的要求,並認為能待晉軍半渡、陷於河中之時出擊,便能將其一舉擊潰。但當前秦大軍開始後退時,先前於襄陽被擒投誠的降將朱序大叫「秦兵敗矣」,秦軍頓時軍心大亂而潰散。苻融親自騎馬入陣中試圖重整亂軍,但反而墮馬被踩死,晉軍於是追擊潰敗的前秦軍,令前秦軍傷亡慘重,連苻堅本人亦中流矢受傷,單騎逃到淮北。

苻堅敗退到淮北時十分飢餓,有平民送他飯菜,苻堅於是給予賞賜,然而該平民卻拒絕,更稱苻堅自取厄困,自己身為其子民即為其子,不圖回報。苻堅因而大感慚愧。及後苻堅與慕容垂的三萬軍隊會合,隨後一直沿途收集逃散的敗兵,到洛陽時聚集了十餘萬人,百官、儀物和軍容都大致齊備了。後苻堅返回長安,哭悼苻融並告罪宗廟後下令大赦,下令鍛煉兵器並監督農務,撫順孤老及陣亡士兵的家屬,試圖重建國家秩序。

隴西鮮卑的乞伏步頹在苻堅出兵東晉時乘機反叛,苻堅派乞伏步頹的侄兒、原降於前秦的乞伏司繁子乞伏國仁出兵討伐,但二人卻相結。淝水戰敗後,乞伏國仁於是裹脅隴西鮮卑諸部叛變,後建立起西秦。而苻堅在洛陽時,不顧權翼的反對,答允讓慕容垂到河北地區安撫民眾及拜謁慕容氏宗廟陵墓。然而慕容垂後來則乘被當時駐鎮鄴城的長樂公苻丕派往鎮壓丁零人翟斌叛亂的機會,聯結丁零人叛秦,並於建元二十年(384年)反與丁零人圍攻鄴城,建立後燕。在圍攻鄴城的同年,慕容泓知道慕容垂的行動亦在關東收集部眾自立,甚為強盛;慕容沖亦在平陽叛變,後投奔慕容泓,慕容泓於是建立西燕,並率眾進攻長安。

為征討大舉叛變的慕容鮮卑,苻堅徵召鉅鹿公苻叡,令其與竇衝及姚萇同討慕容泓,但最終苻叡兵敗戰死,姚萇遣使謝罪卻因苻堅殺其使者而逃到渭北牧馬場,乘機煽動羌族豪帥共五萬餘家歸附,自稱秦王,建立後秦。苻堅自率二萬步騎討伐後秦軍,屢敗後秦軍,更逼得後秦軍中缺水,更有人渴死,但此時天降大雨,後秦軍隊再起,隨後更反敗前秦軍隊。苻堅見慕容沖等已逼近長安,於是回軍長安並組織抵抗,但所派的苻琳、姜宇都兵敗,慕容沖成功佔領阿房城(今陝西西安市西),長安遭圍困。建元二十一年(385年),苻堅在長安宴請群臣,但當時長安已鬧饑荒,發生人食人的事,諸將回家後都吐出宴中吃下的肉來餵饑餓的妻兒。隨後前秦與西燕軍互相攻伐,互有勝負,但在衞將軍楊定被西燕所俘後,苻堅大懼,竟相信他曾經下令禁止的讖諱之言,留太子苻宏留守長安,自己率數百騎及張夫人、苻詵和苻寶、苻錦兩名女兒一同出奔五將山。然而苻堅到五將山後,後秦將領吳忠就來圍攻。苻堅雖見身邊的前秦軍都潰散,但亦神色自若,坐著安然等待吳忠。吳忠及後將苻堅送至新平幽禁。

姚萇及後向苻堅索要傳國玉璽,苻堅張目喝道:「小小羌胡竟敢逼迫天子,五胡的曆數次序,沒有你這個羌人的名字。玉璽已送到晉朝那裏,你得不到的了!」姚萇於是又派人提出苻堅禪讓給他,苻堅亦說:「禪代,是聖賢的事,姚萇是叛賊,有甚麼資格做這事!」苻堅自以平生都待姚萇不薄,甚至在淝水之戰前將「龍驤將軍」這個祖父曾受以及自己殺苻生奪位時有的將軍號珍而重之地封予姚萇,現在姚萇反叛並逼迫他,於是屢次責罵姚萇以求死,並為免姚萇凌辱兩名女兒,於是先殺苻寶和苻錦。八月辛丑日(10月16日),姚萇命人將苻堅絞死於新平佛寺(今彬縣南靜光寺)內,享年四十八歲。張夫人及苻詵亦跟著自殺。

姚萇為掩飾他殺死苻堅的事,故意諡苻堅為壯烈天王。而苻堅去世同年,苻丕得知其死訊,便即位為帝,諡苻堅為宣昭皇帝,上廟號世祖。征西域後回到涼州的呂光得知苻堅去世,亦諡其為文昭皇帝。

苻坚死後就地埋葬,當地人稱“長角塚”。許多人民尊其為苻王爺奉祀之,謂能避免疫病、兵亂。根據《晉書》記載,姚萇被苻堅冤魂作祟,終至發狂,武士欲去救援,竟然打傷其陰部,大出血而死。萇死前還一直跪地叩首,請求苻堅原諒他。

苻堅除了一系列減省奢侈品、鼓勵農業、停止征戰外,更建立學校,重視文教,尤其留心儒學。苻堅曾下令廣收學官,重視經學,郡國弟子員只要通曉一經或以上就獲授官,亦表彰有才德和努力營田之人,令人們都望得朝廷勸勵,崇尚清廉正直,物資亦豐盛。苻堅更每月親臨太學考拔學生,消滅前燕後更在長安祭祀孔子。而王猛亦助苻堅整順風俗,令全國學校漸興。在苻堅治下的關隴地區豐盛安定,地區回復秩序,工商業興盛,一片繁華景象。及至後來王猛去世後,苻堅仍然尊崇儒學,不但命太子、公侯和官員之子以及中外四禁 、二衞、四軍長上 的將士都要受學,連帶後宮亦設有典學,教宮內宦官及宮婢經學。另亦嚴厲禁止老莊以及圖讖學說。後來西域大宛獻馬,苻堅效法西漢漢文帝送還進貢的千里馬,更加命群臣作《止馬詩》送到西域,以示沒有取千里馬的欲望。最終共有四百多人獻詩。

苻堅亦重視生產,遇上天旱不但曾下令節儉及開山澤資源與民共享,亦督導百姓耕種,自己更親身躬耕藉田,讓苟皇后親身養蠶,以示對農業的重視。後又徵集王侯以下及豪門富戶的家僮奴僕共三萬人開通涇水上流,引水灌溉解決關中水旱問題。

苻堅對於前秦這個多民族組成的國家其實沒有作出民族融合的措施。如隴西鮮卑首領乞伏司繁投降後,只遷乞伏司繁到長安,仍留其部眾在隴西地區;前燕鮮卑族人除了慕容氏皇族及部分關東豪族被遷至關中地區外,尚有大部分留在前燕故地,另亦遷原居中山的丁零族人到新安(今河南新安縣);消滅代國後,苻堅雖然由北方匈奴人代領代國遺眾,但仍居北方。在苻洛叛亂被平定後,苻堅則為更好的管理關東以至各地民族,於是從原集中於關中的氐族人分出十五萬戶,各由宗親率領出鎮,如古分封諸侯般管治地方 。然而此舉卻分散了氐族的民族力量,影響對各地的軍事影響力,而移居關中的各少數民族更成前秦的心腹大患 。

史載苻堅「臂垂過膝,目有紫光」。

苻堅與苻法兄弟友好,然而在苻堅即位之初其母苟氏以苻法年長、賢能以及得人心而殺害苻法,苻堅無奈下只有哭著與他訣別,傷心得吐血。後來其子苻陽因憤恨父親無罪遭戮,而謀反,苻堅亦不誅殺。

苻堅寛貸容人,如後趙舊將張平在秦、燕之間搖擺,維持半獨立狀態 ,357年更以并州叛秦,但仍然加以寛貸,署為右將軍。後苻重在洛陽叛變,苻堅也赦而不誅,後更再派他出鎮,終招來苻重聯同苻洛再叛;而苻洛敗後苻堅仍不殺,只流放他到西海郡。另苻堅亦善待亡國貴族,如前涼張氏、前燕慕容氏等都沒有進行屠殺,甚至頗見親待。

苻堅執政前期大推善政,崇尚節儉,然而在王猛死後,苻堅卻因聽後趙前將作功曹熊邈講述後趙宮室器具的規模,下令以其為將作長史,大修舟艦、兵器,並以金銀裝飾,講求精巧,一改之前節儉之風。慕容農亦因而說:「自從王猛死後,秦的法制日漸頹靡,今日又著重奢侈,大禍將來了。」

苻堅初年虛心接納臣下的諫言,如即位初期曾經登龍門,向群臣展現他甚為滿足於關中的穩固。而權翼、薛讚當時則以夏、商、周、秦四個朝代由興盛的基礎而到最終遭他人所滅,表達出修備德行的重要,穩固的地勢並不足以固國。苻堅聽後大喜,隨後就施行一系列新政與民休息。後苻堅在鄴附近狩獵十多日,樂而忘返,亦聽從伶人王洛的勸言,不再出獵。但後來苻堅卻在出兵東晉等事上聽不下諫言,只想聽到支持自己的論調。

苻堅生母因為年輕守寡,於是寵幸將軍李威,當時史官亦記載此事。但苻堅後來看起居注和史官所著的著作發現載有這種事,於是發怒燒書並大檢史官,要加罪於史官,因著作郎趙淵、車敬等已死才了事。

《晉書》史臣曰:「永固雅量瓌姿,變夷從夏,叶魚龍之遙詠,挺莫苻之休徵,克翦姦回,纂承偽曆,遵明王之德教,闡先聖之儒風,撫育黎元,憂勤庶政。……乃平燕定蜀,擒代吞涼,跨三分之二,居九州之七,遐荒慕義,幽險宅心,因止馬而獻歌,託棲以成頌,因以功侔曩烈,豈直化洽當年!雖五胡之盛,莫之比也。既而足己夸世,複諫違謀,輕敵怒鄰,窮兵黷武。懟三正之未叶,恥五運之猶乖,傾率土之師,起滔天之寇,負其犬羊之力,肆其吞噬之能。自謂戰必勝,攻必取,便欲鳴鷥禹穴,駐蹕疑山,疏爵以侯楚材,築館以須歸命。曾鬥知人道助順,神理害盈,雖矜涿野之強,終致昆陽之敗。道使文渠候隙,狡寇伺間,步搖啟其禍先,燒當乘其亂極,宗社遷於他族,身首罄於賊臣,賊戒將來,取笑天下,豈不哀哉!豈不謬哉!」

《晉書》贊曰:「永固禎祥,肇自龍驤。垂旒負扆,竊帝圖王。患生縱敵,難起矜強。」

苻洪:「此兒姿貌瓖偉,質性過人,非常相也。」

徐統:「此兒有霸王之相。」又曰:「苻郎骨相不恒,後當大貴,但僕不見。」

薛禮、權翼:「非常人也!」

苻廋:「苻堅、王猛,皆人傑也。」

司馬光:「夫有功不賞,有罪不誅,雖堯、舜不能為治,況他人乎!秦王堅每得反者輒宥之,使其臣狃於為逆,行險徼幸,力屈被擒,猶不憂死,亂何自而息哉!《書》曰:『威克厥愛,允濟;愛克厥威,允罔功。』《詩》云:『毋縱詭隨,以謹罔極;式遏寇虐,無俾作慝。』今堅違之,能無亡乎!」又言:「論者皆以為秦王堅之亡,由不殺慕容垂、姚萇故也。臣獨以為不然。許劭謂魏武帝治世之能臣,亂世之姦雄。使堅治國無失其道,則垂、萇皆秦之能臣也,烏能為亂哉!堅之所以亡,由驟勝而驕故也。魏文侯問李克,吳之所以亡,對曰:『數戰數勝。』文侯曰:『數戰數勝,國之福也,何故亡?』對曰:『數戰則民疲,數勝則主驕,以驕主御疲民,未有不亡者也。』秦王堅似之矣。」

歷史學家陳登原認為苻堅有四大善事——文學優良,內政修明,大度容人,武功赫赫。后人对待亡国贵族往往以苻坚之仁为戒,选择屠杀殆尽。

呂思勉:「苻堅在諸胡中,尚為稍知治體者,然究非大器。嘗縣珠簾於正殿,以朝群臣。宮宇、車乘、器物、服御、乘以珠璣、琅玕、奇寶、珍怪飾之。雖以尚書裴元略之諫,命去珠簾,且以元略為諫議大夫,然此特好名之為,其諸事不免淫侈,則可想見矣。」後又以苻堅以慕容沖及前燕清河公主姐弟皆有美色而皆寵幸,直斥其「荒淫」。又指其命呂光征西域是「蓋一欲誇耀武功,一亦貪其珍寶也。」又曰:「堅知晉終為秦患,命將出師之不足以晉,而未知躬自入犯之更招大禍,仍是失之於疏;而其疏,亦仍是失之於驕耳。」

著名作家柏楊於柏楊版資治通鑑第25冊的序言中寫到:在大分裂時代中,苻堅大帝以超時代的睿智之姿,出現舞台,為苦難的北中國人民,帶來一個太平盛世。

據說苻堅生母苟氏曾在漳水遊玩,並在西門豹寺祈子,在當晚夢與神交,於是懷有苻堅。十二個月後苻堅才出生,當時天上有神光照耀門庭,苻堅背上亦有紅色字,寫著「草苻臣又土王咸陽」。後苻洪以此及「艸付應王」的讖言改姓苻氏。

姚苌曾把苻坚的屍體挖出来鞭尸,脫掉衣服用荆棘裹起来,再以土坑埋掉。苻坚的冤魂作祟非常顯著,姚萇後來諸事不順,屢屢敗陣,認為是苻堅顯靈,於是也在軍中樹立苻堅像祈求道:「新平之禍,不是臣姚萇的錯啊,臣的兄長姚襄從陝州北渡,順著道路要往西邊去,像狐狸死時把頭朝向原本洞穴一樣,只是想要見一見鄉里啊。陛下與苻眉攔阻於路上攻擊他,害他不能成功就死了,姚襄遺命臣一定要報仇。苻登是陛下的遠親亦想復仇,臣為自己的兄長報仇,又怎麼說是辜負了義理呢?當年陛下封我為龍驤將軍,跟我說:『朕從龍驤將軍當上了皇帝,卿也好好努力罷!』這明明白白的詔諭非常顯然,好像還在耳邊一樣。陛下已經過世成為神明了,怎麼會透過苻登而謀害臣,忘卻當年說的話呢!現在為陛下立神像,請陛下的靈魂進入這裏,不要計較臣的過失了,聽臣至誠的禱告。」 不過姚萇戰況仍未有改善,反而睡不安穩,並招來苻登批評「古今以來,豈有人殺了主公卻反而為主公立神像請求賜福?他期望會有好處嗎?」姚萇終毀了苻堅神像。據說姚萇死前曾夢見過苻堅率天官、鬼兵去襲擊他,期間他被救援自己的士兵誤傷陰部至大量出血。醒後就發現陰部腫脹,醫者刺腫處則如夢中一樣大量出血,一石有餘。,如此嚇得姚萇發狂胡言,又求苻堅原諒,姚萇不久傷重身亡,臨終前跪伏床頭,叩首不已。

據《湧幢小品》言:傳聞死於新平寺之苻堅託夢該寺寺主摩訶,望該寺改為祭祀苻堅及侍衛十餘人的廟宇。住持不從,該寺所在縣鎮,果然死疫相繼,後不得已,即尊其靈示,改廟後,果真無疾。

道教信徒衍其義,逢瘟疫競建祠避禍,稱為苻王爺、苻家神,並於每年正月初二以太牢奉之,稱為祭苻家神。祭苻家神為台灣道教現有祭典之一,祭典日為每年農曆正月初二。

\subsubsection{永光}

\begin{longtable}{|>{\centering\scriptsize}m{2em}|>{\centering\scriptsize}m{1.3em}|>{\centering}m{8.8em}|}
  % \caption{秦王政}\
  \toprule
  \SimHei \normalsize 年数 & \SimHei \scriptsize 公元 & \SimHei 大事件 \tabularnewline
  % \midrule
  \endfirsthead
  \toprule
  \SimHei \normalsize 年数 & \SimHei \scriptsize 公元 & \SimHei 大事件 \tabularnewline
  \midrule
  \endhead
  \midrule
  元年 & 357 & \tabularnewline\hline
  二年 & 358 & \tabularnewline\hline
  三年 & 359 & \tabularnewline
  \bottomrule
\end{longtable}

\subsubsection{甘露}

\begin{longtable}{|>{\centering\scriptsize}m{2em}|>{\centering\scriptsize}m{1.3em}|>{\centering}m{8.8em}|}
  % \caption{秦王政}\
  \toprule
  \SimHei \normalsize 年数 & \SimHei \scriptsize 公元 & \SimHei 大事件 \tabularnewline
  % \midrule
  \endfirsthead
  \toprule
  \SimHei \normalsize 年数 & \SimHei \scriptsize 公元 & \SimHei 大事件 \tabularnewline
  \midrule
  \endhead
  \midrule
  元年 & 359 & \tabularnewline\hline
  二年 & 360 & \tabularnewline\hline
  三年 & 361 & \tabularnewline\hline
  四年 & 362 & \tabularnewline\hline
  五年 & 363 & \tabularnewline\hline
  六年 & 364 & \tabularnewline
  \bottomrule
\end{longtable}

\subsubsection{建元}

\begin{longtable}{|>{\centering\scriptsize}m{2em}|>{\centering\scriptsize}m{1.3em}|>{\centering}m{8.8em}|}
  % \caption{秦王政}\
  \toprule
  \SimHei \normalsize 年数 & \SimHei \scriptsize 公元 & \SimHei 大事件 \tabularnewline
  % \midrule
  \endfirsthead
  \toprule
  \SimHei \normalsize 年数 & \SimHei \scriptsize 公元 & \SimHei 大事件 \tabularnewline
  \midrule
  \endhead
  \midrule
  元年 & 365 & \tabularnewline\hline
  二年 & 366 & \tabularnewline\hline
  三年 & 367 & \tabularnewline\hline
  四年 & 368 & \tabularnewline\hline
  五年 & 369 & \tabularnewline\hline
  六年 & 370 & \tabularnewline\hline
  七年 & 371 & \tabularnewline\hline
  八年 & 372 & \tabularnewline\hline
  九年 & 373 & \tabularnewline\hline
  十年 & 374 & \tabularnewline\hline
  十一年 & 375 & \tabularnewline\hline
  十二年 & 376 & \tabularnewline\hline
  十三年 & 377 & \tabularnewline\hline
  十四年 & 378 & \tabularnewline\hline
  十五年 & 379 & \tabularnewline\hline
  十六年 & 380 & \tabularnewline\hline
  十七年 & 381 & \tabularnewline\hline
  十八年 & 382 & \tabularnewline\hline
  十九年 & 383 & \tabularnewline\hline
  二十年 & 384 & \tabularnewline\hline
  二一年 & 385 & \tabularnewline
  \bottomrule
\end{longtable}


%%% Local Variables:
%%% mode: latex
%%% TeX-engine: xetex
%%% TeX-master: "../../Main"
%%% End:

%% -*- coding: utf-8 -*-
%% Time-stamp: <Chen Wang: 2019-12-19 10:15:33>

\subsection{哀平帝\tiny(385-386)}

\subsubsection{生平}

秦哀平帝苻丕(4世紀-386年),字永叔(或作永敘),略陽臨渭(今甘肅秦安)氐族人。前秦皇帝,宣昭帝苻堅的庶長子,淝水之戰後與後燕君主慕容垂一度相持於鄴城。並在苻堅死後繼承帝位,繼續與後秦、西燕及後燕勢力對抗。最終在進攻洛陽時遭晉軍所殺,死後獲諡為哀平皇帝。

苻丕少時聰慧好學,博通經史。苻堅曾經與他談將略,嘉許了他並命鄧羌教他兵法。苻丕的文武才幹不及叔父苻融,不過他當將領時善於籠絡士卒之心。永興元年(357年)苻堅稱天王時受封為長樂公。

建元四年(368年),苻堅在攻滅叛亂的雍州刺史苻武後,以苻丕為雍州刺史。建元六年(370年)因取消雍州而離任,但次年苻丕就因雍州復置而任使持節、征東大將軍、雍州刺史。後遷征南大將軍,都督征討諸軍事,守尚書令。建元十四年(378年)二月,奉命與苟萇等進攻東晉襄陽。當時前秦軍很快就攻下了襄陽外城,守將朱序只得固守內城,苻丕於是打算急攻內城。然而最終卻聽從了苟萇長期圍困,待其自降的策略,雖然及得慕容垂攻陷南陽郡後與苻丕會合,秦軍仍只一直圍困襄陽;而當時的荊州刺史桓沖以及在次年受命領兵救援襄陽的劉波皆因畏懼秦軍而未敢前進,都沒有起到實質作用。不過,朱序仍一直堅持到年末,前秦御史中丞李柔因而彈劾苻丕等人圍攻襄陽近一年仍未能攻陷,耗費日深而無收效。苻堅亦下詔苻丕要以攻取襄陽贖罪,並命人賜劍苻丕,明言若果不能在下一年春天攻下襄陽就要苻丕以劍自殺。苻丕得詔後惶恐,並下令各軍加緊進攻,終於在次年二月攻下襄陽。

建元十六年(380年),苻堅為加強管理關東領土,於是決定分十五萬戶關中氐族人並分配給宗親重臣,在他們帶領下分駐各重鎮,如同古代諸侯。苻丕則為都督關東諸軍事、征東大將軍、冀州牧,派遣他到鄴鎮守。

苻堅在建元十九年(383年)的淝水之戰大敗給晉軍後返率敗軍回長安,並在洛陽答允讓冠軍將軍慕容垂出撫河北地區。慕容垂到鄴城西南的安陽時修書苻丕,而苻丕知慕容垂北來就已思疑他圖謀作亂,但仍親身迎接,又聽從侍郎姜讓的諫言,放棄襲殺慕容垂的計劃。不久,在新安的丁零人翟斌起兵叛變,苻堅命慕容垂討伐。苻丕當時自覺慕容垂長在鄴城令自己終日都提防他,於是想趁此機會將慕容垂調離鄴城,更希望翟斌和慕容垂打得兩敗俱傷,令自己能從容控制他們,於是給了慕容垂兩千弱兵以及差劣的兵器,並以苻飛龍領一千氐族騎兵作為其副手,作提防監視之用。不久慕容垂請求拜謁前燕在鄴城宗廟遭苻丕拒絕,微服而入亦被亭吏阻止,令其殺掉亭吏,燒亭而去;慕容垂出發後又因知道苻丕想用苻飛龍除掉自己,所以就借機殺了苻飛龍,並開始招集兵士,更密召留鄴的慕容農、慕容楷等出城起兵響應自己。

建元二十年(384年)春,苻丕大宴賓客卻請不來慕容農等,調查三天才知他們已在列人起兵了,而慕容垂及後亦自稱燕王,率兵進攻鄴城。苻丕派了重將石越討伐慕容農等但石越卻兵敗被殺,石越之死更令當地人心騷動。隨著前燕舊臣想相繼響應慕容垂並到鄴城會同慕容垂進攻,慕容垂更寫信給苻丕及苻堅,向其陳述利害,想苻堅放棄鄴城,送苻丕回長安,但遭二人憤怒地拒絕,並回信嚴厲指責慕容垂叛秦。二月,慕容垂就開始進攻鄴城,直至八月仍未能攻下鄴城,但城內糧草已盡,要以松木餵飼戰馬。苻丕向張蚝及并州刺史王騰請兵不得,亦不想向東晉求援;此時謝玄率兵北伐,苻丕派兵抵抗但失敗,終令苻丕屈服,寫信給謝玄說:「我想向你求糧,以西赴國難,當我與援軍相接時就會交鄴城給你。若果不能西進而長安失陷,請你領兵助我保護鄴城。」不過姜讓、最早請苻丕南附東晉的司馬楊膺以及擔任使者的焦逵皆認為苻丕至此仍不肯放下身段,認定事必無成,反而自己修改苻丕的信,改成願意在晉軍來後向東晉歸降,更決定若苻丕屆時不肯就想辦法逼他就範。而當時慕容垂亦派兵圍困鄴城,只留西走長安的路,仍願苻丕自願棄城;而謝玄亦答應出兵救鄴,不但派劉牢之等領二萬兵作援,亦運二千斛米以解城中糧荒。

就在次年(385年)劉牢之北行至枋頭時,楊膺等人改寫苻丕書信並想逼苻丕就範的事被揭發,苻丕於是殺害他們,而因焦逵亦向謝玄等提及此事,令劉牢之聞訊後盤桓不進。此時慕容垂亦因鄴城久久未下而想先取冀州,於是調了慕容農到鄴城。及後因應劉牢之進攻黎陽,苻丕趁慕容垂出兵,留慕容農守鄴圍的機會試圖突圍但失敗,慕容垂亦在擊退劉牢之後回軍鄴城。四月,劉牢之在鄴擊敗慕容垂,終解了鄴城之圍,令慕容垂北走。雖然劉牢之追擊燕軍失敗還須苻丕救援,但苻丕最終都能夠率眾到枋頭獲得晉軍糧食,解決部眾缺糧問題。然而苻丕並非真心與晉合作,亦不曾想放棄鄴城,於是在重返鄴城時就與晉將檀玄發生了戰鬥,終由苻丕取勝並重奪鄴城。

燕、秦兩軍至此時已經相持一整年了,弄得幽、冀地區發生饑荒,人食人且城池都蕭條;而且當時長安亦受到西燕軍隊的攻擊,苻丕於是在當地收兵並要西赴長安。幽州刺史王永因為抵抗不了燕軍進攻而率兵退至壺關,並派使者招請苻丕,苻丕於是率鄴城中六萬多人西赴潞川,並獲張蚝和王騰迎至晉陽。苻丕到了晉陽才知苻堅已經被姚萇所殺,於是發喪並於晉陽南即位為帝,改年號為太安。

在王永等人的協助下,關中及隴右的前秦遺眾都相繼起兵響應苻丕,以對抗慕容氏及姚氏的勢力。太安二年(386年),苻丕留戍晉陽及壺關,自率四萬兵進屯平陽。西燕君主慕容永見此擔憂抵抗不了秦軍,於是請求苻丕讓他取道東歸河北會合慕容垂。但苻丕拒絕並命左丞相王永、俱石子等進攻慕容永。西燕軍於是在襄陵與王永所率的秦軍發生戰鬥,王永及俱石子皆兵敗被殺,苻丕以兵敗,更怕他一直猜忌的苻纂趁他新敗而對其不利,於是率數千南奔東垣,更圖進攻當時受東晉控制的洛陽。晉將馮該就從陝城出兵邀擊苻丕,最終苻丕被殺,除了苻纂等率數萬兵出走杏城外,苻丕統下的官員皆為西燕所得。苻丕死後,族子苻登繼位,諡苻丕為哀平皇帝。

\subsubsection{太安}

\begin{longtable}{|>{\centering\scriptsize}m{2em}|>{\centering\scriptsize}m{1.3em}|>{\centering}m{8.8em}|}
  % \caption{秦王政}\
  \toprule
  \SimHei \normalsize 年数 & \SimHei \scriptsize 公元 & \SimHei 大事件 \tabularnewline
  % \midrule
  \endfirsthead
  \toprule
  \SimHei \normalsize 年数 & \SimHei \scriptsize 公元 & \SimHei 大事件 \tabularnewline
  \midrule
  \endhead
  \midrule
  元年 & 385 & \tabularnewline\hline
  二年 & 386 & \tabularnewline
  \bottomrule
\end{longtable}


%%% Local Variables:
%%% mode: latex
%%% TeX-engine: xetex
%%% TeX-master: "../../Main"
%%% End:

%% -*- coding: utf-8 -*-
%% Time-stamp: <Chen Wang: 2019-12-19 10:16:57>

\subsection{高帝\tiny(386-394)}

\subsubsection{生平}

秦高帝苻登(343年-394年),字文高,略陽臨渭(今甘肅秦安)氐族人,十六国前秦皇帝,苻堅族孫,建節將軍苻敞之子。在苻堅遭後秦君主姚萇殺害後,苻登曾領率氐族殘餘力量於關隴地區對抗後秦,後更被擁立為前秦皇帝。苻登起初屢次獲勝,但終敗給後秦,更遭俘殺。死後獲上廟號太宗,諡高皇帝。

苻登年輕時就勇猛威武,有雄壯的氣慨,但為人粗豪好險而不修小節,並不受苻堅重視。苻登長大成人後卻一改舊習,謹慎厚道,亦看典籍。苻堅曾以其為殿上將軍、羽林監、揚武將軍、長安令,後來因過失被降為狄道長。

淝水之戰後,關中地區大亂,苻登逃到河州牧毛興駐守的枹罕(今甘肅臨夏市)。苻登兄苻同成是毛興的長史,於是請毛興以苻登任其司馬。苻登當時表現得器量不凡,喜歡設奇謀,而他對事物的分析連毛興也十分佩服,然而卻因受毛興所憚而沒有獲重用。

前秦太安二年(386年),時與後秦姚碩德對抗的毛興亦同時與同屬前秦的益州牧王廣及秦州牧王統作戰,頻繁的戰事令厭戰的氐人將其殺害。毛興臨死時就表示苻登能夠消滅姚碩德[1]。同年七月,苻登獲眾人推舉,取代被指年老的氐豪衞平統領原毛興部眾,自稱使持節、都督隴右諸軍事、撫軍大將軍、雍河二州牧、「略陽公」,並即率兵五萬攻佔南安,又派使者向當時前秦皇帝苻丕請求任命。苻丕亦應其自稱授官,並以其為征西大將軍、開府儀同三司、南安王。苻登佔據南安後獲當地胡、漢共三萬多戶歸附,聲勢漸盛,於是進攻姚碩德,並在胡奴阜大敗前往救援的姚萇,更令其身受重傷。

十月,苻丕進攻洛陽時被東晉將領所殺,當時苻丕子苻懿及苻昶都被帶到南安,苻登於是打算立苻懿為帝。但部眾都力勸苻登立長君,並指出非苻登一人不可。苻登於是即位為帝,改元「太初」,立了苻懿為太弟。

當時前秦宗室苻纂為另一軍事力量,他支持苻登令前秦聲勢大盛,並曾與楊定於涇陽(今陝西涇陽縣)大敗姚碩德,更圖謀攻取後秦都城長安。不過苻纂不久卻因不肯自立為帝而遭其弟苻師奴殺害,苻師奴亦遭姚萇擊敗,部眾遭後秦吸納,進攻長安行動亦告吹。

太初三年(388年)二月,苻登與姚萇各據朝那(今寧夏彭陽縣)及武都相持不下,互有勝負。當時關西豪傑見後秦久久不能消滅前秦勢力,很多都轉歸前秦。姚萇終於十月退還根據地安定,苻登亦到新平取軍糧以解軍中饑饉的狀況,並自率萬餘人兵圍姚萇軍營,四面以哭聲震動其軍心;不過姚萇亦命軍人以哭聲回應,苻登見不成功就退兵。

太初四年(389年),苻登在大界留下輜重,自率萬多名輕騎兵進攻安定,先後擊敗安定羌密造保及後秦將吳忠等,並於八月進逼安定。但姚萇卻奇兵夜襲大界,殺害留守的毛氏並擒獲數十名名將,擄掠五萬多人。苻登見此唯有退守根據地胡空堡(今陝西省彬縣西南)。太初六年(391年),苻登因苟曜作為內應而進攻後秦,並擊敗姚萇,殺後秦將吳忠,但姚萇立刻重整軍勢再戰,苻登這次大敗,退兵至郿縣(今陝西眉縣)。同年苻登先後進攻新平及安定,但都遭姚萇擊敗。而當時氐族人強金槌叛歸後秦,兩年前勇略過人的羌人雷惡地亦因遭苻登所忌憚而出奔後秦,次年驃騎將軍沒弈干亦叛降後秦,這些事件都削弱了苻登的力量。

太初七年(392年),苻登以姚萇患病而出兵安定,但在城外九十多里就遭姚萇所派的軍隊攻擊,被逼退還;而姚萇更特意在夜裏命軍隊旁出跟隨苻登軍,苻登聽聞姚萇軍營空無一人,驚懼得說:「他究竟是甚麼人,離開時我不知道,來到時我亦不察覺,人說他快死了,突然又來了。朕和這個羌人活在同一年代,根本是不幸。」

太初九年(394年),苻登知姚萇已死,於是十分高興,並盡率大軍進攻後秦。至夏季,苻登進攻廢橋以得水源,但為後秦將尹緯所阻,部分士兵更渴死。苻登因而急攻尹緯,而尹緯卻大敗苻登,兵眾潰散,苻登單騎逃返原由其弟苻廣留守的雍城(今陝西鳳翔縣南)卻發現苻廣已棄城,另一根據地胡空堡亦遭留守的太子苻崇所棄,苻登無處容身,只有逃到平涼(今甘肅平涼市),收集部眾據守馬髦山。苻登及後向乞伏乾歸求救,得其命乞伏益州領兵救援,卻就在七月苻登率兵迎接乞伏益州時就遇上後秦軍,苻登被生擒並處決,享年五十二歲。

其子苻崇在湟中稱帝,追諡苻登為高皇帝,上廟號太宗。

苻登與後秦軍作戰多年,其為史傳所載的生平事跡多為他在軍旅中的事跡。

苻登取代衞平後,銳意出兵,但當年天旱,兵眾都吃不飽,苻登於是都將戰爭中殺死的敵軍都叫做「熟食」,更向軍人說:「你們早上作戰,黃昏就能吃飽肉了,還怕飢餓麼!」士兵於是就吃屍肉為生,吃飽後都有氣力戰鬥,逼得姚萇急召姚碩德:「你再不來,我們就要被苻登吃光了。」

苻登曾在軍中設苻堅神主,每次作戰或有所決定都會向其稟告。而苻登即皇帝位後要出兵後秦,亦向苻堅稟告,發言後欷歔流涕,更感染了將士們,令他們都在鎧甲和矛上都刻上「死休」二字,以作至死方休之志。立神主一事甚至令時屢敗的姚萇認為這真是苻堅神助,也一度在軍中設了苻堅像。

\subsubsection{太初}

\begin{longtable}{|>{\centering\scriptsize}m{2em}|>{\centering\scriptsize}m{1.3em}|>{\centering}m{8.8em}|}
  % \caption{秦王政}\
  \toprule
  \SimHei \normalsize 年数 & \SimHei \scriptsize 公元 & \SimHei 大事件 \tabularnewline
  % \midrule
  \endfirsthead
  \toprule
  \SimHei \normalsize 年数 & \SimHei \scriptsize 公元 & \SimHei 大事件 \tabularnewline
  \midrule
  \endhead
  \midrule
  元年 & 386 & \tabularnewline\hline
  二年 & 387 & \tabularnewline\hline
  三年 & 388 & \tabularnewline\hline
  四年 & 389 & \tabularnewline\hline
  五年 & 390 & \tabularnewline\hline
  六年 & 391 & \tabularnewline\hline
  七年 & 392 & \tabularnewline\hline
  八年 & 393 & \tabularnewline\hline
  九年 & 394 & \tabularnewline
  \bottomrule
\end{longtable}


%%% Local Variables:
%%% mode: latex
%%% TeX-engine: xetex
%%% TeX-master: "../../Main"
%%% End:

%% -*- coding: utf-8 -*-
%% Time-stamp: <Chen Wang: 2019-12-19 10:17:55>

\subsection{苻崇\tiny(394)}

\subsubsection{生平}

苻崇(4世紀-394年),十六国前秦末主。

苻崇是高帝苻登之子,太初三年(388年)八月立為太子。

太初九年(394年)七月,苻登兵敗,被後秦姚興殺死,崇逃到湟中即帝位,改元延初。十月,苻崇被隴西鮮卑的梁王乞伏乾歸(後來的西秦王)驅逐,逃到隴西王楊定那裡。楊定率領二萬人與苻崇共攻乾歸,先勝後大敗,定及崇俱被殺,乾歸盡有隴西之地。

前秦太子苻宣投靠仇池楊盛,不再設置郡縣,前秦亡。

\subsubsection{延初}

\begin{longtable}{|>{\centering\scriptsize}m{2em}|>{\centering\scriptsize}m{1.3em}|>{\centering}m{8.8em}|}
  % \caption{秦王政}\
  \toprule
  \SimHei \normalsize 年数 & \SimHei \scriptsize 公元 & \SimHei 大事件 \tabularnewline
  % \midrule
  \endfirsthead
  \toprule
  \SimHei \normalsize 年数 & \SimHei \scriptsize 公元 & \SimHei 大事件 \tabularnewline
  \midrule
  \endhead
  \midrule
  元年 & 394 & \tabularnewline
  \bottomrule
\end{longtable}


%%% Local Variables:
%%% mode: latex
%%% TeX-engine: xetex
%%% TeX-master: "../../Main"
%%% End:


%%% Local Variables:
%%% mode: latex
%%% TeX-engine: xetex
%%% TeX-master: "../../Main"
%%% End:
  
%% -*- coding: utf-8 -*-
%% Time-stamp: <Chen Wang: 2019-12-19 10:22:14>


\section{后秦\tiny(384-417)}

\subsection{简介}

后秦(384年-417年,或稱姚秦)是十六国时期羌人贵族姚苌建立的政权。

前秦苻坚淝水兵败后,关中空虚,原降于前秦的羌人贵族姚苌在渭北叛秦,晋太元九年(384年)自称“万年秦王”,都北地(今陕西耀县东南)。次年(385年)擒杀苻坚。太元十一年(386年)姚苌称帝于长安(今陕西西安西北),国号秦,史称后秦。

其国号以所统治地区为战国时秦国故地为名。《十六国春秋》始称“后秦”,以别于前秦和西秦,后世袭用之。又以王室姓姚而别称姚秦。

统治地区包括今陕西、甘肃东部和河南部分地区。

后秦建初七年(393年)姚苌卒,子姚兴继位,攻杀前秦苻登,扫除前秦残部;又乘后燕灭西燕,尽占原西燕河东之地;弘始元年(399年)乘东晋内乱,陷洛阳,淮汉以北诸城多请降,国势遂与后燕相当。伐後涼,得鳩摩羅什。是年,法顯從長安出發西行求經。

弘始十八年(416年)姚兴卒,子姚泓继位。國內曾歸降的多族勢力趁機反叛,乘丧发兵。东晋劉裕派檀道濟等北伐,陷洛阳。后秦宗室皇弟為奪位反叛,被姚泓消滅。永和二年(417年)东晋圍攻长安,姚泓舉家投降,竟被劉裕滅族,后秦亡。后秦共存在32年(384-417)。

\subsection{景元帝生平}

姚弋仲(280年-352年),南安郡赤亭(今甘肅省隴西縣西)羌人。西晉末期至五胡十六國前期人物,南安羌族酋長,先後降於前趙、後趙及東晉。姚弋仲亦是後秦開國君主姚萇之父。

姚弋仲為燒當羌後代,漢光武帝建武中元年間其先祖滇虞因侵擾東漢而受東漢朝廷討伐,被逼逃亡出塞。至遷那時內附,至此獲居於南安郡赤亭縣。姚弋仲是遷那的五世孫,其父是曹魏鎮西將軍、西羌都督柯回。

姚弋仲年少聰明而勇猛,英明果斷,雄武剛毅,不治產業而以收容救濟為務,故很受眾人敬服。永嘉六年(312年),時值永嘉之亂次年,姚弋仲舉眾東遷榆眉,胡漢人民扶老攜幼跟隨者有數萬,姚弋仲並於此時自稱護西羌校尉、雍州刺史、「扶風公」。太寧元年(323年),前趙帝劉曜消滅盤據隴西的陳安後,關隴地區的氐、羌部落都向前趙請降,劉曜就以姚弋仲為平西將軍,封平襄公。

劉曜於咸和三年(328年)敗於後趙天王石勒後,留守長安的太子劉熙於次年棄守長安,出奔上邽(今甘肅天水市),導致關中大亂,後趙乘時進取關中。不久石虎更領兵攻下上邽,消滅前趙殘餘勢力,姚弋仲亦於是歸降後趙,並獲石虎推薦行安西將軍、六夷左都督。姚弋仲當時向石虎建議遷移隴上豪族,以削弱其實力並充實京畿地區,得石虎聽從。

至咸和八年(333年),後趙帝石勒去世,石虎以丞相掌握朝權,因著姚弋仲前言及氐酋蒲洪的勸言,於是遷關中豪族及氐、羌共十萬戶到首都襄國(今河北邢台)所在的關東地區,並命姚弋仲為奮武將軍、西羌大都督,封襄平縣公,讓他的部眾遷居於清河郡的灄頭(今河北棗強縣東北)。後又遷持節、十郡六夷大都督、冠軍大將軍。

永和五年(349年),高力督梁犢與其部眾兵變,聲勢浩大,並擊敗石虎派往討伐的李農。石虎當時大為恐懼,並召姚弋仲與燕王石斌討伐梁犢。姚弋仲率其部眾八千餘人輕騎至首都鄴城(今河北省臨漳縣)。當時石虎已重病,不能馬上接見,只先賞賜姚弋仲酒食。姚弋仲怒而不食,說:「召我擊賊,豈來覓食邪!我不知上存亡,若一見,雖死無恨。」石虎接見後加授姚弋仲使持節、侍中、征西大將軍,賜鎧馬。隨後姚弋仲就不辭而出,策馬南奔,大破叛軍,斬梁犢。因功加劍履上殿,入朝不趨,進封西平郡公。

同年,石虎去世,太子石世繼位,而征梁犢歸來的姚弋仲、蒲洪等人亦於此時回軍,並與彭城王石遵相遇於李城(今河南溫縣),並共同勸說石遵起兵奪位。石遵隨後起兵,不久就殺石世繼位,並讓冉閔掌有兵權。然而不久冉閔就廢殺石遵,立石鑒為帝,掌握朝政。新興王石祗於是與姚弋仲及蒲洪連兵,移檄討伐冉閔。次年,冉閔殺石鑒並誅殺石氏宗室,姚弋仲就率眾討伐冉閔,移兵至混橋。不久石祗於襄國即位為後趙帝,以姚弋仲為右丞相,封親趙王,並殊有禮待。永和七年(351年),冉閔圍攻襄國,姚弋仲命其子姚襄率兵救援石祗,並配合後趙太尉張舉的行動,遣使向前燕求援。最終在汝陰王石琨、姚襄、前燕三軍以及襄國守軍夾擊之下,圍城的冉閔兵敗,敗退鄴城。雖然姚襄取勝,但因為沒有應姚弋仲在出發前所要求的擒得冉閔,遭姚弋仲以杖打一百責罰。而同年石祗亦被殺,後趙滅亡,姚弋仲於是遣使向東晉投降,獲授使持節、六夷大都督、都督江淮諸軍事、車騎大將軍、儀同三司、大單于,封高陵郡公。

次年(352年),姚弋仲在患病時向諸子說:「石氏厚待我,我本來想盡力幫助他們。而今天石氏已經滅了,中原無主;我死了以後,你們要盡快歸降晉室,並固守臣節,不要做不義的事呀!」及後去世,享年七十三歲。其五子姚襄續統其眾。

姚襄後為苻生所敗,弋仲的靈柩為其所獲,苻生以王禮葬弋仲於天水冀縣。後來,姚弋仲第二十四子姚萇稱後秦帝時,追諡姚弋仲為景元皇帝,廟號始祖,其墓稱為「高陵」,置园邑五百家。现为天水市域重点文物古迹。

《晉書》載姚弋仲個性「清儉鯁直,不修威儀,屢獻讜言,無所回避」,連殘暴的石虎也敬重三分,334年,石虎廢皇帝石弘自立,弋仲稱病不來朝賀,經石虎不斷召見才至,弋仲正色向石虎說:「奈何把臂受託而反奪之乎!」石虎也因為弋仲正直而不責怪他。後石虎一名寵姬的弟弟任武城左尉,擾亂姚弋仲的部眾,姚弋仲就捕捉並數責他,更命人殺了他,雖然最終因對方叩頭至流血作請求以及左右的諫言而不殺他,但也見姚弋仲為事剛直,毫不顧忌對方背景。後討梁犢前得石虎召見,又責備患病的石虎:「兒死,愁邪,何為而病?兒幼時不擇善人教之,使至於為逆;既為逆而誅之,又何愁焉!且汝久病,所立兒幼,汝若不愈,天下必亂,當先憂此,勿憂賊也!犢等窮困思歸,相聚為盜,所過殘暴,何所能至!老羌為汝一舉了之」除了看見他梗直而言,直指他教子無道而導致石宣殺害太子石韜的事件發生,亦見其不論尊卑皆直稱「汝」的行為,連作為皇帝的石虎也不例外。

姚弋仲曾有一個叫馬何羅的部曲曾在張豺主政時叛歸對方。後因石世被廢,張豺亦遭誅殺,馬何羅於是回到姚弋仲那裏。當時眾人都建議姚弋仲殺了他,但姚弋仲就以「招才納奇」為由寬恕他,不但不作加害,反以其為參軍。

姚弋仲在後趙末年一直顯得忠於石氏,不過《資治通鑑》亦有見載於後趙混亂,冉閔篡權時姚弋仲與蒲洪爭奪關中的行動。

\subsection{魏武王生平}

姚襄(约331-357年),字景國,南安赤亭(今甘肅省隴西縣西)羌族酋長,五胡十六國時期諸侯、軍閥,是姚弋仲的第五子,也是後秦開國君主姚萇之兄。

姚襄父親姚弋仲是南安羌酋長,在後趙滅前趙後歸降後趙並接受其官爵。而姚襄雄健威武,多才多藝,觀察入微且善於安撫人心,故獲得部眾愛戴和敬重,眾人並因此請求姚弋仲立姚襄為繼承人。姚弋仲起初以姚襄不是長子,並不允許,然請求的百姓很多,姚弋仲才開始給姚襄帶兵。

永和六年(350年),冉閔殺後趙皇帝石鑒,建立冉魏。隨後後趙新興王石祗就於襄國即位為後趙帝,並以姚襄為使持節、驃騎將軍、護烏丸校尉、豫州刺史、新昌公。永和七年(351年),姚襄奉父命領二萬八千騎兵營救正遭冉閔圍攻的石祗,姚弋仲並於出發前作訓誡:「冉閔背棄仁義,屠滅石氏。我受了人家的優厚待遇,就應為其復仇,我卻因年老患病而不能親身去做;你才能比冉閔高出十倍,若果不能擒殺他回來,就不要再來見我了!」姚襄雖然聯同前燕、石琨及襄國守軍大敗冉閔,暫時解了襄國的危機,卻因無法擒得冉閔,遭姚弋仲以一百杖作處罰。

同年,後趙為冉魏所滅,姚襄隨其父向東晉投降,獲晉廷任命為為持節、平北將軍、都督并州諸軍事、并州刺史、平鄉縣公。次年(352年)姚弋仲去世,死前命諸子在其死後歸降晉室,作晉的忠臣。姚襄接手統率父親部眾,不發布父親去世的消息,並攻下陽平(今山東莘城)、元城(今河北大名縣)及發干(今山東聊城市東昌府區),駐於碻磝津(今山东省茌平县西南古益河上);但不久卻敗給前秦,南走至滎陽(今河南滎陽市)才發喪,後在滎陽與洛陽之間的麻田與前秦軍作戰時,座騎中箭死亡,因姚萇贈馬及援軍趕到才免於被擒。此時姚襄才以五個弟弟為人質,歸降東晉。東晉以姚襄駐屯譙城(今安徽亳州),而姚襄隨後單人匹馬渡過淮河,於壽春(今安徽壽縣)面見豫州刺史謝尚,當時謝尚對其名氣亦有所聽聞,於是撤去衞士,以代表高雅的幅巾接見他。二人一見如故,又因姚襄博學及善於談論,很得江東人士敬重。

同年,姚襄與謝尚一同進攻據守許昌(今河南省許昌市)的後趙豫州牧張遇,但遭前秦丞相苻雄等擊敗,謝尚因大敗而退守淮南,得姚襄棄輜重而護送至芍陂,故此將善後工作都委託給姚襄。謝尚因戰敗而受貶降,及後更被調回京師建康(今江蘇南京市),而當時駐屯歷陽(今安徽和縣)的姚襄亦以當時佔據北方的前秦及前燕皆強盛,無意北伐,反在淮河兩岸大興屯田,訓練將士,不過當時主政的殷浩就要北伐,更忌憚姚襄兵力強盛,不但囚禁姚襄送去當質子的弟弟,更屢次派刺客行刺姚襄,但刺客不能下手,反告訴姚襄實情。後殷浩再暗中派魏憬率兵襲殺他,但魏憬反被姚襄所殺,於是令殷浩更加厭惡姚襄,遷姚襄到梁國的蠡臺,表授他為梁國內史。

永和九年(353年),殷浩北伐,以姚襄為前軀,而當時姚襄已決心叛離東晉,於是算好殷浩快來到時就假意命部眾乘夜逃遁,其實暗中設伏伏擊殷浩。殷浩聽聞姚襄部眾逃遁,追至山桑(今安徽蒙城縣北)就被姚襄伏兵擊敗,被逼退守譙城,而姚襄就俘殺萬多人,盡收其輜重,南據淮南郡一帶。不久姚襄北屯盱眙(今江蘇盱眙縣),在當地收納流民,令部眾增至七萬人,並分置地方官員,鼓勵農事生產,又遣使到建康狀告殷浩,並作道歉。永和十年(354年),姚襄向前燕歸降。

永和十一年(355年),姚襄因應部眾要求北歸,於是自稱大將軍、大單于,北攻晉冠軍將軍高季,卻為高季所敗。姚襄撫恤散敗的兵眾,重新集結力量,及後乘高季去世而據有許昌。次年,姚襄進攻當時據有洛陽的周成,但用了一個多月都不能攻下,當時長史王亮就勸他放棄進攻,免得被他人有機可乘,危及自己。不過姚襄沒有聽從。然而,桓溫不久就發動北伐,征伐姚襄,姚襄被逼放棄圍城而抵抗桓溫,並在伊水以北的樹林中設下精兵,並聲稱自願歸降,請桓溫稍為退兵。然而桓溫沒有答應,並親身督戰,組以兵陣進攻沿河岸抵抗的姚襄軍,姚襄兵敗而北逃至北芒山。姚襄隨後西逃,桓溫因追不到而放棄。姚襄逃到平陽(今山西臨汾市)時得時為前秦并州刺史的舊部尹赤叛秦歸附,於是據守襄陵(今山西襄汾縣);同據并州的張平因而攻打姚襄,姚襄雖不敵,但與張平結為兄弟,換取兩者和平。

升平元年(357年),姚襄謀取關中,先移鎮北屈(今山西吉縣),後進屯杏城(今陝西黃陵縣西南),命姚蘭進攻敷城(今陝西富縣)、姚益及王欽盧聯結關中一帶的羌胡外族,共收得胡漢共五萬多戶。不過姚襄就與時據關中的前秦軍發生衝突,前秦帝苻生遣苻黃眉、苻堅、鄧羌進攻,姚襄堅守不戰。但鄧羌就在其軍壘門外列陣,激得姚襄不聽僧人智通的勸言,親自率眾出戰,最終被鄧羌詐敗誘至三原(今陝西三原縣)。姚襄遭受到鄧羌及苻黃眉的合擊,所乘駿馬「黧眉騧」倒地,姚襄為前秦軍所擒斬,享年二十七歲。其弟姚萇率餘眾投降。苻生後以公爵之禮葬姚襄。

後來,姚萇稱後秦帝時,追諡姚襄為魏武王。

姚襄深得人心,如在他大敗給桓溫而北逃時,就有五千多人於當晚拋下妻兒去追隨他。前後幾次敗仗,百姓只要知道他在哪裡就奔赴投靠。當時謠傳姚襄重傷而死,被桓溫俘虜的百姓沒有不痛哭流涕的。姚襄部屬楊亮後來歸降桓溫,桓溫向楊亮詢問姚襄的為人,楊亮的評價是:「神明器宇,孫策之儔,而雄武過之。」

楊亮:「神明器宇,孫策之儔,而雄武過之。」

姚萇:「吾不如亡兄有四:身長八尺五寸,臂垂過膝,人望而畏之,一也;當十萬之眾,與天下爭衡,望麾而進,前無橫陣,二也;溫古知今,講論道藝,駕馭英雄,收羅儁異,三也;統率大眾,履險若夷,上下咸允,人盡死力,四也。所以得建立功業,策任群賢者,正望算略中一片耳。」

呂思勉:「其(姚襄)才略或在苻健之上。然寄居晉地,四面追敵,不如健之入關,有施展之地矣。」

2009年12月27日,河南有关方面宣布在安阳发现曹操墓。这一发现,引发许多质疑。 西安市委党校历史教授胡觉照接受记者采访时称,安阳“曹操墓”实则五胡十六国时期军阀姚襄墓穴。


%% -*- coding: utf-8 -*-
%% Time-stamp: <Chen Wang: 2019-12-19 11:13:54>

\subsection{武昭帝\tiny(384-394)}

\subsubsection{生平}

秦武昭帝姚\xpinyin*{苌}(329年-393年),字景茂。南安赤亭(今甘肅省隴西縣西)羌族人。十六国时期后秦政权的开国君主。後趙末年南安羌酋長姚弋仲第二十四子,姚襄之弟。姚萇在姚襄死後率其部眾入秦,成為前秦的將領。淝水之戰後姚萇在關中羌人的推舉下自稱萬年秦王,建立後秦,並與苻堅領導下的前秦作戰。姚萇後來殺害了苻堅,並乘西燕東退而進駐長安,不久稱帝。前秦宗室苻登在關中氐族殘餘力量支持下繼續與姚萇作戰,姚萇一度處於不利形勢,但終大敗苻登,漸處優勢,但在消滅前秦勢力前去世,直至兒子姚興即位後才完全消滅前秦勢力。

姚萇年少時已聰慧明智,多有權略,豁達率性,並沒有專注於德行和學業之上,而其眾位兄長都認為他很特別。後來姚萇跟隨姚襄四處出兵,經常參與重要的決策。永和八年(352年),姚襄在麻田敗於前秦軍,其坐騎更中箭死亡,姚萇冒險將自己的坐騎送給姚襄助其出逃。最後姚萇因援軍趕至才得倖免。

升平元年(357年),姚襄謀取關中失敗,在三原(今陝西三原縣)與前秦將領苻黃眉、鄧羌等的交戰中戰死。姚萇當時就率姚襄餘眾盡降前秦。同年前秦宗室苻堅發動政變推翻皇帝苻生,自任天王,並以姚萇為揚武將軍。

太和二年(367年),姚萇隨同王猛參與討伐以略陽郡叛變的羌人斂岐,並因姚弋仲昔日統領斂岐的部落,大量部眾知道姚萇到來都向前秦歸降,令得前秦順利取下略陽。太和六年(371年)三月,与苻雅、杨安、王统、徐成及朱彤等讨伐據有仇池的氐王杨纂,双方決戰於峡谷,杨纂大败,损失三成兵力,終被逼投降。

宁康元年(373年)十一月,前秦攻下東晉領下的益、梁二州,姚萇出任宁州刺史,屯兵於垫江(今重慶市墊江縣)。後遷任步兵校尉,封益都侯。太元元年(376年)五月,与武卫将军苟苌、左将军毛盛、中书令梁熙等進逼黃河,並於八月對前涼發動攻擊,攻滅前涼。

太元八年(383年),東晉荊州刺史桓沖北伐,其中涪城(今四川綿陽市)受到晉將楊亮攻擊,姚萇遂與張蚝出兵救援,逼楊亮退兵。同年苻堅大舉攻晉,意圖滅掉東晉,統一全國,史稱淝水之戰。當時苻堅就以姚萇為龍驤將軍,督益、梁二州諸軍事,讓其從蜀地率軍進攻東晉西方,更說:「朕昔日就是以龍驤將軍建立大業,這個將軍號從來都沒有改授他人,今天特別對你授予此號,山南之事都交給你了。」

苻堅於淝水之戰中大敗,姚萇返回長安。而前秦在戰敗後國力大衰,其中北地長史慕容泓於戰後第二年在關東起兵叛亂,回屯華陰(今陝西華陰市),響應於河北地區叛變的叔父慕容垂。苻堅於是命雍州牧苻叡出兵討伐,而姚萇則任其司馬。當時慕容泓因畏懼而率眾東逃關東,苻叡因輕敵而決心追去邀擊,不聽姚萇的諫言,最終遭慕容泓擊敗,苻叡亦戰死。姚萇在敗後派長史趙都及參軍姜協向苻堅謝罪,但二人卻被憤怒的苻堅殺死,驚懼的姚萇於是逃到渭北的牧馬場。在當地,尹緯、尹詳及龐演等人聯結羌族豪強共五萬多戶向姚萇歸降,並推姚萇為盟主。姚萇於是在太元九年(384年)自稱大將軍、大單于、萬年秦王,改元「白雀」,建立後秦政權。

姚萇接著進屯北地,華陰、北地、新平及安定各郡共有十多萬名羌胡外族歸附。不久苻堅親自率軍討伐姚萇,姚萇屢敗更遭前秦軍斷絕水源。然而就在後秦軍中有人渴死及在恐懼當中時就遇上天雨,營中水深三尺,解決了水荒,亦令後秦軍心復振。不久姚萇出兵反擊,擊敗前秦將楊璧並俘獲楊璧、徐成及毛盛等數十人,皆禮待而送還。而隨著西燕軍隊逼近長安,苻堅率兵回防長安。雖然姚萇在早前向西燕送質請和,但當時姚萇群臣卻建議姚萇加入戰鬥以奪取長安,建立根本之地。不過姚萇自度慕容氏獲勝並後不會長留關中,必會東歸河北,故此打算北屯九嵕(今陝西乾縣東北)以北一帶地區(嶺北)以積聚實力和資源,待前秦亡國而西燕東歸後自取長安。姚萇隨後親自率軍進攻新平郡城(今陝西省邠縣),卻遭守將苟輔頑強抵抗,有萬多人陣亡。苟輔又詐降誘騙姚萇入城,雖然姚萇入城前就察覺而沒進城,但仍受到苟輔伏兵攻擊,萬多人戰死之餘亦險些被擒。

因為新平久久不下,姚萇於是在白雀二年(385年)正月留兵繼續攻城,自己另外出兵安定郡,擒下前秦安西将军苻珍,亦令嶺北諸城降,唯新平未下。至四月,新平物資匱乏,亦無外援,苟輔接受後秦軍的勸降,率城內五千人出降。姚苌下令將所有人坑殺,奪取了新平。五月,苻堅離開長安,出屯五將山,至七月時後秦將吴忠捕獲苻坚,送至新平。同年八月,姚萇因向苻堅索取傳國玉璽不遂,更遭其出言侮辱,於是縊殺苻堅於新平佛寺(今彬縣南靜光寺)。姚萇為了掩飾他殺死苻堅的行為,諡苻堅為「壯烈天王」。

十月,已據有長安的西燕王慕容沖派高蓋攻伐姚萇,遭後秦軍擊敗並投降。白雀三年(386年),西燕國內政變頻生,並開始棄守長安東歸。時盧水胡郝奴乘虛入據長安並稱帝,更命其弟郝多進攻於馬嵬(今陝西興平市馬嵬鎮)自守的王驎。姚萇此時從安定東攻,逼走王驎並擒得郝多,並進攻長安,令郝奴懼而請降。取長安後姚萇就於同月即位為帝,改年號「建初」,建國號大秦。不久又擊敗了前秦秦州刺史王統,奪取秦州。

但同一年,前秦宗室苻登就在關中氐族殘餘勢力的推舉下與後秦對抗,不久在前秦帝苻丕遇害後更稱帝繼位。起初苻登力量甚盛,在涇陽(今陝西涇陽縣)大敗姚碩德,要姚萇親自出兵救援;更謀攻長安。不過當時前秦重將苻纂為苻師奴所殺,將領蘭櫝遂與苻師奴反目。蘭櫝因受西燕皇帝慕容永攻擊而向後秦求援,姚萇以苻登遲疑慎重而少決斷,不敢出兵深入而冒著遭乘虛後襲的危險,決意親自率軍救援。最終先破苻師奴並盡收其眾,後敗慕容永並生擒蘭櫝。

另姚方成亦擊敗徐嵩,徐嵩雖然被俘仍大罵姚萇不僅背叛對其有恩的苻堅,更將他殺害,不惜恩情就連狗和馬都不如。姚方成殺死徐嵩後,姚萇又掘出苻堅的屍首不斷鞭撻,更脫光屍身的衣服,裹以荊棘並以土坑埋掉,以釋心中憤怨。建初三年(388年),自春季開始夏末,姚、苻兩軍就分別據朝那(今寧夏朝那縣)及武都(今甘肅武都縣)相持並交戰,互有勝負而不能擊倒對方,於是都解兵歸還。但關西豪傑都以後秦久久未能站穩關中,反多次敗給苻登,大多都投向前秦,唯齊難、徐洛生、劉郭單等人仍然忠於後秦,提供軍糧並跟隨姚萇征戰。

建初四年(389年),姚萇屢次敗於苻登,命姚崇襲擊苻登於大界的輜重又不得,而苻登就已威脅安定。面對如此局面,姚萇堅拒與苻登正面決戰,力圖以計取勝,於是乘夜率兵三萬再攻大界,終攻克大界並殺毛皇后等人及生擒數十名前秦名將。姚萇隨後亦不貪勝,堅拒乘勝進擊苻登,苻登於是收餘眾退守胡空堡,但已元氣大傷。

在大敗苻登輜重後的四個月後,姚萇設計讓其將任盆詐降以誘殺苻登,雖然最終因雷惡地識破而事敗,但苻登卻忌憚雷惡地,逼其降於姚萇。次年(390年)魏揭飛攻後秦,雷惡地叛迎魏揭飛,雖然當時苻登正在長安附近的新豐(今陝西西安市臨潼區),但姚萇以雷惡地「智略非常」,於是親自出兵攻伐魏揭飛。魏揭飛見姚萇兵少就讓全軍進擊,姚萇特意示弱不戰,卻派了姚崇從敵軍後方攻擊令其混亂,接著就出兵直擊,大敗對方並陣斬魏揭飛,又再降雷惡地並不減昔日待遇。雷惡地兩度歸於姚萇,終對其心服。另外姚萇亦不怕前秦兗州刺史強金槌詐降,只帶著數百騎兵隨其訪問強金槌的軍營,以坦誠獲得了身為氐族人的強金槌的信任,令其不應其他氐族勢力的計謀而加害姚萇。

至建初六年(391年)十二月,苻登進攻安定,姚萇在安定城東擊敗他。次年三月,前秦將沒弈干亦向後秦歸降,但姚萇不久就患病。苻登得知姚萇患病就乘機進攻安定,至八月姚萇病情轉好就親自率兵抵抗,更乘苻登出營迎擊而命姚熙隆進襲前秦軍營,令苻登懼而退兵。姚萇又讓軍隊旁出跟隨苻登,苻登得知後秦營壘空空如也,失去其影蹤後更為驚懼,只得敗還雍城(今陝西鳳翔縣南)。

建初八年(393年)十月,姚萇病重而回長安。至同年十二月,姚萇召太尉姚旻、僕射尹緯及姚晃、將軍姚大目和尚書狄伯支受遺詔輔政,輔助太子姚興。及後姚萇去世,享年六十四歲。姚興先秘不發喪,至次年才發布死訊,上諡號為武昭皇帝,廟號太祖。

姚萇簡單率直,即使當了君主,屬下有過錯可能還會直加責罵。權翼曾勸他不要這樣對待屬下,但姚萇自以這是自己本性,更稱自己聽正直之言,能知己過。

姚萇甚得苻堅重用,尤以其為龍驤將軍,並以自己從龍驤將軍登位至前秦君主一事作勉勵。但姚萇終殺害苻堅,此行為成了前秦將領反對及討伐他的理由,而姚萇亦曾挖屍洩忿。不過在屢敗於苻登後,卻認為是苻堅亡魂的助力,於是也在軍中樹立苻堅神像祈求道:「新平之禍,不是臣姚萇的錯啊,臣的兄長姚襄從陝州北渡,順著道路要往西邊去,像狐狸死時把頭朝向原本洞穴一樣,只是想要見一見鄉里啊。陛下與苻眉攔阻於路上攻擊他,害他不能成功就死了,姚襄遺命臣一定要報仇。苻登是陛下的遠親亦想復仇,臣為自己的兄長報仇,又怎麼說是辜負了義理呢?當年陛下封我為龍驤將軍,跟我說:『朕從龍驤將軍當上了皇帝,卿也好好努力罷!』這明明白白的詔諭非常顯然,好像還在耳邊一樣。陛下已經過世成為神明了,怎麼會透過苻登而謀害臣,忘卻當年說的話呢!現在為陛下立神像,請陛下的靈魂進入這裏,聽臣至誠的禱告。」 不過戰況仍未有改善,反時有夜驚,並招來苻登批評,終毀了苻堅像。據說姚萇死前曾夢見過苻堅率天官、鬼兵去襲擊他(《晉書》「將天官使者、鬼兵數百突入營中」),期間他被救援自己的士兵誤傷陰部至大量出血。醒後就發現陰部腫脹,醫者刺腫處則如夢中一樣大量出血(《晉書》「誤中萇陰,出血石餘」),如此嚇得姚萇發狂胡言,又求苻堅原諒,姚萇不久傷重身亡,臨終前跪伏床頭,叩首不已。

即使姚萇在位期間皆與前秦等勢力戰鬥,但仍設立太學,禮遇先賢後代;又曾命各鎮都要設置學官,由他們評核人才優劣再隨其才能擢用,皆可見其重視文教和吸納文人的行為。而他在安定亦修治德政,大行教化,省卻不必要的支出,亦表彰平民戶中有善行的人。

姚萇長期征戰,雖為君主亦不貪圖逸樂,於與前秦相持不下,部分豪族轉為支持前秦時更寫書自責,並賣掉後宮珍寶去支持軍事,而自己與妻子都力行簡約,對為國戰死的將士皆有所褒揚和追贈。


\subsubsection{白雀}

\begin{longtable}{|>{\centering\scriptsize}m{2em}|>{\centering\scriptsize}m{1.3em}|>{\centering}m{8.8em}|}
  % \caption{秦王政}\
  \toprule
  \SimHei \normalsize 年数 & \SimHei \scriptsize 公元 & \SimHei 大事件 \tabularnewline
  % \midrule
  \endfirsthead
  \toprule
  \SimHei \normalsize 年数 & \SimHei \scriptsize 公元 & \SimHei 大事件 \tabularnewline
  \midrule
  \endhead
  \midrule
  元年 & 384 & \tabularnewline\hline
  二年 & 385 & \tabularnewline\hline
  三年 & 386 & \tabularnewline
  \bottomrule
\end{longtable}

\subsubsection{建初}

\begin{longtable}{|>{\centering\scriptsize}m{2em}|>{\centering\scriptsize}m{1.3em}|>{\centering}m{8.8em}|}
  % \caption{秦王政}\
  \toprule
  \SimHei \normalsize 年数 & \SimHei \scriptsize 公元 & \SimHei 大事件 \tabularnewline
  % \midrule
  \endfirsthead
  \toprule
  \SimHei \normalsize 年数 & \SimHei \scriptsize 公元 & \SimHei 大事件 \tabularnewline
  \midrule
  \endhead
  \midrule
  元年 & 386 & \tabularnewline\hline
  二年 & 387 & \tabularnewline\hline
  三年 & 388 & \tabularnewline\hline
  四年 & 389 & \tabularnewline\hline
  五年 & 390 & \tabularnewline\hline
  六年 & 391 & \tabularnewline\hline
  七年 & 392 & \tabularnewline\hline
  八年 & 393 & \tabularnewline\hline
  九年 & 394 & \tabularnewline
  \bottomrule
\end{longtable}

%%% Local Variables:
%%% mode: latex
%%% TeX-engine: xetex
%%% TeX-master: "../../Main"
%%% End:

%% -*- coding: utf-8 -*-
%% Time-stamp: <Chen Wang: 2021-11-01 11:58:34>

\subsection{文桓帝姚兴\tiny(394-416)}

\subsubsection{生平}

秦文桓帝姚兴(366年-416年),字子略,南安赤亭(今甘肅省隴西縣西)羌族人。十六国时期后秦皇帝,后秦武昭帝姚苌长子。姚興即位之初就俘殺了父親在位時面對的最強對手苻登,基本覆滅了前秦。後又出兵後涼,令其投降之餘亦令盤據秦涼一帶的政權如北涼、南涼及西秦等政權臣服,還率兵進攻東晉,一舉攻取洛陽等地,使統治疆域迅速擴大。不過隨後與北魏在柴壁之戰中卻大敗,面對新興的夏國亦不能有效對付,反屢遭侵擾;且國內出現兒子姚弼與太子姚泓爭位的事件,令後秦國勢漸弱。弘始十八年(416年),姚興病逝,諡文桓皇帝,庙号高祖,下葬偶陵。

姚興在前秦時任太子舍人。白雀元年(384年),姚萇在渭北馬牧稱萬年秦王,建後秦,姚興時在長安,冒險出走與父親會合。建初元年(386年),姚萇在奪得長安(今陝西西安)後稱帝,就立了姚興為皇太子。其時姚萇屢次在外與前秦對抗,姚興就經常留鎮長安以統後事。其時又與太子中舍人梁喜及太子洗馬范勖講論經籍,不以兵戎廢業,當時的人亦受他們影響。

建初八年(393年)十二月,姚萇去世,死前命太尉姚旻、僕射尹緯、姚晃、將軍姚大目及尚書狄伯支為輔政大臣,並向姚興說:「若有人謗毀這幾位大臣,小心不要聽信。你以仁管教子女,以禮對待大臣,以信處事,以恩治民,這四項你能做到,我就不憂心了。」姚萇死後,姚興秘不發喪,分命姚緒、姚碩德及姚崇駐安定、陰密及長安,自己就自稱大將軍,領兵進攻前秦。

次年春,前秦皇帝苻登聽聞姚萇已死即十分高興,又輕視姚興,隨即率眾東進。至夏季,苻登要進攻廢橋,尹緯則受命支援守馬嵬堡的姚詳,尹緯於是據守廢橋等待前秦軍。前秦軍因無法取得水源而缺水,兩三成士兵更因而渴死,於是急攻尹緯希望能奪取水源。姚興當時認為苻登已是窮寇,於是派狄伯支命令尹緯要持重拒戰,不要輕易與前秦軍決戰。不過尹緯認為姚萇新死,人心恐懼不安,應當用盡力量消滅敵人,安定眾心。尹緯於是與苻登決戰,終大敗前秦軍,苻登因兵眾潰散而逃走,逃到馬毛山。戰後,姚興才正式發喪,並在槐里(今陝西興平東南)即位為帝,改元「皇初」。七月,姚興進攻苻登並在馬毛山南作戰,擒殺苻登,並解散其部眾。不久繼位的前秦皇帝苻崇因被乞伏乾歸逼逐而聯結楊定進攻乞伏乾歸,卻遭對方所殺,前秦正式滅亡。

皇初七年(397年),姚興率兵進攻東晉控制的湖城,弘農太守陶仲山及華山太守董邁都投降。姚興於是進至陝城(今河南陝縣),並攻下上洛(今陝西商洛市)。另又分遣姚崇進攻洛陽(今河南洛陽),因晉河南太守夏侯宗之守金鏞城而未能攻克,於是改攻柏谷,強遷兩萬多戶流民西歸。及至皇初九年(399年),姚興命姚崇及楊佛嵩再攻洛陽,守將辛恭靖堅守一百多日後失守,後秦奪得洛陽。取洛陽後,淮河、漢水以北各城大多都向後秦請降,並送人質。

弘始二年(400年),姚碩德進攻西秦,西秦王乞伏乾歸率眾抵抗,兩軍對峙期間姚碩德軍中柴草缺乏,姚興就暗中領兵支援。乞伏乾歸知道姚興派軍前來,於是命慕兀率二萬中軍屯柏楊(今甘肅清水縣西南),羅敦率外軍屯侯辰谷,自己領數千輕騎等候秦軍。不過其夜遇上大風和大霧,乞伏乾歸與慕兀的中軍失去聯絡,被逼與外軍會合。天亮後,乞伏乾歸就與後秦軍作戰,終大敗並逃返苑川(今甘肅榆中縣北),後秦軍受降共三萬六千多人,姚興則進軍枹罕(今甘肅臨夏市)。乞伏乾歸初降禿髮利鹿孤,但因怕不為對方所容,最終決定歸降後秦。

弘始三年(401年),姚興命姚碩德進攻後涼,並兵圍後涼首都姑臧(今甘肅武威)。後涼王呂隆被逼請降。而在後秦攻涼時,西涼李暠、南涼禿髮利鹿孤及北涼沮渠蒙遜都遣使向後秦請降。直至弘始五年(403年),後涼被南涼和北涼所逼,最終請後秦派軍迎來歸附,姚興因而派了齊難等人到姑臧,駐兵當地並送呂氏宗族內徙長安,吞併後涼。另外在攻打後涼姑臧時,連帶的將名僧鳩摩羅什請回長安。爾後為鳩摩羅什講解《法華經》,建造「長安大寺」。鳩摩羅什於長安圓寂,其生前將大乘佛教的主要經典(如《中論》、《法華經》、《維摩詰經》等)譯為漢文。

北魏君主拓跋珪曾經送一千匹馬到後秦請婚,姚興原先答應,但知拓跋珪已立了后,於是拒絕並留下使者賀狄干。弘始四年(402年),北魏將領拓跋遵進攻高平(今甘肅固原),沒弈干拋棄部眾,帶著數千騎兵及赫連勃勃逃到秦州。北魏軍追擊至瓦亭仍未追上,於是盡遷高平的物資回國;及後北魏平陽太守貮塵又進攻河東。北魏的一系列軍事行動震動長安,關中各城日間也緊閉城門,姚興於是在城西閱兵,並做好戰爭準備。同年,姚興派姚平及狄伯支等率四萬步騎兵進攻北魏,姚興則親率大軍在後。北魏帝拓跋珪則命拓跋順及長孫肥統六萬騎兵為先鋒,自己也率大軍在後以作抵抗。姚平用了六十多天攻陷了北魏屯積糧食的乾壁,又派二百精騎偵察魏軍,卻為長孫肥襲擊,所有人都被生擒。姚平因而後撤,又遭拓跋珪追擊,並在柴壁(今山西襄汾縣西南)被追上;姚平當時據柴壁城固守,北魏軍則圍困城池。姚興於是自領四萬七千兵營救姚平,並打算佔領天渡以運糧支援姚平。不過北魏加強了包圍圈,又在汾水建浮橋,在汾水西岸築圍堵截姚興援軍,務求引姚興取道汾東,經長達三百多里而缺乏小路通行的蒙坑進攻。而姚興到蒲阪後因怕魏軍強盛,很久才正式進攻。及後姚興在蒙坑以南與拓跋珪所率三萬步騎兵作戰,後秦軍共千多人被殺,姚興被逼退走四十多里,而姚平亦未能突圍。接著拓跋珪分兵各據險要,不讓後秦軍接近柴壁。姚興駐屯汾西,在汾水上游放木材打算沖毀北魏浮橋,但木材都被魏軍截取。至十月,姚平軍需用盡,在夜間試圖向西南方突圍,姚興列兵汾西,點起烽火和擂鼓響應,不過姚興欲救姚平盡力突陣,姚平反望姚興攻圍接應,兩軍雖然能夠以叫喊相通,但始終都沒能壓逼圍城魏軍。姚平最終無法成功突圍,於是率眾投水自殺,然而拓跋珪卻都派人潛下水捕捉,赴水諸將與城中狄伯支、唐小方等人及兩萬多兵眾都被俘。姚興只能見城中軍隊束手就擒而無力支援,全軍都哀傷痛哭,哭聲震動山谷。接著姚興數度派遣使者求和,但都被拒,魏軍更乘勝進攻蒲阪。防禦蒲阪的姚緒固守不戰,又正因柔然要進攻北魏,拓跋珪才撤兵。

弘始九年(407年),北魏歸還柴壁之戰中被俘的唐小方等人,姚興又以良馬千匹贖回狄伯支,與北魏通和。赫連勃勃因後秦與北魏連和而大怒,竟搶奪了柔然送給後秦的八千匹馬,並襲殺沒弈干叛變,稱大夏天王,建夏國。赫連勃勃隨後又攻破鮮卑薛干等三部,並進攻後秦三城以北諸戍,後秦將楊丕、姚石生等都被殺,接著又侵掠嶺北,令嶺北各城城門白天也要緊閉。姚興此時感嘆:「我不聽黃兒(姚興弟姚邕小字)的話,才弄成今天這樣子。」

隨後禿髮傉檀大敗於赫連勃勃,名將折損達六七成,接著成七兒及梁裒、邊憲等又先後謀反,姚興見其並受外憂外患夾擊,不顧尚書郎韋宗的勸阻和吏部尚書尹昭命北涼及西涼進攻禿髮傉檀的建議,堅持分兵兩道進攻夏和禿髮傉檀。姚興於弘始十年(408年)派了齊難領二萬騎兵攻夏,又派姚弼、斂成及乞伏乾歸攻禿髮傉檀,更寫信給禿髮傉檀聲稱姚弼等其實只是配合齊難進攻夏國的行動,禿髮傉檀不作防備。不過姚弼等到姑臧後反被禿髮傉檀的奇兵擊敗,後又特地釋放牛羊引誘後秦軍掠奪,大敗秦軍。作為後繼的姚顯知姚弼兵敗,加快趕到姑臧,並命孟欽等五名擅長射擊的人於涼風門挑戰,卻遭南涼材官將軍宋益擊殺。姚顯見此委罪於斂成,派使者向禿髮傉檀謝罪,撫慰河西後就撤還。而禿髮傉檀亦派使者徐宿向後秦謝罪。不過在當年又再稱涼王。

而赫連勃勃知齊難來攻,於是退守河曲。齊難見赫連勃勃仍在很遠,於是先行縱兵野略;赫連勃勃因而潛軍來襲,俘殺七千多人,齊難逃走但在木城遭赫連勃勃生擒,其餘將士亦被俘。戰後嶺北共計有數萬人歸附赫連勃勃。弘始十一年(409年),姚興再派姚沖及狄伯支率四萬騎再攻夏,但姚沖竟圖謀反,並殺了不肯支持的狄伯支,姚興終賜死姚沖。同年,姚興親自率軍攻夏,至貮城後就派姚詳、斂曼嵬及彭白狼分督租運。其時諸軍未集合,而赫連勃勃乘虛來襲,姚興恐懼之下打算逃到姚詳那裏,但被右僕射韋華勸止。姚興派姚文宗等迎戰,雖將領姚榆生被擒,但在姚文宗力戰之下也成功擊退赫連勃勃。姚興唯有留五千禁軍助姚詳守貮城,自己撤還長安。

赫連勃勃攻破了敕奇堡、黃石固及我羅城。次年又派胡金纂攻平涼,雖然姚興親自率軍擊殺胡金纂,但赫連勃勃侄赫連羅提又攻下定陽,殺四千多人並俘姚廣都。當時秦將曹熾、曹雲及王肆佛等被逼領數千戶內徙,姚興就讓他們住在湟山及陳倉。接著赫連勃勃又進攻隴,攻略陽太守姚壽都守的清水城,姚壽都棄城奔上邽,而赫連勃勃就遷了城中一萬六千戶人到大城。姚興試圖從安定追擊赫連勃勃,但追不上。及後赫連勃勃仍屢屢侵擾後秦,但姚興都無法消滅夏國。

姚興子廣平公姚弼得父親寵愛,任雍州刺史,出鎮安定時天水人姜紀接近姚弼,並勸他巴結姚興左右以望還朝,姚弼於是巴結常山公姚顯。至弘始十三年(411年),姚興就召了姚弼回長安,讓他為尚書令、侍中、大將軍。姚弼於是擔當將相要職,更心引見人才,收結朝士,形成了一股比太子姚泓更大的勢力,更有圖取其太子之位。後來姚弼因為厭惡姚泓親信姚文宗,就誣陷他有所怨言,並讓侍御史廉桃生作證。姚興信以為真,一怒之下就賜死姚文宗。朝中大臣於是都不敢再說姚弼不是了。

因著對姚弼的寵愛,姚興對姚弼幾乎言聽計從,於是機要職位都由姚弼親信出任。當時右僕射梁喜、侍中任謙及京兆尹尹昭就找機會向姚興表示姚弼有奪嫡的志向,指出姚興不當的寵愛他,令傾險無賴的人都在其身邊,又說民間都說姚興有廢立之意,三人同時表示反對易儲。姚興立即否認有易儲計劃,三人就是更勸姚興削減姚弼權力並除去其身邊黨羽,既保姚弼,亦保國家。姚興聽後就沉默不言。

弘始十六年(414年),姚興病重,太子姚泓屯兵東華門,並在諮議堂侍疾。當時姚弼卻意圖作亂,招集了數千人並藏匿在其府中。姚裕當時與任謙、梁喜等人都掌禁軍守衞皇宮,而姚裕就派使者將姚弼謀反的行狀告知各個外藩,於是駐蒲阪的姚懿、洛陽的姚洸及雍城的姚諶都將要領兵入長安討伐姚弼。此時姚興病情好轉,召見了群臣,征虜將軍劉羌向姚興泣告姚弼謀反之事,尹昭等都建議姚興即使不按法處死,也應削其權力,讓他散居藩國。姚興仍然欣賞才兼文武的姚弼,不忍殺他,於是免去其尚書令職位,以大將軍、廣平公身份還第。

及後姚懿、姚洸、姚宣及姚諶來朝,見面時姚宣哭請姚興按法處置姚弼,但姚興拒絕。撫軍東曹屬姜虬也上書指姚弼雖然被姑息,但其黨羽仍然活躍,姚弼變亂的心是不會變的,更請消除姚弼黨羽,以絕禍根。姚興就問梁喜:「天下的人全都以我兒子作為口實,要如何處理?」梁喜則說:「真的如姜虬所言,陛下應該早點有個決定。」姚興又沉默不言。

弘始十七年(415年),姚弼知姚宣在父親面前說自己不是,十分憤恨,於是就向姚興誣陷姚宣。姚興又相信,並召見當時到了長安的姚宣司馬權丕,責怪他沒有好好匡輔姚宣並要處死他。但權丕竟然捏造了姚宣的罪狀報告姚興。姚興於是大怒,收捕了姚宣並派姚弼率兵三萬出鎮秦州。尹昭知道後向姚興指讓姚弼統大軍在外,一旦姚興去世,就會是太子姚泓的大大威脅,試圖勸止姚興,但姚興不聽。

同年,姚興食五石散中毒,姚弼卻稱病不朝,又再次在府中招集兵眾。姚興知道後大怒,殺了姚弼黨羽殿中侍御史唐盛及孫元。姚泓卻在怪責自己,請姚興殺了他,或處之外藩。姚興於是召了姚讚、梁喜、尹昭及斂曼嵬,和他們討論後囚禁了姚弼並準備殺了他,又要將姚弼黨羽全部治罪。不過姚泓請命之下,都將他們寛恕。

弘始十八年(416年),姚興出行華陰,留姚泓監國。及後姚興病重回長安,姚弼黨羽尹沖等仍想發難,想趁姚泓出迎姚興而將其殺害,但姚泓只在黃龍門拜迎。其時尚書姚沙彌更意圖劫奪姚興到廣平公府,以姚興招引眾人支持,從而從姚泓手中奪去儲君之位。尹沖雖不從,但仍然考慮隨姚興乘輿入宮中作亂,只是未知姚興生死而不敢行動。姚興則命姚泓錄尚書事,並命姚紹及胡翼度掌禁軍,又命斂曼嵬收去姚弼府中的武器。

不久,姚興病情更趨嚴重,姚興妹南安長公主去探望他也得不到回應,姚興幼子姚耕兒就向哥哥姚愔報告姚興已死,叫他快點做決定。姚愔於是就帶其他的士兵攻端門,斂曼嵬領兵抵禦,而胡翼度就關上宮中四門。姚愔派壯士爬上門並進入宮內,並走到馬道。時在諮議堂侍疾的姚泓命斂曼嵬登武庫抵禦,而太子右衞率姚和都亦已率東宮士兵在馬道南駐屯。姚愔無法前進,只得燒毀端門。姚興此時竭力走到前殿,並下令賜死姚弼。禁軍見到姚興士氣大振,向姚愔軍發動進攻,姚和都也在後夾擊,最終姚愔軍潰敗,姚愔逃到驪山,呂隆則逃到雍城,尹沖及尹泓就南奔東晉。

姚興召姚紹、姚讚、梁喜、尹昭和斂曼嵬入寢宮,遺命他們為輔政大臣。姚興即逝世,享年五十一歲。諡文桓皇帝,庙号高祖,下葬偶陵。

姚興曾命各郡國每年都上報一個品行純潔的孝廉,又留心政事,廣納百言,包容各種意見。即使只是說了一句姚興認為有益的建言,都會得到特別禮待。如杜瑾、吉默和周寶就曾因向姚興陳述當時國中大事而獲授要職。姚興又重文教,當時有姜龕、淳于岐及郭高等有大德的老儒士在長安教學,各有數百門生,其中有不少門生更遠道而來。而姚興就在處理政務以外的時間請姜龕等到東堂和他談論學問和技藝。當時一叫胡辯的人在當時仍是東晉佔領的洛陽授學,很多關中人都去拜師,姚興更下令各關守長盡量方便這些求學的人出入。種種措施都令後秦儒學興盛。

皇初九年(399年),姚興以国内天灾频频,於是自降帝号,称秦王;另又下令郡國將因戰亂而賣身為奴婢的人變回良人,更將幾個貪財官員誅殺,整頓官員風氣。及後又在長安建立法律學校,讓各郡縣散吏入讀,學成者就送回郡縣以處理形獄事項,又下令郡縣無法裁決的都上交廷尉處理。姚興更經常到諮議堂聽訟和作判決,大大減少了冤獄。

弘始三年(401年)呂隆向後秦請降後,姚興就迎在後涼的僧人鳩摩羅什入長安,並奉其為國師,奉之如神。鳩摩羅什在長安組織了大規模的翻譯佛經事業,姚興亦信了佛,於是群下都跟著信奉佛教,又吸引了五千多個僧人遠道而來。姚興又在永貴里建了佛塔、在中宮建了波若臺,佛教興盛,各州郡都受到佛教影響,至「求佛者十室而九。」

姚興在位後期,國庫不足,曾增加關隘和渡口的稅,又向鹽、竹、山林和木材徵稅。群臣曾勸諫但姚興認為能夠常出入關隘及取利於山水資源的都是富人,現在增稅其實只是取富人多餘的而彌補國家不足,並無不妥。

姚興生性儉約,所乘車馬都沒有黃金或玉石裝飾,以身作則之下,群下都崇尚清正廉潔。不過姚興卻喜歡打獵,常傷及農作物。杜挻及相雲曾分別作《豐草詩》及《德獵賦》以作暗示,姚興雖然明白並以黃金及布帛作賞賜,但始終改變不了打獵的習慣。

每當大臣去世,姚興都不會只按慣例在東堂發哀,而會親身去臨喪。

姚興十分看重親族,更對兩名叔叔姚碩德及姚緒十分恭敬。姚興降號為王時,本為王爵的姚碩德及姚緒應當降為公爵,但姚興不允,在二人再三辭讓下才得允許。姚興又曾下令所有官員取名時不得犯二人名諱,所有車馬、衣服及器玩都先給二人,自己只用次一等的,見面時行家人之禮,朝中大事亦必定先諮詢二人。姚沖叛變不遂殺了顧命大臣之一的狄伯支,姚興仍然顧念他是最小的弟弟,雄武絕人,還想對他寬容一次,不過在斂成規勸下,姚興想到他殺了狄伯支,就下書賜死姚沖。

\subsubsection{皇初}

\begin{longtable}{|>{\centering\scriptsize}m{2em}|>{\centering\scriptsize}m{1.3em}|>{\centering}m{8.8em}|}
  % \caption{秦王政}\
  \toprule
  \SimHei \normalsize 年数 & \SimHei \scriptsize 公元 & \SimHei 大事件 \tabularnewline
  % \midrule
  \endfirsthead
  \toprule
  \SimHei \normalsize 年数 & \SimHei \scriptsize 公元 & \SimHei 大事件 \tabularnewline
  \midrule
  \endhead
  \midrule
  元年 & 394 & \tabularnewline\hline
  二年 & 395 & \tabularnewline\hline
  三年 & 396 & \tabularnewline\hline
  四年 & 397 & \tabularnewline\hline
  五年 & 398 & \tabularnewline\hline
  六年 & 399 & \tabularnewline
  \bottomrule
\end{longtable}

\subsubsection{弘始}

\begin{longtable}{|>{\centering\scriptsize}m{2em}|>{\centering\scriptsize}m{1.3em}|>{\centering}m{8.8em}|}
  % \caption{秦王政}\
  \toprule
  \SimHei \normalsize 年数 & \SimHei \scriptsize 公元 & \SimHei 大事件 \tabularnewline
  % \midrule
  \endfirsthead
  \toprule
  \SimHei \normalsize 年数 & \SimHei \scriptsize 公元 & \SimHei 大事件 \tabularnewline
  \midrule
  \endhead
  \midrule
  元年 & 399 & \tabularnewline\hline
  二年 & 400 & \tabularnewline\hline
  三年 & 401 & \tabularnewline\hline
  四年 & 402 & \tabularnewline\hline
  五年 & 403 & \tabularnewline\hline
  六年 & 404 & \tabularnewline\hline
  七年 & 405 & \tabularnewline\hline
  八年 & 406 & \tabularnewline\hline
  九年 & 407 & \tabularnewline\hline
  十年 & 408 & \tabularnewline\hline
  十一年 & 409 & \tabularnewline\hline
  十二年 & 410 & \tabularnewline\hline
  十三年 & 411 & \tabularnewline\hline
  十四年 & 412 & \tabularnewline\hline
  十五年 & 413 & \tabularnewline\hline
  十六年 & 414 & \tabularnewline\hline
  十七年 & 415 & \tabularnewline\hline
  十八年 & 416 & \tabularnewline
  \bottomrule
\end{longtable}

%%% Local Variables:
%%% mode: latex
%%% TeX-engine: xetex
%%% TeX-master: "../../Main"
%%% End:

%% -*- coding: utf-8 -*-
%% Time-stamp: <Chen Wang: 2019-12-19 15:09:10>

\subsection{姚泓\tiny(416-417)}

\subsubsection{生平}

姚泓(388年-417年),字元子,十六国时期后秦末主,后秦文桓帝姚兴长子。

后秦弘始十八年(416年)正月,姚兴卒,姚泓即位。兄弟相争,国中大乱。八月东晋刘裕起兵伐秦。后秦疲于应敌之际,国内又相继发生姚懿、姚恢的叛乱。

次年八月癸亥(417年9月20日),刘裕帐下大将王镇恶攻入长安平朔门。姚泓无计可出,准备出降,他十一岁的儿子姚佛念说,晋朝人“将逞其欲”,(即使投降)我们也一定不能保全自己,我愿自杀。姚泓怃然不知所对。佛念登上城墙自投而死。姚泓率一家老小至王镇恶大營投降,其堂叔姚赞也率宗室子弟一百余人投降。刘裕將後秦王室全部处死,其余宗族成员迁往江南。姚泓被押往建康斩首。后秦亡。

《晋书》载,泓孝友宽和,而无经世之用,又多疾病,兴将以为嗣而疑焉。久之,乃立为太子。兴每征伐巡游,常留总后事。博学善谈论,尤好诗咏。

\subsubsection{永和}

\begin{longtable}{|>{\centering\scriptsize}m{2em}|>{\centering\scriptsize}m{1.3em}|>{\centering}m{8.8em}|}
  % \caption{秦王政}\
  \toprule
  \SimHei \normalsize 年数 & \SimHei \scriptsize 公元 & \SimHei 大事件 \tabularnewline
  % \midrule
  \endfirsthead
  \toprule
  \SimHei \normalsize 年数 & \SimHei \scriptsize 公元 & \SimHei 大事件 \tabularnewline
  \midrule
  \endhead
  \midrule
  元年 & 416 & \tabularnewline\hline
  二年 & 417 & \tabularnewline
  \bottomrule
\end{longtable}


%%% Local Variables:
%%% mode: latex
%%% TeX-engine: xetex
%%% TeX-master: "../../Main"
%%% End:



%%% Local Variables:
%%% mode: latex
%%% TeX-engine: xetex
%%% TeX-master: "../../Main"
%%% End:

%% -*- coding: utf-8 -*-
%% Time-stamp: <Chen Wang: 2019-12-19 15:30:59>


\section{后燕\tiny(384-407)}

\subsection{简介}

後燕(384年-407年或409年)是中国五胡十六国時慕容氏諸燕之一,由鮮卑人前燕文明帝慕容皝第五子慕容垂所建立的政權。

後燕建國之初定都中山(今河北省定州市),後遷往龍城(今遼寧省朝陽市)。全盛時統治範圍「南至琅琊,東訖遼海,西屆河汾,北暨燕代」(《讀史方輿紀要》),即今河北、山東、山西和河南、遼寧的一部分。自384年慕容垂稱燕王到407年慕容熙被殺(或到409年慕容雲被杀),立國凡24年(一说26年)。

《十六国春秋》始称后燕,以别于慕容氏諸燕,后世袭用之。

重建燕國(383年─385年):前秦在383年淝水之戰大敗後,投降前秦的前燕貴族慕容垂在苻堅同意下回到鄴城。時丁零族翟斌在洛陽新安一帶起兵反秦,鎮守鄴城苻堅庶長子苻丕撥兵2千給慕容垂,派宗室苻飛龍領兵1千為慕容垂的副手,前去對付翟斌,但慕容垂於行軍中襲殺苻飛龍,與前秦正式決裂。

384年正月,慕容垂渡過黃河移至洛陽附近,與翟斌聯兵攻洛陽。後引兵東下,在滎陽自稱大將軍、大都督、燕王,建元燕元元年。後自石門渡黃河,向鄴前進,時有眾20餘萬。因苻丕堅守鄴城,慕容垂久攻不下,因战争,河北經濟受到很大的破壞。到了385年八月,苻丕撤出鄴城,退往晉陽,整個河北,皆落入慕容垂手中。386年正月,慕容垂稱帝,定都中山(今河北省定州),改元建興,史稱後燕。

攻滅西燕(386年─394年):386年十月,西燕慕容永進至長子(今山西省長子縣西),稱帝,改元中興,佔有今山西省一帶。由於後燕不容許作為宗室一方的西燕「僭舉位號,惑民視聽」,與後燕爭奪燕國領導權。在392年消滅翟魏後,出兵攻伐西燕。

393年冬,慕容垂徵發步騎兵7萬,命丹陽王慕容瓚出井陘關(今河北省井陘縣井陘山),攻晉陽,西燕守將慕容友領兵5萬防守潞川。明年春,慕容垂增調司、冀、青、兗四州兵,分兵三路出滏口(今河北省磁縣西北石鼓山)、壺關、沙亭,西燕分兵拒守。後慕容垂在鄴城西南屯兵月餘,慕容永懷疑後燕欲從太行山南口進兵,將大部兵力調往軹關。夏,慕容垂率大軍出滏口,由天井關向南直趨臺壁(今山西省黎城縣西南),慕容永倉卒集結5萬精兵,與後燕軍大戰於臺壁南,西燕軍中伏大敗,慕容永逃回長子。後燕攻下晉陽,進圍長子,八月間滅西燕。

慕容垂滅西燕後,趁東晉衰亂之際,略地青、兗,把疆域向南擴展到今山東的臨沂、棗莊一帶。

燕魏對峙(394年─396年):386年,拓跋珪建立北魏。起初後燕與北魏的關係本來是友好的,因後燕戰馬缺乏,屢求於魏,甚至發生扣留北魏使者以求名馬的事,兩國關係告結。而北魏採取聯西燕拒後燕的政策,對付後燕。394年西燕危急時,北魏派兵5萬為西燕聲援。次年五月,慕容垂命太子慕容寶、趙王慕容麟率兵8萬伐魏,遣范陽王慕容德率步騎1.8萬為後繼。北魏聽說燕軍北上,把部落、畜產及大軍轉移至黃河以南(今內蒙古伊克昭盟),避開燕軍。到十月,由於塞外嚴寒、士氣低落,後燕不得不撤退。這時北魏派拓跋遵領騎兵7萬,堵塞燕軍南歸之路。拓跋珪自領2萬,進擊後燕軍,後燕軍大敗,亂不成軍,四、五萬兵投降,北魏俘虜了後燕文武將吏數千人,繳獲了兵器、衣甲、糧食無數,拓跋珪將後燕降兵全部坑殺於參合陂,慕容寶等單騎逃回。史稱參合陂之役。

慕容寶等逃回中山後,屢請求再次伐魏,慕容德也勸說慕容垂趁自己尚健在時親征,以免遺留後患。慕容垂接受了他們的意見。396年三月,慕容垂率大軍再次伐魏,敗北魏陳留公拓跋虔,其後慕容垂病情加重,急忙退兵。四月,慕容垂病死。慕容垂此次的北伐,並沒有能夠挽回後燕軍事上的頹勢。此後,拓跋珪就挾其三、四十萬騎兵,長驅進入中原。

衰落滅亡(396年─409年):396年四月,慕容垂病死,子慕容寶繼承帝位。後燕在全國重要的戰略及政治中心有五處,即中山、龍城、鄴、晉陽、薊。八、九月間,北魏拓跋珪率領40餘萬大軍,攻取晉陽。十一月,攻下常山、信都,河北許多郡縣的官員,不是逃亡就是投降。這時慕容寶在中山有步兵12萬、騎兵3.7萬,悉數出抗拒魏軍,大敗而還。魏軍進軍包圍了中山,397年三月,慕容寶率軍突圍,退往中山。十月,魏軍攻下中山,後燕官吏兵投降兩萬餘人,後燕的疆域被切斷為南、北二部。

398年,慕容德在滑臺稱燕王,建立南燕。蘭汗殺死慕容寶,自稱大將軍、大單于、昌黎王。慕容盛殺蘭汗自立,後來討伐高麗及庫莫奚有功,然因治下太嚴,刑罰殘忍,在401年為大臣段璣所暗殺。鮮卑貴族立慕容垂少子慕容熙為帝,他採行了胡漢分治的政策來統治國家。這時的後燕疆域,僅有遼西一帶,疆域狹小,民戶不多,但他卻大興土木,營建宮苑殿閣,給人民帶來無窮的災難。407年,馮跋兄弟趁慕容熙送葬苻后時起事,推高雲(慕容雲)為燕王,殺死慕容熙。409年高雲的禁衛離班、桃仁殺死高雲,馮跋稱燕天王,後燕滅亡。

384年 建国。慕容垂包圍鄴。

385年 打敗高句麗,進入遼東。

394年 攻滅西燕。

395年 燕軍與北魏軍在参合陂大戦,燕軍大敗。

396年 慕容垂親率大軍進攻北魏,途中病故。太子慕容宝即位。

397年 魏拓跋珪攻擊燕都中山。燕王慕容宝逃往龍城。

398年 慕容宝被殺,慕容盛推翻弑君的蘭汗,奪取皇位。

401年 慕容盛被暗殺。太后丁氏擁立慕容熙。

407年 漢人將軍馮跋擁立高雲為燕天王,殺死慕容熙。有些史學家把這年認定為後燕滅亡,北燕建立之年。

409年 高雲被殺,馮跋繼位。有些史學家把這年認定為後燕滅亡,北燕建立之年。


%% -*- coding: utf-8 -*-
%% Time-stamp: <Chen Wang: 2019-12-19 15:20:37>

\subsection{成武帝\tiny(384-396)}

\subsubsection{生平}

燕成武帝慕容垂(326年-396年6月2日),字道明,原名霸,字道業,一說字叔仁,鮮卑名阿六敦,昌黎棘城(今遼寧義縣)鮮卑族人。十六國後燕開國君主。前燕文明帝慕容皝的第五子。在前燕時屢有戰功,更加曾擊退東晉桓溫的北伐軍。然而因為受到當政的慕容評排擠而被逼出走前秦,但很受前秦君主苻堅的寵信。淝水之戰後慕容垂乘時而起,復建燕國,建立後燕,後又滅了同為慕容氏所建的西燕。參合陂之戰戰敗後率軍再攻北魏,在期間發病病重,並在退軍時去世。

原名慕容霸的慕容垂甚得父親慕容皝寵愛,甚至比起身為世子的哥哥慕容儁更多,故此慕容儁忿忿不平。咸康八年(342年),慕容皝進攻高句麗,慕容霸與慕容翰作前鋒,終攻陷高句麗都城丸都(今吉林集安西)。建元二年(344年),慕容皝攻伐宇文逸豆歸,慕容翰為前鋒都督,慕容霸與慕容軍、慕容恪及慕輿根則受命兵分三道進攻。當時逸豆歸派遣涉奕于率領精兵抵禦,慕容翰決意以擊敗涉奕于以摧毀宇文部士氣,令宇文部自潰,於是主動進攻,涉奕于親自迎戰,慕容霸於是在側邀擊,與慕容翰擊敗涉奕于。宇文部士兵於戰後果然自潰,宇文逸豆歸出逃敗死漠北,成功消滅了宇文部。慕容霸則以此功封都鄉侯。永和元年(345年),後趙將領鄧恆領兵數萬駐屯樂安(今河北樂亭縣東北),意圖併吞前燕。慕容皝以慕容霸為平狄將軍,駐軍徒河(今遼寧錦州西北),鄧恆因為畏懼慕容霸而不敢進犯。

永和四年(348年),慕容皝去世,慕容儁繼位燕王,就以慕容霸曾經墮馬而撞斷了牙齒為由改其名為「慕容𡙇」,後更去「夬」而改名慕容垂。次年後趙皇帝石虎去世,國內因諸子爭位而大亂,慕容垂於是上書慕容儁建議出兵後趙。慕容儁初以慕容皝新死而不允,但慕容垂親往都城龍城(今遼寧朝陽市)勸說慕容儁,更自請為前驅領兵威逼鄧恆。在封奕等人的支持下,慕容儁以慕容垂為前鋒都督、建鋒將軍,選二十多萬精兵準備伐趙。

永和六年(350年)二月,慕容儁命慕容垂領二萬兵經循東路經徒河伐趙,另遣慕輿于出西道,自率中軍,兵分三路伐趙。慕容垂到三陘(今河北撫寧縣矛石山),鄧恆驚懼而燒倉庫出逃,退保薊城(今北京)。慕容垂到後盡收樂安、北平兩郡兵糧,與慕容儁會合共攻薊城。三月,燕軍攻下薊城,慕容垂勸止了慕容儁阬殺後趙士卒的決定。不久慕容儁又親率軍隊進攻鄧恆,至清梁(今河北清苑縣西南)時趙將鹿勃早率數千人夜襲燕軍,突入慕容垂幕下,慕容垂於是奮力反擊,手刃了十多人,遏制了鹿勃早的攻擊,及後慕輿根等人領兵擊敗鹿勃早,成功擊退了來襲。

元璽元年(352年),慕容儁稱帝,任黃門侍郎,又遷安東將軍、冀州刺史,鎮常山。至元璽三年(354年)封慕容垂為吳王,並移鎮信都(今河北冀縣)。後召為侍中、右禁將軍、錄留臺事,轉鎮龍城,但因慕容垂在當地很得人心,故被慕容儁召還。後又轉撫軍將軍,並於光壽元年(357年)與中軍將軍慕容虔等率軍大敗敕勒。

光壽二年(358年),中常侍涅皓知慕容儁不喜歡慕容垂,又因可足渾皇后不滿慕容垂妻段氏,於是誣稱段氏與吳國典書令高弼行巫蠱之術,意圖以此牽連慕容垂。段氏寧死不屈,雖然最終死在獄中,但都沒有將慕容垂牽連到事件中,後慕容垂遷鎮東將軍、平州刺史,外鎮遼東。

建熙元年(360年),慕容儁去世,太子慕容暐繼位,以慕容垂為河南大都督、征南將軍、兗州牧、荊州刺史,領護南蠻校尉,鎮梁國。建熙六年(365年),慕容垂與慕容恪共攻東晉控制的洛陽(今河南洛陽市),擊敗並俘虜晉將沈勁,攻下了洛陽,隨後遷都督荊揚洛徐兗豫雍益涼秦十州諸軍事、征南大將軍、荊州牧,鎮魯陽。

太宰慕容恪深知慕容垂的才能,故此在建熙八年(367年)病死前向樂安王慕容臧指出應以慕容垂擔任大司馬一職,又向慕容暐推薦慕容垂在其死後接替自己,將政事都交給慕容垂處理。慕容臧雖將慕容恪的話告訴主政的太傅慕容評,但慕容評沒有按慕容恪的意思做,以慕容沖為大司馬,又調慕容垂為侍中、車騎大將軍、儀同三司。

建熙十年(369年)四月,東晉大司馬桓溫北伐前燕,諸將都無法抵抗晉軍,讓晉軍於七月進駐枋頭(今河南浚縣)。當時慕容暐及慕容評皆大驚,想逃回故都龍城避難。慕容垂於是請求讓他出戰。慕容暐就任命他接替慕容臧擔任南討大都督,率慕容德等五萬兵出戰。慕容垂又請了黃門侍郎封孚、司徒左長史申胤及尚書郎悉羅騰從軍。桓溫當時以降人段思為響導,悉羅騰與晉軍接戰,生擒了段思;接著桓溫派李述進攻,又被悉羅騰所敗,李述更戰死,晉軍士氣於是下降。同時慕容德等又至石門阻止晉軍開通漕運,豫州刺史李邽又斷晉軍糧道,桓溫屢戰不利,糧食又不足,終於九月循陸路撤軍。當時諸將打算立刻追擊,但慕容垂以晉軍初退,必定嚴加戒備,以精銳軍隊斷後,於是打算遲點才追擊,待晉軍乘追兵未至而加速行軍,令兵士筋疲力盡時才進攻。慕容垂因而率領八千騎兵緩緩尾隨晉軍,發現桓溫果然在看不見追兵後加速。數日後慕容垂下令進攻,騎兵於是加速,於襄邑(今河南睢縣西)趕上晉軍,配合慕容德所率埋伏於襄邑的伏兵夾擊桓溫,於是大敗晉軍,殺三萬人。桓溫只有收拾殘軍南退。

枋頭之戰大勝後,慕容垂威名大振,卻令慕容評更加嫌忌他,慕容垂上請有戰功的將領獲得封賞都沒得批准,兩人就因此事在廷上互相爭論,更加深化了兩人的嫌隙。時為太后的可足渾皇后亦厭惡慕容垂,於是與慕容評密謀誅除他。慕容恪子慕容楷及慕容垂舅舅蘭建得悉陰謀,於是建議慕容垂先發制人,除去慕容臧及慕容評。然而慕容垂卻表示寧願出奔國外亦不想骨肉相殘。世子慕容令得知後建議慕容垂北奔龍城,並向慕容暐謝罪,盼望慕容暐感悟召還;即使不然,仍可以固守當地以求自保。慕容垂聽從,於同年十一月就上請到大陸澤狩獵,微服潛歸龍城。然而到邯鄲(今河北邯鄲)時,向來不得寵的兒子慕容麟卻逃還鄴城(今河北臨漳西)告發父親的意圖,於是跟隨慕容垂的人大多都逃走,慕容強亦奉命追捕慕容垂。至范陽(今河北涿縣)時慕容強追上慕容垂,但因慕容令親自斷後,慕容強也不敢進逼。日落後,慕容令表示原本的計劃已不再可行,又建議投奔前秦,慕容垂計窮,亦得接受,於是棄用馬匹以免留下蹤跡,悄悄回鄴城並躲於顯原陵。不久竟有數百個獵人從四方向他們所在聚集,慕容垂等人敵不過他們,卻又無處可逃,甚麼也做不了。就在此時,獵人的獵鷹卻同時飛起,獵人於是散去,慕容垂因而殺白馬祭天,與隨行者誓盟。慕容令在那時又建議讓他回鄴城襲殺慕容評,並以慕容垂的名望取而代之,入輔朝廷。但慕容垂以此危險而否決,於是與妻段氏、慕容令、慕容寶、慕容農、慕容楷及蘭建、高弼等西奔前秦。前秦天王苻堅得知慕容垂來奔,十分高興並親自迎接,以慕容垂為冠軍將軍,封賓徒侯。

慕容垂奔秦次年,前秦就滅了前燕,而慕容垂在前秦官至京兆尹,進封為泉州侯。建元十八年(382年),苻堅執意要攻伐東晉,苻融、石越、苻宏等人都反對,而慕容垂卻說:「弱者被強者所吞,小的被大的兼併,這是合乎自然的,並不難理解。以陛下神武,順應天期,聲威布於海外,百萬衞士,滿朝韓信、白起那樣的良將,晉這個於江南的小國獨獨違抗王命,怎可以再留她給子孫。《詩經》說:「谋夫孔多,是用不集」陛下自己決定就夠了,又何必詢問一眾朝臣!晉武帝平滅東吳,也不過只有張華、杜預幾個臣子支持而已,若果他順從朝臣主流意見,又怎能成就統一大業!」苻堅聽後大喜,更說:「和我一起平定天下的人,就只有你呀。」建元十九年(383年)五月,東晉荊州刺史桓沖北伐,親率主力進攻襄陽(今湖北襄陽市),慕容垂就與苻叡率兵救援。苻叡以慕容垂為前鋒進至沔水,慕容垂在夜間命士兵每人拿十個火把,將它們縛在樹枝上,讓桓沖以為援軍兵力很強,成功逼使他撤還。同年八月,苻堅正式出兵伐晉,並命苻融及慕容垂率二十五萬兵作為前鋒。苻融攻下了壽春(今安徽壽縣),而慕容垂就率別軍攻下了鄖城(今湖北鄖縣)。

十一月,苻堅於淝水大敗給晉軍,前線的前秦軍隊潰敗,就只有沒有參加淝水之戰的慕容垂一軍是完整的,故此苻堅就率殘軍投靠他。當時慕容寶等人就勸慕容垂殺了苻堅,但慕容垂不肯,更分兵給苻堅。苻堅到了洛陽後已經又招聚了十多萬人,一直到了澠池(今河南澠池縣西),慕容垂表示想去安撫河北,並想去拜謁宗廟。苻堅不顧權翼反對而准許慕容垂所請。

當時駐守鄴城的苻丕知道慕容垂要來,懷疑他意圖作亂,更想襲擊他,只是姜讓以慕容垂未有謀反舉動,勸苻丕先嚴兵守衞,注意其舉動,苻丕才安置慕容垂住在鄴城西部。慕容垂當時雖然不肯乘機殺死苻丕,但仍暗中聯結前燕舊臣,密謀復國。此時,丁零人翟斌起兵,苻堅命慕容垂討伐,苻丕一直怕慕容垂於鄴城作亂,正就打算借此機會送走他,更期望他與翟斌打得兩敗俱傷,好讓自己消滅兩股勢力。於是給了慕容垂二千弱兵及差劣的兵器鎧甲,更派了苻飛龍為副手,意圖以他解決慕容垂。

慕容垂留了慕容農、慕容楷及慕容紹於鄴,在行軍途中閔亮和李毗就從鄴來到,並告知苻丕與苻飛龍的圖謀。慕容垂於是以此激怒士眾,又以兵少為由留於河內郡募兵,十日間就令部眾增至八千人。及後正受翟斌攻擊的豫州刺史苻暉請慕容垂快點進兵,慕容垂向苻飛龍說要改在夜裏行軍,出其不意,然而其實就已與諸子計劃襲殺苻飛龍,終在晚上襲殺了苻飛龍及他手下的一千氐兵。第二日,慕容垂命田山回鄴告知留於鄴城的慕容農等起兵響應自己,三人於是與數十騎微服出走,在列人(今河北肥鄉縣東北)起兵。

燕元元年(384年),慕容垂圖攻洛陽,當時翟斌帳下有前燕宗室慕容鳳及前燕舊臣之子段延等,都勸翟斌奉慕容垂為盟主,慕容垂原本不知翟斌究竟是否真心歸附,並沒答允,但到洛陽後苻暉因知苻飛龍遇害而拒絕以營救苻暉為名的慕容垂進城,至此慕容垂才接受了翟斌。不久慕容垂以洛陽是四戰之地,於是改攻鄴城,至滎陽(今河南滎陽)時,群下請慕容垂稱帝。正月丙戌(384年2月9日),慕容垂則以晉元帝的先例,先稱大將軍、大都督,燕王,承制行事。接著率二十多萬大軍直攻鄴城。慕容垂至鄴後改元「燕元」。

慕容垂接著引兵攻鄴,苻丕派了姜讓去責備慕容垂,又勸他放棄叛變。然而慕容垂卻表示只想苻丕和平離開,獻出鄴城,並允諾與前秦世代友好;又恐嚇若果苻丕不從,將要以兵力強攻,怕苻丕到時即使想全身而退也不能。姜讓聽後指責慕容垂背叛王室,不顧昔日前秦收留自己的恩德,現在要做叛逆的鬼。慕容垂聽後沉默,但沒有聽從旁人所說將姜讓殺害,反表示尊敬,讓他回去。然而最終仍然陳述利害,勸苻丕棄城出走,激得苻堅及苻丕再寫書指責。游說不果後,燕軍開始進攻鄴城,並攻下其外城,苻丕退守中城。接著慕容垂又用二十多萬丁零及烏桓人用梯及地道戰術攻城,但都不成功,於是下令修築長圍作防守,築新興城放置輜重,作長期戰。不久又以漳水灌城,仍不能攻下,於是改為圍困鄴城,只留西邊缺口讓秦軍西走。

燕元二年(385年)四月,東晉將領劉牢之入援鄴城,慕容垂詐敗誘敵,於是撤圍退屯新城,不久再北撤,劉牢之於是追擊,苻丕聞訊亦率軍後繼,劉牢之一路追擊至五橋澤,因為軍隊忙於搶奪燕軍輜重而遭慕容垂擊敗。至八月,苻丕棄守鄴城,燕軍終成功佔領鄴城。十二月,慕容垂正式定都中山(今河北定州市)。燕元三年(386年)正月,慕容垂稱帝,二月改元「建興」,始置百官。八月,慕容垂率兵南征以擴疆土,並於次年正月襲河東地區,擊敗晉濟北太守溫詳。

慕容柔、慕容盛及慕容會於建興三年(387年)從西燕都城長子(今山西長子縣西)到達中山,投奔後燕,當時慕容垂就問當地情況,意圖攻取。不久,慕容永将治下慕容儁、慕容垂子孙不问男女全部杀死。建興八年(392年),慕容垂率軍擊潰了丁零人翟釗,吞併了其部眾。次年十一月,慕容垂就親率七萬兵西征西燕;次年二月慕容垂大發司、冀、青、兗四州兵,分置各兵準備進攻。至五月,燕軍經天井關進攻臺壁,先後擊敗大逸豆歸及小逸豆歸,圍困了臺壁。慕容永自太行回軍臺壁,慕容垂亦率軍到臺壁,兩軍於是交戰。事前慕容垂派了驍騎將軍慕容國在澗下設伏,於是假裝撤退引慕容永追擊,數里後慕容國伏兵出現斷慕容永後路,燕軍於是四面進攻,大敗慕容永。慕容永敗後逃回長子,慕容垂就於六月追至,並圍困城池。至八月,被圍的慕容永困急,先後向東晉及北魏求援,但在援軍到來前大逸豆歸部將伐勤就開城門迎燕軍,慕容垂於是俘虜慕容永並將其殺害,吞併了西燕。

建興二年(386年),拓跋珪復代國,不久改稱魏王,建立了北魏。同年因國內不穩而請後燕援軍,慕容垂派慕容麟救援,終助拓跋珪解決事件。事後雖然拓跋珪不接受後燕封爵,但燕魏兩國每年都有使臣往來。建興七年(391年),拓跋珪派弟弟拓跋觚出使後燕,但當時主事的慕容垂諸子為求良馬,竟扣留了拓跋觚,如此令拓跋珪中斷兩國交往。至建興十一年(395年)五月,慕容垂因北魏侵擾邊塞諸郡而命太子慕容寶等人率兵伐魏。當時魏軍率眾迴避,燕軍於七月到了五原(今內蒙古包頭西北),收降三萬多家及大量糧食,但未與魏軍決戰。而拓跋珪乘當時慕容垂患病,故意阻截燕軍通往中山的道通,捕捉後燕使者,令燕軍與其國內通訊斷絕,從而以慕容垂已死的假消息擾動燕軍軍心。兩軍自九月臨五原河相持至十月,慕容寶及慕容麟因為慕容麟部將慕輿嵩相信慕容垂死訊而圖謀作亂的事件而互相猜疑,終於燒船乘夜撤退。當時河面尚未結冰,慕容寶認為魏軍不能渡河追擊,於是不設斥候監視魏軍。至十一月,魏軍因暴風令河面結冰而追擊,在參合陂追上燕軍,並發動突襲大敗燕軍,大量文武官員及四五萬人的燕軍士兵都被俘,後北魏更阬殺全數燕軍士兵。

慕容寶敗逃回中山,並以參合陂之戰為恥,再請進攻北魏。當時司徒慕容德建言說慕容寶大敗後已被北魏輕視,想要慕容垂親自率兵征服他們,以免留為後患。慕容垂於是命幽州牧慕容隆及留守薊城行臺的慕容盛率手下精兵到中山,決定次年再度伐魏。

三月,慕容垂秘密出兵,跨越青嶺(今河北易縣西南五廻山),經天門(今河北淶源縣)鑿山開路,出魏軍不意直攻雲中郡。慕容垂率軍至獵嶺(今山西代縣夏屋山)時就命慕容隆及慕容農為前鋒,進襲平城(今山西大同市)。當時燕國軍隊都因參合陂之戰大敗而畏懼魏軍,就只有慕容隆這批來自龍城的士兵仍然奮勇進攻;而留守平城的魏將拓跋虔亦沒作防備,故此在閏三月慕容隆兵臨平城時才發現燕軍,率眾抵抗,最終敗死,部眾都被燕軍接收。拓跋虔戰死的消息令身處盛樂(今內蒙古和林格爾北)的拓跋珪感到恐懼,打算出走迴避,但各諸知拓跋虔死訊亦各懷二心,令拓跋珪不知何去何從。

慕容垂經過參合陂戰場時看見被阬殺的士兵骸骨堆積如山,就為他們置祭,士兵們見此皆傷心痛哭,這令慕容垂既慚愧又憤恨,終因而嘔血病發,要坐馬車前進,到平城西北三十里處停駐。當時慕容寶已領兵至雲中,聞訊亦退兵。有叛燕軍人就因而向北魏報告慕容垂已死的消息,拓跋珪想去追擊,但知平城陷落後就打消念頭。慕容垂在平城停留了十日後病情加重,於是修築燕昌城而南歸,至四月癸未日(6月2日)於沮陽(今河北懷來縣)去世,享年七十一歲。諡號為成武皇帝,廟號世祖。

崔浩:「垂藉父兄之資,修復舊業,國人歸之,若夜蟲之就火,少加倚仗,易以立功。」(《資治通鑑·卷一百一十八·晉紀四十》)

\subsubsection{燕元}

\begin{longtable}{|>{\centering\scriptsize}m{2em}|>{\centering\scriptsize}m{1.3em}|>{\centering}m{8.8em}|}
  % \caption{秦王政}\
  \toprule
  \SimHei \normalsize 年数 & \SimHei \scriptsize 公元 & \SimHei 大事件 \tabularnewline
  % \midrule
  \endfirsthead
  \toprule
  \SimHei \normalsize 年数 & \SimHei \scriptsize 公元 & \SimHei 大事件 \tabularnewline
  \midrule
  \endhead
  \midrule
  元年 & 384 & \tabularnewline\hline
  二年 & 385 & \tabularnewline\hline
  三年 & 386 & \tabularnewline
  \bottomrule
\end{longtable}

\subsubsection{建兴}

\begin{longtable}{|>{\centering\scriptsize}m{2em}|>{\centering\scriptsize}m{1.3em}|>{\centering}m{8.8em}|}
  % \caption{秦王政}\
  \toprule
  \SimHei \normalsize 年数 & \SimHei \scriptsize 公元 & \SimHei 大事件 \tabularnewline
  % \midrule
  \endfirsthead
  \toprule
  \SimHei \normalsize 年数 & \SimHei \scriptsize 公元 & \SimHei 大事件 \tabularnewline
  \midrule
  \endhead
  \midrule
  元年 & 386 & \tabularnewline\hline
  二年 & 387 & \tabularnewline\hline
  三年 & 388 & \tabularnewline\hline
  四年 & 389 & \tabularnewline\hline
  五年 & 390 & \tabularnewline\hline
  六年 & 391 & \tabularnewline\hline
  七年 & 392 & \tabularnewline\hline
  八年 & 393 & \tabularnewline\hline
  九年 & 394 & \tabularnewline\hline
  十年 & 395 & \tabularnewline\hline
  十一年 & 396 & \tabularnewline
  \bottomrule
\end{longtable}


%%% Local Variables:
%%% mode: latex
%%% TeX-engine: xetex
%%% TeX-master: "../../Main"
%%% End:

%% -*- coding: utf-8 -*-
%% Time-stamp: <Chen Wang: 2021-11-01 11:59:08>

\subsection{惠愍帝慕容寶\tiny(396-398)}

\subsubsection{生平}

燕惠愍帝慕容寶(355年-398年5月27日),字道祐,昌黎郡棘城县(今辽宁省锦州市义县西北)人,後燕第二任君主,慕容垂的第四子,母親是先段后。慕容垂建後燕後,立慕容寶為太子,曾領燕軍攻伐北魏,但在參合陂之戰慘敗。慕容垂死後慕容寶繼位為帝,但就面對北魏南侵,最終慕容寶沒能保住後燕在中原的土地,率眾北走龍城(今遼寧朝陽市),但先後遇上兒子慕容會及大臣段速骨的叛亂。慕容寶出走後為蘭汗所誘而歸龍城,最終被其殺害。

369年,慕容寶隨父親慕容垂等人自前燕逃亡至前秦,在前秦曾任太子洗馬及萬年令。

《太平御覽》載慕容寶玩樗蒲時向神祈禱富貴,擲出機率只有1/32768的三次「盧」采,讓他決心復國。

383年,前秦天王苻堅南伐東晉,慕容寶任陵江將軍。同年苻堅於淝水之戰大敗,軍隊潰散,只有未參與戰事的慕容垂軍隊仍然完整,於是前往投奔。慕容寶於是向父親建議趁機殺掉苻堅,復興燕國,不過慕容垂不肯。慕容垂終於384年稱燕王,立慕容寶為太子,建後燕。

慕容寶隨後經常留守後燕首都中山(今河北定州市),並在慕容垂在外時留守。387年,時慕容垂南征翟遼,井陘人賈鮑招引北山丁零翟瑤等夜襲中山,並攻下外城。章武王慕容宙率奇兵出外,而慕容寶在內鳴鼓抗敵,兩人夾擊之下大敗賈鮑等人,盡俘其眾,賈鮑及翟瑤隻身逃走。

395年五月,因北魏侵略後燕附塞諸部,慕容垂派慕容寶與慕容農、慕容麟等率八萬進攻北魏。拓跋珪率眾西渡黃河作迴避,並在河南治軍。慕容寶率眾到黃河邊就建造船隻打算渡河進攻,不過就在九月要列兵渡河時就遇上大風,船隻都被吹到南岸去。拓跋珪又派人從後阻截慕容寶與後燕國內的通訊,更派抓來的後燕使者訛稱慕容垂已死,令得軍心不穩,慕容寶亦都相當恐懼。十月辛末(11月23日),慕容寶燒船乘夜逃走,當時黃河尚未結冰,慕容寶以為北魏軍隊不能即時渡河追擊,故此不設斥候監察。不過八日後黃河面就因大風而結了冰,拓跋珪率眾渡河,並派二萬騎兵追擊。燕軍至參合陂時有遇上大風,更有一大片黑色塵土從後而來。僧人支曇猛認為這些都預示魏軍將來,建議慕容寶派兵防禦,但慕容寶以為已經走得很遠,笑而不答。慕容麟更奉承地說:「以殿下神武及強盛的兵眾,足以橫行沙漠了,索虜怎敢遠來呀!曇猛亂說話動搖眾心,應該處死呀!」支曇猛堅持,慕容德亦勸慕容寶聽從,慕容寶於是就派了慕容麟率三萬兵在後防備。不過慕容麟根本沒有防備的心,只顧著打獵。最終魏軍於參合陂突襲燕軍,大量兵眾不是在驚慌下互相踐踏或在河中遇溺而死就是束手就擒。慕容寶等人就帶著數千騎兵一同逃返後燕。戰後魏軍更盡坑俘獲的燕軍。

慕容寶回國後以此敗為恥,屢請慕容垂再次攻魏,慕容垂於是於次年(396年)大舉伐魏,並攻下平城(今山西大同市)。不過慕容垂在經過參合陂時看到被坑殺的燕兵骸骨堆積如山,士兵的痛哭聲又遍布山谷,令慕容垂在愧疚及憤恨下患病,被逼終止北伐。時慕容寶等人正率軍至雲中,追擊迴避的拓跋珪,但知慕容垂患病亦只好撤還。

慕容垂在回軍途中去世,慕容寶待回到中山時才為父發喪,並即位為帝,改元為永康。慕容寶年少無大志,喜歡別人奉承。但當太子後則磨煉自己,崇尚儒學,變得善談論,能作文,又卑委地討好慕容垂身邊小臣,以求得美譽。當時朝野都稱許慕容寶,而慕容垂亦認為他能夠保住家業,相當敬重他。後慕容垂為其建承華觀,又於388年以他錄尚書事,授予處理政務的權力,自己只處理一些重要的事務;又以其領大單于職位。不過慕容垂皇后段氏就曾指出太子才能不足,建議慕容垂立遼西王慕容農或高陽王慕容隆。又指出慕容麟為人奸詐而不肯屈於人下,有輕視太子之心,建議慕容垂早日除去他。不過慕容垂並不接納。慕容寶及慕容麟聽聞段皇后有這番話更是十分痛恨。慕容寶即位後,便派了慕容麟去逼令段后自殺。段后憤怒地說:「你們兄弟連逼殺嫡母的事也做,怎能保護國家!我怎會怕死,就可惜國家快滅亡了。」隨後便自殺。段后死後,慕容寶更因痛恨段后,以其無母后之道而打算不為其行居喪之禮,不過計劃最終在中書令眭邃反對之下擱置。

慕容寶繼位不久,北魏就出兵進攻後燕,並進攻中山,但被慕容隆擊退。及後北魏大人沒根因被拓跋珪厭惡而投降後燕,並請還攻北魏。慕容寶不敢給他重兵,只分了數百騎兵給他。沒根接著夜襲魏營,拓跋珪發覺有變而狼狽逃走,但沒根礙於兵少,無法對魏軍造成大傷害。永康二年(397年),任北魏并州監軍的沒根侄兒醜提因沒根降燕而害怕被株連,於是率部眾回國預備作亂。拓跋珪聞訊就想北返,派使者向後燕求和,但其時慕容寶知北魏有內亂,故此不肯答允,並率步兵十二萬及騎兵三萬七千的大軍到柏肆預備截擊返兵的魏軍。不久,魏軍到了滹沱水南岸紥營,慕容寶就率兵在夜間渡河,並招募了勇士一萬多人夜襲魏營,而慕容寶就在營北列陣作支援。夜襲部隊乘風縱火並迅速發動進攻,魏軍大亂,拓跋珪亦在驚惶中棄營出逃,燕軍到來帳中只得其衣物。不過接著燕軍竟然自亂,互相攻擊。拓跋珪於營外看見這情況就鳴鼓收整部眾,終大敗夜襲軍,更轉攻慕容寶軍,慕容寶只得回到北岸。次日,魏軍已經重整並與燕軍對峙,相反燕軍就士氣盡失。慕容寶最終只得退還中山,北魏軍跟著追擊,屢敗燕軍。慕容寶因屢敗而恐懼,竟拋棄大軍,自率二萬騎兵速速退回中山,又命士兵拋棄戰袍武器,以加快速度,丟失了大量軍需品,而且其時正遇大風雪,大量士兵凍死道上。拓跋珪及後再派兵進圍中山,駐屯在芳林園。當時中山城中將士都想出戰擊退圍城魏軍,慕容隆亦向慕容寶建議乘城中將士的鬥志進攻。慕容寶原本同意,但慕容麟卻多次反對,令慕容寶反悔,慕容隆於是多次列兵備戰都被逼罷兵。後慕容寶又意圖求和,以交還拓跋觚及割常山以西土地為條件,但不久即反悔,氣得拓跋珪親自率軍圍攻中山。當時有數千將士都自願請戰,但慕容隆披甲上馬,正待命令與魏軍決戰時,慕容麟再次勸止慕容寶,令兵眾忿恨,慕容隆亦痛心哭泣。

早前,慕輿皓謀弒慕容寶而改立慕容麟,失敗出逃,但令慕容麟內心不安。就在慕容麟勸止慕容寶派兵出戰當晚,以兵劫逼左衞將軍北地王慕容精,要他率禁軍弒慕容寶。慕容精拒絕,慕容麟就殺害慕容精,出奔西山依附丁零餘眾。其時慕容寶知慕容會正領兵前來,怕慕容麟劫奪慕容會的軍隊,先一步據有龍城,於是召見慕容隆及慕容農,想放棄中山,退保龍城,最終就與太子慕容策、慕容農、慕容隆、慕容盛等人率萬餘騎出城與慕容會軍會合。慕容寶到薊城時身邊的近衞已經散盡,只餘慕容隆的數百騎守。慕容會率眾於薊南迎接後,慕容寶削減慕容會的軍隊而分給慕容農及慕容隆,不久使率眾北歸龍城。當時慕容會整兵與慕容隆及慕容農的騎兵擊敗前來追擊的魏將石河頭,而其時慕容會的兵眾都不想歸屬於慕容農及慕容隆,於是向慕容寶提議讓慕容會率兵解中山之圍,然後還都中山。不過慕容寶拒絕,而慕容寶身邊的人則勸慕容寶殺掉慕容會,慕容寶亦感到慕容會謀反之心,意圖除去他,只因慕容農及慕容隆反對而作罷。慕容會恐懼,就派了仇尼歸襲擊慕容隆及慕容農,殺了慕容隆並重創慕容農。慕容會自宣稱二人謀逆,已經被殺,慕容寶一心要殺慕容會,於是出言讓他安心,接著就暗中命慕輿騰斬殺慕容會,但失敗。慕容會回到其軍中,接著進攻慕容寶,慕容寶就率數百騎直奔龍城。慕容寶及後拒絕慕容會誅除左右,立其為皇太子的要求,於是引來慕容會進攻龍城。慕容寶更在西門特意責罵慕容會,令慕容會下令士兵向慕容寶鼓譟揚威,藉此激起城中士兵憤怒。慕容寶軍於是在黃昏大敗慕容會,接著又派了高雲率敢死隊夜襲慕容會,再敗慕容會,令其逃奔中山。

永康三年(398年),慕容德派李延北上告知拓跋珪北歸的消息,慕容寶於是決意南征。慕容寶率兵至乙連時,長上段速骨、宋赤眉等人因為兵眾恐懼出征作亂,先逼高陽王慕容崇為主,殺害慕容宙及段誼等人。慕容寶與慕容農及慕輿騰會合,試圖討伐段速骨,但因為士兵厭戰,兵眾都潰散,慕容寶等人唯有奔還龍城。其時蘭汗暗中與段速骨勾結,將龍城軍隊帶到龍城以東,大大削弱了龍城的防禦,而慕容盛則內徙附近的人民,選取了一萬多個男丁守城。段速骨攻城時,慕容農因受蘭汗所誘,竟然叛歸段速骨。原本龍城守軍戰鬥力尚足以抵禦段速骨,令段速骨軍死傷甚大,但段速骨讓守軍看見慕容農後就瓦解了軍心,最終令龍城失守,慕容寶等人出走。

慕容寶到薊城後,在慕容盛等人反對下沒有回龍城,轉而想南投慕容德,可是在知道慕容德已稱燕王後就不敢繼續前進。當時慕容盛等人在冀州成功招集了一些支持慕容寶的力量,但其時蘭汗又派人來迎慕容寶,慕容寶以為蘭汗是忠臣,又想到蘭汗是父親慕容垂的舅舅,於是都不再懷疑,決意回龍城。慕容寶快到龍城時,蘭汗就派了弟蘭加難去迎接,但同時又命兄蘭堤封閉城門,最終蘭加難引慕容寶到龍城外邸並將之殺害,享年四十四歲。蘭汗殺太子慕容策及王公大臣,自稱大都督、大將軍、大單于,昌黎王。不久慕容盛殺蘭汗,改慕容寶諡號為惠愍皇帝,上廟號烈宗。

\subsubsection{永康}

\begin{longtable}{|>{\centering\scriptsize}m{2em}|>{\centering\scriptsize}m{1.3em}|>{\centering}m{8.8em}|}
  % \caption{秦王政}\
  \toprule
  \SimHei \normalsize 年数 & \SimHei \scriptsize 公元 & \SimHei 大事件 \tabularnewline
  % \midrule
  \endfirsthead
  \toprule
  \SimHei \normalsize 年数 & \SimHei \scriptsize 公元 & \SimHei 大事件 \tabularnewline
  \midrule
  \endhead
  \midrule
  元年 & 396 & \tabularnewline\hline
  二年 & 397 & \tabularnewline\hline
  三年 & 398 & \tabularnewline
  \bottomrule
\end{longtable}


%%% Local Variables:
%%% mode: latex
%%% TeX-engine: xetex
%%% TeX-master: "../../Main"
%%% End:

%% -*- coding: utf-8 -*-
%% Time-stamp: <Chen Wang: 2021-11-01 11:59:56>

\subsection{昭武帝慕容盛\tiny(398-401)}

\subsubsection{开封公慕容详生平}

慕容詳(?-397年),昌黎郡棘城县人(今辽宁省锦州市义县)人,追尊燕文明帝慕容皝的曾孙,后燕宗室,封开封公,後一度稱燕帝。

北魏君主、魏王拓跋珪率領魏軍圍攻后燕首都中山,後燕永康二年(397年),後燕不敵北魏的進攻,皇帝慕容寶等撤出都城中山(今中國河北省定州市),出逃龍城(今辽宁省朝阳市)。依然使用永康年號至398年。城內大亂,慕容詳當時不及跟隨撤退,因此被推為盟主以抵禦北魏的攻擊。然而,慕容詳為鞏固自己的地位,不斷翦除城內其他勢力。同年稍後不久,魏軍退卻后,慕容詳即皇帝位,改元建始。

由於慕容詳嗜酒好殺,不恤士民。七月,中山城民遂迎趙王慕容麟入城,慕容麟入城後,慕容詳被逮捕後處死。慕容麟自立,改元延平。

\subsubsection{趙王慕容麟生平}

慕容麟(4世纪-398年),昌黎郡棘城县(今辽宁省锦州市义县)人,後燕成武帝慕容垂之子,婢妾所生。惠愍帝慕容寶庶弟。原為後燕的趙王,後來一度稱燕帝。

早年慕容垂於前燕時期,叛前燕奔前秦時,慕容麟曾逃回前燕告發(369年)。其嫡長兄慕容令被前燕放逐後,欲偷襲龍城(今中國遼寧省遼陽縣),亦是被慕容麟告發,事敗身死(370年)。雖然慕容麟屢次出賣父兄,但後來前秦統一華北,慕容垂回到前燕故地時,還是不忍心殺掉慕容麟,最後是殺了慕容麟的母親頂罪,而把他放逐在外,很少見面。

383年,慕容垂於前秦淝水之戰敗後,陰謀背叛,慕容麟從中貢獻不少計策,慕容垂大為讚賞,待慕容麟開始與其他兒子相同。384年,慕容垂建後燕,慕容麟被任命為撫軍大將軍。同年,率軍攻陷中山(今中國河北省定州市),聲威大振,遂留守中山。386年,慕容垂稱帝後,慕容麟被封為趙王。其後數年,帶領燕軍南征北討,立下不少戰功。396年,慕容垂去世,太子慕容寶繼位,慕容麟被任命為尚書左僕射。

395年的參合陂之战及397年的柏肆之战,後燕二度慘敗給北魏,國力大衰。397年,北魏君主魏王拓跋珪率領魏軍進圍後燕都城中山,慕容麟謀叛,遂以武力威脅北地王慕容精,命其率領禁軍謀殺慕容寶,慕容精拒絕,慕容麟於是殺慕容精,逃出中山,依附丁零遺眾。不久,慕容寶等率領部下撤出中山,出逃龍城,依然使用永康年號至398年。城內大亂,開封公慕容詳被推為盟主以抵禦北魏的攻擊,後來魏軍退卻后,慕容詳即皇帝位,改元建始。但由於慕容詳嗜酒好殺,不恤士民,中山城民遂迎慕容麟入城,慕容麟入城後,殺慕容詳,亦稱帝,改元延平。但隨後北魏再攻中山,又被北魏擊敗,慕容麟自去年號,南奔鄴城(今中國河南省臨漳縣)投靠范陽王慕容德,並不再稱帝。

398年,慕容麟向慕容德上尊號,慕容德於是稱燕王,建立南燕,但不久慕容麟又陰謀推翻慕容德,因此被慕容德所殺。

\subsubsection{兰汗生平}

蘭汗(?-398年8月15日),昌黎郡棘城县(今辽宁省锦州市义县)人,慕容垂堂舅。

昌黎王蘭汗與段速骨密謀叛亂,后又杀段速骨,派兰加难诱杀慕容寶,改元青龍。夺位当年即为慕容盛所杀。蘭穆、蘭堤、蘭加難、蘭和、蘭揚也都被杀。慕容盛继位。

\subsubsection{昭武帝慕容盛生平}

燕昭武帝慕容盛(373年-401年9月13日),字道運,十六国后燕国主,慕容寶之庶長子。

年少時沈實敏銳,富謀略。當前秦天王苻堅誅殺慕容氏時,与叔父慕容柔潛逃投靠慕容沖。385年,慕容冲在阿房宫即皇帝位。慕容盛对慕容柔说:“夫十人之长,亦须才过九人,然后得安。今中山王才不逮人,功未有成,而骄汰已甚,殆难济乎!”慕容冲果然很快被杀,慕容柔、慕容盛以及慕容盛的弟弟慕容会又投靠慕容永。慕容盛指出三人是慕容垂子孙,正被世系疏远的慕容永猜疑,不如投奔祖父慕容垂。387年,他们从长子县一起逃回了后燕。不久慕容永果然尽杀治下的慕容儁、慕容垂子孙。

其父慕容宝登基后,慕容盛反对立祖父慕容垂所爱的庶弟慕容会为储,而支持嫡出的三弟慕容策。慕容会后来谋反被诛。

段速骨叛变后,慕容寶想南奔投靠叔叔慕容德,被慕容盛劝阻。慕容德已自称燕王,不但无意迎接慕容宝还意图谋害,慕容宝又回到龙城,受兰汗诓骗而遭殺害,慕容策亦遇害。慕容盛因為是蘭汗的女婿,与妻子關係甚篤,非但得以不死,還被蘭汗封做侍中。慕容盛乘机离间蘭汗、兰堤和兰加难三兄弟,派遣太原王慕容奇(兰汗外孙)在建安聚眾討伐蘭汗,当兰汗派出其兄长太尉兰堤讨伐时,慕容盛又反間蘭汗称慕容奇实力不足,兰堤才是慕容奇的幕后主谋。于是,兰汗将兰堤之职务转予抚军将军仇尼慕。如此种种,导致兰堤和兰加难兄弟生惧而背叛兰汗。太子兰穆提醒兰汗,慕容盛是仇家,必与慕容奇勾结,兰汗因而召见慕容盛,但慕容盛在妻兰王妃告密下佯病不出,躲过一劫。蘭汗派遣兄子蘭全反擊慕容奇卻反被滅。兰穆出兵讨伐兰堤、兰加难前大宴将士,兰汗父子喝得酩酊大醉,慕容盛与李旱等人趁機杀死兰穆,又引兵將兰汗乱刀砍死。又遣李旱及张真袭杀兰汗子鲁公兰和于令支及陈公兰扬于白狼,並捕杀兰堤和兰加难。

為父復仇之後,慕容盛一度想斬草除根殺死妻子蘭氏,母后丁氏不忍,進行勸阻,因此只廢黜蘭氏,终身未曾立后。慕容盛於398年8月19日(七月廿一辛亥)只改元建平,仍以长乐王称制,诸王皆降为公。又命慕容奇停止用兵,慕容奇抗命且带兵来攻,被慕容盛打败並赐死。11月12日(十月十七丙子)即皇帝位,並誅殺幽州刺史慕容豪、尚書左僕射張通及昌黎尹張順等人。399年改年號為長樂。400年2月11日(正月初一壬子)慕容盛自贬号为庶人天王。慕容盛後來討伐高丽及庫莫奚有功,然治法太嚴,刑罰殘忍,401年9月13日(八月二十壬辰),左將軍慕容國與殿中將軍秦輿、段贊等人密謀暗殺慕容盛,卻東窗事發,眾人皆被誅,軍中大亂。最終平亂時,慕容盛本人卻身中暗器,傷重不治,享年29歲,在位僅三年,慕容熙繼其位。

父親慕容寶。妻子蘭汗的女兒蘭氏。兒子慕容定在慕容盛被殺害後的時候還是年紀幼小。

慕容盛于401年闰八月十九葬于兴平陵(具体方位不详),庙号中宗,谥号昭武皇帝。

\subsubsection{建平}

\begin{longtable}{|>{\centering\scriptsize}m{2em}|>{\centering\scriptsize}m{1.3em}|>{\centering}m{8.8em}|}
  % \caption{秦王政}\
  \toprule
  \SimHei \normalsize 年数 & \SimHei \scriptsize 公元 & \SimHei 大事件 \tabularnewline
  % \midrule
  \endfirsthead
  \toprule
  \SimHei \normalsize 年数 & \SimHei \scriptsize 公元 & \SimHei 大事件 \tabularnewline
  \midrule
  \endhead
  \midrule
  元年 & 396 & \tabularnewline
  \bottomrule
\end{longtable}

\subsubsection{长乐}

\begin{longtable}{|>{\centering\scriptsize}m{2em}|>{\centering\scriptsize}m{1.3em}|>{\centering}m{8.8em}|}
  % \caption{秦王政}\
  \toprule
  \SimHei \normalsize 年数 & \SimHei \scriptsize 公元 & \SimHei 大事件 \tabularnewline
  % \midrule
  \endfirsthead
  \toprule
  \SimHei \normalsize 年数 & \SimHei \scriptsize 公元 & \SimHei 大事件 \tabularnewline
  \midrule
  \endhead
  \midrule
  元年 & 399 & \tabularnewline\hline
  二年 & 400 & \tabularnewline\hline
  三年 & 401 & \tabularnewline
  \bottomrule
\end{longtable}


%%% Local Variables:
%%% mode: latex
%%% TeX-engine: xetex
%%% TeX-master: "../../Main"
%%% End:

%% -*- coding: utf-8 -*-
%% Time-stamp: <Chen Wang: 2021-11-01 12:00:09>

\subsection{昭文帝慕容熙\tiny(401-407)}

\subsubsection{生平}

燕昭文帝慕容熙(385年-407年9月14日),字道文,一字長生,十六國時期後燕國君主,鮮卑人,成武帝慕容垂的幼子,惠愍帝慕容寶之弟,母親是貴嬪段氏。原封河間王,蘭汗之亂時曾被封為遼東公,慕容盛即位後,封河間公。

後燕長樂三年(401年),慕容盛被變軍殺害。慕容盛有兒子慕容定,年紀幼小。群臣希望慕容盛之弟慕容元繼位。但慕容熙因与慕容盛之母丁太后有私情,備受她寵愛,八月癸巳日(9月14日),遂被密迎入宮即天王位,慕容元被賜死,不久慕容熙改元光始。次年(402年),慕容熙又害死了慕容定,且娶了苻秦中山尹苻謨的兩個女兒苻娀娥為貴人、苻訓英為貴嬪,苻訓英極其受寵。丁太后怨恨,遂謀廢慕容熙,事洩,丁太后被殺。

慕容熙立苻訓英為皇后,苻娀娥為貴人,非常寵愛苻氏姐妹,因此興築宮殿、遊玩打獵,導致軍民死亡的數以萬計。苻娀娥生病,有人自稱能醫,結果醫死了,慕容熙遂將醫生支解後焚燒,追封苻娀娥為愍皇后。慕容熙與苻訓英更是玩樂不知節制。元始五年(405年)攻高句麗遼東城,原本城將攻陷,慕容熙只為了要與苻后一同坐輦車進城,因而命軍暫緩登城,以致延誤戰機,不能攻下遼東。次年(406年),後燕攻契丹未果而回師,又為了苻后想要觀戰而臨時拋棄輜重轉而偷襲高句麗,致士卒馬匹,疲累寒冷,沿路死亡不可勝數。又如苻后夏天想要吃凍魚,冬天要吃生地黃,官員也因不能取得而被斬首。

建始元年(407年),苻后去世,慕容熙痛不欲生,喪禮上命檢查百官有無哭泣,規定未哭者給予處罰,群臣只好口含辣物以刺激流淚。又賜死高陽王慕容隆的王妃張氏以殉葬,右仆射韦璆等人都害怕自己去殉葬,每天都洗澡换衣等候命令。此外規定家家戶戶都要參與建造苻后陵墓的工程,更使得國家財政揮霍一空。臨葬,慕容熙竟打開棺材,与苻訓英的屍體親熱一番,才准下葬。

由於早先中衛將軍馮跋與其弟馮素弗曾因事獲罪於後燕帝慕容熙,因此慕容熙一直有殺馮跋兄弟之意。七月甲子日(407年9月14日),馮跋兄弟於是趁慕容熙送葬苻后時起事,推高雲(慕容雲)為燕王,慕容熙被生擒後斬首,和苻訓英合葬。後來被諡昭文皇帝。

\subsubsection{光始}

\begin{longtable}{|>{\centering\scriptsize}m{2em}|>{\centering\scriptsize}m{1.3em}|>{\centering}m{8.8em}|}
  % \caption{秦王政}\
  \toprule
  \SimHei \normalsize 年数 & \SimHei \scriptsize 公元 & \SimHei 大事件 \tabularnewline
  % \midrule
  \endfirsthead
  \toprule
  \SimHei \normalsize 年数 & \SimHei \scriptsize 公元 & \SimHei 大事件 \tabularnewline
  \midrule
  \endhead
  \midrule
  元年 & 401 & \tabularnewline\hline
  二年 & 402 & \tabularnewline\hline
  三年 & 403 & \tabularnewline\hline
  四年 & 404 & \tabularnewline\hline
  五年 & 405 & \tabularnewline\hline
  六年 & 406 & \tabularnewline
  \bottomrule
\end{longtable}

\subsubsection{建始}

\begin{longtable}{|>{\centering\scriptsize}m{2em}|>{\centering\scriptsize}m{1.3em}|>{\centering}m{8.8em}|}
  % \caption{秦王政}\
  \toprule
  \SimHei \normalsize 年数 & \SimHei \scriptsize 公元 & \SimHei 大事件 \tabularnewline
  % \midrule
  \endfirsthead
  \toprule
  \SimHei \normalsize 年数 & \SimHei \scriptsize 公元 & \SimHei 大事件 \tabularnewline
  \midrule
  \endhead
  \midrule
  元年 & 407 & \tabularnewline
  \bottomrule
\end{longtable}


%%% Local Variables:
%%% mode: latex
%%% TeX-engine: xetex
%%% TeX-master: "../../Main"
%%% End:

%% -*- coding: utf-8 -*-
%% Time-stamp: <Chen Wang: 2021-11-01 12:00:17>

\subsection{惠懿帝高雲\tiny(401-407)}

\subsubsection{生平}

燕惠懿帝高雲(4世纪-409年11月6日),曾改名慕容雲,字子雨,高句驪人。十六国時期後燕末代君主,一說為北燕开国国主,称号天王。

早期的高雲於後燕時沉默寡言,並沒有什麼名氣,只有中衛將軍馮跋看出他的氣度與他結交。

後燕永康二年(397年),高雲因率軍擊敗慕容寶之子慕容會的叛軍,被慕容寶收養,賜姓慕容氏,封夕陽公。

後燕建初元年(407年)馮跋反,殺皇帝慕容熙,在馮跋支持之下,慕容雲即天王位,改元曰正始,國號大燕,恢復原本的高姓。高雲自知無功而登大位,因此培養一批禁衛保護自己,但後來反被禁衛離班和桃仁所殺,高雲死後被諡惠懿皇帝。

由於對高雲是否屬後燕慕容氏一族成員的看法不同,因此有人認為高雲是後燕末任君主,也有人把他視為北燕立國君主。

\subsubsection{正始}

\begin{longtable}{|>{\centering\scriptsize}m{2em}|>{\centering\scriptsize}m{1.3em}|>{\centering}m{8.8em}|}
  % \caption{秦王政}\
  \toprule
  \SimHei \normalsize 年数 & \SimHei \scriptsize 公元 & \SimHei 大事件 \tabularnewline
  % \midrule
  \endfirsthead
  \toprule
  \SimHei \normalsize 年数 & \SimHei \scriptsize 公元 & \SimHei 大事件 \tabularnewline
  \midrule
  \endhead
  \midrule
  元年 & 407 & \tabularnewline\hline
  二年 & 409 & \tabularnewline
  \bottomrule
\end{longtable}


%%% Local Variables:
%%% mode: latex
%%% TeX-engine: xetex
%%% TeX-master: "../../Main"
%%% End:



%%% Local Variables:
%%% mode: latex
%%% TeX-engine: xetex
%%% TeX-master: "../../Main"
%%% End:

%% -*- coding: utf-8 -*-
%% Time-stamp: <Chen Wang: 2019-12-19 15:37:04>


\section{西秦\tiny(385-431)}

\subsection{简介}

西秦(385年-400年,409年-431年)是中国历史上十六国时期鲜卑人乞伏國仁建立的政权。其国号“秦”以地处战国时秦国故地为名。《十六国春秋》始用西秦之称,以别于前秦和后秦,后世袭用之。

公元385年,鲜卑酋长乞伏国仁在陇西称大单于,又被前秦封为苑川王,都勇士川(今甘肃榆中)。388年,其弟乞伏乾歸立,称大单于,河南王,迁都金城(今甘肃兰州西)。400年為後秦所滅。409年,二月,乞伏乾归自后秦返回苑川。七月,西秦复国,复都苑川。412年,乞伏熾磐又迁都枹罕(今甘肃临夏市东北)。

最盛时期,其统治范围包括甘肃西南部,青海部分地区。

431年被夏國所灭。

%% -*- coding: utf-8 -*-
%% Time-stamp: <Chen Wang: 2019-12-19 15:40:51>

\subsection{乞伏国仁\tiny(385-388)}

\subsubsection{生平}

乞伏國仁(?-388年),陇西鲜卑人。十六国时期西秦政權奠定者。在前秦官至前將軍,淝水之戰後乘機自立,但仍與前秦保持一定關係。雖然一般認為乞伏國仁是西秦建立者,惟其在位期間,只受前秦封為苑川王,尚未正式稱秦王。一直至394年,國仁繼承人乞伏乾歸才稱秦王。

其父乞伏司繁受前秦天王苻堅封為南單于,並駐鎮勇士川(今甘肅榆中)。秦建元十二年(376年),司繁死,乞伏國仁繼位。前秦建元十九年(383年)淝水之戰時,苻堅原命國仁为前将军,领先锋骑,後國仁叔父乞伏步頹叛于陇西,苻堅派國仁回師討伐,步頹反而迎接國仁。及前秦淝水之戰失利,國仁即趁機吞併其他部族,聚眾共十多萬。前秦太安元年(385年),苻堅為姚萇所殺後,國仁自称大都督、大将军、大单于、领秦、河二州牧,改元建义,建都勇士城(今甘肅榆中)。

就在乞伏國仁自立次年,南安郡豪族祕宜就率領五萬羌、胡人進攻乞伏國仁,並四面來攻。乞伏國仁決意先聲奪人,於是自率五千人突襲祕宜,並大敗對方。祕宜於是逃奔南安郡,同年便率三萬多戶人口歸降。建義三年(387年),前秦皇帝苻登以乞伏國仁為大都督、都督雜夷諸軍事、大將軍、大單于、苑川王。同年乞伏國仁率軍進攻密貴、裕苟及提倫三位鮮卑大人,又大敗來攻的高平鮮卑首領沒弈干及東胡金熙,密貴等三人於是大懼,率部歸降。建義四年,乞伏國仁又擊敗了鮮卑人越質叱黎。同年国仁死,谥宣烈王,庙号烈祖,弟乞伏乾歸繼位。

\subsubsection{建义}

\begin{longtable}{|>{\centering\scriptsize}m{2em}|>{\centering\scriptsize}m{1.3em}|>{\centering}m{8.8em}|}
  % \caption{秦王政}\
  \toprule
  \SimHei \normalsize 年数 & \SimHei \scriptsize 公元 & \SimHei 大事件 \tabularnewline
  % \midrule
  \endfirsthead
  \toprule
  \SimHei \normalsize 年数 & \SimHei \scriptsize 公元 & \SimHei 大事件 \tabularnewline
  \midrule
  \endhead
  \midrule
  元年 & 385 & \tabularnewline\hline
  二年 & 386 & \tabularnewline\hline
  三年 & 387 & \tabularnewline\hline
  四年 & 388 & \tabularnewline
  \bottomrule
\end{longtable}


%%% Local Variables:
%%% mode: latex
%%% TeX-engine: xetex
%%% TeX-master: "../../Main"
%%% End:

%% -*- coding: utf-8 -*-
%% Time-stamp: <Chen Wang: 2019-12-19 15:41:44>

\subsection{武元王\tiny(388-412)}

\subsubsection{生平}

秦武元王乞伏乾歸(?-412年),陇西鲜卑人。十六国时期西秦開國君王,苑川王乞伏國仁弟。乾歸在位初期曾受前秦官爵,並曾響應前秦號召領兵協助,但皇帝苻登敗死後就逼逐繼承的苻崇,後苻崇討伐乾歸時更敗死,令前秦亡國,並乘機併吞其隴西土地,後稱「秦王」,西秦故此得名。後乾歸敗給後秦,被逼投降南涼,最終向後秦歸降,暫時亡國。但因後秦王姚興將其放回原地,並將部眾還給他,令其有機會復興,最終趁後秦漸漸衰弱時復國,並進攻鄰近的南涼、後秦、吐谷渾及其他胡人部落。乞伏乾歸於412年被侄兒乞伏公府所殺,其太子乞伏熾磐討平後繼位。

建义元年(385年)乞伏國仁自称大都督、大将军、单于,领秦、河二州牧。任命乾歸為上將軍。建義四年(388年)國仁去世,群臣認為國仁子乞伏公府年幼,乃推乾归为大都督、大将军、大单于、河南王,改元太初,遷都金城(今甘肅蘭州)。太初二年(389年)受前秦帝苻登封為金城王。

乾歸於太初二年(389年)即討平了休官部落的阿敦及侯年二部,盡降其眾,於是威振西部,鮮卑的豆留螱奇、叱豆渾、南丘鹿結、休官部的曷呼奴及盧水尉地跋都率眾歸降,而乾歸亦各署官爵;枹罕羌彭奚念亦來歸附,乾歸以其為北河州刺史。次年(390年),吐谷渾亦遣使上貢,乾歸又以吐谷渾君主視連為白蘭王、沙州牧。

太初四年(391年),沒弈干遣使結好,並派兩個兒子為人質請兵一共進攻鮮卑大兜,乾歸答允並領兵進攻大兜的安陽城,大兜退守鳴蟬堡但還是被乾歸攻陷,乾歸於是收擄其部眾回國。戰後乾歸歸還了沒弈干的兩個兒子,但沒弈干不久又改結劉衞辰,乾歸於是率兵一萬攻伐沒弈干,並在他樓城射傷沒弈干的眼睛。

太初七年(394年),苻登知後秦皇帝姚萇去世,認為滅後秦時機已到,於是起兵進攻後秦,又拜乾歸為左丞相、河南王、領秦梁益涼沙五州牧,加賜九錫。可是苻登卻遭姚萇太子姚興擊敗,退屯馬毛山,並派了兒子苻宗為質子,向乾歸請兵,並進封乾歸為梁王。乾歸於是派了乞伏益州率兵一萬營救,但苻登要出迎乞伏益州時被姚興擊敗,更被俘殺。苻登太子苻崇於湟中繼位,但不久乾歸就驅逐苻崇,苻崇只好投奔氐族仇池部隴西王楊定。二人組成聯軍反攻乾歸,乾歸派兵抵抗,終擊敗聯軍,斬楊定及苻崇,前秦滅亡,西秦自此盡有隴西。不久,乾歸自稱秦王,又於次年(395年)遷都苑川西城(今甘肅靖遠)。

早於太初五年(392年),呂光就曾派呂方及呂寶進攻乾歸,乾歸初敗於鳴雀峽,退屯青岸。而呂方屯黃河北,呂寶則渡河追擊,乾歸於是派彭奚念斷絕呂寶歸路,率兵反擊,屢敗呂寶,終呂寶等一萬多人戰死。至太初八年(395年),呂光親自率十萬軍進攻乾歸,左輔將軍密貴周及莫者羖羝就勸乾歸向呂光稱藩,乾歸終聽從並以兒子乞伏敕勃作為人質,呂光亦率軍退還。可是不久乾歸就後悔了,殺了密貴周及莫者羖羝。

太初九年(396年),涼州牧乞伏軻彈因與秦州牧乞伏益州不睦,故出奔呂光,呂光於是以乾歸多次反覆而興兵討伐。其時眾臣都請乾歸出奔成紀迴避,但乾歸不願。呂光派呂延等人攻下了臨洮、武始、河關,又命呂纂進攻金城,乾歸率兵救援,但呂光派了王寶及徐炅率兵五千邊擊,令乾歸恐懼不敢前進,終令金城陷落。乾歸於是行反間計,傳出假消息稱乾歸部眾已潰散,乾歸已東逃到成紀。呂延信以為真,於是輕軍進攻,最終被乾歸擊敗,呂延更戰死。呂延敗後,呂光亦退兵。

太初三年(390年),視連去世,視羆繼位,拒絕接受乾歸的封號。乾歸知道後大怒,但因為忌憚吐谷渾強盛,於是暫時容忍,仍然交好。至太初十一年(398年)就派了乞伏益州、慕兀及翟瑥率二萬騎進攻吐谷渾,在度周川大敗視羆,逼其送兒子宕豈為質求和。

太初十三年(400年),乾歸復遷都苑川(今甘肅榆中縣北)。同年,後秦姚碩德來攻,乾歸率眾到隴西對抗。兩軍對峙期間,姚碩德軍柴草缺乏,後秦王姚興於是親自出軍。乾歸見已是國家存亡的危機,於是放手一搏,決定集中力量消滅姚興軍隊,殺死姚興,欲求消除危機之餘更吞併後秦。乾歸因而命慕兀率二萬兵為中軍,駐柏楊(今甘肅清水縣西南);羅敦率四萬兵為外軍,駐侯辰谷。而自己就率數千騎等候姚興軍。但一晚,乾歸遇上大風和大霧,與中軍失去聯絡,被逼與外軍會合。天亮後乾歸就與姚興軍交戰,大敗。乾歸敗歸苑川,接著又逃到金城,並命手下各豪帥留下來歸降後秦,自己西走允吾(今甘肅皋蘭縣西北),望一天復興國家時再見。西秦滅亡。乾歸到允吾後向禿髮利鹿孤投降,被禿髮傉檀迎到晉興,待以上賓之禮。

後秦退兵後,南羌梁戈等人招引乾歸,乾歸打算前赴,但事情卻洩漏給禿髮利鹿孤知道,禿髮吐雷因而出屯捫天嶺。乾歸恐為禿髮利鹿孤所殺,於是送妻子及乞伏熾磐等諸子到西平為人質,自己出奔枹罕(今甘肅臨夏市),向後秦投降。

乾歸到長安後,受封為持节、都督河南诸军事、镇远将军、河州刺史、归义侯,隔年(401年)更被派還西秦故都苑川鎮守,並歸還其部眾。至後秦弘始四年(402年),乞伏熾磐逃奔後秦,姚興也授他官位,不久更加乾歸散騎常侍、左賢王。乾歸於降後秦時期,曾經受命與齊難等後秦將領到姑臧(今甘肅武威)接受後涼王呂隆投降。乾歸又屢攻仇池,先後攻破仇池所領的皮氏堡和西陽堡。乾歸更於405年攻破吐谷渾,其中吐谷渾君主大孩更在敗走後不久去世,乾歸俘擄了一萬多人。

弘始九年(407年),姚興認為乾归的勢力逐漸強大,難以控制,於是趁其入朝的機會將其留在長安當主客尚書,讓其子乞伏熾磐代領其眾。弘始十一年(409年),乞伏熾磐攻伐彭奚念,攻陷其佔領的枹罕。其時乾歸正隨姚興在平涼,得到熾磐的通報後就逃回苑川。乾歸回去後不久到枹罕聚集三萬部眾,並帶他們遷居度堅山,留熾磐守枹罕,接著乾歸更稱秦王,改元「更始」,再次置官爵並讓手下恢復原來在西秦的職位,正式復國。

乾歸復國後,先派兵進攻薄地延,將其部落遷至苑川,後又派兵攻下後秦的金城郡,並置守戍,從而於更始二年(410年)遷都回苑川。略陽、南安、隴西等後秦轄郡都先後遭西秦軍攻下。當時後秦無力討伐,只得任命乾歸為使持節、散騎常侍、都督隴西北匈奴雜胡諸軍事征西大將軍、河州牧、大單于、河南王。乾歸當時正欲攻取河西地區,於是暫時接受。

乾歸又派兵攻伐南涼,擊敗了南涼太子禿髮虎台。另又率兵攻下後秦略陽太守姚龍的柏龍堡及南平太守王憬的水洛城。後又攻殺襲據枹罕的彭利髮,收復了枹罕。更始四年(412年),乾歸更率二萬騎攻破吐谷渾支統阿若干,令吐谷渾向其投降。

同年六月,乾歸為其侄乞伏公府所弒,十余个儿子一并遇害。乞伏熾磐消滅乞伏公府後繼位,諡乾歸為武元王,庙號高祖,葬於枹罕。

\subsubsection{太初}

\begin{longtable}{|>{\centering\scriptsize}m{2em}|>{\centering\scriptsize}m{1.3em}|>{\centering}m{8.8em}|}
  % \caption{秦王政}\
  \toprule
  \SimHei \normalsize 年数 & \SimHei \scriptsize 公元 & \SimHei 大事件 \tabularnewline
  % \midrule
  \endfirsthead
  \toprule
  \SimHei \normalsize 年数 & \SimHei \scriptsize 公元 & \SimHei 大事件 \tabularnewline
  \midrule
  \endhead
  \midrule
  元年 & 388 & \tabularnewline\hline
  二年 & 389 & \tabularnewline\hline
  三年 & 390 & \tabularnewline\hline
  四年 & 391 & \tabularnewline\hline
  五年 & 392 & \tabularnewline\hline
  六年 & 393 & \tabularnewline\hline
  七年 & 394 & \tabularnewline\hline
  八年 & 395 & \tabularnewline\hline
  九年 & 396 & \tabularnewline\hline
  十年 & 397 & \tabularnewline\hline
  十一年 & 398 & \tabularnewline\hline
  十二年 & 399 & \tabularnewline\hline
  十三年 & 400 & \tabularnewline
  \bottomrule
\end{longtable}

\subsubsection{更始}

\begin{longtable}{|>{\centering\scriptsize}m{2em}|>{\centering\scriptsize}m{1.3em}|>{\centering}m{8.8em}|}
  % \caption{秦王政}\
  \toprule
  \SimHei \normalsize 年数 & \SimHei \scriptsize 公元 & \SimHei 大事件 \tabularnewline
  % \midrule
  \endfirsthead
  \toprule
  \SimHei \normalsize 年数 & \SimHei \scriptsize 公元 & \SimHei 大事件 \tabularnewline
  \midrule
  \endhead
  \midrule
  元年 & 409 & \tabularnewline\hline
  二年 & 410 & \tabularnewline\hline
  三年 & 411 & \tabularnewline\hline
  四年 & 412 & \tabularnewline
  \bottomrule
\end{longtable}


%%% Local Variables:
%%% mode: latex
%%% TeX-engine: xetex
%%% TeX-master: "../../Main"
%%% End:

%% -*- coding: utf-8 -*-
%% Time-stamp: <Chen Wang: 2021-11-01 12:02:19>

\subsection{文昭王乞伏熾磐\tiny(412-428)}

\subsubsection{生平}

文昭王乞伏熾磐(?-428年),十六国时期西秦国君主,乞伏乾歸長子。

熾磐個性勇略過人,400年,西秦第一次亡國後,被送往南涼為人質。後秦弘始四年(402年)熾磐自南涼奔後秦與乾歸會合。熾磐於後秦期間,召集軍隊據地自立。弘始十一年(409年)乾歸逃回西秦舊地,再稱秦王,西秦復國,熾磐又被立為太子。西秦更始四年(412年)乾歸為侄乞伏公府所弒,熾磐擒殺公府,繼位,稱河南王,改元永康。永康三年(414年)滅南涼,復稱秦王,其後主要與北涼爭戰。建弘九年(428年)病死,諡文昭王,廟號太祖。其子乞伏暮末继位。

\subsubsection{永康}

\begin{longtable}{|>{\centering\scriptsize}m{2em}|>{\centering\scriptsize}m{1.3em}|>{\centering}m{8.8em}|}
  % \caption{秦王政}\
  \toprule
  \SimHei \normalsize 年数 & \SimHei \scriptsize 公元 & \SimHei 大事件 \tabularnewline
  % \midrule
  \endfirsthead
  \toprule
  \SimHei \normalsize 年数 & \SimHei \scriptsize 公元 & \SimHei 大事件 \tabularnewline
  \midrule
  \endhead
  \midrule
  元年 & 412 & \tabularnewline\hline
  二年 & 413 & \tabularnewline\hline
  三年 & 414 & \tabularnewline\hline
  四年 & 415 & \tabularnewline\hline
  五年 & 416 & \tabularnewline\hline
  六年 & 417 & \tabularnewline\hline
  七年 & 418 & \tabularnewline\hline
  八年 & 419 & \tabularnewline
  \bottomrule
\end{longtable}

\subsubsection{建弘}

\begin{longtable}{|>{\centering\scriptsize}m{2em}|>{\centering\scriptsize}m{1.3em}|>{\centering}m{8.8em}|}
  % \caption{秦王政}\
  \toprule
  \SimHei \normalsize 年数 & \SimHei \scriptsize 公元 & \SimHei 大事件 \tabularnewline
  % \midrule
  \endfirsthead
  \toprule
  \SimHei \normalsize 年数 & \SimHei \scriptsize 公元 & \SimHei 大事件 \tabularnewline
  \midrule
  \endhead
  \midrule
  元年 & 420 & \tabularnewline\hline
  二年 & 421 & \tabularnewline\hline
  三年 & 422 & \tabularnewline\hline
  四年 & 423 & \tabularnewline\hline
  五年 & 424 & \tabularnewline\hline
  六年 & 425 & \tabularnewline\hline
  七年 & 426 & \tabularnewline\hline
  八年 & 427 & \tabularnewline\hline
  九年 & 428 & \tabularnewline
  \bottomrule
\end{longtable}


%%% Local Variables:
%%% mode: latex
%%% TeX-engine: xetex
%%% TeX-master: "../../Main"
%%% End:

%% -*- coding: utf-8 -*-
%% Time-stamp: <Chen Wang: 2019-12-19 15:45:01>

\subsection{乞伏暮末\tiny(428-431)}

\subsubsection{生平}

乞伏暮末(-431年),一名慕末,十六国时期西秦国君主,乞伏熾磐二子。

西秦建弘九年(428年)熾磐去世,暮末繼秦王位,改元永弘。暮末在位期間濫刑好殺,於是人心思叛。永弘三年(430年)因受北涼所迫,暮末擬歸附北魏,未料為夏國所阻。永弘四年(431年)夏國攻西秦都城南安,暮末出降,西秦亡。不久,暮末為夏國皇帝赫連定所殺。

\subsubsection{永弘}

\begin{longtable}{|>{\centering\scriptsize}m{2em}|>{\centering\scriptsize}m{1.3em}|>{\centering}m{8.8em}|}
  % \caption{秦王政}\
  \toprule
  \SimHei \normalsize 年数 & \SimHei \scriptsize 公元 & \SimHei 大事件 \tabularnewline
  % \midrule
  \endfirsthead
  \toprule
  \SimHei \normalsize 年数 & \SimHei \scriptsize 公元 & \SimHei 大事件 \tabularnewline
  \midrule
  \endhead
  \midrule
  元年 & 428 & \tabularnewline\hline
  二年 & 429 & \tabularnewline\hline
  三年 & 430 & \tabularnewline\hline
  四年 & 431 & \tabularnewline
  \bottomrule
\end{longtable}


%%% Local Variables:
%%% mode: latex
%%% TeX-engine: xetex
%%% TeX-master: "../../Main"
%%% End:


%%% Local Variables:
%%% mode: latex
%%% TeX-engine: xetex
%%% TeX-master: "../../Main"
%%% End:

%% -*- coding: utf-8 -*-
%% Time-stamp: <Chen Wang: 2019-12-19 15:47:32>


\section{后凉\tiny(389-403)}

\subsection{简介}

後凉(386年-403年)是十六国时期氐人贵族吕光建立的政权。

其国号以地处凉州为名。《十六国春秋》始称“后凉”,以别于其他以“凉”为国号的政权,后世袭用之。

東晋太元八年(383年)前秦将军吕光受命率7万餘众讨平西域。苻坚淝水兵败後前秦瓦解,吕光据有姑臧(今甘肃武威)于太元十一年(386年)称大将军、凉州牧。太元十四年(389年)吕光称三河王,後改称天王,史称後凉。

统治范围包括甘肃西部和宁夏、青海、新疆一部分。

後凉以氐人军事力量为基础,势力孤弱,刑法峻重,社会局势不稳,叛者连城。

後凉龙飞四年(399年)吕光卒,子吕绍继位,庶长子吕纂又杀吕绍自立。後凉咸宁三年(401年)吕隆(吕光弟吕宝之子)又杀吕纂自立,国势益衰。连年战争,经济凋敝,太元十二年(403年),穀价昂贵,人相食。

神鼎三年(403年),吕隆因后秦、南凉、北凉交相攻逼,降于後秦,後凉亡。

%% -*- coding: utf-8 -*-
%% Time-stamp: <Chen Wang: 2021-11-01 12:02:38>

\subsection{懿武帝呂光\tiny(386-399)}

\subsubsection{生平}

涼懿武帝呂光(338年-399年),字世明,略陽(今甘肅天水)氐人,前秦太尉呂婆樓之子。十六國時期後涼建立者。呂光初為前秦將領,屢立戰功,前秦天王苻堅就派了他出兵西域。呂光降服西域,但當時前秦因淝水之戰戰敗而國亂,回軍時為涼州刺史梁熙所阻,呂光消滅了梁熙而入主涼州,遂在當地建立政權。

呂光得王猛看重,並將他推薦給苻堅,苻堅於是以呂光為美陽令,任內呂光得當地人民愛戴信服。呂光後遷鷹揚將軍,以功封關內侯,並於永興二年(358年)隨苻堅等討伐張平。苻堅與張平於銅壁決戰,張平驍勇大力的養子張蚝單騎屢次進出前秦軍陣中,呂光於是去襲擊張蚝並成功擊傷他。張蚝受傷被擒,張平潰敗,呂光亦因而聲名大噪。

建元四年(368年),呂光與王鑒等因應楊成世討伐上邽叛變的苻雙失敗而率軍再行討伐,王鑒到後打算與苻雙前鋒苟興速戰速決,但呂光慮及對方因剛獲勝而士氣高漲,建議謹慎待敵,讓其糧盡退兵時就是進攻的時機。二十日後苟興退兵,王鑒追擊並擊敗苟興,隨後又大敗苻雙,終攻下上邽,斬殺苻雙。建元六年(370年),呂光隨軍攻滅前燕,獲封都亭侯。後苻重出鎮洛陽,呂光擔任其長史。苻重於建元十四年(378年)謀反,苻堅以呂光忠誠正直,不會與苻重連謀,於是下令呂光收捕苻重,呂光聽命並以檻車押送苻重回長安。後呂光遷太子右率,頗受敬重。次年呂光又以破虜將軍身份率兵擊敗進攻成都的李烏,遷步兵校尉。建元十六年(380年)呂光又奉命與左將軍竇衝共領四萬兵討伐叛亂的苻重,又將其生擒,戰後獲授驍騎將軍。

前秦十八年(382年),呂光受命征討西域,以使持節都督西討諸軍事身份率領姜飛等將領、七萬兵及五千鐵騎出發。呂光越過三百多里長的沙漠到達西域,降服焉耆等西域各國,又擊破唯一拒守的龜茲,威震西域。苻堅知呂光征服西域,即任命其為使持節、散騎常侍、都督玉門以西諸軍事、安西將軍、西域校尉,封順鄉侯,但因前秦於淝水之戰後國內大亂而道路不通,未能傳達。呂光本來想要留在龜茲,但是受到名僧鳩摩羅什勸阻,而且部眾們也想回到中原,遂回師。

太安元年(385年),呂光軍抵宜禾(今新疆安西南),高昌太守楊翰告訴涼州刺史梁熙,稱呂光還軍必定別有所圖,建議關閉天險要道,拒之於外,但梁熙沒有聽從。呂光最初知道楊翰的計劃時曾打算不再前進,但在杜進勸告下還是繼續,楊翰即在呂光到達高昌時向呂光請降。梁熙在呂光到遠玉門時傳檄指責呂光擅自班師,又派其子梁胤等率軍五萬往酒泉阻擊呂光。呂光也傳檄指責梁熙沒有為前秦赴國難的忠誠,還阻攔歸國軍隊,並派了姜飛等為前鋒進攻梁胤。姜飛等在安彌大破梁胤並生擒他,於是周邊外族都紛紛依附呂光,武威太守彭濟更將梁熙抓起來叛歸呂光。呂光殺死梁熙,入主姑臧,自領涼州刺史、護羌校尉。

386年,呂光收到苻堅死訊,改元太安,並自稱使持節、侍中、中外大都督、督隴右河西諸軍事、大將軍、涼州牧、酒泉公。呂光入主涼州時,因尉祐與彭濟共謀抓住梁熙的功勞而寵任他,但呂光卻在尉祐中傷下殺了姚皓、尹景等十多個名士,人心見離。當時國內米價也高漲至一斗五百,饑荒中更發生人吃人事件,死了很多人。呂光與群僚在飲宴中談及為政時用嚴峻刑法的問題,在參軍段業勸言下終下令自省並行寬簡之政。

呂光於太安二年(387年)殺了進逼姑臧的張大豫,但王穆尚據酒泉;西平太守康寧也叛變,阻兵據守,呂光試圖討伐但都不果。及後連呂光部將徐炅及張掖太守彭晃都謀叛,並聯結了王穆及康寧。呂光力排眾議親率三萬兵速攻彭晃,二十日後攻破張掖,殺了彭晃。不久,呂光乘王穆進攻其將索嘏的機會率二萬兵襲破酒泉,王穆率兵東返但部眾在途中就潰散,王穆隻身逃走但為騂馬令郭文所殺。

389年,呂光稱三河王,改元麟嘉。396年六月又改稱天王,國號大涼,改元龍飛。呂光曾先後多次進攻西秦,其中呂光弟呂延於龍飛二年(397年)的進攻中兵敗被殺。呂光聽信讒言,怪罪從軍的尚書沮渠羅仇及三河太守沮渠麴粥,並殺二人。二人歸葬時,因諸部聯姻而共計有萬多人參與葬禮,羅仇之侄沮渠蒙遜遂反,蒙遜堂兄沮渠男成舉兵響應,並推建康太守段業為主,建北涼與後涼對抗,呂光曾派呂纂討伐,但最終無法消滅北涼。

同年,善於天文術數的太常郭黁與僕射王詳認為呂光年老、太子闇弱而呂纂等凶悍,料定呂光死後必會有禍亂,並禍及自己,故圖謀攻奪姑臧東西苑城,推王乞基為主。不過王詳因事泄而被殺,郭黁遂據東苑叛變,當時民間還有很多人支持郭黁。呂光召呂纂回兵討伐郭黁,呂纂遂屢破郭黁,令其於龍飛三年(398年)出走西秦,平定亂事。

龍飛四年(399年),呂光病重,立太子呂紹為天王,自號太上皇帝(太上天王)。呂光又讓呂纂及呂弘分任太尉及司徒,告誡呂紹要倚重二人,放權讓他們處理軍政大事才能保國家安穩;另也對呂纂及呂弘說二人要與天王呂紹同心合力才能保全國家,否則禍亂必會來。呂光於不久去世,享年六十三歲,諡懿武皇帝,廟號太祖。

呂光年輕時已展現其軍事能力,十歲時與其他小童一起玩耍時就創制戰爭陣法,於是同年的人都推其為主,而呂光處事平允,更令眾小童佩服。呂光也不喜歡讀書,只好打獵。

呂光高八尺四寸,雙目重瞳,為人沈著堅毅,凝重且寛大有度量,喜怒不形於色,故王猛賞識他,稱:「此非常人。」

呂光出生於枋頭(今河南浚縣西南),當夜有神光,全家覺得奇怪,遂以光为名。

呂光左肘有一肉印,據說在一次戰爭中肉印隱約顯出「巨霸」兩字。

\subsubsection{太安}

\begin{longtable}{|>{\centering\scriptsize}m{2em}|>{\centering\scriptsize}m{1.3em}|>{\centering}m{8.8em}|}
  % \caption{秦王政}\
  \toprule
  \SimHei \normalsize 年数 & \SimHei \scriptsize 公元 & \SimHei 大事件 \tabularnewline
  % \midrule
  \endfirsthead
  \toprule
  \SimHei \normalsize 年数 & \SimHei \scriptsize 公元 & \SimHei 大事件 \tabularnewline
  \midrule
  \endhead
  \midrule
  元年 & 386 & \tabularnewline\hline
  二年 & 387 & \tabularnewline\hline
  三年 & 388 & \tabularnewline\hline
  四年 & 389 & \tabularnewline
  \bottomrule
\end{longtable}

\subsubsection{麟嘉}

\begin{longtable}{|>{\centering\scriptsize}m{2em}|>{\centering\scriptsize}m{1.3em}|>{\centering}m{8.8em}|}
  % \caption{秦王政}\
  \toprule
  \SimHei \normalsize 年数 & \SimHei \scriptsize 公元 & \SimHei 大事件 \tabularnewline
  % \midrule
  \endfirsthead
  \toprule
  \SimHei \normalsize 年数 & \SimHei \scriptsize 公元 & \SimHei 大事件 \tabularnewline
  \midrule
  \endhead
  \midrule
  元年 & 389 & \tabularnewline\hline
  二年 & 390 & \tabularnewline\hline
  三年 & 391 & \tabularnewline\hline
  四年 & 392 & \tabularnewline\hline
  五年 & 393 & \tabularnewline\hline
  六年 & 394 & \tabularnewline\hline
  七年 & 395 & \tabularnewline\hline
  八年 & 396 & \tabularnewline
  \bottomrule
\end{longtable}

\subsubsection{龙飞}

\begin{longtable}{|>{\centering\scriptsize}m{2em}|>{\centering\scriptsize}m{1.3em}|>{\centering}m{8.8em}|}
  % \caption{秦王政}\
  \toprule
  \SimHei \normalsize 年数 & \SimHei \scriptsize 公元 & \SimHei 大事件 \tabularnewline
  % \midrule
  \endfirsthead
  \toprule
  \SimHei \normalsize 年数 & \SimHei \scriptsize 公元 & \SimHei 大事件 \tabularnewline
  \midrule
  \endhead
  \midrule
  元年 & 396 & \tabularnewline\hline
  二年 & 397 & \tabularnewline\hline
  三年 & 398 & \tabularnewline\hline
  四年 & 399 & \tabularnewline
  \bottomrule
\end{longtable}

%%% Local Variables:
%%% mode: latex
%%% TeX-engine: xetex
%%% TeX-master: "../../Main"
%%% End:

%% -*- coding: utf-8 -*-
%% Time-stamp: <Chen Wang: 2019-12-19 15:50:55>

\subsection{灵帝\tiny(399-401)}

\subsubsection{隐王生平}

涼隱王呂紹(380年代-399年),字永業,略陽(今甘肅天水)氐人。十六國時期後涼國第二任君主,後涼懿武帝呂光嫡子。呂紹登位不久即被呂纂及呂弘兩位兄長發動政變所推翻,呂紹自殺。

呂光出征西域時,呂紹與石氏等人留在前秦。淝水之戰後,前秦因戰敗而國亂,長安亦構亂,呂紹等人於是出奔仇池,直至麟嘉元年(389年)才到後涼,甫稱三河王的呂光遂立吕紹為世子。龍飛元年(396年),呂光立其為太子。

吕绍唯一一次有记载的亲自指挥的军事行动在龍飛四年(399年),当时他与庶兄吕纂攻打北凉天王段业,段业求助于南凉天王秃发乌孤。秃发乌孤的弟弟秃发利鹿孤率援军赶到,吕绍和吕纂只得撤退。

同年年末,呂光病重,立呂紹為天王,以吕纂为太尉,吕弘為司徒,臨終前叮囑呂紹說:「如今三寇(乞伏乾歸、段業和禿髮烏孤)未平,我死之後,呂纂帶領軍隊,呂弘治理朝政,你自己無為而治,把重任交給兩個哥哥」。也对两位长子有所嘱咐:“永业并非治理乱世的人才,只不过因嫡长的规举才让其处元首之位。现在外有强寇,人心不定,你们兄弟和睦则会让国家流传万世;若果自己内斗,则祸乱立即就会来了。”还对吕纂说:“你本性粗豪勇武,很令我担心。开展基业本来就艰难,守成也不容易。好好辅助永业,不要听谗言呀。”不久呂光去世,呂紹秘不發喪,呂纂推門入殿哭喪,竭盡哀思才出來。呂紹害怕被殺害,想讓位給他,但呂纂以呂紹是嫡子身份推辭,呂紹固請也不獲呂纂答允,於是即位。呂光侄子呂超勸呂紹及早除去既有兵權,又有極高威名的呂纂,但呂紹雖也憂心呂纂,但仍以父親遺命及袁尚兄弟相爭之事一再拒絕對付呂纂,令呂超很失望。呂紹在湛露堂面見呂纂時,呂超持刀在側侍候,用眼神請求呂紹收捕呂纂,但呂紹都不肯。

在呂紹到後涼前,呂光曾經想立呂弘為世子,不過因為知道呂紹在仇池而打消念頭。可是呂弘一直記恨在心,不久即派尚書姜紀唆使呂纂和他一起叛變。呂纂順從,於是在一夜率軍攻入宮廷,呂紹試圖出兵抵抗,但兵眾都因為忌憚呂纂威名而潰散。呂紹見此便在紫閣自殺。呂纂即位後諡呂紹為隱王。

\subsubsection{灵帝生平}

涼靈帝呂纂(4世紀?-401年),字永緒,略陽(今甘肅天水)氐人。十六國時期後涼國君主,後涼開國君主呂光庶長子,母親是趙淑媛,隱王呂紹兄。呂纂在呂光死後不久即以政變逼死呂紹登位,但在位一年多就在呂超等人的變亂被殺。

呂纂年少時已熟練弓馬,雖然入了太學,但不愛讀書,只會交結公侯。淝水之戰後前秦國亂,呂纂逃到上邽(今甘肅天水市),至太安元年(386年)才到達後涼都城姑臧(今甘肅武威市),拜虎賁中郎將。麟嘉四年(392年),呂光派了呂纂進攻南羌彭奚念,但在盤夷大敗而還。呂光遂親率大軍再攻,讓呂纂及楊軌、沮渠羅仇進軍左南(今青海西寧市東),逼得彭奚念憑湟河自守,然呂光還是派兵渡過湟河,攻下枹罕(今甘肅臨夏市),令彭奚念敗走甘松(今甘肅叠部縣東南)。

龍飛元年(396年),呂光稱天王,以呂纂為太原公。次年,呂光攻西秦,派呂纂、楊軌及竇苟等率三萬兵攻金城(今甘肅蘭州市),攻陷了金城。同年,呂光殺沮渠羅仇及沮渠麴粥,令得羅仇侄沮渠蒙遜反叛。蒙遜堂兄沮渠男成也推了建康太守段業為主,呂纂奉命討伐段業,然而因為沮渠蒙遜率眾到臨洮為聲援段業,呂纂在合離大敗給段業。同時,太常郭黁在姑臧作亂,呂光立即召回呂纂,當時諸將顧慮段業會乘大軍撤退而從後跟隨,建議乘夜暗中撤走,不過呂纂看准段業無謀略,乘夜退走只會助長敵人,於是在退兵時前派了使者向段業說:「郭黁作亂,吾今還都。卿能決者,可出戰。」段業果然不敢追擊。郭黁派軍於白石邀擊呂纂,呂纂大敗,但不久因西安太守石元良率兵援救才得以擊敗郭黁,攻入姑臧。呂纂隨後在城西擊破郭黁將王斐,令郭黁勢力開始衰敗。不過郭黁卻推了楊軌為盟主,讓楊軌前赴姑臧支援自己。時呂弘為段業所逼,呂纂就前去迎接呂弘,楊軌認為這是機會,於是率兵邀擊,但卻為呂纂所敗,郭黁於是出奔西秦,楊軌隨後亦奔廉川,亂事終告平定。

龍飛四年(399年),吕纂与吕绍一同统兵攻打北凉天王段业,段业求救于南凉天王秃发乌孤,秃发乌孤之弟秃发利鹿孤率援军赶到,段业坚守不战,吕纂、吕绍于是退兵。

同年,呂光病重,立呂紹為天王,以呂纂為太尉,掌握軍權。呂光死前曾向呂纂及呂弘說:「永業並非治理亂世的人才,只不過因嫡長的規舉才讓其處元首之位。現在外有強寇,人心不定,你們兄弟和睦則會讓國家流傳萬世;若果自己內鬥,則禍亂立即就會來了。」另也特別對呂纂說:「你本性粗豪勇武,很令我擔心。開展基業本來就艱難,守成也不容易。好好輔助永業,不要聽讒言呀。」不久呂光去世,呂紹懼怕呂纂,曾經想要讓位給呂纂,然而呂纂以嫡庶之別拒絕;另呂光侄呂超又勸呂紹殺了呂纂,但呂紹不肯。可是不久吕纂就在呂弘的煽動下反叛,夜裏率壯士數百進攻廣夏門,守融明觀的齊從抽劍攻擊呂纂,擊中其額,但為呂纂部眾制服。呂紹所派部隊因懼怕呂纂而潰散,吕紹被逼自殺。呂纂遂即天王位,改年號咸寧。

咸寧二年(400年),呂弘舉兵反叛,但為呂纂將焦辨擊敗,出奔廣武(今甘肅永登縣),不久為呂方所捕,被殺。呂纂隨後縱兵大掠,以原屬呂弘的東苑中之婦女賞給軍士,呂弘的妻兒都被士兵侵辱。呂纂笑著對群臣說:「今日一戰怎樣呀?」侍中房晷卻答:「天要降禍給涼室,故藩王起兵釁。先帝駕崩不久,隱王幽逼而死,山陵才剛建好,大司馬就因驚懼疑惑而反叛肆逆,京邑成了兄弟交戰的戰場。雖然呂弘自取滅亡,亦是因為陛下沒有棠棣所說的兄弟之義。現在應該反思自省,以為向百姓謝過,卻反而縱容士兵大肆掠奪,侮辱士女。兵釁因呂弘而起,百姓有甚麼錯!而且呂弘的妻子是陛下的弟婦,女兒也是陛下的姪女,怎能讓她們成為無賴小人的婢妾。天地神明怎會忍心見到這樣!」呂纂聽後向房晷道歉,又接回呂弘的妻兒到東宮。

隨後,呂纂不顧中書令楊穎反對堅決攻伐南涼,卻為南涼將禿髮傉檀所敗。呂纂不久又不聽姜紀諫言而攻北涼,圍攻張掖(今甘肅張掖)並攻略建康郡地,然而禿髮傉檀果如姜紀所言進攻姑臧,呂纂亦被逼退兵。呂纂在位時沉溺於酒色,又常常出獵,諸大臣皆曾勸阻,然而呂纂皆不能聽從。

咸寧三年(401年)呂纂因番禾太守呂超擅攻鮮卑思盤一事召呂超及思盤入朝,呂超因恐懼而事先結交了殿中監杜尚。呂纂憤怒地斥責呂超,更聲言「要當斬卿,然後天下可定」,嚇得呂超叩頭稱不敢。不過呂纂及後就和呂超及眾大臣宴會,呂超兄呂隆於是頻頻向呂纂勸酒要灌醉他。呂纂飲至昏醉便乘坐步輓車與呂超等人在宮內遊走,在到琨華殿東閤時步輓車過不了去,呂纂親將竇川及駱騰於是放下配劍推車。呂超乘此機會拿起二人配劍襲擊呂纂,呂纂試圖下車抓住呂超但被對方刺穿胸部;呂超又殺了竇川和駱騰。呂纂后楊氏下令禁軍討伐呂超,但杜尚卻命禁軍放下武器。將軍魏益多遂斬下呂纂的頭,聲言:「呂纂違反先帝遺命,殺害太子、沉溺飲酒和田獵、親近小人、輕易殺害忠良、視百姓為草芥。番禾太守呂超以骨肉之親,恐懼國家傾覆,已經除去他了。上可以安寧宗廟,下可為太子報仇。但凡國人都應歡慶。」

呂隆不久繼位,諡呂纂為靈皇帝,葬白石陵。

即序胡安據曾盜張駿的墓,獲得大量珍寶,呂纂誅殺安據和其親黨五十多家人,派使者弔祭張駿,並復修其陵墓。

咸寧二年,有母豬生下小豬,一身三頭,又有飛龍夜裡從東廂的井中出現,名僧鳩摩羅什以為不祥,勸纂廣施仁德。一日羅什與呂纂玩博戲,呂纂吃多子,玩笑道:“砍胡奴頭!”羅什糾正說:“不斫胡奴頭,胡奴斫人頭。”預言了呂纂因小字「胡奴」的呂超而被殺的命運。

\subsubsection{咸宁}

\begin{longtable}{|>{\centering\scriptsize}m{2em}|>{\centering\scriptsize}m{1.3em}|>{\centering}m{8.8em}|}
  % \caption{秦王政}\
  \toprule
  \SimHei \normalsize 年数 & \SimHei \scriptsize 公元 & \SimHei 大事件 \tabularnewline
  % \midrule
  \endfirsthead
  \toprule
  \SimHei \normalsize 年数 & \SimHei \scriptsize 公元 & \SimHei 大事件 \tabularnewline
  \midrule
  \endhead
  \midrule
  元年 & 399 & \tabularnewline\hline
  二年 & 400 & \tabularnewline\hline
  三年 & 401 & \tabularnewline
  \bottomrule
\end{longtable}


%%% Local Variables:
%%% mode: latex
%%% TeX-engine: xetex
%%% TeX-master: "../../Main"
%%% End:

%% -*- coding: utf-8 -*-
%% Time-stamp: <Chen Wang: 2019-12-19 15:51:37>

\subsection{吕隆\tiny(401-403)}

\subsubsection{生平}

呂隆(4世紀?-416年),字永基,略陽(今甘肅天水)氐人。十六國時期後涼最後一位君主,後涼開國君主呂光之弟呂寶子。呂隆即位不久即遭後秦攻擊,被逼向後秦請降,其在位時間亦不斷遭南涼及北涼二國攻擊,國力大衰,最終呂隆向後秦請求迎其東遷,後涼遂為後秦所併。

呂隆長得俊美,擅長騎射。呂光時曾任北部護軍。咸寧三年(401年),呂隆弟呂超以兵變弒殺天王呂纂,隨後就擁立呂隆。呂隆面有難色,但呂超說:「現在就好像騎著龍飛在天上,豈可以中途下來!」呂隆於是登位,改元神鼎。

呂隆登位後多殺豪望以圖立威,反不得人心,令人人自危。魏安人焦朗遂招請後秦將領姚碩德攻涼,姚碩德聽從並率軍進攻,兵臨姑臧。呂隆派了呂超及呂邈抵抗但大敗而還,呂邈更戰死,呂隆只得嬰城固守。不過,後秦軍接著數月的圍困令城中原來自東面的人圖謀叛變,將軍魏益多更煽動人們殺呂隆及呂超,呂隆遂在事件被揭發後誅殺共三百多家人。當時後涼群臣勸呂隆和後秦請和,呂隆原本不肯,但在呂超勸諫下向後秦請降。姚碩德於是表呂隆為鎮西大將軍、涼州刺史、建康公。

神鼎二年(402年),北涼沮渠蒙遜率兵進攻姑臧,呂隆請得南涼將禿髮傉檀援救,但傉檀未到呂隆就擊敗蒙遜。蒙遜於是與呂隆結盟,並留下萬多斛穀。但其時姑臧穀價已經高達五千文一斗,發生人吃人事件,死了十多萬人。百姓因為姑臧整天關上城門而無法出城找食物,於是每日都有數百人請求出城當別人奴婢以求生,呂隆怕他們會動搖人心,遂將這些人都盡數殺害,屍體堆滿路上。然而,接著後涼仍不斷受到北涼及南涼攻擊,呂隆被逼於神鼎三年(403年)借後秦徵呂超入侍的機會命其帶著珍寶,請後秦派兵迎其離開。秦將齊難等於該年八月到達姑臧,呂隆率眾隨之東遷長安,呂隆獲後秦授散騎常侍,後涼至此滅亡。後秦弘始十八年(416年),受後秦皇帝姚興子廣平公姚弼謀反案牽連,被殺。

\subsubsection{神鼎}

\begin{longtable}{|>{\centering\scriptsize}m{2em}|>{\centering\scriptsize}m{1.3em}|>{\centering}m{8.8em}|}
  % \caption{秦王政}\
  \toprule
  \SimHei \normalsize 年数 & \SimHei \scriptsize 公元 & \SimHei 大事件 \tabularnewline
  % \midrule
  \endfirsthead
  \toprule
  \SimHei \normalsize 年数 & \SimHei \scriptsize 公元 & \SimHei 大事件 \tabularnewline
  \midrule
  \endhead
  \midrule
  元年 & 401 & \tabularnewline\hline
  二年 & 402 & \tabularnewline\hline
  三年 & 403 & \tabularnewline
  \bottomrule
\end{longtable}


%%% Local Variables:
%%% mode: latex
%%% TeX-engine: xetex
%%% TeX-master: "../../Main"
%%% End:


%%% Local Variables:
%%% mode: latex
%%% TeX-engine: xetex
%%% TeX-master: "../../Main"
%%% End:

%% -*- coding: utf-8 -*-
%% Time-stamp: <Chen Wang: 2019-12-19 15:57:26>


\section{南凉\tiny(397-414)}

\subsection{简介}

南凉(397年-414年)是十六国时期河西鲜卑贵族秃发乌孤建立的政权。

历史
河西鲜卑秃发氏是塞北拓跋氏鲜卑之一支,汉魏时徙至河西,聚族而居。至十六国时,秃发乌孤继位,务农桑,修邻好,境内安定。

東晋隆安元年(397年)乌孤据廉川堡(今青海西宁)称西平王,后改称武威王,徙都乐都(今属青海),建立南凉政权。南凉太初三年(399年)秃发乌孤卒,弟利鹿孤继位,徙都西平(今青海西宁),后改称河西王。

建和三年(402年)利鹿孤卒,弟傉檀继位,回徙乐都,改称凉王。一度降附后秦,镇姑臧(今甘肃武威),后势力既强,又与后秦决裂。连年用兵,先败于北凉沮渠蒙逊,后又败于夏国赫连勃勃,名臣勇将损失十之六七,只好又迁回乐都。

南凉嘉平七年(414年)傉檀率兵袭青海乙弗部,西秦乞伏炽磐乘虚袭取乐都。傉檀降西秦,南凉亡。南凉共存在18年(397-414)。统治地区包括甘肃西部和青海一部分。

其国号源于所处为凉州故名。又其所处为凉州南部,也为区别其他国号为“凉”的政权,故史称“南凉”。又以其王室姓拓跋,又称拓跋凉。

%% -*- coding: utf-8 -*-
%% Time-stamp: <Chen Wang: 2021-11-01 14:52:45>

\subsection{武威武王禿髮烏孤\tiny(397-399)}

\subsubsection{生平}

武威武王\xpinyin*{禿髮烏孤}(4世紀-399年),河西鮮卑人,十六国时期南涼政權建立者。禿髮鮮卑首領,稱武威王,其父禿髮思復鞬亦為禿髮鮮卑族首領。

禿髮思復鞬死後,禿髮烏孤接任禿髮鮮卑首領,他勇猛威武,且有大志,並圖謀奪取時由後涼控制的涼州。禿髮烏孤於是致力發展農業,與鄰邦修好,禮待賢士,以積聚力量。後涼麟嘉六年(394年),涼王吕光封為冠军大将军、河西鲜卑大都统、广武县侯,其部屬石真若留認為當時禿髮烏孤的根基未穩,尚未是後涼的對手,建議禿髮烏孤暫時接受,等待機會。禿髮烏孤因而接受。

次年(395年)破乙弗、折掘二部,並自建廉川堡(今青海民和縣西北)作都城,此後再受後涼封為广武郡公。後涼龍飛元年(396年)呂光稱天王,又遣使署征南大将军、益州牧、左贤王,禿髮乌孤指呂光諸子貪淫,甥子暴虐,以不違百姓之心及不受不義爵位為由拒絕不受。及於次年正式叛後涼自立,自称大都督、大将军、大单于、西平王,改年号太初,並攻克後涼控制的金城(今甘肅蘭州市西北),後更於街亭(今甘肅秦安縣东北)擊敗前來討伐的後涼軍。次年(398年),後秦樂都、湟河、澆河三郡、嶺南羌胡數萬落及後涼將領楊軌、王乞基皆向禿髮烏孤歸降。禿髮烏孤於同年改稱武威王。太初三年(399年),烏孤遷都樂都(今青海樂都)。當時禿髮烏孤選任官員包括了胡人豪族、當地有德望之士、文武才俊、中原遷來的有才之士以及秦雍世族子弟,皆以其才授官。同年禿髮烏孤因酒後坠马伤及肋骨,傷重而死,死前向臣下表示應當立年長新君,故由其弟禿髮利鹿孤繼位。諡號為武王,廟號烈祖。

\subsubsection{太初}

\begin{longtable}{|>{\centering\scriptsize}m{2em}|>{\centering\scriptsize}m{1.3em}|>{\centering}m{8.8em}|}
  % \caption{秦王政}\
  \toprule
  \SimHei \normalsize 年数 & \SimHei \scriptsize 公元 & \SimHei 大事件 \tabularnewline
  % \midrule
  \endfirsthead
  \toprule
  \SimHei \normalsize 年数 & \SimHei \scriptsize 公元 & \SimHei 大事件 \tabularnewline
  \midrule
  \endhead
  \midrule
  元年 & 397 & \tabularnewline\hline
  二年 & 398 & \tabularnewline\hline
  三年 & 399 & \tabularnewline
  \bottomrule
\end{longtable}


%%% Local Variables:
%%% mode: latex
%%% TeX-engine: xetex
%%% TeX-master: "../../Main"
%%% End:

%% -*- coding: utf-8 -*-
%% Time-stamp: <Chen Wang: 2019-12-19 16:19:43>

\subsection{河西康王\tiny(399-402)}

\subsubsection{生平}

河西康王禿髮利鹿孤(4世紀?-402年),河西鮮卑人,十六國時期南涼國君主(河西王)。禿髮烏孤之弟。

太初三年(399年),禿髮烏孤遷都樂都(今青海樂都),並署利鹿孤為驃騎大將軍、西平公,駐鎮安夷(今青海平安區)。同年,後涼呂紹及呂纂進攻北涼,禿髮烏孤應北涼王段業求援,命利鹿孤及楊軌率軍救援。呂紹等人最終撤退,利鹿孤就以涼州牧改鎮西平(今青海湟中)。同年禿髮烏孤去世,利鹿孤繼位,就將都城遷至西平。

建和二年(401年)以祥瑞為由打算稱帝,但在安國將軍鍮勿崙的勸喻下改稱河西王。同年率軍攻伐後涼,大敗涼軍,俘獲楊桓及強遷其二千戶人口。建和三年(402年),利鹿孤又派兵攻破魏安,俘獲佔據當地的焦朗。同年利鹿孤去世,諡康王,葬於西平東南。因著利鹿孤父禿髮思復鞬向來疼愛並重視弟禿髮傉檀,而利鹿孤在位期間很多軍國大事都是由禿髮傉檀處理,故就以禿髮傉檀繼位。

\subsubsection{建和}

\begin{longtable}{|>{\centering\scriptsize}m{2em}|>{\centering\scriptsize}m{1.3em}|>{\centering}m{8.8em}|}
  % \caption{秦王政}\
  \toprule
  \SimHei \normalsize 年数 & \SimHei \scriptsize 公元 & \SimHei 大事件 \tabularnewline
  % \midrule
  \endfirsthead
  \toprule
  \SimHei \normalsize 年数 & \SimHei \scriptsize 公元 & \SimHei 大事件 \tabularnewline
  \midrule
  \endhead
  \midrule
  元年 & 400 & \tabularnewline\hline
  二年 & 401 & \tabularnewline\hline
  三年 & 402 & \tabularnewline
  \bottomrule
\end{longtable}


%%% Local Variables:
%%% mode: latex
%%% TeX-engine: xetex
%%% TeX-master: "../../Main"
%%% End:

%% -*- coding: utf-8 -*-
%% Time-stamp: <Chen Wang: 2021-11-01 14:53:04>

\subsection{景王禿髮傉檀\tiny(402-414)}

\subsubsection{生平}

涼景王禿髮傉檀(365年-415年),河西鮮卑人。十六國時期南涼國君主,他也是第一位正式稱「涼王」的君主。禿髮鮮卑首領禿髮思復鞬之子,南涼君主禿髮烏孤、利鹿孤之弟。

禿髮傉檀機警有才略。太初三年(399年),自稱武威王的禿髮烏孤移都樂都(今青海樂都),以傉檀為車騎大將軍、廣武公,鎮守西平(治今青海西寧),不久又改讓禿髮利鹿孤鎮守西平,召還傉檀錄府國事。同年去世,傳位予利鹿孤。

建和元年(400年),後涼王呂纂進攻南涼,利鹿孤以傉檀抵抗,傉檀在三堆(大通河以南,今甘肅永登縣境)擊敗後涼軍隊,殺二千多人。不久,呂纂又攻北涼王段業,傉檀聞訊就率一萬騎兵進襲後涼都城姑臧(今甘肅武威市)。當時呂纂弟呂緯據北城防禦,傉檀就置酒於姑臧南門朱明門,嗚鐘鼓,大宴將士並在東門青陽門展示兵力,終掠奪八千戶回去。呂纂知姑臧受襲,亦得退兵撤還。

建和二年(401年),利鹿孤稱河西王,以傉檀為都督中外諸軍事、涼州牧、錄尚書事。同年,後涼呂超攻擊據有魏安的焦朗。焦朗派了侄兒焦嵩為質向南涼求援,利鹿孤就是派傉檀率軍救援。但傉檀到後,呂超已撤退,焦朗卻閉門拒守。傉檀因而大怒,打算攻城,但為鎮北將軍俱延所諫止,於是改與焦朗連和,接著又到姑臧展示兵力,並在姑臧西的胡阬駐防。傉檀知道呂超必定會來攻,於是準備好火把。呂超隨後果然派了王集領二千精兵進攻傉檀,傉檀待王集闖進傉檀營壘中時命營壘內外將士都舉起燃著的火把,令營中十分光亮,接著就命軍隊進攻王集軍,絡終斬殺王集及殺三百多人。後涼王呂隆聞訊恐懼,於是假意與傉檀通和,並請他到苑內結盟。傉檀於是派了俱延去參加結盟,但遭呂超伏兵襲擊。傉檀因而大怒,進攻後涼昌松太守孟禕所駐的顯美(今甘肅永昌縣東南),呂隆雖派苟安國及石可救援,但二人都因表懼傉檀兵強而撤還。傉檀攻下顯美後生擒孟禕,初怪摃他不早早投降,但孟緯辯解說他只是盡了為後涼呂氏守衞疆土的職責,令傉檀改以禮待。接著傉檀遷二千多戶回國,想以孟禕為左司馬,又因孟禕表示想為國盡忠到最後,不欲失守城池反獲對方授予顯職而將其送還後涼。建和三年(402年),北涼沮渠蒙遜進攻後涼,後掠因而向利鹿孤求援,利鹿孤就派傉檀領兵一萬救援。傉檀到昌松時知沮渠蒙遜已退兵,就是遷涼澤、段冢五百多戶人回國。不久又受命進攻魏安的焦朗,逼令其出降。

傉檀父親禿髮思復鞬在傉檀年輕時就已喜愛他,更向其諸子說:「傉檀明識榦藝,非汝等輩也。」因此其兄禿髮烏孤以立長君為由命弟禿髮利鹿孤繼位,利鹿孤就在建和三年(402年)病逝前遺命傉檀繼位,兩兄皆傳弟不傳子,最終將君主之位傳傉檀。而其實利鹿孤在位時,軍國大事都交了給傉檀處理。傉檀繼位後,自稱涼王,改元弘昌,並把都城遷回樂都,並在次年正月大肆修築樂都城。後秦王姚興遣使拜傉檀為車騎將軍、廣武公。

傉檀繼位當年十月就率軍進攻後涼,至次年(403年),呂隆因不堪沮渠蒙遜及傉檀的接連進攻,認為再難固守姑臧,決定投歸後秦,向後秦請兵迎接。後秦王姚興於是派了齊難等領兵迎接,並吞併後涼領地,設置守宰。傉檀則攝昌松及魏安二戍作迴避。於傉檀進攻後涼時,其弟禿髮文真曾魏安攻擊後秦派往為後涼協防姑臧的王松怱軍,並俘擄王松怱。傉檀得知後大怒,送王松怱回長安並懇切地向後秦道歉。及至弘昌三年(404年)二月,傉檀更因畏懼後秦強大,自去年號,罷去尚書各官,並派參軍關尚出使後秦。姚興當時曾經以傉檀擅興戰事及大築城池而向關尚表示傉檀無為臣之道;關尚則答禿髮傉檀有羌人及沮渠蒙遜等強敵在附近,這些舉動都是為了守著後秦的門戶,希望姚興不要疑忌。姚興也對這答覆甚為滿意。後傉檀派禿髮文支大破南羌、西虜,接著就上表求姚興讓他領涼州,但被拒絕。後獲加官散騎常侍及增食邑二千戶,更於後秦弘始八年(406年)率兵進攻沮渠蒙遜。沮渠蒙遜當時嬰城固守,傉檀則割了其莊稼,攻至赤泉退兵。接著,傉檀又向後秦進獻三千匹馬及三萬頭羊。姚興至此認為傉檀是忠心的,於是以傉檀為使持節、都督河右諸軍事、車騎大將軍、領護匈奴中郎將、涼州刺史,鎮守姑臧,並召還涼州刺史王尚。傉檀終於得到涼州治權,但其時涼州人申屠英等派了主簿胡威力勸姚興不要召還王尚,放棄河西土地,終令姚興後悔,命車普阻止王尚離開,又派使者告知傉檀。傉檀率其三萬兵到姑臧南的五澗時遇上車普並得知情況,於是立即逼走王尚,還是得以成功入主涼州。原涼州別駕宗敞送王尚回去,傉檀一直都很欣賞他,而臨行前宗敞進薦了多位文武人材,亦得傉檀接納。同年八月,傉檀命禿髮文支留守姑臧,自回都城樂鄉,至十一月正式遷都至姑臧。而傉檀當時雖然是受後秦任命的官員,但車駕、服飾及禮儀都是國王格式。

及後,傉檀進襲西平、湟河各個羌人部落,並遷他們到武興、番禾、武威及昌松四郡。後又於弘始九年(407年)徵集士兵五萬多人,在方亭閱兵後就進攻沮渠蒙遜。沮渠蒙遜率兵迎擊,兩軍在均石(今甘肅張掖市東)交戰,傉檀戰敗。接著傉檀率二萬騎兵運四萬石穀到西郡,但蒙遜就進攻西郡治所日勒(今甘肅山丹縣東南),西郡太守楊統投降。

同年,夏國君主赫連勃勃因向傉檀求結姻親不遂,自率二萬兵進攻傉檀,進軍至支陽(今甘肅會寧縣)時已殺傷一萬多人,並掠二萬七千多人及數十萬頭牲畜回去。傉檀當時親自率兵追擊,焦朗認為赫連勃勃不可輕視,建議經溫圍水北渡黃河,奪萬斛堆(今寧夏中衞縣與甘肅靖遠縣交界),並阻水結營,扼其咽喉;不過將領賀連卻以為赫連勃勃只是烏合之眾,根本不需迴避其軍,應該快點追擊。傉檀聽從賀連所言但在陽武(今甘肅靖遠縣)遭赫連勃勃擊敗,更被追擊了八十多里,死傷數以萬計,損失了南涼六至七成的名臣勇將。傉檀自己就帶著數個騎兵逃至枝陽以南的南山,差點還被追兵抓住。此戰大敗後,傉檀恐懼外離侵逼,於是逼遷方圓三百里以內所有平民到姑臧城內,此舉令人民既驚且怨。故此屠各成七兒就乘著百姓混亂而起兵叛變,一夜之間部眾增至數千人。其時殿中都尉張猛勸說眾人,請其懸崖勒馬,竟成功令眾人散去,成七兒逃亡時間被殺。另一方面,軍諮祭酒梁裒及輔國司馬邊憲等共七人亦謀反,被傉檀誅殺。

弘始十年(408年),姚興見傉檀剛剛大敗給赫連勃勃,又接連發生內亂,想乘機內憂外患的時機消滅他,於是就派了姚弼、斂成及乞伏乾歸領兵三萬進攻傉檀。其時姚興也派了齊難進攻赫連勃勃,姚興因而寫信給傉檀,聲稱姚弼等軍只是用來截擊可能西逃的赫連勃勃。傉檀信以為真,沒有對姚弼軍設防。姚弼於是一直率大軍進攻,俘殺了昌松太守蘇霸並進攻至姑臧,屯兵西苑,傉檀只能嬰城固守。當時涼州人王鍾、宋鍾及王娥等人偷偷去為後秦做內應,但東窗事發,傉檀原本只想殺主事的幾個人,但終也接納伊力延侯的建議,將涉及事件的共五千人全部殺害,並將他們的妻女賞給將士。傉檀又下令郡縣都將牛羊放出城外,引誘了斂成出兵搶掠,傉檀將俱延及敬歸於是趁機進攻,大敗秦軍,殺了七千多人。姚弼此時只得堅守營壘,傉檀主動進攻,但未能攻下。七月,領二萬騎兵作為後援的姚顯還在高平(今甘肅固原),知姚弼進攻失敗,於是加速趕到姑臧。姚顯派了孟欽等五個神射手在涼風門挑戰,但箭還未射就被傉檀的材官將軍宋益擊殺。姚顯見無法取勝,唯有將罪責推給斂成,派使者向傉檀道歉,並在安撫河西人民引兵退還。傉檀亦派使者徐宿到後秦謝罪。可是,同年十一月,傉檀就再度稱涼王,並設年號「嘉平」,又設百官。

及後,傉檀與沮渠蒙遜互相攻伐,至嘉平三年(410年),傉檀又自率五萬騎進攻沮渠蒙遜,但在窮泉大敗,只得隻身騎馬逃歸姑臧;蒙遜更乘勝進攻姑臧。當時姑臧人仍想起兩年前傉檀大殺王鍾等五千人的事,都十分恐懼,於是漢、胡共一萬多戶人都向蒙遜投降。傉檀恐懼之下派了敬歸及敬佗父子作為人質,向蒙遜請和。蒙遜走時雖然敬歸逃回姑臧,但仍強遷八千多戶人。另一方面,右衞將軍折掘奇鎮據石驢山(今青海西寧北川西北)叛變。傉檀害怕沮渠蒙遜進逼,又怕南部領地被折掘奇鎮佔領,於是遷都回樂都,讓成公緖留守姑臧。可是傉檀甫出城,侯諶等人就閉門作亂,推了焦朗為主,向沮渠蒙遜投降。及後沮渠蒙遜於411年攻克姑臧。

沮渠蒙遜乘著取姑臧威勢,於是進攻傉檀,傉檀派將段苟及雲連出兵番禾襲其後方,遷了三千多戶到西平,但蒙遜依然進圍樂都。傉檀堅守三十日仍未失守,蒙遜就是派使者誘傉檀以寵愛的兒子作人質換取自己退兵,但遭傉檀拒絕。蒙遜憤怒之下決定建屋並進行耕作,預備持久圍困樂都。群臣於是請傉檀考慮蒙遜的條件,最終傉檀被逼以兒子禿髮安周為人質,蒙遜亦退兵。不久,傉檀不聽孟愷諫言進攻沮渠蒙遜,五路俱進,掠番禾、苕藋兩地共五千多戶人回國。當時將軍屈右顧慮蒙遜輕兵來襲,建議傉檀加快行軍,早早回到險要能守之地。不過傉檀聽伊力延所言,認為沮渠蒙遜的步兵趕不上傉檀的騎兵,且快速行軍會丟損戰利品,並非良策。可是一夜就遇上迷霧和風雨,沮渠蒙遜大軍趕到,又打得傉檀大敗。蒙遜再次圍攻樂都,傉檀唯有再以兒子禿髮染干為人質求和。

嘉平六年(413年),傉檀再攻蒙遜,在若厚塢兵敗,蒙遜於是又再圍攻樂都,攻了二十日未能攻克就退兵。但時為鎮南將軍、湟河太守的兒子禿髮文支卻向蒙遜投降。不久蒙遜再攻,傉檀只得以太尉俱延為質請和。

嘉平七年(414年),乙弗部落叛變,傉檀堅持進攻乙弗,當時孟愷以當時南涼國內連年糧食失收,而且南有乞伏熾磐,北有沮渠蒙遜這些大敵,都令百姓不安,認為這次遠征即使克捷,但也是後患無窮,建議與乞伏熾磐結盟,請其資給糧食以解厄困,並積聚實力,待合適時機才出兵。但傉檀並不聽信。於是傉檀親領七千騎大破乙弗部,奪得牛馬羊共四十多萬頭。不過,臨行前傉檀曾囑咐留守的太子禿髮虎台要小心的乞伏熾磐果然來攻,虎台試圖據守城池但遭熾磐四面攻擊,十日就已告失陷。

傉檀得知樂都陷落後,對部眾說希望借著從乙弗部奪取的物資攻取契汗部,並贖回眾人被乞伏熾磐俘擄的妻兒,否則投降乞伏熾磐就只成奴僕。接著傉檀就率眾西進,但很多部眾知樂都陷落都逃走了,連傉檀派去追回逃兵的段苟也逃了,於是傉檀部眾幾乎全部潰散。傉檀至此,唯有向乞伏熾磐投降。傉檀到西平時,乞伏熾磐遣使出城迎接,並以上賓之禮接待,又拜其為驃騎大將軍,封左南公,南涼亡。

一年多後,乞伏熾磐毒死傉檀,當時身邊的人都給傉檀找解藥,但傉檀卻說:「我的病哪該醫治呀!」於是中毒去世,享年五十一歲。其死後獲諡為景王。

\subsubsection{弘昌}

\begin{longtable}{|>{\centering\scriptsize}m{2em}|>{\centering\scriptsize}m{1.3em}|>{\centering}m{8.8em}|}
  % \caption{秦王政}\
  \toprule
  \SimHei \normalsize 年数 & \SimHei \scriptsize 公元 & \SimHei 大事件 \tabularnewline
  % \midrule
  \endfirsthead
  \toprule
  \SimHei \normalsize 年数 & \SimHei \scriptsize 公元 & \SimHei 大事件 \tabularnewline
  \midrule
  \endhead
  \midrule
  元年 & 402 & \tabularnewline\hline
  二年 & 403 & \tabularnewline\hline
  三年 & 404 & \tabularnewline
  \bottomrule
\end{longtable}

\subsubsection{嘉平}

\begin{longtable}{|>{\centering\scriptsize}m{2em}|>{\centering\scriptsize}m{1.3em}|>{\centering}m{8.8em}|}
  % \caption{秦王政}\
  \toprule
  \SimHei \normalsize 年数 & \SimHei \scriptsize 公元 & \SimHei 大事件 \tabularnewline
  % \midrule
  \endfirsthead
  \toprule
  \SimHei \normalsize 年数 & \SimHei \scriptsize 公元 & \SimHei 大事件 \tabularnewline
  \midrule
  \endhead
  \midrule
  元年 & 408 & \tabularnewline\hline
  二年 & 409 & \tabularnewline\hline
  三年 & 410 & \tabularnewline\hline
  四年 & 411 & \tabularnewline\hline
  五年 & 412 & \tabularnewline\hline
  六年 & 413 & \tabularnewline\hline
  七年 & 414 & \tabularnewline
  \bottomrule
\end{longtable}


%%% Local Variables:
%%% mode: latex
%%% TeX-engine: xetex
%%% TeX-master: "../../Main"
%%% End:



%%% Local Variables:
%%% mode: latex
%%% TeX-engine: xetex
%%% TeX-master: "../../Main"
%%% End:

%% -*- coding: utf-8 -*-
%% Time-stamp: <Chen Wang: 2019-12-19 16:23:37>


\section{南燕\tiny(398-405)}

\subsection{简介}

南燕(398年-410年)是中國南北朝時五胡十六国中,由鮮卑慕容部的慕容德所建立的國家,是慕容氏諸燕之一。国号燕。

慕容德原是後燕宗室范陽王。397年,當後燕君主慕容宝於參合陂之戰為北魏所敗之後,后燕被截成南北两部分。次年慕容德於滑台(今河南滑县)自稱燕王,拒绝接纳逃难的慕容宝,甚至险些将其杀害。公元400年迁广固(今山东青州西北)稱帝。南燕的国土,东到大海,南达泗上,西至巨野泽,北临黄河,共有十五个郡、八十二个县,约三十三万户,基本上就是原西晋的青州。统治范围包括今山东、河南、江苏各一部分。慕容德将南燕国土一分为五:青州,治所设在东莱(今山东莱州);幽州,治所设在发干(今山东沂水县西北);徐州,治所设在东莞(今山东莒县);兖州,治所设在梁父(今山东泰安南);并州,治所设在阴平(今江苏沭阳)。所以南燕官方在提到本国疆域时,常自称“五州之地”。

后主慕容超在位時,被東晋的劉裕擊敗,經歷兩代後滅國。

“南燕”之别称,始于当时人张诠所写《南燕书》(已佚),因相对于北燕位于南方故名。

%% -*- coding: utf-8 -*-
%% Time-stamp: <Chen Wang: 2021-11-01 14:53:51>

\subsection{献武帝慕容德\tiny(398-405)}

\subsubsection{生平}

燕献武帝慕容德(336年-405年11月17日),後改名慕容備德,字玄明,十六國時期南燕皇帝,鮮卑人,是前燕文明帝慕容皝之幼子,前燕景昭帝慕容儁、後燕成武帝慕容垂皆為其兄。《晉書》載其「年未弱冠,身長八尺二寸,姿貌雄偉」。又「博觀群書,性清慎,多才藝」。

前燕時期慕容儁在位時,慕容德被封為梁公。後來慕容儁之子慕容暐繼帝位,再被改封為范陽王。369年,曾與慕容垂一同大敗東晉桓溫的進攻。370年,前燕為前秦所滅後,一度被前秦帝苻堅任命為張掖(今中國甘肅省張掖市)太守,數年後被免職。

後來苻堅欲南征東晉,慕容德被任命為奮威將軍隨軍,留下金刀拜別母親公孫氏及胞兄原北海王慕容納而去。383年,前秦於淝水之战敗北,苻坚宠妃张夫人走失投靠慕容暐,慕容暐送她回京,慕容德阻止并劝他趁乱复国,未果。

后来慕容垂趁機起兵建後燕,慕容德嚮應之,被命為車騎大將軍,重新受封為范陽王,然其諸子及慕容纳皆因留在後方而被前秦所殺。

396年,慕容垂臨終,遺命太子慕容寶將鄴城(今中國河南省臨漳縣)委由慕容德鎮守。慕容寶繼位後,即以慕容德為使持節、都督冀、兗、青、徐、荊、豫六州諸軍事、特進、車騎大將軍、冀州牧,領南蠻校尉,鎮守鄴城。

397年,北魏攻擊後燕,後燕兵敗如山倒,皇帝慕容寶向北方故地逃亡,後燕國土被截為南北二部,位在南方的慕容德被慕容寶任命為丞相,領冀州牧。不久,慕容垂另一子趙王慕容麟來逃至鄴城,以鄴城難守,建議慕容德南遷滑台(今中國河南省滑縣)。398年正月,慕容德又受慕容麟建議先稱燕王,稱燕王元年,史稱此一政權為南燕。次年(399年),再遷廣固(今中國山東省青州市),以為都城。

400年,慕容德正式稱帝,改元建平,並在此時把自己名字改為慕容備德,以便臣民避諱。慕容德正式登基時年已65歲,在中國歷史上僅次於唐朝的武則天,武則天登基時已經67歲了。

慕容备德不知道母亲公孙夫人和胞兄慕容纳都已经不在人世,曾于建平二年(401年)十月派平原人杜弘去长安寻访。杜弘说:“臣至长安,若不能得知太后动止,当西往张掖,以死效力。”并为自己年逾六十的父亲杜雄乞求本县县令之职。慕容备德不顾中书令张华反对,认为杜弘“为君迎母,为父求禄,忠孝备矣,何罪之有!”以杜雄为平原令。杜弘到张掖为盗贼所杀。四年(403年),慕容备德旧部赵融从长安前来,告知公孙夫人和慕容纳的死讯,慕容备德放声痛哭以至于吐血,因而卧病不起,从此健康恶化。

慕容备德有女兒無兒子,為繼承人心焦,慕容納之子慕容超持當年慕容德拜別母親的金刀來歸,慕容备德遂以慕容超袭封北海王,后立為太子。

建平六年九月戊午(405年11月17日),慕容备德去世,慕容超繼位。去世當晚從四方城門抬出十餘口棺木,秘密埋葬在山谷之中,因此實際上他並未葬於其陵寢「東陽陵」,後人遂不知其安葬之處。慕容德後來被諡為獻武皇帝,廟號世宗。

\subsubsection{燕平}

\begin{longtable}{|>{\centering\scriptsize}m{2em}|>{\centering\scriptsize}m{1.3em}|>{\centering}m{8.8em}|}
  % \caption{秦王政}\
  \toprule
  \SimHei \normalsize 年数 & \SimHei \scriptsize 公元 & \SimHei 大事件 \tabularnewline
  % \midrule
  \endfirsthead
  \toprule
  \SimHei \normalsize 年数 & \SimHei \scriptsize 公元 & \SimHei 大事件 \tabularnewline
  \midrule
  \endhead
  \midrule
  元年 & 398 & \tabularnewline\hline
  二年 & 399 & \tabularnewline
  \bottomrule
\end{longtable}

\subsubsection{建平}

\begin{longtable}{|>{\centering\scriptsize}m{2em}|>{\centering\scriptsize}m{1.3em}|>{\centering}m{8.8em}|}
  % \caption{秦王政}\
  \toprule
  \SimHei \normalsize 年数 & \SimHei \scriptsize 公元 & \SimHei 大事件 \tabularnewline
  % \midrule
  \endfirsthead
  \toprule
  \SimHei \normalsize 年数 & \SimHei \scriptsize 公元 & \SimHei 大事件 \tabularnewline
  \midrule
  \endhead
  \midrule
  元年 & 400 & \tabularnewline\hline
  二年 & 401 & \tabularnewline\hline
  三年 & 402 & \tabularnewline\hline
  四年 & 403 & \tabularnewline\hline
  五年 & 404 & \tabularnewline\hline
  六年 & 405 & \tabularnewline
  \bottomrule
\end{longtable}


%%% Local Variables:
%%% mode: latex
%%% TeX-engine: xetex
%%% TeX-master: "../../Main"
%%% End:

%% -*- coding: utf-8 -*-
%% Time-stamp: <Chen Wang: 2021-11-01 14:54:03>

\subsection{末主慕容超\tiny(405-410)}

\subsubsection{生平}

燕末主慕容超(385年-410年),字祖明,十六國南燕末代皇帝,鮮卑人。

前秦建元六年(370年),前燕為前秦所滅後,慕容超之父慕容納一度仕於前秦,後來遷居於張掖(今中國甘肅省張掖市)。慕容納之弟慕容德受前秦帝苻堅之命隨軍南征東晉,留下金刀拜別母親公孫氏而去。

前秦建元十九年(383年),前秦於肥水之戰敗北,慕容納、德之兄慕容垂趁機起兵建後燕,前秦遂殺慕容納本人及慕容德諸子。公孫氏因年老而免死,慕容納之妻段氏正好懷孕,暫不執行死刑,羈押於獄中。有個叫呼延平的獄卒,是慕容德以前的下屬,慕容德曾免其死罪,對其有恩。因此,他幫助公孫氏及段氏逃至羌地,慕容超即於該處誕生。

慕容超十歲時,祖母公孫氏去世,臨終前把金刀給慕容超,說:「如果天下太平,你能夠向東回到故土,可以將這把刀還給你叔叔(慕容德)。」呼延平後來又讓慕容超母子逃亡到呂光在位時的後涼。後來的後涼王呂隆向後秦姚興投降,慕容超母子又被遷往長安(今中國陝西省西安市)。呼延平去世後,慕容超之母段氏讓慕容超娶呼延平之女。

慕容超認為幾位伯叔父先後在東方稱帝,恐怕被後秦知道身分,所以就裝成神智失常之人,並以行乞維生。後秦人都看不起他,遂對他不起疑,所以行動自由不受限制。當時已登上南燕帝位的慕容德聽說這件事,立即派使者迎接他,慕容超不告別母親、妻子即東行。後來到達南燕,呈獻金刀給慕容德,並告以其祖母也就是慕容德之母臨終的遺言,慕容德聽了之後哀傷不已。將慕容超封為北海王(即慕容纳在前燕的王爵),任命為侍中、驃騎大將軍、司隸校尉,開王府置僚佐。

史載「慕容超身高八尺,腰帶九圍,姿器魁傑」,和慕容德頗為相似,而且「精彩秀髮,容止可觀」,《晉書》和《十六國春秋》皆載此時他才被取名為慕容超。慕容德由於年輕時生的兒子已經在前秦被殺害,晚年只有女兒沒有兒子,所以動了讓慕容超繼承之心。而慕容超亦深知慕容德的意思,因此「入則盡歡承奉,出則傾身下士」,於是輿論一致稱讚,不久即被立為太子。

南燕建平六年九月戊午(405年11月17日),慕容德去世,九月己未(11月18日),慕容超即皇帝位,改元太上。慕容超登位後,寵信舊部公孫五樓,聽信其言,大殺功臣,時稱“欲得侯,事五樓”。又喜好遊獵,使得人民苦不堪言。他的嬸母、皇太后段季妃等密謀廢掉他立慕容鐘,事發,慕容超殺了相關諸臣,廢黜了段季妃。

太上三年(407年),因母段氏、妻呼延氏尚留在後秦,遂向後秦稱藩,後秦就將其母、妻送還。慕容超追尊其父慕容納為穆皇帝,立其母為皇太后,妻為皇后。

南燕向後秦稱藩後,慕容超即計畫南下攻擊淮北,使得東晉不堪其擾。太上五年(409年),東晉將領劉裕率軍進攻南燕反擊。次年二月丁亥日(410年3月25日),南燕都城廣固(今中國山東省青州市)陷落,慕容超被俘,被送往東晉都城建康(今中國江蘇省南京市)斬首。死後無諡號及廟號,有史家稱他為南燕末主。

慕容超同時也是除了系出同源的吐谷渾外,五胡十六國時期源自鲜卑慕容部的最後一位帝王。

\subsubsection{太上}

\begin{longtable}{|>{\centering\scriptsize}m{2em}|>{\centering\scriptsize}m{1.3em}|>{\centering}m{8.8em}|}
  % \caption{秦王政}\
  \toprule
  \SimHei \normalsize 年数 & \SimHei \scriptsize 公元 & \SimHei 大事件 \tabularnewline
  % \midrule
  \endfirsthead
  \toprule
  \SimHei \normalsize 年数 & \SimHei \scriptsize 公元 & \SimHei 大事件 \tabularnewline
  \midrule
  \endhead
  \midrule
  元年 & 405 & \tabularnewline\hline
  二年 & 406 & \tabularnewline\hline
  三年 & 407 & \tabularnewline\hline
  四年 & 408 & \tabularnewline\hline
  五年 & 409 & \tabularnewline\hline
  六年 & 410 & \tabularnewline
  \bottomrule
\end{longtable}


%%% Local Variables:
%%% mode: latex
%%% TeX-engine: xetex
%%% TeX-master: "../../Main"
%%% End:



%%% Local Variables:
%%% mode: latex
%%% TeX-engine: xetex
%%% TeX-master: "../../Main"
%%% End:

%% -*- coding: utf-8 -*-
%% Time-stamp: <Chen Wang: 2019-12-19 16:26:47>


\section{西凉\tiny(400-417)}

\subsection{简介}

西涼(400年—421年)是十六國之一。

400年李暠在敦煌郡称“凉公”。405年遷都酒泉郡,逼近北涼。疆域在今中國甘肅西部及新疆部分。417年,李暠卒,子李歆嗣位。420年,李歆與北涼交戰被殺,其弟敦煌太守李恂在敦煌嗣位。但次年,北涼軍圍敦煌,李恂戰敗,乞降不成後自殺。西涼因此亡於北涼。二十余年之后,李恂的侄子李宝趁北魏攻灭北凉之际,一度恢复先人的基业,同年向北魏投诚,该政权被称为后西凉。

因其统治地区古为凉州,故国号为“凉”,又位于凉州西部,故名“西凉”。

%% -*- coding: utf-8 -*-
%% Time-stamp: <Chen Wang: 2021-11-01 14:54:20>

\subsection{武昭王李暠\tiny(400-417)}

\subsubsection{生平}

涼武昭王李\xpinyin*{暠}(351年-417年),字玄盛,小字長生,陇西郡狄道县(今甘肃省定西市临洮县)人,是李昶的遺腹子,十六國時期西涼的建立者。自稱是西漢將領李廣之十六世孫。李暠的后代形成了陇西李氏的镇远将军房、平凉房、武阳房、姑臧房、敦煌房、仆射房和绛郡房,唐朝皇室和诗人李白亦稱李暠為其先祖。天宝二年(743),唐玄宗追尊李暠为兴圣皇帝。

李暠年少好學,性格寬和,讀遍經史,尤能理解文章的義理。李暠長大後也學習武藝,讀孫吳兵法。北涼神璽元年(397年),後涼建康太守段業自立,次年孟敏降北涼獲授沙州刺史,以李暠為效穀縣令。不久孟敏去世,敦煌护军郭谦及沙洲治中索仙認為李暠在任縣令期間治績頗可取,故推舉李暠为敦煌太守及甯朔將軍,李暠遂向段業請命,獲授安西將軍、敦煌太守,領護西胡校尉。後來北涼右衞將軍索嗣向段業中傷李暠,讓其改以自己擔當敦煌太守。索嗣率五百騎到敦煌外二十里時才通告李暠去迎接自己,李暠聞訊驚訝疑惑,一度想順從出迎,但在張邈及宋繇勸阻下改為派兵抵抗,索嗣兵敗退還張掖。李暠昔日與索嗣十分友好,但知道他在段業面前中傷他並奪去其官位後就相當痛恨他,遂上陳索嗣罪狀。在沮渠男成的勸說下,段業就將索嗣殺了,並派使者向李暠陳謝,又分劃出涼興郡,進李暠為持節,都督涼興以西諸軍事、鎮西將軍,領西夷校尉。

北涼天璽二年(400年),晉昌太守唐瑤移檄六郡,推李暠為大都督、大將軍、護羌校尉、領秦涼二州牧、涼公,改元庚子,以敦煌為都城,建立西涼。李暠又派兵東伐涼興,又西攻玉門西諸城,令疆域廣及西域。次年,北涼將沮渠蒙遜殺段業,自立為北涼君主,李暠又派唐瑤攻酒泉,擒北涼酒泉太守沮渠益生。建初元年(405年),李暠改元並遣使奉表於晉,又遷都酒泉,與北涼長期爭戰。

李暠立國以後鼓勵農事,為對抗北涼積聚軍資,亦令百姓安居樂業。他亦喜好讀書,因此在位時注重文化教育,境內文風頗盛。

建初十三年(417年),李暠過世,享年六十七歲,臨終遺命宋繇輔助諸子。西涼諡武昭王,廟號太祖,次子李歆繼位。

\subsubsection{庚子}

\begin{longtable}{|>{\centering\scriptsize}m{2em}|>{\centering\scriptsize}m{1.3em}|>{\centering}m{8.8em}|}
  % \caption{秦王政}\
  \toprule
  \SimHei \normalsize 年数 & \SimHei \scriptsize 公元 & \SimHei 大事件 \tabularnewline
  % \midrule
  \endfirsthead
  \toprule
  \SimHei \normalsize 年数 & \SimHei \scriptsize 公元 & \SimHei 大事件 \tabularnewline
  \midrule
  \endhead
  \midrule
  元年 & 400 & \tabularnewline\hline
  二年 & 401 & \tabularnewline\hline
  三年 & 402 & \tabularnewline\hline
  四年 & 403 & \tabularnewline\hline
  五年 & 404 & \tabularnewline
  \bottomrule
\end{longtable}

\subsubsection{建初}

\begin{longtable}{|>{\centering\scriptsize}m{2em}|>{\centering\scriptsize}m{1.3em}|>{\centering}m{8.8em}|}
  % \caption{秦王政}\
  \toprule
  \SimHei \normalsize 年数 & \SimHei \scriptsize 公元 & \SimHei 大事件 \tabularnewline
  % \midrule
  \endfirsthead
  \toprule
  \SimHei \normalsize 年数 & \SimHei \scriptsize 公元 & \SimHei 大事件 \tabularnewline
  \midrule
  \endhead
  \midrule
  元年 & 405 & \tabularnewline\hline
  二年 & 406 & \tabularnewline\hline
  三年 & 407 & \tabularnewline\hline
  四年 & 408 & \tabularnewline\hline
  五年 & 409 & \tabularnewline\hline
  六年 & 410 & \tabularnewline\hline
  七年 & 411 & \tabularnewline\hline
  八年 & 412 & \tabularnewline\hline
  九年 & 413 & \tabularnewline\hline
  十年 & 414 & \tabularnewline\hline
  十一年 & 415 & \tabularnewline\hline
  十二年 & 416 & \tabularnewline\hline
  十三年 & 417 & \tabularnewline
  \bottomrule
\end{longtable}


%%% Local Variables:
%%% mode: latex
%%% TeX-engine: xetex
%%% TeX-master: "../../Main"
%%% End:

%% -*- coding: utf-8 -*-
%% Time-stamp: <Chen Wang: 2021-11-01 14:54:43>

\subsection{酒泉公李歆\tiny(417-420)}

\subsubsection{生平}

李\xpinyin*{歆}(?-420年),字士業,小字桐椎,陇西狄道(今甘肃临洮县)人,十六國西涼公,為李暠世子。其第三子李重耳是李唐王朝皇室的直系祖先。

西涼建初十三年(417年),李暠過世,李歆被部下擁護為大都督、大將軍、涼公、涼州牧,改元嘉興。李歆在位時,繼承其父稱臣於東晉的政策,因此東晉封其為酒泉公。

李歆用刑頗嚴,又喜歡建築宮殿,臣屬多有勸諫,然而李歆並不能接納。嘉興四年(420年),北涼佯攻西秦以誘西涼,李歆因此出兵攻擊,戰敗被殺。

\subsubsection{建兴}

\begin{longtable}{|>{\centering\scriptsize}m{2em}|>{\centering\scriptsize}m{1.3em}|>{\centering}m{8.8em}|}
  % \caption{秦王政}\
  \toprule
  \SimHei \normalsize 年数 & \SimHei \scriptsize 公元 & \SimHei 大事件 \tabularnewline
  % \midrule
  \endfirsthead
  \toprule
  \SimHei \normalsize 年数 & \SimHei \scriptsize 公元 & \SimHei 大事件 \tabularnewline
  \midrule
  \endhead
  \midrule
  元年 & 417 & \tabularnewline\hline
  二年 & 418 & \tabularnewline\hline
  三年 & 419 & \tabularnewline\hline
  四年 & 420 & \tabularnewline
  \bottomrule
\end{longtable}


%%% Local Variables:
%%% mode: latex
%%% TeX-engine: xetex
%%% TeX-master: "../../Main"
%%% End:

%% -*- coding: utf-8 -*-
%% Time-stamp: <Chen Wang: 2019-12-19 16:28:45>

\subsection{李恂\tiny(420-421)}

\subsubsection{生平}

李恂(?-421年),字士如,陇西狄道(今甘肃临洮县)人,十六國時期西涼的君主,凉武昭王李暠第五子,李歆之弟,李歆在位時任敦煌太守。

西涼嘉興四年(420年),北涼敗西涼軍殺李歆,隨即攻佔西涼都城酒泉,李恂及其他諸弟逃往北山。數月後,因北涼王沮渠蒙遜所派敦煌太守索元緒凶險好殺,大失人心,而李恂在敦煌施政名聲卓著,敦煌人民遂密迎李恂,李恂率數十騎入敦煌,索元緒逃走,李恂被推為冠軍將軍、涼州刺史,改元永建。不久,沮渠蒙遜派軍討伐。隔年(421年),北涼軍引水灌敦煌,李恂乞降不成,部下投降,李恂遂自殺,西涼亦亡。

\subsubsection{永建}

\begin{longtable}{|>{\centering\scriptsize}m{2em}|>{\centering\scriptsize}m{1.3em}|>{\centering}m{8.8em}|}
  % \caption{秦王政}\
  \toprule
  \SimHei \normalsize 年数 & \SimHei \scriptsize 公元 & \SimHei 大事件 \tabularnewline
  % \midrule
  \endfirsthead
  \toprule
  \SimHei \normalsize 年数 & \SimHei \scriptsize 公元 & \SimHei 大事件 \tabularnewline
  \midrule
  \endhead
  \midrule
  元年 & 420 & \tabularnewline\hline
  二年 & 421 & \tabularnewline
  \bottomrule
\end{longtable}


%%% Local Variables:
%%% mode: latex
%%% TeX-engine: xetex
%%% TeX-master: "../../Main"
%%% End:


%%% Local Variables:
%%% mode: latex
%%% TeX-engine: xetex
%%% TeX-master: "../../Main"
%%% End:

%% -*- coding: utf-8 -*-
%% Time-stamp: <Chen Wang: 2019-12-19 16:30:29>


\section{夏\tiny(407-431)}

\subsection{简介}

夏(407年-431年)又称为大夏或北夏,因为建立者赫连勃勃是匈奴铁弗部人,故又称胡夏或赫连夏,是407年到431年存在于关中与河套地区的一个国家,国都在大部分时间都位于统万城(今陕西省靖边县红墩界乡白城子村无定河北岸)。夏是匈奴铁弗部首领赫连勃勃在407年自称大单于后所建。418年,赫连勃勃在攻陷长安后称帝。427年,北魏太武帝出兵攻陷统万城并于428年俘虏夏国第二代君主赫连昌。赫连昌之弟赫连定随后被拥立为君主并在430年灭亡西秦。但431年赫连定被吐谷浑俘虏,之后在432年被送往北魏处死。夏国共存在25年,历经三代君主。

夏国是五胡十六国中最晚建立的政权,其都城统万城遗址是至今唯一保存基本完好的早期北方王国都城遗址,也是匈奴人历史上留下的唯一都城遗址。

夏国最初占有大城(今内蒙古杭锦旗东南),之后于413年又修建统万城作为首都。至417年后秦灭亡前,夏国占据河套至陇东与陕西北部,

夏国的地方行政主要分为州、城(县)两级。有史可考的有幽、朔、秦、北秦、雍、并、梁、豫、荆九州。文献记载最多的只有幽州。

%% -*- coding: utf-8 -*-
%% Time-stamp: <Chen Wang: 2019-12-19 16:31:30>

\subsection{武烈帝\tiny(407-425)}

\subsubsection{生平}

夏武烈帝赫连勃勃(381年-425年),字屈孑,匈奴铁弗部人,原名劉勃勃,中國十六国时期夏國建立者。勃勃是南匈奴單于的後裔,其父劉衞辰死於北魏進攻後,勃勃依靠後秦高平公沒弈干,又得後秦君主姚興賞識。及後就以後秦與魏通好而叛秦,殺害沒奕干並自立,建北夏國,屢度進攻後秦。隨後更乘東晉滅後秦後班師的機會佔領關中。

赫連勃勃曾祖父劉虎領導鐵弗部,並曾與代國發生戰鬥,為代國所敗。祖父劉務桓重整部眾,重新壯大鐵弗部,並受後趙封為平北將軍、左賢王。父刘卫辰繼位後搖擺於前秦及代國之間,前秦皇帝苻坚滅代國後更任命劉衞辰为西单于,屯駐代來城(今內蒙古伊克昭盟東勝區西),督摄河西诸部族。前秦瓦解後,劉衞辰一度據有朔方一帶,但在391年受到北魏的攻擊,代來城被攻陷,劉衛辰被殺。年幼的劉勃勃逃奔薛干部,薛干酋長把劉勃勃送給後秦高平公沒弈干,沒弈干就把女兒嫁給劉勃勃。勃勃受到後秦姚興的寵遇,任為安北將軍、五原公,鎮朔方。此後一直从属后秦。

後秦弘始九年(407年),赫連勃勃因怨恨後秦與北魏通訊,決意背叛後秦。於是先扣押起柔然可汗送給後秦的八千匹馬,然後假裝在高平川(今寧夏南清水河)狩獵,襲殺沒弈干,併吞其部眾。勃勃自以是匈奴夏后氏後裔,建國號「大夏」,自立为天王,大单于,国号夏,改年號龙升。

勃勃自立後不久就出兵薛干部等三部,收降數萬人後轉攻後秦三城(今陝西綏德縣)以北諸戍。當時諸將都反對出兵後秦,建議勃勃先固守高平,穩固根本,然後才圖長安。但勃勃認為夏國初建,實力仍弱,關中仍因後秦強大而未能攻取,若果自守一城,必會引來後秦各鎮的聯手攻伐,終兵敗亡國,故此特意不長居一處,以游擊戰術,出其不意,讓對方疲於奔命,以取嶺兵、河東之地,再待後秦君主姚興死後才攻取長安。最終勃勃的進襲令到嶺兵各城日間也要緊閉城門。勃勃稱天王後曾向南涼王禿髮傉檀請婚但遭拒,於是怒而率兵進攻南涼,殺傷一萬多人並掠奪二萬七千人及數十萬頭牲畜。後更在陽武(今甘肅靖遠縣境)大敗來攻的禿髮傉檀,殺傷甚眾,很多南涼的名將都戰死。

大破南涼後,勃勃與後秦屢有戰事。勃勃先於青石原(今甘肅涇川縣境)擊敗秦將張佛生,次年(408年)後秦將齊難來攻,勃勃先退守河曲,待齊難縱兵掠奪時就進軍,並追擊至木城(今陝西榆林市榆陽區),生擒齊難,俘獲大量兵眾及戰馬。戰後,歸降夏國的嶺北胡漢數以萬計,勃勃更置地方官員安撫他們。龍升三年(409年),勃勃又率兵攻秦,掠奪平涼雜胡共七千多戶配給後援軍隊,進據依力川。同年姚興親征勃勃,但勃勃乘秦軍未集,率軍進攻姚興所駐的貮城,秦軍兵敗,姚興只得退回長安。接著勃勃又攻下了敕奇堡、黃石固及我羅城。龍升四年(410年),勃勃派兵攻平涼,為姚興所敗,但進攻定陽(今陝西宜川縣西北)的一軍卻取勝,接著勃勃親自率軍進攻隴右,攻破白崖堡,並進逼清水城,令後秦略陽太守姚壽都棄城逃走。龍升五年(411年),勃勃攻安定,在青石以北平原擊破楊佛嵩,又攻下東鄉。

鳳翔元年(413年),勃勃下令修築夏國都城统万城(今陕西靖边北白城子),並任用了殘忍的叱干阿利為將作大匠。工人以蒸土築城,而巡工发现墙面能用铁锥子刺入一寸,便把修築那處的人处死,尸体也被筑入墙内,因此,统万城的城墙坚硬如铁。其時又用銅鑄造了大鼓、飛廉、翁仲、銅駝、龍獸等裝飾物,並用黃金裝飾,排在宮殿前,但製作這些東西又殺了數千個工匠。同時,又以其祖輩跟隨母系姓劉不合禮,於是改姓赫連,表示「徽赫實與天連」;又將非皇族的其他鐵弗部眾改姓「鐵伐」,以示「剛銳如鐵,皆堪伐人。」

此後,勃勃仍然繼續侵襲後秦,先於鳳翔三年(415年)攻下杏城(今陝西黃陵縣西南);次年又乘後秦與仇池楊盛爭戰的時機先後攻下上邽(仱甘肅天水市)及陰密(今陝西靈臺縣西),更令駐守安定的姚恢棄城出走,當地人胡儼等於是獻城夏國。勃勃隨後又進攻雍城(今陝西鳳翔縣南)及郿城(今陝西郿縣),但在郿城遭姚紹抵抗而未能攻下,於是退回安定。胡儼等人此時卻殺勃勃所命留守安定的羊苟兒,轉而降秦。勃勃唯有退返杏城。不過,其時勃勃知東晉劉裕要進攻後秦,他估計劉裕必定能攻滅後秦,但肯定很快班師,留下子弟及將領守關中。勃勃認定這就是他奪取關中的好時機,並且十分輕易,故此不必耗費兵力與後秦作戰。故此秣馬厲兵,休養士卒。不久勃勃再引兵佔據安定,後秦在嶺北的各戍及郡縣都向夏國投降,嶺北全境盡入夏國。另一方面,勃勃又先後與北燕及北涼結盟。

凤翔五年(417年),東晉大将刘裕滅後秦,同年年末班師,留兒子劉義真及王鎮惡、沈田子、傅弘之等諸將守關中。勃勃聞訊十分高興,就派兒子赫連璝督前鋒攻長安、赫連昌出兵堵塞潼關,又派王買德阻斷青泥,然後自率大軍在後。次年,赫連璝行軍至渭陽時已經有很多人在路邊請降,其時晉軍沈田子作戰失利,更因與王鎮惡不和而殺了他,沈田子隨後亦被劉義真長史王脩處死。劉義真於是召集外軍入城並閉門拒守,關中各郡縣於是都降夏。勃勃隨後進據咸陽,令長安城無法獲得物資補給。劉裕見此唯有派朱齡石接替劉義真,並命劉義真東歸。當時劉義真部眾大肆掠奪物資才離開,令關中人民驅逐朱齡石,迎勃勃入主長安。勃勃入長安後大宴將士,不久就在灞上(今陝西蓝田县)称帝,改元昌武。及後群臣都勸勃勃遷都長安,但勃勃慮及全國中心南遷長安後,北魏會易於攻擊距邊界才百里的統萬,認為定都統萬才能阻遏北魏侵襲北境。於是在次年(419年)於長安置南臺,留太子赫連璝留守。不久回師,因統萬宮殿完工而刻石於城南,歌功頌德。

真兴六年(424年),勃勃想要廢黜太子赫連璝,改立幼子赫連倫。赫連璝知道後率兵七萬自長安攻伐赫連倫,終在高平一戰中擊敗並殺死對方。赫連倫兄赫連昌則率軍襲擊赫連璝,將其殺死,勃勃於是立赫連昌為太子。勃勃於真興七年(425年)死于帝位,諡号武烈皇帝,廟號世祖。

勃勃身材魁武,高八尺五寸,且聰慧有儀態,有辯才的機悟。

勃勃頗有權謀智術,當劉裕滅後秦、占關中之時,勃勃一度震攝於劉裕的兵鋒威勢,答應劉裕「約為兄弟」的和平要求,但勃勃為了在氣勢上勝過劉裕,在接見劉裕的使者之前,先讓文才優異的部下皇甫徽寫好給劉裕的回書,再將文字背的爛熟,然後才接見使者,命令部下將自己當場背出的回書寫下並交付給使者。結果使者就以為勃勃真的即席創作出文采斐然的回書,將此事回報給劉裕,果然讓老粗一名的劉裕敬佩勃勃的文思敏捷,邊讀回書邊感嘆說:「吾不如也!」

勃勃凶暴好殺,在陽武大敗南涼軍及關中大敗東晉軍時曾將屍體或人頭堆積起來,建起「髑髏臺」,當作景觀觀賞。也常常在城上,身旁準備好弓箭刀劍,一旦對人有所不滿就會動手殺人。而群臣若敢直接與其對視就會被弄瞎,敢笑就割下其嘴唇,敢進諫就先割下其舌頭再斬殺。這令當時人們十分不安。

勃勃十分自大,建的統萬城四個城門,東門叫招魏門,南門叫朝宋門,西門叫服涼門,北門叫平朔門。

後秦君臣姚興、姚邕兩方的意見:「姚邕說:『勃勃不可近也。』姚興說:『勃勃有濟世之才,吾方與之平天下,柰何逆忌之?』姚邕說:『勃勃奉上慢,御眾殘,貪猾不仁,輕為去就;寵之踰分,恐終為邊患。』」後來勃勃反叛後秦並成為大患,姚興因此感嘆說:「吾不用黃兒之言,以至於此!」(按:姚邕小字黃兒)

南涼大臣焦朗評論:「勃勃天姿雄健,御軍嚴整,未可輕也。」

\subsubsection{龙昇}

\begin{longtable}{|>{\centering\scriptsize}m{2em}|>{\centering\scriptsize}m{1.3em}|>{\centering}m{8.8em}|}
  % \caption{秦王政}\
  \toprule
  \SimHei \normalsize 年数 & \SimHei \scriptsize 公元 & \SimHei 大事件 \tabularnewline
  % \midrule
  \endfirsthead
  \toprule
  \SimHei \normalsize 年数 & \SimHei \scriptsize 公元 & \SimHei 大事件 \tabularnewline
  \midrule
  \endhead
  \midrule
  元年 & 407 & \tabularnewline\hline
  二年 & 408 & \tabularnewline\hline
  三年 & 409 & \tabularnewline\hline
  四年 & 410 & \tabularnewline\hline
  五年 & 411 & \tabularnewline\hline
  六年 & 412 & \tabularnewline\hline
  七年 & 413 & \tabularnewline
  \bottomrule
\end{longtable}

\subsubsection{凤翔}

\begin{longtable}{|>{\centering\scriptsize}m{2em}|>{\centering\scriptsize}m{1.3em}|>{\centering}m{8.8em}|}
  % \caption{秦王政}\
  \toprule
  \SimHei \normalsize 年数 & \SimHei \scriptsize 公元 & \SimHei 大事件 \tabularnewline
  % \midrule
  \endfirsthead
  \toprule
  \SimHei \normalsize 年数 & \SimHei \scriptsize 公元 & \SimHei 大事件 \tabularnewline
  \midrule
  \endhead
  \midrule
  元年 & 413 & \tabularnewline\hline
  二年 & 414 & \tabularnewline\hline
  三年 & 415 & \tabularnewline\hline
  四年 & 416 & \tabularnewline\hline
  五年 & 417 & \tabularnewline\hline
  六年 & 418 & \tabularnewline
  \bottomrule
\end{longtable}

\subsubsection{昌武}

\begin{longtable}{|>{\centering\scriptsize}m{2em}|>{\centering\scriptsize}m{1.3em}|>{\centering}m{8.8em}|}
  % \caption{秦王政}\
  \toprule
  \SimHei \normalsize 年数 & \SimHei \scriptsize 公元 & \SimHei 大事件 \tabularnewline
  % \midrule
  \endfirsthead
  \toprule
  \SimHei \normalsize 年数 & \SimHei \scriptsize 公元 & \SimHei 大事件 \tabularnewline
  \midrule
  \endhead
  \midrule
  元年 & 418 & \tabularnewline\hline
  二年 & 419 & \tabularnewline
  \bottomrule
\end{longtable}

\subsubsection{真兴}

\begin{longtable}{|>{\centering\scriptsize}m{2em}|>{\centering\scriptsize}m{1.3em}|>{\centering}m{8.8em}|}
  % \caption{秦王政}\
  \toprule
  \SimHei \normalsize 年数 & \SimHei \scriptsize 公元 & \SimHei 大事件 \tabularnewline
  % \midrule
  \endfirsthead
  \toprule
  \SimHei \normalsize 年数 & \SimHei \scriptsize 公元 & \SimHei 大事件 \tabularnewline
  \midrule
  \endhead
  \midrule
  元年 & 419 & \tabularnewline\hline
  二年 & 420 & \tabularnewline\hline
  三年 & 421 & \tabularnewline\hline
  四年 & 422 & \tabularnewline\hline
  五年 & 423 & \tabularnewline\hline
  六年 & 424 & \tabularnewline\hline
  七年 & 425 & \tabularnewline
  \bottomrule
\end{longtable}


%%% Local Variables:
%%% mode: latex
%%% TeX-engine: xetex
%%% TeX-master: "../../Main"
%%% End:

%% -*- coding: utf-8 -*-
%% Time-stamp: <Chen Wang: 2021-11-01 14:59:16>

\subsection{赫连昌\tiny(425-428)}

\subsubsection{生平}

赫連昌(?-434年),一名折,字還國,十六國時期夏國皇帝,匈奴鐵弗部人,赫連勃勃三子,赫連勃勃在位時被封太原公。

夏真興六年(424年),赫連勃勃欲廢太子赫連璝,改立酒泉公赫連倫,赫連璝發兵攻殺赫連倫,後赫連昌再襲殺赫連璝平亂,赫連勃勃遂以赫連昌為太子。真興七年(425年)赫連勃勃去世,赫連昌繼位,改元承光。

承光二年(426年),北魏大舉攻夏,克長安。次年(427年),占領夏國都城統萬(今內蒙古烏審旗南白城子),赫連昌逃往上邽(今甘肅天水)。承光四年(428年),北魏攻上邽,會戰中赫連昌因馬失前蹄墜地而被生擒。

赫連昌被俘後,北魏太武帝拓跋燾十分禮遇他,不僅使其住在西宮,更把皇妹嫁給他,並封會稽公。拓跋燾亦常命赫連昌隨待在側,打獵時二人有時亦單獨並騎,赫連昌素有勇名,因此拓跋燾可說對赫連昌十分信任。北魏神䴥三年(430年)三月又被封為秦王。

北魏延和三年(434年)閏三月,赫連昌叛魏西逃,途中被抓獲格殺。

\subsubsection{承光}

\begin{longtable}{|>{\centering\scriptsize}m{2em}|>{\centering\scriptsize}m{1.3em}|>{\centering}m{8.8em}|}
  % \caption{秦王政}\
  \toprule
  \SimHei \normalsize 年数 & \SimHei \scriptsize 公元 & \SimHei 大事件 \tabularnewline
  % \midrule
  \endfirsthead
  \toprule
  \SimHei \normalsize 年数 & \SimHei \scriptsize 公元 & \SimHei 大事件 \tabularnewline
  \midrule
  \endhead
  \midrule
  元年 & 425 & \tabularnewline\hline
  二年 & 426 & \tabularnewline\hline
  三年 & 427 & \tabularnewline\hline
  四年 & 428 & \tabularnewline
  \bottomrule
\end{longtable}


%%% Local Variables:
%%% mode: latex
%%% TeX-engine: xetex
%%% TeX-master: "../../Main"
%%% End:

%% -*- coding: utf-8 -*-
%% Time-stamp: <Chen Wang: 2019-12-19 16:32:20>

\subsection{郝连定\tiny(428-437)}

\subsubsection{生平}

赫連定(?-432年),小字直獖,十六國時期夏國君主,匈奴鐵弗部人,赫連勃勃五子,赫連昌之弟,赫連勃勃在位時被封平原公,鎮守長安。

夏國皇帝赫連昌承光二年(426年),北魏大舉攻夏,赫連定與北魏軍對峙於長安一帶。次年(427年),夏國都城統萬(今內蒙古烏審旗南白城子)陷落,赫連定逃往上邽(今甘肅天水)與赫連昌會合,被進封平原王。承光四年(428年),北魏攻上邽,赫連昌被擒,赫連定逃奔平涼(今甘肅平涼),即皇帝位,改年號勝光。

赫連定繼位時夏國已侷促一隅,情勢窘迫,不復當年,因此欲與正在北伐的南朝宋結盟,北魏得到消息後決定一舉滅夏國。勝光四年(431年),一路敗退的赫連定無路可退,遂向西攻滅為北涼所逼情勢更加窘迫的西秦。數月後欲再攻北涼,於半渡黃河時,被吐谷渾首領慕容慕璝派軍襲擊,赫連定被俘。次年(432年),赫連定被吐谷渾送往北魏,北魏將其處死。

\subsubsection{胜光}

\begin{longtable}{|>{\centering\scriptsize}m{2em}|>{\centering\scriptsize}m{1.3em}|>{\centering}m{8.8em}|}
  % \caption{秦王政}\
  \toprule
  \SimHei \normalsize 年数 & \SimHei \scriptsize 公元 & \SimHei 大事件 \tabularnewline
  % \midrule
  \endfirsthead
  \toprule
  \SimHei \normalsize 年数 & \SimHei \scriptsize 公元 & \SimHei 大事件 \tabularnewline
  \midrule
  \endhead
  \midrule
  元年 & 428 & \tabularnewline\hline
  二年 & 429 & \tabularnewline\hline
  三年 & 430 & \tabularnewline\hline
  四年 & 431 & \tabularnewline
  \bottomrule
\end{longtable}


%%% Local Variables:
%%% mode: latex
%%% TeX-engine: xetex
%%% TeX-master: "../../Main"
%%% End:



%%% Local Variables:
%%% mode: latex
%%% TeX-engine: xetex
%%% TeX-master: "../../Main"
%%% End:

%% -*- coding: utf-8 -*-
%% Time-stamp: <Chen Wang: 2019-12-19 16:34:34>


\section{北燕\tiny(407-436)}

\subsection{简介}

北燕(407年或409年-436年)是十六國时期汉人馮跋建立的政权。407年,馮跋灭后燕,拥立高云(慕容云)为天王,建都龙城(今遼寧省朝陽市),仍旧沿用后燕国号。409年,高云被部下離班、桃仁所杀,馮跋平定政变後即天王位于昌黎(今辽宁省义县)。據有今遼寧省西南部和河北省东北部。436年被北魏所灭。

因其都龙城,又名黄龙,故南朝宋称其为黄龙国。也有史书因其地处东北地区南部,又称其为东燕,但较为罕见。

\input{09_ShiLiuGuo/15_BeiYan/01_HuiYiDi}
%% -*- coding: utf-8 -*-
%% Time-stamp: <Chen Wang: 2019-12-19 16:35:44>

\subsection{文成帝\tiny(409-430)}

\subsubsection{生平}

北燕文成帝冯\xpinyin*{跋}(4世紀?-430年),十六國時期北燕君主,字文起,小名乞直伐,是胡化的漢族人,长乐信都(今河北省衡水市冀州区)人。

冯跋是馮和之孫,其父馮安曾任西燕將軍。西燕亡,馮跋東遷後燕,於後燕帝慕容寶在位時被任命為中卫将军。

馮跋與其弟馮素弗先前曾因事獲罪於後燕帝慕容熙,因此慕容熙有殺馮跋兄弟之意,馮跋兄弟遂逃匿深山。馮跋兄弟商量說:「熙今昏虐,兼忌吾兄弟,既還首無路,不可坐受誅滅。當及時而起,立公侯之業。事若不成,死其晚乎!」於是與從兄萬泥等二十二人合謀。後燕建始元年(407年)馮跋兄弟乘車,由婦人禦,潛入都城和龙(今辽宁朝阳),匿於北部司馬孫護家。趁慕容熙送葬苻后之際起事,推高雲(慕容雲)為燕王,改元正始,不久擒殺慕容熙。高雲登位後以馮跋為侍中、征北大將軍、開府儀同三司,封武邑公,政事皆決於馮跋兄弟。

正始三年(409年),高雲為寵臣離班、桃仁所殺,亂事平定後,眾人推馮跋為主,馮跋遂即天王位,改元太平。馮跋勤於政事,獎勵農桑,輕薄徭役,因此人民喜悅,雖外有強大的北魏相逼,也維持20餘年的安定。

北燕太平二十二年(430年),馮跋病重,命太子馮翼攝理國家大事,未料宋夫人有為其子馮受居圖謀王位之意,馮跋之弟馮弘於是帶兵進宮平變,倉促間馮跋於驚懼中去世。後被諡文成皇帝,廟號太祖。冯弘篡位,将包括冯翼在内的冯跋之子一百余人一并杀死。

\subsubsection{太平}

\begin{longtable}{|>{\centering\scriptsize}m{2em}|>{\centering\scriptsize}m{1.3em}|>{\centering}m{8.8em}|}
  % \caption{秦王政}\
  \toprule
  \SimHei \normalsize 年数 & \SimHei \scriptsize 公元 & \SimHei 大事件 \tabularnewline
  % \midrule
  \endfirsthead
  \toprule
  \SimHei \normalsize 年数 & \SimHei \scriptsize 公元 & \SimHei 大事件 \tabularnewline
  \midrule
  \endhead
  \midrule
  元年 & 409 & \tabularnewline\hline
  二年 & 410 & \tabularnewline\hline
  三年 & 411 & \tabularnewline\hline
  四年 & 412 & \tabularnewline\hline
  五年 & 413 & \tabularnewline\hline
  六年 & 414 & \tabularnewline\hline
  七年 & 415 & \tabularnewline\hline
  八年 & 416 & \tabularnewline\hline
  九年 & 417 & \tabularnewline\hline
  十年 & 418 & \tabularnewline\hline
  十一年 & 419 & \tabularnewline\hline
  十二年 & 420 & \tabularnewline\hline
  十三年 & 421 & \tabularnewline\hline
  十四年 & 422 & \tabularnewline\hline
  十五年 & 423 & \tabularnewline\hline
  十六年 & 424 & \tabularnewline\hline
  十七年 & 425 & \tabularnewline\hline
  十八年 & 426 & \tabularnewline\hline
  十九年 & 427 & \tabularnewline\hline
  二十年 & 428 & \tabularnewline\hline
  二一年 & 429 & \tabularnewline\hline
  二二年 & 430 & \tabularnewline
  \bottomrule
\end{longtable}


%%% Local Variables:
%%% mode: latex
%%% TeX-engine: xetex
%%% TeX-master: "../../Main"
%%% End:

%% -*- coding: utf-8 -*-
%% Time-stamp: <Chen Wang: 2019-12-19 16:36:16>

\subsection{昭成帝\tiny(430-436)}

\subsubsection{生平}

北燕昭成帝馮弘(?-438年),十六國時期北燕國君主,字文通,長樂信都(今河北省衡水市冀州区)人,北燕太祖馮跋之弟。

馮跋在位時,馮弘被封中山公司徒錄尚書事,輔政。

馮跋病重時,宋夫人有為其子馮受居圖謀王位之意,馮弘於是帶兵入宮平變,倉促間馮跋於驚懼中去世,馮弘遂即天王位,並下詔書說:「天降凶禍,大行崩背,太子不侍疾,群公不奔喪,疑有逆謀,社稷將危。吾備介弟之親,遂攝大位以寧國家;百官叩門入者,進陛二等。」盡殺包括太子馮翼在內的馮跋諸子百人。

翌年(431年),改元太興。將自己元配夫人王氏及其所生之子、太子馮崇廢掉。於第二年四月,冊立後燕皇族之女慕容氏為天后,藉以抬高其身價。於是,長樂公馮崇,以及馮崇之同母弟、廣平公馮朗,樂陵公馮邈也懼繼母迫害,禍及自身,於是舉郡向北魏投誠。第三年,春正月,馮弘冊立「後妻慕容氏子馮王仁為世子」。

北燕國小民弱,馮弘在位時,因北魏屢次攻伐,數次向北魏朝貢請和,但仍持續受到攻擊,因此亦曾遣使向南朝宋稱藩納貢。

太興六年(436年),北魏再攻北燕,馮弘於五月乙卯日(6月4日),馮弘帶領子女、後宮、宗族,及龍城之百姓,隨高句麗援軍從都城龍城(今遼寧朝陽)撤退,臨行焚其宮室、城邑,大火一旬不滅,北燕亡。

馮弘在高句麗號令如在本國,引起高句麗長壽王高璉嫌惡,長壽王將其侍衛撤走,又將其太子馮王仁押回興京,扣作人質。復有歸刘宋之意,於是又派使者帶著三百人出使建康,請求宋文帝允許其全家移居建康;宋文帝答應,並派遣將軍王白駒,率兵七千,北上迎接。當時,高句麗也向刘宋稱臣。高句麗王不欲馮弘南下成仇,好言規勸,而馮弘不聽。遂於438年殺馮弘及其妻子,並為其上諡號曰昭成皇帝,一作昭文皇帝。

\subsubsection{太兴}

\begin{longtable}{|>{\centering\scriptsize}m{2em}|>{\centering\scriptsize}m{1.3em}|>{\centering}m{8.8em}|}
  % \caption{秦王政}\
  \toprule
  \SimHei \normalsize 年数 & \SimHei \scriptsize 公元 & \SimHei 大事件 \tabularnewline
  % \midrule
  \endfirsthead
  \toprule
  \SimHei \normalsize 年数 & \SimHei \scriptsize 公元 & \SimHei 大事件 \tabularnewline
  \midrule
  \endhead
  \midrule
  元年 & 431 & \tabularnewline\hline
  二年 & 432 & \tabularnewline\hline
  三年 & 433 & \tabularnewline\hline
  四年 & 434 & \tabularnewline\hline
  五年 & 435 & \tabularnewline\hline
  六年 & 436 & \tabularnewline
  \bottomrule
\end{longtable}


%%% Local Variables:
%%% mode: latex
%%% TeX-engine: xetex
%%% TeX-master: "../../Main"
%%% End:



%%% Local Variables:
%%% mode: latex
%%% TeX-engine: xetex
%%% TeX-master: "../../Main"
%%% End:

%% -*- coding: utf-8 -*-
%% Time-stamp: <Chen Wang: 2019-12-19 16:38:03>


\section{北凉\tiny(397-439)}

\subsection{简介}

北凉(397年或401年-439年)是十六国之一。由匈奴支系盧水胡族的首領沮渠蒙逊所建立;另有一種看法認為建立者為段業,此說是以蒙遜堂兄沮渠男成擁立段業稱涼州牧,並改元神璽為立國之始(397年)。

401年蒙遜誣男成謀反,段業斬男成,蒙遜以此為藉口攻滅段業,仍稱涼州牧,改元永安,因此亦有人以此為北涼立國之時。

北涼首都为张掖,蒙遜自称张掖公。412年迁都姑臧(今甘肃武威),称河西王。最强盛的时候控制今甘肃西部、宁夏、新疆、青海的一部分,是河西一帶最強大的勢力。420年灭西凉。433年蒙逊去世,其子沮渠牧犍继位。439年北魏大军围攻姑臧,沮渠牧犍出降,北涼亡,北魏統一華北。

後牧犍弟沮渠無諱西行至高昌,建立高昌北涼,一般認為已脫離五胡十六國時代之範圍,460年高昌北涼為柔然所攻滅,無諱弟沮渠安周被殺,高昌北涼亦亡。

%% -*- coding: utf-8 -*-
%% Time-stamp: <Chen Wang: 2019-12-19 16:39:00>

\subsection{段业\tiny(397-401)}

\subsubsection{生平}

段業(?-401年),京兆郡(治今陝西西安)漢人。十六国时期北涼国開國君主,但其本身只是為盧水胡沮渠蒙遜及沮渠男成所推,他也很忌憚沮渠蒙遜,蒙遜亦十分不安,最終沮渠蒙遜發動兵變推翻並殺害段業。

段業博覽史傳,有文辭才學,原是前秦將領吕光部將杜進僚屬,從征西域。後呂光建後涼,出任建康太守。後涼龍飛二年(397年),沮渠蒙遜叛後涼,其堂兄沮渠男成亦叛,並進攻段業所守的建康。男成派使者勸段業支持自己,段業最初不肯,但在男成圍困二十日後還是答應了,遂被推為大都督、龍驤大將軍、涼州牧、建康公,改年號神璽。

神璽元年(397年)以沮渠男成為輔國將軍、沮渠男成的堂弟沮渠蒙遜為張掖太守,委以軍國重任。

神璽二年(398年),段業在沮渠蒙遜的支持下,力排眾議命蒙遜進攻西郡,終擒太守呂純,隨後晉昌太守王德及敦煌太守孟敏都向段業投降。不久段業又攻呂弘鎮守的張掖,呂弘率兵棄城東歸,段業不聽沮渠蒙遜歸師勿遏、窮寇勿追的諫言,執意追擊,終大敗而還。神璽三年(399年),稱涼王,改元天璽。同年後涼太子呂紹及呂纂來攻,段業請得禿髮烏孤派楊軌等協助,就打算進攻結陣迎戰的後涼軍。蒙遜卻認為楊軌伺機圖謀北涼,而且後涼軍兵處死地,肯定會奮戰求生,故段業不要出戰,免陷入危機。段業同意,最終按兵不戰,後涼軍也退兵。

段業本來只是一個有德望的儒者,因緣際會被推上王位,其實本人並沒有權謀,無法約束下屬,只信任卜卦、巫術。而一直以來,段業對於沮渠蒙遜的勇略就頗為忌憚,最初就讓沮渠蒙遜由尚書左丞外調到臨池郡任太守,想疏遠他。段業又親近信任門下侍郎馬權,以其代替蒙遜張掖太守之位,但蒙遜怨恨馬權常輕侮自己,於是向段業中傷馬權,段業遂殺馬權。另段業因索嗣認為李暠不能留在敦煌,任由其發展其勢力的建議而派索嗣接替李暠任敦煌太守,然李暠擊敗了索嗣,並上請段業誅殺索嗣,段業在沮渠男成勸告下就將索嗣殺了。此時,蒙遜有除掉段業之意,遂和男成表示既馬權、索嗣二人已死,應當殺死段業,改奉男成為主,但為男成拒絕。蒙遜因段業忌憚自己而愈見不安,遂自請任西安太守,段業亦怕蒙遜很快會反叛,答應了其請求。

天璽三年(401年),沮渠蒙遜誣沮渠男成謀反,段業收捕沮渠男成並命其自殺,男成死前對段業說:「蒙遜早就和臣說過他要叛亂了,只是臣以兄弟緣故才不說出來。蒙遜以臣還在,怕部眾不聽從他,於是約臣與其祭山,反派人誣告臣。臣若果死了,蒙遜肯定很快就起兵了。請假稱臣死了,宣告臣的罪行,蒙遜肯定會起兵叛亂,而臣立即就會討伐他,必會成功。」可是,段業沒有聽信。男成死後,沮渠蒙遜以此為藉口激怒將士,並率領他們攻擊段業,連羌胡都起兵響應。段業見此就讓田昂及梁中庸率兵攻蒙遜,當時將領王豐孫警告稱西平田氏世代都有反叛者,而田昂「貌恭而心狠,志大而情險」,並不可信;但段業自以只能倚仗他對抗蒙遜,還是沒有聽從。最終田昂果然臨陣降於蒙遜,梁中庸亦被逼投降。接著田昂侄田承愛在蒙遜兵臨張掖時讓蒙遜入城,段業左右潰散,段業請求蒙遜饒他一命,讓其東歸與妻兒見面,但蒙遜還是殺了他。沮渠蒙遜隨後獲推為張掖公,繼立為北涼君主。

\subsubsection{神玺}

\begin{longtable}{|>{\centering\scriptsize}m{2em}|>{\centering\scriptsize}m{1.3em}|>{\centering}m{8.8em}|}
  % \caption{秦王政}\
  \toprule
  \SimHei \normalsize 年数 & \SimHei \scriptsize 公元 & \SimHei 大事件 \tabularnewline
  % \midrule
  \endfirsthead
  \toprule
  \SimHei \normalsize 年数 & \SimHei \scriptsize 公元 & \SimHei 大事件 \tabularnewline
  \midrule
  \endhead
  \midrule
  元年 & 397 & \tabularnewline\hline
  二年 & 398 & \tabularnewline\hline
  三年 & 399 & \tabularnewline
  \bottomrule
\end{longtable}


\subsubsection{天玺}

\begin{longtable}{|>{\centering\scriptsize}m{2em}|>{\centering\scriptsize}m{1.3em}|>{\centering}m{8.8em}|}
  % \caption{秦王政}\
  \toprule
  \SimHei \normalsize 年数 & \SimHei \scriptsize 公元 & \SimHei 大事件 \tabularnewline
  % \midrule
  \endfirsthead
  \toprule
  \SimHei \normalsize 年数 & \SimHei \scriptsize 公元 & \SimHei 大事件 \tabularnewline
  \midrule
  \endhead
  \midrule
  元年 & 399 & \tabularnewline\hline
  二年 & 400 & \tabularnewline\hline
  三年 & 401 & \tabularnewline
  \bottomrule
\end{longtable}


%%% Local Variables:
%%% mode: latex
%%% TeX-engine: xetex
%%% TeX-master: "../../Main"
%%% End:

%% -*- coding: utf-8 -*-
%% Time-stamp: <Chen Wang: 2019-12-19 16:41:54>

\subsection{武宣王\tiny(401-433)}

\subsubsection{生平}

沮渠蒙遜(368年-433年),臨松匈奴人,十六国时期北涼第二任君主。原係匈奴支系卢水胡族首領,曾反叛後涼並推段業建北涼,後攻殺段業自己登位。沮渠蒙遜有勇略,在位期間,北涼於強敵環伺之際擴張成為河西一帶最強大的勢力。

沮渠原是匈奴官名,分為左沮渠與右沮渠。沮渠蒙遜出身匈奴貴族,為盧水胡領袖。

沮渠蒙遜博覽史籍,知曉天文,才智出眾又有謀略,為人圓滑又靈活變通,故前秦將領如梁熙及呂光都對其才能既感驚異,亦生畏懼。沮渠蒙遜知道後亦常飲酒出遊,故作低調。前秦亡後,蒙遜一族依附呂光建立的後涼。397年,蒙遜伯父后凉尚书沮渠罗仇和三河太守沮渠麹粥随从后凉进攻西秦的乞伏乾歸,吕光弟吕延轻敌,兵败被杀,后凉军被迫撤退。呂光以败军之罪杀罗仇、麹粥二人,蒙遜在宗族聚集參加二人喪禮的機會舉眾叛涼,斬後涼中田護軍馬邃及臨松令井祥與眾盟誓,十日之間就招合了萬多人,屯兵金山。同年,蒙遜堂兄沮渠男成擁立段業稱涼州牧,建北涼,蒙遜附之,獲授鎮西將軍、張掖太守。

398年,蒙遜深知西郡戰略價值高,遂大力支持段業進攻該郡的決定,並受命進攻。然而蒙遜攻郡城十餘日不下,改為引水灌城,終逮獲太守呂純而返,晉昌郡王德及敦煌郡孟敏戰後皆向北涼歸降。蒙遜以功封為臨池侯。同年後涼張掖守將呂弘率眾棄城東歸,蒙遜以「歸師勿遏,窮寇弗追」為理反對段業追擊,但段業不聽,終為呂弘所敗,段業更因蒙遜才得安全撤退,因而嘆道:「我沒有聽從子房的話,才會有此結果!」後蒙遜又反對段業以將領臧莫孩擔任新建西安城的太守,稱臧莫孩「勇而無謀,知進忘退」,必會失敗。段業又不聽,不久臧莫孩就被後涼呂纂擊敗。天璽元年(399年),段業稱涼王,以蒙遜為尚書左丞。不久,后凉太子呂紹及呂纂來攻,段業請得南涼禿髮烏孤派楊軌等救援,就打算迎擊,蒙遜就說:「楊軌恃著騎兵戰力強,有伺機圖謀我們的意圖。而呂紹和呂纂在死地,肯定會與我們決戰以求生。拒絕對戰將有如泰山般安穩,出戰則像疊起的蛋一樣危險。」段業同意,遂按兵拒絕接戰,後涼軍沒有辦法,亦退兵。

雖然蒙遜屢次建言協助段業,但卻害怕對方容不下自己,所以每每特意不顯露自己的智謀。段業也畏懼蒙遜的能力,故此調蒙遜為臨池太守,改以門下侍郎馬權為張掖太守。馬權得段業信任和重用,其人亦有過人軍事謀略,卻輕視並常欺侮蒙遜,令蒙遜對他又恨又怕,於是向段業進言中傷馬權,卻令段業將馬權殺死。蒙遜隨後向沮渠男成建議除去段業,改奉男成為主,但被男成拒絕。蒙遜心中不安,自求外任西安太守,也得段業批准。

不過蒙遜天玺三年(401年)四月约男成一同去祭告兰门山(甘肃省山丹县西南)時,暗中派人告诉段业说男成准备发动变乱,段業斬男成,男成死前对段业说:“蒙逊早就和臣说过他要叛乱了,只是臣以兄弟缘故才不说出来。蒙逊以臣还在,怕部众不听从他,于是约臣与其祭山,反派人诬告臣。臣若果死了,蒙逊肯定很快就起兵了。请假称臣死了,宣告臣的罪行,蒙逊肯定会起兵叛乱,而臣立即就会讨伐他,必会成功。”段业不听。蒙遜以此為藉口出兵攻段,並进屯侯坞,段业急派右将军田昂、武威将军梁中庸反击蒙逊,田昂、梁中庸至侯坞反降蒙逊,五月,蒙逊大军抵张掖(今甘肃张掖西北),田昂侄子田承爱开城门内应,蒙逊入城,殺段业,遂稱大都督、大将军、凉州牧、张掖公,改年號永安。

後秦亦在永安二年(402年)任命沮渠蒙遜為鎮西將軍、沙州刺史、西海侯。蒙遜登位後提拔人才,得文武官員支持。

蒙遜曾經送子沮渠奚念到南涼做人質,想與其結好,然而南涼主禿髮利鹿孤嫌奚念年幼,要求改以蒙遜弟沮渠挐為質。蒙遜寫信表示不願,竟惹怒利鹿孤並遭進攻,蒙遜唯有答應利鹿孤的要求。永安七年(407年),禿髮傉檀率兵五萬進攻蒙遜,蒙遜於均石擊敗傉檀,並進攻南涼西郡太守楊統。永安十年(410年),蒙遜因之前南涼枯木及胡康攻掠臨松而攻南涼,至顯美強遷數千戶人退兵。傉檀率兵追擊,並在窮泉追上蒙遜,蒙遜大敗傉檀,更乘勝攻至姑臧,萬多戶姑臧人民向蒙遜歸降。蒙遜隨後接受傉檀求和,遷八千多戶人離開。傉檀不久就遷都至樂都,焦朗等人乘勢據姑臧自立,蒙遜遂率三萬兵進攻,奪取了姑臧。412年,蒙遜遷都姑臧,稱河西王,改元玄始。

西涼在沮渠蒙遜殺段業登位前一年自立,蒙遜曾於永安十一年(411年)輕兵襲擊西涼,西涼君主李暠閉門拒戰,蒙遜撤兵時更被西涼世子李歆擊敗。至玄始六年(417年)李歆即位,蒙遜命張掖太守沮渠廣宗詐降西涼,李歆中計出兵迎接但及後卻發現蒙遜所領的三萬伏兵而撤走,蒙遜追擊卻在鮮支澗一戰中大敗予李歆。蒙遜一度想重結敗兵再戰,但為沮渠成都勸止,在增築建康城後班師。玄始九年(420年),李歆乘蒙遜攻西秦浩亹的機會進攻,蒙遜聞訊時正自浩亹回師至川巖,於是發布浩亹已下,即將進攻黃谷的假消息,讓李歆以為蒙遜仍在外,實質正暗中回援。李歆果然繼續進攻,兩軍遂於懷城決戰,李歆兵敗但不肯撤退,堅持再戰,於是在蓼泉再敗並被殺。蒙遜因而乘勢攻陷西涼都城酒泉,滅亡了西涼。次年,蒙遜率軍進攻李恂領導之西涼殘餘勢力所據的敦煌,成功攻陷,徹底滅亡西涼勢力。

朱齡石滅蜀後曾與蒙遜有使者往來,蒙遜亦上表表示其臣服於東晉,晉廷亦授予涼州刺史。玄始十年(421年),把持東晉軍政的劉裕代晉建南朝宋後,於十月任命沮渠蒙遜為使持節、散騎常侍、都督涼州諸軍事、鎮軍大將軍、開府儀同三司、涼州刺史、張掖公。玄始十二年(423年)二月,蒙遜遣使南朝宋,宋廷進蒙遜侍持節、開府、侍中、都督涼秦河沙四州諸軍事、驃騎大將軍、領護匈奴中郎將、西夷校尉、涼州牧,河西王。玄始十五年(426年)五月又獲改授車騎大將軍。承玄四年(431年),蒙遜又曾命人出使北魏,更派兒子沮渠安周入魏,北魏遂命其為假節,侍中,都督涼州西域羌戎諸軍事,太傅,行征西大將軍,涼州牧,涼王。

義和三年(433年),蒙逊去世,享年六十六,諡武宣王,庙號太祖。因他生前所立继承人沮渠菩提年幼,贵族拥立其年长之子沮渠牧犍继位。

沮渠蒙遜有軍事才能,故屢次向段業提供意見助其解兵厄,亦讓其國能立於河西諸國間。登位後,蒙遜伯父中田護軍沮渠親信及臨松太守沮渠孔篤驕橫奢侈,侵害人民,蒙遜說:「禍亂我國家的就是兩位伯父呀,還怎治理百姓呀!」於是命二人自殺。不過他用計陷害堂兄男成,接著攻殺他推舉的段業,令《晉書》評價他「見利忘義,苞禍滅親。」蒙遜知劉裕滅後秦的消息後十分憤怒,門下校郎劉詳其時有事報告,蒙遜卻回應:「你知道劉裕入關,竟敢這樣得意!」就將劉詳殺了,亦見其嚴酷殘暴一面。

據《晉書》所載,蒙遜頗信天象,並寫其多次憑天象指引而勝利。亦有載蒙遜曾祭祀西王母寺,並命中書侍郎張穆為寺內的《玄石神圖》作賦,銘於寺前;蒙遜又曾派世子沮渠興國到南朝宋借《周易》等書,又曾向南朝宋司徒王弘求《搜神記》。沮渠蒙逊曾在母车太后病重时引咎于己,同时大赦死罪以下,车太后仍然去世。当旱灾时,他也有同样举动,次日就下大雨了。

蒙遜亦信佛,其時有一名自西域東來的僧人曇無讖在涼州譯經,又「以男女交接之術教授婦人」,時蒙遜諸女及子媳都信奉他。曇無讖亦通術數和咒術,屢次準確說出其他國家的事,沮渠蒙逊遂奉昙无谶为国师,每以国事谘之。後北魏聽聞曇無讖的事跡,要求蒙遜將曇無讖送到北魏,蒙遜不肯,及後還將他殺了。

\subsubsection{永安}

\begin{longtable}{|>{\centering\scriptsize}m{2em}|>{\centering\scriptsize}m{1.3em}|>{\centering}m{8.8em}|}
  % \caption{秦王政}\
  \toprule
  \SimHei \normalsize 年数 & \SimHei \scriptsize 公元 & \SimHei 大事件 \tabularnewline
  % \midrule
  \endfirsthead
  \toprule
  \SimHei \normalsize 年数 & \SimHei \scriptsize 公元 & \SimHei 大事件 \tabularnewline
  \midrule
  \endhead
  \midrule
  元年 & 401 & \tabularnewline\hline
  二年 & 402 & \tabularnewline\hline
  三年 & 403 & \tabularnewline\hline
  四年 & 404 & \tabularnewline\hline
  五年 & 405 & \tabularnewline\hline
  六年 & 406 & \tabularnewline\hline
  七年 & 407 & \tabularnewline\hline
  八年 & 408 & \tabularnewline\hline
  九年 & 409 & \tabularnewline\hline
  十年 & 410 & \tabularnewline\hline
  十一年 & 411 & \tabularnewline\hline
  十二年 & 412 & \tabularnewline
  \bottomrule
\end{longtable}


\subsubsection{玄始}

\begin{longtable}{|>{\centering\scriptsize}m{2em}|>{\centering\scriptsize}m{1.3em}|>{\centering}m{8.8em}|}
  % \caption{秦王政}\
  \toprule
  \SimHei \normalsize 年数 & \SimHei \scriptsize 公元 & \SimHei 大事件 \tabularnewline
  % \midrule
  \endfirsthead
  \toprule
  \SimHei \normalsize 年数 & \SimHei \scriptsize 公元 & \SimHei 大事件 \tabularnewline
  \midrule
  \endhead
  \midrule
  元年 & 412 & \tabularnewline\hline
  二年 & 413 & \tabularnewline\hline
  三年 & 414 & \tabularnewline\hline
  四年 & 415 & \tabularnewline\hline
  五年 & 416 & \tabularnewline\hline
  六年 & 417 & \tabularnewline\hline
  七年 & 418 & \tabularnewline\hline
  八年 & 419 & \tabularnewline\hline
  九年 & 420 & \tabularnewline\hline
  十年 & 421 & \tabularnewline\hline
  十一年 & 422 & \tabularnewline\hline
  十二年 & 423 & \tabularnewline\hline
  十三年 & 424 & \tabularnewline\hline
  十四年 & 425 & \tabularnewline\hline
  十五年 & 426 & \tabularnewline\hline
  十六年 & 427 & \tabularnewline\hline
  十七年 & 428 & \tabularnewline
  \bottomrule
\end{longtable}

\subsubsection{承玄}

\begin{longtable}{|>{\centering\scriptsize}m{2em}|>{\centering\scriptsize}m{1.3em}|>{\centering}m{8.8em}|}
  % \caption{秦王政}\
  \toprule
  \SimHei \normalsize 年数 & \SimHei \scriptsize 公元 & \SimHei 大事件 \tabularnewline
  % \midrule
  \endfirsthead
  \toprule
  \SimHei \normalsize 年数 & \SimHei \scriptsize 公元 & \SimHei 大事件 \tabularnewline
  \midrule
  \endhead
  \midrule
  元年 & 428 & \tabularnewline\hline
  二年 & 429 & \tabularnewline\hline
  三年 & 430 & \tabularnewline\hline
  四年 & 431 & \tabularnewline
  \bottomrule
\end{longtable}

\subsubsection{义和}

\begin{longtable}{|>{\centering\scriptsize}m{2em}|>{\centering\scriptsize}m{1.3em}|>{\centering}m{8.8em}|}
  % \caption{秦王政}\
  \toprule
  \SimHei \normalsize 年数 & \SimHei \scriptsize 公元 & \SimHei 大事件 \tabularnewline
  % \midrule
  \endfirsthead
  \toprule
  \SimHei \normalsize 年数 & \SimHei \scriptsize 公元 & \SimHei 大事件 \tabularnewline
  \midrule
  \endhead
  \midrule
  元年 & 431 & \tabularnewline\hline
  二年 & 432 & \tabularnewline\hline
  三年 & 433 & \tabularnewline
  \bottomrule
\end{longtable}


%%% Local Variables:
%%% mode: latex
%%% TeX-engine: xetex
%%% TeX-master: "../../Main"
%%% End:

%% -*- coding: utf-8 -*-
%% Time-stamp: <Chen Wang: 2021-11-01 15:01:19>

\subsection{哀王沮渠牧犍\tiny(433-439)}

\subsubsection{生平}

沮渠牧犍(?-447年),一名茂虔,匈奴支系盧水胡族人,沮渠蒙遜之子。十六國時期北涼國末代君主。沮渠牧犍原非蒙遜指定的繼承人,因國內眾臣推舉而登位,任內保持了父親一貫與北魏及南朝宋的關係,然而北魏既滅北燕,魏太武帝亦因毒殺武威公主圖謀和西域使者之言對牧犍不滿,遂出兵攻涼。牧犍初堅守姑臧城不降,但終在北魏軍圍攻下城陷,被逼投降,北涼亡。牧犍弟沮渠無諱帶領北涼殘餘勢力西走,後立起高昌北涼以承涼祚。

沮渠牧犍生于368年,歷任酒泉太守及敦煌太守,北涼義和三年(433年)沮渠蒙遜去世,因继承人沮渠菩提尚幼,眾臣在蒙遜病重時就推較年長的沮渠牧犍為世子,加中外都督、大將軍、錄尚書事。牧犍在蒙遜死後即襲河西王位,改元永和。牧犍隨後向北魏請求任命,獲授都督涼河沙三州西域諸羌戎諸軍事、車騎將軍、開府儀同三司、涼州刺史、河西王。又以父親遺願為由,將妹妹興平公主嫁給魏太武帝拓跋焘。另一方面,牧犍亦向南朝宋上表告知繼位一事,又獲授持節、散騎常侍、都督涼秦河沙四州諸軍事、征西大將軍、領護匈奴中郎將、西夷校尉、涼州刺史、河西王。

永和五年(437年),拓跋焘將其妹武威公主嫁予牧犍,牧犍與嫂子李氏偷情,李氏既得寵,竟與牧犍之姊共毒殺武威公主,幸得解藥不死。拓跋燾要求押解李氏至北魏,牧犍不肯,把李氏安置到酒泉。另北魏西域使者從北涼官員口中得知柔然可汗宣稱他們擊敗了北魏及牧犍聞言大喜並向國內宣傳的事,並向拓跋燾報告,拓跋燾特意派尚書賀多羅去探聽北涼國內的情況,賀多羅回來時亦稱牧犍表面上臣服於魏,實質上並不服從。拓跋焘遂於永和七年(439年)下詔列牧犍十二項罪狀並大舉進攻北涼,詔中亦勸導牧犍自動請降。牧犍聞訊大驚,聽從左丞姚定國計謀不出城迎降,反向柔然求援並命弟弟沮渠董來率兵在城南抗擊魏軍,可是董來軍卻望風潰敗。魏軍兵臨姑臧,牧犍當時聽聞柔然會進攻北魏,於是期望魏軍會因而東還,故此決意固守不降。不過,當時牧犍侄沮渠祖出降並將牧犍的想法告知拓跋燾,拓跋壽遂分兵圍困姑臧,又派源賀招撫北涼諸部,以專心攻城。姑臧最終失守,沮渠牧犍率領文武百官五千人面缚请降,北涼亡,北魏遂統一北方。

拓跋燾將沮渠牧犍及其宗族官民共三萬戶遷至魏都平城,仍以妹婿身份對待他,仍任征西大將軍及河西王爵。北魏太平真君八年(447年),牧犍親族及守護國庫者告發牧犍在姑臧城陷前將國庫中的金銀財寶都拿走,其餘則任由平民搶奪,最終魏人在牧犍家中果然搜得那些財寶;牧犍父子又被指曾毒死数以百计的无辜者,同时在他家找到毒药,姐妹又習曇無讖之術,行為放蕩無愧色;還有指牧犍與北涼舊臣聯絡,意圖謀反,太武帝派遣太常卿崔浩至牧犍家中,将其賜死,諡哀王,其他宗族除沮渠祖外亦被處死。

自西晉末年大亂,不少中原文士都去河西一帶避亂,前涼張氏主政時亦禮遇他們,故涼州文士傳承,號稱「多士」。牧犍亦喜好文學,任用不少文士。任內又曾獻書南朝,亦向南朝求晉、趙《起居注》等書。

\subsubsection{承和}

\begin{longtable}{|>{\centering\scriptsize}m{2em}|>{\centering\scriptsize}m{1.3em}|>{\centering}m{8.8em}|}
  % \caption{秦王政}\
  \toprule
  \SimHei \normalsize 年数 & \SimHei \scriptsize 公元 & \SimHei 大事件 \tabularnewline
  % \midrule
  \endfirsthead
  \toprule
  \SimHei \normalsize 年数 & \SimHei \scriptsize 公元 & \SimHei 大事件 \tabularnewline
  \midrule
  \endhead
  \midrule
  元年 & 433 & \tabularnewline\hline
  二年 & 434 & \tabularnewline\hline
  三年 & 435 & \tabularnewline\hline
  四年 & 436 & \tabularnewline\hline
  五年 & 437 & \tabularnewline\hline
  六年 & 438 & \tabularnewline\hline
  七年 & 439 & \tabularnewline
  \bottomrule
\end{longtable}


%%% Local Variables:
%%% mode: latex
%%% TeX-engine: xetex
%%% TeX-master: "../../Main"
%%% End:



%%% Local Variables:
%%% mode: latex
%%% TeX-engine: xetex
%%% TeX-master: "../../Main"
%%% End:


%%% Local Variables:
%%% mode: latex
%%% TeX-engine: xetex
%%% TeX-master: "../Main"
%%% End:
