%% -*- coding: utf-8 -*-
%% Time-stamp: <Chen Wang: 2019-12-18 16:24:45>

\subsection{武帝\tiny(304-334)}

\subsubsection{生平}

成武帝李雄(274年-334年),字仲儁,氐人,十六国时期成漢開國皇帝(304年至334年在位)。李特第三子,母羅氏。304年李雄自稱成都王,建年號建興。306年正式稱帝,國號大成,史稱成漢。

李雄身高八尺三寸,容貌俊美。少年時以剛烈氣概聞名,常常在鄉里間周旋,有見識的人士都很器重他。有個叫劉化的人,是道家術士,常對人說:“關、隴一帶的士人都將往南去,李家兒子中只有仲俊有非凡的儀表,終歸會成為人主的。 ”

李特在蜀地率流民起義,承皇帝旨意,任命李雄為前將軍。西晉太安二年(303年),李特被益州刺史羅尚擊殺。繼任者李流旋亦病故,李雄自稱大都督、大將軍、益州牧,住在郫城。羅尚派部將攻打李雄,李雄將其擊跑。叔父李驤攻打犍為,切斷羅尚運糧路錢,羅尚的軍隊非常缺糧,攻打得又很急,於是留下牙門羅特固守,羅尚棄城在夜晚逃走。羅特打開城門迎李雄進城,接著攻克成都。在當時李雄的軍隊非常飢餓,於是就率部眾到郪地去就食,挖掘野芋頭來吃。蜀人流亡逃散,往東下到江陽,往南進入七郡。李雄因為西山的范長生居住在山崖洞穴裡,求道養志,想要迎他來立為君而自己做他的臣子。范長生執意推辭。李雄於是盡量避讓,不敢稱制,無論大小事情,都由李國、李離兄弟決斷。李國等人事奉李雄更加恭謹。

永興元年(304年),將領們執意請李雄即尊位,於是李雄自稱成都王,赦免境內罪犯,建年號建興,廢除晉朝法律,約法七章。任命叔父李驤為太傅,兄長李始為太保,折衝將軍李離為太尉,建威將軍李雲為司徒,翊軍將軍李璜為司空,材官李國為太宰,其餘的人委任各自不同。追尊他的曾祖父李虎(即李武)為巴郡桓公,祖父李慕為隴西襄王,父親李特為成都景王,母親羅氏為王太后。范長生從西山乘坐素車來到成都,李雄在門前迎接,執版讓坐,拜為丞相,尊稱為範賢。建興三年(306年),范長生勸李雄稱帝,李雄於是即皇帝位,赦免境內罪犯,改年號為晏平,國號大成,史称成漢。追尊父親李特為景皇帝,廟號始祖,母親羅氏為太后。加授范長生為天地太師,封為西山侯,允許他的部下不參與軍事征伐,租稅全部歸入​​他的家裡。李雄當時建國初始,本來沒有法紀禮儀,將軍們仗著恩情,各自爭奪班次位置。他的尚書令閻式上疏說:“凡是治理國家製定法紀,總是以遵循舊制度為好。漢、晉舊例,只有太尉、大司馬執掌兵權,太傅、太保是父兄一樣的官,講論道義的職位,司徒、司空掌管五教九土的事情。秦代設置丞相,統掌各類政務。漢武末期,破例讓大將軍統掌政務。如今國家的基業剛剛建立,百事還沒有周全,諸公大將們的班列位次有不同,隨之競相請求設置官職,和典章舊制不相符合,應該建立制度來作為楷模法式。”李雄聽從了他的建議。

晏平二年(307年),李雄派李離、李國、李雲等率領二萬徒眾攻入漢中,梁州刺史張殷逃奔到長安。李國等人攻陷南鄭,將漢中人全部遷到蜀地。晏平四年(309年),當時李離駐鎮梓潼,他的部將訇琦、羅羕、張金苟等殺了李離和閻式,以梓潼歸降晉益州刺史羅尚。羅尚派他的部將向奮屯兵在安漢的宜福來威逼李雄,李雄率兵攻打向奮,但是不能克敵。晏平五年(310年),鎮守巴西的李國也被他帳下的文碩殺死,並以巴西投降羅尚。面對如此情況,李雄於是率眾退回,但派他的部將張寶以殺了人逃亡的名義進入了梓潼,並取得訇琦等人的信任。不久,張寶趁訇琦等人出迎羅尚使者的機會關了城門,成功重奪梓潼。正逢羅尚去世,巴郡混亂,李驤攻打涪城。玉衡元年(311年)正月,李驤攻陷涪城,擒獲梓潼太守譙登,接著乘勝進軍討伐文碩,將文碩殺死。李雄很高興,赦免境內罪犯,改年號為玉衡。

玉衡四年(314年),成漢南得漢嘉、涪陵二城,遠方的人相繼歸附,李雄於是下了有關寬大的命令,對投降依附的人都寬免他們的徭役賦稅。虛心而愛惜人才,授職任用都符合接受者的才能,益州於是安定下來。玉衡五年(315年),李雄立其妻任氏為皇后。當時氐王楊難敵兄弟被前趙劉曜打敗,逃奔葭萌,派兒子來成漢作人質。隴西賊人的統帥陳安又依附了李雄。

王衡九年(319年),李雄派李驤征伐越巂郡,於次年逼降越巂太守李釗。李驤進兵從小會攻打寧州刺史王遜,王遜讓他的部將姚岳率全部兵眾迎戰。李驤的軍隊失利,又遇上連日大雨,李驤領軍隊撤回,爭著渡過瀘水,士卒死了很多。李釗到了成都,李雄對待他非常優厚,朝廷的儀式,喪期的禮節,都由李釗決定。

楊難敵、楊堅頭兄弟因敗予前趙而逃奔葭萌時,李雄的侄兒安北將軍李稚優厚地撫慰他們,沒有送其到成都,反待前趙退兵時放他們兄弟回武都,楊難敵於是仗著天險幹了很多不守法紀的事,李稚請求討伐他。李雄不聽群臣諫言,派李稚的長兄中領軍李琀和將軍樂次、費他、李乾等從白水橋進攻下辯,征東將軍李壽督統李琀的弟弟李玝攻打陰平。楊難敵派軍隊抵禦他們,李壽不能推進,可是李琀、李稚長驅直入到達武街。楊難敵派兵切斷他們的後路,四面圍攻,俘虜李琀、李稚,死了數千人。李琀和李稚都是李雄的兄長李蕩的兒子。李雄深深痛悼他們,幾天不吃飯,說起來就流淚,深深地責備自己。

玉衡十四年(324年),李雄打算立兄李蕩之子李班為太子。李雄有十多個兒子,群臣都想立李雄親生的。李雄說:“當初起兵,好比常人舉手保護腦袋一樣,本來不希求帝王的基業。適逢天下喪亂,西晉皇室流離,群情舉兵起義,志在拯救塗炭的生靈,而各位於是推舉我,處在王公的地位之上。這一份基業的建立,功勞本來是先帝的。我兄長是嫡親血統,大柞應歸他繼承,恢弘懿美明智聰睿,就像是上天賦予了他這一使命,大事垂成,死於戰場。李班姿質性情仁厚孝順,好學素有所成,必定會成為大器。”李驤和司徒王達諫阻說:“先王樹立太子的原因,是用來防止篡位奪權的萌芽產生,不能不慎重。吳子捨棄他的兒子而立他的弟弟,所以會有專諸行刺的大禍;宋宣公不立與夷而立宋穆公,終於導致宋督的事變。說到像兒子的話,哪裡比得上真兒子呢?懇請陛下深思。”李雄不聽從,終於立了李班。李驤退下後流著淚說:“禍亂從此開始了!”

前涼文王張駿派遣使者給李雄一封信,勸他去掉皇帝尊號,向晉朝稱藩做屬臣。李雄回信說: “我以前被士大夫們推舉,卻原本無心做帝王,進一步說想成為晉室有大功的臣子,退一步說想和你一樣同為守禦邊藩的將領,掃除亂氛塵埃,以使皇帝的天下安康太平。可是晉室衰微頹敗,恩德聲譽都沒有,我引領東望,有些年月了。正好收到你的來信,在暗室獨處時體會你的真情,感慨無限。知道你想要按照古時候楚漢的舊事,尊奉楚義帝,《春秋》的大義,在這方面沒有人比得上你。”張駿很重視他的話,不斷派使者來往。巴郡曾告急,說有東面來的軍隊。李雄說:“我曾憂慮石勒飛揚跋扈,侵犯威逼琅邪,為這點耿耿於懷。沒想到竟然能夠舉兵,使人感到欣然。”李雄平時清談,有很多類似這樣的話。

李雄因為中原地區喪亡禍亂,就頻繁派遣使者朝貢,和晉穆帝分割天下。張駿統領秦梁二州,在這之前,派傅穎向成漢借道,以便向京師報送表章,李雄不答應。張駿又派治中從事張淳向成漢自稱藩屬,以此來借道。李雄很高興,對張淳說:“貴主英名蓋世,地形險要兵馬強盛,為什麼不自己在一方稱帝?”張淳說:“寡君因為先祖世代是忠良,沒能夠為天下雪恥,解眾人於倒懸,因而日頭偏西還想不起吃飯,枕戈待旦。想憑藉琅邪來中興江東,所以遠隔萬里仍然翼戴朝廷,打算成就齊桓公、晉文公一樣的事業,說什麼自取天下呢!”李雄表情慚愧,說:“我的先祖先父也是晉朝臣民,從前和六郡人避難到此,被同盟的人推舉,才有今天。琅邪如果能在中原使大晉中興,我也會率眾人助他一臂之力。”張淳回去後,向京師報送了表章,天子讚揚了他們。

當時李驤去世,李雄任命李驤的兒子李壽為大將軍、西夷校尉。玉衡二十年(330年)十月,李壽督率征南將軍費黑、征東將軍任巳攻陷巴東,太守楊謙退守建平。李壽另派費黑侵擾建平,東晉巴東監軍毌丘奧退守宜都。

玉衡二十一年(331年)七月,李壽進攻陰平、武都,氐王楊難敵投降。。玉衡二十三年(333年),李雄再派李壽進攻朱提,任命費黑、仰攀為先鋒,又派鎮南將軍任回征伐木落,分散寧州的援兵。寧州刺史尹奉投降,於是佔有南中地區。李雄在這種情況下赦免境內罪犯,派李班討伐平定寧州的夷人,任命李班為撫軍。

玉衡二十四年(334年),李雄頭上生毒瘡。六月二十五日,李雄去世,時年六十一歲,在位三十一年。諡號武皇帝,廟號太宗。葬於安都陵。

李雄的母親羅氏,夢見兩道彩虹從門口升向天空,其中一道虹中間斷開,而後生下李蕩。後來羅氏因為去打水,忽然間像是睡著了,又夢見大蛇繞在她的身上,於是有了身孕,十四個月之後才生下李雄。羅氏常常說:「我的兩個兒子如果有先死的,活著的必定有大富貴。」最終李盪死在李雄前面。

李雄的母親羅氏去世時,李雄相信巫師的話,有很多忌諱,以至於想不入葬。他的司空趙肅諫阻他,李雄才聽從了。李雄想行三年守喪之禮,群臣執意諫阻,李雄不聽。李驤對司空上官惇說:“如今正有急難還沒有消解,我想堅持諫阻,不讓主上最終守居喪之禮,你認為怎麼樣?”上官惇說:“三年的喪制,從天子直到庶人,所以孔子說:'不一定是高宗,古時候的人都是這樣。'但是漢魏以後,天下多難,宗廟是最重要的,不能長時間無人管理,所以不行衰絰一類的禮,盡哀就罷了。”李驤說:“任回將要到來,這個人在處事方面很有決斷,而且主上常常很難不聽他的話,等他到了,就和他一起去請求。”任回抵達後,李驤和任回一同去見李雄。李驤脫去冠流著淚,一再請求因公除去喪服。李雄大哭不答應。任回跪著上前說:“如今王業剛剛開始建立,各種事情都在草創階段,一天沒有主上,天下人心惶惶。從前周武王披著素甲檢閱軍隊,晉襄公繫著墨絰出征,難道是他們希望做的嗎?是為了天下人而委屈自己的原故呀!希望陛下割捨親情順從權宜的方法,以使國運永遠興隆。”於是強行扶李雄起來,脫去喪服親理政事。


\subsubsection{建兴}

\begin{longtable}{|>{\centering\scriptsize}m{2em}|>{\centering\scriptsize}m{1.3em}|>{\centering}m{8.8em}|}
  % \caption{秦王政}\
  \toprule
  \SimHei \normalsize 年数 & \SimHei \scriptsize 公元 & \SimHei 大事件 \tabularnewline
  % \midrule
  \endfirsthead
  \toprule
  \SimHei \normalsize 年数 & \SimHei \scriptsize 公元 & \SimHei 大事件 \tabularnewline
  \midrule
  \endhead
  \midrule
  元年 & 304 & \tabularnewline\hline
  二年 & 305 & \tabularnewline\hline
  三年 & 306 & \tabularnewline
  \bottomrule
\end{longtable}

\subsubsection{晏平}

\begin{longtable}{|>{\centering\scriptsize}m{2em}|>{\centering\scriptsize}m{1.3em}|>{\centering}m{8.8em}|}
  % \caption{秦王政}\
  \toprule
  \SimHei \normalsize 年数 & \SimHei \scriptsize 公元 & \SimHei 大事件 \tabularnewline
  % \midrule
  \endfirsthead
  \toprule
  \SimHei \normalsize 年数 & \SimHei \scriptsize 公元 & \SimHei 大事件 \tabularnewline
  \midrule
  \endhead
  \midrule
  元年 & 306 & \tabularnewline\hline
  二年 & 307 & \tabularnewline\hline
  三年 & 308 & \tabularnewline\hline
  四年 & 309 & \tabularnewline\hline
  五年 & 310 & \tabularnewline
  \bottomrule
\end{longtable}

\subsubsection{玉衡}

\begin{longtable}{|>{\centering\scriptsize}m{2em}|>{\centering\scriptsize}m{1.3em}|>{\centering}m{8.8em}|}
  % \caption{秦王政}\
  \toprule
  \SimHei \normalsize 年数 & \SimHei \scriptsize 公元 & \SimHei 大事件 \tabularnewline
  % \midrule
  \endfirsthead
  \toprule
  \SimHei \normalsize 年数 & \SimHei \scriptsize 公元 & \SimHei 大事件 \tabularnewline
  \midrule
  \endhead
  \midrule
  元年 & 311 & \tabularnewline\hline
  二年 & 312 & \tabularnewline\hline
  三年 & 313 & \tabularnewline\hline
  四年 & 314 & \tabularnewline\hline
  五年 & 315 & \tabularnewline\hline
  六年 & 316 & \tabularnewline\hline
  七年 & 317 & \tabularnewline\hline
  八年 & 318 & \tabularnewline\hline
  九年 & 319 & \tabularnewline\hline
  十年 & 320 & \tabularnewline\hline
  十一年 & 321 & \tabularnewline\hline
  十二年 & 322 & \tabularnewline\hline
  十三年 & 323 & \tabularnewline\hline
  十四年 & 324 & \tabularnewline\hline
  十五年 & 325 & \tabularnewline\hline
  十六年 & 326 & \tabularnewline\hline
  十七年 & 327 & \tabularnewline\hline
  十八年 & 328 & \tabularnewline\hline
  十九年 & 329 & \tabularnewline\hline
  二十年 & 330 & \tabularnewline\hline
  二一年 & 331 & \tabularnewline\hline
  二二年 & 332 & \tabularnewline\hline
  二三年 & 333 & \tabularnewline\hline
  二四年 & 334 & \tabularnewline
  \bottomrule
\end{longtable}


%%% Local Variables:
%%% mode: latex
%%% TeX-engine: xetex
%%% TeX-master: "../../Main"
%%% End:
