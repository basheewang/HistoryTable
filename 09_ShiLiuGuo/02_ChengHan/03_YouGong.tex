%% -*- coding: utf-8 -*-
%% Time-stamp: <Chen Wang: 2019-12-18 16:30:33>

\subsection{幽公\tiny(334-338)}

\subsubsection{哀帝生平}

成哀帝李班(288年-334年),字世文。十六国时期成汉政权的皇帝。为李雄之兄李荡之子。

李班初任平南將軍。李班的叔父李雄雖然有十個兒子,但都不成氣候,所以李雄捨棄自己的兒子而立李班為太子。

李班為人謙虛能廣泛採納意見,尊敬愛護儒士賢人,從何點、李釗以下,李班皆以老師的禮節對待他們,又接納名士王嘏和隴西人董融、天水人文夔等作為賓客朋友。常常對董融等人說:“看到周景王的太子晉、曹魏的太子曹丕、東吳的太子孫登,文章審察辨識的能力,超然出群,自己總是感到慚愧。怎麼古代的賢人那樣高明,而後人就是望塵莫及呀!”李班為人性情博愛,行為符合軌範法度。當時李氏的子弟都崇尚奢侈靡費,可是李班常常自省自勉。每當朝廷上有重大問題要討論,叔父李雄總是讓他參與。李班認為:“古時候開墾的田地平均分配,不論貧富可以一樣獲得土地,如今顯貴人物佔有大面積的荒田,貧苦人想耕種卻沒有土地,佔地多的人將自己多餘的土地出售給他們,這哪裡是王者使天下均等的大義呀!”李雄採納了他的意見。

玉衡二十四年(334年),李雄臥病不起,李班日夜侍奉在身邊。李雄年輕時頻頻作戰,受了很多傷,到這時病重,疤痕全部化膿潰爛,李雄的兒子李越等人都因厭惡而遠遠躲開。李班替他吸吮膿汁。毫無為難的表情,往往在嘗藥時流淚,不脫衣冠地服侍,他的孝心誠意大多如此。

同年六月二十五日,李雄去世,李班即位。任命堂叔建寧王李壽為錄尚書事,來輔佐朝政。李班在宮中依禮服喪,政事都委託給李壽和司徒何點、尚書令王瑰等人。當時李越鎮守江陽,因為李班不是父親李雄的兒子,心中很是不滿。同年九月,李越回到成都奔喪,和他的弟弟安東將軍李期密謀除掉李班。李班的兄弟李玝勸李班遣送李越回江陽,任命李期為梁州刺史,鎮守葭萌。李班認為李雄還未下葬,不忍心讓他們走,推誠待人而心地仁厚,沒有一點嫌隙。當時有兩道白氣出現在天空中,太史令韓豹奏道:“宮中有秘密陰謀的殺氣,要對親戚加以戒備。”李班沒有明白。十月,李班因為夜晚去哭靈,李越在殯宮殺了李班,時年四十七歲,李班共在位一年,於是群臣立李雄的兒子李期繼位。

\subsubsection{幽公生平}

李期(314年-338年),字世運,是十六国時期成汉政权的皇帝。为李雄第四子。

李期聰慧好學,二十歲時就能作文章,輕財物而好施捨,虛心招納人才。初任建威將軍,其父李雄讓兒子們和宗室的子弟們各自憑恩德信義聚集徒眾,多的不到數百人,可是李期單單招到了上千人。他推薦的人,李雄多半任用,所以長史、列署有不少出自他的門下。

玉衡二十四年(334年),李雄死,太子李班繼位。李雄之子李越回成都奔喪時與李期殺掉李班。

因李期多才多兿、并由皇后任氏(李雄正妻)養大而被擁立,即位改元玉恆。李期即位後,首先誅殺李班的弟弟李都。派堂叔李壽到涪城討伐李都之弟李玝,李玝棄城投降東晉。李期封李壽為漢王,任命他為梁州刺史、東羌校尉、中護軍、錄尚書事;封兄長李越為建寧王,任命為相國、大將軍、錄尚書事。立妻子閻氏為皇后。任命衛將軍尹奉為右丞相、驃騎將軍、尚書令,王瑰為司徒。李期自認為圖謀大事已經成功,不重視各位舊臣,在外則信任尚書令景騫、尚書姚華、田褒。田褒沒有別的才能,李雄在位時期,曾勸其立李期為太子,所以李期非常寵幸厚待他。對內則相信宦官許涪等人。國家的刑獄政事,很少讓卿相過問,獎賞和刑罰,都由這幾個人決定,於是國家的法紀紊亂。竟然誣陷尚書僕射、武陵公李載謀反,致使李載被下獄而死。在此之前,東晉建威將軍司馬勛屯兵漢中,李期派李壽攻陷漢中,於是設置守官,設防於南鄭。

李雄的兒子李霸、李保都無病而死,都說是李期毒死了他們,於是大臣們心懷恐懼,人人不能心安。李期誅殺夷滅了很多人家,抄沒他們的婦女和財物來充實自己的後庭,宮內宮外人心惶惶,路上相見也只敢用目光打招呼,勸諫的人都定了罪,人人只想苟且免禍。李期又毒死李壽的養弟安北將軍李攸,和李越、景騫、田褒、姚華商議襲擊李壽等人,打算燒毀市橋而發兵。李期又多次派中常侍許涪到李壽那裡去,察看他的動靜。

李攸死後,李壽非常害怕,又疑心許涪往來頻繁的情況。於玉恒四年(338年),率領一萬步兵、騎兵,從涪城出發前往成都,聲稱景騫、田褒擾亂朝政,所以發動晉陽兵士,以清除李期身邊的惡人。李壽到達成都,李期、李越沒料到他會來,一向不加防備,李壽於是佔領成都,駐兵到宮門前。李期派侍中慰勞李壽,李壽上奏章說李越、景騫,田褒、姚華、許涪、征西將軍李遐、將軍李西等人都心懷奸詐擾亂朝政,圖謀傾覆社稷,大逆不道,罪該誅殺。李期順從了李壽的意見,於是殺死李越、景騫等人。李壽假託太后任氏的名義下令,將李期廢為“邛都縣公”,幽禁在別宮內。李期嘆息說天下的君主竟然成了一個小小的縣公,真是生不如死。同年(338年),李期自缢而死,時年25歲,諡號幽公。

\subsubsection{玉恒}

\begin{longtable}{|>{\centering\scriptsize}m{2em}|>{\centering\scriptsize}m{1.3em}|>{\centering}m{8.8em}|}
  % \caption{秦王政}\
  \toprule
  \SimHei \normalsize 年数 & \SimHei \scriptsize 公元 & \SimHei 大事件 \tabularnewline
  % \midrule
  \endfirsthead
  \toprule
  \SimHei \normalsize 年数 & \SimHei \scriptsize 公元 & \SimHei 大事件 \tabularnewline
  \midrule
  \endhead
  \midrule
  元年 & 335 & \tabularnewline\hline
  二年 & 336 & \tabularnewline\hline
  三年 & 337 & \tabularnewline\hline
  四年 & 338 & \tabularnewline
  \bottomrule
\end{longtable}


%%% Local Variables:
%%% mode: latex
%%% TeX-engine: xetex
%%% TeX-master: "../../Main"
%%% End:
