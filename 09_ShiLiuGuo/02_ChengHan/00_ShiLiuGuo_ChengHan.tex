%% -*- coding: utf-8 -*-
%% Time-stamp: <Chen Wang: 2019-12-18 16:28:29>


\section{成汉\tiny(306-347)}

\subsection{简介}

成汉(304年-347年)也称成、后蜀,是中国历史上五胡十六国时期之割据政權之一。

301年益州的蜀郡的巴氐族领袖李特在蜀郡地领导西北難民反抗西晋的統治,304年其子李雄称成都王,306年李雄称帝,建国号“成”,建都蜀郡的治所成都。338年李寿改国号为“汉”。其领土疆域为益州全部。347年为东晋桓温攻破成都。

在成漢建國之前,李雄之父李特就在益州發展勢力。297年,李特率領關中流民團南下漢中。302年,李特招集流民團起兵,自稱為使持節、大都督、鎮北大將軍,第二年定年號建初元年(因此有人认为303年也可作为建国年)。率軍攻打成都,益州刺史羅尚拒守,李特敗亡,其弟李流繼續統領流民作戰,然不久後病死。李特之子李雄繼位,並於304年攻下成都,開始稱王,國號「大成」,年號建興。:159

334年,李雄病死,其兄之子李班繼位,不久之後李雄之子李期即殺李班自立。338年,李驤之子李壽又殺了李期自立為帝,將國號改為「漢」。大修宮殿,生活奢侈荒淫,人民受到嚴酷的徭役壓迫。李壽死後,其子李勢繼位,大肆殺伐,國勢更加衰弱。347年,東晉桓溫率兵入蜀,李勢投降,成漢滅亡,立國共44年,两年后残余力量也被东晋消灭。:160

其国号先为“成”,史书也有称“大成”,或以为“大”是尊称。国号“成”来自于成都这个地名,也有说是袭用公孙述的旧称(成家)。后李特弟李骧之子汉王李寿发动兵变夺取政权,改国号为“汉”,史书上又合称为“成汉”,以区别于其他称为“汉”的政权。又其统治地区主要为益州的蜀地,故又被一些史书称为“蜀”(例如《十六国春秋·蜀录》)。《晋书》又称之为“后蜀”,以别于三国时期刘备的前蜀,唐代以后,已基本不使用“后蜀”來指称,而专用作五代十国时期后蜀政权的专称。


%% -*- coding: utf-8 -*-
%% Time-stamp: <Chen Wang: 2019-12-18 15:58:30>

\subsection{李特\tiny(303)}

\subsubsection{生平}

李特(3世紀?-303年),字玄休,西晉末年巴氐人(一說為賨人),其父為李慕;十六國時期成漢國建立者李雄之父,是成漢政權的奠基者。後來李特之子李雄稱王時,追諡李特為成都景王,等到稱帝時,再追諡為景皇帝,廟號為始祖。

李特祖籍為巴西郡宕渠縣(今中國四川省渠縣),其先祖後於曹魏時被遷至略陽(今中國甘肅省秦安縣)。李特身長八尺,在兄弟間排行第二,並與其兄弟都精於騎射,以武略聞名,鄉里紛紛歸附李氏兄弟。西晉元康八年(298年),因齊萬年叛亂使得關中混亂,加上多年饑荒,李特兄弟於是與關中人民一同入蜀。原本朝廷不容許他們進入蜀地,僅讓他們留駐漢中等地,並派侍御史李苾前往慰勞並監察,不容許他們經劍閣入蜀。但因李苾受賄並上奏朝廷,故此李特和一眾中流民都得以在益州和梁州一帶居住。

永康元年(300年),益州刺史趙廞被朝廷徵召為大長秋,原職由成都內史耿滕接任。趙廞身為皇后賈南風姻親,但當年趙王司馬倫就廢黜賈南風並執掌朝政,趙廞因此害怕會因為自己與賈南風的關係而受逼害;而且趙廞亦見晉室宗室相殘,暗有割據巴蜀之意,於是決心叛晉,不旦開倉賑擠流民以收買人心,亦因李特兄弟和其黨眾都強壯勇猛,趙廞於是厚待他們並作為自己爪牙。李特等人亦恃仗趙廞的勢力,聚眾為盜,蜀人視為大患。及後趙廞擊殺耿滕,自稱大都督、大將軍、益州牧。當時李特三弟李庠率親族、黨眾及四千騎兵歸附趙廞,但趙廞因李庠通曉兵法,軍容齊整而感到不快,最終於次年(301年)殺害李庠。

趙廞雖然歸還李庠屍體給李特,並任用李特兄弟為督護以作安撫,但李特兄弟都怨恨趙廞,引兵北歸緜竹。李特後秘密地招收到七千多名兵眾,夜襲並大破趙廞所派北防晉兵的軍隊,並進攻成都(今四川省成都市)。趙廞猝不及防,逃亡被殺,李特則攻陷成都,縱兵大掠,殺趙廞屬官及任命的官員,並派牙門王角及李基向西晉朝廷陳述趙廞罪狀。

在趙廞叛變之時,朝廷另派梁州刺史羅尚入蜀任益州刺史。李特知道羅尚入蜀的消息後十分畏懼,特意派其弟李驤帶著寶物迎接,令羅尚十分高興。李流及後在緜竹為羅尚勞軍,但廣漢太守辛冉和羅尚牙門將王敦卻勸羅尚殺李流。羅尚雖未接納,但李流已經十分畏懼。

及後,朝廷命秦、雍二州召還入蜀的流民。但李特在後來才入蜀的兄長李輔口中得知中國已亂,因此不欲回到關中,於是派閻式請求羅尚,又賄賂羅尚及監督流民回州的御史馮該等,成功讓他們延遲到秋天才起行。同時,朝廷以平定趙廞之功封賞李特,拜李特為宣威將軍,封長樂鄉侯。同時下詔命州府列出當地與李特平定趙廞的流人以作封賞,但辛冉卻沒有如實上報,意圖將平定趙廞作為自己的功勳,於是招來眾人的怨恨。

七月,羅尚再催逼流民起程,然而流民都不願歸去,而且未收割穀物,未有旅費,於是深感憂慮。李特於是再派閻式請求再延遲至冬季才起行,但羅尚聽從辛冉和李苾之言,不再答允。辛冉當時又打算殺害流民首領以獲得他們的物資,於是以當日趙廞敗死時流民大掠成都為由,要在關口搜奪經過的流民的物資財寶。閻式看到這些情形,於是回到李特所駐的緜竹,並勸李特防備可能進襲的辛冉。而當時李特亦因多次為流民發聲,於是獲流民歸心和歸附。而常時李特又將辛冉懸紅捕殺李特兄弟的文告全部收下並改為求取當地李氏、任氏、閻氏等豪族和氐、叟侯王首級,於是令流民大懼,短時間內就有超過二萬人在李特麾下。李特於是特定將部眾分為兩營,分別由自己和李流統率。

不久,辛冉就派广漢都尉曾元、牙門張顯等領兵三萬進攻李特,羅尚亦派督護田佐助戰,而李特因早有準備,下令戒嚴等待曾元等到來。曾元等人到後,李特仍安然躺臥著,沒有任何動作,但當約半數軍隊進入營壘時,李特就命伏兵突擊曾元,大敗敵軍,並殺死曾元、張顯和田佐,並送首給羅尚。李特至此反叛。

當地流民於是共推李特為主,並上書請行鎮北大將軍,承制封拜。隨後便領兵進攻辛冉所在的廣漢,辛冉不敵而退奔德陽。李特在攻佔廣漢後便進攻成都。因羅尚貪婪殘暴,對比李特與蜀人約法三章,並且施捨人民,賑濟借貸,禮賢下士,拔擢人才,軍紀及施政肅然,人民都支持李特。李特屢次擊敗羅尚,羅尚唯有死守成都,並向梁州及南夷都尉李毅求救。

永寧二年(302年),平西將軍、河間王司馬顒派衙博及張微討伐李特,李毅亦派兵支援羅尚,羅尚亦派張龜進攻李特。但李特自領兵擊潰張龜,並命李蕩和李雄攻衙博,不但擊退對方,並收降了巴西郡和葭萌。同年,李特自稱為大將軍、益州牧,都督梁、益二州諸軍事。及後李特就進攻張微,但張微居高據險防守,並趁李特營壘空虛時派兵進攻李特。當時李特處於劣勢,幸李蕩援軍趕到並拼死一戰擊潰張微,才令李特脫險;及後更進攻並斬殺張微。當時羅尚繼續進攻城外李特等軍,但多次交戰皆戰敗,更令李特軍獲得大量兵器和盔甲。及後又多次擊敗梁州刺史許雄所派的軍隊。

太安二年(303年),李特擊潰羅尚駐紥在郫水的水軍,並再進攻成都,蜀郡太守徐儉於是以成都少城投降。但李特進城後僅取用馬匹作軍隊使用,並沒有進行搶掠,並且改元建初。當時蜀人聚居成各個塢自守,都款待李特,李特亦派人安撫並讓流民到各個塢內取食以節省軍糧開支。當時李流和上官惇都勸李特小心各塢都不是誠心支持自己,提防他們反叛,但李特決意安民,不去提防他們。

及後荊州刺史宗岱和建平太守孫阜率水軍救援據守成都太城的羅尚,李特派李蕩等與任臧合兵抵禦。其時宗岱等軍軍勢強盛,令各塢生有二心;同時羅尚又派益州從事任叡又詐降李特,暗中聯結塢主與羅尚一同舉兵,並假稱成都太城內糧食將盡。二月,羅尚率兵乘虛襲擊李特,各塢都響應,於是李特大敗,收兵駐守新繁。後李特見羅尚退兵,於是追擊,最終被羅尚出大軍反擊,李特及李輔和李遠都戰死,屍體被焚毀並送首級到首都洛陽。其弟李流接管其部眾。

\subsubsection{建初}

\begin{longtable}{|>{\centering\scriptsize}m{2em}|>{\centering\scriptsize}m{1.3em}|>{\centering}m{8.8em}|}
  % \caption{秦王政}\
  \toprule
  \SimHei \normalsize 年数 & \SimHei \scriptsize 公元 & \SimHei 大事件 \tabularnewline
  % \midrule
  \endfirsthead
  \toprule
  \SimHei \normalsize 年数 & \SimHei \scriptsize 公元 & \SimHei 大事件 \tabularnewline
  \midrule
  \endhead
  \midrule
  元年 & 303 & \tabularnewline\hline
  二年 & 304 & \tabularnewline
  \bottomrule
\end{longtable}


%%% Local Variables:
%%% mode: latex
%%% TeX-engine: xetex
%%% TeX-master: "../../Main"
%%% End:

%% -*- coding: utf-8 -*-
%% Time-stamp: <Chen Wang: 2019-12-18 16:24:45>

\subsection{武帝\tiny(304-334)}

\subsubsection{生平}

成武帝李雄(274年-334年),字仲儁,氐人,十六国时期成漢開國皇帝(304年至334年在位)。李特第三子,母羅氏。304年李雄自稱成都王,建年號建興。306年正式稱帝,國號大成,史稱成漢。

李雄身高八尺三寸,容貌俊美。少年時以剛烈氣概聞名,常常在鄉里間周旋,有見識的人士都很器重他。有個叫劉化的人,是道家術士,常對人說:“關、隴一帶的士人都將往南去,李家兒子中只有仲俊有非凡的儀表,終歸會成為人主的。 ”

李特在蜀地率流民起義,承皇帝旨意,任命李雄為前將軍。西晉太安二年(303年),李特被益州刺史羅尚擊殺。繼任者李流旋亦病故,李雄自稱大都督、大將軍、益州牧,住在郫城。羅尚派部將攻打李雄,李雄將其擊跑。叔父李驤攻打犍為,切斷羅尚運糧路錢,羅尚的軍隊非常缺糧,攻打得又很急,於是留下牙門羅特固守,羅尚棄城在夜晚逃走。羅特打開城門迎李雄進城,接著攻克成都。在當時李雄的軍隊非常飢餓,於是就率部眾到郪地去就食,挖掘野芋頭來吃。蜀人流亡逃散,往東下到江陽,往南進入七郡。李雄因為西山的范長生居住在山崖洞穴裡,求道養志,想要迎他來立為君而自己做他的臣子。范長生執意推辭。李雄於是盡量避讓,不敢稱制,無論大小事情,都由李國、李離兄弟決斷。李國等人事奉李雄更加恭謹。

永興元年(304年),將領們執意請李雄即尊位,於是李雄自稱成都王,赦免境內罪犯,建年號建興,廢除晉朝法律,約法七章。任命叔父李驤為太傅,兄長李始為太保,折衝將軍李離為太尉,建威將軍李雲為司徒,翊軍將軍李璜為司空,材官李國為太宰,其餘的人委任各自不同。追尊他的曾祖父李虎(即李武)為巴郡桓公,祖父李慕為隴西襄王,父親李特為成都景王,母親羅氏為王太后。范長生從西山乘坐素車來到成都,李雄在門前迎接,執版讓坐,拜為丞相,尊稱為範賢。建興三年(306年),范長生勸李雄稱帝,李雄於是即皇帝位,赦免境內罪犯,改年號為晏平,國號大成,史称成漢。追尊父親李特為景皇帝,廟號始祖,母親羅氏為太后。加授范長生為天地太師,封為西山侯,允許他的部下不參與軍事征伐,租稅全部歸入​​他的家裡。李雄當時建國初始,本來沒有法紀禮儀,將軍們仗著恩情,各自爭奪班次位置。他的尚書令閻式上疏說:“凡是治理國家製定法紀,總是以遵循舊制度為好。漢、晉舊例,只有太尉、大司馬執掌兵權,太傅、太保是父兄一樣的官,講論道義的職位,司徒、司空掌管五教九土的事情。秦代設置丞相,統掌各類政務。漢武末期,破例讓大將軍統掌政務。如今國家的基業剛剛建立,百事還沒有周全,諸公大將們的班列位次有不同,隨之競相請求設置官職,和典章舊制不相符合,應該建立制度來作為楷模法式。”李雄聽從了他的建議。

晏平二年(307年),李雄派李離、李國、李雲等率領二萬徒眾攻入漢中,梁州刺史張殷逃奔到長安。李國等人攻陷南鄭,將漢中人全部遷到蜀地。晏平四年(309年),當時李離駐鎮梓潼,他的部將訇琦、羅羕、張金苟等殺了李離和閻式,以梓潼歸降晉益州刺史羅尚。羅尚派他的部將向奮屯兵在安漢的宜福來威逼李雄,李雄率兵攻打向奮,但是不能克敵。晏平五年(310年),鎮守巴西的李國也被他帳下的文碩殺死,並以巴西投降羅尚。面對如此情況,李雄於是率眾退回,但派他的部將張寶以殺了人逃亡的名義進入了梓潼,並取得訇琦等人的信任。不久,張寶趁訇琦等人出迎羅尚使者的機會關了城門,成功重奪梓潼。正逢羅尚去世,巴郡混亂,李驤攻打涪城。玉衡元年(311年)正月,李驤攻陷涪城,擒獲梓潼太守譙登,接著乘勝進軍討伐文碩,將文碩殺死。李雄很高興,赦免境內罪犯,改年號為玉衡。

玉衡四年(314年),成漢南得漢嘉、涪陵二城,遠方的人相繼歸附,李雄於是下了有關寬大的命令,對投降依附的人都寬免他們的徭役賦稅。虛心而愛惜人才,授職任用都符合接受者的才能,益州於是安定下來。玉衡五年(315年),李雄立其妻任氏為皇后。當時氐王楊難敵兄弟被前趙劉曜打敗,逃奔葭萌,派兒子來成漢作人質。隴西賊人的統帥陳安又依附了李雄。

王衡九年(319年),李雄派李驤征伐越巂郡,於次年逼降越巂太守李釗。李驤進兵從小會攻打寧州刺史王遜,王遜讓他的部將姚岳率全部兵眾迎戰。李驤的軍隊失利,又遇上連日大雨,李驤領軍隊撤回,爭著渡過瀘水,士卒死了很多。李釗到了成都,李雄對待他非常優厚,朝廷的儀式,喪期的禮節,都由李釗決定。

楊難敵、楊堅頭兄弟因敗予前趙而逃奔葭萌時,李雄的侄兒安北將軍李稚優厚地撫慰他們,沒有送其到成都,反待前趙退兵時放他們兄弟回武都,楊難敵於是仗著天險幹了很多不守法紀的事,李稚請求討伐他。李雄不聽群臣諫言,派李稚的長兄中領軍李琀和將軍樂次、費他、李乾等從白水橋進攻下辯,征東將軍李壽督統李琀的弟弟李玝攻打陰平。楊難敵派軍隊抵禦他們,李壽不能推進,可是李琀、李稚長驅直入到達武街。楊難敵派兵切斷他們的後路,四面圍攻,俘虜李琀、李稚,死了數千人。李琀和李稚都是李雄的兄長李蕩的兒子。李雄深深痛悼他們,幾天不吃飯,說起來就流淚,深深地責備自己。

玉衡十四年(324年),李雄打算立兄李蕩之子李班為太子。李雄有十多個兒子,群臣都想立李雄親生的。李雄說:“當初起兵,好比常人舉手保護腦袋一樣,本來不希求帝王的基業。適逢天下喪亂,西晉皇室流離,群情舉兵起義,志在拯救塗炭的生靈,而各位於是推舉我,處在王公的地位之上。這一份基業的建立,功勞本來是先帝的。我兄長是嫡親血統,大柞應歸他繼承,恢弘懿美明智聰睿,就像是上天賦予了他這一使命,大事垂成,死於戰場。李班姿質性情仁厚孝順,好學素有所成,必定會成為大器。”李驤和司徒王達諫阻說:“先王樹立太子的原因,是用來防止篡位奪權的萌芽產生,不能不慎重。吳子捨棄他的兒子而立他的弟弟,所以會有專諸行刺的大禍;宋宣公不立與夷而立宋穆公,終於導致宋督的事變。說到像兒子的話,哪裡比得上真兒子呢?懇請陛下深思。”李雄不聽從,終於立了李班。李驤退下後流著淚說:“禍亂從此開始了!”

前涼文王張駿派遣使者給李雄一封信,勸他去掉皇帝尊號,向晉朝稱藩做屬臣。李雄回信說: “我以前被士大夫們推舉,卻原本無心做帝王,進一步說想成為晉室有大功的臣子,退一步說想和你一樣同為守禦邊藩的將領,掃除亂氛塵埃,以使皇帝的天下安康太平。可是晉室衰微頹敗,恩德聲譽都沒有,我引領東望,有些年月了。正好收到你的來信,在暗室獨處時體會你的真情,感慨無限。知道你想要按照古時候楚漢的舊事,尊奉楚義帝,《春秋》的大義,在這方面沒有人比得上你。”張駿很重視他的話,不斷派使者來往。巴郡曾告急,說有東面來的軍隊。李雄說:“我曾憂慮石勒飛揚跋扈,侵犯威逼琅邪,為這點耿耿於懷。沒想到竟然能夠舉兵,使人感到欣然。”李雄平時清談,有很多類似這樣的話。

李雄因為中原地區喪亡禍亂,就頻繁派遣使者朝貢,和晉穆帝分割天下。張駿統領秦梁二州,在這之前,派傅穎向成漢借道,以便向京師報送表章,李雄不答應。張駿又派治中從事張淳向成漢自稱藩屬,以此來借道。李雄很高興,對張淳說:“貴主英名蓋世,地形險要兵馬強盛,為什麼不自己在一方稱帝?”張淳說:“寡君因為先祖世代是忠良,沒能夠為天下雪恥,解眾人於倒懸,因而日頭偏西還想不起吃飯,枕戈待旦。想憑藉琅邪來中興江東,所以遠隔萬里仍然翼戴朝廷,打算成就齊桓公、晉文公一樣的事業,說什麼自取天下呢!”李雄表情慚愧,說:“我的先祖先父也是晉朝臣民,從前和六郡人避難到此,被同盟的人推舉,才有今天。琅邪如果能在中原使大晉中興,我也會率眾人助他一臂之力。”張淳回去後,向京師報送了表章,天子讚揚了他們。

當時李驤去世,李雄任命李驤的兒子李壽為大將軍、西夷校尉。玉衡二十年(330年)十月,李壽督率征南將軍費黑、征東將軍任巳攻陷巴東,太守楊謙退守建平。李壽另派費黑侵擾建平,東晉巴東監軍毌丘奧退守宜都。

玉衡二十一年(331年)七月,李壽進攻陰平、武都,氐王楊難敵投降。。玉衡二十三年(333年),李雄再派李壽進攻朱提,任命費黑、仰攀為先鋒,又派鎮南將軍任回征伐木落,分散寧州的援兵。寧州刺史尹奉投降,於是佔有南中地區。李雄在這種情況下赦免境內罪犯,派李班討伐平定寧州的夷人,任命李班為撫軍。

玉衡二十四年(334年),李雄頭上生毒瘡。六月二十五日,李雄去世,時年六十一歲,在位三十一年。諡號武皇帝,廟號太宗。葬於安都陵。

李雄的母親羅氏,夢見兩道彩虹從門口升向天空,其中一道虹中間斷開,而後生下李蕩。後來羅氏因為去打水,忽然間像是睡著了,又夢見大蛇繞在她的身上,於是有了身孕,十四個月之後才生下李雄。羅氏常常說:「我的兩個兒子如果有先死的,活著的必定有大富貴。」最終李盪死在李雄前面。

李雄的母親羅氏去世時,李雄相信巫師的話,有很多忌諱,以至於想不入葬。他的司空趙肅諫阻他,李雄才聽從了。李雄想行三年守喪之禮,群臣執意諫阻,李雄不聽。李驤對司空上官惇說:“如今正有急難還沒有消解,我想堅持諫阻,不讓主上最終守居喪之禮,你認為怎麼樣?”上官惇說:“三年的喪制,從天子直到庶人,所以孔子說:'不一定是高宗,古時候的人都是這樣。'但是漢魏以後,天下多難,宗廟是最重要的,不能長時間無人管理,所以不行衰絰一類的禮,盡哀就罷了。”李驤說:“任回將要到來,這個人在處事方面很有決斷,而且主上常常很難不聽他的話,等他到了,就和他一起去請求。”任回抵達後,李驤和任回一同去見李雄。李驤脫去冠流著淚,一再請求因公除去喪服。李雄大哭不答應。任回跪著上前說:“如今王業剛剛開始建立,各種事情都在草創階段,一天沒有主上,天下人心惶惶。從前周武王披著素甲檢閱軍隊,晉襄公繫著墨絰出征,難道是他們希望做的嗎?是為了天下人而委屈自己的原故呀!希望陛下割捨親情順從權宜的方法,以使國運永遠興隆。”於是強行扶李雄起來,脫去喪服親理政事。


\subsubsection{建兴}

\begin{longtable}{|>{\centering\scriptsize}m{2em}|>{\centering\scriptsize}m{1.3em}|>{\centering}m{8.8em}|}
  % \caption{秦王政}\
  \toprule
  \SimHei \normalsize 年数 & \SimHei \scriptsize 公元 & \SimHei 大事件 \tabularnewline
  % \midrule
  \endfirsthead
  \toprule
  \SimHei \normalsize 年数 & \SimHei \scriptsize 公元 & \SimHei 大事件 \tabularnewline
  \midrule
  \endhead
  \midrule
  元年 & 304 & \tabularnewline\hline
  二年 & 305 & \tabularnewline\hline
  三年 & 306 & \tabularnewline
  \bottomrule
\end{longtable}

\subsubsection{晏平}

\begin{longtable}{|>{\centering\scriptsize}m{2em}|>{\centering\scriptsize}m{1.3em}|>{\centering}m{8.8em}|}
  % \caption{秦王政}\
  \toprule
  \SimHei \normalsize 年数 & \SimHei \scriptsize 公元 & \SimHei 大事件 \tabularnewline
  % \midrule
  \endfirsthead
  \toprule
  \SimHei \normalsize 年数 & \SimHei \scriptsize 公元 & \SimHei 大事件 \tabularnewline
  \midrule
  \endhead
  \midrule
  元年 & 306 & \tabularnewline\hline
  二年 & 307 & \tabularnewline\hline
  三年 & 308 & \tabularnewline\hline
  四年 & 309 & \tabularnewline\hline
  五年 & 310 & \tabularnewline
  \bottomrule
\end{longtable}

\subsubsection{玉衡}

\begin{longtable}{|>{\centering\scriptsize}m{2em}|>{\centering\scriptsize}m{1.3em}|>{\centering}m{8.8em}|}
  % \caption{秦王政}\
  \toprule
  \SimHei \normalsize 年数 & \SimHei \scriptsize 公元 & \SimHei 大事件 \tabularnewline
  % \midrule
  \endfirsthead
  \toprule
  \SimHei \normalsize 年数 & \SimHei \scriptsize 公元 & \SimHei 大事件 \tabularnewline
  \midrule
  \endhead
  \midrule
  元年 & 311 & \tabularnewline\hline
  二年 & 312 & \tabularnewline\hline
  三年 & 313 & \tabularnewline\hline
  四年 & 314 & \tabularnewline\hline
  五年 & 315 & \tabularnewline\hline
  六年 & 316 & \tabularnewline\hline
  七年 & 317 & \tabularnewline\hline
  八年 & 318 & \tabularnewline\hline
  九年 & 319 & \tabularnewline\hline
  十年 & 320 & \tabularnewline\hline
  十一年 & 321 & \tabularnewline\hline
  十二年 & 322 & \tabularnewline\hline
  十三年 & 323 & \tabularnewline\hline
  十四年 & 324 & \tabularnewline\hline
  十五年 & 325 & \tabularnewline\hline
  十六年 & 326 & \tabularnewline\hline
  十七年 & 327 & \tabularnewline\hline
  十八年 & 328 & \tabularnewline\hline
  十九年 & 329 & \tabularnewline\hline
  二十年 & 330 & \tabularnewline\hline
  二一年 & 331 & \tabularnewline\hline
  二二年 & 332 & \tabularnewline\hline
  二三年 & 333 & \tabularnewline\hline
  二四年 & 334 & \tabularnewline
  \bottomrule
\end{longtable}


%%% Local Variables:
%%% mode: latex
%%% TeX-engine: xetex
%%% TeX-master: "../../Main"
%%% End:

%% -*- coding: utf-8 -*-
%% Time-stamp: <Chen Wang: 2021-11-01 11:52:53>

\subsection{幽公\tiny(334-338)}

\subsubsection{哀帝李班生平}

成哀帝李班(288年-334年),字世文。十六国时期成汉政权的皇帝。为李雄之兄李荡之子。

李班初任平南將軍。李班的叔父李雄雖然有十個兒子,但都不成氣候,所以李雄捨棄自己的兒子而立李班為太子。

李班為人謙虛能廣泛採納意見,尊敬愛護儒士賢人,從何點、李釗以下,李班皆以老師的禮節對待他們,又接納名士王嘏和隴西人董融、天水人文夔等作為賓客朋友。常常對董融等人說:“看到周景王的太子晉、曹魏的太子曹丕、東吳的太子孫登,文章審察辨識的能力,超然出群,自己總是感到慚愧。怎麼古代的賢人那樣高明,而後人就是望塵莫及呀!”李班為人性情博愛,行為符合軌範法度。當時李氏的子弟都崇尚奢侈靡費,可是李班常常自省自勉。每當朝廷上有重大問題要討論,叔父李雄總是讓他參與。李班認為:“古時候開墾的田地平均分配,不論貧富可以一樣獲得土地,如今顯貴人物佔有大面積的荒田,貧苦人想耕種卻沒有土地,佔地多的人將自己多餘的土地出售給他們,這哪裡是王者使天下均等的大義呀!”李雄採納了他的意見。

玉衡二十四年(334年),李雄臥病不起,李班日夜侍奉在身邊。李雄年輕時頻頻作戰,受了很多傷,到這時病重,疤痕全部化膿潰爛,李雄的兒子李越等人都因厭惡而遠遠躲開。李班替他吸吮膿汁。毫無為難的表情,往往在嘗藥時流淚,不脫衣冠地服侍,他的孝心誠意大多如此。

同年六月二十五日,李雄去世,李班即位。任命堂叔建寧王李壽為錄尚書事,來輔佐朝政。李班在宮中依禮服喪,政事都委託給李壽和司徒何點、尚書令王瑰等人。當時李越鎮守江陽,因為李班不是父親李雄的兒子,心中很是不滿。同年九月,李越回到成都奔喪,和他的弟弟安東將軍李期密謀除掉李班。李班的兄弟李玝勸李班遣送李越回江陽,任命李期為梁州刺史,鎮守葭萌。李班認為李雄還未下葬,不忍心讓他們走,推誠待人而心地仁厚,沒有一點嫌隙。當時有兩道白氣出現在天空中,太史令韓豹奏道:“宮中有秘密陰謀的殺氣,要對親戚加以戒備。”李班沒有明白。十月,李班因為夜晚去哭靈,李越在殯宮殺了李班,時年四十七歲,李班共在位一年,於是群臣立李雄的兒子李期繼位。

\subsubsection{幽公李期生平}

李期(314年-338年),字世運,是十六国時期成汉政权的皇帝。为李雄第四子。

李期聰慧好學,二十歲時就能作文章,輕財物而好施捨,虛心招納人才。初任建威將軍,其父李雄讓兒子們和宗室的子弟們各自憑恩德信義聚集徒眾,多的不到數百人,可是李期單單招到了上千人。他推薦的人,李雄多半任用,所以長史、列署有不少出自他的門下。

玉衡二十四年(334年),李雄死,太子李班繼位。李雄之子李越回成都奔喪時與李期殺掉李班。

因李期多才多兿、并由皇后任氏(李雄正妻)養大而被擁立,即位改元玉恆。李期即位後,首先誅殺李班的弟弟李都。派堂叔李壽到涪城討伐李都之弟李玝,李玝棄城投降東晉。李期封李壽為漢王,任命他為梁州刺史、東羌校尉、中護軍、錄尚書事;封兄長李越為建寧王,任命為相國、大將軍、錄尚書事。立妻子閻氏為皇后。任命衛將軍尹奉為右丞相、驃騎將軍、尚書令,王瑰為司徒。李期自認為圖謀大事已經成功,不重視各位舊臣,在外則信任尚書令景騫、尚書姚華、田褒。田褒沒有別的才能,李雄在位時期,曾勸其立李期為太子,所以李期非常寵幸厚待他。對內則相信宦官許涪等人。國家的刑獄政事,很少讓卿相過問,獎賞和刑罰,都由這幾個人決定,於是國家的法紀紊亂。竟然誣陷尚書僕射、武陵公李載謀反,致使李載被下獄而死。在此之前,東晉建威將軍司馬勛屯兵漢中,李期派李壽攻陷漢中,於是設置守官,設防於南鄭。

李雄的兒子李霸、李保都無病而死,都說是李期毒死了他們,於是大臣們心懷恐懼,人人不能心安。李期誅殺夷滅了很多人家,抄沒他們的婦女和財物來充實自己的後庭,宮內宮外人心惶惶,路上相見也只敢用目光打招呼,勸諫的人都定了罪,人人只想苟且免禍。李期又毒死李壽的養弟安北將軍李攸,和李越、景騫、田褒、姚華商議襲擊李壽等人,打算燒毀市橋而發兵。李期又多次派中常侍許涪到李壽那裡去,察看他的動靜。

李攸死後,李壽非常害怕,又疑心許涪往來頻繁的情況。於玉恒四年(338年),率領一萬步兵、騎兵,從涪城出發前往成都,聲稱景騫、田褒擾亂朝政,所以發動晉陽兵士,以清除李期身邊的惡人。李壽到達成都,李期、李越沒料到他會來,一向不加防備,李壽於是佔領成都,駐兵到宮門前。李期派侍中慰勞李壽,李壽上奏章說李越、景騫,田褒、姚華、許涪、征西將軍李遐、將軍李西等人都心懷奸詐擾亂朝政,圖謀傾覆社稷,大逆不道,罪該誅殺。李期順從了李壽的意見,於是殺死李越、景騫等人。李壽假託太后任氏的名義下令,將李期廢為“邛都縣公”,幽禁在別宮內。李期嘆息說天下的君主竟然成了一個小小的縣公,真是生不如死。同年(338年),李期自缢而死,時年25歲,諡號幽公。

\subsubsection{玉恒}

\begin{longtable}{|>{\centering\scriptsize}m{2em}|>{\centering\scriptsize}m{1.3em}|>{\centering}m{8.8em}|}
  % \caption{秦王政}\
  \toprule
  \SimHei \normalsize 年数 & \SimHei \scriptsize 公元 & \SimHei 大事件 \tabularnewline
  % \midrule
  \endfirsthead
  \toprule
  \SimHei \normalsize 年数 & \SimHei \scriptsize 公元 & \SimHei 大事件 \tabularnewline
  \midrule
  \endhead
  \midrule
  元年 & 335 & \tabularnewline\hline
  二年 & 336 & \tabularnewline\hline
  三年 & 337 & \tabularnewline\hline
  四年 & 338 & \tabularnewline
  \bottomrule
\end{longtable}


%%% Local Variables:
%%% mode: latex
%%% TeX-engine: xetex
%%% TeX-master: "../../Main"
%%% End:

%% -*- coding: utf-8 -*-
%% Time-stamp: <Chen Wang: 2021-11-01 11:53:07>

\subsection{昭文帝李寿\tiny(338-343)}

\subsubsection{生平}

汉昭文帝李寿(300年-343年),字武考,十六国时期成汉政权的皇帝。为李特之弟李骧少子。

338年即位后改国号为“汉”。343年病死。

李壽天生聰敏好學,少尚禮容,在李氏諸子中相當突出,受到李雄欣賞,認為他足以擔當大任,乃授以前將軍,統領巴蜀軍事,彼遷征東將軍,當時年僅十九歲。在任期間以處士譙秀為謀主,對其言聽計從,令他在巴蜀威德日隆。

李驤卒,李壽再先後升遷為大將軍、大都督、侍中、並封扶風公、錄尚書事。在出征寧州時,圍攻百餘日,最終悉數平定諸郡,李雄大悅,再加封建寧王。

李雄卒,受遺命輔政,李期繼位,改封漢王,兼任梁州刺史,獲賜封梁州五郡。

自此,李壽威名遠播,卻同時深為李越、景騫等所忌憚,令李壽深為擔憂。在暫代李玝屯田涪水期間,每次朝覲日期到來,往趁以邊景賊寇橫行,不可放鬆戒備離開而推卻。同時李壽又因為李期、李越兄弟等十餘人年紀漸長,又擁有精兵,擔心不能自全,便數次欲聘得龔壯為其效命。龔壯雖然不答應,不過仍多次與李壽見面。時值岷山崩塌,江水因此枯竭,李壽認為此乃上天預示災劫,因而非常厭惡,便問龔壯自安之法。龔壯的父親及叔父,被李特殺害,為了假借李壽之手報仇,便向李壽提出起兵自立以自保的建議,最終獲得李壽採納,之後便暗中與長史略陽羅恆、巴西解思明共同謀奪首都成都,並得數千人加入。李壽軍起兵突襲成都,將其攻克,縱兵擄掠,甚至姦污李雄女兒及李氏諸婦,並將之殘殺。羅恆、解思明、李奕、王利等人乃勸李壽自稱鎮西將軍、益州牧、成都王,並向晉朝稱藩。

成玉恆四年(338年),大臣任調、司馬蔡興、侍中李艷以及張烈等勸李壽自立。李壽命巫師卜卦,得出「可當數年天子」的預示,任調大喜,進言「一日尚且滿足,何況數年!」(一日尚為足,而況數年乎!)李壽以「有道是「早上聽到警世的道理,就算當晚要死亦無悔無憾」(朝聞道,夕死可矣),任調的進言,實在是上乘之策!」乃自稱為帝,舉國大赦,並改元漢興,以董皎為相國、羅恆、馬當為股肱,李奕、任調、李閎為親信,解思明為謀主。李壽本想向龔壯,授安車束帛以命為大師,然而龔壯拒絕,僅接收縞巾素帶,以師友之位自居。同時拔擢幽滯,授以顯位。並追尊李驤為獻帝、母昝氏為太后、妻阎氏為皇后、世子李勢為太子。

李壽稱帝後,有人狀告廣漢太守李乾與大臣串通,密謀廢帝。李壽命兒子李廣與大臣齊集殿前,將李乾徙為漢嘉太守。一次遇上狂風暴雨,震動大履門柱,李壽為此深自悔責,下命郡臣要盡忠進言,切切拘泥忌諱。

後來後趙石虎向李壽提議結盟出兵晉朝,事成後兩人並分天下,李壽大悅,先是大修船艦,嚴兵繕甲,又令吏卒準備充足糧草。繼而以尚書令馬當為六軍都督,準備以七萬人兵力,乘舟溯江而上。當船隊經過成都時,鼓聲震天,李壽登城檢閱,群臣趁機以國小眾寡,吳越、會稽路遠,不易成功為由出言阻止,尤其解思明更是切誎懇至,於是李壽便讓群臣力陳利害。龔壯誎曰﹕「陞下與胡人互通,是否會比與晉朝更好呢? 胡人素來是豺狼一樣的國家。晉朝被滅,才不得不北面事之。如果與他們爭奪天下,結果只會令強弱更加懸殊,昔日虞國、虢國(成語「假途滅虢」的典故)的教訓在前,希望陛下可以深思熟慮。」群臣都同意龔壯的進言,更叩頭泣誎,終使李壽放棄,士眾大喜,更連聲萬歲。

之後李壽派遣鎮東將軍李奕征討牂柯,太守謝恕據城堅守多日未能攻克,適逢李奕糧盡,因而撤兵。同時以太子李勢為大將軍、錄尚書事。

李壽繼承李雄,同樣為政寬儉,在篡位之初,亦未表現其欲望。某次李閎、王嘏從鄴城歸來,盛讚石季龍的宮殿華麗,鄴中戶口殷實。卻同時聽聞石季龍濫用刑法,王侯表現不遜,亦以殺罰懲戒,反而能夠控制各地邦域,令李壽相當羨慕,決心效法他。下臣每有小過,動輒處死以立威。又以都城空虛、鄉效戶口未至充實、工匠器械仍未滿盈為由,遷徙鄰郡戶有三名男丁以上的家戶到成都,又建造尚方御府,派遣各州郡能工巧匠以充實之,並廣修宮室、引水入城,極盡奢華,又擴充太學、建立宴殿等,令百姓疲於奔命,悲呼嗟嘆怨聲載道,以致人心思亂者,竟有十之八九。左僕射蔡興進誎阻止,李壽以其散播謗言為由,將他處死。右僕射李嶷因為經常直言忤逆意旨,李壽對他素有積怨,便假以他罪將他收監然後處死。

後來李壽患有重病,經常夢見李期、蔡興索命。解思明等復議再次尊奉皇室,李壽不從。李演亦由越雟上書,勸他歸正返本,放棄稱帝,復稱為王,李壽大怒殺之,以警告龔壯、解思明等。龔壯於是作詩七篇,假借應璩之口諷刺李壽,李壽便回應道﹕「有道是「反省詩詞便可知其意思」,如果這篇是今人所作,就是賢哲之話語; 假若是古人所作,便只是死去鬼魂的平常辭令!」

漢興六年(343年),最終在憂患之中病死,享年四十四歲。李壽在位五年,谥昭文皇帝,廟號中宗,葬安昌陵。

李壽為帝之初,好學愛士,即使庶民小兒也對他稱道不已。每次閱到良將賢相建功立業的事蹟時,沒有一次不會反覆誦讀,故能征伐四克,開闢千里疆土。未稱帝之前,相對李雄一心求上,李壽亦能盡誠於下,因此被稱為賢相。到即位之後,改立宗廟,以父李驤為漢始祖廟、李特、李雄為大成廟,又下旨强調與李期、李越並非同族,大凡期、越時定制,都有所改動。公卿之下,悉數任用自己的幕僚輔佐。李雄時的舊臣以及六郡士人,全部罷黜。而李壽相當仰慕漢武帝、魏明帝的所為,同時恥於聞說父兄之事,禁止上書者妄言前任教化功績,而只能提及李壽在位時的當世事功。

\subsubsection{汉兴}

\begin{longtable}{|>{\centering\scriptsize}m{2em}|>{\centering\scriptsize}m{1.3em}|>{\centering}m{8.8em}|}
  % \caption{秦王政}\
  \toprule
  \SimHei \normalsize 年数 & \SimHei \scriptsize 公元 & \SimHei 大事件 \tabularnewline
  % \midrule
  \endfirsthead
  \toprule
  \SimHei \normalsize 年数 & \SimHei \scriptsize 公元 & \SimHei 大事件 \tabularnewline
  \midrule
  \endhead
  \midrule
  元年 & 338 & \tabularnewline\hline
  二年 & 339 & \tabularnewline\hline
  三年 & 340 & \tabularnewline\hline
  四年 & 341 & \tabularnewline\hline
  五年 & 342 & \tabularnewline\hline
  六年 & 343 & \tabularnewline
  \bottomrule
\end{longtable}


%%% Local Variables:
%%% mode: latex
%%% TeX-engine: xetex
%%% TeX-master: "../../Main"
%%% End:

%% -*- coding: utf-8 -*-
%% Time-stamp: <Chen Wang: 2019-12-18 16:34:50>

\subsection{李势\tiny(343-347)}

\subsubsection{生平}

李势(4世紀-361年),字子仁,十六国成汉末主。李寿长子,母李氏。降晉後,封歸義侯,卒於建康,後世稱「後主」。無子。

初,李壽妻閻氏無子,李驤殺李鳳,為李壽納李鳳之女為妻,生李勢。李期愛李勢姿貌,拜他為翊軍將軍、漢王世子。李勢身長七尺九寸,腰帶十四圍,善於俯仰,時人異之。成漢漢興六年(343年),李壽死,李勢嗣偽位,赦其境內,改元曰太和。尊母閻氏為太后,妻李氏為皇后。

太史令韓皓奏熒惑守心,以過廟禮廢,勢命群臣議之。其相國董皎、侍中王嘏等以為景武昌業,獻文承基,至親不遠,無宜疏絕。勢更令祭特、雄,同號曰漢王。

李勢弟大將軍、漢王李廣以李勢無子,求為太弟,李勢不許。馬當、解思明以李勢兄弟不多,若有所廢,則益孤危,固勸李勢准許。李勢疑當等與李廣有叛謀,遣其太保李奕襲廣於涪城,命董皎收馬當、解思明二人斬殺,夷其三族。貶李廣為臨邛侯,李廣自殺。解思明有計謀,強作諫諍,馬當甚得人心。自此之後,無複紀綱及諫諍者。

李奕自晉壽舉兵反之,蜀人多有從奕者,眾至數萬。勢登城距戰。奕單騎突門,門者射而殺之,眾乃潰散。勢既誅奕,大赦境內,改年嘉寧。

初,蜀土無獠,至此,始從山而出,北至犍為,梓潼,布在山谷,十余萬落,不可禁制,大為百姓之患。勢既驕吝,而性愛財色,常殺人而取其妻,荒淫不恤國事。夷獠叛亂,軍守離缺,境宇日蹙。加之荒儉,性多忌害,誅殘大臣,刑獄濫加,人懷危懼。斥外父祖臣佐,親任左右小人,群小因行威福。又常居內,少見公卿。史官屢陳災譴,乃加董皎太師,以名位優之,實欲與分災眚。

晉永和二年(346年)末,晉大司馬桓溫率水軍伐勢。桓溫次青衣,李勢大發軍距守,又遣李福與昝堅等數千人從山陽趣合水距溫。謂溫從步道而上,諸將皆欲設伏於江南以待王師,昝堅不從,率諸軍從江北鴛鴦碕渡向犍為,而桓溫從山陽出江南,昝堅到犍為,方知與溫異道,乃回從沙頭津北渡。及昝堅至,溫已造成都之十裏陌,昝堅之兵眾自潰。桓溫至城下,縱火燒其大城諸門。李勢的兵眾惶懼,無複固志,其中書監王嘏、散騎常侍常璩等勸李勢投降。

李勢以問侍中馮孚,馮孚言:「昔吳漢征蜀,盡誅公孫氏。今晉下書,不赦諸李,雖降,恐無全理。」勢乃夜出東門,與昝堅走至晉壽(今四川广元),然後送降文于溫曰:「偽嘉寧二年三月十七日,略陽李勢叩頭死罪。伏惟大將軍節下,先人播流,恃險因釁,竊自汶、蜀。勢以暗弱,複統未緒,偷安荏苒,未能改圖。猥煩硃軒,踐冒險阻。將士狂愚,干犯天威。仰慚俯愧,精魂飛散,甘受斧鑕,以釁軍鼓。伏惟大晉,天網恢弘,澤及四海,恩過陽日。逼迫倉卒,自投草野。即日到白水城,謹遣私署散騎常侍王幼奉箋以聞,並敕州郡投戈釋杖。窮池之魚,待命漏刻。」勢尋輿櫬面縛軍門,溫解其縛,焚其櫬,遷勢及弟福、從兄權親族十余人于建康,封勢歸義侯。升平五年(361年),卒於建康。在位五年而敗。

《妒记》記載晉時宣武候桓溫平蜀後,娶了李勢的妹妹為妾。桓溫妻南康公主知道後大為妒忌,乃拔刃往李氏居所,準備砍人。結果看見李氏正在窗前梳頭,「姿貌端丽,徐徐结发,敛手向主,神色闲正,辞甚凄惋。」見美而生憐生愛,於是擲刀前抱之曰:「阿子,我見汝亦憐,何況老奴。」意思是說連我看了都會心動了,更何況是那老傢伙。成語「我見猶憐」因此而來。

\subsubsection{太和}

\begin{longtable}{|>{\centering\scriptsize}m{2em}|>{\centering\scriptsize}m{1.3em}|>{\centering}m{8.8em}|}
  % \caption{秦王政}\
  \toprule
  \SimHei \normalsize 年数 & \SimHei \scriptsize 公元 & \SimHei 大事件 \tabularnewline
  % \midrule
  \endfirsthead
  \toprule
  \SimHei \normalsize 年数 & \SimHei \scriptsize 公元 & \SimHei 大事件 \tabularnewline
  \midrule
  \endhead
  \midrule
  元年 & 344 & \tabularnewline\hline
  二年 & 345 & \tabularnewline\hline
  三年 & 346 & \tabularnewline
  \bottomrule
\end{longtable}

\subsubsection{嘉宁}

\begin{longtable}{|>{\centering\scriptsize}m{2em}|>{\centering\scriptsize}m{1.3em}|>{\centering}m{8.8em}|}
  % \caption{秦王政}\
  \toprule
  \SimHei \normalsize 年数 & \SimHei \scriptsize 公元 & \SimHei 大事件 \tabularnewline
  % \midrule
  \endfirsthead
  \toprule
  \SimHei \normalsize 年数 & \SimHei \scriptsize 公元 & \SimHei 大事件 \tabularnewline
  \midrule
  \endhead
  \midrule
  元年 & 346 & \tabularnewline\hline
  二年 & 347 & \tabularnewline
  \bottomrule
\end{longtable}

\subsubsection{范賁生平}

范賁(3世紀?-349年),中國十六國初期東晉境內蜀地(今中國四川省)的民變領袖之一,是成漢丞相范長生之子。曾任成漢的侍中一職,318年范長生去世後,接任丞相。

范長生博學多聞,年近百歲才去世,而被蜀地之人敬若神明。347年,成漢被東晉所滅,成漢將領因此推范賁為帝,根據史書記載,范賁「以妖異惑眾」,因此蜀地很多人歸附。

349年,東晉益州刺史周撫、龍驤將軍朱燾攻擊范賁,范賁被殺,遂平定益州。


%%% Local Variables:
%%% mode: latex
%%% TeX-engine: xetex
%%% TeX-master: "../../Main"
%%% End:


%%% Local Variables:
%%% mode: latex
%%% TeX-engine: xetex
%%% TeX-master: "../../Main"
%%% End:
