%% -*- coding: utf-8 -*-
%% Time-stamp: <Chen Wang: 2019-12-18 16:34:50>

\subsection{李势\tiny(343-347)}

\subsubsection{生平}

李势(4世紀-361年),字子仁,十六国成汉末主。李寿长子,母李氏。降晉後,封歸義侯,卒於建康,後世稱「後主」。無子。

初,李壽妻閻氏無子,李驤殺李鳳,為李壽納李鳳之女為妻,生李勢。李期愛李勢姿貌,拜他為翊軍將軍、漢王世子。李勢身長七尺九寸,腰帶十四圍,善於俯仰,時人異之。成漢漢興六年(343年),李壽死,李勢嗣偽位,赦其境內,改元曰太和。尊母閻氏為太后,妻李氏為皇后。

太史令韓皓奏熒惑守心,以過廟禮廢,勢命群臣議之。其相國董皎、侍中王嘏等以為景武昌業,獻文承基,至親不遠,無宜疏絕。勢更令祭特、雄,同號曰漢王。

李勢弟大將軍、漢王李廣以李勢無子,求為太弟,李勢不許。馬當、解思明以李勢兄弟不多,若有所廢,則益孤危,固勸李勢准許。李勢疑當等與李廣有叛謀,遣其太保李奕襲廣於涪城,命董皎收馬當、解思明二人斬殺,夷其三族。貶李廣為臨邛侯,李廣自殺。解思明有計謀,強作諫諍,馬當甚得人心。自此之後,無複紀綱及諫諍者。

李奕自晉壽舉兵反之,蜀人多有從奕者,眾至數萬。勢登城距戰。奕單騎突門,門者射而殺之,眾乃潰散。勢既誅奕,大赦境內,改年嘉寧。

初,蜀土無獠,至此,始從山而出,北至犍為,梓潼,布在山谷,十余萬落,不可禁制,大為百姓之患。勢既驕吝,而性愛財色,常殺人而取其妻,荒淫不恤國事。夷獠叛亂,軍守離缺,境宇日蹙。加之荒儉,性多忌害,誅殘大臣,刑獄濫加,人懷危懼。斥外父祖臣佐,親任左右小人,群小因行威福。又常居內,少見公卿。史官屢陳災譴,乃加董皎太師,以名位優之,實欲與分災眚。

晉永和二年(346年)末,晉大司馬桓溫率水軍伐勢。桓溫次青衣,李勢大發軍距守,又遣李福與昝堅等數千人從山陽趣合水距溫。謂溫從步道而上,諸將皆欲設伏於江南以待王師,昝堅不從,率諸軍從江北鴛鴦碕渡向犍為,而桓溫從山陽出江南,昝堅到犍為,方知與溫異道,乃回從沙頭津北渡。及昝堅至,溫已造成都之十裏陌,昝堅之兵眾自潰。桓溫至城下,縱火燒其大城諸門。李勢的兵眾惶懼,無複固志,其中書監王嘏、散騎常侍常璩等勸李勢投降。

李勢以問侍中馮孚,馮孚言:「昔吳漢征蜀,盡誅公孫氏。今晉下書,不赦諸李,雖降,恐無全理。」勢乃夜出東門,與昝堅走至晉壽(今四川广元),然後送降文于溫曰:「偽嘉寧二年三月十七日,略陽李勢叩頭死罪。伏惟大將軍節下,先人播流,恃險因釁,竊自汶、蜀。勢以暗弱,複統未緒,偷安荏苒,未能改圖。猥煩硃軒,踐冒險阻。將士狂愚,干犯天威。仰慚俯愧,精魂飛散,甘受斧鑕,以釁軍鼓。伏惟大晉,天網恢弘,澤及四海,恩過陽日。逼迫倉卒,自投草野。即日到白水城,謹遣私署散騎常侍王幼奉箋以聞,並敕州郡投戈釋杖。窮池之魚,待命漏刻。」勢尋輿櫬面縛軍門,溫解其縛,焚其櫬,遷勢及弟福、從兄權親族十余人于建康,封勢歸義侯。升平五年(361年),卒於建康。在位五年而敗。

《妒记》記載晉時宣武候桓溫平蜀後,娶了李勢的妹妹為妾。桓溫妻南康公主知道後大為妒忌,乃拔刃往李氏居所,準備砍人。結果看見李氏正在窗前梳頭,「姿貌端丽,徐徐结发,敛手向主,神色闲正,辞甚凄惋。」見美而生憐生愛,於是擲刀前抱之曰:「阿子,我見汝亦憐,何況老奴。」意思是說連我看了都會心動了,更何況是那老傢伙。成語「我見猶憐」因此而來。

\subsubsection{太和}

\begin{longtable}{|>{\centering\scriptsize}m{2em}|>{\centering\scriptsize}m{1.3em}|>{\centering}m{8.8em}|}
  % \caption{秦王政}\
  \toprule
  \SimHei \normalsize 年数 & \SimHei \scriptsize 公元 & \SimHei 大事件 \tabularnewline
  % \midrule
  \endfirsthead
  \toprule
  \SimHei \normalsize 年数 & \SimHei \scriptsize 公元 & \SimHei 大事件 \tabularnewline
  \midrule
  \endhead
  \midrule
  元年 & 344 & \tabularnewline\hline
  二年 & 345 & \tabularnewline\hline
  三年 & 346 & \tabularnewline
  \bottomrule
\end{longtable}

\subsubsection{嘉宁}

\begin{longtable}{|>{\centering\scriptsize}m{2em}|>{\centering\scriptsize}m{1.3em}|>{\centering}m{8.8em}|}
  % \caption{秦王政}\
  \toprule
  \SimHei \normalsize 年数 & \SimHei \scriptsize 公元 & \SimHei 大事件 \tabularnewline
  % \midrule
  \endfirsthead
  \toprule
  \SimHei \normalsize 年数 & \SimHei \scriptsize 公元 & \SimHei 大事件 \tabularnewline
  \midrule
  \endhead
  \midrule
  元年 & 346 & \tabularnewline\hline
  二年 & 347 & \tabularnewline
  \bottomrule
\end{longtable}

\subsubsection{范賁生平}

范賁(3世紀?-349年),中國十六國初期東晉境內蜀地(今中國四川省)的民變領袖之一,是成漢丞相范長生之子。曾任成漢的侍中一職,318年范長生去世後,接任丞相。

范長生博學多聞,年近百歲才去世,而被蜀地之人敬若神明。347年,成漢被東晉所滅,成漢將領因此推范賁為帝,根據史書記載,范賁「以妖異惑眾」,因此蜀地很多人歸附。

349年,東晉益州刺史周撫、龍驤將軍朱燾攻擊范賁,范賁被殺,遂平定益州。


%%% Local Variables:
%%% mode: latex
%%% TeX-engine: xetex
%%% TeX-master: "../../Main"
%%% End:
