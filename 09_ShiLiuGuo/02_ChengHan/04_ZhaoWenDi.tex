%% -*- coding: utf-8 -*-
%% Time-stamp: <Chen Wang: 2021-11-01 11:53:07>

\subsection{昭文帝李寿\tiny(338-343)}

\subsubsection{生平}

汉昭文帝李寿(300年-343年),字武考,十六国时期成汉政权的皇帝。为李特之弟李骧少子。

338年即位后改国号为“汉”。343年病死。

李壽天生聰敏好學,少尚禮容,在李氏諸子中相當突出,受到李雄欣賞,認為他足以擔當大任,乃授以前將軍,統領巴蜀軍事,彼遷征東將軍,當時年僅十九歲。在任期間以處士譙秀為謀主,對其言聽計從,令他在巴蜀威德日隆。

李驤卒,李壽再先後升遷為大將軍、大都督、侍中、並封扶風公、錄尚書事。在出征寧州時,圍攻百餘日,最終悉數平定諸郡,李雄大悅,再加封建寧王。

李雄卒,受遺命輔政,李期繼位,改封漢王,兼任梁州刺史,獲賜封梁州五郡。

自此,李壽威名遠播,卻同時深為李越、景騫等所忌憚,令李壽深為擔憂。在暫代李玝屯田涪水期間,每次朝覲日期到來,往趁以邊景賊寇橫行,不可放鬆戒備離開而推卻。同時李壽又因為李期、李越兄弟等十餘人年紀漸長,又擁有精兵,擔心不能自全,便數次欲聘得龔壯為其效命。龔壯雖然不答應,不過仍多次與李壽見面。時值岷山崩塌,江水因此枯竭,李壽認為此乃上天預示災劫,因而非常厭惡,便問龔壯自安之法。龔壯的父親及叔父,被李特殺害,為了假借李壽之手報仇,便向李壽提出起兵自立以自保的建議,最終獲得李壽採納,之後便暗中與長史略陽羅恆、巴西解思明共同謀奪首都成都,並得數千人加入。李壽軍起兵突襲成都,將其攻克,縱兵擄掠,甚至姦污李雄女兒及李氏諸婦,並將之殘殺。羅恆、解思明、李奕、王利等人乃勸李壽自稱鎮西將軍、益州牧、成都王,並向晉朝稱藩。

成玉恆四年(338年),大臣任調、司馬蔡興、侍中李艷以及張烈等勸李壽自立。李壽命巫師卜卦,得出「可當數年天子」的預示,任調大喜,進言「一日尚且滿足,何況數年!」(一日尚為足,而況數年乎!)李壽以「有道是「早上聽到警世的道理,就算當晚要死亦無悔無憾」(朝聞道,夕死可矣),任調的進言,實在是上乘之策!」乃自稱為帝,舉國大赦,並改元漢興,以董皎為相國、羅恆、馬當為股肱,李奕、任調、李閎為親信,解思明為謀主。李壽本想向龔壯,授安車束帛以命為大師,然而龔壯拒絕,僅接收縞巾素帶,以師友之位自居。同時拔擢幽滯,授以顯位。並追尊李驤為獻帝、母昝氏為太后、妻阎氏為皇后、世子李勢為太子。

李壽稱帝後,有人狀告廣漢太守李乾與大臣串通,密謀廢帝。李壽命兒子李廣與大臣齊集殿前,將李乾徙為漢嘉太守。一次遇上狂風暴雨,震動大履門柱,李壽為此深自悔責,下命郡臣要盡忠進言,切切拘泥忌諱。

後來後趙石虎向李壽提議結盟出兵晉朝,事成後兩人並分天下,李壽大悅,先是大修船艦,嚴兵繕甲,又令吏卒準備充足糧草。繼而以尚書令馬當為六軍都督,準備以七萬人兵力,乘舟溯江而上。當船隊經過成都時,鼓聲震天,李壽登城檢閱,群臣趁機以國小眾寡,吳越、會稽路遠,不易成功為由出言阻止,尤其解思明更是切誎懇至,於是李壽便讓群臣力陳利害。龔壯誎曰﹕「陞下與胡人互通,是否會比與晉朝更好呢? 胡人素來是豺狼一樣的國家。晉朝被滅,才不得不北面事之。如果與他們爭奪天下,結果只會令強弱更加懸殊,昔日虞國、虢國(成語「假途滅虢」的典故)的教訓在前,希望陛下可以深思熟慮。」群臣都同意龔壯的進言,更叩頭泣誎,終使李壽放棄,士眾大喜,更連聲萬歲。

之後李壽派遣鎮東將軍李奕征討牂柯,太守謝恕據城堅守多日未能攻克,適逢李奕糧盡,因而撤兵。同時以太子李勢為大將軍、錄尚書事。

李壽繼承李雄,同樣為政寬儉,在篡位之初,亦未表現其欲望。某次李閎、王嘏從鄴城歸來,盛讚石季龍的宮殿華麗,鄴中戶口殷實。卻同時聽聞石季龍濫用刑法,王侯表現不遜,亦以殺罰懲戒,反而能夠控制各地邦域,令李壽相當羨慕,決心效法他。下臣每有小過,動輒處死以立威。又以都城空虛、鄉效戶口未至充實、工匠器械仍未滿盈為由,遷徙鄰郡戶有三名男丁以上的家戶到成都,又建造尚方御府,派遣各州郡能工巧匠以充實之,並廣修宮室、引水入城,極盡奢華,又擴充太學、建立宴殿等,令百姓疲於奔命,悲呼嗟嘆怨聲載道,以致人心思亂者,竟有十之八九。左僕射蔡興進誎阻止,李壽以其散播謗言為由,將他處死。右僕射李嶷因為經常直言忤逆意旨,李壽對他素有積怨,便假以他罪將他收監然後處死。

後來李壽患有重病,經常夢見李期、蔡興索命。解思明等復議再次尊奉皇室,李壽不從。李演亦由越雟上書,勸他歸正返本,放棄稱帝,復稱為王,李壽大怒殺之,以警告龔壯、解思明等。龔壯於是作詩七篇,假借應璩之口諷刺李壽,李壽便回應道﹕「有道是「反省詩詞便可知其意思」,如果這篇是今人所作,就是賢哲之話語; 假若是古人所作,便只是死去鬼魂的平常辭令!」

漢興六年(343年),最終在憂患之中病死,享年四十四歲。李壽在位五年,谥昭文皇帝,廟號中宗,葬安昌陵。

李壽為帝之初,好學愛士,即使庶民小兒也對他稱道不已。每次閱到良將賢相建功立業的事蹟時,沒有一次不會反覆誦讀,故能征伐四克,開闢千里疆土。未稱帝之前,相對李雄一心求上,李壽亦能盡誠於下,因此被稱為賢相。到即位之後,改立宗廟,以父李驤為漢始祖廟、李特、李雄為大成廟,又下旨强調與李期、李越並非同族,大凡期、越時定制,都有所改動。公卿之下,悉數任用自己的幕僚輔佐。李雄時的舊臣以及六郡士人,全部罷黜。而李壽相當仰慕漢武帝、魏明帝的所為,同時恥於聞說父兄之事,禁止上書者妄言前任教化功績,而只能提及李壽在位時的當世事功。

\subsubsection{汉兴}

\begin{longtable}{|>{\centering\scriptsize}m{2em}|>{\centering\scriptsize}m{1.3em}|>{\centering}m{8.8em}|}
  % \caption{秦王政}\
  \toprule
  \SimHei \normalsize 年数 & \SimHei \scriptsize 公元 & \SimHei 大事件 \tabularnewline
  % \midrule
  \endfirsthead
  \toprule
  \SimHei \normalsize 年数 & \SimHei \scriptsize 公元 & \SimHei 大事件 \tabularnewline
  \midrule
  \endhead
  \midrule
  元年 & 338 & \tabularnewline\hline
  二年 & 339 & \tabularnewline\hline
  三年 & 340 & \tabularnewline\hline
  四年 & 341 & \tabularnewline\hline
  五年 & 342 & \tabularnewline\hline
  六年 & 343 & \tabularnewline
  \bottomrule
\end{longtable}


%%% Local Variables:
%%% mode: latex
%%% TeX-engine: xetex
%%% TeX-master: "../../Main"
%%% End:
