%% -*- coding: utf-8 -*-
%% Time-stamp: <Chen Wang: 2021-11-01 11:55:43>

\subsection{义阳王石鑒\tiny(349-350)}

\subsubsection{少帝石世生平}

石世(339年-349年),字元安,十六國時期後趙國君主,後世稱「少帝」,為石虎之子。母為前趙帝劉曜幼女安定公主,後趙太和二年(329年),前趙被後趙所滅,石虎將年僅12歲的安定公主強占為妾,十年後安定公主生了石世。石虎在天王位時,石世被封為齊公,安定公主封為昭儀。

後趙建武十三年(348年),石虎廢殺了太子石宣之後,受石世之母昭儀劉氏及她的死黨將軍張豺的教唆鼓動,將劉氏立為皇后,年僅10歲的石世立為太子。次年(349年),石虎正式稱帝,並改元太寧。不久,石虎去世,石世遂即帝位,然而大權皆握在劉太后及張豺之手。

彭城王石遵得知石虎去世後,立即率軍攻回都城鄴城(今河北臨漳縣),殺張豺。數日後,石遵自即帝位,石世被改封為譙王,劉太后被廢為太妃,石世在位僅33日。不久,石世與劉太妃皆被殺。

\subsubsection{彭城王石遵生平}

石遵(?-349年),字大祗,十六國時期後趙皇帝,為石虎第九子,石世之兄,母為鄭櫻桃。後趙建平三年(333年),後趙帝石勒去世,石虎掌控大權,石遵當時被封為齊王。建武三年(337年),石虎改稱天王後,被降封為彭城公。太寧元年(349年),石虎稱帝後,再被進封為彭城王。

太尉张举曾建议石虎立石遵或燕公石斌為太子,然而因昭儀劉氏及戎昭将军張豺從中作梗,石虎遂立劉氏之子石世為太子。太寧元年(349年),石虎病重,石遵被任命為大將軍,鎮守關右。石遵从幽州来朝,被打发走,石虎知道后说“恨不见之”。

不久,石虎去世,石世即位,大權握於劉太后及張豺之手。石遵與姚弋仲、蒲洪、石閔等人商量後決定反擊,遂以石閔為前鋒,攻打都城鄴(今河北臨漳縣),不久,鄴城陷,劉太后不得已只好任命石遵為丞相、领大司马、大都督中外诸军、录尚书事,加黄钺、九锡,增封十郡。數日後,石遵假刘太后令廢石世、立石遵为帝,假装再三辞让后在群臣劝进下自登帝位于太武前殿。封石世为谯王,邑万户,待以不臣之礼,废刘太后为太妃,不久皆杀之。石遵兄沛王石冲讨伐石遵,石遵派将军王擢骑马以书信说和不成,派石闵、司空李农击败石冲于平棘,在元氏俘获石冲并赐死。

石遵可能没有儿子,當初在謀反前,曾答應事成後以石閔為太子,可是等到石遵登帝位後,太子卻是石遵之姪石衍,因此石閔頗為不滿,有反叛之意。經過旁人提醒,石遵遂召其兄石鑒、弟石苞與母親鄭櫻桃等人商議,不料會後卻被石鑒出賣,將此事告知石閔。不久,石閔即率軍入宮,派将军苏彦、周成率领披甲士兵三千人去南台的如意观抓石遵。石遵正在和女人弹棋,问周成:“造反的是谁?”周成说:“义阳王石鉴当立。”石遵说:“我尚且如此,石鉴又能支撑多长时间!”被殺,在位僅183日。

\subsubsection{义阳王石鑒生平}

石鑒(?-350年),字大郎,一作大朗,十六國時期後趙國君主,為石虎第三子,石遵、石世之兄。後趙建平三年(333年),後趙帝石勒去世,石虎掌控大權,石鑒當時被封為代王。建武三年(337年),石虎改稱天王後,被降封為義陽公。

建武五年(339年)九月,东晋征西将军庾亮镇武昌,让豫州刺史毛宝、西阳太守樊峻以一万精兵戍守邾城。石虎厌恶晋军如此动向,以夔安为大都督,率石鉴、养孙石闵、李农、张贺度、李菟五将军及兵五万人攻打荆、扬北境,以二万骑攻邾城。张贺度攻陷邾城,杀死六千人,又败毛宝于邾西,杀死万余人。赵军进犯江夏、义阳,毛宝、樊峻及东晋义阳太守郑进皆死。夔安等进围石城,被竟陵太守李阳所破才退兵。

太寧元年(349年),石虎稱帝後,再被進封為義陽王。

石鑒在鎮守關中的時候,賦役繁重,文武官員只要頭髮長得比較長,就會被拔下來做帽帶,有剩下的會給宮女,曾因為這種荒唐的行徑,被石虎召回都城鄴城(今河北臨漳縣)。

太寧元年(349年),石遵廢皇帝石世,自登帝位,石鑒被命為侍中、太傅。石遵因石閔有叛變之意,召两位兄弟石鑒、乐平王石苞與太后鄭櫻桃等人商議,不料會後石鑒出賣其他人,將此事告知石閔。不久,石閔即率軍入宮,殺石遵,石鑒因此被擁立為帝。石遵被杀时说:“我尚且如此,石鉴能长久吗?”

然而石鑒登位後,處處受制於大將軍石閔,於是派石苞和将军李松、张才暗殺之,然而卻事敗,他装作自己不知情,杀死石苞三人;后又鼓励将军孙伏都攻打石闵,不果,又对石闵说孙伏都谋反,命石闵讨灭。石閔知道石鑒有殺己之意,遂頒殺胡令,被殺的人共有20餘萬;并软禁石鉴于御龙观,派尚书王简、少府王郁率数千人看守,用绳子把食物吊给他。

次年(350年),完全控制國政的石閔將後趙國號改為魏(衛),石閔也将包括自己在内的后赵皇族改姓为李,並改年號為青龍。不久,石鑒為求擺脫控制,遂趁李閔外出作戰,秘密派宦官告知在外的將軍抚军将军张沈等,命他们趁虛攻都城鄴城,但宦官告知李閔此事,李閔因而回軍,石鑒遂被誅殺,在位僅103日。

\subsubsection{青龙}

\begin{longtable}{|>{\centering\scriptsize}m{2em}|>{\centering\scriptsize}m{1.3em}|>{\centering}m{8.8em}|}
  % \caption{秦王政}\
  \toprule
  \SimHei \normalsize 年数 & \SimHei \scriptsize 公元 & \SimHei 大事件 \tabularnewline
  % \midrule
  \endfirsthead
  \toprule
  \SimHei \normalsize 年数 & \SimHei \scriptsize 公元 & \SimHei 大事件 \tabularnewline
  \midrule
  \endhead
  \midrule
  元年 & 350 & \tabularnewline
  \bottomrule
\end{longtable}


%%% Local Variables:
%%% mode: latex
%%% TeX-engine: xetex
%%% TeX-master: "../../Main"
%%% End:
