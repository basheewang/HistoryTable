%% -*- coding: utf-8 -*-
%% Time-stamp: <Chen Wang: 2019-12-18 17:41:04>

\subsection{石弘\tiny(333-334)}

\subsubsection{生平}

石弘(314年-335年),字大雅,是中國五胡十六國時代後趙的君王。上黨武鄉(今山西榆社)人,后赵明帝石勒二子,母程氏。

史載石弘「幼有孝行,以恭謹自守」,受经于杜嘏,诵律于续咸。石勒觉得他不似将门之子,派刘征、任播授以兵书,王阳教之击刺。石勒病重时,中山王石虎与石弘、中常侍严震在宫中侍候,石虎矫诏断绝内外消息。建平四年(333年)九月,石勒一死,石弘繼位,立嫡母劉氏為皇太后。石虎下達第一個“詔令”,將石弘舅父右光祿大夫程遐、中書令徐光論罪誅斬,拜石虎為丞相、魏王、大單于,加九錫,以魏郡等十三郡為邑。石弘恐懼丞相石虎,欲讓位於石虎。石虎拒絕:“君薨而世子立,臣安敢亂之!”遂即位,拜石虎为丞相。

刘太后与石勒养子彭城王石堪谋除石虎,擁皇弟南陽王石恢為盟主。石堪單騎出逃,直奔兗州。到達廩丘時,因事機不密,逮送至襄國,被活活烤死。劉太后被石虎发现参与其中,遭废黜弒害,石虎改尊石弘生母程氏为皇太后。河東王石生在關中起兵,石朗在洛陽起兵,聲言滅石虎。石虎擒下石朗,他先砍掉石朗的雙腳,再斬首。長安一戰,石虎大敗,“枕尸三百余里”,此時石生同盟的鮮卑人竟然反叛,石虎重振軍勢,石生被部下斬首,獻給石虎。延熙元年(334年)十月石弘持玺绶向石虎表明願意禅位。石虎说:“天下人自当有议,何为自论此也!”意思是只能自己逼石弘退位,而不能接受石弘禅位。石弘哭着回宫对程太后说:“先帝真要灭种了!”不久石虎称石弘居丧不孝,废为海阳王,与程太后及弟秦王石宏、石恢一同幽禁崇训宫,不久皆殺之。

\subsubsection{延熙}

\begin{longtable}{|>{\centering\scriptsize}m{2em}|>{\centering\scriptsize}m{1.3em}|>{\centering}m{8.8em}|}
  % \caption{秦王政}\
  \toprule
  \SimHei \normalsize 年数 & \SimHei \scriptsize 公元 & \SimHei 大事件 \tabularnewline
  % \midrule
  \endfirsthead
  \toprule
  \SimHei \normalsize 年数 & \SimHei \scriptsize 公元 & \SimHei 大事件 \tabularnewline
  \midrule
  \endhead
  \midrule
  元年 & 334 & \tabularnewline
  \bottomrule
\end{longtable}


%%% Local Variables:
%%% mode: latex
%%% TeX-engine: xetex
%%% TeX-master: "../../Main"
%%% End:
