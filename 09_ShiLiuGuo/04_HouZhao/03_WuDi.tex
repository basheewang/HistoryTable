%% -*- coding: utf-8 -*-
%% Time-stamp: <Chen Wang: 2019-12-18 17:43:31>

\subsection{武帝\tiny(334-349)}

\subsubsection{孝帝生平}

石寇覓(3世紀-?),是後趙武帝石虎的父親。他早逝,因此石虎被石勒的父親石周曷朱收养,所以又有人稱石虎是石勒的弟弟。

石虎稱帝後,追封他為皇帝,諡號孝皇帝,廟號太宗。

\subsubsection{武帝生平}

趙武帝石虎(295年-349年5月26日),字季龍,上黨武鄉(今山西榆社)人。中國五胡十六國時代中,後趙的第三位皇帝。廟號太祖,諡號武帝。石虎是後趙開國君主石勒的侄兒。石虎生性殘忍,發家前,不僅用殘酷的手段先後殺死兩位妻子,即使在軍隊中如果遇到與他一樣強健的戰士,他會以打獵戲鬥為由,借機將對手殺死,以解心頭之快;戰鬥中,對俘獲的俘虜,不分好壞,不分男女一律坑殺,很少有俘虜生還。

333年,石勒駕崩,其皇位由兒子石弘繼承。因石虎掌握兵權勢大,石勒妻刘太后與養子彭城王石堪擁立石勒子南陽王石恢欲舉兵反對石虎,不幸事洩,劉太后被殺,石堪被捕活活烤死,石恢被召回,咸康元年(334年)十月石弘持璽綬向石虎表明願意禪位,石虎拒绝。十一月,石虎称居摄赵天王,石弘被廢為海陽王,同年石虎殺海陽王石弘、弘母程氏、石弘弟秦王石宏、南陽王石恢。至335年,其首都由襄國(今中國河北邢台)遷至鄴(今河北邯郸市臨漳县城西南20公里邺城遗址),並特地派人到洛陽將九龍、翁仲、銅駝、飛廉轉運到鄴裝點宮殿。337年4月11日(二月辛巳),石虎称大赵天王,349年2月4日(正月初一辛未朔)正式即皇帝位。同年5月26日(四月己巳),患病而死,随后,他的儿子争夺皇位,后赵很快灭亡。石虎在位期間,表現了其殘暴好色的一面,如史書載石虎曾經下達過一條命令:全國二十歲以下、十三歲以上的女子,不論是否嫁人,都要做好準備隨時成為他後宮佳麗中的一員,「百姓妻有美色,豪勢因而脅之,率多自殺」,因此被評為五胡十六國中的暴君。

生性殘暴的石虎,少年時喜歡用彈弓打人為樂。十八歲時,由於其武藝超凡且勇猛過人,因此受到石勒的寵信,被封為征虜將軍。石勒其後又為石虎納聘將軍郭榮的妹妹為妻,但石虎心儀的是當時的雜技名角鄭櫻桃。於是便把郭氏殺死,而後迎娶鄭氏。之後,石虎又娶了崔氏,但崔氏最後因鄭氏的挑撥而死於石虎手中。

在軍中,凡是比石虎有才藝或有武藝的,石虎就會設法把他們殺死,死於他手上的人不可計數。石虎是好殺的人,每次攻下一座城後,不論男女都一律殺死。一次,石虎攻下青州後又下令屠城。此次血腥屠城,僅餘七百多人保全性命。

太和三年(330年)二月,石勒称大赵天王,行皇帝事;以妃刘氏为王后,世子石弘为皇太子,程遐为右仆射、领吏部尚书。中山王石虎怒,秘密对长子齐王石邃说:“我亲冒矢石随主上征战二十余年,是成大赵之业者,应该做大单于,主上却授予‘黄吻婢儿’,想起来就令人气塞,不能寝食!待主上晏驾之后,我不会给他留种。”

石勒臨終前,石虎威迫太子石弘把曾勸石勒除掉自己的大臣程遐和徐光逮捕入獄并杀死。又命兒子石邃率兵入宿衛,文武百官害怕不已,太子石弘也嚇得連忙對石虎說道自己不是治天下的人材,石虎才是真正的天子。但石虎明白石勒屍骨未寒,就這樣強登上皇帝只會眾叛親離,並受後世人的唾罵。因此寧願有點耐性,演齣曹操的「挾天子以令諸侯」的戲,由這位太子登位。

石弘坐上寶座後,成為了傀儡皇帝。石弘登基後便被石虎所逼,将程遐、徐光论罪诛斩,封石虎為丞相、魏王、大單于,再封土地,封邦建土。而他的三名兒子都被封為擁有軍權的職位,至於他的親人和親信都放排在有大權的職位上,而之前石勒的文武百官就放置在毫無權力的閑職上。這時後趙已真正的形成「挾天子以令諸侯」的局面。刘太后与石勒养子石堪合谋起兵拥戴石弘的弟弟石恢为盟主,石堪兵败被杀,石恢被征召回京,刘太后被石虎废黜杀害。石弘生母程氏被尊为太后,也没有实权。延熙元年(334年)十月石弘持玺绶向石虎表明愿意禅位。石虎说:“天下人自当有议,何为自论此也!”意思是只能自己逼石弘退位,而不能接受石弘禅位。石弘哭着回宫对程太后说:“先帝真要灭种了!”不久石虎称石弘居丧不孝,废为海阳王,自称天王,並把石弘、程太后和石弘的弟弟石宏、石恢都幽禁于崇训宫,旋即殺死他們。

石虎稱天王後,石邃為太子(之前为魏太子),並開始他極為奢侈的統治。石虎不顧人民負擔到處征殺,使人民的兵役和力役負擔相當重大,他又下令凡是有免兵役特權的家族,五丁取二,四丁取其二,而沒有特權的家族則所有丁壯都需服役。為了攻打東晉,在全國征調士兵的物品:每五人出車一乘、牛兩頭、米穀五十斛、絹十份,不交者格殺勿論。無數的百姓為了安全,不得不把自己的子女賣掉。

後趙建武二年(336年),石虎為了裝飾鄴城,令牙門將張彌把洛陽的鐘虞、九龍、翁仲、銅駝、飛廉等相生物運到去鄴城。在運送途中,一隻鐘虞沒入了黃河,於是張彌便下令三百多名人潛到水中,把鐘虞繫上繩,再利用百多頭牛和許多架轆轤把鐘虞拉上來,之後就地造了可裝萬斛的大船,把這些相生運過黃河。其後又製造了特大的車子以把相生運送到鄴城,這次的行動單是運送就足足用了人民千千萬萬的勞力和血汗了。

在鄴城以西三里,有石虎所建的桑梓苑,苑內臨漳水修建了很多座豪華的宮殿,下令从民间强行掠夺十三岁至二十岁的女子三万余人。仅在345年一年间,各郡县官吏为搜罗美女上交差事,公然抢掠貌美的有夫之妇九千余人,不忍受夺妻之辱而反抗的男人均遭残杀,被夺女子为避免受辱也大多自杀,一大批家庭夫妻离散,家破人亡。但石虎征集女人倒不完全是好色,石虎内置女官十有八等,教宫人星占及马步射。置女太史于灵台,仰观灾祥,以考外太史之虚实(《晋书·石季龙载记》)。石虎还鉴于东汉太监专权的危害,不信任太监,因此宫中没有太监,相关职务只能由女人充当。苑內養有奇珍異獸,石虎經常在此遊玩設宴。從襄國至鄴城的二百里內,每隔四十里使建一行宮,每宮都有一位夫人,數十位的侍婢居住,由黃門官守門。

而在浴室上,更是別出心裁:在皇后浴室中,門窗都是由木刻成的鏤孔圖案,石虎就是在這兒和皇后梳洗。而每年的4月8日,在這裏精工製造的九龍吐水浴太子之像。在太武殿前,溝的中間有多層以紗等的「過濾器」。

「鳳詔」也是石虎的發明之一,石虎處理政事時會和皇后一起坐在高高在上的樓觀上,並用五色紙上寫下詔書,把詔書放在一只由木雕刻成、外施漆畫、金腿的「鳳凰」口中。金鳳凰繫在轆轤牽引的繩上。當下詔時,待人把轆轤搖動,「鳳凰」就像從天空飛下來般,大臣們都要跪下接詔。

每隔不久,石虎便會大會群臣,每次都頭戴通天冠、身佩玉璽、循周禮的規定禮樂一番,然後觀賞雜技表演,群臣大會幾乎都有美酒佳釀給自己和群臣所飲用。殿上掛著了大鐵燈一百二十支。在燈下有數千戴金銀佩飾的宮女和石虎觀看表演。在殿外,三十部鼓吹同時演奏,鼓樂震天,場面極為震撼。

石虎好射獵,但因體胖而無法騎馬,因而改為用獵輦。而他的獵輦裝有豪華的華蓋羽葆,由二十人推行,座下有轉軸裝置,可以根據獵物的所在地轉動。在出獵時,石虎會戴上由金鏤織成的合歡帽、穿上合歡褲,手拿著弓箭。而石虎為了方便行獵,於是把黃河以北的大片良田為獵區,派御史監督,规定除自己外有敢在獵區獵獸者处死。而这“犯兽”的刑法,又被各官员用来欺压百姓,若百姓家有美女或好的牛马等家畜,官员要求不给,就诬陷其“犯獸”,因此被判死刑者甚多。

石虎像他伯父石勒一样崇拜大和尚佛圖澄,石勒因信佛圖澄之言而減少了很多殺虐。有次石虎向佛圖澄問甚麼是佛法,佛圖澄只說了四字:「佛法不殺」.石虎沒有聽取佛圖澄的勸告,後來倒是聽了一個叫吳進的假和尚說胡人的氣數已衰,而晉人的氣數開始恢復,一定要苦役晉人才能壓著他們的氣數。結果石虎下令強徵鄴城附近各郡的男女百姓十六萬多人、車十萬乘在鄴城東修華林苑,並圍苑建數十里的長牆。

在中國歷史上還記載著石虎父子的相互殘殺。

事緣石虎兒子石邃不滿父親寵愛其餘的兩個兒子石宣和石韜,漸漸地,這種不滿轉化為仇恨,對父亲石虎恨之入骨,恨不得弒父奪位。石虎得知後,把石邃的手下李顏捉來審問,李顏嚇得不知如何是好,便一五一十地都事情告訴石虎:石邃密謀殺石宣和弒石虎奪位。石虎得知後把李顏及其家人三十多人斬首處死,再把石邃幽禁於東宮。石邃被幽禁後仍然目中無人,石虎一怒之下,下令把石邃和他的妻子、家人殺死,再塞進同一口棺材內,同一時間又把石邃的黨羽二百多人殺死。

石邃死後,石宣為皇太子,石宣之母杜昭儀為天王皇后,鄭櫻桃廢為東海太妃。同时又让石韬掌握军政大权,打算让石宣和石韬之间达成一定的平衡。结果却引发新一轮内讧。

到了其後,石宣因不滿其父石虎較寵愛石韜而要除掉石韜。不久之後,兩兄弟經常發生衝突,石宣於是把石韜砍掉手足、雙眼刺爛、破肚慘死。石宣並計劃在石韜的喪禮上弒父,以奪皇位。

石虎得知愛兒石韜死了,昏迷了好一段時間,他本想出席兒子的喪禮,幸而大臣提醒,沒有出席喪禮。後來,石虎得到知情人的報告,得知皇太子石宣殺了石韜。憤怒到極點的石虎在设计控制石宣后,下令用鐵環穿透石宣下巴鎖著,又將他的飯菜倒入大木槽,使石宣進食時像豬、狗般。石虎逼石宣用舌頭舐著殺石韜的劍上的血,石宣發出了震動宮殿的哀聲。石虎下令在鄴城城北埋起柴堆,上面設置了木竿、竿上安裝了轆轤。並讓石韜生前最寵的宦官,郝稚和劉霸二人拽著石宣的舌头和頭髮,沿著梯子拉上柴堆,之後用轆轤把他絞起來,再用一模一樣的方法向石宣施刑。當石宣已奄奄一息時在柴堆四處點火,石宣被燒成了灰燼。這還未能平熄石虎的怒火,再下令把灰燼分散到名門道中,任人、馬、馬車的輾踏,又將石宣的妻、子九人殺死,又把石宣的衛士、宦官等數百人車裂,將屍體投進漳河。石宣的一个年幼的儿子抱着石虎的大腿求饶,石虎心生怜悯想赦免但大臣们却将其夺走处死,石虎的腰带都被孙子扯断。东宫卫兵十余万被流放边疆,途中举行暴动,石虎急忙调集重兵镇压了下去。但后赵统治基础动摇了。

连杀两位太子后,太尉张举认为燕公石斌、彭城公石遵都有武艺文德,建议从二人中选择储君。但戎昭将军张豺先前曾将刘曜的女儿献给石虎,生有齐公石世,于是他说服石虎立石世,这样刘氏成为太后,他可以辅政。石虎说:“太子二十多岁就想弑父,石世才十岁,等他二十岁了,我已经老了。”于是与张举、李农定议,敕令公卿上书请立石世为太子,于是立石世为太子,其母刘氏为皇后。

石虎病重时,以石遵为大将军,镇关右,石斌为丞相、录尚书事,张豺为镇卫大将军、领军将军、吏部尚书,同受遗诏辅政。刘皇后怕石斌辅政不利于石世,就与张豺合谋,派使者诈称石虎病愈,石斌性好酒猎,于是又恣意而为。刘皇后便矫命称石斌无忠孝之心,免其官,以王归第,派张豺弟张雄率龙腾五百人看守。石遵从幽州来朝,被打发走,石虎知道后说“恨不见之”。一次石虎驾临西阁,龙腾将军、中郎二百余人列拜于前,说宜令燕王石斌入宿卫,典兵马,也有请求以石斌为皇太子。石虎不知石斌已被罢官囚禁,命召石斌来,左右说石斌饮酒得病不能入。石虎又命以辇迎之,要将其玺绶交给他,最后也没人前去。不久石虎昏眩入内。张豺让张雄等矫石虎命杀石斌,刘皇后又矫命以张豺为太保、都督中外诸军、录尚书事,加千兵百骑,一依霍光辅汉故事。

石虎死后,石世继位,不久就被推翻,石虎诸子石遵、石鉴、石祗相继登基,又相继被杀。石虎死后三年,后赵就灭亡了。

\subsubsection{建武}

\begin{longtable}{|>{\centering\scriptsize}m{2em}|>{\centering\scriptsize}m{1.3em}|>{\centering}m{8.8em}|}
  % \caption{秦王政}\
  \toprule
  \SimHei \normalsize 年数 & \SimHei \scriptsize 公元 & \SimHei 大事件 \tabularnewline
  % \midrule
  \endfirsthead
  \toprule
  \SimHei \normalsize 年数 & \SimHei \scriptsize 公元 & \SimHei 大事件 \tabularnewline
  \midrule
  \endhead
  \midrule
  元年 & 335 & \tabularnewline\hline
  二年 & 336 & \tabularnewline\hline
  三年 & 337 & \tabularnewline\hline
  四年 & 338 & \tabularnewline\hline
  五年 & 339 & \tabularnewline\hline
  六年 & 340 & \tabularnewline\hline
  七年 & 341 & \tabularnewline\hline
  八年 & 342 & \tabularnewline\hline
  九年 & 343 & \tabularnewline\hline
  十年 & 344 & \tabularnewline\hline
  十一年 & 345 & \tabularnewline\hline
  十二年 & 346 & \tabularnewline\hline
  十三年 & 347 & \tabularnewline\hline
  十四年 & 348 & \tabularnewline
  \bottomrule
\end{longtable}

\subsubsection{太宁}

\begin{longtable}{|>{\centering\scriptsize}m{2em}|>{\centering\scriptsize}m{1.3em}|>{\centering}m{8.8em}|}
  % \caption{秦王政}\
  \toprule
  \SimHei \normalsize 年数 & \SimHei \scriptsize 公元 & \SimHei 大事件 \tabularnewline
  % \midrule
  \endfirsthead
  \toprule
  \SimHei \normalsize 年数 & \SimHei \scriptsize 公元 & \SimHei 大事件 \tabularnewline
  \midrule
  \endhead
  \midrule
  元年 & 349 & \tabularnewline
  \bottomrule
\end{longtable}


%%% Local Variables:
%%% mode: latex
%%% TeX-engine: xetex
%%% TeX-master: "../../Main"
%%% End:
