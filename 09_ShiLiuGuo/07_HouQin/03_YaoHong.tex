%% -*- coding: utf-8 -*-
%% Time-stamp: <Chen Wang: 2019-12-19 15:09:10>

\subsection{姚泓\tiny(416-417)}

\subsubsection{生平}

姚泓(388年-417年),字元子,十六国时期后秦末主,后秦文桓帝姚兴长子。

后秦弘始十八年(416年)正月,姚兴卒,姚泓即位。兄弟相争,国中大乱。八月东晋刘裕起兵伐秦。后秦疲于应敌之际,国内又相继发生姚懿、姚恢的叛乱。

次年八月癸亥(417年9月20日),刘裕帐下大将王镇恶攻入长安平朔门。姚泓无计可出,准备出降,他十一岁的儿子姚佛念说,晋朝人“将逞其欲”,(即使投降)我们也一定不能保全自己,我愿自杀。姚泓怃然不知所对。佛念登上城墙自投而死。姚泓率一家老小至王镇恶大營投降,其堂叔姚赞也率宗室子弟一百余人投降。刘裕將後秦王室全部处死,其余宗族成员迁往江南。姚泓被押往建康斩首。后秦亡。

《晋书》载,泓孝友宽和,而无经世之用,又多疾病,兴将以为嗣而疑焉。久之,乃立为太子。兴每征伐巡游,常留总后事。博学善谈论,尤好诗咏。

\subsubsection{永和}

\begin{longtable}{|>{\centering\scriptsize}m{2em}|>{\centering\scriptsize}m{1.3em}|>{\centering}m{8.8em}|}
  % \caption{秦王政}\
  \toprule
  \SimHei \normalsize 年数 & \SimHei \scriptsize 公元 & \SimHei 大事件 \tabularnewline
  % \midrule
  \endfirsthead
  \toprule
  \SimHei \normalsize 年数 & \SimHei \scriptsize 公元 & \SimHei 大事件 \tabularnewline
  \midrule
  \endhead
  \midrule
  元年 & 416 & \tabularnewline\hline
  二年 & 417 & \tabularnewline
  \bottomrule
\end{longtable}


%%% Local Variables:
%%% mode: latex
%%% TeX-engine: xetex
%%% TeX-master: "../../Main"
%%% End:
