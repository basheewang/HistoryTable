%% -*- coding: utf-8 -*-
%% Time-stamp: <Chen Wang: 2021-11-01 11:58:29>

\subsection{武昭帝姚苌\tiny(384-394)}

\subsubsection{生平}

秦武昭帝姚\xpinyin*{苌}(329年-393年),字景茂。南安赤亭(今甘肅省隴西縣西)羌族人。十六国时期后秦政权的开国君主。後趙末年南安羌酋長姚弋仲第二十四子,姚襄之弟。姚萇在姚襄死後率其部眾入秦,成為前秦的將領。淝水之戰後姚萇在關中羌人的推舉下自稱萬年秦王,建立後秦,並與苻堅領導下的前秦作戰。姚萇後來殺害了苻堅,並乘西燕東退而進駐長安,不久稱帝。前秦宗室苻登在關中氐族殘餘力量支持下繼續與姚萇作戰,姚萇一度處於不利形勢,但終大敗苻登,漸處優勢,但在消滅前秦勢力前去世,直至兒子姚興即位後才完全消滅前秦勢力。

姚萇年少時已聰慧明智,多有權略,豁達率性,並沒有專注於德行和學業之上,而其眾位兄長都認為他很特別。後來姚萇跟隨姚襄四處出兵,經常參與重要的決策。永和八年(352年),姚襄在麻田敗於前秦軍,其坐騎更中箭死亡,姚萇冒險將自己的坐騎送給姚襄助其出逃。最後姚萇因援軍趕至才得倖免。

升平元年(357年),姚襄謀取關中失敗,在三原(今陝西三原縣)與前秦將領苻黃眉、鄧羌等的交戰中戰死。姚萇當時就率姚襄餘眾盡降前秦。同年前秦宗室苻堅發動政變推翻皇帝苻生,自任天王,並以姚萇為揚武將軍。

太和二年(367年),姚萇隨同王猛參與討伐以略陽郡叛變的羌人斂岐,並因姚弋仲昔日統領斂岐的部落,大量部眾知道姚萇到來都向前秦歸降,令得前秦順利取下略陽。太和六年(371年)三月,与苻雅、杨安、王统、徐成及朱彤等讨伐據有仇池的氐王杨纂,双方決戰於峡谷,杨纂大败,损失三成兵力,終被逼投降。

宁康元年(373年)十一月,前秦攻下東晉領下的益、梁二州,姚萇出任宁州刺史,屯兵於垫江(今重慶市墊江縣)。後遷任步兵校尉,封益都侯。太元元年(376年)五月,与武卫将军苟苌、左将军毛盛、中书令梁熙等進逼黃河,並於八月對前涼發動攻擊,攻滅前涼。

太元八年(383年),東晉荊州刺史桓沖北伐,其中涪城(今四川綿陽市)受到晉將楊亮攻擊,姚萇遂與張蚝出兵救援,逼楊亮退兵。同年苻堅大舉攻晉,意圖滅掉東晉,統一全國,史稱淝水之戰。當時苻堅就以姚萇為龍驤將軍,督益、梁二州諸軍事,讓其從蜀地率軍進攻東晉西方,更說:「朕昔日就是以龍驤將軍建立大業,這個將軍號從來都沒有改授他人,今天特別對你授予此號,山南之事都交給你了。」

苻堅於淝水之戰中大敗,姚萇返回長安。而前秦在戰敗後國力大衰,其中北地長史慕容泓於戰後第二年在關東起兵叛亂,回屯華陰(今陝西華陰市),響應於河北地區叛變的叔父慕容垂。苻堅於是命雍州牧苻叡出兵討伐,而姚萇則任其司馬。當時慕容泓因畏懼而率眾東逃關東,苻叡因輕敵而決心追去邀擊,不聽姚萇的諫言,最終遭慕容泓擊敗,苻叡亦戰死。姚萇在敗後派長史趙都及參軍姜協向苻堅謝罪,但二人卻被憤怒的苻堅殺死,驚懼的姚萇於是逃到渭北的牧馬場。在當地,尹緯、尹詳及龐演等人聯結羌族豪強共五萬多戶向姚萇歸降,並推姚萇為盟主。姚萇於是在太元九年(384年)自稱大將軍、大單于、萬年秦王,改元「白雀」,建立後秦政權。

姚萇接著進屯北地,華陰、北地、新平及安定各郡共有十多萬名羌胡外族歸附。不久苻堅親自率軍討伐姚萇,姚萇屢敗更遭前秦軍斷絕水源。然而就在後秦軍中有人渴死及在恐懼當中時就遇上天雨,營中水深三尺,解決了水荒,亦令後秦軍心復振。不久姚萇出兵反擊,擊敗前秦將楊璧並俘獲楊璧、徐成及毛盛等數十人,皆禮待而送還。而隨著西燕軍隊逼近長安,苻堅率兵回防長安。雖然姚萇在早前向西燕送質請和,但當時姚萇群臣卻建議姚萇加入戰鬥以奪取長安,建立根本之地。不過姚萇自度慕容氏獲勝並後不會長留關中,必會東歸河北,故此打算北屯九嵕(今陝西乾縣東北)以北一帶地區(嶺北)以積聚實力和資源,待前秦亡國而西燕東歸後自取長安。姚萇隨後親自率軍進攻新平郡城(今陝西省邠縣),卻遭守將苟輔頑強抵抗,有萬多人陣亡。苟輔又詐降誘騙姚萇入城,雖然姚萇入城前就察覺而沒進城,但仍受到苟輔伏兵攻擊,萬多人戰死之餘亦險些被擒。

因為新平久久不下,姚萇於是在白雀二年(385年)正月留兵繼續攻城,自己另外出兵安定郡,擒下前秦安西将军苻珍,亦令嶺北諸城降,唯新平未下。至四月,新平物資匱乏,亦無外援,苟輔接受後秦軍的勸降,率城內五千人出降。姚苌下令將所有人坑殺,奪取了新平。五月,苻堅離開長安,出屯五將山,至七月時後秦將吴忠捕獲苻坚,送至新平。同年八月,姚萇因向苻堅索取傳國玉璽不遂,更遭其出言侮辱,於是縊殺苻堅於新平佛寺(今彬縣南靜光寺)。姚萇為了掩飾他殺死苻堅的行為,諡苻堅為「壯烈天王」。

十月,已據有長安的西燕王慕容沖派高蓋攻伐姚萇,遭後秦軍擊敗並投降。白雀三年(386年),西燕國內政變頻生,並開始棄守長安東歸。時盧水胡郝奴乘虛入據長安並稱帝,更命其弟郝多進攻於馬嵬(今陝西興平市馬嵬鎮)自守的王驎。姚萇此時從安定東攻,逼走王驎並擒得郝多,並進攻長安,令郝奴懼而請降。取長安後姚萇就於同月即位為帝,改年號「建初」,建國號大秦。不久又擊敗了前秦秦州刺史王統,奪取秦州。

但同一年,前秦宗室苻登就在關中氐族殘餘勢力的推舉下與後秦對抗,不久在前秦帝苻丕遇害後更稱帝繼位。起初苻登力量甚盛,在涇陽(今陝西涇陽縣)大敗姚碩德,要姚萇親自出兵救援;更謀攻長安。不過當時前秦重將苻纂為苻師奴所殺,將領蘭櫝遂與苻師奴反目。蘭櫝因受西燕皇帝慕容永攻擊而向後秦求援,姚萇以苻登遲疑慎重而少決斷,不敢出兵深入而冒著遭乘虛後襲的危險,決意親自率軍救援。最終先破苻師奴並盡收其眾,後敗慕容永並生擒蘭櫝。

另姚方成亦擊敗徐嵩,徐嵩雖然被俘仍大罵姚萇不僅背叛對其有恩的苻堅,更將他殺害,不惜恩情就連狗和馬都不如。姚方成殺死徐嵩後,姚萇又掘出苻堅的屍首不斷鞭撻,更脫光屍身的衣服,裹以荊棘並以土坑埋掉,以釋心中憤怨。建初三年(388年),自春季開始夏末,姚、苻兩軍就分別據朝那(今寧夏朝那縣)及武都(今甘肅武都縣)相持並交戰,互有勝負而不能擊倒對方,於是都解兵歸還。但關西豪傑都以後秦久久未能站穩關中,反多次敗給苻登,大多都投向前秦,唯齊難、徐洛生、劉郭單等人仍然忠於後秦,提供軍糧並跟隨姚萇征戰。

建初四年(389年),姚萇屢次敗於苻登,命姚崇襲擊苻登於大界的輜重又不得,而苻登就已威脅安定。面對如此局面,姚萇堅拒與苻登正面決戰,力圖以計取勝,於是乘夜率兵三萬再攻大界,終攻克大界並殺毛皇后等人及生擒數十名前秦名將。姚萇隨後亦不貪勝,堅拒乘勝進擊苻登,苻登於是收餘眾退守胡空堡,但已元氣大傷。

在大敗苻登輜重後的四個月後,姚萇設計讓其將任盆詐降以誘殺苻登,雖然最終因雷惡地識破而事敗,但苻登卻忌憚雷惡地,逼其降於姚萇。次年(390年)魏揭飛攻後秦,雷惡地叛迎魏揭飛,雖然當時苻登正在長安附近的新豐(今陝西西安市臨潼區),但姚萇以雷惡地「智略非常」,於是親自出兵攻伐魏揭飛。魏揭飛見姚萇兵少就讓全軍進擊,姚萇特意示弱不戰,卻派了姚崇從敵軍後方攻擊令其混亂,接著就出兵直擊,大敗對方並陣斬魏揭飛,又再降雷惡地並不減昔日待遇。雷惡地兩度歸於姚萇,終對其心服。另外姚萇亦不怕前秦兗州刺史強金槌詐降,只帶著數百騎兵隨其訪問強金槌的軍營,以坦誠獲得了身為氐族人的強金槌的信任,令其不應其他氐族勢力的計謀而加害姚萇。

至建初六年(391年)十二月,苻登進攻安定,姚萇在安定城東擊敗他。次年三月,前秦將沒弈干亦向後秦歸降,但姚萇不久就患病。苻登得知姚萇患病就乘機進攻安定,至八月姚萇病情轉好就親自率兵抵抗,更乘苻登出營迎擊而命姚熙隆進襲前秦軍營,令苻登懼而退兵。姚萇又讓軍隊旁出跟隨苻登,苻登得知後秦營壘空空如也,失去其影蹤後更為驚懼,只得敗還雍城(今陝西鳳翔縣南)。

建初八年(393年)十月,姚萇病重而回長安。至同年十二月,姚萇召太尉姚旻、僕射尹緯及姚晃、將軍姚大目和尚書狄伯支受遺詔輔政,輔助太子姚興。及後姚萇去世,享年六十四歲。姚興先秘不發喪,至次年才發布死訊,上諡號為武昭皇帝,廟號太祖。

姚萇簡單率直,即使當了君主,屬下有過錯可能還會直加責罵。權翼曾勸他不要這樣對待屬下,但姚萇自以這是自己本性,更稱自己聽正直之言,能知己過。

姚萇甚得苻堅重用,尤以其為龍驤將軍,並以自己從龍驤將軍登位至前秦君主一事作勉勵。但姚萇終殺害苻堅,此行為成了前秦將領反對及討伐他的理由,而姚萇亦曾挖屍洩忿。不過在屢敗於苻登後,卻認為是苻堅亡魂的助力,於是也在軍中樹立苻堅神像祈求道:「新平之禍,不是臣姚萇的錯啊,臣的兄長姚襄從陝州北渡,順著道路要往西邊去,像狐狸死時把頭朝向原本洞穴一樣,只是想要見一見鄉里啊。陛下與苻眉攔阻於路上攻擊他,害他不能成功就死了,姚襄遺命臣一定要報仇。苻登是陛下的遠親亦想復仇,臣為自己的兄長報仇,又怎麼說是辜負了義理呢?當年陛下封我為龍驤將軍,跟我說:『朕從龍驤將軍當上了皇帝,卿也好好努力罷!』這明明白白的詔諭非常顯然,好像還在耳邊一樣。陛下已經過世成為神明了,怎麼會透過苻登而謀害臣,忘卻當年說的話呢!現在為陛下立神像,請陛下的靈魂進入這裏,聽臣至誠的禱告。」 不過戰況仍未有改善,反時有夜驚,並招來苻登批評,終毀了苻堅像。據說姚萇死前曾夢見過苻堅率天官、鬼兵去襲擊他(《晉書》「將天官使者、鬼兵數百突入營中」),期間他被救援自己的士兵誤傷陰部至大量出血。醒後就發現陰部腫脹,醫者刺腫處則如夢中一樣大量出血(《晉書》「誤中萇陰,出血石餘」),如此嚇得姚萇發狂胡言,又求苻堅原諒,姚萇不久傷重身亡,臨終前跪伏床頭,叩首不已。

即使姚萇在位期間皆與前秦等勢力戰鬥,但仍設立太學,禮遇先賢後代;又曾命各鎮都要設置學官,由他們評核人才優劣再隨其才能擢用,皆可見其重視文教和吸納文人的行為。而他在安定亦修治德政,大行教化,省卻不必要的支出,亦表彰平民戶中有善行的人。

姚萇長期征戰,雖為君主亦不貪圖逸樂,於與前秦相持不下,部分豪族轉為支持前秦時更寫書自責,並賣掉後宮珍寶去支持軍事,而自己與妻子都力行簡約,對為國戰死的將士皆有所褒揚和追贈。


\subsubsection{白雀}

\begin{longtable}{|>{\centering\scriptsize}m{2em}|>{\centering\scriptsize}m{1.3em}|>{\centering}m{8.8em}|}
  % \caption{秦王政}\
  \toprule
  \SimHei \normalsize 年数 & \SimHei \scriptsize 公元 & \SimHei 大事件 \tabularnewline
  % \midrule
  \endfirsthead
  \toprule
  \SimHei \normalsize 年数 & \SimHei \scriptsize 公元 & \SimHei 大事件 \tabularnewline
  \midrule
  \endhead
  \midrule
  元年 & 384 & \tabularnewline\hline
  二年 & 385 & \tabularnewline\hline
  三年 & 386 & \tabularnewline
  \bottomrule
\end{longtable}

\subsubsection{建初}

\begin{longtable}{|>{\centering\scriptsize}m{2em}|>{\centering\scriptsize}m{1.3em}|>{\centering}m{8.8em}|}
  % \caption{秦王政}\
  \toprule
  \SimHei \normalsize 年数 & \SimHei \scriptsize 公元 & \SimHei 大事件 \tabularnewline
  % \midrule
  \endfirsthead
  \toprule
  \SimHei \normalsize 年数 & \SimHei \scriptsize 公元 & \SimHei 大事件 \tabularnewline
  \midrule
  \endhead
  \midrule
  元年 & 386 & \tabularnewline\hline
  二年 & 387 & \tabularnewline\hline
  三年 & 388 & \tabularnewline\hline
  四年 & 389 & \tabularnewline\hline
  五年 & 390 & \tabularnewline\hline
  六年 & 391 & \tabularnewline\hline
  七年 & 392 & \tabularnewline\hline
  八年 & 393 & \tabularnewline\hline
  九年 & 394 & \tabularnewline
  \bottomrule
\end{longtable}

%%% Local Variables:
%%% mode: latex
%%% TeX-engine: xetex
%%% TeX-master: "../../Main"
%%% End:
