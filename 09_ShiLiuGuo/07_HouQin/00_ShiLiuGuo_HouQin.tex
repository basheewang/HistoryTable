%% -*- coding: utf-8 -*-
%% Time-stamp: <Chen Wang: 2019-12-19 10:22:14>


\section{后秦\tiny(384-417)}

\subsection{简介}

后秦(384年-417年,或稱姚秦)是十六国时期羌人贵族姚苌建立的政权。

前秦苻坚淝水兵败后,关中空虚,原降于前秦的羌人贵族姚苌在渭北叛秦,晋太元九年(384年)自称“万年秦王”,都北地(今陕西耀县东南)。次年(385年)擒杀苻坚。太元十一年(386年)姚苌称帝于长安(今陕西西安西北),国号秦,史称后秦。

其国号以所统治地区为战国时秦国故地为名。《十六国春秋》始称“后秦”,以别于前秦和西秦,后世袭用之。又以王室姓姚而别称姚秦。

统治地区包括今陕西、甘肃东部和河南部分地区。

后秦建初七年(393年)姚苌卒,子姚兴继位,攻杀前秦苻登,扫除前秦残部;又乘后燕灭西燕,尽占原西燕河东之地;弘始元年(399年)乘东晋内乱,陷洛阳,淮汉以北诸城多请降,国势遂与后燕相当。伐後涼,得鳩摩羅什。是年,法顯從長安出發西行求經。

弘始十八年(416年)姚兴卒,子姚泓继位。國內曾歸降的多族勢力趁機反叛,乘丧发兵。东晋劉裕派檀道濟等北伐,陷洛阳。后秦宗室皇弟為奪位反叛,被姚泓消滅。永和二年(417年)东晋圍攻长安,姚泓舉家投降,竟被劉裕滅族,后秦亡。后秦共存在32年(384-417)。

\subsection{景元帝生平}

姚弋仲(280年-352年),南安郡赤亭(今甘肅省隴西縣西)羌人。西晉末期至五胡十六國前期人物,南安羌族酋長,先後降於前趙、後趙及東晉。姚弋仲亦是後秦開國君主姚萇之父。

姚弋仲為燒當羌後代,漢光武帝建武中元年間其先祖滇虞因侵擾東漢而受東漢朝廷討伐,被逼逃亡出塞。至遷那時內附,至此獲居於南安郡赤亭縣。姚弋仲是遷那的五世孫,其父是曹魏鎮西將軍、西羌都督柯回。

姚弋仲年少聰明而勇猛,英明果斷,雄武剛毅,不治產業而以收容救濟為務,故很受眾人敬服。永嘉六年(312年),時值永嘉之亂次年,姚弋仲舉眾東遷榆眉,胡漢人民扶老攜幼跟隨者有數萬,姚弋仲並於此時自稱護西羌校尉、雍州刺史、「扶風公」。太寧元年(323年),前趙帝劉曜消滅盤據隴西的陳安後,關隴地區的氐、羌部落都向前趙請降,劉曜就以姚弋仲為平西將軍,封平襄公。

劉曜於咸和三年(328年)敗於後趙天王石勒後,留守長安的太子劉熙於次年棄守長安,出奔上邽(今甘肅天水市),導致關中大亂,後趙乘時進取關中。不久石虎更領兵攻下上邽,消滅前趙殘餘勢力,姚弋仲亦於是歸降後趙,並獲石虎推薦行安西將軍、六夷左都督。姚弋仲當時向石虎建議遷移隴上豪族,以削弱其實力並充實京畿地區,得石虎聽從。

至咸和八年(333年),後趙帝石勒去世,石虎以丞相掌握朝權,因著姚弋仲前言及氐酋蒲洪的勸言,於是遷關中豪族及氐、羌共十萬戶到首都襄國(今河北邢台)所在的關東地區,並命姚弋仲為奮武將軍、西羌大都督,封襄平縣公,讓他的部眾遷居於清河郡的灄頭(今河北棗強縣東北)。後又遷持節、十郡六夷大都督、冠軍大將軍。

永和五年(349年),高力督梁犢與其部眾兵變,聲勢浩大,並擊敗石虎派往討伐的李農。石虎當時大為恐懼,並召姚弋仲與燕王石斌討伐梁犢。姚弋仲率其部眾八千餘人輕騎至首都鄴城(今河北省臨漳縣)。當時石虎已重病,不能馬上接見,只先賞賜姚弋仲酒食。姚弋仲怒而不食,說:「召我擊賊,豈來覓食邪!我不知上存亡,若一見,雖死無恨。」石虎接見後加授姚弋仲使持節、侍中、征西大將軍,賜鎧馬。隨後姚弋仲就不辭而出,策馬南奔,大破叛軍,斬梁犢。因功加劍履上殿,入朝不趨,進封西平郡公。

同年,石虎去世,太子石世繼位,而征梁犢歸來的姚弋仲、蒲洪等人亦於此時回軍,並與彭城王石遵相遇於李城(今河南溫縣),並共同勸說石遵起兵奪位。石遵隨後起兵,不久就殺石世繼位,並讓冉閔掌有兵權。然而不久冉閔就廢殺石遵,立石鑒為帝,掌握朝政。新興王石祗於是與姚弋仲及蒲洪連兵,移檄討伐冉閔。次年,冉閔殺石鑒並誅殺石氏宗室,姚弋仲就率眾討伐冉閔,移兵至混橋。不久石祗於襄國即位為後趙帝,以姚弋仲為右丞相,封親趙王,並殊有禮待。永和七年(351年),冉閔圍攻襄國,姚弋仲命其子姚襄率兵救援石祗,並配合後趙太尉張舉的行動,遣使向前燕求援。最終在汝陰王石琨、姚襄、前燕三軍以及襄國守軍夾擊之下,圍城的冉閔兵敗,敗退鄴城。雖然姚襄取勝,但因為沒有應姚弋仲在出發前所要求的擒得冉閔,遭姚弋仲以杖打一百責罰。而同年石祗亦被殺,後趙滅亡,姚弋仲於是遣使向東晉投降,獲授使持節、六夷大都督、都督江淮諸軍事、車騎大將軍、儀同三司、大單于,封高陵郡公。

次年(352年),姚弋仲在患病時向諸子說:「石氏厚待我,我本來想盡力幫助他們。而今天石氏已經滅了,中原無主;我死了以後,你們要盡快歸降晉室,並固守臣節,不要做不義的事呀!」及後去世,享年七十三歲。其五子姚襄續統其眾。

姚襄後為苻生所敗,弋仲的靈柩為其所獲,苻生以王禮葬弋仲於天水冀縣。後來,姚弋仲第二十四子姚萇稱後秦帝時,追諡姚弋仲為景元皇帝,廟號始祖,其墓稱為「高陵」,置园邑五百家。现为天水市域重点文物古迹。

《晉書》載姚弋仲個性「清儉鯁直,不修威儀,屢獻讜言,無所回避」,連殘暴的石虎也敬重三分,334年,石虎廢皇帝石弘自立,弋仲稱病不來朝賀,經石虎不斷召見才至,弋仲正色向石虎說:「奈何把臂受託而反奪之乎!」石虎也因為弋仲正直而不責怪他。後石虎一名寵姬的弟弟任武城左尉,擾亂姚弋仲的部眾,姚弋仲就捕捉並數責他,更命人殺了他,雖然最終因對方叩頭至流血作請求以及左右的諫言而不殺他,但也見姚弋仲為事剛直,毫不顧忌對方背景。後討梁犢前得石虎召見,又責備患病的石虎:「兒死,愁邪,何為而病?兒幼時不擇善人教之,使至於為逆;既為逆而誅之,又何愁焉!且汝久病,所立兒幼,汝若不愈,天下必亂,當先憂此,勿憂賊也!犢等窮困思歸,相聚為盜,所過殘暴,何所能至!老羌為汝一舉了之」除了看見他梗直而言,直指他教子無道而導致石宣殺害太子石韜的事件發生,亦見其不論尊卑皆直稱「汝」的行為,連作為皇帝的石虎也不例外。

姚弋仲曾有一個叫馬何羅的部曲曾在張豺主政時叛歸對方。後因石世被廢,張豺亦遭誅殺,馬何羅於是回到姚弋仲那裏。當時眾人都建議姚弋仲殺了他,但姚弋仲就以「招才納奇」為由寬恕他,不但不作加害,反以其為參軍。

姚弋仲在後趙末年一直顯得忠於石氏,不過《資治通鑑》亦有見載於後趙混亂,冉閔篡權時姚弋仲與蒲洪爭奪關中的行動。

\subsection{魏武王生平}

姚襄(约331-357年),字景國,南安赤亭(今甘肅省隴西縣西)羌族酋長,五胡十六國時期諸侯、軍閥,是姚弋仲的第五子,也是後秦開國君主姚萇之兄。

姚襄父親姚弋仲是南安羌酋長,在後趙滅前趙後歸降後趙並接受其官爵。而姚襄雄健威武,多才多藝,觀察入微且善於安撫人心,故獲得部眾愛戴和敬重,眾人並因此請求姚弋仲立姚襄為繼承人。姚弋仲起初以姚襄不是長子,並不允許,然請求的百姓很多,姚弋仲才開始給姚襄帶兵。

永和六年(350年),冉閔殺後趙皇帝石鑒,建立冉魏。隨後後趙新興王石祗就於襄國即位為後趙帝,並以姚襄為使持節、驃騎將軍、護烏丸校尉、豫州刺史、新昌公。永和七年(351年),姚襄奉父命領二萬八千騎兵營救正遭冉閔圍攻的石祗,姚弋仲並於出發前作訓誡:「冉閔背棄仁義,屠滅石氏。我受了人家的優厚待遇,就應為其復仇,我卻因年老患病而不能親身去做;你才能比冉閔高出十倍,若果不能擒殺他回來,就不要再來見我了!」姚襄雖然聯同前燕、石琨及襄國守軍大敗冉閔,暫時解了襄國的危機,卻因無法擒得冉閔,遭姚弋仲以一百杖作處罰。

同年,後趙為冉魏所滅,姚襄隨其父向東晉投降,獲晉廷任命為為持節、平北將軍、都督并州諸軍事、并州刺史、平鄉縣公。次年(352年)姚弋仲去世,死前命諸子在其死後歸降晉室,作晉的忠臣。姚襄接手統率父親部眾,不發布父親去世的消息,並攻下陽平(今山東莘城)、元城(今河北大名縣)及發干(今山東聊城市東昌府區),駐於碻磝津(今山东省茌平县西南古益河上);但不久卻敗給前秦,南走至滎陽(今河南滎陽市)才發喪,後在滎陽與洛陽之間的麻田與前秦軍作戰時,座騎中箭死亡,因姚萇贈馬及援軍趕到才免於被擒。此時姚襄才以五個弟弟為人質,歸降東晉。東晉以姚襄駐屯譙城(今安徽亳州),而姚襄隨後單人匹馬渡過淮河,於壽春(今安徽壽縣)面見豫州刺史謝尚,當時謝尚對其名氣亦有所聽聞,於是撤去衞士,以代表高雅的幅巾接見他。二人一見如故,又因姚襄博學及善於談論,很得江東人士敬重。

同年,姚襄與謝尚一同進攻據守許昌(今河南省許昌市)的後趙豫州牧張遇,但遭前秦丞相苻雄等擊敗,謝尚因大敗而退守淮南,得姚襄棄輜重而護送至芍陂,故此將善後工作都委託給姚襄。謝尚因戰敗而受貶降,及後更被調回京師建康(今江蘇南京市),而當時駐屯歷陽(今安徽和縣)的姚襄亦以當時佔據北方的前秦及前燕皆強盛,無意北伐,反在淮河兩岸大興屯田,訓練將士,不過當時主政的殷浩就要北伐,更忌憚姚襄兵力強盛,不但囚禁姚襄送去當質子的弟弟,更屢次派刺客行刺姚襄,但刺客不能下手,反告訴姚襄實情。後殷浩再暗中派魏憬率兵襲殺他,但魏憬反被姚襄所殺,於是令殷浩更加厭惡姚襄,遷姚襄到梁國的蠡臺,表授他為梁國內史。

永和九年(353年),殷浩北伐,以姚襄為前軀,而當時姚襄已決心叛離東晉,於是算好殷浩快來到時就假意命部眾乘夜逃遁,其實暗中設伏伏擊殷浩。殷浩聽聞姚襄部眾逃遁,追至山桑(今安徽蒙城縣北)就被姚襄伏兵擊敗,被逼退守譙城,而姚襄就俘殺萬多人,盡收其輜重,南據淮南郡一帶。不久姚襄北屯盱眙(今江蘇盱眙縣),在當地收納流民,令部眾增至七萬人,並分置地方官員,鼓勵農事生產,又遣使到建康狀告殷浩,並作道歉。永和十年(354年),姚襄向前燕歸降。

永和十一年(355年),姚襄因應部眾要求北歸,於是自稱大將軍、大單于,北攻晉冠軍將軍高季,卻為高季所敗。姚襄撫恤散敗的兵眾,重新集結力量,及後乘高季去世而據有許昌。次年,姚襄進攻當時據有洛陽的周成,但用了一個多月都不能攻下,當時長史王亮就勸他放棄進攻,免得被他人有機可乘,危及自己。不過姚襄沒有聽從。然而,桓溫不久就發動北伐,征伐姚襄,姚襄被逼放棄圍城而抵抗桓溫,並在伊水以北的樹林中設下精兵,並聲稱自願歸降,請桓溫稍為退兵。然而桓溫沒有答應,並親身督戰,組以兵陣進攻沿河岸抵抗的姚襄軍,姚襄兵敗而北逃至北芒山。姚襄隨後西逃,桓溫因追不到而放棄。姚襄逃到平陽(今山西臨汾市)時得時為前秦并州刺史的舊部尹赤叛秦歸附,於是據守襄陵(今山西襄汾縣);同據并州的張平因而攻打姚襄,姚襄雖不敵,但與張平結為兄弟,換取兩者和平。

升平元年(357年),姚襄謀取關中,先移鎮北屈(今山西吉縣),後進屯杏城(今陝西黃陵縣西南),命姚蘭進攻敷城(今陝西富縣)、姚益及王欽盧聯結關中一帶的羌胡外族,共收得胡漢共五萬多戶。不過姚襄就與時據關中的前秦軍發生衝突,前秦帝苻生遣苻黃眉、苻堅、鄧羌進攻,姚襄堅守不戰。但鄧羌就在其軍壘門外列陣,激得姚襄不聽僧人智通的勸言,親自率眾出戰,最終被鄧羌詐敗誘至三原(今陝西三原縣)。姚襄遭受到鄧羌及苻黃眉的合擊,所乘駿馬「黧眉騧」倒地,姚襄為前秦軍所擒斬,享年二十七歲。其弟姚萇率餘眾投降。苻生後以公爵之禮葬姚襄。

後來,姚萇稱後秦帝時,追諡姚襄為魏武王。

姚襄深得人心,如在他大敗給桓溫而北逃時,就有五千多人於當晚拋下妻兒去追隨他。前後幾次敗仗,百姓只要知道他在哪裡就奔赴投靠。當時謠傳姚襄重傷而死,被桓溫俘虜的百姓沒有不痛哭流涕的。姚襄部屬楊亮後來歸降桓溫,桓溫向楊亮詢問姚襄的為人,楊亮的評價是:「神明器宇,孫策之儔,而雄武過之。」

楊亮:「神明器宇,孫策之儔,而雄武過之。」

姚萇:「吾不如亡兄有四:身長八尺五寸,臂垂過膝,人望而畏之,一也;當十萬之眾,與天下爭衡,望麾而進,前無橫陣,二也;溫古知今,講論道藝,駕馭英雄,收羅儁異,三也;統率大眾,履險若夷,上下咸允,人盡死力,四也。所以得建立功業,策任群賢者,正望算略中一片耳。」

呂思勉:「其(姚襄)才略或在苻健之上。然寄居晉地,四面追敵,不如健之入關,有施展之地矣。」

2009年12月27日,河南有关方面宣布在安阳发现曹操墓。这一发现,引发许多质疑。 西安市委党校历史教授胡觉照接受记者采访时称,安阳“曹操墓”实则五胡十六国时期军阀姚襄墓穴。


%% -*- coding: utf-8 -*-
%% Time-stamp: <Chen Wang: 2021-11-01 11:58:29>

\subsection{武昭帝姚苌\tiny(384-394)}

\subsubsection{生平}

秦武昭帝姚\xpinyin*{苌}(329年-393年),字景茂。南安赤亭(今甘肅省隴西縣西)羌族人。十六国时期后秦政权的开国君主。後趙末年南安羌酋長姚弋仲第二十四子,姚襄之弟。姚萇在姚襄死後率其部眾入秦,成為前秦的將領。淝水之戰後姚萇在關中羌人的推舉下自稱萬年秦王,建立後秦,並與苻堅領導下的前秦作戰。姚萇後來殺害了苻堅,並乘西燕東退而進駐長安,不久稱帝。前秦宗室苻登在關中氐族殘餘力量支持下繼續與姚萇作戰,姚萇一度處於不利形勢,但終大敗苻登,漸處優勢,但在消滅前秦勢力前去世,直至兒子姚興即位後才完全消滅前秦勢力。

姚萇年少時已聰慧明智,多有權略,豁達率性,並沒有專注於德行和學業之上,而其眾位兄長都認為他很特別。後來姚萇跟隨姚襄四處出兵,經常參與重要的決策。永和八年(352年),姚襄在麻田敗於前秦軍,其坐騎更中箭死亡,姚萇冒險將自己的坐騎送給姚襄助其出逃。最後姚萇因援軍趕至才得倖免。

升平元年(357年),姚襄謀取關中失敗,在三原(今陝西三原縣)與前秦將領苻黃眉、鄧羌等的交戰中戰死。姚萇當時就率姚襄餘眾盡降前秦。同年前秦宗室苻堅發動政變推翻皇帝苻生,自任天王,並以姚萇為揚武將軍。

太和二年(367年),姚萇隨同王猛參與討伐以略陽郡叛變的羌人斂岐,並因姚弋仲昔日統領斂岐的部落,大量部眾知道姚萇到來都向前秦歸降,令得前秦順利取下略陽。太和六年(371年)三月,与苻雅、杨安、王统、徐成及朱彤等讨伐據有仇池的氐王杨纂,双方決戰於峡谷,杨纂大败,损失三成兵力,終被逼投降。

宁康元年(373年)十一月,前秦攻下東晉領下的益、梁二州,姚萇出任宁州刺史,屯兵於垫江(今重慶市墊江縣)。後遷任步兵校尉,封益都侯。太元元年(376年)五月,与武卫将军苟苌、左将军毛盛、中书令梁熙等進逼黃河,並於八月對前涼發動攻擊,攻滅前涼。

太元八年(383年),東晉荊州刺史桓沖北伐,其中涪城(今四川綿陽市)受到晉將楊亮攻擊,姚萇遂與張蚝出兵救援,逼楊亮退兵。同年苻堅大舉攻晉,意圖滅掉東晉,統一全國,史稱淝水之戰。當時苻堅就以姚萇為龍驤將軍,督益、梁二州諸軍事,讓其從蜀地率軍進攻東晉西方,更說:「朕昔日就是以龍驤將軍建立大業,這個將軍號從來都沒有改授他人,今天特別對你授予此號,山南之事都交給你了。」

苻堅於淝水之戰中大敗,姚萇返回長安。而前秦在戰敗後國力大衰,其中北地長史慕容泓於戰後第二年在關東起兵叛亂,回屯華陰(今陝西華陰市),響應於河北地區叛變的叔父慕容垂。苻堅於是命雍州牧苻叡出兵討伐,而姚萇則任其司馬。當時慕容泓因畏懼而率眾東逃關東,苻叡因輕敵而決心追去邀擊,不聽姚萇的諫言,最終遭慕容泓擊敗,苻叡亦戰死。姚萇在敗後派長史趙都及參軍姜協向苻堅謝罪,但二人卻被憤怒的苻堅殺死,驚懼的姚萇於是逃到渭北的牧馬場。在當地,尹緯、尹詳及龐演等人聯結羌族豪強共五萬多戶向姚萇歸降,並推姚萇為盟主。姚萇於是在太元九年(384年)自稱大將軍、大單于、萬年秦王,改元「白雀」,建立後秦政權。

姚萇接著進屯北地,華陰、北地、新平及安定各郡共有十多萬名羌胡外族歸附。不久苻堅親自率軍討伐姚萇,姚萇屢敗更遭前秦軍斷絕水源。然而就在後秦軍中有人渴死及在恐懼當中時就遇上天雨,營中水深三尺,解決了水荒,亦令後秦軍心復振。不久姚萇出兵反擊,擊敗前秦將楊璧並俘獲楊璧、徐成及毛盛等數十人,皆禮待而送還。而隨著西燕軍隊逼近長安,苻堅率兵回防長安。雖然姚萇在早前向西燕送質請和,但當時姚萇群臣卻建議姚萇加入戰鬥以奪取長安,建立根本之地。不過姚萇自度慕容氏獲勝並後不會長留關中,必會東歸河北,故此打算北屯九嵕(今陝西乾縣東北)以北一帶地區(嶺北)以積聚實力和資源,待前秦亡國而西燕東歸後自取長安。姚萇隨後親自率軍進攻新平郡城(今陝西省邠縣),卻遭守將苟輔頑強抵抗,有萬多人陣亡。苟輔又詐降誘騙姚萇入城,雖然姚萇入城前就察覺而沒進城,但仍受到苟輔伏兵攻擊,萬多人戰死之餘亦險些被擒。

因為新平久久不下,姚萇於是在白雀二年(385年)正月留兵繼續攻城,自己另外出兵安定郡,擒下前秦安西将军苻珍,亦令嶺北諸城降,唯新平未下。至四月,新平物資匱乏,亦無外援,苟輔接受後秦軍的勸降,率城內五千人出降。姚苌下令將所有人坑殺,奪取了新平。五月,苻堅離開長安,出屯五將山,至七月時後秦將吴忠捕獲苻坚,送至新平。同年八月,姚萇因向苻堅索取傳國玉璽不遂,更遭其出言侮辱,於是縊殺苻堅於新平佛寺(今彬縣南靜光寺)。姚萇為了掩飾他殺死苻堅的行為,諡苻堅為「壯烈天王」。

十月,已據有長安的西燕王慕容沖派高蓋攻伐姚萇,遭後秦軍擊敗並投降。白雀三年(386年),西燕國內政變頻生,並開始棄守長安東歸。時盧水胡郝奴乘虛入據長安並稱帝,更命其弟郝多進攻於馬嵬(今陝西興平市馬嵬鎮)自守的王驎。姚萇此時從安定東攻,逼走王驎並擒得郝多,並進攻長安,令郝奴懼而請降。取長安後姚萇就於同月即位為帝,改年號「建初」,建國號大秦。不久又擊敗了前秦秦州刺史王統,奪取秦州。

但同一年,前秦宗室苻登就在關中氐族殘餘勢力的推舉下與後秦對抗,不久在前秦帝苻丕遇害後更稱帝繼位。起初苻登力量甚盛,在涇陽(今陝西涇陽縣)大敗姚碩德,要姚萇親自出兵救援;更謀攻長安。不過當時前秦重將苻纂為苻師奴所殺,將領蘭櫝遂與苻師奴反目。蘭櫝因受西燕皇帝慕容永攻擊而向後秦求援,姚萇以苻登遲疑慎重而少決斷,不敢出兵深入而冒著遭乘虛後襲的危險,決意親自率軍救援。最終先破苻師奴並盡收其眾,後敗慕容永並生擒蘭櫝。

另姚方成亦擊敗徐嵩,徐嵩雖然被俘仍大罵姚萇不僅背叛對其有恩的苻堅,更將他殺害,不惜恩情就連狗和馬都不如。姚方成殺死徐嵩後,姚萇又掘出苻堅的屍首不斷鞭撻,更脫光屍身的衣服,裹以荊棘並以土坑埋掉,以釋心中憤怨。建初三年(388年),自春季開始夏末,姚、苻兩軍就分別據朝那(今寧夏朝那縣)及武都(今甘肅武都縣)相持並交戰,互有勝負而不能擊倒對方,於是都解兵歸還。但關西豪傑都以後秦久久未能站穩關中,反多次敗給苻登,大多都投向前秦,唯齊難、徐洛生、劉郭單等人仍然忠於後秦,提供軍糧並跟隨姚萇征戰。

建初四年(389年),姚萇屢次敗於苻登,命姚崇襲擊苻登於大界的輜重又不得,而苻登就已威脅安定。面對如此局面,姚萇堅拒與苻登正面決戰,力圖以計取勝,於是乘夜率兵三萬再攻大界,終攻克大界並殺毛皇后等人及生擒數十名前秦名將。姚萇隨後亦不貪勝,堅拒乘勝進擊苻登,苻登於是收餘眾退守胡空堡,但已元氣大傷。

在大敗苻登輜重後的四個月後,姚萇設計讓其將任盆詐降以誘殺苻登,雖然最終因雷惡地識破而事敗,但苻登卻忌憚雷惡地,逼其降於姚萇。次年(390年)魏揭飛攻後秦,雷惡地叛迎魏揭飛,雖然當時苻登正在長安附近的新豐(今陝西西安市臨潼區),但姚萇以雷惡地「智略非常」,於是親自出兵攻伐魏揭飛。魏揭飛見姚萇兵少就讓全軍進擊,姚萇特意示弱不戰,卻派了姚崇從敵軍後方攻擊令其混亂,接著就出兵直擊,大敗對方並陣斬魏揭飛,又再降雷惡地並不減昔日待遇。雷惡地兩度歸於姚萇,終對其心服。另外姚萇亦不怕前秦兗州刺史強金槌詐降,只帶著數百騎兵隨其訪問強金槌的軍營,以坦誠獲得了身為氐族人的強金槌的信任,令其不應其他氐族勢力的計謀而加害姚萇。

至建初六年(391年)十二月,苻登進攻安定,姚萇在安定城東擊敗他。次年三月,前秦將沒弈干亦向後秦歸降,但姚萇不久就患病。苻登得知姚萇患病就乘機進攻安定,至八月姚萇病情轉好就親自率兵抵抗,更乘苻登出營迎擊而命姚熙隆進襲前秦軍營,令苻登懼而退兵。姚萇又讓軍隊旁出跟隨苻登,苻登得知後秦營壘空空如也,失去其影蹤後更為驚懼,只得敗還雍城(今陝西鳳翔縣南)。

建初八年(393年)十月,姚萇病重而回長安。至同年十二月,姚萇召太尉姚旻、僕射尹緯及姚晃、將軍姚大目和尚書狄伯支受遺詔輔政,輔助太子姚興。及後姚萇去世,享年六十四歲。姚興先秘不發喪,至次年才發布死訊,上諡號為武昭皇帝,廟號太祖。

姚萇簡單率直,即使當了君主,屬下有過錯可能還會直加責罵。權翼曾勸他不要這樣對待屬下,但姚萇自以這是自己本性,更稱自己聽正直之言,能知己過。

姚萇甚得苻堅重用,尤以其為龍驤將軍,並以自己從龍驤將軍登位至前秦君主一事作勉勵。但姚萇終殺害苻堅,此行為成了前秦將領反對及討伐他的理由,而姚萇亦曾挖屍洩忿。不過在屢敗於苻登後,卻認為是苻堅亡魂的助力,於是也在軍中樹立苻堅神像祈求道:「新平之禍,不是臣姚萇的錯啊,臣的兄長姚襄從陝州北渡,順著道路要往西邊去,像狐狸死時把頭朝向原本洞穴一樣,只是想要見一見鄉里啊。陛下與苻眉攔阻於路上攻擊他,害他不能成功就死了,姚襄遺命臣一定要報仇。苻登是陛下的遠親亦想復仇,臣為自己的兄長報仇,又怎麼說是辜負了義理呢?當年陛下封我為龍驤將軍,跟我說:『朕從龍驤將軍當上了皇帝,卿也好好努力罷!』這明明白白的詔諭非常顯然,好像還在耳邊一樣。陛下已經過世成為神明了,怎麼會透過苻登而謀害臣,忘卻當年說的話呢!現在為陛下立神像,請陛下的靈魂進入這裏,聽臣至誠的禱告。」 不過戰況仍未有改善,反時有夜驚,並招來苻登批評,終毀了苻堅像。據說姚萇死前曾夢見過苻堅率天官、鬼兵去襲擊他(《晉書》「將天官使者、鬼兵數百突入營中」),期間他被救援自己的士兵誤傷陰部至大量出血。醒後就發現陰部腫脹,醫者刺腫處則如夢中一樣大量出血(《晉書》「誤中萇陰,出血石餘」),如此嚇得姚萇發狂胡言,又求苻堅原諒,姚萇不久傷重身亡,臨終前跪伏床頭,叩首不已。

即使姚萇在位期間皆與前秦等勢力戰鬥,但仍設立太學,禮遇先賢後代;又曾命各鎮都要設置學官,由他們評核人才優劣再隨其才能擢用,皆可見其重視文教和吸納文人的行為。而他在安定亦修治德政,大行教化,省卻不必要的支出,亦表彰平民戶中有善行的人。

姚萇長期征戰,雖為君主亦不貪圖逸樂,於與前秦相持不下,部分豪族轉為支持前秦時更寫書自責,並賣掉後宮珍寶去支持軍事,而自己與妻子都力行簡約,對為國戰死的將士皆有所褒揚和追贈。


\subsubsection{白雀}

\begin{longtable}{|>{\centering\scriptsize}m{2em}|>{\centering\scriptsize}m{1.3em}|>{\centering}m{8.8em}|}
  % \caption{秦王政}\
  \toprule
  \SimHei \normalsize 年数 & \SimHei \scriptsize 公元 & \SimHei 大事件 \tabularnewline
  % \midrule
  \endfirsthead
  \toprule
  \SimHei \normalsize 年数 & \SimHei \scriptsize 公元 & \SimHei 大事件 \tabularnewline
  \midrule
  \endhead
  \midrule
  元年 & 384 & \tabularnewline\hline
  二年 & 385 & \tabularnewline\hline
  三年 & 386 & \tabularnewline
  \bottomrule
\end{longtable}

\subsubsection{建初}

\begin{longtable}{|>{\centering\scriptsize}m{2em}|>{\centering\scriptsize}m{1.3em}|>{\centering}m{8.8em}|}
  % \caption{秦王政}\
  \toprule
  \SimHei \normalsize 年数 & \SimHei \scriptsize 公元 & \SimHei 大事件 \tabularnewline
  % \midrule
  \endfirsthead
  \toprule
  \SimHei \normalsize 年数 & \SimHei \scriptsize 公元 & \SimHei 大事件 \tabularnewline
  \midrule
  \endhead
  \midrule
  元年 & 386 & \tabularnewline\hline
  二年 & 387 & \tabularnewline\hline
  三年 & 388 & \tabularnewline\hline
  四年 & 389 & \tabularnewline\hline
  五年 & 390 & \tabularnewline\hline
  六年 & 391 & \tabularnewline\hline
  七年 & 392 & \tabularnewline\hline
  八年 & 393 & \tabularnewline\hline
  九年 & 394 & \tabularnewline
  \bottomrule
\end{longtable}

%%% Local Variables:
%%% mode: latex
%%% TeX-engine: xetex
%%% TeX-master: "../../Main"
%%% End:

%% -*- coding: utf-8 -*-
%% Time-stamp: <Chen Wang: 2021-11-01 11:58:34>

\subsection{文桓帝姚兴\tiny(394-416)}

\subsubsection{生平}

秦文桓帝姚兴(366年-416年),字子略,南安赤亭(今甘肅省隴西縣西)羌族人。十六国时期后秦皇帝,后秦武昭帝姚苌长子。姚興即位之初就俘殺了父親在位時面對的最強對手苻登,基本覆滅了前秦。後又出兵後涼,令其投降之餘亦令盤據秦涼一帶的政權如北涼、南涼及西秦等政權臣服,還率兵進攻東晉,一舉攻取洛陽等地,使統治疆域迅速擴大。不過隨後與北魏在柴壁之戰中卻大敗,面對新興的夏國亦不能有效對付,反屢遭侵擾;且國內出現兒子姚弼與太子姚泓爭位的事件,令後秦國勢漸弱。弘始十八年(416年),姚興病逝,諡文桓皇帝,庙号高祖,下葬偶陵。

姚興在前秦時任太子舍人。白雀元年(384年),姚萇在渭北馬牧稱萬年秦王,建後秦,姚興時在長安,冒險出走與父親會合。建初元年(386年),姚萇在奪得長安(今陝西西安)後稱帝,就立了姚興為皇太子。其時姚萇屢次在外與前秦對抗,姚興就經常留鎮長安以統後事。其時又與太子中舍人梁喜及太子洗馬范勖講論經籍,不以兵戎廢業,當時的人亦受他們影響。

建初八年(393年)十二月,姚萇去世,死前命太尉姚旻、僕射尹緯、姚晃、將軍姚大目及尚書狄伯支為輔政大臣,並向姚興說:「若有人謗毀這幾位大臣,小心不要聽信。你以仁管教子女,以禮對待大臣,以信處事,以恩治民,這四項你能做到,我就不憂心了。」姚萇死後,姚興秘不發喪,分命姚緒、姚碩德及姚崇駐安定、陰密及長安,自己就自稱大將軍,領兵進攻前秦。

次年春,前秦皇帝苻登聽聞姚萇已死即十分高興,又輕視姚興,隨即率眾東進。至夏季,苻登要進攻廢橋,尹緯則受命支援守馬嵬堡的姚詳,尹緯於是據守廢橋等待前秦軍。前秦軍因無法取得水源而缺水,兩三成士兵更因而渴死,於是急攻尹緯希望能奪取水源。姚興當時認為苻登已是窮寇,於是派狄伯支命令尹緯要持重拒戰,不要輕易與前秦軍決戰。不過尹緯認為姚萇新死,人心恐懼不安,應當用盡力量消滅敵人,安定眾心。尹緯於是與苻登決戰,終大敗前秦軍,苻登因兵眾潰散而逃走,逃到馬毛山。戰後,姚興才正式發喪,並在槐里(今陝西興平東南)即位為帝,改元「皇初」。七月,姚興進攻苻登並在馬毛山南作戰,擒殺苻登,並解散其部眾。不久繼位的前秦皇帝苻崇因被乞伏乾歸逼逐而聯結楊定進攻乞伏乾歸,卻遭對方所殺,前秦正式滅亡。

皇初七年(397年),姚興率兵進攻東晉控制的湖城,弘農太守陶仲山及華山太守董邁都投降。姚興於是進至陝城(今河南陝縣),並攻下上洛(今陝西商洛市)。另又分遣姚崇進攻洛陽(今河南洛陽),因晉河南太守夏侯宗之守金鏞城而未能攻克,於是改攻柏谷,強遷兩萬多戶流民西歸。及至皇初九年(399年),姚興命姚崇及楊佛嵩再攻洛陽,守將辛恭靖堅守一百多日後失守,後秦奪得洛陽。取洛陽後,淮河、漢水以北各城大多都向後秦請降,並送人質。

弘始二年(400年),姚碩德進攻西秦,西秦王乞伏乾歸率眾抵抗,兩軍對峙期間姚碩德軍中柴草缺乏,姚興就暗中領兵支援。乞伏乾歸知道姚興派軍前來,於是命慕兀率二萬中軍屯柏楊(今甘肅清水縣西南),羅敦率外軍屯侯辰谷,自己領數千輕騎等候秦軍。不過其夜遇上大風和大霧,乞伏乾歸與慕兀的中軍失去聯絡,被逼與外軍會合。天亮後,乞伏乾歸就與後秦軍作戰,終大敗並逃返苑川(今甘肅榆中縣北),後秦軍受降共三萬六千多人,姚興則進軍枹罕(今甘肅臨夏市)。乞伏乾歸初降禿髮利鹿孤,但因怕不為對方所容,最終決定歸降後秦。

弘始三年(401年),姚興命姚碩德進攻後涼,並兵圍後涼首都姑臧(今甘肅武威)。後涼王呂隆被逼請降。而在後秦攻涼時,西涼李暠、南涼禿髮利鹿孤及北涼沮渠蒙遜都遣使向後秦請降。直至弘始五年(403年),後涼被南涼和北涼所逼,最終請後秦派軍迎來歸附,姚興因而派了齊難等人到姑臧,駐兵當地並送呂氏宗族內徙長安,吞併後涼。另外在攻打後涼姑臧時,連帶的將名僧鳩摩羅什請回長安。爾後為鳩摩羅什講解《法華經》,建造「長安大寺」。鳩摩羅什於長安圓寂,其生前將大乘佛教的主要經典(如《中論》、《法華經》、《維摩詰經》等)譯為漢文。

北魏君主拓跋珪曾經送一千匹馬到後秦請婚,姚興原先答應,但知拓跋珪已立了后,於是拒絕並留下使者賀狄干。弘始四年(402年),北魏將領拓跋遵進攻高平(今甘肅固原),沒弈干拋棄部眾,帶著數千騎兵及赫連勃勃逃到秦州。北魏軍追擊至瓦亭仍未追上,於是盡遷高平的物資回國;及後北魏平陽太守貮塵又進攻河東。北魏的一系列軍事行動震動長安,關中各城日間也緊閉城門,姚興於是在城西閱兵,並做好戰爭準備。同年,姚興派姚平及狄伯支等率四萬步騎兵進攻北魏,姚興則親率大軍在後。北魏帝拓跋珪則命拓跋順及長孫肥統六萬騎兵為先鋒,自己也率大軍在後以作抵抗。姚平用了六十多天攻陷了北魏屯積糧食的乾壁,又派二百精騎偵察魏軍,卻為長孫肥襲擊,所有人都被生擒。姚平因而後撤,又遭拓跋珪追擊,並在柴壁(今山西襄汾縣西南)被追上;姚平當時據柴壁城固守,北魏軍則圍困城池。姚興於是自領四萬七千兵營救姚平,並打算佔領天渡以運糧支援姚平。不過北魏加強了包圍圈,又在汾水建浮橋,在汾水西岸築圍堵截姚興援軍,務求引姚興取道汾東,經長達三百多里而缺乏小路通行的蒙坑進攻。而姚興到蒲阪後因怕魏軍強盛,很久才正式進攻。及後姚興在蒙坑以南與拓跋珪所率三萬步騎兵作戰,後秦軍共千多人被殺,姚興被逼退走四十多里,而姚平亦未能突圍。接著拓跋珪分兵各據險要,不讓後秦軍接近柴壁。姚興駐屯汾西,在汾水上游放木材打算沖毀北魏浮橋,但木材都被魏軍截取。至十月,姚平軍需用盡,在夜間試圖向西南方突圍,姚興列兵汾西,點起烽火和擂鼓響應,不過姚興欲救姚平盡力突陣,姚平反望姚興攻圍接應,兩軍雖然能夠以叫喊相通,但始終都沒能壓逼圍城魏軍。姚平最終無法成功突圍,於是率眾投水自殺,然而拓跋珪卻都派人潛下水捕捉,赴水諸將與城中狄伯支、唐小方等人及兩萬多兵眾都被俘。姚興只能見城中軍隊束手就擒而無力支援,全軍都哀傷痛哭,哭聲震動山谷。接著姚興數度派遣使者求和,但都被拒,魏軍更乘勝進攻蒲阪。防禦蒲阪的姚緒固守不戰,又正因柔然要進攻北魏,拓跋珪才撤兵。

弘始九年(407年),北魏歸還柴壁之戰中被俘的唐小方等人,姚興又以良馬千匹贖回狄伯支,與北魏通和。赫連勃勃因後秦與北魏連和而大怒,竟搶奪了柔然送給後秦的八千匹馬,並襲殺沒弈干叛變,稱大夏天王,建夏國。赫連勃勃隨後又攻破鮮卑薛干等三部,並進攻後秦三城以北諸戍,後秦將楊丕、姚石生等都被殺,接著又侵掠嶺北,令嶺北各城城門白天也要緊閉。姚興此時感嘆:「我不聽黃兒(姚興弟姚邕小字)的話,才弄成今天這樣子。」

隨後禿髮傉檀大敗於赫連勃勃,名將折損達六七成,接著成七兒及梁裒、邊憲等又先後謀反,姚興見其並受外憂外患夾擊,不顧尚書郎韋宗的勸阻和吏部尚書尹昭命北涼及西涼進攻禿髮傉檀的建議,堅持分兵兩道進攻夏和禿髮傉檀。姚興於弘始十年(408年)派了齊難領二萬騎兵攻夏,又派姚弼、斂成及乞伏乾歸攻禿髮傉檀,更寫信給禿髮傉檀聲稱姚弼等其實只是配合齊難進攻夏國的行動,禿髮傉檀不作防備。不過姚弼等到姑臧後反被禿髮傉檀的奇兵擊敗,後又特地釋放牛羊引誘後秦軍掠奪,大敗秦軍。作為後繼的姚顯知姚弼兵敗,加快趕到姑臧,並命孟欽等五名擅長射擊的人於涼風門挑戰,卻遭南涼材官將軍宋益擊殺。姚顯見此委罪於斂成,派使者向禿髮傉檀謝罪,撫慰河西後就撤還。而禿髮傉檀亦派使者徐宿向後秦謝罪。不過在當年又再稱涼王。

而赫連勃勃知齊難來攻,於是退守河曲。齊難見赫連勃勃仍在很遠,於是先行縱兵野略;赫連勃勃因而潛軍來襲,俘殺七千多人,齊難逃走但在木城遭赫連勃勃生擒,其餘將士亦被俘。戰後嶺北共計有數萬人歸附赫連勃勃。弘始十一年(409年),姚興再派姚沖及狄伯支率四萬騎再攻夏,但姚沖竟圖謀反,並殺了不肯支持的狄伯支,姚興終賜死姚沖。同年,姚興親自率軍攻夏,至貮城後就派姚詳、斂曼嵬及彭白狼分督租運。其時諸軍未集合,而赫連勃勃乘虛來襲,姚興恐懼之下打算逃到姚詳那裏,但被右僕射韋華勸止。姚興派姚文宗等迎戰,雖將領姚榆生被擒,但在姚文宗力戰之下也成功擊退赫連勃勃。姚興唯有留五千禁軍助姚詳守貮城,自己撤還長安。

赫連勃勃攻破了敕奇堡、黃石固及我羅城。次年又派胡金纂攻平涼,雖然姚興親自率軍擊殺胡金纂,但赫連勃勃侄赫連羅提又攻下定陽,殺四千多人並俘姚廣都。當時秦將曹熾、曹雲及王肆佛等被逼領數千戶內徙,姚興就讓他們住在湟山及陳倉。接著赫連勃勃又進攻隴,攻略陽太守姚壽都守的清水城,姚壽都棄城奔上邽,而赫連勃勃就遷了城中一萬六千戶人到大城。姚興試圖從安定追擊赫連勃勃,但追不上。及後赫連勃勃仍屢屢侵擾後秦,但姚興都無法消滅夏國。

姚興子廣平公姚弼得父親寵愛,任雍州刺史,出鎮安定時天水人姜紀接近姚弼,並勸他巴結姚興左右以望還朝,姚弼於是巴結常山公姚顯。至弘始十三年(411年),姚興就召了姚弼回長安,讓他為尚書令、侍中、大將軍。姚弼於是擔當將相要職,更心引見人才,收結朝士,形成了一股比太子姚泓更大的勢力,更有圖取其太子之位。後來姚弼因為厭惡姚泓親信姚文宗,就誣陷他有所怨言,並讓侍御史廉桃生作證。姚興信以為真,一怒之下就賜死姚文宗。朝中大臣於是都不敢再說姚弼不是了。

因著對姚弼的寵愛,姚興對姚弼幾乎言聽計從,於是機要職位都由姚弼親信出任。當時右僕射梁喜、侍中任謙及京兆尹尹昭就找機會向姚興表示姚弼有奪嫡的志向,指出姚興不當的寵愛他,令傾險無賴的人都在其身邊,又說民間都說姚興有廢立之意,三人同時表示反對易儲。姚興立即否認有易儲計劃,三人就是更勸姚興削減姚弼權力並除去其身邊黨羽,既保姚弼,亦保國家。姚興聽後就沉默不言。

弘始十六年(414年),姚興病重,太子姚泓屯兵東華門,並在諮議堂侍疾。當時姚弼卻意圖作亂,招集了數千人並藏匿在其府中。姚裕當時與任謙、梁喜等人都掌禁軍守衞皇宮,而姚裕就派使者將姚弼謀反的行狀告知各個外藩,於是駐蒲阪的姚懿、洛陽的姚洸及雍城的姚諶都將要領兵入長安討伐姚弼。此時姚興病情好轉,召見了群臣,征虜將軍劉羌向姚興泣告姚弼謀反之事,尹昭等都建議姚興即使不按法處死,也應削其權力,讓他散居藩國。姚興仍然欣賞才兼文武的姚弼,不忍殺他,於是免去其尚書令職位,以大將軍、廣平公身份還第。

及後姚懿、姚洸、姚宣及姚諶來朝,見面時姚宣哭請姚興按法處置姚弼,但姚興拒絕。撫軍東曹屬姜虬也上書指姚弼雖然被姑息,但其黨羽仍然活躍,姚弼變亂的心是不會變的,更請消除姚弼黨羽,以絕禍根。姚興就問梁喜:「天下的人全都以我兒子作為口實,要如何處理?」梁喜則說:「真的如姜虬所言,陛下應該早點有個決定。」姚興又沉默不言。

弘始十七年(415年),姚弼知姚宣在父親面前說自己不是,十分憤恨,於是就向姚興誣陷姚宣。姚興又相信,並召見當時到了長安的姚宣司馬權丕,責怪他沒有好好匡輔姚宣並要處死他。但權丕竟然捏造了姚宣的罪狀報告姚興。姚興於是大怒,收捕了姚宣並派姚弼率兵三萬出鎮秦州。尹昭知道後向姚興指讓姚弼統大軍在外,一旦姚興去世,就會是太子姚泓的大大威脅,試圖勸止姚興,但姚興不聽。

同年,姚興食五石散中毒,姚弼卻稱病不朝,又再次在府中招集兵眾。姚興知道後大怒,殺了姚弼黨羽殿中侍御史唐盛及孫元。姚泓卻在怪責自己,請姚興殺了他,或處之外藩。姚興於是召了姚讚、梁喜、尹昭及斂曼嵬,和他們討論後囚禁了姚弼並準備殺了他,又要將姚弼黨羽全部治罪。不過姚泓請命之下,都將他們寛恕。

弘始十八年(416年),姚興出行華陰,留姚泓監國。及後姚興病重回長安,姚弼黨羽尹沖等仍想發難,想趁姚泓出迎姚興而將其殺害,但姚泓只在黃龍門拜迎。其時尚書姚沙彌更意圖劫奪姚興到廣平公府,以姚興招引眾人支持,從而從姚泓手中奪去儲君之位。尹沖雖不從,但仍然考慮隨姚興乘輿入宮中作亂,只是未知姚興生死而不敢行動。姚興則命姚泓錄尚書事,並命姚紹及胡翼度掌禁軍,又命斂曼嵬收去姚弼府中的武器。

不久,姚興病情更趨嚴重,姚興妹南安長公主去探望他也得不到回應,姚興幼子姚耕兒就向哥哥姚愔報告姚興已死,叫他快點做決定。姚愔於是就帶其他的士兵攻端門,斂曼嵬領兵抵禦,而胡翼度就關上宮中四門。姚愔派壯士爬上門並進入宮內,並走到馬道。時在諮議堂侍疾的姚泓命斂曼嵬登武庫抵禦,而太子右衞率姚和都亦已率東宮士兵在馬道南駐屯。姚愔無法前進,只得燒毀端門。姚興此時竭力走到前殿,並下令賜死姚弼。禁軍見到姚興士氣大振,向姚愔軍發動進攻,姚和都也在後夾擊,最終姚愔軍潰敗,姚愔逃到驪山,呂隆則逃到雍城,尹沖及尹泓就南奔東晉。

姚興召姚紹、姚讚、梁喜、尹昭和斂曼嵬入寢宮,遺命他們為輔政大臣。姚興即逝世,享年五十一歲。諡文桓皇帝,庙号高祖,下葬偶陵。

姚興曾命各郡國每年都上報一個品行純潔的孝廉,又留心政事,廣納百言,包容各種意見。即使只是說了一句姚興認為有益的建言,都會得到特別禮待。如杜瑾、吉默和周寶就曾因向姚興陳述當時國中大事而獲授要職。姚興又重文教,當時有姜龕、淳于岐及郭高等有大德的老儒士在長安教學,各有數百門生,其中有不少門生更遠道而來。而姚興就在處理政務以外的時間請姜龕等到東堂和他談論學問和技藝。當時一叫胡辯的人在當時仍是東晉佔領的洛陽授學,很多關中人都去拜師,姚興更下令各關守長盡量方便這些求學的人出入。種種措施都令後秦儒學興盛。

皇初九年(399年),姚興以国内天灾频频,於是自降帝号,称秦王;另又下令郡國將因戰亂而賣身為奴婢的人變回良人,更將幾個貪財官員誅殺,整頓官員風氣。及後又在長安建立法律學校,讓各郡縣散吏入讀,學成者就送回郡縣以處理形獄事項,又下令郡縣無法裁決的都上交廷尉處理。姚興更經常到諮議堂聽訟和作判決,大大減少了冤獄。

弘始三年(401年)呂隆向後秦請降後,姚興就迎在後涼的僧人鳩摩羅什入長安,並奉其為國師,奉之如神。鳩摩羅什在長安組織了大規模的翻譯佛經事業,姚興亦信了佛,於是群下都跟著信奉佛教,又吸引了五千多個僧人遠道而來。姚興又在永貴里建了佛塔、在中宮建了波若臺,佛教興盛,各州郡都受到佛教影響,至「求佛者十室而九。」

姚興在位後期,國庫不足,曾增加關隘和渡口的稅,又向鹽、竹、山林和木材徵稅。群臣曾勸諫但姚興認為能夠常出入關隘及取利於山水資源的都是富人,現在增稅其實只是取富人多餘的而彌補國家不足,並無不妥。

姚興生性儉約,所乘車馬都沒有黃金或玉石裝飾,以身作則之下,群下都崇尚清正廉潔。不過姚興卻喜歡打獵,常傷及農作物。杜挻及相雲曾分別作《豐草詩》及《德獵賦》以作暗示,姚興雖然明白並以黃金及布帛作賞賜,但始終改變不了打獵的習慣。

每當大臣去世,姚興都不會只按慣例在東堂發哀,而會親身去臨喪。

姚興十分看重親族,更對兩名叔叔姚碩德及姚緒十分恭敬。姚興降號為王時,本為王爵的姚碩德及姚緒應當降為公爵,但姚興不允,在二人再三辭讓下才得允許。姚興又曾下令所有官員取名時不得犯二人名諱,所有車馬、衣服及器玩都先給二人,自己只用次一等的,見面時行家人之禮,朝中大事亦必定先諮詢二人。姚沖叛變不遂殺了顧命大臣之一的狄伯支,姚興仍然顧念他是最小的弟弟,雄武絕人,還想對他寬容一次,不過在斂成規勸下,姚興想到他殺了狄伯支,就下書賜死姚沖。

\subsubsection{皇初}

\begin{longtable}{|>{\centering\scriptsize}m{2em}|>{\centering\scriptsize}m{1.3em}|>{\centering}m{8.8em}|}
  % \caption{秦王政}\
  \toprule
  \SimHei \normalsize 年数 & \SimHei \scriptsize 公元 & \SimHei 大事件 \tabularnewline
  % \midrule
  \endfirsthead
  \toprule
  \SimHei \normalsize 年数 & \SimHei \scriptsize 公元 & \SimHei 大事件 \tabularnewline
  \midrule
  \endhead
  \midrule
  元年 & 394 & \tabularnewline\hline
  二年 & 395 & \tabularnewline\hline
  三年 & 396 & \tabularnewline\hline
  四年 & 397 & \tabularnewline\hline
  五年 & 398 & \tabularnewline\hline
  六年 & 399 & \tabularnewline
  \bottomrule
\end{longtable}

\subsubsection{弘始}

\begin{longtable}{|>{\centering\scriptsize}m{2em}|>{\centering\scriptsize}m{1.3em}|>{\centering}m{8.8em}|}
  % \caption{秦王政}\
  \toprule
  \SimHei \normalsize 年数 & \SimHei \scriptsize 公元 & \SimHei 大事件 \tabularnewline
  % \midrule
  \endfirsthead
  \toprule
  \SimHei \normalsize 年数 & \SimHei \scriptsize 公元 & \SimHei 大事件 \tabularnewline
  \midrule
  \endhead
  \midrule
  元年 & 399 & \tabularnewline\hline
  二年 & 400 & \tabularnewline\hline
  三年 & 401 & \tabularnewline\hline
  四年 & 402 & \tabularnewline\hline
  五年 & 403 & \tabularnewline\hline
  六年 & 404 & \tabularnewline\hline
  七年 & 405 & \tabularnewline\hline
  八年 & 406 & \tabularnewline\hline
  九年 & 407 & \tabularnewline\hline
  十年 & 408 & \tabularnewline\hline
  十一年 & 409 & \tabularnewline\hline
  十二年 & 410 & \tabularnewline\hline
  十三年 & 411 & \tabularnewline\hline
  十四年 & 412 & \tabularnewline\hline
  十五年 & 413 & \tabularnewline\hline
  十六年 & 414 & \tabularnewline\hline
  十七年 & 415 & \tabularnewline\hline
  十八年 & 416 & \tabularnewline
  \bottomrule
\end{longtable}

%%% Local Variables:
%%% mode: latex
%%% TeX-engine: xetex
%%% TeX-master: "../../Main"
%%% End:

%% -*- coding: utf-8 -*-
%% Time-stamp: <Chen Wang: 2019-12-19 15:09:10>

\subsection{姚泓\tiny(416-417)}

\subsubsection{生平}

姚泓(388年-417年),字元子,十六国时期后秦末主,后秦文桓帝姚兴长子。

后秦弘始十八年(416年)正月,姚兴卒,姚泓即位。兄弟相争,国中大乱。八月东晋刘裕起兵伐秦。后秦疲于应敌之际,国内又相继发生姚懿、姚恢的叛乱。

次年八月癸亥(417年9月20日),刘裕帐下大将王镇恶攻入长安平朔门。姚泓无计可出,准备出降,他十一岁的儿子姚佛念说,晋朝人“将逞其欲”,(即使投降)我们也一定不能保全自己,我愿自杀。姚泓怃然不知所对。佛念登上城墙自投而死。姚泓率一家老小至王镇恶大營投降,其堂叔姚赞也率宗室子弟一百余人投降。刘裕將後秦王室全部处死,其余宗族成员迁往江南。姚泓被押往建康斩首。后秦亡。

《晋书》载,泓孝友宽和,而无经世之用,又多疾病,兴将以为嗣而疑焉。久之,乃立为太子。兴每征伐巡游,常留总后事。博学善谈论,尤好诗咏。

\subsubsection{永和}

\begin{longtable}{|>{\centering\scriptsize}m{2em}|>{\centering\scriptsize}m{1.3em}|>{\centering}m{8.8em}|}
  % \caption{秦王政}\
  \toprule
  \SimHei \normalsize 年数 & \SimHei \scriptsize 公元 & \SimHei 大事件 \tabularnewline
  % \midrule
  \endfirsthead
  \toprule
  \SimHei \normalsize 年数 & \SimHei \scriptsize 公元 & \SimHei 大事件 \tabularnewline
  \midrule
  \endhead
  \midrule
  元年 & 416 & \tabularnewline\hline
  二年 & 417 & \tabularnewline
  \bottomrule
\end{longtable}


%%% Local Variables:
%%% mode: latex
%%% TeX-engine: xetex
%%% TeX-master: "../../Main"
%%% End:



%%% Local Variables:
%%% mode: latex
%%% TeX-engine: xetex
%%% TeX-master: "../../Main"
%%% End:
