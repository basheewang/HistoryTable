%% -*- coding: utf-8 -*-
%% Time-stamp: <Chen Wang: 2021-11-01 12:01:59>

\subsection{宣烈王乞伏国仁\tiny(385-388)}

\subsubsection{生平}

乞伏國仁(?-388年),陇西鲜卑人。十六国时期西秦政權奠定者。在前秦官至前將軍,淝水之戰後乘機自立,但仍與前秦保持一定關係。雖然一般認為乞伏國仁是西秦建立者,惟其在位期間,只受前秦封為苑川王,尚未正式稱秦王。一直至394年,國仁繼承人乞伏乾歸才稱秦王。

其父乞伏司繁受前秦天王苻堅封為南單于,並駐鎮勇士川(今甘肅榆中)。秦建元十二年(376年),司繁死,乞伏國仁繼位。前秦建元十九年(383年)淝水之戰時,苻堅原命國仁为前将军,领先锋骑,後國仁叔父乞伏步頹叛于陇西,苻堅派國仁回師討伐,步頹反而迎接國仁。及前秦淝水之戰失利,國仁即趁機吞併其他部族,聚眾共十多萬。前秦太安元年(385年),苻堅為姚萇所殺後,國仁自称大都督、大将军、大单于、领秦、河二州牧,改元建义,建都勇士城(今甘肅榆中)。

就在乞伏國仁自立次年,南安郡豪族祕宜就率領五萬羌、胡人進攻乞伏國仁,並四面來攻。乞伏國仁決意先聲奪人,於是自率五千人突襲祕宜,並大敗對方。祕宜於是逃奔南安郡,同年便率三萬多戶人口歸降。建義三年(387年),前秦皇帝苻登以乞伏國仁為大都督、都督雜夷諸軍事、大將軍、大單于、苑川王。同年乞伏國仁率軍進攻密貴、裕苟及提倫三位鮮卑大人,又大敗來攻的高平鮮卑首領沒弈干及東胡金熙,密貴等三人於是大懼,率部歸降。建義四年,乞伏國仁又擊敗了鮮卑人越質叱黎。同年国仁死,谥宣烈王,庙号烈祖,弟乞伏乾歸繼位。

\subsubsection{建义}

\begin{longtable}{|>{\centering\scriptsize}m{2em}|>{\centering\scriptsize}m{1.3em}|>{\centering}m{8.8em}|}
  % \caption{秦王政}\
  \toprule
  \SimHei \normalsize 年数 & \SimHei \scriptsize 公元 & \SimHei 大事件 \tabularnewline
  % \midrule
  \endfirsthead
  \toprule
  \SimHei \normalsize 年数 & \SimHei \scriptsize 公元 & \SimHei 大事件 \tabularnewline
  \midrule
  \endhead
  \midrule
  元年 & 385 & \tabularnewline\hline
  二年 & 386 & \tabularnewline\hline
  三年 & 387 & \tabularnewline\hline
  四年 & 388 & \tabularnewline
  \bottomrule
\end{longtable}


%%% Local Variables:
%%% mode: latex
%%% TeX-engine: xetex
%%% TeX-master: "../../Main"
%%% End:
