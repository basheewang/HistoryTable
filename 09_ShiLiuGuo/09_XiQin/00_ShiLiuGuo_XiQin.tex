%% -*- coding: utf-8 -*-
%% Time-stamp: <Chen Wang: 2019-12-19 15:37:04>


\section{西秦\tiny(385-431)}

\subsection{简介}

西秦(385年-400年,409年-431年)是中国历史上十六国时期鲜卑人乞伏國仁建立的政权。其国号“秦”以地处战国时秦国故地为名。《十六国春秋》始用西秦之称,以别于前秦和后秦,后世袭用之。

公元385年,鲜卑酋长乞伏国仁在陇西称大单于,又被前秦封为苑川王,都勇士川(今甘肃榆中)。388年,其弟乞伏乾歸立,称大单于,河南王,迁都金城(今甘肃兰州西)。400年為後秦所滅。409年,二月,乞伏乾归自后秦返回苑川。七月,西秦复国,复都苑川。412年,乞伏熾磐又迁都枹罕(今甘肃临夏市东北)。

最盛时期,其统治范围包括甘肃西南部,青海部分地区。

431年被夏國所灭。

%% -*- coding: utf-8 -*-
%% Time-stamp: <Chen Wang: 2019-12-19 15:40:51>

\subsection{乞伏国仁\tiny(385-388)}

\subsubsection{生平}

乞伏國仁(?-388年),陇西鲜卑人。十六国时期西秦政權奠定者。在前秦官至前將軍,淝水之戰後乘機自立,但仍與前秦保持一定關係。雖然一般認為乞伏國仁是西秦建立者,惟其在位期間,只受前秦封為苑川王,尚未正式稱秦王。一直至394年,國仁繼承人乞伏乾歸才稱秦王。

其父乞伏司繁受前秦天王苻堅封為南單于,並駐鎮勇士川(今甘肅榆中)。秦建元十二年(376年),司繁死,乞伏國仁繼位。前秦建元十九年(383年)淝水之戰時,苻堅原命國仁为前将军,领先锋骑,後國仁叔父乞伏步頹叛于陇西,苻堅派國仁回師討伐,步頹反而迎接國仁。及前秦淝水之戰失利,國仁即趁機吞併其他部族,聚眾共十多萬。前秦太安元年(385年),苻堅為姚萇所殺後,國仁自称大都督、大将军、大单于、领秦、河二州牧,改元建义,建都勇士城(今甘肅榆中)。

就在乞伏國仁自立次年,南安郡豪族祕宜就率領五萬羌、胡人進攻乞伏國仁,並四面來攻。乞伏國仁決意先聲奪人,於是自率五千人突襲祕宜,並大敗對方。祕宜於是逃奔南安郡,同年便率三萬多戶人口歸降。建義三年(387年),前秦皇帝苻登以乞伏國仁為大都督、都督雜夷諸軍事、大將軍、大單于、苑川王。同年乞伏國仁率軍進攻密貴、裕苟及提倫三位鮮卑大人,又大敗來攻的高平鮮卑首領沒弈干及東胡金熙,密貴等三人於是大懼,率部歸降。建義四年,乞伏國仁又擊敗了鮮卑人越質叱黎。同年国仁死,谥宣烈王,庙号烈祖,弟乞伏乾歸繼位。

\subsubsection{建义}

\begin{longtable}{|>{\centering\scriptsize}m{2em}|>{\centering\scriptsize}m{1.3em}|>{\centering}m{8.8em}|}
  % \caption{秦王政}\
  \toprule
  \SimHei \normalsize 年数 & \SimHei \scriptsize 公元 & \SimHei 大事件 \tabularnewline
  % \midrule
  \endfirsthead
  \toprule
  \SimHei \normalsize 年数 & \SimHei \scriptsize 公元 & \SimHei 大事件 \tabularnewline
  \midrule
  \endhead
  \midrule
  元年 & 385 & \tabularnewline\hline
  二年 & 386 & \tabularnewline\hline
  三年 & 387 & \tabularnewline\hline
  四年 & 388 & \tabularnewline
  \bottomrule
\end{longtable}


%%% Local Variables:
%%% mode: latex
%%% TeX-engine: xetex
%%% TeX-master: "../../Main"
%%% End:

%% -*- coding: utf-8 -*-
%% Time-stamp: <Chen Wang: 2019-12-19 15:41:44>

\subsection{武元王\tiny(388-412)}

\subsubsection{生平}

秦武元王乞伏乾歸(?-412年),陇西鲜卑人。十六国时期西秦開國君王,苑川王乞伏國仁弟。乾歸在位初期曾受前秦官爵,並曾響應前秦號召領兵協助,但皇帝苻登敗死後就逼逐繼承的苻崇,後苻崇討伐乾歸時更敗死,令前秦亡國,並乘機併吞其隴西土地,後稱「秦王」,西秦故此得名。後乾歸敗給後秦,被逼投降南涼,最終向後秦歸降,暫時亡國。但因後秦王姚興將其放回原地,並將部眾還給他,令其有機會復興,最終趁後秦漸漸衰弱時復國,並進攻鄰近的南涼、後秦、吐谷渾及其他胡人部落。乞伏乾歸於412年被侄兒乞伏公府所殺,其太子乞伏熾磐討平後繼位。

建义元年(385年)乞伏國仁自称大都督、大将军、单于,领秦、河二州牧。任命乾歸為上將軍。建義四年(388年)國仁去世,群臣認為國仁子乞伏公府年幼,乃推乾归为大都督、大将军、大单于、河南王,改元太初,遷都金城(今甘肅蘭州)。太初二年(389年)受前秦帝苻登封為金城王。

乾歸於太初二年(389年)即討平了休官部落的阿敦及侯年二部,盡降其眾,於是威振西部,鮮卑的豆留螱奇、叱豆渾、南丘鹿結、休官部的曷呼奴及盧水尉地跋都率眾歸降,而乾歸亦各署官爵;枹罕羌彭奚念亦來歸附,乾歸以其為北河州刺史。次年(390年),吐谷渾亦遣使上貢,乾歸又以吐谷渾君主視連為白蘭王、沙州牧。

太初四年(391年),沒弈干遣使結好,並派兩個兒子為人質請兵一共進攻鮮卑大兜,乾歸答允並領兵進攻大兜的安陽城,大兜退守鳴蟬堡但還是被乾歸攻陷,乾歸於是收擄其部眾回國。戰後乾歸歸還了沒弈干的兩個兒子,但沒弈干不久又改結劉衞辰,乾歸於是率兵一萬攻伐沒弈干,並在他樓城射傷沒弈干的眼睛。

太初七年(394年),苻登知後秦皇帝姚萇去世,認為滅後秦時機已到,於是起兵進攻後秦,又拜乾歸為左丞相、河南王、領秦梁益涼沙五州牧,加賜九錫。可是苻登卻遭姚萇太子姚興擊敗,退屯馬毛山,並派了兒子苻宗為質子,向乾歸請兵,並進封乾歸為梁王。乾歸於是派了乞伏益州率兵一萬營救,但苻登要出迎乞伏益州時被姚興擊敗,更被俘殺。苻登太子苻崇於湟中繼位,但不久乾歸就驅逐苻崇,苻崇只好投奔氐族仇池部隴西王楊定。二人組成聯軍反攻乾歸,乾歸派兵抵抗,終擊敗聯軍,斬楊定及苻崇,前秦滅亡,西秦自此盡有隴西。不久,乾歸自稱秦王,又於次年(395年)遷都苑川西城(今甘肅靖遠)。

早於太初五年(392年),呂光就曾派呂方及呂寶進攻乾歸,乾歸初敗於鳴雀峽,退屯青岸。而呂方屯黃河北,呂寶則渡河追擊,乾歸於是派彭奚念斷絕呂寶歸路,率兵反擊,屢敗呂寶,終呂寶等一萬多人戰死。至太初八年(395年),呂光親自率十萬軍進攻乾歸,左輔將軍密貴周及莫者羖羝就勸乾歸向呂光稱藩,乾歸終聽從並以兒子乞伏敕勃作為人質,呂光亦率軍退還。可是不久乾歸就後悔了,殺了密貴周及莫者羖羝。

太初九年(396年),涼州牧乞伏軻彈因與秦州牧乞伏益州不睦,故出奔呂光,呂光於是以乾歸多次反覆而興兵討伐。其時眾臣都請乾歸出奔成紀迴避,但乾歸不願。呂光派呂延等人攻下了臨洮、武始、河關,又命呂纂進攻金城,乾歸率兵救援,但呂光派了王寶及徐炅率兵五千邊擊,令乾歸恐懼不敢前進,終令金城陷落。乾歸於是行反間計,傳出假消息稱乾歸部眾已潰散,乾歸已東逃到成紀。呂延信以為真,於是輕軍進攻,最終被乾歸擊敗,呂延更戰死。呂延敗後,呂光亦退兵。

太初三年(390年),視連去世,視羆繼位,拒絕接受乾歸的封號。乾歸知道後大怒,但因為忌憚吐谷渾強盛,於是暫時容忍,仍然交好。至太初十一年(398年)就派了乞伏益州、慕兀及翟瑥率二萬騎進攻吐谷渾,在度周川大敗視羆,逼其送兒子宕豈為質求和。

太初十三年(400年),乾歸復遷都苑川(今甘肅榆中縣北)。同年,後秦姚碩德來攻,乾歸率眾到隴西對抗。兩軍對峙期間,姚碩德軍柴草缺乏,後秦王姚興於是親自出軍。乾歸見已是國家存亡的危機,於是放手一搏,決定集中力量消滅姚興軍隊,殺死姚興,欲求消除危機之餘更吞併後秦。乾歸因而命慕兀率二萬兵為中軍,駐柏楊(今甘肅清水縣西南);羅敦率四萬兵為外軍,駐侯辰谷。而自己就率數千騎等候姚興軍。但一晚,乾歸遇上大風和大霧,與中軍失去聯絡,被逼與外軍會合。天亮後乾歸就與姚興軍交戰,大敗。乾歸敗歸苑川,接著又逃到金城,並命手下各豪帥留下來歸降後秦,自己西走允吾(今甘肅皋蘭縣西北),望一天復興國家時再見。西秦滅亡。乾歸到允吾後向禿髮利鹿孤投降,被禿髮傉檀迎到晉興,待以上賓之禮。

後秦退兵後,南羌梁戈等人招引乾歸,乾歸打算前赴,但事情卻洩漏給禿髮利鹿孤知道,禿髮吐雷因而出屯捫天嶺。乾歸恐為禿髮利鹿孤所殺,於是送妻子及乞伏熾磐等諸子到西平為人質,自己出奔枹罕(今甘肅臨夏市),向後秦投降。

乾歸到長安後,受封為持节、都督河南诸军事、镇远将军、河州刺史、归义侯,隔年(401年)更被派還西秦故都苑川鎮守,並歸還其部眾。至後秦弘始四年(402年),乞伏熾磐逃奔後秦,姚興也授他官位,不久更加乾歸散騎常侍、左賢王。乾歸於降後秦時期,曾經受命與齊難等後秦將領到姑臧(今甘肅武威)接受後涼王呂隆投降。乾歸又屢攻仇池,先後攻破仇池所領的皮氏堡和西陽堡。乾歸更於405年攻破吐谷渾,其中吐谷渾君主大孩更在敗走後不久去世,乾歸俘擄了一萬多人。

弘始九年(407年),姚興認為乾归的勢力逐漸強大,難以控制,於是趁其入朝的機會將其留在長安當主客尚書,讓其子乞伏熾磐代領其眾。弘始十一年(409年),乞伏熾磐攻伐彭奚念,攻陷其佔領的枹罕。其時乾歸正隨姚興在平涼,得到熾磐的通報後就逃回苑川。乾歸回去後不久到枹罕聚集三萬部眾,並帶他們遷居度堅山,留熾磐守枹罕,接著乾歸更稱秦王,改元「更始」,再次置官爵並讓手下恢復原來在西秦的職位,正式復國。

乾歸復國後,先派兵進攻薄地延,將其部落遷至苑川,後又派兵攻下後秦的金城郡,並置守戍,從而於更始二年(410年)遷都回苑川。略陽、南安、隴西等後秦轄郡都先後遭西秦軍攻下。當時後秦無力討伐,只得任命乾歸為使持節、散騎常侍、都督隴西北匈奴雜胡諸軍事征西大將軍、河州牧、大單于、河南王。乾歸當時正欲攻取河西地區,於是暫時接受。

乾歸又派兵攻伐南涼,擊敗了南涼太子禿髮虎台。另又率兵攻下後秦略陽太守姚龍的柏龍堡及南平太守王憬的水洛城。後又攻殺襲據枹罕的彭利髮,收復了枹罕。更始四年(412年),乾歸更率二萬騎攻破吐谷渾支統阿若干,令吐谷渾向其投降。

同年六月,乾歸為其侄乞伏公府所弒,十余个儿子一并遇害。乞伏熾磐消滅乞伏公府後繼位,諡乾歸為武元王,庙號高祖,葬於枹罕。

\subsubsection{太初}

\begin{longtable}{|>{\centering\scriptsize}m{2em}|>{\centering\scriptsize}m{1.3em}|>{\centering}m{8.8em}|}
  % \caption{秦王政}\
  \toprule
  \SimHei \normalsize 年数 & \SimHei \scriptsize 公元 & \SimHei 大事件 \tabularnewline
  % \midrule
  \endfirsthead
  \toprule
  \SimHei \normalsize 年数 & \SimHei \scriptsize 公元 & \SimHei 大事件 \tabularnewline
  \midrule
  \endhead
  \midrule
  元年 & 388 & \tabularnewline\hline
  二年 & 389 & \tabularnewline\hline
  三年 & 390 & \tabularnewline\hline
  四年 & 391 & \tabularnewline\hline
  五年 & 392 & \tabularnewline\hline
  六年 & 393 & \tabularnewline\hline
  七年 & 394 & \tabularnewline\hline
  八年 & 395 & \tabularnewline\hline
  九年 & 396 & \tabularnewline\hline
  十年 & 397 & \tabularnewline\hline
  十一年 & 398 & \tabularnewline\hline
  十二年 & 399 & \tabularnewline\hline
  十三年 & 400 & \tabularnewline
  \bottomrule
\end{longtable}

\subsubsection{更始}

\begin{longtable}{|>{\centering\scriptsize}m{2em}|>{\centering\scriptsize}m{1.3em}|>{\centering}m{8.8em}|}
  % \caption{秦王政}\
  \toprule
  \SimHei \normalsize 年数 & \SimHei \scriptsize 公元 & \SimHei 大事件 \tabularnewline
  % \midrule
  \endfirsthead
  \toprule
  \SimHei \normalsize 年数 & \SimHei \scriptsize 公元 & \SimHei 大事件 \tabularnewline
  \midrule
  \endhead
  \midrule
  元年 & 409 & \tabularnewline\hline
  二年 & 410 & \tabularnewline\hline
  三年 & 411 & \tabularnewline\hline
  四年 & 412 & \tabularnewline
  \bottomrule
\end{longtable}


%%% Local Variables:
%%% mode: latex
%%% TeX-engine: xetex
%%% TeX-master: "../../Main"
%%% End:

%% -*- coding: utf-8 -*-
%% Time-stamp: <Chen Wang: 2021-11-01 12:02:19>

\subsection{文昭王乞伏熾磐\tiny(412-428)}

\subsubsection{生平}

文昭王乞伏熾磐(?-428年),十六国时期西秦国君主,乞伏乾歸長子。

熾磐個性勇略過人,400年,西秦第一次亡國後,被送往南涼為人質。後秦弘始四年(402年)熾磐自南涼奔後秦與乾歸會合。熾磐於後秦期間,召集軍隊據地自立。弘始十一年(409年)乾歸逃回西秦舊地,再稱秦王,西秦復國,熾磐又被立為太子。西秦更始四年(412年)乾歸為侄乞伏公府所弒,熾磐擒殺公府,繼位,稱河南王,改元永康。永康三年(414年)滅南涼,復稱秦王,其後主要與北涼爭戰。建弘九年(428年)病死,諡文昭王,廟號太祖。其子乞伏暮末继位。

\subsubsection{永康}

\begin{longtable}{|>{\centering\scriptsize}m{2em}|>{\centering\scriptsize}m{1.3em}|>{\centering}m{8.8em}|}
  % \caption{秦王政}\
  \toprule
  \SimHei \normalsize 年数 & \SimHei \scriptsize 公元 & \SimHei 大事件 \tabularnewline
  % \midrule
  \endfirsthead
  \toprule
  \SimHei \normalsize 年数 & \SimHei \scriptsize 公元 & \SimHei 大事件 \tabularnewline
  \midrule
  \endhead
  \midrule
  元年 & 412 & \tabularnewline\hline
  二年 & 413 & \tabularnewline\hline
  三年 & 414 & \tabularnewline\hline
  四年 & 415 & \tabularnewline\hline
  五年 & 416 & \tabularnewline\hline
  六年 & 417 & \tabularnewline\hline
  七年 & 418 & \tabularnewline\hline
  八年 & 419 & \tabularnewline
  \bottomrule
\end{longtable}

\subsubsection{建弘}

\begin{longtable}{|>{\centering\scriptsize}m{2em}|>{\centering\scriptsize}m{1.3em}|>{\centering}m{8.8em}|}
  % \caption{秦王政}\
  \toprule
  \SimHei \normalsize 年数 & \SimHei \scriptsize 公元 & \SimHei 大事件 \tabularnewline
  % \midrule
  \endfirsthead
  \toprule
  \SimHei \normalsize 年数 & \SimHei \scriptsize 公元 & \SimHei 大事件 \tabularnewline
  \midrule
  \endhead
  \midrule
  元年 & 420 & \tabularnewline\hline
  二年 & 421 & \tabularnewline\hline
  三年 & 422 & \tabularnewline\hline
  四年 & 423 & \tabularnewline\hline
  五年 & 424 & \tabularnewline\hline
  六年 & 425 & \tabularnewline\hline
  七年 & 426 & \tabularnewline\hline
  八年 & 427 & \tabularnewline\hline
  九年 & 428 & \tabularnewline
  \bottomrule
\end{longtable}


%%% Local Variables:
%%% mode: latex
%%% TeX-engine: xetex
%%% TeX-master: "../../Main"
%%% End:

%% -*- coding: utf-8 -*-
%% Time-stamp: <Chen Wang: 2019-12-19 15:45:01>

\subsection{乞伏暮末\tiny(428-431)}

\subsubsection{生平}

乞伏暮末(-431年),一名慕末,十六国时期西秦国君主,乞伏熾磐二子。

西秦建弘九年(428年)熾磐去世,暮末繼秦王位,改元永弘。暮末在位期間濫刑好殺,於是人心思叛。永弘三年(430年)因受北涼所迫,暮末擬歸附北魏,未料為夏國所阻。永弘四年(431年)夏國攻西秦都城南安,暮末出降,西秦亡。不久,暮末為夏國皇帝赫連定所殺。

\subsubsection{永弘}

\begin{longtable}{|>{\centering\scriptsize}m{2em}|>{\centering\scriptsize}m{1.3em}|>{\centering}m{8.8em}|}
  % \caption{秦王政}\
  \toprule
  \SimHei \normalsize 年数 & \SimHei \scriptsize 公元 & \SimHei 大事件 \tabularnewline
  % \midrule
  \endfirsthead
  \toprule
  \SimHei \normalsize 年数 & \SimHei \scriptsize 公元 & \SimHei 大事件 \tabularnewline
  \midrule
  \endhead
  \midrule
  元年 & 428 & \tabularnewline\hline
  二年 & 429 & \tabularnewline\hline
  三年 & 430 & \tabularnewline\hline
  四年 & 431 & \tabularnewline
  \bottomrule
\end{longtable}


%%% Local Variables:
%%% mode: latex
%%% TeX-engine: xetex
%%% TeX-master: "../../Main"
%%% End:


%%% Local Variables:
%%% mode: latex
%%% TeX-engine: xetex
%%% TeX-master: "../../Main"
%%% End:
