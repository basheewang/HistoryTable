%% -*- coding: utf-8 -*-
%% Time-stamp: <Chen Wang: 2021-11-01 14:52:45>

\subsection{武威武王禿髮烏孤\tiny(397-399)}

\subsubsection{生平}

武威武王\xpinyin*{禿髮烏孤}(4世紀-399年),河西鮮卑人,十六国时期南涼政權建立者。禿髮鮮卑首領,稱武威王,其父禿髮思復鞬亦為禿髮鮮卑族首領。

禿髮思復鞬死後,禿髮烏孤接任禿髮鮮卑首領,他勇猛威武,且有大志,並圖謀奪取時由後涼控制的涼州。禿髮烏孤於是致力發展農業,與鄰邦修好,禮待賢士,以積聚力量。後涼麟嘉六年(394年),涼王吕光封為冠军大将军、河西鲜卑大都统、广武县侯,其部屬石真若留認為當時禿髮烏孤的根基未穩,尚未是後涼的對手,建議禿髮烏孤暫時接受,等待機會。禿髮烏孤因而接受。

次年(395年)破乙弗、折掘二部,並自建廉川堡(今青海民和縣西北)作都城,此後再受後涼封為广武郡公。後涼龍飛元年(396年)呂光稱天王,又遣使署征南大将军、益州牧、左贤王,禿髮乌孤指呂光諸子貪淫,甥子暴虐,以不違百姓之心及不受不義爵位為由拒絕不受。及於次年正式叛後涼自立,自称大都督、大将军、大单于、西平王,改年号太初,並攻克後涼控制的金城(今甘肅蘭州市西北),後更於街亭(今甘肅秦安縣东北)擊敗前來討伐的後涼軍。次年(398年),後秦樂都、湟河、澆河三郡、嶺南羌胡數萬落及後涼將領楊軌、王乞基皆向禿髮烏孤歸降。禿髮烏孤於同年改稱武威王。太初三年(399年),烏孤遷都樂都(今青海樂都)。當時禿髮烏孤選任官員包括了胡人豪族、當地有德望之士、文武才俊、中原遷來的有才之士以及秦雍世族子弟,皆以其才授官。同年禿髮烏孤因酒後坠马伤及肋骨,傷重而死,死前向臣下表示應當立年長新君,故由其弟禿髮利鹿孤繼位。諡號為武王,廟號烈祖。

\subsubsection{太初}

\begin{longtable}{|>{\centering\scriptsize}m{2em}|>{\centering\scriptsize}m{1.3em}|>{\centering}m{8.8em}|}
  % \caption{秦王政}\
  \toprule
  \SimHei \normalsize 年数 & \SimHei \scriptsize 公元 & \SimHei 大事件 \tabularnewline
  % \midrule
  \endfirsthead
  \toprule
  \SimHei \normalsize 年数 & \SimHei \scriptsize 公元 & \SimHei 大事件 \tabularnewline
  \midrule
  \endhead
  \midrule
  元年 & 397 & \tabularnewline\hline
  二年 & 398 & \tabularnewline\hline
  三年 & 399 & \tabularnewline
  \bottomrule
\end{longtable}


%%% Local Variables:
%%% mode: latex
%%% TeX-engine: xetex
%%% TeX-master: "../../Main"
%%% End:
