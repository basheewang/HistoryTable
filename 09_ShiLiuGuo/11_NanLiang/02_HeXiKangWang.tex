%% -*- coding: utf-8 -*-
%% Time-stamp: <Chen Wang: 2019-12-19 16:19:43>

\subsection{河西康王\tiny(399-402)}

\subsubsection{生平}

河西康王禿髮利鹿孤(4世紀?-402年),河西鮮卑人,十六國時期南涼國君主(河西王)。禿髮烏孤之弟。

太初三年(399年),禿髮烏孤遷都樂都(今青海樂都),並署利鹿孤為驃騎大將軍、西平公,駐鎮安夷(今青海平安區)。同年,後涼呂紹及呂纂進攻北涼,禿髮烏孤應北涼王段業求援,命利鹿孤及楊軌率軍救援。呂紹等人最終撤退,利鹿孤就以涼州牧改鎮西平(今青海湟中)。同年禿髮烏孤去世,利鹿孤繼位,就將都城遷至西平。

建和二年(401年)以祥瑞為由打算稱帝,但在安國將軍鍮勿崙的勸喻下改稱河西王。同年率軍攻伐後涼,大敗涼軍,俘獲楊桓及強遷其二千戶人口。建和三年(402年),利鹿孤又派兵攻破魏安,俘獲佔據當地的焦朗。同年利鹿孤去世,諡康王,葬於西平東南。因著利鹿孤父禿髮思復鞬向來疼愛並重視弟禿髮傉檀,而利鹿孤在位期間很多軍國大事都是由禿髮傉檀處理,故就以禿髮傉檀繼位。

\subsubsection{建和}

\begin{longtable}{|>{\centering\scriptsize}m{2em}|>{\centering\scriptsize}m{1.3em}|>{\centering}m{8.8em}|}
  % \caption{秦王政}\
  \toprule
  \SimHei \normalsize 年数 & \SimHei \scriptsize 公元 & \SimHei 大事件 \tabularnewline
  % \midrule
  \endfirsthead
  \toprule
  \SimHei \normalsize 年数 & \SimHei \scriptsize 公元 & \SimHei 大事件 \tabularnewline
  \midrule
  \endhead
  \midrule
  元年 & 400 & \tabularnewline\hline
  二年 & 401 & \tabularnewline\hline
  三年 & 402 & \tabularnewline
  \bottomrule
\end{longtable}


%%% Local Variables:
%%% mode: latex
%%% TeX-engine: xetex
%%% TeX-master: "../../Main"
%%% End:
