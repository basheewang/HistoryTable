%% -*- coding: utf-8 -*-
%% Time-stamp: <Chen Wang: 2021-11-01 14:59:55>

\subsection{惠懿帝高雲\tiny(407-409)}

\subsubsection{生平}

燕惠懿帝高雲(4世纪-409年11月6日),曾改名慕容雲,字子雨,高句驪人。十六国時期後燕末代君主,一說為北燕开国国主,称号天王。

早期的高雲於後燕時沉默寡言,並沒有什麼名氣,只有中衛將軍馮跋看出他的氣度與他結交。

後燕永康二年(397年),高雲因率軍擊敗慕容寶之子慕容會的叛軍,被慕容寶收養,賜姓慕容氏,封夕陽公。

後燕建初元年(407年)馮跋反,殺皇帝慕容熙,在馮跋支持之下,慕容雲即天王位,改元曰正始,國號大燕,恢復原本的高姓。高雲自知無功而登大位,因此培養一批禁衛保護自己,但後來反被禁衛離班和桃仁所殺,高雲死後被諡惠懿皇帝。

由於對高雲是否屬後燕慕容氏一族成員的看法不同,因此有人認為高雲是後燕末任君主,也有人把他視為北燕立國君主。

\subsubsection{正始}

\begin{longtable}{|>{\centering\scriptsize}m{2em}|>{\centering\scriptsize}m{1.3em}|>{\centering}m{8.8em}|}
  % \caption{秦王政}\
  \toprule
  \SimHei \normalsize 年数 & \SimHei \scriptsize 公元 & \SimHei 大事件 \tabularnewline
  % \midrule
  \endfirsthead
  \toprule
  \SimHei \normalsize 年数 & \SimHei \scriptsize 公元 & \SimHei 大事件 \tabularnewline
  \midrule
  \endhead
  \midrule
  元年 & 407 & \tabularnewline\hline
  二年 & 408 & \tabularnewline\hline
  三年 & 409 & \tabularnewline
  \bottomrule
\end{longtable}


%%% Local Variables:
%%% mode: latex
%%% TeX-engine: xetex
%%% TeX-master: "../../Main"
%%% End:
