%% -*- coding: utf-8 -*-
%% Time-stamp: <Chen Wang: 2019-12-19 16:35:44>

\subsection{文成帝\tiny(409-430)}

\subsubsection{生平}

北燕文成帝冯\xpinyin*{跋}(4世紀?-430年),十六國時期北燕君主,字文起,小名乞直伐,是胡化的漢族人,长乐信都(今河北省衡水市冀州区)人。

冯跋是馮和之孫,其父馮安曾任西燕將軍。西燕亡,馮跋東遷後燕,於後燕帝慕容寶在位時被任命為中卫将军。

馮跋與其弟馮素弗先前曾因事獲罪於後燕帝慕容熙,因此慕容熙有殺馮跋兄弟之意,馮跋兄弟遂逃匿深山。馮跋兄弟商量說:「熙今昏虐,兼忌吾兄弟,既還首無路,不可坐受誅滅。當及時而起,立公侯之業。事若不成,死其晚乎!」於是與從兄萬泥等二十二人合謀。後燕建始元年(407年)馮跋兄弟乘車,由婦人禦,潛入都城和龙(今辽宁朝阳),匿於北部司馬孫護家。趁慕容熙送葬苻后之際起事,推高雲(慕容雲)為燕王,改元正始,不久擒殺慕容熙。高雲登位後以馮跋為侍中、征北大將軍、開府儀同三司,封武邑公,政事皆決於馮跋兄弟。

正始三年(409年),高雲為寵臣離班、桃仁所殺,亂事平定後,眾人推馮跋為主,馮跋遂即天王位,改元太平。馮跋勤於政事,獎勵農桑,輕薄徭役,因此人民喜悅,雖外有強大的北魏相逼,也維持20餘年的安定。

北燕太平二十二年(430年),馮跋病重,命太子馮翼攝理國家大事,未料宋夫人有為其子馮受居圖謀王位之意,馮跋之弟馮弘於是帶兵進宮平變,倉促間馮跋於驚懼中去世。後被諡文成皇帝,廟號太祖。冯弘篡位,将包括冯翼在内的冯跋之子一百余人一并杀死。

\subsubsection{太平}

\begin{longtable}{|>{\centering\scriptsize}m{2em}|>{\centering\scriptsize}m{1.3em}|>{\centering}m{8.8em}|}
  % \caption{秦王政}\
  \toprule
  \SimHei \normalsize 年数 & \SimHei \scriptsize 公元 & \SimHei 大事件 \tabularnewline
  % \midrule
  \endfirsthead
  \toprule
  \SimHei \normalsize 年数 & \SimHei \scriptsize 公元 & \SimHei 大事件 \tabularnewline
  \midrule
  \endhead
  \midrule
  元年 & 409 & \tabularnewline\hline
  二年 & 410 & \tabularnewline\hline
  三年 & 411 & \tabularnewline\hline
  四年 & 412 & \tabularnewline\hline
  五年 & 413 & \tabularnewline\hline
  六年 & 414 & \tabularnewline\hline
  七年 & 415 & \tabularnewline\hline
  八年 & 416 & \tabularnewline\hline
  九年 & 417 & \tabularnewline\hline
  十年 & 418 & \tabularnewline\hline
  十一年 & 419 & \tabularnewline\hline
  十二年 & 420 & \tabularnewline\hline
  十三年 & 421 & \tabularnewline\hline
  十四年 & 422 & \tabularnewline\hline
  十五年 & 423 & \tabularnewline\hline
  十六年 & 424 & \tabularnewline\hline
  十七年 & 425 & \tabularnewline\hline
  十八年 & 426 & \tabularnewline\hline
  十九年 & 427 & \tabularnewline\hline
  二十年 & 428 & \tabularnewline\hline
  二一年 & 429 & \tabularnewline\hline
  二二年 & 430 & \tabularnewline
  \bottomrule
\end{longtable}


%%% Local Variables:
%%% mode: latex
%%% TeX-engine: xetex
%%% TeX-master: "../../Main"
%%% End:
