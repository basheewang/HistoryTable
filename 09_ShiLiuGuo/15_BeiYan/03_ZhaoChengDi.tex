%% -*- coding: utf-8 -*-
%% Time-stamp: <Chen Wang: 2019-12-19 16:36:16>

\subsection{昭成帝\tiny(430-436)}

\subsubsection{生平}

北燕昭成帝馮弘(?-438年),十六國時期北燕國君主,字文通,長樂信都(今河北省衡水市冀州区)人,北燕太祖馮跋之弟。

馮跋在位時,馮弘被封中山公司徒錄尚書事,輔政。

馮跋病重時,宋夫人有為其子馮受居圖謀王位之意,馮弘於是帶兵入宮平變,倉促間馮跋於驚懼中去世,馮弘遂即天王位,並下詔書說:「天降凶禍,大行崩背,太子不侍疾,群公不奔喪,疑有逆謀,社稷將危。吾備介弟之親,遂攝大位以寧國家;百官叩門入者,進陛二等。」盡殺包括太子馮翼在內的馮跋諸子百人。

翌年(431年),改元太興。將自己元配夫人王氏及其所生之子、太子馮崇廢掉。於第二年四月,冊立後燕皇族之女慕容氏為天后,藉以抬高其身價。於是,長樂公馮崇,以及馮崇之同母弟、廣平公馮朗,樂陵公馮邈也懼繼母迫害,禍及自身,於是舉郡向北魏投誠。第三年,春正月,馮弘冊立「後妻慕容氏子馮王仁為世子」。

北燕國小民弱,馮弘在位時,因北魏屢次攻伐,數次向北魏朝貢請和,但仍持續受到攻擊,因此亦曾遣使向南朝宋稱藩納貢。

太興六年(436年),北魏再攻北燕,馮弘於五月乙卯日(6月4日),馮弘帶領子女、後宮、宗族,及龍城之百姓,隨高句麗援軍從都城龍城(今遼寧朝陽)撤退,臨行焚其宮室、城邑,大火一旬不滅,北燕亡。

馮弘在高句麗號令如在本國,引起高句麗長壽王高璉嫌惡,長壽王將其侍衛撤走,又將其太子馮王仁押回興京,扣作人質。復有歸刘宋之意,於是又派使者帶著三百人出使建康,請求宋文帝允許其全家移居建康;宋文帝答應,並派遣將軍王白駒,率兵七千,北上迎接。當時,高句麗也向刘宋稱臣。高句麗王不欲馮弘南下成仇,好言規勸,而馮弘不聽。遂於438年殺馮弘及其妻子,並為其上諡號曰昭成皇帝,一作昭文皇帝。

\subsubsection{太兴}

\begin{longtable}{|>{\centering\scriptsize}m{2em}|>{\centering\scriptsize}m{1.3em}|>{\centering}m{8.8em}|}
  % \caption{秦王政}\
  \toprule
  \SimHei \normalsize 年数 & \SimHei \scriptsize 公元 & \SimHei 大事件 \tabularnewline
  % \midrule
  \endfirsthead
  \toprule
  \SimHei \normalsize 年数 & \SimHei \scriptsize 公元 & \SimHei 大事件 \tabularnewline
  \midrule
  \endhead
  \midrule
  元年 & 431 & \tabularnewline\hline
  二年 & 432 & \tabularnewline\hline
  三年 & 433 & \tabularnewline\hline
  四年 & 434 & \tabularnewline\hline
  五年 & 435 & \tabularnewline\hline
  六年 & 436 & \tabularnewline
  \bottomrule
\end{longtable}


%%% Local Variables:
%%% mode: latex
%%% TeX-engine: xetex
%%% TeX-master: "../../Main"
%%% End:
