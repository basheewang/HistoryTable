%% -*- coding: utf-8 -*-
%% Time-stamp: <Chen Wang: 2019-12-18 17:06:58>

\subsection{威王\tiny(353-355)}

\subsubsection{生平}

涼威王张祚(327年前-355年),字太伯,安定烏氏人。十六国时期前凉皇帝,前凉文王张骏庶长子,前凉桓王张重华异母兄。張祚與張重華寵臣勾結,又與太后通姦,得以在張曜靈繼位不久即廢其自立,更曾經稱帝。然而在稱帝翌年就被政變推翻及被殺。

張祚受封長寧侯,他博學且強壯勇武,又有政治才能,可是為人狡詐善於奉承,與張重華寵臣趙長、尉緝等人勾結並結為異姓兄弟。建興四十一年(353年)張重華病重時,曾下手令召酒泉太守謝艾入朝輔政,但為趙長等壓下。同年張重華死,由其年幼的长子張曜靈繼位,趙長等就假稱張重華遺令,以張祚為持節、都督中外諸軍事、撫軍將軍身份輔政。時趙長等以張曜靈年幼,稱國家需要年長君主,张祚因与张重华之母馬太后通奸,遂煽动马太后废黜了張曜靈,立张祚為主。張祚於是自稱大都督、大將軍、涼州牧、涼公。張祚位後即淫亂張重華的妻妾及其未嫁女兒。

和平元年(354年),张祚称帝,改元「和平」,設宗廟、八佾舞,並置百官,尚書馬岌因切諫被免官,郎中丁琪進諫更被殺,又殺謝艾。張祚又曾進攻驪靬,但大敗而還。同年東晉桓溫北伐,也有配合北伐的秦州刺史王擢派人報告張祚稱桓溫善於用兵,軍勢難測。張祚聞訊恐懼,但還擔憂王擢會倒過來進攻自己,於是派人暗殺他,但因被王擢發現而不成。張祚在暗殺失敗後更加恐懼,於是出兵聲稱要東征,實則是想西退至敦煌自保,只是遇上桓溫退兵才取消行動。不過,張祚仍繼續打擊王擢,派了牛霸率三千兵打敗王擢,逼使王擢投降前秦。

张祚治国不道,曾置五都尉去專抓別人過失,又限定四品以下官員不得送贈衣布,庶人不能畜養奴婢及乘坐車馬。张祚为人荒淫暴虐,国人无不侧目,都作諷刺其淫亂的詩。和平二年(355年),張祚因不欲河州刺史張瓘強大,於是命令他去討伐叛胡,其實已派易揣及張玲率三千兵襲擊張瓘。王鸞識術數,向張祚說:「這支軍隊出去,肯定不會回來,涼國會陷於危險。」更上陳張祚三不道。張祚聞言大怒,認定王鸞所說是妖言,將他處斬。王鸞臨死前就說:「我死後,軍隊在外面戰敗,大王在內死亡,肯定會發生的!」張祚更誅殺王鸞一族人。不過,張瓘就殺了張祚派去代其守枹罕的索孚,易揣等渡過黃河途中就被張瓘攻擊,張瓘更出兵跟隨單騎逃還的易揣,兵向姑臧。張瓘軍前來的消息震動姑臧人心,時宋混、宋澄兄弟因其兄宋修與張祚有前嫌,就出城聚眾響應張瓘,並反攻姑臧。時張瓘傳檄州郡,要復立張曜靈,故張祚就派楊秋胡殺害張曜靈;另又收捕並處死張瓘的兩個弟弟張琚及張嵩。二人知要被捕時卻在市招募數百,大叫張瓘大軍已經到達姑臧城東,恐嚇敢動手的人要被誅三族。收捕的人果被嚇退,然後二人西城門迎宋混等入城。趙長等人懼怕因擁立張祚獲罪,於是請馬太后出殿,改立張玄靚為主,不過易揣等人卻引兵入殿,收殺趙長等人。宋混等入城後,張祚按劍命令部眾死戰,但因為他失眾心,將士根本毫無鬥志,張祚於是為宋混等殺死,頭被斬下宣示內外,更遭曝屍在大道左邊,城內人民都大呼萬歲。

事後張祚以庶人的禮儀下葬,直至其弟張天錫即位時,才改葬到愍陵,追諡為威王。

\subsubsection{和平}

\begin{longtable}{|>{\centering\scriptsize}m{2em}|>{\centering\scriptsize}m{1.3em}|>{\centering}m{8.8em}|}
  % \caption{秦王政}\
  \toprule
  \SimHei \normalsize 年数 & \SimHei \scriptsize 公元 & \SimHei 大事件 \tabularnewline
  % \midrule
  \endfirsthead
  \toprule
  \SimHei \normalsize 年数 & \SimHei \scriptsize 公元 & \SimHei 大事件 \tabularnewline
  \midrule
  \endhead
  \midrule
  元年 & 354 & \tabularnewline\hline
  二年 & 355 & \tabularnewline
  \bottomrule
\end{longtable}


%%% Local Variables:
%%% mode: latex
%%% TeX-engine: xetex
%%% TeX-master: "../../Main"
%%% End:
