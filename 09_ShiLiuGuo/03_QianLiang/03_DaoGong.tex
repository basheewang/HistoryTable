%% -*- coding: utf-8 -*-
%% Time-stamp: <Chen Wang: 2021-11-01 11:53:51>

\subsection{悼公張天錫\tiny(363-376)}

\subsubsection{生平}

張天錫(346年-406年),字純嘏,本字公純嘏,因被人嘲笑是三字而自行改字,小名獨活,安定烏氏人。中國十六國時期前涼政權的最後一位君主。張天錫為前涼文王張駿少子,前涼桓王張重華之弟。張天錫在位時前秦國力強盛,雖曾主動斷絕與前秦關係,但最終仍逼於軍事力量而再度稱藩。及後張天錫反抗前秦徵召入朝的命令並射殺使者,前秦大軍遂攻伐前涼,張天錫不敵投降,前涼國於是滅亡。淝水之戰後張天錫南歸東晉,並在東晉終老。

和平元年(354年),張祚封張天錫為長寧王。建興四十九年(361年),張邕殺死當政的宋澄,當時的前涼君主張玄靚就以張天錫為中領軍,與張邕共輔朝政。不過,張邕因樹立黨羽專權,經常濫用刑法殺人,很不得人心,張天錫親信劉肅則與其共謀除去他。十一月,張天錫與張邕一同入朝,劉肅就與趙白駒跟著張天錫行動,二人先後襲擊張邕但都失敗,於是與張天錫一同走入宮中。逃走的張邕率三百軍人進攻宮門,張天錫登門樓指責張邕凶惡悖逆,聲言自己是在冒死保衞國家社稷,並只會針對張邕而已。張邕兵眾聞言都逃散,張邕自殺,張天錫又誅殺了張邕黨羽,專掌朝政。

升平七年(363年),郭太妃以張天錫專政,與張欽密謀誅殺張天錫,事洩,欽等皆死;右將軍劉肅於是勸張天錫自立,天錫遂會劉肅夜襲皇宮,殺張玄靚。張天錫自稱使持節、大都督、大將軍、護羌校尉、涼州牧、西平公,並派使者出使建康請命,東晉於是在366年授張天錫為大將軍、大都督、督隴右關中諸軍事、護羌校尉、涼州刺史,封西平公。前秦亦派大鴻臚授張天錫大將軍、涼州牧、西平公。

張天錫登位後多次在園池設宴,又沉迷於歌舞和女色,荒廢政事。張天錫更將兩個親信劉肅及梁景收為養子,讓二人參與朝政,令人們有怨言和恐懼,索商及天錫堂弟張憲曾經勸諫他但不獲授納。張天錫於升平十年(366年)與前秦斷交,並在進攻李儼時與前秦發生軍事衝突,並俘獲了陰據和他率領的五千兵。升平十五年(371年),前秦攻滅仇池,送還陰據及其士兵回國,並派梁殊及閻負隨行,順道送達前秦丞相王猛的書信,暗示要張天錫別和前秦作對。張天錫看後十分恐懼,於是派使者向前秦謝罪,向前秦稱藩,前秦天王苻堅任命其為使持節、散騎常侍、都督河右諸軍事、驃騎大將軍、開府儀同三司,涼州刺史、西域都護、西平公。然而因張天錫因為懼怕前秦吞併,於同年在姑臧設壇,遙與晉三公盟誓,又派從事中郎韓博出使東晉,並寫信給東晉大司馬桓溫,約定大舉出兵北伐,會師上邽。

升平二十年(376年),苻堅徵召張天錫入朝任武衞將軍,同時派了苟萇、毛盛、梁熙及姚萇等率十三萬步騎至西河郡,預備一旦張天錫拒絕應命就進攻前涼。張天錫接到梁殊、閻負送來的詔命後問及眾僚意見,除席仂建議送貨款和質子,徐圖後計外,大部份人都認為涼州有精兵及天險,可以取勝。張天錫於是決定反抗,派馬建率兵二萬抵抗秦軍,並命人射殺兩名前秦使節。面對秦軍進攻,馬建懼而退守清塞,張天錫又派掌據率兵三萬與馬建屯於洪池,自率五萬屯金昌城。可是,苟萇隨後進攻掌據時馬建就投降前秦,掌據戰死,張天錫又派趙充哲為前鋒,率五萬兵與苟萇等作戰,但又在赤岸大敗,張天錫出城意圖再戰,但因金昌城中反叛而被逼逃回姑臧並請降。苟萇等到姑臧後受降,並送張天錫到長安,其他郡縣都降秦,前涼滅亡。苻堅在長安為張天錫建了府邸,任命他為侍中、北部尚書,封歸義侯。

晉太元八年(383年),晉軍於淝水之戰擊潰來攻的前秦軍,當時張天錫任征南大將軍苻融的司馬隨軍,趁機南奔東晉,東晉朝廷下詔以張天錫為散騎常侍左員外,復封為西平郡公。後轉拜金紫光祿大夫。後曾加授廬江太守,桓玄掌政時為了招撫四方而任命張天錫為護羌校尉、涼州刺史。義熙二年(406年),張天錫去世,享年六十一歲,追贈為鎮西將軍,諡號悼公。

張天錫因文才而聲名遠著,回歸晉廷後亦甚得晉孝武帝知遇,可是朝中官員卻以其曾經亡國被俘而中傷他。會稽王司馬道子曾經問及涼州有甚出產,張天錫立即就答道:「桑葚甘甜、鴟鴞會變聲音、乳酪養生、人沒有嫉妒之心。」不過,後來張天錫表現得昏亂喪志,雖然有公爵爵位也得不到別人禮遇。至晉安帝隆安年間,當政的會稽王世子司馬元顯更常常請他來戲弄他。擔任廬江太守亦因為其家貧而獲授。

\subsubsection{升平}

\begin{longtable}{|>{\centering\scriptsize}m{2em}|>{\centering\scriptsize}m{1.3em}|>{\centering}m{8.8em}|}
  % \caption{秦王政}\
  \toprule
  \SimHei \normalsize 年数 & \SimHei \scriptsize 公元 & \SimHei 大事件 \tabularnewline
  % \midrule
  \endfirsthead
  \toprule
  \SimHei \normalsize 年数 & \SimHei \scriptsize 公元 & \SimHei 大事件 \tabularnewline
  \midrule
  \endhead
  \midrule
  七年 & 363 & \tabularnewline\hline
  八年 & 364 & \tabularnewline\hline
  九年 & 365 & \tabularnewline\hline
  十年 & 366 & \tabularnewline\hline
  十一年 & 367 & \tabularnewline\hline
  十二年 & 368 & \tabularnewline\hline
  十三年 & 369 & \tabularnewline\hline
  十四年 & 370 & \tabularnewline\hline
  十五年 & 370 & \tabularnewline\hline
  十六年 & 372 & \tabularnewline\hline
  十七年 & 373 & \tabularnewline\hline
  十八年 & 374 & \tabularnewline\hline
  十九年 & 375 & \tabularnewline\hline
  二十年 & 376 & \tabularnewline
  \bottomrule
\end{longtable}


%%% Local Variables:
%%% mode: latex
%%% TeX-engine: xetex
%%% TeX-master: "../../Main"
%%% End:
