%% -*- coding: utf-8 -*-
%% Time-stamp: <Chen Wang: 2019-12-18 17:03:15>


\section{前凉\tiny(301-376)}

\subsection{简介}

前凉(320年-376年)是十六国政权之一。都姑臧(今甘肃武威)。 301年,凉州大姓汉人张轨被晋朝封为凉州刺史,313年封西平公,課農桑、立學校,多所建樹。又鑄五銖錢,全境通行。314年张轨病死,其子张\xpinyin*{寔}袭位。西晋灭亡后,仍然据守凉州,使用司马邺(晉愍帝)的建興年號,成为割据政权。

320年,张茂改元永元,前凉遂彻底成为独立政权。

345年,张寔子张骏称凉王,都姑臧,以所在地凉州为国号“凉”,史称“前凉”,以别于其他以“凉”为国号的政权。張駿、張重華父子統治時期,前涼極盛。353年張重華病死,宗室內亂不止,國勢大衰。

前涼極盛之時,统治范围包括甘肃、宁夏西部以及新疆大部。史載“南逾河、湟,東至秦、隴,西包蔥嶺,北暨居延”。張天錫時已失去甘肅南部。

376年,前秦天王苻堅以十三萬步騎大舉進攻,張天錫投降,前涼滅亡。

\subsection{武王生平}

張軌(255年-314年),字士彥,安定郡烏氏縣(今甘肅平涼市西北)人。西漢常山王張耳的十七世孫。晉朝時任涼州牧,是前涼政權奠定者,張寔、張茂皆為其子。314年去世,晉諡曰武公。至其曾孫張祚時,被追諡為武王,廟號太祖。

張軌其家世孝廉,以儒學著稱。張軌年少時已聰明好學,甚有名望,曾隱居於宜陽郡的女几山上。西晉建立後入朝任官,因與中書監張華議論經籍意義和政事而深得對方的器重。張軌歷任太子舍人、尚書郎、太子洗馬、太子中庶子、散騎常侍,征西將軍司馬。

晉惠帝元康元年(291年),「八王之亂」開始,天下大亂,張軌於是想佔據河西之地(今甘肅西部、新疆東部一帶),於是就要求調任涼州。在朝中官員的支持之下,張軌於永寧元年(301年)被任命為護羌校尉、涼州刺史。張軌到任後,使立刻領兵擊敗當時在涼州叛亂的鮮卑族,又消滅橫行當地的盜賊,斬首萬多人,從此威震西土,亦安定了涼州。張軌任用有才幹的涼州大姓如宋配、陰充、氾瑗和陰澹為股肱謀主,共同治理涼州。他又勸農桑,立學校,又設與州別駕同等的崇文祭酒、春秋行鄉射之禮,在涼州大行教化。

永興二年(305年),鮮卑若羅拔能侵襲涼州,張軌派司馬宋配討伐,最終斬殺若羅拔能,並俘據十多萬人,因而聲名大振。晉惠帝亦因此加張軌安西將軍,封安樂鄉侯,邑千戶。同時又大修涼州治所姑臧(今甘肅武威市)。此時,東羌校尉韓稚殺害秦州刺史張輔,張軌少府司馬楊胤主張討伐韓稚,亦勸張軌效法齊桓公主持地方,對韓稚擅殺刺史的行為予以嚴懲。張軌於是命中督護領二萬兵討伐,並先寫信給韓稚勸降。韓稚拉到書信後就向張軌投降。張軌報告南陽王司馬模後,司馬模十分高興,並將皇帝賜的劍送給張軌,並將隴西地區交給張軌管理。

張軌始終對西晉表示忠誠,以維繫民心。如太安三年(304年)河間王司馬顒和成都王司馬穎到洛陽討伐掌權的司馬乂,張軌亦曾派三千兵支援朝廷。永嘉二年(308年),劉淵部將王彌進攻洛陽,張軌派北宮純、張纂、馬魴和陰濬等領兵入衛洛陽,北宮純及後派百多名勇士突擊王彌軍,協助朝廷擊退王彌。不久北宮純在河東擊敗劉淵兒子劉聰,晉懷帝於是詔封張軌為西平郡公,但張辭讓。西晉自八王之亂起,天下大亂,各州都不再向西晉朝廷賦貢,亦惟有張軌貢獻不絕。

永嘉二年(308年),張軌因患風搐而不能說話,命兒子張茂代管涼州。張越是涼州大族,聽說有預言說張氏會興盛涼州,自以為自己就是預言中的張氏,於是不惜放下梁州刺史的職務告病回涼州,更與兄長酒泉太守張鎮等人合謀要除去張軌。張越兄弟意圖以秦州刺史賈龕取代張軌,於是派密使到洛陽請尚書侍郎曹祛任西平太守,作為援助。張軌別駕麴晁亦意圖借機弄權,派使者到長安告訴司馬模,說張軌已病得不再能繼續行使刺史職權,要求以賈龕代替張軌。賈龕原打算應命,但被兄長勸止。

張鎮和曹祛知道賈龕拒絕應命後,再上表請求新派刺史,但未上呈就已率先以軍司杜耽代領州事,讓杜耽支持並表張越為新任刺史。張軌見此,打算退避,想要回到曾經隱居的宜陽,但長史王融和參軍孟暢接到張鎮等人以杜耽代理涼州的檄命後並不服氣,決意支持張軌,於是領兵戒嚴,又命剛從洛陽回來的張軌長子張寔為中督護,領兵討伐張鎮。同時又派張鎮甥子令狐亞遊說張鎮。最終張鎮聽從,哭著說受了誤導,將事情都推給功曹魯連,更將魯連殺死向張寔請罪。張寔及後攻打曹祛,曹祛逃走。在王融舉兵同時,武威太守張琠亦派兒子張坦到洛陽上表支持張軌;而治中楊澹亦到長安向司馬模控訴張軌被誣,令司馬模上表停止選調新任刺史。張坦到洛陽後,晉懷帝慰勞張軌,又下令誅殺曹祛。張軌知道後十分高興,又命張寔領兵三萬討伐曹祛,最終將曹祛擊敗並殺死。

張軌及後命治中張閬送五千義兵和大量物資到洛陽。永嘉五年(311年)光祿大夫傅祗和太常摯虞及後寫信給張軌說洛陽物資缺乏,張軌又立刻派參軍杜勵進獻五百匹馬和氈布三萬匹。晉懷帝於是進拜張軌為鎮西將軍、都督隴右諸軍事,封霸城侯,並進車騎將軍、開府儀同三司。但使者還未到,王彌就再次進逼洛陽,張軌派將軍張斐、北宮純和郭敷等率五千名精銳騎兵保衛洛陽,但洛陽最終都被漢國大將劉曜攻克。

永嘉之亂後,洛陽和長安兩大重鎮都先後被漢國軍隊攻陷,中原和關中地區的很多百姓流入涼州避難,張軌在姑臧西北置武興郡;又分西平郡(今青海西寧市)界置晉興郡以收容流民。同時,張軌亦繼續支持西晉,晉懷帝被擄到平陽後,張軌曾打算傾一州之力進攻平陽。不久秦王司馬鄴入關,張軌又派兵支持。次年司馬鄴被擁立為皇太子,張軌獲拜驃騎大將軍、儀同三司,張軌辭讓。同時張軌又協助消滅在附近地區叛亂的勢力,如秦州刺史裴苞、西平郡的麴恪、鞠儒等。司馬鄴及後再度任命,但張軌亦再次辭讓。

永嘉七年(313年),晉懷帝被殺,司馬鄴繼位為晉愍帝,並升張軌為司空,張軌再辭讓。同時又聽從索輔的建議,復鑄五銖錢,恢復境內的錢幣流通,大大便利了當地人的生活,不必再以布匹作貨幣。同時,劉曜進逼長安,張軌又派參軍麴陶領三千兵入衛長安。

建興二年(314年),晉愍帝任命張軌為侍中、太尉、涼州牧,封西平公,但張軌仍然辭讓。五月己丑日,張軌病死,享年六十歲,諡曰武公。張軌的親信部下及後擁立張軌長子張寔繼任了涼州牧之職。

张轨墓在今凉州区境内,史称“建陵”,前凉国主的陵墓位置,学术界有三种推测:陵墓上方筑台,可能在今灵钧台、雷台等古台下;或依照汉制,重臣死后多陪葬君主墓旁,根据已出土的“梁舒墓”的方位,可能在今武威城西北太平滩一带;或因山为陵,可能在武威城南祁连山山坡地带。

\subsection{昭王生平}

张寔(271年-320年),字安遜,安定烏氏人。十六国时期前凉政权的君主。为张轨长子。張寔任內保持與晉廷關係,也支持晉元帝即位為帝,但仍一直在割據狀態,即使西晉亡後仍然用晉愍帝「建興」年號。

張寔高學識,觀察入微,而且敬重並愛惜有才德的人,獲舉為秀才,授命為郎中。永嘉初年,張寔辭讓驍騎將軍,並請求回當時由父親任刺史的涼州。朝廷准許張寔所求,遂改授張寔議郎。不過,張寔回到涼州治所姑臧(今甘肅武威市)時正遇上涼州大姓張鎮、張越兄弟與曹祛等人圖謀逼走父親,奪取涼州控制權的行動。張軌長史王融及參軍孟暢支持張軌,決意作出反擊,於是讓張寔為中督護,率兵討伐張鎮。張鎮恐懼並委罪予功曹魯連,將其處死後便向張寔請罪。張寔隨後攻伐曹祛,曹祛逃走。時前往洛陽為張軌陳情的張坦帶著晉懷帝慰問張軌和誅殺曹祛的詔命回來,張軌於是命張寔率尹員、宋配等領三萬步騎兵攻曹祛。曹祛派麴晃到黃阪抵禦,張寔就用計騙了麴晃,令自己得以進至浩亹(今青海海東市樂都區東),並戰於破羌(今青海海東市樂都區西)。曹祛等終為張軌所殺。戰後張寔獲封建武亭侯。

不久,張寔遷西中郎將,封福祿縣侯。晉愍帝即位後,又以西中郎將領護羌校尉。建興二年(314年),張軌去世,長史張璽等人表張寔代行張軌官位。晉愍帝及後下詔授予張寔持節都督涼州諸軍事、西中郎將、涼州刺史、領護羌校尉,封西平公。張寔接掌涼州後鼓勵諫言,當面進諫的賞布帛;書面進諫的賞竹器;在坊間論政的賞羊和米。另又聽從賊曹佐隗瑾所言設立諫官,處理大小事務時都與部屬們討論,廣納眾言,從而鼓勵吏民進言。

建興四年(316年),前趙將領劉曜率軍逼近長安(今陝西西安),張寔派王該率兵救援,晉愍帝於是加授張寔都督陝西諸軍事。同年晉愍帝被圍困被逼投降,降前下詔進張寔為大都督、涼州牧、侍中、司空,承制行事。張寔受詔後以愍帝被俘為由辭讓。及後晉愍帝遇害的消息傳至涼州,南陽王司馬保卻圖謀稱帝,張寔則支持在江東的司馬睿,並在建興六年(318年)派牙門蔡忠上表勸進。同年,司馬睿即位為帝,即晉元帝。

建興七年(319年),司馬保自稱晉王,置百官並改年號,又以張寔為征西大將軍、儀同三司,增食邑三千戶。但不久司馬保就因部將陳安叛變而陷險境,張寔先後派兵協助。次年(320年),司馬保因劉曜逼近而遷至桑城(今甘中肅臨洮县附近),並意圖到涼州避難,但張寔考慮到司馬保宗室的身份,若果到涼州肯定會對當地人心有所影響,於是派將領陰監派兵迎接司馬保,聲稱是保衞,其實是想阻止他前來。不久,司馬保被其部將張春所殺,餘眾離散並有萬多人逃至涼州,張寔至此自恃涼州險遠,頗為驕傲放縱。

當時天梯山上有一個叫劉弘的人,因為法術而有上千信眾,連張寔身邊的人都有其信眾。當時帳下閻涉及牙門趙卬都是劉弘同鄉,而劉弘向閻涉說:「上天賜我神璽,要我治理涼州。」二人深信不疑,於是秘密聯結張寔身邊十多人,意圖行刺張寔,奉劉弘為主。張寔已經從張茂口中得知這圖謀,於是派了牙門將史初收捕劉弘,但閻涉等人不知道,依計眾人懷刀而入,在外寢殺死張寔。劉弘見史初來,還說:「使君已經死了,還殺我做甚麼!」史初憤怒,割了他的舌頭然後囚禁,及後張茂更施車裂之刑,閻涉及其黨羽數百人亦被誅殺。享年五十歲。私諡為昭公,晉元帝則赐諡號元公。張祚稱帝時以張寔為昭王。

張寔死後,因兒子張駿年幼,由弟弟張茂繼位。墓葬称“宁陵”,亦在今凉州区境内。


\subsection{成王生平}

张茂(277年-324年),字成遜,安定烏氏人。中國十六国时期前凉政权的君主,为昭王张寔之同母弟。張茂任內前涼遭前趙出兵威壓,被逼向前趙稱藩,並接受其官爵。

永嘉二年(308年)張軌患病不能說話,時兄長張寔仍在朝中,故張茂就代父管理涼州事務。建興元年(312年),南陽王司馬保曾經請張茂作自己的從事中郎,後又推薦他任散騎侍郎、中壘將軍,但張茂都不應命。次年,張茂被徵召為侍中,但張茂以父親年老為由推辭。不久,改拜平西將軍、秦州刺史。

建興八年(320年),張寔被部下所殺,因其子駿年幼,张茂就代攝其位,殺劉弘數百名同黨。時涼州府推舉張茂為大都督、太尉、涼州牧,但張茂不肯受,只以使持節平西將軍、涼州牧職位,又以張寔子張駿為撫軍將軍、武威太守、西平公。。

建興十年(322年),張茂派韓璞率兵佔領隴西南安郡境,並在當地置秦州。建興十一年(323年),前趙劉曜派部將劉咸攻冀城,呼延晏攻桑壁,時臨洮人翟楷及石琮等又驅逐其地方官員響應劉曜,劉曜本人更发兵二十八万五千人沿黃河列陣百多里,張茂設在黃河沿岸守戍的軍隊望風奔逃,劉曜更聲言面率大軍渡河,直攻姑臧,遂震動河西。張茂聽從參軍馬岌所言出屯姑臧城東的石頭,在聽參軍陳珍分析劉曜其實不會盡力攻涼後,便派陳珍出兵救援在冀城的韓璞。劉曜亦自知其強大的兵力有三分之二是因為人們怯於其聲威而來,主力軍隊已經相當疲累,難以渡河進攻,於是一直按兵不動,想用聲威脅服張茂。張茂最終派使者向前趙稱藩,進獻大量物品,劉則授張茂使持節、假黃鉞、侍中、都督涼南北秦梁益巴漢隴右西域雜夷匈奴諸軍事、太師、領大司馬、涼州牧、領西域大都護、護氐羌校尉。封涼王。

建興十二年(324年)張茂病死,享年四十八歲。前涼私下為張茂上諡號成公,刘曜則遣使赠太宰,諡成烈王。後來张祚稱帝,追尊張茂為成王,庙号太宗。张茂临终时交代张骏“谨守人臣之节,无或失坠”。因張茂無子,張駿就被推舉繼位。

張茂為人謙虛恭敬,且又好學,不因為利祿而動心,又有志向及節操,有決定大事的能力。當時有一個人叫賈摹,不但出身涼州大姓,他也是張寔妻子的弟弟,勢力很大。曾經有一首童謠這樣說:「手莫頭,圖涼州。」張茂以此誘殺賈摹,終令涼州豪門大族不敢橫行,更有助前涼張氏對涼州的管治。


\subsection{文王生平}

张骏(307年-346年),字公庭,安定烏氏人。中國十六国时期前凉政权的君主,在位二十二年。为前凉明王张寔之子,前凉成王张茂之侄。張駿任內前涼國力提升,也乘前趙滅亡而盡得河南(甘肃地区黄河以南)隴西之地,又進攻西域。張駿亦先後接受後趙及東晉的官位,在位晚期亦建設起天子規格器物、儀式及官職架構。

建興四年(316年),張駿受封霸城侯。建興八年(320年),張寔去世,涼州州府推舉其叔張茂繼位,張茂於是以張駿為撫軍將軍、武威太守,襲爵西平公。建興十二年(324年),張駿在張茂死後繼位,並暗示時滯留在姑臧的晉愍帝使者史淑以晉廷名義授予自己使持節大都督、大將軍、涼州牧、領護羌校尉,封西平公。時前趙皇帝劉曜也授張駿大將軍、涼州牧,封涼王。

張駿繼位時,守枹罕的涼州將領辛晏據城反對張駿,不服其統治。張駿打算討伐但為從事劉慶勸止,而翌年辛晏也向張駿投降,收服了河南之地。建興十五年(327年),張駿聽聞前趙軍隊敗於後趙,於是除去前趙所授的官爵,用回晉朝的官爵,並派兵進攻前趙秦州。可時前涼軍敗於前趙南陽王劉胤所率的軍隊,劉胤更乘勝渡過黃河,攻陷令居並殺二萬多人,又進佔振武,震動河西,張駿派皇甫該前往防禦。金城太守張閬及枹罕護軍辛晏都向前趙投降,河南之地復失。建興十七年(329年),前趙亡於後趙,張駿於次年就乘機重奪河南地,進軍至狄道,置武街、石門、侯和、漒川及甘松五屯護軍。不久,後趙派了鴻臚孟毅授予張駿征西大將軍、涼州牧,但張駿恥於為後趙之臣,不接受並留下孟毅。但不久就因畏懼後趙強大而向其稱臣,送還孟毅。東晉朝廷也進張駿鎮西大將軍,仍授涼州刺史、領護羌校尉並封西平公,詔命於建興二十一年(333年)到達前涼,張駿接受任命,派王豐等人陳謝並上疏稱臣,但仍然用建興年號,不用東晉年號。次年東晉又進張駿為大將軍,此後晉涼每年都有使者往來。

張駿又曾派將領楊宣率兵進攻龜茲及鄯善,終令西域諸國都歸附前涼,焉耆前部王及于寘王都派人進貢。也曾上疏晉廷請求配合司空郗鑒及征西將軍庾亮進行北伐。

西域長史李柏敗於不肯服從張駿的戊己校尉趙貞,有人認為是李柏自設計謀導致失敗,請張駿誅殺他,然而張駿終免李柏一死,更得眾人歡心。張駿也改易原本犯下死罪者的親屬不得留在朝中的律令,只是限制他們不能參與宿衞,於是令前涼刑法清明,國家富強,群僚更於建興二十年(332年)勸張駿稱「涼王」,自領秦涼二州牧,置公卿百官。雖然張駿嚴詞拒絕,但其實前涼境內都用涼王去稱呼張駿。後張駿更努力改變自己,勤於庶政,統掌涼州文武事務,治績不錯,得四方稱頌,叫他做「積賢君」。而涼州自晉末以來連年都有戰事,至張駿在位時漸見平穩安定。建興二十七年(339年),張駿又設辟雍、明堂以行禮教。

張駿攻西域後,在涼州西界劃出設沙州,又將涼州東界劃出設河州,時屬官們都稱臣。張駿亦在姑臧附近增築新城,又修建用金玉和五色畫裝飾的謙光殿,極盡珍貴精巧,其四面都各建一殿,四季各居一殿。他又自稱大都督、大將軍、假涼王,督攝涼沙河三州,設六佾之舞,設天子的豹尾車,所設祭酒、郎中、大夫、舍人、謁者等官職官號都模仿晉朝體系,只是稍稍改了名字。

建興三十四年(346年),張駿去世,享年四十歲,前涼私諡為文公,晉廷則賜諡號忠成,贈大司馬,歸葬大陵。其子張祚稱帝時,追尊為文王,廟號世祖。

后凉年间,有名叫安据的即序胡人盗张骏墓,见张骏貌如生前,并盗得真珠簏、琉璃榼、白玉樽、赤玉箫、紫玉笛、珊瑚鞭、马脑钟、水陆奇珍不可胜数。后凉皇帝吕纂诛安据党徒五十余家,遣使吊祭张骏,并缮修其墓。

張駿年輕就已顯得奇特雄偉,十歲時就能寫文章,為人卓越不羈,但曾經縱情淫慾,常夜遊城邑里巷。《魏書·張寔傳》記載張駿為人貪婪,為求進圖秦隴而給予人民穀物布帛,一年後收一倍稅收,不夠的都要用田地屋宅抵償。然《晉書·張軌傳》則載是譚詳建議將倉庫的穀贈予百姓,然後在秋季收三倍稅收,為陰據所諫而放棄實行。《魏書》又寫其因畏懼大姓陰氏勢力大而逼陰澹弟陰鑒自殺,大失人心。此事《晉書》亦無載。


\subsection{桓王生平}

张重华(327年-353年),字泰臨,是十六国时期前凉政权的君主。为前凉文王张骏次子,353年病死。張重華統治時期,前涼國勢達於極盛,多次擊敗後趙石虎的進攻,後更乘後趙末年國亂而進取秦州。在位七年病死,年僅二十七歲。

建興二十年(332年),群僚請張駿立世子,張駿最初不肯,但在中堅將軍宋輯的勸說下,張駿還是立了張重華為世子。建興三十三年(345年),張駿從涼州分劃出沙州及河州,以武威、武興、西平、張掖、酒泉、建康、西郡、湟河、晉興、須武及安故十一郡仍為涼州,由張重華任五官中郎將、涼州刺史。

建興三十四年(346年),張駿去世,涼州官屬推張重華為使持節大都督、太尉、護羌校尉、涼州牧,襲爵西平公,假涼王。張重華即位後減輕賦斂,免除關稅,減省園囿,以撫恤貧窮者。同年後趙派麻秋、王擢等侵涼,金城太守張沖降趙,涼州震動。張重華任用謝艾抵抗,終大破趙軍,殺五千人。翌年,後趙再派石寧領二萬兵作為麻秋後援,前涼將領宋秦更加率二萬戶人投降後趙。張重華再度起用謝艾,命其率三萬步騎進軍臨河,又破趙軍,斬殺趙將杜勳、汲魚,一萬三千人被俘或陣亡。不久,石寧聯合麻秋等率十二萬兵進屯河南,再度進攻,張重華想要親自出擊,但為謝艾及索遐所勸止,遂派二人率兵二萬抵抗。時後趙將孫伏都、劉渾率步騎二萬增援麻秋,眾人渡過黃河並屯於長最。謝艾等進軍至神鳥,擊敗王擢前鋒,令其退回黃河以南,接著就進攻長最,再敗趙軍,麻秋等退還金城。石虎聞麻秋戰敗,也嘆息道:「我以偏師就平定了九個州,現在用九個州的力量卻在枹罕寸步難行,真是對方有能人,還不可以謀取呀。」不過,麻秋隨後擊敗了張瑁,枹罕護軍李逵降趙,於是河南地區羌、氐族人都附趙。

建興三十五年(347年),東晉派侍御史俞歸到涼州,授予張重華假節、侍中、大都督、督隴右關中諸軍事、護羌校尉、大將軍、涼州刺史,封西平公。俞歸到涼時,時張重華想稱涼王,故未受詔,更命親信沈猛向俞歸表示,但為俞歸拒絕,並言:「今天你的主公剛剛繼位就要稱王,若果率領河右部眾平定東方的胡、羯,修復晉朝帝陵及宗廟,迎天子還都洛陽,還有甚麼可以嘉獎呀?」張重華於是不圖稱王。

建興四十年(352年),因後趙國亂,苻健乘時於關中建立前秦,時任後趙西中郎將的王擢向東晉請降,獲授征西將軍、秦州刺史,但同年就被前秦將領苻雄擊破,於是出奔涼州,向前涼歸降。張重華厚待他,任命他為征虜將軍、秦州刺史。張重華更派了將軍張弘及宋修率一萬五千兵與王擢會合,讓他進攻前秦。次年(353年)兩軍交戰,王擢大敗逃奔姑臧,張弘及宋修都戰死。張重華素服為陣亡將士舉哀,也安慰其家人,更加再命王擢進攻前秦秦州,最終取勝,奪取秦州。張重華因而上疏東晉請求伐秦,東晉則進張重華涼州牧。

建興四十一年(353年),張重華因病去世,享年二十七,葬顯陵。私諡為昭公,後改桓公,東晉則賜諡號敬烈公。重華病重時曾下手令徵召謝艾為衞將軍、監中外諸軍事以輔政,但最終為重華兄張祚等人壓下,終由張祚輔政,不久更廢掉張重華的世子張曜靈,自己登位。張祚稱帝,追諡張重華為桓王,上廟號世宗。

張重華寬厚平和,深沉穩重,又寡言。不過在擊退後趙連番進攻後表現怠惰,疏於政事且很少親身接見賓客,司直索遐曾經進言勸諫,張重華雖然大感高興,但沒有改變。他又喜和身邊小人玩樂,更多次向左右近臣賞賜金錢。


\subsection{哀公生平}

張曜靈(344年-355年),字元舒,是十六国時期前凉政權的君主,前凉桓王張重華子。張曜靈即位不久就被伯父張祚奪位,及後更被殺害。

建興四十一年(353年),張重華患病,遂立張曜靈為世子。同年張重華去世,實歲仅九岁的張曜靈继位,稱大司馬、護羌校尉、涼州刺史、西平公。張重華原本想以謝艾輔政,但遭其兄張祚與寵臣趙長、尉緝等勾結而壓下張重華的命令,於是張曜靈繼位後,張祚就矯令擔任輔政工作。不久,趙長等以張曜靈太過年輕,建議立年長君主,其祖母马氏與張祚私通,遂废曜靈爲涼寧侯,由張祚繼位。

和平二年八月(355年),张瓘等大臣试图废黜张祚、迎張曜靈复位,未成,張曜靈被張祚派遣杨秋胡暗杀,匿尸沙坑。同年张祚被杀,私谥為哀公。

%% -*- coding: utf-8 -*-
%% Time-stamp: <Chen Wang: 2021-11-01 11:53:31>

\subsection{威王张祚\tiny(353-355)}

\subsubsection{生平}

涼威王张祚(327年前-355年),字太伯,安定烏氏人。十六国时期前凉皇帝,前凉文王张骏庶长子,前凉桓王张重华异母兄。張祚與張重華寵臣勾結,又與太后通姦,得以在張曜靈繼位不久即廢其自立,更曾經稱帝。然而在稱帝翌年就被政變推翻及被殺。

張祚受封長寧侯,他博學且強壯勇武,又有政治才能,可是為人狡詐善於奉承,與張重華寵臣趙長、尉緝等人勾結並結為異姓兄弟。建興四十一年(353年)張重華病重時,曾下手令召酒泉太守謝艾入朝輔政,但為趙長等壓下。同年張重華死,由其年幼的长子張曜靈繼位,趙長等就假稱張重華遺令,以張祚為持節、都督中外諸軍事、撫軍將軍身份輔政。時趙長等以張曜靈年幼,稱國家需要年長君主,张祚因与张重华之母馬太后通奸,遂煽动马太后废黜了張曜靈,立张祚為主。張祚於是自稱大都督、大將軍、涼州牧、涼公。張祚位後即淫亂張重華的妻妾及其未嫁女兒。

和平元年(354年),张祚称帝,改元「和平」,設宗廟、八佾舞,並置百官,尚書馬岌因切諫被免官,郎中丁琪進諫更被殺,又殺謝艾。張祚又曾進攻驪靬,但大敗而還。同年東晉桓溫北伐,也有配合北伐的秦州刺史王擢派人報告張祚稱桓溫善於用兵,軍勢難測。張祚聞訊恐懼,但還擔憂王擢會倒過來進攻自己,於是派人暗殺他,但因被王擢發現而不成。張祚在暗殺失敗後更加恐懼,於是出兵聲稱要東征,實則是想西退至敦煌自保,只是遇上桓溫退兵才取消行動。不過,張祚仍繼續打擊王擢,派了牛霸率三千兵打敗王擢,逼使王擢投降前秦。

张祚治国不道,曾置五都尉去專抓別人過失,又限定四品以下官員不得送贈衣布,庶人不能畜養奴婢及乘坐車馬。张祚为人荒淫暴虐,国人无不侧目,都作諷刺其淫亂的詩。和平二年(355年),張祚因不欲河州刺史張瓘強大,於是命令他去討伐叛胡,其實已派易揣及張玲率三千兵襲擊張瓘。王鸞識術數,向張祚說:「這支軍隊出去,肯定不會回來,涼國會陷於危險。」更上陳張祚三不道。張祚聞言大怒,認定王鸞所說是妖言,將他處斬。王鸞臨死前就說:「我死後,軍隊在外面戰敗,大王在內死亡,肯定會發生的!」張祚更誅殺王鸞一族人。不過,張瓘就殺了張祚派去代其守枹罕的索孚,易揣等渡過黃河途中就被張瓘攻擊,張瓘更出兵跟隨單騎逃還的易揣,兵向姑臧。張瓘軍前來的消息震動姑臧人心,時宋混、宋澄兄弟因其兄宋修與張祚有前嫌,就出城聚眾響應張瓘,並反攻姑臧。時張瓘傳檄州郡,要復立張曜靈,故張祚就派楊秋胡殺害張曜靈;另又收捕並處死張瓘的兩個弟弟張琚及張嵩。二人知要被捕時卻在市招募數百,大叫張瓘大軍已經到達姑臧城東,恐嚇敢動手的人要被誅三族。收捕的人果被嚇退,然後二人西城門迎宋混等入城。趙長等人懼怕因擁立張祚獲罪,於是請馬太后出殿,改立張玄靚為主,不過易揣等人卻引兵入殿,收殺趙長等人。宋混等入城後,張祚按劍命令部眾死戰,但因為他失眾心,將士根本毫無鬥志,張祚於是為宋混等殺死,頭被斬下宣示內外,更遭曝屍在大道左邊,城內人民都大呼萬歲。

事後張祚以庶人的禮儀下葬,直至其弟張天錫即位時,才改葬到愍陵,追諡為威王。

\subsubsection{和平}

\begin{longtable}{|>{\centering\scriptsize}m{2em}|>{\centering\scriptsize}m{1.3em}|>{\centering}m{8.8em}|}
  % \caption{秦王政}\
  \toprule
  \SimHei \normalsize 年数 & \SimHei \scriptsize 公元 & \SimHei 大事件 \tabularnewline
  % \midrule
  \endfirsthead
  \toprule
  \SimHei \normalsize 年数 & \SimHei \scriptsize 公元 & \SimHei 大事件 \tabularnewline
  \midrule
  \endhead
  \midrule
  元年 & 354 & \tabularnewline\hline
  二年 & 355 & \tabularnewline
  \bottomrule
\end{longtable}


%%% Local Variables:
%%% mode: latex
%%% TeX-engine: xetex
%%% TeX-master: "../../Main"
%%% End:

%% -*- coding: utf-8 -*-
%% Time-stamp: <Chen Wang: 2019-12-18 17:31:16>

\subsection{冲王\tiny(355-363)}

\subsubsection{生平}

涼沖王張玄靚(350年-363年),字元安,十六國時期前涼國君主,為張重華之子,張曜靈之弟。張玄靚年幼繼位,前涼國政先後在張瓘、宋混、宋澄、張邕及張天錫手中掌握,期間政變頻仍,張玄靚最終亦因張天錫政變而被殺。

張玄靚於和平元年(354年)獲張祚封為涼武侯。和平二年(355年),張祚被殺,張玄靚被宋混、張琚推為大將軍、涼州牧、護羌校尉、西平公,恢復年號為建興四十三年。不久,河州刺史張瓘返都城姑臧(今甘肅武威),張玄靚再被推為涼王,政事決於張瓘。次年(356年)前秦派使者閻負、梁殊前來,要勸說前涼臣服於前秦,張瓘恐懼,於是勸導張玄靚向前秦稱藩,而前秦亦以張玄靚所稱的官爵授命。

張玄靚繼位後,前涼國內先後有李儼、衞綝和馬基等人反叛,張瓘擊敗了衞綝並討平馬基。其時張瓘、張琚兄弟賞罰都依從自己愛惡,無視綱紀,又不聽諫言,故並不得人心。可是他們自以勢力強大,且有功勳,所以有篡位的意圖,然而就忌憚忠心剛直的宋混。建興四十七年(359年),張瓘徵集了數萬兵並會聚於姑臧,想要消滅宋混兄弟,宋混及宋澄知道後就率領壯士楊和等四十多騎到南城,並向各個兵營宣稱張瓘謀反,太后下令誅除他,很快就召集到二千多人。隨後宋混率眾與張瓘決戰,張瓘兵敗,其部眾都背棄張瓘,向宋混投降,張瓘兄弟於是自殺。事後宋混代替張瓘掌政,張玄靚為宋混所建議去涼王稱號,改稱涼州牧。建興四十九年(361年),宋混去世,張玄靚順從宋混遺言而讓宋澄掌政,不過右司馬張邕不滿宋澄專政,同年即起兵攻滅宋澄,並誅殺宋氏一族。張玄靚隨後又改讓張邕與叔父張天錫共同掌政。可是,張邕自恃功勳大而行事驕縱,濫用刑法,更與馬太后私通,樹立黨羽,很不得人心,張天錫就是再次發動政變,張邕兵敗自殺,其黨眾皆被張天錫誅殺。張玄靚遂以張天錫一人掌政。十二月,張天錫讓張玄靚改奉當時東晉的升平年號,稱升平五年。晉廷則授張玄靚大都督隴右諸軍事、護羌校尉、涼州刺史,西平公。

升平七年(363年),馬太后去世,張玄靚以其母郭夫人為太妃,而郭夫人因不滿張天錫專政而與張欽圖謀發動政變,可是圖謀外泄,張欽等人都被張天錫殺害。張天錫隨後便發動政變,派兵入宮殺死張玄靚,向外宣稱張玄靚暴斃,享年十四歲。

張玄靚被下葬平陵。張天錫私諡為沖公。東晉孝武帝司馬曜賜諡號敬悼。

\subsubsection{建兴}

\begin{longtable}{|>{\centering\scriptsize}m{2em}|>{\centering\scriptsize}m{1.3em}|>{\centering}m{8.8em}|}
  % \caption{秦王政}\
  \toprule
  \SimHei \normalsize 年数 & \SimHei \scriptsize 公元 & \SimHei 大事件 \tabularnewline
  % \midrule
  \endfirsthead
  \toprule
  \SimHei \normalsize 年数 & \SimHei \scriptsize 公元 & \SimHei 大事件 \tabularnewline
  \midrule
  \endhead
  \midrule
  四三年 & 355 & \tabularnewline\hline
  四四年 & 356 & \tabularnewline\hline
  四五年 & 357 & \tabularnewline\hline
  四六年 & 358 & \tabularnewline\hline
  四七年 & 359 & \tabularnewline\hline
  四八年 & 360 & \tabularnewline\hline
  四九年 & 361 & \tabularnewline
  \bottomrule
\end{longtable}

\subsubsection{升平}

\begin{longtable}{|>{\centering\scriptsize}m{2em}|>{\centering\scriptsize}m{1.3em}|>{\centering}m{8.8em}|}
  % \caption{秦王政}\
  \toprule
  \SimHei \normalsize 年数 & \SimHei \scriptsize 公元 & \SimHei 大事件 \tabularnewline
  % \midrule
  \endfirsthead
  \toprule
  \SimHei \normalsize 年数 & \SimHei \scriptsize 公元 & \SimHei 大事件 \tabularnewline
  \midrule
  \endhead
  \midrule
  五年 & 361 & \tabularnewline\hline
  六年 & 362 & \tabularnewline\hline
  七年 & 363 & \tabularnewline
  \bottomrule
\end{longtable}


%%% Local Variables:
%%% mode: latex
%%% TeX-engine: xetex
%%% TeX-master: "../../Main"
%%% End:

%% -*- coding: utf-8 -*-
%% Time-stamp: <Chen Wang: 2021-11-01 11:53:51>

\subsection{悼公張天錫\tiny(363-376)}

\subsubsection{生平}

張天錫(346年-406年),字純嘏,本字公純嘏,因被人嘲笑是三字而自行改字,小名獨活,安定烏氏人。中國十六國時期前涼政權的最後一位君主。張天錫為前涼文王張駿少子,前涼桓王張重華之弟。張天錫在位時前秦國力強盛,雖曾主動斷絕與前秦關係,但最終仍逼於軍事力量而再度稱藩。及後張天錫反抗前秦徵召入朝的命令並射殺使者,前秦大軍遂攻伐前涼,張天錫不敵投降,前涼國於是滅亡。淝水之戰後張天錫南歸東晉,並在東晉終老。

和平元年(354年),張祚封張天錫為長寧王。建興四十九年(361年),張邕殺死當政的宋澄,當時的前涼君主張玄靚就以張天錫為中領軍,與張邕共輔朝政。不過,張邕因樹立黨羽專權,經常濫用刑法殺人,很不得人心,張天錫親信劉肅則與其共謀除去他。十一月,張天錫與張邕一同入朝,劉肅就與趙白駒跟著張天錫行動,二人先後襲擊張邕但都失敗,於是與張天錫一同走入宮中。逃走的張邕率三百軍人進攻宮門,張天錫登門樓指責張邕凶惡悖逆,聲言自己是在冒死保衞國家社稷,並只會針對張邕而已。張邕兵眾聞言都逃散,張邕自殺,張天錫又誅殺了張邕黨羽,專掌朝政。

升平七年(363年),郭太妃以張天錫專政,與張欽密謀誅殺張天錫,事洩,欽等皆死;右將軍劉肅於是勸張天錫自立,天錫遂會劉肅夜襲皇宮,殺張玄靚。張天錫自稱使持節、大都督、大將軍、護羌校尉、涼州牧、西平公,並派使者出使建康請命,東晉於是在366年授張天錫為大將軍、大都督、督隴右關中諸軍事、護羌校尉、涼州刺史,封西平公。前秦亦派大鴻臚授張天錫大將軍、涼州牧、西平公。

張天錫登位後多次在園池設宴,又沉迷於歌舞和女色,荒廢政事。張天錫更將兩個親信劉肅及梁景收為養子,讓二人參與朝政,令人們有怨言和恐懼,索商及天錫堂弟張憲曾經勸諫他但不獲授納。張天錫於升平十年(366年)與前秦斷交,並在進攻李儼時與前秦發生軍事衝突,並俘獲了陰據和他率領的五千兵。升平十五年(371年),前秦攻滅仇池,送還陰據及其士兵回國,並派梁殊及閻負隨行,順道送達前秦丞相王猛的書信,暗示要張天錫別和前秦作對。張天錫看後十分恐懼,於是派使者向前秦謝罪,向前秦稱藩,前秦天王苻堅任命其為使持節、散騎常侍、都督河右諸軍事、驃騎大將軍、開府儀同三司,涼州刺史、西域都護、西平公。然而因張天錫因為懼怕前秦吞併,於同年在姑臧設壇,遙與晉三公盟誓,又派從事中郎韓博出使東晉,並寫信給東晉大司馬桓溫,約定大舉出兵北伐,會師上邽。

升平二十年(376年),苻堅徵召張天錫入朝任武衞將軍,同時派了苟萇、毛盛、梁熙及姚萇等率十三萬步騎至西河郡,預備一旦張天錫拒絕應命就進攻前涼。張天錫接到梁殊、閻負送來的詔命後問及眾僚意見,除席仂建議送貨款和質子,徐圖後計外,大部份人都認為涼州有精兵及天險,可以取勝。張天錫於是決定反抗,派馬建率兵二萬抵抗秦軍,並命人射殺兩名前秦使節。面對秦軍進攻,馬建懼而退守清塞,張天錫又派掌據率兵三萬與馬建屯於洪池,自率五萬屯金昌城。可是,苟萇隨後進攻掌據時馬建就投降前秦,掌據戰死,張天錫又派趙充哲為前鋒,率五萬兵與苟萇等作戰,但又在赤岸大敗,張天錫出城意圖再戰,但因金昌城中反叛而被逼逃回姑臧並請降。苟萇等到姑臧後受降,並送張天錫到長安,其他郡縣都降秦,前涼滅亡。苻堅在長安為張天錫建了府邸,任命他為侍中、北部尚書,封歸義侯。

晉太元八年(383年),晉軍於淝水之戰擊潰來攻的前秦軍,當時張天錫任征南大將軍苻融的司馬隨軍,趁機南奔東晉,東晉朝廷下詔以張天錫為散騎常侍左員外,復封為西平郡公。後轉拜金紫光祿大夫。後曾加授廬江太守,桓玄掌政時為了招撫四方而任命張天錫為護羌校尉、涼州刺史。義熙二年(406年),張天錫去世,享年六十一歲,追贈為鎮西將軍,諡號悼公。

張天錫因文才而聲名遠著,回歸晉廷後亦甚得晉孝武帝知遇,可是朝中官員卻以其曾經亡國被俘而中傷他。會稽王司馬道子曾經問及涼州有甚出產,張天錫立即就答道:「桑葚甘甜、鴟鴞會變聲音、乳酪養生、人沒有嫉妒之心。」不過,後來張天錫表現得昏亂喪志,雖然有公爵爵位也得不到別人禮遇。至晉安帝隆安年間,當政的會稽王世子司馬元顯更常常請他來戲弄他。擔任廬江太守亦因為其家貧而獲授。

\subsubsection{升平}

\begin{longtable}{|>{\centering\scriptsize}m{2em}|>{\centering\scriptsize}m{1.3em}|>{\centering}m{8.8em}|}
  % \caption{秦王政}\
  \toprule
  \SimHei \normalsize 年数 & \SimHei \scriptsize 公元 & \SimHei 大事件 \tabularnewline
  % \midrule
  \endfirsthead
  \toprule
  \SimHei \normalsize 年数 & \SimHei \scriptsize 公元 & \SimHei 大事件 \tabularnewline
  \midrule
  \endhead
  \midrule
  七年 & 363 & \tabularnewline\hline
  八年 & 364 & \tabularnewline\hline
  九年 & 365 & \tabularnewline\hline
  十年 & 366 & \tabularnewline\hline
  十一年 & 367 & \tabularnewline\hline
  十二年 & 368 & \tabularnewline\hline
  十三年 & 369 & \tabularnewline\hline
  十四年 & 370 & \tabularnewline\hline
  十五年 & 370 & \tabularnewline\hline
  十六年 & 372 & \tabularnewline\hline
  十七年 & 373 & \tabularnewline\hline
  十八年 & 374 & \tabularnewline\hline
  十九年 & 375 & \tabularnewline\hline
  二十年 & 376 & \tabularnewline
  \bottomrule
\end{longtable}


%%% Local Variables:
%%% mode: latex
%%% TeX-engine: xetex
%%% TeX-master: "../../Main"
%%% End:



%%% Local Variables:
%%% mode: latex
%%% TeX-engine: xetex
%%% TeX-master: "../../Main"
%%% End:
