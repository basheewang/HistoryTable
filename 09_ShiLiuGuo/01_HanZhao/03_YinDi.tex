%% -*- coding: utf-8 -*-
%% Time-stamp: <Chen Wang: 2019-12-18 15:50:29>

\subsection{隐帝\tiny(318)}

\subsubsection{生平}

汉隐帝刘粲(?-318年),字士光,新興(今山西忻州市)匈奴人,是十六国时汉赵国君。汉昭武帝刘聪子。劉粲即位後便沉醉於酒色,更與其父的四位皇后亂倫,又大殺輔政大臣,將軍國大事全交給靳準。最終令靳準成功在平陽叛亂,劉粲亦在其中被殺。

劉粲才兼文武,年輕時即為當時俊傑。光興元年(晉永嘉四年,310年)劉聰即位為帝後,封劉粲為河內王,任命為撫軍大將軍,都督中外諸軍事。

永嘉之亂後,因晉牙門趙染叛晉歸降,劉聰命趙染等進攻鎮守長安的南陽王司馬模,劉粲與劉曜則領大軍作趙染後繼。同年攻陷長安,司馬模投降並被劉粲所殺。劉粲於是與劉曜等留守關中地區。但不久,司馬模從事中郎索綝等人圖謀復興晉室,聯合一些不肯投降的郡守起兵進攻長安,並於新豐擊敗劉粲,劉粲被逼撤還首都平陽。

次年,劉聰命劉粲與劉曜領兵進攻晉并州刺史劉琨所在的并州,並成功攻陷治所晉陽。不久劉琨與拓跋猗盧領大軍反攻晉陽,於汾河以東擊敗劉曜。劉曜回晉陽後,與劉粲等擄晉陽城中平民撤退,但被拓跋猗盧追及並於藍谷大戰,最終漢軍大敗,屍橫遍野,但劉粲等人成功撤退。

嘉平四年(建興二年,314年),劉聰升劉粲為丞相、領大將軍、錄尚書事,並進封晉王。年末再升相國、大單于,總管百事。劉聰於是將朝事都交給劉粲等人,漸漸不理朝政。劉粲亦專橫放肆,親近中護軍靳準和中常侍王沈等人而疏遠朝中如陳元禮等忠良官員。性格刻薄無恩,又不聽勸諫。而且又喜好營造宮室,將相國府建得像皇宮一般華麗,國民都開始厭惡他。

及後,因宦官郭猗和靳準都與皇太弟劉乂有積怨,於是建議劉粲誣陷劉乂謀反,以讓劉粲奪去儲君的地位。劉粲聽從,於是命卜抽領兵到東宮監視劉乂。麟嘉二年(建武元年,317年),劉粲命黨羽王平向劉乂說有詔稱京師平陽將有事變,要劉乂要穿護甲在衣內以作防備。劉乂信以為真,更命東宮臣下都穿護甲衣在衣服內。消息被劉粲知道後就派人報告王沈和靳準,靳準於是向劉聰稱劉乂將謀反作亂。劉聰初時不信,但王沈等都說:「臣等早就聽聞了,但怕說出來陛下不相信而已。」劉聰於是派劉粲領兵包圍東宮。同時劉粲又派王沈和靳準收捕十多個氐族和羌族酋長並對他們審問,更加將他們吊起在高處,並用燒熱的鐵灼他們的眼,逼他們誣陷自己與劉乂串通作亂。劉聰於是認定劉乂謀反而靳準等盡忠於他,於是廢劉乂為北部王。劉粲及後就派靳準殺死劉乂。事後劉粲被立為皇太子。

麒嘉三年(318年),劉聰病逝,死前遺命太宰劉景、大司馬劉驥、太師劉顗、太傅朱紀、太保呼延晏、守尚書令范隆和大司空靳準輔政。劉粲隨後繼位。靳準心有異志,於是先打算剷除朝中劉氏勢力,於是向劉粲誣稱一眾王公大臣想行廢立之事,謀圖誅殺皇太后靳月華及自己,改以劉粲弟劉驥掌權,勸劉粲盡早行動。但劉粲不接納。靳準為了令劉粲聽從自己,於是恐嚇靳月華和皇后靳氏,稱一旦劉粲被廢,靳氏一族就會遭到誅殺。二人於是趁劉粲寵幸之機勸說劉粲,終令劉粲聽從,並殺害劉景、劉顗、劉驥、齊王劉勱和大將軍劉逞等人,朱紀和范隆則被逼出奔長安投靠劉曜。八月,劉粲於上林苑閱兵,謀圖進攻擁兵在外的石勒,又以靳準為大將軍,錄尚書事。而劉粲又繼續貪圖酒色歡樂,將軍政大權都交給靳準。而靳準亦扶植宗族勢力,命堂弟靳明為車騎將軍,靳康為衞將軍。

後來,靳準即將作亂,於是招攬年長有德而且有聲望的金紫光祿大夫王延。但王廷不肯與他一同叛亂,並立刻趕去向劉粲報告,但途中遇到靳康並被對方抓去。靳準及後便領兵入宮,在光極前殿命士兵去將劉粲抓來,盡數其罪後將他殺害。諡劉粲為隱皇帝。


\subsubsection{汉昌}

\begin{longtable}{|>{\centering\scriptsize}m{2em}|>{\centering\scriptsize}m{1.3em}|>{\centering}m{8.8em}|}
  % \caption{秦王政}\
  \toprule
  \SimHei \normalsize 年数 & \SimHei \scriptsize 公元 & \SimHei 大事件 \tabularnewline
  % \midrule
  \endfirsthead
  \toprule
  \SimHei \normalsize 年数 & \SimHei \scriptsize 公元 & \SimHei 大事件 \tabularnewline
  \midrule
  \endhead
  \midrule
  元年 & 318 & \tabularnewline
  \bottomrule
\end{longtable}


%%% Local Variables:
%%% mode: latex
%%% TeX-engine: xetex
%%% TeX-master: "../../Main"
%%% End:
