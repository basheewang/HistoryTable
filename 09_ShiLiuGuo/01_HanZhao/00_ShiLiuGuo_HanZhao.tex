%% -*- coding: utf-8 -*-
%% Time-stamp: <Chen Wang: 2019-12-18 14:02:10>


\section{汉赵\tiny(304-329)}

\subsection{简介}

漢趙(304年-329年),又称前趙,是匈奴人劉渊所建的君主制割据政权,都平阳郡(今山西临汾西北),這是十六国時期建立的第一個政權。

304年,劉淵起兵,称漢王。308年称帝,国号“汉”。310年劉聰即位,311年和316年兩次攻破西晋都城洛陽、長安。318年劉曜即位,殺死靳準,次年改国号為「趙」。329年被後趙所滅,立國凡26年。其统治地区包含并州刺史部、雍州刺史部、秦州刺史部、豫州刺史部、司隶校尉部、冀州刺史部部分地区。

劉淵以自己祖先與漢朝宗室劉氏約為兄弟而自稱漢王,并自称继承汉朝,故以“汉”为国号,史稱「前汉」;以多为匈奴人,又称「胡漢」或「匈奴汉」;又统治地区位于中原北方,故称「北汉」,但此稱呼因易于與五代十国时期的北汉混淆而很少使用。劉曜以其发迹之地为战国时赵国之地,改国号为赵,为别于石勒的后赵,而史称「前趙」,或合稱之為「漢趙」。

劉淵為南匈奴單于的後裔,其父劉豹為匈奴左部帥,在五部中勢力最強。劉豹卒后,代父為左部帥。西晉有意削弱他與部落的關係,後二遷為離石將兵都尉,劉淵則利用此職位的權限,暗中擴展勢力。楊駿輔政時,為了拉攏劉淵,命他為建威將軍、五部大都督,封漢光乡侯,給予統率匈奴五部軍事的大權。到元康末年,成都王司馬穎為了擴大自己的勢力,極力拉攏劉淵,表其為「行寧朔將軍,監五部軍事」,加強劉淵在匈奴五部中的地位,並命劉淵居鄴城,以便控制。

到晉惠帝太安中(302年─303年),因河間王司馬顒、成都王司馬穎、齊王司馬冏、長沙王司馬乂等諸王相互殘殺,益州刺史部流民起義爆發,各地局勢不穩,在并州刺史部的匈奴五部右賢王劉宣等人也醞釀著反晋兴匈奴。右賢王劉宣與各部貴族商議共推劉淵為大單于,并派呼延攸告诉在邺城的刘渊,劉淵让呼延攸先回去告诉劉宣等召集各部,聲言聚集五部協助司馬穎,實際是為反晉作準備。

晋惠帝永興元年(304年)三月,司馬穎等攻占洛陽,司馬越挾持晉惠帝攻鄴,司马颖打敗司馬越,並虜獲晉惠帝。八月,司马越势力王浚、司馬騰攻鄴城,刘渊请求带领匈奴五部帮助司马颖抵御,司馬穎同意,并拜劉淵為北單于,派遣回并州刺史部的平阳郡調發匈奴五部為援。劉淵返回并州離石,眾人共推劉淵為大單于,并聚集五萬之眾。刘渊得知王浚军队已攻破邺城,司马颖南逃洛阳。刘渊还想遵守先前承诺帮助司马颖,劉宣等劝说刘渊起兵反晋。十月,劉淵從離石遷于左國城,稱漢王,改年號為元熙,置百官,大赦境內,並以復漢為名義,正式建立政權。

漢元熙元年(304年)十二月,晉并州刺史司馬騰遣兵攻漢,雙方大戰于大陵(今山西省文水北),劉淵大勝,並遣劉曜等攻取上黨、太原、西河各郡縣。當時在青、徐二州的王彌,魏郡的汲桑、石勒,上郡四部鮮卑陸逐延,氐族酋長單徵等人均擁立劉淵為共主。劉淵命王彌、石勒等人攻取河北各郡縣,並一度攻入西晉的重鎮許昌,其兵鋒進抵至西晉的首都洛陽城下。308年十月,劉淵正式稱帝,改年號為永鳳。309年,劉淵遣將攻占黎陽(今河南省浚縣東北),擊敗晉將王湛於延津(今河南省延津縣北),沉殺男女三萬人,又派遣四子劉聰進攻包圍洛陽。

310年,劉淵病重,命劉聰輔佐太子劉和。劉淵病死,劉和繼位,不久劉聰殺死劉和自立為帝。

劉聰繼位後,派遣族弟劉曜、大將王彌等率領四萬大軍攻取洛陽周邊的郡縣,以孤立斷絕洛陽。311年,石勒在苦縣(今河南鹿邑)消滅西晉主力部隊十多萬人。同年夏季,劉曜、王彌攻破洛陽,虜走晉懷帝,殺害官員百姓三萬餘人,史稱永嘉之亂。晉懷帝於次年被殺後,晉愍帝於長安即位。316年,劉聰派遣劉曜攻破長安,俘晉愍帝,西晉滅亡。隨著西晉的滅亡,中原廣大的地區,皆成為漢政權的統治範圍。

雖然劉聰名義上是中原的共主,但隨著领域的擴大,地方的割據迅速形成,漢國統治的地區實際上只有一小部分。

318年,劉聰病死,太子劉粲繼位。匈奴貴族靳準殺死劉粲奪權,在平陽的劉氏男女不分老少全部被殺,靳準自立為漢天王。鎮守長安的劉聰族弟劉曜得知平陽有變,自立為皇帝,派遣軍隊至平陽,族滅靳氏。與此同時,石勒亦以討伐靳準為名,率軍至漢都平陽,于是,平陽、洛陽以東的地區,皆落入石勒勢力之中。漢國於是遷都到長安。

319年,劉曜改國號「漢」為「趙」,史稱「前趙」或「漢趙」。同年,石勒在襄國自稱趙王,從前趙中分離出來,史稱「後趙」,双方決裂。後數年,關中地區連年叛亂及大疫,百姓死者眾多,劉曜撲滅了關中各地氐、羌人的反抗,於是遷徙上郡氐、羌二十萬人及隴西大姓楊、姜等一萬多戶到關中以充實人口。

前趙政權初步鞏固後,即向外擴張,平定隴右一帶的陳安,並向西進擊前涼。雙方在黃河沿岸僵持,張茂稱藩,並獻貢。前趙全盛时,擁兵二十八万五千餘人,據有司隶州、雍州、并州、豫州、秦州各一部,時關隴氐、羌,莫不降附。

324年,前赵军队開始向東挺进,意图夺取石勒所占的河南。325年,劉曜命劉岳率兵一萬五千人圍攻後趙石生於洛陽金墉城,石勒命從子石虎率軍救援,與劉岳在洛水西岸交戰,劉岳兵敗,退守石梁戌,石虎包圍石梁戌。劉曜率軍救援,屯兵于金谷(今河南省洛陽市西北),夜中前赵軍中哗变,士卒潰散,劉曜退歸長安。不久,石虎攻下石梁戌,生擒劉岳等人。

328年,石勒命石虎率大軍四萬從軹關(今河南省濟源市西北十五里)西攻蒲坂(今山西省永濟市蒲州鎮),劉曜親自率領水陸大軍從潼關渡河救援,石虎引兵撤退,劉曜追及並大破,石虎逃奔朝歌。劉曜取得這次大勝之後,從大陽關(今山西省平陸縣茅津渡)南渡,在洛陽金墉城圍攻石生。後趙的滎陽郡太守尹矩、野王郡太守張進等人相繼投降。这次战败震動了後趙。石勒認為洛陽一失守,劉曜必定會進攻河北,於是集結步兵六萬,騎兵二萬七千,從鞏縣渡洛水,進抵洛陽城下。

劉曜得知石勒親率大軍增援,撤走包圍金墉城的軍隊,在洛陽之西列陣十多萬軍隊,南北距離十多里。石勒率軍進入洛陽。到了決戰當天,由石虎率步兵三萬,從洛陽北方向西移動,攻擊劉曜的中軍;石堪、石聰各率騎兵八千,從洛陽西方向北移動,攻擊劉曜的前鋒。雙方大戰於洛陽西面的宣陽門外,交戰之後,石勒親自帶領主力,從西北大門出城,夾擊前趙軍,前趙軍大潰。劉曜飲酒過量,在昏醉中退走,為石堪所擒,這一仗前趙軍被斬首五萬人,主力部隊損失殆盡。

劉曜戰敗被擒,不久被殺。石勒軍乘勝西進,劉曜子劉熙、劉胤等人放棄長安,逃奔上邽(今甘肅省天水市)。329年九月,後趙出兵攻占上邽,殺趙太子劉熙及諸王公侯、將相卿校以下三千餘人,又在洛陽坑殺其王公及五郡屠各五千多人,並遷徙其百官、關東流民、秦雍大族九千多人到襄國,前趙滅亡。

在劉淵、劉聰時期,其範圍控有冀州刺史部、兖州刺史部、青州刺史部、徐州刺史部、豫州刺史部、并州刺史部、雍州刺史部、司隶校尉部、秦州刺史部一帶,然而實際控制範圍不大,劉聰時期,只局限在并州的一角(其餘部分在劉琨手中)和由劉曜坐鎮的關中一部分地區。黄河以北地区一帶由石勒所有,王彌的部將曹嶷控有青州、兗州、徐州一帶,慕容鲜卑更是趁机向南统治到幽州的一帶。

劉曜時期,史稱「東不踰太行,南不越嵩、洛,西不踰隴坻,北不出汾、晉」(引顧祖禹《讀史方輿紀要》),疆域範圍包括雍州、司隶州的渭水流域以及并州、豫州、秦州黃河以東一帶。

基本上,前趙的政治制度承襲漢魏以來的制度而又雜以舊俗。漢國的官制,自304年劉淵稱漢王建立割据的君主制政權後,即採取漢朝的官制,設丞相、御史大夫、太尉及六卿等中樞之官。軍事之官有大司馬、太尉、大將軍等高級將軍以及雜號將軍。而地方之官則沿習魏晉以來的州郡制,採用漢胡分治的政策來進行統治。大單于的權力極大,僅次於皇帝。到劉聰嘉平四年(314年),達到了較為完善的階段。而劉曜的前趙,繼承漢國之制度,小有改革。劉曜繼承君主制的前赵政權胡、汉分治的政策。以子劉胤為大司馬、大單于,置單于台于渭城(今陝西咸陽),自左、右賢王以下皆用少數族豪酋充當。另方面又大体沿用魏晉九品官人法(見九品中正制),設立學校,肯定士族特權,笼络漢人的世家大族、士族,以巩固其統治。

劉淵時,設單于台,最高長官為大單于,統率六夷部落,單于台的設置,是沿匈奴舊制而來。劉聰時,在統治區內設置左、右司隸,各領戶20多萬,每1萬戶設置一名內史,內史共有43人。在大單于下設置單于左、右輔,各主六夷十萬落,萬落叟置一名都尉。

前趙的社會經濟主要是農業,其次是畜牧業,其生產方式,沿襲漢魏以來的生產方式。

在前趙社會中,從事農業、手工業、牧業生產的還有奴隸。奴隸的來源主要是戰俘,其次是犯罪的官吏。國內還有大量從事遊牧及畜牧業的「六夷」部落,因歸降及征服的部落日益增多,故設單于台進行管理。

漢在劉聰時(310—318年),杂夷戶口大約有六十三萬戶,人口大約有三四百萬人以上;汉户未详。

在劉曜全盛時期,有兵力二十八萬五千人,在他出兵時,史稱「臨河列陣,百餘里中,鍾鼓之聲沸河動地,自古軍旅之盛未有斯比」(《晉書》.劉曜載記》)。

%% -*- coding: utf-8 -*-
%% Time-stamp: <Chen Wang: 2019-12-18 14:03:32>

\subsection{光文帝\tiny(304-310)}

\subsubsection{生平}

漢趙光文帝劉淵(249年至254年間-310年8月19日),字元海,新興匈奴人(今山西忻州市北),出身匈奴屠各部。為五胡十六國時代中,汉赵的開國君王。西晉末年八王之亂時諸王互相攻伐,南匈奴族人擁立其為大單于。304年,劉淵乘朝廷內亂而在并州自立,稱漢王,国号为漢(后改为趙,史称前漢、前趙或漢趙),5年後稱帝,改元永鳳。310年,劉淵在位六年病死,諡光文皇帝。

劉淵出身屠各族(南匈奴),是西漢冒頓單于的後代挛鞮家族的人,該家族因西漢劉邦以來,長期與漢朝王室通婚,同時兼具漢朝王室與匈奴貴族的血脈,故漢名多採取漢朝王族的劉姓為姓氏。

東漢獻帝年間,曹操統一華北地區後,重整匈奴五部,劉淵父親劉豹原是匈奴王族的左賢王,在此一時期被曹操任命為「左部元帥」;而劉淵的母親呼延氏,亦是《史記》紀載下的三大匈奴貴族姓氏之一,足見劉淵身份之高貴。

劉淵童稚時已十分聰明,七歲時母親呼延氏逝世,劉淵傷心得捶胸頓足地號叫,旁人都被其哀傷所感染,宗族部落的人都因其表現而對他十分欣賞。連當時曹魏司空王昶聽聞其行為後都讚賞他,又派人弔唁和送禮物。劉淵亦十分好學,拜崔游為師,學習《毛詩》、《京氏易》和《馬氏尚書》,劉淵尤其喜歡《春秋左氏傳》及《孫吳兵法》,《史記》、《漢書》等歷史典籍亦一一看過。同時,劉淵自以書傳中都因隨何、陸賈無武跡;周勃、灌嬰沒文才而都遭後人看不起,認為文武兼備才能獲世人欣賞,因而習武。劉淵臂力過人,善於射擊,可謂文武雙全。崔懿之、公師彧、王渾等都與他結交。

咸熙年間,劉淵到洛陽作任子,受到當時曹魏權臣司馬昭厚待。司馬炎篡魏建立西晉後,王渾向晉武帝司馬炎推薦劉淵,武帝接見劉淵後亦對他十分欣賞,更打算任命他參與平滅東吳的事,但因孔恂和楊珧以「非我族類,其心必異」為由,擔心一旦向劉淵委以重任並平滅東吳,他會在當地叛晉自立。武帝聽後才將擱置這打算。及後禿髮樹機能先後擊敗秦州刺史胡烈及涼州刺史楊欣,李熹建議任用劉淵討伐,但孔恂仍指劉淵可能會作亂涼州,武帝因而又否決了建議。當時在洛陽流浪的王彌正要回故鄉東萊,與劉淵餞別時,劉淵泣訴被人屢進讒言中傷,恐怕將會在洛陽遇害而不能再見到他。劉淵於是縱酒長嘯,同坐的都因他流淚。齊王司馬攸見劉淵後,更建議武帝殺劉淵,以免日後回匈奴五部所在的并州後會禍亂當地,但王渾反對。武帝同意王渾所言,最終沒有殺劉淵。

正巧任匈奴左部帥的父親劉豹於當時逝世,劉淵於是回到并州接替父親左部帥之位。太康末年劉淵官拜北部都尉。劉淵在當地申明刑法,禁止奸邪惡行,而且誠心與人交往,於是匈奴五部中的俊才都投歸劉淵,連幽州和冀州的名儒和寒門秀士都前來與他結交。永熙元年(290年),晉惠帝司馬衷繼位,由外戚楊駿輔政。楊駿為了拉攏遠人,樹立私恩,便任命劉淵為建威將軍、五部大都督,封漢光鄉侯。但至元康末年劉淵便因部下族人叛變出塞而免官。不久成都王司馬穎出鎮鄴城(今河北臨漳縣西南),為拉攏劉淵而表他行寧朔將軍、監五部軍事,並召他至鄴城。

當時八王之亂戰火再起,趙王司馬倫、齊王司馬冏及長沙王司馬乂先後以軍事力量上台掌權,司馬倫更曾篡位稱帝,天下大亂,盜賊蜂起。劉淵叔祖父劉宣見此,決心乘著西晉朝政混亂振興匈奴,於是秘密與族人推舉劉淵為大單于,又派遣呼延攸到鄴城通知劉淵。劉淵向司馬穎請歸不果,於是派呼延攸先回并州,命劉宣召集五部匈奴和在宜陽的一眾胡人,名為支持司馬穎,實質上卻圖謀叛變。

永安元年(304年)司馬穎擊敗司馬乂,成為皇太弟,任命劉淵為屯騎校尉。不久東海王司馬越和陳昣等與惠帝征討司馬穎,司馬穎又任命劉淵為輔國將軍、督北城守事。及至惠帝兵敗蕩陰(今河南湯陰縣)被俘至鄴城,司馬穎再任命劉淵為冠軍將軍,封"盧奴伯"。但在蕩陰之戰後不久,東嬴公司馬騰和安北將軍王淩等就起兵討伐司馬穎,劉淵趁機向司馬穎建議讓他回匈奴五部領部眾支援司馬穎,共同抵抗司馬騰和王淩的討伐部隊。司馬穎同意並拜劉淵為北單于、參丞相軍事。

劉淵回左國城(今山西吕梁市离石区)後,劉宣便為劉淵上大單于稱號,二十日之間就聚眾五萬,定都離石。及後劉淵被司馬騰盟友拓跋猗㐌和拓跋猗盧擊敗,同時司馬穎亦因受不住王淩大軍的進逼而棄守鄴城,帶惠帝逃回洛陽。劉淵在劉宣的反對下,最終決定不援救司馬穎,遷至左國城(今山西吕梁市离石区東北),又吸引數萬人歸附。

永興元年(304年)十一月,劉淵以自己祖先與漢朝宗室劉氏約為兄弟而自稱“漢王”,建國號漢,改元元熙,並追尊蜀漢後主劉禪為孝懷皇帝,又設漢高祖劉邦、漢世祖劉秀、漢昭烈帝劉備、漢文帝劉恆、漢武帝劉徹、漢宣帝劉詢、漢明帝劉莊和漢章帝劉炟等八位西漢、東漢和蜀漢皇帝的牌位;前三者為三祖,後五者為五宗,以漢室繼承者自居。同時自置百官,正式建立一個脫離西晉朝廷的獨立政權。

劉淵稱王後,身為并州刺史的司馬騰便派將軍聶玄討伐,但遭劉淵於大陵(今山西文水縣)擊敗。司馬騰知道聶玄兵敗後十分恐懼,率并州二萬多戶人南下山東地區。劉淵亦派劉曜先後攻陷太原、泫氏、屯留、長子、中都等地方,擴闊領土。次年(305年),劉淵所派將領劉欽再度擊敗司馬騰所派的討伐軍。同年并州爆發大饑荒,離石亦受影響,劉淵於是遷都黎亭。永嘉元年(307年),劉淵已攻陷并州大部份郡縣,並派兵進攻新任并州刺史劉琨。但劉琨擊敗漢軍,成功保著治所晉陽(今山西太原市)。戰後劉琨努力經營并州,更離間收降劉淵部下雜虜,漢軍向并州北部擴張的計劃因而受阻。劉淵於是聽從侍中劉殷和王育派兵進攻其他州郡,南侵進據長安(今陝西西安市未央區)和洛陽(今河南洛陽市)的建議;同時,汲桑、石勒、王彌、鮮卑陸逐延和氐酋大單于單徵數個在其他地方的軍事力量都相繼歸降劉淵,劉淵亦一一任官封爵,令漢國力量更為壯大;亦因這些加入者起事和影響的地方在冀州、徐州、青州等地,西晉受漢國侵襲的地區大大增加。永嘉二年(308年),劉淵攻破司州河東郡的蒲阪和平陽郡的平陽城(今山西臨汾市),更遷都蒲子(今山西交口縣),令兩郡屬下各縣抵抗劉淵的營壘都全部投降。同時亦派劉聰、石勒等南攻太行山、趙、魏地區。

十月甲戌日(308年11月2日),劉淵稱帝,改元永鳳。永嘉三年(309年),太史令宣于脩之認為都城蒲子所處崎嶇難以久安,建議遷都平陽。劉淵聽從並立刻遷都至平陽,改元河瑞。劉淵及後派劉聰、王彌等進攻壺關,先破劉琨所派援軍,後於長平擊敗晉東海王司馬越所派的援軍,成功攻陷壺關。劉淵於是先後於當年八月和十月派劉聰等領兵進攻洛陽,但都被晉軍擊敗,劉淵唯有撤軍。

次年劉淵病重,命太宰劉歡樂、太傅劉洋等宗室重臣入宮接受遺詔輔政。七月己卯日(8月19日),劉淵逝世,由太子劉和繼位。九月辛未日(10月20日)下葬永光陵,諡光文皇帝,廟號高祖,後改太祖。

劉淵對部眾的暴行顯得不能容忍,如一次派遣喬晞進攻西河郡,喬晞先殺不肯投降的介休縣令賈渾,後殺哭罵他的賈渾妻宗氏。劉淵知道後大怒,將喬晞追回並降秩四等,又為賈渾收葬。又將領劉景一次進攻黎陽,在延津擊敗晉將王堪後在黃河將三萬多人溺死,劉淵知道後大怒,更說:「劉景還有何顏面見朕!天道又怎能接受這種事!朕想消滅的只是司馬氏,平民有何罪!」於是貶劉景的官位。

根據《晉書》所載,劉淵膂力過人,姿儀魁偉奇特,身長超過兩米,鬍鬚長三尺有餘,其中雜有少量赤色毛髮。

\subsubsection{元熙}

\begin{longtable}{|>{\centering\scriptsize}m{2em}|>{\centering\scriptsize}m{1.3em}|>{\centering}m{8.8em}|}
  % \caption{秦王政}\
  \toprule
  \SimHei \normalsize 年数 & \SimHei \scriptsize 公元 & \SimHei 大事件 \tabularnewline
  % \midrule
  \endfirsthead
  \toprule
  \SimHei \normalsize 年数 & \SimHei \scriptsize 公元 & \SimHei 大事件 \tabularnewline
  \midrule
  \endhead
  \midrule
  元年 & 304 & \tabularnewline\hline
  二年 & 305 & \tabularnewline\hline
  三年 & 306 & \tabularnewline\hline
  四年 & 307 & \tabularnewline\hline
  五年 & 308 & \tabularnewline
  \bottomrule
\end{longtable}

\subsubsection{永凤}

\begin{longtable}{|>{\centering\scriptsize}m{2em}|>{\centering\scriptsize}m{1.3em}|>{\centering}m{8.8em}|}
  % \caption{秦王政}\
  \toprule
  \SimHei \normalsize 年数 & \SimHei \scriptsize 公元 & \SimHei 大事件 \tabularnewline
  % \midrule
  \endfirsthead
  \toprule
  \SimHei \normalsize 年数 & \SimHei \scriptsize 公元 & \SimHei 大事件 \tabularnewline
  \midrule
  \endhead
  \midrule
  元年 & 308 & \tabularnewline\hline
  二年 & 309 & \tabularnewline
  \bottomrule
\end{longtable}

\subsubsection{河瑞}

\begin{longtable}{|>{\centering\scriptsize}m{2em}|>{\centering\scriptsize}m{1.3em}|>{\centering}m{8.8em}|}
  % \caption{秦王政}\
  \toprule
  \SimHei \normalsize 年数 & \SimHei \scriptsize 公元 & \SimHei 大事件 \tabularnewline
  % \midrule
  \endfirsthead
  \toprule
  \SimHei \normalsize 年数 & \SimHei \scriptsize 公元 & \SimHei 大事件 \tabularnewline
  \midrule
  \endhead
  \midrule
  元年 & 309 & \tabularnewline\hline
  二年 & 310 & \tabularnewline
  \bottomrule
\end{longtable}


%%% Local Variables:
%%% mode: latex
%%% TeX-engine: xetex
%%% TeX-master: "../../Main"
%%% End:

%% -*- coding: utf-8 -*-
%% Time-stamp: <Chen Wang: 2019-12-18 14:07:20>

\subsection{昭武帝\tiny(310-318)}

\subsubsection{戾太子生平}

刘和(?-310年),字玄泰,新兴(今山西忻州市)匈奴人。十六國時汉赵國君,光文帝劉淵長子,呼延皇后所生。劉淵死後以太子身份繼位,但即位後即試圖剷除劉聰等勢力,反被劉聰所殺。

劉和身长八尺,雄毅美姿仪,好學,從小开始学习《毛诗》、《左氏春秋》、《郑氏易》。但性格多作猜忌,對屬下無恩德。

永鳳元年(晉永嘉二年,308年),劉淵稱帝,任命劉和為大將軍。兩個月後遷大司馬,封梁王。河瑞二年(晉永嘉四年,310年)被刘渊立为太子。同年刘渊病死,刘和即位,由太宰劉歡樂、太傅劉洋等人輔政。

即位后,卫尉刘锐和劉和舅父宗正呼延攸怨恨自己不被任命為輔政大臣;侍中刘乘則厭惡握有重兵的楚王劉聰,於是共同合謀,向劉和進讒,稱諸王擁兵於都城平陽內外,其中劉聰更加擁兵十萬,嚴重影響劉和的皇權,要劉和有所行動。劉和聽信,及後召其領軍安昌王劉盛和安邑王劉欽將意圖告知。劉盛聽後勸諫劉和不要懷疑兄弟們,但遭呼延攸和劉銳命左右殺死;劉欽見此畏懼,只好對劉和唯命是從。

翌日,劉銳率馬景領兵攻劉聰,呼延攸率永安王劉安國攻齊王劉裕,劉乘則率劉欽攻魯王劉隆,劉和又派尚書田密、武衞將軍劉璿攻北海王劉乂。但田密和劉璿命人攻破城門,並歸降劉聰;而劉銳知劉聰早作準備於是聯合呼延攸等攻擊並殺害劉隆及劉裕,又因害怕劉安國和劉欽有異心而將二人殺害。此時劉聰率軍攻克西明門入宮,劉銳等在劉聰軍前鋒緊追下逃到南宮。劉和則在光極殿西室被殺,劉銳等皆被收捕並被斬首示眾。

劉聰及後自立为帝,改元“光兴”,即昭武皇帝。

\subsubsection{武帝生平}

漢昭武帝刘聪(?-318年8月31日),字玄明,新兴(今山西忻州市)匈奴人。十六国时汉赵国君。汉光文帝劉淵第四子,母张夫人。劉聰學習漢人典籍,深受漢化。執政時期先後派兵攻破洛陽和長安,俘虜並殺害晉懷帝及晉愍帝,覆滅西晉政權並拓展大片疆土。政治上创建了一套胡、汉分治的政治体制。但同時大行殺戮,又寵信宦官和靳準等人,甚至在在位晚期疏於朝政,只顧情色享樂。其執政末期甚至出現「三后並立」的情況。

劉聰年幼時就已經很聰明和好學,令到博士朱紀都覺得十分驚奇。劉聰非但通曉經史和百家之學,更熟讀《孫吳兵法》,而且善寫文章,又習書法,擅長草書和隸書;另外,劉聰亦學習武藝,擅長射箭,能張開三百斤的弓,勇猛矯捷,冠絕一時。可謂文武皆能。

劉聰二十歲後到洛陽遊歷,得到大量名士結交。後擔任新興太守郭頤的主簿。及後遷任右部都尉,因安撫接納得宜而得到匈奴五部豪族的歸心。河間王司馬顒表劉聰為赤沙中郎將,但當時劉淵在鄴城任官,因害怕駐守鄴城的成都王司馬穎加害父親,於是投奔司馬穎,任右積弩將軍,參前鋒戰事。

永安元年(304年),司馬穎任命劉淵為北單于,劉聰於是被立為右賢王,並與父親應命回到匈奴五部為司馬穎帶來匈奴援軍。但劉淵回到五部後就稱大單于,劉聰亦改拜鹿蠡王。劉淵聚眾自立,同年即稱漢王,建立漢國。後來任命劉聰為撫軍將軍。

元熙五年(晉永嘉二年,308年),劉聰被派遣南據太行山。同年年末淵稱帝,劉聰升任車騎大將軍。不久封楚王。次年與王彌和石勒等進攻壺關,擊敗司馬越派去抵抗的施融和曹超,攻破屯留和長子,令上黨太守龐淳獻壺關投降。數月後又領兵攻洛陽,擊敗平北將軍曹武,長驅直進至宜陽。但劉聰因連番勝利而輕敵,被詐降的弘農太守垣延率兵乘夜偷襲劉聰,最終劉聰大敗而還。兩月後劉聰再與王彌、劉曜、呼延翼等進攻洛陽。晉室以為漢國剛遭大敗,短時間不會再南侵,於是疏於防備,知道劉聰等來攻十分畏懼,劉聰更一度進兵至洛陽附近的洛水。當時晉將北宮純率兵夜襲漢國軍壁壘,斬殺將領呼延顥;及後呼延翼更被部下所殺,所率部隊因喪失主帥而潰退,劉淵於是下令撤兵。劉聰則上表稱晉朝軍隊又少又弱,不能因呼延翼等人之死而放棄進攻,堅持要留下來。劉淵允許。而面對漢軍,防守洛陽的司馬越唯有嬰城固守。但及後司馬越乘劉聰到嵩山祭祀的機會派兵進攻留守的漢軍,斬殺呼延朗。安陽王劉厲見此,害怕劉聰怪罪自己而跳進洛水自殺。王彌此時以洛陽守備仍堅固和糧食不繼勸劉聰撤軍,但劉聰因為是自己請求留下,不敢自行撤軍。劉淵及後聽從宣于脩之之言,命劉聰領軍撤退,劉聰見此才撤軍。

劉淵回到平陽後,任命劉聰為大司徒。河瑞二年(晉永嘉四年,310年),劉淵患病,任命劉聰為大司馬、大單于,與太宰劉歡樂和太傅劉洋共錄尚書事,並在都城平陽西置單于臺。不久劉淵逝世,由太子刘和即位。

劉和即位後,受宗正呼延攸、衞尉劉銳及素來厭惡劉聰的侍中劉乘進言唆擺,決意要消除諸王勢力,尤其當時擁兵十萬的劉聰。劉和不久就採取行動,但因劉聰有備而戰,最終劉聰率軍從西明門攻進皇宮,並於光極殿西室殺害劉和,又收捕逃到南宮的呼延攸等人,並將他們斬首示眾。

劉和死後,群臣請劉聰繼位,劉聰以其弟北海王劉乂是單皇后之子而讓位給他,但劉乂仍堅持由劉聰繼位。劉聰最終答應,並說要在劉乂長大後將皇位讓給他,登位後即立劉乂為皇太弟。

刘聪为了稳固地位,又杀死嫡兄刘恭。

劉聰即位後三個月,即派劉曜、王彌和其子河內王劉粲領兵進攻洛陽,因與石勒於大陽會師並在澠池擊敗晉將裴邈,因此直入洛川,擄掠梁、陳、汝南、潁川之間大片土地,並攻陷百多個壁壘。次年,又派前軍大將軍呼延晏領二萬七千人進攻洛陽,行軍至河南時就已十二度擊敗抵抗的晉軍,殺三萬多人。後劉曜、王彌和石勒都奉命與呼延晏會合。呼延晏在劉曜等人未到時就先行進攻洛陽城,攻陷平昌門並大肆搶掠,更於洛水焚毀晉懷帝打算出逃用的船隻。劉曜等人到達後就一起攻進洛陽城,並攻進皇宮縱兵搶掠,盡收皇宮中的宮人和珍寶,又大殺官員和宗室。另外更俘擄晉怀帝和羊皇后,將他們移送到平陽。

永嘉之亂後,劉聰又因晉牙門趙染叛晉歸降而命劉曜和劉粲攻打關中,最終攻陷長安並殺晉南陽王司馬模,並讓劉曜據守長安。但不久就被晉馮翊太守索綝、安定太守賈疋和雍州刺史麴特等反擊,劉曜等兵敗,劉曜更被圍困於長安。終於嘉平二年(永嘉六年,312年)被逼退出長安,撤回平陽。

嘉平二年(312年)年初,劉聰曾派靳沖和卜翊圍困晉并州治所晉陽,但因拓跋猗盧率兵營救而失敗。不久,令狐泥因其父令狐盛被晉并州刺史劉琨殺害而投奔漢國,並說出晉陽虛實。劉聰十分高興,便派劉粲和劉曜攻晉陽,由令狐泥作嚮導。劉琨知道漢國來攻後就到中山郡和常山郡招兵,並向拓跋猗盧求救;同時由張喬和郝詵領兵擋住漢軍。但張、郝皆敗死,劉粲於是乘劉琨未及救援而攻陷並佔領晉陽。但不久拓跋猗盧則親率大軍與劉琨反攻晉陽,劉曜兵敗,唯有棄守晉陽,撤走時遭拓跋猗盧追及,在藍谷交戰但大敗。晉陽得而復失。

晉懷帝被擄至平陽後,就被劉聰任命為特進、左光祿大夫、平阿公。後來改封會稽郡公。劉聰曾與懷帝回憶昔日與王濟造訪他的往事,亦談到西晉八王之亂,宗室相殘之事。劉聰談得十分高興,更賜小劉貴人給懷帝。但於嘉平三年(313年)正月,劉聰在與群臣的宴會中命懷帝以青衣行酒,晉朝舊臣庾珉和王儁見此忍不住心中悲憤而號哭,令劉聰十分厭惡。當時又有人流傳庾珉等會作劉琨的內應以助他攻取平陽,於是殺害懷帝和庾珉等十多名晉朝舊臣。

晉懷帝被殺的消息於四月傳至長安後,在長安的皇太子司馬鄴便即位為晉愍帝。劉聰則派趙染與劉曜和司隸校尉喬智明等進攻長安,多次擊敗抵抗的麴允。趙染後更乘夜攻進長安外城縱火搶掠,至天亮才因麴鑒救援長安而撤出長安,但麴鑒追擊時又遭劉曜擊敗。後因劉曜輕敵而被麴允偷襲,喬智明被殺,劉曜唯有撤兵回平陽。

次年,再派劉曜與趙染出兵長安,索綝領兵抵抗,但趙染初戰於新豐城西因輕敵而敗北。不久二人與將軍殷凱再攻長安,在馮翊擊敗麴允,但當晚又被麴允夜襲殷凱軍營,殷凱戰死。隨後劉曜到懷縣轉攻晉河內太守郭默,但郭默固守不降。在新鄭的李矩此時還到劉琨所派的鮮卑騎兵,說服帶領他們的張肇進攻劉曜。漢國士兵看見鮮卑騎兵就不戰而走,劉聰見進攻不成,打算先消滅劉琨,故命令劉曜撤軍。

建元元年(晉建興三年,315年),劉曜在襄垣擊敗劉琨所派軍隊,並打算進攻陽曲。但此時劉聰又認為要先攻取長安,於是命劉曜撤軍回蒲阪。

劉聰命劉曜撤回蒲阪後數月即派劉曜進攻北地,劉曜先攻破馮翊,後攻上郡,麴允雖然領兵在靈武抵抗但因兵少而不敢進攻。建元二年(316年),劉曜攻取北地,後即進逼長安。雖然有多批援兵救援長安,但都因畏懼漢國軍隊而不敢進擊。而司馬保所派將領胡崧雖然在靈臺擊敗劉曜,但卻因不願見擊退劉曜後麴允和索綝勢力變得強大,竟然勒兵退還槐里。劉曜因而得以攻佔長安外城並圍困愍帝所在小城。在爆發飢荒的小城內死守兩個月後,愍帝決定出降,被送至平陽。西晉正式滅亡。次年出獵時命愍帝穿戎服執戟作前導,被認出後有老人哭泣。劉粲勸劉聰殺愍帝但劉聰想再作觀望。及後又命愍帝行酒、洗爵和執蓋等僕役工作,令晉朝舊臣流淚哭泣,辛賓更抱著愍帝大哭。劉聰終也殺害愍帝。

劉聰自嘉平三年(晉建興二年,314年)十一月立劉粲為相國、大單于,總管各事務後,就將國事委託給他。自己則開始貪圖享樂,次年更設上皇后、左皇后和右皇后以封妃嬪,造成「三后並立」。後來更立中皇后。在委託政務給劉粲的同時,劉聰亦寵信中常侍王沈、宣懷、俞容等人,劉聰因於後宮享樂而長時間不去朝會,群臣有事都會向王沈等人報告而不是上表送呈劉聰。而王沈亦大多不報告劉聰,只以自己喜惡去議決事項。王沈等人又貶抑朝中賢良,任命奸佞小人任官。劉聰又聽信王沈等人的讒言,於建元二年(316年)二月殺特進綦毋達、太中大夫公師彧、尚書王琰等七名宦官厭惡的官員,侍中卜幹哭著勸諫但就遭劉聰免為庶人。

太宰劉易、御史大夫陳元達、金紫光祿大夫劉延和劉聰子大將軍劉敷都曾上表勸諫劉聰不要寵信宦官。但劉聰完全相信王沈等,都不聽從。劉粲與王沈等人勾結,因此向劉聰大讚王沈等人,劉聰聽後即將王沈等人封列侯。劉易見此又上表進諫,終令劉聰發怒,更親手毀壞劉易的諫書,劉易於是怨憤而死;陳元達見劉易之死,亦對劉聰失望,憤而自殺。朝廷在王沈和劉粲等人把持之下綱紀全無,而且貪污盛行,臣下只會奉承上級;對後宮妃嬪宮人的賞賜豐盛,反而在外軍隊卻資源不足。劉敷見此就曾多次勸諫,劉聰卻責罵劉敷常常在他面前哭諫,令劉敷憂憤得病,不久逝世。

因為劉聰的完全信任,王沈和劉粲等人又與靳準聯手誣稱皇太弟劉乂叛變,不但廢掉並殺害劉乂,更趁機誅除一些自己討厭的官員,又坑殺平陽城中一萬五千多名士兵。劉粲在劉乂死後被立為皇太子。

麟嘉三年(318年),刘聪患病,以太宰劉景、大司馬劉驥、太師劉顗、太傅朱紀和太保呼延晏並錄尚書事,又命范隆為守尚書令、儀同三司,靳準為大司空,二人皆決尚書奏事,以作輔政。七月癸亥日(8月31日)逝世,在位九年。諡為昭武皇帝,庙号烈宗。

據說劉聰出生時形體非常,左耳有一白毛,長逾二尺,有光澤。

劉聰雖然因眾意而登位,但仍認為自己是不依長幼次序而被擁立,於是忌憚兄長劉恭,並乘他睡覺時將他刺殺。

劉聰因單太后的絕美姿貌而與她亂倫,單太后子皇太弟劉乂曾多次規勸母親,單太后因而慚愧憤恨而死。雖然及後知道劉乂曾作規勸間接令單太后逝世,但因懷念單太后而沒有廢去其皇太弟身份。

劉聰曾濫殺大臣,如左都水使者王攄就曾因魚蟹供應不足而被劉聰殺害;將作大匠靳陵就因未能如期建成「溫明」、「徽光」二殿而被殺。王彰曾勸諫劉聰不要游獵過度,要劉聰念及劉淵建國艱難,應專心朝政。但劉聰聽後大怒,又要殺王彰,只因太后張夫人絕食以及劉乂和劉粲死諫才赦免王彰。後來設立中皇后時,尚書令王鑒和中書監崔懿之等又諫止劉聰濫封皇后,亦被劉聰所殺。

\subsubsection{光兴}

\begin{longtable}{|>{\centering\scriptsize}m{2em}|>{\centering\scriptsize}m{1.3em}|>{\centering}m{8.8em}|}
  % \caption{秦王政}\
  \toprule
  \SimHei \normalsize 年数 & \SimHei \scriptsize 公元 & \SimHei 大事件 \tabularnewline
  % \midrule
  \endfirsthead
  \toprule
  \SimHei \normalsize 年数 & \SimHei \scriptsize 公元 & \SimHei 大事件 \tabularnewline
  \midrule
  \endhead
  \midrule
  元年 & 310 & \tabularnewline\hline
  二年 & 311 & \tabularnewline
  \bottomrule
\end{longtable}

\subsubsection{嘉平}

\begin{longtable}{|>{\centering\scriptsize}m{2em}|>{\centering\scriptsize}m{1.3em}|>{\centering}m{8.8em}|}
  % \caption{秦王政}\
  \toprule
  \SimHei \normalsize 年数 & \SimHei \scriptsize 公元 & \SimHei 大事件 \tabularnewline
  % \midrule
  \endfirsthead
  \toprule
  \SimHei \normalsize 年数 & \SimHei \scriptsize 公元 & \SimHei 大事件 \tabularnewline
  \midrule
  \endhead
  \midrule
  元年 & 311 & \tabularnewline\hline
  二年 & 312 & \tabularnewline\hline
  三年 & 313 & \tabularnewline\hline
  四年 & 314 & \tabularnewline\hline
  五年 & 315 & \tabularnewline
  \bottomrule
\end{longtable}

\subsubsection{建元}

\begin{longtable}{|>{\centering\scriptsize}m{2em}|>{\centering\scriptsize}m{1.3em}|>{\centering}m{8.8em}|}
  % \caption{秦王政}\
  \toprule
  \SimHei \normalsize 年数 & \SimHei \scriptsize 公元 & \SimHei 大事件 \tabularnewline
  % \midrule
  \endfirsthead
  \toprule
  \SimHei \normalsize 年数 & \SimHei \scriptsize 公元 & \SimHei 大事件 \tabularnewline
  \midrule
  \endhead
  \midrule
  元年 & 315 & \tabularnewline\hline
  二年 & 316 & \tabularnewline
  \bottomrule
\end{longtable}

\subsubsection{麟嘉}

\begin{longtable}{|>{\centering\scriptsize}m{2em}|>{\centering\scriptsize}m{1.3em}|>{\centering}m{8.8em}|}
  % \caption{秦王政}\
  \toprule
  \SimHei \normalsize 年数 & \SimHei \scriptsize 公元 & \SimHei 大事件 \tabularnewline
  % \midrule
  \endfirsthead
  \toprule
  \SimHei \normalsize 年数 & \SimHei \scriptsize 公元 & \SimHei 大事件 \tabularnewline
  \midrule
  \endhead
  \midrule
  元年 & 316 & \tabularnewline\hline
  二年 & 317 & \tabularnewline\hline
  三年 & 318 & \tabularnewline
  \bottomrule
\end{longtable}


%%% Local Variables:
%%% mode: latex
%%% TeX-engine: xetex
%%% TeX-master: "../../Main"
%%% End:

%% -*- coding: utf-8 -*-
%% Time-stamp: <Chen Wang: 2019-12-18 15:50:29>

\subsection{隐帝\tiny(318)}

\subsubsection{生平}

汉隐帝刘粲(?-318年),字士光,新興(今山西忻州市)匈奴人,是十六国时汉赵国君。汉昭武帝刘聪子。劉粲即位後便沉醉於酒色,更與其父的四位皇后亂倫,又大殺輔政大臣,將軍國大事全交給靳準。最終令靳準成功在平陽叛亂,劉粲亦在其中被殺。

劉粲才兼文武,年輕時即為當時俊傑。光興元年(晉永嘉四年,310年)劉聰即位為帝後,封劉粲為河內王,任命為撫軍大將軍,都督中外諸軍事。

永嘉之亂後,因晉牙門趙染叛晉歸降,劉聰命趙染等進攻鎮守長安的南陽王司馬模,劉粲與劉曜則領大軍作趙染後繼。同年攻陷長安,司馬模投降並被劉粲所殺。劉粲於是與劉曜等留守關中地區。但不久,司馬模從事中郎索綝等人圖謀復興晉室,聯合一些不肯投降的郡守起兵進攻長安,並於新豐擊敗劉粲,劉粲被逼撤還首都平陽。

次年,劉聰命劉粲與劉曜領兵進攻晉并州刺史劉琨所在的并州,並成功攻陷治所晉陽。不久劉琨與拓跋猗盧領大軍反攻晉陽,於汾河以東擊敗劉曜。劉曜回晉陽後,與劉粲等擄晉陽城中平民撤退,但被拓跋猗盧追及並於藍谷大戰,最終漢軍大敗,屍橫遍野,但劉粲等人成功撤退。

嘉平四年(建興二年,314年),劉聰升劉粲為丞相、領大將軍、錄尚書事,並進封晉王。年末再升相國、大單于,總管百事。劉聰於是將朝事都交給劉粲等人,漸漸不理朝政。劉粲亦專橫放肆,親近中護軍靳準和中常侍王沈等人而疏遠朝中如陳元禮等忠良官員。性格刻薄無恩,又不聽勸諫。而且又喜好營造宮室,將相國府建得像皇宮一般華麗,國民都開始厭惡他。

及後,因宦官郭猗和靳準都與皇太弟劉乂有積怨,於是建議劉粲誣陷劉乂謀反,以讓劉粲奪去儲君的地位。劉粲聽從,於是命卜抽領兵到東宮監視劉乂。麟嘉二年(建武元年,317年),劉粲命黨羽王平向劉乂說有詔稱京師平陽將有事變,要劉乂要穿護甲在衣內以作防備。劉乂信以為真,更命東宮臣下都穿護甲衣在衣服內。消息被劉粲知道後就派人報告王沈和靳準,靳準於是向劉聰稱劉乂將謀反作亂。劉聰初時不信,但王沈等都說:「臣等早就聽聞了,但怕說出來陛下不相信而已。」劉聰於是派劉粲領兵包圍東宮。同時劉粲又派王沈和靳準收捕十多個氐族和羌族酋長並對他們審問,更加將他們吊起在高處,並用燒熱的鐵灼他們的眼,逼他們誣陷自己與劉乂串通作亂。劉聰於是認定劉乂謀反而靳準等盡忠於他,於是廢劉乂為北部王。劉粲及後就派靳準殺死劉乂。事後劉粲被立為皇太子。

麒嘉三年(318年),劉聰病逝,死前遺命太宰劉景、大司馬劉驥、太師劉顗、太傅朱紀、太保呼延晏、守尚書令范隆和大司空靳準輔政。劉粲隨後繼位。靳準心有異志,於是先打算剷除朝中劉氏勢力,於是向劉粲誣稱一眾王公大臣想行廢立之事,謀圖誅殺皇太后靳月華及自己,改以劉粲弟劉驥掌權,勸劉粲盡早行動。但劉粲不接納。靳準為了令劉粲聽從自己,於是恐嚇靳月華和皇后靳氏,稱一旦劉粲被廢,靳氏一族就會遭到誅殺。二人於是趁劉粲寵幸之機勸說劉粲,終令劉粲聽從,並殺害劉景、劉顗、劉驥、齊王劉勱和大將軍劉逞等人,朱紀和范隆則被逼出奔長安投靠劉曜。八月,劉粲於上林苑閱兵,謀圖進攻擁兵在外的石勒,又以靳準為大將軍,錄尚書事。而劉粲又繼續貪圖酒色歡樂,將軍政大權都交給靳準。而靳準亦扶植宗族勢力,命堂弟靳明為車騎將軍,靳康為衞將軍。

後來,靳準即將作亂,於是招攬年長有德而且有聲望的金紫光祿大夫王延。但王廷不肯與他一同叛亂,並立刻趕去向劉粲報告,但途中遇到靳康並被對方抓去。靳準及後便領兵入宮,在光極前殿命士兵去將劉粲抓來,盡數其罪後將他殺害。諡劉粲為隱皇帝。


\subsubsection{汉昌}

\begin{longtable}{|>{\centering\scriptsize}m{2em}|>{\centering\scriptsize}m{1.3em}|>{\centering}m{8.8em}|}
  % \caption{秦王政}\
  \toprule
  \SimHei \normalsize 年数 & \SimHei \scriptsize 公元 & \SimHei 大事件 \tabularnewline
  % \midrule
  \endfirsthead
  \toprule
  \SimHei \normalsize 年数 & \SimHei \scriptsize 公元 & \SimHei 大事件 \tabularnewline
  \midrule
  \endhead
  \midrule
  元年 & 318 & \tabularnewline
  \bottomrule
\end{longtable}


%%% Local Variables:
%%% mode: latex
%%% TeX-engine: xetex
%%% TeX-master: "../../Main"
%%% End:

%% -*- coding: utf-8 -*-
%% Time-stamp: <Chen Wang: 2019-12-18 15:53:14>

\subsection{刘曜\tiny(318-328)}

\subsubsection{生平}

刘曜(?-329年),字永明,新興(今山西忻州市)匈奴人。是十六国时汉赵(又称前趙)国君。漢趙光文帝劉淵族子。劉曜由漢趙建國開始就經已為國征戰,參與覆滅西晉的戰爭,並於西晉亡後駐鎮長安(今陝西西安市)。後於靳準之亂中登上帝位,後遷都長安。但登位後不久,將領石勒就自立後趙,國家分裂。劉曜在其在位期間多番出兵平定和招降西戎和西方的割據勢力如仇池和前涼等。在國內亦提倡漢學,設立學校。及後與後趙交戰,一度大敗後趙軍並圍攻洛陽(今河南洛陽市),但終被石勒擊敗並被俘。劉曜及後被殺,死後不久前趙亦被後趙所滅。

劉曜年幼喪父,於是由劉淵撫養。年幼聰慧,有非凡氣度。八歲時隨劉淵到西山狩獵,其間因天雨而在一棵樹下避雨,突然一下雷電令該樹震動,旁邊的人都嚇得跌倒,但劉曜卻神色自若,因而得到劉淵欣賞。劉曜喜歡看書,但志在廣泛涉獵而非精讀文句,尤其喜愛兵書,大致都熟讀。劉曜亦擅长写作和書法,習草書和隸書。另一方面劉曜亦雄健威武,箭术娴熟,能一箭射穿寸余厚的铁板,號稱神射。劉曜亦时常自比乐毅、蕭何和曹參,当时人們都不認同,唯刘聪知道其才能。

二十歲時到洛陽遊歷,但期間就被定罪而要被誅殺,於是逃亡到朝鮮,後來遇到朝廷大赦才敢回來。劉曜亦覺得自己外表異於常人,怕不被世人所接納,於是在管涔山隱居。

晉永興元年(304年),劉淵自稱漢王,國號漢,改元元熙任命劉曜為建武將軍。劉曜當年就被派往進攻并州郡縣以開拓疆土。漢永鳳元年(晉永嘉二年,308年),劉淵稱帝,拜劉曜為龍驤大將軍。後封為始安王。漢河瑞元年(晉永嘉三年,309年),劉曜與劉聰等進攻洛陽,但被晉軍乘虛擊敗。河瑞二年(310年),劉淵患病,命劉曜為征討大都督、領單于左輔。不久劉淵逝世,太子劉和繼位。劉和後又被劉聰所殺,劉聰及後登位為帝。

劉聰登位後,不久就命劉曜與河內王劉粲等進攻洛陽,並擊敗晉將裴邈,在梁、陳、汝南、潁川之間大肆搶掠。次年,劉聰命呼延晏領兵攻洛陽,劉曜奉命領兵與其會合,並於六月壬辰日(7月8日)抵達西明門。五日後,劉曜等便攻入洛陽大肆搶掠和殺害大臣,並擄晉懷帝等人,將他們送到平陽(今山西臨汾市)。史稱「永嘉之亂」。當時王彌認為洛陽城池和宮室都完好,建議劉曜向劉聰建議遷都洛陽,但劉曜認為天下未定而洛陽四面受敵,並不可守,於是焚毀洛陽宮殿。

永嘉之亂後,鎮守長安的南陽王司馬模命牙門趙染領兵在蒲阪(今山西省永濟市)守備,但趙染因請求馮翊太守一職被拒絕而投降漢國,劉聰於是在八月命趙染攻取長安,又命劉曜和劉粲領兵跟隨。司馬模兵敗投降,並於九月被劉粲所殺,劉曜則獲任命為車騎大將軍、雍州牧,並改封中山王,鎮守長安。

劉曜取得長安後,司馬模的從事中郎索綝投靠安定太守賈疋,並與賈疋等人圖謀復興晉室,於是推舉賈疋為平西將軍,率五萬兵攻向長安。當時拒降漢國的晉雍州刺史麴特等人亦領兵與賈疋會合。劉曜於是領兵在黃丘與賈疋大戰,但被擊敗。梁州刺史彭蕩仲和駐守新豐(今陝西西安市臨潼區)的劉粲都先後被賈疋等人所擊敗,彭蕩仲死而劉粲北歸平陽,賈疋等人於是聲勢大振,關西胡人和漢人都響應。劉曜只得據守長安。嘉平二年(晉永嘉六年,312年),劉曜因賈疋圍困長安經已數月,且連續戰敗,於是掠長安八萬多名平民棄守長安,逃奔平陽。劉曜及後因長安失守而被貶為龍驤大將軍,行大司馬。

同年,晉并州刺史劉琨部下令狐泥叛歸漢國,劉聰於是命令狐泥作嚮導,以劉粲和劉曜領兵進攻并州治所晉陽(今山西太原市)。二人最終乘虛攻陷晉陽,奪取劉琨的根據地。因此功績,劉曜復任車騎大將軍。但兩個月後,劉琨即與拓跋猗盧聯手反攻晉陽,劉曜在汾河以東與拓跋六脩交戰,但兵敗墜馬並受重傷,因討虜將軍傅虎協助才得以逃回晉陽。劉曜及後掠晉陽城中人民逃歸平陽,但遭拓跋猗盧追及,在藍谷交戰但慘敗。但仍成功回到平陽。

嘉平三年(晉建興元年,313年),劉曜與司隸校尉喬智明等進攻長安,但遭麴允擊敗。次年又與趙染和殷凱進攻長安,但殷凱被麴允擊殺。劉曜於是轉攻河內太守郭默但不能攻破,後更被鮮卑騎兵所嚇退。建元元年(晉建興三年,315年),劉曜一度轉戰并州,雖曾獲勝,但不久又再回到蒲坂準備再次進攻長安。及後劉曜即被派往進攻北地,先攻馮翊而再攻上郡,前去抵抗的麴允不敢進擊。

建元二年(316年),劉曜圍困並攻陷北地,並逼近長安。九月,劉曜雖被司馬保將領胡崧所敗,但胡崧並沒有進一步攻擊,反而退守槐里,而其他援軍亦因懼怕漢國軍而不敢進逼,劉曜於是成功攻陷長安外城,逼得麴允和索綝只好據守城內小城。終於在十一月,因為城內被圍困三個月而食糧嚴重困乏,晉愍帝被逼向劉曜投降。劉曜受降並於隨後遷晉愍帝和眾官員到平陽,西晉正式滅亡。劉曜因此功而獲任命以假黃鉞、大都督、督陝西諸軍事、太宰。並被改封為秦王,再度鎮守長安。

麒嘉三年(晉太興元年,318年),劉聰患病,徵召劉曜為丞相,錄尚書事,與石勒一同受遺詔輔政。但劉曜和石勒都辭讓。劉聰於是任命劉曜為丞相、領雍州牧。同年劉聰死,太子劉粲登位。八月升劉曜為相國、都督中外諸軍事,仍舊鎮守長安。但當月大將軍靳準就叛變,殺害劉粲和大殺劉氏,並自稱漢天王,向東晉稱藩。劉曜知道靳準作亂,於是進兵平陽。

十月,劉曜進佔赤壁(今山西河津縣西北赤石川),太保呼延晏等人從平陽前來歸附,並興早前因靳準誅殺王公而逃至長安的太傅朱紀等共推劉曜為帝。劉曜稱帝後,派征北將軍劉雅和鎮北將軍劉策進屯汾陰(今山西萬榮),與石勒有掎角之勢,共同討伐靳準。

靳準先前已敗於石勒,見劉曜和石勒現在共同討伐自己,於是在十一月派侍中卜泰向石勒請和,但石勒將卜泰囚禁被送交劉曜。劉曜於是向卜泰說:「先帝劉粲在位時確實亂了倫常,司空靳準你執行伊尹和霍光廢立之權,令我得以登位,實在是很大的功勳。若你早早迎接我入平陽,我就要將朝政大事都全部委託給你了,何止免死?你就為我人入城傳話吧。」於是將卜泰送返平陽。靳準聽到卜泰的傳話後,因為自知當日奪權時殺了劉曜母親胡氏和劉曜兄長,於是猶豫不決。十二月,靳康聯結喬泰和王騰等人殺死靳準,共推尚書令靳明為主,又命卜泰帶六顆傳國璽向劉曜投降。此舉令石勒十分憤怒,領兵進攻靳明,靳明大敗而只得退入平陽,嬰城固守。隨後石勒與石虎一同進攻平陽,靳明於是向劉曜求救,劉曜於是派劉雅和劉策迎接,靳明於是帶著一萬五千名平陽人民逃出平陽。劉曜及後卻大殺靳氏,一如靳準殺劉氏一樣。在其欲纳靳康女为妾时,靳女说及家族被灭,号泣请死,刘曜出于哀怜才放过了靳康的一个儿子。

石勒在靳明逃離後亦攻入平陽,留兵戍守後東歸,並於光初元年(晉太興二年,319年)年初命左長史王脩獻捷報給劉曜。劉曜於是派司徒郭汜授予他趙王和太宰、領大將軍的職位,並加如同曹操輔東漢時的特殊禮待。但留仕劉曜的王脩舍人曹平樂卻向劉曜稱王脩此行其實是要來探聽劉曜虛實,以讓石勒趁機襲擊劉曜。劉曜眼見其軍隊疲憊不堪,於是聽信曹平樂之言,追還郭汜並處斬王脩。石勒及後從逃亡回來的王脩副手劉茂口中得知王脩被殺,因此大怒,開始與劉曜交惡。

劉曜回到長安後,即遷都長安,並設立宗廟、社稷壇和祭天地的南北郊。又改國號為「趙」,史稱「前趙」。同年,石勒自稱趙王,正式建立「後趙」。漢國就此一分為二。

及後,黃石屠各人路松多在新平郡和扶風郡起兵,依附南陽王司馬保。司馬保又讓雍州刺史楊曼及扶風太守王連據守陳倉(今陝西寶雞市東),路松多據守草壁。劉曜派劉雅等人進攻但不能攻下。光初二年(320年),劉曜親自率軍進攻陳倉,擊殺王連並逼楊曼投奔氐族。接著接連攻下草壁和安定,令司馬保恐懼而遷守桑城(今甘肃临洮县东)。不久司馬保被部下張春所殺,已向劉曜投降的司馬保部將陳安則請求進攻張春等,劉曜於是任命陳安為大將軍,進攻張春。陳安最終令張春逃至枹罕(今甘肅臨夏),並殺死張春同黨楊次,消滅司馬保殘餘勢力。

不久,前趙將領解虎和長水校尉尹車與巴氐酋長句徐和庫彭等聯結,意圖謀反。但事敗露,劉曜於是誅殺解虎和尹車,並囚禁句徐和庫彭等五十多人,打算誅殺。光祿大夫游子遠極力勸阻,但劉曜都不聽,游子遠一直叩頭至流血,更惹怒劉曜而將他囚禁;劉曜後盡殺句徐等人,更在長安市內將曝屍十日,然後丟進河中。此舉終令巴氐悉數反叛,自稱大秦,並得其他少數民族共三十多萬人響應,於是關中大亂,城門都日夜緊閉。在獄中的游子遠再度上書勸諫劉曜,劉曜看後大怒,命人要立刻殺害游子遠,幸得劉雅等人勸止劉曜,游子遠才得被赦免。劉曜下令內外戒嚴,打算親自討伐叛亂首領句渠知。但此時游子遠向劉曜進言獻策,認為出兵強行鎮壓會耗費太多時間和資源,建議劉曜大赦叛民,讓他們重回正常生活,讓他們自動歸降。又請給兵五千人讓他討伐可能不肯歸降的句渠知。劉曜聽從。隨著大赦令下達,游子遠所到之都有大批人歸降,游子遠又於陰密平定不肯投降的句氏宗族黨眾。及後游子遠更進兵隴右,擊敗自號秦王的虛除權渠,並令他歸降。由於虛除權渠一部是西戎中力量最強的,故此其他西戎部族都相繼歸降前趙。

光初五年(322年),劉曜親征仇池,仇池首領楊難敵率兵迎擊但被擊敗,被逼退保仇池城。此時仇池轄下的氐羌部落大多都向前趙投降。及後劉曜轉攻楊韜,楊韜因畏懼而與隴西太守梁勛等人投降。劉曜於是再攻仇池,但此時劉曜患病,而且軍中有疫症,被逼退兵。劉曜因怕楊難敵乘機追擊,於是派光國中郎將王獷游說楊難敵,最終令楊難敵投降。劉曜於是臣服仇池,並領兵撤回長安。

此時,秦州刺史陳安請求朝見劉曜,但劉曜以患病為由推辭,陳安於是大怒,以為劉曜已死,於是決心反叛。劉曜此時病情卻愈來愈嚴重,改乘馬輿先回長安,而命呼延寔在後守護輜重。但陳安卻領騎兵邀截,俘獲呼延寔並奪取輜重,後更將呼延寔殺害。陳安又派其弟陳集等領騎兵三萬追劉曜車駕,劉曜則派呼延瑜擊殺陳集並盡俘部眾。陳安見此感到恐懼,退還上邽(今甘肃天水市),但隨後又佔領汧城,並得到隴上少數民族的歸附,於是自稱涼王。次年,陳安圍攻前趙征西將軍劉貢,但被歸附前趙的休屠王石武與劉貢的聯軍擊敗,只得收拾兵眾退保隴城(今秦安縣東北)。不久劉曜親自率軍圍困隴城,並派別軍進攻陳安根據地上邽和平襄。陳安於是出城,試圖領上邽和平襄的軍隊解圍,當知道上邽被圍而平襄被攻破後,改為南逃陝中,最終被前趙將領呼延清追及並殺害。上邽和隴城都先後投降,原本歸附陳安的隴上部落都歸降前趙。

平定陳安後,劉曜於當年即進攻前涼,親自率兵臨西河並命二十八萬兵眾沿黃河立營,延綿百多里,軍中鐘鼓之聲震動河水和大地,嚇得前涼沿河的軍旅都望風奔退。劉曜又聲言讓軍隊分百道一同渡河進攻前涼都城姑臧,令前涼震動。前涼君主張茂於是向前趙稱藩。劉曜亦達成目的,領兵退還。

光初七年(324年),後趙司州刺史石生在新安擊斬前趙河南太守尹平,並掠五千多戶東歸。自此前趙和後趙在河東、弘農之間就常有戰事。光初八年(325年),後趙將領石佗攻前趙北羌王盆句徐,大掠而歸。劉曜因而大怒,派中山王劉岳追擊,自己更移屯富平作為聲援,終大敗後趙軍並斬殺石佗。不久後趙西夷中郎將王騰以并州投降前趙。

五月,晉司州刺史李矩等因多次被後趙石生所攻,投靠前趙。劉曜於是派劉岳和呼延謨領兵與李矩等人共同進攻石生。但劉岳圍困石生於金鏞城時,被救援石生的石虎擊敗,退保石梁,更反被石虎所圍;呼延謨亦被石虎所殺。劉曜於是親自率兵救援劉岳,但及後卻因軍中夜驚而被逼退回長安。劉岳因無援而且物資缺乏,終被石虎所俘並送往後趙都城襄國(今河北邢台)。王騰亦為石虎擊敗並殺害,郭默和李矩亦被逼南歸東晉,李矩長史崔宣則向後趙投降。此戰令後趙盡得司州。

光初十一年(328年),石虎領四萬人進攻河東,獲五十多縣反叛響應,於是進攻蒲阪。因楊難敵先於光初八年(325年)反攻前趙於光初六年(323年)所佔領的仇池;又成功抵抗前趙於光初十年(327年)的攻擊。另一方面前涼於光初十年知道前趙光初八年被後趙擊敗後,即恢復其晉朝的官爵,並侵略前趙。劉曜於是派河間王劉述領氐族和羌族兵眾守備秦州以防仇池和前涼從後偷襲,自己則親率全國精銳救援蒲阪。石虎恐懼退軍,劉曜追擊並在高候大敗石虎,斬殺石曕。後劉曜又進攻石生所駐的金鏞城,以千金堨之水灌城,又派兵攻汲郡和河內,令後趙滎陽太守尹矩和野王太守張進等投降。這次大敗震動後趙人心。而劉曜此時卻不安撫士眾,只與寵臣飲酒博戲。

三個月後,石勒親率大軍救援石生,並命石堪等人在滎陽與石勒會師。劉曜在得悉石勒已渡黃河,才建議增加滎陽守戍和封鎖黃馬關以阻後趙軍。不久洛水斥候與石勒前鋒交戰,劉曜從俘獲的羯人口中得知石勒來攻的軍隊強盛才感懼怕,於是解金鏞之圍,在洛水以西佈陣。石勒則領兵進入洛陽城。

後前趙前鋒在西陽門與後趙軍大戰,劉曜親自出戰,但未出戰就已飲酒數斗;出戰後再飲酒一斗多。後趙將石堪乘其酒醉大敗趙軍,劉曜在昏醉中退走,期間墮馬重傷,被石堪俘獲。

劉曜被俘後被送往襄國,途中石勒派李永醫治劉曜。到襄國後,石勒让他住在永豐小城,給予侍姬,更命令劉岳等人去探望劉曜。石勒後來命劉曜寫信勸留守長安的太子劉熙儘快投降,但劉曜卻在信中命令劉熙和大臣們匡正和維護國家,不要因為自己而放棄。石勒看見後感到厭惡,後來劉曜還是被石勒所殺。

刘曜在霸陵西南建寿陵,侍中乔豫、和苞上疏进谏,刘曜对规谏还听得进去。但是刘曜身死国灭,他的实际墓葬地不详。

劉曜高九尺三寸(2.2米以上),垂手過膝,目有赤光,眉色發白,鬚髯雖長卻相當稀疏。劉曜自少就酗酒,及至後來就更加嚴重。在其在洛陽兵敗被俘一戰中臨陣昏醉,可謂其戰敗的其中一個原因。劉曜亦好殺,如靳準之亂中報復性盡誅靳氏和誅殺句徐等人等都可見。對大臣亦時見殺戮,差點殺了游子遠;又以毒酒殺害進言勸諫的大臣郝述和支當。

劉淵:「此吾家千里駒也,從兄為不亡矣。」

劉聰:「永明,世祖、魏武之流,何數公足道哉!」

晉書評:「曜則天資虓勇,運偶時艱,用兵則王翦之倫,好殺亦董公之亞。而承基醜類,或有可稱。」

张茂:“曜可方吕布、关羽,而云孟德不及,岂不过哉。”(《十六国春秋》)

\subsubsection{光初}

\begin{longtable}{|>{\centering\scriptsize}m{2em}|>{\centering\scriptsize}m{1.3em}|>{\centering}m{8.8em}|}
  % \caption{秦王政}\
  \toprule
  \SimHei \normalsize 年数 & \SimHei \scriptsize 公元 & \SimHei 大事件 \tabularnewline
  % \midrule
  \endfirsthead
  \toprule
  \SimHei \normalsize 年数 & \SimHei \scriptsize 公元 & \SimHei 大事件 \tabularnewline
  \midrule
  \endhead
  \midrule
  元年 & 318 & \tabularnewline\hline
  二年 & 319 & \tabularnewline\hline
  三年 & 320 & \tabularnewline\hline
  四年 & 321 & \tabularnewline\hline
  五年 & 322 & \tabularnewline\hline
  六年 & 323 & \tabularnewline\hline
  七年 & 324 & \tabularnewline\hline
  八年 & 325 & \tabularnewline\hline
  九年 & 326 & \tabularnewline\hline
  十年 & 327 & \tabularnewline\hline
  十一年 & 328 & \tabularnewline\hline
  十二年 & 329 & \tabularnewline
  \bottomrule
\end{longtable}

%%% Local Variables:
%%% mode: latex
%%% TeX-engine: xetex
%%% TeX-master: "../../Main"
%%% End:


%%% Local Variables:
%%% mode: latex
%%% TeX-engine: xetex
%%% TeX-master: "../../Main"
%%% End:
