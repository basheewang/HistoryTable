%% -*- coding: utf-8 -*-
%% Time-stamp: <Chen Wang: 2019-12-19 09:47:53>


\section{前燕\tiny(337-370)}

\subsection{简介}

前燕(337年 - 370年)是十六國时代由鮮卑人首領慕容皝所建立的政權,至慕容儁正式稱帝建國,其國號為「燕」。其全盛时的统治地区包括冀州、兖州、青州、并州、豫州、徐州、幽州等部分。

以其所在地为战国时燕国旧地,故国号为“燕”。《十六国春秋》始用“前燕”之名,為區別同期的慕容氏諸燕,歷史學家遂袭用之。又以其王室姓慕容,又称为“慕容燕”,而其他慕容氏諸燕都不用这个称呼,「慕容燕」成为前燕的专称。

西晉時,慕容廆為鮮卑族慕容氏的首領,曾效忠西晉,與鮮卑族以外的民族作戰。後來其兒子慕容皝於337年自稱為燕王。342年擊敗了後趙的二十萬大軍,解除了來自中原的壓力,建都龍城(今遼寧省朝陽市)。東破夫餘及高句麗,攻滅鮮卑宇文部,成為遼西唯一的武裝勢力,為慕容儁攻占中原奠定了坚实的基礎。

352年,皝子慕容儁滅冉魏稱帝,遷都薊,并隨後的幾年平定了北方的局勢,於357年遷都鄴。其地「南至汝颍,東盡青齊,西抵崤黽,北守雲中」,與關中的前秦平分黃河流域。358年,慕容儁下令全國州郡檢查戶口,每戶僅留一丁,此外全部徵發當兵,擬拼集150萬大軍以滅東晉、前秦以統一天下。

360年正月,慕容儁在鄴檢閱軍隊,但隨即逝世。其子慕容暐即位,改元「建熙」,此後宮廷裡發生了一次內訌。

慕容儁死時命弟慕容恪輔政,後慕容恪阻止了宮廷的內訌。從360年慕容恪輔政到367年病死其間,為前燕政治較為穩定的時期。自建熙二年、東晉升平五年(361年),以前燕河內太守呂護倒戈反覆为导火索,前燕与东晋在中原展开了连绵的战事。

前燕建熙四年、東晉興寧元年(公元363年)前燕全面的攻势开始发动,四月,慕容忠攻滎陽(今河南滎陽東北),東晉滎陽太守逃到魯陽(今河南魯山)。建熙五年、東晉興寧二年(公元364年)二月,前燕李洪開始略地河南,四月,前燕攻許昌、汝南、陳郡,徙上述三地萬餘戶於幽州,遣鎮南將軍慕容塵屯許昌。七月,太宰慕容恪親自領兵攻打洛陽,東晉洛陽守軍战败逃离。建熙六年、東晉興寧三年(公元365年)三月,前燕攻克洛陽。

这一系列的战役后,前燕从东晋手中获得了中原的控制权。但东晋仍然不放弃收复失地的计划。东晋太和四年(公元369年)东晋大将桓温北伐,前燕一度陷入危机。不久,东晋军中绝粮,桓温被迫撤退,途中遭前燕军队伏击,损失三万余人,大败而归。

太宰慕容恪死後,太輔慕容评掌握大權。慕容评為人貪婪,並得太后所信任,得以掌握大權。

東晉乘慕容恪死時,由大司馬桓溫領兵北上,被慕容皝的兒子、少帝慕容暐之叔吳王慕容垂击败,慕容垂却被掌权者慕容评所猜忌。慕容垂被逼无奈,出走前秦,被苻坚收留。苻坚早就想消灭前燕,一直忌惮慕容垂,如今最大劲敌已经投誠,苻坚遂开始讨伐前燕的计划。前燕军团起初并未处于下风,但由于當權的慕容评为人贪鄙,致使军心离散,结果前燕15万主力部队被王猛所率领的前秦军歼灭。苻坚趁势率10万军队包围前燕的首都邺城。“散骑侍郎徐蔚等率扶余、高句丽及上党质子五百余人,夜开城门以纳坚军。”公元370年十一月,慕容暐逃出邺城,试图返回辽东的根据地龙城,中途被前秦军抓获,前燕灭亡。

\subsection{武宣帝生平}

慕容\xpinyin*{廆}(269年-333年6月4日),字弈洛瓌,昌黎棘城(今遼寧義縣)人。晉朝時鮮卑人,慕容部首領慕容涉歸之子,前燕建立者慕容皝之父,吐谷渾第一代首領慕容吐谷渾是其庶兄。

慕容廆年紀輕輕就已經長得魁梧高大,才能出眾且有雄大抱負。張華出鎮北方時慕容廆曾去拜訪他,雖然當時慕容廆仍是兒童,但張華卻十分欣賞他,更與他結交。

西晉武帝太康四年(283年)慕容涉歸死,其弟慕容删篡奪政權,更意圖殺害慕容廆,慕容廆於是投奔躲藏於遼東郡人徐郁家。至太康六年(285年),慕容删被其部下所殺,其部眾於是迎慕容廆繼位。

由于宇文部鲜卑和慕容涉歸有仇,慕容廆繼位後就请求晋朝政府允许其出兵討伐,但遭到拒绝。慕容廆于是反叛晋朝,出兵劫掠遼西郡,令當地傷亡和財物損失都十分嚴重。不久雖然為晋軍所敗,但仍常常侵掠昌黎郡,又進攻東邊的扶餘國,逼死其王依虑,更毀滅了扶餘國都。晋东夷校尉派部將贾沈援助扶餘,助扶餘王子依羅復國,當時慕容廆派兵截擊但被擊敗,扶餘亦成功復國。

慕容廆後自以先世世代臣服於中原王朝,而且力量懸殊,不能與當時是統一王朝的晉朝爭鋒,又稱不能因與晉朝不和而令當地百姓受戰禍之苦,遂於太康十年(289年)重新归顺晋朝。晉廷受降並封慕容廆為鮮卑都督。慕容廆當時就去東夷校尉府拜見東夷校尉何龕,以士大夫禮,穿著巾衣前去;但到後見何龕嚴兵以待,慕容廆於是改穿戎服,又稱主人不以禮待客,客亦不以禮相待。何龕聽聞慕容廆這樣說,慚愧之餘亦敬重他。

慕容廆又因當時勢弱和聲威日上而受宇文部鮮卑與段部鮮卑不斷侵擾,採取忍讓政策,以卑下的言辭和大量金錢去討好對方,段部鮮卑酋長段階於是將女兒下嫁慕容廆。慕容廆認為遼東郡過於僻遠,遂向西遷徙至徒河縣(今遼寧省錦州市)境的青山(今遼寧省義縣東)。元康四年(西元294年),慕容廆遷居大棘城(今遼寧省義縣西)。慕容廆又於勢力範圍內推廣農桑,並且施行與晉朝一樣的法制。至永寧二年(302年)兗、豫、徐、冀四州發生水災,鄰近冀州的幽州亦受影響,慕容廆則開倉賑災,助幽州人民渡過困境。

太安元年(西元302年),宇文部鮮卑酋長宇文莫圭命其弟宇文屈雲率軍進攻慕容廆,慕容廆避其主力反擊重創其別部將領宇文素怒延;宇文素怒延因羞憤而動員十萬人包圍慕容廆所在的大棘城,當時城內部眾都十分恐懼,沒有抵抗的意志,然而慕容廆稱這是在其計劃之中,勉勵部眾作戰,並親自領軍出擊,再度重創宇文素怒延兵團,追擊一百華里並俘虜及斬殺近萬人。原在宇文部下的遼東郡人孟暉率眾數千家歸降慕容廆,慕容廆任命孟暉當建威將軍。

永嘉元年(307年),慕容廆自稱鮮卑大單于。永嘉三年(309年)遼東郡太守龐本因私怨而殺害東夷校尉李臻。當地的附塞鮮卑素喜連和木丸津以為李臻報仇為名起兵,但卻沒有因新任東夷校尉封釋設計殺死龐本而罷兵,竟乘機攻略遼東郡中諸縣。當地晉兵更屢次兵敗。亂事持續了兩年,封釋已經無力再戰,請和但不果。而遼東百姓期間大多因戰火而投靠慕容廆。永嘉五年(311年),慕容廆面對這個情況,接納兒子慕容翰的建議,起兵討伐素喜連等人,將兩人殺害並吞併其部眾,將他們所掠的三千多家人及早前歸附自己的遼東郡人送還本郡,保全了遼東郡。

永嘉之亂後,大司馬王浚承制假立太子,並以慕容廆為散騎常侍、冠軍將軍、前鋒大都督、大單于,但慕容廆以不是王命所授而拒絕。及至建武元年(317年),時為晉王的晉元帝司馬睿承制拜慕容廆為假節、散騎常侍、都督遼左雜夷流人諸軍事、龍驤將軍、大單于、昌黎公。但慕容廆辭讓。當時魯昌勸說慕容廆支持司馬睿為帝,並以司馬睿晉朝正統之名討伐其他擁兵的鮮卑部落。慕容廆接納並命人循海路到建康勸進。至次年司馬睿即位為帝,再次要授予上一年慕容廆拒絕的官位,慕容廆這次就接受昌黎公以外的職位。

當時慕容廆政事修明,愛護人才,在北方紛亂的環境下,士大夫和民眾多歸附之,好像永嘉五年(311年)東夷校尉封釋病死前就托付孫兒封奕給慕容廆,其子封悛和封抽前來奔喪後因道路不通而不能返回,亦願留在當地,被慕容廆任命為長史和參軍。建興元年(313年),据有乐浪、带方二郡的张统因不堪长期孤军与高句丽作战而率千余家投靠慕容廆,慕容廆为其在侨置乐浪郡。為著管理大批的流人,他為冀州人設冀陽郡、豫州人設成周郡、青州人設營丘郡、并州人設唐國郡。同時又任用大批漢人賢才去處理政事和作自己的參謀。同時又推行儒學,除了讓世子慕容皝受學以外,自己在有餘暇時也會去聽講,故此令他統領的地方到處都有頌讚之聲,守禮謙讓之風亦流行。

但慕容廆如此受流徙當地的漢人支持,受到出身清河崔氏的平州刺史、东夷校尉崔毖的妒忌。崔毖曾數度遣使招請慕容廆前去但都不果,於是打算以武力拘禁他。崔毖於太興二年(319年)成功游说宇文部鮮卑、段部鲜卑和高句丽联合攻伐慕容廆,並許約戰後瓜分其領地。

當時面對三國聯軍來攻,慕容廆拒絕諸將出擊,認為他們新聚而銳不可擋,反而應該固守去令他們漸漸互相猜忌,待人心離異後才一舉擊破。及後三國聯軍包圍棘城,慕容廆閉門自守,卻特意送牛酒去宇文部那裏勞軍,以離間計挑起其餘兩國對宇文部的懷疑,最終令兩國各自率軍離去。但當時宇文部大人宇文悉獨官自以兵強,仍然留下進攻棘城。面對當時宇文部數十萬兵力,連營四十里的軍勢,慕容廆打算召留守徒河的兒子慕容翰入援,然而慕容翰卻認為棘城守軍足以守城,派使者向父親表示自己應該作為奇兵伺機突襲,配合城中守軍出擊就能夠大破對手;若自己也進去守城,那宇文部就能專心攻城,而且更示以部眾勢弱,將會削弱士氣。慕容廆在韓壽的進言下接受慕容翰的建言,不再召他回防。而此時宇文悉獨官亦聽聞慕容翰沒有入援棘城,擔憂不久成為後方大患,於是分兵先行消滅慕容翰。但慕容翰則設計打敗來攻的軍隊,更乘勝進攻宇文部大軍,慕容廆在接到慕容翰的消息後亦從城內出兵,成功大敗宇文部。

戰後,三國都遣使請和,崔毖亦因畏懼而派侄兒崔燾前來假意祝賀,以消對方對自己的怨恨。慕容廆卻借由崔燾傳話,要崔毖投降或出走,崔毖最終出奔高句麗,慕容廆就併吞其部眾。同時,主簿宋該亦勸慕容廆向東晉獻捷報,慕容廆於是命其作表,由長史裴嶷出使,同時將大敗宇文部時獲得的三顆印璽送呈建康。明年,裴嶷到建康時盛讚慕容廆,晉元帝於是拜慕容廆為安北將軍、平州刺史。太興四年(321年)再升慕容廆為都督幽、平二州及東夷諸軍事、車騎將軍、平州牧,封遼東郡公,賜丹書鐵券,允許他承制選置平州官員。

太寧元年(323年),後趙王石勒派使者來與慕容廆結好,但慕容廆卻收捕使者並押送到建康。石勒知道後大怒,於太寧三年(325年)命宇文乞得歸進攻慕容廆,卻被慕容廆所派去抵抗的軍隊擊敗,慕容仁等更乘勝攻破宇文部國都並掠奪其大量物資和人馬。

後來,慕容廆與太尉陶侃通信,稱讚王導和庾亮,並稱陶侃是「海內之望中唯足為楚漢輕重者」,表示願意為復興晉朝作出努力,只是礙於自己孤軍進攻難有成果,期待東晉大舉北伐時響應。同時還附著封抽、韓矯等建議封慕容廆為燕王、行大將軍事的上疏。陶侃將封抽的上疏報告朝廷,讓朝議定奪。咸和八年五月甲寅日(333年6月4日),慕容廆去世,享年六十五歲,當時朝議仍未有定論,知道慕容廆去世後就停止了。東晉遣使贈慕容廆大將軍、開府儀同三司,諡號為襄。咸康三年(337年)慕容皝自稱燕王時追諡為武宣王。至永和八年(352年)慕容廆孫慕容儁稱帝時,追諡為武宣皇帝。

\subsection{文明帝生平}

燕文明帝慕容\xpinyin*{皝}(297年-348年10月25日),字元真,小字万年,昌黎棘城(今遼寧義縣)鲜卑族人。中國五胡十六國時代前燕的開國君主,不過當時仍名義上臣屬於東晉,直至其子慕容儁正式稱帝後,才追尊廟號太祖,諡號為文明皇帝。其父為慕容部落的首領、遼東公慕容廆,其母段夫人。其庶長兄為建威将军慕容翰。

慕容皝勇武剛毅且多有謀略,崇尚經學,熟悉天文。建武初年拜冠軍將軍、左賢王、封望平侯。太兴四年(321年)十二月,慕容廆封遼東郡公,立身為嫡子的慕容皝为世子。其曾率眾出征,累有戰功,如於永昌元年(322年)率眾入侵段末柸的都城令支(今河北遷安縣西)。太寧末年,慕容皝拜平北將軍,封朝鮮公。咸和八年五月甲寅(333年6月4日),慕容廆去世。六月,慕容皝嗣辽东郡公,以平北将军行平州刺史,督摄部内,统治辽东。

同年,宇文乞得歸被宇文逸豆歸逼逐而在外去世,慕容皝出兵討伐,令宇文逸豆歸畏懼請和,慕容皝於是修築了榆陰和安晉二城後回軍。慕容皝弟征虜將軍慕容仁和廣武將軍慕容昭很得慕容廆寵愛,惹來慕容皝不滿,而二人在慕容皝登位後怕慕容皝不能接納自己,於是在慕容仁在平郭(今遼寧熊岳城)舉兵西行至棘城以攻慕容皝,並以慕容昭為內應。不過,慕容仁尚在途中,其計劃就被揭發,慕容昭被慕容皝賜死,慕容仁唯有回軍據守平郭。慕容皝於是派兵讨伐,却大败于汶城以北。及後孫機更以遼東郡向慕容仁投降,令其盡得遼東之地,而且獲得段部鮮卑首領段遼和鮮卑諸部的支持,遙遙相援。

咸和九年(334年),慕容皝接連派兵攻殺鮮卑木堤和烏丸悉羅侯。段遼亦攻徒何,不能攻破後更派段蘭和在上一年因懼慕容皝猜忌而出奔段部的慕容翰進攻柳城(今遼寧朝陽市),守將石琮死守,終於退軍。同年,派往東晉報喪的隊伍回遼東,慕容皝獲東晉授予鎮軍大將軍、平州刺史、大單于、遼東公,持節,並因以往慕容廆之事,都督幽、平二州及東夷諸軍事並承制置百官,但皆被慕容仁所留,慕容皝一直至次年隊伍被放回棘城才獲受命。慕容皝亦於同年率军讨辽东,成功奪取襄平(今遼寧遼陽市),居就、新昌兩縣亦歸降,慕容皝置和陽、武次和西樂三縣就撤軍,又將遼東大姓分徒於棘城。

咸康二年(336年)正月,慕容皝堅持趁海面結冰而從海路進攻慕容仁,於是在壬午日(2月17日)自昌黎東出發,經結冰海面走三百多里,至歷林口就放下輜重輕兵直取平郭。慕容皝軍行至平郭七里以外時,慕容仁斥候才向慕容仁報告,令慕容仁狼狽到城西北迎戰。當時慕容軍率部向慕容皝投降,震動慕容仁軍心,慕容皝於是趁機進攻,大敗對方,慕容仁亦被擒和被賜死。

平定慕容仁後,至六月,段遼又派兵進攻慕容皝,分別攻擊武興以及柳城,當時宇文逸豆歸亦攻進安晉以作聲援,慕容皝別將擊破攻武興之軍,而自己率兵增援柳城,逼走屯於城西的段蘭後轉攻安晉,並派封奕大敗逃走的宇文逸豆歸部眾。及後慕容皝預料二部會再來,於是命封奕在馬兜山設伏,成攻大敗下月來攻的段遼。其後又命世子慕容儁和封奕分別進攻段部和宇文部,皆大勝。慕容又下令在乙連東築好城並置戍,又建曲水城作好城之援,以威逼乙連。當時乙連大饑,段遼命人輸送糧食,但就被戍守好城的蘭勃所敗。後段遼部將段屈雲進攻興國,又被慕容皝將慕容遵擊敗並盡俘其部眾。

咸康三年十月丁卯(337年11月23日),慕容皝聽從封奕的勸告,自称燕王,建前燕,追慕容廆为武宣王,夫人段氏为武宣后,立世子慕容儁为王太子。當年又因段部鮮卑多番入侵,於是派宋回向後趙稱藩,以其弟慕容汗為人質,請求後趙與其聯兵進攻段部鮮卑。後趙天王石虎大悅,答允並送還慕容汗,約定明年進攻。随后在咸康四年(338年),石虎率眾進攻段部鮮卑,慕容皝則出兵進掠令支以北諸城,並大敗追擊的段蘭,大掠而還。而因石虎一直進攻,四十多座城被石虎所得,段遼於是棄守令支而逃至密雲山。石虎入令支後,不滿慕容皝自掠人民牲畜後回軍,不與其會師,於是下令進攻慕容皝。後趙軍一直進攻,至五月戊子日(6月12日)攻至棘城時,慕容皝打算逃亡,但被慕輿根勸阻,當時玄菟太守劉佩更率敢死隊數百騎出城衝擊後趙軍,所向披靡,令城中士氣大增;封奕亦勸慕容皝堅守,終令慕容皝安心不降。兩軍相持十多日後,後趙軍引兵退還,慕容皝派慕容恪率二千騎進攻後趙軍,驚擾敵軍而令其棄甲潰散,殺三萬餘人。及後慕容皝分兵收復原本叛歸後趙的各個郡縣,並擴境至凡城,置戍而還。十二月,段遼降後趙,不久又悔而轉投慕容皝,而後趙已派麻秋支援段遼,慕容皝於是命慕容恪設伏於密雲山,大敗麻秋,並帶著段遼和其部眾撤還。

咸康五年(339年),慕容皝守將擊退來攻凡城的後趙軍隊,又因自稱燕王未受東晉朝命,於是命長史劉翔向建康獻捷,兼求假燕王璽綬,又請大舉出兵平定中原。不過當時朝廷議論未肯容讓慕容皝稱王。此時慕容皝得知庾亮去世,其弟庾冰及庾翼分掌朝廷中樞及荊州要地,於是上書要晉成帝以史為鑑,親近賢達,不要親信外戚。又寫信給庾冰,指責他掌握朝權,卻未能為國雪恥,只「安枕逍遙,雅談卒歲」。庾冰知道慕容皝的上表和書信後十分恐懼,自以道遠而不能控制他,於是奏請順應慕容皝的請求。咸康七年(341年),慕容皝獲東晉任命為使持節、大將軍、都督河北諸軍事、幽州牧、大單于,封燕王。

在受封燕王的同一年,慕容皝下令在柳城以北,龍山以西修建龍城,並改柳城為龍城。至次年(342年)正式遷入龍城。遷都後,慕容皝聽從早前歸國的慕容翰建議,先襲破高句麗,後才再攻取宇文鮮卑,以解後顧之憂,專心圖取中原土地。慕容皝並自率精兵四萬從險狹的南道進攻高句麗,以慕容翰及慕容垂為前鋒,以王寓領偏師五千走平廣開闊的北道引誘敵軍,終出其不意,成功攻陷高句麗都城丸都(今吉林集安),高句麗王高釗出逃,慕容皝招引不出,且因王寓敗沒而沒有追擊,於是挖出高釗父高乙弗利的屍體,連同丸都城中府庫收藏的珍寶、高釗母親和妻子及擄掠的五萬多人一同西還,更毀丸都。高句麗因而於翌年(343年)向慕容皝稱臣,慕容皝於是送還其父親屍體,留其母為人質。

高句麗稱臣於慕容皝後,慕容皝又擊敗了宇文逸豆歸派來進攻的國相莫淺渾。建元二年(344年),慕容皝親自率二萬騎兵討伐宇文鮮卑,又派慕容翰為前鋒,慕容軍、慕容恪、慕容垂及慕輿根兵分三路一同進攻。慕容翰與宇文逸豆歸大將涉奕于大戰,涉奕于戰死,宇文部軍心瓦解,被慕容皝所敗,都城紫蒙川陷落,宇文逸豆歸敗死漠北,宇文鮮卑至此被慕容皝所併。

此战后慕容皝终究不能对慕容翰放心,将其赐死。

永和元年(345年),慕容皝又派慕容恪攻高句麗,攻克南蘇並置戍而還。永和二年(346年)又命慕容儁與慕容軍、慕容恪及慕輿根率一萬七千兵東襲夫餘,成功俘虜夫餘王餘玄等五萬多人回國。

而早在咸康六年(340年),後趙已大舉徵兵,大行屯田,並收集戰馬,準備進攻慕容皝。當時慕容皝認為薊城因樂安得重兵駐守而防禦空虛,突襲薊城,守城的石光驚懼而不出擊,慕容皝攻陷高陽並焚毀積聚的軍糧,更掠奪了三萬餘戶。此舉打亂了後趙進攻計劃,而慕容皝平定高句麗和宇文部等主要對手後,前燕就能更集中對抗後趙,終令前燕得以專心在永和六年(350年)乘後趙內亂出兵中原。

永和四年九月丙申日(348年10月25日),慕容皝去世,时年五十二,諡為文明王。

永和元年(345年),慕容皝自以古時諸侯即位皆稱元年,故此不再用晉朝年號,追咸和八年(333年)登位起計,改稱十二年。

慕容皝汉化较深,崇尚儒学,设东庠(学校),以大臣子弟为官学生,号高门生。亲临讲授,每月考试优劣。

慕容皝鼓勵農耕,例如就曾在朝陽門東設籍田,置官主理。後又親自巡行各郡縣,鼓勵和督察農業活動。更加罷園林供沒有土地的農民耕種,更贈送一頭牧牛給沒有牛的農民。

慕容皝曾樹立納諫之木,以示他願意接受正直諫言。

史載慕容皝身長七尺八寸(約191厘米)。

慕容皝好文學典籍,故他勸於到東庠講授,學生多達千餘人。慕容皝更親作《太上章》以取代《急就篇》作學生識字的書籍,又寫了《典誡》共十五篇,皆用來教授學生。

《晉書》載慕容皝一次在國境西邊畋獵,將渡河時見一個騎白馬,穿紅衣的老人,舉手指揮著慕容皝,說該處不是狩獵場,要慕容皝離開。不過慕容皝沒有將事件說出來,更渡河狩獵,接連幾日大有收獲。慕容皝及後見到一隻白兔,於是策馬追射,但馬匹卻跌倒,慕容皝亦墮馬受傷,這時才說出他看見老人一事。慕容皝回龍城後將後事託付給世子慕容儁,後就死去了。王隱《晉書》亦有相近記載,不過是老人說話後就不見了,而後追獵白兔時墮馬撼石,當場死亡。


%% -*- coding: utf-8 -*-
%% Time-stamp: <Chen Wang: 2021-11-01 11:56:21>

\subsection{景昭帝慕容儁\tiny(348-359)}

\subsubsection{生平}

燕景昭帝慕容\xpinyin*{儁}(319年-360年2月23日),字宣英,鮮卑名賀賴跋,昌黎棘城(今遼寧義縣)鲜卑人,五胡十六國時代前燕的君主。前燕文明帝慕容皝次子。慕容儁即位時仍名義上為東晉的燕王,然而於永和八年(352年)正式稱帝獨立。慕容儁在位期間消滅了冉魏,入據原本由後趙所佔領的中原地區,勢力大增,並移都鄴城,終與南方的東晉和關中的前秦政權三足鼎立。

慕容儁博覽群書,有文武才幹,曾領兵攻略段部鮮卑並大勝而還。咸康七年(341年),東晉封慕容皝為燕王,亦以慕容儁為燕王世子,假節、安北將軍、東夷校尉、左賢王。

永和四年九月丙申日(348年10月25日),慕容皝去世。十一月甲辰日(349年1月1日),太子慕容儁繼襲燕王爵位。派使臣到建康向東晉報告了喪事。他還任命弟弟慕容友為左賢王,任命左長史陽鶩為郎中令。次年(349年)稱元年,仍不用東晉年號。同年後趙皇帝石虎去世,諸子爭位令國內大亂,慕容儁圖謀奪取中原土地,於是以慕容垂為前鋒都督、建鋒將軍,另外任命慕容恪為輔國將軍、慕容評為輔弼將軍和陽騖為輔義將軍,人稱三輔。挑選了二十多萬精兵等待時機。而同年東晉朝廷亦任命慕容儁為使持節、侍中、大都督、都督河北諸軍事、幽冀并平四州牧、大將軍、大單于、燕王,並依慕容廆和慕容皝的先例能承制封拜官員,在東晉授命下正式繼承了對遼東的管治。

永和六年(350年),後趙大將軍冉閔在鄴城稱帝,慕容儁亦乘機兵分三路南攻,自己親自率中軍出兵盧龍,攻下了薊城,並遷都至薊。因慕容儁聽從慕容垂不要坑殺薊城士卒的勸言,故得中原士民歸附。其他幽州郡縣多亦奪取,慕容儁於是設置幽州諸郡縣的官員。後慕容儁意圖進攻後趙幽州刺史王午和征東將軍鄧恆所守的魯口,不過被其將鹿勃早夜襲,雖然最終成功擊退對方,不過軍隊鋒銳已因這次突襲而受挫,只得暫緩戰事,返回薊城。不久代郡人趙榼率三百餘家叛歸後趙,慕容儁於是遷廣寧、上谷二郡人到徐無,代郡人到凡城,以防其再次叛歸後趙。不過,慕容儁亦南攻冀州,攻下了章武、河間二郡。

另一方面,守襄國的後趙皇帝石祗自永和六年起就被冉閔所圍攻。圍困百多日後,石祗被逼於永和七年(351年)向前燕求援,並許以傳國璽作交換。慕容儁欲得傳國璽,於是相信了後趙並派了悅綰救援襄國。同年冉閔被擊敗,襄國之圍解除,但悅綰沒有獲得傳國璽,慕容儁於是殺掉當日前來求援的後趙太尉張舉。慕容儁又派兵奪取中山和趙郡,又進攻魯口,擊敗王午派來迎擊的軍隊。

永和八年(352年),前燕王慕容儁派廣威將軍慕容軍、殿中將軍慕輿根、右司馬皇甫真等人率二萬人步、騎兵協助慕容評攻打冉魏鄴城。 永和八年(352年),冉閔攻陷襄國,將殺後趙皇帝石祗的將領劉顯勢力消滅。同年四月甲子日(5月5日),慕容儁命慕容恪等攻伐冉魏,最終擊敗冉閔並將其俘虜。己卯日(5月20日),冉閔被押送到薊城,慕容儁指責冉閔:「你只是配當奴僕的低下才幹,憑甚麼去稱帝?」冉閔卻說:「天下大亂,你這些夷狄禽獸都能稱帝,那我這種中土英雄,怎能不稱帝呀!」慕容儁聽後大怒,鞭打他三百下並送到龍城處死。同時,先前叛燕的段勤既受慕容垂進攻據地繹幕,看見慕容恪進據常山後就因畏懼而請降。

甲申日(5月25日),慕容儁命慕容評等進攻鄴城,冉魏太子冉智與將領蔣幹閉城門自守,得晉將戴施率百餘人入鄴助守,並以傳國璽向東晉請糧。不過,慕容評終於八月庚午日(9月8日)攻下鄴城,俘冉智等人至中山。冉魏亡後,當時擁兵據守州郡的後趙官員都派使者向前燕請降。

攻下鄴城後,慕容儁假稱冉閔皇后董氏獻傳國璽予他,賜董氏號「奉璽君」。十一月丁卯日(353年1月3日),慕容儁置百官,次日即位為皇帝,改年號為「元璽」,追尊慕容廆和慕容皝為皇帝並上廟號。當時東晉使者到了前燕,慕容儁就對他說:「你回去告訴你的天子,中原無主,我被士民推舉為主,已經做了皇帝了!」

前燕南侵幽州時據守魯口的王午在永和八年(352年)自稱安國王,同年被殺,由呂護承襲稱號並繼續據守魯口。永和九年(353年),衛將軍慕容恪、撫軍將軍慕容軍、左將軍慕容彪等人屢次薦舉給事黃門侍郎慕容霸,說他有顯赫於世之才,應總攬重任。前燕皇帝慕容儁任命慕容霸為使持節、安東將軍、北冀州刺史、鎮守常山。永和九年(353年),慕容儁派慕容恪進兵討伐,終令呂護於永和十年(354年)歸降。後慕容儁又命慕容恪鎮守洛水,以慕容強為前鋒都督,進據黃河以南地方。永和十年(354年),慕容儁封弟弟慕容恪為太原王,慕容評為上庸王,封左將軍慕容彭為武昌王,封撫軍將軍慕容軍為襄陽王,封安東將軍慕容霸為吳王,左賢王慕容友為范陽王,散騎常侍慕容厲為下邳王,散騎常侍慕容宜為廬江王,寧北將軍慕容度為樂浪王。慕容桓為宜都王,慕容遵為臨賀王,慕容徽為河間王,慕容龍為歷陽王,慕容納為北海王,慕容秀為蘭陵王,慕容岳為安豐王,慕容德為梁公,慕容默為始安公,慕容僂為南康公。兒子慕容臧為樂安王,慕容亮為勃海王,慕容溫為帶方王,慕容涉為漁陽王,慕容暐為中山王。

永和十一年(355年),東晉蘭陵太守孫黑、濟北太守高柱、建興太守高甕及前秦河內太守王會、黎陽太守韓高都以所在郡投降前燕。而先前屯據蕕城,歸降前秦的前車騎將軍劉寧亦率二千戶人到薊城歸降請罪,慕容儁亦任命劉寧為後將軍。高句麗王高釗亦向前燕進貢。同年,據守廣固並向東晉稱藩的段龕寫信非議慕容儁稱帝之事,觸怒了慕容儁並令他派了慕容恪進討段龕,終於在次年攻陷廣固,俘虜了段龕。升平元年(357年),慕容儁又命慕容垂等率八萬兵到塞北進攻丁零敕勒,大敗對方並俘殺十多萬人,奪去十三萬匹馬和億萬頭牛羊。及後匈奴單于賀賴頭率部歸降前燕。

升平元年十一月癸酉日(357年12月14日),慕容儁遷都鄴城。升平二年(358年),東晉泰山太守諸葛攸進攻東郡,被慕容恪等擊敗,慕容恪更乘機掠奪河南土地。不久東晉北中郎將荀羨攻陷山茌,處死太守賈堅,亦被前燕軍隊擊敗並收復失地。升平三年(359年)諸葛攸再攻前燕,在東阿被慕容評等人擊敗。同年十月,東晉西中郎將謝萬與北中郎將郗曇北伐,但因郗曇因病退兵以及謝萬統率失誤而令軍隊驚潰敗退,前燕得以乘機奪取許昌、穎川、譙及沛諸郡各城。

另一方面,前秦平州刺史劉特率眾向前燕投降。慕容儁又於升平二年(358年)派了司徒慕容評等人進攻盤據并州自立的將領張平、李歷等,令張平的部下諸葛驤、蘇象等率當地一百三十八個壁壘歸降前燕。及後張平等先後出奔,前燕於是收降了其部眾。

此時前燕正与东晋、前秦形成三足鼎立之势,并且在当时是国力最强的。

升平二年(358年),慕容儁因於擴張領土的戰爭中屢次獲勝,於是更圖謀消滅東晉以及前秦。為此下令州郡核實男丁數目,每戶只留下一個男丁,其餘都被徴為士兵,務求令全國步兵達至一百五十萬人。慕容儁更命士兵於明年就要集合,並攻取洛陽。在劉貴的諫止下,慕容儁才與官員議論,最終改為「三五占兵」,並將集合期限寬貸至一年後,定於下一年冬季於鄴城集合。

不過慕容儁於升平三年(359年)就患病,他向弟弟慕容恪表示他擔心自己一病不起,而前秦和東晉尚未滅亡,憂心皇太子慕容暐未有足夠能力治理國家,於是打算仿效宋宣公,以慕容恪繼位。不過慕容恪堅決拒絕,更矢言會輔助慕容暐。升平四年(360年)正月,慕容儁於鄴城閱兵後不久就於當月甲午日(2月23日)病死,臨終遺命大司馬太原王慕容恪、司徒上庸王慕容評、司空陽騖、領軍將軍慕輿根為輔政大臣,虚龄四十二,諡為景昭皇帝,廟號烈祖。

慕容儁于建熙元年三月葬于龙城(今辽宁省朝阳市)的龙陵(具体方位不详)。

慕容儁長子,獻懷太子慕容曄於永和十二年(356年)去世,慕容儁對此十分傷心。一次慕容儁在蒲池與群臣飲宴,因為談及東周時周靈王的太子晉,竟流下淚來,更表示自己在慕容曄死後「鬚髮中白」,更明白為何曹操和孫權昔日要分別為兒子曹沖和孫登早逝而痛惜不已。

慕容儁喜好文學典籍,即位以來都講論不斷,處理政務以外都是和侍臣交流典籍的義理,更有四十多篇著述。慕容儁亦於顯賢里設小學教育冑子。

慕容儁曾夢見石虎咬他的手臂,令慕容儁十分厭惡,於是下令挖開石虎的墓穴,罵道:「死胡竟然敢夢中嚇天子!」於是命御史中尉陽約數其殘酷之罪,鞭屍後丟到漳水去。《資治通鑑》更謂慕容儁在石虎墓找不到石虎屍首,於是懸賞百金求屍;後因鄴城女子李菟報告,在東明觀找到石虎屍首,發現他竟僵硬不腐;石虎屍首被投進漳水後,更靠在柱邊不流走。

\subsubsection{元玺}

\begin{longtable}{|>{\centering\scriptsize}m{2em}|>{\centering\scriptsize}m{1.3em}|>{\centering}m{8.8em}|}
  % \caption{秦王政}\
  \toprule
  \SimHei \normalsize 年数 & \SimHei \scriptsize 公元 & \SimHei 大事件 \tabularnewline
  % \midrule
  \endfirsthead
  \toprule
  \SimHei \normalsize 年数 & \SimHei \scriptsize 公元 & \SimHei 大事件 \tabularnewline
  \midrule
  \endhead
  \midrule
  元年 & 352 & \tabularnewline\hline
  二年 & 353 & \tabularnewline\hline
  三年 & 354 & \tabularnewline\hline
  四年 & 355 & \tabularnewline\hline
  五年 & 356 & \tabularnewline\hline
  六年 & 357 & \tabularnewline
  \bottomrule
\end{longtable}

\subsubsection{光寿}

\begin{longtable}{|>{\centering\scriptsize}m{2em}|>{\centering\scriptsize}m{1.3em}|>{\centering}m{8.8em}|}
  % \caption{秦王政}\
  \toprule
  \SimHei \normalsize 年数 & \SimHei \scriptsize 公元 & \SimHei 大事件 \tabularnewline
  % \midrule
  \endfirsthead
  \toprule
  \SimHei \normalsize 年数 & \SimHei \scriptsize 公元 & \SimHei 大事件 \tabularnewline
  \midrule
  \endhead
  \midrule
  元年 & 357 & \tabularnewline\hline
  二年 & 358 & \tabularnewline\hline
  三年 & 359 & \tabularnewline
  \bottomrule
\end{longtable}


%%% Local Variables:
%%% mode: latex
%%% TeX-engine: xetex
%%% TeX-master: "../../Main"
%%% End:

%% -*- coding: utf-8 -*-
%% Time-stamp: <Chen Wang: 2019-12-19 09:51:40>

\subsection{幽帝\tiny(360-370)}

\subsubsection{生平}

燕幽帝慕容\xpinyin*{暐}(350年-384年),字景茂,昌黎棘城(今遼寧義縣)鮮卑人。五胡十六國時代前燕的最後一位君主,前燕景昭帝慕容儁第三子。前期在慕容恪攝政之下仍能保持國家穩定,但後期在慕容評主政之下就漸漸衰落,最終被前秦所滅。慕容暐在前燕亡後成為前秦的臣下,獲封為新興侯。前秦於淝水之戰後崩潰,慕容垂、慕容泓先後舉兵建立「後燕」和「西燕」,慕容暐亦在西燕進攻前秦都城長安(今陝西西安)時意圖殺死苻堅並令城內混亂,響應外軍,但失敗被殺。

慕容暐最初獲封中山王。永和十二年(356年),皇太子慕容曄去世,慕容儁於是在次年立八歲的慕容暐為皇太子。升平四年(360年),慕容儁去世,臨終時遺命大司馬慕容恪、司空陽騖、司徒慕容評及領軍將軍慕輿根輔政。當時群臣打算立作為慕容儁弟弟的慕容恪繼位,但被慕容恪拒絕,而支持作為儲君的慕容暐即位。

慕容暐即位後便以慕容恪為太宰,讓他專攝朝政,而慕容評、陽騖和慕輿根則分別獲授太傅、太保及太師,參輔朝政。不過,當時慕輿根就自恃自己屢有戰功,顯得高傲自大,心中不服慕容恪。當時慕輿根打算作亂,初以可足渾太后干政煽動慕容恪謀反失敗,於是改向可足渾太后及慕容暐中傷慕容恪,想要他們誅殺慕容恪及慕容評。不過此時慕容暐卻信任慕容恪,勸止打算聽從的可足渾太后。及後慕容恪及慕容評密奏慕輿根罪狀,慕容暐於是命侍中皇甫真、右衞將軍傅顏等收捕慕輿根,並將其家人黨羽一併誅殺。

此時前燕國內正因慕容儁之死而混亂,原本徵集在鄴城的大軍都常常私下逃散,但在慕容恪的輔助下,最終都成功穩定了國家。在慕容恪攝政之下,先擊敗據守野王叛變的寧南將軍呂護,後更進侵當時為東晉所控的洛陽,終於興寧三年(365年)攻下洛陽。後又攻取了東晉的兗州諸郡。

不過,慕容恪於太和二年(367年)去世,死前想以吳王慕容垂代替自己為大司馬,但最終慕容評改以慕容暐弟慕容沖接替慕容恪。慕容恪死後,陽騖在同年亦死,唯一仍在世的輔政大臣慕容評就以太傅主政。當時僕射悅綰上奏盡罷軍封蔭戶,以釋放人口以充實國家地方,防止人口隱匿。慕容暐同意之下,最終在悅綰的規劃下釋放了二十多萬戶人,政令亦令朝野震驚,慕容評更是十分不滿,派人暗殺了悅綰。

太和四年(369年),東晉桓溫發動北伐戰爭,主動進攻前燕,慕容暐所派的慕容厲、傅顏及慕容臧皆不能抵抗桓溫進攻,於是令慕容暐及慕容評十分恐懼,向前秦求援以外還打算逃回和龍(今遼寧錦州)。這時慕容垂自請進攻,最終成功扭轉局勢,更在逼桓溫撤軍時大敗晉兵。然而慕容評在後十分忌憚剛取得大功的慕容垂,二人更因將領孫蓋軍功問題發生爭論。因可足渾太后亦討厭慕容垂,於是就與慕容評謀殺慕容垂,慕容垂只得與家人逃奔前秦。

不久,出使前秦的黃門侍郎梁琛歸國,報告前秦國內揚兵講武,而且運糧至陝東,更逢慕容垂出奔前秦,表示擔憂前秦和和前燕開戰,建議朝廷早作防備。然而慕容評不認為前秦會破壞和前燕的和平,慕容暐於是和慕容評都沒重視梁琛的話。及後皇甫真又上言表示擔憂前秦對前燕有所圖謀,建議增強洛陽、并州和壺關(今山西長治東南)各城的軍力。慕容暐於是召慕容評討論,但因慕容評認為前秦「國小力弱」,要倚靠前燕為援,前秦天王苻堅也不會因慕容垂而攻燕,勸慕容暐不要自亂陣腳。慕容暐於是亦沒有聽從皇甫真的話。

當日前燕向前秦求援時,允諾割讓虎牢(今河南滎陽西北汜水鎮)以西的土地給前秦,但戰後反悔。苻堅於是以此派王猛等進攻前燕,進攻洛陽。慕容暐於是派了慕容臧救援洛陽,然而卻在滎陽大敗給前秦軍,無法有效營救洛陽,洛陽守將慕容筑唯有向前秦投降,洛陽陷落。慕容臧只得築新樂城而退。面對當時的軍事形勢,而且太后干政、慕容評貪污,尚書左丞申紹上疏要改革,提出令將士用命,對士兵「習兵教戰」、「從戎之外,足營私業」等。又要君臣「罷浮華,禁絕奢,峻明婚姻喪葬之條」以及增加重地守備軍隊等措施,但慕容暐都沒聽從。

洛陽陷落的同年(370年),前秦再攻前燕,王猛攻壺關而楊安攻晉陽(今山西太原)。慕容暐命慕容評等率中外精兵三十多萬抵禦。不過,慕容評竟禁止士兵取水和柴,而自據水源和山,向士兵販賣柴水以斂財,導致軍心全無。最終被前秦軍夜燒輜重,火光連鄴城都看得見。慕容暐見狀十分恐懼,下令慕容評將金錢財帛都分給士兵,命他們作戰,慕容評因恐懼而向前秦請戰。最終前秦軍大敗前燕軍,俘殺超過十五萬人,慕容評單騎奔鄴城。

王猛在戰後追擊至鄴,苻堅亦派大軍後繼。面對前秦軍兵臨城下,慕容暐只得與慕容評等人逃奔龍城(今遼寧朝陽),但隨行衞士一出城就散走,只餘十多名仍然隨行。當時前奏將領郭慶亦在後追擊慕容暐,途中道路艱險難行而且時有盜賊,保衞慕容暐的左衞將軍孟高、殿中將軍艾朗皆戰死,慕容暐更因失去馬匹而只得徒步逃亡,最終在高陽被郭慶所俘。慕容暐隨後被押見苻堅,苻堅質問慕容暐為何不降而逃走,慕容暐答:「狐狸快要死時,也會將頭朝向自己出生的山丘,我都是想死在先人墳墓那裏而已。」苻堅憐憫慕容暐而將他釋放,命他回去率文武百官出降。另外逃奔遼東的前燕殘餘勢力不久亦被消滅,前燕正式滅亡。

同年十二月,慕容暐與慕容皇族及鮮卑族四萬戶一同被苻堅遷往長安安置,並受封為新興侯,署為尚書。

太元八年(383年),前秦大舉南侵東晉,即淝水之戰,慕容暐亦以平南將軍、別部都督隨軍。前秦於淝水之戰大敗後,時駐鄖城的慕容暐隨前秦軍北撤,並護送苻堅的張夫人。至滎陽時,叔父慕容德勸慕容暐乘前秦軍力大損而復國,但慕容暐又不聽從。慕容暐終與苻堅回到長安,但當時前秦對全國的控制已不如前。太元九年(384年),慕容垂在河北叛變建立後燕,不久,慕容暐之弟慕容泓也在關中叛變建立西燕。當時慕容泓向苻堅要求送還慕容暐以換取燕秦兩國和平,但為苻堅所拒絕。苻堅亦因此召慕容暐來斥責,終在慕容暐叩頭陳謝之下原諒他,並命他寫信招撫慕容垂、慕容泓和慕容沖。不過慕容暐就暗中派密使向慕容泓說:「我是鐵籠裡的人,肯定無法回去了;而且,我也是帝國的罪人,無必要顧慮我了。你就建立大業,以吳王慕容垂為相國,中山王慕容沖為太宰、領大司馬,你可以做大將軍、領司徒,承制封拜,收到我去世的消息後,你就自己稱帝吧。」及後西燕與前秦在長安多有戰事,慕容暐與慕容肅共謀聯同長安城中數千鮮卑人作亂,以應進攻長安的西燕軍,於是借兒子新婚為由設計在其家殺害苻堅。然而苻堅因大雨而沒有去,事情洩露,苻堅召慕容暐和慕容肅並殺害二人,更誅連城中的鮮卑人。慕容暐死時三十五歲。

西燕、后燕均没有追谥慕容暐。慕容德建立南燕時,諡慕容暐為幽皇帝。

慕容暐很在意他人對其的批評,如李績曾經向慕容儁表示慕容暐的缺點是「雅好遊田,娛心絲竹」,慕容儁亦因而要慕容暐好好記著李績的話,好作改善。但慕容暐登位後,慕容恪雖然屢請以李績為尚書右僕射,但慕容暐都不同意,更說:「萬機之事都交由叔父處理,但李績一人,我想自己決定。」最終李績憂死。

\subsubsection{建熙}

\begin{longtable}{|>{\centering\scriptsize}m{2em}|>{\centering\scriptsize}m{1.3em}|>{\centering}m{8.8em}|}
  % \caption{秦王政}\
  \toprule
  \SimHei \normalsize 年数 & \SimHei \scriptsize 公元 & \SimHei 大事件 \tabularnewline
  % \midrule
  \endfirsthead
  \toprule
  \SimHei \normalsize 年数 & \SimHei \scriptsize 公元 & \SimHei 大事件 \tabularnewline
  \midrule
  \endhead
  \midrule
  元年 & 360 & \tabularnewline\hline
  二年 & 361 & \tabularnewline\hline
  三年 & 362 & \tabularnewline\hline
  四年 & 363 & \tabularnewline\hline
  五年 & 364 & \tabularnewline\hline
  六年 & 365 & \tabularnewline\hline
  七年 & 366 & \tabularnewline\hline
  八年 & 367 & \tabularnewline\hline
  九年 & 368 & \tabularnewline\hline
  十年 & 369 & \tabularnewline\hline
  十一年 & 370 & \tabularnewline
  \bottomrule
\end{longtable}


%%% Local Variables:
%%% mode: latex
%%% TeX-engine: xetex
%%% TeX-master: "../../Main"
%%% End:


%%% Local Variables:
%%% mode: latex
%%% TeX-engine: xetex
%%% TeX-master: "../../Main"
%%% End:
