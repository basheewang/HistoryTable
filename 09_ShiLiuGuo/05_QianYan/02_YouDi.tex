%% -*- coding: utf-8 -*-
%% Time-stamp: <Chen Wang: 2019-12-19 09:51:40>

\subsection{幽帝\tiny(360-370)}

\subsubsection{生平}

燕幽帝慕容\xpinyin*{暐}(350年-384年),字景茂,昌黎棘城(今遼寧義縣)鮮卑人。五胡十六國時代前燕的最後一位君主,前燕景昭帝慕容儁第三子。前期在慕容恪攝政之下仍能保持國家穩定,但後期在慕容評主政之下就漸漸衰落,最終被前秦所滅。慕容暐在前燕亡後成為前秦的臣下,獲封為新興侯。前秦於淝水之戰後崩潰,慕容垂、慕容泓先後舉兵建立「後燕」和「西燕」,慕容暐亦在西燕進攻前秦都城長安(今陝西西安)時意圖殺死苻堅並令城內混亂,響應外軍,但失敗被殺。

慕容暐最初獲封中山王。永和十二年(356年),皇太子慕容曄去世,慕容儁於是在次年立八歲的慕容暐為皇太子。升平四年(360年),慕容儁去世,臨終時遺命大司馬慕容恪、司空陽騖、司徒慕容評及領軍將軍慕輿根輔政。當時群臣打算立作為慕容儁弟弟的慕容恪繼位,但被慕容恪拒絕,而支持作為儲君的慕容暐即位。

慕容暐即位後便以慕容恪為太宰,讓他專攝朝政,而慕容評、陽騖和慕輿根則分別獲授太傅、太保及太師,參輔朝政。不過,當時慕輿根就自恃自己屢有戰功,顯得高傲自大,心中不服慕容恪。當時慕輿根打算作亂,初以可足渾太后干政煽動慕容恪謀反失敗,於是改向可足渾太后及慕容暐中傷慕容恪,想要他們誅殺慕容恪及慕容評。不過此時慕容暐卻信任慕容恪,勸止打算聽從的可足渾太后。及後慕容恪及慕容評密奏慕輿根罪狀,慕容暐於是命侍中皇甫真、右衞將軍傅顏等收捕慕輿根,並將其家人黨羽一併誅殺。

此時前燕國內正因慕容儁之死而混亂,原本徵集在鄴城的大軍都常常私下逃散,但在慕容恪的輔助下,最終都成功穩定了國家。在慕容恪攝政之下,先擊敗據守野王叛變的寧南將軍呂護,後更進侵當時為東晉所控的洛陽,終於興寧三年(365年)攻下洛陽。後又攻取了東晉的兗州諸郡。

不過,慕容恪於太和二年(367年)去世,死前想以吳王慕容垂代替自己為大司馬,但最終慕容評改以慕容暐弟慕容沖接替慕容恪。慕容恪死後,陽騖在同年亦死,唯一仍在世的輔政大臣慕容評就以太傅主政。當時僕射悅綰上奏盡罷軍封蔭戶,以釋放人口以充實國家地方,防止人口隱匿。慕容暐同意之下,最終在悅綰的規劃下釋放了二十多萬戶人,政令亦令朝野震驚,慕容評更是十分不滿,派人暗殺了悅綰。

太和四年(369年),東晉桓溫發動北伐戰爭,主動進攻前燕,慕容暐所派的慕容厲、傅顏及慕容臧皆不能抵抗桓溫進攻,於是令慕容暐及慕容評十分恐懼,向前秦求援以外還打算逃回和龍(今遼寧錦州)。這時慕容垂自請進攻,最終成功扭轉局勢,更在逼桓溫撤軍時大敗晉兵。然而慕容評在後十分忌憚剛取得大功的慕容垂,二人更因將領孫蓋軍功問題發生爭論。因可足渾太后亦討厭慕容垂,於是就與慕容評謀殺慕容垂,慕容垂只得與家人逃奔前秦。

不久,出使前秦的黃門侍郎梁琛歸國,報告前秦國內揚兵講武,而且運糧至陝東,更逢慕容垂出奔前秦,表示擔憂前秦和和前燕開戰,建議朝廷早作防備。然而慕容評不認為前秦會破壞和前燕的和平,慕容暐於是和慕容評都沒重視梁琛的話。及後皇甫真又上言表示擔憂前秦對前燕有所圖謀,建議增強洛陽、并州和壺關(今山西長治東南)各城的軍力。慕容暐於是召慕容評討論,但因慕容評認為前秦「國小力弱」,要倚靠前燕為援,前秦天王苻堅也不會因慕容垂而攻燕,勸慕容暐不要自亂陣腳。慕容暐於是亦沒有聽從皇甫真的話。

當日前燕向前秦求援時,允諾割讓虎牢(今河南滎陽西北汜水鎮)以西的土地給前秦,但戰後反悔。苻堅於是以此派王猛等進攻前燕,進攻洛陽。慕容暐於是派了慕容臧救援洛陽,然而卻在滎陽大敗給前秦軍,無法有效營救洛陽,洛陽守將慕容筑唯有向前秦投降,洛陽陷落。慕容臧只得築新樂城而退。面對當時的軍事形勢,而且太后干政、慕容評貪污,尚書左丞申紹上疏要改革,提出令將士用命,對士兵「習兵教戰」、「從戎之外,足營私業」等。又要君臣「罷浮華,禁絕奢,峻明婚姻喪葬之條」以及增加重地守備軍隊等措施,但慕容暐都沒聽從。

洛陽陷落的同年(370年),前秦再攻前燕,王猛攻壺關而楊安攻晉陽(今山西太原)。慕容暐命慕容評等率中外精兵三十多萬抵禦。不過,慕容評竟禁止士兵取水和柴,而自據水源和山,向士兵販賣柴水以斂財,導致軍心全無。最終被前秦軍夜燒輜重,火光連鄴城都看得見。慕容暐見狀十分恐懼,下令慕容評將金錢財帛都分給士兵,命他們作戰,慕容評因恐懼而向前秦請戰。最終前秦軍大敗前燕軍,俘殺超過十五萬人,慕容評單騎奔鄴城。

王猛在戰後追擊至鄴,苻堅亦派大軍後繼。面對前秦軍兵臨城下,慕容暐只得與慕容評等人逃奔龍城(今遼寧朝陽),但隨行衞士一出城就散走,只餘十多名仍然隨行。當時前奏將領郭慶亦在後追擊慕容暐,途中道路艱險難行而且時有盜賊,保衞慕容暐的左衞將軍孟高、殿中將軍艾朗皆戰死,慕容暐更因失去馬匹而只得徒步逃亡,最終在高陽被郭慶所俘。慕容暐隨後被押見苻堅,苻堅質問慕容暐為何不降而逃走,慕容暐答:「狐狸快要死時,也會將頭朝向自己出生的山丘,我都是想死在先人墳墓那裏而已。」苻堅憐憫慕容暐而將他釋放,命他回去率文武百官出降。另外逃奔遼東的前燕殘餘勢力不久亦被消滅,前燕正式滅亡。

同年十二月,慕容暐與慕容皇族及鮮卑族四萬戶一同被苻堅遷往長安安置,並受封為新興侯,署為尚書。

太元八年(383年),前秦大舉南侵東晉,即淝水之戰,慕容暐亦以平南將軍、別部都督隨軍。前秦於淝水之戰大敗後,時駐鄖城的慕容暐隨前秦軍北撤,並護送苻堅的張夫人。至滎陽時,叔父慕容德勸慕容暐乘前秦軍力大損而復國,但慕容暐又不聽從。慕容暐終與苻堅回到長安,但當時前秦對全國的控制已不如前。太元九年(384年),慕容垂在河北叛變建立後燕,不久,慕容暐之弟慕容泓也在關中叛變建立西燕。當時慕容泓向苻堅要求送還慕容暐以換取燕秦兩國和平,但為苻堅所拒絕。苻堅亦因此召慕容暐來斥責,終在慕容暐叩頭陳謝之下原諒他,並命他寫信招撫慕容垂、慕容泓和慕容沖。不過慕容暐就暗中派密使向慕容泓說:「我是鐵籠裡的人,肯定無法回去了;而且,我也是帝國的罪人,無必要顧慮我了。你就建立大業,以吳王慕容垂為相國,中山王慕容沖為太宰、領大司馬,你可以做大將軍、領司徒,承制封拜,收到我去世的消息後,你就自己稱帝吧。」及後西燕與前秦在長安多有戰事,慕容暐與慕容肅共謀聯同長安城中數千鮮卑人作亂,以應進攻長安的西燕軍,於是借兒子新婚為由設計在其家殺害苻堅。然而苻堅因大雨而沒有去,事情洩露,苻堅召慕容暐和慕容肅並殺害二人,更誅連城中的鮮卑人。慕容暐死時三十五歲。

西燕、后燕均没有追谥慕容暐。慕容德建立南燕時,諡慕容暐為幽皇帝。

慕容暐很在意他人對其的批評,如李績曾經向慕容儁表示慕容暐的缺點是「雅好遊田,娛心絲竹」,慕容儁亦因而要慕容暐好好記著李績的話,好作改善。但慕容暐登位後,慕容恪雖然屢請以李績為尚書右僕射,但慕容暐都不同意,更說:「萬機之事都交由叔父處理,但李績一人,我想自己決定。」最終李績憂死。

\subsubsection{建熙}

\begin{longtable}{|>{\centering\scriptsize}m{2em}|>{\centering\scriptsize}m{1.3em}|>{\centering}m{8.8em}|}
  % \caption{秦王政}\
  \toprule
  \SimHei \normalsize 年数 & \SimHei \scriptsize 公元 & \SimHei 大事件 \tabularnewline
  % \midrule
  \endfirsthead
  \toprule
  \SimHei \normalsize 年数 & \SimHei \scriptsize 公元 & \SimHei 大事件 \tabularnewline
  \midrule
  \endhead
  \midrule
  元年 & 360 & \tabularnewline\hline
  二年 & 361 & \tabularnewline\hline
  三年 & 362 & \tabularnewline\hline
  四年 & 363 & \tabularnewline\hline
  五年 & 364 & \tabularnewline\hline
  六年 & 365 & \tabularnewline\hline
  七年 & 366 & \tabularnewline\hline
  八年 & 367 & \tabularnewline\hline
  九年 & 368 & \tabularnewline\hline
  十年 & 369 & \tabularnewline\hline
  十一年 & 370 & \tabularnewline
  \bottomrule
\end{longtable}


%%% Local Variables:
%%% mode: latex
%%% TeX-engine: xetex
%%% TeX-master: "../../Main"
%%% End:
